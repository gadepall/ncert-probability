\begin{enumerate}[label=\thesection.\arabic*,ref=\thesection.\theenumi]
	\item One card is drawn from a well-shuffled deck of 52 cards. Find the probability of getting
\begin{enumerate}
\item A king of red colour 
\item A face card 
\item A red face card
\item The jack of hearts
\item A spade
\item The queen of diamonds

\end{enumerate}
\solution
		%\begin{table}[H]
	\centering
\begin{tabular}{|c|c|c|}
\hline
Random variable &Value &Definition\\ \hline
\multirow{3}{*}{X} &0 &Slips of Rs 1\\
&1 &Slips of Rs 5\\
&2 &Slips of Rs 13\\ \hline
\multirow{2}{*}{Y} &0 &Box A\\
&1 &Box B\\\hline
\end{tabular}
\caption{}
\label{tab:Distribution}
\end{table}
See \tabref{tab:Distribution}.
\begin{align}
p_{Y}\brak{k}= \begin{cases} 
      \frac{1}{3} & {k=0} \\
      \frac{2}{3 }& {k=1} 
   \end{cases}
   \\
p_{Y|X}\brak{0|0} = \frac{19}{25}\, 
p_{Y|X}\brak{0|1} = \frac{6}{25}\,
p_{Y|X}\brak{1|0} = \frac{45}{50}\,
p_{Y|X}\brak{1|2} = \frac{5}{50}
\end{align}
The desired probability is the probability that a slip drawn at random is marked other than Rs 1,
\begin{align}
&=1-p_X\brak{0}\\
&= p_X(1) + p_X(2)
\end{align}
Using Bayes theorem,
\begin{align}
&= p_Y\brak{0} \times \pr{Y=0 | X=1} + p_Y\brak{1} \times \pr{Y=1|X=2}\\
&=\frac{1}{3} \times \frac{6}{25} + \frac{2}{3} \times \frac{5}{50}\\
&=\frac{11}{75}
\end{align}

\newpage

%\tableofcontents

\bigskip

\renewcommand{\thefigure}{\theenumi}
\renewcommand{\thetable}{\theenumi}
%\renewcommand{\theequation}{\theenumi}

%\begin{abstract}
%%\boldmath
%In this letter, an algorithm for evaluating the exact analytical bit error rate  (BER)  for the piecewise linear (PL) combiner for  multiple relays is presented. Previous results were available only for upto three relays. The algorithm is unique in the sense that  the actual mathematical expressions, that are prohibitively large, need not be explicitly obtained. The diversity gain due to multiple relays is shown through plots of the analytical BER, well supported by simulations. 
%
%\end{abstract}
% IEEEtran.cls defaults to using nonbold math in the Abstract.
% This preserves the distinction between vectors and scalars. However,
% if the journal you are submitting to favors bold math in the abstract,
% then you can use LaTeX's standard command \boldmath at the very start
% of the abstract to achieve this. Many IEEE journals frown on math
% in the abstract anyway.

% Note that keywords are not normally used for peerreview papers.
%\begin{IEEEkeywords}
%Cooperative diversity, decode and forward, piecewise linear
%\end{IEEEkeywords}



% For peer review papers, you can put extra information on the cover
% page as needed:
% \ifCLASSOPTIONpeerreview
% \begin{center} \bfseries EDICS Category: 3-BBND \end{center}
% \fi
%
% For peerreview papers, this IEEEtran command inserts a page break and
% creates the second title. It will be ignored for other modes.
%\IEEEpeerreviewmaketitle




	\item Five cards—the ten, jack, queen, king and ace of diamonds, are well-shuffled with their face downwards. One card is then picked up at random.
\begin{enumerate}
\item
What is the probability that the card is the queen? 
\item
If the queen is drawn and put aside, what is the probability that the second card picked up is (a) an ace? (b) a queen?\\
\end{enumerate}
\solution
		%\begin{enumerate}[label=\thesection.\arabic*,ref=\thesection.\theenumi]
	\item One card is drawn from a well-shuffled deck of 52 cards. Find the probability of getting
\begin{enumerate}
\item A king of red colour 
\item A face card 
\item A red face card
\item The jack of hearts
\item A spade
\item The queen of diamonds

\end{enumerate}
\solution
		%\begin{table}[H]
	\centering
\begin{tabular}{|c|c|c|}
\hline
Random variable &Value &Definition\\ \hline
\multirow{3}{*}{X} &0 &Slips of Rs 1\\
&1 &Slips of Rs 5\\
&2 &Slips of Rs 13\\ \hline
\multirow{2}{*}{Y} &0 &Box A\\
&1 &Box B\\\hline
\end{tabular}
\caption{}
\label{tab:Distribution}
\end{table}
See \tabref{tab:Distribution}.
\begin{align}
p_{Y}\brak{k}= \begin{cases} 
      \frac{1}{3} & {k=0} \\
      \frac{2}{3 }& {k=1} 
   \end{cases}
   \\
p_{Y|X}\brak{0|0} = \frac{19}{25}\, 
p_{Y|X}\brak{0|1} = \frac{6}{25}\,
p_{Y|X}\brak{1|0} = \frac{45}{50}\,
p_{Y|X}\brak{1|2} = \frac{5}{50}
\end{align}
The desired probability is the probability that a slip drawn at random is marked other than Rs 1,
\begin{align}
&=1-p_X\brak{0}\\
&= p_X(1) + p_X(2)
\end{align}
Using Bayes theorem,
\begin{align}
&= p_Y\brak{0} \times \pr{Y=0 | X=1} + p_Y\brak{1} \times \pr{Y=1|X=2}\\
&=\frac{1}{3} \times \frac{6}{25} + \frac{2}{3} \times \frac{5}{50}\\
&=\frac{11}{75}
\end{align}

\newpage

%\tableofcontents

\bigskip

\renewcommand{\thefigure}{\theenumi}
\renewcommand{\thetable}{\theenumi}
%\renewcommand{\theequation}{\theenumi}

%\begin{abstract}
%%\boldmath
%In this letter, an algorithm for evaluating the exact analytical bit error rate  (BER)  for the piecewise linear (PL) combiner for  multiple relays is presented. Previous results were available only for upto three relays. The algorithm is unique in the sense that  the actual mathematical expressions, that are prohibitively large, need not be explicitly obtained. The diversity gain due to multiple relays is shown through plots of the analytical BER, well supported by simulations. 
%
%\end{abstract}
% IEEEtran.cls defaults to using nonbold math in the Abstract.
% This preserves the distinction between vectors and scalars. However,
% if the journal you are submitting to favors bold math in the abstract,
% then you can use LaTeX's standard command \boldmath at the very start
% of the abstract to achieve this. Many IEEE journals frown on math
% in the abstract anyway.

% Note that keywords are not normally used for peerreview papers.
%\begin{IEEEkeywords}
%Cooperative diversity, decode and forward, piecewise linear
%\end{IEEEkeywords}



% For peer review papers, you can put extra information on the cover
% page as needed:
% \ifCLASSOPTIONpeerreview
% \begin{center} \bfseries EDICS Category: 3-BBND \end{center}
% \fi
%
% For peerreview papers, this IEEEtran command inserts a page break and
% creates the second title. It will be ignored for other modes.
%\IEEEpeerreviewmaketitle




	\item Five cards—the ten, jack, queen, king and ace of diamonds, are well-shuffled with their face downwards. One card is then picked up at random.
\begin{enumerate}
\item
What is the probability that the card is the queen? 
\item
If the queen is drawn and put aside, what is the probability that the second card picked up is (a) an ace? (b) a queen?\\
\end{enumerate}
\solution
		%\begin{enumerate}[label=\thesection.\arabic*,ref=\thesection.\theenumi]
	\item One card is drawn from a well-shuffled deck of 52 cards. Find the probability of getting
\begin{enumerate}
\item A king of red colour 
\item A face card 
\item A red face card
\item The jack of hearts
\item A spade
\item The queen of diamonds

\end{enumerate}
\solution
		%\begin{table}[H]
	\centering
\begin{tabular}{|c|c|c|}
\hline
Random variable &Value &Definition\\ \hline
\multirow{3}{*}{X} &0 &Slips of Rs 1\\
&1 &Slips of Rs 5\\
&2 &Slips of Rs 13\\ \hline
\multirow{2}{*}{Y} &0 &Box A\\
&1 &Box B\\\hline
\end{tabular}
\caption{}
\label{tab:Distribution}
\end{table}
See \tabref{tab:Distribution}.
\begin{align}
p_{Y}\brak{k}= \begin{cases} 
      \frac{1}{3} & {k=0} \\
      \frac{2}{3 }& {k=1} 
   \end{cases}
   \\
p_{Y|X}\brak{0|0} = \frac{19}{25}\, 
p_{Y|X}\brak{0|1} = \frac{6}{25}\,
p_{Y|X}\brak{1|0} = \frac{45}{50}\,
p_{Y|X}\brak{1|2} = \frac{5}{50}
\end{align}
The desired probability is the probability that a slip drawn at random is marked other than Rs 1,
\begin{align}
&=1-p_X\brak{0}\\
&= p_X(1) + p_X(2)
\end{align}
Using Bayes theorem,
\begin{align}
&= p_Y\brak{0} \times \pr{Y=0 | X=1} + p_Y\brak{1} \times \pr{Y=1|X=2}\\
&=\frac{1}{3} \times \frac{6}{25} + \frac{2}{3} \times \frac{5}{50}\\
&=\frac{11}{75}
\end{align}

\newpage

%\tableofcontents

\bigskip

\renewcommand{\thefigure}{\theenumi}
\renewcommand{\thetable}{\theenumi}
%\renewcommand{\theequation}{\theenumi}

%\begin{abstract}
%%\boldmath
%In this letter, an algorithm for evaluating the exact analytical bit error rate  (BER)  for the piecewise linear (PL) combiner for  multiple relays is presented. Previous results were available only for upto three relays. The algorithm is unique in the sense that  the actual mathematical expressions, that are prohibitively large, need not be explicitly obtained. The diversity gain due to multiple relays is shown through plots of the analytical BER, well supported by simulations. 
%
%\end{abstract}
% IEEEtran.cls defaults to using nonbold math in the Abstract.
% This preserves the distinction between vectors and scalars. However,
% if the journal you are submitting to favors bold math in the abstract,
% then you can use LaTeX's standard command \boldmath at the very start
% of the abstract to achieve this. Many IEEE journals frown on math
% in the abstract anyway.

% Note that keywords are not normally used for peerreview papers.
%\begin{IEEEkeywords}
%Cooperative diversity, decode and forward, piecewise linear
%\end{IEEEkeywords}



% For peer review papers, you can put extra information on the cover
% page as needed:
% \ifCLASSOPTIONpeerreview
% \begin{center} \bfseries EDICS Category: 3-BBND \end{center}
% \fi
%
% For peerreview papers, this IEEEtran command inserts a page break and
% creates the second title. It will be ignored for other modes.
%\IEEEpeerreviewmaketitle




	\item Five cards—the ten, jack, queen, king and ace of diamonds, are well-shuffled with their face downwards. One card is then picked up at random.
\begin{enumerate}
\item
What is the probability that the card is the queen? 
\item
If the queen is drawn and put aside, what is the probability that the second card picked up is (a) an ace? (b) a queen?\\
\end{enumerate}
\solution
		%\begin{enumerate}[label=\thesection.\arabic*,ref=\thesection.\theenumi]
	\item One card is drawn from a well-shuffled deck of 52 cards. Find the probability of getting
\begin{enumerate}
\item A king of red colour 
\item A face card 
\item A red face card
\item The jack of hearts
\item A spade
\item The queen of diamonds

\end{enumerate}
\solution
		%\input{ncert/10/15/1/14/main.tex}
	\item Five cards—the ten, jack, queen, king and ace of diamonds, are well-shuffled with their face downwards. One card is then picked up at random.
\begin{enumerate}
\item
What is the probability that the card is the queen? 
\item
If the queen is drawn and put aside, what is the probability that the second card picked up is (a) an ace? (b) a queen?\\
\end{enumerate}
\solution
		%\input{ncert/10/15/1/15/defs.tex}
	\item A bag contains $5$ red balls and some blue balls. If the probability of drawing a blue ball is double that if a red ball, determine the number of blue balls in the bag. 
		\\
\solution
		%\input{ncert/10/15/2/3/defs.tex}
	\item A card is selected from a pack of 52 cards.
 \begin{enumerate}[label=(\alph*)] 
                 \item How many points are there in the sample space?
                 \item Calculate the probability that the card is an ace of spades.
                 \item Calculate the probability that the card is (i) an ace and (ii) black card.
 \end{enumerate}
\solution
		%\input{ncert/11/16/3/4/main.tex}
\item Four cards are drawn from a well-shuffled deck of 52 cards. What is the probability of obtaining 3 diamonds and one spade.
\\
\solution
		%\input{ncert/11/16/4/2/defs.tex}
\item In a certain lottery 10,000 tickets are sold and ten equal prizes are awarded. What is the probability of not getting a prize if you buy (a) one ticket (b) two tickets (c) 10 tickets ?	
\\
\solution
		%\input{ncert/11/16/4/4/defs.tex}
		%
\item 
Out of 100 students, two sections of 40 and 60 are formed. If you and your friend are among the 100 students, what is the probability that
\begin{enumerate}
\item you both enter the same section?
\item you both enter the different sections?
\end{enumerate}
\solution
		%\input{ncert/11/16/4/5/defs.tex}
	\item 
The number lock of a suitcase has 4 wheels each labelled with ten digits i.e. from 0 to 9.The lock opens with a sequence of four digits with no repeats.What is the probability of a person getting the right sequence to open the suitcase.
\\
\solution
		%\input{ncert/11/16/4/10/defs.tex}
		%
\item 
Two cards are drawn at random and without replacement from a pack of 52 playing cards. Find the probability that both the cards are black.
\\
\solution
		%\input{ncert/12/13/2/2/defs.tex}
		\item A box of oranges is inspected by examining three randomly selected oranges drawn without replacement. If all the three oranges are good, the box is approved for sale, otherwise, it is rejected. Find the probability that a box containing 15 oranges out of which 12 are good and 3 are bad ones will be approved for sale.
		\label{ncert/12/13/2/3/defs.tex}
		\item Two balls are drawn at random with replacement from a box containing 10 black and 8 red balls. Find the probability that
		\label{ncert/12/13/2/12}
\begin{enumerate}
\item both balls are red.
\item first ball is black and second is red.
\item one of them is black and other is red.
\end{enumerate}

\item In a hostel, 60\% of the students read Hindi newspaper, 40\% read English newspaper and 20\% read both Hindi and English newspapers. A student is selected at random.
		\label{ncert/12/13/2/15}
\begin{enumerate}
\item Find the probability that she reads neither Hindi nor English newspapers.
\item If she reads Hindi newspaper, find the probability that she reads English newspaper.
\item If she reads English newspaper, find the probability that she reads Hindi newspaper.\\
\end{enumerate}
\item The probability of obtaining an even prime number on each die, when a pair of dice is rolled is 
\begin{enumerate}
    \item $0$ 
    
    \item $\frac{1}{3}$ 
    
    \item $\frac{1}{12}$ 
    
    \item $\frac{1}{36}$ 
\end{enumerate}
\solution
		%\input{ncert/12/13/2/17/defs.tex}
	\item A bag contains 4 red and 4 black balls, another bag contains 2 red and 6 black balls. One of the two bags is selected at random and a ball is drawn from the bag which is found to be red. Find the probability that the ball is drawn from the first bag.
\\
\solution
		%\input{ncert/12/13/3/2/main.tex}
  \item
  Cards with numbers 2 to 101 are placed in a box. A card is selected at random.Find the probability that the card has
\begin{enumerate}[label=(\roman*)]
	\item an even number 
	\item a square number
\end{enumerate}
\solution
%\input{exemplar/10/13/3/32/main.tex}
\item
The king, queen and jack of clubs are removed from a deck of 52 playing cards and then well shuffled. Now one card is drawn at random from the remaining cards.  Determine the probability that the card is
\begin{enumerate}[label=(\roman*)]
\item a club
\item 10 of hearts
\end{enumerate}
\solution
%\input{exemplar/10/13/3/29/main.tex}
\item A team of medical students doing their internship have to assist during surgeries
at a city hospital. The probabilities of surgeries rated as very complex, complex,
routine, simple or very simple are respectively, 0.15, 0.20, 0.31, 0.26, .08. Find
the probabilities that a particular surgery will be rated
\begin{enumerate}
	\item complex or very complex;
	\item neither very complex nor very simple;
	\item routine or complex
	\item routine or simple
\end{enumerate}
\solution
%\input{exemplar/11/16/3/8(1)/main.tex}
\item A card is selected from a pack of 52 cards.
\begin{enumerate}[label=(\alph*)]
    \item How many points are there in the sample space?
    \item Calculate the probability that the card is an ace of spades.
    \item Calculate the probability that the card is (i) an ace and (ii) black card.
\end{enumerate}
\solution
%\input{exemplar/11/16/3/4/main2.tex}
\item The probability that a non leap year selected at random will contain 53 sundays.
\\
\solution
%\input{exemplar/10/13/1/19/main.tex}
\item One of the four persons John, Rita, Aslam or Gurpreet will be promoted next
month. Consequently the sample space consists of four elementary outcomes
S = {John promoted, Rita promoted, Aslam promoted, Gurpreet promoted}
You are told that the chances of John’s promotion is same as that of Gurpreet,
Rita’s chances of promotion are twice as likely as Johns. Aslam’s chances are
four times that of John.
\begin{enumerate}
	\item Determine
	\begin{enumerate}
		\item P (John promoted)
		\item P (Rita promoted)
		\item P (Aslam promoted)
		\item P (Gurpreet promoted)
	\end{enumerate}
	\item If A = {John promoted or Gurpreet promoted}, find P (A).
\end{enumerate}
\solution
%\input{exemplar/11/16/3/10/main.tex}
\item A card is drawn from a deck of 52 cards. Find the probability of getting a king or a heart or a red card.\\
\solution
%\input{exemplar/11/16/3/15/main.tex}
\item The probability that a student will pass his examination is 0.73, the probability of
the student getting a compartment is 0.13, and the probability that the student will
either pass or get compartment is 0.96. State True or False.\\
\solution
%\input{exemplar/11/16/3/31/main.tex}
\item A card is selected from a pack of 52 cards\\
\begin{enumerate}[label=(\alph*)]
\item How many points are there in the sample space?
\item Calculate the probability that the cards is an ace of spades.
\item Calculate the probability that the card is (i) an ace (ii)black card.\\
\end{enumerate}
%\input{ncert/11/16/3/4_1/Prob_4.tex}
\item In a non-leap year, the probability of having 53 tuesdays or 53 wednesdays is\\
\solution
%\input{exemplar/11/16/3/18/main.tex}
\item There are 1000 sealed envelopes in a box, 10 of them contain a cash prize of
Rs 100 each, 100 of them contain a cash prize of Rs 50 each and 200 of them
contain a cash prize of Rs 10 each and rest do not contain any cash prize. If they
are well shuffled and an envelope is picked up out, what is the probability that it
contains no cash prize?\\
\solution
%\input{exemplar/10/13/3/34/main.tex}
\item 
A die is thrown and a card is selected at random from a deck of 52 playing cards. The probability of getting an even number on the die and a spade card.\\
\solution
%\input{exemplar/12/13/3/78/main.tex}
\item
If 4-digit numbers greater than 5,000 are randomly formed from the digits 0, 1, 3, 5, and 7, what is the probability of forming a number divisible by 5 when:
\begin{enumerate}
    \item The digits are repeated?
    \item The repetition of digits is not allowed?
\end{enumerate}
\solution
%\input{ncert/11/16/4/9/main.tex}
\item Consider the probability space $\brak{\Omega, \mathcal{G}, P}$ where $\Omega = [0,2]$ and $\mathcal{G} = \cbrak{\phi, \Omega, [0,1], (1,2]}$. Let $X$ and $Y$ be two functions on $\Omega$ defined as
\begin{align*}
    X(\omega) = 
    \begin{cases}
        1 & \text{if }\omega \in [0, 1]\\
        2 & \text{if }\omega \in (1, 2]
    \end{cases}
\end{align*}
and
\begin{align*}
    Y(\omega) = 
    \begin{cases}
        2 & \text{if }\omega \in [0, 1.5]\\
        3 & \text{if }\omega \in (1.5, 2].
    \end{cases}
\end{align*}
Then which one of the following statements is true?
\begin{enumerate}
    \item [(A)] $X$ is a random variable with respect to $\mathcal{G}$, but $Y$ is not a random variable with respect to $\mathcal{G}$.
    \item [(B)] $Y$ is a random variable with respect to $\mathcal{G}$, but $X$ is not a random variable with respect to $\mathcal{G}$.
    \item [(C)] Neither $X$ nor $Y$ is a random variable with respect to $\mathcal{G}$.
    \item [(D)] Both $X$ and $Y$ are random variables with respect to $\mathcal{G}$.
\end{enumerate} \hfill (GATE ST 2023)\\
\solution
%\input{gate/ST/2023/14/main.tex}
	\item  A die is loaded in such a way that each odd number is twice as likely to occur as
each even number. Find $P(G)$, where $G$ is the event that a number greater than
3 occurs on a single roll of the die.
\\
\solution
		%\input{exemplar/11/16/3/5/main.tex}
	\item All the jacks, queens and kings are removed from a deck of 52 playing cards. The remaining cards are well shuffled and then one card is drawn at random. Giving ace a value 1 similar value for other cards, find the probability that the card has a value 
		\begin{enumerate}
			\item 7
			\item greater than 7
			\item less than 7
		\end{enumerate}
		%\input{exemplar/10/13/3/30/main.tex}
  \item A Lot consists of 48 mobile phones of which 42 are good, 3 have only minor defects and 3 have major defects.Varnika will buy a phone if it is good but the trader will only buy a mobile if it has no major defects. One phone is selected at random from the lot. What is the probability that it is
\begin{enumerate}
	\item acceptable to Varnika?
            \item acceptable to the trader?
\end{enumerate}
\solution
	%\input{exemplar/10/13/3/40/main.tex}
 \item A student says that if you throw a die, it will show up 1 or not 1. Therefore, the probability of getting 1 and the probability of getting 'not 1' each is equal to $\frac{1}{2}$. Is this correct? Give reasons.\\
 \solution
        %\input{exemplar/10/13/2/9/main.tex}
   \item Four candidates A, B, C, D have ap-
plied for the assignment to coach a school cricket
team. If A is twice as likely to be selected as B, and
B and C are given about the same chance of being
selected, while C is twice as likely to be selected
as D, what are the probabilities that
\begin{enumerate}
\item C will be selected?
\item A will not be selected?
\end{enumerate}
	%\input{exemplar/11/16/3/9/main.tex}
 \item A bag contain 24 balls of which $x$ balls are red, $2x$ are white and $3x$ are blue. A ball is selected at random, What is the probability that it is
\begin{enumerate}[label=\alph*)]
\item not red ?
\item white ?
\end{enumerate}
%\input{exemplar/10/13/3/41/main.tex}
If the letters of the word ASSASSINATION are arranged at random. Find the Probability that
\begin{enumerate}[label=(\alph*)]
\item Four $S's$ come consecutively in the word
\item Two  $I's$ and two $N's$ come together
\item All $A's$ are not coming together
\item No two $A's$ are coming together
\end{enumerate}
%\input{exemplar/11/16/3/14/main.tex}
	\item One urn contains two black balls (labelled B1 and B2) and one white ball. A
	second urn contains one black ball and two white balls (labelled W1 and W2).
	Suppose the following experiment is performed. One of the two urns is chosen
	at random. Next a ball is randomly chosen from the urn. Then a second ball is
	chosen at random from the same urn without replacing the first ball.
	
	\begin{enumerate}
	\item What is the probability that two black balls are chosen?
	
	\item What is the probability that two balls of opposite colour are chosen?
	\end{enumerate}
	\solution
	%\input{exemplar/11/16/3/12/main1.tex}
\end{enumerate}

	\item A bag contains $5$ red balls and some blue balls. If the probability of drawing a blue ball is double that if a red ball, determine the number of blue balls in the bag. 
		\\
\solution
		%\begin{enumerate}[label=\thesection.\arabic*,ref=\thesection.\theenumi]
	\item One card is drawn from a well-shuffled deck of 52 cards. Find the probability of getting
\begin{enumerate}
\item A king of red colour 
\item A face card 
\item A red face card
\item The jack of hearts
\item A spade
\item The queen of diamonds

\end{enumerate}
\solution
		%\input{ncert/10/15/1/14/main.tex}
	\item Five cards—the ten, jack, queen, king and ace of diamonds, are well-shuffled with their face downwards. One card is then picked up at random.
\begin{enumerate}
\item
What is the probability that the card is the queen? 
\item
If the queen is drawn and put aside, what is the probability that the second card picked up is (a) an ace? (b) a queen?\\
\end{enumerate}
\solution
		%\input{ncert/10/15/1/15/defs.tex}
	\item A bag contains $5$ red balls and some blue balls. If the probability of drawing a blue ball is double that if a red ball, determine the number of blue balls in the bag. 
		\\
\solution
		%\input{ncert/10/15/2/3/defs.tex}
	\item A card is selected from a pack of 52 cards.
 \begin{enumerate}[label=(\alph*)] 
                 \item How many points are there in the sample space?
                 \item Calculate the probability that the card is an ace of spades.
                 \item Calculate the probability that the card is (i) an ace and (ii) black card.
 \end{enumerate}
\solution
		%\input{ncert/11/16/3/4/main.tex}
\item Four cards are drawn from a well-shuffled deck of 52 cards. What is the probability of obtaining 3 diamonds and one spade.
\\
\solution
		%\input{ncert/11/16/4/2/defs.tex}
\item In a certain lottery 10,000 tickets are sold and ten equal prizes are awarded. What is the probability of not getting a prize if you buy (a) one ticket (b) two tickets (c) 10 tickets ?	
\\
\solution
		%\input{ncert/11/16/4/4/defs.tex}
		%
\item 
Out of 100 students, two sections of 40 and 60 are formed. If you and your friend are among the 100 students, what is the probability that
\begin{enumerate}
\item you both enter the same section?
\item you both enter the different sections?
\end{enumerate}
\solution
		%\input{ncert/11/16/4/5/defs.tex}
	\item 
The number lock of a suitcase has 4 wheels each labelled with ten digits i.e. from 0 to 9.The lock opens with a sequence of four digits with no repeats.What is the probability of a person getting the right sequence to open the suitcase.
\\
\solution
		%\input{ncert/11/16/4/10/defs.tex}
		%
\item 
Two cards are drawn at random and without replacement from a pack of 52 playing cards. Find the probability that both the cards are black.
\\
\solution
		%\input{ncert/12/13/2/2/defs.tex}
		\item A box of oranges is inspected by examining three randomly selected oranges drawn without replacement. If all the three oranges are good, the box is approved for sale, otherwise, it is rejected. Find the probability that a box containing 15 oranges out of which 12 are good and 3 are bad ones will be approved for sale.
		\label{ncert/12/13/2/3/defs.tex}
		\item Two balls are drawn at random with replacement from a box containing 10 black and 8 red balls. Find the probability that
		\label{ncert/12/13/2/12}
\begin{enumerate}
\item both balls are red.
\item first ball is black and second is red.
\item one of them is black and other is red.
\end{enumerate}

\item In a hostel, 60\% of the students read Hindi newspaper, 40\% read English newspaper and 20\% read both Hindi and English newspapers. A student is selected at random.
		\label{ncert/12/13/2/15}
\begin{enumerate}
\item Find the probability that she reads neither Hindi nor English newspapers.
\item If she reads Hindi newspaper, find the probability that she reads English newspaper.
\item If she reads English newspaper, find the probability that she reads Hindi newspaper.\\
\end{enumerate}
\item The probability of obtaining an even prime number on each die, when a pair of dice is rolled is 
\begin{enumerate}
    \item $0$ 
    
    \item $\frac{1}{3}$ 
    
    \item $\frac{1}{12}$ 
    
    \item $\frac{1}{36}$ 
\end{enumerate}
\solution
		%\input{ncert/12/13/2/17/defs.tex}
	\item A bag contains 4 red and 4 black balls, another bag contains 2 red and 6 black balls. One of the two bags is selected at random and a ball is drawn from the bag which is found to be red. Find the probability that the ball is drawn from the first bag.
\\
\solution
		%\input{ncert/12/13/3/2/main.tex}
  \item
  Cards with numbers 2 to 101 are placed in a box. A card is selected at random.Find the probability that the card has
\begin{enumerate}[label=(\roman*)]
	\item an even number 
	\item a square number
\end{enumerate}
\solution
%\input{exemplar/10/13/3/32/main.tex}
\item
The king, queen and jack of clubs are removed from a deck of 52 playing cards and then well shuffled. Now one card is drawn at random from the remaining cards.  Determine the probability that the card is
\begin{enumerate}[label=(\roman*)]
\item a club
\item 10 of hearts
\end{enumerate}
\solution
%\input{exemplar/10/13/3/29/main.tex}
\item A team of medical students doing their internship have to assist during surgeries
at a city hospital. The probabilities of surgeries rated as very complex, complex,
routine, simple or very simple are respectively, 0.15, 0.20, 0.31, 0.26, .08. Find
the probabilities that a particular surgery will be rated
\begin{enumerate}
	\item complex or very complex;
	\item neither very complex nor very simple;
	\item routine or complex
	\item routine or simple
\end{enumerate}
\solution
%\input{exemplar/11/16/3/8(1)/main.tex}
\item A card is selected from a pack of 52 cards.
\begin{enumerate}[label=(\alph*)]
    \item How many points are there in the sample space?
    \item Calculate the probability that the card is an ace of spades.
    \item Calculate the probability that the card is (i) an ace and (ii) black card.
\end{enumerate}
\solution
%\input{exemplar/11/16/3/4/main2.tex}
\item The probability that a non leap year selected at random will contain 53 sundays.
\\
\solution
%\input{exemplar/10/13/1/19/main.tex}
\item One of the four persons John, Rita, Aslam or Gurpreet will be promoted next
month. Consequently the sample space consists of four elementary outcomes
S = {John promoted, Rita promoted, Aslam promoted, Gurpreet promoted}
You are told that the chances of John’s promotion is same as that of Gurpreet,
Rita’s chances of promotion are twice as likely as Johns. Aslam’s chances are
four times that of John.
\begin{enumerate}
	\item Determine
	\begin{enumerate}
		\item P (John promoted)
		\item P (Rita promoted)
		\item P (Aslam promoted)
		\item P (Gurpreet promoted)
	\end{enumerate}
	\item If A = {John promoted or Gurpreet promoted}, find P (A).
\end{enumerate}
\solution
%\input{exemplar/11/16/3/10/main.tex}
\item A card is drawn from a deck of 52 cards. Find the probability of getting a king or a heart or a red card.\\
\solution
%\input{exemplar/11/16/3/15/main.tex}
\item The probability that a student will pass his examination is 0.73, the probability of
the student getting a compartment is 0.13, and the probability that the student will
either pass or get compartment is 0.96. State True or False.\\
\solution
%\input{exemplar/11/16/3/31/main.tex}
\item A card is selected from a pack of 52 cards\\
\begin{enumerate}[label=(\alph*)]
\item How many points are there in the sample space?
\item Calculate the probability that the cards is an ace of spades.
\item Calculate the probability that the card is (i) an ace (ii)black card.\\
\end{enumerate}
%\input{ncert/11/16/3/4_1/Prob_4.tex}
\item In a non-leap year, the probability of having 53 tuesdays or 53 wednesdays is\\
\solution
%\input{exemplar/11/16/3/18/main.tex}
\item There are 1000 sealed envelopes in a box, 10 of them contain a cash prize of
Rs 100 each, 100 of them contain a cash prize of Rs 50 each and 200 of them
contain a cash prize of Rs 10 each and rest do not contain any cash prize. If they
are well shuffled and an envelope is picked up out, what is the probability that it
contains no cash prize?\\
\solution
%\input{exemplar/10/13/3/34/main.tex}
\item 
A die is thrown and a card is selected at random from a deck of 52 playing cards. The probability of getting an even number on the die and a spade card.\\
\solution
%\input{exemplar/12/13/3/78/main.tex}
\item
If 4-digit numbers greater than 5,000 are randomly formed from the digits 0, 1, 3, 5, and 7, what is the probability of forming a number divisible by 5 when:
\begin{enumerate}
    \item The digits are repeated?
    \item The repetition of digits is not allowed?
\end{enumerate}
\solution
%\input{ncert/11/16/4/9/main.tex}
\item Consider the probability space $\brak{\Omega, \mathcal{G}, P}$ where $\Omega = [0,2]$ and $\mathcal{G} = \cbrak{\phi, \Omega, [0,1], (1,2]}$. Let $X$ and $Y$ be two functions on $\Omega$ defined as
\begin{align*}
    X(\omega) = 
    \begin{cases}
        1 & \text{if }\omega \in [0, 1]\\
        2 & \text{if }\omega \in (1, 2]
    \end{cases}
\end{align*}
and
\begin{align*}
    Y(\omega) = 
    \begin{cases}
        2 & \text{if }\omega \in [0, 1.5]\\
        3 & \text{if }\omega \in (1.5, 2].
    \end{cases}
\end{align*}
Then which one of the following statements is true?
\begin{enumerate}
    \item [(A)] $X$ is a random variable with respect to $\mathcal{G}$, but $Y$ is not a random variable with respect to $\mathcal{G}$.
    \item [(B)] $Y$ is a random variable with respect to $\mathcal{G}$, but $X$ is not a random variable with respect to $\mathcal{G}$.
    \item [(C)] Neither $X$ nor $Y$ is a random variable with respect to $\mathcal{G}$.
    \item [(D)] Both $X$ and $Y$ are random variables with respect to $\mathcal{G}$.
\end{enumerate} \hfill (GATE ST 2023)\\
\solution
%\input{gate/ST/2023/14/main.tex}
	\item  A die is loaded in such a way that each odd number is twice as likely to occur as
each even number. Find $P(G)$, where $G$ is the event that a number greater than
3 occurs on a single roll of the die.
\\
\solution
		%\input{exemplar/11/16/3/5/main.tex}
	\item All the jacks, queens and kings are removed from a deck of 52 playing cards. The remaining cards are well shuffled and then one card is drawn at random. Giving ace a value 1 similar value for other cards, find the probability that the card has a value 
		\begin{enumerate}
			\item 7
			\item greater than 7
			\item less than 7
		\end{enumerate}
		%\input{exemplar/10/13/3/30/main.tex}
  \item A Lot consists of 48 mobile phones of which 42 are good, 3 have only minor defects and 3 have major defects.Varnika will buy a phone if it is good but the trader will only buy a mobile if it has no major defects. One phone is selected at random from the lot. What is the probability that it is
\begin{enumerate}
	\item acceptable to Varnika?
            \item acceptable to the trader?
\end{enumerate}
\solution
	%\input{exemplar/10/13/3/40/main.tex}
 \item A student says that if you throw a die, it will show up 1 or not 1. Therefore, the probability of getting 1 and the probability of getting 'not 1' each is equal to $\frac{1}{2}$. Is this correct? Give reasons.\\
 \solution
        %\input{exemplar/10/13/2/9/main.tex}
   \item Four candidates A, B, C, D have ap-
plied for the assignment to coach a school cricket
team. If A is twice as likely to be selected as B, and
B and C are given about the same chance of being
selected, while C is twice as likely to be selected
as D, what are the probabilities that
\begin{enumerate}
\item C will be selected?
\item A will not be selected?
\end{enumerate}
	%\input{exemplar/11/16/3/9/main.tex}
 \item A bag contain 24 balls of which $x$ balls are red, $2x$ are white and $3x$ are blue. A ball is selected at random, What is the probability that it is
\begin{enumerate}[label=\alph*)]
\item not red ?
\item white ?
\end{enumerate}
%\input{exemplar/10/13/3/41/main.tex}
If the letters of the word ASSASSINATION are arranged at random. Find the Probability that
\begin{enumerate}[label=(\alph*)]
\item Four $S's$ come consecutively in the word
\item Two  $I's$ and two $N's$ come together
\item All $A's$ are not coming together
\item No two $A's$ are coming together
\end{enumerate}
%\input{exemplar/11/16/3/14/main.tex}
	\item One urn contains two black balls (labelled B1 and B2) and one white ball. A
	second urn contains one black ball and two white balls (labelled W1 and W2).
	Suppose the following experiment is performed. One of the two urns is chosen
	at random. Next a ball is randomly chosen from the urn. Then a second ball is
	chosen at random from the same urn without replacing the first ball.
	
	\begin{enumerate}
	\item What is the probability that two black balls are chosen?
	
	\item What is the probability that two balls of opposite colour are chosen?
	\end{enumerate}
	\solution
	%\input{exemplar/11/16/3/12/main1.tex}
\end{enumerate}

	\item A card is selected from a pack of 52 cards.
 \begin{enumerate}[label=(\alph*)] 
                 \item How many points are there in the sample space?
                 \item Calculate the probability that the card is an ace of spades.
                 \item Calculate the probability that the card is (i) an ace and (ii) black card.
 \end{enumerate}
\solution
		%\begin{table}[H]
	\centering
\begin{tabular}{|c|c|c|}
\hline
Random variable &Value &Definition\\ \hline
\multirow{3}{*}{X} &0 &Slips of Rs 1\\
&1 &Slips of Rs 5\\
&2 &Slips of Rs 13\\ \hline
\multirow{2}{*}{Y} &0 &Box A\\
&1 &Box B\\\hline
\end{tabular}
\caption{}
\label{tab:Distribution}
\end{table}
See \tabref{tab:Distribution}.
\begin{align}
p_{Y}\brak{k}= \begin{cases} 
      \frac{1}{3} & {k=0} \\
      \frac{2}{3 }& {k=1} 
   \end{cases}
   \\
p_{Y|X}\brak{0|0} = \frac{19}{25}\, 
p_{Y|X}\brak{0|1} = \frac{6}{25}\,
p_{Y|X}\brak{1|0} = \frac{45}{50}\,
p_{Y|X}\brak{1|2} = \frac{5}{50}
\end{align}
The desired probability is the probability that a slip drawn at random is marked other than Rs 1,
\begin{align}
&=1-p_X\brak{0}\\
&= p_X(1) + p_X(2)
\end{align}
Using Bayes theorem,
\begin{align}
&= p_Y\brak{0} \times \pr{Y=0 | X=1} + p_Y\brak{1} \times \pr{Y=1|X=2}\\
&=\frac{1}{3} \times \frac{6}{25} + \frac{2}{3} \times \frac{5}{50}\\
&=\frac{11}{75}
\end{align}

\newpage

%\tableofcontents

\bigskip

\renewcommand{\thefigure}{\theenumi}
\renewcommand{\thetable}{\theenumi}
%\renewcommand{\theequation}{\theenumi}

%\begin{abstract}
%%\boldmath
%In this letter, an algorithm for evaluating the exact analytical bit error rate  (BER)  for the piecewise linear (PL) combiner for  multiple relays is presented. Previous results were available only for upto three relays. The algorithm is unique in the sense that  the actual mathematical expressions, that are prohibitively large, need not be explicitly obtained. The diversity gain due to multiple relays is shown through plots of the analytical BER, well supported by simulations. 
%
%\end{abstract}
% IEEEtran.cls defaults to using nonbold math in the Abstract.
% This preserves the distinction between vectors and scalars. However,
% if the journal you are submitting to favors bold math in the abstract,
% then you can use LaTeX's standard command \boldmath at the very start
% of the abstract to achieve this. Many IEEE journals frown on math
% in the abstract anyway.

% Note that keywords are not normally used for peerreview papers.
%\begin{IEEEkeywords}
%Cooperative diversity, decode and forward, piecewise linear
%\end{IEEEkeywords}



% For peer review papers, you can put extra information on the cover
% page as needed:
% \ifCLASSOPTIONpeerreview
% \begin{center} \bfseries EDICS Category: 3-BBND \end{center}
% \fi
%
% For peerreview papers, this IEEEtran command inserts a page break and
% creates the second title. It will be ignored for other modes.
%\IEEEpeerreviewmaketitle




\item Four cards are drawn from a well-shuffled deck of 52 cards. What is the probability of obtaining 3 diamonds and one spade.
\\
\solution
		%\begin{enumerate}[label=\thesection.\arabic*,ref=\thesection.\theenumi]
	\item One card is drawn from a well-shuffled deck of 52 cards. Find the probability of getting
\begin{enumerate}
\item A king of red colour 
\item A face card 
\item A red face card
\item The jack of hearts
\item A spade
\item The queen of diamonds

\end{enumerate}
\solution
		%\input{ncert/10/15/1/14/main.tex}
	\item Five cards—the ten, jack, queen, king and ace of diamonds, are well-shuffled with their face downwards. One card is then picked up at random.
\begin{enumerate}
\item
What is the probability that the card is the queen? 
\item
If the queen is drawn and put aside, what is the probability that the second card picked up is (a) an ace? (b) a queen?\\
\end{enumerate}
\solution
		%\input{ncert/10/15/1/15/defs.tex}
	\item A bag contains $5$ red balls and some blue balls. If the probability of drawing a blue ball is double that if a red ball, determine the number of blue balls in the bag. 
		\\
\solution
		%\input{ncert/10/15/2/3/defs.tex}
	\item A card is selected from a pack of 52 cards.
 \begin{enumerate}[label=(\alph*)] 
                 \item How many points are there in the sample space?
                 \item Calculate the probability that the card is an ace of spades.
                 \item Calculate the probability that the card is (i) an ace and (ii) black card.
 \end{enumerate}
\solution
		%\input{ncert/11/16/3/4/main.tex}
\item Four cards are drawn from a well-shuffled deck of 52 cards. What is the probability of obtaining 3 diamonds and one spade.
\\
\solution
		%\input{ncert/11/16/4/2/defs.tex}
\item In a certain lottery 10,000 tickets are sold and ten equal prizes are awarded. What is the probability of not getting a prize if you buy (a) one ticket (b) two tickets (c) 10 tickets ?	
\\
\solution
		%\input{ncert/11/16/4/4/defs.tex}
		%
\item 
Out of 100 students, two sections of 40 and 60 are formed. If you and your friend are among the 100 students, what is the probability that
\begin{enumerate}
\item you both enter the same section?
\item you both enter the different sections?
\end{enumerate}
\solution
		%\input{ncert/11/16/4/5/defs.tex}
	\item 
The number lock of a suitcase has 4 wheels each labelled with ten digits i.e. from 0 to 9.The lock opens with a sequence of four digits with no repeats.What is the probability of a person getting the right sequence to open the suitcase.
\\
\solution
		%\input{ncert/11/16/4/10/defs.tex}
		%
\item 
Two cards are drawn at random and without replacement from a pack of 52 playing cards. Find the probability that both the cards are black.
\\
\solution
		%\input{ncert/12/13/2/2/defs.tex}
		\item A box of oranges is inspected by examining three randomly selected oranges drawn without replacement. If all the three oranges are good, the box is approved for sale, otherwise, it is rejected. Find the probability that a box containing 15 oranges out of which 12 are good and 3 are bad ones will be approved for sale.
		\label{ncert/12/13/2/3/defs.tex}
		\item Two balls are drawn at random with replacement from a box containing 10 black and 8 red balls. Find the probability that
		\label{ncert/12/13/2/12}
\begin{enumerate}
\item both balls are red.
\item first ball is black and second is red.
\item one of them is black and other is red.
\end{enumerate}

\item In a hostel, 60\% of the students read Hindi newspaper, 40\% read English newspaper and 20\% read both Hindi and English newspapers. A student is selected at random.
		\label{ncert/12/13/2/15}
\begin{enumerate}
\item Find the probability that she reads neither Hindi nor English newspapers.
\item If she reads Hindi newspaper, find the probability that she reads English newspaper.
\item If she reads English newspaper, find the probability that she reads Hindi newspaper.\\
\end{enumerate}
\item The probability of obtaining an even prime number on each die, when a pair of dice is rolled is 
\begin{enumerate}
    \item $0$ 
    
    \item $\frac{1}{3}$ 
    
    \item $\frac{1}{12}$ 
    
    \item $\frac{1}{36}$ 
\end{enumerate}
\solution
		%\input{ncert/12/13/2/17/defs.tex}
	\item A bag contains 4 red and 4 black balls, another bag contains 2 red and 6 black balls. One of the two bags is selected at random and a ball is drawn from the bag which is found to be red. Find the probability that the ball is drawn from the first bag.
\\
\solution
		%\input{ncert/12/13/3/2/main.tex}
  \item
  Cards with numbers 2 to 101 are placed in a box. A card is selected at random.Find the probability that the card has
\begin{enumerate}[label=(\roman*)]
	\item an even number 
	\item a square number
\end{enumerate}
\solution
%\input{exemplar/10/13/3/32/main.tex}
\item
The king, queen and jack of clubs are removed from a deck of 52 playing cards and then well shuffled. Now one card is drawn at random from the remaining cards.  Determine the probability that the card is
\begin{enumerate}[label=(\roman*)]
\item a club
\item 10 of hearts
\end{enumerate}
\solution
%\input{exemplar/10/13/3/29/main.tex}
\item A team of medical students doing their internship have to assist during surgeries
at a city hospital. The probabilities of surgeries rated as very complex, complex,
routine, simple or very simple are respectively, 0.15, 0.20, 0.31, 0.26, .08. Find
the probabilities that a particular surgery will be rated
\begin{enumerate}
	\item complex or very complex;
	\item neither very complex nor very simple;
	\item routine or complex
	\item routine or simple
\end{enumerate}
\solution
%\input{exemplar/11/16/3/8(1)/main.tex}
\item A card is selected from a pack of 52 cards.
\begin{enumerate}[label=(\alph*)]
    \item How many points are there in the sample space?
    \item Calculate the probability that the card is an ace of spades.
    \item Calculate the probability that the card is (i) an ace and (ii) black card.
\end{enumerate}
\solution
%\input{exemplar/11/16/3/4/main2.tex}
\item The probability that a non leap year selected at random will contain 53 sundays.
\\
\solution
%\input{exemplar/10/13/1/19/main.tex}
\item One of the four persons John, Rita, Aslam or Gurpreet will be promoted next
month. Consequently the sample space consists of four elementary outcomes
S = {John promoted, Rita promoted, Aslam promoted, Gurpreet promoted}
You are told that the chances of John’s promotion is same as that of Gurpreet,
Rita’s chances of promotion are twice as likely as Johns. Aslam’s chances are
four times that of John.
\begin{enumerate}
	\item Determine
	\begin{enumerate}
		\item P (John promoted)
		\item P (Rita promoted)
		\item P (Aslam promoted)
		\item P (Gurpreet promoted)
	\end{enumerate}
	\item If A = {John promoted or Gurpreet promoted}, find P (A).
\end{enumerate}
\solution
%\input{exemplar/11/16/3/10/main.tex}
\item A card is drawn from a deck of 52 cards. Find the probability of getting a king or a heart or a red card.\\
\solution
%\input{exemplar/11/16/3/15/main.tex}
\item The probability that a student will pass his examination is 0.73, the probability of
the student getting a compartment is 0.13, and the probability that the student will
either pass or get compartment is 0.96. State True or False.\\
\solution
%\input{exemplar/11/16/3/31/main.tex}
\item A card is selected from a pack of 52 cards\\
\begin{enumerate}[label=(\alph*)]
\item How many points are there in the sample space?
\item Calculate the probability that the cards is an ace of spades.
\item Calculate the probability that the card is (i) an ace (ii)black card.\\
\end{enumerate}
%\input{ncert/11/16/3/4_1/Prob_4.tex}
\item In a non-leap year, the probability of having 53 tuesdays or 53 wednesdays is\\
\solution
%\input{exemplar/11/16/3/18/main.tex}
\item There are 1000 sealed envelopes in a box, 10 of them contain a cash prize of
Rs 100 each, 100 of them contain a cash prize of Rs 50 each and 200 of them
contain a cash prize of Rs 10 each and rest do not contain any cash prize. If they
are well shuffled and an envelope is picked up out, what is the probability that it
contains no cash prize?\\
\solution
%\input{exemplar/10/13/3/34/main.tex}
\item 
A die is thrown and a card is selected at random from a deck of 52 playing cards. The probability of getting an even number on the die and a spade card.\\
\solution
%\input{exemplar/12/13/3/78/main.tex}
\item
If 4-digit numbers greater than 5,000 are randomly formed from the digits 0, 1, 3, 5, and 7, what is the probability of forming a number divisible by 5 when:
\begin{enumerate}
    \item The digits are repeated?
    \item The repetition of digits is not allowed?
\end{enumerate}
\solution
%\input{ncert/11/16/4/9/main.tex}
\item Consider the probability space $\brak{\Omega, \mathcal{G}, P}$ where $\Omega = [0,2]$ and $\mathcal{G} = \cbrak{\phi, \Omega, [0,1], (1,2]}$. Let $X$ and $Y$ be two functions on $\Omega$ defined as
\begin{align*}
    X(\omega) = 
    \begin{cases}
        1 & \text{if }\omega \in [0, 1]\\
        2 & \text{if }\omega \in (1, 2]
    \end{cases}
\end{align*}
and
\begin{align*}
    Y(\omega) = 
    \begin{cases}
        2 & \text{if }\omega \in [0, 1.5]\\
        3 & \text{if }\omega \in (1.5, 2].
    \end{cases}
\end{align*}
Then which one of the following statements is true?
\begin{enumerate}
    \item [(A)] $X$ is a random variable with respect to $\mathcal{G}$, but $Y$ is not a random variable with respect to $\mathcal{G}$.
    \item [(B)] $Y$ is a random variable with respect to $\mathcal{G}$, but $X$ is not a random variable with respect to $\mathcal{G}$.
    \item [(C)] Neither $X$ nor $Y$ is a random variable with respect to $\mathcal{G}$.
    \item [(D)] Both $X$ and $Y$ are random variables with respect to $\mathcal{G}$.
\end{enumerate} \hfill (GATE ST 2023)\\
\solution
%\input{gate/ST/2023/14/main.tex}
	\item  A die is loaded in such a way that each odd number is twice as likely to occur as
each even number. Find $P(G)$, where $G$ is the event that a number greater than
3 occurs on a single roll of the die.
\\
\solution
		%\input{exemplar/11/16/3/5/main.tex}
	\item All the jacks, queens and kings are removed from a deck of 52 playing cards. The remaining cards are well shuffled and then one card is drawn at random. Giving ace a value 1 similar value for other cards, find the probability that the card has a value 
		\begin{enumerate}
			\item 7
			\item greater than 7
			\item less than 7
		\end{enumerate}
		%\input{exemplar/10/13/3/30/main.tex}
  \item A Lot consists of 48 mobile phones of which 42 are good, 3 have only minor defects and 3 have major defects.Varnika will buy a phone if it is good but the trader will only buy a mobile if it has no major defects. One phone is selected at random from the lot. What is the probability that it is
\begin{enumerate}
	\item acceptable to Varnika?
            \item acceptable to the trader?
\end{enumerate}
\solution
	%\input{exemplar/10/13/3/40/main.tex}
 \item A student says that if you throw a die, it will show up 1 or not 1. Therefore, the probability of getting 1 and the probability of getting 'not 1' each is equal to $\frac{1}{2}$. Is this correct? Give reasons.\\
 \solution
        %\input{exemplar/10/13/2/9/main.tex}
   \item Four candidates A, B, C, D have ap-
plied for the assignment to coach a school cricket
team. If A is twice as likely to be selected as B, and
B and C are given about the same chance of being
selected, while C is twice as likely to be selected
as D, what are the probabilities that
\begin{enumerate}
\item C will be selected?
\item A will not be selected?
\end{enumerate}
	%\input{exemplar/11/16/3/9/main.tex}
 \item A bag contain 24 balls of which $x$ balls are red, $2x$ are white and $3x$ are blue. A ball is selected at random, What is the probability that it is
\begin{enumerate}[label=\alph*)]
\item not red ?
\item white ?
\end{enumerate}
%\input{exemplar/10/13/3/41/main.tex}
If the letters of the word ASSASSINATION are arranged at random. Find the Probability that
\begin{enumerate}[label=(\alph*)]
\item Four $S's$ come consecutively in the word
\item Two  $I's$ and two $N's$ come together
\item All $A's$ are not coming together
\item No two $A's$ are coming together
\end{enumerate}
%\input{exemplar/11/16/3/14/main.tex}
	\item One urn contains two black balls (labelled B1 and B2) and one white ball. A
	second urn contains one black ball and two white balls (labelled W1 and W2).
	Suppose the following experiment is performed. One of the two urns is chosen
	at random. Next a ball is randomly chosen from the urn. Then a second ball is
	chosen at random from the same urn without replacing the first ball.
	
	\begin{enumerate}
	\item What is the probability that two black balls are chosen?
	
	\item What is the probability that two balls of opposite colour are chosen?
	\end{enumerate}
	\solution
	%\input{exemplar/11/16/3/12/main1.tex}
\end{enumerate}

\item In a certain lottery 10,000 tickets are sold and ten equal prizes are awarded. What is the probability of not getting a prize if you buy (a) one ticket (b) two tickets (c) 10 tickets ?	
\\
\solution
		%\begin{enumerate}[label=\thesection.\arabic*,ref=\thesection.\theenumi]
	\item One card is drawn from a well-shuffled deck of 52 cards. Find the probability of getting
\begin{enumerate}
\item A king of red colour 
\item A face card 
\item A red face card
\item The jack of hearts
\item A spade
\item The queen of diamonds

\end{enumerate}
\solution
		%\input{ncert/10/15/1/14/main.tex}
	\item Five cards—the ten, jack, queen, king and ace of diamonds, are well-shuffled with their face downwards. One card is then picked up at random.
\begin{enumerate}
\item
What is the probability that the card is the queen? 
\item
If the queen is drawn and put aside, what is the probability that the second card picked up is (a) an ace? (b) a queen?\\
\end{enumerate}
\solution
		%\input{ncert/10/15/1/15/defs.tex}
	\item A bag contains $5$ red balls and some blue balls. If the probability of drawing a blue ball is double that if a red ball, determine the number of blue balls in the bag. 
		\\
\solution
		%\input{ncert/10/15/2/3/defs.tex}
	\item A card is selected from a pack of 52 cards.
 \begin{enumerate}[label=(\alph*)] 
                 \item How many points are there in the sample space?
                 \item Calculate the probability that the card is an ace of spades.
                 \item Calculate the probability that the card is (i) an ace and (ii) black card.
 \end{enumerate}
\solution
		%\input{ncert/11/16/3/4/main.tex}
\item Four cards are drawn from a well-shuffled deck of 52 cards. What is the probability of obtaining 3 diamonds and one spade.
\\
\solution
		%\input{ncert/11/16/4/2/defs.tex}
\item In a certain lottery 10,000 tickets are sold and ten equal prizes are awarded. What is the probability of not getting a prize if you buy (a) one ticket (b) two tickets (c) 10 tickets ?	
\\
\solution
		%\input{ncert/11/16/4/4/defs.tex}
		%
\item 
Out of 100 students, two sections of 40 and 60 are formed. If you and your friend are among the 100 students, what is the probability that
\begin{enumerate}
\item you both enter the same section?
\item you both enter the different sections?
\end{enumerate}
\solution
		%\input{ncert/11/16/4/5/defs.tex}
	\item 
The number lock of a suitcase has 4 wheels each labelled with ten digits i.e. from 0 to 9.The lock opens with a sequence of four digits with no repeats.What is the probability of a person getting the right sequence to open the suitcase.
\\
\solution
		%\input{ncert/11/16/4/10/defs.tex}
		%
\item 
Two cards are drawn at random and without replacement from a pack of 52 playing cards. Find the probability that both the cards are black.
\\
\solution
		%\input{ncert/12/13/2/2/defs.tex}
		\item A box of oranges is inspected by examining three randomly selected oranges drawn without replacement. If all the three oranges are good, the box is approved for sale, otherwise, it is rejected. Find the probability that a box containing 15 oranges out of which 12 are good and 3 are bad ones will be approved for sale.
		\label{ncert/12/13/2/3/defs.tex}
		\item Two balls are drawn at random with replacement from a box containing 10 black and 8 red balls. Find the probability that
		\label{ncert/12/13/2/12}
\begin{enumerate}
\item both balls are red.
\item first ball is black and second is red.
\item one of them is black and other is red.
\end{enumerate}

\item In a hostel, 60\% of the students read Hindi newspaper, 40\% read English newspaper and 20\% read both Hindi and English newspapers. A student is selected at random.
		\label{ncert/12/13/2/15}
\begin{enumerate}
\item Find the probability that she reads neither Hindi nor English newspapers.
\item If she reads Hindi newspaper, find the probability that she reads English newspaper.
\item If she reads English newspaper, find the probability that she reads Hindi newspaper.\\
\end{enumerate}
\item The probability of obtaining an even prime number on each die, when a pair of dice is rolled is 
\begin{enumerate}
    \item $0$ 
    
    \item $\frac{1}{3}$ 
    
    \item $\frac{1}{12}$ 
    
    \item $\frac{1}{36}$ 
\end{enumerate}
\solution
		%\input{ncert/12/13/2/17/defs.tex}
	\item A bag contains 4 red and 4 black balls, another bag contains 2 red and 6 black balls. One of the two bags is selected at random and a ball is drawn from the bag which is found to be red. Find the probability that the ball is drawn from the first bag.
\\
\solution
		%\input{ncert/12/13/3/2/main.tex}
  \item
  Cards with numbers 2 to 101 are placed in a box. A card is selected at random.Find the probability that the card has
\begin{enumerate}[label=(\roman*)]
	\item an even number 
	\item a square number
\end{enumerate}
\solution
%\input{exemplar/10/13/3/32/main.tex}
\item
The king, queen and jack of clubs are removed from a deck of 52 playing cards and then well shuffled. Now one card is drawn at random from the remaining cards.  Determine the probability that the card is
\begin{enumerate}[label=(\roman*)]
\item a club
\item 10 of hearts
\end{enumerate}
\solution
%\input{exemplar/10/13/3/29/main.tex}
\item A team of medical students doing their internship have to assist during surgeries
at a city hospital. The probabilities of surgeries rated as very complex, complex,
routine, simple or very simple are respectively, 0.15, 0.20, 0.31, 0.26, .08. Find
the probabilities that a particular surgery will be rated
\begin{enumerate}
	\item complex or very complex;
	\item neither very complex nor very simple;
	\item routine or complex
	\item routine or simple
\end{enumerate}
\solution
%\input{exemplar/11/16/3/8(1)/main.tex}
\item A card is selected from a pack of 52 cards.
\begin{enumerate}[label=(\alph*)]
    \item How many points are there in the sample space?
    \item Calculate the probability that the card is an ace of spades.
    \item Calculate the probability that the card is (i) an ace and (ii) black card.
\end{enumerate}
\solution
%\input{exemplar/11/16/3/4/main2.tex}
\item The probability that a non leap year selected at random will contain 53 sundays.
\\
\solution
%\input{exemplar/10/13/1/19/main.tex}
\item One of the four persons John, Rita, Aslam or Gurpreet will be promoted next
month. Consequently the sample space consists of four elementary outcomes
S = {John promoted, Rita promoted, Aslam promoted, Gurpreet promoted}
You are told that the chances of John’s promotion is same as that of Gurpreet,
Rita’s chances of promotion are twice as likely as Johns. Aslam’s chances are
four times that of John.
\begin{enumerate}
	\item Determine
	\begin{enumerate}
		\item P (John promoted)
		\item P (Rita promoted)
		\item P (Aslam promoted)
		\item P (Gurpreet promoted)
	\end{enumerate}
	\item If A = {John promoted or Gurpreet promoted}, find P (A).
\end{enumerate}
\solution
%\input{exemplar/11/16/3/10/main.tex}
\item A card is drawn from a deck of 52 cards. Find the probability of getting a king or a heart or a red card.\\
\solution
%\input{exemplar/11/16/3/15/main.tex}
\item The probability that a student will pass his examination is 0.73, the probability of
the student getting a compartment is 0.13, and the probability that the student will
either pass or get compartment is 0.96. State True or False.\\
\solution
%\input{exemplar/11/16/3/31/main.tex}
\item A card is selected from a pack of 52 cards\\
\begin{enumerate}[label=(\alph*)]
\item How many points are there in the sample space?
\item Calculate the probability that the cards is an ace of spades.
\item Calculate the probability that the card is (i) an ace (ii)black card.\\
\end{enumerate}
%\input{ncert/11/16/3/4_1/Prob_4.tex}
\item In a non-leap year, the probability of having 53 tuesdays or 53 wednesdays is\\
\solution
%\input{exemplar/11/16/3/18/main.tex}
\item There are 1000 sealed envelopes in a box, 10 of them contain a cash prize of
Rs 100 each, 100 of them contain a cash prize of Rs 50 each and 200 of them
contain a cash prize of Rs 10 each and rest do not contain any cash prize. If they
are well shuffled and an envelope is picked up out, what is the probability that it
contains no cash prize?\\
\solution
%\input{exemplar/10/13/3/34/main.tex}
\item 
A die is thrown and a card is selected at random from a deck of 52 playing cards. The probability of getting an even number on the die and a spade card.\\
\solution
%\input{exemplar/12/13/3/78/main.tex}
\item
If 4-digit numbers greater than 5,000 are randomly formed from the digits 0, 1, 3, 5, and 7, what is the probability of forming a number divisible by 5 when:
\begin{enumerate}
    \item The digits are repeated?
    \item The repetition of digits is not allowed?
\end{enumerate}
\solution
%\input{ncert/11/16/4/9/main.tex}
\item Consider the probability space $\brak{\Omega, \mathcal{G}, P}$ where $\Omega = [0,2]$ and $\mathcal{G} = \cbrak{\phi, \Omega, [0,1], (1,2]}$. Let $X$ and $Y$ be two functions on $\Omega$ defined as
\begin{align*}
    X(\omega) = 
    \begin{cases}
        1 & \text{if }\omega \in [0, 1]\\
        2 & \text{if }\omega \in (1, 2]
    \end{cases}
\end{align*}
and
\begin{align*}
    Y(\omega) = 
    \begin{cases}
        2 & \text{if }\omega \in [0, 1.5]\\
        3 & \text{if }\omega \in (1.5, 2].
    \end{cases}
\end{align*}
Then which one of the following statements is true?
\begin{enumerate}
    \item [(A)] $X$ is a random variable with respect to $\mathcal{G}$, but $Y$ is not a random variable with respect to $\mathcal{G}$.
    \item [(B)] $Y$ is a random variable with respect to $\mathcal{G}$, but $X$ is not a random variable with respect to $\mathcal{G}$.
    \item [(C)] Neither $X$ nor $Y$ is a random variable with respect to $\mathcal{G}$.
    \item [(D)] Both $X$ and $Y$ are random variables with respect to $\mathcal{G}$.
\end{enumerate} \hfill (GATE ST 2023)\\
\solution
%\input{gate/ST/2023/14/main.tex}
	\item  A die is loaded in such a way that each odd number is twice as likely to occur as
each even number. Find $P(G)$, where $G$ is the event that a number greater than
3 occurs on a single roll of the die.
\\
\solution
		%\input{exemplar/11/16/3/5/main.tex}
	\item All the jacks, queens and kings are removed from a deck of 52 playing cards. The remaining cards are well shuffled and then one card is drawn at random. Giving ace a value 1 similar value for other cards, find the probability that the card has a value 
		\begin{enumerate}
			\item 7
			\item greater than 7
			\item less than 7
		\end{enumerate}
		%\input{exemplar/10/13/3/30/main.tex}
  \item A Lot consists of 48 mobile phones of which 42 are good, 3 have only minor defects and 3 have major defects.Varnika will buy a phone if it is good but the trader will only buy a mobile if it has no major defects. One phone is selected at random from the lot. What is the probability that it is
\begin{enumerate}
	\item acceptable to Varnika?
            \item acceptable to the trader?
\end{enumerate}
\solution
	%\input{exemplar/10/13/3/40/main.tex}
 \item A student says that if you throw a die, it will show up 1 or not 1. Therefore, the probability of getting 1 and the probability of getting 'not 1' each is equal to $\frac{1}{2}$. Is this correct? Give reasons.\\
 \solution
        %\input{exemplar/10/13/2/9/main.tex}
   \item Four candidates A, B, C, D have ap-
plied for the assignment to coach a school cricket
team. If A is twice as likely to be selected as B, and
B and C are given about the same chance of being
selected, while C is twice as likely to be selected
as D, what are the probabilities that
\begin{enumerate}
\item C will be selected?
\item A will not be selected?
\end{enumerate}
	%\input{exemplar/11/16/3/9/main.tex}
 \item A bag contain 24 balls of which $x$ balls are red, $2x$ are white and $3x$ are blue. A ball is selected at random, What is the probability that it is
\begin{enumerate}[label=\alph*)]
\item not red ?
\item white ?
\end{enumerate}
%\input{exemplar/10/13/3/41/main.tex}
If the letters of the word ASSASSINATION are arranged at random. Find the Probability that
\begin{enumerate}[label=(\alph*)]
\item Four $S's$ come consecutively in the word
\item Two  $I's$ and two $N's$ come together
\item All $A's$ are not coming together
\item No two $A's$ are coming together
\end{enumerate}
%\input{exemplar/11/16/3/14/main.tex}
	\item One urn contains two black balls (labelled B1 and B2) and one white ball. A
	second urn contains one black ball and two white balls (labelled W1 and W2).
	Suppose the following experiment is performed. One of the two urns is chosen
	at random. Next a ball is randomly chosen from the urn. Then a second ball is
	chosen at random from the same urn without replacing the first ball.
	
	\begin{enumerate}
	\item What is the probability that two black balls are chosen?
	
	\item What is the probability that two balls of opposite colour are chosen?
	\end{enumerate}
	\solution
	%\input{exemplar/11/16/3/12/main1.tex}
\end{enumerate}

		%
\item 
Out of 100 students, two sections of 40 and 60 are formed. If you and your friend are among the 100 students, what is the probability that
\begin{enumerate}
\item you both enter the same section?
\item you both enter the different sections?
\end{enumerate}
\solution
		%\begin{enumerate}[label=\thesection.\arabic*,ref=\thesection.\theenumi]
	\item One card is drawn from a well-shuffled deck of 52 cards. Find the probability of getting
\begin{enumerate}
\item A king of red colour 
\item A face card 
\item A red face card
\item The jack of hearts
\item A spade
\item The queen of diamonds

\end{enumerate}
\solution
		%\input{ncert/10/15/1/14/main.tex}
	\item Five cards—the ten, jack, queen, king and ace of diamonds, are well-shuffled with their face downwards. One card is then picked up at random.
\begin{enumerate}
\item
What is the probability that the card is the queen? 
\item
If the queen is drawn and put aside, what is the probability that the second card picked up is (a) an ace? (b) a queen?\\
\end{enumerate}
\solution
		%\input{ncert/10/15/1/15/defs.tex}
	\item A bag contains $5$ red balls and some blue balls. If the probability of drawing a blue ball is double that if a red ball, determine the number of blue balls in the bag. 
		\\
\solution
		%\input{ncert/10/15/2/3/defs.tex}
	\item A card is selected from a pack of 52 cards.
 \begin{enumerate}[label=(\alph*)] 
                 \item How many points are there in the sample space?
                 \item Calculate the probability that the card is an ace of spades.
                 \item Calculate the probability that the card is (i) an ace and (ii) black card.
 \end{enumerate}
\solution
		%\input{ncert/11/16/3/4/main.tex}
\item Four cards are drawn from a well-shuffled deck of 52 cards. What is the probability of obtaining 3 diamonds and one spade.
\\
\solution
		%\input{ncert/11/16/4/2/defs.tex}
\item In a certain lottery 10,000 tickets are sold and ten equal prizes are awarded. What is the probability of not getting a prize if you buy (a) one ticket (b) two tickets (c) 10 tickets ?	
\\
\solution
		%\input{ncert/11/16/4/4/defs.tex}
		%
\item 
Out of 100 students, two sections of 40 and 60 are formed. If you and your friend are among the 100 students, what is the probability that
\begin{enumerate}
\item you both enter the same section?
\item you both enter the different sections?
\end{enumerate}
\solution
		%\input{ncert/11/16/4/5/defs.tex}
	\item 
The number lock of a suitcase has 4 wheels each labelled with ten digits i.e. from 0 to 9.The lock opens with a sequence of four digits with no repeats.What is the probability of a person getting the right sequence to open the suitcase.
\\
\solution
		%\input{ncert/11/16/4/10/defs.tex}
		%
\item 
Two cards are drawn at random and without replacement from a pack of 52 playing cards. Find the probability that both the cards are black.
\\
\solution
		%\input{ncert/12/13/2/2/defs.tex}
		\item A box of oranges is inspected by examining three randomly selected oranges drawn without replacement. If all the three oranges are good, the box is approved for sale, otherwise, it is rejected. Find the probability that a box containing 15 oranges out of which 12 are good and 3 are bad ones will be approved for sale.
		\label{ncert/12/13/2/3/defs.tex}
		\item Two balls are drawn at random with replacement from a box containing 10 black and 8 red balls. Find the probability that
		\label{ncert/12/13/2/12}
\begin{enumerate}
\item both balls are red.
\item first ball is black and second is red.
\item one of them is black and other is red.
\end{enumerate}

\item In a hostel, 60\% of the students read Hindi newspaper, 40\% read English newspaper and 20\% read both Hindi and English newspapers. A student is selected at random.
		\label{ncert/12/13/2/15}
\begin{enumerate}
\item Find the probability that she reads neither Hindi nor English newspapers.
\item If she reads Hindi newspaper, find the probability that she reads English newspaper.
\item If she reads English newspaper, find the probability that she reads Hindi newspaper.\\
\end{enumerate}
\item The probability of obtaining an even prime number on each die, when a pair of dice is rolled is 
\begin{enumerate}
    \item $0$ 
    
    \item $\frac{1}{3}$ 
    
    \item $\frac{1}{12}$ 
    
    \item $\frac{1}{36}$ 
\end{enumerate}
\solution
		%\input{ncert/12/13/2/17/defs.tex}
	\item A bag contains 4 red and 4 black balls, another bag contains 2 red and 6 black balls. One of the two bags is selected at random and a ball is drawn from the bag which is found to be red. Find the probability that the ball is drawn from the first bag.
\\
\solution
		%\input{ncert/12/13/3/2/main.tex}
  \item
  Cards with numbers 2 to 101 are placed in a box. A card is selected at random.Find the probability that the card has
\begin{enumerate}[label=(\roman*)]
	\item an even number 
	\item a square number
\end{enumerate}
\solution
%\input{exemplar/10/13/3/32/main.tex}
\item
The king, queen and jack of clubs are removed from a deck of 52 playing cards and then well shuffled. Now one card is drawn at random from the remaining cards.  Determine the probability that the card is
\begin{enumerate}[label=(\roman*)]
\item a club
\item 10 of hearts
\end{enumerate}
\solution
%\input{exemplar/10/13/3/29/main.tex}
\item A team of medical students doing their internship have to assist during surgeries
at a city hospital. The probabilities of surgeries rated as very complex, complex,
routine, simple or very simple are respectively, 0.15, 0.20, 0.31, 0.26, .08. Find
the probabilities that a particular surgery will be rated
\begin{enumerate}
	\item complex or very complex;
	\item neither very complex nor very simple;
	\item routine or complex
	\item routine or simple
\end{enumerate}
\solution
%\input{exemplar/11/16/3/8(1)/main.tex}
\item A card is selected from a pack of 52 cards.
\begin{enumerate}[label=(\alph*)]
    \item How many points are there in the sample space?
    \item Calculate the probability that the card is an ace of spades.
    \item Calculate the probability that the card is (i) an ace and (ii) black card.
\end{enumerate}
\solution
%\input{exemplar/11/16/3/4/main2.tex}
\item The probability that a non leap year selected at random will contain 53 sundays.
\\
\solution
%\input{exemplar/10/13/1/19/main.tex}
\item One of the four persons John, Rita, Aslam or Gurpreet will be promoted next
month. Consequently the sample space consists of four elementary outcomes
S = {John promoted, Rita promoted, Aslam promoted, Gurpreet promoted}
You are told that the chances of John’s promotion is same as that of Gurpreet,
Rita’s chances of promotion are twice as likely as Johns. Aslam’s chances are
four times that of John.
\begin{enumerate}
	\item Determine
	\begin{enumerate}
		\item P (John promoted)
		\item P (Rita promoted)
		\item P (Aslam promoted)
		\item P (Gurpreet promoted)
	\end{enumerate}
	\item If A = {John promoted or Gurpreet promoted}, find P (A).
\end{enumerate}
\solution
%\input{exemplar/11/16/3/10/main.tex}
\item A card is drawn from a deck of 52 cards. Find the probability of getting a king or a heart or a red card.\\
\solution
%\input{exemplar/11/16/3/15/main.tex}
\item The probability that a student will pass his examination is 0.73, the probability of
the student getting a compartment is 0.13, and the probability that the student will
either pass or get compartment is 0.96. State True or False.\\
\solution
%\input{exemplar/11/16/3/31/main.tex}
\item A card is selected from a pack of 52 cards\\
\begin{enumerate}[label=(\alph*)]
\item How many points are there in the sample space?
\item Calculate the probability that the cards is an ace of spades.
\item Calculate the probability that the card is (i) an ace (ii)black card.\\
\end{enumerate}
%\input{ncert/11/16/3/4_1/Prob_4.tex}
\item In a non-leap year, the probability of having 53 tuesdays or 53 wednesdays is\\
\solution
%\input{exemplar/11/16/3/18/main.tex}
\item There are 1000 sealed envelopes in a box, 10 of them contain a cash prize of
Rs 100 each, 100 of them contain a cash prize of Rs 50 each and 200 of them
contain a cash prize of Rs 10 each and rest do not contain any cash prize. If they
are well shuffled and an envelope is picked up out, what is the probability that it
contains no cash prize?\\
\solution
%\input{exemplar/10/13/3/34/main.tex}
\item 
A die is thrown and a card is selected at random from a deck of 52 playing cards. The probability of getting an even number on the die and a spade card.\\
\solution
%\input{exemplar/12/13/3/78/main.tex}
\item
If 4-digit numbers greater than 5,000 are randomly formed from the digits 0, 1, 3, 5, and 7, what is the probability of forming a number divisible by 5 when:
\begin{enumerate}
    \item The digits are repeated?
    \item The repetition of digits is not allowed?
\end{enumerate}
\solution
%\input{ncert/11/16/4/9/main.tex}
\item Consider the probability space $\brak{\Omega, \mathcal{G}, P}$ where $\Omega = [0,2]$ and $\mathcal{G} = \cbrak{\phi, \Omega, [0,1], (1,2]}$. Let $X$ and $Y$ be two functions on $\Omega$ defined as
\begin{align*}
    X(\omega) = 
    \begin{cases}
        1 & \text{if }\omega \in [0, 1]\\
        2 & \text{if }\omega \in (1, 2]
    \end{cases}
\end{align*}
and
\begin{align*}
    Y(\omega) = 
    \begin{cases}
        2 & \text{if }\omega \in [0, 1.5]\\
        3 & \text{if }\omega \in (1.5, 2].
    \end{cases}
\end{align*}
Then which one of the following statements is true?
\begin{enumerate}
    \item [(A)] $X$ is a random variable with respect to $\mathcal{G}$, but $Y$ is not a random variable with respect to $\mathcal{G}$.
    \item [(B)] $Y$ is a random variable with respect to $\mathcal{G}$, but $X$ is not a random variable with respect to $\mathcal{G}$.
    \item [(C)] Neither $X$ nor $Y$ is a random variable with respect to $\mathcal{G}$.
    \item [(D)] Both $X$ and $Y$ are random variables with respect to $\mathcal{G}$.
\end{enumerate} \hfill (GATE ST 2023)\\
\solution
%\input{gate/ST/2023/14/main.tex}
	\item  A die is loaded in such a way that each odd number is twice as likely to occur as
each even number. Find $P(G)$, where $G$ is the event that a number greater than
3 occurs on a single roll of the die.
\\
\solution
		%\input{exemplar/11/16/3/5/main.tex}
	\item All the jacks, queens and kings are removed from a deck of 52 playing cards. The remaining cards are well shuffled and then one card is drawn at random. Giving ace a value 1 similar value for other cards, find the probability that the card has a value 
		\begin{enumerate}
			\item 7
			\item greater than 7
			\item less than 7
		\end{enumerate}
		%\input{exemplar/10/13/3/30/main.tex}
  \item A Lot consists of 48 mobile phones of which 42 are good, 3 have only minor defects and 3 have major defects.Varnika will buy a phone if it is good but the trader will only buy a mobile if it has no major defects. One phone is selected at random from the lot. What is the probability that it is
\begin{enumerate}
	\item acceptable to Varnika?
            \item acceptable to the trader?
\end{enumerate}
\solution
	%\input{exemplar/10/13/3/40/main.tex}
 \item A student says that if you throw a die, it will show up 1 or not 1. Therefore, the probability of getting 1 and the probability of getting 'not 1' each is equal to $\frac{1}{2}$. Is this correct? Give reasons.\\
 \solution
        %\input{exemplar/10/13/2/9/main.tex}
   \item Four candidates A, B, C, D have ap-
plied for the assignment to coach a school cricket
team. If A is twice as likely to be selected as B, and
B and C are given about the same chance of being
selected, while C is twice as likely to be selected
as D, what are the probabilities that
\begin{enumerate}
\item C will be selected?
\item A will not be selected?
\end{enumerate}
	%\input{exemplar/11/16/3/9/main.tex}
 \item A bag contain 24 balls of which $x$ balls are red, $2x$ are white and $3x$ are blue. A ball is selected at random, What is the probability that it is
\begin{enumerate}[label=\alph*)]
\item not red ?
\item white ?
\end{enumerate}
%\input{exemplar/10/13/3/41/main.tex}
If the letters of the word ASSASSINATION are arranged at random. Find the Probability that
\begin{enumerate}[label=(\alph*)]
\item Four $S's$ come consecutively in the word
\item Two  $I's$ and two $N's$ come together
\item All $A's$ are not coming together
\item No two $A's$ are coming together
\end{enumerate}
%\input{exemplar/11/16/3/14/main.tex}
	\item One urn contains two black balls (labelled B1 and B2) and one white ball. A
	second urn contains one black ball and two white balls (labelled W1 and W2).
	Suppose the following experiment is performed. One of the two urns is chosen
	at random. Next a ball is randomly chosen from the urn. Then a second ball is
	chosen at random from the same urn without replacing the first ball.
	
	\begin{enumerate}
	\item What is the probability that two black balls are chosen?
	
	\item What is the probability that two balls of opposite colour are chosen?
	\end{enumerate}
	\solution
	%\input{exemplar/11/16/3/12/main1.tex}
\end{enumerate}

	\item 
The number lock of a suitcase has 4 wheels each labelled with ten digits i.e. from 0 to 9.The lock opens with a sequence of four digits with no repeats.What is the probability of a person getting the right sequence to open the suitcase.
\\
\solution
		%\begin{enumerate}[label=\thesection.\arabic*,ref=\thesection.\theenumi]
	\item One card is drawn from a well-shuffled deck of 52 cards. Find the probability of getting
\begin{enumerate}
\item A king of red colour 
\item A face card 
\item A red face card
\item The jack of hearts
\item A spade
\item The queen of diamonds

\end{enumerate}
\solution
		%\input{ncert/10/15/1/14/main.tex}
	\item Five cards—the ten, jack, queen, king and ace of diamonds, are well-shuffled with their face downwards. One card is then picked up at random.
\begin{enumerate}
\item
What is the probability that the card is the queen? 
\item
If the queen is drawn and put aside, what is the probability that the second card picked up is (a) an ace? (b) a queen?\\
\end{enumerate}
\solution
		%\input{ncert/10/15/1/15/defs.tex}
	\item A bag contains $5$ red balls and some blue balls. If the probability of drawing a blue ball is double that if a red ball, determine the number of blue balls in the bag. 
		\\
\solution
		%\input{ncert/10/15/2/3/defs.tex}
	\item A card is selected from a pack of 52 cards.
 \begin{enumerate}[label=(\alph*)] 
                 \item How many points are there in the sample space?
                 \item Calculate the probability that the card is an ace of spades.
                 \item Calculate the probability that the card is (i) an ace and (ii) black card.
 \end{enumerate}
\solution
		%\input{ncert/11/16/3/4/main.tex}
\item Four cards are drawn from a well-shuffled deck of 52 cards. What is the probability of obtaining 3 diamonds and one spade.
\\
\solution
		%\input{ncert/11/16/4/2/defs.tex}
\item In a certain lottery 10,000 tickets are sold and ten equal prizes are awarded. What is the probability of not getting a prize if you buy (a) one ticket (b) two tickets (c) 10 tickets ?	
\\
\solution
		%\input{ncert/11/16/4/4/defs.tex}
		%
\item 
Out of 100 students, two sections of 40 and 60 are formed. If you and your friend are among the 100 students, what is the probability that
\begin{enumerate}
\item you both enter the same section?
\item you both enter the different sections?
\end{enumerate}
\solution
		%\input{ncert/11/16/4/5/defs.tex}
	\item 
The number lock of a suitcase has 4 wheels each labelled with ten digits i.e. from 0 to 9.The lock opens with a sequence of four digits with no repeats.What is the probability of a person getting the right sequence to open the suitcase.
\\
\solution
		%\input{ncert/11/16/4/10/defs.tex}
		%
\item 
Two cards are drawn at random and without replacement from a pack of 52 playing cards. Find the probability that both the cards are black.
\\
\solution
		%\input{ncert/12/13/2/2/defs.tex}
		\item A box of oranges is inspected by examining three randomly selected oranges drawn without replacement. If all the three oranges are good, the box is approved for sale, otherwise, it is rejected. Find the probability that a box containing 15 oranges out of which 12 are good and 3 are bad ones will be approved for sale.
		\label{ncert/12/13/2/3/defs.tex}
		\item Two balls are drawn at random with replacement from a box containing 10 black and 8 red balls. Find the probability that
		\label{ncert/12/13/2/12}
\begin{enumerate}
\item both balls are red.
\item first ball is black and second is red.
\item one of them is black and other is red.
\end{enumerate}

\item In a hostel, 60\% of the students read Hindi newspaper, 40\% read English newspaper and 20\% read both Hindi and English newspapers. A student is selected at random.
		\label{ncert/12/13/2/15}
\begin{enumerate}
\item Find the probability that she reads neither Hindi nor English newspapers.
\item If she reads Hindi newspaper, find the probability that she reads English newspaper.
\item If she reads English newspaper, find the probability that she reads Hindi newspaper.\\
\end{enumerate}
\item The probability of obtaining an even prime number on each die, when a pair of dice is rolled is 
\begin{enumerate}
    \item $0$ 
    
    \item $\frac{1}{3}$ 
    
    \item $\frac{1}{12}$ 
    
    \item $\frac{1}{36}$ 
\end{enumerate}
\solution
		%\input{ncert/12/13/2/17/defs.tex}
	\item A bag contains 4 red and 4 black balls, another bag contains 2 red and 6 black balls. One of the two bags is selected at random and a ball is drawn from the bag which is found to be red. Find the probability that the ball is drawn from the first bag.
\\
\solution
		%\input{ncert/12/13/3/2/main.tex}
  \item
  Cards with numbers 2 to 101 are placed in a box. A card is selected at random.Find the probability that the card has
\begin{enumerate}[label=(\roman*)]
	\item an even number 
	\item a square number
\end{enumerate}
\solution
%\input{exemplar/10/13/3/32/main.tex}
\item
The king, queen and jack of clubs are removed from a deck of 52 playing cards and then well shuffled. Now one card is drawn at random from the remaining cards.  Determine the probability that the card is
\begin{enumerate}[label=(\roman*)]
\item a club
\item 10 of hearts
\end{enumerate}
\solution
%\input{exemplar/10/13/3/29/main.tex}
\item A team of medical students doing their internship have to assist during surgeries
at a city hospital. The probabilities of surgeries rated as very complex, complex,
routine, simple or very simple are respectively, 0.15, 0.20, 0.31, 0.26, .08. Find
the probabilities that a particular surgery will be rated
\begin{enumerate}
	\item complex or very complex;
	\item neither very complex nor very simple;
	\item routine or complex
	\item routine or simple
\end{enumerate}
\solution
%\input{exemplar/11/16/3/8(1)/main.tex}
\item A card is selected from a pack of 52 cards.
\begin{enumerate}[label=(\alph*)]
    \item How many points are there in the sample space?
    \item Calculate the probability that the card is an ace of spades.
    \item Calculate the probability that the card is (i) an ace and (ii) black card.
\end{enumerate}
\solution
%\input{exemplar/11/16/3/4/main2.tex}
\item The probability that a non leap year selected at random will contain 53 sundays.
\\
\solution
%\input{exemplar/10/13/1/19/main.tex}
\item One of the four persons John, Rita, Aslam or Gurpreet will be promoted next
month. Consequently the sample space consists of four elementary outcomes
S = {John promoted, Rita promoted, Aslam promoted, Gurpreet promoted}
You are told that the chances of John’s promotion is same as that of Gurpreet,
Rita’s chances of promotion are twice as likely as Johns. Aslam’s chances are
four times that of John.
\begin{enumerate}
	\item Determine
	\begin{enumerate}
		\item P (John promoted)
		\item P (Rita promoted)
		\item P (Aslam promoted)
		\item P (Gurpreet promoted)
	\end{enumerate}
	\item If A = {John promoted or Gurpreet promoted}, find P (A).
\end{enumerate}
\solution
%\input{exemplar/11/16/3/10/main.tex}
\item A card is drawn from a deck of 52 cards. Find the probability of getting a king or a heart or a red card.\\
\solution
%\input{exemplar/11/16/3/15/main.tex}
\item The probability that a student will pass his examination is 0.73, the probability of
the student getting a compartment is 0.13, and the probability that the student will
either pass or get compartment is 0.96. State True or False.\\
\solution
%\input{exemplar/11/16/3/31/main.tex}
\item A card is selected from a pack of 52 cards\\
\begin{enumerate}[label=(\alph*)]
\item How many points are there in the sample space?
\item Calculate the probability that the cards is an ace of spades.
\item Calculate the probability that the card is (i) an ace (ii)black card.\\
\end{enumerate}
%\input{ncert/11/16/3/4_1/Prob_4.tex}
\item In a non-leap year, the probability of having 53 tuesdays or 53 wednesdays is\\
\solution
%\input{exemplar/11/16/3/18/main.tex}
\item There are 1000 sealed envelopes in a box, 10 of them contain a cash prize of
Rs 100 each, 100 of them contain a cash prize of Rs 50 each and 200 of them
contain a cash prize of Rs 10 each and rest do not contain any cash prize. If they
are well shuffled and an envelope is picked up out, what is the probability that it
contains no cash prize?\\
\solution
%\input{exemplar/10/13/3/34/main.tex}
\item 
A die is thrown and a card is selected at random from a deck of 52 playing cards. The probability of getting an even number on the die and a spade card.\\
\solution
%\input{exemplar/12/13/3/78/main.tex}
\item
If 4-digit numbers greater than 5,000 are randomly formed from the digits 0, 1, 3, 5, and 7, what is the probability of forming a number divisible by 5 when:
\begin{enumerate}
    \item The digits are repeated?
    \item The repetition of digits is not allowed?
\end{enumerate}
\solution
%\input{ncert/11/16/4/9/main.tex}
\item Consider the probability space $\brak{\Omega, \mathcal{G}, P}$ where $\Omega = [0,2]$ and $\mathcal{G} = \cbrak{\phi, \Omega, [0,1], (1,2]}$. Let $X$ and $Y$ be two functions on $\Omega$ defined as
\begin{align*}
    X(\omega) = 
    \begin{cases}
        1 & \text{if }\omega \in [0, 1]\\
        2 & \text{if }\omega \in (1, 2]
    \end{cases}
\end{align*}
and
\begin{align*}
    Y(\omega) = 
    \begin{cases}
        2 & \text{if }\omega \in [0, 1.5]\\
        3 & \text{if }\omega \in (1.5, 2].
    \end{cases}
\end{align*}
Then which one of the following statements is true?
\begin{enumerate}
    \item [(A)] $X$ is a random variable with respect to $\mathcal{G}$, but $Y$ is not a random variable with respect to $\mathcal{G}$.
    \item [(B)] $Y$ is a random variable with respect to $\mathcal{G}$, but $X$ is not a random variable with respect to $\mathcal{G}$.
    \item [(C)] Neither $X$ nor $Y$ is a random variable with respect to $\mathcal{G}$.
    \item [(D)] Both $X$ and $Y$ are random variables with respect to $\mathcal{G}$.
\end{enumerate} \hfill (GATE ST 2023)\\
\solution
%\input{gate/ST/2023/14/main.tex}
	\item  A die is loaded in such a way that each odd number is twice as likely to occur as
each even number. Find $P(G)$, where $G$ is the event that a number greater than
3 occurs on a single roll of the die.
\\
\solution
		%\input{exemplar/11/16/3/5/main.tex}
	\item All the jacks, queens and kings are removed from a deck of 52 playing cards. The remaining cards are well shuffled and then one card is drawn at random. Giving ace a value 1 similar value for other cards, find the probability that the card has a value 
		\begin{enumerate}
			\item 7
			\item greater than 7
			\item less than 7
		\end{enumerate}
		%\input{exemplar/10/13/3/30/main.tex}
  \item A Lot consists of 48 mobile phones of which 42 are good, 3 have only minor defects and 3 have major defects.Varnika will buy a phone if it is good but the trader will only buy a mobile if it has no major defects. One phone is selected at random from the lot. What is the probability that it is
\begin{enumerate}
	\item acceptable to Varnika?
            \item acceptable to the trader?
\end{enumerate}
\solution
	%\input{exemplar/10/13/3/40/main.tex}
 \item A student says that if you throw a die, it will show up 1 or not 1. Therefore, the probability of getting 1 and the probability of getting 'not 1' each is equal to $\frac{1}{2}$. Is this correct? Give reasons.\\
 \solution
        %\input{exemplar/10/13/2/9/main.tex}
   \item Four candidates A, B, C, D have ap-
plied for the assignment to coach a school cricket
team. If A is twice as likely to be selected as B, and
B and C are given about the same chance of being
selected, while C is twice as likely to be selected
as D, what are the probabilities that
\begin{enumerate}
\item C will be selected?
\item A will not be selected?
\end{enumerate}
	%\input{exemplar/11/16/3/9/main.tex}
 \item A bag contain 24 balls of which $x$ balls are red, $2x$ are white and $3x$ are blue. A ball is selected at random, What is the probability that it is
\begin{enumerate}[label=\alph*)]
\item not red ?
\item white ?
\end{enumerate}
%\input{exemplar/10/13/3/41/main.tex}
If the letters of the word ASSASSINATION are arranged at random. Find the Probability that
\begin{enumerate}[label=(\alph*)]
\item Four $S's$ come consecutively in the word
\item Two  $I's$ and two $N's$ come together
\item All $A's$ are not coming together
\item No two $A's$ are coming together
\end{enumerate}
%\input{exemplar/11/16/3/14/main.tex}
	\item One urn contains two black balls (labelled B1 and B2) and one white ball. A
	second urn contains one black ball and two white balls (labelled W1 and W2).
	Suppose the following experiment is performed. One of the two urns is chosen
	at random. Next a ball is randomly chosen from the urn. Then a second ball is
	chosen at random from the same urn without replacing the first ball.
	
	\begin{enumerate}
	\item What is the probability that two black balls are chosen?
	
	\item What is the probability that two balls of opposite colour are chosen?
	\end{enumerate}
	\solution
	%\input{exemplar/11/16/3/12/main1.tex}
\end{enumerate}

		%
\item 
Two cards are drawn at random and without replacement from a pack of 52 playing cards. Find the probability that both the cards are black.
\\
\solution
		%\begin{enumerate}[label=\thesection.\arabic*,ref=\thesection.\theenumi]
	\item One card is drawn from a well-shuffled deck of 52 cards. Find the probability of getting
\begin{enumerate}
\item A king of red colour 
\item A face card 
\item A red face card
\item The jack of hearts
\item A spade
\item The queen of diamonds

\end{enumerate}
\solution
		%\input{ncert/10/15/1/14/main.tex}
	\item Five cards—the ten, jack, queen, king and ace of diamonds, are well-shuffled with their face downwards. One card is then picked up at random.
\begin{enumerate}
\item
What is the probability that the card is the queen? 
\item
If the queen is drawn and put aside, what is the probability that the second card picked up is (a) an ace? (b) a queen?\\
\end{enumerate}
\solution
		%\input{ncert/10/15/1/15/defs.tex}
	\item A bag contains $5$ red balls and some blue balls. If the probability of drawing a blue ball is double that if a red ball, determine the number of blue balls in the bag. 
		\\
\solution
		%\input{ncert/10/15/2/3/defs.tex}
	\item A card is selected from a pack of 52 cards.
 \begin{enumerate}[label=(\alph*)] 
                 \item How many points are there in the sample space?
                 \item Calculate the probability that the card is an ace of spades.
                 \item Calculate the probability that the card is (i) an ace and (ii) black card.
 \end{enumerate}
\solution
		%\input{ncert/11/16/3/4/main.tex}
\item Four cards are drawn from a well-shuffled deck of 52 cards. What is the probability of obtaining 3 diamonds and one spade.
\\
\solution
		%\input{ncert/11/16/4/2/defs.tex}
\item In a certain lottery 10,000 tickets are sold and ten equal prizes are awarded. What is the probability of not getting a prize if you buy (a) one ticket (b) two tickets (c) 10 tickets ?	
\\
\solution
		%\input{ncert/11/16/4/4/defs.tex}
		%
\item 
Out of 100 students, two sections of 40 and 60 are formed. If you and your friend are among the 100 students, what is the probability that
\begin{enumerate}
\item you both enter the same section?
\item you both enter the different sections?
\end{enumerate}
\solution
		%\input{ncert/11/16/4/5/defs.tex}
	\item 
The number lock of a suitcase has 4 wheels each labelled with ten digits i.e. from 0 to 9.The lock opens with a sequence of four digits with no repeats.What is the probability of a person getting the right sequence to open the suitcase.
\\
\solution
		%\input{ncert/11/16/4/10/defs.tex}
		%
\item 
Two cards are drawn at random and without replacement from a pack of 52 playing cards. Find the probability that both the cards are black.
\\
\solution
		%\input{ncert/12/13/2/2/defs.tex}
		\item A box of oranges is inspected by examining three randomly selected oranges drawn without replacement. If all the three oranges are good, the box is approved for sale, otherwise, it is rejected. Find the probability that a box containing 15 oranges out of which 12 are good and 3 are bad ones will be approved for sale.
		\label{ncert/12/13/2/3/defs.tex}
		\item Two balls are drawn at random with replacement from a box containing 10 black and 8 red balls. Find the probability that
		\label{ncert/12/13/2/12}
\begin{enumerate}
\item both balls are red.
\item first ball is black and second is red.
\item one of them is black and other is red.
\end{enumerate}

\item In a hostel, 60\% of the students read Hindi newspaper, 40\% read English newspaper and 20\% read both Hindi and English newspapers. A student is selected at random.
		\label{ncert/12/13/2/15}
\begin{enumerate}
\item Find the probability that she reads neither Hindi nor English newspapers.
\item If she reads Hindi newspaper, find the probability that she reads English newspaper.
\item If she reads English newspaper, find the probability that she reads Hindi newspaper.\\
\end{enumerate}
\item The probability of obtaining an even prime number on each die, when a pair of dice is rolled is 
\begin{enumerate}
    \item $0$ 
    
    \item $\frac{1}{3}$ 
    
    \item $\frac{1}{12}$ 
    
    \item $\frac{1}{36}$ 
\end{enumerate}
\solution
		%\input{ncert/12/13/2/17/defs.tex}
	\item A bag contains 4 red and 4 black balls, another bag contains 2 red and 6 black balls. One of the two bags is selected at random and a ball is drawn from the bag which is found to be red. Find the probability that the ball is drawn from the first bag.
\\
\solution
		%\input{ncert/12/13/3/2/main.tex}
  \item
  Cards with numbers 2 to 101 are placed in a box. A card is selected at random.Find the probability that the card has
\begin{enumerate}[label=(\roman*)]
	\item an even number 
	\item a square number
\end{enumerate}
\solution
%\input{exemplar/10/13/3/32/main.tex}
\item
The king, queen and jack of clubs are removed from a deck of 52 playing cards and then well shuffled. Now one card is drawn at random from the remaining cards.  Determine the probability that the card is
\begin{enumerate}[label=(\roman*)]
\item a club
\item 10 of hearts
\end{enumerate}
\solution
%\input{exemplar/10/13/3/29/main.tex}
\item A team of medical students doing their internship have to assist during surgeries
at a city hospital. The probabilities of surgeries rated as very complex, complex,
routine, simple or very simple are respectively, 0.15, 0.20, 0.31, 0.26, .08. Find
the probabilities that a particular surgery will be rated
\begin{enumerate}
	\item complex or very complex;
	\item neither very complex nor very simple;
	\item routine or complex
	\item routine or simple
\end{enumerate}
\solution
%\input{exemplar/11/16/3/8(1)/main.tex}
\item A card is selected from a pack of 52 cards.
\begin{enumerate}[label=(\alph*)]
    \item How many points are there in the sample space?
    \item Calculate the probability that the card is an ace of spades.
    \item Calculate the probability that the card is (i) an ace and (ii) black card.
\end{enumerate}
\solution
%\input{exemplar/11/16/3/4/main2.tex}
\item The probability that a non leap year selected at random will contain 53 sundays.
\\
\solution
%\input{exemplar/10/13/1/19/main.tex}
\item One of the four persons John, Rita, Aslam or Gurpreet will be promoted next
month. Consequently the sample space consists of four elementary outcomes
S = {John promoted, Rita promoted, Aslam promoted, Gurpreet promoted}
You are told that the chances of John’s promotion is same as that of Gurpreet,
Rita’s chances of promotion are twice as likely as Johns. Aslam’s chances are
four times that of John.
\begin{enumerate}
	\item Determine
	\begin{enumerate}
		\item P (John promoted)
		\item P (Rita promoted)
		\item P (Aslam promoted)
		\item P (Gurpreet promoted)
	\end{enumerate}
	\item If A = {John promoted or Gurpreet promoted}, find P (A).
\end{enumerate}
\solution
%\input{exemplar/11/16/3/10/main.tex}
\item A card is drawn from a deck of 52 cards. Find the probability of getting a king or a heart or a red card.\\
\solution
%\input{exemplar/11/16/3/15/main.tex}
\item The probability that a student will pass his examination is 0.73, the probability of
the student getting a compartment is 0.13, and the probability that the student will
either pass or get compartment is 0.96. State True or False.\\
\solution
%\input{exemplar/11/16/3/31/main.tex}
\item A card is selected from a pack of 52 cards\\
\begin{enumerate}[label=(\alph*)]
\item How many points are there in the sample space?
\item Calculate the probability that the cards is an ace of spades.
\item Calculate the probability that the card is (i) an ace (ii)black card.\\
\end{enumerate}
%\input{ncert/11/16/3/4_1/Prob_4.tex}
\item In a non-leap year, the probability of having 53 tuesdays or 53 wednesdays is\\
\solution
%\input{exemplar/11/16/3/18/main.tex}
\item There are 1000 sealed envelopes in a box, 10 of them contain a cash prize of
Rs 100 each, 100 of them contain a cash prize of Rs 50 each and 200 of them
contain a cash prize of Rs 10 each and rest do not contain any cash prize. If they
are well shuffled and an envelope is picked up out, what is the probability that it
contains no cash prize?\\
\solution
%\input{exemplar/10/13/3/34/main.tex}
\item 
A die is thrown and a card is selected at random from a deck of 52 playing cards. The probability of getting an even number on the die and a spade card.\\
\solution
%\input{exemplar/12/13/3/78/main.tex}
\item
If 4-digit numbers greater than 5,000 are randomly formed from the digits 0, 1, 3, 5, and 7, what is the probability of forming a number divisible by 5 when:
\begin{enumerate}
    \item The digits are repeated?
    \item The repetition of digits is not allowed?
\end{enumerate}
\solution
%\input{ncert/11/16/4/9/main.tex}
\item Consider the probability space $\brak{\Omega, \mathcal{G}, P}$ where $\Omega = [0,2]$ and $\mathcal{G} = \cbrak{\phi, \Omega, [0,1], (1,2]}$. Let $X$ and $Y$ be two functions on $\Omega$ defined as
\begin{align*}
    X(\omega) = 
    \begin{cases}
        1 & \text{if }\omega \in [0, 1]\\
        2 & \text{if }\omega \in (1, 2]
    \end{cases}
\end{align*}
and
\begin{align*}
    Y(\omega) = 
    \begin{cases}
        2 & \text{if }\omega \in [0, 1.5]\\
        3 & \text{if }\omega \in (1.5, 2].
    \end{cases}
\end{align*}
Then which one of the following statements is true?
\begin{enumerate}
    \item [(A)] $X$ is a random variable with respect to $\mathcal{G}$, but $Y$ is not a random variable with respect to $\mathcal{G}$.
    \item [(B)] $Y$ is a random variable with respect to $\mathcal{G}$, but $X$ is not a random variable with respect to $\mathcal{G}$.
    \item [(C)] Neither $X$ nor $Y$ is a random variable with respect to $\mathcal{G}$.
    \item [(D)] Both $X$ and $Y$ are random variables with respect to $\mathcal{G}$.
\end{enumerate} \hfill (GATE ST 2023)\\
\solution
%\input{gate/ST/2023/14/main.tex}
	\item  A die is loaded in such a way that each odd number is twice as likely to occur as
each even number. Find $P(G)$, where $G$ is the event that a number greater than
3 occurs on a single roll of the die.
\\
\solution
		%\input{exemplar/11/16/3/5/main.tex}
	\item All the jacks, queens and kings are removed from a deck of 52 playing cards. The remaining cards are well shuffled and then one card is drawn at random. Giving ace a value 1 similar value for other cards, find the probability that the card has a value 
		\begin{enumerate}
			\item 7
			\item greater than 7
			\item less than 7
		\end{enumerate}
		%\input{exemplar/10/13/3/30/main.tex}
  \item A Lot consists of 48 mobile phones of which 42 are good, 3 have only minor defects and 3 have major defects.Varnika will buy a phone if it is good but the trader will only buy a mobile if it has no major defects. One phone is selected at random from the lot. What is the probability that it is
\begin{enumerate}
	\item acceptable to Varnika?
            \item acceptable to the trader?
\end{enumerate}
\solution
	%\input{exemplar/10/13/3/40/main.tex}
 \item A student says that if you throw a die, it will show up 1 or not 1. Therefore, the probability of getting 1 and the probability of getting 'not 1' each is equal to $\frac{1}{2}$. Is this correct? Give reasons.\\
 \solution
        %\input{exemplar/10/13/2/9/main.tex}
   \item Four candidates A, B, C, D have ap-
plied for the assignment to coach a school cricket
team. If A is twice as likely to be selected as B, and
B and C are given about the same chance of being
selected, while C is twice as likely to be selected
as D, what are the probabilities that
\begin{enumerate}
\item C will be selected?
\item A will not be selected?
\end{enumerate}
	%\input{exemplar/11/16/3/9/main.tex}
 \item A bag contain 24 balls of which $x$ balls are red, $2x$ are white and $3x$ are blue. A ball is selected at random, What is the probability that it is
\begin{enumerate}[label=\alph*)]
\item not red ?
\item white ?
\end{enumerate}
%\input{exemplar/10/13/3/41/main.tex}
If the letters of the word ASSASSINATION are arranged at random. Find the Probability that
\begin{enumerate}[label=(\alph*)]
\item Four $S's$ come consecutively in the word
\item Two  $I's$ and two $N's$ come together
\item All $A's$ are not coming together
\item No two $A's$ are coming together
\end{enumerate}
%\input{exemplar/11/16/3/14/main.tex}
	\item One urn contains two black balls (labelled B1 and B2) and one white ball. A
	second urn contains one black ball and two white balls (labelled W1 and W2).
	Suppose the following experiment is performed. One of the two urns is chosen
	at random. Next a ball is randomly chosen from the urn. Then a second ball is
	chosen at random from the same urn without replacing the first ball.
	
	\begin{enumerate}
	\item What is the probability that two black balls are chosen?
	
	\item What is the probability that two balls of opposite colour are chosen?
	\end{enumerate}
	\solution
	%\input{exemplar/11/16/3/12/main1.tex}
\end{enumerate}

		\item A box of oranges is inspected by examining three randomly selected oranges drawn without replacement. If all the three oranges are good, the box is approved for sale, otherwise, it is rejected. Find the probability that a box containing 15 oranges out of which 12 are good and 3 are bad ones will be approved for sale.
		\label{ncert/12/13/2/3/defs.tex}
		\item Two balls are drawn at random with replacement from a box containing 10 black and 8 red balls. Find the probability that
		\label{ncert/12/13/2/12}
\begin{enumerate}
\item both balls are red.
\item first ball is black and second is red.
\item one of them is black and other is red.
\end{enumerate}

\item In a hostel, 60\% of the students read Hindi newspaper, 40\% read English newspaper and 20\% read both Hindi and English newspapers. A student is selected at random.
		\label{ncert/12/13/2/15}
\begin{enumerate}
\item Find the probability that she reads neither Hindi nor English newspapers.
\item If she reads Hindi newspaper, find the probability that she reads English newspaper.
\item If she reads English newspaper, find the probability that she reads Hindi newspaper.\\
\end{enumerate}
\item The probability of obtaining an even prime number on each die, when a pair of dice is rolled is 
\begin{enumerate}
    \item $0$ 
    
    \item $\frac{1}{3}$ 
    
    \item $\frac{1}{12}$ 
    
    \item $\frac{1}{36}$ 
\end{enumerate}
\solution
		%\begin{enumerate}[label=\thesection.\arabic*,ref=\thesection.\theenumi]
	\item One card is drawn from a well-shuffled deck of 52 cards. Find the probability of getting
\begin{enumerate}
\item A king of red colour 
\item A face card 
\item A red face card
\item The jack of hearts
\item A spade
\item The queen of diamonds

\end{enumerate}
\solution
		%\input{ncert/10/15/1/14/main.tex}
	\item Five cards—the ten, jack, queen, king and ace of diamonds, are well-shuffled with their face downwards. One card is then picked up at random.
\begin{enumerate}
\item
What is the probability that the card is the queen? 
\item
If the queen is drawn and put aside, what is the probability that the second card picked up is (a) an ace? (b) a queen?\\
\end{enumerate}
\solution
		%\input{ncert/10/15/1/15/defs.tex}
	\item A bag contains $5$ red balls and some blue balls. If the probability of drawing a blue ball is double that if a red ball, determine the number of blue balls in the bag. 
		\\
\solution
		%\input{ncert/10/15/2/3/defs.tex}
	\item A card is selected from a pack of 52 cards.
 \begin{enumerate}[label=(\alph*)] 
                 \item How many points are there in the sample space?
                 \item Calculate the probability that the card is an ace of spades.
                 \item Calculate the probability that the card is (i) an ace and (ii) black card.
 \end{enumerate}
\solution
		%\input{ncert/11/16/3/4/main.tex}
\item Four cards are drawn from a well-shuffled deck of 52 cards. What is the probability of obtaining 3 diamonds and one spade.
\\
\solution
		%\input{ncert/11/16/4/2/defs.tex}
\item In a certain lottery 10,000 tickets are sold and ten equal prizes are awarded. What is the probability of not getting a prize if you buy (a) one ticket (b) two tickets (c) 10 tickets ?	
\\
\solution
		%\input{ncert/11/16/4/4/defs.tex}
		%
\item 
Out of 100 students, two sections of 40 and 60 are formed. If you and your friend are among the 100 students, what is the probability that
\begin{enumerate}
\item you both enter the same section?
\item you both enter the different sections?
\end{enumerate}
\solution
		%\input{ncert/11/16/4/5/defs.tex}
	\item 
The number lock of a suitcase has 4 wheels each labelled with ten digits i.e. from 0 to 9.The lock opens with a sequence of four digits with no repeats.What is the probability of a person getting the right sequence to open the suitcase.
\\
\solution
		%\input{ncert/11/16/4/10/defs.tex}
		%
\item 
Two cards are drawn at random and without replacement from a pack of 52 playing cards. Find the probability that both the cards are black.
\\
\solution
		%\input{ncert/12/13/2/2/defs.tex}
		\item A box of oranges is inspected by examining three randomly selected oranges drawn without replacement. If all the three oranges are good, the box is approved for sale, otherwise, it is rejected. Find the probability that a box containing 15 oranges out of which 12 are good and 3 are bad ones will be approved for sale.
		\label{ncert/12/13/2/3/defs.tex}
		\item Two balls are drawn at random with replacement from a box containing 10 black and 8 red balls. Find the probability that
		\label{ncert/12/13/2/12}
\begin{enumerate}
\item both balls are red.
\item first ball is black and second is red.
\item one of them is black and other is red.
\end{enumerate}

\item In a hostel, 60\% of the students read Hindi newspaper, 40\% read English newspaper and 20\% read both Hindi and English newspapers. A student is selected at random.
		\label{ncert/12/13/2/15}
\begin{enumerate}
\item Find the probability that she reads neither Hindi nor English newspapers.
\item If she reads Hindi newspaper, find the probability that she reads English newspaper.
\item If she reads English newspaper, find the probability that she reads Hindi newspaper.\\
\end{enumerate}
\item The probability of obtaining an even prime number on each die, when a pair of dice is rolled is 
\begin{enumerate}
    \item $0$ 
    
    \item $\frac{1}{3}$ 
    
    \item $\frac{1}{12}$ 
    
    \item $\frac{1}{36}$ 
\end{enumerate}
\solution
		%\input{ncert/12/13/2/17/defs.tex}
	\item A bag contains 4 red and 4 black balls, another bag contains 2 red and 6 black balls. One of the two bags is selected at random and a ball is drawn from the bag which is found to be red. Find the probability that the ball is drawn from the first bag.
\\
\solution
		%\input{ncert/12/13/3/2/main.tex}
  \item
  Cards with numbers 2 to 101 are placed in a box. A card is selected at random.Find the probability that the card has
\begin{enumerate}[label=(\roman*)]
	\item an even number 
	\item a square number
\end{enumerate}
\solution
%\input{exemplar/10/13/3/32/main.tex}
\item
The king, queen and jack of clubs are removed from a deck of 52 playing cards and then well shuffled. Now one card is drawn at random from the remaining cards.  Determine the probability that the card is
\begin{enumerate}[label=(\roman*)]
\item a club
\item 10 of hearts
\end{enumerate}
\solution
%\input{exemplar/10/13/3/29/main.tex}
\item A team of medical students doing their internship have to assist during surgeries
at a city hospital. The probabilities of surgeries rated as very complex, complex,
routine, simple or very simple are respectively, 0.15, 0.20, 0.31, 0.26, .08. Find
the probabilities that a particular surgery will be rated
\begin{enumerate}
	\item complex or very complex;
	\item neither very complex nor very simple;
	\item routine or complex
	\item routine or simple
\end{enumerate}
\solution
%\input{exemplar/11/16/3/8(1)/main.tex}
\item A card is selected from a pack of 52 cards.
\begin{enumerate}[label=(\alph*)]
    \item How many points are there in the sample space?
    \item Calculate the probability that the card is an ace of spades.
    \item Calculate the probability that the card is (i) an ace and (ii) black card.
\end{enumerate}
\solution
%\input{exemplar/11/16/3/4/main2.tex}
\item The probability that a non leap year selected at random will contain 53 sundays.
\\
\solution
%\input{exemplar/10/13/1/19/main.tex}
\item One of the four persons John, Rita, Aslam or Gurpreet will be promoted next
month. Consequently the sample space consists of four elementary outcomes
S = {John promoted, Rita promoted, Aslam promoted, Gurpreet promoted}
You are told that the chances of John’s promotion is same as that of Gurpreet,
Rita’s chances of promotion are twice as likely as Johns. Aslam’s chances are
four times that of John.
\begin{enumerate}
	\item Determine
	\begin{enumerate}
		\item P (John promoted)
		\item P (Rita promoted)
		\item P (Aslam promoted)
		\item P (Gurpreet promoted)
	\end{enumerate}
	\item If A = {John promoted or Gurpreet promoted}, find P (A).
\end{enumerate}
\solution
%\input{exemplar/11/16/3/10/main.tex}
\item A card is drawn from a deck of 52 cards. Find the probability of getting a king or a heart or a red card.\\
\solution
%\input{exemplar/11/16/3/15/main.tex}
\item The probability that a student will pass his examination is 0.73, the probability of
the student getting a compartment is 0.13, and the probability that the student will
either pass or get compartment is 0.96. State True or False.\\
\solution
%\input{exemplar/11/16/3/31/main.tex}
\item A card is selected from a pack of 52 cards\\
\begin{enumerate}[label=(\alph*)]
\item How many points are there in the sample space?
\item Calculate the probability that the cards is an ace of spades.
\item Calculate the probability that the card is (i) an ace (ii)black card.\\
\end{enumerate}
%\input{ncert/11/16/3/4_1/Prob_4.tex}
\item In a non-leap year, the probability of having 53 tuesdays or 53 wednesdays is\\
\solution
%\input{exemplar/11/16/3/18/main.tex}
\item There are 1000 sealed envelopes in a box, 10 of them contain a cash prize of
Rs 100 each, 100 of them contain a cash prize of Rs 50 each and 200 of them
contain a cash prize of Rs 10 each and rest do not contain any cash prize. If they
are well shuffled and an envelope is picked up out, what is the probability that it
contains no cash prize?\\
\solution
%\input{exemplar/10/13/3/34/main.tex}
\item 
A die is thrown and a card is selected at random from a deck of 52 playing cards. The probability of getting an even number on the die and a spade card.\\
\solution
%\input{exemplar/12/13/3/78/main.tex}
\item
If 4-digit numbers greater than 5,000 are randomly formed from the digits 0, 1, 3, 5, and 7, what is the probability of forming a number divisible by 5 when:
\begin{enumerate}
    \item The digits are repeated?
    \item The repetition of digits is not allowed?
\end{enumerate}
\solution
%\input{ncert/11/16/4/9/main.tex}
\item Consider the probability space $\brak{\Omega, \mathcal{G}, P}$ where $\Omega = [0,2]$ and $\mathcal{G} = \cbrak{\phi, \Omega, [0,1], (1,2]}$. Let $X$ and $Y$ be two functions on $\Omega$ defined as
\begin{align*}
    X(\omega) = 
    \begin{cases}
        1 & \text{if }\omega \in [0, 1]\\
        2 & \text{if }\omega \in (1, 2]
    \end{cases}
\end{align*}
and
\begin{align*}
    Y(\omega) = 
    \begin{cases}
        2 & \text{if }\omega \in [0, 1.5]\\
        3 & \text{if }\omega \in (1.5, 2].
    \end{cases}
\end{align*}
Then which one of the following statements is true?
\begin{enumerate}
    \item [(A)] $X$ is a random variable with respect to $\mathcal{G}$, but $Y$ is not a random variable with respect to $\mathcal{G}$.
    \item [(B)] $Y$ is a random variable with respect to $\mathcal{G}$, but $X$ is not a random variable with respect to $\mathcal{G}$.
    \item [(C)] Neither $X$ nor $Y$ is a random variable with respect to $\mathcal{G}$.
    \item [(D)] Both $X$ and $Y$ are random variables with respect to $\mathcal{G}$.
\end{enumerate} \hfill (GATE ST 2023)\\
\solution
%\input{gate/ST/2023/14/main.tex}
	\item  A die is loaded in such a way that each odd number is twice as likely to occur as
each even number. Find $P(G)$, where $G$ is the event that a number greater than
3 occurs on a single roll of the die.
\\
\solution
		%\input{exemplar/11/16/3/5/main.tex}
	\item All the jacks, queens and kings are removed from a deck of 52 playing cards. The remaining cards are well shuffled and then one card is drawn at random. Giving ace a value 1 similar value for other cards, find the probability that the card has a value 
		\begin{enumerate}
			\item 7
			\item greater than 7
			\item less than 7
		\end{enumerate}
		%\input{exemplar/10/13/3/30/main.tex}
  \item A Lot consists of 48 mobile phones of which 42 are good, 3 have only minor defects and 3 have major defects.Varnika will buy a phone if it is good but the trader will only buy a mobile if it has no major defects. One phone is selected at random from the lot. What is the probability that it is
\begin{enumerate}
	\item acceptable to Varnika?
            \item acceptable to the trader?
\end{enumerate}
\solution
	%\input{exemplar/10/13/3/40/main.tex}
 \item A student says that if you throw a die, it will show up 1 or not 1. Therefore, the probability of getting 1 and the probability of getting 'not 1' each is equal to $\frac{1}{2}$. Is this correct? Give reasons.\\
 \solution
        %\input{exemplar/10/13/2/9/main.tex}
   \item Four candidates A, B, C, D have ap-
plied for the assignment to coach a school cricket
team. If A is twice as likely to be selected as B, and
B and C are given about the same chance of being
selected, while C is twice as likely to be selected
as D, what are the probabilities that
\begin{enumerate}
\item C will be selected?
\item A will not be selected?
\end{enumerate}
	%\input{exemplar/11/16/3/9/main.tex}
 \item A bag contain 24 balls of which $x$ balls are red, $2x$ are white and $3x$ are blue. A ball is selected at random, What is the probability that it is
\begin{enumerate}[label=\alph*)]
\item not red ?
\item white ?
\end{enumerate}
%\input{exemplar/10/13/3/41/main.tex}
If the letters of the word ASSASSINATION are arranged at random. Find the Probability that
\begin{enumerate}[label=(\alph*)]
\item Four $S's$ come consecutively in the word
\item Two  $I's$ and two $N's$ come together
\item All $A's$ are not coming together
\item No two $A's$ are coming together
\end{enumerate}
%\input{exemplar/11/16/3/14/main.tex}
	\item One urn contains two black balls (labelled B1 and B2) and one white ball. A
	second urn contains one black ball and two white balls (labelled W1 and W2).
	Suppose the following experiment is performed. One of the two urns is chosen
	at random. Next a ball is randomly chosen from the urn. Then a second ball is
	chosen at random from the same urn without replacing the first ball.
	
	\begin{enumerate}
	\item What is the probability that two black balls are chosen?
	
	\item What is the probability that two balls of opposite colour are chosen?
	\end{enumerate}
	\solution
	%\input{exemplar/11/16/3/12/main1.tex}
\end{enumerate}

	\item A bag contains 4 red and 4 black balls, another bag contains 2 red and 6 black balls. One of the two bags is selected at random and a ball is drawn from the bag which is found to be red. Find the probability that the ball is drawn from the first bag.
\\
\solution
		%\begin{table}[H]
	\centering
\begin{tabular}{|c|c|c|}
\hline
Random variable &Value &Definition\\ \hline
\multirow{3}{*}{X} &0 &Slips of Rs 1\\
&1 &Slips of Rs 5\\
&2 &Slips of Rs 13\\ \hline
\multirow{2}{*}{Y} &0 &Box A\\
&1 &Box B\\\hline
\end{tabular}
\caption{}
\label{tab:Distribution}
\end{table}
See \tabref{tab:Distribution}.
\begin{align}
p_{Y}\brak{k}= \begin{cases} 
      \frac{1}{3} & {k=0} \\
      \frac{2}{3 }& {k=1} 
   \end{cases}
   \\
p_{Y|X}\brak{0|0} = \frac{19}{25}\, 
p_{Y|X}\brak{0|1} = \frac{6}{25}\,
p_{Y|X}\brak{1|0} = \frac{45}{50}\,
p_{Y|X}\brak{1|2} = \frac{5}{50}
\end{align}
The desired probability is the probability that a slip drawn at random is marked other than Rs 1,
\begin{align}
&=1-p_X\brak{0}\\
&= p_X(1) + p_X(2)
\end{align}
Using Bayes theorem,
\begin{align}
&= p_Y\brak{0} \times \pr{Y=0 | X=1} + p_Y\brak{1} \times \pr{Y=1|X=2}\\
&=\frac{1}{3} \times \frac{6}{25} + \frac{2}{3} \times \frac{5}{50}\\
&=\frac{11}{75}
\end{align}

\newpage

%\tableofcontents

\bigskip

\renewcommand{\thefigure}{\theenumi}
\renewcommand{\thetable}{\theenumi}
%\renewcommand{\theequation}{\theenumi}

%\begin{abstract}
%%\boldmath
%In this letter, an algorithm for evaluating the exact analytical bit error rate  (BER)  for the piecewise linear (PL) combiner for  multiple relays is presented. Previous results were available only for upto three relays. The algorithm is unique in the sense that  the actual mathematical expressions, that are prohibitively large, need not be explicitly obtained. The diversity gain due to multiple relays is shown through plots of the analytical BER, well supported by simulations. 
%
%\end{abstract}
% IEEEtran.cls defaults to using nonbold math in the Abstract.
% This preserves the distinction between vectors and scalars. However,
% if the journal you are submitting to favors bold math in the abstract,
% then you can use LaTeX's standard command \boldmath at the very start
% of the abstract to achieve this. Many IEEE journals frown on math
% in the abstract anyway.

% Note that keywords are not normally used for peerreview papers.
%\begin{IEEEkeywords}
%Cooperative diversity, decode and forward, piecewise linear
%\end{IEEEkeywords}



% For peer review papers, you can put extra information on the cover
% page as needed:
% \ifCLASSOPTIONpeerreview
% \begin{center} \bfseries EDICS Category: 3-BBND \end{center}
% \fi
%
% For peerreview papers, this IEEEtran command inserts a page break and
% creates the second title. It will be ignored for other modes.
%\IEEEpeerreviewmaketitle




  \item
  Cards with numbers 2 to 101 are placed in a box. A card is selected at random.Find the probability that the card has
\begin{enumerate}[label=(\roman*)]
	\item an even number 
	\item a square number
\end{enumerate}
\solution
%\begin{table}[H]
	\centering
\begin{tabular}{|c|c|c|}
\hline
Random variable &Value &Definition\\ \hline
\multirow{3}{*}{X} &0 &Slips of Rs 1\\
&1 &Slips of Rs 5\\
&2 &Slips of Rs 13\\ \hline
\multirow{2}{*}{Y} &0 &Box A\\
&1 &Box B\\\hline
\end{tabular}
\caption{}
\label{tab:Distribution}
\end{table}
See \tabref{tab:Distribution}.
\begin{align}
p_{Y}\brak{k}= \begin{cases} 
      \frac{1}{3} & {k=0} \\
      \frac{2}{3 }& {k=1} 
   \end{cases}
   \\
p_{Y|X}\brak{0|0} = \frac{19}{25}\, 
p_{Y|X}\brak{0|1} = \frac{6}{25}\,
p_{Y|X}\brak{1|0} = \frac{45}{50}\,
p_{Y|X}\brak{1|2} = \frac{5}{50}
\end{align}
The desired probability is the probability that a slip drawn at random is marked other than Rs 1,
\begin{align}
&=1-p_X\brak{0}\\
&= p_X(1) + p_X(2)
\end{align}
Using Bayes theorem,
\begin{align}
&= p_Y\brak{0} \times \pr{Y=0 | X=1} + p_Y\brak{1} \times \pr{Y=1|X=2}\\
&=\frac{1}{3} \times \frac{6}{25} + \frac{2}{3} \times \frac{5}{50}\\
&=\frac{11}{75}
\end{align}

\newpage

%\tableofcontents

\bigskip

\renewcommand{\thefigure}{\theenumi}
\renewcommand{\thetable}{\theenumi}
%\renewcommand{\theequation}{\theenumi}

%\begin{abstract}
%%\boldmath
%In this letter, an algorithm for evaluating the exact analytical bit error rate  (BER)  for the piecewise linear (PL) combiner for  multiple relays is presented. Previous results were available only for upto three relays. The algorithm is unique in the sense that  the actual mathematical expressions, that are prohibitively large, need not be explicitly obtained. The diversity gain due to multiple relays is shown through plots of the analytical BER, well supported by simulations. 
%
%\end{abstract}
% IEEEtran.cls defaults to using nonbold math in the Abstract.
% This preserves the distinction between vectors and scalars. However,
% if the journal you are submitting to favors bold math in the abstract,
% then you can use LaTeX's standard command \boldmath at the very start
% of the abstract to achieve this. Many IEEE journals frown on math
% in the abstract anyway.

% Note that keywords are not normally used for peerreview papers.
%\begin{IEEEkeywords}
%Cooperative diversity, decode and forward, piecewise linear
%\end{IEEEkeywords}



% For peer review papers, you can put extra information on the cover
% page as needed:
% \ifCLASSOPTIONpeerreview
% \begin{center} \bfseries EDICS Category: 3-BBND \end{center}
% \fi
%
% For peerreview papers, this IEEEtran command inserts a page break and
% creates the second title. It will be ignored for other modes.
%\IEEEpeerreviewmaketitle




\item
The king, queen and jack of clubs are removed from a deck of 52 playing cards and then well shuffled. Now one card is drawn at random from the remaining cards.  Determine the probability that the card is
\begin{enumerate}[label=(\roman*)]
\item a club
\item 10 of hearts
\end{enumerate}
\solution
%\begin{table}[H]
	\centering
\begin{tabular}{|c|c|c|}
\hline
Random variable &Value &Definition\\ \hline
\multirow{3}{*}{X} &0 &Slips of Rs 1\\
&1 &Slips of Rs 5\\
&2 &Slips of Rs 13\\ \hline
\multirow{2}{*}{Y} &0 &Box A\\
&1 &Box B\\\hline
\end{tabular}
\caption{}
\label{tab:Distribution}
\end{table}
See \tabref{tab:Distribution}.
\begin{align}
p_{Y}\brak{k}= \begin{cases} 
      \frac{1}{3} & {k=0} \\
      \frac{2}{3 }& {k=1} 
   \end{cases}
   \\
p_{Y|X}\brak{0|0} = \frac{19}{25}\, 
p_{Y|X}\brak{0|1} = \frac{6}{25}\,
p_{Y|X}\brak{1|0} = \frac{45}{50}\,
p_{Y|X}\brak{1|2} = \frac{5}{50}
\end{align}
The desired probability is the probability that a slip drawn at random is marked other than Rs 1,
\begin{align}
&=1-p_X\brak{0}\\
&= p_X(1) + p_X(2)
\end{align}
Using Bayes theorem,
\begin{align}
&= p_Y\brak{0} \times \pr{Y=0 | X=1} + p_Y\brak{1} \times \pr{Y=1|X=2}\\
&=\frac{1}{3} \times \frac{6}{25} + \frac{2}{3} \times \frac{5}{50}\\
&=\frac{11}{75}
\end{align}

\newpage

%\tableofcontents

\bigskip

\renewcommand{\thefigure}{\theenumi}
\renewcommand{\thetable}{\theenumi}
%\renewcommand{\theequation}{\theenumi}

%\begin{abstract}
%%\boldmath
%In this letter, an algorithm for evaluating the exact analytical bit error rate  (BER)  for the piecewise linear (PL) combiner for  multiple relays is presented. Previous results were available only for upto three relays. The algorithm is unique in the sense that  the actual mathematical expressions, that are prohibitively large, need not be explicitly obtained. The diversity gain due to multiple relays is shown through plots of the analytical BER, well supported by simulations. 
%
%\end{abstract}
% IEEEtran.cls defaults to using nonbold math in the Abstract.
% This preserves the distinction between vectors and scalars. However,
% if the journal you are submitting to favors bold math in the abstract,
% then you can use LaTeX's standard command \boldmath at the very start
% of the abstract to achieve this. Many IEEE journals frown on math
% in the abstract anyway.

% Note that keywords are not normally used for peerreview papers.
%\begin{IEEEkeywords}
%Cooperative diversity, decode and forward, piecewise linear
%\end{IEEEkeywords}



% For peer review papers, you can put extra information on the cover
% page as needed:
% \ifCLASSOPTIONpeerreview
% \begin{center} \bfseries EDICS Category: 3-BBND \end{center}
% \fi
%
% For peerreview papers, this IEEEtran command inserts a page break and
% creates the second title. It will be ignored for other modes.
%\IEEEpeerreviewmaketitle




\item A team of medical students doing their internship have to assist during surgeries
at a city hospital. The probabilities of surgeries rated as very complex, complex,
routine, simple or very simple are respectively, 0.15, 0.20, 0.31, 0.26, .08. Find
the probabilities that a particular surgery will be rated
\begin{enumerate}
	\item complex or very complex;
	\item neither very complex nor very simple;
	\item routine or complex
	\item routine or simple
\end{enumerate}
\solution
%\begin{table}[H]
	\centering
\begin{tabular}{|c|c|c|}
\hline
Random variable &Value &Definition\\ \hline
\multirow{3}{*}{X} &0 &Slips of Rs 1\\
&1 &Slips of Rs 5\\
&2 &Slips of Rs 13\\ \hline
\multirow{2}{*}{Y} &0 &Box A\\
&1 &Box B\\\hline
\end{tabular}
\caption{}
\label{tab:Distribution}
\end{table}
See \tabref{tab:Distribution}.
\begin{align}
p_{Y}\brak{k}= \begin{cases} 
      \frac{1}{3} & {k=0} \\
      \frac{2}{3 }& {k=1} 
   \end{cases}
   \\
p_{Y|X}\brak{0|0} = \frac{19}{25}\, 
p_{Y|X}\brak{0|1} = \frac{6}{25}\,
p_{Y|X}\brak{1|0} = \frac{45}{50}\,
p_{Y|X}\brak{1|2} = \frac{5}{50}
\end{align}
The desired probability is the probability that a slip drawn at random is marked other than Rs 1,
\begin{align}
&=1-p_X\brak{0}\\
&= p_X(1) + p_X(2)
\end{align}
Using Bayes theorem,
\begin{align}
&= p_Y\brak{0} \times \pr{Y=0 | X=1} + p_Y\brak{1} \times \pr{Y=1|X=2}\\
&=\frac{1}{3} \times \frac{6}{25} + \frac{2}{3} \times \frac{5}{50}\\
&=\frac{11}{75}
\end{align}

\newpage

%\tableofcontents

\bigskip

\renewcommand{\thefigure}{\theenumi}
\renewcommand{\thetable}{\theenumi}
%\renewcommand{\theequation}{\theenumi}

%\begin{abstract}
%%\boldmath
%In this letter, an algorithm for evaluating the exact analytical bit error rate  (BER)  for the piecewise linear (PL) combiner for  multiple relays is presented. Previous results were available only for upto three relays. The algorithm is unique in the sense that  the actual mathematical expressions, that are prohibitively large, need not be explicitly obtained. The diversity gain due to multiple relays is shown through plots of the analytical BER, well supported by simulations. 
%
%\end{abstract}
% IEEEtran.cls defaults to using nonbold math in the Abstract.
% This preserves the distinction between vectors and scalars. However,
% if the journal you are submitting to favors bold math in the abstract,
% then you can use LaTeX's standard command \boldmath at the very start
% of the abstract to achieve this. Many IEEE journals frown on math
% in the abstract anyway.

% Note that keywords are not normally used for peerreview papers.
%\begin{IEEEkeywords}
%Cooperative diversity, decode and forward, piecewise linear
%\end{IEEEkeywords}



% For peer review papers, you can put extra information on the cover
% page as needed:
% \ifCLASSOPTIONpeerreview
% \begin{center} \bfseries EDICS Category: 3-BBND \end{center}
% \fi
%
% For peerreview papers, this IEEEtran command inserts a page break and
% creates the second title. It will be ignored for other modes.
%\IEEEpeerreviewmaketitle




\item A card is selected from a pack of 52 cards.
\begin{enumerate}[label=(\alph*)]
    \item How many points are there in the sample space?
    \item Calculate the probability that the card is an ace of spades.
    \item Calculate the probability that the card is (i) an ace and (ii) black card.
\end{enumerate}
\solution
%Let $X$ be an bernoulli rv defined as in \tabref{tab:exemplar/11/16/3/26}.  Then, 
\begin{equation}
    p =
        \frac{4}{11} 
\end{equation}
\begin{table}[H]
	\centering
	\input{exemplar/11/16/3/26/tables/Table2.tex}
	\caption{}
        \label{tab:exemplar/11/16/3/26}
\end{table}

\item The probability that a non leap year selected at random will contain 53 sundays.
\\
\solution
%\begin{table}[H]
	\centering
\begin{tabular}{|c|c|c|}
\hline
Random variable &Value &Definition\\ \hline
\multirow{3}{*}{X} &0 &Slips of Rs 1\\
&1 &Slips of Rs 5\\
&2 &Slips of Rs 13\\ \hline
\multirow{2}{*}{Y} &0 &Box A\\
&1 &Box B\\\hline
\end{tabular}
\caption{}
\label{tab:Distribution}
\end{table}
See \tabref{tab:Distribution}.
\begin{align}
p_{Y}\brak{k}= \begin{cases} 
      \frac{1}{3} & {k=0} \\
      \frac{2}{3 }& {k=1} 
   \end{cases}
   \\
p_{Y|X}\brak{0|0} = \frac{19}{25}\, 
p_{Y|X}\brak{0|1} = \frac{6}{25}\,
p_{Y|X}\brak{1|0} = \frac{45}{50}\,
p_{Y|X}\brak{1|2} = \frac{5}{50}
\end{align}
The desired probability is the probability that a slip drawn at random is marked other than Rs 1,
\begin{align}
&=1-p_X\brak{0}\\
&= p_X(1) + p_X(2)
\end{align}
Using Bayes theorem,
\begin{align}
&= p_Y\brak{0} \times \pr{Y=0 | X=1} + p_Y\brak{1} \times \pr{Y=1|X=2}\\
&=\frac{1}{3} \times \frac{6}{25} + \frac{2}{3} \times \frac{5}{50}\\
&=\frac{11}{75}
\end{align}

\newpage

%\tableofcontents

\bigskip

\renewcommand{\thefigure}{\theenumi}
\renewcommand{\thetable}{\theenumi}
%\renewcommand{\theequation}{\theenumi}

%\begin{abstract}
%%\boldmath
%In this letter, an algorithm for evaluating the exact analytical bit error rate  (BER)  for the piecewise linear (PL) combiner for  multiple relays is presented. Previous results were available only for upto three relays. The algorithm is unique in the sense that  the actual mathematical expressions, that are prohibitively large, need not be explicitly obtained. The diversity gain due to multiple relays is shown through plots of the analytical BER, well supported by simulations. 
%
%\end{abstract}
% IEEEtran.cls defaults to using nonbold math in the Abstract.
% This preserves the distinction between vectors and scalars. However,
% if the journal you are submitting to favors bold math in the abstract,
% then you can use LaTeX's standard command \boldmath at the very start
% of the abstract to achieve this. Many IEEE journals frown on math
% in the abstract anyway.

% Note that keywords are not normally used for peerreview papers.
%\begin{IEEEkeywords}
%Cooperative diversity, decode and forward, piecewise linear
%\end{IEEEkeywords}



% For peer review papers, you can put extra information on the cover
% page as needed:
% \ifCLASSOPTIONpeerreview
% \begin{center} \bfseries EDICS Category: 3-BBND \end{center}
% \fi
%
% For peerreview papers, this IEEEtran command inserts a page break and
% creates the second title. It will be ignored for other modes.
%\IEEEpeerreviewmaketitle




\item One of the four persons John, Rita, Aslam or Gurpreet will be promoted next
month. Consequently the sample space consists of four elementary outcomes
S = {John promoted, Rita promoted, Aslam promoted, Gurpreet promoted}
You are told that the chances of John’s promotion is same as that of Gurpreet,
Rita’s chances of promotion are twice as likely as Johns. Aslam’s chances are
four times that of John.
\begin{enumerate}
	\item Determine
	\begin{enumerate}
		\item P (John promoted)
		\item P (Rita promoted)
		\item P (Aslam promoted)
		\item P (Gurpreet promoted)
	\end{enumerate}
	\item If A = {John promoted or Gurpreet promoted}, find P (A).
\end{enumerate}
\solution
%\begin{table}[H]
	\centering
\begin{tabular}{|c|c|c|}
\hline
Random variable &Value &Definition\\ \hline
\multirow{3}{*}{X} &0 &Slips of Rs 1\\
&1 &Slips of Rs 5\\
&2 &Slips of Rs 13\\ \hline
\multirow{2}{*}{Y} &0 &Box A\\
&1 &Box B\\\hline
\end{tabular}
\caption{}
\label{tab:Distribution}
\end{table}
See \tabref{tab:Distribution}.
\begin{align}
p_{Y}\brak{k}= \begin{cases} 
      \frac{1}{3} & {k=0} \\
      \frac{2}{3 }& {k=1} 
   \end{cases}
   \\
p_{Y|X}\brak{0|0} = \frac{19}{25}\, 
p_{Y|X}\brak{0|1} = \frac{6}{25}\,
p_{Y|X}\brak{1|0} = \frac{45}{50}\,
p_{Y|X}\brak{1|2} = \frac{5}{50}
\end{align}
The desired probability is the probability that a slip drawn at random is marked other than Rs 1,
\begin{align}
&=1-p_X\brak{0}\\
&= p_X(1) + p_X(2)
\end{align}
Using Bayes theorem,
\begin{align}
&= p_Y\brak{0} \times \pr{Y=0 | X=1} + p_Y\brak{1} \times \pr{Y=1|X=2}\\
&=\frac{1}{3} \times \frac{6}{25} + \frac{2}{3} \times \frac{5}{50}\\
&=\frac{11}{75}
\end{align}

\newpage

%\tableofcontents

\bigskip

\renewcommand{\thefigure}{\theenumi}
\renewcommand{\thetable}{\theenumi}
%\renewcommand{\theequation}{\theenumi}

%\begin{abstract}
%%\boldmath
%In this letter, an algorithm for evaluating the exact analytical bit error rate  (BER)  for the piecewise linear (PL) combiner for  multiple relays is presented. Previous results were available only for upto three relays. The algorithm is unique in the sense that  the actual mathematical expressions, that are prohibitively large, need not be explicitly obtained. The diversity gain due to multiple relays is shown through plots of the analytical BER, well supported by simulations. 
%
%\end{abstract}
% IEEEtran.cls defaults to using nonbold math in the Abstract.
% This preserves the distinction between vectors and scalars. However,
% if the journal you are submitting to favors bold math in the abstract,
% then you can use LaTeX's standard command \boldmath at the very start
% of the abstract to achieve this. Many IEEE journals frown on math
% in the abstract anyway.

% Note that keywords are not normally used for peerreview papers.
%\begin{IEEEkeywords}
%Cooperative diversity, decode and forward, piecewise linear
%\end{IEEEkeywords}



% For peer review papers, you can put extra information on the cover
% page as needed:
% \ifCLASSOPTIONpeerreview
% \begin{center} \bfseries EDICS Category: 3-BBND \end{center}
% \fi
%
% For peerreview papers, this IEEEtran command inserts a page break and
% creates the second title. It will be ignored for other modes.
%\IEEEpeerreviewmaketitle




\item A card is drawn from a deck of 52 cards. Find the probability of getting a king or a heart or a red card.\\
\solution
%\begin{table}[H]
	\centering
\begin{tabular}{|c|c|c|}
\hline
Random variable &Value &Definition\\ \hline
\multirow{3}{*}{X} &0 &Slips of Rs 1\\
&1 &Slips of Rs 5\\
&2 &Slips of Rs 13\\ \hline
\multirow{2}{*}{Y} &0 &Box A\\
&1 &Box B\\\hline
\end{tabular}
\caption{}
\label{tab:Distribution}
\end{table}
See \tabref{tab:Distribution}.
\begin{align}
p_{Y}\brak{k}= \begin{cases} 
      \frac{1}{3} & {k=0} \\
      \frac{2}{3 }& {k=1} 
   \end{cases}
   \\
p_{Y|X}\brak{0|0} = \frac{19}{25}\, 
p_{Y|X}\brak{0|1} = \frac{6}{25}\,
p_{Y|X}\brak{1|0} = \frac{45}{50}\,
p_{Y|X}\brak{1|2} = \frac{5}{50}
\end{align}
The desired probability is the probability that a slip drawn at random is marked other than Rs 1,
\begin{align}
&=1-p_X\brak{0}\\
&= p_X(1) + p_X(2)
\end{align}
Using Bayes theorem,
\begin{align}
&= p_Y\brak{0} \times \pr{Y=0 | X=1} + p_Y\brak{1} \times \pr{Y=1|X=2}\\
&=\frac{1}{3} \times \frac{6}{25} + \frac{2}{3} \times \frac{5}{50}\\
&=\frac{11}{75}
\end{align}

\newpage

%\tableofcontents

\bigskip

\renewcommand{\thefigure}{\theenumi}
\renewcommand{\thetable}{\theenumi}
%\renewcommand{\theequation}{\theenumi}

%\begin{abstract}
%%\boldmath
%In this letter, an algorithm for evaluating the exact analytical bit error rate  (BER)  for the piecewise linear (PL) combiner for  multiple relays is presented. Previous results were available only for upto three relays. The algorithm is unique in the sense that  the actual mathematical expressions, that are prohibitively large, need not be explicitly obtained. The diversity gain due to multiple relays is shown through plots of the analytical BER, well supported by simulations. 
%
%\end{abstract}
% IEEEtran.cls defaults to using nonbold math in the Abstract.
% This preserves the distinction between vectors and scalars. However,
% if the journal you are submitting to favors bold math in the abstract,
% then you can use LaTeX's standard command \boldmath at the very start
% of the abstract to achieve this. Many IEEE journals frown on math
% in the abstract anyway.

% Note that keywords are not normally used for peerreview papers.
%\begin{IEEEkeywords}
%Cooperative diversity, decode and forward, piecewise linear
%\end{IEEEkeywords}



% For peer review papers, you can put extra information on the cover
% page as needed:
% \ifCLASSOPTIONpeerreview
% \begin{center} \bfseries EDICS Category: 3-BBND \end{center}
% \fi
%
% For peerreview papers, this IEEEtran command inserts a page break and
% creates the second title. It will be ignored for other modes.
%\IEEEpeerreviewmaketitle




\item The probability that a student will pass his examination is 0.73, the probability of
the student getting a compartment is 0.13, and the probability that the student will
either pass or get compartment is 0.96. State True or False.\\
\solution
%\begin{table}[H]
	\centering
\begin{tabular}{|c|c|c|}
\hline
Random variable &Value &Definition\\ \hline
\multirow{3}{*}{X} &0 &Slips of Rs 1\\
&1 &Slips of Rs 5\\
&2 &Slips of Rs 13\\ \hline
\multirow{2}{*}{Y} &0 &Box A\\
&1 &Box B\\\hline
\end{tabular}
\caption{}
\label{tab:Distribution}
\end{table}
See \tabref{tab:Distribution}.
\begin{align}
p_{Y}\brak{k}= \begin{cases} 
      \frac{1}{3} & {k=0} \\
      \frac{2}{3 }& {k=1} 
   \end{cases}
   \\
p_{Y|X}\brak{0|0} = \frac{19}{25}\, 
p_{Y|X}\brak{0|1} = \frac{6}{25}\,
p_{Y|X}\brak{1|0} = \frac{45}{50}\,
p_{Y|X}\brak{1|2} = \frac{5}{50}
\end{align}
The desired probability is the probability that a slip drawn at random is marked other than Rs 1,
\begin{align}
&=1-p_X\brak{0}\\
&= p_X(1) + p_X(2)
\end{align}
Using Bayes theorem,
\begin{align}
&= p_Y\brak{0} \times \pr{Y=0 | X=1} + p_Y\brak{1} \times \pr{Y=1|X=2}\\
&=\frac{1}{3} \times \frac{6}{25} + \frac{2}{3} \times \frac{5}{50}\\
&=\frac{11}{75}
\end{align}

\newpage

%\tableofcontents

\bigskip

\renewcommand{\thefigure}{\theenumi}
\renewcommand{\thetable}{\theenumi}
%\renewcommand{\theequation}{\theenumi}

%\begin{abstract}
%%\boldmath
%In this letter, an algorithm for evaluating the exact analytical bit error rate  (BER)  for the piecewise linear (PL) combiner for  multiple relays is presented. Previous results were available only for upto three relays. The algorithm is unique in the sense that  the actual mathematical expressions, that are prohibitively large, need not be explicitly obtained. The diversity gain due to multiple relays is shown through plots of the analytical BER, well supported by simulations. 
%
%\end{abstract}
% IEEEtran.cls defaults to using nonbold math in the Abstract.
% This preserves the distinction between vectors and scalars. However,
% if the journal you are submitting to favors bold math in the abstract,
% then you can use LaTeX's standard command \boldmath at the very start
% of the abstract to achieve this. Many IEEE journals frown on math
% in the abstract anyway.

% Note that keywords are not normally used for peerreview papers.
%\begin{IEEEkeywords}
%Cooperative diversity, decode and forward, piecewise linear
%\end{IEEEkeywords}



% For peer review papers, you can put extra information on the cover
% page as needed:
% \ifCLASSOPTIONpeerreview
% \begin{center} \bfseries EDICS Category: 3-BBND \end{center}
% \fi
%
% For peerreview papers, this IEEEtran command inserts a page break and
% creates the second title. It will be ignored for other modes.
%\IEEEpeerreviewmaketitle




\item A card is selected from a pack of 52 cards\\
\begin{enumerate}[label=(\alph*)]
\item How many points are there in the sample space?
\item Calculate the probability that the cards is an ace of spades.
\item Calculate the probability that the card is (i) an ace (ii)black card.\\
\end{enumerate}
%\input{ncert/11/16/3/4_1/Prob_4.tex}
\item In a non-leap year, the probability of having 53 tuesdays or 53 wednesdays is\\
\solution
%A non-leap year has a total of 365 days, and a week has 7 days.\\
So it can be expressed as 
\begin{align}
365\text{days} &=52\times 7+1 \text{day}
\end{align}
$\implies$ 52 tuesdays or wednesdays\\
Random variable X denotes the days of a week
\begin{align}
p_X\brak{k}&=\frac{1}{7}; \quad \brak{1<k<7}
\end{align}
So the probability of extra day being tuesday or wednesday is
\begin{align}
p_X\brak{3}+p_X\brak{4}&=\frac{1}{7}+\frac{1}{7}=\frac{2}{7}
\end{align}



\item There are 1000 sealed envelopes in a box, 10 of them contain a cash prize of
Rs 100 each, 100 of them contain a cash prize of Rs 50 each and 200 of them
contain a cash prize of Rs 10 each and rest do not contain any cash prize. If they
are well shuffled and an envelope is picked up out, what is the probability that it
contains no cash prize?\\
\solution
%\begin{table}[H]
	\centering
\begin{tabular}{|c|c|c|}
\hline
Random variable &Value &Definition\\ \hline
\multirow{3}{*}{X} &0 &Slips of Rs 1\\
&1 &Slips of Rs 5\\
&2 &Slips of Rs 13\\ \hline
\multirow{2}{*}{Y} &0 &Box A\\
&1 &Box B\\\hline
\end{tabular}
\caption{}
\label{tab:Distribution}
\end{table}
See \tabref{tab:Distribution}.
\begin{align}
p_{Y}\brak{k}= \begin{cases} 
      \frac{1}{3} & {k=0} \\
      \frac{2}{3 }& {k=1} 
   \end{cases}
   \\
p_{Y|X}\brak{0|0} = \frac{19}{25}\, 
p_{Y|X}\brak{0|1} = \frac{6}{25}\,
p_{Y|X}\brak{1|0} = \frac{45}{50}\,
p_{Y|X}\brak{1|2} = \frac{5}{50}
\end{align}
The desired probability is the probability that a slip drawn at random is marked other than Rs 1,
\begin{align}
&=1-p_X\brak{0}\\
&= p_X(1) + p_X(2)
\end{align}
Using Bayes theorem,
\begin{align}
&= p_Y\brak{0} \times \pr{Y=0 | X=1} + p_Y\brak{1} \times \pr{Y=1|X=2}\\
&=\frac{1}{3} \times \frac{6}{25} + \frac{2}{3} \times \frac{5}{50}\\
&=\frac{11}{75}
\end{align}

\newpage

%\tableofcontents

\bigskip

\renewcommand{\thefigure}{\theenumi}
\renewcommand{\thetable}{\theenumi}
%\renewcommand{\theequation}{\theenumi}

%\begin{abstract}
%%\boldmath
%In this letter, an algorithm for evaluating the exact analytical bit error rate  (BER)  for the piecewise linear (PL) combiner for  multiple relays is presented. Previous results were available only for upto three relays. The algorithm is unique in the sense that  the actual mathematical expressions, that are prohibitively large, need not be explicitly obtained. The diversity gain due to multiple relays is shown through plots of the analytical BER, well supported by simulations. 
%
%\end{abstract}
% IEEEtran.cls defaults to using nonbold math in the Abstract.
% This preserves the distinction between vectors and scalars. However,
% if the journal you are submitting to favors bold math in the abstract,
% then you can use LaTeX's standard command \boldmath at the very start
% of the abstract to achieve this. Many IEEE journals frown on math
% in the abstract anyway.

% Note that keywords are not normally used for peerreview papers.
%\begin{IEEEkeywords}
%Cooperative diversity, decode and forward, piecewise linear
%\end{IEEEkeywords}



% For peer review papers, you can put extra information on the cover
% page as needed:
% \ifCLASSOPTIONpeerreview
% \begin{center} \bfseries EDICS Category: 3-BBND \end{center}
% \fi
%
% For peerreview papers, this IEEEtran command inserts a page break and
% creates the second title. It will be ignored for other modes.
%\IEEEpeerreviewmaketitle




\item 
A die is thrown and a card is selected at random from a deck of 52 playing cards. The probability of getting an even number on the die and a spade card.\\
\solution
%\begin{table}[H]
	\centering
\begin{tabular}{|c|c|c|}
\hline
Random variable &Value &Definition\\ \hline
\multirow{3}{*}{X} &0 &Slips of Rs 1\\
&1 &Slips of Rs 5\\
&2 &Slips of Rs 13\\ \hline
\multirow{2}{*}{Y} &0 &Box A\\
&1 &Box B\\\hline
\end{tabular}
\caption{}
\label{tab:Distribution}
\end{table}
See \tabref{tab:Distribution}.
\begin{align}
p_{Y}\brak{k}= \begin{cases} 
      \frac{1}{3} & {k=0} \\
      \frac{2}{3 }& {k=1} 
   \end{cases}
   \\
p_{Y|X}\brak{0|0} = \frac{19}{25}\, 
p_{Y|X}\brak{0|1} = \frac{6}{25}\,
p_{Y|X}\brak{1|0} = \frac{45}{50}\,
p_{Y|X}\brak{1|2} = \frac{5}{50}
\end{align}
The desired probability is the probability that a slip drawn at random is marked other than Rs 1,
\begin{align}
&=1-p_X\brak{0}\\
&= p_X(1) + p_X(2)
\end{align}
Using Bayes theorem,
\begin{align}
&= p_Y\brak{0} \times \pr{Y=0 | X=1} + p_Y\brak{1} \times \pr{Y=1|X=2}\\
&=\frac{1}{3} \times \frac{6}{25} + \frac{2}{3} \times \frac{5}{50}\\
&=\frac{11}{75}
\end{align}

\newpage

%\tableofcontents

\bigskip

\renewcommand{\thefigure}{\theenumi}
\renewcommand{\thetable}{\theenumi}
%\renewcommand{\theequation}{\theenumi}

%\begin{abstract}
%%\boldmath
%In this letter, an algorithm for evaluating the exact analytical bit error rate  (BER)  for the piecewise linear (PL) combiner for  multiple relays is presented. Previous results were available only for upto three relays. The algorithm is unique in the sense that  the actual mathematical expressions, that are prohibitively large, need not be explicitly obtained. The diversity gain due to multiple relays is shown through plots of the analytical BER, well supported by simulations. 
%
%\end{abstract}
% IEEEtran.cls defaults to using nonbold math in the Abstract.
% This preserves the distinction between vectors and scalars. However,
% if the journal you are submitting to favors bold math in the abstract,
% then you can use LaTeX's standard command \boldmath at the very start
% of the abstract to achieve this. Many IEEE journals frown on math
% in the abstract anyway.

% Note that keywords are not normally used for peerreview papers.
%\begin{IEEEkeywords}
%Cooperative diversity, decode and forward, piecewise linear
%\end{IEEEkeywords}



% For peer review papers, you can put extra information on the cover
% page as needed:
% \ifCLASSOPTIONpeerreview
% \begin{center} \bfseries EDICS Category: 3-BBND \end{center}
% \fi
%
% For peerreview papers, this IEEEtran command inserts a page break and
% creates the second title. It will be ignored for other modes.
%\IEEEpeerreviewmaketitle




\item
If 4-digit numbers greater than 5,000 are randomly formed from the digits 0, 1, 3, 5, and 7, what is the probability of forming a number divisible by 5 when:
\begin{enumerate}
    \item The digits are repeated?
    \item The repetition of digits is not allowed?
\end{enumerate}
\solution
%\begin{table}[H]
	\centering
\begin{tabular}{|c|c|c|}
\hline
Random variable &Value &Definition\\ \hline
\multirow{3}{*}{X} &0 &Slips of Rs 1\\
&1 &Slips of Rs 5\\
&2 &Slips of Rs 13\\ \hline
\multirow{2}{*}{Y} &0 &Box A\\
&1 &Box B\\\hline
\end{tabular}
\caption{}
\label{tab:Distribution}
\end{table}
See \tabref{tab:Distribution}.
\begin{align}
p_{Y}\brak{k}= \begin{cases} 
      \frac{1}{3} & {k=0} \\
      \frac{2}{3 }& {k=1} 
   \end{cases}
   \\
p_{Y|X}\brak{0|0} = \frac{19}{25}\, 
p_{Y|X}\brak{0|1} = \frac{6}{25}\,
p_{Y|X}\brak{1|0} = \frac{45}{50}\,
p_{Y|X}\brak{1|2} = \frac{5}{50}
\end{align}
The desired probability is the probability that a slip drawn at random is marked other than Rs 1,
\begin{align}
&=1-p_X\brak{0}\\
&= p_X(1) + p_X(2)
\end{align}
Using Bayes theorem,
\begin{align}
&= p_Y\brak{0} \times \pr{Y=0 | X=1} + p_Y\brak{1} \times \pr{Y=1|X=2}\\
&=\frac{1}{3} \times \frac{6}{25} + \frac{2}{3} \times \frac{5}{50}\\
&=\frac{11}{75}
\end{align}

\newpage

%\tableofcontents

\bigskip

\renewcommand{\thefigure}{\theenumi}
\renewcommand{\thetable}{\theenumi}
%\renewcommand{\theequation}{\theenumi}

%\begin{abstract}
%%\boldmath
%In this letter, an algorithm for evaluating the exact analytical bit error rate  (BER)  for the piecewise linear (PL) combiner for  multiple relays is presented. Previous results were available only for upto three relays. The algorithm is unique in the sense that  the actual mathematical expressions, that are prohibitively large, need not be explicitly obtained. The diversity gain due to multiple relays is shown through plots of the analytical BER, well supported by simulations. 
%
%\end{abstract}
% IEEEtran.cls defaults to using nonbold math in the Abstract.
% This preserves the distinction between vectors and scalars. However,
% if the journal you are submitting to favors bold math in the abstract,
% then you can use LaTeX's standard command \boldmath at the very start
% of the abstract to achieve this. Many IEEE journals frown on math
% in the abstract anyway.

% Note that keywords are not normally used for peerreview papers.
%\begin{IEEEkeywords}
%Cooperative diversity, decode and forward, piecewise linear
%\end{IEEEkeywords}



% For peer review papers, you can put extra information on the cover
% page as needed:
% \ifCLASSOPTIONpeerreview
% \begin{center} \bfseries EDICS Category: 3-BBND \end{center}
% \fi
%
% For peerreview papers, this IEEEtran command inserts a page break and
% creates the second title. It will be ignored for other modes.
%\IEEEpeerreviewmaketitle




\item Consider the probability space $\brak{\Omega, \mathcal{G}, P}$ where $\Omega = [0,2]$ and $\mathcal{G} = \cbrak{\phi, \Omega, [0,1], (1,2]}$. Let $X$ and $Y$ be two functions on $\Omega$ defined as
\begin{align*}
    X(\omega) = 
    \begin{cases}
        1 & \text{if }\omega \in [0, 1]\\
        2 & \text{if }\omega \in (1, 2]
    \end{cases}
\end{align*}
and
\begin{align*}
    Y(\omega) = 
    \begin{cases}
        2 & \text{if }\omega \in [0, 1.5]\\
        3 & \text{if }\omega \in (1.5, 2].
    \end{cases}
\end{align*}
Then which one of the following statements is true?
\begin{enumerate}
    \item [(A)] $X$ is a random variable with respect to $\mathcal{G}$, but $Y$ is not a random variable with respect to $\mathcal{G}$.
    \item [(B)] $Y$ is a random variable with respect to $\mathcal{G}$, but $X$ is not a random variable with respect to $\mathcal{G}$.
    \item [(C)] Neither $X$ nor $Y$ is a random variable with respect to $\mathcal{G}$.
    \item [(D)] Both $X$ and $Y$ are random variables with respect to $\mathcal{G}$.
\end{enumerate} \hfill (GATE ST 2023)\\
\solution
%\begin{table}[H]
	\centering
\begin{tabular}{|c|c|c|}
\hline
Random variable &Value &Definition\\ \hline
\multirow{3}{*}{X} &0 &Slips of Rs 1\\
&1 &Slips of Rs 5\\
&2 &Slips of Rs 13\\ \hline
\multirow{2}{*}{Y} &0 &Box A\\
&1 &Box B\\\hline
\end{tabular}
\caption{}
\label{tab:Distribution}
\end{table}
See \tabref{tab:Distribution}.
\begin{align}
p_{Y}\brak{k}= \begin{cases} 
      \frac{1}{3} & {k=0} \\
      \frac{2}{3 }& {k=1} 
   \end{cases}
   \\
p_{Y|X}\brak{0|0} = \frac{19}{25}\, 
p_{Y|X}\brak{0|1} = \frac{6}{25}\,
p_{Y|X}\brak{1|0} = \frac{45}{50}\,
p_{Y|X}\brak{1|2} = \frac{5}{50}
\end{align}
The desired probability is the probability that a slip drawn at random is marked other than Rs 1,
\begin{align}
&=1-p_X\brak{0}\\
&= p_X(1) + p_X(2)
\end{align}
Using Bayes theorem,
\begin{align}
&= p_Y\brak{0} \times \pr{Y=0 | X=1} + p_Y\brak{1} \times \pr{Y=1|X=2}\\
&=\frac{1}{3} \times \frac{6}{25} + \frac{2}{3} \times \frac{5}{50}\\
&=\frac{11}{75}
\end{align}

\newpage

%\tableofcontents

\bigskip

\renewcommand{\thefigure}{\theenumi}
\renewcommand{\thetable}{\theenumi}
%\renewcommand{\theequation}{\theenumi}

%\begin{abstract}
%%\boldmath
%In this letter, an algorithm for evaluating the exact analytical bit error rate  (BER)  for the piecewise linear (PL) combiner for  multiple relays is presented. Previous results were available only for upto three relays. The algorithm is unique in the sense that  the actual mathematical expressions, that are prohibitively large, need not be explicitly obtained. The diversity gain due to multiple relays is shown through plots of the analytical BER, well supported by simulations. 
%
%\end{abstract}
% IEEEtran.cls defaults to using nonbold math in the Abstract.
% This preserves the distinction between vectors and scalars. However,
% if the journal you are submitting to favors bold math in the abstract,
% then you can use LaTeX's standard command \boldmath at the very start
% of the abstract to achieve this. Many IEEE journals frown on math
% in the abstract anyway.

% Note that keywords are not normally used for peerreview papers.
%\begin{IEEEkeywords}
%Cooperative diversity, decode and forward, piecewise linear
%\end{IEEEkeywords}



% For peer review papers, you can put extra information on the cover
% page as needed:
% \ifCLASSOPTIONpeerreview
% \begin{center} \bfseries EDICS Category: 3-BBND \end{center}
% \fi
%
% For peerreview papers, this IEEEtran command inserts a page break and
% creates the second title. It will be ignored for other modes.
%\IEEEpeerreviewmaketitle




	\item  A die is loaded in such a way that each odd number is twice as likely to occur as
each even number. Find $P(G)$, where $G$ is the event that a number greater than
3 occurs on a single roll of the die.
\\
\solution
		%\begin{table}[H]
	\centering
\begin{tabular}{|c|c|c|}
\hline
Random variable &Value &Definition\\ \hline
\multirow{3}{*}{X} &0 &Slips of Rs 1\\
&1 &Slips of Rs 5\\
&2 &Slips of Rs 13\\ \hline
\multirow{2}{*}{Y} &0 &Box A\\
&1 &Box B\\\hline
\end{tabular}
\caption{}
\label{tab:Distribution}
\end{table}
See \tabref{tab:Distribution}.
\begin{align}
p_{Y}\brak{k}= \begin{cases} 
      \frac{1}{3} & {k=0} \\
      \frac{2}{3 }& {k=1} 
   \end{cases}
   \\
p_{Y|X}\brak{0|0} = \frac{19}{25}\, 
p_{Y|X}\brak{0|1} = \frac{6}{25}\,
p_{Y|X}\brak{1|0} = \frac{45}{50}\,
p_{Y|X}\brak{1|2} = \frac{5}{50}
\end{align}
The desired probability is the probability that a slip drawn at random is marked other than Rs 1,
\begin{align}
&=1-p_X\brak{0}\\
&= p_X(1) + p_X(2)
\end{align}
Using Bayes theorem,
\begin{align}
&= p_Y\brak{0} \times \pr{Y=0 | X=1} + p_Y\brak{1} \times \pr{Y=1|X=2}\\
&=\frac{1}{3} \times \frac{6}{25} + \frac{2}{3} \times \frac{5}{50}\\
&=\frac{11}{75}
\end{align}

\newpage

%\tableofcontents

\bigskip

\renewcommand{\thefigure}{\theenumi}
\renewcommand{\thetable}{\theenumi}
%\renewcommand{\theequation}{\theenumi}

%\begin{abstract}
%%\boldmath
%In this letter, an algorithm for evaluating the exact analytical bit error rate  (BER)  for the piecewise linear (PL) combiner for  multiple relays is presented. Previous results were available only for upto three relays. The algorithm is unique in the sense that  the actual mathematical expressions, that are prohibitively large, need not be explicitly obtained. The diversity gain due to multiple relays is shown through plots of the analytical BER, well supported by simulations. 
%
%\end{abstract}
% IEEEtran.cls defaults to using nonbold math in the Abstract.
% This preserves the distinction between vectors and scalars. However,
% if the journal you are submitting to favors bold math in the abstract,
% then you can use LaTeX's standard command \boldmath at the very start
% of the abstract to achieve this. Many IEEE journals frown on math
% in the abstract anyway.

% Note that keywords are not normally used for peerreview papers.
%\begin{IEEEkeywords}
%Cooperative diversity, decode and forward, piecewise linear
%\end{IEEEkeywords}



% For peer review papers, you can put extra information on the cover
% page as needed:
% \ifCLASSOPTIONpeerreview
% \begin{center} \bfseries EDICS Category: 3-BBND \end{center}
% \fi
%
% For peerreview papers, this IEEEtran command inserts a page break and
% creates the second title. It will be ignored for other modes.
%\IEEEpeerreviewmaketitle




	\item All the jacks, queens and kings are removed from a deck of 52 playing cards. The remaining cards are well shuffled and then one card is drawn at random. Giving ace a value 1 similar value for other cards, find the probability that the card has a value 
		\begin{enumerate}
			\item 7
			\item greater than 7
			\item less than 7
		\end{enumerate}
		%Number of cards left after removing all jacks, queens and kings 
\begin{align}
N	= 52 - 4\times 3
	= 40
\end{align}
%\begin{table}[H]
%\def\arraystretch{1.2}
%\begin{tabular}{|c|c|c|}
%\hline
%	\textbf{Parameter} &\textbf{Value} &\textbf{Description}\\ \hline
%	$X$ &1-10 &Represents the value of the card picked \\ \hline
%\end{tabular}
%\end{table}
Let $1 \le X \le 10$ be the value of the card picked.  Then,
\begin{align}
	p_X(k) &= \Pr(X=k)\ \forall\ 1 \leq k \leq 10\\
	&= \frac{4\times 1}{40}\\
	&= \frac{1}{10}\\
	\therefore p_X(k) &= 
	\begin{cases}
		\frac{1}{10} & 1 \leq k \leq 10\\
		0 & \text{otherwise}
	\end{cases}
\end{align}
and
\begin{align}
	F_{X}(k) &= \sum_{m=0}^{k}p_{X}(m) \quad 1 \leq k \leq 10\\
	&= \frac{k}{10}\\
	\therefore F_{X}(k) &= 
	\begin{cases}
		0 & k \leq 0\\
		\frac{k}{10} & 1\leq k \leq 10\\
		1 & k > 10 
	\end{cases}
\end{align}
\begin{enumerate}
	\item Probability that card has value equal to 7 is
		\begin{align}
			 p_{X}(7)
			= \frac{1}{10}
		\end{align}
	\item Probability that card has value greater than 7 is
		\begin{align}
			1 - F_X(7)
			&= 1 - \frac{7}{10}
			\\
			&= \frac{3}{10}
		\end{align}
	\item Probability that card has value less than 7 is
		\begin{align}
			 F_{X}(6)
			=\frac{6}{10}
		\end{align}
\end{enumerate}

  \item A Lot consists of 48 mobile phones of which 42 are good, 3 have only minor defects and 3 have major defects.Varnika will buy a phone if it is good but the trader will only buy a mobile if it has no major defects. One phone is selected at random from the lot. What is the probability that it is
\begin{enumerate}
	\item acceptable to Varnika?
            \item acceptable to the trader?
\end{enumerate}
\solution
	%\begin{table}[H]
	\centering
\begin{tabular}{|c|c|c|}
\hline
Random variable &Value &Definition\\ \hline
\multirow{3}{*}{X} &0 &Slips of Rs 1\\
&1 &Slips of Rs 5\\
&2 &Slips of Rs 13\\ \hline
\multirow{2}{*}{Y} &0 &Box A\\
&1 &Box B\\\hline
\end{tabular}
\caption{}
\label{tab:Distribution}
\end{table}
See \tabref{tab:Distribution}.
\begin{align}
p_{Y}\brak{k}= \begin{cases} 
      \frac{1}{3} & {k=0} \\
      \frac{2}{3 }& {k=1} 
   \end{cases}
   \\
p_{Y|X}\brak{0|0} = \frac{19}{25}\, 
p_{Y|X}\brak{0|1} = \frac{6}{25}\,
p_{Y|X}\brak{1|0} = \frac{45}{50}\,
p_{Y|X}\brak{1|2} = \frac{5}{50}
\end{align}
The desired probability is the probability that a slip drawn at random is marked other than Rs 1,
\begin{align}
&=1-p_X\brak{0}\\
&= p_X(1) + p_X(2)
\end{align}
Using Bayes theorem,
\begin{align}
&= p_Y\brak{0} \times \pr{Y=0 | X=1} + p_Y\brak{1} \times \pr{Y=1|X=2}\\
&=\frac{1}{3} \times \frac{6}{25} + \frac{2}{3} \times \frac{5}{50}\\
&=\frac{11}{75}
\end{align}

\newpage

%\tableofcontents

\bigskip

\renewcommand{\thefigure}{\theenumi}
\renewcommand{\thetable}{\theenumi}
%\renewcommand{\theequation}{\theenumi}

%\begin{abstract}
%%\boldmath
%In this letter, an algorithm for evaluating the exact analytical bit error rate  (BER)  for the piecewise linear (PL) combiner for  multiple relays is presented. Previous results were available only for upto three relays. The algorithm is unique in the sense that  the actual mathematical expressions, that are prohibitively large, need not be explicitly obtained. The diversity gain due to multiple relays is shown through plots of the analytical BER, well supported by simulations. 
%
%\end{abstract}
% IEEEtran.cls defaults to using nonbold math in the Abstract.
% This preserves the distinction between vectors and scalars. However,
% if the journal you are submitting to favors bold math in the abstract,
% then you can use LaTeX's standard command \boldmath at the very start
% of the abstract to achieve this. Many IEEE journals frown on math
% in the abstract anyway.

% Note that keywords are not normally used for peerreview papers.
%\begin{IEEEkeywords}
%Cooperative diversity, decode and forward, piecewise linear
%\end{IEEEkeywords}



% For peer review papers, you can put extra information on the cover
% page as needed:
% \ifCLASSOPTIONpeerreview
% \begin{center} \bfseries EDICS Category: 3-BBND \end{center}
% \fi
%
% For peerreview papers, this IEEEtran command inserts a page break and
% creates the second title. It will be ignored for other modes.
%\IEEEpeerreviewmaketitle




 \item A student says that if you throw a die, it will show up 1 or not 1. Therefore, the probability of getting 1 and the probability of getting 'not 1' each is equal to $\frac{1}{2}$. Is this correct? Give reasons.\\
 \solution
        %\begin{table}[H]
	\centering
\begin{tabular}{|c|c|c|}
\hline
Random variable &Value &Definition\\ \hline
\multirow{3}{*}{X} &0 &Slips of Rs 1\\
&1 &Slips of Rs 5\\
&2 &Slips of Rs 13\\ \hline
\multirow{2}{*}{Y} &0 &Box A\\
&1 &Box B\\\hline
\end{tabular}
\caption{}
\label{tab:Distribution}
\end{table}
See \tabref{tab:Distribution}.
\begin{align}
p_{Y}\brak{k}= \begin{cases} 
      \frac{1}{3} & {k=0} \\
      \frac{2}{3 }& {k=1} 
   \end{cases}
   \\
p_{Y|X}\brak{0|0} = \frac{19}{25}\, 
p_{Y|X}\brak{0|1} = \frac{6}{25}\,
p_{Y|X}\brak{1|0} = \frac{45}{50}\,
p_{Y|X}\brak{1|2} = \frac{5}{50}
\end{align}
The desired probability is the probability that a slip drawn at random is marked other than Rs 1,
\begin{align}
&=1-p_X\brak{0}\\
&= p_X(1) + p_X(2)
\end{align}
Using Bayes theorem,
\begin{align}
&= p_Y\brak{0} \times \pr{Y=0 | X=1} + p_Y\brak{1} \times \pr{Y=1|X=2}\\
&=\frac{1}{3} \times \frac{6}{25} + \frac{2}{3} \times \frac{5}{50}\\
&=\frac{11}{75}
\end{align}

\newpage

%\tableofcontents

\bigskip

\renewcommand{\thefigure}{\theenumi}
\renewcommand{\thetable}{\theenumi}
%\renewcommand{\theequation}{\theenumi}

%\begin{abstract}
%%\boldmath
%In this letter, an algorithm for evaluating the exact analytical bit error rate  (BER)  for the piecewise linear (PL) combiner for  multiple relays is presented. Previous results were available only for upto three relays. The algorithm is unique in the sense that  the actual mathematical expressions, that are prohibitively large, need not be explicitly obtained. The diversity gain due to multiple relays is shown through plots of the analytical BER, well supported by simulations. 
%
%\end{abstract}
% IEEEtran.cls defaults to using nonbold math in the Abstract.
% This preserves the distinction between vectors and scalars. However,
% if the journal you are submitting to favors bold math in the abstract,
% then you can use LaTeX's standard command \boldmath at the very start
% of the abstract to achieve this. Many IEEE journals frown on math
% in the abstract anyway.

% Note that keywords are not normally used for peerreview papers.
%\begin{IEEEkeywords}
%Cooperative diversity, decode and forward, piecewise linear
%\end{IEEEkeywords}



% For peer review papers, you can put extra information on the cover
% page as needed:
% \ifCLASSOPTIONpeerreview
% \begin{center} \bfseries EDICS Category: 3-BBND \end{center}
% \fi
%
% For peerreview papers, this IEEEtran command inserts a page break and
% creates the second title. It will be ignored for other modes.
%\IEEEpeerreviewmaketitle




   \item Four candidates A, B, C, D have ap-
plied for the assignment to coach a school cricket
team. If A is twice as likely to be selected as B, and
B and C are given about the same chance of being
selected, while C is twice as likely to be selected
as D, what are the probabilities that
\begin{enumerate}
\item C will be selected?
\item A will not be selected?
\end{enumerate}
	%\begin{table}[H]
	\centering
\begin{tabular}{|c|c|c|}
\hline
Random variable &Value &Definition\\ \hline
\multirow{3}{*}{X} &0 &Slips of Rs 1\\
&1 &Slips of Rs 5\\
&2 &Slips of Rs 13\\ \hline
\multirow{2}{*}{Y} &0 &Box A\\
&1 &Box B\\\hline
\end{tabular}
\caption{}
\label{tab:Distribution}
\end{table}
See \tabref{tab:Distribution}.
\begin{align}
p_{Y}\brak{k}= \begin{cases} 
      \frac{1}{3} & {k=0} \\
      \frac{2}{3 }& {k=1} 
   \end{cases}
   \\
p_{Y|X}\brak{0|0} = \frac{19}{25}\, 
p_{Y|X}\brak{0|1} = \frac{6}{25}\,
p_{Y|X}\brak{1|0} = \frac{45}{50}\,
p_{Y|X}\brak{1|2} = \frac{5}{50}
\end{align}
The desired probability is the probability that a slip drawn at random is marked other than Rs 1,
\begin{align}
&=1-p_X\brak{0}\\
&= p_X(1) + p_X(2)
\end{align}
Using Bayes theorem,
\begin{align}
&= p_Y\brak{0} \times \pr{Y=0 | X=1} + p_Y\brak{1} \times \pr{Y=1|X=2}\\
&=\frac{1}{3} \times \frac{6}{25} + \frac{2}{3} \times \frac{5}{50}\\
&=\frac{11}{75}
\end{align}

\newpage

%\tableofcontents

\bigskip

\renewcommand{\thefigure}{\theenumi}
\renewcommand{\thetable}{\theenumi}
%\renewcommand{\theequation}{\theenumi}

%\begin{abstract}
%%\boldmath
%In this letter, an algorithm for evaluating the exact analytical bit error rate  (BER)  for the piecewise linear (PL) combiner for  multiple relays is presented. Previous results were available only for upto three relays. The algorithm is unique in the sense that  the actual mathematical expressions, that are prohibitively large, need not be explicitly obtained. The diversity gain due to multiple relays is shown through plots of the analytical BER, well supported by simulations. 
%
%\end{abstract}
% IEEEtran.cls defaults to using nonbold math in the Abstract.
% This preserves the distinction between vectors and scalars. However,
% if the journal you are submitting to favors bold math in the abstract,
% then you can use LaTeX's standard command \boldmath at the very start
% of the abstract to achieve this. Many IEEE journals frown on math
% in the abstract anyway.

% Note that keywords are not normally used for peerreview papers.
%\begin{IEEEkeywords}
%Cooperative diversity, decode and forward, piecewise linear
%\end{IEEEkeywords}



% For peer review papers, you can put extra information on the cover
% page as needed:
% \ifCLASSOPTIONpeerreview
% \begin{center} \bfseries EDICS Category: 3-BBND \end{center}
% \fi
%
% For peerreview papers, this IEEEtran command inserts a page break and
% creates the second title. It will be ignored for other modes.
%\IEEEpeerreviewmaketitle




 \item A bag contain 24 balls of which $x$ balls are red, $2x$ are white and $3x$ are blue. A ball is selected at random, What is the probability that it is
\begin{enumerate}[label=\alph*)]
\item not red ?
\item white ?
\end{enumerate}
%\begin{table}[H]
	\centering
\begin{tabular}{|c|c|c|}
\hline
Random variable &Value &Definition\\ \hline
\multirow{3}{*}{X} &0 &Slips of Rs 1\\
&1 &Slips of Rs 5\\
&2 &Slips of Rs 13\\ \hline
\multirow{2}{*}{Y} &0 &Box A\\
&1 &Box B\\\hline
\end{tabular}
\caption{}
\label{tab:Distribution}
\end{table}
See \tabref{tab:Distribution}.
\begin{align}
p_{Y}\brak{k}= \begin{cases} 
      \frac{1}{3} & {k=0} \\
      \frac{2}{3 }& {k=1} 
   \end{cases}
   \\
p_{Y|X}\brak{0|0} = \frac{19}{25}\, 
p_{Y|X}\brak{0|1} = \frac{6}{25}\,
p_{Y|X}\brak{1|0} = \frac{45}{50}\,
p_{Y|X}\brak{1|2} = \frac{5}{50}
\end{align}
The desired probability is the probability that a slip drawn at random is marked other than Rs 1,
\begin{align}
&=1-p_X\brak{0}\\
&= p_X(1) + p_X(2)
\end{align}
Using Bayes theorem,
\begin{align}
&= p_Y\brak{0} \times \pr{Y=0 | X=1} + p_Y\brak{1} \times \pr{Y=1|X=2}\\
&=\frac{1}{3} \times \frac{6}{25} + \frac{2}{3} \times \frac{5}{50}\\
&=\frac{11}{75}
\end{align}

\newpage

%\tableofcontents

\bigskip

\renewcommand{\thefigure}{\theenumi}
\renewcommand{\thetable}{\theenumi}
%\renewcommand{\theequation}{\theenumi}

%\begin{abstract}
%%\boldmath
%In this letter, an algorithm for evaluating the exact analytical bit error rate  (BER)  for the piecewise linear (PL) combiner for  multiple relays is presented. Previous results were available only for upto three relays. The algorithm is unique in the sense that  the actual mathematical expressions, that are prohibitively large, need not be explicitly obtained. The diversity gain due to multiple relays is shown through plots of the analytical BER, well supported by simulations. 
%
%\end{abstract}
% IEEEtran.cls defaults to using nonbold math in the Abstract.
% This preserves the distinction between vectors and scalars. However,
% if the journal you are submitting to favors bold math in the abstract,
% then you can use LaTeX's standard command \boldmath at the very start
% of the abstract to achieve this. Many IEEE journals frown on math
% in the abstract anyway.

% Note that keywords are not normally used for peerreview papers.
%\begin{IEEEkeywords}
%Cooperative diversity, decode and forward, piecewise linear
%\end{IEEEkeywords}



% For peer review papers, you can put extra information on the cover
% page as needed:
% \ifCLASSOPTIONpeerreview
% \begin{center} \bfseries EDICS Category: 3-BBND \end{center}
% \fi
%
% For peerreview papers, this IEEEtran command inserts a page break and
% creates the second title. It will be ignored for other modes.
%\IEEEpeerreviewmaketitle




If the letters of the word ASSASSINATION are arranged at random. Find the Probability that
\begin{enumerate}[label=(\alph*)]
\item Four $S's$ come consecutively in the word
\item Two  $I's$ and two $N's$ come together
\item All $A's$ are not coming together
\item No two $A's$ are coming together
\end{enumerate}
%\begin{table}[H]
	\centering
\begin{tabular}{|c|c|c|}
\hline
Random variable &Value &Definition\\ \hline
\multirow{3}{*}{X} &0 &Slips of Rs 1\\
&1 &Slips of Rs 5\\
&2 &Slips of Rs 13\\ \hline
\multirow{2}{*}{Y} &0 &Box A\\
&1 &Box B\\\hline
\end{tabular}
\caption{}
\label{tab:Distribution}
\end{table}
See \tabref{tab:Distribution}.
\begin{align}
p_{Y}\brak{k}= \begin{cases} 
      \frac{1}{3} & {k=0} \\
      \frac{2}{3 }& {k=1} 
   \end{cases}
   \\
p_{Y|X}\brak{0|0} = \frac{19}{25}\, 
p_{Y|X}\brak{0|1} = \frac{6}{25}\,
p_{Y|X}\brak{1|0} = \frac{45}{50}\,
p_{Y|X}\brak{1|2} = \frac{5}{50}
\end{align}
The desired probability is the probability that a slip drawn at random is marked other than Rs 1,
\begin{align}
&=1-p_X\brak{0}\\
&= p_X(1) + p_X(2)
\end{align}
Using Bayes theorem,
\begin{align}
&= p_Y\brak{0} \times \pr{Y=0 | X=1} + p_Y\brak{1} \times \pr{Y=1|X=2}\\
&=\frac{1}{3} \times \frac{6}{25} + \frac{2}{3} \times \frac{5}{50}\\
&=\frac{11}{75}
\end{align}

\newpage

%\tableofcontents

\bigskip

\renewcommand{\thefigure}{\theenumi}
\renewcommand{\thetable}{\theenumi}
%\renewcommand{\theequation}{\theenumi}

%\begin{abstract}
%%\boldmath
%In this letter, an algorithm for evaluating the exact analytical bit error rate  (BER)  for the piecewise linear (PL) combiner for  multiple relays is presented. Previous results were available only for upto three relays. The algorithm is unique in the sense that  the actual mathematical expressions, that are prohibitively large, need not be explicitly obtained. The diversity gain due to multiple relays is shown through plots of the analytical BER, well supported by simulations. 
%
%\end{abstract}
% IEEEtran.cls defaults to using nonbold math in the Abstract.
% This preserves the distinction between vectors and scalars. However,
% if the journal you are submitting to favors bold math in the abstract,
% then you can use LaTeX's standard command \boldmath at the very start
% of the abstract to achieve this. Many IEEE journals frown on math
% in the abstract anyway.

% Note that keywords are not normally used for peerreview papers.
%\begin{IEEEkeywords}
%Cooperative diversity, decode and forward, piecewise linear
%\end{IEEEkeywords}



% For peer review papers, you can put extra information on the cover
% page as needed:
% \ifCLASSOPTIONpeerreview
% \begin{center} \bfseries EDICS Category: 3-BBND \end{center}
% \fi
%
% For peerreview papers, this IEEEtran command inserts a page break and
% creates the second title. It will be ignored for other modes.
%\IEEEpeerreviewmaketitle




	\item One urn contains two black balls (labelled B1 and B2) and one white ball. A
	second urn contains one black ball and two white balls (labelled W1 and W2).
	Suppose the following experiment is performed. One of the two urns is chosen
	at random. Next a ball is randomly chosen from the urn. Then a second ball is
	chosen at random from the same urn without replacing the first ball.
	
	\begin{enumerate}
	\item What is the probability that two black balls are chosen?
	
	\item What is the probability that two balls of opposite colour are chosen?
	\end{enumerate}
	\solution
	%\begin{align}
    \label{eq:12.13.6.18.1}
	\because	\pr{A|B} &> \pr{A},\
\frac{\pr{AB}}{\pr{B}} > \pr{A}
\\
    \label{eq:12.13.6.18.2}
	\implies \pr{AB} &> \pr{A}\pr{B}
	\\
	\text{or, } \frac{\pr{AB}}{\pr{A}} &=\pr{B|A} > \pr{A}
\end{align}

\end{enumerate}

	\item A bag contains $5$ red balls and some blue balls. If the probability of drawing a blue ball is double that if a red ball, determine the number of blue balls in the bag. 
		\\
\solution
		%\begin{enumerate}[label=\thesection.\arabic*,ref=\thesection.\theenumi]
	\item One card is drawn from a well-shuffled deck of 52 cards. Find the probability of getting
\begin{enumerate}
\item A king of red colour 
\item A face card 
\item A red face card
\item The jack of hearts
\item A spade
\item The queen of diamonds

\end{enumerate}
\solution
		%\begin{table}[H]
	\centering
\begin{tabular}{|c|c|c|}
\hline
Random variable &Value &Definition\\ \hline
\multirow{3}{*}{X} &0 &Slips of Rs 1\\
&1 &Slips of Rs 5\\
&2 &Slips of Rs 13\\ \hline
\multirow{2}{*}{Y} &0 &Box A\\
&1 &Box B\\\hline
\end{tabular}
\caption{}
\label{tab:Distribution}
\end{table}
See \tabref{tab:Distribution}.
\begin{align}
p_{Y}\brak{k}= \begin{cases} 
      \frac{1}{3} & {k=0} \\
      \frac{2}{3 }& {k=1} 
   \end{cases}
   \\
p_{Y|X}\brak{0|0} = \frac{19}{25}\, 
p_{Y|X}\brak{0|1} = \frac{6}{25}\,
p_{Y|X}\brak{1|0} = \frac{45}{50}\,
p_{Y|X}\brak{1|2} = \frac{5}{50}
\end{align}
The desired probability is the probability that a slip drawn at random is marked other than Rs 1,
\begin{align}
&=1-p_X\brak{0}\\
&= p_X(1) + p_X(2)
\end{align}
Using Bayes theorem,
\begin{align}
&= p_Y\brak{0} \times \pr{Y=0 | X=1} + p_Y\brak{1} \times \pr{Y=1|X=2}\\
&=\frac{1}{3} \times \frac{6}{25} + \frac{2}{3} \times \frac{5}{50}\\
&=\frac{11}{75}
\end{align}

\newpage

%\tableofcontents

\bigskip

\renewcommand{\thefigure}{\theenumi}
\renewcommand{\thetable}{\theenumi}
%\renewcommand{\theequation}{\theenumi}

%\begin{abstract}
%%\boldmath
%In this letter, an algorithm for evaluating the exact analytical bit error rate  (BER)  for the piecewise linear (PL) combiner for  multiple relays is presented. Previous results were available only for upto three relays. The algorithm is unique in the sense that  the actual mathematical expressions, that are prohibitively large, need not be explicitly obtained. The diversity gain due to multiple relays is shown through plots of the analytical BER, well supported by simulations. 
%
%\end{abstract}
% IEEEtran.cls defaults to using nonbold math in the Abstract.
% This preserves the distinction between vectors and scalars. However,
% if the journal you are submitting to favors bold math in the abstract,
% then you can use LaTeX's standard command \boldmath at the very start
% of the abstract to achieve this. Many IEEE journals frown on math
% in the abstract anyway.

% Note that keywords are not normally used for peerreview papers.
%\begin{IEEEkeywords}
%Cooperative diversity, decode and forward, piecewise linear
%\end{IEEEkeywords}



% For peer review papers, you can put extra information on the cover
% page as needed:
% \ifCLASSOPTIONpeerreview
% \begin{center} \bfseries EDICS Category: 3-BBND \end{center}
% \fi
%
% For peerreview papers, this IEEEtran command inserts a page break and
% creates the second title. It will be ignored for other modes.
%\IEEEpeerreviewmaketitle




	\item Five cards—the ten, jack, queen, king and ace of diamonds, are well-shuffled with their face downwards. One card is then picked up at random.
\begin{enumerate}
\item
What is the probability that the card is the queen? 
\item
If the queen is drawn and put aside, what is the probability that the second card picked up is (a) an ace? (b) a queen?\\
\end{enumerate}
\solution
		%\begin{enumerate}[label=\thesection.\arabic*,ref=\thesection.\theenumi]
	\item One card is drawn from a well-shuffled deck of 52 cards. Find the probability of getting
\begin{enumerate}
\item A king of red colour 
\item A face card 
\item A red face card
\item The jack of hearts
\item A spade
\item The queen of diamonds

\end{enumerate}
\solution
		%\input{ncert/10/15/1/14/main.tex}
	\item Five cards—the ten, jack, queen, king and ace of diamonds, are well-shuffled with their face downwards. One card is then picked up at random.
\begin{enumerate}
\item
What is the probability that the card is the queen? 
\item
If the queen is drawn and put aside, what is the probability that the second card picked up is (a) an ace? (b) a queen?\\
\end{enumerate}
\solution
		%\input{ncert/10/15/1/15/defs.tex}
	\item A bag contains $5$ red balls and some blue balls. If the probability of drawing a blue ball is double that if a red ball, determine the number of blue balls in the bag. 
		\\
\solution
		%\input{ncert/10/15/2/3/defs.tex}
	\item A card is selected from a pack of 52 cards.
 \begin{enumerate}[label=(\alph*)] 
                 \item How many points are there in the sample space?
                 \item Calculate the probability that the card is an ace of spades.
                 \item Calculate the probability that the card is (i) an ace and (ii) black card.
 \end{enumerate}
\solution
		%\input{ncert/11/16/3/4/main.tex}
\item Four cards are drawn from a well-shuffled deck of 52 cards. What is the probability of obtaining 3 diamonds and one spade.
\\
\solution
		%\input{ncert/11/16/4/2/defs.tex}
\item In a certain lottery 10,000 tickets are sold and ten equal prizes are awarded. What is the probability of not getting a prize if you buy (a) one ticket (b) two tickets (c) 10 tickets ?	
\\
\solution
		%\input{ncert/11/16/4/4/defs.tex}
		%
\item 
Out of 100 students, two sections of 40 and 60 are formed. If you and your friend are among the 100 students, what is the probability that
\begin{enumerate}
\item you both enter the same section?
\item you both enter the different sections?
\end{enumerate}
\solution
		%\input{ncert/11/16/4/5/defs.tex}
	\item 
The number lock of a suitcase has 4 wheels each labelled with ten digits i.e. from 0 to 9.The lock opens with a sequence of four digits with no repeats.What is the probability of a person getting the right sequence to open the suitcase.
\\
\solution
		%\input{ncert/11/16/4/10/defs.tex}
		%
\item 
Two cards are drawn at random and without replacement from a pack of 52 playing cards. Find the probability that both the cards are black.
\\
\solution
		%\input{ncert/12/13/2/2/defs.tex}
		\item A box of oranges is inspected by examining three randomly selected oranges drawn without replacement. If all the three oranges are good, the box is approved for sale, otherwise, it is rejected. Find the probability that a box containing 15 oranges out of which 12 are good and 3 are bad ones will be approved for sale.
		\label{ncert/12/13/2/3/defs.tex}
		\item Two balls are drawn at random with replacement from a box containing 10 black and 8 red balls. Find the probability that
		\label{ncert/12/13/2/12}
\begin{enumerate}
\item both balls are red.
\item first ball is black and second is red.
\item one of them is black and other is red.
\end{enumerate}

\item In a hostel, 60\% of the students read Hindi newspaper, 40\% read English newspaper and 20\% read both Hindi and English newspapers. A student is selected at random.
		\label{ncert/12/13/2/15}
\begin{enumerate}
\item Find the probability that she reads neither Hindi nor English newspapers.
\item If she reads Hindi newspaper, find the probability that she reads English newspaper.
\item If she reads English newspaper, find the probability that she reads Hindi newspaper.\\
\end{enumerate}
\item The probability of obtaining an even prime number on each die, when a pair of dice is rolled is 
\begin{enumerate}
    \item $0$ 
    
    \item $\frac{1}{3}$ 
    
    \item $\frac{1}{12}$ 
    
    \item $\frac{1}{36}$ 
\end{enumerate}
\solution
		%\input{ncert/12/13/2/17/defs.tex}
	\item A bag contains 4 red and 4 black balls, another bag contains 2 red and 6 black balls. One of the two bags is selected at random and a ball is drawn from the bag which is found to be red. Find the probability that the ball is drawn from the first bag.
\\
\solution
		%\input{ncert/12/13/3/2/main.tex}
  \item
  Cards with numbers 2 to 101 are placed in a box. A card is selected at random.Find the probability that the card has
\begin{enumerate}[label=(\roman*)]
	\item an even number 
	\item a square number
\end{enumerate}
\solution
%\input{exemplar/10/13/3/32/main.tex}
\item
The king, queen and jack of clubs are removed from a deck of 52 playing cards and then well shuffled. Now one card is drawn at random from the remaining cards.  Determine the probability that the card is
\begin{enumerate}[label=(\roman*)]
\item a club
\item 10 of hearts
\end{enumerate}
\solution
%\input{exemplar/10/13/3/29/main.tex}
\item A team of medical students doing their internship have to assist during surgeries
at a city hospital. The probabilities of surgeries rated as very complex, complex,
routine, simple or very simple are respectively, 0.15, 0.20, 0.31, 0.26, .08. Find
the probabilities that a particular surgery will be rated
\begin{enumerate}
	\item complex or very complex;
	\item neither very complex nor very simple;
	\item routine or complex
	\item routine or simple
\end{enumerate}
\solution
%\input{exemplar/11/16/3/8(1)/main.tex}
\item A card is selected from a pack of 52 cards.
\begin{enumerate}[label=(\alph*)]
    \item How many points are there in the sample space?
    \item Calculate the probability that the card is an ace of spades.
    \item Calculate the probability that the card is (i) an ace and (ii) black card.
\end{enumerate}
\solution
%\input{exemplar/11/16/3/4/main2.tex}
\item The probability that a non leap year selected at random will contain 53 sundays.
\\
\solution
%\input{exemplar/10/13/1/19/main.tex}
\item One of the four persons John, Rita, Aslam or Gurpreet will be promoted next
month. Consequently the sample space consists of four elementary outcomes
S = {John promoted, Rita promoted, Aslam promoted, Gurpreet promoted}
You are told that the chances of John’s promotion is same as that of Gurpreet,
Rita’s chances of promotion are twice as likely as Johns. Aslam’s chances are
four times that of John.
\begin{enumerate}
	\item Determine
	\begin{enumerate}
		\item P (John promoted)
		\item P (Rita promoted)
		\item P (Aslam promoted)
		\item P (Gurpreet promoted)
	\end{enumerate}
	\item If A = {John promoted or Gurpreet promoted}, find P (A).
\end{enumerate}
\solution
%\input{exemplar/11/16/3/10/main.tex}
\item A card is drawn from a deck of 52 cards. Find the probability of getting a king or a heart or a red card.\\
\solution
%\input{exemplar/11/16/3/15/main.tex}
\item The probability that a student will pass his examination is 0.73, the probability of
the student getting a compartment is 0.13, and the probability that the student will
either pass or get compartment is 0.96. State True or False.\\
\solution
%\input{exemplar/11/16/3/31/main.tex}
\item A card is selected from a pack of 52 cards\\
\begin{enumerate}[label=(\alph*)]
\item How many points are there in the sample space?
\item Calculate the probability that the cards is an ace of spades.
\item Calculate the probability that the card is (i) an ace (ii)black card.\\
\end{enumerate}
%\input{ncert/11/16/3/4_1/Prob_4.tex}
\item In a non-leap year, the probability of having 53 tuesdays or 53 wednesdays is\\
\solution
%\input{exemplar/11/16/3/18/main.tex}
\item There are 1000 sealed envelopes in a box, 10 of them contain a cash prize of
Rs 100 each, 100 of them contain a cash prize of Rs 50 each and 200 of them
contain a cash prize of Rs 10 each and rest do not contain any cash prize. If they
are well shuffled and an envelope is picked up out, what is the probability that it
contains no cash prize?\\
\solution
%\input{exemplar/10/13/3/34/main.tex}
\item 
A die is thrown and a card is selected at random from a deck of 52 playing cards. The probability of getting an even number on the die and a spade card.\\
\solution
%\input{exemplar/12/13/3/78/main.tex}
\item
If 4-digit numbers greater than 5,000 are randomly formed from the digits 0, 1, 3, 5, and 7, what is the probability of forming a number divisible by 5 when:
\begin{enumerate}
    \item The digits are repeated?
    \item The repetition of digits is not allowed?
\end{enumerate}
\solution
%\input{ncert/11/16/4/9/main.tex}
\item Consider the probability space $\brak{\Omega, \mathcal{G}, P}$ where $\Omega = [0,2]$ and $\mathcal{G} = \cbrak{\phi, \Omega, [0,1], (1,2]}$. Let $X$ and $Y$ be two functions on $\Omega$ defined as
\begin{align*}
    X(\omega) = 
    \begin{cases}
        1 & \text{if }\omega \in [0, 1]\\
        2 & \text{if }\omega \in (1, 2]
    \end{cases}
\end{align*}
and
\begin{align*}
    Y(\omega) = 
    \begin{cases}
        2 & \text{if }\omega \in [0, 1.5]\\
        3 & \text{if }\omega \in (1.5, 2].
    \end{cases}
\end{align*}
Then which one of the following statements is true?
\begin{enumerate}
    \item [(A)] $X$ is a random variable with respect to $\mathcal{G}$, but $Y$ is not a random variable with respect to $\mathcal{G}$.
    \item [(B)] $Y$ is a random variable with respect to $\mathcal{G}$, but $X$ is not a random variable with respect to $\mathcal{G}$.
    \item [(C)] Neither $X$ nor $Y$ is a random variable with respect to $\mathcal{G}$.
    \item [(D)] Both $X$ and $Y$ are random variables with respect to $\mathcal{G}$.
\end{enumerate} \hfill (GATE ST 2023)\\
\solution
%\input{gate/ST/2023/14/main.tex}
	\item  A die is loaded in such a way that each odd number is twice as likely to occur as
each even number. Find $P(G)$, where $G$ is the event that a number greater than
3 occurs on a single roll of the die.
\\
\solution
		%\input{exemplar/11/16/3/5/main.tex}
	\item All the jacks, queens and kings are removed from a deck of 52 playing cards. The remaining cards are well shuffled and then one card is drawn at random. Giving ace a value 1 similar value for other cards, find the probability that the card has a value 
		\begin{enumerate}
			\item 7
			\item greater than 7
			\item less than 7
		\end{enumerate}
		%\input{exemplar/10/13/3/30/main.tex}
  \item A Lot consists of 48 mobile phones of which 42 are good, 3 have only minor defects and 3 have major defects.Varnika will buy a phone if it is good but the trader will only buy a mobile if it has no major defects. One phone is selected at random from the lot. What is the probability that it is
\begin{enumerate}
	\item acceptable to Varnika?
            \item acceptable to the trader?
\end{enumerate}
\solution
	%\input{exemplar/10/13/3/40/main.tex}
 \item A student says that if you throw a die, it will show up 1 or not 1. Therefore, the probability of getting 1 and the probability of getting 'not 1' each is equal to $\frac{1}{2}$. Is this correct? Give reasons.\\
 \solution
        %\input{exemplar/10/13/2/9/main.tex}
   \item Four candidates A, B, C, D have ap-
plied for the assignment to coach a school cricket
team. If A is twice as likely to be selected as B, and
B and C are given about the same chance of being
selected, while C is twice as likely to be selected
as D, what are the probabilities that
\begin{enumerate}
\item C will be selected?
\item A will not be selected?
\end{enumerate}
	%\input{exemplar/11/16/3/9/main.tex}
 \item A bag contain 24 balls of which $x$ balls are red, $2x$ are white and $3x$ are blue. A ball is selected at random, What is the probability that it is
\begin{enumerate}[label=\alph*)]
\item not red ?
\item white ?
\end{enumerate}
%\input{exemplar/10/13/3/41/main.tex}
If the letters of the word ASSASSINATION are arranged at random. Find the Probability that
\begin{enumerate}[label=(\alph*)]
\item Four $S's$ come consecutively in the word
\item Two  $I's$ and two $N's$ come together
\item All $A's$ are not coming together
\item No two $A's$ are coming together
\end{enumerate}
%\input{exemplar/11/16/3/14/main.tex}
	\item One urn contains two black balls (labelled B1 and B2) and one white ball. A
	second urn contains one black ball and two white balls (labelled W1 and W2).
	Suppose the following experiment is performed. One of the two urns is chosen
	at random. Next a ball is randomly chosen from the urn. Then a second ball is
	chosen at random from the same urn without replacing the first ball.
	
	\begin{enumerate}
	\item What is the probability that two black balls are chosen?
	
	\item What is the probability that two balls of opposite colour are chosen?
	\end{enumerate}
	\solution
	%\input{exemplar/11/16/3/12/main1.tex}
\end{enumerate}

	\item A bag contains $5$ red balls and some blue balls. If the probability of drawing a blue ball is double that if a red ball, determine the number of blue balls in the bag. 
		\\
\solution
		%\begin{enumerate}[label=\thesection.\arabic*,ref=\thesection.\theenumi]
	\item One card is drawn from a well-shuffled deck of 52 cards. Find the probability of getting
\begin{enumerate}
\item A king of red colour 
\item A face card 
\item A red face card
\item The jack of hearts
\item A spade
\item The queen of diamonds

\end{enumerate}
\solution
		%\input{ncert/10/15/1/14/main.tex}
	\item Five cards—the ten, jack, queen, king and ace of diamonds, are well-shuffled with their face downwards. One card is then picked up at random.
\begin{enumerate}
\item
What is the probability that the card is the queen? 
\item
If the queen is drawn and put aside, what is the probability that the second card picked up is (a) an ace? (b) a queen?\\
\end{enumerate}
\solution
		%\input{ncert/10/15/1/15/defs.tex}
	\item A bag contains $5$ red balls and some blue balls. If the probability of drawing a blue ball is double that if a red ball, determine the number of blue balls in the bag. 
		\\
\solution
		%\input{ncert/10/15/2/3/defs.tex}
	\item A card is selected from a pack of 52 cards.
 \begin{enumerate}[label=(\alph*)] 
                 \item How many points are there in the sample space?
                 \item Calculate the probability that the card is an ace of spades.
                 \item Calculate the probability that the card is (i) an ace and (ii) black card.
 \end{enumerate}
\solution
		%\input{ncert/11/16/3/4/main.tex}
\item Four cards are drawn from a well-shuffled deck of 52 cards. What is the probability of obtaining 3 diamonds and one spade.
\\
\solution
		%\input{ncert/11/16/4/2/defs.tex}
\item In a certain lottery 10,000 tickets are sold and ten equal prizes are awarded. What is the probability of not getting a prize if you buy (a) one ticket (b) two tickets (c) 10 tickets ?	
\\
\solution
		%\input{ncert/11/16/4/4/defs.tex}
		%
\item 
Out of 100 students, two sections of 40 and 60 are formed. If you and your friend are among the 100 students, what is the probability that
\begin{enumerate}
\item you both enter the same section?
\item you both enter the different sections?
\end{enumerate}
\solution
		%\input{ncert/11/16/4/5/defs.tex}
	\item 
The number lock of a suitcase has 4 wheels each labelled with ten digits i.e. from 0 to 9.The lock opens with a sequence of four digits with no repeats.What is the probability of a person getting the right sequence to open the suitcase.
\\
\solution
		%\input{ncert/11/16/4/10/defs.tex}
		%
\item 
Two cards are drawn at random and without replacement from a pack of 52 playing cards. Find the probability that both the cards are black.
\\
\solution
		%\input{ncert/12/13/2/2/defs.tex}
		\item A box of oranges is inspected by examining three randomly selected oranges drawn without replacement. If all the three oranges are good, the box is approved for sale, otherwise, it is rejected. Find the probability that a box containing 15 oranges out of which 12 are good and 3 are bad ones will be approved for sale.
		\label{ncert/12/13/2/3/defs.tex}
		\item Two balls are drawn at random with replacement from a box containing 10 black and 8 red balls. Find the probability that
		\label{ncert/12/13/2/12}
\begin{enumerate}
\item both balls are red.
\item first ball is black and second is red.
\item one of them is black and other is red.
\end{enumerate}

\item In a hostel, 60\% of the students read Hindi newspaper, 40\% read English newspaper and 20\% read both Hindi and English newspapers. A student is selected at random.
		\label{ncert/12/13/2/15}
\begin{enumerate}
\item Find the probability that she reads neither Hindi nor English newspapers.
\item If she reads Hindi newspaper, find the probability that she reads English newspaper.
\item If she reads English newspaper, find the probability that she reads Hindi newspaper.\\
\end{enumerate}
\item The probability of obtaining an even prime number on each die, when a pair of dice is rolled is 
\begin{enumerate}
    \item $0$ 
    
    \item $\frac{1}{3}$ 
    
    \item $\frac{1}{12}$ 
    
    \item $\frac{1}{36}$ 
\end{enumerate}
\solution
		%\input{ncert/12/13/2/17/defs.tex}
	\item A bag contains 4 red and 4 black balls, another bag contains 2 red and 6 black balls. One of the two bags is selected at random and a ball is drawn from the bag which is found to be red. Find the probability that the ball is drawn from the first bag.
\\
\solution
		%\input{ncert/12/13/3/2/main.tex}
  \item
  Cards with numbers 2 to 101 are placed in a box. A card is selected at random.Find the probability that the card has
\begin{enumerate}[label=(\roman*)]
	\item an even number 
	\item a square number
\end{enumerate}
\solution
%\input{exemplar/10/13/3/32/main.tex}
\item
The king, queen and jack of clubs are removed from a deck of 52 playing cards and then well shuffled. Now one card is drawn at random from the remaining cards.  Determine the probability that the card is
\begin{enumerate}[label=(\roman*)]
\item a club
\item 10 of hearts
\end{enumerate}
\solution
%\input{exemplar/10/13/3/29/main.tex}
\item A team of medical students doing their internship have to assist during surgeries
at a city hospital. The probabilities of surgeries rated as very complex, complex,
routine, simple or very simple are respectively, 0.15, 0.20, 0.31, 0.26, .08. Find
the probabilities that a particular surgery will be rated
\begin{enumerate}
	\item complex or very complex;
	\item neither very complex nor very simple;
	\item routine or complex
	\item routine or simple
\end{enumerate}
\solution
%\input{exemplar/11/16/3/8(1)/main.tex}
\item A card is selected from a pack of 52 cards.
\begin{enumerate}[label=(\alph*)]
    \item How many points are there in the sample space?
    \item Calculate the probability that the card is an ace of spades.
    \item Calculate the probability that the card is (i) an ace and (ii) black card.
\end{enumerate}
\solution
%\input{exemplar/11/16/3/4/main2.tex}
\item The probability that a non leap year selected at random will contain 53 sundays.
\\
\solution
%\input{exemplar/10/13/1/19/main.tex}
\item One of the four persons John, Rita, Aslam or Gurpreet will be promoted next
month. Consequently the sample space consists of four elementary outcomes
S = {John promoted, Rita promoted, Aslam promoted, Gurpreet promoted}
You are told that the chances of John’s promotion is same as that of Gurpreet,
Rita’s chances of promotion are twice as likely as Johns. Aslam’s chances are
four times that of John.
\begin{enumerate}
	\item Determine
	\begin{enumerate}
		\item P (John promoted)
		\item P (Rita promoted)
		\item P (Aslam promoted)
		\item P (Gurpreet promoted)
	\end{enumerate}
	\item If A = {John promoted or Gurpreet promoted}, find P (A).
\end{enumerate}
\solution
%\input{exemplar/11/16/3/10/main.tex}
\item A card is drawn from a deck of 52 cards. Find the probability of getting a king or a heart or a red card.\\
\solution
%\input{exemplar/11/16/3/15/main.tex}
\item The probability that a student will pass his examination is 0.73, the probability of
the student getting a compartment is 0.13, and the probability that the student will
either pass or get compartment is 0.96. State True or False.\\
\solution
%\input{exemplar/11/16/3/31/main.tex}
\item A card is selected from a pack of 52 cards\\
\begin{enumerate}[label=(\alph*)]
\item How many points are there in the sample space?
\item Calculate the probability that the cards is an ace of spades.
\item Calculate the probability that the card is (i) an ace (ii)black card.\\
\end{enumerate}
%\input{ncert/11/16/3/4_1/Prob_4.tex}
\item In a non-leap year, the probability of having 53 tuesdays or 53 wednesdays is\\
\solution
%\input{exemplar/11/16/3/18/main.tex}
\item There are 1000 sealed envelopes in a box, 10 of them contain a cash prize of
Rs 100 each, 100 of them contain a cash prize of Rs 50 each and 200 of them
contain a cash prize of Rs 10 each and rest do not contain any cash prize. If they
are well shuffled and an envelope is picked up out, what is the probability that it
contains no cash prize?\\
\solution
%\input{exemplar/10/13/3/34/main.tex}
\item 
A die is thrown and a card is selected at random from a deck of 52 playing cards. The probability of getting an even number on the die and a spade card.\\
\solution
%\input{exemplar/12/13/3/78/main.tex}
\item
If 4-digit numbers greater than 5,000 are randomly formed from the digits 0, 1, 3, 5, and 7, what is the probability of forming a number divisible by 5 when:
\begin{enumerate}
    \item The digits are repeated?
    \item The repetition of digits is not allowed?
\end{enumerate}
\solution
%\input{ncert/11/16/4/9/main.tex}
\item Consider the probability space $\brak{\Omega, \mathcal{G}, P}$ where $\Omega = [0,2]$ and $\mathcal{G} = \cbrak{\phi, \Omega, [0,1], (1,2]}$. Let $X$ and $Y$ be two functions on $\Omega$ defined as
\begin{align*}
    X(\omega) = 
    \begin{cases}
        1 & \text{if }\omega \in [0, 1]\\
        2 & \text{if }\omega \in (1, 2]
    \end{cases}
\end{align*}
and
\begin{align*}
    Y(\omega) = 
    \begin{cases}
        2 & \text{if }\omega \in [0, 1.5]\\
        3 & \text{if }\omega \in (1.5, 2].
    \end{cases}
\end{align*}
Then which one of the following statements is true?
\begin{enumerate}
    \item [(A)] $X$ is a random variable with respect to $\mathcal{G}$, but $Y$ is not a random variable with respect to $\mathcal{G}$.
    \item [(B)] $Y$ is a random variable with respect to $\mathcal{G}$, but $X$ is not a random variable with respect to $\mathcal{G}$.
    \item [(C)] Neither $X$ nor $Y$ is a random variable with respect to $\mathcal{G}$.
    \item [(D)] Both $X$ and $Y$ are random variables with respect to $\mathcal{G}$.
\end{enumerate} \hfill (GATE ST 2023)\\
\solution
%\input{gate/ST/2023/14/main.tex}
	\item  A die is loaded in such a way that each odd number is twice as likely to occur as
each even number. Find $P(G)$, where $G$ is the event that a number greater than
3 occurs on a single roll of the die.
\\
\solution
		%\input{exemplar/11/16/3/5/main.tex}
	\item All the jacks, queens and kings are removed from a deck of 52 playing cards. The remaining cards are well shuffled and then one card is drawn at random. Giving ace a value 1 similar value for other cards, find the probability that the card has a value 
		\begin{enumerate}
			\item 7
			\item greater than 7
			\item less than 7
		\end{enumerate}
		%\input{exemplar/10/13/3/30/main.tex}
  \item A Lot consists of 48 mobile phones of which 42 are good, 3 have only minor defects and 3 have major defects.Varnika will buy a phone if it is good but the trader will only buy a mobile if it has no major defects. One phone is selected at random from the lot. What is the probability that it is
\begin{enumerate}
	\item acceptable to Varnika?
            \item acceptable to the trader?
\end{enumerate}
\solution
	%\input{exemplar/10/13/3/40/main.tex}
 \item A student says that if you throw a die, it will show up 1 or not 1. Therefore, the probability of getting 1 and the probability of getting 'not 1' each is equal to $\frac{1}{2}$. Is this correct? Give reasons.\\
 \solution
        %\input{exemplar/10/13/2/9/main.tex}
   \item Four candidates A, B, C, D have ap-
plied for the assignment to coach a school cricket
team. If A is twice as likely to be selected as B, and
B and C are given about the same chance of being
selected, while C is twice as likely to be selected
as D, what are the probabilities that
\begin{enumerate}
\item C will be selected?
\item A will not be selected?
\end{enumerate}
	%\input{exemplar/11/16/3/9/main.tex}
 \item A bag contain 24 balls of which $x$ balls are red, $2x$ are white and $3x$ are blue. A ball is selected at random, What is the probability that it is
\begin{enumerate}[label=\alph*)]
\item not red ?
\item white ?
\end{enumerate}
%\input{exemplar/10/13/3/41/main.tex}
If the letters of the word ASSASSINATION are arranged at random. Find the Probability that
\begin{enumerate}[label=(\alph*)]
\item Four $S's$ come consecutively in the word
\item Two  $I's$ and two $N's$ come together
\item All $A's$ are not coming together
\item No two $A's$ are coming together
\end{enumerate}
%\input{exemplar/11/16/3/14/main.tex}
	\item One urn contains two black balls (labelled B1 and B2) and one white ball. A
	second urn contains one black ball and two white balls (labelled W1 and W2).
	Suppose the following experiment is performed. One of the two urns is chosen
	at random. Next a ball is randomly chosen from the urn. Then a second ball is
	chosen at random from the same urn without replacing the first ball.
	
	\begin{enumerate}
	\item What is the probability that two black balls are chosen?
	
	\item What is the probability that two balls of opposite colour are chosen?
	\end{enumerate}
	\solution
	%\input{exemplar/11/16/3/12/main1.tex}
\end{enumerate}

	\item A card is selected from a pack of 52 cards.
 \begin{enumerate}[label=(\alph*)] 
                 \item How many points are there in the sample space?
                 \item Calculate the probability that the card is an ace of spades.
                 \item Calculate the probability that the card is (i) an ace and (ii) black card.
 \end{enumerate}
\solution
		%\begin{table}[H]
	\centering
\begin{tabular}{|c|c|c|}
\hline
Random variable &Value &Definition\\ \hline
\multirow{3}{*}{X} &0 &Slips of Rs 1\\
&1 &Slips of Rs 5\\
&2 &Slips of Rs 13\\ \hline
\multirow{2}{*}{Y} &0 &Box A\\
&1 &Box B\\\hline
\end{tabular}
\caption{}
\label{tab:Distribution}
\end{table}
See \tabref{tab:Distribution}.
\begin{align}
p_{Y}\brak{k}= \begin{cases} 
      \frac{1}{3} & {k=0} \\
      \frac{2}{3 }& {k=1} 
   \end{cases}
   \\
p_{Y|X}\brak{0|0} = \frac{19}{25}\, 
p_{Y|X}\brak{0|1} = \frac{6}{25}\,
p_{Y|X}\brak{1|0} = \frac{45}{50}\,
p_{Y|X}\brak{1|2} = \frac{5}{50}
\end{align}
The desired probability is the probability that a slip drawn at random is marked other than Rs 1,
\begin{align}
&=1-p_X\brak{0}\\
&= p_X(1) + p_X(2)
\end{align}
Using Bayes theorem,
\begin{align}
&= p_Y\brak{0} \times \pr{Y=0 | X=1} + p_Y\brak{1} \times \pr{Y=1|X=2}\\
&=\frac{1}{3} \times \frac{6}{25} + \frac{2}{3} \times \frac{5}{50}\\
&=\frac{11}{75}
\end{align}

\newpage

%\tableofcontents

\bigskip

\renewcommand{\thefigure}{\theenumi}
\renewcommand{\thetable}{\theenumi}
%\renewcommand{\theequation}{\theenumi}

%\begin{abstract}
%%\boldmath
%In this letter, an algorithm for evaluating the exact analytical bit error rate  (BER)  for the piecewise linear (PL) combiner for  multiple relays is presented. Previous results were available only for upto three relays. The algorithm is unique in the sense that  the actual mathematical expressions, that are prohibitively large, need not be explicitly obtained. The diversity gain due to multiple relays is shown through plots of the analytical BER, well supported by simulations. 
%
%\end{abstract}
% IEEEtran.cls defaults to using nonbold math in the Abstract.
% This preserves the distinction between vectors and scalars. However,
% if the journal you are submitting to favors bold math in the abstract,
% then you can use LaTeX's standard command \boldmath at the very start
% of the abstract to achieve this. Many IEEE journals frown on math
% in the abstract anyway.

% Note that keywords are not normally used for peerreview papers.
%\begin{IEEEkeywords}
%Cooperative diversity, decode and forward, piecewise linear
%\end{IEEEkeywords}



% For peer review papers, you can put extra information on the cover
% page as needed:
% \ifCLASSOPTIONpeerreview
% \begin{center} \bfseries EDICS Category: 3-BBND \end{center}
% \fi
%
% For peerreview papers, this IEEEtran command inserts a page break and
% creates the second title. It will be ignored for other modes.
%\IEEEpeerreviewmaketitle




\item Four cards are drawn from a well-shuffled deck of 52 cards. What is the probability of obtaining 3 diamonds and one spade.
\\
\solution
		%\begin{enumerate}[label=\thesection.\arabic*,ref=\thesection.\theenumi]
	\item One card is drawn from a well-shuffled deck of 52 cards. Find the probability of getting
\begin{enumerate}
\item A king of red colour 
\item A face card 
\item A red face card
\item The jack of hearts
\item A spade
\item The queen of diamonds

\end{enumerate}
\solution
		%\input{ncert/10/15/1/14/main.tex}
	\item Five cards—the ten, jack, queen, king and ace of diamonds, are well-shuffled with their face downwards. One card is then picked up at random.
\begin{enumerate}
\item
What is the probability that the card is the queen? 
\item
If the queen is drawn and put aside, what is the probability that the second card picked up is (a) an ace? (b) a queen?\\
\end{enumerate}
\solution
		%\input{ncert/10/15/1/15/defs.tex}
	\item A bag contains $5$ red balls and some blue balls. If the probability of drawing a blue ball is double that if a red ball, determine the number of blue balls in the bag. 
		\\
\solution
		%\input{ncert/10/15/2/3/defs.tex}
	\item A card is selected from a pack of 52 cards.
 \begin{enumerate}[label=(\alph*)] 
                 \item How many points are there in the sample space?
                 \item Calculate the probability that the card is an ace of spades.
                 \item Calculate the probability that the card is (i) an ace and (ii) black card.
 \end{enumerate}
\solution
		%\input{ncert/11/16/3/4/main.tex}
\item Four cards are drawn from a well-shuffled deck of 52 cards. What is the probability of obtaining 3 diamonds and one spade.
\\
\solution
		%\input{ncert/11/16/4/2/defs.tex}
\item In a certain lottery 10,000 tickets are sold and ten equal prizes are awarded. What is the probability of not getting a prize if you buy (a) one ticket (b) two tickets (c) 10 tickets ?	
\\
\solution
		%\input{ncert/11/16/4/4/defs.tex}
		%
\item 
Out of 100 students, two sections of 40 and 60 are formed. If you and your friend are among the 100 students, what is the probability that
\begin{enumerate}
\item you both enter the same section?
\item you both enter the different sections?
\end{enumerate}
\solution
		%\input{ncert/11/16/4/5/defs.tex}
	\item 
The number lock of a suitcase has 4 wheels each labelled with ten digits i.e. from 0 to 9.The lock opens with a sequence of four digits with no repeats.What is the probability of a person getting the right sequence to open the suitcase.
\\
\solution
		%\input{ncert/11/16/4/10/defs.tex}
		%
\item 
Two cards are drawn at random and without replacement from a pack of 52 playing cards. Find the probability that both the cards are black.
\\
\solution
		%\input{ncert/12/13/2/2/defs.tex}
		\item A box of oranges is inspected by examining three randomly selected oranges drawn without replacement. If all the three oranges are good, the box is approved for sale, otherwise, it is rejected. Find the probability that a box containing 15 oranges out of which 12 are good and 3 are bad ones will be approved for sale.
		\label{ncert/12/13/2/3/defs.tex}
		\item Two balls are drawn at random with replacement from a box containing 10 black and 8 red balls. Find the probability that
		\label{ncert/12/13/2/12}
\begin{enumerate}
\item both balls are red.
\item first ball is black and second is red.
\item one of them is black and other is red.
\end{enumerate}

\item In a hostel, 60\% of the students read Hindi newspaper, 40\% read English newspaper and 20\% read both Hindi and English newspapers. A student is selected at random.
		\label{ncert/12/13/2/15}
\begin{enumerate}
\item Find the probability that she reads neither Hindi nor English newspapers.
\item If she reads Hindi newspaper, find the probability that she reads English newspaper.
\item If she reads English newspaper, find the probability that she reads Hindi newspaper.\\
\end{enumerate}
\item The probability of obtaining an even prime number on each die, when a pair of dice is rolled is 
\begin{enumerate}
    \item $0$ 
    
    \item $\frac{1}{3}$ 
    
    \item $\frac{1}{12}$ 
    
    \item $\frac{1}{36}$ 
\end{enumerate}
\solution
		%\input{ncert/12/13/2/17/defs.tex}
	\item A bag contains 4 red and 4 black balls, another bag contains 2 red and 6 black balls. One of the two bags is selected at random and a ball is drawn from the bag which is found to be red. Find the probability that the ball is drawn from the first bag.
\\
\solution
		%\input{ncert/12/13/3/2/main.tex}
  \item
  Cards with numbers 2 to 101 are placed in a box. A card is selected at random.Find the probability that the card has
\begin{enumerate}[label=(\roman*)]
	\item an even number 
	\item a square number
\end{enumerate}
\solution
%\input{exemplar/10/13/3/32/main.tex}
\item
The king, queen and jack of clubs are removed from a deck of 52 playing cards and then well shuffled. Now one card is drawn at random from the remaining cards.  Determine the probability that the card is
\begin{enumerate}[label=(\roman*)]
\item a club
\item 10 of hearts
\end{enumerate}
\solution
%\input{exemplar/10/13/3/29/main.tex}
\item A team of medical students doing their internship have to assist during surgeries
at a city hospital. The probabilities of surgeries rated as very complex, complex,
routine, simple or very simple are respectively, 0.15, 0.20, 0.31, 0.26, .08. Find
the probabilities that a particular surgery will be rated
\begin{enumerate}
	\item complex or very complex;
	\item neither very complex nor very simple;
	\item routine or complex
	\item routine or simple
\end{enumerate}
\solution
%\input{exemplar/11/16/3/8(1)/main.tex}
\item A card is selected from a pack of 52 cards.
\begin{enumerate}[label=(\alph*)]
    \item How many points are there in the sample space?
    \item Calculate the probability that the card is an ace of spades.
    \item Calculate the probability that the card is (i) an ace and (ii) black card.
\end{enumerate}
\solution
%\input{exemplar/11/16/3/4/main2.tex}
\item The probability that a non leap year selected at random will contain 53 sundays.
\\
\solution
%\input{exemplar/10/13/1/19/main.tex}
\item One of the four persons John, Rita, Aslam or Gurpreet will be promoted next
month. Consequently the sample space consists of four elementary outcomes
S = {John promoted, Rita promoted, Aslam promoted, Gurpreet promoted}
You are told that the chances of John’s promotion is same as that of Gurpreet,
Rita’s chances of promotion are twice as likely as Johns. Aslam’s chances are
four times that of John.
\begin{enumerate}
	\item Determine
	\begin{enumerate}
		\item P (John promoted)
		\item P (Rita promoted)
		\item P (Aslam promoted)
		\item P (Gurpreet promoted)
	\end{enumerate}
	\item If A = {John promoted or Gurpreet promoted}, find P (A).
\end{enumerate}
\solution
%\input{exemplar/11/16/3/10/main.tex}
\item A card is drawn from a deck of 52 cards. Find the probability of getting a king or a heart or a red card.\\
\solution
%\input{exemplar/11/16/3/15/main.tex}
\item The probability that a student will pass his examination is 0.73, the probability of
the student getting a compartment is 0.13, and the probability that the student will
either pass or get compartment is 0.96. State True or False.\\
\solution
%\input{exemplar/11/16/3/31/main.tex}
\item A card is selected from a pack of 52 cards\\
\begin{enumerate}[label=(\alph*)]
\item How many points are there in the sample space?
\item Calculate the probability that the cards is an ace of spades.
\item Calculate the probability that the card is (i) an ace (ii)black card.\\
\end{enumerate}
%\input{ncert/11/16/3/4_1/Prob_4.tex}
\item In a non-leap year, the probability of having 53 tuesdays or 53 wednesdays is\\
\solution
%\input{exemplar/11/16/3/18/main.tex}
\item There are 1000 sealed envelopes in a box, 10 of them contain a cash prize of
Rs 100 each, 100 of them contain a cash prize of Rs 50 each and 200 of them
contain a cash prize of Rs 10 each and rest do not contain any cash prize. If they
are well shuffled and an envelope is picked up out, what is the probability that it
contains no cash prize?\\
\solution
%\input{exemplar/10/13/3/34/main.tex}
\item 
A die is thrown and a card is selected at random from a deck of 52 playing cards. The probability of getting an even number on the die and a spade card.\\
\solution
%\input{exemplar/12/13/3/78/main.tex}
\item
If 4-digit numbers greater than 5,000 are randomly formed from the digits 0, 1, 3, 5, and 7, what is the probability of forming a number divisible by 5 when:
\begin{enumerate}
    \item The digits are repeated?
    \item The repetition of digits is not allowed?
\end{enumerate}
\solution
%\input{ncert/11/16/4/9/main.tex}
\item Consider the probability space $\brak{\Omega, \mathcal{G}, P}$ where $\Omega = [0,2]$ and $\mathcal{G} = \cbrak{\phi, \Omega, [0,1], (1,2]}$. Let $X$ and $Y$ be two functions on $\Omega$ defined as
\begin{align*}
    X(\omega) = 
    \begin{cases}
        1 & \text{if }\omega \in [0, 1]\\
        2 & \text{if }\omega \in (1, 2]
    \end{cases}
\end{align*}
and
\begin{align*}
    Y(\omega) = 
    \begin{cases}
        2 & \text{if }\omega \in [0, 1.5]\\
        3 & \text{if }\omega \in (1.5, 2].
    \end{cases}
\end{align*}
Then which one of the following statements is true?
\begin{enumerate}
    \item [(A)] $X$ is a random variable with respect to $\mathcal{G}$, but $Y$ is not a random variable with respect to $\mathcal{G}$.
    \item [(B)] $Y$ is a random variable with respect to $\mathcal{G}$, but $X$ is not a random variable with respect to $\mathcal{G}$.
    \item [(C)] Neither $X$ nor $Y$ is a random variable with respect to $\mathcal{G}$.
    \item [(D)] Both $X$ and $Y$ are random variables with respect to $\mathcal{G}$.
\end{enumerate} \hfill (GATE ST 2023)\\
\solution
%\input{gate/ST/2023/14/main.tex}
	\item  A die is loaded in such a way that each odd number is twice as likely to occur as
each even number. Find $P(G)$, where $G$ is the event that a number greater than
3 occurs on a single roll of the die.
\\
\solution
		%\input{exemplar/11/16/3/5/main.tex}
	\item All the jacks, queens and kings are removed from a deck of 52 playing cards. The remaining cards are well shuffled and then one card is drawn at random. Giving ace a value 1 similar value for other cards, find the probability that the card has a value 
		\begin{enumerate}
			\item 7
			\item greater than 7
			\item less than 7
		\end{enumerate}
		%\input{exemplar/10/13/3/30/main.tex}
  \item A Lot consists of 48 mobile phones of which 42 are good, 3 have only minor defects and 3 have major defects.Varnika will buy a phone if it is good but the trader will only buy a mobile if it has no major defects. One phone is selected at random from the lot. What is the probability that it is
\begin{enumerate}
	\item acceptable to Varnika?
            \item acceptable to the trader?
\end{enumerate}
\solution
	%\input{exemplar/10/13/3/40/main.tex}
 \item A student says that if you throw a die, it will show up 1 or not 1. Therefore, the probability of getting 1 and the probability of getting 'not 1' each is equal to $\frac{1}{2}$. Is this correct? Give reasons.\\
 \solution
        %\input{exemplar/10/13/2/9/main.tex}
   \item Four candidates A, B, C, D have ap-
plied for the assignment to coach a school cricket
team. If A is twice as likely to be selected as B, and
B and C are given about the same chance of being
selected, while C is twice as likely to be selected
as D, what are the probabilities that
\begin{enumerate}
\item C will be selected?
\item A will not be selected?
\end{enumerate}
	%\input{exemplar/11/16/3/9/main.tex}
 \item A bag contain 24 balls of which $x$ balls are red, $2x$ are white and $3x$ are blue. A ball is selected at random, What is the probability that it is
\begin{enumerate}[label=\alph*)]
\item not red ?
\item white ?
\end{enumerate}
%\input{exemplar/10/13/3/41/main.tex}
If the letters of the word ASSASSINATION are arranged at random. Find the Probability that
\begin{enumerate}[label=(\alph*)]
\item Four $S's$ come consecutively in the word
\item Two  $I's$ and two $N's$ come together
\item All $A's$ are not coming together
\item No two $A's$ are coming together
\end{enumerate}
%\input{exemplar/11/16/3/14/main.tex}
	\item One urn contains two black balls (labelled B1 and B2) and one white ball. A
	second urn contains one black ball and two white balls (labelled W1 and W2).
	Suppose the following experiment is performed. One of the two urns is chosen
	at random. Next a ball is randomly chosen from the urn. Then a second ball is
	chosen at random from the same urn without replacing the first ball.
	
	\begin{enumerate}
	\item What is the probability that two black balls are chosen?
	
	\item What is the probability that two balls of opposite colour are chosen?
	\end{enumerate}
	\solution
	%\input{exemplar/11/16/3/12/main1.tex}
\end{enumerate}

\item In a certain lottery 10,000 tickets are sold and ten equal prizes are awarded. What is the probability of not getting a prize if you buy (a) one ticket (b) two tickets (c) 10 tickets ?	
\\
\solution
		%\begin{enumerate}[label=\thesection.\arabic*,ref=\thesection.\theenumi]
	\item One card is drawn from a well-shuffled deck of 52 cards. Find the probability of getting
\begin{enumerate}
\item A king of red colour 
\item A face card 
\item A red face card
\item The jack of hearts
\item A spade
\item The queen of diamonds

\end{enumerate}
\solution
		%\input{ncert/10/15/1/14/main.tex}
	\item Five cards—the ten, jack, queen, king and ace of diamonds, are well-shuffled with their face downwards. One card is then picked up at random.
\begin{enumerate}
\item
What is the probability that the card is the queen? 
\item
If the queen is drawn and put aside, what is the probability that the second card picked up is (a) an ace? (b) a queen?\\
\end{enumerate}
\solution
		%\input{ncert/10/15/1/15/defs.tex}
	\item A bag contains $5$ red balls and some blue balls. If the probability of drawing a blue ball is double that if a red ball, determine the number of blue balls in the bag. 
		\\
\solution
		%\input{ncert/10/15/2/3/defs.tex}
	\item A card is selected from a pack of 52 cards.
 \begin{enumerate}[label=(\alph*)] 
                 \item How many points are there in the sample space?
                 \item Calculate the probability that the card is an ace of spades.
                 \item Calculate the probability that the card is (i) an ace and (ii) black card.
 \end{enumerate}
\solution
		%\input{ncert/11/16/3/4/main.tex}
\item Four cards are drawn from a well-shuffled deck of 52 cards. What is the probability of obtaining 3 diamonds and one spade.
\\
\solution
		%\input{ncert/11/16/4/2/defs.tex}
\item In a certain lottery 10,000 tickets are sold and ten equal prizes are awarded. What is the probability of not getting a prize if you buy (a) one ticket (b) two tickets (c) 10 tickets ?	
\\
\solution
		%\input{ncert/11/16/4/4/defs.tex}
		%
\item 
Out of 100 students, two sections of 40 and 60 are formed. If you and your friend are among the 100 students, what is the probability that
\begin{enumerate}
\item you both enter the same section?
\item you both enter the different sections?
\end{enumerate}
\solution
		%\input{ncert/11/16/4/5/defs.tex}
	\item 
The number lock of a suitcase has 4 wheels each labelled with ten digits i.e. from 0 to 9.The lock opens with a sequence of four digits with no repeats.What is the probability of a person getting the right sequence to open the suitcase.
\\
\solution
		%\input{ncert/11/16/4/10/defs.tex}
		%
\item 
Two cards are drawn at random and without replacement from a pack of 52 playing cards. Find the probability that both the cards are black.
\\
\solution
		%\input{ncert/12/13/2/2/defs.tex}
		\item A box of oranges is inspected by examining three randomly selected oranges drawn without replacement. If all the three oranges are good, the box is approved for sale, otherwise, it is rejected. Find the probability that a box containing 15 oranges out of which 12 are good and 3 are bad ones will be approved for sale.
		\label{ncert/12/13/2/3/defs.tex}
		\item Two balls are drawn at random with replacement from a box containing 10 black and 8 red balls. Find the probability that
		\label{ncert/12/13/2/12}
\begin{enumerate}
\item both balls are red.
\item first ball is black and second is red.
\item one of them is black and other is red.
\end{enumerate}

\item In a hostel, 60\% of the students read Hindi newspaper, 40\% read English newspaper and 20\% read both Hindi and English newspapers. A student is selected at random.
		\label{ncert/12/13/2/15}
\begin{enumerate}
\item Find the probability that she reads neither Hindi nor English newspapers.
\item If she reads Hindi newspaper, find the probability that she reads English newspaper.
\item If she reads English newspaper, find the probability that she reads Hindi newspaper.\\
\end{enumerate}
\item The probability of obtaining an even prime number on each die, when a pair of dice is rolled is 
\begin{enumerate}
    \item $0$ 
    
    \item $\frac{1}{3}$ 
    
    \item $\frac{1}{12}$ 
    
    \item $\frac{1}{36}$ 
\end{enumerate}
\solution
		%\input{ncert/12/13/2/17/defs.tex}
	\item A bag contains 4 red and 4 black balls, another bag contains 2 red and 6 black balls. One of the two bags is selected at random and a ball is drawn from the bag which is found to be red. Find the probability that the ball is drawn from the first bag.
\\
\solution
		%\input{ncert/12/13/3/2/main.tex}
  \item
  Cards with numbers 2 to 101 are placed in a box. A card is selected at random.Find the probability that the card has
\begin{enumerate}[label=(\roman*)]
	\item an even number 
	\item a square number
\end{enumerate}
\solution
%\input{exemplar/10/13/3/32/main.tex}
\item
The king, queen and jack of clubs are removed from a deck of 52 playing cards and then well shuffled. Now one card is drawn at random from the remaining cards.  Determine the probability that the card is
\begin{enumerate}[label=(\roman*)]
\item a club
\item 10 of hearts
\end{enumerate}
\solution
%\input{exemplar/10/13/3/29/main.tex}
\item A team of medical students doing their internship have to assist during surgeries
at a city hospital. The probabilities of surgeries rated as very complex, complex,
routine, simple or very simple are respectively, 0.15, 0.20, 0.31, 0.26, .08. Find
the probabilities that a particular surgery will be rated
\begin{enumerate}
	\item complex or very complex;
	\item neither very complex nor very simple;
	\item routine or complex
	\item routine or simple
\end{enumerate}
\solution
%\input{exemplar/11/16/3/8(1)/main.tex}
\item A card is selected from a pack of 52 cards.
\begin{enumerate}[label=(\alph*)]
    \item How many points are there in the sample space?
    \item Calculate the probability that the card is an ace of spades.
    \item Calculate the probability that the card is (i) an ace and (ii) black card.
\end{enumerate}
\solution
%\input{exemplar/11/16/3/4/main2.tex}
\item The probability that a non leap year selected at random will contain 53 sundays.
\\
\solution
%\input{exemplar/10/13/1/19/main.tex}
\item One of the four persons John, Rita, Aslam or Gurpreet will be promoted next
month. Consequently the sample space consists of four elementary outcomes
S = {John promoted, Rita promoted, Aslam promoted, Gurpreet promoted}
You are told that the chances of John’s promotion is same as that of Gurpreet,
Rita’s chances of promotion are twice as likely as Johns. Aslam’s chances are
four times that of John.
\begin{enumerate}
	\item Determine
	\begin{enumerate}
		\item P (John promoted)
		\item P (Rita promoted)
		\item P (Aslam promoted)
		\item P (Gurpreet promoted)
	\end{enumerate}
	\item If A = {John promoted or Gurpreet promoted}, find P (A).
\end{enumerate}
\solution
%\input{exemplar/11/16/3/10/main.tex}
\item A card is drawn from a deck of 52 cards. Find the probability of getting a king or a heart or a red card.\\
\solution
%\input{exemplar/11/16/3/15/main.tex}
\item The probability that a student will pass his examination is 0.73, the probability of
the student getting a compartment is 0.13, and the probability that the student will
either pass or get compartment is 0.96. State True or False.\\
\solution
%\input{exemplar/11/16/3/31/main.tex}
\item A card is selected from a pack of 52 cards\\
\begin{enumerate}[label=(\alph*)]
\item How many points are there in the sample space?
\item Calculate the probability that the cards is an ace of spades.
\item Calculate the probability that the card is (i) an ace (ii)black card.\\
\end{enumerate}
%\input{ncert/11/16/3/4_1/Prob_4.tex}
\item In a non-leap year, the probability of having 53 tuesdays or 53 wednesdays is\\
\solution
%\input{exemplar/11/16/3/18/main.tex}
\item There are 1000 sealed envelopes in a box, 10 of them contain a cash prize of
Rs 100 each, 100 of them contain a cash prize of Rs 50 each and 200 of them
contain a cash prize of Rs 10 each and rest do not contain any cash prize. If they
are well shuffled and an envelope is picked up out, what is the probability that it
contains no cash prize?\\
\solution
%\input{exemplar/10/13/3/34/main.tex}
\item 
A die is thrown and a card is selected at random from a deck of 52 playing cards. The probability of getting an even number on the die and a spade card.\\
\solution
%\input{exemplar/12/13/3/78/main.tex}
\item
If 4-digit numbers greater than 5,000 are randomly formed from the digits 0, 1, 3, 5, and 7, what is the probability of forming a number divisible by 5 when:
\begin{enumerate}
    \item The digits are repeated?
    \item The repetition of digits is not allowed?
\end{enumerate}
\solution
%\input{ncert/11/16/4/9/main.tex}
\item Consider the probability space $\brak{\Omega, \mathcal{G}, P}$ where $\Omega = [0,2]$ and $\mathcal{G} = \cbrak{\phi, \Omega, [0,1], (1,2]}$. Let $X$ and $Y$ be two functions on $\Omega$ defined as
\begin{align*}
    X(\omega) = 
    \begin{cases}
        1 & \text{if }\omega \in [0, 1]\\
        2 & \text{if }\omega \in (1, 2]
    \end{cases}
\end{align*}
and
\begin{align*}
    Y(\omega) = 
    \begin{cases}
        2 & \text{if }\omega \in [0, 1.5]\\
        3 & \text{if }\omega \in (1.5, 2].
    \end{cases}
\end{align*}
Then which one of the following statements is true?
\begin{enumerate}
    \item [(A)] $X$ is a random variable with respect to $\mathcal{G}$, but $Y$ is not a random variable with respect to $\mathcal{G}$.
    \item [(B)] $Y$ is a random variable with respect to $\mathcal{G}$, but $X$ is not a random variable with respect to $\mathcal{G}$.
    \item [(C)] Neither $X$ nor $Y$ is a random variable with respect to $\mathcal{G}$.
    \item [(D)] Both $X$ and $Y$ are random variables with respect to $\mathcal{G}$.
\end{enumerate} \hfill (GATE ST 2023)\\
\solution
%\input{gate/ST/2023/14/main.tex}
	\item  A die is loaded in such a way that each odd number is twice as likely to occur as
each even number. Find $P(G)$, where $G$ is the event that a number greater than
3 occurs on a single roll of the die.
\\
\solution
		%\input{exemplar/11/16/3/5/main.tex}
	\item All the jacks, queens and kings are removed from a deck of 52 playing cards. The remaining cards are well shuffled and then one card is drawn at random. Giving ace a value 1 similar value for other cards, find the probability that the card has a value 
		\begin{enumerate}
			\item 7
			\item greater than 7
			\item less than 7
		\end{enumerate}
		%\input{exemplar/10/13/3/30/main.tex}
  \item A Lot consists of 48 mobile phones of which 42 are good, 3 have only minor defects and 3 have major defects.Varnika will buy a phone if it is good but the trader will only buy a mobile if it has no major defects. One phone is selected at random from the lot. What is the probability that it is
\begin{enumerate}
	\item acceptable to Varnika?
            \item acceptable to the trader?
\end{enumerate}
\solution
	%\input{exemplar/10/13/3/40/main.tex}
 \item A student says that if you throw a die, it will show up 1 or not 1. Therefore, the probability of getting 1 and the probability of getting 'not 1' each is equal to $\frac{1}{2}$. Is this correct? Give reasons.\\
 \solution
        %\input{exemplar/10/13/2/9/main.tex}
   \item Four candidates A, B, C, D have ap-
plied for the assignment to coach a school cricket
team. If A is twice as likely to be selected as B, and
B and C are given about the same chance of being
selected, while C is twice as likely to be selected
as D, what are the probabilities that
\begin{enumerate}
\item C will be selected?
\item A will not be selected?
\end{enumerate}
	%\input{exemplar/11/16/3/9/main.tex}
 \item A bag contain 24 balls of which $x$ balls are red, $2x$ are white and $3x$ are blue. A ball is selected at random, What is the probability that it is
\begin{enumerate}[label=\alph*)]
\item not red ?
\item white ?
\end{enumerate}
%\input{exemplar/10/13/3/41/main.tex}
If the letters of the word ASSASSINATION are arranged at random. Find the Probability that
\begin{enumerate}[label=(\alph*)]
\item Four $S's$ come consecutively in the word
\item Two  $I's$ and two $N's$ come together
\item All $A's$ are not coming together
\item No two $A's$ are coming together
\end{enumerate}
%\input{exemplar/11/16/3/14/main.tex}
	\item One urn contains two black balls (labelled B1 and B2) and one white ball. A
	second urn contains one black ball and two white balls (labelled W1 and W2).
	Suppose the following experiment is performed. One of the two urns is chosen
	at random. Next a ball is randomly chosen from the urn. Then a second ball is
	chosen at random from the same urn without replacing the first ball.
	
	\begin{enumerate}
	\item What is the probability that two black balls are chosen?
	
	\item What is the probability that two balls of opposite colour are chosen?
	\end{enumerate}
	\solution
	%\input{exemplar/11/16/3/12/main1.tex}
\end{enumerate}

		%
\item 
Out of 100 students, two sections of 40 and 60 are formed. If you and your friend are among the 100 students, what is the probability that
\begin{enumerate}
\item you both enter the same section?
\item you both enter the different sections?
\end{enumerate}
\solution
		%\begin{enumerate}[label=\thesection.\arabic*,ref=\thesection.\theenumi]
	\item One card is drawn from a well-shuffled deck of 52 cards. Find the probability of getting
\begin{enumerate}
\item A king of red colour 
\item A face card 
\item A red face card
\item The jack of hearts
\item A spade
\item The queen of diamonds

\end{enumerate}
\solution
		%\input{ncert/10/15/1/14/main.tex}
	\item Five cards—the ten, jack, queen, king and ace of diamonds, are well-shuffled with their face downwards. One card is then picked up at random.
\begin{enumerate}
\item
What is the probability that the card is the queen? 
\item
If the queen is drawn and put aside, what is the probability that the second card picked up is (a) an ace? (b) a queen?\\
\end{enumerate}
\solution
		%\input{ncert/10/15/1/15/defs.tex}
	\item A bag contains $5$ red balls and some blue balls. If the probability of drawing a blue ball is double that if a red ball, determine the number of blue balls in the bag. 
		\\
\solution
		%\input{ncert/10/15/2/3/defs.tex}
	\item A card is selected from a pack of 52 cards.
 \begin{enumerate}[label=(\alph*)] 
                 \item How many points are there in the sample space?
                 \item Calculate the probability that the card is an ace of spades.
                 \item Calculate the probability that the card is (i) an ace and (ii) black card.
 \end{enumerate}
\solution
		%\input{ncert/11/16/3/4/main.tex}
\item Four cards are drawn from a well-shuffled deck of 52 cards. What is the probability of obtaining 3 diamonds and one spade.
\\
\solution
		%\input{ncert/11/16/4/2/defs.tex}
\item In a certain lottery 10,000 tickets are sold and ten equal prizes are awarded. What is the probability of not getting a prize if you buy (a) one ticket (b) two tickets (c) 10 tickets ?	
\\
\solution
		%\input{ncert/11/16/4/4/defs.tex}
		%
\item 
Out of 100 students, two sections of 40 and 60 are formed. If you and your friend are among the 100 students, what is the probability that
\begin{enumerate}
\item you both enter the same section?
\item you both enter the different sections?
\end{enumerate}
\solution
		%\input{ncert/11/16/4/5/defs.tex}
	\item 
The number lock of a suitcase has 4 wheels each labelled with ten digits i.e. from 0 to 9.The lock opens with a sequence of four digits with no repeats.What is the probability of a person getting the right sequence to open the suitcase.
\\
\solution
		%\input{ncert/11/16/4/10/defs.tex}
		%
\item 
Two cards are drawn at random and without replacement from a pack of 52 playing cards. Find the probability that both the cards are black.
\\
\solution
		%\input{ncert/12/13/2/2/defs.tex}
		\item A box of oranges is inspected by examining three randomly selected oranges drawn without replacement. If all the three oranges are good, the box is approved for sale, otherwise, it is rejected. Find the probability that a box containing 15 oranges out of which 12 are good and 3 are bad ones will be approved for sale.
		\label{ncert/12/13/2/3/defs.tex}
		\item Two balls are drawn at random with replacement from a box containing 10 black and 8 red balls. Find the probability that
		\label{ncert/12/13/2/12}
\begin{enumerate}
\item both balls are red.
\item first ball is black and second is red.
\item one of them is black and other is red.
\end{enumerate}

\item In a hostel, 60\% of the students read Hindi newspaper, 40\% read English newspaper and 20\% read both Hindi and English newspapers. A student is selected at random.
		\label{ncert/12/13/2/15}
\begin{enumerate}
\item Find the probability that she reads neither Hindi nor English newspapers.
\item If she reads Hindi newspaper, find the probability that she reads English newspaper.
\item If she reads English newspaper, find the probability that she reads Hindi newspaper.\\
\end{enumerate}
\item The probability of obtaining an even prime number on each die, when a pair of dice is rolled is 
\begin{enumerate}
    \item $0$ 
    
    \item $\frac{1}{3}$ 
    
    \item $\frac{1}{12}$ 
    
    \item $\frac{1}{36}$ 
\end{enumerate}
\solution
		%\input{ncert/12/13/2/17/defs.tex}
	\item A bag contains 4 red and 4 black balls, another bag contains 2 red and 6 black balls. One of the two bags is selected at random and a ball is drawn from the bag which is found to be red. Find the probability that the ball is drawn from the first bag.
\\
\solution
		%\input{ncert/12/13/3/2/main.tex}
  \item
  Cards with numbers 2 to 101 are placed in a box. A card is selected at random.Find the probability that the card has
\begin{enumerate}[label=(\roman*)]
	\item an even number 
	\item a square number
\end{enumerate}
\solution
%\input{exemplar/10/13/3/32/main.tex}
\item
The king, queen and jack of clubs are removed from a deck of 52 playing cards and then well shuffled. Now one card is drawn at random from the remaining cards.  Determine the probability that the card is
\begin{enumerate}[label=(\roman*)]
\item a club
\item 10 of hearts
\end{enumerate}
\solution
%\input{exemplar/10/13/3/29/main.tex}
\item A team of medical students doing their internship have to assist during surgeries
at a city hospital. The probabilities of surgeries rated as very complex, complex,
routine, simple or very simple are respectively, 0.15, 0.20, 0.31, 0.26, .08. Find
the probabilities that a particular surgery will be rated
\begin{enumerate}
	\item complex or very complex;
	\item neither very complex nor very simple;
	\item routine or complex
	\item routine or simple
\end{enumerate}
\solution
%\input{exemplar/11/16/3/8(1)/main.tex}
\item A card is selected from a pack of 52 cards.
\begin{enumerate}[label=(\alph*)]
    \item How many points are there in the sample space?
    \item Calculate the probability that the card is an ace of spades.
    \item Calculate the probability that the card is (i) an ace and (ii) black card.
\end{enumerate}
\solution
%\input{exemplar/11/16/3/4/main2.tex}
\item The probability that a non leap year selected at random will contain 53 sundays.
\\
\solution
%\input{exemplar/10/13/1/19/main.tex}
\item One of the four persons John, Rita, Aslam or Gurpreet will be promoted next
month. Consequently the sample space consists of four elementary outcomes
S = {John promoted, Rita promoted, Aslam promoted, Gurpreet promoted}
You are told that the chances of John’s promotion is same as that of Gurpreet,
Rita’s chances of promotion are twice as likely as Johns. Aslam’s chances are
four times that of John.
\begin{enumerate}
	\item Determine
	\begin{enumerate}
		\item P (John promoted)
		\item P (Rita promoted)
		\item P (Aslam promoted)
		\item P (Gurpreet promoted)
	\end{enumerate}
	\item If A = {John promoted or Gurpreet promoted}, find P (A).
\end{enumerate}
\solution
%\input{exemplar/11/16/3/10/main.tex}
\item A card is drawn from a deck of 52 cards. Find the probability of getting a king or a heart or a red card.\\
\solution
%\input{exemplar/11/16/3/15/main.tex}
\item The probability that a student will pass his examination is 0.73, the probability of
the student getting a compartment is 0.13, and the probability that the student will
either pass or get compartment is 0.96. State True or False.\\
\solution
%\input{exemplar/11/16/3/31/main.tex}
\item A card is selected from a pack of 52 cards\\
\begin{enumerate}[label=(\alph*)]
\item How many points are there in the sample space?
\item Calculate the probability that the cards is an ace of spades.
\item Calculate the probability that the card is (i) an ace (ii)black card.\\
\end{enumerate}
%\input{ncert/11/16/3/4_1/Prob_4.tex}
\item In a non-leap year, the probability of having 53 tuesdays or 53 wednesdays is\\
\solution
%\input{exemplar/11/16/3/18/main.tex}
\item There are 1000 sealed envelopes in a box, 10 of them contain a cash prize of
Rs 100 each, 100 of them contain a cash prize of Rs 50 each and 200 of them
contain a cash prize of Rs 10 each and rest do not contain any cash prize. If they
are well shuffled and an envelope is picked up out, what is the probability that it
contains no cash prize?\\
\solution
%\input{exemplar/10/13/3/34/main.tex}
\item 
A die is thrown and a card is selected at random from a deck of 52 playing cards. The probability of getting an even number on the die and a spade card.\\
\solution
%\input{exemplar/12/13/3/78/main.tex}
\item
If 4-digit numbers greater than 5,000 are randomly formed from the digits 0, 1, 3, 5, and 7, what is the probability of forming a number divisible by 5 when:
\begin{enumerate}
    \item The digits are repeated?
    \item The repetition of digits is not allowed?
\end{enumerate}
\solution
%\input{ncert/11/16/4/9/main.tex}
\item Consider the probability space $\brak{\Omega, \mathcal{G}, P}$ where $\Omega = [0,2]$ and $\mathcal{G} = \cbrak{\phi, \Omega, [0,1], (1,2]}$. Let $X$ and $Y$ be two functions on $\Omega$ defined as
\begin{align*}
    X(\omega) = 
    \begin{cases}
        1 & \text{if }\omega \in [0, 1]\\
        2 & \text{if }\omega \in (1, 2]
    \end{cases}
\end{align*}
and
\begin{align*}
    Y(\omega) = 
    \begin{cases}
        2 & \text{if }\omega \in [0, 1.5]\\
        3 & \text{if }\omega \in (1.5, 2].
    \end{cases}
\end{align*}
Then which one of the following statements is true?
\begin{enumerate}
    \item [(A)] $X$ is a random variable with respect to $\mathcal{G}$, but $Y$ is not a random variable with respect to $\mathcal{G}$.
    \item [(B)] $Y$ is a random variable with respect to $\mathcal{G}$, but $X$ is not a random variable with respect to $\mathcal{G}$.
    \item [(C)] Neither $X$ nor $Y$ is a random variable with respect to $\mathcal{G}$.
    \item [(D)] Both $X$ and $Y$ are random variables with respect to $\mathcal{G}$.
\end{enumerate} \hfill (GATE ST 2023)\\
\solution
%\input{gate/ST/2023/14/main.tex}
	\item  A die is loaded in such a way that each odd number is twice as likely to occur as
each even number. Find $P(G)$, where $G$ is the event that a number greater than
3 occurs on a single roll of the die.
\\
\solution
		%\input{exemplar/11/16/3/5/main.tex}
	\item All the jacks, queens and kings are removed from a deck of 52 playing cards. The remaining cards are well shuffled and then one card is drawn at random. Giving ace a value 1 similar value for other cards, find the probability that the card has a value 
		\begin{enumerate}
			\item 7
			\item greater than 7
			\item less than 7
		\end{enumerate}
		%\input{exemplar/10/13/3/30/main.tex}
  \item A Lot consists of 48 mobile phones of which 42 are good, 3 have only minor defects and 3 have major defects.Varnika will buy a phone if it is good but the trader will only buy a mobile if it has no major defects. One phone is selected at random from the lot. What is the probability that it is
\begin{enumerate}
	\item acceptable to Varnika?
            \item acceptable to the trader?
\end{enumerate}
\solution
	%\input{exemplar/10/13/3/40/main.tex}
 \item A student says that if you throw a die, it will show up 1 or not 1. Therefore, the probability of getting 1 and the probability of getting 'not 1' each is equal to $\frac{1}{2}$. Is this correct? Give reasons.\\
 \solution
        %\input{exemplar/10/13/2/9/main.tex}
   \item Four candidates A, B, C, D have ap-
plied for the assignment to coach a school cricket
team. If A is twice as likely to be selected as B, and
B and C are given about the same chance of being
selected, while C is twice as likely to be selected
as D, what are the probabilities that
\begin{enumerate}
\item C will be selected?
\item A will not be selected?
\end{enumerate}
	%\input{exemplar/11/16/3/9/main.tex}
 \item A bag contain 24 balls of which $x$ balls are red, $2x$ are white and $3x$ are blue. A ball is selected at random, What is the probability that it is
\begin{enumerate}[label=\alph*)]
\item not red ?
\item white ?
\end{enumerate}
%\input{exemplar/10/13/3/41/main.tex}
If the letters of the word ASSASSINATION are arranged at random. Find the Probability that
\begin{enumerate}[label=(\alph*)]
\item Four $S's$ come consecutively in the word
\item Two  $I's$ and two $N's$ come together
\item All $A's$ are not coming together
\item No two $A's$ are coming together
\end{enumerate}
%\input{exemplar/11/16/3/14/main.tex}
	\item One urn contains two black balls (labelled B1 and B2) and one white ball. A
	second urn contains one black ball and two white balls (labelled W1 and W2).
	Suppose the following experiment is performed. One of the two urns is chosen
	at random. Next a ball is randomly chosen from the urn. Then a second ball is
	chosen at random from the same urn without replacing the first ball.
	
	\begin{enumerate}
	\item What is the probability that two black balls are chosen?
	
	\item What is the probability that two balls of opposite colour are chosen?
	\end{enumerate}
	\solution
	%\input{exemplar/11/16/3/12/main1.tex}
\end{enumerate}

	\item 
The number lock of a suitcase has 4 wheels each labelled with ten digits i.e. from 0 to 9.The lock opens with a sequence of four digits with no repeats.What is the probability of a person getting the right sequence to open the suitcase.
\\
\solution
		%\begin{enumerate}[label=\thesection.\arabic*,ref=\thesection.\theenumi]
	\item One card is drawn from a well-shuffled deck of 52 cards. Find the probability of getting
\begin{enumerate}
\item A king of red colour 
\item A face card 
\item A red face card
\item The jack of hearts
\item A spade
\item The queen of diamonds

\end{enumerate}
\solution
		%\input{ncert/10/15/1/14/main.tex}
	\item Five cards—the ten, jack, queen, king and ace of diamonds, are well-shuffled with their face downwards. One card is then picked up at random.
\begin{enumerate}
\item
What is the probability that the card is the queen? 
\item
If the queen is drawn and put aside, what is the probability that the second card picked up is (a) an ace? (b) a queen?\\
\end{enumerate}
\solution
		%\input{ncert/10/15/1/15/defs.tex}
	\item A bag contains $5$ red balls and some blue balls. If the probability of drawing a blue ball is double that if a red ball, determine the number of blue balls in the bag. 
		\\
\solution
		%\input{ncert/10/15/2/3/defs.tex}
	\item A card is selected from a pack of 52 cards.
 \begin{enumerate}[label=(\alph*)] 
                 \item How many points are there in the sample space?
                 \item Calculate the probability that the card is an ace of spades.
                 \item Calculate the probability that the card is (i) an ace and (ii) black card.
 \end{enumerate}
\solution
		%\input{ncert/11/16/3/4/main.tex}
\item Four cards are drawn from a well-shuffled deck of 52 cards. What is the probability of obtaining 3 diamonds and one spade.
\\
\solution
		%\input{ncert/11/16/4/2/defs.tex}
\item In a certain lottery 10,000 tickets are sold and ten equal prizes are awarded. What is the probability of not getting a prize if you buy (a) one ticket (b) two tickets (c) 10 tickets ?	
\\
\solution
		%\input{ncert/11/16/4/4/defs.tex}
		%
\item 
Out of 100 students, two sections of 40 and 60 are formed. If you and your friend are among the 100 students, what is the probability that
\begin{enumerate}
\item you both enter the same section?
\item you both enter the different sections?
\end{enumerate}
\solution
		%\input{ncert/11/16/4/5/defs.tex}
	\item 
The number lock of a suitcase has 4 wheels each labelled with ten digits i.e. from 0 to 9.The lock opens with a sequence of four digits with no repeats.What is the probability of a person getting the right sequence to open the suitcase.
\\
\solution
		%\input{ncert/11/16/4/10/defs.tex}
		%
\item 
Two cards are drawn at random and without replacement from a pack of 52 playing cards. Find the probability that both the cards are black.
\\
\solution
		%\input{ncert/12/13/2/2/defs.tex}
		\item A box of oranges is inspected by examining three randomly selected oranges drawn without replacement. If all the three oranges are good, the box is approved for sale, otherwise, it is rejected. Find the probability that a box containing 15 oranges out of which 12 are good and 3 are bad ones will be approved for sale.
		\label{ncert/12/13/2/3/defs.tex}
		\item Two balls are drawn at random with replacement from a box containing 10 black and 8 red balls. Find the probability that
		\label{ncert/12/13/2/12}
\begin{enumerate}
\item both balls are red.
\item first ball is black and second is red.
\item one of them is black and other is red.
\end{enumerate}

\item In a hostel, 60\% of the students read Hindi newspaper, 40\% read English newspaper and 20\% read both Hindi and English newspapers. A student is selected at random.
		\label{ncert/12/13/2/15}
\begin{enumerate}
\item Find the probability that she reads neither Hindi nor English newspapers.
\item If she reads Hindi newspaper, find the probability that she reads English newspaper.
\item If she reads English newspaper, find the probability that she reads Hindi newspaper.\\
\end{enumerate}
\item The probability of obtaining an even prime number on each die, when a pair of dice is rolled is 
\begin{enumerate}
    \item $0$ 
    
    \item $\frac{1}{3}$ 
    
    \item $\frac{1}{12}$ 
    
    \item $\frac{1}{36}$ 
\end{enumerate}
\solution
		%\input{ncert/12/13/2/17/defs.tex}
	\item A bag contains 4 red and 4 black balls, another bag contains 2 red and 6 black balls. One of the two bags is selected at random and a ball is drawn from the bag which is found to be red. Find the probability that the ball is drawn from the first bag.
\\
\solution
		%\input{ncert/12/13/3/2/main.tex}
  \item
  Cards with numbers 2 to 101 are placed in a box. A card is selected at random.Find the probability that the card has
\begin{enumerate}[label=(\roman*)]
	\item an even number 
	\item a square number
\end{enumerate}
\solution
%\input{exemplar/10/13/3/32/main.tex}
\item
The king, queen and jack of clubs are removed from a deck of 52 playing cards and then well shuffled. Now one card is drawn at random from the remaining cards.  Determine the probability that the card is
\begin{enumerate}[label=(\roman*)]
\item a club
\item 10 of hearts
\end{enumerate}
\solution
%\input{exemplar/10/13/3/29/main.tex}
\item A team of medical students doing their internship have to assist during surgeries
at a city hospital. The probabilities of surgeries rated as very complex, complex,
routine, simple or very simple are respectively, 0.15, 0.20, 0.31, 0.26, .08. Find
the probabilities that a particular surgery will be rated
\begin{enumerate}
	\item complex or very complex;
	\item neither very complex nor very simple;
	\item routine or complex
	\item routine or simple
\end{enumerate}
\solution
%\input{exemplar/11/16/3/8(1)/main.tex}
\item A card is selected from a pack of 52 cards.
\begin{enumerate}[label=(\alph*)]
    \item How many points are there in the sample space?
    \item Calculate the probability that the card is an ace of spades.
    \item Calculate the probability that the card is (i) an ace and (ii) black card.
\end{enumerate}
\solution
%\input{exemplar/11/16/3/4/main2.tex}
\item The probability that a non leap year selected at random will contain 53 sundays.
\\
\solution
%\input{exemplar/10/13/1/19/main.tex}
\item One of the four persons John, Rita, Aslam or Gurpreet will be promoted next
month. Consequently the sample space consists of four elementary outcomes
S = {John promoted, Rita promoted, Aslam promoted, Gurpreet promoted}
You are told that the chances of John’s promotion is same as that of Gurpreet,
Rita’s chances of promotion are twice as likely as Johns. Aslam’s chances are
four times that of John.
\begin{enumerate}
	\item Determine
	\begin{enumerate}
		\item P (John promoted)
		\item P (Rita promoted)
		\item P (Aslam promoted)
		\item P (Gurpreet promoted)
	\end{enumerate}
	\item If A = {John promoted or Gurpreet promoted}, find P (A).
\end{enumerate}
\solution
%\input{exemplar/11/16/3/10/main.tex}
\item A card is drawn from a deck of 52 cards. Find the probability of getting a king or a heart or a red card.\\
\solution
%\input{exemplar/11/16/3/15/main.tex}
\item The probability that a student will pass his examination is 0.73, the probability of
the student getting a compartment is 0.13, and the probability that the student will
either pass or get compartment is 0.96. State True or False.\\
\solution
%\input{exemplar/11/16/3/31/main.tex}
\item A card is selected from a pack of 52 cards\\
\begin{enumerate}[label=(\alph*)]
\item How many points are there in the sample space?
\item Calculate the probability that the cards is an ace of spades.
\item Calculate the probability that the card is (i) an ace (ii)black card.\\
\end{enumerate}
%\input{ncert/11/16/3/4_1/Prob_4.tex}
\item In a non-leap year, the probability of having 53 tuesdays or 53 wednesdays is\\
\solution
%\input{exemplar/11/16/3/18/main.tex}
\item There are 1000 sealed envelopes in a box, 10 of them contain a cash prize of
Rs 100 each, 100 of them contain a cash prize of Rs 50 each and 200 of them
contain a cash prize of Rs 10 each and rest do not contain any cash prize. If they
are well shuffled and an envelope is picked up out, what is the probability that it
contains no cash prize?\\
\solution
%\input{exemplar/10/13/3/34/main.tex}
\item 
A die is thrown and a card is selected at random from a deck of 52 playing cards. The probability of getting an even number on the die and a spade card.\\
\solution
%\input{exemplar/12/13/3/78/main.tex}
\item
If 4-digit numbers greater than 5,000 are randomly formed from the digits 0, 1, 3, 5, and 7, what is the probability of forming a number divisible by 5 when:
\begin{enumerate}
    \item The digits are repeated?
    \item The repetition of digits is not allowed?
\end{enumerate}
\solution
%\input{ncert/11/16/4/9/main.tex}
\item Consider the probability space $\brak{\Omega, \mathcal{G}, P}$ where $\Omega = [0,2]$ and $\mathcal{G} = \cbrak{\phi, \Omega, [0,1], (1,2]}$. Let $X$ and $Y$ be two functions on $\Omega$ defined as
\begin{align*}
    X(\omega) = 
    \begin{cases}
        1 & \text{if }\omega \in [0, 1]\\
        2 & \text{if }\omega \in (1, 2]
    \end{cases}
\end{align*}
and
\begin{align*}
    Y(\omega) = 
    \begin{cases}
        2 & \text{if }\omega \in [0, 1.5]\\
        3 & \text{if }\omega \in (1.5, 2].
    \end{cases}
\end{align*}
Then which one of the following statements is true?
\begin{enumerate}
    \item [(A)] $X$ is a random variable with respect to $\mathcal{G}$, but $Y$ is not a random variable with respect to $\mathcal{G}$.
    \item [(B)] $Y$ is a random variable with respect to $\mathcal{G}$, but $X$ is not a random variable with respect to $\mathcal{G}$.
    \item [(C)] Neither $X$ nor $Y$ is a random variable with respect to $\mathcal{G}$.
    \item [(D)] Both $X$ and $Y$ are random variables with respect to $\mathcal{G}$.
\end{enumerate} \hfill (GATE ST 2023)\\
\solution
%\input{gate/ST/2023/14/main.tex}
	\item  A die is loaded in such a way that each odd number is twice as likely to occur as
each even number. Find $P(G)$, where $G$ is the event that a number greater than
3 occurs on a single roll of the die.
\\
\solution
		%\input{exemplar/11/16/3/5/main.tex}
	\item All the jacks, queens and kings are removed from a deck of 52 playing cards. The remaining cards are well shuffled and then one card is drawn at random. Giving ace a value 1 similar value for other cards, find the probability that the card has a value 
		\begin{enumerate}
			\item 7
			\item greater than 7
			\item less than 7
		\end{enumerate}
		%\input{exemplar/10/13/3/30/main.tex}
  \item A Lot consists of 48 mobile phones of which 42 are good, 3 have only minor defects and 3 have major defects.Varnika will buy a phone if it is good but the trader will only buy a mobile if it has no major defects. One phone is selected at random from the lot. What is the probability that it is
\begin{enumerate}
	\item acceptable to Varnika?
            \item acceptable to the trader?
\end{enumerate}
\solution
	%\input{exemplar/10/13/3/40/main.tex}
 \item A student says that if you throw a die, it will show up 1 or not 1. Therefore, the probability of getting 1 and the probability of getting 'not 1' each is equal to $\frac{1}{2}$. Is this correct? Give reasons.\\
 \solution
        %\input{exemplar/10/13/2/9/main.tex}
   \item Four candidates A, B, C, D have ap-
plied for the assignment to coach a school cricket
team. If A is twice as likely to be selected as B, and
B and C are given about the same chance of being
selected, while C is twice as likely to be selected
as D, what are the probabilities that
\begin{enumerate}
\item C will be selected?
\item A will not be selected?
\end{enumerate}
	%\input{exemplar/11/16/3/9/main.tex}
 \item A bag contain 24 balls of which $x$ balls are red, $2x$ are white and $3x$ are blue. A ball is selected at random, What is the probability that it is
\begin{enumerate}[label=\alph*)]
\item not red ?
\item white ?
\end{enumerate}
%\input{exemplar/10/13/3/41/main.tex}
If the letters of the word ASSASSINATION are arranged at random. Find the Probability that
\begin{enumerate}[label=(\alph*)]
\item Four $S's$ come consecutively in the word
\item Two  $I's$ and two $N's$ come together
\item All $A's$ are not coming together
\item No two $A's$ are coming together
\end{enumerate}
%\input{exemplar/11/16/3/14/main.tex}
	\item One urn contains two black balls (labelled B1 and B2) and one white ball. A
	second urn contains one black ball and two white balls (labelled W1 and W2).
	Suppose the following experiment is performed. One of the two urns is chosen
	at random. Next a ball is randomly chosen from the urn. Then a second ball is
	chosen at random from the same urn without replacing the first ball.
	
	\begin{enumerate}
	\item What is the probability that two black balls are chosen?
	
	\item What is the probability that two balls of opposite colour are chosen?
	\end{enumerate}
	\solution
	%\input{exemplar/11/16/3/12/main1.tex}
\end{enumerate}

		%
\item 
Two cards are drawn at random and without replacement from a pack of 52 playing cards. Find the probability that both the cards are black.
\\
\solution
		%\begin{enumerate}[label=\thesection.\arabic*,ref=\thesection.\theenumi]
	\item One card is drawn from a well-shuffled deck of 52 cards. Find the probability of getting
\begin{enumerate}
\item A king of red colour 
\item A face card 
\item A red face card
\item The jack of hearts
\item A spade
\item The queen of diamonds

\end{enumerate}
\solution
		%\input{ncert/10/15/1/14/main.tex}
	\item Five cards—the ten, jack, queen, king and ace of diamonds, are well-shuffled with their face downwards. One card is then picked up at random.
\begin{enumerate}
\item
What is the probability that the card is the queen? 
\item
If the queen is drawn and put aside, what is the probability that the second card picked up is (a) an ace? (b) a queen?\\
\end{enumerate}
\solution
		%\input{ncert/10/15/1/15/defs.tex}
	\item A bag contains $5$ red balls and some blue balls. If the probability of drawing a blue ball is double that if a red ball, determine the number of blue balls in the bag. 
		\\
\solution
		%\input{ncert/10/15/2/3/defs.tex}
	\item A card is selected from a pack of 52 cards.
 \begin{enumerate}[label=(\alph*)] 
                 \item How many points are there in the sample space?
                 \item Calculate the probability that the card is an ace of spades.
                 \item Calculate the probability that the card is (i) an ace and (ii) black card.
 \end{enumerate}
\solution
		%\input{ncert/11/16/3/4/main.tex}
\item Four cards are drawn from a well-shuffled deck of 52 cards. What is the probability of obtaining 3 diamonds and one spade.
\\
\solution
		%\input{ncert/11/16/4/2/defs.tex}
\item In a certain lottery 10,000 tickets are sold and ten equal prizes are awarded. What is the probability of not getting a prize if you buy (a) one ticket (b) two tickets (c) 10 tickets ?	
\\
\solution
		%\input{ncert/11/16/4/4/defs.tex}
		%
\item 
Out of 100 students, two sections of 40 and 60 are formed. If you and your friend are among the 100 students, what is the probability that
\begin{enumerate}
\item you both enter the same section?
\item you both enter the different sections?
\end{enumerate}
\solution
		%\input{ncert/11/16/4/5/defs.tex}
	\item 
The number lock of a suitcase has 4 wheels each labelled with ten digits i.e. from 0 to 9.The lock opens with a sequence of four digits with no repeats.What is the probability of a person getting the right sequence to open the suitcase.
\\
\solution
		%\input{ncert/11/16/4/10/defs.tex}
		%
\item 
Two cards are drawn at random and without replacement from a pack of 52 playing cards. Find the probability that both the cards are black.
\\
\solution
		%\input{ncert/12/13/2/2/defs.tex}
		\item A box of oranges is inspected by examining three randomly selected oranges drawn without replacement. If all the three oranges are good, the box is approved for sale, otherwise, it is rejected. Find the probability that a box containing 15 oranges out of which 12 are good and 3 are bad ones will be approved for sale.
		\label{ncert/12/13/2/3/defs.tex}
		\item Two balls are drawn at random with replacement from a box containing 10 black and 8 red balls. Find the probability that
		\label{ncert/12/13/2/12}
\begin{enumerate}
\item both balls are red.
\item first ball is black and second is red.
\item one of them is black and other is red.
\end{enumerate}

\item In a hostel, 60\% of the students read Hindi newspaper, 40\% read English newspaper and 20\% read both Hindi and English newspapers. A student is selected at random.
		\label{ncert/12/13/2/15}
\begin{enumerate}
\item Find the probability that she reads neither Hindi nor English newspapers.
\item If she reads Hindi newspaper, find the probability that she reads English newspaper.
\item If she reads English newspaper, find the probability that she reads Hindi newspaper.\\
\end{enumerate}
\item The probability of obtaining an even prime number on each die, when a pair of dice is rolled is 
\begin{enumerate}
    \item $0$ 
    
    \item $\frac{1}{3}$ 
    
    \item $\frac{1}{12}$ 
    
    \item $\frac{1}{36}$ 
\end{enumerate}
\solution
		%\input{ncert/12/13/2/17/defs.tex}
	\item A bag contains 4 red and 4 black balls, another bag contains 2 red and 6 black balls. One of the two bags is selected at random and a ball is drawn from the bag which is found to be red. Find the probability that the ball is drawn from the first bag.
\\
\solution
		%\input{ncert/12/13/3/2/main.tex}
  \item
  Cards with numbers 2 to 101 are placed in a box. A card is selected at random.Find the probability that the card has
\begin{enumerate}[label=(\roman*)]
	\item an even number 
	\item a square number
\end{enumerate}
\solution
%\input{exemplar/10/13/3/32/main.tex}
\item
The king, queen and jack of clubs are removed from a deck of 52 playing cards and then well shuffled. Now one card is drawn at random from the remaining cards.  Determine the probability that the card is
\begin{enumerate}[label=(\roman*)]
\item a club
\item 10 of hearts
\end{enumerate}
\solution
%\input{exemplar/10/13/3/29/main.tex}
\item A team of medical students doing their internship have to assist during surgeries
at a city hospital. The probabilities of surgeries rated as very complex, complex,
routine, simple or very simple are respectively, 0.15, 0.20, 0.31, 0.26, .08. Find
the probabilities that a particular surgery will be rated
\begin{enumerate}
	\item complex or very complex;
	\item neither very complex nor very simple;
	\item routine or complex
	\item routine or simple
\end{enumerate}
\solution
%\input{exemplar/11/16/3/8(1)/main.tex}
\item A card is selected from a pack of 52 cards.
\begin{enumerate}[label=(\alph*)]
    \item How many points are there in the sample space?
    \item Calculate the probability that the card is an ace of spades.
    \item Calculate the probability that the card is (i) an ace and (ii) black card.
\end{enumerate}
\solution
%\input{exemplar/11/16/3/4/main2.tex}
\item The probability that a non leap year selected at random will contain 53 sundays.
\\
\solution
%\input{exemplar/10/13/1/19/main.tex}
\item One of the four persons John, Rita, Aslam or Gurpreet will be promoted next
month. Consequently the sample space consists of four elementary outcomes
S = {John promoted, Rita promoted, Aslam promoted, Gurpreet promoted}
You are told that the chances of John’s promotion is same as that of Gurpreet,
Rita’s chances of promotion are twice as likely as Johns. Aslam’s chances are
four times that of John.
\begin{enumerate}
	\item Determine
	\begin{enumerate}
		\item P (John promoted)
		\item P (Rita promoted)
		\item P (Aslam promoted)
		\item P (Gurpreet promoted)
	\end{enumerate}
	\item If A = {John promoted or Gurpreet promoted}, find P (A).
\end{enumerate}
\solution
%\input{exemplar/11/16/3/10/main.tex}
\item A card is drawn from a deck of 52 cards. Find the probability of getting a king or a heart or a red card.\\
\solution
%\input{exemplar/11/16/3/15/main.tex}
\item The probability that a student will pass his examination is 0.73, the probability of
the student getting a compartment is 0.13, and the probability that the student will
either pass or get compartment is 0.96. State True or False.\\
\solution
%\input{exemplar/11/16/3/31/main.tex}
\item A card is selected from a pack of 52 cards\\
\begin{enumerate}[label=(\alph*)]
\item How many points are there in the sample space?
\item Calculate the probability that the cards is an ace of spades.
\item Calculate the probability that the card is (i) an ace (ii)black card.\\
\end{enumerate}
%\input{ncert/11/16/3/4_1/Prob_4.tex}
\item In a non-leap year, the probability of having 53 tuesdays or 53 wednesdays is\\
\solution
%\input{exemplar/11/16/3/18/main.tex}
\item There are 1000 sealed envelopes in a box, 10 of them contain a cash prize of
Rs 100 each, 100 of them contain a cash prize of Rs 50 each and 200 of them
contain a cash prize of Rs 10 each and rest do not contain any cash prize. If they
are well shuffled and an envelope is picked up out, what is the probability that it
contains no cash prize?\\
\solution
%\input{exemplar/10/13/3/34/main.tex}
\item 
A die is thrown and a card is selected at random from a deck of 52 playing cards. The probability of getting an even number on the die and a spade card.\\
\solution
%\input{exemplar/12/13/3/78/main.tex}
\item
If 4-digit numbers greater than 5,000 are randomly formed from the digits 0, 1, 3, 5, and 7, what is the probability of forming a number divisible by 5 when:
\begin{enumerate}
    \item The digits are repeated?
    \item The repetition of digits is not allowed?
\end{enumerate}
\solution
%\input{ncert/11/16/4/9/main.tex}
\item Consider the probability space $\brak{\Omega, \mathcal{G}, P}$ where $\Omega = [0,2]$ and $\mathcal{G} = \cbrak{\phi, \Omega, [0,1], (1,2]}$. Let $X$ and $Y$ be two functions on $\Omega$ defined as
\begin{align*}
    X(\omega) = 
    \begin{cases}
        1 & \text{if }\omega \in [0, 1]\\
        2 & \text{if }\omega \in (1, 2]
    \end{cases}
\end{align*}
and
\begin{align*}
    Y(\omega) = 
    \begin{cases}
        2 & \text{if }\omega \in [0, 1.5]\\
        3 & \text{if }\omega \in (1.5, 2].
    \end{cases}
\end{align*}
Then which one of the following statements is true?
\begin{enumerate}
    \item [(A)] $X$ is a random variable with respect to $\mathcal{G}$, but $Y$ is not a random variable with respect to $\mathcal{G}$.
    \item [(B)] $Y$ is a random variable with respect to $\mathcal{G}$, but $X$ is not a random variable with respect to $\mathcal{G}$.
    \item [(C)] Neither $X$ nor $Y$ is a random variable with respect to $\mathcal{G}$.
    \item [(D)] Both $X$ and $Y$ are random variables with respect to $\mathcal{G}$.
\end{enumerate} \hfill (GATE ST 2023)\\
\solution
%\input{gate/ST/2023/14/main.tex}
	\item  A die is loaded in such a way that each odd number is twice as likely to occur as
each even number. Find $P(G)$, where $G$ is the event that a number greater than
3 occurs on a single roll of the die.
\\
\solution
		%\input{exemplar/11/16/3/5/main.tex}
	\item All the jacks, queens and kings are removed from a deck of 52 playing cards. The remaining cards are well shuffled and then one card is drawn at random. Giving ace a value 1 similar value for other cards, find the probability that the card has a value 
		\begin{enumerate}
			\item 7
			\item greater than 7
			\item less than 7
		\end{enumerate}
		%\input{exemplar/10/13/3/30/main.tex}
  \item A Lot consists of 48 mobile phones of which 42 are good, 3 have only minor defects and 3 have major defects.Varnika will buy a phone if it is good but the trader will only buy a mobile if it has no major defects. One phone is selected at random from the lot. What is the probability that it is
\begin{enumerate}
	\item acceptable to Varnika?
            \item acceptable to the trader?
\end{enumerate}
\solution
	%\input{exemplar/10/13/3/40/main.tex}
 \item A student says that if you throw a die, it will show up 1 or not 1. Therefore, the probability of getting 1 and the probability of getting 'not 1' each is equal to $\frac{1}{2}$. Is this correct? Give reasons.\\
 \solution
        %\input{exemplar/10/13/2/9/main.tex}
   \item Four candidates A, B, C, D have ap-
plied for the assignment to coach a school cricket
team. If A is twice as likely to be selected as B, and
B and C are given about the same chance of being
selected, while C is twice as likely to be selected
as D, what are the probabilities that
\begin{enumerate}
\item C will be selected?
\item A will not be selected?
\end{enumerate}
	%\input{exemplar/11/16/3/9/main.tex}
 \item A bag contain 24 balls of which $x$ balls are red, $2x$ are white and $3x$ are blue. A ball is selected at random, What is the probability that it is
\begin{enumerate}[label=\alph*)]
\item not red ?
\item white ?
\end{enumerate}
%\input{exemplar/10/13/3/41/main.tex}
If the letters of the word ASSASSINATION are arranged at random. Find the Probability that
\begin{enumerate}[label=(\alph*)]
\item Four $S's$ come consecutively in the word
\item Two  $I's$ and two $N's$ come together
\item All $A's$ are not coming together
\item No two $A's$ are coming together
\end{enumerate}
%\input{exemplar/11/16/3/14/main.tex}
	\item One urn contains two black balls (labelled B1 and B2) and one white ball. A
	second urn contains one black ball and two white balls (labelled W1 and W2).
	Suppose the following experiment is performed. One of the two urns is chosen
	at random. Next a ball is randomly chosen from the urn. Then a second ball is
	chosen at random from the same urn without replacing the first ball.
	
	\begin{enumerate}
	\item What is the probability that two black balls are chosen?
	
	\item What is the probability that two balls of opposite colour are chosen?
	\end{enumerate}
	\solution
	%\input{exemplar/11/16/3/12/main1.tex}
\end{enumerate}

		\item A box of oranges is inspected by examining three randomly selected oranges drawn without replacement. If all the three oranges are good, the box is approved for sale, otherwise, it is rejected. Find the probability that a box containing 15 oranges out of which 12 are good and 3 are bad ones will be approved for sale.
		\label{ncert/12/13/2/3/defs.tex}
		\item Two balls are drawn at random with replacement from a box containing 10 black and 8 red balls. Find the probability that
		\label{ncert/12/13/2/12}
\begin{enumerate}
\item both balls are red.
\item first ball is black and second is red.
\item one of them is black and other is red.
\end{enumerate}

\item In a hostel, 60\% of the students read Hindi newspaper, 40\% read English newspaper and 20\% read both Hindi and English newspapers. A student is selected at random.
		\label{ncert/12/13/2/15}
\begin{enumerate}
\item Find the probability that she reads neither Hindi nor English newspapers.
\item If she reads Hindi newspaper, find the probability that she reads English newspaper.
\item If she reads English newspaper, find the probability that she reads Hindi newspaper.\\
\end{enumerate}
\item The probability of obtaining an even prime number on each die, when a pair of dice is rolled is 
\begin{enumerate}
    \item $0$ 
    
    \item $\frac{1}{3}$ 
    
    \item $\frac{1}{12}$ 
    
    \item $\frac{1}{36}$ 
\end{enumerate}
\solution
		%\begin{enumerate}[label=\thesection.\arabic*,ref=\thesection.\theenumi]
	\item One card is drawn from a well-shuffled deck of 52 cards. Find the probability of getting
\begin{enumerate}
\item A king of red colour 
\item A face card 
\item A red face card
\item The jack of hearts
\item A spade
\item The queen of diamonds

\end{enumerate}
\solution
		%\input{ncert/10/15/1/14/main.tex}
	\item Five cards—the ten, jack, queen, king and ace of diamonds, are well-shuffled with their face downwards. One card is then picked up at random.
\begin{enumerate}
\item
What is the probability that the card is the queen? 
\item
If the queen is drawn and put aside, what is the probability that the second card picked up is (a) an ace? (b) a queen?\\
\end{enumerate}
\solution
		%\input{ncert/10/15/1/15/defs.tex}
	\item A bag contains $5$ red balls and some blue balls. If the probability of drawing a blue ball is double that if a red ball, determine the number of blue balls in the bag. 
		\\
\solution
		%\input{ncert/10/15/2/3/defs.tex}
	\item A card is selected from a pack of 52 cards.
 \begin{enumerate}[label=(\alph*)] 
                 \item How many points are there in the sample space?
                 \item Calculate the probability that the card is an ace of spades.
                 \item Calculate the probability that the card is (i) an ace and (ii) black card.
 \end{enumerate}
\solution
		%\input{ncert/11/16/3/4/main.tex}
\item Four cards are drawn from a well-shuffled deck of 52 cards. What is the probability of obtaining 3 diamonds and one spade.
\\
\solution
		%\input{ncert/11/16/4/2/defs.tex}
\item In a certain lottery 10,000 tickets are sold and ten equal prizes are awarded. What is the probability of not getting a prize if you buy (a) one ticket (b) two tickets (c) 10 tickets ?	
\\
\solution
		%\input{ncert/11/16/4/4/defs.tex}
		%
\item 
Out of 100 students, two sections of 40 and 60 are formed. If you and your friend are among the 100 students, what is the probability that
\begin{enumerate}
\item you both enter the same section?
\item you both enter the different sections?
\end{enumerate}
\solution
		%\input{ncert/11/16/4/5/defs.tex}
	\item 
The number lock of a suitcase has 4 wheels each labelled with ten digits i.e. from 0 to 9.The lock opens with a sequence of four digits with no repeats.What is the probability of a person getting the right sequence to open the suitcase.
\\
\solution
		%\input{ncert/11/16/4/10/defs.tex}
		%
\item 
Two cards are drawn at random and without replacement from a pack of 52 playing cards. Find the probability that both the cards are black.
\\
\solution
		%\input{ncert/12/13/2/2/defs.tex}
		\item A box of oranges is inspected by examining three randomly selected oranges drawn without replacement. If all the three oranges are good, the box is approved for sale, otherwise, it is rejected. Find the probability that a box containing 15 oranges out of which 12 are good and 3 are bad ones will be approved for sale.
		\label{ncert/12/13/2/3/defs.tex}
		\item Two balls are drawn at random with replacement from a box containing 10 black and 8 red balls. Find the probability that
		\label{ncert/12/13/2/12}
\begin{enumerate}
\item both balls are red.
\item first ball is black and second is red.
\item one of them is black and other is red.
\end{enumerate}

\item In a hostel, 60\% of the students read Hindi newspaper, 40\% read English newspaper and 20\% read both Hindi and English newspapers. A student is selected at random.
		\label{ncert/12/13/2/15}
\begin{enumerate}
\item Find the probability that she reads neither Hindi nor English newspapers.
\item If she reads Hindi newspaper, find the probability that she reads English newspaper.
\item If she reads English newspaper, find the probability that she reads Hindi newspaper.\\
\end{enumerate}
\item The probability of obtaining an even prime number on each die, when a pair of dice is rolled is 
\begin{enumerate}
    \item $0$ 
    
    \item $\frac{1}{3}$ 
    
    \item $\frac{1}{12}$ 
    
    \item $\frac{1}{36}$ 
\end{enumerate}
\solution
		%\input{ncert/12/13/2/17/defs.tex}
	\item A bag contains 4 red and 4 black balls, another bag contains 2 red and 6 black balls. One of the two bags is selected at random and a ball is drawn from the bag which is found to be red. Find the probability that the ball is drawn from the first bag.
\\
\solution
		%\input{ncert/12/13/3/2/main.tex}
  \item
  Cards with numbers 2 to 101 are placed in a box. A card is selected at random.Find the probability that the card has
\begin{enumerate}[label=(\roman*)]
	\item an even number 
	\item a square number
\end{enumerate}
\solution
%\input{exemplar/10/13/3/32/main.tex}
\item
The king, queen and jack of clubs are removed from a deck of 52 playing cards and then well shuffled. Now one card is drawn at random from the remaining cards.  Determine the probability that the card is
\begin{enumerate}[label=(\roman*)]
\item a club
\item 10 of hearts
\end{enumerate}
\solution
%\input{exemplar/10/13/3/29/main.tex}
\item A team of medical students doing their internship have to assist during surgeries
at a city hospital. The probabilities of surgeries rated as very complex, complex,
routine, simple or very simple are respectively, 0.15, 0.20, 0.31, 0.26, .08. Find
the probabilities that a particular surgery will be rated
\begin{enumerate}
	\item complex or very complex;
	\item neither very complex nor very simple;
	\item routine or complex
	\item routine or simple
\end{enumerate}
\solution
%\input{exemplar/11/16/3/8(1)/main.tex}
\item A card is selected from a pack of 52 cards.
\begin{enumerate}[label=(\alph*)]
    \item How many points are there in the sample space?
    \item Calculate the probability that the card is an ace of spades.
    \item Calculate the probability that the card is (i) an ace and (ii) black card.
\end{enumerate}
\solution
%\input{exemplar/11/16/3/4/main2.tex}
\item The probability that a non leap year selected at random will contain 53 sundays.
\\
\solution
%\input{exemplar/10/13/1/19/main.tex}
\item One of the four persons John, Rita, Aslam or Gurpreet will be promoted next
month. Consequently the sample space consists of four elementary outcomes
S = {John promoted, Rita promoted, Aslam promoted, Gurpreet promoted}
You are told that the chances of John’s promotion is same as that of Gurpreet,
Rita’s chances of promotion are twice as likely as Johns. Aslam’s chances are
four times that of John.
\begin{enumerate}
	\item Determine
	\begin{enumerate}
		\item P (John promoted)
		\item P (Rita promoted)
		\item P (Aslam promoted)
		\item P (Gurpreet promoted)
	\end{enumerate}
	\item If A = {John promoted or Gurpreet promoted}, find P (A).
\end{enumerate}
\solution
%\input{exemplar/11/16/3/10/main.tex}
\item A card is drawn from a deck of 52 cards. Find the probability of getting a king or a heart or a red card.\\
\solution
%\input{exemplar/11/16/3/15/main.tex}
\item The probability that a student will pass his examination is 0.73, the probability of
the student getting a compartment is 0.13, and the probability that the student will
either pass or get compartment is 0.96. State True or False.\\
\solution
%\input{exemplar/11/16/3/31/main.tex}
\item A card is selected from a pack of 52 cards\\
\begin{enumerate}[label=(\alph*)]
\item How many points are there in the sample space?
\item Calculate the probability that the cards is an ace of spades.
\item Calculate the probability that the card is (i) an ace (ii)black card.\\
\end{enumerate}
%\input{ncert/11/16/3/4_1/Prob_4.tex}
\item In a non-leap year, the probability of having 53 tuesdays or 53 wednesdays is\\
\solution
%\input{exemplar/11/16/3/18/main.tex}
\item There are 1000 sealed envelopes in a box, 10 of them contain a cash prize of
Rs 100 each, 100 of them contain a cash prize of Rs 50 each and 200 of them
contain a cash prize of Rs 10 each and rest do not contain any cash prize. If they
are well shuffled and an envelope is picked up out, what is the probability that it
contains no cash prize?\\
\solution
%\input{exemplar/10/13/3/34/main.tex}
\item 
A die is thrown and a card is selected at random from a deck of 52 playing cards. The probability of getting an even number on the die and a spade card.\\
\solution
%\input{exemplar/12/13/3/78/main.tex}
\item
If 4-digit numbers greater than 5,000 are randomly formed from the digits 0, 1, 3, 5, and 7, what is the probability of forming a number divisible by 5 when:
\begin{enumerate}
    \item The digits are repeated?
    \item The repetition of digits is not allowed?
\end{enumerate}
\solution
%\input{ncert/11/16/4/9/main.tex}
\item Consider the probability space $\brak{\Omega, \mathcal{G}, P}$ where $\Omega = [0,2]$ and $\mathcal{G} = \cbrak{\phi, \Omega, [0,1], (1,2]}$. Let $X$ and $Y$ be two functions on $\Omega$ defined as
\begin{align*}
    X(\omega) = 
    \begin{cases}
        1 & \text{if }\omega \in [0, 1]\\
        2 & \text{if }\omega \in (1, 2]
    \end{cases}
\end{align*}
and
\begin{align*}
    Y(\omega) = 
    \begin{cases}
        2 & \text{if }\omega \in [0, 1.5]\\
        3 & \text{if }\omega \in (1.5, 2].
    \end{cases}
\end{align*}
Then which one of the following statements is true?
\begin{enumerate}
    \item [(A)] $X$ is a random variable with respect to $\mathcal{G}$, but $Y$ is not a random variable with respect to $\mathcal{G}$.
    \item [(B)] $Y$ is a random variable with respect to $\mathcal{G}$, but $X$ is not a random variable with respect to $\mathcal{G}$.
    \item [(C)] Neither $X$ nor $Y$ is a random variable with respect to $\mathcal{G}$.
    \item [(D)] Both $X$ and $Y$ are random variables with respect to $\mathcal{G}$.
\end{enumerate} \hfill (GATE ST 2023)\\
\solution
%\input{gate/ST/2023/14/main.tex}
	\item  A die is loaded in such a way that each odd number is twice as likely to occur as
each even number. Find $P(G)$, where $G$ is the event that a number greater than
3 occurs on a single roll of the die.
\\
\solution
		%\input{exemplar/11/16/3/5/main.tex}
	\item All the jacks, queens and kings are removed from a deck of 52 playing cards. The remaining cards are well shuffled and then one card is drawn at random. Giving ace a value 1 similar value for other cards, find the probability that the card has a value 
		\begin{enumerate}
			\item 7
			\item greater than 7
			\item less than 7
		\end{enumerate}
		%\input{exemplar/10/13/3/30/main.tex}
  \item A Lot consists of 48 mobile phones of which 42 are good, 3 have only minor defects and 3 have major defects.Varnika will buy a phone if it is good but the trader will only buy a mobile if it has no major defects. One phone is selected at random from the lot. What is the probability that it is
\begin{enumerate}
	\item acceptable to Varnika?
            \item acceptable to the trader?
\end{enumerate}
\solution
	%\input{exemplar/10/13/3/40/main.tex}
 \item A student says that if you throw a die, it will show up 1 or not 1. Therefore, the probability of getting 1 and the probability of getting 'not 1' each is equal to $\frac{1}{2}$. Is this correct? Give reasons.\\
 \solution
        %\input{exemplar/10/13/2/9/main.tex}
   \item Four candidates A, B, C, D have ap-
plied for the assignment to coach a school cricket
team. If A is twice as likely to be selected as B, and
B and C are given about the same chance of being
selected, while C is twice as likely to be selected
as D, what are the probabilities that
\begin{enumerate}
\item C will be selected?
\item A will not be selected?
\end{enumerate}
	%\input{exemplar/11/16/3/9/main.tex}
 \item A bag contain 24 balls of which $x$ balls are red, $2x$ are white and $3x$ are blue. A ball is selected at random, What is the probability that it is
\begin{enumerate}[label=\alph*)]
\item not red ?
\item white ?
\end{enumerate}
%\input{exemplar/10/13/3/41/main.tex}
If the letters of the word ASSASSINATION are arranged at random. Find the Probability that
\begin{enumerate}[label=(\alph*)]
\item Four $S's$ come consecutively in the word
\item Two  $I's$ and two $N's$ come together
\item All $A's$ are not coming together
\item No two $A's$ are coming together
\end{enumerate}
%\input{exemplar/11/16/3/14/main.tex}
	\item One urn contains two black balls (labelled B1 and B2) and one white ball. A
	second urn contains one black ball and two white balls (labelled W1 and W2).
	Suppose the following experiment is performed. One of the two urns is chosen
	at random. Next a ball is randomly chosen from the urn. Then a second ball is
	chosen at random from the same urn without replacing the first ball.
	
	\begin{enumerate}
	\item What is the probability that two black balls are chosen?
	
	\item What is the probability that two balls of opposite colour are chosen?
	\end{enumerate}
	\solution
	%\input{exemplar/11/16/3/12/main1.tex}
\end{enumerate}

	\item A bag contains 4 red and 4 black balls, another bag contains 2 red and 6 black balls. One of the two bags is selected at random and a ball is drawn from the bag which is found to be red. Find the probability that the ball is drawn from the first bag.
\\
\solution
		%\begin{table}[H]
	\centering
\begin{tabular}{|c|c|c|}
\hline
Random variable &Value &Definition\\ \hline
\multirow{3}{*}{X} &0 &Slips of Rs 1\\
&1 &Slips of Rs 5\\
&2 &Slips of Rs 13\\ \hline
\multirow{2}{*}{Y} &0 &Box A\\
&1 &Box B\\\hline
\end{tabular}
\caption{}
\label{tab:Distribution}
\end{table}
See \tabref{tab:Distribution}.
\begin{align}
p_{Y}\brak{k}= \begin{cases} 
      \frac{1}{3} & {k=0} \\
      \frac{2}{3 }& {k=1} 
   \end{cases}
   \\
p_{Y|X}\brak{0|0} = \frac{19}{25}\, 
p_{Y|X}\brak{0|1} = \frac{6}{25}\,
p_{Y|X}\brak{1|0} = \frac{45}{50}\,
p_{Y|X}\brak{1|2} = \frac{5}{50}
\end{align}
The desired probability is the probability that a slip drawn at random is marked other than Rs 1,
\begin{align}
&=1-p_X\brak{0}\\
&= p_X(1) + p_X(2)
\end{align}
Using Bayes theorem,
\begin{align}
&= p_Y\brak{0} \times \pr{Y=0 | X=1} + p_Y\brak{1} \times \pr{Y=1|X=2}\\
&=\frac{1}{3} \times \frac{6}{25} + \frac{2}{3} \times \frac{5}{50}\\
&=\frac{11}{75}
\end{align}

\newpage

%\tableofcontents

\bigskip

\renewcommand{\thefigure}{\theenumi}
\renewcommand{\thetable}{\theenumi}
%\renewcommand{\theequation}{\theenumi}

%\begin{abstract}
%%\boldmath
%In this letter, an algorithm for evaluating the exact analytical bit error rate  (BER)  for the piecewise linear (PL) combiner for  multiple relays is presented. Previous results were available only for upto three relays. The algorithm is unique in the sense that  the actual mathematical expressions, that are prohibitively large, need not be explicitly obtained. The diversity gain due to multiple relays is shown through plots of the analytical BER, well supported by simulations. 
%
%\end{abstract}
% IEEEtran.cls defaults to using nonbold math in the Abstract.
% This preserves the distinction between vectors and scalars. However,
% if the journal you are submitting to favors bold math in the abstract,
% then you can use LaTeX's standard command \boldmath at the very start
% of the abstract to achieve this. Many IEEE journals frown on math
% in the abstract anyway.

% Note that keywords are not normally used for peerreview papers.
%\begin{IEEEkeywords}
%Cooperative diversity, decode and forward, piecewise linear
%\end{IEEEkeywords}



% For peer review papers, you can put extra information on the cover
% page as needed:
% \ifCLASSOPTIONpeerreview
% \begin{center} \bfseries EDICS Category: 3-BBND \end{center}
% \fi
%
% For peerreview papers, this IEEEtran command inserts a page break and
% creates the second title. It will be ignored for other modes.
%\IEEEpeerreviewmaketitle




  \item
  Cards with numbers 2 to 101 are placed in a box. A card is selected at random.Find the probability that the card has
\begin{enumerate}[label=(\roman*)]
	\item an even number 
	\item a square number
\end{enumerate}
\solution
%\begin{table}[H]
	\centering
\begin{tabular}{|c|c|c|}
\hline
Random variable &Value &Definition\\ \hline
\multirow{3}{*}{X} &0 &Slips of Rs 1\\
&1 &Slips of Rs 5\\
&2 &Slips of Rs 13\\ \hline
\multirow{2}{*}{Y} &0 &Box A\\
&1 &Box B\\\hline
\end{tabular}
\caption{}
\label{tab:Distribution}
\end{table}
See \tabref{tab:Distribution}.
\begin{align}
p_{Y}\brak{k}= \begin{cases} 
      \frac{1}{3} & {k=0} \\
      \frac{2}{3 }& {k=1} 
   \end{cases}
   \\
p_{Y|X}\brak{0|0} = \frac{19}{25}\, 
p_{Y|X}\brak{0|1} = \frac{6}{25}\,
p_{Y|X}\brak{1|0} = \frac{45}{50}\,
p_{Y|X}\brak{1|2} = \frac{5}{50}
\end{align}
The desired probability is the probability that a slip drawn at random is marked other than Rs 1,
\begin{align}
&=1-p_X\brak{0}\\
&= p_X(1) + p_X(2)
\end{align}
Using Bayes theorem,
\begin{align}
&= p_Y\brak{0} \times \pr{Y=0 | X=1} + p_Y\brak{1} \times \pr{Y=1|X=2}\\
&=\frac{1}{3} \times \frac{6}{25} + \frac{2}{3} \times \frac{5}{50}\\
&=\frac{11}{75}
\end{align}

\newpage

%\tableofcontents

\bigskip

\renewcommand{\thefigure}{\theenumi}
\renewcommand{\thetable}{\theenumi}
%\renewcommand{\theequation}{\theenumi}

%\begin{abstract}
%%\boldmath
%In this letter, an algorithm for evaluating the exact analytical bit error rate  (BER)  for the piecewise linear (PL) combiner for  multiple relays is presented. Previous results were available only for upto three relays. The algorithm is unique in the sense that  the actual mathematical expressions, that are prohibitively large, need not be explicitly obtained. The diversity gain due to multiple relays is shown through plots of the analytical BER, well supported by simulations. 
%
%\end{abstract}
% IEEEtran.cls defaults to using nonbold math in the Abstract.
% This preserves the distinction between vectors and scalars. However,
% if the journal you are submitting to favors bold math in the abstract,
% then you can use LaTeX's standard command \boldmath at the very start
% of the abstract to achieve this. Many IEEE journals frown on math
% in the abstract anyway.

% Note that keywords are not normally used for peerreview papers.
%\begin{IEEEkeywords}
%Cooperative diversity, decode and forward, piecewise linear
%\end{IEEEkeywords}



% For peer review papers, you can put extra information on the cover
% page as needed:
% \ifCLASSOPTIONpeerreview
% \begin{center} \bfseries EDICS Category: 3-BBND \end{center}
% \fi
%
% For peerreview papers, this IEEEtran command inserts a page break and
% creates the second title. It will be ignored for other modes.
%\IEEEpeerreviewmaketitle




\item
The king, queen and jack of clubs are removed from a deck of 52 playing cards and then well shuffled. Now one card is drawn at random from the remaining cards.  Determine the probability that the card is
\begin{enumerate}[label=(\roman*)]
\item a club
\item 10 of hearts
\end{enumerate}
\solution
%\begin{table}[H]
	\centering
\begin{tabular}{|c|c|c|}
\hline
Random variable &Value &Definition\\ \hline
\multirow{3}{*}{X} &0 &Slips of Rs 1\\
&1 &Slips of Rs 5\\
&2 &Slips of Rs 13\\ \hline
\multirow{2}{*}{Y} &0 &Box A\\
&1 &Box B\\\hline
\end{tabular}
\caption{}
\label{tab:Distribution}
\end{table}
See \tabref{tab:Distribution}.
\begin{align}
p_{Y}\brak{k}= \begin{cases} 
      \frac{1}{3} & {k=0} \\
      \frac{2}{3 }& {k=1} 
   \end{cases}
   \\
p_{Y|X}\brak{0|0} = \frac{19}{25}\, 
p_{Y|X}\brak{0|1} = \frac{6}{25}\,
p_{Y|X}\brak{1|0} = \frac{45}{50}\,
p_{Y|X}\brak{1|2} = \frac{5}{50}
\end{align}
The desired probability is the probability that a slip drawn at random is marked other than Rs 1,
\begin{align}
&=1-p_X\brak{0}\\
&= p_X(1) + p_X(2)
\end{align}
Using Bayes theorem,
\begin{align}
&= p_Y\brak{0} \times \pr{Y=0 | X=1} + p_Y\brak{1} \times \pr{Y=1|X=2}\\
&=\frac{1}{3} \times \frac{6}{25} + \frac{2}{3} \times \frac{5}{50}\\
&=\frac{11}{75}
\end{align}

\newpage

%\tableofcontents

\bigskip

\renewcommand{\thefigure}{\theenumi}
\renewcommand{\thetable}{\theenumi}
%\renewcommand{\theequation}{\theenumi}

%\begin{abstract}
%%\boldmath
%In this letter, an algorithm for evaluating the exact analytical bit error rate  (BER)  for the piecewise linear (PL) combiner for  multiple relays is presented. Previous results were available only for upto three relays. The algorithm is unique in the sense that  the actual mathematical expressions, that are prohibitively large, need not be explicitly obtained. The diversity gain due to multiple relays is shown through plots of the analytical BER, well supported by simulations. 
%
%\end{abstract}
% IEEEtran.cls defaults to using nonbold math in the Abstract.
% This preserves the distinction between vectors and scalars. However,
% if the journal you are submitting to favors bold math in the abstract,
% then you can use LaTeX's standard command \boldmath at the very start
% of the abstract to achieve this. Many IEEE journals frown on math
% in the abstract anyway.

% Note that keywords are not normally used for peerreview papers.
%\begin{IEEEkeywords}
%Cooperative diversity, decode and forward, piecewise linear
%\end{IEEEkeywords}



% For peer review papers, you can put extra information on the cover
% page as needed:
% \ifCLASSOPTIONpeerreview
% \begin{center} \bfseries EDICS Category: 3-BBND \end{center}
% \fi
%
% For peerreview papers, this IEEEtran command inserts a page break and
% creates the second title. It will be ignored for other modes.
%\IEEEpeerreviewmaketitle




\item A team of medical students doing their internship have to assist during surgeries
at a city hospital. The probabilities of surgeries rated as very complex, complex,
routine, simple or very simple are respectively, 0.15, 0.20, 0.31, 0.26, .08. Find
the probabilities that a particular surgery will be rated
\begin{enumerate}
	\item complex or very complex;
	\item neither very complex nor very simple;
	\item routine or complex
	\item routine or simple
\end{enumerate}
\solution
%\begin{table}[H]
	\centering
\begin{tabular}{|c|c|c|}
\hline
Random variable &Value &Definition\\ \hline
\multirow{3}{*}{X} &0 &Slips of Rs 1\\
&1 &Slips of Rs 5\\
&2 &Slips of Rs 13\\ \hline
\multirow{2}{*}{Y} &0 &Box A\\
&1 &Box B\\\hline
\end{tabular}
\caption{}
\label{tab:Distribution}
\end{table}
See \tabref{tab:Distribution}.
\begin{align}
p_{Y}\brak{k}= \begin{cases} 
      \frac{1}{3} & {k=0} \\
      \frac{2}{3 }& {k=1} 
   \end{cases}
   \\
p_{Y|X}\brak{0|0} = \frac{19}{25}\, 
p_{Y|X}\brak{0|1} = \frac{6}{25}\,
p_{Y|X}\brak{1|0} = \frac{45}{50}\,
p_{Y|X}\brak{1|2} = \frac{5}{50}
\end{align}
The desired probability is the probability that a slip drawn at random is marked other than Rs 1,
\begin{align}
&=1-p_X\brak{0}\\
&= p_X(1) + p_X(2)
\end{align}
Using Bayes theorem,
\begin{align}
&= p_Y\brak{0} \times \pr{Y=0 | X=1} + p_Y\brak{1} \times \pr{Y=1|X=2}\\
&=\frac{1}{3} \times \frac{6}{25} + \frac{2}{3} \times \frac{5}{50}\\
&=\frac{11}{75}
\end{align}

\newpage

%\tableofcontents

\bigskip

\renewcommand{\thefigure}{\theenumi}
\renewcommand{\thetable}{\theenumi}
%\renewcommand{\theequation}{\theenumi}

%\begin{abstract}
%%\boldmath
%In this letter, an algorithm for evaluating the exact analytical bit error rate  (BER)  for the piecewise linear (PL) combiner for  multiple relays is presented. Previous results were available only for upto three relays. The algorithm is unique in the sense that  the actual mathematical expressions, that are prohibitively large, need not be explicitly obtained. The diversity gain due to multiple relays is shown through plots of the analytical BER, well supported by simulations. 
%
%\end{abstract}
% IEEEtran.cls defaults to using nonbold math in the Abstract.
% This preserves the distinction between vectors and scalars. However,
% if the journal you are submitting to favors bold math in the abstract,
% then you can use LaTeX's standard command \boldmath at the very start
% of the abstract to achieve this. Many IEEE journals frown on math
% in the abstract anyway.

% Note that keywords are not normally used for peerreview papers.
%\begin{IEEEkeywords}
%Cooperative diversity, decode and forward, piecewise linear
%\end{IEEEkeywords}



% For peer review papers, you can put extra information on the cover
% page as needed:
% \ifCLASSOPTIONpeerreview
% \begin{center} \bfseries EDICS Category: 3-BBND \end{center}
% \fi
%
% For peerreview papers, this IEEEtran command inserts a page break and
% creates the second title. It will be ignored for other modes.
%\IEEEpeerreviewmaketitle




\item A card is selected from a pack of 52 cards.
\begin{enumerate}[label=(\alph*)]
    \item How many points are there in the sample space?
    \item Calculate the probability that the card is an ace of spades.
    \item Calculate the probability that the card is (i) an ace and (ii) black card.
\end{enumerate}
\solution
%Let $X$ be an bernoulli rv defined as in \tabref{tab:exemplar/11/16/3/26}.  Then, 
\begin{equation}
    p =
        \frac{4}{11} 
\end{equation}
\begin{table}[H]
	\centering
	\input{exemplar/11/16/3/26/tables/Table2.tex}
	\caption{}
        \label{tab:exemplar/11/16/3/26}
\end{table}

\item The probability that a non leap year selected at random will contain 53 sundays.
\\
\solution
%\begin{table}[H]
	\centering
\begin{tabular}{|c|c|c|}
\hline
Random variable &Value &Definition\\ \hline
\multirow{3}{*}{X} &0 &Slips of Rs 1\\
&1 &Slips of Rs 5\\
&2 &Slips of Rs 13\\ \hline
\multirow{2}{*}{Y} &0 &Box A\\
&1 &Box B\\\hline
\end{tabular}
\caption{}
\label{tab:Distribution}
\end{table}
See \tabref{tab:Distribution}.
\begin{align}
p_{Y}\brak{k}= \begin{cases} 
      \frac{1}{3} & {k=0} \\
      \frac{2}{3 }& {k=1} 
   \end{cases}
   \\
p_{Y|X}\brak{0|0} = \frac{19}{25}\, 
p_{Y|X}\brak{0|1} = \frac{6}{25}\,
p_{Y|X}\brak{1|0} = \frac{45}{50}\,
p_{Y|X}\brak{1|2} = \frac{5}{50}
\end{align}
The desired probability is the probability that a slip drawn at random is marked other than Rs 1,
\begin{align}
&=1-p_X\brak{0}\\
&= p_X(1) + p_X(2)
\end{align}
Using Bayes theorem,
\begin{align}
&= p_Y\brak{0} \times \pr{Y=0 | X=1} + p_Y\brak{1} \times \pr{Y=1|X=2}\\
&=\frac{1}{3} \times \frac{6}{25} + \frac{2}{3} \times \frac{5}{50}\\
&=\frac{11}{75}
\end{align}

\newpage

%\tableofcontents

\bigskip

\renewcommand{\thefigure}{\theenumi}
\renewcommand{\thetable}{\theenumi}
%\renewcommand{\theequation}{\theenumi}

%\begin{abstract}
%%\boldmath
%In this letter, an algorithm for evaluating the exact analytical bit error rate  (BER)  for the piecewise linear (PL) combiner for  multiple relays is presented. Previous results were available only for upto three relays. The algorithm is unique in the sense that  the actual mathematical expressions, that are prohibitively large, need not be explicitly obtained. The diversity gain due to multiple relays is shown through plots of the analytical BER, well supported by simulations. 
%
%\end{abstract}
% IEEEtran.cls defaults to using nonbold math in the Abstract.
% This preserves the distinction between vectors and scalars. However,
% if the journal you are submitting to favors bold math in the abstract,
% then you can use LaTeX's standard command \boldmath at the very start
% of the abstract to achieve this. Many IEEE journals frown on math
% in the abstract anyway.

% Note that keywords are not normally used for peerreview papers.
%\begin{IEEEkeywords}
%Cooperative diversity, decode and forward, piecewise linear
%\end{IEEEkeywords}



% For peer review papers, you can put extra information on the cover
% page as needed:
% \ifCLASSOPTIONpeerreview
% \begin{center} \bfseries EDICS Category: 3-BBND \end{center}
% \fi
%
% For peerreview papers, this IEEEtran command inserts a page break and
% creates the second title. It will be ignored for other modes.
%\IEEEpeerreviewmaketitle




\item One of the four persons John, Rita, Aslam or Gurpreet will be promoted next
month. Consequently the sample space consists of four elementary outcomes
S = {John promoted, Rita promoted, Aslam promoted, Gurpreet promoted}
You are told that the chances of John’s promotion is same as that of Gurpreet,
Rita’s chances of promotion are twice as likely as Johns. Aslam’s chances are
four times that of John.
\begin{enumerate}
	\item Determine
	\begin{enumerate}
		\item P (John promoted)
		\item P (Rita promoted)
		\item P (Aslam promoted)
		\item P (Gurpreet promoted)
	\end{enumerate}
	\item If A = {John promoted or Gurpreet promoted}, find P (A).
\end{enumerate}
\solution
%\begin{table}[H]
	\centering
\begin{tabular}{|c|c|c|}
\hline
Random variable &Value &Definition\\ \hline
\multirow{3}{*}{X} &0 &Slips of Rs 1\\
&1 &Slips of Rs 5\\
&2 &Slips of Rs 13\\ \hline
\multirow{2}{*}{Y} &0 &Box A\\
&1 &Box B\\\hline
\end{tabular}
\caption{}
\label{tab:Distribution}
\end{table}
See \tabref{tab:Distribution}.
\begin{align}
p_{Y}\brak{k}= \begin{cases} 
      \frac{1}{3} & {k=0} \\
      \frac{2}{3 }& {k=1} 
   \end{cases}
   \\
p_{Y|X}\brak{0|0} = \frac{19}{25}\, 
p_{Y|X}\brak{0|1} = \frac{6}{25}\,
p_{Y|X}\brak{1|0} = \frac{45}{50}\,
p_{Y|X}\brak{1|2} = \frac{5}{50}
\end{align}
The desired probability is the probability that a slip drawn at random is marked other than Rs 1,
\begin{align}
&=1-p_X\brak{0}\\
&= p_X(1) + p_X(2)
\end{align}
Using Bayes theorem,
\begin{align}
&= p_Y\brak{0} \times \pr{Y=0 | X=1} + p_Y\brak{1} \times \pr{Y=1|X=2}\\
&=\frac{1}{3} \times \frac{6}{25} + \frac{2}{3} \times \frac{5}{50}\\
&=\frac{11}{75}
\end{align}

\newpage

%\tableofcontents

\bigskip

\renewcommand{\thefigure}{\theenumi}
\renewcommand{\thetable}{\theenumi}
%\renewcommand{\theequation}{\theenumi}

%\begin{abstract}
%%\boldmath
%In this letter, an algorithm for evaluating the exact analytical bit error rate  (BER)  for the piecewise linear (PL) combiner for  multiple relays is presented. Previous results were available only for upto three relays. The algorithm is unique in the sense that  the actual mathematical expressions, that are prohibitively large, need not be explicitly obtained. The diversity gain due to multiple relays is shown through plots of the analytical BER, well supported by simulations. 
%
%\end{abstract}
% IEEEtran.cls defaults to using nonbold math in the Abstract.
% This preserves the distinction between vectors and scalars. However,
% if the journal you are submitting to favors bold math in the abstract,
% then you can use LaTeX's standard command \boldmath at the very start
% of the abstract to achieve this. Many IEEE journals frown on math
% in the abstract anyway.

% Note that keywords are not normally used for peerreview papers.
%\begin{IEEEkeywords}
%Cooperative diversity, decode and forward, piecewise linear
%\end{IEEEkeywords}



% For peer review papers, you can put extra information on the cover
% page as needed:
% \ifCLASSOPTIONpeerreview
% \begin{center} \bfseries EDICS Category: 3-BBND \end{center}
% \fi
%
% For peerreview papers, this IEEEtran command inserts a page break and
% creates the second title. It will be ignored for other modes.
%\IEEEpeerreviewmaketitle




\item A card is drawn from a deck of 52 cards. Find the probability of getting a king or a heart or a red card.\\
\solution
%\begin{table}[H]
	\centering
\begin{tabular}{|c|c|c|}
\hline
Random variable &Value &Definition\\ \hline
\multirow{3}{*}{X} &0 &Slips of Rs 1\\
&1 &Slips of Rs 5\\
&2 &Slips of Rs 13\\ \hline
\multirow{2}{*}{Y} &0 &Box A\\
&1 &Box B\\\hline
\end{tabular}
\caption{}
\label{tab:Distribution}
\end{table}
See \tabref{tab:Distribution}.
\begin{align}
p_{Y}\brak{k}= \begin{cases} 
      \frac{1}{3} & {k=0} \\
      \frac{2}{3 }& {k=1} 
   \end{cases}
   \\
p_{Y|X}\brak{0|0} = \frac{19}{25}\, 
p_{Y|X}\brak{0|1} = \frac{6}{25}\,
p_{Y|X}\brak{1|0} = \frac{45}{50}\,
p_{Y|X}\brak{1|2} = \frac{5}{50}
\end{align}
The desired probability is the probability that a slip drawn at random is marked other than Rs 1,
\begin{align}
&=1-p_X\brak{0}\\
&= p_X(1) + p_X(2)
\end{align}
Using Bayes theorem,
\begin{align}
&= p_Y\brak{0} \times \pr{Y=0 | X=1} + p_Y\brak{1} \times \pr{Y=1|X=2}\\
&=\frac{1}{3} \times \frac{6}{25} + \frac{2}{3} \times \frac{5}{50}\\
&=\frac{11}{75}
\end{align}

\newpage

%\tableofcontents

\bigskip

\renewcommand{\thefigure}{\theenumi}
\renewcommand{\thetable}{\theenumi}
%\renewcommand{\theequation}{\theenumi}

%\begin{abstract}
%%\boldmath
%In this letter, an algorithm for evaluating the exact analytical bit error rate  (BER)  for the piecewise linear (PL) combiner for  multiple relays is presented. Previous results were available only for upto three relays. The algorithm is unique in the sense that  the actual mathematical expressions, that are prohibitively large, need not be explicitly obtained. The diversity gain due to multiple relays is shown through plots of the analytical BER, well supported by simulations. 
%
%\end{abstract}
% IEEEtran.cls defaults to using nonbold math in the Abstract.
% This preserves the distinction between vectors and scalars. However,
% if the journal you are submitting to favors bold math in the abstract,
% then you can use LaTeX's standard command \boldmath at the very start
% of the abstract to achieve this. Many IEEE journals frown on math
% in the abstract anyway.

% Note that keywords are not normally used for peerreview papers.
%\begin{IEEEkeywords}
%Cooperative diversity, decode and forward, piecewise linear
%\end{IEEEkeywords}



% For peer review papers, you can put extra information on the cover
% page as needed:
% \ifCLASSOPTIONpeerreview
% \begin{center} \bfseries EDICS Category: 3-BBND \end{center}
% \fi
%
% For peerreview papers, this IEEEtran command inserts a page break and
% creates the second title. It will be ignored for other modes.
%\IEEEpeerreviewmaketitle




\item The probability that a student will pass his examination is 0.73, the probability of
the student getting a compartment is 0.13, and the probability that the student will
either pass or get compartment is 0.96. State True or False.\\
\solution
%\begin{table}[H]
	\centering
\begin{tabular}{|c|c|c|}
\hline
Random variable &Value &Definition\\ \hline
\multirow{3}{*}{X} &0 &Slips of Rs 1\\
&1 &Slips of Rs 5\\
&2 &Slips of Rs 13\\ \hline
\multirow{2}{*}{Y} &0 &Box A\\
&1 &Box B\\\hline
\end{tabular}
\caption{}
\label{tab:Distribution}
\end{table}
See \tabref{tab:Distribution}.
\begin{align}
p_{Y}\brak{k}= \begin{cases} 
      \frac{1}{3} & {k=0} \\
      \frac{2}{3 }& {k=1} 
   \end{cases}
   \\
p_{Y|X}\brak{0|0} = \frac{19}{25}\, 
p_{Y|X}\brak{0|1} = \frac{6}{25}\,
p_{Y|X}\brak{1|0} = \frac{45}{50}\,
p_{Y|X}\brak{1|2} = \frac{5}{50}
\end{align}
The desired probability is the probability that a slip drawn at random is marked other than Rs 1,
\begin{align}
&=1-p_X\brak{0}\\
&= p_X(1) + p_X(2)
\end{align}
Using Bayes theorem,
\begin{align}
&= p_Y\brak{0} \times \pr{Y=0 | X=1} + p_Y\brak{1} \times \pr{Y=1|X=2}\\
&=\frac{1}{3} \times \frac{6}{25} + \frac{2}{3} \times \frac{5}{50}\\
&=\frac{11}{75}
\end{align}

\newpage

%\tableofcontents

\bigskip

\renewcommand{\thefigure}{\theenumi}
\renewcommand{\thetable}{\theenumi}
%\renewcommand{\theequation}{\theenumi}

%\begin{abstract}
%%\boldmath
%In this letter, an algorithm for evaluating the exact analytical bit error rate  (BER)  for the piecewise linear (PL) combiner for  multiple relays is presented. Previous results were available only for upto three relays. The algorithm is unique in the sense that  the actual mathematical expressions, that are prohibitively large, need not be explicitly obtained. The diversity gain due to multiple relays is shown through plots of the analytical BER, well supported by simulations. 
%
%\end{abstract}
% IEEEtran.cls defaults to using nonbold math in the Abstract.
% This preserves the distinction between vectors and scalars. However,
% if the journal you are submitting to favors bold math in the abstract,
% then you can use LaTeX's standard command \boldmath at the very start
% of the abstract to achieve this. Many IEEE journals frown on math
% in the abstract anyway.

% Note that keywords are not normally used for peerreview papers.
%\begin{IEEEkeywords}
%Cooperative diversity, decode and forward, piecewise linear
%\end{IEEEkeywords}



% For peer review papers, you can put extra information on the cover
% page as needed:
% \ifCLASSOPTIONpeerreview
% \begin{center} \bfseries EDICS Category: 3-BBND \end{center}
% \fi
%
% For peerreview papers, this IEEEtran command inserts a page break and
% creates the second title. It will be ignored for other modes.
%\IEEEpeerreviewmaketitle




\item A card is selected from a pack of 52 cards\\
\begin{enumerate}[label=(\alph*)]
\item How many points are there in the sample space?
\item Calculate the probability that the cards is an ace of spades.
\item Calculate the probability that the card is (i) an ace (ii)black card.\\
\end{enumerate}
%\input{ncert/11/16/3/4_1/Prob_4.tex}
\item In a non-leap year, the probability of having 53 tuesdays or 53 wednesdays is\\
\solution
%A non-leap year has a total of 365 days, and a week has 7 days.\\
So it can be expressed as 
\begin{align}
365\text{days} &=52\times 7+1 \text{day}
\end{align}
$\implies$ 52 tuesdays or wednesdays\\
Random variable X denotes the days of a week
\begin{align}
p_X\brak{k}&=\frac{1}{7}; \quad \brak{1<k<7}
\end{align}
So the probability of extra day being tuesday or wednesday is
\begin{align}
p_X\brak{3}+p_X\brak{4}&=\frac{1}{7}+\frac{1}{7}=\frac{2}{7}
\end{align}



\item There are 1000 sealed envelopes in a box, 10 of them contain a cash prize of
Rs 100 each, 100 of them contain a cash prize of Rs 50 each and 200 of them
contain a cash prize of Rs 10 each and rest do not contain any cash prize. If they
are well shuffled and an envelope is picked up out, what is the probability that it
contains no cash prize?\\
\solution
%\begin{table}[H]
	\centering
\begin{tabular}{|c|c|c|}
\hline
Random variable &Value &Definition\\ \hline
\multirow{3}{*}{X} &0 &Slips of Rs 1\\
&1 &Slips of Rs 5\\
&2 &Slips of Rs 13\\ \hline
\multirow{2}{*}{Y} &0 &Box A\\
&1 &Box B\\\hline
\end{tabular}
\caption{}
\label{tab:Distribution}
\end{table}
See \tabref{tab:Distribution}.
\begin{align}
p_{Y}\brak{k}= \begin{cases} 
      \frac{1}{3} & {k=0} \\
      \frac{2}{3 }& {k=1} 
   \end{cases}
   \\
p_{Y|X}\brak{0|0} = \frac{19}{25}\, 
p_{Y|X}\brak{0|1} = \frac{6}{25}\,
p_{Y|X}\brak{1|0} = \frac{45}{50}\,
p_{Y|X}\brak{1|2} = \frac{5}{50}
\end{align}
The desired probability is the probability that a slip drawn at random is marked other than Rs 1,
\begin{align}
&=1-p_X\brak{0}\\
&= p_X(1) + p_X(2)
\end{align}
Using Bayes theorem,
\begin{align}
&= p_Y\brak{0} \times \pr{Y=0 | X=1} + p_Y\brak{1} \times \pr{Y=1|X=2}\\
&=\frac{1}{3} \times \frac{6}{25} + \frac{2}{3} \times \frac{5}{50}\\
&=\frac{11}{75}
\end{align}

\newpage

%\tableofcontents

\bigskip

\renewcommand{\thefigure}{\theenumi}
\renewcommand{\thetable}{\theenumi}
%\renewcommand{\theequation}{\theenumi}

%\begin{abstract}
%%\boldmath
%In this letter, an algorithm for evaluating the exact analytical bit error rate  (BER)  for the piecewise linear (PL) combiner for  multiple relays is presented. Previous results were available only for upto three relays. The algorithm is unique in the sense that  the actual mathematical expressions, that are prohibitively large, need not be explicitly obtained. The diversity gain due to multiple relays is shown through plots of the analytical BER, well supported by simulations. 
%
%\end{abstract}
% IEEEtran.cls defaults to using nonbold math in the Abstract.
% This preserves the distinction between vectors and scalars. However,
% if the journal you are submitting to favors bold math in the abstract,
% then you can use LaTeX's standard command \boldmath at the very start
% of the abstract to achieve this. Many IEEE journals frown on math
% in the abstract anyway.

% Note that keywords are not normally used for peerreview papers.
%\begin{IEEEkeywords}
%Cooperative diversity, decode and forward, piecewise linear
%\end{IEEEkeywords}



% For peer review papers, you can put extra information on the cover
% page as needed:
% \ifCLASSOPTIONpeerreview
% \begin{center} \bfseries EDICS Category: 3-BBND \end{center}
% \fi
%
% For peerreview papers, this IEEEtran command inserts a page break and
% creates the second title. It will be ignored for other modes.
%\IEEEpeerreviewmaketitle




\item 
A die is thrown and a card is selected at random from a deck of 52 playing cards. The probability of getting an even number on the die and a spade card.\\
\solution
%\begin{table}[H]
	\centering
\begin{tabular}{|c|c|c|}
\hline
Random variable &Value &Definition\\ \hline
\multirow{3}{*}{X} &0 &Slips of Rs 1\\
&1 &Slips of Rs 5\\
&2 &Slips of Rs 13\\ \hline
\multirow{2}{*}{Y} &0 &Box A\\
&1 &Box B\\\hline
\end{tabular}
\caption{}
\label{tab:Distribution}
\end{table}
See \tabref{tab:Distribution}.
\begin{align}
p_{Y}\brak{k}= \begin{cases} 
      \frac{1}{3} & {k=0} \\
      \frac{2}{3 }& {k=1} 
   \end{cases}
   \\
p_{Y|X}\brak{0|0} = \frac{19}{25}\, 
p_{Y|X}\brak{0|1} = \frac{6}{25}\,
p_{Y|X}\brak{1|0} = \frac{45}{50}\,
p_{Y|X}\brak{1|2} = \frac{5}{50}
\end{align}
The desired probability is the probability that a slip drawn at random is marked other than Rs 1,
\begin{align}
&=1-p_X\brak{0}\\
&= p_X(1) + p_X(2)
\end{align}
Using Bayes theorem,
\begin{align}
&= p_Y\brak{0} \times \pr{Y=0 | X=1} + p_Y\brak{1} \times \pr{Y=1|X=2}\\
&=\frac{1}{3} \times \frac{6}{25} + \frac{2}{3} \times \frac{5}{50}\\
&=\frac{11}{75}
\end{align}

\newpage

%\tableofcontents

\bigskip

\renewcommand{\thefigure}{\theenumi}
\renewcommand{\thetable}{\theenumi}
%\renewcommand{\theequation}{\theenumi}

%\begin{abstract}
%%\boldmath
%In this letter, an algorithm for evaluating the exact analytical bit error rate  (BER)  for the piecewise linear (PL) combiner for  multiple relays is presented. Previous results were available only for upto three relays. The algorithm is unique in the sense that  the actual mathematical expressions, that are prohibitively large, need not be explicitly obtained. The diversity gain due to multiple relays is shown through plots of the analytical BER, well supported by simulations. 
%
%\end{abstract}
% IEEEtran.cls defaults to using nonbold math in the Abstract.
% This preserves the distinction between vectors and scalars. However,
% if the journal you are submitting to favors bold math in the abstract,
% then you can use LaTeX's standard command \boldmath at the very start
% of the abstract to achieve this. Many IEEE journals frown on math
% in the abstract anyway.

% Note that keywords are not normally used for peerreview papers.
%\begin{IEEEkeywords}
%Cooperative diversity, decode and forward, piecewise linear
%\end{IEEEkeywords}



% For peer review papers, you can put extra information on the cover
% page as needed:
% \ifCLASSOPTIONpeerreview
% \begin{center} \bfseries EDICS Category: 3-BBND \end{center}
% \fi
%
% For peerreview papers, this IEEEtran command inserts a page break and
% creates the second title. It will be ignored for other modes.
%\IEEEpeerreviewmaketitle




\item
If 4-digit numbers greater than 5,000 are randomly formed from the digits 0, 1, 3, 5, and 7, what is the probability of forming a number divisible by 5 when:
\begin{enumerate}
    \item The digits are repeated?
    \item The repetition of digits is not allowed?
\end{enumerate}
\solution
%\begin{table}[H]
	\centering
\begin{tabular}{|c|c|c|}
\hline
Random variable &Value &Definition\\ \hline
\multirow{3}{*}{X} &0 &Slips of Rs 1\\
&1 &Slips of Rs 5\\
&2 &Slips of Rs 13\\ \hline
\multirow{2}{*}{Y} &0 &Box A\\
&1 &Box B\\\hline
\end{tabular}
\caption{}
\label{tab:Distribution}
\end{table}
See \tabref{tab:Distribution}.
\begin{align}
p_{Y}\brak{k}= \begin{cases} 
      \frac{1}{3} & {k=0} \\
      \frac{2}{3 }& {k=1} 
   \end{cases}
   \\
p_{Y|X}\brak{0|0} = \frac{19}{25}\, 
p_{Y|X}\brak{0|1} = \frac{6}{25}\,
p_{Y|X}\brak{1|0} = \frac{45}{50}\,
p_{Y|X}\brak{1|2} = \frac{5}{50}
\end{align}
The desired probability is the probability that a slip drawn at random is marked other than Rs 1,
\begin{align}
&=1-p_X\brak{0}\\
&= p_X(1) + p_X(2)
\end{align}
Using Bayes theorem,
\begin{align}
&= p_Y\brak{0} \times \pr{Y=0 | X=1} + p_Y\brak{1} \times \pr{Y=1|X=2}\\
&=\frac{1}{3} \times \frac{6}{25} + \frac{2}{3} \times \frac{5}{50}\\
&=\frac{11}{75}
\end{align}

\newpage

%\tableofcontents

\bigskip

\renewcommand{\thefigure}{\theenumi}
\renewcommand{\thetable}{\theenumi}
%\renewcommand{\theequation}{\theenumi}

%\begin{abstract}
%%\boldmath
%In this letter, an algorithm for evaluating the exact analytical bit error rate  (BER)  for the piecewise linear (PL) combiner for  multiple relays is presented. Previous results were available only for upto three relays. The algorithm is unique in the sense that  the actual mathematical expressions, that are prohibitively large, need not be explicitly obtained. The diversity gain due to multiple relays is shown through plots of the analytical BER, well supported by simulations. 
%
%\end{abstract}
% IEEEtran.cls defaults to using nonbold math in the Abstract.
% This preserves the distinction between vectors and scalars. However,
% if the journal you are submitting to favors bold math in the abstract,
% then you can use LaTeX's standard command \boldmath at the very start
% of the abstract to achieve this. Many IEEE journals frown on math
% in the abstract anyway.

% Note that keywords are not normally used for peerreview papers.
%\begin{IEEEkeywords}
%Cooperative diversity, decode and forward, piecewise linear
%\end{IEEEkeywords}



% For peer review papers, you can put extra information on the cover
% page as needed:
% \ifCLASSOPTIONpeerreview
% \begin{center} \bfseries EDICS Category: 3-BBND \end{center}
% \fi
%
% For peerreview papers, this IEEEtran command inserts a page break and
% creates the second title. It will be ignored for other modes.
%\IEEEpeerreviewmaketitle




\item Consider the probability space $\brak{\Omega, \mathcal{G}, P}$ where $\Omega = [0,2]$ and $\mathcal{G} = \cbrak{\phi, \Omega, [0,1], (1,2]}$. Let $X$ and $Y$ be two functions on $\Omega$ defined as
\begin{align*}
    X(\omega) = 
    \begin{cases}
        1 & \text{if }\omega \in [0, 1]\\
        2 & \text{if }\omega \in (1, 2]
    \end{cases}
\end{align*}
and
\begin{align*}
    Y(\omega) = 
    \begin{cases}
        2 & \text{if }\omega \in [0, 1.5]\\
        3 & \text{if }\omega \in (1.5, 2].
    \end{cases}
\end{align*}
Then which one of the following statements is true?
\begin{enumerate}
    \item [(A)] $X$ is a random variable with respect to $\mathcal{G}$, but $Y$ is not a random variable with respect to $\mathcal{G}$.
    \item [(B)] $Y$ is a random variable with respect to $\mathcal{G}$, but $X$ is not a random variable with respect to $\mathcal{G}$.
    \item [(C)] Neither $X$ nor $Y$ is a random variable with respect to $\mathcal{G}$.
    \item [(D)] Both $X$ and $Y$ are random variables with respect to $\mathcal{G}$.
\end{enumerate} \hfill (GATE ST 2023)\\
\solution
%\begin{table}[H]
	\centering
\begin{tabular}{|c|c|c|}
\hline
Random variable &Value &Definition\\ \hline
\multirow{3}{*}{X} &0 &Slips of Rs 1\\
&1 &Slips of Rs 5\\
&2 &Slips of Rs 13\\ \hline
\multirow{2}{*}{Y} &0 &Box A\\
&1 &Box B\\\hline
\end{tabular}
\caption{}
\label{tab:Distribution}
\end{table}
See \tabref{tab:Distribution}.
\begin{align}
p_{Y}\brak{k}= \begin{cases} 
      \frac{1}{3} & {k=0} \\
      \frac{2}{3 }& {k=1} 
   \end{cases}
   \\
p_{Y|X}\brak{0|0} = \frac{19}{25}\, 
p_{Y|X}\brak{0|1} = \frac{6}{25}\,
p_{Y|X}\brak{1|0} = \frac{45}{50}\,
p_{Y|X}\brak{1|2} = \frac{5}{50}
\end{align}
The desired probability is the probability that a slip drawn at random is marked other than Rs 1,
\begin{align}
&=1-p_X\brak{0}\\
&= p_X(1) + p_X(2)
\end{align}
Using Bayes theorem,
\begin{align}
&= p_Y\brak{0} \times \pr{Y=0 | X=1} + p_Y\brak{1} \times \pr{Y=1|X=2}\\
&=\frac{1}{3} \times \frac{6}{25} + \frac{2}{3} \times \frac{5}{50}\\
&=\frac{11}{75}
\end{align}

\newpage

%\tableofcontents

\bigskip

\renewcommand{\thefigure}{\theenumi}
\renewcommand{\thetable}{\theenumi}
%\renewcommand{\theequation}{\theenumi}

%\begin{abstract}
%%\boldmath
%In this letter, an algorithm for evaluating the exact analytical bit error rate  (BER)  for the piecewise linear (PL) combiner for  multiple relays is presented. Previous results were available only for upto three relays. The algorithm is unique in the sense that  the actual mathematical expressions, that are prohibitively large, need not be explicitly obtained. The diversity gain due to multiple relays is shown through plots of the analytical BER, well supported by simulations. 
%
%\end{abstract}
% IEEEtran.cls defaults to using nonbold math in the Abstract.
% This preserves the distinction between vectors and scalars. However,
% if the journal you are submitting to favors bold math in the abstract,
% then you can use LaTeX's standard command \boldmath at the very start
% of the abstract to achieve this. Many IEEE journals frown on math
% in the abstract anyway.

% Note that keywords are not normally used for peerreview papers.
%\begin{IEEEkeywords}
%Cooperative diversity, decode and forward, piecewise linear
%\end{IEEEkeywords}



% For peer review papers, you can put extra information on the cover
% page as needed:
% \ifCLASSOPTIONpeerreview
% \begin{center} \bfseries EDICS Category: 3-BBND \end{center}
% \fi
%
% For peerreview papers, this IEEEtran command inserts a page break and
% creates the second title. It will be ignored for other modes.
%\IEEEpeerreviewmaketitle




	\item  A die is loaded in such a way that each odd number is twice as likely to occur as
each even number. Find $P(G)$, where $G$ is the event that a number greater than
3 occurs on a single roll of the die.
\\
\solution
		%\begin{table}[H]
	\centering
\begin{tabular}{|c|c|c|}
\hline
Random variable &Value &Definition\\ \hline
\multirow{3}{*}{X} &0 &Slips of Rs 1\\
&1 &Slips of Rs 5\\
&2 &Slips of Rs 13\\ \hline
\multirow{2}{*}{Y} &0 &Box A\\
&1 &Box B\\\hline
\end{tabular}
\caption{}
\label{tab:Distribution}
\end{table}
See \tabref{tab:Distribution}.
\begin{align}
p_{Y}\brak{k}= \begin{cases} 
      \frac{1}{3} & {k=0} \\
      \frac{2}{3 }& {k=1} 
   \end{cases}
   \\
p_{Y|X}\brak{0|0} = \frac{19}{25}\, 
p_{Y|X}\brak{0|1} = \frac{6}{25}\,
p_{Y|X}\brak{1|0} = \frac{45}{50}\,
p_{Y|X}\brak{1|2} = \frac{5}{50}
\end{align}
The desired probability is the probability that a slip drawn at random is marked other than Rs 1,
\begin{align}
&=1-p_X\brak{0}\\
&= p_X(1) + p_X(2)
\end{align}
Using Bayes theorem,
\begin{align}
&= p_Y\brak{0} \times \pr{Y=0 | X=1} + p_Y\brak{1} \times \pr{Y=1|X=2}\\
&=\frac{1}{3} \times \frac{6}{25} + \frac{2}{3} \times \frac{5}{50}\\
&=\frac{11}{75}
\end{align}

\newpage

%\tableofcontents

\bigskip

\renewcommand{\thefigure}{\theenumi}
\renewcommand{\thetable}{\theenumi}
%\renewcommand{\theequation}{\theenumi}

%\begin{abstract}
%%\boldmath
%In this letter, an algorithm for evaluating the exact analytical bit error rate  (BER)  for the piecewise linear (PL) combiner for  multiple relays is presented. Previous results were available only for upto three relays. The algorithm is unique in the sense that  the actual mathematical expressions, that are prohibitively large, need not be explicitly obtained. The diversity gain due to multiple relays is shown through plots of the analytical BER, well supported by simulations. 
%
%\end{abstract}
% IEEEtran.cls defaults to using nonbold math in the Abstract.
% This preserves the distinction between vectors and scalars. However,
% if the journal you are submitting to favors bold math in the abstract,
% then you can use LaTeX's standard command \boldmath at the very start
% of the abstract to achieve this. Many IEEE journals frown on math
% in the abstract anyway.

% Note that keywords are not normally used for peerreview papers.
%\begin{IEEEkeywords}
%Cooperative diversity, decode and forward, piecewise linear
%\end{IEEEkeywords}



% For peer review papers, you can put extra information on the cover
% page as needed:
% \ifCLASSOPTIONpeerreview
% \begin{center} \bfseries EDICS Category: 3-BBND \end{center}
% \fi
%
% For peerreview papers, this IEEEtran command inserts a page break and
% creates the second title. It will be ignored for other modes.
%\IEEEpeerreviewmaketitle




	\item All the jacks, queens and kings are removed from a deck of 52 playing cards. The remaining cards are well shuffled and then one card is drawn at random. Giving ace a value 1 similar value for other cards, find the probability that the card has a value 
		\begin{enumerate}
			\item 7
			\item greater than 7
			\item less than 7
		\end{enumerate}
		%Number of cards left after removing all jacks, queens and kings 
\begin{align}
N	= 52 - 4\times 3
	= 40
\end{align}
%\begin{table}[H]
%\def\arraystretch{1.2}
%\begin{tabular}{|c|c|c|}
%\hline
%	\textbf{Parameter} &\textbf{Value} &\textbf{Description}\\ \hline
%	$X$ &1-10 &Represents the value of the card picked \\ \hline
%\end{tabular}
%\end{table}
Let $1 \le X \le 10$ be the value of the card picked.  Then,
\begin{align}
	p_X(k) &= \Pr(X=k)\ \forall\ 1 \leq k \leq 10\\
	&= \frac{4\times 1}{40}\\
	&= \frac{1}{10}\\
	\therefore p_X(k) &= 
	\begin{cases}
		\frac{1}{10} & 1 \leq k \leq 10\\
		0 & \text{otherwise}
	\end{cases}
\end{align}
and
\begin{align}
	F_{X}(k) &= \sum_{m=0}^{k}p_{X}(m) \quad 1 \leq k \leq 10\\
	&= \frac{k}{10}\\
	\therefore F_{X}(k) &= 
	\begin{cases}
		0 & k \leq 0\\
		\frac{k}{10} & 1\leq k \leq 10\\
		1 & k > 10 
	\end{cases}
\end{align}
\begin{enumerate}
	\item Probability that card has value equal to 7 is
		\begin{align}
			 p_{X}(7)
			= \frac{1}{10}
		\end{align}
	\item Probability that card has value greater than 7 is
		\begin{align}
			1 - F_X(7)
			&= 1 - \frac{7}{10}
			\\
			&= \frac{3}{10}
		\end{align}
	\item Probability that card has value less than 7 is
		\begin{align}
			 F_{X}(6)
			=\frac{6}{10}
		\end{align}
\end{enumerate}

  \item A Lot consists of 48 mobile phones of which 42 are good, 3 have only minor defects and 3 have major defects.Varnika will buy a phone if it is good but the trader will only buy a mobile if it has no major defects. One phone is selected at random from the lot. What is the probability that it is
\begin{enumerate}
	\item acceptable to Varnika?
            \item acceptable to the trader?
\end{enumerate}
\solution
	%\begin{table}[H]
	\centering
\begin{tabular}{|c|c|c|}
\hline
Random variable &Value &Definition\\ \hline
\multirow{3}{*}{X} &0 &Slips of Rs 1\\
&1 &Slips of Rs 5\\
&2 &Slips of Rs 13\\ \hline
\multirow{2}{*}{Y} &0 &Box A\\
&1 &Box B\\\hline
\end{tabular}
\caption{}
\label{tab:Distribution}
\end{table}
See \tabref{tab:Distribution}.
\begin{align}
p_{Y}\brak{k}= \begin{cases} 
      \frac{1}{3} & {k=0} \\
      \frac{2}{3 }& {k=1} 
   \end{cases}
   \\
p_{Y|X}\brak{0|0} = \frac{19}{25}\, 
p_{Y|X}\brak{0|1} = \frac{6}{25}\,
p_{Y|X}\brak{1|0} = \frac{45}{50}\,
p_{Y|X}\brak{1|2} = \frac{5}{50}
\end{align}
The desired probability is the probability that a slip drawn at random is marked other than Rs 1,
\begin{align}
&=1-p_X\brak{0}\\
&= p_X(1) + p_X(2)
\end{align}
Using Bayes theorem,
\begin{align}
&= p_Y\brak{0} \times \pr{Y=0 | X=1} + p_Y\brak{1} \times \pr{Y=1|X=2}\\
&=\frac{1}{3} \times \frac{6}{25} + \frac{2}{3} \times \frac{5}{50}\\
&=\frac{11}{75}
\end{align}

\newpage

%\tableofcontents

\bigskip

\renewcommand{\thefigure}{\theenumi}
\renewcommand{\thetable}{\theenumi}
%\renewcommand{\theequation}{\theenumi}

%\begin{abstract}
%%\boldmath
%In this letter, an algorithm for evaluating the exact analytical bit error rate  (BER)  for the piecewise linear (PL) combiner for  multiple relays is presented. Previous results were available only for upto three relays. The algorithm is unique in the sense that  the actual mathematical expressions, that are prohibitively large, need not be explicitly obtained. The diversity gain due to multiple relays is shown through plots of the analytical BER, well supported by simulations. 
%
%\end{abstract}
% IEEEtran.cls defaults to using nonbold math in the Abstract.
% This preserves the distinction between vectors and scalars. However,
% if the journal you are submitting to favors bold math in the abstract,
% then you can use LaTeX's standard command \boldmath at the very start
% of the abstract to achieve this. Many IEEE journals frown on math
% in the abstract anyway.

% Note that keywords are not normally used for peerreview papers.
%\begin{IEEEkeywords}
%Cooperative diversity, decode and forward, piecewise linear
%\end{IEEEkeywords}



% For peer review papers, you can put extra information on the cover
% page as needed:
% \ifCLASSOPTIONpeerreview
% \begin{center} \bfseries EDICS Category: 3-BBND \end{center}
% \fi
%
% For peerreview papers, this IEEEtran command inserts a page break and
% creates the second title. It will be ignored for other modes.
%\IEEEpeerreviewmaketitle




 \item A student says that if you throw a die, it will show up 1 or not 1. Therefore, the probability of getting 1 and the probability of getting 'not 1' each is equal to $\frac{1}{2}$. Is this correct? Give reasons.\\
 \solution
        %\begin{table}[H]
	\centering
\begin{tabular}{|c|c|c|}
\hline
Random variable &Value &Definition\\ \hline
\multirow{3}{*}{X} &0 &Slips of Rs 1\\
&1 &Slips of Rs 5\\
&2 &Slips of Rs 13\\ \hline
\multirow{2}{*}{Y} &0 &Box A\\
&1 &Box B\\\hline
\end{tabular}
\caption{}
\label{tab:Distribution}
\end{table}
See \tabref{tab:Distribution}.
\begin{align}
p_{Y}\brak{k}= \begin{cases} 
      \frac{1}{3} & {k=0} \\
      \frac{2}{3 }& {k=1} 
   \end{cases}
   \\
p_{Y|X}\brak{0|0} = \frac{19}{25}\, 
p_{Y|X}\brak{0|1} = \frac{6}{25}\,
p_{Y|X}\brak{1|0} = \frac{45}{50}\,
p_{Y|X}\brak{1|2} = \frac{5}{50}
\end{align}
The desired probability is the probability that a slip drawn at random is marked other than Rs 1,
\begin{align}
&=1-p_X\brak{0}\\
&= p_X(1) + p_X(2)
\end{align}
Using Bayes theorem,
\begin{align}
&= p_Y\brak{0} \times \pr{Y=0 | X=1} + p_Y\brak{1} \times \pr{Y=1|X=2}\\
&=\frac{1}{3} \times \frac{6}{25} + \frac{2}{3} \times \frac{5}{50}\\
&=\frac{11}{75}
\end{align}

\newpage

%\tableofcontents

\bigskip

\renewcommand{\thefigure}{\theenumi}
\renewcommand{\thetable}{\theenumi}
%\renewcommand{\theequation}{\theenumi}

%\begin{abstract}
%%\boldmath
%In this letter, an algorithm for evaluating the exact analytical bit error rate  (BER)  for the piecewise linear (PL) combiner for  multiple relays is presented. Previous results were available only for upto three relays. The algorithm is unique in the sense that  the actual mathematical expressions, that are prohibitively large, need not be explicitly obtained. The diversity gain due to multiple relays is shown through plots of the analytical BER, well supported by simulations. 
%
%\end{abstract}
% IEEEtran.cls defaults to using nonbold math in the Abstract.
% This preserves the distinction between vectors and scalars. However,
% if the journal you are submitting to favors bold math in the abstract,
% then you can use LaTeX's standard command \boldmath at the very start
% of the abstract to achieve this. Many IEEE journals frown on math
% in the abstract anyway.

% Note that keywords are not normally used for peerreview papers.
%\begin{IEEEkeywords}
%Cooperative diversity, decode and forward, piecewise linear
%\end{IEEEkeywords}



% For peer review papers, you can put extra information on the cover
% page as needed:
% \ifCLASSOPTIONpeerreview
% \begin{center} \bfseries EDICS Category: 3-BBND \end{center}
% \fi
%
% For peerreview papers, this IEEEtran command inserts a page break and
% creates the second title. It will be ignored for other modes.
%\IEEEpeerreviewmaketitle




   \item Four candidates A, B, C, D have ap-
plied for the assignment to coach a school cricket
team. If A is twice as likely to be selected as B, and
B and C are given about the same chance of being
selected, while C is twice as likely to be selected
as D, what are the probabilities that
\begin{enumerate}
\item C will be selected?
\item A will not be selected?
\end{enumerate}
	%\begin{table}[H]
	\centering
\begin{tabular}{|c|c|c|}
\hline
Random variable &Value &Definition\\ \hline
\multirow{3}{*}{X} &0 &Slips of Rs 1\\
&1 &Slips of Rs 5\\
&2 &Slips of Rs 13\\ \hline
\multirow{2}{*}{Y} &0 &Box A\\
&1 &Box B\\\hline
\end{tabular}
\caption{}
\label{tab:Distribution}
\end{table}
See \tabref{tab:Distribution}.
\begin{align}
p_{Y}\brak{k}= \begin{cases} 
      \frac{1}{3} & {k=0} \\
      \frac{2}{3 }& {k=1} 
   \end{cases}
   \\
p_{Y|X}\brak{0|0} = \frac{19}{25}\, 
p_{Y|X}\brak{0|1} = \frac{6}{25}\,
p_{Y|X}\brak{1|0} = \frac{45}{50}\,
p_{Y|X}\brak{1|2} = \frac{5}{50}
\end{align}
The desired probability is the probability that a slip drawn at random is marked other than Rs 1,
\begin{align}
&=1-p_X\brak{0}\\
&= p_X(1) + p_X(2)
\end{align}
Using Bayes theorem,
\begin{align}
&= p_Y\brak{0} \times \pr{Y=0 | X=1} + p_Y\brak{1} \times \pr{Y=1|X=2}\\
&=\frac{1}{3} \times \frac{6}{25} + \frac{2}{3} \times \frac{5}{50}\\
&=\frac{11}{75}
\end{align}

\newpage

%\tableofcontents

\bigskip

\renewcommand{\thefigure}{\theenumi}
\renewcommand{\thetable}{\theenumi}
%\renewcommand{\theequation}{\theenumi}

%\begin{abstract}
%%\boldmath
%In this letter, an algorithm for evaluating the exact analytical bit error rate  (BER)  for the piecewise linear (PL) combiner for  multiple relays is presented. Previous results were available only for upto three relays. The algorithm is unique in the sense that  the actual mathematical expressions, that are prohibitively large, need not be explicitly obtained. The diversity gain due to multiple relays is shown through plots of the analytical BER, well supported by simulations. 
%
%\end{abstract}
% IEEEtran.cls defaults to using nonbold math in the Abstract.
% This preserves the distinction between vectors and scalars. However,
% if the journal you are submitting to favors bold math in the abstract,
% then you can use LaTeX's standard command \boldmath at the very start
% of the abstract to achieve this. Many IEEE journals frown on math
% in the abstract anyway.

% Note that keywords are not normally used for peerreview papers.
%\begin{IEEEkeywords}
%Cooperative diversity, decode and forward, piecewise linear
%\end{IEEEkeywords}



% For peer review papers, you can put extra information on the cover
% page as needed:
% \ifCLASSOPTIONpeerreview
% \begin{center} \bfseries EDICS Category: 3-BBND \end{center}
% \fi
%
% For peerreview papers, this IEEEtran command inserts a page break and
% creates the second title. It will be ignored for other modes.
%\IEEEpeerreviewmaketitle




 \item A bag contain 24 balls of which $x$ balls are red, $2x$ are white and $3x$ are blue. A ball is selected at random, What is the probability that it is
\begin{enumerate}[label=\alph*)]
\item not red ?
\item white ?
\end{enumerate}
%\begin{table}[H]
	\centering
\begin{tabular}{|c|c|c|}
\hline
Random variable &Value &Definition\\ \hline
\multirow{3}{*}{X} &0 &Slips of Rs 1\\
&1 &Slips of Rs 5\\
&2 &Slips of Rs 13\\ \hline
\multirow{2}{*}{Y} &0 &Box A\\
&1 &Box B\\\hline
\end{tabular}
\caption{}
\label{tab:Distribution}
\end{table}
See \tabref{tab:Distribution}.
\begin{align}
p_{Y}\brak{k}= \begin{cases} 
      \frac{1}{3} & {k=0} \\
      \frac{2}{3 }& {k=1} 
   \end{cases}
   \\
p_{Y|X}\brak{0|0} = \frac{19}{25}\, 
p_{Y|X}\brak{0|1} = \frac{6}{25}\,
p_{Y|X}\brak{1|0} = \frac{45}{50}\,
p_{Y|X}\brak{1|2} = \frac{5}{50}
\end{align}
The desired probability is the probability that a slip drawn at random is marked other than Rs 1,
\begin{align}
&=1-p_X\brak{0}\\
&= p_X(1) + p_X(2)
\end{align}
Using Bayes theorem,
\begin{align}
&= p_Y\brak{0} \times \pr{Y=0 | X=1} + p_Y\brak{1} \times \pr{Y=1|X=2}\\
&=\frac{1}{3} \times \frac{6}{25} + \frac{2}{3} \times \frac{5}{50}\\
&=\frac{11}{75}
\end{align}

\newpage

%\tableofcontents

\bigskip

\renewcommand{\thefigure}{\theenumi}
\renewcommand{\thetable}{\theenumi}
%\renewcommand{\theequation}{\theenumi}

%\begin{abstract}
%%\boldmath
%In this letter, an algorithm for evaluating the exact analytical bit error rate  (BER)  for the piecewise linear (PL) combiner for  multiple relays is presented. Previous results were available only for upto three relays. The algorithm is unique in the sense that  the actual mathematical expressions, that are prohibitively large, need not be explicitly obtained. The diversity gain due to multiple relays is shown through plots of the analytical BER, well supported by simulations. 
%
%\end{abstract}
% IEEEtran.cls defaults to using nonbold math in the Abstract.
% This preserves the distinction between vectors and scalars. However,
% if the journal you are submitting to favors bold math in the abstract,
% then you can use LaTeX's standard command \boldmath at the very start
% of the abstract to achieve this. Many IEEE journals frown on math
% in the abstract anyway.

% Note that keywords are not normally used for peerreview papers.
%\begin{IEEEkeywords}
%Cooperative diversity, decode and forward, piecewise linear
%\end{IEEEkeywords}



% For peer review papers, you can put extra information on the cover
% page as needed:
% \ifCLASSOPTIONpeerreview
% \begin{center} \bfseries EDICS Category: 3-BBND \end{center}
% \fi
%
% For peerreview papers, this IEEEtran command inserts a page break and
% creates the second title. It will be ignored for other modes.
%\IEEEpeerreviewmaketitle




If the letters of the word ASSASSINATION are arranged at random. Find the Probability that
\begin{enumerate}[label=(\alph*)]
\item Four $S's$ come consecutively in the word
\item Two  $I's$ and two $N's$ come together
\item All $A's$ are not coming together
\item No two $A's$ are coming together
\end{enumerate}
%\begin{table}[H]
	\centering
\begin{tabular}{|c|c|c|}
\hline
Random variable &Value &Definition\\ \hline
\multirow{3}{*}{X} &0 &Slips of Rs 1\\
&1 &Slips of Rs 5\\
&2 &Slips of Rs 13\\ \hline
\multirow{2}{*}{Y} &0 &Box A\\
&1 &Box B\\\hline
\end{tabular}
\caption{}
\label{tab:Distribution}
\end{table}
See \tabref{tab:Distribution}.
\begin{align}
p_{Y}\brak{k}= \begin{cases} 
      \frac{1}{3} & {k=0} \\
      \frac{2}{3 }& {k=1} 
   \end{cases}
   \\
p_{Y|X}\brak{0|0} = \frac{19}{25}\, 
p_{Y|X}\brak{0|1} = \frac{6}{25}\,
p_{Y|X}\brak{1|0} = \frac{45}{50}\,
p_{Y|X}\brak{1|2} = \frac{5}{50}
\end{align}
The desired probability is the probability that a slip drawn at random is marked other than Rs 1,
\begin{align}
&=1-p_X\brak{0}\\
&= p_X(1) + p_X(2)
\end{align}
Using Bayes theorem,
\begin{align}
&= p_Y\brak{0} \times \pr{Y=0 | X=1} + p_Y\brak{1} \times \pr{Y=1|X=2}\\
&=\frac{1}{3} \times \frac{6}{25} + \frac{2}{3} \times \frac{5}{50}\\
&=\frac{11}{75}
\end{align}

\newpage

%\tableofcontents

\bigskip

\renewcommand{\thefigure}{\theenumi}
\renewcommand{\thetable}{\theenumi}
%\renewcommand{\theequation}{\theenumi}

%\begin{abstract}
%%\boldmath
%In this letter, an algorithm for evaluating the exact analytical bit error rate  (BER)  for the piecewise linear (PL) combiner for  multiple relays is presented. Previous results were available only for upto three relays. The algorithm is unique in the sense that  the actual mathematical expressions, that are prohibitively large, need not be explicitly obtained. The diversity gain due to multiple relays is shown through plots of the analytical BER, well supported by simulations. 
%
%\end{abstract}
% IEEEtran.cls defaults to using nonbold math in the Abstract.
% This preserves the distinction between vectors and scalars. However,
% if the journal you are submitting to favors bold math in the abstract,
% then you can use LaTeX's standard command \boldmath at the very start
% of the abstract to achieve this. Many IEEE journals frown on math
% in the abstract anyway.

% Note that keywords are not normally used for peerreview papers.
%\begin{IEEEkeywords}
%Cooperative diversity, decode and forward, piecewise linear
%\end{IEEEkeywords}



% For peer review papers, you can put extra information on the cover
% page as needed:
% \ifCLASSOPTIONpeerreview
% \begin{center} \bfseries EDICS Category: 3-BBND \end{center}
% \fi
%
% For peerreview papers, this IEEEtran command inserts a page break and
% creates the second title. It will be ignored for other modes.
%\IEEEpeerreviewmaketitle




	\item One urn contains two black balls (labelled B1 and B2) and one white ball. A
	second urn contains one black ball and two white balls (labelled W1 and W2).
	Suppose the following experiment is performed. One of the two urns is chosen
	at random. Next a ball is randomly chosen from the urn. Then a second ball is
	chosen at random from the same urn without replacing the first ball.
	
	\begin{enumerate}
	\item What is the probability that two black balls are chosen?
	
	\item What is the probability that two balls of opposite colour are chosen?
	\end{enumerate}
	\solution
	%\begin{align}
    \label{eq:12.13.6.18.1}
	\because	\pr{A|B} &> \pr{A},\
\frac{\pr{AB}}{\pr{B}} > \pr{A}
\\
    \label{eq:12.13.6.18.2}
	\implies \pr{AB} &> \pr{A}\pr{B}
	\\
	\text{or, } \frac{\pr{AB}}{\pr{A}} &=\pr{B|A} > \pr{A}
\end{align}

\end{enumerate}

	\item A card is selected from a pack of 52 cards.
 \begin{enumerate}[label=(\alph*)] 
                 \item How many points are there in the sample space?
                 \item Calculate the probability that the card is an ace of spades.
                 \item Calculate the probability that the card is (i) an ace and (ii) black card.
 \end{enumerate}
\solution
		%\begin{table}[H]
	\centering
\begin{tabular}{|c|c|c|}
\hline
Random variable &Value &Definition\\ \hline
\multirow{3}{*}{X} &0 &Slips of Rs 1\\
&1 &Slips of Rs 5\\
&2 &Slips of Rs 13\\ \hline
\multirow{2}{*}{Y} &0 &Box A\\
&1 &Box B\\\hline
\end{tabular}
\caption{}
\label{tab:Distribution}
\end{table}
See \tabref{tab:Distribution}.
\begin{align}
p_{Y}\brak{k}= \begin{cases} 
      \frac{1}{3} & {k=0} \\
      \frac{2}{3 }& {k=1} 
   \end{cases}
   \\
p_{Y|X}\brak{0|0} = \frac{19}{25}\, 
p_{Y|X}\brak{0|1} = \frac{6}{25}\,
p_{Y|X}\brak{1|0} = \frac{45}{50}\,
p_{Y|X}\brak{1|2} = \frac{5}{50}
\end{align}
The desired probability is the probability that a slip drawn at random is marked other than Rs 1,
\begin{align}
&=1-p_X\brak{0}\\
&= p_X(1) + p_X(2)
\end{align}
Using Bayes theorem,
\begin{align}
&= p_Y\brak{0} \times \pr{Y=0 | X=1} + p_Y\brak{1} \times \pr{Y=1|X=2}\\
&=\frac{1}{3} \times \frac{6}{25} + \frac{2}{3} \times \frac{5}{50}\\
&=\frac{11}{75}
\end{align}

\newpage

%\tableofcontents

\bigskip

\renewcommand{\thefigure}{\theenumi}
\renewcommand{\thetable}{\theenumi}
%\renewcommand{\theequation}{\theenumi}

%\begin{abstract}
%%\boldmath
%In this letter, an algorithm for evaluating the exact analytical bit error rate  (BER)  for the piecewise linear (PL) combiner for  multiple relays is presented. Previous results were available only for upto three relays. The algorithm is unique in the sense that  the actual mathematical expressions, that are prohibitively large, need not be explicitly obtained. The diversity gain due to multiple relays is shown through plots of the analytical BER, well supported by simulations. 
%
%\end{abstract}
% IEEEtran.cls defaults to using nonbold math in the Abstract.
% This preserves the distinction between vectors and scalars. However,
% if the journal you are submitting to favors bold math in the abstract,
% then you can use LaTeX's standard command \boldmath at the very start
% of the abstract to achieve this. Many IEEE journals frown on math
% in the abstract anyway.

% Note that keywords are not normally used for peerreview papers.
%\begin{IEEEkeywords}
%Cooperative diversity, decode and forward, piecewise linear
%\end{IEEEkeywords}



% For peer review papers, you can put extra information on the cover
% page as needed:
% \ifCLASSOPTIONpeerreview
% \begin{center} \bfseries EDICS Category: 3-BBND \end{center}
% \fi
%
% For peerreview papers, this IEEEtran command inserts a page break and
% creates the second title. It will be ignored for other modes.
%\IEEEpeerreviewmaketitle




\item Four cards are drawn from a well-shuffled deck of 52 cards. What is the probability of obtaining 3 diamonds and one spade.
\\
\solution
		%\begin{enumerate}[label=\thesection.\arabic*,ref=\thesection.\theenumi]
	\item One card is drawn from a well-shuffled deck of 52 cards. Find the probability of getting
\begin{enumerate}
\item A king of red colour 
\item A face card 
\item A red face card
\item The jack of hearts
\item A spade
\item The queen of diamonds

\end{enumerate}
\solution
		%\begin{table}[H]
	\centering
\begin{tabular}{|c|c|c|}
\hline
Random variable &Value &Definition\\ \hline
\multirow{3}{*}{X} &0 &Slips of Rs 1\\
&1 &Slips of Rs 5\\
&2 &Slips of Rs 13\\ \hline
\multirow{2}{*}{Y} &0 &Box A\\
&1 &Box B\\\hline
\end{tabular}
\caption{}
\label{tab:Distribution}
\end{table}
See \tabref{tab:Distribution}.
\begin{align}
p_{Y}\brak{k}= \begin{cases} 
      \frac{1}{3} & {k=0} \\
      \frac{2}{3 }& {k=1} 
   \end{cases}
   \\
p_{Y|X}\brak{0|0} = \frac{19}{25}\, 
p_{Y|X}\brak{0|1} = \frac{6}{25}\,
p_{Y|X}\brak{1|0} = \frac{45}{50}\,
p_{Y|X}\brak{1|2} = \frac{5}{50}
\end{align}
The desired probability is the probability that a slip drawn at random is marked other than Rs 1,
\begin{align}
&=1-p_X\brak{0}\\
&= p_X(1) + p_X(2)
\end{align}
Using Bayes theorem,
\begin{align}
&= p_Y\brak{0} \times \pr{Y=0 | X=1} + p_Y\brak{1} \times \pr{Y=1|X=2}\\
&=\frac{1}{3} \times \frac{6}{25} + \frac{2}{3} \times \frac{5}{50}\\
&=\frac{11}{75}
\end{align}

\newpage

%\tableofcontents

\bigskip

\renewcommand{\thefigure}{\theenumi}
\renewcommand{\thetable}{\theenumi}
%\renewcommand{\theequation}{\theenumi}

%\begin{abstract}
%%\boldmath
%In this letter, an algorithm for evaluating the exact analytical bit error rate  (BER)  for the piecewise linear (PL) combiner for  multiple relays is presented. Previous results were available only for upto three relays. The algorithm is unique in the sense that  the actual mathematical expressions, that are prohibitively large, need not be explicitly obtained. The diversity gain due to multiple relays is shown through plots of the analytical BER, well supported by simulations. 
%
%\end{abstract}
% IEEEtran.cls defaults to using nonbold math in the Abstract.
% This preserves the distinction between vectors and scalars. However,
% if the journal you are submitting to favors bold math in the abstract,
% then you can use LaTeX's standard command \boldmath at the very start
% of the abstract to achieve this. Many IEEE journals frown on math
% in the abstract anyway.

% Note that keywords are not normally used for peerreview papers.
%\begin{IEEEkeywords}
%Cooperative diversity, decode and forward, piecewise linear
%\end{IEEEkeywords}



% For peer review papers, you can put extra information on the cover
% page as needed:
% \ifCLASSOPTIONpeerreview
% \begin{center} \bfseries EDICS Category: 3-BBND \end{center}
% \fi
%
% For peerreview papers, this IEEEtran command inserts a page break and
% creates the second title. It will be ignored for other modes.
%\IEEEpeerreviewmaketitle




	\item Five cards—the ten, jack, queen, king and ace of diamonds, are well-shuffled with their face downwards. One card is then picked up at random.
\begin{enumerate}
\item
What is the probability that the card is the queen? 
\item
If the queen is drawn and put aside, what is the probability that the second card picked up is (a) an ace? (b) a queen?\\
\end{enumerate}
\solution
		%\begin{enumerate}[label=\thesection.\arabic*,ref=\thesection.\theenumi]
	\item One card is drawn from a well-shuffled deck of 52 cards. Find the probability of getting
\begin{enumerate}
\item A king of red colour 
\item A face card 
\item A red face card
\item The jack of hearts
\item A spade
\item The queen of diamonds

\end{enumerate}
\solution
		%\input{ncert/10/15/1/14/main.tex}
	\item Five cards—the ten, jack, queen, king and ace of diamonds, are well-shuffled with their face downwards. One card is then picked up at random.
\begin{enumerate}
\item
What is the probability that the card is the queen? 
\item
If the queen is drawn and put aside, what is the probability that the second card picked up is (a) an ace? (b) a queen?\\
\end{enumerate}
\solution
		%\input{ncert/10/15/1/15/defs.tex}
	\item A bag contains $5$ red balls and some blue balls. If the probability of drawing a blue ball is double that if a red ball, determine the number of blue balls in the bag. 
		\\
\solution
		%\input{ncert/10/15/2/3/defs.tex}
	\item A card is selected from a pack of 52 cards.
 \begin{enumerate}[label=(\alph*)] 
                 \item How many points are there in the sample space?
                 \item Calculate the probability that the card is an ace of spades.
                 \item Calculate the probability that the card is (i) an ace and (ii) black card.
 \end{enumerate}
\solution
		%\input{ncert/11/16/3/4/main.tex}
\item Four cards are drawn from a well-shuffled deck of 52 cards. What is the probability of obtaining 3 diamonds and one spade.
\\
\solution
		%\input{ncert/11/16/4/2/defs.tex}
\item In a certain lottery 10,000 tickets are sold and ten equal prizes are awarded. What is the probability of not getting a prize if you buy (a) one ticket (b) two tickets (c) 10 tickets ?	
\\
\solution
		%\input{ncert/11/16/4/4/defs.tex}
		%
\item 
Out of 100 students, two sections of 40 and 60 are formed. If you and your friend are among the 100 students, what is the probability that
\begin{enumerate}
\item you both enter the same section?
\item you both enter the different sections?
\end{enumerate}
\solution
		%\input{ncert/11/16/4/5/defs.tex}
	\item 
The number lock of a suitcase has 4 wheels each labelled with ten digits i.e. from 0 to 9.The lock opens with a sequence of four digits with no repeats.What is the probability of a person getting the right sequence to open the suitcase.
\\
\solution
		%\input{ncert/11/16/4/10/defs.tex}
		%
\item 
Two cards are drawn at random and without replacement from a pack of 52 playing cards. Find the probability that both the cards are black.
\\
\solution
		%\input{ncert/12/13/2/2/defs.tex}
		\item A box of oranges is inspected by examining three randomly selected oranges drawn without replacement. If all the three oranges are good, the box is approved for sale, otherwise, it is rejected. Find the probability that a box containing 15 oranges out of which 12 are good and 3 are bad ones will be approved for sale.
		\label{ncert/12/13/2/3/defs.tex}
		\item Two balls are drawn at random with replacement from a box containing 10 black and 8 red balls. Find the probability that
		\label{ncert/12/13/2/12}
\begin{enumerate}
\item both balls are red.
\item first ball is black and second is red.
\item one of them is black and other is red.
\end{enumerate}

\item In a hostel, 60\% of the students read Hindi newspaper, 40\% read English newspaper and 20\% read both Hindi and English newspapers. A student is selected at random.
		\label{ncert/12/13/2/15}
\begin{enumerate}
\item Find the probability that she reads neither Hindi nor English newspapers.
\item If she reads Hindi newspaper, find the probability that she reads English newspaper.
\item If she reads English newspaper, find the probability that she reads Hindi newspaper.\\
\end{enumerate}
\item The probability of obtaining an even prime number on each die, when a pair of dice is rolled is 
\begin{enumerate}
    \item $0$ 
    
    \item $\frac{1}{3}$ 
    
    \item $\frac{1}{12}$ 
    
    \item $\frac{1}{36}$ 
\end{enumerate}
\solution
		%\input{ncert/12/13/2/17/defs.tex}
	\item A bag contains 4 red and 4 black balls, another bag contains 2 red and 6 black balls. One of the two bags is selected at random and a ball is drawn from the bag which is found to be red. Find the probability that the ball is drawn from the first bag.
\\
\solution
		%\input{ncert/12/13/3/2/main.tex}
  \item
  Cards with numbers 2 to 101 are placed in a box. A card is selected at random.Find the probability that the card has
\begin{enumerate}[label=(\roman*)]
	\item an even number 
	\item a square number
\end{enumerate}
\solution
%\input{exemplar/10/13/3/32/main.tex}
\item
The king, queen and jack of clubs are removed from a deck of 52 playing cards and then well shuffled. Now one card is drawn at random from the remaining cards.  Determine the probability that the card is
\begin{enumerate}[label=(\roman*)]
\item a club
\item 10 of hearts
\end{enumerate}
\solution
%\input{exemplar/10/13/3/29/main.tex}
\item A team of medical students doing their internship have to assist during surgeries
at a city hospital. The probabilities of surgeries rated as very complex, complex,
routine, simple or very simple are respectively, 0.15, 0.20, 0.31, 0.26, .08. Find
the probabilities that a particular surgery will be rated
\begin{enumerate}
	\item complex or very complex;
	\item neither very complex nor very simple;
	\item routine or complex
	\item routine or simple
\end{enumerate}
\solution
%\input{exemplar/11/16/3/8(1)/main.tex}
\item A card is selected from a pack of 52 cards.
\begin{enumerate}[label=(\alph*)]
    \item How many points are there in the sample space?
    \item Calculate the probability that the card is an ace of spades.
    \item Calculate the probability that the card is (i) an ace and (ii) black card.
\end{enumerate}
\solution
%\input{exemplar/11/16/3/4/main2.tex}
\item The probability that a non leap year selected at random will contain 53 sundays.
\\
\solution
%\input{exemplar/10/13/1/19/main.tex}
\item One of the four persons John, Rita, Aslam or Gurpreet will be promoted next
month. Consequently the sample space consists of four elementary outcomes
S = {John promoted, Rita promoted, Aslam promoted, Gurpreet promoted}
You are told that the chances of John’s promotion is same as that of Gurpreet,
Rita’s chances of promotion are twice as likely as Johns. Aslam’s chances are
four times that of John.
\begin{enumerate}
	\item Determine
	\begin{enumerate}
		\item P (John promoted)
		\item P (Rita promoted)
		\item P (Aslam promoted)
		\item P (Gurpreet promoted)
	\end{enumerate}
	\item If A = {John promoted or Gurpreet promoted}, find P (A).
\end{enumerate}
\solution
%\input{exemplar/11/16/3/10/main.tex}
\item A card is drawn from a deck of 52 cards. Find the probability of getting a king or a heart or a red card.\\
\solution
%\input{exemplar/11/16/3/15/main.tex}
\item The probability that a student will pass his examination is 0.73, the probability of
the student getting a compartment is 0.13, and the probability that the student will
either pass or get compartment is 0.96. State True or False.\\
\solution
%\input{exemplar/11/16/3/31/main.tex}
\item A card is selected from a pack of 52 cards\\
\begin{enumerate}[label=(\alph*)]
\item How many points are there in the sample space?
\item Calculate the probability that the cards is an ace of spades.
\item Calculate the probability that the card is (i) an ace (ii)black card.\\
\end{enumerate}
%\input{ncert/11/16/3/4_1/Prob_4.tex}
\item In a non-leap year, the probability of having 53 tuesdays or 53 wednesdays is\\
\solution
%\input{exemplar/11/16/3/18/main.tex}
\item There are 1000 sealed envelopes in a box, 10 of them contain a cash prize of
Rs 100 each, 100 of them contain a cash prize of Rs 50 each and 200 of them
contain a cash prize of Rs 10 each and rest do not contain any cash prize. If they
are well shuffled and an envelope is picked up out, what is the probability that it
contains no cash prize?\\
\solution
%\input{exemplar/10/13/3/34/main.tex}
\item 
A die is thrown and a card is selected at random from a deck of 52 playing cards. The probability of getting an even number on the die and a spade card.\\
\solution
%\input{exemplar/12/13/3/78/main.tex}
\item
If 4-digit numbers greater than 5,000 are randomly formed from the digits 0, 1, 3, 5, and 7, what is the probability of forming a number divisible by 5 when:
\begin{enumerate}
    \item The digits are repeated?
    \item The repetition of digits is not allowed?
\end{enumerate}
\solution
%\input{ncert/11/16/4/9/main.tex}
\item Consider the probability space $\brak{\Omega, \mathcal{G}, P}$ where $\Omega = [0,2]$ and $\mathcal{G} = \cbrak{\phi, \Omega, [0,1], (1,2]}$. Let $X$ and $Y$ be two functions on $\Omega$ defined as
\begin{align*}
    X(\omega) = 
    \begin{cases}
        1 & \text{if }\omega \in [0, 1]\\
        2 & \text{if }\omega \in (1, 2]
    \end{cases}
\end{align*}
and
\begin{align*}
    Y(\omega) = 
    \begin{cases}
        2 & \text{if }\omega \in [0, 1.5]\\
        3 & \text{if }\omega \in (1.5, 2].
    \end{cases}
\end{align*}
Then which one of the following statements is true?
\begin{enumerate}
    \item [(A)] $X$ is a random variable with respect to $\mathcal{G}$, but $Y$ is not a random variable with respect to $\mathcal{G}$.
    \item [(B)] $Y$ is a random variable with respect to $\mathcal{G}$, but $X$ is not a random variable with respect to $\mathcal{G}$.
    \item [(C)] Neither $X$ nor $Y$ is a random variable with respect to $\mathcal{G}$.
    \item [(D)] Both $X$ and $Y$ are random variables with respect to $\mathcal{G}$.
\end{enumerate} \hfill (GATE ST 2023)\\
\solution
%\input{gate/ST/2023/14/main.tex}
	\item  A die is loaded in such a way that each odd number is twice as likely to occur as
each even number. Find $P(G)$, where $G$ is the event that a number greater than
3 occurs on a single roll of the die.
\\
\solution
		%\input{exemplar/11/16/3/5/main.tex}
	\item All the jacks, queens and kings are removed from a deck of 52 playing cards. The remaining cards are well shuffled and then one card is drawn at random. Giving ace a value 1 similar value for other cards, find the probability that the card has a value 
		\begin{enumerate}
			\item 7
			\item greater than 7
			\item less than 7
		\end{enumerate}
		%\input{exemplar/10/13/3/30/main.tex}
  \item A Lot consists of 48 mobile phones of which 42 are good, 3 have only minor defects and 3 have major defects.Varnika will buy a phone if it is good but the trader will only buy a mobile if it has no major defects. One phone is selected at random from the lot. What is the probability that it is
\begin{enumerate}
	\item acceptable to Varnika?
            \item acceptable to the trader?
\end{enumerate}
\solution
	%\input{exemplar/10/13/3/40/main.tex}
 \item A student says that if you throw a die, it will show up 1 or not 1. Therefore, the probability of getting 1 and the probability of getting 'not 1' each is equal to $\frac{1}{2}$. Is this correct? Give reasons.\\
 \solution
        %\input{exemplar/10/13/2/9/main.tex}
   \item Four candidates A, B, C, D have ap-
plied for the assignment to coach a school cricket
team. If A is twice as likely to be selected as B, and
B and C are given about the same chance of being
selected, while C is twice as likely to be selected
as D, what are the probabilities that
\begin{enumerate}
\item C will be selected?
\item A will not be selected?
\end{enumerate}
	%\input{exemplar/11/16/3/9/main.tex}
 \item A bag contain 24 balls of which $x$ balls are red, $2x$ are white and $3x$ are blue. A ball is selected at random, What is the probability that it is
\begin{enumerate}[label=\alph*)]
\item not red ?
\item white ?
\end{enumerate}
%\input{exemplar/10/13/3/41/main.tex}
If the letters of the word ASSASSINATION are arranged at random. Find the Probability that
\begin{enumerate}[label=(\alph*)]
\item Four $S's$ come consecutively in the word
\item Two  $I's$ and two $N's$ come together
\item All $A's$ are not coming together
\item No two $A's$ are coming together
\end{enumerate}
%\input{exemplar/11/16/3/14/main.tex}
	\item One urn contains two black balls (labelled B1 and B2) and one white ball. A
	second urn contains one black ball and two white balls (labelled W1 and W2).
	Suppose the following experiment is performed. One of the two urns is chosen
	at random. Next a ball is randomly chosen from the urn. Then a second ball is
	chosen at random from the same urn without replacing the first ball.
	
	\begin{enumerate}
	\item What is the probability that two black balls are chosen?
	
	\item What is the probability that two balls of opposite colour are chosen?
	\end{enumerate}
	\solution
	%\input{exemplar/11/16/3/12/main1.tex}
\end{enumerate}

	\item A bag contains $5$ red balls and some blue balls. If the probability of drawing a blue ball is double that if a red ball, determine the number of blue balls in the bag. 
		\\
\solution
		%\begin{enumerate}[label=\thesection.\arabic*,ref=\thesection.\theenumi]
	\item One card is drawn from a well-shuffled deck of 52 cards. Find the probability of getting
\begin{enumerate}
\item A king of red colour 
\item A face card 
\item A red face card
\item The jack of hearts
\item A spade
\item The queen of diamonds

\end{enumerate}
\solution
		%\input{ncert/10/15/1/14/main.tex}
	\item Five cards—the ten, jack, queen, king and ace of diamonds, are well-shuffled with their face downwards. One card is then picked up at random.
\begin{enumerate}
\item
What is the probability that the card is the queen? 
\item
If the queen is drawn and put aside, what is the probability that the second card picked up is (a) an ace? (b) a queen?\\
\end{enumerate}
\solution
		%\input{ncert/10/15/1/15/defs.tex}
	\item A bag contains $5$ red balls and some blue balls. If the probability of drawing a blue ball is double that if a red ball, determine the number of blue balls in the bag. 
		\\
\solution
		%\input{ncert/10/15/2/3/defs.tex}
	\item A card is selected from a pack of 52 cards.
 \begin{enumerate}[label=(\alph*)] 
                 \item How many points are there in the sample space?
                 \item Calculate the probability that the card is an ace of spades.
                 \item Calculate the probability that the card is (i) an ace and (ii) black card.
 \end{enumerate}
\solution
		%\input{ncert/11/16/3/4/main.tex}
\item Four cards are drawn from a well-shuffled deck of 52 cards. What is the probability of obtaining 3 diamonds and one spade.
\\
\solution
		%\input{ncert/11/16/4/2/defs.tex}
\item In a certain lottery 10,000 tickets are sold and ten equal prizes are awarded. What is the probability of not getting a prize if you buy (a) one ticket (b) two tickets (c) 10 tickets ?	
\\
\solution
		%\input{ncert/11/16/4/4/defs.tex}
		%
\item 
Out of 100 students, two sections of 40 and 60 are formed. If you and your friend are among the 100 students, what is the probability that
\begin{enumerate}
\item you both enter the same section?
\item you both enter the different sections?
\end{enumerate}
\solution
		%\input{ncert/11/16/4/5/defs.tex}
	\item 
The number lock of a suitcase has 4 wheels each labelled with ten digits i.e. from 0 to 9.The lock opens with a sequence of four digits with no repeats.What is the probability of a person getting the right sequence to open the suitcase.
\\
\solution
		%\input{ncert/11/16/4/10/defs.tex}
		%
\item 
Two cards are drawn at random and without replacement from a pack of 52 playing cards. Find the probability that both the cards are black.
\\
\solution
		%\input{ncert/12/13/2/2/defs.tex}
		\item A box of oranges is inspected by examining three randomly selected oranges drawn without replacement. If all the three oranges are good, the box is approved for sale, otherwise, it is rejected. Find the probability that a box containing 15 oranges out of which 12 are good and 3 are bad ones will be approved for sale.
		\label{ncert/12/13/2/3/defs.tex}
		\item Two balls are drawn at random with replacement from a box containing 10 black and 8 red balls. Find the probability that
		\label{ncert/12/13/2/12}
\begin{enumerate}
\item both balls are red.
\item first ball is black and second is red.
\item one of them is black and other is red.
\end{enumerate}

\item In a hostel, 60\% of the students read Hindi newspaper, 40\% read English newspaper and 20\% read both Hindi and English newspapers. A student is selected at random.
		\label{ncert/12/13/2/15}
\begin{enumerate}
\item Find the probability that she reads neither Hindi nor English newspapers.
\item If she reads Hindi newspaper, find the probability that she reads English newspaper.
\item If she reads English newspaper, find the probability that she reads Hindi newspaper.\\
\end{enumerate}
\item The probability of obtaining an even prime number on each die, when a pair of dice is rolled is 
\begin{enumerate}
    \item $0$ 
    
    \item $\frac{1}{3}$ 
    
    \item $\frac{1}{12}$ 
    
    \item $\frac{1}{36}$ 
\end{enumerate}
\solution
		%\input{ncert/12/13/2/17/defs.tex}
	\item A bag contains 4 red and 4 black balls, another bag contains 2 red and 6 black balls. One of the two bags is selected at random and a ball is drawn from the bag which is found to be red. Find the probability that the ball is drawn from the first bag.
\\
\solution
		%\input{ncert/12/13/3/2/main.tex}
  \item
  Cards with numbers 2 to 101 are placed in a box. A card is selected at random.Find the probability that the card has
\begin{enumerate}[label=(\roman*)]
	\item an even number 
	\item a square number
\end{enumerate}
\solution
%\input{exemplar/10/13/3/32/main.tex}
\item
The king, queen and jack of clubs are removed from a deck of 52 playing cards and then well shuffled. Now one card is drawn at random from the remaining cards.  Determine the probability that the card is
\begin{enumerate}[label=(\roman*)]
\item a club
\item 10 of hearts
\end{enumerate}
\solution
%\input{exemplar/10/13/3/29/main.tex}
\item A team of medical students doing their internship have to assist during surgeries
at a city hospital. The probabilities of surgeries rated as very complex, complex,
routine, simple or very simple are respectively, 0.15, 0.20, 0.31, 0.26, .08. Find
the probabilities that a particular surgery will be rated
\begin{enumerate}
	\item complex or very complex;
	\item neither very complex nor very simple;
	\item routine or complex
	\item routine or simple
\end{enumerate}
\solution
%\input{exemplar/11/16/3/8(1)/main.tex}
\item A card is selected from a pack of 52 cards.
\begin{enumerate}[label=(\alph*)]
    \item How many points are there in the sample space?
    \item Calculate the probability that the card is an ace of spades.
    \item Calculate the probability that the card is (i) an ace and (ii) black card.
\end{enumerate}
\solution
%\input{exemplar/11/16/3/4/main2.tex}
\item The probability that a non leap year selected at random will contain 53 sundays.
\\
\solution
%\input{exemplar/10/13/1/19/main.tex}
\item One of the four persons John, Rita, Aslam or Gurpreet will be promoted next
month. Consequently the sample space consists of four elementary outcomes
S = {John promoted, Rita promoted, Aslam promoted, Gurpreet promoted}
You are told that the chances of John’s promotion is same as that of Gurpreet,
Rita’s chances of promotion are twice as likely as Johns. Aslam’s chances are
four times that of John.
\begin{enumerate}
	\item Determine
	\begin{enumerate}
		\item P (John promoted)
		\item P (Rita promoted)
		\item P (Aslam promoted)
		\item P (Gurpreet promoted)
	\end{enumerate}
	\item If A = {John promoted or Gurpreet promoted}, find P (A).
\end{enumerate}
\solution
%\input{exemplar/11/16/3/10/main.tex}
\item A card is drawn from a deck of 52 cards. Find the probability of getting a king or a heart or a red card.\\
\solution
%\input{exemplar/11/16/3/15/main.tex}
\item The probability that a student will pass his examination is 0.73, the probability of
the student getting a compartment is 0.13, and the probability that the student will
either pass or get compartment is 0.96. State True or False.\\
\solution
%\input{exemplar/11/16/3/31/main.tex}
\item A card is selected from a pack of 52 cards\\
\begin{enumerate}[label=(\alph*)]
\item How many points are there in the sample space?
\item Calculate the probability that the cards is an ace of spades.
\item Calculate the probability that the card is (i) an ace (ii)black card.\\
\end{enumerate}
%\input{ncert/11/16/3/4_1/Prob_4.tex}
\item In a non-leap year, the probability of having 53 tuesdays or 53 wednesdays is\\
\solution
%\input{exemplar/11/16/3/18/main.tex}
\item There are 1000 sealed envelopes in a box, 10 of them contain a cash prize of
Rs 100 each, 100 of them contain a cash prize of Rs 50 each and 200 of them
contain a cash prize of Rs 10 each and rest do not contain any cash prize. If they
are well shuffled and an envelope is picked up out, what is the probability that it
contains no cash prize?\\
\solution
%\input{exemplar/10/13/3/34/main.tex}
\item 
A die is thrown and a card is selected at random from a deck of 52 playing cards. The probability of getting an even number on the die and a spade card.\\
\solution
%\input{exemplar/12/13/3/78/main.tex}
\item
If 4-digit numbers greater than 5,000 are randomly formed from the digits 0, 1, 3, 5, and 7, what is the probability of forming a number divisible by 5 when:
\begin{enumerate}
    \item The digits are repeated?
    \item The repetition of digits is not allowed?
\end{enumerate}
\solution
%\input{ncert/11/16/4/9/main.tex}
\item Consider the probability space $\brak{\Omega, \mathcal{G}, P}$ where $\Omega = [0,2]$ and $\mathcal{G} = \cbrak{\phi, \Omega, [0,1], (1,2]}$. Let $X$ and $Y$ be two functions on $\Omega$ defined as
\begin{align*}
    X(\omega) = 
    \begin{cases}
        1 & \text{if }\omega \in [0, 1]\\
        2 & \text{if }\omega \in (1, 2]
    \end{cases}
\end{align*}
and
\begin{align*}
    Y(\omega) = 
    \begin{cases}
        2 & \text{if }\omega \in [0, 1.5]\\
        3 & \text{if }\omega \in (1.5, 2].
    \end{cases}
\end{align*}
Then which one of the following statements is true?
\begin{enumerate}
    \item [(A)] $X$ is a random variable with respect to $\mathcal{G}$, but $Y$ is not a random variable with respect to $\mathcal{G}$.
    \item [(B)] $Y$ is a random variable with respect to $\mathcal{G}$, but $X$ is not a random variable with respect to $\mathcal{G}$.
    \item [(C)] Neither $X$ nor $Y$ is a random variable with respect to $\mathcal{G}$.
    \item [(D)] Both $X$ and $Y$ are random variables with respect to $\mathcal{G}$.
\end{enumerate} \hfill (GATE ST 2023)\\
\solution
%\input{gate/ST/2023/14/main.tex}
	\item  A die is loaded in such a way that each odd number is twice as likely to occur as
each even number. Find $P(G)$, where $G$ is the event that a number greater than
3 occurs on a single roll of the die.
\\
\solution
		%\input{exemplar/11/16/3/5/main.tex}
	\item All the jacks, queens and kings are removed from a deck of 52 playing cards. The remaining cards are well shuffled and then one card is drawn at random. Giving ace a value 1 similar value for other cards, find the probability that the card has a value 
		\begin{enumerate}
			\item 7
			\item greater than 7
			\item less than 7
		\end{enumerate}
		%\input{exemplar/10/13/3/30/main.tex}
  \item A Lot consists of 48 mobile phones of which 42 are good, 3 have only minor defects and 3 have major defects.Varnika will buy a phone if it is good but the trader will only buy a mobile if it has no major defects. One phone is selected at random from the lot. What is the probability that it is
\begin{enumerate}
	\item acceptable to Varnika?
            \item acceptable to the trader?
\end{enumerate}
\solution
	%\input{exemplar/10/13/3/40/main.tex}
 \item A student says that if you throw a die, it will show up 1 or not 1. Therefore, the probability of getting 1 and the probability of getting 'not 1' each is equal to $\frac{1}{2}$. Is this correct? Give reasons.\\
 \solution
        %\input{exemplar/10/13/2/9/main.tex}
   \item Four candidates A, B, C, D have ap-
plied for the assignment to coach a school cricket
team. If A is twice as likely to be selected as B, and
B and C are given about the same chance of being
selected, while C is twice as likely to be selected
as D, what are the probabilities that
\begin{enumerate}
\item C will be selected?
\item A will not be selected?
\end{enumerate}
	%\input{exemplar/11/16/3/9/main.tex}
 \item A bag contain 24 balls of which $x$ balls are red, $2x$ are white and $3x$ are blue. A ball is selected at random, What is the probability that it is
\begin{enumerate}[label=\alph*)]
\item not red ?
\item white ?
\end{enumerate}
%\input{exemplar/10/13/3/41/main.tex}
If the letters of the word ASSASSINATION are arranged at random. Find the Probability that
\begin{enumerate}[label=(\alph*)]
\item Four $S's$ come consecutively in the word
\item Two  $I's$ and two $N's$ come together
\item All $A's$ are not coming together
\item No two $A's$ are coming together
\end{enumerate}
%\input{exemplar/11/16/3/14/main.tex}
	\item One urn contains two black balls (labelled B1 and B2) and one white ball. A
	second urn contains one black ball and two white balls (labelled W1 and W2).
	Suppose the following experiment is performed. One of the two urns is chosen
	at random. Next a ball is randomly chosen from the urn. Then a second ball is
	chosen at random from the same urn without replacing the first ball.
	
	\begin{enumerate}
	\item What is the probability that two black balls are chosen?
	
	\item What is the probability that two balls of opposite colour are chosen?
	\end{enumerate}
	\solution
	%\input{exemplar/11/16/3/12/main1.tex}
\end{enumerate}

	\item A card is selected from a pack of 52 cards.
 \begin{enumerate}[label=(\alph*)] 
                 \item How many points are there in the sample space?
                 \item Calculate the probability that the card is an ace of spades.
                 \item Calculate the probability that the card is (i) an ace and (ii) black card.
 \end{enumerate}
\solution
		%\begin{table}[H]
	\centering
\begin{tabular}{|c|c|c|}
\hline
Random variable &Value &Definition\\ \hline
\multirow{3}{*}{X} &0 &Slips of Rs 1\\
&1 &Slips of Rs 5\\
&2 &Slips of Rs 13\\ \hline
\multirow{2}{*}{Y} &0 &Box A\\
&1 &Box B\\\hline
\end{tabular}
\caption{}
\label{tab:Distribution}
\end{table}
See \tabref{tab:Distribution}.
\begin{align}
p_{Y}\brak{k}= \begin{cases} 
      \frac{1}{3} & {k=0} \\
      \frac{2}{3 }& {k=1} 
   \end{cases}
   \\
p_{Y|X}\brak{0|0} = \frac{19}{25}\, 
p_{Y|X}\brak{0|1} = \frac{6}{25}\,
p_{Y|X}\brak{1|0} = \frac{45}{50}\,
p_{Y|X}\brak{1|2} = \frac{5}{50}
\end{align}
The desired probability is the probability that a slip drawn at random is marked other than Rs 1,
\begin{align}
&=1-p_X\brak{0}\\
&= p_X(1) + p_X(2)
\end{align}
Using Bayes theorem,
\begin{align}
&= p_Y\brak{0} \times \pr{Y=0 | X=1} + p_Y\brak{1} \times \pr{Y=1|X=2}\\
&=\frac{1}{3} \times \frac{6}{25} + \frac{2}{3} \times \frac{5}{50}\\
&=\frac{11}{75}
\end{align}

\newpage

%\tableofcontents

\bigskip

\renewcommand{\thefigure}{\theenumi}
\renewcommand{\thetable}{\theenumi}
%\renewcommand{\theequation}{\theenumi}

%\begin{abstract}
%%\boldmath
%In this letter, an algorithm for evaluating the exact analytical bit error rate  (BER)  for the piecewise linear (PL) combiner for  multiple relays is presented. Previous results were available only for upto three relays. The algorithm is unique in the sense that  the actual mathematical expressions, that are prohibitively large, need not be explicitly obtained. The diversity gain due to multiple relays is shown through plots of the analytical BER, well supported by simulations. 
%
%\end{abstract}
% IEEEtran.cls defaults to using nonbold math in the Abstract.
% This preserves the distinction between vectors and scalars. However,
% if the journal you are submitting to favors bold math in the abstract,
% then you can use LaTeX's standard command \boldmath at the very start
% of the abstract to achieve this. Many IEEE journals frown on math
% in the abstract anyway.

% Note that keywords are not normally used for peerreview papers.
%\begin{IEEEkeywords}
%Cooperative diversity, decode and forward, piecewise linear
%\end{IEEEkeywords}



% For peer review papers, you can put extra information on the cover
% page as needed:
% \ifCLASSOPTIONpeerreview
% \begin{center} \bfseries EDICS Category: 3-BBND \end{center}
% \fi
%
% For peerreview papers, this IEEEtran command inserts a page break and
% creates the second title. It will be ignored for other modes.
%\IEEEpeerreviewmaketitle




\item Four cards are drawn from a well-shuffled deck of 52 cards. What is the probability of obtaining 3 diamonds and one spade.
\\
\solution
		%\begin{enumerate}[label=\thesection.\arabic*,ref=\thesection.\theenumi]
	\item One card is drawn from a well-shuffled deck of 52 cards. Find the probability of getting
\begin{enumerate}
\item A king of red colour 
\item A face card 
\item A red face card
\item The jack of hearts
\item A spade
\item The queen of diamonds

\end{enumerate}
\solution
		%\input{ncert/10/15/1/14/main.tex}
	\item Five cards—the ten, jack, queen, king and ace of diamonds, are well-shuffled with their face downwards. One card is then picked up at random.
\begin{enumerate}
\item
What is the probability that the card is the queen? 
\item
If the queen is drawn and put aside, what is the probability that the second card picked up is (a) an ace? (b) a queen?\\
\end{enumerate}
\solution
		%\input{ncert/10/15/1/15/defs.tex}
	\item A bag contains $5$ red balls and some blue balls. If the probability of drawing a blue ball is double that if a red ball, determine the number of blue balls in the bag. 
		\\
\solution
		%\input{ncert/10/15/2/3/defs.tex}
	\item A card is selected from a pack of 52 cards.
 \begin{enumerate}[label=(\alph*)] 
                 \item How many points are there in the sample space?
                 \item Calculate the probability that the card is an ace of spades.
                 \item Calculate the probability that the card is (i) an ace and (ii) black card.
 \end{enumerate}
\solution
		%\input{ncert/11/16/3/4/main.tex}
\item Four cards are drawn from a well-shuffled deck of 52 cards. What is the probability of obtaining 3 diamonds and one spade.
\\
\solution
		%\input{ncert/11/16/4/2/defs.tex}
\item In a certain lottery 10,000 tickets are sold and ten equal prizes are awarded. What is the probability of not getting a prize if you buy (a) one ticket (b) two tickets (c) 10 tickets ?	
\\
\solution
		%\input{ncert/11/16/4/4/defs.tex}
		%
\item 
Out of 100 students, two sections of 40 and 60 are formed. If you and your friend are among the 100 students, what is the probability that
\begin{enumerate}
\item you both enter the same section?
\item you both enter the different sections?
\end{enumerate}
\solution
		%\input{ncert/11/16/4/5/defs.tex}
	\item 
The number lock of a suitcase has 4 wheels each labelled with ten digits i.e. from 0 to 9.The lock opens with a sequence of four digits with no repeats.What is the probability of a person getting the right sequence to open the suitcase.
\\
\solution
		%\input{ncert/11/16/4/10/defs.tex}
		%
\item 
Two cards are drawn at random and without replacement from a pack of 52 playing cards. Find the probability that both the cards are black.
\\
\solution
		%\input{ncert/12/13/2/2/defs.tex}
		\item A box of oranges is inspected by examining three randomly selected oranges drawn without replacement. If all the three oranges are good, the box is approved for sale, otherwise, it is rejected. Find the probability that a box containing 15 oranges out of which 12 are good and 3 are bad ones will be approved for sale.
		\label{ncert/12/13/2/3/defs.tex}
		\item Two balls are drawn at random with replacement from a box containing 10 black and 8 red balls. Find the probability that
		\label{ncert/12/13/2/12}
\begin{enumerate}
\item both balls are red.
\item first ball is black and second is red.
\item one of them is black and other is red.
\end{enumerate}

\item In a hostel, 60\% of the students read Hindi newspaper, 40\% read English newspaper and 20\% read both Hindi and English newspapers. A student is selected at random.
		\label{ncert/12/13/2/15}
\begin{enumerate}
\item Find the probability that she reads neither Hindi nor English newspapers.
\item If she reads Hindi newspaper, find the probability that she reads English newspaper.
\item If she reads English newspaper, find the probability that she reads Hindi newspaper.\\
\end{enumerate}
\item The probability of obtaining an even prime number on each die, when a pair of dice is rolled is 
\begin{enumerate}
    \item $0$ 
    
    \item $\frac{1}{3}$ 
    
    \item $\frac{1}{12}$ 
    
    \item $\frac{1}{36}$ 
\end{enumerate}
\solution
		%\input{ncert/12/13/2/17/defs.tex}
	\item A bag contains 4 red and 4 black balls, another bag contains 2 red and 6 black balls. One of the two bags is selected at random and a ball is drawn from the bag which is found to be red. Find the probability that the ball is drawn from the first bag.
\\
\solution
		%\input{ncert/12/13/3/2/main.tex}
  \item
  Cards with numbers 2 to 101 are placed in a box. A card is selected at random.Find the probability that the card has
\begin{enumerate}[label=(\roman*)]
	\item an even number 
	\item a square number
\end{enumerate}
\solution
%\input{exemplar/10/13/3/32/main.tex}
\item
The king, queen and jack of clubs are removed from a deck of 52 playing cards and then well shuffled. Now one card is drawn at random from the remaining cards.  Determine the probability that the card is
\begin{enumerate}[label=(\roman*)]
\item a club
\item 10 of hearts
\end{enumerate}
\solution
%\input{exemplar/10/13/3/29/main.tex}
\item A team of medical students doing their internship have to assist during surgeries
at a city hospital. The probabilities of surgeries rated as very complex, complex,
routine, simple or very simple are respectively, 0.15, 0.20, 0.31, 0.26, .08. Find
the probabilities that a particular surgery will be rated
\begin{enumerate}
	\item complex or very complex;
	\item neither very complex nor very simple;
	\item routine or complex
	\item routine or simple
\end{enumerate}
\solution
%\input{exemplar/11/16/3/8(1)/main.tex}
\item A card is selected from a pack of 52 cards.
\begin{enumerate}[label=(\alph*)]
    \item How many points are there in the sample space?
    \item Calculate the probability that the card is an ace of spades.
    \item Calculate the probability that the card is (i) an ace and (ii) black card.
\end{enumerate}
\solution
%\input{exemplar/11/16/3/4/main2.tex}
\item The probability that a non leap year selected at random will contain 53 sundays.
\\
\solution
%\input{exemplar/10/13/1/19/main.tex}
\item One of the four persons John, Rita, Aslam or Gurpreet will be promoted next
month. Consequently the sample space consists of four elementary outcomes
S = {John promoted, Rita promoted, Aslam promoted, Gurpreet promoted}
You are told that the chances of John’s promotion is same as that of Gurpreet,
Rita’s chances of promotion are twice as likely as Johns. Aslam’s chances are
four times that of John.
\begin{enumerate}
	\item Determine
	\begin{enumerate}
		\item P (John promoted)
		\item P (Rita promoted)
		\item P (Aslam promoted)
		\item P (Gurpreet promoted)
	\end{enumerate}
	\item If A = {John promoted or Gurpreet promoted}, find P (A).
\end{enumerate}
\solution
%\input{exemplar/11/16/3/10/main.tex}
\item A card is drawn from a deck of 52 cards. Find the probability of getting a king or a heart or a red card.\\
\solution
%\input{exemplar/11/16/3/15/main.tex}
\item The probability that a student will pass his examination is 0.73, the probability of
the student getting a compartment is 0.13, and the probability that the student will
either pass or get compartment is 0.96. State True or False.\\
\solution
%\input{exemplar/11/16/3/31/main.tex}
\item A card is selected from a pack of 52 cards\\
\begin{enumerate}[label=(\alph*)]
\item How many points are there in the sample space?
\item Calculate the probability that the cards is an ace of spades.
\item Calculate the probability that the card is (i) an ace (ii)black card.\\
\end{enumerate}
%\input{ncert/11/16/3/4_1/Prob_4.tex}
\item In a non-leap year, the probability of having 53 tuesdays or 53 wednesdays is\\
\solution
%\input{exemplar/11/16/3/18/main.tex}
\item There are 1000 sealed envelopes in a box, 10 of them contain a cash prize of
Rs 100 each, 100 of them contain a cash prize of Rs 50 each and 200 of them
contain a cash prize of Rs 10 each and rest do not contain any cash prize. If they
are well shuffled and an envelope is picked up out, what is the probability that it
contains no cash prize?\\
\solution
%\input{exemplar/10/13/3/34/main.tex}
\item 
A die is thrown and a card is selected at random from a deck of 52 playing cards. The probability of getting an even number on the die and a spade card.\\
\solution
%\input{exemplar/12/13/3/78/main.tex}
\item
If 4-digit numbers greater than 5,000 are randomly formed from the digits 0, 1, 3, 5, and 7, what is the probability of forming a number divisible by 5 when:
\begin{enumerate}
    \item The digits are repeated?
    \item The repetition of digits is not allowed?
\end{enumerate}
\solution
%\input{ncert/11/16/4/9/main.tex}
\item Consider the probability space $\brak{\Omega, \mathcal{G}, P}$ where $\Omega = [0,2]$ and $\mathcal{G} = \cbrak{\phi, \Omega, [0,1], (1,2]}$. Let $X$ and $Y$ be two functions on $\Omega$ defined as
\begin{align*}
    X(\omega) = 
    \begin{cases}
        1 & \text{if }\omega \in [0, 1]\\
        2 & \text{if }\omega \in (1, 2]
    \end{cases}
\end{align*}
and
\begin{align*}
    Y(\omega) = 
    \begin{cases}
        2 & \text{if }\omega \in [0, 1.5]\\
        3 & \text{if }\omega \in (1.5, 2].
    \end{cases}
\end{align*}
Then which one of the following statements is true?
\begin{enumerate}
    \item [(A)] $X$ is a random variable with respect to $\mathcal{G}$, but $Y$ is not a random variable with respect to $\mathcal{G}$.
    \item [(B)] $Y$ is a random variable with respect to $\mathcal{G}$, but $X$ is not a random variable with respect to $\mathcal{G}$.
    \item [(C)] Neither $X$ nor $Y$ is a random variable with respect to $\mathcal{G}$.
    \item [(D)] Both $X$ and $Y$ are random variables with respect to $\mathcal{G}$.
\end{enumerate} \hfill (GATE ST 2023)\\
\solution
%\input{gate/ST/2023/14/main.tex}
	\item  A die is loaded in such a way that each odd number is twice as likely to occur as
each even number. Find $P(G)$, where $G$ is the event that a number greater than
3 occurs on a single roll of the die.
\\
\solution
		%\input{exemplar/11/16/3/5/main.tex}
	\item All the jacks, queens and kings are removed from a deck of 52 playing cards. The remaining cards are well shuffled and then one card is drawn at random. Giving ace a value 1 similar value for other cards, find the probability that the card has a value 
		\begin{enumerate}
			\item 7
			\item greater than 7
			\item less than 7
		\end{enumerate}
		%\input{exemplar/10/13/3/30/main.tex}
  \item A Lot consists of 48 mobile phones of which 42 are good, 3 have only minor defects and 3 have major defects.Varnika will buy a phone if it is good but the trader will only buy a mobile if it has no major defects. One phone is selected at random from the lot. What is the probability that it is
\begin{enumerate}
	\item acceptable to Varnika?
            \item acceptable to the trader?
\end{enumerate}
\solution
	%\input{exemplar/10/13/3/40/main.tex}
 \item A student says that if you throw a die, it will show up 1 or not 1. Therefore, the probability of getting 1 and the probability of getting 'not 1' each is equal to $\frac{1}{2}$. Is this correct? Give reasons.\\
 \solution
        %\input{exemplar/10/13/2/9/main.tex}
   \item Four candidates A, B, C, D have ap-
plied for the assignment to coach a school cricket
team. If A is twice as likely to be selected as B, and
B and C are given about the same chance of being
selected, while C is twice as likely to be selected
as D, what are the probabilities that
\begin{enumerate}
\item C will be selected?
\item A will not be selected?
\end{enumerate}
	%\input{exemplar/11/16/3/9/main.tex}
 \item A bag contain 24 balls of which $x$ balls are red, $2x$ are white and $3x$ are blue. A ball is selected at random, What is the probability that it is
\begin{enumerate}[label=\alph*)]
\item not red ?
\item white ?
\end{enumerate}
%\input{exemplar/10/13/3/41/main.tex}
If the letters of the word ASSASSINATION are arranged at random. Find the Probability that
\begin{enumerate}[label=(\alph*)]
\item Four $S's$ come consecutively in the word
\item Two  $I's$ and two $N's$ come together
\item All $A's$ are not coming together
\item No two $A's$ are coming together
\end{enumerate}
%\input{exemplar/11/16/3/14/main.tex}
	\item One urn contains two black balls (labelled B1 and B2) and one white ball. A
	second urn contains one black ball and two white balls (labelled W1 and W2).
	Suppose the following experiment is performed. One of the two urns is chosen
	at random. Next a ball is randomly chosen from the urn. Then a second ball is
	chosen at random from the same urn without replacing the first ball.
	
	\begin{enumerate}
	\item What is the probability that two black balls are chosen?
	
	\item What is the probability that two balls of opposite colour are chosen?
	\end{enumerate}
	\solution
	%\input{exemplar/11/16/3/12/main1.tex}
\end{enumerate}

\item In a certain lottery 10,000 tickets are sold and ten equal prizes are awarded. What is the probability of not getting a prize if you buy (a) one ticket (b) two tickets (c) 10 tickets ?	
\\
\solution
		%\begin{enumerate}[label=\thesection.\arabic*,ref=\thesection.\theenumi]
	\item One card is drawn from a well-shuffled deck of 52 cards. Find the probability of getting
\begin{enumerate}
\item A king of red colour 
\item A face card 
\item A red face card
\item The jack of hearts
\item A spade
\item The queen of diamonds

\end{enumerate}
\solution
		%\input{ncert/10/15/1/14/main.tex}
	\item Five cards—the ten, jack, queen, king and ace of diamonds, are well-shuffled with their face downwards. One card is then picked up at random.
\begin{enumerate}
\item
What is the probability that the card is the queen? 
\item
If the queen is drawn and put aside, what is the probability that the second card picked up is (a) an ace? (b) a queen?\\
\end{enumerate}
\solution
		%\input{ncert/10/15/1/15/defs.tex}
	\item A bag contains $5$ red balls and some blue balls. If the probability of drawing a blue ball is double that if a red ball, determine the number of blue balls in the bag. 
		\\
\solution
		%\input{ncert/10/15/2/3/defs.tex}
	\item A card is selected from a pack of 52 cards.
 \begin{enumerate}[label=(\alph*)] 
                 \item How many points are there in the sample space?
                 \item Calculate the probability that the card is an ace of spades.
                 \item Calculate the probability that the card is (i) an ace and (ii) black card.
 \end{enumerate}
\solution
		%\input{ncert/11/16/3/4/main.tex}
\item Four cards are drawn from a well-shuffled deck of 52 cards. What is the probability of obtaining 3 diamonds and one spade.
\\
\solution
		%\input{ncert/11/16/4/2/defs.tex}
\item In a certain lottery 10,000 tickets are sold and ten equal prizes are awarded. What is the probability of not getting a prize if you buy (a) one ticket (b) two tickets (c) 10 tickets ?	
\\
\solution
		%\input{ncert/11/16/4/4/defs.tex}
		%
\item 
Out of 100 students, two sections of 40 and 60 are formed. If you and your friend are among the 100 students, what is the probability that
\begin{enumerate}
\item you both enter the same section?
\item you both enter the different sections?
\end{enumerate}
\solution
		%\input{ncert/11/16/4/5/defs.tex}
	\item 
The number lock of a suitcase has 4 wheels each labelled with ten digits i.e. from 0 to 9.The lock opens with a sequence of four digits with no repeats.What is the probability of a person getting the right sequence to open the suitcase.
\\
\solution
		%\input{ncert/11/16/4/10/defs.tex}
		%
\item 
Two cards are drawn at random and without replacement from a pack of 52 playing cards. Find the probability that both the cards are black.
\\
\solution
		%\input{ncert/12/13/2/2/defs.tex}
		\item A box of oranges is inspected by examining three randomly selected oranges drawn without replacement. If all the three oranges are good, the box is approved for sale, otherwise, it is rejected. Find the probability that a box containing 15 oranges out of which 12 are good and 3 are bad ones will be approved for sale.
		\label{ncert/12/13/2/3/defs.tex}
		\item Two balls are drawn at random with replacement from a box containing 10 black and 8 red balls. Find the probability that
		\label{ncert/12/13/2/12}
\begin{enumerate}
\item both balls are red.
\item first ball is black and second is red.
\item one of them is black and other is red.
\end{enumerate}

\item In a hostel, 60\% of the students read Hindi newspaper, 40\% read English newspaper and 20\% read both Hindi and English newspapers. A student is selected at random.
		\label{ncert/12/13/2/15}
\begin{enumerate}
\item Find the probability that she reads neither Hindi nor English newspapers.
\item If she reads Hindi newspaper, find the probability that she reads English newspaper.
\item If she reads English newspaper, find the probability that she reads Hindi newspaper.\\
\end{enumerate}
\item The probability of obtaining an even prime number on each die, when a pair of dice is rolled is 
\begin{enumerate}
    \item $0$ 
    
    \item $\frac{1}{3}$ 
    
    \item $\frac{1}{12}$ 
    
    \item $\frac{1}{36}$ 
\end{enumerate}
\solution
		%\input{ncert/12/13/2/17/defs.tex}
	\item A bag contains 4 red and 4 black balls, another bag contains 2 red and 6 black balls. One of the two bags is selected at random and a ball is drawn from the bag which is found to be red. Find the probability that the ball is drawn from the first bag.
\\
\solution
		%\input{ncert/12/13/3/2/main.tex}
  \item
  Cards with numbers 2 to 101 are placed in a box. A card is selected at random.Find the probability that the card has
\begin{enumerate}[label=(\roman*)]
	\item an even number 
	\item a square number
\end{enumerate}
\solution
%\input{exemplar/10/13/3/32/main.tex}
\item
The king, queen and jack of clubs are removed from a deck of 52 playing cards and then well shuffled. Now one card is drawn at random from the remaining cards.  Determine the probability that the card is
\begin{enumerate}[label=(\roman*)]
\item a club
\item 10 of hearts
\end{enumerate}
\solution
%\input{exemplar/10/13/3/29/main.tex}
\item A team of medical students doing their internship have to assist during surgeries
at a city hospital. The probabilities of surgeries rated as very complex, complex,
routine, simple or very simple are respectively, 0.15, 0.20, 0.31, 0.26, .08. Find
the probabilities that a particular surgery will be rated
\begin{enumerate}
	\item complex or very complex;
	\item neither very complex nor very simple;
	\item routine or complex
	\item routine or simple
\end{enumerate}
\solution
%\input{exemplar/11/16/3/8(1)/main.tex}
\item A card is selected from a pack of 52 cards.
\begin{enumerate}[label=(\alph*)]
    \item How many points are there in the sample space?
    \item Calculate the probability that the card is an ace of spades.
    \item Calculate the probability that the card is (i) an ace and (ii) black card.
\end{enumerate}
\solution
%\input{exemplar/11/16/3/4/main2.tex}
\item The probability that a non leap year selected at random will contain 53 sundays.
\\
\solution
%\input{exemplar/10/13/1/19/main.tex}
\item One of the four persons John, Rita, Aslam or Gurpreet will be promoted next
month. Consequently the sample space consists of four elementary outcomes
S = {John promoted, Rita promoted, Aslam promoted, Gurpreet promoted}
You are told that the chances of John’s promotion is same as that of Gurpreet,
Rita’s chances of promotion are twice as likely as Johns. Aslam’s chances are
four times that of John.
\begin{enumerate}
	\item Determine
	\begin{enumerate}
		\item P (John promoted)
		\item P (Rita promoted)
		\item P (Aslam promoted)
		\item P (Gurpreet promoted)
	\end{enumerate}
	\item If A = {John promoted or Gurpreet promoted}, find P (A).
\end{enumerate}
\solution
%\input{exemplar/11/16/3/10/main.tex}
\item A card is drawn from a deck of 52 cards. Find the probability of getting a king or a heart or a red card.\\
\solution
%\input{exemplar/11/16/3/15/main.tex}
\item The probability that a student will pass his examination is 0.73, the probability of
the student getting a compartment is 0.13, and the probability that the student will
either pass or get compartment is 0.96. State True or False.\\
\solution
%\input{exemplar/11/16/3/31/main.tex}
\item A card is selected from a pack of 52 cards\\
\begin{enumerate}[label=(\alph*)]
\item How many points are there in the sample space?
\item Calculate the probability that the cards is an ace of spades.
\item Calculate the probability that the card is (i) an ace (ii)black card.\\
\end{enumerate}
%\input{ncert/11/16/3/4_1/Prob_4.tex}
\item In a non-leap year, the probability of having 53 tuesdays or 53 wednesdays is\\
\solution
%\input{exemplar/11/16/3/18/main.tex}
\item There are 1000 sealed envelopes in a box, 10 of them contain a cash prize of
Rs 100 each, 100 of them contain a cash prize of Rs 50 each and 200 of them
contain a cash prize of Rs 10 each and rest do not contain any cash prize. If they
are well shuffled and an envelope is picked up out, what is the probability that it
contains no cash prize?\\
\solution
%\input{exemplar/10/13/3/34/main.tex}
\item 
A die is thrown and a card is selected at random from a deck of 52 playing cards. The probability of getting an even number on the die and a spade card.\\
\solution
%\input{exemplar/12/13/3/78/main.tex}
\item
If 4-digit numbers greater than 5,000 are randomly formed from the digits 0, 1, 3, 5, and 7, what is the probability of forming a number divisible by 5 when:
\begin{enumerate}
    \item The digits are repeated?
    \item The repetition of digits is not allowed?
\end{enumerate}
\solution
%\input{ncert/11/16/4/9/main.tex}
\item Consider the probability space $\brak{\Omega, \mathcal{G}, P}$ where $\Omega = [0,2]$ and $\mathcal{G} = \cbrak{\phi, \Omega, [0,1], (1,2]}$. Let $X$ and $Y$ be two functions on $\Omega$ defined as
\begin{align*}
    X(\omega) = 
    \begin{cases}
        1 & \text{if }\omega \in [0, 1]\\
        2 & \text{if }\omega \in (1, 2]
    \end{cases}
\end{align*}
and
\begin{align*}
    Y(\omega) = 
    \begin{cases}
        2 & \text{if }\omega \in [0, 1.5]\\
        3 & \text{if }\omega \in (1.5, 2].
    \end{cases}
\end{align*}
Then which one of the following statements is true?
\begin{enumerate}
    \item [(A)] $X$ is a random variable with respect to $\mathcal{G}$, but $Y$ is not a random variable with respect to $\mathcal{G}$.
    \item [(B)] $Y$ is a random variable with respect to $\mathcal{G}$, but $X$ is not a random variable with respect to $\mathcal{G}$.
    \item [(C)] Neither $X$ nor $Y$ is a random variable with respect to $\mathcal{G}$.
    \item [(D)] Both $X$ and $Y$ are random variables with respect to $\mathcal{G}$.
\end{enumerate} \hfill (GATE ST 2023)\\
\solution
%\input{gate/ST/2023/14/main.tex}
	\item  A die is loaded in such a way that each odd number is twice as likely to occur as
each even number. Find $P(G)$, where $G$ is the event that a number greater than
3 occurs on a single roll of the die.
\\
\solution
		%\input{exemplar/11/16/3/5/main.tex}
	\item All the jacks, queens and kings are removed from a deck of 52 playing cards. The remaining cards are well shuffled and then one card is drawn at random. Giving ace a value 1 similar value for other cards, find the probability that the card has a value 
		\begin{enumerate}
			\item 7
			\item greater than 7
			\item less than 7
		\end{enumerate}
		%\input{exemplar/10/13/3/30/main.tex}
  \item A Lot consists of 48 mobile phones of which 42 are good, 3 have only minor defects and 3 have major defects.Varnika will buy a phone if it is good but the trader will only buy a mobile if it has no major defects. One phone is selected at random from the lot. What is the probability that it is
\begin{enumerate}
	\item acceptable to Varnika?
            \item acceptable to the trader?
\end{enumerate}
\solution
	%\input{exemplar/10/13/3/40/main.tex}
 \item A student says that if you throw a die, it will show up 1 or not 1. Therefore, the probability of getting 1 and the probability of getting 'not 1' each is equal to $\frac{1}{2}$. Is this correct? Give reasons.\\
 \solution
        %\input{exemplar/10/13/2/9/main.tex}
   \item Four candidates A, B, C, D have ap-
plied for the assignment to coach a school cricket
team. If A is twice as likely to be selected as B, and
B and C are given about the same chance of being
selected, while C is twice as likely to be selected
as D, what are the probabilities that
\begin{enumerate}
\item C will be selected?
\item A will not be selected?
\end{enumerate}
	%\input{exemplar/11/16/3/9/main.tex}
 \item A bag contain 24 balls of which $x$ balls are red, $2x$ are white and $3x$ are blue. A ball is selected at random, What is the probability that it is
\begin{enumerate}[label=\alph*)]
\item not red ?
\item white ?
\end{enumerate}
%\input{exemplar/10/13/3/41/main.tex}
If the letters of the word ASSASSINATION are arranged at random. Find the Probability that
\begin{enumerate}[label=(\alph*)]
\item Four $S's$ come consecutively in the word
\item Two  $I's$ and two $N's$ come together
\item All $A's$ are not coming together
\item No two $A's$ are coming together
\end{enumerate}
%\input{exemplar/11/16/3/14/main.tex}
	\item One urn contains two black balls (labelled B1 and B2) and one white ball. A
	second urn contains one black ball and two white balls (labelled W1 and W2).
	Suppose the following experiment is performed. One of the two urns is chosen
	at random. Next a ball is randomly chosen from the urn. Then a second ball is
	chosen at random from the same urn without replacing the first ball.
	
	\begin{enumerate}
	\item What is the probability that two black balls are chosen?
	
	\item What is the probability that two balls of opposite colour are chosen?
	\end{enumerate}
	\solution
	%\input{exemplar/11/16/3/12/main1.tex}
\end{enumerate}

		%
\item 
Out of 100 students, two sections of 40 and 60 are formed. If you and your friend are among the 100 students, what is the probability that
\begin{enumerate}
\item you both enter the same section?
\item you both enter the different sections?
\end{enumerate}
\solution
		%\begin{enumerate}[label=\thesection.\arabic*,ref=\thesection.\theenumi]
	\item One card is drawn from a well-shuffled deck of 52 cards. Find the probability of getting
\begin{enumerate}
\item A king of red colour 
\item A face card 
\item A red face card
\item The jack of hearts
\item A spade
\item The queen of diamonds

\end{enumerate}
\solution
		%\input{ncert/10/15/1/14/main.tex}
	\item Five cards—the ten, jack, queen, king and ace of diamonds, are well-shuffled with their face downwards. One card is then picked up at random.
\begin{enumerate}
\item
What is the probability that the card is the queen? 
\item
If the queen is drawn and put aside, what is the probability that the second card picked up is (a) an ace? (b) a queen?\\
\end{enumerate}
\solution
		%\input{ncert/10/15/1/15/defs.tex}
	\item A bag contains $5$ red balls and some blue balls. If the probability of drawing a blue ball is double that if a red ball, determine the number of blue balls in the bag. 
		\\
\solution
		%\input{ncert/10/15/2/3/defs.tex}
	\item A card is selected from a pack of 52 cards.
 \begin{enumerate}[label=(\alph*)] 
                 \item How many points are there in the sample space?
                 \item Calculate the probability that the card is an ace of spades.
                 \item Calculate the probability that the card is (i) an ace and (ii) black card.
 \end{enumerate}
\solution
		%\input{ncert/11/16/3/4/main.tex}
\item Four cards are drawn from a well-shuffled deck of 52 cards. What is the probability of obtaining 3 diamonds and one spade.
\\
\solution
		%\input{ncert/11/16/4/2/defs.tex}
\item In a certain lottery 10,000 tickets are sold and ten equal prizes are awarded. What is the probability of not getting a prize if you buy (a) one ticket (b) two tickets (c) 10 tickets ?	
\\
\solution
		%\input{ncert/11/16/4/4/defs.tex}
		%
\item 
Out of 100 students, two sections of 40 and 60 are formed. If you and your friend are among the 100 students, what is the probability that
\begin{enumerate}
\item you both enter the same section?
\item you both enter the different sections?
\end{enumerate}
\solution
		%\input{ncert/11/16/4/5/defs.tex}
	\item 
The number lock of a suitcase has 4 wheels each labelled with ten digits i.e. from 0 to 9.The lock opens with a sequence of four digits with no repeats.What is the probability of a person getting the right sequence to open the suitcase.
\\
\solution
		%\input{ncert/11/16/4/10/defs.tex}
		%
\item 
Two cards are drawn at random and without replacement from a pack of 52 playing cards. Find the probability that both the cards are black.
\\
\solution
		%\input{ncert/12/13/2/2/defs.tex}
		\item A box of oranges is inspected by examining three randomly selected oranges drawn without replacement. If all the three oranges are good, the box is approved for sale, otherwise, it is rejected. Find the probability that a box containing 15 oranges out of which 12 are good and 3 are bad ones will be approved for sale.
		\label{ncert/12/13/2/3/defs.tex}
		\item Two balls are drawn at random with replacement from a box containing 10 black and 8 red balls. Find the probability that
		\label{ncert/12/13/2/12}
\begin{enumerate}
\item both balls are red.
\item first ball is black and second is red.
\item one of them is black and other is red.
\end{enumerate}

\item In a hostel, 60\% of the students read Hindi newspaper, 40\% read English newspaper and 20\% read both Hindi and English newspapers. A student is selected at random.
		\label{ncert/12/13/2/15}
\begin{enumerate}
\item Find the probability that she reads neither Hindi nor English newspapers.
\item If she reads Hindi newspaper, find the probability that she reads English newspaper.
\item If she reads English newspaper, find the probability that she reads Hindi newspaper.\\
\end{enumerate}
\item The probability of obtaining an even prime number on each die, when a pair of dice is rolled is 
\begin{enumerate}
    \item $0$ 
    
    \item $\frac{1}{3}$ 
    
    \item $\frac{1}{12}$ 
    
    \item $\frac{1}{36}$ 
\end{enumerate}
\solution
		%\input{ncert/12/13/2/17/defs.tex}
	\item A bag contains 4 red and 4 black balls, another bag contains 2 red and 6 black balls. One of the two bags is selected at random and a ball is drawn from the bag which is found to be red. Find the probability that the ball is drawn from the first bag.
\\
\solution
		%\input{ncert/12/13/3/2/main.tex}
  \item
  Cards with numbers 2 to 101 are placed in a box. A card is selected at random.Find the probability that the card has
\begin{enumerate}[label=(\roman*)]
	\item an even number 
	\item a square number
\end{enumerate}
\solution
%\input{exemplar/10/13/3/32/main.tex}
\item
The king, queen and jack of clubs are removed from a deck of 52 playing cards and then well shuffled. Now one card is drawn at random from the remaining cards.  Determine the probability that the card is
\begin{enumerate}[label=(\roman*)]
\item a club
\item 10 of hearts
\end{enumerate}
\solution
%\input{exemplar/10/13/3/29/main.tex}
\item A team of medical students doing their internship have to assist during surgeries
at a city hospital. The probabilities of surgeries rated as very complex, complex,
routine, simple or very simple are respectively, 0.15, 0.20, 0.31, 0.26, .08. Find
the probabilities that a particular surgery will be rated
\begin{enumerate}
	\item complex or very complex;
	\item neither very complex nor very simple;
	\item routine or complex
	\item routine or simple
\end{enumerate}
\solution
%\input{exemplar/11/16/3/8(1)/main.tex}
\item A card is selected from a pack of 52 cards.
\begin{enumerate}[label=(\alph*)]
    \item How many points are there in the sample space?
    \item Calculate the probability that the card is an ace of spades.
    \item Calculate the probability that the card is (i) an ace and (ii) black card.
\end{enumerate}
\solution
%\input{exemplar/11/16/3/4/main2.tex}
\item The probability that a non leap year selected at random will contain 53 sundays.
\\
\solution
%\input{exemplar/10/13/1/19/main.tex}
\item One of the four persons John, Rita, Aslam or Gurpreet will be promoted next
month. Consequently the sample space consists of four elementary outcomes
S = {John promoted, Rita promoted, Aslam promoted, Gurpreet promoted}
You are told that the chances of John’s promotion is same as that of Gurpreet,
Rita’s chances of promotion are twice as likely as Johns. Aslam’s chances are
four times that of John.
\begin{enumerate}
	\item Determine
	\begin{enumerate}
		\item P (John promoted)
		\item P (Rita promoted)
		\item P (Aslam promoted)
		\item P (Gurpreet promoted)
	\end{enumerate}
	\item If A = {John promoted or Gurpreet promoted}, find P (A).
\end{enumerate}
\solution
%\input{exemplar/11/16/3/10/main.tex}
\item A card is drawn from a deck of 52 cards. Find the probability of getting a king or a heart or a red card.\\
\solution
%\input{exemplar/11/16/3/15/main.tex}
\item The probability that a student will pass his examination is 0.73, the probability of
the student getting a compartment is 0.13, and the probability that the student will
either pass or get compartment is 0.96. State True or False.\\
\solution
%\input{exemplar/11/16/3/31/main.tex}
\item A card is selected from a pack of 52 cards\\
\begin{enumerate}[label=(\alph*)]
\item How many points are there in the sample space?
\item Calculate the probability that the cards is an ace of spades.
\item Calculate the probability that the card is (i) an ace (ii)black card.\\
\end{enumerate}
%\input{ncert/11/16/3/4_1/Prob_4.tex}
\item In a non-leap year, the probability of having 53 tuesdays or 53 wednesdays is\\
\solution
%\input{exemplar/11/16/3/18/main.tex}
\item There are 1000 sealed envelopes in a box, 10 of them contain a cash prize of
Rs 100 each, 100 of them contain a cash prize of Rs 50 each and 200 of them
contain a cash prize of Rs 10 each and rest do not contain any cash prize. If they
are well shuffled and an envelope is picked up out, what is the probability that it
contains no cash prize?\\
\solution
%\input{exemplar/10/13/3/34/main.tex}
\item 
A die is thrown and a card is selected at random from a deck of 52 playing cards. The probability of getting an even number on the die and a spade card.\\
\solution
%\input{exemplar/12/13/3/78/main.tex}
\item
If 4-digit numbers greater than 5,000 are randomly formed from the digits 0, 1, 3, 5, and 7, what is the probability of forming a number divisible by 5 when:
\begin{enumerate}
    \item The digits are repeated?
    \item The repetition of digits is not allowed?
\end{enumerate}
\solution
%\input{ncert/11/16/4/9/main.tex}
\item Consider the probability space $\brak{\Omega, \mathcal{G}, P}$ where $\Omega = [0,2]$ and $\mathcal{G} = \cbrak{\phi, \Omega, [0,1], (1,2]}$. Let $X$ and $Y$ be two functions on $\Omega$ defined as
\begin{align*}
    X(\omega) = 
    \begin{cases}
        1 & \text{if }\omega \in [0, 1]\\
        2 & \text{if }\omega \in (1, 2]
    \end{cases}
\end{align*}
and
\begin{align*}
    Y(\omega) = 
    \begin{cases}
        2 & \text{if }\omega \in [0, 1.5]\\
        3 & \text{if }\omega \in (1.5, 2].
    \end{cases}
\end{align*}
Then which one of the following statements is true?
\begin{enumerate}
    \item [(A)] $X$ is a random variable with respect to $\mathcal{G}$, but $Y$ is not a random variable with respect to $\mathcal{G}$.
    \item [(B)] $Y$ is a random variable with respect to $\mathcal{G}$, but $X$ is not a random variable with respect to $\mathcal{G}$.
    \item [(C)] Neither $X$ nor $Y$ is a random variable with respect to $\mathcal{G}$.
    \item [(D)] Both $X$ and $Y$ are random variables with respect to $\mathcal{G}$.
\end{enumerate} \hfill (GATE ST 2023)\\
\solution
%\input{gate/ST/2023/14/main.tex}
	\item  A die is loaded in such a way that each odd number is twice as likely to occur as
each even number. Find $P(G)$, where $G$ is the event that a number greater than
3 occurs on a single roll of the die.
\\
\solution
		%\input{exemplar/11/16/3/5/main.tex}
	\item All the jacks, queens and kings are removed from a deck of 52 playing cards. The remaining cards are well shuffled and then one card is drawn at random. Giving ace a value 1 similar value for other cards, find the probability that the card has a value 
		\begin{enumerate}
			\item 7
			\item greater than 7
			\item less than 7
		\end{enumerate}
		%\input{exemplar/10/13/3/30/main.tex}
  \item A Lot consists of 48 mobile phones of which 42 are good, 3 have only minor defects and 3 have major defects.Varnika will buy a phone if it is good but the trader will only buy a mobile if it has no major defects. One phone is selected at random from the lot. What is the probability that it is
\begin{enumerate}
	\item acceptable to Varnika?
            \item acceptable to the trader?
\end{enumerate}
\solution
	%\input{exemplar/10/13/3/40/main.tex}
 \item A student says that if you throw a die, it will show up 1 or not 1. Therefore, the probability of getting 1 and the probability of getting 'not 1' each is equal to $\frac{1}{2}$. Is this correct? Give reasons.\\
 \solution
        %\input{exemplar/10/13/2/9/main.tex}
   \item Four candidates A, B, C, D have ap-
plied for the assignment to coach a school cricket
team. If A is twice as likely to be selected as B, and
B and C are given about the same chance of being
selected, while C is twice as likely to be selected
as D, what are the probabilities that
\begin{enumerate}
\item C will be selected?
\item A will not be selected?
\end{enumerate}
	%\input{exemplar/11/16/3/9/main.tex}
 \item A bag contain 24 balls of which $x$ balls are red, $2x$ are white and $3x$ are blue. A ball is selected at random, What is the probability that it is
\begin{enumerate}[label=\alph*)]
\item not red ?
\item white ?
\end{enumerate}
%\input{exemplar/10/13/3/41/main.tex}
If the letters of the word ASSASSINATION are arranged at random. Find the Probability that
\begin{enumerate}[label=(\alph*)]
\item Four $S's$ come consecutively in the word
\item Two  $I's$ and two $N's$ come together
\item All $A's$ are not coming together
\item No two $A's$ are coming together
\end{enumerate}
%\input{exemplar/11/16/3/14/main.tex}
	\item One urn contains two black balls (labelled B1 and B2) and one white ball. A
	second urn contains one black ball and two white balls (labelled W1 and W2).
	Suppose the following experiment is performed. One of the two urns is chosen
	at random. Next a ball is randomly chosen from the urn. Then a second ball is
	chosen at random from the same urn without replacing the first ball.
	
	\begin{enumerate}
	\item What is the probability that two black balls are chosen?
	
	\item What is the probability that two balls of opposite colour are chosen?
	\end{enumerate}
	\solution
	%\input{exemplar/11/16/3/12/main1.tex}
\end{enumerate}

	\item 
The number lock of a suitcase has 4 wheels each labelled with ten digits i.e. from 0 to 9.The lock opens with a sequence of four digits with no repeats.What is the probability of a person getting the right sequence to open the suitcase.
\\
\solution
		%\begin{enumerate}[label=\thesection.\arabic*,ref=\thesection.\theenumi]
	\item One card is drawn from a well-shuffled deck of 52 cards. Find the probability of getting
\begin{enumerate}
\item A king of red colour 
\item A face card 
\item A red face card
\item The jack of hearts
\item A spade
\item The queen of diamonds

\end{enumerate}
\solution
		%\input{ncert/10/15/1/14/main.tex}
	\item Five cards—the ten, jack, queen, king and ace of diamonds, are well-shuffled with their face downwards. One card is then picked up at random.
\begin{enumerate}
\item
What is the probability that the card is the queen? 
\item
If the queen is drawn and put aside, what is the probability that the second card picked up is (a) an ace? (b) a queen?\\
\end{enumerate}
\solution
		%\input{ncert/10/15/1/15/defs.tex}
	\item A bag contains $5$ red balls and some blue balls. If the probability of drawing a blue ball is double that if a red ball, determine the number of blue balls in the bag. 
		\\
\solution
		%\input{ncert/10/15/2/3/defs.tex}
	\item A card is selected from a pack of 52 cards.
 \begin{enumerate}[label=(\alph*)] 
                 \item How many points are there in the sample space?
                 \item Calculate the probability that the card is an ace of spades.
                 \item Calculate the probability that the card is (i) an ace and (ii) black card.
 \end{enumerate}
\solution
		%\input{ncert/11/16/3/4/main.tex}
\item Four cards are drawn from a well-shuffled deck of 52 cards. What is the probability of obtaining 3 diamonds and one spade.
\\
\solution
		%\input{ncert/11/16/4/2/defs.tex}
\item In a certain lottery 10,000 tickets are sold and ten equal prizes are awarded. What is the probability of not getting a prize if you buy (a) one ticket (b) two tickets (c) 10 tickets ?	
\\
\solution
		%\input{ncert/11/16/4/4/defs.tex}
		%
\item 
Out of 100 students, two sections of 40 and 60 are formed. If you and your friend are among the 100 students, what is the probability that
\begin{enumerate}
\item you both enter the same section?
\item you both enter the different sections?
\end{enumerate}
\solution
		%\input{ncert/11/16/4/5/defs.tex}
	\item 
The number lock of a suitcase has 4 wheels each labelled with ten digits i.e. from 0 to 9.The lock opens with a sequence of four digits with no repeats.What is the probability of a person getting the right sequence to open the suitcase.
\\
\solution
		%\input{ncert/11/16/4/10/defs.tex}
		%
\item 
Two cards are drawn at random and without replacement from a pack of 52 playing cards. Find the probability that both the cards are black.
\\
\solution
		%\input{ncert/12/13/2/2/defs.tex}
		\item A box of oranges is inspected by examining three randomly selected oranges drawn without replacement. If all the three oranges are good, the box is approved for sale, otherwise, it is rejected. Find the probability that a box containing 15 oranges out of which 12 are good and 3 are bad ones will be approved for sale.
		\label{ncert/12/13/2/3/defs.tex}
		\item Two balls are drawn at random with replacement from a box containing 10 black and 8 red balls. Find the probability that
		\label{ncert/12/13/2/12}
\begin{enumerate}
\item both balls are red.
\item first ball is black and second is red.
\item one of them is black and other is red.
\end{enumerate}

\item In a hostel, 60\% of the students read Hindi newspaper, 40\% read English newspaper and 20\% read both Hindi and English newspapers. A student is selected at random.
		\label{ncert/12/13/2/15}
\begin{enumerate}
\item Find the probability that she reads neither Hindi nor English newspapers.
\item If she reads Hindi newspaper, find the probability that she reads English newspaper.
\item If she reads English newspaper, find the probability that she reads Hindi newspaper.\\
\end{enumerate}
\item The probability of obtaining an even prime number on each die, when a pair of dice is rolled is 
\begin{enumerate}
    \item $0$ 
    
    \item $\frac{1}{3}$ 
    
    \item $\frac{1}{12}$ 
    
    \item $\frac{1}{36}$ 
\end{enumerate}
\solution
		%\input{ncert/12/13/2/17/defs.tex}
	\item A bag contains 4 red and 4 black balls, another bag contains 2 red and 6 black balls. One of the two bags is selected at random and a ball is drawn from the bag which is found to be red. Find the probability that the ball is drawn from the first bag.
\\
\solution
		%\input{ncert/12/13/3/2/main.tex}
  \item
  Cards with numbers 2 to 101 are placed in a box. A card is selected at random.Find the probability that the card has
\begin{enumerate}[label=(\roman*)]
	\item an even number 
	\item a square number
\end{enumerate}
\solution
%\input{exemplar/10/13/3/32/main.tex}
\item
The king, queen and jack of clubs are removed from a deck of 52 playing cards and then well shuffled. Now one card is drawn at random from the remaining cards.  Determine the probability that the card is
\begin{enumerate}[label=(\roman*)]
\item a club
\item 10 of hearts
\end{enumerate}
\solution
%\input{exemplar/10/13/3/29/main.tex}
\item A team of medical students doing their internship have to assist during surgeries
at a city hospital. The probabilities of surgeries rated as very complex, complex,
routine, simple or very simple are respectively, 0.15, 0.20, 0.31, 0.26, .08. Find
the probabilities that a particular surgery will be rated
\begin{enumerate}
	\item complex or very complex;
	\item neither very complex nor very simple;
	\item routine or complex
	\item routine or simple
\end{enumerate}
\solution
%\input{exemplar/11/16/3/8(1)/main.tex}
\item A card is selected from a pack of 52 cards.
\begin{enumerate}[label=(\alph*)]
    \item How many points are there in the sample space?
    \item Calculate the probability that the card is an ace of spades.
    \item Calculate the probability that the card is (i) an ace and (ii) black card.
\end{enumerate}
\solution
%\input{exemplar/11/16/3/4/main2.tex}
\item The probability that a non leap year selected at random will contain 53 sundays.
\\
\solution
%\input{exemplar/10/13/1/19/main.tex}
\item One of the four persons John, Rita, Aslam or Gurpreet will be promoted next
month. Consequently the sample space consists of four elementary outcomes
S = {John promoted, Rita promoted, Aslam promoted, Gurpreet promoted}
You are told that the chances of John’s promotion is same as that of Gurpreet,
Rita’s chances of promotion are twice as likely as Johns. Aslam’s chances are
four times that of John.
\begin{enumerate}
	\item Determine
	\begin{enumerate}
		\item P (John promoted)
		\item P (Rita promoted)
		\item P (Aslam promoted)
		\item P (Gurpreet promoted)
	\end{enumerate}
	\item If A = {John promoted or Gurpreet promoted}, find P (A).
\end{enumerate}
\solution
%\input{exemplar/11/16/3/10/main.tex}
\item A card is drawn from a deck of 52 cards. Find the probability of getting a king or a heart or a red card.\\
\solution
%\input{exemplar/11/16/3/15/main.tex}
\item The probability that a student will pass his examination is 0.73, the probability of
the student getting a compartment is 0.13, and the probability that the student will
either pass or get compartment is 0.96. State True or False.\\
\solution
%\input{exemplar/11/16/3/31/main.tex}
\item A card is selected from a pack of 52 cards\\
\begin{enumerate}[label=(\alph*)]
\item How many points are there in the sample space?
\item Calculate the probability that the cards is an ace of spades.
\item Calculate the probability that the card is (i) an ace (ii)black card.\\
\end{enumerate}
%\input{ncert/11/16/3/4_1/Prob_4.tex}
\item In a non-leap year, the probability of having 53 tuesdays or 53 wednesdays is\\
\solution
%\input{exemplar/11/16/3/18/main.tex}
\item There are 1000 sealed envelopes in a box, 10 of them contain a cash prize of
Rs 100 each, 100 of them contain a cash prize of Rs 50 each and 200 of them
contain a cash prize of Rs 10 each and rest do not contain any cash prize. If they
are well shuffled and an envelope is picked up out, what is the probability that it
contains no cash prize?\\
\solution
%\input{exemplar/10/13/3/34/main.tex}
\item 
A die is thrown and a card is selected at random from a deck of 52 playing cards. The probability of getting an even number on the die and a spade card.\\
\solution
%\input{exemplar/12/13/3/78/main.tex}
\item
If 4-digit numbers greater than 5,000 are randomly formed from the digits 0, 1, 3, 5, and 7, what is the probability of forming a number divisible by 5 when:
\begin{enumerate}
    \item The digits are repeated?
    \item The repetition of digits is not allowed?
\end{enumerate}
\solution
%\input{ncert/11/16/4/9/main.tex}
\item Consider the probability space $\brak{\Omega, \mathcal{G}, P}$ where $\Omega = [0,2]$ and $\mathcal{G} = \cbrak{\phi, \Omega, [0,1], (1,2]}$. Let $X$ and $Y$ be two functions on $\Omega$ defined as
\begin{align*}
    X(\omega) = 
    \begin{cases}
        1 & \text{if }\omega \in [0, 1]\\
        2 & \text{if }\omega \in (1, 2]
    \end{cases}
\end{align*}
and
\begin{align*}
    Y(\omega) = 
    \begin{cases}
        2 & \text{if }\omega \in [0, 1.5]\\
        3 & \text{if }\omega \in (1.5, 2].
    \end{cases}
\end{align*}
Then which one of the following statements is true?
\begin{enumerate}
    \item [(A)] $X$ is a random variable with respect to $\mathcal{G}$, but $Y$ is not a random variable with respect to $\mathcal{G}$.
    \item [(B)] $Y$ is a random variable with respect to $\mathcal{G}$, but $X$ is not a random variable with respect to $\mathcal{G}$.
    \item [(C)] Neither $X$ nor $Y$ is a random variable with respect to $\mathcal{G}$.
    \item [(D)] Both $X$ and $Y$ are random variables with respect to $\mathcal{G}$.
\end{enumerate} \hfill (GATE ST 2023)\\
\solution
%\input{gate/ST/2023/14/main.tex}
	\item  A die is loaded in such a way that each odd number is twice as likely to occur as
each even number. Find $P(G)$, where $G$ is the event that a number greater than
3 occurs on a single roll of the die.
\\
\solution
		%\input{exemplar/11/16/3/5/main.tex}
	\item All the jacks, queens and kings are removed from a deck of 52 playing cards. The remaining cards are well shuffled and then one card is drawn at random. Giving ace a value 1 similar value for other cards, find the probability that the card has a value 
		\begin{enumerate}
			\item 7
			\item greater than 7
			\item less than 7
		\end{enumerate}
		%\input{exemplar/10/13/3/30/main.tex}
  \item A Lot consists of 48 mobile phones of which 42 are good, 3 have only minor defects and 3 have major defects.Varnika will buy a phone if it is good but the trader will only buy a mobile if it has no major defects. One phone is selected at random from the lot. What is the probability that it is
\begin{enumerate}
	\item acceptable to Varnika?
            \item acceptable to the trader?
\end{enumerate}
\solution
	%\input{exemplar/10/13/3/40/main.tex}
 \item A student says that if you throw a die, it will show up 1 or not 1. Therefore, the probability of getting 1 and the probability of getting 'not 1' each is equal to $\frac{1}{2}$. Is this correct? Give reasons.\\
 \solution
        %\input{exemplar/10/13/2/9/main.tex}
   \item Four candidates A, B, C, D have ap-
plied for the assignment to coach a school cricket
team. If A is twice as likely to be selected as B, and
B and C are given about the same chance of being
selected, while C is twice as likely to be selected
as D, what are the probabilities that
\begin{enumerate}
\item C will be selected?
\item A will not be selected?
\end{enumerate}
	%\input{exemplar/11/16/3/9/main.tex}
 \item A bag contain 24 balls of which $x$ balls are red, $2x$ are white and $3x$ are blue. A ball is selected at random, What is the probability that it is
\begin{enumerate}[label=\alph*)]
\item not red ?
\item white ?
\end{enumerate}
%\input{exemplar/10/13/3/41/main.tex}
If the letters of the word ASSASSINATION are arranged at random. Find the Probability that
\begin{enumerate}[label=(\alph*)]
\item Four $S's$ come consecutively in the word
\item Two  $I's$ and two $N's$ come together
\item All $A's$ are not coming together
\item No two $A's$ are coming together
\end{enumerate}
%\input{exemplar/11/16/3/14/main.tex}
	\item One urn contains two black balls (labelled B1 and B2) and one white ball. A
	second urn contains one black ball and two white balls (labelled W1 and W2).
	Suppose the following experiment is performed. One of the two urns is chosen
	at random. Next a ball is randomly chosen from the urn. Then a second ball is
	chosen at random from the same urn without replacing the first ball.
	
	\begin{enumerate}
	\item What is the probability that two black balls are chosen?
	
	\item What is the probability that two balls of opposite colour are chosen?
	\end{enumerate}
	\solution
	%\input{exemplar/11/16/3/12/main1.tex}
\end{enumerate}

		%
\item 
Two cards are drawn at random and without replacement from a pack of 52 playing cards. Find the probability that both the cards are black.
\\
\solution
		%\begin{enumerate}[label=\thesection.\arabic*,ref=\thesection.\theenumi]
	\item One card is drawn from a well-shuffled deck of 52 cards. Find the probability of getting
\begin{enumerate}
\item A king of red colour 
\item A face card 
\item A red face card
\item The jack of hearts
\item A spade
\item The queen of diamonds

\end{enumerate}
\solution
		%\input{ncert/10/15/1/14/main.tex}
	\item Five cards—the ten, jack, queen, king and ace of diamonds, are well-shuffled with their face downwards. One card is then picked up at random.
\begin{enumerate}
\item
What is the probability that the card is the queen? 
\item
If the queen is drawn and put aside, what is the probability that the second card picked up is (a) an ace? (b) a queen?\\
\end{enumerate}
\solution
		%\input{ncert/10/15/1/15/defs.tex}
	\item A bag contains $5$ red balls and some blue balls. If the probability of drawing a blue ball is double that if a red ball, determine the number of blue balls in the bag. 
		\\
\solution
		%\input{ncert/10/15/2/3/defs.tex}
	\item A card is selected from a pack of 52 cards.
 \begin{enumerate}[label=(\alph*)] 
                 \item How many points are there in the sample space?
                 \item Calculate the probability that the card is an ace of spades.
                 \item Calculate the probability that the card is (i) an ace and (ii) black card.
 \end{enumerate}
\solution
		%\input{ncert/11/16/3/4/main.tex}
\item Four cards are drawn from a well-shuffled deck of 52 cards. What is the probability of obtaining 3 diamonds and one spade.
\\
\solution
		%\input{ncert/11/16/4/2/defs.tex}
\item In a certain lottery 10,000 tickets are sold and ten equal prizes are awarded. What is the probability of not getting a prize if you buy (a) one ticket (b) two tickets (c) 10 tickets ?	
\\
\solution
		%\input{ncert/11/16/4/4/defs.tex}
		%
\item 
Out of 100 students, two sections of 40 and 60 are formed. If you and your friend are among the 100 students, what is the probability that
\begin{enumerate}
\item you both enter the same section?
\item you both enter the different sections?
\end{enumerate}
\solution
		%\input{ncert/11/16/4/5/defs.tex}
	\item 
The number lock of a suitcase has 4 wheels each labelled with ten digits i.e. from 0 to 9.The lock opens with a sequence of four digits with no repeats.What is the probability of a person getting the right sequence to open the suitcase.
\\
\solution
		%\input{ncert/11/16/4/10/defs.tex}
		%
\item 
Two cards are drawn at random and without replacement from a pack of 52 playing cards. Find the probability that both the cards are black.
\\
\solution
		%\input{ncert/12/13/2/2/defs.tex}
		\item A box of oranges is inspected by examining three randomly selected oranges drawn without replacement. If all the three oranges are good, the box is approved for sale, otherwise, it is rejected. Find the probability that a box containing 15 oranges out of which 12 are good and 3 are bad ones will be approved for sale.
		\label{ncert/12/13/2/3/defs.tex}
		\item Two balls are drawn at random with replacement from a box containing 10 black and 8 red balls. Find the probability that
		\label{ncert/12/13/2/12}
\begin{enumerate}
\item both balls are red.
\item first ball is black and second is red.
\item one of them is black and other is red.
\end{enumerate}

\item In a hostel, 60\% of the students read Hindi newspaper, 40\% read English newspaper and 20\% read both Hindi and English newspapers. A student is selected at random.
		\label{ncert/12/13/2/15}
\begin{enumerate}
\item Find the probability that she reads neither Hindi nor English newspapers.
\item If she reads Hindi newspaper, find the probability that she reads English newspaper.
\item If she reads English newspaper, find the probability that she reads Hindi newspaper.\\
\end{enumerate}
\item The probability of obtaining an even prime number on each die, when a pair of dice is rolled is 
\begin{enumerate}
    \item $0$ 
    
    \item $\frac{1}{3}$ 
    
    \item $\frac{1}{12}$ 
    
    \item $\frac{1}{36}$ 
\end{enumerate}
\solution
		%\input{ncert/12/13/2/17/defs.tex}
	\item A bag contains 4 red and 4 black balls, another bag contains 2 red and 6 black balls. One of the two bags is selected at random and a ball is drawn from the bag which is found to be red. Find the probability that the ball is drawn from the first bag.
\\
\solution
		%\input{ncert/12/13/3/2/main.tex}
  \item
  Cards with numbers 2 to 101 are placed in a box. A card is selected at random.Find the probability that the card has
\begin{enumerate}[label=(\roman*)]
	\item an even number 
	\item a square number
\end{enumerate}
\solution
%\input{exemplar/10/13/3/32/main.tex}
\item
The king, queen and jack of clubs are removed from a deck of 52 playing cards and then well shuffled. Now one card is drawn at random from the remaining cards.  Determine the probability that the card is
\begin{enumerate}[label=(\roman*)]
\item a club
\item 10 of hearts
\end{enumerate}
\solution
%\input{exemplar/10/13/3/29/main.tex}
\item A team of medical students doing their internship have to assist during surgeries
at a city hospital. The probabilities of surgeries rated as very complex, complex,
routine, simple or very simple are respectively, 0.15, 0.20, 0.31, 0.26, .08. Find
the probabilities that a particular surgery will be rated
\begin{enumerate}
	\item complex or very complex;
	\item neither very complex nor very simple;
	\item routine or complex
	\item routine or simple
\end{enumerate}
\solution
%\input{exemplar/11/16/3/8(1)/main.tex}
\item A card is selected from a pack of 52 cards.
\begin{enumerate}[label=(\alph*)]
    \item How many points are there in the sample space?
    \item Calculate the probability that the card is an ace of spades.
    \item Calculate the probability that the card is (i) an ace and (ii) black card.
\end{enumerate}
\solution
%\input{exemplar/11/16/3/4/main2.tex}
\item The probability that a non leap year selected at random will contain 53 sundays.
\\
\solution
%\input{exemplar/10/13/1/19/main.tex}
\item One of the four persons John, Rita, Aslam or Gurpreet will be promoted next
month. Consequently the sample space consists of four elementary outcomes
S = {John promoted, Rita promoted, Aslam promoted, Gurpreet promoted}
You are told that the chances of John’s promotion is same as that of Gurpreet,
Rita’s chances of promotion are twice as likely as Johns. Aslam’s chances are
four times that of John.
\begin{enumerate}
	\item Determine
	\begin{enumerate}
		\item P (John promoted)
		\item P (Rita promoted)
		\item P (Aslam promoted)
		\item P (Gurpreet promoted)
	\end{enumerate}
	\item If A = {John promoted or Gurpreet promoted}, find P (A).
\end{enumerate}
\solution
%\input{exemplar/11/16/3/10/main.tex}
\item A card is drawn from a deck of 52 cards. Find the probability of getting a king or a heart or a red card.\\
\solution
%\input{exemplar/11/16/3/15/main.tex}
\item The probability that a student will pass his examination is 0.73, the probability of
the student getting a compartment is 0.13, and the probability that the student will
either pass or get compartment is 0.96. State True or False.\\
\solution
%\input{exemplar/11/16/3/31/main.tex}
\item A card is selected from a pack of 52 cards\\
\begin{enumerate}[label=(\alph*)]
\item How many points are there in the sample space?
\item Calculate the probability that the cards is an ace of spades.
\item Calculate the probability that the card is (i) an ace (ii)black card.\\
\end{enumerate}
%\input{ncert/11/16/3/4_1/Prob_4.tex}
\item In a non-leap year, the probability of having 53 tuesdays or 53 wednesdays is\\
\solution
%\input{exemplar/11/16/3/18/main.tex}
\item There are 1000 sealed envelopes in a box, 10 of them contain a cash prize of
Rs 100 each, 100 of them contain a cash prize of Rs 50 each and 200 of them
contain a cash prize of Rs 10 each and rest do not contain any cash prize. If they
are well shuffled and an envelope is picked up out, what is the probability that it
contains no cash prize?\\
\solution
%\input{exemplar/10/13/3/34/main.tex}
\item 
A die is thrown and a card is selected at random from a deck of 52 playing cards. The probability of getting an even number on the die and a spade card.\\
\solution
%\input{exemplar/12/13/3/78/main.tex}
\item
If 4-digit numbers greater than 5,000 are randomly formed from the digits 0, 1, 3, 5, and 7, what is the probability of forming a number divisible by 5 when:
\begin{enumerate}
    \item The digits are repeated?
    \item The repetition of digits is not allowed?
\end{enumerate}
\solution
%\input{ncert/11/16/4/9/main.tex}
\item Consider the probability space $\brak{\Omega, \mathcal{G}, P}$ where $\Omega = [0,2]$ and $\mathcal{G} = \cbrak{\phi, \Omega, [0,1], (1,2]}$. Let $X$ and $Y$ be two functions on $\Omega$ defined as
\begin{align*}
    X(\omega) = 
    \begin{cases}
        1 & \text{if }\omega \in [0, 1]\\
        2 & \text{if }\omega \in (1, 2]
    \end{cases}
\end{align*}
and
\begin{align*}
    Y(\omega) = 
    \begin{cases}
        2 & \text{if }\omega \in [0, 1.5]\\
        3 & \text{if }\omega \in (1.5, 2].
    \end{cases}
\end{align*}
Then which one of the following statements is true?
\begin{enumerate}
    \item [(A)] $X$ is a random variable with respect to $\mathcal{G}$, but $Y$ is not a random variable with respect to $\mathcal{G}$.
    \item [(B)] $Y$ is a random variable with respect to $\mathcal{G}$, but $X$ is not a random variable with respect to $\mathcal{G}$.
    \item [(C)] Neither $X$ nor $Y$ is a random variable with respect to $\mathcal{G}$.
    \item [(D)] Both $X$ and $Y$ are random variables with respect to $\mathcal{G}$.
\end{enumerate} \hfill (GATE ST 2023)\\
\solution
%\input{gate/ST/2023/14/main.tex}
	\item  A die is loaded in such a way that each odd number is twice as likely to occur as
each even number. Find $P(G)$, where $G$ is the event that a number greater than
3 occurs on a single roll of the die.
\\
\solution
		%\input{exemplar/11/16/3/5/main.tex}
	\item All the jacks, queens and kings are removed from a deck of 52 playing cards. The remaining cards are well shuffled and then one card is drawn at random. Giving ace a value 1 similar value for other cards, find the probability that the card has a value 
		\begin{enumerate}
			\item 7
			\item greater than 7
			\item less than 7
		\end{enumerate}
		%\input{exemplar/10/13/3/30/main.tex}
  \item A Lot consists of 48 mobile phones of which 42 are good, 3 have only minor defects and 3 have major defects.Varnika will buy a phone if it is good but the trader will only buy a mobile if it has no major defects. One phone is selected at random from the lot. What is the probability that it is
\begin{enumerate}
	\item acceptable to Varnika?
            \item acceptable to the trader?
\end{enumerate}
\solution
	%\input{exemplar/10/13/3/40/main.tex}
 \item A student says that if you throw a die, it will show up 1 or not 1. Therefore, the probability of getting 1 and the probability of getting 'not 1' each is equal to $\frac{1}{2}$. Is this correct? Give reasons.\\
 \solution
        %\input{exemplar/10/13/2/9/main.tex}
   \item Four candidates A, B, C, D have ap-
plied for the assignment to coach a school cricket
team. If A is twice as likely to be selected as B, and
B and C are given about the same chance of being
selected, while C is twice as likely to be selected
as D, what are the probabilities that
\begin{enumerate}
\item C will be selected?
\item A will not be selected?
\end{enumerate}
	%\input{exemplar/11/16/3/9/main.tex}
 \item A bag contain 24 balls of which $x$ balls are red, $2x$ are white and $3x$ are blue. A ball is selected at random, What is the probability that it is
\begin{enumerate}[label=\alph*)]
\item not red ?
\item white ?
\end{enumerate}
%\input{exemplar/10/13/3/41/main.tex}
If the letters of the word ASSASSINATION are arranged at random. Find the Probability that
\begin{enumerate}[label=(\alph*)]
\item Four $S's$ come consecutively in the word
\item Two  $I's$ and two $N's$ come together
\item All $A's$ are not coming together
\item No two $A's$ are coming together
\end{enumerate}
%\input{exemplar/11/16/3/14/main.tex}
	\item One urn contains two black balls (labelled B1 and B2) and one white ball. A
	second urn contains one black ball and two white balls (labelled W1 and W2).
	Suppose the following experiment is performed. One of the two urns is chosen
	at random. Next a ball is randomly chosen from the urn. Then a second ball is
	chosen at random from the same urn without replacing the first ball.
	
	\begin{enumerate}
	\item What is the probability that two black balls are chosen?
	
	\item What is the probability that two balls of opposite colour are chosen?
	\end{enumerate}
	\solution
	%\input{exemplar/11/16/3/12/main1.tex}
\end{enumerate}

		\item A box of oranges is inspected by examining three randomly selected oranges drawn without replacement. If all the three oranges are good, the box is approved for sale, otherwise, it is rejected. Find the probability that a box containing 15 oranges out of which 12 are good and 3 are bad ones will be approved for sale.
		\label{ncert/12/13/2/3/defs.tex}
		\item Two balls are drawn at random with replacement from a box containing 10 black and 8 red balls. Find the probability that
		\label{ncert/12/13/2/12}
\begin{enumerate}
\item both balls are red.
\item first ball is black and second is red.
\item one of them is black and other is red.
\end{enumerate}

\item In a hostel, 60\% of the students read Hindi newspaper, 40\% read English newspaper and 20\% read both Hindi and English newspapers. A student is selected at random.
		\label{ncert/12/13/2/15}
\begin{enumerate}
\item Find the probability that she reads neither Hindi nor English newspapers.
\item If she reads Hindi newspaper, find the probability that she reads English newspaper.
\item If she reads English newspaper, find the probability that she reads Hindi newspaper.\\
\end{enumerate}
\item The probability of obtaining an even prime number on each die, when a pair of dice is rolled is 
\begin{enumerate}
    \item $0$ 
    
    \item $\frac{1}{3}$ 
    
    \item $\frac{1}{12}$ 
    
    \item $\frac{1}{36}$ 
\end{enumerate}
\solution
		%\begin{enumerate}[label=\thesection.\arabic*,ref=\thesection.\theenumi]
	\item One card is drawn from a well-shuffled deck of 52 cards. Find the probability of getting
\begin{enumerate}
\item A king of red colour 
\item A face card 
\item A red face card
\item The jack of hearts
\item A spade
\item The queen of diamonds

\end{enumerate}
\solution
		%\input{ncert/10/15/1/14/main.tex}
	\item Five cards—the ten, jack, queen, king and ace of diamonds, are well-shuffled with their face downwards. One card is then picked up at random.
\begin{enumerate}
\item
What is the probability that the card is the queen? 
\item
If the queen is drawn and put aside, what is the probability that the second card picked up is (a) an ace? (b) a queen?\\
\end{enumerate}
\solution
		%\input{ncert/10/15/1/15/defs.tex}
	\item A bag contains $5$ red balls and some blue balls. If the probability of drawing a blue ball is double that if a red ball, determine the number of blue balls in the bag. 
		\\
\solution
		%\input{ncert/10/15/2/3/defs.tex}
	\item A card is selected from a pack of 52 cards.
 \begin{enumerate}[label=(\alph*)] 
                 \item How many points are there in the sample space?
                 \item Calculate the probability that the card is an ace of spades.
                 \item Calculate the probability that the card is (i) an ace and (ii) black card.
 \end{enumerate}
\solution
		%\input{ncert/11/16/3/4/main.tex}
\item Four cards are drawn from a well-shuffled deck of 52 cards. What is the probability of obtaining 3 diamonds and one spade.
\\
\solution
		%\input{ncert/11/16/4/2/defs.tex}
\item In a certain lottery 10,000 tickets are sold and ten equal prizes are awarded. What is the probability of not getting a prize if you buy (a) one ticket (b) two tickets (c) 10 tickets ?	
\\
\solution
		%\input{ncert/11/16/4/4/defs.tex}
		%
\item 
Out of 100 students, two sections of 40 and 60 are formed. If you and your friend are among the 100 students, what is the probability that
\begin{enumerate}
\item you both enter the same section?
\item you both enter the different sections?
\end{enumerate}
\solution
		%\input{ncert/11/16/4/5/defs.tex}
	\item 
The number lock of a suitcase has 4 wheels each labelled with ten digits i.e. from 0 to 9.The lock opens with a sequence of four digits with no repeats.What is the probability of a person getting the right sequence to open the suitcase.
\\
\solution
		%\input{ncert/11/16/4/10/defs.tex}
		%
\item 
Two cards are drawn at random and without replacement from a pack of 52 playing cards. Find the probability that both the cards are black.
\\
\solution
		%\input{ncert/12/13/2/2/defs.tex}
		\item A box of oranges is inspected by examining three randomly selected oranges drawn without replacement. If all the three oranges are good, the box is approved for sale, otherwise, it is rejected. Find the probability that a box containing 15 oranges out of which 12 are good and 3 are bad ones will be approved for sale.
		\label{ncert/12/13/2/3/defs.tex}
		\item Two balls are drawn at random with replacement from a box containing 10 black and 8 red balls. Find the probability that
		\label{ncert/12/13/2/12}
\begin{enumerate}
\item both balls are red.
\item first ball is black and second is red.
\item one of them is black and other is red.
\end{enumerate}

\item In a hostel, 60\% of the students read Hindi newspaper, 40\% read English newspaper and 20\% read both Hindi and English newspapers. A student is selected at random.
		\label{ncert/12/13/2/15}
\begin{enumerate}
\item Find the probability that she reads neither Hindi nor English newspapers.
\item If she reads Hindi newspaper, find the probability that she reads English newspaper.
\item If she reads English newspaper, find the probability that she reads Hindi newspaper.\\
\end{enumerate}
\item The probability of obtaining an even prime number on each die, when a pair of dice is rolled is 
\begin{enumerate}
    \item $0$ 
    
    \item $\frac{1}{3}$ 
    
    \item $\frac{1}{12}$ 
    
    \item $\frac{1}{36}$ 
\end{enumerate}
\solution
		%\input{ncert/12/13/2/17/defs.tex}
	\item A bag contains 4 red and 4 black balls, another bag contains 2 red and 6 black balls. One of the two bags is selected at random and a ball is drawn from the bag which is found to be red. Find the probability that the ball is drawn from the first bag.
\\
\solution
		%\input{ncert/12/13/3/2/main.tex}
  \item
  Cards with numbers 2 to 101 are placed in a box. A card is selected at random.Find the probability that the card has
\begin{enumerate}[label=(\roman*)]
	\item an even number 
	\item a square number
\end{enumerate}
\solution
%\input{exemplar/10/13/3/32/main.tex}
\item
The king, queen and jack of clubs are removed from a deck of 52 playing cards and then well shuffled. Now one card is drawn at random from the remaining cards.  Determine the probability that the card is
\begin{enumerate}[label=(\roman*)]
\item a club
\item 10 of hearts
\end{enumerate}
\solution
%\input{exemplar/10/13/3/29/main.tex}
\item A team of medical students doing their internship have to assist during surgeries
at a city hospital. The probabilities of surgeries rated as very complex, complex,
routine, simple or very simple are respectively, 0.15, 0.20, 0.31, 0.26, .08. Find
the probabilities that a particular surgery will be rated
\begin{enumerate}
	\item complex or very complex;
	\item neither very complex nor very simple;
	\item routine or complex
	\item routine or simple
\end{enumerate}
\solution
%\input{exemplar/11/16/3/8(1)/main.tex}
\item A card is selected from a pack of 52 cards.
\begin{enumerate}[label=(\alph*)]
    \item How many points are there in the sample space?
    \item Calculate the probability that the card is an ace of spades.
    \item Calculate the probability that the card is (i) an ace and (ii) black card.
\end{enumerate}
\solution
%\input{exemplar/11/16/3/4/main2.tex}
\item The probability that a non leap year selected at random will contain 53 sundays.
\\
\solution
%\input{exemplar/10/13/1/19/main.tex}
\item One of the four persons John, Rita, Aslam or Gurpreet will be promoted next
month. Consequently the sample space consists of four elementary outcomes
S = {John promoted, Rita promoted, Aslam promoted, Gurpreet promoted}
You are told that the chances of John’s promotion is same as that of Gurpreet,
Rita’s chances of promotion are twice as likely as Johns. Aslam’s chances are
four times that of John.
\begin{enumerate}
	\item Determine
	\begin{enumerate}
		\item P (John promoted)
		\item P (Rita promoted)
		\item P (Aslam promoted)
		\item P (Gurpreet promoted)
	\end{enumerate}
	\item If A = {John promoted or Gurpreet promoted}, find P (A).
\end{enumerate}
\solution
%\input{exemplar/11/16/3/10/main.tex}
\item A card is drawn from a deck of 52 cards. Find the probability of getting a king or a heart or a red card.\\
\solution
%\input{exemplar/11/16/3/15/main.tex}
\item The probability that a student will pass his examination is 0.73, the probability of
the student getting a compartment is 0.13, and the probability that the student will
either pass or get compartment is 0.96. State True or False.\\
\solution
%\input{exemplar/11/16/3/31/main.tex}
\item A card is selected from a pack of 52 cards\\
\begin{enumerate}[label=(\alph*)]
\item How many points are there in the sample space?
\item Calculate the probability that the cards is an ace of spades.
\item Calculate the probability that the card is (i) an ace (ii)black card.\\
\end{enumerate}
%\input{ncert/11/16/3/4_1/Prob_4.tex}
\item In a non-leap year, the probability of having 53 tuesdays or 53 wednesdays is\\
\solution
%\input{exemplar/11/16/3/18/main.tex}
\item There are 1000 sealed envelopes in a box, 10 of them contain a cash prize of
Rs 100 each, 100 of them contain a cash prize of Rs 50 each and 200 of them
contain a cash prize of Rs 10 each and rest do not contain any cash prize. If they
are well shuffled and an envelope is picked up out, what is the probability that it
contains no cash prize?\\
\solution
%\input{exemplar/10/13/3/34/main.tex}
\item 
A die is thrown and a card is selected at random from a deck of 52 playing cards. The probability of getting an even number on the die and a spade card.\\
\solution
%\input{exemplar/12/13/3/78/main.tex}
\item
If 4-digit numbers greater than 5,000 are randomly formed from the digits 0, 1, 3, 5, and 7, what is the probability of forming a number divisible by 5 when:
\begin{enumerate}
    \item The digits are repeated?
    \item The repetition of digits is not allowed?
\end{enumerate}
\solution
%\input{ncert/11/16/4/9/main.tex}
\item Consider the probability space $\brak{\Omega, \mathcal{G}, P}$ where $\Omega = [0,2]$ and $\mathcal{G} = \cbrak{\phi, \Omega, [0,1], (1,2]}$. Let $X$ and $Y$ be two functions on $\Omega$ defined as
\begin{align*}
    X(\omega) = 
    \begin{cases}
        1 & \text{if }\omega \in [0, 1]\\
        2 & \text{if }\omega \in (1, 2]
    \end{cases}
\end{align*}
and
\begin{align*}
    Y(\omega) = 
    \begin{cases}
        2 & \text{if }\omega \in [0, 1.5]\\
        3 & \text{if }\omega \in (1.5, 2].
    \end{cases}
\end{align*}
Then which one of the following statements is true?
\begin{enumerate}
    \item [(A)] $X$ is a random variable with respect to $\mathcal{G}$, but $Y$ is not a random variable with respect to $\mathcal{G}$.
    \item [(B)] $Y$ is a random variable with respect to $\mathcal{G}$, but $X$ is not a random variable with respect to $\mathcal{G}$.
    \item [(C)] Neither $X$ nor $Y$ is a random variable with respect to $\mathcal{G}$.
    \item [(D)] Both $X$ and $Y$ are random variables with respect to $\mathcal{G}$.
\end{enumerate} \hfill (GATE ST 2023)\\
\solution
%\input{gate/ST/2023/14/main.tex}
	\item  A die is loaded in such a way that each odd number is twice as likely to occur as
each even number. Find $P(G)$, where $G$ is the event that a number greater than
3 occurs on a single roll of the die.
\\
\solution
		%\input{exemplar/11/16/3/5/main.tex}
	\item All the jacks, queens and kings are removed from a deck of 52 playing cards. The remaining cards are well shuffled and then one card is drawn at random. Giving ace a value 1 similar value for other cards, find the probability that the card has a value 
		\begin{enumerate}
			\item 7
			\item greater than 7
			\item less than 7
		\end{enumerate}
		%\input{exemplar/10/13/3/30/main.tex}
  \item A Lot consists of 48 mobile phones of which 42 are good, 3 have only minor defects and 3 have major defects.Varnika will buy a phone if it is good but the trader will only buy a mobile if it has no major defects. One phone is selected at random from the lot. What is the probability that it is
\begin{enumerate}
	\item acceptable to Varnika?
            \item acceptable to the trader?
\end{enumerate}
\solution
	%\input{exemplar/10/13/3/40/main.tex}
 \item A student says that if you throw a die, it will show up 1 or not 1. Therefore, the probability of getting 1 and the probability of getting 'not 1' each is equal to $\frac{1}{2}$. Is this correct? Give reasons.\\
 \solution
        %\input{exemplar/10/13/2/9/main.tex}
   \item Four candidates A, B, C, D have ap-
plied for the assignment to coach a school cricket
team. If A is twice as likely to be selected as B, and
B and C are given about the same chance of being
selected, while C is twice as likely to be selected
as D, what are the probabilities that
\begin{enumerate}
\item C will be selected?
\item A will not be selected?
\end{enumerate}
	%\input{exemplar/11/16/3/9/main.tex}
 \item A bag contain 24 balls of which $x$ balls are red, $2x$ are white and $3x$ are blue. A ball is selected at random, What is the probability that it is
\begin{enumerate}[label=\alph*)]
\item not red ?
\item white ?
\end{enumerate}
%\input{exemplar/10/13/3/41/main.tex}
If the letters of the word ASSASSINATION are arranged at random. Find the Probability that
\begin{enumerate}[label=(\alph*)]
\item Four $S's$ come consecutively in the word
\item Two  $I's$ and two $N's$ come together
\item All $A's$ are not coming together
\item No two $A's$ are coming together
\end{enumerate}
%\input{exemplar/11/16/3/14/main.tex}
	\item One urn contains two black balls (labelled B1 and B2) and one white ball. A
	second urn contains one black ball and two white balls (labelled W1 and W2).
	Suppose the following experiment is performed. One of the two urns is chosen
	at random. Next a ball is randomly chosen from the urn. Then a second ball is
	chosen at random from the same urn without replacing the first ball.
	
	\begin{enumerate}
	\item What is the probability that two black balls are chosen?
	
	\item What is the probability that two balls of opposite colour are chosen?
	\end{enumerate}
	\solution
	%\input{exemplar/11/16/3/12/main1.tex}
\end{enumerate}

	\item A bag contains 4 red and 4 black balls, another bag contains 2 red and 6 black balls. One of the two bags is selected at random and a ball is drawn from the bag which is found to be red. Find the probability that the ball is drawn from the first bag.
\\
\solution
		%\begin{table}[H]
	\centering
\begin{tabular}{|c|c|c|}
\hline
Random variable &Value &Definition\\ \hline
\multirow{3}{*}{X} &0 &Slips of Rs 1\\
&1 &Slips of Rs 5\\
&2 &Slips of Rs 13\\ \hline
\multirow{2}{*}{Y} &0 &Box A\\
&1 &Box B\\\hline
\end{tabular}
\caption{}
\label{tab:Distribution}
\end{table}
See \tabref{tab:Distribution}.
\begin{align}
p_{Y}\brak{k}= \begin{cases} 
      \frac{1}{3} & {k=0} \\
      \frac{2}{3 }& {k=1} 
   \end{cases}
   \\
p_{Y|X}\brak{0|0} = \frac{19}{25}\, 
p_{Y|X}\brak{0|1} = \frac{6}{25}\,
p_{Y|X}\brak{1|0} = \frac{45}{50}\,
p_{Y|X}\brak{1|2} = \frac{5}{50}
\end{align}
The desired probability is the probability that a slip drawn at random is marked other than Rs 1,
\begin{align}
&=1-p_X\brak{0}\\
&= p_X(1) + p_X(2)
\end{align}
Using Bayes theorem,
\begin{align}
&= p_Y\brak{0} \times \pr{Y=0 | X=1} + p_Y\brak{1} \times \pr{Y=1|X=2}\\
&=\frac{1}{3} \times \frac{6}{25} + \frac{2}{3} \times \frac{5}{50}\\
&=\frac{11}{75}
\end{align}

\newpage

%\tableofcontents

\bigskip

\renewcommand{\thefigure}{\theenumi}
\renewcommand{\thetable}{\theenumi}
%\renewcommand{\theequation}{\theenumi}

%\begin{abstract}
%%\boldmath
%In this letter, an algorithm for evaluating the exact analytical bit error rate  (BER)  for the piecewise linear (PL) combiner for  multiple relays is presented. Previous results were available only for upto three relays. The algorithm is unique in the sense that  the actual mathematical expressions, that are prohibitively large, need not be explicitly obtained. The diversity gain due to multiple relays is shown through plots of the analytical BER, well supported by simulations. 
%
%\end{abstract}
% IEEEtran.cls defaults to using nonbold math in the Abstract.
% This preserves the distinction between vectors and scalars. However,
% if the journal you are submitting to favors bold math in the abstract,
% then you can use LaTeX's standard command \boldmath at the very start
% of the abstract to achieve this. Many IEEE journals frown on math
% in the abstract anyway.

% Note that keywords are not normally used for peerreview papers.
%\begin{IEEEkeywords}
%Cooperative diversity, decode and forward, piecewise linear
%\end{IEEEkeywords}



% For peer review papers, you can put extra information on the cover
% page as needed:
% \ifCLASSOPTIONpeerreview
% \begin{center} \bfseries EDICS Category: 3-BBND \end{center}
% \fi
%
% For peerreview papers, this IEEEtran command inserts a page break and
% creates the second title. It will be ignored for other modes.
%\IEEEpeerreviewmaketitle




  \item
  Cards with numbers 2 to 101 are placed in a box. A card is selected at random.Find the probability that the card has
\begin{enumerate}[label=(\roman*)]
	\item an even number 
	\item a square number
\end{enumerate}
\solution
%\begin{table}[H]
	\centering
\begin{tabular}{|c|c|c|}
\hline
Random variable &Value &Definition\\ \hline
\multirow{3}{*}{X} &0 &Slips of Rs 1\\
&1 &Slips of Rs 5\\
&2 &Slips of Rs 13\\ \hline
\multirow{2}{*}{Y} &0 &Box A\\
&1 &Box B\\\hline
\end{tabular}
\caption{}
\label{tab:Distribution}
\end{table}
See \tabref{tab:Distribution}.
\begin{align}
p_{Y}\brak{k}= \begin{cases} 
      \frac{1}{3} & {k=0} \\
      \frac{2}{3 }& {k=1} 
   \end{cases}
   \\
p_{Y|X}\brak{0|0} = \frac{19}{25}\, 
p_{Y|X}\brak{0|1} = \frac{6}{25}\,
p_{Y|X}\brak{1|0} = \frac{45}{50}\,
p_{Y|X}\brak{1|2} = \frac{5}{50}
\end{align}
The desired probability is the probability that a slip drawn at random is marked other than Rs 1,
\begin{align}
&=1-p_X\brak{0}\\
&= p_X(1) + p_X(2)
\end{align}
Using Bayes theorem,
\begin{align}
&= p_Y\brak{0} \times \pr{Y=0 | X=1} + p_Y\brak{1} \times \pr{Y=1|X=2}\\
&=\frac{1}{3} \times \frac{6}{25} + \frac{2}{3} \times \frac{5}{50}\\
&=\frac{11}{75}
\end{align}

\newpage

%\tableofcontents

\bigskip

\renewcommand{\thefigure}{\theenumi}
\renewcommand{\thetable}{\theenumi}
%\renewcommand{\theequation}{\theenumi}

%\begin{abstract}
%%\boldmath
%In this letter, an algorithm for evaluating the exact analytical bit error rate  (BER)  for the piecewise linear (PL) combiner for  multiple relays is presented. Previous results were available only for upto three relays. The algorithm is unique in the sense that  the actual mathematical expressions, that are prohibitively large, need not be explicitly obtained. The diversity gain due to multiple relays is shown through plots of the analytical BER, well supported by simulations. 
%
%\end{abstract}
% IEEEtran.cls defaults to using nonbold math in the Abstract.
% This preserves the distinction between vectors and scalars. However,
% if the journal you are submitting to favors bold math in the abstract,
% then you can use LaTeX's standard command \boldmath at the very start
% of the abstract to achieve this. Many IEEE journals frown on math
% in the abstract anyway.

% Note that keywords are not normally used for peerreview papers.
%\begin{IEEEkeywords}
%Cooperative diversity, decode and forward, piecewise linear
%\end{IEEEkeywords}



% For peer review papers, you can put extra information on the cover
% page as needed:
% \ifCLASSOPTIONpeerreview
% \begin{center} \bfseries EDICS Category: 3-BBND \end{center}
% \fi
%
% For peerreview papers, this IEEEtran command inserts a page break and
% creates the second title. It will be ignored for other modes.
%\IEEEpeerreviewmaketitle




\item
The king, queen and jack of clubs are removed from a deck of 52 playing cards and then well shuffled. Now one card is drawn at random from the remaining cards.  Determine the probability that the card is
\begin{enumerate}[label=(\roman*)]
\item a club
\item 10 of hearts
\end{enumerate}
\solution
%\begin{table}[H]
	\centering
\begin{tabular}{|c|c|c|}
\hline
Random variable &Value &Definition\\ \hline
\multirow{3}{*}{X} &0 &Slips of Rs 1\\
&1 &Slips of Rs 5\\
&2 &Slips of Rs 13\\ \hline
\multirow{2}{*}{Y} &0 &Box A\\
&1 &Box B\\\hline
\end{tabular}
\caption{}
\label{tab:Distribution}
\end{table}
See \tabref{tab:Distribution}.
\begin{align}
p_{Y}\brak{k}= \begin{cases} 
      \frac{1}{3} & {k=0} \\
      \frac{2}{3 }& {k=1} 
   \end{cases}
   \\
p_{Y|X}\brak{0|0} = \frac{19}{25}\, 
p_{Y|X}\brak{0|1} = \frac{6}{25}\,
p_{Y|X}\brak{1|0} = \frac{45}{50}\,
p_{Y|X}\brak{1|2} = \frac{5}{50}
\end{align}
The desired probability is the probability that a slip drawn at random is marked other than Rs 1,
\begin{align}
&=1-p_X\brak{0}\\
&= p_X(1) + p_X(2)
\end{align}
Using Bayes theorem,
\begin{align}
&= p_Y\brak{0} \times \pr{Y=0 | X=1} + p_Y\brak{1} \times \pr{Y=1|X=2}\\
&=\frac{1}{3} \times \frac{6}{25} + \frac{2}{3} \times \frac{5}{50}\\
&=\frac{11}{75}
\end{align}

\newpage

%\tableofcontents

\bigskip

\renewcommand{\thefigure}{\theenumi}
\renewcommand{\thetable}{\theenumi}
%\renewcommand{\theequation}{\theenumi}

%\begin{abstract}
%%\boldmath
%In this letter, an algorithm for evaluating the exact analytical bit error rate  (BER)  for the piecewise linear (PL) combiner for  multiple relays is presented. Previous results were available only for upto three relays. The algorithm is unique in the sense that  the actual mathematical expressions, that are prohibitively large, need not be explicitly obtained. The diversity gain due to multiple relays is shown through plots of the analytical BER, well supported by simulations. 
%
%\end{abstract}
% IEEEtran.cls defaults to using nonbold math in the Abstract.
% This preserves the distinction between vectors and scalars. However,
% if the journal you are submitting to favors bold math in the abstract,
% then you can use LaTeX's standard command \boldmath at the very start
% of the abstract to achieve this. Many IEEE journals frown on math
% in the abstract anyway.

% Note that keywords are not normally used for peerreview papers.
%\begin{IEEEkeywords}
%Cooperative diversity, decode and forward, piecewise linear
%\end{IEEEkeywords}



% For peer review papers, you can put extra information on the cover
% page as needed:
% \ifCLASSOPTIONpeerreview
% \begin{center} \bfseries EDICS Category: 3-BBND \end{center}
% \fi
%
% For peerreview papers, this IEEEtran command inserts a page break and
% creates the second title. It will be ignored for other modes.
%\IEEEpeerreviewmaketitle




\item A team of medical students doing their internship have to assist during surgeries
at a city hospital. The probabilities of surgeries rated as very complex, complex,
routine, simple or very simple are respectively, 0.15, 0.20, 0.31, 0.26, .08. Find
the probabilities that a particular surgery will be rated
\begin{enumerate}
	\item complex or very complex;
	\item neither very complex nor very simple;
	\item routine or complex
	\item routine or simple
\end{enumerate}
\solution
%\begin{table}[H]
	\centering
\begin{tabular}{|c|c|c|}
\hline
Random variable &Value &Definition\\ \hline
\multirow{3}{*}{X} &0 &Slips of Rs 1\\
&1 &Slips of Rs 5\\
&2 &Slips of Rs 13\\ \hline
\multirow{2}{*}{Y} &0 &Box A\\
&1 &Box B\\\hline
\end{tabular}
\caption{}
\label{tab:Distribution}
\end{table}
See \tabref{tab:Distribution}.
\begin{align}
p_{Y}\brak{k}= \begin{cases} 
      \frac{1}{3} & {k=0} \\
      \frac{2}{3 }& {k=1} 
   \end{cases}
   \\
p_{Y|X}\brak{0|0} = \frac{19}{25}\, 
p_{Y|X}\brak{0|1} = \frac{6}{25}\,
p_{Y|X}\brak{1|0} = \frac{45}{50}\,
p_{Y|X}\brak{1|2} = \frac{5}{50}
\end{align}
The desired probability is the probability that a slip drawn at random is marked other than Rs 1,
\begin{align}
&=1-p_X\brak{0}\\
&= p_X(1) + p_X(2)
\end{align}
Using Bayes theorem,
\begin{align}
&= p_Y\brak{0} \times \pr{Y=0 | X=1} + p_Y\brak{1} \times \pr{Y=1|X=2}\\
&=\frac{1}{3} \times \frac{6}{25} + \frac{2}{3} \times \frac{5}{50}\\
&=\frac{11}{75}
\end{align}

\newpage

%\tableofcontents

\bigskip

\renewcommand{\thefigure}{\theenumi}
\renewcommand{\thetable}{\theenumi}
%\renewcommand{\theequation}{\theenumi}

%\begin{abstract}
%%\boldmath
%In this letter, an algorithm for evaluating the exact analytical bit error rate  (BER)  for the piecewise linear (PL) combiner for  multiple relays is presented. Previous results were available only for upto three relays. The algorithm is unique in the sense that  the actual mathematical expressions, that are prohibitively large, need not be explicitly obtained. The diversity gain due to multiple relays is shown through plots of the analytical BER, well supported by simulations. 
%
%\end{abstract}
% IEEEtran.cls defaults to using nonbold math in the Abstract.
% This preserves the distinction between vectors and scalars. However,
% if the journal you are submitting to favors bold math in the abstract,
% then you can use LaTeX's standard command \boldmath at the very start
% of the abstract to achieve this. Many IEEE journals frown on math
% in the abstract anyway.

% Note that keywords are not normally used for peerreview papers.
%\begin{IEEEkeywords}
%Cooperative diversity, decode and forward, piecewise linear
%\end{IEEEkeywords}



% For peer review papers, you can put extra information on the cover
% page as needed:
% \ifCLASSOPTIONpeerreview
% \begin{center} \bfseries EDICS Category: 3-BBND \end{center}
% \fi
%
% For peerreview papers, this IEEEtran command inserts a page break and
% creates the second title. It will be ignored for other modes.
%\IEEEpeerreviewmaketitle




\item A card is selected from a pack of 52 cards.
\begin{enumerate}[label=(\alph*)]
    \item How many points are there in the sample space?
    \item Calculate the probability that the card is an ace of spades.
    \item Calculate the probability that the card is (i) an ace and (ii) black card.
\end{enumerate}
\solution
%Let $X$ be an bernoulli rv defined as in \tabref{tab:exemplar/11/16/3/26}.  Then, 
\begin{equation}
    p =
        \frac{4}{11} 
\end{equation}
\begin{table}[H]
	\centering
	\input{exemplar/11/16/3/26/tables/Table2.tex}
	\caption{}
        \label{tab:exemplar/11/16/3/26}
\end{table}

\item The probability that a non leap year selected at random will contain 53 sundays.
\\
\solution
%\begin{table}[H]
	\centering
\begin{tabular}{|c|c|c|}
\hline
Random variable &Value &Definition\\ \hline
\multirow{3}{*}{X} &0 &Slips of Rs 1\\
&1 &Slips of Rs 5\\
&2 &Slips of Rs 13\\ \hline
\multirow{2}{*}{Y} &0 &Box A\\
&1 &Box B\\\hline
\end{tabular}
\caption{}
\label{tab:Distribution}
\end{table}
See \tabref{tab:Distribution}.
\begin{align}
p_{Y}\brak{k}= \begin{cases} 
      \frac{1}{3} & {k=0} \\
      \frac{2}{3 }& {k=1} 
   \end{cases}
   \\
p_{Y|X}\brak{0|0} = \frac{19}{25}\, 
p_{Y|X}\brak{0|1} = \frac{6}{25}\,
p_{Y|X}\brak{1|0} = \frac{45}{50}\,
p_{Y|X}\brak{1|2} = \frac{5}{50}
\end{align}
The desired probability is the probability that a slip drawn at random is marked other than Rs 1,
\begin{align}
&=1-p_X\brak{0}\\
&= p_X(1) + p_X(2)
\end{align}
Using Bayes theorem,
\begin{align}
&= p_Y\brak{0} \times \pr{Y=0 | X=1} + p_Y\brak{1} \times \pr{Y=1|X=2}\\
&=\frac{1}{3} \times \frac{6}{25} + \frac{2}{3} \times \frac{5}{50}\\
&=\frac{11}{75}
\end{align}

\newpage

%\tableofcontents

\bigskip

\renewcommand{\thefigure}{\theenumi}
\renewcommand{\thetable}{\theenumi}
%\renewcommand{\theequation}{\theenumi}

%\begin{abstract}
%%\boldmath
%In this letter, an algorithm for evaluating the exact analytical bit error rate  (BER)  for the piecewise linear (PL) combiner for  multiple relays is presented. Previous results were available only for upto three relays. The algorithm is unique in the sense that  the actual mathematical expressions, that are prohibitively large, need not be explicitly obtained. The diversity gain due to multiple relays is shown through plots of the analytical BER, well supported by simulations. 
%
%\end{abstract}
% IEEEtran.cls defaults to using nonbold math in the Abstract.
% This preserves the distinction between vectors and scalars. However,
% if the journal you are submitting to favors bold math in the abstract,
% then you can use LaTeX's standard command \boldmath at the very start
% of the abstract to achieve this. Many IEEE journals frown on math
% in the abstract anyway.

% Note that keywords are not normally used for peerreview papers.
%\begin{IEEEkeywords}
%Cooperative diversity, decode and forward, piecewise linear
%\end{IEEEkeywords}



% For peer review papers, you can put extra information on the cover
% page as needed:
% \ifCLASSOPTIONpeerreview
% \begin{center} \bfseries EDICS Category: 3-BBND \end{center}
% \fi
%
% For peerreview papers, this IEEEtran command inserts a page break and
% creates the second title. It will be ignored for other modes.
%\IEEEpeerreviewmaketitle




\item One of the four persons John, Rita, Aslam or Gurpreet will be promoted next
month. Consequently the sample space consists of four elementary outcomes
S = {John promoted, Rita promoted, Aslam promoted, Gurpreet promoted}
You are told that the chances of John’s promotion is same as that of Gurpreet,
Rita’s chances of promotion are twice as likely as Johns. Aslam’s chances are
four times that of John.
\begin{enumerate}
	\item Determine
	\begin{enumerate}
		\item P (John promoted)
		\item P (Rita promoted)
		\item P (Aslam promoted)
		\item P (Gurpreet promoted)
	\end{enumerate}
	\item If A = {John promoted or Gurpreet promoted}, find P (A).
\end{enumerate}
\solution
%\begin{table}[H]
	\centering
\begin{tabular}{|c|c|c|}
\hline
Random variable &Value &Definition\\ \hline
\multirow{3}{*}{X} &0 &Slips of Rs 1\\
&1 &Slips of Rs 5\\
&2 &Slips of Rs 13\\ \hline
\multirow{2}{*}{Y} &0 &Box A\\
&1 &Box B\\\hline
\end{tabular}
\caption{}
\label{tab:Distribution}
\end{table}
See \tabref{tab:Distribution}.
\begin{align}
p_{Y}\brak{k}= \begin{cases} 
      \frac{1}{3} & {k=0} \\
      \frac{2}{3 }& {k=1} 
   \end{cases}
   \\
p_{Y|X}\brak{0|0} = \frac{19}{25}\, 
p_{Y|X}\brak{0|1} = \frac{6}{25}\,
p_{Y|X}\brak{1|0} = \frac{45}{50}\,
p_{Y|X}\brak{1|2} = \frac{5}{50}
\end{align}
The desired probability is the probability that a slip drawn at random is marked other than Rs 1,
\begin{align}
&=1-p_X\brak{0}\\
&= p_X(1) + p_X(2)
\end{align}
Using Bayes theorem,
\begin{align}
&= p_Y\brak{0} \times \pr{Y=0 | X=1} + p_Y\brak{1} \times \pr{Y=1|X=2}\\
&=\frac{1}{3} \times \frac{6}{25} + \frac{2}{3} \times \frac{5}{50}\\
&=\frac{11}{75}
\end{align}

\newpage

%\tableofcontents

\bigskip

\renewcommand{\thefigure}{\theenumi}
\renewcommand{\thetable}{\theenumi}
%\renewcommand{\theequation}{\theenumi}

%\begin{abstract}
%%\boldmath
%In this letter, an algorithm for evaluating the exact analytical bit error rate  (BER)  for the piecewise linear (PL) combiner for  multiple relays is presented. Previous results were available only for upto three relays. The algorithm is unique in the sense that  the actual mathematical expressions, that are prohibitively large, need not be explicitly obtained. The diversity gain due to multiple relays is shown through plots of the analytical BER, well supported by simulations. 
%
%\end{abstract}
% IEEEtran.cls defaults to using nonbold math in the Abstract.
% This preserves the distinction between vectors and scalars. However,
% if the journal you are submitting to favors bold math in the abstract,
% then you can use LaTeX's standard command \boldmath at the very start
% of the abstract to achieve this. Many IEEE journals frown on math
% in the abstract anyway.

% Note that keywords are not normally used for peerreview papers.
%\begin{IEEEkeywords}
%Cooperative diversity, decode and forward, piecewise linear
%\end{IEEEkeywords}



% For peer review papers, you can put extra information on the cover
% page as needed:
% \ifCLASSOPTIONpeerreview
% \begin{center} \bfseries EDICS Category: 3-BBND \end{center}
% \fi
%
% For peerreview papers, this IEEEtran command inserts a page break and
% creates the second title. It will be ignored for other modes.
%\IEEEpeerreviewmaketitle




\item A card is drawn from a deck of 52 cards. Find the probability of getting a king or a heart or a red card.\\
\solution
%\begin{table}[H]
	\centering
\begin{tabular}{|c|c|c|}
\hline
Random variable &Value &Definition\\ \hline
\multirow{3}{*}{X} &0 &Slips of Rs 1\\
&1 &Slips of Rs 5\\
&2 &Slips of Rs 13\\ \hline
\multirow{2}{*}{Y} &0 &Box A\\
&1 &Box B\\\hline
\end{tabular}
\caption{}
\label{tab:Distribution}
\end{table}
See \tabref{tab:Distribution}.
\begin{align}
p_{Y}\brak{k}= \begin{cases} 
      \frac{1}{3} & {k=0} \\
      \frac{2}{3 }& {k=1} 
   \end{cases}
   \\
p_{Y|X}\brak{0|0} = \frac{19}{25}\, 
p_{Y|X}\brak{0|1} = \frac{6}{25}\,
p_{Y|X}\brak{1|0} = \frac{45}{50}\,
p_{Y|X}\brak{1|2} = \frac{5}{50}
\end{align}
The desired probability is the probability that a slip drawn at random is marked other than Rs 1,
\begin{align}
&=1-p_X\brak{0}\\
&= p_X(1) + p_X(2)
\end{align}
Using Bayes theorem,
\begin{align}
&= p_Y\brak{0} \times \pr{Y=0 | X=1} + p_Y\brak{1} \times \pr{Y=1|X=2}\\
&=\frac{1}{3} \times \frac{6}{25} + \frac{2}{3} \times \frac{5}{50}\\
&=\frac{11}{75}
\end{align}

\newpage

%\tableofcontents

\bigskip

\renewcommand{\thefigure}{\theenumi}
\renewcommand{\thetable}{\theenumi}
%\renewcommand{\theequation}{\theenumi}

%\begin{abstract}
%%\boldmath
%In this letter, an algorithm for evaluating the exact analytical bit error rate  (BER)  for the piecewise linear (PL) combiner for  multiple relays is presented. Previous results were available only for upto three relays. The algorithm is unique in the sense that  the actual mathematical expressions, that are prohibitively large, need not be explicitly obtained. The diversity gain due to multiple relays is shown through plots of the analytical BER, well supported by simulations. 
%
%\end{abstract}
% IEEEtran.cls defaults to using nonbold math in the Abstract.
% This preserves the distinction between vectors and scalars. However,
% if the journal you are submitting to favors bold math in the abstract,
% then you can use LaTeX's standard command \boldmath at the very start
% of the abstract to achieve this. Many IEEE journals frown on math
% in the abstract anyway.

% Note that keywords are not normally used for peerreview papers.
%\begin{IEEEkeywords}
%Cooperative diversity, decode and forward, piecewise linear
%\end{IEEEkeywords}



% For peer review papers, you can put extra information on the cover
% page as needed:
% \ifCLASSOPTIONpeerreview
% \begin{center} \bfseries EDICS Category: 3-BBND \end{center}
% \fi
%
% For peerreview papers, this IEEEtran command inserts a page break and
% creates the second title. It will be ignored for other modes.
%\IEEEpeerreviewmaketitle




\item The probability that a student will pass his examination is 0.73, the probability of
the student getting a compartment is 0.13, and the probability that the student will
either pass or get compartment is 0.96. State True or False.\\
\solution
%\begin{table}[H]
	\centering
\begin{tabular}{|c|c|c|}
\hline
Random variable &Value &Definition\\ \hline
\multirow{3}{*}{X} &0 &Slips of Rs 1\\
&1 &Slips of Rs 5\\
&2 &Slips of Rs 13\\ \hline
\multirow{2}{*}{Y} &0 &Box A\\
&1 &Box B\\\hline
\end{tabular}
\caption{}
\label{tab:Distribution}
\end{table}
See \tabref{tab:Distribution}.
\begin{align}
p_{Y}\brak{k}= \begin{cases} 
      \frac{1}{3} & {k=0} \\
      \frac{2}{3 }& {k=1} 
   \end{cases}
   \\
p_{Y|X}\brak{0|0} = \frac{19}{25}\, 
p_{Y|X}\brak{0|1} = \frac{6}{25}\,
p_{Y|X}\brak{1|0} = \frac{45}{50}\,
p_{Y|X}\brak{1|2} = \frac{5}{50}
\end{align}
The desired probability is the probability that a slip drawn at random is marked other than Rs 1,
\begin{align}
&=1-p_X\brak{0}\\
&= p_X(1) + p_X(2)
\end{align}
Using Bayes theorem,
\begin{align}
&= p_Y\brak{0} \times \pr{Y=0 | X=1} + p_Y\brak{1} \times \pr{Y=1|X=2}\\
&=\frac{1}{3} \times \frac{6}{25} + \frac{2}{3} \times \frac{5}{50}\\
&=\frac{11}{75}
\end{align}

\newpage

%\tableofcontents

\bigskip

\renewcommand{\thefigure}{\theenumi}
\renewcommand{\thetable}{\theenumi}
%\renewcommand{\theequation}{\theenumi}

%\begin{abstract}
%%\boldmath
%In this letter, an algorithm for evaluating the exact analytical bit error rate  (BER)  for the piecewise linear (PL) combiner for  multiple relays is presented. Previous results were available only for upto three relays. The algorithm is unique in the sense that  the actual mathematical expressions, that are prohibitively large, need not be explicitly obtained. The diversity gain due to multiple relays is shown through plots of the analytical BER, well supported by simulations. 
%
%\end{abstract}
% IEEEtran.cls defaults to using nonbold math in the Abstract.
% This preserves the distinction between vectors and scalars. However,
% if the journal you are submitting to favors bold math in the abstract,
% then you can use LaTeX's standard command \boldmath at the very start
% of the abstract to achieve this. Many IEEE journals frown on math
% in the abstract anyway.

% Note that keywords are not normally used for peerreview papers.
%\begin{IEEEkeywords}
%Cooperative diversity, decode and forward, piecewise linear
%\end{IEEEkeywords}



% For peer review papers, you can put extra information on the cover
% page as needed:
% \ifCLASSOPTIONpeerreview
% \begin{center} \bfseries EDICS Category: 3-BBND \end{center}
% \fi
%
% For peerreview papers, this IEEEtran command inserts a page break and
% creates the second title. It will be ignored for other modes.
%\IEEEpeerreviewmaketitle




\item A card is selected from a pack of 52 cards\\
\begin{enumerate}[label=(\alph*)]
\item How many points are there in the sample space?
\item Calculate the probability that the cards is an ace of spades.
\item Calculate the probability that the card is (i) an ace (ii)black card.\\
\end{enumerate}
%\input{ncert/11/16/3/4_1/Prob_4.tex}
\item In a non-leap year, the probability of having 53 tuesdays or 53 wednesdays is\\
\solution
%A non-leap year has a total of 365 days, and a week has 7 days.\\
So it can be expressed as 
\begin{align}
365\text{days} &=52\times 7+1 \text{day}
\end{align}
$\implies$ 52 tuesdays or wednesdays\\
Random variable X denotes the days of a week
\begin{align}
p_X\brak{k}&=\frac{1}{7}; \quad \brak{1<k<7}
\end{align}
So the probability of extra day being tuesday or wednesday is
\begin{align}
p_X\brak{3}+p_X\brak{4}&=\frac{1}{7}+\frac{1}{7}=\frac{2}{7}
\end{align}



\item There are 1000 sealed envelopes in a box, 10 of them contain a cash prize of
Rs 100 each, 100 of them contain a cash prize of Rs 50 each and 200 of them
contain a cash prize of Rs 10 each and rest do not contain any cash prize. If they
are well shuffled and an envelope is picked up out, what is the probability that it
contains no cash prize?\\
\solution
%\begin{table}[H]
	\centering
\begin{tabular}{|c|c|c|}
\hline
Random variable &Value &Definition\\ \hline
\multirow{3}{*}{X} &0 &Slips of Rs 1\\
&1 &Slips of Rs 5\\
&2 &Slips of Rs 13\\ \hline
\multirow{2}{*}{Y} &0 &Box A\\
&1 &Box B\\\hline
\end{tabular}
\caption{}
\label{tab:Distribution}
\end{table}
See \tabref{tab:Distribution}.
\begin{align}
p_{Y}\brak{k}= \begin{cases} 
      \frac{1}{3} & {k=0} \\
      \frac{2}{3 }& {k=1} 
   \end{cases}
   \\
p_{Y|X}\brak{0|0} = \frac{19}{25}\, 
p_{Y|X}\brak{0|1} = \frac{6}{25}\,
p_{Y|X}\brak{1|0} = \frac{45}{50}\,
p_{Y|X}\brak{1|2} = \frac{5}{50}
\end{align}
The desired probability is the probability that a slip drawn at random is marked other than Rs 1,
\begin{align}
&=1-p_X\brak{0}\\
&= p_X(1) + p_X(2)
\end{align}
Using Bayes theorem,
\begin{align}
&= p_Y\brak{0} \times \pr{Y=0 | X=1} + p_Y\brak{1} \times \pr{Y=1|X=2}\\
&=\frac{1}{3} \times \frac{6}{25} + \frac{2}{3} \times \frac{5}{50}\\
&=\frac{11}{75}
\end{align}

\newpage

%\tableofcontents

\bigskip

\renewcommand{\thefigure}{\theenumi}
\renewcommand{\thetable}{\theenumi}
%\renewcommand{\theequation}{\theenumi}

%\begin{abstract}
%%\boldmath
%In this letter, an algorithm for evaluating the exact analytical bit error rate  (BER)  for the piecewise linear (PL) combiner for  multiple relays is presented. Previous results were available only for upto three relays. The algorithm is unique in the sense that  the actual mathematical expressions, that are prohibitively large, need not be explicitly obtained. The diversity gain due to multiple relays is shown through plots of the analytical BER, well supported by simulations. 
%
%\end{abstract}
% IEEEtran.cls defaults to using nonbold math in the Abstract.
% This preserves the distinction between vectors and scalars. However,
% if the journal you are submitting to favors bold math in the abstract,
% then you can use LaTeX's standard command \boldmath at the very start
% of the abstract to achieve this. Many IEEE journals frown on math
% in the abstract anyway.

% Note that keywords are not normally used for peerreview papers.
%\begin{IEEEkeywords}
%Cooperative diversity, decode and forward, piecewise linear
%\end{IEEEkeywords}



% For peer review papers, you can put extra information on the cover
% page as needed:
% \ifCLASSOPTIONpeerreview
% \begin{center} \bfseries EDICS Category: 3-BBND \end{center}
% \fi
%
% For peerreview papers, this IEEEtran command inserts a page break and
% creates the second title. It will be ignored for other modes.
%\IEEEpeerreviewmaketitle




\item 
A die is thrown and a card is selected at random from a deck of 52 playing cards. The probability of getting an even number on the die and a spade card.\\
\solution
%\begin{table}[H]
	\centering
\begin{tabular}{|c|c|c|}
\hline
Random variable &Value &Definition\\ \hline
\multirow{3}{*}{X} &0 &Slips of Rs 1\\
&1 &Slips of Rs 5\\
&2 &Slips of Rs 13\\ \hline
\multirow{2}{*}{Y} &0 &Box A\\
&1 &Box B\\\hline
\end{tabular}
\caption{}
\label{tab:Distribution}
\end{table}
See \tabref{tab:Distribution}.
\begin{align}
p_{Y}\brak{k}= \begin{cases} 
      \frac{1}{3} & {k=0} \\
      \frac{2}{3 }& {k=1} 
   \end{cases}
   \\
p_{Y|X}\brak{0|0} = \frac{19}{25}\, 
p_{Y|X}\brak{0|1} = \frac{6}{25}\,
p_{Y|X}\brak{1|0} = \frac{45}{50}\,
p_{Y|X}\brak{1|2} = \frac{5}{50}
\end{align}
The desired probability is the probability that a slip drawn at random is marked other than Rs 1,
\begin{align}
&=1-p_X\brak{0}\\
&= p_X(1) + p_X(2)
\end{align}
Using Bayes theorem,
\begin{align}
&= p_Y\brak{0} \times \pr{Y=0 | X=1} + p_Y\brak{1} \times \pr{Y=1|X=2}\\
&=\frac{1}{3} \times \frac{6}{25} + \frac{2}{3} \times \frac{5}{50}\\
&=\frac{11}{75}
\end{align}

\newpage

%\tableofcontents

\bigskip

\renewcommand{\thefigure}{\theenumi}
\renewcommand{\thetable}{\theenumi}
%\renewcommand{\theequation}{\theenumi}

%\begin{abstract}
%%\boldmath
%In this letter, an algorithm for evaluating the exact analytical bit error rate  (BER)  for the piecewise linear (PL) combiner for  multiple relays is presented. Previous results were available only for upto three relays. The algorithm is unique in the sense that  the actual mathematical expressions, that are prohibitively large, need not be explicitly obtained. The diversity gain due to multiple relays is shown through plots of the analytical BER, well supported by simulations. 
%
%\end{abstract}
% IEEEtran.cls defaults to using nonbold math in the Abstract.
% This preserves the distinction between vectors and scalars. However,
% if the journal you are submitting to favors bold math in the abstract,
% then you can use LaTeX's standard command \boldmath at the very start
% of the abstract to achieve this. Many IEEE journals frown on math
% in the abstract anyway.

% Note that keywords are not normally used for peerreview papers.
%\begin{IEEEkeywords}
%Cooperative diversity, decode and forward, piecewise linear
%\end{IEEEkeywords}



% For peer review papers, you can put extra information on the cover
% page as needed:
% \ifCLASSOPTIONpeerreview
% \begin{center} \bfseries EDICS Category: 3-BBND \end{center}
% \fi
%
% For peerreview papers, this IEEEtran command inserts a page break and
% creates the second title. It will be ignored for other modes.
%\IEEEpeerreviewmaketitle




\item
If 4-digit numbers greater than 5,000 are randomly formed from the digits 0, 1, 3, 5, and 7, what is the probability of forming a number divisible by 5 when:
\begin{enumerate}
    \item The digits are repeated?
    \item The repetition of digits is not allowed?
\end{enumerate}
\solution
%\begin{table}[H]
	\centering
\begin{tabular}{|c|c|c|}
\hline
Random variable &Value &Definition\\ \hline
\multirow{3}{*}{X} &0 &Slips of Rs 1\\
&1 &Slips of Rs 5\\
&2 &Slips of Rs 13\\ \hline
\multirow{2}{*}{Y} &0 &Box A\\
&1 &Box B\\\hline
\end{tabular}
\caption{}
\label{tab:Distribution}
\end{table}
See \tabref{tab:Distribution}.
\begin{align}
p_{Y}\brak{k}= \begin{cases} 
      \frac{1}{3} & {k=0} \\
      \frac{2}{3 }& {k=1} 
   \end{cases}
   \\
p_{Y|X}\brak{0|0} = \frac{19}{25}\, 
p_{Y|X}\brak{0|1} = \frac{6}{25}\,
p_{Y|X}\brak{1|0} = \frac{45}{50}\,
p_{Y|X}\brak{1|2} = \frac{5}{50}
\end{align}
The desired probability is the probability that a slip drawn at random is marked other than Rs 1,
\begin{align}
&=1-p_X\brak{0}\\
&= p_X(1) + p_X(2)
\end{align}
Using Bayes theorem,
\begin{align}
&= p_Y\brak{0} \times \pr{Y=0 | X=1} + p_Y\brak{1} \times \pr{Y=1|X=2}\\
&=\frac{1}{3} \times \frac{6}{25} + \frac{2}{3} \times \frac{5}{50}\\
&=\frac{11}{75}
\end{align}

\newpage

%\tableofcontents

\bigskip

\renewcommand{\thefigure}{\theenumi}
\renewcommand{\thetable}{\theenumi}
%\renewcommand{\theequation}{\theenumi}

%\begin{abstract}
%%\boldmath
%In this letter, an algorithm for evaluating the exact analytical bit error rate  (BER)  for the piecewise linear (PL) combiner for  multiple relays is presented. Previous results were available only for upto three relays. The algorithm is unique in the sense that  the actual mathematical expressions, that are prohibitively large, need not be explicitly obtained. The diversity gain due to multiple relays is shown through plots of the analytical BER, well supported by simulations. 
%
%\end{abstract}
% IEEEtran.cls defaults to using nonbold math in the Abstract.
% This preserves the distinction between vectors and scalars. However,
% if the journal you are submitting to favors bold math in the abstract,
% then you can use LaTeX's standard command \boldmath at the very start
% of the abstract to achieve this. Many IEEE journals frown on math
% in the abstract anyway.

% Note that keywords are not normally used for peerreview papers.
%\begin{IEEEkeywords}
%Cooperative diversity, decode and forward, piecewise linear
%\end{IEEEkeywords}



% For peer review papers, you can put extra information on the cover
% page as needed:
% \ifCLASSOPTIONpeerreview
% \begin{center} \bfseries EDICS Category: 3-BBND \end{center}
% \fi
%
% For peerreview papers, this IEEEtran command inserts a page break and
% creates the second title. It will be ignored for other modes.
%\IEEEpeerreviewmaketitle




\item Consider the probability space $\brak{\Omega, \mathcal{G}, P}$ where $\Omega = [0,2]$ and $\mathcal{G} = \cbrak{\phi, \Omega, [0,1], (1,2]}$. Let $X$ and $Y$ be two functions on $\Omega$ defined as
\begin{align*}
    X(\omega) = 
    \begin{cases}
        1 & \text{if }\omega \in [0, 1]\\
        2 & \text{if }\omega \in (1, 2]
    \end{cases}
\end{align*}
and
\begin{align*}
    Y(\omega) = 
    \begin{cases}
        2 & \text{if }\omega \in [0, 1.5]\\
        3 & \text{if }\omega \in (1.5, 2].
    \end{cases}
\end{align*}
Then which one of the following statements is true?
\begin{enumerate}
    \item [(A)] $X$ is a random variable with respect to $\mathcal{G}$, but $Y$ is not a random variable with respect to $\mathcal{G}$.
    \item [(B)] $Y$ is a random variable with respect to $\mathcal{G}$, but $X$ is not a random variable with respect to $\mathcal{G}$.
    \item [(C)] Neither $X$ nor $Y$ is a random variable with respect to $\mathcal{G}$.
    \item [(D)] Both $X$ and $Y$ are random variables with respect to $\mathcal{G}$.
\end{enumerate} \hfill (GATE ST 2023)\\
\solution
%\begin{table}[H]
	\centering
\begin{tabular}{|c|c|c|}
\hline
Random variable &Value &Definition\\ \hline
\multirow{3}{*}{X} &0 &Slips of Rs 1\\
&1 &Slips of Rs 5\\
&2 &Slips of Rs 13\\ \hline
\multirow{2}{*}{Y} &0 &Box A\\
&1 &Box B\\\hline
\end{tabular}
\caption{}
\label{tab:Distribution}
\end{table}
See \tabref{tab:Distribution}.
\begin{align}
p_{Y}\brak{k}= \begin{cases} 
      \frac{1}{3} & {k=0} \\
      \frac{2}{3 }& {k=1} 
   \end{cases}
   \\
p_{Y|X}\brak{0|0} = \frac{19}{25}\, 
p_{Y|X}\brak{0|1} = \frac{6}{25}\,
p_{Y|X}\brak{1|0} = \frac{45}{50}\,
p_{Y|X}\brak{1|2} = \frac{5}{50}
\end{align}
The desired probability is the probability that a slip drawn at random is marked other than Rs 1,
\begin{align}
&=1-p_X\brak{0}\\
&= p_X(1) + p_X(2)
\end{align}
Using Bayes theorem,
\begin{align}
&= p_Y\brak{0} \times \pr{Y=0 | X=1} + p_Y\brak{1} \times \pr{Y=1|X=2}\\
&=\frac{1}{3} \times \frac{6}{25} + \frac{2}{3} \times \frac{5}{50}\\
&=\frac{11}{75}
\end{align}

\newpage

%\tableofcontents

\bigskip

\renewcommand{\thefigure}{\theenumi}
\renewcommand{\thetable}{\theenumi}
%\renewcommand{\theequation}{\theenumi}

%\begin{abstract}
%%\boldmath
%In this letter, an algorithm for evaluating the exact analytical bit error rate  (BER)  for the piecewise linear (PL) combiner for  multiple relays is presented. Previous results were available only for upto three relays. The algorithm is unique in the sense that  the actual mathematical expressions, that are prohibitively large, need not be explicitly obtained. The diversity gain due to multiple relays is shown through plots of the analytical BER, well supported by simulations. 
%
%\end{abstract}
% IEEEtran.cls defaults to using nonbold math in the Abstract.
% This preserves the distinction between vectors and scalars. However,
% if the journal you are submitting to favors bold math in the abstract,
% then you can use LaTeX's standard command \boldmath at the very start
% of the abstract to achieve this. Many IEEE journals frown on math
% in the abstract anyway.

% Note that keywords are not normally used for peerreview papers.
%\begin{IEEEkeywords}
%Cooperative diversity, decode and forward, piecewise linear
%\end{IEEEkeywords}



% For peer review papers, you can put extra information on the cover
% page as needed:
% \ifCLASSOPTIONpeerreview
% \begin{center} \bfseries EDICS Category: 3-BBND \end{center}
% \fi
%
% For peerreview papers, this IEEEtran command inserts a page break and
% creates the second title. It will be ignored for other modes.
%\IEEEpeerreviewmaketitle




	\item  A die is loaded in such a way that each odd number is twice as likely to occur as
each even number. Find $P(G)$, where $G$ is the event that a number greater than
3 occurs on a single roll of the die.
\\
\solution
		%\begin{table}[H]
	\centering
\begin{tabular}{|c|c|c|}
\hline
Random variable &Value &Definition\\ \hline
\multirow{3}{*}{X} &0 &Slips of Rs 1\\
&1 &Slips of Rs 5\\
&2 &Slips of Rs 13\\ \hline
\multirow{2}{*}{Y} &0 &Box A\\
&1 &Box B\\\hline
\end{tabular}
\caption{}
\label{tab:Distribution}
\end{table}
See \tabref{tab:Distribution}.
\begin{align}
p_{Y}\brak{k}= \begin{cases} 
      \frac{1}{3} & {k=0} \\
      \frac{2}{3 }& {k=1} 
   \end{cases}
   \\
p_{Y|X}\brak{0|0} = \frac{19}{25}\, 
p_{Y|X}\brak{0|1} = \frac{6}{25}\,
p_{Y|X}\brak{1|0} = \frac{45}{50}\,
p_{Y|X}\brak{1|2} = \frac{5}{50}
\end{align}
The desired probability is the probability that a slip drawn at random is marked other than Rs 1,
\begin{align}
&=1-p_X\brak{0}\\
&= p_X(1) + p_X(2)
\end{align}
Using Bayes theorem,
\begin{align}
&= p_Y\brak{0} \times \pr{Y=0 | X=1} + p_Y\brak{1} \times \pr{Y=1|X=2}\\
&=\frac{1}{3} \times \frac{6}{25} + \frac{2}{3} \times \frac{5}{50}\\
&=\frac{11}{75}
\end{align}

\newpage

%\tableofcontents

\bigskip

\renewcommand{\thefigure}{\theenumi}
\renewcommand{\thetable}{\theenumi}
%\renewcommand{\theequation}{\theenumi}

%\begin{abstract}
%%\boldmath
%In this letter, an algorithm for evaluating the exact analytical bit error rate  (BER)  for the piecewise linear (PL) combiner for  multiple relays is presented. Previous results were available only for upto three relays. The algorithm is unique in the sense that  the actual mathematical expressions, that are prohibitively large, need not be explicitly obtained. The diversity gain due to multiple relays is shown through plots of the analytical BER, well supported by simulations. 
%
%\end{abstract}
% IEEEtran.cls defaults to using nonbold math in the Abstract.
% This preserves the distinction between vectors and scalars. However,
% if the journal you are submitting to favors bold math in the abstract,
% then you can use LaTeX's standard command \boldmath at the very start
% of the abstract to achieve this. Many IEEE journals frown on math
% in the abstract anyway.

% Note that keywords are not normally used for peerreview papers.
%\begin{IEEEkeywords}
%Cooperative diversity, decode and forward, piecewise linear
%\end{IEEEkeywords}



% For peer review papers, you can put extra information on the cover
% page as needed:
% \ifCLASSOPTIONpeerreview
% \begin{center} \bfseries EDICS Category: 3-BBND \end{center}
% \fi
%
% For peerreview papers, this IEEEtran command inserts a page break and
% creates the second title. It will be ignored for other modes.
%\IEEEpeerreviewmaketitle




	\item All the jacks, queens and kings are removed from a deck of 52 playing cards. The remaining cards are well shuffled and then one card is drawn at random. Giving ace a value 1 similar value for other cards, find the probability that the card has a value 
		\begin{enumerate}
			\item 7
			\item greater than 7
			\item less than 7
		\end{enumerate}
		%Number of cards left after removing all jacks, queens and kings 
\begin{align}
N	= 52 - 4\times 3
	= 40
\end{align}
%\begin{table}[H]
%\def\arraystretch{1.2}
%\begin{tabular}{|c|c|c|}
%\hline
%	\textbf{Parameter} &\textbf{Value} &\textbf{Description}\\ \hline
%	$X$ &1-10 &Represents the value of the card picked \\ \hline
%\end{tabular}
%\end{table}
Let $1 \le X \le 10$ be the value of the card picked.  Then,
\begin{align}
	p_X(k) &= \Pr(X=k)\ \forall\ 1 \leq k \leq 10\\
	&= \frac{4\times 1}{40}\\
	&= \frac{1}{10}\\
	\therefore p_X(k) &= 
	\begin{cases}
		\frac{1}{10} & 1 \leq k \leq 10\\
		0 & \text{otherwise}
	\end{cases}
\end{align}
and
\begin{align}
	F_{X}(k) &= \sum_{m=0}^{k}p_{X}(m) \quad 1 \leq k \leq 10\\
	&= \frac{k}{10}\\
	\therefore F_{X}(k) &= 
	\begin{cases}
		0 & k \leq 0\\
		\frac{k}{10} & 1\leq k \leq 10\\
		1 & k > 10 
	\end{cases}
\end{align}
\begin{enumerate}
	\item Probability that card has value equal to 7 is
		\begin{align}
			 p_{X}(7)
			= \frac{1}{10}
		\end{align}
	\item Probability that card has value greater than 7 is
		\begin{align}
			1 - F_X(7)
			&= 1 - \frac{7}{10}
			\\
			&= \frac{3}{10}
		\end{align}
	\item Probability that card has value less than 7 is
		\begin{align}
			 F_{X}(6)
			=\frac{6}{10}
		\end{align}
\end{enumerate}

  \item A Lot consists of 48 mobile phones of which 42 are good, 3 have only minor defects and 3 have major defects.Varnika will buy a phone if it is good but the trader will only buy a mobile if it has no major defects. One phone is selected at random from the lot. What is the probability that it is
\begin{enumerate}
	\item acceptable to Varnika?
            \item acceptable to the trader?
\end{enumerate}
\solution
	%\begin{table}[H]
	\centering
\begin{tabular}{|c|c|c|}
\hline
Random variable &Value &Definition\\ \hline
\multirow{3}{*}{X} &0 &Slips of Rs 1\\
&1 &Slips of Rs 5\\
&2 &Slips of Rs 13\\ \hline
\multirow{2}{*}{Y} &0 &Box A\\
&1 &Box B\\\hline
\end{tabular}
\caption{}
\label{tab:Distribution}
\end{table}
See \tabref{tab:Distribution}.
\begin{align}
p_{Y}\brak{k}= \begin{cases} 
      \frac{1}{3} & {k=0} \\
      \frac{2}{3 }& {k=1} 
   \end{cases}
   \\
p_{Y|X}\brak{0|0} = \frac{19}{25}\, 
p_{Y|X}\brak{0|1} = \frac{6}{25}\,
p_{Y|X}\brak{1|0} = \frac{45}{50}\,
p_{Y|X}\brak{1|2} = \frac{5}{50}
\end{align}
The desired probability is the probability that a slip drawn at random is marked other than Rs 1,
\begin{align}
&=1-p_X\brak{0}\\
&= p_X(1) + p_X(2)
\end{align}
Using Bayes theorem,
\begin{align}
&= p_Y\brak{0} \times \pr{Y=0 | X=1} + p_Y\brak{1} \times \pr{Y=1|X=2}\\
&=\frac{1}{3} \times \frac{6}{25} + \frac{2}{3} \times \frac{5}{50}\\
&=\frac{11}{75}
\end{align}

\newpage

%\tableofcontents

\bigskip

\renewcommand{\thefigure}{\theenumi}
\renewcommand{\thetable}{\theenumi}
%\renewcommand{\theequation}{\theenumi}

%\begin{abstract}
%%\boldmath
%In this letter, an algorithm for evaluating the exact analytical bit error rate  (BER)  for the piecewise linear (PL) combiner for  multiple relays is presented. Previous results were available only for upto three relays. The algorithm is unique in the sense that  the actual mathematical expressions, that are prohibitively large, need not be explicitly obtained. The diversity gain due to multiple relays is shown through plots of the analytical BER, well supported by simulations. 
%
%\end{abstract}
% IEEEtran.cls defaults to using nonbold math in the Abstract.
% This preserves the distinction between vectors and scalars. However,
% if the journal you are submitting to favors bold math in the abstract,
% then you can use LaTeX's standard command \boldmath at the very start
% of the abstract to achieve this. Many IEEE journals frown on math
% in the abstract anyway.

% Note that keywords are not normally used for peerreview papers.
%\begin{IEEEkeywords}
%Cooperative diversity, decode and forward, piecewise linear
%\end{IEEEkeywords}



% For peer review papers, you can put extra information on the cover
% page as needed:
% \ifCLASSOPTIONpeerreview
% \begin{center} \bfseries EDICS Category: 3-BBND \end{center}
% \fi
%
% For peerreview papers, this IEEEtran command inserts a page break and
% creates the second title. It will be ignored for other modes.
%\IEEEpeerreviewmaketitle




 \item A student says that if you throw a die, it will show up 1 or not 1. Therefore, the probability of getting 1 and the probability of getting 'not 1' each is equal to $\frac{1}{2}$. Is this correct? Give reasons.\\
 \solution
        %\begin{table}[H]
	\centering
\begin{tabular}{|c|c|c|}
\hline
Random variable &Value &Definition\\ \hline
\multirow{3}{*}{X} &0 &Slips of Rs 1\\
&1 &Slips of Rs 5\\
&2 &Slips of Rs 13\\ \hline
\multirow{2}{*}{Y} &0 &Box A\\
&1 &Box B\\\hline
\end{tabular}
\caption{}
\label{tab:Distribution}
\end{table}
See \tabref{tab:Distribution}.
\begin{align}
p_{Y}\brak{k}= \begin{cases} 
      \frac{1}{3} & {k=0} \\
      \frac{2}{3 }& {k=1} 
   \end{cases}
   \\
p_{Y|X}\brak{0|0} = \frac{19}{25}\, 
p_{Y|X}\brak{0|1} = \frac{6}{25}\,
p_{Y|X}\brak{1|0} = \frac{45}{50}\,
p_{Y|X}\brak{1|2} = \frac{5}{50}
\end{align}
The desired probability is the probability that a slip drawn at random is marked other than Rs 1,
\begin{align}
&=1-p_X\brak{0}\\
&= p_X(1) + p_X(2)
\end{align}
Using Bayes theorem,
\begin{align}
&= p_Y\brak{0} \times \pr{Y=0 | X=1} + p_Y\brak{1} \times \pr{Y=1|X=2}\\
&=\frac{1}{3} \times \frac{6}{25} + \frac{2}{3} \times \frac{5}{50}\\
&=\frac{11}{75}
\end{align}

\newpage

%\tableofcontents

\bigskip

\renewcommand{\thefigure}{\theenumi}
\renewcommand{\thetable}{\theenumi}
%\renewcommand{\theequation}{\theenumi}

%\begin{abstract}
%%\boldmath
%In this letter, an algorithm for evaluating the exact analytical bit error rate  (BER)  for the piecewise linear (PL) combiner for  multiple relays is presented. Previous results were available only for upto three relays. The algorithm is unique in the sense that  the actual mathematical expressions, that are prohibitively large, need not be explicitly obtained. The diversity gain due to multiple relays is shown through plots of the analytical BER, well supported by simulations. 
%
%\end{abstract}
% IEEEtran.cls defaults to using nonbold math in the Abstract.
% This preserves the distinction between vectors and scalars. However,
% if the journal you are submitting to favors bold math in the abstract,
% then you can use LaTeX's standard command \boldmath at the very start
% of the abstract to achieve this. Many IEEE journals frown on math
% in the abstract anyway.

% Note that keywords are not normally used for peerreview papers.
%\begin{IEEEkeywords}
%Cooperative diversity, decode and forward, piecewise linear
%\end{IEEEkeywords}



% For peer review papers, you can put extra information on the cover
% page as needed:
% \ifCLASSOPTIONpeerreview
% \begin{center} \bfseries EDICS Category: 3-BBND \end{center}
% \fi
%
% For peerreview papers, this IEEEtran command inserts a page break and
% creates the second title. It will be ignored for other modes.
%\IEEEpeerreviewmaketitle




   \item Four candidates A, B, C, D have ap-
plied for the assignment to coach a school cricket
team. If A is twice as likely to be selected as B, and
B and C are given about the same chance of being
selected, while C is twice as likely to be selected
as D, what are the probabilities that
\begin{enumerate}
\item C will be selected?
\item A will not be selected?
\end{enumerate}
	%\begin{table}[H]
	\centering
\begin{tabular}{|c|c|c|}
\hline
Random variable &Value &Definition\\ \hline
\multirow{3}{*}{X} &0 &Slips of Rs 1\\
&1 &Slips of Rs 5\\
&2 &Slips of Rs 13\\ \hline
\multirow{2}{*}{Y} &0 &Box A\\
&1 &Box B\\\hline
\end{tabular}
\caption{}
\label{tab:Distribution}
\end{table}
See \tabref{tab:Distribution}.
\begin{align}
p_{Y}\brak{k}= \begin{cases} 
      \frac{1}{3} & {k=0} \\
      \frac{2}{3 }& {k=1} 
   \end{cases}
   \\
p_{Y|X}\brak{0|0} = \frac{19}{25}\, 
p_{Y|X}\brak{0|1} = \frac{6}{25}\,
p_{Y|X}\brak{1|0} = \frac{45}{50}\,
p_{Y|X}\brak{1|2} = \frac{5}{50}
\end{align}
The desired probability is the probability that a slip drawn at random is marked other than Rs 1,
\begin{align}
&=1-p_X\brak{0}\\
&= p_X(1) + p_X(2)
\end{align}
Using Bayes theorem,
\begin{align}
&= p_Y\brak{0} \times \pr{Y=0 | X=1} + p_Y\brak{1} \times \pr{Y=1|X=2}\\
&=\frac{1}{3} \times \frac{6}{25} + \frac{2}{3} \times \frac{5}{50}\\
&=\frac{11}{75}
\end{align}

\newpage

%\tableofcontents

\bigskip

\renewcommand{\thefigure}{\theenumi}
\renewcommand{\thetable}{\theenumi}
%\renewcommand{\theequation}{\theenumi}

%\begin{abstract}
%%\boldmath
%In this letter, an algorithm for evaluating the exact analytical bit error rate  (BER)  for the piecewise linear (PL) combiner for  multiple relays is presented. Previous results were available only for upto three relays. The algorithm is unique in the sense that  the actual mathematical expressions, that are prohibitively large, need not be explicitly obtained. The diversity gain due to multiple relays is shown through plots of the analytical BER, well supported by simulations. 
%
%\end{abstract}
% IEEEtran.cls defaults to using nonbold math in the Abstract.
% This preserves the distinction between vectors and scalars. However,
% if the journal you are submitting to favors bold math in the abstract,
% then you can use LaTeX's standard command \boldmath at the very start
% of the abstract to achieve this. Many IEEE journals frown on math
% in the abstract anyway.

% Note that keywords are not normally used for peerreview papers.
%\begin{IEEEkeywords}
%Cooperative diversity, decode and forward, piecewise linear
%\end{IEEEkeywords}



% For peer review papers, you can put extra information on the cover
% page as needed:
% \ifCLASSOPTIONpeerreview
% \begin{center} \bfseries EDICS Category: 3-BBND \end{center}
% \fi
%
% For peerreview papers, this IEEEtran command inserts a page break and
% creates the second title. It will be ignored for other modes.
%\IEEEpeerreviewmaketitle




 \item A bag contain 24 balls of which $x$ balls are red, $2x$ are white and $3x$ are blue. A ball is selected at random, What is the probability that it is
\begin{enumerate}[label=\alph*)]
\item not red ?
\item white ?
\end{enumerate}
%\begin{table}[H]
	\centering
\begin{tabular}{|c|c|c|}
\hline
Random variable &Value &Definition\\ \hline
\multirow{3}{*}{X} &0 &Slips of Rs 1\\
&1 &Slips of Rs 5\\
&2 &Slips of Rs 13\\ \hline
\multirow{2}{*}{Y} &0 &Box A\\
&1 &Box B\\\hline
\end{tabular}
\caption{}
\label{tab:Distribution}
\end{table}
See \tabref{tab:Distribution}.
\begin{align}
p_{Y}\brak{k}= \begin{cases} 
      \frac{1}{3} & {k=0} \\
      \frac{2}{3 }& {k=1} 
   \end{cases}
   \\
p_{Y|X}\brak{0|0} = \frac{19}{25}\, 
p_{Y|X}\brak{0|1} = \frac{6}{25}\,
p_{Y|X}\brak{1|0} = \frac{45}{50}\,
p_{Y|X}\brak{1|2} = \frac{5}{50}
\end{align}
The desired probability is the probability that a slip drawn at random is marked other than Rs 1,
\begin{align}
&=1-p_X\brak{0}\\
&= p_X(1) + p_X(2)
\end{align}
Using Bayes theorem,
\begin{align}
&= p_Y\brak{0} \times \pr{Y=0 | X=1} + p_Y\brak{1} \times \pr{Y=1|X=2}\\
&=\frac{1}{3} \times \frac{6}{25} + \frac{2}{3} \times \frac{5}{50}\\
&=\frac{11}{75}
\end{align}

\newpage

%\tableofcontents

\bigskip

\renewcommand{\thefigure}{\theenumi}
\renewcommand{\thetable}{\theenumi}
%\renewcommand{\theequation}{\theenumi}

%\begin{abstract}
%%\boldmath
%In this letter, an algorithm for evaluating the exact analytical bit error rate  (BER)  for the piecewise linear (PL) combiner for  multiple relays is presented. Previous results were available only for upto three relays. The algorithm is unique in the sense that  the actual mathematical expressions, that are prohibitively large, need not be explicitly obtained. The diversity gain due to multiple relays is shown through plots of the analytical BER, well supported by simulations. 
%
%\end{abstract}
% IEEEtran.cls defaults to using nonbold math in the Abstract.
% This preserves the distinction between vectors and scalars. However,
% if the journal you are submitting to favors bold math in the abstract,
% then you can use LaTeX's standard command \boldmath at the very start
% of the abstract to achieve this. Many IEEE journals frown on math
% in the abstract anyway.

% Note that keywords are not normally used for peerreview papers.
%\begin{IEEEkeywords}
%Cooperative diversity, decode and forward, piecewise linear
%\end{IEEEkeywords}



% For peer review papers, you can put extra information on the cover
% page as needed:
% \ifCLASSOPTIONpeerreview
% \begin{center} \bfseries EDICS Category: 3-BBND \end{center}
% \fi
%
% For peerreview papers, this IEEEtran command inserts a page break and
% creates the second title. It will be ignored for other modes.
%\IEEEpeerreviewmaketitle




If the letters of the word ASSASSINATION are arranged at random. Find the Probability that
\begin{enumerate}[label=(\alph*)]
\item Four $S's$ come consecutively in the word
\item Two  $I's$ and two $N's$ come together
\item All $A's$ are not coming together
\item No two $A's$ are coming together
\end{enumerate}
%\begin{table}[H]
	\centering
\begin{tabular}{|c|c|c|}
\hline
Random variable &Value &Definition\\ \hline
\multirow{3}{*}{X} &0 &Slips of Rs 1\\
&1 &Slips of Rs 5\\
&2 &Slips of Rs 13\\ \hline
\multirow{2}{*}{Y} &0 &Box A\\
&1 &Box B\\\hline
\end{tabular}
\caption{}
\label{tab:Distribution}
\end{table}
See \tabref{tab:Distribution}.
\begin{align}
p_{Y}\brak{k}= \begin{cases} 
      \frac{1}{3} & {k=0} \\
      \frac{2}{3 }& {k=1} 
   \end{cases}
   \\
p_{Y|X}\brak{0|0} = \frac{19}{25}\, 
p_{Y|X}\brak{0|1} = \frac{6}{25}\,
p_{Y|X}\brak{1|0} = \frac{45}{50}\,
p_{Y|X}\brak{1|2} = \frac{5}{50}
\end{align}
The desired probability is the probability that a slip drawn at random is marked other than Rs 1,
\begin{align}
&=1-p_X\brak{0}\\
&= p_X(1) + p_X(2)
\end{align}
Using Bayes theorem,
\begin{align}
&= p_Y\brak{0} \times \pr{Y=0 | X=1} + p_Y\brak{1} \times \pr{Y=1|X=2}\\
&=\frac{1}{3} \times \frac{6}{25} + \frac{2}{3} \times \frac{5}{50}\\
&=\frac{11}{75}
\end{align}

\newpage

%\tableofcontents

\bigskip

\renewcommand{\thefigure}{\theenumi}
\renewcommand{\thetable}{\theenumi}
%\renewcommand{\theequation}{\theenumi}

%\begin{abstract}
%%\boldmath
%In this letter, an algorithm for evaluating the exact analytical bit error rate  (BER)  for the piecewise linear (PL) combiner for  multiple relays is presented. Previous results were available only for upto three relays. The algorithm is unique in the sense that  the actual mathematical expressions, that are prohibitively large, need not be explicitly obtained. The diversity gain due to multiple relays is shown through plots of the analytical BER, well supported by simulations. 
%
%\end{abstract}
% IEEEtran.cls defaults to using nonbold math in the Abstract.
% This preserves the distinction between vectors and scalars. However,
% if the journal you are submitting to favors bold math in the abstract,
% then you can use LaTeX's standard command \boldmath at the very start
% of the abstract to achieve this. Many IEEE journals frown on math
% in the abstract anyway.

% Note that keywords are not normally used for peerreview papers.
%\begin{IEEEkeywords}
%Cooperative diversity, decode and forward, piecewise linear
%\end{IEEEkeywords}



% For peer review papers, you can put extra information on the cover
% page as needed:
% \ifCLASSOPTIONpeerreview
% \begin{center} \bfseries EDICS Category: 3-BBND \end{center}
% \fi
%
% For peerreview papers, this IEEEtran command inserts a page break and
% creates the second title. It will be ignored for other modes.
%\IEEEpeerreviewmaketitle




	\item One urn contains two black balls (labelled B1 and B2) and one white ball. A
	second urn contains one black ball and two white balls (labelled W1 and W2).
	Suppose the following experiment is performed. One of the two urns is chosen
	at random. Next a ball is randomly chosen from the urn. Then a second ball is
	chosen at random from the same urn without replacing the first ball.
	
	\begin{enumerate}
	\item What is the probability that two black balls are chosen?
	
	\item What is the probability that two balls of opposite colour are chosen?
	\end{enumerate}
	\solution
	%\begin{align}
    \label{eq:12.13.6.18.1}
	\because	\pr{A|B} &> \pr{A},\
\frac{\pr{AB}}{\pr{B}} > \pr{A}
\\
    \label{eq:12.13.6.18.2}
	\implies \pr{AB} &> \pr{A}\pr{B}
	\\
	\text{or, } \frac{\pr{AB}}{\pr{A}} &=\pr{B|A} > \pr{A}
\end{align}

\end{enumerate}

\item In a certain lottery 10,000 tickets are sold and ten equal prizes are awarded. What is the probability of not getting a prize if you buy (a) one ticket (b) two tickets (c) 10 tickets ?	
\\
\solution
		%\begin{enumerate}[label=\thesection.\arabic*,ref=\thesection.\theenumi]
	\item One card is drawn from a well-shuffled deck of 52 cards. Find the probability of getting
\begin{enumerate}
\item A king of red colour 
\item A face card 
\item A red face card
\item The jack of hearts
\item A spade
\item The queen of diamonds

\end{enumerate}
\solution
		%\begin{table}[H]
	\centering
\begin{tabular}{|c|c|c|}
\hline
Random variable &Value &Definition\\ \hline
\multirow{3}{*}{X} &0 &Slips of Rs 1\\
&1 &Slips of Rs 5\\
&2 &Slips of Rs 13\\ \hline
\multirow{2}{*}{Y} &0 &Box A\\
&1 &Box B\\\hline
\end{tabular}
\caption{}
\label{tab:Distribution}
\end{table}
See \tabref{tab:Distribution}.
\begin{align}
p_{Y}\brak{k}= \begin{cases} 
      \frac{1}{3} & {k=0} \\
      \frac{2}{3 }& {k=1} 
   \end{cases}
   \\
p_{Y|X}\brak{0|0} = \frac{19}{25}\, 
p_{Y|X}\brak{0|1} = \frac{6}{25}\,
p_{Y|X}\brak{1|0} = \frac{45}{50}\,
p_{Y|X}\brak{1|2} = \frac{5}{50}
\end{align}
The desired probability is the probability that a slip drawn at random is marked other than Rs 1,
\begin{align}
&=1-p_X\brak{0}\\
&= p_X(1) + p_X(2)
\end{align}
Using Bayes theorem,
\begin{align}
&= p_Y\brak{0} \times \pr{Y=0 | X=1} + p_Y\brak{1} \times \pr{Y=1|X=2}\\
&=\frac{1}{3} \times \frac{6}{25} + \frac{2}{3} \times \frac{5}{50}\\
&=\frac{11}{75}
\end{align}

\newpage

%\tableofcontents

\bigskip

\renewcommand{\thefigure}{\theenumi}
\renewcommand{\thetable}{\theenumi}
%\renewcommand{\theequation}{\theenumi}

%\begin{abstract}
%%\boldmath
%In this letter, an algorithm for evaluating the exact analytical bit error rate  (BER)  for the piecewise linear (PL) combiner for  multiple relays is presented. Previous results were available only for upto three relays. The algorithm is unique in the sense that  the actual mathematical expressions, that are prohibitively large, need not be explicitly obtained. The diversity gain due to multiple relays is shown through plots of the analytical BER, well supported by simulations. 
%
%\end{abstract}
% IEEEtran.cls defaults to using nonbold math in the Abstract.
% This preserves the distinction between vectors and scalars. However,
% if the journal you are submitting to favors bold math in the abstract,
% then you can use LaTeX's standard command \boldmath at the very start
% of the abstract to achieve this. Many IEEE journals frown on math
% in the abstract anyway.

% Note that keywords are not normally used for peerreview papers.
%\begin{IEEEkeywords}
%Cooperative diversity, decode and forward, piecewise linear
%\end{IEEEkeywords}



% For peer review papers, you can put extra information on the cover
% page as needed:
% \ifCLASSOPTIONpeerreview
% \begin{center} \bfseries EDICS Category: 3-BBND \end{center}
% \fi
%
% For peerreview papers, this IEEEtran command inserts a page break and
% creates the second title. It will be ignored for other modes.
%\IEEEpeerreviewmaketitle




	\item Five cards—the ten, jack, queen, king and ace of diamonds, are well-shuffled with their face downwards. One card is then picked up at random.
\begin{enumerate}
\item
What is the probability that the card is the queen? 
\item
If the queen is drawn and put aside, what is the probability that the second card picked up is (a) an ace? (b) a queen?\\
\end{enumerate}
\solution
		%\begin{enumerate}[label=\thesection.\arabic*,ref=\thesection.\theenumi]
	\item One card is drawn from a well-shuffled deck of 52 cards. Find the probability of getting
\begin{enumerate}
\item A king of red colour 
\item A face card 
\item A red face card
\item The jack of hearts
\item A spade
\item The queen of diamonds

\end{enumerate}
\solution
		%\input{ncert/10/15/1/14/main.tex}
	\item Five cards—the ten, jack, queen, king and ace of diamonds, are well-shuffled with their face downwards. One card is then picked up at random.
\begin{enumerate}
\item
What is the probability that the card is the queen? 
\item
If the queen is drawn and put aside, what is the probability that the second card picked up is (a) an ace? (b) a queen?\\
\end{enumerate}
\solution
		%\input{ncert/10/15/1/15/defs.tex}
	\item A bag contains $5$ red balls and some blue balls. If the probability of drawing a blue ball is double that if a red ball, determine the number of blue balls in the bag. 
		\\
\solution
		%\input{ncert/10/15/2/3/defs.tex}
	\item A card is selected from a pack of 52 cards.
 \begin{enumerate}[label=(\alph*)] 
                 \item How many points are there in the sample space?
                 \item Calculate the probability that the card is an ace of spades.
                 \item Calculate the probability that the card is (i) an ace and (ii) black card.
 \end{enumerate}
\solution
		%\input{ncert/11/16/3/4/main.tex}
\item Four cards are drawn from a well-shuffled deck of 52 cards. What is the probability of obtaining 3 diamonds and one spade.
\\
\solution
		%\input{ncert/11/16/4/2/defs.tex}
\item In a certain lottery 10,000 tickets are sold and ten equal prizes are awarded. What is the probability of not getting a prize if you buy (a) one ticket (b) two tickets (c) 10 tickets ?	
\\
\solution
		%\input{ncert/11/16/4/4/defs.tex}
		%
\item 
Out of 100 students, two sections of 40 and 60 are formed. If you and your friend are among the 100 students, what is the probability that
\begin{enumerate}
\item you both enter the same section?
\item you both enter the different sections?
\end{enumerate}
\solution
		%\input{ncert/11/16/4/5/defs.tex}
	\item 
The number lock of a suitcase has 4 wheels each labelled with ten digits i.e. from 0 to 9.The lock opens with a sequence of four digits with no repeats.What is the probability of a person getting the right sequence to open the suitcase.
\\
\solution
		%\input{ncert/11/16/4/10/defs.tex}
		%
\item 
Two cards are drawn at random and without replacement from a pack of 52 playing cards. Find the probability that both the cards are black.
\\
\solution
		%\input{ncert/12/13/2/2/defs.tex}
		\item A box of oranges is inspected by examining three randomly selected oranges drawn without replacement. If all the three oranges are good, the box is approved for sale, otherwise, it is rejected. Find the probability that a box containing 15 oranges out of which 12 are good and 3 are bad ones will be approved for sale.
		\label{ncert/12/13/2/3/defs.tex}
		\item Two balls are drawn at random with replacement from a box containing 10 black and 8 red balls. Find the probability that
		\label{ncert/12/13/2/12}
\begin{enumerate}
\item both balls are red.
\item first ball is black and second is red.
\item one of them is black and other is red.
\end{enumerate}

\item In a hostel, 60\% of the students read Hindi newspaper, 40\% read English newspaper and 20\% read both Hindi and English newspapers. A student is selected at random.
		\label{ncert/12/13/2/15}
\begin{enumerate}
\item Find the probability that she reads neither Hindi nor English newspapers.
\item If she reads Hindi newspaper, find the probability that she reads English newspaper.
\item If she reads English newspaper, find the probability that she reads Hindi newspaper.\\
\end{enumerate}
\item The probability of obtaining an even prime number on each die, when a pair of dice is rolled is 
\begin{enumerate}
    \item $0$ 
    
    \item $\frac{1}{3}$ 
    
    \item $\frac{1}{12}$ 
    
    \item $\frac{1}{36}$ 
\end{enumerate}
\solution
		%\input{ncert/12/13/2/17/defs.tex}
	\item A bag contains 4 red and 4 black balls, another bag contains 2 red and 6 black balls. One of the two bags is selected at random and a ball is drawn from the bag which is found to be red. Find the probability that the ball is drawn from the first bag.
\\
\solution
		%\input{ncert/12/13/3/2/main.tex}
  \item
  Cards with numbers 2 to 101 are placed in a box. A card is selected at random.Find the probability that the card has
\begin{enumerate}[label=(\roman*)]
	\item an even number 
	\item a square number
\end{enumerate}
\solution
%\input{exemplar/10/13/3/32/main.tex}
\item
The king, queen and jack of clubs are removed from a deck of 52 playing cards and then well shuffled. Now one card is drawn at random from the remaining cards.  Determine the probability that the card is
\begin{enumerate}[label=(\roman*)]
\item a club
\item 10 of hearts
\end{enumerate}
\solution
%\input{exemplar/10/13/3/29/main.tex}
\item A team of medical students doing their internship have to assist during surgeries
at a city hospital. The probabilities of surgeries rated as very complex, complex,
routine, simple or very simple are respectively, 0.15, 0.20, 0.31, 0.26, .08. Find
the probabilities that a particular surgery will be rated
\begin{enumerate}
	\item complex or very complex;
	\item neither very complex nor very simple;
	\item routine or complex
	\item routine or simple
\end{enumerate}
\solution
%\input{exemplar/11/16/3/8(1)/main.tex}
\item A card is selected from a pack of 52 cards.
\begin{enumerate}[label=(\alph*)]
    \item How many points are there in the sample space?
    \item Calculate the probability that the card is an ace of spades.
    \item Calculate the probability that the card is (i) an ace and (ii) black card.
\end{enumerate}
\solution
%\input{exemplar/11/16/3/4/main2.tex}
\item The probability that a non leap year selected at random will contain 53 sundays.
\\
\solution
%\input{exemplar/10/13/1/19/main.tex}
\item One of the four persons John, Rita, Aslam or Gurpreet will be promoted next
month. Consequently the sample space consists of four elementary outcomes
S = {John promoted, Rita promoted, Aslam promoted, Gurpreet promoted}
You are told that the chances of John’s promotion is same as that of Gurpreet,
Rita’s chances of promotion are twice as likely as Johns. Aslam’s chances are
four times that of John.
\begin{enumerate}
	\item Determine
	\begin{enumerate}
		\item P (John promoted)
		\item P (Rita promoted)
		\item P (Aslam promoted)
		\item P (Gurpreet promoted)
	\end{enumerate}
	\item If A = {John promoted or Gurpreet promoted}, find P (A).
\end{enumerate}
\solution
%\input{exemplar/11/16/3/10/main.tex}
\item A card is drawn from a deck of 52 cards. Find the probability of getting a king or a heart or a red card.\\
\solution
%\input{exemplar/11/16/3/15/main.tex}
\item The probability that a student will pass his examination is 0.73, the probability of
the student getting a compartment is 0.13, and the probability that the student will
either pass or get compartment is 0.96. State True or False.\\
\solution
%\input{exemplar/11/16/3/31/main.tex}
\item A card is selected from a pack of 52 cards\\
\begin{enumerate}[label=(\alph*)]
\item How many points are there in the sample space?
\item Calculate the probability that the cards is an ace of spades.
\item Calculate the probability that the card is (i) an ace (ii)black card.\\
\end{enumerate}
%\input{ncert/11/16/3/4_1/Prob_4.tex}
\item In a non-leap year, the probability of having 53 tuesdays or 53 wednesdays is\\
\solution
%\input{exemplar/11/16/3/18/main.tex}
\item There are 1000 sealed envelopes in a box, 10 of them contain a cash prize of
Rs 100 each, 100 of them contain a cash prize of Rs 50 each and 200 of them
contain a cash prize of Rs 10 each and rest do not contain any cash prize. If they
are well shuffled and an envelope is picked up out, what is the probability that it
contains no cash prize?\\
\solution
%\input{exemplar/10/13/3/34/main.tex}
\item 
A die is thrown and a card is selected at random from a deck of 52 playing cards. The probability of getting an even number on the die and a spade card.\\
\solution
%\input{exemplar/12/13/3/78/main.tex}
\item
If 4-digit numbers greater than 5,000 are randomly formed from the digits 0, 1, 3, 5, and 7, what is the probability of forming a number divisible by 5 when:
\begin{enumerate}
    \item The digits are repeated?
    \item The repetition of digits is not allowed?
\end{enumerate}
\solution
%\input{ncert/11/16/4/9/main.tex}
\item Consider the probability space $\brak{\Omega, \mathcal{G}, P}$ where $\Omega = [0,2]$ and $\mathcal{G} = \cbrak{\phi, \Omega, [0,1], (1,2]}$. Let $X$ and $Y$ be two functions on $\Omega$ defined as
\begin{align*}
    X(\omega) = 
    \begin{cases}
        1 & \text{if }\omega \in [0, 1]\\
        2 & \text{if }\omega \in (1, 2]
    \end{cases}
\end{align*}
and
\begin{align*}
    Y(\omega) = 
    \begin{cases}
        2 & \text{if }\omega \in [0, 1.5]\\
        3 & \text{if }\omega \in (1.5, 2].
    \end{cases}
\end{align*}
Then which one of the following statements is true?
\begin{enumerate}
    \item [(A)] $X$ is a random variable with respect to $\mathcal{G}$, but $Y$ is not a random variable with respect to $\mathcal{G}$.
    \item [(B)] $Y$ is a random variable with respect to $\mathcal{G}$, but $X$ is not a random variable with respect to $\mathcal{G}$.
    \item [(C)] Neither $X$ nor $Y$ is a random variable with respect to $\mathcal{G}$.
    \item [(D)] Both $X$ and $Y$ are random variables with respect to $\mathcal{G}$.
\end{enumerate} \hfill (GATE ST 2023)\\
\solution
%\input{gate/ST/2023/14/main.tex}
	\item  A die is loaded in such a way that each odd number is twice as likely to occur as
each even number. Find $P(G)$, where $G$ is the event that a number greater than
3 occurs on a single roll of the die.
\\
\solution
		%\input{exemplar/11/16/3/5/main.tex}
	\item All the jacks, queens and kings are removed from a deck of 52 playing cards. The remaining cards are well shuffled and then one card is drawn at random. Giving ace a value 1 similar value for other cards, find the probability that the card has a value 
		\begin{enumerate}
			\item 7
			\item greater than 7
			\item less than 7
		\end{enumerate}
		%\input{exemplar/10/13/3/30/main.tex}
  \item A Lot consists of 48 mobile phones of which 42 are good, 3 have only minor defects and 3 have major defects.Varnika will buy a phone if it is good but the trader will only buy a mobile if it has no major defects. One phone is selected at random from the lot. What is the probability that it is
\begin{enumerate}
	\item acceptable to Varnika?
            \item acceptable to the trader?
\end{enumerate}
\solution
	%\input{exemplar/10/13/3/40/main.tex}
 \item A student says that if you throw a die, it will show up 1 or not 1. Therefore, the probability of getting 1 and the probability of getting 'not 1' each is equal to $\frac{1}{2}$. Is this correct? Give reasons.\\
 \solution
        %\input{exemplar/10/13/2/9/main.tex}
   \item Four candidates A, B, C, D have ap-
plied for the assignment to coach a school cricket
team. If A is twice as likely to be selected as B, and
B and C are given about the same chance of being
selected, while C is twice as likely to be selected
as D, what are the probabilities that
\begin{enumerate}
\item C will be selected?
\item A will not be selected?
\end{enumerate}
	%\input{exemplar/11/16/3/9/main.tex}
 \item A bag contain 24 balls of which $x$ balls are red, $2x$ are white and $3x$ are blue. A ball is selected at random, What is the probability that it is
\begin{enumerate}[label=\alph*)]
\item not red ?
\item white ?
\end{enumerate}
%\input{exemplar/10/13/3/41/main.tex}
If the letters of the word ASSASSINATION are arranged at random. Find the Probability that
\begin{enumerate}[label=(\alph*)]
\item Four $S's$ come consecutively in the word
\item Two  $I's$ and two $N's$ come together
\item All $A's$ are not coming together
\item No two $A's$ are coming together
\end{enumerate}
%\input{exemplar/11/16/3/14/main.tex}
	\item One urn contains two black balls (labelled B1 and B2) and one white ball. A
	second urn contains one black ball and two white balls (labelled W1 and W2).
	Suppose the following experiment is performed. One of the two urns is chosen
	at random. Next a ball is randomly chosen from the urn. Then a second ball is
	chosen at random from the same urn without replacing the first ball.
	
	\begin{enumerate}
	\item What is the probability that two black balls are chosen?
	
	\item What is the probability that two balls of opposite colour are chosen?
	\end{enumerate}
	\solution
	%\input{exemplar/11/16/3/12/main1.tex}
\end{enumerate}

	\item A bag contains $5$ red balls and some blue balls. If the probability of drawing a blue ball is double that if a red ball, determine the number of blue balls in the bag. 
		\\
\solution
		%\begin{enumerate}[label=\thesection.\arabic*,ref=\thesection.\theenumi]
	\item One card is drawn from a well-shuffled deck of 52 cards. Find the probability of getting
\begin{enumerate}
\item A king of red colour 
\item A face card 
\item A red face card
\item The jack of hearts
\item A spade
\item The queen of diamonds

\end{enumerate}
\solution
		%\input{ncert/10/15/1/14/main.tex}
	\item Five cards—the ten, jack, queen, king and ace of diamonds, are well-shuffled with their face downwards. One card is then picked up at random.
\begin{enumerate}
\item
What is the probability that the card is the queen? 
\item
If the queen is drawn and put aside, what is the probability that the second card picked up is (a) an ace? (b) a queen?\\
\end{enumerate}
\solution
		%\input{ncert/10/15/1/15/defs.tex}
	\item A bag contains $5$ red balls and some blue balls. If the probability of drawing a blue ball is double that if a red ball, determine the number of blue balls in the bag. 
		\\
\solution
		%\input{ncert/10/15/2/3/defs.tex}
	\item A card is selected from a pack of 52 cards.
 \begin{enumerate}[label=(\alph*)] 
                 \item How many points are there in the sample space?
                 \item Calculate the probability that the card is an ace of spades.
                 \item Calculate the probability that the card is (i) an ace and (ii) black card.
 \end{enumerate}
\solution
		%\input{ncert/11/16/3/4/main.tex}
\item Four cards are drawn from a well-shuffled deck of 52 cards. What is the probability of obtaining 3 diamonds and one spade.
\\
\solution
		%\input{ncert/11/16/4/2/defs.tex}
\item In a certain lottery 10,000 tickets are sold and ten equal prizes are awarded. What is the probability of not getting a prize if you buy (a) one ticket (b) two tickets (c) 10 tickets ?	
\\
\solution
		%\input{ncert/11/16/4/4/defs.tex}
		%
\item 
Out of 100 students, two sections of 40 and 60 are formed. If you and your friend are among the 100 students, what is the probability that
\begin{enumerate}
\item you both enter the same section?
\item you both enter the different sections?
\end{enumerate}
\solution
		%\input{ncert/11/16/4/5/defs.tex}
	\item 
The number lock of a suitcase has 4 wheels each labelled with ten digits i.e. from 0 to 9.The lock opens with a sequence of four digits with no repeats.What is the probability of a person getting the right sequence to open the suitcase.
\\
\solution
		%\input{ncert/11/16/4/10/defs.tex}
		%
\item 
Two cards are drawn at random and without replacement from a pack of 52 playing cards. Find the probability that both the cards are black.
\\
\solution
		%\input{ncert/12/13/2/2/defs.tex}
		\item A box of oranges is inspected by examining three randomly selected oranges drawn without replacement. If all the three oranges are good, the box is approved for sale, otherwise, it is rejected. Find the probability that a box containing 15 oranges out of which 12 are good and 3 are bad ones will be approved for sale.
		\label{ncert/12/13/2/3/defs.tex}
		\item Two balls are drawn at random with replacement from a box containing 10 black and 8 red balls. Find the probability that
		\label{ncert/12/13/2/12}
\begin{enumerate}
\item both balls are red.
\item first ball is black and second is red.
\item one of them is black and other is red.
\end{enumerate}

\item In a hostel, 60\% of the students read Hindi newspaper, 40\% read English newspaper and 20\% read both Hindi and English newspapers. A student is selected at random.
		\label{ncert/12/13/2/15}
\begin{enumerate}
\item Find the probability that she reads neither Hindi nor English newspapers.
\item If she reads Hindi newspaper, find the probability that she reads English newspaper.
\item If she reads English newspaper, find the probability that she reads Hindi newspaper.\\
\end{enumerate}
\item The probability of obtaining an even prime number on each die, when a pair of dice is rolled is 
\begin{enumerate}
    \item $0$ 
    
    \item $\frac{1}{3}$ 
    
    \item $\frac{1}{12}$ 
    
    \item $\frac{1}{36}$ 
\end{enumerate}
\solution
		%\input{ncert/12/13/2/17/defs.tex}
	\item A bag contains 4 red and 4 black balls, another bag contains 2 red and 6 black balls. One of the two bags is selected at random and a ball is drawn from the bag which is found to be red. Find the probability that the ball is drawn from the first bag.
\\
\solution
		%\input{ncert/12/13/3/2/main.tex}
  \item
  Cards with numbers 2 to 101 are placed in a box. A card is selected at random.Find the probability that the card has
\begin{enumerate}[label=(\roman*)]
	\item an even number 
	\item a square number
\end{enumerate}
\solution
%\input{exemplar/10/13/3/32/main.tex}
\item
The king, queen and jack of clubs are removed from a deck of 52 playing cards and then well shuffled. Now one card is drawn at random from the remaining cards.  Determine the probability that the card is
\begin{enumerate}[label=(\roman*)]
\item a club
\item 10 of hearts
\end{enumerate}
\solution
%\input{exemplar/10/13/3/29/main.tex}
\item A team of medical students doing their internship have to assist during surgeries
at a city hospital. The probabilities of surgeries rated as very complex, complex,
routine, simple or very simple are respectively, 0.15, 0.20, 0.31, 0.26, .08. Find
the probabilities that a particular surgery will be rated
\begin{enumerate}
	\item complex or very complex;
	\item neither very complex nor very simple;
	\item routine or complex
	\item routine or simple
\end{enumerate}
\solution
%\input{exemplar/11/16/3/8(1)/main.tex}
\item A card is selected from a pack of 52 cards.
\begin{enumerate}[label=(\alph*)]
    \item How many points are there in the sample space?
    \item Calculate the probability that the card is an ace of spades.
    \item Calculate the probability that the card is (i) an ace and (ii) black card.
\end{enumerate}
\solution
%\input{exemplar/11/16/3/4/main2.tex}
\item The probability that a non leap year selected at random will contain 53 sundays.
\\
\solution
%\input{exemplar/10/13/1/19/main.tex}
\item One of the four persons John, Rita, Aslam or Gurpreet will be promoted next
month. Consequently the sample space consists of four elementary outcomes
S = {John promoted, Rita promoted, Aslam promoted, Gurpreet promoted}
You are told that the chances of John’s promotion is same as that of Gurpreet,
Rita’s chances of promotion are twice as likely as Johns. Aslam’s chances are
four times that of John.
\begin{enumerate}
	\item Determine
	\begin{enumerate}
		\item P (John promoted)
		\item P (Rita promoted)
		\item P (Aslam promoted)
		\item P (Gurpreet promoted)
	\end{enumerate}
	\item If A = {John promoted or Gurpreet promoted}, find P (A).
\end{enumerate}
\solution
%\input{exemplar/11/16/3/10/main.tex}
\item A card is drawn from a deck of 52 cards. Find the probability of getting a king or a heart or a red card.\\
\solution
%\input{exemplar/11/16/3/15/main.tex}
\item The probability that a student will pass his examination is 0.73, the probability of
the student getting a compartment is 0.13, and the probability that the student will
either pass or get compartment is 0.96. State True or False.\\
\solution
%\input{exemplar/11/16/3/31/main.tex}
\item A card is selected from a pack of 52 cards\\
\begin{enumerate}[label=(\alph*)]
\item How many points are there in the sample space?
\item Calculate the probability that the cards is an ace of spades.
\item Calculate the probability that the card is (i) an ace (ii)black card.\\
\end{enumerate}
%\input{ncert/11/16/3/4_1/Prob_4.tex}
\item In a non-leap year, the probability of having 53 tuesdays or 53 wednesdays is\\
\solution
%\input{exemplar/11/16/3/18/main.tex}
\item There are 1000 sealed envelopes in a box, 10 of them contain a cash prize of
Rs 100 each, 100 of them contain a cash prize of Rs 50 each and 200 of them
contain a cash prize of Rs 10 each and rest do not contain any cash prize. If they
are well shuffled and an envelope is picked up out, what is the probability that it
contains no cash prize?\\
\solution
%\input{exemplar/10/13/3/34/main.tex}
\item 
A die is thrown and a card is selected at random from a deck of 52 playing cards. The probability of getting an even number on the die and a spade card.\\
\solution
%\input{exemplar/12/13/3/78/main.tex}
\item
If 4-digit numbers greater than 5,000 are randomly formed from the digits 0, 1, 3, 5, and 7, what is the probability of forming a number divisible by 5 when:
\begin{enumerate}
    \item The digits are repeated?
    \item The repetition of digits is not allowed?
\end{enumerate}
\solution
%\input{ncert/11/16/4/9/main.tex}
\item Consider the probability space $\brak{\Omega, \mathcal{G}, P}$ where $\Omega = [0,2]$ and $\mathcal{G} = \cbrak{\phi, \Omega, [0,1], (1,2]}$. Let $X$ and $Y$ be two functions on $\Omega$ defined as
\begin{align*}
    X(\omega) = 
    \begin{cases}
        1 & \text{if }\omega \in [0, 1]\\
        2 & \text{if }\omega \in (1, 2]
    \end{cases}
\end{align*}
and
\begin{align*}
    Y(\omega) = 
    \begin{cases}
        2 & \text{if }\omega \in [0, 1.5]\\
        3 & \text{if }\omega \in (1.5, 2].
    \end{cases}
\end{align*}
Then which one of the following statements is true?
\begin{enumerate}
    \item [(A)] $X$ is a random variable with respect to $\mathcal{G}$, but $Y$ is not a random variable with respect to $\mathcal{G}$.
    \item [(B)] $Y$ is a random variable with respect to $\mathcal{G}$, but $X$ is not a random variable with respect to $\mathcal{G}$.
    \item [(C)] Neither $X$ nor $Y$ is a random variable with respect to $\mathcal{G}$.
    \item [(D)] Both $X$ and $Y$ are random variables with respect to $\mathcal{G}$.
\end{enumerate} \hfill (GATE ST 2023)\\
\solution
%\input{gate/ST/2023/14/main.tex}
	\item  A die is loaded in such a way that each odd number is twice as likely to occur as
each even number. Find $P(G)$, where $G$ is the event that a number greater than
3 occurs on a single roll of the die.
\\
\solution
		%\input{exemplar/11/16/3/5/main.tex}
	\item All the jacks, queens and kings are removed from a deck of 52 playing cards. The remaining cards are well shuffled and then one card is drawn at random. Giving ace a value 1 similar value for other cards, find the probability that the card has a value 
		\begin{enumerate}
			\item 7
			\item greater than 7
			\item less than 7
		\end{enumerate}
		%\input{exemplar/10/13/3/30/main.tex}
  \item A Lot consists of 48 mobile phones of which 42 are good, 3 have only minor defects and 3 have major defects.Varnika will buy a phone if it is good but the trader will only buy a mobile if it has no major defects. One phone is selected at random from the lot. What is the probability that it is
\begin{enumerate}
	\item acceptable to Varnika?
            \item acceptable to the trader?
\end{enumerate}
\solution
	%\input{exemplar/10/13/3/40/main.tex}
 \item A student says that if you throw a die, it will show up 1 or not 1. Therefore, the probability of getting 1 and the probability of getting 'not 1' each is equal to $\frac{1}{2}$. Is this correct? Give reasons.\\
 \solution
        %\input{exemplar/10/13/2/9/main.tex}
   \item Four candidates A, B, C, D have ap-
plied for the assignment to coach a school cricket
team. If A is twice as likely to be selected as B, and
B and C are given about the same chance of being
selected, while C is twice as likely to be selected
as D, what are the probabilities that
\begin{enumerate}
\item C will be selected?
\item A will not be selected?
\end{enumerate}
	%\input{exemplar/11/16/3/9/main.tex}
 \item A bag contain 24 balls of which $x$ balls are red, $2x$ are white and $3x$ are blue. A ball is selected at random, What is the probability that it is
\begin{enumerate}[label=\alph*)]
\item not red ?
\item white ?
\end{enumerate}
%\input{exemplar/10/13/3/41/main.tex}
If the letters of the word ASSASSINATION are arranged at random. Find the Probability that
\begin{enumerate}[label=(\alph*)]
\item Four $S's$ come consecutively in the word
\item Two  $I's$ and two $N's$ come together
\item All $A's$ are not coming together
\item No two $A's$ are coming together
\end{enumerate}
%\input{exemplar/11/16/3/14/main.tex}
	\item One urn contains two black balls (labelled B1 and B2) and one white ball. A
	second urn contains one black ball and two white balls (labelled W1 and W2).
	Suppose the following experiment is performed. One of the two urns is chosen
	at random. Next a ball is randomly chosen from the urn. Then a second ball is
	chosen at random from the same urn without replacing the first ball.
	
	\begin{enumerate}
	\item What is the probability that two black balls are chosen?
	
	\item What is the probability that two balls of opposite colour are chosen?
	\end{enumerate}
	\solution
	%\input{exemplar/11/16/3/12/main1.tex}
\end{enumerate}

	\item A card is selected from a pack of 52 cards.
 \begin{enumerate}[label=(\alph*)] 
                 \item How many points are there in the sample space?
                 \item Calculate the probability that the card is an ace of spades.
                 \item Calculate the probability that the card is (i) an ace and (ii) black card.
 \end{enumerate}
\solution
		%\begin{table}[H]
	\centering
\begin{tabular}{|c|c|c|}
\hline
Random variable &Value &Definition\\ \hline
\multirow{3}{*}{X} &0 &Slips of Rs 1\\
&1 &Slips of Rs 5\\
&2 &Slips of Rs 13\\ \hline
\multirow{2}{*}{Y} &0 &Box A\\
&1 &Box B\\\hline
\end{tabular}
\caption{}
\label{tab:Distribution}
\end{table}
See \tabref{tab:Distribution}.
\begin{align}
p_{Y}\brak{k}= \begin{cases} 
      \frac{1}{3} & {k=0} \\
      \frac{2}{3 }& {k=1} 
   \end{cases}
   \\
p_{Y|X}\brak{0|0} = \frac{19}{25}\, 
p_{Y|X}\brak{0|1} = \frac{6}{25}\,
p_{Y|X}\brak{1|0} = \frac{45}{50}\,
p_{Y|X}\brak{1|2} = \frac{5}{50}
\end{align}
The desired probability is the probability that a slip drawn at random is marked other than Rs 1,
\begin{align}
&=1-p_X\brak{0}\\
&= p_X(1) + p_X(2)
\end{align}
Using Bayes theorem,
\begin{align}
&= p_Y\brak{0} \times \pr{Y=0 | X=1} + p_Y\brak{1} \times \pr{Y=1|X=2}\\
&=\frac{1}{3} \times \frac{6}{25} + \frac{2}{3} \times \frac{5}{50}\\
&=\frac{11}{75}
\end{align}

\newpage

%\tableofcontents

\bigskip

\renewcommand{\thefigure}{\theenumi}
\renewcommand{\thetable}{\theenumi}
%\renewcommand{\theequation}{\theenumi}

%\begin{abstract}
%%\boldmath
%In this letter, an algorithm for evaluating the exact analytical bit error rate  (BER)  for the piecewise linear (PL) combiner for  multiple relays is presented. Previous results were available only for upto three relays. The algorithm is unique in the sense that  the actual mathematical expressions, that are prohibitively large, need not be explicitly obtained. The diversity gain due to multiple relays is shown through plots of the analytical BER, well supported by simulations. 
%
%\end{abstract}
% IEEEtran.cls defaults to using nonbold math in the Abstract.
% This preserves the distinction between vectors and scalars. However,
% if the journal you are submitting to favors bold math in the abstract,
% then you can use LaTeX's standard command \boldmath at the very start
% of the abstract to achieve this. Many IEEE journals frown on math
% in the abstract anyway.

% Note that keywords are not normally used for peerreview papers.
%\begin{IEEEkeywords}
%Cooperative diversity, decode and forward, piecewise linear
%\end{IEEEkeywords}



% For peer review papers, you can put extra information on the cover
% page as needed:
% \ifCLASSOPTIONpeerreview
% \begin{center} \bfseries EDICS Category: 3-BBND \end{center}
% \fi
%
% For peerreview papers, this IEEEtran command inserts a page break and
% creates the second title. It will be ignored for other modes.
%\IEEEpeerreviewmaketitle




\item Four cards are drawn from a well-shuffled deck of 52 cards. What is the probability of obtaining 3 diamonds and one spade.
\\
\solution
		%\begin{enumerate}[label=\thesection.\arabic*,ref=\thesection.\theenumi]
	\item One card is drawn from a well-shuffled deck of 52 cards. Find the probability of getting
\begin{enumerate}
\item A king of red colour 
\item A face card 
\item A red face card
\item The jack of hearts
\item A spade
\item The queen of diamonds

\end{enumerate}
\solution
		%\input{ncert/10/15/1/14/main.tex}
	\item Five cards—the ten, jack, queen, king and ace of diamonds, are well-shuffled with their face downwards. One card is then picked up at random.
\begin{enumerate}
\item
What is the probability that the card is the queen? 
\item
If the queen is drawn and put aside, what is the probability that the second card picked up is (a) an ace? (b) a queen?\\
\end{enumerate}
\solution
		%\input{ncert/10/15/1/15/defs.tex}
	\item A bag contains $5$ red balls and some blue balls. If the probability of drawing a blue ball is double that if a red ball, determine the number of blue balls in the bag. 
		\\
\solution
		%\input{ncert/10/15/2/3/defs.tex}
	\item A card is selected from a pack of 52 cards.
 \begin{enumerate}[label=(\alph*)] 
                 \item How many points are there in the sample space?
                 \item Calculate the probability that the card is an ace of spades.
                 \item Calculate the probability that the card is (i) an ace and (ii) black card.
 \end{enumerate}
\solution
		%\input{ncert/11/16/3/4/main.tex}
\item Four cards are drawn from a well-shuffled deck of 52 cards. What is the probability of obtaining 3 diamonds and one spade.
\\
\solution
		%\input{ncert/11/16/4/2/defs.tex}
\item In a certain lottery 10,000 tickets are sold and ten equal prizes are awarded. What is the probability of not getting a prize if you buy (a) one ticket (b) two tickets (c) 10 tickets ?	
\\
\solution
		%\input{ncert/11/16/4/4/defs.tex}
		%
\item 
Out of 100 students, two sections of 40 and 60 are formed. If you and your friend are among the 100 students, what is the probability that
\begin{enumerate}
\item you both enter the same section?
\item you both enter the different sections?
\end{enumerate}
\solution
		%\input{ncert/11/16/4/5/defs.tex}
	\item 
The number lock of a suitcase has 4 wheels each labelled with ten digits i.e. from 0 to 9.The lock opens with a sequence of four digits with no repeats.What is the probability of a person getting the right sequence to open the suitcase.
\\
\solution
		%\input{ncert/11/16/4/10/defs.tex}
		%
\item 
Two cards are drawn at random and without replacement from a pack of 52 playing cards. Find the probability that both the cards are black.
\\
\solution
		%\input{ncert/12/13/2/2/defs.tex}
		\item A box of oranges is inspected by examining three randomly selected oranges drawn without replacement. If all the three oranges are good, the box is approved for sale, otherwise, it is rejected. Find the probability that a box containing 15 oranges out of which 12 are good and 3 are bad ones will be approved for sale.
		\label{ncert/12/13/2/3/defs.tex}
		\item Two balls are drawn at random with replacement from a box containing 10 black and 8 red balls. Find the probability that
		\label{ncert/12/13/2/12}
\begin{enumerate}
\item both balls are red.
\item first ball is black and second is red.
\item one of them is black and other is red.
\end{enumerate}

\item In a hostel, 60\% of the students read Hindi newspaper, 40\% read English newspaper and 20\% read both Hindi and English newspapers. A student is selected at random.
		\label{ncert/12/13/2/15}
\begin{enumerate}
\item Find the probability that she reads neither Hindi nor English newspapers.
\item If she reads Hindi newspaper, find the probability that she reads English newspaper.
\item If she reads English newspaper, find the probability that she reads Hindi newspaper.\\
\end{enumerate}
\item The probability of obtaining an even prime number on each die, when a pair of dice is rolled is 
\begin{enumerate}
    \item $0$ 
    
    \item $\frac{1}{3}$ 
    
    \item $\frac{1}{12}$ 
    
    \item $\frac{1}{36}$ 
\end{enumerate}
\solution
		%\input{ncert/12/13/2/17/defs.tex}
	\item A bag contains 4 red and 4 black balls, another bag contains 2 red and 6 black balls. One of the two bags is selected at random and a ball is drawn from the bag which is found to be red. Find the probability that the ball is drawn from the first bag.
\\
\solution
		%\input{ncert/12/13/3/2/main.tex}
  \item
  Cards with numbers 2 to 101 are placed in a box. A card is selected at random.Find the probability that the card has
\begin{enumerate}[label=(\roman*)]
	\item an even number 
	\item a square number
\end{enumerate}
\solution
%\input{exemplar/10/13/3/32/main.tex}
\item
The king, queen and jack of clubs are removed from a deck of 52 playing cards and then well shuffled. Now one card is drawn at random from the remaining cards.  Determine the probability that the card is
\begin{enumerate}[label=(\roman*)]
\item a club
\item 10 of hearts
\end{enumerate}
\solution
%\input{exemplar/10/13/3/29/main.tex}
\item A team of medical students doing their internship have to assist during surgeries
at a city hospital. The probabilities of surgeries rated as very complex, complex,
routine, simple or very simple are respectively, 0.15, 0.20, 0.31, 0.26, .08. Find
the probabilities that a particular surgery will be rated
\begin{enumerate}
	\item complex or very complex;
	\item neither very complex nor very simple;
	\item routine or complex
	\item routine or simple
\end{enumerate}
\solution
%\input{exemplar/11/16/3/8(1)/main.tex}
\item A card is selected from a pack of 52 cards.
\begin{enumerate}[label=(\alph*)]
    \item How many points are there in the sample space?
    \item Calculate the probability that the card is an ace of spades.
    \item Calculate the probability that the card is (i) an ace and (ii) black card.
\end{enumerate}
\solution
%\input{exemplar/11/16/3/4/main2.tex}
\item The probability that a non leap year selected at random will contain 53 sundays.
\\
\solution
%\input{exemplar/10/13/1/19/main.tex}
\item One of the four persons John, Rita, Aslam or Gurpreet will be promoted next
month. Consequently the sample space consists of four elementary outcomes
S = {John promoted, Rita promoted, Aslam promoted, Gurpreet promoted}
You are told that the chances of John’s promotion is same as that of Gurpreet,
Rita’s chances of promotion are twice as likely as Johns. Aslam’s chances are
four times that of John.
\begin{enumerate}
	\item Determine
	\begin{enumerate}
		\item P (John promoted)
		\item P (Rita promoted)
		\item P (Aslam promoted)
		\item P (Gurpreet promoted)
	\end{enumerate}
	\item If A = {John promoted or Gurpreet promoted}, find P (A).
\end{enumerate}
\solution
%\input{exemplar/11/16/3/10/main.tex}
\item A card is drawn from a deck of 52 cards. Find the probability of getting a king or a heart or a red card.\\
\solution
%\input{exemplar/11/16/3/15/main.tex}
\item The probability that a student will pass his examination is 0.73, the probability of
the student getting a compartment is 0.13, and the probability that the student will
either pass or get compartment is 0.96. State True or False.\\
\solution
%\input{exemplar/11/16/3/31/main.tex}
\item A card is selected from a pack of 52 cards\\
\begin{enumerate}[label=(\alph*)]
\item How many points are there in the sample space?
\item Calculate the probability that the cards is an ace of spades.
\item Calculate the probability that the card is (i) an ace (ii)black card.\\
\end{enumerate}
%\input{ncert/11/16/3/4_1/Prob_4.tex}
\item In a non-leap year, the probability of having 53 tuesdays or 53 wednesdays is\\
\solution
%\input{exemplar/11/16/3/18/main.tex}
\item There are 1000 sealed envelopes in a box, 10 of them contain a cash prize of
Rs 100 each, 100 of them contain a cash prize of Rs 50 each and 200 of them
contain a cash prize of Rs 10 each and rest do not contain any cash prize. If they
are well shuffled and an envelope is picked up out, what is the probability that it
contains no cash prize?\\
\solution
%\input{exemplar/10/13/3/34/main.tex}
\item 
A die is thrown and a card is selected at random from a deck of 52 playing cards. The probability of getting an even number on the die and a spade card.\\
\solution
%\input{exemplar/12/13/3/78/main.tex}
\item
If 4-digit numbers greater than 5,000 are randomly formed from the digits 0, 1, 3, 5, and 7, what is the probability of forming a number divisible by 5 when:
\begin{enumerate}
    \item The digits are repeated?
    \item The repetition of digits is not allowed?
\end{enumerate}
\solution
%\input{ncert/11/16/4/9/main.tex}
\item Consider the probability space $\brak{\Omega, \mathcal{G}, P}$ where $\Omega = [0,2]$ and $\mathcal{G} = \cbrak{\phi, \Omega, [0,1], (1,2]}$. Let $X$ and $Y$ be two functions on $\Omega$ defined as
\begin{align*}
    X(\omega) = 
    \begin{cases}
        1 & \text{if }\omega \in [0, 1]\\
        2 & \text{if }\omega \in (1, 2]
    \end{cases}
\end{align*}
and
\begin{align*}
    Y(\omega) = 
    \begin{cases}
        2 & \text{if }\omega \in [0, 1.5]\\
        3 & \text{if }\omega \in (1.5, 2].
    \end{cases}
\end{align*}
Then which one of the following statements is true?
\begin{enumerate}
    \item [(A)] $X$ is a random variable with respect to $\mathcal{G}$, but $Y$ is not a random variable with respect to $\mathcal{G}$.
    \item [(B)] $Y$ is a random variable with respect to $\mathcal{G}$, but $X$ is not a random variable with respect to $\mathcal{G}$.
    \item [(C)] Neither $X$ nor $Y$ is a random variable with respect to $\mathcal{G}$.
    \item [(D)] Both $X$ and $Y$ are random variables with respect to $\mathcal{G}$.
\end{enumerate} \hfill (GATE ST 2023)\\
\solution
%\input{gate/ST/2023/14/main.tex}
	\item  A die is loaded in such a way that each odd number is twice as likely to occur as
each even number. Find $P(G)$, where $G$ is the event that a number greater than
3 occurs on a single roll of the die.
\\
\solution
		%\input{exemplar/11/16/3/5/main.tex}
	\item All the jacks, queens and kings are removed from a deck of 52 playing cards. The remaining cards are well shuffled and then one card is drawn at random. Giving ace a value 1 similar value for other cards, find the probability that the card has a value 
		\begin{enumerate}
			\item 7
			\item greater than 7
			\item less than 7
		\end{enumerate}
		%\input{exemplar/10/13/3/30/main.tex}
  \item A Lot consists of 48 mobile phones of which 42 are good, 3 have only minor defects and 3 have major defects.Varnika will buy a phone if it is good but the trader will only buy a mobile if it has no major defects. One phone is selected at random from the lot. What is the probability that it is
\begin{enumerate}
	\item acceptable to Varnika?
            \item acceptable to the trader?
\end{enumerate}
\solution
	%\input{exemplar/10/13/3/40/main.tex}
 \item A student says that if you throw a die, it will show up 1 or not 1. Therefore, the probability of getting 1 and the probability of getting 'not 1' each is equal to $\frac{1}{2}$. Is this correct? Give reasons.\\
 \solution
        %\input{exemplar/10/13/2/9/main.tex}
   \item Four candidates A, B, C, D have ap-
plied for the assignment to coach a school cricket
team. If A is twice as likely to be selected as B, and
B and C are given about the same chance of being
selected, while C is twice as likely to be selected
as D, what are the probabilities that
\begin{enumerate}
\item C will be selected?
\item A will not be selected?
\end{enumerate}
	%\input{exemplar/11/16/3/9/main.tex}
 \item A bag contain 24 balls of which $x$ balls are red, $2x$ are white and $3x$ are blue. A ball is selected at random, What is the probability that it is
\begin{enumerate}[label=\alph*)]
\item not red ?
\item white ?
\end{enumerate}
%\input{exemplar/10/13/3/41/main.tex}
If the letters of the word ASSASSINATION are arranged at random. Find the Probability that
\begin{enumerate}[label=(\alph*)]
\item Four $S's$ come consecutively in the word
\item Two  $I's$ and two $N's$ come together
\item All $A's$ are not coming together
\item No two $A's$ are coming together
\end{enumerate}
%\input{exemplar/11/16/3/14/main.tex}
	\item One urn contains two black balls (labelled B1 and B2) and one white ball. A
	second urn contains one black ball and two white balls (labelled W1 and W2).
	Suppose the following experiment is performed. One of the two urns is chosen
	at random. Next a ball is randomly chosen from the urn. Then a second ball is
	chosen at random from the same urn without replacing the first ball.
	
	\begin{enumerate}
	\item What is the probability that two black balls are chosen?
	
	\item What is the probability that two balls of opposite colour are chosen?
	\end{enumerate}
	\solution
	%\input{exemplar/11/16/3/12/main1.tex}
\end{enumerate}

\item In a certain lottery 10,000 tickets are sold and ten equal prizes are awarded. What is the probability of not getting a prize if you buy (a) one ticket (b) two tickets (c) 10 tickets ?	
\\
\solution
		%\begin{enumerate}[label=\thesection.\arabic*,ref=\thesection.\theenumi]
	\item One card is drawn from a well-shuffled deck of 52 cards. Find the probability of getting
\begin{enumerate}
\item A king of red colour 
\item A face card 
\item A red face card
\item The jack of hearts
\item A spade
\item The queen of diamonds

\end{enumerate}
\solution
		%\input{ncert/10/15/1/14/main.tex}
	\item Five cards—the ten, jack, queen, king and ace of diamonds, are well-shuffled with their face downwards. One card is then picked up at random.
\begin{enumerate}
\item
What is the probability that the card is the queen? 
\item
If the queen is drawn and put aside, what is the probability that the second card picked up is (a) an ace? (b) a queen?\\
\end{enumerate}
\solution
		%\input{ncert/10/15/1/15/defs.tex}
	\item A bag contains $5$ red balls and some blue balls. If the probability of drawing a blue ball is double that if a red ball, determine the number of blue balls in the bag. 
		\\
\solution
		%\input{ncert/10/15/2/3/defs.tex}
	\item A card is selected from a pack of 52 cards.
 \begin{enumerate}[label=(\alph*)] 
                 \item How many points are there in the sample space?
                 \item Calculate the probability that the card is an ace of spades.
                 \item Calculate the probability that the card is (i) an ace and (ii) black card.
 \end{enumerate}
\solution
		%\input{ncert/11/16/3/4/main.tex}
\item Four cards are drawn from a well-shuffled deck of 52 cards. What is the probability of obtaining 3 diamonds and one spade.
\\
\solution
		%\input{ncert/11/16/4/2/defs.tex}
\item In a certain lottery 10,000 tickets are sold and ten equal prizes are awarded. What is the probability of not getting a prize if you buy (a) one ticket (b) two tickets (c) 10 tickets ?	
\\
\solution
		%\input{ncert/11/16/4/4/defs.tex}
		%
\item 
Out of 100 students, two sections of 40 and 60 are formed. If you and your friend are among the 100 students, what is the probability that
\begin{enumerate}
\item you both enter the same section?
\item you both enter the different sections?
\end{enumerate}
\solution
		%\input{ncert/11/16/4/5/defs.tex}
	\item 
The number lock of a suitcase has 4 wheels each labelled with ten digits i.e. from 0 to 9.The lock opens with a sequence of four digits with no repeats.What is the probability of a person getting the right sequence to open the suitcase.
\\
\solution
		%\input{ncert/11/16/4/10/defs.tex}
		%
\item 
Two cards are drawn at random and without replacement from a pack of 52 playing cards. Find the probability that both the cards are black.
\\
\solution
		%\input{ncert/12/13/2/2/defs.tex}
		\item A box of oranges is inspected by examining three randomly selected oranges drawn without replacement. If all the three oranges are good, the box is approved for sale, otherwise, it is rejected. Find the probability that a box containing 15 oranges out of which 12 are good and 3 are bad ones will be approved for sale.
		\label{ncert/12/13/2/3/defs.tex}
		\item Two balls are drawn at random with replacement from a box containing 10 black and 8 red balls. Find the probability that
		\label{ncert/12/13/2/12}
\begin{enumerate}
\item both balls are red.
\item first ball is black and second is red.
\item one of them is black and other is red.
\end{enumerate}

\item In a hostel, 60\% of the students read Hindi newspaper, 40\% read English newspaper and 20\% read both Hindi and English newspapers. A student is selected at random.
		\label{ncert/12/13/2/15}
\begin{enumerate}
\item Find the probability that she reads neither Hindi nor English newspapers.
\item If she reads Hindi newspaper, find the probability that she reads English newspaper.
\item If she reads English newspaper, find the probability that she reads Hindi newspaper.\\
\end{enumerate}
\item The probability of obtaining an even prime number on each die, when a pair of dice is rolled is 
\begin{enumerate}
    \item $0$ 
    
    \item $\frac{1}{3}$ 
    
    \item $\frac{1}{12}$ 
    
    \item $\frac{1}{36}$ 
\end{enumerate}
\solution
		%\input{ncert/12/13/2/17/defs.tex}
	\item A bag contains 4 red and 4 black balls, another bag contains 2 red and 6 black balls. One of the two bags is selected at random and a ball is drawn from the bag which is found to be red. Find the probability that the ball is drawn from the first bag.
\\
\solution
		%\input{ncert/12/13/3/2/main.tex}
  \item
  Cards with numbers 2 to 101 are placed in a box. A card is selected at random.Find the probability that the card has
\begin{enumerate}[label=(\roman*)]
	\item an even number 
	\item a square number
\end{enumerate}
\solution
%\input{exemplar/10/13/3/32/main.tex}
\item
The king, queen and jack of clubs are removed from a deck of 52 playing cards and then well shuffled. Now one card is drawn at random from the remaining cards.  Determine the probability that the card is
\begin{enumerate}[label=(\roman*)]
\item a club
\item 10 of hearts
\end{enumerate}
\solution
%\input{exemplar/10/13/3/29/main.tex}
\item A team of medical students doing their internship have to assist during surgeries
at a city hospital. The probabilities of surgeries rated as very complex, complex,
routine, simple or very simple are respectively, 0.15, 0.20, 0.31, 0.26, .08. Find
the probabilities that a particular surgery will be rated
\begin{enumerate}
	\item complex or very complex;
	\item neither very complex nor very simple;
	\item routine or complex
	\item routine or simple
\end{enumerate}
\solution
%\input{exemplar/11/16/3/8(1)/main.tex}
\item A card is selected from a pack of 52 cards.
\begin{enumerate}[label=(\alph*)]
    \item How many points are there in the sample space?
    \item Calculate the probability that the card is an ace of spades.
    \item Calculate the probability that the card is (i) an ace and (ii) black card.
\end{enumerate}
\solution
%\input{exemplar/11/16/3/4/main2.tex}
\item The probability that a non leap year selected at random will contain 53 sundays.
\\
\solution
%\input{exemplar/10/13/1/19/main.tex}
\item One of the four persons John, Rita, Aslam or Gurpreet will be promoted next
month. Consequently the sample space consists of four elementary outcomes
S = {John promoted, Rita promoted, Aslam promoted, Gurpreet promoted}
You are told that the chances of John’s promotion is same as that of Gurpreet,
Rita’s chances of promotion are twice as likely as Johns. Aslam’s chances are
four times that of John.
\begin{enumerate}
	\item Determine
	\begin{enumerate}
		\item P (John promoted)
		\item P (Rita promoted)
		\item P (Aslam promoted)
		\item P (Gurpreet promoted)
	\end{enumerate}
	\item If A = {John promoted or Gurpreet promoted}, find P (A).
\end{enumerate}
\solution
%\input{exemplar/11/16/3/10/main.tex}
\item A card is drawn from a deck of 52 cards. Find the probability of getting a king or a heart or a red card.\\
\solution
%\input{exemplar/11/16/3/15/main.tex}
\item The probability that a student will pass his examination is 0.73, the probability of
the student getting a compartment is 0.13, and the probability that the student will
either pass or get compartment is 0.96. State True or False.\\
\solution
%\input{exemplar/11/16/3/31/main.tex}
\item A card is selected from a pack of 52 cards\\
\begin{enumerate}[label=(\alph*)]
\item How many points are there in the sample space?
\item Calculate the probability that the cards is an ace of spades.
\item Calculate the probability that the card is (i) an ace (ii)black card.\\
\end{enumerate}
%\input{ncert/11/16/3/4_1/Prob_4.tex}
\item In a non-leap year, the probability of having 53 tuesdays or 53 wednesdays is\\
\solution
%\input{exemplar/11/16/3/18/main.tex}
\item There are 1000 sealed envelopes in a box, 10 of them contain a cash prize of
Rs 100 each, 100 of them contain a cash prize of Rs 50 each and 200 of them
contain a cash prize of Rs 10 each and rest do not contain any cash prize. If they
are well shuffled and an envelope is picked up out, what is the probability that it
contains no cash prize?\\
\solution
%\input{exemplar/10/13/3/34/main.tex}
\item 
A die is thrown and a card is selected at random from a deck of 52 playing cards. The probability of getting an even number on the die and a spade card.\\
\solution
%\input{exemplar/12/13/3/78/main.tex}
\item
If 4-digit numbers greater than 5,000 are randomly formed from the digits 0, 1, 3, 5, and 7, what is the probability of forming a number divisible by 5 when:
\begin{enumerate}
    \item The digits are repeated?
    \item The repetition of digits is not allowed?
\end{enumerate}
\solution
%\input{ncert/11/16/4/9/main.tex}
\item Consider the probability space $\brak{\Omega, \mathcal{G}, P}$ where $\Omega = [0,2]$ and $\mathcal{G} = \cbrak{\phi, \Omega, [0,1], (1,2]}$. Let $X$ and $Y$ be two functions on $\Omega$ defined as
\begin{align*}
    X(\omega) = 
    \begin{cases}
        1 & \text{if }\omega \in [0, 1]\\
        2 & \text{if }\omega \in (1, 2]
    \end{cases}
\end{align*}
and
\begin{align*}
    Y(\omega) = 
    \begin{cases}
        2 & \text{if }\omega \in [0, 1.5]\\
        3 & \text{if }\omega \in (1.5, 2].
    \end{cases}
\end{align*}
Then which one of the following statements is true?
\begin{enumerate}
    \item [(A)] $X$ is a random variable with respect to $\mathcal{G}$, but $Y$ is not a random variable with respect to $\mathcal{G}$.
    \item [(B)] $Y$ is a random variable with respect to $\mathcal{G}$, but $X$ is not a random variable with respect to $\mathcal{G}$.
    \item [(C)] Neither $X$ nor $Y$ is a random variable with respect to $\mathcal{G}$.
    \item [(D)] Both $X$ and $Y$ are random variables with respect to $\mathcal{G}$.
\end{enumerate} \hfill (GATE ST 2023)\\
\solution
%\input{gate/ST/2023/14/main.tex}
	\item  A die is loaded in such a way that each odd number is twice as likely to occur as
each even number. Find $P(G)$, where $G$ is the event that a number greater than
3 occurs on a single roll of the die.
\\
\solution
		%\input{exemplar/11/16/3/5/main.tex}
	\item All the jacks, queens and kings are removed from a deck of 52 playing cards. The remaining cards are well shuffled and then one card is drawn at random. Giving ace a value 1 similar value for other cards, find the probability that the card has a value 
		\begin{enumerate}
			\item 7
			\item greater than 7
			\item less than 7
		\end{enumerate}
		%\input{exemplar/10/13/3/30/main.tex}
  \item A Lot consists of 48 mobile phones of which 42 are good, 3 have only minor defects and 3 have major defects.Varnika will buy a phone if it is good but the trader will only buy a mobile if it has no major defects. One phone is selected at random from the lot. What is the probability that it is
\begin{enumerate}
	\item acceptable to Varnika?
            \item acceptable to the trader?
\end{enumerate}
\solution
	%\input{exemplar/10/13/3/40/main.tex}
 \item A student says that if you throw a die, it will show up 1 or not 1. Therefore, the probability of getting 1 and the probability of getting 'not 1' each is equal to $\frac{1}{2}$. Is this correct? Give reasons.\\
 \solution
        %\input{exemplar/10/13/2/9/main.tex}
   \item Four candidates A, B, C, D have ap-
plied for the assignment to coach a school cricket
team. If A is twice as likely to be selected as B, and
B and C are given about the same chance of being
selected, while C is twice as likely to be selected
as D, what are the probabilities that
\begin{enumerate}
\item C will be selected?
\item A will not be selected?
\end{enumerate}
	%\input{exemplar/11/16/3/9/main.tex}
 \item A bag contain 24 balls of which $x$ balls are red, $2x$ are white and $3x$ are blue. A ball is selected at random, What is the probability that it is
\begin{enumerate}[label=\alph*)]
\item not red ?
\item white ?
\end{enumerate}
%\input{exemplar/10/13/3/41/main.tex}
If the letters of the word ASSASSINATION are arranged at random. Find the Probability that
\begin{enumerate}[label=(\alph*)]
\item Four $S's$ come consecutively in the word
\item Two  $I's$ and two $N's$ come together
\item All $A's$ are not coming together
\item No two $A's$ are coming together
\end{enumerate}
%\input{exemplar/11/16/3/14/main.tex}
	\item One urn contains two black balls (labelled B1 and B2) and one white ball. A
	second urn contains one black ball and two white balls (labelled W1 and W2).
	Suppose the following experiment is performed. One of the two urns is chosen
	at random. Next a ball is randomly chosen from the urn. Then a second ball is
	chosen at random from the same urn without replacing the first ball.
	
	\begin{enumerate}
	\item What is the probability that two black balls are chosen?
	
	\item What is the probability that two balls of opposite colour are chosen?
	\end{enumerate}
	\solution
	%\input{exemplar/11/16/3/12/main1.tex}
\end{enumerate}

		%
\item 
Out of 100 students, two sections of 40 and 60 are formed. If you and your friend are among the 100 students, what is the probability that
\begin{enumerate}
\item you both enter the same section?
\item you both enter the different sections?
\end{enumerate}
\solution
		%\begin{enumerate}[label=\thesection.\arabic*,ref=\thesection.\theenumi]
	\item One card is drawn from a well-shuffled deck of 52 cards. Find the probability of getting
\begin{enumerate}
\item A king of red colour 
\item A face card 
\item A red face card
\item The jack of hearts
\item A spade
\item The queen of diamonds

\end{enumerate}
\solution
		%\input{ncert/10/15/1/14/main.tex}
	\item Five cards—the ten, jack, queen, king and ace of diamonds, are well-shuffled with their face downwards. One card is then picked up at random.
\begin{enumerate}
\item
What is the probability that the card is the queen? 
\item
If the queen is drawn and put aside, what is the probability that the second card picked up is (a) an ace? (b) a queen?\\
\end{enumerate}
\solution
		%\input{ncert/10/15/1/15/defs.tex}
	\item A bag contains $5$ red balls and some blue balls. If the probability of drawing a blue ball is double that if a red ball, determine the number of blue balls in the bag. 
		\\
\solution
		%\input{ncert/10/15/2/3/defs.tex}
	\item A card is selected from a pack of 52 cards.
 \begin{enumerate}[label=(\alph*)] 
                 \item How many points are there in the sample space?
                 \item Calculate the probability that the card is an ace of spades.
                 \item Calculate the probability that the card is (i) an ace and (ii) black card.
 \end{enumerate}
\solution
		%\input{ncert/11/16/3/4/main.tex}
\item Four cards are drawn from a well-shuffled deck of 52 cards. What is the probability of obtaining 3 diamonds and one spade.
\\
\solution
		%\input{ncert/11/16/4/2/defs.tex}
\item In a certain lottery 10,000 tickets are sold and ten equal prizes are awarded. What is the probability of not getting a prize if you buy (a) one ticket (b) two tickets (c) 10 tickets ?	
\\
\solution
		%\input{ncert/11/16/4/4/defs.tex}
		%
\item 
Out of 100 students, two sections of 40 and 60 are formed. If you and your friend are among the 100 students, what is the probability that
\begin{enumerate}
\item you both enter the same section?
\item you both enter the different sections?
\end{enumerate}
\solution
		%\input{ncert/11/16/4/5/defs.tex}
	\item 
The number lock of a suitcase has 4 wheels each labelled with ten digits i.e. from 0 to 9.The lock opens with a sequence of four digits with no repeats.What is the probability of a person getting the right sequence to open the suitcase.
\\
\solution
		%\input{ncert/11/16/4/10/defs.tex}
		%
\item 
Two cards are drawn at random and without replacement from a pack of 52 playing cards. Find the probability that both the cards are black.
\\
\solution
		%\input{ncert/12/13/2/2/defs.tex}
		\item A box of oranges is inspected by examining three randomly selected oranges drawn without replacement. If all the three oranges are good, the box is approved for sale, otherwise, it is rejected. Find the probability that a box containing 15 oranges out of which 12 are good and 3 are bad ones will be approved for sale.
		\label{ncert/12/13/2/3/defs.tex}
		\item Two balls are drawn at random with replacement from a box containing 10 black and 8 red balls. Find the probability that
		\label{ncert/12/13/2/12}
\begin{enumerate}
\item both balls are red.
\item first ball is black and second is red.
\item one of them is black and other is red.
\end{enumerate}

\item In a hostel, 60\% of the students read Hindi newspaper, 40\% read English newspaper and 20\% read both Hindi and English newspapers. A student is selected at random.
		\label{ncert/12/13/2/15}
\begin{enumerate}
\item Find the probability that she reads neither Hindi nor English newspapers.
\item If she reads Hindi newspaper, find the probability that she reads English newspaper.
\item If she reads English newspaper, find the probability that she reads Hindi newspaper.\\
\end{enumerate}
\item The probability of obtaining an even prime number on each die, when a pair of dice is rolled is 
\begin{enumerate}
    \item $0$ 
    
    \item $\frac{1}{3}$ 
    
    \item $\frac{1}{12}$ 
    
    \item $\frac{1}{36}$ 
\end{enumerate}
\solution
		%\input{ncert/12/13/2/17/defs.tex}
	\item A bag contains 4 red and 4 black balls, another bag contains 2 red and 6 black balls. One of the two bags is selected at random and a ball is drawn from the bag which is found to be red. Find the probability that the ball is drawn from the first bag.
\\
\solution
		%\input{ncert/12/13/3/2/main.tex}
  \item
  Cards with numbers 2 to 101 are placed in a box. A card is selected at random.Find the probability that the card has
\begin{enumerate}[label=(\roman*)]
	\item an even number 
	\item a square number
\end{enumerate}
\solution
%\input{exemplar/10/13/3/32/main.tex}
\item
The king, queen and jack of clubs are removed from a deck of 52 playing cards and then well shuffled. Now one card is drawn at random from the remaining cards.  Determine the probability that the card is
\begin{enumerate}[label=(\roman*)]
\item a club
\item 10 of hearts
\end{enumerate}
\solution
%\input{exemplar/10/13/3/29/main.tex}
\item A team of medical students doing their internship have to assist during surgeries
at a city hospital. The probabilities of surgeries rated as very complex, complex,
routine, simple or very simple are respectively, 0.15, 0.20, 0.31, 0.26, .08. Find
the probabilities that a particular surgery will be rated
\begin{enumerate}
	\item complex or very complex;
	\item neither very complex nor very simple;
	\item routine or complex
	\item routine or simple
\end{enumerate}
\solution
%\input{exemplar/11/16/3/8(1)/main.tex}
\item A card is selected from a pack of 52 cards.
\begin{enumerate}[label=(\alph*)]
    \item How many points are there in the sample space?
    \item Calculate the probability that the card is an ace of spades.
    \item Calculate the probability that the card is (i) an ace and (ii) black card.
\end{enumerate}
\solution
%\input{exemplar/11/16/3/4/main2.tex}
\item The probability that a non leap year selected at random will contain 53 sundays.
\\
\solution
%\input{exemplar/10/13/1/19/main.tex}
\item One of the four persons John, Rita, Aslam or Gurpreet will be promoted next
month. Consequently the sample space consists of four elementary outcomes
S = {John promoted, Rita promoted, Aslam promoted, Gurpreet promoted}
You are told that the chances of John’s promotion is same as that of Gurpreet,
Rita’s chances of promotion are twice as likely as Johns. Aslam’s chances are
four times that of John.
\begin{enumerate}
	\item Determine
	\begin{enumerate}
		\item P (John promoted)
		\item P (Rita promoted)
		\item P (Aslam promoted)
		\item P (Gurpreet promoted)
	\end{enumerate}
	\item If A = {John promoted or Gurpreet promoted}, find P (A).
\end{enumerate}
\solution
%\input{exemplar/11/16/3/10/main.tex}
\item A card is drawn from a deck of 52 cards. Find the probability of getting a king or a heart or a red card.\\
\solution
%\input{exemplar/11/16/3/15/main.tex}
\item The probability that a student will pass his examination is 0.73, the probability of
the student getting a compartment is 0.13, and the probability that the student will
either pass or get compartment is 0.96. State True or False.\\
\solution
%\input{exemplar/11/16/3/31/main.tex}
\item A card is selected from a pack of 52 cards\\
\begin{enumerate}[label=(\alph*)]
\item How many points are there in the sample space?
\item Calculate the probability that the cards is an ace of spades.
\item Calculate the probability that the card is (i) an ace (ii)black card.\\
\end{enumerate}
%\input{ncert/11/16/3/4_1/Prob_4.tex}
\item In a non-leap year, the probability of having 53 tuesdays or 53 wednesdays is\\
\solution
%\input{exemplar/11/16/3/18/main.tex}
\item There are 1000 sealed envelopes in a box, 10 of them contain a cash prize of
Rs 100 each, 100 of them contain a cash prize of Rs 50 each and 200 of them
contain a cash prize of Rs 10 each and rest do not contain any cash prize. If they
are well shuffled and an envelope is picked up out, what is the probability that it
contains no cash prize?\\
\solution
%\input{exemplar/10/13/3/34/main.tex}
\item 
A die is thrown and a card is selected at random from a deck of 52 playing cards. The probability of getting an even number on the die and a spade card.\\
\solution
%\input{exemplar/12/13/3/78/main.tex}
\item
If 4-digit numbers greater than 5,000 are randomly formed from the digits 0, 1, 3, 5, and 7, what is the probability of forming a number divisible by 5 when:
\begin{enumerate}
    \item The digits are repeated?
    \item The repetition of digits is not allowed?
\end{enumerate}
\solution
%\input{ncert/11/16/4/9/main.tex}
\item Consider the probability space $\brak{\Omega, \mathcal{G}, P}$ where $\Omega = [0,2]$ and $\mathcal{G} = \cbrak{\phi, \Omega, [0,1], (1,2]}$. Let $X$ and $Y$ be two functions on $\Omega$ defined as
\begin{align*}
    X(\omega) = 
    \begin{cases}
        1 & \text{if }\omega \in [0, 1]\\
        2 & \text{if }\omega \in (1, 2]
    \end{cases}
\end{align*}
and
\begin{align*}
    Y(\omega) = 
    \begin{cases}
        2 & \text{if }\omega \in [0, 1.5]\\
        3 & \text{if }\omega \in (1.5, 2].
    \end{cases}
\end{align*}
Then which one of the following statements is true?
\begin{enumerate}
    \item [(A)] $X$ is a random variable with respect to $\mathcal{G}$, but $Y$ is not a random variable with respect to $\mathcal{G}$.
    \item [(B)] $Y$ is a random variable with respect to $\mathcal{G}$, but $X$ is not a random variable with respect to $\mathcal{G}$.
    \item [(C)] Neither $X$ nor $Y$ is a random variable with respect to $\mathcal{G}$.
    \item [(D)] Both $X$ and $Y$ are random variables with respect to $\mathcal{G}$.
\end{enumerate} \hfill (GATE ST 2023)\\
\solution
%\input{gate/ST/2023/14/main.tex}
	\item  A die is loaded in such a way that each odd number is twice as likely to occur as
each even number. Find $P(G)$, where $G$ is the event that a number greater than
3 occurs on a single roll of the die.
\\
\solution
		%\input{exemplar/11/16/3/5/main.tex}
	\item All the jacks, queens and kings are removed from a deck of 52 playing cards. The remaining cards are well shuffled and then one card is drawn at random. Giving ace a value 1 similar value for other cards, find the probability that the card has a value 
		\begin{enumerate}
			\item 7
			\item greater than 7
			\item less than 7
		\end{enumerate}
		%\input{exemplar/10/13/3/30/main.tex}
  \item A Lot consists of 48 mobile phones of which 42 are good, 3 have only minor defects and 3 have major defects.Varnika will buy a phone if it is good but the trader will only buy a mobile if it has no major defects. One phone is selected at random from the lot. What is the probability that it is
\begin{enumerate}
	\item acceptable to Varnika?
            \item acceptable to the trader?
\end{enumerate}
\solution
	%\input{exemplar/10/13/3/40/main.tex}
 \item A student says that if you throw a die, it will show up 1 or not 1. Therefore, the probability of getting 1 and the probability of getting 'not 1' each is equal to $\frac{1}{2}$. Is this correct? Give reasons.\\
 \solution
        %\input{exemplar/10/13/2/9/main.tex}
   \item Four candidates A, B, C, D have ap-
plied for the assignment to coach a school cricket
team. If A is twice as likely to be selected as B, and
B and C are given about the same chance of being
selected, while C is twice as likely to be selected
as D, what are the probabilities that
\begin{enumerate}
\item C will be selected?
\item A will not be selected?
\end{enumerate}
	%\input{exemplar/11/16/3/9/main.tex}
 \item A bag contain 24 balls of which $x$ balls are red, $2x$ are white and $3x$ are blue. A ball is selected at random, What is the probability that it is
\begin{enumerate}[label=\alph*)]
\item not red ?
\item white ?
\end{enumerate}
%\input{exemplar/10/13/3/41/main.tex}
If the letters of the word ASSASSINATION are arranged at random. Find the Probability that
\begin{enumerate}[label=(\alph*)]
\item Four $S's$ come consecutively in the word
\item Two  $I's$ and two $N's$ come together
\item All $A's$ are not coming together
\item No two $A's$ are coming together
\end{enumerate}
%\input{exemplar/11/16/3/14/main.tex}
	\item One urn contains two black balls (labelled B1 and B2) and one white ball. A
	second urn contains one black ball and two white balls (labelled W1 and W2).
	Suppose the following experiment is performed. One of the two urns is chosen
	at random. Next a ball is randomly chosen from the urn. Then a second ball is
	chosen at random from the same urn without replacing the first ball.
	
	\begin{enumerate}
	\item What is the probability that two black balls are chosen?
	
	\item What is the probability that two balls of opposite colour are chosen?
	\end{enumerate}
	\solution
	%\input{exemplar/11/16/3/12/main1.tex}
\end{enumerate}

	\item 
The number lock of a suitcase has 4 wheels each labelled with ten digits i.e. from 0 to 9.The lock opens with a sequence of four digits with no repeats.What is the probability of a person getting the right sequence to open the suitcase.
\\
\solution
		%\begin{enumerate}[label=\thesection.\arabic*,ref=\thesection.\theenumi]
	\item One card is drawn from a well-shuffled deck of 52 cards. Find the probability of getting
\begin{enumerate}
\item A king of red colour 
\item A face card 
\item A red face card
\item The jack of hearts
\item A spade
\item The queen of diamonds

\end{enumerate}
\solution
		%\input{ncert/10/15/1/14/main.tex}
	\item Five cards—the ten, jack, queen, king and ace of diamonds, are well-shuffled with their face downwards. One card is then picked up at random.
\begin{enumerate}
\item
What is the probability that the card is the queen? 
\item
If the queen is drawn and put aside, what is the probability that the second card picked up is (a) an ace? (b) a queen?\\
\end{enumerate}
\solution
		%\input{ncert/10/15/1/15/defs.tex}
	\item A bag contains $5$ red balls and some blue balls. If the probability of drawing a blue ball is double that if a red ball, determine the number of blue balls in the bag. 
		\\
\solution
		%\input{ncert/10/15/2/3/defs.tex}
	\item A card is selected from a pack of 52 cards.
 \begin{enumerate}[label=(\alph*)] 
                 \item How many points are there in the sample space?
                 \item Calculate the probability that the card is an ace of spades.
                 \item Calculate the probability that the card is (i) an ace and (ii) black card.
 \end{enumerate}
\solution
		%\input{ncert/11/16/3/4/main.tex}
\item Four cards are drawn from a well-shuffled deck of 52 cards. What is the probability of obtaining 3 diamonds and one spade.
\\
\solution
		%\input{ncert/11/16/4/2/defs.tex}
\item In a certain lottery 10,000 tickets are sold and ten equal prizes are awarded. What is the probability of not getting a prize if you buy (a) one ticket (b) two tickets (c) 10 tickets ?	
\\
\solution
		%\input{ncert/11/16/4/4/defs.tex}
		%
\item 
Out of 100 students, two sections of 40 and 60 are formed. If you and your friend are among the 100 students, what is the probability that
\begin{enumerate}
\item you both enter the same section?
\item you both enter the different sections?
\end{enumerate}
\solution
		%\input{ncert/11/16/4/5/defs.tex}
	\item 
The number lock of a suitcase has 4 wheels each labelled with ten digits i.e. from 0 to 9.The lock opens with a sequence of four digits with no repeats.What is the probability of a person getting the right sequence to open the suitcase.
\\
\solution
		%\input{ncert/11/16/4/10/defs.tex}
		%
\item 
Two cards are drawn at random and without replacement from a pack of 52 playing cards. Find the probability that both the cards are black.
\\
\solution
		%\input{ncert/12/13/2/2/defs.tex}
		\item A box of oranges is inspected by examining three randomly selected oranges drawn without replacement. If all the three oranges are good, the box is approved for sale, otherwise, it is rejected. Find the probability that a box containing 15 oranges out of which 12 are good and 3 are bad ones will be approved for sale.
		\label{ncert/12/13/2/3/defs.tex}
		\item Two balls are drawn at random with replacement from a box containing 10 black and 8 red balls. Find the probability that
		\label{ncert/12/13/2/12}
\begin{enumerate}
\item both balls are red.
\item first ball is black and second is red.
\item one of them is black and other is red.
\end{enumerate}

\item In a hostel, 60\% of the students read Hindi newspaper, 40\% read English newspaper and 20\% read both Hindi and English newspapers. A student is selected at random.
		\label{ncert/12/13/2/15}
\begin{enumerate}
\item Find the probability that she reads neither Hindi nor English newspapers.
\item If she reads Hindi newspaper, find the probability that she reads English newspaper.
\item If she reads English newspaper, find the probability that she reads Hindi newspaper.\\
\end{enumerate}
\item The probability of obtaining an even prime number on each die, when a pair of dice is rolled is 
\begin{enumerate}
    \item $0$ 
    
    \item $\frac{1}{3}$ 
    
    \item $\frac{1}{12}$ 
    
    \item $\frac{1}{36}$ 
\end{enumerate}
\solution
		%\input{ncert/12/13/2/17/defs.tex}
	\item A bag contains 4 red and 4 black balls, another bag contains 2 red and 6 black balls. One of the two bags is selected at random and a ball is drawn from the bag which is found to be red. Find the probability that the ball is drawn from the first bag.
\\
\solution
		%\input{ncert/12/13/3/2/main.tex}
  \item
  Cards with numbers 2 to 101 are placed in a box. A card is selected at random.Find the probability that the card has
\begin{enumerate}[label=(\roman*)]
	\item an even number 
	\item a square number
\end{enumerate}
\solution
%\input{exemplar/10/13/3/32/main.tex}
\item
The king, queen and jack of clubs are removed from a deck of 52 playing cards and then well shuffled. Now one card is drawn at random from the remaining cards.  Determine the probability that the card is
\begin{enumerate}[label=(\roman*)]
\item a club
\item 10 of hearts
\end{enumerate}
\solution
%\input{exemplar/10/13/3/29/main.tex}
\item A team of medical students doing their internship have to assist during surgeries
at a city hospital. The probabilities of surgeries rated as very complex, complex,
routine, simple or very simple are respectively, 0.15, 0.20, 0.31, 0.26, .08. Find
the probabilities that a particular surgery will be rated
\begin{enumerate}
	\item complex or very complex;
	\item neither very complex nor very simple;
	\item routine or complex
	\item routine or simple
\end{enumerate}
\solution
%\input{exemplar/11/16/3/8(1)/main.tex}
\item A card is selected from a pack of 52 cards.
\begin{enumerate}[label=(\alph*)]
    \item How many points are there in the sample space?
    \item Calculate the probability that the card is an ace of spades.
    \item Calculate the probability that the card is (i) an ace and (ii) black card.
\end{enumerate}
\solution
%\input{exemplar/11/16/3/4/main2.tex}
\item The probability that a non leap year selected at random will contain 53 sundays.
\\
\solution
%\input{exemplar/10/13/1/19/main.tex}
\item One of the four persons John, Rita, Aslam or Gurpreet will be promoted next
month. Consequently the sample space consists of four elementary outcomes
S = {John promoted, Rita promoted, Aslam promoted, Gurpreet promoted}
You are told that the chances of John’s promotion is same as that of Gurpreet,
Rita’s chances of promotion are twice as likely as Johns. Aslam’s chances are
four times that of John.
\begin{enumerate}
	\item Determine
	\begin{enumerate}
		\item P (John promoted)
		\item P (Rita promoted)
		\item P (Aslam promoted)
		\item P (Gurpreet promoted)
	\end{enumerate}
	\item If A = {John promoted or Gurpreet promoted}, find P (A).
\end{enumerate}
\solution
%\input{exemplar/11/16/3/10/main.tex}
\item A card is drawn from a deck of 52 cards. Find the probability of getting a king or a heart or a red card.\\
\solution
%\input{exemplar/11/16/3/15/main.tex}
\item The probability that a student will pass his examination is 0.73, the probability of
the student getting a compartment is 0.13, and the probability that the student will
either pass or get compartment is 0.96. State True or False.\\
\solution
%\input{exemplar/11/16/3/31/main.tex}
\item A card is selected from a pack of 52 cards\\
\begin{enumerate}[label=(\alph*)]
\item How many points are there in the sample space?
\item Calculate the probability that the cards is an ace of spades.
\item Calculate the probability that the card is (i) an ace (ii)black card.\\
\end{enumerate}
%\input{ncert/11/16/3/4_1/Prob_4.tex}
\item In a non-leap year, the probability of having 53 tuesdays or 53 wednesdays is\\
\solution
%\input{exemplar/11/16/3/18/main.tex}
\item There are 1000 sealed envelopes in a box, 10 of them contain a cash prize of
Rs 100 each, 100 of them contain a cash prize of Rs 50 each and 200 of them
contain a cash prize of Rs 10 each and rest do not contain any cash prize. If they
are well shuffled and an envelope is picked up out, what is the probability that it
contains no cash prize?\\
\solution
%\input{exemplar/10/13/3/34/main.tex}
\item 
A die is thrown and a card is selected at random from a deck of 52 playing cards. The probability of getting an even number on the die and a spade card.\\
\solution
%\input{exemplar/12/13/3/78/main.tex}
\item
If 4-digit numbers greater than 5,000 are randomly formed from the digits 0, 1, 3, 5, and 7, what is the probability of forming a number divisible by 5 when:
\begin{enumerate}
    \item The digits are repeated?
    \item The repetition of digits is not allowed?
\end{enumerate}
\solution
%\input{ncert/11/16/4/9/main.tex}
\item Consider the probability space $\brak{\Omega, \mathcal{G}, P}$ where $\Omega = [0,2]$ and $\mathcal{G} = \cbrak{\phi, \Omega, [0,1], (1,2]}$. Let $X$ and $Y$ be two functions on $\Omega$ defined as
\begin{align*}
    X(\omega) = 
    \begin{cases}
        1 & \text{if }\omega \in [0, 1]\\
        2 & \text{if }\omega \in (1, 2]
    \end{cases}
\end{align*}
and
\begin{align*}
    Y(\omega) = 
    \begin{cases}
        2 & \text{if }\omega \in [0, 1.5]\\
        3 & \text{if }\omega \in (1.5, 2].
    \end{cases}
\end{align*}
Then which one of the following statements is true?
\begin{enumerate}
    \item [(A)] $X$ is a random variable with respect to $\mathcal{G}$, but $Y$ is not a random variable with respect to $\mathcal{G}$.
    \item [(B)] $Y$ is a random variable with respect to $\mathcal{G}$, but $X$ is not a random variable with respect to $\mathcal{G}$.
    \item [(C)] Neither $X$ nor $Y$ is a random variable with respect to $\mathcal{G}$.
    \item [(D)] Both $X$ and $Y$ are random variables with respect to $\mathcal{G}$.
\end{enumerate} \hfill (GATE ST 2023)\\
\solution
%\input{gate/ST/2023/14/main.tex}
	\item  A die is loaded in such a way that each odd number is twice as likely to occur as
each even number. Find $P(G)$, where $G$ is the event that a number greater than
3 occurs on a single roll of the die.
\\
\solution
		%\input{exemplar/11/16/3/5/main.tex}
	\item All the jacks, queens and kings are removed from a deck of 52 playing cards. The remaining cards are well shuffled and then one card is drawn at random. Giving ace a value 1 similar value for other cards, find the probability that the card has a value 
		\begin{enumerate}
			\item 7
			\item greater than 7
			\item less than 7
		\end{enumerate}
		%\input{exemplar/10/13/3/30/main.tex}
  \item A Lot consists of 48 mobile phones of which 42 are good, 3 have only minor defects and 3 have major defects.Varnika will buy a phone if it is good but the trader will only buy a mobile if it has no major defects. One phone is selected at random from the lot. What is the probability that it is
\begin{enumerate}
	\item acceptable to Varnika?
            \item acceptable to the trader?
\end{enumerate}
\solution
	%\input{exemplar/10/13/3/40/main.tex}
 \item A student says that if you throw a die, it will show up 1 or not 1. Therefore, the probability of getting 1 and the probability of getting 'not 1' each is equal to $\frac{1}{2}$. Is this correct? Give reasons.\\
 \solution
        %\input{exemplar/10/13/2/9/main.tex}
   \item Four candidates A, B, C, D have ap-
plied for the assignment to coach a school cricket
team. If A is twice as likely to be selected as B, and
B and C are given about the same chance of being
selected, while C is twice as likely to be selected
as D, what are the probabilities that
\begin{enumerate}
\item C will be selected?
\item A will not be selected?
\end{enumerate}
	%\input{exemplar/11/16/3/9/main.tex}
 \item A bag contain 24 balls of which $x$ balls are red, $2x$ are white and $3x$ are blue. A ball is selected at random, What is the probability that it is
\begin{enumerate}[label=\alph*)]
\item not red ?
\item white ?
\end{enumerate}
%\input{exemplar/10/13/3/41/main.tex}
If the letters of the word ASSASSINATION are arranged at random. Find the Probability that
\begin{enumerate}[label=(\alph*)]
\item Four $S's$ come consecutively in the word
\item Two  $I's$ and two $N's$ come together
\item All $A's$ are not coming together
\item No two $A's$ are coming together
\end{enumerate}
%\input{exemplar/11/16/3/14/main.tex}
	\item One urn contains two black balls (labelled B1 and B2) and one white ball. A
	second urn contains one black ball and two white balls (labelled W1 and W2).
	Suppose the following experiment is performed. One of the two urns is chosen
	at random. Next a ball is randomly chosen from the urn. Then a second ball is
	chosen at random from the same urn without replacing the first ball.
	
	\begin{enumerate}
	\item What is the probability that two black balls are chosen?
	
	\item What is the probability that two balls of opposite colour are chosen?
	\end{enumerate}
	\solution
	%\input{exemplar/11/16/3/12/main1.tex}
\end{enumerate}

		%
\item 
Two cards are drawn at random and without replacement from a pack of 52 playing cards. Find the probability that both the cards are black.
\\
\solution
		%\begin{enumerate}[label=\thesection.\arabic*,ref=\thesection.\theenumi]
	\item One card is drawn from a well-shuffled deck of 52 cards. Find the probability of getting
\begin{enumerate}
\item A king of red colour 
\item A face card 
\item A red face card
\item The jack of hearts
\item A spade
\item The queen of diamonds

\end{enumerate}
\solution
		%\input{ncert/10/15/1/14/main.tex}
	\item Five cards—the ten, jack, queen, king and ace of diamonds, are well-shuffled with their face downwards. One card is then picked up at random.
\begin{enumerate}
\item
What is the probability that the card is the queen? 
\item
If the queen is drawn and put aside, what is the probability that the second card picked up is (a) an ace? (b) a queen?\\
\end{enumerate}
\solution
		%\input{ncert/10/15/1/15/defs.tex}
	\item A bag contains $5$ red balls and some blue balls. If the probability of drawing a blue ball is double that if a red ball, determine the number of blue balls in the bag. 
		\\
\solution
		%\input{ncert/10/15/2/3/defs.tex}
	\item A card is selected from a pack of 52 cards.
 \begin{enumerate}[label=(\alph*)] 
                 \item How many points are there in the sample space?
                 \item Calculate the probability that the card is an ace of spades.
                 \item Calculate the probability that the card is (i) an ace and (ii) black card.
 \end{enumerate}
\solution
		%\input{ncert/11/16/3/4/main.tex}
\item Four cards are drawn from a well-shuffled deck of 52 cards. What is the probability of obtaining 3 diamonds and one spade.
\\
\solution
		%\input{ncert/11/16/4/2/defs.tex}
\item In a certain lottery 10,000 tickets are sold and ten equal prizes are awarded. What is the probability of not getting a prize if you buy (a) one ticket (b) two tickets (c) 10 tickets ?	
\\
\solution
		%\input{ncert/11/16/4/4/defs.tex}
		%
\item 
Out of 100 students, two sections of 40 and 60 are formed. If you and your friend are among the 100 students, what is the probability that
\begin{enumerate}
\item you both enter the same section?
\item you both enter the different sections?
\end{enumerate}
\solution
		%\input{ncert/11/16/4/5/defs.tex}
	\item 
The number lock of a suitcase has 4 wheels each labelled with ten digits i.e. from 0 to 9.The lock opens with a sequence of four digits with no repeats.What is the probability of a person getting the right sequence to open the suitcase.
\\
\solution
		%\input{ncert/11/16/4/10/defs.tex}
		%
\item 
Two cards are drawn at random and without replacement from a pack of 52 playing cards. Find the probability that both the cards are black.
\\
\solution
		%\input{ncert/12/13/2/2/defs.tex}
		\item A box of oranges is inspected by examining three randomly selected oranges drawn without replacement. If all the three oranges are good, the box is approved for sale, otherwise, it is rejected. Find the probability that a box containing 15 oranges out of which 12 are good and 3 are bad ones will be approved for sale.
		\label{ncert/12/13/2/3/defs.tex}
		\item Two balls are drawn at random with replacement from a box containing 10 black and 8 red balls. Find the probability that
		\label{ncert/12/13/2/12}
\begin{enumerate}
\item both balls are red.
\item first ball is black and second is red.
\item one of them is black and other is red.
\end{enumerate}

\item In a hostel, 60\% of the students read Hindi newspaper, 40\% read English newspaper and 20\% read both Hindi and English newspapers. A student is selected at random.
		\label{ncert/12/13/2/15}
\begin{enumerate}
\item Find the probability that she reads neither Hindi nor English newspapers.
\item If she reads Hindi newspaper, find the probability that she reads English newspaper.
\item If she reads English newspaper, find the probability that she reads Hindi newspaper.\\
\end{enumerate}
\item The probability of obtaining an even prime number on each die, when a pair of dice is rolled is 
\begin{enumerate}
    \item $0$ 
    
    \item $\frac{1}{3}$ 
    
    \item $\frac{1}{12}$ 
    
    \item $\frac{1}{36}$ 
\end{enumerate}
\solution
		%\input{ncert/12/13/2/17/defs.tex}
	\item A bag contains 4 red and 4 black balls, another bag contains 2 red and 6 black balls. One of the two bags is selected at random and a ball is drawn from the bag which is found to be red. Find the probability that the ball is drawn from the first bag.
\\
\solution
		%\input{ncert/12/13/3/2/main.tex}
  \item
  Cards with numbers 2 to 101 are placed in a box. A card is selected at random.Find the probability that the card has
\begin{enumerate}[label=(\roman*)]
	\item an even number 
	\item a square number
\end{enumerate}
\solution
%\input{exemplar/10/13/3/32/main.tex}
\item
The king, queen and jack of clubs are removed from a deck of 52 playing cards and then well shuffled. Now one card is drawn at random from the remaining cards.  Determine the probability that the card is
\begin{enumerate}[label=(\roman*)]
\item a club
\item 10 of hearts
\end{enumerate}
\solution
%\input{exemplar/10/13/3/29/main.tex}
\item A team of medical students doing their internship have to assist during surgeries
at a city hospital. The probabilities of surgeries rated as very complex, complex,
routine, simple or very simple are respectively, 0.15, 0.20, 0.31, 0.26, .08. Find
the probabilities that a particular surgery will be rated
\begin{enumerate}
	\item complex or very complex;
	\item neither very complex nor very simple;
	\item routine or complex
	\item routine or simple
\end{enumerate}
\solution
%\input{exemplar/11/16/3/8(1)/main.tex}
\item A card is selected from a pack of 52 cards.
\begin{enumerate}[label=(\alph*)]
    \item How many points are there in the sample space?
    \item Calculate the probability that the card is an ace of spades.
    \item Calculate the probability that the card is (i) an ace and (ii) black card.
\end{enumerate}
\solution
%\input{exemplar/11/16/3/4/main2.tex}
\item The probability that a non leap year selected at random will contain 53 sundays.
\\
\solution
%\input{exemplar/10/13/1/19/main.tex}
\item One of the four persons John, Rita, Aslam or Gurpreet will be promoted next
month. Consequently the sample space consists of four elementary outcomes
S = {John promoted, Rita promoted, Aslam promoted, Gurpreet promoted}
You are told that the chances of John’s promotion is same as that of Gurpreet,
Rita’s chances of promotion are twice as likely as Johns. Aslam’s chances are
four times that of John.
\begin{enumerate}
	\item Determine
	\begin{enumerate}
		\item P (John promoted)
		\item P (Rita promoted)
		\item P (Aslam promoted)
		\item P (Gurpreet promoted)
	\end{enumerate}
	\item If A = {John promoted or Gurpreet promoted}, find P (A).
\end{enumerate}
\solution
%\input{exemplar/11/16/3/10/main.tex}
\item A card is drawn from a deck of 52 cards. Find the probability of getting a king or a heart or a red card.\\
\solution
%\input{exemplar/11/16/3/15/main.tex}
\item The probability that a student will pass his examination is 0.73, the probability of
the student getting a compartment is 0.13, and the probability that the student will
either pass or get compartment is 0.96. State True or False.\\
\solution
%\input{exemplar/11/16/3/31/main.tex}
\item A card is selected from a pack of 52 cards\\
\begin{enumerate}[label=(\alph*)]
\item How many points are there in the sample space?
\item Calculate the probability that the cards is an ace of spades.
\item Calculate the probability that the card is (i) an ace (ii)black card.\\
\end{enumerate}
%\input{ncert/11/16/3/4_1/Prob_4.tex}
\item In a non-leap year, the probability of having 53 tuesdays or 53 wednesdays is\\
\solution
%\input{exemplar/11/16/3/18/main.tex}
\item There are 1000 sealed envelopes in a box, 10 of them contain a cash prize of
Rs 100 each, 100 of them contain a cash prize of Rs 50 each and 200 of them
contain a cash prize of Rs 10 each and rest do not contain any cash prize. If they
are well shuffled and an envelope is picked up out, what is the probability that it
contains no cash prize?\\
\solution
%\input{exemplar/10/13/3/34/main.tex}
\item 
A die is thrown and a card is selected at random from a deck of 52 playing cards. The probability of getting an even number on the die and a spade card.\\
\solution
%\input{exemplar/12/13/3/78/main.tex}
\item
If 4-digit numbers greater than 5,000 are randomly formed from the digits 0, 1, 3, 5, and 7, what is the probability of forming a number divisible by 5 when:
\begin{enumerate}
    \item The digits are repeated?
    \item The repetition of digits is not allowed?
\end{enumerate}
\solution
%\input{ncert/11/16/4/9/main.tex}
\item Consider the probability space $\brak{\Omega, \mathcal{G}, P}$ where $\Omega = [0,2]$ and $\mathcal{G} = \cbrak{\phi, \Omega, [0,1], (1,2]}$. Let $X$ and $Y$ be two functions on $\Omega$ defined as
\begin{align*}
    X(\omega) = 
    \begin{cases}
        1 & \text{if }\omega \in [0, 1]\\
        2 & \text{if }\omega \in (1, 2]
    \end{cases}
\end{align*}
and
\begin{align*}
    Y(\omega) = 
    \begin{cases}
        2 & \text{if }\omega \in [0, 1.5]\\
        3 & \text{if }\omega \in (1.5, 2].
    \end{cases}
\end{align*}
Then which one of the following statements is true?
\begin{enumerate}
    \item [(A)] $X$ is a random variable with respect to $\mathcal{G}$, but $Y$ is not a random variable with respect to $\mathcal{G}$.
    \item [(B)] $Y$ is a random variable with respect to $\mathcal{G}$, but $X$ is not a random variable with respect to $\mathcal{G}$.
    \item [(C)] Neither $X$ nor $Y$ is a random variable with respect to $\mathcal{G}$.
    \item [(D)] Both $X$ and $Y$ are random variables with respect to $\mathcal{G}$.
\end{enumerate} \hfill (GATE ST 2023)\\
\solution
%\input{gate/ST/2023/14/main.tex}
	\item  A die is loaded in such a way that each odd number is twice as likely to occur as
each even number. Find $P(G)$, where $G$ is the event that a number greater than
3 occurs on a single roll of the die.
\\
\solution
		%\input{exemplar/11/16/3/5/main.tex}
	\item All the jacks, queens and kings are removed from a deck of 52 playing cards. The remaining cards are well shuffled and then one card is drawn at random. Giving ace a value 1 similar value for other cards, find the probability that the card has a value 
		\begin{enumerate}
			\item 7
			\item greater than 7
			\item less than 7
		\end{enumerate}
		%\input{exemplar/10/13/3/30/main.tex}
  \item A Lot consists of 48 mobile phones of which 42 are good, 3 have only minor defects and 3 have major defects.Varnika will buy a phone if it is good but the trader will only buy a mobile if it has no major defects. One phone is selected at random from the lot. What is the probability that it is
\begin{enumerate}
	\item acceptable to Varnika?
            \item acceptable to the trader?
\end{enumerate}
\solution
	%\input{exemplar/10/13/3/40/main.tex}
 \item A student says that if you throw a die, it will show up 1 or not 1. Therefore, the probability of getting 1 and the probability of getting 'not 1' each is equal to $\frac{1}{2}$. Is this correct? Give reasons.\\
 \solution
        %\input{exemplar/10/13/2/9/main.tex}
   \item Four candidates A, B, C, D have ap-
plied for the assignment to coach a school cricket
team. If A is twice as likely to be selected as B, and
B and C are given about the same chance of being
selected, while C is twice as likely to be selected
as D, what are the probabilities that
\begin{enumerate}
\item C will be selected?
\item A will not be selected?
\end{enumerate}
	%\input{exemplar/11/16/3/9/main.tex}
 \item A bag contain 24 balls of which $x$ balls are red, $2x$ are white and $3x$ are blue. A ball is selected at random, What is the probability that it is
\begin{enumerate}[label=\alph*)]
\item not red ?
\item white ?
\end{enumerate}
%\input{exemplar/10/13/3/41/main.tex}
If the letters of the word ASSASSINATION are arranged at random. Find the Probability that
\begin{enumerate}[label=(\alph*)]
\item Four $S's$ come consecutively in the word
\item Two  $I's$ and two $N's$ come together
\item All $A's$ are not coming together
\item No two $A's$ are coming together
\end{enumerate}
%\input{exemplar/11/16/3/14/main.tex}
	\item One urn contains two black balls (labelled B1 and B2) and one white ball. A
	second urn contains one black ball and two white balls (labelled W1 and W2).
	Suppose the following experiment is performed. One of the two urns is chosen
	at random. Next a ball is randomly chosen from the urn. Then a second ball is
	chosen at random from the same urn without replacing the first ball.
	
	\begin{enumerate}
	\item What is the probability that two black balls are chosen?
	
	\item What is the probability that two balls of opposite colour are chosen?
	\end{enumerate}
	\solution
	%\input{exemplar/11/16/3/12/main1.tex}
\end{enumerate}

		\item A box of oranges is inspected by examining three randomly selected oranges drawn without replacement. If all the three oranges are good, the box is approved for sale, otherwise, it is rejected. Find the probability that a box containing 15 oranges out of which 12 are good and 3 are bad ones will be approved for sale.
		\label{ncert/12/13/2/3/defs.tex}
		\item Two balls are drawn at random with replacement from a box containing 10 black and 8 red balls. Find the probability that
		\label{ncert/12/13/2/12}
\begin{enumerate}
\item both balls are red.
\item first ball is black and second is red.
\item one of them is black and other is red.
\end{enumerate}

\item In a hostel, 60\% of the students read Hindi newspaper, 40\% read English newspaper and 20\% read both Hindi and English newspapers. A student is selected at random.
		\label{ncert/12/13/2/15}
\begin{enumerate}
\item Find the probability that she reads neither Hindi nor English newspapers.
\item If she reads Hindi newspaper, find the probability that she reads English newspaper.
\item If she reads English newspaper, find the probability that she reads Hindi newspaper.\\
\end{enumerate}
\item The probability of obtaining an even prime number on each die, when a pair of dice is rolled is 
\begin{enumerate}
    \item $0$ 
    
    \item $\frac{1}{3}$ 
    
    \item $\frac{1}{12}$ 
    
    \item $\frac{1}{36}$ 
\end{enumerate}
\solution
		%\begin{enumerate}[label=\thesection.\arabic*,ref=\thesection.\theenumi]
	\item One card is drawn from a well-shuffled deck of 52 cards. Find the probability of getting
\begin{enumerate}
\item A king of red colour 
\item A face card 
\item A red face card
\item The jack of hearts
\item A spade
\item The queen of diamonds

\end{enumerate}
\solution
		%\input{ncert/10/15/1/14/main.tex}
	\item Five cards—the ten, jack, queen, king and ace of diamonds, are well-shuffled with their face downwards. One card is then picked up at random.
\begin{enumerate}
\item
What is the probability that the card is the queen? 
\item
If the queen is drawn and put aside, what is the probability that the second card picked up is (a) an ace? (b) a queen?\\
\end{enumerate}
\solution
		%\input{ncert/10/15/1/15/defs.tex}
	\item A bag contains $5$ red balls and some blue balls. If the probability of drawing a blue ball is double that if a red ball, determine the number of blue balls in the bag. 
		\\
\solution
		%\input{ncert/10/15/2/3/defs.tex}
	\item A card is selected from a pack of 52 cards.
 \begin{enumerate}[label=(\alph*)] 
                 \item How many points are there in the sample space?
                 \item Calculate the probability that the card is an ace of spades.
                 \item Calculate the probability that the card is (i) an ace and (ii) black card.
 \end{enumerate}
\solution
		%\input{ncert/11/16/3/4/main.tex}
\item Four cards are drawn from a well-shuffled deck of 52 cards. What is the probability of obtaining 3 diamonds and one spade.
\\
\solution
		%\input{ncert/11/16/4/2/defs.tex}
\item In a certain lottery 10,000 tickets are sold and ten equal prizes are awarded. What is the probability of not getting a prize if you buy (a) one ticket (b) two tickets (c) 10 tickets ?	
\\
\solution
		%\input{ncert/11/16/4/4/defs.tex}
		%
\item 
Out of 100 students, two sections of 40 and 60 are formed. If you and your friend are among the 100 students, what is the probability that
\begin{enumerate}
\item you both enter the same section?
\item you both enter the different sections?
\end{enumerate}
\solution
		%\input{ncert/11/16/4/5/defs.tex}
	\item 
The number lock of a suitcase has 4 wheels each labelled with ten digits i.e. from 0 to 9.The lock opens with a sequence of four digits with no repeats.What is the probability of a person getting the right sequence to open the suitcase.
\\
\solution
		%\input{ncert/11/16/4/10/defs.tex}
		%
\item 
Two cards are drawn at random and without replacement from a pack of 52 playing cards. Find the probability that both the cards are black.
\\
\solution
		%\input{ncert/12/13/2/2/defs.tex}
		\item A box of oranges is inspected by examining three randomly selected oranges drawn without replacement. If all the three oranges are good, the box is approved for sale, otherwise, it is rejected. Find the probability that a box containing 15 oranges out of which 12 are good and 3 are bad ones will be approved for sale.
		\label{ncert/12/13/2/3/defs.tex}
		\item Two balls are drawn at random with replacement from a box containing 10 black and 8 red balls. Find the probability that
		\label{ncert/12/13/2/12}
\begin{enumerate}
\item both balls are red.
\item first ball is black and second is red.
\item one of them is black and other is red.
\end{enumerate}

\item In a hostel, 60\% of the students read Hindi newspaper, 40\% read English newspaper and 20\% read both Hindi and English newspapers. A student is selected at random.
		\label{ncert/12/13/2/15}
\begin{enumerate}
\item Find the probability that she reads neither Hindi nor English newspapers.
\item If she reads Hindi newspaper, find the probability that she reads English newspaper.
\item If she reads English newspaper, find the probability that she reads Hindi newspaper.\\
\end{enumerate}
\item The probability of obtaining an even prime number on each die, when a pair of dice is rolled is 
\begin{enumerate}
    \item $0$ 
    
    \item $\frac{1}{3}$ 
    
    \item $\frac{1}{12}$ 
    
    \item $\frac{1}{36}$ 
\end{enumerate}
\solution
		%\input{ncert/12/13/2/17/defs.tex}
	\item A bag contains 4 red and 4 black balls, another bag contains 2 red and 6 black balls. One of the two bags is selected at random and a ball is drawn from the bag which is found to be red. Find the probability that the ball is drawn from the first bag.
\\
\solution
		%\input{ncert/12/13/3/2/main.tex}
  \item
  Cards with numbers 2 to 101 are placed in a box. A card is selected at random.Find the probability that the card has
\begin{enumerate}[label=(\roman*)]
	\item an even number 
	\item a square number
\end{enumerate}
\solution
%\input{exemplar/10/13/3/32/main.tex}
\item
The king, queen and jack of clubs are removed from a deck of 52 playing cards and then well shuffled. Now one card is drawn at random from the remaining cards.  Determine the probability that the card is
\begin{enumerate}[label=(\roman*)]
\item a club
\item 10 of hearts
\end{enumerate}
\solution
%\input{exemplar/10/13/3/29/main.tex}
\item A team of medical students doing their internship have to assist during surgeries
at a city hospital. The probabilities of surgeries rated as very complex, complex,
routine, simple or very simple are respectively, 0.15, 0.20, 0.31, 0.26, .08. Find
the probabilities that a particular surgery will be rated
\begin{enumerate}
	\item complex or very complex;
	\item neither very complex nor very simple;
	\item routine or complex
	\item routine or simple
\end{enumerate}
\solution
%\input{exemplar/11/16/3/8(1)/main.tex}
\item A card is selected from a pack of 52 cards.
\begin{enumerate}[label=(\alph*)]
    \item How many points are there in the sample space?
    \item Calculate the probability that the card is an ace of spades.
    \item Calculate the probability that the card is (i) an ace and (ii) black card.
\end{enumerate}
\solution
%\input{exemplar/11/16/3/4/main2.tex}
\item The probability that a non leap year selected at random will contain 53 sundays.
\\
\solution
%\input{exemplar/10/13/1/19/main.tex}
\item One of the four persons John, Rita, Aslam or Gurpreet will be promoted next
month. Consequently the sample space consists of four elementary outcomes
S = {John promoted, Rita promoted, Aslam promoted, Gurpreet promoted}
You are told that the chances of John’s promotion is same as that of Gurpreet,
Rita’s chances of promotion are twice as likely as Johns. Aslam’s chances are
four times that of John.
\begin{enumerate}
	\item Determine
	\begin{enumerate}
		\item P (John promoted)
		\item P (Rita promoted)
		\item P (Aslam promoted)
		\item P (Gurpreet promoted)
	\end{enumerate}
	\item If A = {John promoted or Gurpreet promoted}, find P (A).
\end{enumerate}
\solution
%\input{exemplar/11/16/3/10/main.tex}
\item A card is drawn from a deck of 52 cards. Find the probability of getting a king or a heart or a red card.\\
\solution
%\input{exemplar/11/16/3/15/main.tex}
\item The probability that a student will pass his examination is 0.73, the probability of
the student getting a compartment is 0.13, and the probability that the student will
either pass or get compartment is 0.96. State True or False.\\
\solution
%\input{exemplar/11/16/3/31/main.tex}
\item A card is selected from a pack of 52 cards\\
\begin{enumerate}[label=(\alph*)]
\item How many points are there in the sample space?
\item Calculate the probability that the cards is an ace of spades.
\item Calculate the probability that the card is (i) an ace (ii)black card.\\
\end{enumerate}
%\input{ncert/11/16/3/4_1/Prob_4.tex}
\item In a non-leap year, the probability of having 53 tuesdays or 53 wednesdays is\\
\solution
%\input{exemplar/11/16/3/18/main.tex}
\item There are 1000 sealed envelopes in a box, 10 of them contain a cash prize of
Rs 100 each, 100 of them contain a cash prize of Rs 50 each and 200 of them
contain a cash prize of Rs 10 each and rest do not contain any cash prize. If they
are well shuffled and an envelope is picked up out, what is the probability that it
contains no cash prize?\\
\solution
%\input{exemplar/10/13/3/34/main.tex}
\item 
A die is thrown and a card is selected at random from a deck of 52 playing cards. The probability of getting an even number on the die and a spade card.\\
\solution
%\input{exemplar/12/13/3/78/main.tex}
\item
If 4-digit numbers greater than 5,000 are randomly formed from the digits 0, 1, 3, 5, and 7, what is the probability of forming a number divisible by 5 when:
\begin{enumerate}
    \item The digits are repeated?
    \item The repetition of digits is not allowed?
\end{enumerate}
\solution
%\input{ncert/11/16/4/9/main.tex}
\item Consider the probability space $\brak{\Omega, \mathcal{G}, P}$ where $\Omega = [0,2]$ and $\mathcal{G} = \cbrak{\phi, \Omega, [0,1], (1,2]}$. Let $X$ and $Y$ be two functions on $\Omega$ defined as
\begin{align*}
    X(\omega) = 
    \begin{cases}
        1 & \text{if }\omega \in [0, 1]\\
        2 & \text{if }\omega \in (1, 2]
    \end{cases}
\end{align*}
and
\begin{align*}
    Y(\omega) = 
    \begin{cases}
        2 & \text{if }\omega \in [0, 1.5]\\
        3 & \text{if }\omega \in (1.5, 2].
    \end{cases}
\end{align*}
Then which one of the following statements is true?
\begin{enumerate}
    \item [(A)] $X$ is a random variable with respect to $\mathcal{G}$, but $Y$ is not a random variable with respect to $\mathcal{G}$.
    \item [(B)] $Y$ is a random variable with respect to $\mathcal{G}$, but $X$ is not a random variable with respect to $\mathcal{G}$.
    \item [(C)] Neither $X$ nor $Y$ is a random variable with respect to $\mathcal{G}$.
    \item [(D)] Both $X$ and $Y$ are random variables with respect to $\mathcal{G}$.
\end{enumerate} \hfill (GATE ST 2023)\\
\solution
%\input{gate/ST/2023/14/main.tex}
	\item  A die is loaded in such a way that each odd number is twice as likely to occur as
each even number. Find $P(G)$, where $G$ is the event that a number greater than
3 occurs on a single roll of the die.
\\
\solution
		%\input{exemplar/11/16/3/5/main.tex}
	\item All the jacks, queens and kings are removed from a deck of 52 playing cards. The remaining cards are well shuffled and then one card is drawn at random. Giving ace a value 1 similar value for other cards, find the probability that the card has a value 
		\begin{enumerate}
			\item 7
			\item greater than 7
			\item less than 7
		\end{enumerate}
		%\input{exemplar/10/13/3/30/main.tex}
  \item A Lot consists of 48 mobile phones of which 42 are good, 3 have only minor defects and 3 have major defects.Varnika will buy a phone if it is good but the trader will only buy a mobile if it has no major defects. One phone is selected at random from the lot. What is the probability that it is
\begin{enumerate}
	\item acceptable to Varnika?
            \item acceptable to the trader?
\end{enumerate}
\solution
	%\input{exemplar/10/13/3/40/main.tex}
 \item A student says that if you throw a die, it will show up 1 or not 1. Therefore, the probability of getting 1 and the probability of getting 'not 1' each is equal to $\frac{1}{2}$. Is this correct? Give reasons.\\
 \solution
        %\input{exemplar/10/13/2/9/main.tex}
   \item Four candidates A, B, C, D have ap-
plied for the assignment to coach a school cricket
team. If A is twice as likely to be selected as B, and
B and C are given about the same chance of being
selected, while C is twice as likely to be selected
as D, what are the probabilities that
\begin{enumerate}
\item C will be selected?
\item A will not be selected?
\end{enumerate}
	%\input{exemplar/11/16/3/9/main.tex}
 \item A bag contain 24 balls of which $x$ balls are red, $2x$ are white and $3x$ are blue. A ball is selected at random, What is the probability that it is
\begin{enumerate}[label=\alph*)]
\item not red ?
\item white ?
\end{enumerate}
%\input{exemplar/10/13/3/41/main.tex}
If the letters of the word ASSASSINATION are arranged at random. Find the Probability that
\begin{enumerate}[label=(\alph*)]
\item Four $S's$ come consecutively in the word
\item Two  $I's$ and two $N's$ come together
\item All $A's$ are not coming together
\item No two $A's$ are coming together
\end{enumerate}
%\input{exemplar/11/16/3/14/main.tex}
	\item One urn contains two black balls (labelled B1 and B2) and one white ball. A
	second urn contains one black ball and two white balls (labelled W1 and W2).
	Suppose the following experiment is performed. One of the two urns is chosen
	at random. Next a ball is randomly chosen from the urn. Then a second ball is
	chosen at random from the same urn without replacing the first ball.
	
	\begin{enumerate}
	\item What is the probability that two black balls are chosen?
	
	\item What is the probability that two balls of opposite colour are chosen?
	\end{enumerate}
	\solution
	%\input{exemplar/11/16/3/12/main1.tex}
\end{enumerate}

	\item A bag contains 4 red and 4 black balls, another bag contains 2 red and 6 black balls. One of the two bags is selected at random and a ball is drawn from the bag which is found to be red. Find the probability that the ball is drawn from the first bag.
\\
\solution
		%\begin{table}[H]
	\centering
\begin{tabular}{|c|c|c|}
\hline
Random variable &Value &Definition\\ \hline
\multirow{3}{*}{X} &0 &Slips of Rs 1\\
&1 &Slips of Rs 5\\
&2 &Slips of Rs 13\\ \hline
\multirow{2}{*}{Y} &0 &Box A\\
&1 &Box B\\\hline
\end{tabular}
\caption{}
\label{tab:Distribution}
\end{table}
See \tabref{tab:Distribution}.
\begin{align}
p_{Y}\brak{k}= \begin{cases} 
      \frac{1}{3} & {k=0} \\
      \frac{2}{3 }& {k=1} 
   \end{cases}
   \\
p_{Y|X}\brak{0|0} = \frac{19}{25}\, 
p_{Y|X}\brak{0|1} = \frac{6}{25}\,
p_{Y|X}\brak{1|0} = \frac{45}{50}\,
p_{Y|X}\brak{1|2} = \frac{5}{50}
\end{align}
The desired probability is the probability that a slip drawn at random is marked other than Rs 1,
\begin{align}
&=1-p_X\brak{0}\\
&= p_X(1) + p_X(2)
\end{align}
Using Bayes theorem,
\begin{align}
&= p_Y\brak{0} \times \pr{Y=0 | X=1} + p_Y\brak{1} \times \pr{Y=1|X=2}\\
&=\frac{1}{3} \times \frac{6}{25} + \frac{2}{3} \times \frac{5}{50}\\
&=\frac{11}{75}
\end{align}

\newpage

%\tableofcontents

\bigskip

\renewcommand{\thefigure}{\theenumi}
\renewcommand{\thetable}{\theenumi}
%\renewcommand{\theequation}{\theenumi}

%\begin{abstract}
%%\boldmath
%In this letter, an algorithm for evaluating the exact analytical bit error rate  (BER)  for the piecewise linear (PL) combiner for  multiple relays is presented. Previous results were available only for upto three relays. The algorithm is unique in the sense that  the actual mathematical expressions, that are prohibitively large, need not be explicitly obtained. The diversity gain due to multiple relays is shown through plots of the analytical BER, well supported by simulations. 
%
%\end{abstract}
% IEEEtran.cls defaults to using nonbold math in the Abstract.
% This preserves the distinction between vectors and scalars. However,
% if the journal you are submitting to favors bold math in the abstract,
% then you can use LaTeX's standard command \boldmath at the very start
% of the abstract to achieve this. Many IEEE journals frown on math
% in the abstract anyway.

% Note that keywords are not normally used for peerreview papers.
%\begin{IEEEkeywords}
%Cooperative diversity, decode and forward, piecewise linear
%\end{IEEEkeywords}



% For peer review papers, you can put extra information on the cover
% page as needed:
% \ifCLASSOPTIONpeerreview
% \begin{center} \bfseries EDICS Category: 3-BBND \end{center}
% \fi
%
% For peerreview papers, this IEEEtran command inserts a page break and
% creates the second title. It will be ignored for other modes.
%\IEEEpeerreviewmaketitle




  \item
  Cards with numbers 2 to 101 are placed in a box. A card is selected at random.Find the probability that the card has
\begin{enumerate}[label=(\roman*)]
	\item an even number 
	\item a square number
\end{enumerate}
\solution
%\begin{table}[H]
	\centering
\begin{tabular}{|c|c|c|}
\hline
Random variable &Value &Definition\\ \hline
\multirow{3}{*}{X} &0 &Slips of Rs 1\\
&1 &Slips of Rs 5\\
&2 &Slips of Rs 13\\ \hline
\multirow{2}{*}{Y} &0 &Box A\\
&1 &Box B\\\hline
\end{tabular}
\caption{}
\label{tab:Distribution}
\end{table}
See \tabref{tab:Distribution}.
\begin{align}
p_{Y}\brak{k}= \begin{cases} 
      \frac{1}{3} & {k=0} \\
      \frac{2}{3 }& {k=1} 
   \end{cases}
   \\
p_{Y|X}\brak{0|0} = \frac{19}{25}\, 
p_{Y|X}\brak{0|1} = \frac{6}{25}\,
p_{Y|X}\brak{1|0} = \frac{45}{50}\,
p_{Y|X}\brak{1|2} = \frac{5}{50}
\end{align}
The desired probability is the probability that a slip drawn at random is marked other than Rs 1,
\begin{align}
&=1-p_X\brak{0}\\
&= p_X(1) + p_X(2)
\end{align}
Using Bayes theorem,
\begin{align}
&= p_Y\brak{0} \times \pr{Y=0 | X=1} + p_Y\brak{1} \times \pr{Y=1|X=2}\\
&=\frac{1}{3} \times \frac{6}{25} + \frac{2}{3} \times \frac{5}{50}\\
&=\frac{11}{75}
\end{align}

\newpage

%\tableofcontents

\bigskip

\renewcommand{\thefigure}{\theenumi}
\renewcommand{\thetable}{\theenumi}
%\renewcommand{\theequation}{\theenumi}

%\begin{abstract}
%%\boldmath
%In this letter, an algorithm for evaluating the exact analytical bit error rate  (BER)  for the piecewise linear (PL) combiner for  multiple relays is presented. Previous results were available only for upto three relays. The algorithm is unique in the sense that  the actual mathematical expressions, that are prohibitively large, need not be explicitly obtained. The diversity gain due to multiple relays is shown through plots of the analytical BER, well supported by simulations. 
%
%\end{abstract}
% IEEEtran.cls defaults to using nonbold math in the Abstract.
% This preserves the distinction between vectors and scalars. However,
% if the journal you are submitting to favors bold math in the abstract,
% then you can use LaTeX's standard command \boldmath at the very start
% of the abstract to achieve this. Many IEEE journals frown on math
% in the abstract anyway.

% Note that keywords are not normally used for peerreview papers.
%\begin{IEEEkeywords}
%Cooperative diversity, decode and forward, piecewise linear
%\end{IEEEkeywords}



% For peer review papers, you can put extra information on the cover
% page as needed:
% \ifCLASSOPTIONpeerreview
% \begin{center} \bfseries EDICS Category: 3-BBND \end{center}
% \fi
%
% For peerreview papers, this IEEEtran command inserts a page break and
% creates the second title. It will be ignored for other modes.
%\IEEEpeerreviewmaketitle




\item
The king, queen and jack of clubs are removed from a deck of 52 playing cards and then well shuffled. Now one card is drawn at random from the remaining cards.  Determine the probability that the card is
\begin{enumerate}[label=(\roman*)]
\item a club
\item 10 of hearts
\end{enumerate}
\solution
%\begin{table}[H]
	\centering
\begin{tabular}{|c|c|c|}
\hline
Random variable &Value &Definition\\ \hline
\multirow{3}{*}{X} &0 &Slips of Rs 1\\
&1 &Slips of Rs 5\\
&2 &Slips of Rs 13\\ \hline
\multirow{2}{*}{Y} &0 &Box A\\
&1 &Box B\\\hline
\end{tabular}
\caption{}
\label{tab:Distribution}
\end{table}
See \tabref{tab:Distribution}.
\begin{align}
p_{Y}\brak{k}= \begin{cases} 
      \frac{1}{3} & {k=0} \\
      \frac{2}{3 }& {k=1} 
   \end{cases}
   \\
p_{Y|X}\brak{0|0} = \frac{19}{25}\, 
p_{Y|X}\brak{0|1} = \frac{6}{25}\,
p_{Y|X}\brak{1|0} = \frac{45}{50}\,
p_{Y|X}\brak{1|2} = \frac{5}{50}
\end{align}
The desired probability is the probability that a slip drawn at random is marked other than Rs 1,
\begin{align}
&=1-p_X\brak{0}\\
&= p_X(1) + p_X(2)
\end{align}
Using Bayes theorem,
\begin{align}
&= p_Y\brak{0} \times \pr{Y=0 | X=1} + p_Y\brak{1} \times \pr{Y=1|X=2}\\
&=\frac{1}{3} \times \frac{6}{25} + \frac{2}{3} \times \frac{5}{50}\\
&=\frac{11}{75}
\end{align}

\newpage

%\tableofcontents

\bigskip

\renewcommand{\thefigure}{\theenumi}
\renewcommand{\thetable}{\theenumi}
%\renewcommand{\theequation}{\theenumi}

%\begin{abstract}
%%\boldmath
%In this letter, an algorithm for evaluating the exact analytical bit error rate  (BER)  for the piecewise linear (PL) combiner for  multiple relays is presented. Previous results were available only for upto three relays. The algorithm is unique in the sense that  the actual mathematical expressions, that are prohibitively large, need not be explicitly obtained. The diversity gain due to multiple relays is shown through plots of the analytical BER, well supported by simulations. 
%
%\end{abstract}
% IEEEtran.cls defaults to using nonbold math in the Abstract.
% This preserves the distinction between vectors and scalars. However,
% if the journal you are submitting to favors bold math in the abstract,
% then you can use LaTeX's standard command \boldmath at the very start
% of the abstract to achieve this. Many IEEE journals frown on math
% in the abstract anyway.

% Note that keywords are not normally used for peerreview papers.
%\begin{IEEEkeywords}
%Cooperative diversity, decode and forward, piecewise linear
%\end{IEEEkeywords}



% For peer review papers, you can put extra information on the cover
% page as needed:
% \ifCLASSOPTIONpeerreview
% \begin{center} \bfseries EDICS Category: 3-BBND \end{center}
% \fi
%
% For peerreview papers, this IEEEtran command inserts a page break and
% creates the second title. It will be ignored for other modes.
%\IEEEpeerreviewmaketitle




\item A team of medical students doing their internship have to assist during surgeries
at a city hospital. The probabilities of surgeries rated as very complex, complex,
routine, simple or very simple are respectively, 0.15, 0.20, 0.31, 0.26, .08. Find
the probabilities that a particular surgery will be rated
\begin{enumerate}
	\item complex or very complex;
	\item neither very complex nor very simple;
	\item routine or complex
	\item routine or simple
\end{enumerate}
\solution
%\begin{table}[H]
	\centering
\begin{tabular}{|c|c|c|}
\hline
Random variable &Value &Definition\\ \hline
\multirow{3}{*}{X} &0 &Slips of Rs 1\\
&1 &Slips of Rs 5\\
&2 &Slips of Rs 13\\ \hline
\multirow{2}{*}{Y} &0 &Box A\\
&1 &Box B\\\hline
\end{tabular}
\caption{}
\label{tab:Distribution}
\end{table}
See \tabref{tab:Distribution}.
\begin{align}
p_{Y}\brak{k}= \begin{cases} 
      \frac{1}{3} & {k=0} \\
      \frac{2}{3 }& {k=1} 
   \end{cases}
   \\
p_{Y|X}\brak{0|0} = \frac{19}{25}\, 
p_{Y|X}\brak{0|1} = \frac{6}{25}\,
p_{Y|X}\brak{1|0} = \frac{45}{50}\,
p_{Y|X}\brak{1|2} = \frac{5}{50}
\end{align}
The desired probability is the probability that a slip drawn at random is marked other than Rs 1,
\begin{align}
&=1-p_X\brak{0}\\
&= p_X(1) + p_X(2)
\end{align}
Using Bayes theorem,
\begin{align}
&= p_Y\brak{0} \times \pr{Y=0 | X=1} + p_Y\brak{1} \times \pr{Y=1|X=2}\\
&=\frac{1}{3} \times \frac{6}{25} + \frac{2}{3} \times \frac{5}{50}\\
&=\frac{11}{75}
\end{align}

\newpage

%\tableofcontents

\bigskip

\renewcommand{\thefigure}{\theenumi}
\renewcommand{\thetable}{\theenumi}
%\renewcommand{\theequation}{\theenumi}

%\begin{abstract}
%%\boldmath
%In this letter, an algorithm for evaluating the exact analytical bit error rate  (BER)  for the piecewise linear (PL) combiner for  multiple relays is presented. Previous results were available only for upto three relays. The algorithm is unique in the sense that  the actual mathematical expressions, that are prohibitively large, need not be explicitly obtained. The diversity gain due to multiple relays is shown through plots of the analytical BER, well supported by simulations. 
%
%\end{abstract}
% IEEEtran.cls defaults to using nonbold math in the Abstract.
% This preserves the distinction between vectors and scalars. However,
% if the journal you are submitting to favors bold math in the abstract,
% then you can use LaTeX's standard command \boldmath at the very start
% of the abstract to achieve this. Many IEEE journals frown on math
% in the abstract anyway.

% Note that keywords are not normally used for peerreview papers.
%\begin{IEEEkeywords}
%Cooperative diversity, decode and forward, piecewise linear
%\end{IEEEkeywords}



% For peer review papers, you can put extra information on the cover
% page as needed:
% \ifCLASSOPTIONpeerreview
% \begin{center} \bfseries EDICS Category: 3-BBND \end{center}
% \fi
%
% For peerreview papers, this IEEEtran command inserts a page break and
% creates the second title. It will be ignored for other modes.
%\IEEEpeerreviewmaketitle




\item A card is selected from a pack of 52 cards.
\begin{enumerate}[label=(\alph*)]
    \item How many points are there in the sample space?
    \item Calculate the probability that the card is an ace of spades.
    \item Calculate the probability that the card is (i) an ace and (ii) black card.
\end{enumerate}
\solution
%Let $X$ be an bernoulli rv defined as in \tabref{tab:exemplar/11/16/3/26}.  Then, 
\begin{equation}
    p =
        \frac{4}{11} 
\end{equation}
\begin{table}[H]
	\centering
	\input{exemplar/11/16/3/26/tables/Table2.tex}
	\caption{}
        \label{tab:exemplar/11/16/3/26}
\end{table}

\item The probability that a non leap year selected at random will contain 53 sundays.
\\
\solution
%\begin{table}[H]
	\centering
\begin{tabular}{|c|c|c|}
\hline
Random variable &Value &Definition\\ \hline
\multirow{3}{*}{X} &0 &Slips of Rs 1\\
&1 &Slips of Rs 5\\
&2 &Slips of Rs 13\\ \hline
\multirow{2}{*}{Y} &0 &Box A\\
&1 &Box B\\\hline
\end{tabular}
\caption{}
\label{tab:Distribution}
\end{table}
See \tabref{tab:Distribution}.
\begin{align}
p_{Y}\brak{k}= \begin{cases} 
      \frac{1}{3} & {k=0} \\
      \frac{2}{3 }& {k=1} 
   \end{cases}
   \\
p_{Y|X}\brak{0|0} = \frac{19}{25}\, 
p_{Y|X}\brak{0|1} = \frac{6}{25}\,
p_{Y|X}\brak{1|0} = \frac{45}{50}\,
p_{Y|X}\brak{1|2} = \frac{5}{50}
\end{align}
The desired probability is the probability that a slip drawn at random is marked other than Rs 1,
\begin{align}
&=1-p_X\brak{0}\\
&= p_X(1) + p_X(2)
\end{align}
Using Bayes theorem,
\begin{align}
&= p_Y\brak{0} \times \pr{Y=0 | X=1} + p_Y\brak{1} \times \pr{Y=1|X=2}\\
&=\frac{1}{3} \times \frac{6}{25} + \frac{2}{3} \times \frac{5}{50}\\
&=\frac{11}{75}
\end{align}

\newpage

%\tableofcontents

\bigskip

\renewcommand{\thefigure}{\theenumi}
\renewcommand{\thetable}{\theenumi}
%\renewcommand{\theequation}{\theenumi}

%\begin{abstract}
%%\boldmath
%In this letter, an algorithm for evaluating the exact analytical bit error rate  (BER)  for the piecewise linear (PL) combiner for  multiple relays is presented. Previous results were available only for upto three relays. The algorithm is unique in the sense that  the actual mathematical expressions, that are prohibitively large, need not be explicitly obtained. The diversity gain due to multiple relays is shown through plots of the analytical BER, well supported by simulations. 
%
%\end{abstract}
% IEEEtran.cls defaults to using nonbold math in the Abstract.
% This preserves the distinction between vectors and scalars. However,
% if the journal you are submitting to favors bold math in the abstract,
% then you can use LaTeX's standard command \boldmath at the very start
% of the abstract to achieve this. Many IEEE journals frown on math
% in the abstract anyway.

% Note that keywords are not normally used for peerreview papers.
%\begin{IEEEkeywords}
%Cooperative diversity, decode and forward, piecewise linear
%\end{IEEEkeywords}



% For peer review papers, you can put extra information on the cover
% page as needed:
% \ifCLASSOPTIONpeerreview
% \begin{center} \bfseries EDICS Category: 3-BBND \end{center}
% \fi
%
% For peerreview papers, this IEEEtran command inserts a page break and
% creates the second title. It will be ignored for other modes.
%\IEEEpeerreviewmaketitle




\item One of the four persons John, Rita, Aslam or Gurpreet will be promoted next
month. Consequently the sample space consists of four elementary outcomes
S = {John promoted, Rita promoted, Aslam promoted, Gurpreet promoted}
You are told that the chances of John’s promotion is same as that of Gurpreet,
Rita’s chances of promotion are twice as likely as Johns. Aslam’s chances are
four times that of John.
\begin{enumerate}
	\item Determine
	\begin{enumerate}
		\item P (John promoted)
		\item P (Rita promoted)
		\item P (Aslam promoted)
		\item P (Gurpreet promoted)
	\end{enumerate}
	\item If A = {John promoted or Gurpreet promoted}, find P (A).
\end{enumerate}
\solution
%\begin{table}[H]
	\centering
\begin{tabular}{|c|c|c|}
\hline
Random variable &Value &Definition\\ \hline
\multirow{3}{*}{X} &0 &Slips of Rs 1\\
&1 &Slips of Rs 5\\
&2 &Slips of Rs 13\\ \hline
\multirow{2}{*}{Y} &0 &Box A\\
&1 &Box B\\\hline
\end{tabular}
\caption{}
\label{tab:Distribution}
\end{table}
See \tabref{tab:Distribution}.
\begin{align}
p_{Y}\brak{k}= \begin{cases} 
      \frac{1}{3} & {k=0} \\
      \frac{2}{3 }& {k=1} 
   \end{cases}
   \\
p_{Y|X}\brak{0|0} = \frac{19}{25}\, 
p_{Y|X}\brak{0|1} = \frac{6}{25}\,
p_{Y|X}\brak{1|0} = \frac{45}{50}\,
p_{Y|X}\brak{1|2} = \frac{5}{50}
\end{align}
The desired probability is the probability that a slip drawn at random is marked other than Rs 1,
\begin{align}
&=1-p_X\brak{0}\\
&= p_X(1) + p_X(2)
\end{align}
Using Bayes theorem,
\begin{align}
&= p_Y\brak{0} \times \pr{Y=0 | X=1} + p_Y\brak{1} \times \pr{Y=1|X=2}\\
&=\frac{1}{3} \times \frac{6}{25} + \frac{2}{3} \times \frac{5}{50}\\
&=\frac{11}{75}
\end{align}

\newpage

%\tableofcontents

\bigskip

\renewcommand{\thefigure}{\theenumi}
\renewcommand{\thetable}{\theenumi}
%\renewcommand{\theequation}{\theenumi}

%\begin{abstract}
%%\boldmath
%In this letter, an algorithm for evaluating the exact analytical bit error rate  (BER)  for the piecewise linear (PL) combiner for  multiple relays is presented. Previous results were available only for upto three relays. The algorithm is unique in the sense that  the actual mathematical expressions, that are prohibitively large, need not be explicitly obtained. The diversity gain due to multiple relays is shown through plots of the analytical BER, well supported by simulations. 
%
%\end{abstract}
% IEEEtran.cls defaults to using nonbold math in the Abstract.
% This preserves the distinction between vectors and scalars. However,
% if the journal you are submitting to favors bold math in the abstract,
% then you can use LaTeX's standard command \boldmath at the very start
% of the abstract to achieve this. Many IEEE journals frown on math
% in the abstract anyway.

% Note that keywords are not normally used for peerreview papers.
%\begin{IEEEkeywords}
%Cooperative diversity, decode and forward, piecewise linear
%\end{IEEEkeywords}



% For peer review papers, you can put extra information on the cover
% page as needed:
% \ifCLASSOPTIONpeerreview
% \begin{center} \bfseries EDICS Category: 3-BBND \end{center}
% \fi
%
% For peerreview papers, this IEEEtran command inserts a page break and
% creates the second title. It will be ignored for other modes.
%\IEEEpeerreviewmaketitle




\item A card is drawn from a deck of 52 cards. Find the probability of getting a king or a heart or a red card.\\
\solution
%\begin{table}[H]
	\centering
\begin{tabular}{|c|c|c|}
\hline
Random variable &Value &Definition\\ \hline
\multirow{3}{*}{X} &0 &Slips of Rs 1\\
&1 &Slips of Rs 5\\
&2 &Slips of Rs 13\\ \hline
\multirow{2}{*}{Y} &0 &Box A\\
&1 &Box B\\\hline
\end{tabular}
\caption{}
\label{tab:Distribution}
\end{table}
See \tabref{tab:Distribution}.
\begin{align}
p_{Y}\brak{k}= \begin{cases} 
      \frac{1}{3} & {k=0} \\
      \frac{2}{3 }& {k=1} 
   \end{cases}
   \\
p_{Y|X}\brak{0|0} = \frac{19}{25}\, 
p_{Y|X}\brak{0|1} = \frac{6}{25}\,
p_{Y|X}\brak{1|0} = \frac{45}{50}\,
p_{Y|X}\brak{1|2} = \frac{5}{50}
\end{align}
The desired probability is the probability that a slip drawn at random is marked other than Rs 1,
\begin{align}
&=1-p_X\brak{0}\\
&= p_X(1) + p_X(2)
\end{align}
Using Bayes theorem,
\begin{align}
&= p_Y\brak{0} \times \pr{Y=0 | X=1} + p_Y\brak{1} \times \pr{Y=1|X=2}\\
&=\frac{1}{3} \times \frac{6}{25} + \frac{2}{3} \times \frac{5}{50}\\
&=\frac{11}{75}
\end{align}

\newpage

%\tableofcontents

\bigskip

\renewcommand{\thefigure}{\theenumi}
\renewcommand{\thetable}{\theenumi}
%\renewcommand{\theequation}{\theenumi}

%\begin{abstract}
%%\boldmath
%In this letter, an algorithm for evaluating the exact analytical bit error rate  (BER)  for the piecewise linear (PL) combiner for  multiple relays is presented. Previous results were available only for upto three relays. The algorithm is unique in the sense that  the actual mathematical expressions, that are prohibitively large, need not be explicitly obtained. The diversity gain due to multiple relays is shown through plots of the analytical BER, well supported by simulations. 
%
%\end{abstract}
% IEEEtran.cls defaults to using nonbold math in the Abstract.
% This preserves the distinction between vectors and scalars. However,
% if the journal you are submitting to favors bold math in the abstract,
% then you can use LaTeX's standard command \boldmath at the very start
% of the abstract to achieve this. Many IEEE journals frown on math
% in the abstract anyway.

% Note that keywords are not normally used for peerreview papers.
%\begin{IEEEkeywords}
%Cooperative diversity, decode and forward, piecewise linear
%\end{IEEEkeywords}



% For peer review papers, you can put extra information on the cover
% page as needed:
% \ifCLASSOPTIONpeerreview
% \begin{center} \bfseries EDICS Category: 3-BBND \end{center}
% \fi
%
% For peerreview papers, this IEEEtran command inserts a page break and
% creates the second title. It will be ignored for other modes.
%\IEEEpeerreviewmaketitle




\item The probability that a student will pass his examination is 0.73, the probability of
the student getting a compartment is 0.13, and the probability that the student will
either pass or get compartment is 0.96. State True or False.\\
\solution
%\begin{table}[H]
	\centering
\begin{tabular}{|c|c|c|}
\hline
Random variable &Value &Definition\\ \hline
\multirow{3}{*}{X} &0 &Slips of Rs 1\\
&1 &Slips of Rs 5\\
&2 &Slips of Rs 13\\ \hline
\multirow{2}{*}{Y} &0 &Box A\\
&1 &Box B\\\hline
\end{tabular}
\caption{}
\label{tab:Distribution}
\end{table}
See \tabref{tab:Distribution}.
\begin{align}
p_{Y}\brak{k}= \begin{cases} 
      \frac{1}{3} & {k=0} \\
      \frac{2}{3 }& {k=1} 
   \end{cases}
   \\
p_{Y|X}\brak{0|0} = \frac{19}{25}\, 
p_{Y|X}\brak{0|1} = \frac{6}{25}\,
p_{Y|X}\brak{1|0} = \frac{45}{50}\,
p_{Y|X}\brak{1|2} = \frac{5}{50}
\end{align}
The desired probability is the probability that a slip drawn at random is marked other than Rs 1,
\begin{align}
&=1-p_X\brak{0}\\
&= p_X(1) + p_X(2)
\end{align}
Using Bayes theorem,
\begin{align}
&= p_Y\brak{0} \times \pr{Y=0 | X=1} + p_Y\brak{1} \times \pr{Y=1|X=2}\\
&=\frac{1}{3} \times \frac{6}{25} + \frac{2}{3} \times \frac{5}{50}\\
&=\frac{11}{75}
\end{align}

\newpage

%\tableofcontents

\bigskip

\renewcommand{\thefigure}{\theenumi}
\renewcommand{\thetable}{\theenumi}
%\renewcommand{\theequation}{\theenumi}

%\begin{abstract}
%%\boldmath
%In this letter, an algorithm for evaluating the exact analytical bit error rate  (BER)  for the piecewise linear (PL) combiner for  multiple relays is presented. Previous results were available only for upto three relays. The algorithm is unique in the sense that  the actual mathematical expressions, that are prohibitively large, need not be explicitly obtained. The diversity gain due to multiple relays is shown through plots of the analytical BER, well supported by simulations. 
%
%\end{abstract}
% IEEEtran.cls defaults to using nonbold math in the Abstract.
% This preserves the distinction between vectors and scalars. However,
% if the journal you are submitting to favors bold math in the abstract,
% then you can use LaTeX's standard command \boldmath at the very start
% of the abstract to achieve this. Many IEEE journals frown on math
% in the abstract anyway.

% Note that keywords are not normally used for peerreview papers.
%\begin{IEEEkeywords}
%Cooperative diversity, decode and forward, piecewise linear
%\end{IEEEkeywords}



% For peer review papers, you can put extra information on the cover
% page as needed:
% \ifCLASSOPTIONpeerreview
% \begin{center} \bfseries EDICS Category: 3-BBND \end{center}
% \fi
%
% For peerreview papers, this IEEEtran command inserts a page break and
% creates the second title. It will be ignored for other modes.
%\IEEEpeerreviewmaketitle




\item A card is selected from a pack of 52 cards\\
\begin{enumerate}[label=(\alph*)]
\item How many points are there in the sample space?
\item Calculate the probability that the cards is an ace of spades.
\item Calculate the probability that the card is (i) an ace (ii)black card.\\
\end{enumerate}
%\input{ncert/11/16/3/4_1/Prob_4.tex}
\item In a non-leap year, the probability of having 53 tuesdays or 53 wednesdays is\\
\solution
%A non-leap year has a total of 365 days, and a week has 7 days.\\
So it can be expressed as 
\begin{align}
365\text{days} &=52\times 7+1 \text{day}
\end{align}
$\implies$ 52 tuesdays or wednesdays\\
Random variable X denotes the days of a week
\begin{align}
p_X\brak{k}&=\frac{1}{7}; \quad \brak{1<k<7}
\end{align}
So the probability of extra day being tuesday or wednesday is
\begin{align}
p_X\brak{3}+p_X\brak{4}&=\frac{1}{7}+\frac{1}{7}=\frac{2}{7}
\end{align}



\item There are 1000 sealed envelopes in a box, 10 of them contain a cash prize of
Rs 100 each, 100 of them contain a cash prize of Rs 50 each and 200 of them
contain a cash prize of Rs 10 each and rest do not contain any cash prize. If they
are well shuffled and an envelope is picked up out, what is the probability that it
contains no cash prize?\\
\solution
%\begin{table}[H]
	\centering
\begin{tabular}{|c|c|c|}
\hline
Random variable &Value &Definition\\ \hline
\multirow{3}{*}{X} &0 &Slips of Rs 1\\
&1 &Slips of Rs 5\\
&2 &Slips of Rs 13\\ \hline
\multirow{2}{*}{Y} &0 &Box A\\
&1 &Box B\\\hline
\end{tabular}
\caption{}
\label{tab:Distribution}
\end{table}
See \tabref{tab:Distribution}.
\begin{align}
p_{Y}\brak{k}= \begin{cases} 
      \frac{1}{3} & {k=0} \\
      \frac{2}{3 }& {k=1} 
   \end{cases}
   \\
p_{Y|X}\brak{0|0} = \frac{19}{25}\, 
p_{Y|X}\brak{0|1} = \frac{6}{25}\,
p_{Y|X}\brak{1|0} = \frac{45}{50}\,
p_{Y|X}\brak{1|2} = \frac{5}{50}
\end{align}
The desired probability is the probability that a slip drawn at random is marked other than Rs 1,
\begin{align}
&=1-p_X\brak{0}\\
&= p_X(1) + p_X(2)
\end{align}
Using Bayes theorem,
\begin{align}
&= p_Y\brak{0} \times \pr{Y=0 | X=1} + p_Y\brak{1} \times \pr{Y=1|X=2}\\
&=\frac{1}{3} \times \frac{6}{25} + \frac{2}{3} \times \frac{5}{50}\\
&=\frac{11}{75}
\end{align}

\newpage

%\tableofcontents

\bigskip

\renewcommand{\thefigure}{\theenumi}
\renewcommand{\thetable}{\theenumi}
%\renewcommand{\theequation}{\theenumi}

%\begin{abstract}
%%\boldmath
%In this letter, an algorithm for evaluating the exact analytical bit error rate  (BER)  for the piecewise linear (PL) combiner for  multiple relays is presented. Previous results were available only for upto three relays. The algorithm is unique in the sense that  the actual mathematical expressions, that are prohibitively large, need not be explicitly obtained. The diversity gain due to multiple relays is shown through plots of the analytical BER, well supported by simulations. 
%
%\end{abstract}
% IEEEtran.cls defaults to using nonbold math in the Abstract.
% This preserves the distinction between vectors and scalars. However,
% if the journal you are submitting to favors bold math in the abstract,
% then you can use LaTeX's standard command \boldmath at the very start
% of the abstract to achieve this. Many IEEE journals frown on math
% in the abstract anyway.

% Note that keywords are not normally used for peerreview papers.
%\begin{IEEEkeywords}
%Cooperative diversity, decode and forward, piecewise linear
%\end{IEEEkeywords}



% For peer review papers, you can put extra information on the cover
% page as needed:
% \ifCLASSOPTIONpeerreview
% \begin{center} \bfseries EDICS Category: 3-BBND \end{center}
% \fi
%
% For peerreview papers, this IEEEtran command inserts a page break and
% creates the second title. It will be ignored for other modes.
%\IEEEpeerreviewmaketitle




\item 
A die is thrown and a card is selected at random from a deck of 52 playing cards. The probability of getting an even number on the die and a spade card.\\
\solution
%\begin{table}[H]
	\centering
\begin{tabular}{|c|c|c|}
\hline
Random variable &Value &Definition\\ \hline
\multirow{3}{*}{X} &0 &Slips of Rs 1\\
&1 &Slips of Rs 5\\
&2 &Slips of Rs 13\\ \hline
\multirow{2}{*}{Y} &0 &Box A\\
&1 &Box B\\\hline
\end{tabular}
\caption{}
\label{tab:Distribution}
\end{table}
See \tabref{tab:Distribution}.
\begin{align}
p_{Y}\brak{k}= \begin{cases} 
      \frac{1}{3} & {k=0} \\
      \frac{2}{3 }& {k=1} 
   \end{cases}
   \\
p_{Y|X}\brak{0|0} = \frac{19}{25}\, 
p_{Y|X}\brak{0|1} = \frac{6}{25}\,
p_{Y|X}\brak{1|0} = \frac{45}{50}\,
p_{Y|X}\brak{1|2} = \frac{5}{50}
\end{align}
The desired probability is the probability that a slip drawn at random is marked other than Rs 1,
\begin{align}
&=1-p_X\brak{0}\\
&= p_X(1) + p_X(2)
\end{align}
Using Bayes theorem,
\begin{align}
&= p_Y\brak{0} \times \pr{Y=0 | X=1} + p_Y\brak{1} \times \pr{Y=1|X=2}\\
&=\frac{1}{3} \times \frac{6}{25} + \frac{2}{3} \times \frac{5}{50}\\
&=\frac{11}{75}
\end{align}

\newpage

%\tableofcontents

\bigskip

\renewcommand{\thefigure}{\theenumi}
\renewcommand{\thetable}{\theenumi}
%\renewcommand{\theequation}{\theenumi}

%\begin{abstract}
%%\boldmath
%In this letter, an algorithm for evaluating the exact analytical bit error rate  (BER)  for the piecewise linear (PL) combiner for  multiple relays is presented. Previous results were available only for upto three relays. The algorithm is unique in the sense that  the actual mathematical expressions, that are prohibitively large, need not be explicitly obtained. The diversity gain due to multiple relays is shown through plots of the analytical BER, well supported by simulations. 
%
%\end{abstract}
% IEEEtran.cls defaults to using nonbold math in the Abstract.
% This preserves the distinction between vectors and scalars. However,
% if the journal you are submitting to favors bold math in the abstract,
% then you can use LaTeX's standard command \boldmath at the very start
% of the abstract to achieve this. Many IEEE journals frown on math
% in the abstract anyway.

% Note that keywords are not normally used for peerreview papers.
%\begin{IEEEkeywords}
%Cooperative diversity, decode and forward, piecewise linear
%\end{IEEEkeywords}



% For peer review papers, you can put extra information on the cover
% page as needed:
% \ifCLASSOPTIONpeerreview
% \begin{center} \bfseries EDICS Category: 3-BBND \end{center}
% \fi
%
% For peerreview papers, this IEEEtran command inserts a page break and
% creates the second title. It will be ignored for other modes.
%\IEEEpeerreviewmaketitle




\item
If 4-digit numbers greater than 5,000 are randomly formed from the digits 0, 1, 3, 5, and 7, what is the probability of forming a number divisible by 5 when:
\begin{enumerate}
    \item The digits are repeated?
    \item The repetition of digits is not allowed?
\end{enumerate}
\solution
%\begin{table}[H]
	\centering
\begin{tabular}{|c|c|c|}
\hline
Random variable &Value &Definition\\ \hline
\multirow{3}{*}{X} &0 &Slips of Rs 1\\
&1 &Slips of Rs 5\\
&2 &Slips of Rs 13\\ \hline
\multirow{2}{*}{Y} &0 &Box A\\
&1 &Box B\\\hline
\end{tabular}
\caption{}
\label{tab:Distribution}
\end{table}
See \tabref{tab:Distribution}.
\begin{align}
p_{Y}\brak{k}= \begin{cases} 
      \frac{1}{3} & {k=0} \\
      \frac{2}{3 }& {k=1} 
   \end{cases}
   \\
p_{Y|X}\brak{0|0} = \frac{19}{25}\, 
p_{Y|X}\brak{0|1} = \frac{6}{25}\,
p_{Y|X}\brak{1|0} = \frac{45}{50}\,
p_{Y|X}\brak{1|2} = \frac{5}{50}
\end{align}
The desired probability is the probability that a slip drawn at random is marked other than Rs 1,
\begin{align}
&=1-p_X\brak{0}\\
&= p_X(1) + p_X(2)
\end{align}
Using Bayes theorem,
\begin{align}
&= p_Y\brak{0} \times \pr{Y=0 | X=1} + p_Y\brak{1} \times \pr{Y=1|X=2}\\
&=\frac{1}{3} \times \frac{6}{25} + \frac{2}{3} \times \frac{5}{50}\\
&=\frac{11}{75}
\end{align}

\newpage

%\tableofcontents

\bigskip

\renewcommand{\thefigure}{\theenumi}
\renewcommand{\thetable}{\theenumi}
%\renewcommand{\theequation}{\theenumi}

%\begin{abstract}
%%\boldmath
%In this letter, an algorithm for evaluating the exact analytical bit error rate  (BER)  for the piecewise linear (PL) combiner for  multiple relays is presented. Previous results were available only for upto three relays. The algorithm is unique in the sense that  the actual mathematical expressions, that are prohibitively large, need not be explicitly obtained. The diversity gain due to multiple relays is shown through plots of the analytical BER, well supported by simulations. 
%
%\end{abstract}
% IEEEtran.cls defaults to using nonbold math in the Abstract.
% This preserves the distinction between vectors and scalars. However,
% if the journal you are submitting to favors bold math in the abstract,
% then you can use LaTeX's standard command \boldmath at the very start
% of the abstract to achieve this. Many IEEE journals frown on math
% in the abstract anyway.

% Note that keywords are not normally used for peerreview papers.
%\begin{IEEEkeywords}
%Cooperative diversity, decode and forward, piecewise linear
%\end{IEEEkeywords}



% For peer review papers, you can put extra information on the cover
% page as needed:
% \ifCLASSOPTIONpeerreview
% \begin{center} \bfseries EDICS Category: 3-BBND \end{center}
% \fi
%
% For peerreview papers, this IEEEtran command inserts a page break and
% creates the second title. It will be ignored for other modes.
%\IEEEpeerreviewmaketitle




\item Consider the probability space $\brak{\Omega, \mathcal{G}, P}$ where $\Omega = [0,2]$ and $\mathcal{G} = \cbrak{\phi, \Omega, [0,1], (1,2]}$. Let $X$ and $Y$ be two functions on $\Omega$ defined as
\begin{align*}
    X(\omega) = 
    \begin{cases}
        1 & \text{if }\omega \in [0, 1]\\
        2 & \text{if }\omega \in (1, 2]
    \end{cases}
\end{align*}
and
\begin{align*}
    Y(\omega) = 
    \begin{cases}
        2 & \text{if }\omega \in [0, 1.5]\\
        3 & \text{if }\omega \in (1.5, 2].
    \end{cases}
\end{align*}
Then which one of the following statements is true?
\begin{enumerate}
    \item [(A)] $X$ is a random variable with respect to $\mathcal{G}$, but $Y$ is not a random variable with respect to $\mathcal{G}$.
    \item [(B)] $Y$ is a random variable with respect to $\mathcal{G}$, but $X$ is not a random variable with respect to $\mathcal{G}$.
    \item [(C)] Neither $X$ nor $Y$ is a random variable with respect to $\mathcal{G}$.
    \item [(D)] Both $X$ and $Y$ are random variables with respect to $\mathcal{G}$.
\end{enumerate} \hfill (GATE ST 2023)\\
\solution
%\begin{table}[H]
	\centering
\begin{tabular}{|c|c|c|}
\hline
Random variable &Value &Definition\\ \hline
\multirow{3}{*}{X} &0 &Slips of Rs 1\\
&1 &Slips of Rs 5\\
&2 &Slips of Rs 13\\ \hline
\multirow{2}{*}{Y} &0 &Box A\\
&1 &Box B\\\hline
\end{tabular}
\caption{}
\label{tab:Distribution}
\end{table}
See \tabref{tab:Distribution}.
\begin{align}
p_{Y}\brak{k}= \begin{cases} 
      \frac{1}{3} & {k=0} \\
      \frac{2}{3 }& {k=1} 
   \end{cases}
   \\
p_{Y|X}\brak{0|0} = \frac{19}{25}\, 
p_{Y|X}\brak{0|1} = \frac{6}{25}\,
p_{Y|X}\brak{1|0} = \frac{45}{50}\,
p_{Y|X}\brak{1|2} = \frac{5}{50}
\end{align}
The desired probability is the probability that a slip drawn at random is marked other than Rs 1,
\begin{align}
&=1-p_X\brak{0}\\
&= p_X(1) + p_X(2)
\end{align}
Using Bayes theorem,
\begin{align}
&= p_Y\brak{0} \times \pr{Y=0 | X=1} + p_Y\brak{1} \times \pr{Y=1|X=2}\\
&=\frac{1}{3} \times \frac{6}{25} + \frac{2}{3} \times \frac{5}{50}\\
&=\frac{11}{75}
\end{align}

\newpage

%\tableofcontents

\bigskip

\renewcommand{\thefigure}{\theenumi}
\renewcommand{\thetable}{\theenumi}
%\renewcommand{\theequation}{\theenumi}

%\begin{abstract}
%%\boldmath
%In this letter, an algorithm for evaluating the exact analytical bit error rate  (BER)  for the piecewise linear (PL) combiner for  multiple relays is presented. Previous results were available only for upto three relays. The algorithm is unique in the sense that  the actual mathematical expressions, that are prohibitively large, need not be explicitly obtained. The diversity gain due to multiple relays is shown through plots of the analytical BER, well supported by simulations. 
%
%\end{abstract}
% IEEEtran.cls defaults to using nonbold math in the Abstract.
% This preserves the distinction between vectors and scalars. However,
% if the journal you are submitting to favors bold math in the abstract,
% then you can use LaTeX's standard command \boldmath at the very start
% of the abstract to achieve this. Many IEEE journals frown on math
% in the abstract anyway.

% Note that keywords are not normally used for peerreview papers.
%\begin{IEEEkeywords}
%Cooperative diversity, decode and forward, piecewise linear
%\end{IEEEkeywords}



% For peer review papers, you can put extra information on the cover
% page as needed:
% \ifCLASSOPTIONpeerreview
% \begin{center} \bfseries EDICS Category: 3-BBND \end{center}
% \fi
%
% For peerreview papers, this IEEEtran command inserts a page break and
% creates the second title. It will be ignored for other modes.
%\IEEEpeerreviewmaketitle




	\item  A die is loaded in such a way that each odd number is twice as likely to occur as
each even number. Find $P(G)$, where $G$ is the event that a number greater than
3 occurs on a single roll of the die.
\\
\solution
		%\begin{table}[H]
	\centering
\begin{tabular}{|c|c|c|}
\hline
Random variable &Value &Definition\\ \hline
\multirow{3}{*}{X} &0 &Slips of Rs 1\\
&1 &Slips of Rs 5\\
&2 &Slips of Rs 13\\ \hline
\multirow{2}{*}{Y} &0 &Box A\\
&1 &Box B\\\hline
\end{tabular}
\caption{}
\label{tab:Distribution}
\end{table}
See \tabref{tab:Distribution}.
\begin{align}
p_{Y}\brak{k}= \begin{cases} 
      \frac{1}{3} & {k=0} \\
      \frac{2}{3 }& {k=1} 
   \end{cases}
   \\
p_{Y|X}\brak{0|0} = \frac{19}{25}\, 
p_{Y|X}\brak{0|1} = \frac{6}{25}\,
p_{Y|X}\brak{1|0} = \frac{45}{50}\,
p_{Y|X}\brak{1|2} = \frac{5}{50}
\end{align}
The desired probability is the probability that a slip drawn at random is marked other than Rs 1,
\begin{align}
&=1-p_X\brak{0}\\
&= p_X(1) + p_X(2)
\end{align}
Using Bayes theorem,
\begin{align}
&= p_Y\brak{0} \times \pr{Y=0 | X=1} + p_Y\brak{1} \times \pr{Y=1|X=2}\\
&=\frac{1}{3} \times \frac{6}{25} + \frac{2}{3} \times \frac{5}{50}\\
&=\frac{11}{75}
\end{align}

\newpage

%\tableofcontents

\bigskip

\renewcommand{\thefigure}{\theenumi}
\renewcommand{\thetable}{\theenumi}
%\renewcommand{\theequation}{\theenumi}

%\begin{abstract}
%%\boldmath
%In this letter, an algorithm for evaluating the exact analytical bit error rate  (BER)  for the piecewise linear (PL) combiner for  multiple relays is presented. Previous results were available only for upto three relays. The algorithm is unique in the sense that  the actual mathematical expressions, that are prohibitively large, need not be explicitly obtained. The diversity gain due to multiple relays is shown through plots of the analytical BER, well supported by simulations. 
%
%\end{abstract}
% IEEEtran.cls defaults to using nonbold math in the Abstract.
% This preserves the distinction between vectors and scalars. However,
% if the journal you are submitting to favors bold math in the abstract,
% then you can use LaTeX's standard command \boldmath at the very start
% of the abstract to achieve this. Many IEEE journals frown on math
% in the abstract anyway.

% Note that keywords are not normally used for peerreview papers.
%\begin{IEEEkeywords}
%Cooperative diversity, decode and forward, piecewise linear
%\end{IEEEkeywords}



% For peer review papers, you can put extra information on the cover
% page as needed:
% \ifCLASSOPTIONpeerreview
% \begin{center} \bfseries EDICS Category: 3-BBND \end{center}
% \fi
%
% For peerreview papers, this IEEEtran command inserts a page break and
% creates the second title. It will be ignored for other modes.
%\IEEEpeerreviewmaketitle




	\item All the jacks, queens and kings are removed from a deck of 52 playing cards. The remaining cards are well shuffled and then one card is drawn at random. Giving ace a value 1 similar value for other cards, find the probability that the card has a value 
		\begin{enumerate}
			\item 7
			\item greater than 7
			\item less than 7
		\end{enumerate}
		%Number of cards left after removing all jacks, queens and kings 
\begin{align}
N	= 52 - 4\times 3
	= 40
\end{align}
%\begin{table}[H]
%\def\arraystretch{1.2}
%\begin{tabular}{|c|c|c|}
%\hline
%	\textbf{Parameter} &\textbf{Value} &\textbf{Description}\\ \hline
%	$X$ &1-10 &Represents the value of the card picked \\ \hline
%\end{tabular}
%\end{table}
Let $1 \le X \le 10$ be the value of the card picked.  Then,
\begin{align}
	p_X(k) &= \Pr(X=k)\ \forall\ 1 \leq k \leq 10\\
	&= \frac{4\times 1}{40}\\
	&= \frac{1}{10}\\
	\therefore p_X(k) &= 
	\begin{cases}
		\frac{1}{10} & 1 \leq k \leq 10\\
		0 & \text{otherwise}
	\end{cases}
\end{align}
and
\begin{align}
	F_{X}(k) &= \sum_{m=0}^{k}p_{X}(m) \quad 1 \leq k \leq 10\\
	&= \frac{k}{10}\\
	\therefore F_{X}(k) &= 
	\begin{cases}
		0 & k \leq 0\\
		\frac{k}{10} & 1\leq k \leq 10\\
		1 & k > 10 
	\end{cases}
\end{align}
\begin{enumerate}
	\item Probability that card has value equal to 7 is
		\begin{align}
			 p_{X}(7)
			= \frac{1}{10}
		\end{align}
	\item Probability that card has value greater than 7 is
		\begin{align}
			1 - F_X(7)
			&= 1 - \frac{7}{10}
			\\
			&= \frac{3}{10}
		\end{align}
	\item Probability that card has value less than 7 is
		\begin{align}
			 F_{X}(6)
			=\frac{6}{10}
		\end{align}
\end{enumerate}

  \item A Lot consists of 48 mobile phones of which 42 are good, 3 have only minor defects and 3 have major defects.Varnika will buy a phone if it is good but the trader will only buy a mobile if it has no major defects. One phone is selected at random from the lot. What is the probability that it is
\begin{enumerate}
	\item acceptable to Varnika?
            \item acceptable to the trader?
\end{enumerate}
\solution
	%\begin{table}[H]
	\centering
\begin{tabular}{|c|c|c|}
\hline
Random variable &Value &Definition\\ \hline
\multirow{3}{*}{X} &0 &Slips of Rs 1\\
&1 &Slips of Rs 5\\
&2 &Slips of Rs 13\\ \hline
\multirow{2}{*}{Y} &0 &Box A\\
&1 &Box B\\\hline
\end{tabular}
\caption{}
\label{tab:Distribution}
\end{table}
See \tabref{tab:Distribution}.
\begin{align}
p_{Y}\brak{k}= \begin{cases} 
      \frac{1}{3} & {k=0} \\
      \frac{2}{3 }& {k=1} 
   \end{cases}
   \\
p_{Y|X}\brak{0|0} = \frac{19}{25}\, 
p_{Y|X}\brak{0|1} = \frac{6}{25}\,
p_{Y|X}\brak{1|0} = \frac{45}{50}\,
p_{Y|X}\brak{1|2} = \frac{5}{50}
\end{align}
The desired probability is the probability that a slip drawn at random is marked other than Rs 1,
\begin{align}
&=1-p_X\brak{0}\\
&= p_X(1) + p_X(2)
\end{align}
Using Bayes theorem,
\begin{align}
&= p_Y\brak{0} \times \pr{Y=0 | X=1} + p_Y\brak{1} \times \pr{Y=1|X=2}\\
&=\frac{1}{3} \times \frac{6}{25} + \frac{2}{3} \times \frac{5}{50}\\
&=\frac{11}{75}
\end{align}

\newpage

%\tableofcontents

\bigskip

\renewcommand{\thefigure}{\theenumi}
\renewcommand{\thetable}{\theenumi}
%\renewcommand{\theequation}{\theenumi}

%\begin{abstract}
%%\boldmath
%In this letter, an algorithm for evaluating the exact analytical bit error rate  (BER)  for the piecewise linear (PL) combiner for  multiple relays is presented. Previous results were available only for upto three relays. The algorithm is unique in the sense that  the actual mathematical expressions, that are prohibitively large, need not be explicitly obtained. The diversity gain due to multiple relays is shown through plots of the analytical BER, well supported by simulations. 
%
%\end{abstract}
% IEEEtran.cls defaults to using nonbold math in the Abstract.
% This preserves the distinction between vectors and scalars. However,
% if the journal you are submitting to favors bold math in the abstract,
% then you can use LaTeX's standard command \boldmath at the very start
% of the abstract to achieve this. Many IEEE journals frown on math
% in the abstract anyway.

% Note that keywords are not normally used for peerreview papers.
%\begin{IEEEkeywords}
%Cooperative diversity, decode and forward, piecewise linear
%\end{IEEEkeywords}



% For peer review papers, you can put extra information on the cover
% page as needed:
% \ifCLASSOPTIONpeerreview
% \begin{center} \bfseries EDICS Category: 3-BBND \end{center}
% \fi
%
% For peerreview papers, this IEEEtran command inserts a page break and
% creates the second title. It will be ignored for other modes.
%\IEEEpeerreviewmaketitle




 \item A student says that if you throw a die, it will show up 1 or not 1. Therefore, the probability of getting 1 and the probability of getting 'not 1' each is equal to $\frac{1}{2}$. Is this correct? Give reasons.\\
 \solution
        %\begin{table}[H]
	\centering
\begin{tabular}{|c|c|c|}
\hline
Random variable &Value &Definition\\ \hline
\multirow{3}{*}{X} &0 &Slips of Rs 1\\
&1 &Slips of Rs 5\\
&2 &Slips of Rs 13\\ \hline
\multirow{2}{*}{Y} &0 &Box A\\
&1 &Box B\\\hline
\end{tabular}
\caption{}
\label{tab:Distribution}
\end{table}
See \tabref{tab:Distribution}.
\begin{align}
p_{Y}\brak{k}= \begin{cases} 
      \frac{1}{3} & {k=0} \\
      \frac{2}{3 }& {k=1} 
   \end{cases}
   \\
p_{Y|X}\brak{0|0} = \frac{19}{25}\, 
p_{Y|X}\brak{0|1} = \frac{6}{25}\,
p_{Y|X}\brak{1|0} = \frac{45}{50}\,
p_{Y|X}\brak{1|2} = \frac{5}{50}
\end{align}
The desired probability is the probability that a slip drawn at random is marked other than Rs 1,
\begin{align}
&=1-p_X\brak{0}\\
&= p_X(1) + p_X(2)
\end{align}
Using Bayes theorem,
\begin{align}
&= p_Y\brak{0} \times \pr{Y=0 | X=1} + p_Y\brak{1} \times \pr{Y=1|X=2}\\
&=\frac{1}{3} \times \frac{6}{25} + \frac{2}{3} \times \frac{5}{50}\\
&=\frac{11}{75}
\end{align}

\newpage

%\tableofcontents

\bigskip

\renewcommand{\thefigure}{\theenumi}
\renewcommand{\thetable}{\theenumi}
%\renewcommand{\theequation}{\theenumi}

%\begin{abstract}
%%\boldmath
%In this letter, an algorithm for evaluating the exact analytical bit error rate  (BER)  for the piecewise linear (PL) combiner for  multiple relays is presented. Previous results were available only for upto three relays. The algorithm is unique in the sense that  the actual mathematical expressions, that are prohibitively large, need not be explicitly obtained. The diversity gain due to multiple relays is shown through plots of the analytical BER, well supported by simulations. 
%
%\end{abstract}
% IEEEtran.cls defaults to using nonbold math in the Abstract.
% This preserves the distinction between vectors and scalars. However,
% if the journal you are submitting to favors bold math in the abstract,
% then you can use LaTeX's standard command \boldmath at the very start
% of the abstract to achieve this. Many IEEE journals frown on math
% in the abstract anyway.

% Note that keywords are not normally used for peerreview papers.
%\begin{IEEEkeywords}
%Cooperative diversity, decode and forward, piecewise linear
%\end{IEEEkeywords}



% For peer review papers, you can put extra information on the cover
% page as needed:
% \ifCLASSOPTIONpeerreview
% \begin{center} \bfseries EDICS Category: 3-BBND \end{center}
% \fi
%
% For peerreview papers, this IEEEtran command inserts a page break and
% creates the second title. It will be ignored for other modes.
%\IEEEpeerreviewmaketitle




   \item Four candidates A, B, C, D have ap-
plied for the assignment to coach a school cricket
team. If A is twice as likely to be selected as B, and
B and C are given about the same chance of being
selected, while C is twice as likely to be selected
as D, what are the probabilities that
\begin{enumerate}
\item C will be selected?
\item A will not be selected?
\end{enumerate}
	%\begin{table}[H]
	\centering
\begin{tabular}{|c|c|c|}
\hline
Random variable &Value &Definition\\ \hline
\multirow{3}{*}{X} &0 &Slips of Rs 1\\
&1 &Slips of Rs 5\\
&2 &Slips of Rs 13\\ \hline
\multirow{2}{*}{Y} &0 &Box A\\
&1 &Box B\\\hline
\end{tabular}
\caption{}
\label{tab:Distribution}
\end{table}
See \tabref{tab:Distribution}.
\begin{align}
p_{Y}\brak{k}= \begin{cases} 
      \frac{1}{3} & {k=0} \\
      \frac{2}{3 }& {k=1} 
   \end{cases}
   \\
p_{Y|X}\brak{0|0} = \frac{19}{25}\, 
p_{Y|X}\brak{0|1} = \frac{6}{25}\,
p_{Y|X}\brak{1|0} = \frac{45}{50}\,
p_{Y|X}\brak{1|2} = \frac{5}{50}
\end{align}
The desired probability is the probability that a slip drawn at random is marked other than Rs 1,
\begin{align}
&=1-p_X\brak{0}\\
&= p_X(1) + p_X(2)
\end{align}
Using Bayes theorem,
\begin{align}
&= p_Y\brak{0} \times \pr{Y=0 | X=1} + p_Y\brak{1} \times \pr{Y=1|X=2}\\
&=\frac{1}{3} \times \frac{6}{25} + \frac{2}{3} \times \frac{5}{50}\\
&=\frac{11}{75}
\end{align}

\newpage

%\tableofcontents

\bigskip

\renewcommand{\thefigure}{\theenumi}
\renewcommand{\thetable}{\theenumi}
%\renewcommand{\theequation}{\theenumi}

%\begin{abstract}
%%\boldmath
%In this letter, an algorithm for evaluating the exact analytical bit error rate  (BER)  for the piecewise linear (PL) combiner for  multiple relays is presented. Previous results were available only for upto three relays. The algorithm is unique in the sense that  the actual mathematical expressions, that are prohibitively large, need not be explicitly obtained. The diversity gain due to multiple relays is shown through plots of the analytical BER, well supported by simulations. 
%
%\end{abstract}
% IEEEtran.cls defaults to using nonbold math in the Abstract.
% This preserves the distinction between vectors and scalars. However,
% if the journal you are submitting to favors bold math in the abstract,
% then you can use LaTeX's standard command \boldmath at the very start
% of the abstract to achieve this. Many IEEE journals frown on math
% in the abstract anyway.

% Note that keywords are not normally used for peerreview papers.
%\begin{IEEEkeywords}
%Cooperative diversity, decode and forward, piecewise linear
%\end{IEEEkeywords}



% For peer review papers, you can put extra information on the cover
% page as needed:
% \ifCLASSOPTIONpeerreview
% \begin{center} \bfseries EDICS Category: 3-BBND \end{center}
% \fi
%
% For peerreview papers, this IEEEtran command inserts a page break and
% creates the second title. It will be ignored for other modes.
%\IEEEpeerreviewmaketitle




 \item A bag contain 24 balls of which $x$ balls are red, $2x$ are white and $3x$ are blue. A ball is selected at random, What is the probability that it is
\begin{enumerate}[label=\alph*)]
\item not red ?
\item white ?
\end{enumerate}
%\begin{table}[H]
	\centering
\begin{tabular}{|c|c|c|}
\hline
Random variable &Value &Definition\\ \hline
\multirow{3}{*}{X} &0 &Slips of Rs 1\\
&1 &Slips of Rs 5\\
&2 &Slips of Rs 13\\ \hline
\multirow{2}{*}{Y} &0 &Box A\\
&1 &Box B\\\hline
\end{tabular}
\caption{}
\label{tab:Distribution}
\end{table}
See \tabref{tab:Distribution}.
\begin{align}
p_{Y}\brak{k}= \begin{cases} 
      \frac{1}{3} & {k=0} \\
      \frac{2}{3 }& {k=1} 
   \end{cases}
   \\
p_{Y|X}\brak{0|0} = \frac{19}{25}\, 
p_{Y|X}\brak{0|1} = \frac{6}{25}\,
p_{Y|X}\brak{1|0} = \frac{45}{50}\,
p_{Y|X}\brak{1|2} = \frac{5}{50}
\end{align}
The desired probability is the probability that a slip drawn at random is marked other than Rs 1,
\begin{align}
&=1-p_X\brak{0}\\
&= p_X(1) + p_X(2)
\end{align}
Using Bayes theorem,
\begin{align}
&= p_Y\brak{0} \times \pr{Y=0 | X=1} + p_Y\brak{1} \times \pr{Y=1|X=2}\\
&=\frac{1}{3} \times \frac{6}{25} + \frac{2}{3} \times \frac{5}{50}\\
&=\frac{11}{75}
\end{align}

\newpage

%\tableofcontents

\bigskip

\renewcommand{\thefigure}{\theenumi}
\renewcommand{\thetable}{\theenumi}
%\renewcommand{\theequation}{\theenumi}

%\begin{abstract}
%%\boldmath
%In this letter, an algorithm for evaluating the exact analytical bit error rate  (BER)  for the piecewise linear (PL) combiner for  multiple relays is presented. Previous results were available only for upto three relays. The algorithm is unique in the sense that  the actual mathematical expressions, that are prohibitively large, need not be explicitly obtained. The diversity gain due to multiple relays is shown through plots of the analytical BER, well supported by simulations. 
%
%\end{abstract}
% IEEEtran.cls defaults to using nonbold math in the Abstract.
% This preserves the distinction between vectors and scalars. However,
% if the journal you are submitting to favors bold math in the abstract,
% then you can use LaTeX's standard command \boldmath at the very start
% of the abstract to achieve this. Many IEEE journals frown on math
% in the abstract anyway.

% Note that keywords are not normally used for peerreview papers.
%\begin{IEEEkeywords}
%Cooperative diversity, decode and forward, piecewise linear
%\end{IEEEkeywords}



% For peer review papers, you can put extra information on the cover
% page as needed:
% \ifCLASSOPTIONpeerreview
% \begin{center} \bfseries EDICS Category: 3-BBND \end{center}
% \fi
%
% For peerreview papers, this IEEEtran command inserts a page break and
% creates the second title. It will be ignored for other modes.
%\IEEEpeerreviewmaketitle




If the letters of the word ASSASSINATION are arranged at random. Find the Probability that
\begin{enumerate}[label=(\alph*)]
\item Four $S's$ come consecutively in the word
\item Two  $I's$ and two $N's$ come together
\item All $A's$ are not coming together
\item No two $A's$ are coming together
\end{enumerate}
%\begin{table}[H]
	\centering
\begin{tabular}{|c|c|c|}
\hline
Random variable &Value &Definition\\ \hline
\multirow{3}{*}{X} &0 &Slips of Rs 1\\
&1 &Slips of Rs 5\\
&2 &Slips of Rs 13\\ \hline
\multirow{2}{*}{Y} &0 &Box A\\
&1 &Box B\\\hline
\end{tabular}
\caption{}
\label{tab:Distribution}
\end{table}
See \tabref{tab:Distribution}.
\begin{align}
p_{Y}\brak{k}= \begin{cases} 
      \frac{1}{3} & {k=0} \\
      \frac{2}{3 }& {k=1} 
   \end{cases}
   \\
p_{Y|X}\brak{0|0} = \frac{19}{25}\, 
p_{Y|X}\brak{0|1} = \frac{6}{25}\,
p_{Y|X}\brak{1|0} = \frac{45}{50}\,
p_{Y|X}\brak{1|2} = \frac{5}{50}
\end{align}
The desired probability is the probability that a slip drawn at random is marked other than Rs 1,
\begin{align}
&=1-p_X\brak{0}\\
&= p_X(1) + p_X(2)
\end{align}
Using Bayes theorem,
\begin{align}
&= p_Y\brak{0} \times \pr{Y=0 | X=1} + p_Y\brak{1} \times \pr{Y=1|X=2}\\
&=\frac{1}{3} \times \frac{6}{25} + \frac{2}{3} \times \frac{5}{50}\\
&=\frac{11}{75}
\end{align}

\newpage

%\tableofcontents

\bigskip

\renewcommand{\thefigure}{\theenumi}
\renewcommand{\thetable}{\theenumi}
%\renewcommand{\theequation}{\theenumi}

%\begin{abstract}
%%\boldmath
%In this letter, an algorithm for evaluating the exact analytical bit error rate  (BER)  for the piecewise linear (PL) combiner for  multiple relays is presented. Previous results were available only for upto three relays. The algorithm is unique in the sense that  the actual mathematical expressions, that are prohibitively large, need not be explicitly obtained. The diversity gain due to multiple relays is shown through plots of the analytical BER, well supported by simulations. 
%
%\end{abstract}
% IEEEtran.cls defaults to using nonbold math in the Abstract.
% This preserves the distinction between vectors and scalars. However,
% if the journal you are submitting to favors bold math in the abstract,
% then you can use LaTeX's standard command \boldmath at the very start
% of the abstract to achieve this. Many IEEE journals frown on math
% in the abstract anyway.

% Note that keywords are not normally used for peerreview papers.
%\begin{IEEEkeywords}
%Cooperative diversity, decode and forward, piecewise linear
%\end{IEEEkeywords}



% For peer review papers, you can put extra information on the cover
% page as needed:
% \ifCLASSOPTIONpeerreview
% \begin{center} \bfseries EDICS Category: 3-BBND \end{center}
% \fi
%
% For peerreview papers, this IEEEtran command inserts a page break and
% creates the second title. It will be ignored for other modes.
%\IEEEpeerreviewmaketitle




	\item One urn contains two black balls (labelled B1 and B2) and one white ball. A
	second urn contains one black ball and two white balls (labelled W1 and W2).
	Suppose the following experiment is performed. One of the two urns is chosen
	at random. Next a ball is randomly chosen from the urn. Then a second ball is
	chosen at random from the same urn without replacing the first ball.
	
	\begin{enumerate}
	\item What is the probability that two black balls are chosen?
	
	\item What is the probability that two balls of opposite colour are chosen?
	\end{enumerate}
	\solution
	%\begin{align}
    \label{eq:12.13.6.18.1}
	\because	\pr{A|B} &> \pr{A},\
\frac{\pr{AB}}{\pr{B}} > \pr{A}
\\
    \label{eq:12.13.6.18.2}
	\implies \pr{AB} &> \pr{A}\pr{B}
	\\
	\text{or, } \frac{\pr{AB}}{\pr{A}} &=\pr{B|A} > \pr{A}
\end{align}

\end{enumerate}

		%
\item 
Out of 100 students, two sections of 40 and 60 are formed. If you and your friend are among the 100 students, what is the probability that
\begin{enumerate}
\item you both enter the same section?
\item you both enter the different sections?
\end{enumerate}
\solution
		%\begin{enumerate}[label=\thesection.\arabic*,ref=\thesection.\theenumi]
	\item One card is drawn from a well-shuffled deck of 52 cards. Find the probability of getting
\begin{enumerate}
\item A king of red colour 
\item A face card 
\item A red face card
\item The jack of hearts
\item A spade
\item The queen of diamonds

\end{enumerate}
\solution
		%\begin{table}[H]
	\centering
\begin{tabular}{|c|c|c|}
\hline
Random variable &Value &Definition\\ \hline
\multirow{3}{*}{X} &0 &Slips of Rs 1\\
&1 &Slips of Rs 5\\
&2 &Slips of Rs 13\\ \hline
\multirow{2}{*}{Y} &0 &Box A\\
&1 &Box B\\\hline
\end{tabular}
\caption{}
\label{tab:Distribution}
\end{table}
See \tabref{tab:Distribution}.
\begin{align}
p_{Y}\brak{k}= \begin{cases} 
      \frac{1}{3} & {k=0} \\
      \frac{2}{3 }& {k=1} 
   \end{cases}
   \\
p_{Y|X}\brak{0|0} = \frac{19}{25}\, 
p_{Y|X}\brak{0|1} = \frac{6}{25}\,
p_{Y|X}\brak{1|0} = \frac{45}{50}\,
p_{Y|X}\brak{1|2} = \frac{5}{50}
\end{align}
The desired probability is the probability that a slip drawn at random is marked other than Rs 1,
\begin{align}
&=1-p_X\brak{0}\\
&= p_X(1) + p_X(2)
\end{align}
Using Bayes theorem,
\begin{align}
&= p_Y\brak{0} \times \pr{Y=0 | X=1} + p_Y\brak{1} \times \pr{Y=1|X=2}\\
&=\frac{1}{3} \times \frac{6}{25} + \frac{2}{3} \times \frac{5}{50}\\
&=\frac{11}{75}
\end{align}

\newpage

%\tableofcontents

\bigskip

\renewcommand{\thefigure}{\theenumi}
\renewcommand{\thetable}{\theenumi}
%\renewcommand{\theequation}{\theenumi}

%\begin{abstract}
%%\boldmath
%In this letter, an algorithm for evaluating the exact analytical bit error rate  (BER)  for the piecewise linear (PL) combiner for  multiple relays is presented. Previous results were available only for upto three relays. The algorithm is unique in the sense that  the actual mathematical expressions, that are prohibitively large, need not be explicitly obtained. The diversity gain due to multiple relays is shown through plots of the analytical BER, well supported by simulations. 
%
%\end{abstract}
% IEEEtran.cls defaults to using nonbold math in the Abstract.
% This preserves the distinction between vectors and scalars. However,
% if the journal you are submitting to favors bold math in the abstract,
% then you can use LaTeX's standard command \boldmath at the very start
% of the abstract to achieve this. Many IEEE journals frown on math
% in the abstract anyway.

% Note that keywords are not normally used for peerreview papers.
%\begin{IEEEkeywords}
%Cooperative diversity, decode and forward, piecewise linear
%\end{IEEEkeywords}



% For peer review papers, you can put extra information on the cover
% page as needed:
% \ifCLASSOPTIONpeerreview
% \begin{center} \bfseries EDICS Category: 3-BBND \end{center}
% \fi
%
% For peerreview papers, this IEEEtran command inserts a page break and
% creates the second title. It will be ignored for other modes.
%\IEEEpeerreviewmaketitle




	\item Five cards—the ten, jack, queen, king and ace of diamonds, are well-shuffled with their face downwards. One card is then picked up at random.
\begin{enumerate}
\item
What is the probability that the card is the queen? 
\item
If the queen is drawn and put aside, what is the probability that the second card picked up is (a) an ace? (b) a queen?\\
\end{enumerate}
\solution
		%\begin{enumerate}[label=\thesection.\arabic*,ref=\thesection.\theenumi]
	\item One card is drawn from a well-shuffled deck of 52 cards. Find the probability of getting
\begin{enumerate}
\item A king of red colour 
\item A face card 
\item A red face card
\item The jack of hearts
\item A spade
\item The queen of diamonds

\end{enumerate}
\solution
		%\input{ncert/10/15/1/14/main.tex}
	\item Five cards—the ten, jack, queen, king and ace of diamonds, are well-shuffled with their face downwards. One card is then picked up at random.
\begin{enumerate}
\item
What is the probability that the card is the queen? 
\item
If the queen is drawn and put aside, what is the probability that the second card picked up is (a) an ace? (b) a queen?\\
\end{enumerate}
\solution
		%\input{ncert/10/15/1/15/defs.tex}
	\item A bag contains $5$ red balls and some blue balls. If the probability of drawing a blue ball is double that if a red ball, determine the number of blue balls in the bag. 
		\\
\solution
		%\input{ncert/10/15/2/3/defs.tex}
	\item A card is selected from a pack of 52 cards.
 \begin{enumerate}[label=(\alph*)] 
                 \item How many points are there in the sample space?
                 \item Calculate the probability that the card is an ace of spades.
                 \item Calculate the probability that the card is (i) an ace and (ii) black card.
 \end{enumerate}
\solution
		%\input{ncert/11/16/3/4/main.tex}
\item Four cards are drawn from a well-shuffled deck of 52 cards. What is the probability of obtaining 3 diamonds and one spade.
\\
\solution
		%\input{ncert/11/16/4/2/defs.tex}
\item In a certain lottery 10,000 tickets are sold and ten equal prizes are awarded. What is the probability of not getting a prize if you buy (a) one ticket (b) two tickets (c) 10 tickets ?	
\\
\solution
		%\input{ncert/11/16/4/4/defs.tex}
		%
\item 
Out of 100 students, two sections of 40 and 60 are formed. If you and your friend are among the 100 students, what is the probability that
\begin{enumerate}
\item you both enter the same section?
\item you both enter the different sections?
\end{enumerate}
\solution
		%\input{ncert/11/16/4/5/defs.tex}
	\item 
The number lock of a suitcase has 4 wheels each labelled with ten digits i.e. from 0 to 9.The lock opens with a sequence of four digits with no repeats.What is the probability of a person getting the right sequence to open the suitcase.
\\
\solution
		%\input{ncert/11/16/4/10/defs.tex}
		%
\item 
Two cards are drawn at random and without replacement from a pack of 52 playing cards. Find the probability that both the cards are black.
\\
\solution
		%\input{ncert/12/13/2/2/defs.tex}
		\item A box of oranges is inspected by examining three randomly selected oranges drawn without replacement. If all the three oranges are good, the box is approved for sale, otherwise, it is rejected. Find the probability that a box containing 15 oranges out of which 12 are good and 3 are bad ones will be approved for sale.
		\label{ncert/12/13/2/3/defs.tex}
		\item Two balls are drawn at random with replacement from a box containing 10 black and 8 red balls. Find the probability that
		\label{ncert/12/13/2/12}
\begin{enumerate}
\item both balls are red.
\item first ball is black and second is red.
\item one of them is black and other is red.
\end{enumerate}

\item In a hostel, 60\% of the students read Hindi newspaper, 40\% read English newspaper and 20\% read both Hindi and English newspapers. A student is selected at random.
		\label{ncert/12/13/2/15}
\begin{enumerate}
\item Find the probability that she reads neither Hindi nor English newspapers.
\item If she reads Hindi newspaper, find the probability that she reads English newspaper.
\item If she reads English newspaper, find the probability that she reads Hindi newspaper.\\
\end{enumerate}
\item The probability of obtaining an even prime number on each die, when a pair of dice is rolled is 
\begin{enumerate}
    \item $0$ 
    
    \item $\frac{1}{3}$ 
    
    \item $\frac{1}{12}$ 
    
    \item $\frac{1}{36}$ 
\end{enumerate}
\solution
		%\input{ncert/12/13/2/17/defs.tex}
	\item A bag contains 4 red and 4 black balls, another bag contains 2 red and 6 black balls. One of the two bags is selected at random and a ball is drawn from the bag which is found to be red. Find the probability that the ball is drawn from the first bag.
\\
\solution
		%\input{ncert/12/13/3/2/main.tex}
  \item
  Cards with numbers 2 to 101 are placed in a box. A card is selected at random.Find the probability that the card has
\begin{enumerate}[label=(\roman*)]
	\item an even number 
	\item a square number
\end{enumerate}
\solution
%\input{exemplar/10/13/3/32/main.tex}
\item
The king, queen and jack of clubs are removed from a deck of 52 playing cards and then well shuffled. Now one card is drawn at random from the remaining cards.  Determine the probability that the card is
\begin{enumerate}[label=(\roman*)]
\item a club
\item 10 of hearts
\end{enumerate}
\solution
%\input{exemplar/10/13/3/29/main.tex}
\item A team of medical students doing their internship have to assist during surgeries
at a city hospital. The probabilities of surgeries rated as very complex, complex,
routine, simple or very simple are respectively, 0.15, 0.20, 0.31, 0.26, .08. Find
the probabilities that a particular surgery will be rated
\begin{enumerate}
	\item complex or very complex;
	\item neither very complex nor very simple;
	\item routine or complex
	\item routine or simple
\end{enumerate}
\solution
%\input{exemplar/11/16/3/8(1)/main.tex}
\item A card is selected from a pack of 52 cards.
\begin{enumerate}[label=(\alph*)]
    \item How many points are there in the sample space?
    \item Calculate the probability that the card is an ace of spades.
    \item Calculate the probability that the card is (i) an ace and (ii) black card.
\end{enumerate}
\solution
%\input{exemplar/11/16/3/4/main2.tex}
\item The probability that a non leap year selected at random will contain 53 sundays.
\\
\solution
%\input{exemplar/10/13/1/19/main.tex}
\item One of the four persons John, Rita, Aslam or Gurpreet will be promoted next
month. Consequently the sample space consists of four elementary outcomes
S = {John promoted, Rita promoted, Aslam promoted, Gurpreet promoted}
You are told that the chances of John’s promotion is same as that of Gurpreet,
Rita’s chances of promotion are twice as likely as Johns. Aslam’s chances are
four times that of John.
\begin{enumerate}
	\item Determine
	\begin{enumerate}
		\item P (John promoted)
		\item P (Rita promoted)
		\item P (Aslam promoted)
		\item P (Gurpreet promoted)
	\end{enumerate}
	\item If A = {John promoted or Gurpreet promoted}, find P (A).
\end{enumerate}
\solution
%\input{exemplar/11/16/3/10/main.tex}
\item A card is drawn from a deck of 52 cards. Find the probability of getting a king or a heart or a red card.\\
\solution
%\input{exemplar/11/16/3/15/main.tex}
\item The probability that a student will pass his examination is 0.73, the probability of
the student getting a compartment is 0.13, and the probability that the student will
either pass or get compartment is 0.96. State True or False.\\
\solution
%\input{exemplar/11/16/3/31/main.tex}
\item A card is selected from a pack of 52 cards\\
\begin{enumerate}[label=(\alph*)]
\item How many points are there in the sample space?
\item Calculate the probability that the cards is an ace of spades.
\item Calculate the probability that the card is (i) an ace (ii)black card.\\
\end{enumerate}
%\input{ncert/11/16/3/4_1/Prob_4.tex}
\item In a non-leap year, the probability of having 53 tuesdays or 53 wednesdays is\\
\solution
%\input{exemplar/11/16/3/18/main.tex}
\item There are 1000 sealed envelopes in a box, 10 of them contain a cash prize of
Rs 100 each, 100 of them contain a cash prize of Rs 50 each and 200 of them
contain a cash prize of Rs 10 each and rest do not contain any cash prize. If they
are well shuffled and an envelope is picked up out, what is the probability that it
contains no cash prize?\\
\solution
%\input{exemplar/10/13/3/34/main.tex}
\item 
A die is thrown and a card is selected at random from a deck of 52 playing cards. The probability of getting an even number on the die and a spade card.\\
\solution
%\input{exemplar/12/13/3/78/main.tex}
\item
If 4-digit numbers greater than 5,000 are randomly formed from the digits 0, 1, 3, 5, and 7, what is the probability of forming a number divisible by 5 when:
\begin{enumerate}
    \item The digits are repeated?
    \item The repetition of digits is not allowed?
\end{enumerate}
\solution
%\input{ncert/11/16/4/9/main.tex}
\item Consider the probability space $\brak{\Omega, \mathcal{G}, P}$ where $\Omega = [0,2]$ and $\mathcal{G} = \cbrak{\phi, \Omega, [0,1], (1,2]}$. Let $X$ and $Y$ be two functions on $\Omega$ defined as
\begin{align*}
    X(\omega) = 
    \begin{cases}
        1 & \text{if }\omega \in [0, 1]\\
        2 & \text{if }\omega \in (1, 2]
    \end{cases}
\end{align*}
and
\begin{align*}
    Y(\omega) = 
    \begin{cases}
        2 & \text{if }\omega \in [0, 1.5]\\
        3 & \text{if }\omega \in (1.5, 2].
    \end{cases}
\end{align*}
Then which one of the following statements is true?
\begin{enumerate}
    \item [(A)] $X$ is a random variable with respect to $\mathcal{G}$, but $Y$ is not a random variable with respect to $\mathcal{G}$.
    \item [(B)] $Y$ is a random variable with respect to $\mathcal{G}$, but $X$ is not a random variable with respect to $\mathcal{G}$.
    \item [(C)] Neither $X$ nor $Y$ is a random variable with respect to $\mathcal{G}$.
    \item [(D)] Both $X$ and $Y$ are random variables with respect to $\mathcal{G}$.
\end{enumerate} \hfill (GATE ST 2023)\\
\solution
%\input{gate/ST/2023/14/main.tex}
	\item  A die is loaded in such a way that each odd number is twice as likely to occur as
each even number. Find $P(G)$, where $G$ is the event that a number greater than
3 occurs on a single roll of the die.
\\
\solution
		%\input{exemplar/11/16/3/5/main.tex}
	\item All the jacks, queens and kings are removed from a deck of 52 playing cards. The remaining cards are well shuffled and then one card is drawn at random. Giving ace a value 1 similar value for other cards, find the probability that the card has a value 
		\begin{enumerate}
			\item 7
			\item greater than 7
			\item less than 7
		\end{enumerate}
		%\input{exemplar/10/13/3/30/main.tex}
  \item A Lot consists of 48 mobile phones of which 42 are good, 3 have only minor defects and 3 have major defects.Varnika will buy a phone if it is good but the trader will only buy a mobile if it has no major defects. One phone is selected at random from the lot. What is the probability that it is
\begin{enumerate}
	\item acceptable to Varnika?
            \item acceptable to the trader?
\end{enumerate}
\solution
	%\input{exemplar/10/13/3/40/main.tex}
 \item A student says that if you throw a die, it will show up 1 or not 1. Therefore, the probability of getting 1 and the probability of getting 'not 1' each is equal to $\frac{1}{2}$. Is this correct? Give reasons.\\
 \solution
        %\input{exemplar/10/13/2/9/main.tex}
   \item Four candidates A, B, C, D have ap-
plied for the assignment to coach a school cricket
team. If A is twice as likely to be selected as B, and
B and C are given about the same chance of being
selected, while C is twice as likely to be selected
as D, what are the probabilities that
\begin{enumerate}
\item C will be selected?
\item A will not be selected?
\end{enumerate}
	%\input{exemplar/11/16/3/9/main.tex}
 \item A bag contain 24 balls of which $x$ balls are red, $2x$ are white and $3x$ are blue. A ball is selected at random, What is the probability that it is
\begin{enumerate}[label=\alph*)]
\item not red ?
\item white ?
\end{enumerate}
%\input{exemplar/10/13/3/41/main.tex}
If the letters of the word ASSASSINATION are arranged at random. Find the Probability that
\begin{enumerate}[label=(\alph*)]
\item Four $S's$ come consecutively in the word
\item Two  $I's$ and two $N's$ come together
\item All $A's$ are not coming together
\item No two $A's$ are coming together
\end{enumerate}
%\input{exemplar/11/16/3/14/main.tex}
	\item One urn contains two black balls (labelled B1 and B2) and one white ball. A
	second urn contains one black ball and two white balls (labelled W1 and W2).
	Suppose the following experiment is performed. One of the two urns is chosen
	at random. Next a ball is randomly chosen from the urn. Then a second ball is
	chosen at random from the same urn without replacing the first ball.
	
	\begin{enumerate}
	\item What is the probability that two black balls are chosen?
	
	\item What is the probability that two balls of opposite colour are chosen?
	\end{enumerate}
	\solution
	%\input{exemplar/11/16/3/12/main1.tex}
\end{enumerate}

	\item A bag contains $5$ red balls and some blue balls. If the probability of drawing a blue ball is double that if a red ball, determine the number of blue balls in the bag. 
		\\
\solution
		%\begin{enumerate}[label=\thesection.\arabic*,ref=\thesection.\theenumi]
	\item One card is drawn from a well-shuffled deck of 52 cards. Find the probability of getting
\begin{enumerate}
\item A king of red colour 
\item A face card 
\item A red face card
\item The jack of hearts
\item A spade
\item The queen of diamonds

\end{enumerate}
\solution
		%\input{ncert/10/15/1/14/main.tex}
	\item Five cards—the ten, jack, queen, king and ace of diamonds, are well-shuffled with their face downwards. One card is then picked up at random.
\begin{enumerate}
\item
What is the probability that the card is the queen? 
\item
If the queen is drawn and put aside, what is the probability that the second card picked up is (a) an ace? (b) a queen?\\
\end{enumerate}
\solution
		%\input{ncert/10/15/1/15/defs.tex}
	\item A bag contains $5$ red balls and some blue balls. If the probability of drawing a blue ball is double that if a red ball, determine the number of blue balls in the bag. 
		\\
\solution
		%\input{ncert/10/15/2/3/defs.tex}
	\item A card is selected from a pack of 52 cards.
 \begin{enumerate}[label=(\alph*)] 
                 \item How many points are there in the sample space?
                 \item Calculate the probability that the card is an ace of spades.
                 \item Calculate the probability that the card is (i) an ace and (ii) black card.
 \end{enumerate}
\solution
		%\input{ncert/11/16/3/4/main.tex}
\item Four cards are drawn from a well-shuffled deck of 52 cards. What is the probability of obtaining 3 diamonds and one spade.
\\
\solution
		%\input{ncert/11/16/4/2/defs.tex}
\item In a certain lottery 10,000 tickets are sold and ten equal prizes are awarded. What is the probability of not getting a prize if you buy (a) one ticket (b) two tickets (c) 10 tickets ?	
\\
\solution
		%\input{ncert/11/16/4/4/defs.tex}
		%
\item 
Out of 100 students, two sections of 40 and 60 are formed. If you and your friend are among the 100 students, what is the probability that
\begin{enumerate}
\item you both enter the same section?
\item you both enter the different sections?
\end{enumerate}
\solution
		%\input{ncert/11/16/4/5/defs.tex}
	\item 
The number lock of a suitcase has 4 wheels each labelled with ten digits i.e. from 0 to 9.The lock opens with a sequence of four digits with no repeats.What is the probability of a person getting the right sequence to open the suitcase.
\\
\solution
		%\input{ncert/11/16/4/10/defs.tex}
		%
\item 
Two cards are drawn at random and without replacement from a pack of 52 playing cards. Find the probability that both the cards are black.
\\
\solution
		%\input{ncert/12/13/2/2/defs.tex}
		\item A box of oranges is inspected by examining three randomly selected oranges drawn without replacement. If all the three oranges are good, the box is approved for sale, otherwise, it is rejected. Find the probability that a box containing 15 oranges out of which 12 are good and 3 are bad ones will be approved for sale.
		\label{ncert/12/13/2/3/defs.tex}
		\item Two balls are drawn at random with replacement from a box containing 10 black and 8 red balls. Find the probability that
		\label{ncert/12/13/2/12}
\begin{enumerate}
\item both balls are red.
\item first ball is black and second is red.
\item one of them is black and other is red.
\end{enumerate}

\item In a hostel, 60\% of the students read Hindi newspaper, 40\% read English newspaper and 20\% read both Hindi and English newspapers. A student is selected at random.
		\label{ncert/12/13/2/15}
\begin{enumerate}
\item Find the probability that she reads neither Hindi nor English newspapers.
\item If she reads Hindi newspaper, find the probability that she reads English newspaper.
\item If she reads English newspaper, find the probability that she reads Hindi newspaper.\\
\end{enumerate}
\item The probability of obtaining an even prime number on each die, when a pair of dice is rolled is 
\begin{enumerate}
    \item $0$ 
    
    \item $\frac{1}{3}$ 
    
    \item $\frac{1}{12}$ 
    
    \item $\frac{1}{36}$ 
\end{enumerate}
\solution
		%\input{ncert/12/13/2/17/defs.tex}
	\item A bag contains 4 red and 4 black balls, another bag contains 2 red and 6 black balls. One of the two bags is selected at random and a ball is drawn from the bag which is found to be red. Find the probability that the ball is drawn from the first bag.
\\
\solution
		%\input{ncert/12/13/3/2/main.tex}
  \item
  Cards with numbers 2 to 101 are placed in a box. A card is selected at random.Find the probability that the card has
\begin{enumerate}[label=(\roman*)]
	\item an even number 
	\item a square number
\end{enumerate}
\solution
%\input{exemplar/10/13/3/32/main.tex}
\item
The king, queen and jack of clubs are removed from a deck of 52 playing cards and then well shuffled. Now one card is drawn at random from the remaining cards.  Determine the probability that the card is
\begin{enumerate}[label=(\roman*)]
\item a club
\item 10 of hearts
\end{enumerate}
\solution
%\input{exemplar/10/13/3/29/main.tex}
\item A team of medical students doing their internship have to assist during surgeries
at a city hospital. The probabilities of surgeries rated as very complex, complex,
routine, simple or very simple are respectively, 0.15, 0.20, 0.31, 0.26, .08. Find
the probabilities that a particular surgery will be rated
\begin{enumerate}
	\item complex or very complex;
	\item neither very complex nor very simple;
	\item routine or complex
	\item routine or simple
\end{enumerate}
\solution
%\input{exemplar/11/16/3/8(1)/main.tex}
\item A card is selected from a pack of 52 cards.
\begin{enumerate}[label=(\alph*)]
    \item How many points are there in the sample space?
    \item Calculate the probability that the card is an ace of spades.
    \item Calculate the probability that the card is (i) an ace and (ii) black card.
\end{enumerate}
\solution
%\input{exemplar/11/16/3/4/main2.tex}
\item The probability that a non leap year selected at random will contain 53 sundays.
\\
\solution
%\input{exemplar/10/13/1/19/main.tex}
\item One of the four persons John, Rita, Aslam or Gurpreet will be promoted next
month. Consequently the sample space consists of four elementary outcomes
S = {John promoted, Rita promoted, Aslam promoted, Gurpreet promoted}
You are told that the chances of John’s promotion is same as that of Gurpreet,
Rita’s chances of promotion are twice as likely as Johns. Aslam’s chances are
four times that of John.
\begin{enumerate}
	\item Determine
	\begin{enumerate}
		\item P (John promoted)
		\item P (Rita promoted)
		\item P (Aslam promoted)
		\item P (Gurpreet promoted)
	\end{enumerate}
	\item If A = {John promoted or Gurpreet promoted}, find P (A).
\end{enumerate}
\solution
%\input{exemplar/11/16/3/10/main.tex}
\item A card is drawn from a deck of 52 cards. Find the probability of getting a king or a heart or a red card.\\
\solution
%\input{exemplar/11/16/3/15/main.tex}
\item The probability that a student will pass his examination is 0.73, the probability of
the student getting a compartment is 0.13, and the probability that the student will
either pass or get compartment is 0.96. State True or False.\\
\solution
%\input{exemplar/11/16/3/31/main.tex}
\item A card is selected from a pack of 52 cards\\
\begin{enumerate}[label=(\alph*)]
\item How many points are there in the sample space?
\item Calculate the probability that the cards is an ace of spades.
\item Calculate the probability that the card is (i) an ace (ii)black card.\\
\end{enumerate}
%\input{ncert/11/16/3/4_1/Prob_4.tex}
\item In a non-leap year, the probability of having 53 tuesdays or 53 wednesdays is\\
\solution
%\input{exemplar/11/16/3/18/main.tex}
\item There are 1000 sealed envelopes in a box, 10 of them contain a cash prize of
Rs 100 each, 100 of them contain a cash prize of Rs 50 each and 200 of them
contain a cash prize of Rs 10 each and rest do not contain any cash prize. If they
are well shuffled and an envelope is picked up out, what is the probability that it
contains no cash prize?\\
\solution
%\input{exemplar/10/13/3/34/main.tex}
\item 
A die is thrown and a card is selected at random from a deck of 52 playing cards. The probability of getting an even number on the die and a spade card.\\
\solution
%\input{exemplar/12/13/3/78/main.tex}
\item
If 4-digit numbers greater than 5,000 are randomly formed from the digits 0, 1, 3, 5, and 7, what is the probability of forming a number divisible by 5 when:
\begin{enumerate}
    \item The digits are repeated?
    \item The repetition of digits is not allowed?
\end{enumerate}
\solution
%\input{ncert/11/16/4/9/main.tex}
\item Consider the probability space $\brak{\Omega, \mathcal{G}, P}$ where $\Omega = [0,2]$ and $\mathcal{G} = \cbrak{\phi, \Omega, [0,1], (1,2]}$. Let $X$ and $Y$ be two functions on $\Omega$ defined as
\begin{align*}
    X(\omega) = 
    \begin{cases}
        1 & \text{if }\omega \in [0, 1]\\
        2 & \text{if }\omega \in (1, 2]
    \end{cases}
\end{align*}
and
\begin{align*}
    Y(\omega) = 
    \begin{cases}
        2 & \text{if }\omega \in [0, 1.5]\\
        3 & \text{if }\omega \in (1.5, 2].
    \end{cases}
\end{align*}
Then which one of the following statements is true?
\begin{enumerate}
    \item [(A)] $X$ is a random variable with respect to $\mathcal{G}$, but $Y$ is not a random variable with respect to $\mathcal{G}$.
    \item [(B)] $Y$ is a random variable with respect to $\mathcal{G}$, but $X$ is not a random variable with respect to $\mathcal{G}$.
    \item [(C)] Neither $X$ nor $Y$ is a random variable with respect to $\mathcal{G}$.
    \item [(D)] Both $X$ and $Y$ are random variables with respect to $\mathcal{G}$.
\end{enumerate} \hfill (GATE ST 2023)\\
\solution
%\input{gate/ST/2023/14/main.tex}
	\item  A die is loaded in such a way that each odd number is twice as likely to occur as
each even number. Find $P(G)$, where $G$ is the event that a number greater than
3 occurs on a single roll of the die.
\\
\solution
		%\input{exemplar/11/16/3/5/main.tex}
	\item All the jacks, queens and kings are removed from a deck of 52 playing cards. The remaining cards are well shuffled and then one card is drawn at random. Giving ace a value 1 similar value for other cards, find the probability that the card has a value 
		\begin{enumerate}
			\item 7
			\item greater than 7
			\item less than 7
		\end{enumerate}
		%\input{exemplar/10/13/3/30/main.tex}
  \item A Lot consists of 48 mobile phones of which 42 are good, 3 have only minor defects and 3 have major defects.Varnika will buy a phone if it is good but the trader will only buy a mobile if it has no major defects. One phone is selected at random from the lot. What is the probability that it is
\begin{enumerate}
	\item acceptable to Varnika?
            \item acceptable to the trader?
\end{enumerate}
\solution
	%\input{exemplar/10/13/3/40/main.tex}
 \item A student says that if you throw a die, it will show up 1 or not 1. Therefore, the probability of getting 1 and the probability of getting 'not 1' each is equal to $\frac{1}{2}$. Is this correct? Give reasons.\\
 \solution
        %\input{exemplar/10/13/2/9/main.tex}
   \item Four candidates A, B, C, D have ap-
plied for the assignment to coach a school cricket
team. If A is twice as likely to be selected as B, and
B and C are given about the same chance of being
selected, while C is twice as likely to be selected
as D, what are the probabilities that
\begin{enumerate}
\item C will be selected?
\item A will not be selected?
\end{enumerate}
	%\input{exemplar/11/16/3/9/main.tex}
 \item A bag contain 24 balls of which $x$ balls are red, $2x$ are white and $3x$ are blue. A ball is selected at random, What is the probability that it is
\begin{enumerate}[label=\alph*)]
\item not red ?
\item white ?
\end{enumerate}
%\input{exemplar/10/13/3/41/main.tex}
If the letters of the word ASSASSINATION are arranged at random. Find the Probability that
\begin{enumerate}[label=(\alph*)]
\item Four $S's$ come consecutively in the word
\item Two  $I's$ and two $N's$ come together
\item All $A's$ are not coming together
\item No two $A's$ are coming together
\end{enumerate}
%\input{exemplar/11/16/3/14/main.tex}
	\item One urn contains two black balls (labelled B1 and B2) and one white ball. A
	second urn contains one black ball and two white balls (labelled W1 and W2).
	Suppose the following experiment is performed. One of the two urns is chosen
	at random. Next a ball is randomly chosen from the urn. Then a second ball is
	chosen at random from the same urn without replacing the first ball.
	
	\begin{enumerate}
	\item What is the probability that two black balls are chosen?
	
	\item What is the probability that two balls of opposite colour are chosen?
	\end{enumerate}
	\solution
	%\input{exemplar/11/16/3/12/main1.tex}
\end{enumerate}

	\item A card is selected from a pack of 52 cards.
 \begin{enumerate}[label=(\alph*)] 
                 \item How many points are there in the sample space?
                 \item Calculate the probability that the card is an ace of spades.
                 \item Calculate the probability that the card is (i) an ace and (ii) black card.
 \end{enumerate}
\solution
		%\begin{table}[H]
	\centering
\begin{tabular}{|c|c|c|}
\hline
Random variable &Value &Definition\\ \hline
\multirow{3}{*}{X} &0 &Slips of Rs 1\\
&1 &Slips of Rs 5\\
&2 &Slips of Rs 13\\ \hline
\multirow{2}{*}{Y} &0 &Box A\\
&1 &Box B\\\hline
\end{tabular}
\caption{}
\label{tab:Distribution}
\end{table}
See \tabref{tab:Distribution}.
\begin{align}
p_{Y}\brak{k}= \begin{cases} 
      \frac{1}{3} & {k=0} \\
      \frac{2}{3 }& {k=1} 
   \end{cases}
   \\
p_{Y|X}\brak{0|0} = \frac{19}{25}\, 
p_{Y|X}\brak{0|1} = \frac{6}{25}\,
p_{Y|X}\brak{1|0} = \frac{45}{50}\,
p_{Y|X}\brak{1|2} = \frac{5}{50}
\end{align}
The desired probability is the probability that a slip drawn at random is marked other than Rs 1,
\begin{align}
&=1-p_X\brak{0}\\
&= p_X(1) + p_X(2)
\end{align}
Using Bayes theorem,
\begin{align}
&= p_Y\brak{0} \times \pr{Y=0 | X=1} + p_Y\brak{1} \times \pr{Y=1|X=2}\\
&=\frac{1}{3} \times \frac{6}{25} + \frac{2}{3} \times \frac{5}{50}\\
&=\frac{11}{75}
\end{align}

\newpage

%\tableofcontents

\bigskip

\renewcommand{\thefigure}{\theenumi}
\renewcommand{\thetable}{\theenumi}
%\renewcommand{\theequation}{\theenumi}

%\begin{abstract}
%%\boldmath
%In this letter, an algorithm for evaluating the exact analytical bit error rate  (BER)  for the piecewise linear (PL) combiner for  multiple relays is presented. Previous results were available only for upto three relays. The algorithm is unique in the sense that  the actual mathematical expressions, that are prohibitively large, need not be explicitly obtained. The diversity gain due to multiple relays is shown through plots of the analytical BER, well supported by simulations. 
%
%\end{abstract}
% IEEEtran.cls defaults to using nonbold math in the Abstract.
% This preserves the distinction between vectors and scalars. However,
% if the journal you are submitting to favors bold math in the abstract,
% then you can use LaTeX's standard command \boldmath at the very start
% of the abstract to achieve this. Many IEEE journals frown on math
% in the abstract anyway.

% Note that keywords are not normally used for peerreview papers.
%\begin{IEEEkeywords}
%Cooperative diversity, decode and forward, piecewise linear
%\end{IEEEkeywords}



% For peer review papers, you can put extra information on the cover
% page as needed:
% \ifCLASSOPTIONpeerreview
% \begin{center} \bfseries EDICS Category: 3-BBND \end{center}
% \fi
%
% For peerreview papers, this IEEEtran command inserts a page break and
% creates the second title. It will be ignored for other modes.
%\IEEEpeerreviewmaketitle




\item Four cards are drawn from a well-shuffled deck of 52 cards. What is the probability of obtaining 3 diamonds and one spade.
\\
\solution
		%\begin{enumerate}[label=\thesection.\arabic*,ref=\thesection.\theenumi]
	\item One card is drawn from a well-shuffled deck of 52 cards. Find the probability of getting
\begin{enumerate}
\item A king of red colour 
\item A face card 
\item A red face card
\item The jack of hearts
\item A spade
\item The queen of diamonds

\end{enumerate}
\solution
		%\input{ncert/10/15/1/14/main.tex}
	\item Five cards—the ten, jack, queen, king and ace of diamonds, are well-shuffled with their face downwards. One card is then picked up at random.
\begin{enumerate}
\item
What is the probability that the card is the queen? 
\item
If the queen is drawn and put aside, what is the probability that the second card picked up is (a) an ace? (b) a queen?\\
\end{enumerate}
\solution
		%\input{ncert/10/15/1/15/defs.tex}
	\item A bag contains $5$ red balls and some blue balls. If the probability of drawing a blue ball is double that if a red ball, determine the number of blue balls in the bag. 
		\\
\solution
		%\input{ncert/10/15/2/3/defs.tex}
	\item A card is selected from a pack of 52 cards.
 \begin{enumerate}[label=(\alph*)] 
                 \item How many points are there in the sample space?
                 \item Calculate the probability that the card is an ace of spades.
                 \item Calculate the probability that the card is (i) an ace and (ii) black card.
 \end{enumerate}
\solution
		%\input{ncert/11/16/3/4/main.tex}
\item Four cards are drawn from a well-shuffled deck of 52 cards. What is the probability of obtaining 3 diamonds and one spade.
\\
\solution
		%\input{ncert/11/16/4/2/defs.tex}
\item In a certain lottery 10,000 tickets are sold and ten equal prizes are awarded. What is the probability of not getting a prize if you buy (a) one ticket (b) two tickets (c) 10 tickets ?	
\\
\solution
		%\input{ncert/11/16/4/4/defs.tex}
		%
\item 
Out of 100 students, two sections of 40 and 60 are formed. If you and your friend are among the 100 students, what is the probability that
\begin{enumerate}
\item you both enter the same section?
\item you both enter the different sections?
\end{enumerate}
\solution
		%\input{ncert/11/16/4/5/defs.tex}
	\item 
The number lock of a suitcase has 4 wheels each labelled with ten digits i.e. from 0 to 9.The lock opens with a sequence of four digits with no repeats.What is the probability of a person getting the right sequence to open the suitcase.
\\
\solution
		%\input{ncert/11/16/4/10/defs.tex}
		%
\item 
Two cards are drawn at random and without replacement from a pack of 52 playing cards. Find the probability that both the cards are black.
\\
\solution
		%\input{ncert/12/13/2/2/defs.tex}
		\item A box of oranges is inspected by examining three randomly selected oranges drawn without replacement. If all the three oranges are good, the box is approved for sale, otherwise, it is rejected. Find the probability that a box containing 15 oranges out of which 12 are good and 3 are bad ones will be approved for sale.
		\label{ncert/12/13/2/3/defs.tex}
		\item Two balls are drawn at random with replacement from a box containing 10 black and 8 red balls. Find the probability that
		\label{ncert/12/13/2/12}
\begin{enumerate}
\item both balls are red.
\item first ball is black and second is red.
\item one of them is black and other is red.
\end{enumerate}

\item In a hostel, 60\% of the students read Hindi newspaper, 40\% read English newspaper and 20\% read both Hindi and English newspapers. A student is selected at random.
		\label{ncert/12/13/2/15}
\begin{enumerate}
\item Find the probability that she reads neither Hindi nor English newspapers.
\item If she reads Hindi newspaper, find the probability that she reads English newspaper.
\item If she reads English newspaper, find the probability that she reads Hindi newspaper.\\
\end{enumerate}
\item The probability of obtaining an even prime number on each die, when a pair of dice is rolled is 
\begin{enumerate}
    \item $0$ 
    
    \item $\frac{1}{3}$ 
    
    \item $\frac{1}{12}$ 
    
    \item $\frac{1}{36}$ 
\end{enumerate}
\solution
		%\input{ncert/12/13/2/17/defs.tex}
	\item A bag contains 4 red and 4 black balls, another bag contains 2 red and 6 black balls. One of the two bags is selected at random and a ball is drawn from the bag which is found to be red. Find the probability that the ball is drawn from the first bag.
\\
\solution
		%\input{ncert/12/13/3/2/main.tex}
  \item
  Cards with numbers 2 to 101 are placed in a box. A card is selected at random.Find the probability that the card has
\begin{enumerate}[label=(\roman*)]
	\item an even number 
	\item a square number
\end{enumerate}
\solution
%\input{exemplar/10/13/3/32/main.tex}
\item
The king, queen and jack of clubs are removed from a deck of 52 playing cards and then well shuffled. Now one card is drawn at random from the remaining cards.  Determine the probability that the card is
\begin{enumerate}[label=(\roman*)]
\item a club
\item 10 of hearts
\end{enumerate}
\solution
%\input{exemplar/10/13/3/29/main.tex}
\item A team of medical students doing their internship have to assist during surgeries
at a city hospital. The probabilities of surgeries rated as very complex, complex,
routine, simple or very simple are respectively, 0.15, 0.20, 0.31, 0.26, .08. Find
the probabilities that a particular surgery will be rated
\begin{enumerate}
	\item complex or very complex;
	\item neither very complex nor very simple;
	\item routine or complex
	\item routine or simple
\end{enumerate}
\solution
%\input{exemplar/11/16/3/8(1)/main.tex}
\item A card is selected from a pack of 52 cards.
\begin{enumerate}[label=(\alph*)]
    \item How many points are there in the sample space?
    \item Calculate the probability that the card is an ace of spades.
    \item Calculate the probability that the card is (i) an ace and (ii) black card.
\end{enumerate}
\solution
%\input{exemplar/11/16/3/4/main2.tex}
\item The probability that a non leap year selected at random will contain 53 sundays.
\\
\solution
%\input{exemplar/10/13/1/19/main.tex}
\item One of the four persons John, Rita, Aslam or Gurpreet will be promoted next
month. Consequently the sample space consists of four elementary outcomes
S = {John promoted, Rita promoted, Aslam promoted, Gurpreet promoted}
You are told that the chances of John’s promotion is same as that of Gurpreet,
Rita’s chances of promotion are twice as likely as Johns. Aslam’s chances are
four times that of John.
\begin{enumerate}
	\item Determine
	\begin{enumerate}
		\item P (John promoted)
		\item P (Rita promoted)
		\item P (Aslam promoted)
		\item P (Gurpreet promoted)
	\end{enumerate}
	\item If A = {John promoted or Gurpreet promoted}, find P (A).
\end{enumerate}
\solution
%\input{exemplar/11/16/3/10/main.tex}
\item A card is drawn from a deck of 52 cards. Find the probability of getting a king or a heart or a red card.\\
\solution
%\input{exemplar/11/16/3/15/main.tex}
\item The probability that a student will pass his examination is 0.73, the probability of
the student getting a compartment is 0.13, and the probability that the student will
either pass or get compartment is 0.96. State True or False.\\
\solution
%\input{exemplar/11/16/3/31/main.tex}
\item A card is selected from a pack of 52 cards\\
\begin{enumerate}[label=(\alph*)]
\item How many points are there in the sample space?
\item Calculate the probability that the cards is an ace of spades.
\item Calculate the probability that the card is (i) an ace (ii)black card.\\
\end{enumerate}
%\input{ncert/11/16/3/4_1/Prob_4.tex}
\item In a non-leap year, the probability of having 53 tuesdays or 53 wednesdays is\\
\solution
%\input{exemplar/11/16/3/18/main.tex}
\item There are 1000 sealed envelopes in a box, 10 of them contain a cash prize of
Rs 100 each, 100 of them contain a cash prize of Rs 50 each and 200 of them
contain a cash prize of Rs 10 each and rest do not contain any cash prize. If they
are well shuffled and an envelope is picked up out, what is the probability that it
contains no cash prize?\\
\solution
%\input{exemplar/10/13/3/34/main.tex}
\item 
A die is thrown and a card is selected at random from a deck of 52 playing cards. The probability of getting an even number on the die and a spade card.\\
\solution
%\input{exemplar/12/13/3/78/main.tex}
\item
If 4-digit numbers greater than 5,000 are randomly formed from the digits 0, 1, 3, 5, and 7, what is the probability of forming a number divisible by 5 when:
\begin{enumerate}
    \item The digits are repeated?
    \item The repetition of digits is not allowed?
\end{enumerate}
\solution
%\input{ncert/11/16/4/9/main.tex}
\item Consider the probability space $\brak{\Omega, \mathcal{G}, P}$ where $\Omega = [0,2]$ and $\mathcal{G} = \cbrak{\phi, \Omega, [0,1], (1,2]}$. Let $X$ and $Y$ be two functions on $\Omega$ defined as
\begin{align*}
    X(\omega) = 
    \begin{cases}
        1 & \text{if }\omega \in [0, 1]\\
        2 & \text{if }\omega \in (1, 2]
    \end{cases}
\end{align*}
and
\begin{align*}
    Y(\omega) = 
    \begin{cases}
        2 & \text{if }\omega \in [0, 1.5]\\
        3 & \text{if }\omega \in (1.5, 2].
    \end{cases}
\end{align*}
Then which one of the following statements is true?
\begin{enumerate}
    \item [(A)] $X$ is a random variable with respect to $\mathcal{G}$, but $Y$ is not a random variable with respect to $\mathcal{G}$.
    \item [(B)] $Y$ is a random variable with respect to $\mathcal{G}$, but $X$ is not a random variable with respect to $\mathcal{G}$.
    \item [(C)] Neither $X$ nor $Y$ is a random variable with respect to $\mathcal{G}$.
    \item [(D)] Both $X$ and $Y$ are random variables with respect to $\mathcal{G}$.
\end{enumerate} \hfill (GATE ST 2023)\\
\solution
%\input{gate/ST/2023/14/main.tex}
	\item  A die is loaded in such a way that each odd number is twice as likely to occur as
each even number. Find $P(G)$, where $G$ is the event that a number greater than
3 occurs on a single roll of the die.
\\
\solution
		%\input{exemplar/11/16/3/5/main.tex}
	\item All the jacks, queens and kings are removed from a deck of 52 playing cards. The remaining cards are well shuffled and then one card is drawn at random. Giving ace a value 1 similar value for other cards, find the probability that the card has a value 
		\begin{enumerate}
			\item 7
			\item greater than 7
			\item less than 7
		\end{enumerate}
		%\input{exemplar/10/13/3/30/main.tex}
  \item A Lot consists of 48 mobile phones of which 42 are good, 3 have only minor defects and 3 have major defects.Varnika will buy a phone if it is good but the trader will only buy a mobile if it has no major defects. One phone is selected at random from the lot. What is the probability that it is
\begin{enumerate}
	\item acceptable to Varnika?
            \item acceptable to the trader?
\end{enumerate}
\solution
	%\input{exemplar/10/13/3/40/main.tex}
 \item A student says that if you throw a die, it will show up 1 or not 1. Therefore, the probability of getting 1 and the probability of getting 'not 1' each is equal to $\frac{1}{2}$. Is this correct? Give reasons.\\
 \solution
        %\input{exemplar/10/13/2/9/main.tex}
   \item Four candidates A, B, C, D have ap-
plied for the assignment to coach a school cricket
team. If A is twice as likely to be selected as B, and
B and C are given about the same chance of being
selected, while C is twice as likely to be selected
as D, what are the probabilities that
\begin{enumerate}
\item C will be selected?
\item A will not be selected?
\end{enumerate}
	%\input{exemplar/11/16/3/9/main.tex}
 \item A bag contain 24 balls of which $x$ balls are red, $2x$ are white and $3x$ are blue. A ball is selected at random, What is the probability that it is
\begin{enumerate}[label=\alph*)]
\item not red ?
\item white ?
\end{enumerate}
%\input{exemplar/10/13/3/41/main.tex}
If the letters of the word ASSASSINATION are arranged at random. Find the Probability that
\begin{enumerate}[label=(\alph*)]
\item Four $S's$ come consecutively in the word
\item Two  $I's$ and two $N's$ come together
\item All $A's$ are not coming together
\item No two $A's$ are coming together
\end{enumerate}
%\input{exemplar/11/16/3/14/main.tex}
	\item One urn contains two black balls (labelled B1 and B2) and one white ball. A
	second urn contains one black ball and two white balls (labelled W1 and W2).
	Suppose the following experiment is performed. One of the two urns is chosen
	at random. Next a ball is randomly chosen from the urn. Then a second ball is
	chosen at random from the same urn without replacing the first ball.
	
	\begin{enumerate}
	\item What is the probability that two black balls are chosen?
	
	\item What is the probability that two balls of opposite colour are chosen?
	\end{enumerate}
	\solution
	%\input{exemplar/11/16/3/12/main1.tex}
\end{enumerate}

\item In a certain lottery 10,000 tickets are sold and ten equal prizes are awarded. What is the probability of not getting a prize if you buy (a) one ticket (b) two tickets (c) 10 tickets ?	
\\
\solution
		%\begin{enumerate}[label=\thesection.\arabic*,ref=\thesection.\theenumi]
	\item One card is drawn from a well-shuffled deck of 52 cards. Find the probability of getting
\begin{enumerate}
\item A king of red colour 
\item A face card 
\item A red face card
\item The jack of hearts
\item A spade
\item The queen of diamonds

\end{enumerate}
\solution
		%\input{ncert/10/15/1/14/main.tex}
	\item Five cards—the ten, jack, queen, king and ace of diamonds, are well-shuffled with their face downwards. One card is then picked up at random.
\begin{enumerate}
\item
What is the probability that the card is the queen? 
\item
If the queen is drawn and put aside, what is the probability that the second card picked up is (a) an ace? (b) a queen?\\
\end{enumerate}
\solution
		%\input{ncert/10/15/1/15/defs.tex}
	\item A bag contains $5$ red balls and some blue balls. If the probability of drawing a blue ball is double that if a red ball, determine the number of blue balls in the bag. 
		\\
\solution
		%\input{ncert/10/15/2/3/defs.tex}
	\item A card is selected from a pack of 52 cards.
 \begin{enumerate}[label=(\alph*)] 
                 \item How many points are there in the sample space?
                 \item Calculate the probability that the card is an ace of spades.
                 \item Calculate the probability that the card is (i) an ace and (ii) black card.
 \end{enumerate}
\solution
		%\input{ncert/11/16/3/4/main.tex}
\item Four cards are drawn from a well-shuffled deck of 52 cards. What is the probability of obtaining 3 diamonds and one spade.
\\
\solution
		%\input{ncert/11/16/4/2/defs.tex}
\item In a certain lottery 10,000 tickets are sold and ten equal prizes are awarded. What is the probability of not getting a prize if you buy (a) one ticket (b) two tickets (c) 10 tickets ?	
\\
\solution
		%\input{ncert/11/16/4/4/defs.tex}
		%
\item 
Out of 100 students, two sections of 40 and 60 are formed. If you and your friend are among the 100 students, what is the probability that
\begin{enumerate}
\item you both enter the same section?
\item you both enter the different sections?
\end{enumerate}
\solution
		%\input{ncert/11/16/4/5/defs.tex}
	\item 
The number lock of a suitcase has 4 wheels each labelled with ten digits i.e. from 0 to 9.The lock opens with a sequence of four digits with no repeats.What is the probability of a person getting the right sequence to open the suitcase.
\\
\solution
		%\input{ncert/11/16/4/10/defs.tex}
		%
\item 
Two cards are drawn at random and without replacement from a pack of 52 playing cards. Find the probability that both the cards are black.
\\
\solution
		%\input{ncert/12/13/2/2/defs.tex}
		\item A box of oranges is inspected by examining three randomly selected oranges drawn without replacement. If all the three oranges are good, the box is approved for sale, otherwise, it is rejected. Find the probability that a box containing 15 oranges out of which 12 are good and 3 are bad ones will be approved for sale.
		\label{ncert/12/13/2/3/defs.tex}
		\item Two balls are drawn at random with replacement from a box containing 10 black and 8 red balls. Find the probability that
		\label{ncert/12/13/2/12}
\begin{enumerate}
\item both balls are red.
\item first ball is black and second is red.
\item one of them is black and other is red.
\end{enumerate}

\item In a hostel, 60\% of the students read Hindi newspaper, 40\% read English newspaper and 20\% read both Hindi and English newspapers. A student is selected at random.
		\label{ncert/12/13/2/15}
\begin{enumerate}
\item Find the probability that she reads neither Hindi nor English newspapers.
\item If she reads Hindi newspaper, find the probability that she reads English newspaper.
\item If she reads English newspaper, find the probability that she reads Hindi newspaper.\\
\end{enumerate}
\item The probability of obtaining an even prime number on each die, when a pair of dice is rolled is 
\begin{enumerate}
    \item $0$ 
    
    \item $\frac{1}{3}$ 
    
    \item $\frac{1}{12}$ 
    
    \item $\frac{1}{36}$ 
\end{enumerate}
\solution
		%\input{ncert/12/13/2/17/defs.tex}
	\item A bag contains 4 red and 4 black balls, another bag contains 2 red and 6 black balls. One of the two bags is selected at random and a ball is drawn from the bag which is found to be red. Find the probability that the ball is drawn from the first bag.
\\
\solution
		%\input{ncert/12/13/3/2/main.tex}
  \item
  Cards with numbers 2 to 101 are placed in a box. A card is selected at random.Find the probability that the card has
\begin{enumerate}[label=(\roman*)]
	\item an even number 
	\item a square number
\end{enumerate}
\solution
%\input{exemplar/10/13/3/32/main.tex}
\item
The king, queen and jack of clubs are removed from a deck of 52 playing cards and then well shuffled. Now one card is drawn at random from the remaining cards.  Determine the probability that the card is
\begin{enumerate}[label=(\roman*)]
\item a club
\item 10 of hearts
\end{enumerate}
\solution
%\input{exemplar/10/13/3/29/main.tex}
\item A team of medical students doing their internship have to assist during surgeries
at a city hospital. The probabilities of surgeries rated as very complex, complex,
routine, simple or very simple are respectively, 0.15, 0.20, 0.31, 0.26, .08. Find
the probabilities that a particular surgery will be rated
\begin{enumerate}
	\item complex or very complex;
	\item neither very complex nor very simple;
	\item routine or complex
	\item routine or simple
\end{enumerate}
\solution
%\input{exemplar/11/16/3/8(1)/main.tex}
\item A card is selected from a pack of 52 cards.
\begin{enumerate}[label=(\alph*)]
    \item How many points are there in the sample space?
    \item Calculate the probability that the card is an ace of spades.
    \item Calculate the probability that the card is (i) an ace and (ii) black card.
\end{enumerate}
\solution
%\input{exemplar/11/16/3/4/main2.tex}
\item The probability that a non leap year selected at random will contain 53 sundays.
\\
\solution
%\input{exemplar/10/13/1/19/main.tex}
\item One of the four persons John, Rita, Aslam or Gurpreet will be promoted next
month. Consequently the sample space consists of four elementary outcomes
S = {John promoted, Rita promoted, Aslam promoted, Gurpreet promoted}
You are told that the chances of John’s promotion is same as that of Gurpreet,
Rita’s chances of promotion are twice as likely as Johns. Aslam’s chances are
four times that of John.
\begin{enumerate}
	\item Determine
	\begin{enumerate}
		\item P (John promoted)
		\item P (Rita promoted)
		\item P (Aslam promoted)
		\item P (Gurpreet promoted)
	\end{enumerate}
	\item If A = {John promoted or Gurpreet promoted}, find P (A).
\end{enumerate}
\solution
%\input{exemplar/11/16/3/10/main.tex}
\item A card is drawn from a deck of 52 cards. Find the probability of getting a king or a heart or a red card.\\
\solution
%\input{exemplar/11/16/3/15/main.tex}
\item The probability that a student will pass his examination is 0.73, the probability of
the student getting a compartment is 0.13, and the probability that the student will
either pass or get compartment is 0.96. State True or False.\\
\solution
%\input{exemplar/11/16/3/31/main.tex}
\item A card is selected from a pack of 52 cards\\
\begin{enumerate}[label=(\alph*)]
\item How many points are there in the sample space?
\item Calculate the probability that the cards is an ace of spades.
\item Calculate the probability that the card is (i) an ace (ii)black card.\\
\end{enumerate}
%\input{ncert/11/16/3/4_1/Prob_4.tex}
\item In a non-leap year, the probability of having 53 tuesdays or 53 wednesdays is\\
\solution
%\input{exemplar/11/16/3/18/main.tex}
\item There are 1000 sealed envelopes in a box, 10 of them contain a cash prize of
Rs 100 each, 100 of them contain a cash prize of Rs 50 each and 200 of them
contain a cash prize of Rs 10 each and rest do not contain any cash prize. If they
are well shuffled and an envelope is picked up out, what is the probability that it
contains no cash prize?\\
\solution
%\input{exemplar/10/13/3/34/main.tex}
\item 
A die is thrown and a card is selected at random from a deck of 52 playing cards. The probability of getting an even number on the die and a spade card.\\
\solution
%\input{exemplar/12/13/3/78/main.tex}
\item
If 4-digit numbers greater than 5,000 are randomly formed from the digits 0, 1, 3, 5, and 7, what is the probability of forming a number divisible by 5 when:
\begin{enumerate}
    \item The digits are repeated?
    \item The repetition of digits is not allowed?
\end{enumerate}
\solution
%\input{ncert/11/16/4/9/main.tex}
\item Consider the probability space $\brak{\Omega, \mathcal{G}, P}$ where $\Omega = [0,2]$ and $\mathcal{G} = \cbrak{\phi, \Omega, [0,1], (1,2]}$. Let $X$ and $Y$ be two functions on $\Omega$ defined as
\begin{align*}
    X(\omega) = 
    \begin{cases}
        1 & \text{if }\omega \in [0, 1]\\
        2 & \text{if }\omega \in (1, 2]
    \end{cases}
\end{align*}
and
\begin{align*}
    Y(\omega) = 
    \begin{cases}
        2 & \text{if }\omega \in [0, 1.5]\\
        3 & \text{if }\omega \in (1.5, 2].
    \end{cases}
\end{align*}
Then which one of the following statements is true?
\begin{enumerate}
    \item [(A)] $X$ is a random variable with respect to $\mathcal{G}$, but $Y$ is not a random variable with respect to $\mathcal{G}$.
    \item [(B)] $Y$ is a random variable with respect to $\mathcal{G}$, but $X$ is not a random variable with respect to $\mathcal{G}$.
    \item [(C)] Neither $X$ nor $Y$ is a random variable with respect to $\mathcal{G}$.
    \item [(D)] Both $X$ and $Y$ are random variables with respect to $\mathcal{G}$.
\end{enumerate} \hfill (GATE ST 2023)\\
\solution
%\input{gate/ST/2023/14/main.tex}
	\item  A die is loaded in such a way that each odd number is twice as likely to occur as
each even number. Find $P(G)$, where $G$ is the event that a number greater than
3 occurs on a single roll of the die.
\\
\solution
		%\input{exemplar/11/16/3/5/main.tex}
	\item All the jacks, queens and kings are removed from a deck of 52 playing cards. The remaining cards are well shuffled and then one card is drawn at random. Giving ace a value 1 similar value for other cards, find the probability that the card has a value 
		\begin{enumerate}
			\item 7
			\item greater than 7
			\item less than 7
		\end{enumerate}
		%\input{exemplar/10/13/3/30/main.tex}
  \item A Lot consists of 48 mobile phones of which 42 are good, 3 have only minor defects and 3 have major defects.Varnika will buy a phone if it is good but the trader will only buy a mobile if it has no major defects. One phone is selected at random from the lot. What is the probability that it is
\begin{enumerate}
	\item acceptable to Varnika?
            \item acceptable to the trader?
\end{enumerate}
\solution
	%\input{exemplar/10/13/3/40/main.tex}
 \item A student says that if you throw a die, it will show up 1 or not 1. Therefore, the probability of getting 1 and the probability of getting 'not 1' each is equal to $\frac{1}{2}$. Is this correct? Give reasons.\\
 \solution
        %\input{exemplar/10/13/2/9/main.tex}
   \item Four candidates A, B, C, D have ap-
plied for the assignment to coach a school cricket
team. If A is twice as likely to be selected as B, and
B and C are given about the same chance of being
selected, while C is twice as likely to be selected
as D, what are the probabilities that
\begin{enumerate}
\item C will be selected?
\item A will not be selected?
\end{enumerate}
	%\input{exemplar/11/16/3/9/main.tex}
 \item A bag contain 24 balls of which $x$ balls are red, $2x$ are white and $3x$ are blue. A ball is selected at random, What is the probability that it is
\begin{enumerate}[label=\alph*)]
\item not red ?
\item white ?
\end{enumerate}
%\input{exemplar/10/13/3/41/main.tex}
If the letters of the word ASSASSINATION are arranged at random. Find the Probability that
\begin{enumerate}[label=(\alph*)]
\item Four $S's$ come consecutively in the word
\item Two  $I's$ and two $N's$ come together
\item All $A's$ are not coming together
\item No two $A's$ are coming together
\end{enumerate}
%\input{exemplar/11/16/3/14/main.tex}
	\item One urn contains two black balls (labelled B1 and B2) and one white ball. A
	second urn contains one black ball and two white balls (labelled W1 and W2).
	Suppose the following experiment is performed. One of the two urns is chosen
	at random. Next a ball is randomly chosen from the urn. Then a second ball is
	chosen at random from the same urn without replacing the first ball.
	
	\begin{enumerate}
	\item What is the probability that two black balls are chosen?
	
	\item What is the probability that two balls of opposite colour are chosen?
	\end{enumerate}
	\solution
	%\input{exemplar/11/16/3/12/main1.tex}
\end{enumerate}

		%
\item 
Out of 100 students, two sections of 40 and 60 are formed. If you and your friend are among the 100 students, what is the probability that
\begin{enumerate}
\item you both enter the same section?
\item you both enter the different sections?
\end{enumerate}
\solution
		%\begin{enumerate}[label=\thesection.\arabic*,ref=\thesection.\theenumi]
	\item One card is drawn from a well-shuffled deck of 52 cards. Find the probability of getting
\begin{enumerate}
\item A king of red colour 
\item A face card 
\item A red face card
\item The jack of hearts
\item A spade
\item The queen of diamonds

\end{enumerate}
\solution
		%\input{ncert/10/15/1/14/main.tex}
	\item Five cards—the ten, jack, queen, king and ace of diamonds, are well-shuffled with their face downwards. One card is then picked up at random.
\begin{enumerate}
\item
What is the probability that the card is the queen? 
\item
If the queen is drawn and put aside, what is the probability that the second card picked up is (a) an ace? (b) a queen?\\
\end{enumerate}
\solution
		%\input{ncert/10/15/1/15/defs.tex}
	\item A bag contains $5$ red balls and some blue balls. If the probability of drawing a blue ball is double that if a red ball, determine the number of blue balls in the bag. 
		\\
\solution
		%\input{ncert/10/15/2/3/defs.tex}
	\item A card is selected from a pack of 52 cards.
 \begin{enumerate}[label=(\alph*)] 
                 \item How many points are there in the sample space?
                 \item Calculate the probability that the card is an ace of spades.
                 \item Calculate the probability that the card is (i) an ace and (ii) black card.
 \end{enumerate}
\solution
		%\input{ncert/11/16/3/4/main.tex}
\item Four cards are drawn from a well-shuffled deck of 52 cards. What is the probability of obtaining 3 diamonds and one spade.
\\
\solution
		%\input{ncert/11/16/4/2/defs.tex}
\item In a certain lottery 10,000 tickets are sold and ten equal prizes are awarded. What is the probability of not getting a prize if you buy (a) one ticket (b) two tickets (c) 10 tickets ?	
\\
\solution
		%\input{ncert/11/16/4/4/defs.tex}
		%
\item 
Out of 100 students, two sections of 40 and 60 are formed. If you and your friend are among the 100 students, what is the probability that
\begin{enumerate}
\item you both enter the same section?
\item you both enter the different sections?
\end{enumerate}
\solution
		%\input{ncert/11/16/4/5/defs.tex}
	\item 
The number lock of a suitcase has 4 wheels each labelled with ten digits i.e. from 0 to 9.The lock opens with a sequence of four digits with no repeats.What is the probability of a person getting the right sequence to open the suitcase.
\\
\solution
		%\input{ncert/11/16/4/10/defs.tex}
		%
\item 
Two cards are drawn at random and without replacement from a pack of 52 playing cards. Find the probability that both the cards are black.
\\
\solution
		%\input{ncert/12/13/2/2/defs.tex}
		\item A box of oranges is inspected by examining three randomly selected oranges drawn without replacement. If all the three oranges are good, the box is approved for sale, otherwise, it is rejected. Find the probability that a box containing 15 oranges out of which 12 are good and 3 are bad ones will be approved for sale.
		\label{ncert/12/13/2/3/defs.tex}
		\item Two balls are drawn at random with replacement from a box containing 10 black and 8 red balls. Find the probability that
		\label{ncert/12/13/2/12}
\begin{enumerate}
\item both balls are red.
\item first ball is black and second is red.
\item one of them is black and other is red.
\end{enumerate}

\item In a hostel, 60\% of the students read Hindi newspaper, 40\% read English newspaper and 20\% read both Hindi and English newspapers. A student is selected at random.
		\label{ncert/12/13/2/15}
\begin{enumerate}
\item Find the probability that she reads neither Hindi nor English newspapers.
\item If she reads Hindi newspaper, find the probability that she reads English newspaper.
\item If she reads English newspaper, find the probability that she reads Hindi newspaper.\\
\end{enumerate}
\item The probability of obtaining an even prime number on each die, when a pair of dice is rolled is 
\begin{enumerate}
    \item $0$ 
    
    \item $\frac{1}{3}$ 
    
    \item $\frac{1}{12}$ 
    
    \item $\frac{1}{36}$ 
\end{enumerate}
\solution
		%\input{ncert/12/13/2/17/defs.tex}
	\item A bag contains 4 red and 4 black balls, another bag contains 2 red and 6 black balls. One of the two bags is selected at random and a ball is drawn from the bag which is found to be red. Find the probability that the ball is drawn from the first bag.
\\
\solution
		%\input{ncert/12/13/3/2/main.tex}
  \item
  Cards with numbers 2 to 101 are placed in a box. A card is selected at random.Find the probability that the card has
\begin{enumerate}[label=(\roman*)]
	\item an even number 
	\item a square number
\end{enumerate}
\solution
%\input{exemplar/10/13/3/32/main.tex}
\item
The king, queen and jack of clubs are removed from a deck of 52 playing cards and then well shuffled. Now one card is drawn at random from the remaining cards.  Determine the probability that the card is
\begin{enumerate}[label=(\roman*)]
\item a club
\item 10 of hearts
\end{enumerate}
\solution
%\input{exemplar/10/13/3/29/main.tex}
\item A team of medical students doing their internship have to assist during surgeries
at a city hospital. The probabilities of surgeries rated as very complex, complex,
routine, simple or very simple are respectively, 0.15, 0.20, 0.31, 0.26, .08. Find
the probabilities that a particular surgery will be rated
\begin{enumerate}
	\item complex or very complex;
	\item neither very complex nor very simple;
	\item routine or complex
	\item routine or simple
\end{enumerate}
\solution
%\input{exemplar/11/16/3/8(1)/main.tex}
\item A card is selected from a pack of 52 cards.
\begin{enumerate}[label=(\alph*)]
    \item How many points are there in the sample space?
    \item Calculate the probability that the card is an ace of spades.
    \item Calculate the probability that the card is (i) an ace and (ii) black card.
\end{enumerate}
\solution
%\input{exemplar/11/16/3/4/main2.tex}
\item The probability that a non leap year selected at random will contain 53 sundays.
\\
\solution
%\input{exemplar/10/13/1/19/main.tex}
\item One of the four persons John, Rita, Aslam or Gurpreet will be promoted next
month. Consequently the sample space consists of four elementary outcomes
S = {John promoted, Rita promoted, Aslam promoted, Gurpreet promoted}
You are told that the chances of John’s promotion is same as that of Gurpreet,
Rita’s chances of promotion are twice as likely as Johns. Aslam’s chances are
four times that of John.
\begin{enumerate}
	\item Determine
	\begin{enumerate}
		\item P (John promoted)
		\item P (Rita promoted)
		\item P (Aslam promoted)
		\item P (Gurpreet promoted)
	\end{enumerate}
	\item If A = {John promoted or Gurpreet promoted}, find P (A).
\end{enumerate}
\solution
%\input{exemplar/11/16/3/10/main.tex}
\item A card is drawn from a deck of 52 cards. Find the probability of getting a king or a heart or a red card.\\
\solution
%\input{exemplar/11/16/3/15/main.tex}
\item The probability that a student will pass his examination is 0.73, the probability of
the student getting a compartment is 0.13, and the probability that the student will
either pass or get compartment is 0.96. State True or False.\\
\solution
%\input{exemplar/11/16/3/31/main.tex}
\item A card is selected from a pack of 52 cards\\
\begin{enumerate}[label=(\alph*)]
\item How many points are there in the sample space?
\item Calculate the probability that the cards is an ace of spades.
\item Calculate the probability that the card is (i) an ace (ii)black card.\\
\end{enumerate}
%\input{ncert/11/16/3/4_1/Prob_4.tex}
\item In a non-leap year, the probability of having 53 tuesdays or 53 wednesdays is\\
\solution
%\input{exemplar/11/16/3/18/main.tex}
\item There are 1000 sealed envelopes in a box, 10 of them contain a cash prize of
Rs 100 each, 100 of them contain a cash prize of Rs 50 each and 200 of them
contain a cash prize of Rs 10 each and rest do not contain any cash prize. If they
are well shuffled and an envelope is picked up out, what is the probability that it
contains no cash prize?\\
\solution
%\input{exemplar/10/13/3/34/main.tex}
\item 
A die is thrown and a card is selected at random from a deck of 52 playing cards. The probability of getting an even number on the die and a spade card.\\
\solution
%\input{exemplar/12/13/3/78/main.tex}
\item
If 4-digit numbers greater than 5,000 are randomly formed from the digits 0, 1, 3, 5, and 7, what is the probability of forming a number divisible by 5 when:
\begin{enumerate}
    \item The digits are repeated?
    \item The repetition of digits is not allowed?
\end{enumerate}
\solution
%\input{ncert/11/16/4/9/main.tex}
\item Consider the probability space $\brak{\Omega, \mathcal{G}, P}$ where $\Omega = [0,2]$ and $\mathcal{G} = \cbrak{\phi, \Omega, [0,1], (1,2]}$. Let $X$ and $Y$ be two functions on $\Omega$ defined as
\begin{align*}
    X(\omega) = 
    \begin{cases}
        1 & \text{if }\omega \in [0, 1]\\
        2 & \text{if }\omega \in (1, 2]
    \end{cases}
\end{align*}
and
\begin{align*}
    Y(\omega) = 
    \begin{cases}
        2 & \text{if }\omega \in [0, 1.5]\\
        3 & \text{if }\omega \in (1.5, 2].
    \end{cases}
\end{align*}
Then which one of the following statements is true?
\begin{enumerate}
    \item [(A)] $X$ is a random variable with respect to $\mathcal{G}$, but $Y$ is not a random variable with respect to $\mathcal{G}$.
    \item [(B)] $Y$ is a random variable with respect to $\mathcal{G}$, but $X$ is not a random variable with respect to $\mathcal{G}$.
    \item [(C)] Neither $X$ nor $Y$ is a random variable with respect to $\mathcal{G}$.
    \item [(D)] Both $X$ and $Y$ are random variables with respect to $\mathcal{G}$.
\end{enumerate} \hfill (GATE ST 2023)\\
\solution
%\input{gate/ST/2023/14/main.tex}
	\item  A die is loaded in such a way that each odd number is twice as likely to occur as
each even number. Find $P(G)$, where $G$ is the event that a number greater than
3 occurs on a single roll of the die.
\\
\solution
		%\input{exemplar/11/16/3/5/main.tex}
	\item All the jacks, queens and kings are removed from a deck of 52 playing cards. The remaining cards are well shuffled and then one card is drawn at random. Giving ace a value 1 similar value for other cards, find the probability that the card has a value 
		\begin{enumerate}
			\item 7
			\item greater than 7
			\item less than 7
		\end{enumerate}
		%\input{exemplar/10/13/3/30/main.tex}
  \item A Lot consists of 48 mobile phones of which 42 are good, 3 have only minor defects and 3 have major defects.Varnika will buy a phone if it is good but the trader will only buy a mobile if it has no major defects. One phone is selected at random from the lot. What is the probability that it is
\begin{enumerate}
	\item acceptable to Varnika?
            \item acceptable to the trader?
\end{enumerate}
\solution
	%\input{exemplar/10/13/3/40/main.tex}
 \item A student says that if you throw a die, it will show up 1 or not 1. Therefore, the probability of getting 1 and the probability of getting 'not 1' each is equal to $\frac{1}{2}$. Is this correct? Give reasons.\\
 \solution
        %\input{exemplar/10/13/2/9/main.tex}
   \item Four candidates A, B, C, D have ap-
plied for the assignment to coach a school cricket
team. If A is twice as likely to be selected as B, and
B and C are given about the same chance of being
selected, while C is twice as likely to be selected
as D, what are the probabilities that
\begin{enumerate}
\item C will be selected?
\item A will not be selected?
\end{enumerate}
	%\input{exemplar/11/16/3/9/main.tex}
 \item A bag contain 24 balls of which $x$ balls are red, $2x$ are white and $3x$ are blue. A ball is selected at random, What is the probability that it is
\begin{enumerate}[label=\alph*)]
\item not red ?
\item white ?
\end{enumerate}
%\input{exemplar/10/13/3/41/main.tex}
If the letters of the word ASSASSINATION are arranged at random. Find the Probability that
\begin{enumerate}[label=(\alph*)]
\item Four $S's$ come consecutively in the word
\item Two  $I's$ and two $N's$ come together
\item All $A's$ are not coming together
\item No two $A's$ are coming together
\end{enumerate}
%\input{exemplar/11/16/3/14/main.tex}
	\item One urn contains two black balls (labelled B1 and B2) and one white ball. A
	second urn contains one black ball and two white balls (labelled W1 and W2).
	Suppose the following experiment is performed. One of the two urns is chosen
	at random. Next a ball is randomly chosen from the urn. Then a second ball is
	chosen at random from the same urn without replacing the first ball.
	
	\begin{enumerate}
	\item What is the probability that two black balls are chosen?
	
	\item What is the probability that two balls of opposite colour are chosen?
	\end{enumerate}
	\solution
	%\input{exemplar/11/16/3/12/main1.tex}
\end{enumerate}

	\item 
The number lock of a suitcase has 4 wheels each labelled with ten digits i.e. from 0 to 9.The lock opens with a sequence of four digits with no repeats.What is the probability of a person getting the right sequence to open the suitcase.
\\
\solution
		%\begin{enumerate}[label=\thesection.\arabic*,ref=\thesection.\theenumi]
	\item One card is drawn from a well-shuffled deck of 52 cards. Find the probability of getting
\begin{enumerate}
\item A king of red colour 
\item A face card 
\item A red face card
\item The jack of hearts
\item A spade
\item The queen of diamonds

\end{enumerate}
\solution
		%\input{ncert/10/15/1/14/main.tex}
	\item Five cards—the ten, jack, queen, king and ace of diamonds, are well-shuffled with their face downwards. One card is then picked up at random.
\begin{enumerate}
\item
What is the probability that the card is the queen? 
\item
If the queen is drawn and put aside, what is the probability that the second card picked up is (a) an ace? (b) a queen?\\
\end{enumerate}
\solution
		%\input{ncert/10/15/1/15/defs.tex}
	\item A bag contains $5$ red balls and some blue balls. If the probability of drawing a blue ball is double that if a red ball, determine the number of blue balls in the bag. 
		\\
\solution
		%\input{ncert/10/15/2/3/defs.tex}
	\item A card is selected from a pack of 52 cards.
 \begin{enumerate}[label=(\alph*)] 
                 \item How many points are there in the sample space?
                 \item Calculate the probability that the card is an ace of spades.
                 \item Calculate the probability that the card is (i) an ace and (ii) black card.
 \end{enumerate}
\solution
		%\input{ncert/11/16/3/4/main.tex}
\item Four cards are drawn from a well-shuffled deck of 52 cards. What is the probability of obtaining 3 diamonds and one spade.
\\
\solution
		%\input{ncert/11/16/4/2/defs.tex}
\item In a certain lottery 10,000 tickets are sold and ten equal prizes are awarded. What is the probability of not getting a prize if you buy (a) one ticket (b) two tickets (c) 10 tickets ?	
\\
\solution
		%\input{ncert/11/16/4/4/defs.tex}
		%
\item 
Out of 100 students, two sections of 40 and 60 are formed. If you and your friend are among the 100 students, what is the probability that
\begin{enumerate}
\item you both enter the same section?
\item you both enter the different sections?
\end{enumerate}
\solution
		%\input{ncert/11/16/4/5/defs.tex}
	\item 
The number lock of a suitcase has 4 wheels each labelled with ten digits i.e. from 0 to 9.The lock opens with a sequence of four digits with no repeats.What is the probability of a person getting the right sequence to open the suitcase.
\\
\solution
		%\input{ncert/11/16/4/10/defs.tex}
		%
\item 
Two cards are drawn at random and without replacement from a pack of 52 playing cards. Find the probability that both the cards are black.
\\
\solution
		%\input{ncert/12/13/2/2/defs.tex}
		\item A box of oranges is inspected by examining three randomly selected oranges drawn without replacement. If all the three oranges are good, the box is approved for sale, otherwise, it is rejected. Find the probability that a box containing 15 oranges out of which 12 are good and 3 are bad ones will be approved for sale.
		\label{ncert/12/13/2/3/defs.tex}
		\item Two balls are drawn at random with replacement from a box containing 10 black and 8 red balls. Find the probability that
		\label{ncert/12/13/2/12}
\begin{enumerate}
\item both balls are red.
\item first ball is black and second is red.
\item one of them is black and other is red.
\end{enumerate}

\item In a hostel, 60\% of the students read Hindi newspaper, 40\% read English newspaper and 20\% read both Hindi and English newspapers. A student is selected at random.
		\label{ncert/12/13/2/15}
\begin{enumerate}
\item Find the probability that she reads neither Hindi nor English newspapers.
\item If she reads Hindi newspaper, find the probability that she reads English newspaper.
\item If she reads English newspaper, find the probability that she reads Hindi newspaper.\\
\end{enumerate}
\item The probability of obtaining an even prime number on each die, when a pair of dice is rolled is 
\begin{enumerate}
    \item $0$ 
    
    \item $\frac{1}{3}$ 
    
    \item $\frac{1}{12}$ 
    
    \item $\frac{1}{36}$ 
\end{enumerate}
\solution
		%\input{ncert/12/13/2/17/defs.tex}
	\item A bag contains 4 red and 4 black balls, another bag contains 2 red and 6 black balls. One of the two bags is selected at random and a ball is drawn from the bag which is found to be red. Find the probability that the ball is drawn from the first bag.
\\
\solution
		%\input{ncert/12/13/3/2/main.tex}
  \item
  Cards with numbers 2 to 101 are placed in a box. A card is selected at random.Find the probability that the card has
\begin{enumerate}[label=(\roman*)]
	\item an even number 
	\item a square number
\end{enumerate}
\solution
%\input{exemplar/10/13/3/32/main.tex}
\item
The king, queen and jack of clubs are removed from a deck of 52 playing cards and then well shuffled. Now one card is drawn at random from the remaining cards.  Determine the probability that the card is
\begin{enumerate}[label=(\roman*)]
\item a club
\item 10 of hearts
\end{enumerate}
\solution
%\input{exemplar/10/13/3/29/main.tex}
\item A team of medical students doing their internship have to assist during surgeries
at a city hospital. The probabilities of surgeries rated as very complex, complex,
routine, simple or very simple are respectively, 0.15, 0.20, 0.31, 0.26, .08. Find
the probabilities that a particular surgery will be rated
\begin{enumerate}
	\item complex or very complex;
	\item neither very complex nor very simple;
	\item routine or complex
	\item routine or simple
\end{enumerate}
\solution
%\input{exemplar/11/16/3/8(1)/main.tex}
\item A card is selected from a pack of 52 cards.
\begin{enumerate}[label=(\alph*)]
    \item How many points are there in the sample space?
    \item Calculate the probability that the card is an ace of spades.
    \item Calculate the probability that the card is (i) an ace and (ii) black card.
\end{enumerate}
\solution
%\input{exemplar/11/16/3/4/main2.tex}
\item The probability that a non leap year selected at random will contain 53 sundays.
\\
\solution
%\input{exemplar/10/13/1/19/main.tex}
\item One of the four persons John, Rita, Aslam or Gurpreet will be promoted next
month. Consequently the sample space consists of four elementary outcomes
S = {John promoted, Rita promoted, Aslam promoted, Gurpreet promoted}
You are told that the chances of John’s promotion is same as that of Gurpreet,
Rita’s chances of promotion are twice as likely as Johns. Aslam’s chances are
four times that of John.
\begin{enumerate}
	\item Determine
	\begin{enumerate}
		\item P (John promoted)
		\item P (Rita promoted)
		\item P (Aslam promoted)
		\item P (Gurpreet promoted)
	\end{enumerate}
	\item If A = {John promoted or Gurpreet promoted}, find P (A).
\end{enumerate}
\solution
%\input{exemplar/11/16/3/10/main.tex}
\item A card is drawn from a deck of 52 cards. Find the probability of getting a king or a heart or a red card.\\
\solution
%\input{exemplar/11/16/3/15/main.tex}
\item The probability that a student will pass his examination is 0.73, the probability of
the student getting a compartment is 0.13, and the probability that the student will
either pass or get compartment is 0.96. State True or False.\\
\solution
%\input{exemplar/11/16/3/31/main.tex}
\item A card is selected from a pack of 52 cards\\
\begin{enumerate}[label=(\alph*)]
\item How many points are there in the sample space?
\item Calculate the probability that the cards is an ace of spades.
\item Calculate the probability that the card is (i) an ace (ii)black card.\\
\end{enumerate}
%\input{ncert/11/16/3/4_1/Prob_4.tex}
\item In a non-leap year, the probability of having 53 tuesdays or 53 wednesdays is\\
\solution
%\input{exemplar/11/16/3/18/main.tex}
\item There are 1000 sealed envelopes in a box, 10 of them contain a cash prize of
Rs 100 each, 100 of them contain a cash prize of Rs 50 each and 200 of them
contain a cash prize of Rs 10 each and rest do not contain any cash prize. If they
are well shuffled and an envelope is picked up out, what is the probability that it
contains no cash prize?\\
\solution
%\input{exemplar/10/13/3/34/main.tex}
\item 
A die is thrown and a card is selected at random from a deck of 52 playing cards. The probability of getting an even number on the die and a spade card.\\
\solution
%\input{exemplar/12/13/3/78/main.tex}
\item
If 4-digit numbers greater than 5,000 are randomly formed from the digits 0, 1, 3, 5, and 7, what is the probability of forming a number divisible by 5 when:
\begin{enumerate}
    \item The digits are repeated?
    \item The repetition of digits is not allowed?
\end{enumerate}
\solution
%\input{ncert/11/16/4/9/main.tex}
\item Consider the probability space $\brak{\Omega, \mathcal{G}, P}$ where $\Omega = [0,2]$ and $\mathcal{G} = \cbrak{\phi, \Omega, [0,1], (1,2]}$. Let $X$ and $Y$ be two functions on $\Omega$ defined as
\begin{align*}
    X(\omega) = 
    \begin{cases}
        1 & \text{if }\omega \in [0, 1]\\
        2 & \text{if }\omega \in (1, 2]
    \end{cases}
\end{align*}
and
\begin{align*}
    Y(\omega) = 
    \begin{cases}
        2 & \text{if }\omega \in [0, 1.5]\\
        3 & \text{if }\omega \in (1.5, 2].
    \end{cases}
\end{align*}
Then which one of the following statements is true?
\begin{enumerate}
    \item [(A)] $X$ is a random variable with respect to $\mathcal{G}$, but $Y$ is not a random variable with respect to $\mathcal{G}$.
    \item [(B)] $Y$ is a random variable with respect to $\mathcal{G}$, but $X$ is not a random variable with respect to $\mathcal{G}$.
    \item [(C)] Neither $X$ nor $Y$ is a random variable with respect to $\mathcal{G}$.
    \item [(D)] Both $X$ and $Y$ are random variables with respect to $\mathcal{G}$.
\end{enumerate} \hfill (GATE ST 2023)\\
\solution
%\input{gate/ST/2023/14/main.tex}
	\item  A die is loaded in such a way that each odd number is twice as likely to occur as
each even number. Find $P(G)$, where $G$ is the event that a number greater than
3 occurs on a single roll of the die.
\\
\solution
		%\input{exemplar/11/16/3/5/main.tex}
	\item All the jacks, queens and kings are removed from a deck of 52 playing cards. The remaining cards are well shuffled and then one card is drawn at random. Giving ace a value 1 similar value for other cards, find the probability that the card has a value 
		\begin{enumerate}
			\item 7
			\item greater than 7
			\item less than 7
		\end{enumerate}
		%\input{exemplar/10/13/3/30/main.tex}
  \item A Lot consists of 48 mobile phones of which 42 are good, 3 have only minor defects and 3 have major defects.Varnika will buy a phone if it is good but the trader will only buy a mobile if it has no major defects. One phone is selected at random from the lot. What is the probability that it is
\begin{enumerate}
	\item acceptable to Varnika?
            \item acceptable to the trader?
\end{enumerate}
\solution
	%\input{exemplar/10/13/3/40/main.tex}
 \item A student says that if you throw a die, it will show up 1 or not 1. Therefore, the probability of getting 1 and the probability of getting 'not 1' each is equal to $\frac{1}{2}$. Is this correct? Give reasons.\\
 \solution
        %\input{exemplar/10/13/2/9/main.tex}
   \item Four candidates A, B, C, D have ap-
plied for the assignment to coach a school cricket
team. If A is twice as likely to be selected as B, and
B and C are given about the same chance of being
selected, while C is twice as likely to be selected
as D, what are the probabilities that
\begin{enumerate}
\item C will be selected?
\item A will not be selected?
\end{enumerate}
	%\input{exemplar/11/16/3/9/main.tex}
 \item A bag contain 24 balls of which $x$ balls are red, $2x$ are white and $3x$ are blue. A ball is selected at random, What is the probability that it is
\begin{enumerate}[label=\alph*)]
\item not red ?
\item white ?
\end{enumerate}
%\input{exemplar/10/13/3/41/main.tex}
If the letters of the word ASSASSINATION are arranged at random. Find the Probability that
\begin{enumerate}[label=(\alph*)]
\item Four $S's$ come consecutively in the word
\item Two  $I's$ and two $N's$ come together
\item All $A's$ are not coming together
\item No two $A's$ are coming together
\end{enumerate}
%\input{exemplar/11/16/3/14/main.tex}
	\item One urn contains two black balls (labelled B1 and B2) and one white ball. A
	second urn contains one black ball and two white balls (labelled W1 and W2).
	Suppose the following experiment is performed. One of the two urns is chosen
	at random. Next a ball is randomly chosen from the urn. Then a second ball is
	chosen at random from the same urn without replacing the first ball.
	
	\begin{enumerate}
	\item What is the probability that two black balls are chosen?
	
	\item What is the probability that two balls of opposite colour are chosen?
	\end{enumerate}
	\solution
	%\input{exemplar/11/16/3/12/main1.tex}
\end{enumerate}

		%
\item 
Two cards are drawn at random and without replacement from a pack of 52 playing cards. Find the probability that both the cards are black.
\\
\solution
		%\begin{enumerate}[label=\thesection.\arabic*,ref=\thesection.\theenumi]
	\item One card is drawn from a well-shuffled deck of 52 cards. Find the probability of getting
\begin{enumerate}
\item A king of red colour 
\item A face card 
\item A red face card
\item The jack of hearts
\item A spade
\item The queen of diamonds

\end{enumerate}
\solution
		%\input{ncert/10/15/1/14/main.tex}
	\item Five cards—the ten, jack, queen, king and ace of diamonds, are well-shuffled with their face downwards. One card is then picked up at random.
\begin{enumerate}
\item
What is the probability that the card is the queen? 
\item
If the queen is drawn and put aside, what is the probability that the second card picked up is (a) an ace? (b) a queen?\\
\end{enumerate}
\solution
		%\input{ncert/10/15/1/15/defs.tex}
	\item A bag contains $5$ red balls and some blue balls. If the probability of drawing a blue ball is double that if a red ball, determine the number of blue balls in the bag. 
		\\
\solution
		%\input{ncert/10/15/2/3/defs.tex}
	\item A card is selected from a pack of 52 cards.
 \begin{enumerate}[label=(\alph*)] 
                 \item How many points are there in the sample space?
                 \item Calculate the probability that the card is an ace of spades.
                 \item Calculate the probability that the card is (i) an ace and (ii) black card.
 \end{enumerate}
\solution
		%\input{ncert/11/16/3/4/main.tex}
\item Four cards are drawn from a well-shuffled deck of 52 cards. What is the probability of obtaining 3 diamonds and one spade.
\\
\solution
		%\input{ncert/11/16/4/2/defs.tex}
\item In a certain lottery 10,000 tickets are sold and ten equal prizes are awarded. What is the probability of not getting a prize if you buy (a) one ticket (b) two tickets (c) 10 tickets ?	
\\
\solution
		%\input{ncert/11/16/4/4/defs.tex}
		%
\item 
Out of 100 students, two sections of 40 and 60 are formed. If you and your friend are among the 100 students, what is the probability that
\begin{enumerate}
\item you both enter the same section?
\item you both enter the different sections?
\end{enumerate}
\solution
		%\input{ncert/11/16/4/5/defs.tex}
	\item 
The number lock of a suitcase has 4 wheels each labelled with ten digits i.e. from 0 to 9.The lock opens with a sequence of four digits with no repeats.What is the probability of a person getting the right sequence to open the suitcase.
\\
\solution
		%\input{ncert/11/16/4/10/defs.tex}
		%
\item 
Two cards are drawn at random and without replacement from a pack of 52 playing cards. Find the probability that both the cards are black.
\\
\solution
		%\input{ncert/12/13/2/2/defs.tex}
		\item A box of oranges is inspected by examining three randomly selected oranges drawn without replacement. If all the three oranges are good, the box is approved for sale, otherwise, it is rejected. Find the probability that a box containing 15 oranges out of which 12 are good and 3 are bad ones will be approved for sale.
		\label{ncert/12/13/2/3/defs.tex}
		\item Two balls are drawn at random with replacement from a box containing 10 black and 8 red balls. Find the probability that
		\label{ncert/12/13/2/12}
\begin{enumerate}
\item both balls are red.
\item first ball is black and second is red.
\item one of them is black and other is red.
\end{enumerate}

\item In a hostel, 60\% of the students read Hindi newspaper, 40\% read English newspaper and 20\% read both Hindi and English newspapers. A student is selected at random.
		\label{ncert/12/13/2/15}
\begin{enumerate}
\item Find the probability that she reads neither Hindi nor English newspapers.
\item If she reads Hindi newspaper, find the probability that she reads English newspaper.
\item If she reads English newspaper, find the probability that she reads Hindi newspaper.\\
\end{enumerate}
\item The probability of obtaining an even prime number on each die, when a pair of dice is rolled is 
\begin{enumerate}
    \item $0$ 
    
    \item $\frac{1}{3}$ 
    
    \item $\frac{1}{12}$ 
    
    \item $\frac{1}{36}$ 
\end{enumerate}
\solution
		%\input{ncert/12/13/2/17/defs.tex}
	\item A bag contains 4 red and 4 black balls, another bag contains 2 red and 6 black balls. One of the two bags is selected at random and a ball is drawn from the bag which is found to be red. Find the probability that the ball is drawn from the first bag.
\\
\solution
		%\input{ncert/12/13/3/2/main.tex}
  \item
  Cards with numbers 2 to 101 are placed in a box. A card is selected at random.Find the probability that the card has
\begin{enumerate}[label=(\roman*)]
	\item an even number 
	\item a square number
\end{enumerate}
\solution
%\input{exemplar/10/13/3/32/main.tex}
\item
The king, queen and jack of clubs are removed from a deck of 52 playing cards and then well shuffled. Now one card is drawn at random from the remaining cards.  Determine the probability that the card is
\begin{enumerate}[label=(\roman*)]
\item a club
\item 10 of hearts
\end{enumerate}
\solution
%\input{exemplar/10/13/3/29/main.tex}
\item A team of medical students doing their internship have to assist during surgeries
at a city hospital. The probabilities of surgeries rated as very complex, complex,
routine, simple or very simple are respectively, 0.15, 0.20, 0.31, 0.26, .08. Find
the probabilities that a particular surgery will be rated
\begin{enumerate}
	\item complex or very complex;
	\item neither very complex nor very simple;
	\item routine or complex
	\item routine or simple
\end{enumerate}
\solution
%\input{exemplar/11/16/3/8(1)/main.tex}
\item A card is selected from a pack of 52 cards.
\begin{enumerate}[label=(\alph*)]
    \item How many points are there in the sample space?
    \item Calculate the probability that the card is an ace of spades.
    \item Calculate the probability that the card is (i) an ace and (ii) black card.
\end{enumerate}
\solution
%\input{exemplar/11/16/3/4/main2.tex}
\item The probability that a non leap year selected at random will contain 53 sundays.
\\
\solution
%\input{exemplar/10/13/1/19/main.tex}
\item One of the four persons John, Rita, Aslam or Gurpreet will be promoted next
month. Consequently the sample space consists of four elementary outcomes
S = {John promoted, Rita promoted, Aslam promoted, Gurpreet promoted}
You are told that the chances of John’s promotion is same as that of Gurpreet,
Rita’s chances of promotion are twice as likely as Johns. Aslam’s chances are
four times that of John.
\begin{enumerate}
	\item Determine
	\begin{enumerate}
		\item P (John promoted)
		\item P (Rita promoted)
		\item P (Aslam promoted)
		\item P (Gurpreet promoted)
	\end{enumerate}
	\item If A = {John promoted or Gurpreet promoted}, find P (A).
\end{enumerate}
\solution
%\input{exemplar/11/16/3/10/main.tex}
\item A card is drawn from a deck of 52 cards. Find the probability of getting a king or a heart or a red card.\\
\solution
%\input{exemplar/11/16/3/15/main.tex}
\item The probability that a student will pass his examination is 0.73, the probability of
the student getting a compartment is 0.13, and the probability that the student will
either pass or get compartment is 0.96. State True or False.\\
\solution
%\input{exemplar/11/16/3/31/main.tex}
\item A card is selected from a pack of 52 cards\\
\begin{enumerate}[label=(\alph*)]
\item How many points are there in the sample space?
\item Calculate the probability that the cards is an ace of spades.
\item Calculate the probability that the card is (i) an ace (ii)black card.\\
\end{enumerate}
%\input{ncert/11/16/3/4_1/Prob_4.tex}
\item In a non-leap year, the probability of having 53 tuesdays or 53 wednesdays is\\
\solution
%\input{exemplar/11/16/3/18/main.tex}
\item There are 1000 sealed envelopes in a box, 10 of them contain a cash prize of
Rs 100 each, 100 of them contain a cash prize of Rs 50 each and 200 of them
contain a cash prize of Rs 10 each and rest do not contain any cash prize. If they
are well shuffled and an envelope is picked up out, what is the probability that it
contains no cash prize?\\
\solution
%\input{exemplar/10/13/3/34/main.tex}
\item 
A die is thrown and a card is selected at random from a deck of 52 playing cards. The probability of getting an even number on the die and a spade card.\\
\solution
%\input{exemplar/12/13/3/78/main.tex}
\item
If 4-digit numbers greater than 5,000 are randomly formed from the digits 0, 1, 3, 5, and 7, what is the probability of forming a number divisible by 5 when:
\begin{enumerate}
    \item The digits are repeated?
    \item The repetition of digits is not allowed?
\end{enumerate}
\solution
%\input{ncert/11/16/4/9/main.tex}
\item Consider the probability space $\brak{\Omega, \mathcal{G}, P}$ where $\Omega = [0,2]$ and $\mathcal{G} = \cbrak{\phi, \Omega, [0,1], (1,2]}$. Let $X$ and $Y$ be two functions on $\Omega$ defined as
\begin{align*}
    X(\omega) = 
    \begin{cases}
        1 & \text{if }\omega \in [0, 1]\\
        2 & \text{if }\omega \in (1, 2]
    \end{cases}
\end{align*}
and
\begin{align*}
    Y(\omega) = 
    \begin{cases}
        2 & \text{if }\omega \in [0, 1.5]\\
        3 & \text{if }\omega \in (1.5, 2].
    \end{cases}
\end{align*}
Then which one of the following statements is true?
\begin{enumerate}
    \item [(A)] $X$ is a random variable with respect to $\mathcal{G}$, but $Y$ is not a random variable with respect to $\mathcal{G}$.
    \item [(B)] $Y$ is a random variable with respect to $\mathcal{G}$, but $X$ is not a random variable with respect to $\mathcal{G}$.
    \item [(C)] Neither $X$ nor $Y$ is a random variable with respect to $\mathcal{G}$.
    \item [(D)] Both $X$ and $Y$ are random variables with respect to $\mathcal{G}$.
\end{enumerate} \hfill (GATE ST 2023)\\
\solution
%\input{gate/ST/2023/14/main.tex}
	\item  A die is loaded in such a way that each odd number is twice as likely to occur as
each even number. Find $P(G)$, where $G$ is the event that a number greater than
3 occurs on a single roll of the die.
\\
\solution
		%\input{exemplar/11/16/3/5/main.tex}
	\item All the jacks, queens and kings are removed from a deck of 52 playing cards. The remaining cards are well shuffled and then one card is drawn at random. Giving ace a value 1 similar value for other cards, find the probability that the card has a value 
		\begin{enumerate}
			\item 7
			\item greater than 7
			\item less than 7
		\end{enumerate}
		%\input{exemplar/10/13/3/30/main.tex}
  \item A Lot consists of 48 mobile phones of which 42 are good, 3 have only minor defects and 3 have major defects.Varnika will buy a phone if it is good but the trader will only buy a mobile if it has no major defects. One phone is selected at random from the lot. What is the probability that it is
\begin{enumerate}
	\item acceptable to Varnika?
            \item acceptable to the trader?
\end{enumerate}
\solution
	%\input{exemplar/10/13/3/40/main.tex}
 \item A student says that if you throw a die, it will show up 1 or not 1. Therefore, the probability of getting 1 and the probability of getting 'not 1' each is equal to $\frac{1}{2}$. Is this correct? Give reasons.\\
 \solution
        %\input{exemplar/10/13/2/9/main.tex}
   \item Four candidates A, B, C, D have ap-
plied for the assignment to coach a school cricket
team. If A is twice as likely to be selected as B, and
B and C are given about the same chance of being
selected, while C is twice as likely to be selected
as D, what are the probabilities that
\begin{enumerate}
\item C will be selected?
\item A will not be selected?
\end{enumerate}
	%\input{exemplar/11/16/3/9/main.tex}
 \item A bag contain 24 balls of which $x$ balls are red, $2x$ are white and $3x$ are blue. A ball is selected at random, What is the probability that it is
\begin{enumerate}[label=\alph*)]
\item not red ?
\item white ?
\end{enumerate}
%\input{exemplar/10/13/3/41/main.tex}
If the letters of the word ASSASSINATION are arranged at random. Find the Probability that
\begin{enumerate}[label=(\alph*)]
\item Four $S's$ come consecutively in the word
\item Two  $I's$ and two $N's$ come together
\item All $A's$ are not coming together
\item No two $A's$ are coming together
\end{enumerate}
%\input{exemplar/11/16/3/14/main.tex}
	\item One urn contains two black balls (labelled B1 and B2) and one white ball. A
	second urn contains one black ball and two white balls (labelled W1 and W2).
	Suppose the following experiment is performed. One of the two urns is chosen
	at random. Next a ball is randomly chosen from the urn. Then a second ball is
	chosen at random from the same urn without replacing the first ball.
	
	\begin{enumerate}
	\item What is the probability that two black balls are chosen?
	
	\item What is the probability that two balls of opposite colour are chosen?
	\end{enumerate}
	\solution
	%\input{exemplar/11/16/3/12/main1.tex}
\end{enumerate}

		\item A box of oranges is inspected by examining three randomly selected oranges drawn without replacement. If all the three oranges are good, the box is approved for sale, otherwise, it is rejected. Find the probability that a box containing 15 oranges out of which 12 are good and 3 are bad ones will be approved for sale.
		\label{ncert/12/13/2/3/defs.tex}
		\item Two balls are drawn at random with replacement from a box containing 10 black and 8 red balls. Find the probability that
		\label{ncert/12/13/2/12}
\begin{enumerate}
\item both balls are red.
\item first ball is black and second is red.
\item one of them is black and other is red.
\end{enumerate}

\item In a hostel, 60\% of the students read Hindi newspaper, 40\% read English newspaper and 20\% read both Hindi and English newspapers. A student is selected at random.
		\label{ncert/12/13/2/15}
\begin{enumerate}
\item Find the probability that she reads neither Hindi nor English newspapers.
\item If she reads Hindi newspaper, find the probability that she reads English newspaper.
\item If she reads English newspaper, find the probability that she reads Hindi newspaper.\\
\end{enumerate}
\item The probability of obtaining an even prime number on each die, when a pair of dice is rolled is 
\begin{enumerate}
    \item $0$ 
    
    \item $\frac{1}{3}$ 
    
    \item $\frac{1}{12}$ 
    
    \item $\frac{1}{36}$ 
\end{enumerate}
\solution
		%\begin{enumerate}[label=\thesection.\arabic*,ref=\thesection.\theenumi]
	\item One card is drawn from a well-shuffled deck of 52 cards. Find the probability of getting
\begin{enumerate}
\item A king of red colour 
\item A face card 
\item A red face card
\item The jack of hearts
\item A spade
\item The queen of diamonds

\end{enumerate}
\solution
		%\input{ncert/10/15/1/14/main.tex}
	\item Five cards—the ten, jack, queen, king and ace of diamonds, are well-shuffled with their face downwards. One card is then picked up at random.
\begin{enumerate}
\item
What is the probability that the card is the queen? 
\item
If the queen is drawn and put aside, what is the probability that the second card picked up is (a) an ace? (b) a queen?\\
\end{enumerate}
\solution
		%\input{ncert/10/15/1/15/defs.tex}
	\item A bag contains $5$ red balls and some blue balls. If the probability of drawing a blue ball is double that if a red ball, determine the number of blue balls in the bag. 
		\\
\solution
		%\input{ncert/10/15/2/3/defs.tex}
	\item A card is selected from a pack of 52 cards.
 \begin{enumerate}[label=(\alph*)] 
                 \item How many points are there in the sample space?
                 \item Calculate the probability that the card is an ace of spades.
                 \item Calculate the probability that the card is (i) an ace and (ii) black card.
 \end{enumerate}
\solution
		%\input{ncert/11/16/3/4/main.tex}
\item Four cards are drawn from a well-shuffled deck of 52 cards. What is the probability of obtaining 3 diamonds and one spade.
\\
\solution
		%\input{ncert/11/16/4/2/defs.tex}
\item In a certain lottery 10,000 tickets are sold and ten equal prizes are awarded. What is the probability of not getting a prize if you buy (a) one ticket (b) two tickets (c) 10 tickets ?	
\\
\solution
		%\input{ncert/11/16/4/4/defs.tex}
		%
\item 
Out of 100 students, two sections of 40 and 60 are formed. If you and your friend are among the 100 students, what is the probability that
\begin{enumerate}
\item you both enter the same section?
\item you both enter the different sections?
\end{enumerate}
\solution
		%\input{ncert/11/16/4/5/defs.tex}
	\item 
The number lock of a suitcase has 4 wheels each labelled with ten digits i.e. from 0 to 9.The lock opens with a sequence of four digits with no repeats.What is the probability of a person getting the right sequence to open the suitcase.
\\
\solution
		%\input{ncert/11/16/4/10/defs.tex}
		%
\item 
Two cards are drawn at random and without replacement from a pack of 52 playing cards. Find the probability that both the cards are black.
\\
\solution
		%\input{ncert/12/13/2/2/defs.tex}
		\item A box of oranges is inspected by examining three randomly selected oranges drawn without replacement. If all the three oranges are good, the box is approved for sale, otherwise, it is rejected. Find the probability that a box containing 15 oranges out of which 12 are good and 3 are bad ones will be approved for sale.
		\label{ncert/12/13/2/3/defs.tex}
		\item Two balls are drawn at random with replacement from a box containing 10 black and 8 red balls. Find the probability that
		\label{ncert/12/13/2/12}
\begin{enumerate}
\item both balls are red.
\item first ball is black and second is red.
\item one of them is black and other is red.
\end{enumerate}

\item In a hostel, 60\% of the students read Hindi newspaper, 40\% read English newspaper and 20\% read both Hindi and English newspapers. A student is selected at random.
		\label{ncert/12/13/2/15}
\begin{enumerate}
\item Find the probability that she reads neither Hindi nor English newspapers.
\item If she reads Hindi newspaper, find the probability that she reads English newspaper.
\item If she reads English newspaper, find the probability that she reads Hindi newspaper.\\
\end{enumerate}
\item The probability of obtaining an even prime number on each die, when a pair of dice is rolled is 
\begin{enumerate}
    \item $0$ 
    
    \item $\frac{1}{3}$ 
    
    \item $\frac{1}{12}$ 
    
    \item $\frac{1}{36}$ 
\end{enumerate}
\solution
		%\input{ncert/12/13/2/17/defs.tex}
	\item A bag contains 4 red and 4 black balls, another bag contains 2 red and 6 black balls. One of the two bags is selected at random and a ball is drawn from the bag which is found to be red. Find the probability that the ball is drawn from the first bag.
\\
\solution
		%\input{ncert/12/13/3/2/main.tex}
  \item
  Cards with numbers 2 to 101 are placed in a box. A card is selected at random.Find the probability that the card has
\begin{enumerate}[label=(\roman*)]
	\item an even number 
	\item a square number
\end{enumerate}
\solution
%\input{exemplar/10/13/3/32/main.tex}
\item
The king, queen and jack of clubs are removed from a deck of 52 playing cards and then well shuffled. Now one card is drawn at random from the remaining cards.  Determine the probability that the card is
\begin{enumerate}[label=(\roman*)]
\item a club
\item 10 of hearts
\end{enumerate}
\solution
%\input{exemplar/10/13/3/29/main.tex}
\item A team of medical students doing their internship have to assist during surgeries
at a city hospital. The probabilities of surgeries rated as very complex, complex,
routine, simple or very simple are respectively, 0.15, 0.20, 0.31, 0.26, .08. Find
the probabilities that a particular surgery will be rated
\begin{enumerate}
	\item complex or very complex;
	\item neither very complex nor very simple;
	\item routine or complex
	\item routine or simple
\end{enumerate}
\solution
%\input{exemplar/11/16/3/8(1)/main.tex}
\item A card is selected from a pack of 52 cards.
\begin{enumerate}[label=(\alph*)]
    \item How many points are there in the sample space?
    \item Calculate the probability that the card is an ace of spades.
    \item Calculate the probability that the card is (i) an ace and (ii) black card.
\end{enumerate}
\solution
%\input{exemplar/11/16/3/4/main2.tex}
\item The probability that a non leap year selected at random will contain 53 sundays.
\\
\solution
%\input{exemplar/10/13/1/19/main.tex}
\item One of the four persons John, Rita, Aslam or Gurpreet will be promoted next
month. Consequently the sample space consists of four elementary outcomes
S = {John promoted, Rita promoted, Aslam promoted, Gurpreet promoted}
You are told that the chances of John’s promotion is same as that of Gurpreet,
Rita’s chances of promotion are twice as likely as Johns. Aslam’s chances are
four times that of John.
\begin{enumerate}
	\item Determine
	\begin{enumerate}
		\item P (John promoted)
		\item P (Rita promoted)
		\item P (Aslam promoted)
		\item P (Gurpreet promoted)
	\end{enumerate}
	\item If A = {John promoted or Gurpreet promoted}, find P (A).
\end{enumerate}
\solution
%\input{exemplar/11/16/3/10/main.tex}
\item A card is drawn from a deck of 52 cards. Find the probability of getting a king or a heart or a red card.\\
\solution
%\input{exemplar/11/16/3/15/main.tex}
\item The probability that a student will pass his examination is 0.73, the probability of
the student getting a compartment is 0.13, and the probability that the student will
either pass or get compartment is 0.96. State True or False.\\
\solution
%\input{exemplar/11/16/3/31/main.tex}
\item A card is selected from a pack of 52 cards\\
\begin{enumerate}[label=(\alph*)]
\item How many points are there in the sample space?
\item Calculate the probability that the cards is an ace of spades.
\item Calculate the probability that the card is (i) an ace (ii)black card.\\
\end{enumerate}
%\input{ncert/11/16/3/4_1/Prob_4.tex}
\item In a non-leap year, the probability of having 53 tuesdays or 53 wednesdays is\\
\solution
%\input{exemplar/11/16/3/18/main.tex}
\item There are 1000 sealed envelopes in a box, 10 of them contain a cash prize of
Rs 100 each, 100 of them contain a cash prize of Rs 50 each and 200 of them
contain a cash prize of Rs 10 each and rest do not contain any cash prize. If they
are well shuffled and an envelope is picked up out, what is the probability that it
contains no cash prize?\\
\solution
%\input{exemplar/10/13/3/34/main.tex}
\item 
A die is thrown and a card is selected at random from a deck of 52 playing cards. The probability of getting an even number on the die and a spade card.\\
\solution
%\input{exemplar/12/13/3/78/main.tex}
\item
If 4-digit numbers greater than 5,000 are randomly formed from the digits 0, 1, 3, 5, and 7, what is the probability of forming a number divisible by 5 when:
\begin{enumerate}
    \item The digits are repeated?
    \item The repetition of digits is not allowed?
\end{enumerate}
\solution
%\input{ncert/11/16/4/9/main.tex}
\item Consider the probability space $\brak{\Omega, \mathcal{G}, P}$ where $\Omega = [0,2]$ and $\mathcal{G} = \cbrak{\phi, \Omega, [0,1], (1,2]}$. Let $X$ and $Y$ be two functions on $\Omega$ defined as
\begin{align*}
    X(\omega) = 
    \begin{cases}
        1 & \text{if }\omega \in [0, 1]\\
        2 & \text{if }\omega \in (1, 2]
    \end{cases}
\end{align*}
and
\begin{align*}
    Y(\omega) = 
    \begin{cases}
        2 & \text{if }\omega \in [0, 1.5]\\
        3 & \text{if }\omega \in (1.5, 2].
    \end{cases}
\end{align*}
Then which one of the following statements is true?
\begin{enumerate}
    \item [(A)] $X$ is a random variable with respect to $\mathcal{G}$, but $Y$ is not a random variable with respect to $\mathcal{G}$.
    \item [(B)] $Y$ is a random variable with respect to $\mathcal{G}$, but $X$ is not a random variable with respect to $\mathcal{G}$.
    \item [(C)] Neither $X$ nor $Y$ is a random variable with respect to $\mathcal{G}$.
    \item [(D)] Both $X$ and $Y$ are random variables with respect to $\mathcal{G}$.
\end{enumerate} \hfill (GATE ST 2023)\\
\solution
%\input{gate/ST/2023/14/main.tex}
	\item  A die is loaded in such a way that each odd number is twice as likely to occur as
each even number. Find $P(G)$, where $G$ is the event that a number greater than
3 occurs on a single roll of the die.
\\
\solution
		%\input{exemplar/11/16/3/5/main.tex}
	\item All the jacks, queens and kings are removed from a deck of 52 playing cards. The remaining cards are well shuffled and then one card is drawn at random. Giving ace a value 1 similar value for other cards, find the probability that the card has a value 
		\begin{enumerate}
			\item 7
			\item greater than 7
			\item less than 7
		\end{enumerate}
		%\input{exemplar/10/13/3/30/main.tex}
  \item A Lot consists of 48 mobile phones of which 42 are good, 3 have only minor defects and 3 have major defects.Varnika will buy a phone if it is good but the trader will only buy a mobile if it has no major defects. One phone is selected at random from the lot. What is the probability that it is
\begin{enumerate}
	\item acceptable to Varnika?
            \item acceptable to the trader?
\end{enumerate}
\solution
	%\input{exemplar/10/13/3/40/main.tex}
 \item A student says that if you throw a die, it will show up 1 or not 1. Therefore, the probability of getting 1 and the probability of getting 'not 1' each is equal to $\frac{1}{2}$. Is this correct? Give reasons.\\
 \solution
        %\input{exemplar/10/13/2/9/main.tex}
   \item Four candidates A, B, C, D have ap-
plied for the assignment to coach a school cricket
team. If A is twice as likely to be selected as B, and
B and C are given about the same chance of being
selected, while C is twice as likely to be selected
as D, what are the probabilities that
\begin{enumerate}
\item C will be selected?
\item A will not be selected?
\end{enumerate}
	%\input{exemplar/11/16/3/9/main.tex}
 \item A bag contain 24 balls of which $x$ balls are red, $2x$ are white and $3x$ are blue. A ball is selected at random, What is the probability that it is
\begin{enumerate}[label=\alph*)]
\item not red ?
\item white ?
\end{enumerate}
%\input{exemplar/10/13/3/41/main.tex}
If the letters of the word ASSASSINATION are arranged at random. Find the Probability that
\begin{enumerate}[label=(\alph*)]
\item Four $S's$ come consecutively in the word
\item Two  $I's$ and two $N's$ come together
\item All $A's$ are not coming together
\item No two $A's$ are coming together
\end{enumerate}
%\input{exemplar/11/16/3/14/main.tex}
	\item One urn contains two black balls (labelled B1 and B2) and one white ball. A
	second urn contains one black ball and two white balls (labelled W1 and W2).
	Suppose the following experiment is performed. One of the two urns is chosen
	at random. Next a ball is randomly chosen from the urn. Then a second ball is
	chosen at random from the same urn without replacing the first ball.
	
	\begin{enumerate}
	\item What is the probability that two black balls are chosen?
	
	\item What is the probability that two balls of opposite colour are chosen?
	\end{enumerate}
	\solution
	%\input{exemplar/11/16/3/12/main1.tex}
\end{enumerate}

	\item A bag contains 4 red and 4 black balls, another bag contains 2 red and 6 black balls. One of the two bags is selected at random and a ball is drawn from the bag which is found to be red. Find the probability that the ball is drawn from the first bag.
\\
\solution
		%\begin{table}[H]
	\centering
\begin{tabular}{|c|c|c|}
\hline
Random variable &Value &Definition\\ \hline
\multirow{3}{*}{X} &0 &Slips of Rs 1\\
&1 &Slips of Rs 5\\
&2 &Slips of Rs 13\\ \hline
\multirow{2}{*}{Y} &0 &Box A\\
&1 &Box B\\\hline
\end{tabular}
\caption{}
\label{tab:Distribution}
\end{table}
See \tabref{tab:Distribution}.
\begin{align}
p_{Y}\brak{k}= \begin{cases} 
      \frac{1}{3} & {k=0} \\
      \frac{2}{3 }& {k=1} 
   \end{cases}
   \\
p_{Y|X}\brak{0|0} = \frac{19}{25}\, 
p_{Y|X}\brak{0|1} = \frac{6}{25}\,
p_{Y|X}\brak{1|0} = \frac{45}{50}\,
p_{Y|X}\brak{1|2} = \frac{5}{50}
\end{align}
The desired probability is the probability that a slip drawn at random is marked other than Rs 1,
\begin{align}
&=1-p_X\brak{0}\\
&= p_X(1) + p_X(2)
\end{align}
Using Bayes theorem,
\begin{align}
&= p_Y\brak{0} \times \pr{Y=0 | X=1} + p_Y\brak{1} \times \pr{Y=1|X=2}\\
&=\frac{1}{3} \times \frac{6}{25} + \frac{2}{3} \times \frac{5}{50}\\
&=\frac{11}{75}
\end{align}

\newpage

%\tableofcontents

\bigskip

\renewcommand{\thefigure}{\theenumi}
\renewcommand{\thetable}{\theenumi}
%\renewcommand{\theequation}{\theenumi}

%\begin{abstract}
%%\boldmath
%In this letter, an algorithm for evaluating the exact analytical bit error rate  (BER)  for the piecewise linear (PL) combiner for  multiple relays is presented. Previous results were available only for upto three relays. The algorithm is unique in the sense that  the actual mathematical expressions, that are prohibitively large, need not be explicitly obtained. The diversity gain due to multiple relays is shown through plots of the analytical BER, well supported by simulations. 
%
%\end{abstract}
% IEEEtran.cls defaults to using nonbold math in the Abstract.
% This preserves the distinction between vectors and scalars. However,
% if the journal you are submitting to favors bold math in the abstract,
% then you can use LaTeX's standard command \boldmath at the very start
% of the abstract to achieve this. Many IEEE journals frown on math
% in the abstract anyway.

% Note that keywords are not normally used for peerreview papers.
%\begin{IEEEkeywords}
%Cooperative diversity, decode and forward, piecewise linear
%\end{IEEEkeywords}



% For peer review papers, you can put extra information on the cover
% page as needed:
% \ifCLASSOPTIONpeerreview
% \begin{center} \bfseries EDICS Category: 3-BBND \end{center}
% \fi
%
% For peerreview papers, this IEEEtran command inserts a page break and
% creates the second title. It will be ignored for other modes.
%\IEEEpeerreviewmaketitle




  \item
  Cards with numbers 2 to 101 are placed in a box. A card is selected at random.Find the probability that the card has
\begin{enumerate}[label=(\roman*)]
	\item an even number 
	\item a square number
\end{enumerate}
\solution
%\begin{table}[H]
	\centering
\begin{tabular}{|c|c|c|}
\hline
Random variable &Value &Definition\\ \hline
\multirow{3}{*}{X} &0 &Slips of Rs 1\\
&1 &Slips of Rs 5\\
&2 &Slips of Rs 13\\ \hline
\multirow{2}{*}{Y} &0 &Box A\\
&1 &Box B\\\hline
\end{tabular}
\caption{}
\label{tab:Distribution}
\end{table}
See \tabref{tab:Distribution}.
\begin{align}
p_{Y}\brak{k}= \begin{cases} 
      \frac{1}{3} & {k=0} \\
      \frac{2}{3 }& {k=1} 
   \end{cases}
   \\
p_{Y|X}\brak{0|0} = \frac{19}{25}\, 
p_{Y|X}\brak{0|1} = \frac{6}{25}\,
p_{Y|X}\brak{1|0} = \frac{45}{50}\,
p_{Y|X}\brak{1|2} = \frac{5}{50}
\end{align}
The desired probability is the probability that a slip drawn at random is marked other than Rs 1,
\begin{align}
&=1-p_X\brak{0}\\
&= p_X(1) + p_X(2)
\end{align}
Using Bayes theorem,
\begin{align}
&= p_Y\brak{0} \times \pr{Y=0 | X=1} + p_Y\brak{1} \times \pr{Y=1|X=2}\\
&=\frac{1}{3} \times \frac{6}{25} + \frac{2}{3} \times \frac{5}{50}\\
&=\frac{11}{75}
\end{align}

\newpage

%\tableofcontents

\bigskip

\renewcommand{\thefigure}{\theenumi}
\renewcommand{\thetable}{\theenumi}
%\renewcommand{\theequation}{\theenumi}

%\begin{abstract}
%%\boldmath
%In this letter, an algorithm for evaluating the exact analytical bit error rate  (BER)  for the piecewise linear (PL) combiner for  multiple relays is presented. Previous results were available only for upto three relays. The algorithm is unique in the sense that  the actual mathematical expressions, that are prohibitively large, need not be explicitly obtained. The diversity gain due to multiple relays is shown through plots of the analytical BER, well supported by simulations. 
%
%\end{abstract}
% IEEEtran.cls defaults to using nonbold math in the Abstract.
% This preserves the distinction between vectors and scalars. However,
% if the journal you are submitting to favors bold math in the abstract,
% then you can use LaTeX's standard command \boldmath at the very start
% of the abstract to achieve this. Many IEEE journals frown on math
% in the abstract anyway.

% Note that keywords are not normally used for peerreview papers.
%\begin{IEEEkeywords}
%Cooperative diversity, decode and forward, piecewise linear
%\end{IEEEkeywords}



% For peer review papers, you can put extra information on the cover
% page as needed:
% \ifCLASSOPTIONpeerreview
% \begin{center} \bfseries EDICS Category: 3-BBND \end{center}
% \fi
%
% For peerreview papers, this IEEEtran command inserts a page break and
% creates the second title. It will be ignored for other modes.
%\IEEEpeerreviewmaketitle




\item
The king, queen and jack of clubs are removed from a deck of 52 playing cards and then well shuffled. Now one card is drawn at random from the remaining cards.  Determine the probability that the card is
\begin{enumerate}[label=(\roman*)]
\item a club
\item 10 of hearts
\end{enumerate}
\solution
%\begin{table}[H]
	\centering
\begin{tabular}{|c|c|c|}
\hline
Random variable &Value &Definition\\ \hline
\multirow{3}{*}{X} &0 &Slips of Rs 1\\
&1 &Slips of Rs 5\\
&2 &Slips of Rs 13\\ \hline
\multirow{2}{*}{Y} &0 &Box A\\
&1 &Box B\\\hline
\end{tabular}
\caption{}
\label{tab:Distribution}
\end{table}
See \tabref{tab:Distribution}.
\begin{align}
p_{Y}\brak{k}= \begin{cases} 
      \frac{1}{3} & {k=0} \\
      \frac{2}{3 }& {k=1} 
   \end{cases}
   \\
p_{Y|X}\brak{0|0} = \frac{19}{25}\, 
p_{Y|X}\brak{0|1} = \frac{6}{25}\,
p_{Y|X}\brak{1|0} = \frac{45}{50}\,
p_{Y|X}\brak{1|2} = \frac{5}{50}
\end{align}
The desired probability is the probability that a slip drawn at random is marked other than Rs 1,
\begin{align}
&=1-p_X\brak{0}\\
&= p_X(1) + p_X(2)
\end{align}
Using Bayes theorem,
\begin{align}
&= p_Y\brak{0} \times \pr{Y=0 | X=1} + p_Y\brak{1} \times \pr{Y=1|X=2}\\
&=\frac{1}{3} \times \frac{6}{25} + \frac{2}{3} \times \frac{5}{50}\\
&=\frac{11}{75}
\end{align}

\newpage

%\tableofcontents

\bigskip

\renewcommand{\thefigure}{\theenumi}
\renewcommand{\thetable}{\theenumi}
%\renewcommand{\theequation}{\theenumi}

%\begin{abstract}
%%\boldmath
%In this letter, an algorithm for evaluating the exact analytical bit error rate  (BER)  for the piecewise linear (PL) combiner for  multiple relays is presented. Previous results were available only for upto three relays. The algorithm is unique in the sense that  the actual mathematical expressions, that are prohibitively large, need not be explicitly obtained. The diversity gain due to multiple relays is shown through plots of the analytical BER, well supported by simulations. 
%
%\end{abstract}
% IEEEtran.cls defaults to using nonbold math in the Abstract.
% This preserves the distinction between vectors and scalars. However,
% if the journal you are submitting to favors bold math in the abstract,
% then you can use LaTeX's standard command \boldmath at the very start
% of the abstract to achieve this. Many IEEE journals frown on math
% in the abstract anyway.

% Note that keywords are not normally used for peerreview papers.
%\begin{IEEEkeywords}
%Cooperative diversity, decode and forward, piecewise linear
%\end{IEEEkeywords}



% For peer review papers, you can put extra information on the cover
% page as needed:
% \ifCLASSOPTIONpeerreview
% \begin{center} \bfseries EDICS Category: 3-BBND \end{center}
% \fi
%
% For peerreview papers, this IEEEtran command inserts a page break and
% creates the second title. It will be ignored for other modes.
%\IEEEpeerreviewmaketitle




\item A team of medical students doing their internship have to assist during surgeries
at a city hospital. The probabilities of surgeries rated as very complex, complex,
routine, simple or very simple are respectively, 0.15, 0.20, 0.31, 0.26, .08. Find
the probabilities that a particular surgery will be rated
\begin{enumerate}
	\item complex or very complex;
	\item neither very complex nor very simple;
	\item routine or complex
	\item routine or simple
\end{enumerate}
\solution
%\begin{table}[H]
	\centering
\begin{tabular}{|c|c|c|}
\hline
Random variable &Value &Definition\\ \hline
\multirow{3}{*}{X} &0 &Slips of Rs 1\\
&1 &Slips of Rs 5\\
&2 &Slips of Rs 13\\ \hline
\multirow{2}{*}{Y} &0 &Box A\\
&1 &Box B\\\hline
\end{tabular}
\caption{}
\label{tab:Distribution}
\end{table}
See \tabref{tab:Distribution}.
\begin{align}
p_{Y}\brak{k}= \begin{cases} 
      \frac{1}{3} & {k=0} \\
      \frac{2}{3 }& {k=1} 
   \end{cases}
   \\
p_{Y|X}\brak{0|0} = \frac{19}{25}\, 
p_{Y|X}\brak{0|1} = \frac{6}{25}\,
p_{Y|X}\brak{1|0} = \frac{45}{50}\,
p_{Y|X}\brak{1|2} = \frac{5}{50}
\end{align}
The desired probability is the probability that a slip drawn at random is marked other than Rs 1,
\begin{align}
&=1-p_X\brak{0}\\
&= p_X(1) + p_X(2)
\end{align}
Using Bayes theorem,
\begin{align}
&= p_Y\brak{0} \times \pr{Y=0 | X=1} + p_Y\brak{1} \times \pr{Y=1|X=2}\\
&=\frac{1}{3} \times \frac{6}{25} + \frac{2}{3} \times \frac{5}{50}\\
&=\frac{11}{75}
\end{align}

\newpage

%\tableofcontents

\bigskip

\renewcommand{\thefigure}{\theenumi}
\renewcommand{\thetable}{\theenumi}
%\renewcommand{\theequation}{\theenumi}

%\begin{abstract}
%%\boldmath
%In this letter, an algorithm for evaluating the exact analytical bit error rate  (BER)  for the piecewise linear (PL) combiner for  multiple relays is presented. Previous results were available only for upto three relays. The algorithm is unique in the sense that  the actual mathematical expressions, that are prohibitively large, need not be explicitly obtained. The diversity gain due to multiple relays is shown through plots of the analytical BER, well supported by simulations. 
%
%\end{abstract}
% IEEEtran.cls defaults to using nonbold math in the Abstract.
% This preserves the distinction between vectors and scalars. However,
% if the journal you are submitting to favors bold math in the abstract,
% then you can use LaTeX's standard command \boldmath at the very start
% of the abstract to achieve this. Many IEEE journals frown on math
% in the abstract anyway.

% Note that keywords are not normally used for peerreview papers.
%\begin{IEEEkeywords}
%Cooperative diversity, decode and forward, piecewise linear
%\end{IEEEkeywords}



% For peer review papers, you can put extra information on the cover
% page as needed:
% \ifCLASSOPTIONpeerreview
% \begin{center} \bfseries EDICS Category: 3-BBND \end{center}
% \fi
%
% For peerreview papers, this IEEEtran command inserts a page break and
% creates the second title. It will be ignored for other modes.
%\IEEEpeerreviewmaketitle




\item A card is selected from a pack of 52 cards.
\begin{enumerate}[label=(\alph*)]
    \item How many points are there in the sample space?
    \item Calculate the probability that the card is an ace of spades.
    \item Calculate the probability that the card is (i) an ace and (ii) black card.
\end{enumerate}
\solution
%Let $X$ be an bernoulli rv defined as in \tabref{tab:exemplar/11/16/3/26}.  Then, 
\begin{equation}
    p =
        \frac{4}{11} 
\end{equation}
\begin{table}[H]
	\centering
	\input{exemplar/11/16/3/26/tables/Table2.tex}
	\caption{}
        \label{tab:exemplar/11/16/3/26}
\end{table}

\item The probability that a non leap year selected at random will contain 53 sundays.
\\
\solution
%\begin{table}[H]
	\centering
\begin{tabular}{|c|c|c|}
\hline
Random variable &Value &Definition\\ \hline
\multirow{3}{*}{X} &0 &Slips of Rs 1\\
&1 &Slips of Rs 5\\
&2 &Slips of Rs 13\\ \hline
\multirow{2}{*}{Y} &0 &Box A\\
&1 &Box B\\\hline
\end{tabular}
\caption{}
\label{tab:Distribution}
\end{table}
See \tabref{tab:Distribution}.
\begin{align}
p_{Y}\brak{k}= \begin{cases} 
      \frac{1}{3} & {k=0} \\
      \frac{2}{3 }& {k=1} 
   \end{cases}
   \\
p_{Y|X}\brak{0|0} = \frac{19}{25}\, 
p_{Y|X}\brak{0|1} = \frac{6}{25}\,
p_{Y|X}\brak{1|0} = \frac{45}{50}\,
p_{Y|X}\brak{1|2} = \frac{5}{50}
\end{align}
The desired probability is the probability that a slip drawn at random is marked other than Rs 1,
\begin{align}
&=1-p_X\brak{0}\\
&= p_X(1) + p_X(2)
\end{align}
Using Bayes theorem,
\begin{align}
&= p_Y\brak{0} \times \pr{Y=0 | X=1} + p_Y\brak{1} \times \pr{Y=1|X=2}\\
&=\frac{1}{3} \times \frac{6}{25} + \frac{2}{3} \times \frac{5}{50}\\
&=\frac{11}{75}
\end{align}

\newpage

%\tableofcontents

\bigskip

\renewcommand{\thefigure}{\theenumi}
\renewcommand{\thetable}{\theenumi}
%\renewcommand{\theequation}{\theenumi}

%\begin{abstract}
%%\boldmath
%In this letter, an algorithm for evaluating the exact analytical bit error rate  (BER)  for the piecewise linear (PL) combiner for  multiple relays is presented. Previous results were available only for upto three relays. The algorithm is unique in the sense that  the actual mathematical expressions, that are prohibitively large, need not be explicitly obtained. The diversity gain due to multiple relays is shown through plots of the analytical BER, well supported by simulations. 
%
%\end{abstract}
% IEEEtran.cls defaults to using nonbold math in the Abstract.
% This preserves the distinction between vectors and scalars. However,
% if the journal you are submitting to favors bold math in the abstract,
% then you can use LaTeX's standard command \boldmath at the very start
% of the abstract to achieve this. Many IEEE journals frown on math
% in the abstract anyway.

% Note that keywords are not normally used for peerreview papers.
%\begin{IEEEkeywords}
%Cooperative diversity, decode and forward, piecewise linear
%\end{IEEEkeywords}



% For peer review papers, you can put extra information on the cover
% page as needed:
% \ifCLASSOPTIONpeerreview
% \begin{center} \bfseries EDICS Category: 3-BBND \end{center}
% \fi
%
% For peerreview papers, this IEEEtran command inserts a page break and
% creates the second title. It will be ignored for other modes.
%\IEEEpeerreviewmaketitle




\item One of the four persons John, Rita, Aslam or Gurpreet will be promoted next
month. Consequently the sample space consists of four elementary outcomes
S = {John promoted, Rita promoted, Aslam promoted, Gurpreet promoted}
You are told that the chances of John’s promotion is same as that of Gurpreet,
Rita’s chances of promotion are twice as likely as Johns. Aslam’s chances are
four times that of John.
\begin{enumerate}
	\item Determine
	\begin{enumerate}
		\item P (John promoted)
		\item P (Rita promoted)
		\item P (Aslam promoted)
		\item P (Gurpreet promoted)
	\end{enumerate}
	\item If A = {John promoted or Gurpreet promoted}, find P (A).
\end{enumerate}
\solution
%\begin{table}[H]
	\centering
\begin{tabular}{|c|c|c|}
\hline
Random variable &Value &Definition\\ \hline
\multirow{3}{*}{X} &0 &Slips of Rs 1\\
&1 &Slips of Rs 5\\
&2 &Slips of Rs 13\\ \hline
\multirow{2}{*}{Y} &0 &Box A\\
&1 &Box B\\\hline
\end{tabular}
\caption{}
\label{tab:Distribution}
\end{table}
See \tabref{tab:Distribution}.
\begin{align}
p_{Y}\brak{k}= \begin{cases} 
      \frac{1}{3} & {k=0} \\
      \frac{2}{3 }& {k=1} 
   \end{cases}
   \\
p_{Y|X}\brak{0|0} = \frac{19}{25}\, 
p_{Y|X}\brak{0|1} = \frac{6}{25}\,
p_{Y|X}\brak{1|0} = \frac{45}{50}\,
p_{Y|X}\brak{1|2} = \frac{5}{50}
\end{align}
The desired probability is the probability that a slip drawn at random is marked other than Rs 1,
\begin{align}
&=1-p_X\brak{0}\\
&= p_X(1) + p_X(2)
\end{align}
Using Bayes theorem,
\begin{align}
&= p_Y\brak{0} \times \pr{Y=0 | X=1} + p_Y\brak{1} \times \pr{Y=1|X=2}\\
&=\frac{1}{3} \times \frac{6}{25} + \frac{2}{3} \times \frac{5}{50}\\
&=\frac{11}{75}
\end{align}

\newpage

%\tableofcontents

\bigskip

\renewcommand{\thefigure}{\theenumi}
\renewcommand{\thetable}{\theenumi}
%\renewcommand{\theequation}{\theenumi}

%\begin{abstract}
%%\boldmath
%In this letter, an algorithm for evaluating the exact analytical bit error rate  (BER)  for the piecewise linear (PL) combiner for  multiple relays is presented. Previous results were available only for upto three relays. The algorithm is unique in the sense that  the actual mathematical expressions, that are prohibitively large, need not be explicitly obtained. The diversity gain due to multiple relays is shown through plots of the analytical BER, well supported by simulations. 
%
%\end{abstract}
% IEEEtran.cls defaults to using nonbold math in the Abstract.
% This preserves the distinction between vectors and scalars. However,
% if the journal you are submitting to favors bold math in the abstract,
% then you can use LaTeX's standard command \boldmath at the very start
% of the abstract to achieve this. Many IEEE journals frown on math
% in the abstract anyway.

% Note that keywords are not normally used for peerreview papers.
%\begin{IEEEkeywords}
%Cooperative diversity, decode and forward, piecewise linear
%\end{IEEEkeywords}



% For peer review papers, you can put extra information on the cover
% page as needed:
% \ifCLASSOPTIONpeerreview
% \begin{center} \bfseries EDICS Category: 3-BBND \end{center}
% \fi
%
% For peerreview papers, this IEEEtran command inserts a page break and
% creates the second title. It will be ignored for other modes.
%\IEEEpeerreviewmaketitle




\item A card is drawn from a deck of 52 cards. Find the probability of getting a king or a heart or a red card.\\
\solution
%\begin{table}[H]
	\centering
\begin{tabular}{|c|c|c|}
\hline
Random variable &Value &Definition\\ \hline
\multirow{3}{*}{X} &0 &Slips of Rs 1\\
&1 &Slips of Rs 5\\
&2 &Slips of Rs 13\\ \hline
\multirow{2}{*}{Y} &0 &Box A\\
&1 &Box B\\\hline
\end{tabular}
\caption{}
\label{tab:Distribution}
\end{table}
See \tabref{tab:Distribution}.
\begin{align}
p_{Y}\brak{k}= \begin{cases} 
      \frac{1}{3} & {k=0} \\
      \frac{2}{3 }& {k=1} 
   \end{cases}
   \\
p_{Y|X}\brak{0|0} = \frac{19}{25}\, 
p_{Y|X}\brak{0|1} = \frac{6}{25}\,
p_{Y|X}\brak{1|0} = \frac{45}{50}\,
p_{Y|X}\brak{1|2} = \frac{5}{50}
\end{align}
The desired probability is the probability that a slip drawn at random is marked other than Rs 1,
\begin{align}
&=1-p_X\brak{0}\\
&= p_X(1) + p_X(2)
\end{align}
Using Bayes theorem,
\begin{align}
&= p_Y\brak{0} \times \pr{Y=0 | X=1} + p_Y\brak{1} \times \pr{Y=1|X=2}\\
&=\frac{1}{3} \times \frac{6}{25} + \frac{2}{3} \times \frac{5}{50}\\
&=\frac{11}{75}
\end{align}

\newpage

%\tableofcontents

\bigskip

\renewcommand{\thefigure}{\theenumi}
\renewcommand{\thetable}{\theenumi}
%\renewcommand{\theequation}{\theenumi}

%\begin{abstract}
%%\boldmath
%In this letter, an algorithm for evaluating the exact analytical bit error rate  (BER)  for the piecewise linear (PL) combiner for  multiple relays is presented. Previous results were available only for upto three relays. The algorithm is unique in the sense that  the actual mathematical expressions, that are prohibitively large, need not be explicitly obtained. The diversity gain due to multiple relays is shown through plots of the analytical BER, well supported by simulations. 
%
%\end{abstract}
% IEEEtran.cls defaults to using nonbold math in the Abstract.
% This preserves the distinction between vectors and scalars. However,
% if the journal you are submitting to favors bold math in the abstract,
% then you can use LaTeX's standard command \boldmath at the very start
% of the abstract to achieve this. Many IEEE journals frown on math
% in the abstract anyway.

% Note that keywords are not normally used for peerreview papers.
%\begin{IEEEkeywords}
%Cooperative diversity, decode and forward, piecewise linear
%\end{IEEEkeywords}



% For peer review papers, you can put extra information on the cover
% page as needed:
% \ifCLASSOPTIONpeerreview
% \begin{center} \bfseries EDICS Category: 3-BBND \end{center}
% \fi
%
% For peerreview papers, this IEEEtran command inserts a page break and
% creates the second title. It will be ignored for other modes.
%\IEEEpeerreviewmaketitle




\item The probability that a student will pass his examination is 0.73, the probability of
the student getting a compartment is 0.13, and the probability that the student will
either pass or get compartment is 0.96. State True or False.\\
\solution
%\begin{table}[H]
	\centering
\begin{tabular}{|c|c|c|}
\hline
Random variable &Value &Definition\\ \hline
\multirow{3}{*}{X} &0 &Slips of Rs 1\\
&1 &Slips of Rs 5\\
&2 &Slips of Rs 13\\ \hline
\multirow{2}{*}{Y} &0 &Box A\\
&1 &Box B\\\hline
\end{tabular}
\caption{}
\label{tab:Distribution}
\end{table}
See \tabref{tab:Distribution}.
\begin{align}
p_{Y}\brak{k}= \begin{cases} 
      \frac{1}{3} & {k=0} \\
      \frac{2}{3 }& {k=1} 
   \end{cases}
   \\
p_{Y|X}\brak{0|0} = \frac{19}{25}\, 
p_{Y|X}\brak{0|1} = \frac{6}{25}\,
p_{Y|X}\brak{1|0} = \frac{45}{50}\,
p_{Y|X}\brak{1|2} = \frac{5}{50}
\end{align}
The desired probability is the probability that a slip drawn at random is marked other than Rs 1,
\begin{align}
&=1-p_X\brak{0}\\
&= p_X(1) + p_X(2)
\end{align}
Using Bayes theorem,
\begin{align}
&= p_Y\brak{0} \times \pr{Y=0 | X=1} + p_Y\brak{1} \times \pr{Y=1|X=2}\\
&=\frac{1}{3} \times \frac{6}{25} + \frac{2}{3} \times \frac{5}{50}\\
&=\frac{11}{75}
\end{align}

\newpage

%\tableofcontents

\bigskip

\renewcommand{\thefigure}{\theenumi}
\renewcommand{\thetable}{\theenumi}
%\renewcommand{\theequation}{\theenumi}

%\begin{abstract}
%%\boldmath
%In this letter, an algorithm for evaluating the exact analytical bit error rate  (BER)  for the piecewise linear (PL) combiner for  multiple relays is presented. Previous results were available only for upto three relays. The algorithm is unique in the sense that  the actual mathematical expressions, that are prohibitively large, need not be explicitly obtained. The diversity gain due to multiple relays is shown through plots of the analytical BER, well supported by simulations. 
%
%\end{abstract}
% IEEEtran.cls defaults to using nonbold math in the Abstract.
% This preserves the distinction between vectors and scalars. However,
% if the journal you are submitting to favors bold math in the abstract,
% then you can use LaTeX's standard command \boldmath at the very start
% of the abstract to achieve this. Many IEEE journals frown on math
% in the abstract anyway.

% Note that keywords are not normally used for peerreview papers.
%\begin{IEEEkeywords}
%Cooperative diversity, decode and forward, piecewise linear
%\end{IEEEkeywords}



% For peer review papers, you can put extra information on the cover
% page as needed:
% \ifCLASSOPTIONpeerreview
% \begin{center} \bfseries EDICS Category: 3-BBND \end{center}
% \fi
%
% For peerreview papers, this IEEEtran command inserts a page break and
% creates the second title. It will be ignored for other modes.
%\IEEEpeerreviewmaketitle




\item A card is selected from a pack of 52 cards\\
\begin{enumerate}[label=(\alph*)]
\item How many points are there in the sample space?
\item Calculate the probability that the cards is an ace of spades.
\item Calculate the probability that the card is (i) an ace (ii)black card.\\
\end{enumerate}
%\input{ncert/11/16/3/4_1/Prob_4.tex}
\item In a non-leap year, the probability of having 53 tuesdays or 53 wednesdays is\\
\solution
%A non-leap year has a total of 365 days, and a week has 7 days.\\
So it can be expressed as 
\begin{align}
365\text{days} &=52\times 7+1 \text{day}
\end{align}
$\implies$ 52 tuesdays or wednesdays\\
Random variable X denotes the days of a week
\begin{align}
p_X\brak{k}&=\frac{1}{7}; \quad \brak{1<k<7}
\end{align}
So the probability of extra day being tuesday or wednesday is
\begin{align}
p_X\brak{3}+p_X\brak{4}&=\frac{1}{7}+\frac{1}{7}=\frac{2}{7}
\end{align}



\item There are 1000 sealed envelopes in a box, 10 of them contain a cash prize of
Rs 100 each, 100 of them contain a cash prize of Rs 50 each and 200 of them
contain a cash prize of Rs 10 each and rest do not contain any cash prize. If they
are well shuffled and an envelope is picked up out, what is the probability that it
contains no cash prize?\\
\solution
%\begin{table}[H]
	\centering
\begin{tabular}{|c|c|c|}
\hline
Random variable &Value &Definition\\ \hline
\multirow{3}{*}{X} &0 &Slips of Rs 1\\
&1 &Slips of Rs 5\\
&2 &Slips of Rs 13\\ \hline
\multirow{2}{*}{Y} &0 &Box A\\
&1 &Box B\\\hline
\end{tabular}
\caption{}
\label{tab:Distribution}
\end{table}
See \tabref{tab:Distribution}.
\begin{align}
p_{Y}\brak{k}= \begin{cases} 
      \frac{1}{3} & {k=0} \\
      \frac{2}{3 }& {k=1} 
   \end{cases}
   \\
p_{Y|X}\brak{0|0} = \frac{19}{25}\, 
p_{Y|X}\brak{0|1} = \frac{6}{25}\,
p_{Y|X}\brak{1|0} = \frac{45}{50}\,
p_{Y|X}\brak{1|2} = \frac{5}{50}
\end{align}
The desired probability is the probability that a slip drawn at random is marked other than Rs 1,
\begin{align}
&=1-p_X\brak{0}\\
&= p_X(1) + p_X(2)
\end{align}
Using Bayes theorem,
\begin{align}
&= p_Y\brak{0} \times \pr{Y=0 | X=1} + p_Y\brak{1} \times \pr{Y=1|X=2}\\
&=\frac{1}{3} \times \frac{6}{25} + \frac{2}{3} \times \frac{5}{50}\\
&=\frac{11}{75}
\end{align}

\newpage

%\tableofcontents

\bigskip

\renewcommand{\thefigure}{\theenumi}
\renewcommand{\thetable}{\theenumi}
%\renewcommand{\theequation}{\theenumi}

%\begin{abstract}
%%\boldmath
%In this letter, an algorithm for evaluating the exact analytical bit error rate  (BER)  for the piecewise linear (PL) combiner for  multiple relays is presented. Previous results were available only for upto three relays. The algorithm is unique in the sense that  the actual mathematical expressions, that are prohibitively large, need not be explicitly obtained. The diversity gain due to multiple relays is shown through plots of the analytical BER, well supported by simulations. 
%
%\end{abstract}
% IEEEtran.cls defaults to using nonbold math in the Abstract.
% This preserves the distinction between vectors and scalars. However,
% if the journal you are submitting to favors bold math in the abstract,
% then you can use LaTeX's standard command \boldmath at the very start
% of the abstract to achieve this. Many IEEE journals frown on math
% in the abstract anyway.

% Note that keywords are not normally used for peerreview papers.
%\begin{IEEEkeywords}
%Cooperative diversity, decode and forward, piecewise linear
%\end{IEEEkeywords}



% For peer review papers, you can put extra information on the cover
% page as needed:
% \ifCLASSOPTIONpeerreview
% \begin{center} \bfseries EDICS Category: 3-BBND \end{center}
% \fi
%
% For peerreview papers, this IEEEtran command inserts a page break and
% creates the second title. It will be ignored for other modes.
%\IEEEpeerreviewmaketitle




\item 
A die is thrown and a card is selected at random from a deck of 52 playing cards. The probability of getting an even number on the die and a spade card.\\
\solution
%\begin{table}[H]
	\centering
\begin{tabular}{|c|c|c|}
\hline
Random variable &Value &Definition\\ \hline
\multirow{3}{*}{X} &0 &Slips of Rs 1\\
&1 &Slips of Rs 5\\
&2 &Slips of Rs 13\\ \hline
\multirow{2}{*}{Y} &0 &Box A\\
&1 &Box B\\\hline
\end{tabular}
\caption{}
\label{tab:Distribution}
\end{table}
See \tabref{tab:Distribution}.
\begin{align}
p_{Y}\brak{k}= \begin{cases} 
      \frac{1}{3} & {k=0} \\
      \frac{2}{3 }& {k=1} 
   \end{cases}
   \\
p_{Y|X}\brak{0|0} = \frac{19}{25}\, 
p_{Y|X}\brak{0|1} = \frac{6}{25}\,
p_{Y|X}\brak{1|0} = \frac{45}{50}\,
p_{Y|X}\brak{1|2} = \frac{5}{50}
\end{align}
The desired probability is the probability that a slip drawn at random is marked other than Rs 1,
\begin{align}
&=1-p_X\brak{0}\\
&= p_X(1) + p_X(2)
\end{align}
Using Bayes theorem,
\begin{align}
&= p_Y\brak{0} \times \pr{Y=0 | X=1} + p_Y\brak{1} \times \pr{Y=1|X=2}\\
&=\frac{1}{3} \times \frac{6}{25} + \frac{2}{3} \times \frac{5}{50}\\
&=\frac{11}{75}
\end{align}

\newpage

%\tableofcontents

\bigskip

\renewcommand{\thefigure}{\theenumi}
\renewcommand{\thetable}{\theenumi}
%\renewcommand{\theequation}{\theenumi}

%\begin{abstract}
%%\boldmath
%In this letter, an algorithm for evaluating the exact analytical bit error rate  (BER)  for the piecewise linear (PL) combiner for  multiple relays is presented. Previous results were available only for upto three relays. The algorithm is unique in the sense that  the actual mathematical expressions, that are prohibitively large, need not be explicitly obtained. The diversity gain due to multiple relays is shown through plots of the analytical BER, well supported by simulations. 
%
%\end{abstract}
% IEEEtran.cls defaults to using nonbold math in the Abstract.
% This preserves the distinction between vectors and scalars. However,
% if the journal you are submitting to favors bold math in the abstract,
% then you can use LaTeX's standard command \boldmath at the very start
% of the abstract to achieve this. Many IEEE journals frown on math
% in the abstract anyway.

% Note that keywords are not normally used for peerreview papers.
%\begin{IEEEkeywords}
%Cooperative diversity, decode and forward, piecewise linear
%\end{IEEEkeywords}



% For peer review papers, you can put extra information on the cover
% page as needed:
% \ifCLASSOPTIONpeerreview
% \begin{center} \bfseries EDICS Category: 3-BBND \end{center}
% \fi
%
% For peerreview papers, this IEEEtran command inserts a page break and
% creates the second title. It will be ignored for other modes.
%\IEEEpeerreviewmaketitle




\item
If 4-digit numbers greater than 5,000 are randomly formed from the digits 0, 1, 3, 5, and 7, what is the probability of forming a number divisible by 5 when:
\begin{enumerate}
    \item The digits are repeated?
    \item The repetition of digits is not allowed?
\end{enumerate}
\solution
%\begin{table}[H]
	\centering
\begin{tabular}{|c|c|c|}
\hline
Random variable &Value &Definition\\ \hline
\multirow{3}{*}{X} &0 &Slips of Rs 1\\
&1 &Slips of Rs 5\\
&2 &Slips of Rs 13\\ \hline
\multirow{2}{*}{Y} &0 &Box A\\
&1 &Box B\\\hline
\end{tabular}
\caption{}
\label{tab:Distribution}
\end{table}
See \tabref{tab:Distribution}.
\begin{align}
p_{Y}\brak{k}= \begin{cases} 
      \frac{1}{3} & {k=0} \\
      \frac{2}{3 }& {k=1} 
   \end{cases}
   \\
p_{Y|X}\brak{0|0} = \frac{19}{25}\, 
p_{Y|X}\brak{0|1} = \frac{6}{25}\,
p_{Y|X}\brak{1|0} = \frac{45}{50}\,
p_{Y|X}\brak{1|2} = \frac{5}{50}
\end{align}
The desired probability is the probability that a slip drawn at random is marked other than Rs 1,
\begin{align}
&=1-p_X\brak{0}\\
&= p_X(1) + p_X(2)
\end{align}
Using Bayes theorem,
\begin{align}
&= p_Y\brak{0} \times \pr{Y=0 | X=1} + p_Y\brak{1} \times \pr{Y=1|X=2}\\
&=\frac{1}{3} \times \frac{6}{25} + \frac{2}{3} \times \frac{5}{50}\\
&=\frac{11}{75}
\end{align}

\newpage

%\tableofcontents

\bigskip

\renewcommand{\thefigure}{\theenumi}
\renewcommand{\thetable}{\theenumi}
%\renewcommand{\theequation}{\theenumi}

%\begin{abstract}
%%\boldmath
%In this letter, an algorithm for evaluating the exact analytical bit error rate  (BER)  for the piecewise linear (PL) combiner for  multiple relays is presented. Previous results were available only for upto three relays. The algorithm is unique in the sense that  the actual mathematical expressions, that are prohibitively large, need not be explicitly obtained. The diversity gain due to multiple relays is shown through plots of the analytical BER, well supported by simulations. 
%
%\end{abstract}
% IEEEtran.cls defaults to using nonbold math in the Abstract.
% This preserves the distinction between vectors and scalars. However,
% if the journal you are submitting to favors bold math in the abstract,
% then you can use LaTeX's standard command \boldmath at the very start
% of the abstract to achieve this. Many IEEE journals frown on math
% in the abstract anyway.

% Note that keywords are not normally used for peerreview papers.
%\begin{IEEEkeywords}
%Cooperative diversity, decode and forward, piecewise linear
%\end{IEEEkeywords}



% For peer review papers, you can put extra information on the cover
% page as needed:
% \ifCLASSOPTIONpeerreview
% \begin{center} \bfseries EDICS Category: 3-BBND \end{center}
% \fi
%
% For peerreview papers, this IEEEtran command inserts a page break and
% creates the second title. It will be ignored for other modes.
%\IEEEpeerreviewmaketitle




\item Consider the probability space $\brak{\Omega, \mathcal{G}, P}$ where $\Omega = [0,2]$ and $\mathcal{G} = \cbrak{\phi, \Omega, [0,1], (1,2]}$. Let $X$ and $Y$ be two functions on $\Omega$ defined as
\begin{align*}
    X(\omega) = 
    \begin{cases}
        1 & \text{if }\omega \in [0, 1]\\
        2 & \text{if }\omega \in (1, 2]
    \end{cases}
\end{align*}
and
\begin{align*}
    Y(\omega) = 
    \begin{cases}
        2 & \text{if }\omega \in [0, 1.5]\\
        3 & \text{if }\omega \in (1.5, 2].
    \end{cases}
\end{align*}
Then which one of the following statements is true?
\begin{enumerate}
    \item [(A)] $X$ is a random variable with respect to $\mathcal{G}$, but $Y$ is not a random variable with respect to $\mathcal{G}$.
    \item [(B)] $Y$ is a random variable with respect to $\mathcal{G}$, but $X$ is not a random variable with respect to $\mathcal{G}$.
    \item [(C)] Neither $X$ nor $Y$ is a random variable with respect to $\mathcal{G}$.
    \item [(D)] Both $X$ and $Y$ are random variables with respect to $\mathcal{G}$.
\end{enumerate} \hfill (GATE ST 2023)\\
\solution
%\begin{table}[H]
	\centering
\begin{tabular}{|c|c|c|}
\hline
Random variable &Value &Definition\\ \hline
\multirow{3}{*}{X} &0 &Slips of Rs 1\\
&1 &Slips of Rs 5\\
&2 &Slips of Rs 13\\ \hline
\multirow{2}{*}{Y} &0 &Box A\\
&1 &Box B\\\hline
\end{tabular}
\caption{}
\label{tab:Distribution}
\end{table}
See \tabref{tab:Distribution}.
\begin{align}
p_{Y}\brak{k}= \begin{cases} 
      \frac{1}{3} & {k=0} \\
      \frac{2}{3 }& {k=1} 
   \end{cases}
   \\
p_{Y|X}\brak{0|0} = \frac{19}{25}\, 
p_{Y|X}\brak{0|1} = \frac{6}{25}\,
p_{Y|X}\brak{1|0} = \frac{45}{50}\,
p_{Y|X}\brak{1|2} = \frac{5}{50}
\end{align}
The desired probability is the probability that a slip drawn at random is marked other than Rs 1,
\begin{align}
&=1-p_X\brak{0}\\
&= p_X(1) + p_X(2)
\end{align}
Using Bayes theorem,
\begin{align}
&= p_Y\brak{0} \times \pr{Y=0 | X=1} + p_Y\brak{1} \times \pr{Y=1|X=2}\\
&=\frac{1}{3} \times \frac{6}{25} + \frac{2}{3} \times \frac{5}{50}\\
&=\frac{11}{75}
\end{align}

\newpage

%\tableofcontents

\bigskip

\renewcommand{\thefigure}{\theenumi}
\renewcommand{\thetable}{\theenumi}
%\renewcommand{\theequation}{\theenumi}

%\begin{abstract}
%%\boldmath
%In this letter, an algorithm for evaluating the exact analytical bit error rate  (BER)  for the piecewise linear (PL) combiner for  multiple relays is presented. Previous results were available only for upto three relays. The algorithm is unique in the sense that  the actual mathematical expressions, that are prohibitively large, need not be explicitly obtained. The diversity gain due to multiple relays is shown through plots of the analytical BER, well supported by simulations. 
%
%\end{abstract}
% IEEEtran.cls defaults to using nonbold math in the Abstract.
% This preserves the distinction between vectors and scalars. However,
% if the journal you are submitting to favors bold math in the abstract,
% then you can use LaTeX's standard command \boldmath at the very start
% of the abstract to achieve this. Many IEEE journals frown on math
% in the abstract anyway.

% Note that keywords are not normally used for peerreview papers.
%\begin{IEEEkeywords}
%Cooperative diversity, decode and forward, piecewise linear
%\end{IEEEkeywords}



% For peer review papers, you can put extra information on the cover
% page as needed:
% \ifCLASSOPTIONpeerreview
% \begin{center} \bfseries EDICS Category: 3-BBND \end{center}
% \fi
%
% For peerreview papers, this IEEEtran command inserts a page break and
% creates the second title. It will be ignored for other modes.
%\IEEEpeerreviewmaketitle




	\item  A die is loaded in such a way that each odd number is twice as likely to occur as
each even number. Find $P(G)$, where $G$ is the event that a number greater than
3 occurs on a single roll of the die.
\\
\solution
		%\begin{table}[H]
	\centering
\begin{tabular}{|c|c|c|}
\hline
Random variable &Value &Definition\\ \hline
\multirow{3}{*}{X} &0 &Slips of Rs 1\\
&1 &Slips of Rs 5\\
&2 &Slips of Rs 13\\ \hline
\multirow{2}{*}{Y} &0 &Box A\\
&1 &Box B\\\hline
\end{tabular}
\caption{}
\label{tab:Distribution}
\end{table}
See \tabref{tab:Distribution}.
\begin{align}
p_{Y}\brak{k}= \begin{cases} 
      \frac{1}{3} & {k=0} \\
      \frac{2}{3 }& {k=1} 
   \end{cases}
   \\
p_{Y|X}\brak{0|0} = \frac{19}{25}\, 
p_{Y|X}\brak{0|1} = \frac{6}{25}\,
p_{Y|X}\brak{1|0} = \frac{45}{50}\,
p_{Y|X}\brak{1|2} = \frac{5}{50}
\end{align}
The desired probability is the probability that a slip drawn at random is marked other than Rs 1,
\begin{align}
&=1-p_X\brak{0}\\
&= p_X(1) + p_X(2)
\end{align}
Using Bayes theorem,
\begin{align}
&= p_Y\brak{0} \times \pr{Y=0 | X=1} + p_Y\brak{1} \times \pr{Y=1|X=2}\\
&=\frac{1}{3} \times \frac{6}{25} + \frac{2}{3} \times \frac{5}{50}\\
&=\frac{11}{75}
\end{align}

\newpage

%\tableofcontents

\bigskip

\renewcommand{\thefigure}{\theenumi}
\renewcommand{\thetable}{\theenumi}
%\renewcommand{\theequation}{\theenumi}

%\begin{abstract}
%%\boldmath
%In this letter, an algorithm for evaluating the exact analytical bit error rate  (BER)  for the piecewise linear (PL) combiner for  multiple relays is presented. Previous results were available only for upto three relays. The algorithm is unique in the sense that  the actual mathematical expressions, that are prohibitively large, need not be explicitly obtained. The diversity gain due to multiple relays is shown through plots of the analytical BER, well supported by simulations. 
%
%\end{abstract}
% IEEEtran.cls defaults to using nonbold math in the Abstract.
% This preserves the distinction between vectors and scalars. However,
% if the journal you are submitting to favors bold math in the abstract,
% then you can use LaTeX's standard command \boldmath at the very start
% of the abstract to achieve this. Many IEEE journals frown on math
% in the abstract anyway.

% Note that keywords are not normally used for peerreview papers.
%\begin{IEEEkeywords}
%Cooperative diversity, decode and forward, piecewise linear
%\end{IEEEkeywords}



% For peer review papers, you can put extra information on the cover
% page as needed:
% \ifCLASSOPTIONpeerreview
% \begin{center} \bfseries EDICS Category: 3-BBND \end{center}
% \fi
%
% For peerreview papers, this IEEEtran command inserts a page break and
% creates the second title. It will be ignored for other modes.
%\IEEEpeerreviewmaketitle




	\item All the jacks, queens and kings are removed from a deck of 52 playing cards. The remaining cards are well shuffled and then one card is drawn at random. Giving ace a value 1 similar value for other cards, find the probability that the card has a value 
		\begin{enumerate}
			\item 7
			\item greater than 7
			\item less than 7
		\end{enumerate}
		%Number of cards left after removing all jacks, queens and kings 
\begin{align}
N	= 52 - 4\times 3
	= 40
\end{align}
%\begin{table}[H]
%\def\arraystretch{1.2}
%\begin{tabular}{|c|c|c|}
%\hline
%	\textbf{Parameter} &\textbf{Value} &\textbf{Description}\\ \hline
%	$X$ &1-10 &Represents the value of the card picked \\ \hline
%\end{tabular}
%\end{table}
Let $1 \le X \le 10$ be the value of the card picked.  Then,
\begin{align}
	p_X(k) &= \Pr(X=k)\ \forall\ 1 \leq k \leq 10\\
	&= \frac{4\times 1}{40}\\
	&= \frac{1}{10}\\
	\therefore p_X(k) &= 
	\begin{cases}
		\frac{1}{10} & 1 \leq k \leq 10\\
		0 & \text{otherwise}
	\end{cases}
\end{align}
and
\begin{align}
	F_{X}(k) &= \sum_{m=0}^{k}p_{X}(m) \quad 1 \leq k \leq 10\\
	&= \frac{k}{10}\\
	\therefore F_{X}(k) &= 
	\begin{cases}
		0 & k \leq 0\\
		\frac{k}{10} & 1\leq k \leq 10\\
		1 & k > 10 
	\end{cases}
\end{align}
\begin{enumerate}
	\item Probability that card has value equal to 7 is
		\begin{align}
			 p_{X}(7)
			= \frac{1}{10}
		\end{align}
	\item Probability that card has value greater than 7 is
		\begin{align}
			1 - F_X(7)
			&= 1 - \frac{7}{10}
			\\
			&= \frac{3}{10}
		\end{align}
	\item Probability that card has value less than 7 is
		\begin{align}
			 F_{X}(6)
			=\frac{6}{10}
		\end{align}
\end{enumerate}

  \item A Lot consists of 48 mobile phones of which 42 are good, 3 have only minor defects and 3 have major defects.Varnika will buy a phone if it is good but the trader will only buy a mobile if it has no major defects. One phone is selected at random from the lot. What is the probability that it is
\begin{enumerate}
	\item acceptable to Varnika?
            \item acceptable to the trader?
\end{enumerate}
\solution
	%\begin{table}[H]
	\centering
\begin{tabular}{|c|c|c|}
\hline
Random variable &Value &Definition\\ \hline
\multirow{3}{*}{X} &0 &Slips of Rs 1\\
&1 &Slips of Rs 5\\
&2 &Slips of Rs 13\\ \hline
\multirow{2}{*}{Y} &0 &Box A\\
&1 &Box B\\\hline
\end{tabular}
\caption{}
\label{tab:Distribution}
\end{table}
See \tabref{tab:Distribution}.
\begin{align}
p_{Y}\brak{k}= \begin{cases} 
      \frac{1}{3} & {k=0} \\
      \frac{2}{3 }& {k=1} 
   \end{cases}
   \\
p_{Y|X}\brak{0|0} = \frac{19}{25}\, 
p_{Y|X}\brak{0|1} = \frac{6}{25}\,
p_{Y|X}\brak{1|0} = \frac{45}{50}\,
p_{Y|X}\brak{1|2} = \frac{5}{50}
\end{align}
The desired probability is the probability that a slip drawn at random is marked other than Rs 1,
\begin{align}
&=1-p_X\brak{0}\\
&= p_X(1) + p_X(2)
\end{align}
Using Bayes theorem,
\begin{align}
&= p_Y\brak{0} \times \pr{Y=0 | X=1} + p_Y\brak{1} \times \pr{Y=1|X=2}\\
&=\frac{1}{3} \times \frac{6}{25} + \frac{2}{3} \times \frac{5}{50}\\
&=\frac{11}{75}
\end{align}

\newpage

%\tableofcontents

\bigskip

\renewcommand{\thefigure}{\theenumi}
\renewcommand{\thetable}{\theenumi}
%\renewcommand{\theequation}{\theenumi}

%\begin{abstract}
%%\boldmath
%In this letter, an algorithm for evaluating the exact analytical bit error rate  (BER)  for the piecewise linear (PL) combiner for  multiple relays is presented. Previous results were available only for upto three relays. The algorithm is unique in the sense that  the actual mathematical expressions, that are prohibitively large, need not be explicitly obtained. The diversity gain due to multiple relays is shown through plots of the analytical BER, well supported by simulations. 
%
%\end{abstract}
% IEEEtran.cls defaults to using nonbold math in the Abstract.
% This preserves the distinction between vectors and scalars. However,
% if the journal you are submitting to favors bold math in the abstract,
% then you can use LaTeX's standard command \boldmath at the very start
% of the abstract to achieve this. Many IEEE journals frown on math
% in the abstract anyway.

% Note that keywords are not normally used for peerreview papers.
%\begin{IEEEkeywords}
%Cooperative diversity, decode and forward, piecewise linear
%\end{IEEEkeywords}



% For peer review papers, you can put extra information on the cover
% page as needed:
% \ifCLASSOPTIONpeerreview
% \begin{center} \bfseries EDICS Category: 3-BBND \end{center}
% \fi
%
% For peerreview papers, this IEEEtran command inserts a page break and
% creates the second title. It will be ignored for other modes.
%\IEEEpeerreviewmaketitle




 \item A student says that if you throw a die, it will show up 1 or not 1. Therefore, the probability of getting 1 and the probability of getting 'not 1' each is equal to $\frac{1}{2}$. Is this correct? Give reasons.\\
 \solution
        %\begin{table}[H]
	\centering
\begin{tabular}{|c|c|c|}
\hline
Random variable &Value &Definition\\ \hline
\multirow{3}{*}{X} &0 &Slips of Rs 1\\
&1 &Slips of Rs 5\\
&2 &Slips of Rs 13\\ \hline
\multirow{2}{*}{Y} &0 &Box A\\
&1 &Box B\\\hline
\end{tabular}
\caption{}
\label{tab:Distribution}
\end{table}
See \tabref{tab:Distribution}.
\begin{align}
p_{Y}\brak{k}= \begin{cases} 
      \frac{1}{3} & {k=0} \\
      \frac{2}{3 }& {k=1} 
   \end{cases}
   \\
p_{Y|X}\brak{0|0} = \frac{19}{25}\, 
p_{Y|X}\brak{0|1} = \frac{6}{25}\,
p_{Y|X}\brak{1|0} = \frac{45}{50}\,
p_{Y|X}\brak{1|2} = \frac{5}{50}
\end{align}
The desired probability is the probability that a slip drawn at random is marked other than Rs 1,
\begin{align}
&=1-p_X\brak{0}\\
&= p_X(1) + p_X(2)
\end{align}
Using Bayes theorem,
\begin{align}
&= p_Y\brak{0} \times \pr{Y=0 | X=1} + p_Y\brak{1} \times \pr{Y=1|X=2}\\
&=\frac{1}{3} \times \frac{6}{25} + \frac{2}{3} \times \frac{5}{50}\\
&=\frac{11}{75}
\end{align}

\newpage

%\tableofcontents

\bigskip

\renewcommand{\thefigure}{\theenumi}
\renewcommand{\thetable}{\theenumi}
%\renewcommand{\theequation}{\theenumi}

%\begin{abstract}
%%\boldmath
%In this letter, an algorithm for evaluating the exact analytical bit error rate  (BER)  for the piecewise linear (PL) combiner for  multiple relays is presented. Previous results were available only for upto three relays. The algorithm is unique in the sense that  the actual mathematical expressions, that are prohibitively large, need not be explicitly obtained. The diversity gain due to multiple relays is shown through plots of the analytical BER, well supported by simulations. 
%
%\end{abstract}
% IEEEtran.cls defaults to using nonbold math in the Abstract.
% This preserves the distinction between vectors and scalars. However,
% if the journal you are submitting to favors bold math in the abstract,
% then you can use LaTeX's standard command \boldmath at the very start
% of the abstract to achieve this. Many IEEE journals frown on math
% in the abstract anyway.

% Note that keywords are not normally used for peerreview papers.
%\begin{IEEEkeywords}
%Cooperative diversity, decode and forward, piecewise linear
%\end{IEEEkeywords}



% For peer review papers, you can put extra information on the cover
% page as needed:
% \ifCLASSOPTIONpeerreview
% \begin{center} \bfseries EDICS Category: 3-BBND \end{center}
% \fi
%
% For peerreview papers, this IEEEtran command inserts a page break and
% creates the second title. It will be ignored for other modes.
%\IEEEpeerreviewmaketitle




   \item Four candidates A, B, C, D have ap-
plied for the assignment to coach a school cricket
team. If A is twice as likely to be selected as B, and
B and C are given about the same chance of being
selected, while C is twice as likely to be selected
as D, what are the probabilities that
\begin{enumerate}
\item C will be selected?
\item A will not be selected?
\end{enumerate}
	%\begin{table}[H]
	\centering
\begin{tabular}{|c|c|c|}
\hline
Random variable &Value &Definition\\ \hline
\multirow{3}{*}{X} &0 &Slips of Rs 1\\
&1 &Slips of Rs 5\\
&2 &Slips of Rs 13\\ \hline
\multirow{2}{*}{Y} &0 &Box A\\
&1 &Box B\\\hline
\end{tabular}
\caption{}
\label{tab:Distribution}
\end{table}
See \tabref{tab:Distribution}.
\begin{align}
p_{Y}\brak{k}= \begin{cases} 
      \frac{1}{3} & {k=0} \\
      \frac{2}{3 }& {k=1} 
   \end{cases}
   \\
p_{Y|X}\brak{0|0} = \frac{19}{25}\, 
p_{Y|X}\brak{0|1} = \frac{6}{25}\,
p_{Y|X}\brak{1|0} = \frac{45}{50}\,
p_{Y|X}\brak{1|2} = \frac{5}{50}
\end{align}
The desired probability is the probability that a slip drawn at random is marked other than Rs 1,
\begin{align}
&=1-p_X\brak{0}\\
&= p_X(1) + p_X(2)
\end{align}
Using Bayes theorem,
\begin{align}
&= p_Y\brak{0} \times \pr{Y=0 | X=1} + p_Y\brak{1} \times \pr{Y=1|X=2}\\
&=\frac{1}{3} \times \frac{6}{25} + \frac{2}{3} \times \frac{5}{50}\\
&=\frac{11}{75}
\end{align}

\newpage

%\tableofcontents

\bigskip

\renewcommand{\thefigure}{\theenumi}
\renewcommand{\thetable}{\theenumi}
%\renewcommand{\theequation}{\theenumi}

%\begin{abstract}
%%\boldmath
%In this letter, an algorithm for evaluating the exact analytical bit error rate  (BER)  for the piecewise linear (PL) combiner for  multiple relays is presented. Previous results were available only for upto three relays. The algorithm is unique in the sense that  the actual mathematical expressions, that are prohibitively large, need not be explicitly obtained. The diversity gain due to multiple relays is shown through plots of the analytical BER, well supported by simulations. 
%
%\end{abstract}
% IEEEtran.cls defaults to using nonbold math in the Abstract.
% This preserves the distinction between vectors and scalars. However,
% if the journal you are submitting to favors bold math in the abstract,
% then you can use LaTeX's standard command \boldmath at the very start
% of the abstract to achieve this. Many IEEE journals frown on math
% in the abstract anyway.

% Note that keywords are not normally used for peerreview papers.
%\begin{IEEEkeywords}
%Cooperative diversity, decode and forward, piecewise linear
%\end{IEEEkeywords}



% For peer review papers, you can put extra information on the cover
% page as needed:
% \ifCLASSOPTIONpeerreview
% \begin{center} \bfseries EDICS Category: 3-BBND \end{center}
% \fi
%
% For peerreview papers, this IEEEtran command inserts a page break and
% creates the second title. It will be ignored for other modes.
%\IEEEpeerreviewmaketitle




 \item A bag contain 24 balls of which $x$ balls are red, $2x$ are white and $3x$ are blue. A ball is selected at random, What is the probability that it is
\begin{enumerate}[label=\alph*)]
\item not red ?
\item white ?
\end{enumerate}
%\begin{table}[H]
	\centering
\begin{tabular}{|c|c|c|}
\hline
Random variable &Value &Definition\\ \hline
\multirow{3}{*}{X} &0 &Slips of Rs 1\\
&1 &Slips of Rs 5\\
&2 &Slips of Rs 13\\ \hline
\multirow{2}{*}{Y} &0 &Box A\\
&1 &Box B\\\hline
\end{tabular}
\caption{}
\label{tab:Distribution}
\end{table}
See \tabref{tab:Distribution}.
\begin{align}
p_{Y}\brak{k}= \begin{cases} 
      \frac{1}{3} & {k=0} \\
      \frac{2}{3 }& {k=1} 
   \end{cases}
   \\
p_{Y|X}\brak{0|0} = \frac{19}{25}\, 
p_{Y|X}\brak{0|1} = \frac{6}{25}\,
p_{Y|X}\brak{1|0} = \frac{45}{50}\,
p_{Y|X}\brak{1|2} = \frac{5}{50}
\end{align}
The desired probability is the probability that a slip drawn at random is marked other than Rs 1,
\begin{align}
&=1-p_X\brak{0}\\
&= p_X(1) + p_X(2)
\end{align}
Using Bayes theorem,
\begin{align}
&= p_Y\brak{0} \times \pr{Y=0 | X=1} + p_Y\brak{1} \times \pr{Y=1|X=2}\\
&=\frac{1}{3} \times \frac{6}{25} + \frac{2}{3} \times \frac{5}{50}\\
&=\frac{11}{75}
\end{align}

\newpage

%\tableofcontents

\bigskip

\renewcommand{\thefigure}{\theenumi}
\renewcommand{\thetable}{\theenumi}
%\renewcommand{\theequation}{\theenumi}

%\begin{abstract}
%%\boldmath
%In this letter, an algorithm for evaluating the exact analytical bit error rate  (BER)  for the piecewise linear (PL) combiner for  multiple relays is presented. Previous results were available only for upto three relays. The algorithm is unique in the sense that  the actual mathematical expressions, that are prohibitively large, need not be explicitly obtained. The diversity gain due to multiple relays is shown through plots of the analytical BER, well supported by simulations. 
%
%\end{abstract}
% IEEEtran.cls defaults to using nonbold math in the Abstract.
% This preserves the distinction between vectors and scalars. However,
% if the journal you are submitting to favors bold math in the abstract,
% then you can use LaTeX's standard command \boldmath at the very start
% of the abstract to achieve this. Many IEEE journals frown on math
% in the abstract anyway.

% Note that keywords are not normally used for peerreview papers.
%\begin{IEEEkeywords}
%Cooperative diversity, decode and forward, piecewise linear
%\end{IEEEkeywords}



% For peer review papers, you can put extra information on the cover
% page as needed:
% \ifCLASSOPTIONpeerreview
% \begin{center} \bfseries EDICS Category: 3-BBND \end{center}
% \fi
%
% For peerreview papers, this IEEEtran command inserts a page break and
% creates the second title. It will be ignored for other modes.
%\IEEEpeerreviewmaketitle




If the letters of the word ASSASSINATION are arranged at random. Find the Probability that
\begin{enumerate}[label=(\alph*)]
\item Four $S's$ come consecutively in the word
\item Two  $I's$ and two $N's$ come together
\item All $A's$ are not coming together
\item No two $A's$ are coming together
\end{enumerate}
%\begin{table}[H]
	\centering
\begin{tabular}{|c|c|c|}
\hline
Random variable &Value &Definition\\ \hline
\multirow{3}{*}{X} &0 &Slips of Rs 1\\
&1 &Slips of Rs 5\\
&2 &Slips of Rs 13\\ \hline
\multirow{2}{*}{Y} &0 &Box A\\
&1 &Box B\\\hline
\end{tabular}
\caption{}
\label{tab:Distribution}
\end{table}
See \tabref{tab:Distribution}.
\begin{align}
p_{Y}\brak{k}= \begin{cases} 
      \frac{1}{3} & {k=0} \\
      \frac{2}{3 }& {k=1} 
   \end{cases}
   \\
p_{Y|X}\brak{0|0} = \frac{19}{25}\, 
p_{Y|X}\brak{0|1} = \frac{6}{25}\,
p_{Y|X}\brak{1|0} = \frac{45}{50}\,
p_{Y|X}\brak{1|2} = \frac{5}{50}
\end{align}
The desired probability is the probability that a slip drawn at random is marked other than Rs 1,
\begin{align}
&=1-p_X\brak{0}\\
&= p_X(1) + p_X(2)
\end{align}
Using Bayes theorem,
\begin{align}
&= p_Y\brak{0} \times \pr{Y=0 | X=1} + p_Y\brak{1} \times \pr{Y=1|X=2}\\
&=\frac{1}{3} \times \frac{6}{25} + \frac{2}{3} \times \frac{5}{50}\\
&=\frac{11}{75}
\end{align}

\newpage

%\tableofcontents

\bigskip

\renewcommand{\thefigure}{\theenumi}
\renewcommand{\thetable}{\theenumi}
%\renewcommand{\theequation}{\theenumi}

%\begin{abstract}
%%\boldmath
%In this letter, an algorithm for evaluating the exact analytical bit error rate  (BER)  for the piecewise linear (PL) combiner for  multiple relays is presented. Previous results were available only for upto three relays. The algorithm is unique in the sense that  the actual mathematical expressions, that are prohibitively large, need not be explicitly obtained. The diversity gain due to multiple relays is shown through plots of the analytical BER, well supported by simulations. 
%
%\end{abstract}
% IEEEtran.cls defaults to using nonbold math in the Abstract.
% This preserves the distinction between vectors and scalars. However,
% if the journal you are submitting to favors bold math in the abstract,
% then you can use LaTeX's standard command \boldmath at the very start
% of the abstract to achieve this. Many IEEE journals frown on math
% in the abstract anyway.

% Note that keywords are not normally used for peerreview papers.
%\begin{IEEEkeywords}
%Cooperative diversity, decode and forward, piecewise linear
%\end{IEEEkeywords}



% For peer review papers, you can put extra information on the cover
% page as needed:
% \ifCLASSOPTIONpeerreview
% \begin{center} \bfseries EDICS Category: 3-BBND \end{center}
% \fi
%
% For peerreview papers, this IEEEtran command inserts a page break and
% creates the second title. It will be ignored for other modes.
%\IEEEpeerreviewmaketitle




	\item One urn contains two black balls (labelled B1 and B2) and one white ball. A
	second urn contains one black ball and two white balls (labelled W1 and W2).
	Suppose the following experiment is performed. One of the two urns is chosen
	at random. Next a ball is randomly chosen from the urn. Then a second ball is
	chosen at random from the same urn without replacing the first ball.
	
	\begin{enumerate}
	\item What is the probability that two black balls are chosen?
	
	\item What is the probability that two balls of opposite colour are chosen?
	\end{enumerate}
	\solution
	%\begin{align}
    \label{eq:12.13.6.18.1}
	\because	\pr{A|B} &> \pr{A},\
\frac{\pr{AB}}{\pr{B}} > \pr{A}
\\
    \label{eq:12.13.6.18.2}
	\implies \pr{AB} &> \pr{A}\pr{B}
	\\
	\text{or, } \frac{\pr{AB}}{\pr{A}} &=\pr{B|A} > \pr{A}
\end{align}

\end{enumerate}

	\item 
The number lock of a suitcase has 4 wheels each labelled with ten digits i.e. from 0 to 9.The lock opens with a sequence of four digits with no repeats.What is the probability of a person getting the right sequence to open the suitcase.
\\
\solution
		%\begin{enumerate}[label=\thesection.\arabic*,ref=\thesection.\theenumi]
	\item One card is drawn from a well-shuffled deck of 52 cards. Find the probability of getting
\begin{enumerate}
\item A king of red colour 
\item A face card 
\item A red face card
\item The jack of hearts
\item A spade
\item The queen of diamonds

\end{enumerate}
\solution
		%\begin{table}[H]
	\centering
\begin{tabular}{|c|c|c|}
\hline
Random variable &Value &Definition\\ \hline
\multirow{3}{*}{X} &0 &Slips of Rs 1\\
&1 &Slips of Rs 5\\
&2 &Slips of Rs 13\\ \hline
\multirow{2}{*}{Y} &0 &Box A\\
&1 &Box B\\\hline
\end{tabular}
\caption{}
\label{tab:Distribution}
\end{table}
See \tabref{tab:Distribution}.
\begin{align}
p_{Y}\brak{k}= \begin{cases} 
      \frac{1}{3} & {k=0} \\
      \frac{2}{3 }& {k=1} 
   \end{cases}
   \\
p_{Y|X}\brak{0|0} = \frac{19}{25}\, 
p_{Y|X}\brak{0|1} = \frac{6}{25}\,
p_{Y|X}\brak{1|0} = \frac{45}{50}\,
p_{Y|X}\brak{1|2} = \frac{5}{50}
\end{align}
The desired probability is the probability that a slip drawn at random is marked other than Rs 1,
\begin{align}
&=1-p_X\brak{0}\\
&= p_X(1) + p_X(2)
\end{align}
Using Bayes theorem,
\begin{align}
&= p_Y\brak{0} \times \pr{Y=0 | X=1} + p_Y\brak{1} \times \pr{Y=1|X=2}\\
&=\frac{1}{3} \times \frac{6}{25} + \frac{2}{3} \times \frac{5}{50}\\
&=\frac{11}{75}
\end{align}

\newpage

%\tableofcontents

\bigskip

\renewcommand{\thefigure}{\theenumi}
\renewcommand{\thetable}{\theenumi}
%\renewcommand{\theequation}{\theenumi}

%\begin{abstract}
%%\boldmath
%In this letter, an algorithm for evaluating the exact analytical bit error rate  (BER)  for the piecewise linear (PL) combiner for  multiple relays is presented. Previous results were available only for upto three relays. The algorithm is unique in the sense that  the actual mathematical expressions, that are prohibitively large, need not be explicitly obtained. The diversity gain due to multiple relays is shown through plots of the analytical BER, well supported by simulations. 
%
%\end{abstract}
% IEEEtran.cls defaults to using nonbold math in the Abstract.
% This preserves the distinction between vectors and scalars. However,
% if the journal you are submitting to favors bold math in the abstract,
% then you can use LaTeX's standard command \boldmath at the very start
% of the abstract to achieve this. Many IEEE journals frown on math
% in the abstract anyway.

% Note that keywords are not normally used for peerreview papers.
%\begin{IEEEkeywords}
%Cooperative diversity, decode and forward, piecewise linear
%\end{IEEEkeywords}



% For peer review papers, you can put extra information on the cover
% page as needed:
% \ifCLASSOPTIONpeerreview
% \begin{center} \bfseries EDICS Category: 3-BBND \end{center}
% \fi
%
% For peerreview papers, this IEEEtran command inserts a page break and
% creates the second title. It will be ignored for other modes.
%\IEEEpeerreviewmaketitle




	\item Five cards—the ten, jack, queen, king and ace of diamonds, are well-shuffled with their face downwards. One card is then picked up at random.
\begin{enumerate}
\item
What is the probability that the card is the queen? 
\item
If the queen is drawn and put aside, what is the probability that the second card picked up is (a) an ace? (b) a queen?\\
\end{enumerate}
\solution
		%\begin{enumerate}[label=\thesection.\arabic*,ref=\thesection.\theenumi]
	\item One card is drawn from a well-shuffled deck of 52 cards. Find the probability of getting
\begin{enumerate}
\item A king of red colour 
\item A face card 
\item A red face card
\item The jack of hearts
\item A spade
\item The queen of diamonds

\end{enumerate}
\solution
		%\input{ncert/10/15/1/14/main.tex}
	\item Five cards—the ten, jack, queen, king and ace of diamonds, are well-shuffled with their face downwards. One card is then picked up at random.
\begin{enumerate}
\item
What is the probability that the card is the queen? 
\item
If the queen is drawn and put aside, what is the probability that the second card picked up is (a) an ace? (b) a queen?\\
\end{enumerate}
\solution
		%\input{ncert/10/15/1/15/defs.tex}
	\item A bag contains $5$ red balls and some blue balls. If the probability of drawing a blue ball is double that if a red ball, determine the number of blue balls in the bag. 
		\\
\solution
		%\input{ncert/10/15/2/3/defs.tex}
	\item A card is selected from a pack of 52 cards.
 \begin{enumerate}[label=(\alph*)] 
                 \item How many points are there in the sample space?
                 \item Calculate the probability that the card is an ace of spades.
                 \item Calculate the probability that the card is (i) an ace and (ii) black card.
 \end{enumerate}
\solution
		%\input{ncert/11/16/3/4/main.tex}
\item Four cards are drawn from a well-shuffled deck of 52 cards. What is the probability of obtaining 3 diamonds and one spade.
\\
\solution
		%\input{ncert/11/16/4/2/defs.tex}
\item In a certain lottery 10,000 tickets are sold and ten equal prizes are awarded. What is the probability of not getting a prize if you buy (a) one ticket (b) two tickets (c) 10 tickets ?	
\\
\solution
		%\input{ncert/11/16/4/4/defs.tex}
		%
\item 
Out of 100 students, two sections of 40 and 60 are formed. If you and your friend are among the 100 students, what is the probability that
\begin{enumerate}
\item you both enter the same section?
\item you both enter the different sections?
\end{enumerate}
\solution
		%\input{ncert/11/16/4/5/defs.tex}
	\item 
The number lock of a suitcase has 4 wheels each labelled with ten digits i.e. from 0 to 9.The lock opens with a sequence of four digits with no repeats.What is the probability of a person getting the right sequence to open the suitcase.
\\
\solution
		%\input{ncert/11/16/4/10/defs.tex}
		%
\item 
Two cards are drawn at random and without replacement from a pack of 52 playing cards. Find the probability that both the cards are black.
\\
\solution
		%\input{ncert/12/13/2/2/defs.tex}
		\item A box of oranges is inspected by examining three randomly selected oranges drawn without replacement. If all the three oranges are good, the box is approved for sale, otherwise, it is rejected. Find the probability that a box containing 15 oranges out of which 12 are good and 3 are bad ones will be approved for sale.
		\label{ncert/12/13/2/3/defs.tex}
		\item Two balls are drawn at random with replacement from a box containing 10 black and 8 red balls. Find the probability that
		\label{ncert/12/13/2/12}
\begin{enumerate}
\item both balls are red.
\item first ball is black and second is red.
\item one of them is black and other is red.
\end{enumerate}

\item In a hostel, 60\% of the students read Hindi newspaper, 40\% read English newspaper and 20\% read both Hindi and English newspapers. A student is selected at random.
		\label{ncert/12/13/2/15}
\begin{enumerate}
\item Find the probability that she reads neither Hindi nor English newspapers.
\item If she reads Hindi newspaper, find the probability that she reads English newspaper.
\item If she reads English newspaper, find the probability that she reads Hindi newspaper.\\
\end{enumerate}
\item The probability of obtaining an even prime number on each die, when a pair of dice is rolled is 
\begin{enumerate}
    \item $0$ 
    
    \item $\frac{1}{3}$ 
    
    \item $\frac{1}{12}$ 
    
    \item $\frac{1}{36}$ 
\end{enumerate}
\solution
		%\input{ncert/12/13/2/17/defs.tex}
	\item A bag contains 4 red and 4 black balls, another bag contains 2 red and 6 black balls. One of the two bags is selected at random and a ball is drawn from the bag which is found to be red. Find the probability that the ball is drawn from the first bag.
\\
\solution
		%\input{ncert/12/13/3/2/main.tex}
  \item
  Cards with numbers 2 to 101 are placed in a box. A card is selected at random.Find the probability that the card has
\begin{enumerate}[label=(\roman*)]
	\item an even number 
	\item a square number
\end{enumerate}
\solution
%\input{exemplar/10/13/3/32/main.tex}
\item
The king, queen and jack of clubs are removed from a deck of 52 playing cards and then well shuffled. Now one card is drawn at random from the remaining cards.  Determine the probability that the card is
\begin{enumerate}[label=(\roman*)]
\item a club
\item 10 of hearts
\end{enumerate}
\solution
%\input{exemplar/10/13/3/29/main.tex}
\item A team of medical students doing their internship have to assist during surgeries
at a city hospital. The probabilities of surgeries rated as very complex, complex,
routine, simple or very simple are respectively, 0.15, 0.20, 0.31, 0.26, .08. Find
the probabilities that a particular surgery will be rated
\begin{enumerate}
	\item complex or very complex;
	\item neither very complex nor very simple;
	\item routine or complex
	\item routine or simple
\end{enumerate}
\solution
%\input{exemplar/11/16/3/8(1)/main.tex}
\item A card is selected from a pack of 52 cards.
\begin{enumerate}[label=(\alph*)]
    \item How many points are there in the sample space?
    \item Calculate the probability that the card is an ace of spades.
    \item Calculate the probability that the card is (i) an ace and (ii) black card.
\end{enumerate}
\solution
%\input{exemplar/11/16/3/4/main2.tex}
\item The probability that a non leap year selected at random will contain 53 sundays.
\\
\solution
%\input{exemplar/10/13/1/19/main.tex}
\item One of the four persons John, Rita, Aslam or Gurpreet will be promoted next
month. Consequently the sample space consists of four elementary outcomes
S = {John promoted, Rita promoted, Aslam promoted, Gurpreet promoted}
You are told that the chances of John’s promotion is same as that of Gurpreet,
Rita’s chances of promotion are twice as likely as Johns. Aslam’s chances are
four times that of John.
\begin{enumerate}
	\item Determine
	\begin{enumerate}
		\item P (John promoted)
		\item P (Rita promoted)
		\item P (Aslam promoted)
		\item P (Gurpreet promoted)
	\end{enumerate}
	\item If A = {John promoted or Gurpreet promoted}, find P (A).
\end{enumerate}
\solution
%\input{exemplar/11/16/3/10/main.tex}
\item A card is drawn from a deck of 52 cards. Find the probability of getting a king or a heart or a red card.\\
\solution
%\input{exemplar/11/16/3/15/main.tex}
\item The probability that a student will pass his examination is 0.73, the probability of
the student getting a compartment is 0.13, and the probability that the student will
either pass or get compartment is 0.96. State True or False.\\
\solution
%\input{exemplar/11/16/3/31/main.tex}
\item A card is selected from a pack of 52 cards\\
\begin{enumerate}[label=(\alph*)]
\item How many points are there in the sample space?
\item Calculate the probability that the cards is an ace of spades.
\item Calculate the probability that the card is (i) an ace (ii)black card.\\
\end{enumerate}
%\input{ncert/11/16/3/4_1/Prob_4.tex}
\item In a non-leap year, the probability of having 53 tuesdays or 53 wednesdays is\\
\solution
%\input{exemplar/11/16/3/18/main.tex}
\item There are 1000 sealed envelopes in a box, 10 of them contain a cash prize of
Rs 100 each, 100 of them contain a cash prize of Rs 50 each and 200 of them
contain a cash prize of Rs 10 each and rest do not contain any cash prize. If they
are well shuffled and an envelope is picked up out, what is the probability that it
contains no cash prize?\\
\solution
%\input{exemplar/10/13/3/34/main.tex}
\item 
A die is thrown and a card is selected at random from a deck of 52 playing cards. The probability of getting an even number on the die and a spade card.\\
\solution
%\input{exemplar/12/13/3/78/main.tex}
\item
If 4-digit numbers greater than 5,000 are randomly formed from the digits 0, 1, 3, 5, and 7, what is the probability of forming a number divisible by 5 when:
\begin{enumerate}
    \item The digits are repeated?
    \item The repetition of digits is not allowed?
\end{enumerate}
\solution
%\input{ncert/11/16/4/9/main.tex}
\item Consider the probability space $\brak{\Omega, \mathcal{G}, P}$ where $\Omega = [0,2]$ and $\mathcal{G} = \cbrak{\phi, \Omega, [0,1], (1,2]}$. Let $X$ and $Y$ be two functions on $\Omega$ defined as
\begin{align*}
    X(\omega) = 
    \begin{cases}
        1 & \text{if }\omega \in [0, 1]\\
        2 & \text{if }\omega \in (1, 2]
    \end{cases}
\end{align*}
and
\begin{align*}
    Y(\omega) = 
    \begin{cases}
        2 & \text{if }\omega \in [0, 1.5]\\
        3 & \text{if }\omega \in (1.5, 2].
    \end{cases}
\end{align*}
Then which one of the following statements is true?
\begin{enumerate}
    \item [(A)] $X$ is a random variable with respect to $\mathcal{G}$, but $Y$ is not a random variable with respect to $\mathcal{G}$.
    \item [(B)] $Y$ is a random variable with respect to $\mathcal{G}$, but $X$ is not a random variable with respect to $\mathcal{G}$.
    \item [(C)] Neither $X$ nor $Y$ is a random variable with respect to $\mathcal{G}$.
    \item [(D)] Both $X$ and $Y$ are random variables with respect to $\mathcal{G}$.
\end{enumerate} \hfill (GATE ST 2023)\\
\solution
%\input{gate/ST/2023/14/main.tex}
	\item  A die is loaded in such a way that each odd number is twice as likely to occur as
each even number. Find $P(G)$, where $G$ is the event that a number greater than
3 occurs on a single roll of the die.
\\
\solution
		%\input{exemplar/11/16/3/5/main.tex}
	\item All the jacks, queens and kings are removed from a deck of 52 playing cards. The remaining cards are well shuffled and then one card is drawn at random. Giving ace a value 1 similar value for other cards, find the probability that the card has a value 
		\begin{enumerate}
			\item 7
			\item greater than 7
			\item less than 7
		\end{enumerate}
		%\input{exemplar/10/13/3/30/main.tex}
  \item A Lot consists of 48 mobile phones of which 42 are good, 3 have only minor defects and 3 have major defects.Varnika will buy a phone if it is good but the trader will only buy a mobile if it has no major defects. One phone is selected at random from the lot. What is the probability that it is
\begin{enumerate}
	\item acceptable to Varnika?
            \item acceptable to the trader?
\end{enumerate}
\solution
	%\input{exemplar/10/13/3/40/main.tex}
 \item A student says that if you throw a die, it will show up 1 or not 1. Therefore, the probability of getting 1 and the probability of getting 'not 1' each is equal to $\frac{1}{2}$. Is this correct? Give reasons.\\
 \solution
        %\input{exemplar/10/13/2/9/main.tex}
   \item Four candidates A, B, C, D have ap-
plied for the assignment to coach a school cricket
team. If A is twice as likely to be selected as B, and
B and C are given about the same chance of being
selected, while C is twice as likely to be selected
as D, what are the probabilities that
\begin{enumerate}
\item C will be selected?
\item A will not be selected?
\end{enumerate}
	%\input{exemplar/11/16/3/9/main.tex}
 \item A bag contain 24 balls of which $x$ balls are red, $2x$ are white and $3x$ are blue. A ball is selected at random, What is the probability that it is
\begin{enumerate}[label=\alph*)]
\item not red ?
\item white ?
\end{enumerate}
%\input{exemplar/10/13/3/41/main.tex}
If the letters of the word ASSASSINATION are arranged at random. Find the Probability that
\begin{enumerate}[label=(\alph*)]
\item Four $S's$ come consecutively in the word
\item Two  $I's$ and two $N's$ come together
\item All $A's$ are not coming together
\item No two $A's$ are coming together
\end{enumerate}
%\input{exemplar/11/16/3/14/main.tex}
	\item One urn contains two black balls (labelled B1 and B2) and one white ball. A
	second urn contains one black ball and two white balls (labelled W1 and W2).
	Suppose the following experiment is performed. One of the two urns is chosen
	at random. Next a ball is randomly chosen from the urn. Then a second ball is
	chosen at random from the same urn without replacing the first ball.
	
	\begin{enumerate}
	\item What is the probability that two black balls are chosen?
	
	\item What is the probability that two balls of opposite colour are chosen?
	\end{enumerate}
	\solution
	%\input{exemplar/11/16/3/12/main1.tex}
\end{enumerate}

	\item A bag contains $5$ red balls and some blue balls. If the probability of drawing a blue ball is double that if a red ball, determine the number of blue balls in the bag. 
		\\
\solution
		%\begin{enumerate}[label=\thesection.\arabic*,ref=\thesection.\theenumi]
	\item One card is drawn from a well-shuffled deck of 52 cards. Find the probability of getting
\begin{enumerate}
\item A king of red colour 
\item A face card 
\item A red face card
\item The jack of hearts
\item A spade
\item The queen of diamonds

\end{enumerate}
\solution
		%\input{ncert/10/15/1/14/main.tex}
	\item Five cards—the ten, jack, queen, king and ace of diamonds, are well-shuffled with their face downwards. One card is then picked up at random.
\begin{enumerate}
\item
What is the probability that the card is the queen? 
\item
If the queen is drawn and put aside, what is the probability that the second card picked up is (a) an ace? (b) a queen?\\
\end{enumerate}
\solution
		%\input{ncert/10/15/1/15/defs.tex}
	\item A bag contains $5$ red balls and some blue balls. If the probability of drawing a blue ball is double that if a red ball, determine the number of blue balls in the bag. 
		\\
\solution
		%\input{ncert/10/15/2/3/defs.tex}
	\item A card is selected from a pack of 52 cards.
 \begin{enumerate}[label=(\alph*)] 
                 \item How many points are there in the sample space?
                 \item Calculate the probability that the card is an ace of spades.
                 \item Calculate the probability that the card is (i) an ace and (ii) black card.
 \end{enumerate}
\solution
		%\input{ncert/11/16/3/4/main.tex}
\item Four cards are drawn from a well-shuffled deck of 52 cards. What is the probability of obtaining 3 diamonds and one spade.
\\
\solution
		%\input{ncert/11/16/4/2/defs.tex}
\item In a certain lottery 10,000 tickets are sold and ten equal prizes are awarded. What is the probability of not getting a prize if you buy (a) one ticket (b) two tickets (c) 10 tickets ?	
\\
\solution
		%\input{ncert/11/16/4/4/defs.tex}
		%
\item 
Out of 100 students, two sections of 40 and 60 are formed. If you and your friend are among the 100 students, what is the probability that
\begin{enumerate}
\item you both enter the same section?
\item you both enter the different sections?
\end{enumerate}
\solution
		%\input{ncert/11/16/4/5/defs.tex}
	\item 
The number lock of a suitcase has 4 wheels each labelled with ten digits i.e. from 0 to 9.The lock opens with a sequence of four digits with no repeats.What is the probability of a person getting the right sequence to open the suitcase.
\\
\solution
		%\input{ncert/11/16/4/10/defs.tex}
		%
\item 
Two cards are drawn at random and without replacement from a pack of 52 playing cards. Find the probability that both the cards are black.
\\
\solution
		%\input{ncert/12/13/2/2/defs.tex}
		\item A box of oranges is inspected by examining three randomly selected oranges drawn without replacement. If all the three oranges are good, the box is approved for sale, otherwise, it is rejected. Find the probability that a box containing 15 oranges out of which 12 are good and 3 are bad ones will be approved for sale.
		\label{ncert/12/13/2/3/defs.tex}
		\item Two balls are drawn at random with replacement from a box containing 10 black and 8 red balls. Find the probability that
		\label{ncert/12/13/2/12}
\begin{enumerate}
\item both balls are red.
\item first ball is black and second is red.
\item one of them is black and other is red.
\end{enumerate}

\item In a hostel, 60\% of the students read Hindi newspaper, 40\% read English newspaper and 20\% read both Hindi and English newspapers. A student is selected at random.
		\label{ncert/12/13/2/15}
\begin{enumerate}
\item Find the probability that she reads neither Hindi nor English newspapers.
\item If she reads Hindi newspaper, find the probability that she reads English newspaper.
\item If she reads English newspaper, find the probability that she reads Hindi newspaper.\\
\end{enumerate}
\item The probability of obtaining an even prime number on each die, when a pair of dice is rolled is 
\begin{enumerate}
    \item $0$ 
    
    \item $\frac{1}{3}$ 
    
    \item $\frac{1}{12}$ 
    
    \item $\frac{1}{36}$ 
\end{enumerate}
\solution
		%\input{ncert/12/13/2/17/defs.tex}
	\item A bag contains 4 red and 4 black balls, another bag contains 2 red and 6 black balls. One of the two bags is selected at random and a ball is drawn from the bag which is found to be red. Find the probability that the ball is drawn from the first bag.
\\
\solution
		%\input{ncert/12/13/3/2/main.tex}
  \item
  Cards with numbers 2 to 101 are placed in a box. A card is selected at random.Find the probability that the card has
\begin{enumerate}[label=(\roman*)]
	\item an even number 
	\item a square number
\end{enumerate}
\solution
%\input{exemplar/10/13/3/32/main.tex}
\item
The king, queen and jack of clubs are removed from a deck of 52 playing cards and then well shuffled. Now one card is drawn at random from the remaining cards.  Determine the probability that the card is
\begin{enumerate}[label=(\roman*)]
\item a club
\item 10 of hearts
\end{enumerate}
\solution
%\input{exemplar/10/13/3/29/main.tex}
\item A team of medical students doing their internship have to assist during surgeries
at a city hospital. The probabilities of surgeries rated as very complex, complex,
routine, simple or very simple are respectively, 0.15, 0.20, 0.31, 0.26, .08. Find
the probabilities that a particular surgery will be rated
\begin{enumerate}
	\item complex or very complex;
	\item neither very complex nor very simple;
	\item routine or complex
	\item routine or simple
\end{enumerate}
\solution
%\input{exemplar/11/16/3/8(1)/main.tex}
\item A card is selected from a pack of 52 cards.
\begin{enumerate}[label=(\alph*)]
    \item How many points are there in the sample space?
    \item Calculate the probability that the card is an ace of spades.
    \item Calculate the probability that the card is (i) an ace and (ii) black card.
\end{enumerate}
\solution
%\input{exemplar/11/16/3/4/main2.tex}
\item The probability that a non leap year selected at random will contain 53 sundays.
\\
\solution
%\input{exemplar/10/13/1/19/main.tex}
\item One of the four persons John, Rita, Aslam or Gurpreet will be promoted next
month. Consequently the sample space consists of four elementary outcomes
S = {John promoted, Rita promoted, Aslam promoted, Gurpreet promoted}
You are told that the chances of John’s promotion is same as that of Gurpreet,
Rita’s chances of promotion are twice as likely as Johns. Aslam’s chances are
four times that of John.
\begin{enumerate}
	\item Determine
	\begin{enumerate}
		\item P (John promoted)
		\item P (Rita promoted)
		\item P (Aslam promoted)
		\item P (Gurpreet promoted)
	\end{enumerate}
	\item If A = {John promoted or Gurpreet promoted}, find P (A).
\end{enumerate}
\solution
%\input{exemplar/11/16/3/10/main.tex}
\item A card is drawn from a deck of 52 cards. Find the probability of getting a king or a heart or a red card.\\
\solution
%\input{exemplar/11/16/3/15/main.tex}
\item The probability that a student will pass his examination is 0.73, the probability of
the student getting a compartment is 0.13, and the probability that the student will
either pass or get compartment is 0.96. State True or False.\\
\solution
%\input{exemplar/11/16/3/31/main.tex}
\item A card is selected from a pack of 52 cards\\
\begin{enumerate}[label=(\alph*)]
\item How many points are there in the sample space?
\item Calculate the probability that the cards is an ace of spades.
\item Calculate the probability that the card is (i) an ace (ii)black card.\\
\end{enumerate}
%\input{ncert/11/16/3/4_1/Prob_4.tex}
\item In a non-leap year, the probability of having 53 tuesdays or 53 wednesdays is\\
\solution
%\input{exemplar/11/16/3/18/main.tex}
\item There are 1000 sealed envelopes in a box, 10 of them contain a cash prize of
Rs 100 each, 100 of them contain a cash prize of Rs 50 each and 200 of them
contain a cash prize of Rs 10 each and rest do not contain any cash prize. If they
are well shuffled and an envelope is picked up out, what is the probability that it
contains no cash prize?\\
\solution
%\input{exemplar/10/13/3/34/main.tex}
\item 
A die is thrown and a card is selected at random from a deck of 52 playing cards. The probability of getting an even number on the die and a spade card.\\
\solution
%\input{exemplar/12/13/3/78/main.tex}
\item
If 4-digit numbers greater than 5,000 are randomly formed from the digits 0, 1, 3, 5, and 7, what is the probability of forming a number divisible by 5 when:
\begin{enumerate}
    \item The digits are repeated?
    \item The repetition of digits is not allowed?
\end{enumerate}
\solution
%\input{ncert/11/16/4/9/main.tex}
\item Consider the probability space $\brak{\Omega, \mathcal{G}, P}$ where $\Omega = [0,2]$ and $\mathcal{G} = \cbrak{\phi, \Omega, [0,1], (1,2]}$. Let $X$ and $Y$ be two functions on $\Omega$ defined as
\begin{align*}
    X(\omega) = 
    \begin{cases}
        1 & \text{if }\omega \in [0, 1]\\
        2 & \text{if }\omega \in (1, 2]
    \end{cases}
\end{align*}
and
\begin{align*}
    Y(\omega) = 
    \begin{cases}
        2 & \text{if }\omega \in [0, 1.5]\\
        3 & \text{if }\omega \in (1.5, 2].
    \end{cases}
\end{align*}
Then which one of the following statements is true?
\begin{enumerate}
    \item [(A)] $X$ is a random variable with respect to $\mathcal{G}$, but $Y$ is not a random variable with respect to $\mathcal{G}$.
    \item [(B)] $Y$ is a random variable with respect to $\mathcal{G}$, but $X$ is not a random variable with respect to $\mathcal{G}$.
    \item [(C)] Neither $X$ nor $Y$ is a random variable with respect to $\mathcal{G}$.
    \item [(D)] Both $X$ and $Y$ are random variables with respect to $\mathcal{G}$.
\end{enumerate} \hfill (GATE ST 2023)\\
\solution
%\input{gate/ST/2023/14/main.tex}
	\item  A die is loaded in such a way that each odd number is twice as likely to occur as
each even number. Find $P(G)$, where $G$ is the event that a number greater than
3 occurs on a single roll of the die.
\\
\solution
		%\input{exemplar/11/16/3/5/main.tex}
	\item All the jacks, queens and kings are removed from a deck of 52 playing cards. The remaining cards are well shuffled and then one card is drawn at random. Giving ace a value 1 similar value for other cards, find the probability that the card has a value 
		\begin{enumerate}
			\item 7
			\item greater than 7
			\item less than 7
		\end{enumerate}
		%\input{exemplar/10/13/3/30/main.tex}
  \item A Lot consists of 48 mobile phones of which 42 are good, 3 have only minor defects and 3 have major defects.Varnika will buy a phone if it is good but the trader will only buy a mobile if it has no major defects. One phone is selected at random from the lot. What is the probability that it is
\begin{enumerate}
	\item acceptable to Varnika?
            \item acceptable to the trader?
\end{enumerate}
\solution
	%\input{exemplar/10/13/3/40/main.tex}
 \item A student says that if you throw a die, it will show up 1 or not 1. Therefore, the probability of getting 1 and the probability of getting 'not 1' each is equal to $\frac{1}{2}$. Is this correct? Give reasons.\\
 \solution
        %\input{exemplar/10/13/2/9/main.tex}
   \item Four candidates A, B, C, D have ap-
plied for the assignment to coach a school cricket
team. If A is twice as likely to be selected as B, and
B and C are given about the same chance of being
selected, while C is twice as likely to be selected
as D, what are the probabilities that
\begin{enumerate}
\item C will be selected?
\item A will not be selected?
\end{enumerate}
	%\input{exemplar/11/16/3/9/main.tex}
 \item A bag contain 24 balls of which $x$ balls are red, $2x$ are white and $3x$ are blue. A ball is selected at random, What is the probability that it is
\begin{enumerate}[label=\alph*)]
\item not red ?
\item white ?
\end{enumerate}
%\input{exemplar/10/13/3/41/main.tex}
If the letters of the word ASSASSINATION are arranged at random. Find the Probability that
\begin{enumerate}[label=(\alph*)]
\item Four $S's$ come consecutively in the word
\item Two  $I's$ and two $N's$ come together
\item All $A's$ are not coming together
\item No two $A's$ are coming together
\end{enumerate}
%\input{exemplar/11/16/3/14/main.tex}
	\item One urn contains two black balls (labelled B1 and B2) and one white ball. A
	second urn contains one black ball and two white balls (labelled W1 and W2).
	Suppose the following experiment is performed. One of the two urns is chosen
	at random. Next a ball is randomly chosen from the urn. Then a second ball is
	chosen at random from the same urn without replacing the first ball.
	
	\begin{enumerate}
	\item What is the probability that two black balls are chosen?
	
	\item What is the probability that two balls of opposite colour are chosen?
	\end{enumerate}
	\solution
	%\input{exemplar/11/16/3/12/main1.tex}
\end{enumerate}

	\item A card is selected from a pack of 52 cards.
 \begin{enumerate}[label=(\alph*)] 
                 \item How many points are there in the sample space?
                 \item Calculate the probability that the card is an ace of spades.
                 \item Calculate the probability that the card is (i) an ace and (ii) black card.
 \end{enumerate}
\solution
		%\begin{table}[H]
	\centering
\begin{tabular}{|c|c|c|}
\hline
Random variable &Value &Definition\\ \hline
\multirow{3}{*}{X} &0 &Slips of Rs 1\\
&1 &Slips of Rs 5\\
&2 &Slips of Rs 13\\ \hline
\multirow{2}{*}{Y} &0 &Box A\\
&1 &Box B\\\hline
\end{tabular}
\caption{}
\label{tab:Distribution}
\end{table}
See \tabref{tab:Distribution}.
\begin{align}
p_{Y}\brak{k}= \begin{cases} 
      \frac{1}{3} & {k=0} \\
      \frac{2}{3 }& {k=1} 
   \end{cases}
   \\
p_{Y|X}\brak{0|0} = \frac{19}{25}\, 
p_{Y|X}\brak{0|1} = \frac{6}{25}\,
p_{Y|X}\brak{1|0} = \frac{45}{50}\,
p_{Y|X}\brak{1|2} = \frac{5}{50}
\end{align}
The desired probability is the probability that a slip drawn at random is marked other than Rs 1,
\begin{align}
&=1-p_X\brak{0}\\
&= p_X(1) + p_X(2)
\end{align}
Using Bayes theorem,
\begin{align}
&= p_Y\brak{0} \times \pr{Y=0 | X=1} + p_Y\brak{1} \times \pr{Y=1|X=2}\\
&=\frac{1}{3} \times \frac{6}{25} + \frac{2}{3} \times \frac{5}{50}\\
&=\frac{11}{75}
\end{align}

\newpage

%\tableofcontents

\bigskip

\renewcommand{\thefigure}{\theenumi}
\renewcommand{\thetable}{\theenumi}
%\renewcommand{\theequation}{\theenumi}

%\begin{abstract}
%%\boldmath
%In this letter, an algorithm for evaluating the exact analytical bit error rate  (BER)  for the piecewise linear (PL) combiner for  multiple relays is presented. Previous results were available only for upto three relays. The algorithm is unique in the sense that  the actual mathematical expressions, that are prohibitively large, need not be explicitly obtained. The diversity gain due to multiple relays is shown through plots of the analytical BER, well supported by simulations. 
%
%\end{abstract}
% IEEEtran.cls defaults to using nonbold math in the Abstract.
% This preserves the distinction between vectors and scalars. However,
% if the journal you are submitting to favors bold math in the abstract,
% then you can use LaTeX's standard command \boldmath at the very start
% of the abstract to achieve this. Many IEEE journals frown on math
% in the abstract anyway.

% Note that keywords are not normally used for peerreview papers.
%\begin{IEEEkeywords}
%Cooperative diversity, decode and forward, piecewise linear
%\end{IEEEkeywords}



% For peer review papers, you can put extra information on the cover
% page as needed:
% \ifCLASSOPTIONpeerreview
% \begin{center} \bfseries EDICS Category: 3-BBND \end{center}
% \fi
%
% For peerreview papers, this IEEEtran command inserts a page break and
% creates the second title. It will be ignored for other modes.
%\IEEEpeerreviewmaketitle




\item Four cards are drawn from a well-shuffled deck of 52 cards. What is the probability of obtaining 3 diamonds and one spade.
\\
\solution
		%\begin{enumerate}[label=\thesection.\arabic*,ref=\thesection.\theenumi]
	\item One card is drawn from a well-shuffled deck of 52 cards. Find the probability of getting
\begin{enumerate}
\item A king of red colour 
\item A face card 
\item A red face card
\item The jack of hearts
\item A spade
\item The queen of diamonds

\end{enumerate}
\solution
		%\input{ncert/10/15/1/14/main.tex}
	\item Five cards—the ten, jack, queen, king and ace of diamonds, are well-shuffled with their face downwards. One card is then picked up at random.
\begin{enumerate}
\item
What is the probability that the card is the queen? 
\item
If the queen is drawn and put aside, what is the probability that the second card picked up is (a) an ace? (b) a queen?\\
\end{enumerate}
\solution
		%\input{ncert/10/15/1/15/defs.tex}
	\item A bag contains $5$ red balls and some blue balls. If the probability of drawing a blue ball is double that if a red ball, determine the number of blue balls in the bag. 
		\\
\solution
		%\input{ncert/10/15/2/3/defs.tex}
	\item A card is selected from a pack of 52 cards.
 \begin{enumerate}[label=(\alph*)] 
                 \item How many points are there in the sample space?
                 \item Calculate the probability that the card is an ace of spades.
                 \item Calculate the probability that the card is (i) an ace and (ii) black card.
 \end{enumerate}
\solution
		%\input{ncert/11/16/3/4/main.tex}
\item Four cards are drawn from a well-shuffled deck of 52 cards. What is the probability of obtaining 3 diamonds and one spade.
\\
\solution
		%\input{ncert/11/16/4/2/defs.tex}
\item In a certain lottery 10,000 tickets are sold and ten equal prizes are awarded. What is the probability of not getting a prize if you buy (a) one ticket (b) two tickets (c) 10 tickets ?	
\\
\solution
		%\input{ncert/11/16/4/4/defs.tex}
		%
\item 
Out of 100 students, two sections of 40 and 60 are formed. If you and your friend are among the 100 students, what is the probability that
\begin{enumerate}
\item you both enter the same section?
\item you both enter the different sections?
\end{enumerate}
\solution
		%\input{ncert/11/16/4/5/defs.tex}
	\item 
The number lock of a suitcase has 4 wheels each labelled with ten digits i.e. from 0 to 9.The lock opens with a sequence of four digits with no repeats.What is the probability of a person getting the right sequence to open the suitcase.
\\
\solution
		%\input{ncert/11/16/4/10/defs.tex}
		%
\item 
Two cards are drawn at random and without replacement from a pack of 52 playing cards. Find the probability that both the cards are black.
\\
\solution
		%\input{ncert/12/13/2/2/defs.tex}
		\item A box of oranges is inspected by examining three randomly selected oranges drawn without replacement. If all the three oranges are good, the box is approved for sale, otherwise, it is rejected. Find the probability that a box containing 15 oranges out of which 12 are good and 3 are bad ones will be approved for sale.
		\label{ncert/12/13/2/3/defs.tex}
		\item Two balls are drawn at random with replacement from a box containing 10 black and 8 red balls. Find the probability that
		\label{ncert/12/13/2/12}
\begin{enumerate}
\item both balls are red.
\item first ball is black and second is red.
\item one of them is black and other is red.
\end{enumerate}

\item In a hostel, 60\% of the students read Hindi newspaper, 40\% read English newspaper and 20\% read both Hindi and English newspapers. A student is selected at random.
		\label{ncert/12/13/2/15}
\begin{enumerate}
\item Find the probability that she reads neither Hindi nor English newspapers.
\item If she reads Hindi newspaper, find the probability that she reads English newspaper.
\item If she reads English newspaper, find the probability that she reads Hindi newspaper.\\
\end{enumerate}
\item The probability of obtaining an even prime number on each die, when a pair of dice is rolled is 
\begin{enumerate}
    \item $0$ 
    
    \item $\frac{1}{3}$ 
    
    \item $\frac{1}{12}$ 
    
    \item $\frac{1}{36}$ 
\end{enumerate}
\solution
		%\input{ncert/12/13/2/17/defs.tex}
	\item A bag contains 4 red and 4 black balls, another bag contains 2 red and 6 black balls. One of the two bags is selected at random and a ball is drawn from the bag which is found to be red. Find the probability that the ball is drawn from the first bag.
\\
\solution
		%\input{ncert/12/13/3/2/main.tex}
  \item
  Cards with numbers 2 to 101 are placed in a box. A card is selected at random.Find the probability that the card has
\begin{enumerate}[label=(\roman*)]
	\item an even number 
	\item a square number
\end{enumerate}
\solution
%\input{exemplar/10/13/3/32/main.tex}
\item
The king, queen and jack of clubs are removed from a deck of 52 playing cards and then well shuffled. Now one card is drawn at random from the remaining cards.  Determine the probability that the card is
\begin{enumerate}[label=(\roman*)]
\item a club
\item 10 of hearts
\end{enumerate}
\solution
%\input{exemplar/10/13/3/29/main.tex}
\item A team of medical students doing their internship have to assist during surgeries
at a city hospital. The probabilities of surgeries rated as very complex, complex,
routine, simple or very simple are respectively, 0.15, 0.20, 0.31, 0.26, .08. Find
the probabilities that a particular surgery will be rated
\begin{enumerate}
	\item complex or very complex;
	\item neither very complex nor very simple;
	\item routine or complex
	\item routine or simple
\end{enumerate}
\solution
%\input{exemplar/11/16/3/8(1)/main.tex}
\item A card is selected from a pack of 52 cards.
\begin{enumerate}[label=(\alph*)]
    \item How many points are there in the sample space?
    \item Calculate the probability that the card is an ace of spades.
    \item Calculate the probability that the card is (i) an ace and (ii) black card.
\end{enumerate}
\solution
%\input{exemplar/11/16/3/4/main2.tex}
\item The probability that a non leap year selected at random will contain 53 sundays.
\\
\solution
%\input{exemplar/10/13/1/19/main.tex}
\item One of the four persons John, Rita, Aslam or Gurpreet will be promoted next
month. Consequently the sample space consists of four elementary outcomes
S = {John promoted, Rita promoted, Aslam promoted, Gurpreet promoted}
You are told that the chances of John’s promotion is same as that of Gurpreet,
Rita’s chances of promotion are twice as likely as Johns. Aslam’s chances are
four times that of John.
\begin{enumerate}
	\item Determine
	\begin{enumerate}
		\item P (John promoted)
		\item P (Rita promoted)
		\item P (Aslam promoted)
		\item P (Gurpreet promoted)
	\end{enumerate}
	\item If A = {John promoted or Gurpreet promoted}, find P (A).
\end{enumerate}
\solution
%\input{exemplar/11/16/3/10/main.tex}
\item A card is drawn from a deck of 52 cards. Find the probability of getting a king or a heart or a red card.\\
\solution
%\input{exemplar/11/16/3/15/main.tex}
\item The probability that a student will pass his examination is 0.73, the probability of
the student getting a compartment is 0.13, and the probability that the student will
either pass or get compartment is 0.96. State True or False.\\
\solution
%\input{exemplar/11/16/3/31/main.tex}
\item A card is selected from a pack of 52 cards\\
\begin{enumerate}[label=(\alph*)]
\item How many points are there in the sample space?
\item Calculate the probability that the cards is an ace of spades.
\item Calculate the probability that the card is (i) an ace (ii)black card.\\
\end{enumerate}
%\input{ncert/11/16/3/4_1/Prob_4.tex}
\item In a non-leap year, the probability of having 53 tuesdays or 53 wednesdays is\\
\solution
%\input{exemplar/11/16/3/18/main.tex}
\item There are 1000 sealed envelopes in a box, 10 of them contain a cash prize of
Rs 100 each, 100 of them contain a cash prize of Rs 50 each and 200 of them
contain a cash prize of Rs 10 each and rest do not contain any cash prize. If they
are well shuffled and an envelope is picked up out, what is the probability that it
contains no cash prize?\\
\solution
%\input{exemplar/10/13/3/34/main.tex}
\item 
A die is thrown and a card is selected at random from a deck of 52 playing cards. The probability of getting an even number on the die and a spade card.\\
\solution
%\input{exemplar/12/13/3/78/main.tex}
\item
If 4-digit numbers greater than 5,000 are randomly formed from the digits 0, 1, 3, 5, and 7, what is the probability of forming a number divisible by 5 when:
\begin{enumerate}
    \item The digits are repeated?
    \item The repetition of digits is not allowed?
\end{enumerate}
\solution
%\input{ncert/11/16/4/9/main.tex}
\item Consider the probability space $\brak{\Omega, \mathcal{G}, P}$ where $\Omega = [0,2]$ and $\mathcal{G} = \cbrak{\phi, \Omega, [0,1], (1,2]}$. Let $X$ and $Y$ be two functions on $\Omega$ defined as
\begin{align*}
    X(\omega) = 
    \begin{cases}
        1 & \text{if }\omega \in [0, 1]\\
        2 & \text{if }\omega \in (1, 2]
    \end{cases}
\end{align*}
and
\begin{align*}
    Y(\omega) = 
    \begin{cases}
        2 & \text{if }\omega \in [0, 1.5]\\
        3 & \text{if }\omega \in (1.5, 2].
    \end{cases}
\end{align*}
Then which one of the following statements is true?
\begin{enumerate}
    \item [(A)] $X$ is a random variable with respect to $\mathcal{G}$, but $Y$ is not a random variable with respect to $\mathcal{G}$.
    \item [(B)] $Y$ is a random variable with respect to $\mathcal{G}$, but $X$ is not a random variable with respect to $\mathcal{G}$.
    \item [(C)] Neither $X$ nor $Y$ is a random variable with respect to $\mathcal{G}$.
    \item [(D)] Both $X$ and $Y$ are random variables with respect to $\mathcal{G}$.
\end{enumerate} \hfill (GATE ST 2023)\\
\solution
%\input{gate/ST/2023/14/main.tex}
	\item  A die is loaded in such a way that each odd number is twice as likely to occur as
each even number. Find $P(G)$, where $G$ is the event that a number greater than
3 occurs on a single roll of the die.
\\
\solution
		%\input{exemplar/11/16/3/5/main.tex}
	\item All the jacks, queens and kings are removed from a deck of 52 playing cards. The remaining cards are well shuffled and then one card is drawn at random. Giving ace a value 1 similar value for other cards, find the probability that the card has a value 
		\begin{enumerate}
			\item 7
			\item greater than 7
			\item less than 7
		\end{enumerate}
		%\input{exemplar/10/13/3/30/main.tex}
  \item A Lot consists of 48 mobile phones of which 42 are good, 3 have only minor defects and 3 have major defects.Varnika will buy a phone if it is good but the trader will only buy a mobile if it has no major defects. One phone is selected at random from the lot. What is the probability that it is
\begin{enumerate}
	\item acceptable to Varnika?
            \item acceptable to the trader?
\end{enumerate}
\solution
	%\input{exemplar/10/13/3/40/main.tex}
 \item A student says that if you throw a die, it will show up 1 or not 1. Therefore, the probability of getting 1 and the probability of getting 'not 1' each is equal to $\frac{1}{2}$. Is this correct? Give reasons.\\
 \solution
        %\input{exemplar/10/13/2/9/main.tex}
   \item Four candidates A, B, C, D have ap-
plied for the assignment to coach a school cricket
team. If A is twice as likely to be selected as B, and
B and C are given about the same chance of being
selected, while C is twice as likely to be selected
as D, what are the probabilities that
\begin{enumerate}
\item C will be selected?
\item A will not be selected?
\end{enumerate}
	%\input{exemplar/11/16/3/9/main.tex}
 \item A bag contain 24 balls of which $x$ balls are red, $2x$ are white and $3x$ are blue. A ball is selected at random, What is the probability that it is
\begin{enumerate}[label=\alph*)]
\item not red ?
\item white ?
\end{enumerate}
%\input{exemplar/10/13/3/41/main.tex}
If the letters of the word ASSASSINATION are arranged at random. Find the Probability that
\begin{enumerate}[label=(\alph*)]
\item Four $S's$ come consecutively in the word
\item Two  $I's$ and two $N's$ come together
\item All $A's$ are not coming together
\item No two $A's$ are coming together
\end{enumerate}
%\input{exemplar/11/16/3/14/main.tex}
	\item One urn contains two black balls (labelled B1 and B2) and one white ball. A
	second urn contains one black ball and two white balls (labelled W1 and W2).
	Suppose the following experiment is performed. One of the two urns is chosen
	at random. Next a ball is randomly chosen from the urn. Then a second ball is
	chosen at random from the same urn without replacing the first ball.
	
	\begin{enumerate}
	\item What is the probability that two black balls are chosen?
	
	\item What is the probability that two balls of opposite colour are chosen?
	\end{enumerate}
	\solution
	%\input{exemplar/11/16/3/12/main1.tex}
\end{enumerate}

\item In a certain lottery 10,000 tickets are sold and ten equal prizes are awarded. What is the probability of not getting a prize if you buy (a) one ticket (b) two tickets (c) 10 tickets ?	
\\
\solution
		%\begin{enumerate}[label=\thesection.\arabic*,ref=\thesection.\theenumi]
	\item One card is drawn from a well-shuffled deck of 52 cards. Find the probability of getting
\begin{enumerate}
\item A king of red colour 
\item A face card 
\item A red face card
\item The jack of hearts
\item A spade
\item The queen of diamonds

\end{enumerate}
\solution
		%\input{ncert/10/15/1/14/main.tex}
	\item Five cards—the ten, jack, queen, king and ace of diamonds, are well-shuffled with their face downwards. One card is then picked up at random.
\begin{enumerate}
\item
What is the probability that the card is the queen? 
\item
If the queen is drawn and put aside, what is the probability that the second card picked up is (a) an ace? (b) a queen?\\
\end{enumerate}
\solution
		%\input{ncert/10/15/1/15/defs.tex}
	\item A bag contains $5$ red balls and some blue balls. If the probability of drawing a blue ball is double that if a red ball, determine the number of blue balls in the bag. 
		\\
\solution
		%\input{ncert/10/15/2/3/defs.tex}
	\item A card is selected from a pack of 52 cards.
 \begin{enumerate}[label=(\alph*)] 
                 \item How many points are there in the sample space?
                 \item Calculate the probability that the card is an ace of spades.
                 \item Calculate the probability that the card is (i) an ace and (ii) black card.
 \end{enumerate}
\solution
		%\input{ncert/11/16/3/4/main.tex}
\item Four cards are drawn from a well-shuffled deck of 52 cards. What is the probability of obtaining 3 diamonds and one spade.
\\
\solution
		%\input{ncert/11/16/4/2/defs.tex}
\item In a certain lottery 10,000 tickets are sold and ten equal prizes are awarded. What is the probability of not getting a prize if you buy (a) one ticket (b) two tickets (c) 10 tickets ?	
\\
\solution
		%\input{ncert/11/16/4/4/defs.tex}
		%
\item 
Out of 100 students, two sections of 40 and 60 are formed. If you and your friend are among the 100 students, what is the probability that
\begin{enumerate}
\item you both enter the same section?
\item you both enter the different sections?
\end{enumerate}
\solution
		%\input{ncert/11/16/4/5/defs.tex}
	\item 
The number lock of a suitcase has 4 wheels each labelled with ten digits i.e. from 0 to 9.The lock opens with a sequence of four digits with no repeats.What is the probability of a person getting the right sequence to open the suitcase.
\\
\solution
		%\input{ncert/11/16/4/10/defs.tex}
		%
\item 
Two cards are drawn at random and without replacement from a pack of 52 playing cards. Find the probability that both the cards are black.
\\
\solution
		%\input{ncert/12/13/2/2/defs.tex}
		\item A box of oranges is inspected by examining three randomly selected oranges drawn without replacement. If all the three oranges are good, the box is approved for sale, otherwise, it is rejected. Find the probability that a box containing 15 oranges out of which 12 are good and 3 are bad ones will be approved for sale.
		\label{ncert/12/13/2/3/defs.tex}
		\item Two balls are drawn at random with replacement from a box containing 10 black and 8 red balls. Find the probability that
		\label{ncert/12/13/2/12}
\begin{enumerate}
\item both balls are red.
\item first ball is black and second is red.
\item one of them is black and other is red.
\end{enumerate}

\item In a hostel, 60\% of the students read Hindi newspaper, 40\% read English newspaper and 20\% read both Hindi and English newspapers. A student is selected at random.
		\label{ncert/12/13/2/15}
\begin{enumerate}
\item Find the probability that she reads neither Hindi nor English newspapers.
\item If she reads Hindi newspaper, find the probability that she reads English newspaper.
\item If she reads English newspaper, find the probability that she reads Hindi newspaper.\\
\end{enumerate}
\item The probability of obtaining an even prime number on each die, when a pair of dice is rolled is 
\begin{enumerate}
    \item $0$ 
    
    \item $\frac{1}{3}$ 
    
    \item $\frac{1}{12}$ 
    
    \item $\frac{1}{36}$ 
\end{enumerate}
\solution
		%\input{ncert/12/13/2/17/defs.tex}
	\item A bag contains 4 red and 4 black balls, another bag contains 2 red and 6 black balls. One of the two bags is selected at random and a ball is drawn from the bag which is found to be red. Find the probability that the ball is drawn from the first bag.
\\
\solution
		%\input{ncert/12/13/3/2/main.tex}
  \item
  Cards with numbers 2 to 101 are placed in a box. A card is selected at random.Find the probability that the card has
\begin{enumerate}[label=(\roman*)]
	\item an even number 
	\item a square number
\end{enumerate}
\solution
%\input{exemplar/10/13/3/32/main.tex}
\item
The king, queen and jack of clubs are removed from a deck of 52 playing cards and then well shuffled. Now one card is drawn at random from the remaining cards.  Determine the probability that the card is
\begin{enumerate}[label=(\roman*)]
\item a club
\item 10 of hearts
\end{enumerate}
\solution
%\input{exemplar/10/13/3/29/main.tex}
\item A team of medical students doing their internship have to assist during surgeries
at a city hospital. The probabilities of surgeries rated as very complex, complex,
routine, simple or very simple are respectively, 0.15, 0.20, 0.31, 0.26, .08. Find
the probabilities that a particular surgery will be rated
\begin{enumerate}
	\item complex or very complex;
	\item neither very complex nor very simple;
	\item routine or complex
	\item routine or simple
\end{enumerate}
\solution
%\input{exemplar/11/16/3/8(1)/main.tex}
\item A card is selected from a pack of 52 cards.
\begin{enumerate}[label=(\alph*)]
    \item How many points are there in the sample space?
    \item Calculate the probability that the card is an ace of spades.
    \item Calculate the probability that the card is (i) an ace and (ii) black card.
\end{enumerate}
\solution
%\input{exemplar/11/16/3/4/main2.tex}
\item The probability that a non leap year selected at random will contain 53 sundays.
\\
\solution
%\input{exemplar/10/13/1/19/main.tex}
\item One of the four persons John, Rita, Aslam or Gurpreet will be promoted next
month. Consequently the sample space consists of four elementary outcomes
S = {John promoted, Rita promoted, Aslam promoted, Gurpreet promoted}
You are told that the chances of John’s promotion is same as that of Gurpreet,
Rita’s chances of promotion are twice as likely as Johns. Aslam’s chances are
four times that of John.
\begin{enumerate}
	\item Determine
	\begin{enumerate}
		\item P (John promoted)
		\item P (Rita promoted)
		\item P (Aslam promoted)
		\item P (Gurpreet promoted)
	\end{enumerate}
	\item If A = {John promoted or Gurpreet promoted}, find P (A).
\end{enumerate}
\solution
%\input{exemplar/11/16/3/10/main.tex}
\item A card is drawn from a deck of 52 cards. Find the probability of getting a king or a heart or a red card.\\
\solution
%\input{exemplar/11/16/3/15/main.tex}
\item The probability that a student will pass his examination is 0.73, the probability of
the student getting a compartment is 0.13, and the probability that the student will
either pass or get compartment is 0.96. State True or False.\\
\solution
%\input{exemplar/11/16/3/31/main.tex}
\item A card is selected from a pack of 52 cards\\
\begin{enumerate}[label=(\alph*)]
\item How many points are there in the sample space?
\item Calculate the probability that the cards is an ace of spades.
\item Calculate the probability that the card is (i) an ace (ii)black card.\\
\end{enumerate}
%\input{ncert/11/16/3/4_1/Prob_4.tex}
\item In a non-leap year, the probability of having 53 tuesdays or 53 wednesdays is\\
\solution
%\input{exemplar/11/16/3/18/main.tex}
\item There are 1000 sealed envelopes in a box, 10 of them contain a cash prize of
Rs 100 each, 100 of them contain a cash prize of Rs 50 each and 200 of them
contain a cash prize of Rs 10 each and rest do not contain any cash prize. If they
are well shuffled and an envelope is picked up out, what is the probability that it
contains no cash prize?\\
\solution
%\input{exemplar/10/13/3/34/main.tex}
\item 
A die is thrown and a card is selected at random from a deck of 52 playing cards. The probability of getting an even number on the die and a spade card.\\
\solution
%\input{exemplar/12/13/3/78/main.tex}
\item
If 4-digit numbers greater than 5,000 are randomly formed from the digits 0, 1, 3, 5, and 7, what is the probability of forming a number divisible by 5 when:
\begin{enumerate}
    \item The digits are repeated?
    \item The repetition of digits is not allowed?
\end{enumerate}
\solution
%\input{ncert/11/16/4/9/main.tex}
\item Consider the probability space $\brak{\Omega, \mathcal{G}, P}$ where $\Omega = [0,2]$ and $\mathcal{G} = \cbrak{\phi, \Omega, [0,1], (1,2]}$. Let $X$ and $Y$ be two functions on $\Omega$ defined as
\begin{align*}
    X(\omega) = 
    \begin{cases}
        1 & \text{if }\omega \in [0, 1]\\
        2 & \text{if }\omega \in (1, 2]
    \end{cases}
\end{align*}
and
\begin{align*}
    Y(\omega) = 
    \begin{cases}
        2 & \text{if }\omega \in [0, 1.5]\\
        3 & \text{if }\omega \in (1.5, 2].
    \end{cases}
\end{align*}
Then which one of the following statements is true?
\begin{enumerate}
    \item [(A)] $X$ is a random variable with respect to $\mathcal{G}$, but $Y$ is not a random variable with respect to $\mathcal{G}$.
    \item [(B)] $Y$ is a random variable with respect to $\mathcal{G}$, but $X$ is not a random variable with respect to $\mathcal{G}$.
    \item [(C)] Neither $X$ nor $Y$ is a random variable with respect to $\mathcal{G}$.
    \item [(D)] Both $X$ and $Y$ are random variables with respect to $\mathcal{G}$.
\end{enumerate} \hfill (GATE ST 2023)\\
\solution
%\input{gate/ST/2023/14/main.tex}
	\item  A die is loaded in such a way that each odd number is twice as likely to occur as
each even number. Find $P(G)$, where $G$ is the event that a number greater than
3 occurs on a single roll of the die.
\\
\solution
		%\input{exemplar/11/16/3/5/main.tex}
	\item All the jacks, queens and kings are removed from a deck of 52 playing cards. The remaining cards are well shuffled and then one card is drawn at random. Giving ace a value 1 similar value for other cards, find the probability that the card has a value 
		\begin{enumerate}
			\item 7
			\item greater than 7
			\item less than 7
		\end{enumerate}
		%\input{exemplar/10/13/3/30/main.tex}
  \item A Lot consists of 48 mobile phones of which 42 are good, 3 have only minor defects and 3 have major defects.Varnika will buy a phone if it is good but the trader will only buy a mobile if it has no major defects. One phone is selected at random from the lot. What is the probability that it is
\begin{enumerate}
	\item acceptable to Varnika?
            \item acceptable to the trader?
\end{enumerate}
\solution
	%\input{exemplar/10/13/3/40/main.tex}
 \item A student says that if you throw a die, it will show up 1 or not 1. Therefore, the probability of getting 1 and the probability of getting 'not 1' each is equal to $\frac{1}{2}$. Is this correct? Give reasons.\\
 \solution
        %\input{exemplar/10/13/2/9/main.tex}
   \item Four candidates A, B, C, D have ap-
plied for the assignment to coach a school cricket
team. If A is twice as likely to be selected as B, and
B and C are given about the same chance of being
selected, while C is twice as likely to be selected
as D, what are the probabilities that
\begin{enumerate}
\item C will be selected?
\item A will not be selected?
\end{enumerate}
	%\input{exemplar/11/16/3/9/main.tex}
 \item A bag contain 24 balls of which $x$ balls are red, $2x$ are white and $3x$ are blue. A ball is selected at random, What is the probability that it is
\begin{enumerate}[label=\alph*)]
\item not red ?
\item white ?
\end{enumerate}
%\input{exemplar/10/13/3/41/main.tex}
If the letters of the word ASSASSINATION are arranged at random. Find the Probability that
\begin{enumerate}[label=(\alph*)]
\item Four $S's$ come consecutively in the word
\item Two  $I's$ and two $N's$ come together
\item All $A's$ are not coming together
\item No two $A's$ are coming together
\end{enumerate}
%\input{exemplar/11/16/3/14/main.tex}
	\item One urn contains two black balls (labelled B1 and B2) and one white ball. A
	second urn contains one black ball and two white balls (labelled W1 and W2).
	Suppose the following experiment is performed. One of the two urns is chosen
	at random. Next a ball is randomly chosen from the urn. Then a second ball is
	chosen at random from the same urn without replacing the first ball.
	
	\begin{enumerate}
	\item What is the probability that two black balls are chosen?
	
	\item What is the probability that two balls of opposite colour are chosen?
	\end{enumerate}
	\solution
	%\input{exemplar/11/16/3/12/main1.tex}
\end{enumerate}

		%
\item 
Out of 100 students, two sections of 40 and 60 are formed. If you and your friend are among the 100 students, what is the probability that
\begin{enumerate}
\item you both enter the same section?
\item you both enter the different sections?
\end{enumerate}
\solution
		%\begin{enumerate}[label=\thesection.\arabic*,ref=\thesection.\theenumi]
	\item One card is drawn from a well-shuffled deck of 52 cards. Find the probability of getting
\begin{enumerate}
\item A king of red colour 
\item A face card 
\item A red face card
\item The jack of hearts
\item A spade
\item The queen of diamonds

\end{enumerate}
\solution
		%\input{ncert/10/15/1/14/main.tex}
	\item Five cards—the ten, jack, queen, king and ace of diamonds, are well-shuffled with their face downwards. One card is then picked up at random.
\begin{enumerate}
\item
What is the probability that the card is the queen? 
\item
If the queen is drawn and put aside, what is the probability that the second card picked up is (a) an ace? (b) a queen?\\
\end{enumerate}
\solution
		%\input{ncert/10/15/1/15/defs.tex}
	\item A bag contains $5$ red balls and some blue balls. If the probability of drawing a blue ball is double that if a red ball, determine the number of blue balls in the bag. 
		\\
\solution
		%\input{ncert/10/15/2/3/defs.tex}
	\item A card is selected from a pack of 52 cards.
 \begin{enumerate}[label=(\alph*)] 
                 \item How many points are there in the sample space?
                 \item Calculate the probability that the card is an ace of spades.
                 \item Calculate the probability that the card is (i) an ace and (ii) black card.
 \end{enumerate}
\solution
		%\input{ncert/11/16/3/4/main.tex}
\item Four cards are drawn from a well-shuffled deck of 52 cards. What is the probability of obtaining 3 diamonds and one spade.
\\
\solution
		%\input{ncert/11/16/4/2/defs.tex}
\item In a certain lottery 10,000 tickets are sold and ten equal prizes are awarded. What is the probability of not getting a prize if you buy (a) one ticket (b) two tickets (c) 10 tickets ?	
\\
\solution
		%\input{ncert/11/16/4/4/defs.tex}
		%
\item 
Out of 100 students, two sections of 40 and 60 are formed. If you and your friend are among the 100 students, what is the probability that
\begin{enumerate}
\item you both enter the same section?
\item you both enter the different sections?
\end{enumerate}
\solution
		%\input{ncert/11/16/4/5/defs.tex}
	\item 
The number lock of a suitcase has 4 wheels each labelled with ten digits i.e. from 0 to 9.The lock opens with a sequence of four digits with no repeats.What is the probability of a person getting the right sequence to open the suitcase.
\\
\solution
		%\input{ncert/11/16/4/10/defs.tex}
		%
\item 
Two cards are drawn at random and without replacement from a pack of 52 playing cards. Find the probability that both the cards are black.
\\
\solution
		%\input{ncert/12/13/2/2/defs.tex}
		\item A box of oranges is inspected by examining three randomly selected oranges drawn without replacement. If all the three oranges are good, the box is approved for sale, otherwise, it is rejected. Find the probability that a box containing 15 oranges out of which 12 are good and 3 are bad ones will be approved for sale.
		\label{ncert/12/13/2/3/defs.tex}
		\item Two balls are drawn at random with replacement from a box containing 10 black and 8 red balls. Find the probability that
		\label{ncert/12/13/2/12}
\begin{enumerate}
\item both balls are red.
\item first ball is black and second is red.
\item one of them is black and other is red.
\end{enumerate}

\item In a hostel, 60\% of the students read Hindi newspaper, 40\% read English newspaper and 20\% read both Hindi and English newspapers. A student is selected at random.
		\label{ncert/12/13/2/15}
\begin{enumerate}
\item Find the probability that she reads neither Hindi nor English newspapers.
\item If she reads Hindi newspaper, find the probability that she reads English newspaper.
\item If she reads English newspaper, find the probability that she reads Hindi newspaper.\\
\end{enumerate}
\item The probability of obtaining an even prime number on each die, when a pair of dice is rolled is 
\begin{enumerate}
    \item $0$ 
    
    \item $\frac{1}{3}$ 
    
    \item $\frac{1}{12}$ 
    
    \item $\frac{1}{36}$ 
\end{enumerate}
\solution
		%\input{ncert/12/13/2/17/defs.tex}
	\item A bag contains 4 red and 4 black balls, another bag contains 2 red and 6 black balls. One of the two bags is selected at random and a ball is drawn from the bag which is found to be red. Find the probability that the ball is drawn from the first bag.
\\
\solution
		%\input{ncert/12/13/3/2/main.tex}
  \item
  Cards with numbers 2 to 101 are placed in a box. A card is selected at random.Find the probability that the card has
\begin{enumerate}[label=(\roman*)]
	\item an even number 
	\item a square number
\end{enumerate}
\solution
%\input{exemplar/10/13/3/32/main.tex}
\item
The king, queen and jack of clubs are removed from a deck of 52 playing cards and then well shuffled. Now one card is drawn at random from the remaining cards.  Determine the probability that the card is
\begin{enumerate}[label=(\roman*)]
\item a club
\item 10 of hearts
\end{enumerate}
\solution
%\input{exemplar/10/13/3/29/main.tex}
\item A team of medical students doing their internship have to assist during surgeries
at a city hospital. The probabilities of surgeries rated as very complex, complex,
routine, simple or very simple are respectively, 0.15, 0.20, 0.31, 0.26, .08. Find
the probabilities that a particular surgery will be rated
\begin{enumerate}
	\item complex or very complex;
	\item neither very complex nor very simple;
	\item routine or complex
	\item routine or simple
\end{enumerate}
\solution
%\input{exemplar/11/16/3/8(1)/main.tex}
\item A card is selected from a pack of 52 cards.
\begin{enumerate}[label=(\alph*)]
    \item How many points are there in the sample space?
    \item Calculate the probability that the card is an ace of spades.
    \item Calculate the probability that the card is (i) an ace and (ii) black card.
\end{enumerate}
\solution
%\input{exemplar/11/16/3/4/main2.tex}
\item The probability that a non leap year selected at random will contain 53 sundays.
\\
\solution
%\input{exemplar/10/13/1/19/main.tex}
\item One of the four persons John, Rita, Aslam or Gurpreet will be promoted next
month. Consequently the sample space consists of four elementary outcomes
S = {John promoted, Rita promoted, Aslam promoted, Gurpreet promoted}
You are told that the chances of John’s promotion is same as that of Gurpreet,
Rita’s chances of promotion are twice as likely as Johns. Aslam’s chances are
four times that of John.
\begin{enumerate}
	\item Determine
	\begin{enumerate}
		\item P (John promoted)
		\item P (Rita promoted)
		\item P (Aslam promoted)
		\item P (Gurpreet promoted)
	\end{enumerate}
	\item If A = {John promoted or Gurpreet promoted}, find P (A).
\end{enumerate}
\solution
%\input{exemplar/11/16/3/10/main.tex}
\item A card is drawn from a deck of 52 cards. Find the probability of getting a king or a heart or a red card.\\
\solution
%\input{exemplar/11/16/3/15/main.tex}
\item The probability that a student will pass his examination is 0.73, the probability of
the student getting a compartment is 0.13, and the probability that the student will
either pass or get compartment is 0.96. State True or False.\\
\solution
%\input{exemplar/11/16/3/31/main.tex}
\item A card is selected from a pack of 52 cards\\
\begin{enumerate}[label=(\alph*)]
\item How many points are there in the sample space?
\item Calculate the probability that the cards is an ace of spades.
\item Calculate the probability that the card is (i) an ace (ii)black card.\\
\end{enumerate}
%\input{ncert/11/16/3/4_1/Prob_4.tex}
\item In a non-leap year, the probability of having 53 tuesdays or 53 wednesdays is\\
\solution
%\input{exemplar/11/16/3/18/main.tex}
\item There are 1000 sealed envelopes in a box, 10 of them contain a cash prize of
Rs 100 each, 100 of them contain a cash prize of Rs 50 each and 200 of them
contain a cash prize of Rs 10 each and rest do not contain any cash prize. If they
are well shuffled and an envelope is picked up out, what is the probability that it
contains no cash prize?\\
\solution
%\input{exemplar/10/13/3/34/main.tex}
\item 
A die is thrown and a card is selected at random from a deck of 52 playing cards. The probability of getting an even number on the die and a spade card.\\
\solution
%\input{exemplar/12/13/3/78/main.tex}
\item
If 4-digit numbers greater than 5,000 are randomly formed from the digits 0, 1, 3, 5, and 7, what is the probability of forming a number divisible by 5 when:
\begin{enumerate}
    \item The digits are repeated?
    \item The repetition of digits is not allowed?
\end{enumerate}
\solution
%\input{ncert/11/16/4/9/main.tex}
\item Consider the probability space $\brak{\Omega, \mathcal{G}, P}$ where $\Omega = [0,2]$ and $\mathcal{G} = \cbrak{\phi, \Omega, [0,1], (1,2]}$. Let $X$ and $Y$ be two functions on $\Omega$ defined as
\begin{align*}
    X(\omega) = 
    \begin{cases}
        1 & \text{if }\omega \in [0, 1]\\
        2 & \text{if }\omega \in (1, 2]
    \end{cases}
\end{align*}
and
\begin{align*}
    Y(\omega) = 
    \begin{cases}
        2 & \text{if }\omega \in [0, 1.5]\\
        3 & \text{if }\omega \in (1.5, 2].
    \end{cases}
\end{align*}
Then which one of the following statements is true?
\begin{enumerate}
    \item [(A)] $X$ is a random variable with respect to $\mathcal{G}$, but $Y$ is not a random variable with respect to $\mathcal{G}$.
    \item [(B)] $Y$ is a random variable with respect to $\mathcal{G}$, but $X$ is not a random variable with respect to $\mathcal{G}$.
    \item [(C)] Neither $X$ nor $Y$ is a random variable with respect to $\mathcal{G}$.
    \item [(D)] Both $X$ and $Y$ are random variables with respect to $\mathcal{G}$.
\end{enumerate} \hfill (GATE ST 2023)\\
\solution
%\input{gate/ST/2023/14/main.tex}
	\item  A die is loaded in such a way that each odd number is twice as likely to occur as
each even number. Find $P(G)$, where $G$ is the event that a number greater than
3 occurs on a single roll of the die.
\\
\solution
		%\input{exemplar/11/16/3/5/main.tex}
	\item All the jacks, queens and kings are removed from a deck of 52 playing cards. The remaining cards are well shuffled and then one card is drawn at random. Giving ace a value 1 similar value for other cards, find the probability that the card has a value 
		\begin{enumerate}
			\item 7
			\item greater than 7
			\item less than 7
		\end{enumerate}
		%\input{exemplar/10/13/3/30/main.tex}
  \item A Lot consists of 48 mobile phones of which 42 are good, 3 have only minor defects and 3 have major defects.Varnika will buy a phone if it is good but the trader will only buy a mobile if it has no major defects. One phone is selected at random from the lot. What is the probability that it is
\begin{enumerate}
	\item acceptable to Varnika?
            \item acceptable to the trader?
\end{enumerate}
\solution
	%\input{exemplar/10/13/3/40/main.tex}
 \item A student says that if you throw a die, it will show up 1 or not 1. Therefore, the probability of getting 1 and the probability of getting 'not 1' each is equal to $\frac{1}{2}$. Is this correct? Give reasons.\\
 \solution
        %\input{exemplar/10/13/2/9/main.tex}
   \item Four candidates A, B, C, D have ap-
plied for the assignment to coach a school cricket
team. If A is twice as likely to be selected as B, and
B and C are given about the same chance of being
selected, while C is twice as likely to be selected
as D, what are the probabilities that
\begin{enumerate}
\item C will be selected?
\item A will not be selected?
\end{enumerate}
	%\input{exemplar/11/16/3/9/main.tex}
 \item A bag contain 24 balls of which $x$ balls are red, $2x$ are white and $3x$ are blue. A ball is selected at random, What is the probability that it is
\begin{enumerate}[label=\alph*)]
\item not red ?
\item white ?
\end{enumerate}
%\input{exemplar/10/13/3/41/main.tex}
If the letters of the word ASSASSINATION are arranged at random. Find the Probability that
\begin{enumerate}[label=(\alph*)]
\item Four $S's$ come consecutively in the word
\item Two  $I's$ and two $N's$ come together
\item All $A's$ are not coming together
\item No two $A's$ are coming together
\end{enumerate}
%\input{exemplar/11/16/3/14/main.tex}
	\item One urn contains two black balls (labelled B1 and B2) and one white ball. A
	second urn contains one black ball and two white balls (labelled W1 and W2).
	Suppose the following experiment is performed. One of the two urns is chosen
	at random. Next a ball is randomly chosen from the urn. Then a second ball is
	chosen at random from the same urn without replacing the first ball.
	
	\begin{enumerate}
	\item What is the probability that two black balls are chosen?
	
	\item What is the probability that two balls of opposite colour are chosen?
	\end{enumerate}
	\solution
	%\input{exemplar/11/16/3/12/main1.tex}
\end{enumerate}

	\item 
The number lock of a suitcase has 4 wheels each labelled with ten digits i.e. from 0 to 9.The lock opens with a sequence of four digits with no repeats.What is the probability of a person getting the right sequence to open the suitcase.
\\
\solution
		%\begin{enumerate}[label=\thesection.\arabic*,ref=\thesection.\theenumi]
	\item One card is drawn from a well-shuffled deck of 52 cards. Find the probability of getting
\begin{enumerate}
\item A king of red colour 
\item A face card 
\item A red face card
\item The jack of hearts
\item A spade
\item The queen of diamonds

\end{enumerate}
\solution
		%\input{ncert/10/15/1/14/main.tex}
	\item Five cards—the ten, jack, queen, king and ace of diamonds, are well-shuffled with their face downwards. One card is then picked up at random.
\begin{enumerate}
\item
What is the probability that the card is the queen? 
\item
If the queen is drawn and put aside, what is the probability that the second card picked up is (a) an ace? (b) a queen?\\
\end{enumerate}
\solution
		%\input{ncert/10/15/1/15/defs.tex}
	\item A bag contains $5$ red balls and some blue balls. If the probability of drawing a blue ball is double that if a red ball, determine the number of blue balls in the bag. 
		\\
\solution
		%\input{ncert/10/15/2/3/defs.tex}
	\item A card is selected from a pack of 52 cards.
 \begin{enumerate}[label=(\alph*)] 
                 \item How many points are there in the sample space?
                 \item Calculate the probability that the card is an ace of spades.
                 \item Calculate the probability that the card is (i) an ace and (ii) black card.
 \end{enumerate}
\solution
		%\input{ncert/11/16/3/4/main.tex}
\item Four cards are drawn from a well-shuffled deck of 52 cards. What is the probability of obtaining 3 diamonds and one spade.
\\
\solution
		%\input{ncert/11/16/4/2/defs.tex}
\item In a certain lottery 10,000 tickets are sold and ten equal prizes are awarded. What is the probability of not getting a prize if you buy (a) one ticket (b) two tickets (c) 10 tickets ?	
\\
\solution
		%\input{ncert/11/16/4/4/defs.tex}
		%
\item 
Out of 100 students, two sections of 40 and 60 are formed. If you and your friend are among the 100 students, what is the probability that
\begin{enumerate}
\item you both enter the same section?
\item you both enter the different sections?
\end{enumerate}
\solution
		%\input{ncert/11/16/4/5/defs.tex}
	\item 
The number lock of a suitcase has 4 wheels each labelled with ten digits i.e. from 0 to 9.The lock opens with a sequence of four digits with no repeats.What is the probability of a person getting the right sequence to open the suitcase.
\\
\solution
		%\input{ncert/11/16/4/10/defs.tex}
		%
\item 
Two cards are drawn at random and without replacement from a pack of 52 playing cards. Find the probability that both the cards are black.
\\
\solution
		%\input{ncert/12/13/2/2/defs.tex}
		\item A box of oranges is inspected by examining three randomly selected oranges drawn without replacement. If all the three oranges are good, the box is approved for sale, otherwise, it is rejected. Find the probability that a box containing 15 oranges out of which 12 are good and 3 are bad ones will be approved for sale.
		\label{ncert/12/13/2/3/defs.tex}
		\item Two balls are drawn at random with replacement from a box containing 10 black and 8 red balls. Find the probability that
		\label{ncert/12/13/2/12}
\begin{enumerate}
\item both balls are red.
\item first ball is black and second is red.
\item one of them is black and other is red.
\end{enumerate}

\item In a hostel, 60\% of the students read Hindi newspaper, 40\% read English newspaper and 20\% read both Hindi and English newspapers. A student is selected at random.
		\label{ncert/12/13/2/15}
\begin{enumerate}
\item Find the probability that she reads neither Hindi nor English newspapers.
\item If she reads Hindi newspaper, find the probability that she reads English newspaper.
\item If she reads English newspaper, find the probability that she reads Hindi newspaper.\\
\end{enumerate}
\item The probability of obtaining an even prime number on each die, when a pair of dice is rolled is 
\begin{enumerate}
    \item $0$ 
    
    \item $\frac{1}{3}$ 
    
    \item $\frac{1}{12}$ 
    
    \item $\frac{1}{36}$ 
\end{enumerate}
\solution
		%\input{ncert/12/13/2/17/defs.tex}
	\item A bag contains 4 red and 4 black balls, another bag contains 2 red and 6 black balls. One of the two bags is selected at random and a ball is drawn from the bag which is found to be red. Find the probability that the ball is drawn from the first bag.
\\
\solution
		%\input{ncert/12/13/3/2/main.tex}
  \item
  Cards with numbers 2 to 101 are placed in a box. A card is selected at random.Find the probability that the card has
\begin{enumerate}[label=(\roman*)]
	\item an even number 
	\item a square number
\end{enumerate}
\solution
%\input{exemplar/10/13/3/32/main.tex}
\item
The king, queen and jack of clubs are removed from a deck of 52 playing cards and then well shuffled. Now one card is drawn at random from the remaining cards.  Determine the probability that the card is
\begin{enumerate}[label=(\roman*)]
\item a club
\item 10 of hearts
\end{enumerate}
\solution
%\input{exemplar/10/13/3/29/main.tex}
\item A team of medical students doing their internship have to assist during surgeries
at a city hospital. The probabilities of surgeries rated as very complex, complex,
routine, simple or very simple are respectively, 0.15, 0.20, 0.31, 0.26, .08. Find
the probabilities that a particular surgery will be rated
\begin{enumerate}
	\item complex or very complex;
	\item neither very complex nor very simple;
	\item routine or complex
	\item routine or simple
\end{enumerate}
\solution
%\input{exemplar/11/16/3/8(1)/main.tex}
\item A card is selected from a pack of 52 cards.
\begin{enumerate}[label=(\alph*)]
    \item How many points are there in the sample space?
    \item Calculate the probability that the card is an ace of spades.
    \item Calculate the probability that the card is (i) an ace and (ii) black card.
\end{enumerate}
\solution
%\input{exemplar/11/16/3/4/main2.tex}
\item The probability that a non leap year selected at random will contain 53 sundays.
\\
\solution
%\input{exemplar/10/13/1/19/main.tex}
\item One of the four persons John, Rita, Aslam or Gurpreet will be promoted next
month. Consequently the sample space consists of four elementary outcomes
S = {John promoted, Rita promoted, Aslam promoted, Gurpreet promoted}
You are told that the chances of John’s promotion is same as that of Gurpreet,
Rita’s chances of promotion are twice as likely as Johns. Aslam’s chances are
four times that of John.
\begin{enumerate}
	\item Determine
	\begin{enumerate}
		\item P (John promoted)
		\item P (Rita promoted)
		\item P (Aslam promoted)
		\item P (Gurpreet promoted)
	\end{enumerate}
	\item If A = {John promoted or Gurpreet promoted}, find P (A).
\end{enumerate}
\solution
%\input{exemplar/11/16/3/10/main.tex}
\item A card is drawn from a deck of 52 cards. Find the probability of getting a king or a heart or a red card.\\
\solution
%\input{exemplar/11/16/3/15/main.tex}
\item The probability that a student will pass his examination is 0.73, the probability of
the student getting a compartment is 0.13, and the probability that the student will
either pass or get compartment is 0.96. State True or False.\\
\solution
%\input{exemplar/11/16/3/31/main.tex}
\item A card is selected from a pack of 52 cards\\
\begin{enumerate}[label=(\alph*)]
\item How many points are there in the sample space?
\item Calculate the probability that the cards is an ace of spades.
\item Calculate the probability that the card is (i) an ace (ii)black card.\\
\end{enumerate}
%\input{ncert/11/16/3/4_1/Prob_4.tex}
\item In a non-leap year, the probability of having 53 tuesdays or 53 wednesdays is\\
\solution
%\input{exemplar/11/16/3/18/main.tex}
\item There are 1000 sealed envelopes in a box, 10 of them contain a cash prize of
Rs 100 each, 100 of them contain a cash prize of Rs 50 each and 200 of them
contain a cash prize of Rs 10 each and rest do not contain any cash prize. If they
are well shuffled and an envelope is picked up out, what is the probability that it
contains no cash prize?\\
\solution
%\input{exemplar/10/13/3/34/main.tex}
\item 
A die is thrown and a card is selected at random from a deck of 52 playing cards. The probability of getting an even number on the die and a spade card.\\
\solution
%\input{exemplar/12/13/3/78/main.tex}
\item
If 4-digit numbers greater than 5,000 are randomly formed from the digits 0, 1, 3, 5, and 7, what is the probability of forming a number divisible by 5 when:
\begin{enumerate}
    \item The digits are repeated?
    \item The repetition of digits is not allowed?
\end{enumerate}
\solution
%\input{ncert/11/16/4/9/main.tex}
\item Consider the probability space $\brak{\Omega, \mathcal{G}, P}$ where $\Omega = [0,2]$ and $\mathcal{G} = \cbrak{\phi, \Omega, [0,1], (1,2]}$. Let $X$ and $Y$ be two functions on $\Omega$ defined as
\begin{align*}
    X(\omega) = 
    \begin{cases}
        1 & \text{if }\omega \in [0, 1]\\
        2 & \text{if }\omega \in (1, 2]
    \end{cases}
\end{align*}
and
\begin{align*}
    Y(\omega) = 
    \begin{cases}
        2 & \text{if }\omega \in [0, 1.5]\\
        3 & \text{if }\omega \in (1.5, 2].
    \end{cases}
\end{align*}
Then which one of the following statements is true?
\begin{enumerate}
    \item [(A)] $X$ is a random variable with respect to $\mathcal{G}$, but $Y$ is not a random variable with respect to $\mathcal{G}$.
    \item [(B)] $Y$ is a random variable with respect to $\mathcal{G}$, but $X$ is not a random variable with respect to $\mathcal{G}$.
    \item [(C)] Neither $X$ nor $Y$ is a random variable with respect to $\mathcal{G}$.
    \item [(D)] Both $X$ and $Y$ are random variables with respect to $\mathcal{G}$.
\end{enumerate} \hfill (GATE ST 2023)\\
\solution
%\input{gate/ST/2023/14/main.tex}
	\item  A die is loaded in such a way that each odd number is twice as likely to occur as
each even number. Find $P(G)$, where $G$ is the event that a number greater than
3 occurs on a single roll of the die.
\\
\solution
		%\input{exemplar/11/16/3/5/main.tex}
	\item All the jacks, queens and kings are removed from a deck of 52 playing cards. The remaining cards are well shuffled and then one card is drawn at random. Giving ace a value 1 similar value for other cards, find the probability that the card has a value 
		\begin{enumerate}
			\item 7
			\item greater than 7
			\item less than 7
		\end{enumerate}
		%\input{exemplar/10/13/3/30/main.tex}
  \item A Lot consists of 48 mobile phones of which 42 are good, 3 have only minor defects and 3 have major defects.Varnika will buy a phone if it is good but the trader will only buy a mobile if it has no major defects. One phone is selected at random from the lot. What is the probability that it is
\begin{enumerate}
	\item acceptable to Varnika?
            \item acceptable to the trader?
\end{enumerate}
\solution
	%\input{exemplar/10/13/3/40/main.tex}
 \item A student says that if you throw a die, it will show up 1 or not 1. Therefore, the probability of getting 1 and the probability of getting 'not 1' each is equal to $\frac{1}{2}$. Is this correct? Give reasons.\\
 \solution
        %\input{exemplar/10/13/2/9/main.tex}
   \item Four candidates A, B, C, D have ap-
plied for the assignment to coach a school cricket
team. If A is twice as likely to be selected as B, and
B and C are given about the same chance of being
selected, while C is twice as likely to be selected
as D, what are the probabilities that
\begin{enumerate}
\item C will be selected?
\item A will not be selected?
\end{enumerate}
	%\input{exemplar/11/16/3/9/main.tex}
 \item A bag contain 24 balls of which $x$ balls are red, $2x$ are white and $3x$ are blue. A ball is selected at random, What is the probability that it is
\begin{enumerate}[label=\alph*)]
\item not red ?
\item white ?
\end{enumerate}
%\input{exemplar/10/13/3/41/main.tex}
If the letters of the word ASSASSINATION are arranged at random. Find the Probability that
\begin{enumerate}[label=(\alph*)]
\item Four $S's$ come consecutively in the word
\item Two  $I's$ and two $N's$ come together
\item All $A's$ are not coming together
\item No two $A's$ are coming together
\end{enumerate}
%\input{exemplar/11/16/3/14/main.tex}
	\item One urn contains two black balls (labelled B1 and B2) and one white ball. A
	second urn contains one black ball and two white balls (labelled W1 and W2).
	Suppose the following experiment is performed. One of the two urns is chosen
	at random. Next a ball is randomly chosen from the urn. Then a second ball is
	chosen at random from the same urn without replacing the first ball.
	
	\begin{enumerate}
	\item What is the probability that two black balls are chosen?
	
	\item What is the probability that two balls of opposite colour are chosen?
	\end{enumerate}
	\solution
	%\input{exemplar/11/16/3/12/main1.tex}
\end{enumerate}

		%
\item 
Two cards are drawn at random and without replacement from a pack of 52 playing cards. Find the probability that both the cards are black.
\\
\solution
		%\begin{enumerate}[label=\thesection.\arabic*,ref=\thesection.\theenumi]
	\item One card is drawn from a well-shuffled deck of 52 cards. Find the probability of getting
\begin{enumerate}
\item A king of red colour 
\item A face card 
\item A red face card
\item The jack of hearts
\item A spade
\item The queen of diamonds

\end{enumerate}
\solution
		%\input{ncert/10/15/1/14/main.tex}
	\item Five cards—the ten, jack, queen, king and ace of diamonds, are well-shuffled with their face downwards. One card is then picked up at random.
\begin{enumerate}
\item
What is the probability that the card is the queen? 
\item
If the queen is drawn and put aside, what is the probability that the second card picked up is (a) an ace? (b) a queen?\\
\end{enumerate}
\solution
		%\input{ncert/10/15/1/15/defs.tex}
	\item A bag contains $5$ red balls and some blue balls. If the probability of drawing a blue ball is double that if a red ball, determine the number of blue balls in the bag. 
		\\
\solution
		%\input{ncert/10/15/2/3/defs.tex}
	\item A card is selected from a pack of 52 cards.
 \begin{enumerate}[label=(\alph*)] 
                 \item How many points are there in the sample space?
                 \item Calculate the probability that the card is an ace of spades.
                 \item Calculate the probability that the card is (i) an ace and (ii) black card.
 \end{enumerate}
\solution
		%\input{ncert/11/16/3/4/main.tex}
\item Four cards are drawn from a well-shuffled deck of 52 cards. What is the probability of obtaining 3 diamonds and one spade.
\\
\solution
		%\input{ncert/11/16/4/2/defs.tex}
\item In a certain lottery 10,000 tickets are sold and ten equal prizes are awarded. What is the probability of not getting a prize if you buy (a) one ticket (b) two tickets (c) 10 tickets ?	
\\
\solution
		%\input{ncert/11/16/4/4/defs.tex}
		%
\item 
Out of 100 students, two sections of 40 and 60 are formed. If you and your friend are among the 100 students, what is the probability that
\begin{enumerate}
\item you both enter the same section?
\item you both enter the different sections?
\end{enumerate}
\solution
		%\input{ncert/11/16/4/5/defs.tex}
	\item 
The number lock of a suitcase has 4 wheels each labelled with ten digits i.e. from 0 to 9.The lock opens with a sequence of four digits with no repeats.What is the probability of a person getting the right sequence to open the suitcase.
\\
\solution
		%\input{ncert/11/16/4/10/defs.tex}
		%
\item 
Two cards are drawn at random and without replacement from a pack of 52 playing cards. Find the probability that both the cards are black.
\\
\solution
		%\input{ncert/12/13/2/2/defs.tex}
		\item A box of oranges is inspected by examining three randomly selected oranges drawn without replacement. If all the three oranges are good, the box is approved for sale, otherwise, it is rejected. Find the probability that a box containing 15 oranges out of which 12 are good and 3 are bad ones will be approved for sale.
		\label{ncert/12/13/2/3/defs.tex}
		\item Two balls are drawn at random with replacement from a box containing 10 black and 8 red balls. Find the probability that
		\label{ncert/12/13/2/12}
\begin{enumerate}
\item both balls are red.
\item first ball is black and second is red.
\item one of them is black and other is red.
\end{enumerate}

\item In a hostel, 60\% of the students read Hindi newspaper, 40\% read English newspaper and 20\% read both Hindi and English newspapers. A student is selected at random.
		\label{ncert/12/13/2/15}
\begin{enumerate}
\item Find the probability that she reads neither Hindi nor English newspapers.
\item If she reads Hindi newspaper, find the probability that she reads English newspaper.
\item If she reads English newspaper, find the probability that she reads Hindi newspaper.\\
\end{enumerate}
\item The probability of obtaining an even prime number on each die, when a pair of dice is rolled is 
\begin{enumerate}
    \item $0$ 
    
    \item $\frac{1}{3}$ 
    
    \item $\frac{1}{12}$ 
    
    \item $\frac{1}{36}$ 
\end{enumerate}
\solution
		%\input{ncert/12/13/2/17/defs.tex}
	\item A bag contains 4 red and 4 black balls, another bag contains 2 red and 6 black balls. One of the two bags is selected at random and a ball is drawn from the bag which is found to be red. Find the probability that the ball is drawn from the first bag.
\\
\solution
		%\input{ncert/12/13/3/2/main.tex}
  \item
  Cards with numbers 2 to 101 are placed in a box. A card is selected at random.Find the probability that the card has
\begin{enumerate}[label=(\roman*)]
	\item an even number 
	\item a square number
\end{enumerate}
\solution
%\input{exemplar/10/13/3/32/main.tex}
\item
The king, queen and jack of clubs are removed from a deck of 52 playing cards and then well shuffled. Now one card is drawn at random from the remaining cards.  Determine the probability that the card is
\begin{enumerate}[label=(\roman*)]
\item a club
\item 10 of hearts
\end{enumerate}
\solution
%\input{exemplar/10/13/3/29/main.tex}
\item A team of medical students doing their internship have to assist during surgeries
at a city hospital. The probabilities of surgeries rated as very complex, complex,
routine, simple or very simple are respectively, 0.15, 0.20, 0.31, 0.26, .08. Find
the probabilities that a particular surgery will be rated
\begin{enumerate}
	\item complex or very complex;
	\item neither very complex nor very simple;
	\item routine or complex
	\item routine or simple
\end{enumerate}
\solution
%\input{exemplar/11/16/3/8(1)/main.tex}
\item A card is selected from a pack of 52 cards.
\begin{enumerate}[label=(\alph*)]
    \item How many points are there in the sample space?
    \item Calculate the probability that the card is an ace of spades.
    \item Calculate the probability that the card is (i) an ace and (ii) black card.
\end{enumerate}
\solution
%\input{exemplar/11/16/3/4/main2.tex}
\item The probability that a non leap year selected at random will contain 53 sundays.
\\
\solution
%\input{exemplar/10/13/1/19/main.tex}
\item One of the four persons John, Rita, Aslam or Gurpreet will be promoted next
month. Consequently the sample space consists of four elementary outcomes
S = {John promoted, Rita promoted, Aslam promoted, Gurpreet promoted}
You are told that the chances of John’s promotion is same as that of Gurpreet,
Rita’s chances of promotion are twice as likely as Johns. Aslam’s chances are
four times that of John.
\begin{enumerate}
	\item Determine
	\begin{enumerate}
		\item P (John promoted)
		\item P (Rita promoted)
		\item P (Aslam promoted)
		\item P (Gurpreet promoted)
	\end{enumerate}
	\item If A = {John promoted or Gurpreet promoted}, find P (A).
\end{enumerate}
\solution
%\input{exemplar/11/16/3/10/main.tex}
\item A card is drawn from a deck of 52 cards. Find the probability of getting a king or a heart or a red card.\\
\solution
%\input{exemplar/11/16/3/15/main.tex}
\item The probability that a student will pass his examination is 0.73, the probability of
the student getting a compartment is 0.13, and the probability that the student will
either pass or get compartment is 0.96. State True or False.\\
\solution
%\input{exemplar/11/16/3/31/main.tex}
\item A card is selected from a pack of 52 cards\\
\begin{enumerate}[label=(\alph*)]
\item How many points are there in the sample space?
\item Calculate the probability that the cards is an ace of spades.
\item Calculate the probability that the card is (i) an ace (ii)black card.\\
\end{enumerate}
%\input{ncert/11/16/3/4_1/Prob_4.tex}
\item In a non-leap year, the probability of having 53 tuesdays or 53 wednesdays is\\
\solution
%\input{exemplar/11/16/3/18/main.tex}
\item There are 1000 sealed envelopes in a box, 10 of them contain a cash prize of
Rs 100 each, 100 of them contain a cash prize of Rs 50 each and 200 of them
contain a cash prize of Rs 10 each and rest do not contain any cash prize. If they
are well shuffled and an envelope is picked up out, what is the probability that it
contains no cash prize?\\
\solution
%\input{exemplar/10/13/3/34/main.tex}
\item 
A die is thrown and a card is selected at random from a deck of 52 playing cards. The probability of getting an even number on the die and a spade card.\\
\solution
%\input{exemplar/12/13/3/78/main.tex}
\item
If 4-digit numbers greater than 5,000 are randomly formed from the digits 0, 1, 3, 5, and 7, what is the probability of forming a number divisible by 5 when:
\begin{enumerate}
    \item The digits are repeated?
    \item The repetition of digits is not allowed?
\end{enumerate}
\solution
%\input{ncert/11/16/4/9/main.tex}
\item Consider the probability space $\brak{\Omega, \mathcal{G}, P}$ where $\Omega = [0,2]$ and $\mathcal{G} = \cbrak{\phi, \Omega, [0,1], (1,2]}$. Let $X$ and $Y$ be two functions on $\Omega$ defined as
\begin{align*}
    X(\omega) = 
    \begin{cases}
        1 & \text{if }\omega \in [0, 1]\\
        2 & \text{if }\omega \in (1, 2]
    \end{cases}
\end{align*}
and
\begin{align*}
    Y(\omega) = 
    \begin{cases}
        2 & \text{if }\omega \in [0, 1.5]\\
        3 & \text{if }\omega \in (1.5, 2].
    \end{cases}
\end{align*}
Then which one of the following statements is true?
\begin{enumerate}
    \item [(A)] $X$ is a random variable with respect to $\mathcal{G}$, but $Y$ is not a random variable with respect to $\mathcal{G}$.
    \item [(B)] $Y$ is a random variable with respect to $\mathcal{G}$, but $X$ is not a random variable with respect to $\mathcal{G}$.
    \item [(C)] Neither $X$ nor $Y$ is a random variable with respect to $\mathcal{G}$.
    \item [(D)] Both $X$ and $Y$ are random variables with respect to $\mathcal{G}$.
\end{enumerate} \hfill (GATE ST 2023)\\
\solution
%\input{gate/ST/2023/14/main.tex}
	\item  A die is loaded in such a way that each odd number is twice as likely to occur as
each even number. Find $P(G)$, where $G$ is the event that a number greater than
3 occurs on a single roll of the die.
\\
\solution
		%\input{exemplar/11/16/3/5/main.tex}
	\item All the jacks, queens and kings are removed from a deck of 52 playing cards. The remaining cards are well shuffled and then one card is drawn at random. Giving ace a value 1 similar value for other cards, find the probability that the card has a value 
		\begin{enumerate}
			\item 7
			\item greater than 7
			\item less than 7
		\end{enumerate}
		%\input{exemplar/10/13/3/30/main.tex}
  \item A Lot consists of 48 mobile phones of which 42 are good, 3 have only minor defects and 3 have major defects.Varnika will buy a phone if it is good but the trader will only buy a mobile if it has no major defects. One phone is selected at random from the lot. What is the probability that it is
\begin{enumerate}
	\item acceptable to Varnika?
            \item acceptable to the trader?
\end{enumerate}
\solution
	%\input{exemplar/10/13/3/40/main.tex}
 \item A student says that if you throw a die, it will show up 1 or not 1. Therefore, the probability of getting 1 and the probability of getting 'not 1' each is equal to $\frac{1}{2}$. Is this correct? Give reasons.\\
 \solution
        %\input{exemplar/10/13/2/9/main.tex}
   \item Four candidates A, B, C, D have ap-
plied for the assignment to coach a school cricket
team. If A is twice as likely to be selected as B, and
B and C are given about the same chance of being
selected, while C is twice as likely to be selected
as D, what are the probabilities that
\begin{enumerate}
\item C will be selected?
\item A will not be selected?
\end{enumerate}
	%\input{exemplar/11/16/3/9/main.tex}
 \item A bag contain 24 balls of which $x$ balls are red, $2x$ are white and $3x$ are blue. A ball is selected at random, What is the probability that it is
\begin{enumerate}[label=\alph*)]
\item not red ?
\item white ?
\end{enumerate}
%\input{exemplar/10/13/3/41/main.tex}
If the letters of the word ASSASSINATION are arranged at random. Find the Probability that
\begin{enumerate}[label=(\alph*)]
\item Four $S's$ come consecutively in the word
\item Two  $I's$ and two $N's$ come together
\item All $A's$ are not coming together
\item No two $A's$ are coming together
\end{enumerate}
%\input{exemplar/11/16/3/14/main.tex}
	\item One urn contains two black balls (labelled B1 and B2) and one white ball. A
	second urn contains one black ball and two white balls (labelled W1 and W2).
	Suppose the following experiment is performed. One of the two urns is chosen
	at random. Next a ball is randomly chosen from the urn. Then a second ball is
	chosen at random from the same urn without replacing the first ball.
	
	\begin{enumerate}
	\item What is the probability that two black balls are chosen?
	
	\item What is the probability that two balls of opposite colour are chosen?
	\end{enumerate}
	\solution
	%\input{exemplar/11/16/3/12/main1.tex}
\end{enumerate}

		\item A box of oranges is inspected by examining three randomly selected oranges drawn without replacement. If all the three oranges are good, the box is approved for sale, otherwise, it is rejected. Find the probability that a box containing 15 oranges out of which 12 are good and 3 are bad ones will be approved for sale.
		\label{ncert/12/13/2/3/defs.tex}
		\item Two balls are drawn at random with replacement from a box containing 10 black and 8 red balls. Find the probability that
		\label{ncert/12/13/2/12}
\begin{enumerate}
\item both balls are red.
\item first ball is black and second is red.
\item one of them is black and other is red.
\end{enumerate}

\item In a hostel, 60\% of the students read Hindi newspaper, 40\% read English newspaper and 20\% read both Hindi and English newspapers. A student is selected at random.
		\label{ncert/12/13/2/15}
\begin{enumerate}
\item Find the probability that she reads neither Hindi nor English newspapers.
\item If she reads Hindi newspaper, find the probability that she reads English newspaper.
\item If she reads English newspaper, find the probability that she reads Hindi newspaper.\\
\end{enumerate}
\item The probability of obtaining an even prime number on each die, when a pair of dice is rolled is 
\begin{enumerate}
    \item $0$ 
    
    \item $\frac{1}{3}$ 
    
    \item $\frac{1}{12}$ 
    
    \item $\frac{1}{36}$ 
\end{enumerate}
\solution
		%\begin{enumerate}[label=\thesection.\arabic*,ref=\thesection.\theenumi]
	\item One card is drawn from a well-shuffled deck of 52 cards. Find the probability of getting
\begin{enumerate}
\item A king of red colour 
\item A face card 
\item A red face card
\item The jack of hearts
\item A spade
\item The queen of diamonds

\end{enumerate}
\solution
		%\input{ncert/10/15/1/14/main.tex}
	\item Five cards—the ten, jack, queen, king and ace of diamonds, are well-shuffled with their face downwards. One card is then picked up at random.
\begin{enumerate}
\item
What is the probability that the card is the queen? 
\item
If the queen is drawn and put aside, what is the probability that the second card picked up is (a) an ace? (b) a queen?\\
\end{enumerate}
\solution
		%\input{ncert/10/15/1/15/defs.tex}
	\item A bag contains $5$ red balls and some blue balls. If the probability of drawing a blue ball is double that if a red ball, determine the number of blue balls in the bag. 
		\\
\solution
		%\input{ncert/10/15/2/3/defs.tex}
	\item A card is selected from a pack of 52 cards.
 \begin{enumerate}[label=(\alph*)] 
                 \item How many points are there in the sample space?
                 \item Calculate the probability that the card is an ace of spades.
                 \item Calculate the probability that the card is (i) an ace and (ii) black card.
 \end{enumerate}
\solution
		%\input{ncert/11/16/3/4/main.tex}
\item Four cards are drawn from a well-shuffled deck of 52 cards. What is the probability of obtaining 3 diamonds and one spade.
\\
\solution
		%\input{ncert/11/16/4/2/defs.tex}
\item In a certain lottery 10,000 tickets are sold and ten equal prizes are awarded. What is the probability of not getting a prize if you buy (a) one ticket (b) two tickets (c) 10 tickets ?	
\\
\solution
		%\input{ncert/11/16/4/4/defs.tex}
		%
\item 
Out of 100 students, two sections of 40 and 60 are formed. If you and your friend are among the 100 students, what is the probability that
\begin{enumerate}
\item you both enter the same section?
\item you both enter the different sections?
\end{enumerate}
\solution
		%\input{ncert/11/16/4/5/defs.tex}
	\item 
The number lock of a suitcase has 4 wheels each labelled with ten digits i.e. from 0 to 9.The lock opens with a sequence of four digits with no repeats.What is the probability of a person getting the right sequence to open the suitcase.
\\
\solution
		%\input{ncert/11/16/4/10/defs.tex}
		%
\item 
Two cards are drawn at random and without replacement from a pack of 52 playing cards. Find the probability that both the cards are black.
\\
\solution
		%\input{ncert/12/13/2/2/defs.tex}
		\item A box of oranges is inspected by examining three randomly selected oranges drawn without replacement. If all the three oranges are good, the box is approved for sale, otherwise, it is rejected. Find the probability that a box containing 15 oranges out of which 12 are good and 3 are bad ones will be approved for sale.
		\label{ncert/12/13/2/3/defs.tex}
		\item Two balls are drawn at random with replacement from a box containing 10 black and 8 red balls. Find the probability that
		\label{ncert/12/13/2/12}
\begin{enumerate}
\item both balls are red.
\item first ball is black and second is red.
\item one of them is black and other is red.
\end{enumerate}

\item In a hostel, 60\% of the students read Hindi newspaper, 40\% read English newspaper and 20\% read both Hindi and English newspapers. A student is selected at random.
		\label{ncert/12/13/2/15}
\begin{enumerate}
\item Find the probability that she reads neither Hindi nor English newspapers.
\item If she reads Hindi newspaper, find the probability that she reads English newspaper.
\item If she reads English newspaper, find the probability that she reads Hindi newspaper.\\
\end{enumerate}
\item The probability of obtaining an even prime number on each die, when a pair of dice is rolled is 
\begin{enumerate}
    \item $0$ 
    
    \item $\frac{1}{3}$ 
    
    \item $\frac{1}{12}$ 
    
    \item $\frac{1}{36}$ 
\end{enumerate}
\solution
		%\input{ncert/12/13/2/17/defs.tex}
	\item A bag contains 4 red and 4 black balls, another bag contains 2 red and 6 black balls. One of the two bags is selected at random and a ball is drawn from the bag which is found to be red. Find the probability that the ball is drawn from the first bag.
\\
\solution
		%\input{ncert/12/13/3/2/main.tex}
  \item
  Cards with numbers 2 to 101 are placed in a box. A card is selected at random.Find the probability that the card has
\begin{enumerate}[label=(\roman*)]
	\item an even number 
	\item a square number
\end{enumerate}
\solution
%\input{exemplar/10/13/3/32/main.tex}
\item
The king, queen and jack of clubs are removed from a deck of 52 playing cards and then well shuffled. Now one card is drawn at random from the remaining cards.  Determine the probability that the card is
\begin{enumerate}[label=(\roman*)]
\item a club
\item 10 of hearts
\end{enumerate}
\solution
%\input{exemplar/10/13/3/29/main.tex}
\item A team of medical students doing their internship have to assist during surgeries
at a city hospital. The probabilities of surgeries rated as very complex, complex,
routine, simple or very simple are respectively, 0.15, 0.20, 0.31, 0.26, .08. Find
the probabilities that a particular surgery will be rated
\begin{enumerate}
	\item complex or very complex;
	\item neither very complex nor very simple;
	\item routine or complex
	\item routine or simple
\end{enumerate}
\solution
%\input{exemplar/11/16/3/8(1)/main.tex}
\item A card is selected from a pack of 52 cards.
\begin{enumerate}[label=(\alph*)]
    \item How many points are there in the sample space?
    \item Calculate the probability that the card is an ace of spades.
    \item Calculate the probability that the card is (i) an ace and (ii) black card.
\end{enumerate}
\solution
%\input{exemplar/11/16/3/4/main2.tex}
\item The probability that a non leap year selected at random will contain 53 sundays.
\\
\solution
%\input{exemplar/10/13/1/19/main.tex}
\item One of the four persons John, Rita, Aslam or Gurpreet will be promoted next
month. Consequently the sample space consists of four elementary outcomes
S = {John promoted, Rita promoted, Aslam promoted, Gurpreet promoted}
You are told that the chances of John’s promotion is same as that of Gurpreet,
Rita’s chances of promotion are twice as likely as Johns. Aslam’s chances are
four times that of John.
\begin{enumerate}
	\item Determine
	\begin{enumerate}
		\item P (John promoted)
		\item P (Rita promoted)
		\item P (Aslam promoted)
		\item P (Gurpreet promoted)
	\end{enumerate}
	\item If A = {John promoted or Gurpreet promoted}, find P (A).
\end{enumerate}
\solution
%\input{exemplar/11/16/3/10/main.tex}
\item A card is drawn from a deck of 52 cards. Find the probability of getting a king or a heart or a red card.\\
\solution
%\input{exemplar/11/16/3/15/main.tex}
\item The probability that a student will pass his examination is 0.73, the probability of
the student getting a compartment is 0.13, and the probability that the student will
either pass or get compartment is 0.96. State True or False.\\
\solution
%\input{exemplar/11/16/3/31/main.tex}
\item A card is selected from a pack of 52 cards\\
\begin{enumerate}[label=(\alph*)]
\item How many points are there in the sample space?
\item Calculate the probability that the cards is an ace of spades.
\item Calculate the probability that the card is (i) an ace (ii)black card.\\
\end{enumerate}
%\input{ncert/11/16/3/4_1/Prob_4.tex}
\item In a non-leap year, the probability of having 53 tuesdays or 53 wednesdays is\\
\solution
%\input{exemplar/11/16/3/18/main.tex}
\item There are 1000 sealed envelopes in a box, 10 of them contain a cash prize of
Rs 100 each, 100 of them contain a cash prize of Rs 50 each and 200 of them
contain a cash prize of Rs 10 each and rest do not contain any cash prize. If they
are well shuffled and an envelope is picked up out, what is the probability that it
contains no cash prize?\\
\solution
%\input{exemplar/10/13/3/34/main.tex}
\item 
A die is thrown and a card is selected at random from a deck of 52 playing cards. The probability of getting an even number on the die and a spade card.\\
\solution
%\input{exemplar/12/13/3/78/main.tex}
\item
If 4-digit numbers greater than 5,000 are randomly formed from the digits 0, 1, 3, 5, and 7, what is the probability of forming a number divisible by 5 when:
\begin{enumerate}
    \item The digits are repeated?
    \item The repetition of digits is not allowed?
\end{enumerate}
\solution
%\input{ncert/11/16/4/9/main.tex}
\item Consider the probability space $\brak{\Omega, \mathcal{G}, P}$ where $\Omega = [0,2]$ and $\mathcal{G} = \cbrak{\phi, \Omega, [0,1], (1,2]}$. Let $X$ and $Y$ be two functions on $\Omega$ defined as
\begin{align*}
    X(\omega) = 
    \begin{cases}
        1 & \text{if }\omega \in [0, 1]\\
        2 & \text{if }\omega \in (1, 2]
    \end{cases}
\end{align*}
and
\begin{align*}
    Y(\omega) = 
    \begin{cases}
        2 & \text{if }\omega \in [0, 1.5]\\
        3 & \text{if }\omega \in (1.5, 2].
    \end{cases}
\end{align*}
Then which one of the following statements is true?
\begin{enumerate}
    \item [(A)] $X$ is a random variable with respect to $\mathcal{G}$, but $Y$ is not a random variable with respect to $\mathcal{G}$.
    \item [(B)] $Y$ is a random variable with respect to $\mathcal{G}$, but $X$ is not a random variable with respect to $\mathcal{G}$.
    \item [(C)] Neither $X$ nor $Y$ is a random variable with respect to $\mathcal{G}$.
    \item [(D)] Both $X$ and $Y$ are random variables with respect to $\mathcal{G}$.
\end{enumerate} \hfill (GATE ST 2023)\\
\solution
%\input{gate/ST/2023/14/main.tex}
	\item  A die is loaded in such a way that each odd number is twice as likely to occur as
each even number. Find $P(G)$, where $G$ is the event that a number greater than
3 occurs on a single roll of the die.
\\
\solution
		%\input{exemplar/11/16/3/5/main.tex}
	\item All the jacks, queens and kings are removed from a deck of 52 playing cards. The remaining cards are well shuffled and then one card is drawn at random. Giving ace a value 1 similar value for other cards, find the probability that the card has a value 
		\begin{enumerate}
			\item 7
			\item greater than 7
			\item less than 7
		\end{enumerate}
		%\input{exemplar/10/13/3/30/main.tex}
  \item A Lot consists of 48 mobile phones of which 42 are good, 3 have only minor defects and 3 have major defects.Varnika will buy a phone if it is good but the trader will only buy a mobile if it has no major defects. One phone is selected at random from the lot. What is the probability that it is
\begin{enumerate}
	\item acceptable to Varnika?
            \item acceptable to the trader?
\end{enumerate}
\solution
	%\input{exemplar/10/13/3/40/main.tex}
 \item A student says that if you throw a die, it will show up 1 or not 1. Therefore, the probability of getting 1 and the probability of getting 'not 1' each is equal to $\frac{1}{2}$. Is this correct? Give reasons.\\
 \solution
        %\input{exemplar/10/13/2/9/main.tex}
   \item Four candidates A, B, C, D have ap-
plied for the assignment to coach a school cricket
team. If A is twice as likely to be selected as B, and
B and C are given about the same chance of being
selected, while C is twice as likely to be selected
as D, what are the probabilities that
\begin{enumerate}
\item C will be selected?
\item A will not be selected?
\end{enumerate}
	%\input{exemplar/11/16/3/9/main.tex}
 \item A bag contain 24 balls of which $x$ balls are red, $2x$ are white and $3x$ are blue. A ball is selected at random, What is the probability that it is
\begin{enumerate}[label=\alph*)]
\item not red ?
\item white ?
\end{enumerate}
%\input{exemplar/10/13/3/41/main.tex}
If the letters of the word ASSASSINATION are arranged at random. Find the Probability that
\begin{enumerate}[label=(\alph*)]
\item Four $S's$ come consecutively in the word
\item Two  $I's$ and two $N's$ come together
\item All $A's$ are not coming together
\item No two $A's$ are coming together
\end{enumerate}
%\input{exemplar/11/16/3/14/main.tex}
	\item One urn contains two black balls (labelled B1 and B2) and one white ball. A
	second urn contains one black ball and two white balls (labelled W1 and W2).
	Suppose the following experiment is performed. One of the two urns is chosen
	at random. Next a ball is randomly chosen from the urn. Then a second ball is
	chosen at random from the same urn without replacing the first ball.
	
	\begin{enumerate}
	\item What is the probability that two black balls are chosen?
	
	\item What is the probability that two balls of opposite colour are chosen?
	\end{enumerate}
	\solution
	%\input{exemplar/11/16/3/12/main1.tex}
\end{enumerate}

	\item A bag contains 4 red and 4 black balls, another bag contains 2 red and 6 black balls. One of the two bags is selected at random and a ball is drawn from the bag which is found to be red. Find the probability that the ball is drawn from the first bag.
\\
\solution
		%\begin{table}[H]
	\centering
\begin{tabular}{|c|c|c|}
\hline
Random variable &Value &Definition\\ \hline
\multirow{3}{*}{X} &0 &Slips of Rs 1\\
&1 &Slips of Rs 5\\
&2 &Slips of Rs 13\\ \hline
\multirow{2}{*}{Y} &0 &Box A\\
&1 &Box B\\\hline
\end{tabular}
\caption{}
\label{tab:Distribution}
\end{table}
See \tabref{tab:Distribution}.
\begin{align}
p_{Y}\brak{k}= \begin{cases} 
      \frac{1}{3} & {k=0} \\
      \frac{2}{3 }& {k=1} 
   \end{cases}
   \\
p_{Y|X}\brak{0|0} = \frac{19}{25}\, 
p_{Y|X}\brak{0|1} = \frac{6}{25}\,
p_{Y|X}\brak{1|0} = \frac{45}{50}\,
p_{Y|X}\brak{1|2} = \frac{5}{50}
\end{align}
The desired probability is the probability that a slip drawn at random is marked other than Rs 1,
\begin{align}
&=1-p_X\brak{0}\\
&= p_X(1) + p_X(2)
\end{align}
Using Bayes theorem,
\begin{align}
&= p_Y\brak{0} \times \pr{Y=0 | X=1} + p_Y\brak{1} \times \pr{Y=1|X=2}\\
&=\frac{1}{3} \times \frac{6}{25} + \frac{2}{3} \times \frac{5}{50}\\
&=\frac{11}{75}
\end{align}

\newpage

%\tableofcontents

\bigskip

\renewcommand{\thefigure}{\theenumi}
\renewcommand{\thetable}{\theenumi}
%\renewcommand{\theequation}{\theenumi}

%\begin{abstract}
%%\boldmath
%In this letter, an algorithm for evaluating the exact analytical bit error rate  (BER)  for the piecewise linear (PL) combiner for  multiple relays is presented. Previous results were available only for upto three relays. The algorithm is unique in the sense that  the actual mathematical expressions, that are prohibitively large, need not be explicitly obtained. The diversity gain due to multiple relays is shown through plots of the analytical BER, well supported by simulations. 
%
%\end{abstract}
% IEEEtran.cls defaults to using nonbold math in the Abstract.
% This preserves the distinction between vectors and scalars. However,
% if the journal you are submitting to favors bold math in the abstract,
% then you can use LaTeX's standard command \boldmath at the very start
% of the abstract to achieve this. Many IEEE journals frown on math
% in the abstract anyway.

% Note that keywords are not normally used for peerreview papers.
%\begin{IEEEkeywords}
%Cooperative diversity, decode and forward, piecewise linear
%\end{IEEEkeywords}



% For peer review papers, you can put extra information on the cover
% page as needed:
% \ifCLASSOPTIONpeerreview
% \begin{center} \bfseries EDICS Category: 3-BBND \end{center}
% \fi
%
% For peerreview papers, this IEEEtran command inserts a page break and
% creates the second title. It will be ignored for other modes.
%\IEEEpeerreviewmaketitle




  \item
  Cards with numbers 2 to 101 are placed in a box. A card is selected at random.Find the probability that the card has
\begin{enumerate}[label=(\roman*)]
	\item an even number 
	\item a square number
\end{enumerate}
\solution
%\begin{table}[H]
	\centering
\begin{tabular}{|c|c|c|}
\hline
Random variable &Value &Definition\\ \hline
\multirow{3}{*}{X} &0 &Slips of Rs 1\\
&1 &Slips of Rs 5\\
&2 &Slips of Rs 13\\ \hline
\multirow{2}{*}{Y} &0 &Box A\\
&1 &Box B\\\hline
\end{tabular}
\caption{}
\label{tab:Distribution}
\end{table}
See \tabref{tab:Distribution}.
\begin{align}
p_{Y}\brak{k}= \begin{cases} 
      \frac{1}{3} & {k=0} \\
      \frac{2}{3 }& {k=1} 
   \end{cases}
   \\
p_{Y|X}\brak{0|0} = \frac{19}{25}\, 
p_{Y|X}\brak{0|1} = \frac{6}{25}\,
p_{Y|X}\brak{1|0} = \frac{45}{50}\,
p_{Y|X}\brak{1|2} = \frac{5}{50}
\end{align}
The desired probability is the probability that a slip drawn at random is marked other than Rs 1,
\begin{align}
&=1-p_X\brak{0}\\
&= p_X(1) + p_X(2)
\end{align}
Using Bayes theorem,
\begin{align}
&= p_Y\brak{0} \times \pr{Y=0 | X=1} + p_Y\brak{1} \times \pr{Y=1|X=2}\\
&=\frac{1}{3} \times \frac{6}{25} + \frac{2}{3} \times \frac{5}{50}\\
&=\frac{11}{75}
\end{align}

\newpage

%\tableofcontents

\bigskip

\renewcommand{\thefigure}{\theenumi}
\renewcommand{\thetable}{\theenumi}
%\renewcommand{\theequation}{\theenumi}

%\begin{abstract}
%%\boldmath
%In this letter, an algorithm for evaluating the exact analytical bit error rate  (BER)  for the piecewise linear (PL) combiner for  multiple relays is presented. Previous results were available only for upto three relays. The algorithm is unique in the sense that  the actual mathematical expressions, that are prohibitively large, need not be explicitly obtained. The diversity gain due to multiple relays is shown through plots of the analytical BER, well supported by simulations. 
%
%\end{abstract}
% IEEEtran.cls defaults to using nonbold math in the Abstract.
% This preserves the distinction between vectors and scalars. However,
% if the journal you are submitting to favors bold math in the abstract,
% then you can use LaTeX's standard command \boldmath at the very start
% of the abstract to achieve this. Many IEEE journals frown on math
% in the abstract anyway.

% Note that keywords are not normally used for peerreview papers.
%\begin{IEEEkeywords}
%Cooperative diversity, decode and forward, piecewise linear
%\end{IEEEkeywords}



% For peer review papers, you can put extra information on the cover
% page as needed:
% \ifCLASSOPTIONpeerreview
% \begin{center} \bfseries EDICS Category: 3-BBND \end{center}
% \fi
%
% For peerreview papers, this IEEEtran command inserts a page break and
% creates the second title. It will be ignored for other modes.
%\IEEEpeerreviewmaketitle




\item
The king, queen and jack of clubs are removed from a deck of 52 playing cards and then well shuffled. Now one card is drawn at random from the remaining cards.  Determine the probability that the card is
\begin{enumerate}[label=(\roman*)]
\item a club
\item 10 of hearts
\end{enumerate}
\solution
%\begin{table}[H]
	\centering
\begin{tabular}{|c|c|c|}
\hline
Random variable &Value &Definition\\ \hline
\multirow{3}{*}{X} &0 &Slips of Rs 1\\
&1 &Slips of Rs 5\\
&2 &Slips of Rs 13\\ \hline
\multirow{2}{*}{Y} &0 &Box A\\
&1 &Box B\\\hline
\end{tabular}
\caption{}
\label{tab:Distribution}
\end{table}
See \tabref{tab:Distribution}.
\begin{align}
p_{Y}\brak{k}= \begin{cases} 
      \frac{1}{3} & {k=0} \\
      \frac{2}{3 }& {k=1} 
   \end{cases}
   \\
p_{Y|X}\brak{0|0} = \frac{19}{25}\, 
p_{Y|X}\brak{0|1} = \frac{6}{25}\,
p_{Y|X}\brak{1|0} = \frac{45}{50}\,
p_{Y|X}\brak{1|2} = \frac{5}{50}
\end{align}
The desired probability is the probability that a slip drawn at random is marked other than Rs 1,
\begin{align}
&=1-p_X\brak{0}\\
&= p_X(1) + p_X(2)
\end{align}
Using Bayes theorem,
\begin{align}
&= p_Y\brak{0} \times \pr{Y=0 | X=1} + p_Y\brak{1} \times \pr{Y=1|X=2}\\
&=\frac{1}{3} \times \frac{6}{25} + \frac{2}{3} \times \frac{5}{50}\\
&=\frac{11}{75}
\end{align}

\newpage

%\tableofcontents

\bigskip

\renewcommand{\thefigure}{\theenumi}
\renewcommand{\thetable}{\theenumi}
%\renewcommand{\theequation}{\theenumi}

%\begin{abstract}
%%\boldmath
%In this letter, an algorithm for evaluating the exact analytical bit error rate  (BER)  for the piecewise linear (PL) combiner for  multiple relays is presented. Previous results were available only for upto three relays. The algorithm is unique in the sense that  the actual mathematical expressions, that are prohibitively large, need not be explicitly obtained. The diversity gain due to multiple relays is shown through plots of the analytical BER, well supported by simulations. 
%
%\end{abstract}
% IEEEtran.cls defaults to using nonbold math in the Abstract.
% This preserves the distinction between vectors and scalars. However,
% if the journal you are submitting to favors bold math in the abstract,
% then you can use LaTeX's standard command \boldmath at the very start
% of the abstract to achieve this. Many IEEE journals frown on math
% in the abstract anyway.

% Note that keywords are not normally used for peerreview papers.
%\begin{IEEEkeywords}
%Cooperative diversity, decode and forward, piecewise linear
%\end{IEEEkeywords}



% For peer review papers, you can put extra information on the cover
% page as needed:
% \ifCLASSOPTIONpeerreview
% \begin{center} \bfseries EDICS Category: 3-BBND \end{center}
% \fi
%
% For peerreview papers, this IEEEtran command inserts a page break and
% creates the second title. It will be ignored for other modes.
%\IEEEpeerreviewmaketitle




\item A team of medical students doing their internship have to assist during surgeries
at a city hospital. The probabilities of surgeries rated as very complex, complex,
routine, simple or very simple are respectively, 0.15, 0.20, 0.31, 0.26, .08. Find
the probabilities that a particular surgery will be rated
\begin{enumerate}
	\item complex or very complex;
	\item neither very complex nor very simple;
	\item routine or complex
	\item routine or simple
\end{enumerate}
\solution
%\begin{table}[H]
	\centering
\begin{tabular}{|c|c|c|}
\hline
Random variable &Value &Definition\\ \hline
\multirow{3}{*}{X} &0 &Slips of Rs 1\\
&1 &Slips of Rs 5\\
&2 &Slips of Rs 13\\ \hline
\multirow{2}{*}{Y} &0 &Box A\\
&1 &Box B\\\hline
\end{tabular}
\caption{}
\label{tab:Distribution}
\end{table}
See \tabref{tab:Distribution}.
\begin{align}
p_{Y}\brak{k}= \begin{cases} 
      \frac{1}{3} & {k=0} \\
      \frac{2}{3 }& {k=1} 
   \end{cases}
   \\
p_{Y|X}\brak{0|0} = \frac{19}{25}\, 
p_{Y|X}\brak{0|1} = \frac{6}{25}\,
p_{Y|X}\brak{1|0} = \frac{45}{50}\,
p_{Y|X}\brak{1|2} = \frac{5}{50}
\end{align}
The desired probability is the probability that a slip drawn at random is marked other than Rs 1,
\begin{align}
&=1-p_X\brak{0}\\
&= p_X(1) + p_X(2)
\end{align}
Using Bayes theorem,
\begin{align}
&= p_Y\brak{0} \times \pr{Y=0 | X=1} + p_Y\brak{1} \times \pr{Y=1|X=2}\\
&=\frac{1}{3} \times \frac{6}{25} + \frac{2}{3} \times \frac{5}{50}\\
&=\frac{11}{75}
\end{align}

\newpage

%\tableofcontents

\bigskip

\renewcommand{\thefigure}{\theenumi}
\renewcommand{\thetable}{\theenumi}
%\renewcommand{\theequation}{\theenumi}

%\begin{abstract}
%%\boldmath
%In this letter, an algorithm for evaluating the exact analytical bit error rate  (BER)  for the piecewise linear (PL) combiner for  multiple relays is presented. Previous results were available only for upto three relays. The algorithm is unique in the sense that  the actual mathematical expressions, that are prohibitively large, need not be explicitly obtained. The diversity gain due to multiple relays is shown through plots of the analytical BER, well supported by simulations. 
%
%\end{abstract}
% IEEEtran.cls defaults to using nonbold math in the Abstract.
% This preserves the distinction between vectors and scalars. However,
% if the journal you are submitting to favors bold math in the abstract,
% then you can use LaTeX's standard command \boldmath at the very start
% of the abstract to achieve this. Many IEEE journals frown on math
% in the abstract anyway.

% Note that keywords are not normally used for peerreview papers.
%\begin{IEEEkeywords}
%Cooperative diversity, decode and forward, piecewise linear
%\end{IEEEkeywords}



% For peer review papers, you can put extra information on the cover
% page as needed:
% \ifCLASSOPTIONpeerreview
% \begin{center} \bfseries EDICS Category: 3-BBND \end{center}
% \fi
%
% For peerreview papers, this IEEEtran command inserts a page break and
% creates the second title. It will be ignored for other modes.
%\IEEEpeerreviewmaketitle




\item A card is selected from a pack of 52 cards.
\begin{enumerate}[label=(\alph*)]
    \item How many points are there in the sample space?
    \item Calculate the probability that the card is an ace of spades.
    \item Calculate the probability that the card is (i) an ace and (ii) black card.
\end{enumerate}
\solution
%Let $X$ be an bernoulli rv defined as in \tabref{tab:exemplar/11/16/3/26}.  Then, 
\begin{equation}
    p =
        \frac{4}{11} 
\end{equation}
\begin{table}[H]
	\centering
	\input{exemplar/11/16/3/26/tables/Table2.tex}
	\caption{}
        \label{tab:exemplar/11/16/3/26}
\end{table}

\item The probability that a non leap year selected at random will contain 53 sundays.
\\
\solution
%\begin{table}[H]
	\centering
\begin{tabular}{|c|c|c|}
\hline
Random variable &Value &Definition\\ \hline
\multirow{3}{*}{X} &0 &Slips of Rs 1\\
&1 &Slips of Rs 5\\
&2 &Slips of Rs 13\\ \hline
\multirow{2}{*}{Y} &0 &Box A\\
&1 &Box B\\\hline
\end{tabular}
\caption{}
\label{tab:Distribution}
\end{table}
See \tabref{tab:Distribution}.
\begin{align}
p_{Y}\brak{k}= \begin{cases} 
      \frac{1}{3} & {k=0} \\
      \frac{2}{3 }& {k=1} 
   \end{cases}
   \\
p_{Y|X}\brak{0|0} = \frac{19}{25}\, 
p_{Y|X}\brak{0|1} = \frac{6}{25}\,
p_{Y|X}\brak{1|0} = \frac{45}{50}\,
p_{Y|X}\brak{1|2} = \frac{5}{50}
\end{align}
The desired probability is the probability that a slip drawn at random is marked other than Rs 1,
\begin{align}
&=1-p_X\brak{0}\\
&= p_X(1) + p_X(2)
\end{align}
Using Bayes theorem,
\begin{align}
&= p_Y\brak{0} \times \pr{Y=0 | X=1} + p_Y\brak{1} \times \pr{Y=1|X=2}\\
&=\frac{1}{3} \times \frac{6}{25} + \frac{2}{3} \times \frac{5}{50}\\
&=\frac{11}{75}
\end{align}

\newpage

%\tableofcontents

\bigskip

\renewcommand{\thefigure}{\theenumi}
\renewcommand{\thetable}{\theenumi}
%\renewcommand{\theequation}{\theenumi}

%\begin{abstract}
%%\boldmath
%In this letter, an algorithm for evaluating the exact analytical bit error rate  (BER)  for the piecewise linear (PL) combiner for  multiple relays is presented. Previous results were available only for upto three relays. The algorithm is unique in the sense that  the actual mathematical expressions, that are prohibitively large, need not be explicitly obtained. The diversity gain due to multiple relays is shown through plots of the analytical BER, well supported by simulations. 
%
%\end{abstract}
% IEEEtran.cls defaults to using nonbold math in the Abstract.
% This preserves the distinction between vectors and scalars. However,
% if the journal you are submitting to favors bold math in the abstract,
% then you can use LaTeX's standard command \boldmath at the very start
% of the abstract to achieve this. Many IEEE journals frown on math
% in the abstract anyway.

% Note that keywords are not normally used for peerreview papers.
%\begin{IEEEkeywords}
%Cooperative diversity, decode and forward, piecewise linear
%\end{IEEEkeywords}



% For peer review papers, you can put extra information on the cover
% page as needed:
% \ifCLASSOPTIONpeerreview
% \begin{center} \bfseries EDICS Category: 3-BBND \end{center}
% \fi
%
% For peerreview papers, this IEEEtran command inserts a page break and
% creates the second title. It will be ignored for other modes.
%\IEEEpeerreviewmaketitle




\item One of the four persons John, Rita, Aslam or Gurpreet will be promoted next
month. Consequently the sample space consists of four elementary outcomes
S = {John promoted, Rita promoted, Aslam promoted, Gurpreet promoted}
You are told that the chances of John’s promotion is same as that of Gurpreet,
Rita’s chances of promotion are twice as likely as Johns. Aslam’s chances are
four times that of John.
\begin{enumerate}
	\item Determine
	\begin{enumerate}
		\item P (John promoted)
		\item P (Rita promoted)
		\item P (Aslam promoted)
		\item P (Gurpreet promoted)
	\end{enumerate}
	\item If A = {John promoted or Gurpreet promoted}, find P (A).
\end{enumerate}
\solution
%\begin{table}[H]
	\centering
\begin{tabular}{|c|c|c|}
\hline
Random variable &Value &Definition\\ \hline
\multirow{3}{*}{X} &0 &Slips of Rs 1\\
&1 &Slips of Rs 5\\
&2 &Slips of Rs 13\\ \hline
\multirow{2}{*}{Y} &0 &Box A\\
&1 &Box B\\\hline
\end{tabular}
\caption{}
\label{tab:Distribution}
\end{table}
See \tabref{tab:Distribution}.
\begin{align}
p_{Y}\brak{k}= \begin{cases} 
      \frac{1}{3} & {k=0} \\
      \frac{2}{3 }& {k=1} 
   \end{cases}
   \\
p_{Y|X}\brak{0|0} = \frac{19}{25}\, 
p_{Y|X}\brak{0|1} = \frac{6}{25}\,
p_{Y|X}\brak{1|0} = \frac{45}{50}\,
p_{Y|X}\brak{1|2} = \frac{5}{50}
\end{align}
The desired probability is the probability that a slip drawn at random is marked other than Rs 1,
\begin{align}
&=1-p_X\brak{0}\\
&= p_X(1) + p_X(2)
\end{align}
Using Bayes theorem,
\begin{align}
&= p_Y\brak{0} \times \pr{Y=0 | X=1} + p_Y\brak{1} \times \pr{Y=1|X=2}\\
&=\frac{1}{3} \times \frac{6}{25} + \frac{2}{3} \times \frac{5}{50}\\
&=\frac{11}{75}
\end{align}

\newpage

%\tableofcontents

\bigskip

\renewcommand{\thefigure}{\theenumi}
\renewcommand{\thetable}{\theenumi}
%\renewcommand{\theequation}{\theenumi}

%\begin{abstract}
%%\boldmath
%In this letter, an algorithm for evaluating the exact analytical bit error rate  (BER)  for the piecewise linear (PL) combiner for  multiple relays is presented. Previous results were available only for upto three relays. The algorithm is unique in the sense that  the actual mathematical expressions, that are prohibitively large, need not be explicitly obtained. The diversity gain due to multiple relays is shown through plots of the analytical BER, well supported by simulations. 
%
%\end{abstract}
% IEEEtran.cls defaults to using nonbold math in the Abstract.
% This preserves the distinction between vectors and scalars. However,
% if the journal you are submitting to favors bold math in the abstract,
% then you can use LaTeX's standard command \boldmath at the very start
% of the abstract to achieve this. Many IEEE journals frown on math
% in the abstract anyway.

% Note that keywords are not normally used for peerreview papers.
%\begin{IEEEkeywords}
%Cooperative diversity, decode and forward, piecewise linear
%\end{IEEEkeywords}



% For peer review papers, you can put extra information on the cover
% page as needed:
% \ifCLASSOPTIONpeerreview
% \begin{center} \bfseries EDICS Category: 3-BBND \end{center}
% \fi
%
% For peerreview papers, this IEEEtran command inserts a page break and
% creates the second title. It will be ignored for other modes.
%\IEEEpeerreviewmaketitle




\item A card is drawn from a deck of 52 cards. Find the probability of getting a king or a heart or a red card.\\
\solution
%\begin{table}[H]
	\centering
\begin{tabular}{|c|c|c|}
\hline
Random variable &Value &Definition\\ \hline
\multirow{3}{*}{X} &0 &Slips of Rs 1\\
&1 &Slips of Rs 5\\
&2 &Slips of Rs 13\\ \hline
\multirow{2}{*}{Y} &0 &Box A\\
&1 &Box B\\\hline
\end{tabular}
\caption{}
\label{tab:Distribution}
\end{table}
See \tabref{tab:Distribution}.
\begin{align}
p_{Y}\brak{k}= \begin{cases} 
      \frac{1}{3} & {k=0} \\
      \frac{2}{3 }& {k=1} 
   \end{cases}
   \\
p_{Y|X}\brak{0|0} = \frac{19}{25}\, 
p_{Y|X}\brak{0|1} = \frac{6}{25}\,
p_{Y|X}\brak{1|0} = \frac{45}{50}\,
p_{Y|X}\brak{1|2} = \frac{5}{50}
\end{align}
The desired probability is the probability that a slip drawn at random is marked other than Rs 1,
\begin{align}
&=1-p_X\brak{0}\\
&= p_X(1) + p_X(2)
\end{align}
Using Bayes theorem,
\begin{align}
&= p_Y\brak{0} \times \pr{Y=0 | X=1} + p_Y\brak{1} \times \pr{Y=1|X=2}\\
&=\frac{1}{3} \times \frac{6}{25} + \frac{2}{3} \times \frac{5}{50}\\
&=\frac{11}{75}
\end{align}

\newpage

%\tableofcontents

\bigskip

\renewcommand{\thefigure}{\theenumi}
\renewcommand{\thetable}{\theenumi}
%\renewcommand{\theequation}{\theenumi}

%\begin{abstract}
%%\boldmath
%In this letter, an algorithm for evaluating the exact analytical bit error rate  (BER)  for the piecewise linear (PL) combiner for  multiple relays is presented. Previous results were available only for upto three relays. The algorithm is unique in the sense that  the actual mathematical expressions, that are prohibitively large, need not be explicitly obtained. The diversity gain due to multiple relays is shown through plots of the analytical BER, well supported by simulations. 
%
%\end{abstract}
% IEEEtran.cls defaults to using nonbold math in the Abstract.
% This preserves the distinction between vectors and scalars. However,
% if the journal you are submitting to favors bold math in the abstract,
% then you can use LaTeX's standard command \boldmath at the very start
% of the abstract to achieve this. Many IEEE journals frown on math
% in the abstract anyway.

% Note that keywords are not normally used for peerreview papers.
%\begin{IEEEkeywords}
%Cooperative diversity, decode and forward, piecewise linear
%\end{IEEEkeywords}



% For peer review papers, you can put extra information on the cover
% page as needed:
% \ifCLASSOPTIONpeerreview
% \begin{center} \bfseries EDICS Category: 3-BBND \end{center}
% \fi
%
% For peerreview papers, this IEEEtran command inserts a page break and
% creates the second title. It will be ignored for other modes.
%\IEEEpeerreviewmaketitle




\item The probability that a student will pass his examination is 0.73, the probability of
the student getting a compartment is 0.13, and the probability that the student will
either pass or get compartment is 0.96. State True or False.\\
\solution
%\begin{table}[H]
	\centering
\begin{tabular}{|c|c|c|}
\hline
Random variable &Value &Definition\\ \hline
\multirow{3}{*}{X} &0 &Slips of Rs 1\\
&1 &Slips of Rs 5\\
&2 &Slips of Rs 13\\ \hline
\multirow{2}{*}{Y} &0 &Box A\\
&1 &Box B\\\hline
\end{tabular}
\caption{}
\label{tab:Distribution}
\end{table}
See \tabref{tab:Distribution}.
\begin{align}
p_{Y}\brak{k}= \begin{cases} 
      \frac{1}{3} & {k=0} \\
      \frac{2}{3 }& {k=1} 
   \end{cases}
   \\
p_{Y|X}\brak{0|0} = \frac{19}{25}\, 
p_{Y|X}\brak{0|1} = \frac{6}{25}\,
p_{Y|X}\brak{1|0} = \frac{45}{50}\,
p_{Y|X}\brak{1|2} = \frac{5}{50}
\end{align}
The desired probability is the probability that a slip drawn at random is marked other than Rs 1,
\begin{align}
&=1-p_X\brak{0}\\
&= p_X(1) + p_X(2)
\end{align}
Using Bayes theorem,
\begin{align}
&= p_Y\brak{0} \times \pr{Y=0 | X=1} + p_Y\brak{1} \times \pr{Y=1|X=2}\\
&=\frac{1}{3} \times \frac{6}{25} + \frac{2}{3} \times \frac{5}{50}\\
&=\frac{11}{75}
\end{align}

\newpage

%\tableofcontents

\bigskip

\renewcommand{\thefigure}{\theenumi}
\renewcommand{\thetable}{\theenumi}
%\renewcommand{\theequation}{\theenumi}

%\begin{abstract}
%%\boldmath
%In this letter, an algorithm for evaluating the exact analytical bit error rate  (BER)  for the piecewise linear (PL) combiner for  multiple relays is presented. Previous results were available only for upto three relays. The algorithm is unique in the sense that  the actual mathematical expressions, that are prohibitively large, need not be explicitly obtained. The diversity gain due to multiple relays is shown through plots of the analytical BER, well supported by simulations. 
%
%\end{abstract}
% IEEEtran.cls defaults to using nonbold math in the Abstract.
% This preserves the distinction between vectors and scalars. However,
% if the journal you are submitting to favors bold math in the abstract,
% then you can use LaTeX's standard command \boldmath at the very start
% of the abstract to achieve this. Many IEEE journals frown on math
% in the abstract anyway.

% Note that keywords are not normally used for peerreview papers.
%\begin{IEEEkeywords}
%Cooperative diversity, decode and forward, piecewise linear
%\end{IEEEkeywords}



% For peer review papers, you can put extra information on the cover
% page as needed:
% \ifCLASSOPTIONpeerreview
% \begin{center} \bfseries EDICS Category: 3-BBND \end{center}
% \fi
%
% For peerreview papers, this IEEEtran command inserts a page break and
% creates the second title. It will be ignored for other modes.
%\IEEEpeerreviewmaketitle




\item A card is selected from a pack of 52 cards\\
\begin{enumerate}[label=(\alph*)]
\item How many points are there in the sample space?
\item Calculate the probability that the cards is an ace of spades.
\item Calculate the probability that the card is (i) an ace (ii)black card.\\
\end{enumerate}
%\input{ncert/11/16/3/4_1/Prob_4.tex}
\item In a non-leap year, the probability of having 53 tuesdays or 53 wednesdays is\\
\solution
%A non-leap year has a total of 365 days, and a week has 7 days.\\
So it can be expressed as 
\begin{align}
365\text{days} &=52\times 7+1 \text{day}
\end{align}
$\implies$ 52 tuesdays or wednesdays\\
Random variable X denotes the days of a week
\begin{align}
p_X\brak{k}&=\frac{1}{7}; \quad \brak{1<k<7}
\end{align}
So the probability of extra day being tuesday or wednesday is
\begin{align}
p_X\brak{3}+p_X\brak{4}&=\frac{1}{7}+\frac{1}{7}=\frac{2}{7}
\end{align}



\item There are 1000 sealed envelopes in a box, 10 of them contain a cash prize of
Rs 100 each, 100 of them contain a cash prize of Rs 50 each and 200 of them
contain a cash prize of Rs 10 each and rest do not contain any cash prize. If they
are well shuffled and an envelope is picked up out, what is the probability that it
contains no cash prize?\\
\solution
%\begin{table}[H]
	\centering
\begin{tabular}{|c|c|c|}
\hline
Random variable &Value &Definition\\ \hline
\multirow{3}{*}{X} &0 &Slips of Rs 1\\
&1 &Slips of Rs 5\\
&2 &Slips of Rs 13\\ \hline
\multirow{2}{*}{Y} &0 &Box A\\
&1 &Box B\\\hline
\end{tabular}
\caption{}
\label{tab:Distribution}
\end{table}
See \tabref{tab:Distribution}.
\begin{align}
p_{Y}\brak{k}= \begin{cases} 
      \frac{1}{3} & {k=0} \\
      \frac{2}{3 }& {k=1} 
   \end{cases}
   \\
p_{Y|X}\brak{0|0} = \frac{19}{25}\, 
p_{Y|X}\brak{0|1} = \frac{6}{25}\,
p_{Y|X}\brak{1|0} = \frac{45}{50}\,
p_{Y|X}\brak{1|2} = \frac{5}{50}
\end{align}
The desired probability is the probability that a slip drawn at random is marked other than Rs 1,
\begin{align}
&=1-p_X\brak{0}\\
&= p_X(1) + p_X(2)
\end{align}
Using Bayes theorem,
\begin{align}
&= p_Y\brak{0} \times \pr{Y=0 | X=1} + p_Y\brak{1} \times \pr{Y=1|X=2}\\
&=\frac{1}{3} \times \frac{6}{25} + \frac{2}{3} \times \frac{5}{50}\\
&=\frac{11}{75}
\end{align}

\newpage

%\tableofcontents

\bigskip

\renewcommand{\thefigure}{\theenumi}
\renewcommand{\thetable}{\theenumi}
%\renewcommand{\theequation}{\theenumi}

%\begin{abstract}
%%\boldmath
%In this letter, an algorithm for evaluating the exact analytical bit error rate  (BER)  for the piecewise linear (PL) combiner for  multiple relays is presented. Previous results were available only for upto three relays. The algorithm is unique in the sense that  the actual mathematical expressions, that are prohibitively large, need not be explicitly obtained. The diversity gain due to multiple relays is shown through plots of the analytical BER, well supported by simulations. 
%
%\end{abstract}
% IEEEtran.cls defaults to using nonbold math in the Abstract.
% This preserves the distinction between vectors and scalars. However,
% if the journal you are submitting to favors bold math in the abstract,
% then you can use LaTeX's standard command \boldmath at the very start
% of the abstract to achieve this. Many IEEE journals frown on math
% in the abstract anyway.

% Note that keywords are not normally used for peerreview papers.
%\begin{IEEEkeywords}
%Cooperative diversity, decode and forward, piecewise linear
%\end{IEEEkeywords}



% For peer review papers, you can put extra information on the cover
% page as needed:
% \ifCLASSOPTIONpeerreview
% \begin{center} \bfseries EDICS Category: 3-BBND \end{center}
% \fi
%
% For peerreview papers, this IEEEtran command inserts a page break and
% creates the second title. It will be ignored for other modes.
%\IEEEpeerreviewmaketitle




\item 
A die is thrown and a card is selected at random from a deck of 52 playing cards. The probability of getting an even number on the die and a spade card.\\
\solution
%\begin{table}[H]
	\centering
\begin{tabular}{|c|c|c|}
\hline
Random variable &Value &Definition\\ \hline
\multirow{3}{*}{X} &0 &Slips of Rs 1\\
&1 &Slips of Rs 5\\
&2 &Slips of Rs 13\\ \hline
\multirow{2}{*}{Y} &0 &Box A\\
&1 &Box B\\\hline
\end{tabular}
\caption{}
\label{tab:Distribution}
\end{table}
See \tabref{tab:Distribution}.
\begin{align}
p_{Y}\brak{k}= \begin{cases} 
      \frac{1}{3} & {k=0} \\
      \frac{2}{3 }& {k=1} 
   \end{cases}
   \\
p_{Y|X}\brak{0|0} = \frac{19}{25}\, 
p_{Y|X}\brak{0|1} = \frac{6}{25}\,
p_{Y|X}\brak{1|0} = \frac{45}{50}\,
p_{Y|X}\brak{1|2} = \frac{5}{50}
\end{align}
The desired probability is the probability that a slip drawn at random is marked other than Rs 1,
\begin{align}
&=1-p_X\brak{0}\\
&= p_X(1) + p_X(2)
\end{align}
Using Bayes theorem,
\begin{align}
&= p_Y\brak{0} \times \pr{Y=0 | X=1} + p_Y\brak{1} \times \pr{Y=1|X=2}\\
&=\frac{1}{3} \times \frac{6}{25} + \frac{2}{3} \times \frac{5}{50}\\
&=\frac{11}{75}
\end{align}

\newpage

%\tableofcontents

\bigskip

\renewcommand{\thefigure}{\theenumi}
\renewcommand{\thetable}{\theenumi}
%\renewcommand{\theequation}{\theenumi}

%\begin{abstract}
%%\boldmath
%In this letter, an algorithm for evaluating the exact analytical bit error rate  (BER)  for the piecewise linear (PL) combiner for  multiple relays is presented. Previous results were available only for upto three relays. The algorithm is unique in the sense that  the actual mathematical expressions, that are prohibitively large, need not be explicitly obtained. The diversity gain due to multiple relays is shown through plots of the analytical BER, well supported by simulations. 
%
%\end{abstract}
% IEEEtran.cls defaults to using nonbold math in the Abstract.
% This preserves the distinction between vectors and scalars. However,
% if the journal you are submitting to favors bold math in the abstract,
% then you can use LaTeX's standard command \boldmath at the very start
% of the abstract to achieve this. Many IEEE journals frown on math
% in the abstract anyway.

% Note that keywords are not normally used for peerreview papers.
%\begin{IEEEkeywords}
%Cooperative diversity, decode and forward, piecewise linear
%\end{IEEEkeywords}



% For peer review papers, you can put extra information on the cover
% page as needed:
% \ifCLASSOPTIONpeerreview
% \begin{center} \bfseries EDICS Category: 3-BBND \end{center}
% \fi
%
% For peerreview papers, this IEEEtran command inserts a page break and
% creates the second title. It will be ignored for other modes.
%\IEEEpeerreviewmaketitle




\item
If 4-digit numbers greater than 5,000 are randomly formed from the digits 0, 1, 3, 5, and 7, what is the probability of forming a number divisible by 5 when:
\begin{enumerate}
    \item The digits are repeated?
    \item The repetition of digits is not allowed?
\end{enumerate}
\solution
%\begin{table}[H]
	\centering
\begin{tabular}{|c|c|c|}
\hline
Random variable &Value &Definition\\ \hline
\multirow{3}{*}{X} &0 &Slips of Rs 1\\
&1 &Slips of Rs 5\\
&2 &Slips of Rs 13\\ \hline
\multirow{2}{*}{Y} &0 &Box A\\
&1 &Box B\\\hline
\end{tabular}
\caption{}
\label{tab:Distribution}
\end{table}
See \tabref{tab:Distribution}.
\begin{align}
p_{Y}\brak{k}= \begin{cases} 
      \frac{1}{3} & {k=0} \\
      \frac{2}{3 }& {k=1} 
   \end{cases}
   \\
p_{Y|X}\brak{0|0} = \frac{19}{25}\, 
p_{Y|X}\brak{0|1} = \frac{6}{25}\,
p_{Y|X}\brak{1|0} = \frac{45}{50}\,
p_{Y|X}\brak{1|2} = \frac{5}{50}
\end{align}
The desired probability is the probability that a slip drawn at random is marked other than Rs 1,
\begin{align}
&=1-p_X\brak{0}\\
&= p_X(1) + p_X(2)
\end{align}
Using Bayes theorem,
\begin{align}
&= p_Y\brak{0} \times \pr{Y=0 | X=1} + p_Y\brak{1} \times \pr{Y=1|X=2}\\
&=\frac{1}{3} \times \frac{6}{25} + \frac{2}{3} \times \frac{5}{50}\\
&=\frac{11}{75}
\end{align}

\newpage

%\tableofcontents

\bigskip

\renewcommand{\thefigure}{\theenumi}
\renewcommand{\thetable}{\theenumi}
%\renewcommand{\theequation}{\theenumi}

%\begin{abstract}
%%\boldmath
%In this letter, an algorithm for evaluating the exact analytical bit error rate  (BER)  for the piecewise linear (PL) combiner for  multiple relays is presented. Previous results were available only for upto three relays. The algorithm is unique in the sense that  the actual mathematical expressions, that are prohibitively large, need not be explicitly obtained. The diversity gain due to multiple relays is shown through plots of the analytical BER, well supported by simulations. 
%
%\end{abstract}
% IEEEtran.cls defaults to using nonbold math in the Abstract.
% This preserves the distinction between vectors and scalars. However,
% if the journal you are submitting to favors bold math in the abstract,
% then you can use LaTeX's standard command \boldmath at the very start
% of the abstract to achieve this. Many IEEE journals frown on math
% in the abstract anyway.

% Note that keywords are not normally used for peerreview papers.
%\begin{IEEEkeywords}
%Cooperative diversity, decode and forward, piecewise linear
%\end{IEEEkeywords}



% For peer review papers, you can put extra information on the cover
% page as needed:
% \ifCLASSOPTIONpeerreview
% \begin{center} \bfseries EDICS Category: 3-BBND \end{center}
% \fi
%
% For peerreview papers, this IEEEtran command inserts a page break and
% creates the second title. It will be ignored for other modes.
%\IEEEpeerreviewmaketitle




\item Consider the probability space $\brak{\Omega, \mathcal{G}, P}$ where $\Omega = [0,2]$ and $\mathcal{G} = \cbrak{\phi, \Omega, [0,1], (1,2]}$. Let $X$ and $Y$ be two functions on $\Omega$ defined as
\begin{align*}
    X(\omega) = 
    \begin{cases}
        1 & \text{if }\omega \in [0, 1]\\
        2 & \text{if }\omega \in (1, 2]
    \end{cases}
\end{align*}
and
\begin{align*}
    Y(\omega) = 
    \begin{cases}
        2 & \text{if }\omega \in [0, 1.5]\\
        3 & \text{if }\omega \in (1.5, 2].
    \end{cases}
\end{align*}
Then which one of the following statements is true?
\begin{enumerate}
    \item [(A)] $X$ is a random variable with respect to $\mathcal{G}$, but $Y$ is not a random variable with respect to $\mathcal{G}$.
    \item [(B)] $Y$ is a random variable with respect to $\mathcal{G}$, but $X$ is not a random variable with respect to $\mathcal{G}$.
    \item [(C)] Neither $X$ nor $Y$ is a random variable with respect to $\mathcal{G}$.
    \item [(D)] Both $X$ and $Y$ are random variables with respect to $\mathcal{G}$.
\end{enumerate} \hfill (GATE ST 2023)\\
\solution
%\begin{table}[H]
	\centering
\begin{tabular}{|c|c|c|}
\hline
Random variable &Value &Definition\\ \hline
\multirow{3}{*}{X} &0 &Slips of Rs 1\\
&1 &Slips of Rs 5\\
&2 &Slips of Rs 13\\ \hline
\multirow{2}{*}{Y} &0 &Box A\\
&1 &Box B\\\hline
\end{tabular}
\caption{}
\label{tab:Distribution}
\end{table}
See \tabref{tab:Distribution}.
\begin{align}
p_{Y}\brak{k}= \begin{cases} 
      \frac{1}{3} & {k=0} \\
      \frac{2}{3 }& {k=1} 
   \end{cases}
   \\
p_{Y|X}\brak{0|0} = \frac{19}{25}\, 
p_{Y|X}\brak{0|1} = \frac{6}{25}\,
p_{Y|X}\brak{1|0} = \frac{45}{50}\,
p_{Y|X}\brak{1|2} = \frac{5}{50}
\end{align}
The desired probability is the probability that a slip drawn at random is marked other than Rs 1,
\begin{align}
&=1-p_X\brak{0}\\
&= p_X(1) + p_X(2)
\end{align}
Using Bayes theorem,
\begin{align}
&= p_Y\brak{0} \times \pr{Y=0 | X=1} + p_Y\brak{1} \times \pr{Y=1|X=2}\\
&=\frac{1}{3} \times \frac{6}{25} + \frac{2}{3} \times \frac{5}{50}\\
&=\frac{11}{75}
\end{align}

\newpage

%\tableofcontents

\bigskip

\renewcommand{\thefigure}{\theenumi}
\renewcommand{\thetable}{\theenumi}
%\renewcommand{\theequation}{\theenumi}

%\begin{abstract}
%%\boldmath
%In this letter, an algorithm for evaluating the exact analytical bit error rate  (BER)  for the piecewise linear (PL) combiner for  multiple relays is presented. Previous results were available only for upto three relays. The algorithm is unique in the sense that  the actual mathematical expressions, that are prohibitively large, need not be explicitly obtained. The diversity gain due to multiple relays is shown through plots of the analytical BER, well supported by simulations. 
%
%\end{abstract}
% IEEEtran.cls defaults to using nonbold math in the Abstract.
% This preserves the distinction between vectors and scalars. However,
% if the journal you are submitting to favors bold math in the abstract,
% then you can use LaTeX's standard command \boldmath at the very start
% of the abstract to achieve this. Many IEEE journals frown on math
% in the abstract anyway.

% Note that keywords are not normally used for peerreview papers.
%\begin{IEEEkeywords}
%Cooperative diversity, decode and forward, piecewise linear
%\end{IEEEkeywords}



% For peer review papers, you can put extra information on the cover
% page as needed:
% \ifCLASSOPTIONpeerreview
% \begin{center} \bfseries EDICS Category: 3-BBND \end{center}
% \fi
%
% For peerreview papers, this IEEEtran command inserts a page break and
% creates the second title. It will be ignored for other modes.
%\IEEEpeerreviewmaketitle




	\item  A die is loaded in such a way that each odd number is twice as likely to occur as
each even number. Find $P(G)$, where $G$ is the event that a number greater than
3 occurs on a single roll of the die.
\\
\solution
		%\begin{table}[H]
	\centering
\begin{tabular}{|c|c|c|}
\hline
Random variable &Value &Definition\\ \hline
\multirow{3}{*}{X} &0 &Slips of Rs 1\\
&1 &Slips of Rs 5\\
&2 &Slips of Rs 13\\ \hline
\multirow{2}{*}{Y} &0 &Box A\\
&1 &Box B\\\hline
\end{tabular}
\caption{}
\label{tab:Distribution}
\end{table}
See \tabref{tab:Distribution}.
\begin{align}
p_{Y}\brak{k}= \begin{cases} 
      \frac{1}{3} & {k=0} \\
      \frac{2}{3 }& {k=1} 
   \end{cases}
   \\
p_{Y|X}\brak{0|0} = \frac{19}{25}\, 
p_{Y|X}\brak{0|1} = \frac{6}{25}\,
p_{Y|X}\brak{1|0} = \frac{45}{50}\,
p_{Y|X}\brak{1|2} = \frac{5}{50}
\end{align}
The desired probability is the probability that a slip drawn at random is marked other than Rs 1,
\begin{align}
&=1-p_X\brak{0}\\
&= p_X(1) + p_X(2)
\end{align}
Using Bayes theorem,
\begin{align}
&= p_Y\brak{0} \times \pr{Y=0 | X=1} + p_Y\brak{1} \times \pr{Y=1|X=2}\\
&=\frac{1}{3} \times \frac{6}{25} + \frac{2}{3} \times \frac{5}{50}\\
&=\frac{11}{75}
\end{align}

\newpage

%\tableofcontents

\bigskip

\renewcommand{\thefigure}{\theenumi}
\renewcommand{\thetable}{\theenumi}
%\renewcommand{\theequation}{\theenumi}

%\begin{abstract}
%%\boldmath
%In this letter, an algorithm for evaluating the exact analytical bit error rate  (BER)  for the piecewise linear (PL) combiner for  multiple relays is presented. Previous results were available only for upto three relays. The algorithm is unique in the sense that  the actual mathematical expressions, that are prohibitively large, need not be explicitly obtained. The diversity gain due to multiple relays is shown through plots of the analytical BER, well supported by simulations. 
%
%\end{abstract}
% IEEEtran.cls defaults to using nonbold math in the Abstract.
% This preserves the distinction between vectors and scalars. However,
% if the journal you are submitting to favors bold math in the abstract,
% then you can use LaTeX's standard command \boldmath at the very start
% of the abstract to achieve this. Many IEEE journals frown on math
% in the abstract anyway.

% Note that keywords are not normally used for peerreview papers.
%\begin{IEEEkeywords}
%Cooperative diversity, decode and forward, piecewise linear
%\end{IEEEkeywords}



% For peer review papers, you can put extra information on the cover
% page as needed:
% \ifCLASSOPTIONpeerreview
% \begin{center} \bfseries EDICS Category: 3-BBND \end{center}
% \fi
%
% For peerreview papers, this IEEEtran command inserts a page break and
% creates the second title. It will be ignored for other modes.
%\IEEEpeerreviewmaketitle




	\item All the jacks, queens and kings are removed from a deck of 52 playing cards. The remaining cards are well shuffled and then one card is drawn at random. Giving ace a value 1 similar value for other cards, find the probability that the card has a value 
		\begin{enumerate}
			\item 7
			\item greater than 7
			\item less than 7
		\end{enumerate}
		%Number of cards left after removing all jacks, queens and kings 
\begin{align}
N	= 52 - 4\times 3
	= 40
\end{align}
%\begin{table}[H]
%\def\arraystretch{1.2}
%\begin{tabular}{|c|c|c|}
%\hline
%	\textbf{Parameter} &\textbf{Value} &\textbf{Description}\\ \hline
%	$X$ &1-10 &Represents the value of the card picked \\ \hline
%\end{tabular}
%\end{table}
Let $1 \le X \le 10$ be the value of the card picked.  Then,
\begin{align}
	p_X(k) &= \Pr(X=k)\ \forall\ 1 \leq k \leq 10\\
	&= \frac{4\times 1}{40}\\
	&= \frac{1}{10}\\
	\therefore p_X(k) &= 
	\begin{cases}
		\frac{1}{10} & 1 \leq k \leq 10\\
		0 & \text{otherwise}
	\end{cases}
\end{align}
and
\begin{align}
	F_{X}(k) &= \sum_{m=0}^{k}p_{X}(m) \quad 1 \leq k \leq 10\\
	&= \frac{k}{10}\\
	\therefore F_{X}(k) &= 
	\begin{cases}
		0 & k \leq 0\\
		\frac{k}{10} & 1\leq k \leq 10\\
		1 & k > 10 
	\end{cases}
\end{align}
\begin{enumerate}
	\item Probability that card has value equal to 7 is
		\begin{align}
			 p_{X}(7)
			= \frac{1}{10}
		\end{align}
	\item Probability that card has value greater than 7 is
		\begin{align}
			1 - F_X(7)
			&= 1 - \frac{7}{10}
			\\
			&= \frac{3}{10}
		\end{align}
	\item Probability that card has value less than 7 is
		\begin{align}
			 F_{X}(6)
			=\frac{6}{10}
		\end{align}
\end{enumerate}

  \item A Lot consists of 48 mobile phones of which 42 are good, 3 have only minor defects and 3 have major defects.Varnika will buy a phone if it is good but the trader will only buy a mobile if it has no major defects. One phone is selected at random from the lot. What is the probability that it is
\begin{enumerate}
	\item acceptable to Varnika?
            \item acceptable to the trader?
\end{enumerate}
\solution
	%\begin{table}[H]
	\centering
\begin{tabular}{|c|c|c|}
\hline
Random variable &Value &Definition\\ \hline
\multirow{3}{*}{X} &0 &Slips of Rs 1\\
&1 &Slips of Rs 5\\
&2 &Slips of Rs 13\\ \hline
\multirow{2}{*}{Y} &0 &Box A\\
&1 &Box B\\\hline
\end{tabular}
\caption{}
\label{tab:Distribution}
\end{table}
See \tabref{tab:Distribution}.
\begin{align}
p_{Y}\brak{k}= \begin{cases} 
      \frac{1}{3} & {k=0} \\
      \frac{2}{3 }& {k=1} 
   \end{cases}
   \\
p_{Y|X}\brak{0|0} = \frac{19}{25}\, 
p_{Y|X}\brak{0|1} = \frac{6}{25}\,
p_{Y|X}\brak{1|0} = \frac{45}{50}\,
p_{Y|X}\brak{1|2} = \frac{5}{50}
\end{align}
The desired probability is the probability that a slip drawn at random is marked other than Rs 1,
\begin{align}
&=1-p_X\brak{0}\\
&= p_X(1) + p_X(2)
\end{align}
Using Bayes theorem,
\begin{align}
&= p_Y\brak{0} \times \pr{Y=0 | X=1} + p_Y\brak{1} \times \pr{Y=1|X=2}\\
&=\frac{1}{3} \times \frac{6}{25} + \frac{2}{3} \times \frac{5}{50}\\
&=\frac{11}{75}
\end{align}

\newpage

%\tableofcontents

\bigskip

\renewcommand{\thefigure}{\theenumi}
\renewcommand{\thetable}{\theenumi}
%\renewcommand{\theequation}{\theenumi}

%\begin{abstract}
%%\boldmath
%In this letter, an algorithm for evaluating the exact analytical bit error rate  (BER)  for the piecewise linear (PL) combiner for  multiple relays is presented. Previous results were available only for upto three relays. The algorithm is unique in the sense that  the actual mathematical expressions, that are prohibitively large, need not be explicitly obtained. The diversity gain due to multiple relays is shown through plots of the analytical BER, well supported by simulations. 
%
%\end{abstract}
% IEEEtran.cls defaults to using nonbold math in the Abstract.
% This preserves the distinction between vectors and scalars. However,
% if the journal you are submitting to favors bold math in the abstract,
% then you can use LaTeX's standard command \boldmath at the very start
% of the abstract to achieve this. Many IEEE journals frown on math
% in the abstract anyway.

% Note that keywords are not normally used for peerreview papers.
%\begin{IEEEkeywords}
%Cooperative diversity, decode and forward, piecewise linear
%\end{IEEEkeywords}



% For peer review papers, you can put extra information on the cover
% page as needed:
% \ifCLASSOPTIONpeerreview
% \begin{center} \bfseries EDICS Category: 3-BBND \end{center}
% \fi
%
% For peerreview papers, this IEEEtran command inserts a page break and
% creates the second title. It will be ignored for other modes.
%\IEEEpeerreviewmaketitle




 \item A student says that if you throw a die, it will show up 1 or not 1. Therefore, the probability of getting 1 and the probability of getting 'not 1' each is equal to $\frac{1}{2}$. Is this correct? Give reasons.\\
 \solution
        %\begin{table}[H]
	\centering
\begin{tabular}{|c|c|c|}
\hline
Random variable &Value &Definition\\ \hline
\multirow{3}{*}{X} &0 &Slips of Rs 1\\
&1 &Slips of Rs 5\\
&2 &Slips of Rs 13\\ \hline
\multirow{2}{*}{Y} &0 &Box A\\
&1 &Box B\\\hline
\end{tabular}
\caption{}
\label{tab:Distribution}
\end{table}
See \tabref{tab:Distribution}.
\begin{align}
p_{Y}\brak{k}= \begin{cases} 
      \frac{1}{3} & {k=0} \\
      \frac{2}{3 }& {k=1} 
   \end{cases}
   \\
p_{Y|X}\brak{0|0} = \frac{19}{25}\, 
p_{Y|X}\brak{0|1} = \frac{6}{25}\,
p_{Y|X}\brak{1|0} = \frac{45}{50}\,
p_{Y|X}\brak{1|2} = \frac{5}{50}
\end{align}
The desired probability is the probability that a slip drawn at random is marked other than Rs 1,
\begin{align}
&=1-p_X\brak{0}\\
&= p_X(1) + p_X(2)
\end{align}
Using Bayes theorem,
\begin{align}
&= p_Y\brak{0} \times \pr{Y=0 | X=1} + p_Y\brak{1} \times \pr{Y=1|X=2}\\
&=\frac{1}{3} \times \frac{6}{25} + \frac{2}{3} \times \frac{5}{50}\\
&=\frac{11}{75}
\end{align}

\newpage

%\tableofcontents

\bigskip

\renewcommand{\thefigure}{\theenumi}
\renewcommand{\thetable}{\theenumi}
%\renewcommand{\theequation}{\theenumi}

%\begin{abstract}
%%\boldmath
%In this letter, an algorithm for evaluating the exact analytical bit error rate  (BER)  for the piecewise linear (PL) combiner for  multiple relays is presented. Previous results were available only for upto three relays. The algorithm is unique in the sense that  the actual mathematical expressions, that are prohibitively large, need not be explicitly obtained. The diversity gain due to multiple relays is shown through plots of the analytical BER, well supported by simulations. 
%
%\end{abstract}
% IEEEtran.cls defaults to using nonbold math in the Abstract.
% This preserves the distinction between vectors and scalars. However,
% if the journal you are submitting to favors bold math in the abstract,
% then you can use LaTeX's standard command \boldmath at the very start
% of the abstract to achieve this. Many IEEE journals frown on math
% in the abstract anyway.

% Note that keywords are not normally used for peerreview papers.
%\begin{IEEEkeywords}
%Cooperative diversity, decode and forward, piecewise linear
%\end{IEEEkeywords}



% For peer review papers, you can put extra information on the cover
% page as needed:
% \ifCLASSOPTIONpeerreview
% \begin{center} \bfseries EDICS Category: 3-BBND \end{center}
% \fi
%
% For peerreview papers, this IEEEtran command inserts a page break and
% creates the second title. It will be ignored for other modes.
%\IEEEpeerreviewmaketitle




   \item Four candidates A, B, C, D have ap-
plied for the assignment to coach a school cricket
team. If A is twice as likely to be selected as B, and
B and C are given about the same chance of being
selected, while C is twice as likely to be selected
as D, what are the probabilities that
\begin{enumerate}
\item C will be selected?
\item A will not be selected?
\end{enumerate}
	%\begin{table}[H]
	\centering
\begin{tabular}{|c|c|c|}
\hline
Random variable &Value &Definition\\ \hline
\multirow{3}{*}{X} &0 &Slips of Rs 1\\
&1 &Slips of Rs 5\\
&2 &Slips of Rs 13\\ \hline
\multirow{2}{*}{Y} &0 &Box A\\
&1 &Box B\\\hline
\end{tabular}
\caption{}
\label{tab:Distribution}
\end{table}
See \tabref{tab:Distribution}.
\begin{align}
p_{Y}\brak{k}= \begin{cases} 
      \frac{1}{3} & {k=0} \\
      \frac{2}{3 }& {k=1} 
   \end{cases}
   \\
p_{Y|X}\brak{0|0} = \frac{19}{25}\, 
p_{Y|X}\brak{0|1} = \frac{6}{25}\,
p_{Y|X}\brak{1|0} = \frac{45}{50}\,
p_{Y|X}\brak{1|2} = \frac{5}{50}
\end{align}
The desired probability is the probability that a slip drawn at random is marked other than Rs 1,
\begin{align}
&=1-p_X\brak{0}\\
&= p_X(1) + p_X(2)
\end{align}
Using Bayes theorem,
\begin{align}
&= p_Y\brak{0} \times \pr{Y=0 | X=1} + p_Y\brak{1} \times \pr{Y=1|X=2}\\
&=\frac{1}{3} \times \frac{6}{25} + \frac{2}{3} \times \frac{5}{50}\\
&=\frac{11}{75}
\end{align}

\newpage

%\tableofcontents

\bigskip

\renewcommand{\thefigure}{\theenumi}
\renewcommand{\thetable}{\theenumi}
%\renewcommand{\theequation}{\theenumi}

%\begin{abstract}
%%\boldmath
%In this letter, an algorithm for evaluating the exact analytical bit error rate  (BER)  for the piecewise linear (PL) combiner for  multiple relays is presented. Previous results were available only for upto three relays. The algorithm is unique in the sense that  the actual mathematical expressions, that are prohibitively large, need not be explicitly obtained. The diversity gain due to multiple relays is shown through plots of the analytical BER, well supported by simulations. 
%
%\end{abstract}
% IEEEtran.cls defaults to using nonbold math in the Abstract.
% This preserves the distinction between vectors and scalars. However,
% if the journal you are submitting to favors bold math in the abstract,
% then you can use LaTeX's standard command \boldmath at the very start
% of the abstract to achieve this. Many IEEE journals frown on math
% in the abstract anyway.

% Note that keywords are not normally used for peerreview papers.
%\begin{IEEEkeywords}
%Cooperative diversity, decode and forward, piecewise linear
%\end{IEEEkeywords}



% For peer review papers, you can put extra information on the cover
% page as needed:
% \ifCLASSOPTIONpeerreview
% \begin{center} \bfseries EDICS Category: 3-BBND \end{center}
% \fi
%
% For peerreview papers, this IEEEtran command inserts a page break and
% creates the second title. It will be ignored for other modes.
%\IEEEpeerreviewmaketitle




 \item A bag contain 24 balls of which $x$ balls are red, $2x$ are white and $3x$ are blue. A ball is selected at random, What is the probability that it is
\begin{enumerate}[label=\alph*)]
\item not red ?
\item white ?
\end{enumerate}
%\begin{table}[H]
	\centering
\begin{tabular}{|c|c|c|}
\hline
Random variable &Value &Definition\\ \hline
\multirow{3}{*}{X} &0 &Slips of Rs 1\\
&1 &Slips of Rs 5\\
&2 &Slips of Rs 13\\ \hline
\multirow{2}{*}{Y} &0 &Box A\\
&1 &Box B\\\hline
\end{tabular}
\caption{}
\label{tab:Distribution}
\end{table}
See \tabref{tab:Distribution}.
\begin{align}
p_{Y}\brak{k}= \begin{cases} 
      \frac{1}{3} & {k=0} \\
      \frac{2}{3 }& {k=1} 
   \end{cases}
   \\
p_{Y|X}\brak{0|0} = \frac{19}{25}\, 
p_{Y|X}\brak{0|1} = \frac{6}{25}\,
p_{Y|X}\brak{1|0} = \frac{45}{50}\,
p_{Y|X}\brak{1|2} = \frac{5}{50}
\end{align}
The desired probability is the probability that a slip drawn at random is marked other than Rs 1,
\begin{align}
&=1-p_X\brak{0}\\
&= p_X(1) + p_X(2)
\end{align}
Using Bayes theorem,
\begin{align}
&= p_Y\brak{0} \times \pr{Y=0 | X=1} + p_Y\brak{1} \times \pr{Y=1|X=2}\\
&=\frac{1}{3} \times \frac{6}{25} + \frac{2}{3} \times \frac{5}{50}\\
&=\frac{11}{75}
\end{align}

\newpage

%\tableofcontents

\bigskip

\renewcommand{\thefigure}{\theenumi}
\renewcommand{\thetable}{\theenumi}
%\renewcommand{\theequation}{\theenumi}

%\begin{abstract}
%%\boldmath
%In this letter, an algorithm for evaluating the exact analytical bit error rate  (BER)  for the piecewise linear (PL) combiner for  multiple relays is presented. Previous results were available only for upto three relays. The algorithm is unique in the sense that  the actual mathematical expressions, that are prohibitively large, need not be explicitly obtained. The diversity gain due to multiple relays is shown through plots of the analytical BER, well supported by simulations. 
%
%\end{abstract}
% IEEEtran.cls defaults to using nonbold math in the Abstract.
% This preserves the distinction between vectors and scalars. However,
% if the journal you are submitting to favors bold math in the abstract,
% then you can use LaTeX's standard command \boldmath at the very start
% of the abstract to achieve this. Many IEEE journals frown on math
% in the abstract anyway.

% Note that keywords are not normally used for peerreview papers.
%\begin{IEEEkeywords}
%Cooperative diversity, decode and forward, piecewise linear
%\end{IEEEkeywords}



% For peer review papers, you can put extra information on the cover
% page as needed:
% \ifCLASSOPTIONpeerreview
% \begin{center} \bfseries EDICS Category: 3-BBND \end{center}
% \fi
%
% For peerreview papers, this IEEEtran command inserts a page break and
% creates the second title. It will be ignored for other modes.
%\IEEEpeerreviewmaketitle




If the letters of the word ASSASSINATION are arranged at random. Find the Probability that
\begin{enumerate}[label=(\alph*)]
\item Four $S's$ come consecutively in the word
\item Two  $I's$ and two $N's$ come together
\item All $A's$ are not coming together
\item No two $A's$ are coming together
\end{enumerate}
%\begin{table}[H]
	\centering
\begin{tabular}{|c|c|c|}
\hline
Random variable &Value &Definition\\ \hline
\multirow{3}{*}{X} &0 &Slips of Rs 1\\
&1 &Slips of Rs 5\\
&2 &Slips of Rs 13\\ \hline
\multirow{2}{*}{Y} &0 &Box A\\
&1 &Box B\\\hline
\end{tabular}
\caption{}
\label{tab:Distribution}
\end{table}
See \tabref{tab:Distribution}.
\begin{align}
p_{Y}\brak{k}= \begin{cases} 
      \frac{1}{3} & {k=0} \\
      \frac{2}{3 }& {k=1} 
   \end{cases}
   \\
p_{Y|X}\brak{0|0} = \frac{19}{25}\, 
p_{Y|X}\brak{0|1} = \frac{6}{25}\,
p_{Y|X}\brak{1|0} = \frac{45}{50}\,
p_{Y|X}\brak{1|2} = \frac{5}{50}
\end{align}
The desired probability is the probability that a slip drawn at random is marked other than Rs 1,
\begin{align}
&=1-p_X\brak{0}\\
&= p_X(1) + p_X(2)
\end{align}
Using Bayes theorem,
\begin{align}
&= p_Y\brak{0} \times \pr{Y=0 | X=1} + p_Y\brak{1} \times \pr{Y=1|X=2}\\
&=\frac{1}{3} \times \frac{6}{25} + \frac{2}{3} \times \frac{5}{50}\\
&=\frac{11}{75}
\end{align}

\newpage

%\tableofcontents

\bigskip

\renewcommand{\thefigure}{\theenumi}
\renewcommand{\thetable}{\theenumi}
%\renewcommand{\theequation}{\theenumi}

%\begin{abstract}
%%\boldmath
%In this letter, an algorithm for evaluating the exact analytical bit error rate  (BER)  for the piecewise linear (PL) combiner for  multiple relays is presented. Previous results were available only for upto three relays. The algorithm is unique in the sense that  the actual mathematical expressions, that are prohibitively large, need not be explicitly obtained. The diversity gain due to multiple relays is shown through plots of the analytical BER, well supported by simulations. 
%
%\end{abstract}
% IEEEtran.cls defaults to using nonbold math in the Abstract.
% This preserves the distinction between vectors and scalars. However,
% if the journal you are submitting to favors bold math in the abstract,
% then you can use LaTeX's standard command \boldmath at the very start
% of the abstract to achieve this. Many IEEE journals frown on math
% in the abstract anyway.

% Note that keywords are not normally used for peerreview papers.
%\begin{IEEEkeywords}
%Cooperative diversity, decode and forward, piecewise linear
%\end{IEEEkeywords}



% For peer review papers, you can put extra information on the cover
% page as needed:
% \ifCLASSOPTIONpeerreview
% \begin{center} \bfseries EDICS Category: 3-BBND \end{center}
% \fi
%
% For peerreview papers, this IEEEtran command inserts a page break and
% creates the second title. It will be ignored for other modes.
%\IEEEpeerreviewmaketitle




	\item One urn contains two black balls (labelled B1 and B2) and one white ball. A
	second urn contains one black ball and two white balls (labelled W1 and W2).
	Suppose the following experiment is performed. One of the two urns is chosen
	at random. Next a ball is randomly chosen from the urn. Then a second ball is
	chosen at random from the same urn without replacing the first ball.
	
	\begin{enumerate}
	\item What is the probability that two black balls are chosen?
	
	\item What is the probability that two balls of opposite colour are chosen?
	\end{enumerate}
	\solution
	%\begin{align}
    \label{eq:12.13.6.18.1}
	\because	\pr{A|B} &> \pr{A},\
\frac{\pr{AB}}{\pr{B}} > \pr{A}
\\
    \label{eq:12.13.6.18.2}
	\implies \pr{AB} &> \pr{A}\pr{B}
	\\
	\text{or, } \frac{\pr{AB}}{\pr{A}} &=\pr{B|A} > \pr{A}
\end{align}

\end{enumerate}

		%
\item 
Two cards are drawn at random and without replacement from a pack of 52 playing cards. Find the probability that both the cards are black.
\\
\solution
		%\begin{enumerate}[label=\thesection.\arabic*,ref=\thesection.\theenumi]
	\item One card is drawn from a well-shuffled deck of 52 cards. Find the probability of getting
\begin{enumerate}
\item A king of red colour 
\item A face card 
\item A red face card
\item The jack of hearts
\item A spade
\item The queen of diamonds

\end{enumerate}
\solution
		%\begin{table}[H]
	\centering
\begin{tabular}{|c|c|c|}
\hline
Random variable &Value &Definition\\ \hline
\multirow{3}{*}{X} &0 &Slips of Rs 1\\
&1 &Slips of Rs 5\\
&2 &Slips of Rs 13\\ \hline
\multirow{2}{*}{Y} &0 &Box A\\
&1 &Box B\\\hline
\end{tabular}
\caption{}
\label{tab:Distribution}
\end{table}
See \tabref{tab:Distribution}.
\begin{align}
p_{Y}\brak{k}= \begin{cases} 
      \frac{1}{3} & {k=0} \\
      \frac{2}{3 }& {k=1} 
   \end{cases}
   \\
p_{Y|X}\brak{0|0} = \frac{19}{25}\, 
p_{Y|X}\brak{0|1} = \frac{6}{25}\,
p_{Y|X}\brak{1|0} = \frac{45}{50}\,
p_{Y|X}\brak{1|2} = \frac{5}{50}
\end{align}
The desired probability is the probability that a slip drawn at random is marked other than Rs 1,
\begin{align}
&=1-p_X\brak{0}\\
&= p_X(1) + p_X(2)
\end{align}
Using Bayes theorem,
\begin{align}
&= p_Y\brak{0} \times \pr{Y=0 | X=1} + p_Y\brak{1} \times \pr{Y=1|X=2}\\
&=\frac{1}{3} \times \frac{6}{25} + \frac{2}{3} \times \frac{5}{50}\\
&=\frac{11}{75}
\end{align}

\newpage

%\tableofcontents

\bigskip

\renewcommand{\thefigure}{\theenumi}
\renewcommand{\thetable}{\theenumi}
%\renewcommand{\theequation}{\theenumi}

%\begin{abstract}
%%\boldmath
%In this letter, an algorithm for evaluating the exact analytical bit error rate  (BER)  for the piecewise linear (PL) combiner for  multiple relays is presented. Previous results were available only for upto three relays. The algorithm is unique in the sense that  the actual mathematical expressions, that are prohibitively large, need not be explicitly obtained. The diversity gain due to multiple relays is shown through plots of the analytical BER, well supported by simulations. 
%
%\end{abstract}
% IEEEtran.cls defaults to using nonbold math in the Abstract.
% This preserves the distinction between vectors and scalars. However,
% if the journal you are submitting to favors bold math in the abstract,
% then you can use LaTeX's standard command \boldmath at the very start
% of the abstract to achieve this. Many IEEE journals frown on math
% in the abstract anyway.

% Note that keywords are not normally used for peerreview papers.
%\begin{IEEEkeywords}
%Cooperative diversity, decode and forward, piecewise linear
%\end{IEEEkeywords}



% For peer review papers, you can put extra information on the cover
% page as needed:
% \ifCLASSOPTIONpeerreview
% \begin{center} \bfseries EDICS Category: 3-BBND \end{center}
% \fi
%
% For peerreview papers, this IEEEtran command inserts a page break and
% creates the second title. It will be ignored for other modes.
%\IEEEpeerreviewmaketitle




	\item Five cards—the ten, jack, queen, king and ace of diamonds, are well-shuffled with their face downwards. One card is then picked up at random.
\begin{enumerate}
\item
What is the probability that the card is the queen? 
\item
If the queen is drawn and put aside, what is the probability that the second card picked up is (a) an ace? (b) a queen?\\
\end{enumerate}
\solution
		%\begin{enumerate}[label=\thesection.\arabic*,ref=\thesection.\theenumi]
	\item One card is drawn from a well-shuffled deck of 52 cards. Find the probability of getting
\begin{enumerate}
\item A king of red colour 
\item A face card 
\item A red face card
\item The jack of hearts
\item A spade
\item The queen of diamonds

\end{enumerate}
\solution
		%\input{ncert/10/15/1/14/main.tex}
	\item Five cards—the ten, jack, queen, king and ace of diamonds, are well-shuffled with their face downwards. One card is then picked up at random.
\begin{enumerate}
\item
What is the probability that the card is the queen? 
\item
If the queen is drawn and put aside, what is the probability that the second card picked up is (a) an ace? (b) a queen?\\
\end{enumerate}
\solution
		%\input{ncert/10/15/1/15/defs.tex}
	\item A bag contains $5$ red balls and some blue balls. If the probability of drawing a blue ball is double that if a red ball, determine the number of blue balls in the bag. 
		\\
\solution
		%\input{ncert/10/15/2/3/defs.tex}
	\item A card is selected from a pack of 52 cards.
 \begin{enumerate}[label=(\alph*)] 
                 \item How many points are there in the sample space?
                 \item Calculate the probability that the card is an ace of spades.
                 \item Calculate the probability that the card is (i) an ace and (ii) black card.
 \end{enumerate}
\solution
		%\input{ncert/11/16/3/4/main.tex}
\item Four cards are drawn from a well-shuffled deck of 52 cards. What is the probability of obtaining 3 diamonds and one spade.
\\
\solution
		%\input{ncert/11/16/4/2/defs.tex}
\item In a certain lottery 10,000 tickets are sold and ten equal prizes are awarded. What is the probability of not getting a prize if you buy (a) one ticket (b) two tickets (c) 10 tickets ?	
\\
\solution
		%\input{ncert/11/16/4/4/defs.tex}
		%
\item 
Out of 100 students, two sections of 40 and 60 are formed. If you and your friend are among the 100 students, what is the probability that
\begin{enumerate}
\item you both enter the same section?
\item you both enter the different sections?
\end{enumerate}
\solution
		%\input{ncert/11/16/4/5/defs.tex}
	\item 
The number lock of a suitcase has 4 wheels each labelled with ten digits i.e. from 0 to 9.The lock opens with a sequence of four digits with no repeats.What is the probability of a person getting the right sequence to open the suitcase.
\\
\solution
		%\input{ncert/11/16/4/10/defs.tex}
		%
\item 
Two cards are drawn at random and without replacement from a pack of 52 playing cards. Find the probability that both the cards are black.
\\
\solution
		%\input{ncert/12/13/2/2/defs.tex}
		\item A box of oranges is inspected by examining three randomly selected oranges drawn without replacement. If all the three oranges are good, the box is approved for sale, otherwise, it is rejected. Find the probability that a box containing 15 oranges out of which 12 are good and 3 are bad ones will be approved for sale.
		\label{ncert/12/13/2/3/defs.tex}
		\item Two balls are drawn at random with replacement from a box containing 10 black and 8 red balls. Find the probability that
		\label{ncert/12/13/2/12}
\begin{enumerate}
\item both balls are red.
\item first ball is black and second is red.
\item one of them is black and other is red.
\end{enumerate}

\item In a hostel, 60\% of the students read Hindi newspaper, 40\% read English newspaper and 20\% read both Hindi and English newspapers. A student is selected at random.
		\label{ncert/12/13/2/15}
\begin{enumerate}
\item Find the probability that she reads neither Hindi nor English newspapers.
\item If she reads Hindi newspaper, find the probability that she reads English newspaper.
\item If she reads English newspaper, find the probability that she reads Hindi newspaper.\\
\end{enumerate}
\item The probability of obtaining an even prime number on each die, when a pair of dice is rolled is 
\begin{enumerate}
    \item $0$ 
    
    \item $\frac{1}{3}$ 
    
    \item $\frac{1}{12}$ 
    
    \item $\frac{1}{36}$ 
\end{enumerate}
\solution
		%\input{ncert/12/13/2/17/defs.tex}
	\item A bag contains 4 red and 4 black balls, another bag contains 2 red and 6 black balls. One of the two bags is selected at random and a ball is drawn from the bag which is found to be red. Find the probability that the ball is drawn from the first bag.
\\
\solution
		%\input{ncert/12/13/3/2/main.tex}
  \item
  Cards with numbers 2 to 101 are placed in a box. A card is selected at random.Find the probability that the card has
\begin{enumerate}[label=(\roman*)]
	\item an even number 
	\item a square number
\end{enumerate}
\solution
%\input{exemplar/10/13/3/32/main.tex}
\item
The king, queen and jack of clubs are removed from a deck of 52 playing cards and then well shuffled. Now one card is drawn at random from the remaining cards.  Determine the probability that the card is
\begin{enumerate}[label=(\roman*)]
\item a club
\item 10 of hearts
\end{enumerate}
\solution
%\input{exemplar/10/13/3/29/main.tex}
\item A team of medical students doing their internship have to assist during surgeries
at a city hospital. The probabilities of surgeries rated as very complex, complex,
routine, simple or very simple are respectively, 0.15, 0.20, 0.31, 0.26, .08. Find
the probabilities that a particular surgery will be rated
\begin{enumerate}
	\item complex or very complex;
	\item neither very complex nor very simple;
	\item routine or complex
	\item routine or simple
\end{enumerate}
\solution
%\input{exemplar/11/16/3/8(1)/main.tex}
\item A card is selected from a pack of 52 cards.
\begin{enumerate}[label=(\alph*)]
    \item How many points are there in the sample space?
    \item Calculate the probability that the card is an ace of spades.
    \item Calculate the probability that the card is (i) an ace and (ii) black card.
\end{enumerate}
\solution
%\input{exemplar/11/16/3/4/main2.tex}
\item The probability that a non leap year selected at random will contain 53 sundays.
\\
\solution
%\input{exemplar/10/13/1/19/main.tex}
\item One of the four persons John, Rita, Aslam or Gurpreet will be promoted next
month. Consequently the sample space consists of four elementary outcomes
S = {John promoted, Rita promoted, Aslam promoted, Gurpreet promoted}
You are told that the chances of John’s promotion is same as that of Gurpreet,
Rita’s chances of promotion are twice as likely as Johns. Aslam’s chances are
four times that of John.
\begin{enumerate}
	\item Determine
	\begin{enumerate}
		\item P (John promoted)
		\item P (Rita promoted)
		\item P (Aslam promoted)
		\item P (Gurpreet promoted)
	\end{enumerate}
	\item If A = {John promoted or Gurpreet promoted}, find P (A).
\end{enumerate}
\solution
%\input{exemplar/11/16/3/10/main.tex}
\item A card is drawn from a deck of 52 cards. Find the probability of getting a king or a heart or a red card.\\
\solution
%\input{exemplar/11/16/3/15/main.tex}
\item The probability that a student will pass his examination is 0.73, the probability of
the student getting a compartment is 0.13, and the probability that the student will
either pass or get compartment is 0.96. State True or False.\\
\solution
%\input{exemplar/11/16/3/31/main.tex}
\item A card is selected from a pack of 52 cards\\
\begin{enumerate}[label=(\alph*)]
\item How many points are there in the sample space?
\item Calculate the probability that the cards is an ace of spades.
\item Calculate the probability that the card is (i) an ace (ii)black card.\\
\end{enumerate}
%\input{ncert/11/16/3/4_1/Prob_4.tex}
\item In a non-leap year, the probability of having 53 tuesdays or 53 wednesdays is\\
\solution
%\input{exemplar/11/16/3/18/main.tex}
\item There are 1000 sealed envelopes in a box, 10 of them contain a cash prize of
Rs 100 each, 100 of them contain a cash prize of Rs 50 each and 200 of them
contain a cash prize of Rs 10 each and rest do not contain any cash prize. If they
are well shuffled and an envelope is picked up out, what is the probability that it
contains no cash prize?\\
\solution
%\input{exemplar/10/13/3/34/main.tex}
\item 
A die is thrown and a card is selected at random from a deck of 52 playing cards. The probability of getting an even number on the die and a spade card.\\
\solution
%\input{exemplar/12/13/3/78/main.tex}
\item
If 4-digit numbers greater than 5,000 are randomly formed from the digits 0, 1, 3, 5, and 7, what is the probability of forming a number divisible by 5 when:
\begin{enumerate}
    \item The digits are repeated?
    \item The repetition of digits is not allowed?
\end{enumerate}
\solution
%\input{ncert/11/16/4/9/main.tex}
\item Consider the probability space $\brak{\Omega, \mathcal{G}, P}$ where $\Omega = [0,2]$ and $\mathcal{G} = \cbrak{\phi, \Omega, [0,1], (1,2]}$. Let $X$ and $Y$ be two functions on $\Omega$ defined as
\begin{align*}
    X(\omega) = 
    \begin{cases}
        1 & \text{if }\omega \in [0, 1]\\
        2 & \text{if }\omega \in (1, 2]
    \end{cases}
\end{align*}
and
\begin{align*}
    Y(\omega) = 
    \begin{cases}
        2 & \text{if }\omega \in [0, 1.5]\\
        3 & \text{if }\omega \in (1.5, 2].
    \end{cases}
\end{align*}
Then which one of the following statements is true?
\begin{enumerate}
    \item [(A)] $X$ is a random variable with respect to $\mathcal{G}$, but $Y$ is not a random variable with respect to $\mathcal{G}$.
    \item [(B)] $Y$ is a random variable with respect to $\mathcal{G}$, but $X$ is not a random variable with respect to $\mathcal{G}$.
    \item [(C)] Neither $X$ nor $Y$ is a random variable with respect to $\mathcal{G}$.
    \item [(D)] Both $X$ and $Y$ are random variables with respect to $\mathcal{G}$.
\end{enumerate} \hfill (GATE ST 2023)\\
\solution
%\input{gate/ST/2023/14/main.tex}
	\item  A die is loaded in such a way that each odd number is twice as likely to occur as
each even number. Find $P(G)$, where $G$ is the event that a number greater than
3 occurs on a single roll of the die.
\\
\solution
		%\input{exemplar/11/16/3/5/main.tex}
	\item All the jacks, queens and kings are removed from a deck of 52 playing cards. The remaining cards are well shuffled and then one card is drawn at random. Giving ace a value 1 similar value for other cards, find the probability that the card has a value 
		\begin{enumerate}
			\item 7
			\item greater than 7
			\item less than 7
		\end{enumerate}
		%\input{exemplar/10/13/3/30/main.tex}
  \item A Lot consists of 48 mobile phones of which 42 are good, 3 have only minor defects and 3 have major defects.Varnika will buy a phone if it is good but the trader will only buy a mobile if it has no major defects. One phone is selected at random from the lot. What is the probability that it is
\begin{enumerate}
	\item acceptable to Varnika?
            \item acceptable to the trader?
\end{enumerate}
\solution
	%\input{exemplar/10/13/3/40/main.tex}
 \item A student says that if you throw a die, it will show up 1 or not 1. Therefore, the probability of getting 1 and the probability of getting 'not 1' each is equal to $\frac{1}{2}$. Is this correct? Give reasons.\\
 \solution
        %\input{exemplar/10/13/2/9/main.tex}
   \item Four candidates A, B, C, D have ap-
plied for the assignment to coach a school cricket
team. If A is twice as likely to be selected as B, and
B and C are given about the same chance of being
selected, while C is twice as likely to be selected
as D, what are the probabilities that
\begin{enumerate}
\item C will be selected?
\item A will not be selected?
\end{enumerate}
	%\input{exemplar/11/16/3/9/main.tex}
 \item A bag contain 24 balls of which $x$ balls are red, $2x$ are white and $3x$ are blue. A ball is selected at random, What is the probability that it is
\begin{enumerate}[label=\alph*)]
\item not red ?
\item white ?
\end{enumerate}
%\input{exemplar/10/13/3/41/main.tex}
If the letters of the word ASSASSINATION are arranged at random. Find the Probability that
\begin{enumerate}[label=(\alph*)]
\item Four $S's$ come consecutively in the word
\item Two  $I's$ and two $N's$ come together
\item All $A's$ are not coming together
\item No two $A's$ are coming together
\end{enumerate}
%\input{exemplar/11/16/3/14/main.tex}
	\item One urn contains two black balls (labelled B1 and B2) and one white ball. A
	second urn contains one black ball and two white balls (labelled W1 and W2).
	Suppose the following experiment is performed. One of the two urns is chosen
	at random. Next a ball is randomly chosen from the urn. Then a second ball is
	chosen at random from the same urn without replacing the first ball.
	
	\begin{enumerate}
	\item What is the probability that two black balls are chosen?
	
	\item What is the probability that two balls of opposite colour are chosen?
	\end{enumerate}
	\solution
	%\input{exemplar/11/16/3/12/main1.tex}
\end{enumerate}

	\item A bag contains $5$ red balls and some blue balls. If the probability of drawing a blue ball is double that if a red ball, determine the number of blue balls in the bag. 
		\\
\solution
		%\begin{enumerate}[label=\thesection.\arabic*,ref=\thesection.\theenumi]
	\item One card is drawn from a well-shuffled deck of 52 cards. Find the probability of getting
\begin{enumerate}
\item A king of red colour 
\item A face card 
\item A red face card
\item The jack of hearts
\item A spade
\item The queen of diamonds

\end{enumerate}
\solution
		%\input{ncert/10/15/1/14/main.tex}
	\item Five cards—the ten, jack, queen, king and ace of diamonds, are well-shuffled with their face downwards. One card is then picked up at random.
\begin{enumerate}
\item
What is the probability that the card is the queen? 
\item
If the queen is drawn and put aside, what is the probability that the second card picked up is (a) an ace? (b) a queen?\\
\end{enumerate}
\solution
		%\input{ncert/10/15/1/15/defs.tex}
	\item A bag contains $5$ red balls and some blue balls. If the probability of drawing a blue ball is double that if a red ball, determine the number of blue balls in the bag. 
		\\
\solution
		%\input{ncert/10/15/2/3/defs.tex}
	\item A card is selected from a pack of 52 cards.
 \begin{enumerate}[label=(\alph*)] 
                 \item How many points are there in the sample space?
                 \item Calculate the probability that the card is an ace of spades.
                 \item Calculate the probability that the card is (i) an ace and (ii) black card.
 \end{enumerate}
\solution
		%\input{ncert/11/16/3/4/main.tex}
\item Four cards are drawn from a well-shuffled deck of 52 cards. What is the probability of obtaining 3 diamonds and one spade.
\\
\solution
		%\input{ncert/11/16/4/2/defs.tex}
\item In a certain lottery 10,000 tickets are sold and ten equal prizes are awarded. What is the probability of not getting a prize if you buy (a) one ticket (b) two tickets (c) 10 tickets ?	
\\
\solution
		%\input{ncert/11/16/4/4/defs.tex}
		%
\item 
Out of 100 students, two sections of 40 and 60 are formed. If you and your friend are among the 100 students, what is the probability that
\begin{enumerate}
\item you both enter the same section?
\item you both enter the different sections?
\end{enumerate}
\solution
		%\input{ncert/11/16/4/5/defs.tex}
	\item 
The number lock of a suitcase has 4 wheels each labelled with ten digits i.e. from 0 to 9.The lock opens with a sequence of four digits with no repeats.What is the probability of a person getting the right sequence to open the suitcase.
\\
\solution
		%\input{ncert/11/16/4/10/defs.tex}
		%
\item 
Two cards are drawn at random and without replacement from a pack of 52 playing cards. Find the probability that both the cards are black.
\\
\solution
		%\input{ncert/12/13/2/2/defs.tex}
		\item A box of oranges is inspected by examining three randomly selected oranges drawn without replacement. If all the three oranges are good, the box is approved for sale, otherwise, it is rejected. Find the probability that a box containing 15 oranges out of which 12 are good and 3 are bad ones will be approved for sale.
		\label{ncert/12/13/2/3/defs.tex}
		\item Two balls are drawn at random with replacement from a box containing 10 black and 8 red balls. Find the probability that
		\label{ncert/12/13/2/12}
\begin{enumerate}
\item both balls are red.
\item first ball is black and second is red.
\item one of them is black and other is red.
\end{enumerate}

\item In a hostel, 60\% of the students read Hindi newspaper, 40\% read English newspaper and 20\% read both Hindi and English newspapers. A student is selected at random.
		\label{ncert/12/13/2/15}
\begin{enumerate}
\item Find the probability that she reads neither Hindi nor English newspapers.
\item If she reads Hindi newspaper, find the probability that she reads English newspaper.
\item If she reads English newspaper, find the probability that she reads Hindi newspaper.\\
\end{enumerate}
\item The probability of obtaining an even prime number on each die, when a pair of dice is rolled is 
\begin{enumerate}
    \item $0$ 
    
    \item $\frac{1}{3}$ 
    
    \item $\frac{1}{12}$ 
    
    \item $\frac{1}{36}$ 
\end{enumerate}
\solution
		%\input{ncert/12/13/2/17/defs.tex}
	\item A bag contains 4 red and 4 black balls, another bag contains 2 red and 6 black balls. One of the two bags is selected at random and a ball is drawn from the bag which is found to be red. Find the probability that the ball is drawn from the first bag.
\\
\solution
		%\input{ncert/12/13/3/2/main.tex}
  \item
  Cards with numbers 2 to 101 are placed in a box. A card is selected at random.Find the probability that the card has
\begin{enumerate}[label=(\roman*)]
	\item an even number 
	\item a square number
\end{enumerate}
\solution
%\input{exemplar/10/13/3/32/main.tex}
\item
The king, queen and jack of clubs are removed from a deck of 52 playing cards and then well shuffled. Now one card is drawn at random from the remaining cards.  Determine the probability that the card is
\begin{enumerate}[label=(\roman*)]
\item a club
\item 10 of hearts
\end{enumerate}
\solution
%\input{exemplar/10/13/3/29/main.tex}
\item A team of medical students doing their internship have to assist during surgeries
at a city hospital. The probabilities of surgeries rated as very complex, complex,
routine, simple or very simple are respectively, 0.15, 0.20, 0.31, 0.26, .08. Find
the probabilities that a particular surgery will be rated
\begin{enumerate}
	\item complex or very complex;
	\item neither very complex nor very simple;
	\item routine or complex
	\item routine or simple
\end{enumerate}
\solution
%\input{exemplar/11/16/3/8(1)/main.tex}
\item A card is selected from a pack of 52 cards.
\begin{enumerate}[label=(\alph*)]
    \item How many points are there in the sample space?
    \item Calculate the probability that the card is an ace of spades.
    \item Calculate the probability that the card is (i) an ace and (ii) black card.
\end{enumerate}
\solution
%\input{exemplar/11/16/3/4/main2.tex}
\item The probability that a non leap year selected at random will contain 53 sundays.
\\
\solution
%\input{exemplar/10/13/1/19/main.tex}
\item One of the four persons John, Rita, Aslam or Gurpreet will be promoted next
month. Consequently the sample space consists of four elementary outcomes
S = {John promoted, Rita promoted, Aslam promoted, Gurpreet promoted}
You are told that the chances of John’s promotion is same as that of Gurpreet,
Rita’s chances of promotion are twice as likely as Johns. Aslam’s chances are
four times that of John.
\begin{enumerate}
	\item Determine
	\begin{enumerate}
		\item P (John promoted)
		\item P (Rita promoted)
		\item P (Aslam promoted)
		\item P (Gurpreet promoted)
	\end{enumerate}
	\item If A = {John promoted or Gurpreet promoted}, find P (A).
\end{enumerate}
\solution
%\input{exemplar/11/16/3/10/main.tex}
\item A card is drawn from a deck of 52 cards. Find the probability of getting a king or a heart or a red card.\\
\solution
%\input{exemplar/11/16/3/15/main.tex}
\item The probability that a student will pass his examination is 0.73, the probability of
the student getting a compartment is 0.13, and the probability that the student will
either pass or get compartment is 0.96. State True or False.\\
\solution
%\input{exemplar/11/16/3/31/main.tex}
\item A card is selected from a pack of 52 cards\\
\begin{enumerate}[label=(\alph*)]
\item How many points are there in the sample space?
\item Calculate the probability that the cards is an ace of spades.
\item Calculate the probability that the card is (i) an ace (ii)black card.\\
\end{enumerate}
%\input{ncert/11/16/3/4_1/Prob_4.tex}
\item In a non-leap year, the probability of having 53 tuesdays or 53 wednesdays is\\
\solution
%\input{exemplar/11/16/3/18/main.tex}
\item There are 1000 sealed envelopes in a box, 10 of them contain a cash prize of
Rs 100 each, 100 of them contain a cash prize of Rs 50 each and 200 of them
contain a cash prize of Rs 10 each and rest do not contain any cash prize. If they
are well shuffled and an envelope is picked up out, what is the probability that it
contains no cash prize?\\
\solution
%\input{exemplar/10/13/3/34/main.tex}
\item 
A die is thrown and a card is selected at random from a deck of 52 playing cards. The probability of getting an even number on the die and a spade card.\\
\solution
%\input{exemplar/12/13/3/78/main.tex}
\item
If 4-digit numbers greater than 5,000 are randomly formed from the digits 0, 1, 3, 5, and 7, what is the probability of forming a number divisible by 5 when:
\begin{enumerate}
    \item The digits are repeated?
    \item The repetition of digits is not allowed?
\end{enumerate}
\solution
%\input{ncert/11/16/4/9/main.tex}
\item Consider the probability space $\brak{\Omega, \mathcal{G}, P}$ where $\Omega = [0,2]$ and $\mathcal{G} = \cbrak{\phi, \Omega, [0,1], (1,2]}$. Let $X$ and $Y$ be two functions on $\Omega$ defined as
\begin{align*}
    X(\omega) = 
    \begin{cases}
        1 & \text{if }\omega \in [0, 1]\\
        2 & \text{if }\omega \in (1, 2]
    \end{cases}
\end{align*}
and
\begin{align*}
    Y(\omega) = 
    \begin{cases}
        2 & \text{if }\omega \in [0, 1.5]\\
        3 & \text{if }\omega \in (1.5, 2].
    \end{cases}
\end{align*}
Then which one of the following statements is true?
\begin{enumerate}
    \item [(A)] $X$ is a random variable with respect to $\mathcal{G}$, but $Y$ is not a random variable with respect to $\mathcal{G}$.
    \item [(B)] $Y$ is a random variable with respect to $\mathcal{G}$, but $X$ is not a random variable with respect to $\mathcal{G}$.
    \item [(C)] Neither $X$ nor $Y$ is a random variable with respect to $\mathcal{G}$.
    \item [(D)] Both $X$ and $Y$ are random variables with respect to $\mathcal{G}$.
\end{enumerate} \hfill (GATE ST 2023)\\
\solution
%\input{gate/ST/2023/14/main.tex}
	\item  A die is loaded in such a way that each odd number is twice as likely to occur as
each even number. Find $P(G)$, where $G$ is the event that a number greater than
3 occurs on a single roll of the die.
\\
\solution
		%\input{exemplar/11/16/3/5/main.tex}
	\item All the jacks, queens and kings are removed from a deck of 52 playing cards. The remaining cards are well shuffled and then one card is drawn at random. Giving ace a value 1 similar value for other cards, find the probability that the card has a value 
		\begin{enumerate}
			\item 7
			\item greater than 7
			\item less than 7
		\end{enumerate}
		%\input{exemplar/10/13/3/30/main.tex}
  \item A Lot consists of 48 mobile phones of which 42 are good, 3 have only minor defects and 3 have major defects.Varnika will buy a phone if it is good but the trader will only buy a mobile if it has no major defects. One phone is selected at random from the lot. What is the probability that it is
\begin{enumerate}
	\item acceptable to Varnika?
            \item acceptable to the trader?
\end{enumerate}
\solution
	%\input{exemplar/10/13/3/40/main.tex}
 \item A student says that if you throw a die, it will show up 1 or not 1. Therefore, the probability of getting 1 and the probability of getting 'not 1' each is equal to $\frac{1}{2}$. Is this correct? Give reasons.\\
 \solution
        %\input{exemplar/10/13/2/9/main.tex}
   \item Four candidates A, B, C, D have ap-
plied for the assignment to coach a school cricket
team. If A is twice as likely to be selected as B, and
B and C are given about the same chance of being
selected, while C is twice as likely to be selected
as D, what are the probabilities that
\begin{enumerate}
\item C will be selected?
\item A will not be selected?
\end{enumerate}
	%\input{exemplar/11/16/3/9/main.tex}
 \item A bag contain 24 balls of which $x$ balls are red, $2x$ are white and $3x$ are blue. A ball is selected at random, What is the probability that it is
\begin{enumerate}[label=\alph*)]
\item not red ?
\item white ?
\end{enumerate}
%\input{exemplar/10/13/3/41/main.tex}
If the letters of the word ASSASSINATION are arranged at random. Find the Probability that
\begin{enumerate}[label=(\alph*)]
\item Four $S's$ come consecutively in the word
\item Two  $I's$ and two $N's$ come together
\item All $A's$ are not coming together
\item No two $A's$ are coming together
\end{enumerate}
%\input{exemplar/11/16/3/14/main.tex}
	\item One urn contains two black balls (labelled B1 and B2) and one white ball. A
	second urn contains one black ball and two white balls (labelled W1 and W2).
	Suppose the following experiment is performed. One of the two urns is chosen
	at random. Next a ball is randomly chosen from the urn. Then a second ball is
	chosen at random from the same urn without replacing the first ball.
	
	\begin{enumerate}
	\item What is the probability that two black balls are chosen?
	
	\item What is the probability that two balls of opposite colour are chosen?
	\end{enumerate}
	\solution
	%\input{exemplar/11/16/3/12/main1.tex}
\end{enumerate}

	\item A card is selected from a pack of 52 cards.
 \begin{enumerate}[label=(\alph*)] 
                 \item How many points are there in the sample space?
                 \item Calculate the probability that the card is an ace of spades.
                 \item Calculate the probability that the card is (i) an ace and (ii) black card.
 \end{enumerate}
\solution
		%\begin{table}[H]
	\centering
\begin{tabular}{|c|c|c|}
\hline
Random variable &Value &Definition\\ \hline
\multirow{3}{*}{X} &0 &Slips of Rs 1\\
&1 &Slips of Rs 5\\
&2 &Slips of Rs 13\\ \hline
\multirow{2}{*}{Y} &0 &Box A\\
&1 &Box B\\\hline
\end{tabular}
\caption{}
\label{tab:Distribution}
\end{table}
See \tabref{tab:Distribution}.
\begin{align}
p_{Y}\brak{k}= \begin{cases} 
      \frac{1}{3} & {k=0} \\
      \frac{2}{3 }& {k=1} 
   \end{cases}
   \\
p_{Y|X}\brak{0|0} = \frac{19}{25}\, 
p_{Y|X}\brak{0|1} = \frac{6}{25}\,
p_{Y|X}\brak{1|0} = \frac{45}{50}\,
p_{Y|X}\brak{1|2} = \frac{5}{50}
\end{align}
The desired probability is the probability that a slip drawn at random is marked other than Rs 1,
\begin{align}
&=1-p_X\brak{0}\\
&= p_X(1) + p_X(2)
\end{align}
Using Bayes theorem,
\begin{align}
&= p_Y\brak{0} \times \pr{Y=0 | X=1} + p_Y\brak{1} \times \pr{Y=1|X=2}\\
&=\frac{1}{3} \times \frac{6}{25} + \frac{2}{3} \times \frac{5}{50}\\
&=\frac{11}{75}
\end{align}

\newpage

%\tableofcontents

\bigskip

\renewcommand{\thefigure}{\theenumi}
\renewcommand{\thetable}{\theenumi}
%\renewcommand{\theequation}{\theenumi}

%\begin{abstract}
%%\boldmath
%In this letter, an algorithm for evaluating the exact analytical bit error rate  (BER)  for the piecewise linear (PL) combiner for  multiple relays is presented. Previous results were available only for upto three relays. The algorithm is unique in the sense that  the actual mathematical expressions, that are prohibitively large, need not be explicitly obtained. The diversity gain due to multiple relays is shown through plots of the analytical BER, well supported by simulations. 
%
%\end{abstract}
% IEEEtran.cls defaults to using nonbold math in the Abstract.
% This preserves the distinction between vectors and scalars. However,
% if the journal you are submitting to favors bold math in the abstract,
% then you can use LaTeX's standard command \boldmath at the very start
% of the abstract to achieve this. Many IEEE journals frown on math
% in the abstract anyway.

% Note that keywords are not normally used for peerreview papers.
%\begin{IEEEkeywords}
%Cooperative diversity, decode and forward, piecewise linear
%\end{IEEEkeywords}



% For peer review papers, you can put extra information on the cover
% page as needed:
% \ifCLASSOPTIONpeerreview
% \begin{center} \bfseries EDICS Category: 3-BBND \end{center}
% \fi
%
% For peerreview papers, this IEEEtran command inserts a page break and
% creates the second title. It will be ignored for other modes.
%\IEEEpeerreviewmaketitle




\item Four cards are drawn from a well-shuffled deck of 52 cards. What is the probability of obtaining 3 diamonds and one spade.
\\
\solution
		%\begin{enumerate}[label=\thesection.\arabic*,ref=\thesection.\theenumi]
	\item One card is drawn from a well-shuffled deck of 52 cards. Find the probability of getting
\begin{enumerate}
\item A king of red colour 
\item A face card 
\item A red face card
\item The jack of hearts
\item A spade
\item The queen of diamonds

\end{enumerate}
\solution
		%\input{ncert/10/15/1/14/main.tex}
	\item Five cards—the ten, jack, queen, king and ace of diamonds, are well-shuffled with their face downwards. One card is then picked up at random.
\begin{enumerate}
\item
What is the probability that the card is the queen? 
\item
If the queen is drawn and put aside, what is the probability that the second card picked up is (a) an ace? (b) a queen?\\
\end{enumerate}
\solution
		%\input{ncert/10/15/1/15/defs.tex}
	\item A bag contains $5$ red balls and some blue balls. If the probability of drawing a blue ball is double that if a red ball, determine the number of blue balls in the bag. 
		\\
\solution
		%\input{ncert/10/15/2/3/defs.tex}
	\item A card is selected from a pack of 52 cards.
 \begin{enumerate}[label=(\alph*)] 
                 \item How many points are there in the sample space?
                 \item Calculate the probability that the card is an ace of spades.
                 \item Calculate the probability that the card is (i) an ace and (ii) black card.
 \end{enumerate}
\solution
		%\input{ncert/11/16/3/4/main.tex}
\item Four cards are drawn from a well-shuffled deck of 52 cards. What is the probability of obtaining 3 diamonds and one spade.
\\
\solution
		%\input{ncert/11/16/4/2/defs.tex}
\item In a certain lottery 10,000 tickets are sold and ten equal prizes are awarded. What is the probability of not getting a prize if you buy (a) one ticket (b) two tickets (c) 10 tickets ?	
\\
\solution
		%\input{ncert/11/16/4/4/defs.tex}
		%
\item 
Out of 100 students, two sections of 40 and 60 are formed. If you and your friend are among the 100 students, what is the probability that
\begin{enumerate}
\item you both enter the same section?
\item you both enter the different sections?
\end{enumerate}
\solution
		%\input{ncert/11/16/4/5/defs.tex}
	\item 
The number lock of a suitcase has 4 wheels each labelled with ten digits i.e. from 0 to 9.The lock opens with a sequence of four digits with no repeats.What is the probability of a person getting the right sequence to open the suitcase.
\\
\solution
		%\input{ncert/11/16/4/10/defs.tex}
		%
\item 
Two cards are drawn at random and without replacement from a pack of 52 playing cards. Find the probability that both the cards are black.
\\
\solution
		%\input{ncert/12/13/2/2/defs.tex}
		\item A box of oranges is inspected by examining three randomly selected oranges drawn without replacement. If all the three oranges are good, the box is approved for sale, otherwise, it is rejected. Find the probability that a box containing 15 oranges out of which 12 are good and 3 are bad ones will be approved for sale.
		\label{ncert/12/13/2/3/defs.tex}
		\item Two balls are drawn at random with replacement from a box containing 10 black and 8 red balls. Find the probability that
		\label{ncert/12/13/2/12}
\begin{enumerate}
\item both balls are red.
\item first ball is black and second is red.
\item one of them is black and other is red.
\end{enumerate}

\item In a hostel, 60\% of the students read Hindi newspaper, 40\% read English newspaper and 20\% read both Hindi and English newspapers. A student is selected at random.
		\label{ncert/12/13/2/15}
\begin{enumerate}
\item Find the probability that she reads neither Hindi nor English newspapers.
\item If she reads Hindi newspaper, find the probability that she reads English newspaper.
\item If she reads English newspaper, find the probability that she reads Hindi newspaper.\\
\end{enumerate}
\item The probability of obtaining an even prime number on each die, when a pair of dice is rolled is 
\begin{enumerate}
    \item $0$ 
    
    \item $\frac{1}{3}$ 
    
    \item $\frac{1}{12}$ 
    
    \item $\frac{1}{36}$ 
\end{enumerate}
\solution
		%\input{ncert/12/13/2/17/defs.tex}
	\item A bag contains 4 red and 4 black balls, another bag contains 2 red and 6 black balls. One of the two bags is selected at random and a ball is drawn from the bag which is found to be red. Find the probability that the ball is drawn from the first bag.
\\
\solution
		%\input{ncert/12/13/3/2/main.tex}
  \item
  Cards with numbers 2 to 101 are placed in a box. A card is selected at random.Find the probability that the card has
\begin{enumerate}[label=(\roman*)]
	\item an even number 
	\item a square number
\end{enumerate}
\solution
%\input{exemplar/10/13/3/32/main.tex}
\item
The king, queen and jack of clubs are removed from a deck of 52 playing cards and then well shuffled. Now one card is drawn at random from the remaining cards.  Determine the probability that the card is
\begin{enumerate}[label=(\roman*)]
\item a club
\item 10 of hearts
\end{enumerate}
\solution
%\input{exemplar/10/13/3/29/main.tex}
\item A team of medical students doing their internship have to assist during surgeries
at a city hospital. The probabilities of surgeries rated as very complex, complex,
routine, simple or very simple are respectively, 0.15, 0.20, 0.31, 0.26, .08. Find
the probabilities that a particular surgery will be rated
\begin{enumerate}
	\item complex or very complex;
	\item neither very complex nor very simple;
	\item routine or complex
	\item routine or simple
\end{enumerate}
\solution
%\input{exemplar/11/16/3/8(1)/main.tex}
\item A card is selected from a pack of 52 cards.
\begin{enumerate}[label=(\alph*)]
    \item How many points are there in the sample space?
    \item Calculate the probability that the card is an ace of spades.
    \item Calculate the probability that the card is (i) an ace and (ii) black card.
\end{enumerate}
\solution
%\input{exemplar/11/16/3/4/main2.tex}
\item The probability that a non leap year selected at random will contain 53 sundays.
\\
\solution
%\input{exemplar/10/13/1/19/main.tex}
\item One of the four persons John, Rita, Aslam or Gurpreet will be promoted next
month. Consequently the sample space consists of four elementary outcomes
S = {John promoted, Rita promoted, Aslam promoted, Gurpreet promoted}
You are told that the chances of John’s promotion is same as that of Gurpreet,
Rita’s chances of promotion are twice as likely as Johns. Aslam’s chances are
four times that of John.
\begin{enumerate}
	\item Determine
	\begin{enumerate}
		\item P (John promoted)
		\item P (Rita promoted)
		\item P (Aslam promoted)
		\item P (Gurpreet promoted)
	\end{enumerate}
	\item If A = {John promoted or Gurpreet promoted}, find P (A).
\end{enumerate}
\solution
%\input{exemplar/11/16/3/10/main.tex}
\item A card is drawn from a deck of 52 cards. Find the probability of getting a king or a heart or a red card.\\
\solution
%\input{exemplar/11/16/3/15/main.tex}
\item The probability that a student will pass his examination is 0.73, the probability of
the student getting a compartment is 0.13, and the probability that the student will
either pass or get compartment is 0.96. State True or False.\\
\solution
%\input{exemplar/11/16/3/31/main.tex}
\item A card is selected from a pack of 52 cards\\
\begin{enumerate}[label=(\alph*)]
\item How many points are there in the sample space?
\item Calculate the probability that the cards is an ace of spades.
\item Calculate the probability that the card is (i) an ace (ii)black card.\\
\end{enumerate}
%\input{ncert/11/16/3/4_1/Prob_4.tex}
\item In a non-leap year, the probability of having 53 tuesdays or 53 wednesdays is\\
\solution
%\input{exemplar/11/16/3/18/main.tex}
\item There are 1000 sealed envelopes in a box, 10 of them contain a cash prize of
Rs 100 each, 100 of them contain a cash prize of Rs 50 each and 200 of them
contain a cash prize of Rs 10 each and rest do not contain any cash prize. If they
are well shuffled and an envelope is picked up out, what is the probability that it
contains no cash prize?\\
\solution
%\input{exemplar/10/13/3/34/main.tex}
\item 
A die is thrown and a card is selected at random from a deck of 52 playing cards. The probability of getting an even number on the die and a spade card.\\
\solution
%\input{exemplar/12/13/3/78/main.tex}
\item
If 4-digit numbers greater than 5,000 are randomly formed from the digits 0, 1, 3, 5, and 7, what is the probability of forming a number divisible by 5 when:
\begin{enumerate}
    \item The digits are repeated?
    \item The repetition of digits is not allowed?
\end{enumerate}
\solution
%\input{ncert/11/16/4/9/main.tex}
\item Consider the probability space $\brak{\Omega, \mathcal{G}, P}$ where $\Omega = [0,2]$ and $\mathcal{G} = \cbrak{\phi, \Omega, [0,1], (1,2]}$. Let $X$ and $Y$ be two functions on $\Omega$ defined as
\begin{align*}
    X(\omega) = 
    \begin{cases}
        1 & \text{if }\omega \in [0, 1]\\
        2 & \text{if }\omega \in (1, 2]
    \end{cases}
\end{align*}
and
\begin{align*}
    Y(\omega) = 
    \begin{cases}
        2 & \text{if }\omega \in [0, 1.5]\\
        3 & \text{if }\omega \in (1.5, 2].
    \end{cases}
\end{align*}
Then which one of the following statements is true?
\begin{enumerate}
    \item [(A)] $X$ is a random variable with respect to $\mathcal{G}$, but $Y$ is not a random variable with respect to $\mathcal{G}$.
    \item [(B)] $Y$ is a random variable with respect to $\mathcal{G}$, but $X$ is not a random variable with respect to $\mathcal{G}$.
    \item [(C)] Neither $X$ nor $Y$ is a random variable with respect to $\mathcal{G}$.
    \item [(D)] Both $X$ and $Y$ are random variables with respect to $\mathcal{G}$.
\end{enumerate} \hfill (GATE ST 2023)\\
\solution
%\input{gate/ST/2023/14/main.tex}
	\item  A die is loaded in such a way that each odd number is twice as likely to occur as
each even number. Find $P(G)$, where $G$ is the event that a number greater than
3 occurs on a single roll of the die.
\\
\solution
		%\input{exemplar/11/16/3/5/main.tex}
	\item All the jacks, queens and kings are removed from a deck of 52 playing cards. The remaining cards are well shuffled and then one card is drawn at random. Giving ace a value 1 similar value for other cards, find the probability that the card has a value 
		\begin{enumerate}
			\item 7
			\item greater than 7
			\item less than 7
		\end{enumerate}
		%\input{exemplar/10/13/3/30/main.tex}
  \item A Lot consists of 48 mobile phones of which 42 are good, 3 have only minor defects and 3 have major defects.Varnika will buy a phone if it is good but the trader will only buy a mobile if it has no major defects. One phone is selected at random from the lot. What is the probability that it is
\begin{enumerate}
	\item acceptable to Varnika?
            \item acceptable to the trader?
\end{enumerate}
\solution
	%\input{exemplar/10/13/3/40/main.tex}
 \item A student says that if you throw a die, it will show up 1 or not 1. Therefore, the probability of getting 1 and the probability of getting 'not 1' each is equal to $\frac{1}{2}$. Is this correct? Give reasons.\\
 \solution
        %\input{exemplar/10/13/2/9/main.tex}
   \item Four candidates A, B, C, D have ap-
plied for the assignment to coach a school cricket
team. If A is twice as likely to be selected as B, and
B and C are given about the same chance of being
selected, while C is twice as likely to be selected
as D, what are the probabilities that
\begin{enumerate}
\item C will be selected?
\item A will not be selected?
\end{enumerate}
	%\input{exemplar/11/16/3/9/main.tex}
 \item A bag contain 24 balls of which $x$ balls are red, $2x$ are white and $3x$ are blue. A ball is selected at random, What is the probability that it is
\begin{enumerate}[label=\alph*)]
\item not red ?
\item white ?
\end{enumerate}
%\input{exemplar/10/13/3/41/main.tex}
If the letters of the word ASSASSINATION are arranged at random. Find the Probability that
\begin{enumerate}[label=(\alph*)]
\item Four $S's$ come consecutively in the word
\item Two  $I's$ and two $N's$ come together
\item All $A's$ are not coming together
\item No two $A's$ are coming together
\end{enumerate}
%\input{exemplar/11/16/3/14/main.tex}
	\item One urn contains two black balls (labelled B1 and B2) and one white ball. A
	second urn contains one black ball and two white balls (labelled W1 and W2).
	Suppose the following experiment is performed. One of the two urns is chosen
	at random. Next a ball is randomly chosen from the urn. Then a second ball is
	chosen at random from the same urn without replacing the first ball.
	
	\begin{enumerate}
	\item What is the probability that two black balls are chosen?
	
	\item What is the probability that two balls of opposite colour are chosen?
	\end{enumerate}
	\solution
	%\input{exemplar/11/16/3/12/main1.tex}
\end{enumerate}

\item In a certain lottery 10,000 tickets are sold and ten equal prizes are awarded. What is the probability of not getting a prize if you buy (a) one ticket (b) two tickets (c) 10 tickets ?	
\\
\solution
		%\begin{enumerate}[label=\thesection.\arabic*,ref=\thesection.\theenumi]
	\item One card is drawn from a well-shuffled deck of 52 cards. Find the probability of getting
\begin{enumerate}
\item A king of red colour 
\item A face card 
\item A red face card
\item The jack of hearts
\item A spade
\item The queen of diamonds

\end{enumerate}
\solution
		%\input{ncert/10/15/1/14/main.tex}
	\item Five cards—the ten, jack, queen, king and ace of diamonds, are well-shuffled with their face downwards. One card is then picked up at random.
\begin{enumerate}
\item
What is the probability that the card is the queen? 
\item
If the queen is drawn and put aside, what is the probability that the second card picked up is (a) an ace? (b) a queen?\\
\end{enumerate}
\solution
		%\input{ncert/10/15/1/15/defs.tex}
	\item A bag contains $5$ red balls and some blue balls. If the probability of drawing a blue ball is double that if a red ball, determine the number of blue balls in the bag. 
		\\
\solution
		%\input{ncert/10/15/2/3/defs.tex}
	\item A card is selected from a pack of 52 cards.
 \begin{enumerate}[label=(\alph*)] 
                 \item How many points are there in the sample space?
                 \item Calculate the probability that the card is an ace of spades.
                 \item Calculate the probability that the card is (i) an ace and (ii) black card.
 \end{enumerate}
\solution
		%\input{ncert/11/16/3/4/main.tex}
\item Four cards are drawn from a well-shuffled deck of 52 cards. What is the probability of obtaining 3 diamonds and one spade.
\\
\solution
		%\input{ncert/11/16/4/2/defs.tex}
\item In a certain lottery 10,000 tickets are sold and ten equal prizes are awarded. What is the probability of not getting a prize if you buy (a) one ticket (b) two tickets (c) 10 tickets ?	
\\
\solution
		%\input{ncert/11/16/4/4/defs.tex}
		%
\item 
Out of 100 students, two sections of 40 and 60 are formed. If you and your friend are among the 100 students, what is the probability that
\begin{enumerate}
\item you both enter the same section?
\item you both enter the different sections?
\end{enumerate}
\solution
		%\input{ncert/11/16/4/5/defs.tex}
	\item 
The number lock of a suitcase has 4 wheels each labelled with ten digits i.e. from 0 to 9.The lock opens with a sequence of four digits with no repeats.What is the probability of a person getting the right sequence to open the suitcase.
\\
\solution
		%\input{ncert/11/16/4/10/defs.tex}
		%
\item 
Two cards are drawn at random and without replacement from a pack of 52 playing cards. Find the probability that both the cards are black.
\\
\solution
		%\input{ncert/12/13/2/2/defs.tex}
		\item A box of oranges is inspected by examining three randomly selected oranges drawn without replacement. If all the three oranges are good, the box is approved for sale, otherwise, it is rejected. Find the probability that a box containing 15 oranges out of which 12 are good and 3 are bad ones will be approved for sale.
		\label{ncert/12/13/2/3/defs.tex}
		\item Two balls are drawn at random with replacement from a box containing 10 black and 8 red balls. Find the probability that
		\label{ncert/12/13/2/12}
\begin{enumerate}
\item both balls are red.
\item first ball is black and second is red.
\item one of them is black and other is red.
\end{enumerate}

\item In a hostel, 60\% of the students read Hindi newspaper, 40\% read English newspaper and 20\% read both Hindi and English newspapers. A student is selected at random.
		\label{ncert/12/13/2/15}
\begin{enumerate}
\item Find the probability that she reads neither Hindi nor English newspapers.
\item If she reads Hindi newspaper, find the probability that she reads English newspaper.
\item If she reads English newspaper, find the probability that she reads Hindi newspaper.\\
\end{enumerate}
\item The probability of obtaining an even prime number on each die, when a pair of dice is rolled is 
\begin{enumerate}
    \item $0$ 
    
    \item $\frac{1}{3}$ 
    
    \item $\frac{1}{12}$ 
    
    \item $\frac{1}{36}$ 
\end{enumerate}
\solution
		%\input{ncert/12/13/2/17/defs.tex}
	\item A bag contains 4 red and 4 black balls, another bag contains 2 red and 6 black balls. One of the two bags is selected at random and a ball is drawn from the bag which is found to be red. Find the probability that the ball is drawn from the first bag.
\\
\solution
		%\input{ncert/12/13/3/2/main.tex}
  \item
  Cards with numbers 2 to 101 are placed in a box. A card is selected at random.Find the probability that the card has
\begin{enumerate}[label=(\roman*)]
	\item an even number 
	\item a square number
\end{enumerate}
\solution
%\input{exemplar/10/13/3/32/main.tex}
\item
The king, queen and jack of clubs are removed from a deck of 52 playing cards and then well shuffled. Now one card is drawn at random from the remaining cards.  Determine the probability that the card is
\begin{enumerate}[label=(\roman*)]
\item a club
\item 10 of hearts
\end{enumerate}
\solution
%\input{exemplar/10/13/3/29/main.tex}
\item A team of medical students doing their internship have to assist during surgeries
at a city hospital. The probabilities of surgeries rated as very complex, complex,
routine, simple or very simple are respectively, 0.15, 0.20, 0.31, 0.26, .08. Find
the probabilities that a particular surgery will be rated
\begin{enumerate}
	\item complex or very complex;
	\item neither very complex nor very simple;
	\item routine or complex
	\item routine or simple
\end{enumerate}
\solution
%\input{exemplar/11/16/3/8(1)/main.tex}
\item A card is selected from a pack of 52 cards.
\begin{enumerate}[label=(\alph*)]
    \item How many points are there in the sample space?
    \item Calculate the probability that the card is an ace of spades.
    \item Calculate the probability that the card is (i) an ace and (ii) black card.
\end{enumerate}
\solution
%\input{exemplar/11/16/3/4/main2.tex}
\item The probability that a non leap year selected at random will contain 53 sundays.
\\
\solution
%\input{exemplar/10/13/1/19/main.tex}
\item One of the four persons John, Rita, Aslam or Gurpreet will be promoted next
month. Consequently the sample space consists of four elementary outcomes
S = {John promoted, Rita promoted, Aslam promoted, Gurpreet promoted}
You are told that the chances of John’s promotion is same as that of Gurpreet,
Rita’s chances of promotion are twice as likely as Johns. Aslam’s chances are
four times that of John.
\begin{enumerate}
	\item Determine
	\begin{enumerate}
		\item P (John promoted)
		\item P (Rita promoted)
		\item P (Aslam promoted)
		\item P (Gurpreet promoted)
	\end{enumerate}
	\item If A = {John promoted or Gurpreet promoted}, find P (A).
\end{enumerate}
\solution
%\input{exemplar/11/16/3/10/main.tex}
\item A card is drawn from a deck of 52 cards. Find the probability of getting a king or a heart or a red card.\\
\solution
%\input{exemplar/11/16/3/15/main.tex}
\item The probability that a student will pass his examination is 0.73, the probability of
the student getting a compartment is 0.13, and the probability that the student will
either pass or get compartment is 0.96. State True or False.\\
\solution
%\input{exemplar/11/16/3/31/main.tex}
\item A card is selected from a pack of 52 cards\\
\begin{enumerate}[label=(\alph*)]
\item How many points are there in the sample space?
\item Calculate the probability that the cards is an ace of spades.
\item Calculate the probability that the card is (i) an ace (ii)black card.\\
\end{enumerate}
%\input{ncert/11/16/3/4_1/Prob_4.tex}
\item In a non-leap year, the probability of having 53 tuesdays or 53 wednesdays is\\
\solution
%\input{exemplar/11/16/3/18/main.tex}
\item There are 1000 sealed envelopes in a box, 10 of them contain a cash prize of
Rs 100 each, 100 of them contain a cash prize of Rs 50 each and 200 of them
contain a cash prize of Rs 10 each and rest do not contain any cash prize. If they
are well shuffled and an envelope is picked up out, what is the probability that it
contains no cash prize?\\
\solution
%\input{exemplar/10/13/3/34/main.tex}
\item 
A die is thrown and a card is selected at random from a deck of 52 playing cards. The probability of getting an even number on the die and a spade card.\\
\solution
%\input{exemplar/12/13/3/78/main.tex}
\item
If 4-digit numbers greater than 5,000 are randomly formed from the digits 0, 1, 3, 5, and 7, what is the probability of forming a number divisible by 5 when:
\begin{enumerate}
    \item The digits are repeated?
    \item The repetition of digits is not allowed?
\end{enumerate}
\solution
%\input{ncert/11/16/4/9/main.tex}
\item Consider the probability space $\brak{\Omega, \mathcal{G}, P}$ where $\Omega = [0,2]$ and $\mathcal{G} = \cbrak{\phi, \Omega, [0,1], (1,2]}$. Let $X$ and $Y$ be two functions on $\Omega$ defined as
\begin{align*}
    X(\omega) = 
    \begin{cases}
        1 & \text{if }\omega \in [0, 1]\\
        2 & \text{if }\omega \in (1, 2]
    \end{cases}
\end{align*}
and
\begin{align*}
    Y(\omega) = 
    \begin{cases}
        2 & \text{if }\omega \in [0, 1.5]\\
        3 & \text{if }\omega \in (1.5, 2].
    \end{cases}
\end{align*}
Then which one of the following statements is true?
\begin{enumerate}
    \item [(A)] $X$ is a random variable with respect to $\mathcal{G}$, but $Y$ is not a random variable with respect to $\mathcal{G}$.
    \item [(B)] $Y$ is a random variable with respect to $\mathcal{G}$, but $X$ is not a random variable with respect to $\mathcal{G}$.
    \item [(C)] Neither $X$ nor $Y$ is a random variable with respect to $\mathcal{G}$.
    \item [(D)] Both $X$ and $Y$ are random variables with respect to $\mathcal{G}$.
\end{enumerate} \hfill (GATE ST 2023)\\
\solution
%\input{gate/ST/2023/14/main.tex}
	\item  A die is loaded in such a way that each odd number is twice as likely to occur as
each even number. Find $P(G)$, where $G$ is the event that a number greater than
3 occurs on a single roll of the die.
\\
\solution
		%\input{exemplar/11/16/3/5/main.tex}
	\item All the jacks, queens and kings are removed from a deck of 52 playing cards. The remaining cards are well shuffled and then one card is drawn at random. Giving ace a value 1 similar value for other cards, find the probability that the card has a value 
		\begin{enumerate}
			\item 7
			\item greater than 7
			\item less than 7
		\end{enumerate}
		%\input{exemplar/10/13/3/30/main.tex}
  \item A Lot consists of 48 mobile phones of which 42 are good, 3 have only minor defects and 3 have major defects.Varnika will buy a phone if it is good but the trader will only buy a mobile if it has no major defects. One phone is selected at random from the lot. What is the probability that it is
\begin{enumerate}
	\item acceptable to Varnika?
            \item acceptable to the trader?
\end{enumerate}
\solution
	%\input{exemplar/10/13/3/40/main.tex}
 \item A student says that if you throw a die, it will show up 1 or not 1. Therefore, the probability of getting 1 and the probability of getting 'not 1' each is equal to $\frac{1}{2}$. Is this correct? Give reasons.\\
 \solution
        %\input{exemplar/10/13/2/9/main.tex}
   \item Four candidates A, B, C, D have ap-
plied for the assignment to coach a school cricket
team. If A is twice as likely to be selected as B, and
B and C are given about the same chance of being
selected, while C is twice as likely to be selected
as D, what are the probabilities that
\begin{enumerate}
\item C will be selected?
\item A will not be selected?
\end{enumerate}
	%\input{exemplar/11/16/3/9/main.tex}
 \item A bag contain 24 balls of which $x$ balls are red, $2x$ are white and $3x$ are blue. A ball is selected at random, What is the probability that it is
\begin{enumerate}[label=\alph*)]
\item not red ?
\item white ?
\end{enumerate}
%\input{exemplar/10/13/3/41/main.tex}
If the letters of the word ASSASSINATION are arranged at random. Find the Probability that
\begin{enumerate}[label=(\alph*)]
\item Four $S's$ come consecutively in the word
\item Two  $I's$ and two $N's$ come together
\item All $A's$ are not coming together
\item No two $A's$ are coming together
\end{enumerate}
%\input{exemplar/11/16/3/14/main.tex}
	\item One urn contains two black balls (labelled B1 and B2) and one white ball. A
	second urn contains one black ball and two white balls (labelled W1 and W2).
	Suppose the following experiment is performed. One of the two urns is chosen
	at random. Next a ball is randomly chosen from the urn. Then a second ball is
	chosen at random from the same urn without replacing the first ball.
	
	\begin{enumerate}
	\item What is the probability that two black balls are chosen?
	
	\item What is the probability that two balls of opposite colour are chosen?
	\end{enumerate}
	\solution
	%\input{exemplar/11/16/3/12/main1.tex}
\end{enumerate}

		%
\item 
Out of 100 students, two sections of 40 and 60 are formed. If you and your friend are among the 100 students, what is the probability that
\begin{enumerate}
\item you both enter the same section?
\item you both enter the different sections?
\end{enumerate}
\solution
		%\begin{enumerate}[label=\thesection.\arabic*,ref=\thesection.\theenumi]
	\item One card is drawn from a well-shuffled deck of 52 cards. Find the probability of getting
\begin{enumerate}
\item A king of red colour 
\item A face card 
\item A red face card
\item The jack of hearts
\item A spade
\item The queen of diamonds

\end{enumerate}
\solution
		%\input{ncert/10/15/1/14/main.tex}
	\item Five cards—the ten, jack, queen, king and ace of diamonds, are well-shuffled with their face downwards. One card is then picked up at random.
\begin{enumerate}
\item
What is the probability that the card is the queen? 
\item
If the queen is drawn and put aside, what is the probability that the second card picked up is (a) an ace? (b) a queen?\\
\end{enumerate}
\solution
		%\input{ncert/10/15/1/15/defs.tex}
	\item A bag contains $5$ red balls and some blue balls. If the probability of drawing a blue ball is double that if a red ball, determine the number of blue balls in the bag. 
		\\
\solution
		%\input{ncert/10/15/2/3/defs.tex}
	\item A card is selected from a pack of 52 cards.
 \begin{enumerate}[label=(\alph*)] 
                 \item How many points are there in the sample space?
                 \item Calculate the probability that the card is an ace of spades.
                 \item Calculate the probability that the card is (i) an ace and (ii) black card.
 \end{enumerate}
\solution
		%\input{ncert/11/16/3/4/main.tex}
\item Four cards are drawn from a well-shuffled deck of 52 cards. What is the probability of obtaining 3 diamonds and one spade.
\\
\solution
		%\input{ncert/11/16/4/2/defs.tex}
\item In a certain lottery 10,000 tickets are sold and ten equal prizes are awarded. What is the probability of not getting a prize if you buy (a) one ticket (b) two tickets (c) 10 tickets ?	
\\
\solution
		%\input{ncert/11/16/4/4/defs.tex}
		%
\item 
Out of 100 students, two sections of 40 and 60 are formed. If you and your friend are among the 100 students, what is the probability that
\begin{enumerate}
\item you both enter the same section?
\item you both enter the different sections?
\end{enumerate}
\solution
		%\input{ncert/11/16/4/5/defs.tex}
	\item 
The number lock of a suitcase has 4 wheels each labelled with ten digits i.e. from 0 to 9.The lock opens with a sequence of four digits with no repeats.What is the probability of a person getting the right sequence to open the suitcase.
\\
\solution
		%\input{ncert/11/16/4/10/defs.tex}
		%
\item 
Two cards are drawn at random and without replacement from a pack of 52 playing cards. Find the probability that both the cards are black.
\\
\solution
		%\input{ncert/12/13/2/2/defs.tex}
		\item A box of oranges is inspected by examining three randomly selected oranges drawn without replacement. If all the three oranges are good, the box is approved for sale, otherwise, it is rejected. Find the probability that a box containing 15 oranges out of which 12 are good and 3 are bad ones will be approved for sale.
		\label{ncert/12/13/2/3/defs.tex}
		\item Two balls are drawn at random with replacement from a box containing 10 black and 8 red balls. Find the probability that
		\label{ncert/12/13/2/12}
\begin{enumerate}
\item both balls are red.
\item first ball is black and second is red.
\item one of them is black and other is red.
\end{enumerate}

\item In a hostel, 60\% of the students read Hindi newspaper, 40\% read English newspaper and 20\% read both Hindi and English newspapers. A student is selected at random.
		\label{ncert/12/13/2/15}
\begin{enumerate}
\item Find the probability that she reads neither Hindi nor English newspapers.
\item If she reads Hindi newspaper, find the probability that she reads English newspaper.
\item If she reads English newspaper, find the probability that she reads Hindi newspaper.\\
\end{enumerate}
\item The probability of obtaining an even prime number on each die, when a pair of dice is rolled is 
\begin{enumerate}
    \item $0$ 
    
    \item $\frac{1}{3}$ 
    
    \item $\frac{1}{12}$ 
    
    \item $\frac{1}{36}$ 
\end{enumerate}
\solution
		%\input{ncert/12/13/2/17/defs.tex}
	\item A bag contains 4 red and 4 black balls, another bag contains 2 red and 6 black balls. One of the two bags is selected at random and a ball is drawn from the bag which is found to be red. Find the probability that the ball is drawn from the first bag.
\\
\solution
		%\input{ncert/12/13/3/2/main.tex}
  \item
  Cards with numbers 2 to 101 are placed in a box. A card is selected at random.Find the probability that the card has
\begin{enumerate}[label=(\roman*)]
	\item an even number 
	\item a square number
\end{enumerate}
\solution
%\input{exemplar/10/13/3/32/main.tex}
\item
The king, queen and jack of clubs are removed from a deck of 52 playing cards and then well shuffled. Now one card is drawn at random from the remaining cards.  Determine the probability that the card is
\begin{enumerate}[label=(\roman*)]
\item a club
\item 10 of hearts
\end{enumerate}
\solution
%\input{exemplar/10/13/3/29/main.tex}
\item A team of medical students doing their internship have to assist during surgeries
at a city hospital. The probabilities of surgeries rated as very complex, complex,
routine, simple or very simple are respectively, 0.15, 0.20, 0.31, 0.26, .08. Find
the probabilities that a particular surgery will be rated
\begin{enumerate}
	\item complex or very complex;
	\item neither very complex nor very simple;
	\item routine or complex
	\item routine or simple
\end{enumerate}
\solution
%\input{exemplar/11/16/3/8(1)/main.tex}
\item A card is selected from a pack of 52 cards.
\begin{enumerate}[label=(\alph*)]
    \item How many points are there in the sample space?
    \item Calculate the probability that the card is an ace of spades.
    \item Calculate the probability that the card is (i) an ace and (ii) black card.
\end{enumerate}
\solution
%\input{exemplar/11/16/3/4/main2.tex}
\item The probability that a non leap year selected at random will contain 53 sundays.
\\
\solution
%\input{exemplar/10/13/1/19/main.tex}
\item One of the four persons John, Rita, Aslam or Gurpreet will be promoted next
month. Consequently the sample space consists of four elementary outcomes
S = {John promoted, Rita promoted, Aslam promoted, Gurpreet promoted}
You are told that the chances of John’s promotion is same as that of Gurpreet,
Rita’s chances of promotion are twice as likely as Johns. Aslam’s chances are
four times that of John.
\begin{enumerate}
	\item Determine
	\begin{enumerate}
		\item P (John promoted)
		\item P (Rita promoted)
		\item P (Aslam promoted)
		\item P (Gurpreet promoted)
	\end{enumerate}
	\item If A = {John promoted or Gurpreet promoted}, find P (A).
\end{enumerate}
\solution
%\input{exemplar/11/16/3/10/main.tex}
\item A card is drawn from a deck of 52 cards. Find the probability of getting a king or a heart or a red card.\\
\solution
%\input{exemplar/11/16/3/15/main.tex}
\item The probability that a student will pass his examination is 0.73, the probability of
the student getting a compartment is 0.13, and the probability that the student will
either pass or get compartment is 0.96. State True or False.\\
\solution
%\input{exemplar/11/16/3/31/main.tex}
\item A card is selected from a pack of 52 cards\\
\begin{enumerate}[label=(\alph*)]
\item How many points are there in the sample space?
\item Calculate the probability that the cards is an ace of spades.
\item Calculate the probability that the card is (i) an ace (ii)black card.\\
\end{enumerate}
%\input{ncert/11/16/3/4_1/Prob_4.tex}
\item In a non-leap year, the probability of having 53 tuesdays or 53 wednesdays is\\
\solution
%\input{exemplar/11/16/3/18/main.tex}
\item There are 1000 sealed envelopes in a box, 10 of them contain a cash prize of
Rs 100 each, 100 of them contain a cash prize of Rs 50 each and 200 of them
contain a cash prize of Rs 10 each and rest do not contain any cash prize. If they
are well shuffled and an envelope is picked up out, what is the probability that it
contains no cash prize?\\
\solution
%\input{exemplar/10/13/3/34/main.tex}
\item 
A die is thrown and a card is selected at random from a deck of 52 playing cards. The probability of getting an even number on the die and a spade card.\\
\solution
%\input{exemplar/12/13/3/78/main.tex}
\item
If 4-digit numbers greater than 5,000 are randomly formed from the digits 0, 1, 3, 5, and 7, what is the probability of forming a number divisible by 5 when:
\begin{enumerate}
    \item The digits are repeated?
    \item The repetition of digits is not allowed?
\end{enumerate}
\solution
%\input{ncert/11/16/4/9/main.tex}
\item Consider the probability space $\brak{\Omega, \mathcal{G}, P}$ where $\Omega = [0,2]$ and $\mathcal{G} = \cbrak{\phi, \Omega, [0,1], (1,2]}$. Let $X$ and $Y$ be two functions on $\Omega$ defined as
\begin{align*}
    X(\omega) = 
    \begin{cases}
        1 & \text{if }\omega \in [0, 1]\\
        2 & \text{if }\omega \in (1, 2]
    \end{cases}
\end{align*}
and
\begin{align*}
    Y(\omega) = 
    \begin{cases}
        2 & \text{if }\omega \in [0, 1.5]\\
        3 & \text{if }\omega \in (1.5, 2].
    \end{cases}
\end{align*}
Then which one of the following statements is true?
\begin{enumerate}
    \item [(A)] $X$ is a random variable with respect to $\mathcal{G}$, but $Y$ is not a random variable with respect to $\mathcal{G}$.
    \item [(B)] $Y$ is a random variable with respect to $\mathcal{G}$, but $X$ is not a random variable with respect to $\mathcal{G}$.
    \item [(C)] Neither $X$ nor $Y$ is a random variable with respect to $\mathcal{G}$.
    \item [(D)] Both $X$ and $Y$ are random variables with respect to $\mathcal{G}$.
\end{enumerate} \hfill (GATE ST 2023)\\
\solution
%\input{gate/ST/2023/14/main.tex}
	\item  A die is loaded in such a way that each odd number is twice as likely to occur as
each even number. Find $P(G)$, where $G$ is the event that a number greater than
3 occurs on a single roll of the die.
\\
\solution
		%\input{exemplar/11/16/3/5/main.tex}
	\item All the jacks, queens and kings are removed from a deck of 52 playing cards. The remaining cards are well shuffled and then one card is drawn at random. Giving ace a value 1 similar value for other cards, find the probability that the card has a value 
		\begin{enumerate}
			\item 7
			\item greater than 7
			\item less than 7
		\end{enumerate}
		%\input{exemplar/10/13/3/30/main.tex}
  \item A Lot consists of 48 mobile phones of which 42 are good, 3 have only minor defects and 3 have major defects.Varnika will buy a phone if it is good but the trader will only buy a mobile if it has no major defects. One phone is selected at random from the lot. What is the probability that it is
\begin{enumerate}
	\item acceptable to Varnika?
            \item acceptable to the trader?
\end{enumerate}
\solution
	%\input{exemplar/10/13/3/40/main.tex}
 \item A student says that if you throw a die, it will show up 1 or not 1. Therefore, the probability of getting 1 and the probability of getting 'not 1' each is equal to $\frac{1}{2}$. Is this correct? Give reasons.\\
 \solution
        %\input{exemplar/10/13/2/9/main.tex}
   \item Four candidates A, B, C, D have ap-
plied for the assignment to coach a school cricket
team. If A is twice as likely to be selected as B, and
B and C are given about the same chance of being
selected, while C is twice as likely to be selected
as D, what are the probabilities that
\begin{enumerate}
\item C will be selected?
\item A will not be selected?
\end{enumerate}
	%\input{exemplar/11/16/3/9/main.tex}
 \item A bag contain 24 balls of which $x$ balls are red, $2x$ are white and $3x$ are blue. A ball is selected at random, What is the probability that it is
\begin{enumerate}[label=\alph*)]
\item not red ?
\item white ?
\end{enumerate}
%\input{exemplar/10/13/3/41/main.tex}
If the letters of the word ASSASSINATION are arranged at random. Find the Probability that
\begin{enumerate}[label=(\alph*)]
\item Four $S's$ come consecutively in the word
\item Two  $I's$ and two $N's$ come together
\item All $A's$ are not coming together
\item No two $A's$ are coming together
\end{enumerate}
%\input{exemplar/11/16/3/14/main.tex}
	\item One urn contains two black balls (labelled B1 and B2) and one white ball. A
	second urn contains one black ball and two white balls (labelled W1 and W2).
	Suppose the following experiment is performed. One of the two urns is chosen
	at random. Next a ball is randomly chosen from the urn. Then a second ball is
	chosen at random from the same urn without replacing the first ball.
	
	\begin{enumerate}
	\item What is the probability that two black balls are chosen?
	
	\item What is the probability that two balls of opposite colour are chosen?
	\end{enumerate}
	\solution
	%\input{exemplar/11/16/3/12/main1.tex}
\end{enumerate}

	\item 
The number lock of a suitcase has 4 wheels each labelled with ten digits i.e. from 0 to 9.The lock opens with a sequence of four digits with no repeats.What is the probability of a person getting the right sequence to open the suitcase.
\\
\solution
		%\begin{enumerate}[label=\thesection.\arabic*,ref=\thesection.\theenumi]
	\item One card is drawn from a well-shuffled deck of 52 cards. Find the probability of getting
\begin{enumerate}
\item A king of red colour 
\item A face card 
\item A red face card
\item The jack of hearts
\item A spade
\item The queen of diamonds

\end{enumerate}
\solution
		%\input{ncert/10/15/1/14/main.tex}
	\item Five cards—the ten, jack, queen, king and ace of diamonds, are well-shuffled with their face downwards. One card is then picked up at random.
\begin{enumerate}
\item
What is the probability that the card is the queen? 
\item
If the queen is drawn and put aside, what is the probability that the second card picked up is (a) an ace? (b) a queen?\\
\end{enumerate}
\solution
		%\input{ncert/10/15/1/15/defs.tex}
	\item A bag contains $5$ red balls and some blue balls. If the probability of drawing a blue ball is double that if a red ball, determine the number of blue balls in the bag. 
		\\
\solution
		%\input{ncert/10/15/2/3/defs.tex}
	\item A card is selected from a pack of 52 cards.
 \begin{enumerate}[label=(\alph*)] 
                 \item How many points are there in the sample space?
                 \item Calculate the probability that the card is an ace of spades.
                 \item Calculate the probability that the card is (i) an ace and (ii) black card.
 \end{enumerate}
\solution
		%\input{ncert/11/16/3/4/main.tex}
\item Four cards are drawn from a well-shuffled deck of 52 cards. What is the probability of obtaining 3 diamonds and one spade.
\\
\solution
		%\input{ncert/11/16/4/2/defs.tex}
\item In a certain lottery 10,000 tickets are sold and ten equal prizes are awarded. What is the probability of not getting a prize if you buy (a) one ticket (b) two tickets (c) 10 tickets ?	
\\
\solution
		%\input{ncert/11/16/4/4/defs.tex}
		%
\item 
Out of 100 students, two sections of 40 and 60 are formed. If you and your friend are among the 100 students, what is the probability that
\begin{enumerate}
\item you both enter the same section?
\item you both enter the different sections?
\end{enumerate}
\solution
		%\input{ncert/11/16/4/5/defs.tex}
	\item 
The number lock of a suitcase has 4 wheels each labelled with ten digits i.e. from 0 to 9.The lock opens with a sequence of four digits with no repeats.What is the probability of a person getting the right sequence to open the suitcase.
\\
\solution
		%\input{ncert/11/16/4/10/defs.tex}
		%
\item 
Two cards are drawn at random and without replacement from a pack of 52 playing cards. Find the probability that both the cards are black.
\\
\solution
		%\input{ncert/12/13/2/2/defs.tex}
		\item A box of oranges is inspected by examining three randomly selected oranges drawn without replacement. If all the three oranges are good, the box is approved for sale, otherwise, it is rejected. Find the probability that a box containing 15 oranges out of which 12 are good and 3 are bad ones will be approved for sale.
		\label{ncert/12/13/2/3/defs.tex}
		\item Two balls are drawn at random with replacement from a box containing 10 black and 8 red balls. Find the probability that
		\label{ncert/12/13/2/12}
\begin{enumerate}
\item both balls are red.
\item first ball is black and second is red.
\item one of them is black and other is red.
\end{enumerate}

\item In a hostel, 60\% of the students read Hindi newspaper, 40\% read English newspaper and 20\% read both Hindi and English newspapers. A student is selected at random.
		\label{ncert/12/13/2/15}
\begin{enumerate}
\item Find the probability that she reads neither Hindi nor English newspapers.
\item If she reads Hindi newspaper, find the probability that she reads English newspaper.
\item If she reads English newspaper, find the probability that she reads Hindi newspaper.\\
\end{enumerate}
\item The probability of obtaining an even prime number on each die, when a pair of dice is rolled is 
\begin{enumerate}
    \item $0$ 
    
    \item $\frac{1}{3}$ 
    
    \item $\frac{1}{12}$ 
    
    \item $\frac{1}{36}$ 
\end{enumerate}
\solution
		%\input{ncert/12/13/2/17/defs.tex}
	\item A bag contains 4 red and 4 black balls, another bag contains 2 red and 6 black balls. One of the two bags is selected at random and a ball is drawn from the bag which is found to be red. Find the probability that the ball is drawn from the first bag.
\\
\solution
		%\input{ncert/12/13/3/2/main.tex}
  \item
  Cards with numbers 2 to 101 are placed in a box. A card is selected at random.Find the probability that the card has
\begin{enumerate}[label=(\roman*)]
	\item an even number 
	\item a square number
\end{enumerate}
\solution
%\input{exemplar/10/13/3/32/main.tex}
\item
The king, queen and jack of clubs are removed from a deck of 52 playing cards and then well shuffled. Now one card is drawn at random from the remaining cards.  Determine the probability that the card is
\begin{enumerate}[label=(\roman*)]
\item a club
\item 10 of hearts
\end{enumerate}
\solution
%\input{exemplar/10/13/3/29/main.tex}
\item A team of medical students doing their internship have to assist during surgeries
at a city hospital. The probabilities of surgeries rated as very complex, complex,
routine, simple or very simple are respectively, 0.15, 0.20, 0.31, 0.26, .08. Find
the probabilities that a particular surgery will be rated
\begin{enumerate}
	\item complex or very complex;
	\item neither very complex nor very simple;
	\item routine or complex
	\item routine or simple
\end{enumerate}
\solution
%\input{exemplar/11/16/3/8(1)/main.tex}
\item A card is selected from a pack of 52 cards.
\begin{enumerate}[label=(\alph*)]
    \item How many points are there in the sample space?
    \item Calculate the probability that the card is an ace of spades.
    \item Calculate the probability that the card is (i) an ace and (ii) black card.
\end{enumerate}
\solution
%\input{exemplar/11/16/3/4/main2.tex}
\item The probability that a non leap year selected at random will contain 53 sundays.
\\
\solution
%\input{exemplar/10/13/1/19/main.tex}
\item One of the four persons John, Rita, Aslam or Gurpreet will be promoted next
month. Consequently the sample space consists of four elementary outcomes
S = {John promoted, Rita promoted, Aslam promoted, Gurpreet promoted}
You are told that the chances of John’s promotion is same as that of Gurpreet,
Rita’s chances of promotion are twice as likely as Johns. Aslam’s chances are
four times that of John.
\begin{enumerate}
	\item Determine
	\begin{enumerate}
		\item P (John promoted)
		\item P (Rita promoted)
		\item P (Aslam promoted)
		\item P (Gurpreet promoted)
	\end{enumerate}
	\item If A = {John promoted or Gurpreet promoted}, find P (A).
\end{enumerate}
\solution
%\input{exemplar/11/16/3/10/main.tex}
\item A card is drawn from a deck of 52 cards. Find the probability of getting a king or a heart or a red card.\\
\solution
%\input{exemplar/11/16/3/15/main.tex}
\item The probability that a student will pass his examination is 0.73, the probability of
the student getting a compartment is 0.13, and the probability that the student will
either pass or get compartment is 0.96. State True or False.\\
\solution
%\input{exemplar/11/16/3/31/main.tex}
\item A card is selected from a pack of 52 cards\\
\begin{enumerate}[label=(\alph*)]
\item How many points are there in the sample space?
\item Calculate the probability that the cards is an ace of spades.
\item Calculate the probability that the card is (i) an ace (ii)black card.\\
\end{enumerate}
%\input{ncert/11/16/3/4_1/Prob_4.tex}
\item In a non-leap year, the probability of having 53 tuesdays or 53 wednesdays is\\
\solution
%\input{exemplar/11/16/3/18/main.tex}
\item There are 1000 sealed envelopes in a box, 10 of them contain a cash prize of
Rs 100 each, 100 of them contain a cash prize of Rs 50 each and 200 of them
contain a cash prize of Rs 10 each and rest do not contain any cash prize. If they
are well shuffled and an envelope is picked up out, what is the probability that it
contains no cash prize?\\
\solution
%\input{exemplar/10/13/3/34/main.tex}
\item 
A die is thrown and a card is selected at random from a deck of 52 playing cards. The probability of getting an even number on the die and a spade card.\\
\solution
%\input{exemplar/12/13/3/78/main.tex}
\item
If 4-digit numbers greater than 5,000 are randomly formed from the digits 0, 1, 3, 5, and 7, what is the probability of forming a number divisible by 5 when:
\begin{enumerate}
    \item The digits are repeated?
    \item The repetition of digits is not allowed?
\end{enumerate}
\solution
%\input{ncert/11/16/4/9/main.tex}
\item Consider the probability space $\brak{\Omega, \mathcal{G}, P}$ where $\Omega = [0,2]$ and $\mathcal{G} = \cbrak{\phi, \Omega, [0,1], (1,2]}$. Let $X$ and $Y$ be two functions on $\Omega$ defined as
\begin{align*}
    X(\omega) = 
    \begin{cases}
        1 & \text{if }\omega \in [0, 1]\\
        2 & \text{if }\omega \in (1, 2]
    \end{cases}
\end{align*}
and
\begin{align*}
    Y(\omega) = 
    \begin{cases}
        2 & \text{if }\omega \in [0, 1.5]\\
        3 & \text{if }\omega \in (1.5, 2].
    \end{cases}
\end{align*}
Then which one of the following statements is true?
\begin{enumerate}
    \item [(A)] $X$ is a random variable with respect to $\mathcal{G}$, but $Y$ is not a random variable with respect to $\mathcal{G}$.
    \item [(B)] $Y$ is a random variable with respect to $\mathcal{G}$, but $X$ is not a random variable with respect to $\mathcal{G}$.
    \item [(C)] Neither $X$ nor $Y$ is a random variable with respect to $\mathcal{G}$.
    \item [(D)] Both $X$ and $Y$ are random variables with respect to $\mathcal{G}$.
\end{enumerate} \hfill (GATE ST 2023)\\
\solution
%\input{gate/ST/2023/14/main.tex}
	\item  A die is loaded in such a way that each odd number is twice as likely to occur as
each even number. Find $P(G)$, where $G$ is the event that a number greater than
3 occurs on a single roll of the die.
\\
\solution
		%\input{exemplar/11/16/3/5/main.tex}
	\item All the jacks, queens and kings are removed from a deck of 52 playing cards. The remaining cards are well shuffled and then one card is drawn at random. Giving ace a value 1 similar value for other cards, find the probability that the card has a value 
		\begin{enumerate}
			\item 7
			\item greater than 7
			\item less than 7
		\end{enumerate}
		%\input{exemplar/10/13/3/30/main.tex}
  \item A Lot consists of 48 mobile phones of which 42 are good, 3 have only minor defects and 3 have major defects.Varnika will buy a phone if it is good but the trader will only buy a mobile if it has no major defects. One phone is selected at random from the lot. What is the probability that it is
\begin{enumerate}
	\item acceptable to Varnika?
            \item acceptable to the trader?
\end{enumerate}
\solution
	%\input{exemplar/10/13/3/40/main.tex}
 \item A student says that if you throw a die, it will show up 1 or not 1. Therefore, the probability of getting 1 and the probability of getting 'not 1' each is equal to $\frac{1}{2}$. Is this correct? Give reasons.\\
 \solution
        %\input{exemplar/10/13/2/9/main.tex}
   \item Four candidates A, B, C, D have ap-
plied for the assignment to coach a school cricket
team. If A is twice as likely to be selected as B, and
B and C are given about the same chance of being
selected, while C is twice as likely to be selected
as D, what are the probabilities that
\begin{enumerate}
\item C will be selected?
\item A will not be selected?
\end{enumerate}
	%\input{exemplar/11/16/3/9/main.tex}
 \item A bag contain 24 balls of which $x$ balls are red, $2x$ are white and $3x$ are blue. A ball is selected at random, What is the probability that it is
\begin{enumerate}[label=\alph*)]
\item not red ?
\item white ?
\end{enumerate}
%\input{exemplar/10/13/3/41/main.tex}
If the letters of the word ASSASSINATION are arranged at random. Find the Probability that
\begin{enumerate}[label=(\alph*)]
\item Four $S's$ come consecutively in the word
\item Two  $I's$ and two $N's$ come together
\item All $A's$ are not coming together
\item No two $A's$ are coming together
\end{enumerate}
%\input{exemplar/11/16/3/14/main.tex}
	\item One urn contains two black balls (labelled B1 and B2) and one white ball. A
	second urn contains one black ball and two white balls (labelled W1 and W2).
	Suppose the following experiment is performed. One of the two urns is chosen
	at random. Next a ball is randomly chosen from the urn. Then a second ball is
	chosen at random from the same urn without replacing the first ball.
	
	\begin{enumerate}
	\item What is the probability that two black balls are chosen?
	
	\item What is the probability that two balls of opposite colour are chosen?
	\end{enumerate}
	\solution
	%\input{exemplar/11/16/3/12/main1.tex}
\end{enumerate}

		%
\item 
Two cards are drawn at random and without replacement from a pack of 52 playing cards. Find the probability that both the cards are black.
\\
\solution
		%\begin{enumerate}[label=\thesection.\arabic*,ref=\thesection.\theenumi]
	\item One card is drawn from a well-shuffled deck of 52 cards. Find the probability of getting
\begin{enumerate}
\item A king of red colour 
\item A face card 
\item A red face card
\item The jack of hearts
\item A spade
\item The queen of diamonds

\end{enumerate}
\solution
		%\input{ncert/10/15/1/14/main.tex}
	\item Five cards—the ten, jack, queen, king and ace of diamonds, are well-shuffled with their face downwards. One card is then picked up at random.
\begin{enumerate}
\item
What is the probability that the card is the queen? 
\item
If the queen is drawn and put aside, what is the probability that the second card picked up is (a) an ace? (b) a queen?\\
\end{enumerate}
\solution
		%\input{ncert/10/15/1/15/defs.tex}
	\item A bag contains $5$ red balls and some blue balls. If the probability of drawing a blue ball is double that if a red ball, determine the number of blue balls in the bag. 
		\\
\solution
		%\input{ncert/10/15/2/3/defs.tex}
	\item A card is selected from a pack of 52 cards.
 \begin{enumerate}[label=(\alph*)] 
                 \item How many points are there in the sample space?
                 \item Calculate the probability that the card is an ace of spades.
                 \item Calculate the probability that the card is (i) an ace and (ii) black card.
 \end{enumerate}
\solution
		%\input{ncert/11/16/3/4/main.tex}
\item Four cards are drawn from a well-shuffled deck of 52 cards. What is the probability of obtaining 3 diamonds and one spade.
\\
\solution
		%\input{ncert/11/16/4/2/defs.tex}
\item In a certain lottery 10,000 tickets are sold and ten equal prizes are awarded. What is the probability of not getting a prize if you buy (a) one ticket (b) two tickets (c) 10 tickets ?	
\\
\solution
		%\input{ncert/11/16/4/4/defs.tex}
		%
\item 
Out of 100 students, two sections of 40 and 60 are formed. If you and your friend are among the 100 students, what is the probability that
\begin{enumerate}
\item you both enter the same section?
\item you both enter the different sections?
\end{enumerate}
\solution
		%\input{ncert/11/16/4/5/defs.tex}
	\item 
The number lock of a suitcase has 4 wheels each labelled with ten digits i.e. from 0 to 9.The lock opens with a sequence of four digits with no repeats.What is the probability of a person getting the right sequence to open the suitcase.
\\
\solution
		%\input{ncert/11/16/4/10/defs.tex}
		%
\item 
Two cards are drawn at random and without replacement from a pack of 52 playing cards. Find the probability that both the cards are black.
\\
\solution
		%\input{ncert/12/13/2/2/defs.tex}
		\item A box of oranges is inspected by examining three randomly selected oranges drawn without replacement. If all the three oranges are good, the box is approved for sale, otherwise, it is rejected. Find the probability that a box containing 15 oranges out of which 12 are good and 3 are bad ones will be approved for sale.
		\label{ncert/12/13/2/3/defs.tex}
		\item Two balls are drawn at random with replacement from a box containing 10 black and 8 red balls. Find the probability that
		\label{ncert/12/13/2/12}
\begin{enumerate}
\item both balls are red.
\item first ball is black and second is red.
\item one of them is black and other is red.
\end{enumerate}

\item In a hostel, 60\% of the students read Hindi newspaper, 40\% read English newspaper and 20\% read both Hindi and English newspapers. A student is selected at random.
		\label{ncert/12/13/2/15}
\begin{enumerate}
\item Find the probability that she reads neither Hindi nor English newspapers.
\item If she reads Hindi newspaper, find the probability that she reads English newspaper.
\item If she reads English newspaper, find the probability that she reads Hindi newspaper.\\
\end{enumerate}
\item The probability of obtaining an even prime number on each die, when a pair of dice is rolled is 
\begin{enumerate}
    \item $0$ 
    
    \item $\frac{1}{3}$ 
    
    \item $\frac{1}{12}$ 
    
    \item $\frac{1}{36}$ 
\end{enumerate}
\solution
		%\input{ncert/12/13/2/17/defs.tex}
	\item A bag contains 4 red and 4 black balls, another bag contains 2 red and 6 black balls. One of the two bags is selected at random and a ball is drawn from the bag which is found to be red. Find the probability that the ball is drawn from the first bag.
\\
\solution
		%\input{ncert/12/13/3/2/main.tex}
  \item
  Cards with numbers 2 to 101 are placed in a box. A card is selected at random.Find the probability that the card has
\begin{enumerate}[label=(\roman*)]
	\item an even number 
	\item a square number
\end{enumerate}
\solution
%\input{exemplar/10/13/3/32/main.tex}
\item
The king, queen and jack of clubs are removed from a deck of 52 playing cards and then well shuffled. Now one card is drawn at random from the remaining cards.  Determine the probability that the card is
\begin{enumerate}[label=(\roman*)]
\item a club
\item 10 of hearts
\end{enumerate}
\solution
%\input{exemplar/10/13/3/29/main.tex}
\item A team of medical students doing their internship have to assist during surgeries
at a city hospital. The probabilities of surgeries rated as very complex, complex,
routine, simple or very simple are respectively, 0.15, 0.20, 0.31, 0.26, .08. Find
the probabilities that a particular surgery will be rated
\begin{enumerate}
	\item complex or very complex;
	\item neither very complex nor very simple;
	\item routine or complex
	\item routine or simple
\end{enumerate}
\solution
%\input{exemplar/11/16/3/8(1)/main.tex}
\item A card is selected from a pack of 52 cards.
\begin{enumerate}[label=(\alph*)]
    \item How many points are there in the sample space?
    \item Calculate the probability that the card is an ace of spades.
    \item Calculate the probability that the card is (i) an ace and (ii) black card.
\end{enumerate}
\solution
%\input{exemplar/11/16/3/4/main2.tex}
\item The probability that a non leap year selected at random will contain 53 sundays.
\\
\solution
%\input{exemplar/10/13/1/19/main.tex}
\item One of the four persons John, Rita, Aslam or Gurpreet will be promoted next
month. Consequently the sample space consists of four elementary outcomes
S = {John promoted, Rita promoted, Aslam promoted, Gurpreet promoted}
You are told that the chances of John’s promotion is same as that of Gurpreet,
Rita’s chances of promotion are twice as likely as Johns. Aslam’s chances are
four times that of John.
\begin{enumerate}
	\item Determine
	\begin{enumerate}
		\item P (John promoted)
		\item P (Rita promoted)
		\item P (Aslam promoted)
		\item P (Gurpreet promoted)
	\end{enumerate}
	\item If A = {John promoted or Gurpreet promoted}, find P (A).
\end{enumerate}
\solution
%\input{exemplar/11/16/3/10/main.tex}
\item A card is drawn from a deck of 52 cards. Find the probability of getting a king or a heart or a red card.\\
\solution
%\input{exemplar/11/16/3/15/main.tex}
\item The probability that a student will pass his examination is 0.73, the probability of
the student getting a compartment is 0.13, and the probability that the student will
either pass or get compartment is 0.96. State True or False.\\
\solution
%\input{exemplar/11/16/3/31/main.tex}
\item A card is selected from a pack of 52 cards\\
\begin{enumerate}[label=(\alph*)]
\item How many points are there in the sample space?
\item Calculate the probability that the cards is an ace of spades.
\item Calculate the probability that the card is (i) an ace (ii)black card.\\
\end{enumerate}
%\input{ncert/11/16/3/4_1/Prob_4.tex}
\item In a non-leap year, the probability of having 53 tuesdays or 53 wednesdays is\\
\solution
%\input{exemplar/11/16/3/18/main.tex}
\item There are 1000 sealed envelopes in a box, 10 of them contain a cash prize of
Rs 100 each, 100 of them contain a cash prize of Rs 50 each and 200 of them
contain a cash prize of Rs 10 each and rest do not contain any cash prize. If they
are well shuffled and an envelope is picked up out, what is the probability that it
contains no cash prize?\\
\solution
%\input{exemplar/10/13/3/34/main.tex}
\item 
A die is thrown and a card is selected at random from a deck of 52 playing cards. The probability of getting an even number on the die and a spade card.\\
\solution
%\input{exemplar/12/13/3/78/main.tex}
\item
If 4-digit numbers greater than 5,000 are randomly formed from the digits 0, 1, 3, 5, and 7, what is the probability of forming a number divisible by 5 when:
\begin{enumerate}
    \item The digits are repeated?
    \item The repetition of digits is not allowed?
\end{enumerate}
\solution
%\input{ncert/11/16/4/9/main.tex}
\item Consider the probability space $\brak{\Omega, \mathcal{G}, P}$ where $\Omega = [0,2]$ and $\mathcal{G} = \cbrak{\phi, \Omega, [0,1], (1,2]}$. Let $X$ and $Y$ be two functions on $\Omega$ defined as
\begin{align*}
    X(\omega) = 
    \begin{cases}
        1 & \text{if }\omega \in [0, 1]\\
        2 & \text{if }\omega \in (1, 2]
    \end{cases}
\end{align*}
and
\begin{align*}
    Y(\omega) = 
    \begin{cases}
        2 & \text{if }\omega \in [0, 1.5]\\
        3 & \text{if }\omega \in (1.5, 2].
    \end{cases}
\end{align*}
Then which one of the following statements is true?
\begin{enumerate}
    \item [(A)] $X$ is a random variable with respect to $\mathcal{G}$, but $Y$ is not a random variable with respect to $\mathcal{G}$.
    \item [(B)] $Y$ is a random variable with respect to $\mathcal{G}$, but $X$ is not a random variable with respect to $\mathcal{G}$.
    \item [(C)] Neither $X$ nor $Y$ is a random variable with respect to $\mathcal{G}$.
    \item [(D)] Both $X$ and $Y$ are random variables with respect to $\mathcal{G}$.
\end{enumerate} \hfill (GATE ST 2023)\\
\solution
%\input{gate/ST/2023/14/main.tex}
	\item  A die is loaded in such a way that each odd number is twice as likely to occur as
each even number. Find $P(G)$, where $G$ is the event that a number greater than
3 occurs on a single roll of the die.
\\
\solution
		%\input{exemplar/11/16/3/5/main.tex}
	\item All the jacks, queens and kings are removed from a deck of 52 playing cards. The remaining cards are well shuffled and then one card is drawn at random. Giving ace a value 1 similar value for other cards, find the probability that the card has a value 
		\begin{enumerate}
			\item 7
			\item greater than 7
			\item less than 7
		\end{enumerate}
		%\input{exemplar/10/13/3/30/main.tex}
  \item A Lot consists of 48 mobile phones of which 42 are good, 3 have only minor defects and 3 have major defects.Varnika will buy a phone if it is good but the trader will only buy a mobile if it has no major defects. One phone is selected at random from the lot. What is the probability that it is
\begin{enumerate}
	\item acceptable to Varnika?
            \item acceptable to the trader?
\end{enumerate}
\solution
	%\input{exemplar/10/13/3/40/main.tex}
 \item A student says that if you throw a die, it will show up 1 or not 1. Therefore, the probability of getting 1 and the probability of getting 'not 1' each is equal to $\frac{1}{2}$. Is this correct? Give reasons.\\
 \solution
        %\input{exemplar/10/13/2/9/main.tex}
   \item Four candidates A, B, C, D have ap-
plied for the assignment to coach a school cricket
team. If A is twice as likely to be selected as B, and
B and C are given about the same chance of being
selected, while C is twice as likely to be selected
as D, what are the probabilities that
\begin{enumerate}
\item C will be selected?
\item A will not be selected?
\end{enumerate}
	%\input{exemplar/11/16/3/9/main.tex}
 \item A bag contain 24 balls of which $x$ balls are red, $2x$ are white and $3x$ are blue. A ball is selected at random, What is the probability that it is
\begin{enumerate}[label=\alph*)]
\item not red ?
\item white ?
\end{enumerate}
%\input{exemplar/10/13/3/41/main.tex}
If the letters of the word ASSASSINATION are arranged at random. Find the Probability that
\begin{enumerate}[label=(\alph*)]
\item Four $S's$ come consecutively in the word
\item Two  $I's$ and two $N's$ come together
\item All $A's$ are not coming together
\item No two $A's$ are coming together
\end{enumerate}
%\input{exemplar/11/16/3/14/main.tex}
	\item One urn contains two black balls (labelled B1 and B2) and one white ball. A
	second urn contains one black ball and two white balls (labelled W1 and W2).
	Suppose the following experiment is performed. One of the two urns is chosen
	at random. Next a ball is randomly chosen from the urn. Then a second ball is
	chosen at random from the same urn without replacing the first ball.
	
	\begin{enumerate}
	\item What is the probability that two black balls are chosen?
	
	\item What is the probability that two balls of opposite colour are chosen?
	\end{enumerate}
	\solution
	%\input{exemplar/11/16/3/12/main1.tex}
\end{enumerate}

		\item A box of oranges is inspected by examining three randomly selected oranges drawn without replacement. If all the three oranges are good, the box is approved for sale, otherwise, it is rejected. Find the probability that a box containing 15 oranges out of which 12 are good and 3 are bad ones will be approved for sale.
		\label{ncert/12/13/2/3/defs.tex}
		\item Two balls are drawn at random with replacement from a box containing 10 black and 8 red balls. Find the probability that
		\label{ncert/12/13/2/12}
\begin{enumerate}
\item both balls are red.
\item first ball is black and second is red.
\item one of them is black and other is red.
\end{enumerate}

\item In a hostel, 60\% of the students read Hindi newspaper, 40\% read English newspaper and 20\% read both Hindi and English newspapers. A student is selected at random.
		\label{ncert/12/13/2/15}
\begin{enumerate}
\item Find the probability that she reads neither Hindi nor English newspapers.
\item If she reads Hindi newspaper, find the probability that she reads English newspaper.
\item If she reads English newspaper, find the probability that she reads Hindi newspaper.\\
\end{enumerate}
\item The probability of obtaining an even prime number on each die, when a pair of dice is rolled is 
\begin{enumerate}
    \item $0$ 
    
    \item $\frac{1}{3}$ 
    
    \item $\frac{1}{12}$ 
    
    \item $\frac{1}{36}$ 
\end{enumerate}
\solution
		%\begin{enumerate}[label=\thesection.\arabic*,ref=\thesection.\theenumi]
	\item One card is drawn from a well-shuffled deck of 52 cards. Find the probability of getting
\begin{enumerate}
\item A king of red colour 
\item A face card 
\item A red face card
\item The jack of hearts
\item A spade
\item The queen of diamonds

\end{enumerate}
\solution
		%\input{ncert/10/15/1/14/main.tex}
	\item Five cards—the ten, jack, queen, king and ace of diamonds, are well-shuffled with their face downwards. One card is then picked up at random.
\begin{enumerate}
\item
What is the probability that the card is the queen? 
\item
If the queen is drawn and put aside, what is the probability that the second card picked up is (a) an ace? (b) a queen?\\
\end{enumerate}
\solution
		%\input{ncert/10/15/1/15/defs.tex}
	\item A bag contains $5$ red balls and some blue balls. If the probability of drawing a blue ball is double that if a red ball, determine the number of blue balls in the bag. 
		\\
\solution
		%\input{ncert/10/15/2/3/defs.tex}
	\item A card is selected from a pack of 52 cards.
 \begin{enumerate}[label=(\alph*)] 
                 \item How many points are there in the sample space?
                 \item Calculate the probability that the card is an ace of spades.
                 \item Calculate the probability that the card is (i) an ace and (ii) black card.
 \end{enumerate}
\solution
		%\input{ncert/11/16/3/4/main.tex}
\item Four cards are drawn from a well-shuffled deck of 52 cards. What is the probability of obtaining 3 diamonds and one spade.
\\
\solution
		%\input{ncert/11/16/4/2/defs.tex}
\item In a certain lottery 10,000 tickets are sold and ten equal prizes are awarded. What is the probability of not getting a prize if you buy (a) one ticket (b) two tickets (c) 10 tickets ?	
\\
\solution
		%\input{ncert/11/16/4/4/defs.tex}
		%
\item 
Out of 100 students, two sections of 40 and 60 are formed. If you and your friend are among the 100 students, what is the probability that
\begin{enumerate}
\item you both enter the same section?
\item you both enter the different sections?
\end{enumerate}
\solution
		%\input{ncert/11/16/4/5/defs.tex}
	\item 
The number lock of a suitcase has 4 wheels each labelled with ten digits i.e. from 0 to 9.The lock opens with a sequence of four digits with no repeats.What is the probability of a person getting the right sequence to open the suitcase.
\\
\solution
		%\input{ncert/11/16/4/10/defs.tex}
		%
\item 
Two cards are drawn at random and without replacement from a pack of 52 playing cards. Find the probability that both the cards are black.
\\
\solution
		%\input{ncert/12/13/2/2/defs.tex}
		\item A box of oranges is inspected by examining three randomly selected oranges drawn without replacement. If all the three oranges are good, the box is approved for sale, otherwise, it is rejected. Find the probability that a box containing 15 oranges out of which 12 are good and 3 are bad ones will be approved for sale.
		\label{ncert/12/13/2/3/defs.tex}
		\item Two balls are drawn at random with replacement from a box containing 10 black and 8 red balls. Find the probability that
		\label{ncert/12/13/2/12}
\begin{enumerate}
\item both balls are red.
\item first ball is black and second is red.
\item one of them is black and other is red.
\end{enumerate}

\item In a hostel, 60\% of the students read Hindi newspaper, 40\% read English newspaper and 20\% read both Hindi and English newspapers. A student is selected at random.
		\label{ncert/12/13/2/15}
\begin{enumerate}
\item Find the probability that she reads neither Hindi nor English newspapers.
\item If she reads Hindi newspaper, find the probability that she reads English newspaper.
\item If she reads English newspaper, find the probability that she reads Hindi newspaper.\\
\end{enumerate}
\item The probability of obtaining an even prime number on each die, when a pair of dice is rolled is 
\begin{enumerate}
    \item $0$ 
    
    \item $\frac{1}{3}$ 
    
    \item $\frac{1}{12}$ 
    
    \item $\frac{1}{36}$ 
\end{enumerate}
\solution
		%\input{ncert/12/13/2/17/defs.tex}
	\item A bag contains 4 red and 4 black balls, another bag contains 2 red and 6 black balls. One of the two bags is selected at random and a ball is drawn from the bag which is found to be red. Find the probability that the ball is drawn from the first bag.
\\
\solution
		%\input{ncert/12/13/3/2/main.tex}
  \item
  Cards with numbers 2 to 101 are placed in a box. A card is selected at random.Find the probability that the card has
\begin{enumerate}[label=(\roman*)]
	\item an even number 
	\item a square number
\end{enumerate}
\solution
%\input{exemplar/10/13/3/32/main.tex}
\item
The king, queen and jack of clubs are removed from a deck of 52 playing cards and then well shuffled. Now one card is drawn at random from the remaining cards.  Determine the probability that the card is
\begin{enumerate}[label=(\roman*)]
\item a club
\item 10 of hearts
\end{enumerate}
\solution
%\input{exemplar/10/13/3/29/main.tex}
\item A team of medical students doing their internship have to assist during surgeries
at a city hospital. The probabilities of surgeries rated as very complex, complex,
routine, simple or very simple are respectively, 0.15, 0.20, 0.31, 0.26, .08. Find
the probabilities that a particular surgery will be rated
\begin{enumerate}
	\item complex or very complex;
	\item neither very complex nor very simple;
	\item routine or complex
	\item routine or simple
\end{enumerate}
\solution
%\input{exemplar/11/16/3/8(1)/main.tex}
\item A card is selected from a pack of 52 cards.
\begin{enumerate}[label=(\alph*)]
    \item How many points are there in the sample space?
    \item Calculate the probability that the card is an ace of spades.
    \item Calculate the probability that the card is (i) an ace and (ii) black card.
\end{enumerate}
\solution
%\input{exemplar/11/16/3/4/main2.tex}
\item The probability that a non leap year selected at random will contain 53 sundays.
\\
\solution
%\input{exemplar/10/13/1/19/main.tex}
\item One of the four persons John, Rita, Aslam or Gurpreet will be promoted next
month. Consequently the sample space consists of four elementary outcomes
S = {John promoted, Rita promoted, Aslam promoted, Gurpreet promoted}
You are told that the chances of John’s promotion is same as that of Gurpreet,
Rita’s chances of promotion are twice as likely as Johns. Aslam’s chances are
four times that of John.
\begin{enumerate}
	\item Determine
	\begin{enumerate}
		\item P (John promoted)
		\item P (Rita promoted)
		\item P (Aslam promoted)
		\item P (Gurpreet promoted)
	\end{enumerate}
	\item If A = {John promoted or Gurpreet promoted}, find P (A).
\end{enumerate}
\solution
%\input{exemplar/11/16/3/10/main.tex}
\item A card is drawn from a deck of 52 cards. Find the probability of getting a king or a heart or a red card.\\
\solution
%\input{exemplar/11/16/3/15/main.tex}
\item The probability that a student will pass his examination is 0.73, the probability of
the student getting a compartment is 0.13, and the probability that the student will
either pass or get compartment is 0.96. State True or False.\\
\solution
%\input{exemplar/11/16/3/31/main.tex}
\item A card is selected from a pack of 52 cards\\
\begin{enumerate}[label=(\alph*)]
\item How many points are there in the sample space?
\item Calculate the probability that the cards is an ace of spades.
\item Calculate the probability that the card is (i) an ace (ii)black card.\\
\end{enumerate}
%\input{ncert/11/16/3/4_1/Prob_4.tex}
\item In a non-leap year, the probability of having 53 tuesdays or 53 wednesdays is\\
\solution
%\input{exemplar/11/16/3/18/main.tex}
\item There are 1000 sealed envelopes in a box, 10 of them contain a cash prize of
Rs 100 each, 100 of them contain a cash prize of Rs 50 each and 200 of them
contain a cash prize of Rs 10 each and rest do not contain any cash prize. If they
are well shuffled and an envelope is picked up out, what is the probability that it
contains no cash prize?\\
\solution
%\input{exemplar/10/13/3/34/main.tex}
\item 
A die is thrown and a card is selected at random from a deck of 52 playing cards. The probability of getting an even number on the die and a spade card.\\
\solution
%\input{exemplar/12/13/3/78/main.tex}
\item
If 4-digit numbers greater than 5,000 are randomly formed from the digits 0, 1, 3, 5, and 7, what is the probability of forming a number divisible by 5 when:
\begin{enumerate}
    \item The digits are repeated?
    \item The repetition of digits is not allowed?
\end{enumerate}
\solution
%\input{ncert/11/16/4/9/main.tex}
\item Consider the probability space $\brak{\Omega, \mathcal{G}, P}$ where $\Omega = [0,2]$ and $\mathcal{G} = \cbrak{\phi, \Omega, [0,1], (1,2]}$. Let $X$ and $Y$ be two functions on $\Omega$ defined as
\begin{align*}
    X(\omega) = 
    \begin{cases}
        1 & \text{if }\omega \in [0, 1]\\
        2 & \text{if }\omega \in (1, 2]
    \end{cases}
\end{align*}
and
\begin{align*}
    Y(\omega) = 
    \begin{cases}
        2 & \text{if }\omega \in [0, 1.5]\\
        3 & \text{if }\omega \in (1.5, 2].
    \end{cases}
\end{align*}
Then which one of the following statements is true?
\begin{enumerate}
    \item [(A)] $X$ is a random variable with respect to $\mathcal{G}$, but $Y$ is not a random variable with respect to $\mathcal{G}$.
    \item [(B)] $Y$ is a random variable with respect to $\mathcal{G}$, but $X$ is not a random variable with respect to $\mathcal{G}$.
    \item [(C)] Neither $X$ nor $Y$ is a random variable with respect to $\mathcal{G}$.
    \item [(D)] Both $X$ and $Y$ are random variables with respect to $\mathcal{G}$.
\end{enumerate} \hfill (GATE ST 2023)\\
\solution
%\input{gate/ST/2023/14/main.tex}
	\item  A die is loaded in such a way that each odd number is twice as likely to occur as
each even number. Find $P(G)$, where $G$ is the event that a number greater than
3 occurs on a single roll of the die.
\\
\solution
		%\input{exemplar/11/16/3/5/main.tex}
	\item All the jacks, queens and kings are removed from a deck of 52 playing cards. The remaining cards are well shuffled and then one card is drawn at random. Giving ace a value 1 similar value for other cards, find the probability that the card has a value 
		\begin{enumerate}
			\item 7
			\item greater than 7
			\item less than 7
		\end{enumerate}
		%\input{exemplar/10/13/3/30/main.tex}
  \item A Lot consists of 48 mobile phones of which 42 are good, 3 have only minor defects and 3 have major defects.Varnika will buy a phone if it is good but the trader will only buy a mobile if it has no major defects. One phone is selected at random from the lot. What is the probability that it is
\begin{enumerate}
	\item acceptable to Varnika?
            \item acceptable to the trader?
\end{enumerate}
\solution
	%\input{exemplar/10/13/3/40/main.tex}
 \item A student says that if you throw a die, it will show up 1 or not 1. Therefore, the probability of getting 1 and the probability of getting 'not 1' each is equal to $\frac{1}{2}$. Is this correct? Give reasons.\\
 \solution
        %\input{exemplar/10/13/2/9/main.tex}
   \item Four candidates A, B, C, D have ap-
plied for the assignment to coach a school cricket
team. If A is twice as likely to be selected as B, and
B and C are given about the same chance of being
selected, while C is twice as likely to be selected
as D, what are the probabilities that
\begin{enumerate}
\item C will be selected?
\item A will not be selected?
\end{enumerate}
	%\input{exemplar/11/16/3/9/main.tex}
 \item A bag contain 24 balls of which $x$ balls are red, $2x$ are white and $3x$ are blue. A ball is selected at random, What is the probability that it is
\begin{enumerate}[label=\alph*)]
\item not red ?
\item white ?
\end{enumerate}
%\input{exemplar/10/13/3/41/main.tex}
If the letters of the word ASSASSINATION are arranged at random. Find the Probability that
\begin{enumerate}[label=(\alph*)]
\item Four $S's$ come consecutively in the word
\item Two  $I's$ and two $N's$ come together
\item All $A's$ are not coming together
\item No two $A's$ are coming together
\end{enumerate}
%\input{exemplar/11/16/3/14/main.tex}
	\item One urn contains two black balls (labelled B1 and B2) and one white ball. A
	second urn contains one black ball and two white balls (labelled W1 and W2).
	Suppose the following experiment is performed. One of the two urns is chosen
	at random. Next a ball is randomly chosen from the urn. Then a second ball is
	chosen at random from the same urn without replacing the first ball.
	
	\begin{enumerate}
	\item What is the probability that two black balls are chosen?
	
	\item What is the probability that two balls of opposite colour are chosen?
	\end{enumerate}
	\solution
	%\input{exemplar/11/16/3/12/main1.tex}
\end{enumerate}

	\item A bag contains 4 red and 4 black balls, another bag contains 2 red and 6 black balls. One of the two bags is selected at random and a ball is drawn from the bag which is found to be red. Find the probability that the ball is drawn from the first bag.
\\
\solution
		%\begin{table}[H]
	\centering
\begin{tabular}{|c|c|c|}
\hline
Random variable &Value &Definition\\ \hline
\multirow{3}{*}{X} &0 &Slips of Rs 1\\
&1 &Slips of Rs 5\\
&2 &Slips of Rs 13\\ \hline
\multirow{2}{*}{Y} &0 &Box A\\
&1 &Box B\\\hline
\end{tabular}
\caption{}
\label{tab:Distribution}
\end{table}
See \tabref{tab:Distribution}.
\begin{align}
p_{Y}\brak{k}= \begin{cases} 
      \frac{1}{3} & {k=0} \\
      \frac{2}{3 }& {k=1} 
   \end{cases}
   \\
p_{Y|X}\brak{0|0} = \frac{19}{25}\, 
p_{Y|X}\brak{0|1} = \frac{6}{25}\,
p_{Y|X}\brak{1|0} = \frac{45}{50}\,
p_{Y|X}\brak{1|2} = \frac{5}{50}
\end{align}
The desired probability is the probability that a slip drawn at random is marked other than Rs 1,
\begin{align}
&=1-p_X\brak{0}\\
&= p_X(1) + p_X(2)
\end{align}
Using Bayes theorem,
\begin{align}
&= p_Y\brak{0} \times \pr{Y=0 | X=1} + p_Y\brak{1} \times \pr{Y=1|X=2}\\
&=\frac{1}{3} \times \frac{6}{25} + \frac{2}{3} \times \frac{5}{50}\\
&=\frac{11}{75}
\end{align}

\newpage

%\tableofcontents

\bigskip

\renewcommand{\thefigure}{\theenumi}
\renewcommand{\thetable}{\theenumi}
%\renewcommand{\theequation}{\theenumi}

%\begin{abstract}
%%\boldmath
%In this letter, an algorithm for evaluating the exact analytical bit error rate  (BER)  for the piecewise linear (PL) combiner for  multiple relays is presented. Previous results were available only for upto three relays. The algorithm is unique in the sense that  the actual mathematical expressions, that are prohibitively large, need not be explicitly obtained. The diversity gain due to multiple relays is shown through plots of the analytical BER, well supported by simulations. 
%
%\end{abstract}
% IEEEtran.cls defaults to using nonbold math in the Abstract.
% This preserves the distinction between vectors and scalars. However,
% if the journal you are submitting to favors bold math in the abstract,
% then you can use LaTeX's standard command \boldmath at the very start
% of the abstract to achieve this. Many IEEE journals frown on math
% in the abstract anyway.

% Note that keywords are not normally used for peerreview papers.
%\begin{IEEEkeywords}
%Cooperative diversity, decode and forward, piecewise linear
%\end{IEEEkeywords}



% For peer review papers, you can put extra information on the cover
% page as needed:
% \ifCLASSOPTIONpeerreview
% \begin{center} \bfseries EDICS Category: 3-BBND \end{center}
% \fi
%
% For peerreview papers, this IEEEtran command inserts a page break and
% creates the second title. It will be ignored for other modes.
%\IEEEpeerreviewmaketitle




  \item
  Cards with numbers 2 to 101 are placed in a box. A card is selected at random.Find the probability that the card has
\begin{enumerate}[label=(\roman*)]
	\item an even number 
	\item a square number
\end{enumerate}
\solution
%\begin{table}[H]
	\centering
\begin{tabular}{|c|c|c|}
\hline
Random variable &Value &Definition\\ \hline
\multirow{3}{*}{X} &0 &Slips of Rs 1\\
&1 &Slips of Rs 5\\
&2 &Slips of Rs 13\\ \hline
\multirow{2}{*}{Y} &0 &Box A\\
&1 &Box B\\\hline
\end{tabular}
\caption{}
\label{tab:Distribution}
\end{table}
See \tabref{tab:Distribution}.
\begin{align}
p_{Y}\brak{k}= \begin{cases} 
      \frac{1}{3} & {k=0} \\
      \frac{2}{3 }& {k=1} 
   \end{cases}
   \\
p_{Y|X}\brak{0|0} = \frac{19}{25}\, 
p_{Y|X}\brak{0|1} = \frac{6}{25}\,
p_{Y|X}\brak{1|0} = \frac{45}{50}\,
p_{Y|X}\brak{1|2} = \frac{5}{50}
\end{align}
The desired probability is the probability that a slip drawn at random is marked other than Rs 1,
\begin{align}
&=1-p_X\brak{0}\\
&= p_X(1) + p_X(2)
\end{align}
Using Bayes theorem,
\begin{align}
&= p_Y\brak{0} \times \pr{Y=0 | X=1} + p_Y\brak{1} \times \pr{Y=1|X=2}\\
&=\frac{1}{3} \times \frac{6}{25} + \frac{2}{3} \times \frac{5}{50}\\
&=\frac{11}{75}
\end{align}

\newpage

%\tableofcontents

\bigskip

\renewcommand{\thefigure}{\theenumi}
\renewcommand{\thetable}{\theenumi}
%\renewcommand{\theequation}{\theenumi}

%\begin{abstract}
%%\boldmath
%In this letter, an algorithm for evaluating the exact analytical bit error rate  (BER)  for the piecewise linear (PL) combiner for  multiple relays is presented. Previous results were available only for upto three relays. The algorithm is unique in the sense that  the actual mathematical expressions, that are prohibitively large, need not be explicitly obtained. The diversity gain due to multiple relays is shown through plots of the analytical BER, well supported by simulations. 
%
%\end{abstract}
% IEEEtran.cls defaults to using nonbold math in the Abstract.
% This preserves the distinction between vectors and scalars. However,
% if the journal you are submitting to favors bold math in the abstract,
% then you can use LaTeX's standard command \boldmath at the very start
% of the abstract to achieve this. Many IEEE journals frown on math
% in the abstract anyway.

% Note that keywords are not normally used for peerreview papers.
%\begin{IEEEkeywords}
%Cooperative diversity, decode and forward, piecewise linear
%\end{IEEEkeywords}



% For peer review papers, you can put extra information on the cover
% page as needed:
% \ifCLASSOPTIONpeerreview
% \begin{center} \bfseries EDICS Category: 3-BBND \end{center}
% \fi
%
% For peerreview papers, this IEEEtran command inserts a page break and
% creates the second title. It will be ignored for other modes.
%\IEEEpeerreviewmaketitle




\item
The king, queen and jack of clubs are removed from a deck of 52 playing cards and then well shuffled. Now one card is drawn at random from the remaining cards.  Determine the probability that the card is
\begin{enumerate}[label=(\roman*)]
\item a club
\item 10 of hearts
\end{enumerate}
\solution
%\begin{table}[H]
	\centering
\begin{tabular}{|c|c|c|}
\hline
Random variable &Value &Definition\\ \hline
\multirow{3}{*}{X} &0 &Slips of Rs 1\\
&1 &Slips of Rs 5\\
&2 &Slips of Rs 13\\ \hline
\multirow{2}{*}{Y} &0 &Box A\\
&1 &Box B\\\hline
\end{tabular}
\caption{}
\label{tab:Distribution}
\end{table}
See \tabref{tab:Distribution}.
\begin{align}
p_{Y}\brak{k}= \begin{cases} 
      \frac{1}{3} & {k=0} \\
      \frac{2}{3 }& {k=1} 
   \end{cases}
   \\
p_{Y|X}\brak{0|0} = \frac{19}{25}\, 
p_{Y|X}\brak{0|1} = \frac{6}{25}\,
p_{Y|X}\brak{1|0} = \frac{45}{50}\,
p_{Y|X}\brak{1|2} = \frac{5}{50}
\end{align}
The desired probability is the probability that a slip drawn at random is marked other than Rs 1,
\begin{align}
&=1-p_X\brak{0}\\
&= p_X(1) + p_X(2)
\end{align}
Using Bayes theorem,
\begin{align}
&= p_Y\brak{0} \times \pr{Y=0 | X=1} + p_Y\brak{1} \times \pr{Y=1|X=2}\\
&=\frac{1}{3} \times \frac{6}{25} + \frac{2}{3} \times \frac{5}{50}\\
&=\frac{11}{75}
\end{align}

\newpage

%\tableofcontents

\bigskip

\renewcommand{\thefigure}{\theenumi}
\renewcommand{\thetable}{\theenumi}
%\renewcommand{\theequation}{\theenumi}

%\begin{abstract}
%%\boldmath
%In this letter, an algorithm for evaluating the exact analytical bit error rate  (BER)  for the piecewise linear (PL) combiner for  multiple relays is presented. Previous results were available only for upto three relays. The algorithm is unique in the sense that  the actual mathematical expressions, that are prohibitively large, need not be explicitly obtained. The diversity gain due to multiple relays is shown through plots of the analytical BER, well supported by simulations. 
%
%\end{abstract}
% IEEEtran.cls defaults to using nonbold math in the Abstract.
% This preserves the distinction between vectors and scalars. However,
% if the journal you are submitting to favors bold math in the abstract,
% then you can use LaTeX's standard command \boldmath at the very start
% of the abstract to achieve this. Many IEEE journals frown on math
% in the abstract anyway.

% Note that keywords are not normally used for peerreview papers.
%\begin{IEEEkeywords}
%Cooperative diversity, decode and forward, piecewise linear
%\end{IEEEkeywords}



% For peer review papers, you can put extra information on the cover
% page as needed:
% \ifCLASSOPTIONpeerreview
% \begin{center} \bfseries EDICS Category: 3-BBND \end{center}
% \fi
%
% For peerreview papers, this IEEEtran command inserts a page break and
% creates the second title. It will be ignored for other modes.
%\IEEEpeerreviewmaketitle




\item A team of medical students doing their internship have to assist during surgeries
at a city hospital. The probabilities of surgeries rated as very complex, complex,
routine, simple or very simple are respectively, 0.15, 0.20, 0.31, 0.26, .08. Find
the probabilities that a particular surgery will be rated
\begin{enumerate}
	\item complex or very complex;
	\item neither very complex nor very simple;
	\item routine or complex
	\item routine or simple
\end{enumerate}
\solution
%\begin{table}[H]
	\centering
\begin{tabular}{|c|c|c|}
\hline
Random variable &Value &Definition\\ \hline
\multirow{3}{*}{X} &0 &Slips of Rs 1\\
&1 &Slips of Rs 5\\
&2 &Slips of Rs 13\\ \hline
\multirow{2}{*}{Y} &0 &Box A\\
&1 &Box B\\\hline
\end{tabular}
\caption{}
\label{tab:Distribution}
\end{table}
See \tabref{tab:Distribution}.
\begin{align}
p_{Y}\brak{k}= \begin{cases} 
      \frac{1}{3} & {k=0} \\
      \frac{2}{3 }& {k=1} 
   \end{cases}
   \\
p_{Y|X}\brak{0|0} = \frac{19}{25}\, 
p_{Y|X}\brak{0|1} = \frac{6}{25}\,
p_{Y|X}\brak{1|0} = \frac{45}{50}\,
p_{Y|X}\brak{1|2} = \frac{5}{50}
\end{align}
The desired probability is the probability that a slip drawn at random is marked other than Rs 1,
\begin{align}
&=1-p_X\brak{0}\\
&= p_X(1) + p_X(2)
\end{align}
Using Bayes theorem,
\begin{align}
&= p_Y\brak{0} \times \pr{Y=0 | X=1} + p_Y\brak{1} \times \pr{Y=1|X=2}\\
&=\frac{1}{3} \times \frac{6}{25} + \frac{2}{3} \times \frac{5}{50}\\
&=\frac{11}{75}
\end{align}

\newpage

%\tableofcontents

\bigskip

\renewcommand{\thefigure}{\theenumi}
\renewcommand{\thetable}{\theenumi}
%\renewcommand{\theequation}{\theenumi}

%\begin{abstract}
%%\boldmath
%In this letter, an algorithm for evaluating the exact analytical bit error rate  (BER)  for the piecewise linear (PL) combiner for  multiple relays is presented. Previous results were available only for upto three relays. The algorithm is unique in the sense that  the actual mathematical expressions, that are prohibitively large, need not be explicitly obtained. The diversity gain due to multiple relays is shown through plots of the analytical BER, well supported by simulations. 
%
%\end{abstract}
% IEEEtran.cls defaults to using nonbold math in the Abstract.
% This preserves the distinction between vectors and scalars. However,
% if the journal you are submitting to favors bold math in the abstract,
% then you can use LaTeX's standard command \boldmath at the very start
% of the abstract to achieve this. Many IEEE journals frown on math
% in the abstract anyway.

% Note that keywords are not normally used for peerreview papers.
%\begin{IEEEkeywords}
%Cooperative diversity, decode and forward, piecewise linear
%\end{IEEEkeywords}



% For peer review papers, you can put extra information on the cover
% page as needed:
% \ifCLASSOPTIONpeerreview
% \begin{center} \bfseries EDICS Category: 3-BBND \end{center}
% \fi
%
% For peerreview papers, this IEEEtran command inserts a page break and
% creates the second title. It will be ignored for other modes.
%\IEEEpeerreviewmaketitle




\item A card is selected from a pack of 52 cards.
\begin{enumerate}[label=(\alph*)]
    \item How many points are there in the sample space?
    \item Calculate the probability that the card is an ace of spades.
    \item Calculate the probability that the card is (i) an ace and (ii) black card.
\end{enumerate}
\solution
%Let $X$ be an bernoulli rv defined as in \tabref{tab:exemplar/11/16/3/26}.  Then, 
\begin{equation}
    p =
        \frac{4}{11} 
\end{equation}
\begin{table}[H]
	\centering
	\input{exemplar/11/16/3/26/tables/Table2.tex}
	\caption{}
        \label{tab:exemplar/11/16/3/26}
\end{table}

\item The probability that a non leap year selected at random will contain 53 sundays.
\\
\solution
%\begin{table}[H]
	\centering
\begin{tabular}{|c|c|c|}
\hline
Random variable &Value &Definition\\ \hline
\multirow{3}{*}{X} &0 &Slips of Rs 1\\
&1 &Slips of Rs 5\\
&2 &Slips of Rs 13\\ \hline
\multirow{2}{*}{Y} &0 &Box A\\
&1 &Box B\\\hline
\end{tabular}
\caption{}
\label{tab:Distribution}
\end{table}
See \tabref{tab:Distribution}.
\begin{align}
p_{Y}\brak{k}= \begin{cases} 
      \frac{1}{3} & {k=0} \\
      \frac{2}{3 }& {k=1} 
   \end{cases}
   \\
p_{Y|X}\brak{0|0} = \frac{19}{25}\, 
p_{Y|X}\brak{0|1} = \frac{6}{25}\,
p_{Y|X}\brak{1|0} = \frac{45}{50}\,
p_{Y|X}\brak{1|2} = \frac{5}{50}
\end{align}
The desired probability is the probability that a slip drawn at random is marked other than Rs 1,
\begin{align}
&=1-p_X\brak{0}\\
&= p_X(1) + p_X(2)
\end{align}
Using Bayes theorem,
\begin{align}
&= p_Y\brak{0} \times \pr{Y=0 | X=1} + p_Y\brak{1} \times \pr{Y=1|X=2}\\
&=\frac{1}{3} \times \frac{6}{25} + \frac{2}{3} \times \frac{5}{50}\\
&=\frac{11}{75}
\end{align}

\newpage

%\tableofcontents

\bigskip

\renewcommand{\thefigure}{\theenumi}
\renewcommand{\thetable}{\theenumi}
%\renewcommand{\theequation}{\theenumi}

%\begin{abstract}
%%\boldmath
%In this letter, an algorithm for evaluating the exact analytical bit error rate  (BER)  for the piecewise linear (PL) combiner for  multiple relays is presented. Previous results were available only for upto three relays. The algorithm is unique in the sense that  the actual mathematical expressions, that are prohibitively large, need not be explicitly obtained. The diversity gain due to multiple relays is shown through plots of the analytical BER, well supported by simulations. 
%
%\end{abstract}
% IEEEtran.cls defaults to using nonbold math in the Abstract.
% This preserves the distinction between vectors and scalars. However,
% if the journal you are submitting to favors bold math in the abstract,
% then you can use LaTeX's standard command \boldmath at the very start
% of the abstract to achieve this. Many IEEE journals frown on math
% in the abstract anyway.

% Note that keywords are not normally used for peerreview papers.
%\begin{IEEEkeywords}
%Cooperative diversity, decode and forward, piecewise linear
%\end{IEEEkeywords}



% For peer review papers, you can put extra information on the cover
% page as needed:
% \ifCLASSOPTIONpeerreview
% \begin{center} \bfseries EDICS Category: 3-BBND \end{center}
% \fi
%
% For peerreview papers, this IEEEtran command inserts a page break and
% creates the second title. It will be ignored for other modes.
%\IEEEpeerreviewmaketitle




\item One of the four persons John, Rita, Aslam or Gurpreet will be promoted next
month. Consequently the sample space consists of four elementary outcomes
S = {John promoted, Rita promoted, Aslam promoted, Gurpreet promoted}
You are told that the chances of John’s promotion is same as that of Gurpreet,
Rita’s chances of promotion are twice as likely as Johns. Aslam’s chances are
four times that of John.
\begin{enumerate}
	\item Determine
	\begin{enumerate}
		\item P (John promoted)
		\item P (Rita promoted)
		\item P (Aslam promoted)
		\item P (Gurpreet promoted)
	\end{enumerate}
	\item If A = {John promoted or Gurpreet promoted}, find P (A).
\end{enumerate}
\solution
%\begin{table}[H]
	\centering
\begin{tabular}{|c|c|c|}
\hline
Random variable &Value &Definition\\ \hline
\multirow{3}{*}{X} &0 &Slips of Rs 1\\
&1 &Slips of Rs 5\\
&2 &Slips of Rs 13\\ \hline
\multirow{2}{*}{Y} &0 &Box A\\
&1 &Box B\\\hline
\end{tabular}
\caption{}
\label{tab:Distribution}
\end{table}
See \tabref{tab:Distribution}.
\begin{align}
p_{Y}\brak{k}= \begin{cases} 
      \frac{1}{3} & {k=0} \\
      \frac{2}{3 }& {k=1} 
   \end{cases}
   \\
p_{Y|X}\brak{0|0} = \frac{19}{25}\, 
p_{Y|X}\brak{0|1} = \frac{6}{25}\,
p_{Y|X}\brak{1|0} = \frac{45}{50}\,
p_{Y|X}\brak{1|2} = \frac{5}{50}
\end{align}
The desired probability is the probability that a slip drawn at random is marked other than Rs 1,
\begin{align}
&=1-p_X\brak{0}\\
&= p_X(1) + p_X(2)
\end{align}
Using Bayes theorem,
\begin{align}
&= p_Y\brak{0} \times \pr{Y=0 | X=1} + p_Y\brak{1} \times \pr{Y=1|X=2}\\
&=\frac{1}{3} \times \frac{6}{25} + \frac{2}{3} \times \frac{5}{50}\\
&=\frac{11}{75}
\end{align}

\newpage

%\tableofcontents

\bigskip

\renewcommand{\thefigure}{\theenumi}
\renewcommand{\thetable}{\theenumi}
%\renewcommand{\theequation}{\theenumi}

%\begin{abstract}
%%\boldmath
%In this letter, an algorithm for evaluating the exact analytical bit error rate  (BER)  for the piecewise linear (PL) combiner for  multiple relays is presented. Previous results were available only for upto three relays. The algorithm is unique in the sense that  the actual mathematical expressions, that are prohibitively large, need not be explicitly obtained. The diversity gain due to multiple relays is shown through plots of the analytical BER, well supported by simulations. 
%
%\end{abstract}
% IEEEtran.cls defaults to using nonbold math in the Abstract.
% This preserves the distinction between vectors and scalars. However,
% if the journal you are submitting to favors bold math in the abstract,
% then you can use LaTeX's standard command \boldmath at the very start
% of the abstract to achieve this. Many IEEE journals frown on math
% in the abstract anyway.

% Note that keywords are not normally used for peerreview papers.
%\begin{IEEEkeywords}
%Cooperative diversity, decode and forward, piecewise linear
%\end{IEEEkeywords}



% For peer review papers, you can put extra information on the cover
% page as needed:
% \ifCLASSOPTIONpeerreview
% \begin{center} \bfseries EDICS Category: 3-BBND \end{center}
% \fi
%
% For peerreview papers, this IEEEtran command inserts a page break and
% creates the second title. It will be ignored for other modes.
%\IEEEpeerreviewmaketitle




\item A card is drawn from a deck of 52 cards. Find the probability of getting a king or a heart or a red card.\\
\solution
%\begin{table}[H]
	\centering
\begin{tabular}{|c|c|c|}
\hline
Random variable &Value &Definition\\ \hline
\multirow{3}{*}{X} &0 &Slips of Rs 1\\
&1 &Slips of Rs 5\\
&2 &Slips of Rs 13\\ \hline
\multirow{2}{*}{Y} &0 &Box A\\
&1 &Box B\\\hline
\end{tabular}
\caption{}
\label{tab:Distribution}
\end{table}
See \tabref{tab:Distribution}.
\begin{align}
p_{Y}\brak{k}= \begin{cases} 
      \frac{1}{3} & {k=0} \\
      \frac{2}{3 }& {k=1} 
   \end{cases}
   \\
p_{Y|X}\brak{0|0} = \frac{19}{25}\, 
p_{Y|X}\brak{0|1} = \frac{6}{25}\,
p_{Y|X}\brak{1|0} = \frac{45}{50}\,
p_{Y|X}\brak{1|2} = \frac{5}{50}
\end{align}
The desired probability is the probability that a slip drawn at random is marked other than Rs 1,
\begin{align}
&=1-p_X\brak{0}\\
&= p_X(1) + p_X(2)
\end{align}
Using Bayes theorem,
\begin{align}
&= p_Y\brak{0} \times \pr{Y=0 | X=1} + p_Y\brak{1} \times \pr{Y=1|X=2}\\
&=\frac{1}{3} \times \frac{6}{25} + \frac{2}{3} \times \frac{5}{50}\\
&=\frac{11}{75}
\end{align}

\newpage

%\tableofcontents

\bigskip

\renewcommand{\thefigure}{\theenumi}
\renewcommand{\thetable}{\theenumi}
%\renewcommand{\theequation}{\theenumi}

%\begin{abstract}
%%\boldmath
%In this letter, an algorithm for evaluating the exact analytical bit error rate  (BER)  for the piecewise linear (PL) combiner for  multiple relays is presented. Previous results were available only for upto three relays. The algorithm is unique in the sense that  the actual mathematical expressions, that are prohibitively large, need not be explicitly obtained. The diversity gain due to multiple relays is shown through plots of the analytical BER, well supported by simulations. 
%
%\end{abstract}
% IEEEtran.cls defaults to using nonbold math in the Abstract.
% This preserves the distinction between vectors and scalars. However,
% if the journal you are submitting to favors bold math in the abstract,
% then you can use LaTeX's standard command \boldmath at the very start
% of the abstract to achieve this. Many IEEE journals frown on math
% in the abstract anyway.

% Note that keywords are not normally used for peerreview papers.
%\begin{IEEEkeywords}
%Cooperative diversity, decode and forward, piecewise linear
%\end{IEEEkeywords}



% For peer review papers, you can put extra information on the cover
% page as needed:
% \ifCLASSOPTIONpeerreview
% \begin{center} \bfseries EDICS Category: 3-BBND \end{center}
% \fi
%
% For peerreview papers, this IEEEtran command inserts a page break and
% creates the second title. It will be ignored for other modes.
%\IEEEpeerreviewmaketitle




\item The probability that a student will pass his examination is 0.73, the probability of
the student getting a compartment is 0.13, and the probability that the student will
either pass or get compartment is 0.96. State True or False.\\
\solution
%\begin{table}[H]
	\centering
\begin{tabular}{|c|c|c|}
\hline
Random variable &Value &Definition\\ \hline
\multirow{3}{*}{X} &0 &Slips of Rs 1\\
&1 &Slips of Rs 5\\
&2 &Slips of Rs 13\\ \hline
\multirow{2}{*}{Y} &0 &Box A\\
&1 &Box B\\\hline
\end{tabular}
\caption{}
\label{tab:Distribution}
\end{table}
See \tabref{tab:Distribution}.
\begin{align}
p_{Y}\brak{k}= \begin{cases} 
      \frac{1}{3} & {k=0} \\
      \frac{2}{3 }& {k=1} 
   \end{cases}
   \\
p_{Y|X}\brak{0|0} = \frac{19}{25}\, 
p_{Y|X}\brak{0|1} = \frac{6}{25}\,
p_{Y|X}\brak{1|0} = \frac{45}{50}\,
p_{Y|X}\brak{1|2} = \frac{5}{50}
\end{align}
The desired probability is the probability that a slip drawn at random is marked other than Rs 1,
\begin{align}
&=1-p_X\brak{0}\\
&= p_X(1) + p_X(2)
\end{align}
Using Bayes theorem,
\begin{align}
&= p_Y\brak{0} \times \pr{Y=0 | X=1} + p_Y\brak{1} \times \pr{Y=1|X=2}\\
&=\frac{1}{3} \times \frac{6}{25} + \frac{2}{3} \times \frac{5}{50}\\
&=\frac{11}{75}
\end{align}

\newpage

%\tableofcontents

\bigskip

\renewcommand{\thefigure}{\theenumi}
\renewcommand{\thetable}{\theenumi}
%\renewcommand{\theequation}{\theenumi}

%\begin{abstract}
%%\boldmath
%In this letter, an algorithm for evaluating the exact analytical bit error rate  (BER)  for the piecewise linear (PL) combiner for  multiple relays is presented. Previous results were available only for upto three relays. The algorithm is unique in the sense that  the actual mathematical expressions, that are prohibitively large, need not be explicitly obtained. The diversity gain due to multiple relays is shown through plots of the analytical BER, well supported by simulations. 
%
%\end{abstract}
% IEEEtran.cls defaults to using nonbold math in the Abstract.
% This preserves the distinction between vectors and scalars. However,
% if the journal you are submitting to favors bold math in the abstract,
% then you can use LaTeX's standard command \boldmath at the very start
% of the abstract to achieve this. Many IEEE journals frown on math
% in the abstract anyway.

% Note that keywords are not normally used for peerreview papers.
%\begin{IEEEkeywords}
%Cooperative diversity, decode and forward, piecewise linear
%\end{IEEEkeywords}



% For peer review papers, you can put extra information on the cover
% page as needed:
% \ifCLASSOPTIONpeerreview
% \begin{center} \bfseries EDICS Category: 3-BBND \end{center}
% \fi
%
% For peerreview papers, this IEEEtran command inserts a page break and
% creates the second title. It will be ignored for other modes.
%\IEEEpeerreviewmaketitle




\item A card is selected from a pack of 52 cards\\
\begin{enumerate}[label=(\alph*)]
\item How many points are there in the sample space?
\item Calculate the probability that the cards is an ace of spades.
\item Calculate the probability that the card is (i) an ace (ii)black card.\\
\end{enumerate}
%\input{ncert/11/16/3/4_1/Prob_4.tex}
\item In a non-leap year, the probability of having 53 tuesdays or 53 wednesdays is\\
\solution
%A non-leap year has a total of 365 days, and a week has 7 days.\\
So it can be expressed as 
\begin{align}
365\text{days} &=52\times 7+1 \text{day}
\end{align}
$\implies$ 52 tuesdays or wednesdays\\
Random variable X denotes the days of a week
\begin{align}
p_X\brak{k}&=\frac{1}{7}; \quad \brak{1<k<7}
\end{align}
So the probability of extra day being tuesday or wednesday is
\begin{align}
p_X\brak{3}+p_X\brak{4}&=\frac{1}{7}+\frac{1}{7}=\frac{2}{7}
\end{align}



\item There are 1000 sealed envelopes in a box, 10 of them contain a cash prize of
Rs 100 each, 100 of them contain a cash prize of Rs 50 each and 200 of them
contain a cash prize of Rs 10 each and rest do not contain any cash prize. If they
are well shuffled and an envelope is picked up out, what is the probability that it
contains no cash prize?\\
\solution
%\begin{table}[H]
	\centering
\begin{tabular}{|c|c|c|}
\hline
Random variable &Value &Definition\\ \hline
\multirow{3}{*}{X} &0 &Slips of Rs 1\\
&1 &Slips of Rs 5\\
&2 &Slips of Rs 13\\ \hline
\multirow{2}{*}{Y} &0 &Box A\\
&1 &Box B\\\hline
\end{tabular}
\caption{}
\label{tab:Distribution}
\end{table}
See \tabref{tab:Distribution}.
\begin{align}
p_{Y}\brak{k}= \begin{cases} 
      \frac{1}{3} & {k=0} \\
      \frac{2}{3 }& {k=1} 
   \end{cases}
   \\
p_{Y|X}\brak{0|0} = \frac{19}{25}\, 
p_{Y|X}\brak{0|1} = \frac{6}{25}\,
p_{Y|X}\brak{1|0} = \frac{45}{50}\,
p_{Y|X}\brak{1|2} = \frac{5}{50}
\end{align}
The desired probability is the probability that a slip drawn at random is marked other than Rs 1,
\begin{align}
&=1-p_X\brak{0}\\
&= p_X(1) + p_X(2)
\end{align}
Using Bayes theorem,
\begin{align}
&= p_Y\brak{0} \times \pr{Y=0 | X=1} + p_Y\brak{1} \times \pr{Y=1|X=2}\\
&=\frac{1}{3} \times \frac{6}{25} + \frac{2}{3} \times \frac{5}{50}\\
&=\frac{11}{75}
\end{align}

\newpage

%\tableofcontents

\bigskip

\renewcommand{\thefigure}{\theenumi}
\renewcommand{\thetable}{\theenumi}
%\renewcommand{\theequation}{\theenumi}

%\begin{abstract}
%%\boldmath
%In this letter, an algorithm for evaluating the exact analytical bit error rate  (BER)  for the piecewise linear (PL) combiner for  multiple relays is presented. Previous results were available only for upto three relays. The algorithm is unique in the sense that  the actual mathematical expressions, that are prohibitively large, need not be explicitly obtained. The diversity gain due to multiple relays is shown through plots of the analytical BER, well supported by simulations. 
%
%\end{abstract}
% IEEEtran.cls defaults to using nonbold math in the Abstract.
% This preserves the distinction between vectors and scalars. However,
% if the journal you are submitting to favors bold math in the abstract,
% then you can use LaTeX's standard command \boldmath at the very start
% of the abstract to achieve this. Many IEEE journals frown on math
% in the abstract anyway.

% Note that keywords are not normally used for peerreview papers.
%\begin{IEEEkeywords}
%Cooperative diversity, decode and forward, piecewise linear
%\end{IEEEkeywords}



% For peer review papers, you can put extra information on the cover
% page as needed:
% \ifCLASSOPTIONpeerreview
% \begin{center} \bfseries EDICS Category: 3-BBND \end{center}
% \fi
%
% For peerreview papers, this IEEEtran command inserts a page break and
% creates the second title. It will be ignored for other modes.
%\IEEEpeerreviewmaketitle




\item 
A die is thrown and a card is selected at random from a deck of 52 playing cards. The probability of getting an even number on the die and a spade card.\\
\solution
%\begin{table}[H]
	\centering
\begin{tabular}{|c|c|c|}
\hline
Random variable &Value &Definition\\ \hline
\multirow{3}{*}{X} &0 &Slips of Rs 1\\
&1 &Slips of Rs 5\\
&2 &Slips of Rs 13\\ \hline
\multirow{2}{*}{Y} &0 &Box A\\
&1 &Box B\\\hline
\end{tabular}
\caption{}
\label{tab:Distribution}
\end{table}
See \tabref{tab:Distribution}.
\begin{align}
p_{Y}\brak{k}= \begin{cases} 
      \frac{1}{3} & {k=0} \\
      \frac{2}{3 }& {k=1} 
   \end{cases}
   \\
p_{Y|X}\brak{0|0} = \frac{19}{25}\, 
p_{Y|X}\brak{0|1} = \frac{6}{25}\,
p_{Y|X}\brak{1|0} = \frac{45}{50}\,
p_{Y|X}\brak{1|2} = \frac{5}{50}
\end{align}
The desired probability is the probability that a slip drawn at random is marked other than Rs 1,
\begin{align}
&=1-p_X\brak{0}\\
&= p_X(1) + p_X(2)
\end{align}
Using Bayes theorem,
\begin{align}
&= p_Y\brak{0} \times \pr{Y=0 | X=1} + p_Y\brak{1} \times \pr{Y=1|X=2}\\
&=\frac{1}{3} \times \frac{6}{25} + \frac{2}{3} \times \frac{5}{50}\\
&=\frac{11}{75}
\end{align}

\newpage

%\tableofcontents

\bigskip

\renewcommand{\thefigure}{\theenumi}
\renewcommand{\thetable}{\theenumi}
%\renewcommand{\theequation}{\theenumi}

%\begin{abstract}
%%\boldmath
%In this letter, an algorithm for evaluating the exact analytical bit error rate  (BER)  for the piecewise linear (PL) combiner for  multiple relays is presented. Previous results were available only for upto three relays. The algorithm is unique in the sense that  the actual mathematical expressions, that are prohibitively large, need not be explicitly obtained. The diversity gain due to multiple relays is shown through plots of the analytical BER, well supported by simulations. 
%
%\end{abstract}
% IEEEtran.cls defaults to using nonbold math in the Abstract.
% This preserves the distinction between vectors and scalars. However,
% if the journal you are submitting to favors bold math in the abstract,
% then you can use LaTeX's standard command \boldmath at the very start
% of the abstract to achieve this. Many IEEE journals frown on math
% in the abstract anyway.

% Note that keywords are not normally used for peerreview papers.
%\begin{IEEEkeywords}
%Cooperative diversity, decode and forward, piecewise linear
%\end{IEEEkeywords}



% For peer review papers, you can put extra information on the cover
% page as needed:
% \ifCLASSOPTIONpeerreview
% \begin{center} \bfseries EDICS Category: 3-BBND \end{center}
% \fi
%
% For peerreview papers, this IEEEtran command inserts a page break and
% creates the second title. It will be ignored for other modes.
%\IEEEpeerreviewmaketitle




\item
If 4-digit numbers greater than 5,000 are randomly formed from the digits 0, 1, 3, 5, and 7, what is the probability of forming a number divisible by 5 when:
\begin{enumerate}
    \item The digits are repeated?
    \item The repetition of digits is not allowed?
\end{enumerate}
\solution
%\begin{table}[H]
	\centering
\begin{tabular}{|c|c|c|}
\hline
Random variable &Value &Definition\\ \hline
\multirow{3}{*}{X} &0 &Slips of Rs 1\\
&1 &Slips of Rs 5\\
&2 &Slips of Rs 13\\ \hline
\multirow{2}{*}{Y} &0 &Box A\\
&1 &Box B\\\hline
\end{tabular}
\caption{}
\label{tab:Distribution}
\end{table}
See \tabref{tab:Distribution}.
\begin{align}
p_{Y}\brak{k}= \begin{cases} 
      \frac{1}{3} & {k=0} \\
      \frac{2}{3 }& {k=1} 
   \end{cases}
   \\
p_{Y|X}\brak{0|0} = \frac{19}{25}\, 
p_{Y|X}\brak{0|1} = \frac{6}{25}\,
p_{Y|X}\brak{1|0} = \frac{45}{50}\,
p_{Y|X}\brak{1|2} = \frac{5}{50}
\end{align}
The desired probability is the probability that a slip drawn at random is marked other than Rs 1,
\begin{align}
&=1-p_X\brak{0}\\
&= p_X(1) + p_X(2)
\end{align}
Using Bayes theorem,
\begin{align}
&= p_Y\brak{0} \times \pr{Y=0 | X=1} + p_Y\brak{1} \times \pr{Y=1|X=2}\\
&=\frac{1}{3} \times \frac{6}{25} + \frac{2}{3} \times \frac{5}{50}\\
&=\frac{11}{75}
\end{align}

\newpage

%\tableofcontents

\bigskip

\renewcommand{\thefigure}{\theenumi}
\renewcommand{\thetable}{\theenumi}
%\renewcommand{\theequation}{\theenumi}

%\begin{abstract}
%%\boldmath
%In this letter, an algorithm for evaluating the exact analytical bit error rate  (BER)  for the piecewise linear (PL) combiner for  multiple relays is presented. Previous results were available only for upto three relays. The algorithm is unique in the sense that  the actual mathematical expressions, that are prohibitively large, need not be explicitly obtained. The diversity gain due to multiple relays is shown through plots of the analytical BER, well supported by simulations. 
%
%\end{abstract}
% IEEEtran.cls defaults to using nonbold math in the Abstract.
% This preserves the distinction between vectors and scalars. However,
% if the journal you are submitting to favors bold math in the abstract,
% then you can use LaTeX's standard command \boldmath at the very start
% of the abstract to achieve this. Many IEEE journals frown on math
% in the abstract anyway.

% Note that keywords are not normally used for peerreview papers.
%\begin{IEEEkeywords}
%Cooperative diversity, decode and forward, piecewise linear
%\end{IEEEkeywords}



% For peer review papers, you can put extra information on the cover
% page as needed:
% \ifCLASSOPTIONpeerreview
% \begin{center} \bfseries EDICS Category: 3-BBND \end{center}
% \fi
%
% For peerreview papers, this IEEEtran command inserts a page break and
% creates the second title. It will be ignored for other modes.
%\IEEEpeerreviewmaketitle




\item Consider the probability space $\brak{\Omega, \mathcal{G}, P}$ where $\Omega = [0,2]$ and $\mathcal{G} = \cbrak{\phi, \Omega, [0,1], (1,2]}$. Let $X$ and $Y$ be two functions on $\Omega$ defined as
\begin{align*}
    X(\omega) = 
    \begin{cases}
        1 & \text{if }\omega \in [0, 1]\\
        2 & \text{if }\omega \in (1, 2]
    \end{cases}
\end{align*}
and
\begin{align*}
    Y(\omega) = 
    \begin{cases}
        2 & \text{if }\omega \in [0, 1.5]\\
        3 & \text{if }\omega \in (1.5, 2].
    \end{cases}
\end{align*}
Then which one of the following statements is true?
\begin{enumerate}
    \item [(A)] $X$ is a random variable with respect to $\mathcal{G}$, but $Y$ is not a random variable with respect to $\mathcal{G}$.
    \item [(B)] $Y$ is a random variable with respect to $\mathcal{G}$, but $X$ is not a random variable with respect to $\mathcal{G}$.
    \item [(C)] Neither $X$ nor $Y$ is a random variable with respect to $\mathcal{G}$.
    \item [(D)] Both $X$ and $Y$ are random variables with respect to $\mathcal{G}$.
\end{enumerate} \hfill (GATE ST 2023)\\
\solution
%\begin{table}[H]
	\centering
\begin{tabular}{|c|c|c|}
\hline
Random variable &Value &Definition\\ \hline
\multirow{3}{*}{X} &0 &Slips of Rs 1\\
&1 &Slips of Rs 5\\
&2 &Slips of Rs 13\\ \hline
\multirow{2}{*}{Y} &0 &Box A\\
&1 &Box B\\\hline
\end{tabular}
\caption{}
\label{tab:Distribution}
\end{table}
See \tabref{tab:Distribution}.
\begin{align}
p_{Y}\brak{k}= \begin{cases} 
      \frac{1}{3} & {k=0} \\
      \frac{2}{3 }& {k=1} 
   \end{cases}
   \\
p_{Y|X}\brak{0|0} = \frac{19}{25}\, 
p_{Y|X}\brak{0|1} = \frac{6}{25}\,
p_{Y|X}\brak{1|0} = \frac{45}{50}\,
p_{Y|X}\brak{1|2} = \frac{5}{50}
\end{align}
The desired probability is the probability that a slip drawn at random is marked other than Rs 1,
\begin{align}
&=1-p_X\brak{0}\\
&= p_X(1) + p_X(2)
\end{align}
Using Bayes theorem,
\begin{align}
&= p_Y\brak{0} \times \pr{Y=0 | X=1} + p_Y\brak{1} \times \pr{Y=1|X=2}\\
&=\frac{1}{3} \times \frac{6}{25} + \frac{2}{3} \times \frac{5}{50}\\
&=\frac{11}{75}
\end{align}

\newpage

%\tableofcontents

\bigskip

\renewcommand{\thefigure}{\theenumi}
\renewcommand{\thetable}{\theenumi}
%\renewcommand{\theequation}{\theenumi}

%\begin{abstract}
%%\boldmath
%In this letter, an algorithm for evaluating the exact analytical bit error rate  (BER)  for the piecewise linear (PL) combiner for  multiple relays is presented. Previous results were available only for upto three relays. The algorithm is unique in the sense that  the actual mathematical expressions, that are prohibitively large, need not be explicitly obtained. The diversity gain due to multiple relays is shown through plots of the analytical BER, well supported by simulations. 
%
%\end{abstract}
% IEEEtran.cls defaults to using nonbold math in the Abstract.
% This preserves the distinction between vectors and scalars. However,
% if the journal you are submitting to favors bold math in the abstract,
% then you can use LaTeX's standard command \boldmath at the very start
% of the abstract to achieve this. Many IEEE journals frown on math
% in the abstract anyway.

% Note that keywords are not normally used for peerreview papers.
%\begin{IEEEkeywords}
%Cooperative diversity, decode and forward, piecewise linear
%\end{IEEEkeywords}



% For peer review papers, you can put extra information on the cover
% page as needed:
% \ifCLASSOPTIONpeerreview
% \begin{center} \bfseries EDICS Category: 3-BBND \end{center}
% \fi
%
% For peerreview papers, this IEEEtran command inserts a page break and
% creates the second title. It will be ignored for other modes.
%\IEEEpeerreviewmaketitle




	\item  A die is loaded in such a way that each odd number is twice as likely to occur as
each even number. Find $P(G)$, where $G$ is the event that a number greater than
3 occurs on a single roll of the die.
\\
\solution
		%\begin{table}[H]
	\centering
\begin{tabular}{|c|c|c|}
\hline
Random variable &Value &Definition\\ \hline
\multirow{3}{*}{X} &0 &Slips of Rs 1\\
&1 &Slips of Rs 5\\
&2 &Slips of Rs 13\\ \hline
\multirow{2}{*}{Y} &0 &Box A\\
&1 &Box B\\\hline
\end{tabular}
\caption{}
\label{tab:Distribution}
\end{table}
See \tabref{tab:Distribution}.
\begin{align}
p_{Y}\brak{k}= \begin{cases} 
      \frac{1}{3} & {k=0} \\
      \frac{2}{3 }& {k=1} 
   \end{cases}
   \\
p_{Y|X}\brak{0|0} = \frac{19}{25}\, 
p_{Y|X}\brak{0|1} = \frac{6}{25}\,
p_{Y|X}\brak{1|0} = \frac{45}{50}\,
p_{Y|X}\brak{1|2} = \frac{5}{50}
\end{align}
The desired probability is the probability that a slip drawn at random is marked other than Rs 1,
\begin{align}
&=1-p_X\brak{0}\\
&= p_X(1) + p_X(2)
\end{align}
Using Bayes theorem,
\begin{align}
&= p_Y\brak{0} \times \pr{Y=0 | X=1} + p_Y\brak{1} \times \pr{Y=1|X=2}\\
&=\frac{1}{3} \times \frac{6}{25} + \frac{2}{3} \times \frac{5}{50}\\
&=\frac{11}{75}
\end{align}

\newpage

%\tableofcontents

\bigskip

\renewcommand{\thefigure}{\theenumi}
\renewcommand{\thetable}{\theenumi}
%\renewcommand{\theequation}{\theenumi}

%\begin{abstract}
%%\boldmath
%In this letter, an algorithm for evaluating the exact analytical bit error rate  (BER)  for the piecewise linear (PL) combiner for  multiple relays is presented. Previous results were available only for upto three relays. The algorithm is unique in the sense that  the actual mathematical expressions, that are prohibitively large, need not be explicitly obtained. The diversity gain due to multiple relays is shown through plots of the analytical BER, well supported by simulations. 
%
%\end{abstract}
% IEEEtran.cls defaults to using nonbold math in the Abstract.
% This preserves the distinction between vectors and scalars. However,
% if the journal you are submitting to favors bold math in the abstract,
% then you can use LaTeX's standard command \boldmath at the very start
% of the abstract to achieve this. Many IEEE journals frown on math
% in the abstract anyway.

% Note that keywords are not normally used for peerreview papers.
%\begin{IEEEkeywords}
%Cooperative diversity, decode and forward, piecewise linear
%\end{IEEEkeywords}



% For peer review papers, you can put extra information on the cover
% page as needed:
% \ifCLASSOPTIONpeerreview
% \begin{center} \bfseries EDICS Category: 3-BBND \end{center}
% \fi
%
% For peerreview papers, this IEEEtran command inserts a page break and
% creates the second title. It will be ignored for other modes.
%\IEEEpeerreviewmaketitle




	\item All the jacks, queens and kings are removed from a deck of 52 playing cards. The remaining cards are well shuffled and then one card is drawn at random. Giving ace a value 1 similar value for other cards, find the probability that the card has a value 
		\begin{enumerate}
			\item 7
			\item greater than 7
			\item less than 7
		\end{enumerate}
		%Number of cards left after removing all jacks, queens and kings 
\begin{align}
N	= 52 - 4\times 3
	= 40
\end{align}
%\begin{table}[H]
%\def\arraystretch{1.2}
%\begin{tabular}{|c|c|c|}
%\hline
%	\textbf{Parameter} &\textbf{Value} &\textbf{Description}\\ \hline
%	$X$ &1-10 &Represents the value of the card picked \\ \hline
%\end{tabular}
%\end{table}
Let $1 \le X \le 10$ be the value of the card picked.  Then,
\begin{align}
	p_X(k) &= \Pr(X=k)\ \forall\ 1 \leq k \leq 10\\
	&= \frac{4\times 1}{40}\\
	&= \frac{1}{10}\\
	\therefore p_X(k) &= 
	\begin{cases}
		\frac{1}{10} & 1 \leq k \leq 10\\
		0 & \text{otherwise}
	\end{cases}
\end{align}
and
\begin{align}
	F_{X}(k) &= \sum_{m=0}^{k}p_{X}(m) \quad 1 \leq k \leq 10\\
	&= \frac{k}{10}\\
	\therefore F_{X}(k) &= 
	\begin{cases}
		0 & k \leq 0\\
		\frac{k}{10} & 1\leq k \leq 10\\
		1 & k > 10 
	\end{cases}
\end{align}
\begin{enumerate}
	\item Probability that card has value equal to 7 is
		\begin{align}
			 p_{X}(7)
			= \frac{1}{10}
		\end{align}
	\item Probability that card has value greater than 7 is
		\begin{align}
			1 - F_X(7)
			&= 1 - \frac{7}{10}
			\\
			&= \frac{3}{10}
		\end{align}
	\item Probability that card has value less than 7 is
		\begin{align}
			 F_{X}(6)
			=\frac{6}{10}
		\end{align}
\end{enumerate}

  \item A Lot consists of 48 mobile phones of which 42 are good, 3 have only minor defects and 3 have major defects.Varnika will buy a phone if it is good but the trader will only buy a mobile if it has no major defects. One phone is selected at random from the lot. What is the probability that it is
\begin{enumerate}
	\item acceptable to Varnika?
            \item acceptable to the trader?
\end{enumerate}
\solution
	%\begin{table}[H]
	\centering
\begin{tabular}{|c|c|c|}
\hline
Random variable &Value &Definition\\ \hline
\multirow{3}{*}{X} &0 &Slips of Rs 1\\
&1 &Slips of Rs 5\\
&2 &Slips of Rs 13\\ \hline
\multirow{2}{*}{Y} &0 &Box A\\
&1 &Box B\\\hline
\end{tabular}
\caption{}
\label{tab:Distribution}
\end{table}
See \tabref{tab:Distribution}.
\begin{align}
p_{Y}\brak{k}= \begin{cases} 
      \frac{1}{3} & {k=0} \\
      \frac{2}{3 }& {k=1} 
   \end{cases}
   \\
p_{Y|X}\brak{0|0} = \frac{19}{25}\, 
p_{Y|X}\brak{0|1} = \frac{6}{25}\,
p_{Y|X}\brak{1|0} = \frac{45}{50}\,
p_{Y|X}\brak{1|2} = \frac{5}{50}
\end{align}
The desired probability is the probability that a slip drawn at random is marked other than Rs 1,
\begin{align}
&=1-p_X\brak{0}\\
&= p_X(1) + p_X(2)
\end{align}
Using Bayes theorem,
\begin{align}
&= p_Y\brak{0} \times \pr{Y=0 | X=1} + p_Y\brak{1} \times \pr{Y=1|X=2}\\
&=\frac{1}{3} \times \frac{6}{25} + \frac{2}{3} \times \frac{5}{50}\\
&=\frac{11}{75}
\end{align}

\newpage

%\tableofcontents

\bigskip

\renewcommand{\thefigure}{\theenumi}
\renewcommand{\thetable}{\theenumi}
%\renewcommand{\theequation}{\theenumi}

%\begin{abstract}
%%\boldmath
%In this letter, an algorithm for evaluating the exact analytical bit error rate  (BER)  for the piecewise linear (PL) combiner for  multiple relays is presented. Previous results were available only for upto three relays. The algorithm is unique in the sense that  the actual mathematical expressions, that are prohibitively large, need not be explicitly obtained. The diversity gain due to multiple relays is shown through plots of the analytical BER, well supported by simulations. 
%
%\end{abstract}
% IEEEtran.cls defaults to using nonbold math in the Abstract.
% This preserves the distinction between vectors and scalars. However,
% if the journal you are submitting to favors bold math in the abstract,
% then you can use LaTeX's standard command \boldmath at the very start
% of the abstract to achieve this. Many IEEE journals frown on math
% in the abstract anyway.

% Note that keywords are not normally used for peerreview papers.
%\begin{IEEEkeywords}
%Cooperative diversity, decode and forward, piecewise linear
%\end{IEEEkeywords}



% For peer review papers, you can put extra information on the cover
% page as needed:
% \ifCLASSOPTIONpeerreview
% \begin{center} \bfseries EDICS Category: 3-BBND \end{center}
% \fi
%
% For peerreview papers, this IEEEtran command inserts a page break and
% creates the second title. It will be ignored for other modes.
%\IEEEpeerreviewmaketitle




 \item A student says that if you throw a die, it will show up 1 or not 1. Therefore, the probability of getting 1 and the probability of getting 'not 1' each is equal to $\frac{1}{2}$. Is this correct? Give reasons.\\
 \solution
        %\begin{table}[H]
	\centering
\begin{tabular}{|c|c|c|}
\hline
Random variable &Value &Definition\\ \hline
\multirow{3}{*}{X} &0 &Slips of Rs 1\\
&1 &Slips of Rs 5\\
&2 &Slips of Rs 13\\ \hline
\multirow{2}{*}{Y} &0 &Box A\\
&1 &Box B\\\hline
\end{tabular}
\caption{}
\label{tab:Distribution}
\end{table}
See \tabref{tab:Distribution}.
\begin{align}
p_{Y}\brak{k}= \begin{cases} 
      \frac{1}{3} & {k=0} \\
      \frac{2}{3 }& {k=1} 
   \end{cases}
   \\
p_{Y|X}\brak{0|0} = \frac{19}{25}\, 
p_{Y|X}\brak{0|1} = \frac{6}{25}\,
p_{Y|X}\brak{1|0} = \frac{45}{50}\,
p_{Y|X}\brak{1|2} = \frac{5}{50}
\end{align}
The desired probability is the probability that a slip drawn at random is marked other than Rs 1,
\begin{align}
&=1-p_X\brak{0}\\
&= p_X(1) + p_X(2)
\end{align}
Using Bayes theorem,
\begin{align}
&= p_Y\brak{0} \times \pr{Y=0 | X=1} + p_Y\brak{1} \times \pr{Y=1|X=2}\\
&=\frac{1}{3} \times \frac{6}{25} + \frac{2}{3} \times \frac{5}{50}\\
&=\frac{11}{75}
\end{align}

\newpage

%\tableofcontents

\bigskip

\renewcommand{\thefigure}{\theenumi}
\renewcommand{\thetable}{\theenumi}
%\renewcommand{\theequation}{\theenumi}

%\begin{abstract}
%%\boldmath
%In this letter, an algorithm for evaluating the exact analytical bit error rate  (BER)  for the piecewise linear (PL) combiner for  multiple relays is presented. Previous results were available only for upto three relays. The algorithm is unique in the sense that  the actual mathematical expressions, that are prohibitively large, need not be explicitly obtained. The diversity gain due to multiple relays is shown through plots of the analytical BER, well supported by simulations. 
%
%\end{abstract}
% IEEEtran.cls defaults to using nonbold math in the Abstract.
% This preserves the distinction between vectors and scalars. However,
% if the journal you are submitting to favors bold math in the abstract,
% then you can use LaTeX's standard command \boldmath at the very start
% of the abstract to achieve this. Many IEEE journals frown on math
% in the abstract anyway.

% Note that keywords are not normally used for peerreview papers.
%\begin{IEEEkeywords}
%Cooperative diversity, decode and forward, piecewise linear
%\end{IEEEkeywords}



% For peer review papers, you can put extra information on the cover
% page as needed:
% \ifCLASSOPTIONpeerreview
% \begin{center} \bfseries EDICS Category: 3-BBND \end{center}
% \fi
%
% For peerreview papers, this IEEEtran command inserts a page break and
% creates the second title. It will be ignored for other modes.
%\IEEEpeerreviewmaketitle




   \item Four candidates A, B, C, D have ap-
plied for the assignment to coach a school cricket
team. If A is twice as likely to be selected as B, and
B and C are given about the same chance of being
selected, while C is twice as likely to be selected
as D, what are the probabilities that
\begin{enumerate}
\item C will be selected?
\item A will not be selected?
\end{enumerate}
	%\begin{table}[H]
	\centering
\begin{tabular}{|c|c|c|}
\hline
Random variable &Value &Definition\\ \hline
\multirow{3}{*}{X} &0 &Slips of Rs 1\\
&1 &Slips of Rs 5\\
&2 &Slips of Rs 13\\ \hline
\multirow{2}{*}{Y} &0 &Box A\\
&1 &Box B\\\hline
\end{tabular}
\caption{}
\label{tab:Distribution}
\end{table}
See \tabref{tab:Distribution}.
\begin{align}
p_{Y}\brak{k}= \begin{cases} 
      \frac{1}{3} & {k=0} \\
      \frac{2}{3 }& {k=1} 
   \end{cases}
   \\
p_{Y|X}\brak{0|0} = \frac{19}{25}\, 
p_{Y|X}\brak{0|1} = \frac{6}{25}\,
p_{Y|X}\brak{1|0} = \frac{45}{50}\,
p_{Y|X}\brak{1|2} = \frac{5}{50}
\end{align}
The desired probability is the probability that a slip drawn at random is marked other than Rs 1,
\begin{align}
&=1-p_X\brak{0}\\
&= p_X(1) + p_X(2)
\end{align}
Using Bayes theorem,
\begin{align}
&= p_Y\brak{0} \times \pr{Y=0 | X=1} + p_Y\brak{1} \times \pr{Y=1|X=2}\\
&=\frac{1}{3} \times \frac{6}{25} + \frac{2}{3} \times \frac{5}{50}\\
&=\frac{11}{75}
\end{align}

\newpage

%\tableofcontents

\bigskip

\renewcommand{\thefigure}{\theenumi}
\renewcommand{\thetable}{\theenumi}
%\renewcommand{\theequation}{\theenumi}

%\begin{abstract}
%%\boldmath
%In this letter, an algorithm for evaluating the exact analytical bit error rate  (BER)  for the piecewise linear (PL) combiner for  multiple relays is presented. Previous results were available only for upto three relays. The algorithm is unique in the sense that  the actual mathematical expressions, that are prohibitively large, need not be explicitly obtained. The diversity gain due to multiple relays is shown through plots of the analytical BER, well supported by simulations. 
%
%\end{abstract}
% IEEEtran.cls defaults to using nonbold math in the Abstract.
% This preserves the distinction between vectors and scalars. However,
% if the journal you are submitting to favors bold math in the abstract,
% then you can use LaTeX's standard command \boldmath at the very start
% of the abstract to achieve this. Many IEEE journals frown on math
% in the abstract anyway.

% Note that keywords are not normally used for peerreview papers.
%\begin{IEEEkeywords}
%Cooperative diversity, decode and forward, piecewise linear
%\end{IEEEkeywords}



% For peer review papers, you can put extra information on the cover
% page as needed:
% \ifCLASSOPTIONpeerreview
% \begin{center} \bfseries EDICS Category: 3-BBND \end{center}
% \fi
%
% For peerreview papers, this IEEEtran command inserts a page break and
% creates the second title. It will be ignored for other modes.
%\IEEEpeerreviewmaketitle




 \item A bag contain 24 balls of which $x$ balls are red, $2x$ are white and $3x$ are blue. A ball is selected at random, What is the probability that it is
\begin{enumerate}[label=\alph*)]
\item not red ?
\item white ?
\end{enumerate}
%\begin{table}[H]
	\centering
\begin{tabular}{|c|c|c|}
\hline
Random variable &Value &Definition\\ \hline
\multirow{3}{*}{X} &0 &Slips of Rs 1\\
&1 &Slips of Rs 5\\
&2 &Slips of Rs 13\\ \hline
\multirow{2}{*}{Y} &0 &Box A\\
&1 &Box B\\\hline
\end{tabular}
\caption{}
\label{tab:Distribution}
\end{table}
See \tabref{tab:Distribution}.
\begin{align}
p_{Y}\brak{k}= \begin{cases} 
      \frac{1}{3} & {k=0} \\
      \frac{2}{3 }& {k=1} 
   \end{cases}
   \\
p_{Y|X}\brak{0|0} = \frac{19}{25}\, 
p_{Y|X}\brak{0|1} = \frac{6}{25}\,
p_{Y|X}\brak{1|0} = \frac{45}{50}\,
p_{Y|X}\brak{1|2} = \frac{5}{50}
\end{align}
The desired probability is the probability that a slip drawn at random is marked other than Rs 1,
\begin{align}
&=1-p_X\brak{0}\\
&= p_X(1) + p_X(2)
\end{align}
Using Bayes theorem,
\begin{align}
&= p_Y\brak{0} \times \pr{Y=0 | X=1} + p_Y\brak{1} \times \pr{Y=1|X=2}\\
&=\frac{1}{3} \times \frac{6}{25} + \frac{2}{3} \times \frac{5}{50}\\
&=\frac{11}{75}
\end{align}

\newpage

%\tableofcontents

\bigskip

\renewcommand{\thefigure}{\theenumi}
\renewcommand{\thetable}{\theenumi}
%\renewcommand{\theequation}{\theenumi}

%\begin{abstract}
%%\boldmath
%In this letter, an algorithm for evaluating the exact analytical bit error rate  (BER)  for the piecewise linear (PL) combiner for  multiple relays is presented. Previous results were available only for upto three relays. The algorithm is unique in the sense that  the actual mathematical expressions, that are prohibitively large, need not be explicitly obtained. The diversity gain due to multiple relays is shown through plots of the analytical BER, well supported by simulations. 
%
%\end{abstract}
% IEEEtran.cls defaults to using nonbold math in the Abstract.
% This preserves the distinction between vectors and scalars. However,
% if the journal you are submitting to favors bold math in the abstract,
% then you can use LaTeX's standard command \boldmath at the very start
% of the abstract to achieve this. Many IEEE journals frown on math
% in the abstract anyway.

% Note that keywords are not normally used for peerreview papers.
%\begin{IEEEkeywords}
%Cooperative diversity, decode and forward, piecewise linear
%\end{IEEEkeywords}



% For peer review papers, you can put extra information on the cover
% page as needed:
% \ifCLASSOPTIONpeerreview
% \begin{center} \bfseries EDICS Category: 3-BBND \end{center}
% \fi
%
% For peerreview papers, this IEEEtran command inserts a page break and
% creates the second title. It will be ignored for other modes.
%\IEEEpeerreviewmaketitle




If the letters of the word ASSASSINATION are arranged at random. Find the Probability that
\begin{enumerate}[label=(\alph*)]
\item Four $S's$ come consecutively in the word
\item Two  $I's$ and two $N's$ come together
\item All $A's$ are not coming together
\item No two $A's$ are coming together
\end{enumerate}
%\begin{table}[H]
	\centering
\begin{tabular}{|c|c|c|}
\hline
Random variable &Value &Definition\\ \hline
\multirow{3}{*}{X} &0 &Slips of Rs 1\\
&1 &Slips of Rs 5\\
&2 &Slips of Rs 13\\ \hline
\multirow{2}{*}{Y} &0 &Box A\\
&1 &Box B\\\hline
\end{tabular}
\caption{}
\label{tab:Distribution}
\end{table}
See \tabref{tab:Distribution}.
\begin{align}
p_{Y}\brak{k}= \begin{cases} 
      \frac{1}{3} & {k=0} \\
      \frac{2}{3 }& {k=1} 
   \end{cases}
   \\
p_{Y|X}\brak{0|0} = \frac{19}{25}\, 
p_{Y|X}\brak{0|1} = \frac{6}{25}\,
p_{Y|X}\brak{1|0} = \frac{45}{50}\,
p_{Y|X}\brak{1|2} = \frac{5}{50}
\end{align}
The desired probability is the probability that a slip drawn at random is marked other than Rs 1,
\begin{align}
&=1-p_X\brak{0}\\
&= p_X(1) + p_X(2)
\end{align}
Using Bayes theorem,
\begin{align}
&= p_Y\brak{0} \times \pr{Y=0 | X=1} + p_Y\brak{1} \times \pr{Y=1|X=2}\\
&=\frac{1}{3} \times \frac{6}{25} + \frac{2}{3} \times \frac{5}{50}\\
&=\frac{11}{75}
\end{align}

\newpage

%\tableofcontents

\bigskip

\renewcommand{\thefigure}{\theenumi}
\renewcommand{\thetable}{\theenumi}
%\renewcommand{\theequation}{\theenumi}

%\begin{abstract}
%%\boldmath
%In this letter, an algorithm for evaluating the exact analytical bit error rate  (BER)  for the piecewise linear (PL) combiner for  multiple relays is presented. Previous results were available only for upto three relays. The algorithm is unique in the sense that  the actual mathematical expressions, that are prohibitively large, need not be explicitly obtained. The diversity gain due to multiple relays is shown through plots of the analytical BER, well supported by simulations. 
%
%\end{abstract}
% IEEEtran.cls defaults to using nonbold math in the Abstract.
% This preserves the distinction between vectors and scalars. However,
% if the journal you are submitting to favors bold math in the abstract,
% then you can use LaTeX's standard command \boldmath at the very start
% of the abstract to achieve this. Many IEEE journals frown on math
% in the abstract anyway.

% Note that keywords are not normally used for peerreview papers.
%\begin{IEEEkeywords}
%Cooperative diversity, decode and forward, piecewise linear
%\end{IEEEkeywords}



% For peer review papers, you can put extra information on the cover
% page as needed:
% \ifCLASSOPTIONpeerreview
% \begin{center} \bfseries EDICS Category: 3-BBND \end{center}
% \fi
%
% For peerreview papers, this IEEEtran command inserts a page break and
% creates the second title. It will be ignored for other modes.
%\IEEEpeerreviewmaketitle




	\item One urn contains two black balls (labelled B1 and B2) and one white ball. A
	second urn contains one black ball and two white balls (labelled W1 and W2).
	Suppose the following experiment is performed. One of the two urns is chosen
	at random. Next a ball is randomly chosen from the urn. Then a second ball is
	chosen at random from the same urn without replacing the first ball.
	
	\begin{enumerate}
	\item What is the probability that two black balls are chosen?
	
	\item What is the probability that two balls of opposite colour are chosen?
	\end{enumerate}
	\solution
	%\begin{align}
    \label{eq:12.13.6.18.1}
	\because	\pr{A|B} &> \pr{A},\
\frac{\pr{AB}}{\pr{B}} > \pr{A}
\\
    \label{eq:12.13.6.18.2}
	\implies \pr{AB} &> \pr{A}\pr{B}
	\\
	\text{or, } \frac{\pr{AB}}{\pr{A}} &=\pr{B|A} > \pr{A}
\end{align}

\end{enumerate}

		\item A box of oranges is inspected by examining three randomly selected oranges drawn without replacement. If all the three oranges are good, the box is approved for sale, otherwise, it is rejected. Find the probability that a box containing 15 oranges out of which 12 are good and 3 are bad ones will be approved for sale.
		\label{ncert/12/13/2/3/defs.tex}
		\item Two balls are drawn at random with replacement from a box containing 10 black and 8 red balls. Find the probability that
		\label{ncert/12/13/2/12}
\begin{enumerate}
\item both balls are red.
\item first ball is black and second is red.
\item one of them is black and other is red.
\end{enumerate}

\item In a hostel, 60\% of the students read Hindi newspaper, 40\% read English newspaper and 20\% read both Hindi and English newspapers. A student is selected at random.
		\label{ncert/12/13/2/15}
\begin{enumerate}
\item Find the probability that she reads neither Hindi nor English newspapers.
\item If she reads Hindi newspaper, find the probability that she reads English newspaper.
\item If she reads English newspaper, find the probability that she reads Hindi newspaper.\\
\end{enumerate}
\item The probability of obtaining an even prime number on each die, when a pair of dice is rolled is 
\begin{enumerate}
    \item $0$ 
    
    \item $\frac{1}{3}$ 
    
    \item $\frac{1}{12}$ 
    
    \item $\frac{1}{36}$ 
\end{enumerate}
\solution
		%\begin{enumerate}[label=\thesection.\arabic*,ref=\thesection.\theenumi]
	\item One card is drawn from a well-shuffled deck of 52 cards. Find the probability of getting
\begin{enumerate}
\item A king of red colour 
\item A face card 
\item A red face card
\item The jack of hearts
\item A spade
\item The queen of diamonds

\end{enumerate}
\solution
		%\begin{table}[H]
	\centering
\begin{tabular}{|c|c|c|}
\hline
Random variable &Value &Definition\\ \hline
\multirow{3}{*}{X} &0 &Slips of Rs 1\\
&1 &Slips of Rs 5\\
&2 &Slips of Rs 13\\ \hline
\multirow{2}{*}{Y} &0 &Box A\\
&1 &Box B\\\hline
\end{tabular}
\caption{}
\label{tab:Distribution}
\end{table}
See \tabref{tab:Distribution}.
\begin{align}
p_{Y}\brak{k}= \begin{cases} 
      \frac{1}{3} & {k=0} \\
      \frac{2}{3 }& {k=1} 
   \end{cases}
   \\
p_{Y|X}\brak{0|0} = \frac{19}{25}\, 
p_{Y|X}\brak{0|1} = \frac{6}{25}\,
p_{Y|X}\brak{1|0} = \frac{45}{50}\,
p_{Y|X}\brak{1|2} = \frac{5}{50}
\end{align}
The desired probability is the probability that a slip drawn at random is marked other than Rs 1,
\begin{align}
&=1-p_X\brak{0}\\
&= p_X(1) + p_X(2)
\end{align}
Using Bayes theorem,
\begin{align}
&= p_Y\brak{0} \times \pr{Y=0 | X=1} + p_Y\brak{1} \times \pr{Y=1|X=2}\\
&=\frac{1}{3} \times \frac{6}{25} + \frac{2}{3} \times \frac{5}{50}\\
&=\frac{11}{75}
\end{align}

\newpage

%\tableofcontents

\bigskip

\renewcommand{\thefigure}{\theenumi}
\renewcommand{\thetable}{\theenumi}
%\renewcommand{\theequation}{\theenumi}

%\begin{abstract}
%%\boldmath
%In this letter, an algorithm for evaluating the exact analytical bit error rate  (BER)  for the piecewise linear (PL) combiner for  multiple relays is presented. Previous results were available only for upto three relays. The algorithm is unique in the sense that  the actual mathematical expressions, that are prohibitively large, need not be explicitly obtained. The diversity gain due to multiple relays is shown through plots of the analytical BER, well supported by simulations. 
%
%\end{abstract}
% IEEEtran.cls defaults to using nonbold math in the Abstract.
% This preserves the distinction between vectors and scalars. However,
% if the journal you are submitting to favors bold math in the abstract,
% then you can use LaTeX's standard command \boldmath at the very start
% of the abstract to achieve this. Many IEEE journals frown on math
% in the abstract anyway.

% Note that keywords are not normally used for peerreview papers.
%\begin{IEEEkeywords}
%Cooperative diversity, decode and forward, piecewise linear
%\end{IEEEkeywords}



% For peer review papers, you can put extra information on the cover
% page as needed:
% \ifCLASSOPTIONpeerreview
% \begin{center} \bfseries EDICS Category: 3-BBND \end{center}
% \fi
%
% For peerreview papers, this IEEEtran command inserts a page break and
% creates the second title. It will be ignored for other modes.
%\IEEEpeerreviewmaketitle




	\item Five cards—the ten, jack, queen, king and ace of diamonds, are well-shuffled with their face downwards. One card is then picked up at random.
\begin{enumerate}
\item
What is the probability that the card is the queen? 
\item
If the queen is drawn and put aside, what is the probability that the second card picked up is (a) an ace? (b) a queen?\\
\end{enumerate}
\solution
		%\begin{enumerate}[label=\thesection.\arabic*,ref=\thesection.\theenumi]
	\item One card is drawn from a well-shuffled deck of 52 cards. Find the probability of getting
\begin{enumerate}
\item A king of red colour 
\item A face card 
\item A red face card
\item The jack of hearts
\item A spade
\item The queen of diamonds

\end{enumerate}
\solution
		%\input{ncert/10/15/1/14/main.tex}
	\item Five cards—the ten, jack, queen, king and ace of diamonds, are well-shuffled with their face downwards. One card is then picked up at random.
\begin{enumerate}
\item
What is the probability that the card is the queen? 
\item
If the queen is drawn and put aside, what is the probability that the second card picked up is (a) an ace? (b) a queen?\\
\end{enumerate}
\solution
		%\input{ncert/10/15/1/15/defs.tex}
	\item A bag contains $5$ red balls and some blue balls. If the probability of drawing a blue ball is double that if a red ball, determine the number of blue balls in the bag. 
		\\
\solution
		%\input{ncert/10/15/2/3/defs.tex}
	\item A card is selected from a pack of 52 cards.
 \begin{enumerate}[label=(\alph*)] 
                 \item How many points are there in the sample space?
                 \item Calculate the probability that the card is an ace of spades.
                 \item Calculate the probability that the card is (i) an ace and (ii) black card.
 \end{enumerate}
\solution
		%\input{ncert/11/16/3/4/main.tex}
\item Four cards are drawn from a well-shuffled deck of 52 cards. What is the probability of obtaining 3 diamonds and one spade.
\\
\solution
		%\input{ncert/11/16/4/2/defs.tex}
\item In a certain lottery 10,000 tickets are sold and ten equal prizes are awarded. What is the probability of not getting a prize if you buy (a) one ticket (b) two tickets (c) 10 tickets ?	
\\
\solution
		%\input{ncert/11/16/4/4/defs.tex}
		%
\item 
Out of 100 students, two sections of 40 and 60 are formed. If you and your friend are among the 100 students, what is the probability that
\begin{enumerate}
\item you both enter the same section?
\item you both enter the different sections?
\end{enumerate}
\solution
		%\input{ncert/11/16/4/5/defs.tex}
	\item 
The number lock of a suitcase has 4 wheels each labelled with ten digits i.e. from 0 to 9.The lock opens with a sequence of four digits with no repeats.What is the probability of a person getting the right sequence to open the suitcase.
\\
\solution
		%\input{ncert/11/16/4/10/defs.tex}
		%
\item 
Two cards are drawn at random and without replacement from a pack of 52 playing cards. Find the probability that both the cards are black.
\\
\solution
		%\input{ncert/12/13/2/2/defs.tex}
		\item A box of oranges is inspected by examining three randomly selected oranges drawn without replacement. If all the three oranges are good, the box is approved for sale, otherwise, it is rejected. Find the probability that a box containing 15 oranges out of which 12 are good and 3 are bad ones will be approved for sale.
		\label{ncert/12/13/2/3/defs.tex}
		\item Two balls are drawn at random with replacement from a box containing 10 black and 8 red balls. Find the probability that
		\label{ncert/12/13/2/12}
\begin{enumerate}
\item both balls are red.
\item first ball is black and second is red.
\item one of them is black and other is red.
\end{enumerate}

\item In a hostel, 60\% of the students read Hindi newspaper, 40\% read English newspaper and 20\% read both Hindi and English newspapers. A student is selected at random.
		\label{ncert/12/13/2/15}
\begin{enumerate}
\item Find the probability that she reads neither Hindi nor English newspapers.
\item If she reads Hindi newspaper, find the probability that she reads English newspaper.
\item If she reads English newspaper, find the probability that she reads Hindi newspaper.\\
\end{enumerate}
\item The probability of obtaining an even prime number on each die, when a pair of dice is rolled is 
\begin{enumerate}
    \item $0$ 
    
    \item $\frac{1}{3}$ 
    
    \item $\frac{1}{12}$ 
    
    \item $\frac{1}{36}$ 
\end{enumerate}
\solution
		%\input{ncert/12/13/2/17/defs.tex}
	\item A bag contains 4 red and 4 black balls, another bag contains 2 red and 6 black balls. One of the two bags is selected at random and a ball is drawn from the bag which is found to be red. Find the probability that the ball is drawn from the first bag.
\\
\solution
		%\input{ncert/12/13/3/2/main.tex}
  \item
  Cards with numbers 2 to 101 are placed in a box. A card is selected at random.Find the probability that the card has
\begin{enumerate}[label=(\roman*)]
	\item an even number 
	\item a square number
\end{enumerate}
\solution
%\input{exemplar/10/13/3/32/main.tex}
\item
The king, queen and jack of clubs are removed from a deck of 52 playing cards and then well shuffled. Now one card is drawn at random from the remaining cards.  Determine the probability that the card is
\begin{enumerate}[label=(\roman*)]
\item a club
\item 10 of hearts
\end{enumerate}
\solution
%\input{exemplar/10/13/3/29/main.tex}
\item A team of medical students doing their internship have to assist during surgeries
at a city hospital. The probabilities of surgeries rated as very complex, complex,
routine, simple or very simple are respectively, 0.15, 0.20, 0.31, 0.26, .08. Find
the probabilities that a particular surgery will be rated
\begin{enumerate}
	\item complex or very complex;
	\item neither very complex nor very simple;
	\item routine or complex
	\item routine or simple
\end{enumerate}
\solution
%\input{exemplar/11/16/3/8(1)/main.tex}
\item A card is selected from a pack of 52 cards.
\begin{enumerate}[label=(\alph*)]
    \item How many points are there in the sample space?
    \item Calculate the probability that the card is an ace of spades.
    \item Calculate the probability that the card is (i) an ace and (ii) black card.
\end{enumerate}
\solution
%\input{exemplar/11/16/3/4/main2.tex}
\item The probability that a non leap year selected at random will contain 53 sundays.
\\
\solution
%\input{exemplar/10/13/1/19/main.tex}
\item One of the four persons John, Rita, Aslam or Gurpreet will be promoted next
month. Consequently the sample space consists of four elementary outcomes
S = {John promoted, Rita promoted, Aslam promoted, Gurpreet promoted}
You are told that the chances of John’s promotion is same as that of Gurpreet,
Rita’s chances of promotion are twice as likely as Johns. Aslam’s chances are
four times that of John.
\begin{enumerate}
	\item Determine
	\begin{enumerate}
		\item P (John promoted)
		\item P (Rita promoted)
		\item P (Aslam promoted)
		\item P (Gurpreet promoted)
	\end{enumerate}
	\item If A = {John promoted or Gurpreet promoted}, find P (A).
\end{enumerate}
\solution
%\input{exemplar/11/16/3/10/main.tex}
\item A card is drawn from a deck of 52 cards. Find the probability of getting a king or a heart or a red card.\\
\solution
%\input{exemplar/11/16/3/15/main.tex}
\item The probability that a student will pass his examination is 0.73, the probability of
the student getting a compartment is 0.13, and the probability that the student will
either pass or get compartment is 0.96. State True or False.\\
\solution
%\input{exemplar/11/16/3/31/main.tex}
\item A card is selected from a pack of 52 cards\\
\begin{enumerate}[label=(\alph*)]
\item How many points are there in the sample space?
\item Calculate the probability that the cards is an ace of spades.
\item Calculate the probability that the card is (i) an ace (ii)black card.\\
\end{enumerate}
%\input{ncert/11/16/3/4_1/Prob_4.tex}
\item In a non-leap year, the probability of having 53 tuesdays or 53 wednesdays is\\
\solution
%\input{exemplar/11/16/3/18/main.tex}
\item There are 1000 sealed envelopes in a box, 10 of them contain a cash prize of
Rs 100 each, 100 of them contain a cash prize of Rs 50 each and 200 of them
contain a cash prize of Rs 10 each and rest do not contain any cash prize. If they
are well shuffled and an envelope is picked up out, what is the probability that it
contains no cash prize?\\
\solution
%\input{exemplar/10/13/3/34/main.tex}
\item 
A die is thrown and a card is selected at random from a deck of 52 playing cards. The probability of getting an even number on the die and a spade card.\\
\solution
%\input{exemplar/12/13/3/78/main.tex}
\item
If 4-digit numbers greater than 5,000 are randomly formed from the digits 0, 1, 3, 5, and 7, what is the probability of forming a number divisible by 5 when:
\begin{enumerate}
    \item The digits are repeated?
    \item The repetition of digits is not allowed?
\end{enumerate}
\solution
%\input{ncert/11/16/4/9/main.tex}
\item Consider the probability space $\brak{\Omega, \mathcal{G}, P}$ where $\Omega = [0,2]$ and $\mathcal{G} = \cbrak{\phi, \Omega, [0,1], (1,2]}$. Let $X$ and $Y$ be two functions on $\Omega$ defined as
\begin{align*}
    X(\omega) = 
    \begin{cases}
        1 & \text{if }\omega \in [0, 1]\\
        2 & \text{if }\omega \in (1, 2]
    \end{cases}
\end{align*}
and
\begin{align*}
    Y(\omega) = 
    \begin{cases}
        2 & \text{if }\omega \in [0, 1.5]\\
        3 & \text{if }\omega \in (1.5, 2].
    \end{cases}
\end{align*}
Then which one of the following statements is true?
\begin{enumerate}
    \item [(A)] $X$ is a random variable with respect to $\mathcal{G}$, but $Y$ is not a random variable with respect to $\mathcal{G}$.
    \item [(B)] $Y$ is a random variable with respect to $\mathcal{G}$, but $X$ is not a random variable with respect to $\mathcal{G}$.
    \item [(C)] Neither $X$ nor $Y$ is a random variable with respect to $\mathcal{G}$.
    \item [(D)] Both $X$ and $Y$ are random variables with respect to $\mathcal{G}$.
\end{enumerate} \hfill (GATE ST 2023)\\
\solution
%\input{gate/ST/2023/14/main.tex}
	\item  A die is loaded in such a way that each odd number is twice as likely to occur as
each even number. Find $P(G)$, where $G$ is the event that a number greater than
3 occurs on a single roll of the die.
\\
\solution
		%\input{exemplar/11/16/3/5/main.tex}
	\item All the jacks, queens and kings are removed from a deck of 52 playing cards. The remaining cards are well shuffled and then one card is drawn at random. Giving ace a value 1 similar value for other cards, find the probability that the card has a value 
		\begin{enumerate}
			\item 7
			\item greater than 7
			\item less than 7
		\end{enumerate}
		%\input{exemplar/10/13/3/30/main.tex}
  \item A Lot consists of 48 mobile phones of which 42 are good, 3 have only minor defects and 3 have major defects.Varnika will buy a phone if it is good but the trader will only buy a mobile if it has no major defects. One phone is selected at random from the lot. What is the probability that it is
\begin{enumerate}
	\item acceptable to Varnika?
            \item acceptable to the trader?
\end{enumerate}
\solution
	%\input{exemplar/10/13/3/40/main.tex}
 \item A student says that if you throw a die, it will show up 1 or not 1. Therefore, the probability of getting 1 and the probability of getting 'not 1' each is equal to $\frac{1}{2}$. Is this correct? Give reasons.\\
 \solution
        %\input{exemplar/10/13/2/9/main.tex}
   \item Four candidates A, B, C, D have ap-
plied for the assignment to coach a school cricket
team. If A is twice as likely to be selected as B, and
B and C are given about the same chance of being
selected, while C is twice as likely to be selected
as D, what are the probabilities that
\begin{enumerate}
\item C will be selected?
\item A will not be selected?
\end{enumerate}
	%\input{exemplar/11/16/3/9/main.tex}
 \item A bag contain 24 balls of which $x$ balls are red, $2x$ are white and $3x$ are blue. A ball is selected at random, What is the probability that it is
\begin{enumerate}[label=\alph*)]
\item not red ?
\item white ?
\end{enumerate}
%\input{exemplar/10/13/3/41/main.tex}
If the letters of the word ASSASSINATION are arranged at random. Find the Probability that
\begin{enumerate}[label=(\alph*)]
\item Four $S's$ come consecutively in the word
\item Two  $I's$ and two $N's$ come together
\item All $A's$ are not coming together
\item No two $A's$ are coming together
\end{enumerate}
%\input{exemplar/11/16/3/14/main.tex}
	\item One urn contains two black balls (labelled B1 and B2) and one white ball. A
	second urn contains one black ball and two white balls (labelled W1 and W2).
	Suppose the following experiment is performed. One of the two urns is chosen
	at random. Next a ball is randomly chosen from the urn. Then a second ball is
	chosen at random from the same urn without replacing the first ball.
	
	\begin{enumerate}
	\item What is the probability that two black balls are chosen?
	
	\item What is the probability that two balls of opposite colour are chosen?
	\end{enumerate}
	\solution
	%\input{exemplar/11/16/3/12/main1.tex}
\end{enumerate}

	\item A bag contains $5$ red balls and some blue balls. If the probability of drawing a blue ball is double that if a red ball, determine the number of blue balls in the bag. 
		\\
\solution
		%\begin{enumerate}[label=\thesection.\arabic*,ref=\thesection.\theenumi]
	\item One card is drawn from a well-shuffled deck of 52 cards. Find the probability of getting
\begin{enumerate}
\item A king of red colour 
\item A face card 
\item A red face card
\item The jack of hearts
\item A spade
\item The queen of diamonds

\end{enumerate}
\solution
		%\input{ncert/10/15/1/14/main.tex}
	\item Five cards—the ten, jack, queen, king and ace of diamonds, are well-shuffled with their face downwards. One card is then picked up at random.
\begin{enumerate}
\item
What is the probability that the card is the queen? 
\item
If the queen is drawn and put aside, what is the probability that the second card picked up is (a) an ace? (b) a queen?\\
\end{enumerate}
\solution
		%\input{ncert/10/15/1/15/defs.tex}
	\item A bag contains $5$ red balls and some blue balls. If the probability of drawing a blue ball is double that if a red ball, determine the number of blue balls in the bag. 
		\\
\solution
		%\input{ncert/10/15/2/3/defs.tex}
	\item A card is selected from a pack of 52 cards.
 \begin{enumerate}[label=(\alph*)] 
                 \item How many points are there in the sample space?
                 \item Calculate the probability that the card is an ace of spades.
                 \item Calculate the probability that the card is (i) an ace and (ii) black card.
 \end{enumerate}
\solution
		%\input{ncert/11/16/3/4/main.tex}
\item Four cards are drawn from a well-shuffled deck of 52 cards. What is the probability of obtaining 3 diamonds and one spade.
\\
\solution
		%\input{ncert/11/16/4/2/defs.tex}
\item In a certain lottery 10,000 tickets are sold and ten equal prizes are awarded. What is the probability of not getting a prize if you buy (a) one ticket (b) two tickets (c) 10 tickets ?	
\\
\solution
		%\input{ncert/11/16/4/4/defs.tex}
		%
\item 
Out of 100 students, two sections of 40 and 60 are formed. If you and your friend are among the 100 students, what is the probability that
\begin{enumerate}
\item you both enter the same section?
\item you both enter the different sections?
\end{enumerate}
\solution
		%\input{ncert/11/16/4/5/defs.tex}
	\item 
The number lock of a suitcase has 4 wheels each labelled with ten digits i.e. from 0 to 9.The lock opens with a sequence of four digits with no repeats.What is the probability of a person getting the right sequence to open the suitcase.
\\
\solution
		%\input{ncert/11/16/4/10/defs.tex}
		%
\item 
Two cards are drawn at random and without replacement from a pack of 52 playing cards. Find the probability that both the cards are black.
\\
\solution
		%\input{ncert/12/13/2/2/defs.tex}
		\item A box of oranges is inspected by examining three randomly selected oranges drawn without replacement. If all the three oranges are good, the box is approved for sale, otherwise, it is rejected. Find the probability that a box containing 15 oranges out of which 12 are good and 3 are bad ones will be approved for sale.
		\label{ncert/12/13/2/3/defs.tex}
		\item Two balls are drawn at random with replacement from a box containing 10 black and 8 red balls. Find the probability that
		\label{ncert/12/13/2/12}
\begin{enumerate}
\item both balls are red.
\item first ball is black and second is red.
\item one of them is black and other is red.
\end{enumerate}

\item In a hostel, 60\% of the students read Hindi newspaper, 40\% read English newspaper and 20\% read both Hindi and English newspapers. A student is selected at random.
		\label{ncert/12/13/2/15}
\begin{enumerate}
\item Find the probability that she reads neither Hindi nor English newspapers.
\item If she reads Hindi newspaper, find the probability that she reads English newspaper.
\item If she reads English newspaper, find the probability that she reads Hindi newspaper.\\
\end{enumerate}
\item The probability of obtaining an even prime number on each die, when a pair of dice is rolled is 
\begin{enumerate}
    \item $0$ 
    
    \item $\frac{1}{3}$ 
    
    \item $\frac{1}{12}$ 
    
    \item $\frac{1}{36}$ 
\end{enumerate}
\solution
		%\input{ncert/12/13/2/17/defs.tex}
	\item A bag contains 4 red and 4 black balls, another bag contains 2 red and 6 black balls. One of the two bags is selected at random and a ball is drawn from the bag which is found to be red. Find the probability that the ball is drawn from the first bag.
\\
\solution
		%\input{ncert/12/13/3/2/main.tex}
  \item
  Cards with numbers 2 to 101 are placed in a box. A card is selected at random.Find the probability that the card has
\begin{enumerate}[label=(\roman*)]
	\item an even number 
	\item a square number
\end{enumerate}
\solution
%\input{exemplar/10/13/3/32/main.tex}
\item
The king, queen and jack of clubs are removed from a deck of 52 playing cards and then well shuffled. Now one card is drawn at random from the remaining cards.  Determine the probability that the card is
\begin{enumerate}[label=(\roman*)]
\item a club
\item 10 of hearts
\end{enumerate}
\solution
%\input{exemplar/10/13/3/29/main.tex}
\item A team of medical students doing their internship have to assist during surgeries
at a city hospital. The probabilities of surgeries rated as very complex, complex,
routine, simple or very simple are respectively, 0.15, 0.20, 0.31, 0.26, .08. Find
the probabilities that a particular surgery will be rated
\begin{enumerate}
	\item complex or very complex;
	\item neither very complex nor very simple;
	\item routine or complex
	\item routine or simple
\end{enumerate}
\solution
%\input{exemplar/11/16/3/8(1)/main.tex}
\item A card is selected from a pack of 52 cards.
\begin{enumerate}[label=(\alph*)]
    \item How many points are there in the sample space?
    \item Calculate the probability that the card is an ace of spades.
    \item Calculate the probability that the card is (i) an ace and (ii) black card.
\end{enumerate}
\solution
%\input{exemplar/11/16/3/4/main2.tex}
\item The probability that a non leap year selected at random will contain 53 sundays.
\\
\solution
%\input{exemplar/10/13/1/19/main.tex}
\item One of the four persons John, Rita, Aslam or Gurpreet will be promoted next
month. Consequently the sample space consists of four elementary outcomes
S = {John promoted, Rita promoted, Aslam promoted, Gurpreet promoted}
You are told that the chances of John’s promotion is same as that of Gurpreet,
Rita’s chances of promotion are twice as likely as Johns. Aslam’s chances are
four times that of John.
\begin{enumerate}
	\item Determine
	\begin{enumerate}
		\item P (John promoted)
		\item P (Rita promoted)
		\item P (Aslam promoted)
		\item P (Gurpreet promoted)
	\end{enumerate}
	\item If A = {John promoted or Gurpreet promoted}, find P (A).
\end{enumerate}
\solution
%\input{exemplar/11/16/3/10/main.tex}
\item A card is drawn from a deck of 52 cards. Find the probability of getting a king or a heart or a red card.\\
\solution
%\input{exemplar/11/16/3/15/main.tex}
\item The probability that a student will pass his examination is 0.73, the probability of
the student getting a compartment is 0.13, and the probability that the student will
either pass or get compartment is 0.96. State True or False.\\
\solution
%\input{exemplar/11/16/3/31/main.tex}
\item A card is selected from a pack of 52 cards\\
\begin{enumerate}[label=(\alph*)]
\item How many points are there in the sample space?
\item Calculate the probability that the cards is an ace of spades.
\item Calculate the probability that the card is (i) an ace (ii)black card.\\
\end{enumerate}
%\input{ncert/11/16/3/4_1/Prob_4.tex}
\item In a non-leap year, the probability of having 53 tuesdays or 53 wednesdays is\\
\solution
%\input{exemplar/11/16/3/18/main.tex}
\item There are 1000 sealed envelopes in a box, 10 of them contain a cash prize of
Rs 100 each, 100 of them contain a cash prize of Rs 50 each and 200 of them
contain a cash prize of Rs 10 each and rest do not contain any cash prize. If they
are well shuffled and an envelope is picked up out, what is the probability that it
contains no cash prize?\\
\solution
%\input{exemplar/10/13/3/34/main.tex}
\item 
A die is thrown and a card is selected at random from a deck of 52 playing cards. The probability of getting an even number on the die and a spade card.\\
\solution
%\input{exemplar/12/13/3/78/main.tex}
\item
If 4-digit numbers greater than 5,000 are randomly formed from the digits 0, 1, 3, 5, and 7, what is the probability of forming a number divisible by 5 when:
\begin{enumerate}
    \item The digits are repeated?
    \item The repetition of digits is not allowed?
\end{enumerate}
\solution
%\input{ncert/11/16/4/9/main.tex}
\item Consider the probability space $\brak{\Omega, \mathcal{G}, P}$ where $\Omega = [0,2]$ and $\mathcal{G} = \cbrak{\phi, \Omega, [0,1], (1,2]}$. Let $X$ and $Y$ be two functions on $\Omega$ defined as
\begin{align*}
    X(\omega) = 
    \begin{cases}
        1 & \text{if }\omega \in [0, 1]\\
        2 & \text{if }\omega \in (1, 2]
    \end{cases}
\end{align*}
and
\begin{align*}
    Y(\omega) = 
    \begin{cases}
        2 & \text{if }\omega \in [0, 1.5]\\
        3 & \text{if }\omega \in (1.5, 2].
    \end{cases}
\end{align*}
Then which one of the following statements is true?
\begin{enumerate}
    \item [(A)] $X$ is a random variable with respect to $\mathcal{G}$, but $Y$ is not a random variable with respect to $\mathcal{G}$.
    \item [(B)] $Y$ is a random variable with respect to $\mathcal{G}$, but $X$ is not a random variable with respect to $\mathcal{G}$.
    \item [(C)] Neither $X$ nor $Y$ is a random variable with respect to $\mathcal{G}$.
    \item [(D)] Both $X$ and $Y$ are random variables with respect to $\mathcal{G}$.
\end{enumerate} \hfill (GATE ST 2023)\\
\solution
%\input{gate/ST/2023/14/main.tex}
	\item  A die is loaded in such a way that each odd number is twice as likely to occur as
each even number. Find $P(G)$, where $G$ is the event that a number greater than
3 occurs on a single roll of the die.
\\
\solution
		%\input{exemplar/11/16/3/5/main.tex}
	\item All the jacks, queens and kings are removed from a deck of 52 playing cards. The remaining cards are well shuffled and then one card is drawn at random. Giving ace a value 1 similar value for other cards, find the probability that the card has a value 
		\begin{enumerate}
			\item 7
			\item greater than 7
			\item less than 7
		\end{enumerate}
		%\input{exemplar/10/13/3/30/main.tex}
  \item A Lot consists of 48 mobile phones of which 42 are good, 3 have only minor defects and 3 have major defects.Varnika will buy a phone if it is good but the trader will only buy a mobile if it has no major defects. One phone is selected at random from the lot. What is the probability that it is
\begin{enumerate}
	\item acceptable to Varnika?
            \item acceptable to the trader?
\end{enumerate}
\solution
	%\input{exemplar/10/13/3/40/main.tex}
 \item A student says that if you throw a die, it will show up 1 or not 1. Therefore, the probability of getting 1 and the probability of getting 'not 1' each is equal to $\frac{1}{2}$. Is this correct? Give reasons.\\
 \solution
        %\input{exemplar/10/13/2/9/main.tex}
   \item Four candidates A, B, C, D have ap-
plied for the assignment to coach a school cricket
team. If A is twice as likely to be selected as B, and
B and C are given about the same chance of being
selected, while C is twice as likely to be selected
as D, what are the probabilities that
\begin{enumerate}
\item C will be selected?
\item A will not be selected?
\end{enumerate}
	%\input{exemplar/11/16/3/9/main.tex}
 \item A bag contain 24 balls of which $x$ balls are red, $2x$ are white and $3x$ are blue. A ball is selected at random, What is the probability that it is
\begin{enumerate}[label=\alph*)]
\item not red ?
\item white ?
\end{enumerate}
%\input{exemplar/10/13/3/41/main.tex}
If the letters of the word ASSASSINATION are arranged at random. Find the Probability that
\begin{enumerate}[label=(\alph*)]
\item Four $S's$ come consecutively in the word
\item Two  $I's$ and two $N's$ come together
\item All $A's$ are not coming together
\item No two $A's$ are coming together
\end{enumerate}
%\input{exemplar/11/16/3/14/main.tex}
	\item One urn contains two black balls (labelled B1 and B2) and one white ball. A
	second urn contains one black ball and two white balls (labelled W1 and W2).
	Suppose the following experiment is performed. One of the two urns is chosen
	at random. Next a ball is randomly chosen from the urn. Then a second ball is
	chosen at random from the same urn without replacing the first ball.
	
	\begin{enumerate}
	\item What is the probability that two black balls are chosen?
	
	\item What is the probability that two balls of opposite colour are chosen?
	\end{enumerate}
	\solution
	%\input{exemplar/11/16/3/12/main1.tex}
\end{enumerate}

	\item A card is selected from a pack of 52 cards.
 \begin{enumerate}[label=(\alph*)] 
                 \item How many points are there in the sample space?
                 \item Calculate the probability that the card is an ace of spades.
                 \item Calculate the probability that the card is (i) an ace and (ii) black card.
 \end{enumerate}
\solution
		%\begin{table}[H]
	\centering
\begin{tabular}{|c|c|c|}
\hline
Random variable &Value &Definition\\ \hline
\multirow{3}{*}{X} &0 &Slips of Rs 1\\
&1 &Slips of Rs 5\\
&2 &Slips of Rs 13\\ \hline
\multirow{2}{*}{Y} &0 &Box A\\
&1 &Box B\\\hline
\end{tabular}
\caption{}
\label{tab:Distribution}
\end{table}
See \tabref{tab:Distribution}.
\begin{align}
p_{Y}\brak{k}= \begin{cases} 
      \frac{1}{3} & {k=0} \\
      \frac{2}{3 }& {k=1} 
   \end{cases}
   \\
p_{Y|X}\brak{0|0} = \frac{19}{25}\, 
p_{Y|X}\brak{0|1} = \frac{6}{25}\,
p_{Y|X}\brak{1|0} = \frac{45}{50}\,
p_{Y|X}\brak{1|2} = \frac{5}{50}
\end{align}
The desired probability is the probability that a slip drawn at random is marked other than Rs 1,
\begin{align}
&=1-p_X\brak{0}\\
&= p_X(1) + p_X(2)
\end{align}
Using Bayes theorem,
\begin{align}
&= p_Y\brak{0} \times \pr{Y=0 | X=1} + p_Y\brak{1} \times \pr{Y=1|X=2}\\
&=\frac{1}{3} \times \frac{6}{25} + \frac{2}{3} \times \frac{5}{50}\\
&=\frac{11}{75}
\end{align}

\newpage

%\tableofcontents

\bigskip

\renewcommand{\thefigure}{\theenumi}
\renewcommand{\thetable}{\theenumi}
%\renewcommand{\theequation}{\theenumi}

%\begin{abstract}
%%\boldmath
%In this letter, an algorithm for evaluating the exact analytical bit error rate  (BER)  for the piecewise linear (PL) combiner for  multiple relays is presented. Previous results were available only for upto three relays. The algorithm is unique in the sense that  the actual mathematical expressions, that are prohibitively large, need not be explicitly obtained. The diversity gain due to multiple relays is shown through plots of the analytical BER, well supported by simulations. 
%
%\end{abstract}
% IEEEtran.cls defaults to using nonbold math in the Abstract.
% This preserves the distinction between vectors and scalars. However,
% if the journal you are submitting to favors bold math in the abstract,
% then you can use LaTeX's standard command \boldmath at the very start
% of the abstract to achieve this. Many IEEE journals frown on math
% in the abstract anyway.

% Note that keywords are not normally used for peerreview papers.
%\begin{IEEEkeywords}
%Cooperative diversity, decode and forward, piecewise linear
%\end{IEEEkeywords}



% For peer review papers, you can put extra information on the cover
% page as needed:
% \ifCLASSOPTIONpeerreview
% \begin{center} \bfseries EDICS Category: 3-BBND \end{center}
% \fi
%
% For peerreview papers, this IEEEtran command inserts a page break and
% creates the second title. It will be ignored for other modes.
%\IEEEpeerreviewmaketitle




\item Four cards are drawn from a well-shuffled deck of 52 cards. What is the probability of obtaining 3 diamonds and one spade.
\\
\solution
		%\begin{enumerate}[label=\thesection.\arabic*,ref=\thesection.\theenumi]
	\item One card is drawn from a well-shuffled deck of 52 cards. Find the probability of getting
\begin{enumerate}
\item A king of red colour 
\item A face card 
\item A red face card
\item The jack of hearts
\item A spade
\item The queen of diamonds

\end{enumerate}
\solution
		%\input{ncert/10/15/1/14/main.tex}
	\item Five cards—the ten, jack, queen, king and ace of diamonds, are well-shuffled with their face downwards. One card is then picked up at random.
\begin{enumerate}
\item
What is the probability that the card is the queen? 
\item
If the queen is drawn and put aside, what is the probability that the second card picked up is (a) an ace? (b) a queen?\\
\end{enumerate}
\solution
		%\input{ncert/10/15/1/15/defs.tex}
	\item A bag contains $5$ red balls and some blue balls. If the probability of drawing a blue ball is double that if a red ball, determine the number of blue balls in the bag. 
		\\
\solution
		%\input{ncert/10/15/2/3/defs.tex}
	\item A card is selected from a pack of 52 cards.
 \begin{enumerate}[label=(\alph*)] 
                 \item How many points are there in the sample space?
                 \item Calculate the probability that the card is an ace of spades.
                 \item Calculate the probability that the card is (i) an ace and (ii) black card.
 \end{enumerate}
\solution
		%\input{ncert/11/16/3/4/main.tex}
\item Four cards are drawn from a well-shuffled deck of 52 cards. What is the probability of obtaining 3 diamonds and one spade.
\\
\solution
		%\input{ncert/11/16/4/2/defs.tex}
\item In a certain lottery 10,000 tickets are sold and ten equal prizes are awarded. What is the probability of not getting a prize if you buy (a) one ticket (b) two tickets (c) 10 tickets ?	
\\
\solution
		%\input{ncert/11/16/4/4/defs.tex}
		%
\item 
Out of 100 students, two sections of 40 and 60 are formed. If you and your friend are among the 100 students, what is the probability that
\begin{enumerate}
\item you both enter the same section?
\item you both enter the different sections?
\end{enumerate}
\solution
		%\input{ncert/11/16/4/5/defs.tex}
	\item 
The number lock of a suitcase has 4 wheels each labelled with ten digits i.e. from 0 to 9.The lock opens with a sequence of four digits with no repeats.What is the probability of a person getting the right sequence to open the suitcase.
\\
\solution
		%\input{ncert/11/16/4/10/defs.tex}
		%
\item 
Two cards are drawn at random and without replacement from a pack of 52 playing cards. Find the probability that both the cards are black.
\\
\solution
		%\input{ncert/12/13/2/2/defs.tex}
		\item A box of oranges is inspected by examining three randomly selected oranges drawn without replacement. If all the three oranges are good, the box is approved for sale, otherwise, it is rejected. Find the probability that a box containing 15 oranges out of which 12 are good and 3 are bad ones will be approved for sale.
		\label{ncert/12/13/2/3/defs.tex}
		\item Two balls are drawn at random with replacement from a box containing 10 black and 8 red balls. Find the probability that
		\label{ncert/12/13/2/12}
\begin{enumerate}
\item both balls are red.
\item first ball is black and second is red.
\item one of them is black and other is red.
\end{enumerate}

\item In a hostel, 60\% of the students read Hindi newspaper, 40\% read English newspaper and 20\% read both Hindi and English newspapers. A student is selected at random.
		\label{ncert/12/13/2/15}
\begin{enumerate}
\item Find the probability that she reads neither Hindi nor English newspapers.
\item If she reads Hindi newspaper, find the probability that she reads English newspaper.
\item If she reads English newspaper, find the probability that she reads Hindi newspaper.\\
\end{enumerate}
\item The probability of obtaining an even prime number on each die, when a pair of dice is rolled is 
\begin{enumerate}
    \item $0$ 
    
    \item $\frac{1}{3}$ 
    
    \item $\frac{1}{12}$ 
    
    \item $\frac{1}{36}$ 
\end{enumerate}
\solution
		%\input{ncert/12/13/2/17/defs.tex}
	\item A bag contains 4 red and 4 black balls, another bag contains 2 red and 6 black balls. One of the two bags is selected at random and a ball is drawn from the bag which is found to be red. Find the probability that the ball is drawn from the first bag.
\\
\solution
		%\input{ncert/12/13/3/2/main.tex}
  \item
  Cards with numbers 2 to 101 are placed in a box. A card is selected at random.Find the probability that the card has
\begin{enumerate}[label=(\roman*)]
	\item an even number 
	\item a square number
\end{enumerate}
\solution
%\input{exemplar/10/13/3/32/main.tex}
\item
The king, queen and jack of clubs are removed from a deck of 52 playing cards and then well shuffled. Now one card is drawn at random from the remaining cards.  Determine the probability that the card is
\begin{enumerate}[label=(\roman*)]
\item a club
\item 10 of hearts
\end{enumerate}
\solution
%\input{exemplar/10/13/3/29/main.tex}
\item A team of medical students doing their internship have to assist during surgeries
at a city hospital. The probabilities of surgeries rated as very complex, complex,
routine, simple or very simple are respectively, 0.15, 0.20, 0.31, 0.26, .08. Find
the probabilities that a particular surgery will be rated
\begin{enumerate}
	\item complex or very complex;
	\item neither very complex nor very simple;
	\item routine or complex
	\item routine or simple
\end{enumerate}
\solution
%\input{exemplar/11/16/3/8(1)/main.tex}
\item A card is selected from a pack of 52 cards.
\begin{enumerate}[label=(\alph*)]
    \item How many points are there in the sample space?
    \item Calculate the probability that the card is an ace of spades.
    \item Calculate the probability that the card is (i) an ace and (ii) black card.
\end{enumerate}
\solution
%\input{exemplar/11/16/3/4/main2.tex}
\item The probability that a non leap year selected at random will contain 53 sundays.
\\
\solution
%\input{exemplar/10/13/1/19/main.tex}
\item One of the four persons John, Rita, Aslam or Gurpreet will be promoted next
month. Consequently the sample space consists of four elementary outcomes
S = {John promoted, Rita promoted, Aslam promoted, Gurpreet promoted}
You are told that the chances of John’s promotion is same as that of Gurpreet,
Rita’s chances of promotion are twice as likely as Johns. Aslam’s chances are
four times that of John.
\begin{enumerate}
	\item Determine
	\begin{enumerate}
		\item P (John promoted)
		\item P (Rita promoted)
		\item P (Aslam promoted)
		\item P (Gurpreet promoted)
	\end{enumerate}
	\item If A = {John promoted or Gurpreet promoted}, find P (A).
\end{enumerate}
\solution
%\input{exemplar/11/16/3/10/main.tex}
\item A card is drawn from a deck of 52 cards. Find the probability of getting a king or a heart or a red card.\\
\solution
%\input{exemplar/11/16/3/15/main.tex}
\item The probability that a student will pass his examination is 0.73, the probability of
the student getting a compartment is 0.13, and the probability that the student will
either pass or get compartment is 0.96. State True or False.\\
\solution
%\input{exemplar/11/16/3/31/main.tex}
\item A card is selected from a pack of 52 cards\\
\begin{enumerate}[label=(\alph*)]
\item How many points are there in the sample space?
\item Calculate the probability that the cards is an ace of spades.
\item Calculate the probability that the card is (i) an ace (ii)black card.\\
\end{enumerate}
%\input{ncert/11/16/3/4_1/Prob_4.tex}
\item In a non-leap year, the probability of having 53 tuesdays or 53 wednesdays is\\
\solution
%\input{exemplar/11/16/3/18/main.tex}
\item There are 1000 sealed envelopes in a box, 10 of them contain a cash prize of
Rs 100 each, 100 of them contain a cash prize of Rs 50 each and 200 of them
contain a cash prize of Rs 10 each and rest do not contain any cash prize. If they
are well shuffled and an envelope is picked up out, what is the probability that it
contains no cash prize?\\
\solution
%\input{exemplar/10/13/3/34/main.tex}
\item 
A die is thrown and a card is selected at random from a deck of 52 playing cards. The probability of getting an even number on the die and a spade card.\\
\solution
%\input{exemplar/12/13/3/78/main.tex}
\item
If 4-digit numbers greater than 5,000 are randomly formed from the digits 0, 1, 3, 5, and 7, what is the probability of forming a number divisible by 5 when:
\begin{enumerate}
    \item The digits are repeated?
    \item The repetition of digits is not allowed?
\end{enumerate}
\solution
%\input{ncert/11/16/4/9/main.tex}
\item Consider the probability space $\brak{\Omega, \mathcal{G}, P}$ where $\Omega = [0,2]$ and $\mathcal{G} = \cbrak{\phi, \Omega, [0,1], (1,2]}$. Let $X$ and $Y$ be two functions on $\Omega$ defined as
\begin{align*}
    X(\omega) = 
    \begin{cases}
        1 & \text{if }\omega \in [0, 1]\\
        2 & \text{if }\omega \in (1, 2]
    \end{cases}
\end{align*}
and
\begin{align*}
    Y(\omega) = 
    \begin{cases}
        2 & \text{if }\omega \in [0, 1.5]\\
        3 & \text{if }\omega \in (1.5, 2].
    \end{cases}
\end{align*}
Then which one of the following statements is true?
\begin{enumerate}
    \item [(A)] $X$ is a random variable with respect to $\mathcal{G}$, but $Y$ is not a random variable with respect to $\mathcal{G}$.
    \item [(B)] $Y$ is a random variable with respect to $\mathcal{G}$, but $X$ is not a random variable with respect to $\mathcal{G}$.
    \item [(C)] Neither $X$ nor $Y$ is a random variable with respect to $\mathcal{G}$.
    \item [(D)] Both $X$ and $Y$ are random variables with respect to $\mathcal{G}$.
\end{enumerate} \hfill (GATE ST 2023)\\
\solution
%\input{gate/ST/2023/14/main.tex}
	\item  A die is loaded in such a way that each odd number is twice as likely to occur as
each even number. Find $P(G)$, where $G$ is the event that a number greater than
3 occurs on a single roll of the die.
\\
\solution
		%\input{exemplar/11/16/3/5/main.tex}
	\item All the jacks, queens and kings are removed from a deck of 52 playing cards. The remaining cards are well shuffled and then one card is drawn at random. Giving ace a value 1 similar value for other cards, find the probability that the card has a value 
		\begin{enumerate}
			\item 7
			\item greater than 7
			\item less than 7
		\end{enumerate}
		%\input{exemplar/10/13/3/30/main.tex}
  \item A Lot consists of 48 mobile phones of which 42 are good, 3 have only minor defects and 3 have major defects.Varnika will buy a phone if it is good but the trader will only buy a mobile if it has no major defects. One phone is selected at random from the lot. What is the probability that it is
\begin{enumerate}
	\item acceptable to Varnika?
            \item acceptable to the trader?
\end{enumerate}
\solution
	%\input{exemplar/10/13/3/40/main.tex}
 \item A student says that if you throw a die, it will show up 1 or not 1. Therefore, the probability of getting 1 and the probability of getting 'not 1' each is equal to $\frac{1}{2}$. Is this correct? Give reasons.\\
 \solution
        %\input{exemplar/10/13/2/9/main.tex}
   \item Four candidates A, B, C, D have ap-
plied for the assignment to coach a school cricket
team. If A is twice as likely to be selected as B, and
B and C are given about the same chance of being
selected, while C is twice as likely to be selected
as D, what are the probabilities that
\begin{enumerate}
\item C will be selected?
\item A will not be selected?
\end{enumerate}
	%\input{exemplar/11/16/3/9/main.tex}
 \item A bag contain 24 balls of which $x$ balls are red, $2x$ are white and $3x$ are blue. A ball is selected at random, What is the probability that it is
\begin{enumerate}[label=\alph*)]
\item not red ?
\item white ?
\end{enumerate}
%\input{exemplar/10/13/3/41/main.tex}
If the letters of the word ASSASSINATION are arranged at random. Find the Probability that
\begin{enumerate}[label=(\alph*)]
\item Four $S's$ come consecutively in the word
\item Two  $I's$ and two $N's$ come together
\item All $A's$ are not coming together
\item No two $A's$ are coming together
\end{enumerate}
%\input{exemplar/11/16/3/14/main.tex}
	\item One urn contains two black balls (labelled B1 and B2) and one white ball. A
	second urn contains one black ball and two white balls (labelled W1 and W2).
	Suppose the following experiment is performed. One of the two urns is chosen
	at random. Next a ball is randomly chosen from the urn. Then a second ball is
	chosen at random from the same urn without replacing the first ball.
	
	\begin{enumerate}
	\item What is the probability that two black balls are chosen?
	
	\item What is the probability that two balls of opposite colour are chosen?
	\end{enumerate}
	\solution
	%\input{exemplar/11/16/3/12/main1.tex}
\end{enumerate}

\item In a certain lottery 10,000 tickets are sold and ten equal prizes are awarded. What is the probability of not getting a prize if you buy (a) one ticket (b) two tickets (c) 10 tickets ?	
\\
\solution
		%\begin{enumerate}[label=\thesection.\arabic*,ref=\thesection.\theenumi]
	\item One card is drawn from a well-shuffled deck of 52 cards. Find the probability of getting
\begin{enumerate}
\item A king of red colour 
\item A face card 
\item A red face card
\item The jack of hearts
\item A spade
\item The queen of diamonds

\end{enumerate}
\solution
		%\input{ncert/10/15/1/14/main.tex}
	\item Five cards—the ten, jack, queen, king and ace of diamonds, are well-shuffled with their face downwards. One card is then picked up at random.
\begin{enumerate}
\item
What is the probability that the card is the queen? 
\item
If the queen is drawn and put aside, what is the probability that the second card picked up is (a) an ace? (b) a queen?\\
\end{enumerate}
\solution
		%\input{ncert/10/15/1/15/defs.tex}
	\item A bag contains $5$ red balls and some blue balls. If the probability of drawing a blue ball is double that if a red ball, determine the number of blue balls in the bag. 
		\\
\solution
		%\input{ncert/10/15/2/3/defs.tex}
	\item A card is selected from a pack of 52 cards.
 \begin{enumerate}[label=(\alph*)] 
                 \item How many points are there in the sample space?
                 \item Calculate the probability that the card is an ace of spades.
                 \item Calculate the probability that the card is (i) an ace and (ii) black card.
 \end{enumerate}
\solution
		%\input{ncert/11/16/3/4/main.tex}
\item Four cards are drawn from a well-shuffled deck of 52 cards. What is the probability of obtaining 3 diamonds and one spade.
\\
\solution
		%\input{ncert/11/16/4/2/defs.tex}
\item In a certain lottery 10,000 tickets are sold and ten equal prizes are awarded. What is the probability of not getting a prize if you buy (a) one ticket (b) two tickets (c) 10 tickets ?	
\\
\solution
		%\input{ncert/11/16/4/4/defs.tex}
		%
\item 
Out of 100 students, two sections of 40 and 60 are formed. If you and your friend are among the 100 students, what is the probability that
\begin{enumerate}
\item you both enter the same section?
\item you both enter the different sections?
\end{enumerate}
\solution
		%\input{ncert/11/16/4/5/defs.tex}
	\item 
The number lock of a suitcase has 4 wheels each labelled with ten digits i.e. from 0 to 9.The lock opens with a sequence of four digits with no repeats.What is the probability of a person getting the right sequence to open the suitcase.
\\
\solution
		%\input{ncert/11/16/4/10/defs.tex}
		%
\item 
Two cards are drawn at random and without replacement from a pack of 52 playing cards. Find the probability that both the cards are black.
\\
\solution
		%\input{ncert/12/13/2/2/defs.tex}
		\item A box of oranges is inspected by examining three randomly selected oranges drawn without replacement. If all the three oranges are good, the box is approved for sale, otherwise, it is rejected. Find the probability that a box containing 15 oranges out of which 12 are good and 3 are bad ones will be approved for sale.
		\label{ncert/12/13/2/3/defs.tex}
		\item Two balls are drawn at random with replacement from a box containing 10 black and 8 red balls. Find the probability that
		\label{ncert/12/13/2/12}
\begin{enumerate}
\item both balls are red.
\item first ball is black and second is red.
\item one of them is black and other is red.
\end{enumerate}

\item In a hostel, 60\% of the students read Hindi newspaper, 40\% read English newspaper and 20\% read both Hindi and English newspapers. A student is selected at random.
		\label{ncert/12/13/2/15}
\begin{enumerate}
\item Find the probability that she reads neither Hindi nor English newspapers.
\item If she reads Hindi newspaper, find the probability that she reads English newspaper.
\item If she reads English newspaper, find the probability that she reads Hindi newspaper.\\
\end{enumerate}
\item The probability of obtaining an even prime number on each die, when a pair of dice is rolled is 
\begin{enumerate}
    \item $0$ 
    
    \item $\frac{1}{3}$ 
    
    \item $\frac{1}{12}$ 
    
    \item $\frac{1}{36}$ 
\end{enumerate}
\solution
		%\input{ncert/12/13/2/17/defs.tex}
	\item A bag contains 4 red and 4 black balls, another bag contains 2 red and 6 black balls. One of the two bags is selected at random and a ball is drawn from the bag which is found to be red. Find the probability that the ball is drawn from the first bag.
\\
\solution
		%\input{ncert/12/13/3/2/main.tex}
  \item
  Cards with numbers 2 to 101 are placed in a box. A card is selected at random.Find the probability that the card has
\begin{enumerate}[label=(\roman*)]
	\item an even number 
	\item a square number
\end{enumerate}
\solution
%\input{exemplar/10/13/3/32/main.tex}
\item
The king, queen and jack of clubs are removed from a deck of 52 playing cards and then well shuffled. Now one card is drawn at random from the remaining cards.  Determine the probability that the card is
\begin{enumerate}[label=(\roman*)]
\item a club
\item 10 of hearts
\end{enumerate}
\solution
%\input{exemplar/10/13/3/29/main.tex}
\item A team of medical students doing their internship have to assist during surgeries
at a city hospital. The probabilities of surgeries rated as very complex, complex,
routine, simple or very simple are respectively, 0.15, 0.20, 0.31, 0.26, .08. Find
the probabilities that a particular surgery will be rated
\begin{enumerate}
	\item complex or very complex;
	\item neither very complex nor very simple;
	\item routine or complex
	\item routine or simple
\end{enumerate}
\solution
%\input{exemplar/11/16/3/8(1)/main.tex}
\item A card is selected from a pack of 52 cards.
\begin{enumerate}[label=(\alph*)]
    \item How many points are there in the sample space?
    \item Calculate the probability that the card is an ace of spades.
    \item Calculate the probability that the card is (i) an ace and (ii) black card.
\end{enumerate}
\solution
%\input{exemplar/11/16/3/4/main2.tex}
\item The probability that a non leap year selected at random will contain 53 sundays.
\\
\solution
%\input{exemplar/10/13/1/19/main.tex}
\item One of the four persons John, Rita, Aslam or Gurpreet will be promoted next
month. Consequently the sample space consists of four elementary outcomes
S = {John promoted, Rita promoted, Aslam promoted, Gurpreet promoted}
You are told that the chances of John’s promotion is same as that of Gurpreet,
Rita’s chances of promotion are twice as likely as Johns. Aslam’s chances are
four times that of John.
\begin{enumerate}
	\item Determine
	\begin{enumerate}
		\item P (John promoted)
		\item P (Rita promoted)
		\item P (Aslam promoted)
		\item P (Gurpreet promoted)
	\end{enumerate}
	\item If A = {John promoted or Gurpreet promoted}, find P (A).
\end{enumerate}
\solution
%\input{exemplar/11/16/3/10/main.tex}
\item A card is drawn from a deck of 52 cards. Find the probability of getting a king or a heart or a red card.\\
\solution
%\input{exemplar/11/16/3/15/main.tex}
\item The probability that a student will pass his examination is 0.73, the probability of
the student getting a compartment is 0.13, and the probability that the student will
either pass or get compartment is 0.96. State True or False.\\
\solution
%\input{exemplar/11/16/3/31/main.tex}
\item A card is selected from a pack of 52 cards\\
\begin{enumerate}[label=(\alph*)]
\item How many points are there in the sample space?
\item Calculate the probability that the cards is an ace of spades.
\item Calculate the probability that the card is (i) an ace (ii)black card.\\
\end{enumerate}
%\input{ncert/11/16/3/4_1/Prob_4.tex}
\item In a non-leap year, the probability of having 53 tuesdays or 53 wednesdays is\\
\solution
%\input{exemplar/11/16/3/18/main.tex}
\item There are 1000 sealed envelopes in a box, 10 of them contain a cash prize of
Rs 100 each, 100 of them contain a cash prize of Rs 50 each and 200 of them
contain a cash prize of Rs 10 each and rest do not contain any cash prize. If they
are well shuffled and an envelope is picked up out, what is the probability that it
contains no cash prize?\\
\solution
%\input{exemplar/10/13/3/34/main.tex}
\item 
A die is thrown and a card is selected at random from a deck of 52 playing cards. The probability of getting an even number on the die and a spade card.\\
\solution
%\input{exemplar/12/13/3/78/main.tex}
\item
If 4-digit numbers greater than 5,000 are randomly formed from the digits 0, 1, 3, 5, and 7, what is the probability of forming a number divisible by 5 when:
\begin{enumerate}
    \item The digits are repeated?
    \item The repetition of digits is not allowed?
\end{enumerate}
\solution
%\input{ncert/11/16/4/9/main.tex}
\item Consider the probability space $\brak{\Omega, \mathcal{G}, P}$ where $\Omega = [0,2]$ and $\mathcal{G} = \cbrak{\phi, \Omega, [0,1], (1,2]}$. Let $X$ and $Y$ be two functions on $\Omega$ defined as
\begin{align*}
    X(\omega) = 
    \begin{cases}
        1 & \text{if }\omega \in [0, 1]\\
        2 & \text{if }\omega \in (1, 2]
    \end{cases}
\end{align*}
and
\begin{align*}
    Y(\omega) = 
    \begin{cases}
        2 & \text{if }\omega \in [0, 1.5]\\
        3 & \text{if }\omega \in (1.5, 2].
    \end{cases}
\end{align*}
Then which one of the following statements is true?
\begin{enumerate}
    \item [(A)] $X$ is a random variable with respect to $\mathcal{G}$, but $Y$ is not a random variable with respect to $\mathcal{G}$.
    \item [(B)] $Y$ is a random variable with respect to $\mathcal{G}$, but $X$ is not a random variable with respect to $\mathcal{G}$.
    \item [(C)] Neither $X$ nor $Y$ is a random variable with respect to $\mathcal{G}$.
    \item [(D)] Both $X$ and $Y$ are random variables with respect to $\mathcal{G}$.
\end{enumerate} \hfill (GATE ST 2023)\\
\solution
%\input{gate/ST/2023/14/main.tex}
	\item  A die is loaded in such a way that each odd number is twice as likely to occur as
each even number. Find $P(G)$, where $G$ is the event that a number greater than
3 occurs on a single roll of the die.
\\
\solution
		%\input{exemplar/11/16/3/5/main.tex}
	\item All the jacks, queens and kings are removed from a deck of 52 playing cards. The remaining cards are well shuffled and then one card is drawn at random. Giving ace a value 1 similar value for other cards, find the probability that the card has a value 
		\begin{enumerate}
			\item 7
			\item greater than 7
			\item less than 7
		\end{enumerate}
		%\input{exemplar/10/13/3/30/main.tex}
  \item A Lot consists of 48 mobile phones of which 42 are good, 3 have only minor defects and 3 have major defects.Varnika will buy a phone if it is good but the trader will only buy a mobile if it has no major defects. One phone is selected at random from the lot. What is the probability that it is
\begin{enumerate}
	\item acceptable to Varnika?
            \item acceptable to the trader?
\end{enumerate}
\solution
	%\input{exemplar/10/13/3/40/main.tex}
 \item A student says that if you throw a die, it will show up 1 or not 1. Therefore, the probability of getting 1 and the probability of getting 'not 1' each is equal to $\frac{1}{2}$. Is this correct? Give reasons.\\
 \solution
        %\input{exemplar/10/13/2/9/main.tex}
   \item Four candidates A, B, C, D have ap-
plied for the assignment to coach a school cricket
team. If A is twice as likely to be selected as B, and
B and C are given about the same chance of being
selected, while C is twice as likely to be selected
as D, what are the probabilities that
\begin{enumerate}
\item C will be selected?
\item A will not be selected?
\end{enumerate}
	%\input{exemplar/11/16/3/9/main.tex}
 \item A bag contain 24 balls of which $x$ balls are red, $2x$ are white and $3x$ are blue. A ball is selected at random, What is the probability that it is
\begin{enumerate}[label=\alph*)]
\item not red ?
\item white ?
\end{enumerate}
%\input{exemplar/10/13/3/41/main.tex}
If the letters of the word ASSASSINATION are arranged at random. Find the Probability that
\begin{enumerate}[label=(\alph*)]
\item Four $S's$ come consecutively in the word
\item Two  $I's$ and two $N's$ come together
\item All $A's$ are not coming together
\item No two $A's$ are coming together
\end{enumerate}
%\input{exemplar/11/16/3/14/main.tex}
	\item One urn contains two black balls (labelled B1 and B2) and one white ball. A
	second urn contains one black ball and two white balls (labelled W1 and W2).
	Suppose the following experiment is performed. One of the two urns is chosen
	at random. Next a ball is randomly chosen from the urn. Then a second ball is
	chosen at random from the same urn without replacing the first ball.
	
	\begin{enumerate}
	\item What is the probability that two black balls are chosen?
	
	\item What is the probability that two balls of opposite colour are chosen?
	\end{enumerate}
	\solution
	%\input{exemplar/11/16/3/12/main1.tex}
\end{enumerate}

		%
\item 
Out of 100 students, two sections of 40 and 60 are formed. If you and your friend are among the 100 students, what is the probability that
\begin{enumerate}
\item you both enter the same section?
\item you both enter the different sections?
\end{enumerate}
\solution
		%\begin{enumerate}[label=\thesection.\arabic*,ref=\thesection.\theenumi]
	\item One card is drawn from a well-shuffled deck of 52 cards. Find the probability of getting
\begin{enumerate}
\item A king of red colour 
\item A face card 
\item A red face card
\item The jack of hearts
\item A spade
\item The queen of diamonds

\end{enumerate}
\solution
		%\input{ncert/10/15/1/14/main.tex}
	\item Five cards—the ten, jack, queen, king and ace of diamonds, are well-shuffled with their face downwards. One card is then picked up at random.
\begin{enumerate}
\item
What is the probability that the card is the queen? 
\item
If the queen is drawn and put aside, what is the probability that the second card picked up is (a) an ace? (b) a queen?\\
\end{enumerate}
\solution
		%\input{ncert/10/15/1/15/defs.tex}
	\item A bag contains $5$ red balls and some blue balls. If the probability of drawing a blue ball is double that if a red ball, determine the number of blue balls in the bag. 
		\\
\solution
		%\input{ncert/10/15/2/3/defs.tex}
	\item A card is selected from a pack of 52 cards.
 \begin{enumerate}[label=(\alph*)] 
                 \item How many points are there in the sample space?
                 \item Calculate the probability that the card is an ace of spades.
                 \item Calculate the probability that the card is (i) an ace and (ii) black card.
 \end{enumerate}
\solution
		%\input{ncert/11/16/3/4/main.tex}
\item Four cards are drawn from a well-shuffled deck of 52 cards. What is the probability of obtaining 3 diamonds and one spade.
\\
\solution
		%\input{ncert/11/16/4/2/defs.tex}
\item In a certain lottery 10,000 tickets are sold and ten equal prizes are awarded. What is the probability of not getting a prize if you buy (a) one ticket (b) two tickets (c) 10 tickets ?	
\\
\solution
		%\input{ncert/11/16/4/4/defs.tex}
		%
\item 
Out of 100 students, two sections of 40 and 60 are formed. If you and your friend are among the 100 students, what is the probability that
\begin{enumerate}
\item you both enter the same section?
\item you both enter the different sections?
\end{enumerate}
\solution
		%\input{ncert/11/16/4/5/defs.tex}
	\item 
The number lock of a suitcase has 4 wheels each labelled with ten digits i.e. from 0 to 9.The lock opens with a sequence of four digits with no repeats.What is the probability of a person getting the right sequence to open the suitcase.
\\
\solution
		%\input{ncert/11/16/4/10/defs.tex}
		%
\item 
Two cards are drawn at random and without replacement from a pack of 52 playing cards. Find the probability that both the cards are black.
\\
\solution
		%\input{ncert/12/13/2/2/defs.tex}
		\item A box of oranges is inspected by examining three randomly selected oranges drawn without replacement. If all the three oranges are good, the box is approved for sale, otherwise, it is rejected. Find the probability that a box containing 15 oranges out of which 12 are good and 3 are bad ones will be approved for sale.
		\label{ncert/12/13/2/3/defs.tex}
		\item Two balls are drawn at random with replacement from a box containing 10 black and 8 red balls. Find the probability that
		\label{ncert/12/13/2/12}
\begin{enumerate}
\item both balls are red.
\item first ball is black and second is red.
\item one of them is black and other is red.
\end{enumerate}

\item In a hostel, 60\% of the students read Hindi newspaper, 40\% read English newspaper and 20\% read both Hindi and English newspapers. A student is selected at random.
		\label{ncert/12/13/2/15}
\begin{enumerate}
\item Find the probability that she reads neither Hindi nor English newspapers.
\item If she reads Hindi newspaper, find the probability that she reads English newspaper.
\item If she reads English newspaper, find the probability that she reads Hindi newspaper.\\
\end{enumerate}
\item The probability of obtaining an even prime number on each die, when a pair of dice is rolled is 
\begin{enumerate}
    \item $0$ 
    
    \item $\frac{1}{3}$ 
    
    \item $\frac{1}{12}$ 
    
    \item $\frac{1}{36}$ 
\end{enumerate}
\solution
		%\input{ncert/12/13/2/17/defs.tex}
	\item A bag contains 4 red and 4 black balls, another bag contains 2 red and 6 black balls. One of the two bags is selected at random and a ball is drawn from the bag which is found to be red. Find the probability that the ball is drawn from the first bag.
\\
\solution
		%\input{ncert/12/13/3/2/main.tex}
  \item
  Cards with numbers 2 to 101 are placed in a box. A card is selected at random.Find the probability that the card has
\begin{enumerate}[label=(\roman*)]
	\item an even number 
	\item a square number
\end{enumerate}
\solution
%\input{exemplar/10/13/3/32/main.tex}
\item
The king, queen and jack of clubs are removed from a deck of 52 playing cards and then well shuffled. Now one card is drawn at random from the remaining cards.  Determine the probability that the card is
\begin{enumerate}[label=(\roman*)]
\item a club
\item 10 of hearts
\end{enumerate}
\solution
%\input{exemplar/10/13/3/29/main.tex}
\item A team of medical students doing their internship have to assist during surgeries
at a city hospital. The probabilities of surgeries rated as very complex, complex,
routine, simple or very simple are respectively, 0.15, 0.20, 0.31, 0.26, .08. Find
the probabilities that a particular surgery will be rated
\begin{enumerate}
	\item complex or very complex;
	\item neither very complex nor very simple;
	\item routine or complex
	\item routine or simple
\end{enumerate}
\solution
%\input{exemplar/11/16/3/8(1)/main.tex}
\item A card is selected from a pack of 52 cards.
\begin{enumerate}[label=(\alph*)]
    \item How many points are there in the sample space?
    \item Calculate the probability that the card is an ace of spades.
    \item Calculate the probability that the card is (i) an ace and (ii) black card.
\end{enumerate}
\solution
%\input{exemplar/11/16/3/4/main2.tex}
\item The probability that a non leap year selected at random will contain 53 sundays.
\\
\solution
%\input{exemplar/10/13/1/19/main.tex}
\item One of the four persons John, Rita, Aslam or Gurpreet will be promoted next
month. Consequently the sample space consists of four elementary outcomes
S = {John promoted, Rita promoted, Aslam promoted, Gurpreet promoted}
You are told that the chances of John’s promotion is same as that of Gurpreet,
Rita’s chances of promotion are twice as likely as Johns. Aslam’s chances are
four times that of John.
\begin{enumerate}
	\item Determine
	\begin{enumerate}
		\item P (John promoted)
		\item P (Rita promoted)
		\item P (Aslam promoted)
		\item P (Gurpreet promoted)
	\end{enumerate}
	\item If A = {John promoted or Gurpreet promoted}, find P (A).
\end{enumerate}
\solution
%\input{exemplar/11/16/3/10/main.tex}
\item A card is drawn from a deck of 52 cards. Find the probability of getting a king or a heart or a red card.\\
\solution
%\input{exemplar/11/16/3/15/main.tex}
\item The probability that a student will pass his examination is 0.73, the probability of
the student getting a compartment is 0.13, and the probability that the student will
either pass or get compartment is 0.96. State True or False.\\
\solution
%\input{exemplar/11/16/3/31/main.tex}
\item A card is selected from a pack of 52 cards\\
\begin{enumerate}[label=(\alph*)]
\item How many points are there in the sample space?
\item Calculate the probability that the cards is an ace of spades.
\item Calculate the probability that the card is (i) an ace (ii)black card.\\
\end{enumerate}
%\input{ncert/11/16/3/4_1/Prob_4.tex}
\item In a non-leap year, the probability of having 53 tuesdays or 53 wednesdays is\\
\solution
%\input{exemplar/11/16/3/18/main.tex}
\item There are 1000 sealed envelopes in a box, 10 of them contain a cash prize of
Rs 100 each, 100 of them contain a cash prize of Rs 50 each and 200 of them
contain a cash prize of Rs 10 each and rest do not contain any cash prize. If they
are well shuffled and an envelope is picked up out, what is the probability that it
contains no cash prize?\\
\solution
%\input{exemplar/10/13/3/34/main.tex}
\item 
A die is thrown and a card is selected at random from a deck of 52 playing cards. The probability of getting an even number on the die and a spade card.\\
\solution
%\input{exemplar/12/13/3/78/main.tex}
\item
If 4-digit numbers greater than 5,000 are randomly formed from the digits 0, 1, 3, 5, and 7, what is the probability of forming a number divisible by 5 when:
\begin{enumerate}
    \item The digits are repeated?
    \item The repetition of digits is not allowed?
\end{enumerate}
\solution
%\input{ncert/11/16/4/9/main.tex}
\item Consider the probability space $\brak{\Omega, \mathcal{G}, P}$ where $\Omega = [0,2]$ and $\mathcal{G} = \cbrak{\phi, \Omega, [0,1], (1,2]}$. Let $X$ and $Y$ be two functions on $\Omega$ defined as
\begin{align*}
    X(\omega) = 
    \begin{cases}
        1 & \text{if }\omega \in [0, 1]\\
        2 & \text{if }\omega \in (1, 2]
    \end{cases}
\end{align*}
and
\begin{align*}
    Y(\omega) = 
    \begin{cases}
        2 & \text{if }\omega \in [0, 1.5]\\
        3 & \text{if }\omega \in (1.5, 2].
    \end{cases}
\end{align*}
Then which one of the following statements is true?
\begin{enumerate}
    \item [(A)] $X$ is a random variable with respect to $\mathcal{G}$, but $Y$ is not a random variable with respect to $\mathcal{G}$.
    \item [(B)] $Y$ is a random variable with respect to $\mathcal{G}$, but $X$ is not a random variable with respect to $\mathcal{G}$.
    \item [(C)] Neither $X$ nor $Y$ is a random variable with respect to $\mathcal{G}$.
    \item [(D)] Both $X$ and $Y$ are random variables with respect to $\mathcal{G}$.
\end{enumerate} \hfill (GATE ST 2023)\\
\solution
%\input{gate/ST/2023/14/main.tex}
	\item  A die is loaded in such a way that each odd number is twice as likely to occur as
each even number. Find $P(G)$, where $G$ is the event that a number greater than
3 occurs on a single roll of the die.
\\
\solution
		%\input{exemplar/11/16/3/5/main.tex}
	\item All the jacks, queens and kings are removed from a deck of 52 playing cards. The remaining cards are well shuffled and then one card is drawn at random. Giving ace a value 1 similar value for other cards, find the probability that the card has a value 
		\begin{enumerate}
			\item 7
			\item greater than 7
			\item less than 7
		\end{enumerate}
		%\input{exemplar/10/13/3/30/main.tex}
  \item A Lot consists of 48 mobile phones of which 42 are good, 3 have only minor defects and 3 have major defects.Varnika will buy a phone if it is good but the trader will only buy a mobile if it has no major defects. One phone is selected at random from the lot. What is the probability that it is
\begin{enumerate}
	\item acceptable to Varnika?
            \item acceptable to the trader?
\end{enumerate}
\solution
	%\input{exemplar/10/13/3/40/main.tex}
 \item A student says that if you throw a die, it will show up 1 or not 1. Therefore, the probability of getting 1 and the probability of getting 'not 1' each is equal to $\frac{1}{2}$. Is this correct? Give reasons.\\
 \solution
        %\input{exemplar/10/13/2/9/main.tex}
   \item Four candidates A, B, C, D have ap-
plied for the assignment to coach a school cricket
team. If A is twice as likely to be selected as B, and
B and C are given about the same chance of being
selected, while C is twice as likely to be selected
as D, what are the probabilities that
\begin{enumerate}
\item C will be selected?
\item A will not be selected?
\end{enumerate}
	%\input{exemplar/11/16/3/9/main.tex}
 \item A bag contain 24 balls of which $x$ balls are red, $2x$ are white and $3x$ are blue. A ball is selected at random, What is the probability that it is
\begin{enumerate}[label=\alph*)]
\item not red ?
\item white ?
\end{enumerate}
%\input{exemplar/10/13/3/41/main.tex}
If the letters of the word ASSASSINATION are arranged at random. Find the Probability that
\begin{enumerate}[label=(\alph*)]
\item Four $S's$ come consecutively in the word
\item Two  $I's$ and two $N's$ come together
\item All $A's$ are not coming together
\item No two $A's$ are coming together
\end{enumerate}
%\input{exemplar/11/16/3/14/main.tex}
	\item One urn contains two black balls (labelled B1 and B2) and one white ball. A
	second urn contains one black ball and two white balls (labelled W1 and W2).
	Suppose the following experiment is performed. One of the two urns is chosen
	at random. Next a ball is randomly chosen from the urn. Then a second ball is
	chosen at random from the same urn without replacing the first ball.
	
	\begin{enumerate}
	\item What is the probability that two black balls are chosen?
	
	\item What is the probability that two balls of opposite colour are chosen?
	\end{enumerate}
	\solution
	%\input{exemplar/11/16/3/12/main1.tex}
\end{enumerate}

	\item 
The number lock of a suitcase has 4 wheels each labelled with ten digits i.e. from 0 to 9.The lock opens with a sequence of four digits with no repeats.What is the probability of a person getting the right sequence to open the suitcase.
\\
\solution
		%\begin{enumerate}[label=\thesection.\arabic*,ref=\thesection.\theenumi]
	\item One card is drawn from a well-shuffled deck of 52 cards. Find the probability of getting
\begin{enumerate}
\item A king of red colour 
\item A face card 
\item A red face card
\item The jack of hearts
\item A spade
\item The queen of diamonds

\end{enumerate}
\solution
		%\input{ncert/10/15/1/14/main.tex}
	\item Five cards—the ten, jack, queen, king and ace of diamonds, are well-shuffled with their face downwards. One card is then picked up at random.
\begin{enumerate}
\item
What is the probability that the card is the queen? 
\item
If the queen is drawn and put aside, what is the probability that the second card picked up is (a) an ace? (b) a queen?\\
\end{enumerate}
\solution
		%\input{ncert/10/15/1/15/defs.tex}
	\item A bag contains $5$ red balls and some blue balls. If the probability of drawing a blue ball is double that if a red ball, determine the number of blue balls in the bag. 
		\\
\solution
		%\input{ncert/10/15/2/3/defs.tex}
	\item A card is selected from a pack of 52 cards.
 \begin{enumerate}[label=(\alph*)] 
                 \item How many points are there in the sample space?
                 \item Calculate the probability that the card is an ace of spades.
                 \item Calculate the probability that the card is (i) an ace and (ii) black card.
 \end{enumerate}
\solution
		%\input{ncert/11/16/3/4/main.tex}
\item Four cards are drawn from a well-shuffled deck of 52 cards. What is the probability of obtaining 3 diamonds and one spade.
\\
\solution
		%\input{ncert/11/16/4/2/defs.tex}
\item In a certain lottery 10,000 tickets are sold and ten equal prizes are awarded. What is the probability of not getting a prize if you buy (a) one ticket (b) two tickets (c) 10 tickets ?	
\\
\solution
		%\input{ncert/11/16/4/4/defs.tex}
		%
\item 
Out of 100 students, two sections of 40 and 60 are formed. If you and your friend are among the 100 students, what is the probability that
\begin{enumerate}
\item you both enter the same section?
\item you both enter the different sections?
\end{enumerate}
\solution
		%\input{ncert/11/16/4/5/defs.tex}
	\item 
The number lock of a suitcase has 4 wheels each labelled with ten digits i.e. from 0 to 9.The lock opens with a sequence of four digits with no repeats.What is the probability of a person getting the right sequence to open the suitcase.
\\
\solution
		%\input{ncert/11/16/4/10/defs.tex}
		%
\item 
Two cards are drawn at random and without replacement from a pack of 52 playing cards. Find the probability that both the cards are black.
\\
\solution
		%\input{ncert/12/13/2/2/defs.tex}
		\item A box of oranges is inspected by examining three randomly selected oranges drawn without replacement. If all the three oranges are good, the box is approved for sale, otherwise, it is rejected. Find the probability that a box containing 15 oranges out of which 12 are good and 3 are bad ones will be approved for sale.
		\label{ncert/12/13/2/3/defs.tex}
		\item Two balls are drawn at random with replacement from a box containing 10 black and 8 red balls. Find the probability that
		\label{ncert/12/13/2/12}
\begin{enumerate}
\item both balls are red.
\item first ball is black and second is red.
\item one of them is black and other is red.
\end{enumerate}

\item In a hostel, 60\% of the students read Hindi newspaper, 40\% read English newspaper and 20\% read both Hindi and English newspapers. A student is selected at random.
		\label{ncert/12/13/2/15}
\begin{enumerate}
\item Find the probability that she reads neither Hindi nor English newspapers.
\item If she reads Hindi newspaper, find the probability that she reads English newspaper.
\item If she reads English newspaper, find the probability that she reads Hindi newspaper.\\
\end{enumerate}
\item The probability of obtaining an even prime number on each die, when a pair of dice is rolled is 
\begin{enumerate}
    \item $0$ 
    
    \item $\frac{1}{3}$ 
    
    \item $\frac{1}{12}$ 
    
    \item $\frac{1}{36}$ 
\end{enumerate}
\solution
		%\input{ncert/12/13/2/17/defs.tex}
	\item A bag contains 4 red and 4 black balls, another bag contains 2 red and 6 black balls. One of the two bags is selected at random and a ball is drawn from the bag which is found to be red. Find the probability that the ball is drawn from the first bag.
\\
\solution
		%\input{ncert/12/13/3/2/main.tex}
  \item
  Cards with numbers 2 to 101 are placed in a box. A card is selected at random.Find the probability that the card has
\begin{enumerate}[label=(\roman*)]
	\item an even number 
	\item a square number
\end{enumerate}
\solution
%\input{exemplar/10/13/3/32/main.tex}
\item
The king, queen and jack of clubs are removed from a deck of 52 playing cards and then well shuffled. Now one card is drawn at random from the remaining cards.  Determine the probability that the card is
\begin{enumerate}[label=(\roman*)]
\item a club
\item 10 of hearts
\end{enumerate}
\solution
%\input{exemplar/10/13/3/29/main.tex}
\item A team of medical students doing their internship have to assist during surgeries
at a city hospital. The probabilities of surgeries rated as very complex, complex,
routine, simple or very simple are respectively, 0.15, 0.20, 0.31, 0.26, .08. Find
the probabilities that a particular surgery will be rated
\begin{enumerate}
	\item complex or very complex;
	\item neither very complex nor very simple;
	\item routine or complex
	\item routine or simple
\end{enumerate}
\solution
%\input{exemplar/11/16/3/8(1)/main.tex}
\item A card is selected from a pack of 52 cards.
\begin{enumerate}[label=(\alph*)]
    \item How many points are there in the sample space?
    \item Calculate the probability that the card is an ace of spades.
    \item Calculate the probability that the card is (i) an ace and (ii) black card.
\end{enumerate}
\solution
%\input{exemplar/11/16/3/4/main2.tex}
\item The probability that a non leap year selected at random will contain 53 sundays.
\\
\solution
%\input{exemplar/10/13/1/19/main.tex}
\item One of the four persons John, Rita, Aslam or Gurpreet will be promoted next
month. Consequently the sample space consists of four elementary outcomes
S = {John promoted, Rita promoted, Aslam promoted, Gurpreet promoted}
You are told that the chances of John’s promotion is same as that of Gurpreet,
Rita’s chances of promotion are twice as likely as Johns. Aslam’s chances are
four times that of John.
\begin{enumerate}
	\item Determine
	\begin{enumerate}
		\item P (John promoted)
		\item P (Rita promoted)
		\item P (Aslam promoted)
		\item P (Gurpreet promoted)
	\end{enumerate}
	\item If A = {John promoted or Gurpreet promoted}, find P (A).
\end{enumerate}
\solution
%\input{exemplar/11/16/3/10/main.tex}
\item A card is drawn from a deck of 52 cards. Find the probability of getting a king or a heart or a red card.\\
\solution
%\input{exemplar/11/16/3/15/main.tex}
\item The probability that a student will pass his examination is 0.73, the probability of
the student getting a compartment is 0.13, and the probability that the student will
either pass or get compartment is 0.96. State True or False.\\
\solution
%\input{exemplar/11/16/3/31/main.tex}
\item A card is selected from a pack of 52 cards\\
\begin{enumerate}[label=(\alph*)]
\item How many points are there in the sample space?
\item Calculate the probability that the cards is an ace of spades.
\item Calculate the probability that the card is (i) an ace (ii)black card.\\
\end{enumerate}
%\input{ncert/11/16/3/4_1/Prob_4.tex}
\item In a non-leap year, the probability of having 53 tuesdays or 53 wednesdays is\\
\solution
%\input{exemplar/11/16/3/18/main.tex}
\item There are 1000 sealed envelopes in a box, 10 of them contain a cash prize of
Rs 100 each, 100 of them contain a cash prize of Rs 50 each and 200 of them
contain a cash prize of Rs 10 each and rest do not contain any cash prize. If they
are well shuffled and an envelope is picked up out, what is the probability that it
contains no cash prize?\\
\solution
%\input{exemplar/10/13/3/34/main.tex}
\item 
A die is thrown and a card is selected at random from a deck of 52 playing cards. The probability of getting an even number on the die and a spade card.\\
\solution
%\input{exemplar/12/13/3/78/main.tex}
\item
If 4-digit numbers greater than 5,000 are randomly formed from the digits 0, 1, 3, 5, and 7, what is the probability of forming a number divisible by 5 when:
\begin{enumerate}
    \item The digits are repeated?
    \item The repetition of digits is not allowed?
\end{enumerate}
\solution
%\input{ncert/11/16/4/9/main.tex}
\item Consider the probability space $\brak{\Omega, \mathcal{G}, P}$ where $\Omega = [0,2]$ and $\mathcal{G} = \cbrak{\phi, \Omega, [0,1], (1,2]}$. Let $X$ and $Y$ be two functions on $\Omega$ defined as
\begin{align*}
    X(\omega) = 
    \begin{cases}
        1 & \text{if }\omega \in [0, 1]\\
        2 & \text{if }\omega \in (1, 2]
    \end{cases}
\end{align*}
and
\begin{align*}
    Y(\omega) = 
    \begin{cases}
        2 & \text{if }\omega \in [0, 1.5]\\
        3 & \text{if }\omega \in (1.5, 2].
    \end{cases}
\end{align*}
Then which one of the following statements is true?
\begin{enumerate}
    \item [(A)] $X$ is a random variable with respect to $\mathcal{G}$, but $Y$ is not a random variable with respect to $\mathcal{G}$.
    \item [(B)] $Y$ is a random variable with respect to $\mathcal{G}$, but $X$ is not a random variable with respect to $\mathcal{G}$.
    \item [(C)] Neither $X$ nor $Y$ is a random variable with respect to $\mathcal{G}$.
    \item [(D)] Both $X$ and $Y$ are random variables with respect to $\mathcal{G}$.
\end{enumerate} \hfill (GATE ST 2023)\\
\solution
%\input{gate/ST/2023/14/main.tex}
	\item  A die is loaded in such a way that each odd number is twice as likely to occur as
each even number. Find $P(G)$, where $G$ is the event that a number greater than
3 occurs on a single roll of the die.
\\
\solution
		%\input{exemplar/11/16/3/5/main.tex}
	\item All the jacks, queens and kings are removed from a deck of 52 playing cards. The remaining cards are well shuffled and then one card is drawn at random. Giving ace a value 1 similar value for other cards, find the probability that the card has a value 
		\begin{enumerate}
			\item 7
			\item greater than 7
			\item less than 7
		\end{enumerate}
		%\input{exemplar/10/13/3/30/main.tex}
  \item A Lot consists of 48 mobile phones of which 42 are good, 3 have only minor defects and 3 have major defects.Varnika will buy a phone if it is good but the trader will only buy a mobile if it has no major defects. One phone is selected at random from the lot. What is the probability that it is
\begin{enumerate}
	\item acceptable to Varnika?
            \item acceptable to the trader?
\end{enumerate}
\solution
	%\input{exemplar/10/13/3/40/main.tex}
 \item A student says that if you throw a die, it will show up 1 or not 1. Therefore, the probability of getting 1 and the probability of getting 'not 1' each is equal to $\frac{1}{2}$. Is this correct? Give reasons.\\
 \solution
        %\input{exemplar/10/13/2/9/main.tex}
   \item Four candidates A, B, C, D have ap-
plied for the assignment to coach a school cricket
team. If A is twice as likely to be selected as B, and
B and C are given about the same chance of being
selected, while C is twice as likely to be selected
as D, what are the probabilities that
\begin{enumerate}
\item C will be selected?
\item A will not be selected?
\end{enumerate}
	%\input{exemplar/11/16/3/9/main.tex}
 \item A bag contain 24 balls of which $x$ balls are red, $2x$ are white and $3x$ are blue. A ball is selected at random, What is the probability that it is
\begin{enumerate}[label=\alph*)]
\item not red ?
\item white ?
\end{enumerate}
%\input{exemplar/10/13/3/41/main.tex}
If the letters of the word ASSASSINATION are arranged at random. Find the Probability that
\begin{enumerate}[label=(\alph*)]
\item Four $S's$ come consecutively in the word
\item Two  $I's$ and two $N's$ come together
\item All $A's$ are not coming together
\item No two $A's$ are coming together
\end{enumerate}
%\input{exemplar/11/16/3/14/main.tex}
	\item One urn contains two black balls (labelled B1 and B2) and one white ball. A
	second urn contains one black ball and two white balls (labelled W1 and W2).
	Suppose the following experiment is performed. One of the two urns is chosen
	at random. Next a ball is randomly chosen from the urn. Then a second ball is
	chosen at random from the same urn without replacing the first ball.
	
	\begin{enumerate}
	\item What is the probability that two black balls are chosen?
	
	\item What is the probability that two balls of opposite colour are chosen?
	\end{enumerate}
	\solution
	%\input{exemplar/11/16/3/12/main1.tex}
\end{enumerate}

		%
\item 
Two cards are drawn at random and without replacement from a pack of 52 playing cards. Find the probability that both the cards are black.
\\
\solution
		%\begin{enumerate}[label=\thesection.\arabic*,ref=\thesection.\theenumi]
	\item One card is drawn from a well-shuffled deck of 52 cards. Find the probability of getting
\begin{enumerate}
\item A king of red colour 
\item A face card 
\item A red face card
\item The jack of hearts
\item A spade
\item The queen of diamonds

\end{enumerate}
\solution
		%\input{ncert/10/15/1/14/main.tex}
	\item Five cards—the ten, jack, queen, king and ace of diamonds, are well-shuffled with their face downwards. One card is then picked up at random.
\begin{enumerate}
\item
What is the probability that the card is the queen? 
\item
If the queen is drawn and put aside, what is the probability that the second card picked up is (a) an ace? (b) a queen?\\
\end{enumerate}
\solution
		%\input{ncert/10/15/1/15/defs.tex}
	\item A bag contains $5$ red balls and some blue balls. If the probability of drawing a blue ball is double that if a red ball, determine the number of blue balls in the bag. 
		\\
\solution
		%\input{ncert/10/15/2/3/defs.tex}
	\item A card is selected from a pack of 52 cards.
 \begin{enumerate}[label=(\alph*)] 
                 \item How many points are there in the sample space?
                 \item Calculate the probability that the card is an ace of spades.
                 \item Calculate the probability that the card is (i) an ace and (ii) black card.
 \end{enumerate}
\solution
		%\input{ncert/11/16/3/4/main.tex}
\item Four cards are drawn from a well-shuffled deck of 52 cards. What is the probability of obtaining 3 diamonds and one spade.
\\
\solution
		%\input{ncert/11/16/4/2/defs.tex}
\item In a certain lottery 10,000 tickets are sold and ten equal prizes are awarded. What is the probability of not getting a prize if you buy (a) one ticket (b) two tickets (c) 10 tickets ?	
\\
\solution
		%\input{ncert/11/16/4/4/defs.tex}
		%
\item 
Out of 100 students, two sections of 40 and 60 are formed. If you and your friend are among the 100 students, what is the probability that
\begin{enumerate}
\item you both enter the same section?
\item you both enter the different sections?
\end{enumerate}
\solution
		%\input{ncert/11/16/4/5/defs.tex}
	\item 
The number lock of a suitcase has 4 wheels each labelled with ten digits i.e. from 0 to 9.The lock opens with a sequence of four digits with no repeats.What is the probability of a person getting the right sequence to open the suitcase.
\\
\solution
		%\input{ncert/11/16/4/10/defs.tex}
		%
\item 
Two cards are drawn at random and without replacement from a pack of 52 playing cards. Find the probability that both the cards are black.
\\
\solution
		%\input{ncert/12/13/2/2/defs.tex}
		\item A box of oranges is inspected by examining three randomly selected oranges drawn without replacement. If all the three oranges are good, the box is approved for sale, otherwise, it is rejected. Find the probability that a box containing 15 oranges out of which 12 are good and 3 are bad ones will be approved for sale.
		\label{ncert/12/13/2/3/defs.tex}
		\item Two balls are drawn at random with replacement from a box containing 10 black and 8 red balls. Find the probability that
		\label{ncert/12/13/2/12}
\begin{enumerate}
\item both balls are red.
\item first ball is black and second is red.
\item one of them is black and other is red.
\end{enumerate}

\item In a hostel, 60\% of the students read Hindi newspaper, 40\% read English newspaper and 20\% read both Hindi and English newspapers. A student is selected at random.
		\label{ncert/12/13/2/15}
\begin{enumerate}
\item Find the probability that she reads neither Hindi nor English newspapers.
\item If she reads Hindi newspaper, find the probability that she reads English newspaper.
\item If she reads English newspaper, find the probability that she reads Hindi newspaper.\\
\end{enumerate}
\item The probability of obtaining an even prime number on each die, when a pair of dice is rolled is 
\begin{enumerate}
    \item $0$ 
    
    \item $\frac{1}{3}$ 
    
    \item $\frac{1}{12}$ 
    
    \item $\frac{1}{36}$ 
\end{enumerate}
\solution
		%\input{ncert/12/13/2/17/defs.tex}
	\item A bag contains 4 red and 4 black balls, another bag contains 2 red and 6 black balls. One of the two bags is selected at random and a ball is drawn from the bag which is found to be red. Find the probability that the ball is drawn from the first bag.
\\
\solution
		%\input{ncert/12/13/3/2/main.tex}
  \item
  Cards with numbers 2 to 101 are placed in a box. A card is selected at random.Find the probability that the card has
\begin{enumerate}[label=(\roman*)]
	\item an even number 
	\item a square number
\end{enumerate}
\solution
%\input{exemplar/10/13/3/32/main.tex}
\item
The king, queen and jack of clubs are removed from a deck of 52 playing cards and then well shuffled. Now one card is drawn at random from the remaining cards.  Determine the probability that the card is
\begin{enumerate}[label=(\roman*)]
\item a club
\item 10 of hearts
\end{enumerate}
\solution
%\input{exemplar/10/13/3/29/main.tex}
\item A team of medical students doing their internship have to assist during surgeries
at a city hospital. The probabilities of surgeries rated as very complex, complex,
routine, simple or very simple are respectively, 0.15, 0.20, 0.31, 0.26, .08. Find
the probabilities that a particular surgery will be rated
\begin{enumerate}
	\item complex or very complex;
	\item neither very complex nor very simple;
	\item routine or complex
	\item routine or simple
\end{enumerate}
\solution
%\input{exemplar/11/16/3/8(1)/main.tex}
\item A card is selected from a pack of 52 cards.
\begin{enumerate}[label=(\alph*)]
    \item How many points are there in the sample space?
    \item Calculate the probability that the card is an ace of spades.
    \item Calculate the probability that the card is (i) an ace and (ii) black card.
\end{enumerate}
\solution
%\input{exemplar/11/16/3/4/main2.tex}
\item The probability that a non leap year selected at random will contain 53 sundays.
\\
\solution
%\input{exemplar/10/13/1/19/main.tex}
\item One of the four persons John, Rita, Aslam or Gurpreet will be promoted next
month. Consequently the sample space consists of four elementary outcomes
S = {John promoted, Rita promoted, Aslam promoted, Gurpreet promoted}
You are told that the chances of John’s promotion is same as that of Gurpreet,
Rita’s chances of promotion are twice as likely as Johns. Aslam’s chances are
four times that of John.
\begin{enumerate}
	\item Determine
	\begin{enumerate}
		\item P (John promoted)
		\item P (Rita promoted)
		\item P (Aslam promoted)
		\item P (Gurpreet promoted)
	\end{enumerate}
	\item If A = {John promoted or Gurpreet promoted}, find P (A).
\end{enumerate}
\solution
%\input{exemplar/11/16/3/10/main.tex}
\item A card is drawn from a deck of 52 cards. Find the probability of getting a king or a heart or a red card.\\
\solution
%\input{exemplar/11/16/3/15/main.tex}
\item The probability that a student will pass his examination is 0.73, the probability of
the student getting a compartment is 0.13, and the probability that the student will
either pass or get compartment is 0.96. State True or False.\\
\solution
%\input{exemplar/11/16/3/31/main.tex}
\item A card is selected from a pack of 52 cards\\
\begin{enumerate}[label=(\alph*)]
\item How many points are there in the sample space?
\item Calculate the probability that the cards is an ace of spades.
\item Calculate the probability that the card is (i) an ace (ii)black card.\\
\end{enumerate}
%\input{ncert/11/16/3/4_1/Prob_4.tex}
\item In a non-leap year, the probability of having 53 tuesdays or 53 wednesdays is\\
\solution
%\input{exemplar/11/16/3/18/main.tex}
\item There are 1000 sealed envelopes in a box, 10 of them contain a cash prize of
Rs 100 each, 100 of them contain a cash prize of Rs 50 each and 200 of them
contain a cash prize of Rs 10 each and rest do not contain any cash prize. If they
are well shuffled and an envelope is picked up out, what is the probability that it
contains no cash prize?\\
\solution
%\input{exemplar/10/13/3/34/main.tex}
\item 
A die is thrown and a card is selected at random from a deck of 52 playing cards. The probability of getting an even number on the die and a spade card.\\
\solution
%\input{exemplar/12/13/3/78/main.tex}
\item
If 4-digit numbers greater than 5,000 are randomly formed from the digits 0, 1, 3, 5, and 7, what is the probability of forming a number divisible by 5 when:
\begin{enumerate}
    \item The digits are repeated?
    \item The repetition of digits is not allowed?
\end{enumerate}
\solution
%\input{ncert/11/16/4/9/main.tex}
\item Consider the probability space $\brak{\Omega, \mathcal{G}, P}$ where $\Omega = [0,2]$ and $\mathcal{G} = \cbrak{\phi, \Omega, [0,1], (1,2]}$. Let $X$ and $Y$ be two functions on $\Omega$ defined as
\begin{align*}
    X(\omega) = 
    \begin{cases}
        1 & \text{if }\omega \in [0, 1]\\
        2 & \text{if }\omega \in (1, 2]
    \end{cases}
\end{align*}
and
\begin{align*}
    Y(\omega) = 
    \begin{cases}
        2 & \text{if }\omega \in [0, 1.5]\\
        3 & \text{if }\omega \in (1.5, 2].
    \end{cases}
\end{align*}
Then which one of the following statements is true?
\begin{enumerate}
    \item [(A)] $X$ is a random variable with respect to $\mathcal{G}$, but $Y$ is not a random variable with respect to $\mathcal{G}$.
    \item [(B)] $Y$ is a random variable with respect to $\mathcal{G}$, but $X$ is not a random variable with respect to $\mathcal{G}$.
    \item [(C)] Neither $X$ nor $Y$ is a random variable with respect to $\mathcal{G}$.
    \item [(D)] Both $X$ and $Y$ are random variables with respect to $\mathcal{G}$.
\end{enumerate} \hfill (GATE ST 2023)\\
\solution
%\input{gate/ST/2023/14/main.tex}
	\item  A die is loaded in such a way that each odd number is twice as likely to occur as
each even number. Find $P(G)$, where $G$ is the event that a number greater than
3 occurs on a single roll of the die.
\\
\solution
		%\input{exemplar/11/16/3/5/main.tex}
	\item All the jacks, queens and kings are removed from a deck of 52 playing cards. The remaining cards are well shuffled and then one card is drawn at random. Giving ace a value 1 similar value for other cards, find the probability that the card has a value 
		\begin{enumerate}
			\item 7
			\item greater than 7
			\item less than 7
		\end{enumerate}
		%\input{exemplar/10/13/3/30/main.tex}
  \item A Lot consists of 48 mobile phones of which 42 are good, 3 have only minor defects and 3 have major defects.Varnika will buy a phone if it is good but the trader will only buy a mobile if it has no major defects. One phone is selected at random from the lot. What is the probability that it is
\begin{enumerate}
	\item acceptable to Varnika?
            \item acceptable to the trader?
\end{enumerate}
\solution
	%\input{exemplar/10/13/3/40/main.tex}
 \item A student says that if you throw a die, it will show up 1 or not 1. Therefore, the probability of getting 1 and the probability of getting 'not 1' each is equal to $\frac{1}{2}$. Is this correct? Give reasons.\\
 \solution
        %\input{exemplar/10/13/2/9/main.tex}
   \item Four candidates A, B, C, D have ap-
plied for the assignment to coach a school cricket
team. If A is twice as likely to be selected as B, and
B and C are given about the same chance of being
selected, while C is twice as likely to be selected
as D, what are the probabilities that
\begin{enumerate}
\item C will be selected?
\item A will not be selected?
\end{enumerate}
	%\input{exemplar/11/16/3/9/main.tex}
 \item A bag contain 24 balls of which $x$ balls are red, $2x$ are white and $3x$ are blue. A ball is selected at random, What is the probability that it is
\begin{enumerate}[label=\alph*)]
\item not red ?
\item white ?
\end{enumerate}
%\input{exemplar/10/13/3/41/main.tex}
If the letters of the word ASSASSINATION are arranged at random. Find the Probability that
\begin{enumerate}[label=(\alph*)]
\item Four $S's$ come consecutively in the word
\item Two  $I's$ and two $N's$ come together
\item All $A's$ are not coming together
\item No two $A's$ are coming together
\end{enumerate}
%\input{exemplar/11/16/3/14/main.tex}
	\item One urn contains two black balls (labelled B1 and B2) and one white ball. A
	second urn contains one black ball and two white balls (labelled W1 and W2).
	Suppose the following experiment is performed. One of the two urns is chosen
	at random. Next a ball is randomly chosen from the urn. Then a second ball is
	chosen at random from the same urn without replacing the first ball.
	
	\begin{enumerate}
	\item What is the probability that two black balls are chosen?
	
	\item What is the probability that two balls of opposite colour are chosen?
	\end{enumerate}
	\solution
	%\input{exemplar/11/16/3/12/main1.tex}
\end{enumerate}

		\item A box of oranges is inspected by examining three randomly selected oranges drawn without replacement. If all the three oranges are good, the box is approved for sale, otherwise, it is rejected. Find the probability that a box containing 15 oranges out of which 12 are good and 3 are bad ones will be approved for sale.
		\label{ncert/12/13/2/3/defs.tex}
		\item Two balls are drawn at random with replacement from a box containing 10 black and 8 red balls. Find the probability that
		\label{ncert/12/13/2/12}
\begin{enumerate}
\item both balls are red.
\item first ball is black and second is red.
\item one of them is black and other is red.
\end{enumerate}

\item In a hostel, 60\% of the students read Hindi newspaper, 40\% read English newspaper and 20\% read both Hindi and English newspapers. A student is selected at random.
		\label{ncert/12/13/2/15}
\begin{enumerate}
\item Find the probability that she reads neither Hindi nor English newspapers.
\item If she reads Hindi newspaper, find the probability that she reads English newspaper.
\item If she reads English newspaper, find the probability that she reads Hindi newspaper.\\
\end{enumerate}
\item The probability of obtaining an even prime number on each die, when a pair of dice is rolled is 
\begin{enumerate}
    \item $0$ 
    
    \item $\frac{1}{3}$ 
    
    \item $\frac{1}{12}$ 
    
    \item $\frac{1}{36}$ 
\end{enumerate}
\solution
		%\begin{enumerate}[label=\thesection.\arabic*,ref=\thesection.\theenumi]
	\item One card is drawn from a well-shuffled deck of 52 cards. Find the probability of getting
\begin{enumerate}
\item A king of red colour 
\item A face card 
\item A red face card
\item The jack of hearts
\item A spade
\item The queen of diamonds

\end{enumerate}
\solution
		%\input{ncert/10/15/1/14/main.tex}
	\item Five cards—the ten, jack, queen, king and ace of diamonds, are well-shuffled with their face downwards. One card is then picked up at random.
\begin{enumerate}
\item
What is the probability that the card is the queen? 
\item
If the queen is drawn and put aside, what is the probability that the second card picked up is (a) an ace? (b) a queen?\\
\end{enumerate}
\solution
		%\input{ncert/10/15/1/15/defs.tex}
	\item A bag contains $5$ red balls and some blue balls. If the probability of drawing a blue ball is double that if a red ball, determine the number of blue balls in the bag. 
		\\
\solution
		%\input{ncert/10/15/2/3/defs.tex}
	\item A card is selected from a pack of 52 cards.
 \begin{enumerate}[label=(\alph*)] 
                 \item How many points are there in the sample space?
                 \item Calculate the probability that the card is an ace of spades.
                 \item Calculate the probability that the card is (i) an ace and (ii) black card.
 \end{enumerate}
\solution
		%\input{ncert/11/16/3/4/main.tex}
\item Four cards are drawn from a well-shuffled deck of 52 cards. What is the probability of obtaining 3 diamonds and one spade.
\\
\solution
		%\input{ncert/11/16/4/2/defs.tex}
\item In a certain lottery 10,000 tickets are sold and ten equal prizes are awarded. What is the probability of not getting a prize if you buy (a) one ticket (b) two tickets (c) 10 tickets ?	
\\
\solution
		%\input{ncert/11/16/4/4/defs.tex}
		%
\item 
Out of 100 students, two sections of 40 and 60 are formed. If you and your friend are among the 100 students, what is the probability that
\begin{enumerate}
\item you both enter the same section?
\item you both enter the different sections?
\end{enumerate}
\solution
		%\input{ncert/11/16/4/5/defs.tex}
	\item 
The number lock of a suitcase has 4 wheels each labelled with ten digits i.e. from 0 to 9.The lock opens with a sequence of four digits with no repeats.What is the probability of a person getting the right sequence to open the suitcase.
\\
\solution
		%\input{ncert/11/16/4/10/defs.tex}
		%
\item 
Two cards are drawn at random and without replacement from a pack of 52 playing cards. Find the probability that both the cards are black.
\\
\solution
		%\input{ncert/12/13/2/2/defs.tex}
		\item A box of oranges is inspected by examining three randomly selected oranges drawn without replacement. If all the three oranges are good, the box is approved for sale, otherwise, it is rejected. Find the probability that a box containing 15 oranges out of which 12 are good and 3 are bad ones will be approved for sale.
		\label{ncert/12/13/2/3/defs.tex}
		\item Two balls are drawn at random with replacement from a box containing 10 black and 8 red balls. Find the probability that
		\label{ncert/12/13/2/12}
\begin{enumerate}
\item both balls are red.
\item first ball is black and second is red.
\item one of them is black and other is red.
\end{enumerate}

\item In a hostel, 60\% of the students read Hindi newspaper, 40\% read English newspaper and 20\% read both Hindi and English newspapers. A student is selected at random.
		\label{ncert/12/13/2/15}
\begin{enumerate}
\item Find the probability that she reads neither Hindi nor English newspapers.
\item If she reads Hindi newspaper, find the probability that she reads English newspaper.
\item If she reads English newspaper, find the probability that she reads Hindi newspaper.\\
\end{enumerate}
\item The probability of obtaining an even prime number on each die, when a pair of dice is rolled is 
\begin{enumerate}
    \item $0$ 
    
    \item $\frac{1}{3}$ 
    
    \item $\frac{1}{12}$ 
    
    \item $\frac{1}{36}$ 
\end{enumerate}
\solution
		%\input{ncert/12/13/2/17/defs.tex}
	\item A bag contains 4 red and 4 black balls, another bag contains 2 red and 6 black balls. One of the two bags is selected at random and a ball is drawn from the bag which is found to be red. Find the probability that the ball is drawn from the first bag.
\\
\solution
		%\input{ncert/12/13/3/2/main.tex}
  \item
  Cards with numbers 2 to 101 are placed in a box. A card is selected at random.Find the probability that the card has
\begin{enumerate}[label=(\roman*)]
	\item an even number 
	\item a square number
\end{enumerate}
\solution
%\input{exemplar/10/13/3/32/main.tex}
\item
The king, queen and jack of clubs are removed from a deck of 52 playing cards and then well shuffled. Now one card is drawn at random from the remaining cards.  Determine the probability that the card is
\begin{enumerate}[label=(\roman*)]
\item a club
\item 10 of hearts
\end{enumerate}
\solution
%\input{exemplar/10/13/3/29/main.tex}
\item A team of medical students doing their internship have to assist during surgeries
at a city hospital. The probabilities of surgeries rated as very complex, complex,
routine, simple or very simple are respectively, 0.15, 0.20, 0.31, 0.26, .08. Find
the probabilities that a particular surgery will be rated
\begin{enumerate}
	\item complex or very complex;
	\item neither very complex nor very simple;
	\item routine or complex
	\item routine or simple
\end{enumerate}
\solution
%\input{exemplar/11/16/3/8(1)/main.tex}
\item A card is selected from a pack of 52 cards.
\begin{enumerate}[label=(\alph*)]
    \item How many points are there in the sample space?
    \item Calculate the probability that the card is an ace of spades.
    \item Calculate the probability that the card is (i) an ace and (ii) black card.
\end{enumerate}
\solution
%\input{exemplar/11/16/3/4/main2.tex}
\item The probability that a non leap year selected at random will contain 53 sundays.
\\
\solution
%\input{exemplar/10/13/1/19/main.tex}
\item One of the four persons John, Rita, Aslam or Gurpreet will be promoted next
month. Consequently the sample space consists of four elementary outcomes
S = {John promoted, Rita promoted, Aslam promoted, Gurpreet promoted}
You are told that the chances of John’s promotion is same as that of Gurpreet,
Rita’s chances of promotion are twice as likely as Johns. Aslam’s chances are
four times that of John.
\begin{enumerate}
	\item Determine
	\begin{enumerate}
		\item P (John promoted)
		\item P (Rita promoted)
		\item P (Aslam promoted)
		\item P (Gurpreet promoted)
	\end{enumerate}
	\item If A = {John promoted or Gurpreet promoted}, find P (A).
\end{enumerate}
\solution
%\input{exemplar/11/16/3/10/main.tex}
\item A card is drawn from a deck of 52 cards. Find the probability of getting a king or a heart or a red card.\\
\solution
%\input{exemplar/11/16/3/15/main.tex}
\item The probability that a student will pass his examination is 0.73, the probability of
the student getting a compartment is 0.13, and the probability that the student will
either pass or get compartment is 0.96. State True or False.\\
\solution
%\input{exemplar/11/16/3/31/main.tex}
\item A card is selected from a pack of 52 cards\\
\begin{enumerate}[label=(\alph*)]
\item How many points are there in the sample space?
\item Calculate the probability that the cards is an ace of spades.
\item Calculate the probability that the card is (i) an ace (ii)black card.\\
\end{enumerate}
%\input{ncert/11/16/3/4_1/Prob_4.tex}
\item In a non-leap year, the probability of having 53 tuesdays or 53 wednesdays is\\
\solution
%\input{exemplar/11/16/3/18/main.tex}
\item There are 1000 sealed envelopes in a box, 10 of them contain a cash prize of
Rs 100 each, 100 of them contain a cash prize of Rs 50 each and 200 of them
contain a cash prize of Rs 10 each and rest do not contain any cash prize. If they
are well shuffled and an envelope is picked up out, what is the probability that it
contains no cash prize?\\
\solution
%\input{exemplar/10/13/3/34/main.tex}
\item 
A die is thrown and a card is selected at random from a deck of 52 playing cards. The probability of getting an even number on the die and a spade card.\\
\solution
%\input{exemplar/12/13/3/78/main.tex}
\item
If 4-digit numbers greater than 5,000 are randomly formed from the digits 0, 1, 3, 5, and 7, what is the probability of forming a number divisible by 5 when:
\begin{enumerate}
    \item The digits are repeated?
    \item The repetition of digits is not allowed?
\end{enumerate}
\solution
%\input{ncert/11/16/4/9/main.tex}
\item Consider the probability space $\brak{\Omega, \mathcal{G}, P}$ where $\Omega = [0,2]$ and $\mathcal{G} = \cbrak{\phi, \Omega, [0,1], (1,2]}$. Let $X$ and $Y$ be two functions on $\Omega$ defined as
\begin{align*}
    X(\omega) = 
    \begin{cases}
        1 & \text{if }\omega \in [0, 1]\\
        2 & \text{if }\omega \in (1, 2]
    \end{cases}
\end{align*}
and
\begin{align*}
    Y(\omega) = 
    \begin{cases}
        2 & \text{if }\omega \in [0, 1.5]\\
        3 & \text{if }\omega \in (1.5, 2].
    \end{cases}
\end{align*}
Then which one of the following statements is true?
\begin{enumerate}
    \item [(A)] $X$ is a random variable with respect to $\mathcal{G}$, but $Y$ is not a random variable with respect to $\mathcal{G}$.
    \item [(B)] $Y$ is a random variable with respect to $\mathcal{G}$, but $X$ is not a random variable with respect to $\mathcal{G}$.
    \item [(C)] Neither $X$ nor $Y$ is a random variable with respect to $\mathcal{G}$.
    \item [(D)] Both $X$ and $Y$ are random variables with respect to $\mathcal{G}$.
\end{enumerate} \hfill (GATE ST 2023)\\
\solution
%\input{gate/ST/2023/14/main.tex}
	\item  A die is loaded in such a way that each odd number is twice as likely to occur as
each even number. Find $P(G)$, where $G$ is the event that a number greater than
3 occurs on a single roll of the die.
\\
\solution
		%\input{exemplar/11/16/3/5/main.tex}
	\item All the jacks, queens and kings are removed from a deck of 52 playing cards. The remaining cards are well shuffled and then one card is drawn at random. Giving ace a value 1 similar value for other cards, find the probability that the card has a value 
		\begin{enumerate}
			\item 7
			\item greater than 7
			\item less than 7
		\end{enumerate}
		%\input{exemplar/10/13/3/30/main.tex}
  \item A Lot consists of 48 mobile phones of which 42 are good, 3 have only minor defects and 3 have major defects.Varnika will buy a phone if it is good but the trader will only buy a mobile if it has no major defects. One phone is selected at random from the lot. What is the probability that it is
\begin{enumerate}
	\item acceptable to Varnika?
            \item acceptable to the trader?
\end{enumerate}
\solution
	%\input{exemplar/10/13/3/40/main.tex}
 \item A student says that if you throw a die, it will show up 1 or not 1. Therefore, the probability of getting 1 and the probability of getting 'not 1' each is equal to $\frac{1}{2}$. Is this correct? Give reasons.\\
 \solution
        %\input{exemplar/10/13/2/9/main.tex}
   \item Four candidates A, B, C, D have ap-
plied for the assignment to coach a school cricket
team. If A is twice as likely to be selected as B, and
B and C are given about the same chance of being
selected, while C is twice as likely to be selected
as D, what are the probabilities that
\begin{enumerate}
\item C will be selected?
\item A will not be selected?
\end{enumerate}
	%\input{exemplar/11/16/3/9/main.tex}
 \item A bag contain 24 balls of which $x$ balls are red, $2x$ are white and $3x$ are blue. A ball is selected at random, What is the probability that it is
\begin{enumerate}[label=\alph*)]
\item not red ?
\item white ?
\end{enumerate}
%\input{exemplar/10/13/3/41/main.tex}
If the letters of the word ASSASSINATION are arranged at random. Find the Probability that
\begin{enumerate}[label=(\alph*)]
\item Four $S's$ come consecutively in the word
\item Two  $I's$ and two $N's$ come together
\item All $A's$ are not coming together
\item No two $A's$ are coming together
\end{enumerate}
%\input{exemplar/11/16/3/14/main.tex}
	\item One urn contains two black balls (labelled B1 and B2) and one white ball. A
	second urn contains one black ball and two white balls (labelled W1 and W2).
	Suppose the following experiment is performed. One of the two urns is chosen
	at random. Next a ball is randomly chosen from the urn. Then a second ball is
	chosen at random from the same urn without replacing the first ball.
	
	\begin{enumerate}
	\item What is the probability that two black balls are chosen?
	
	\item What is the probability that two balls of opposite colour are chosen?
	\end{enumerate}
	\solution
	%\input{exemplar/11/16/3/12/main1.tex}
\end{enumerate}

	\item A bag contains 4 red and 4 black balls, another bag contains 2 red and 6 black balls. One of the two bags is selected at random and a ball is drawn from the bag which is found to be red. Find the probability that the ball is drawn from the first bag.
\\
\solution
		%\begin{table}[H]
	\centering
\begin{tabular}{|c|c|c|}
\hline
Random variable &Value &Definition\\ \hline
\multirow{3}{*}{X} &0 &Slips of Rs 1\\
&1 &Slips of Rs 5\\
&2 &Slips of Rs 13\\ \hline
\multirow{2}{*}{Y} &0 &Box A\\
&1 &Box B\\\hline
\end{tabular}
\caption{}
\label{tab:Distribution}
\end{table}
See \tabref{tab:Distribution}.
\begin{align}
p_{Y}\brak{k}= \begin{cases} 
      \frac{1}{3} & {k=0} \\
      \frac{2}{3 }& {k=1} 
   \end{cases}
   \\
p_{Y|X}\brak{0|0} = \frac{19}{25}\, 
p_{Y|X}\brak{0|1} = \frac{6}{25}\,
p_{Y|X}\brak{1|0} = \frac{45}{50}\,
p_{Y|X}\brak{1|2} = \frac{5}{50}
\end{align}
The desired probability is the probability that a slip drawn at random is marked other than Rs 1,
\begin{align}
&=1-p_X\brak{0}\\
&= p_X(1) + p_X(2)
\end{align}
Using Bayes theorem,
\begin{align}
&= p_Y\brak{0} \times \pr{Y=0 | X=1} + p_Y\brak{1} \times \pr{Y=1|X=2}\\
&=\frac{1}{3} \times \frac{6}{25} + \frac{2}{3} \times \frac{5}{50}\\
&=\frac{11}{75}
\end{align}

\newpage

%\tableofcontents

\bigskip

\renewcommand{\thefigure}{\theenumi}
\renewcommand{\thetable}{\theenumi}
%\renewcommand{\theequation}{\theenumi}

%\begin{abstract}
%%\boldmath
%In this letter, an algorithm for evaluating the exact analytical bit error rate  (BER)  for the piecewise linear (PL) combiner for  multiple relays is presented. Previous results were available only for upto three relays. The algorithm is unique in the sense that  the actual mathematical expressions, that are prohibitively large, need not be explicitly obtained. The diversity gain due to multiple relays is shown through plots of the analytical BER, well supported by simulations. 
%
%\end{abstract}
% IEEEtran.cls defaults to using nonbold math in the Abstract.
% This preserves the distinction between vectors and scalars. However,
% if the journal you are submitting to favors bold math in the abstract,
% then you can use LaTeX's standard command \boldmath at the very start
% of the abstract to achieve this. Many IEEE journals frown on math
% in the abstract anyway.

% Note that keywords are not normally used for peerreview papers.
%\begin{IEEEkeywords}
%Cooperative diversity, decode and forward, piecewise linear
%\end{IEEEkeywords}



% For peer review papers, you can put extra information on the cover
% page as needed:
% \ifCLASSOPTIONpeerreview
% \begin{center} \bfseries EDICS Category: 3-BBND \end{center}
% \fi
%
% For peerreview papers, this IEEEtran command inserts a page break and
% creates the second title. It will be ignored for other modes.
%\IEEEpeerreviewmaketitle




  \item
  Cards with numbers 2 to 101 are placed in a box. A card is selected at random.Find the probability that the card has
\begin{enumerate}[label=(\roman*)]
	\item an even number 
	\item a square number
\end{enumerate}
\solution
%\begin{table}[H]
	\centering
\begin{tabular}{|c|c|c|}
\hline
Random variable &Value &Definition\\ \hline
\multirow{3}{*}{X} &0 &Slips of Rs 1\\
&1 &Slips of Rs 5\\
&2 &Slips of Rs 13\\ \hline
\multirow{2}{*}{Y} &0 &Box A\\
&1 &Box B\\\hline
\end{tabular}
\caption{}
\label{tab:Distribution}
\end{table}
See \tabref{tab:Distribution}.
\begin{align}
p_{Y}\brak{k}= \begin{cases} 
      \frac{1}{3} & {k=0} \\
      \frac{2}{3 }& {k=1} 
   \end{cases}
   \\
p_{Y|X}\brak{0|0} = \frac{19}{25}\, 
p_{Y|X}\brak{0|1} = \frac{6}{25}\,
p_{Y|X}\brak{1|0} = \frac{45}{50}\,
p_{Y|X}\brak{1|2} = \frac{5}{50}
\end{align}
The desired probability is the probability that a slip drawn at random is marked other than Rs 1,
\begin{align}
&=1-p_X\brak{0}\\
&= p_X(1) + p_X(2)
\end{align}
Using Bayes theorem,
\begin{align}
&= p_Y\brak{0} \times \pr{Y=0 | X=1} + p_Y\brak{1} \times \pr{Y=1|X=2}\\
&=\frac{1}{3} \times \frac{6}{25} + \frac{2}{3} \times \frac{5}{50}\\
&=\frac{11}{75}
\end{align}

\newpage

%\tableofcontents

\bigskip

\renewcommand{\thefigure}{\theenumi}
\renewcommand{\thetable}{\theenumi}
%\renewcommand{\theequation}{\theenumi}

%\begin{abstract}
%%\boldmath
%In this letter, an algorithm for evaluating the exact analytical bit error rate  (BER)  for the piecewise linear (PL) combiner for  multiple relays is presented. Previous results were available only for upto three relays. The algorithm is unique in the sense that  the actual mathematical expressions, that are prohibitively large, need not be explicitly obtained. The diversity gain due to multiple relays is shown through plots of the analytical BER, well supported by simulations. 
%
%\end{abstract}
% IEEEtran.cls defaults to using nonbold math in the Abstract.
% This preserves the distinction between vectors and scalars. However,
% if the journal you are submitting to favors bold math in the abstract,
% then you can use LaTeX's standard command \boldmath at the very start
% of the abstract to achieve this. Many IEEE journals frown on math
% in the abstract anyway.

% Note that keywords are not normally used for peerreview papers.
%\begin{IEEEkeywords}
%Cooperative diversity, decode and forward, piecewise linear
%\end{IEEEkeywords}



% For peer review papers, you can put extra information on the cover
% page as needed:
% \ifCLASSOPTIONpeerreview
% \begin{center} \bfseries EDICS Category: 3-BBND \end{center}
% \fi
%
% For peerreview papers, this IEEEtran command inserts a page break and
% creates the second title. It will be ignored for other modes.
%\IEEEpeerreviewmaketitle




\item
The king, queen and jack of clubs are removed from a deck of 52 playing cards and then well shuffled. Now one card is drawn at random from the remaining cards.  Determine the probability that the card is
\begin{enumerate}[label=(\roman*)]
\item a club
\item 10 of hearts
\end{enumerate}
\solution
%\begin{table}[H]
	\centering
\begin{tabular}{|c|c|c|}
\hline
Random variable &Value &Definition\\ \hline
\multirow{3}{*}{X} &0 &Slips of Rs 1\\
&1 &Slips of Rs 5\\
&2 &Slips of Rs 13\\ \hline
\multirow{2}{*}{Y} &0 &Box A\\
&1 &Box B\\\hline
\end{tabular}
\caption{}
\label{tab:Distribution}
\end{table}
See \tabref{tab:Distribution}.
\begin{align}
p_{Y}\brak{k}= \begin{cases} 
      \frac{1}{3} & {k=0} \\
      \frac{2}{3 }& {k=1} 
   \end{cases}
   \\
p_{Y|X}\brak{0|0} = \frac{19}{25}\, 
p_{Y|X}\brak{0|1} = \frac{6}{25}\,
p_{Y|X}\brak{1|0} = \frac{45}{50}\,
p_{Y|X}\brak{1|2} = \frac{5}{50}
\end{align}
The desired probability is the probability that a slip drawn at random is marked other than Rs 1,
\begin{align}
&=1-p_X\brak{0}\\
&= p_X(1) + p_X(2)
\end{align}
Using Bayes theorem,
\begin{align}
&= p_Y\brak{0} \times \pr{Y=0 | X=1} + p_Y\brak{1} \times \pr{Y=1|X=2}\\
&=\frac{1}{3} \times \frac{6}{25} + \frac{2}{3} \times \frac{5}{50}\\
&=\frac{11}{75}
\end{align}

\newpage

%\tableofcontents

\bigskip

\renewcommand{\thefigure}{\theenumi}
\renewcommand{\thetable}{\theenumi}
%\renewcommand{\theequation}{\theenumi}

%\begin{abstract}
%%\boldmath
%In this letter, an algorithm for evaluating the exact analytical bit error rate  (BER)  for the piecewise linear (PL) combiner for  multiple relays is presented. Previous results were available only for upto three relays. The algorithm is unique in the sense that  the actual mathematical expressions, that are prohibitively large, need not be explicitly obtained. The diversity gain due to multiple relays is shown through plots of the analytical BER, well supported by simulations. 
%
%\end{abstract}
% IEEEtran.cls defaults to using nonbold math in the Abstract.
% This preserves the distinction between vectors and scalars. However,
% if the journal you are submitting to favors bold math in the abstract,
% then you can use LaTeX's standard command \boldmath at the very start
% of the abstract to achieve this. Many IEEE journals frown on math
% in the abstract anyway.

% Note that keywords are not normally used for peerreview papers.
%\begin{IEEEkeywords}
%Cooperative diversity, decode and forward, piecewise linear
%\end{IEEEkeywords}



% For peer review papers, you can put extra information on the cover
% page as needed:
% \ifCLASSOPTIONpeerreview
% \begin{center} \bfseries EDICS Category: 3-BBND \end{center}
% \fi
%
% For peerreview papers, this IEEEtran command inserts a page break and
% creates the second title. It will be ignored for other modes.
%\IEEEpeerreviewmaketitle




\item A team of medical students doing their internship have to assist during surgeries
at a city hospital. The probabilities of surgeries rated as very complex, complex,
routine, simple or very simple are respectively, 0.15, 0.20, 0.31, 0.26, .08. Find
the probabilities that a particular surgery will be rated
\begin{enumerate}
	\item complex or very complex;
	\item neither very complex nor very simple;
	\item routine or complex
	\item routine or simple
\end{enumerate}
\solution
%\begin{table}[H]
	\centering
\begin{tabular}{|c|c|c|}
\hline
Random variable &Value &Definition\\ \hline
\multirow{3}{*}{X} &0 &Slips of Rs 1\\
&1 &Slips of Rs 5\\
&2 &Slips of Rs 13\\ \hline
\multirow{2}{*}{Y} &0 &Box A\\
&1 &Box B\\\hline
\end{tabular}
\caption{}
\label{tab:Distribution}
\end{table}
See \tabref{tab:Distribution}.
\begin{align}
p_{Y}\brak{k}= \begin{cases} 
      \frac{1}{3} & {k=0} \\
      \frac{2}{3 }& {k=1} 
   \end{cases}
   \\
p_{Y|X}\brak{0|0} = \frac{19}{25}\, 
p_{Y|X}\brak{0|1} = \frac{6}{25}\,
p_{Y|X}\brak{1|0} = \frac{45}{50}\,
p_{Y|X}\brak{1|2} = \frac{5}{50}
\end{align}
The desired probability is the probability that a slip drawn at random is marked other than Rs 1,
\begin{align}
&=1-p_X\brak{0}\\
&= p_X(1) + p_X(2)
\end{align}
Using Bayes theorem,
\begin{align}
&= p_Y\brak{0} \times \pr{Y=0 | X=1} + p_Y\brak{1} \times \pr{Y=1|X=2}\\
&=\frac{1}{3} \times \frac{6}{25} + \frac{2}{3} \times \frac{5}{50}\\
&=\frac{11}{75}
\end{align}

\newpage

%\tableofcontents

\bigskip

\renewcommand{\thefigure}{\theenumi}
\renewcommand{\thetable}{\theenumi}
%\renewcommand{\theequation}{\theenumi}

%\begin{abstract}
%%\boldmath
%In this letter, an algorithm for evaluating the exact analytical bit error rate  (BER)  for the piecewise linear (PL) combiner for  multiple relays is presented. Previous results were available only for upto three relays. The algorithm is unique in the sense that  the actual mathematical expressions, that are prohibitively large, need not be explicitly obtained. The diversity gain due to multiple relays is shown through plots of the analytical BER, well supported by simulations. 
%
%\end{abstract}
% IEEEtran.cls defaults to using nonbold math in the Abstract.
% This preserves the distinction between vectors and scalars. However,
% if the journal you are submitting to favors bold math in the abstract,
% then you can use LaTeX's standard command \boldmath at the very start
% of the abstract to achieve this. Many IEEE journals frown on math
% in the abstract anyway.

% Note that keywords are not normally used for peerreview papers.
%\begin{IEEEkeywords}
%Cooperative diversity, decode and forward, piecewise linear
%\end{IEEEkeywords}



% For peer review papers, you can put extra information on the cover
% page as needed:
% \ifCLASSOPTIONpeerreview
% \begin{center} \bfseries EDICS Category: 3-BBND \end{center}
% \fi
%
% For peerreview papers, this IEEEtran command inserts a page break and
% creates the second title. It will be ignored for other modes.
%\IEEEpeerreviewmaketitle




\item A card is selected from a pack of 52 cards.
\begin{enumerate}[label=(\alph*)]
    \item How many points are there in the sample space?
    \item Calculate the probability that the card is an ace of spades.
    \item Calculate the probability that the card is (i) an ace and (ii) black card.
\end{enumerate}
\solution
%Let $X$ be an bernoulli rv defined as in \tabref{tab:exemplar/11/16/3/26}.  Then, 
\begin{equation}
    p =
        \frac{4}{11} 
\end{equation}
\begin{table}[H]
	\centering
	\input{exemplar/11/16/3/26/tables/Table2.tex}
	\caption{}
        \label{tab:exemplar/11/16/3/26}
\end{table}

\item The probability that a non leap year selected at random will contain 53 sundays.
\\
\solution
%\begin{table}[H]
	\centering
\begin{tabular}{|c|c|c|}
\hline
Random variable &Value &Definition\\ \hline
\multirow{3}{*}{X} &0 &Slips of Rs 1\\
&1 &Slips of Rs 5\\
&2 &Slips of Rs 13\\ \hline
\multirow{2}{*}{Y} &0 &Box A\\
&1 &Box B\\\hline
\end{tabular}
\caption{}
\label{tab:Distribution}
\end{table}
See \tabref{tab:Distribution}.
\begin{align}
p_{Y}\brak{k}= \begin{cases} 
      \frac{1}{3} & {k=0} \\
      \frac{2}{3 }& {k=1} 
   \end{cases}
   \\
p_{Y|X}\brak{0|0} = \frac{19}{25}\, 
p_{Y|X}\brak{0|1} = \frac{6}{25}\,
p_{Y|X}\brak{1|0} = \frac{45}{50}\,
p_{Y|X}\brak{1|2} = \frac{5}{50}
\end{align}
The desired probability is the probability that a slip drawn at random is marked other than Rs 1,
\begin{align}
&=1-p_X\brak{0}\\
&= p_X(1) + p_X(2)
\end{align}
Using Bayes theorem,
\begin{align}
&= p_Y\brak{0} \times \pr{Y=0 | X=1} + p_Y\brak{1} \times \pr{Y=1|X=2}\\
&=\frac{1}{3} \times \frac{6}{25} + \frac{2}{3} \times \frac{5}{50}\\
&=\frac{11}{75}
\end{align}

\newpage

%\tableofcontents

\bigskip

\renewcommand{\thefigure}{\theenumi}
\renewcommand{\thetable}{\theenumi}
%\renewcommand{\theequation}{\theenumi}

%\begin{abstract}
%%\boldmath
%In this letter, an algorithm for evaluating the exact analytical bit error rate  (BER)  for the piecewise linear (PL) combiner for  multiple relays is presented. Previous results were available only for upto three relays. The algorithm is unique in the sense that  the actual mathematical expressions, that are prohibitively large, need not be explicitly obtained. The diversity gain due to multiple relays is shown through plots of the analytical BER, well supported by simulations. 
%
%\end{abstract}
% IEEEtran.cls defaults to using nonbold math in the Abstract.
% This preserves the distinction between vectors and scalars. However,
% if the journal you are submitting to favors bold math in the abstract,
% then you can use LaTeX's standard command \boldmath at the very start
% of the abstract to achieve this. Many IEEE journals frown on math
% in the abstract anyway.

% Note that keywords are not normally used for peerreview papers.
%\begin{IEEEkeywords}
%Cooperative diversity, decode and forward, piecewise linear
%\end{IEEEkeywords}



% For peer review papers, you can put extra information on the cover
% page as needed:
% \ifCLASSOPTIONpeerreview
% \begin{center} \bfseries EDICS Category: 3-BBND \end{center}
% \fi
%
% For peerreview papers, this IEEEtran command inserts a page break and
% creates the second title. It will be ignored for other modes.
%\IEEEpeerreviewmaketitle




\item One of the four persons John, Rita, Aslam or Gurpreet will be promoted next
month. Consequently the sample space consists of four elementary outcomes
S = {John promoted, Rita promoted, Aslam promoted, Gurpreet promoted}
You are told that the chances of John’s promotion is same as that of Gurpreet,
Rita’s chances of promotion are twice as likely as Johns. Aslam’s chances are
four times that of John.
\begin{enumerate}
	\item Determine
	\begin{enumerate}
		\item P (John promoted)
		\item P (Rita promoted)
		\item P (Aslam promoted)
		\item P (Gurpreet promoted)
	\end{enumerate}
	\item If A = {John promoted or Gurpreet promoted}, find P (A).
\end{enumerate}
\solution
%\begin{table}[H]
	\centering
\begin{tabular}{|c|c|c|}
\hline
Random variable &Value &Definition\\ \hline
\multirow{3}{*}{X} &0 &Slips of Rs 1\\
&1 &Slips of Rs 5\\
&2 &Slips of Rs 13\\ \hline
\multirow{2}{*}{Y} &0 &Box A\\
&1 &Box B\\\hline
\end{tabular}
\caption{}
\label{tab:Distribution}
\end{table}
See \tabref{tab:Distribution}.
\begin{align}
p_{Y}\brak{k}= \begin{cases} 
      \frac{1}{3} & {k=0} \\
      \frac{2}{3 }& {k=1} 
   \end{cases}
   \\
p_{Y|X}\brak{0|0} = \frac{19}{25}\, 
p_{Y|X}\brak{0|1} = \frac{6}{25}\,
p_{Y|X}\brak{1|0} = \frac{45}{50}\,
p_{Y|X}\brak{1|2} = \frac{5}{50}
\end{align}
The desired probability is the probability that a slip drawn at random is marked other than Rs 1,
\begin{align}
&=1-p_X\brak{0}\\
&= p_X(1) + p_X(2)
\end{align}
Using Bayes theorem,
\begin{align}
&= p_Y\brak{0} \times \pr{Y=0 | X=1} + p_Y\brak{1} \times \pr{Y=1|X=2}\\
&=\frac{1}{3} \times \frac{6}{25} + \frac{2}{3} \times \frac{5}{50}\\
&=\frac{11}{75}
\end{align}

\newpage

%\tableofcontents

\bigskip

\renewcommand{\thefigure}{\theenumi}
\renewcommand{\thetable}{\theenumi}
%\renewcommand{\theequation}{\theenumi}

%\begin{abstract}
%%\boldmath
%In this letter, an algorithm for evaluating the exact analytical bit error rate  (BER)  for the piecewise linear (PL) combiner for  multiple relays is presented. Previous results were available only for upto three relays. The algorithm is unique in the sense that  the actual mathematical expressions, that are prohibitively large, need not be explicitly obtained. The diversity gain due to multiple relays is shown through plots of the analytical BER, well supported by simulations. 
%
%\end{abstract}
% IEEEtran.cls defaults to using nonbold math in the Abstract.
% This preserves the distinction between vectors and scalars. However,
% if the journal you are submitting to favors bold math in the abstract,
% then you can use LaTeX's standard command \boldmath at the very start
% of the abstract to achieve this. Many IEEE journals frown on math
% in the abstract anyway.

% Note that keywords are not normally used for peerreview papers.
%\begin{IEEEkeywords}
%Cooperative diversity, decode and forward, piecewise linear
%\end{IEEEkeywords}



% For peer review papers, you can put extra information on the cover
% page as needed:
% \ifCLASSOPTIONpeerreview
% \begin{center} \bfseries EDICS Category: 3-BBND \end{center}
% \fi
%
% For peerreview papers, this IEEEtran command inserts a page break and
% creates the second title. It will be ignored for other modes.
%\IEEEpeerreviewmaketitle




\item A card is drawn from a deck of 52 cards. Find the probability of getting a king or a heart or a red card.\\
\solution
%\begin{table}[H]
	\centering
\begin{tabular}{|c|c|c|}
\hline
Random variable &Value &Definition\\ \hline
\multirow{3}{*}{X} &0 &Slips of Rs 1\\
&1 &Slips of Rs 5\\
&2 &Slips of Rs 13\\ \hline
\multirow{2}{*}{Y} &0 &Box A\\
&1 &Box B\\\hline
\end{tabular}
\caption{}
\label{tab:Distribution}
\end{table}
See \tabref{tab:Distribution}.
\begin{align}
p_{Y}\brak{k}= \begin{cases} 
      \frac{1}{3} & {k=0} \\
      \frac{2}{3 }& {k=1} 
   \end{cases}
   \\
p_{Y|X}\brak{0|0} = \frac{19}{25}\, 
p_{Y|X}\brak{0|1} = \frac{6}{25}\,
p_{Y|X}\brak{1|0} = \frac{45}{50}\,
p_{Y|X}\brak{1|2} = \frac{5}{50}
\end{align}
The desired probability is the probability that a slip drawn at random is marked other than Rs 1,
\begin{align}
&=1-p_X\brak{0}\\
&= p_X(1) + p_X(2)
\end{align}
Using Bayes theorem,
\begin{align}
&= p_Y\brak{0} \times \pr{Y=0 | X=1} + p_Y\brak{1} \times \pr{Y=1|X=2}\\
&=\frac{1}{3} \times \frac{6}{25} + \frac{2}{3} \times \frac{5}{50}\\
&=\frac{11}{75}
\end{align}

\newpage

%\tableofcontents

\bigskip

\renewcommand{\thefigure}{\theenumi}
\renewcommand{\thetable}{\theenumi}
%\renewcommand{\theequation}{\theenumi}

%\begin{abstract}
%%\boldmath
%In this letter, an algorithm for evaluating the exact analytical bit error rate  (BER)  for the piecewise linear (PL) combiner for  multiple relays is presented. Previous results were available only for upto three relays. The algorithm is unique in the sense that  the actual mathematical expressions, that are prohibitively large, need not be explicitly obtained. The diversity gain due to multiple relays is shown through plots of the analytical BER, well supported by simulations. 
%
%\end{abstract}
% IEEEtran.cls defaults to using nonbold math in the Abstract.
% This preserves the distinction between vectors and scalars. However,
% if the journal you are submitting to favors bold math in the abstract,
% then you can use LaTeX's standard command \boldmath at the very start
% of the abstract to achieve this. Many IEEE journals frown on math
% in the abstract anyway.

% Note that keywords are not normally used for peerreview papers.
%\begin{IEEEkeywords}
%Cooperative diversity, decode and forward, piecewise linear
%\end{IEEEkeywords}



% For peer review papers, you can put extra information on the cover
% page as needed:
% \ifCLASSOPTIONpeerreview
% \begin{center} \bfseries EDICS Category: 3-BBND \end{center}
% \fi
%
% For peerreview papers, this IEEEtran command inserts a page break and
% creates the second title. It will be ignored for other modes.
%\IEEEpeerreviewmaketitle




\item The probability that a student will pass his examination is 0.73, the probability of
the student getting a compartment is 0.13, and the probability that the student will
either pass or get compartment is 0.96. State True or False.\\
\solution
%\begin{table}[H]
	\centering
\begin{tabular}{|c|c|c|}
\hline
Random variable &Value &Definition\\ \hline
\multirow{3}{*}{X} &0 &Slips of Rs 1\\
&1 &Slips of Rs 5\\
&2 &Slips of Rs 13\\ \hline
\multirow{2}{*}{Y} &0 &Box A\\
&1 &Box B\\\hline
\end{tabular}
\caption{}
\label{tab:Distribution}
\end{table}
See \tabref{tab:Distribution}.
\begin{align}
p_{Y}\brak{k}= \begin{cases} 
      \frac{1}{3} & {k=0} \\
      \frac{2}{3 }& {k=1} 
   \end{cases}
   \\
p_{Y|X}\brak{0|0} = \frac{19}{25}\, 
p_{Y|X}\brak{0|1} = \frac{6}{25}\,
p_{Y|X}\brak{1|0} = \frac{45}{50}\,
p_{Y|X}\brak{1|2} = \frac{5}{50}
\end{align}
The desired probability is the probability that a slip drawn at random is marked other than Rs 1,
\begin{align}
&=1-p_X\brak{0}\\
&= p_X(1) + p_X(2)
\end{align}
Using Bayes theorem,
\begin{align}
&= p_Y\brak{0} \times \pr{Y=0 | X=1} + p_Y\brak{1} \times \pr{Y=1|X=2}\\
&=\frac{1}{3} \times \frac{6}{25} + \frac{2}{3} \times \frac{5}{50}\\
&=\frac{11}{75}
\end{align}

\newpage

%\tableofcontents

\bigskip

\renewcommand{\thefigure}{\theenumi}
\renewcommand{\thetable}{\theenumi}
%\renewcommand{\theequation}{\theenumi}

%\begin{abstract}
%%\boldmath
%In this letter, an algorithm for evaluating the exact analytical bit error rate  (BER)  for the piecewise linear (PL) combiner for  multiple relays is presented. Previous results were available only for upto three relays. The algorithm is unique in the sense that  the actual mathematical expressions, that are prohibitively large, need not be explicitly obtained. The diversity gain due to multiple relays is shown through plots of the analytical BER, well supported by simulations. 
%
%\end{abstract}
% IEEEtran.cls defaults to using nonbold math in the Abstract.
% This preserves the distinction between vectors and scalars. However,
% if the journal you are submitting to favors bold math in the abstract,
% then you can use LaTeX's standard command \boldmath at the very start
% of the abstract to achieve this. Many IEEE journals frown on math
% in the abstract anyway.

% Note that keywords are not normally used for peerreview papers.
%\begin{IEEEkeywords}
%Cooperative diversity, decode and forward, piecewise linear
%\end{IEEEkeywords}



% For peer review papers, you can put extra information on the cover
% page as needed:
% \ifCLASSOPTIONpeerreview
% \begin{center} \bfseries EDICS Category: 3-BBND \end{center}
% \fi
%
% For peerreview papers, this IEEEtran command inserts a page break and
% creates the second title. It will be ignored for other modes.
%\IEEEpeerreviewmaketitle




\item A card is selected from a pack of 52 cards\\
\begin{enumerate}[label=(\alph*)]
\item How many points are there in the sample space?
\item Calculate the probability that the cards is an ace of spades.
\item Calculate the probability that the card is (i) an ace (ii)black card.\\
\end{enumerate}
%\input{ncert/11/16/3/4_1/Prob_4.tex}
\item In a non-leap year, the probability of having 53 tuesdays or 53 wednesdays is\\
\solution
%A non-leap year has a total of 365 days, and a week has 7 days.\\
So it can be expressed as 
\begin{align}
365\text{days} &=52\times 7+1 \text{day}
\end{align}
$\implies$ 52 tuesdays or wednesdays\\
Random variable X denotes the days of a week
\begin{align}
p_X\brak{k}&=\frac{1}{7}; \quad \brak{1<k<7}
\end{align}
So the probability of extra day being tuesday or wednesday is
\begin{align}
p_X\brak{3}+p_X\brak{4}&=\frac{1}{7}+\frac{1}{7}=\frac{2}{7}
\end{align}



\item There are 1000 sealed envelopes in a box, 10 of them contain a cash prize of
Rs 100 each, 100 of them contain a cash prize of Rs 50 each and 200 of them
contain a cash prize of Rs 10 each and rest do not contain any cash prize. If they
are well shuffled and an envelope is picked up out, what is the probability that it
contains no cash prize?\\
\solution
%\begin{table}[H]
	\centering
\begin{tabular}{|c|c|c|}
\hline
Random variable &Value &Definition\\ \hline
\multirow{3}{*}{X} &0 &Slips of Rs 1\\
&1 &Slips of Rs 5\\
&2 &Slips of Rs 13\\ \hline
\multirow{2}{*}{Y} &0 &Box A\\
&1 &Box B\\\hline
\end{tabular}
\caption{}
\label{tab:Distribution}
\end{table}
See \tabref{tab:Distribution}.
\begin{align}
p_{Y}\brak{k}= \begin{cases} 
      \frac{1}{3} & {k=0} \\
      \frac{2}{3 }& {k=1} 
   \end{cases}
   \\
p_{Y|X}\brak{0|0} = \frac{19}{25}\, 
p_{Y|X}\brak{0|1} = \frac{6}{25}\,
p_{Y|X}\brak{1|0} = \frac{45}{50}\,
p_{Y|X}\brak{1|2} = \frac{5}{50}
\end{align}
The desired probability is the probability that a slip drawn at random is marked other than Rs 1,
\begin{align}
&=1-p_X\brak{0}\\
&= p_X(1) + p_X(2)
\end{align}
Using Bayes theorem,
\begin{align}
&= p_Y\brak{0} \times \pr{Y=0 | X=1} + p_Y\brak{1} \times \pr{Y=1|X=2}\\
&=\frac{1}{3} \times \frac{6}{25} + \frac{2}{3} \times \frac{5}{50}\\
&=\frac{11}{75}
\end{align}

\newpage

%\tableofcontents

\bigskip

\renewcommand{\thefigure}{\theenumi}
\renewcommand{\thetable}{\theenumi}
%\renewcommand{\theequation}{\theenumi}

%\begin{abstract}
%%\boldmath
%In this letter, an algorithm for evaluating the exact analytical bit error rate  (BER)  for the piecewise linear (PL) combiner for  multiple relays is presented. Previous results were available only for upto three relays. The algorithm is unique in the sense that  the actual mathematical expressions, that are prohibitively large, need not be explicitly obtained. The diversity gain due to multiple relays is shown through plots of the analytical BER, well supported by simulations. 
%
%\end{abstract}
% IEEEtran.cls defaults to using nonbold math in the Abstract.
% This preserves the distinction between vectors and scalars. However,
% if the journal you are submitting to favors bold math in the abstract,
% then you can use LaTeX's standard command \boldmath at the very start
% of the abstract to achieve this. Many IEEE journals frown on math
% in the abstract anyway.

% Note that keywords are not normally used for peerreview papers.
%\begin{IEEEkeywords}
%Cooperative diversity, decode and forward, piecewise linear
%\end{IEEEkeywords}



% For peer review papers, you can put extra information on the cover
% page as needed:
% \ifCLASSOPTIONpeerreview
% \begin{center} \bfseries EDICS Category: 3-BBND \end{center}
% \fi
%
% For peerreview papers, this IEEEtran command inserts a page break and
% creates the second title. It will be ignored for other modes.
%\IEEEpeerreviewmaketitle




\item 
A die is thrown and a card is selected at random from a deck of 52 playing cards. The probability of getting an even number on the die and a spade card.\\
\solution
%\begin{table}[H]
	\centering
\begin{tabular}{|c|c|c|}
\hline
Random variable &Value &Definition\\ \hline
\multirow{3}{*}{X} &0 &Slips of Rs 1\\
&1 &Slips of Rs 5\\
&2 &Slips of Rs 13\\ \hline
\multirow{2}{*}{Y} &0 &Box A\\
&1 &Box B\\\hline
\end{tabular}
\caption{}
\label{tab:Distribution}
\end{table}
See \tabref{tab:Distribution}.
\begin{align}
p_{Y}\brak{k}= \begin{cases} 
      \frac{1}{3} & {k=0} \\
      \frac{2}{3 }& {k=1} 
   \end{cases}
   \\
p_{Y|X}\brak{0|0} = \frac{19}{25}\, 
p_{Y|X}\brak{0|1} = \frac{6}{25}\,
p_{Y|X}\brak{1|0} = \frac{45}{50}\,
p_{Y|X}\brak{1|2} = \frac{5}{50}
\end{align}
The desired probability is the probability that a slip drawn at random is marked other than Rs 1,
\begin{align}
&=1-p_X\brak{0}\\
&= p_X(1) + p_X(2)
\end{align}
Using Bayes theorem,
\begin{align}
&= p_Y\brak{0} \times \pr{Y=0 | X=1} + p_Y\brak{1} \times \pr{Y=1|X=2}\\
&=\frac{1}{3} \times \frac{6}{25} + \frac{2}{3} \times \frac{5}{50}\\
&=\frac{11}{75}
\end{align}

\newpage

%\tableofcontents

\bigskip

\renewcommand{\thefigure}{\theenumi}
\renewcommand{\thetable}{\theenumi}
%\renewcommand{\theequation}{\theenumi}

%\begin{abstract}
%%\boldmath
%In this letter, an algorithm for evaluating the exact analytical bit error rate  (BER)  for the piecewise linear (PL) combiner for  multiple relays is presented. Previous results were available only for upto three relays. The algorithm is unique in the sense that  the actual mathematical expressions, that are prohibitively large, need not be explicitly obtained. The diversity gain due to multiple relays is shown through plots of the analytical BER, well supported by simulations. 
%
%\end{abstract}
% IEEEtran.cls defaults to using nonbold math in the Abstract.
% This preserves the distinction between vectors and scalars. However,
% if the journal you are submitting to favors bold math in the abstract,
% then you can use LaTeX's standard command \boldmath at the very start
% of the abstract to achieve this. Many IEEE journals frown on math
% in the abstract anyway.

% Note that keywords are not normally used for peerreview papers.
%\begin{IEEEkeywords}
%Cooperative diversity, decode and forward, piecewise linear
%\end{IEEEkeywords}



% For peer review papers, you can put extra information on the cover
% page as needed:
% \ifCLASSOPTIONpeerreview
% \begin{center} \bfseries EDICS Category: 3-BBND \end{center}
% \fi
%
% For peerreview papers, this IEEEtran command inserts a page break and
% creates the second title. It will be ignored for other modes.
%\IEEEpeerreviewmaketitle




\item
If 4-digit numbers greater than 5,000 are randomly formed from the digits 0, 1, 3, 5, and 7, what is the probability of forming a number divisible by 5 when:
\begin{enumerate}
    \item The digits are repeated?
    \item The repetition of digits is not allowed?
\end{enumerate}
\solution
%\begin{table}[H]
	\centering
\begin{tabular}{|c|c|c|}
\hline
Random variable &Value &Definition\\ \hline
\multirow{3}{*}{X} &0 &Slips of Rs 1\\
&1 &Slips of Rs 5\\
&2 &Slips of Rs 13\\ \hline
\multirow{2}{*}{Y} &0 &Box A\\
&1 &Box B\\\hline
\end{tabular}
\caption{}
\label{tab:Distribution}
\end{table}
See \tabref{tab:Distribution}.
\begin{align}
p_{Y}\brak{k}= \begin{cases} 
      \frac{1}{3} & {k=0} \\
      \frac{2}{3 }& {k=1} 
   \end{cases}
   \\
p_{Y|X}\brak{0|0} = \frac{19}{25}\, 
p_{Y|X}\brak{0|1} = \frac{6}{25}\,
p_{Y|X}\brak{1|0} = \frac{45}{50}\,
p_{Y|X}\brak{1|2} = \frac{5}{50}
\end{align}
The desired probability is the probability that a slip drawn at random is marked other than Rs 1,
\begin{align}
&=1-p_X\brak{0}\\
&= p_X(1) + p_X(2)
\end{align}
Using Bayes theorem,
\begin{align}
&= p_Y\brak{0} \times \pr{Y=0 | X=1} + p_Y\brak{1} \times \pr{Y=1|X=2}\\
&=\frac{1}{3} \times \frac{6}{25} + \frac{2}{3} \times \frac{5}{50}\\
&=\frac{11}{75}
\end{align}

\newpage

%\tableofcontents

\bigskip

\renewcommand{\thefigure}{\theenumi}
\renewcommand{\thetable}{\theenumi}
%\renewcommand{\theequation}{\theenumi}

%\begin{abstract}
%%\boldmath
%In this letter, an algorithm for evaluating the exact analytical bit error rate  (BER)  for the piecewise linear (PL) combiner for  multiple relays is presented. Previous results were available only for upto three relays. The algorithm is unique in the sense that  the actual mathematical expressions, that are prohibitively large, need not be explicitly obtained. The diversity gain due to multiple relays is shown through plots of the analytical BER, well supported by simulations. 
%
%\end{abstract}
% IEEEtran.cls defaults to using nonbold math in the Abstract.
% This preserves the distinction between vectors and scalars. However,
% if the journal you are submitting to favors bold math in the abstract,
% then you can use LaTeX's standard command \boldmath at the very start
% of the abstract to achieve this. Many IEEE journals frown on math
% in the abstract anyway.

% Note that keywords are not normally used for peerreview papers.
%\begin{IEEEkeywords}
%Cooperative diversity, decode and forward, piecewise linear
%\end{IEEEkeywords}



% For peer review papers, you can put extra information on the cover
% page as needed:
% \ifCLASSOPTIONpeerreview
% \begin{center} \bfseries EDICS Category: 3-BBND \end{center}
% \fi
%
% For peerreview papers, this IEEEtran command inserts a page break and
% creates the second title. It will be ignored for other modes.
%\IEEEpeerreviewmaketitle




\item Consider the probability space $\brak{\Omega, \mathcal{G}, P}$ where $\Omega = [0,2]$ and $\mathcal{G} = \cbrak{\phi, \Omega, [0,1], (1,2]}$. Let $X$ and $Y$ be two functions on $\Omega$ defined as
\begin{align*}
    X(\omega) = 
    \begin{cases}
        1 & \text{if }\omega \in [0, 1]\\
        2 & \text{if }\omega \in (1, 2]
    \end{cases}
\end{align*}
and
\begin{align*}
    Y(\omega) = 
    \begin{cases}
        2 & \text{if }\omega \in [0, 1.5]\\
        3 & \text{if }\omega \in (1.5, 2].
    \end{cases}
\end{align*}
Then which one of the following statements is true?
\begin{enumerate}
    \item [(A)] $X$ is a random variable with respect to $\mathcal{G}$, but $Y$ is not a random variable with respect to $\mathcal{G}$.
    \item [(B)] $Y$ is a random variable with respect to $\mathcal{G}$, but $X$ is not a random variable with respect to $\mathcal{G}$.
    \item [(C)] Neither $X$ nor $Y$ is a random variable with respect to $\mathcal{G}$.
    \item [(D)] Both $X$ and $Y$ are random variables with respect to $\mathcal{G}$.
\end{enumerate} \hfill (GATE ST 2023)\\
\solution
%\begin{table}[H]
	\centering
\begin{tabular}{|c|c|c|}
\hline
Random variable &Value &Definition\\ \hline
\multirow{3}{*}{X} &0 &Slips of Rs 1\\
&1 &Slips of Rs 5\\
&2 &Slips of Rs 13\\ \hline
\multirow{2}{*}{Y} &0 &Box A\\
&1 &Box B\\\hline
\end{tabular}
\caption{}
\label{tab:Distribution}
\end{table}
See \tabref{tab:Distribution}.
\begin{align}
p_{Y}\brak{k}= \begin{cases} 
      \frac{1}{3} & {k=0} \\
      \frac{2}{3 }& {k=1} 
   \end{cases}
   \\
p_{Y|X}\brak{0|0} = \frac{19}{25}\, 
p_{Y|X}\brak{0|1} = \frac{6}{25}\,
p_{Y|X}\brak{1|0} = \frac{45}{50}\,
p_{Y|X}\brak{1|2} = \frac{5}{50}
\end{align}
The desired probability is the probability that a slip drawn at random is marked other than Rs 1,
\begin{align}
&=1-p_X\brak{0}\\
&= p_X(1) + p_X(2)
\end{align}
Using Bayes theorem,
\begin{align}
&= p_Y\brak{0} \times \pr{Y=0 | X=1} + p_Y\brak{1} \times \pr{Y=1|X=2}\\
&=\frac{1}{3} \times \frac{6}{25} + \frac{2}{3} \times \frac{5}{50}\\
&=\frac{11}{75}
\end{align}

\newpage

%\tableofcontents

\bigskip

\renewcommand{\thefigure}{\theenumi}
\renewcommand{\thetable}{\theenumi}
%\renewcommand{\theequation}{\theenumi}

%\begin{abstract}
%%\boldmath
%In this letter, an algorithm for evaluating the exact analytical bit error rate  (BER)  for the piecewise linear (PL) combiner for  multiple relays is presented. Previous results were available only for upto three relays. The algorithm is unique in the sense that  the actual mathematical expressions, that are prohibitively large, need not be explicitly obtained. The diversity gain due to multiple relays is shown through plots of the analytical BER, well supported by simulations. 
%
%\end{abstract}
% IEEEtran.cls defaults to using nonbold math in the Abstract.
% This preserves the distinction between vectors and scalars. However,
% if the journal you are submitting to favors bold math in the abstract,
% then you can use LaTeX's standard command \boldmath at the very start
% of the abstract to achieve this. Many IEEE journals frown on math
% in the abstract anyway.

% Note that keywords are not normally used for peerreview papers.
%\begin{IEEEkeywords}
%Cooperative diversity, decode and forward, piecewise linear
%\end{IEEEkeywords}



% For peer review papers, you can put extra information on the cover
% page as needed:
% \ifCLASSOPTIONpeerreview
% \begin{center} \bfseries EDICS Category: 3-BBND \end{center}
% \fi
%
% For peerreview papers, this IEEEtran command inserts a page break and
% creates the second title. It will be ignored for other modes.
%\IEEEpeerreviewmaketitle




	\item  A die is loaded in such a way that each odd number is twice as likely to occur as
each even number. Find $P(G)$, where $G$ is the event that a number greater than
3 occurs on a single roll of the die.
\\
\solution
		%\begin{table}[H]
	\centering
\begin{tabular}{|c|c|c|}
\hline
Random variable &Value &Definition\\ \hline
\multirow{3}{*}{X} &0 &Slips of Rs 1\\
&1 &Slips of Rs 5\\
&2 &Slips of Rs 13\\ \hline
\multirow{2}{*}{Y} &0 &Box A\\
&1 &Box B\\\hline
\end{tabular}
\caption{}
\label{tab:Distribution}
\end{table}
See \tabref{tab:Distribution}.
\begin{align}
p_{Y}\brak{k}= \begin{cases} 
      \frac{1}{3} & {k=0} \\
      \frac{2}{3 }& {k=1} 
   \end{cases}
   \\
p_{Y|X}\brak{0|0} = \frac{19}{25}\, 
p_{Y|X}\brak{0|1} = \frac{6}{25}\,
p_{Y|X}\brak{1|0} = \frac{45}{50}\,
p_{Y|X}\brak{1|2} = \frac{5}{50}
\end{align}
The desired probability is the probability that a slip drawn at random is marked other than Rs 1,
\begin{align}
&=1-p_X\brak{0}\\
&= p_X(1) + p_X(2)
\end{align}
Using Bayes theorem,
\begin{align}
&= p_Y\brak{0} \times \pr{Y=0 | X=1} + p_Y\brak{1} \times \pr{Y=1|X=2}\\
&=\frac{1}{3} \times \frac{6}{25} + \frac{2}{3} \times \frac{5}{50}\\
&=\frac{11}{75}
\end{align}

\newpage

%\tableofcontents

\bigskip

\renewcommand{\thefigure}{\theenumi}
\renewcommand{\thetable}{\theenumi}
%\renewcommand{\theequation}{\theenumi}

%\begin{abstract}
%%\boldmath
%In this letter, an algorithm for evaluating the exact analytical bit error rate  (BER)  for the piecewise linear (PL) combiner for  multiple relays is presented. Previous results were available only for upto three relays. The algorithm is unique in the sense that  the actual mathematical expressions, that are prohibitively large, need not be explicitly obtained. The diversity gain due to multiple relays is shown through plots of the analytical BER, well supported by simulations. 
%
%\end{abstract}
% IEEEtran.cls defaults to using nonbold math in the Abstract.
% This preserves the distinction between vectors and scalars. However,
% if the journal you are submitting to favors bold math in the abstract,
% then you can use LaTeX's standard command \boldmath at the very start
% of the abstract to achieve this. Many IEEE journals frown on math
% in the abstract anyway.

% Note that keywords are not normally used for peerreview papers.
%\begin{IEEEkeywords}
%Cooperative diversity, decode and forward, piecewise linear
%\end{IEEEkeywords}



% For peer review papers, you can put extra information on the cover
% page as needed:
% \ifCLASSOPTIONpeerreview
% \begin{center} \bfseries EDICS Category: 3-BBND \end{center}
% \fi
%
% For peerreview papers, this IEEEtran command inserts a page break and
% creates the second title. It will be ignored for other modes.
%\IEEEpeerreviewmaketitle




	\item All the jacks, queens and kings are removed from a deck of 52 playing cards. The remaining cards are well shuffled and then one card is drawn at random. Giving ace a value 1 similar value for other cards, find the probability that the card has a value 
		\begin{enumerate}
			\item 7
			\item greater than 7
			\item less than 7
		\end{enumerate}
		%Number of cards left after removing all jacks, queens and kings 
\begin{align}
N	= 52 - 4\times 3
	= 40
\end{align}
%\begin{table}[H]
%\def\arraystretch{1.2}
%\begin{tabular}{|c|c|c|}
%\hline
%	\textbf{Parameter} &\textbf{Value} &\textbf{Description}\\ \hline
%	$X$ &1-10 &Represents the value of the card picked \\ \hline
%\end{tabular}
%\end{table}
Let $1 \le X \le 10$ be the value of the card picked.  Then,
\begin{align}
	p_X(k) &= \Pr(X=k)\ \forall\ 1 \leq k \leq 10\\
	&= \frac{4\times 1}{40}\\
	&= \frac{1}{10}\\
	\therefore p_X(k) &= 
	\begin{cases}
		\frac{1}{10} & 1 \leq k \leq 10\\
		0 & \text{otherwise}
	\end{cases}
\end{align}
and
\begin{align}
	F_{X}(k) &= \sum_{m=0}^{k}p_{X}(m) \quad 1 \leq k \leq 10\\
	&= \frac{k}{10}\\
	\therefore F_{X}(k) &= 
	\begin{cases}
		0 & k \leq 0\\
		\frac{k}{10} & 1\leq k \leq 10\\
		1 & k > 10 
	\end{cases}
\end{align}
\begin{enumerate}
	\item Probability that card has value equal to 7 is
		\begin{align}
			 p_{X}(7)
			= \frac{1}{10}
		\end{align}
	\item Probability that card has value greater than 7 is
		\begin{align}
			1 - F_X(7)
			&= 1 - \frac{7}{10}
			\\
			&= \frac{3}{10}
		\end{align}
	\item Probability that card has value less than 7 is
		\begin{align}
			 F_{X}(6)
			=\frac{6}{10}
		\end{align}
\end{enumerate}

  \item A Lot consists of 48 mobile phones of which 42 are good, 3 have only minor defects and 3 have major defects.Varnika will buy a phone if it is good but the trader will only buy a mobile if it has no major defects. One phone is selected at random from the lot. What is the probability that it is
\begin{enumerate}
	\item acceptable to Varnika?
            \item acceptable to the trader?
\end{enumerate}
\solution
	%\begin{table}[H]
	\centering
\begin{tabular}{|c|c|c|}
\hline
Random variable &Value &Definition\\ \hline
\multirow{3}{*}{X} &0 &Slips of Rs 1\\
&1 &Slips of Rs 5\\
&2 &Slips of Rs 13\\ \hline
\multirow{2}{*}{Y} &0 &Box A\\
&1 &Box B\\\hline
\end{tabular}
\caption{}
\label{tab:Distribution}
\end{table}
See \tabref{tab:Distribution}.
\begin{align}
p_{Y}\brak{k}= \begin{cases} 
      \frac{1}{3} & {k=0} \\
      \frac{2}{3 }& {k=1} 
   \end{cases}
   \\
p_{Y|X}\brak{0|0} = \frac{19}{25}\, 
p_{Y|X}\brak{0|1} = \frac{6}{25}\,
p_{Y|X}\brak{1|0} = \frac{45}{50}\,
p_{Y|X}\brak{1|2} = \frac{5}{50}
\end{align}
The desired probability is the probability that a slip drawn at random is marked other than Rs 1,
\begin{align}
&=1-p_X\brak{0}\\
&= p_X(1) + p_X(2)
\end{align}
Using Bayes theorem,
\begin{align}
&= p_Y\brak{0} \times \pr{Y=0 | X=1} + p_Y\brak{1} \times \pr{Y=1|X=2}\\
&=\frac{1}{3} \times \frac{6}{25} + \frac{2}{3} \times \frac{5}{50}\\
&=\frac{11}{75}
\end{align}

\newpage

%\tableofcontents

\bigskip

\renewcommand{\thefigure}{\theenumi}
\renewcommand{\thetable}{\theenumi}
%\renewcommand{\theequation}{\theenumi}

%\begin{abstract}
%%\boldmath
%In this letter, an algorithm for evaluating the exact analytical bit error rate  (BER)  for the piecewise linear (PL) combiner for  multiple relays is presented. Previous results were available only for upto three relays. The algorithm is unique in the sense that  the actual mathematical expressions, that are prohibitively large, need not be explicitly obtained. The diversity gain due to multiple relays is shown through plots of the analytical BER, well supported by simulations. 
%
%\end{abstract}
% IEEEtran.cls defaults to using nonbold math in the Abstract.
% This preserves the distinction between vectors and scalars. However,
% if the journal you are submitting to favors bold math in the abstract,
% then you can use LaTeX's standard command \boldmath at the very start
% of the abstract to achieve this. Many IEEE journals frown on math
% in the abstract anyway.

% Note that keywords are not normally used for peerreview papers.
%\begin{IEEEkeywords}
%Cooperative diversity, decode and forward, piecewise linear
%\end{IEEEkeywords}



% For peer review papers, you can put extra information on the cover
% page as needed:
% \ifCLASSOPTIONpeerreview
% \begin{center} \bfseries EDICS Category: 3-BBND \end{center}
% \fi
%
% For peerreview papers, this IEEEtran command inserts a page break and
% creates the second title. It will be ignored for other modes.
%\IEEEpeerreviewmaketitle




 \item A student says that if you throw a die, it will show up 1 or not 1. Therefore, the probability of getting 1 and the probability of getting 'not 1' each is equal to $\frac{1}{2}$. Is this correct? Give reasons.\\
 \solution
        %\begin{table}[H]
	\centering
\begin{tabular}{|c|c|c|}
\hline
Random variable &Value &Definition\\ \hline
\multirow{3}{*}{X} &0 &Slips of Rs 1\\
&1 &Slips of Rs 5\\
&2 &Slips of Rs 13\\ \hline
\multirow{2}{*}{Y} &0 &Box A\\
&1 &Box B\\\hline
\end{tabular}
\caption{}
\label{tab:Distribution}
\end{table}
See \tabref{tab:Distribution}.
\begin{align}
p_{Y}\brak{k}= \begin{cases} 
      \frac{1}{3} & {k=0} \\
      \frac{2}{3 }& {k=1} 
   \end{cases}
   \\
p_{Y|X}\brak{0|0} = \frac{19}{25}\, 
p_{Y|X}\brak{0|1} = \frac{6}{25}\,
p_{Y|X}\brak{1|0} = \frac{45}{50}\,
p_{Y|X}\brak{1|2} = \frac{5}{50}
\end{align}
The desired probability is the probability that a slip drawn at random is marked other than Rs 1,
\begin{align}
&=1-p_X\brak{0}\\
&= p_X(1) + p_X(2)
\end{align}
Using Bayes theorem,
\begin{align}
&= p_Y\brak{0} \times \pr{Y=0 | X=1} + p_Y\brak{1} \times \pr{Y=1|X=2}\\
&=\frac{1}{3} \times \frac{6}{25} + \frac{2}{3} \times \frac{5}{50}\\
&=\frac{11}{75}
\end{align}

\newpage

%\tableofcontents

\bigskip

\renewcommand{\thefigure}{\theenumi}
\renewcommand{\thetable}{\theenumi}
%\renewcommand{\theequation}{\theenumi}

%\begin{abstract}
%%\boldmath
%In this letter, an algorithm for evaluating the exact analytical bit error rate  (BER)  for the piecewise linear (PL) combiner for  multiple relays is presented. Previous results were available only for upto three relays. The algorithm is unique in the sense that  the actual mathematical expressions, that are prohibitively large, need not be explicitly obtained. The diversity gain due to multiple relays is shown through plots of the analytical BER, well supported by simulations. 
%
%\end{abstract}
% IEEEtran.cls defaults to using nonbold math in the Abstract.
% This preserves the distinction between vectors and scalars. However,
% if the journal you are submitting to favors bold math in the abstract,
% then you can use LaTeX's standard command \boldmath at the very start
% of the abstract to achieve this. Many IEEE journals frown on math
% in the abstract anyway.

% Note that keywords are not normally used for peerreview papers.
%\begin{IEEEkeywords}
%Cooperative diversity, decode and forward, piecewise linear
%\end{IEEEkeywords}



% For peer review papers, you can put extra information on the cover
% page as needed:
% \ifCLASSOPTIONpeerreview
% \begin{center} \bfseries EDICS Category: 3-BBND \end{center}
% \fi
%
% For peerreview papers, this IEEEtran command inserts a page break and
% creates the second title. It will be ignored for other modes.
%\IEEEpeerreviewmaketitle




   \item Four candidates A, B, C, D have ap-
plied for the assignment to coach a school cricket
team. If A is twice as likely to be selected as B, and
B and C are given about the same chance of being
selected, while C is twice as likely to be selected
as D, what are the probabilities that
\begin{enumerate}
\item C will be selected?
\item A will not be selected?
\end{enumerate}
	%\begin{table}[H]
	\centering
\begin{tabular}{|c|c|c|}
\hline
Random variable &Value &Definition\\ \hline
\multirow{3}{*}{X} &0 &Slips of Rs 1\\
&1 &Slips of Rs 5\\
&2 &Slips of Rs 13\\ \hline
\multirow{2}{*}{Y} &0 &Box A\\
&1 &Box B\\\hline
\end{tabular}
\caption{}
\label{tab:Distribution}
\end{table}
See \tabref{tab:Distribution}.
\begin{align}
p_{Y}\brak{k}= \begin{cases} 
      \frac{1}{3} & {k=0} \\
      \frac{2}{3 }& {k=1} 
   \end{cases}
   \\
p_{Y|X}\brak{0|0} = \frac{19}{25}\, 
p_{Y|X}\brak{0|1} = \frac{6}{25}\,
p_{Y|X}\brak{1|0} = \frac{45}{50}\,
p_{Y|X}\brak{1|2} = \frac{5}{50}
\end{align}
The desired probability is the probability that a slip drawn at random is marked other than Rs 1,
\begin{align}
&=1-p_X\brak{0}\\
&= p_X(1) + p_X(2)
\end{align}
Using Bayes theorem,
\begin{align}
&= p_Y\brak{0} \times \pr{Y=0 | X=1} + p_Y\brak{1} \times \pr{Y=1|X=2}\\
&=\frac{1}{3} \times \frac{6}{25} + \frac{2}{3} \times \frac{5}{50}\\
&=\frac{11}{75}
\end{align}

\newpage

%\tableofcontents

\bigskip

\renewcommand{\thefigure}{\theenumi}
\renewcommand{\thetable}{\theenumi}
%\renewcommand{\theequation}{\theenumi}

%\begin{abstract}
%%\boldmath
%In this letter, an algorithm for evaluating the exact analytical bit error rate  (BER)  for the piecewise linear (PL) combiner for  multiple relays is presented. Previous results were available only for upto three relays. The algorithm is unique in the sense that  the actual mathematical expressions, that are prohibitively large, need not be explicitly obtained. The diversity gain due to multiple relays is shown through plots of the analytical BER, well supported by simulations. 
%
%\end{abstract}
% IEEEtran.cls defaults to using nonbold math in the Abstract.
% This preserves the distinction between vectors and scalars. However,
% if the journal you are submitting to favors bold math in the abstract,
% then you can use LaTeX's standard command \boldmath at the very start
% of the abstract to achieve this. Many IEEE journals frown on math
% in the abstract anyway.

% Note that keywords are not normally used for peerreview papers.
%\begin{IEEEkeywords}
%Cooperative diversity, decode and forward, piecewise linear
%\end{IEEEkeywords}



% For peer review papers, you can put extra information on the cover
% page as needed:
% \ifCLASSOPTIONpeerreview
% \begin{center} \bfseries EDICS Category: 3-BBND \end{center}
% \fi
%
% For peerreview papers, this IEEEtran command inserts a page break and
% creates the second title. It will be ignored for other modes.
%\IEEEpeerreviewmaketitle




 \item A bag contain 24 balls of which $x$ balls are red, $2x$ are white and $3x$ are blue. A ball is selected at random, What is the probability that it is
\begin{enumerate}[label=\alph*)]
\item not red ?
\item white ?
\end{enumerate}
%\begin{table}[H]
	\centering
\begin{tabular}{|c|c|c|}
\hline
Random variable &Value &Definition\\ \hline
\multirow{3}{*}{X} &0 &Slips of Rs 1\\
&1 &Slips of Rs 5\\
&2 &Slips of Rs 13\\ \hline
\multirow{2}{*}{Y} &0 &Box A\\
&1 &Box B\\\hline
\end{tabular}
\caption{}
\label{tab:Distribution}
\end{table}
See \tabref{tab:Distribution}.
\begin{align}
p_{Y}\brak{k}= \begin{cases} 
      \frac{1}{3} & {k=0} \\
      \frac{2}{3 }& {k=1} 
   \end{cases}
   \\
p_{Y|X}\brak{0|0} = \frac{19}{25}\, 
p_{Y|X}\brak{0|1} = \frac{6}{25}\,
p_{Y|X}\brak{1|0} = \frac{45}{50}\,
p_{Y|X}\brak{1|2} = \frac{5}{50}
\end{align}
The desired probability is the probability that a slip drawn at random is marked other than Rs 1,
\begin{align}
&=1-p_X\brak{0}\\
&= p_X(1) + p_X(2)
\end{align}
Using Bayes theorem,
\begin{align}
&= p_Y\brak{0} \times \pr{Y=0 | X=1} + p_Y\brak{1} \times \pr{Y=1|X=2}\\
&=\frac{1}{3} \times \frac{6}{25} + \frac{2}{3} \times \frac{5}{50}\\
&=\frac{11}{75}
\end{align}

\newpage

%\tableofcontents

\bigskip

\renewcommand{\thefigure}{\theenumi}
\renewcommand{\thetable}{\theenumi}
%\renewcommand{\theequation}{\theenumi}

%\begin{abstract}
%%\boldmath
%In this letter, an algorithm for evaluating the exact analytical bit error rate  (BER)  for the piecewise linear (PL) combiner for  multiple relays is presented. Previous results were available only for upto three relays. The algorithm is unique in the sense that  the actual mathematical expressions, that are prohibitively large, need not be explicitly obtained. The diversity gain due to multiple relays is shown through plots of the analytical BER, well supported by simulations. 
%
%\end{abstract}
% IEEEtran.cls defaults to using nonbold math in the Abstract.
% This preserves the distinction between vectors and scalars. However,
% if the journal you are submitting to favors bold math in the abstract,
% then you can use LaTeX's standard command \boldmath at the very start
% of the abstract to achieve this. Many IEEE journals frown on math
% in the abstract anyway.

% Note that keywords are not normally used for peerreview papers.
%\begin{IEEEkeywords}
%Cooperative diversity, decode and forward, piecewise linear
%\end{IEEEkeywords}



% For peer review papers, you can put extra information on the cover
% page as needed:
% \ifCLASSOPTIONpeerreview
% \begin{center} \bfseries EDICS Category: 3-BBND \end{center}
% \fi
%
% For peerreview papers, this IEEEtran command inserts a page break and
% creates the second title. It will be ignored for other modes.
%\IEEEpeerreviewmaketitle




If the letters of the word ASSASSINATION are arranged at random. Find the Probability that
\begin{enumerate}[label=(\alph*)]
\item Four $S's$ come consecutively in the word
\item Two  $I's$ and two $N's$ come together
\item All $A's$ are not coming together
\item No two $A's$ are coming together
\end{enumerate}
%\begin{table}[H]
	\centering
\begin{tabular}{|c|c|c|}
\hline
Random variable &Value &Definition\\ \hline
\multirow{3}{*}{X} &0 &Slips of Rs 1\\
&1 &Slips of Rs 5\\
&2 &Slips of Rs 13\\ \hline
\multirow{2}{*}{Y} &0 &Box A\\
&1 &Box B\\\hline
\end{tabular}
\caption{}
\label{tab:Distribution}
\end{table}
See \tabref{tab:Distribution}.
\begin{align}
p_{Y}\brak{k}= \begin{cases} 
      \frac{1}{3} & {k=0} \\
      \frac{2}{3 }& {k=1} 
   \end{cases}
   \\
p_{Y|X}\brak{0|0} = \frac{19}{25}\, 
p_{Y|X}\brak{0|1} = \frac{6}{25}\,
p_{Y|X}\brak{1|0} = \frac{45}{50}\,
p_{Y|X}\brak{1|2} = \frac{5}{50}
\end{align}
The desired probability is the probability that a slip drawn at random is marked other than Rs 1,
\begin{align}
&=1-p_X\brak{0}\\
&= p_X(1) + p_X(2)
\end{align}
Using Bayes theorem,
\begin{align}
&= p_Y\brak{0} \times \pr{Y=0 | X=1} + p_Y\brak{1} \times \pr{Y=1|X=2}\\
&=\frac{1}{3} \times \frac{6}{25} + \frac{2}{3} \times \frac{5}{50}\\
&=\frac{11}{75}
\end{align}

\newpage

%\tableofcontents

\bigskip

\renewcommand{\thefigure}{\theenumi}
\renewcommand{\thetable}{\theenumi}
%\renewcommand{\theequation}{\theenumi}

%\begin{abstract}
%%\boldmath
%In this letter, an algorithm for evaluating the exact analytical bit error rate  (BER)  for the piecewise linear (PL) combiner for  multiple relays is presented. Previous results were available only for upto three relays. The algorithm is unique in the sense that  the actual mathematical expressions, that are prohibitively large, need not be explicitly obtained. The diversity gain due to multiple relays is shown through plots of the analytical BER, well supported by simulations. 
%
%\end{abstract}
% IEEEtran.cls defaults to using nonbold math in the Abstract.
% This preserves the distinction between vectors and scalars. However,
% if the journal you are submitting to favors bold math in the abstract,
% then you can use LaTeX's standard command \boldmath at the very start
% of the abstract to achieve this. Many IEEE journals frown on math
% in the abstract anyway.

% Note that keywords are not normally used for peerreview papers.
%\begin{IEEEkeywords}
%Cooperative diversity, decode and forward, piecewise linear
%\end{IEEEkeywords}



% For peer review papers, you can put extra information on the cover
% page as needed:
% \ifCLASSOPTIONpeerreview
% \begin{center} \bfseries EDICS Category: 3-BBND \end{center}
% \fi
%
% For peerreview papers, this IEEEtran command inserts a page break and
% creates the second title. It will be ignored for other modes.
%\IEEEpeerreviewmaketitle




	\item One urn contains two black balls (labelled B1 and B2) and one white ball. A
	second urn contains one black ball and two white balls (labelled W1 and W2).
	Suppose the following experiment is performed. One of the two urns is chosen
	at random. Next a ball is randomly chosen from the urn. Then a second ball is
	chosen at random from the same urn without replacing the first ball.
	
	\begin{enumerate}
	\item What is the probability that two black balls are chosen?
	
	\item What is the probability that two balls of opposite colour are chosen?
	\end{enumerate}
	\solution
	%\begin{align}
    \label{eq:12.13.6.18.1}
	\because	\pr{A|B} &> \pr{A},\
\frac{\pr{AB}}{\pr{B}} > \pr{A}
\\
    \label{eq:12.13.6.18.2}
	\implies \pr{AB} &> \pr{A}\pr{B}
	\\
	\text{or, } \frac{\pr{AB}}{\pr{A}} &=\pr{B|A} > \pr{A}
\end{align}

\end{enumerate}

	\item A bag contains 4 red and 4 black balls, another bag contains 2 red and 6 black balls. One of the two bags is selected at random and a ball is drawn from the bag which is found to be red. Find the probability that the ball is drawn from the first bag.
\\
\solution
		%\begin{table}[H]
	\centering
\begin{tabular}{|c|c|c|}
\hline
Random variable &Value &Definition\\ \hline
\multirow{3}{*}{X} &0 &Slips of Rs 1\\
&1 &Slips of Rs 5\\
&2 &Slips of Rs 13\\ \hline
\multirow{2}{*}{Y} &0 &Box A\\
&1 &Box B\\\hline
\end{tabular}
\caption{}
\label{tab:Distribution}
\end{table}
See \tabref{tab:Distribution}.
\begin{align}
p_{Y}\brak{k}= \begin{cases} 
      \frac{1}{3} & {k=0} \\
      \frac{2}{3 }& {k=1} 
   \end{cases}
   \\
p_{Y|X}\brak{0|0} = \frac{19}{25}\, 
p_{Y|X}\brak{0|1} = \frac{6}{25}\,
p_{Y|X}\brak{1|0} = \frac{45}{50}\,
p_{Y|X}\brak{1|2} = \frac{5}{50}
\end{align}
The desired probability is the probability that a slip drawn at random is marked other than Rs 1,
\begin{align}
&=1-p_X\brak{0}\\
&= p_X(1) + p_X(2)
\end{align}
Using Bayes theorem,
\begin{align}
&= p_Y\brak{0} \times \pr{Y=0 | X=1} + p_Y\brak{1} \times \pr{Y=1|X=2}\\
&=\frac{1}{3} \times \frac{6}{25} + \frac{2}{3} \times \frac{5}{50}\\
&=\frac{11}{75}
\end{align}

\newpage

%\tableofcontents

\bigskip

\renewcommand{\thefigure}{\theenumi}
\renewcommand{\thetable}{\theenumi}
%\renewcommand{\theequation}{\theenumi}

%\begin{abstract}
%%\boldmath
%In this letter, an algorithm for evaluating the exact analytical bit error rate  (BER)  for the piecewise linear (PL) combiner for  multiple relays is presented. Previous results were available only for upto three relays. The algorithm is unique in the sense that  the actual mathematical expressions, that are prohibitively large, need not be explicitly obtained. The diversity gain due to multiple relays is shown through plots of the analytical BER, well supported by simulations. 
%
%\end{abstract}
% IEEEtran.cls defaults to using nonbold math in the Abstract.
% This preserves the distinction between vectors and scalars. However,
% if the journal you are submitting to favors bold math in the abstract,
% then you can use LaTeX's standard command \boldmath at the very start
% of the abstract to achieve this. Many IEEE journals frown on math
% in the abstract anyway.

% Note that keywords are not normally used for peerreview papers.
%\begin{IEEEkeywords}
%Cooperative diversity, decode and forward, piecewise linear
%\end{IEEEkeywords}



% For peer review papers, you can put extra information on the cover
% page as needed:
% \ifCLASSOPTIONpeerreview
% \begin{center} \bfseries EDICS Category: 3-BBND \end{center}
% \fi
%
% For peerreview papers, this IEEEtran command inserts a page break and
% creates the second title. It will be ignored for other modes.
%\IEEEpeerreviewmaketitle




  \item
  Cards with numbers 2 to 101 are placed in a box. A card is selected at random.Find the probability that the card has
\begin{enumerate}[label=(\roman*)]
	\item an even number 
	\item a square number
\end{enumerate}
\solution
%\begin{table}[H]
	\centering
\begin{tabular}{|c|c|c|}
\hline
Random variable &Value &Definition\\ \hline
\multirow{3}{*}{X} &0 &Slips of Rs 1\\
&1 &Slips of Rs 5\\
&2 &Slips of Rs 13\\ \hline
\multirow{2}{*}{Y} &0 &Box A\\
&1 &Box B\\\hline
\end{tabular}
\caption{}
\label{tab:Distribution}
\end{table}
See \tabref{tab:Distribution}.
\begin{align}
p_{Y}\brak{k}= \begin{cases} 
      \frac{1}{3} & {k=0} \\
      \frac{2}{3 }& {k=1} 
   \end{cases}
   \\
p_{Y|X}\brak{0|0} = \frac{19}{25}\, 
p_{Y|X}\brak{0|1} = \frac{6}{25}\,
p_{Y|X}\brak{1|0} = \frac{45}{50}\,
p_{Y|X}\brak{1|2} = \frac{5}{50}
\end{align}
The desired probability is the probability that a slip drawn at random is marked other than Rs 1,
\begin{align}
&=1-p_X\brak{0}\\
&= p_X(1) + p_X(2)
\end{align}
Using Bayes theorem,
\begin{align}
&= p_Y\brak{0} \times \pr{Y=0 | X=1} + p_Y\brak{1} \times \pr{Y=1|X=2}\\
&=\frac{1}{3} \times \frac{6}{25} + \frac{2}{3} \times \frac{5}{50}\\
&=\frac{11}{75}
\end{align}

\newpage

%\tableofcontents

\bigskip

\renewcommand{\thefigure}{\theenumi}
\renewcommand{\thetable}{\theenumi}
%\renewcommand{\theequation}{\theenumi}

%\begin{abstract}
%%\boldmath
%In this letter, an algorithm for evaluating the exact analytical bit error rate  (BER)  for the piecewise linear (PL) combiner for  multiple relays is presented. Previous results were available only for upto three relays. The algorithm is unique in the sense that  the actual mathematical expressions, that are prohibitively large, need not be explicitly obtained. The diversity gain due to multiple relays is shown through plots of the analytical BER, well supported by simulations. 
%
%\end{abstract}
% IEEEtran.cls defaults to using nonbold math in the Abstract.
% This preserves the distinction between vectors and scalars. However,
% if the journal you are submitting to favors bold math in the abstract,
% then you can use LaTeX's standard command \boldmath at the very start
% of the abstract to achieve this. Many IEEE journals frown on math
% in the abstract anyway.

% Note that keywords are not normally used for peerreview papers.
%\begin{IEEEkeywords}
%Cooperative diversity, decode and forward, piecewise linear
%\end{IEEEkeywords}



% For peer review papers, you can put extra information on the cover
% page as needed:
% \ifCLASSOPTIONpeerreview
% \begin{center} \bfseries EDICS Category: 3-BBND \end{center}
% \fi
%
% For peerreview papers, this IEEEtran command inserts a page break and
% creates the second title. It will be ignored for other modes.
%\IEEEpeerreviewmaketitle




\item
The king, queen and jack of clubs are removed from a deck of 52 playing cards and then well shuffled. Now one card is drawn at random from the remaining cards.  Determine the probability that the card is
\begin{enumerate}[label=(\roman*)]
\item a club
\item 10 of hearts
\end{enumerate}
\solution
%\begin{table}[H]
	\centering
\begin{tabular}{|c|c|c|}
\hline
Random variable &Value &Definition\\ \hline
\multirow{3}{*}{X} &0 &Slips of Rs 1\\
&1 &Slips of Rs 5\\
&2 &Slips of Rs 13\\ \hline
\multirow{2}{*}{Y} &0 &Box A\\
&1 &Box B\\\hline
\end{tabular}
\caption{}
\label{tab:Distribution}
\end{table}
See \tabref{tab:Distribution}.
\begin{align}
p_{Y}\brak{k}= \begin{cases} 
      \frac{1}{3} & {k=0} \\
      \frac{2}{3 }& {k=1} 
   \end{cases}
   \\
p_{Y|X}\brak{0|0} = \frac{19}{25}\, 
p_{Y|X}\brak{0|1} = \frac{6}{25}\,
p_{Y|X}\brak{1|0} = \frac{45}{50}\,
p_{Y|X}\brak{1|2} = \frac{5}{50}
\end{align}
The desired probability is the probability that a slip drawn at random is marked other than Rs 1,
\begin{align}
&=1-p_X\brak{0}\\
&= p_X(1) + p_X(2)
\end{align}
Using Bayes theorem,
\begin{align}
&= p_Y\brak{0} \times \pr{Y=0 | X=1} + p_Y\brak{1} \times \pr{Y=1|X=2}\\
&=\frac{1}{3} \times \frac{6}{25} + \frac{2}{3} \times \frac{5}{50}\\
&=\frac{11}{75}
\end{align}

\newpage

%\tableofcontents

\bigskip

\renewcommand{\thefigure}{\theenumi}
\renewcommand{\thetable}{\theenumi}
%\renewcommand{\theequation}{\theenumi}

%\begin{abstract}
%%\boldmath
%In this letter, an algorithm for evaluating the exact analytical bit error rate  (BER)  for the piecewise linear (PL) combiner for  multiple relays is presented. Previous results were available only for upto three relays. The algorithm is unique in the sense that  the actual mathematical expressions, that are prohibitively large, need not be explicitly obtained. The diversity gain due to multiple relays is shown through plots of the analytical BER, well supported by simulations. 
%
%\end{abstract}
% IEEEtran.cls defaults to using nonbold math in the Abstract.
% This preserves the distinction between vectors and scalars. However,
% if the journal you are submitting to favors bold math in the abstract,
% then you can use LaTeX's standard command \boldmath at the very start
% of the abstract to achieve this. Many IEEE journals frown on math
% in the abstract anyway.

% Note that keywords are not normally used for peerreview papers.
%\begin{IEEEkeywords}
%Cooperative diversity, decode and forward, piecewise linear
%\end{IEEEkeywords}



% For peer review papers, you can put extra information on the cover
% page as needed:
% \ifCLASSOPTIONpeerreview
% \begin{center} \bfseries EDICS Category: 3-BBND \end{center}
% \fi
%
% For peerreview papers, this IEEEtran command inserts a page break and
% creates the second title. It will be ignored for other modes.
%\IEEEpeerreviewmaketitle




\item A team of medical students doing their internship have to assist during surgeries
at a city hospital. The probabilities of surgeries rated as very complex, complex,
routine, simple or very simple are respectively, 0.15, 0.20, 0.31, 0.26, .08. Find
the probabilities that a particular surgery will be rated
\begin{enumerate}
	\item complex or very complex;
	\item neither very complex nor very simple;
	\item routine or complex
	\item routine or simple
\end{enumerate}
\solution
%\begin{table}[H]
	\centering
\begin{tabular}{|c|c|c|}
\hline
Random variable &Value &Definition\\ \hline
\multirow{3}{*}{X} &0 &Slips of Rs 1\\
&1 &Slips of Rs 5\\
&2 &Slips of Rs 13\\ \hline
\multirow{2}{*}{Y} &0 &Box A\\
&1 &Box B\\\hline
\end{tabular}
\caption{}
\label{tab:Distribution}
\end{table}
See \tabref{tab:Distribution}.
\begin{align}
p_{Y}\brak{k}= \begin{cases} 
      \frac{1}{3} & {k=0} \\
      \frac{2}{3 }& {k=1} 
   \end{cases}
   \\
p_{Y|X}\brak{0|0} = \frac{19}{25}\, 
p_{Y|X}\brak{0|1} = \frac{6}{25}\,
p_{Y|X}\brak{1|0} = \frac{45}{50}\,
p_{Y|X}\brak{1|2} = \frac{5}{50}
\end{align}
The desired probability is the probability that a slip drawn at random is marked other than Rs 1,
\begin{align}
&=1-p_X\brak{0}\\
&= p_X(1) + p_X(2)
\end{align}
Using Bayes theorem,
\begin{align}
&= p_Y\brak{0} \times \pr{Y=0 | X=1} + p_Y\brak{1} \times \pr{Y=1|X=2}\\
&=\frac{1}{3} \times \frac{6}{25} + \frac{2}{3} \times \frac{5}{50}\\
&=\frac{11}{75}
\end{align}

\newpage

%\tableofcontents

\bigskip

\renewcommand{\thefigure}{\theenumi}
\renewcommand{\thetable}{\theenumi}
%\renewcommand{\theequation}{\theenumi}

%\begin{abstract}
%%\boldmath
%In this letter, an algorithm for evaluating the exact analytical bit error rate  (BER)  for the piecewise linear (PL) combiner for  multiple relays is presented. Previous results were available only for upto three relays. The algorithm is unique in the sense that  the actual mathematical expressions, that are prohibitively large, need not be explicitly obtained. The diversity gain due to multiple relays is shown through plots of the analytical BER, well supported by simulations. 
%
%\end{abstract}
% IEEEtran.cls defaults to using nonbold math in the Abstract.
% This preserves the distinction between vectors and scalars. However,
% if the journal you are submitting to favors bold math in the abstract,
% then you can use LaTeX's standard command \boldmath at the very start
% of the abstract to achieve this. Many IEEE journals frown on math
% in the abstract anyway.

% Note that keywords are not normally used for peerreview papers.
%\begin{IEEEkeywords}
%Cooperative diversity, decode and forward, piecewise linear
%\end{IEEEkeywords}



% For peer review papers, you can put extra information on the cover
% page as needed:
% \ifCLASSOPTIONpeerreview
% \begin{center} \bfseries EDICS Category: 3-BBND \end{center}
% \fi
%
% For peerreview papers, this IEEEtran command inserts a page break and
% creates the second title. It will be ignored for other modes.
%\IEEEpeerreviewmaketitle




\item A card is selected from a pack of 52 cards.
\begin{enumerate}[label=(\alph*)]
    \item How many points are there in the sample space?
    \item Calculate the probability that the card is an ace of spades.
    \item Calculate the probability that the card is (i) an ace and (ii) black card.
\end{enumerate}
\solution
%Let $X$ be an bernoulli rv defined as in \tabref{tab:exemplar/11/16/3/26}.  Then, 
\begin{equation}
    p =
        \frac{4}{11} 
\end{equation}
\begin{table}[H]
	\centering
	\input{exemplar/11/16/3/26/tables/Table2.tex}
	\caption{}
        \label{tab:exemplar/11/16/3/26}
\end{table}

\item The probability that a non leap year selected at random will contain 53 sundays.
\\
\solution
%\begin{table}[H]
	\centering
\begin{tabular}{|c|c|c|}
\hline
Random variable &Value &Definition\\ \hline
\multirow{3}{*}{X} &0 &Slips of Rs 1\\
&1 &Slips of Rs 5\\
&2 &Slips of Rs 13\\ \hline
\multirow{2}{*}{Y} &0 &Box A\\
&1 &Box B\\\hline
\end{tabular}
\caption{}
\label{tab:Distribution}
\end{table}
See \tabref{tab:Distribution}.
\begin{align}
p_{Y}\brak{k}= \begin{cases} 
      \frac{1}{3} & {k=0} \\
      \frac{2}{3 }& {k=1} 
   \end{cases}
   \\
p_{Y|X}\brak{0|0} = \frac{19}{25}\, 
p_{Y|X}\brak{0|1} = \frac{6}{25}\,
p_{Y|X}\brak{1|0} = \frac{45}{50}\,
p_{Y|X}\brak{1|2} = \frac{5}{50}
\end{align}
The desired probability is the probability that a slip drawn at random is marked other than Rs 1,
\begin{align}
&=1-p_X\brak{0}\\
&= p_X(1) + p_X(2)
\end{align}
Using Bayes theorem,
\begin{align}
&= p_Y\brak{0} \times \pr{Y=0 | X=1} + p_Y\brak{1} \times \pr{Y=1|X=2}\\
&=\frac{1}{3} \times \frac{6}{25} + \frac{2}{3} \times \frac{5}{50}\\
&=\frac{11}{75}
\end{align}

\newpage

%\tableofcontents

\bigskip

\renewcommand{\thefigure}{\theenumi}
\renewcommand{\thetable}{\theenumi}
%\renewcommand{\theequation}{\theenumi}

%\begin{abstract}
%%\boldmath
%In this letter, an algorithm for evaluating the exact analytical bit error rate  (BER)  for the piecewise linear (PL) combiner for  multiple relays is presented. Previous results were available only for upto three relays. The algorithm is unique in the sense that  the actual mathematical expressions, that are prohibitively large, need not be explicitly obtained. The diversity gain due to multiple relays is shown through plots of the analytical BER, well supported by simulations. 
%
%\end{abstract}
% IEEEtran.cls defaults to using nonbold math in the Abstract.
% This preserves the distinction between vectors and scalars. However,
% if the journal you are submitting to favors bold math in the abstract,
% then you can use LaTeX's standard command \boldmath at the very start
% of the abstract to achieve this. Many IEEE journals frown on math
% in the abstract anyway.

% Note that keywords are not normally used for peerreview papers.
%\begin{IEEEkeywords}
%Cooperative diversity, decode and forward, piecewise linear
%\end{IEEEkeywords}



% For peer review papers, you can put extra information on the cover
% page as needed:
% \ifCLASSOPTIONpeerreview
% \begin{center} \bfseries EDICS Category: 3-BBND \end{center}
% \fi
%
% For peerreview papers, this IEEEtran command inserts a page break and
% creates the second title. It will be ignored for other modes.
%\IEEEpeerreviewmaketitle




\item One of the four persons John, Rita, Aslam or Gurpreet will be promoted next
month. Consequently the sample space consists of four elementary outcomes
S = {John promoted, Rita promoted, Aslam promoted, Gurpreet promoted}
You are told that the chances of John’s promotion is same as that of Gurpreet,
Rita’s chances of promotion are twice as likely as Johns. Aslam’s chances are
four times that of John.
\begin{enumerate}
	\item Determine
	\begin{enumerate}
		\item P (John promoted)
		\item P (Rita promoted)
		\item P (Aslam promoted)
		\item P (Gurpreet promoted)
	\end{enumerate}
	\item If A = {John promoted or Gurpreet promoted}, find P (A).
\end{enumerate}
\solution
%\begin{table}[H]
	\centering
\begin{tabular}{|c|c|c|}
\hline
Random variable &Value &Definition\\ \hline
\multirow{3}{*}{X} &0 &Slips of Rs 1\\
&1 &Slips of Rs 5\\
&2 &Slips of Rs 13\\ \hline
\multirow{2}{*}{Y} &0 &Box A\\
&1 &Box B\\\hline
\end{tabular}
\caption{}
\label{tab:Distribution}
\end{table}
See \tabref{tab:Distribution}.
\begin{align}
p_{Y}\brak{k}= \begin{cases} 
      \frac{1}{3} & {k=0} \\
      \frac{2}{3 }& {k=1} 
   \end{cases}
   \\
p_{Y|X}\brak{0|0} = \frac{19}{25}\, 
p_{Y|X}\brak{0|1} = \frac{6}{25}\,
p_{Y|X}\brak{1|0} = \frac{45}{50}\,
p_{Y|X}\brak{1|2} = \frac{5}{50}
\end{align}
The desired probability is the probability that a slip drawn at random is marked other than Rs 1,
\begin{align}
&=1-p_X\brak{0}\\
&= p_X(1) + p_X(2)
\end{align}
Using Bayes theorem,
\begin{align}
&= p_Y\brak{0} \times \pr{Y=0 | X=1} + p_Y\brak{1} \times \pr{Y=1|X=2}\\
&=\frac{1}{3} \times \frac{6}{25} + \frac{2}{3} \times \frac{5}{50}\\
&=\frac{11}{75}
\end{align}

\newpage

%\tableofcontents

\bigskip

\renewcommand{\thefigure}{\theenumi}
\renewcommand{\thetable}{\theenumi}
%\renewcommand{\theequation}{\theenumi}

%\begin{abstract}
%%\boldmath
%In this letter, an algorithm for evaluating the exact analytical bit error rate  (BER)  for the piecewise linear (PL) combiner for  multiple relays is presented. Previous results were available only for upto three relays. The algorithm is unique in the sense that  the actual mathematical expressions, that are prohibitively large, need not be explicitly obtained. The diversity gain due to multiple relays is shown through plots of the analytical BER, well supported by simulations. 
%
%\end{abstract}
% IEEEtran.cls defaults to using nonbold math in the Abstract.
% This preserves the distinction between vectors and scalars. However,
% if the journal you are submitting to favors bold math in the abstract,
% then you can use LaTeX's standard command \boldmath at the very start
% of the abstract to achieve this. Many IEEE journals frown on math
% in the abstract anyway.

% Note that keywords are not normally used for peerreview papers.
%\begin{IEEEkeywords}
%Cooperative diversity, decode and forward, piecewise linear
%\end{IEEEkeywords}



% For peer review papers, you can put extra information on the cover
% page as needed:
% \ifCLASSOPTIONpeerreview
% \begin{center} \bfseries EDICS Category: 3-BBND \end{center}
% \fi
%
% For peerreview papers, this IEEEtran command inserts a page break and
% creates the second title. It will be ignored for other modes.
%\IEEEpeerreviewmaketitle




\item A card is drawn from a deck of 52 cards. Find the probability of getting a king or a heart or a red card.\\
\solution
%\begin{table}[H]
	\centering
\begin{tabular}{|c|c|c|}
\hline
Random variable &Value &Definition\\ \hline
\multirow{3}{*}{X} &0 &Slips of Rs 1\\
&1 &Slips of Rs 5\\
&2 &Slips of Rs 13\\ \hline
\multirow{2}{*}{Y} &0 &Box A\\
&1 &Box B\\\hline
\end{tabular}
\caption{}
\label{tab:Distribution}
\end{table}
See \tabref{tab:Distribution}.
\begin{align}
p_{Y}\brak{k}= \begin{cases} 
      \frac{1}{3} & {k=0} \\
      \frac{2}{3 }& {k=1} 
   \end{cases}
   \\
p_{Y|X}\brak{0|0} = \frac{19}{25}\, 
p_{Y|X}\brak{0|1} = \frac{6}{25}\,
p_{Y|X}\brak{1|0} = \frac{45}{50}\,
p_{Y|X}\brak{1|2} = \frac{5}{50}
\end{align}
The desired probability is the probability that a slip drawn at random is marked other than Rs 1,
\begin{align}
&=1-p_X\brak{0}\\
&= p_X(1) + p_X(2)
\end{align}
Using Bayes theorem,
\begin{align}
&= p_Y\brak{0} \times \pr{Y=0 | X=1} + p_Y\brak{1} \times \pr{Y=1|X=2}\\
&=\frac{1}{3} \times \frac{6}{25} + \frac{2}{3} \times \frac{5}{50}\\
&=\frac{11}{75}
\end{align}

\newpage

%\tableofcontents

\bigskip

\renewcommand{\thefigure}{\theenumi}
\renewcommand{\thetable}{\theenumi}
%\renewcommand{\theequation}{\theenumi}

%\begin{abstract}
%%\boldmath
%In this letter, an algorithm for evaluating the exact analytical bit error rate  (BER)  for the piecewise linear (PL) combiner for  multiple relays is presented. Previous results were available only for upto three relays. The algorithm is unique in the sense that  the actual mathematical expressions, that are prohibitively large, need not be explicitly obtained. The diversity gain due to multiple relays is shown through plots of the analytical BER, well supported by simulations. 
%
%\end{abstract}
% IEEEtran.cls defaults to using nonbold math in the Abstract.
% This preserves the distinction between vectors and scalars. However,
% if the journal you are submitting to favors bold math in the abstract,
% then you can use LaTeX's standard command \boldmath at the very start
% of the abstract to achieve this. Many IEEE journals frown on math
% in the abstract anyway.

% Note that keywords are not normally used for peerreview papers.
%\begin{IEEEkeywords}
%Cooperative diversity, decode and forward, piecewise linear
%\end{IEEEkeywords}



% For peer review papers, you can put extra information on the cover
% page as needed:
% \ifCLASSOPTIONpeerreview
% \begin{center} \bfseries EDICS Category: 3-BBND \end{center}
% \fi
%
% For peerreview papers, this IEEEtran command inserts a page break and
% creates the second title. It will be ignored for other modes.
%\IEEEpeerreviewmaketitle




\item The probability that a student will pass his examination is 0.73, the probability of
the student getting a compartment is 0.13, and the probability that the student will
either pass or get compartment is 0.96. State True or False.\\
\solution
%\begin{table}[H]
	\centering
\begin{tabular}{|c|c|c|}
\hline
Random variable &Value &Definition\\ \hline
\multirow{3}{*}{X} &0 &Slips of Rs 1\\
&1 &Slips of Rs 5\\
&2 &Slips of Rs 13\\ \hline
\multirow{2}{*}{Y} &0 &Box A\\
&1 &Box B\\\hline
\end{tabular}
\caption{}
\label{tab:Distribution}
\end{table}
See \tabref{tab:Distribution}.
\begin{align}
p_{Y}\brak{k}= \begin{cases} 
      \frac{1}{3} & {k=0} \\
      \frac{2}{3 }& {k=1} 
   \end{cases}
   \\
p_{Y|X}\brak{0|0} = \frac{19}{25}\, 
p_{Y|X}\brak{0|1} = \frac{6}{25}\,
p_{Y|X}\brak{1|0} = \frac{45}{50}\,
p_{Y|X}\brak{1|2} = \frac{5}{50}
\end{align}
The desired probability is the probability that a slip drawn at random is marked other than Rs 1,
\begin{align}
&=1-p_X\brak{0}\\
&= p_X(1) + p_X(2)
\end{align}
Using Bayes theorem,
\begin{align}
&= p_Y\brak{0} \times \pr{Y=0 | X=1} + p_Y\brak{1} \times \pr{Y=1|X=2}\\
&=\frac{1}{3} \times \frac{6}{25} + \frac{2}{3} \times \frac{5}{50}\\
&=\frac{11}{75}
\end{align}

\newpage

%\tableofcontents

\bigskip

\renewcommand{\thefigure}{\theenumi}
\renewcommand{\thetable}{\theenumi}
%\renewcommand{\theequation}{\theenumi}

%\begin{abstract}
%%\boldmath
%In this letter, an algorithm for evaluating the exact analytical bit error rate  (BER)  for the piecewise linear (PL) combiner for  multiple relays is presented. Previous results were available only for upto three relays. The algorithm is unique in the sense that  the actual mathematical expressions, that are prohibitively large, need not be explicitly obtained. The diversity gain due to multiple relays is shown through plots of the analytical BER, well supported by simulations. 
%
%\end{abstract}
% IEEEtran.cls defaults to using nonbold math in the Abstract.
% This preserves the distinction between vectors and scalars. However,
% if the journal you are submitting to favors bold math in the abstract,
% then you can use LaTeX's standard command \boldmath at the very start
% of the abstract to achieve this. Many IEEE journals frown on math
% in the abstract anyway.

% Note that keywords are not normally used for peerreview papers.
%\begin{IEEEkeywords}
%Cooperative diversity, decode and forward, piecewise linear
%\end{IEEEkeywords}



% For peer review papers, you can put extra information on the cover
% page as needed:
% \ifCLASSOPTIONpeerreview
% \begin{center} \bfseries EDICS Category: 3-BBND \end{center}
% \fi
%
% For peerreview papers, this IEEEtran command inserts a page break and
% creates the second title. It will be ignored for other modes.
%\IEEEpeerreviewmaketitle




\item A card is selected from a pack of 52 cards\\
\begin{enumerate}[label=(\alph*)]
\item How many points are there in the sample space?
\item Calculate the probability that the cards is an ace of spades.
\item Calculate the probability that the card is (i) an ace (ii)black card.\\
\end{enumerate}
%\input{ncert/11/16/3/4_1/Prob_4.tex}
\item In a non-leap year, the probability of having 53 tuesdays or 53 wednesdays is\\
\solution
%A non-leap year has a total of 365 days, and a week has 7 days.\\
So it can be expressed as 
\begin{align}
365\text{days} &=52\times 7+1 \text{day}
\end{align}
$\implies$ 52 tuesdays or wednesdays\\
Random variable X denotes the days of a week
\begin{align}
p_X\brak{k}&=\frac{1}{7}; \quad \brak{1<k<7}
\end{align}
So the probability of extra day being tuesday or wednesday is
\begin{align}
p_X\brak{3}+p_X\brak{4}&=\frac{1}{7}+\frac{1}{7}=\frac{2}{7}
\end{align}



\item There are 1000 sealed envelopes in a box, 10 of them contain a cash prize of
Rs 100 each, 100 of them contain a cash prize of Rs 50 each and 200 of them
contain a cash prize of Rs 10 each and rest do not contain any cash prize. If they
are well shuffled and an envelope is picked up out, what is the probability that it
contains no cash prize?\\
\solution
%\begin{table}[H]
	\centering
\begin{tabular}{|c|c|c|}
\hline
Random variable &Value &Definition\\ \hline
\multirow{3}{*}{X} &0 &Slips of Rs 1\\
&1 &Slips of Rs 5\\
&2 &Slips of Rs 13\\ \hline
\multirow{2}{*}{Y} &0 &Box A\\
&1 &Box B\\\hline
\end{tabular}
\caption{}
\label{tab:Distribution}
\end{table}
See \tabref{tab:Distribution}.
\begin{align}
p_{Y}\brak{k}= \begin{cases} 
      \frac{1}{3} & {k=0} \\
      \frac{2}{3 }& {k=1} 
   \end{cases}
   \\
p_{Y|X}\brak{0|0} = \frac{19}{25}\, 
p_{Y|X}\brak{0|1} = \frac{6}{25}\,
p_{Y|X}\brak{1|0} = \frac{45}{50}\,
p_{Y|X}\brak{1|2} = \frac{5}{50}
\end{align}
The desired probability is the probability that a slip drawn at random is marked other than Rs 1,
\begin{align}
&=1-p_X\brak{0}\\
&= p_X(1) + p_X(2)
\end{align}
Using Bayes theorem,
\begin{align}
&= p_Y\brak{0} \times \pr{Y=0 | X=1} + p_Y\brak{1} \times \pr{Y=1|X=2}\\
&=\frac{1}{3} \times \frac{6}{25} + \frac{2}{3} \times \frac{5}{50}\\
&=\frac{11}{75}
\end{align}

\newpage

%\tableofcontents

\bigskip

\renewcommand{\thefigure}{\theenumi}
\renewcommand{\thetable}{\theenumi}
%\renewcommand{\theequation}{\theenumi}

%\begin{abstract}
%%\boldmath
%In this letter, an algorithm for evaluating the exact analytical bit error rate  (BER)  for the piecewise linear (PL) combiner for  multiple relays is presented. Previous results were available only for upto three relays. The algorithm is unique in the sense that  the actual mathematical expressions, that are prohibitively large, need not be explicitly obtained. The diversity gain due to multiple relays is shown through plots of the analytical BER, well supported by simulations. 
%
%\end{abstract}
% IEEEtran.cls defaults to using nonbold math in the Abstract.
% This preserves the distinction between vectors and scalars. However,
% if the journal you are submitting to favors bold math in the abstract,
% then you can use LaTeX's standard command \boldmath at the very start
% of the abstract to achieve this. Many IEEE journals frown on math
% in the abstract anyway.

% Note that keywords are not normally used for peerreview papers.
%\begin{IEEEkeywords}
%Cooperative diversity, decode and forward, piecewise linear
%\end{IEEEkeywords}



% For peer review papers, you can put extra information on the cover
% page as needed:
% \ifCLASSOPTIONpeerreview
% \begin{center} \bfseries EDICS Category: 3-BBND \end{center}
% \fi
%
% For peerreview papers, this IEEEtran command inserts a page break and
% creates the second title. It will be ignored for other modes.
%\IEEEpeerreviewmaketitle




\item 
A die is thrown and a card is selected at random from a deck of 52 playing cards. The probability of getting an even number on the die and a spade card.\\
\solution
%\begin{table}[H]
	\centering
\begin{tabular}{|c|c|c|}
\hline
Random variable &Value &Definition\\ \hline
\multirow{3}{*}{X} &0 &Slips of Rs 1\\
&1 &Slips of Rs 5\\
&2 &Slips of Rs 13\\ \hline
\multirow{2}{*}{Y} &0 &Box A\\
&1 &Box B\\\hline
\end{tabular}
\caption{}
\label{tab:Distribution}
\end{table}
See \tabref{tab:Distribution}.
\begin{align}
p_{Y}\brak{k}= \begin{cases} 
      \frac{1}{3} & {k=0} \\
      \frac{2}{3 }& {k=1} 
   \end{cases}
   \\
p_{Y|X}\brak{0|0} = \frac{19}{25}\, 
p_{Y|X}\brak{0|1} = \frac{6}{25}\,
p_{Y|X}\brak{1|0} = \frac{45}{50}\,
p_{Y|X}\brak{1|2} = \frac{5}{50}
\end{align}
The desired probability is the probability that a slip drawn at random is marked other than Rs 1,
\begin{align}
&=1-p_X\brak{0}\\
&= p_X(1) + p_X(2)
\end{align}
Using Bayes theorem,
\begin{align}
&= p_Y\brak{0} \times \pr{Y=0 | X=1} + p_Y\brak{1} \times \pr{Y=1|X=2}\\
&=\frac{1}{3} \times \frac{6}{25} + \frac{2}{3} \times \frac{5}{50}\\
&=\frac{11}{75}
\end{align}

\newpage

%\tableofcontents

\bigskip

\renewcommand{\thefigure}{\theenumi}
\renewcommand{\thetable}{\theenumi}
%\renewcommand{\theequation}{\theenumi}

%\begin{abstract}
%%\boldmath
%In this letter, an algorithm for evaluating the exact analytical bit error rate  (BER)  for the piecewise linear (PL) combiner for  multiple relays is presented. Previous results were available only for upto three relays. The algorithm is unique in the sense that  the actual mathematical expressions, that are prohibitively large, need not be explicitly obtained. The diversity gain due to multiple relays is shown through plots of the analytical BER, well supported by simulations. 
%
%\end{abstract}
% IEEEtran.cls defaults to using nonbold math in the Abstract.
% This preserves the distinction between vectors and scalars. However,
% if the journal you are submitting to favors bold math in the abstract,
% then you can use LaTeX's standard command \boldmath at the very start
% of the abstract to achieve this. Many IEEE journals frown on math
% in the abstract anyway.

% Note that keywords are not normally used for peerreview papers.
%\begin{IEEEkeywords}
%Cooperative diversity, decode and forward, piecewise linear
%\end{IEEEkeywords}



% For peer review papers, you can put extra information on the cover
% page as needed:
% \ifCLASSOPTIONpeerreview
% \begin{center} \bfseries EDICS Category: 3-BBND \end{center}
% \fi
%
% For peerreview papers, this IEEEtran command inserts a page break and
% creates the second title. It will be ignored for other modes.
%\IEEEpeerreviewmaketitle




\item
If 4-digit numbers greater than 5,000 are randomly formed from the digits 0, 1, 3, 5, and 7, what is the probability of forming a number divisible by 5 when:
\begin{enumerate}
    \item The digits are repeated?
    \item The repetition of digits is not allowed?
\end{enumerate}
\solution
%\begin{table}[H]
	\centering
\begin{tabular}{|c|c|c|}
\hline
Random variable &Value &Definition\\ \hline
\multirow{3}{*}{X} &0 &Slips of Rs 1\\
&1 &Slips of Rs 5\\
&2 &Slips of Rs 13\\ \hline
\multirow{2}{*}{Y} &0 &Box A\\
&1 &Box B\\\hline
\end{tabular}
\caption{}
\label{tab:Distribution}
\end{table}
See \tabref{tab:Distribution}.
\begin{align}
p_{Y}\brak{k}= \begin{cases} 
      \frac{1}{3} & {k=0} \\
      \frac{2}{3 }& {k=1} 
   \end{cases}
   \\
p_{Y|X}\brak{0|0} = \frac{19}{25}\, 
p_{Y|X}\brak{0|1} = \frac{6}{25}\,
p_{Y|X}\brak{1|0} = \frac{45}{50}\,
p_{Y|X}\brak{1|2} = \frac{5}{50}
\end{align}
The desired probability is the probability that a slip drawn at random is marked other than Rs 1,
\begin{align}
&=1-p_X\brak{0}\\
&= p_X(1) + p_X(2)
\end{align}
Using Bayes theorem,
\begin{align}
&= p_Y\brak{0} \times \pr{Y=0 | X=1} + p_Y\brak{1} \times \pr{Y=1|X=2}\\
&=\frac{1}{3} \times \frac{6}{25} + \frac{2}{3} \times \frac{5}{50}\\
&=\frac{11}{75}
\end{align}

\newpage

%\tableofcontents

\bigskip

\renewcommand{\thefigure}{\theenumi}
\renewcommand{\thetable}{\theenumi}
%\renewcommand{\theequation}{\theenumi}

%\begin{abstract}
%%\boldmath
%In this letter, an algorithm for evaluating the exact analytical bit error rate  (BER)  for the piecewise linear (PL) combiner for  multiple relays is presented. Previous results were available only for upto three relays. The algorithm is unique in the sense that  the actual mathematical expressions, that are prohibitively large, need not be explicitly obtained. The diversity gain due to multiple relays is shown through plots of the analytical BER, well supported by simulations. 
%
%\end{abstract}
% IEEEtran.cls defaults to using nonbold math in the Abstract.
% This preserves the distinction between vectors and scalars. However,
% if the journal you are submitting to favors bold math in the abstract,
% then you can use LaTeX's standard command \boldmath at the very start
% of the abstract to achieve this. Many IEEE journals frown on math
% in the abstract anyway.

% Note that keywords are not normally used for peerreview papers.
%\begin{IEEEkeywords}
%Cooperative diversity, decode and forward, piecewise linear
%\end{IEEEkeywords}



% For peer review papers, you can put extra information on the cover
% page as needed:
% \ifCLASSOPTIONpeerreview
% \begin{center} \bfseries EDICS Category: 3-BBND \end{center}
% \fi
%
% For peerreview papers, this IEEEtran command inserts a page break and
% creates the second title. It will be ignored for other modes.
%\IEEEpeerreviewmaketitle




\item Consider the probability space $\brak{\Omega, \mathcal{G}, P}$ where $\Omega = [0,2]$ and $\mathcal{G} = \cbrak{\phi, \Omega, [0,1], (1,2]}$. Let $X$ and $Y$ be two functions on $\Omega$ defined as
\begin{align*}
    X(\omega) = 
    \begin{cases}
        1 & \text{if }\omega \in [0, 1]\\
        2 & \text{if }\omega \in (1, 2]
    \end{cases}
\end{align*}
and
\begin{align*}
    Y(\omega) = 
    \begin{cases}
        2 & \text{if }\omega \in [0, 1.5]\\
        3 & \text{if }\omega \in (1.5, 2].
    \end{cases}
\end{align*}
Then which one of the following statements is true?
\begin{enumerate}
    \item [(A)] $X$ is a random variable with respect to $\mathcal{G}$, but $Y$ is not a random variable with respect to $\mathcal{G}$.
    \item [(B)] $Y$ is a random variable with respect to $\mathcal{G}$, but $X$ is not a random variable with respect to $\mathcal{G}$.
    \item [(C)] Neither $X$ nor $Y$ is a random variable with respect to $\mathcal{G}$.
    \item [(D)] Both $X$ and $Y$ are random variables with respect to $\mathcal{G}$.
\end{enumerate} \hfill (GATE ST 2023)\\
\solution
%\begin{table}[H]
	\centering
\begin{tabular}{|c|c|c|}
\hline
Random variable &Value &Definition\\ \hline
\multirow{3}{*}{X} &0 &Slips of Rs 1\\
&1 &Slips of Rs 5\\
&2 &Slips of Rs 13\\ \hline
\multirow{2}{*}{Y} &0 &Box A\\
&1 &Box B\\\hline
\end{tabular}
\caption{}
\label{tab:Distribution}
\end{table}
See \tabref{tab:Distribution}.
\begin{align}
p_{Y}\brak{k}= \begin{cases} 
      \frac{1}{3} & {k=0} \\
      \frac{2}{3 }& {k=1} 
   \end{cases}
   \\
p_{Y|X}\brak{0|0} = \frac{19}{25}\, 
p_{Y|X}\brak{0|1} = \frac{6}{25}\,
p_{Y|X}\brak{1|0} = \frac{45}{50}\,
p_{Y|X}\brak{1|2} = \frac{5}{50}
\end{align}
The desired probability is the probability that a slip drawn at random is marked other than Rs 1,
\begin{align}
&=1-p_X\brak{0}\\
&= p_X(1) + p_X(2)
\end{align}
Using Bayes theorem,
\begin{align}
&= p_Y\brak{0} \times \pr{Y=0 | X=1} + p_Y\brak{1} \times \pr{Y=1|X=2}\\
&=\frac{1}{3} \times \frac{6}{25} + \frac{2}{3} \times \frac{5}{50}\\
&=\frac{11}{75}
\end{align}

\newpage

%\tableofcontents

\bigskip

\renewcommand{\thefigure}{\theenumi}
\renewcommand{\thetable}{\theenumi}
%\renewcommand{\theequation}{\theenumi}

%\begin{abstract}
%%\boldmath
%In this letter, an algorithm for evaluating the exact analytical bit error rate  (BER)  for the piecewise linear (PL) combiner for  multiple relays is presented. Previous results were available only for upto three relays. The algorithm is unique in the sense that  the actual mathematical expressions, that are prohibitively large, need not be explicitly obtained. The diversity gain due to multiple relays is shown through plots of the analytical BER, well supported by simulations. 
%
%\end{abstract}
% IEEEtran.cls defaults to using nonbold math in the Abstract.
% This preserves the distinction between vectors and scalars. However,
% if the journal you are submitting to favors bold math in the abstract,
% then you can use LaTeX's standard command \boldmath at the very start
% of the abstract to achieve this. Many IEEE journals frown on math
% in the abstract anyway.

% Note that keywords are not normally used for peerreview papers.
%\begin{IEEEkeywords}
%Cooperative diversity, decode and forward, piecewise linear
%\end{IEEEkeywords}



% For peer review papers, you can put extra information on the cover
% page as needed:
% \ifCLASSOPTIONpeerreview
% \begin{center} \bfseries EDICS Category: 3-BBND \end{center}
% \fi
%
% For peerreview papers, this IEEEtran command inserts a page break and
% creates the second title. It will be ignored for other modes.
%\IEEEpeerreviewmaketitle




	\item  A die is loaded in such a way that each odd number is twice as likely to occur as
each even number. Find $P(G)$, where $G$ is the event that a number greater than
3 occurs on a single roll of the die.
\\
\solution
		%\begin{table}[H]
	\centering
\begin{tabular}{|c|c|c|}
\hline
Random variable &Value &Definition\\ \hline
\multirow{3}{*}{X} &0 &Slips of Rs 1\\
&1 &Slips of Rs 5\\
&2 &Slips of Rs 13\\ \hline
\multirow{2}{*}{Y} &0 &Box A\\
&1 &Box B\\\hline
\end{tabular}
\caption{}
\label{tab:Distribution}
\end{table}
See \tabref{tab:Distribution}.
\begin{align}
p_{Y}\brak{k}= \begin{cases} 
      \frac{1}{3} & {k=0} \\
      \frac{2}{3 }& {k=1} 
   \end{cases}
   \\
p_{Y|X}\brak{0|0} = \frac{19}{25}\, 
p_{Y|X}\brak{0|1} = \frac{6}{25}\,
p_{Y|X}\brak{1|0} = \frac{45}{50}\,
p_{Y|X}\brak{1|2} = \frac{5}{50}
\end{align}
The desired probability is the probability that a slip drawn at random is marked other than Rs 1,
\begin{align}
&=1-p_X\brak{0}\\
&= p_X(1) + p_X(2)
\end{align}
Using Bayes theorem,
\begin{align}
&= p_Y\brak{0} \times \pr{Y=0 | X=1} + p_Y\brak{1} \times \pr{Y=1|X=2}\\
&=\frac{1}{3} \times \frac{6}{25} + \frac{2}{3} \times \frac{5}{50}\\
&=\frac{11}{75}
\end{align}

\newpage

%\tableofcontents

\bigskip

\renewcommand{\thefigure}{\theenumi}
\renewcommand{\thetable}{\theenumi}
%\renewcommand{\theequation}{\theenumi}

%\begin{abstract}
%%\boldmath
%In this letter, an algorithm for evaluating the exact analytical bit error rate  (BER)  for the piecewise linear (PL) combiner for  multiple relays is presented. Previous results were available only for upto three relays. The algorithm is unique in the sense that  the actual mathematical expressions, that are prohibitively large, need not be explicitly obtained. The diversity gain due to multiple relays is shown through plots of the analytical BER, well supported by simulations. 
%
%\end{abstract}
% IEEEtran.cls defaults to using nonbold math in the Abstract.
% This preserves the distinction between vectors and scalars. However,
% if the journal you are submitting to favors bold math in the abstract,
% then you can use LaTeX's standard command \boldmath at the very start
% of the abstract to achieve this. Many IEEE journals frown on math
% in the abstract anyway.

% Note that keywords are not normally used for peerreview papers.
%\begin{IEEEkeywords}
%Cooperative diversity, decode and forward, piecewise linear
%\end{IEEEkeywords}



% For peer review papers, you can put extra information on the cover
% page as needed:
% \ifCLASSOPTIONpeerreview
% \begin{center} \bfseries EDICS Category: 3-BBND \end{center}
% \fi
%
% For peerreview papers, this IEEEtran command inserts a page break and
% creates the second title. It will be ignored for other modes.
%\IEEEpeerreviewmaketitle




	\item All the jacks, queens and kings are removed from a deck of 52 playing cards. The remaining cards are well shuffled and then one card is drawn at random. Giving ace a value 1 similar value for other cards, find the probability that the card has a value 
		\begin{enumerate}
			\item 7
			\item greater than 7
			\item less than 7
		\end{enumerate}
		%Number of cards left after removing all jacks, queens and kings 
\begin{align}
N	= 52 - 4\times 3
	= 40
\end{align}
%\begin{table}[H]
%\def\arraystretch{1.2}
%\begin{tabular}{|c|c|c|}
%\hline
%	\textbf{Parameter} &\textbf{Value} &\textbf{Description}\\ \hline
%	$X$ &1-10 &Represents the value of the card picked \\ \hline
%\end{tabular}
%\end{table}
Let $1 \le X \le 10$ be the value of the card picked.  Then,
\begin{align}
	p_X(k) &= \Pr(X=k)\ \forall\ 1 \leq k \leq 10\\
	&= \frac{4\times 1}{40}\\
	&= \frac{1}{10}\\
	\therefore p_X(k) &= 
	\begin{cases}
		\frac{1}{10} & 1 \leq k \leq 10\\
		0 & \text{otherwise}
	\end{cases}
\end{align}
and
\begin{align}
	F_{X}(k) &= \sum_{m=0}^{k}p_{X}(m) \quad 1 \leq k \leq 10\\
	&= \frac{k}{10}\\
	\therefore F_{X}(k) &= 
	\begin{cases}
		0 & k \leq 0\\
		\frac{k}{10} & 1\leq k \leq 10\\
		1 & k > 10 
	\end{cases}
\end{align}
\begin{enumerate}
	\item Probability that card has value equal to 7 is
		\begin{align}
			 p_{X}(7)
			= \frac{1}{10}
		\end{align}
	\item Probability that card has value greater than 7 is
		\begin{align}
			1 - F_X(7)
			&= 1 - \frac{7}{10}
			\\
			&= \frac{3}{10}
		\end{align}
	\item Probability that card has value less than 7 is
		\begin{align}
			 F_{X}(6)
			=\frac{6}{10}
		\end{align}
\end{enumerate}

  \item A Lot consists of 48 mobile phones of which 42 are good, 3 have only minor defects and 3 have major defects.Varnika will buy a phone if it is good but the trader will only buy a mobile if it has no major defects. One phone is selected at random from the lot. What is the probability that it is
\begin{enumerate}
	\item acceptable to Varnika?
            \item acceptable to the trader?
\end{enumerate}
\solution
	%\begin{table}[H]
	\centering
\begin{tabular}{|c|c|c|}
\hline
Random variable &Value &Definition\\ \hline
\multirow{3}{*}{X} &0 &Slips of Rs 1\\
&1 &Slips of Rs 5\\
&2 &Slips of Rs 13\\ \hline
\multirow{2}{*}{Y} &0 &Box A\\
&1 &Box B\\\hline
\end{tabular}
\caption{}
\label{tab:Distribution}
\end{table}
See \tabref{tab:Distribution}.
\begin{align}
p_{Y}\brak{k}= \begin{cases} 
      \frac{1}{3} & {k=0} \\
      \frac{2}{3 }& {k=1} 
   \end{cases}
   \\
p_{Y|X}\brak{0|0} = \frac{19}{25}\, 
p_{Y|X}\brak{0|1} = \frac{6}{25}\,
p_{Y|X}\brak{1|0} = \frac{45}{50}\,
p_{Y|X}\brak{1|2} = \frac{5}{50}
\end{align}
The desired probability is the probability that a slip drawn at random is marked other than Rs 1,
\begin{align}
&=1-p_X\brak{0}\\
&= p_X(1) + p_X(2)
\end{align}
Using Bayes theorem,
\begin{align}
&= p_Y\brak{0} \times \pr{Y=0 | X=1} + p_Y\brak{1} \times \pr{Y=1|X=2}\\
&=\frac{1}{3} \times \frac{6}{25} + \frac{2}{3} \times \frac{5}{50}\\
&=\frac{11}{75}
\end{align}

\newpage

%\tableofcontents

\bigskip

\renewcommand{\thefigure}{\theenumi}
\renewcommand{\thetable}{\theenumi}
%\renewcommand{\theequation}{\theenumi}

%\begin{abstract}
%%\boldmath
%In this letter, an algorithm for evaluating the exact analytical bit error rate  (BER)  for the piecewise linear (PL) combiner for  multiple relays is presented. Previous results were available only for upto three relays. The algorithm is unique in the sense that  the actual mathematical expressions, that are prohibitively large, need not be explicitly obtained. The diversity gain due to multiple relays is shown through plots of the analytical BER, well supported by simulations. 
%
%\end{abstract}
% IEEEtran.cls defaults to using nonbold math in the Abstract.
% This preserves the distinction between vectors and scalars. However,
% if the journal you are submitting to favors bold math in the abstract,
% then you can use LaTeX's standard command \boldmath at the very start
% of the abstract to achieve this. Many IEEE journals frown on math
% in the abstract anyway.

% Note that keywords are not normally used for peerreview papers.
%\begin{IEEEkeywords}
%Cooperative diversity, decode and forward, piecewise linear
%\end{IEEEkeywords}



% For peer review papers, you can put extra information on the cover
% page as needed:
% \ifCLASSOPTIONpeerreview
% \begin{center} \bfseries EDICS Category: 3-BBND \end{center}
% \fi
%
% For peerreview papers, this IEEEtran command inserts a page break and
% creates the second title. It will be ignored for other modes.
%\IEEEpeerreviewmaketitle




 \item A student says that if you throw a die, it will show up 1 or not 1. Therefore, the probability of getting 1 and the probability of getting 'not 1' each is equal to $\frac{1}{2}$. Is this correct? Give reasons.\\
 \solution
        %\begin{table}[H]
	\centering
\begin{tabular}{|c|c|c|}
\hline
Random variable &Value &Definition\\ \hline
\multirow{3}{*}{X} &0 &Slips of Rs 1\\
&1 &Slips of Rs 5\\
&2 &Slips of Rs 13\\ \hline
\multirow{2}{*}{Y} &0 &Box A\\
&1 &Box B\\\hline
\end{tabular}
\caption{}
\label{tab:Distribution}
\end{table}
See \tabref{tab:Distribution}.
\begin{align}
p_{Y}\brak{k}= \begin{cases} 
      \frac{1}{3} & {k=0} \\
      \frac{2}{3 }& {k=1} 
   \end{cases}
   \\
p_{Y|X}\brak{0|0} = \frac{19}{25}\, 
p_{Y|X}\brak{0|1} = \frac{6}{25}\,
p_{Y|X}\brak{1|0} = \frac{45}{50}\,
p_{Y|X}\brak{1|2} = \frac{5}{50}
\end{align}
The desired probability is the probability that a slip drawn at random is marked other than Rs 1,
\begin{align}
&=1-p_X\brak{0}\\
&= p_X(1) + p_X(2)
\end{align}
Using Bayes theorem,
\begin{align}
&= p_Y\brak{0} \times \pr{Y=0 | X=1} + p_Y\brak{1} \times \pr{Y=1|X=2}\\
&=\frac{1}{3} \times \frac{6}{25} + \frac{2}{3} \times \frac{5}{50}\\
&=\frac{11}{75}
\end{align}

\newpage

%\tableofcontents

\bigskip

\renewcommand{\thefigure}{\theenumi}
\renewcommand{\thetable}{\theenumi}
%\renewcommand{\theequation}{\theenumi}

%\begin{abstract}
%%\boldmath
%In this letter, an algorithm for evaluating the exact analytical bit error rate  (BER)  for the piecewise linear (PL) combiner for  multiple relays is presented. Previous results were available only for upto three relays. The algorithm is unique in the sense that  the actual mathematical expressions, that are prohibitively large, need not be explicitly obtained. The diversity gain due to multiple relays is shown through plots of the analytical BER, well supported by simulations. 
%
%\end{abstract}
% IEEEtran.cls defaults to using nonbold math in the Abstract.
% This preserves the distinction between vectors and scalars. However,
% if the journal you are submitting to favors bold math in the abstract,
% then you can use LaTeX's standard command \boldmath at the very start
% of the abstract to achieve this. Many IEEE journals frown on math
% in the abstract anyway.

% Note that keywords are not normally used for peerreview papers.
%\begin{IEEEkeywords}
%Cooperative diversity, decode and forward, piecewise linear
%\end{IEEEkeywords}



% For peer review papers, you can put extra information on the cover
% page as needed:
% \ifCLASSOPTIONpeerreview
% \begin{center} \bfseries EDICS Category: 3-BBND \end{center}
% \fi
%
% For peerreview papers, this IEEEtran command inserts a page break and
% creates the second title. It will be ignored for other modes.
%\IEEEpeerreviewmaketitle




   \item Four candidates A, B, C, D have ap-
plied for the assignment to coach a school cricket
team. If A is twice as likely to be selected as B, and
B and C are given about the same chance of being
selected, while C is twice as likely to be selected
as D, what are the probabilities that
\begin{enumerate}
\item C will be selected?
\item A will not be selected?
\end{enumerate}
	%\begin{table}[H]
	\centering
\begin{tabular}{|c|c|c|}
\hline
Random variable &Value &Definition\\ \hline
\multirow{3}{*}{X} &0 &Slips of Rs 1\\
&1 &Slips of Rs 5\\
&2 &Slips of Rs 13\\ \hline
\multirow{2}{*}{Y} &0 &Box A\\
&1 &Box B\\\hline
\end{tabular}
\caption{}
\label{tab:Distribution}
\end{table}
See \tabref{tab:Distribution}.
\begin{align}
p_{Y}\brak{k}= \begin{cases} 
      \frac{1}{3} & {k=0} \\
      \frac{2}{3 }& {k=1} 
   \end{cases}
   \\
p_{Y|X}\brak{0|0} = \frac{19}{25}\, 
p_{Y|X}\brak{0|1} = \frac{6}{25}\,
p_{Y|X}\brak{1|0} = \frac{45}{50}\,
p_{Y|X}\brak{1|2} = \frac{5}{50}
\end{align}
The desired probability is the probability that a slip drawn at random is marked other than Rs 1,
\begin{align}
&=1-p_X\brak{0}\\
&= p_X(1) + p_X(2)
\end{align}
Using Bayes theorem,
\begin{align}
&= p_Y\brak{0} \times \pr{Y=0 | X=1} + p_Y\brak{1} \times \pr{Y=1|X=2}\\
&=\frac{1}{3} \times \frac{6}{25} + \frac{2}{3} \times \frac{5}{50}\\
&=\frac{11}{75}
\end{align}

\newpage

%\tableofcontents

\bigskip

\renewcommand{\thefigure}{\theenumi}
\renewcommand{\thetable}{\theenumi}
%\renewcommand{\theequation}{\theenumi}

%\begin{abstract}
%%\boldmath
%In this letter, an algorithm for evaluating the exact analytical bit error rate  (BER)  for the piecewise linear (PL) combiner for  multiple relays is presented. Previous results were available only for upto three relays. The algorithm is unique in the sense that  the actual mathematical expressions, that are prohibitively large, need not be explicitly obtained. The diversity gain due to multiple relays is shown through plots of the analytical BER, well supported by simulations. 
%
%\end{abstract}
% IEEEtran.cls defaults to using nonbold math in the Abstract.
% This preserves the distinction between vectors and scalars. However,
% if the journal you are submitting to favors bold math in the abstract,
% then you can use LaTeX's standard command \boldmath at the very start
% of the abstract to achieve this. Many IEEE journals frown on math
% in the abstract anyway.

% Note that keywords are not normally used for peerreview papers.
%\begin{IEEEkeywords}
%Cooperative diversity, decode and forward, piecewise linear
%\end{IEEEkeywords}



% For peer review papers, you can put extra information on the cover
% page as needed:
% \ifCLASSOPTIONpeerreview
% \begin{center} \bfseries EDICS Category: 3-BBND \end{center}
% \fi
%
% For peerreview papers, this IEEEtran command inserts a page break and
% creates the second title. It will be ignored for other modes.
%\IEEEpeerreviewmaketitle




 \item A bag contain 24 balls of which $x$ balls are red, $2x$ are white and $3x$ are blue. A ball is selected at random, What is the probability that it is
\begin{enumerate}[label=\alph*)]
\item not red ?
\item white ?
\end{enumerate}
%\begin{table}[H]
	\centering
\begin{tabular}{|c|c|c|}
\hline
Random variable &Value &Definition\\ \hline
\multirow{3}{*}{X} &0 &Slips of Rs 1\\
&1 &Slips of Rs 5\\
&2 &Slips of Rs 13\\ \hline
\multirow{2}{*}{Y} &0 &Box A\\
&1 &Box B\\\hline
\end{tabular}
\caption{}
\label{tab:Distribution}
\end{table}
See \tabref{tab:Distribution}.
\begin{align}
p_{Y}\brak{k}= \begin{cases} 
      \frac{1}{3} & {k=0} \\
      \frac{2}{3 }& {k=1} 
   \end{cases}
   \\
p_{Y|X}\brak{0|0} = \frac{19}{25}\, 
p_{Y|X}\brak{0|1} = \frac{6}{25}\,
p_{Y|X}\brak{1|0} = \frac{45}{50}\,
p_{Y|X}\brak{1|2} = \frac{5}{50}
\end{align}
The desired probability is the probability that a slip drawn at random is marked other than Rs 1,
\begin{align}
&=1-p_X\brak{0}\\
&= p_X(1) + p_X(2)
\end{align}
Using Bayes theorem,
\begin{align}
&= p_Y\brak{0} \times \pr{Y=0 | X=1} + p_Y\brak{1} \times \pr{Y=1|X=2}\\
&=\frac{1}{3} \times \frac{6}{25} + \frac{2}{3} \times \frac{5}{50}\\
&=\frac{11}{75}
\end{align}

\newpage

%\tableofcontents

\bigskip

\renewcommand{\thefigure}{\theenumi}
\renewcommand{\thetable}{\theenumi}
%\renewcommand{\theequation}{\theenumi}

%\begin{abstract}
%%\boldmath
%In this letter, an algorithm for evaluating the exact analytical bit error rate  (BER)  for the piecewise linear (PL) combiner for  multiple relays is presented. Previous results were available only for upto three relays. The algorithm is unique in the sense that  the actual mathematical expressions, that are prohibitively large, need not be explicitly obtained. The diversity gain due to multiple relays is shown through plots of the analytical BER, well supported by simulations. 
%
%\end{abstract}
% IEEEtran.cls defaults to using nonbold math in the Abstract.
% This preserves the distinction between vectors and scalars. However,
% if the journal you are submitting to favors bold math in the abstract,
% then you can use LaTeX's standard command \boldmath at the very start
% of the abstract to achieve this. Many IEEE journals frown on math
% in the abstract anyway.

% Note that keywords are not normally used for peerreview papers.
%\begin{IEEEkeywords}
%Cooperative diversity, decode and forward, piecewise linear
%\end{IEEEkeywords}



% For peer review papers, you can put extra information on the cover
% page as needed:
% \ifCLASSOPTIONpeerreview
% \begin{center} \bfseries EDICS Category: 3-BBND \end{center}
% \fi
%
% For peerreview papers, this IEEEtran command inserts a page break and
% creates the second title. It will be ignored for other modes.
%\IEEEpeerreviewmaketitle




If the letters of the word ASSASSINATION are arranged at random. Find the Probability that
\begin{enumerate}[label=(\alph*)]
\item Four $S's$ come consecutively in the word
\item Two  $I's$ and two $N's$ come together
\item All $A's$ are not coming together
\item No two $A's$ are coming together
\end{enumerate}
%\begin{table}[H]
	\centering
\begin{tabular}{|c|c|c|}
\hline
Random variable &Value &Definition\\ \hline
\multirow{3}{*}{X} &0 &Slips of Rs 1\\
&1 &Slips of Rs 5\\
&2 &Slips of Rs 13\\ \hline
\multirow{2}{*}{Y} &0 &Box A\\
&1 &Box B\\\hline
\end{tabular}
\caption{}
\label{tab:Distribution}
\end{table}
See \tabref{tab:Distribution}.
\begin{align}
p_{Y}\brak{k}= \begin{cases} 
      \frac{1}{3} & {k=0} \\
      \frac{2}{3 }& {k=1} 
   \end{cases}
   \\
p_{Y|X}\brak{0|0} = \frac{19}{25}\, 
p_{Y|X}\brak{0|1} = \frac{6}{25}\,
p_{Y|X}\brak{1|0} = \frac{45}{50}\,
p_{Y|X}\brak{1|2} = \frac{5}{50}
\end{align}
The desired probability is the probability that a slip drawn at random is marked other than Rs 1,
\begin{align}
&=1-p_X\brak{0}\\
&= p_X(1) + p_X(2)
\end{align}
Using Bayes theorem,
\begin{align}
&= p_Y\brak{0} \times \pr{Y=0 | X=1} + p_Y\brak{1} \times \pr{Y=1|X=2}\\
&=\frac{1}{3} \times \frac{6}{25} + \frac{2}{3} \times \frac{5}{50}\\
&=\frac{11}{75}
\end{align}

\newpage

%\tableofcontents

\bigskip

\renewcommand{\thefigure}{\theenumi}
\renewcommand{\thetable}{\theenumi}
%\renewcommand{\theequation}{\theenumi}

%\begin{abstract}
%%\boldmath
%In this letter, an algorithm for evaluating the exact analytical bit error rate  (BER)  for the piecewise linear (PL) combiner for  multiple relays is presented. Previous results were available only for upto three relays. The algorithm is unique in the sense that  the actual mathematical expressions, that are prohibitively large, need not be explicitly obtained. The diversity gain due to multiple relays is shown through plots of the analytical BER, well supported by simulations. 
%
%\end{abstract}
% IEEEtran.cls defaults to using nonbold math in the Abstract.
% This preserves the distinction between vectors and scalars. However,
% if the journal you are submitting to favors bold math in the abstract,
% then you can use LaTeX's standard command \boldmath at the very start
% of the abstract to achieve this. Many IEEE journals frown on math
% in the abstract anyway.

% Note that keywords are not normally used for peerreview papers.
%\begin{IEEEkeywords}
%Cooperative diversity, decode and forward, piecewise linear
%\end{IEEEkeywords}



% For peer review papers, you can put extra information on the cover
% page as needed:
% \ifCLASSOPTIONpeerreview
% \begin{center} \bfseries EDICS Category: 3-BBND \end{center}
% \fi
%
% For peerreview papers, this IEEEtran command inserts a page break and
% creates the second title. It will be ignored for other modes.
%\IEEEpeerreviewmaketitle




	\item One urn contains two black balls (labelled B1 and B2) and one white ball. A
	second urn contains one black ball and two white balls (labelled W1 and W2).
	Suppose the following experiment is performed. One of the two urns is chosen
	at random. Next a ball is randomly chosen from the urn. Then a second ball is
	chosen at random from the same urn without replacing the first ball.
	
	\begin{enumerate}
	\item What is the probability that two black balls are chosen?
	
	\item What is the probability that two balls of opposite colour are chosen?
	\end{enumerate}
	\solution
	%\begin{align}
    \label{eq:12.13.6.18.1}
	\because	\pr{A|B} &> \pr{A},\
\frac{\pr{AB}}{\pr{B}} > \pr{A}
\\
    \label{eq:12.13.6.18.2}
	\implies \pr{AB} &> \pr{A}\pr{B}
	\\
	\text{or, } \frac{\pr{AB}}{\pr{A}} &=\pr{B|A} > \pr{A}
\end{align}

\end{enumerate}

	\item A bag contains $5$ red balls and some blue balls. If the probability of drawing a blue ball is double that if a red ball, determine the number of blue balls in the bag. 
		\\
\solution
		%\begin{enumerate}[label=\thesection.\arabic*,ref=\thesection.\theenumi]
	\item One card is drawn from a well-shuffled deck of 52 cards. Find the probability of getting
\begin{enumerate}
\item A king of red colour 
\item A face card 
\item A red face card
\item The jack of hearts
\item A spade
\item The queen of diamonds

\end{enumerate}
\solution
		%\begin{table}[H]
	\centering
\begin{tabular}{|c|c|c|}
\hline
Random variable &Value &Definition\\ \hline
\multirow{3}{*}{X} &0 &Slips of Rs 1\\
&1 &Slips of Rs 5\\
&2 &Slips of Rs 13\\ \hline
\multirow{2}{*}{Y} &0 &Box A\\
&1 &Box B\\\hline
\end{tabular}
\caption{}
\label{tab:Distribution}
\end{table}
See \tabref{tab:Distribution}.
\begin{align}
p_{Y}\brak{k}= \begin{cases} 
      \frac{1}{3} & {k=0} \\
      \frac{2}{3 }& {k=1} 
   \end{cases}
   \\
p_{Y|X}\brak{0|0} = \frac{19}{25}\, 
p_{Y|X}\brak{0|1} = \frac{6}{25}\,
p_{Y|X}\brak{1|0} = \frac{45}{50}\,
p_{Y|X}\brak{1|2} = \frac{5}{50}
\end{align}
The desired probability is the probability that a slip drawn at random is marked other than Rs 1,
\begin{align}
&=1-p_X\brak{0}\\
&= p_X(1) + p_X(2)
\end{align}
Using Bayes theorem,
\begin{align}
&= p_Y\brak{0} \times \pr{Y=0 | X=1} + p_Y\brak{1} \times \pr{Y=1|X=2}\\
&=\frac{1}{3} \times \frac{6}{25} + \frac{2}{3} \times \frac{5}{50}\\
&=\frac{11}{75}
\end{align}

\newpage

%\tableofcontents

\bigskip

\renewcommand{\thefigure}{\theenumi}
\renewcommand{\thetable}{\theenumi}
%\renewcommand{\theequation}{\theenumi}

%\begin{abstract}
%%\boldmath
%In this letter, an algorithm for evaluating the exact analytical bit error rate  (BER)  for the piecewise linear (PL) combiner for  multiple relays is presented. Previous results were available only for upto three relays. The algorithm is unique in the sense that  the actual mathematical expressions, that are prohibitively large, need not be explicitly obtained. The diversity gain due to multiple relays is shown through plots of the analytical BER, well supported by simulations. 
%
%\end{abstract}
% IEEEtran.cls defaults to using nonbold math in the Abstract.
% This preserves the distinction between vectors and scalars. However,
% if the journal you are submitting to favors bold math in the abstract,
% then you can use LaTeX's standard command \boldmath at the very start
% of the abstract to achieve this. Many IEEE journals frown on math
% in the abstract anyway.

% Note that keywords are not normally used for peerreview papers.
%\begin{IEEEkeywords}
%Cooperative diversity, decode and forward, piecewise linear
%\end{IEEEkeywords}



% For peer review papers, you can put extra information on the cover
% page as needed:
% \ifCLASSOPTIONpeerreview
% \begin{center} \bfseries EDICS Category: 3-BBND \end{center}
% \fi
%
% For peerreview papers, this IEEEtran command inserts a page break and
% creates the second title. It will be ignored for other modes.
%\IEEEpeerreviewmaketitle




	\item Five cards—the ten, jack, queen, king and ace of diamonds, are well-shuffled with their face downwards. One card is then picked up at random.
\begin{enumerate}
\item
What is the probability that the card is the queen? 
\item
If the queen is drawn and put aside, what is the probability that the second card picked up is (a) an ace? (b) a queen?\\
\end{enumerate}
\solution
		%\begin{enumerate}[label=\thesection.\arabic*,ref=\thesection.\theenumi]
	\item One card is drawn from a well-shuffled deck of 52 cards. Find the probability of getting
\begin{enumerate}
\item A king of red colour 
\item A face card 
\item A red face card
\item The jack of hearts
\item A spade
\item The queen of diamonds

\end{enumerate}
\solution
		%\begin{table}[H]
	\centering
\begin{tabular}{|c|c|c|}
\hline
Random variable &Value &Definition\\ \hline
\multirow{3}{*}{X} &0 &Slips of Rs 1\\
&1 &Slips of Rs 5\\
&2 &Slips of Rs 13\\ \hline
\multirow{2}{*}{Y} &0 &Box A\\
&1 &Box B\\\hline
\end{tabular}
\caption{}
\label{tab:Distribution}
\end{table}
See \tabref{tab:Distribution}.
\begin{align}
p_{Y}\brak{k}= \begin{cases} 
      \frac{1}{3} & {k=0} \\
      \frac{2}{3 }& {k=1} 
   \end{cases}
   \\
p_{Y|X}\brak{0|0} = \frac{19}{25}\, 
p_{Y|X}\brak{0|1} = \frac{6}{25}\,
p_{Y|X}\brak{1|0} = \frac{45}{50}\,
p_{Y|X}\brak{1|2} = \frac{5}{50}
\end{align}
The desired probability is the probability that a slip drawn at random is marked other than Rs 1,
\begin{align}
&=1-p_X\brak{0}\\
&= p_X(1) + p_X(2)
\end{align}
Using Bayes theorem,
\begin{align}
&= p_Y\brak{0} \times \pr{Y=0 | X=1} + p_Y\brak{1} \times \pr{Y=1|X=2}\\
&=\frac{1}{3} \times \frac{6}{25} + \frac{2}{3} \times \frac{5}{50}\\
&=\frac{11}{75}
\end{align}

\newpage

%\tableofcontents

\bigskip

\renewcommand{\thefigure}{\theenumi}
\renewcommand{\thetable}{\theenumi}
%\renewcommand{\theequation}{\theenumi}

%\begin{abstract}
%%\boldmath
%In this letter, an algorithm for evaluating the exact analytical bit error rate  (BER)  for the piecewise linear (PL) combiner for  multiple relays is presented. Previous results were available only for upto three relays. The algorithm is unique in the sense that  the actual mathematical expressions, that are prohibitively large, need not be explicitly obtained. The diversity gain due to multiple relays is shown through plots of the analytical BER, well supported by simulations. 
%
%\end{abstract}
% IEEEtran.cls defaults to using nonbold math in the Abstract.
% This preserves the distinction between vectors and scalars. However,
% if the journal you are submitting to favors bold math in the abstract,
% then you can use LaTeX's standard command \boldmath at the very start
% of the abstract to achieve this. Many IEEE journals frown on math
% in the abstract anyway.

% Note that keywords are not normally used for peerreview papers.
%\begin{IEEEkeywords}
%Cooperative diversity, decode and forward, piecewise linear
%\end{IEEEkeywords}



% For peer review papers, you can put extra information on the cover
% page as needed:
% \ifCLASSOPTIONpeerreview
% \begin{center} \bfseries EDICS Category: 3-BBND \end{center}
% \fi
%
% For peerreview papers, this IEEEtran command inserts a page break and
% creates the second title. It will be ignored for other modes.
%\IEEEpeerreviewmaketitle




	\item Five cards—the ten, jack, queen, king and ace of diamonds, are well-shuffled with their face downwards. One card is then picked up at random.
\begin{enumerate}
\item
What is the probability that the card is the queen? 
\item
If the queen is drawn and put aside, what is the probability that the second card picked up is (a) an ace? (b) a queen?\\
\end{enumerate}
\solution
		%\begin{enumerate}[label=\thesection.\arabic*,ref=\thesection.\theenumi]
	\item One card is drawn from a well-shuffled deck of 52 cards. Find the probability of getting
\begin{enumerate}
\item A king of red colour 
\item A face card 
\item A red face card
\item The jack of hearts
\item A spade
\item The queen of diamonds

\end{enumerate}
\solution
		%\input{ncert/10/15/1/14/main.tex}
	\item Five cards—the ten, jack, queen, king and ace of diamonds, are well-shuffled with their face downwards. One card is then picked up at random.
\begin{enumerate}
\item
What is the probability that the card is the queen? 
\item
If the queen is drawn and put aside, what is the probability that the second card picked up is (a) an ace? (b) a queen?\\
\end{enumerate}
\solution
		%\input{ncert/10/15/1/15/defs.tex}
	\item A bag contains $5$ red balls and some blue balls. If the probability of drawing a blue ball is double that if a red ball, determine the number of blue balls in the bag. 
		\\
\solution
		%\input{ncert/10/15/2/3/defs.tex}
	\item A card is selected from a pack of 52 cards.
 \begin{enumerate}[label=(\alph*)] 
                 \item How many points are there in the sample space?
                 \item Calculate the probability that the card is an ace of spades.
                 \item Calculate the probability that the card is (i) an ace and (ii) black card.
 \end{enumerate}
\solution
		%\input{ncert/11/16/3/4/main.tex}
\item Four cards are drawn from a well-shuffled deck of 52 cards. What is the probability of obtaining 3 diamonds and one spade.
\\
\solution
		%\input{ncert/11/16/4/2/defs.tex}
\item In a certain lottery 10,000 tickets are sold and ten equal prizes are awarded. What is the probability of not getting a prize if you buy (a) one ticket (b) two tickets (c) 10 tickets ?	
\\
\solution
		%\input{ncert/11/16/4/4/defs.tex}
		%
\item 
Out of 100 students, two sections of 40 and 60 are formed. If you and your friend are among the 100 students, what is the probability that
\begin{enumerate}
\item you both enter the same section?
\item you both enter the different sections?
\end{enumerate}
\solution
		%\input{ncert/11/16/4/5/defs.tex}
	\item 
The number lock of a suitcase has 4 wheels each labelled with ten digits i.e. from 0 to 9.The lock opens with a sequence of four digits with no repeats.What is the probability of a person getting the right sequence to open the suitcase.
\\
\solution
		%\input{ncert/11/16/4/10/defs.tex}
		%
\item 
Two cards are drawn at random and without replacement from a pack of 52 playing cards. Find the probability that both the cards are black.
\\
\solution
		%\input{ncert/12/13/2/2/defs.tex}
		\item A box of oranges is inspected by examining three randomly selected oranges drawn without replacement. If all the three oranges are good, the box is approved for sale, otherwise, it is rejected. Find the probability that a box containing 15 oranges out of which 12 are good and 3 are bad ones will be approved for sale.
		\label{ncert/12/13/2/3/defs.tex}
		\item Two balls are drawn at random with replacement from a box containing 10 black and 8 red balls. Find the probability that
		\label{ncert/12/13/2/12}
\begin{enumerate}
\item both balls are red.
\item first ball is black and second is red.
\item one of them is black and other is red.
\end{enumerate}

\item In a hostel, 60\% of the students read Hindi newspaper, 40\% read English newspaper and 20\% read both Hindi and English newspapers. A student is selected at random.
		\label{ncert/12/13/2/15}
\begin{enumerate}
\item Find the probability that she reads neither Hindi nor English newspapers.
\item If she reads Hindi newspaper, find the probability that she reads English newspaper.
\item If she reads English newspaper, find the probability that she reads Hindi newspaper.\\
\end{enumerate}
\item The probability of obtaining an even prime number on each die, when a pair of dice is rolled is 
\begin{enumerate}
    \item $0$ 
    
    \item $\frac{1}{3}$ 
    
    \item $\frac{1}{12}$ 
    
    \item $\frac{1}{36}$ 
\end{enumerate}
\solution
		%\input{ncert/12/13/2/17/defs.tex}
	\item A bag contains 4 red and 4 black balls, another bag contains 2 red and 6 black balls. One of the two bags is selected at random and a ball is drawn from the bag which is found to be red. Find the probability that the ball is drawn from the first bag.
\\
\solution
		%\input{ncert/12/13/3/2/main.tex}
  \item
  Cards with numbers 2 to 101 are placed in a box. A card is selected at random.Find the probability that the card has
\begin{enumerate}[label=(\roman*)]
	\item an even number 
	\item a square number
\end{enumerate}
\solution
%\input{exemplar/10/13/3/32/main.tex}
\item
The king, queen and jack of clubs are removed from a deck of 52 playing cards and then well shuffled. Now one card is drawn at random from the remaining cards.  Determine the probability that the card is
\begin{enumerate}[label=(\roman*)]
\item a club
\item 10 of hearts
\end{enumerate}
\solution
%\input{exemplar/10/13/3/29/main.tex}
\item A team of medical students doing their internship have to assist during surgeries
at a city hospital. The probabilities of surgeries rated as very complex, complex,
routine, simple or very simple are respectively, 0.15, 0.20, 0.31, 0.26, .08. Find
the probabilities that a particular surgery will be rated
\begin{enumerate}
	\item complex or very complex;
	\item neither very complex nor very simple;
	\item routine or complex
	\item routine or simple
\end{enumerate}
\solution
%\input{exemplar/11/16/3/8(1)/main.tex}
\item A card is selected from a pack of 52 cards.
\begin{enumerate}[label=(\alph*)]
    \item How many points are there in the sample space?
    \item Calculate the probability that the card is an ace of spades.
    \item Calculate the probability that the card is (i) an ace and (ii) black card.
\end{enumerate}
\solution
%\input{exemplar/11/16/3/4/main2.tex}
\item The probability that a non leap year selected at random will contain 53 sundays.
\\
\solution
%\input{exemplar/10/13/1/19/main.tex}
\item One of the four persons John, Rita, Aslam or Gurpreet will be promoted next
month. Consequently the sample space consists of four elementary outcomes
S = {John promoted, Rita promoted, Aslam promoted, Gurpreet promoted}
You are told that the chances of John’s promotion is same as that of Gurpreet,
Rita’s chances of promotion are twice as likely as Johns. Aslam’s chances are
four times that of John.
\begin{enumerate}
	\item Determine
	\begin{enumerate}
		\item P (John promoted)
		\item P (Rita promoted)
		\item P (Aslam promoted)
		\item P (Gurpreet promoted)
	\end{enumerate}
	\item If A = {John promoted or Gurpreet promoted}, find P (A).
\end{enumerate}
\solution
%\input{exemplar/11/16/3/10/main.tex}
\item A card is drawn from a deck of 52 cards. Find the probability of getting a king or a heart or a red card.\\
\solution
%\input{exemplar/11/16/3/15/main.tex}
\item The probability that a student will pass his examination is 0.73, the probability of
the student getting a compartment is 0.13, and the probability that the student will
either pass or get compartment is 0.96. State True or False.\\
\solution
%\input{exemplar/11/16/3/31/main.tex}
\item A card is selected from a pack of 52 cards\\
\begin{enumerate}[label=(\alph*)]
\item How many points are there in the sample space?
\item Calculate the probability that the cards is an ace of spades.
\item Calculate the probability that the card is (i) an ace (ii)black card.\\
\end{enumerate}
%\input{ncert/11/16/3/4_1/Prob_4.tex}
\item In a non-leap year, the probability of having 53 tuesdays or 53 wednesdays is\\
\solution
%\input{exemplar/11/16/3/18/main.tex}
\item There are 1000 sealed envelopes in a box, 10 of them contain a cash prize of
Rs 100 each, 100 of them contain a cash prize of Rs 50 each and 200 of them
contain a cash prize of Rs 10 each and rest do not contain any cash prize. If they
are well shuffled and an envelope is picked up out, what is the probability that it
contains no cash prize?\\
\solution
%\input{exemplar/10/13/3/34/main.tex}
\item 
A die is thrown and a card is selected at random from a deck of 52 playing cards. The probability of getting an even number on the die and a spade card.\\
\solution
%\input{exemplar/12/13/3/78/main.tex}
\item
If 4-digit numbers greater than 5,000 are randomly formed from the digits 0, 1, 3, 5, and 7, what is the probability of forming a number divisible by 5 when:
\begin{enumerate}
    \item The digits are repeated?
    \item The repetition of digits is not allowed?
\end{enumerate}
\solution
%\input{ncert/11/16/4/9/main.tex}
\item Consider the probability space $\brak{\Omega, \mathcal{G}, P}$ where $\Omega = [0,2]$ and $\mathcal{G} = \cbrak{\phi, \Omega, [0,1], (1,2]}$. Let $X$ and $Y$ be two functions on $\Omega$ defined as
\begin{align*}
    X(\omega) = 
    \begin{cases}
        1 & \text{if }\omega \in [0, 1]\\
        2 & \text{if }\omega \in (1, 2]
    \end{cases}
\end{align*}
and
\begin{align*}
    Y(\omega) = 
    \begin{cases}
        2 & \text{if }\omega \in [0, 1.5]\\
        3 & \text{if }\omega \in (1.5, 2].
    \end{cases}
\end{align*}
Then which one of the following statements is true?
\begin{enumerate}
    \item [(A)] $X$ is a random variable with respect to $\mathcal{G}$, but $Y$ is not a random variable with respect to $\mathcal{G}$.
    \item [(B)] $Y$ is a random variable with respect to $\mathcal{G}$, but $X$ is not a random variable with respect to $\mathcal{G}$.
    \item [(C)] Neither $X$ nor $Y$ is a random variable with respect to $\mathcal{G}$.
    \item [(D)] Both $X$ and $Y$ are random variables with respect to $\mathcal{G}$.
\end{enumerate} \hfill (GATE ST 2023)\\
\solution
%\input{gate/ST/2023/14/main.tex}
	\item  A die is loaded in such a way that each odd number is twice as likely to occur as
each even number. Find $P(G)$, where $G$ is the event that a number greater than
3 occurs on a single roll of the die.
\\
\solution
		%\input{exemplar/11/16/3/5/main.tex}
	\item All the jacks, queens and kings are removed from a deck of 52 playing cards. The remaining cards are well shuffled and then one card is drawn at random. Giving ace a value 1 similar value for other cards, find the probability that the card has a value 
		\begin{enumerate}
			\item 7
			\item greater than 7
			\item less than 7
		\end{enumerate}
		%\input{exemplar/10/13/3/30/main.tex}
  \item A Lot consists of 48 mobile phones of which 42 are good, 3 have only minor defects and 3 have major defects.Varnika will buy a phone if it is good but the trader will only buy a mobile if it has no major defects. One phone is selected at random from the lot. What is the probability that it is
\begin{enumerate}
	\item acceptable to Varnika?
            \item acceptable to the trader?
\end{enumerate}
\solution
	%\input{exemplar/10/13/3/40/main.tex}
 \item A student says that if you throw a die, it will show up 1 or not 1. Therefore, the probability of getting 1 and the probability of getting 'not 1' each is equal to $\frac{1}{2}$. Is this correct? Give reasons.\\
 \solution
        %\input{exemplar/10/13/2/9/main.tex}
   \item Four candidates A, B, C, D have ap-
plied for the assignment to coach a school cricket
team. If A is twice as likely to be selected as B, and
B and C are given about the same chance of being
selected, while C is twice as likely to be selected
as D, what are the probabilities that
\begin{enumerate}
\item C will be selected?
\item A will not be selected?
\end{enumerate}
	%\input{exemplar/11/16/3/9/main.tex}
 \item A bag contain 24 balls of which $x$ balls are red, $2x$ are white and $3x$ are blue. A ball is selected at random, What is the probability that it is
\begin{enumerate}[label=\alph*)]
\item not red ?
\item white ?
\end{enumerate}
%\input{exemplar/10/13/3/41/main.tex}
If the letters of the word ASSASSINATION are arranged at random. Find the Probability that
\begin{enumerate}[label=(\alph*)]
\item Four $S's$ come consecutively in the word
\item Two  $I's$ and two $N's$ come together
\item All $A's$ are not coming together
\item No two $A's$ are coming together
\end{enumerate}
%\input{exemplar/11/16/3/14/main.tex}
	\item One urn contains two black balls (labelled B1 and B2) and one white ball. A
	second urn contains one black ball and two white balls (labelled W1 and W2).
	Suppose the following experiment is performed. One of the two urns is chosen
	at random. Next a ball is randomly chosen from the urn. Then a second ball is
	chosen at random from the same urn without replacing the first ball.
	
	\begin{enumerate}
	\item What is the probability that two black balls are chosen?
	
	\item What is the probability that two balls of opposite colour are chosen?
	\end{enumerate}
	\solution
	%\input{exemplar/11/16/3/12/main1.tex}
\end{enumerate}

	\item A bag contains $5$ red balls and some blue balls. If the probability of drawing a blue ball is double that if a red ball, determine the number of blue balls in the bag. 
		\\
\solution
		%\begin{enumerate}[label=\thesection.\arabic*,ref=\thesection.\theenumi]
	\item One card is drawn from a well-shuffled deck of 52 cards. Find the probability of getting
\begin{enumerate}
\item A king of red colour 
\item A face card 
\item A red face card
\item The jack of hearts
\item A spade
\item The queen of diamonds

\end{enumerate}
\solution
		%\input{ncert/10/15/1/14/main.tex}
	\item Five cards—the ten, jack, queen, king and ace of diamonds, are well-shuffled with their face downwards. One card is then picked up at random.
\begin{enumerate}
\item
What is the probability that the card is the queen? 
\item
If the queen is drawn and put aside, what is the probability that the second card picked up is (a) an ace? (b) a queen?\\
\end{enumerate}
\solution
		%\input{ncert/10/15/1/15/defs.tex}
	\item A bag contains $5$ red balls and some blue balls. If the probability of drawing a blue ball is double that if a red ball, determine the number of blue balls in the bag. 
		\\
\solution
		%\input{ncert/10/15/2/3/defs.tex}
	\item A card is selected from a pack of 52 cards.
 \begin{enumerate}[label=(\alph*)] 
                 \item How many points are there in the sample space?
                 \item Calculate the probability that the card is an ace of spades.
                 \item Calculate the probability that the card is (i) an ace and (ii) black card.
 \end{enumerate}
\solution
		%\input{ncert/11/16/3/4/main.tex}
\item Four cards are drawn from a well-shuffled deck of 52 cards. What is the probability of obtaining 3 diamonds and one spade.
\\
\solution
		%\input{ncert/11/16/4/2/defs.tex}
\item In a certain lottery 10,000 tickets are sold and ten equal prizes are awarded. What is the probability of not getting a prize if you buy (a) one ticket (b) two tickets (c) 10 tickets ?	
\\
\solution
		%\input{ncert/11/16/4/4/defs.tex}
		%
\item 
Out of 100 students, two sections of 40 and 60 are formed. If you and your friend are among the 100 students, what is the probability that
\begin{enumerate}
\item you both enter the same section?
\item you both enter the different sections?
\end{enumerate}
\solution
		%\input{ncert/11/16/4/5/defs.tex}
	\item 
The number lock of a suitcase has 4 wheels each labelled with ten digits i.e. from 0 to 9.The lock opens with a sequence of four digits with no repeats.What is the probability of a person getting the right sequence to open the suitcase.
\\
\solution
		%\input{ncert/11/16/4/10/defs.tex}
		%
\item 
Two cards are drawn at random and without replacement from a pack of 52 playing cards. Find the probability that both the cards are black.
\\
\solution
		%\input{ncert/12/13/2/2/defs.tex}
		\item A box of oranges is inspected by examining three randomly selected oranges drawn without replacement. If all the three oranges are good, the box is approved for sale, otherwise, it is rejected. Find the probability that a box containing 15 oranges out of which 12 are good and 3 are bad ones will be approved for sale.
		\label{ncert/12/13/2/3/defs.tex}
		\item Two balls are drawn at random with replacement from a box containing 10 black and 8 red balls. Find the probability that
		\label{ncert/12/13/2/12}
\begin{enumerate}
\item both balls are red.
\item first ball is black and second is red.
\item one of them is black and other is red.
\end{enumerate}

\item In a hostel, 60\% of the students read Hindi newspaper, 40\% read English newspaper and 20\% read both Hindi and English newspapers. A student is selected at random.
		\label{ncert/12/13/2/15}
\begin{enumerate}
\item Find the probability that she reads neither Hindi nor English newspapers.
\item If she reads Hindi newspaper, find the probability that she reads English newspaper.
\item If she reads English newspaper, find the probability that she reads Hindi newspaper.\\
\end{enumerate}
\item The probability of obtaining an even prime number on each die, when a pair of dice is rolled is 
\begin{enumerate}
    \item $0$ 
    
    \item $\frac{1}{3}$ 
    
    \item $\frac{1}{12}$ 
    
    \item $\frac{1}{36}$ 
\end{enumerate}
\solution
		%\input{ncert/12/13/2/17/defs.tex}
	\item A bag contains 4 red and 4 black balls, another bag contains 2 red and 6 black balls. One of the two bags is selected at random and a ball is drawn from the bag which is found to be red. Find the probability that the ball is drawn from the first bag.
\\
\solution
		%\input{ncert/12/13/3/2/main.tex}
  \item
  Cards with numbers 2 to 101 are placed in a box. A card is selected at random.Find the probability that the card has
\begin{enumerate}[label=(\roman*)]
	\item an even number 
	\item a square number
\end{enumerate}
\solution
%\input{exemplar/10/13/3/32/main.tex}
\item
The king, queen and jack of clubs are removed from a deck of 52 playing cards and then well shuffled. Now one card is drawn at random from the remaining cards.  Determine the probability that the card is
\begin{enumerate}[label=(\roman*)]
\item a club
\item 10 of hearts
\end{enumerate}
\solution
%\input{exemplar/10/13/3/29/main.tex}
\item A team of medical students doing their internship have to assist during surgeries
at a city hospital. The probabilities of surgeries rated as very complex, complex,
routine, simple or very simple are respectively, 0.15, 0.20, 0.31, 0.26, .08. Find
the probabilities that a particular surgery will be rated
\begin{enumerate}
	\item complex or very complex;
	\item neither very complex nor very simple;
	\item routine or complex
	\item routine or simple
\end{enumerate}
\solution
%\input{exemplar/11/16/3/8(1)/main.tex}
\item A card is selected from a pack of 52 cards.
\begin{enumerate}[label=(\alph*)]
    \item How many points are there in the sample space?
    \item Calculate the probability that the card is an ace of spades.
    \item Calculate the probability that the card is (i) an ace and (ii) black card.
\end{enumerate}
\solution
%\input{exemplar/11/16/3/4/main2.tex}
\item The probability that a non leap year selected at random will contain 53 sundays.
\\
\solution
%\input{exemplar/10/13/1/19/main.tex}
\item One of the four persons John, Rita, Aslam or Gurpreet will be promoted next
month. Consequently the sample space consists of four elementary outcomes
S = {John promoted, Rita promoted, Aslam promoted, Gurpreet promoted}
You are told that the chances of John’s promotion is same as that of Gurpreet,
Rita’s chances of promotion are twice as likely as Johns. Aslam’s chances are
four times that of John.
\begin{enumerate}
	\item Determine
	\begin{enumerate}
		\item P (John promoted)
		\item P (Rita promoted)
		\item P (Aslam promoted)
		\item P (Gurpreet promoted)
	\end{enumerate}
	\item If A = {John promoted or Gurpreet promoted}, find P (A).
\end{enumerate}
\solution
%\input{exemplar/11/16/3/10/main.tex}
\item A card is drawn from a deck of 52 cards. Find the probability of getting a king or a heart or a red card.\\
\solution
%\input{exemplar/11/16/3/15/main.tex}
\item The probability that a student will pass his examination is 0.73, the probability of
the student getting a compartment is 0.13, and the probability that the student will
either pass or get compartment is 0.96. State True or False.\\
\solution
%\input{exemplar/11/16/3/31/main.tex}
\item A card is selected from a pack of 52 cards\\
\begin{enumerate}[label=(\alph*)]
\item How many points are there in the sample space?
\item Calculate the probability that the cards is an ace of spades.
\item Calculate the probability that the card is (i) an ace (ii)black card.\\
\end{enumerate}
%\input{ncert/11/16/3/4_1/Prob_4.tex}
\item In a non-leap year, the probability of having 53 tuesdays or 53 wednesdays is\\
\solution
%\input{exemplar/11/16/3/18/main.tex}
\item There are 1000 sealed envelopes in a box, 10 of them contain a cash prize of
Rs 100 each, 100 of them contain a cash prize of Rs 50 each and 200 of them
contain a cash prize of Rs 10 each and rest do not contain any cash prize. If they
are well shuffled and an envelope is picked up out, what is the probability that it
contains no cash prize?\\
\solution
%\input{exemplar/10/13/3/34/main.tex}
\item 
A die is thrown and a card is selected at random from a deck of 52 playing cards. The probability of getting an even number on the die and a spade card.\\
\solution
%\input{exemplar/12/13/3/78/main.tex}
\item
If 4-digit numbers greater than 5,000 are randomly formed from the digits 0, 1, 3, 5, and 7, what is the probability of forming a number divisible by 5 when:
\begin{enumerate}
    \item The digits are repeated?
    \item The repetition of digits is not allowed?
\end{enumerate}
\solution
%\input{ncert/11/16/4/9/main.tex}
\item Consider the probability space $\brak{\Omega, \mathcal{G}, P}$ where $\Omega = [0,2]$ and $\mathcal{G} = \cbrak{\phi, \Omega, [0,1], (1,2]}$. Let $X$ and $Y$ be two functions on $\Omega$ defined as
\begin{align*}
    X(\omega) = 
    \begin{cases}
        1 & \text{if }\omega \in [0, 1]\\
        2 & \text{if }\omega \in (1, 2]
    \end{cases}
\end{align*}
and
\begin{align*}
    Y(\omega) = 
    \begin{cases}
        2 & \text{if }\omega \in [0, 1.5]\\
        3 & \text{if }\omega \in (1.5, 2].
    \end{cases}
\end{align*}
Then which one of the following statements is true?
\begin{enumerate}
    \item [(A)] $X$ is a random variable with respect to $\mathcal{G}$, but $Y$ is not a random variable with respect to $\mathcal{G}$.
    \item [(B)] $Y$ is a random variable with respect to $\mathcal{G}$, but $X$ is not a random variable with respect to $\mathcal{G}$.
    \item [(C)] Neither $X$ nor $Y$ is a random variable with respect to $\mathcal{G}$.
    \item [(D)] Both $X$ and $Y$ are random variables with respect to $\mathcal{G}$.
\end{enumerate} \hfill (GATE ST 2023)\\
\solution
%\input{gate/ST/2023/14/main.tex}
	\item  A die is loaded in such a way that each odd number is twice as likely to occur as
each even number. Find $P(G)$, where $G$ is the event that a number greater than
3 occurs on a single roll of the die.
\\
\solution
		%\input{exemplar/11/16/3/5/main.tex}
	\item All the jacks, queens and kings are removed from a deck of 52 playing cards. The remaining cards are well shuffled and then one card is drawn at random. Giving ace a value 1 similar value for other cards, find the probability that the card has a value 
		\begin{enumerate}
			\item 7
			\item greater than 7
			\item less than 7
		\end{enumerate}
		%\input{exemplar/10/13/3/30/main.tex}
  \item A Lot consists of 48 mobile phones of which 42 are good, 3 have only minor defects and 3 have major defects.Varnika will buy a phone if it is good but the trader will only buy a mobile if it has no major defects. One phone is selected at random from the lot. What is the probability that it is
\begin{enumerate}
	\item acceptable to Varnika?
            \item acceptable to the trader?
\end{enumerate}
\solution
	%\input{exemplar/10/13/3/40/main.tex}
 \item A student says that if you throw a die, it will show up 1 or not 1. Therefore, the probability of getting 1 and the probability of getting 'not 1' each is equal to $\frac{1}{2}$. Is this correct? Give reasons.\\
 \solution
        %\input{exemplar/10/13/2/9/main.tex}
   \item Four candidates A, B, C, D have ap-
plied for the assignment to coach a school cricket
team. If A is twice as likely to be selected as B, and
B and C are given about the same chance of being
selected, while C is twice as likely to be selected
as D, what are the probabilities that
\begin{enumerate}
\item C will be selected?
\item A will not be selected?
\end{enumerate}
	%\input{exemplar/11/16/3/9/main.tex}
 \item A bag contain 24 balls of which $x$ balls are red, $2x$ are white and $3x$ are blue. A ball is selected at random, What is the probability that it is
\begin{enumerate}[label=\alph*)]
\item not red ?
\item white ?
\end{enumerate}
%\input{exemplar/10/13/3/41/main.tex}
If the letters of the word ASSASSINATION are arranged at random. Find the Probability that
\begin{enumerate}[label=(\alph*)]
\item Four $S's$ come consecutively in the word
\item Two  $I's$ and two $N's$ come together
\item All $A's$ are not coming together
\item No two $A's$ are coming together
\end{enumerate}
%\input{exemplar/11/16/3/14/main.tex}
	\item One urn contains two black balls (labelled B1 and B2) and one white ball. A
	second urn contains one black ball and two white balls (labelled W1 and W2).
	Suppose the following experiment is performed. One of the two urns is chosen
	at random. Next a ball is randomly chosen from the urn. Then a second ball is
	chosen at random from the same urn without replacing the first ball.
	
	\begin{enumerate}
	\item What is the probability that two black balls are chosen?
	
	\item What is the probability that two balls of opposite colour are chosen?
	\end{enumerate}
	\solution
	%\input{exemplar/11/16/3/12/main1.tex}
\end{enumerate}

	\item A card is selected from a pack of 52 cards.
 \begin{enumerate}[label=(\alph*)] 
                 \item How many points are there in the sample space?
                 \item Calculate the probability that the card is an ace of spades.
                 \item Calculate the probability that the card is (i) an ace and (ii) black card.
 \end{enumerate}
\solution
		%\begin{table}[H]
	\centering
\begin{tabular}{|c|c|c|}
\hline
Random variable &Value &Definition\\ \hline
\multirow{3}{*}{X} &0 &Slips of Rs 1\\
&1 &Slips of Rs 5\\
&2 &Slips of Rs 13\\ \hline
\multirow{2}{*}{Y} &0 &Box A\\
&1 &Box B\\\hline
\end{tabular}
\caption{}
\label{tab:Distribution}
\end{table}
See \tabref{tab:Distribution}.
\begin{align}
p_{Y}\brak{k}= \begin{cases} 
      \frac{1}{3} & {k=0} \\
      \frac{2}{3 }& {k=1} 
   \end{cases}
   \\
p_{Y|X}\brak{0|0} = \frac{19}{25}\, 
p_{Y|X}\brak{0|1} = \frac{6}{25}\,
p_{Y|X}\brak{1|0} = \frac{45}{50}\,
p_{Y|X}\brak{1|2} = \frac{5}{50}
\end{align}
The desired probability is the probability that a slip drawn at random is marked other than Rs 1,
\begin{align}
&=1-p_X\brak{0}\\
&= p_X(1) + p_X(2)
\end{align}
Using Bayes theorem,
\begin{align}
&= p_Y\brak{0} \times \pr{Y=0 | X=1} + p_Y\brak{1} \times \pr{Y=1|X=2}\\
&=\frac{1}{3} \times \frac{6}{25} + \frac{2}{3} \times \frac{5}{50}\\
&=\frac{11}{75}
\end{align}

\newpage

%\tableofcontents

\bigskip

\renewcommand{\thefigure}{\theenumi}
\renewcommand{\thetable}{\theenumi}
%\renewcommand{\theequation}{\theenumi}

%\begin{abstract}
%%\boldmath
%In this letter, an algorithm for evaluating the exact analytical bit error rate  (BER)  for the piecewise linear (PL) combiner for  multiple relays is presented. Previous results were available only for upto three relays. The algorithm is unique in the sense that  the actual mathematical expressions, that are prohibitively large, need not be explicitly obtained. The diversity gain due to multiple relays is shown through plots of the analytical BER, well supported by simulations. 
%
%\end{abstract}
% IEEEtran.cls defaults to using nonbold math in the Abstract.
% This preserves the distinction between vectors and scalars. However,
% if the journal you are submitting to favors bold math in the abstract,
% then you can use LaTeX's standard command \boldmath at the very start
% of the abstract to achieve this. Many IEEE journals frown on math
% in the abstract anyway.

% Note that keywords are not normally used for peerreview papers.
%\begin{IEEEkeywords}
%Cooperative diversity, decode and forward, piecewise linear
%\end{IEEEkeywords}



% For peer review papers, you can put extra information on the cover
% page as needed:
% \ifCLASSOPTIONpeerreview
% \begin{center} \bfseries EDICS Category: 3-BBND \end{center}
% \fi
%
% For peerreview papers, this IEEEtran command inserts a page break and
% creates the second title. It will be ignored for other modes.
%\IEEEpeerreviewmaketitle




\item Four cards are drawn from a well-shuffled deck of 52 cards. What is the probability of obtaining 3 diamonds and one spade.
\\
\solution
		%\begin{enumerate}[label=\thesection.\arabic*,ref=\thesection.\theenumi]
	\item One card is drawn from a well-shuffled deck of 52 cards. Find the probability of getting
\begin{enumerate}
\item A king of red colour 
\item A face card 
\item A red face card
\item The jack of hearts
\item A spade
\item The queen of diamonds

\end{enumerate}
\solution
		%\input{ncert/10/15/1/14/main.tex}
	\item Five cards—the ten, jack, queen, king and ace of diamonds, are well-shuffled with their face downwards. One card is then picked up at random.
\begin{enumerate}
\item
What is the probability that the card is the queen? 
\item
If the queen is drawn and put aside, what is the probability that the second card picked up is (a) an ace? (b) a queen?\\
\end{enumerate}
\solution
		%\input{ncert/10/15/1/15/defs.tex}
	\item A bag contains $5$ red balls and some blue balls. If the probability of drawing a blue ball is double that if a red ball, determine the number of blue balls in the bag. 
		\\
\solution
		%\input{ncert/10/15/2/3/defs.tex}
	\item A card is selected from a pack of 52 cards.
 \begin{enumerate}[label=(\alph*)] 
                 \item How many points are there in the sample space?
                 \item Calculate the probability that the card is an ace of spades.
                 \item Calculate the probability that the card is (i) an ace and (ii) black card.
 \end{enumerate}
\solution
		%\input{ncert/11/16/3/4/main.tex}
\item Four cards are drawn from a well-shuffled deck of 52 cards. What is the probability of obtaining 3 diamonds and one spade.
\\
\solution
		%\input{ncert/11/16/4/2/defs.tex}
\item In a certain lottery 10,000 tickets are sold and ten equal prizes are awarded. What is the probability of not getting a prize if you buy (a) one ticket (b) two tickets (c) 10 tickets ?	
\\
\solution
		%\input{ncert/11/16/4/4/defs.tex}
		%
\item 
Out of 100 students, two sections of 40 and 60 are formed. If you and your friend are among the 100 students, what is the probability that
\begin{enumerate}
\item you both enter the same section?
\item you both enter the different sections?
\end{enumerate}
\solution
		%\input{ncert/11/16/4/5/defs.tex}
	\item 
The number lock of a suitcase has 4 wheels each labelled with ten digits i.e. from 0 to 9.The lock opens with a sequence of four digits with no repeats.What is the probability of a person getting the right sequence to open the suitcase.
\\
\solution
		%\input{ncert/11/16/4/10/defs.tex}
		%
\item 
Two cards are drawn at random and without replacement from a pack of 52 playing cards. Find the probability that both the cards are black.
\\
\solution
		%\input{ncert/12/13/2/2/defs.tex}
		\item A box of oranges is inspected by examining three randomly selected oranges drawn without replacement. If all the three oranges are good, the box is approved for sale, otherwise, it is rejected. Find the probability that a box containing 15 oranges out of which 12 are good and 3 are bad ones will be approved for sale.
		\label{ncert/12/13/2/3/defs.tex}
		\item Two balls are drawn at random with replacement from a box containing 10 black and 8 red balls. Find the probability that
		\label{ncert/12/13/2/12}
\begin{enumerate}
\item both balls are red.
\item first ball is black and second is red.
\item one of them is black and other is red.
\end{enumerate}

\item In a hostel, 60\% of the students read Hindi newspaper, 40\% read English newspaper and 20\% read both Hindi and English newspapers. A student is selected at random.
		\label{ncert/12/13/2/15}
\begin{enumerate}
\item Find the probability that she reads neither Hindi nor English newspapers.
\item If she reads Hindi newspaper, find the probability that she reads English newspaper.
\item If she reads English newspaper, find the probability that she reads Hindi newspaper.\\
\end{enumerate}
\item The probability of obtaining an even prime number on each die, when a pair of dice is rolled is 
\begin{enumerate}
    \item $0$ 
    
    \item $\frac{1}{3}$ 
    
    \item $\frac{1}{12}$ 
    
    \item $\frac{1}{36}$ 
\end{enumerate}
\solution
		%\input{ncert/12/13/2/17/defs.tex}
	\item A bag contains 4 red and 4 black balls, another bag contains 2 red and 6 black balls. One of the two bags is selected at random and a ball is drawn from the bag which is found to be red. Find the probability that the ball is drawn from the first bag.
\\
\solution
		%\input{ncert/12/13/3/2/main.tex}
  \item
  Cards with numbers 2 to 101 are placed in a box. A card is selected at random.Find the probability that the card has
\begin{enumerate}[label=(\roman*)]
	\item an even number 
	\item a square number
\end{enumerate}
\solution
%\input{exemplar/10/13/3/32/main.tex}
\item
The king, queen and jack of clubs are removed from a deck of 52 playing cards and then well shuffled. Now one card is drawn at random from the remaining cards.  Determine the probability that the card is
\begin{enumerate}[label=(\roman*)]
\item a club
\item 10 of hearts
\end{enumerate}
\solution
%\input{exemplar/10/13/3/29/main.tex}
\item A team of medical students doing their internship have to assist during surgeries
at a city hospital. The probabilities of surgeries rated as very complex, complex,
routine, simple or very simple are respectively, 0.15, 0.20, 0.31, 0.26, .08. Find
the probabilities that a particular surgery will be rated
\begin{enumerate}
	\item complex or very complex;
	\item neither very complex nor very simple;
	\item routine or complex
	\item routine or simple
\end{enumerate}
\solution
%\input{exemplar/11/16/3/8(1)/main.tex}
\item A card is selected from a pack of 52 cards.
\begin{enumerate}[label=(\alph*)]
    \item How many points are there in the sample space?
    \item Calculate the probability that the card is an ace of spades.
    \item Calculate the probability that the card is (i) an ace and (ii) black card.
\end{enumerate}
\solution
%\input{exemplar/11/16/3/4/main2.tex}
\item The probability that a non leap year selected at random will contain 53 sundays.
\\
\solution
%\input{exemplar/10/13/1/19/main.tex}
\item One of the four persons John, Rita, Aslam or Gurpreet will be promoted next
month. Consequently the sample space consists of four elementary outcomes
S = {John promoted, Rita promoted, Aslam promoted, Gurpreet promoted}
You are told that the chances of John’s promotion is same as that of Gurpreet,
Rita’s chances of promotion are twice as likely as Johns. Aslam’s chances are
four times that of John.
\begin{enumerate}
	\item Determine
	\begin{enumerate}
		\item P (John promoted)
		\item P (Rita promoted)
		\item P (Aslam promoted)
		\item P (Gurpreet promoted)
	\end{enumerate}
	\item If A = {John promoted or Gurpreet promoted}, find P (A).
\end{enumerate}
\solution
%\input{exemplar/11/16/3/10/main.tex}
\item A card is drawn from a deck of 52 cards. Find the probability of getting a king or a heart or a red card.\\
\solution
%\input{exemplar/11/16/3/15/main.tex}
\item The probability that a student will pass his examination is 0.73, the probability of
the student getting a compartment is 0.13, and the probability that the student will
either pass or get compartment is 0.96. State True or False.\\
\solution
%\input{exemplar/11/16/3/31/main.tex}
\item A card is selected from a pack of 52 cards\\
\begin{enumerate}[label=(\alph*)]
\item How many points are there in the sample space?
\item Calculate the probability that the cards is an ace of spades.
\item Calculate the probability that the card is (i) an ace (ii)black card.\\
\end{enumerate}
%\input{ncert/11/16/3/4_1/Prob_4.tex}
\item In a non-leap year, the probability of having 53 tuesdays or 53 wednesdays is\\
\solution
%\input{exemplar/11/16/3/18/main.tex}
\item There are 1000 sealed envelopes in a box, 10 of them contain a cash prize of
Rs 100 each, 100 of them contain a cash prize of Rs 50 each and 200 of them
contain a cash prize of Rs 10 each and rest do not contain any cash prize. If they
are well shuffled and an envelope is picked up out, what is the probability that it
contains no cash prize?\\
\solution
%\input{exemplar/10/13/3/34/main.tex}
\item 
A die is thrown and a card is selected at random from a deck of 52 playing cards. The probability of getting an even number on the die and a spade card.\\
\solution
%\input{exemplar/12/13/3/78/main.tex}
\item
If 4-digit numbers greater than 5,000 are randomly formed from the digits 0, 1, 3, 5, and 7, what is the probability of forming a number divisible by 5 when:
\begin{enumerate}
    \item The digits are repeated?
    \item The repetition of digits is not allowed?
\end{enumerate}
\solution
%\input{ncert/11/16/4/9/main.tex}
\item Consider the probability space $\brak{\Omega, \mathcal{G}, P}$ where $\Omega = [0,2]$ and $\mathcal{G} = \cbrak{\phi, \Omega, [0,1], (1,2]}$. Let $X$ and $Y$ be two functions on $\Omega$ defined as
\begin{align*}
    X(\omega) = 
    \begin{cases}
        1 & \text{if }\omega \in [0, 1]\\
        2 & \text{if }\omega \in (1, 2]
    \end{cases}
\end{align*}
and
\begin{align*}
    Y(\omega) = 
    \begin{cases}
        2 & \text{if }\omega \in [0, 1.5]\\
        3 & \text{if }\omega \in (1.5, 2].
    \end{cases}
\end{align*}
Then which one of the following statements is true?
\begin{enumerate}
    \item [(A)] $X$ is a random variable with respect to $\mathcal{G}$, but $Y$ is not a random variable with respect to $\mathcal{G}$.
    \item [(B)] $Y$ is a random variable with respect to $\mathcal{G}$, but $X$ is not a random variable with respect to $\mathcal{G}$.
    \item [(C)] Neither $X$ nor $Y$ is a random variable with respect to $\mathcal{G}$.
    \item [(D)] Both $X$ and $Y$ are random variables with respect to $\mathcal{G}$.
\end{enumerate} \hfill (GATE ST 2023)\\
\solution
%\input{gate/ST/2023/14/main.tex}
	\item  A die is loaded in such a way that each odd number is twice as likely to occur as
each even number. Find $P(G)$, where $G$ is the event that a number greater than
3 occurs on a single roll of the die.
\\
\solution
		%\input{exemplar/11/16/3/5/main.tex}
	\item All the jacks, queens and kings are removed from a deck of 52 playing cards. The remaining cards are well shuffled and then one card is drawn at random. Giving ace a value 1 similar value for other cards, find the probability that the card has a value 
		\begin{enumerate}
			\item 7
			\item greater than 7
			\item less than 7
		\end{enumerate}
		%\input{exemplar/10/13/3/30/main.tex}
  \item A Lot consists of 48 mobile phones of which 42 are good, 3 have only minor defects and 3 have major defects.Varnika will buy a phone if it is good but the trader will only buy a mobile if it has no major defects. One phone is selected at random from the lot. What is the probability that it is
\begin{enumerate}
	\item acceptable to Varnika?
            \item acceptable to the trader?
\end{enumerate}
\solution
	%\input{exemplar/10/13/3/40/main.tex}
 \item A student says that if you throw a die, it will show up 1 or not 1. Therefore, the probability of getting 1 and the probability of getting 'not 1' each is equal to $\frac{1}{2}$. Is this correct? Give reasons.\\
 \solution
        %\input{exemplar/10/13/2/9/main.tex}
   \item Four candidates A, B, C, D have ap-
plied for the assignment to coach a school cricket
team. If A is twice as likely to be selected as B, and
B and C are given about the same chance of being
selected, while C is twice as likely to be selected
as D, what are the probabilities that
\begin{enumerate}
\item C will be selected?
\item A will not be selected?
\end{enumerate}
	%\input{exemplar/11/16/3/9/main.tex}
 \item A bag contain 24 balls of which $x$ balls are red, $2x$ are white and $3x$ are blue. A ball is selected at random, What is the probability that it is
\begin{enumerate}[label=\alph*)]
\item not red ?
\item white ?
\end{enumerate}
%\input{exemplar/10/13/3/41/main.tex}
If the letters of the word ASSASSINATION are arranged at random. Find the Probability that
\begin{enumerate}[label=(\alph*)]
\item Four $S's$ come consecutively in the word
\item Two  $I's$ and two $N's$ come together
\item All $A's$ are not coming together
\item No two $A's$ are coming together
\end{enumerate}
%\input{exemplar/11/16/3/14/main.tex}
	\item One urn contains two black balls (labelled B1 and B2) and one white ball. A
	second urn contains one black ball and two white balls (labelled W1 and W2).
	Suppose the following experiment is performed. One of the two urns is chosen
	at random. Next a ball is randomly chosen from the urn. Then a second ball is
	chosen at random from the same urn without replacing the first ball.
	
	\begin{enumerate}
	\item What is the probability that two black balls are chosen?
	
	\item What is the probability that two balls of opposite colour are chosen?
	\end{enumerate}
	\solution
	%\input{exemplar/11/16/3/12/main1.tex}
\end{enumerate}

\item In a certain lottery 10,000 tickets are sold and ten equal prizes are awarded. What is the probability of not getting a prize if you buy (a) one ticket (b) two tickets (c) 10 tickets ?	
\\
\solution
		%\begin{enumerate}[label=\thesection.\arabic*,ref=\thesection.\theenumi]
	\item One card is drawn from a well-shuffled deck of 52 cards. Find the probability of getting
\begin{enumerate}
\item A king of red colour 
\item A face card 
\item A red face card
\item The jack of hearts
\item A spade
\item The queen of diamonds

\end{enumerate}
\solution
		%\input{ncert/10/15/1/14/main.tex}
	\item Five cards—the ten, jack, queen, king and ace of diamonds, are well-shuffled with their face downwards. One card is then picked up at random.
\begin{enumerate}
\item
What is the probability that the card is the queen? 
\item
If the queen is drawn and put aside, what is the probability that the second card picked up is (a) an ace? (b) a queen?\\
\end{enumerate}
\solution
		%\input{ncert/10/15/1/15/defs.tex}
	\item A bag contains $5$ red balls and some blue balls. If the probability of drawing a blue ball is double that if a red ball, determine the number of blue balls in the bag. 
		\\
\solution
		%\input{ncert/10/15/2/3/defs.tex}
	\item A card is selected from a pack of 52 cards.
 \begin{enumerate}[label=(\alph*)] 
                 \item How many points are there in the sample space?
                 \item Calculate the probability that the card is an ace of spades.
                 \item Calculate the probability that the card is (i) an ace and (ii) black card.
 \end{enumerate}
\solution
		%\input{ncert/11/16/3/4/main.tex}
\item Four cards are drawn from a well-shuffled deck of 52 cards. What is the probability of obtaining 3 diamonds and one spade.
\\
\solution
		%\input{ncert/11/16/4/2/defs.tex}
\item In a certain lottery 10,000 tickets are sold and ten equal prizes are awarded. What is the probability of not getting a prize if you buy (a) one ticket (b) two tickets (c) 10 tickets ?	
\\
\solution
		%\input{ncert/11/16/4/4/defs.tex}
		%
\item 
Out of 100 students, two sections of 40 and 60 are formed. If you and your friend are among the 100 students, what is the probability that
\begin{enumerate}
\item you both enter the same section?
\item you both enter the different sections?
\end{enumerate}
\solution
		%\input{ncert/11/16/4/5/defs.tex}
	\item 
The number lock of a suitcase has 4 wheels each labelled with ten digits i.e. from 0 to 9.The lock opens with a sequence of four digits with no repeats.What is the probability of a person getting the right sequence to open the suitcase.
\\
\solution
		%\input{ncert/11/16/4/10/defs.tex}
		%
\item 
Two cards are drawn at random and without replacement from a pack of 52 playing cards. Find the probability that both the cards are black.
\\
\solution
		%\input{ncert/12/13/2/2/defs.tex}
		\item A box of oranges is inspected by examining three randomly selected oranges drawn without replacement. If all the three oranges are good, the box is approved for sale, otherwise, it is rejected. Find the probability that a box containing 15 oranges out of which 12 are good and 3 are bad ones will be approved for sale.
		\label{ncert/12/13/2/3/defs.tex}
		\item Two balls are drawn at random with replacement from a box containing 10 black and 8 red balls. Find the probability that
		\label{ncert/12/13/2/12}
\begin{enumerate}
\item both balls are red.
\item first ball is black and second is red.
\item one of them is black and other is red.
\end{enumerate}

\item In a hostel, 60\% of the students read Hindi newspaper, 40\% read English newspaper and 20\% read both Hindi and English newspapers. A student is selected at random.
		\label{ncert/12/13/2/15}
\begin{enumerate}
\item Find the probability that she reads neither Hindi nor English newspapers.
\item If she reads Hindi newspaper, find the probability that she reads English newspaper.
\item If she reads English newspaper, find the probability that she reads Hindi newspaper.\\
\end{enumerate}
\item The probability of obtaining an even prime number on each die, when a pair of dice is rolled is 
\begin{enumerate}
    \item $0$ 
    
    \item $\frac{1}{3}$ 
    
    \item $\frac{1}{12}$ 
    
    \item $\frac{1}{36}$ 
\end{enumerate}
\solution
		%\input{ncert/12/13/2/17/defs.tex}
	\item A bag contains 4 red and 4 black balls, another bag contains 2 red and 6 black balls. One of the two bags is selected at random and a ball is drawn from the bag which is found to be red. Find the probability that the ball is drawn from the first bag.
\\
\solution
		%\input{ncert/12/13/3/2/main.tex}
  \item
  Cards with numbers 2 to 101 are placed in a box. A card is selected at random.Find the probability that the card has
\begin{enumerate}[label=(\roman*)]
	\item an even number 
	\item a square number
\end{enumerate}
\solution
%\input{exemplar/10/13/3/32/main.tex}
\item
The king, queen and jack of clubs are removed from a deck of 52 playing cards and then well shuffled. Now one card is drawn at random from the remaining cards.  Determine the probability that the card is
\begin{enumerate}[label=(\roman*)]
\item a club
\item 10 of hearts
\end{enumerate}
\solution
%\input{exemplar/10/13/3/29/main.tex}
\item A team of medical students doing their internship have to assist during surgeries
at a city hospital. The probabilities of surgeries rated as very complex, complex,
routine, simple or very simple are respectively, 0.15, 0.20, 0.31, 0.26, .08. Find
the probabilities that a particular surgery will be rated
\begin{enumerate}
	\item complex or very complex;
	\item neither very complex nor very simple;
	\item routine or complex
	\item routine or simple
\end{enumerate}
\solution
%\input{exemplar/11/16/3/8(1)/main.tex}
\item A card is selected from a pack of 52 cards.
\begin{enumerate}[label=(\alph*)]
    \item How many points are there in the sample space?
    \item Calculate the probability that the card is an ace of spades.
    \item Calculate the probability that the card is (i) an ace and (ii) black card.
\end{enumerate}
\solution
%\input{exemplar/11/16/3/4/main2.tex}
\item The probability that a non leap year selected at random will contain 53 sundays.
\\
\solution
%\input{exemplar/10/13/1/19/main.tex}
\item One of the four persons John, Rita, Aslam or Gurpreet will be promoted next
month. Consequently the sample space consists of four elementary outcomes
S = {John promoted, Rita promoted, Aslam promoted, Gurpreet promoted}
You are told that the chances of John’s promotion is same as that of Gurpreet,
Rita’s chances of promotion are twice as likely as Johns. Aslam’s chances are
four times that of John.
\begin{enumerate}
	\item Determine
	\begin{enumerate}
		\item P (John promoted)
		\item P (Rita promoted)
		\item P (Aslam promoted)
		\item P (Gurpreet promoted)
	\end{enumerate}
	\item If A = {John promoted or Gurpreet promoted}, find P (A).
\end{enumerate}
\solution
%\input{exemplar/11/16/3/10/main.tex}
\item A card is drawn from a deck of 52 cards. Find the probability of getting a king or a heart or a red card.\\
\solution
%\input{exemplar/11/16/3/15/main.tex}
\item The probability that a student will pass his examination is 0.73, the probability of
the student getting a compartment is 0.13, and the probability that the student will
either pass or get compartment is 0.96. State True or False.\\
\solution
%\input{exemplar/11/16/3/31/main.tex}
\item A card is selected from a pack of 52 cards\\
\begin{enumerate}[label=(\alph*)]
\item How many points are there in the sample space?
\item Calculate the probability that the cards is an ace of spades.
\item Calculate the probability that the card is (i) an ace (ii)black card.\\
\end{enumerate}
%\input{ncert/11/16/3/4_1/Prob_4.tex}
\item In a non-leap year, the probability of having 53 tuesdays or 53 wednesdays is\\
\solution
%\input{exemplar/11/16/3/18/main.tex}
\item There are 1000 sealed envelopes in a box, 10 of them contain a cash prize of
Rs 100 each, 100 of them contain a cash prize of Rs 50 each and 200 of them
contain a cash prize of Rs 10 each and rest do not contain any cash prize. If they
are well shuffled and an envelope is picked up out, what is the probability that it
contains no cash prize?\\
\solution
%\input{exemplar/10/13/3/34/main.tex}
\item 
A die is thrown and a card is selected at random from a deck of 52 playing cards. The probability of getting an even number on the die and a spade card.\\
\solution
%\input{exemplar/12/13/3/78/main.tex}
\item
If 4-digit numbers greater than 5,000 are randomly formed from the digits 0, 1, 3, 5, and 7, what is the probability of forming a number divisible by 5 when:
\begin{enumerate}
    \item The digits are repeated?
    \item The repetition of digits is not allowed?
\end{enumerate}
\solution
%\input{ncert/11/16/4/9/main.tex}
\item Consider the probability space $\brak{\Omega, \mathcal{G}, P}$ where $\Omega = [0,2]$ and $\mathcal{G} = \cbrak{\phi, \Omega, [0,1], (1,2]}$. Let $X$ and $Y$ be two functions on $\Omega$ defined as
\begin{align*}
    X(\omega) = 
    \begin{cases}
        1 & \text{if }\omega \in [0, 1]\\
        2 & \text{if }\omega \in (1, 2]
    \end{cases}
\end{align*}
and
\begin{align*}
    Y(\omega) = 
    \begin{cases}
        2 & \text{if }\omega \in [0, 1.5]\\
        3 & \text{if }\omega \in (1.5, 2].
    \end{cases}
\end{align*}
Then which one of the following statements is true?
\begin{enumerate}
    \item [(A)] $X$ is a random variable with respect to $\mathcal{G}$, but $Y$ is not a random variable with respect to $\mathcal{G}$.
    \item [(B)] $Y$ is a random variable with respect to $\mathcal{G}$, but $X$ is not a random variable with respect to $\mathcal{G}$.
    \item [(C)] Neither $X$ nor $Y$ is a random variable with respect to $\mathcal{G}$.
    \item [(D)] Both $X$ and $Y$ are random variables with respect to $\mathcal{G}$.
\end{enumerate} \hfill (GATE ST 2023)\\
\solution
%\input{gate/ST/2023/14/main.tex}
	\item  A die is loaded in such a way that each odd number is twice as likely to occur as
each even number. Find $P(G)$, where $G$ is the event that a number greater than
3 occurs on a single roll of the die.
\\
\solution
		%\input{exemplar/11/16/3/5/main.tex}
	\item All the jacks, queens and kings are removed from a deck of 52 playing cards. The remaining cards are well shuffled and then one card is drawn at random. Giving ace a value 1 similar value for other cards, find the probability that the card has a value 
		\begin{enumerate}
			\item 7
			\item greater than 7
			\item less than 7
		\end{enumerate}
		%\input{exemplar/10/13/3/30/main.tex}
  \item A Lot consists of 48 mobile phones of which 42 are good, 3 have only minor defects and 3 have major defects.Varnika will buy a phone if it is good but the trader will only buy a mobile if it has no major defects. One phone is selected at random from the lot. What is the probability that it is
\begin{enumerate}
	\item acceptable to Varnika?
            \item acceptable to the trader?
\end{enumerate}
\solution
	%\input{exemplar/10/13/3/40/main.tex}
 \item A student says that if you throw a die, it will show up 1 or not 1. Therefore, the probability of getting 1 and the probability of getting 'not 1' each is equal to $\frac{1}{2}$. Is this correct? Give reasons.\\
 \solution
        %\input{exemplar/10/13/2/9/main.tex}
   \item Four candidates A, B, C, D have ap-
plied for the assignment to coach a school cricket
team. If A is twice as likely to be selected as B, and
B and C are given about the same chance of being
selected, while C is twice as likely to be selected
as D, what are the probabilities that
\begin{enumerate}
\item C will be selected?
\item A will not be selected?
\end{enumerate}
	%\input{exemplar/11/16/3/9/main.tex}
 \item A bag contain 24 balls of which $x$ balls are red, $2x$ are white and $3x$ are blue. A ball is selected at random, What is the probability that it is
\begin{enumerate}[label=\alph*)]
\item not red ?
\item white ?
\end{enumerate}
%\input{exemplar/10/13/3/41/main.tex}
If the letters of the word ASSASSINATION are arranged at random. Find the Probability that
\begin{enumerate}[label=(\alph*)]
\item Four $S's$ come consecutively in the word
\item Two  $I's$ and two $N's$ come together
\item All $A's$ are not coming together
\item No two $A's$ are coming together
\end{enumerate}
%\input{exemplar/11/16/3/14/main.tex}
	\item One urn contains two black balls (labelled B1 and B2) and one white ball. A
	second urn contains one black ball and two white balls (labelled W1 and W2).
	Suppose the following experiment is performed. One of the two urns is chosen
	at random. Next a ball is randomly chosen from the urn. Then a second ball is
	chosen at random from the same urn without replacing the first ball.
	
	\begin{enumerate}
	\item What is the probability that two black balls are chosen?
	
	\item What is the probability that two balls of opposite colour are chosen?
	\end{enumerate}
	\solution
	%\input{exemplar/11/16/3/12/main1.tex}
\end{enumerate}

		%
\item 
Out of 100 students, two sections of 40 and 60 are formed. If you and your friend are among the 100 students, what is the probability that
\begin{enumerate}
\item you both enter the same section?
\item you both enter the different sections?
\end{enumerate}
\solution
		%\begin{enumerate}[label=\thesection.\arabic*,ref=\thesection.\theenumi]
	\item One card is drawn from a well-shuffled deck of 52 cards. Find the probability of getting
\begin{enumerate}
\item A king of red colour 
\item A face card 
\item A red face card
\item The jack of hearts
\item A spade
\item The queen of diamonds

\end{enumerate}
\solution
		%\input{ncert/10/15/1/14/main.tex}
	\item Five cards—the ten, jack, queen, king and ace of diamonds, are well-shuffled with their face downwards. One card is then picked up at random.
\begin{enumerate}
\item
What is the probability that the card is the queen? 
\item
If the queen is drawn and put aside, what is the probability that the second card picked up is (a) an ace? (b) a queen?\\
\end{enumerate}
\solution
		%\input{ncert/10/15/1/15/defs.tex}
	\item A bag contains $5$ red balls and some blue balls. If the probability of drawing a blue ball is double that if a red ball, determine the number of blue balls in the bag. 
		\\
\solution
		%\input{ncert/10/15/2/3/defs.tex}
	\item A card is selected from a pack of 52 cards.
 \begin{enumerate}[label=(\alph*)] 
                 \item How many points are there in the sample space?
                 \item Calculate the probability that the card is an ace of spades.
                 \item Calculate the probability that the card is (i) an ace and (ii) black card.
 \end{enumerate}
\solution
		%\input{ncert/11/16/3/4/main.tex}
\item Four cards are drawn from a well-shuffled deck of 52 cards. What is the probability of obtaining 3 diamonds and one spade.
\\
\solution
		%\input{ncert/11/16/4/2/defs.tex}
\item In a certain lottery 10,000 tickets are sold and ten equal prizes are awarded. What is the probability of not getting a prize if you buy (a) one ticket (b) two tickets (c) 10 tickets ?	
\\
\solution
		%\input{ncert/11/16/4/4/defs.tex}
		%
\item 
Out of 100 students, two sections of 40 and 60 are formed. If you and your friend are among the 100 students, what is the probability that
\begin{enumerate}
\item you both enter the same section?
\item you both enter the different sections?
\end{enumerate}
\solution
		%\input{ncert/11/16/4/5/defs.tex}
	\item 
The number lock of a suitcase has 4 wheels each labelled with ten digits i.e. from 0 to 9.The lock opens with a sequence of four digits with no repeats.What is the probability of a person getting the right sequence to open the suitcase.
\\
\solution
		%\input{ncert/11/16/4/10/defs.tex}
		%
\item 
Two cards are drawn at random and without replacement from a pack of 52 playing cards. Find the probability that both the cards are black.
\\
\solution
		%\input{ncert/12/13/2/2/defs.tex}
		\item A box of oranges is inspected by examining three randomly selected oranges drawn without replacement. If all the three oranges are good, the box is approved for sale, otherwise, it is rejected. Find the probability that a box containing 15 oranges out of which 12 are good and 3 are bad ones will be approved for sale.
		\label{ncert/12/13/2/3/defs.tex}
		\item Two balls are drawn at random with replacement from a box containing 10 black and 8 red balls. Find the probability that
		\label{ncert/12/13/2/12}
\begin{enumerate}
\item both balls are red.
\item first ball is black and second is red.
\item one of them is black and other is red.
\end{enumerate}

\item In a hostel, 60\% of the students read Hindi newspaper, 40\% read English newspaper and 20\% read both Hindi and English newspapers. A student is selected at random.
		\label{ncert/12/13/2/15}
\begin{enumerate}
\item Find the probability that she reads neither Hindi nor English newspapers.
\item If she reads Hindi newspaper, find the probability that she reads English newspaper.
\item If she reads English newspaper, find the probability that she reads Hindi newspaper.\\
\end{enumerate}
\item The probability of obtaining an even prime number on each die, when a pair of dice is rolled is 
\begin{enumerate}
    \item $0$ 
    
    \item $\frac{1}{3}$ 
    
    \item $\frac{1}{12}$ 
    
    \item $\frac{1}{36}$ 
\end{enumerate}
\solution
		%\input{ncert/12/13/2/17/defs.tex}
	\item A bag contains 4 red and 4 black balls, another bag contains 2 red and 6 black balls. One of the two bags is selected at random and a ball is drawn from the bag which is found to be red. Find the probability that the ball is drawn from the first bag.
\\
\solution
		%\input{ncert/12/13/3/2/main.tex}
  \item
  Cards with numbers 2 to 101 are placed in a box. A card is selected at random.Find the probability that the card has
\begin{enumerate}[label=(\roman*)]
	\item an even number 
	\item a square number
\end{enumerate}
\solution
%\input{exemplar/10/13/3/32/main.tex}
\item
The king, queen and jack of clubs are removed from a deck of 52 playing cards and then well shuffled. Now one card is drawn at random from the remaining cards.  Determine the probability that the card is
\begin{enumerate}[label=(\roman*)]
\item a club
\item 10 of hearts
\end{enumerate}
\solution
%\input{exemplar/10/13/3/29/main.tex}
\item A team of medical students doing their internship have to assist during surgeries
at a city hospital. The probabilities of surgeries rated as very complex, complex,
routine, simple or very simple are respectively, 0.15, 0.20, 0.31, 0.26, .08. Find
the probabilities that a particular surgery will be rated
\begin{enumerate}
	\item complex or very complex;
	\item neither very complex nor very simple;
	\item routine or complex
	\item routine or simple
\end{enumerate}
\solution
%\input{exemplar/11/16/3/8(1)/main.tex}
\item A card is selected from a pack of 52 cards.
\begin{enumerate}[label=(\alph*)]
    \item How many points are there in the sample space?
    \item Calculate the probability that the card is an ace of spades.
    \item Calculate the probability that the card is (i) an ace and (ii) black card.
\end{enumerate}
\solution
%\input{exemplar/11/16/3/4/main2.tex}
\item The probability that a non leap year selected at random will contain 53 sundays.
\\
\solution
%\input{exemplar/10/13/1/19/main.tex}
\item One of the four persons John, Rita, Aslam or Gurpreet will be promoted next
month. Consequently the sample space consists of four elementary outcomes
S = {John promoted, Rita promoted, Aslam promoted, Gurpreet promoted}
You are told that the chances of John’s promotion is same as that of Gurpreet,
Rita’s chances of promotion are twice as likely as Johns. Aslam’s chances are
four times that of John.
\begin{enumerate}
	\item Determine
	\begin{enumerate}
		\item P (John promoted)
		\item P (Rita promoted)
		\item P (Aslam promoted)
		\item P (Gurpreet promoted)
	\end{enumerate}
	\item If A = {John promoted or Gurpreet promoted}, find P (A).
\end{enumerate}
\solution
%\input{exemplar/11/16/3/10/main.tex}
\item A card is drawn from a deck of 52 cards. Find the probability of getting a king or a heart or a red card.\\
\solution
%\input{exemplar/11/16/3/15/main.tex}
\item The probability that a student will pass his examination is 0.73, the probability of
the student getting a compartment is 0.13, and the probability that the student will
either pass or get compartment is 0.96. State True or False.\\
\solution
%\input{exemplar/11/16/3/31/main.tex}
\item A card is selected from a pack of 52 cards\\
\begin{enumerate}[label=(\alph*)]
\item How many points are there in the sample space?
\item Calculate the probability that the cards is an ace of spades.
\item Calculate the probability that the card is (i) an ace (ii)black card.\\
\end{enumerate}
%\input{ncert/11/16/3/4_1/Prob_4.tex}
\item In a non-leap year, the probability of having 53 tuesdays or 53 wednesdays is\\
\solution
%\input{exemplar/11/16/3/18/main.tex}
\item There are 1000 sealed envelopes in a box, 10 of them contain a cash prize of
Rs 100 each, 100 of them contain a cash prize of Rs 50 each and 200 of them
contain a cash prize of Rs 10 each and rest do not contain any cash prize. If they
are well shuffled and an envelope is picked up out, what is the probability that it
contains no cash prize?\\
\solution
%\input{exemplar/10/13/3/34/main.tex}
\item 
A die is thrown and a card is selected at random from a deck of 52 playing cards. The probability of getting an even number on the die and a spade card.\\
\solution
%\input{exemplar/12/13/3/78/main.tex}
\item
If 4-digit numbers greater than 5,000 are randomly formed from the digits 0, 1, 3, 5, and 7, what is the probability of forming a number divisible by 5 when:
\begin{enumerate}
    \item The digits are repeated?
    \item The repetition of digits is not allowed?
\end{enumerate}
\solution
%\input{ncert/11/16/4/9/main.tex}
\item Consider the probability space $\brak{\Omega, \mathcal{G}, P}$ where $\Omega = [0,2]$ and $\mathcal{G} = \cbrak{\phi, \Omega, [0,1], (1,2]}$. Let $X$ and $Y$ be two functions on $\Omega$ defined as
\begin{align*}
    X(\omega) = 
    \begin{cases}
        1 & \text{if }\omega \in [0, 1]\\
        2 & \text{if }\omega \in (1, 2]
    \end{cases}
\end{align*}
and
\begin{align*}
    Y(\omega) = 
    \begin{cases}
        2 & \text{if }\omega \in [0, 1.5]\\
        3 & \text{if }\omega \in (1.5, 2].
    \end{cases}
\end{align*}
Then which one of the following statements is true?
\begin{enumerate}
    \item [(A)] $X$ is a random variable with respect to $\mathcal{G}$, but $Y$ is not a random variable with respect to $\mathcal{G}$.
    \item [(B)] $Y$ is a random variable with respect to $\mathcal{G}$, but $X$ is not a random variable with respect to $\mathcal{G}$.
    \item [(C)] Neither $X$ nor $Y$ is a random variable with respect to $\mathcal{G}$.
    \item [(D)] Both $X$ and $Y$ are random variables with respect to $\mathcal{G}$.
\end{enumerate} \hfill (GATE ST 2023)\\
\solution
%\input{gate/ST/2023/14/main.tex}
	\item  A die is loaded in such a way that each odd number is twice as likely to occur as
each even number. Find $P(G)$, where $G$ is the event that a number greater than
3 occurs on a single roll of the die.
\\
\solution
		%\input{exemplar/11/16/3/5/main.tex}
	\item All the jacks, queens and kings are removed from a deck of 52 playing cards. The remaining cards are well shuffled and then one card is drawn at random. Giving ace a value 1 similar value for other cards, find the probability that the card has a value 
		\begin{enumerate}
			\item 7
			\item greater than 7
			\item less than 7
		\end{enumerate}
		%\input{exemplar/10/13/3/30/main.tex}
  \item A Lot consists of 48 mobile phones of which 42 are good, 3 have only minor defects and 3 have major defects.Varnika will buy a phone if it is good but the trader will only buy a mobile if it has no major defects. One phone is selected at random from the lot. What is the probability that it is
\begin{enumerate}
	\item acceptable to Varnika?
            \item acceptable to the trader?
\end{enumerate}
\solution
	%\input{exemplar/10/13/3/40/main.tex}
 \item A student says that if you throw a die, it will show up 1 or not 1. Therefore, the probability of getting 1 and the probability of getting 'not 1' each is equal to $\frac{1}{2}$. Is this correct? Give reasons.\\
 \solution
        %\input{exemplar/10/13/2/9/main.tex}
   \item Four candidates A, B, C, D have ap-
plied for the assignment to coach a school cricket
team. If A is twice as likely to be selected as B, and
B and C are given about the same chance of being
selected, while C is twice as likely to be selected
as D, what are the probabilities that
\begin{enumerate}
\item C will be selected?
\item A will not be selected?
\end{enumerate}
	%\input{exemplar/11/16/3/9/main.tex}
 \item A bag contain 24 balls of which $x$ balls are red, $2x$ are white and $3x$ are blue. A ball is selected at random, What is the probability that it is
\begin{enumerate}[label=\alph*)]
\item not red ?
\item white ?
\end{enumerate}
%\input{exemplar/10/13/3/41/main.tex}
If the letters of the word ASSASSINATION are arranged at random. Find the Probability that
\begin{enumerate}[label=(\alph*)]
\item Four $S's$ come consecutively in the word
\item Two  $I's$ and two $N's$ come together
\item All $A's$ are not coming together
\item No two $A's$ are coming together
\end{enumerate}
%\input{exemplar/11/16/3/14/main.tex}
	\item One urn contains two black balls (labelled B1 and B2) and one white ball. A
	second urn contains one black ball and two white balls (labelled W1 and W2).
	Suppose the following experiment is performed. One of the two urns is chosen
	at random. Next a ball is randomly chosen from the urn. Then a second ball is
	chosen at random from the same urn without replacing the first ball.
	
	\begin{enumerate}
	\item What is the probability that two black balls are chosen?
	
	\item What is the probability that two balls of opposite colour are chosen?
	\end{enumerate}
	\solution
	%\input{exemplar/11/16/3/12/main1.tex}
\end{enumerate}

	\item 
The number lock of a suitcase has 4 wheels each labelled with ten digits i.e. from 0 to 9.The lock opens with a sequence of four digits with no repeats.What is the probability of a person getting the right sequence to open the suitcase.
\\
\solution
		%\begin{enumerate}[label=\thesection.\arabic*,ref=\thesection.\theenumi]
	\item One card is drawn from a well-shuffled deck of 52 cards. Find the probability of getting
\begin{enumerate}
\item A king of red colour 
\item A face card 
\item A red face card
\item The jack of hearts
\item A spade
\item The queen of diamonds

\end{enumerate}
\solution
		%\input{ncert/10/15/1/14/main.tex}
	\item Five cards—the ten, jack, queen, king and ace of diamonds, are well-shuffled with their face downwards. One card is then picked up at random.
\begin{enumerate}
\item
What is the probability that the card is the queen? 
\item
If the queen is drawn and put aside, what is the probability that the second card picked up is (a) an ace? (b) a queen?\\
\end{enumerate}
\solution
		%\input{ncert/10/15/1/15/defs.tex}
	\item A bag contains $5$ red balls and some blue balls. If the probability of drawing a blue ball is double that if a red ball, determine the number of blue balls in the bag. 
		\\
\solution
		%\input{ncert/10/15/2/3/defs.tex}
	\item A card is selected from a pack of 52 cards.
 \begin{enumerate}[label=(\alph*)] 
                 \item How many points are there in the sample space?
                 \item Calculate the probability that the card is an ace of spades.
                 \item Calculate the probability that the card is (i) an ace and (ii) black card.
 \end{enumerate}
\solution
		%\input{ncert/11/16/3/4/main.tex}
\item Four cards are drawn from a well-shuffled deck of 52 cards. What is the probability of obtaining 3 diamonds and one spade.
\\
\solution
		%\input{ncert/11/16/4/2/defs.tex}
\item In a certain lottery 10,000 tickets are sold and ten equal prizes are awarded. What is the probability of not getting a prize if you buy (a) one ticket (b) two tickets (c) 10 tickets ?	
\\
\solution
		%\input{ncert/11/16/4/4/defs.tex}
		%
\item 
Out of 100 students, two sections of 40 and 60 are formed. If you and your friend are among the 100 students, what is the probability that
\begin{enumerate}
\item you both enter the same section?
\item you both enter the different sections?
\end{enumerate}
\solution
		%\input{ncert/11/16/4/5/defs.tex}
	\item 
The number lock of a suitcase has 4 wheels each labelled with ten digits i.e. from 0 to 9.The lock opens with a sequence of four digits with no repeats.What is the probability of a person getting the right sequence to open the suitcase.
\\
\solution
		%\input{ncert/11/16/4/10/defs.tex}
		%
\item 
Two cards are drawn at random and without replacement from a pack of 52 playing cards. Find the probability that both the cards are black.
\\
\solution
		%\input{ncert/12/13/2/2/defs.tex}
		\item A box of oranges is inspected by examining three randomly selected oranges drawn without replacement. If all the three oranges are good, the box is approved for sale, otherwise, it is rejected. Find the probability that a box containing 15 oranges out of which 12 are good and 3 are bad ones will be approved for sale.
		\label{ncert/12/13/2/3/defs.tex}
		\item Two balls are drawn at random with replacement from a box containing 10 black and 8 red balls. Find the probability that
		\label{ncert/12/13/2/12}
\begin{enumerate}
\item both balls are red.
\item first ball is black and second is red.
\item one of them is black and other is red.
\end{enumerate}

\item In a hostel, 60\% of the students read Hindi newspaper, 40\% read English newspaper and 20\% read both Hindi and English newspapers. A student is selected at random.
		\label{ncert/12/13/2/15}
\begin{enumerate}
\item Find the probability that she reads neither Hindi nor English newspapers.
\item If she reads Hindi newspaper, find the probability that she reads English newspaper.
\item If she reads English newspaper, find the probability that she reads Hindi newspaper.\\
\end{enumerate}
\item The probability of obtaining an even prime number on each die, when a pair of dice is rolled is 
\begin{enumerate}
    \item $0$ 
    
    \item $\frac{1}{3}$ 
    
    \item $\frac{1}{12}$ 
    
    \item $\frac{1}{36}$ 
\end{enumerate}
\solution
		%\input{ncert/12/13/2/17/defs.tex}
	\item A bag contains 4 red and 4 black balls, another bag contains 2 red and 6 black balls. One of the two bags is selected at random and a ball is drawn from the bag which is found to be red. Find the probability that the ball is drawn from the first bag.
\\
\solution
		%\input{ncert/12/13/3/2/main.tex}
  \item
  Cards with numbers 2 to 101 are placed in a box. A card is selected at random.Find the probability that the card has
\begin{enumerate}[label=(\roman*)]
	\item an even number 
	\item a square number
\end{enumerate}
\solution
%\input{exemplar/10/13/3/32/main.tex}
\item
The king, queen and jack of clubs are removed from a deck of 52 playing cards and then well shuffled. Now one card is drawn at random from the remaining cards.  Determine the probability that the card is
\begin{enumerate}[label=(\roman*)]
\item a club
\item 10 of hearts
\end{enumerate}
\solution
%\input{exemplar/10/13/3/29/main.tex}
\item A team of medical students doing their internship have to assist during surgeries
at a city hospital. The probabilities of surgeries rated as very complex, complex,
routine, simple or very simple are respectively, 0.15, 0.20, 0.31, 0.26, .08. Find
the probabilities that a particular surgery will be rated
\begin{enumerate}
	\item complex or very complex;
	\item neither very complex nor very simple;
	\item routine or complex
	\item routine or simple
\end{enumerate}
\solution
%\input{exemplar/11/16/3/8(1)/main.tex}
\item A card is selected from a pack of 52 cards.
\begin{enumerate}[label=(\alph*)]
    \item How many points are there in the sample space?
    \item Calculate the probability that the card is an ace of spades.
    \item Calculate the probability that the card is (i) an ace and (ii) black card.
\end{enumerate}
\solution
%\input{exemplar/11/16/3/4/main2.tex}
\item The probability that a non leap year selected at random will contain 53 sundays.
\\
\solution
%\input{exemplar/10/13/1/19/main.tex}
\item One of the four persons John, Rita, Aslam or Gurpreet will be promoted next
month. Consequently the sample space consists of four elementary outcomes
S = {John promoted, Rita promoted, Aslam promoted, Gurpreet promoted}
You are told that the chances of John’s promotion is same as that of Gurpreet,
Rita’s chances of promotion are twice as likely as Johns. Aslam’s chances are
four times that of John.
\begin{enumerate}
	\item Determine
	\begin{enumerate}
		\item P (John promoted)
		\item P (Rita promoted)
		\item P (Aslam promoted)
		\item P (Gurpreet promoted)
	\end{enumerate}
	\item If A = {John promoted or Gurpreet promoted}, find P (A).
\end{enumerate}
\solution
%\input{exemplar/11/16/3/10/main.tex}
\item A card is drawn from a deck of 52 cards. Find the probability of getting a king or a heart or a red card.\\
\solution
%\input{exemplar/11/16/3/15/main.tex}
\item The probability that a student will pass his examination is 0.73, the probability of
the student getting a compartment is 0.13, and the probability that the student will
either pass or get compartment is 0.96. State True or False.\\
\solution
%\input{exemplar/11/16/3/31/main.tex}
\item A card is selected from a pack of 52 cards\\
\begin{enumerate}[label=(\alph*)]
\item How many points are there in the sample space?
\item Calculate the probability that the cards is an ace of spades.
\item Calculate the probability that the card is (i) an ace (ii)black card.\\
\end{enumerate}
%\input{ncert/11/16/3/4_1/Prob_4.tex}
\item In a non-leap year, the probability of having 53 tuesdays or 53 wednesdays is\\
\solution
%\input{exemplar/11/16/3/18/main.tex}
\item There are 1000 sealed envelopes in a box, 10 of them contain a cash prize of
Rs 100 each, 100 of them contain a cash prize of Rs 50 each and 200 of them
contain a cash prize of Rs 10 each and rest do not contain any cash prize. If they
are well shuffled and an envelope is picked up out, what is the probability that it
contains no cash prize?\\
\solution
%\input{exemplar/10/13/3/34/main.tex}
\item 
A die is thrown and a card is selected at random from a deck of 52 playing cards. The probability of getting an even number on the die and a spade card.\\
\solution
%\input{exemplar/12/13/3/78/main.tex}
\item
If 4-digit numbers greater than 5,000 are randomly formed from the digits 0, 1, 3, 5, and 7, what is the probability of forming a number divisible by 5 when:
\begin{enumerate}
    \item The digits are repeated?
    \item The repetition of digits is not allowed?
\end{enumerate}
\solution
%\input{ncert/11/16/4/9/main.tex}
\item Consider the probability space $\brak{\Omega, \mathcal{G}, P}$ where $\Omega = [0,2]$ and $\mathcal{G} = \cbrak{\phi, \Omega, [0,1], (1,2]}$. Let $X$ and $Y$ be two functions on $\Omega$ defined as
\begin{align*}
    X(\omega) = 
    \begin{cases}
        1 & \text{if }\omega \in [0, 1]\\
        2 & \text{if }\omega \in (1, 2]
    \end{cases}
\end{align*}
and
\begin{align*}
    Y(\omega) = 
    \begin{cases}
        2 & \text{if }\omega \in [0, 1.5]\\
        3 & \text{if }\omega \in (1.5, 2].
    \end{cases}
\end{align*}
Then which one of the following statements is true?
\begin{enumerate}
    \item [(A)] $X$ is a random variable with respect to $\mathcal{G}$, but $Y$ is not a random variable with respect to $\mathcal{G}$.
    \item [(B)] $Y$ is a random variable with respect to $\mathcal{G}$, but $X$ is not a random variable with respect to $\mathcal{G}$.
    \item [(C)] Neither $X$ nor $Y$ is a random variable with respect to $\mathcal{G}$.
    \item [(D)] Both $X$ and $Y$ are random variables with respect to $\mathcal{G}$.
\end{enumerate} \hfill (GATE ST 2023)\\
\solution
%\input{gate/ST/2023/14/main.tex}
	\item  A die is loaded in such a way that each odd number is twice as likely to occur as
each even number. Find $P(G)$, where $G$ is the event that a number greater than
3 occurs on a single roll of the die.
\\
\solution
		%\input{exemplar/11/16/3/5/main.tex}
	\item All the jacks, queens and kings are removed from a deck of 52 playing cards. The remaining cards are well shuffled and then one card is drawn at random. Giving ace a value 1 similar value for other cards, find the probability that the card has a value 
		\begin{enumerate}
			\item 7
			\item greater than 7
			\item less than 7
		\end{enumerate}
		%\input{exemplar/10/13/3/30/main.tex}
  \item A Lot consists of 48 mobile phones of which 42 are good, 3 have only minor defects and 3 have major defects.Varnika will buy a phone if it is good but the trader will only buy a mobile if it has no major defects. One phone is selected at random from the lot. What is the probability that it is
\begin{enumerate}
	\item acceptable to Varnika?
            \item acceptable to the trader?
\end{enumerate}
\solution
	%\input{exemplar/10/13/3/40/main.tex}
 \item A student says that if you throw a die, it will show up 1 or not 1. Therefore, the probability of getting 1 and the probability of getting 'not 1' each is equal to $\frac{1}{2}$. Is this correct? Give reasons.\\
 \solution
        %\input{exemplar/10/13/2/9/main.tex}
   \item Four candidates A, B, C, D have ap-
plied for the assignment to coach a school cricket
team. If A is twice as likely to be selected as B, and
B and C are given about the same chance of being
selected, while C is twice as likely to be selected
as D, what are the probabilities that
\begin{enumerate}
\item C will be selected?
\item A will not be selected?
\end{enumerate}
	%\input{exemplar/11/16/3/9/main.tex}
 \item A bag contain 24 balls of which $x$ balls are red, $2x$ are white and $3x$ are blue. A ball is selected at random, What is the probability that it is
\begin{enumerate}[label=\alph*)]
\item not red ?
\item white ?
\end{enumerate}
%\input{exemplar/10/13/3/41/main.tex}
If the letters of the word ASSASSINATION are arranged at random. Find the Probability that
\begin{enumerate}[label=(\alph*)]
\item Four $S's$ come consecutively in the word
\item Two  $I's$ and two $N's$ come together
\item All $A's$ are not coming together
\item No two $A's$ are coming together
\end{enumerate}
%\input{exemplar/11/16/3/14/main.tex}
	\item One urn contains two black balls (labelled B1 and B2) and one white ball. A
	second urn contains one black ball and two white balls (labelled W1 and W2).
	Suppose the following experiment is performed. One of the two urns is chosen
	at random. Next a ball is randomly chosen from the urn. Then a second ball is
	chosen at random from the same urn without replacing the first ball.
	
	\begin{enumerate}
	\item What is the probability that two black balls are chosen?
	
	\item What is the probability that two balls of opposite colour are chosen?
	\end{enumerate}
	\solution
	%\input{exemplar/11/16/3/12/main1.tex}
\end{enumerate}

		%
\item 
Two cards are drawn at random and without replacement from a pack of 52 playing cards. Find the probability that both the cards are black.
\\
\solution
		%\begin{enumerate}[label=\thesection.\arabic*,ref=\thesection.\theenumi]
	\item One card is drawn from a well-shuffled deck of 52 cards. Find the probability of getting
\begin{enumerate}
\item A king of red colour 
\item A face card 
\item A red face card
\item The jack of hearts
\item A spade
\item The queen of diamonds

\end{enumerate}
\solution
		%\input{ncert/10/15/1/14/main.tex}
	\item Five cards—the ten, jack, queen, king and ace of diamonds, are well-shuffled with their face downwards. One card is then picked up at random.
\begin{enumerate}
\item
What is the probability that the card is the queen? 
\item
If the queen is drawn and put aside, what is the probability that the second card picked up is (a) an ace? (b) a queen?\\
\end{enumerate}
\solution
		%\input{ncert/10/15/1/15/defs.tex}
	\item A bag contains $5$ red balls and some blue balls. If the probability of drawing a blue ball is double that if a red ball, determine the number of blue balls in the bag. 
		\\
\solution
		%\input{ncert/10/15/2/3/defs.tex}
	\item A card is selected from a pack of 52 cards.
 \begin{enumerate}[label=(\alph*)] 
                 \item How many points are there in the sample space?
                 \item Calculate the probability that the card is an ace of spades.
                 \item Calculate the probability that the card is (i) an ace and (ii) black card.
 \end{enumerate}
\solution
		%\input{ncert/11/16/3/4/main.tex}
\item Four cards are drawn from a well-shuffled deck of 52 cards. What is the probability of obtaining 3 diamonds and one spade.
\\
\solution
		%\input{ncert/11/16/4/2/defs.tex}
\item In a certain lottery 10,000 tickets are sold and ten equal prizes are awarded. What is the probability of not getting a prize if you buy (a) one ticket (b) two tickets (c) 10 tickets ?	
\\
\solution
		%\input{ncert/11/16/4/4/defs.tex}
		%
\item 
Out of 100 students, two sections of 40 and 60 are formed. If you and your friend are among the 100 students, what is the probability that
\begin{enumerate}
\item you both enter the same section?
\item you both enter the different sections?
\end{enumerate}
\solution
		%\input{ncert/11/16/4/5/defs.tex}
	\item 
The number lock of a suitcase has 4 wheels each labelled with ten digits i.e. from 0 to 9.The lock opens with a sequence of four digits with no repeats.What is the probability of a person getting the right sequence to open the suitcase.
\\
\solution
		%\input{ncert/11/16/4/10/defs.tex}
		%
\item 
Two cards are drawn at random and without replacement from a pack of 52 playing cards. Find the probability that both the cards are black.
\\
\solution
		%\input{ncert/12/13/2/2/defs.tex}
		\item A box of oranges is inspected by examining three randomly selected oranges drawn without replacement. If all the three oranges are good, the box is approved for sale, otherwise, it is rejected. Find the probability that a box containing 15 oranges out of which 12 are good and 3 are bad ones will be approved for sale.
		\label{ncert/12/13/2/3/defs.tex}
		\item Two balls are drawn at random with replacement from a box containing 10 black and 8 red balls. Find the probability that
		\label{ncert/12/13/2/12}
\begin{enumerate}
\item both balls are red.
\item first ball is black and second is red.
\item one of them is black and other is red.
\end{enumerate}

\item In a hostel, 60\% of the students read Hindi newspaper, 40\% read English newspaper and 20\% read both Hindi and English newspapers. A student is selected at random.
		\label{ncert/12/13/2/15}
\begin{enumerate}
\item Find the probability that she reads neither Hindi nor English newspapers.
\item If she reads Hindi newspaper, find the probability that she reads English newspaper.
\item If she reads English newspaper, find the probability that she reads Hindi newspaper.\\
\end{enumerate}
\item The probability of obtaining an even prime number on each die, when a pair of dice is rolled is 
\begin{enumerate}
    \item $0$ 
    
    \item $\frac{1}{3}$ 
    
    \item $\frac{1}{12}$ 
    
    \item $\frac{1}{36}$ 
\end{enumerate}
\solution
		%\input{ncert/12/13/2/17/defs.tex}
	\item A bag contains 4 red and 4 black balls, another bag contains 2 red and 6 black balls. One of the two bags is selected at random and a ball is drawn from the bag which is found to be red. Find the probability that the ball is drawn from the first bag.
\\
\solution
		%\input{ncert/12/13/3/2/main.tex}
  \item
  Cards with numbers 2 to 101 are placed in a box. A card is selected at random.Find the probability that the card has
\begin{enumerate}[label=(\roman*)]
	\item an even number 
	\item a square number
\end{enumerate}
\solution
%\input{exemplar/10/13/3/32/main.tex}
\item
The king, queen and jack of clubs are removed from a deck of 52 playing cards and then well shuffled. Now one card is drawn at random from the remaining cards.  Determine the probability that the card is
\begin{enumerate}[label=(\roman*)]
\item a club
\item 10 of hearts
\end{enumerate}
\solution
%\input{exemplar/10/13/3/29/main.tex}
\item A team of medical students doing their internship have to assist during surgeries
at a city hospital. The probabilities of surgeries rated as very complex, complex,
routine, simple or very simple are respectively, 0.15, 0.20, 0.31, 0.26, .08. Find
the probabilities that a particular surgery will be rated
\begin{enumerate}
	\item complex or very complex;
	\item neither very complex nor very simple;
	\item routine or complex
	\item routine or simple
\end{enumerate}
\solution
%\input{exemplar/11/16/3/8(1)/main.tex}
\item A card is selected from a pack of 52 cards.
\begin{enumerate}[label=(\alph*)]
    \item How many points are there in the sample space?
    \item Calculate the probability that the card is an ace of spades.
    \item Calculate the probability that the card is (i) an ace and (ii) black card.
\end{enumerate}
\solution
%\input{exemplar/11/16/3/4/main2.tex}
\item The probability that a non leap year selected at random will contain 53 sundays.
\\
\solution
%\input{exemplar/10/13/1/19/main.tex}
\item One of the four persons John, Rita, Aslam or Gurpreet will be promoted next
month. Consequently the sample space consists of four elementary outcomes
S = {John promoted, Rita promoted, Aslam promoted, Gurpreet promoted}
You are told that the chances of John’s promotion is same as that of Gurpreet,
Rita’s chances of promotion are twice as likely as Johns. Aslam’s chances are
four times that of John.
\begin{enumerate}
	\item Determine
	\begin{enumerate}
		\item P (John promoted)
		\item P (Rita promoted)
		\item P (Aslam promoted)
		\item P (Gurpreet promoted)
	\end{enumerate}
	\item If A = {John promoted or Gurpreet promoted}, find P (A).
\end{enumerate}
\solution
%\input{exemplar/11/16/3/10/main.tex}
\item A card is drawn from a deck of 52 cards. Find the probability of getting a king or a heart or a red card.\\
\solution
%\input{exemplar/11/16/3/15/main.tex}
\item The probability that a student will pass his examination is 0.73, the probability of
the student getting a compartment is 0.13, and the probability that the student will
either pass or get compartment is 0.96. State True or False.\\
\solution
%\input{exemplar/11/16/3/31/main.tex}
\item A card is selected from a pack of 52 cards\\
\begin{enumerate}[label=(\alph*)]
\item How many points are there in the sample space?
\item Calculate the probability that the cards is an ace of spades.
\item Calculate the probability that the card is (i) an ace (ii)black card.\\
\end{enumerate}
%\input{ncert/11/16/3/4_1/Prob_4.tex}
\item In a non-leap year, the probability of having 53 tuesdays or 53 wednesdays is\\
\solution
%\input{exemplar/11/16/3/18/main.tex}
\item There are 1000 sealed envelopes in a box, 10 of them contain a cash prize of
Rs 100 each, 100 of them contain a cash prize of Rs 50 each and 200 of them
contain a cash prize of Rs 10 each and rest do not contain any cash prize. If they
are well shuffled and an envelope is picked up out, what is the probability that it
contains no cash prize?\\
\solution
%\input{exemplar/10/13/3/34/main.tex}
\item 
A die is thrown and a card is selected at random from a deck of 52 playing cards. The probability of getting an even number on the die and a spade card.\\
\solution
%\input{exemplar/12/13/3/78/main.tex}
\item
If 4-digit numbers greater than 5,000 are randomly formed from the digits 0, 1, 3, 5, and 7, what is the probability of forming a number divisible by 5 when:
\begin{enumerate}
    \item The digits are repeated?
    \item The repetition of digits is not allowed?
\end{enumerate}
\solution
%\input{ncert/11/16/4/9/main.tex}
\item Consider the probability space $\brak{\Omega, \mathcal{G}, P}$ where $\Omega = [0,2]$ and $\mathcal{G} = \cbrak{\phi, \Omega, [0,1], (1,2]}$. Let $X$ and $Y$ be two functions on $\Omega$ defined as
\begin{align*}
    X(\omega) = 
    \begin{cases}
        1 & \text{if }\omega \in [0, 1]\\
        2 & \text{if }\omega \in (1, 2]
    \end{cases}
\end{align*}
and
\begin{align*}
    Y(\omega) = 
    \begin{cases}
        2 & \text{if }\omega \in [0, 1.5]\\
        3 & \text{if }\omega \in (1.5, 2].
    \end{cases}
\end{align*}
Then which one of the following statements is true?
\begin{enumerate}
    \item [(A)] $X$ is a random variable with respect to $\mathcal{G}$, but $Y$ is not a random variable with respect to $\mathcal{G}$.
    \item [(B)] $Y$ is a random variable with respect to $\mathcal{G}$, but $X$ is not a random variable with respect to $\mathcal{G}$.
    \item [(C)] Neither $X$ nor $Y$ is a random variable with respect to $\mathcal{G}$.
    \item [(D)] Both $X$ and $Y$ are random variables with respect to $\mathcal{G}$.
\end{enumerate} \hfill (GATE ST 2023)\\
\solution
%\input{gate/ST/2023/14/main.tex}
	\item  A die is loaded in such a way that each odd number is twice as likely to occur as
each even number. Find $P(G)$, where $G$ is the event that a number greater than
3 occurs on a single roll of the die.
\\
\solution
		%\input{exemplar/11/16/3/5/main.tex}
	\item All the jacks, queens and kings are removed from a deck of 52 playing cards. The remaining cards are well shuffled and then one card is drawn at random. Giving ace a value 1 similar value for other cards, find the probability that the card has a value 
		\begin{enumerate}
			\item 7
			\item greater than 7
			\item less than 7
		\end{enumerate}
		%\input{exemplar/10/13/3/30/main.tex}
  \item A Lot consists of 48 mobile phones of which 42 are good, 3 have only minor defects and 3 have major defects.Varnika will buy a phone if it is good but the trader will only buy a mobile if it has no major defects. One phone is selected at random from the lot. What is the probability that it is
\begin{enumerate}
	\item acceptable to Varnika?
            \item acceptable to the trader?
\end{enumerate}
\solution
	%\input{exemplar/10/13/3/40/main.tex}
 \item A student says that if you throw a die, it will show up 1 or not 1. Therefore, the probability of getting 1 and the probability of getting 'not 1' each is equal to $\frac{1}{2}$. Is this correct? Give reasons.\\
 \solution
        %\input{exemplar/10/13/2/9/main.tex}
   \item Four candidates A, B, C, D have ap-
plied for the assignment to coach a school cricket
team. If A is twice as likely to be selected as B, and
B and C are given about the same chance of being
selected, while C is twice as likely to be selected
as D, what are the probabilities that
\begin{enumerate}
\item C will be selected?
\item A will not be selected?
\end{enumerate}
	%\input{exemplar/11/16/3/9/main.tex}
 \item A bag contain 24 balls of which $x$ balls are red, $2x$ are white and $3x$ are blue. A ball is selected at random, What is the probability that it is
\begin{enumerate}[label=\alph*)]
\item not red ?
\item white ?
\end{enumerate}
%\input{exemplar/10/13/3/41/main.tex}
If the letters of the word ASSASSINATION are arranged at random. Find the Probability that
\begin{enumerate}[label=(\alph*)]
\item Four $S's$ come consecutively in the word
\item Two  $I's$ and two $N's$ come together
\item All $A's$ are not coming together
\item No two $A's$ are coming together
\end{enumerate}
%\input{exemplar/11/16/3/14/main.tex}
	\item One urn contains two black balls (labelled B1 and B2) and one white ball. A
	second urn contains one black ball and two white balls (labelled W1 and W2).
	Suppose the following experiment is performed. One of the two urns is chosen
	at random. Next a ball is randomly chosen from the urn. Then a second ball is
	chosen at random from the same urn without replacing the first ball.
	
	\begin{enumerate}
	\item What is the probability that two black balls are chosen?
	
	\item What is the probability that two balls of opposite colour are chosen?
	\end{enumerate}
	\solution
	%\input{exemplar/11/16/3/12/main1.tex}
\end{enumerate}

		\item A box of oranges is inspected by examining three randomly selected oranges drawn without replacement. If all the three oranges are good, the box is approved for sale, otherwise, it is rejected. Find the probability that a box containing 15 oranges out of which 12 are good and 3 are bad ones will be approved for sale.
		\label{ncert/12/13/2/3/defs.tex}
		\item Two balls are drawn at random with replacement from a box containing 10 black and 8 red balls. Find the probability that
		\label{ncert/12/13/2/12}
\begin{enumerate}
\item both balls are red.
\item first ball is black and second is red.
\item one of them is black and other is red.
\end{enumerate}

\item In a hostel, 60\% of the students read Hindi newspaper, 40\% read English newspaper and 20\% read both Hindi and English newspapers. A student is selected at random.
		\label{ncert/12/13/2/15}
\begin{enumerate}
\item Find the probability that she reads neither Hindi nor English newspapers.
\item If she reads Hindi newspaper, find the probability that she reads English newspaper.
\item If she reads English newspaper, find the probability that she reads Hindi newspaper.\\
\end{enumerate}
\item The probability of obtaining an even prime number on each die, when a pair of dice is rolled is 
\begin{enumerate}
    \item $0$ 
    
    \item $\frac{1}{3}$ 
    
    \item $\frac{1}{12}$ 
    
    \item $\frac{1}{36}$ 
\end{enumerate}
\solution
		%\begin{enumerate}[label=\thesection.\arabic*,ref=\thesection.\theenumi]
	\item One card is drawn from a well-shuffled deck of 52 cards. Find the probability of getting
\begin{enumerate}
\item A king of red colour 
\item A face card 
\item A red face card
\item The jack of hearts
\item A spade
\item The queen of diamonds

\end{enumerate}
\solution
		%\input{ncert/10/15/1/14/main.tex}
	\item Five cards—the ten, jack, queen, king and ace of diamonds, are well-shuffled with their face downwards. One card is then picked up at random.
\begin{enumerate}
\item
What is the probability that the card is the queen? 
\item
If the queen is drawn and put aside, what is the probability that the second card picked up is (a) an ace? (b) a queen?\\
\end{enumerate}
\solution
		%\input{ncert/10/15/1/15/defs.tex}
	\item A bag contains $5$ red balls and some blue balls. If the probability of drawing a blue ball is double that if a red ball, determine the number of blue balls in the bag. 
		\\
\solution
		%\input{ncert/10/15/2/3/defs.tex}
	\item A card is selected from a pack of 52 cards.
 \begin{enumerate}[label=(\alph*)] 
                 \item How many points are there in the sample space?
                 \item Calculate the probability that the card is an ace of spades.
                 \item Calculate the probability that the card is (i) an ace and (ii) black card.
 \end{enumerate}
\solution
		%\input{ncert/11/16/3/4/main.tex}
\item Four cards are drawn from a well-shuffled deck of 52 cards. What is the probability of obtaining 3 diamonds and one spade.
\\
\solution
		%\input{ncert/11/16/4/2/defs.tex}
\item In a certain lottery 10,000 tickets are sold and ten equal prizes are awarded. What is the probability of not getting a prize if you buy (a) one ticket (b) two tickets (c) 10 tickets ?	
\\
\solution
		%\input{ncert/11/16/4/4/defs.tex}
		%
\item 
Out of 100 students, two sections of 40 and 60 are formed. If you and your friend are among the 100 students, what is the probability that
\begin{enumerate}
\item you both enter the same section?
\item you both enter the different sections?
\end{enumerate}
\solution
		%\input{ncert/11/16/4/5/defs.tex}
	\item 
The number lock of a suitcase has 4 wheels each labelled with ten digits i.e. from 0 to 9.The lock opens with a sequence of four digits with no repeats.What is the probability of a person getting the right sequence to open the suitcase.
\\
\solution
		%\input{ncert/11/16/4/10/defs.tex}
		%
\item 
Two cards are drawn at random and without replacement from a pack of 52 playing cards. Find the probability that both the cards are black.
\\
\solution
		%\input{ncert/12/13/2/2/defs.tex}
		\item A box of oranges is inspected by examining three randomly selected oranges drawn without replacement. If all the three oranges are good, the box is approved for sale, otherwise, it is rejected. Find the probability that a box containing 15 oranges out of which 12 are good and 3 are bad ones will be approved for sale.
		\label{ncert/12/13/2/3/defs.tex}
		\item Two balls are drawn at random with replacement from a box containing 10 black and 8 red balls. Find the probability that
		\label{ncert/12/13/2/12}
\begin{enumerate}
\item both balls are red.
\item first ball is black and second is red.
\item one of them is black and other is red.
\end{enumerate}

\item In a hostel, 60\% of the students read Hindi newspaper, 40\% read English newspaper and 20\% read both Hindi and English newspapers. A student is selected at random.
		\label{ncert/12/13/2/15}
\begin{enumerate}
\item Find the probability that she reads neither Hindi nor English newspapers.
\item If she reads Hindi newspaper, find the probability that she reads English newspaper.
\item If she reads English newspaper, find the probability that she reads Hindi newspaper.\\
\end{enumerate}
\item The probability of obtaining an even prime number on each die, when a pair of dice is rolled is 
\begin{enumerate}
    \item $0$ 
    
    \item $\frac{1}{3}$ 
    
    \item $\frac{1}{12}$ 
    
    \item $\frac{1}{36}$ 
\end{enumerate}
\solution
		%\input{ncert/12/13/2/17/defs.tex}
	\item A bag contains 4 red and 4 black balls, another bag contains 2 red and 6 black balls. One of the two bags is selected at random and a ball is drawn from the bag which is found to be red. Find the probability that the ball is drawn from the first bag.
\\
\solution
		%\input{ncert/12/13/3/2/main.tex}
  \item
  Cards with numbers 2 to 101 are placed in a box. A card is selected at random.Find the probability that the card has
\begin{enumerate}[label=(\roman*)]
	\item an even number 
	\item a square number
\end{enumerate}
\solution
%\input{exemplar/10/13/3/32/main.tex}
\item
The king, queen and jack of clubs are removed from a deck of 52 playing cards and then well shuffled. Now one card is drawn at random from the remaining cards.  Determine the probability that the card is
\begin{enumerate}[label=(\roman*)]
\item a club
\item 10 of hearts
\end{enumerate}
\solution
%\input{exemplar/10/13/3/29/main.tex}
\item A team of medical students doing their internship have to assist during surgeries
at a city hospital. The probabilities of surgeries rated as very complex, complex,
routine, simple or very simple are respectively, 0.15, 0.20, 0.31, 0.26, .08. Find
the probabilities that a particular surgery will be rated
\begin{enumerate}
	\item complex or very complex;
	\item neither very complex nor very simple;
	\item routine or complex
	\item routine or simple
\end{enumerate}
\solution
%\input{exemplar/11/16/3/8(1)/main.tex}
\item A card is selected from a pack of 52 cards.
\begin{enumerate}[label=(\alph*)]
    \item How many points are there in the sample space?
    \item Calculate the probability that the card is an ace of spades.
    \item Calculate the probability that the card is (i) an ace and (ii) black card.
\end{enumerate}
\solution
%\input{exemplar/11/16/3/4/main2.tex}
\item The probability that a non leap year selected at random will contain 53 sundays.
\\
\solution
%\input{exemplar/10/13/1/19/main.tex}
\item One of the four persons John, Rita, Aslam or Gurpreet will be promoted next
month. Consequently the sample space consists of four elementary outcomes
S = {John promoted, Rita promoted, Aslam promoted, Gurpreet promoted}
You are told that the chances of John’s promotion is same as that of Gurpreet,
Rita’s chances of promotion are twice as likely as Johns. Aslam’s chances are
four times that of John.
\begin{enumerate}
	\item Determine
	\begin{enumerate}
		\item P (John promoted)
		\item P (Rita promoted)
		\item P (Aslam promoted)
		\item P (Gurpreet promoted)
	\end{enumerate}
	\item If A = {John promoted or Gurpreet promoted}, find P (A).
\end{enumerate}
\solution
%\input{exemplar/11/16/3/10/main.tex}
\item A card is drawn from a deck of 52 cards. Find the probability of getting a king or a heart or a red card.\\
\solution
%\input{exemplar/11/16/3/15/main.tex}
\item The probability that a student will pass his examination is 0.73, the probability of
the student getting a compartment is 0.13, and the probability that the student will
either pass or get compartment is 0.96. State True or False.\\
\solution
%\input{exemplar/11/16/3/31/main.tex}
\item A card is selected from a pack of 52 cards\\
\begin{enumerate}[label=(\alph*)]
\item How many points are there in the sample space?
\item Calculate the probability that the cards is an ace of spades.
\item Calculate the probability that the card is (i) an ace (ii)black card.\\
\end{enumerate}
%\input{ncert/11/16/3/4_1/Prob_4.tex}
\item In a non-leap year, the probability of having 53 tuesdays or 53 wednesdays is\\
\solution
%\input{exemplar/11/16/3/18/main.tex}
\item There are 1000 sealed envelopes in a box, 10 of them contain a cash prize of
Rs 100 each, 100 of them contain a cash prize of Rs 50 each and 200 of them
contain a cash prize of Rs 10 each and rest do not contain any cash prize. If they
are well shuffled and an envelope is picked up out, what is the probability that it
contains no cash prize?\\
\solution
%\input{exemplar/10/13/3/34/main.tex}
\item 
A die is thrown and a card is selected at random from a deck of 52 playing cards. The probability of getting an even number on the die and a spade card.\\
\solution
%\input{exemplar/12/13/3/78/main.tex}
\item
If 4-digit numbers greater than 5,000 are randomly formed from the digits 0, 1, 3, 5, and 7, what is the probability of forming a number divisible by 5 when:
\begin{enumerate}
    \item The digits are repeated?
    \item The repetition of digits is not allowed?
\end{enumerate}
\solution
%\input{ncert/11/16/4/9/main.tex}
\item Consider the probability space $\brak{\Omega, \mathcal{G}, P}$ where $\Omega = [0,2]$ and $\mathcal{G} = \cbrak{\phi, \Omega, [0,1], (1,2]}$. Let $X$ and $Y$ be two functions on $\Omega$ defined as
\begin{align*}
    X(\omega) = 
    \begin{cases}
        1 & \text{if }\omega \in [0, 1]\\
        2 & \text{if }\omega \in (1, 2]
    \end{cases}
\end{align*}
and
\begin{align*}
    Y(\omega) = 
    \begin{cases}
        2 & \text{if }\omega \in [0, 1.5]\\
        3 & \text{if }\omega \in (1.5, 2].
    \end{cases}
\end{align*}
Then which one of the following statements is true?
\begin{enumerate}
    \item [(A)] $X$ is a random variable with respect to $\mathcal{G}$, but $Y$ is not a random variable with respect to $\mathcal{G}$.
    \item [(B)] $Y$ is a random variable with respect to $\mathcal{G}$, but $X$ is not a random variable with respect to $\mathcal{G}$.
    \item [(C)] Neither $X$ nor $Y$ is a random variable with respect to $\mathcal{G}$.
    \item [(D)] Both $X$ and $Y$ are random variables with respect to $\mathcal{G}$.
\end{enumerate} \hfill (GATE ST 2023)\\
\solution
%\input{gate/ST/2023/14/main.tex}
	\item  A die is loaded in such a way that each odd number is twice as likely to occur as
each even number. Find $P(G)$, where $G$ is the event that a number greater than
3 occurs on a single roll of the die.
\\
\solution
		%\input{exemplar/11/16/3/5/main.tex}
	\item All the jacks, queens and kings are removed from a deck of 52 playing cards. The remaining cards are well shuffled and then one card is drawn at random. Giving ace a value 1 similar value for other cards, find the probability that the card has a value 
		\begin{enumerate}
			\item 7
			\item greater than 7
			\item less than 7
		\end{enumerate}
		%\input{exemplar/10/13/3/30/main.tex}
  \item A Lot consists of 48 mobile phones of which 42 are good, 3 have only minor defects and 3 have major defects.Varnika will buy a phone if it is good but the trader will only buy a mobile if it has no major defects. One phone is selected at random from the lot. What is the probability that it is
\begin{enumerate}
	\item acceptable to Varnika?
            \item acceptable to the trader?
\end{enumerate}
\solution
	%\input{exemplar/10/13/3/40/main.tex}
 \item A student says that if you throw a die, it will show up 1 or not 1. Therefore, the probability of getting 1 and the probability of getting 'not 1' each is equal to $\frac{1}{2}$. Is this correct? Give reasons.\\
 \solution
        %\input{exemplar/10/13/2/9/main.tex}
   \item Four candidates A, B, C, D have ap-
plied for the assignment to coach a school cricket
team. If A is twice as likely to be selected as B, and
B and C are given about the same chance of being
selected, while C is twice as likely to be selected
as D, what are the probabilities that
\begin{enumerate}
\item C will be selected?
\item A will not be selected?
\end{enumerate}
	%\input{exemplar/11/16/3/9/main.tex}
 \item A bag contain 24 balls of which $x$ balls are red, $2x$ are white and $3x$ are blue. A ball is selected at random, What is the probability that it is
\begin{enumerate}[label=\alph*)]
\item not red ?
\item white ?
\end{enumerate}
%\input{exemplar/10/13/3/41/main.tex}
If the letters of the word ASSASSINATION are arranged at random. Find the Probability that
\begin{enumerate}[label=(\alph*)]
\item Four $S's$ come consecutively in the word
\item Two  $I's$ and two $N's$ come together
\item All $A's$ are not coming together
\item No two $A's$ are coming together
\end{enumerate}
%\input{exemplar/11/16/3/14/main.tex}
	\item One urn contains two black balls (labelled B1 and B2) and one white ball. A
	second urn contains one black ball and two white balls (labelled W1 and W2).
	Suppose the following experiment is performed. One of the two urns is chosen
	at random. Next a ball is randomly chosen from the urn. Then a second ball is
	chosen at random from the same urn without replacing the first ball.
	
	\begin{enumerate}
	\item What is the probability that two black balls are chosen?
	
	\item What is the probability that two balls of opposite colour are chosen?
	\end{enumerate}
	\solution
	%\input{exemplar/11/16/3/12/main1.tex}
\end{enumerate}

	\item A bag contains 4 red and 4 black balls, another bag contains 2 red and 6 black balls. One of the two bags is selected at random and a ball is drawn from the bag which is found to be red. Find the probability that the ball is drawn from the first bag.
\\
\solution
		%\begin{table}[H]
	\centering
\begin{tabular}{|c|c|c|}
\hline
Random variable &Value &Definition\\ \hline
\multirow{3}{*}{X} &0 &Slips of Rs 1\\
&1 &Slips of Rs 5\\
&2 &Slips of Rs 13\\ \hline
\multirow{2}{*}{Y} &0 &Box A\\
&1 &Box B\\\hline
\end{tabular}
\caption{}
\label{tab:Distribution}
\end{table}
See \tabref{tab:Distribution}.
\begin{align}
p_{Y}\brak{k}= \begin{cases} 
      \frac{1}{3} & {k=0} \\
      \frac{2}{3 }& {k=1} 
   \end{cases}
   \\
p_{Y|X}\brak{0|0} = \frac{19}{25}\, 
p_{Y|X}\brak{0|1} = \frac{6}{25}\,
p_{Y|X}\brak{1|0} = \frac{45}{50}\,
p_{Y|X}\brak{1|2} = \frac{5}{50}
\end{align}
The desired probability is the probability that a slip drawn at random is marked other than Rs 1,
\begin{align}
&=1-p_X\brak{0}\\
&= p_X(1) + p_X(2)
\end{align}
Using Bayes theorem,
\begin{align}
&= p_Y\brak{0} \times \pr{Y=0 | X=1} + p_Y\brak{1} \times \pr{Y=1|X=2}\\
&=\frac{1}{3} \times \frac{6}{25} + \frac{2}{3} \times \frac{5}{50}\\
&=\frac{11}{75}
\end{align}

\newpage

%\tableofcontents

\bigskip

\renewcommand{\thefigure}{\theenumi}
\renewcommand{\thetable}{\theenumi}
%\renewcommand{\theequation}{\theenumi}

%\begin{abstract}
%%\boldmath
%In this letter, an algorithm for evaluating the exact analytical bit error rate  (BER)  for the piecewise linear (PL) combiner for  multiple relays is presented. Previous results were available only for upto three relays. The algorithm is unique in the sense that  the actual mathematical expressions, that are prohibitively large, need not be explicitly obtained. The diversity gain due to multiple relays is shown through plots of the analytical BER, well supported by simulations. 
%
%\end{abstract}
% IEEEtran.cls defaults to using nonbold math in the Abstract.
% This preserves the distinction between vectors and scalars. However,
% if the journal you are submitting to favors bold math in the abstract,
% then you can use LaTeX's standard command \boldmath at the very start
% of the abstract to achieve this. Many IEEE journals frown on math
% in the abstract anyway.

% Note that keywords are not normally used for peerreview papers.
%\begin{IEEEkeywords}
%Cooperative diversity, decode and forward, piecewise linear
%\end{IEEEkeywords}



% For peer review papers, you can put extra information on the cover
% page as needed:
% \ifCLASSOPTIONpeerreview
% \begin{center} \bfseries EDICS Category: 3-BBND \end{center}
% \fi
%
% For peerreview papers, this IEEEtran command inserts a page break and
% creates the second title. It will be ignored for other modes.
%\IEEEpeerreviewmaketitle




  \item
  Cards with numbers 2 to 101 are placed in a box. A card is selected at random.Find the probability that the card has
\begin{enumerate}[label=(\roman*)]
	\item an even number 
	\item a square number
\end{enumerate}
\solution
%\begin{table}[H]
	\centering
\begin{tabular}{|c|c|c|}
\hline
Random variable &Value &Definition\\ \hline
\multirow{3}{*}{X} &0 &Slips of Rs 1\\
&1 &Slips of Rs 5\\
&2 &Slips of Rs 13\\ \hline
\multirow{2}{*}{Y} &0 &Box A\\
&1 &Box B\\\hline
\end{tabular}
\caption{}
\label{tab:Distribution}
\end{table}
See \tabref{tab:Distribution}.
\begin{align}
p_{Y}\brak{k}= \begin{cases} 
      \frac{1}{3} & {k=0} \\
      \frac{2}{3 }& {k=1} 
   \end{cases}
   \\
p_{Y|X}\brak{0|0} = \frac{19}{25}\, 
p_{Y|X}\brak{0|1} = \frac{6}{25}\,
p_{Y|X}\brak{1|0} = \frac{45}{50}\,
p_{Y|X}\brak{1|2} = \frac{5}{50}
\end{align}
The desired probability is the probability that a slip drawn at random is marked other than Rs 1,
\begin{align}
&=1-p_X\brak{0}\\
&= p_X(1) + p_X(2)
\end{align}
Using Bayes theorem,
\begin{align}
&= p_Y\brak{0} \times \pr{Y=0 | X=1} + p_Y\brak{1} \times \pr{Y=1|X=2}\\
&=\frac{1}{3} \times \frac{6}{25} + \frac{2}{3} \times \frac{5}{50}\\
&=\frac{11}{75}
\end{align}

\newpage

%\tableofcontents

\bigskip

\renewcommand{\thefigure}{\theenumi}
\renewcommand{\thetable}{\theenumi}
%\renewcommand{\theequation}{\theenumi}

%\begin{abstract}
%%\boldmath
%In this letter, an algorithm for evaluating the exact analytical bit error rate  (BER)  for the piecewise linear (PL) combiner for  multiple relays is presented. Previous results were available only for upto three relays. The algorithm is unique in the sense that  the actual mathematical expressions, that are prohibitively large, need not be explicitly obtained. The diversity gain due to multiple relays is shown through plots of the analytical BER, well supported by simulations. 
%
%\end{abstract}
% IEEEtran.cls defaults to using nonbold math in the Abstract.
% This preserves the distinction between vectors and scalars. However,
% if the journal you are submitting to favors bold math in the abstract,
% then you can use LaTeX's standard command \boldmath at the very start
% of the abstract to achieve this. Many IEEE journals frown on math
% in the abstract anyway.

% Note that keywords are not normally used for peerreview papers.
%\begin{IEEEkeywords}
%Cooperative diversity, decode and forward, piecewise linear
%\end{IEEEkeywords}



% For peer review papers, you can put extra information on the cover
% page as needed:
% \ifCLASSOPTIONpeerreview
% \begin{center} \bfseries EDICS Category: 3-BBND \end{center}
% \fi
%
% For peerreview papers, this IEEEtran command inserts a page break and
% creates the second title. It will be ignored for other modes.
%\IEEEpeerreviewmaketitle




\item
The king, queen and jack of clubs are removed from a deck of 52 playing cards and then well shuffled. Now one card is drawn at random from the remaining cards.  Determine the probability that the card is
\begin{enumerate}[label=(\roman*)]
\item a club
\item 10 of hearts
\end{enumerate}
\solution
%\begin{table}[H]
	\centering
\begin{tabular}{|c|c|c|}
\hline
Random variable &Value &Definition\\ \hline
\multirow{3}{*}{X} &0 &Slips of Rs 1\\
&1 &Slips of Rs 5\\
&2 &Slips of Rs 13\\ \hline
\multirow{2}{*}{Y} &0 &Box A\\
&1 &Box B\\\hline
\end{tabular}
\caption{}
\label{tab:Distribution}
\end{table}
See \tabref{tab:Distribution}.
\begin{align}
p_{Y}\brak{k}= \begin{cases} 
      \frac{1}{3} & {k=0} \\
      \frac{2}{3 }& {k=1} 
   \end{cases}
   \\
p_{Y|X}\brak{0|0} = \frac{19}{25}\, 
p_{Y|X}\brak{0|1} = \frac{6}{25}\,
p_{Y|X}\brak{1|0} = \frac{45}{50}\,
p_{Y|X}\brak{1|2} = \frac{5}{50}
\end{align}
The desired probability is the probability that a slip drawn at random is marked other than Rs 1,
\begin{align}
&=1-p_X\brak{0}\\
&= p_X(1) + p_X(2)
\end{align}
Using Bayes theorem,
\begin{align}
&= p_Y\brak{0} \times \pr{Y=0 | X=1} + p_Y\brak{1} \times \pr{Y=1|X=2}\\
&=\frac{1}{3} \times \frac{6}{25} + \frac{2}{3} \times \frac{5}{50}\\
&=\frac{11}{75}
\end{align}

\newpage

%\tableofcontents

\bigskip

\renewcommand{\thefigure}{\theenumi}
\renewcommand{\thetable}{\theenumi}
%\renewcommand{\theequation}{\theenumi}

%\begin{abstract}
%%\boldmath
%In this letter, an algorithm for evaluating the exact analytical bit error rate  (BER)  for the piecewise linear (PL) combiner for  multiple relays is presented. Previous results were available only for upto three relays. The algorithm is unique in the sense that  the actual mathematical expressions, that are prohibitively large, need not be explicitly obtained. The diversity gain due to multiple relays is shown through plots of the analytical BER, well supported by simulations. 
%
%\end{abstract}
% IEEEtran.cls defaults to using nonbold math in the Abstract.
% This preserves the distinction between vectors and scalars. However,
% if the journal you are submitting to favors bold math in the abstract,
% then you can use LaTeX's standard command \boldmath at the very start
% of the abstract to achieve this. Many IEEE journals frown on math
% in the abstract anyway.

% Note that keywords are not normally used for peerreview papers.
%\begin{IEEEkeywords}
%Cooperative diversity, decode and forward, piecewise linear
%\end{IEEEkeywords}



% For peer review papers, you can put extra information on the cover
% page as needed:
% \ifCLASSOPTIONpeerreview
% \begin{center} \bfseries EDICS Category: 3-BBND \end{center}
% \fi
%
% For peerreview papers, this IEEEtran command inserts a page break and
% creates the second title. It will be ignored for other modes.
%\IEEEpeerreviewmaketitle




\item A team of medical students doing their internship have to assist during surgeries
at a city hospital. The probabilities of surgeries rated as very complex, complex,
routine, simple or very simple are respectively, 0.15, 0.20, 0.31, 0.26, .08. Find
the probabilities that a particular surgery will be rated
\begin{enumerate}
	\item complex or very complex;
	\item neither very complex nor very simple;
	\item routine or complex
	\item routine or simple
\end{enumerate}
\solution
%\begin{table}[H]
	\centering
\begin{tabular}{|c|c|c|}
\hline
Random variable &Value &Definition\\ \hline
\multirow{3}{*}{X} &0 &Slips of Rs 1\\
&1 &Slips of Rs 5\\
&2 &Slips of Rs 13\\ \hline
\multirow{2}{*}{Y} &0 &Box A\\
&1 &Box B\\\hline
\end{tabular}
\caption{}
\label{tab:Distribution}
\end{table}
See \tabref{tab:Distribution}.
\begin{align}
p_{Y}\brak{k}= \begin{cases} 
      \frac{1}{3} & {k=0} \\
      \frac{2}{3 }& {k=1} 
   \end{cases}
   \\
p_{Y|X}\brak{0|0} = \frac{19}{25}\, 
p_{Y|X}\brak{0|1} = \frac{6}{25}\,
p_{Y|X}\brak{1|0} = \frac{45}{50}\,
p_{Y|X}\brak{1|2} = \frac{5}{50}
\end{align}
The desired probability is the probability that a slip drawn at random is marked other than Rs 1,
\begin{align}
&=1-p_X\brak{0}\\
&= p_X(1) + p_X(2)
\end{align}
Using Bayes theorem,
\begin{align}
&= p_Y\brak{0} \times \pr{Y=0 | X=1} + p_Y\brak{1} \times \pr{Y=1|X=2}\\
&=\frac{1}{3} \times \frac{6}{25} + \frac{2}{3} \times \frac{5}{50}\\
&=\frac{11}{75}
\end{align}

\newpage

%\tableofcontents

\bigskip

\renewcommand{\thefigure}{\theenumi}
\renewcommand{\thetable}{\theenumi}
%\renewcommand{\theequation}{\theenumi}

%\begin{abstract}
%%\boldmath
%In this letter, an algorithm for evaluating the exact analytical bit error rate  (BER)  for the piecewise linear (PL) combiner for  multiple relays is presented. Previous results were available only for upto three relays. The algorithm is unique in the sense that  the actual mathematical expressions, that are prohibitively large, need not be explicitly obtained. The diversity gain due to multiple relays is shown through plots of the analytical BER, well supported by simulations. 
%
%\end{abstract}
% IEEEtran.cls defaults to using nonbold math in the Abstract.
% This preserves the distinction between vectors and scalars. However,
% if the journal you are submitting to favors bold math in the abstract,
% then you can use LaTeX's standard command \boldmath at the very start
% of the abstract to achieve this. Many IEEE journals frown on math
% in the abstract anyway.

% Note that keywords are not normally used for peerreview papers.
%\begin{IEEEkeywords}
%Cooperative diversity, decode and forward, piecewise linear
%\end{IEEEkeywords}



% For peer review papers, you can put extra information on the cover
% page as needed:
% \ifCLASSOPTIONpeerreview
% \begin{center} \bfseries EDICS Category: 3-BBND \end{center}
% \fi
%
% For peerreview papers, this IEEEtran command inserts a page break and
% creates the second title. It will be ignored for other modes.
%\IEEEpeerreviewmaketitle




\item A card is selected from a pack of 52 cards.
\begin{enumerate}[label=(\alph*)]
    \item How many points are there in the sample space?
    \item Calculate the probability that the card is an ace of spades.
    \item Calculate the probability that the card is (i) an ace and (ii) black card.
\end{enumerate}
\solution
%Let $X$ be an bernoulli rv defined as in \tabref{tab:exemplar/11/16/3/26}.  Then, 
\begin{equation}
    p =
        \frac{4}{11} 
\end{equation}
\begin{table}[H]
	\centering
	\input{exemplar/11/16/3/26/tables/Table2.tex}
	\caption{}
        \label{tab:exemplar/11/16/3/26}
\end{table}

\item The probability that a non leap year selected at random will contain 53 sundays.
\\
\solution
%\begin{table}[H]
	\centering
\begin{tabular}{|c|c|c|}
\hline
Random variable &Value &Definition\\ \hline
\multirow{3}{*}{X} &0 &Slips of Rs 1\\
&1 &Slips of Rs 5\\
&2 &Slips of Rs 13\\ \hline
\multirow{2}{*}{Y} &0 &Box A\\
&1 &Box B\\\hline
\end{tabular}
\caption{}
\label{tab:Distribution}
\end{table}
See \tabref{tab:Distribution}.
\begin{align}
p_{Y}\brak{k}= \begin{cases} 
      \frac{1}{3} & {k=0} \\
      \frac{2}{3 }& {k=1} 
   \end{cases}
   \\
p_{Y|X}\brak{0|0} = \frac{19}{25}\, 
p_{Y|X}\brak{0|1} = \frac{6}{25}\,
p_{Y|X}\brak{1|0} = \frac{45}{50}\,
p_{Y|X}\brak{1|2} = \frac{5}{50}
\end{align}
The desired probability is the probability that a slip drawn at random is marked other than Rs 1,
\begin{align}
&=1-p_X\brak{0}\\
&= p_X(1) + p_X(2)
\end{align}
Using Bayes theorem,
\begin{align}
&= p_Y\brak{0} \times \pr{Y=0 | X=1} + p_Y\brak{1} \times \pr{Y=1|X=2}\\
&=\frac{1}{3} \times \frac{6}{25} + \frac{2}{3} \times \frac{5}{50}\\
&=\frac{11}{75}
\end{align}

\newpage

%\tableofcontents

\bigskip

\renewcommand{\thefigure}{\theenumi}
\renewcommand{\thetable}{\theenumi}
%\renewcommand{\theequation}{\theenumi}

%\begin{abstract}
%%\boldmath
%In this letter, an algorithm for evaluating the exact analytical bit error rate  (BER)  for the piecewise linear (PL) combiner for  multiple relays is presented. Previous results were available only for upto three relays. The algorithm is unique in the sense that  the actual mathematical expressions, that are prohibitively large, need not be explicitly obtained. The diversity gain due to multiple relays is shown through plots of the analytical BER, well supported by simulations. 
%
%\end{abstract}
% IEEEtran.cls defaults to using nonbold math in the Abstract.
% This preserves the distinction between vectors and scalars. However,
% if the journal you are submitting to favors bold math in the abstract,
% then you can use LaTeX's standard command \boldmath at the very start
% of the abstract to achieve this. Many IEEE journals frown on math
% in the abstract anyway.

% Note that keywords are not normally used for peerreview papers.
%\begin{IEEEkeywords}
%Cooperative diversity, decode and forward, piecewise linear
%\end{IEEEkeywords}



% For peer review papers, you can put extra information on the cover
% page as needed:
% \ifCLASSOPTIONpeerreview
% \begin{center} \bfseries EDICS Category: 3-BBND \end{center}
% \fi
%
% For peerreview papers, this IEEEtran command inserts a page break and
% creates the second title. It will be ignored for other modes.
%\IEEEpeerreviewmaketitle




\item One of the four persons John, Rita, Aslam or Gurpreet will be promoted next
month. Consequently the sample space consists of four elementary outcomes
S = {John promoted, Rita promoted, Aslam promoted, Gurpreet promoted}
You are told that the chances of John’s promotion is same as that of Gurpreet,
Rita’s chances of promotion are twice as likely as Johns. Aslam’s chances are
four times that of John.
\begin{enumerate}
	\item Determine
	\begin{enumerate}
		\item P (John promoted)
		\item P (Rita promoted)
		\item P (Aslam promoted)
		\item P (Gurpreet promoted)
	\end{enumerate}
	\item If A = {John promoted or Gurpreet promoted}, find P (A).
\end{enumerate}
\solution
%\begin{table}[H]
	\centering
\begin{tabular}{|c|c|c|}
\hline
Random variable &Value &Definition\\ \hline
\multirow{3}{*}{X} &0 &Slips of Rs 1\\
&1 &Slips of Rs 5\\
&2 &Slips of Rs 13\\ \hline
\multirow{2}{*}{Y} &0 &Box A\\
&1 &Box B\\\hline
\end{tabular}
\caption{}
\label{tab:Distribution}
\end{table}
See \tabref{tab:Distribution}.
\begin{align}
p_{Y}\brak{k}= \begin{cases} 
      \frac{1}{3} & {k=0} \\
      \frac{2}{3 }& {k=1} 
   \end{cases}
   \\
p_{Y|X}\brak{0|0} = \frac{19}{25}\, 
p_{Y|X}\brak{0|1} = \frac{6}{25}\,
p_{Y|X}\brak{1|0} = \frac{45}{50}\,
p_{Y|X}\brak{1|2} = \frac{5}{50}
\end{align}
The desired probability is the probability that a slip drawn at random is marked other than Rs 1,
\begin{align}
&=1-p_X\brak{0}\\
&= p_X(1) + p_X(2)
\end{align}
Using Bayes theorem,
\begin{align}
&= p_Y\brak{0} \times \pr{Y=0 | X=1} + p_Y\brak{1} \times \pr{Y=1|X=2}\\
&=\frac{1}{3} \times \frac{6}{25} + \frac{2}{3} \times \frac{5}{50}\\
&=\frac{11}{75}
\end{align}

\newpage

%\tableofcontents

\bigskip

\renewcommand{\thefigure}{\theenumi}
\renewcommand{\thetable}{\theenumi}
%\renewcommand{\theequation}{\theenumi}

%\begin{abstract}
%%\boldmath
%In this letter, an algorithm for evaluating the exact analytical bit error rate  (BER)  for the piecewise linear (PL) combiner for  multiple relays is presented. Previous results were available only for upto three relays. The algorithm is unique in the sense that  the actual mathematical expressions, that are prohibitively large, need not be explicitly obtained. The diversity gain due to multiple relays is shown through plots of the analytical BER, well supported by simulations. 
%
%\end{abstract}
% IEEEtran.cls defaults to using nonbold math in the Abstract.
% This preserves the distinction between vectors and scalars. However,
% if the journal you are submitting to favors bold math in the abstract,
% then you can use LaTeX's standard command \boldmath at the very start
% of the abstract to achieve this. Many IEEE journals frown on math
% in the abstract anyway.

% Note that keywords are not normally used for peerreview papers.
%\begin{IEEEkeywords}
%Cooperative diversity, decode and forward, piecewise linear
%\end{IEEEkeywords}



% For peer review papers, you can put extra information on the cover
% page as needed:
% \ifCLASSOPTIONpeerreview
% \begin{center} \bfseries EDICS Category: 3-BBND \end{center}
% \fi
%
% For peerreview papers, this IEEEtran command inserts a page break and
% creates the second title. It will be ignored for other modes.
%\IEEEpeerreviewmaketitle




\item A card is drawn from a deck of 52 cards. Find the probability of getting a king or a heart or a red card.\\
\solution
%\begin{table}[H]
	\centering
\begin{tabular}{|c|c|c|}
\hline
Random variable &Value &Definition\\ \hline
\multirow{3}{*}{X} &0 &Slips of Rs 1\\
&1 &Slips of Rs 5\\
&2 &Slips of Rs 13\\ \hline
\multirow{2}{*}{Y} &0 &Box A\\
&1 &Box B\\\hline
\end{tabular}
\caption{}
\label{tab:Distribution}
\end{table}
See \tabref{tab:Distribution}.
\begin{align}
p_{Y}\brak{k}= \begin{cases} 
      \frac{1}{3} & {k=0} \\
      \frac{2}{3 }& {k=1} 
   \end{cases}
   \\
p_{Y|X}\brak{0|0} = \frac{19}{25}\, 
p_{Y|X}\brak{0|1} = \frac{6}{25}\,
p_{Y|X}\brak{1|0} = \frac{45}{50}\,
p_{Y|X}\brak{1|2} = \frac{5}{50}
\end{align}
The desired probability is the probability that a slip drawn at random is marked other than Rs 1,
\begin{align}
&=1-p_X\brak{0}\\
&= p_X(1) + p_X(2)
\end{align}
Using Bayes theorem,
\begin{align}
&= p_Y\brak{0} \times \pr{Y=0 | X=1} + p_Y\brak{1} \times \pr{Y=1|X=2}\\
&=\frac{1}{3} \times \frac{6}{25} + \frac{2}{3} \times \frac{5}{50}\\
&=\frac{11}{75}
\end{align}

\newpage

%\tableofcontents

\bigskip

\renewcommand{\thefigure}{\theenumi}
\renewcommand{\thetable}{\theenumi}
%\renewcommand{\theequation}{\theenumi}

%\begin{abstract}
%%\boldmath
%In this letter, an algorithm for evaluating the exact analytical bit error rate  (BER)  for the piecewise linear (PL) combiner for  multiple relays is presented. Previous results were available only for upto three relays. The algorithm is unique in the sense that  the actual mathematical expressions, that are prohibitively large, need not be explicitly obtained. The diversity gain due to multiple relays is shown through plots of the analytical BER, well supported by simulations. 
%
%\end{abstract}
% IEEEtran.cls defaults to using nonbold math in the Abstract.
% This preserves the distinction between vectors and scalars. However,
% if the journal you are submitting to favors bold math in the abstract,
% then you can use LaTeX's standard command \boldmath at the very start
% of the abstract to achieve this. Many IEEE journals frown on math
% in the abstract anyway.

% Note that keywords are not normally used for peerreview papers.
%\begin{IEEEkeywords}
%Cooperative diversity, decode and forward, piecewise linear
%\end{IEEEkeywords}



% For peer review papers, you can put extra information on the cover
% page as needed:
% \ifCLASSOPTIONpeerreview
% \begin{center} \bfseries EDICS Category: 3-BBND \end{center}
% \fi
%
% For peerreview papers, this IEEEtran command inserts a page break and
% creates the second title. It will be ignored for other modes.
%\IEEEpeerreviewmaketitle




\item The probability that a student will pass his examination is 0.73, the probability of
the student getting a compartment is 0.13, and the probability that the student will
either pass or get compartment is 0.96. State True or False.\\
\solution
%\begin{table}[H]
	\centering
\begin{tabular}{|c|c|c|}
\hline
Random variable &Value &Definition\\ \hline
\multirow{3}{*}{X} &0 &Slips of Rs 1\\
&1 &Slips of Rs 5\\
&2 &Slips of Rs 13\\ \hline
\multirow{2}{*}{Y} &0 &Box A\\
&1 &Box B\\\hline
\end{tabular}
\caption{}
\label{tab:Distribution}
\end{table}
See \tabref{tab:Distribution}.
\begin{align}
p_{Y}\brak{k}= \begin{cases} 
      \frac{1}{3} & {k=0} \\
      \frac{2}{3 }& {k=1} 
   \end{cases}
   \\
p_{Y|X}\brak{0|0} = \frac{19}{25}\, 
p_{Y|X}\brak{0|1} = \frac{6}{25}\,
p_{Y|X}\brak{1|0} = \frac{45}{50}\,
p_{Y|X}\brak{1|2} = \frac{5}{50}
\end{align}
The desired probability is the probability that a slip drawn at random is marked other than Rs 1,
\begin{align}
&=1-p_X\brak{0}\\
&= p_X(1) + p_X(2)
\end{align}
Using Bayes theorem,
\begin{align}
&= p_Y\brak{0} \times \pr{Y=0 | X=1} + p_Y\brak{1} \times \pr{Y=1|X=2}\\
&=\frac{1}{3} \times \frac{6}{25} + \frac{2}{3} \times \frac{5}{50}\\
&=\frac{11}{75}
\end{align}

\newpage

%\tableofcontents

\bigskip

\renewcommand{\thefigure}{\theenumi}
\renewcommand{\thetable}{\theenumi}
%\renewcommand{\theequation}{\theenumi}

%\begin{abstract}
%%\boldmath
%In this letter, an algorithm for evaluating the exact analytical bit error rate  (BER)  for the piecewise linear (PL) combiner for  multiple relays is presented. Previous results were available only for upto three relays. The algorithm is unique in the sense that  the actual mathematical expressions, that are prohibitively large, need not be explicitly obtained. The diversity gain due to multiple relays is shown through plots of the analytical BER, well supported by simulations. 
%
%\end{abstract}
% IEEEtran.cls defaults to using nonbold math in the Abstract.
% This preserves the distinction between vectors and scalars. However,
% if the journal you are submitting to favors bold math in the abstract,
% then you can use LaTeX's standard command \boldmath at the very start
% of the abstract to achieve this. Many IEEE journals frown on math
% in the abstract anyway.

% Note that keywords are not normally used for peerreview papers.
%\begin{IEEEkeywords}
%Cooperative diversity, decode and forward, piecewise linear
%\end{IEEEkeywords}



% For peer review papers, you can put extra information on the cover
% page as needed:
% \ifCLASSOPTIONpeerreview
% \begin{center} \bfseries EDICS Category: 3-BBND \end{center}
% \fi
%
% For peerreview papers, this IEEEtran command inserts a page break and
% creates the second title. It will be ignored for other modes.
%\IEEEpeerreviewmaketitle




\item A card is selected from a pack of 52 cards\\
\begin{enumerate}[label=(\alph*)]
\item How many points are there in the sample space?
\item Calculate the probability that the cards is an ace of spades.
\item Calculate the probability that the card is (i) an ace (ii)black card.\\
\end{enumerate}
%\input{ncert/11/16/3/4_1/Prob_4.tex}
\item In a non-leap year, the probability of having 53 tuesdays or 53 wednesdays is\\
\solution
%A non-leap year has a total of 365 days, and a week has 7 days.\\
So it can be expressed as 
\begin{align}
365\text{days} &=52\times 7+1 \text{day}
\end{align}
$\implies$ 52 tuesdays or wednesdays\\
Random variable X denotes the days of a week
\begin{align}
p_X\brak{k}&=\frac{1}{7}; \quad \brak{1<k<7}
\end{align}
So the probability of extra day being tuesday or wednesday is
\begin{align}
p_X\brak{3}+p_X\brak{4}&=\frac{1}{7}+\frac{1}{7}=\frac{2}{7}
\end{align}



\item There are 1000 sealed envelopes in a box, 10 of them contain a cash prize of
Rs 100 each, 100 of them contain a cash prize of Rs 50 each and 200 of them
contain a cash prize of Rs 10 each and rest do not contain any cash prize. If they
are well shuffled and an envelope is picked up out, what is the probability that it
contains no cash prize?\\
\solution
%\begin{table}[H]
	\centering
\begin{tabular}{|c|c|c|}
\hline
Random variable &Value &Definition\\ \hline
\multirow{3}{*}{X} &0 &Slips of Rs 1\\
&1 &Slips of Rs 5\\
&2 &Slips of Rs 13\\ \hline
\multirow{2}{*}{Y} &0 &Box A\\
&1 &Box B\\\hline
\end{tabular}
\caption{}
\label{tab:Distribution}
\end{table}
See \tabref{tab:Distribution}.
\begin{align}
p_{Y}\brak{k}= \begin{cases} 
      \frac{1}{3} & {k=0} \\
      \frac{2}{3 }& {k=1} 
   \end{cases}
   \\
p_{Y|X}\brak{0|0} = \frac{19}{25}\, 
p_{Y|X}\brak{0|1} = \frac{6}{25}\,
p_{Y|X}\brak{1|0} = \frac{45}{50}\,
p_{Y|X}\brak{1|2} = \frac{5}{50}
\end{align}
The desired probability is the probability that a slip drawn at random is marked other than Rs 1,
\begin{align}
&=1-p_X\brak{0}\\
&= p_X(1) + p_X(2)
\end{align}
Using Bayes theorem,
\begin{align}
&= p_Y\brak{0} \times \pr{Y=0 | X=1} + p_Y\brak{1} \times \pr{Y=1|X=2}\\
&=\frac{1}{3} \times \frac{6}{25} + \frac{2}{3} \times \frac{5}{50}\\
&=\frac{11}{75}
\end{align}

\newpage

%\tableofcontents

\bigskip

\renewcommand{\thefigure}{\theenumi}
\renewcommand{\thetable}{\theenumi}
%\renewcommand{\theequation}{\theenumi}

%\begin{abstract}
%%\boldmath
%In this letter, an algorithm for evaluating the exact analytical bit error rate  (BER)  for the piecewise linear (PL) combiner for  multiple relays is presented. Previous results were available only for upto three relays. The algorithm is unique in the sense that  the actual mathematical expressions, that are prohibitively large, need not be explicitly obtained. The diversity gain due to multiple relays is shown through plots of the analytical BER, well supported by simulations. 
%
%\end{abstract}
% IEEEtran.cls defaults to using nonbold math in the Abstract.
% This preserves the distinction between vectors and scalars. However,
% if the journal you are submitting to favors bold math in the abstract,
% then you can use LaTeX's standard command \boldmath at the very start
% of the abstract to achieve this. Many IEEE journals frown on math
% in the abstract anyway.

% Note that keywords are not normally used for peerreview papers.
%\begin{IEEEkeywords}
%Cooperative diversity, decode and forward, piecewise linear
%\end{IEEEkeywords}



% For peer review papers, you can put extra information on the cover
% page as needed:
% \ifCLASSOPTIONpeerreview
% \begin{center} \bfseries EDICS Category: 3-BBND \end{center}
% \fi
%
% For peerreview papers, this IEEEtran command inserts a page break and
% creates the second title. It will be ignored for other modes.
%\IEEEpeerreviewmaketitle




\item 
A die is thrown and a card is selected at random from a deck of 52 playing cards. The probability of getting an even number on the die and a spade card.\\
\solution
%\begin{table}[H]
	\centering
\begin{tabular}{|c|c|c|}
\hline
Random variable &Value &Definition\\ \hline
\multirow{3}{*}{X} &0 &Slips of Rs 1\\
&1 &Slips of Rs 5\\
&2 &Slips of Rs 13\\ \hline
\multirow{2}{*}{Y} &0 &Box A\\
&1 &Box B\\\hline
\end{tabular}
\caption{}
\label{tab:Distribution}
\end{table}
See \tabref{tab:Distribution}.
\begin{align}
p_{Y}\brak{k}= \begin{cases} 
      \frac{1}{3} & {k=0} \\
      \frac{2}{3 }& {k=1} 
   \end{cases}
   \\
p_{Y|X}\brak{0|0} = \frac{19}{25}\, 
p_{Y|X}\brak{0|1} = \frac{6}{25}\,
p_{Y|X}\brak{1|0} = \frac{45}{50}\,
p_{Y|X}\brak{1|2} = \frac{5}{50}
\end{align}
The desired probability is the probability that a slip drawn at random is marked other than Rs 1,
\begin{align}
&=1-p_X\brak{0}\\
&= p_X(1) + p_X(2)
\end{align}
Using Bayes theorem,
\begin{align}
&= p_Y\brak{0} \times \pr{Y=0 | X=1} + p_Y\brak{1} \times \pr{Y=1|X=2}\\
&=\frac{1}{3} \times \frac{6}{25} + \frac{2}{3} \times \frac{5}{50}\\
&=\frac{11}{75}
\end{align}

\newpage

%\tableofcontents

\bigskip

\renewcommand{\thefigure}{\theenumi}
\renewcommand{\thetable}{\theenumi}
%\renewcommand{\theequation}{\theenumi}

%\begin{abstract}
%%\boldmath
%In this letter, an algorithm for evaluating the exact analytical bit error rate  (BER)  for the piecewise linear (PL) combiner for  multiple relays is presented. Previous results were available only for upto three relays. The algorithm is unique in the sense that  the actual mathematical expressions, that are prohibitively large, need not be explicitly obtained. The diversity gain due to multiple relays is shown through plots of the analytical BER, well supported by simulations. 
%
%\end{abstract}
% IEEEtran.cls defaults to using nonbold math in the Abstract.
% This preserves the distinction between vectors and scalars. However,
% if the journal you are submitting to favors bold math in the abstract,
% then you can use LaTeX's standard command \boldmath at the very start
% of the abstract to achieve this. Many IEEE journals frown on math
% in the abstract anyway.

% Note that keywords are not normally used for peerreview papers.
%\begin{IEEEkeywords}
%Cooperative diversity, decode and forward, piecewise linear
%\end{IEEEkeywords}



% For peer review papers, you can put extra information on the cover
% page as needed:
% \ifCLASSOPTIONpeerreview
% \begin{center} \bfseries EDICS Category: 3-BBND \end{center}
% \fi
%
% For peerreview papers, this IEEEtran command inserts a page break and
% creates the second title. It will be ignored for other modes.
%\IEEEpeerreviewmaketitle




\item
If 4-digit numbers greater than 5,000 are randomly formed from the digits 0, 1, 3, 5, and 7, what is the probability of forming a number divisible by 5 when:
\begin{enumerate}
    \item The digits are repeated?
    \item The repetition of digits is not allowed?
\end{enumerate}
\solution
%\begin{table}[H]
	\centering
\begin{tabular}{|c|c|c|}
\hline
Random variable &Value &Definition\\ \hline
\multirow{3}{*}{X} &0 &Slips of Rs 1\\
&1 &Slips of Rs 5\\
&2 &Slips of Rs 13\\ \hline
\multirow{2}{*}{Y} &0 &Box A\\
&1 &Box B\\\hline
\end{tabular}
\caption{}
\label{tab:Distribution}
\end{table}
See \tabref{tab:Distribution}.
\begin{align}
p_{Y}\brak{k}= \begin{cases} 
      \frac{1}{3} & {k=0} \\
      \frac{2}{3 }& {k=1} 
   \end{cases}
   \\
p_{Y|X}\brak{0|0} = \frac{19}{25}\, 
p_{Y|X}\brak{0|1} = \frac{6}{25}\,
p_{Y|X}\brak{1|0} = \frac{45}{50}\,
p_{Y|X}\brak{1|2} = \frac{5}{50}
\end{align}
The desired probability is the probability that a slip drawn at random is marked other than Rs 1,
\begin{align}
&=1-p_X\brak{0}\\
&= p_X(1) + p_X(2)
\end{align}
Using Bayes theorem,
\begin{align}
&= p_Y\brak{0} \times \pr{Y=0 | X=1} + p_Y\brak{1} \times \pr{Y=1|X=2}\\
&=\frac{1}{3} \times \frac{6}{25} + \frac{2}{3} \times \frac{5}{50}\\
&=\frac{11}{75}
\end{align}

\newpage

%\tableofcontents

\bigskip

\renewcommand{\thefigure}{\theenumi}
\renewcommand{\thetable}{\theenumi}
%\renewcommand{\theequation}{\theenumi}

%\begin{abstract}
%%\boldmath
%In this letter, an algorithm for evaluating the exact analytical bit error rate  (BER)  for the piecewise linear (PL) combiner for  multiple relays is presented. Previous results were available only for upto three relays. The algorithm is unique in the sense that  the actual mathematical expressions, that are prohibitively large, need not be explicitly obtained. The diversity gain due to multiple relays is shown through plots of the analytical BER, well supported by simulations. 
%
%\end{abstract}
% IEEEtran.cls defaults to using nonbold math in the Abstract.
% This preserves the distinction between vectors and scalars. However,
% if the journal you are submitting to favors bold math in the abstract,
% then you can use LaTeX's standard command \boldmath at the very start
% of the abstract to achieve this. Many IEEE journals frown on math
% in the abstract anyway.

% Note that keywords are not normally used for peerreview papers.
%\begin{IEEEkeywords}
%Cooperative diversity, decode and forward, piecewise linear
%\end{IEEEkeywords}



% For peer review papers, you can put extra information on the cover
% page as needed:
% \ifCLASSOPTIONpeerreview
% \begin{center} \bfseries EDICS Category: 3-BBND \end{center}
% \fi
%
% For peerreview papers, this IEEEtran command inserts a page break and
% creates the second title. It will be ignored for other modes.
%\IEEEpeerreviewmaketitle




\item Consider the probability space $\brak{\Omega, \mathcal{G}, P}$ where $\Omega = [0,2]$ and $\mathcal{G} = \cbrak{\phi, \Omega, [0,1], (1,2]}$. Let $X$ and $Y$ be two functions on $\Omega$ defined as
\begin{align*}
    X(\omega) = 
    \begin{cases}
        1 & \text{if }\omega \in [0, 1]\\
        2 & \text{if }\omega \in (1, 2]
    \end{cases}
\end{align*}
and
\begin{align*}
    Y(\omega) = 
    \begin{cases}
        2 & \text{if }\omega \in [0, 1.5]\\
        3 & \text{if }\omega \in (1.5, 2].
    \end{cases}
\end{align*}
Then which one of the following statements is true?
\begin{enumerate}
    \item [(A)] $X$ is a random variable with respect to $\mathcal{G}$, but $Y$ is not a random variable with respect to $\mathcal{G}$.
    \item [(B)] $Y$ is a random variable with respect to $\mathcal{G}$, but $X$ is not a random variable with respect to $\mathcal{G}$.
    \item [(C)] Neither $X$ nor $Y$ is a random variable with respect to $\mathcal{G}$.
    \item [(D)] Both $X$ and $Y$ are random variables with respect to $\mathcal{G}$.
\end{enumerate} \hfill (GATE ST 2023)\\
\solution
%\begin{table}[H]
	\centering
\begin{tabular}{|c|c|c|}
\hline
Random variable &Value &Definition\\ \hline
\multirow{3}{*}{X} &0 &Slips of Rs 1\\
&1 &Slips of Rs 5\\
&2 &Slips of Rs 13\\ \hline
\multirow{2}{*}{Y} &0 &Box A\\
&1 &Box B\\\hline
\end{tabular}
\caption{}
\label{tab:Distribution}
\end{table}
See \tabref{tab:Distribution}.
\begin{align}
p_{Y}\brak{k}= \begin{cases} 
      \frac{1}{3} & {k=0} \\
      \frac{2}{3 }& {k=1} 
   \end{cases}
   \\
p_{Y|X}\brak{0|0} = \frac{19}{25}\, 
p_{Y|X}\brak{0|1} = \frac{6}{25}\,
p_{Y|X}\brak{1|0} = \frac{45}{50}\,
p_{Y|X}\brak{1|2} = \frac{5}{50}
\end{align}
The desired probability is the probability that a slip drawn at random is marked other than Rs 1,
\begin{align}
&=1-p_X\brak{0}\\
&= p_X(1) + p_X(2)
\end{align}
Using Bayes theorem,
\begin{align}
&= p_Y\brak{0} \times \pr{Y=0 | X=1} + p_Y\brak{1} \times \pr{Y=1|X=2}\\
&=\frac{1}{3} \times \frac{6}{25} + \frac{2}{3} \times \frac{5}{50}\\
&=\frac{11}{75}
\end{align}

\newpage

%\tableofcontents

\bigskip

\renewcommand{\thefigure}{\theenumi}
\renewcommand{\thetable}{\theenumi}
%\renewcommand{\theequation}{\theenumi}

%\begin{abstract}
%%\boldmath
%In this letter, an algorithm for evaluating the exact analytical bit error rate  (BER)  for the piecewise linear (PL) combiner for  multiple relays is presented. Previous results were available only for upto three relays. The algorithm is unique in the sense that  the actual mathematical expressions, that are prohibitively large, need not be explicitly obtained. The diversity gain due to multiple relays is shown through plots of the analytical BER, well supported by simulations. 
%
%\end{abstract}
% IEEEtran.cls defaults to using nonbold math in the Abstract.
% This preserves the distinction between vectors and scalars. However,
% if the journal you are submitting to favors bold math in the abstract,
% then you can use LaTeX's standard command \boldmath at the very start
% of the abstract to achieve this. Many IEEE journals frown on math
% in the abstract anyway.

% Note that keywords are not normally used for peerreview papers.
%\begin{IEEEkeywords}
%Cooperative diversity, decode and forward, piecewise linear
%\end{IEEEkeywords}



% For peer review papers, you can put extra information on the cover
% page as needed:
% \ifCLASSOPTIONpeerreview
% \begin{center} \bfseries EDICS Category: 3-BBND \end{center}
% \fi
%
% For peerreview papers, this IEEEtran command inserts a page break and
% creates the second title. It will be ignored for other modes.
%\IEEEpeerreviewmaketitle




	\item  A die is loaded in such a way that each odd number is twice as likely to occur as
each even number. Find $P(G)$, where $G$ is the event that a number greater than
3 occurs on a single roll of the die.
\\
\solution
		%\begin{table}[H]
	\centering
\begin{tabular}{|c|c|c|}
\hline
Random variable &Value &Definition\\ \hline
\multirow{3}{*}{X} &0 &Slips of Rs 1\\
&1 &Slips of Rs 5\\
&2 &Slips of Rs 13\\ \hline
\multirow{2}{*}{Y} &0 &Box A\\
&1 &Box B\\\hline
\end{tabular}
\caption{}
\label{tab:Distribution}
\end{table}
See \tabref{tab:Distribution}.
\begin{align}
p_{Y}\brak{k}= \begin{cases} 
      \frac{1}{3} & {k=0} \\
      \frac{2}{3 }& {k=1} 
   \end{cases}
   \\
p_{Y|X}\brak{0|0} = \frac{19}{25}\, 
p_{Y|X}\brak{0|1} = \frac{6}{25}\,
p_{Y|X}\brak{1|0} = \frac{45}{50}\,
p_{Y|X}\brak{1|2} = \frac{5}{50}
\end{align}
The desired probability is the probability that a slip drawn at random is marked other than Rs 1,
\begin{align}
&=1-p_X\brak{0}\\
&= p_X(1) + p_X(2)
\end{align}
Using Bayes theorem,
\begin{align}
&= p_Y\brak{0} \times \pr{Y=0 | X=1} + p_Y\brak{1} \times \pr{Y=1|X=2}\\
&=\frac{1}{3} \times \frac{6}{25} + \frac{2}{3} \times \frac{5}{50}\\
&=\frac{11}{75}
\end{align}

\newpage

%\tableofcontents

\bigskip

\renewcommand{\thefigure}{\theenumi}
\renewcommand{\thetable}{\theenumi}
%\renewcommand{\theequation}{\theenumi}

%\begin{abstract}
%%\boldmath
%In this letter, an algorithm for evaluating the exact analytical bit error rate  (BER)  for the piecewise linear (PL) combiner for  multiple relays is presented. Previous results were available only for upto three relays. The algorithm is unique in the sense that  the actual mathematical expressions, that are prohibitively large, need not be explicitly obtained. The diversity gain due to multiple relays is shown through plots of the analytical BER, well supported by simulations. 
%
%\end{abstract}
% IEEEtran.cls defaults to using nonbold math in the Abstract.
% This preserves the distinction between vectors and scalars. However,
% if the journal you are submitting to favors bold math in the abstract,
% then you can use LaTeX's standard command \boldmath at the very start
% of the abstract to achieve this. Many IEEE journals frown on math
% in the abstract anyway.

% Note that keywords are not normally used for peerreview papers.
%\begin{IEEEkeywords}
%Cooperative diversity, decode and forward, piecewise linear
%\end{IEEEkeywords}



% For peer review papers, you can put extra information on the cover
% page as needed:
% \ifCLASSOPTIONpeerreview
% \begin{center} \bfseries EDICS Category: 3-BBND \end{center}
% \fi
%
% For peerreview papers, this IEEEtran command inserts a page break and
% creates the second title. It will be ignored for other modes.
%\IEEEpeerreviewmaketitle




	\item All the jacks, queens and kings are removed from a deck of 52 playing cards. The remaining cards are well shuffled and then one card is drawn at random. Giving ace a value 1 similar value for other cards, find the probability that the card has a value 
		\begin{enumerate}
			\item 7
			\item greater than 7
			\item less than 7
		\end{enumerate}
		%Number of cards left after removing all jacks, queens and kings 
\begin{align}
N	= 52 - 4\times 3
	= 40
\end{align}
%\begin{table}[H]
%\def\arraystretch{1.2}
%\begin{tabular}{|c|c|c|}
%\hline
%	\textbf{Parameter} &\textbf{Value} &\textbf{Description}\\ \hline
%	$X$ &1-10 &Represents the value of the card picked \\ \hline
%\end{tabular}
%\end{table}
Let $1 \le X \le 10$ be the value of the card picked.  Then,
\begin{align}
	p_X(k) &= \Pr(X=k)\ \forall\ 1 \leq k \leq 10\\
	&= \frac{4\times 1}{40}\\
	&= \frac{1}{10}\\
	\therefore p_X(k) &= 
	\begin{cases}
		\frac{1}{10} & 1 \leq k \leq 10\\
		0 & \text{otherwise}
	\end{cases}
\end{align}
and
\begin{align}
	F_{X}(k) &= \sum_{m=0}^{k}p_{X}(m) \quad 1 \leq k \leq 10\\
	&= \frac{k}{10}\\
	\therefore F_{X}(k) &= 
	\begin{cases}
		0 & k \leq 0\\
		\frac{k}{10} & 1\leq k \leq 10\\
		1 & k > 10 
	\end{cases}
\end{align}
\begin{enumerate}
	\item Probability that card has value equal to 7 is
		\begin{align}
			 p_{X}(7)
			= \frac{1}{10}
		\end{align}
	\item Probability that card has value greater than 7 is
		\begin{align}
			1 - F_X(7)
			&= 1 - \frac{7}{10}
			\\
			&= \frac{3}{10}
		\end{align}
	\item Probability that card has value less than 7 is
		\begin{align}
			 F_{X}(6)
			=\frac{6}{10}
		\end{align}
\end{enumerate}

  \item A Lot consists of 48 mobile phones of which 42 are good, 3 have only minor defects and 3 have major defects.Varnika will buy a phone if it is good but the trader will only buy a mobile if it has no major defects. One phone is selected at random from the lot. What is the probability that it is
\begin{enumerate}
	\item acceptable to Varnika?
            \item acceptable to the trader?
\end{enumerate}
\solution
	%\begin{table}[H]
	\centering
\begin{tabular}{|c|c|c|}
\hline
Random variable &Value &Definition\\ \hline
\multirow{3}{*}{X} &0 &Slips of Rs 1\\
&1 &Slips of Rs 5\\
&2 &Slips of Rs 13\\ \hline
\multirow{2}{*}{Y} &0 &Box A\\
&1 &Box B\\\hline
\end{tabular}
\caption{}
\label{tab:Distribution}
\end{table}
See \tabref{tab:Distribution}.
\begin{align}
p_{Y}\brak{k}= \begin{cases} 
      \frac{1}{3} & {k=0} \\
      \frac{2}{3 }& {k=1} 
   \end{cases}
   \\
p_{Y|X}\brak{0|0} = \frac{19}{25}\, 
p_{Y|X}\brak{0|1} = \frac{6}{25}\,
p_{Y|X}\brak{1|0} = \frac{45}{50}\,
p_{Y|X}\brak{1|2} = \frac{5}{50}
\end{align}
The desired probability is the probability that a slip drawn at random is marked other than Rs 1,
\begin{align}
&=1-p_X\brak{0}\\
&= p_X(1) + p_X(2)
\end{align}
Using Bayes theorem,
\begin{align}
&= p_Y\brak{0} \times \pr{Y=0 | X=1} + p_Y\brak{1} \times \pr{Y=1|X=2}\\
&=\frac{1}{3} \times \frac{6}{25} + \frac{2}{3} \times \frac{5}{50}\\
&=\frac{11}{75}
\end{align}

\newpage

%\tableofcontents

\bigskip

\renewcommand{\thefigure}{\theenumi}
\renewcommand{\thetable}{\theenumi}
%\renewcommand{\theequation}{\theenumi}

%\begin{abstract}
%%\boldmath
%In this letter, an algorithm for evaluating the exact analytical bit error rate  (BER)  for the piecewise linear (PL) combiner for  multiple relays is presented. Previous results were available only for upto three relays. The algorithm is unique in the sense that  the actual mathematical expressions, that are prohibitively large, need not be explicitly obtained. The diversity gain due to multiple relays is shown through plots of the analytical BER, well supported by simulations. 
%
%\end{abstract}
% IEEEtran.cls defaults to using nonbold math in the Abstract.
% This preserves the distinction between vectors and scalars. However,
% if the journal you are submitting to favors bold math in the abstract,
% then you can use LaTeX's standard command \boldmath at the very start
% of the abstract to achieve this. Many IEEE journals frown on math
% in the abstract anyway.

% Note that keywords are not normally used for peerreview papers.
%\begin{IEEEkeywords}
%Cooperative diversity, decode and forward, piecewise linear
%\end{IEEEkeywords}



% For peer review papers, you can put extra information on the cover
% page as needed:
% \ifCLASSOPTIONpeerreview
% \begin{center} \bfseries EDICS Category: 3-BBND \end{center}
% \fi
%
% For peerreview papers, this IEEEtran command inserts a page break and
% creates the second title. It will be ignored for other modes.
%\IEEEpeerreviewmaketitle




 \item A student says that if you throw a die, it will show up 1 or not 1. Therefore, the probability of getting 1 and the probability of getting 'not 1' each is equal to $\frac{1}{2}$. Is this correct? Give reasons.\\
 \solution
        %\begin{table}[H]
	\centering
\begin{tabular}{|c|c|c|}
\hline
Random variable &Value &Definition\\ \hline
\multirow{3}{*}{X} &0 &Slips of Rs 1\\
&1 &Slips of Rs 5\\
&2 &Slips of Rs 13\\ \hline
\multirow{2}{*}{Y} &0 &Box A\\
&1 &Box B\\\hline
\end{tabular}
\caption{}
\label{tab:Distribution}
\end{table}
See \tabref{tab:Distribution}.
\begin{align}
p_{Y}\brak{k}= \begin{cases} 
      \frac{1}{3} & {k=0} \\
      \frac{2}{3 }& {k=1} 
   \end{cases}
   \\
p_{Y|X}\brak{0|0} = \frac{19}{25}\, 
p_{Y|X}\brak{0|1} = \frac{6}{25}\,
p_{Y|X}\brak{1|0} = \frac{45}{50}\,
p_{Y|X}\brak{1|2} = \frac{5}{50}
\end{align}
The desired probability is the probability that a slip drawn at random is marked other than Rs 1,
\begin{align}
&=1-p_X\brak{0}\\
&= p_X(1) + p_X(2)
\end{align}
Using Bayes theorem,
\begin{align}
&= p_Y\brak{0} \times \pr{Y=0 | X=1} + p_Y\brak{1} \times \pr{Y=1|X=2}\\
&=\frac{1}{3} \times \frac{6}{25} + \frac{2}{3} \times \frac{5}{50}\\
&=\frac{11}{75}
\end{align}

\newpage

%\tableofcontents

\bigskip

\renewcommand{\thefigure}{\theenumi}
\renewcommand{\thetable}{\theenumi}
%\renewcommand{\theequation}{\theenumi}

%\begin{abstract}
%%\boldmath
%In this letter, an algorithm for evaluating the exact analytical bit error rate  (BER)  for the piecewise linear (PL) combiner for  multiple relays is presented. Previous results were available only for upto three relays. The algorithm is unique in the sense that  the actual mathematical expressions, that are prohibitively large, need not be explicitly obtained. The diversity gain due to multiple relays is shown through plots of the analytical BER, well supported by simulations. 
%
%\end{abstract}
% IEEEtran.cls defaults to using nonbold math in the Abstract.
% This preserves the distinction between vectors and scalars. However,
% if the journal you are submitting to favors bold math in the abstract,
% then you can use LaTeX's standard command \boldmath at the very start
% of the abstract to achieve this. Many IEEE journals frown on math
% in the abstract anyway.

% Note that keywords are not normally used for peerreview papers.
%\begin{IEEEkeywords}
%Cooperative diversity, decode and forward, piecewise linear
%\end{IEEEkeywords}



% For peer review papers, you can put extra information on the cover
% page as needed:
% \ifCLASSOPTIONpeerreview
% \begin{center} \bfseries EDICS Category: 3-BBND \end{center}
% \fi
%
% For peerreview papers, this IEEEtran command inserts a page break and
% creates the second title. It will be ignored for other modes.
%\IEEEpeerreviewmaketitle




   \item Four candidates A, B, C, D have ap-
plied for the assignment to coach a school cricket
team. If A is twice as likely to be selected as B, and
B and C are given about the same chance of being
selected, while C is twice as likely to be selected
as D, what are the probabilities that
\begin{enumerate}
\item C will be selected?
\item A will not be selected?
\end{enumerate}
	%\begin{table}[H]
	\centering
\begin{tabular}{|c|c|c|}
\hline
Random variable &Value &Definition\\ \hline
\multirow{3}{*}{X} &0 &Slips of Rs 1\\
&1 &Slips of Rs 5\\
&2 &Slips of Rs 13\\ \hline
\multirow{2}{*}{Y} &0 &Box A\\
&1 &Box B\\\hline
\end{tabular}
\caption{}
\label{tab:Distribution}
\end{table}
See \tabref{tab:Distribution}.
\begin{align}
p_{Y}\brak{k}= \begin{cases} 
      \frac{1}{3} & {k=0} \\
      \frac{2}{3 }& {k=1} 
   \end{cases}
   \\
p_{Y|X}\brak{0|0} = \frac{19}{25}\, 
p_{Y|X}\brak{0|1} = \frac{6}{25}\,
p_{Y|X}\brak{1|0} = \frac{45}{50}\,
p_{Y|X}\brak{1|2} = \frac{5}{50}
\end{align}
The desired probability is the probability that a slip drawn at random is marked other than Rs 1,
\begin{align}
&=1-p_X\brak{0}\\
&= p_X(1) + p_X(2)
\end{align}
Using Bayes theorem,
\begin{align}
&= p_Y\brak{0} \times \pr{Y=0 | X=1} + p_Y\brak{1} \times \pr{Y=1|X=2}\\
&=\frac{1}{3} \times \frac{6}{25} + \frac{2}{3} \times \frac{5}{50}\\
&=\frac{11}{75}
\end{align}

\newpage

%\tableofcontents

\bigskip

\renewcommand{\thefigure}{\theenumi}
\renewcommand{\thetable}{\theenumi}
%\renewcommand{\theequation}{\theenumi}

%\begin{abstract}
%%\boldmath
%In this letter, an algorithm for evaluating the exact analytical bit error rate  (BER)  for the piecewise linear (PL) combiner for  multiple relays is presented. Previous results were available only for upto three relays. The algorithm is unique in the sense that  the actual mathematical expressions, that are prohibitively large, need not be explicitly obtained. The diversity gain due to multiple relays is shown through plots of the analytical BER, well supported by simulations. 
%
%\end{abstract}
% IEEEtran.cls defaults to using nonbold math in the Abstract.
% This preserves the distinction between vectors and scalars. However,
% if the journal you are submitting to favors bold math in the abstract,
% then you can use LaTeX's standard command \boldmath at the very start
% of the abstract to achieve this. Many IEEE journals frown on math
% in the abstract anyway.

% Note that keywords are not normally used for peerreview papers.
%\begin{IEEEkeywords}
%Cooperative diversity, decode and forward, piecewise linear
%\end{IEEEkeywords}



% For peer review papers, you can put extra information on the cover
% page as needed:
% \ifCLASSOPTIONpeerreview
% \begin{center} \bfseries EDICS Category: 3-BBND \end{center}
% \fi
%
% For peerreview papers, this IEEEtran command inserts a page break and
% creates the second title. It will be ignored for other modes.
%\IEEEpeerreviewmaketitle




 \item A bag contain 24 balls of which $x$ balls are red, $2x$ are white and $3x$ are blue. A ball is selected at random, What is the probability that it is
\begin{enumerate}[label=\alph*)]
\item not red ?
\item white ?
\end{enumerate}
%\begin{table}[H]
	\centering
\begin{tabular}{|c|c|c|}
\hline
Random variable &Value &Definition\\ \hline
\multirow{3}{*}{X} &0 &Slips of Rs 1\\
&1 &Slips of Rs 5\\
&2 &Slips of Rs 13\\ \hline
\multirow{2}{*}{Y} &0 &Box A\\
&1 &Box B\\\hline
\end{tabular}
\caption{}
\label{tab:Distribution}
\end{table}
See \tabref{tab:Distribution}.
\begin{align}
p_{Y}\brak{k}= \begin{cases} 
      \frac{1}{3} & {k=0} \\
      \frac{2}{3 }& {k=1} 
   \end{cases}
   \\
p_{Y|X}\brak{0|0} = \frac{19}{25}\, 
p_{Y|X}\brak{0|1} = \frac{6}{25}\,
p_{Y|X}\brak{1|0} = \frac{45}{50}\,
p_{Y|X}\brak{1|2} = \frac{5}{50}
\end{align}
The desired probability is the probability that a slip drawn at random is marked other than Rs 1,
\begin{align}
&=1-p_X\brak{0}\\
&= p_X(1) + p_X(2)
\end{align}
Using Bayes theorem,
\begin{align}
&= p_Y\brak{0} \times \pr{Y=0 | X=1} + p_Y\brak{1} \times \pr{Y=1|X=2}\\
&=\frac{1}{3} \times \frac{6}{25} + \frac{2}{3} \times \frac{5}{50}\\
&=\frac{11}{75}
\end{align}

\newpage

%\tableofcontents

\bigskip

\renewcommand{\thefigure}{\theenumi}
\renewcommand{\thetable}{\theenumi}
%\renewcommand{\theequation}{\theenumi}

%\begin{abstract}
%%\boldmath
%In this letter, an algorithm for evaluating the exact analytical bit error rate  (BER)  for the piecewise linear (PL) combiner for  multiple relays is presented. Previous results were available only for upto three relays. The algorithm is unique in the sense that  the actual mathematical expressions, that are prohibitively large, need not be explicitly obtained. The diversity gain due to multiple relays is shown through plots of the analytical BER, well supported by simulations. 
%
%\end{abstract}
% IEEEtran.cls defaults to using nonbold math in the Abstract.
% This preserves the distinction between vectors and scalars. However,
% if the journal you are submitting to favors bold math in the abstract,
% then you can use LaTeX's standard command \boldmath at the very start
% of the abstract to achieve this. Many IEEE journals frown on math
% in the abstract anyway.

% Note that keywords are not normally used for peerreview papers.
%\begin{IEEEkeywords}
%Cooperative diversity, decode and forward, piecewise linear
%\end{IEEEkeywords}



% For peer review papers, you can put extra information on the cover
% page as needed:
% \ifCLASSOPTIONpeerreview
% \begin{center} \bfseries EDICS Category: 3-BBND \end{center}
% \fi
%
% For peerreview papers, this IEEEtran command inserts a page break and
% creates the second title. It will be ignored for other modes.
%\IEEEpeerreviewmaketitle




If the letters of the word ASSASSINATION are arranged at random. Find the Probability that
\begin{enumerate}[label=(\alph*)]
\item Four $S's$ come consecutively in the word
\item Two  $I's$ and two $N's$ come together
\item All $A's$ are not coming together
\item No two $A's$ are coming together
\end{enumerate}
%\begin{table}[H]
	\centering
\begin{tabular}{|c|c|c|}
\hline
Random variable &Value &Definition\\ \hline
\multirow{3}{*}{X} &0 &Slips of Rs 1\\
&1 &Slips of Rs 5\\
&2 &Slips of Rs 13\\ \hline
\multirow{2}{*}{Y} &0 &Box A\\
&1 &Box B\\\hline
\end{tabular}
\caption{}
\label{tab:Distribution}
\end{table}
See \tabref{tab:Distribution}.
\begin{align}
p_{Y}\brak{k}= \begin{cases} 
      \frac{1}{3} & {k=0} \\
      \frac{2}{3 }& {k=1} 
   \end{cases}
   \\
p_{Y|X}\brak{0|0} = \frac{19}{25}\, 
p_{Y|X}\brak{0|1} = \frac{6}{25}\,
p_{Y|X}\brak{1|0} = \frac{45}{50}\,
p_{Y|X}\brak{1|2} = \frac{5}{50}
\end{align}
The desired probability is the probability that a slip drawn at random is marked other than Rs 1,
\begin{align}
&=1-p_X\brak{0}\\
&= p_X(1) + p_X(2)
\end{align}
Using Bayes theorem,
\begin{align}
&= p_Y\brak{0} \times \pr{Y=0 | X=1} + p_Y\brak{1} \times \pr{Y=1|X=2}\\
&=\frac{1}{3} \times \frac{6}{25} + \frac{2}{3} \times \frac{5}{50}\\
&=\frac{11}{75}
\end{align}

\newpage

%\tableofcontents

\bigskip

\renewcommand{\thefigure}{\theenumi}
\renewcommand{\thetable}{\theenumi}
%\renewcommand{\theequation}{\theenumi}

%\begin{abstract}
%%\boldmath
%In this letter, an algorithm for evaluating the exact analytical bit error rate  (BER)  for the piecewise linear (PL) combiner for  multiple relays is presented. Previous results were available only for upto three relays. The algorithm is unique in the sense that  the actual mathematical expressions, that are prohibitively large, need not be explicitly obtained. The diversity gain due to multiple relays is shown through plots of the analytical BER, well supported by simulations. 
%
%\end{abstract}
% IEEEtran.cls defaults to using nonbold math in the Abstract.
% This preserves the distinction between vectors and scalars. However,
% if the journal you are submitting to favors bold math in the abstract,
% then you can use LaTeX's standard command \boldmath at the very start
% of the abstract to achieve this. Many IEEE journals frown on math
% in the abstract anyway.

% Note that keywords are not normally used for peerreview papers.
%\begin{IEEEkeywords}
%Cooperative diversity, decode and forward, piecewise linear
%\end{IEEEkeywords}



% For peer review papers, you can put extra information on the cover
% page as needed:
% \ifCLASSOPTIONpeerreview
% \begin{center} \bfseries EDICS Category: 3-BBND \end{center}
% \fi
%
% For peerreview papers, this IEEEtran command inserts a page break and
% creates the second title. It will be ignored for other modes.
%\IEEEpeerreviewmaketitle




	\item One urn contains two black balls (labelled B1 and B2) and one white ball. A
	second urn contains one black ball and two white balls (labelled W1 and W2).
	Suppose the following experiment is performed. One of the two urns is chosen
	at random. Next a ball is randomly chosen from the urn. Then a second ball is
	chosen at random from the same urn without replacing the first ball.
	
	\begin{enumerate}
	\item What is the probability that two black balls are chosen?
	
	\item What is the probability that two balls of opposite colour are chosen?
	\end{enumerate}
	\solution
	%\begin{align}
    \label{eq:12.13.6.18.1}
	\because	\pr{A|B} &> \pr{A},\
\frac{\pr{AB}}{\pr{B}} > \pr{A}
\\
    \label{eq:12.13.6.18.2}
	\implies \pr{AB} &> \pr{A}\pr{B}
	\\
	\text{or, } \frac{\pr{AB}}{\pr{A}} &=\pr{B|A} > \pr{A}
\end{align}

\end{enumerate}

	\item A bag contains $5$ red balls and some blue balls. If the probability of drawing a blue ball is double that if a red ball, determine the number of blue balls in the bag. 
		\\
\solution
		%\begin{enumerate}[label=\thesection.\arabic*,ref=\thesection.\theenumi]
	\item One card is drawn from a well-shuffled deck of 52 cards. Find the probability of getting
\begin{enumerate}
\item A king of red colour 
\item A face card 
\item A red face card
\item The jack of hearts
\item A spade
\item The queen of diamonds

\end{enumerate}
\solution
		%\begin{table}[H]
	\centering
\begin{tabular}{|c|c|c|}
\hline
Random variable &Value &Definition\\ \hline
\multirow{3}{*}{X} &0 &Slips of Rs 1\\
&1 &Slips of Rs 5\\
&2 &Slips of Rs 13\\ \hline
\multirow{2}{*}{Y} &0 &Box A\\
&1 &Box B\\\hline
\end{tabular}
\caption{}
\label{tab:Distribution}
\end{table}
See \tabref{tab:Distribution}.
\begin{align}
p_{Y}\brak{k}= \begin{cases} 
      \frac{1}{3} & {k=0} \\
      \frac{2}{3 }& {k=1} 
   \end{cases}
   \\
p_{Y|X}\brak{0|0} = \frac{19}{25}\, 
p_{Y|X}\brak{0|1} = \frac{6}{25}\,
p_{Y|X}\brak{1|0} = \frac{45}{50}\,
p_{Y|X}\brak{1|2} = \frac{5}{50}
\end{align}
The desired probability is the probability that a slip drawn at random is marked other than Rs 1,
\begin{align}
&=1-p_X\brak{0}\\
&= p_X(1) + p_X(2)
\end{align}
Using Bayes theorem,
\begin{align}
&= p_Y\brak{0} \times \pr{Y=0 | X=1} + p_Y\brak{1} \times \pr{Y=1|X=2}\\
&=\frac{1}{3} \times \frac{6}{25} + \frac{2}{3} \times \frac{5}{50}\\
&=\frac{11}{75}
\end{align}

\newpage

%\tableofcontents

\bigskip

\renewcommand{\thefigure}{\theenumi}
\renewcommand{\thetable}{\theenumi}
%\renewcommand{\theequation}{\theenumi}

%\begin{abstract}
%%\boldmath
%In this letter, an algorithm for evaluating the exact analytical bit error rate  (BER)  for the piecewise linear (PL) combiner for  multiple relays is presented. Previous results were available only for upto three relays. The algorithm is unique in the sense that  the actual mathematical expressions, that are prohibitively large, need not be explicitly obtained. The diversity gain due to multiple relays is shown through plots of the analytical BER, well supported by simulations. 
%
%\end{abstract}
% IEEEtran.cls defaults to using nonbold math in the Abstract.
% This preserves the distinction between vectors and scalars. However,
% if the journal you are submitting to favors bold math in the abstract,
% then you can use LaTeX's standard command \boldmath at the very start
% of the abstract to achieve this. Many IEEE journals frown on math
% in the abstract anyway.

% Note that keywords are not normally used for peerreview papers.
%\begin{IEEEkeywords}
%Cooperative diversity, decode and forward, piecewise linear
%\end{IEEEkeywords}



% For peer review papers, you can put extra information on the cover
% page as needed:
% \ifCLASSOPTIONpeerreview
% \begin{center} \bfseries EDICS Category: 3-BBND \end{center}
% \fi
%
% For peerreview papers, this IEEEtran command inserts a page break and
% creates the second title. It will be ignored for other modes.
%\IEEEpeerreviewmaketitle




	\item Five cards—the ten, jack, queen, king and ace of diamonds, are well-shuffled with their face downwards. One card is then picked up at random.
\begin{enumerate}
\item
What is the probability that the card is the queen? 
\item
If the queen is drawn and put aside, what is the probability that the second card picked up is (a) an ace? (b) a queen?\\
\end{enumerate}
\solution
		%\begin{enumerate}[label=\thesection.\arabic*,ref=\thesection.\theenumi]
	\item One card is drawn from a well-shuffled deck of 52 cards. Find the probability of getting
\begin{enumerate}
\item A king of red colour 
\item A face card 
\item A red face card
\item The jack of hearts
\item A spade
\item The queen of diamonds

\end{enumerate}
\solution
		%\input{ncert/10/15/1/14/main.tex}
	\item Five cards—the ten, jack, queen, king and ace of diamonds, are well-shuffled with their face downwards. One card is then picked up at random.
\begin{enumerate}
\item
What is the probability that the card is the queen? 
\item
If the queen is drawn and put aside, what is the probability that the second card picked up is (a) an ace? (b) a queen?\\
\end{enumerate}
\solution
		%\input{ncert/10/15/1/15/defs.tex}
	\item A bag contains $5$ red balls and some blue balls. If the probability of drawing a blue ball is double that if a red ball, determine the number of blue balls in the bag. 
		\\
\solution
		%\input{ncert/10/15/2/3/defs.tex}
	\item A card is selected from a pack of 52 cards.
 \begin{enumerate}[label=(\alph*)] 
                 \item How many points are there in the sample space?
                 \item Calculate the probability that the card is an ace of spades.
                 \item Calculate the probability that the card is (i) an ace and (ii) black card.
 \end{enumerate}
\solution
		%\input{ncert/11/16/3/4/main.tex}
\item Four cards are drawn from a well-shuffled deck of 52 cards. What is the probability of obtaining 3 diamonds and one spade.
\\
\solution
		%\input{ncert/11/16/4/2/defs.tex}
\item In a certain lottery 10,000 tickets are sold and ten equal prizes are awarded. What is the probability of not getting a prize if you buy (a) one ticket (b) two tickets (c) 10 tickets ?	
\\
\solution
		%\input{ncert/11/16/4/4/defs.tex}
		%
\item 
Out of 100 students, two sections of 40 and 60 are formed. If you and your friend are among the 100 students, what is the probability that
\begin{enumerate}
\item you both enter the same section?
\item you both enter the different sections?
\end{enumerate}
\solution
		%\input{ncert/11/16/4/5/defs.tex}
	\item 
The number lock of a suitcase has 4 wheels each labelled with ten digits i.e. from 0 to 9.The lock opens with a sequence of four digits with no repeats.What is the probability of a person getting the right sequence to open the suitcase.
\\
\solution
		%\input{ncert/11/16/4/10/defs.tex}
		%
\item 
Two cards are drawn at random and without replacement from a pack of 52 playing cards. Find the probability that both the cards are black.
\\
\solution
		%\input{ncert/12/13/2/2/defs.tex}
		\item A box of oranges is inspected by examining three randomly selected oranges drawn without replacement. If all the three oranges are good, the box is approved for sale, otherwise, it is rejected. Find the probability that a box containing 15 oranges out of which 12 are good and 3 are bad ones will be approved for sale.
		\label{ncert/12/13/2/3/defs.tex}
		\item Two balls are drawn at random with replacement from a box containing 10 black and 8 red balls. Find the probability that
		\label{ncert/12/13/2/12}
\begin{enumerate}
\item both balls are red.
\item first ball is black and second is red.
\item one of them is black and other is red.
\end{enumerate}

\item In a hostel, 60\% of the students read Hindi newspaper, 40\% read English newspaper and 20\% read both Hindi and English newspapers. A student is selected at random.
		\label{ncert/12/13/2/15}
\begin{enumerate}
\item Find the probability that she reads neither Hindi nor English newspapers.
\item If she reads Hindi newspaper, find the probability that she reads English newspaper.
\item If she reads English newspaper, find the probability that she reads Hindi newspaper.\\
\end{enumerate}
\item The probability of obtaining an even prime number on each die, when a pair of dice is rolled is 
\begin{enumerate}
    \item $0$ 
    
    \item $\frac{1}{3}$ 
    
    \item $\frac{1}{12}$ 
    
    \item $\frac{1}{36}$ 
\end{enumerate}
\solution
		%\input{ncert/12/13/2/17/defs.tex}
	\item A bag contains 4 red and 4 black balls, another bag contains 2 red and 6 black balls. One of the two bags is selected at random and a ball is drawn from the bag which is found to be red. Find the probability that the ball is drawn from the first bag.
\\
\solution
		%\input{ncert/12/13/3/2/main.tex}
  \item
  Cards with numbers 2 to 101 are placed in a box. A card is selected at random.Find the probability that the card has
\begin{enumerate}[label=(\roman*)]
	\item an even number 
	\item a square number
\end{enumerate}
\solution
%\input{exemplar/10/13/3/32/main.tex}
\item
The king, queen and jack of clubs are removed from a deck of 52 playing cards and then well shuffled. Now one card is drawn at random from the remaining cards.  Determine the probability that the card is
\begin{enumerate}[label=(\roman*)]
\item a club
\item 10 of hearts
\end{enumerate}
\solution
%\input{exemplar/10/13/3/29/main.tex}
\item A team of medical students doing their internship have to assist during surgeries
at a city hospital. The probabilities of surgeries rated as very complex, complex,
routine, simple or very simple are respectively, 0.15, 0.20, 0.31, 0.26, .08. Find
the probabilities that a particular surgery will be rated
\begin{enumerate}
	\item complex or very complex;
	\item neither very complex nor very simple;
	\item routine or complex
	\item routine or simple
\end{enumerate}
\solution
%\input{exemplar/11/16/3/8(1)/main.tex}
\item A card is selected from a pack of 52 cards.
\begin{enumerate}[label=(\alph*)]
    \item How many points are there in the sample space?
    \item Calculate the probability that the card is an ace of spades.
    \item Calculate the probability that the card is (i) an ace and (ii) black card.
\end{enumerate}
\solution
%\input{exemplar/11/16/3/4/main2.tex}
\item The probability that a non leap year selected at random will contain 53 sundays.
\\
\solution
%\input{exemplar/10/13/1/19/main.tex}
\item One of the four persons John, Rita, Aslam or Gurpreet will be promoted next
month. Consequently the sample space consists of four elementary outcomes
S = {John promoted, Rita promoted, Aslam promoted, Gurpreet promoted}
You are told that the chances of John’s promotion is same as that of Gurpreet,
Rita’s chances of promotion are twice as likely as Johns. Aslam’s chances are
four times that of John.
\begin{enumerate}
	\item Determine
	\begin{enumerate}
		\item P (John promoted)
		\item P (Rita promoted)
		\item P (Aslam promoted)
		\item P (Gurpreet promoted)
	\end{enumerate}
	\item If A = {John promoted or Gurpreet promoted}, find P (A).
\end{enumerate}
\solution
%\input{exemplar/11/16/3/10/main.tex}
\item A card is drawn from a deck of 52 cards. Find the probability of getting a king or a heart or a red card.\\
\solution
%\input{exemplar/11/16/3/15/main.tex}
\item The probability that a student will pass his examination is 0.73, the probability of
the student getting a compartment is 0.13, and the probability that the student will
either pass or get compartment is 0.96. State True or False.\\
\solution
%\input{exemplar/11/16/3/31/main.tex}
\item A card is selected from a pack of 52 cards\\
\begin{enumerate}[label=(\alph*)]
\item How many points are there in the sample space?
\item Calculate the probability that the cards is an ace of spades.
\item Calculate the probability that the card is (i) an ace (ii)black card.\\
\end{enumerate}
%\input{ncert/11/16/3/4_1/Prob_4.tex}
\item In a non-leap year, the probability of having 53 tuesdays or 53 wednesdays is\\
\solution
%\input{exemplar/11/16/3/18/main.tex}
\item There are 1000 sealed envelopes in a box, 10 of them contain a cash prize of
Rs 100 each, 100 of them contain a cash prize of Rs 50 each and 200 of them
contain a cash prize of Rs 10 each and rest do not contain any cash prize. If they
are well shuffled and an envelope is picked up out, what is the probability that it
contains no cash prize?\\
\solution
%\input{exemplar/10/13/3/34/main.tex}
\item 
A die is thrown and a card is selected at random from a deck of 52 playing cards. The probability of getting an even number on the die and a spade card.\\
\solution
%\input{exemplar/12/13/3/78/main.tex}
\item
If 4-digit numbers greater than 5,000 are randomly formed from the digits 0, 1, 3, 5, and 7, what is the probability of forming a number divisible by 5 when:
\begin{enumerate}
    \item The digits are repeated?
    \item The repetition of digits is not allowed?
\end{enumerate}
\solution
%\input{ncert/11/16/4/9/main.tex}
\item Consider the probability space $\brak{\Omega, \mathcal{G}, P}$ where $\Omega = [0,2]$ and $\mathcal{G} = \cbrak{\phi, \Omega, [0,1], (1,2]}$. Let $X$ and $Y$ be two functions on $\Omega$ defined as
\begin{align*}
    X(\omega) = 
    \begin{cases}
        1 & \text{if }\omega \in [0, 1]\\
        2 & \text{if }\omega \in (1, 2]
    \end{cases}
\end{align*}
and
\begin{align*}
    Y(\omega) = 
    \begin{cases}
        2 & \text{if }\omega \in [0, 1.5]\\
        3 & \text{if }\omega \in (1.5, 2].
    \end{cases}
\end{align*}
Then which one of the following statements is true?
\begin{enumerate}
    \item [(A)] $X$ is a random variable with respect to $\mathcal{G}$, but $Y$ is not a random variable with respect to $\mathcal{G}$.
    \item [(B)] $Y$ is a random variable with respect to $\mathcal{G}$, but $X$ is not a random variable with respect to $\mathcal{G}$.
    \item [(C)] Neither $X$ nor $Y$ is a random variable with respect to $\mathcal{G}$.
    \item [(D)] Both $X$ and $Y$ are random variables with respect to $\mathcal{G}$.
\end{enumerate} \hfill (GATE ST 2023)\\
\solution
%\input{gate/ST/2023/14/main.tex}
	\item  A die is loaded in such a way that each odd number is twice as likely to occur as
each even number. Find $P(G)$, where $G$ is the event that a number greater than
3 occurs on a single roll of the die.
\\
\solution
		%\input{exemplar/11/16/3/5/main.tex}
	\item All the jacks, queens and kings are removed from a deck of 52 playing cards. The remaining cards are well shuffled and then one card is drawn at random. Giving ace a value 1 similar value for other cards, find the probability that the card has a value 
		\begin{enumerate}
			\item 7
			\item greater than 7
			\item less than 7
		\end{enumerate}
		%\input{exemplar/10/13/3/30/main.tex}
  \item A Lot consists of 48 mobile phones of which 42 are good, 3 have only minor defects and 3 have major defects.Varnika will buy a phone if it is good but the trader will only buy a mobile if it has no major defects. One phone is selected at random from the lot. What is the probability that it is
\begin{enumerate}
	\item acceptable to Varnika?
            \item acceptable to the trader?
\end{enumerate}
\solution
	%\input{exemplar/10/13/3/40/main.tex}
 \item A student says that if you throw a die, it will show up 1 or not 1. Therefore, the probability of getting 1 and the probability of getting 'not 1' each is equal to $\frac{1}{2}$. Is this correct? Give reasons.\\
 \solution
        %\input{exemplar/10/13/2/9/main.tex}
   \item Four candidates A, B, C, D have ap-
plied for the assignment to coach a school cricket
team. If A is twice as likely to be selected as B, and
B and C are given about the same chance of being
selected, while C is twice as likely to be selected
as D, what are the probabilities that
\begin{enumerate}
\item C will be selected?
\item A will not be selected?
\end{enumerate}
	%\input{exemplar/11/16/3/9/main.tex}
 \item A bag contain 24 balls of which $x$ balls are red, $2x$ are white and $3x$ are blue. A ball is selected at random, What is the probability that it is
\begin{enumerate}[label=\alph*)]
\item not red ?
\item white ?
\end{enumerate}
%\input{exemplar/10/13/3/41/main.tex}
If the letters of the word ASSASSINATION are arranged at random. Find the Probability that
\begin{enumerate}[label=(\alph*)]
\item Four $S's$ come consecutively in the word
\item Two  $I's$ and two $N's$ come together
\item All $A's$ are not coming together
\item No two $A's$ are coming together
\end{enumerate}
%\input{exemplar/11/16/3/14/main.tex}
	\item One urn contains two black balls (labelled B1 and B2) and one white ball. A
	second urn contains one black ball and two white balls (labelled W1 and W2).
	Suppose the following experiment is performed. One of the two urns is chosen
	at random. Next a ball is randomly chosen from the urn. Then a second ball is
	chosen at random from the same urn without replacing the first ball.
	
	\begin{enumerate}
	\item What is the probability that two black balls are chosen?
	
	\item What is the probability that two balls of opposite colour are chosen?
	\end{enumerate}
	\solution
	%\input{exemplar/11/16/3/12/main1.tex}
\end{enumerate}

	\item A bag contains $5$ red balls and some blue balls. If the probability of drawing a blue ball is double that if a red ball, determine the number of blue balls in the bag. 
		\\
\solution
		%\begin{enumerate}[label=\thesection.\arabic*,ref=\thesection.\theenumi]
	\item One card is drawn from a well-shuffled deck of 52 cards. Find the probability of getting
\begin{enumerate}
\item A king of red colour 
\item A face card 
\item A red face card
\item The jack of hearts
\item A spade
\item The queen of diamonds

\end{enumerate}
\solution
		%\input{ncert/10/15/1/14/main.tex}
	\item Five cards—the ten, jack, queen, king and ace of diamonds, are well-shuffled with their face downwards. One card is then picked up at random.
\begin{enumerate}
\item
What is the probability that the card is the queen? 
\item
If the queen is drawn and put aside, what is the probability that the second card picked up is (a) an ace? (b) a queen?\\
\end{enumerate}
\solution
		%\input{ncert/10/15/1/15/defs.tex}
	\item A bag contains $5$ red balls and some blue balls. If the probability of drawing a blue ball is double that if a red ball, determine the number of blue balls in the bag. 
		\\
\solution
		%\input{ncert/10/15/2/3/defs.tex}
	\item A card is selected from a pack of 52 cards.
 \begin{enumerate}[label=(\alph*)] 
                 \item How many points are there in the sample space?
                 \item Calculate the probability that the card is an ace of spades.
                 \item Calculate the probability that the card is (i) an ace and (ii) black card.
 \end{enumerate}
\solution
		%\input{ncert/11/16/3/4/main.tex}
\item Four cards are drawn from a well-shuffled deck of 52 cards. What is the probability of obtaining 3 diamonds and one spade.
\\
\solution
		%\input{ncert/11/16/4/2/defs.tex}
\item In a certain lottery 10,000 tickets are sold and ten equal prizes are awarded. What is the probability of not getting a prize if you buy (a) one ticket (b) two tickets (c) 10 tickets ?	
\\
\solution
		%\input{ncert/11/16/4/4/defs.tex}
		%
\item 
Out of 100 students, two sections of 40 and 60 are formed. If you and your friend are among the 100 students, what is the probability that
\begin{enumerate}
\item you both enter the same section?
\item you both enter the different sections?
\end{enumerate}
\solution
		%\input{ncert/11/16/4/5/defs.tex}
	\item 
The number lock of a suitcase has 4 wheels each labelled with ten digits i.e. from 0 to 9.The lock opens with a sequence of four digits with no repeats.What is the probability of a person getting the right sequence to open the suitcase.
\\
\solution
		%\input{ncert/11/16/4/10/defs.tex}
		%
\item 
Two cards are drawn at random and without replacement from a pack of 52 playing cards. Find the probability that both the cards are black.
\\
\solution
		%\input{ncert/12/13/2/2/defs.tex}
		\item A box of oranges is inspected by examining three randomly selected oranges drawn without replacement. If all the three oranges are good, the box is approved for sale, otherwise, it is rejected. Find the probability that a box containing 15 oranges out of which 12 are good and 3 are bad ones will be approved for sale.
		\label{ncert/12/13/2/3/defs.tex}
		\item Two balls are drawn at random with replacement from a box containing 10 black and 8 red balls. Find the probability that
		\label{ncert/12/13/2/12}
\begin{enumerate}
\item both balls are red.
\item first ball is black and second is red.
\item one of them is black and other is red.
\end{enumerate}

\item In a hostel, 60\% of the students read Hindi newspaper, 40\% read English newspaper and 20\% read both Hindi and English newspapers. A student is selected at random.
		\label{ncert/12/13/2/15}
\begin{enumerate}
\item Find the probability that she reads neither Hindi nor English newspapers.
\item If she reads Hindi newspaper, find the probability that she reads English newspaper.
\item If she reads English newspaper, find the probability that she reads Hindi newspaper.\\
\end{enumerate}
\item The probability of obtaining an even prime number on each die, when a pair of dice is rolled is 
\begin{enumerate}
    \item $0$ 
    
    \item $\frac{1}{3}$ 
    
    \item $\frac{1}{12}$ 
    
    \item $\frac{1}{36}$ 
\end{enumerate}
\solution
		%\input{ncert/12/13/2/17/defs.tex}
	\item A bag contains 4 red and 4 black balls, another bag contains 2 red and 6 black balls. One of the two bags is selected at random and a ball is drawn from the bag which is found to be red. Find the probability that the ball is drawn from the first bag.
\\
\solution
		%\input{ncert/12/13/3/2/main.tex}
  \item
  Cards with numbers 2 to 101 are placed in a box. A card is selected at random.Find the probability that the card has
\begin{enumerate}[label=(\roman*)]
	\item an even number 
	\item a square number
\end{enumerate}
\solution
%\input{exemplar/10/13/3/32/main.tex}
\item
The king, queen and jack of clubs are removed from a deck of 52 playing cards and then well shuffled. Now one card is drawn at random from the remaining cards.  Determine the probability that the card is
\begin{enumerate}[label=(\roman*)]
\item a club
\item 10 of hearts
\end{enumerate}
\solution
%\input{exemplar/10/13/3/29/main.tex}
\item A team of medical students doing their internship have to assist during surgeries
at a city hospital. The probabilities of surgeries rated as very complex, complex,
routine, simple or very simple are respectively, 0.15, 0.20, 0.31, 0.26, .08. Find
the probabilities that a particular surgery will be rated
\begin{enumerate}
	\item complex or very complex;
	\item neither very complex nor very simple;
	\item routine or complex
	\item routine or simple
\end{enumerate}
\solution
%\input{exemplar/11/16/3/8(1)/main.tex}
\item A card is selected from a pack of 52 cards.
\begin{enumerate}[label=(\alph*)]
    \item How many points are there in the sample space?
    \item Calculate the probability that the card is an ace of spades.
    \item Calculate the probability that the card is (i) an ace and (ii) black card.
\end{enumerate}
\solution
%\input{exemplar/11/16/3/4/main2.tex}
\item The probability that a non leap year selected at random will contain 53 sundays.
\\
\solution
%\input{exemplar/10/13/1/19/main.tex}
\item One of the four persons John, Rita, Aslam or Gurpreet will be promoted next
month. Consequently the sample space consists of four elementary outcomes
S = {John promoted, Rita promoted, Aslam promoted, Gurpreet promoted}
You are told that the chances of John’s promotion is same as that of Gurpreet,
Rita’s chances of promotion are twice as likely as Johns. Aslam’s chances are
four times that of John.
\begin{enumerate}
	\item Determine
	\begin{enumerate}
		\item P (John promoted)
		\item P (Rita promoted)
		\item P (Aslam promoted)
		\item P (Gurpreet promoted)
	\end{enumerate}
	\item If A = {John promoted or Gurpreet promoted}, find P (A).
\end{enumerate}
\solution
%\input{exemplar/11/16/3/10/main.tex}
\item A card is drawn from a deck of 52 cards. Find the probability of getting a king or a heart or a red card.\\
\solution
%\input{exemplar/11/16/3/15/main.tex}
\item The probability that a student will pass his examination is 0.73, the probability of
the student getting a compartment is 0.13, and the probability that the student will
either pass or get compartment is 0.96. State True or False.\\
\solution
%\input{exemplar/11/16/3/31/main.tex}
\item A card is selected from a pack of 52 cards\\
\begin{enumerate}[label=(\alph*)]
\item How many points are there in the sample space?
\item Calculate the probability that the cards is an ace of spades.
\item Calculate the probability that the card is (i) an ace (ii)black card.\\
\end{enumerate}
%\input{ncert/11/16/3/4_1/Prob_4.tex}
\item In a non-leap year, the probability of having 53 tuesdays or 53 wednesdays is\\
\solution
%\input{exemplar/11/16/3/18/main.tex}
\item There are 1000 sealed envelopes in a box, 10 of them contain a cash prize of
Rs 100 each, 100 of them contain a cash prize of Rs 50 each and 200 of them
contain a cash prize of Rs 10 each and rest do not contain any cash prize. If they
are well shuffled and an envelope is picked up out, what is the probability that it
contains no cash prize?\\
\solution
%\input{exemplar/10/13/3/34/main.tex}
\item 
A die is thrown and a card is selected at random from a deck of 52 playing cards. The probability of getting an even number on the die and a spade card.\\
\solution
%\input{exemplar/12/13/3/78/main.tex}
\item
If 4-digit numbers greater than 5,000 are randomly formed from the digits 0, 1, 3, 5, and 7, what is the probability of forming a number divisible by 5 when:
\begin{enumerate}
    \item The digits are repeated?
    \item The repetition of digits is not allowed?
\end{enumerate}
\solution
%\input{ncert/11/16/4/9/main.tex}
\item Consider the probability space $\brak{\Omega, \mathcal{G}, P}$ where $\Omega = [0,2]$ and $\mathcal{G} = \cbrak{\phi, \Omega, [0,1], (1,2]}$. Let $X$ and $Y$ be two functions on $\Omega$ defined as
\begin{align*}
    X(\omega) = 
    \begin{cases}
        1 & \text{if }\omega \in [0, 1]\\
        2 & \text{if }\omega \in (1, 2]
    \end{cases}
\end{align*}
and
\begin{align*}
    Y(\omega) = 
    \begin{cases}
        2 & \text{if }\omega \in [0, 1.5]\\
        3 & \text{if }\omega \in (1.5, 2].
    \end{cases}
\end{align*}
Then which one of the following statements is true?
\begin{enumerate}
    \item [(A)] $X$ is a random variable with respect to $\mathcal{G}$, but $Y$ is not a random variable with respect to $\mathcal{G}$.
    \item [(B)] $Y$ is a random variable with respect to $\mathcal{G}$, but $X$ is not a random variable with respect to $\mathcal{G}$.
    \item [(C)] Neither $X$ nor $Y$ is a random variable with respect to $\mathcal{G}$.
    \item [(D)] Both $X$ and $Y$ are random variables with respect to $\mathcal{G}$.
\end{enumerate} \hfill (GATE ST 2023)\\
\solution
%\input{gate/ST/2023/14/main.tex}
	\item  A die is loaded in such a way that each odd number is twice as likely to occur as
each even number. Find $P(G)$, where $G$ is the event that a number greater than
3 occurs on a single roll of the die.
\\
\solution
		%\input{exemplar/11/16/3/5/main.tex}
	\item All the jacks, queens and kings are removed from a deck of 52 playing cards. The remaining cards are well shuffled and then one card is drawn at random. Giving ace a value 1 similar value for other cards, find the probability that the card has a value 
		\begin{enumerate}
			\item 7
			\item greater than 7
			\item less than 7
		\end{enumerate}
		%\input{exemplar/10/13/3/30/main.tex}
  \item A Lot consists of 48 mobile phones of which 42 are good, 3 have only minor defects and 3 have major defects.Varnika will buy a phone if it is good but the trader will only buy a mobile if it has no major defects. One phone is selected at random from the lot. What is the probability that it is
\begin{enumerate}
	\item acceptable to Varnika?
            \item acceptable to the trader?
\end{enumerate}
\solution
	%\input{exemplar/10/13/3/40/main.tex}
 \item A student says that if you throw a die, it will show up 1 or not 1. Therefore, the probability of getting 1 and the probability of getting 'not 1' each is equal to $\frac{1}{2}$. Is this correct? Give reasons.\\
 \solution
        %\input{exemplar/10/13/2/9/main.tex}
   \item Four candidates A, B, C, D have ap-
plied for the assignment to coach a school cricket
team. If A is twice as likely to be selected as B, and
B and C are given about the same chance of being
selected, while C is twice as likely to be selected
as D, what are the probabilities that
\begin{enumerate}
\item C will be selected?
\item A will not be selected?
\end{enumerate}
	%\input{exemplar/11/16/3/9/main.tex}
 \item A bag contain 24 balls of which $x$ balls are red, $2x$ are white and $3x$ are blue. A ball is selected at random, What is the probability that it is
\begin{enumerate}[label=\alph*)]
\item not red ?
\item white ?
\end{enumerate}
%\input{exemplar/10/13/3/41/main.tex}
If the letters of the word ASSASSINATION are arranged at random. Find the Probability that
\begin{enumerate}[label=(\alph*)]
\item Four $S's$ come consecutively in the word
\item Two  $I's$ and two $N's$ come together
\item All $A's$ are not coming together
\item No two $A's$ are coming together
\end{enumerate}
%\input{exemplar/11/16/3/14/main.tex}
	\item One urn contains two black balls (labelled B1 and B2) and one white ball. A
	second urn contains one black ball and two white balls (labelled W1 and W2).
	Suppose the following experiment is performed. One of the two urns is chosen
	at random. Next a ball is randomly chosen from the urn. Then a second ball is
	chosen at random from the same urn without replacing the first ball.
	
	\begin{enumerate}
	\item What is the probability that two black balls are chosen?
	
	\item What is the probability that two balls of opposite colour are chosen?
	\end{enumerate}
	\solution
	%\input{exemplar/11/16/3/12/main1.tex}
\end{enumerate}

	\item A card is selected from a pack of 52 cards.
 \begin{enumerate}[label=(\alph*)] 
                 \item How many points are there in the sample space?
                 \item Calculate the probability that the card is an ace of spades.
                 \item Calculate the probability that the card is (i) an ace and (ii) black card.
 \end{enumerate}
\solution
		%\begin{table}[H]
	\centering
\begin{tabular}{|c|c|c|}
\hline
Random variable &Value &Definition\\ \hline
\multirow{3}{*}{X} &0 &Slips of Rs 1\\
&1 &Slips of Rs 5\\
&2 &Slips of Rs 13\\ \hline
\multirow{2}{*}{Y} &0 &Box A\\
&1 &Box B\\\hline
\end{tabular}
\caption{}
\label{tab:Distribution}
\end{table}
See \tabref{tab:Distribution}.
\begin{align}
p_{Y}\brak{k}= \begin{cases} 
      \frac{1}{3} & {k=0} \\
      \frac{2}{3 }& {k=1} 
   \end{cases}
   \\
p_{Y|X}\brak{0|0} = \frac{19}{25}\, 
p_{Y|X}\brak{0|1} = \frac{6}{25}\,
p_{Y|X}\brak{1|0} = \frac{45}{50}\,
p_{Y|X}\brak{1|2} = \frac{5}{50}
\end{align}
The desired probability is the probability that a slip drawn at random is marked other than Rs 1,
\begin{align}
&=1-p_X\brak{0}\\
&= p_X(1) + p_X(2)
\end{align}
Using Bayes theorem,
\begin{align}
&= p_Y\brak{0} \times \pr{Y=0 | X=1} + p_Y\brak{1} \times \pr{Y=1|X=2}\\
&=\frac{1}{3} \times \frac{6}{25} + \frac{2}{3} \times \frac{5}{50}\\
&=\frac{11}{75}
\end{align}

\newpage

%\tableofcontents

\bigskip

\renewcommand{\thefigure}{\theenumi}
\renewcommand{\thetable}{\theenumi}
%\renewcommand{\theequation}{\theenumi}

%\begin{abstract}
%%\boldmath
%In this letter, an algorithm for evaluating the exact analytical bit error rate  (BER)  for the piecewise linear (PL) combiner for  multiple relays is presented. Previous results were available only for upto three relays. The algorithm is unique in the sense that  the actual mathematical expressions, that are prohibitively large, need not be explicitly obtained. The diversity gain due to multiple relays is shown through plots of the analytical BER, well supported by simulations. 
%
%\end{abstract}
% IEEEtran.cls defaults to using nonbold math in the Abstract.
% This preserves the distinction between vectors and scalars. However,
% if the journal you are submitting to favors bold math in the abstract,
% then you can use LaTeX's standard command \boldmath at the very start
% of the abstract to achieve this. Many IEEE journals frown on math
% in the abstract anyway.

% Note that keywords are not normally used for peerreview papers.
%\begin{IEEEkeywords}
%Cooperative diversity, decode and forward, piecewise linear
%\end{IEEEkeywords}



% For peer review papers, you can put extra information on the cover
% page as needed:
% \ifCLASSOPTIONpeerreview
% \begin{center} \bfseries EDICS Category: 3-BBND \end{center}
% \fi
%
% For peerreview papers, this IEEEtran command inserts a page break and
% creates the second title. It will be ignored for other modes.
%\IEEEpeerreviewmaketitle




\item Four cards are drawn from a well-shuffled deck of 52 cards. What is the probability of obtaining 3 diamonds and one spade.
\\
\solution
		%\begin{enumerate}[label=\thesection.\arabic*,ref=\thesection.\theenumi]
	\item One card is drawn from a well-shuffled deck of 52 cards. Find the probability of getting
\begin{enumerate}
\item A king of red colour 
\item A face card 
\item A red face card
\item The jack of hearts
\item A spade
\item The queen of diamonds

\end{enumerate}
\solution
		%\input{ncert/10/15/1/14/main.tex}
	\item Five cards—the ten, jack, queen, king and ace of diamonds, are well-shuffled with their face downwards. One card is then picked up at random.
\begin{enumerate}
\item
What is the probability that the card is the queen? 
\item
If the queen is drawn and put aside, what is the probability that the second card picked up is (a) an ace? (b) a queen?\\
\end{enumerate}
\solution
		%\input{ncert/10/15/1/15/defs.tex}
	\item A bag contains $5$ red balls and some blue balls. If the probability of drawing a blue ball is double that if a red ball, determine the number of blue balls in the bag. 
		\\
\solution
		%\input{ncert/10/15/2/3/defs.tex}
	\item A card is selected from a pack of 52 cards.
 \begin{enumerate}[label=(\alph*)] 
                 \item How many points are there in the sample space?
                 \item Calculate the probability that the card is an ace of spades.
                 \item Calculate the probability that the card is (i) an ace and (ii) black card.
 \end{enumerate}
\solution
		%\input{ncert/11/16/3/4/main.tex}
\item Four cards are drawn from a well-shuffled deck of 52 cards. What is the probability of obtaining 3 diamonds and one spade.
\\
\solution
		%\input{ncert/11/16/4/2/defs.tex}
\item In a certain lottery 10,000 tickets are sold and ten equal prizes are awarded. What is the probability of not getting a prize if you buy (a) one ticket (b) two tickets (c) 10 tickets ?	
\\
\solution
		%\input{ncert/11/16/4/4/defs.tex}
		%
\item 
Out of 100 students, two sections of 40 and 60 are formed. If you and your friend are among the 100 students, what is the probability that
\begin{enumerate}
\item you both enter the same section?
\item you both enter the different sections?
\end{enumerate}
\solution
		%\input{ncert/11/16/4/5/defs.tex}
	\item 
The number lock of a suitcase has 4 wheels each labelled with ten digits i.e. from 0 to 9.The lock opens with a sequence of four digits with no repeats.What is the probability of a person getting the right sequence to open the suitcase.
\\
\solution
		%\input{ncert/11/16/4/10/defs.tex}
		%
\item 
Two cards are drawn at random and without replacement from a pack of 52 playing cards. Find the probability that both the cards are black.
\\
\solution
		%\input{ncert/12/13/2/2/defs.tex}
		\item A box of oranges is inspected by examining three randomly selected oranges drawn without replacement. If all the three oranges are good, the box is approved for sale, otherwise, it is rejected. Find the probability that a box containing 15 oranges out of which 12 are good and 3 are bad ones will be approved for sale.
		\label{ncert/12/13/2/3/defs.tex}
		\item Two balls are drawn at random with replacement from a box containing 10 black and 8 red balls. Find the probability that
		\label{ncert/12/13/2/12}
\begin{enumerate}
\item both balls are red.
\item first ball is black and second is red.
\item one of them is black and other is red.
\end{enumerate}

\item In a hostel, 60\% of the students read Hindi newspaper, 40\% read English newspaper and 20\% read both Hindi and English newspapers. A student is selected at random.
		\label{ncert/12/13/2/15}
\begin{enumerate}
\item Find the probability that she reads neither Hindi nor English newspapers.
\item If she reads Hindi newspaper, find the probability that she reads English newspaper.
\item If she reads English newspaper, find the probability that she reads Hindi newspaper.\\
\end{enumerate}
\item The probability of obtaining an even prime number on each die, when a pair of dice is rolled is 
\begin{enumerate}
    \item $0$ 
    
    \item $\frac{1}{3}$ 
    
    \item $\frac{1}{12}$ 
    
    \item $\frac{1}{36}$ 
\end{enumerate}
\solution
		%\input{ncert/12/13/2/17/defs.tex}
	\item A bag contains 4 red and 4 black balls, another bag contains 2 red and 6 black balls. One of the two bags is selected at random and a ball is drawn from the bag which is found to be red. Find the probability that the ball is drawn from the first bag.
\\
\solution
		%\input{ncert/12/13/3/2/main.tex}
  \item
  Cards with numbers 2 to 101 are placed in a box. A card is selected at random.Find the probability that the card has
\begin{enumerate}[label=(\roman*)]
	\item an even number 
	\item a square number
\end{enumerate}
\solution
%\input{exemplar/10/13/3/32/main.tex}
\item
The king, queen and jack of clubs are removed from a deck of 52 playing cards and then well shuffled. Now one card is drawn at random from the remaining cards.  Determine the probability that the card is
\begin{enumerate}[label=(\roman*)]
\item a club
\item 10 of hearts
\end{enumerate}
\solution
%\input{exemplar/10/13/3/29/main.tex}
\item A team of medical students doing their internship have to assist during surgeries
at a city hospital. The probabilities of surgeries rated as very complex, complex,
routine, simple or very simple are respectively, 0.15, 0.20, 0.31, 0.26, .08. Find
the probabilities that a particular surgery will be rated
\begin{enumerate}
	\item complex or very complex;
	\item neither very complex nor very simple;
	\item routine or complex
	\item routine or simple
\end{enumerate}
\solution
%\input{exemplar/11/16/3/8(1)/main.tex}
\item A card is selected from a pack of 52 cards.
\begin{enumerate}[label=(\alph*)]
    \item How many points are there in the sample space?
    \item Calculate the probability that the card is an ace of spades.
    \item Calculate the probability that the card is (i) an ace and (ii) black card.
\end{enumerate}
\solution
%\input{exemplar/11/16/3/4/main2.tex}
\item The probability that a non leap year selected at random will contain 53 sundays.
\\
\solution
%\input{exemplar/10/13/1/19/main.tex}
\item One of the four persons John, Rita, Aslam or Gurpreet will be promoted next
month. Consequently the sample space consists of four elementary outcomes
S = {John promoted, Rita promoted, Aslam promoted, Gurpreet promoted}
You are told that the chances of John’s promotion is same as that of Gurpreet,
Rita’s chances of promotion are twice as likely as Johns. Aslam’s chances are
four times that of John.
\begin{enumerate}
	\item Determine
	\begin{enumerate}
		\item P (John promoted)
		\item P (Rita promoted)
		\item P (Aslam promoted)
		\item P (Gurpreet promoted)
	\end{enumerate}
	\item If A = {John promoted or Gurpreet promoted}, find P (A).
\end{enumerate}
\solution
%\input{exemplar/11/16/3/10/main.tex}
\item A card is drawn from a deck of 52 cards. Find the probability of getting a king or a heart or a red card.\\
\solution
%\input{exemplar/11/16/3/15/main.tex}
\item The probability that a student will pass his examination is 0.73, the probability of
the student getting a compartment is 0.13, and the probability that the student will
either pass or get compartment is 0.96. State True or False.\\
\solution
%\input{exemplar/11/16/3/31/main.tex}
\item A card is selected from a pack of 52 cards\\
\begin{enumerate}[label=(\alph*)]
\item How many points are there in the sample space?
\item Calculate the probability that the cards is an ace of spades.
\item Calculate the probability that the card is (i) an ace (ii)black card.\\
\end{enumerate}
%\input{ncert/11/16/3/4_1/Prob_4.tex}
\item In a non-leap year, the probability of having 53 tuesdays or 53 wednesdays is\\
\solution
%\input{exemplar/11/16/3/18/main.tex}
\item There are 1000 sealed envelopes in a box, 10 of them contain a cash prize of
Rs 100 each, 100 of them contain a cash prize of Rs 50 each and 200 of them
contain a cash prize of Rs 10 each and rest do not contain any cash prize. If they
are well shuffled and an envelope is picked up out, what is the probability that it
contains no cash prize?\\
\solution
%\input{exemplar/10/13/3/34/main.tex}
\item 
A die is thrown and a card is selected at random from a deck of 52 playing cards. The probability of getting an even number on the die and a spade card.\\
\solution
%\input{exemplar/12/13/3/78/main.tex}
\item
If 4-digit numbers greater than 5,000 are randomly formed from the digits 0, 1, 3, 5, and 7, what is the probability of forming a number divisible by 5 when:
\begin{enumerate}
    \item The digits are repeated?
    \item The repetition of digits is not allowed?
\end{enumerate}
\solution
%\input{ncert/11/16/4/9/main.tex}
\item Consider the probability space $\brak{\Omega, \mathcal{G}, P}$ where $\Omega = [0,2]$ and $\mathcal{G} = \cbrak{\phi, \Omega, [0,1], (1,2]}$. Let $X$ and $Y$ be two functions on $\Omega$ defined as
\begin{align*}
    X(\omega) = 
    \begin{cases}
        1 & \text{if }\omega \in [0, 1]\\
        2 & \text{if }\omega \in (1, 2]
    \end{cases}
\end{align*}
and
\begin{align*}
    Y(\omega) = 
    \begin{cases}
        2 & \text{if }\omega \in [0, 1.5]\\
        3 & \text{if }\omega \in (1.5, 2].
    \end{cases}
\end{align*}
Then which one of the following statements is true?
\begin{enumerate}
    \item [(A)] $X$ is a random variable with respect to $\mathcal{G}$, but $Y$ is not a random variable with respect to $\mathcal{G}$.
    \item [(B)] $Y$ is a random variable with respect to $\mathcal{G}$, but $X$ is not a random variable with respect to $\mathcal{G}$.
    \item [(C)] Neither $X$ nor $Y$ is a random variable with respect to $\mathcal{G}$.
    \item [(D)] Both $X$ and $Y$ are random variables with respect to $\mathcal{G}$.
\end{enumerate} \hfill (GATE ST 2023)\\
\solution
%\input{gate/ST/2023/14/main.tex}
	\item  A die is loaded in such a way that each odd number is twice as likely to occur as
each even number. Find $P(G)$, where $G$ is the event that a number greater than
3 occurs on a single roll of the die.
\\
\solution
		%\input{exemplar/11/16/3/5/main.tex}
	\item All the jacks, queens and kings are removed from a deck of 52 playing cards. The remaining cards are well shuffled and then one card is drawn at random. Giving ace a value 1 similar value for other cards, find the probability that the card has a value 
		\begin{enumerate}
			\item 7
			\item greater than 7
			\item less than 7
		\end{enumerate}
		%\input{exemplar/10/13/3/30/main.tex}
  \item A Lot consists of 48 mobile phones of which 42 are good, 3 have only minor defects and 3 have major defects.Varnika will buy a phone if it is good but the trader will only buy a mobile if it has no major defects. One phone is selected at random from the lot. What is the probability that it is
\begin{enumerate}
	\item acceptable to Varnika?
            \item acceptable to the trader?
\end{enumerate}
\solution
	%\input{exemplar/10/13/3/40/main.tex}
 \item A student says that if you throw a die, it will show up 1 or not 1. Therefore, the probability of getting 1 and the probability of getting 'not 1' each is equal to $\frac{1}{2}$. Is this correct? Give reasons.\\
 \solution
        %\input{exemplar/10/13/2/9/main.tex}
   \item Four candidates A, B, C, D have ap-
plied for the assignment to coach a school cricket
team. If A is twice as likely to be selected as B, and
B and C are given about the same chance of being
selected, while C is twice as likely to be selected
as D, what are the probabilities that
\begin{enumerate}
\item C will be selected?
\item A will not be selected?
\end{enumerate}
	%\input{exemplar/11/16/3/9/main.tex}
 \item A bag contain 24 balls of which $x$ balls are red, $2x$ are white and $3x$ are blue. A ball is selected at random, What is the probability that it is
\begin{enumerate}[label=\alph*)]
\item not red ?
\item white ?
\end{enumerate}
%\input{exemplar/10/13/3/41/main.tex}
If the letters of the word ASSASSINATION are arranged at random. Find the Probability that
\begin{enumerate}[label=(\alph*)]
\item Four $S's$ come consecutively in the word
\item Two  $I's$ and two $N's$ come together
\item All $A's$ are not coming together
\item No two $A's$ are coming together
\end{enumerate}
%\input{exemplar/11/16/3/14/main.tex}
	\item One urn contains two black balls (labelled B1 and B2) and one white ball. A
	second urn contains one black ball and two white balls (labelled W1 and W2).
	Suppose the following experiment is performed. One of the two urns is chosen
	at random. Next a ball is randomly chosen from the urn. Then a second ball is
	chosen at random from the same urn without replacing the first ball.
	
	\begin{enumerate}
	\item What is the probability that two black balls are chosen?
	
	\item What is the probability that two balls of opposite colour are chosen?
	\end{enumerate}
	\solution
	%\input{exemplar/11/16/3/12/main1.tex}
\end{enumerate}

\item In a certain lottery 10,000 tickets are sold and ten equal prizes are awarded. What is the probability of not getting a prize if you buy (a) one ticket (b) two tickets (c) 10 tickets ?	
\\
\solution
		%\begin{enumerate}[label=\thesection.\arabic*,ref=\thesection.\theenumi]
	\item One card is drawn from a well-shuffled deck of 52 cards. Find the probability of getting
\begin{enumerate}
\item A king of red colour 
\item A face card 
\item A red face card
\item The jack of hearts
\item A spade
\item The queen of diamonds

\end{enumerate}
\solution
		%\input{ncert/10/15/1/14/main.tex}
	\item Five cards—the ten, jack, queen, king and ace of diamonds, are well-shuffled with their face downwards. One card is then picked up at random.
\begin{enumerate}
\item
What is the probability that the card is the queen? 
\item
If the queen is drawn and put aside, what is the probability that the second card picked up is (a) an ace? (b) a queen?\\
\end{enumerate}
\solution
		%\input{ncert/10/15/1/15/defs.tex}
	\item A bag contains $5$ red balls and some blue balls. If the probability of drawing a blue ball is double that if a red ball, determine the number of blue balls in the bag. 
		\\
\solution
		%\input{ncert/10/15/2/3/defs.tex}
	\item A card is selected from a pack of 52 cards.
 \begin{enumerate}[label=(\alph*)] 
                 \item How many points are there in the sample space?
                 \item Calculate the probability that the card is an ace of spades.
                 \item Calculate the probability that the card is (i) an ace and (ii) black card.
 \end{enumerate}
\solution
		%\input{ncert/11/16/3/4/main.tex}
\item Four cards are drawn from a well-shuffled deck of 52 cards. What is the probability of obtaining 3 diamonds and one spade.
\\
\solution
		%\input{ncert/11/16/4/2/defs.tex}
\item In a certain lottery 10,000 tickets are sold and ten equal prizes are awarded. What is the probability of not getting a prize if you buy (a) one ticket (b) two tickets (c) 10 tickets ?	
\\
\solution
		%\input{ncert/11/16/4/4/defs.tex}
		%
\item 
Out of 100 students, two sections of 40 and 60 are formed. If you and your friend are among the 100 students, what is the probability that
\begin{enumerate}
\item you both enter the same section?
\item you both enter the different sections?
\end{enumerate}
\solution
		%\input{ncert/11/16/4/5/defs.tex}
	\item 
The number lock of a suitcase has 4 wheels each labelled with ten digits i.e. from 0 to 9.The lock opens with a sequence of four digits with no repeats.What is the probability of a person getting the right sequence to open the suitcase.
\\
\solution
		%\input{ncert/11/16/4/10/defs.tex}
		%
\item 
Two cards are drawn at random and without replacement from a pack of 52 playing cards. Find the probability that both the cards are black.
\\
\solution
		%\input{ncert/12/13/2/2/defs.tex}
		\item A box of oranges is inspected by examining three randomly selected oranges drawn without replacement. If all the three oranges are good, the box is approved for sale, otherwise, it is rejected. Find the probability that a box containing 15 oranges out of which 12 are good and 3 are bad ones will be approved for sale.
		\label{ncert/12/13/2/3/defs.tex}
		\item Two balls are drawn at random with replacement from a box containing 10 black and 8 red balls. Find the probability that
		\label{ncert/12/13/2/12}
\begin{enumerate}
\item both balls are red.
\item first ball is black and second is red.
\item one of them is black and other is red.
\end{enumerate}

\item In a hostel, 60\% of the students read Hindi newspaper, 40\% read English newspaper and 20\% read both Hindi and English newspapers. A student is selected at random.
		\label{ncert/12/13/2/15}
\begin{enumerate}
\item Find the probability that she reads neither Hindi nor English newspapers.
\item If she reads Hindi newspaper, find the probability that she reads English newspaper.
\item If she reads English newspaper, find the probability that she reads Hindi newspaper.\\
\end{enumerate}
\item The probability of obtaining an even prime number on each die, when a pair of dice is rolled is 
\begin{enumerate}
    \item $0$ 
    
    \item $\frac{1}{3}$ 
    
    \item $\frac{1}{12}$ 
    
    \item $\frac{1}{36}$ 
\end{enumerate}
\solution
		%\input{ncert/12/13/2/17/defs.tex}
	\item A bag contains 4 red and 4 black balls, another bag contains 2 red and 6 black balls. One of the two bags is selected at random and a ball is drawn from the bag which is found to be red. Find the probability that the ball is drawn from the first bag.
\\
\solution
		%\input{ncert/12/13/3/2/main.tex}
  \item
  Cards with numbers 2 to 101 are placed in a box. A card is selected at random.Find the probability that the card has
\begin{enumerate}[label=(\roman*)]
	\item an even number 
	\item a square number
\end{enumerate}
\solution
%\input{exemplar/10/13/3/32/main.tex}
\item
The king, queen and jack of clubs are removed from a deck of 52 playing cards and then well shuffled. Now one card is drawn at random from the remaining cards.  Determine the probability that the card is
\begin{enumerate}[label=(\roman*)]
\item a club
\item 10 of hearts
\end{enumerate}
\solution
%\input{exemplar/10/13/3/29/main.tex}
\item A team of medical students doing their internship have to assist during surgeries
at a city hospital. The probabilities of surgeries rated as very complex, complex,
routine, simple or very simple are respectively, 0.15, 0.20, 0.31, 0.26, .08. Find
the probabilities that a particular surgery will be rated
\begin{enumerate}
	\item complex or very complex;
	\item neither very complex nor very simple;
	\item routine or complex
	\item routine or simple
\end{enumerate}
\solution
%\input{exemplar/11/16/3/8(1)/main.tex}
\item A card is selected from a pack of 52 cards.
\begin{enumerate}[label=(\alph*)]
    \item How many points are there in the sample space?
    \item Calculate the probability that the card is an ace of spades.
    \item Calculate the probability that the card is (i) an ace and (ii) black card.
\end{enumerate}
\solution
%\input{exemplar/11/16/3/4/main2.tex}
\item The probability that a non leap year selected at random will contain 53 sundays.
\\
\solution
%\input{exemplar/10/13/1/19/main.tex}
\item One of the four persons John, Rita, Aslam or Gurpreet will be promoted next
month. Consequently the sample space consists of four elementary outcomes
S = {John promoted, Rita promoted, Aslam promoted, Gurpreet promoted}
You are told that the chances of John’s promotion is same as that of Gurpreet,
Rita’s chances of promotion are twice as likely as Johns. Aslam’s chances are
four times that of John.
\begin{enumerate}
	\item Determine
	\begin{enumerate}
		\item P (John promoted)
		\item P (Rita promoted)
		\item P (Aslam promoted)
		\item P (Gurpreet promoted)
	\end{enumerate}
	\item If A = {John promoted or Gurpreet promoted}, find P (A).
\end{enumerate}
\solution
%\input{exemplar/11/16/3/10/main.tex}
\item A card is drawn from a deck of 52 cards. Find the probability of getting a king or a heart or a red card.\\
\solution
%\input{exemplar/11/16/3/15/main.tex}
\item The probability that a student will pass his examination is 0.73, the probability of
the student getting a compartment is 0.13, and the probability that the student will
either pass or get compartment is 0.96. State True or False.\\
\solution
%\input{exemplar/11/16/3/31/main.tex}
\item A card is selected from a pack of 52 cards\\
\begin{enumerate}[label=(\alph*)]
\item How many points are there in the sample space?
\item Calculate the probability that the cards is an ace of spades.
\item Calculate the probability that the card is (i) an ace (ii)black card.\\
\end{enumerate}
%\input{ncert/11/16/3/4_1/Prob_4.tex}
\item In a non-leap year, the probability of having 53 tuesdays or 53 wednesdays is\\
\solution
%\input{exemplar/11/16/3/18/main.tex}
\item There are 1000 sealed envelopes in a box, 10 of them contain a cash prize of
Rs 100 each, 100 of them contain a cash prize of Rs 50 each and 200 of them
contain a cash prize of Rs 10 each and rest do not contain any cash prize. If they
are well shuffled and an envelope is picked up out, what is the probability that it
contains no cash prize?\\
\solution
%\input{exemplar/10/13/3/34/main.tex}
\item 
A die is thrown and a card is selected at random from a deck of 52 playing cards. The probability of getting an even number on the die and a spade card.\\
\solution
%\input{exemplar/12/13/3/78/main.tex}
\item
If 4-digit numbers greater than 5,000 are randomly formed from the digits 0, 1, 3, 5, and 7, what is the probability of forming a number divisible by 5 when:
\begin{enumerate}
    \item The digits are repeated?
    \item The repetition of digits is not allowed?
\end{enumerate}
\solution
%\input{ncert/11/16/4/9/main.tex}
\item Consider the probability space $\brak{\Omega, \mathcal{G}, P}$ where $\Omega = [0,2]$ and $\mathcal{G} = \cbrak{\phi, \Omega, [0,1], (1,2]}$. Let $X$ and $Y$ be two functions on $\Omega$ defined as
\begin{align*}
    X(\omega) = 
    \begin{cases}
        1 & \text{if }\omega \in [0, 1]\\
        2 & \text{if }\omega \in (1, 2]
    \end{cases}
\end{align*}
and
\begin{align*}
    Y(\omega) = 
    \begin{cases}
        2 & \text{if }\omega \in [0, 1.5]\\
        3 & \text{if }\omega \in (1.5, 2].
    \end{cases}
\end{align*}
Then which one of the following statements is true?
\begin{enumerate}
    \item [(A)] $X$ is a random variable with respect to $\mathcal{G}$, but $Y$ is not a random variable with respect to $\mathcal{G}$.
    \item [(B)] $Y$ is a random variable with respect to $\mathcal{G}$, but $X$ is not a random variable with respect to $\mathcal{G}$.
    \item [(C)] Neither $X$ nor $Y$ is a random variable with respect to $\mathcal{G}$.
    \item [(D)] Both $X$ and $Y$ are random variables with respect to $\mathcal{G}$.
\end{enumerate} \hfill (GATE ST 2023)\\
\solution
%\input{gate/ST/2023/14/main.tex}
	\item  A die is loaded in such a way that each odd number is twice as likely to occur as
each even number. Find $P(G)$, where $G$ is the event that a number greater than
3 occurs on a single roll of the die.
\\
\solution
		%\input{exemplar/11/16/3/5/main.tex}
	\item All the jacks, queens and kings are removed from a deck of 52 playing cards. The remaining cards are well shuffled and then one card is drawn at random. Giving ace a value 1 similar value for other cards, find the probability that the card has a value 
		\begin{enumerate}
			\item 7
			\item greater than 7
			\item less than 7
		\end{enumerate}
		%\input{exemplar/10/13/3/30/main.tex}
  \item A Lot consists of 48 mobile phones of which 42 are good, 3 have only minor defects and 3 have major defects.Varnika will buy a phone if it is good but the trader will only buy a mobile if it has no major defects. One phone is selected at random from the lot. What is the probability that it is
\begin{enumerate}
	\item acceptable to Varnika?
            \item acceptable to the trader?
\end{enumerate}
\solution
	%\input{exemplar/10/13/3/40/main.tex}
 \item A student says that if you throw a die, it will show up 1 or not 1. Therefore, the probability of getting 1 and the probability of getting 'not 1' each is equal to $\frac{1}{2}$. Is this correct? Give reasons.\\
 \solution
        %\input{exemplar/10/13/2/9/main.tex}
   \item Four candidates A, B, C, D have ap-
plied for the assignment to coach a school cricket
team. If A is twice as likely to be selected as B, and
B and C are given about the same chance of being
selected, while C is twice as likely to be selected
as D, what are the probabilities that
\begin{enumerate}
\item C will be selected?
\item A will not be selected?
\end{enumerate}
	%\input{exemplar/11/16/3/9/main.tex}
 \item A bag contain 24 balls of which $x$ balls are red, $2x$ are white and $3x$ are blue. A ball is selected at random, What is the probability that it is
\begin{enumerate}[label=\alph*)]
\item not red ?
\item white ?
\end{enumerate}
%\input{exemplar/10/13/3/41/main.tex}
If the letters of the word ASSASSINATION are arranged at random. Find the Probability that
\begin{enumerate}[label=(\alph*)]
\item Four $S's$ come consecutively in the word
\item Two  $I's$ and two $N's$ come together
\item All $A's$ are not coming together
\item No two $A's$ are coming together
\end{enumerate}
%\input{exemplar/11/16/3/14/main.tex}
	\item One urn contains two black balls (labelled B1 and B2) and one white ball. A
	second urn contains one black ball and two white balls (labelled W1 and W2).
	Suppose the following experiment is performed. One of the two urns is chosen
	at random. Next a ball is randomly chosen from the urn. Then a second ball is
	chosen at random from the same urn without replacing the first ball.
	
	\begin{enumerate}
	\item What is the probability that two black balls are chosen?
	
	\item What is the probability that two balls of opposite colour are chosen?
	\end{enumerate}
	\solution
	%\input{exemplar/11/16/3/12/main1.tex}
\end{enumerate}

		%
\item 
Out of 100 students, two sections of 40 and 60 are formed. If you and your friend are among the 100 students, what is the probability that
\begin{enumerate}
\item you both enter the same section?
\item you both enter the different sections?
\end{enumerate}
\solution
		%\begin{enumerate}[label=\thesection.\arabic*,ref=\thesection.\theenumi]
	\item One card is drawn from a well-shuffled deck of 52 cards. Find the probability of getting
\begin{enumerate}
\item A king of red colour 
\item A face card 
\item A red face card
\item The jack of hearts
\item A spade
\item The queen of diamonds

\end{enumerate}
\solution
		%\input{ncert/10/15/1/14/main.tex}
	\item Five cards—the ten, jack, queen, king and ace of diamonds, are well-shuffled with their face downwards. One card is then picked up at random.
\begin{enumerate}
\item
What is the probability that the card is the queen? 
\item
If the queen is drawn and put aside, what is the probability that the second card picked up is (a) an ace? (b) a queen?\\
\end{enumerate}
\solution
		%\input{ncert/10/15/1/15/defs.tex}
	\item A bag contains $5$ red balls and some blue balls. If the probability of drawing a blue ball is double that if a red ball, determine the number of blue balls in the bag. 
		\\
\solution
		%\input{ncert/10/15/2/3/defs.tex}
	\item A card is selected from a pack of 52 cards.
 \begin{enumerate}[label=(\alph*)] 
                 \item How many points are there in the sample space?
                 \item Calculate the probability that the card is an ace of spades.
                 \item Calculate the probability that the card is (i) an ace and (ii) black card.
 \end{enumerate}
\solution
		%\input{ncert/11/16/3/4/main.tex}
\item Four cards are drawn from a well-shuffled deck of 52 cards. What is the probability of obtaining 3 diamonds and one spade.
\\
\solution
		%\input{ncert/11/16/4/2/defs.tex}
\item In a certain lottery 10,000 tickets are sold and ten equal prizes are awarded. What is the probability of not getting a prize if you buy (a) one ticket (b) two tickets (c) 10 tickets ?	
\\
\solution
		%\input{ncert/11/16/4/4/defs.tex}
		%
\item 
Out of 100 students, two sections of 40 and 60 are formed. If you and your friend are among the 100 students, what is the probability that
\begin{enumerate}
\item you both enter the same section?
\item you both enter the different sections?
\end{enumerate}
\solution
		%\input{ncert/11/16/4/5/defs.tex}
	\item 
The number lock of a suitcase has 4 wheels each labelled with ten digits i.e. from 0 to 9.The lock opens with a sequence of four digits with no repeats.What is the probability of a person getting the right sequence to open the suitcase.
\\
\solution
		%\input{ncert/11/16/4/10/defs.tex}
		%
\item 
Two cards are drawn at random and without replacement from a pack of 52 playing cards. Find the probability that both the cards are black.
\\
\solution
		%\input{ncert/12/13/2/2/defs.tex}
		\item A box of oranges is inspected by examining three randomly selected oranges drawn without replacement. If all the three oranges are good, the box is approved for sale, otherwise, it is rejected. Find the probability that a box containing 15 oranges out of which 12 are good and 3 are bad ones will be approved for sale.
		\label{ncert/12/13/2/3/defs.tex}
		\item Two balls are drawn at random with replacement from a box containing 10 black and 8 red balls. Find the probability that
		\label{ncert/12/13/2/12}
\begin{enumerate}
\item both balls are red.
\item first ball is black and second is red.
\item one of them is black and other is red.
\end{enumerate}

\item In a hostel, 60\% of the students read Hindi newspaper, 40\% read English newspaper and 20\% read both Hindi and English newspapers. A student is selected at random.
		\label{ncert/12/13/2/15}
\begin{enumerate}
\item Find the probability that she reads neither Hindi nor English newspapers.
\item If she reads Hindi newspaper, find the probability that she reads English newspaper.
\item If she reads English newspaper, find the probability that she reads Hindi newspaper.\\
\end{enumerate}
\item The probability of obtaining an even prime number on each die, when a pair of dice is rolled is 
\begin{enumerate}
    \item $0$ 
    
    \item $\frac{1}{3}$ 
    
    \item $\frac{1}{12}$ 
    
    \item $\frac{1}{36}$ 
\end{enumerate}
\solution
		%\input{ncert/12/13/2/17/defs.tex}
	\item A bag contains 4 red and 4 black balls, another bag contains 2 red and 6 black balls. One of the two bags is selected at random and a ball is drawn from the bag which is found to be red. Find the probability that the ball is drawn from the first bag.
\\
\solution
		%\input{ncert/12/13/3/2/main.tex}
  \item
  Cards with numbers 2 to 101 are placed in a box. A card is selected at random.Find the probability that the card has
\begin{enumerate}[label=(\roman*)]
	\item an even number 
	\item a square number
\end{enumerate}
\solution
%\input{exemplar/10/13/3/32/main.tex}
\item
The king, queen and jack of clubs are removed from a deck of 52 playing cards and then well shuffled. Now one card is drawn at random from the remaining cards.  Determine the probability that the card is
\begin{enumerate}[label=(\roman*)]
\item a club
\item 10 of hearts
\end{enumerate}
\solution
%\input{exemplar/10/13/3/29/main.tex}
\item A team of medical students doing their internship have to assist during surgeries
at a city hospital. The probabilities of surgeries rated as very complex, complex,
routine, simple or very simple are respectively, 0.15, 0.20, 0.31, 0.26, .08. Find
the probabilities that a particular surgery will be rated
\begin{enumerate}
	\item complex or very complex;
	\item neither very complex nor very simple;
	\item routine or complex
	\item routine or simple
\end{enumerate}
\solution
%\input{exemplar/11/16/3/8(1)/main.tex}
\item A card is selected from a pack of 52 cards.
\begin{enumerate}[label=(\alph*)]
    \item How many points are there in the sample space?
    \item Calculate the probability that the card is an ace of spades.
    \item Calculate the probability that the card is (i) an ace and (ii) black card.
\end{enumerate}
\solution
%\input{exemplar/11/16/3/4/main2.tex}
\item The probability that a non leap year selected at random will contain 53 sundays.
\\
\solution
%\input{exemplar/10/13/1/19/main.tex}
\item One of the four persons John, Rita, Aslam or Gurpreet will be promoted next
month. Consequently the sample space consists of four elementary outcomes
S = {John promoted, Rita promoted, Aslam promoted, Gurpreet promoted}
You are told that the chances of John’s promotion is same as that of Gurpreet,
Rita’s chances of promotion are twice as likely as Johns. Aslam’s chances are
four times that of John.
\begin{enumerate}
	\item Determine
	\begin{enumerate}
		\item P (John promoted)
		\item P (Rita promoted)
		\item P (Aslam promoted)
		\item P (Gurpreet promoted)
	\end{enumerate}
	\item If A = {John promoted or Gurpreet promoted}, find P (A).
\end{enumerate}
\solution
%\input{exemplar/11/16/3/10/main.tex}
\item A card is drawn from a deck of 52 cards. Find the probability of getting a king or a heart or a red card.\\
\solution
%\input{exemplar/11/16/3/15/main.tex}
\item The probability that a student will pass his examination is 0.73, the probability of
the student getting a compartment is 0.13, and the probability that the student will
either pass or get compartment is 0.96. State True or False.\\
\solution
%\input{exemplar/11/16/3/31/main.tex}
\item A card is selected from a pack of 52 cards\\
\begin{enumerate}[label=(\alph*)]
\item How many points are there in the sample space?
\item Calculate the probability that the cards is an ace of spades.
\item Calculate the probability that the card is (i) an ace (ii)black card.\\
\end{enumerate}
%\input{ncert/11/16/3/4_1/Prob_4.tex}
\item In a non-leap year, the probability of having 53 tuesdays or 53 wednesdays is\\
\solution
%\input{exemplar/11/16/3/18/main.tex}
\item There are 1000 sealed envelopes in a box, 10 of them contain a cash prize of
Rs 100 each, 100 of them contain a cash prize of Rs 50 each and 200 of them
contain a cash prize of Rs 10 each and rest do not contain any cash prize. If they
are well shuffled and an envelope is picked up out, what is the probability that it
contains no cash prize?\\
\solution
%\input{exemplar/10/13/3/34/main.tex}
\item 
A die is thrown and a card is selected at random from a deck of 52 playing cards. The probability of getting an even number on the die and a spade card.\\
\solution
%\input{exemplar/12/13/3/78/main.tex}
\item
If 4-digit numbers greater than 5,000 are randomly formed from the digits 0, 1, 3, 5, and 7, what is the probability of forming a number divisible by 5 when:
\begin{enumerate}
    \item The digits are repeated?
    \item The repetition of digits is not allowed?
\end{enumerate}
\solution
%\input{ncert/11/16/4/9/main.tex}
\item Consider the probability space $\brak{\Omega, \mathcal{G}, P}$ where $\Omega = [0,2]$ and $\mathcal{G} = \cbrak{\phi, \Omega, [0,1], (1,2]}$. Let $X$ and $Y$ be two functions on $\Omega$ defined as
\begin{align*}
    X(\omega) = 
    \begin{cases}
        1 & \text{if }\omega \in [0, 1]\\
        2 & \text{if }\omega \in (1, 2]
    \end{cases}
\end{align*}
and
\begin{align*}
    Y(\omega) = 
    \begin{cases}
        2 & \text{if }\omega \in [0, 1.5]\\
        3 & \text{if }\omega \in (1.5, 2].
    \end{cases}
\end{align*}
Then which one of the following statements is true?
\begin{enumerate}
    \item [(A)] $X$ is a random variable with respect to $\mathcal{G}$, but $Y$ is not a random variable with respect to $\mathcal{G}$.
    \item [(B)] $Y$ is a random variable with respect to $\mathcal{G}$, but $X$ is not a random variable with respect to $\mathcal{G}$.
    \item [(C)] Neither $X$ nor $Y$ is a random variable with respect to $\mathcal{G}$.
    \item [(D)] Both $X$ and $Y$ are random variables with respect to $\mathcal{G}$.
\end{enumerate} \hfill (GATE ST 2023)\\
\solution
%\input{gate/ST/2023/14/main.tex}
	\item  A die is loaded in such a way that each odd number is twice as likely to occur as
each even number. Find $P(G)$, where $G$ is the event that a number greater than
3 occurs on a single roll of the die.
\\
\solution
		%\input{exemplar/11/16/3/5/main.tex}
	\item All the jacks, queens and kings are removed from a deck of 52 playing cards. The remaining cards are well shuffled and then one card is drawn at random. Giving ace a value 1 similar value for other cards, find the probability that the card has a value 
		\begin{enumerate}
			\item 7
			\item greater than 7
			\item less than 7
		\end{enumerate}
		%\input{exemplar/10/13/3/30/main.tex}
  \item A Lot consists of 48 mobile phones of which 42 are good, 3 have only minor defects and 3 have major defects.Varnika will buy a phone if it is good but the trader will only buy a mobile if it has no major defects. One phone is selected at random from the lot. What is the probability that it is
\begin{enumerate}
	\item acceptable to Varnika?
            \item acceptable to the trader?
\end{enumerate}
\solution
	%\input{exemplar/10/13/3/40/main.tex}
 \item A student says that if you throw a die, it will show up 1 or not 1. Therefore, the probability of getting 1 and the probability of getting 'not 1' each is equal to $\frac{1}{2}$. Is this correct? Give reasons.\\
 \solution
        %\input{exemplar/10/13/2/9/main.tex}
   \item Four candidates A, B, C, D have ap-
plied for the assignment to coach a school cricket
team. If A is twice as likely to be selected as B, and
B and C are given about the same chance of being
selected, while C is twice as likely to be selected
as D, what are the probabilities that
\begin{enumerate}
\item C will be selected?
\item A will not be selected?
\end{enumerate}
	%\input{exemplar/11/16/3/9/main.tex}
 \item A bag contain 24 balls of which $x$ balls are red, $2x$ are white and $3x$ are blue. A ball is selected at random, What is the probability that it is
\begin{enumerate}[label=\alph*)]
\item not red ?
\item white ?
\end{enumerate}
%\input{exemplar/10/13/3/41/main.tex}
If the letters of the word ASSASSINATION are arranged at random. Find the Probability that
\begin{enumerate}[label=(\alph*)]
\item Four $S's$ come consecutively in the word
\item Two  $I's$ and two $N's$ come together
\item All $A's$ are not coming together
\item No two $A's$ are coming together
\end{enumerate}
%\input{exemplar/11/16/3/14/main.tex}
	\item One urn contains two black balls (labelled B1 and B2) and one white ball. A
	second urn contains one black ball and two white balls (labelled W1 and W2).
	Suppose the following experiment is performed. One of the two urns is chosen
	at random. Next a ball is randomly chosen from the urn. Then a second ball is
	chosen at random from the same urn without replacing the first ball.
	
	\begin{enumerate}
	\item What is the probability that two black balls are chosen?
	
	\item What is the probability that two balls of opposite colour are chosen?
	\end{enumerate}
	\solution
	%\input{exemplar/11/16/3/12/main1.tex}
\end{enumerate}

	\item 
The number lock of a suitcase has 4 wheels each labelled with ten digits i.e. from 0 to 9.The lock opens with a sequence of four digits with no repeats.What is the probability of a person getting the right sequence to open the suitcase.
\\
\solution
		%\begin{enumerate}[label=\thesection.\arabic*,ref=\thesection.\theenumi]
	\item One card is drawn from a well-shuffled deck of 52 cards. Find the probability of getting
\begin{enumerate}
\item A king of red colour 
\item A face card 
\item A red face card
\item The jack of hearts
\item A spade
\item The queen of diamonds

\end{enumerate}
\solution
		%\input{ncert/10/15/1/14/main.tex}
	\item Five cards—the ten, jack, queen, king and ace of diamonds, are well-shuffled with their face downwards. One card is then picked up at random.
\begin{enumerate}
\item
What is the probability that the card is the queen? 
\item
If the queen is drawn and put aside, what is the probability that the second card picked up is (a) an ace? (b) a queen?\\
\end{enumerate}
\solution
		%\input{ncert/10/15/1/15/defs.tex}
	\item A bag contains $5$ red balls and some blue balls. If the probability of drawing a blue ball is double that if a red ball, determine the number of blue balls in the bag. 
		\\
\solution
		%\input{ncert/10/15/2/3/defs.tex}
	\item A card is selected from a pack of 52 cards.
 \begin{enumerate}[label=(\alph*)] 
                 \item How many points are there in the sample space?
                 \item Calculate the probability that the card is an ace of spades.
                 \item Calculate the probability that the card is (i) an ace and (ii) black card.
 \end{enumerate}
\solution
		%\input{ncert/11/16/3/4/main.tex}
\item Four cards are drawn from a well-shuffled deck of 52 cards. What is the probability of obtaining 3 diamonds and one spade.
\\
\solution
		%\input{ncert/11/16/4/2/defs.tex}
\item In a certain lottery 10,000 tickets are sold and ten equal prizes are awarded. What is the probability of not getting a prize if you buy (a) one ticket (b) two tickets (c) 10 tickets ?	
\\
\solution
		%\input{ncert/11/16/4/4/defs.tex}
		%
\item 
Out of 100 students, two sections of 40 and 60 are formed. If you and your friend are among the 100 students, what is the probability that
\begin{enumerate}
\item you both enter the same section?
\item you both enter the different sections?
\end{enumerate}
\solution
		%\input{ncert/11/16/4/5/defs.tex}
	\item 
The number lock of a suitcase has 4 wheels each labelled with ten digits i.e. from 0 to 9.The lock opens with a sequence of four digits with no repeats.What is the probability of a person getting the right sequence to open the suitcase.
\\
\solution
		%\input{ncert/11/16/4/10/defs.tex}
		%
\item 
Two cards are drawn at random and without replacement from a pack of 52 playing cards. Find the probability that both the cards are black.
\\
\solution
		%\input{ncert/12/13/2/2/defs.tex}
		\item A box of oranges is inspected by examining three randomly selected oranges drawn without replacement. If all the three oranges are good, the box is approved for sale, otherwise, it is rejected. Find the probability that a box containing 15 oranges out of which 12 are good and 3 are bad ones will be approved for sale.
		\label{ncert/12/13/2/3/defs.tex}
		\item Two balls are drawn at random with replacement from a box containing 10 black and 8 red balls. Find the probability that
		\label{ncert/12/13/2/12}
\begin{enumerate}
\item both balls are red.
\item first ball is black and second is red.
\item one of them is black and other is red.
\end{enumerate}

\item In a hostel, 60\% of the students read Hindi newspaper, 40\% read English newspaper and 20\% read both Hindi and English newspapers. A student is selected at random.
		\label{ncert/12/13/2/15}
\begin{enumerate}
\item Find the probability that she reads neither Hindi nor English newspapers.
\item If she reads Hindi newspaper, find the probability that she reads English newspaper.
\item If she reads English newspaper, find the probability that she reads Hindi newspaper.\\
\end{enumerate}
\item The probability of obtaining an even prime number on each die, when a pair of dice is rolled is 
\begin{enumerate}
    \item $0$ 
    
    \item $\frac{1}{3}$ 
    
    \item $\frac{1}{12}$ 
    
    \item $\frac{1}{36}$ 
\end{enumerate}
\solution
		%\input{ncert/12/13/2/17/defs.tex}
	\item A bag contains 4 red and 4 black balls, another bag contains 2 red and 6 black balls. One of the two bags is selected at random and a ball is drawn from the bag which is found to be red. Find the probability that the ball is drawn from the first bag.
\\
\solution
		%\input{ncert/12/13/3/2/main.tex}
  \item
  Cards with numbers 2 to 101 are placed in a box. A card is selected at random.Find the probability that the card has
\begin{enumerate}[label=(\roman*)]
	\item an even number 
	\item a square number
\end{enumerate}
\solution
%\input{exemplar/10/13/3/32/main.tex}
\item
The king, queen and jack of clubs are removed from a deck of 52 playing cards and then well shuffled. Now one card is drawn at random from the remaining cards.  Determine the probability that the card is
\begin{enumerate}[label=(\roman*)]
\item a club
\item 10 of hearts
\end{enumerate}
\solution
%\input{exemplar/10/13/3/29/main.tex}
\item A team of medical students doing their internship have to assist during surgeries
at a city hospital. The probabilities of surgeries rated as very complex, complex,
routine, simple or very simple are respectively, 0.15, 0.20, 0.31, 0.26, .08. Find
the probabilities that a particular surgery will be rated
\begin{enumerate}
	\item complex or very complex;
	\item neither very complex nor very simple;
	\item routine or complex
	\item routine or simple
\end{enumerate}
\solution
%\input{exemplar/11/16/3/8(1)/main.tex}
\item A card is selected from a pack of 52 cards.
\begin{enumerate}[label=(\alph*)]
    \item How many points are there in the sample space?
    \item Calculate the probability that the card is an ace of spades.
    \item Calculate the probability that the card is (i) an ace and (ii) black card.
\end{enumerate}
\solution
%\input{exemplar/11/16/3/4/main2.tex}
\item The probability that a non leap year selected at random will contain 53 sundays.
\\
\solution
%\input{exemplar/10/13/1/19/main.tex}
\item One of the four persons John, Rita, Aslam or Gurpreet will be promoted next
month. Consequently the sample space consists of four elementary outcomes
S = {John promoted, Rita promoted, Aslam promoted, Gurpreet promoted}
You are told that the chances of John’s promotion is same as that of Gurpreet,
Rita’s chances of promotion are twice as likely as Johns. Aslam’s chances are
four times that of John.
\begin{enumerate}
	\item Determine
	\begin{enumerate}
		\item P (John promoted)
		\item P (Rita promoted)
		\item P (Aslam promoted)
		\item P (Gurpreet promoted)
	\end{enumerate}
	\item If A = {John promoted or Gurpreet promoted}, find P (A).
\end{enumerate}
\solution
%\input{exemplar/11/16/3/10/main.tex}
\item A card is drawn from a deck of 52 cards. Find the probability of getting a king or a heart or a red card.\\
\solution
%\input{exemplar/11/16/3/15/main.tex}
\item The probability that a student will pass his examination is 0.73, the probability of
the student getting a compartment is 0.13, and the probability that the student will
either pass or get compartment is 0.96. State True or False.\\
\solution
%\input{exemplar/11/16/3/31/main.tex}
\item A card is selected from a pack of 52 cards\\
\begin{enumerate}[label=(\alph*)]
\item How many points are there in the sample space?
\item Calculate the probability that the cards is an ace of spades.
\item Calculate the probability that the card is (i) an ace (ii)black card.\\
\end{enumerate}
%\input{ncert/11/16/3/4_1/Prob_4.tex}
\item In a non-leap year, the probability of having 53 tuesdays or 53 wednesdays is\\
\solution
%\input{exemplar/11/16/3/18/main.tex}
\item There are 1000 sealed envelopes in a box, 10 of them contain a cash prize of
Rs 100 each, 100 of them contain a cash prize of Rs 50 each and 200 of them
contain a cash prize of Rs 10 each and rest do not contain any cash prize. If they
are well shuffled and an envelope is picked up out, what is the probability that it
contains no cash prize?\\
\solution
%\input{exemplar/10/13/3/34/main.tex}
\item 
A die is thrown and a card is selected at random from a deck of 52 playing cards. The probability of getting an even number on the die and a spade card.\\
\solution
%\input{exemplar/12/13/3/78/main.tex}
\item
If 4-digit numbers greater than 5,000 are randomly formed from the digits 0, 1, 3, 5, and 7, what is the probability of forming a number divisible by 5 when:
\begin{enumerate}
    \item The digits are repeated?
    \item The repetition of digits is not allowed?
\end{enumerate}
\solution
%\input{ncert/11/16/4/9/main.tex}
\item Consider the probability space $\brak{\Omega, \mathcal{G}, P}$ where $\Omega = [0,2]$ and $\mathcal{G} = \cbrak{\phi, \Omega, [0,1], (1,2]}$. Let $X$ and $Y$ be two functions on $\Omega$ defined as
\begin{align*}
    X(\omega) = 
    \begin{cases}
        1 & \text{if }\omega \in [0, 1]\\
        2 & \text{if }\omega \in (1, 2]
    \end{cases}
\end{align*}
and
\begin{align*}
    Y(\omega) = 
    \begin{cases}
        2 & \text{if }\omega \in [0, 1.5]\\
        3 & \text{if }\omega \in (1.5, 2].
    \end{cases}
\end{align*}
Then which one of the following statements is true?
\begin{enumerate}
    \item [(A)] $X$ is a random variable with respect to $\mathcal{G}$, but $Y$ is not a random variable with respect to $\mathcal{G}$.
    \item [(B)] $Y$ is a random variable with respect to $\mathcal{G}$, but $X$ is not a random variable with respect to $\mathcal{G}$.
    \item [(C)] Neither $X$ nor $Y$ is a random variable with respect to $\mathcal{G}$.
    \item [(D)] Both $X$ and $Y$ are random variables with respect to $\mathcal{G}$.
\end{enumerate} \hfill (GATE ST 2023)\\
\solution
%\input{gate/ST/2023/14/main.tex}
	\item  A die is loaded in such a way that each odd number is twice as likely to occur as
each even number. Find $P(G)$, where $G$ is the event that a number greater than
3 occurs on a single roll of the die.
\\
\solution
		%\input{exemplar/11/16/3/5/main.tex}
	\item All the jacks, queens and kings are removed from a deck of 52 playing cards. The remaining cards are well shuffled and then one card is drawn at random. Giving ace a value 1 similar value for other cards, find the probability that the card has a value 
		\begin{enumerate}
			\item 7
			\item greater than 7
			\item less than 7
		\end{enumerate}
		%\input{exemplar/10/13/3/30/main.tex}
  \item A Lot consists of 48 mobile phones of which 42 are good, 3 have only minor defects and 3 have major defects.Varnika will buy a phone if it is good but the trader will only buy a mobile if it has no major defects. One phone is selected at random from the lot. What is the probability that it is
\begin{enumerate}
	\item acceptable to Varnika?
            \item acceptable to the trader?
\end{enumerate}
\solution
	%\input{exemplar/10/13/3/40/main.tex}
 \item A student says that if you throw a die, it will show up 1 or not 1. Therefore, the probability of getting 1 and the probability of getting 'not 1' each is equal to $\frac{1}{2}$. Is this correct? Give reasons.\\
 \solution
        %\input{exemplar/10/13/2/9/main.tex}
   \item Four candidates A, B, C, D have ap-
plied for the assignment to coach a school cricket
team. If A is twice as likely to be selected as B, and
B and C are given about the same chance of being
selected, while C is twice as likely to be selected
as D, what are the probabilities that
\begin{enumerate}
\item C will be selected?
\item A will not be selected?
\end{enumerate}
	%\input{exemplar/11/16/3/9/main.tex}
 \item A bag contain 24 balls of which $x$ balls are red, $2x$ are white and $3x$ are blue. A ball is selected at random, What is the probability that it is
\begin{enumerate}[label=\alph*)]
\item not red ?
\item white ?
\end{enumerate}
%\input{exemplar/10/13/3/41/main.tex}
If the letters of the word ASSASSINATION are arranged at random. Find the Probability that
\begin{enumerate}[label=(\alph*)]
\item Four $S's$ come consecutively in the word
\item Two  $I's$ and two $N's$ come together
\item All $A's$ are not coming together
\item No two $A's$ are coming together
\end{enumerate}
%\input{exemplar/11/16/3/14/main.tex}
	\item One urn contains two black balls (labelled B1 and B2) and one white ball. A
	second urn contains one black ball and two white balls (labelled W1 and W2).
	Suppose the following experiment is performed. One of the two urns is chosen
	at random. Next a ball is randomly chosen from the urn. Then a second ball is
	chosen at random from the same urn without replacing the first ball.
	
	\begin{enumerate}
	\item What is the probability that two black balls are chosen?
	
	\item What is the probability that two balls of opposite colour are chosen?
	\end{enumerate}
	\solution
	%\input{exemplar/11/16/3/12/main1.tex}
\end{enumerate}

		%
\item 
Two cards are drawn at random and without replacement from a pack of 52 playing cards. Find the probability that both the cards are black.
\\
\solution
		%\begin{enumerate}[label=\thesection.\arabic*,ref=\thesection.\theenumi]
	\item One card is drawn from a well-shuffled deck of 52 cards. Find the probability of getting
\begin{enumerate}
\item A king of red colour 
\item A face card 
\item A red face card
\item The jack of hearts
\item A spade
\item The queen of diamonds

\end{enumerate}
\solution
		%\input{ncert/10/15/1/14/main.tex}
	\item Five cards—the ten, jack, queen, king and ace of diamonds, are well-shuffled with their face downwards. One card is then picked up at random.
\begin{enumerate}
\item
What is the probability that the card is the queen? 
\item
If the queen is drawn and put aside, what is the probability that the second card picked up is (a) an ace? (b) a queen?\\
\end{enumerate}
\solution
		%\input{ncert/10/15/1/15/defs.tex}
	\item A bag contains $5$ red balls and some blue balls. If the probability of drawing a blue ball is double that if a red ball, determine the number of blue balls in the bag. 
		\\
\solution
		%\input{ncert/10/15/2/3/defs.tex}
	\item A card is selected from a pack of 52 cards.
 \begin{enumerate}[label=(\alph*)] 
                 \item How many points are there in the sample space?
                 \item Calculate the probability that the card is an ace of spades.
                 \item Calculate the probability that the card is (i) an ace and (ii) black card.
 \end{enumerate}
\solution
		%\input{ncert/11/16/3/4/main.tex}
\item Four cards are drawn from a well-shuffled deck of 52 cards. What is the probability of obtaining 3 diamonds and one spade.
\\
\solution
		%\input{ncert/11/16/4/2/defs.tex}
\item In a certain lottery 10,000 tickets are sold and ten equal prizes are awarded. What is the probability of not getting a prize if you buy (a) one ticket (b) two tickets (c) 10 tickets ?	
\\
\solution
		%\input{ncert/11/16/4/4/defs.tex}
		%
\item 
Out of 100 students, two sections of 40 and 60 are formed. If you and your friend are among the 100 students, what is the probability that
\begin{enumerate}
\item you both enter the same section?
\item you both enter the different sections?
\end{enumerate}
\solution
		%\input{ncert/11/16/4/5/defs.tex}
	\item 
The number lock of a suitcase has 4 wheels each labelled with ten digits i.e. from 0 to 9.The lock opens with a sequence of four digits with no repeats.What is the probability of a person getting the right sequence to open the suitcase.
\\
\solution
		%\input{ncert/11/16/4/10/defs.tex}
		%
\item 
Two cards are drawn at random and without replacement from a pack of 52 playing cards. Find the probability that both the cards are black.
\\
\solution
		%\input{ncert/12/13/2/2/defs.tex}
		\item A box of oranges is inspected by examining three randomly selected oranges drawn without replacement. If all the three oranges are good, the box is approved for sale, otherwise, it is rejected. Find the probability that a box containing 15 oranges out of which 12 are good and 3 are bad ones will be approved for sale.
		\label{ncert/12/13/2/3/defs.tex}
		\item Two balls are drawn at random with replacement from a box containing 10 black and 8 red balls. Find the probability that
		\label{ncert/12/13/2/12}
\begin{enumerate}
\item both balls are red.
\item first ball is black and second is red.
\item one of them is black and other is red.
\end{enumerate}

\item In a hostel, 60\% of the students read Hindi newspaper, 40\% read English newspaper and 20\% read both Hindi and English newspapers. A student is selected at random.
		\label{ncert/12/13/2/15}
\begin{enumerate}
\item Find the probability that she reads neither Hindi nor English newspapers.
\item If she reads Hindi newspaper, find the probability that she reads English newspaper.
\item If she reads English newspaper, find the probability that she reads Hindi newspaper.\\
\end{enumerate}
\item The probability of obtaining an even prime number on each die, when a pair of dice is rolled is 
\begin{enumerate}
    \item $0$ 
    
    \item $\frac{1}{3}$ 
    
    \item $\frac{1}{12}$ 
    
    \item $\frac{1}{36}$ 
\end{enumerate}
\solution
		%\input{ncert/12/13/2/17/defs.tex}
	\item A bag contains 4 red and 4 black balls, another bag contains 2 red and 6 black balls. One of the two bags is selected at random and a ball is drawn from the bag which is found to be red. Find the probability that the ball is drawn from the first bag.
\\
\solution
		%\input{ncert/12/13/3/2/main.tex}
  \item
  Cards with numbers 2 to 101 are placed in a box. A card is selected at random.Find the probability that the card has
\begin{enumerate}[label=(\roman*)]
	\item an even number 
	\item a square number
\end{enumerate}
\solution
%\input{exemplar/10/13/3/32/main.tex}
\item
The king, queen and jack of clubs are removed from a deck of 52 playing cards and then well shuffled. Now one card is drawn at random from the remaining cards.  Determine the probability that the card is
\begin{enumerate}[label=(\roman*)]
\item a club
\item 10 of hearts
\end{enumerate}
\solution
%\input{exemplar/10/13/3/29/main.tex}
\item A team of medical students doing their internship have to assist during surgeries
at a city hospital. The probabilities of surgeries rated as very complex, complex,
routine, simple or very simple are respectively, 0.15, 0.20, 0.31, 0.26, .08. Find
the probabilities that a particular surgery will be rated
\begin{enumerate}
	\item complex or very complex;
	\item neither very complex nor very simple;
	\item routine or complex
	\item routine or simple
\end{enumerate}
\solution
%\input{exemplar/11/16/3/8(1)/main.tex}
\item A card is selected from a pack of 52 cards.
\begin{enumerate}[label=(\alph*)]
    \item How many points are there in the sample space?
    \item Calculate the probability that the card is an ace of spades.
    \item Calculate the probability that the card is (i) an ace and (ii) black card.
\end{enumerate}
\solution
%\input{exemplar/11/16/3/4/main2.tex}
\item The probability that a non leap year selected at random will contain 53 sundays.
\\
\solution
%\input{exemplar/10/13/1/19/main.tex}
\item One of the four persons John, Rita, Aslam or Gurpreet will be promoted next
month. Consequently the sample space consists of four elementary outcomes
S = {John promoted, Rita promoted, Aslam promoted, Gurpreet promoted}
You are told that the chances of John’s promotion is same as that of Gurpreet,
Rita’s chances of promotion are twice as likely as Johns. Aslam’s chances are
four times that of John.
\begin{enumerate}
	\item Determine
	\begin{enumerate}
		\item P (John promoted)
		\item P (Rita promoted)
		\item P (Aslam promoted)
		\item P (Gurpreet promoted)
	\end{enumerate}
	\item If A = {John promoted or Gurpreet promoted}, find P (A).
\end{enumerate}
\solution
%\input{exemplar/11/16/3/10/main.tex}
\item A card is drawn from a deck of 52 cards. Find the probability of getting a king or a heart or a red card.\\
\solution
%\input{exemplar/11/16/3/15/main.tex}
\item The probability that a student will pass his examination is 0.73, the probability of
the student getting a compartment is 0.13, and the probability that the student will
either pass or get compartment is 0.96. State True or False.\\
\solution
%\input{exemplar/11/16/3/31/main.tex}
\item A card is selected from a pack of 52 cards\\
\begin{enumerate}[label=(\alph*)]
\item How many points are there in the sample space?
\item Calculate the probability that the cards is an ace of spades.
\item Calculate the probability that the card is (i) an ace (ii)black card.\\
\end{enumerate}
%\input{ncert/11/16/3/4_1/Prob_4.tex}
\item In a non-leap year, the probability of having 53 tuesdays or 53 wednesdays is\\
\solution
%\input{exemplar/11/16/3/18/main.tex}
\item There are 1000 sealed envelopes in a box, 10 of them contain a cash prize of
Rs 100 each, 100 of them contain a cash prize of Rs 50 each and 200 of them
contain a cash prize of Rs 10 each and rest do not contain any cash prize. If they
are well shuffled and an envelope is picked up out, what is the probability that it
contains no cash prize?\\
\solution
%\input{exemplar/10/13/3/34/main.tex}
\item 
A die is thrown and a card is selected at random from a deck of 52 playing cards. The probability of getting an even number on the die and a spade card.\\
\solution
%\input{exemplar/12/13/3/78/main.tex}
\item
If 4-digit numbers greater than 5,000 are randomly formed from the digits 0, 1, 3, 5, and 7, what is the probability of forming a number divisible by 5 when:
\begin{enumerate}
    \item The digits are repeated?
    \item The repetition of digits is not allowed?
\end{enumerate}
\solution
%\input{ncert/11/16/4/9/main.tex}
\item Consider the probability space $\brak{\Omega, \mathcal{G}, P}$ where $\Omega = [0,2]$ and $\mathcal{G} = \cbrak{\phi, \Omega, [0,1], (1,2]}$. Let $X$ and $Y$ be two functions on $\Omega$ defined as
\begin{align*}
    X(\omega) = 
    \begin{cases}
        1 & \text{if }\omega \in [0, 1]\\
        2 & \text{if }\omega \in (1, 2]
    \end{cases}
\end{align*}
and
\begin{align*}
    Y(\omega) = 
    \begin{cases}
        2 & \text{if }\omega \in [0, 1.5]\\
        3 & \text{if }\omega \in (1.5, 2].
    \end{cases}
\end{align*}
Then which one of the following statements is true?
\begin{enumerate}
    \item [(A)] $X$ is a random variable with respect to $\mathcal{G}$, but $Y$ is not a random variable with respect to $\mathcal{G}$.
    \item [(B)] $Y$ is a random variable with respect to $\mathcal{G}$, but $X$ is not a random variable with respect to $\mathcal{G}$.
    \item [(C)] Neither $X$ nor $Y$ is a random variable with respect to $\mathcal{G}$.
    \item [(D)] Both $X$ and $Y$ are random variables with respect to $\mathcal{G}$.
\end{enumerate} \hfill (GATE ST 2023)\\
\solution
%\input{gate/ST/2023/14/main.tex}
	\item  A die is loaded in such a way that each odd number is twice as likely to occur as
each even number. Find $P(G)$, where $G$ is the event that a number greater than
3 occurs on a single roll of the die.
\\
\solution
		%\input{exemplar/11/16/3/5/main.tex}
	\item All the jacks, queens and kings are removed from a deck of 52 playing cards. The remaining cards are well shuffled and then one card is drawn at random. Giving ace a value 1 similar value for other cards, find the probability that the card has a value 
		\begin{enumerate}
			\item 7
			\item greater than 7
			\item less than 7
		\end{enumerate}
		%\input{exemplar/10/13/3/30/main.tex}
  \item A Lot consists of 48 mobile phones of which 42 are good, 3 have only minor defects and 3 have major defects.Varnika will buy a phone if it is good but the trader will only buy a mobile if it has no major defects. One phone is selected at random from the lot. What is the probability that it is
\begin{enumerate}
	\item acceptable to Varnika?
            \item acceptable to the trader?
\end{enumerate}
\solution
	%\input{exemplar/10/13/3/40/main.tex}
 \item A student says that if you throw a die, it will show up 1 or not 1. Therefore, the probability of getting 1 and the probability of getting 'not 1' each is equal to $\frac{1}{2}$. Is this correct? Give reasons.\\
 \solution
        %\input{exemplar/10/13/2/9/main.tex}
   \item Four candidates A, B, C, D have ap-
plied for the assignment to coach a school cricket
team. If A is twice as likely to be selected as B, and
B and C are given about the same chance of being
selected, while C is twice as likely to be selected
as D, what are the probabilities that
\begin{enumerate}
\item C will be selected?
\item A will not be selected?
\end{enumerate}
	%\input{exemplar/11/16/3/9/main.tex}
 \item A bag contain 24 balls of which $x$ balls are red, $2x$ are white and $3x$ are blue. A ball is selected at random, What is the probability that it is
\begin{enumerate}[label=\alph*)]
\item not red ?
\item white ?
\end{enumerate}
%\input{exemplar/10/13/3/41/main.tex}
If the letters of the word ASSASSINATION are arranged at random. Find the Probability that
\begin{enumerate}[label=(\alph*)]
\item Four $S's$ come consecutively in the word
\item Two  $I's$ and two $N's$ come together
\item All $A's$ are not coming together
\item No two $A's$ are coming together
\end{enumerate}
%\input{exemplar/11/16/3/14/main.tex}
	\item One urn contains two black balls (labelled B1 and B2) and one white ball. A
	second urn contains one black ball and two white balls (labelled W1 and W2).
	Suppose the following experiment is performed. One of the two urns is chosen
	at random. Next a ball is randomly chosen from the urn. Then a second ball is
	chosen at random from the same urn without replacing the first ball.
	
	\begin{enumerate}
	\item What is the probability that two black balls are chosen?
	
	\item What is the probability that two balls of opposite colour are chosen?
	\end{enumerate}
	\solution
	%\input{exemplar/11/16/3/12/main1.tex}
\end{enumerate}

		\item A box of oranges is inspected by examining three randomly selected oranges drawn without replacement. If all the three oranges are good, the box is approved for sale, otherwise, it is rejected. Find the probability that a box containing 15 oranges out of which 12 are good and 3 are bad ones will be approved for sale.
		\label{ncert/12/13/2/3/defs.tex}
		\item Two balls are drawn at random with replacement from a box containing 10 black and 8 red balls. Find the probability that
		\label{ncert/12/13/2/12}
\begin{enumerate}
\item both balls are red.
\item first ball is black and second is red.
\item one of them is black and other is red.
\end{enumerate}

\item In a hostel, 60\% of the students read Hindi newspaper, 40\% read English newspaper and 20\% read both Hindi and English newspapers. A student is selected at random.
		\label{ncert/12/13/2/15}
\begin{enumerate}
\item Find the probability that she reads neither Hindi nor English newspapers.
\item If she reads Hindi newspaper, find the probability that she reads English newspaper.
\item If she reads English newspaper, find the probability that she reads Hindi newspaper.\\
\end{enumerate}
\item The probability of obtaining an even prime number on each die, when a pair of dice is rolled is 
\begin{enumerate}
    \item $0$ 
    
    \item $\frac{1}{3}$ 
    
    \item $\frac{1}{12}$ 
    
    \item $\frac{1}{36}$ 
\end{enumerate}
\solution
		%\begin{enumerate}[label=\thesection.\arabic*,ref=\thesection.\theenumi]
	\item One card is drawn from a well-shuffled deck of 52 cards. Find the probability of getting
\begin{enumerate}
\item A king of red colour 
\item A face card 
\item A red face card
\item The jack of hearts
\item A spade
\item The queen of diamonds

\end{enumerate}
\solution
		%\input{ncert/10/15/1/14/main.tex}
	\item Five cards—the ten, jack, queen, king and ace of diamonds, are well-shuffled with their face downwards. One card is then picked up at random.
\begin{enumerate}
\item
What is the probability that the card is the queen? 
\item
If the queen is drawn and put aside, what is the probability that the second card picked up is (a) an ace? (b) a queen?\\
\end{enumerate}
\solution
		%\input{ncert/10/15/1/15/defs.tex}
	\item A bag contains $5$ red balls and some blue balls. If the probability of drawing a blue ball is double that if a red ball, determine the number of blue balls in the bag. 
		\\
\solution
		%\input{ncert/10/15/2/3/defs.tex}
	\item A card is selected from a pack of 52 cards.
 \begin{enumerate}[label=(\alph*)] 
                 \item How many points are there in the sample space?
                 \item Calculate the probability that the card is an ace of spades.
                 \item Calculate the probability that the card is (i) an ace and (ii) black card.
 \end{enumerate}
\solution
		%\input{ncert/11/16/3/4/main.tex}
\item Four cards are drawn from a well-shuffled deck of 52 cards. What is the probability of obtaining 3 diamonds and one spade.
\\
\solution
		%\input{ncert/11/16/4/2/defs.tex}
\item In a certain lottery 10,000 tickets are sold and ten equal prizes are awarded. What is the probability of not getting a prize if you buy (a) one ticket (b) two tickets (c) 10 tickets ?	
\\
\solution
		%\input{ncert/11/16/4/4/defs.tex}
		%
\item 
Out of 100 students, two sections of 40 and 60 are formed. If you and your friend are among the 100 students, what is the probability that
\begin{enumerate}
\item you both enter the same section?
\item you both enter the different sections?
\end{enumerate}
\solution
		%\input{ncert/11/16/4/5/defs.tex}
	\item 
The number lock of a suitcase has 4 wheels each labelled with ten digits i.e. from 0 to 9.The lock opens with a sequence of four digits with no repeats.What is the probability of a person getting the right sequence to open the suitcase.
\\
\solution
		%\input{ncert/11/16/4/10/defs.tex}
		%
\item 
Two cards are drawn at random and without replacement from a pack of 52 playing cards. Find the probability that both the cards are black.
\\
\solution
		%\input{ncert/12/13/2/2/defs.tex}
		\item A box of oranges is inspected by examining three randomly selected oranges drawn without replacement. If all the three oranges are good, the box is approved for sale, otherwise, it is rejected. Find the probability that a box containing 15 oranges out of which 12 are good and 3 are bad ones will be approved for sale.
		\label{ncert/12/13/2/3/defs.tex}
		\item Two balls are drawn at random with replacement from a box containing 10 black and 8 red balls. Find the probability that
		\label{ncert/12/13/2/12}
\begin{enumerate}
\item both balls are red.
\item first ball is black and second is red.
\item one of them is black and other is red.
\end{enumerate}

\item In a hostel, 60\% of the students read Hindi newspaper, 40\% read English newspaper and 20\% read both Hindi and English newspapers. A student is selected at random.
		\label{ncert/12/13/2/15}
\begin{enumerate}
\item Find the probability that she reads neither Hindi nor English newspapers.
\item If she reads Hindi newspaper, find the probability that she reads English newspaper.
\item If she reads English newspaper, find the probability that she reads Hindi newspaper.\\
\end{enumerate}
\item The probability of obtaining an even prime number on each die, when a pair of dice is rolled is 
\begin{enumerate}
    \item $0$ 
    
    \item $\frac{1}{3}$ 
    
    \item $\frac{1}{12}$ 
    
    \item $\frac{1}{36}$ 
\end{enumerate}
\solution
		%\input{ncert/12/13/2/17/defs.tex}
	\item A bag contains 4 red and 4 black balls, another bag contains 2 red and 6 black balls. One of the two bags is selected at random and a ball is drawn from the bag which is found to be red. Find the probability that the ball is drawn from the first bag.
\\
\solution
		%\input{ncert/12/13/3/2/main.tex}
  \item
  Cards with numbers 2 to 101 are placed in a box. A card is selected at random.Find the probability that the card has
\begin{enumerate}[label=(\roman*)]
	\item an even number 
	\item a square number
\end{enumerate}
\solution
%\input{exemplar/10/13/3/32/main.tex}
\item
The king, queen and jack of clubs are removed from a deck of 52 playing cards and then well shuffled. Now one card is drawn at random from the remaining cards.  Determine the probability that the card is
\begin{enumerate}[label=(\roman*)]
\item a club
\item 10 of hearts
\end{enumerate}
\solution
%\input{exemplar/10/13/3/29/main.tex}
\item A team of medical students doing their internship have to assist during surgeries
at a city hospital. The probabilities of surgeries rated as very complex, complex,
routine, simple or very simple are respectively, 0.15, 0.20, 0.31, 0.26, .08. Find
the probabilities that a particular surgery will be rated
\begin{enumerate}
	\item complex or very complex;
	\item neither very complex nor very simple;
	\item routine or complex
	\item routine or simple
\end{enumerate}
\solution
%\input{exemplar/11/16/3/8(1)/main.tex}
\item A card is selected from a pack of 52 cards.
\begin{enumerate}[label=(\alph*)]
    \item How many points are there in the sample space?
    \item Calculate the probability that the card is an ace of spades.
    \item Calculate the probability that the card is (i) an ace and (ii) black card.
\end{enumerate}
\solution
%\input{exemplar/11/16/3/4/main2.tex}
\item The probability that a non leap year selected at random will contain 53 sundays.
\\
\solution
%\input{exemplar/10/13/1/19/main.tex}
\item One of the four persons John, Rita, Aslam or Gurpreet will be promoted next
month. Consequently the sample space consists of four elementary outcomes
S = {John promoted, Rita promoted, Aslam promoted, Gurpreet promoted}
You are told that the chances of John’s promotion is same as that of Gurpreet,
Rita’s chances of promotion are twice as likely as Johns. Aslam’s chances are
four times that of John.
\begin{enumerate}
	\item Determine
	\begin{enumerate}
		\item P (John promoted)
		\item P (Rita promoted)
		\item P (Aslam promoted)
		\item P (Gurpreet promoted)
	\end{enumerate}
	\item If A = {John promoted or Gurpreet promoted}, find P (A).
\end{enumerate}
\solution
%\input{exemplar/11/16/3/10/main.tex}
\item A card is drawn from a deck of 52 cards. Find the probability of getting a king or a heart or a red card.\\
\solution
%\input{exemplar/11/16/3/15/main.tex}
\item The probability that a student will pass his examination is 0.73, the probability of
the student getting a compartment is 0.13, and the probability that the student will
either pass or get compartment is 0.96. State True or False.\\
\solution
%\input{exemplar/11/16/3/31/main.tex}
\item A card is selected from a pack of 52 cards\\
\begin{enumerate}[label=(\alph*)]
\item How many points are there in the sample space?
\item Calculate the probability that the cards is an ace of spades.
\item Calculate the probability that the card is (i) an ace (ii)black card.\\
\end{enumerate}
%\input{ncert/11/16/3/4_1/Prob_4.tex}
\item In a non-leap year, the probability of having 53 tuesdays or 53 wednesdays is\\
\solution
%\input{exemplar/11/16/3/18/main.tex}
\item There are 1000 sealed envelopes in a box, 10 of them contain a cash prize of
Rs 100 each, 100 of them contain a cash prize of Rs 50 each and 200 of them
contain a cash prize of Rs 10 each and rest do not contain any cash prize. If they
are well shuffled and an envelope is picked up out, what is the probability that it
contains no cash prize?\\
\solution
%\input{exemplar/10/13/3/34/main.tex}
\item 
A die is thrown and a card is selected at random from a deck of 52 playing cards. The probability of getting an even number on the die and a spade card.\\
\solution
%\input{exemplar/12/13/3/78/main.tex}
\item
If 4-digit numbers greater than 5,000 are randomly formed from the digits 0, 1, 3, 5, and 7, what is the probability of forming a number divisible by 5 when:
\begin{enumerate}
    \item The digits are repeated?
    \item The repetition of digits is not allowed?
\end{enumerate}
\solution
%\input{ncert/11/16/4/9/main.tex}
\item Consider the probability space $\brak{\Omega, \mathcal{G}, P}$ where $\Omega = [0,2]$ and $\mathcal{G} = \cbrak{\phi, \Omega, [0,1], (1,2]}$. Let $X$ and $Y$ be two functions on $\Omega$ defined as
\begin{align*}
    X(\omega) = 
    \begin{cases}
        1 & \text{if }\omega \in [0, 1]\\
        2 & \text{if }\omega \in (1, 2]
    \end{cases}
\end{align*}
and
\begin{align*}
    Y(\omega) = 
    \begin{cases}
        2 & \text{if }\omega \in [0, 1.5]\\
        3 & \text{if }\omega \in (1.5, 2].
    \end{cases}
\end{align*}
Then which one of the following statements is true?
\begin{enumerate}
    \item [(A)] $X$ is a random variable with respect to $\mathcal{G}$, but $Y$ is not a random variable with respect to $\mathcal{G}$.
    \item [(B)] $Y$ is a random variable with respect to $\mathcal{G}$, but $X$ is not a random variable with respect to $\mathcal{G}$.
    \item [(C)] Neither $X$ nor $Y$ is a random variable with respect to $\mathcal{G}$.
    \item [(D)] Both $X$ and $Y$ are random variables with respect to $\mathcal{G}$.
\end{enumerate} \hfill (GATE ST 2023)\\
\solution
%\input{gate/ST/2023/14/main.tex}
	\item  A die is loaded in such a way that each odd number is twice as likely to occur as
each even number. Find $P(G)$, where $G$ is the event that a number greater than
3 occurs on a single roll of the die.
\\
\solution
		%\input{exemplar/11/16/3/5/main.tex}
	\item All the jacks, queens and kings are removed from a deck of 52 playing cards. The remaining cards are well shuffled and then one card is drawn at random. Giving ace a value 1 similar value for other cards, find the probability that the card has a value 
		\begin{enumerate}
			\item 7
			\item greater than 7
			\item less than 7
		\end{enumerate}
		%\input{exemplar/10/13/3/30/main.tex}
  \item A Lot consists of 48 mobile phones of which 42 are good, 3 have only minor defects and 3 have major defects.Varnika will buy a phone if it is good but the trader will only buy a mobile if it has no major defects. One phone is selected at random from the lot. What is the probability that it is
\begin{enumerate}
	\item acceptable to Varnika?
            \item acceptable to the trader?
\end{enumerate}
\solution
	%\input{exemplar/10/13/3/40/main.tex}
 \item A student says that if you throw a die, it will show up 1 or not 1. Therefore, the probability of getting 1 and the probability of getting 'not 1' each is equal to $\frac{1}{2}$. Is this correct? Give reasons.\\
 \solution
        %\input{exemplar/10/13/2/9/main.tex}
   \item Four candidates A, B, C, D have ap-
plied for the assignment to coach a school cricket
team. If A is twice as likely to be selected as B, and
B and C are given about the same chance of being
selected, while C is twice as likely to be selected
as D, what are the probabilities that
\begin{enumerate}
\item C will be selected?
\item A will not be selected?
\end{enumerate}
	%\input{exemplar/11/16/3/9/main.tex}
 \item A bag contain 24 balls of which $x$ balls are red, $2x$ are white and $3x$ are blue. A ball is selected at random, What is the probability that it is
\begin{enumerate}[label=\alph*)]
\item not red ?
\item white ?
\end{enumerate}
%\input{exemplar/10/13/3/41/main.tex}
If the letters of the word ASSASSINATION are arranged at random. Find the Probability that
\begin{enumerate}[label=(\alph*)]
\item Four $S's$ come consecutively in the word
\item Two  $I's$ and two $N's$ come together
\item All $A's$ are not coming together
\item No two $A's$ are coming together
\end{enumerate}
%\input{exemplar/11/16/3/14/main.tex}
	\item One urn contains two black balls (labelled B1 and B2) and one white ball. A
	second urn contains one black ball and two white balls (labelled W1 and W2).
	Suppose the following experiment is performed. One of the two urns is chosen
	at random. Next a ball is randomly chosen from the urn. Then a second ball is
	chosen at random from the same urn without replacing the first ball.
	
	\begin{enumerate}
	\item What is the probability that two black balls are chosen?
	
	\item What is the probability that two balls of opposite colour are chosen?
	\end{enumerate}
	\solution
	%\input{exemplar/11/16/3/12/main1.tex}
\end{enumerate}

	\item A bag contains 4 red and 4 black balls, another bag contains 2 red and 6 black balls. One of the two bags is selected at random and a ball is drawn from the bag which is found to be red. Find the probability that the ball is drawn from the first bag.
\\
\solution
		%\begin{table}[H]
	\centering
\begin{tabular}{|c|c|c|}
\hline
Random variable &Value &Definition\\ \hline
\multirow{3}{*}{X} &0 &Slips of Rs 1\\
&1 &Slips of Rs 5\\
&2 &Slips of Rs 13\\ \hline
\multirow{2}{*}{Y} &0 &Box A\\
&1 &Box B\\\hline
\end{tabular}
\caption{}
\label{tab:Distribution}
\end{table}
See \tabref{tab:Distribution}.
\begin{align}
p_{Y}\brak{k}= \begin{cases} 
      \frac{1}{3} & {k=0} \\
      \frac{2}{3 }& {k=1} 
   \end{cases}
   \\
p_{Y|X}\brak{0|0} = \frac{19}{25}\, 
p_{Y|X}\brak{0|1} = \frac{6}{25}\,
p_{Y|X}\brak{1|0} = \frac{45}{50}\,
p_{Y|X}\brak{1|2} = \frac{5}{50}
\end{align}
The desired probability is the probability that a slip drawn at random is marked other than Rs 1,
\begin{align}
&=1-p_X\brak{0}\\
&= p_X(1) + p_X(2)
\end{align}
Using Bayes theorem,
\begin{align}
&= p_Y\brak{0} \times \pr{Y=0 | X=1} + p_Y\brak{1} \times \pr{Y=1|X=2}\\
&=\frac{1}{3} \times \frac{6}{25} + \frac{2}{3} \times \frac{5}{50}\\
&=\frac{11}{75}
\end{align}

\newpage

%\tableofcontents

\bigskip

\renewcommand{\thefigure}{\theenumi}
\renewcommand{\thetable}{\theenumi}
%\renewcommand{\theequation}{\theenumi}

%\begin{abstract}
%%\boldmath
%In this letter, an algorithm for evaluating the exact analytical bit error rate  (BER)  for the piecewise linear (PL) combiner for  multiple relays is presented. Previous results were available only for upto three relays. The algorithm is unique in the sense that  the actual mathematical expressions, that are prohibitively large, need not be explicitly obtained. The diversity gain due to multiple relays is shown through plots of the analytical BER, well supported by simulations. 
%
%\end{abstract}
% IEEEtran.cls defaults to using nonbold math in the Abstract.
% This preserves the distinction between vectors and scalars. However,
% if the journal you are submitting to favors bold math in the abstract,
% then you can use LaTeX's standard command \boldmath at the very start
% of the abstract to achieve this. Many IEEE journals frown on math
% in the abstract anyway.

% Note that keywords are not normally used for peerreview papers.
%\begin{IEEEkeywords}
%Cooperative diversity, decode and forward, piecewise linear
%\end{IEEEkeywords}



% For peer review papers, you can put extra information on the cover
% page as needed:
% \ifCLASSOPTIONpeerreview
% \begin{center} \bfseries EDICS Category: 3-BBND \end{center}
% \fi
%
% For peerreview papers, this IEEEtran command inserts a page break and
% creates the second title. It will be ignored for other modes.
%\IEEEpeerreviewmaketitle




  \item
  Cards with numbers 2 to 101 are placed in a box. A card is selected at random.Find the probability that the card has
\begin{enumerate}[label=(\roman*)]
	\item an even number 
	\item a square number
\end{enumerate}
\solution
%\begin{table}[H]
	\centering
\begin{tabular}{|c|c|c|}
\hline
Random variable &Value &Definition\\ \hline
\multirow{3}{*}{X} &0 &Slips of Rs 1\\
&1 &Slips of Rs 5\\
&2 &Slips of Rs 13\\ \hline
\multirow{2}{*}{Y} &0 &Box A\\
&1 &Box B\\\hline
\end{tabular}
\caption{}
\label{tab:Distribution}
\end{table}
See \tabref{tab:Distribution}.
\begin{align}
p_{Y}\brak{k}= \begin{cases} 
      \frac{1}{3} & {k=0} \\
      \frac{2}{3 }& {k=1} 
   \end{cases}
   \\
p_{Y|X}\brak{0|0} = \frac{19}{25}\, 
p_{Y|X}\brak{0|1} = \frac{6}{25}\,
p_{Y|X}\brak{1|0} = \frac{45}{50}\,
p_{Y|X}\brak{1|2} = \frac{5}{50}
\end{align}
The desired probability is the probability that a slip drawn at random is marked other than Rs 1,
\begin{align}
&=1-p_X\brak{0}\\
&= p_X(1) + p_X(2)
\end{align}
Using Bayes theorem,
\begin{align}
&= p_Y\brak{0} \times \pr{Y=0 | X=1} + p_Y\brak{1} \times \pr{Y=1|X=2}\\
&=\frac{1}{3} \times \frac{6}{25} + \frac{2}{3} \times \frac{5}{50}\\
&=\frac{11}{75}
\end{align}

\newpage

%\tableofcontents

\bigskip

\renewcommand{\thefigure}{\theenumi}
\renewcommand{\thetable}{\theenumi}
%\renewcommand{\theequation}{\theenumi}

%\begin{abstract}
%%\boldmath
%In this letter, an algorithm for evaluating the exact analytical bit error rate  (BER)  for the piecewise linear (PL) combiner for  multiple relays is presented. Previous results were available only for upto three relays. The algorithm is unique in the sense that  the actual mathematical expressions, that are prohibitively large, need not be explicitly obtained. The diversity gain due to multiple relays is shown through plots of the analytical BER, well supported by simulations. 
%
%\end{abstract}
% IEEEtran.cls defaults to using nonbold math in the Abstract.
% This preserves the distinction between vectors and scalars. However,
% if the journal you are submitting to favors bold math in the abstract,
% then you can use LaTeX's standard command \boldmath at the very start
% of the abstract to achieve this. Many IEEE journals frown on math
% in the abstract anyway.

% Note that keywords are not normally used for peerreview papers.
%\begin{IEEEkeywords}
%Cooperative diversity, decode and forward, piecewise linear
%\end{IEEEkeywords}



% For peer review papers, you can put extra information on the cover
% page as needed:
% \ifCLASSOPTIONpeerreview
% \begin{center} \bfseries EDICS Category: 3-BBND \end{center}
% \fi
%
% For peerreview papers, this IEEEtran command inserts a page break and
% creates the second title. It will be ignored for other modes.
%\IEEEpeerreviewmaketitle




\item
The king, queen and jack of clubs are removed from a deck of 52 playing cards and then well shuffled. Now one card is drawn at random from the remaining cards.  Determine the probability that the card is
\begin{enumerate}[label=(\roman*)]
\item a club
\item 10 of hearts
\end{enumerate}
\solution
%\begin{table}[H]
	\centering
\begin{tabular}{|c|c|c|}
\hline
Random variable &Value &Definition\\ \hline
\multirow{3}{*}{X} &0 &Slips of Rs 1\\
&1 &Slips of Rs 5\\
&2 &Slips of Rs 13\\ \hline
\multirow{2}{*}{Y} &0 &Box A\\
&1 &Box B\\\hline
\end{tabular}
\caption{}
\label{tab:Distribution}
\end{table}
See \tabref{tab:Distribution}.
\begin{align}
p_{Y}\brak{k}= \begin{cases} 
      \frac{1}{3} & {k=0} \\
      \frac{2}{3 }& {k=1} 
   \end{cases}
   \\
p_{Y|X}\brak{0|0} = \frac{19}{25}\, 
p_{Y|X}\brak{0|1} = \frac{6}{25}\,
p_{Y|X}\brak{1|0} = \frac{45}{50}\,
p_{Y|X}\brak{1|2} = \frac{5}{50}
\end{align}
The desired probability is the probability that a slip drawn at random is marked other than Rs 1,
\begin{align}
&=1-p_X\brak{0}\\
&= p_X(1) + p_X(2)
\end{align}
Using Bayes theorem,
\begin{align}
&= p_Y\brak{0} \times \pr{Y=0 | X=1} + p_Y\brak{1} \times \pr{Y=1|X=2}\\
&=\frac{1}{3} \times \frac{6}{25} + \frac{2}{3} \times \frac{5}{50}\\
&=\frac{11}{75}
\end{align}

\newpage

%\tableofcontents

\bigskip

\renewcommand{\thefigure}{\theenumi}
\renewcommand{\thetable}{\theenumi}
%\renewcommand{\theequation}{\theenumi}

%\begin{abstract}
%%\boldmath
%In this letter, an algorithm for evaluating the exact analytical bit error rate  (BER)  for the piecewise linear (PL) combiner for  multiple relays is presented. Previous results were available only for upto three relays. The algorithm is unique in the sense that  the actual mathematical expressions, that are prohibitively large, need not be explicitly obtained. The diversity gain due to multiple relays is shown through plots of the analytical BER, well supported by simulations. 
%
%\end{abstract}
% IEEEtran.cls defaults to using nonbold math in the Abstract.
% This preserves the distinction between vectors and scalars. However,
% if the journal you are submitting to favors bold math in the abstract,
% then you can use LaTeX's standard command \boldmath at the very start
% of the abstract to achieve this. Many IEEE journals frown on math
% in the abstract anyway.

% Note that keywords are not normally used for peerreview papers.
%\begin{IEEEkeywords}
%Cooperative diversity, decode and forward, piecewise linear
%\end{IEEEkeywords}



% For peer review papers, you can put extra information on the cover
% page as needed:
% \ifCLASSOPTIONpeerreview
% \begin{center} \bfseries EDICS Category: 3-BBND \end{center}
% \fi
%
% For peerreview papers, this IEEEtran command inserts a page break and
% creates the second title. It will be ignored for other modes.
%\IEEEpeerreviewmaketitle




\item A team of medical students doing their internship have to assist during surgeries
at a city hospital. The probabilities of surgeries rated as very complex, complex,
routine, simple or very simple are respectively, 0.15, 0.20, 0.31, 0.26, .08. Find
the probabilities that a particular surgery will be rated
\begin{enumerate}
	\item complex or very complex;
	\item neither very complex nor very simple;
	\item routine or complex
	\item routine or simple
\end{enumerate}
\solution
%\begin{table}[H]
	\centering
\begin{tabular}{|c|c|c|}
\hline
Random variable &Value &Definition\\ \hline
\multirow{3}{*}{X} &0 &Slips of Rs 1\\
&1 &Slips of Rs 5\\
&2 &Slips of Rs 13\\ \hline
\multirow{2}{*}{Y} &0 &Box A\\
&1 &Box B\\\hline
\end{tabular}
\caption{}
\label{tab:Distribution}
\end{table}
See \tabref{tab:Distribution}.
\begin{align}
p_{Y}\brak{k}= \begin{cases} 
      \frac{1}{3} & {k=0} \\
      \frac{2}{3 }& {k=1} 
   \end{cases}
   \\
p_{Y|X}\brak{0|0} = \frac{19}{25}\, 
p_{Y|X}\brak{0|1} = \frac{6}{25}\,
p_{Y|X}\brak{1|0} = \frac{45}{50}\,
p_{Y|X}\brak{1|2} = \frac{5}{50}
\end{align}
The desired probability is the probability that a slip drawn at random is marked other than Rs 1,
\begin{align}
&=1-p_X\brak{0}\\
&= p_X(1) + p_X(2)
\end{align}
Using Bayes theorem,
\begin{align}
&= p_Y\brak{0} \times \pr{Y=0 | X=1} + p_Y\brak{1} \times \pr{Y=1|X=2}\\
&=\frac{1}{3} \times \frac{6}{25} + \frac{2}{3} \times \frac{5}{50}\\
&=\frac{11}{75}
\end{align}

\newpage

%\tableofcontents

\bigskip

\renewcommand{\thefigure}{\theenumi}
\renewcommand{\thetable}{\theenumi}
%\renewcommand{\theequation}{\theenumi}

%\begin{abstract}
%%\boldmath
%In this letter, an algorithm for evaluating the exact analytical bit error rate  (BER)  for the piecewise linear (PL) combiner for  multiple relays is presented. Previous results were available only for upto three relays. The algorithm is unique in the sense that  the actual mathematical expressions, that are prohibitively large, need not be explicitly obtained. The diversity gain due to multiple relays is shown through plots of the analytical BER, well supported by simulations. 
%
%\end{abstract}
% IEEEtran.cls defaults to using nonbold math in the Abstract.
% This preserves the distinction between vectors and scalars. However,
% if the journal you are submitting to favors bold math in the abstract,
% then you can use LaTeX's standard command \boldmath at the very start
% of the abstract to achieve this. Many IEEE journals frown on math
% in the abstract anyway.

% Note that keywords are not normally used for peerreview papers.
%\begin{IEEEkeywords}
%Cooperative diversity, decode and forward, piecewise linear
%\end{IEEEkeywords}



% For peer review papers, you can put extra information on the cover
% page as needed:
% \ifCLASSOPTIONpeerreview
% \begin{center} \bfseries EDICS Category: 3-BBND \end{center}
% \fi
%
% For peerreview papers, this IEEEtran command inserts a page break and
% creates the second title. It will be ignored for other modes.
%\IEEEpeerreviewmaketitle




\item A card is selected from a pack of 52 cards.
\begin{enumerate}[label=(\alph*)]
    \item How many points are there in the sample space?
    \item Calculate the probability that the card is an ace of spades.
    \item Calculate the probability that the card is (i) an ace and (ii) black card.
\end{enumerate}
\solution
%Let $X$ be an bernoulli rv defined as in \tabref{tab:exemplar/11/16/3/26}.  Then, 
\begin{equation}
    p =
        \frac{4}{11} 
\end{equation}
\begin{table}[H]
	\centering
	\input{exemplar/11/16/3/26/tables/Table2.tex}
	\caption{}
        \label{tab:exemplar/11/16/3/26}
\end{table}

\item The probability that a non leap year selected at random will contain 53 sundays.
\\
\solution
%\begin{table}[H]
	\centering
\begin{tabular}{|c|c|c|}
\hline
Random variable &Value &Definition\\ \hline
\multirow{3}{*}{X} &0 &Slips of Rs 1\\
&1 &Slips of Rs 5\\
&2 &Slips of Rs 13\\ \hline
\multirow{2}{*}{Y} &0 &Box A\\
&1 &Box B\\\hline
\end{tabular}
\caption{}
\label{tab:Distribution}
\end{table}
See \tabref{tab:Distribution}.
\begin{align}
p_{Y}\brak{k}= \begin{cases} 
      \frac{1}{3} & {k=0} \\
      \frac{2}{3 }& {k=1} 
   \end{cases}
   \\
p_{Y|X}\brak{0|0} = \frac{19}{25}\, 
p_{Y|X}\brak{0|1} = \frac{6}{25}\,
p_{Y|X}\brak{1|0} = \frac{45}{50}\,
p_{Y|X}\brak{1|2} = \frac{5}{50}
\end{align}
The desired probability is the probability that a slip drawn at random is marked other than Rs 1,
\begin{align}
&=1-p_X\brak{0}\\
&= p_X(1) + p_X(2)
\end{align}
Using Bayes theorem,
\begin{align}
&= p_Y\brak{0} \times \pr{Y=0 | X=1} + p_Y\brak{1} \times \pr{Y=1|X=2}\\
&=\frac{1}{3} \times \frac{6}{25} + \frac{2}{3} \times \frac{5}{50}\\
&=\frac{11}{75}
\end{align}

\newpage

%\tableofcontents

\bigskip

\renewcommand{\thefigure}{\theenumi}
\renewcommand{\thetable}{\theenumi}
%\renewcommand{\theequation}{\theenumi}

%\begin{abstract}
%%\boldmath
%In this letter, an algorithm for evaluating the exact analytical bit error rate  (BER)  for the piecewise linear (PL) combiner for  multiple relays is presented. Previous results were available only for upto three relays. The algorithm is unique in the sense that  the actual mathematical expressions, that are prohibitively large, need not be explicitly obtained. The diversity gain due to multiple relays is shown through plots of the analytical BER, well supported by simulations. 
%
%\end{abstract}
% IEEEtran.cls defaults to using nonbold math in the Abstract.
% This preserves the distinction between vectors and scalars. However,
% if the journal you are submitting to favors bold math in the abstract,
% then you can use LaTeX's standard command \boldmath at the very start
% of the abstract to achieve this. Many IEEE journals frown on math
% in the abstract anyway.

% Note that keywords are not normally used for peerreview papers.
%\begin{IEEEkeywords}
%Cooperative diversity, decode and forward, piecewise linear
%\end{IEEEkeywords}



% For peer review papers, you can put extra information on the cover
% page as needed:
% \ifCLASSOPTIONpeerreview
% \begin{center} \bfseries EDICS Category: 3-BBND \end{center}
% \fi
%
% For peerreview papers, this IEEEtran command inserts a page break and
% creates the second title. It will be ignored for other modes.
%\IEEEpeerreviewmaketitle




\item One of the four persons John, Rita, Aslam or Gurpreet will be promoted next
month. Consequently the sample space consists of four elementary outcomes
S = {John promoted, Rita promoted, Aslam promoted, Gurpreet promoted}
You are told that the chances of John’s promotion is same as that of Gurpreet,
Rita’s chances of promotion are twice as likely as Johns. Aslam’s chances are
four times that of John.
\begin{enumerate}
	\item Determine
	\begin{enumerate}
		\item P (John promoted)
		\item P (Rita promoted)
		\item P (Aslam promoted)
		\item P (Gurpreet promoted)
	\end{enumerate}
	\item If A = {John promoted or Gurpreet promoted}, find P (A).
\end{enumerate}
\solution
%\begin{table}[H]
	\centering
\begin{tabular}{|c|c|c|}
\hline
Random variable &Value &Definition\\ \hline
\multirow{3}{*}{X} &0 &Slips of Rs 1\\
&1 &Slips of Rs 5\\
&2 &Slips of Rs 13\\ \hline
\multirow{2}{*}{Y} &0 &Box A\\
&1 &Box B\\\hline
\end{tabular}
\caption{}
\label{tab:Distribution}
\end{table}
See \tabref{tab:Distribution}.
\begin{align}
p_{Y}\brak{k}= \begin{cases} 
      \frac{1}{3} & {k=0} \\
      \frac{2}{3 }& {k=1} 
   \end{cases}
   \\
p_{Y|X}\brak{0|0} = \frac{19}{25}\, 
p_{Y|X}\brak{0|1} = \frac{6}{25}\,
p_{Y|X}\brak{1|0} = \frac{45}{50}\,
p_{Y|X}\brak{1|2} = \frac{5}{50}
\end{align}
The desired probability is the probability that a slip drawn at random is marked other than Rs 1,
\begin{align}
&=1-p_X\brak{0}\\
&= p_X(1) + p_X(2)
\end{align}
Using Bayes theorem,
\begin{align}
&= p_Y\brak{0} \times \pr{Y=0 | X=1} + p_Y\brak{1} \times \pr{Y=1|X=2}\\
&=\frac{1}{3} \times \frac{6}{25} + \frac{2}{3} \times \frac{5}{50}\\
&=\frac{11}{75}
\end{align}

\newpage

%\tableofcontents

\bigskip

\renewcommand{\thefigure}{\theenumi}
\renewcommand{\thetable}{\theenumi}
%\renewcommand{\theequation}{\theenumi}

%\begin{abstract}
%%\boldmath
%In this letter, an algorithm for evaluating the exact analytical bit error rate  (BER)  for the piecewise linear (PL) combiner for  multiple relays is presented. Previous results were available only for upto three relays. The algorithm is unique in the sense that  the actual mathematical expressions, that are prohibitively large, need not be explicitly obtained. The diversity gain due to multiple relays is shown through plots of the analytical BER, well supported by simulations. 
%
%\end{abstract}
% IEEEtran.cls defaults to using nonbold math in the Abstract.
% This preserves the distinction between vectors and scalars. However,
% if the journal you are submitting to favors bold math in the abstract,
% then you can use LaTeX's standard command \boldmath at the very start
% of the abstract to achieve this. Many IEEE journals frown on math
% in the abstract anyway.

% Note that keywords are not normally used for peerreview papers.
%\begin{IEEEkeywords}
%Cooperative diversity, decode and forward, piecewise linear
%\end{IEEEkeywords}



% For peer review papers, you can put extra information on the cover
% page as needed:
% \ifCLASSOPTIONpeerreview
% \begin{center} \bfseries EDICS Category: 3-BBND \end{center}
% \fi
%
% For peerreview papers, this IEEEtran command inserts a page break and
% creates the second title. It will be ignored for other modes.
%\IEEEpeerreviewmaketitle




\item A card is drawn from a deck of 52 cards. Find the probability of getting a king or a heart or a red card.\\
\solution
%\begin{table}[H]
	\centering
\begin{tabular}{|c|c|c|}
\hline
Random variable &Value &Definition\\ \hline
\multirow{3}{*}{X} &0 &Slips of Rs 1\\
&1 &Slips of Rs 5\\
&2 &Slips of Rs 13\\ \hline
\multirow{2}{*}{Y} &0 &Box A\\
&1 &Box B\\\hline
\end{tabular}
\caption{}
\label{tab:Distribution}
\end{table}
See \tabref{tab:Distribution}.
\begin{align}
p_{Y}\brak{k}= \begin{cases} 
      \frac{1}{3} & {k=0} \\
      \frac{2}{3 }& {k=1} 
   \end{cases}
   \\
p_{Y|X}\brak{0|0} = \frac{19}{25}\, 
p_{Y|X}\brak{0|1} = \frac{6}{25}\,
p_{Y|X}\brak{1|0} = \frac{45}{50}\,
p_{Y|X}\brak{1|2} = \frac{5}{50}
\end{align}
The desired probability is the probability that a slip drawn at random is marked other than Rs 1,
\begin{align}
&=1-p_X\brak{0}\\
&= p_X(1) + p_X(2)
\end{align}
Using Bayes theorem,
\begin{align}
&= p_Y\brak{0} \times \pr{Y=0 | X=1} + p_Y\brak{1} \times \pr{Y=1|X=2}\\
&=\frac{1}{3} \times \frac{6}{25} + \frac{2}{3} \times \frac{5}{50}\\
&=\frac{11}{75}
\end{align}

\newpage

%\tableofcontents

\bigskip

\renewcommand{\thefigure}{\theenumi}
\renewcommand{\thetable}{\theenumi}
%\renewcommand{\theequation}{\theenumi}

%\begin{abstract}
%%\boldmath
%In this letter, an algorithm for evaluating the exact analytical bit error rate  (BER)  for the piecewise linear (PL) combiner for  multiple relays is presented. Previous results were available only for upto three relays. The algorithm is unique in the sense that  the actual mathematical expressions, that are prohibitively large, need not be explicitly obtained. The diversity gain due to multiple relays is shown through plots of the analytical BER, well supported by simulations. 
%
%\end{abstract}
% IEEEtran.cls defaults to using nonbold math in the Abstract.
% This preserves the distinction between vectors and scalars. However,
% if the journal you are submitting to favors bold math in the abstract,
% then you can use LaTeX's standard command \boldmath at the very start
% of the abstract to achieve this. Many IEEE journals frown on math
% in the abstract anyway.

% Note that keywords are not normally used for peerreview papers.
%\begin{IEEEkeywords}
%Cooperative diversity, decode and forward, piecewise linear
%\end{IEEEkeywords}



% For peer review papers, you can put extra information on the cover
% page as needed:
% \ifCLASSOPTIONpeerreview
% \begin{center} \bfseries EDICS Category: 3-BBND \end{center}
% \fi
%
% For peerreview papers, this IEEEtran command inserts a page break and
% creates the second title. It will be ignored for other modes.
%\IEEEpeerreviewmaketitle




\item The probability that a student will pass his examination is 0.73, the probability of
the student getting a compartment is 0.13, and the probability that the student will
either pass or get compartment is 0.96. State True or False.\\
\solution
%\begin{table}[H]
	\centering
\begin{tabular}{|c|c|c|}
\hline
Random variable &Value &Definition\\ \hline
\multirow{3}{*}{X} &0 &Slips of Rs 1\\
&1 &Slips of Rs 5\\
&2 &Slips of Rs 13\\ \hline
\multirow{2}{*}{Y} &0 &Box A\\
&1 &Box B\\\hline
\end{tabular}
\caption{}
\label{tab:Distribution}
\end{table}
See \tabref{tab:Distribution}.
\begin{align}
p_{Y}\brak{k}= \begin{cases} 
      \frac{1}{3} & {k=0} \\
      \frac{2}{3 }& {k=1} 
   \end{cases}
   \\
p_{Y|X}\brak{0|0} = \frac{19}{25}\, 
p_{Y|X}\brak{0|1} = \frac{6}{25}\,
p_{Y|X}\brak{1|0} = \frac{45}{50}\,
p_{Y|X}\brak{1|2} = \frac{5}{50}
\end{align}
The desired probability is the probability that a slip drawn at random is marked other than Rs 1,
\begin{align}
&=1-p_X\brak{0}\\
&= p_X(1) + p_X(2)
\end{align}
Using Bayes theorem,
\begin{align}
&= p_Y\brak{0} \times \pr{Y=0 | X=1} + p_Y\brak{1} \times \pr{Y=1|X=2}\\
&=\frac{1}{3} \times \frac{6}{25} + \frac{2}{3} \times \frac{5}{50}\\
&=\frac{11}{75}
\end{align}

\newpage

%\tableofcontents

\bigskip

\renewcommand{\thefigure}{\theenumi}
\renewcommand{\thetable}{\theenumi}
%\renewcommand{\theequation}{\theenumi}

%\begin{abstract}
%%\boldmath
%In this letter, an algorithm for evaluating the exact analytical bit error rate  (BER)  for the piecewise linear (PL) combiner for  multiple relays is presented. Previous results were available only for upto three relays. The algorithm is unique in the sense that  the actual mathematical expressions, that are prohibitively large, need not be explicitly obtained. The diversity gain due to multiple relays is shown through plots of the analytical BER, well supported by simulations. 
%
%\end{abstract}
% IEEEtran.cls defaults to using nonbold math in the Abstract.
% This preserves the distinction between vectors and scalars. However,
% if the journal you are submitting to favors bold math in the abstract,
% then you can use LaTeX's standard command \boldmath at the very start
% of the abstract to achieve this. Many IEEE journals frown on math
% in the abstract anyway.

% Note that keywords are not normally used for peerreview papers.
%\begin{IEEEkeywords}
%Cooperative diversity, decode and forward, piecewise linear
%\end{IEEEkeywords}



% For peer review papers, you can put extra information on the cover
% page as needed:
% \ifCLASSOPTIONpeerreview
% \begin{center} \bfseries EDICS Category: 3-BBND \end{center}
% \fi
%
% For peerreview papers, this IEEEtran command inserts a page break and
% creates the second title. It will be ignored for other modes.
%\IEEEpeerreviewmaketitle




\item A card is selected from a pack of 52 cards\\
\begin{enumerate}[label=(\alph*)]
\item How many points are there in the sample space?
\item Calculate the probability that the cards is an ace of spades.
\item Calculate the probability that the card is (i) an ace (ii)black card.\\
\end{enumerate}
%\input{ncert/11/16/3/4_1/Prob_4.tex}
\item In a non-leap year, the probability of having 53 tuesdays or 53 wednesdays is\\
\solution
%A non-leap year has a total of 365 days, and a week has 7 days.\\
So it can be expressed as 
\begin{align}
365\text{days} &=52\times 7+1 \text{day}
\end{align}
$\implies$ 52 tuesdays or wednesdays\\
Random variable X denotes the days of a week
\begin{align}
p_X\brak{k}&=\frac{1}{7}; \quad \brak{1<k<7}
\end{align}
So the probability of extra day being tuesday or wednesday is
\begin{align}
p_X\brak{3}+p_X\brak{4}&=\frac{1}{7}+\frac{1}{7}=\frac{2}{7}
\end{align}



\item There are 1000 sealed envelopes in a box, 10 of them contain a cash prize of
Rs 100 each, 100 of them contain a cash prize of Rs 50 each and 200 of them
contain a cash prize of Rs 10 each and rest do not contain any cash prize. If they
are well shuffled and an envelope is picked up out, what is the probability that it
contains no cash prize?\\
\solution
%\begin{table}[H]
	\centering
\begin{tabular}{|c|c|c|}
\hline
Random variable &Value &Definition\\ \hline
\multirow{3}{*}{X} &0 &Slips of Rs 1\\
&1 &Slips of Rs 5\\
&2 &Slips of Rs 13\\ \hline
\multirow{2}{*}{Y} &0 &Box A\\
&1 &Box B\\\hline
\end{tabular}
\caption{}
\label{tab:Distribution}
\end{table}
See \tabref{tab:Distribution}.
\begin{align}
p_{Y}\brak{k}= \begin{cases} 
      \frac{1}{3} & {k=0} \\
      \frac{2}{3 }& {k=1} 
   \end{cases}
   \\
p_{Y|X}\brak{0|0} = \frac{19}{25}\, 
p_{Y|X}\brak{0|1} = \frac{6}{25}\,
p_{Y|X}\brak{1|0} = \frac{45}{50}\,
p_{Y|X}\brak{1|2} = \frac{5}{50}
\end{align}
The desired probability is the probability that a slip drawn at random is marked other than Rs 1,
\begin{align}
&=1-p_X\brak{0}\\
&= p_X(1) + p_X(2)
\end{align}
Using Bayes theorem,
\begin{align}
&= p_Y\brak{0} \times \pr{Y=0 | X=1} + p_Y\brak{1} \times \pr{Y=1|X=2}\\
&=\frac{1}{3} \times \frac{6}{25} + \frac{2}{3} \times \frac{5}{50}\\
&=\frac{11}{75}
\end{align}

\newpage

%\tableofcontents

\bigskip

\renewcommand{\thefigure}{\theenumi}
\renewcommand{\thetable}{\theenumi}
%\renewcommand{\theequation}{\theenumi}

%\begin{abstract}
%%\boldmath
%In this letter, an algorithm for evaluating the exact analytical bit error rate  (BER)  for the piecewise linear (PL) combiner for  multiple relays is presented. Previous results were available only for upto three relays. The algorithm is unique in the sense that  the actual mathematical expressions, that are prohibitively large, need not be explicitly obtained. The diversity gain due to multiple relays is shown through plots of the analytical BER, well supported by simulations. 
%
%\end{abstract}
% IEEEtran.cls defaults to using nonbold math in the Abstract.
% This preserves the distinction between vectors and scalars. However,
% if the journal you are submitting to favors bold math in the abstract,
% then you can use LaTeX's standard command \boldmath at the very start
% of the abstract to achieve this. Many IEEE journals frown on math
% in the abstract anyway.

% Note that keywords are not normally used for peerreview papers.
%\begin{IEEEkeywords}
%Cooperative diversity, decode and forward, piecewise linear
%\end{IEEEkeywords}



% For peer review papers, you can put extra information on the cover
% page as needed:
% \ifCLASSOPTIONpeerreview
% \begin{center} \bfseries EDICS Category: 3-BBND \end{center}
% \fi
%
% For peerreview papers, this IEEEtran command inserts a page break and
% creates the second title. It will be ignored for other modes.
%\IEEEpeerreviewmaketitle




\item 
A die is thrown and a card is selected at random from a deck of 52 playing cards. The probability of getting an even number on the die and a spade card.\\
\solution
%\begin{table}[H]
	\centering
\begin{tabular}{|c|c|c|}
\hline
Random variable &Value &Definition\\ \hline
\multirow{3}{*}{X} &0 &Slips of Rs 1\\
&1 &Slips of Rs 5\\
&2 &Slips of Rs 13\\ \hline
\multirow{2}{*}{Y} &0 &Box A\\
&1 &Box B\\\hline
\end{tabular}
\caption{}
\label{tab:Distribution}
\end{table}
See \tabref{tab:Distribution}.
\begin{align}
p_{Y}\brak{k}= \begin{cases} 
      \frac{1}{3} & {k=0} \\
      \frac{2}{3 }& {k=1} 
   \end{cases}
   \\
p_{Y|X}\brak{0|0} = \frac{19}{25}\, 
p_{Y|X}\brak{0|1} = \frac{6}{25}\,
p_{Y|X}\brak{1|0} = \frac{45}{50}\,
p_{Y|X}\brak{1|2} = \frac{5}{50}
\end{align}
The desired probability is the probability that a slip drawn at random is marked other than Rs 1,
\begin{align}
&=1-p_X\brak{0}\\
&= p_X(1) + p_X(2)
\end{align}
Using Bayes theorem,
\begin{align}
&= p_Y\brak{0} \times \pr{Y=0 | X=1} + p_Y\brak{1} \times \pr{Y=1|X=2}\\
&=\frac{1}{3} \times \frac{6}{25} + \frac{2}{3} \times \frac{5}{50}\\
&=\frac{11}{75}
\end{align}

\newpage

%\tableofcontents

\bigskip

\renewcommand{\thefigure}{\theenumi}
\renewcommand{\thetable}{\theenumi}
%\renewcommand{\theequation}{\theenumi}

%\begin{abstract}
%%\boldmath
%In this letter, an algorithm for evaluating the exact analytical bit error rate  (BER)  for the piecewise linear (PL) combiner for  multiple relays is presented. Previous results were available only for upto three relays. The algorithm is unique in the sense that  the actual mathematical expressions, that are prohibitively large, need not be explicitly obtained. The diversity gain due to multiple relays is shown through plots of the analytical BER, well supported by simulations. 
%
%\end{abstract}
% IEEEtran.cls defaults to using nonbold math in the Abstract.
% This preserves the distinction between vectors and scalars. However,
% if the journal you are submitting to favors bold math in the abstract,
% then you can use LaTeX's standard command \boldmath at the very start
% of the abstract to achieve this. Many IEEE journals frown on math
% in the abstract anyway.

% Note that keywords are not normally used for peerreview papers.
%\begin{IEEEkeywords}
%Cooperative diversity, decode and forward, piecewise linear
%\end{IEEEkeywords}



% For peer review papers, you can put extra information on the cover
% page as needed:
% \ifCLASSOPTIONpeerreview
% \begin{center} \bfseries EDICS Category: 3-BBND \end{center}
% \fi
%
% For peerreview papers, this IEEEtran command inserts a page break and
% creates the second title. It will be ignored for other modes.
%\IEEEpeerreviewmaketitle




\item
If 4-digit numbers greater than 5,000 are randomly formed from the digits 0, 1, 3, 5, and 7, what is the probability of forming a number divisible by 5 when:
\begin{enumerate}
    \item The digits are repeated?
    \item The repetition of digits is not allowed?
\end{enumerate}
\solution
%\begin{table}[H]
	\centering
\begin{tabular}{|c|c|c|}
\hline
Random variable &Value &Definition\\ \hline
\multirow{3}{*}{X} &0 &Slips of Rs 1\\
&1 &Slips of Rs 5\\
&2 &Slips of Rs 13\\ \hline
\multirow{2}{*}{Y} &0 &Box A\\
&1 &Box B\\\hline
\end{tabular}
\caption{}
\label{tab:Distribution}
\end{table}
See \tabref{tab:Distribution}.
\begin{align}
p_{Y}\brak{k}= \begin{cases} 
      \frac{1}{3} & {k=0} \\
      \frac{2}{3 }& {k=1} 
   \end{cases}
   \\
p_{Y|X}\brak{0|0} = \frac{19}{25}\, 
p_{Y|X}\brak{0|1} = \frac{6}{25}\,
p_{Y|X}\brak{1|0} = \frac{45}{50}\,
p_{Y|X}\brak{1|2} = \frac{5}{50}
\end{align}
The desired probability is the probability that a slip drawn at random is marked other than Rs 1,
\begin{align}
&=1-p_X\brak{0}\\
&= p_X(1) + p_X(2)
\end{align}
Using Bayes theorem,
\begin{align}
&= p_Y\brak{0} \times \pr{Y=0 | X=1} + p_Y\brak{1} \times \pr{Y=1|X=2}\\
&=\frac{1}{3} \times \frac{6}{25} + \frac{2}{3} \times \frac{5}{50}\\
&=\frac{11}{75}
\end{align}

\newpage

%\tableofcontents

\bigskip

\renewcommand{\thefigure}{\theenumi}
\renewcommand{\thetable}{\theenumi}
%\renewcommand{\theequation}{\theenumi}

%\begin{abstract}
%%\boldmath
%In this letter, an algorithm for evaluating the exact analytical bit error rate  (BER)  for the piecewise linear (PL) combiner for  multiple relays is presented. Previous results were available only for upto three relays. The algorithm is unique in the sense that  the actual mathematical expressions, that are prohibitively large, need not be explicitly obtained. The diversity gain due to multiple relays is shown through plots of the analytical BER, well supported by simulations. 
%
%\end{abstract}
% IEEEtran.cls defaults to using nonbold math in the Abstract.
% This preserves the distinction between vectors and scalars. However,
% if the journal you are submitting to favors bold math in the abstract,
% then you can use LaTeX's standard command \boldmath at the very start
% of the abstract to achieve this. Many IEEE journals frown on math
% in the abstract anyway.

% Note that keywords are not normally used for peerreview papers.
%\begin{IEEEkeywords}
%Cooperative diversity, decode and forward, piecewise linear
%\end{IEEEkeywords}



% For peer review papers, you can put extra information on the cover
% page as needed:
% \ifCLASSOPTIONpeerreview
% \begin{center} \bfseries EDICS Category: 3-BBND \end{center}
% \fi
%
% For peerreview papers, this IEEEtran command inserts a page break and
% creates the second title. It will be ignored for other modes.
%\IEEEpeerreviewmaketitle




\item Consider the probability space $\brak{\Omega, \mathcal{G}, P}$ where $\Omega = [0,2]$ and $\mathcal{G} = \cbrak{\phi, \Omega, [0,1], (1,2]}$. Let $X$ and $Y$ be two functions on $\Omega$ defined as
\begin{align*}
    X(\omega) = 
    \begin{cases}
        1 & \text{if }\omega \in [0, 1]\\
        2 & \text{if }\omega \in (1, 2]
    \end{cases}
\end{align*}
and
\begin{align*}
    Y(\omega) = 
    \begin{cases}
        2 & \text{if }\omega \in [0, 1.5]\\
        3 & \text{if }\omega \in (1.5, 2].
    \end{cases}
\end{align*}
Then which one of the following statements is true?
\begin{enumerate}
    \item [(A)] $X$ is a random variable with respect to $\mathcal{G}$, but $Y$ is not a random variable with respect to $\mathcal{G}$.
    \item [(B)] $Y$ is a random variable with respect to $\mathcal{G}$, but $X$ is not a random variable with respect to $\mathcal{G}$.
    \item [(C)] Neither $X$ nor $Y$ is a random variable with respect to $\mathcal{G}$.
    \item [(D)] Both $X$ and $Y$ are random variables with respect to $\mathcal{G}$.
\end{enumerate} \hfill (GATE ST 2023)\\
\solution
%\begin{table}[H]
	\centering
\begin{tabular}{|c|c|c|}
\hline
Random variable &Value &Definition\\ \hline
\multirow{3}{*}{X} &0 &Slips of Rs 1\\
&1 &Slips of Rs 5\\
&2 &Slips of Rs 13\\ \hline
\multirow{2}{*}{Y} &0 &Box A\\
&1 &Box B\\\hline
\end{tabular}
\caption{}
\label{tab:Distribution}
\end{table}
See \tabref{tab:Distribution}.
\begin{align}
p_{Y}\brak{k}= \begin{cases} 
      \frac{1}{3} & {k=0} \\
      \frac{2}{3 }& {k=1} 
   \end{cases}
   \\
p_{Y|X}\brak{0|0} = \frac{19}{25}\, 
p_{Y|X}\brak{0|1} = \frac{6}{25}\,
p_{Y|X}\brak{1|0} = \frac{45}{50}\,
p_{Y|X}\brak{1|2} = \frac{5}{50}
\end{align}
The desired probability is the probability that a slip drawn at random is marked other than Rs 1,
\begin{align}
&=1-p_X\brak{0}\\
&= p_X(1) + p_X(2)
\end{align}
Using Bayes theorem,
\begin{align}
&= p_Y\brak{0} \times \pr{Y=0 | X=1} + p_Y\brak{1} \times \pr{Y=1|X=2}\\
&=\frac{1}{3} \times \frac{6}{25} + \frac{2}{3} \times \frac{5}{50}\\
&=\frac{11}{75}
\end{align}

\newpage

%\tableofcontents

\bigskip

\renewcommand{\thefigure}{\theenumi}
\renewcommand{\thetable}{\theenumi}
%\renewcommand{\theequation}{\theenumi}

%\begin{abstract}
%%\boldmath
%In this letter, an algorithm for evaluating the exact analytical bit error rate  (BER)  for the piecewise linear (PL) combiner for  multiple relays is presented. Previous results were available only for upto three relays. The algorithm is unique in the sense that  the actual mathematical expressions, that are prohibitively large, need not be explicitly obtained. The diversity gain due to multiple relays is shown through plots of the analytical BER, well supported by simulations. 
%
%\end{abstract}
% IEEEtran.cls defaults to using nonbold math in the Abstract.
% This preserves the distinction between vectors and scalars. However,
% if the journal you are submitting to favors bold math in the abstract,
% then you can use LaTeX's standard command \boldmath at the very start
% of the abstract to achieve this. Many IEEE journals frown on math
% in the abstract anyway.

% Note that keywords are not normally used for peerreview papers.
%\begin{IEEEkeywords}
%Cooperative diversity, decode and forward, piecewise linear
%\end{IEEEkeywords}



% For peer review papers, you can put extra information on the cover
% page as needed:
% \ifCLASSOPTIONpeerreview
% \begin{center} \bfseries EDICS Category: 3-BBND \end{center}
% \fi
%
% For peerreview papers, this IEEEtran command inserts a page break and
% creates the second title. It will be ignored for other modes.
%\IEEEpeerreviewmaketitle




	\item  A die is loaded in such a way that each odd number is twice as likely to occur as
each even number. Find $P(G)$, where $G$ is the event that a number greater than
3 occurs on a single roll of the die.
\\
\solution
		%\begin{table}[H]
	\centering
\begin{tabular}{|c|c|c|}
\hline
Random variable &Value &Definition\\ \hline
\multirow{3}{*}{X} &0 &Slips of Rs 1\\
&1 &Slips of Rs 5\\
&2 &Slips of Rs 13\\ \hline
\multirow{2}{*}{Y} &0 &Box A\\
&1 &Box B\\\hline
\end{tabular}
\caption{}
\label{tab:Distribution}
\end{table}
See \tabref{tab:Distribution}.
\begin{align}
p_{Y}\brak{k}= \begin{cases} 
      \frac{1}{3} & {k=0} \\
      \frac{2}{3 }& {k=1} 
   \end{cases}
   \\
p_{Y|X}\brak{0|0} = \frac{19}{25}\, 
p_{Y|X}\brak{0|1} = \frac{6}{25}\,
p_{Y|X}\brak{1|0} = \frac{45}{50}\,
p_{Y|X}\brak{1|2} = \frac{5}{50}
\end{align}
The desired probability is the probability that a slip drawn at random is marked other than Rs 1,
\begin{align}
&=1-p_X\brak{0}\\
&= p_X(1) + p_X(2)
\end{align}
Using Bayes theorem,
\begin{align}
&= p_Y\brak{0} \times \pr{Y=0 | X=1} + p_Y\brak{1} \times \pr{Y=1|X=2}\\
&=\frac{1}{3} \times \frac{6}{25} + \frac{2}{3} \times \frac{5}{50}\\
&=\frac{11}{75}
\end{align}

\newpage

%\tableofcontents

\bigskip

\renewcommand{\thefigure}{\theenumi}
\renewcommand{\thetable}{\theenumi}
%\renewcommand{\theequation}{\theenumi}

%\begin{abstract}
%%\boldmath
%In this letter, an algorithm for evaluating the exact analytical bit error rate  (BER)  for the piecewise linear (PL) combiner for  multiple relays is presented. Previous results were available only for upto three relays. The algorithm is unique in the sense that  the actual mathematical expressions, that are prohibitively large, need not be explicitly obtained. The diversity gain due to multiple relays is shown through plots of the analytical BER, well supported by simulations. 
%
%\end{abstract}
% IEEEtran.cls defaults to using nonbold math in the Abstract.
% This preserves the distinction between vectors and scalars. However,
% if the journal you are submitting to favors bold math in the abstract,
% then you can use LaTeX's standard command \boldmath at the very start
% of the abstract to achieve this. Many IEEE journals frown on math
% in the abstract anyway.

% Note that keywords are not normally used for peerreview papers.
%\begin{IEEEkeywords}
%Cooperative diversity, decode and forward, piecewise linear
%\end{IEEEkeywords}



% For peer review papers, you can put extra information on the cover
% page as needed:
% \ifCLASSOPTIONpeerreview
% \begin{center} \bfseries EDICS Category: 3-BBND \end{center}
% \fi
%
% For peerreview papers, this IEEEtran command inserts a page break and
% creates the second title. It will be ignored for other modes.
%\IEEEpeerreviewmaketitle




	\item All the jacks, queens and kings are removed from a deck of 52 playing cards. The remaining cards are well shuffled and then one card is drawn at random. Giving ace a value 1 similar value for other cards, find the probability that the card has a value 
		\begin{enumerate}
			\item 7
			\item greater than 7
			\item less than 7
		\end{enumerate}
		%Number of cards left after removing all jacks, queens and kings 
\begin{align}
N	= 52 - 4\times 3
	= 40
\end{align}
%\begin{table}[H]
%\def\arraystretch{1.2}
%\begin{tabular}{|c|c|c|}
%\hline
%	\textbf{Parameter} &\textbf{Value} &\textbf{Description}\\ \hline
%	$X$ &1-10 &Represents the value of the card picked \\ \hline
%\end{tabular}
%\end{table}
Let $1 \le X \le 10$ be the value of the card picked.  Then,
\begin{align}
	p_X(k) &= \Pr(X=k)\ \forall\ 1 \leq k \leq 10\\
	&= \frac{4\times 1}{40}\\
	&= \frac{1}{10}\\
	\therefore p_X(k) &= 
	\begin{cases}
		\frac{1}{10} & 1 \leq k \leq 10\\
		0 & \text{otherwise}
	\end{cases}
\end{align}
and
\begin{align}
	F_{X}(k) &= \sum_{m=0}^{k}p_{X}(m) \quad 1 \leq k \leq 10\\
	&= \frac{k}{10}\\
	\therefore F_{X}(k) &= 
	\begin{cases}
		0 & k \leq 0\\
		\frac{k}{10} & 1\leq k \leq 10\\
		1 & k > 10 
	\end{cases}
\end{align}
\begin{enumerate}
	\item Probability that card has value equal to 7 is
		\begin{align}
			 p_{X}(7)
			= \frac{1}{10}
		\end{align}
	\item Probability that card has value greater than 7 is
		\begin{align}
			1 - F_X(7)
			&= 1 - \frac{7}{10}
			\\
			&= \frac{3}{10}
		\end{align}
	\item Probability that card has value less than 7 is
		\begin{align}
			 F_{X}(6)
			=\frac{6}{10}
		\end{align}
\end{enumerate}

  \item A Lot consists of 48 mobile phones of which 42 are good, 3 have only minor defects and 3 have major defects.Varnika will buy a phone if it is good but the trader will only buy a mobile if it has no major defects. One phone is selected at random from the lot. What is the probability that it is
\begin{enumerate}
	\item acceptable to Varnika?
            \item acceptable to the trader?
\end{enumerate}
\solution
	%\begin{table}[H]
	\centering
\begin{tabular}{|c|c|c|}
\hline
Random variable &Value &Definition\\ \hline
\multirow{3}{*}{X} &0 &Slips of Rs 1\\
&1 &Slips of Rs 5\\
&2 &Slips of Rs 13\\ \hline
\multirow{2}{*}{Y} &0 &Box A\\
&1 &Box B\\\hline
\end{tabular}
\caption{}
\label{tab:Distribution}
\end{table}
See \tabref{tab:Distribution}.
\begin{align}
p_{Y}\brak{k}= \begin{cases} 
      \frac{1}{3} & {k=0} \\
      \frac{2}{3 }& {k=1} 
   \end{cases}
   \\
p_{Y|X}\brak{0|0} = \frac{19}{25}\, 
p_{Y|X}\brak{0|1} = \frac{6}{25}\,
p_{Y|X}\brak{1|0} = \frac{45}{50}\,
p_{Y|X}\brak{1|2} = \frac{5}{50}
\end{align}
The desired probability is the probability that a slip drawn at random is marked other than Rs 1,
\begin{align}
&=1-p_X\brak{0}\\
&= p_X(1) + p_X(2)
\end{align}
Using Bayes theorem,
\begin{align}
&= p_Y\brak{0} \times \pr{Y=0 | X=1} + p_Y\brak{1} \times \pr{Y=1|X=2}\\
&=\frac{1}{3} \times \frac{6}{25} + \frac{2}{3} \times \frac{5}{50}\\
&=\frac{11}{75}
\end{align}

\newpage

%\tableofcontents

\bigskip

\renewcommand{\thefigure}{\theenumi}
\renewcommand{\thetable}{\theenumi}
%\renewcommand{\theequation}{\theenumi}

%\begin{abstract}
%%\boldmath
%In this letter, an algorithm for evaluating the exact analytical bit error rate  (BER)  for the piecewise linear (PL) combiner for  multiple relays is presented. Previous results were available only for upto three relays. The algorithm is unique in the sense that  the actual mathematical expressions, that are prohibitively large, need not be explicitly obtained. The diversity gain due to multiple relays is shown through plots of the analytical BER, well supported by simulations. 
%
%\end{abstract}
% IEEEtran.cls defaults to using nonbold math in the Abstract.
% This preserves the distinction between vectors and scalars. However,
% if the journal you are submitting to favors bold math in the abstract,
% then you can use LaTeX's standard command \boldmath at the very start
% of the abstract to achieve this. Many IEEE journals frown on math
% in the abstract anyway.

% Note that keywords are not normally used for peerreview papers.
%\begin{IEEEkeywords}
%Cooperative diversity, decode and forward, piecewise linear
%\end{IEEEkeywords}



% For peer review papers, you can put extra information on the cover
% page as needed:
% \ifCLASSOPTIONpeerreview
% \begin{center} \bfseries EDICS Category: 3-BBND \end{center}
% \fi
%
% For peerreview papers, this IEEEtran command inserts a page break and
% creates the second title. It will be ignored for other modes.
%\IEEEpeerreviewmaketitle




 \item A student says that if you throw a die, it will show up 1 or not 1. Therefore, the probability of getting 1 and the probability of getting 'not 1' each is equal to $\frac{1}{2}$. Is this correct? Give reasons.\\
 \solution
        %\begin{table}[H]
	\centering
\begin{tabular}{|c|c|c|}
\hline
Random variable &Value &Definition\\ \hline
\multirow{3}{*}{X} &0 &Slips of Rs 1\\
&1 &Slips of Rs 5\\
&2 &Slips of Rs 13\\ \hline
\multirow{2}{*}{Y} &0 &Box A\\
&1 &Box B\\\hline
\end{tabular}
\caption{}
\label{tab:Distribution}
\end{table}
See \tabref{tab:Distribution}.
\begin{align}
p_{Y}\brak{k}= \begin{cases} 
      \frac{1}{3} & {k=0} \\
      \frac{2}{3 }& {k=1} 
   \end{cases}
   \\
p_{Y|X}\brak{0|0} = \frac{19}{25}\, 
p_{Y|X}\brak{0|1} = \frac{6}{25}\,
p_{Y|X}\brak{1|0} = \frac{45}{50}\,
p_{Y|X}\brak{1|2} = \frac{5}{50}
\end{align}
The desired probability is the probability that a slip drawn at random is marked other than Rs 1,
\begin{align}
&=1-p_X\brak{0}\\
&= p_X(1) + p_X(2)
\end{align}
Using Bayes theorem,
\begin{align}
&= p_Y\brak{0} \times \pr{Y=0 | X=1} + p_Y\brak{1} \times \pr{Y=1|X=2}\\
&=\frac{1}{3} \times \frac{6}{25} + \frac{2}{3} \times \frac{5}{50}\\
&=\frac{11}{75}
\end{align}

\newpage

%\tableofcontents

\bigskip

\renewcommand{\thefigure}{\theenumi}
\renewcommand{\thetable}{\theenumi}
%\renewcommand{\theequation}{\theenumi}

%\begin{abstract}
%%\boldmath
%In this letter, an algorithm for evaluating the exact analytical bit error rate  (BER)  for the piecewise linear (PL) combiner for  multiple relays is presented. Previous results were available only for upto three relays. The algorithm is unique in the sense that  the actual mathematical expressions, that are prohibitively large, need not be explicitly obtained. The diversity gain due to multiple relays is shown through plots of the analytical BER, well supported by simulations. 
%
%\end{abstract}
% IEEEtran.cls defaults to using nonbold math in the Abstract.
% This preserves the distinction between vectors and scalars. However,
% if the journal you are submitting to favors bold math in the abstract,
% then you can use LaTeX's standard command \boldmath at the very start
% of the abstract to achieve this. Many IEEE journals frown on math
% in the abstract anyway.

% Note that keywords are not normally used for peerreview papers.
%\begin{IEEEkeywords}
%Cooperative diversity, decode and forward, piecewise linear
%\end{IEEEkeywords}



% For peer review papers, you can put extra information on the cover
% page as needed:
% \ifCLASSOPTIONpeerreview
% \begin{center} \bfseries EDICS Category: 3-BBND \end{center}
% \fi
%
% For peerreview papers, this IEEEtran command inserts a page break and
% creates the second title. It will be ignored for other modes.
%\IEEEpeerreviewmaketitle




   \item Four candidates A, B, C, D have ap-
plied for the assignment to coach a school cricket
team. If A is twice as likely to be selected as B, and
B and C are given about the same chance of being
selected, while C is twice as likely to be selected
as D, what are the probabilities that
\begin{enumerate}
\item C will be selected?
\item A will not be selected?
\end{enumerate}
	%\begin{table}[H]
	\centering
\begin{tabular}{|c|c|c|}
\hline
Random variable &Value &Definition\\ \hline
\multirow{3}{*}{X} &0 &Slips of Rs 1\\
&1 &Slips of Rs 5\\
&2 &Slips of Rs 13\\ \hline
\multirow{2}{*}{Y} &0 &Box A\\
&1 &Box B\\\hline
\end{tabular}
\caption{}
\label{tab:Distribution}
\end{table}
See \tabref{tab:Distribution}.
\begin{align}
p_{Y}\brak{k}= \begin{cases} 
      \frac{1}{3} & {k=0} \\
      \frac{2}{3 }& {k=1} 
   \end{cases}
   \\
p_{Y|X}\brak{0|0} = \frac{19}{25}\, 
p_{Y|X}\brak{0|1} = \frac{6}{25}\,
p_{Y|X}\brak{1|0} = \frac{45}{50}\,
p_{Y|X}\brak{1|2} = \frac{5}{50}
\end{align}
The desired probability is the probability that a slip drawn at random is marked other than Rs 1,
\begin{align}
&=1-p_X\brak{0}\\
&= p_X(1) + p_X(2)
\end{align}
Using Bayes theorem,
\begin{align}
&= p_Y\brak{0} \times \pr{Y=0 | X=1} + p_Y\brak{1} \times \pr{Y=1|X=2}\\
&=\frac{1}{3} \times \frac{6}{25} + \frac{2}{3} \times \frac{5}{50}\\
&=\frac{11}{75}
\end{align}

\newpage

%\tableofcontents

\bigskip

\renewcommand{\thefigure}{\theenumi}
\renewcommand{\thetable}{\theenumi}
%\renewcommand{\theequation}{\theenumi}

%\begin{abstract}
%%\boldmath
%In this letter, an algorithm for evaluating the exact analytical bit error rate  (BER)  for the piecewise linear (PL) combiner for  multiple relays is presented. Previous results were available only for upto three relays. The algorithm is unique in the sense that  the actual mathematical expressions, that are prohibitively large, need not be explicitly obtained. The diversity gain due to multiple relays is shown through plots of the analytical BER, well supported by simulations. 
%
%\end{abstract}
% IEEEtran.cls defaults to using nonbold math in the Abstract.
% This preserves the distinction between vectors and scalars. However,
% if the journal you are submitting to favors bold math in the abstract,
% then you can use LaTeX's standard command \boldmath at the very start
% of the abstract to achieve this. Many IEEE journals frown on math
% in the abstract anyway.

% Note that keywords are not normally used for peerreview papers.
%\begin{IEEEkeywords}
%Cooperative diversity, decode and forward, piecewise linear
%\end{IEEEkeywords}



% For peer review papers, you can put extra information on the cover
% page as needed:
% \ifCLASSOPTIONpeerreview
% \begin{center} \bfseries EDICS Category: 3-BBND \end{center}
% \fi
%
% For peerreview papers, this IEEEtran command inserts a page break and
% creates the second title. It will be ignored for other modes.
%\IEEEpeerreviewmaketitle




 \item A bag contain 24 balls of which $x$ balls are red, $2x$ are white and $3x$ are blue. A ball is selected at random, What is the probability that it is
\begin{enumerate}[label=\alph*)]
\item not red ?
\item white ?
\end{enumerate}
%\begin{table}[H]
	\centering
\begin{tabular}{|c|c|c|}
\hline
Random variable &Value &Definition\\ \hline
\multirow{3}{*}{X} &0 &Slips of Rs 1\\
&1 &Slips of Rs 5\\
&2 &Slips of Rs 13\\ \hline
\multirow{2}{*}{Y} &0 &Box A\\
&1 &Box B\\\hline
\end{tabular}
\caption{}
\label{tab:Distribution}
\end{table}
See \tabref{tab:Distribution}.
\begin{align}
p_{Y}\brak{k}= \begin{cases} 
      \frac{1}{3} & {k=0} \\
      \frac{2}{3 }& {k=1} 
   \end{cases}
   \\
p_{Y|X}\brak{0|0} = \frac{19}{25}\, 
p_{Y|X}\brak{0|1} = \frac{6}{25}\,
p_{Y|X}\brak{1|0} = \frac{45}{50}\,
p_{Y|X}\brak{1|2} = \frac{5}{50}
\end{align}
The desired probability is the probability that a slip drawn at random is marked other than Rs 1,
\begin{align}
&=1-p_X\brak{0}\\
&= p_X(1) + p_X(2)
\end{align}
Using Bayes theorem,
\begin{align}
&= p_Y\brak{0} \times \pr{Y=0 | X=1} + p_Y\brak{1} \times \pr{Y=1|X=2}\\
&=\frac{1}{3} \times \frac{6}{25} + \frac{2}{3} \times \frac{5}{50}\\
&=\frac{11}{75}
\end{align}

\newpage

%\tableofcontents

\bigskip

\renewcommand{\thefigure}{\theenumi}
\renewcommand{\thetable}{\theenumi}
%\renewcommand{\theequation}{\theenumi}

%\begin{abstract}
%%\boldmath
%In this letter, an algorithm for evaluating the exact analytical bit error rate  (BER)  for the piecewise linear (PL) combiner for  multiple relays is presented. Previous results were available only for upto three relays. The algorithm is unique in the sense that  the actual mathematical expressions, that are prohibitively large, need not be explicitly obtained. The diversity gain due to multiple relays is shown through plots of the analytical BER, well supported by simulations. 
%
%\end{abstract}
% IEEEtran.cls defaults to using nonbold math in the Abstract.
% This preserves the distinction between vectors and scalars. However,
% if the journal you are submitting to favors bold math in the abstract,
% then you can use LaTeX's standard command \boldmath at the very start
% of the abstract to achieve this. Many IEEE journals frown on math
% in the abstract anyway.

% Note that keywords are not normally used for peerreview papers.
%\begin{IEEEkeywords}
%Cooperative diversity, decode and forward, piecewise linear
%\end{IEEEkeywords}



% For peer review papers, you can put extra information on the cover
% page as needed:
% \ifCLASSOPTIONpeerreview
% \begin{center} \bfseries EDICS Category: 3-BBND \end{center}
% \fi
%
% For peerreview papers, this IEEEtran command inserts a page break and
% creates the second title. It will be ignored for other modes.
%\IEEEpeerreviewmaketitle




If the letters of the word ASSASSINATION are arranged at random. Find the Probability that
\begin{enumerate}[label=(\alph*)]
\item Four $S's$ come consecutively in the word
\item Two  $I's$ and two $N's$ come together
\item All $A's$ are not coming together
\item No two $A's$ are coming together
\end{enumerate}
%\begin{table}[H]
	\centering
\begin{tabular}{|c|c|c|}
\hline
Random variable &Value &Definition\\ \hline
\multirow{3}{*}{X} &0 &Slips of Rs 1\\
&1 &Slips of Rs 5\\
&2 &Slips of Rs 13\\ \hline
\multirow{2}{*}{Y} &0 &Box A\\
&1 &Box B\\\hline
\end{tabular}
\caption{}
\label{tab:Distribution}
\end{table}
See \tabref{tab:Distribution}.
\begin{align}
p_{Y}\brak{k}= \begin{cases} 
      \frac{1}{3} & {k=0} \\
      \frac{2}{3 }& {k=1} 
   \end{cases}
   \\
p_{Y|X}\brak{0|0} = \frac{19}{25}\, 
p_{Y|X}\brak{0|1} = \frac{6}{25}\,
p_{Y|X}\brak{1|0} = \frac{45}{50}\,
p_{Y|X}\brak{1|2} = \frac{5}{50}
\end{align}
The desired probability is the probability that a slip drawn at random is marked other than Rs 1,
\begin{align}
&=1-p_X\brak{0}\\
&= p_X(1) + p_X(2)
\end{align}
Using Bayes theorem,
\begin{align}
&= p_Y\brak{0} \times \pr{Y=0 | X=1} + p_Y\brak{1} \times \pr{Y=1|X=2}\\
&=\frac{1}{3} \times \frac{6}{25} + \frac{2}{3} \times \frac{5}{50}\\
&=\frac{11}{75}
\end{align}

\newpage

%\tableofcontents

\bigskip

\renewcommand{\thefigure}{\theenumi}
\renewcommand{\thetable}{\theenumi}
%\renewcommand{\theequation}{\theenumi}

%\begin{abstract}
%%\boldmath
%In this letter, an algorithm for evaluating the exact analytical bit error rate  (BER)  for the piecewise linear (PL) combiner for  multiple relays is presented. Previous results were available only for upto three relays. The algorithm is unique in the sense that  the actual mathematical expressions, that are prohibitively large, need not be explicitly obtained. The diversity gain due to multiple relays is shown through plots of the analytical BER, well supported by simulations. 
%
%\end{abstract}
% IEEEtran.cls defaults to using nonbold math in the Abstract.
% This preserves the distinction between vectors and scalars. However,
% if the journal you are submitting to favors bold math in the abstract,
% then you can use LaTeX's standard command \boldmath at the very start
% of the abstract to achieve this. Many IEEE journals frown on math
% in the abstract anyway.

% Note that keywords are not normally used for peerreview papers.
%\begin{IEEEkeywords}
%Cooperative diversity, decode and forward, piecewise linear
%\end{IEEEkeywords}



% For peer review papers, you can put extra information on the cover
% page as needed:
% \ifCLASSOPTIONpeerreview
% \begin{center} \bfseries EDICS Category: 3-BBND \end{center}
% \fi
%
% For peerreview papers, this IEEEtran command inserts a page break and
% creates the second title. It will be ignored for other modes.
%\IEEEpeerreviewmaketitle




	\item One urn contains two black balls (labelled B1 and B2) and one white ball. A
	second urn contains one black ball and two white balls (labelled W1 and W2).
	Suppose the following experiment is performed. One of the two urns is chosen
	at random. Next a ball is randomly chosen from the urn. Then a second ball is
	chosen at random from the same urn without replacing the first ball.
	
	\begin{enumerate}
	\item What is the probability that two black balls are chosen?
	
	\item What is the probability that two balls of opposite colour are chosen?
	\end{enumerate}
	\solution
	%\begin{align}
    \label{eq:12.13.6.18.1}
	\because	\pr{A|B} &> \pr{A},\
\frac{\pr{AB}}{\pr{B}} > \pr{A}
\\
    \label{eq:12.13.6.18.2}
	\implies \pr{AB} &> \pr{A}\pr{B}
	\\
	\text{or, } \frac{\pr{AB}}{\pr{A}} &=\pr{B|A} > \pr{A}
\end{align}

\end{enumerate}

	\item A card is selected from a pack of 52 cards.
 \begin{enumerate}[label=(\alph*)] 
                 \item How many points are there in the sample space?
                 \item Calculate the probability that the card is an ace of spades.
                 \item Calculate the probability that the card is (i) an ace and (ii) black card.
 \end{enumerate}
\solution
		%\begin{table}[H]
	\centering
\begin{tabular}{|c|c|c|}
\hline
Random variable &Value &Definition\\ \hline
\multirow{3}{*}{X} &0 &Slips of Rs 1\\
&1 &Slips of Rs 5\\
&2 &Slips of Rs 13\\ \hline
\multirow{2}{*}{Y} &0 &Box A\\
&1 &Box B\\\hline
\end{tabular}
\caption{}
\label{tab:Distribution}
\end{table}
See \tabref{tab:Distribution}.
\begin{align}
p_{Y}\brak{k}= \begin{cases} 
      \frac{1}{3} & {k=0} \\
      \frac{2}{3 }& {k=1} 
   \end{cases}
   \\
p_{Y|X}\brak{0|0} = \frac{19}{25}\, 
p_{Y|X}\brak{0|1} = \frac{6}{25}\,
p_{Y|X}\brak{1|0} = \frac{45}{50}\,
p_{Y|X}\brak{1|2} = \frac{5}{50}
\end{align}
The desired probability is the probability that a slip drawn at random is marked other than Rs 1,
\begin{align}
&=1-p_X\brak{0}\\
&= p_X(1) + p_X(2)
\end{align}
Using Bayes theorem,
\begin{align}
&= p_Y\brak{0} \times \pr{Y=0 | X=1} + p_Y\brak{1} \times \pr{Y=1|X=2}\\
&=\frac{1}{3} \times \frac{6}{25} + \frac{2}{3} \times \frac{5}{50}\\
&=\frac{11}{75}
\end{align}

\newpage

%\tableofcontents

\bigskip

\renewcommand{\thefigure}{\theenumi}
\renewcommand{\thetable}{\theenumi}
%\renewcommand{\theequation}{\theenumi}

%\begin{abstract}
%%\boldmath
%In this letter, an algorithm for evaluating the exact analytical bit error rate  (BER)  for the piecewise linear (PL) combiner for  multiple relays is presented. Previous results were available only for upto three relays. The algorithm is unique in the sense that  the actual mathematical expressions, that are prohibitively large, need not be explicitly obtained. The diversity gain due to multiple relays is shown through plots of the analytical BER, well supported by simulations. 
%
%\end{abstract}
% IEEEtran.cls defaults to using nonbold math in the Abstract.
% This preserves the distinction between vectors and scalars. However,
% if the journal you are submitting to favors bold math in the abstract,
% then you can use LaTeX's standard command \boldmath at the very start
% of the abstract to achieve this. Many IEEE journals frown on math
% in the abstract anyway.

% Note that keywords are not normally used for peerreview papers.
%\begin{IEEEkeywords}
%Cooperative diversity, decode and forward, piecewise linear
%\end{IEEEkeywords}



% For peer review papers, you can put extra information on the cover
% page as needed:
% \ifCLASSOPTIONpeerreview
% \begin{center} \bfseries EDICS Category: 3-BBND \end{center}
% \fi
%
% For peerreview papers, this IEEEtran command inserts a page break and
% creates the second title. It will be ignored for other modes.
%\IEEEpeerreviewmaketitle




\item Four cards are drawn from a well-shuffled deck of 52 cards. What is the probability of obtaining 3 diamonds and one spade.
\\
\solution
		%\begin{enumerate}[label=\thesection.\arabic*,ref=\thesection.\theenumi]
	\item One card is drawn from a well-shuffled deck of 52 cards. Find the probability of getting
\begin{enumerate}
\item A king of red colour 
\item A face card 
\item A red face card
\item The jack of hearts
\item A spade
\item The queen of diamonds

\end{enumerate}
\solution
		%\begin{table}[H]
	\centering
\begin{tabular}{|c|c|c|}
\hline
Random variable &Value &Definition\\ \hline
\multirow{3}{*}{X} &0 &Slips of Rs 1\\
&1 &Slips of Rs 5\\
&2 &Slips of Rs 13\\ \hline
\multirow{2}{*}{Y} &0 &Box A\\
&1 &Box B\\\hline
\end{tabular}
\caption{}
\label{tab:Distribution}
\end{table}
See \tabref{tab:Distribution}.
\begin{align}
p_{Y}\brak{k}= \begin{cases} 
      \frac{1}{3} & {k=0} \\
      \frac{2}{3 }& {k=1} 
   \end{cases}
   \\
p_{Y|X}\brak{0|0} = \frac{19}{25}\, 
p_{Y|X}\brak{0|1} = \frac{6}{25}\,
p_{Y|X}\brak{1|0} = \frac{45}{50}\,
p_{Y|X}\brak{1|2} = \frac{5}{50}
\end{align}
The desired probability is the probability that a slip drawn at random is marked other than Rs 1,
\begin{align}
&=1-p_X\brak{0}\\
&= p_X(1) + p_X(2)
\end{align}
Using Bayes theorem,
\begin{align}
&= p_Y\brak{0} \times \pr{Y=0 | X=1} + p_Y\brak{1} \times \pr{Y=1|X=2}\\
&=\frac{1}{3} \times \frac{6}{25} + \frac{2}{3} \times \frac{5}{50}\\
&=\frac{11}{75}
\end{align}

\newpage

%\tableofcontents

\bigskip

\renewcommand{\thefigure}{\theenumi}
\renewcommand{\thetable}{\theenumi}
%\renewcommand{\theequation}{\theenumi}

%\begin{abstract}
%%\boldmath
%In this letter, an algorithm for evaluating the exact analytical bit error rate  (BER)  for the piecewise linear (PL) combiner for  multiple relays is presented. Previous results were available only for upto three relays. The algorithm is unique in the sense that  the actual mathematical expressions, that are prohibitively large, need not be explicitly obtained. The diversity gain due to multiple relays is shown through plots of the analytical BER, well supported by simulations. 
%
%\end{abstract}
% IEEEtran.cls defaults to using nonbold math in the Abstract.
% This preserves the distinction between vectors and scalars. However,
% if the journal you are submitting to favors bold math in the abstract,
% then you can use LaTeX's standard command \boldmath at the very start
% of the abstract to achieve this. Many IEEE journals frown on math
% in the abstract anyway.

% Note that keywords are not normally used for peerreview papers.
%\begin{IEEEkeywords}
%Cooperative diversity, decode and forward, piecewise linear
%\end{IEEEkeywords}



% For peer review papers, you can put extra information on the cover
% page as needed:
% \ifCLASSOPTIONpeerreview
% \begin{center} \bfseries EDICS Category: 3-BBND \end{center}
% \fi
%
% For peerreview papers, this IEEEtran command inserts a page break and
% creates the second title. It will be ignored for other modes.
%\IEEEpeerreviewmaketitle




	\item Five cards—the ten, jack, queen, king and ace of diamonds, are well-shuffled with their face downwards. One card is then picked up at random.
\begin{enumerate}
\item
What is the probability that the card is the queen? 
\item
If the queen is drawn and put aside, what is the probability that the second card picked up is (a) an ace? (b) a queen?\\
\end{enumerate}
\solution
		%\begin{enumerate}[label=\thesection.\arabic*,ref=\thesection.\theenumi]
	\item One card is drawn from a well-shuffled deck of 52 cards. Find the probability of getting
\begin{enumerate}
\item A king of red colour 
\item A face card 
\item A red face card
\item The jack of hearts
\item A spade
\item The queen of diamonds

\end{enumerate}
\solution
		%\input{ncert/10/15/1/14/main.tex}
	\item Five cards—the ten, jack, queen, king and ace of diamonds, are well-shuffled with their face downwards. One card is then picked up at random.
\begin{enumerate}
\item
What is the probability that the card is the queen? 
\item
If the queen is drawn and put aside, what is the probability that the second card picked up is (a) an ace? (b) a queen?\\
\end{enumerate}
\solution
		%\input{ncert/10/15/1/15/defs.tex}
	\item A bag contains $5$ red balls and some blue balls. If the probability of drawing a blue ball is double that if a red ball, determine the number of blue balls in the bag. 
		\\
\solution
		%\input{ncert/10/15/2/3/defs.tex}
	\item A card is selected from a pack of 52 cards.
 \begin{enumerate}[label=(\alph*)] 
                 \item How many points are there in the sample space?
                 \item Calculate the probability that the card is an ace of spades.
                 \item Calculate the probability that the card is (i) an ace and (ii) black card.
 \end{enumerate}
\solution
		%\input{ncert/11/16/3/4/main.tex}
\item Four cards are drawn from a well-shuffled deck of 52 cards. What is the probability of obtaining 3 diamonds and one spade.
\\
\solution
		%\input{ncert/11/16/4/2/defs.tex}
\item In a certain lottery 10,000 tickets are sold and ten equal prizes are awarded. What is the probability of not getting a prize if you buy (a) one ticket (b) two tickets (c) 10 tickets ?	
\\
\solution
		%\input{ncert/11/16/4/4/defs.tex}
		%
\item 
Out of 100 students, two sections of 40 and 60 are formed. If you and your friend are among the 100 students, what is the probability that
\begin{enumerate}
\item you both enter the same section?
\item you both enter the different sections?
\end{enumerate}
\solution
		%\input{ncert/11/16/4/5/defs.tex}
	\item 
The number lock of a suitcase has 4 wheels each labelled with ten digits i.e. from 0 to 9.The lock opens with a sequence of four digits with no repeats.What is the probability of a person getting the right sequence to open the suitcase.
\\
\solution
		%\input{ncert/11/16/4/10/defs.tex}
		%
\item 
Two cards are drawn at random and without replacement from a pack of 52 playing cards. Find the probability that both the cards are black.
\\
\solution
		%\input{ncert/12/13/2/2/defs.tex}
		\item A box of oranges is inspected by examining three randomly selected oranges drawn without replacement. If all the three oranges are good, the box is approved for sale, otherwise, it is rejected. Find the probability that a box containing 15 oranges out of which 12 are good and 3 are bad ones will be approved for sale.
		\label{ncert/12/13/2/3/defs.tex}
		\item Two balls are drawn at random with replacement from a box containing 10 black and 8 red balls. Find the probability that
		\label{ncert/12/13/2/12}
\begin{enumerate}
\item both balls are red.
\item first ball is black and second is red.
\item one of them is black and other is red.
\end{enumerate}

\item In a hostel, 60\% of the students read Hindi newspaper, 40\% read English newspaper and 20\% read both Hindi and English newspapers. A student is selected at random.
		\label{ncert/12/13/2/15}
\begin{enumerate}
\item Find the probability that she reads neither Hindi nor English newspapers.
\item If she reads Hindi newspaper, find the probability that she reads English newspaper.
\item If she reads English newspaper, find the probability that she reads Hindi newspaper.\\
\end{enumerate}
\item The probability of obtaining an even prime number on each die, when a pair of dice is rolled is 
\begin{enumerate}
    \item $0$ 
    
    \item $\frac{1}{3}$ 
    
    \item $\frac{1}{12}$ 
    
    \item $\frac{1}{36}$ 
\end{enumerate}
\solution
		%\input{ncert/12/13/2/17/defs.tex}
	\item A bag contains 4 red and 4 black balls, another bag contains 2 red and 6 black balls. One of the two bags is selected at random and a ball is drawn from the bag which is found to be red. Find the probability that the ball is drawn from the first bag.
\\
\solution
		%\input{ncert/12/13/3/2/main.tex}
  \item
  Cards with numbers 2 to 101 are placed in a box. A card is selected at random.Find the probability that the card has
\begin{enumerate}[label=(\roman*)]
	\item an even number 
	\item a square number
\end{enumerate}
\solution
%\input{exemplar/10/13/3/32/main.tex}
\item
The king, queen and jack of clubs are removed from a deck of 52 playing cards and then well shuffled. Now one card is drawn at random from the remaining cards.  Determine the probability that the card is
\begin{enumerate}[label=(\roman*)]
\item a club
\item 10 of hearts
\end{enumerate}
\solution
%\input{exemplar/10/13/3/29/main.tex}
\item A team of medical students doing their internship have to assist during surgeries
at a city hospital. The probabilities of surgeries rated as very complex, complex,
routine, simple or very simple are respectively, 0.15, 0.20, 0.31, 0.26, .08. Find
the probabilities that a particular surgery will be rated
\begin{enumerate}
	\item complex or very complex;
	\item neither very complex nor very simple;
	\item routine or complex
	\item routine or simple
\end{enumerate}
\solution
%\input{exemplar/11/16/3/8(1)/main.tex}
\item A card is selected from a pack of 52 cards.
\begin{enumerate}[label=(\alph*)]
    \item How many points are there in the sample space?
    \item Calculate the probability that the card is an ace of spades.
    \item Calculate the probability that the card is (i) an ace and (ii) black card.
\end{enumerate}
\solution
%\input{exemplar/11/16/3/4/main2.tex}
\item The probability that a non leap year selected at random will contain 53 sundays.
\\
\solution
%\input{exemplar/10/13/1/19/main.tex}
\item One of the four persons John, Rita, Aslam or Gurpreet will be promoted next
month. Consequently the sample space consists of four elementary outcomes
S = {John promoted, Rita promoted, Aslam promoted, Gurpreet promoted}
You are told that the chances of John’s promotion is same as that of Gurpreet,
Rita’s chances of promotion are twice as likely as Johns. Aslam’s chances are
four times that of John.
\begin{enumerate}
	\item Determine
	\begin{enumerate}
		\item P (John promoted)
		\item P (Rita promoted)
		\item P (Aslam promoted)
		\item P (Gurpreet promoted)
	\end{enumerate}
	\item If A = {John promoted or Gurpreet promoted}, find P (A).
\end{enumerate}
\solution
%\input{exemplar/11/16/3/10/main.tex}
\item A card is drawn from a deck of 52 cards. Find the probability of getting a king or a heart or a red card.\\
\solution
%\input{exemplar/11/16/3/15/main.tex}
\item The probability that a student will pass his examination is 0.73, the probability of
the student getting a compartment is 0.13, and the probability that the student will
either pass or get compartment is 0.96. State True or False.\\
\solution
%\input{exemplar/11/16/3/31/main.tex}
\item A card is selected from a pack of 52 cards\\
\begin{enumerate}[label=(\alph*)]
\item How many points are there in the sample space?
\item Calculate the probability that the cards is an ace of spades.
\item Calculate the probability that the card is (i) an ace (ii)black card.\\
\end{enumerate}
%\input{ncert/11/16/3/4_1/Prob_4.tex}
\item In a non-leap year, the probability of having 53 tuesdays or 53 wednesdays is\\
\solution
%\input{exemplar/11/16/3/18/main.tex}
\item There are 1000 sealed envelopes in a box, 10 of them contain a cash prize of
Rs 100 each, 100 of them contain a cash prize of Rs 50 each and 200 of them
contain a cash prize of Rs 10 each and rest do not contain any cash prize. If they
are well shuffled and an envelope is picked up out, what is the probability that it
contains no cash prize?\\
\solution
%\input{exemplar/10/13/3/34/main.tex}
\item 
A die is thrown and a card is selected at random from a deck of 52 playing cards. The probability of getting an even number on the die and a spade card.\\
\solution
%\input{exemplar/12/13/3/78/main.tex}
\item
If 4-digit numbers greater than 5,000 are randomly formed from the digits 0, 1, 3, 5, and 7, what is the probability of forming a number divisible by 5 when:
\begin{enumerate}
    \item The digits are repeated?
    \item The repetition of digits is not allowed?
\end{enumerate}
\solution
%\input{ncert/11/16/4/9/main.tex}
\item Consider the probability space $\brak{\Omega, \mathcal{G}, P}$ where $\Omega = [0,2]$ and $\mathcal{G} = \cbrak{\phi, \Omega, [0,1], (1,2]}$. Let $X$ and $Y$ be two functions on $\Omega$ defined as
\begin{align*}
    X(\omega) = 
    \begin{cases}
        1 & \text{if }\omega \in [0, 1]\\
        2 & \text{if }\omega \in (1, 2]
    \end{cases}
\end{align*}
and
\begin{align*}
    Y(\omega) = 
    \begin{cases}
        2 & \text{if }\omega \in [0, 1.5]\\
        3 & \text{if }\omega \in (1.5, 2].
    \end{cases}
\end{align*}
Then which one of the following statements is true?
\begin{enumerate}
    \item [(A)] $X$ is a random variable with respect to $\mathcal{G}$, but $Y$ is not a random variable with respect to $\mathcal{G}$.
    \item [(B)] $Y$ is a random variable with respect to $\mathcal{G}$, but $X$ is not a random variable with respect to $\mathcal{G}$.
    \item [(C)] Neither $X$ nor $Y$ is a random variable with respect to $\mathcal{G}$.
    \item [(D)] Both $X$ and $Y$ are random variables with respect to $\mathcal{G}$.
\end{enumerate} \hfill (GATE ST 2023)\\
\solution
%\input{gate/ST/2023/14/main.tex}
	\item  A die is loaded in such a way that each odd number is twice as likely to occur as
each even number. Find $P(G)$, where $G$ is the event that a number greater than
3 occurs on a single roll of the die.
\\
\solution
		%\input{exemplar/11/16/3/5/main.tex}
	\item All the jacks, queens and kings are removed from a deck of 52 playing cards. The remaining cards are well shuffled and then one card is drawn at random. Giving ace a value 1 similar value for other cards, find the probability that the card has a value 
		\begin{enumerate}
			\item 7
			\item greater than 7
			\item less than 7
		\end{enumerate}
		%\input{exemplar/10/13/3/30/main.tex}
  \item A Lot consists of 48 mobile phones of which 42 are good, 3 have only minor defects and 3 have major defects.Varnika will buy a phone if it is good but the trader will only buy a mobile if it has no major defects. One phone is selected at random from the lot. What is the probability that it is
\begin{enumerate}
	\item acceptable to Varnika?
            \item acceptable to the trader?
\end{enumerate}
\solution
	%\input{exemplar/10/13/3/40/main.tex}
 \item A student says that if you throw a die, it will show up 1 or not 1. Therefore, the probability of getting 1 and the probability of getting 'not 1' each is equal to $\frac{1}{2}$. Is this correct? Give reasons.\\
 \solution
        %\input{exemplar/10/13/2/9/main.tex}
   \item Four candidates A, B, C, D have ap-
plied for the assignment to coach a school cricket
team. If A is twice as likely to be selected as B, and
B and C are given about the same chance of being
selected, while C is twice as likely to be selected
as D, what are the probabilities that
\begin{enumerate}
\item C will be selected?
\item A will not be selected?
\end{enumerate}
	%\input{exemplar/11/16/3/9/main.tex}
 \item A bag contain 24 balls of which $x$ balls are red, $2x$ are white and $3x$ are blue. A ball is selected at random, What is the probability that it is
\begin{enumerate}[label=\alph*)]
\item not red ?
\item white ?
\end{enumerate}
%\input{exemplar/10/13/3/41/main.tex}
If the letters of the word ASSASSINATION are arranged at random. Find the Probability that
\begin{enumerate}[label=(\alph*)]
\item Four $S's$ come consecutively in the word
\item Two  $I's$ and two $N's$ come together
\item All $A's$ are not coming together
\item No two $A's$ are coming together
\end{enumerate}
%\input{exemplar/11/16/3/14/main.tex}
	\item One urn contains two black balls (labelled B1 and B2) and one white ball. A
	second urn contains one black ball and two white balls (labelled W1 and W2).
	Suppose the following experiment is performed. One of the two urns is chosen
	at random. Next a ball is randomly chosen from the urn. Then a second ball is
	chosen at random from the same urn without replacing the first ball.
	
	\begin{enumerate}
	\item What is the probability that two black balls are chosen?
	
	\item What is the probability that two balls of opposite colour are chosen?
	\end{enumerate}
	\solution
	%\input{exemplar/11/16/3/12/main1.tex}
\end{enumerate}

	\item A bag contains $5$ red balls and some blue balls. If the probability of drawing a blue ball is double that if a red ball, determine the number of blue balls in the bag. 
		\\
\solution
		%\begin{enumerate}[label=\thesection.\arabic*,ref=\thesection.\theenumi]
	\item One card is drawn from a well-shuffled deck of 52 cards. Find the probability of getting
\begin{enumerate}
\item A king of red colour 
\item A face card 
\item A red face card
\item The jack of hearts
\item A spade
\item The queen of diamonds

\end{enumerate}
\solution
		%\input{ncert/10/15/1/14/main.tex}
	\item Five cards—the ten, jack, queen, king and ace of diamonds, are well-shuffled with their face downwards. One card is then picked up at random.
\begin{enumerate}
\item
What is the probability that the card is the queen? 
\item
If the queen is drawn and put aside, what is the probability that the second card picked up is (a) an ace? (b) a queen?\\
\end{enumerate}
\solution
		%\input{ncert/10/15/1/15/defs.tex}
	\item A bag contains $5$ red balls and some blue balls. If the probability of drawing a blue ball is double that if a red ball, determine the number of blue balls in the bag. 
		\\
\solution
		%\input{ncert/10/15/2/3/defs.tex}
	\item A card is selected from a pack of 52 cards.
 \begin{enumerate}[label=(\alph*)] 
                 \item How many points are there in the sample space?
                 \item Calculate the probability that the card is an ace of spades.
                 \item Calculate the probability that the card is (i) an ace and (ii) black card.
 \end{enumerate}
\solution
		%\input{ncert/11/16/3/4/main.tex}
\item Four cards are drawn from a well-shuffled deck of 52 cards. What is the probability of obtaining 3 diamonds and one spade.
\\
\solution
		%\input{ncert/11/16/4/2/defs.tex}
\item In a certain lottery 10,000 tickets are sold and ten equal prizes are awarded. What is the probability of not getting a prize if you buy (a) one ticket (b) two tickets (c) 10 tickets ?	
\\
\solution
		%\input{ncert/11/16/4/4/defs.tex}
		%
\item 
Out of 100 students, two sections of 40 and 60 are formed. If you and your friend are among the 100 students, what is the probability that
\begin{enumerate}
\item you both enter the same section?
\item you both enter the different sections?
\end{enumerate}
\solution
		%\input{ncert/11/16/4/5/defs.tex}
	\item 
The number lock of a suitcase has 4 wheels each labelled with ten digits i.e. from 0 to 9.The lock opens with a sequence of four digits with no repeats.What is the probability of a person getting the right sequence to open the suitcase.
\\
\solution
		%\input{ncert/11/16/4/10/defs.tex}
		%
\item 
Two cards are drawn at random and without replacement from a pack of 52 playing cards. Find the probability that both the cards are black.
\\
\solution
		%\input{ncert/12/13/2/2/defs.tex}
		\item A box of oranges is inspected by examining three randomly selected oranges drawn without replacement. If all the three oranges are good, the box is approved for sale, otherwise, it is rejected. Find the probability that a box containing 15 oranges out of which 12 are good and 3 are bad ones will be approved for sale.
		\label{ncert/12/13/2/3/defs.tex}
		\item Two balls are drawn at random with replacement from a box containing 10 black and 8 red balls. Find the probability that
		\label{ncert/12/13/2/12}
\begin{enumerate}
\item both balls are red.
\item first ball is black and second is red.
\item one of them is black and other is red.
\end{enumerate}

\item In a hostel, 60\% of the students read Hindi newspaper, 40\% read English newspaper and 20\% read both Hindi and English newspapers. A student is selected at random.
		\label{ncert/12/13/2/15}
\begin{enumerate}
\item Find the probability that she reads neither Hindi nor English newspapers.
\item If she reads Hindi newspaper, find the probability that she reads English newspaper.
\item If she reads English newspaper, find the probability that she reads Hindi newspaper.\\
\end{enumerate}
\item The probability of obtaining an even prime number on each die, when a pair of dice is rolled is 
\begin{enumerate}
    \item $0$ 
    
    \item $\frac{1}{3}$ 
    
    \item $\frac{1}{12}$ 
    
    \item $\frac{1}{36}$ 
\end{enumerate}
\solution
		%\input{ncert/12/13/2/17/defs.tex}
	\item A bag contains 4 red and 4 black balls, another bag contains 2 red and 6 black balls. One of the two bags is selected at random and a ball is drawn from the bag which is found to be red. Find the probability that the ball is drawn from the first bag.
\\
\solution
		%\input{ncert/12/13/3/2/main.tex}
  \item
  Cards with numbers 2 to 101 are placed in a box. A card is selected at random.Find the probability that the card has
\begin{enumerate}[label=(\roman*)]
	\item an even number 
	\item a square number
\end{enumerate}
\solution
%\input{exemplar/10/13/3/32/main.tex}
\item
The king, queen and jack of clubs are removed from a deck of 52 playing cards and then well shuffled. Now one card is drawn at random from the remaining cards.  Determine the probability that the card is
\begin{enumerate}[label=(\roman*)]
\item a club
\item 10 of hearts
\end{enumerate}
\solution
%\input{exemplar/10/13/3/29/main.tex}
\item A team of medical students doing their internship have to assist during surgeries
at a city hospital. The probabilities of surgeries rated as very complex, complex,
routine, simple or very simple are respectively, 0.15, 0.20, 0.31, 0.26, .08. Find
the probabilities that a particular surgery will be rated
\begin{enumerate}
	\item complex or very complex;
	\item neither very complex nor very simple;
	\item routine or complex
	\item routine or simple
\end{enumerate}
\solution
%\input{exemplar/11/16/3/8(1)/main.tex}
\item A card is selected from a pack of 52 cards.
\begin{enumerate}[label=(\alph*)]
    \item How many points are there in the sample space?
    \item Calculate the probability that the card is an ace of spades.
    \item Calculate the probability that the card is (i) an ace and (ii) black card.
\end{enumerate}
\solution
%\input{exemplar/11/16/3/4/main2.tex}
\item The probability that a non leap year selected at random will contain 53 sundays.
\\
\solution
%\input{exemplar/10/13/1/19/main.tex}
\item One of the four persons John, Rita, Aslam or Gurpreet will be promoted next
month. Consequently the sample space consists of four elementary outcomes
S = {John promoted, Rita promoted, Aslam promoted, Gurpreet promoted}
You are told that the chances of John’s promotion is same as that of Gurpreet,
Rita’s chances of promotion are twice as likely as Johns. Aslam’s chances are
four times that of John.
\begin{enumerate}
	\item Determine
	\begin{enumerate}
		\item P (John promoted)
		\item P (Rita promoted)
		\item P (Aslam promoted)
		\item P (Gurpreet promoted)
	\end{enumerate}
	\item If A = {John promoted or Gurpreet promoted}, find P (A).
\end{enumerate}
\solution
%\input{exemplar/11/16/3/10/main.tex}
\item A card is drawn from a deck of 52 cards. Find the probability of getting a king or a heart or a red card.\\
\solution
%\input{exemplar/11/16/3/15/main.tex}
\item The probability that a student will pass his examination is 0.73, the probability of
the student getting a compartment is 0.13, and the probability that the student will
either pass or get compartment is 0.96. State True or False.\\
\solution
%\input{exemplar/11/16/3/31/main.tex}
\item A card is selected from a pack of 52 cards\\
\begin{enumerate}[label=(\alph*)]
\item How many points are there in the sample space?
\item Calculate the probability that the cards is an ace of spades.
\item Calculate the probability that the card is (i) an ace (ii)black card.\\
\end{enumerate}
%\input{ncert/11/16/3/4_1/Prob_4.tex}
\item In a non-leap year, the probability of having 53 tuesdays or 53 wednesdays is\\
\solution
%\input{exemplar/11/16/3/18/main.tex}
\item There are 1000 sealed envelopes in a box, 10 of them contain a cash prize of
Rs 100 each, 100 of them contain a cash prize of Rs 50 each and 200 of them
contain a cash prize of Rs 10 each and rest do not contain any cash prize. If they
are well shuffled and an envelope is picked up out, what is the probability that it
contains no cash prize?\\
\solution
%\input{exemplar/10/13/3/34/main.tex}
\item 
A die is thrown and a card is selected at random from a deck of 52 playing cards. The probability of getting an even number on the die and a spade card.\\
\solution
%\input{exemplar/12/13/3/78/main.tex}
\item
If 4-digit numbers greater than 5,000 are randomly formed from the digits 0, 1, 3, 5, and 7, what is the probability of forming a number divisible by 5 when:
\begin{enumerate}
    \item The digits are repeated?
    \item The repetition of digits is not allowed?
\end{enumerate}
\solution
%\input{ncert/11/16/4/9/main.tex}
\item Consider the probability space $\brak{\Omega, \mathcal{G}, P}$ where $\Omega = [0,2]$ and $\mathcal{G} = \cbrak{\phi, \Omega, [0,1], (1,2]}$. Let $X$ and $Y$ be two functions on $\Omega$ defined as
\begin{align*}
    X(\omega) = 
    \begin{cases}
        1 & \text{if }\omega \in [0, 1]\\
        2 & \text{if }\omega \in (1, 2]
    \end{cases}
\end{align*}
and
\begin{align*}
    Y(\omega) = 
    \begin{cases}
        2 & \text{if }\omega \in [0, 1.5]\\
        3 & \text{if }\omega \in (1.5, 2].
    \end{cases}
\end{align*}
Then which one of the following statements is true?
\begin{enumerate}
    \item [(A)] $X$ is a random variable with respect to $\mathcal{G}$, but $Y$ is not a random variable with respect to $\mathcal{G}$.
    \item [(B)] $Y$ is a random variable with respect to $\mathcal{G}$, but $X$ is not a random variable with respect to $\mathcal{G}$.
    \item [(C)] Neither $X$ nor $Y$ is a random variable with respect to $\mathcal{G}$.
    \item [(D)] Both $X$ and $Y$ are random variables with respect to $\mathcal{G}$.
\end{enumerate} \hfill (GATE ST 2023)\\
\solution
%\input{gate/ST/2023/14/main.tex}
	\item  A die is loaded in such a way that each odd number is twice as likely to occur as
each even number. Find $P(G)$, where $G$ is the event that a number greater than
3 occurs on a single roll of the die.
\\
\solution
		%\input{exemplar/11/16/3/5/main.tex}
	\item All the jacks, queens and kings are removed from a deck of 52 playing cards. The remaining cards are well shuffled and then one card is drawn at random. Giving ace a value 1 similar value for other cards, find the probability that the card has a value 
		\begin{enumerate}
			\item 7
			\item greater than 7
			\item less than 7
		\end{enumerate}
		%\input{exemplar/10/13/3/30/main.tex}
  \item A Lot consists of 48 mobile phones of which 42 are good, 3 have only minor defects and 3 have major defects.Varnika will buy a phone if it is good but the trader will only buy a mobile if it has no major defects. One phone is selected at random from the lot. What is the probability that it is
\begin{enumerate}
	\item acceptable to Varnika?
            \item acceptable to the trader?
\end{enumerate}
\solution
	%\input{exemplar/10/13/3/40/main.tex}
 \item A student says that if you throw a die, it will show up 1 or not 1. Therefore, the probability of getting 1 and the probability of getting 'not 1' each is equal to $\frac{1}{2}$. Is this correct? Give reasons.\\
 \solution
        %\input{exemplar/10/13/2/9/main.tex}
   \item Four candidates A, B, C, D have ap-
plied for the assignment to coach a school cricket
team. If A is twice as likely to be selected as B, and
B and C are given about the same chance of being
selected, while C is twice as likely to be selected
as D, what are the probabilities that
\begin{enumerate}
\item C will be selected?
\item A will not be selected?
\end{enumerate}
	%\input{exemplar/11/16/3/9/main.tex}
 \item A bag contain 24 balls of which $x$ balls are red, $2x$ are white and $3x$ are blue. A ball is selected at random, What is the probability that it is
\begin{enumerate}[label=\alph*)]
\item not red ?
\item white ?
\end{enumerate}
%\input{exemplar/10/13/3/41/main.tex}
If the letters of the word ASSASSINATION are arranged at random. Find the Probability that
\begin{enumerate}[label=(\alph*)]
\item Four $S's$ come consecutively in the word
\item Two  $I's$ and two $N's$ come together
\item All $A's$ are not coming together
\item No two $A's$ are coming together
\end{enumerate}
%\input{exemplar/11/16/3/14/main.tex}
	\item One urn contains two black balls (labelled B1 and B2) and one white ball. A
	second urn contains one black ball and two white balls (labelled W1 and W2).
	Suppose the following experiment is performed. One of the two urns is chosen
	at random. Next a ball is randomly chosen from the urn. Then a second ball is
	chosen at random from the same urn without replacing the first ball.
	
	\begin{enumerate}
	\item What is the probability that two black balls are chosen?
	
	\item What is the probability that two balls of opposite colour are chosen?
	\end{enumerate}
	\solution
	%\input{exemplar/11/16/3/12/main1.tex}
\end{enumerate}

	\item A card is selected from a pack of 52 cards.
 \begin{enumerate}[label=(\alph*)] 
                 \item How many points are there in the sample space?
                 \item Calculate the probability that the card is an ace of spades.
                 \item Calculate the probability that the card is (i) an ace and (ii) black card.
 \end{enumerate}
\solution
		%\begin{table}[H]
	\centering
\begin{tabular}{|c|c|c|}
\hline
Random variable &Value &Definition\\ \hline
\multirow{3}{*}{X} &0 &Slips of Rs 1\\
&1 &Slips of Rs 5\\
&2 &Slips of Rs 13\\ \hline
\multirow{2}{*}{Y} &0 &Box A\\
&1 &Box B\\\hline
\end{tabular}
\caption{}
\label{tab:Distribution}
\end{table}
See \tabref{tab:Distribution}.
\begin{align}
p_{Y}\brak{k}= \begin{cases} 
      \frac{1}{3} & {k=0} \\
      \frac{2}{3 }& {k=1} 
   \end{cases}
   \\
p_{Y|X}\brak{0|0} = \frac{19}{25}\, 
p_{Y|X}\brak{0|1} = \frac{6}{25}\,
p_{Y|X}\brak{1|0} = \frac{45}{50}\,
p_{Y|X}\brak{1|2} = \frac{5}{50}
\end{align}
The desired probability is the probability that a slip drawn at random is marked other than Rs 1,
\begin{align}
&=1-p_X\brak{0}\\
&= p_X(1) + p_X(2)
\end{align}
Using Bayes theorem,
\begin{align}
&= p_Y\brak{0} \times \pr{Y=0 | X=1} + p_Y\brak{1} \times \pr{Y=1|X=2}\\
&=\frac{1}{3} \times \frac{6}{25} + \frac{2}{3} \times \frac{5}{50}\\
&=\frac{11}{75}
\end{align}

\newpage

%\tableofcontents

\bigskip

\renewcommand{\thefigure}{\theenumi}
\renewcommand{\thetable}{\theenumi}
%\renewcommand{\theequation}{\theenumi}

%\begin{abstract}
%%\boldmath
%In this letter, an algorithm for evaluating the exact analytical bit error rate  (BER)  for the piecewise linear (PL) combiner for  multiple relays is presented. Previous results were available only for upto three relays. The algorithm is unique in the sense that  the actual mathematical expressions, that are prohibitively large, need not be explicitly obtained. The diversity gain due to multiple relays is shown through plots of the analytical BER, well supported by simulations. 
%
%\end{abstract}
% IEEEtran.cls defaults to using nonbold math in the Abstract.
% This preserves the distinction between vectors and scalars. However,
% if the journal you are submitting to favors bold math in the abstract,
% then you can use LaTeX's standard command \boldmath at the very start
% of the abstract to achieve this. Many IEEE journals frown on math
% in the abstract anyway.

% Note that keywords are not normally used for peerreview papers.
%\begin{IEEEkeywords}
%Cooperative diversity, decode and forward, piecewise linear
%\end{IEEEkeywords}



% For peer review papers, you can put extra information on the cover
% page as needed:
% \ifCLASSOPTIONpeerreview
% \begin{center} \bfseries EDICS Category: 3-BBND \end{center}
% \fi
%
% For peerreview papers, this IEEEtran command inserts a page break and
% creates the second title. It will be ignored for other modes.
%\IEEEpeerreviewmaketitle




\item Four cards are drawn from a well-shuffled deck of 52 cards. What is the probability of obtaining 3 diamonds and one spade.
\\
\solution
		%\begin{enumerate}[label=\thesection.\arabic*,ref=\thesection.\theenumi]
	\item One card is drawn from a well-shuffled deck of 52 cards. Find the probability of getting
\begin{enumerate}
\item A king of red colour 
\item A face card 
\item A red face card
\item The jack of hearts
\item A spade
\item The queen of diamonds

\end{enumerate}
\solution
		%\input{ncert/10/15/1/14/main.tex}
	\item Five cards—the ten, jack, queen, king and ace of diamonds, are well-shuffled with their face downwards. One card is then picked up at random.
\begin{enumerate}
\item
What is the probability that the card is the queen? 
\item
If the queen is drawn and put aside, what is the probability that the second card picked up is (a) an ace? (b) a queen?\\
\end{enumerate}
\solution
		%\input{ncert/10/15/1/15/defs.tex}
	\item A bag contains $5$ red balls and some blue balls. If the probability of drawing a blue ball is double that if a red ball, determine the number of blue balls in the bag. 
		\\
\solution
		%\input{ncert/10/15/2/3/defs.tex}
	\item A card is selected from a pack of 52 cards.
 \begin{enumerate}[label=(\alph*)] 
                 \item How many points are there in the sample space?
                 \item Calculate the probability that the card is an ace of spades.
                 \item Calculate the probability that the card is (i) an ace and (ii) black card.
 \end{enumerate}
\solution
		%\input{ncert/11/16/3/4/main.tex}
\item Four cards are drawn from a well-shuffled deck of 52 cards. What is the probability of obtaining 3 diamonds and one spade.
\\
\solution
		%\input{ncert/11/16/4/2/defs.tex}
\item In a certain lottery 10,000 tickets are sold and ten equal prizes are awarded. What is the probability of not getting a prize if you buy (a) one ticket (b) two tickets (c) 10 tickets ?	
\\
\solution
		%\input{ncert/11/16/4/4/defs.tex}
		%
\item 
Out of 100 students, two sections of 40 and 60 are formed. If you and your friend are among the 100 students, what is the probability that
\begin{enumerate}
\item you both enter the same section?
\item you both enter the different sections?
\end{enumerate}
\solution
		%\input{ncert/11/16/4/5/defs.tex}
	\item 
The number lock of a suitcase has 4 wheels each labelled with ten digits i.e. from 0 to 9.The lock opens with a sequence of four digits with no repeats.What is the probability of a person getting the right sequence to open the suitcase.
\\
\solution
		%\input{ncert/11/16/4/10/defs.tex}
		%
\item 
Two cards are drawn at random and without replacement from a pack of 52 playing cards. Find the probability that both the cards are black.
\\
\solution
		%\input{ncert/12/13/2/2/defs.tex}
		\item A box of oranges is inspected by examining three randomly selected oranges drawn without replacement. If all the three oranges are good, the box is approved for sale, otherwise, it is rejected. Find the probability that a box containing 15 oranges out of which 12 are good and 3 are bad ones will be approved for sale.
		\label{ncert/12/13/2/3/defs.tex}
		\item Two balls are drawn at random with replacement from a box containing 10 black and 8 red balls. Find the probability that
		\label{ncert/12/13/2/12}
\begin{enumerate}
\item both balls are red.
\item first ball is black and second is red.
\item one of them is black and other is red.
\end{enumerate}

\item In a hostel, 60\% of the students read Hindi newspaper, 40\% read English newspaper and 20\% read both Hindi and English newspapers. A student is selected at random.
		\label{ncert/12/13/2/15}
\begin{enumerate}
\item Find the probability that she reads neither Hindi nor English newspapers.
\item If she reads Hindi newspaper, find the probability that she reads English newspaper.
\item If she reads English newspaper, find the probability that she reads Hindi newspaper.\\
\end{enumerate}
\item The probability of obtaining an even prime number on each die, when a pair of dice is rolled is 
\begin{enumerate}
    \item $0$ 
    
    \item $\frac{1}{3}$ 
    
    \item $\frac{1}{12}$ 
    
    \item $\frac{1}{36}$ 
\end{enumerate}
\solution
		%\input{ncert/12/13/2/17/defs.tex}
	\item A bag contains 4 red and 4 black balls, another bag contains 2 red and 6 black balls. One of the two bags is selected at random and a ball is drawn from the bag which is found to be red. Find the probability that the ball is drawn from the first bag.
\\
\solution
		%\input{ncert/12/13/3/2/main.tex}
  \item
  Cards with numbers 2 to 101 are placed in a box. A card is selected at random.Find the probability that the card has
\begin{enumerate}[label=(\roman*)]
	\item an even number 
	\item a square number
\end{enumerate}
\solution
%\input{exemplar/10/13/3/32/main.tex}
\item
The king, queen and jack of clubs are removed from a deck of 52 playing cards and then well shuffled. Now one card is drawn at random from the remaining cards.  Determine the probability that the card is
\begin{enumerate}[label=(\roman*)]
\item a club
\item 10 of hearts
\end{enumerate}
\solution
%\input{exemplar/10/13/3/29/main.tex}
\item A team of medical students doing their internship have to assist during surgeries
at a city hospital. The probabilities of surgeries rated as very complex, complex,
routine, simple or very simple are respectively, 0.15, 0.20, 0.31, 0.26, .08. Find
the probabilities that a particular surgery will be rated
\begin{enumerate}
	\item complex or very complex;
	\item neither very complex nor very simple;
	\item routine or complex
	\item routine or simple
\end{enumerate}
\solution
%\input{exemplar/11/16/3/8(1)/main.tex}
\item A card is selected from a pack of 52 cards.
\begin{enumerate}[label=(\alph*)]
    \item How many points are there in the sample space?
    \item Calculate the probability that the card is an ace of spades.
    \item Calculate the probability that the card is (i) an ace and (ii) black card.
\end{enumerate}
\solution
%\input{exemplar/11/16/3/4/main2.tex}
\item The probability that a non leap year selected at random will contain 53 sundays.
\\
\solution
%\input{exemplar/10/13/1/19/main.tex}
\item One of the four persons John, Rita, Aslam or Gurpreet will be promoted next
month. Consequently the sample space consists of four elementary outcomes
S = {John promoted, Rita promoted, Aslam promoted, Gurpreet promoted}
You are told that the chances of John’s promotion is same as that of Gurpreet,
Rita’s chances of promotion are twice as likely as Johns. Aslam’s chances are
four times that of John.
\begin{enumerate}
	\item Determine
	\begin{enumerate}
		\item P (John promoted)
		\item P (Rita promoted)
		\item P (Aslam promoted)
		\item P (Gurpreet promoted)
	\end{enumerate}
	\item If A = {John promoted or Gurpreet promoted}, find P (A).
\end{enumerate}
\solution
%\input{exemplar/11/16/3/10/main.tex}
\item A card is drawn from a deck of 52 cards. Find the probability of getting a king or a heart or a red card.\\
\solution
%\input{exemplar/11/16/3/15/main.tex}
\item The probability that a student will pass his examination is 0.73, the probability of
the student getting a compartment is 0.13, and the probability that the student will
either pass or get compartment is 0.96. State True or False.\\
\solution
%\input{exemplar/11/16/3/31/main.tex}
\item A card is selected from a pack of 52 cards\\
\begin{enumerate}[label=(\alph*)]
\item How many points are there in the sample space?
\item Calculate the probability that the cards is an ace of spades.
\item Calculate the probability that the card is (i) an ace (ii)black card.\\
\end{enumerate}
%\input{ncert/11/16/3/4_1/Prob_4.tex}
\item In a non-leap year, the probability of having 53 tuesdays or 53 wednesdays is\\
\solution
%\input{exemplar/11/16/3/18/main.tex}
\item There are 1000 sealed envelopes in a box, 10 of them contain a cash prize of
Rs 100 each, 100 of them contain a cash prize of Rs 50 each and 200 of them
contain a cash prize of Rs 10 each and rest do not contain any cash prize. If they
are well shuffled and an envelope is picked up out, what is the probability that it
contains no cash prize?\\
\solution
%\input{exemplar/10/13/3/34/main.tex}
\item 
A die is thrown and a card is selected at random from a deck of 52 playing cards. The probability of getting an even number on the die and a spade card.\\
\solution
%\input{exemplar/12/13/3/78/main.tex}
\item
If 4-digit numbers greater than 5,000 are randomly formed from the digits 0, 1, 3, 5, and 7, what is the probability of forming a number divisible by 5 when:
\begin{enumerate}
    \item The digits are repeated?
    \item The repetition of digits is not allowed?
\end{enumerate}
\solution
%\input{ncert/11/16/4/9/main.tex}
\item Consider the probability space $\brak{\Omega, \mathcal{G}, P}$ where $\Omega = [0,2]$ and $\mathcal{G} = \cbrak{\phi, \Omega, [0,1], (1,2]}$. Let $X$ and $Y$ be two functions on $\Omega$ defined as
\begin{align*}
    X(\omega) = 
    \begin{cases}
        1 & \text{if }\omega \in [0, 1]\\
        2 & \text{if }\omega \in (1, 2]
    \end{cases}
\end{align*}
and
\begin{align*}
    Y(\omega) = 
    \begin{cases}
        2 & \text{if }\omega \in [0, 1.5]\\
        3 & \text{if }\omega \in (1.5, 2].
    \end{cases}
\end{align*}
Then which one of the following statements is true?
\begin{enumerate}
    \item [(A)] $X$ is a random variable with respect to $\mathcal{G}$, but $Y$ is not a random variable with respect to $\mathcal{G}$.
    \item [(B)] $Y$ is a random variable with respect to $\mathcal{G}$, but $X$ is not a random variable with respect to $\mathcal{G}$.
    \item [(C)] Neither $X$ nor $Y$ is a random variable with respect to $\mathcal{G}$.
    \item [(D)] Both $X$ and $Y$ are random variables with respect to $\mathcal{G}$.
\end{enumerate} \hfill (GATE ST 2023)\\
\solution
%\input{gate/ST/2023/14/main.tex}
	\item  A die is loaded in such a way that each odd number is twice as likely to occur as
each even number. Find $P(G)$, where $G$ is the event that a number greater than
3 occurs on a single roll of the die.
\\
\solution
		%\input{exemplar/11/16/3/5/main.tex}
	\item All the jacks, queens and kings are removed from a deck of 52 playing cards. The remaining cards are well shuffled and then one card is drawn at random. Giving ace a value 1 similar value for other cards, find the probability that the card has a value 
		\begin{enumerate}
			\item 7
			\item greater than 7
			\item less than 7
		\end{enumerate}
		%\input{exemplar/10/13/3/30/main.tex}
  \item A Lot consists of 48 mobile phones of which 42 are good, 3 have only minor defects and 3 have major defects.Varnika will buy a phone if it is good but the trader will only buy a mobile if it has no major defects. One phone is selected at random from the lot. What is the probability that it is
\begin{enumerate}
	\item acceptable to Varnika?
            \item acceptable to the trader?
\end{enumerate}
\solution
	%\input{exemplar/10/13/3/40/main.tex}
 \item A student says that if you throw a die, it will show up 1 or not 1. Therefore, the probability of getting 1 and the probability of getting 'not 1' each is equal to $\frac{1}{2}$. Is this correct? Give reasons.\\
 \solution
        %\input{exemplar/10/13/2/9/main.tex}
   \item Four candidates A, B, C, D have ap-
plied for the assignment to coach a school cricket
team. If A is twice as likely to be selected as B, and
B and C are given about the same chance of being
selected, while C is twice as likely to be selected
as D, what are the probabilities that
\begin{enumerate}
\item C will be selected?
\item A will not be selected?
\end{enumerate}
	%\input{exemplar/11/16/3/9/main.tex}
 \item A bag contain 24 balls of which $x$ balls are red, $2x$ are white and $3x$ are blue. A ball is selected at random, What is the probability that it is
\begin{enumerate}[label=\alph*)]
\item not red ?
\item white ?
\end{enumerate}
%\input{exemplar/10/13/3/41/main.tex}
If the letters of the word ASSASSINATION are arranged at random. Find the Probability that
\begin{enumerate}[label=(\alph*)]
\item Four $S's$ come consecutively in the word
\item Two  $I's$ and two $N's$ come together
\item All $A's$ are not coming together
\item No two $A's$ are coming together
\end{enumerate}
%\input{exemplar/11/16/3/14/main.tex}
	\item One urn contains two black balls (labelled B1 and B2) and one white ball. A
	second urn contains one black ball and two white balls (labelled W1 and W2).
	Suppose the following experiment is performed. One of the two urns is chosen
	at random. Next a ball is randomly chosen from the urn. Then a second ball is
	chosen at random from the same urn without replacing the first ball.
	
	\begin{enumerate}
	\item What is the probability that two black balls are chosen?
	
	\item What is the probability that two balls of opposite colour are chosen?
	\end{enumerate}
	\solution
	%\input{exemplar/11/16/3/12/main1.tex}
\end{enumerate}

\item In a certain lottery 10,000 tickets are sold and ten equal prizes are awarded. What is the probability of not getting a prize if you buy (a) one ticket (b) two tickets (c) 10 tickets ?	
\\
\solution
		%\begin{enumerate}[label=\thesection.\arabic*,ref=\thesection.\theenumi]
	\item One card is drawn from a well-shuffled deck of 52 cards. Find the probability of getting
\begin{enumerate}
\item A king of red colour 
\item A face card 
\item A red face card
\item The jack of hearts
\item A spade
\item The queen of diamonds

\end{enumerate}
\solution
		%\input{ncert/10/15/1/14/main.tex}
	\item Five cards—the ten, jack, queen, king and ace of diamonds, are well-shuffled with their face downwards. One card is then picked up at random.
\begin{enumerate}
\item
What is the probability that the card is the queen? 
\item
If the queen is drawn and put aside, what is the probability that the second card picked up is (a) an ace? (b) a queen?\\
\end{enumerate}
\solution
		%\input{ncert/10/15/1/15/defs.tex}
	\item A bag contains $5$ red balls and some blue balls. If the probability of drawing a blue ball is double that if a red ball, determine the number of blue balls in the bag. 
		\\
\solution
		%\input{ncert/10/15/2/3/defs.tex}
	\item A card is selected from a pack of 52 cards.
 \begin{enumerate}[label=(\alph*)] 
                 \item How many points are there in the sample space?
                 \item Calculate the probability that the card is an ace of spades.
                 \item Calculate the probability that the card is (i) an ace and (ii) black card.
 \end{enumerate}
\solution
		%\input{ncert/11/16/3/4/main.tex}
\item Four cards are drawn from a well-shuffled deck of 52 cards. What is the probability of obtaining 3 diamonds and one spade.
\\
\solution
		%\input{ncert/11/16/4/2/defs.tex}
\item In a certain lottery 10,000 tickets are sold and ten equal prizes are awarded. What is the probability of not getting a prize if you buy (a) one ticket (b) two tickets (c) 10 tickets ?	
\\
\solution
		%\input{ncert/11/16/4/4/defs.tex}
		%
\item 
Out of 100 students, two sections of 40 and 60 are formed. If you and your friend are among the 100 students, what is the probability that
\begin{enumerate}
\item you both enter the same section?
\item you both enter the different sections?
\end{enumerate}
\solution
		%\input{ncert/11/16/4/5/defs.tex}
	\item 
The number lock of a suitcase has 4 wheels each labelled with ten digits i.e. from 0 to 9.The lock opens with a sequence of four digits with no repeats.What is the probability of a person getting the right sequence to open the suitcase.
\\
\solution
		%\input{ncert/11/16/4/10/defs.tex}
		%
\item 
Two cards are drawn at random and without replacement from a pack of 52 playing cards. Find the probability that both the cards are black.
\\
\solution
		%\input{ncert/12/13/2/2/defs.tex}
		\item A box of oranges is inspected by examining three randomly selected oranges drawn without replacement. If all the three oranges are good, the box is approved for sale, otherwise, it is rejected. Find the probability that a box containing 15 oranges out of which 12 are good and 3 are bad ones will be approved for sale.
		\label{ncert/12/13/2/3/defs.tex}
		\item Two balls are drawn at random with replacement from a box containing 10 black and 8 red balls. Find the probability that
		\label{ncert/12/13/2/12}
\begin{enumerate}
\item both balls are red.
\item first ball is black and second is red.
\item one of them is black and other is red.
\end{enumerate}

\item In a hostel, 60\% of the students read Hindi newspaper, 40\% read English newspaper and 20\% read both Hindi and English newspapers. A student is selected at random.
		\label{ncert/12/13/2/15}
\begin{enumerate}
\item Find the probability that she reads neither Hindi nor English newspapers.
\item If she reads Hindi newspaper, find the probability that she reads English newspaper.
\item If she reads English newspaper, find the probability that she reads Hindi newspaper.\\
\end{enumerate}
\item The probability of obtaining an even prime number on each die, when a pair of dice is rolled is 
\begin{enumerate}
    \item $0$ 
    
    \item $\frac{1}{3}$ 
    
    \item $\frac{1}{12}$ 
    
    \item $\frac{1}{36}$ 
\end{enumerate}
\solution
		%\input{ncert/12/13/2/17/defs.tex}
	\item A bag contains 4 red and 4 black balls, another bag contains 2 red and 6 black balls. One of the two bags is selected at random and a ball is drawn from the bag which is found to be red. Find the probability that the ball is drawn from the first bag.
\\
\solution
		%\input{ncert/12/13/3/2/main.tex}
  \item
  Cards with numbers 2 to 101 are placed in a box. A card is selected at random.Find the probability that the card has
\begin{enumerate}[label=(\roman*)]
	\item an even number 
	\item a square number
\end{enumerate}
\solution
%\input{exemplar/10/13/3/32/main.tex}
\item
The king, queen and jack of clubs are removed from a deck of 52 playing cards and then well shuffled. Now one card is drawn at random from the remaining cards.  Determine the probability that the card is
\begin{enumerate}[label=(\roman*)]
\item a club
\item 10 of hearts
\end{enumerate}
\solution
%\input{exemplar/10/13/3/29/main.tex}
\item A team of medical students doing their internship have to assist during surgeries
at a city hospital. The probabilities of surgeries rated as very complex, complex,
routine, simple or very simple are respectively, 0.15, 0.20, 0.31, 0.26, .08. Find
the probabilities that a particular surgery will be rated
\begin{enumerate}
	\item complex or very complex;
	\item neither very complex nor very simple;
	\item routine or complex
	\item routine or simple
\end{enumerate}
\solution
%\input{exemplar/11/16/3/8(1)/main.tex}
\item A card is selected from a pack of 52 cards.
\begin{enumerate}[label=(\alph*)]
    \item How many points are there in the sample space?
    \item Calculate the probability that the card is an ace of spades.
    \item Calculate the probability that the card is (i) an ace and (ii) black card.
\end{enumerate}
\solution
%\input{exemplar/11/16/3/4/main2.tex}
\item The probability that a non leap year selected at random will contain 53 sundays.
\\
\solution
%\input{exemplar/10/13/1/19/main.tex}
\item One of the four persons John, Rita, Aslam or Gurpreet will be promoted next
month. Consequently the sample space consists of four elementary outcomes
S = {John promoted, Rita promoted, Aslam promoted, Gurpreet promoted}
You are told that the chances of John’s promotion is same as that of Gurpreet,
Rita’s chances of promotion are twice as likely as Johns. Aslam’s chances are
four times that of John.
\begin{enumerate}
	\item Determine
	\begin{enumerate}
		\item P (John promoted)
		\item P (Rita promoted)
		\item P (Aslam promoted)
		\item P (Gurpreet promoted)
	\end{enumerate}
	\item If A = {John promoted or Gurpreet promoted}, find P (A).
\end{enumerate}
\solution
%\input{exemplar/11/16/3/10/main.tex}
\item A card is drawn from a deck of 52 cards. Find the probability of getting a king or a heart or a red card.\\
\solution
%\input{exemplar/11/16/3/15/main.tex}
\item The probability that a student will pass his examination is 0.73, the probability of
the student getting a compartment is 0.13, and the probability that the student will
either pass or get compartment is 0.96. State True or False.\\
\solution
%\input{exemplar/11/16/3/31/main.tex}
\item A card is selected from a pack of 52 cards\\
\begin{enumerate}[label=(\alph*)]
\item How many points are there in the sample space?
\item Calculate the probability that the cards is an ace of spades.
\item Calculate the probability that the card is (i) an ace (ii)black card.\\
\end{enumerate}
%\input{ncert/11/16/3/4_1/Prob_4.tex}
\item In a non-leap year, the probability of having 53 tuesdays or 53 wednesdays is\\
\solution
%\input{exemplar/11/16/3/18/main.tex}
\item There are 1000 sealed envelopes in a box, 10 of them contain a cash prize of
Rs 100 each, 100 of them contain a cash prize of Rs 50 each and 200 of them
contain a cash prize of Rs 10 each and rest do not contain any cash prize. If they
are well shuffled and an envelope is picked up out, what is the probability that it
contains no cash prize?\\
\solution
%\input{exemplar/10/13/3/34/main.tex}
\item 
A die is thrown and a card is selected at random from a deck of 52 playing cards. The probability of getting an even number on the die and a spade card.\\
\solution
%\input{exemplar/12/13/3/78/main.tex}
\item
If 4-digit numbers greater than 5,000 are randomly formed from the digits 0, 1, 3, 5, and 7, what is the probability of forming a number divisible by 5 when:
\begin{enumerate}
    \item The digits are repeated?
    \item The repetition of digits is not allowed?
\end{enumerate}
\solution
%\input{ncert/11/16/4/9/main.tex}
\item Consider the probability space $\brak{\Omega, \mathcal{G}, P}$ where $\Omega = [0,2]$ and $\mathcal{G} = \cbrak{\phi, \Omega, [0,1], (1,2]}$. Let $X$ and $Y$ be two functions on $\Omega$ defined as
\begin{align*}
    X(\omega) = 
    \begin{cases}
        1 & \text{if }\omega \in [0, 1]\\
        2 & \text{if }\omega \in (1, 2]
    \end{cases}
\end{align*}
and
\begin{align*}
    Y(\omega) = 
    \begin{cases}
        2 & \text{if }\omega \in [0, 1.5]\\
        3 & \text{if }\omega \in (1.5, 2].
    \end{cases}
\end{align*}
Then which one of the following statements is true?
\begin{enumerate}
    \item [(A)] $X$ is a random variable with respect to $\mathcal{G}$, but $Y$ is not a random variable with respect to $\mathcal{G}$.
    \item [(B)] $Y$ is a random variable with respect to $\mathcal{G}$, but $X$ is not a random variable with respect to $\mathcal{G}$.
    \item [(C)] Neither $X$ nor $Y$ is a random variable with respect to $\mathcal{G}$.
    \item [(D)] Both $X$ and $Y$ are random variables with respect to $\mathcal{G}$.
\end{enumerate} \hfill (GATE ST 2023)\\
\solution
%\input{gate/ST/2023/14/main.tex}
	\item  A die is loaded in such a way that each odd number is twice as likely to occur as
each even number. Find $P(G)$, where $G$ is the event that a number greater than
3 occurs on a single roll of the die.
\\
\solution
		%\input{exemplar/11/16/3/5/main.tex}
	\item All the jacks, queens and kings are removed from a deck of 52 playing cards. The remaining cards are well shuffled and then one card is drawn at random. Giving ace a value 1 similar value for other cards, find the probability that the card has a value 
		\begin{enumerate}
			\item 7
			\item greater than 7
			\item less than 7
		\end{enumerate}
		%\input{exemplar/10/13/3/30/main.tex}
  \item A Lot consists of 48 mobile phones of which 42 are good, 3 have only minor defects and 3 have major defects.Varnika will buy a phone if it is good but the trader will only buy a mobile if it has no major defects. One phone is selected at random from the lot. What is the probability that it is
\begin{enumerate}
	\item acceptable to Varnika?
            \item acceptable to the trader?
\end{enumerate}
\solution
	%\input{exemplar/10/13/3/40/main.tex}
 \item A student says that if you throw a die, it will show up 1 or not 1. Therefore, the probability of getting 1 and the probability of getting 'not 1' each is equal to $\frac{1}{2}$. Is this correct? Give reasons.\\
 \solution
        %\input{exemplar/10/13/2/9/main.tex}
   \item Four candidates A, B, C, D have ap-
plied for the assignment to coach a school cricket
team. If A is twice as likely to be selected as B, and
B and C are given about the same chance of being
selected, while C is twice as likely to be selected
as D, what are the probabilities that
\begin{enumerate}
\item C will be selected?
\item A will not be selected?
\end{enumerate}
	%\input{exemplar/11/16/3/9/main.tex}
 \item A bag contain 24 balls of which $x$ balls are red, $2x$ are white and $3x$ are blue. A ball is selected at random, What is the probability that it is
\begin{enumerate}[label=\alph*)]
\item not red ?
\item white ?
\end{enumerate}
%\input{exemplar/10/13/3/41/main.tex}
If the letters of the word ASSASSINATION are arranged at random. Find the Probability that
\begin{enumerate}[label=(\alph*)]
\item Four $S's$ come consecutively in the word
\item Two  $I's$ and two $N's$ come together
\item All $A's$ are not coming together
\item No two $A's$ are coming together
\end{enumerate}
%\input{exemplar/11/16/3/14/main.tex}
	\item One urn contains two black balls (labelled B1 and B2) and one white ball. A
	second urn contains one black ball and two white balls (labelled W1 and W2).
	Suppose the following experiment is performed. One of the two urns is chosen
	at random. Next a ball is randomly chosen from the urn. Then a second ball is
	chosen at random from the same urn without replacing the first ball.
	
	\begin{enumerate}
	\item What is the probability that two black balls are chosen?
	
	\item What is the probability that two balls of opposite colour are chosen?
	\end{enumerate}
	\solution
	%\input{exemplar/11/16/3/12/main1.tex}
\end{enumerate}

		%
\item 
Out of 100 students, two sections of 40 and 60 are formed. If you and your friend are among the 100 students, what is the probability that
\begin{enumerate}
\item you both enter the same section?
\item you both enter the different sections?
\end{enumerate}
\solution
		%\begin{enumerate}[label=\thesection.\arabic*,ref=\thesection.\theenumi]
	\item One card is drawn from a well-shuffled deck of 52 cards. Find the probability of getting
\begin{enumerate}
\item A king of red colour 
\item A face card 
\item A red face card
\item The jack of hearts
\item A spade
\item The queen of diamonds

\end{enumerate}
\solution
		%\input{ncert/10/15/1/14/main.tex}
	\item Five cards—the ten, jack, queen, king and ace of diamonds, are well-shuffled with their face downwards. One card is then picked up at random.
\begin{enumerate}
\item
What is the probability that the card is the queen? 
\item
If the queen is drawn and put aside, what is the probability that the second card picked up is (a) an ace? (b) a queen?\\
\end{enumerate}
\solution
		%\input{ncert/10/15/1/15/defs.tex}
	\item A bag contains $5$ red balls and some blue balls. If the probability of drawing a blue ball is double that if a red ball, determine the number of blue balls in the bag. 
		\\
\solution
		%\input{ncert/10/15/2/3/defs.tex}
	\item A card is selected from a pack of 52 cards.
 \begin{enumerate}[label=(\alph*)] 
                 \item How many points are there in the sample space?
                 \item Calculate the probability that the card is an ace of spades.
                 \item Calculate the probability that the card is (i) an ace and (ii) black card.
 \end{enumerate}
\solution
		%\input{ncert/11/16/3/4/main.tex}
\item Four cards are drawn from a well-shuffled deck of 52 cards. What is the probability of obtaining 3 diamonds and one spade.
\\
\solution
		%\input{ncert/11/16/4/2/defs.tex}
\item In a certain lottery 10,000 tickets are sold and ten equal prizes are awarded. What is the probability of not getting a prize if you buy (a) one ticket (b) two tickets (c) 10 tickets ?	
\\
\solution
		%\input{ncert/11/16/4/4/defs.tex}
		%
\item 
Out of 100 students, two sections of 40 and 60 are formed. If you and your friend are among the 100 students, what is the probability that
\begin{enumerate}
\item you both enter the same section?
\item you both enter the different sections?
\end{enumerate}
\solution
		%\input{ncert/11/16/4/5/defs.tex}
	\item 
The number lock of a suitcase has 4 wheels each labelled with ten digits i.e. from 0 to 9.The lock opens with a sequence of four digits with no repeats.What is the probability of a person getting the right sequence to open the suitcase.
\\
\solution
		%\input{ncert/11/16/4/10/defs.tex}
		%
\item 
Two cards are drawn at random and without replacement from a pack of 52 playing cards. Find the probability that both the cards are black.
\\
\solution
		%\input{ncert/12/13/2/2/defs.tex}
		\item A box of oranges is inspected by examining three randomly selected oranges drawn without replacement. If all the three oranges are good, the box is approved for sale, otherwise, it is rejected. Find the probability that a box containing 15 oranges out of which 12 are good and 3 are bad ones will be approved for sale.
		\label{ncert/12/13/2/3/defs.tex}
		\item Two balls are drawn at random with replacement from a box containing 10 black and 8 red balls. Find the probability that
		\label{ncert/12/13/2/12}
\begin{enumerate}
\item both balls are red.
\item first ball is black and second is red.
\item one of them is black and other is red.
\end{enumerate}

\item In a hostel, 60\% of the students read Hindi newspaper, 40\% read English newspaper and 20\% read both Hindi and English newspapers. A student is selected at random.
		\label{ncert/12/13/2/15}
\begin{enumerate}
\item Find the probability that she reads neither Hindi nor English newspapers.
\item If she reads Hindi newspaper, find the probability that she reads English newspaper.
\item If she reads English newspaper, find the probability that she reads Hindi newspaper.\\
\end{enumerate}
\item The probability of obtaining an even prime number on each die, when a pair of dice is rolled is 
\begin{enumerate}
    \item $0$ 
    
    \item $\frac{1}{3}$ 
    
    \item $\frac{1}{12}$ 
    
    \item $\frac{1}{36}$ 
\end{enumerate}
\solution
		%\input{ncert/12/13/2/17/defs.tex}
	\item A bag contains 4 red and 4 black balls, another bag contains 2 red and 6 black balls. One of the two bags is selected at random and a ball is drawn from the bag which is found to be red. Find the probability that the ball is drawn from the first bag.
\\
\solution
		%\input{ncert/12/13/3/2/main.tex}
  \item
  Cards with numbers 2 to 101 are placed in a box. A card is selected at random.Find the probability that the card has
\begin{enumerate}[label=(\roman*)]
	\item an even number 
	\item a square number
\end{enumerate}
\solution
%\input{exemplar/10/13/3/32/main.tex}
\item
The king, queen and jack of clubs are removed from a deck of 52 playing cards and then well shuffled. Now one card is drawn at random from the remaining cards.  Determine the probability that the card is
\begin{enumerate}[label=(\roman*)]
\item a club
\item 10 of hearts
\end{enumerate}
\solution
%\input{exemplar/10/13/3/29/main.tex}
\item A team of medical students doing their internship have to assist during surgeries
at a city hospital. The probabilities of surgeries rated as very complex, complex,
routine, simple or very simple are respectively, 0.15, 0.20, 0.31, 0.26, .08. Find
the probabilities that a particular surgery will be rated
\begin{enumerate}
	\item complex or very complex;
	\item neither very complex nor very simple;
	\item routine or complex
	\item routine or simple
\end{enumerate}
\solution
%\input{exemplar/11/16/3/8(1)/main.tex}
\item A card is selected from a pack of 52 cards.
\begin{enumerate}[label=(\alph*)]
    \item How many points are there in the sample space?
    \item Calculate the probability that the card is an ace of spades.
    \item Calculate the probability that the card is (i) an ace and (ii) black card.
\end{enumerate}
\solution
%\input{exemplar/11/16/3/4/main2.tex}
\item The probability that a non leap year selected at random will contain 53 sundays.
\\
\solution
%\input{exemplar/10/13/1/19/main.tex}
\item One of the four persons John, Rita, Aslam or Gurpreet will be promoted next
month. Consequently the sample space consists of four elementary outcomes
S = {John promoted, Rita promoted, Aslam promoted, Gurpreet promoted}
You are told that the chances of John’s promotion is same as that of Gurpreet,
Rita’s chances of promotion are twice as likely as Johns. Aslam’s chances are
four times that of John.
\begin{enumerate}
	\item Determine
	\begin{enumerate}
		\item P (John promoted)
		\item P (Rita promoted)
		\item P (Aslam promoted)
		\item P (Gurpreet promoted)
	\end{enumerate}
	\item If A = {John promoted or Gurpreet promoted}, find P (A).
\end{enumerate}
\solution
%\input{exemplar/11/16/3/10/main.tex}
\item A card is drawn from a deck of 52 cards. Find the probability of getting a king or a heart or a red card.\\
\solution
%\input{exemplar/11/16/3/15/main.tex}
\item The probability that a student will pass his examination is 0.73, the probability of
the student getting a compartment is 0.13, and the probability that the student will
either pass or get compartment is 0.96. State True or False.\\
\solution
%\input{exemplar/11/16/3/31/main.tex}
\item A card is selected from a pack of 52 cards\\
\begin{enumerate}[label=(\alph*)]
\item How many points are there in the sample space?
\item Calculate the probability that the cards is an ace of spades.
\item Calculate the probability that the card is (i) an ace (ii)black card.\\
\end{enumerate}
%\input{ncert/11/16/3/4_1/Prob_4.tex}
\item In a non-leap year, the probability of having 53 tuesdays or 53 wednesdays is\\
\solution
%\input{exemplar/11/16/3/18/main.tex}
\item There are 1000 sealed envelopes in a box, 10 of them contain a cash prize of
Rs 100 each, 100 of them contain a cash prize of Rs 50 each and 200 of them
contain a cash prize of Rs 10 each and rest do not contain any cash prize. If they
are well shuffled and an envelope is picked up out, what is the probability that it
contains no cash prize?\\
\solution
%\input{exemplar/10/13/3/34/main.tex}
\item 
A die is thrown and a card is selected at random from a deck of 52 playing cards. The probability of getting an even number on the die and a spade card.\\
\solution
%\input{exemplar/12/13/3/78/main.tex}
\item
If 4-digit numbers greater than 5,000 are randomly formed from the digits 0, 1, 3, 5, and 7, what is the probability of forming a number divisible by 5 when:
\begin{enumerate}
    \item The digits are repeated?
    \item The repetition of digits is not allowed?
\end{enumerate}
\solution
%\input{ncert/11/16/4/9/main.tex}
\item Consider the probability space $\brak{\Omega, \mathcal{G}, P}$ where $\Omega = [0,2]$ and $\mathcal{G} = \cbrak{\phi, \Omega, [0,1], (1,2]}$. Let $X$ and $Y$ be two functions on $\Omega$ defined as
\begin{align*}
    X(\omega) = 
    \begin{cases}
        1 & \text{if }\omega \in [0, 1]\\
        2 & \text{if }\omega \in (1, 2]
    \end{cases}
\end{align*}
and
\begin{align*}
    Y(\omega) = 
    \begin{cases}
        2 & \text{if }\omega \in [0, 1.5]\\
        3 & \text{if }\omega \in (1.5, 2].
    \end{cases}
\end{align*}
Then which one of the following statements is true?
\begin{enumerate}
    \item [(A)] $X$ is a random variable with respect to $\mathcal{G}$, but $Y$ is not a random variable with respect to $\mathcal{G}$.
    \item [(B)] $Y$ is a random variable with respect to $\mathcal{G}$, but $X$ is not a random variable with respect to $\mathcal{G}$.
    \item [(C)] Neither $X$ nor $Y$ is a random variable with respect to $\mathcal{G}$.
    \item [(D)] Both $X$ and $Y$ are random variables with respect to $\mathcal{G}$.
\end{enumerate} \hfill (GATE ST 2023)\\
\solution
%\input{gate/ST/2023/14/main.tex}
	\item  A die is loaded in such a way that each odd number is twice as likely to occur as
each even number. Find $P(G)$, where $G$ is the event that a number greater than
3 occurs on a single roll of the die.
\\
\solution
		%\input{exemplar/11/16/3/5/main.tex}
	\item All the jacks, queens and kings are removed from a deck of 52 playing cards. The remaining cards are well shuffled and then one card is drawn at random. Giving ace a value 1 similar value for other cards, find the probability that the card has a value 
		\begin{enumerate}
			\item 7
			\item greater than 7
			\item less than 7
		\end{enumerate}
		%\input{exemplar/10/13/3/30/main.tex}
  \item A Lot consists of 48 mobile phones of which 42 are good, 3 have only minor defects and 3 have major defects.Varnika will buy a phone if it is good but the trader will only buy a mobile if it has no major defects. One phone is selected at random from the lot. What is the probability that it is
\begin{enumerate}
	\item acceptable to Varnika?
            \item acceptable to the trader?
\end{enumerate}
\solution
	%\input{exemplar/10/13/3/40/main.tex}
 \item A student says that if you throw a die, it will show up 1 or not 1. Therefore, the probability of getting 1 and the probability of getting 'not 1' each is equal to $\frac{1}{2}$. Is this correct? Give reasons.\\
 \solution
        %\input{exemplar/10/13/2/9/main.tex}
   \item Four candidates A, B, C, D have ap-
plied for the assignment to coach a school cricket
team. If A is twice as likely to be selected as B, and
B and C are given about the same chance of being
selected, while C is twice as likely to be selected
as D, what are the probabilities that
\begin{enumerate}
\item C will be selected?
\item A will not be selected?
\end{enumerate}
	%\input{exemplar/11/16/3/9/main.tex}
 \item A bag contain 24 balls of which $x$ balls are red, $2x$ are white and $3x$ are blue. A ball is selected at random, What is the probability that it is
\begin{enumerate}[label=\alph*)]
\item not red ?
\item white ?
\end{enumerate}
%\input{exemplar/10/13/3/41/main.tex}
If the letters of the word ASSASSINATION are arranged at random. Find the Probability that
\begin{enumerate}[label=(\alph*)]
\item Four $S's$ come consecutively in the word
\item Two  $I's$ and two $N's$ come together
\item All $A's$ are not coming together
\item No two $A's$ are coming together
\end{enumerate}
%\input{exemplar/11/16/3/14/main.tex}
	\item One urn contains two black balls (labelled B1 and B2) and one white ball. A
	second urn contains one black ball and two white balls (labelled W1 and W2).
	Suppose the following experiment is performed. One of the two urns is chosen
	at random. Next a ball is randomly chosen from the urn. Then a second ball is
	chosen at random from the same urn without replacing the first ball.
	
	\begin{enumerate}
	\item What is the probability that two black balls are chosen?
	
	\item What is the probability that two balls of opposite colour are chosen?
	\end{enumerate}
	\solution
	%\input{exemplar/11/16/3/12/main1.tex}
\end{enumerate}

	\item 
The number lock of a suitcase has 4 wheels each labelled with ten digits i.e. from 0 to 9.The lock opens with a sequence of four digits with no repeats.What is the probability of a person getting the right sequence to open the suitcase.
\\
\solution
		%\begin{enumerate}[label=\thesection.\arabic*,ref=\thesection.\theenumi]
	\item One card is drawn from a well-shuffled deck of 52 cards. Find the probability of getting
\begin{enumerate}
\item A king of red colour 
\item A face card 
\item A red face card
\item The jack of hearts
\item A spade
\item The queen of diamonds

\end{enumerate}
\solution
		%\input{ncert/10/15/1/14/main.tex}
	\item Five cards—the ten, jack, queen, king and ace of diamonds, are well-shuffled with their face downwards. One card is then picked up at random.
\begin{enumerate}
\item
What is the probability that the card is the queen? 
\item
If the queen is drawn and put aside, what is the probability that the second card picked up is (a) an ace? (b) a queen?\\
\end{enumerate}
\solution
		%\input{ncert/10/15/1/15/defs.tex}
	\item A bag contains $5$ red balls and some blue balls. If the probability of drawing a blue ball is double that if a red ball, determine the number of blue balls in the bag. 
		\\
\solution
		%\input{ncert/10/15/2/3/defs.tex}
	\item A card is selected from a pack of 52 cards.
 \begin{enumerate}[label=(\alph*)] 
                 \item How many points are there in the sample space?
                 \item Calculate the probability that the card is an ace of spades.
                 \item Calculate the probability that the card is (i) an ace and (ii) black card.
 \end{enumerate}
\solution
		%\input{ncert/11/16/3/4/main.tex}
\item Four cards are drawn from a well-shuffled deck of 52 cards. What is the probability of obtaining 3 diamonds and one spade.
\\
\solution
		%\input{ncert/11/16/4/2/defs.tex}
\item In a certain lottery 10,000 tickets are sold and ten equal prizes are awarded. What is the probability of not getting a prize if you buy (a) one ticket (b) two tickets (c) 10 tickets ?	
\\
\solution
		%\input{ncert/11/16/4/4/defs.tex}
		%
\item 
Out of 100 students, two sections of 40 and 60 are formed. If you and your friend are among the 100 students, what is the probability that
\begin{enumerate}
\item you both enter the same section?
\item you both enter the different sections?
\end{enumerate}
\solution
		%\input{ncert/11/16/4/5/defs.tex}
	\item 
The number lock of a suitcase has 4 wheels each labelled with ten digits i.e. from 0 to 9.The lock opens with a sequence of four digits with no repeats.What is the probability of a person getting the right sequence to open the suitcase.
\\
\solution
		%\input{ncert/11/16/4/10/defs.tex}
		%
\item 
Two cards are drawn at random and without replacement from a pack of 52 playing cards. Find the probability that both the cards are black.
\\
\solution
		%\input{ncert/12/13/2/2/defs.tex}
		\item A box of oranges is inspected by examining three randomly selected oranges drawn without replacement. If all the three oranges are good, the box is approved for sale, otherwise, it is rejected. Find the probability that a box containing 15 oranges out of which 12 are good and 3 are bad ones will be approved for sale.
		\label{ncert/12/13/2/3/defs.tex}
		\item Two balls are drawn at random with replacement from a box containing 10 black and 8 red balls. Find the probability that
		\label{ncert/12/13/2/12}
\begin{enumerate}
\item both balls are red.
\item first ball is black and second is red.
\item one of them is black and other is red.
\end{enumerate}

\item In a hostel, 60\% of the students read Hindi newspaper, 40\% read English newspaper and 20\% read both Hindi and English newspapers. A student is selected at random.
		\label{ncert/12/13/2/15}
\begin{enumerate}
\item Find the probability that she reads neither Hindi nor English newspapers.
\item If she reads Hindi newspaper, find the probability that she reads English newspaper.
\item If she reads English newspaper, find the probability that she reads Hindi newspaper.\\
\end{enumerate}
\item The probability of obtaining an even prime number on each die, when a pair of dice is rolled is 
\begin{enumerate}
    \item $0$ 
    
    \item $\frac{1}{3}$ 
    
    \item $\frac{1}{12}$ 
    
    \item $\frac{1}{36}$ 
\end{enumerate}
\solution
		%\input{ncert/12/13/2/17/defs.tex}
	\item A bag contains 4 red and 4 black balls, another bag contains 2 red and 6 black balls. One of the two bags is selected at random and a ball is drawn from the bag which is found to be red. Find the probability that the ball is drawn from the first bag.
\\
\solution
		%\input{ncert/12/13/3/2/main.tex}
  \item
  Cards with numbers 2 to 101 are placed in a box. A card is selected at random.Find the probability that the card has
\begin{enumerate}[label=(\roman*)]
	\item an even number 
	\item a square number
\end{enumerate}
\solution
%\input{exemplar/10/13/3/32/main.tex}
\item
The king, queen and jack of clubs are removed from a deck of 52 playing cards and then well shuffled. Now one card is drawn at random from the remaining cards.  Determine the probability that the card is
\begin{enumerate}[label=(\roman*)]
\item a club
\item 10 of hearts
\end{enumerate}
\solution
%\input{exemplar/10/13/3/29/main.tex}
\item A team of medical students doing their internship have to assist during surgeries
at a city hospital. The probabilities of surgeries rated as very complex, complex,
routine, simple or very simple are respectively, 0.15, 0.20, 0.31, 0.26, .08. Find
the probabilities that a particular surgery will be rated
\begin{enumerate}
	\item complex or very complex;
	\item neither very complex nor very simple;
	\item routine or complex
	\item routine or simple
\end{enumerate}
\solution
%\input{exemplar/11/16/3/8(1)/main.tex}
\item A card is selected from a pack of 52 cards.
\begin{enumerate}[label=(\alph*)]
    \item How many points are there in the sample space?
    \item Calculate the probability that the card is an ace of spades.
    \item Calculate the probability that the card is (i) an ace and (ii) black card.
\end{enumerate}
\solution
%\input{exemplar/11/16/3/4/main2.tex}
\item The probability that a non leap year selected at random will contain 53 sundays.
\\
\solution
%\input{exemplar/10/13/1/19/main.tex}
\item One of the four persons John, Rita, Aslam or Gurpreet will be promoted next
month. Consequently the sample space consists of four elementary outcomes
S = {John promoted, Rita promoted, Aslam promoted, Gurpreet promoted}
You are told that the chances of John’s promotion is same as that of Gurpreet,
Rita’s chances of promotion are twice as likely as Johns. Aslam’s chances are
four times that of John.
\begin{enumerate}
	\item Determine
	\begin{enumerate}
		\item P (John promoted)
		\item P (Rita promoted)
		\item P (Aslam promoted)
		\item P (Gurpreet promoted)
	\end{enumerate}
	\item If A = {John promoted or Gurpreet promoted}, find P (A).
\end{enumerate}
\solution
%\input{exemplar/11/16/3/10/main.tex}
\item A card is drawn from a deck of 52 cards. Find the probability of getting a king or a heart or a red card.\\
\solution
%\input{exemplar/11/16/3/15/main.tex}
\item The probability that a student will pass his examination is 0.73, the probability of
the student getting a compartment is 0.13, and the probability that the student will
either pass or get compartment is 0.96. State True or False.\\
\solution
%\input{exemplar/11/16/3/31/main.tex}
\item A card is selected from a pack of 52 cards\\
\begin{enumerate}[label=(\alph*)]
\item How many points are there in the sample space?
\item Calculate the probability that the cards is an ace of spades.
\item Calculate the probability that the card is (i) an ace (ii)black card.\\
\end{enumerate}
%\input{ncert/11/16/3/4_1/Prob_4.tex}
\item In a non-leap year, the probability of having 53 tuesdays or 53 wednesdays is\\
\solution
%\input{exemplar/11/16/3/18/main.tex}
\item There are 1000 sealed envelopes in a box, 10 of them contain a cash prize of
Rs 100 each, 100 of them contain a cash prize of Rs 50 each and 200 of them
contain a cash prize of Rs 10 each and rest do not contain any cash prize. If they
are well shuffled and an envelope is picked up out, what is the probability that it
contains no cash prize?\\
\solution
%\input{exemplar/10/13/3/34/main.tex}
\item 
A die is thrown and a card is selected at random from a deck of 52 playing cards. The probability of getting an even number on the die and a spade card.\\
\solution
%\input{exemplar/12/13/3/78/main.tex}
\item
If 4-digit numbers greater than 5,000 are randomly formed from the digits 0, 1, 3, 5, and 7, what is the probability of forming a number divisible by 5 when:
\begin{enumerate}
    \item The digits are repeated?
    \item The repetition of digits is not allowed?
\end{enumerate}
\solution
%\input{ncert/11/16/4/9/main.tex}
\item Consider the probability space $\brak{\Omega, \mathcal{G}, P}$ where $\Omega = [0,2]$ and $\mathcal{G} = \cbrak{\phi, \Omega, [0,1], (1,2]}$. Let $X$ and $Y$ be two functions on $\Omega$ defined as
\begin{align*}
    X(\omega) = 
    \begin{cases}
        1 & \text{if }\omega \in [0, 1]\\
        2 & \text{if }\omega \in (1, 2]
    \end{cases}
\end{align*}
and
\begin{align*}
    Y(\omega) = 
    \begin{cases}
        2 & \text{if }\omega \in [0, 1.5]\\
        3 & \text{if }\omega \in (1.5, 2].
    \end{cases}
\end{align*}
Then which one of the following statements is true?
\begin{enumerate}
    \item [(A)] $X$ is a random variable with respect to $\mathcal{G}$, but $Y$ is not a random variable with respect to $\mathcal{G}$.
    \item [(B)] $Y$ is a random variable with respect to $\mathcal{G}$, but $X$ is not a random variable with respect to $\mathcal{G}$.
    \item [(C)] Neither $X$ nor $Y$ is a random variable with respect to $\mathcal{G}$.
    \item [(D)] Both $X$ and $Y$ are random variables with respect to $\mathcal{G}$.
\end{enumerate} \hfill (GATE ST 2023)\\
\solution
%\input{gate/ST/2023/14/main.tex}
	\item  A die is loaded in such a way that each odd number is twice as likely to occur as
each even number. Find $P(G)$, where $G$ is the event that a number greater than
3 occurs on a single roll of the die.
\\
\solution
		%\input{exemplar/11/16/3/5/main.tex}
	\item All the jacks, queens and kings are removed from a deck of 52 playing cards. The remaining cards are well shuffled and then one card is drawn at random. Giving ace a value 1 similar value for other cards, find the probability that the card has a value 
		\begin{enumerate}
			\item 7
			\item greater than 7
			\item less than 7
		\end{enumerate}
		%\input{exemplar/10/13/3/30/main.tex}
  \item A Lot consists of 48 mobile phones of which 42 are good, 3 have only minor defects and 3 have major defects.Varnika will buy a phone if it is good but the trader will only buy a mobile if it has no major defects. One phone is selected at random from the lot. What is the probability that it is
\begin{enumerate}
	\item acceptable to Varnika?
            \item acceptable to the trader?
\end{enumerate}
\solution
	%\input{exemplar/10/13/3/40/main.tex}
 \item A student says that if you throw a die, it will show up 1 or not 1. Therefore, the probability of getting 1 and the probability of getting 'not 1' each is equal to $\frac{1}{2}$. Is this correct? Give reasons.\\
 \solution
        %\input{exemplar/10/13/2/9/main.tex}
   \item Four candidates A, B, C, D have ap-
plied for the assignment to coach a school cricket
team. If A is twice as likely to be selected as B, and
B and C are given about the same chance of being
selected, while C is twice as likely to be selected
as D, what are the probabilities that
\begin{enumerate}
\item C will be selected?
\item A will not be selected?
\end{enumerate}
	%\input{exemplar/11/16/3/9/main.tex}
 \item A bag contain 24 balls of which $x$ balls are red, $2x$ are white and $3x$ are blue. A ball is selected at random, What is the probability that it is
\begin{enumerate}[label=\alph*)]
\item not red ?
\item white ?
\end{enumerate}
%\input{exemplar/10/13/3/41/main.tex}
If the letters of the word ASSASSINATION are arranged at random. Find the Probability that
\begin{enumerate}[label=(\alph*)]
\item Four $S's$ come consecutively in the word
\item Two  $I's$ and two $N's$ come together
\item All $A's$ are not coming together
\item No two $A's$ are coming together
\end{enumerate}
%\input{exemplar/11/16/3/14/main.tex}
	\item One urn contains two black balls (labelled B1 and B2) and one white ball. A
	second urn contains one black ball and two white balls (labelled W1 and W2).
	Suppose the following experiment is performed. One of the two urns is chosen
	at random. Next a ball is randomly chosen from the urn. Then a second ball is
	chosen at random from the same urn without replacing the first ball.
	
	\begin{enumerate}
	\item What is the probability that two black balls are chosen?
	
	\item What is the probability that two balls of opposite colour are chosen?
	\end{enumerate}
	\solution
	%\input{exemplar/11/16/3/12/main1.tex}
\end{enumerate}

		%
\item 
Two cards are drawn at random and without replacement from a pack of 52 playing cards. Find the probability that both the cards are black.
\\
\solution
		%\begin{enumerate}[label=\thesection.\arabic*,ref=\thesection.\theenumi]
	\item One card is drawn from a well-shuffled deck of 52 cards. Find the probability of getting
\begin{enumerate}
\item A king of red colour 
\item A face card 
\item A red face card
\item The jack of hearts
\item A spade
\item The queen of diamonds

\end{enumerate}
\solution
		%\input{ncert/10/15/1/14/main.tex}
	\item Five cards—the ten, jack, queen, king and ace of diamonds, are well-shuffled with their face downwards. One card is then picked up at random.
\begin{enumerate}
\item
What is the probability that the card is the queen? 
\item
If the queen is drawn and put aside, what is the probability that the second card picked up is (a) an ace? (b) a queen?\\
\end{enumerate}
\solution
		%\input{ncert/10/15/1/15/defs.tex}
	\item A bag contains $5$ red balls and some blue balls. If the probability of drawing a blue ball is double that if a red ball, determine the number of blue balls in the bag. 
		\\
\solution
		%\input{ncert/10/15/2/3/defs.tex}
	\item A card is selected from a pack of 52 cards.
 \begin{enumerate}[label=(\alph*)] 
                 \item How many points are there in the sample space?
                 \item Calculate the probability that the card is an ace of spades.
                 \item Calculate the probability that the card is (i) an ace and (ii) black card.
 \end{enumerate}
\solution
		%\input{ncert/11/16/3/4/main.tex}
\item Four cards are drawn from a well-shuffled deck of 52 cards. What is the probability of obtaining 3 diamonds and one spade.
\\
\solution
		%\input{ncert/11/16/4/2/defs.tex}
\item In a certain lottery 10,000 tickets are sold and ten equal prizes are awarded. What is the probability of not getting a prize if you buy (a) one ticket (b) two tickets (c) 10 tickets ?	
\\
\solution
		%\input{ncert/11/16/4/4/defs.tex}
		%
\item 
Out of 100 students, two sections of 40 and 60 are formed. If you and your friend are among the 100 students, what is the probability that
\begin{enumerate}
\item you both enter the same section?
\item you both enter the different sections?
\end{enumerate}
\solution
		%\input{ncert/11/16/4/5/defs.tex}
	\item 
The number lock of a suitcase has 4 wheels each labelled with ten digits i.e. from 0 to 9.The lock opens with a sequence of four digits with no repeats.What is the probability of a person getting the right sequence to open the suitcase.
\\
\solution
		%\input{ncert/11/16/4/10/defs.tex}
		%
\item 
Two cards are drawn at random and without replacement from a pack of 52 playing cards. Find the probability that both the cards are black.
\\
\solution
		%\input{ncert/12/13/2/2/defs.tex}
		\item A box of oranges is inspected by examining three randomly selected oranges drawn without replacement. If all the three oranges are good, the box is approved for sale, otherwise, it is rejected. Find the probability that a box containing 15 oranges out of which 12 are good and 3 are bad ones will be approved for sale.
		\label{ncert/12/13/2/3/defs.tex}
		\item Two balls are drawn at random with replacement from a box containing 10 black and 8 red balls. Find the probability that
		\label{ncert/12/13/2/12}
\begin{enumerate}
\item both balls are red.
\item first ball is black and second is red.
\item one of them is black and other is red.
\end{enumerate}

\item In a hostel, 60\% of the students read Hindi newspaper, 40\% read English newspaper and 20\% read both Hindi and English newspapers. A student is selected at random.
		\label{ncert/12/13/2/15}
\begin{enumerate}
\item Find the probability that she reads neither Hindi nor English newspapers.
\item If she reads Hindi newspaper, find the probability that she reads English newspaper.
\item If she reads English newspaper, find the probability that she reads Hindi newspaper.\\
\end{enumerate}
\item The probability of obtaining an even prime number on each die, when a pair of dice is rolled is 
\begin{enumerate}
    \item $0$ 
    
    \item $\frac{1}{3}$ 
    
    \item $\frac{1}{12}$ 
    
    \item $\frac{1}{36}$ 
\end{enumerate}
\solution
		%\input{ncert/12/13/2/17/defs.tex}
	\item A bag contains 4 red and 4 black balls, another bag contains 2 red and 6 black balls. One of the two bags is selected at random and a ball is drawn from the bag which is found to be red. Find the probability that the ball is drawn from the first bag.
\\
\solution
		%\input{ncert/12/13/3/2/main.tex}
  \item
  Cards with numbers 2 to 101 are placed in a box. A card is selected at random.Find the probability that the card has
\begin{enumerate}[label=(\roman*)]
	\item an even number 
	\item a square number
\end{enumerate}
\solution
%\input{exemplar/10/13/3/32/main.tex}
\item
The king, queen and jack of clubs are removed from a deck of 52 playing cards and then well shuffled. Now one card is drawn at random from the remaining cards.  Determine the probability that the card is
\begin{enumerate}[label=(\roman*)]
\item a club
\item 10 of hearts
\end{enumerate}
\solution
%\input{exemplar/10/13/3/29/main.tex}
\item A team of medical students doing their internship have to assist during surgeries
at a city hospital. The probabilities of surgeries rated as very complex, complex,
routine, simple or very simple are respectively, 0.15, 0.20, 0.31, 0.26, .08. Find
the probabilities that a particular surgery will be rated
\begin{enumerate}
	\item complex or very complex;
	\item neither very complex nor very simple;
	\item routine or complex
	\item routine or simple
\end{enumerate}
\solution
%\input{exemplar/11/16/3/8(1)/main.tex}
\item A card is selected from a pack of 52 cards.
\begin{enumerate}[label=(\alph*)]
    \item How many points are there in the sample space?
    \item Calculate the probability that the card is an ace of spades.
    \item Calculate the probability that the card is (i) an ace and (ii) black card.
\end{enumerate}
\solution
%\input{exemplar/11/16/3/4/main2.tex}
\item The probability that a non leap year selected at random will contain 53 sundays.
\\
\solution
%\input{exemplar/10/13/1/19/main.tex}
\item One of the four persons John, Rita, Aslam or Gurpreet will be promoted next
month. Consequently the sample space consists of four elementary outcomes
S = {John promoted, Rita promoted, Aslam promoted, Gurpreet promoted}
You are told that the chances of John’s promotion is same as that of Gurpreet,
Rita’s chances of promotion are twice as likely as Johns. Aslam’s chances are
four times that of John.
\begin{enumerate}
	\item Determine
	\begin{enumerate}
		\item P (John promoted)
		\item P (Rita promoted)
		\item P (Aslam promoted)
		\item P (Gurpreet promoted)
	\end{enumerate}
	\item If A = {John promoted or Gurpreet promoted}, find P (A).
\end{enumerate}
\solution
%\input{exemplar/11/16/3/10/main.tex}
\item A card is drawn from a deck of 52 cards. Find the probability of getting a king or a heart or a red card.\\
\solution
%\input{exemplar/11/16/3/15/main.tex}
\item The probability that a student will pass his examination is 0.73, the probability of
the student getting a compartment is 0.13, and the probability that the student will
either pass or get compartment is 0.96. State True or False.\\
\solution
%\input{exemplar/11/16/3/31/main.tex}
\item A card is selected from a pack of 52 cards\\
\begin{enumerate}[label=(\alph*)]
\item How many points are there in the sample space?
\item Calculate the probability that the cards is an ace of spades.
\item Calculate the probability that the card is (i) an ace (ii)black card.\\
\end{enumerate}
%\input{ncert/11/16/3/4_1/Prob_4.tex}
\item In a non-leap year, the probability of having 53 tuesdays or 53 wednesdays is\\
\solution
%\input{exemplar/11/16/3/18/main.tex}
\item There are 1000 sealed envelopes in a box, 10 of them contain a cash prize of
Rs 100 each, 100 of them contain a cash prize of Rs 50 each and 200 of them
contain a cash prize of Rs 10 each and rest do not contain any cash prize. If they
are well shuffled and an envelope is picked up out, what is the probability that it
contains no cash prize?\\
\solution
%\input{exemplar/10/13/3/34/main.tex}
\item 
A die is thrown and a card is selected at random from a deck of 52 playing cards. The probability of getting an even number on the die and a spade card.\\
\solution
%\input{exemplar/12/13/3/78/main.tex}
\item
If 4-digit numbers greater than 5,000 are randomly formed from the digits 0, 1, 3, 5, and 7, what is the probability of forming a number divisible by 5 when:
\begin{enumerate}
    \item The digits are repeated?
    \item The repetition of digits is not allowed?
\end{enumerate}
\solution
%\input{ncert/11/16/4/9/main.tex}
\item Consider the probability space $\brak{\Omega, \mathcal{G}, P}$ where $\Omega = [0,2]$ and $\mathcal{G} = \cbrak{\phi, \Omega, [0,1], (1,2]}$. Let $X$ and $Y$ be two functions on $\Omega$ defined as
\begin{align*}
    X(\omega) = 
    \begin{cases}
        1 & \text{if }\omega \in [0, 1]\\
        2 & \text{if }\omega \in (1, 2]
    \end{cases}
\end{align*}
and
\begin{align*}
    Y(\omega) = 
    \begin{cases}
        2 & \text{if }\omega \in [0, 1.5]\\
        3 & \text{if }\omega \in (1.5, 2].
    \end{cases}
\end{align*}
Then which one of the following statements is true?
\begin{enumerate}
    \item [(A)] $X$ is a random variable with respect to $\mathcal{G}$, but $Y$ is not a random variable with respect to $\mathcal{G}$.
    \item [(B)] $Y$ is a random variable with respect to $\mathcal{G}$, but $X$ is not a random variable with respect to $\mathcal{G}$.
    \item [(C)] Neither $X$ nor $Y$ is a random variable with respect to $\mathcal{G}$.
    \item [(D)] Both $X$ and $Y$ are random variables with respect to $\mathcal{G}$.
\end{enumerate} \hfill (GATE ST 2023)\\
\solution
%\input{gate/ST/2023/14/main.tex}
	\item  A die is loaded in such a way that each odd number is twice as likely to occur as
each even number. Find $P(G)$, where $G$ is the event that a number greater than
3 occurs on a single roll of the die.
\\
\solution
		%\input{exemplar/11/16/3/5/main.tex}
	\item All the jacks, queens and kings are removed from a deck of 52 playing cards. The remaining cards are well shuffled and then one card is drawn at random. Giving ace a value 1 similar value for other cards, find the probability that the card has a value 
		\begin{enumerate}
			\item 7
			\item greater than 7
			\item less than 7
		\end{enumerate}
		%\input{exemplar/10/13/3/30/main.tex}
  \item A Lot consists of 48 mobile phones of which 42 are good, 3 have only minor defects and 3 have major defects.Varnika will buy a phone if it is good but the trader will only buy a mobile if it has no major defects. One phone is selected at random from the lot. What is the probability that it is
\begin{enumerate}
	\item acceptable to Varnika?
            \item acceptable to the trader?
\end{enumerate}
\solution
	%\input{exemplar/10/13/3/40/main.tex}
 \item A student says that if you throw a die, it will show up 1 or not 1. Therefore, the probability of getting 1 and the probability of getting 'not 1' each is equal to $\frac{1}{2}$. Is this correct? Give reasons.\\
 \solution
        %\input{exemplar/10/13/2/9/main.tex}
   \item Four candidates A, B, C, D have ap-
plied for the assignment to coach a school cricket
team. If A is twice as likely to be selected as B, and
B and C are given about the same chance of being
selected, while C is twice as likely to be selected
as D, what are the probabilities that
\begin{enumerate}
\item C will be selected?
\item A will not be selected?
\end{enumerate}
	%\input{exemplar/11/16/3/9/main.tex}
 \item A bag contain 24 balls of which $x$ balls are red, $2x$ are white and $3x$ are blue. A ball is selected at random, What is the probability that it is
\begin{enumerate}[label=\alph*)]
\item not red ?
\item white ?
\end{enumerate}
%\input{exemplar/10/13/3/41/main.tex}
If the letters of the word ASSASSINATION are arranged at random. Find the Probability that
\begin{enumerate}[label=(\alph*)]
\item Four $S's$ come consecutively in the word
\item Two  $I's$ and two $N's$ come together
\item All $A's$ are not coming together
\item No two $A's$ are coming together
\end{enumerate}
%\input{exemplar/11/16/3/14/main.tex}
	\item One urn contains two black balls (labelled B1 and B2) and one white ball. A
	second urn contains one black ball and two white balls (labelled W1 and W2).
	Suppose the following experiment is performed. One of the two urns is chosen
	at random. Next a ball is randomly chosen from the urn. Then a second ball is
	chosen at random from the same urn without replacing the first ball.
	
	\begin{enumerate}
	\item What is the probability that two black balls are chosen?
	
	\item What is the probability that two balls of opposite colour are chosen?
	\end{enumerate}
	\solution
	%\input{exemplar/11/16/3/12/main1.tex}
\end{enumerate}

		\item A box of oranges is inspected by examining three randomly selected oranges drawn without replacement. If all the three oranges are good, the box is approved for sale, otherwise, it is rejected. Find the probability that a box containing 15 oranges out of which 12 are good and 3 are bad ones will be approved for sale.
		\label{ncert/12/13/2/3/defs.tex}
		\item Two balls are drawn at random with replacement from a box containing 10 black and 8 red balls. Find the probability that
		\label{ncert/12/13/2/12}
\begin{enumerate}
\item both balls are red.
\item first ball is black and second is red.
\item one of them is black and other is red.
\end{enumerate}

\item In a hostel, 60\% of the students read Hindi newspaper, 40\% read English newspaper and 20\% read both Hindi and English newspapers. A student is selected at random.
		\label{ncert/12/13/2/15}
\begin{enumerate}
\item Find the probability that she reads neither Hindi nor English newspapers.
\item If she reads Hindi newspaper, find the probability that she reads English newspaper.
\item If she reads English newspaper, find the probability that she reads Hindi newspaper.\\
\end{enumerate}
\item The probability of obtaining an even prime number on each die, when a pair of dice is rolled is 
\begin{enumerate}
    \item $0$ 
    
    \item $\frac{1}{3}$ 
    
    \item $\frac{1}{12}$ 
    
    \item $\frac{1}{36}$ 
\end{enumerate}
\solution
		%\begin{enumerate}[label=\thesection.\arabic*,ref=\thesection.\theenumi]
	\item One card is drawn from a well-shuffled deck of 52 cards. Find the probability of getting
\begin{enumerate}
\item A king of red colour 
\item A face card 
\item A red face card
\item The jack of hearts
\item A spade
\item The queen of diamonds

\end{enumerate}
\solution
		%\input{ncert/10/15/1/14/main.tex}
	\item Five cards—the ten, jack, queen, king and ace of diamonds, are well-shuffled with their face downwards. One card is then picked up at random.
\begin{enumerate}
\item
What is the probability that the card is the queen? 
\item
If the queen is drawn and put aside, what is the probability that the second card picked up is (a) an ace? (b) a queen?\\
\end{enumerate}
\solution
		%\input{ncert/10/15/1/15/defs.tex}
	\item A bag contains $5$ red balls and some blue balls. If the probability of drawing a blue ball is double that if a red ball, determine the number of blue balls in the bag. 
		\\
\solution
		%\input{ncert/10/15/2/3/defs.tex}
	\item A card is selected from a pack of 52 cards.
 \begin{enumerate}[label=(\alph*)] 
                 \item How many points are there in the sample space?
                 \item Calculate the probability that the card is an ace of spades.
                 \item Calculate the probability that the card is (i) an ace and (ii) black card.
 \end{enumerate}
\solution
		%\input{ncert/11/16/3/4/main.tex}
\item Four cards are drawn from a well-shuffled deck of 52 cards. What is the probability of obtaining 3 diamonds and one spade.
\\
\solution
		%\input{ncert/11/16/4/2/defs.tex}
\item In a certain lottery 10,000 tickets are sold and ten equal prizes are awarded. What is the probability of not getting a prize if you buy (a) one ticket (b) two tickets (c) 10 tickets ?	
\\
\solution
		%\input{ncert/11/16/4/4/defs.tex}
		%
\item 
Out of 100 students, two sections of 40 and 60 are formed. If you and your friend are among the 100 students, what is the probability that
\begin{enumerate}
\item you both enter the same section?
\item you both enter the different sections?
\end{enumerate}
\solution
		%\input{ncert/11/16/4/5/defs.tex}
	\item 
The number lock of a suitcase has 4 wheels each labelled with ten digits i.e. from 0 to 9.The lock opens with a sequence of four digits with no repeats.What is the probability of a person getting the right sequence to open the suitcase.
\\
\solution
		%\input{ncert/11/16/4/10/defs.tex}
		%
\item 
Two cards are drawn at random and without replacement from a pack of 52 playing cards. Find the probability that both the cards are black.
\\
\solution
		%\input{ncert/12/13/2/2/defs.tex}
		\item A box of oranges is inspected by examining three randomly selected oranges drawn without replacement. If all the three oranges are good, the box is approved for sale, otherwise, it is rejected. Find the probability that a box containing 15 oranges out of which 12 are good and 3 are bad ones will be approved for sale.
		\label{ncert/12/13/2/3/defs.tex}
		\item Two balls are drawn at random with replacement from a box containing 10 black and 8 red balls. Find the probability that
		\label{ncert/12/13/2/12}
\begin{enumerate}
\item both balls are red.
\item first ball is black and second is red.
\item one of them is black and other is red.
\end{enumerate}

\item In a hostel, 60\% of the students read Hindi newspaper, 40\% read English newspaper and 20\% read both Hindi and English newspapers. A student is selected at random.
		\label{ncert/12/13/2/15}
\begin{enumerate}
\item Find the probability that she reads neither Hindi nor English newspapers.
\item If she reads Hindi newspaper, find the probability that she reads English newspaper.
\item If she reads English newspaper, find the probability that she reads Hindi newspaper.\\
\end{enumerate}
\item The probability of obtaining an even prime number on each die, when a pair of dice is rolled is 
\begin{enumerate}
    \item $0$ 
    
    \item $\frac{1}{3}$ 
    
    \item $\frac{1}{12}$ 
    
    \item $\frac{1}{36}$ 
\end{enumerate}
\solution
		%\input{ncert/12/13/2/17/defs.tex}
	\item A bag contains 4 red and 4 black balls, another bag contains 2 red and 6 black balls. One of the two bags is selected at random and a ball is drawn from the bag which is found to be red. Find the probability that the ball is drawn from the first bag.
\\
\solution
		%\input{ncert/12/13/3/2/main.tex}
  \item
  Cards with numbers 2 to 101 are placed in a box. A card is selected at random.Find the probability that the card has
\begin{enumerate}[label=(\roman*)]
	\item an even number 
	\item a square number
\end{enumerate}
\solution
%\input{exemplar/10/13/3/32/main.tex}
\item
The king, queen and jack of clubs are removed from a deck of 52 playing cards and then well shuffled. Now one card is drawn at random from the remaining cards.  Determine the probability that the card is
\begin{enumerate}[label=(\roman*)]
\item a club
\item 10 of hearts
\end{enumerate}
\solution
%\input{exemplar/10/13/3/29/main.tex}
\item A team of medical students doing their internship have to assist during surgeries
at a city hospital. The probabilities of surgeries rated as very complex, complex,
routine, simple or very simple are respectively, 0.15, 0.20, 0.31, 0.26, .08. Find
the probabilities that a particular surgery will be rated
\begin{enumerate}
	\item complex or very complex;
	\item neither very complex nor very simple;
	\item routine or complex
	\item routine or simple
\end{enumerate}
\solution
%\input{exemplar/11/16/3/8(1)/main.tex}
\item A card is selected from a pack of 52 cards.
\begin{enumerate}[label=(\alph*)]
    \item How many points are there in the sample space?
    \item Calculate the probability that the card is an ace of spades.
    \item Calculate the probability that the card is (i) an ace and (ii) black card.
\end{enumerate}
\solution
%\input{exemplar/11/16/3/4/main2.tex}
\item The probability that a non leap year selected at random will contain 53 sundays.
\\
\solution
%\input{exemplar/10/13/1/19/main.tex}
\item One of the four persons John, Rita, Aslam or Gurpreet will be promoted next
month. Consequently the sample space consists of four elementary outcomes
S = {John promoted, Rita promoted, Aslam promoted, Gurpreet promoted}
You are told that the chances of John’s promotion is same as that of Gurpreet,
Rita’s chances of promotion are twice as likely as Johns. Aslam’s chances are
four times that of John.
\begin{enumerate}
	\item Determine
	\begin{enumerate}
		\item P (John promoted)
		\item P (Rita promoted)
		\item P (Aslam promoted)
		\item P (Gurpreet promoted)
	\end{enumerate}
	\item If A = {John promoted or Gurpreet promoted}, find P (A).
\end{enumerate}
\solution
%\input{exemplar/11/16/3/10/main.tex}
\item A card is drawn from a deck of 52 cards. Find the probability of getting a king or a heart or a red card.\\
\solution
%\input{exemplar/11/16/3/15/main.tex}
\item The probability that a student will pass his examination is 0.73, the probability of
the student getting a compartment is 0.13, and the probability that the student will
either pass or get compartment is 0.96. State True or False.\\
\solution
%\input{exemplar/11/16/3/31/main.tex}
\item A card is selected from a pack of 52 cards\\
\begin{enumerate}[label=(\alph*)]
\item How many points are there in the sample space?
\item Calculate the probability that the cards is an ace of spades.
\item Calculate the probability that the card is (i) an ace (ii)black card.\\
\end{enumerate}
%\input{ncert/11/16/3/4_1/Prob_4.tex}
\item In a non-leap year, the probability of having 53 tuesdays or 53 wednesdays is\\
\solution
%\input{exemplar/11/16/3/18/main.tex}
\item There are 1000 sealed envelopes in a box, 10 of them contain a cash prize of
Rs 100 each, 100 of them contain a cash prize of Rs 50 each and 200 of them
contain a cash prize of Rs 10 each and rest do not contain any cash prize. If they
are well shuffled and an envelope is picked up out, what is the probability that it
contains no cash prize?\\
\solution
%\input{exemplar/10/13/3/34/main.tex}
\item 
A die is thrown and a card is selected at random from a deck of 52 playing cards. The probability of getting an even number on the die and a spade card.\\
\solution
%\input{exemplar/12/13/3/78/main.tex}
\item
If 4-digit numbers greater than 5,000 are randomly formed from the digits 0, 1, 3, 5, and 7, what is the probability of forming a number divisible by 5 when:
\begin{enumerate}
    \item The digits are repeated?
    \item The repetition of digits is not allowed?
\end{enumerate}
\solution
%\input{ncert/11/16/4/9/main.tex}
\item Consider the probability space $\brak{\Omega, \mathcal{G}, P}$ where $\Omega = [0,2]$ and $\mathcal{G} = \cbrak{\phi, \Omega, [0,1], (1,2]}$. Let $X$ and $Y$ be two functions on $\Omega$ defined as
\begin{align*}
    X(\omega) = 
    \begin{cases}
        1 & \text{if }\omega \in [0, 1]\\
        2 & \text{if }\omega \in (1, 2]
    \end{cases}
\end{align*}
and
\begin{align*}
    Y(\omega) = 
    \begin{cases}
        2 & \text{if }\omega \in [0, 1.5]\\
        3 & \text{if }\omega \in (1.5, 2].
    \end{cases}
\end{align*}
Then which one of the following statements is true?
\begin{enumerate}
    \item [(A)] $X$ is a random variable with respect to $\mathcal{G}$, but $Y$ is not a random variable with respect to $\mathcal{G}$.
    \item [(B)] $Y$ is a random variable with respect to $\mathcal{G}$, but $X$ is not a random variable with respect to $\mathcal{G}$.
    \item [(C)] Neither $X$ nor $Y$ is a random variable with respect to $\mathcal{G}$.
    \item [(D)] Both $X$ and $Y$ are random variables with respect to $\mathcal{G}$.
\end{enumerate} \hfill (GATE ST 2023)\\
\solution
%\input{gate/ST/2023/14/main.tex}
	\item  A die is loaded in such a way that each odd number is twice as likely to occur as
each even number. Find $P(G)$, where $G$ is the event that a number greater than
3 occurs on a single roll of the die.
\\
\solution
		%\input{exemplar/11/16/3/5/main.tex}
	\item All the jacks, queens and kings are removed from a deck of 52 playing cards. The remaining cards are well shuffled and then one card is drawn at random. Giving ace a value 1 similar value for other cards, find the probability that the card has a value 
		\begin{enumerate}
			\item 7
			\item greater than 7
			\item less than 7
		\end{enumerate}
		%\input{exemplar/10/13/3/30/main.tex}
  \item A Lot consists of 48 mobile phones of which 42 are good, 3 have only minor defects and 3 have major defects.Varnika will buy a phone if it is good but the trader will only buy a mobile if it has no major defects. One phone is selected at random from the lot. What is the probability that it is
\begin{enumerate}
	\item acceptable to Varnika?
            \item acceptable to the trader?
\end{enumerate}
\solution
	%\input{exemplar/10/13/3/40/main.tex}
 \item A student says that if you throw a die, it will show up 1 or not 1. Therefore, the probability of getting 1 and the probability of getting 'not 1' each is equal to $\frac{1}{2}$. Is this correct? Give reasons.\\
 \solution
        %\input{exemplar/10/13/2/9/main.tex}
   \item Four candidates A, B, C, D have ap-
plied for the assignment to coach a school cricket
team. If A is twice as likely to be selected as B, and
B and C are given about the same chance of being
selected, while C is twice as likely to be selected
as D, what are the probabilities that
\begin{enumerate}
\item C will be selected?
\item A will not be selected?
\end{enumerate}
	%\input{exemplar/11/16/3/9/main.tex}
 \item A bag contain 24 balls of which $x$ balls are red, $2x$ are white and $3x$ are blue. A ball is selected at random, What is the probability that it is
\begin{enumerate}[label=\alph*)]
\item not red ?
\item white ?
\end{enumerate}
%\input{exemplar/10/13/3/41/main.tex}
If the letters of the word ASSASSINATION are arranged at random. Find the Probability that
\begin{enumerate}[label=(\alph*)]
\item Four $S's$ come consecutively in the word
\item Two  $I's$ and two $N's$ come together
\item All $A's$ are not coming together
\item No two $A's$ are coming together
\end{enumerate}
%\input{exemplar/11/16/3/14/main.tex}
	\item One urn contains two black balls (labelled B1 and B2) and one white ball. A
	second urn contains one black ball and two white balls (labelled W1 and W2).
	Suppose the following experiment is performed. One of the two urns is chosen
	at random. Next a ball is randomly chosen from the urn. Then a second ball is
	chosen at random from the same urn without replacing the first ball.
	
	\begin{enumerate}
	\item What is the probability that two black balls are chosen?
	
	\item What is the probability that two balls of opposite colour are chosen?
	\end{enumerate}
	\solution
	%\input{exemplar/11/16/3/12/main1.tex}
\end{enumerate}

	\item A bag contains 4 red and 4 black balls, another bag contains 2 red and 6 black balls. One of the two bags is selected at random and a ball is drawn from the bag which is found to be red. Find the probability that the ball is drawn from the first bag.
\\
\solution
		%\begin{table}[H]
	\centering
\begin{tabular}{|c|c|c|}
\hline
Random variable &Value &Definition\\ \hline
\multirow{3}{*}{X} &0 &Slips of Rs 1\\
&1 &Slips of Rs 5\\
&2 &Slips of Rs 13\\ \hline
\multirow{2}{*}{Y} &0 &Box A\\
&1 &Box B\\\hline
\end{tabular}
\caption{}
\label{tab:Distribution}
\end{table}
See \tabref{tab:Distribution}.
\begin{align}
p_{Y}\brak{k}= \begin{cases} 
      \frac{1}{3} & {k=0} \\
      \frac{2}{3 }& {k=1} 
   \end{cases}
   \\
p_{Y|X}\brak{0|0} = \frac{19}{25}\, 
p_{Y|X}\brak{0|1} = \frac{6}{25}\,
p_{Y|X}\brak{1|0} = \frac{45}{50}\,
p_{Y|X}\brak{1|2} = \frac{5}{50}
\end{align}
The desired probability is the probability that a slip drawn at random is marked other than Rs 1,
\begin{align}
&=1-p_X\brak{0}\\
&= p_X(1) + p_X(2)
\end{align}
Using Bayes theorem,
\begin{align}
&= p_Y\brak{0} \times \pr{Y=0 | X=1} + p_Y\brak{1} \times \pr{Y=1|X=2}\\
&=\frac{1}{3} \times \frac{6}{25} + \frac{2}{3} \times \frac{5}{50}\\
&=\frac{11}{75}
\end{align}

\newpage

%\tableofcontents

\bigskip

\renewcommand{\thefigure}{\theenumi}
\renewcommand{\thetable}{\theenumi}
%\renewcommand{\theequation}{\theenumi}

%\begin{abstract}
%%\boldmath
%In this letter, an algorithm for evaluating the exact analytical bit error rate  (BER)  for the piecewise linear (PL) combiner for  multiple relays is presented. Previous results were available only for upto three relays. The algorithm is unique in the sense that  the actual mathematical expressions, that are prohibitively large, need not be explicitly obtained. The diversity gain due to multiple relays is shown through plots of the analytical BER, well supported by simulations. 
%
%\end{abstract}
% IEEEtran.cls defaults to using nonbold math in the Abstract.
% This preserves the distinction between vectors and scalars. However,
% if the journal you are submitting to favors bold math in the abstract,
% then you can use LaTeX's standard command \boldmath at the very start
% of the abstract to achieve this. Many IEEE journals frown on math
% in the abstract anyway.

% Note that keywords are not normally used for peerreview papers.
%\begin{IEEEkeywords}
%Cooperative diversity, decode and forward, piecewise linear
%\end{IEEEkeywords}



% For peer review papers, you can put extra information on the cover
% page as needed:
% \ifCLASSOPTIONpeerreview
% \begin{center} \bfseries EDICS Category: 3-BBND \end{center}
% \fi
%
% For peerreview papers, this IEEEtran command inserts a page break and
% creates the second title. It will be ignored for other modes.
%\IEEEpeerreviewmaketitle




  \item
  Cards with numbers 2 to 101 are placed in a box. A card is selected at random.Find the probability that the card has
\begin{enumerate}[label=(\roman*)]
	\item an even number 
	\item a square number
\end{enumerate}
\solution
%\begin{table}[H]
	\centering
\begin{tabular}{|c|c|c|}
\hline
Random variable &Value &Definition\\ \hline
\multirow{3}{*}{X} &0 &Slips of Rs 1\\
&1 &Slips of Rs 5\\
&2 &Slips of Rs 13\\ \hline
\multirow{2}{*}{Y} &0 &Box A\\
&1 &Box B\\\hline
\end{tabular}
\caption{}
\label{tab:Distribution}
\end{table}
See \tabref{tab:Distribution}.
\begin{align}
p_{Y}\brak{k}= \begin{cases} 
      \frac{1}{3} & {k=0} \\
      \frac{2}{3 }& {k=1} 
   \end{cases}
   \\
p_{Y|X}\brak{0|0} = \frac{19}{25}\, 
p_{Y|X}\brak{0|1} = \frac{6}{25}\,
p_{Y|X}\brak{1|0} = \frac{45}{50}\,
p_{Y|X}\brak{1|2} = \frac{5}{50}
\end{align}
The desired probability is the probability that a slip drawn at random is marked other than Rs 1,
\begin{align}
&=1-p_X\brak{0}\\
&= p_X(1) + p_X(2)
\end{align}
Using Bayes theorem,
\begin{align}
&= p_Y\brak{0} \times \pr{Y=0 | X=1} + p_Y\brak{1} \times \pr{Y=1|X=2}\\
&=\frac{1}{3} \times \frac{6}{25} + \frac{2}{3} \times \frac{5}{50}\\
&=\frac{11}{75}
\end{align}

\newpage

%\tableofcontents

\bigskip

\renewcommand{\thefigure}{\theenumi}
\renewcommand{\thetable}{\theenumi}
%\renewcommand{\theequation}{\theenumi}

%\begin{abstract}
%%\boldmath
%In this letter, an algorithm for evaluating the exact analytical bit error rate  (BER)  for the piecewise linear (PL) combiner for  multiple relays is presented. Previous results were available only for upto three relays. The algorithm is unique in the sense that  the actual mathematical expressions, that are prohibitively large, need not be explicitly obtained. The diversity gain due to multiple relays is shown through plots of the analytical BER, well supported by simulations. 
%
%\end{abstract}
% IEEEtran.cls defaults to using nonbold math in the Abstract.
% This preserves the distinction between vectors and scalars. However,
% if the journal you are submitting to favors bold math in the abstract,
% then you can use LaTeX's standard command \boldmath at the very start
% of the abstract to achieve this. Many IEEE journals frown on math
% in the abstract anyway.

% Note that keywords are not normally used for peerreview papers.
%\begin{IEEEkeywords}
%Cooperative diversity, decode and forward, piecewise linear
%\end{IEEEkeywords}



% For peer review papers, you can put extra information on the cover
% page as needed:
% \ifCLASSOPTIONpeerreview
% \begin{center} \bfseries EDICS Category: 3-BBND \end{center}
% \fi
%
% For peerreview papers, this IEEEtran command inserts a page break and
% creates the second title. It will be ignored for other modes.
%\IEEEpeerreviewmaketitle




\item
The king, queen and jack of clubs are removed from a deck of 52 playing cards and then well shuffled. Now one card is drawn at random from the remaining cards.  Determine the probability that the card is
\begin{enumerate}[label=(\roman*)]
\item a club
\item 10 of hearts
\end{enumerate}
\solution
%\begin{table}[H]
	\centering
\begin{tabular}{|c|c|c|}
\hline
Random variable &Value &Definition\\ \hline
\multirow{3}{*}{X} &0 &Slips of Rs 1\\
&1 &Slips of Rs 5\\
&2 &Slips of Rs 13\\ \hline
\multirow{2}{*}{Y} &0 &Box A\\
&1 &Box B\\\hline
\end{tabular}
\caption{}
\label{tab:Distribution}
\end{table}
See \tabref{tab:Distribution}.
\begin{align}
p_{Y}\brak{k}= \begin{cases} 
      \frac{1}{3} & {k=0} \\
      \frac{2}{3 }& {k=1} 
   \end{cases}
   \\
p_{Y|X}\brak{0|0} = \frac{19}{25}\, 
p_{Y|X}\brak{0|1} = \frac{6}{25}\,
p_{Y|X}\brak{1|0} = \frac{45}{50}\,
p_{Y|X}\brak{1|2} = \frac{5}{50}
\end{align}
The desired probability is the probability that a slip drawn at random is marked other than Rs 1,
\begin{align}
&=1-p_X\brak{0}\\
&= p_X(1) + p_X(2)
\end{align}
Using Bayes theorem,
\begin{align}
&= p_Y\brak{0} \times \pr{Y=0 | X=1} + p_Y\brak{1} \times \pr{Y=1|X=2}\\
&=\frac{1}{3} \times \frac{6}{25} + \frac{2}{3} \times \frac{5}{50}\\
&=\frac{11}{75}
\end{align}

\newpage

%\tableofcontents

\bigskip

\renewcommand{\thefigure}{\theenumi}
\renewcommand{\thetable}{\theenumi}
%\renewcommand{\theequation}{\theenumi}

%\begin{abstract}
%%\boldmath
%In this letter, an algorithm for evaluating the exact analytical bit error rate  (BER)  for the piecewise linear (PL) combiner for  multiple relays is presented. Previous results were available only for upto three relays. The algorithm is unique in the sense that  the actual mathematical expressions, that are prohibitively large, need not be explicitly obtained. The diversity gain due to multiple relays is shown through plots of the analytical BER, well supported by simulations. 
%
%\end{abstract}
% IEEEtran.cls defaults to using nonbold math in the Abstract.
% This preserves the distinction between vectors and scalars. However,
% if the journal you are submitting to favors bold math in the abstract,
% then you can use LaTeX's standard command \boldmath at the very start
% of the abstract to achieve this. Many IEEE journals frown on math
% in the abstract anyway.

% Note that keywords are not normally used for peerreview papers.
%\begin{IEEEkeywords}
%Cooperative diversity, decode and forward, piecewise linear
%\end{IEEEkeywords}



% For peer review papers, you can put extra information on the cover
% page as needed:
% \ifCLASSOPTIONpeerreview
% \begin{center} \bfseries EDICS Category: 3-BBND \end{center}
% \fi
%
% For peerreview papers, this IEEEtran command inserts a page break and
% creates the second title. It will be ignored for other modes.
%\IEEEpeerreviewmaketitle




\item A team of medical students doing their internship have to assist during surgeries
at a city hospital. The probabilities of surgeries rated as very complex, complex,
routine, simple or very simple are respectively, 0.15, 0.20, 0.31, 0.26, .08. Find
the probabilities that a particular surgery will be rated
\begin{enumerate}
	\item complex or very complex;
	\item neither very complex nor very simple;
	\item routine or complex
	\item routine or simple
\end{enumerate}
\solution
%\begin{table}[H]
	\centering
\begin{tabular}{|c|c|c|}
\hline
Random variable &Value &Definition\\ \hline
\multirow{3}{*}{X} &0 &Slips of Rs 1\\
&1 &Slips of Rs 5\\
&2 &Slips of Rs 13\\ \hline
\multirow{2}{*}{Y} &0 &Box A\\
&1 &Box B\\\hline
\end{tabular}
\caption{}
\label{tab:Distribution}
\end{table}
See \tabref{tab:Distribution}.
\begin{align}
p_{Y}\brak{k}= \begin{cases} 
      \frac{1}{3} & {k=0} \\
      \frac{2}{3 }& {k=1} 
   \end{cases}
   \\
p_{Y|X}\brak{0|0} = \frac{19}{25}\, 
p_{Y|X}\brak{0|1} = \frac{6}{25}\,
p_{Y|X}\brak{1|0} = \frac{45}{50}\,
p_{Y|X}\brak{1|2} = \frac{5}{50}
\end{align}
The desired probability is the probability that a slip drawn at random is marked other than Rs 1,
\begin{align}
&=1-p_X\brak{0}\\
&= p_X(1) + p_X(2)
\end{align}
Using Bayes theorem,
\begin{align}
&= p_Y\brak{0} \times \pr{Y=0 | X=1} + p_Y\brak{1} \times \pr{Y=1|X=2}\\
&=\frac{1}{3} \times \frac{6}{25} + \frac{2}{3} \times \frac{5}{50}\\
&=\frac{11}{75}
\end{align}

\newpage

%\tableofcontents

\bigskip

\renewcommand{\thefigure}{\theenumi}
\renewcommand{\thetable}{\theenumi}
%\renewcommand{\theequation}{\theenumi}

%\begin{abstract}
%%\boldmath
%In this letter, an algorithm for evaluating the exact analytical bit error rate  (BER)  for the piecewise linear (PL) combiner for  multiple relays is presented. Previous results were available only for upto three relays. The algorithm is unique in the sense that  the actual mathematical expressions, that are prohibitively large, need not be explicitly obtained. The diversity gain due to multiple relays is shown through plots of the analytical BER, well supported by simulations. 
%
%\end{abstract}
% IEEEtran.cls defaults to using nonbold math in the Abstract.
% This preserves the distinction between vectors and scalars. However,
% if the journal you are submitting to favors bold math in the abstract,
% then you can use LaTeX's standard command \boldmath at the very start
% of the abstract to achieve this. Many IEEE journals frown on math
% in the abstract anyway.

% Note that keywords are not normally used for peerreview papers.
%\begin{IEEEkeywords}
%Cooperative diversity, decode and forward, piecewise linear
%\end{IEEEkeywords}



% For peer review papers, you can put extra information on the cover
% page as needed:
% \ifCLASSOPTIONpeerreview
% \begin{center} \bfseries EDICS Category: 3-BBND \end{center}
% \fi
%
% For peerreview papers, this IEEEtran command inserts a page break and
% creates the second title. It will be ignored for other modes.
%\IEEEpeerreviewmaketitle




\item A card is selected from a pack of 52 cards.
\begin{enumerate}[label=(\alph*)]
    \item How many points are there in the sample space?
    \item Calculate the probability that the card is an ace of spades.
    \item Calculate the probability that the card is (i) an ace and (ii) black card.
\end{enumerate}
\solution
%Let $X$ be an bernoulli rv defined as in \tabref{tab:exemplar/11/16/3/26}.  Then, 
\begin{equation}
    p =
        \frac{4}{11} 
\end{equation}
\begin{table}[H]
	\centering
	\input{exemplar/11/16/3/26/tables/Table2.tex}
	\caption{}
        \label{tab:exemplar/11/16/3/26}
\end{table}

\item The probability that a non leap year selected at random will contain 53 sundays.
\\
\solution
%\begin{table}[H]
	\centering
\begin{tabular}{|c|c|c|}
\hline
Random variable &Value &Definition\\ \hline
\multirow{3}{*}{X} &0 &Slips of Rs 1\\
&1 &Slips of Rs 5\\
&2 &Slips of Rs 13\\ \hline
\multirow{2}{*}{Y} &0 &Box A\\
&1 &Box B\\\hline
\end{tabular}
\caption{}
\label{tab:Distribution}
\end{table}
See \tabref{tab:Distribution}.
\begin{align}
p_{Y}\brak{k}= \begin{cases} 
      \frac{1}{3} & {k=0} \\
      \frac{2}{3 }& {k=1} 
   \end{cases}
   \\
p_{Y|X}\brak{0|0} = \frac{19}{25}\, 
p_{Y|X}\brak{0|1} = \frac{6}{25}\,
p_{Y|X}\brak{1|0} = \frac{45}{50}\,
p_{Y|X}\brak{1|2} = \frac{5}{50}
\end{align}
The desired probability is the probability that a slip drawn at random is marked other than Rs 1,
\begin{align}
&=1-p_X\brak{0}\\
&= p_X(1) + p_X(2)
\end{align}
Using Bayes theorem,
\begin{align}
&= p_Y\brak{0} \times \pr{Y=0 | X=1} + p_Y\brak{1} \times \pr{Y=1|X=2}\\
&=\frac{1}{3} \times \frac{6}{25} + \frac{2}{3} \times \frac{5}{50}\\
&=\frac{11}{75}
\end{align}

\newpage

%\tableofcontents

\bigskip

\renewcommand{\thefigure}{\theenumi}
\renewcommand{\thetable}{\theenumi}
%\renewcommand{\theequation}{\theenumi}

%\begin{abstract}
%%\boldmath
%In this letter, an algorithm for evaluating the exact analytical bit error rate  (BER)  for the piecewise linear (PL) combiner for  multiple relays is presented. Previous results were available only for upto three relays. The algorithm is unique in the sense that  the actual mathematical expressions, that are prohibitively large, need not be explicitly obtained. The diversity gain due to multiple relays is shown through plots of the analytical BER, well supported by simulations. 
%
%\end{abstract}
% IEEEtran.cls defaults to using nonbold math in the Abstract.
% This preserves the distinction between vectors and scalars. However,
% if the journal you are submitting to favors bold math in the abstract,
% then you can use LaTeX's standard command \boldmath at the very start
% of the abstract to achieve this. Many IEEE journals frown on math
% in the abstract anyway.

% Note that keywords are not normally used for peerreview papers.
%\begin{IEEEkeywords}
%Cooperative diversity, decode and forward, piecewise linear
%\end{IEEEkeywords}



% For peer review papers, you can put extra information on the cover
% page as needed:
% \ifCLASSOPTIONpeerreview
% \begin{center} \bfseries EDICS Category: 3-BBND \end{center}
% \fi
%
% For peerreview papers, this IEEEtran command inserts a page break and
% creates the second title. It will be ignored for other modes.
%\IEEEpeerreviewmaketitle




\item One of the four persons John, Rita, Aslam or Gurpreet will be promoted next
month. Consequently the sample space consists of four elementary outcomes
S = {John promoted, Rita promoted, Aslam promoted, Gurpreet promoted}
You are told that the chances of John’s promotion is same as that of Gurpreet,
Rita’s chances of promotion are twice as likely as Johns. Aslam’s chances are
four times that of John.
\begin{enumerate}
	\item Determine
	\begin{enumerate}
		\item P (John promoted)
		\item P (Rita promoted)
		\item P (Aslam promoted)
		\item P (Gurpreet promoted)
	\end{enumerate}
	\item If A = {John promoted or Gurpreet promoted}, find P (A).
\end{enumerate}
\solution
%\begin{table}[H]
	\centering
\begin{tabular}{|c|c|c|}
\hline
Random variable &Value &Definition\\ \hline
\multirow{3}{*}{X} &0 &Slips of Rs 1\\
&1 &Slips of Rs 5\\
&2 &Slips of Rs 13\\ \hline
\multirow{2}{*}{Y} &0 &Box A\\
&1 &Box B\\\hline
\end{tabular}
\caption{}
\label{tab:Distribution}
\end{table}
See \tabref{tab:Distribution}.
\begin{align}
p_{Y}\brak{k}= \begin{cases} 
      \frac{1}{3} & {k=0} \\
      \frac{2}{3 }& {k=1} 
   \end{cases}
   \\
p_{Y|X}\brak{0|0} = \frac{19}{25}\, 
p_{Y|X}\brak{0|1} = \frac{6}{25}\,
p_{Y|X}\brak{1|0} = \frac{45}{50}\,
p_{Y|X}\brak{1|2} = \frac{5}{50}
\end{align}
The desired probability is the probability that a slip drawn at random is marked other than Rs 1,
\begin{align}
&=1-p_X\brak{0}\\
&= p_X(1) + p_X(2)
\end{align}
Using Bayes theorem,
\begin{align}
&= p_Y\brak{0} \times \pr{Y=0 | X=1} + p_Y\brak{1} \times \pr{Y=1|X=2}\\
&=\frac{1}{3} \times \frac{6}{25} + \frac{2}{3} \times \frac{5}{50}\\
&=\frac{11}{75}
\end{align}

\newpage

%\tableofcontents

\bigskip

\renewcommand{\thefigure}{\theenumi}
\renewcommand{\thetable}{\theenumi}
%\renewcommand{\theequation}{\theenumi}

%\begin{abstract}
%%\boldmath
%In this letter, an algorithm for evaluating the exact analytical bit error rate  (BER)  for the piecewise linear (PL) combiner for  multiple relays is presented. Previous results were available only for upto three relays. The algorithm is unique in the sense that  the actual mathematical expressions, that are prohibitively large, need not be explicitly obtained. The diversity gain due to multiple relays is shown through plots of the analytical BER, well supported by simulations. 
%
%\end{abstract}
% IEEEtran.cls defaults to using nonbold math in the Abstract.
% This preserves the distinction between vectors and scalars. However,
% if the journal you are submitting to favors bold math in the abstract,
% then you can use LaTeX's standard command \boldmath at the very start
% of the abstract to achieve this. Many IEEE journals frown on math
% in the abstract anyway.

% Note that keywords are not normally used for peerreview papers.
%\begin{IEEEkeywords}
%Cooperative diversity, decode and forward, piecewise linear
%\end{IEEEkeywords}



% For peer review papers, you can put extra information on the cover
% page as needed:
% \ifCLASSOPTIONpeerreview
% \begin{center} \bfseries EDICS Category: 3-BBND \end{center}
% \fi
%
% For peerreview papers, this IEEEtran command inserts a page break and
% creates the second title. It will be ignored for other modes.
%\IEEEpeerreviewmaketitle




\item A card is drawn from a deck of 52 cards. Find the probability of getting a king or a heart or a red card.\\
\solution
%\begin{table}[H]
	\centering
\begin{tabular}{|c|c|c|}
\hline
Random variable &Value &Definition\\ \hline
\multirow{3}{*}{X} &0 &Slips of Rs 1\\
&1 &Slips of Rs 5\\
&2 &Slips of Rs 13\\ \hline
\multirow{2}{*}{Y} &0 &Box A\\
&1 &Box B\\\hline
\end{tabular}
\caption{}
\label{tab:Distribution}
\end{table}
See \tabref{tab:Distribution}.
\begin{align}
p_{Y}\brak{k}= \begin{cases} 
      \frac{1}{3} & {k=0} \\
      \frac{2}{3 }& {k=1} 
   \end{cases}
   \\
p_{Y|X}\brak{0|0} = \frac{19}{25}\, 
p_{Y|X}\brak{0|1} = \frac{6}{25}\,
p_{Y|X}\brak{1|0} = \frac{45}{50}\,
p_{Y|X}\brak{1|2} = \frac{5}{50}
\end{align}
The desired probability is the probability that a slip drawn at random is marked other than Rs 1,
\begin{align}
&=1-p_X\brak{0}\\
&= p_X(1) + p_X(2)
\end{align}
Using Bayes theorem,
\begin{align}
&= p_Y\brak{0} \times \pr{Y=0 | X=1} + p_Y\brak{1} \times \pr{Y=1|X=2}\\
&=\frac{1}{3} \times \frac{6}{25} + \frac{2}{3} \times \frac{5}{50}\\
&=\frac{11}{75}
\end{align}

\newpage

%\tableofcontents

\bigskip

\renewcommand{\thefigure}{\theenumi}
\renewcommand{\thetable}{\theenumi}
%\renewcommand{\theequation}{\theenumi}

%\begin{abstract}
%%\boldmath
%In this letter, an algorithm for evaluating the exact analytical bit error rate  (BER)  for the piecewise linear (PL) combiner for  multiple relays is presented. Previous results were available only for upto three relays. The algorithm is unique in the sense that  the actual mathematical expressions, that are prohibitively large, need not be explicitly obtained. The diversity gain due to multiple relays is shown through plots of the analytical BER, well supported by simulations. 
%
%\end{abstract}
% IEEEtran.cls defaults to using nonbold math in the Abstract.
% This preserves the distinction between vectors and scalars. However,
% if the journal you are submitting to favors bold math in the abstract,
% then you can use LaTeX's standard command \boldmath at the very start
% of the abstract to achieve this. Many IEEE journals frown on math
% in the abstract anyway.

% Note that keywords are not normally used for peerreview papers.
%\begin{IEEEkeywords}
%Cooperative diversity, decode and forward, piecewise linear
%\end{IEEEkeywords}



% For peer review papers, you can put extra information on the cover
% page as needed:
% \ifCLASSOPTIONpeerreview
% \begin{center} \bfseries EDICS Category: 3-BBND \end{center}
% \fi
%
% For peerreview papers, this IEEEtran command inserts a page break and
% creates the second title. It will be ignored for other modes.
%\IEEEpeerreviewmaketitle




\item The probability that a student will pass his examination is 0.73, the probability of
the student getting a compartment is 0.13, and the probability that the student will
either pass or get compartment is 0.96. State True or False.\\
\solution
%\begin{table}[H]
	\centering
\begin{tabular}{|c|c|c|}
\hline
Random variable &Value &Definition\\ \hline
\multirow{3}{*}{X} &0 &Slips of Rs 1\\
&1 &Slips of Rs 5\\
&2 &Slips of Rs 13\\ \hline
\multirow{2}{*}{Y} &0 &Box A\\
&1 &Box B\\\hline
\end{tabular}
\caption{}
\label{tab:Distribution}
\end{table}
See \tabref{tab:Distribution}.
\begin{align}
p_{Y}\brak{k}= \begin{cases} 
      \frac{1}{3} & {k=0} \\
      \frac{2}{3 }& {k=1} 
   \end{cases}
   \\
p_{Y|X}\brak{0|0} = \frac{19}{25}\, 
p_{Y|X}\brak{0|1} = \frac{6}{25}\,
p_{Y|X}\brak{1|0} = \frac{45}{50}\,
p_{Y|X}\brak{1|2} = \frac{5}{50}
\end{align}
The desired probability is the probability that a slip drawn at random is marked other than Rs 1,
\begin{align}
&=1-p_X\brak{0}\\
&= p_X(1) + p_X(2)
\end{align}
Using Bayes theorem,
\begin{align}
&= p_Y\brak{0} \times \pr{Y=0 | X=1} + p_Y\brak{1} \times \pr{Y=1|X=2}\\
&=\frac{1}{3} \times \frac{6}{25} + \frac{2}{3} \times \frac{5}{50}\\
&=\frac{11}{75}
\end{align}

\newpage

%\tableofcontents

\bigskip

\renewcommand{\thefigure}{\theenumi}
\renewcommand{\thetable}{\theenumi}
%\renewcommand{\theequation}{\theenumi}

%\begin{abstract}
%%\boldmath
%In this letter, an algorithm for evaluating the exact analytical bit error rate  (BER)  for the piecewise linear (PL) combiner for  multiple relays is presented. Previous results were available only for upto three relays. The algorithm is unique in the sense that  the actual mathematical expressions, that are prohibitively large, need not be explicitly obtained. The diversity gain due to multiple relays is shown through plots of the analytical BER, well supported by simulations. 
%
%\end{abstract}
% IEEEtran.cls defaults to using nonbold math in the Abstract.
% This preserves the distinction between vectors and scalars. However,
% if the journal you are submitting to favors bold math in the abstract,
% then you can use LaTeX's standard command \boldmath at the very start
% of the abstract to achieve this. Many IEEE journals frown on math
% in the abstract anyway.

% Note that keywords are not normally used for peerreview papers.
%\begin{IEEEkeywords}
%Cooperative diversity, decode and forward, piecewise linear
%\end{IEEEkeywords}



% For peer review papers, you can put extra information on the cover
% page as needed:
% \ifCLASSOPTIONpeerreview
% \begin{center} \bfseries EDICS Category: 3-BBND \end{center}
% \fi
%
% For peerreview papers, this IEEEtran command inserts a page break and
% creates the second title. It will be ignored for other modes.
%\IEEEpeerreviewmaketitle




\item A card is selected from a pack of 52 cards\\
\begin{enumerate}[label=(\alph*)]
\item How many points are there in the sample space?
\item Calculate the probability that the cards is an ace of spades.
\item Calculate the probability that the card is (i) an ace (ii)black card.\\
\end{enumerate}
%\input{ncert/11/16/3/4_1/Prob_4.tex}
\item In a non-leap year, the probability of having 53 tuesdays or 53 wednesdays is\\
\solution
%A non-leap year has a total of 365 days, and a week has 7 days.\\
So it can be expressed as 
\begin{align}
365\text{days} &=52\times 7+1 \text{day}
\end{align}
$\implies$ 52 tuesdays or wednesdays\\
Random variable X denotes the days of a week
\begin{align}
p_X\brak{k}&=\frac{1}{7}; \quad \brak{1<k<7}
\end{align}
So the probability of extra day being tuesday or wednesday is
\begin{align}
p_X\brak{3}+p_X\brak{4}&=\frac{1}{7}+\frac{1}{7}=\frac{2}{7}
\end{align}



\item There are 1000 sealed envelopes in a box, 10 of them contain a cash prize of
Rs 100 each, 100 of them contain a cash prize of Rs 50 each and 200 of them
contain a cash prize of Rs 10 each and rest do not contain any cash prize. If they
are well shuffled and an envelope is picked up out, what is the probability that it
contains no cash prize?\\
\solution
%\begin{table}[H]
	\centering
\begin{tabular}{|c|c|c|}
\hline
Random variable &Value &Definition\\ \hline
\multirow{3}{*}{X} &0 &Slips of Rs 1\\
&1 &Slips of Rs 5\\
&2 &Slips of Rs 13\\ \hline
\multirow{2}{*}{Y} &0 &Box A\\
&1 &Box B\\\hline
\end{tabular}
\caption{}
\label{tab:Distribution}
\end{table}
See \tabref{tab:Distribution}.
\begin{align}
p_{Y}\brak{k}= \begin{cases} 
      \frac{1}{3} & {k=0} \\
      \frac{2}{3 }& {k=1} 
   \end{cases}
   \\
p_{Y|X}\brak{0|0} = \frac{19}{25}\, 
p_{Y|X}\brak{0|1} = \frac{6}{25}\,
p_{Y|X}\brak{1|0} = \frac{45}{50}\,
p_{Y|X}\brak{1|2} = \frac{5}{50}
\end{align}
The desired probability is the probability that a slip drawn at random is marked other than Rs 1,
\begin{align}
&=1-p_X\brak{0}\\
&= p_X(1) + p_X(2)
\end{align}
Using Bayes theorem,
\begin{align}
&= p_Y\brak{0} \times \pr{Y=0 | X=1} + p_Y\brak{1} \times \pr{Y=1|X=2}\\
&=\frac{1}{3} \times \frac{6}{25} + \frac{2}{3} \times \frac{5}{50}\\
&=\frac{11}{75}
\end{align}

\newpage

%\tableofcontents

\bigskip

\renewcommand{\thefigure}{\theenumi}
\renewcommand{\thetable}{\theenumi}
%\renewcommand{\theequation}{\theenumi}

%\begin{abstract}
%%\boldmath
%In this letter, an algorithm for evaluating the exact analytical bit error rate  (BER)  for the piecewise linear (PL) combiner for  multiple relays is presented. Previous results were available only for upto three relays. The algorithm is unique in the sense that  the actual mathematical expressions, that are prohibitively large, need not be explicitly obtained. The diversity gain due to multiple relays is shown through plots of the analytical BER, well supported by simulations. 
%
%\end{abstract}
% IEEEtran.cls defaults to using nonbold math in the Abstract.
% This preserves the distinction between vectors and scalars. However,
% if the journal you are submitting to favors bold math in the abstract,
% then you can use LaTeX's standard command \boldmath at the very start
% of the abstract to achieve this. Many IEEE journals frown on math
% in the abstract anyway.

% Note that keywords are not normally used for peerreview papers.
%\begin{IEEEkeywords}
%Cooperative diversity, decode and forward, piecewise linear
%\end{IEEEkeywords}



% For peer review papers, you can put extra information on the cover
% page as needed:
% \ifCLASSOPTIONpeerreview
% \begin{center} \bfseries EDICS Category: 3-BBND \end{center}
% \fi
%
% For peerreview papers, this IEEEtran command inserts a page break and
% creates the second title. It will be ignored for other modes.
%\IEEEpeerreviewmaketitle




\item 
A die is thrown and a card is selected at random from a deck of 52 playing cards. The probability of getting an even number on the die and a spade card.\\
\solution
%\begin{table}[H]
	\centering
\begin{tabular}{|c|c|c|}
\hline
Random variable &Value &Definition\\ \hline
\multirow{3}{*}{X} &0 &Slips of Rs 1\\
&1 &Slips of Rs 5\\
&2 &Slips of Rs 13\\ \hline
\multirow{2}{*}{Y} &0 &Box A\\
&1 &Box B\\\hline
\end{tabular}
\caption{}
\label{tab:Distribution}
\end{table}
See \tabref{tab:Distribution}.
\begin{align}
p_{Y}\brak{k}= \begin{cases} 
      \frac{1}{3} & {k=0} \\
      \frac{2}{3 }& {k=1} 
   \end{cases}
   \\
p_{Y|X}\brak{0|0} = \frac{19}{25}\, 
p_{Y|X}\brak{0|1} = \frac{6}{25}\,
p_{Y|X}\brak{1|0} = \frac{45}{50}\,
p_{Y|X}\brak{1|2} = \frac{5}{50}
\end{align}
The desired probability is the probability that a slip drawn at random is marked other than Rs 1,
\begin{align}
&=1-p_X\brak{0}\\
&= p_X(1) + p_X(2)
\end{align}
Using Bayes theorem,
\begin{align}
&= p_Y\brak{0} \times \pr{Y=0 | X=1} + p_Y\brak{1} \times \pr{Y=1|X=2}\\
&=\frac{1}{3} \times \frac{6}{25} + \frac{2}{3} \times \frac{5}{50}\\
&=\frac{11}{75}
\end{align}

\newpage

%\tableofcontents

\bigskip

\renewcommand{\thefigure}{\theenumi}
\renewcommand{\thetable}{\theenumi}
%\renewcommand{\theequation}{\theenumi}

%\begin{abstract}
%%\boldmath
%In this letter, an algorithm for evaluating the exact analytical bit error rate  (BER)  for the piecewise linear (PL) combiner for  multiple relays is presented. Previous results were available only for upto three relays. The algorithm is unique in the sense that  the actual mathematical expressions, that are prohibitively large, need not be explicitly obtained. The diversity gain due to multiple relays is shown through plots of the analytical BER, well supported by simulations. 
%
%\end{abstract}
% IEEEtran.cls defaults to using nonbold math in the Abstract.
% This preserves the distinction between vectors and scalars. However,
% if the journal you are submitting to favors bold math in the abstract,
% then you can use LaTeX's standard command \boldmath at the very start
% of the abstract to achieve this. Many IEEE journals frown on math
% in the abstract anyway.

% Note that keywords are not normally used for peerreview papers.
%\begin{IEEEkeywords}
%Cooperative diversity, decode and forward, piecewise linear
%\end{IEEEkeywords}



% For peer review papers, you can put extra information on the cover
% page as needed:
% \ifCLASSOPTIONpeerreview
% \begin{center} \bfseries EDICS Category: 3-BBND \end{center}
% \fi
%
% For peerreview papers, this IEEEtran command inserts a page break and
% creates the second title. It will be ignored for other modes.
%\IEEEpeerreviewmaketitle




\item
If 4-digit numbers greater than 5,000 are randomly formed from the digits 0, 1, 3, 5, and 7, what is the probability of forming a number divisible by 5 when:
\begin{enumerate}
    \item The digits are repeated?
    \item The repetition of digits is not allowed?
\end{enumerate}
\solution
%\begin{table}[H]
	\centering
\begin{tabular}{|c|c|c|}
\hline
Random variable &Value &Definition\\ \hline
\multirow{3}{*}{X} &0 &Slips of Rs 1\\
&1 &Slips of Rs 5\\
&2 &Slips of Rs 13\\ \hline
\multirow{2}{*}{Y} &0 &Box A\\
&1 &Box B\\\hline
\end{tabular}
\caption{}
\label{tab:Distribution}
\end{table}
See \tabref{tab:Distribution}.
\begin{align}
p_{Y}\brak{k}= \begin{cases} 
      \frac{1}{3} & {k=0} \\
      \frac{2}{3 }& {k=1} 
   \end{cases}
   \\
p_{Y|X}\brak{0|0} = \frac{19}{25}\, 
p_{Y|X}\brak{0|1} = \frac{6}{25}\,
p_{Y|X}\brak{1|0} = \frac{45}{50}\,
p_{Y|X}\brak{1|2} = \frac{5}{50}
\end{align}
The desired probability is the probability that a slip drawn at random is marked other than Rs 1,
\begin{align}
&=1-p_X\brak{0}\\
&= p_X(1) + p_X(2)
\end{align}
Using Bayes theorem,
\begin{align}
&= p_Y\brak{0} \times \pr{Y=0 | X=1} + p_Y\brak{1} \times \pr{Y=1|X=2}\\
&=\frac{1}{3} \times \frac{6}{25} + \frac{2}{3} \times \frac{5}{50}\\
&=\frac{11}{75}
\end{align}

\newpage

%\tableofcontents

\bigskip

\renewcommand{\thefigure}{\theenumi}
\renewcommand{\thetable}{\theenumi}
%\renewcommand{\theequation}{\theenumi}

%\begin{abstract}
%%\boldmath
%In this letter, an algorithm for evaluating the exact analytical bit error rate  (BER)  for the piecewise linear (PL) combiner for  multiple relays is presented. Previous results were available only for upto three relays. The algorithm is unique in the sense that  the actual mathematical expressions, that are prohibitively large, need not be explicitly obtained. The diversity gain due to multiple relays is shown through plots of the analytical BER, well supported by simulations. 
%
%\end{abstract}
% IEEEtran.cls defaults to using nonbold math in the Abstract.
% This preserves the distinction between vectors and scalars. However,
% if the journal you are submitting to favors bold math in the abstract,
% then you can use LaTeX's standard command \boldmath at the very start
% of the abstract to achieve this. Many IEEE journals frown on math
% in the abstract anyway.

% Note that keywords are not normally used for peerreview papers.
%\begin{IEEEkeywords}
%Cooperative diversity, decode and forward, piecewise linear
%\end{IEEEkeywords}



% For peer review papers, you can put extra information on the cover
% page as needed:
% \ifCLASSOPTIONpeerreview
% \begin{center} \bfseries EDICS Category: 3-BBND \end{center}
% \fi
%
% For peerreview papers, this IEEEtran command inserts a page break and
% creates the second title. It will be ignored for other modes.
%\IEEEpeerreviewmaketitle




\item Consider the probability space $\brak{\Omega, \mathcal{G}, P}$ where $\Omega = [0,2]$ and $\mathcal{G} = \cbrak{\phi, \Omega, [0,1], (1,2]}$. Let $X$ and $Y$ be two functions on $\Omega$ defined as
\begin{align*}
    X(\omega) = 
    \begin{cases}
        1 & \text{if }\omega \in [0, 1]\\
        2 & \text{if }\omega \in (1, 2]
    \end{cases}
\end{align*}
and
\begin{align*}
    Y(\omega) = 
    \begin{cases}
        2 & \text{if }\omega \in [0, 1.5]\\
        3 & \text{if }\omega \in (1.5, 2].
    \end{cases}
\end{align*}
Then which one of the following statements is true?
\begin{enumerate}
    \item [(A)] $X$ is a random variable with respect to $\mathcal{G}$, but $Y$ is not a random variable with respect to $\mathcal{G}$.
    \item [(B)] $Y$ is a random variable with respect to $\mathcal{G}$, but $X$ is not a random variable with respect to $\mathcal{G}$.
    \item [(C)] Neither $X$ nor $Y$ is a random variable with respect to $\mathcal{G}$.
    \item [(D)] Both $X$ and $Y$ are random variables with respect to $\mathcal{G}$.
\end{enumerate} \hfill (GATE ST 2023)\\
\solution
%\begin{table}[H]
	\centering
\begin{tabular}{|c|c|c|}
\hline
Random variable &Value &Definition\\ \hline
\multirow{3}{*}{X} &0 &Slips of Rs 1\\
&1 &Slips of Rs 5\\
&2 &Slips of Rs 13\\ \hline
\multirow{2}{*}{Y} &0 &Box A\\
&1 &Box B\\\hline
\end{tabular}
\caption{}
\label{tab:Distribution}
\end{table}
See \tabref{tab:Distribution}.
\begin{align}
p_{Y}\brak{k}= \begin{cases} 
      \frac{1}{3} & {k=0} \\
      \frac{2}{3 }& {k=1} 
   \end{cases}
   \\
p_{Y|X}\brak{0|0} = \frac{19}{25}\, 
p_{Y|X}\brak{0|1} = \frac{6}{25}\,
p_{Y|X}\brak{1|0} = \frac{45}{50}\,
p_{Y|X}\brak{1|2} = \frac{5}{50}
\end{align}
The desired probability is the probability that a slip drawn at random is marked other than Rs 1,
\begin{align}
&=1-p_X\brak{0}\\
&= p_X(1) + p_X(2)
\end{align}
Using Bayes theorem,
\begin{align}
&= p_Y\brak{0} \times \pr{Y=0 | X=1} + p_Y\brak{1} \times \pr{Y=1|X=2}\\
&=\frac{1}{3} \times \frac{6}{25} + \frac{2}{3} \times \frac{5}{50}\\
&=\frac{11}{75}
\end{align}

\newpage

%\tableofcontents

\bigskip

\renewcommand{\thefigure}{\theenumi}
\renewcommand{\thetable}{\theenumi}
%\renewcommand{\theequation}{\theenumi}

%\begin{abstract}
%%\boldmath
%In this letter, an algorithm for evaluating the exact analytical bit error rate  (BER)  for the piecewise linear (PL) combiner for  multiple relays is presented. Previous results were available only for upto three relays. The algorithm is unique in the sense that  the actual mathematical expressions, that are prohibitively large, need not be explicitly obtained. The diversity gain due to multiple relays is shown through plots of the analytical BER, well supported by simulations. 
%
%\end{abstract}
% IEEEtran.cls defaults to using nonbold math in the Abstract.
% This preserves the distinction between vectors and scalars. However,
% if the journal you are submitting to favors bold math in the abstract,
% then you can use LaTeX's standard command \boldmath at the very start
% of the abstract to achieve this. Many IEEE journals frown on math
% in the abstract anyway.

% Note that keywords are not normally used for peerreview papers.
%\begin{IEEEkeywords}
%Cooperative diversity, decode and forward, piecewise linear
%\end{IEEEkeywords}



% For peer review papers, you can put extra information on the cover
% page as needed:
% \ifCLASSOPTIONpeerreview
% \begin{center} \bfseries EDICS Category: 3-BBND \end{center}
% \fi
%
% For peerreview papers, this IEEEtran command inserts a page break and
% creates the second title. It will be ignored for other modes.
%\IEEEpeerreviewmaketitle




	\item  A die is loaded in such a way that each odd number is twice as likely to occur as
each even number. Find $P(G)$, where $G$ is the event that a number greater than
3 occurs on a single roll of the die.
\\
\solution
		%\begin{table}[H]
	\centering
\begin{tabular}{|c|c|c|}
\hline
Random variable &Value &Definition\\ \hline
\multirow{3}{*}{X} &0 &Slips of Rs 1\\
&1 &Slips of Rs 5\\
&2 &Slips of Rs 13\\ \hline
\multirow{2}{*}{Y} &0 &Box A\\
&1 &Box B\\\hline
\end{tabular}
\caption{}
\label{tab:Distribution}
\end{table}
See \tabref{tab:Distribution}.
\begin{align}
p_{Y}\brak{k}= \begin{cases} 
      \frac{1}{3} & {k=0} \\
      \frac{2}{3 }& {k=1} 
   \end{cases}
   \\
p_{Y|X}\brak{0|0} = \frac{19}{25}\, 
p_{Y|X}\brak{0|1} = \frac{6}{25}\,
p_{Y|X}\brak{1|0} = \frac{45}{50}\,
p_{Y|X}\brak{1|2} = \frac{5}{50}
\end{align}
The desired probability is the probability that a slip drawn at random is marked other than Rs 1,
\begin{align}
&=1-p_X\brak{0}\\
&= p_X(1) + p_X(2)
\end{align}
Using Bayes theorem,
\begin{align}
&= p_Y\brak{0} \times \pr{Y=0 | X=1} + p_Y\brak{1} \times \pr{Y=1|X=2}\\
&=\frac{1}{3} \times \frac{6}{25} + \frac{2}{3} \times \frac{5}{50}\\
&=\frac{11}{75}
\end{align}

\newpage

%\tableofcontents

\bigskip

\renewcommand{\thefigure}{\theenumi}
\renewcommand{\thetable}{\theenumi}
%\renewcommand{\theequation}{\theenumi}

%\begin{abstract}
%%\boldmath
%In this letter, an algorithm for evaluating the exact analytical bit error rate  (BER)  for the piecewise linear (PL) combiner for  multiple relays is presented. Previous results were available only for upto three relays. The algorithm is unique in the sense that  the actual mathematical expressions, that are prohibitively large, need not be explicitly obtained. The diversity gain due to multiple relays is shown through plots of the analytical BER, well supported by simulations. 
%
%\end{abstract}
% IEEEtran.cls defaults to using nonbold math in the Abstract.
% This preserves the distinction between vectors and scalars. However,
% if the journal you are submitting to favors bold math in the abstract,
% then you can use LaTeX's standard command \boldmath at the very start
% of the abstract to achieve this. Many IEEE journals frown on math
% in the abstract anyway.

% Note that keywords are not normally used for peerreview papers.
%\begin{IEEEkeywords}
%Cooperative diversity, decode and forward, piecewise linear
%\end{IEEEkeywords}



% For peer review papers, you can put extra information on the cover
% page as needed:
% \ifCLASSOPTIONpeerreview
% \begin{center} \bfseries EDICS Category: 3-BBND \end{center}
% \fi
%
% For peerreview papers, this IEEEtran command inserts a page break and
% creates the second title. It will be ignored for other modes.
%\IEEEpeerreviewmaketitle




	\item All the jacks, queens and kings are removed from a deck of 52 playing cards. The remaining cards are well shuffled and then one card is drawn at random. Giving ace a value 1 similar value for other cards, find the probability that the card has a value 
		\begin{enumerate}
			\item 7
			\item greater than 7
			\item less than 7
		\end{enumerate}
		%Number of cards left after removing all jacks, queens and kings 
\begin{align}
N	= 52 - 4\times 3
	= 40
\end{align}
%\begin{table}[H]
%\def\arraystretch{1.2}
%\begin{tabular}{|c|c|c|}
%\hline
%	\textbf{Parameter} &\textbf{Value} &\textbf{Description}\\ \hline
%	$X$ &1-10 &Represents the value of the card picked \\ \hline
%\end{tabular}
%\end{table}
Let $1 \le X \le 10$ be the value of the card picked.  Then,
\begin{align}
	p_X(k) &= \Pr(X=k)\ \forall\ 1 \leq k \leq 10\\
	&= \frac{4\times 1}{40}\\
	&= \frac{1}{10}\\
	\therefore p_X(k) &= 
	\begin{cases}
		\frac{1}{10} & 1 \leq k \leq 10\\
		0 & \text{otherwise}
	\end{cases}
\end{align}
and
\begin{align}
	F_{X}(k) &= \sum_{m=0}^{k}p_{X}(m) \quad 1 \leq k \leq 10\\
	&= \frac{k}{10}\\
	\therefore F_{X}(k) &= 
	\begin{cases}
		0 & k \leq 0\\
		\frac{k}{10} & 1\leq k \leq 10\\
		1 & k > 10 
	\end{cases}
\end{align}
\begin{enumerate}
	\item Probability that card has value equal to 7 is
		\begin{align}
			 p_{X}(7)
			= \frac{1}{10}
		\end{align}
	\item Probability that card has value greater than 7 is
		\begin{align}
			1 - F_X(7)
			&= 1 - \frac{7}{10}
			\\
			&= \frac{3}{10}
		\end{align}
	\item Probability that card has value less than 7 is
		\begin{align}
			 F_{X}(6)
			=\frac{6}{10}
		\end{align}
\end{enumerate}

  \item A Lot consists of 48 mobile phones of which 42 are good, 3 have only minor defects and 3 have major defects.Varnika will buy a phone if it is good but the trader will only buy a mobile if it has no major defects. One phone is selected at random from the lot. What is the probability that it is
\begin{enumerate}
	\item acceptable to Varnika?
            \item acceptable to the trader?
\end{enumerate}
\solution
	%\begin{table}[H]
	\centering
\begin{tabular}{|c|c|c|}
\hline
Random variable &Value &Definition\\ \hline
\multirow{3}{*}{X} &0 &Slips of Rs 1\\
&1 &Slips of Rs 5\\
&2 &Slips of Rs 13\\ \hline
\multirow{2}{*}{Y} &0 &Box A\\
&1 &Box B\\\hline
\end{tabular}
\caption{}
\label{tab:Distribution}
\end{table}
See \tabref{tab:Distribution}.
\begin{align}
p_{Y}\brak{k}= \begin{cases} 
      \frac{1}{3} & {k=0} \\
      \frac{2}{3 }& {k=1} 
   \end{cases}
   \\
p_{Y|X}\brak{0|0} = \frac{19}{25}\, 
p_{Y|X}\brak{0|1} = \frac{6}{25}\,
p_{Y|X}\brak{1|0} = \frac{45}{50}\,
p_{Y|X}\brak{1|2} = \frac{5}{50}
\end{align}
The desired probability is the probability that a slip drawn at random is marked other than Rs 1,
\begin{align}
&=1-p_X\brak{0}\\
&= p_X(1) + p_X(2)
\end{align}
Using Bayes theorem,
\begin{align}
&= p_Y\brak{0} \times \pr{Y=0 | X=1} + p_Y\brak{1} \times \pr{Y=1|X=2}\\
&=\frac{1}{3} \times \frac{6}{25} + \frac{2}{3} \times \frac{5}{50}\\
&=\frac{11}{75}
\end{align}

\newpage

%\tableofcontents

\bigskip

\renewcommand{\thefigure}{\theenumi}
\renewcommand{\thetable}{\theenumi}
%\renewcommand{\theequation}{\theenumi}

%\begin{abstract}
%%\boldmath
%In this letter, an algorithm for evaluating the exact analytical bit error rate  (BER)  for the piecewise linear (PL) combiner for  multiple relays is presented. Previous results were available only for upto three relays. The algorithm is unique in the sense that  the actual mathematical expressions, that are prohibitively large, need not be explicitly obtained. The diversity gain due to multiple relays is shown through plots of the analytical BER, well supported by simulations. 
%
%\end{abstract}
% IEEEtran.cls defaults to using nonbold math in the Abstract.
% This preserves the distinction between vectors and scalars. However,
% if the journal you are submitting to favors bold math in the abstract,
% then you can use LaTeX's standard command \boldmath at the very start
% of the abstract to achieve this. Many IEEE journals frown on math
% in the abstract anyway.

% Note that keywords are not normally used for peerreview papers.
%\begin{IEEEkeywords}
%Cooperative diversity, decode and forward, piecewise linear
%\end{IEEEkeywords}



% For peer review papers, you can put extra information on the cover
% page as needed:
% \ifCLASSOPTIONpeerreview
% \begin{center} \bfseries EDICS Category: 3-BBND \end{center}
% \fi
%
% For peerreview papers, this IEEEtran command inserts a page break and
% creates the second title. It will be ignored for other modes.
%\IEEEpeerreviewmaketitle




 \item A student says that if you throw a die, it will show up 1 or not 1. Therefore, the probability of getting 1 and the probability of getting 'not 1' each is equal to $\frac{1}{2}$. Is this correct? Give reasons.\\
 \solution
        %\begin{table}[H]
	\centering
\begin{tabular}{|c|c|c|}
\hline
Random variable &Value &Definition\\ \hline
\multirow{3}{*}{X} &0 &Slips of Rs 1\\
&1 &Slips of Rs 5\\
&2 &Slips of Rs 13\\ \hline
\multirow{2}{*}{Y} &0 &Box A\\
&1 &Box B\\\hline
\end{tabular}
\caption{}
\label{tab:Distribution}
\end{table}
See \tabref{tab:Distribution}.
\begin{align}
p_{Y}\brak{k}= \begin{cases} 
      \frac{1}{3} & {k=0} \\
      \frac{2}{3 }& {k=1} 
   \end{cases}
   \\
p_{Y|X}\brak{0|0} = \frac{19}{25}\, 
p_{Y|X}\brak{0|1} = \frac{6}{25}\,
p_{Y|X}\brak{1|0} = \frac{45}{50}\,
p_{Y|X}\brak{1|2} = \frac{5}{50}
\end{align}
The desired probability is the probability that a slip drawn at random is marked other than Rs 1,
\begin{align}
&=1-p_X\brak{0}\\
&= p_X(1) + p_X(2)
\end{align}
Using Bayes theorem,
\begin{align}
&= p_Y\brak{0} \times \pr{Y=0 | X=1} + p_Y\brak{1} \times \pr{Y=1|X=2}\\
&=\frac{1}{3} \times \frac{6}{25} + \frac{2}{3} \times \frac{5}{50}\\
&=\frac{11}{75}
\end{align}

\newpage

%\tableofcontents

\bigskip

\renewcommand{\thefigure}{\theenumi}
\renewcommand{\thetable}{\theenumi}
%\renewcommand{\theequation}{\theenumi}

%\begin{abstract}
%%\boldmath
%In this letter, an algorithm for evaluating the exact analytical bit error rate  (BER)  for the piecewise linear (PL) combiner for  multiple relays is presented. Previous results were available only for upto three relays. The algorithm is unique in the sense that  the actual mathematical expressions, that are prohibitively large, need not be explicitly obtained. The diversity gain due to multiple relays is shown through plots of the analytical BER, well supported by simulations. 
%
%\end{abstract}
% IEEEtran.cls defaults to using nonbold math in the Abstract.
% This preserves the distinction between vectors and scalars. However,
% if the journal you are submitting to favors bold math in the abstract,
% then you can use LaTeX's standard command \boldmath at the very start
% of the abstract to achieve this. Many IEEE journals frown on math
% in the abstract anyway.

% Note that keywords are not normally used for peerreview papers.
%\begin{IEEEkeywords}
%Cooperative diversity, decode and forward, piecewise linear
%\end{IEEEkeywords}



% For peer review papers, you can put extra information on the cover
% page as needed:
% \ifCLASSOPTIONpeerreview
% \begin{center} \bfseries EDICS Category: 3-BBND \end{center}
% \fi
%
% For peerreview papers, this IEEEtran command inserts a page break and
% creates the second title. It will be ignored for other modes.
%\IEEEpeerreviewmaketitle




   \item Four candidates A, B, C, D have ap-
plied for the assignment to coach a school cricket
team. If A is twice as likely to be selected as B, and
B and C are given about the same chance of being
selected, while C is twice as likely to be selected
as D, what are the probabilities that
\begin{enumerate}
\item C will be selected?
\item A will not be selected?
\end{enumerate}
	%\begin{table}[H]
	\centering
\begin{tabular}{|c|c|c|}
\hline
Random variable &Value &Definition\\ \hline
\multirow{3}{*}{X} &0 &Slips of Rs 1\\
&1 &Slips of Rs 5\\
&2 &Slips of Rs 13\\ \hline
\multirow{2}{*}{Y} &0 &Box A\\
&1 &Box B\\\hline
\end{tabular}
\caption{}
\label{tab:Distribution}
\end{table}
See \tabref{tab:Distribution}.
\begin{align}
p_{Y}\brak{k}= \begin{cases} 
      \frac{1}{3} & {k=0} \\
      \frac{2}{3 }& {k=1} 
   \end{cases}
   \\
p_{Y|X}\brak{0|0} = \frac{19}{25}\, 
p_{Y|X}\brak{0|1} = \frac{6}{25}\,
p_{Y|X}\brak{1|0} = \frac{45}{50}\,
p_{Y|X}\brak{1|2} = \frac{5}{50}
\end{align}
The desired probability is the probability that a slip drawn at random is marked other than Rs 1,
\begin{align}
&=1-p_X\brak{0}\\
&= p_X(1) + p_X(2)
\end{align}
Using Bayes theorem,
\begin{align}
&= p_Y\brak{0} \times \pr{Y=0 | X=1} + p_Y\brak{1} \times \pr{Y=1|X=2}\\
&=\frac{1}{3} \times \frac{6}{25} + \frac{2}{3} \times \frac{5}{50}\\
&=\frac{11}{75}
\end{align}

\newpage

%\tableofcontents

\bigskip

\renewcommand{\thefigure}{\theenumi}
\renewcommand{\thetable}{\theenumi}
%\renewcommand{\theequation}{\theenumi}

%\begin{abstract}
%%\boldmath
%In this letter, an algorithm for evaluating the exact analytical bit error rate  (BER)  for the piecewise linear (PL) combiner for  multiple relays is presented. Previous results were available only for upto three relays. The algorithm is unique in the sense that  the actual mathematical expressions, that are prohibitively large, need not be explicitly obtained. The diversity gain due to multiple relays is shown through plots of the analytical BER, well supported by simulations. 
%
%\end{abstract}
% IEEEtran.cls defaults to using nonbold math in the Abstract.
% This preserves the distinction between vectors and scalars. However,
% if the journal you are submitting to favors bold math in the abstract,
% then you can use LaTeX's standard command \boldmath at the very start
% of the abstract to achieve this. Many IEEE journals frown on math
% in the abstract anyway.

% Note that keywords are not normally used for peerreview papers.
%\begin{IEEEkeywords}
%Cooperative diversity, decode and forward, piecewise linear
%\end{IEEEkeywords}



% For peer review papers, you can put extra information on the cover
% page as needed:
% \ifCLASSOPTIONpeerreview
% \begin{center} \bfseries EDICS Category: 3-BBND \end{center}
% \fi
%
% For peerreview papers, this IEEEtran command inserts a page break and
% creates the second title. It will be ignored for other modes.
%\IEEEpeerreviewmaketitle




 \item A bag contain 24 balls of which $x$ balls are red, $2x$ are white and $3x$ are blue. A ball is selected at random, What is the probability that it is
\begin{enumerate}[label=\alph*)]
\item not red ?
\item white ?
\end{enumerate}
%\begin{table}[H]
	\centering
\begin{tabular}{|c|c|c|}
\hline
Random variable &Value &Definition\\ \hline
\multirow{3}{*}{X} &0 &Slips of Rs 1\\
&1 &Slips of Rs 5\\
&2 &Slips of Rs 13\\ \hline
\multirow{2}{*}{Y} &0 &Box A\\
&1 &Box B\\\hline
\end{tabular}
\caption{}
\label{tab:Distribution}
\end{table}
See \tabref{tab:Distribution}.
\begin{align}
p_{Y}\brak{k}= \begin{cases} 
      \frac{1}{3} & {k=0} \\
      \frac{2}{3 }& {k=1} 
   \end{cases}
   \\
p_{Y|X}\brak{0|0} = \frac{19}{25}\, 
p_{Y|X}\brak{0|1} = \frac{6}{25}\,
p_{Y|X}\brak{1|0} = \frac{45}{50}\,
p_{Y|X}\brak{1|2} = \frac{5}{50}
\end{align}
The desired probability is the probability that a slip drawn at random is marked other than Rs 1,
\begin{align}
&=1-p_X\brak{0}\\
&= p_X(1) + p_X(2)
\end{align}
Using Bayes theorem,
\begin{align}
&= p_Y\brak{0} \times \pr{Y=0 | X=1} + p_Y\brak{1} \times \pr{Y=1|X=2}\\
&=\frac{1}{3} \times \frac{6}{25} + \frac{2}{3} \times \frac{5}{50}\\
&=\frac{11}{75}
\end{align}

\newpage

%\tableofcontents

\bigskip

\renewcommand{\thefigure}{\theenumi}
\renewcommand{\thetable}{\theenumi}
%\renewcommand{\theequation}{\theenumi}

%\begin{abstract}
%%\boldmath
%In this letter, an algorithm for evaluating the exact analytical bit error rate  (BER)  for the piecewise linear (PL) combiner for  multiple relays is presented. Previous results were available only for upto three relays. The algorithm is unique in the sense that  the actual mathematical expressions, that are prohibitively large, need not be explicitly obtained. The diversity gain due to multiple relays is shown through plots of the analytical BER, well supported by simulations. 
%
%\end{abstract}
% IEEEtran.cls defaults to using nonbold math in the Abstract.
% This preserves the distinction between vectors and scalars. However,
% if the journal you are submitting to favors bold math in the abstract,
% then you can use LaTeX's standard command \boldmath at the very start
% of the abstract to achieve this. Many IEEE journals frown on math
% in the abstract anyway.

% Note that keywords are not normally used for peerreview papers.
%\begin{IEEEkeywords}
%Cooperative diversity, decode and forward, piecewise linear
%\end{IEEEkeywords}



% For peer review papers, you can put extra information on the cover
% page as needed:
% \ifCLASSOPTIONpeerreview
% \begin{center} \bfseries EDICS Category: 3-BBND \end{center}
% \fi
%
% For peerreview papers, this IEEEtran command inserts a page break and
% creates the second title. It will be ignored for other modes.
%\IEEEpeerreviewmaketitle




If the letters of the word ASSASSINATION are arranged at random. Find the Probability that
\begin{enumerate}[label=(\alph*)]
\item Four $S's$ come consecutively in the word
\item Two  $I's$ and two $N's$ come together
\item All $A's$ are not coming together
\item No two $A's$ are coming together
\end{enumerate}
%\begin{table}[H]
	\centering
\begin{tabular}{|c|c|c|}
\hline
Random variable &Value &Definition\\ \hline
\multirow{3}{*}{X} &0 &Slips of Rs 1\\
&1 &Slips of Rs 5\\
&2 &Slips of Rs 13\\ \hline
\multirow{2}{*}{Y} &0 &Box A\\
&1 &Box B\\\hline
\end{tabular}
\caption{}
\label{tab:Distribution}
\end{table}
See \tabref{tab:Distribution}.
\begin{align}
p_{Y}\brak{k}= \begin{cases} 
      \frac{1}{3} & {k=0} \\
      \frac{2}{3 }& {k=1} 
   \end{cases}
   \\
p_{Y|X}\brak{0|0} = \frac{19}{25}\, 
p_{Y|X}\brak{0|1} = \frac{6}{25}\,
p_{Y|X}\brak{1|0} = \frac{45}{50}\,
p_{Y|X}\brak{1|2} = \frac{5}{50}
\end{align}
The desired probability is the probability that a slip drawn at random is marked other than Rs 1,
\begin{align}
&=1-p_X\brak{0}\\
&= p_X(1) + p_X(2)
\end{align}
Using Bayes theorem,
\begin{align}
&= p_Y\brak{0} \times \pr{Y=0 | X=1} + p_Y\brak{1} \times \pr{Y=1|X=2}\\
&=\frac{1}{3} \times \frac{6}{25} + \frac{2}{3} \times \frac{5}{50}\\
&=\frac{11}{75}
\end{align}

\newpage

%\tableofcontents

\bigskip

\renewcommand{\thefigure}{\theenumi}
\renewcommand{\thetable}{\theenumi}
%\renewcommand{\theequation}{\theenumi}

%\begin{abstract}
%%\boldmath
%In this letter, an algorithm for evaluating the exact analytical bit error rate  (BER)  for the piecewise linear (PL) combiner for  multiple relays is presented. Previous results were available only for upto three relays. The algorithm is unique in the sense that  the actual mathematical expressions, that are prohibitively large, need not be explicitly obtained. The diversity gain due to multiple relays is shown through plots of the analytical BER, well supported by simulations. 
%
%\end{abstract}
% IEEEtran.cls defaults to using nonbold math in the Abstract.
% This preserves the distinction between vectors and scalars. However,
% if the journal you are submitting to favors bold math in the abstract,
% then you can use LaTeX's standard command \boldmath at the very start
% of the abstract to achieve this. Many IEEE journals frown on math
% in the abstract anyway.

% Note that keywords are not normally used for peerreview papers.
%\begin{IEEEkeywords}
%Cooperative diversity, decode and forward, piecewise linear
%\end{IEEEkeywords}



% For peer review papers, you can put extra information on the cover
% page as needed:
% \ifCLASSOPTIONpeerreview
% \begin{center} \bfseries EDICS Category: 3-BBND \end{center}
% \fi
%
% For peerreview papers, this IEEEtran command inserts a page break and
% creates the second title. It will be ignored for other modes.
%\IEEEpeerreviewmaketitle




	\item One urn contains two black balls (labelled B1 and B2) and one white ball. A
	second urn contains one black ball and two white balls (labelled W1 and W2).
	Suppose the following experiment is performed. One of the two urns is chosen
	at random. Next a ball is randomly chosen from the urn. Then a second ball is
	chosen at random from the same urn without replacing the first ball.
	
	\begin{enumerate}
	\item What is the probability that two black balls are chosen?
	
	\item What is the probability that two balls of opposite colour are chosen?
	\end{enumerate}
	\solution
	%\begin{align}
    \label{eq:12.13.6.18.1}
	\because	\pr{A|B} &> \pr{A},\
\frac{\pr{AB}}{\pr{B}} > \pr{A}
\\
    \label{eq:12.13.6.18.2}
	\implies \pr{AB} &> \pr{A}\pr{B}
	\\
	\text{or, } \frac{\pr{AB}}{\pr{A}} &=\pr{B|A} > \pr{A}
\end{align}

\end{enumerate}

\item In a certain lottery 10,000 tickets are sold and ten equal prizes are awarded. What is the probability of not getting a prize if you buy (a) one ticket (b) two tickets (c) 10 tickets ?	
\\
\solution
		%\begin{enumerate}[label=\thesection.\arabic*,ref=\thesection.\theenumi]
	\item One card is drawn from a well-shuffled deck of 52 cards. Find the probability of getting
\begin{enumerate}
\item A king of red colour 
\item A face card 
\item A red face card
\item The jack of hearts
\item A spade
\item The queen of diamonds

\end{enumerate}
\solution
		%\begin{table}[H]
	\centering
\begin{tabular}{|c|c|c|}
\hline
Random variable &Value &Definition\\ \hline
\multirow{3}{*}{X} &0 &Slips of Rs 1\\
&1 &Slips of Rs 5\\
&2 &Slips of Rs 13\\ \hline
\multirow{2}{*}{Y} &0 &Box A\\
&1 &Box B\\\hline
\end{tabular}
\caption{}
\label{tab:Distribution}
\end{table}
See \tabref{tab:Distribution}.
\begin{align}
p_{Y}\brak{k}= \begin{cases} 
      \frac{1}{3} & {k=0} \\
      \frac{2}{3 }& {k=1} 
   \end{cases}
   \\
p_{Y|X}\brak{0|0} = \frac{19}{25}\, 
p_{Y|X}\brak{0|1} = \frac{6}{25}\,
p_{Y|X}\brak{1|0} = \frac{45}{50}\,
p_{Y|X}\brak{1|2} = \frac{5}{50}
\end{align}
The desired probability is the probability that a slip drawn at random is marked other than Rs 1,
\begin{align}
&=1-p_X\brak{0}\\
&= p_X(1) + p_X(2)
\end{align}
Using Bayes theorem,
\begin{align}
&= p_Y\brak{0} \times \pr{Y=0 | X=1} + p_Y\brak{1} \times \pr{Y=1|X=2}\\
&=\frac{1}{3} \times \frac{6}{25} + \frac{2}{3} \times \frac{5}{50}\\
&=\frac{11}{75}
\end{align}

\newpage

%\tableofcontents

\bigskip

\renewcommand{\thefigure}{\theenumi}
\renewcommand{\thetable}{\theenumi}
%\renewcommand{\theequation}{\theenumi}

%\begin{abstract}
%%\boldmath
%In this letter, an algorithm for evaluating the exact analytical bit error rate  (BER)  for the piecewise linear (PL) combiner for  multiple relays is presented. Previous results were available only for upto three relays. The algorithm is unique in the sense that  the actual mathematical expressions, that are prohibitively large, need not be explicitly obtained. The diversity gain due to multiple relays is shown through plots of the analytical BER, well supported by simulations. 
%
%\end{abstract}
% IEEEtran.cls defaults to using nonbold math in the Abstract.
% This preserves the distinction between vectors and scalars. However,
% if the journal you are submitting to favors bold math in the abstract,
% then you can use LaTeX's standard command \boldmath at the very start
% of the abstract to achieve this. Many IEEE journals frown on math
% in the abstract anyway.

% Note that keywords are not normally used for peerreview papers.
%\begin{IEEEkeywords}
%Cooperative diversity, decode and forward, piecewise linear
%\end{IEEEkeywords}



% For peer review papers, you can put extra information on the cover
% page as needed:
% \ifCLASSOPTIONpeerreview
% \begin{center} \bfseries EDICS Category: 3-BBND \end{center}
% \fi
%
% For peerreview papers, this IEEEtran command inserts a page break and
% creates the second title. It will be ignored for other modes.
%\IEEEpeerreviewmaketitle




	\item Five cards—the ten, jack, queen, king and ace of diamonds, are well-shuffled with their face downwards. One card is then picked up at random.
\begin{enumerate}
\item
What is the probability that the card is the queen? 
\item
If the queen is drawn and put aside, what is the probability that the second card picked up is (a) an ace? (b) a queen?\\
\end{enumerate}
\solution
		%\begin{enumerate}[label=\thesection.\arabic*,ref=\thesection.\theenumi]
	\item One card is drawn from a well-shuffled deck of 52 cards. Find the probability of getting
\begin{enumerate}
\item A king of red colour 
\item A face card 
\item A red face card
\item The jack of hearts
\item A spade
\item The queen of diamonds

\end{enumerate}
\solution
		%\input{ncert/10/15/1/14/main.tex}
	\item Five cards—the ten, jack, queen, king and ace of diamonds, are well-shuffled with their face downwards. One card is then picked up at random.
\begin{enumerate}
\item
What is the probability that the card is the queen? 
\item
If the queen is drawn and put aside, what is the probability that the second card picked up is (a) an ace? (b) a queen?\\
\end{enumerate}
\solution
		%\input{ncert/10/15/1/15/defs.tex}
	\item A bag contains $5$ red balls and some blue balls. If the probability of drawing a blue ball is double that if a red ball, determine the number of blue balls in the bag. 
		\\
\solution
		%\input{ncert/10/15/2/3/defs.tex}
	\item A card is selected from a pack of 52 cards.
 \begin{enumerate}[label=(\alph*)] 
                 \item How many points are there in the sample space?
                 \item Calculate the probability that the card is an ace of spades.
                 \item Calculate the probability that the card is (i) an ace and (ii) black card.
 \end{enumerate}
\solution
		%\input{ncert/11/16/3/4/main.tex}
\item Four cards are drawn from a well-shuffled deck of 52 cards. What is the probability of obtaining 3 diamonds and one spade.
\\
\solution
		%\input{ncert/11/16/4/2/defs.tex}
\item In a certain lottery 10,000 tickets are sold and ten equal prizes are awarded. What is the probability of not getting a prize if you buy (a) one ticket (b) two tickets (c) 10 tickets ?	
\\
\solution
		%\input{ncert/11/16/4/4/defs.tex}
		%
\item 
Out of 100 students, two sections of 40 and 60 are formed. If you and your friend are among the 100 students, what is the probability that
\begin{enumerate}
\item you both enter the same section?
\item you both enter the different sections?
\end{enumerate}
\solution
		%\input{ncert/11/16/4/5/defs.tex}
	\item 
The number lock of a suitcase has 4 wheels each labelled with ten digits i.e. from 0 to 9.The lock opens with a sequence of four digits with no repeats.What is the probability of a person getting the right sequence to open the suitcase.
\\
\solution
		%\input{ncert/11/16/4/10/defs.tex}
		%
\item 
Two cards are drawn at random and without replacement from a pack of 52 playing cards. Find the probability that both the cards are black.
\\
\solution
		%\input{ncert/12/13/2/2/defs.tex}
		\item A box of oranges is inspected by examining three randomly selected oranges drawn without replacement. If all the three oranges are good, the box is approved for sale, otherwise, it is rejected. Find the probability that a box containing 15 oranges out of which 12 are good and 3 are bad ones will be approved for sale.
		\label{ncert/12/13/2/3/defs.tex}
		\item Two balls are drawn at random with replacement from a box containing 10 black and 8 red balls. Find the probability that
		\label{ncert/12/13/2/12}
\begin{enumerate}
\item both balls are red.
\item first ball is black and second is red.
\item one of them is black and other is red.
\end{enumerate}

\item In a hostel, 60\% of the students read Hindi newspaper, 40\% read English newspaper and 20\% read both Hindi and English newspapers. A student is selected at random.
		\label{ncert/12/13/2/15}
\begin{enumerate}
\item Find the probability that she reads neither Hindi nor English newspapers.
\item If she reads Hindi newspaper, find the probability that she reads English newspaper.
\item If she reads English newspaper, find the probability that she reads Hindi newspaper.\\
\end{enumerate}
\item The probability of obtaining an even prime number on each die, when a pair of dice is rolled is 
\begin{enumerate}
    \item $0$ 
    
    \item $\frac{1}{3}$ 
    
    \item $\frac{1}{12}$ 
    
    \item $\frac{1}{36}$ 
\end{enumerate}
\solution
		%\input{ncert/12/13/2/17/defs.tex}
	\item A bag contains 4 red and 4 black balls, another bag contains 2 red and 6 black balls. One of the two bags is selected at random and a ball is drawn from the bag which is found to be red. Find the probability that the ball is drawn from the first bag.
\\
\solution
		%\input{ncert/12/13/3/2/main.tex}
  \item
  Cards with numbers 2 to 101 are placed in a box. A card is selected at random.Find the probability that the card has
\begin{enumerate}[label=(\roman*)]
	\item an even number 
	\item a square number
\end{enumerate}
\solution
%\input{exemplar/10/13/3/32/main.tex}
\item
The king, queen and jack of clubs are removed from a deck of 52 playing cards and then well shuffled. Now one card is drawn at random from the remaining cards.  Determine the probability that the card is
\begin{enumerate}[label=(\roman*)]
\item a club
\item 10 of hearts
\end{enumerate}
\solution
%\input{exemplar/10/13/3/29/main.tex}
\item A team of medical students doing their internship have to assist during surgeries
at a city hospital. The probabilities of surgeries rated as very complex, complex,
routine, simple or very simple are respectively, 0.15, 0.20, 0.31, 0.26, .08. Find
the probabilities that a particular surgery will be rated
\begin{enumerate}
	\item complex or very complex;
	\item neither very complex nor very simple;
	\item routine or complex
	\item routine or simple
\end{enumerate}
\solution
%\input{exemplar/11/16/3/8(1)/main.tex}
\item A card is selected from a pack of 52 cards.
\begin{enumerate}[label=(\alph*)]
    \item How many points are there in the sample space?
    \item Calculate the probability that the card is an ace of spades.
    \item Calculate the probability that the card is (i) an ace and (ii) black card.
\end{enumerate}
\solution
%\input{exemplar/11/16/3/4/main2.tex}
\item The probability that a non leap year selected at random will contain 53 sundays.
\\
\solution
%\input{exemplar/10/13/1/19/main.tex}
\item One of the four persons John, Rita, Aslam or Gurpreet will be promoted next
month. Consequently the sample space consists of four elementary outcomes
S = {John promoted, Rita promoted, Aslam promoted, Gurpreet promoted}
You are told that the chances of John’s promotion is same as that of Gurpreet,
Rita’s chances of promotion are twice as likely as Johns. Aslam’s chances are
four times that of John.
\begin{enumerate}
	\item Determine
	\begin{enumerate}
		\item P (John promoted)
		\item P (Rita promoted)
		\item P (Aslam promoted)
		\item P (Gurpreet promoted)
	\end{enumerate}
	\item If A = {John promoted or Gurpreet promoted}, find P (A).
\end{enumerate}
\solution
%\input{exemplar/11/16/3/10/main.tex}
\item A card is drawn from a deck of 52 cards. Find the probability of getting a king or a heart or a red card.\\
\solution
%\input{exemplar/11/16/3/15/main.tex}
\item The probability that a student will pass his examination is 0.73, the probability of
the student getting a compartment is 0.13, and the probability that the student will
either pass or get compartment is 0.96. State True or False.\\
\solution
%\input{exemplar/11/16/3/31/main.tex}
\item A card is selected from a pack of 52 cards\\
\begin{enumerate}[label=(\alph*)]
\item How many points are there in the sample space?
\item Calculate the probability that the cards is an ace of spades.
\item Calculate the probability that the card is (i) an ace (ii)black card.\\
\end{enumerate}
%\input{ncert/11/16/3/4_1/Prob_4.tex}
\item In a non-leap year, the probability of having 53 tuesdays or 53 wednesdays is\\
\solution
%\input{exemplar/11/16/3/18/main.tex}
\item There are 1000 sealed envelopes in a box, 10 of them contain a cash prize of
Rs 100 each, 100 of them contain a cash prize of Rs 50 each and 200 of them
contain a cash prize of Rs 10 each and rest do not contain any cash prize. If they
are well shuffled and an envelope is picked up out, what is the probability that it
contains no cash prize?\\
\solution
%\input{exemplar/10/13/3/34/main.tex}
\item 
A die is thrown and a card is selected at random from a deck of 52 playing cards. The probability of getting an even number on the die and a spade card.\\
\solution
%\input{exemplar/12/13/3/78/main.tex}
\item
If 4-digit numbers greater than 5,000 are randomly formed from the digits 0, 1, 3, 5, and 7, what is the probability of forming a number divisible by 5 when:
\begin{enumerate}
    \item The digits are repeated?
    \item The repetition of digits is not allowed?
\end{enumerate}
\solution
%\input{ncert/11/16/4/9/main.tex}
\item Consider the probability space $\brak{\Omega, \mathcal{G}, P}$ where $\Omega = [0,2]$ and $\mathcal{G} = \cbrak{\phi, \Omega, [0,1], (1,2]}$. Let $X$ and $Y$ be two functions on $\Omega$ defined as
\begin{align*}
    X(\omega) = 
    \begin{cases}
        1 & \text{if }\omega \in [0, 1]\\
        2 & \text{if }\omega \in (1, 2]
    \end{cases}
\end{align*}
and
\begin{align*}
    Y(\omega) = 
    \begin{cases}
        2 & \text{if }\omega \in [0, 1.5]\\
        3 & \text{if }\omega \in (1.5, 2].
    \end{cases}
\end{align*}
Then which one of the following statements is true?
\begin{enumerate}
    \item [(A)] $X$ is a random variable with respect to $\mathcal{G}$, but $Y$ is not a random variable with respect to $\mathcal{G}$.
    \item [(B)] $Y$ is a random variable with respect to $\mathcal{G}$, but $X$ is not a random variable with respect to $\mathcal{G}$.
    \item [(C)] Neither $X$ nor $Y$ is a random variable with respect to $\mathcal{G}$.
    \item [(D)] Both $X$ and $Y$ are random variables with respect to $\mathcal{G}$.
\end{enumerate} \hfill (GATE ST 2023)\\
\solution
%\input{gate/ST/2023/14/main.tex}
	\item  A die is loaded in such a way that each odd number is twice as likely to occur as
each even number. Find $P(G)$, where $G$ is the event that a number greater than
3 occurs on a single roll of the die.
\\
\solution
		%\input{exemplar/11/16/3/5/main.tex}
	\item All the jacks, queens and kings are removed from a deck of 52 playing cards. The remaining cards are well shuffled and then one card is drawn at random. Giving ace a value 1 similar value for other cards, find the probability that the card has a value 
		\begin{enumerate}
			\item 7
			\item greater than 7
			\item less than 7
		\end{enumerate}
		%\input{exemplar/10/13/3/30/main.tex}
  \item A Lot consists of 48 mobile phones of which 42 are good, 3 have only minor defects and 3 have major defects.Varnika will buy a phone if it is good but the trader will only buy a mobile if it has no major defects. One phone is selected at random from the lot. What is the probability that it is
\begin{enumerate}
	\item acceptable to Varnika?
            \item acceptable to the trader?
\end{enumerate}
\solution
	%\input{exemplar/10/13/3/40/main.tex}
 \item A student says that if you throw a die, it will show up 1 or not 1. Therefore, the probability of getting 1 and the probability of getting 'not 1' each is equal to $\frac{1}{2}$. Is this correct? Give reasons.\\
 \solution
        %\input{exemplar/10/13/2/9/main.tex}
   \item Four candidates A, B, C, D have ap-
plied for the assignment to coach a school cricket
team. If A is twice as likely to be selected as B, and
B and C are given about the same chance of being
selected, while C is twice as likely to be selected
as D, what are the probabilities that
\begin{enumerate}
\item C will be selected?
\item A will not be selected?
\end{enumerate}
	%\input{exemplar/11/16/3/9/main.tex}
 \item A bag contain 24 balls of which $x$ balls are red, $2x$ are white and $3x$ are blue. A ball is selected at random, What is the probability that it is
\begin{enumerate}[label=\alph*)]
\item not red ?
\item white ?
\end{enumerate}
%\input{exemplar/10/13/3/41/main.tex}
If the letters of the word ASSASSINATION are arranged at random. Find the Probability that
\begin{enumerate}[label=(\alph*)]
\item Four $S's$ come consecutively in the word
\item Two  $I's$ and two $N's$ come together
\item All $A's$ are not coming together
\item No two $A's$ are coming together
\end{enumerate}
%\input{exemplar/11/16/3/14/main.tex}
	\item One urn contains two black balls (labelled B1 and B2) and one white ball. A
	second urn contains one black ball and two white balls (labelled W1 and W2).
	Suppose the following experiment is performed. One of the two urns is chosen
	at random. Next a ball is randomly chosen from the urn. Then a second ball is
	chosen at random from the same urn without replacing the first ball.
	
	\begin{enumerate}
	\item What is the probability that two black balls are chosen?
	
	\item What is the probability that two balls of opposite colour are chosen?
	\end{enumerate}
	\solution
	%\input{exemplar/11/16/3/12/main1.tex}
\end{enumerate}

	\item A bag contains $5$ red balls and some blue balls. If the probability of drawing a blue ball is double that if a red ball, determine the number of blue balls in the bag. 
		\\
\solution
		%\begin{enumerate}[label=\thesection.\arabic*,ref=\thesection.\theenumi]
	\item One card is drawn from a well-shuffled deck of 52 cards. Find the probability of getting
\begin{enumerate}
\item A king of red colour 
\item A face card 
\item A red face card
\item The jack of hearts
\item A spade
\item The queen of diamonds

\end{enumerate}
\solution
		%\input{ncert/10/15/1/14/main.tex}
	\item Five cards—the ten, jack, queen, king and ace of diamonds, are well-shuffled with their face downwards. One card is then picked up at random.
\begin{enumerate}
\item
What is the probability that the card is the queen? 
\item
If the queen is drawn and put aside, what is the probability that the second card picked up is (a) an ace? (b) a queen?\\
\end{enumerate}
\solution
		%\input{ncert/10/15/1/15/defs.tex}
	\item A bag contains $5$ red balls and some blue balls. If the probability of drawing a blue ball is double that if a red ball, determine the number of blue balls in the bag. 
		\\
\solution
		%\input{ncert/10/15/2/3/defs.tex}
	\item A card is selected from a pack of 52 cards.
 \begin{enumerate}[label=(\alph*)] 
                 \item How many points are there in the sample space?
                 \item Calculate the probability that the card is an ace of spades.
                 \item Calculate the probability that the card is (i) an ace and (ii) black card.
 \end{enumerate}
\solution
		%\input{ncert/11/16/3/4/main.tex}
\item Four cards are drawn from a well-shuffled deck of 52 cards. What is the probability of obtaining 3 diamonds and one spade.
\\
\solution
		%\input{ncert/11/16/4/2/defs.tex}
\item In a certain lottery 10,000 tickets are sold and ten equal prizes are awarded. What is the probability of not getting a prize if you buy (a) one ticket (b) two tickets (c) 10 tickets ?	
\\
\solution
		%\input{ncert/11/16/4/4/defs.tex}
		%
\item 
Out of 100 students, two sections of 40 and 60 are formed. If you and your friend are among the 100 students, what is the probability that
\begin{enumerate}
\item you both enter the same section?
\item you both enter the different sections?
\end{enumerate}
\solution
		%\input{ncert/11/16/4/5/defs.tex}
	\item 
The number lock of a suitcase has 4 wheels each labelled with ten digits i.e. from 0 to 9.The lock opens with a sequence of four digits with no repeats.What is the probability of a person getting the right sequence to open the suitcase.
\\
\solution
		%\input{ncert/11/16/4/10/defs.tex}
		%
\item 
Two cards are drawn at random and without replacement from a pack of 52 playing cards. Find the probability that both the cards are black.
\\
\solution
		%\input{ncert/12/13/2/2/defs.tex}
		\item A box of oranges is inspected by examining three randomly selected oranges drawn without replacement. If all the three oranges are good, the box is approved for sale, otherwise, it is rejected. Find the probability that a box containing 15 oranges out of which 12 are good and 3 are bad ones will be approved for sale.
		\label{ncert/12/13/2/3/defs.tex}
		\item Two balls are drawn at random with replacement from a box containing 10 black and 8 red balls. Find the probability that
		\label{ncert/12/13/2/12}
\begin{enumerate}
\item both balls are red.
\item first ball is black and second is red.
\item one of them is black and other is red.
\end{enumerate}

\item In a hostel, 60\% of the students read Hindi newspaper, 40\% read English newspaper and 20\% read both Hindi and English newspapers. A student is selected at random.
		\label{ncert/12/13/2/15}
\begin{enumerate}
\item Find the probability that she reads neither Hindi nor English newspapers.
\item If she reads Hindi newspaper, find the probability that she reads English newspaper.
\item If she reads English newspaper, find the probability that she reads Hindi newspaper.\\
\end{enumerate}
\item The probability of obtaining an even prime number on each die, when a pair of dice is rolled is 
\begin{enumerate}
    \item $0$ 
    
    \item $\frac{1}{3}$ 
    
    \item $\frac{1}{12}$ 
    
    \item $\frac{1}{36}$ 
\end{enumerate}
\solution
		%\input{ncert/12/13/2/17/defs.tex}
	\item A bag contains 4 red and 4 black balls, another bag contains 2 red and 6 black balls. One of the two bags is selected at random and a ball is drawn from the bag which is found to be red. Find the probability that the ball is drawn from the first bag.
\\
\solution
		%\input{ncert/12/13/3/2/main.tex}
  \item
  Cards with numbers 2 to 101 are placed in a box. A card is selected at random.Find the probability that the card has
\begin{enumerate}[label=(\roman*)]
	\item an even number 
	\item a square number
\end{enumerate}
\solution
%\input{exemplar/10/13/3/32/main.tex}
\item
The king, queen and jack of clubs are removed from a deck of 52 playing cards and then well shuffled. Now one card is drawn at random from the remaining cards.  Determine the probability that the card is
\begin{enumerate}[label=(\roman*)]
\item a club
\item 10 of hearts
\end{enumerate}
\solution
%\input{exemplar/10/13/3/29/main.tex}
\item A team of medical students doing their internship have to assist during surgeries
at a city hospital. The probabilities of surgeries rated as very complex, complex,
routine, simple or very simple are respectively, 0.15, 0.20, 0.31, 0.26, .08. Find
the probabilities that a particular surgery will be rated
\begin{enumerate}
	\item complex or very complex;
	\item neither very complex nor very simple;
	\item routine or complex
	\item routine or simple
\end{enumerate}
\solution
%\input{exemplar/11/16/3/8(1)/main.tex}
\item A card is selected from a pack of 52 cards.
\begin{enumerate}[label=(\alph*)]
    \item How many points are there in the sample space?
    \item Calculate the probability that the card is an ace of spades.
    \item Calculate the probability that the card is (i) an ace and (ii) black card.
\end{enumerate}
\solution
%\input{exemplar/11/16/3/4/main2.tex}
\item The probability that a non leap year selected at random will contain 53 sundays.
\\
\solution
%\input{exemplar/10/13/1/19/main.tex}
\item One of the four persons John, Rita, Aslam or Gurpreet will be promoted next
month. Consequently the sample space consists of four elementary outcomes
S = {John promoted, Rita promoted, Aslam promoted, Gurpreet promoted}
You are told that the chances of John’s promotion is same as that of Gurpreet,
Rita’s chances of promotion are twice as likely as Johns. Aslam’s chances are
four times that of John.
\begin{enumerate}
	\item Determine
	\begin{enumerate}
		\item P (John promoted)
		\item P (Rita promoted)
		\item P (Aslam promoted)
		\item P (Gurpreet promoted)
	\end{enumerate}
	\item If A = {John promoted or Gurpreet promoted}, find P (A).
\end{enumerate}
\solution
%\input{exemplar/11/16/3/10/main.tex}
\item A card is drawn from a deck of 52 cards. Find the probability of getting a king or a heart or a red card.\\
\solution
%\input{exemplar/11/16/3/15/main.tex}
\item The probability that a student will pass his examination is 0.73, the probability of
the student getting a compartment is 0.13, and the probability that the student will
either pass or get compartment is 0.96. State True or False.\\
\solution
%\input{exemplar/11/16/3/31/main.tex}
\item A card is selected from a pack of 52 cards\\
\begin{enumerate}[label=(\alph*)]
\item How many points are there in the sample space?
\item Calculate the probability that the cards is an ace of spades.
\item Calculate the probability that the card is (i) an ace (ii)black card.\\
\end{enumerate}
%\input{ncert/11/16/3/4_1/Prob_4.tex}
\item In a non-leap year, the probability of having 53 tuesdays or 53 wednesdays is\\
\solution
%\input{exemplar/11/16/3/18/main.tex}
\item There are 1000 sealed envelopes in a box, 10 of them contain a cash prize of
Rs 100 each, 100 of them contain a cash prize of Rs 50 each and 200 of them
contain a cash prize of Rs 10 each and rest do not contain any cash prize. If they
are well shuffled and an envelope is picked up out, what is the probability that it
contains no cash prize?\\
\solution
%\input{exemplar/10/13/3/34/main.tex}
\item 
A die is thrown and a card is selected at random from a deck of 52 playing cards. The probability of getting an even number on the die and a spade card.\\
\solution
%\input{exemplar/12/13/3/78/main.tex}
\item
If 4-digit numbers greater than 5,000 are randomly formed from the digits 0, 1, 3, 5, and 7, what is the probability of forming a number divisible by 5 when:
\begin{enumerate}
    \item The digits are repeated?
    \item The repetition of digits is not allowed?
\end{enumerate}
\solution
%\input{ncert/11/16/4/9/main.tex}
\item Consider the probability space $\brak{\Omega, \mathcal{G}, P}$ where $\Omega = [0,2]$ and $\mathcal{G} = \cbrak{\phi, \Omega, [0,1], (1,2]}$. Let $X$ and $Y$ be two functions on $\Omega$ defined as
\begin{align*}
    X(\omega) = 
    \begin{cases}
        1 & \text{if }\omega \in [0, 1]\\
        2 & \text{if }\omega \in (1, 2]
    \end{cases}
\end{align*}
and
\begin{align*}
    Y(\omega) = 
    \begin{cases}
        2 & \text{if }\omega \in [0, 1.5]\\
        3 & \text{if }\omega \in (1.5, 2].
    \end{cases}
\end{align*}
Then which one of the following statements is true?
\begin{enumerate}
    \item [(A)] $X$ is a random variable with respect to $\mathcal{G}$, but $Y$ is not a random variable with respect to $\mathcal{G}$.
    \item [(B)] $Y$ is a random variable with respect to $\mathcal{G}$, but $X$ is not a random variable with respect to $\mathcal{G}$.
    \item [(C)] Neither $X$ nor $Y$ is a random variable with respect to $\mathcal{G}$.
    \item [(D)] Both $X$ and $Y$ are random variables with respect to $\mathcal{G}$.
\end{enumerate} \hfill (GATE ST 2023)\\
\solution
%\input{gate/ST/2023/14/main.tex}
	\item  A die is loaded in such a way that each odd number is twice as likely to occur as
each even number. Find $P(G)$, where $G$ is the event that a number greater than
3 occurs on a single roll of the die.
\\
\solution
		%\input{exemplar/11/16/3/5/main.tex}
	\item All the jacks, queens and kings are removed from a deck of 52 playing cards. The remaining cards are well shuffled and then one card is drawn at random. Giving ace a value 1 similar value for other cards, find the probability that the card has a value 
		\begin{enumerate}
			\item 7
			\item greater than 7
			\item less than 7
		\end{enumerate}
		%\input{exemplar/10/13/3/30/main.tex}
  \item A Lot consists of 48 mobile phones of which 42 are good, 3 have only minor defects and 3 have major defects.Varnika will buy a phone if it is good but the trader will only buy a mobile if it has no major defects. One phone is selected at random from the lot. What is the probability that it is
\begin{enumerate}
	\item acceptable to Varnika?
            \item acceptable to the trader?
\end{enumerate}
\solution
	%\input{exemplar/10/13/3/40/main.tex}
 \item A student says that if you throw a die, it will show up 1 or not 1. Therefore, the probability of getting 1 and the probability of getting 'not 1' each is equal to $\frac{1}{2}$. Is this correct? Give reasons.\\
 \solution
        %\input{exemplar/10/13/2/9/main.tex}
   \item Four candidates A, B, C, D have ap-
plied for the assignment to coach a school cricket
team. If A is twice as likely to be selected as B, and
B and C are given about the same chance of being
selected, while C is twice as likely to be selected
as D, what are the probabilities that
\begin{enumerate}
\item C will be selected?
\item A will not be selected?
\end{enumerate}
	%\input{exemplar/11/16/3/9/main.tex}
 \item A bag contain 24 balls of which $x$ balls are red, $2x$ are white and $3x$ are blue. A ball is selected at random, What is the probability that it is
\begin{enumerate}[label=\alph*)]
\item not red ?
\item white ?
\end{enumerate}
%\input{exemplar/10/13/3/41/main.tex}
If the letters of the word ASSASSINATION are arranged at random. Find the Probability that
\begin{enumerate}[label=(\alph*)]
\item Four $S's$ come consecutively in the word
\item Two  $I's$ and two $N's$ come together
\item All $A's$ are not coming together
\item No two $A's$ are coming together
\end{enumerate}
%\input{exemplar/11/16/3/14/main.tex}
	\item One urn contains two black balls (labelled B1 and B2) and one white ball. A
	second urn contains one black ball and two white balls (labelled W1 and W2).
	Suppose the following experiment is performed. One of the two urns is chosen
	at random. Next a ball is randomly chosen from the urn. Then a second ball is
	chosen at random from the same urn without replacing the first ball.
	
	\begin{enumerate}
	\item What is the probability that two black balls are chosen?
	
	\item What is the probability that two balls of opposite colour are chosen?
	\end{enumerate}
	\solution
	%\input{exemplar/11/16/3/12/main1.tex}
\end{enumerate}

	\item A card is selected from a pack of 52 cards.
 \begin{enumerate}[label=(\alph*)] 
                 \item How many points are there in the sample space?
                 \item Calculate the probability that the card is an ace of spades.
                 \item Calculate the probability that the card is (i) an ace and (ii) black card.
 \end{enumerate}
\solution
		%\begin{table}[H]
	\centering
\begin{tabular}{|c|c|c|}
\hline
Random variable &Value &Definition\\ \hline
\multirow{3}{*}{X} &0 &Slips of Rs 1\\
&1 &Slips of Rs 5\\
&2 &Slips of Rs 13\\ \hline
\multirow{2}{*}{Y} &0 &Box A\\
&1 &Box B\\\hline
\end{tabular}
\caption{}
\label{tab:Distribution}
\end{table}
See \tabref{tab:Distribution}.
\begin{align}
p_{Y}\brak{k}= \begin{cases} 
      \frac{1}{3} & {k=0} \\
      \frac{2}{3 }& {k=1} 
   \end{cases}
   \\
p_{Y|X}\brak{0|0} = \frac{19}{25}\, 
p_{Y|X}\brak{0|1} = \frac{6}{25}\,
p_{Y|X}\brak{1|0} = \frac{45}{50}\,
p_{Y|X}\brak{1|2} = \frac{5}{50}
\end{align}
The desired probability is the probability that a slip drawn at random is marked other than Rs 1,
\begin{align}
&=1-p_X\brak{0}\\
&= p_X(1) + p_X(2)
\end{align}
Using Bayes theorem,
\begin{align}
&= p_Y\brak{0} \times \pr{Y=0 | X=1} + p_Y\brak{1} \times \pr{Y=1|X=2}\\
&=\frac{1}{3} \times \frac{6}{25} + \frac{2}{3} \times \frac{5}{50}\\
&=\frac{11}{75}
\end{align}

\newpage

%\tableofcontents

\bigskip

\renewcommand{\thefigure}{\theenumi}
\renewcommand{\thetable}{\theenumi}
%\renewcommand{\theequation}{\theenumi}

%\begin{abstract}
%%\boldmath
%In this letter, an algorithm for evaluating the exact analytical bit error rate  (BER)  for the piecewise linear (PL) combiner for  multiple relays is presented. Previous results were available only for upto three relays. The algorithm is unique in the sense that  the actual mathematical expressions, that are prohibitively large, need not be explicitly obtained. The diversity gain due to multiple relays is shown through plots of the analytical BER, well supported by simulations. 
%
%\end{abstract}
% IEEEtran.cls defaults to using nonbold math in the Abstract.
% This preserves the distinction between vectors and scalars. However,
% if the journal you are submitting to favors bold math in the abstract,
% then you can use LaTeX's standard command \boldmath at the very start
% of the abstract to achieve this. Many IEEE journals frown on math
% in the abstract anyway.

% Note that keywords are not normally used for peerreview papers.
%\begin{IEEEkeywords}
%Cooperative diversity, decode and forward, piecewise linear
%\end{IEEEkeywords}



% For peer review papers, you can put extra information on the cover
% page as needed:
% \ifCLASSOPTIONpeerreview
% \begin{center} \bfseries EDICS Category: 3-BBND \end{center}
% \fi
%
% For peerreview papers, this IEEEtran command inserts a page break and
% creates the second title. It will be ignored for other modes.
%\IEEEpeerreviewmaketitle




\item Four cards are drawn from a well-shuffled deck of 52 cards. What is the probability of obtaining 3 diamonds and one spade.
\\
\solution
		%\begin{enumerate}[label=\thesection.\arabic*,ref=\thesection.\theenumi]
	\item One card is drawn from a well-shuffled deck of 52 cards. Find the probability of getting
\begin{enumerate}
\item A king of red colour 
\item A face card 
\item A red face card
\item The jack of hearts
\item A spade
\item The queen of diamonds

\end{enumerate}
\solution
		%\input{ncert/10/15/1/14/main.tex}
	\item Five cards—the ten, jack, queen, king and ace of diamonds, are well-shuffled with their face downwards. One card is then picked up at random.
\begin{enumerate}
\item
What is the probability that the card is the queen? 
\item
If the queen is drawn and put aside, what is the probability that the second card picked up is (a) an ace? (b) a queen?\\
\end{enumerate}
\solution
		%\input{ncert/10/15/1/15/defs.tex}
	\item A bag contains $5$ red balls and some blue balls. If the probability of drawing a blue ball is double that if a red ball, determine the number of blue balls in the bag. 
		\\
\solution
		%\input{ncert/10/15/2/3/defs.tex}
	\item A card is selected from a pack of 52 cards.
 \begin{enumerate}[label=(\alph*)] 
                 \item How many points are there in the sample space?
                 \item Calculate the probability that the card is an ace of spades.
                 \item Calculate the probability that the card is (i) an ace and (ii) black card.
 \end{enumerate}
\solution
		%\input{ncert/11/16/3/4/main.tex}
\item Four cards are drawn from a well-shuffled deck of 52 cards. What is the probability of obtaining 3 diamonds and one spade.
\\
\solution
		%\input{ncert/11/16/4/2/defs.tex}
\item In a certain lottery 10,000 tickets are sold and ten equal prizes are awarded. What is the probability of not getting a prize if you buy (a) one ticket (b) two tickets (c) 10 tickets ?	
\\
\solution
		%\input{ncert/11/16/4/4/defs.tex}
		%
\item 
Out of 100 students, two sections of 40 and 60 are formed. If you and your friend are among the 100 students, what is the probability that
\begin{enumerate}
\item you both enter the same section?
\item you both enter the different sections?
\end{enumerate}
\solution
		%\input{ncert/11/16/4/5/defs.tex}
	\item 
The number lock of a suitcase has 4 wheels each labelled with ten digits i.e. from 0 to 9.The lock opens with a sequence of four digits with no repeats.What is the probability of a person getting the right sequence to open the suitcase.
\\
\solution
		%\input{ncert/11/16/4/10/defs.tex}
		%
\item 
Two cards are drawn at random and without replacement from a pack of 52 playing cards. Find the probability that both the cards are black.
\\
\solution
		%\input{ncert/12/13/2/2/defs.tex}
		\item A box of oranges is inspected by examining three randomly selected oranges drawn without replacement. If all the three oranges are good, the box is approved for sale, otherwise, it is rejected. Find the probability that a box containing 15 oranges out of which 12 are good and 3 are bad ones will be approved for sale.
		\label{ncert/12/13/2/3/defs.tex}
		\item Two balls are drawn at random with replacement from a box containing 10 black and 8 red balls. Find the probability that
		\label{ncert/12/13/2/12}
\begin{enumerate}
\item both balls are red.
\item first ball is black and second is red.
\item one of them is black and other is red.
\end{enumerate}

\item In a hostel, 60\% of the students read Hindi newspaper, 40\% read English newspaper and 20\% read both Hindi and English newspapers. A student is selected at random.
		\label{ncert/12/13/2/15}
\begin{enumerate}
\item Find the probability that she reads neither Hindi nor English newspapers.
\item If she reads Hindi newspaper, find the probability that she reads English newspaper.
\item If she reads English newspaper, find the probability that she reads Hindi newspaper.\\
\end{enumerate}
\item The probability of obtaining an even prime number on each die, when a pair of dice is rolled is 
\begin{enumerate}
    \item $0$ 
    
    \item $\frac{1}{3}$ 
    
    \item $\frac{1}{12}$ 
    
    \item $\frac{1}{36}$ 
\end{enumerate}
\solution
		%\input{ncert/12/13/2/17/defs.tex}
	\item A bag contains 4 red and 4 black balls, another bag contains 2 red and 6 black balls. One of the two bags is selected at random and a ball is drawn from the bag which is found to be red. Find the probability that the ball is drawn from the first bag.
\\
\solution
		%\input{ncert/12/13/3/2/main.tex}
  \item
  Cards with numbers 2 to 101 are placed in a box. A card is selected at random.Find the probability that the card has
\begin{enumerate}[label=(\roman*)]
	\item an even number 
	\item a square number
\end{enumerate}
\solution
%\input{exemplar/10/13/3/32/main.tex}
\item
The king, queen and jack of clubs are removed from a deck of 52 playing cards and then well shuffled. Now one card is drawn at random from the remaining cards.  Determine the probability that the card is
\begin{enumerate}[label=(\roman*)]
\item a club
\item 10 of hearts
\end{enumerate}
\solution
%\input{exemplar/10/13/3/29/main.tex}
\item A team of medical students doing their internship have to assist during surgeries
at a city hospital. The probabilities of surgeries rated as very complex, complex,
routine, simple or very simple are respectively, 0.15, 0.20, 0.31, 0.26, .08. Find
the probabilities that a particular surgery will be rated
\begin{enumerate}
	\item complex or very complex;
	\item neither very complex nor very simple;
	\item routine or complex
	\item routine or simple
\end{enumerate}
\solution
%\input{exemplar/11/16/3/8(1)/main.tex}
\item A card is selected from a pack of 52 cards.
\begin{enumerate}[label=(\alph*)]
    \item How many points are there in the sample space?
    \item Calculate the probability that the card is an ace of spades.
    \item Calculate the probability that the card is (i) an ace and (ii) black card.
\end{enumerate}
\solution
%\input{exemplar/11/16/3/4/main2.tex}
\item The probability that a non leap year selected at random will contain 53 sundays.
\\
\solution
%\input{exemplar/10/13/1/19/main.tex}
\item One of the four persons John, Rita, Aslam or Gurpreet will be promoted next
month. Consequently the sample space consists of four elementary outcomes
S = {John promoted, Rita promoted, Aslam promoted, Gurpreet promoted}
You are told that the chances of John’s promotion is same as that of Gurpreet,
Rita’s chances of promotion are twice as likely as Johns. Aslam’s chances are
four times that of John.
\begin{enumerate}
	\item Determine
	\begin{enumerate}
		\item P (John promoted)
		\item P (Rita promoted)
		\item P (Aslam promoted)
		\item P (Gurpreet promoted)
	\end{enumerate}
	\item If A = {John promoted or Gurpreet promoted}, find P (A).
\end{enumerate}
\solution
%\input{exemplar/11/16/3/10/main.tex}
\item A card is drawn from a deck of 52 cards. Find the probability of getting a king or a heart or a red card.\\
\solution
%\input{exemplar/11/16/3/15/main.tex}
\item The probability that a student will pass his examination is 0.73, the probability of
the student getting a compartment is 0.13, and the probability that the student will
either pass or get compartment is 0.96. State True or False.\\
\solution
%\input{exemplar/11/16/3/31/main.tex}
\item A card is selected from a pack of 52 cards\\
\begin{enumerate}[label=(\alph*)]
\item How many points are there in the sample space?
\item Calculate the probability that the cards is an ace of spades.
\item Calculate the probability that the card is (i) an ace (ii)black card.\\
\end{enumerate}
%\input{ncert/11/16/3/4_1/Prob_4.tex}
\item In a non-leap year, the probability of having 53 tuesdays or 53 wednesdays is\\
\solution
%\input{exemplar/11/16/3/18/main.tex}
\item There are 1000 sealed envelopes in a box, 10 of them contain a cash prize of
Rs 100 each, 100 of them contain a cash prize of Rs 50 each and 200 of them
contain a cash prize of Rs 10 each and rest do not contain any cash prize. If they
are well shuffled and an envelope is picked up out, what is the probability that it
contains no cash prize?\\
\solution
%\input{exemplar/10/13/3/34/main.tex}
\item 
A die is thrown and a card is selected at random from a deck of 52 playing cards. The probability of getting an even number on the die and a spade card.\\
\solution
%\input{exemplar/12/13/3/78/main.tex}
\item
If 4-digit numbers greater than 5,000 are randomly formed from the digits 0, 1, 3, 5, and 7, what is the probability of forming a number divisible by 5 when:
\begin{enumerate}
    \item The digits are repeated?
    \item The repetition of digits is not allowed?
\end{enumerate}
\solution
%\input{ncert/11/16/4/9/main.tex}
\item Consider the probability space $\brak{\Omega, \mathcal{G}, P}$ where $\Omega = [0,2]$ and $\mathcal{G} = \cbrak{\phi, \Omega, [0,1], (1,2]}$. Let $X$ and $Y$ be two functions on $\Omega$ defined as
\begin{align*}
    X(\omega) = 
    \begin{cases}
        1 & \text{if }\omega \in [0, 1]\\
        2 & \text{if }\omega \in (1, 2]
    \end{cases}
\end{align*}
and
\begin{align*}
    Y(\omega) = 
    \begin{cases}
        2 & \text{if }\omega \in [0, 1.5]\\
        3 & \text{if }\omega \in (1.5, 2].
    \end{cases}
\end{align*}
Then which one of the following statements is true?
\begin{enumerate}
    \item [(A)] $X$ is a random variable with respect to $\mathcal{G}$, but $Y$ is not a random variable with respect to $\mathcal{G}$.
    \item [(B)] $Y$ is a random variable with respect to $\mathcal{G}$, but $X$ is not a random variable with respect to $\mathcal{G}$.
    \item [(C)] Neither $X$ nor $Y$ is a random variable with respect to $\mathcal{G}$.
    \item [(D)] Both $X$ and $Y$ are random variables with respect to $\mathcal{G}$.
\end{enumerate} \hfill (GATE ST 2023)\\
\solution
%\input{gate/ST/2023/14/main.tex}
	\item  A die is loaded in such a way that each odd number is twice as likely to occur as
each even number. Find $P(G)$, where $G$ is the event that a number greater than
3 occurs on a single roll of the die.
\\
\solution
		%\input{exemplar/11/16/3/5/main.tex}
	\item All the jacks, queens and kings are removed from a deck of 52 playing cards. The remaining cards are well shuffled and then one card is drawn at random. Giving ace a value 1 similar value for other cards, find the probability that the card has a value 
		\begin{enumerate}
			\item 7
			\item greater than 7
			\item less than 7
		\end{enumerate}
		%\input{exemplar/10/13/3/30/main.tex}
  \item A Lot consists of 48 mobile phones of which 42 are good, 3 have only minor defects and 3 have major defects.Varnika will buy a phone if it is good but the trader will only buy a mobile if it has no major defects. One phone is selected at random from the lot. What is the probability that it is
\begin{enumerate}
	\item acceptable to Varnika?
            \item acceptable to the trader?
\end{enumerate}
\solution
	%\input{exemplar/10/13/3/40/main.tex}
 \item A student says that if you throw a die, it will show up 1 or not 1. Therefore, the probability of getting 1 and the probability of getting 'not 1' each is equal to $\frac{1}{2}$. Is this correct? Give reasons.\\
 \solution
        %\input{exemplar/10/13/2/9/main.tex}
   \item Four candidates A, B, C, D have ap-
plied for the assignment to coach a school cricket
team. If A is twice as likely to be selected as B, and
B and C are given about the same chance of being
selected, while C is twice as likely to be selected
as D, what are the probabilities that
\begin{enumerate}
\item C will be selected?
\item A will not be selected?
\end{enumerate}
	%\input{exemplar/11/16/3/9/main.tex}
 \item A bag contain 24 balls of which $x$ balls are red, $2x$ are white and $3x$ are blue. A ball is selected at random, What is the probability that it is
\begin{enumerate}[label=\alph*)]
\item not red ?
\item white ?
\end{enumerate}
%\input{exemplar/10/13/3/41/main.tex}
If the letters of the word ASSASSINATION are arranged at random. Find the Probability that
\begin{enumerate}[label=(\alph*)]
\item Four $S's$ come consecutively in the word
\item Two  $I's$ and two $N's$ come together
\item All $A's$ are not coming together
\item No two $A's$ are coming together
\end{enumerate}
%\input{exemplar/11/16/3/14/main.tex}
	\item One urn contains two black balls (labelled B1 and B2) and one white ball. A
	second urn contains one black ball and two white balls (labelled W1 and W2).
	Suppose the following experiment is performed. One of the two urns is chosen
	at random. Next a ball is randomly chosen from the urn. Then a second ball is
	chosen at random from the same urn without replacing the first ball.
	
	\begin{enumerate}
	\item What is the probability that two black balls are chosen?
	
	\item What is the probability that two balls of opposite colour are chosen?
	\end{enumerate}
	\solution
	%\input{exemplar/11/16/3/12/main1.tex}
\end{enumerate}

\item In a certain lottery 10,000 tickets are sold and ten equal prizes are awarded. What is the probability of not getting a prize if you buy (a) one ticket (b) two tickets (c) 10 tickets ?	
\\
\solution
		%\begin{enumerate}[label=\thesection.\arabic*,ref=\thesection.\theenumi]
	\item One card is drawn from a well-shuffled deck of 52 cards. Find the probability of getting
\begin{enumerate}
\item A king of red colour 
\item A face card 
\item A red face card
\item The jack of hearts
\item A spade
\item The queen of diamonds

\end{enumerate}
\solution
		%\input{ncert/10/15/1/14/main.tex}
	\item Five cards—the ten, jack, queen, king and ace of diamonds, are well-shuffled with their face downwards. One card is then picked up at random.
\begin{enumerate}
\item
What is the probability that the card is the queen? 
\item
If the queen is drawn and put aside, what is the probability that the second card picked up is (a) an ace? (b) a queen?\\
\end{enumerate}
\solution
		%\input{ncert/10/15/1/15/defs.tex}
	\item A bag contains $5$ red balls and some blue balls. If the probability of drawing a blue ball is double that if a red ball, determine the number of blue balls in the bag. 
		\\
\solution
		%\input{ncert/10/15/2/3/defs.tex}
	\item A card is selected from a pack of 52 cards.
 \begin{enumerate}[label=(\alph*)] 
                 \item How many points are there in the sample space?
                 \item Calculate the probability that the card is an ace of spades.
                 \item Calculate the probability that the card is (i) an ace and (ii) black card.
 \end{enumerate}
\solution
		%\input{ncert/11/16/3/4/main.tex}
\item Four cards are drawn from a well-shuffled deck of 52 cards. What is the probability of obtaining 3 diamonds and one spade.
\\
\solution
		%\input{ncert/11/16/4/2/defs.tex}
\item In a certain lottery 10,000 tickets are sold and ten equal prizes are awarded. What is the probability of not getting a prize if you buy (a) one ticket (b) two tickets (c) 10 tickets ?	
\\
\solution
		%\input{ncert/11/16/4/4/defs.tex}
		%
\item 
Out of 100 students, two sections of 40 and 60 are formed. If you and your friend are among the 100 students, what is the probability that
\begin{enumerate}
\item you both enter the same section?
\item you both enter the different sections?
\end{enumerate}
\solution
		%\input{ncert/11/16/4/5/defs.tex}
	\item 
The number lock of a suitcase has 4 wheels each labelled with ten digits i.e. from 0 to 9.The lock opens with a sequence of four digits with no repeats.What is the probability of a person getting the right sequence to open the suitcase.
\\
\solution
		%\input{ncert/11/16/4/10/defs.tex}
		%
\item 
Two cards are drawn at random and without replacement from a pack of 52 playing cards. Find the probability that both the cards are black.
\\
\solution
		%\input{ncert/12/13/2/2/defs.tex}
		\item A box of oranges is inspected by examining three randomly selected oranges drawn without replacement. If all the three oranges are good, the box is approved for sale, otherwise, it is rejected. Find the probability that a box containing 15 oranges out of which 12 are good and 3 are bad ones will be approved for sale.
		\label{ncert/12/13/2/3/defs.tex}
		\item Two balls are drawn at random with replacement from a box containing 10 black and 8 red balls. Find the probability that
		\label{ncert/12/13/2/12}
\begin{enumerate}
\item both balls are red.
\item first ball is black and second is red.
\item one of them is black and other is red.
\end{enumerate}

\item In a hostel, 60\% of the students read Hindi newspaper, 40\% read English newspaper and 20\% read both Hindi and English newspapers. A student is selected at random.
		\label{ncert/12/13/2/15}
\begin{enumerate}
\item Find the probability that she reads neither Hindi nor English newspapers.
\item If she reads Hindi newspaper, find the probability that she reads English newspaper.
\item If she reads English newspaper, find the probability that she reads Hindi newspaper.\\
\end{enumerate}
\item The probability of obtaining an even prime number on each die, when a pair of dice is rolled is 
\begin{enumerate}
    \item $0$ 
    
    \item $\frac{1}{3}$ 
    
    \item $\frac{1}{12}$ 
    
    \item $\frac{1}{36}$ 
\end{enumerate}
\solution
		%\input{ncert/12/13/2/17/defs.tex}
	\item A bag contains 4 red and 4 black balls, another bag contains 2 red and 6 black balls. One of the two bags is selected at random and a ball is drawn from the bag which is found to be red. Find the probability that the ball is drawn from the first bag.
\\
\solution
		%\input{ncert/12/13/3/2/main.tex}
  \item
  Cards with numbers 2 to 101 are placed in a box. A card is selected at random.Find the probability that the card has
\begin{enumerate}[label=(\roman*)]
	\item an even number 
	\item a square number
\end{enumerate}
\solution
%\input{exemplar/10/13/3/32/main.tex}
\item
The king, queen and jack of clubs are removed from a deck of 52 playing cards and then well shuffled. Now one card is drawn at random from the remaining cards.  Determine the probability that the card is
\begin{enumerate}[label=(\roman*)]
\item a club
\item 10 of hearts
\end{enumerate}
\solution
%\input{exemplar/10/13/3/29/main.tex}
\item A team of medical students doing their internship have to assist during surgeries
at a city hospital. The probabilities of surgeries rated as very complex, complex,
routine, simple or very simple are respectively, 0.15, 0.20, 0.31, 0.26, .08. Find
the probabilities that a particular surgery will be rated
\begin{enumerate}
	\item complex or very complex;
	\item neither very complex nor very simple;
	\item routine or complex
	\item routine or simple
\end{enumerate}
\solution
%\input{exemplar/11/16/3/8(1)/main.tex}
\item A card is selected from a pack of 52 cards.
\begin{enumerate}[label=(\alph*)]
    \item How many points are there in the sample space?
    \item Calculate the probability that the card is an ace of spades.
    \item Calculate the probability that the card is (i) an ace and (ii) black card.
\end{enumerate}
\solution
%\input{exemplar/11/16/3/4/main2.tex}
\item The probability that a non leap year selected at random will contain 53 sundays.
\\
\solution
%\input{exemplar/10/13/1/19/main.tex}
\item One of the four persons John, Rita, Aslam or Gurpreet will be promoted next
month. Consequently the sample space consists of four elementary outcomes
S = {John promoted, Rita promoted, Aslam promoted, Gurpreet promoted}
You are told that the chances of John’s promotion is same as that of Gurpreet,
Rita’s chances of promotion are twice as likely as Johns. Aslam’s chances are
four times that of John.
\begin{enumerate}
	\item Determine
	\begin{enumerate}
		\item P (John promoted)
		\item P (Rita promoted)
		\item P (Aslam promoted)
		\item P (Gurpreet promoted)
	\end{enumerate}
	\item If A = {John promoted or Gurpreet promoted}, find P (A).
\end{enumerate}
\solution
%\input{exemplar/11/16/3/10/main.tex}
\item A card is drawn from a deck of 52 cards. Find the probability of getting a king or a heart or a red card.\\
\solution
%\input{exemplar/11/16/3/15/main.tex}
\item The probability that a student will pass his examination is 0.73, the probability of
the student getting a compartment is 0.13, and the probability that the student will
either pass or get compartment is 0.96. State True or False.\\
\solution
%\input{exemplar/11/16/3/31/main.tex}
\item A card is selected from a pack of 52 cards\\
\begin{enumerate}[label=(\alph*)]
\item How many points are there in the sample space?
\item Calculate the probability that the cards is an ace of spades.
\item Calculate the probability that the card is (i) an ace (ii)black card.\\
\end{enumerate}
%\input{ncert/11/16/3/4_1/Prob_4.tex}
\item In a non-leap year, the probability of having 53 tuesdays or 53 wednesdays is\\
\solution
%\input{exemplar/11/16/3/18/main.tex}
\item There are 1000 sealed envelopes in a box, 10 of them contain a cash prize of
Rs 100 each, 100 of them contain a cash prize of Rs 50 each and 200 of them
contain a cash prize of Rs 10 each and rest do not contain any cash prize. If they
are well shuffled and an envelope is picked up out, what is the probability that it
contains no cash prize?\\
\solution
%\input{exemplar/10/13/3/34/main.tex}
\item 
A die is thrown and a card is selected at random from a deck of 52 playing cards. The probability of getting an even number on the die and a spade card.\\
\solution
%\input{exemplar/12/13/3/78/main.tex}
\item
If 4-digit numbers greater than 5,000 are randomly formed from the digits 0, 1, 3, 5, and 7, what is the probability of forming a number divisible by 5 when:
\begin{enumerate}
    \item The digits are repeated?
    \item The repetition of digits is not allowed?
\end{enumerate}
\solution
%\input{ncert/11/16/4/9/main.tex}
\item Consider the probability space $\brak{\Omega, \mathcal{G}, P}$ where $\Omega = [0,2]$ and $\mathcal{G} = \cbrak{\phi, \Omega, [0,1], (1,2]}$. Let $X$ and $Y$ be two functions on $\Omega$ defined as
\begin{align*}
    X(\omega) = 
    \begin{cases}
        1 & \text{if }\omega \in [0, 1]\\
        2 & \text{if }\omega \in (1, 2]
    \end{cases}
\end{align*}
and
\begin{align*}
    Y(\omega) = 
    \begin{cases}
        2 & \text{if }\omega \in [0, 1.5]\\
        3 & \text{if }\omega \in (1.5, 2].
    \end{cases}
\end{align*}
Then which one of the following statements is true?
\begin{enumerate}
    \item [(A)] $X$ is a random variable with respect to $\mathcal{G}$, but $Y$ is not a random variable with respect to $\mathcal{G}$.
    \item [(B)] $Y$ is a random variable with respect to $\mathcal{G}$, but $X$ is not a random variable with respect to $\mathcal{G}$.
    \item [(C)] Neither $X$ nor $Y$ is a random variable with respect to $\mathcal{G}$.
    \item [(D)] Both $X$ and $Y$ are random variables with respect to $\mathcal{G}$.
\end{enumerate} \hfill (GATE ST 2023)\\
\solution
%\input{gate/ST/2023/14/main.tex}
	\item  A die is loaded in such a way that each odd number is twice as likely to occur as
each even number. Find $P(G)$, where $G$ is the event that a number greater than
3 occurs on a single roll of the die.
\\
\solution
		%\input{exemplar/11/16/3/5/main.tex}
	\item All the jacks, queens and kings are removed from a deck of 52 playing cards. The remaining cards are well shuffled and then one card is drawn at random. Giving ace a value 1 similar value for other cards, find the probability that the card has a value 
		\begin{enumerate}
			\item 7
			\item greater than 7
			\item less than 7
		\end{enumerate}
		%\input{exemplar/10/13/3/30/main.tex}
  \item A Lot consists of 48 mobile phones of which 42 are good, 3 have only minor defects and 3 have major defects.Varnika will buy a phone if it is good but the trader will only buy a mobile if it has no major defects. One phone is selected at random from the lot. What is the probability that it is
\begin{enumerate}
	\item acceptable to Varnika?
            \item acceptable to the trader?
\end{enumerate}
\solution
	%\input{exemplar/10/13/3/40/main.tex}
 \item A student says that if you throw a die, it will show up 1 or not 1. Therefore, the probability of getting 1 and the probability of getting 'not 1' each is equal to $\frac{1}{2}$. Is this correct? Give reasons.\\
 \solution
        %\input{exemplar/10/13/2/9/main.tex}
   \item Four candidates A, B, C, D have ap-
plied for the assignment to coach a school cricket
team. If A is twice as likely to be selected as B, and
B and C are given about the same chance of being
selected, while C is twice as likely to be selected
as D, what are the probabilities that
\begin{enumerate}
\item C will be selected?
\item A will not be selected?
\end{enumerate}
	%\input{exemplar/11/16/3/9/main.tex}
 \item A bag contain 24 balls of which $x$ balls are red, $2x$ are white and $3x$ are blue. A ball is selected at random, What is the probability that it is
\begin{enumerate}[label=\alph*)]
\item not red ?
\item white ?
\end{enumerate}
%\input{exemplar/10/13/3/41/main.tex}
If the letters of the word ASSASSINATION are arranged at random. Find the Probability that
\begin{enumerate}[label=(\alph*)]
\item Four $S's$ come consecutively in the word
\item Two  $I's$ and two $N's$ come together
\item All $A's$ are not coming together
\item No two $A's$ are coming together
\end{enumerate}
%\input{exemplar/11/16/3/14/main.tex}
	\item One urn contains two black balls (labelled B1 and B2) and one white ball. A
	second urn contains one black ball and two white balls (labelled W1 and W2).
	Suppose the following experiment is performed. One of the two urns is chosen
	at random. Next a ball is randomly chosen from the urn. Then a second ball is
	chosen at random from the same urn without replacing the first ball.
	
	\begin{enumerate}
	\item What is the probability that two black balls are chosen?
	
	\item What is the probability that two balls of opposite colour are chosen?
	\end{enumerate}
	\solution
	%\input{exemplar/11/16/3/12/main1.tex}
\end{enumerate}

		%
\item 
Out of 100 students, two sections of 40 and 60 are formed. If you and your friend are among the 100 students, what is the probability that
\begin{enumerate}
\item you both enter the same section?
\item you both enter the different sections?
\end{enumerate}
\solution
		%\begin{enumerate}[label=\thesection.\arabic*,ref=\thesection.\theenumi]
	\item One card is drawn from a well-shuffled deck of 52 cards. Find the probability of getting
\begin{enumerate}
\item A king of red colour 
\item A face card 
\item A red face card
\item The jack of hearts
\item A spade
\item The queen of diamonds

\end{enumerate}
\solution
		%\input{ncert/10/15/1/14/main.tex}
	\item Five cards—the ten, jack, queen, king and ace of diamonds, are well-shuffled with their face downwards. One card is then picked up at random.
\begin{enumerate}
\item
What is the probability that the card is the queen? 
\item
If the queen is drawn and put aside, what is the probability that the second card picked up is (a) an ace? (b) a queen?\\
\end{enumerate}
\solution
		%\input{ncert/10/15/1/15/defs.tex}
	\item A bag contains $5$ red balls and some blue balls. If the probability of drawing a blue ball is double that if a red ball, determine the number of blue balls in the bag. 
		\\
\solution
		%\input{ncert/10/15/2/3/defs.tex}
	\item A card is selected from a pack of 52 cards.
 \begin{enumerate}[label=(\alph*)] 
                 \item How many points are there in the sample space?
                 \item Calculate the probability that the card is an ace of spades.
                 \item Calculate the probability that the card is (i) an ace and (ii) black card.
 \end{enumerate}
\solution
		%\input{ncert/11/16/3/4/main.tex}
\item Four cards are drawn from a well-shuffled deck of 52 cards. What is the probability of obtaining 3 diamonds and one spade.
\\
\solution
		%\input{ncert/11/16/4/2/defs.tex}
\item In a certain lottery 10,000 tickets are sold and ten equal prizes are awarded. What is the probability of not getting a prize if you buy (a) one ticket (b) two tickets (c) 10 tickets ?	
\\
\solution
		%\input{ncert/11/16/4/4/defs.tex}
		%
\item 
Out of 100 students, two sections of 40 and 60 are formed. If you and your friend are among the 100 students, what is the probability that
\begin{enumerate}
\item you both enter the same section?
\item you both enter the different sections?
\end{enumerate}
\solution
		%\input{ncert/11/16/4/5/defs.tex}
	\item 
The number lock of a suitcase has 4 wheels each labelled with ten digits i.e. from 0 to 9.The lock opens with a sequence of four digits with no repeats.What is the probability of a person getting the right sequence to open the suitcase.
\\
\solution
		%\input{ncert/11/16/4/10/defs.tex}
		%
\item 
Two cards are drawn at random and without replacement from a pack of 52 playing cards. Find the probability that both the cards are black.
\\
\solution
		%\input{ncert/12/13/2/2/defs.tex}
		\item A box of oranges is inspected by examining three randomly selected oranges drawn without replacement. If all the three oranges are good, the box is approved for sale, otherwise, it is rejected. Find the probability that a box containing 15 oranges out of which 12 are good and 3 are bad ones will be approved for sale.
		\label{ncert/12/13/2/3/defs.tex}
		\item Two balls are drawn at random with replacement from a box containing 10 black and 8 red balls. Find the probability that
		\label{ncert/12/13/2/12}
\begin{enumerate}
\item both balls are red.
\item first ball is black and second is red.
\item one of them is black and other is red.
\end{enumerate}

\item In a hostel, 60\% of the students read Hindi newspaper, 40\% read English newspaper and 20\% read both Hindi and English newspapers. A student is selected at random.
		\label{ncert/12/13/2/15}
\begin{enumerate}
\item Find the probability that she reads neither Hindi nor English newspapers.
\item If she reads Hindi newspaper, find the probability that she reads English newspaper.
\item If she reads English newspaper, find the probability that she reads Hindi newspaper.\\
\end{enumerate}
\item The probability of obtaining an even prime number on each die, when a pair of dice is rolled is 
\begin{enumerate}
    \item $0$ 
    
    \item $\frac{1}{3}$ 
    
    \item $\frac{1}{12}$ 
    
    \item $\frac{1}{36}$ 
\end{enumerate}
\solution
		%\input{ncert/12/13/2/17/defs.tex}
	\item A bag contains 4 red and 4 black balls, another bag contains 2 red and 6 black balls. One of the two bags is selected at random and a ball is drawn from the bag which is found to be red. Find the probability that the ball is drawn from the first bag.
\\
\solution
		%\input{ncert/12/13/3/2/main.tex}
  \item
  Cards with numbers 2 to 101 are placed in a box. A card is selected at random.Find the probability that the card has
\begin{enumerate}[label=(\roman*)]
	\item an even number 
	\item a square number
\end{enumerate}
\solution
%\input{exemplar/10/13/3/32/main.tex}
\item
The king, queen and jack of clubs are removed from a deck of 52 playing cards and then well shuffled. Now one card is drawn at random from the remaining cards.  Determine the probability that the card is
\begin{enumerate}[label=(\roman*)]
\item a club
\item 10 of hearts
\end{enumerate}
\solution
%\input{exemplar/10/13/3/29/main.tex}
\item A team of medical students doing their internship have to assist during surgeries
at a city hospital. The probabilities of surgeries rated as very complex, complex,
routine, simple or very simple are respectively, 0.15, 0.20, 0.31, 0.26, .08. Find
the probabilities that a particular surgery will be rated
\begin{enumerate}
	\item complex or very complex;
	\item neither very complex nor very simple;
	\item routine or complex
	\item routine or simple
\end{enumerate}
\solution
%\input{exemplar/11/16/3/8(1)/main.tex}
\item A card is selected from a pack of 52 cards.
\begin{enumerate}[label=(\alph*)]
    \item How many points are there in the sample space?
    \item Calculate the probability that the card is an ace of spades.
    \item Calculate the probability that the card is (i) an ace and (ii) black card.
\end{enumerate}
\solution
%\input{exemplar/11/16/3/4/main2.tex}
\item The probability that a non leap year selected at random will contain 53 sundays.
\\
\solution
%\input{exemplar/10/13/1/19/main.tex}
\item One of the four persons John, Rita, Aslam or Gurpreet will be promoted next
month. Consequently the sample space consists of four elementary outcomes
S = {John promoted, Rita promoted, Aslam promoted, Gurpreet promoted}
You are told that the chances of John’s promotion is same as that of Gurpreet,
Rita’s chances of promotion are twice as likely as Johns. Aslam’s chances are
four times that of John.
\begin{enumerate}
	\item Determine
	\begin{enumerate}
		\item P (John promoted)
		\item P (Rita promoted)
		\item P (Aslam promoted)
		\item P (Gurpreet promoted)
	\end{enumerate}
	\item If A = {John promoted or Gurpreet promoted}, find P (A).
\end{enumerate}
\solution
%\input{exemplar/11/16/3/10/main.tex}
\item A card is drawn from a deck of 52 cards. Find the probability of getting a king or a heart or a red card.\\
\solution
%\input{exemplar/11/16/3/15/main.tex}
\item The probability that a student will pass his examination is 0.73, the probability of
the student getting a compartment is 0.13, and the probability that the student will
either pass or get compartment is 0.96. State True or False.\\
\solution
%\input{exemplar/11/16/3/31/main.tex}
\item A card is selected from a pack of 52 cards\\
\begin{enumerate}[label=(\alph*)]
\item How many points are there in the sample space?
\item Calculate the probability that the cards is an ace of spades.
\item Calculate the probability that the card is (i) an ace (ii)black card.\\
\end{enumerate}
%\input{ncert/11/16/3/4_1/Prob_4.tex}
\item In a non-leap year, the probability of having 53 tuesdays or 53 wednesdays is\\
\solution
%\input{exemplar/11/16/3/18/main.tex}
\item There are 1000 sealed envelopes in a box, 10 of them contain a cash prize of
Rs 100 each, 100 of them contain a cash prize of Rs 50 each and 200 of them
contain a cash prize of Rs 10 each and rest do not contain any cash prize. If they
are well shuffled and an envelope is picked up out, what is the probability that it
contains no cash prize?\\
\solution
%\input{exemplar/10/13/3/34/main.tex}
\item 
A die is thrown and a card is selected at random from a deck of 52 playing cards. The probability of getting an even number on the die and a spade card.\\
\solution
%\input{exemplar/12/13/3/78/main.tex}
\item
If 4-digit numbers greater than 5,000 are randomly formed from the digits 0, 1, 3, 5, and 7, what is the probability of forming a number divisible by 5 when:
\begin{enumerate}
    \item The digits are repeated?
    \item The repetition of digits is not allowed?
\end{enumerate}
\solution
%\input{ncert/11/16/4/9/main.tex}
\item Consider the probability space $\brak{\Omega, \mathcal{G}, P}$ where $\Omega = [0,2]$ and $\mathcal{G} = \cbrak{\phi, \Omega, [0,1], (1,2]}$. Let $X$ and $Y$ be two functions on $\Omega$ defined as
\begin{align*}
    X(\omega) = 
    \begin{cases}
        1 & \text{if }\omega \in [0, 1]\\
        2 & \text{if }\omega \in (1, 2]
    \end{cases}
\end{align*}
and
\begin{align*}
    Y(\omega) = 
    \begin{cases}
        2 & \text{if }\omega \in [0, 1.5]\\
        3 & \text{if }\omega \in (1.5, 2].
    \end{cases}
\end{align*}
Then which one of the following statements is true?
\begin{enumerate}
    \item [(A)] $X$ is a random variable with respect to $\mathcal{G}$, but $Y$ is not a random variable with respect to $\mathcal{G}$.
    \item [(B)] $Y$ is a random variable with respect to $\mathcal{G}$, but $X$ is not a random variable with respect to $\mathcal{G}$.
    \item [(C)] Neither $X$ nor $Y$ is a random variable with respect to $\mathcal{G}$.
    \item [(D)] Both $X$ and $Y$ are random variables with respect to $\mathcal{G}$.
\end{enumerate} \hfill (GATE ST 2023)\\
\solution
%\input{gate/ST/2023/14/main.tex}
	\item  A die is loaded in such a way that each odd number is twice as likely to occur as
each even number. Find $P(G)$, where $G$ is the event that a number greater than
3 occurs on a single roll of the die.
\\
\solution
		%\input{exemplar/11/16/3/5/main.tex}
	\item All the jacks, queens and kings are removed from a deck of 52 playing cards. The remaining cards are well shuffled and then one card is drawn at random. Giving ace a value 1 similar value for other cards, find the probability that the card has a value 
		\begin{enumerate}
			\item 7
			\item greater than 7
			\item less than 7
		\end{enumerate}
		%\input{exemplar/10/13/3/30/main.tex}
  \item A Lot consists of 48 mobile phones of which 42 are good, 3 have only minor defects and 3 have major defects.Varnika will buy a phone if it is good but the trader will only buy a mobile if it has no major defects. One phone is selected at random from the lot. What is the probability that it is
\begin{enumerate}
	\item acceptable to Varnika?
            \item acceptable to the trader?
\end{enumerate}
\solution
	%\input{exemplar/10/13/3/40/main.tex}
 \item A student says that if you throw a die, it will show up 1 or not 1. Therefore, the probability of getting 1 and the probability of getting 'not 1' each is equal to $\frac{1}{2}$. Is this correct? Give reasons.\\
 \solution
        %\input{exemplar/10/13/2/9/main.tex}
   \item Four candidates A, B, C, D have ap-
plied for the assignment to coach a school cricket
team. If A is twice as likely to be selected as B, and
B and C are given about the same chance of being
selected, while C is twice as likely to be selected
as D, what are the probabilities that
\begin{enumerate}
\item C will be selected?
\item A will not be selected?
\end{enumerate}
	%\input{exemplar/11/16/3/9/main.tex}
 \item A bag contain 24 balls of which $x$ balls are red, $2x$ are white and $3x$ are blue. A ball is selected at random, What is the probability that it is
\begin{enumerate}[label=\alph*)]
\item not red ?
\item white ?
\end{enumerate}
%\input{exemplar/10/13/3/41/main.tex}
If the letters of the word ASSASSINATION are arranged at random. Find the Probability that
\begin{enumerate}[label=(\alph*)]
\item Four $S's$ come consecutively in the word
\item Two  $I's$ and two $N's$ come together
\item All $A's$ are not coming together
\item No two $A's$ are coming together
\end{enumerate}
%\input{exemplar/11/16/3/14/main.tex}
	\item One urn contains two black balls (labelled B1 and B2) and one white ball. A
	second urn contains one black ball and two white balls (labelled W1 and W2).
	Suppose the following experiment is performed. One of the two urns is chosen
	at random. Next a ball is randomly chosen from the urn. Then a second ball is
	chosen at random from the same urn without replacing the first ball.
	
	\begin{enumerate}
	\item What is the probability that two black balls are chosen?
	
	\item What is the probability that two balls of opposite colour are chosen?
	\end{enumerate}
	\solution
	%\input{exemplar/11/16/3/12/main1.tex}
\end{enumerate}

	\item 
The number lock of a suitcase has 4 wheels each labelled with ten digits i.e. from 0 to 9.The lock opens with a sequence of four digits with no repeats.What is the probability of a person getting the right sequence to open the suitcase.
\\
\solution
		%\begin{enumerate}[label=\thesection.\arabic*,ref=\thesection.\theenumi]
	\item One card is drawn from a well-shuffled deck of 52 cards. Find the probability of getting
\begin{enumerate}
\item A king of red colour 
\item A face card 
\item A red face card
\item The jack of hearts
\item A spade
\item The queen of diamonds

\end{enumerate}
\solution
		%\input{ncert/10/15/1/14/main.tex}
	\item Five cards—the ten, jack, queen, king and ace of diamonds, are well-shuffled with their face downwards. One card is then picked up at random.
\begin{enumerate}
\item
What is the probability that the card is the queen? 
\item
If the queen is drawn and put aside, what is the probability that the second card picked up is (a) an ace? (b) a queen?\\
\end{enumerate}
\solution
		%\input{ncert/10/15/1/15/defs.tex}
	\item A bag contains $5$ red balls and some blue balls. If the probability of drawing a blue ball is double that if a red ball, determine the number of blue balls in the bag. 
		\\
\solution
		%\input{ncert/10/15/2/3/defs.tex}
	\item A card is selected from a pack of 52 cards.
 \begin{enumerate}[label=(\alph*)] 
                 \item How many points are there in the sample space?
                 \item Calculate the probability that the card is an ace of spades.
                 \item Calculate the probability that the card is (i) an ace and (ii) black card.
 \end{enumerate}
\solution
		%\input{ncert/11/16/3/4/main.tex}
\item Four cards are drawn from a well-shuffled deck of 52 cards. What is the probability of obtaining 3 diamonds and one spade.
\\
\solution
		%\input{ncert/11/16/4/2/defs.tex}
\item In a certain lottery 10,000 tickets are sold and ten equal prizes are awarded. What is the probability of not getting a prize if you buy (a) one ticket (b) two tickets (c) 10 tickets ?	
\\
\solution
		%\input{ncert/11/16/4/4/defs.tex}
		%
\item 
Out of 100 students, two sections of 40 and 60 are formed. If you and your friend are among the 100 students, what is the probability that
\begin{enumerate}
\item you both enter the same section?
\item you both enter the different sections?
\end{enumerate}
\solution
		%\input{ncert/11/16/4/5/defs.tex}
	\item 
The number lock of a suitcase has 4 wheels each labelled with ten digits i.e. from 0 to 9.The lock opens with a sequence of four digits with no repeats.What is the probability of a person getting the right sequence to open the suitcase.
\\
\solution
		%\input{ncert/11/16/4/10/defs.tex}
		%
\item 
Two cards are drawn at random and without replacement from a pack of 52 playing cards. Find the probability that both the cards are black.
\\
\solution
		%\input{ncert/12/13/2/2/defs.tex}
		\item A box of oranges is inspected by examining three randomly selected oranges drawn without replacement. If all the three oranges are good, the box is approved for sale, otherwise, it is rejected. Find the probability that a box containing 15 oranges out of which 12 are good and 3 are bad ones will be approved for sale.
		\label{ncert/12/13/2/3/defs.tex}
		\item Two balls are drawn at random with replacement from a box containing 10 black and 8 red balls. Find the probability that
		\label{ncert/12/13/2/12}
\begin{enumerate}
\item both balls are red.
\item first ball is black and second is red.
\item one of them is black and other is red.
\end{enumerate}

\item In a hostel, 60\% of the students read Hindi newspaper, 40\% read English newspaper and 20\% read both Hindi and English newspapers. A student is selected at random.
		\label{ncert/12/13/2/15}
\begin{enumerate}
\item Find the probability that she reads neither Hindi nor English newspapers.
\item If she reads Hindi newspaper, find the probability that she reads English newspaper.
\item If she reads English newspaper, find the probability that she reads Hindi newspaper.\\
\end{enumerate}
\item The probability of obtaining an even prime number on each die, when a pair of dice is rolled is 
\begin{enumerate}
    \item $0$ 
    
    \item $\frac{1}{3}$ 
    
    \item $\frac{1}{12}$ 
    
    \item $\frac{1}{36}$ 
\end{enumerate}
\solution
		%\input{ncert/12/13/2/17/defs.tex}
	\item A bag contains 4 red and 4 black balls, another bag contains 2 red and 6 black balls. One of the two bags is selected at random and a ball is drawn from the bag which is found to be red. Find the probability that the ball is drawn from the first bag.
\\
\solution
		%\input{ncert/12/13/3/2/main.tex}
  \item
  Cards with numbers 2 to 101 are placed in a box. A card is selected at random.Find the probability that the card has
\begin{enumerate}[label=(\roman*)]
	\item an even number 
	\item a square number
\end{enumerate}
\solution
%\input{exemplar/10/13/3/32/main.tex}
\item
The king, queen and jack of clubs are removed from a deck of 52 playing cards and then well shuffled. Now one card is drawn at random from the remaining cards.  Determine the probability that the card is
\begin{enumerate}[label=(\roman*)]
\item a club
\item 10 of hearts
\end{enumerate}
\solution
%\input{exemplar/10/13/3/29/main.tex}
\item A team of medical students doing their internship have to assist during surgeries
at a city hospital. The probabilities of surgeries rated as very complex, complex,
routine, simple or very simple are respectively, 0.15, 0.20, 0.31, 0.26, .08. Find
the probabilities that a particular surgery will be rated
\begin{enumerate}
	\item complex or very complex;
	\item neither very complex nor very simple;
	\item routine or complex
	\item routine or simple
\end{enumerate}
\solution
%\input{exemplar/11/16/3/8(1)/main.tex}
\item A card is selected from a pack of 52 cards.
\begin{enumerate}[label=(\alph*)]
    \item How many points are there in the sample space?
    \item Calculate the probability that the card is an ace of spades.
    \item Calculate the probability that the card is (i) an ace and (ii) black card.
\end{enumerate}
\solution
%\input{exemplar/11/16/3/4/main2.tex}
\item The probability that a non leap year selected at random will contain 53 sundays.
\\
\solution
%\input{exemplar/10/13/1/19/main.tex}
\item One of the four persons John, Rita, Aslam or Gurpreet will be promoted next
month. Consequently the sample space consists of four elementary outcomes
S = {John promoted, Rita promoted, Aslam promoted, Gurpreet promoted}
You are told that the chances of John’s promotion is same as that of Gurpreet,
Rita’s chances of promotion are twice as likely as Johns. Aslam’s chances are
four times that of John.
\begin{enumerate}
	\item Determine
	\begin{enumerate}
		\item P (John promoted)
		\item P (Rita promoted)
		\item P (Aslam promoted)
		\item P (Gurpreet promoted)
	\end{enumerate}
	\item If A = {John promoted or Gurpreet promoted}, find P (A).
\end{enumerate}
\solution
%\input{exemplar/11/16/3/10/main.tex}
\item A card is drawn from a deck of 52 cards. Find the probability of getting a king or a heart or a red card.\\
\solution
%\input{exemplar/11/16/3/15/main.tex}
\item The probability that a student will pass his examination is 0.73, the probability of
the student getting a compartment is 0.13, and the probability that the student will
either pass or get compartment is 0.96. State True or False.\\
\solution
%\input{exemplar/11/16/3/31/main.tex}
\item A card is selected from a pack of 52 cards\\
\begin{enumerate}[label=(\alph*)]
\item How many points are there in the sample space?
\item Calculate the probability that the cards is an ace of spades.
\item Calculate the probability that the card is (i) an ace (ii)black card.\\
\end{enumerate}
%\input{ncert/11/16/3/4_1/Prob_4.tex}
\item In a non-leap year, the probability of having 53 tuesdays or 53 wednesdays is\\
\solution
%\input{exemplar/11/16/3/18/main.tex}
\item There are 1000 sealed envelopes in a box, 10 of them contain a cash prize of
Rs 100 each, 100 of them contain a cash prize of Rs 50 each and 200 of them
contain a cash prize of Rs 10 each and rest do not contain any cash prize. If they
are well shuffled and an envelope is picked up out, what is the probability that it
contains no cash prize?\\
\solution
%\input{exemplar/10/13/3/34/main.tex}
\item 
A die is thrown and a card is selected at random from a deck of 52 playing cards. The probability of getting an even number on the die and a spade card.\\
\solution
%\input{exemplar/12/13/3/78/main.tex}
\item
If 4-digit numbers greater than 5,000 are randomly formed from the digits 0, 1, 3, 5, and 7, what is the probability of forming a number divisible by 5 when:
\begin{enumerate}
    \item The digits are repeated?
    \item The repetition of digits is not allowed?
\end{enumerate}
\solution
%\input{ncert/11/16/4/9/main.tex}
\item Consider the probability space $\brak{\Omega, \mathcal{G}, P}$ where $\Omega = [0,2]$ and $\mathcal{G} = \cbrak{\phi, \Omega, [0,1], (1,2]}$. Let $X$ and $Y$ be two functions on $\Omega$ defined as
\begin{align*}
    X(\omega) = 
    \begin{cases}
        1 & \text{if }\omega \in [0, 1]\\
        2 & \text{if }\omega \in (1, 2]
    \end{cases}
\end{align*}
and
\begin{align*}
    Y(\omega) = 
    \begin{cases}
        2 & \text{if }\omega \in [0, 1.5]\\
        3 & \text{if }\omega \in (1.5, 2].
    \end{cases}
\end{align*}
Then which one of the following statements is true?
\begin{enumerate}
    \item [(A)] $X$ is a random variable with respect to $\mathcal{G}$, but $Y$ is not a random variable with respect to $\mathcal{G}$.
    \item [(B)] $Y$ is a random variable with respect to $\mathcal{G}$, but $X$ is not a random variable with respect to $\mathcal{G}$.
    \item [(C)] Neither $X$ nor $Y$ is a random variable with respect to $\mathcal{G}$.
    \item [(D)] Both $X$ and $Y$ are random variables with respect to $\mathcal{G}$.
\end{enumerate} \hfill (GATE ST 2023)\\
\solution
%\input{gate/ST/2023/14/main.tex}
	\item  A die is loaded in such a way that each odd number is twice as likely to occur as
each even number. Find $P(G)$, where $G$ is the event that a number greater than
3 occurs on a single roll of the die.
\\
\solution
		%\input{exemplar/11/16/3/5/main.tex}
	\item All the jacks, queens and kings are removed from a deck of 52 playing cards. The remaining cards are well shuffled and then one card is drawn at random. Giving ace a value 1 similar value for other cards, find the probability that the card has a value 
		\begin{enumerate}
			\item 7
			\item greater than 7
			\item less than 7
		\end{enumerate}
		%\input{exemplar/10/13/3/30/main.tex}
  \item A Lot consists of 48 mobile phones of which 42 are good, 3 have only minor defects and 3 have major defects.Varnika will buy a phone if it is good but the trader will only buy a mobile if it has no major defects. One phone is selected at random from the lot. What is the probability that it is
\begin{enumerate}
	\item acceptable to Varnika?
            \item acceptable to the trader?
\end{enumerate}
\solution
	%\input{exemplar/10/13/3/40/main.tex}
 \item A student says that if you throw a die, it will show up 1 or not 1. Therefore, the probability of getting 1 and the probability of getting 'not 1' each is equal to $\frac{1}{2}$. Is this correct? Give reasons.\\
 \solution
        %\input{exemplar/10/13/2/9/main.tex}
   \item Four candidates A, B, C, D have ap-
plied for the assignment to coach a school cricket
team. If A is twice as likely to be selected as B, and
B and C are given about the same chance of being
selected, while C is twice as likely to be selected
as D, what are the probabilities that
\begin{enumerate}
\item C will be selected?
\item A will not be selected?
\end{enumerate}
	%\input{exemplar/11/16/3/9/main.tex}
 \item A bag contain 24 balls of which $x$ balls are red, $2x$ are white and $3x$ are blue. A ball is selected at random, What is the probability that it is
\begin{enumerate}[label=\alph*)]
\item not red ?
\item white ?
\end{enumerate}
%\input{exemplar/10/13/3/41/main.tex}
If the letters of the word ASSASSINATION are arranged at random. Find the Probability that
\begin{enumerate}[label=(\alph*)]
\item Four $S's$ come consecutively in the word
\item Two  $I's$ and two $N's$ come together
\item All $A's$ are not coming together
\item No two $A's$ are coming together
\end{enumerate}
%\input{exemplar/11/16/3/14/main.tex}
	\item One urn contains two black balls (labelled B1 and B2) and one white ball. A
	second urn contains one black ball and two white balls (labelled W1 and W2).
	Suppose the following experiment is performed. One of the two urns is chosen
	at random. Next a ball is randomly chosen from the urn. Then a second ball is
	chosen at random from the same urn without replacing the first ball.
	
	\begin{enumerate}
	\item What is the probability that two black balls are chosen?
	
	\item What is the probability that two balls of opposite colour are chosen?
	\end{enumerate}
	\solution
	%\input{exemplar/11/16/3/12/main1.tex}
\end{enumerate}

		%
\item 
Two cards are drawn at random and without replacement from a pack of 52 playing cards. Find the probability that both the cards are black.
\\
\solution
		%\begin{enumerate}[label=\thesection.\arabic*,ref=\thesection.\theenumi]
	\item One card is drawn from a well-shuffled deck of 52 cards. Find the probability of getting
\begin{enumerate}
\item A king of red colour 
\item A face card 
\item A red face card
\item The jack of hearts
\item A spade
\item The queen of diamonds

\end{enumerate}
\solution
		%\input{ncert/10/15/1/14/main.tex}
	\item Five cards—the ten, jack, queen, king and ace of diamonds, are well-shuffled with their face downwards. One card is then picked up at random.
\begin{enumerate}
\item
What is the probability that the card is the queen? 
\item
If the queen is drawn and put aside, what is the probability that the second card picked up is (a) an ace? (b) a queen?\\
\end{enumerate}
\solution
		%\input{ncert/10/15/1/15/defs.tex}
	\item A bag contains $5$ red balls and some blue balls. If the probability of drawing a blue ball is double that if a red ball, determine the number of blue balls in the bag. 
		\\
\solution
		%\input{ncert/10/15/2/3/defs.tex}
	\item A card is selected from a pack of 52 cards.
 \begin{enumerate}[label=(\alph*)] 
                 \item How many points are there in the sample space?
                 \item Calculate the probability that the card is an ace of spades.
                 \item Calculate the probability that the card is (i) an ace and (ii) black card.
 \end{enumerate}
\solution
		%\input{ncert/11/16/3/4/main.tex}
\item Four cards are drawn from a well-shuffled deck of 52 cards. What is the probability of obtaining 3 diamonds and one spade.
\\
\solution
		%\input{ncert/11/16/4/2/defs.tex}
\item In a certain lottery 10,000 tickets are sold and ten equal prizes are awarded. What is the probability of not getting a prize if you buy (a) one ticket (b) two tickets (c) 10 tickets ?	
\\
\solution
		%\input{ncert/11/16/4/4/defs.tex}
		%
\item 
Out of 100 students, two sections of 40 and 60 are formed. If you and your friend are among the 100 students, what is the probability that
\begin{enumerate}
\item you both enter the same section?
\item you both enter the different sections?
\end{enumerate}
\solution
		%\input{ncert/11/16/4/5/defs.tex}
	\item 
The number lock of a suitcase has 4 wheels each labelled with ten digits i.e. from 0 to 9.The lock opens with a sequence of four digits with no repeats.What is the probability of a person getting the right sequence to open the suitcase.
\\
\solution
		%\input{ncert/11/16/4/10/defs.tex}
		%
\item 
Two cards are drawn at random and without replacement from a pack of 52 playing cards. Find the probability that both the cards are black.
\\
\solution
		%\input{ncert/12/13/2/2/defs.tex}
		\item A box of oranges is inspected by examining three randomly selected oranges drawn without replacement. If all the three oranges are good, the box is approved for sale, otherwise, it is rejected. Find the probability that a box containing 15 oranges out of which 12 are good and 3 are bad ones will be approved for sale.
		\label{ncert/12/13/2/3/defs.tex}
		\item Two balls are drawn at random with replacement from a box containing 10 black and 8 red balls. Find the probability that
		\label{ncert/12/13/2/12}
\begin{enumerate}
\item both balls are red.
\item first ball is black and second is red.
\item one of them is black and other is red.
\end{enumerate}

\item In a hostel, 60\% of the students read Hindi newspaper, 40\% read English newspaper and 20\% read both Hindi and English newspapers. A student is selected at random.
		\label{ncert/12/13/2/15}
\begin{enumerate}
\item Find the probability that she reads neither Hindi nor English newspapers.
\item If she reads Hindi newspaper, find the probability that she reads English newspaper.
\item If she reads English newspaper, find the probability that she reads Hindi newspaper.\\
\end{enumerate}
\item The probability of obtaining an even prime number on each die, when a pair of dice is rolled is 
\begin{enumerate}
    \item $0$ 
    
    \item $\frac{1}{3}$ 
    
    \item $\frac{1}{12}$ 
    
    \item $\frac{1}{36}$ 
\end{enumerate}
\solution
		%\input{ncert/12/13/2/17/defs.tex}
	\item A bag contains 4 red and 4 black balls, another bag contains 2 red and 6 black balls. One of the two bags is selected at random and a ball is drawn from the bag which is found to be red. Find the probability that the ball is drawn from the first bag.
\\
\solution
		%\input{ncert/12/13/3/2/main.tex}
  \item
  Cards with numbers 2 to 101 are placed in a box. A card is selected at random.Find the probability that the card has
\begin{enumerate}[label=(\roman*)]
	\item an even number 
	\item a square number
\end{enumerate}
\solution
%\input{exemplar/10/13/3/32/main.tex}
\item
The king, queen and jack of clubs are removed from a deck of 52 playing cards and then well shuffled. Now one card is drawn at random from the remaining cards.  Determine the probability that the card is
\begin{enumerate}[label=(\roman*)]
\item a club
\item 10 of hearts
\end{enumerate}
\solution
%\input{exemplar/10/13/3/29/main.tex}
\item A team of medical students doing their internship have to assist during surgeries
at a city hospital. The probabilities of surgeries rated as very complex, complex,
routine, simple or very simple are respectively, 0.15, 0.20, 0.31, 0.26, .08. Find
the probabilities that a particular surgery will be rated
\begin{enumerate}
	\item complex or very complex;
	\item neither very complex nor very simple;
	\item routine or complex
	\item routine or simple
\end{enumerate}
\solution
%\input{exemplar/11/16/3/8(1)/main.tex}
\item A card is selected from a pack of 52 cards.
\begin{enumerate}[label=(\alph*)]
    \item How many points are there in the sample space?
    \item Calculate the probability that the card is an ace of spades.
    \item Calculate the probability that the card is (i) an ace and (ii) black card.
\end{enumerate}
\solution
%\input{exemplar/11/16/3/4/main2.tex}
\item The probability that a non leap year selected at random will contain 53 sundays.
\\
\solution
%\input{exemplar/10/13/1/19/main.tex}
\item One of the four persons John, Rita, Aslam or Gurpreet will be promoted next
month. Consequently the sample space consists of four elementary outcomes
S = {John promoted, Rita promoted, Aslam promoted, Gurpreet promoted}
You are told that the chances of John’s promotion is same as that of Gurpreet,
Rita’s chances of promotion are twice as likely as Johns. Aslam’s chances are
four times that of John.
\begin{enumerate}
	\item Determine
	\begin{enumerate}
		\item P (John promoted)
		\item P (Rita promoted)
		\item P (Aslam promoted)
		\item P (Gurpreet promoted)
	\end{enumerate}
	\item If A = {John promoted or Gurpreet promoted}, find P (A).
\end{enumerate}
\solution
%\input{exemplar/11/16/3/10/main.tex}
\item A card is drawn from a deck of 52 cards. Find the probability of getting a king or a heart or a red card.\\
\solution
%\input{exemplar/11/16/3/15/main.tex}
\item The probability that a student will pass his examination is 0.73, the probability of
the student getting a compartment is 0.13, and the probability that the student will
either pass or get compartment is 0.96. State True or False.\\
\solution
%\input{exemplar/11/16/3/31/main.tex}
\item A card is selected from a pack of 52 cards\\
\begin{enumerate}[label=(\alph*)]
\item How many points are there in the sample space?
\item Calculate the probability that the cards is an ace of spades.
\item Calculate the probability that the card is (i) an ace (ii)black card.\\
\end{enumerate}
%\input{ncert/11/16/3/4_1/Prob_4.tex}
\item In a non-leap year, the probability of having 53 tuesdays or 53 wednesdays is\\
\solution
%\input{exemplar/11/16/3/18/main.tex}
\item There are 1000 sealed envelopes in a box, 10 of them contain a cash prize of
Rs 100 each, 100 of them contain a cash prize of Rs 50 each and 200 of them
contain a cash prize of Rs 10 each and rest do not contain any cash prize. If they
are well shuffled and an envelope is picked up out, what is the probability that it
contains no cash prize?\\
\solution
%\input{exemplar/10/13/3/34/main.tex}
\item 
A die is thrown and a card is selected at random from a deck of 52 playing cards. The probability of getting an even number on the die and a spade card.\\
\solution
%\input{exemplar/12/13/3/78/main.tex}
\item
If 4-digit numbers greater than 5,000 are randomly formed from the digits 0, 1, 3, 5, and 7, what is the probability of forming a number divisible by 5 when:
\begin{enumerate}
    \item The digits are repeated?
    \item The repetition of digits is not allowed?
\end{enumerate}
\solution
%\input{ncert/11/16/4/9/main.tex}
\item Consider the probability space $\brak{\Omega, \mathcal{G}, P}$ where $\Omega = [0,2]$ and $\mathcal{G} = \cbrak{\phi, \Omega, [0,1], (1,2]}$. Let $X$ and $Y$ be two functions on $\Omega$ defined as
\begin{align*}
    X(\omega) = 
    \begin{cases}
        1 & \text{if }\omega \in [0, 1]\\
        2 & \text{if }\omega \in (1, 2]
    \end{cases}
\end{align*}
and
\begin{align*}
    Y(\omega) = 
    \begin{cases}
        2 & \text{if }\omega \in [0, 1.5]\\
        3 & \text{if }\omega \in (1.5, 2].
    \end{cases}
\end{align*}
Then which one of the following statements is true?
\begin{enumerate}
    \item [(A)] $X$ is a random variable with respect to $\mathcal{G}$, but $Y$ is not a random variable with respect to $\mathcal{G}$.
    \item [(B)] $Y$ is a random variable with respect to $\mathcal{G}$, but $X$ is not a random variable with respect to $\mathcal{G}$.
    \item [(C)] Neither $X$ nor $Y$ is a random variable with respect to $\mathcal{G}$.
    \item [(D)] Both $X$ and $Y$ are random variables with respect to $\mathcal{G}$.
\end{enumerate} \hfill (GATE ST 2023)\\
\solution
%\input{gate/ST/2023/14/main.tex}
	\item  A die is loaded in such a way that each odd number is twice as likely to occur as
each even number. Find $P(G)$, where $G$ is the event that a number greater than
3 occurs on a single roll of the die.
\\
\solution
		%\input{exemplar/11/16/3/5/main.tex}
	\item All the jacks, queens and kings are removed from a deck of 52 playing cards. The remaining cards are well shuffled and then one card is drawn at random. Giving ace a value 1 similar value for other cards, find the probability that the card has a value 
		\begin{enumerate}
			\item 7
			\item greater than 7
			\item less than 7
		\end{enumerate}
		%\input{exemplar/10/13/3/30/main.tex}
  \item A Lot consists of 48 mobile phones of which 42 are good, 3 have only minor defects and 3 have major defects.Varnika will buy a phone if it is good but the trader will only buy a mobile if it has no major defects. One phone is selected at random from the lot. What is the probability that it is
\begin{enumerate}
	\item acceptable to Varnika?
            \item acceptable to the trader?
\end{enumerate}
\solution
	%\input{exemplar/10/13/3/40/main.tex}
 \item A student says that if you throw a die, it will show up 1 or not 1. Therefore, the probability of getting 1 and the probability of getting 'not 1' each is equal to $\frac{1}{2}$. Is this correct? Give reasons.\\
 \solution
        %\input{exemplar/10/13/2/9/main.tex}
   \item Four candidates A, B, C, D have ap-
plied for the assignment to coach a school cricket
team. If A is twice as likely to be selected as B, and
B and C are given about the same chance of being
selected, while C is twice as likely to be selected
as D, what are the probabilities that
\begin{enumerate}
\item C will be selected?
\item A will not be selected?
\end{enumerate}
	%\input{exemplar/11/16/3/9/main.tex}
 \item A bag contain 24 balls of which $x$ balls are red, $2x$ are white and $3x$ are blue. A ball is selected at random, What is the probability that it is
\begin{enumerate}[label=\alph*)]
\item not red ?
\item white ?
\end{enumerate}
%\input{exemplar/10/13/3/41/main.tex}
If the letters of the word ASSASSINATION are arranged at random. Find the Probability that
\begin{enumerate}[label=(\alph*)]
\item Four $S's$ come consecutively in the word
\item Two  $I's$ and two $N's$ come together
\item All $A's$ are not coming together
\item No two $A's$ are coming together
\end{enumerate}
%\input{exemplar/11/16/3/14/main.tex}
	\item One urn contains two black balls (labelled B1 and B2) and one white ball. A
	second urn contains one black ball and two white balls (labelled W1 and W2).
	Suppose the following experiment is performed. One of the two urns is chosen
	at random. Next a ball is randomly chosen from the urn. Then a second ball is
	chosen at random from the same urn without replacing the first ball.
	
	\begin{enumerate}
	\item What is the probability that two black balls are chosen?
	
	\item What is the probability that two balls of opposite colour are chosen?
	\end{enumerate}
	\solution
	%\input{exemplar/11/16/3/12/main1.tex}
\end{enumerate}

		\item A box of oranges is inspected by examining three randomly selected oranges drawn without replacement. If all the three oranges are good, the box is approved for sale, otherwise, it is rejected. Find the probability that a box containing 15 oranges out of which 12 are good and 3 are bad ones will be approved for sale.
		\label{ncert/12/13/2/3/defs.tex}
		\item Two balls are drawn at random with replacement from a box containing 10 black and 8 red balls. Find the probability that
		\label{ncert/12/13/2/12}
\begin{enumerate}
\item both balls are red.
\item first ball is black and second is red.
\item one of them is black and other is red.
\end{enumerate}

\item In a hostel, 60\% of the students read Hindi newspaper, 40\% read English newspaper and 20\% read both Hindi and English newspapers. A student is selected at random.
		\label{ncert/12/13/2/15}
\begin{enumerate}
\item Find the probability that she reads neither Hindi nor English newspapers.
\item If she reads Hindi newspaper, find the probability that she reads English newspaper.
\item If she reads English newspaper, find the probability that she reads Hindi newspaper.\\
\end{enumerate}
\item The probability of obtaining an even prime number on each die, when a pair of dice is rolled is 
\begin{enumerate}
    \item $0$ 
    
    \item $\frac{1}{3}$ 
    
    \item $\frac{1}{12}$ 
    
    \item $\frac{1}{36}$ 
\end{enumerate}
\solution
		%\begin{enumerate}[label=\thesection.\arabic*,ref=\thesection.\theenumi]
	\item One card is drawn from a well-shuffled deck of 52 cards. Find the probability of getting
\begin{enumerate}
\item A king of red colour 
\item A face card 
\item A red face card
\item The jack of hearts
\item A spade
\item The queen of diamonds

\end{enumerate}
\solution
		%\input{ncert/10/15/1/14/main.tex}
	\item Five cards—the ten, jack, queen, king and ace of diamonds, are well-shuffled with their face downwards. One card is then picked up at random.
\begin{enumerate}
\item
What is the probability that the card is the queen? 
\item
If the queen is drawn and put aside, what is the probability that the second card picked up is (a) an ace? (b) a queen?\\
\end{enumerate}
\solution
		%\input{ncert/10/15/1/15/defs.tex}
	\item A bag contains $5$ red balls and some blue balls. If the probability of drawing a blue ball is double that if a red ball, determine the number of blue balls in the bag. 
		\\
\solution
		%\input{ncert/10/15/2/3/defs.tex}
	\item A card is selected from a pack of 52 cards.
 \begin{enumerate}[label=(\alph*)] 
                 \item How many points are there in the sample space?
                 \item Calculate the probability that the card is an ace of spades.
                 \item Calculate the probability that the card is (i) an ace and (ii) black card.
 \end{enumerate}
\solution
		%\input{ncert/11/16/3/4/main.tex}
\item Four cards are drawn from a well-shuffled deck of 52 cards. What is the probability of obtaining 3 diamonds and one spade.
\\
\solution
		%\input{ncert/11/16/4/2/defs.tex}
\item In a certain lottery 10,000 tickets are sold and ten equal prizes are awarded. What is the probability of not getting a prize if you buy (a) one ticket (b) two tickets (c) 10 tickets ?	
\\
\solution
		%\input{ncert/11/16/4/4/defs.tex}
		%
\item 
Out of 100 students, two sections of 40 and 60 are formed. If you and your friend are among the 100 students, what is the probability that
\begin{enumerate}
\item you both enter the same section?
\item you both enter the different sections?
\end{enumerate}
\solution
		%\input{ncert/11/16/4/5/defs.tex}
	\item 
The number lock of a suitcase has 4 wheels each labelled with ten digits i.e. from 0 to 9.The lock opens with a sequence of four digits with no repeats.What is the probability of a person getting the right sequence to open the suitcase.
\\
\solution
		%\input{ncert/11/16/4/10/defs.tex}
		%
\item 
Two cards are drawn at random and without replacement from a pack of 52 playing cards. Find the probability that both the cards are black.
\\
\solution
		%\input{ncert/12/13/2/2/defs.tex}
		\item A box of oranges is inspected by examining three randomly selected oranges drawn without replacement. If all the three oranges are good, the box is approved for sale, otherwise, it is rejected. Find the probability that a box containing 15 oranges out of which 12 are good and 3 are bad ones will be approved for sale.
		\label{ncert/12/13/2/3/defs.tex}
		\item Two balls are drawn at random with replacement from a box containing 10 black and 8 red balls. Find the probability that
		\label{ncert/12/13/2/12}
\begin{enumerate}
\item both balls are red.
\item first ball is black and second is red.
\item one of them is black and other is red.
\end{enumerate}

\item In a hostel, 60\% of the students read Hindi newspaper, 40\% read English newspaper and 20\% read both Hindi and English newspapers. A student is selected at random.
		\label{ncert/12/13/2/15}
\begin{enumerate}
\item Find the probability that she reads neither Hindi nor English newspapers.
\item If she reads Hindi newspaper, find the probability that she reads English newspaper.
\item If she reads English newspaper, find the probability that she reads Hindi newspaper.\\
\end{enumerate}
\item The probability of obtaining an even prime number on each die, when a pair of dice is rolled is 
\begin{enumerate}
    \item $0$ 
    
    \item $\frac{1}{3}$ 
    
    \item $\frac{1}{12}$ 
    
    \item $\frac{1}{36}$ 
\end{enumerate}
\solution
		%\input{ncert/12/13/2/17/defs.tex}
	\item A bag contains 4 red and 4 black balls, another bag contains 2 red and 6 black balls. One of the two bags is selected at random and a ball is drawn from the bag which is found to be red. Find the probability that the ball is drawn from the first bag.
\\
\solution
		%\input{ncert/12/13/3/2/main.tex}
  \item
  Cards with numbers 2 to 101 are placed in a box. A card is selected at random.Find the probability that the card has
\begin{enumerate}[label=(\roman*)]
	\item an even number 
	\item a square number
\end{enumerate}
\solution
%\input{exemplar/10/13/3/32/main.tex}
\item
The king, queen and jack of clubs are removed from a deck of 52 playing cards and then well shuffled. Now one card is drawn at random from the remaining cards.  Determine the probability that the card is
\begin{enumerate}[label=(\roman*)]
\item a club
\item 10 of hearts
\end{enumerate}
\solution
%\input{exemplar/10/13/3/29/main.tex}
\item A team of medical students doing their internship have to assist during surgeries
at a city hospital. The probabilities of surgeries rated as very complex, complex,
routine, simple or very simple are respectively, 0.15, 0.20, 0.31, 0.26, .08. Find
the probabilities that a particular surgery will be rated
\begin{enumerate}
	\item complex or very complex;
	\item neither very complex nor very simple;
	\item routine or complex
	\item routine or simple
\end{enumerate}
\solution
%\input{exemplar/11/16/3/8(1)/main.tex}
\item A card is selected from a pack of 52 cards.
\begin{enumerate}[label=(\alph*)]
    \item How many points are there in the sample space?
    \item Calculate the probability that the card is an ace of spades.
    \item Calculate the probability that the card is (i) an ace and (ii) black card.
\end{enumerate}
\solution
%\input{exemplar/11/16/3/4/main2.tex}
\item The probability that a non leap year selected at random will contain 53 sundays.
\\
\solution
%\input{exemplar/10/13/1/19/main.tex}
\item One of the four persons John, Rita, Aslam or Gurpreet will be promoted next
month. Consequently the sample space consists of four elementary outcomes
S = {John promoted, Rita promoted, Aslam promoted, Gurpreet promoted}
You are told that the chances of John’s promotion is same as that of Gurpreet,
Rita’s chances of promotion are twice as likely as Johns. Aslam’s chances are
four times that of John.
\begin{enumerate}
	\item Determine
	\begin{enumerate}
		\item P (John promoted)
		\item P (Rita promoted)
		\item P (Aslam promoted)
		\item P (Gurpreet promoted)
	\end{enumerate}
	\item If A = {John promoted or Gurpreet promoted}, find P (A).
\end{enumerate}
\solution
%\input{exemplar/11/16/3/10/main.tex}
\item A card is drawn from a deck of 52 cards. Find the probability of getting a king or a heart or a red card.\\
\solution
%\input{exemplar/11/16/3/15/main.tex}
\item The probability that a student will pass his examination is 0.73, the probability of
the student getting a compartment is 0.13, and the probability that the student will
either pass or get compartment is 0.96. State True or False.\\
\solution
%\input{exemplar/11/16/3/31/main.tex}
\item A card is selected from a pack of 52 cards\\
\begin{enumerate}[label=(\alph*)]
\item How many points are there in the sample space?
\item Calculate the probability that the cards is an ace of spades.
\item Calculate the probability that the card is (i) an ace (ii)black card.\\
\end{enumerate}
%\input{ncert/11/16/3/4_1/Prob_4.tex}
\item In a non-leap year, the probability of having 53 tuesdays or 53 wednesdays is\\
\solution
%\input{exemplar/11/16/3/18/main.tex}
\item There are 1000 sealed envelopes in a box, 10 of them contain a cash prize of
Rs 100 each, 100 of them contain a cash prize of Rs 50 each and 200 of them
contain a cash prize of Rs 10 each and rest do not contain any cash prize. If they
are well shuffled and an envelope is picked up out, what is the probability that it
contains no cash prize?\\
\solution
%\input{exemplar/10/13/3/34/main.tex}
\item 
A die is thrown and a card is selected at random from a deck of 52 playing cards. The probability of getting an even number on the die and a spade card.\\
\solution
%\input{exemplar/12/13/3/78/main.tex}
\item
If 4-digit numbers greater than 5,000 are randomly formed from the digits 0, 1, 3, 5, and 7, what is the probability of forming a number divisible by 5 when:
\begin{enumerate}
    \item The digits are repeated?
    \item The repetition of digits is not allowed?
\end{enumerate}
\solution
%\input{ncert/11/16/4/9/main.tex}
\item Consider the probability space $\brak{\Omega, \mathcal{G}, P}$ where $\Omega = [0,2]$ and $\mathcal{G} = \cbrak{\phi, \Omega, [0,1], (1,2]}$. Let $X$ and $Y$ be two functions on $\Omega$ defined as
\begin{align*}
    X(\omega) = 
    \begin{cases}
        1 & \text{if }\omega \in [0, 1]\\
        2 & \text{if }\omega \in (1, 2]
    \end{cases}
\end{align*}
and
\begin{align*}
    Y(\omega) = 
    \begin{cases}
        2 & \text{if }\omega \in [0, 1.5]\\
        3 & \text{if }\omega \in (1.5, 2].
    \end{cases}
\end{align*}
Then which one of the following statements is true?
\begin{enumerate}
    \item [(A)] $X$ is a random variable with respect to $\mathcal{G}$, but $Y$ is not a random variable with respect to $\mathcal{G}$.
    \item [(B)] $Y$ is a random variable with respect to $\mathcal{G}$, but $X$ is not a random variable with respect to $\mathcal{G}$.
    \item [(C)] Neither $X$ nor $Y$ is a random variable with respect to $\mathcal{G}$.
    \item [(D)] Both $X$ and $Y$ are random variables with respect to $\mathcal{G}$.
\end{enumerate} \hfill (GATE ST 2023)\\
\solution
%\input{gate/ST/2023/14/main.tex}
	\item  A die is loaded in such a way that each odd number is twice as likely to occur as
each even number. Find $P(G)$, where $G$ is the event that a number greater than
3 occurs on a single roll of the die.
\\
\solution
		%\input{exemplar/11/16/3/5/main.tex}
	\item All the jacks, queens and kings are removed from a deck of 52 playing cards. The remaining cards are well shuffled and then one card is drawn at random. Giving ace a value 1 similar value for other cards, find the probability that the card has a value 
		\begin{enumerate}
			\item 7
			\item greater than 7
			\item less than 7
		\end{enumerate}
		%\input{exemplar/10/13/3/30/main.tex}
  \item A Lot consists of 48 mobile phones of which 42 are good, 3 have only minor defects and 3 have major defects.Varnika will buy a phone if it is good but the trader will only buy a mobile if it has no major defects. One phone is selected at random from the lot. What is the probability that it is
\begin{enumerate}
	\item acceptable to Varnika?
            \item acceptable to the trader?
\end{enumerate}
\solution
	%\input{exemplar/10/13/3/40/main.tex}
 \item A student says that if you throw a die, it will show up 1 or not 1. Therefore, the probability of getting 1 and the probability of getting 'not 1' each is equal to $\frac{1}{2}$. Is this correct? Give reasons.\\
 \solution
        %\input{exemplar/10/13/2/9/main.tex}
   \item Four candidates A, B, C, D have ap-
plied for the assignment to coach a school cricket
team. If A is twice as likely to be selected as B, and
B and C are given about the same chance of being
selected, while C is twice as likely to be selected
as D, what are the probabilities that
\begin{enumerate}
\item C will be selected?
\item A will not be selected?
\end{enumerate}
	%\input{exemplar/11/16/3/9/main.tex}
 \item A bag contain 24 balls of which $x$ balls are red, $2x$ are white and $3x$ are blue. A ball is selected at random, What is the probability that it is
\begin{enumerate}[label=\alph*)]
\item not red ?
\item white ?
\end{enumerate}
%\input{exemplar/10/13/3/41/main.tex}
If the letters of the word ASSASSINATION are arranged at random. Find the Probability that
\begin{enumerate}[label=(\alph*)]
\item Four $S's$ come consecutively in the word
\item Two  $I's$ and two $N's$ come together
\item All $A's$ are not coming together
\item No two $A's$ are coming together
\end{enumerate}
%\input{exemplar/11/16/3/14/main.tex}
	\item One urn contains two black balls (labelled B1 and B2) and one white ball. A
	second urn contains one black ball and two white balls (labelled W1 and W2).
	Suppose the following experiment is performed. One of the two urns is chosen
	at random. Next a ball is randomly chosen from the urn. Then a second ball is
	chosen at random from the same urn without replacing the first ball.
	
	\begin{enumerate}
	\item What is the probability that two black balls are chosen?
	
	\item What is the probability that two balls of opposite colour are chosen?
	\end{enumerate}
	\solution
	%\input{exemplar/11/16/3/12/main1.tex}
\end{enumerate}

	\item A bag contains 4 red and 4 black balls, another bag contains 2 red and 6 black balls. One of the two bags is selected at random and a ball is drawn from the bag which is found to be red. Find the probability that the ball is drawn from the first bag.
\\
\solution
		%\begin{table}[H]
	\centering
\begin{tabular}{|c|c|c|}
\hline
Random variable &Value &Definition\\ \hline
\multirow{3}{*}{X} &0 &Slips of Rs 1\\
&1 &Slips of Rs 5\\
&2 &Slips of Rs 13\\ \hline
\multirow{2}{*}{Y} &0 &Box A\\
&1 &Box B\\\hline
\end{tabular}
\caption{}
\label{tab:Distribution}
\end{table}
See \tabref{tab:Distribution}.
\begin{align}
p_{Y}\brak{k}= \begin{cases} 
      \frac{1}{3} & {k=0} \\
      \frac{2}{3 }& {k=1} 
   \end{cases}
   \\
p_{Y|X}\brak{0|0} = \frac{19}{25}\, 
p_{Y|X}\brak{0|1} = \frac{6}{25}\,
p_{Y|X}\brak{1|0} = \frac{45}{50}\,
p_{Y|X}\brak{1|2} = \frac{5}{50}
\end{align}
The desired probability is the probability that a slip drawn at random is marked other than Rs 1,
\begin{align}
&=1-p_X\brak{0}\\
&= p_X(1) + p_X(2)
\end{align}
Using Bayes theorem,
\begin{align}
&= p_Y\brak{0} \times \pr{Y=0 | X=1} + p_Y\brak{1} \times \pr{Y=1|X=2}\\
&=\frac{1}{3} \times \frac{6}{25} + \frac{2}{3} \times \frac{5}{50}\\
&=\frac{11}{75}
\end{align}

\newpage

%\tableofcontents

\bigskip

\renewcommand{\thefigure}{\theenumi}
\renewcommand{\thetable}{\theenumi}
%\renewcommand{\theequation}{\theenumi}

%\begin{abstract}
%%\boldmath
%In this letter, an algorithm for evaluating the exact analytical bit error rate  (BER)  for the piecewise linear (PL) combiner for  multiple relays is presented. Previous results were available only for upto three relays. The algorithm is unique in the sense that  the actual mathematical expressions, that are prohibitively large, need not be explicitly obtained. The diversity gain due to multiple relays is shown through plots of the analytical BER, well supported by simulations. 
%
%\end{abstract}
% IEEEtran.cls defaults to using nonbold math in the Abstract.
% This preserves the distinction between vectors and scalars. However,
% if the journal you are submitting to favors bold math in the abstract,
% then you can use LaTeX's standard command \boldmath at the very start
% of the abstract to achieve this. Many IEEE journals frown on math
% in the abstract anyway.

% Note that keywords are not normally used for peerreview papers.
%\begin{IEEEkeywords}
%Cooperative diversity, decode and forward, piecewise linear
%\end{IEEEkeywords}



% For peer review papers, you can put extra information on the cover
% page as needed:
% \ifCLASSOPTIONpeerreview
% \begin{center} \bfseries EDICS Category: 3-BBND \end{center}
% \fi
%
% For peerreview papers, this IEEEtran command inserts a page break and
% creates the second title. It will be ignored for other modes.
%\IEEEpeerreviewmaketitle




  \item
  Cards with numbers 2 to 101 are placed in a box. A card is selected at random.Find the probability that the card has
\begin{enumerate}[label=(\roman*)]
	\item an even number 
	\item a square number
\end{enumerate}
\solution
%\begin{table}[H]
	\centering
\begin{tabular}{|c|c|c|}
\hline
Random variable &Value &Definition\\ \hline
\multirow{3}{*}{X} &0 &Slips of Rs 1\\
&1 &Slips of Rs 5\\
&2 &Slips of Rs 13\\ \hline
\multirow{2}{*}{Y} &0 &Box A\\
&1 &Box B\\\hline
\end{tabular}
\caption{}
\label{tab:Distribution}
\end{table}
See \tabref{tab:Distribution}.
\begin{align}
p_{Y}\brak{k}= \begin{cases} 
      \frac{1}{3} & {k=0} \\
      \frac{2}{3 }& {k=1} 
   \end{cases}
   \\
p_{Y|X}\brak{0|0} = \frac{19}{25}\, 
p_{Y|X}\brak{0|1} = \frac{6}{25}\,
p_{Y|X}\brak{1|0} = \frac{45}{50}\,
p_{Y|X}\brak{1|2} = \frac{5}{50}
\end{align}
The desired probability is the probability that a slip drawn at random is marked other than Rs 1,
\begin{align}
&=1-p_X\brak{0}\\
&= p_X(1) + p_X(2)
\end{align}
Using Bayes theorem,
\begin{align}
&= p_Y\brak{0} \times \pr{Y=0 | X=1} + p_Y\brak{1} \times \pr{Y=1|X=2}\\
&=\frac{1}{3} \times \frac{6}{25} + \frac{2}{3} \times \frac{5}{50}\\
&=\frac{11}{75}
\end{align}

\newpage

%\tableofcontents

\bigskip

\renewcommand{\thefigure}{\theenumi}
\renewcommand{\thetable}{\theenumi}
%\renewcommand{\theequation}{\theenumi}

%\begin{abstract}
%%\boldmath
%In this letter, an algorithm for evaluating the exact analytical bit error rate  (BER)  for the piecewise linear (PL) combiner for  multiple relays is presented. Previous results were available only for upto three relays. The algorithm is unique in the sense that  the actual mathematical expressions, that are prohibitively large, need not be explicitly obtained. The diversity gain due to multiple relays is shown through plots of the analytical BER, well supported by simulations. 
%
%\end{abstract}
% IEEEtran.cls defaults to using nonbold math in the Abstract.
% This preserves the distinction between vectors and scalars. However,
% if the journal you are submitting to favors bold math in the abstract,
% then you can use LaTeX's standard command \boldmath at the very start
% of the abstract to achieve this. Many IEEE journals frown on math
% in the abstract anyway.

% Note that keywords are not normally used for peerreview papers.
%\begin{IEEEkeywords}
%Cooperative diversity, decode and forward, piecewise linear
%\end{IEEEkeywords}



% For peer review papers, you can put extra information on the cover
% page as needed:
% \ifCLASSOPTIONpeerreview
% \begin{center} \bfseries EDICS Category: 3-BBND \end{center}
% \fi
%
% For peerreview papers, this IEEEtran command inserts a page break and
% creates the second title. It will be ignored for other modes.
%\IEEEpeerreviewmaketitle




\item
The king, queen and jack of clubs are removed from a deck of 52 playing cards and then well shuffled. Now one card is drawn at random from the remaining cards.  Determine the probability that the card is
\begin{enumerate}[label=(\roman*)]
\item a club
\item 10 of hearts
\end{enumerate}
\solution
%\begin{table}[H]
	\centering
\begin{tabular}{|c|c|c|}
\hline
Random variable &Value &Definition\\ \hline
\multirow{3}{*}{X} &0 &Slips of Rs 1\\
&1 &Slips of Rs 5\\
&2 &Slips of Rs 13\\ \hline
\multirow{2}{*}{Y} &0 &Box A\\
&1 &Box B\\\hline
\end{tabular}
\caption{}
\label{tab:Distribution}
\end{table}
See \tabref{tab:Distribution}.
\begin{align}
p_{Y}\brak{k}= \begin{cases} 
      \frac{1}{3} & {k=0} \\
      \frac{2}{3 }& {k=1} 
   \end{cases}
   \\
p_{Y|X}\brak{0|0} = \frac{19}{25}\, 
p_{Y|X}\brak{0|1} = \frac{6}{25}\,
p_{Y|X}\brak{1|0} = \frac{45}{50}\,
p_{Y|X}\brak{1|2} = \frac{5}{50}
\end{align}
The desired probability is the probability that a slip drawn at random is marked other than Rs 1,
\begin{align}
&=1-p_X\brak{0}\\
&= p_X(1) + p_X(2)
\end{align}
Using Bayes theorem,
\begin{align}
&= p_Y\brak{0} \times \pr{Y=0 | X=1} + p_Y\brak{1} \times \pr{Y=1|X=2}\\
&=\frac{1}{3} \times \frac{6}{25} + \frac{2}{3} \times \frac{5}{50}\\
&=\frac{11}{75}
\end{align}

\newpage

%\tableofcontents

\bigskip

\renewcommand{\thefigure}{\theenumi}
\renewcommand{\thetable}{\theenumi}
%\renewcommand{\theequation}{\theenumi}

%\begin{abstract}
%%\boldmath
%In this letter, an algorithm for evaluating the exact analytical bit error rate  (BER)  for the piecewise linear (PL) combiner for  multiple relays is presented. Previous results were available only for upto three relays. The algorithm is unique in the sense that  the actual mathematical expressions, that are prohibitively large, need not be explicitly obtained. The diversity gain due to multiple relays is shown through plots of the analytical BER, well supported by simulations. 
%
%\end{abstract}
% IEEEtran.cls defaults to using nonbold math in the Abstract.
% This preserves the distinction between vectors and scalars. However,
% if the journal you are submitting to favors bold math in the abstract,
% then you can use LaTeX's standard command \boldmath at the very start
% of the abstract to achieve this. Many IEEE journals frown on math
% in the abstract anyway.

% Note that keywords are not normally used for peerreview papers.
%\begin{IEEEkeywords}
%Cooperative diversity, decode and forward, piecewise linear
%\end{IEEEkeywords}



% For peer review papers, you can put extra information on the cover
% page as needed:
% \ifCLASSOPTIONpeerreview
% \begin{center} \bfseries EDICS Category: 3-BBND \end{center}
% \fi
%
% For peerreview papers, this IEEEtran command inserts a page break and
% creates the second title. It will be ignored for other modes.
%\IEEEpeerreviewmaketitle




\item A team of medical students doing their internship have to assist during surgeries
at a city hospital. The probabilities of surgeries rated as very complex, complex,
routine, simple or very simple are respectively, 0.15, 0.20, 0.31, 0.26, .08. Find
the probabilities that a particular surgery will be rated
\begin{enumerate}
	\item complex or very complex;
	\item neither very complex nor very simple;
	\item routine or complex
	\item routine or simple
\end{enumerate}
\solution
%\begin{table}[H]
	\centering
\begin{tabular}{|c|c|c|}
\hline
Random variable &Value &Definition\\ \hline
\multirow{3}{*}{X} &0 &Slips of Rs 1\\
&1 &Slips of Rs 5\\
&2 &Slips of Rs 13\\ \hline
\multirow{2}{*}{Y} &0 &Box A\\
&1 &Box B\\\hline
\end{tabular}
\caption{}
\label{tab:Distribution}
\end{table}
See \tabref{tab:Distribution}.
\begin{align}
p_{Y}\brak{k}= \begin{cases} 
      \frac{1}{3} & {k=0} \\
      \frac{2}{3 }& {k=1} 
   \end{cases}
   \\
p_{Y|X}\brak{0|0} = \frac{19}{25}\, 
p_{Y|X}\brak{0|1} = \frac{6}{25}\,
p_{Y|X}\brak{1|0} = \frac{45}{50}\,
p_{Y|X}\brak{1|2} = \frac{5}{50}
\end{align}
The desired probability is the probability that a slip drawn at random is marked other than Rs 1,
\begin{align}
&=1-p_X\brak{0}\\
&= p_X(1) + p_X(2)
\end{align}
Using Bayes theorem,
\begin{align}
&= p_Y\brak{0} \times \pr{Y=0 | X=1} + p_Y\brak{1} \times \pr{Y=1|X=2}\\
&=\frac{1}{3} \times \frac{6}{25} + \frac{2}{3} \times \frac{5}{50}\\
&=\frac{11}{75}
\end{align}

\newpage

%\tableofcontents

\bigskip

\renewcommand{\thefigure}{\theenumi}
\renewcommand{\thetable}{\theenumi}
%\renewcommand{\theequation}{\theenumi}

%\begin{abstract}
%%\boldmath
%In this letter, an algorithm for evaluating the exact analytical bit error rate  (BER)  for the piecewise linear (PL) combiner for  multiple relays is presented. Previous results were available only for upto three relays. The algorithm is unique in the sense that  the actual mathematical expressions, that are prohibitively large, need not be explicitly obtained. The diversity gain due to multiple relays is shown through plots of the analytical BER, well supported by simulations. 
%
%\end{abstract}
% IEEEtran.cls defaults to using nonbold math in the Abstract.
% This preserves the distinction between vectors and scalars. However,
% if the journal you are submitting to favors bold math in the abstract,
% then you can use LaTeX's standard command \boldmath at the very start
% of the abstract to achieve this. Many IEEE journals frown on math
% in the abstract anyway.

% Note that keywords are not normally used for peerreview papers.
%\begin{IEEEkeywords}
%Cooperative diversity, decode and forward, piecewise linear
%\end{IEEEkeywords}



% For peer review papers, you can put extra information on the cover
% page as needed:
% \ifCLASSOPTIONpeerreview
% \begin{center} \bfseries EDICS Category: 3-BBND \end{center}
% \fi
%
% For peerreview papers, this IEEEtran command inserts a page break and
% creates the second title. It will be ignored for other modes.
%\IEEEpeerreviewmaketitle




\item A card is selected from a pack of 52 cards.
\begin{enumerate}[label=(\alph*)]
    \item How many points are there in the sample space?
    \item Calculate the probability that the card is an ace of spades.
    \item Calculate the probability that the card is (i) an ace and (ii) black card.
\end{enumerate}
\solution
%Let $X$ be an bernoulli rv defined as in \tabref{tab:exemplar/11/16/3/26}.  Then, 
\begin{equation}
    p =
        \frac{4}{11} 
\end{equation}
\begin{table}[H]
	\centering
	\input{exemplar/11/16/3/26/tables/Table2.tex}
	\caption{}
        \label{tab:exemplar/11/16/3/26}
\end{table}

\item The probability that a non leap year selected at random will contain 53 sundays.
\\
\solution
%\begin{table}[H]
	\centering
\begin{tabular}{|c|c|c|}
\hline
Random variable &Value &Definition\\ \hline
\multirow{3}{*}{X} &0 &Slips of Rs 1\\
&1 &Slips of Rs 5\\
&2 &Slips of Rs 13\\ \hline
\multirow{2}{*}{Y} &0 &Box A\\
&1 &Box B\\\hline
\end{tabular}
\caption{}
\label{tab:Distribution}
\end{table}
See \tabref{tab:Distribution}.
\begin{align}
p_{Y}\brak{k}= \begin{cases} 
      \frac{1}{3} & {k=0} \\
      \frac{2}{3 }& {k=1} 
   \end{cases}
   \\
p_{Y|X}\brak{0|0} = \frac{19}{25}\, 
p_{Y|X}\brak{0|1} = \frac{6}{25}\,
p_{Y|X}\brak{1|0} = \frac{45}{50}\,
p_{Y|X}\brak{1|2} = \frac{5}{50}
\end{align}
The desired probability is the probability that a slip drawn at random is marked other than Rs 1,
\begin{align}
&=1-p_X\brak{0}\\
&= p_X(1) + p_X(2)
\end{align}
Using Bayes theorem,
\begin{align}
&= p_Y\brak{0} \times \pr{Y=0 | X=1} + p_Y\brak{1} \times \pr{Y=1|X=2}\\
&=\frac{1}{3} \times \frac{6}{25} + \frac{2}{3} \times \frac{5}{50}\\
&=\frac{11}{75}
\end{align}

\newpage

%\tableofcontents

\bigskip

\renewcommand{\thefigure}{\theenumi}
\renewcommand{\thetable}{\theenumi}
%\renewcommand{\theequation}{\theenumi}

%\begin{abstract}
%%\boldmath
%In this letter, an algorithm for evaluating the exact analytical bit error rate  (BER)  for the piecewise linear (PL) combiner for  multiple relays is presented. Previous results were available only for upto three relays. The algorithm is unique in the sense that  the actual mathematical expressions, that are prohibitively large, need not be explicitly obtained. The diversity gain due to multiple relays is shown through plots of the analytical BER, well supported by simulations. 
%
%\end{abstract}
% IEEEtran.cls defaults to using nonbold math in the Abstract.
% This preserves the distinction between vectors and scalars. However,
% if the journal you are submitting to favors bold math in the abstract,
% then you can use LaTeX's standard command \boldmath at the very start
% of the abstract to achieve this. Many IEEE journals frown on math
% in the abstract anyway.

% Note that keywords are not normally used for peerreview papers.
%\begin{IEEEkeywords}
%Cooperative diversity, decode and forward, piecewise linear
%\end{IEEEkeywords}



% For peer review papers, you can put extra information on the cover
% page as needed:
% \ifCLASSOPTIONpeerreview
% \begin{center} \bfseries EDICS Category: 3-BBND \end{center}
% \fi
%
% For peerreview papers, this IEEEtran command inserts a page break and
% creates the second title. It will be ignored for other modes.
%\IEEEpeerreviewmaketitle




\item One of the four persons John, Rita, Aslam or Gurpreet will be promoted next
month. Consequently the sample space consists of four elementary outcomes
S = {John promoted, Rita promoted, Aslam promoted, Gurpreet promoted}
You are told that the chances of John’s promotion is same as that of Gurpreet,
Rita’s chances of promotion are twice as likely as Johns. Aslam’s chances are
four times that of John.
\begin{enumerate}
	\item Determine
	\begin{enumerate}
		\item P (John promoted)
		\item P (Rita promoted)
		\item P (Aslam promoted)
		\item P (Gurpreet promoted)
	\end{enumerate}
	\item If A = {John promoted or Gurpreet promoted}, find P (A).
\end{enumerate}
\solution
%\begin{table}[H]
	\centering
\begin{tabular}{|c|c|c|}
\hline
Random variable &Value &Definition\\ \hline
\multirow{3}{*}{X} &0 &Slips of Rs 1\\
&1 &Slips of Rs 5\\
&2 &Slips of Rs 13\\ \hline
\multirow{2}{*}{Y} &0 &Box A\\
&1 &Box B\\\hline
\end{tabular}
\caption{}
\label{tab:Distribution}
\end{table}
See \tabref{tab:Distribution}.
\begin{align}
p_{Y}\brak{k}= \begin{cases} 
      \frac{1}{3} & {k=0} \\
      \frac{2}{3 }& {k=1} 
   \end{cases}
   \\
p_{Y|X}\brak{0|0} = \frac{19}{25}\, 
p_{Y|X}\brak{0|1} = \frac{6}{25}\,
p_{Y|X}\brak{1|0} = \frac{45}{50}\,
p_{Y|X}\brak{1|2} = \frac{5}{50}
\end{align}
The desired probability is the probability that a slip drawn at random is marked other than Rs 1,
\begin{align}
&=1-p_X\brak{0}\\
&= p_X(1) + p_X(2)
\end{align}
Using Bayes theorem,
\begin{align}
&= p_Y\brak{0} \times \pr{Y=0 | X=1} + p_Y\brak{1} \times \pr{Y=1|X=2}\\
&=\frac{1}{3} \times \frac{6}{25} + \frac{2}{3} \times \frac{5}{50}\\
&=\frac{11}{75}
\end{align}

\newpage

%\tableofcontents

\bigskip

\renewcommand{\thefigure}{\theenumi}
\renewcommand{\thetable}{\theenumi}
%\renewcommand{\theequation}{\theenumi}

%\begin{abstract}
%%\boldmath
%In this letter, an algorithm for evaluating the exact analytical bit error rate  (BER)  for the piecewise linear (PL) combiner for  multiple relays is presented. Previous results were available only for upto three relays. The algorithm is unique in the sense that  the actual mathematical expressions, that are prohibitively large, need not be explicitly obtained. The diversity gain due to multiple relays is shown through plots of the analytical BER, well supported by simulations. 
%
%\end{abstract}
% IEEEtran.cls defaults to using nonbold math in the Abstract.
% This preserves the distinction between vectors and scalars. However,
% if the journal you are submitting to favors bold math in the abstract,
% then you can use LaTeX's standard command \boldmath at the very start
% of the abstract to achieve this. Many IEEE journals frown on math
% in the abstract anyway.

% Note that keywords are not normally used for peerreview papers.
%\begin{IEEEkeywords}
%Cooperative diversity, decode and forward, piecewise linear
%\end{IEEEkeywords}



% For peer review papers, you can put extra information on the cover
% page as needed:
% \ifCLASSOPTIONpeerreview
% \begin{center} \bfseries EDICS Category: 3-BBND \end{center}
% \fi
%
% For peerreview papers, this IEEEtran command inserts a page break and
% creates the second title. It will be ignored for other modes.
%\IEEEpeerreviewmaketitle




\item A card is drawn from a deck of 52 cards. Find the probability of getting a king or a heart or a red card.\\
\solution
%\begin{table}[H]
	\centering
\begin{tabular}{|c|c|c|}
\hline
Random variable &Value &Definition\\ \hline
\multirow{3}{*}{X} &0 &Slips of Rs 1\\
&1 &Slips of Rs 5\\
&2 &Slips of Rs 13\\ \hline
\multirow{2}{*}{Y} &0 &Box A\\
&1 &Box B\\\hline
\end{tabular}
\caption{}
\label{tab:Distribution}
\end{table}
See \tabref{tab:Distribution}.
\begin{align}
p_{Y}\brak{k}= \begin{cases} 
      \frac{1}{3} & {k=0} \\
      \frac{2}{3 }& {k=1} 
   \end{cases}
   \\
p_{Y|X}\brak{0|0} = \frac{19}{25}\, 
p_{Y|X}\brak{0|1} = \frac{6}{25}\,
p_{Y|X}\brak{1|0} = \frac{45}{50}\,
p_{Y|X}\brak{1|2} = \frac{5}{50}
\end{align}
The desired probability is the probability that a slip drawn at random is marked other than Rs 1,
\begin{align}
&=1-p_X\brak{0}\\
&= p_X(1) + p_X(2)
\end{align}
Using Bayes theorem,
\begin{align}
&= p_Y\brak{0} \times \pr{Y=0 | X=1} + p_Y\brak{1} \times \pr{Y=1|X=2}\\
&=\frac{1}{3} \times \frac{6}{25} + \frac{2}{3} \times \frac{5}{50}\\
&=\frac{11}{75}
\end{align}

\newpage

%\tableofcontents

\bigskip

\renewcommand{\thefigure}{\theenumi}
\renewcommand{\thetable}{\theenumi}
%\renewcommand{\theequation}{\theenumi}

%\begin{abstract}
%%\boldmath
%In this letter, an algorithm for evaluating the exact analytical bit error rate  (BER)  for the piecewise linear (PL) combiner for  multiple relays is presented. Previous results were available only for upto three relays. The algorithm is unique in the sense that  the actual mathematical expressions, that are prohibitively large, need not be explicitly obtained. The diversity gain due to multiple relays is shown through plots of the analytical BER, well supported by simulations. 
%
%\end{abstract}
% IEEEtran.cls defaults to using nonbold math in the Abstract.
% This preserves the distinction between vectors and scalars. However,
% if the journal you are submitting to favors bold math in the abstract,
% then you can use LaTeX's standard command \boldmath at the very start
% of the abstract to achieve this. Many IEEE journals frown on math
% in the abstract anyway.

% Note that keywords are not normally used for peerreview papers.
%\begin{IEEEkeywords}
%Cooperative diversity, decode and forward, piecewise linear
%\end{IEEEkeywords}



% For peer review papers, you can put extra information on the cover
% page as needed:
% \ifCLASSOPTIONpeerreview
% \begin{center} \bfseries EDICS Category: 3-BBND \end{center}
% \fi
%
% For peerreview papers, this IEEEtran command inserts a page break and
% creates the second title. It will be ignored for other modes.
%\IEEEpeerreviewmaketitle




\item The probability that a student will pass his examination is 0.73, the probability of
the student getting a compartment is 0.13, and the probability that the student will
either pass or get compartment is 0.96. State True or False.\\
\solution
%\begin{table}[H]
	\centering
\begin{tabular}{|c|c|c|}
\hline
Random variable &Value &Definition\\ \hline
\multirow{3}{*}{X} &0 &Slips of Rs 1\\
&1 &Slips of Rs 5\\
&2 &Slips of Rs 13\\ \hline
\multirow{2}{*}{Y} &0 &Box A\\
&1 &Box B\\\hline
\end{tabular}
\caption{}
\label{tab:Distribution}
\end{table}
See \tabref{tab:Distribution}.
\begin{align}
p_{Y}\brak{k}= \begin{cases} 
      \frac{1}{3} & {k=0} \\
      \frac{2}{3 }& {k=1} 
   \end{cases}
   \\
p_{Y|X}\brak{0|0} = \frac{19}{25}\, 
p_{Y|X}\brak{0|1} = \frac{6}{25}\,
p_{Y|X}\brak{1|0} = \frac{45}{50}\,
p_{Y|X}\brak{1|2} = \frac{5}{50}
\end{align}
The desired probability is the probability that a slip drawn at random is marked other than Rs 1,
\begin{align}
&=1-p_X\brak{0}\\
&= p_X(1) + p_X(2)
\end{align}
Using Bayes theorem,
\begin{align}
&= p_Y\brak{0} \times \pr{Y=0 | X=1} + p_Y\brak{1} \times \pr{Y=1|X=2}\\
&=\frac{1}{3} \times \frac{6}{25} + \frac{2}{3} \times \frac{5}{50}\\
&=\frac{11}{75}
\end{align}

\newpage

%\tableofcontents

\bigskip

\renewcommand{\thefigure}{\theenumi}
\renewcommand{\thetable}{\theenumi}
%\renewcommand{\theequation}{\theenumi}

%\begin{abstract}
%%\boldmath
%In this letter, an algorithm for evaluating the exact analytical bit error rate  (BER)  for the piecewise linear (PL) combiner for  multiple relays is presented. Previous results were available only for upto three relays. The algorithm is unique in the sense that  the actual mathematical expressions, that are prohibitively large, need not be explicitly obtained. The diversity gain due to multiple relays is shown through plots of the analytical BER, well supported by simulations. 
%
%\end{abstract}
% IEEEtran.cls defaults to using nonbold math in the Abstract.
% This preserves the distinction between vectors and scalars. However,
% if the journal you are submitting to favors bold math in the abstract,
% then you can use LaTeX's standard command \boldmath at the very start
% of the abstract to achieve this. Many IEEE journals frown on math
% in the abstract anyway.

% Note that keywords are not normally used for peerreview papers.
%\begin{IEEEkeywords}
%Cooperative diversity, decode and forward, piecewise linear
%\end{IEEEkeywords}



% For peer review papers, you can put extra information on the cover
% page as needed:
% \ifCLASSOPTIONpeerreview
% \begin{center} \bfseries EDICS Category: 3-BBND \end{center}
% \fi
%
% For peerreview papers, this IEEEtran command inserts a page break and
% creates the second title. It will be ignored for other modes.
%\IEEEpeerreviewmaketitle




\item A card is selected from a pack of 52 cards\\
\begin{enumerate}[label=(\alph*)]
\item How many points are there in the sample space?
\item Calculate the probability that the cards is an ace of spades.
\item Calculate the probability that the card is (i) an ace (ii)black card.\\
\end{enumerate}
%\input{ncert/11/16/3/4_1/Prob_4.tex}
\item In a non-leap year, the probability of having 53 tuesdays or 53 wednesdays is\\
\solution
%A non-leap year has a total of 365 days, and a week has 7 days.\\
So it can be expressed as 
\begin{align}
365\text{days} &=52\times 7+1 \text{day}
\end{align}
$\implies$ 52 tuesdays or wednesdays\\
Random variable X denotes the days of a week
\begin{align}
p_X\brak{k}&=\frac{1}{7}; \quad \brak{1<k<7}
\end{align}
So the probability of extra day being tuesday or wednesday is
\begin{align}
p_X\brak{3}+p_X\brak{4}&=\frac{1}{7}+\frac{1}{7}=\frac{2}{7}
\end{align}



\item There are 1000 sealed envelopes in a box, 10 of them contain a cash prize of
Rs 100 each, 100 of them contain a cash prize of Rs 50 each and 200 of them
contain a cash prize of Rs 10 each and rest do not contain any cash prize. If they
are well shuffled and an envelope is picked up out, what is the probability that it
contains no cash prize?\\
\solution
%\begin{table}[H]
	\centering
\begin{tabular}{|c|c|c|}
\hline
Random variable &Value &Definition\\ \hline
\multirow{3}{*}{X} &0 &Slips of Rs 1\\
&1 &Slips of Rs 5\\
&2 &Slips of Rs 13\\ \hline
\multirow{2}{*}{Y} &0 &Box A\\
&1 &Box B\\\hline
\end{tabular}
\caption{}
\label{tab:Distribution}
\end{table}
See \tabref{tab:Distribution}.
\begin{align}
p_{Y}\brak{k}= \begin{cases} 
      \frac{1}{3} & {k=0} \\
      \frac{2}{3 }& {k=1} 
   \end{cases}
   \\
p_{Y|X}\brak{0|0} = \frac{19}{25}\, 
p_{Y|X}\brak{0|1} = \frac{6}{25}\,
p_{Y|X}\brak{1|0} = \frac{45}{50}\,
p_{Y|X}\brak{1|2} = \frac{5}{50}
\end{align}
The desired probability is the probability that a slip drawn at random is marked other than Rs 1,
\begin{align}
&=1-p_X\brak{0}\\
&= p_X(1) + p_X(2)
\end{align}
Using Bayes theorem,
\begin{align}
&= p_Y\brak{0} \times \pr{Y=0 | X=1} + p_Y\brak{1} \times \pr{Y=1|X=2}\\
&=\frac{1}{3} \times \frac{6}{25} + \frac{2}{3} \times \frac{5}{50}\\
&=\frac{11}{75}
\end{align}

\newpage

%\tableofcontents

\bigskip

\renewcommand{\thefigure}{\theenumi}
\renewcommand{\thetable}{\theenumi}
%\renewcommand{\theequation}{\theenumi}

%\begin{abstract}
%%\boldmath
%In this letter, an algorithm for evaluating the exact analytical bit error rate  (BER)  for the piecewise linear (PL) combiner for  multiple relays is presented. Previous results were available only for upto three relays. The algorithm is unique in the sense that  the actual mathematical expressions, that are prohibitively large, need not be explicitly obtained. The diversity gain due to multiple relays is shown through plots of the analytical BER, well supported by simulations. 
%
%\end{abstract}
% IEEEtran.cls defaults to using nonbold math in the Abstract.
% This preserves the distinction between vectors and scalars. However,
% if the journal you are submitting to favors bold math in the abstract,
% then you can use LaTeX's standard command \boldmath at the very start
% of the abstract to achieve this. Many IEEE journals frown on math
% in the abstract anyway.

% Note that keywords are not normally used for peerreview papers.
%\begin{IEEEkeywords}
%Cooperative diversity, decode and forward, piecewise linear
%\end{IEEEkeywords}



% For peer review papers, you can put extra information on the cover
% page as needed:
% \ifCLASSOPTIONpeerreview
% \begin{center} \bfseries EDICS Category: 3-BBND \end{center}
% \fi
%
% For peerreview papers, this IEEEtran command inserts a page break and
% creates the second title. It will be ignored for other modes.
%\IEEEpeerreviewmaketitle




\item 
A die is thrown and a card is selected at random from a deck of 52 playing cards. The probability of getting an even number on the die and a spade card.\\
\solution
%\begin{table}[H]
	\centering
\begin{tabular}{|c|c|c|}
\hline
Random variable &Value &Definition\\ \hline
\multirow{3}{*}{X} &0 &Slips of Rs 1\\
&1 &Slips of Rs 5\\
&2 &Slips of Rs 13\\ \hline
\multirow{2}{*}{Y} &0 &Box A\\
&1 &Box B\\\hline
\end{tabular}
\caption{}
\label{tab:Distribution}
\end{table}
See \tabref{tab:Distribution}.
\begin{align}
p_{Y}\brak{k}= \begin{cases} 
      \frac{1}{3} & {k=0} \\
      \frac{2}{3 }& {k=1} 
   \end{cases}
   \\
p_{Y|X}\brak{0|0} = \frac{19}{25}\, 
p_{Y|X}\brak{0|1} = \frac{6}{25}\,
p_{Y|X}\brak{1|0} = \frac{45}{50}\,
p_{Y|X}\brak{1|2} = \frac{5}{50}
\end{align}
The desired probability is the probability that a slip drawn at random is marked other than Rs 1,
\begin{align}
&=1-p_X\brak{0}\\
&= p_X(1) + p_X(2)
\end{align}
Using Bayes theorem,
\begin{align}
&= p_Y\brak{0} \times \pr{Y=0 | X=1} + p_Y\brak{1} \times \pr{Y=1|X=2}\\
&=\frac{1}{3} \times \frac{6}{25} + \frac{2}{3} \times \frac{5}{50}\\
&=\frac{11}{75}
\end{align}

\newpage

%\tableofcontents

\bigskip

\renewcommand{\thefigure}{\theenumi}
\renewcommand{\thetable}{\theenumi}
%\renewcommand{\theequation}{\theenumi}

%\begin{abstract}
%%\boldmath
%In this letter, an algorithm for evaluating the exact analytical bit error rate  (BER)  for the piecewise linear (PL) combiner for  multiple relays is presented. Previous results were available only for upto three relays. The algorithm is unique in the sense that  the actual mathematical expressions, that are prohibitively large, need not be explicitly obtained. The diversity gain due to multiple relays is shown through plots of the analytical BER, well supported by simulations. 
%
%\end{abstract}
% IEEEtran.cls defaults to using nonbold math in the Abstract.
% This preserves the distinction between vectors and scalars. However,
% if the journal you are submitting to favors bold math in the abstract,
% then you can use LaTeX's standard command \boldmath at the very start
% of the abstract to achieve this. Many IEEE journals frown on math
% in the abstract anyway.

% Note that keywords are not normally used for peerreview papers.
%\begin{IEEEkeywords}
%Cooperative diversity, decode and forward, piecewise linear
%\end{IEEEkeywords}



% For peer review papers, you can put extra information on the cover
% page as needed:
% \ifCLASSOPTIONpeerreview
% \begin{center} \bfseries EDICS Category: 3-BBND \end{center}
% \fi
%
% For peerreview papers, this IEEEtran command inserts a page break and
% creates the second title. It will be ignored for other modes.
%\IEEEpeerreviewmaketitle




\item
If 4-digit numbers greater than 5,000 are randomly formed from the digits 0, 1, 3, 5, and 7, what is the probability of forming a number divisible by 5 when:
\begin{enumerate}
    \item The digits are repeated?
    \item The repetition of digits is not allowed?
\end{enumerate}
\solution
%\begin{table}[H]
	\centering
\begin{tabular}{|c|c|c|}
\hline
Random variable &Value &Definition\\ \hline
\multirow{3}{*}{X} &0 &Slips of Rs 1\\
&1 &Slips of Rs 5\\
&2 &Slips of Rs 13\\ \hline
\multirow{2}{*}{Y} &0 &Box A\\
&1 &Box B\\\hline
\end{tabular}
\caption{}
\label{tab:Distribution}
\end{table}
See \tabref{tab:Distribution}.
\begin{align}
p_{Y}\brak{k}= \begin{cases} 
      \frac{1}{3} & {k=0} \\
      \frac{2}{3 }& {k=1} 
   \end{cases}
   \\
p_{Y|X}\brak{0|0} = \frac{19}{25}\, 
p_{Y|X}\brak{0|1} = \frac{6}{25}\,
p_{Y|X}\brak{1|0} = \frac{45}{50}\,
p_{Y|X}\brak{1|2} = \frac{5}{50}
\end{align}
The desired probability is the probability that a slip drawn at random is marked other than Rs 1,
\begin{align}
&=1-p_X\brak{0}\\
&= p_X(1) + p_X(2)
\end{align}
Using Bayes theorem,
\begin{align}
&= p_Y\brak{0} \times \pr{Y=0 | X=1} + p_Y\brak{1} \times \pr{Y=1|X=2}\\
&=\frac{1}{3} \times \frac{6}{25} + \frac{2}{3} \times \frac{5}{50}\\
&=\frac{11}{75}
\end{align}

\newpage

%\tableofcontents

\bigskip

\renewcommand{\thefigure}{\theenumi}
\renewcommand{\thetable}{\theenumi}
%\renewcommand{\theequation}{\theenumi}

%\begin{abstract}
%%\boldmath
%In this letter, an algorithm for evaluating the exact analytical bit error rate  (BER)  for the piecewise linear (PL) combiner for  multiple relays is presented. Previous results were available only for upto three relays. The algorithm is unique in the sense that  the actual mathematical expressions, that are prohibitively large, need not be explicitly obtained. The diversity gain due to multiple relays is shown through plots of the analytical BER, well supported by simulations. 
%
%\end{abstract}
% IEEEtran.cls defaults to using nonbold math in the Abstract.
% This preserves the distinction between vectors and scalars. However,
% if the journal you are submitting to favors bold math in the abstract,
% then you can use LaTeX's standard command \boldmath at the very start
% of the abstract to achieve this. Many IEEE journals frown on math
% in the abstract anyway.

% Note that keywords are not normally used for peerreview papers.
%\begin{IEEEkeywords}
%Cooperative diversity, decode and forward, piecewise linear
%\end{IEEEkeywords}



% For peer review papers, you can put extra information on the cover
% page as needed:
% \ifCLASSOPTIONpeerreview
% \begin{center} \bfseries EDICS Category: 3-BBND \end{center}
% \fi
%
% For peerreview papers, this IEEEtran command inserts a page break and
% creates the second title. It will be ignored for other modes.
%\IEEEpeerreviewmaketitle




\item Consider the probability space $\brak{\Omega, \mathcal{G}, P}$ where $\Omega = [0,2]$ and $\mathcal{G} = \cbrak{\phi, \Omega, [0,1], (1,2]}$. Let $X$ and $Y$ be two functions on $\Omega$ defined as
\begin{align*}
    X(\omega) = 
    \begin{cases}
        1 & \text{if }\omega \in [0, 1]\\
        2 & \text{if }\omega \in (1, 2]
    \end{cases}
\end{align*}
and
\begin{align*}
    Y(\omega) = 
    \begin{cases}
        2 & \text{if }\omega \in [0, 1.5]\\
        3 & \text{if }\omega \in (1.5, 2].
    \end{cases}
\end{align*}
Then which one of the following statements is true?
\begin{enumerate}
    \item [(A)] $X$ is a random variable with respect to $\mathcal{G}$, but $Y$ is not a random variable with respect to $\mathcal{G}$.
    \item [(B)] $Y$ is a random variable with respect to $\mathcal{G}$, but $X$ is not a random variable with respect to $\mathcal{G}$.
    \item [(C)] Neither $X$ nor $Y$ is a random variable with respect to $\mathcal{G}$.
    \item [(D)] Both $X$ and $Y$ are random variables with respect to $\mathcal{G}$.
\end{enumerate} \hfill (GATE ST 2023)\\
\solution
%\begin{table}[H]
	\centering
\begin{tabular}{|c|c|c|}
\hline
Random variable &Value &Definition\\ \hline
\multirow{3}{*}{X} &0 &Slips of Rs 1\\
&1 &Slips of Rs 5\\
&2 &Slips of Rs 13\\ \hline
\multirow{2}{*}{Y} &0 &Box A\\
&1 &Box B\\\hline
\end{tabular}
\caption{}
\label{tab:Distribution}
\end{table}
See \tabref{tab:Distribution}.
\begin{align}
p_{Y}\brak{k}= \begin{cases} 
      \frac{1}{3} & {k=0} \\
      \frac{2}{3 }& {k=1} 
   \end{cases}
   \\
p_{Y|X}\brak{0|0} = \frac{19}{25}\, 
p_{Y|X}\brak{0|1} = \frac{6}{25}\,
p_{Y|X}\brak{1|0} = \frac{45}{50}\,
p_{Y|X}\brak{1|2} = \frac{5}{50}
\end{align}
The desired probability is the probability that a slip drawn at random is marked other than Rs 1,
\begin{align}
&=1-p_X\brak{0}\\
&= p_X(1) + p_X(2)
\end{align}
Using Bayes theorem,
\begin{align}
&= p_Y\brak{0} \times \pr{Y=0 | X=1} + p_Y\brak{1} \times \pr{Y=1|X=2}\\
&=\frac{1}{3} \times \frac{6}{25} + \frac{2}{3} \times \frac{5}{50}\\
&=\frac{11}{75}
\end{align}

\newpage

%\tableofcontents

\bigskip

\renewcommand{\thefigure}{\theenumi}
\renewcommand{\thetable}{\theenumi}
%\renewcommand{\theequation}{\theenumi}

%\begin{abstract}
%%\boldmath
%In this letter, an algorithm for evaluating the exact analytical bit error rate  (BER)  for the piecewise linear (PL) combiner for  multiple relays is presented. Previous results were available only for upto three relays. The algorithm is unique in the sense that  the actual mathematical expressions, that are prohibitively large, need not be explicitly obtained. The diversity gain due to multiple relays is shown through plots of the analytical BER, well supported by simulations. 
%
%\end{abstract}
% IEEEtran.cls defaults to using nonbold math in the Abstract.
% This preserves the distinction between vectors and scalars. However,
% if the journal you are submitting to favors bold math in the abstract,
% then you can use LaTeX's standard command \boldmath at the very start
% of the abstract to achieve this. Many IEEE journals frown on math
% in the abstract anyway.

% Note that keywords are not normally used for peerreview papers.
%\begin{IEEEkeywords}
%Cooperative diversity, decode and forward, piecewise linear
%\end{IEEEkeywords}



% For peer review papers, you can put extra information on the cover
% page as needed:
% \ifCLASSOPTIONpeerreview
% \begin{center} \bfseries EDICS Category: 3-BBND \end{center}
% \fi
%
% For peerreview papers, this IEEEtran command inserts a page break and
% creates the second title. It will be ignored for other modes.
%\IEEEpeerreviewmaketitle




	\item  A die is loaded in such a way that each odd number is twice as likely to occur as
each even number. Find $P(G)$, where $G$ is the event that a number greater than
3 occurs on a single roll of the die.
\\
\solution
		%\begin{table}[H]
	\centering
\begin{tabular}{|c|c|c|}
\hline
Random variable &Value &Definition\\ \hline
\multirow{3}{*}{X} &0 &Slips of Rs 1\\
&1 &Slips of Rs 5\\
&2 &Slips of Rs 13\\ \hline
\multirow{2}{*}{Y} &0 &Box A\\
&1 &Box B\\\hline
\end{tabular}
\caption{}
\label{tab:Distribution}
\end{table}
See \tabref{tab:Distribution}.
\begin{align}
p_{Y}\brak{k}= \begin{cases} 
      \frac{1}{3} & {k=0} \\
      \frac{2}{3 }& {k=1} 
   \end{cases}
   \\
p_{Y|X}\brak{0|0} = \frac{19}{25}\, 
p_{Y|X}\brak{0|1} = \frac{6}{25}\,
p_{Y|X}\brak{1|0} = \frac{45}{50}\,
p_{Y|X}\brak{1|2} = \frac{5}{50}
\end{align}
The desired probability is the probability that a slip drawn at random is marked other than Rs 1,
\begin{align}
&=1-p_X\brak{0}\\
&= p_X(1) + p_X(2)
\end{align}
Using Bayes theorem,
\begin{align}
&= p_Y\brak{0} \times \pr{Y=0 | X=1} + p_Y\brak{1} \times \pr{Y=1|X=2}\\
&=\frac{1}{3} \times \frac{6}{25} + \frac{2}{3} \times \frac{5}{50}\\
&=\frac{11}{75}
\end{align}

\newpage

%\tableofcontents

\bigskip

\renewcommand{\thefigure}{\theenumi}
\renewcommand{\thetable}{\theenumi}
%\renewcommand{\theequation}{\theenumi}

%\begin{abstract}
%%\boldmath
%In this letter, an algorithm for evaluating the exact analytical bit error rate  (BER)  for the piecewise linear (PL) combiner for  multiple relays is presented. Previous results were available only for upto three relays. The algorithm is unique in the sense that  the actual mathematical expressions, that are prohibitively large, need not be explicitly obtained. The diversity gain due to multiple relays is shown through plots of the analytical BER, well supported by simulations. 
%
%\end{abstract}
% IEEEtran.cls defaults to using nonbold math in the Abstract.
% This preserves the distinction between vectors and scalars. However,
% if the journal you are submitting to favors bold math in the abstract,
% then you can use LaTeX's standard command \boldmath at the very start
% of the abstract to achieve this. Many IEEE journals frown on math
% in the abstract anyway.

% Note that keywords are not normally used for peerreview papers.
%\begin{IEEEkeywords}
%Cooperative diversity, decode and forward, piecewise linear
%\end{IEEEkeywords}



% For peer review papers, you can put extra information on the cover
% page as needed:
% \ifCLASSOPTIONpeerreview
% \begin{center} \bfseries EDICS Category: 3-BBND \end{center}
% \fi
%
% For peerreview papers, this IEEEtran command inserts a page break and
% creates the second title. It will be ignored for other modes.
%\IEEEpeerreviewmaketitle




	\item All the jacks, queens and kings are removed from a deck of 52 playing cards. The remaining cards are well shuffled and then one card is drawn at random. Giving ace a value 1 similar value for other cards, find the probability that the card has a value 
		\begin{enumerate}
			\item 7
			\item greater than 7
			\item less than 7
		\end{enumerate}
		%Number of cards left after removing all jacks, queens and kings 
\begin{align}
N	= 52 - 4\times 3
	= 40
\end{align}
%\begin{table}[H]
%\def\arraystretch{1.2}
%\begin{tabular}{|c|c|c|}
%\hline
%	\textbf{Parameter} &\textbf{Value} &\textbf{Description}\\ \hline
%	$X$ &1-10 &Represents the value of the card picked \\ \hline
%\end{tabular}
%\end{table}
Let $1 \le X \le 10$ be the value of the card picked.  Then,
\begin{align}
	p_X(k) &= \Pr(X=k)\ \forall\ 1 \leq k \leq 10\\
	&= \frac{4\times 1}{40}\\
	&= \frac{1}{10}\\
	\therefore p_X(k) &= 
	\begin{cases}
		\frac{1}{10} & 1 \leq k \leq 10\\
		0 & \text{otherwise}
	\end{cases}
\end{align}
and
\begin{align}
	F_{X}(k) &= \sum_{m=0}^{k}p_{X}(m) \quad 1 \leq k \leq 10\\
	&= \frac{k}{10}\\
	\therefore F_{X}(k) &= 
	\begin{cases}
		0 & k \leq 0\\
		\frac{k}{10} & 1\leq k \leq 10\\
		1 & k > 10 
	\end{cases}
\end{align}
\begin{enumerate}
	\item Probability that card has value equal to 7 is
		\begin{align}
			 p_{X}(7)
			= \frac{1}{10}
		\end{align}
	\item Probability that card has value greater than 7 is
		\begin{align}
			1 - F_X(7)
			&= 1 - \frac{7}{10}
			\\
			&= \frac{3}{10}
		\end{align}
	\item Probability that card has value less than 7 is
		\begin{align}
			 F_{X}(6)
			=\frac{6}{10}
		\end{align}
\end{enumerate}

  \item A Lot consists of 48 mobile phones of which 42 are good, 3 have only minor defects and 3 have major defects.Varnika will buy a phone if it is good but the trader will only buy a mobile if it has no major defects. One phone is selected at random from the lot. What is the probability that it is
\begin{enumerate}
	\item acceptable to Varnika?
            \item acceptable to the trader?
\end{enumerate}
\solution
	%\begin{table}[H]
	\centering
\begin{tabular}{|c|c|c|}
\hline
Random variable &Value &Definition\\ \hline
\multirow{3}{*}{X} &0 &Slips of Rs 1\\
&1 &Slips of Rs 5\\
&2 &Slips of Rs 13\\ \hline
\multirow{2}{*}{Y} &0 &Box A\\
&1 &Box B\\\hline
\end{tabular}
\caption{}
\label{tab:Distribution}
\end{table}
See \tabref{tab:Distribution}.
\begin{align}
p_{Y}\brak{k}= \begin{cases} 
      \frac{1}{3} & {k=0} \\
      \frac{2}{3 }& {k=1} 
   \end{cases}
   \\
p_{Y|X}\brak{0|0} = \frac{19}{25}\, 
p_{Y|X}\brak{0|1} = \frac{6}{25}\,
p_{Y|X}\brak{1|0} = \frac{45}{50}\,
p_{Y|X}\brak{1|2} = \frac{5}{50}
\end{align}
The desired probability is the probability that a slip drawn at random is marked other than Rs 1,
\begin{align}
&=1-p_X\brak{0}\\
&= p_X(1) + p_X(2)
\end{align}
Using Bayes theorem,
\begin{align}
&= p_Y\brak{0} \times \pr{Y=0 | X=1} + p_Y\brak{1} \times \pr{Y=1|X=2}\\
&=\frac{1}{3} \times \frac{6}{25} + \frac{2}{3} \times \frac{5}{50}\\
&=\frac{11}{75}
\end{align}

\newpage

%\tableofcontents

\bigskip

\renewcommand{\thefigure}{\theenumi}
\renewcommand{\thetable}{\theenumi}
%\renewcommand{\theequation}{\theenumi}

%\begin{abstract}
%%\boldmath
%In this letter, an algorithm for evaluating the exact analytical bit error rate  (BER)  for the piecewise linear (PL) combiner for  multiple relays is presented. Previous results were available only for upto three relays. The algorithm is unique in the sense that  the actual mathematical expressions, that are prohibitively large, need not be explicitly obtained. The diversity gain due to multiple relays is shown through plots of the analytical BER, well supported by simulations. 
%
%\end{abstract}
% IEEEtran.cls defaults to using nonbold math in the Abstract.
% This preserves the distinction between vectors and scalars. However,
% if the journal you are submitting to favors bold math in the abstract,
% then you can use LaTeX's standard command \boldmath at the very start
% of the abstract to achieve this. Many IEEE journals frown on math
% in the abstract anyway.

% Note that keywords are not normally used for peerreview papers.
%\begin{IEEEkeywords}
%Cooperative diversity, decode and forward, piecewise linear
%\end{IEEEkeywords}



% For peer review papers, you can put extra information on the cover
% page as needed:
% \ifCLASSOPTIONpeerreview
% \begin{center} \bfseries EDICS Category: 3-BBND \end{center}
% \fi
%
% For peerreview papers, this IEEEtran command inserts a page break and
% creates the second title. It will be ignored for other modes.
%\IEEEpeerreviewmaketitle




 \item A student says that if you throw a die, it will show up 1 or not 1. Therefore, the probability of getting 1 and the probability of getting 'not 1' each is equal to $\frac{1}{2}$. Is this correct? Give reasons.\\
 \solution
        %\begin{table}[H]
	\centering
\begin{tabular}{|c|c|c|}
\hline
Random variable &Value &Definition\\ \hline
\multirow{3}{*}{X} &0 &Slips of Rs 1\\
&1 &Slips of Rs 5\\
&2 &Slips of Rs 13\\ \hline
\multirow{2}{*}{Y} &0 &Box A\\
&1 &Box B\\\hline
\end{tabular}
\caption{}
\label{tab:Distribution}
\end{table}
See \tabref{tab:Distribution}.
\begin{align}
p_{Y}\brak{k}= \begin{cases} 
      \frac{1}{3} & {k=0} \\
      \frac{2}{3 }& {k=1} 
   \end{cases}
   \\
p_{Y|X}\brak{0|0} = \frac{19}{25}\, 
p_{Y|X}\brak{0|1} = \frac{6}{25}\,
p_{Y|X}\brak{1|0} = \frac{45}{50}\,
p_{Y|X}\brak{1|2} = \frac{5}{50}
\end{align}
The desired probability is the probability that a slip drawn at random is marked other than Rs 1,
\begin{align}
&=1-p_X\brak{0}\\
&= p_X(1) + p_X(2)
\end{align}
Using Bayes theorem,
\begin{align}
&= p_Y\brak{0} \times \pr{Y=0 | X=1} + p_Y\brak{1} \times \pr{Y=1|X=2}\\
&=\frac{1}{3} \times \frac{6}{25} + \frac{2}{3} \times \frac{5}{50}\\
&=\frac{11}{75}
\end{align}

\newpage

%\tableofcontents

\bigskip

\renewcommand{\thefigure}{\theenumi}
\renewcommand{\thetable}{\theenumi}
%\renewcommand{\theequation}{\theenumi}

%\begin{abstract}
%%\boldmath
%In this letter, an algorithm for evaluating the exact analytical bit error rate  (BER)  for the piecewise linear (PL) combiner for  multiple relays is presented. Previous results were available only for upto three relays. The algorithm is unique in the sense that  the actual mathematical expressions, that are prohibitively large, need not be explicitly obtained. The diversity gain due to multiple relays is shown through plots of the analytical BER, well supported by simulations. 
%
%\end{abstract}
% IEEEtran.cls defaults to using nonbold math in the Abstract.
% This preserves the distinction between vectors and scalars. However,
% if the journal you are submitting to favors bold math in the abstract,
% then you can use LaTeX's standard command \boldmath at the very start
% of the abstract to achieve this. Many IEEE journals frown on math
% in the abstract anyway.

% Note that keywords are not normally used for peerreview papers.
%\begin{IEEEkeywords}
%Cooperative diversity, decode and forward, piecewise linear
%\end{IEEEkeywords}



% For peer review papers, you can put extra information on the cover
% page as needed:
% \ifCLASSOPTIONpeerreview
% \begin{center} \bfseries EDICS Category: 3-BBND \end{center}
% \fi
%
% For peerreview papers, this IEEEtran command inserts a page break and
% creates the second title. It will be ignored for other modes.
%\IEEEpeerreviewmaketitle




   \item Four candidates A, B, C, D have ap-
plied for the assignment to coach a school cricket
team. If A is twice as likely to be selected as B, and
B and C are given about the same chance of being
selected, while C is twice as likely to be selected
as D, what are the probabilities that
\begin{enumerate}
\item C will be selected?
\item A will not be selected?
\end{enumerate}
	%\begin{table}[H]
	\centering
\begin{tabular}{|c|c|c|}
\hline
Random variable &Value &Definition\\ \hline
\multirow{3}{*}{X} &0 &Slips of Rs 1\\
&1 &Slips of Rs 5\\
&2 &Slips of Rs 13\\ \hline
\multirow{2}{*}{Y} &0 &Box A\\
&1 &Box B\\\hline
\end{tabular}
\caption{}
\label{tab:Distribution}
\end{table}
See \tabref{tab:Distribution}.
\begin{align}
p_{Y}\brak{k}= \begin{cases} 
      \frac{1}{3} & {k=0} \\
      \frac{2}{3 }& {k=1} 
   \end{cases}
   \\
p_{Y|X}\brak{0|0} = \frac{19}{25}\, 
p_{Y|X}\brak{0|1} = \frac{6}{25}\,
p_{Y|X}\brak{1|0} = \frac{45}{50}\,
p_{Y|X}\brak{1|2} = \frac{5}{50}
\end{align}
The desired probability is the probability that a slip drawn at random is marked other than Rs 1,
\begin{align}
&=1-p_X\brak{0}\\
&= p_X(1) + p_X(2)
\end{align}
Using Bayes theorem,
\begin{align}
&= p_Y\brak{0} \times \pr{Y=0 | X=1} + p_Y\brak{1} \times \pr{Y=1|X=2}\\
&=\frac{1}{3} \times \frac{6}{25} + \frac{2}{3} \times \frac{5}{50}\\
&=\frac{11}{75}
\end{align}

\newpage

%\tableofcontents

\bigskip

\renewcommand{\thefigure}{\theenumi}
\renewcommand{\thetable}{\theenumi}
%\renewcommand{\theequation}{\theenumi}

%\begin{abstract}
%%\boldmath
%In this letter, an algorithm for evaluating the exact analytical bit error rate  (BER)  for the piecewise linear (PL) combiner for  multiple relays is presented. Previous results were available only for upto three relays. The algorithm is unique in the sense that  the actual mathematical expressions, that are prohibitively large, need not be explicitly obtained. The diversity gain due to multiple relays is shown through plots of the analytical BER, well supported by simulations. 
%
%\end{abstract}
% IEEEtran.cls defaults to using nonbold math in the Abstract.
% This preserves the distinction between vectors and scalars. However,
% if the journal you are submitting to favors bold math in the abstract,
% then you can use LaTeX's standard command \boldmath at the very start
% of the abstract to achieve this. Many IEEE journals frown on math
% in the abstract anyway.

% Note that keywords are not normally used for peerreview papers.
%\begin{IEEEkeywords}
%Cooperative diversity, decode and forward, piecewise linear
%\end{IEEEkeywords}



% For peer review papers, you can put extra information on the cover
% page as needed:
% \ifCLASSOPTIONpeerreview
% \begin{center} \bfseries EDICS Category: 3-BBND \end{center}
% \fi
%
% For peerreview papers, this IEEEtran command inserts a page break and
% creates the second title. It will be ignored for other modes.
%\IEEEpeerreviewmaketitle




 \item A bag contain 24 balls of which $x$ balls are red, $2x$ are white and $3x$ are blue. A ball is selected at random, What is the probability that it is
\begin{enumerate}[label=\alph*)]
\item not red ?
\item white ?
\end{enumerate}
%\begin{table}[H]
	\centering
\begin{tabular}{|c|c|c|}
\hline
Random variable &Value &Definition\\ \hline
\multirow{3}{*}{X} &0 &Slips of Rs 1\\
&1 &Slips of Rs 5\\
&2 &Slips of Rs 13\\ \hline
\multirow{2}{*}{Y} &0 &Box A\\
&1 &Box B\\\hline
\end{tabular}
\caption{}
\label{tab:Distribution}
\end{table}
See \tabref{tab:Distribution}.
\begin{align}
p_{Y}\brak{k}= \begin{cases} 
      \frac{1}{3} & {k=0} \\
      \frac{2}{3 }& {k=1} 
   \end{cases}
   \\
p_{Y|X}\brak{0|0} = \frac{19}{25}\, 
p_{Y|X}\brak{0|1} = \frac{6}{25}\,
p_{Y|X}\brak{1|0} = \frac{45}{50}\,
p_{Y|X}\brak{1|2} = \frac{5}{50}
\end{align}
The desired probability is the probability that a slip drawn at random is marked other than Rs 1,
\begin{align}
&=1-p_X\brak{0}\\
&= p_X(1) + p_X(2)
\end{align}
Using Bayes theorem,
\begin{align}
&= p_Y\brak{0} \times \pr{Y=0 | X=1} + p_Y\brak{1} \times \pr{Y=1|X=2}\\
&=\frac{1}{3} \times \frac{6}{25} + \frac{2}{3} \times \frac{5}{50}\\
&=\frac{11}{75}
\end{align}

\newpage

%\tableofcontents

\bigskip

\renewcommand{\thefigure}{\theenumi}
\renewcommand{\thetable}{\theenumi}
%\renewcommand{\theequation}{\theenumi}

%\begin{abstract}
%%\boldmath
%In this letter, an algorithm for evaluating the exact analytical bit error rate  (BER)  for the piecewise linear (PL) combiner for  multiple relays is presented. Previous results were available only for upto three relays. The algorithm is unique in the sense that  the actual mathematical expressions, that are prohibitively large, need not be explicitly obtained. The diversity gain due to multiple relays is shown through plots of the analytical BER, well supported by simulations. 
%
%\end{abstract}
% IEEEtran.cls defaults to using nonbold math in the Abstract.
% This preserves the distinction between vectors and scalars. However,
% if the journal you are submitting to favors bold math in the abstract,
% then you can use LaTeX's standard command \boldmath at the very start
% of the abstract to achieve this. Many IEEE journals frown on math
% in the abstract anyway.

% Note that keywords are not normally used for peerreview papers.
%\begin{IEEEkeywords}
%Cooperative diversity, decode and forward, piecewise linear
%\end{IEEEkeywords}



% For peer review papers, you can put extra information on the cover
% page as needed:
% \ifCLASSOPTIONpeerreview
% \begin{center} \bfseries EDICS Category: 3-BBND \end{center}
% \fi
%
% For peerreview papers, this IEEEtran command inserts a page break and
% creates the second title. It will be ignored for other modes.
%\IEEEpeerreviewmaketitle




If the letters of the word ASSASSINATION are arranged at random. Find the Probability that
\begin{enumerate}[label=(\alph*)]
\item Four $S's$ come consecutively in the word
\item Two  $I's$ and two $N's$ come together
\item All $A's$ are not coming together
\item No two $A's$ are coming together
\end{enumerate}
%\begin{table}[H]
	\centering
\begin{tabular}{|c|c|c|}
\hline
Random variable &Value &Definition\\ \hline
\multirow{3}{*}{X} &0 &Slips of Rs 1\\
&1 &Slips of Rs 5\\
&2 &Slips of Rs 13\\ \hline
\multirow{2}{*}{Y} &0 &Box A\\
&1 &Box B\\\hline
\end{tabular}
\caption{}
\label{tab:Distribution}
\end{table}
See \tabref{tab:Distribution}.
\begin{align}
p_{Y}\brak{k}= \begin{cases} 
      \frac{1}{3} & {k=0} \\
      \frac{2}{3 }& {k=1} 
   \end{cases}
   \\
p_{Y|X}\brak{0|0} = \frac{19}{25}\, 
p_{Y|X}\brak{0|1} = \frac{6}{25}\,
p_{Y|X}\brak{1|0} = \frac{45}{50}\,
p_{Y|X}\brak{1|2} = \frac{5}{50}
\end{align}
The desired probability is the probability that a slip drawn at random is marked other than Rs 1,
\begin{align}
&=1-p_X\brak{0}\\
&= p_X(1) + p_X(2)
\end{align}
Using Bayes theorem,
\begin{align}
&= p_Y\brak{0} \times \pr{Y=0 | X=1} + p_Y\brak{1} \times \pr{Y=1|X=2}\\
&=\frac{1}{3} \times \frac{6}{25} + \frac{2}{3} \times \frac{5}{50}\\
&=\frac{11}{75}
\end{align}

\newpage

%\tableofcontents

\bigskip

\renewcommand{\thefigure}{\theenumi}
\renewcommand{\thetable}{\theenumi}
%\renewcommand{\theequation}{\theenumi}

%\begin{abstract}
%%\boldmath
%In this letter, an algorithm for evaluating the exact analytical bit error rate  (BER)  for the piecewise linear (PL) combiner for  multiple relays is presented. Previous results were available only for upto three relays. The algorithm is unique in the sense that  the actual mathematical expressions, that are prohibitively large, need not be explicitly obtained. The diversity gain due to multiple relays is shown through plots of the analytical BER, well supported by simulations. 
%
%\end{abstract}
% IEEEtran.cls defaults to using nonbold math in the Abstract.
% This preserves the distinction between vectors and scalars. However,
% if the journal you are submitting to favors bold math in the abstract,
% then you can use LaTeX's standard command \boldmath at the very start
% of the abstract to achieve this. Many IEEE journals frown on math
% in the abstract anyway.

% Note that keywords are not normally used for peerreview papers.
%\begin{IEEEkeywords}
%Cooperative diversity, decode and forward, piecewise linear
%\end{IEEEkeywords}



% For peer review papers, you can put extra information on the cover
% page as needed:
% \ifCLASSOPTIONpeerreview
% \begin{center} \bfseries EDICS Category: 3-BBND \end{center}
% \fi
%
% For peerreview papers, this IEEEtran command inserts a page break and
% creates the second title. It will be ignored for other modes.
%\IEEEpeerreviewmaketitle




	\item One urn contains two black balls (labelled B1 and B2) and one white ball. A
	second urn contains one black ball and two white balls (labelled W1 and W2).
	Suppose the following experiment is performed. One of the two urns is chosen
	at random. Next a ball is randomly chosen from the urn. Then a second ball is
	chosen at random from the same urn without replacing the first ball.
	
	\begin{enumerate}
	\item What is the probability that two black balls are chosen?
	
	\item What is the probability that two balls of opposite colour are chosen?
	\end{enumerate}
	\solution
	%\begin{align}
    \label{eq:12.13.6.18.1}
	\because	\pr{A|B} &> \pr{A},\
\frac{\pr{AB}}{\pr{B}} > \pr{A}
\\
    \label{eq:12.13.6.18.2}
	\implies \pr{AB} &> \pr{A}\pr{B}
	\\
	\text{or, } \frac{\pr{AB}}{\pr{A}} &=\pr{B|A} > \pr{A}
\end{align}

\end{enumerate}

		%
\item 
Out of 100 students, two sections of 40 and 60 are formed. If you and your friend are among the 100 students, what is the probability that
\begin{enumerate}
\item you both enter the same section?
\item you both enter the different sections?
\end{enumerate}
\solution
		%\begin{enumerate}[label=\thesection.\arabic*,ref=\thesection.\theenumi]
	\item One card is drawn from a well-shuffled deck of 52 cards. Find the probability of getting
\begin{enumerate}
\item A king of red colour 
\item A face card 
\item A red face card
\item The jack of hearts
\item A spade
\item The queen of diamonds

\end{enumerate}
\solution
		%\begin{table}[H]
	\centering
\begin{tabular}{|c|c|c|}
\hline
Random variable &Value &Definition\\ \hline
\multirow{3}{*}{X} &0 &Slips of Rs 1\\
&1 &Slips of Rs 5\\
&2 &Slips of Rs 13\\ \hline
\multirow{2}{*}{Y} &0 &Box A\\
&1 &Box B\\\hline
\end{tabular}
\caption{}
\label{tab:Distribution}
\end{table}
See \tabref{tab:Distribution}.
\begin{align}
p_{Y}\brak{k}= \begin{cases} 
      \frac{1}{3} & {k=0} \\
      \frac{2}{3 }& {k=1} 
   \end{cases}
   \\
p_{Y|X}\brak{0|0} = \frac{19}{25}\, 
p_{Y|X}\brak{0|1} = \frac{6}{25}\,
p_{Y|X}\brak{1|0} = \frac{45}{50}\,
p_{Y|X}\brak{1|2} = \frac{5}{50}
\end{align}
The desired probability is the probability that a slip drawn at random is marked other than Rs 1,
\begin{align}
&=1-p_X\brak{0}\\
&= p_X(1) + p_X(2)
\end{align}
Using Bayes theorem,
\begin{align}
&= p_Y\brak{0} \times \pr{Y=0 | X=1} + p_Y\brak{1} \times \pr{Y=1|X=2}\\
&=\frac{1}{3} \times \frac{6}{25} + \frac{2}{3} \times \frac{5}{50}\\
&=\frac{11}{75}
\end{align}

\newpage

%\tableofcontents

\bigskip

\renewcommand{\thefigure}{\theenumi}
\renewcommand{\thetable}{\theenumi}
%\renewcommand{\theequation}{\theenumi}

%\begin{abstract}
%%\boldmath
%In this letter, an algorithm for evaluating the exact analytical bit error rate  (BER)  for the piecewise linear (PL) combiner for  multiple relays is presented. Previous results were available only for upto three relays. The algorithm is unique in the sense that  the actual mathematical expressions, that are prohibitively large, need not be explicitly obtained. The diversity gain due to multiple relays is shown through plots of the analytical BER, well supported by simulations. 
%
%\end{abstract}
% IEEEtran.cls defaults to using nonbold math in the Abstract.
% This preserves the distinction between vectors and scalars. However,
% if the journal you are submitting to favors bold math in the abstract,
% then you can use LaTeX's standard command \boldmath at the very start
% of the abstract to achieve this. Many IEEE journals frown on math
% in the abstract anyway.

% Note that keywords are not normally used for peerreview papers.
%\begin{IEEEkeywords}
%Cooperative diversity, decode and forward, piecewise linear
%\end{IEEEkeywords}



% For peer review papers, you can put extra information on the cover
% page as needed:
% \ifCLASSOPTIONpeerreview
% \begin{center} \bfseries EDICS Category: 3-BBND \end{center}
% \fi
%
% For peerreview papers, this IEEEtran command inserts a page break and
% creates the second title. It will be ignored for other modes.
%\IEEEpeerreviewmaketitle




	\item Five cards—the ten, jack, queen, king and ace of diamonds, are well-shuffled with their face downwards. One card is then picked up at random.
\begin{enumerate}
\item
What is the probability that the card is the queen? 
\item
If the queen is drawn and put aside, what is the probability that the second card picked up is (a) an ace? (b) a queen?\\
\end{enumerate}
\solution
		%\begin{enumerate}[label=\thesection.\arabic*,ref=\thesection.\theenumi]
	\item One card is drawn from a well-shuffled deck of 52 cards. Find the probability of getting
\begin{enumerate}
\item A king of red colour 
\item A face card 
\item A red face card
\item The jack of hearts
\item A spade
\item The queen of diamonds

\end{enumerate}
\solution
		%\input{ncert/10/15/1/14/main.tex}
	\item Five cards—the ten, jack, queen, king and ace of diamonds, are well-shuffled with their face downwards. One card is then picked up at random.
\begin{enumerate}
\item
What is the probability that the card is the queen? 
\item
If the queen is drawn and put aside, what is the probability that the second card picked up is (a) an ace? (b) a queen?\\
\end{enumerate}
\solution
		%\input{ncert/10/15/1/15/defs.tex}
	\item A bag contains $5$ red balls and some blue balls. If the probability of drawing a blue ball is double that if a red ball, determine the number of blue balls in the bag. 
		\\
\solution
		%\input{ncert/10/15/2/3/defs.tex}
	\item A card is selected from a pack of 52 cards.
 \begin{enumerate}[label=(\alph*)] 
                 \item How many points are there in the sample space?
                 \item Calculate the probability that the card is an ace of spades.
                 \item Calculate the probability that the card is (i) an ace and (ii) black card.
 \end{enumerate}
\solution
		%\input{ncert/11/16/3/4/main.tex}
\item Four cards are drawn from a well-shuffled deck of 52 cards. What is the probability of obtaining 3 diamonds and one spade.
\\
\solution
		%\input{ncert/11/16/4/2/defs.tex}
\item In a certain lottery 10,000 tickets are sold and ten equal prizes are awarded. What is the probability of not getting a prize if you buy (a) one ticket (b) two tickets (c) 10 tickets ?	
\\
\solution
		%\input{ncert/11/16/4/4/defs.tex}
		%
\item 
Out of 100 students, two sections of 40 and 60 are formed. If you and your friend are among the 100 students, what is the probability that
\begin{enumerate}
\item you both enter the same section?
\item you both enter the different sections?
\end{enumerate}
\solution
		%\input{ncert/11/16/4/5/defs.tex}
	\item 
The number lock of a suitcase has 4 wheels each labelled with ten digits i.e. from 0 to 9.The lock opens with a sequence of four digits with no repeats.What is the probability of a person getting the right sequence to open the suitcase.
\\
\solution
		%\input{ncert/11/16/4/10/defs.tex}
		%
\item 
Two cards are drawn at random and without replacement from a pack of 52 playing cards. Find the probability that both the cards are black.
\\
\solution
		%\input{ncert/12/13/2/2/defs.tex}
		\item A box of oranges is inspected by examining three randomly selected oranges drawn without replacement. If all the three oranges are good, the box is approved for sale, otherwise, it is rejected. Find the probability that a box containing 15 oranges out of which 12 are good and 3 are bad ones will be approved for sale.
		\label{ncert/12/13/2/3/defs.tex}
		\item Two balls are drawn at random with replacement from a box containing 10 black and 8 red balls. Find the probability that
		\label{ncert/12/13/2/12}
\begin{enumerate}
\item both balls are red.
\item first ball is black and second is red.
\item one of them is black and other is red.
\end{enumerate}

\item In a hostel, 60\% of the students read Hindi newspaper, 40\% read English newspaper and 20\% read both Hindi and English newspapers. A student is selected at random.
		\label{ncert/12/13/2/15}
\begin{enumerate}
\item Find the probability that she reads neither Hindi nor English newspapers.
\item If she reads Hindi newspaper, find the probability that she reads English newspaper.
\item If she reads English newspaper, find the probability that she reads Hindi newspaper.\\
\end{enumerate}
\item The probability of obtaining an even prime number on each die, when a pair of dice is rolled is 
\begin{enumerate}
    \item $0$ 
    
    \item $\frac{1}{3}$ 
    
    \item $\frac{1}{12}$ 
    
    \item $\frac{1}{36}$ 
\end{enumerate}
\solution
		%\input{ncert/12/13/2/17/defs.tex}
	\item A bag contains 4 red and 4 black balls, another bag contains 2 red and 6 black balls. One of the two bags is selected at random and a ball is drawn from the bag which is found to be red. Find the probability that the ball is drawn from the first bag.
\\
\solution
		%\input{ncert/12/13/3/2/main.tex}
  \item
  Cards with numbers 2 to 101 are placed in a box. A card is selected at random.Find the probability that the card has
\begin{enumerate}[label=(\roman*)]
	\item an even number 
	\item a square number
\end{enumerate}
\solution
%\input{exemplar/10/13/3/32/main.tex}
\item
The king, queen and jack of clubs are removed from a deck of 52 playing cards and then well shuffled. Now one card is drawn at random from the remaining cards.  Determine the probability that the card is
\begin{enumerate}[label=(\roman*)]
\item a club
\item 10 of hearts
\end{enumerate}
\solution
%\input{exemplar/10/13/3/29/main.tex}
\item A team of medical students doing their internship have to assist during surgeries
at a city hospital. The probabilities of surgeries rated as very complex, complex,
routine, simple or very simple are respectively, 0.15, 0.20, 0.31, 0.26, .08. Find
the probabilities that a particular surgery will be rated
\begin{enumerate}
	\item complex or very complex;
	\item neither very complex nor very simple;
	\item routine or complex
	\item routine or simple
\end{enumerate}
\solution
%\input{exemplar/11/16/3/8(1)/main.tex}
\item A card is selected from a pack of 52 cards.
\begin{enumerate}[label=(\alph*)]
    \item How many points are there in the sample space?
    \item Calculate the probability that the card is an ace of spades.
    \item Calculate the probability that the card is (i) an ace and (ii) black card.
\end{enumerate}
\solution
%\input{exemplar/11/16/3/4/main2.tex}
\item The probability that a non leap year selected at random will contain 53 sundays.
\\
\solution
%\input{exemplar/10/13/1/19/main.tex}
\item One of the four persons John, Rita, Aslam or Gurpreet will be promoted next
month. Consequently the sample space consists of four elementary outcomes
S = {John promoted, Rita promoted, Aslam promoted, Gurpreet promoted}
You are told that the chances of John’s promotion is same as that of Gurpreet,
Rita’s chances of promotion are twice as likely as Johns. Aslam’s chances are
four times that of John.
\begin{enumerate}
	\item Determine
	\begin{enumerate}
		\item P (John promoted)
		\item P (Rita promoted)
		\item P (Aslam promoted)
		\item P (Gurpreet promoted)
	\end{enumerate}
	\item If A = {John promoted or Gurpreet promoted}, find P (A).
\end{enumerate}
\solution
%\input{exemplar/11/16/3/10/main.tex}
\item A card is drawn from a deck of 52 cards. Find the probability of getting a king or a heart or a red card.\\
\solution
%\input{exemplar/11/16/3/15/main.tex}
\item The probability that a student will pass his examination is 0.73, the probability of
the student getting a compartment is 0.13, and the probability that the student will
either pass or get compartment is 0.96. State True or False.\\
\solution
%\input{exemplar/11/16/3/31/main.tex}
\item A card is selected from a pack of 52 cards\\
\begin{enumerate}[label=(\alph*)]
\item How many points are there in the sample space?
\item Calculate the probability that the cards is an ace of spades.
\item Calculate the probability that the card is (i) an ace (ii)black card.\\
\end{enumerate}
%\input{ncert/11/16/3/4_1/Prob_4.tex}
\item In a non-leap year, the probability of having 53 tuesdays or 53 wednesdays is\\
\solution
%\input{exemplar/11/16/3/18/main.tex}
\item There are 1000 sealed envelopes in a box, 10 of them contain a cash prize of
Rs 100 each, 100 of them contain a cash prize of Rs 50 each and 200 of them
contain a cash prize of Rs 10 each and rest do not contain any cash prize. If they
are well shuffled and an envelope is picked up out, what is the probability that it
contains no cash prize?\\
\solution
%\input{exemplar/10/13/3/34/main.tex}
\item 
A die is thrown and a card is selected at random from a deck of 52 playing cards. The probability of getting an even number on the die and a spade card.\\
\solution
%\input{exemplar/12/13/3/78/main.tex}
\item
If 4-digit numbers greater than 5,000 are randomly formed from the digits 0, 1, 3, 5, and 7, what is the probability of forming a number divisible by 5 when:
\begin{enumerate}
    \item The digits are repeated?
    \item The repetition of digits is not allowed?
\end{enumerate}
\solution
%\input{ncert/11/16/4/9/main.tex}
\item Consider the probability space $\brak{\Omega, \mathcal{G}, P}$ where $\Omega = [0,2]$ and $\mathcal{G} = \cbrak{\phi, \Omega, [0,1], (1,2]}$. Let $X$ and $Y$ be two functions on $\Omega$ defined as
\begin{align*}
    X(\omega) = 
    \begin{cases}
        1 & \text{if }\omega \in [0, 1]\\
        2 & \text{if }\omega \in (1, 2]
    \end{cases}
\end{align*}
and
\begin{align*}
    Y(\omega) = 
    \begin{cases}
        2 & \text{if }\omega \in [0, 1.5]\\
        3 & \text{if }\omega \in (1.5, 2].
    \end{cases}
\end{align*}
Then which one of the following statements is true?
\begin{enumerate}
    \item [(A)] $X$ is a random variable with respect to $\mathcal{G}$, but $Y$ is not a random variable with respect to $\mathcal{G}$.
    \item [(B)] $Y$ is a random variable with respect to $\mathcal{G}$, but $X$ is not a random variable with respect to $\mathcal{G}$.
    \item [(C)] Neither $X$ nor $Y$ is a random variable with respect to $\mathcal{G}$.
    \item [(D)] Both $X$ and $Y$ are random variables with respect to $\mathcal{G}$.
\end{enumerate} \hfill (GATE ST 2023)\\
\solution
%\input{gate/ST/2023/14/main.tex}
	\item  A die is loaded in such a way that each odd number is twice as likely to occur as
each even number. Find $P(G)$, where $G$ is the event that a number greater than
3 occurs on a single roll of the die.
\\
\solution
		%\input{exemplar/11/16/3/5/main.tex}
	\item All the jacks, queens and kings are removed from a deck of 52 playing cards. The remaining cards are well shuffled and then one card is drawn at random. Giving ace a value 1 similar value for other cards, find the probability that the card has a value 
		\begin{enumerate}
			\item 7
			\item greater than 7
			\item less than 7
		\end{enumerate}
		%\input{exemplar/10/13/3/30/main.tex}
  \item A Lot consists of 48 mobile phones of which 42 are good, 3 have only minor defects and 3 have major defects.Varnika will buy a phone if it is good but the trader will only buy a mobile if it has no major defects. One phone is selected at random from the lot. What is the probability that it is
\begin{enumerate}
	\item acceptable to Varnika?
            \item acceptable to the trader?
\end{enumerate}
\solution
	%\input{exemplar/10/13/3/40/main.tex}
 \item A student says that if you throw a die, it will show up 1 or not 1. Therefore, the probability of getting 1 and the probability of getting 'not 1' each is equal to $\frac{1}{2}$. Is this correct? Give reasons.\\
 \solution
        %\input{exemplar/10/13/2/9/main.tex}
   \item Four candidates A, B, C, D have ap-
plied for the assignment to coach a school cricket
team. If A is twice as likely to be selected as B, and
B and C are given about the same chance of being
selected, while C is twice as likely to be selected
as D, what are the probabilities that
\begin{enumerate}
\item C will be selected?
\item A will not be selected?
\end{enumerate}
	%\input{exemplar/11/16/3/9/main.tex}
 \item A bag contain 24 balls of which $x$ balls are red, $2x$ are white and $3x$ are blue. A ball is selected at random, What is the probability that it is
\begin{enumerate}[label=\alph*)]
\item not red ?
\item white ?
\end{enumerate}
%\input{exemplar/10/13/3/41/main.tex}
If the letters of the word ASSASSINATION are arranged at random. Find the Probability that
\begin{enumerate}[label=(\alph*)]
\item Four $S's$ come consecutively in the word
\item Two  $I's$ and two $N's$ come together
\item All $A's$ are not coming together
\item No two $A's$ are coming together
\end{enumerate}
%\input{exemplar/11/16/3/14/main.tex}
	\item One urn contains two black balls (labelled B1 and B2) and one white ball. A
	second urn contains one black ball and two white balls (labelled W1 and W2).
	Suppose the following experiment is performed. One of the two urns is chosen
	at random. Next a ball is randomly chosen from the urn. Then a second ball is
	chosen at random from the same urn without replacing the first ball.
	
	\begin{enumerate}
	\item What is the probability that two black balls are chosen?
	
	\item What is the probability that two balls of opposite colour are chosen?
	\end{enumerate}
	\solution
	%\input{exemplar/11/16/3/12/main1.tex}
\end{enumerate}

	\item A bag contains $5$ red balls and some blue balls. If the probability of drawing a blue ball is double that if a red ball, determine the number of blue balls in the bag. 
		\\
\solution
		%\begin{enumerate}[label=\thesection.\arabic*,ref=\thesection.\theenumi]
	\item One card is drawn from a well-shuffled deck of 52 cards. Find the probability of getting
\begin{enumerate}
\item A king of red colour 
\item A face card 
\item A red face card
\item The jack of hearts
\item A spade
\item The queen of diamonds

\end{enumerate}
\solution
		%\input{ncert/10/15/1/14/main.tex}
	\item Five cards—the ten, jack, queen, king and ace of diamonds, are well-shuffled with their face downwards. One card is then picked up at random.
\begin{enumerate}
\item
What is the probability that the card is the queen? 
\item
If the queen is drawn and put aside, what is the probability that the second card picked up is (a) an ace? (b) a queen?\\
\end{enumerate}
\solution
		%\input{ncert/10/15/1/15/defs.tex}
	\item A bag contains $5$ red balls and some blue balls. If the probability of drawing a blue ball is double that if a red ball, determine the number of blue balls in the bag. 
		\\
\solution
		%\input{ncert/10/15/2/3/defs.tex}
	\item A card is selected from a pack of 52 cards.
 \begin{enumerate}[label=(\alph*)] 
                 \item How many points are there in the sample space?
                 \item Calculate the probability that the card is an ace of spades.
                 \item Calculate the probability that the card is (i) an ace and (ii) black card.
 \end{enumerate}
\solution
		%\input{ncert/11/16/3/4/main.tex}
\item Four cards are drawn from a well-shuffled deck of 52 cards. What is the probability of obtaining 3 diamonds and one spade.
\\
\solution
		%\input{ncert/11/16/4/2/defs.tex}
\item In a certain lottery 10,000 tickets are sold and ten equal prizes are awarded. What is the probability of not getting a prize if you buy (a) one ticket (b) two tickets (c) 10 tickets ?	
\\
\solution
		%\input{ncert/11/16/4/4/defs.tex}
		%
\item 
Out of 100 students, two sections of 40 and 60 are formed. If you and your friend are among the 100 students, what is the probability that
\begin{enumerate}
\item you both enter the same section?
\item you both enter the different sections?
\end{enumerate}
\solution
		%\input{ncert/11/16/4/5/defs.tex}
	\item 
The number lock of a suitcase has 4 wheels each labelled with ten digits i.e. from 0 to 9.The lock opens with a sequence of four digits with no repeats.What is the probability of a person getting the right sequence to open the suitcase.
\\
\solution
		%\input{ncert/11/16/4/10/defs.tex}
		%
\item 
Two cards are drawn at random and without replacement from a pack of 52 playing cards. Find the probability that both the cards are black.
\\
\solution
		%\input{ncert/12/13/2/2/defs.tex}
		\item A box of oranges is inspected by examining three randomly selected oranges drawn without replacement. If all the three oranges are good, the box is approved for sale, otherwise, it is rejected. Find the probability that a box containing 15 oranges out of which 12 are good and 3 are bad ones will be approved for sale.
		\label{ncert/12/13/2/3/defs.tex}
		\item Two balls are drawn at random with replacement from a box containing 10 black and 8 red balls. Find the probability that
		\label{ncert/12/13/2/12}
\begin{enumerate}
\item both balls are red.
\item first ball is black and second is red.
\item one of them is black and other is red.
\end{enumerate}

\item In a hostel, 60\% of the students read Hindi newspaper, 40\% read English newspaper and 20\% read both Hindi and English newspapers. A student is selected at random.
		\label{ncert/12/13/2/15}
\begin{enumerate}
\item Find the probability that she reads neither Hindi nor English newspapers.
\item If she reads Hindi newspaper, find the probability that she reads English newspaper.
\item If she reads English newspaper, find the probability that she reads Hindi newspaper.\\
\end{enumerate}
\item The probability of obtaining an even prime number on each die, when a pair of dice is rolled is 
\begin{enumerate}
    \item $0$ 
    
    \item $\frac{1}{3}$ 
    
    \item $\frac{1}{12}$ 
    
    \item $\frac{1}{36}$ 
\end{enumerate}
\solution
		%\input{ncert/12/13/2/17/defs.tex}
	\item A bag contains 4 red and 4 black balls, another bag contains 2 red and 6 black balls. One of the two bags is selected at random and a ball is drawn from the bag which is found to be red. Find the probability that the ball is drawn from the first bag.
\\
\solution
		%\input{ncert/12/13/3/2/main.tex}
  \item
  Cards with numbers 2 to 101 are placed in a box. A card is selected at random.Find the probability that the card has
\begin{enumerate}[label=(\roman*)]
	\item an even number 
	\item a square number
\end{enumerate}
\solution
%\input{exemplar/10/13/3/32/main.tex}
\item
The king, queen and jack of clubs are removed from a deck of 52 playing cards and then well shuffled. Now one card is drawn at random from the remaining cards.  Determine the probability that the card is
\begin{enumerate}[label=(\roman*)]
\item a club
\item 10 of hearts
\end{enumerate}
\solution
%\input{exemplar/10/13/3/29/main.tex}
\item A team of medical students doing their internship have to assist during surgeries
at a city hospital. The probabilities of surgeries rated as very complex, complex,
routine, simple or very simple are respectively, 0.15, 0.20, 0.31, 0.26, .08. Find
the probabilities that a particular surgery will be rated
\begin{enumerate}
	\item complex or very complex;
	\item neither very complex nor very simple;
	\item routine or complex
	\item routine or simple
\end{enumerate}
\solution
%\input{exemplar/11/16/3/8(1)/main.tex}
\item A card is selected from a pack of 52 cards.
\begin{enumerate}[label=(\alph*)]
    \item How many points are there in the sample space?
    \item Calculate the probability that the card is an ace of spades.
    \item Calculate the probability that the card is (i) an ace and (ii) black card.
\end{enumerate}
\solution
%\input{exemplar/11/16/3/4/main2.tex}
\item The probability that a non leap year selected at random will contain 53 sundays.
\\
\solution
%\input{exemplar/10/13/1/19/main.tex}
\item One of the four persons John, Rita, Aslam or Gurpreet will be promoted next
month. Consequently the sample space consists of four elementary outcomes
S = {John promoted, Rita promoted, Aslam promoted, Gurpreet promoted}
You are told that the chances of John’s promotion is same as that of Gurpreet,
Rita’s chances of promotion are twice as likely as Johns. Aslam’s chances are
four times that of John.
\begin{enumerate}
	\item Determine
	\begin{enumerate}
		\item P (John promoted)
		\item P (Rita promoted)
		\item P (Aslam promoted)
		\item P (Gurpreet promoted)
	\end{enumerate}
	\item If A = {John promoted or Gurpreet promoted}, find P (A).
\end{enumerate}
\solution
%\input{exemplar/11/16/3/10/main.tex}
\item A card is drawn from a deck of 52 cards. Find the probability of getting a king or a heart or a red card.\\
\solution
%\input{exemplar/11/16/3/15/main.tex}
\item The probability that a student will pass his examination is 0.73, the probability of
the student getting a compartment is 0.13, and the probability that the student will
either pass or get compartment is 0.96. State True or False.\\
\solution
%\input{exemplar/11/16/3/31/main.tex}
\item A card is selected from a pack of 52 cards\\
\begin{enumerate}[label=(\alph*)]
\item How many points are there in the sample space?
\item Calculate the probability that the cards is an ace of spades.
\item Calculate the probability that the card is (i) an ace (ii)black card.\\
\end{enumerate}
%\input{ncert/11/16/3/4_1/Prob_4.tex}
\item In a non-leap year, the probability of having 53 tuesdays or 53 wednesdays is\\
\solution
%\input{exemplar/11/16/3/18/main.tex}
\item There are 1000 sealed envelopes in a box, 10 of them contain a cash prize of
Rs 100 each, 100 of them contain a cash prize of Rs 50 each and 200 of them
contain a cash prize of Rs 10 each and rest do not contain any cash prize. If they
are well shuffled and an envelope is picked up out, what is the probability that it
contains no cash prize?\\
\solution
%\input{exemplar/10/13/3/34/main.tex}
\item 
A die is thrown and a card is selected at random from a deck of 52 playing cards. The probability of getting an even number on the die and a spade card.\\
\solution
%\input{exemplar/12/13/3/78/main.tex}
\item
If 4-digit numbers greater than 5,000 are randomly formed from the digits 0, 1, 3, 5, and 7, what is the probability of forming a number divisible by 5 when:
\begin{enumerate}
    \item The digits are repeated?
    \item The repetition of digits is not allowed?
\end{enumerate}
\solution
%\input{ncert/11/16/4/9/main.tex}
\item Consider the probability space $\brak{\Omega, \mathcal{G}, P}$ where $\Omega = [0,2]$ and $\mathcal{G} = \cbrak{\phi, \Omega, [0,1], (1,2]}$. Let $X$ and $Y$ be two functions on $\Omega$ defined as
\begin{align*}
    X(\omega) = 
    \begin{cases}
        1 & \text{if }\omega \in [0, 1]\\
        2 & \text{if }\omega \in (1, 2]
    \end{cases}
\end{align*}
and
\begin{align*}
    Y(\omega) = 
    \begin{cases}
        2 & \text{if }\omega \in [0, 1.5]\\
        3 & \text{if }\omega \in (1.5, 2].
    \end{cases}
\end{align*}
Then which one of the following statements is true?
\begin{enumerate}
    \item [(A)] $X$ is a random variable with respect to $\mathcal{G}$, but $Y$ is not a random variable with respect to $\mathcal{G}$.
    \item [(B)] $Y$ is a random variable with respect to $\mathcal{G}$, but $X$ is not a random variable with respect to $\mathcal{G}$.
    \item [(C)] Neither $X$ nor $Y$ is a random variable with respect to $\mathcal{G}$.
    \item [(D)] Both $X$ and $Y$ are random variables with respect to $\mathcal{G}$.
\end{enumerate} \hfill (GATE ST 2023)\\
\solution
%\input{gate/ST/2023/14/main.tex}
	\item  A die is loaded in such a way that each odd number is twice as likely to occur as
each even number. Find $P(G)$, where $G$ is the event that a number greater than
3 occurs on a single roll of the die.
\\
\solution
		%\input{exemplar/11/16/3/5/main.tex}
	\item All the jacks, queens and kings are removed from a deck of 52 playing cards. The remaining cards are well shuffled and then one card is drawn at random. Giving ace a value 1 similar value for other cards, find the probability that the card has a value 
		\begin{enumerate}
			\item 7
			\item greater than 7
			\item less than 7
		\end{enumerate}
		%\input{exemplar/10/13/3/30/main.tex}
  \item A Lot consists of 48 mobile phones of which 42 are good, 3 have only minor defects and 3 have major defects.Varnika will buy a phone if it is good but the trader will only buy a mobile if it has no major defects. One phone is selected at random from the lot. What is the probability that it is
\begin{enumerate}
	\item acceptable to Varnika?
            \item acceptable to the trader?
\end{enumerate}
\solution
	%\input{exemplar/10/13/3/40/main.tex}
 \item A student says that if you throw a die, it will show up 1 or not 1. Therefore, the probability of getting 1 and the probability of getting 'not 1' each is equal to $\frac{1}{2}$. Is this correct? Give reasons.\\
 \solution
        %\input{exemplar/10/13/2/9/main.tex}
   \item Four candidates A, B, C, D have ap-
plied for the assignment to coach a school cricket
team. If A is twice as likely to be selected as B, and
B and C are given about the same chance of being
selected, while C is twice as likely to be selected
as D, what are the probabilities that
\begin{enumerate}
\item C will be selected?
\item A will not be selected?
\end{enumerate}
	%\input{exemplar/11/16/3/9/main.tex}
 \item A bag contain 24 balls of which $x$ balls are red, $2x$ are white and $3x$ are blue. A ball is selected at random, What is the probability that it is
\begin{enumerate}[label=\alph*)]
\item not red ?
\item white ?
\end{enumerate}
%\input{exemplar/10/13/3/41/main.tex}
If the letters of the word ASSASSINATION are arranged at random. Find the Probability that
\begin{enumerate}[label=(\alph*)]
\item Four $S's$ come consecutively in the word
\item Two  $I's$ and two $N's$ come together
\item All $A's$ are not coming together
\item No two $A's$ are coming together
\end{enumerate}
%\input{exemplar/11/16/3/14/main.tex}
	\item One urn contains two black balls (labelled B1 and B2) and one white ball. A
	second urn contains one black ball and two white balls (labelled W1 and W2).
	Suppose the following experiment is performed. One of the two urns is chosen
	at random. Next a ball is randomly chosen from the urn. Then a second ball is
	chosen at random from the same urn without replacing the first ball.
	
	\begin{enumerate}
	\item What is the probability that two black balls are chosen?
	
	\item What is the probability that two balls of opposite colour are chosen?
	\end{enumerate}
	\solution
	%\input{exemplar/11/16/3/12/main1.tex}
\end{enumerate}

	\item A card is selected from a pack of 52 cards.
 \begin{enumerate}[label=(\alph*)] 
                 \item How many points are there in the sample space?
                 \item Calculate the probability that the card is an ace of spades.
                 \item Calculate the probability that the card is (i) an ace and (ii) black card.
 \end{enumerate}
\solution
		%\begin{table}[H]
	\centering
\begin{tabular}{|c|c|c|}
\hline
Random variable &Value &Definition\\ \hline
\multirow{3}{*}{X} &0 &Slips of Rs 1\\
&1 &Slips of Rs 5\\
&2 &Slips of Rs 13\\ \hline
\multirow{2}{*}{Y} &0 &Box A\\
&1 &Box B\\\hline
\end{tabular}
\caption{}
\label{tab:Distribution}
\end{table}
See \tabref{tab:Distribution}.
\begin{align}
p_{Y}\brak{k}= \begin{cases} 
      \frac{1}{3} & {k=0} \\
      \frac{2}{3 }& {k=1} 
   \end{cases}
   \\
p_{Y|X}\brak{0|0} = \frac{19}{25}\, 
p_{Y|X}\brak{0|1} = \frac{6}{25}\,
p_{Y|X}\brak{1|0} = \frac{45}{50}\,
p_{Y|X}\brak{1|2} = \frac{5}{50}
\end{align}
The desired probability is the probability that a slip drawn at random is marked other than Rs 1,
\begin{align}
&=1-p_X\brak{0}\\
&= p_X(1) + p_X(2)
\end{align}
Using Bayes theorem,
\begin{align}
&= p_Y\brak{0} \times \pr{Y=0 | X=1} + p_Y\brak{1} \times \pr{Y=1|X=2}\\
&=\frac{1}{3} \times \frac{6}{25} + \frac{2}{3} \times \frac{5}{50}\\
&=\frac{11}{75}
\end{align}

\newpage

%\tableofcontents

\bigskip

\renewcommand{\thefigure}{\theenumi}
\renewcommand{\thetable}{\theenumi}
%\renewcommand{\theequation}{\theenumi}

%\begin{abstract}
%%\boldmath
%In this letter, an algorithm for evaluating the exact analytical bit error rate  (BER)  for the piecewise linear (PL) combiner for  multiple relays is presented. Previous results were available only for upto three relays. The algorithm is unique in the sense that  the actual mathematical expressions, that are prohibitively large, need not be explicitly obtained. The diversity gain due to multiple relays is shown through plots of the analytical BER, well supported by simulations. 
%
%\end{abstract}
% IEEEtran.cls defaults to using nonbold math in the Abstract.
% This preserves the distinction between vectors and scalars. However,
% if the journal you are submitting to favors bold math in the abstract,
% then you can use LaTeX's standard command \boldmath at the very start
% of the abstract to achieve this. Many IEEE journals frown on math
% in the abstract anyway.

% Note that keywords are not normally used for peerreview papers.
%\begin{IEEEkeywords}
%Cooperative diversity, decode and forward, piecewise linear
%\end{IEEEkeywords}



% For peer review papers, you can put extra information on the cover
% page as needed:
% \ifCLASSOPTIONpeerreview
% \begin{center} \bfseries EDICS Category: 3-BBND \end{center}
% \fi
%
% For peerreview papers, this IEEEtran command inserts a page break and
% creates the second title. It will be ignored for other modes.
%\IEEEpeerreviewmaketitle




\item Four cards are drawn from a well-shuffled deck of 52 cards. What is the probability of obtaining 3 diamonds and one spade.
\\
\solution
		%\begin{enumerate}[label=\thesection.\arabic*,ref=\thesection.\theenumi]
	\item One card is drawn from a well-shuffled deck of 52 cards. Find the probability of getting
\begin{enumerate}
\item A king of red colour 
\item A face card 
\item A red face card
\item The jack of hearts
\item A spade
\item The queen of diamonds

\end{enumerate}
\solution
		%\input{ncert/10/15/1/14/main.tex}
	\item Five cards—the ten, jack, queen, king and ace of diamonds, are well-shuffled with their face downwards. One card is then picked up at random.
\begin{enumerate}
\item
What is the probability that the card is the queen? 
\item
If the queen is drawn and put aside, what is the probability that the second card picked up is (a) an ace? (b) a queen?\\
\end{enumerate}
\solution
		%\input{ncert/10/15/1/15/defs.tex}
	\item A bag contains $5$ red balls and some blue balls. If the probability of drawing a blue ball is double that if a red ball, determine the number of blue balls in the bag. 
		\\
\solution
		%\input{ncert/10/15/2/3/defs.tex}
	\item A card is selected from a pack of 52 cards.
 \begin{enumerate}[label=(\alph*)] 
                 \item How many points are there in the sample space?
                 \item Calculate the probability that the card is an ace of spades.
                 \item Calculate the probability that the card is (i) an ace and (ii) black card.
 \end{enumerate}
\solution
		%\input{ncert/11/16/3/4/main.tex}
\item Four cards are drawn from a well-shuffled deck of 52 cards. What is the probability of obtaining 3 diamonds and one spade.
\\
\solution
		%\input{ncert/11/16/4/2/defs.tex}
\item In a certain lottery 10,000 tickets are sold and ten equal prizes are awarded. What is the probability of not getting a prize if you buy (a) one ticket (b) two tickets (c) 10 tickets ?	
\\
\solution
		%\input{ncert/11/16/4/4/defs.tex}
		%
\item 
Out of 100 students, two sections of 40 and 60 are formed. If you and your friend are among the 100 students, what is the probability that
\begin{enumerate}
\item you both enter the same section?
\item you both enter the different sections?
\end{enumerate}
\solution
		%\input{ncert/11/16/4/5/defs.tex}
	\item 
The number lock of a suitcase has 4 wheels each labelled with ten digits i.e. from 0 to 9.The lock opens with a sequence of four digits with no repeats.What is the probability of a person getting the right sequence to open the suitcase.
\\
\solution
		%\input{ncert/11/16/4/10/defs.tex}
		%
\item 
Two cards are drawn at random and without replacement from a pack of 52 playing cards. Find the probability that both the cards are black.
\\
\solution
		%\input{ncert/12/13/2/2/defs.tex}
		\item A box of oranges is inspected by examining three randomly selected oranges drawn without replacement. If all the three oranges are good, the box is approved for sale, otherwise, it is rejected. Find the probability that a box containing 15 oranges out of which 12 are good and 3 are bad ones will be approved for sale.
		\label{ncert/12/13/2/3/defs.tex}
		\item Two balls are drawn at random with replacement from a box containing 10 black and 8 red balls. Find the probability that
		\label{ncert/12/13/2/12}
\begin{enumerate}
\item both balls are red.
\item first ball is black and second is red.
\item one of them is black and other is red.
\end{enumerate}

\item In a hostel, 60\% of the students read Hindi newspaper, 40\% read English newspaper and 20\% read both Hindi and English newspapers. A student is selected at random.
		\label{ncert/12/13/2/15}
\begin{enumerate}
\item Find the probability that she reads neither Hindi nor English newspapers.
\item If she reads Hindi newspaper, find the probability that she reads English newspaper.
\item If she reads English newspaper, find the probability that she reads Hindi newspaper.\\
\end{enumerate}
\item The probability of obtaining an even prime number on each die, when a pair of dice is rolled is 
\begin{enumerate}
    \item $0$ 
    
    \item $\frac{1}{3}$ 
    
    \item $\frac{1}{12}$ 
    
    \item $\frac{1}{36}$ 
\end{enumerate}
\solution
		%\input{ncert/12/13/2/17/defs.tex}
	\item A bag contains 4 red and 4 black balls, another bag contains 2 red and 6 black balls. One of the two bags is selected at random and a ball is drawn from the bag which is found to be red. Find the probability that the ball is drawn from the first bag.
\\
\solution
		%\input{ncert/12/13/3/2/main.tex}
  \item
  Cards with numbers 2 to 101 are placed in a box. A card is selected at random.Find the probability that the card has
\begin{enumerate}[label=(\roman*)]
	\item an even number 
	\item a square number
\end{enumerate}
\solution
%\input{exemplar/10/13/3/32/main.tex}
\item
The king, queen and jack of clubs are removed from a deck of 52 playing cards and then well shuffled. Now one card is drawn at random from the remaining cards.  Determine the probability that the card is
\begin{enumerate}[label=(\roman*)]
\item a club
\item 10 of hearts
\end{enumerate}
\solution
%\input{exemplar/10/13/3/29/main.tex}
\item A team of medical students doing their internship have to assist during surgeries
at a city hospital. The probabilities of surgeries rated as very complex, complex,
routine, simple or very simple are respectively, 0.15, 0.20, 0.31, 0.26, .08. Find
the probabilities that a particular surgery will be rated
\begin{enumerate}
	\item complex or very complex;
	\item neither very complex nor very simple;
	\item routine or complex
	\item routine or simple
\end{enumerate}
\solution
%\input{exemplar/11/16/3/8(1)/main.tex}
\item A card is selected from a pack of 52 cards.
\begin{enumerate}[label=(\alph*)]
    \item How many points are there in the sample space?
    \item Calculate the probability that the card is an ace of spades.
    \item Calculate the probability that the card is (i) an ace and (ii) black card.
\end{enumerate}
\solution
%\input{exemplar/11/16/3/4/main2.tex}
\item The probability that a non leap year selected at random will contain 53 sundays.
\\
\solution
%\input{exemplar/10/13/1/19/main.tex}
\item One of the four persons John, Rita, Aslam or Gurpreet will be promoted next
month. Consequently the sample space consists of four elementary outcomes
S = {John promoted, Rita promoted, Aslam promoted, Gurpreet promoted}
You are told that the chances of John’s promotion is same as that of Gurpreet,
Rita’s chances of promotion are twice as likely as Johns. Aslam’s chances are
four times that of John.
\begin{enumerate}
	\item Determine
	\begin{enumerate}
		\item P (John promoted)
		\item P (Rita promoted)
		\item P (Aslam promoted)
		\item P (Gurpreet promoted)
	\end{enumerate}
	\item If A = {John promoted or Gurpreet promoted}, find P (A).
\end{enumerate}
\solution
%\input{exemplar/11/16/3/10/main.tex}
\item A card is drawn from a deck of 52 cards. Find the probability of getting a king or a heart or a red card.\\
\solution
%\input{exemplar/11/16/3/15/main.tex}
\item The probability that a student will pass his examination is 0.73, the probability of
the student getting a compartment is 0.13, and the probability that the student will
either pass or get compartment is 0.96. State True or False.\\
\solution
%\input{exemplar/11/16/3/31/main.tex}
\item A card is selected from a pack of 52 cards\\
\begin{enumerate}[label=(\alph*)]
\item How many points are there in the sample space?
\item Calculate the probability that the cards is an ace of spades.
\item Calculate the probability that the card is (i) an ace (ii)black card.\\
\end{enumerate}
%\input{ncert/11/16/3/4_1/Prob_4.tex}
\item In a non-leap year, the probability of having 53 tuesdays or 53 wednesdays is\\
\solution
%\input{exemplar/11/16/3/18/main.tex}
\item There are 1000 sealed envelopes in a box, 10 of them contain a cash prize of
Rs 100 each, 100 of them contain a cash prize of Rs 50 each and 200 of them
contain a cash prize of Rs 10 each and rest do not contain any cash prize. If they
are well shuffled and an envelope is picked up out, what is the probability that it
contains no cash prize?\\
\solution
%\input{exemplar/10/13/3/34/main.tex}
\item 
A die is thrown and a card is selected at random from a deck of 52 playing cards. The probability of getting an even number on the die and a spade card.\\
\solution
%\input{exemplar/12/13/3/78/main.tex}
\item
If 4-digit numbers greater than 5,000 are randomly formed from the digits 0, 1, 3, 5, and 7, what is the probability of forming a number divisible by 5 when:
\begin{enumerate}
    \item The digits are repeated?
    \item The repetition of digits is not allowed?
\end{enumerate}
\solution
%\input{ncert/11/16/4/9/main.tex}
\item Consider the probability space $\brak{\Omega, \mathcal{G}, P}$ where $\Omega = [0,2]$ and $\mathcal{G} = \cbrak{\phi, \Omega, [0,1], (1,2]}$. Let $X$ and $Y$ be two functions on $\Omega$ defined as
\begin{align*}
    X(\omega) = 
    \begin{cases}
        1 & \text{if }\omega \in [0, 1]\\
        2 & \text{if }\omega \in (1, 2]
    \end{cases}
\end{align*}
and
\begin{align*}
    Y(\omega) = 
    \begin{cases}
        2 & \text{if }\omega \in [0, 1.5]\\
        3 & \text{if }\omega \in (1.5, 2].
    \end{cases}
\end{align*}
Then which one of the following statements is true?
\begin{enumerate}
    \item [(A)] $X$ is a random variable with respect to $\mathcal{G}$, but $Y$ is not a random variable with respect to $\mathcal{G}$.
    \item [(B)] $Y$ is a random variable with respect to $\mathcal{G}$, but $X$ is not a random variable with respect to $\mathcal{G}$.
    \item [(C)] Neither $X$ nor $Y$ is a random variable with respect to $\mathcal{G}$.
    \item [(D)] Both $X$ and $Y$ are random variables with respect to $\mathcal{G}$.
\end{enumerate} \hfill (GATE ST 2023)\\
\solution
%\input{gate/ST/2023/14/main.tex}
	\item  A die is loaded in such a way that each odd number is twice as likely to occur as
each even number. Find $P(G)$, where $G$ is the event that a number greater than
3 occurs on a single roll of the die.
\\
\solution
		%\input{exemplar/11/16/3/5/main.tex}
	\item All the jacks, queens and kings are removed from a deck of 52 playing cards. The remaining cards are well shuffled and then one card is drawn at random. Giving ace a value 1 similar value for other cards, find the probability that the card has a value 
		\begin{enumerate}
			\item 7
			\item greater than 7
			\item less than 7
		\end{enumerate}
		%\input{exemplar/10/13/3/30/main.tex}
  \item A Lot consists of 48 mobile phones of which 42 are good, 3 have only minor defects and 3 have major defects.Varnika will buy a phone if it is good but the trader will only buy a mobile if it has no major defects. One phone is selected at random from the lot. What is the probability that it is
\begin{enumerate}
	\item acceptable to Varnika?
            \item acceptable to the trader?
\end{enumerate}
\solution
	%\input{exemplar/10/13/3/40/main.tex}
 \item A student says that if you throw a die, it will show up 1 or not 1. Therefore, the probability of getting 1 and the probability of getting 'not 1' each is equal to $\frac{1}{2}$. Is this correct? Give reasons.\\
 \solution
        %\input{exemplar/10/13/2/9/main.tex}
   \item Four candidates A, B, C, D have ap-
plied for the assignment to coach a school cricket
team. If A is twice as likely to be selected as B, and
B and C are given about the same chance of being
selected, while C is twice as likely to be selected
as D, what are the probabilities that
\begin{enumerate}
\item C will be selected?
\item A will not be selected?
\end{enumerate}
	%\input{exemplar/11/16/3/9/main.tex}
 \item A bag contain 24 balls of which $x$ balls are red, $2x$ are white and $3x$ are blue. A ball is selected at random, What is the probability that it is
\begin{enumerate}[label=\alph*)]
\item not red ?
\item white ?
\end{enumerate}
%\input{exemplar/10/13/3/41/main.tex}
If the letters of the word ASSASSINATION are arranged at random. Find the Probability that
\begin{enumerate}[label=(\alph*)]
\item Four $S's$ come consecutively in the word
\item Two  $I's$ and two $N's$ come together
\item All $A's$ are not coming together
\item No two $A's$ are coming together
\end{enumerate}
%\input{exemplar/11/16/3/14/main.tex}
	\item One urn contains two black balls (labelled B1 and B2) and one white ball. A
	second urn contains one black ball and two white balls (labelled W1 and W2).
	Suppose the following experiment is performed. One of the two urns is chosen
	at random. Next a ball is randomly chosen from the urn. Then a second ball is
	chosen at random from the same urn without replacing the first ball.
	
	\begin{enumerate}
	\item What is the probability that two black balls are chosen?
	
	\item What is the probability that two balls of opposite colour are chosen?
	\end{enumerate}
	\solution
	%\input{exemplar/11/16/3/12/main1.tex}
\end{enumerate}

\item In a certain lottery 10,000 tickets are sold and ten equal prizes are awarded. What is the probability of not getting a prize if you buy (a) one ticket (b) two tickets (c) 10 tickets ?	
\\
\solution
		%\begin{enumerate}[label=\thesection.\arabic*,ref=\thesection.\theenumi]
	\item One card is drawn from a well-shuffled deck of 52 cards. Find the probability of getting
\begin{enumerate}
\item A king of red colour 
\item A face card 
\item A red face card
\item The jack of hearts
\item A spade
\item The queen of diamonds

\end{enumerate}
\solution
		%\input{ncert/10/15/1/14/main.tex}
	\item Five cards—the ten, jack, queen, king and ace of diamonds, are well-shuffled with their face downwards. One card is then picked up at random.
\begin{enumerate}
\item
What is the probability that the card is the queen? 
\item
If the queen is drawn and put aside, what is the probability that the second card picked up is (a) an ace? (b) a queen?\\
\end{enumerate}
\solution
		%\input{ncert/10/15/1/15/defs.tex}
	\item A bag contains $5$ red balls and some blue balls. If the probability of drawing a blue ball is double that if a red ball, determine the number of blue balls in the bag. 
		\\
\solution
		%\input{ncert/10/15/2/3/defs.tex}
	\item A card is selected from a pack of 52 cards.
 \begin{enumerate}[label=(\alph*)] 
                 \item How many points are there in the sample space?
                 \item Calculate the probability that the card is an ace of spades.
                 \item Calculate the probability that the card is (i) an ace and (ii) black card.
 \end{enumerate}
\solution
		%\input{ncert/11/16/3/4/main.tex}
\item Four cards are drawn from a well-shuffled deck of 52 cards. What is the probability of obtaining 3 diamonds and one spade.
\\
\solution
		%\input{ncert/11/16/4/2/defs.tex}
\item In a certain lottery 10,000 tickets are sold and ten equal prizes are awarded. What is the probability of not getting a prize if you buy (a) one ticket (b) two tickets (c) 10 tickets ?	
\\
\solution
		%\input{ncert/11/16/4/4/defs.tex}
		%
\item 
Out of 100 students, two sections of 40 and 60 are formed. If you and your friend are among the 100 students, what is the probability that
\begin{enumerate}
\item you both enter the same section?
\item you both enter the different sections?
\end{enumerate}
\solution
		%\input{ncert/11/16/4/5/defs.tex}
	\item 
The number lock of a suitcase has 4 wheels each labelled with ten digits i.e. from 0 to 9.The lock opens with a sequence of four digits with no repeats.What is the probability of a person getting the right sequence to open the suitcase.
\\
\solution
		%\input{ncert/11/16/4/10/defs.tex}
		%
\item 
Two cards are drawn at random and without replacement from a pack of 52 playing cards. Find the probability that both the cards are black.
\\
\solution
		%\input{ncert/12/13/2/2/defs.tex}
		\item A box of oranges is inspected by examining three randomly selected oranges drawn without replacement. If all the three oranges are good, the box is approved for sale, otherwise, it is rejected. Find the probability that a box containing 15 oranges out of which 12 are good and 3 are bad ones will be approved for sale.
		\label{ncert/12/13/2/3/defs.tex}
		\item Two balls are drawn at random with replacement from a box containing 10 black and 8 red balls. Find the probability that
		\label{ncert/12/13/2/12}
\begin{enumerate}
\item both balls are red.
\item first ball is black and second is red.
\item one of them is black and other is red.
\end{enumerate}

\item In a hostel, 60\% of the students read Hindi newspaper, 40\% read English newspaper and 20\% read both Hindi and English newspapers. A student is selected at random.
		\label{ncert/12/13/2/15}
\begin{enumerate}
\item Find the probability that she reads neither Hindi nor English newspapers.
\item If she reads Hindi newspaper, find the probability that she reads English newspaper.
\item If she reads English newspaper, find the probability that she reads Hindi newspaper.\\
\end{enumerate}
\item The probability of obtaining an even prime number on each die, when a pair of dice is rolled is 
\begin{enumerate}
    \item $0$ 
    
    \item $\frac{1}{3}$ 
    
    \item $\frac{1}{12}$ 
    
    \item $\frac{1}{36}$ 
\end{enumerate}
\solution
		%\input{ncert/12/13/2/17/defs.tex}
	\item A bag contains 4 red and 4 black balls, another bag contains 2 red and 6 black balls. One of the two bags is selected at random and a ball is drawn from the bag which is found to be red. Find the probability that the ball is drawn from the first bag.
\\
\solution
		%\input{ncert/12/13/3/2/main.tex}
  \item
  Cards with numbers 2 to 101 are placed in a box. A card is selected at random.Find the probability that the card has
\begin{enumerate}[label=(\roman*)]
	\item an even number 
	\item a square number
\end{enumerate}
\solution
%\input{exemplar/10/13/3/32/main.tex}
\item
The king, queen and jack of clubs are removed from a deck of 52 playing cards and then well shuffled. Now one card is drawn at random from the remaining cards.  Determine the probability that the card is
\begin{enumerate}[label=(\roman*)]
\item a club
\item 10 of hearts
\end{enumerate}
\solution
%\input{exemplar/10/13/3/29/main.tex}
\item A team of medical students doing their internship have to assist during surgeries
at a city hospital. The probabilities of surgeries rated as very complex, complex,
routine, simple or very simple are respectively, 0.15, 0.20, 0.31, 0.26, .08. Find
the probabilities that a particular surgery will be rated
\begin{enumerate}
	\item complex or very complex;
	\item neither very complex nor very simple;
	\item routine or complex
	\item routine or simple
\end{enumerate}
\solution
%\input{exemplar/11/16/3/8(1)/main.tex}
\item A card is selected from a pack of 52 cards.
\begin{enumerate}[label=(\alph*)]
    \item How many points are there in the sample space?
    \item Calculate the probability that the card is an ace of spades.
    \item Calculate the probability that the card is (i) an ace and (ii) black card.
\end{enumerate}
\solution
%\input{exemplar/11/16/3/4/main2.tex}
\item The probability that a non leap year selected at random will contain 53 sundays.
\\
\solution
%\input{exemplar/10/13/1/19/main.tex}
\item One of the four persons John, Rita, Aslam or Gurpreet will be promoted next
month. Consequently the sample space consists of four elementary outcomes
S = {John promoted, Rita promoted, Aslam promoted, Gurpreet promoted}
You are told that the chances of John’s promotion is same as that of Gurpreet,
Rita’s chances of promotion are twice as likely as Johns. Aslam’s chances are
four times that of John.
\begin{enumerate}
	\item Determine
	\begin{enumerate}
		\item P (John promoted)
		\item P (Rita promoted)
		\item P (Aslam promoted)
		\item P (Gurpreet promoted)
	\end{enumerate}
	\item If A = {John promoted or Gurpreet promoted}, find P (A).
\end{enumerate}
\solution
%\input{exemplar/11/16/3/10/main.tex}
\item A card is drawn from a deck of 52 cards. Find the probability of getting a king or a heart or a red card.\\
\solution
%\input{exemplar/11/16/3/15/main.tex}
\item The probability that a student will pass his examination is 0.73, the probability of
the student getting a compartment is 0.13, and the probability that the student will
either pass or get compartment is 0.96. State True or False.\\
\solution
%\input{exemplar/11/16/3/31/main.tex}
\item A card is selected from a pack of 52 cards\\
\begin{enumerate}[label=(\alph*)]
\item How many points are there in the sample space?
\item Calculate the probability that the cards is an ace of spades.
\item Calculate the probability that the card is (i) an ace (ii)black card.\\
\end{enumerate}
%\input{ncert/11/16/3/4_1/Prob_4.tex}
\item In a non-leap year, the probability of having 53 tuesdays or 53 wednesdays is\\
\solution
%\input{exemplar/11/16/3/18/main.tex}
\item There are 1000 sealed envelopes in a box, 10 of them contain a cash prize of
Rs 100 each, 100 of them contain a cash prize of Rs 50 each and 200 of them
contain a cash prize of Rs 10 each and rest do not contain any cash prize. If they
are well shuffled and an envelope is picked up out, what is the probability that it
contains no cash prize?\\
\solution
%\input{exemplar/10/13/3/34/main.tex}
\item 
A die is thrown and a card is selected at random from a deck of 52 playing cards. The probability of getting an even number on the die and a spade card.\\
\solution
%\input{exemplar/12/13/3/78/main.tex}
\item
If 4-digit numbers greater than 5,000 are randomly formed from the digits 0, 1, 3, 5, and 7, what is the probability of forming a number divisible by 5 when:
\begin{enumerate}
    \item The digits are repeated?
    \item The repetition of digits is not allowed?
\end{enumerate}
\solution
%\input{ncert/11/16/4/9/main.tex}
\item Consider the probability space $\brak{\Omega, \mathcal{G}, P}$ where $\Omega = [0,2]$ and $\mathcal{G} = \cbrak{\phi, \Omega, [0,1], (1,2]}$. Let $X$ and $Y$ be two functions on $\Omega$ defined as
\begin{align*}
    X(\omega) = 
    \begin{cases}
        1 & \text{if }\omega \in [0, 1]\\
        2 & \text{if }\omega \in (1, 2]
    \end{cases}
\end{align*}
and
\begin{align*}
    Y(\omega) = 
    \begin{cases}
        2 & \text{if }\omega \in [0, 1.5]\\
        3 & \text{if }\omega \in (1.5, 2].
    \end{cases}
\end{align*}
Then which one of the following statements is true?
\begin{enumerate}
    \item [(A)] $X$ is a random variable with respect to $\mathcal{G}$, but $Y$ is not a random variable with respect to $\mathcal{G}$.
    \item [(B)] $Y$ is a random variable with respect to $\mathcal{G}$, but $X$ is not a random variable with respect to $\mathcal{G}$.
    \item [(C)] Neither $X$ nor $Y$ is a random variable with respect to $\mathcal{G}$.
    \item [(D)] Both $X$ and $Y$ are random variables with respect to $\mathcal{G}$.
\end{enumerate} \hfill (GATE ST 2023)\\
\solution
%\input{gate/ST/2023/14/main.tex}
	\item  A die is loaded in such a way that each odd number is twice as likely to occur as
each even number. Find $P(G)$, where $G$ is the event that a number greater than
3 occurs on a single roll of the die.
\\
\solution
		%\input{exemplar/11/16/3/5/main.tex}
	\item All the jacks, queens and kings are removed from a deck of 52 playing cards. The remaining cards are well shuffled and then one card is drawn at random. Giving ace a value 1 similar value for other cards, find the probability that the card has a value 
		\begin{enumerate}
			\item 7
			\item greater than 7
			\item less than 7
		\end{enumerate}
		%\input{exemplar/10/13/3/30/main.tex}
  \item A Lot consists of 48 mobile phones of which 42 are good, 3 have only minor defects and 3 have major defects.Varnika will buy a phone if it is good but the trader will only buy a mobile if it has no major defects. One phone is selected at random from the lot. What is the probability that it is
\begin{enumerate}
	\item acceptable to Varnika?
            \item acceptable to the trader?
\end{enumerate}
\solution
	%\input{exemplar/10/13/3/40/main.tex}
 \item A student says that if you throw a die, it will show up 1 or not 1. Therefore, the probability of getting 1 and the probability of getting 'not 1' each is equal to $\frac{1}{2}$. Is this correct? Give reasons.\\
 \solution
        %\input{exemplar/10/13/2/9/main.tex}
   \item Four candidates A, B, C, D have ap-
plied for the assignment to coach a school cricket
team. If A is twice as likely to be selected as B, and
B and C are given about the same chance of being
selected, while C is twice as likely to be selected
as D, what are the probabilities that
\begin{enumerate}
\item C will be selected?
\item A will not be selected?
\end{enumerate}
	%\input{exemplar/11/16/3/9/main.tex}
 \item A bag contain 24 balls of which $x$ balls are red, $2x$ are white and $3x$ are blue. A ball is selected at random, What is the probability that it is
\begin{enumerate}[label=\alph*)]
\item not red ?
\item white ?
\end{enumerate}
%\input{exemplar/10/13/3/41/main.tex}
If the letters of the word ASSASSINATION are arranged at random. Find the Probability that
\begin{enumerate}[label=(\alph*)]
\item Four $S's$ come consecutively in the word
\item Two  $I's$ and two $N's$ come together
\item All $A's$ are not coming together
\item No two $A's$ are coming together
\end{enumerate}
%\input{exemplar/11/16/3/14/main.tex}
	\item One urn contains two black balls (labelled B1 and B2) and one white ball. A
	second urn contains one black ball and two white balls (labelled W1 and W2).
	Suppose the following experiment is performed. One of the two urns is chosen
	at random. Next a ball is randomly chosen from the urn. Then a second ball is
	chosen at random from the same urn without replacing the first ball.
	
	\begin{enumerate}
	\item What is the probability that two black balls are chosen?
	
	\item What is the probability that two balls of opposite colour are chosen?
	\end{enumerate}
	\solution
	%\input{exemplar/11/16/3/12/main1.tex}
\end{enumerate}

		%
\item 
Out of 100 students, two sections of 40 and 60 are formed. If you and your friend are among the 100 students, what is the probability that
\begin{enumerate}
\item you both enter the same section?
\item you both enter the different sections?
\end{enumerate}
\solution
		%\begin{enumerate}[label=\thesection.\arabic*,ref=\thesection.\theenumi]
	\item One card is drawn from a well-shuffled deck of 52 cards. Find the probability of getting
\begin{enumerate}
\item A king of red colour 
\item A face card 
\item A red face card
\item The jack of hearts
\item A spade
\item The queen of diamonds

\end{enumerate}
\solution
		%\input{ncert/10/15/1/14/main.tex}
	\item Five cards—the ten, jack, queen, king and ace of diamonds, are well-shuffled with their face downwards. One card is then picked up at random.
\begin{enumerate}
\item
What is the probability that the card is the queen? 
\item
If the queen is drawn and put aside, what is the probability that the second card picked up is (a) an ace? (b) a queen?\\
\end{enumerate}
\solution
		%\input{ncert/10/15/1/15/defs.tex}
	\item A bag contains $5$ red balls and some blue balls. If the probability of drawing a blue ball is double that if a red ball, determine the number of blue balls in the bag. 
		\\
\solution
		%\input{ncert/10/15/2/3/defs.tex}
	\item A card is selected from a pack of 52 cards.
 \begin{enumerate}[label=(\alph*)] 
                 \item How many points are there in the sample space?
                 \item Calculate the probability that the card is an ace of spades.
                 \item Calculate the probability that the card is (i) an ace and (ii) black card.
 \end{enumerate}
\solution
		%\input{ncert/11/16/3/4/main.tex}
\item Four cards are drawn from a well-shuffled deck of 52 cards. What is the probability of obtaining 3 diamonds and one spade.
\\
\solution
		%\input{ncert/11/16/4/2/defs.tex}
\item In a certain lottery 10,000 tickets are sold and ten equal prizes are awarded. What is the probability of not getting a prize if you buy (a) one ticket (b) two tickets (c) 10 tickets ?	
\\
\solution
		%\input{ncert/11/16/4/4/defs.tex}
		%
\item 
Out of 100 students, two sections of 40 and 60 are formed. If you and your friend are among the 100 students, what is the probability that
\begin{enumerate}
\item you both enter the same section?
\item you both enter the different sections?
\end{enumerate}
\solution
		%\input{ncert/11/16/4/5/defs.tex}
	\item 
The number lock of a suitcase has 4 wheels each labelled with ten digits i.e. from 0 to 9.The lock opens with a sequence of four digits with no repeats.What is the probability of a person getting the right sequence to open the suitcase.
\\
\solution
		%\input{ncert/11/16/4/10/defs.tex}
		%
\item 
Two cards are drawn at random and without replacement from a pack of 52 playing cards. Find the probability that both the cards are black.
\\
\solution
		%\input{ncert/12/13/2/2/defs.tex}
		\item A box of oranges is inspected by examining three randomly selected oranges drawn without replacement. If all the three oranges are good, the box is approved for sale, otherwise, it is rejected. Find the probability that a box containing 15 oranges out of which 12 are good and 3 are bad ones will be approved for sale.
		\label{ncert/12/13/2/3/defs.tex}
		\item Two balls are drawn at random with replacement from a box containing 10 black and 8 red balls. Find the probability that
		\label{ncert/12/13/2/12}
\begin{enumerate}
\item both balls are red.
\item first ball is black and second is red.
\item one of them is black and other is red.
\end{enumerate}

\item In a hostel, 60\% of the students read Hindi newspaper, 40\% read English newspaper and 20\% read both Hindi and English newspapers. A student is selected at random.
		\label{ncert/12/13/2/15}
\begin{enumerate}
\item Find the probability that she reads neither Hindi nor English newspapers.
\item If she reads Hindi newspaper, find the probability that she reads English newspaper.
\item If she reads English newspaper, find the probability that she reads Hindi newspaper.\\
\end{enumerate}
\item The probability of obtaining an even prime number on each die, when a pair of dice is rolled is 
\begin{enumerate}
    \item $0$ 
    
    \item $\frac{1}{3}$ 
    
    \item $\frac{1}{12}$ 
    
    \item $\frac{1}{36}$ 
\end{enumerate}
\solution
		%\input{ncert/12/13/2/17/defs.tex}
	\item A bag contains 4 red and 4 black balls, another bag contains 2 red and 6 black balls. One of the two bags is selected at random and a ball is drawn from the bag which is found to be red. Find the probability that the ball is drawn from the first bag.
\\
\solution
		%\input{ncert/12/13/3/2/main.tex}
  \item
  Cards with numbers 2 to 101 are placed in a box. A card is selected at random.Find the probability that the card has
\begin{enumerate}[label=(\roman*)]
	\item an even number 
	\item a square number
\end{enumerate}
\solution
%\input{exemplar/10/13/3/32/main.tex}
\item
The king, queen and jack of clubs are removed from a deck of 52 playing cards and then well shuffled. Now one card is drawn at random from the remaining cards.  Determine the probability that the card is
\begin{enumerate}[label=(\roman*)]
\item a club
\item 10 of hearts
\end{enumerate}
\solution
%\input{exemplar/10/13/3/29/main.tex}
\item A team of medical students doing their internship have to assist during surgeries
at a city hospital. The probabilities of surgeries rated as very complex, complex,
routine, simple or very simple are respectively, 0.15, 0.20, 0.31, 0.26, .08. Find
the probabilities that a particular surgery will be rated
\begin{enumerate}
	\item complex or very complex;
	\item neither very complex nor very simple;
	\item routine or complex
	\item routine or simple
\end{enumerate}
\solution
%\input{exemplar/11/16/3/8(1)/main.tex}
\item A card is selected from a pack of 52 cards.
\begin{enumerate}[label=(\alph*)]
    \item How many points are there in the sample space?
    \item Calculate the probability that the card is an ace of spades.
    \item Calculate the probability that the card is (i) an ace and (ii) black card.
\end{enumerate}
\solution
%\input{exemplar/11/16/3/4/main2.tex}
\item The probability that a non leap year selected at random will contain 53 sundays.
\\
\solution
%\input{exemplar/10/13/1/19/main.tex}
\item One of the four persons John, Rita, Aslam or Gurpreet will be promoted next
month. Consequently the sample space consists of four elementary outcomes
S = {John promoted, Rita promoted, Aslam promoted, Gurpreet promoted}
You are told that the chances of John’s promotion is same as that of Gurpreet,
Rita’s chances of promotion are twice as likely as Johns. Aslam’s chances are
four times that of John.
\begin{enumerate}
	\item Determine
	\begin{enumerate}
		\item P (John promoted)
		\item P (Rita promoted)
		\item P (Aslam promoted)
		\item P (Gurpreet promoted)
	\end{enumerate}
	\item If A = {John promoted or Gurpreet promoted}, find P (A).
\end{enumerate}
\solution
%\input{exemplar/11/16/3/10/main.tex}
\item A card is drawn from a deck of 52 cards. Find the probability of getting a king or a heart or a red card.\\
\solution
%\input{exemplar/11/16/3/15/main.tex}
\item The probability that a student will pass his examination is 0.73, the probability of
the student getting a compartment is 0.13, and the probability that the student will
either pass or get compartment is 0.96. State True or False.\\
\solution
%\input{exemplar/11/16/3/31/main.tex}
\item A card is selected from a pack of 52 cards\\
\begin{enumerate}[label=(\alph*)]
\item How many points are there in the sample space?
\item Calculate the probability that the cards is an ace of spades.
\item Calculate the probability that the card is (i) an ace (ii)black card.\\
\end{enumerate}
%\input{ncert/11/16/3/4_1/Prob_4.tex}
\item In a non-leap year, the probability of having 53 tuesdays or 53 wednesdays is\\
\solution
%\input{exemplar/11/16/3/18/main.tex}
\item There are 1000 sealed envelopes in a box, 10 of them contain a cash prize of
Rs 100 each, 100 of them contain a cash prize of Rs 50 each and 200 of them
contain a cash prize of Rs 10 each and rest do not contain any cash prize. If they
are well shuffled and an envelope is picked up out, what is the probability that it
contains no cash prize?\\
\solution
%\input{exemplar/10/13/3/34/main.tex}
\item 
A die is thrown and a card is selected at random from a deck of 52 playing cards. The probability of getting an even number on the die and a spade card.\\
\solution
%\input{exemplar/12/13/3/78/main.tex}
\item
If 4-digit numbers greater than 5,000 are randomly formed from the digits 0, 1, 3, 5, and 7, what is the probability of forming a number divisible by 5 when:
\begin{enumerate}
    \item The digits are repeated?
    \item The repetition of digits is not allowed?
\end{enumerate}
\solution
%\input{ncert/11/16/4/9/main.tex}
\item Consider the probability space $\brak{\Omega, \mathcal{G}, P}$ where $\Omega = [0,2]$ and $\mathcal{G} = \cbrak{\phi, \Omega, [0,1], (1,2]}$. Let $X$ and $Y$ be two functions on $\Omega$ defined as
\begin{align*}
    X(\omega) = 
    \begin{cases}
        1 & \text{if }\omega \in [0, 1]\\
        2 & \text{if }\omega \in (1, 2]
    \end{cases}
\end{align*}
and
\begin{align*}
    Y(\omega) = 
    \begin{cases}
        2 & \text{if }\omega \in [0, 1.5]\\
        3 & \text{if }\omega \in (1.5, 2].
    \end{cases}
\end{align*}
Then which one of the following statements is true?
\begin{enumerate}
    \item [(A)] $X$ is a random variable with respect to $\mathcal{G}$, but $Y$ is not a random variable with respect to $\mathcal{G}$.
    \item [(B)] $Y$ is a random variable with respect to $\mathcal{G}$, but $X$ is not a random variable with respect to $\mathcal{G}$.
    \item [(C)] Neither $X$ nor $Y$ is a random variable with respect to $\mathcal{G}$.
    \item [(D)] Both $X$ and $Y$ are random variables with respect to $\mathcal{G}$.
\end{enumerate} \hfill (GATE ST 2023)\\
\solution
%\input{gate/ST/2023/14/main.tex}
	\item  A die is loaded in such a way that each odd number is twice as likely to occur as
each even number. Find $P(G)$, where $G$ is the event that a number greater than
3 occurs on a single roll of the die.
\\
\solution
		%\input{exemplar/11/16/3/5/main.tex}
	\item All the jacks, queens and kings are removed from a deck of 52 playing cards. The remaining cards are well shuffled and then one card is drawn at random. Giving ace a value 1 similar value for other cards, find the probability that the card has a value 
		\begin{enumerate}
			\item 7
			\item greater than 7
			\item less than 7
		\end{enumerate}
		%\input{exemplar/10/13/3/30/main.tex}
  \item A Lot consists of 48 mobile phones of which 42 are good, 3 have only minor defects and 3 have major defects.Varnika will buy a phone if it is good but the trader will only buy a mobile if it has no major defects. One phone is selected at random from the lot. What is the probability that it is
\begin{enumerate}
	\item acceptable to Varnika?
            \item acceptable to the trader?
\end{enumerate}
\solution
	%\input{exemplar/10/13/3/40/main.tex}
 \item A student says that if you throw a die, it will show up 1 or not 1. Therefore, the probability of getting 1 and the probability of getting 'not 1' each is equal to $\frac{1}{2}$. Is this correct? Give reasons.\\
 \solution
        %\input{exemplar/10/13/2/9/main.tex}
   \item Four candidates A, B, C, D have ap-
plied for the assignment to coach a school cricket
team. If A is twice as likely to be selected as B, and
B and C are given about the same chance of being
selected, while C is twice as likely to be selected
as D, what are the probabilities that
\begin{enumerate}
\item C will be selected?
\item A will not be selected?
\end{enumerate}
	%\input{exemplar/11/16/3/9/main.tex}
 \item A bag contain 24 balls of which $x$ balls are red, $2x$ are white and $3x$ are blue. A ball is selected at random, What is the probability that it is
\begin{enumerate}[label=\alph*)]
\item not red ?
\item white ?
\end{enumerate}
%\input{exemplar/10/13/3/41/main.tex}
If the letters of the word ASSASSINATION are arranged at random. Find the Probability that
\begin{enumerate}[label=(\alph*)]
\item Four $S's$ come consecutively in the word
\item Two  $I's$ and two $N's$ come together
\item All $A's$ are not coming together
\item No two $A's$ are coming together
\end{enumerate}
%\input{exemplar/11/16/3/14/main.tex}
	\item One urn contains two black balls (labelled B1 and B2) and one white ball. A
	second urn contains one black ball and two white balls (labelled W1 and W2).
	Suppose the following experiment is performed. One of the two urns is chosen
	at random. Next a ball is randomly chosen from the urn. Then a second ball is
	chosen at random from the same urn without replacing the first ball.
	
	\begin{enumerate}
	\item What is the probability that two black balls are chosen?
	
	\item What is the probability that two balls of opposite colour are chosen?
	\end{enumerate}
	\solution
	%\input{exemplar/11/16/3/12/main1.tex}
\end{enumerate}

	\item 
The number lock of a suitcase has 4 wheels each labelled with ten digits i.e. from 0 to 9.The lock opens with a sequence of four digits with no repeats.What is the probability of a person getting the right sequence to open the suitcase.
\\
\solution
		%\begin{enumerate}[label=\thesection.\arabic*,ref=\thesection.\theenumi]
	\item One card is drawn from a well-shuffled deck of 52 cards. Find the probability of getting
\begin{enumerate}
\item A king of red colour 
\item A face card 
\item A red face card
\item The jack of hearts
\item A spade
\item The queen of diamonds

\end{enumerate}
\solution
		%\input{ncert/10/15/1/14/main.tex}
	\item Five cards—the ten, jack, queen, king and ace of diamonds, are well-shuffled with their face downwards. One card is then picked up at random.
\begin{enumerate}
\item
What is the probability that the card is the queen? 
\item
If the queen is drawn and put aside, what is the probability that the second card picked up is (a) an ace? (b) a queen?\\
\end{enumerate}
\solution
		%\input{ncert/10/15/1/15/defs.tex}
	\item A bag contains $5$ red balls and some blue balls. If the probability of drawing a blue ball is double that if a red ball, determine the number of blue balls in the bag. 
		\\
\solution
		%\input{ncert/10/15/2/3/defs.tex}
	\item A card is selected from a pack of 52 cards.
 \begin{enumerate}[label=(\alph*)] 
                 \item How many points are there in the sample space?
                 \item Calculate the probability that the card is an ace of spades.
                 \item Calculate the probability that the card is (i) an ace and (ii) black card.
 \end{enumerate}
\solution
		%\input{ncert/11/16/3/4/main.tex}
\item Four cards are drawn from a well-shuffled deck of 52 cards. What is the probability of obtaining 3 diamonds and one spade.
\\
\solution
		%\input{ncert/11/16/4/2/defs.tex}
\item In a certain lottery 10,000 tickets are sold and ten equal prizes are awarded. What is the probability of not getting a prize if you buy (a) one ticket (b) two tickets (c) 10 tickets ?	
\\
\solution
		%\input{ncert/11/16/4/4/defs.tex}
		%
\item 
Out of 100 students, two sections of 40 and 60 are formed. If you and your friend are among the 100 students, what is the probability that
\begin{enumerate}
\item you both enter the same section?
\item you both enter the different sections?
\end{enumerate}
\solution
		%\input{ncert/11/16/4/5/defs.tex}
	\item 
The number lock of a suitcase has 4 wheels each labelled with ten digits i.e. from 0 to 9.The lock opens with a sequence of four digits with no repeats.What is the probability of a person getting the right sequence to open the suitcase.
\\
\solution
		%\input{ncert/11/16/4/10/defs.tex}
		%
\item 
Two cards are drawn at random and without replacement from a pack of 52 playing cards. Find the probability that both the cards are black.
\\
\solution
		%\input{ncert/12/13/2/2/defs.tex}
		\item A box of oranges is inspected by examining three randomly selected oranges drawn without replacement. If all the three oranges are good, the box is approved for sale, otherwise, it is rejected. Find the probability that a box containing 15 oranges out of which 12 are good and 3 are bad ones will be approved for sale.
		\label{ncert/12/13/2/3/defs.tex}
		\item Two balls are drawn at random with replacement from a box containing 10 black and 8 red balls. Find the probability that
		\label{ncert/12/13/2/12}
\begin{enumerate}
\item both balls are red.
\item first ball is black and second is red.
\item one of them is black and other is red.
\end{enumerate}

\item In a hostel, 60\% of the students read Hindi newspaper, 40\% read English newspaper and 20\% read both Hindi and English newspapers. A student is selected at random.
		\label{ncert/12/13/2/15}
\begin{enumerate}
\item Find the probability that she reads neither Hindi nor English newspapers.
\item If she reads Hindi newspaper, find the probability that she reads English newspaper.
\item If she reads English newspaper, find the probability that she reads Hindi newspaper.\\
\end{enumerate}
\item The probability of obtaining an even prime number on each die, when a pair of dice is rolled is 
\begin{enumerate}
    \item $0$ 
    
    \item $\frac{1}{3}$ 
    
    \item $\frac{1}{12}$ 
    
    \item $\frac{1}{36}$ 
\end{enumerate}
\solution
		%\input{ncert/12/13/2/17/defs.tex}
	\item A bag contains 4 red and 4 black balls, another bag contains 2 red and 6 black balls. One of the two bags is selected at random and a ball is drawn from the bag which is found to be red. Find the probability that the ball is drawn from the first bag.
\\
\solution
		%\input{ncert/12/13/3/2/main.tex}
  \item
  Cards with numbers 2 to 101 are placed in a box. A card is selected at random.Find the probability that the card has
\begin{enumerate}[label=(\roman*)]
	\item an even number 
	\item a square number
\end{enumerate}
\solution
%\input{exemplar/10/13/3/32/main.tex}
\item
The king, queen and jack of clubs are removed from a deck of 52 playing cards and then well shuffled. Now one card is drawn at random from the remaining cards.  Determine the probability that the card is
\begin{enumerate}[label=(\roman*)]
\item a club
\item 10 of hearts
\end{enumerate}
\solution
%\input{exemplar/10/13/3/29/main.tex}
\item A team of medical students doing their internship have to assist during surgeries
at a city hospital. The probabilities of surgeries rated as very complex, complex,
routine, simple or very simple are respectively, 0.15, 0.20, 0.31, 0.26, .08. Find
the probabilities that a particular surgery will be rated
\begin{enumerate}
	\item complex or very complex;
	\item neither very complex nor very simple;
	\item routine or complex
	\item routine or simple
\end{enumerate}
\solution
%\input{exemplar/11/16/3/8(1)/main.tex}
\item A card is selected from a pack of 52 cards.
\begin{enumerate}[label=(\alph*)]
    \item How many points are there in the sample space?
    \item Calculate the probability that the card is an ace of spades.
    \item Calculate the probability that the card is (i) an ace and (ii) black card.
\end{enumerate}
\solution
%\input{exemplar/11/16/3/4/main2.tex}
\item The probability that a non leap year selected at random will contain 53 sundays.
\\
\solution
%\input{exemplar/10/13/1/19/main.tex}
\item One of the four persons John, Rita, Aslam or Gurpreet will be promoted next
month. Consequently the sample space consists of four elementary outcomes
S = {John promoted, Rita promoted, Aslam promoted, Gurpreet promoted}
You are told that the chances of John’s promotion is same as that of Gurpreet,
Rita’s chances of promotion are twice as likely as Johns. Aslam’s chances are
four times that of John.
\begin{enumerate}
	\item Determine
	\begin{enumerate}
		\item P (John promoted)
		\item P (Rita promoted)
		\item P (Aslam promoted)
		\item P (Gurpreet promoted)
	\end{enumerate}
	\item If A = {John promoted or Gurpreet promoted}, find P (A).
\end{enumerate}
\solution
%\input{exemplar/11/16/3/10/main.tex}
\item A card is drawn from a deck of 52 cards. Find the probability of getting a king or a heart or a red card.\\
\solution
%\input{exemplar/11/16/3/15/main.tex}
\item The probability that a student will pass his examination is 0.73, the probability of
the student getting a compartment is 0.13, and the probability that the student will
either pass or get compartment is 0.96. State True or False.\\
\solution
%\input{exemplar/11/16/3/31/main.tex}
\item A card is selected from a pack of 52 cards\\
\begin{enumerate}[label=(\alph*)]
\item How many points are there in the sample space?
\item Calculate the probability that the cards is an ace of spades.
\item Calculate the probability that the card is (i) an ace (ii)black card.\\
\end{enumerate}
%\input{ncert/11/16/3/4_1/Prob_4.tex}
\item In a non-leap year, the probability of having 53 tuesdays or 53 wednesdays is\\
\solution
%\input{exemplar/11/16/3/18/main.tex}
\item There are 1000 sealed envelopes in a box, 10 of them contain a cash prize of
Rs 100 each, 100 of them contain a cash prize of Rs 50 each and 200 of them
contain a cash prize of Rs 10 each and rest do not contain any cash prize. If they
are well shuffled and an envelope is picked up out, what is the probability that it
contains no cash prize?\\
\solution
%\input{exemplar/10/13/3/34/main.tex}
\item 
A die is thrown and a card is selected at random from a deck of 52 playing cards. The probability of getting an even number on the die and a spade card.\\
\solution
%\input{exemplar/12/13/3/78/main.tex}
\item
If 4-digit numbers greater than 5,000 are randomly formed from the digits 0, 1, 3, 5, and 7, what is the probability of forming a number divisible by 5 when:
\begin{enumerate}
    \item The digits are repeated?
    \item The repetition of digits is not allowed?
\end{enumerate}
\solution
%\input{ncert/11/16/4/9/main.tex}
\item Consider the probability space $\brak{\Omega, \mathcal{G}, P}$ where $\Omega = [0,2]$ and $\mathcal{G} = \cbrak{\phi, \Omega, [0,1], (1,2]}$. Let $X$ and $Y$ be two functions on $\Omega$ defined as
\begin{align*}
    X(\omega) = 
    \begin{cases}
        1 & \text{if }\omega \in [0, 1]\\
        2 & \text{if }\omega \in (1, 2]
    \end{cases}
\end{align*}
and
\begin{align*}
    Y(\omega) = 
    \begin{cases}
        2 & \text{if }\omega \in [0, 1.5]\\
        3 & \text{if }\omega \in (1.5, 2].
    \end{cases}
\end{align*}
Then which one of the following statements is true?
\begin{enumerate}
    \item [(A)] $X$ is a random variable with respect to $\mathcal{G}$, but $Y$ is not a random variable with respect to $\mathcal{G}$.
    \item [(B)] $Y$ is a random variable with respect to $\mathcal{G}$, but $X$ is not a random variable with respect to $\mathcal{G}$.
    \item [(C)] Neither $X$ nor $Y$ is a random variable with respect to $\mathcal{G}$.
    \item [(D)] Both $X$ and $Y$ are random variables with respect to $\mathcal{G}$.
\end{enumerate} \hfill (GATE ST 2023)\\
\solution
%\input{gate/ST/2023/14/main.tex}
	\item  A die is loaded in such a way that each odd number is twice as likely to occur as
each even number. Find $P(G)$, where $G$ is the event that a number greater than
3 occurs on a single roll of the die.
\\
\solution
		%\input{exemplar/11/16/3/5/main.tex}
	\item All the jacks, queens and kings are removed from a deck of 52 playing cards. The remaining cards are well shuffled and then one card is drawn at random. Giving ace a value 1 similar value for other cards, find the probability that the card has a value 
		\begin{enumerate}
			\item 7
			\item greater than 7
			\item less than 7
		\end{enumerate}
		%\input{exemplar/10/13/3/30/main.tex}
  \item A Lot consists of 48 mobile phones of which 42 are good, 3 have only minor defects and 3 have major defects.Varnika will buy a phone if it is good but the trader will only buy a mobile if it has no major defects. One phone is selected at random from the lot. What is the probability that it is
\begin{enumerate}
	\item acceptable to Varnika?
            \item acceptable to the trader?
\end{enumerate}
\solution
	%\input{exemplar/10/13/3/40/main.tex}
 \item A student says that if you throw a die, it will show up 1 or not 1. Therefore, the probability of getting 1 and the probability of getting 'not 1' each is equal to $\frac{1}{2}$. Is this correct? Give reasons.\\
 \solution
        %\input{exemplar/10/13/2/9/main.tex}
   \item Four candidates A, B, C, D have ap-
plied for the assignment to coach a school cricket
team. If A is twice as likely to be selected as B, and
B and C are given about the same chance of being
selected, while C is twice as likely to be selected
as D, what are the probabilities that
\begin{enumerate}
\item C will be selected?
\item A will not be selected?
\end{enumerate}
	%\input{exemplar/11/16/3/9/main.tex}
 \item A bag contain 24 balls of which $x$ balls are red, $2x$ are white and $3x$ are blue. A ball is selected at random, What is the probability that it is
\begin{enumerate}[label=\alph*)]
\item not red ?
\item white ?
\end{enumerate}
%\input{exemplar/10/13/3/41/main.tex}
If the letters of the word ASSASSINATION are arranged at random. Find the Probability that
\begin{enumerate}[label=(\alph*)]
\item Four $S's$ come consecutively in the word
\item Two  $I's$ and two $N's$ come together
\item All $A's$ are not coming together
\item No two $A's$ are coming together
\end{enumerate}
%\input{exemplar/11/16/3/14/main.tex}
	\item One urn contains two black balls (labelled B1 and B2) and one white ball. A
	second urn contains one black ball and two white balls (labelled W1 and W2).
	Suppose the following experiment is performed. One of the two urns is chosen
	at random. Next a ball is randomly chosen from the urn. Then a second ball is
	chosen at random from the same urn without replacing the first ball.
	
	\begin{enumerate}
	\item What is the probability that two black balls are chosen?
	
	\item What is the probability that two balls of opposite colour are chosen?
	\end{enumerate}
	\solution
	%\input{exemplar/11/16/3/12/main1.tex}
\end{enumerate}

		%
\item 
Two cards are drawn at random and without replacement from a pack of 52 playing cards. Find the probability that both the cards are black.
\\
\solution
		%\begin{enumerate}[label=\thesection.\arabic*,ref=\thesection.\theenumi]
	\item One card is drawn from a well-shuffled deck of 52 cards. Find the probability of getting
\begin{enumerate}
\item A king of red colour 
\item A face card 
\item A red face card
\item The jack of hearts
\item A spade
\item The queen of diamonds

\end{enumerate}
\solution
		%\input{ncert/10/15/1/14/main.tex}
	\item Five cards—the ten, jack, queen, king and ace of diamonds, are well-shuffled with their face downwards. One card is then picked up at random.
\begin{enumerate}
\item
What is the probability that the card is the queen? 
\item
If the queen is drawn and put aside, what is the probability that the second card picked up is (a) an ace? (b) a queen?\\
\end{enumerate}
\solution
		%\input{ncert/10/15/1/15/defs.tex}
	\item A bag contains $5$ red balls and some blue balls. If the probability of drawing a blue ball is double that if a red ball, determine the number of blue balls in the bag. 
		\\
\solution
		%\input{ncert/10/15/2/3/defs.tex}
	\item A card is selected from a pack of 52 cards.
 \begin{enumerate}[label=(\alph*)] 
                 \item How many points are there in the sample space?
                 \item Calculate the probability that the card is an ace of spades.
                 \item Calculate the probability that the card is (i) an ace and (ii) black card.
 \end{enumerate}
\solution
		%\input{ncert/11/16/3/4/main.tex}
\item Four cards are drawn from a well-shuffled deck of 52 cards. What is the probability of obtaining 3 diamonds and one spade.
\\
\solution
		%\input{ncert/11/16/4/2/defs.tex}
\item In a certain lottery 10,000 tickets are sold and ten equal prizes are awarded. What is the probability of not getting a prize if you buy (a) one ticket (b) two tickets (c) 10 tickets ?	
\\
\solution
		%\input{ncert/11/16/4/4/defs.tex}
		%
\item 
Out of 100 students, two sections of 40 and 60 are formed. If you and your friend are among the 100 students, what is the probability that
\begin{enumerate}
\item you both enter the same section?
\item you both enter the different sections?
\end{enumerate}
\solution
		%\input{ncert/11/16/4/5/defs.tex}
	\item 
The number lock of a suitcase has 4 wheels each labelled with ten digits i.e. from 0 to 9.The lock opens with a sequence of four digits with no repeats.What is the probability of a person getting the right sequence to open the suitcase.
\\
\solution
		%\input{ncert/11/16/4/10/defs.tex}
		%
\item 
Two cards are drawn at random and without replacement from a pack of 52 playing cards. Find the probability that both the cards are black.
\\
\solution
		%\input{ncert/12/13/2/2/defs.tex}
		\item A box of oranges is inspected by examining three randomly selected oranges drawn without replacement. If all the three oranges are good, the box is approved for sale, otherwise, it is rejected. Find the probability that a box containing 15 oranges out of which 12 are good and 3 are bad ones will be approved for sale.
		\label{ncert/12/13/2/3/defs.tex}
		\item Two balls are drawn at random with replacement from a box containing 10 black and 8 red balls. Find the probability that
		\label{ncert/12/13/2/12}
\begin{enumerate}
\item both balls are red.
\item first ball is black and second is red.
\item one of them is black and other is red.
\end{enumerate}

\item In a hostel, 60\% of the students read Hindi newspaper, 40\% read English newspaper and 20\% read both Hindi and English newspapers. A student is selected at random.
		\label{ncert/12/13/2/15}
\begin{enumerate}
\item Find the probability that she reads neither Hindi nor English newspapers.
\item If she reads Hindi newspaper, find the probability that she reads English newspaper.
\item If she reads English newspaper, find the probability that she reads Hindi newspaper.\\
\end{enumerate}
\item The probability of obtaining an even prime number on each die, when a pair of dice is rolled is 
\begin{enumerate}
    \item $0$ 
    
    \item $\frac{1}{3}$ 
    
    \item $\frac{1}{12}$ 
    
    \item $\frac{1}{36}$ 
\end{enumerate}
\solution
		%\input{ncert/12/13/2/17/defs.tex}
	\item A bag contains 4 red and 4 black balls, another bag contains 2 red and 6 black balls. One of the two bags is selected at random and a ball is drawn from the bag which is found to be red. Find the probability that the ball is drawn from the first bag.
\\
\solution
		%\input{ncert/12/13/3/2/main.tex}
  \item
  Cards with numbers 2 to 101 are placed in a box. A card is selected at random.Find the probability that the card has
\begin{enumerate}[label=(\roman*)]
	\item an even number 
	\item a square number
\end{enumerate}
\solution
%\input{exemplar/10/13/3/32/main.tex}
\item
The king, queen and jack of clubs are removed from a deck of 52 playing cards and then well shuffled. Now one card is drawn at random from the remaining cards.  Determine the probability that the card is
\begin{enumerate}[label=(\roman*)]
\item a club
\item 10 of hearts
\end{enumerate}
\solution
%\input{exemplar/10/13/3/29/main.tex}
\item A team of medical students doing their internship have to assist during surgeries
at a city hospital. The probabilities of surgeries rated as very complex, complex,
routine, simple or very simple are respectively, 0.15, 0.20, 0.31, 0.26, .08. Find
the probabilities that a particular surgery will be rated
\begin{enumerate}
	\item complex or very complex;
	\item neither very complex nor very simple;
	\item routine or complex
	\item routine or simple
\end{enumerate}
\solution
%\input{exemplar/11/16/3/8(1)/main.tex}
\item A card is selected from a pack of 52 cards.
\begin{enumerate}[label=(\alph*)]
    \item How many points are there in the sample space?
    \item Calculate the probability that the card is an ace of spades.
    \item Calculate the probability that the card is (i) an ace and (ii) black card.
\end{enumerate}
\solution
%\input{exemplar/11/16/3/4/main2.tex}
\item The probability that a non leap year selected at random will contain 53 sundays.
\\
\solution
%\input{exemplar/10/13/1/19/main.tex}
\item One of the four persons John, Rita, Aslam or Gurpreet will be promoted next
month. Consequently the sample space consists of four elementary outcomes
S = {John promoted, Rita promoted, Aslam promoted, Gurpreet promoted}
You are told that the chances of John’s promotion is same as that of Gurpreet,
Rita’s chances of promotion are twice as likely as Johns. Aslam’s chances are
four times that of John.
\begin{enumerate}
	\item Determine
	\begin{enumerate}
		\item P (John promoted)
		\item P (Rita promoted)
		\item P (Aslam promoted)
		\item P (Gurpreet promoted)
	\end{enumerate}
	\item If A = {John promoted or Gurpreet promoted}, find P (A).
\end{enumerate}
\solution
%\input{exemplar/11/16/3/10/main.tex}
\item A card is drawn from a deck of 52 cards. Find the probability of getting a king or a heart or a red card.\\
\solution
%\input{exemplar/11/16/3/15/main.tex}
\item The probability that a student will pass his examination is 0.73, the probability of
the student getting a compartment is 0.13, and the probability that the student will
either pass or get compartment is 0.96. State True or False.\\
\solution
%\input{exemplar/11/16/3/31/main.tex}
\item A card is selected from a pack of 52 cards\\
\begin{enumerate}[label=(\alph*)]
\item How many points are there in the sample space?
\item Calculate the probability that the cards is an ace of spades.
\item Calculate the probability that the card is (i) an ace (ii)black card.\\
\end{enumerate}
%\input{ncert/11/16/3/4_1/Prob_4.tex}
\item In a non-leap year, the probability of having 53 tuesdays or 53 wednesdays is\\
\solution
%\input{exemplar/11/16/3/18/main.tex}
\item There are 1000 sealed envelopes in a box, 10 of them contain a cash prize of
Rs 100 each, 100 of them contain a cash prize of Rs 50 each and 200 of them
contain a cash prize of Rs 10 each and rest do not contain any cash prize. If they
are well shuffled and an envelope is picked up out, what is the probability that it
contains no cash prize?\\
\solution
%\input{exemplar/10/13/3/34/main.tex}
\item 
A die is thrown and a card is selected at random from a deck of 52 playing cards. The probability of getting an even number on the die and a spade card.\\
\solution
%\input{exemplar/12/13/3/78/main.tex}
\item
If 4-digit numbers greater than 5,000 are randomly formed from the digits 0, 1, 3, 5, and 7, what is the probability of forming a number divisible by 5 when:
\begin{enumerate}
    \item The digits are repeated?
    \item The repetition of digits is not allowed?
\end{enumerate}
\solution
%\input{ncert/11/16/4/9/main.tex}
\item Consider the probability space $\brak{\Omega, \mathcal{G}, P}$ where $\Omega = [0,2]$ and $\mathcal{G} = \cbrak{\phi, \Omega, [0,1], (1,2]}$. Let $X$ and $Y$ be two functions on $\Omega$ defined as
\begin{align*}
    X(\omega) = 
    \begin{cases}
        1 & \text{if }\omega \in [0, 1]\\
        2 & \text{if }\omega \in (1, 2]
    \end{cases}
\end{align*}
and
\begin{align*}
    Y(\omega) = 
    \begin{cases}
        2 & \text{if }\omega \in [0, 1.5]\\
        3 & \text{if }\omega \in (1.5, 2].
    \end{cases}
\end{align*}
Then which one of the following statements is true?
\begin{enumerate}
    \item [(A)] $X$ is a random variable with respect to $\mathcal{G}$, but $Y$ is not a random variable with respect to $\mathcal{G}$.
    \item [(B)] $Y$ is a random variable with respect to $\mathcal{G}$, but $X$ is not a random variable with respect to $\mathcal{G}$.
    \item [(C)] Neither $X$ nor $Y$ is a random variable with respect to $\mathcal{G}$.
    \item [(D)] Both $X$ and $Y$ are random variables with respect to $\mathcal{G}$.
\end{enumerate} \hfill (GATE ST 2023)\\
\solution
%\input{gate/ST/2023/14/main.tex}
	\item  A die is loaded in such a way that each odd number is twice as likely to occur as
each even number. Find $P(G)$, where $G$ is the event that a number greater than
3 occurs on a single roll of the die.
\\
\solution
		%\input{exemplar/11/16/3/5/main.tex}
	\item All the jacks, queens and kings are removed from a deck of 52 playing cards. The remaining cards are well shuffled and then one card is drawn at random. Giving ace a value 1 similar value for other cards, find the probability that the card has a value 
		\begin{enumerate}
			\item 7
			\item greater than 7
			\item less than 7
		\end{enumerate}
		%\input{exemplar/10/13/3/30/main.tex}
  \item A Lot consists of 48 mobile phones of which 42 are good, 3 have only minor defects and 3 have major defects.Varnika will buy a phone if it is good but the trader will only buy a mobile if it has no major defects. One phone is selected at random from the lot. What is the probability that it is
\begin{enumerate}
	\item acceptable to Varnika?
            \item acceptable to the trader?
\end{enumerate}
\solution
	%\input{exemplar/10/13/3/40/main.tex}
 \item A student says that if you throw a die, it will show up 1 or not 1. Therefore, the probability of getting 1 and the probability of getting 'not 1' each is equal to $\frac{1}{2}$. Is this correct? Give reasons.\\
 \solution
        %\input{exemplar/10/13/2/9/main.tex}
   \item Four candidates A, B, C, D have ap-
plied for the assignment to coach a school cricket
team. If A is twice as likely to be selected as B, and
B and C are given about the same chance of being
selected, while C is twice as likely to be selected
as D, what are the probabilities that
\begin{enumerate}
\item C will be selected?
\item A will not be selected?
\end{enumerate}
	%\input{exemplar/11/16/3/9/main.tex}
 \item A bag contain 24 balls of which $x$ balls are red, $2x$ are white and $3x$ are blue. A ball is selected at random, What is the probability that it is
\begin{enumerate}[label=\alph*)]
\item not red ?
\item white ?
\end{enumerate}
%\input{exemplar/10/13/3/41/main.tex}
If the letters of the word ASSASSINATION are arranged at random. Find the Probability that
\begin{enumerate}[label=(\alph*)]
\item Four $S's$ come consecutively in the word
\item Two  $I's$ and two $N's$ come together
\item All $A's$ are not coming together
\item No two $A's$ are coming together
\end{enumerate}
%\input{exemplar/11/16/3/14/main.tex}
	\item One urn contains two black balls (labelled B1 and B2) and one white ball. A
	second urn contains one black ball and two white balls (labelled W1 and W2).
	Suppose the following experiment is performed. One of the two urns is chosen
	at random. Next a ball is randomly chosen from the urn. Then a second ball is
	chosen at random from the same urn without replacing the first ball.
	
	\begin{enumerate}
	\item What is the probability that two black balls are chosen?
	
	\item What is the probability that two balls of opposite colour are chosen?
	\end{enumerate}
	\solution
	%\input{exemplar/11/16/3/12/main1.tex}
\end{enumerate}

		\item A box of oranges is inspected by examining three randomly selected oranges drawn without replacement. If all the three oranges are good, the box is approved for sale, otherwise, it is rejected. Find the probability that a box containing 15 oranges out of which 12 are good and 3 are bad ones will be approved for sale.
		\label{ncert/12/13/2/3/defs.tex}
		\item Two balls are drawn at random with replacement from a box containing 10 black and 8 red balls. Find the probability that
		\label{ncert/12/13/2/12}
\begin{enumerate}
\item both balls are red.
\item first ball is black and second is red.
\item one of them is black and other is red.
\end{enumerate}

\item In a hostel, 60\% of the students read Hindi newspaper, 40\% read English newspaper and 20\% read both Hindi and English newspapers. A student is selected at random.
		\label{ncert/12/13/2/15}
\begin{enumerate}
\item Find the probability that she reads neither Hindi nor English newspapers.
\item If she reads Hindi newspaper, find the probability that she reads English newspaper.
\item If she reads English newspaper, find the probability that she reads Hindi newspaper.\\
\end{enumerate}
\item The probability of obtaining an even prime number on each die, when a pair of dice is rolled is 
\begin{enumerate}
    \item $0$ 
    
    \item $\frac{1}{3}$ 
    
    \item $\frac{1}{12}$ 
    
    \item $\frac{1}{36}$ 
\end{enumerate}
\solution
		%\begin{enumerate}[label=\thesection.\arabic*,ref=\thesection.\theenumi]
	\item One card is drawn from a well-shuffled deck of 52 cards. Find the probability of getting
\begin{enumerate}
\item A king of red colour 
\item A face card 
\item A red face card
\item The jack of hearts
\item A spade
\item The queen of diamonds

\end{enumerate}
\solution
		%\input{ncert/10/15/1/14/main.tex}
	\item Five cards—the ten, jack, queen, king and ace of diamonds, are well-shuffled with their face downwards. One card is then picked up at random.
\begin{enumerate}
\item
What is the probability that the card is the queen? 
\item
If the queen is drawn and put aside, what is the probability that the second card picked up is (a) an ace? (b) a queen?\\
\end{enumerate}
\solution
		%\input{ncert/10/15/1/15/defs.tex}
	\item A bag contains $5$ red balls and some blue balls. If the probability of drawing a blue ball is double that if a red ball, determine the number of blue balls in the bag. 
		\\
\solution
		%\input{ncert/10/15/2/3/defs.tex}
	\item A card is selected from a pack of 52 cards.
 \begin{enumerate}[label=(\alph*)] 
                 \item How many points are there in the sample space?
                 \item Calculate the probability that the card is an ace of spades.
                 \item Calculate the probability that the card is (i) an ace and (ii) black card.
 \end{enumerate}
\solution
		%\input{ncert/11/16/3/4/main.tex}
\item Four cards are drawn from a well-shuffled deck of 52 cards. What is the probability of obtaining 3 diamonds and one spade.
\\
\solution
		%\input{ncert/11/16/4/2/defs.tex}
\item In a certain lottery 10,000 tickets are sold and ten equal prizes are awarded. What is the probability of not getting a prize if you buy (a) one ticket (b) two tickets (c) 10 tickets ?	
\\
\solution
		%\input{ncert/11/16/4/4/defs.tex}
		%
\item 
Out of 100 students, two sections of 40 and 60 are formed. If you and your friend are among the 100 students, what is the probability that
\begin{enumerate}
\item you both enter the same section?
\item you both enter the different sections?
\end{enumerate}
\solution
		%\input{ncert/11/16/4/5/defs.tex}
	\item 
The number lock of a suitcase has 4 wheels each labelled with ten digits i.e. from 0 to 9.The lock opens with a sequence of four digits with no repeats.What is the probability of a person getting the right sequence to open the suitcase.
\\
\solution
		%\input{ncert/11/16/4/10/defs.tex}
		%
\item 
Two cards are drawn at random and without replacement from a pack of 52 playing cards. Find the probability that both the cards are black.
\\
\solution
		%\input{ncert/12/13/2/2/defs.tex}
		\item A box of oranges is inspected by examining three randomly selected oranges drawn without replacement. If all the three oranges are good, the box is approved for sale, otherwise, it is rejected. Find the probability that a box containing 15 oranges out of which 12 are good and 3 are bad ones will be approved for sale.
		\label{ncert/12/13/2/3/defs.tex}
		\item Two balls are drawn at random with replacement from a box containing 10 black and 8 red balls. Find the probability that
		\label{ncert/12/13/2/12}
\begin{enumerate}
\item both balls are red.
\item first ball is black and second is red.
\item one of them is black and other is red.
\end{enumerate}

\item In a hostel, 60\% of the students read Hindi newspaper, 40\% read English newspaper and 20\% read both Hindi and English newspapers. A student is selected at random.
		\label{ncert/12/13/2/15}
\begin{enumerate}
\item Find the probability that she reads neither Hindi nor English newspapers.
\item If she reads Hindi newspaper, find the probability that she reads English newspaper.
\item If she reads English newspaper, find the probability that she reads Hindi newspaper.\\
\end{enumerate}
\item The probability of obtaining an even prime number on each die, when a pair of dice is rolled is 
\begin{enumerate}
    \item $0$ 
    
    \item $\frac{1}{3}$ 
    
    \item $\frac{1}{12}$ 
    
    \item $\frac{1}{36}$ 
\end{enumerate}
\solution
		%\input{ncert/12/13/2/17/defs.tex}
	\item A bag contains 4 red and 4 black balls, another bag contains 2 red and 6 black balls. One of the two bags is selected at random and a ball is drawn from the bag which is found to be red. Find the probability that the ball is drawn from the first bag.
\\
\solution
		%\input{ncert/12/13/3/2/main.tex}
  \item
  Cards with numbers 2 to 101 are placed in a box. A card is selected at random.Find the probability that the card has
\begin{enumerate}[label=(\roman*)]
	\item an even number 
	\item a square number
\end{enumerate}
\solution
%\input{exemplar/10/13/3/32/main.tex}
\item
The king, queen and jack of clubs are removed from a deck of 52 playing cards and then well shuffled. Now one card is drawn at random from the remaining cards.  Determine the probability that the card is
\begin{enumerate}[label=(\roman*)]
\item a club
\item 10 of hearts
\end{enumerate}
\solution
%\input{exemplar/10/13/3/29/main.tex}
\item A team of medical students doing their internship have to assist during surgeries
at a city hospital. The probabilities of surgeries rated as very complex, complex,
routine, simple or very simple are respectively, 0.15, 0.20, 0.31, 0.26, .08. Find
the probabilities that a particular surgery will be rated
\begin{enumerate}
	\item complex or very complex;
	\item neither very complex nor very simple;
	\item routine or complex
	\item routine or simple
\end{enumerate}
\solution
%\input{exemplar/11/16/3/8(1)/main.tex}
\item A card is selected from a pack of 52 cards.
\begin{enumerate}[label=(\alph*)]
    \item How many points are there in the sample space?
    \item Calculate the probability that the card is an ace of spades.
    \item Calculate the probability that the card is (i) an ace and (ii) black card.
\end{enumerate}
\solution
%\input{exemplar/11/16/3/4/main2.tex}
\item The probability that a non leap year selected at random will contain 53 sundays.
\\
\solution
%\input{exemplar/10/13/1/19/main.tex}
\item One of the four persons John, Rita, Aslam or Gurpreet will be promoted next
month. Consequently the sample space consists of four elementary outcomes
S = {John promoted, Rita promoted, Aslam promoted, Gurpreet promoted}
You are told that the chances of John’s promotion is same as that of Gurpreet,
Rita’s chances of promotion are twice as likely as Johns. Aslam’s chances are
four times that of John.
\begin{enumerate}
	\item Determine
	\begin{enumerate}
		\item P (John promoted)
		\item P (Rita promoted)
		\item P (Aslam promoted)
		\item P (Gurpreet promoted)
	\end{enumerate}
	\item If A = {John promoted or Gurpreet promoted}, find P (A).
\end{enumerate}
\solution
%\input{exemplar/11/16/3/10/main.tex}
\item A card is drawn from a deck of 52 cards. Find the probability of getting a king or a heart or a red card.\\
\solution
%\input{exemplar/11/16/3/15/main.tex}
\item The probability that a student will pass his examination is 0.73, the probability of
the student getting a compartment is 0.13, and the probability that the student will
either pass or get compartment is 0.96. State True or False.\\
\solution
%\input{exemplar/11/16/3/31/main.tex}
\item A card is selected from a pack of 52 cards\\
\begin{enumerate}[label=(\alph*)]
\item How many points are there in the sample space?
\item Calculate the probability that the cards is an ace of spades.
\item Calculate the probability that the card is (i) an ace (ii)black card.\\
\end{enumerate}
%\input{ncert/11/16/3/4_1/Prob_4.tex}
\item In a non-leap year, the probability of having 53 tuesdays or 53 wednesdays is\\
\solution
%\input{exemplar/11/16/3/18/main.tex}
\item There are 1000 sealed envelopes in a box, 10 of them contain a cash prize of
Rs 100 each, 100 of them contain a cash prize of Rs 50 each and 200 of them
contain a cash prize of Rs 10 each and rest do not contain any cash prize. If they
are well shuffled and an envelope is picked up out, what is the probability that it
contains no cash prize?\\
\solution
%\input{exemplar/10/13/3/34/main.tex}
\item 
A die is thrown and a card is selected at random from a deck of 52 playing cards. The probability of getting an even number on the die and a spade card.\\
\solution
%\input{exemplar/12/13/3/78/main.tex}
\item
If 4-digit numbers greater than 5,000 are randomly formed from the digits 0, 1, 3, 5, and 7, what is the probability of forming a number divisible by 5 when:
\begin{enumerate}
    \item The digits are repeated?
    \item The repetition of digits is not allowed?
\end{enumerate}
\solution
%\input{ncert/11/16/4/9/main.tex}
\item Consider the probability space $\brak{\Omega, \mathcal{G}, P}$ where $\Omega = [0,2]$ and $\mathcal{G} = \cbrak{\phi, \Omega, [0,1], (1,2]}$. Let $X$ and $Y$ be two functions on $\Omega$ defined as
\begin{align*}
    X(\omega) = 
    \begin{cases}
        1 & \text{if }\omega \in [0, 1]\\
        2 & \text{if }\omega \in (1, 2]
    \end{cases}
\end{align*}
and
\begin{align*}
    Y(\omega) = 
    \begin{cases}
        2 & \text{if }\omega \in [0, 1.5]\\
        3 & \text{if }\omega \in (1.5, 2].
    \end{cases}
\end{align*}
Then which one of the following statements is true?
\begin{enumerate}
    \item [(A)] $X$ is a random variable with respect to $\mathcal{G}$, but $Y$ is not a random variable with respect to $\mathcal{G}$.
    \item [(B)] $Y$ is a random variable with respect to $\mathcal{G}$, but $X$ is not a random variable with respect to $\mathcal{G}$.
    \item [(C)] Neither $X$ nor $Y$ is a random variable with respect to $\mathcal{G}$.
    \item [(D)] Both $X$ and $Y$ are random variables with respect to $\mathcal{G}$.
\end{enumerate} \hfill (GATE ST 2023)\\
\solution
%\input{gate/ST/2023/14/main.tex}
	\item  A die is loaded in such a way that each odd number is twice as likely to occur as
each even number. Find $P(G)$, where $G$ is the event that a number greater than
3 occurs on a single roll of the die.
\\
\solution
		%\input{exemplar/11/16/3/5/main.tex}
	\item All the jacks, queens and kings are removed from a deck of 52 playing cards. The remaining cards are well shuffled and then one card is drawn at random. Giving ace a value 1 similar value for other cards, find the probability that the card has a value 
		\begin{enumerate}
			\item 7
			\item greater than 7
			\item less than 7
		\end{enumerate}
		%\input{exemplar/10/13/3/30/main.tex}
  \item A Lot consists of 48 mobile phones of which 42 are good, 3 have only minor defects and 3 have major defects.Varnika will buy a phone if it is good but the trader will only buy a mobile if it has no major defects. One phone is selected at random from the lot. What is the probability that it is
\begin{enumerate}
	\item acceptable to Varnika?
            \item acceptable to the trader?
\end{enumerate}
\solution
	%\input{exemplar/10/13/3/40/main.tex}
 \item A student says that if you throw a die, it will show up 1 or not 1. Therefore, the probability of getting 1 and the probability of getting 'not 1' each is equal to $\frac{1}{2}$. Is this correct? Give reasons.\\
 \solution
        %\input{exemplar/10/13/2/9/main.tex}
   \item Four candidates A, B, C, D have ap-
plied for the assignment to coach a school cricket
team. If A is twice as likely to be selected as B, and
B and C are given about the same chance of being
selected, while C is twice as likely to be selected
as D, what are the probabilities that
\begin{enumerate}
\item C will be selected?
\item A will not be selected?
\end{enumerate}
	%\input{exemplar/11/16/3/9/main.tex}
 \item A bag contain 24 balls of which $x$ balls are red, $2x$ are white and $3x$ are blue. A ball is selected at random, What is the probability that it is
\begin{enumerate}[label=\alph*)]
\item not red ?
\item white ?
\end{enumerate}
%\input{exemplar/10/13/3/41/main.tex}
If the letters of the word ASSASSINATION are arranged at random. Find the Probability that
\begin{enumerate}[label=(\alph*)]
\item Four $S's$ come consecutively in the word
\item Two  $I's$ and two $N's$ come together
\item All $A's$ are not coming together
\item No two $A's$ are coming together
\end{enumerate}
%\input{exemplar/11/16/3/14/main.tex}
	\item One urn contains two black balls (labelled B1 and B2) and one white ball. A
	second urn contains one black ball and two white balls (labelled W1 and W2).
	Suppose the following experiment is performed. One of the two urns is chosen
	at random. Next a ball is randomly chosen from the urn. Then a second ball is
	chosen at random from the same urn without replacing the first ball.
	
	\begin{enumerate}
	\item What is the probability that two black balls are chosen?
	
	\item What is the probability that two balls of opposite colour are chosen?
	\end{enumerate}
	\solution
	%\input{exemplar/11/16/3/12/main1.tex}
\end{enumerate}

	\item A bag contains 4 red and 4 black balls, another bag contains 2 red and 6 black balls. One of the two bags is selected at random and a ball is drawn from the bag which is found to be red. Find the probability that the ball is drawn from the first bag.
\\
\solution
		%\begin{table}[H]
	\centering
\begin{tabular}{|c|c|c|}
\hline
Random variable &Value &Definition\\ \hline
\multirow{3}{*}{X} &0 &Slips of Rs 1\\
&1 &Slips of Rs 5\\
&2 &Slips of Rs 13\\ \hline
\multirow{2}{*}{Y} &0 &Box A\\
&1 &Box B\\\hline
\end{tabular}
\caption{}
\label{tab:Distribution}
\end{table}
See \tabref{tab:Distribution}.
\begin{align}
p_{Y}\brak{k}= \begin{cases} 
      \frac{1}{3} & {k=0} \\
      \frac{2}{3 }& {k=1} 
   \end{cases}
   \\
p_{Y|X}\brak{0|0} = \frac{19}{25}\, 
p_{Y|X}\brak{0|1} = \frac{6}{25}\,
p_{Y|X}\brak{1|0} = \frac{45}{50}\,
p_{Y|X}\brak{1|2} = \frac{5}{50}
\end{align}
The desired probability is the probability that a slip drawn at random is marked other than Rs 1,
\begin{align}
&=1-p_X\brak{0}\\
&= p_X(1) + p_X(2)
\end{align}
Using Bayes theorem,
\begin{align}
&= p_Y\brak{0} \times \pr{Y=0 | X=1} + p_Y\brak{1} \times \pr{Y=1|X=2}\\
&=\frac{1}{3} \times \frac{6}{25} + \frac{2}{3} \times \frac{5}{50}\\
&=\frac{11}{75}
\end{align}

\newpage

%\tableofcontents

\bigskip

\renewcommand{\thefigure}{\theenumi}
\renewcommand{\thetable}{\theenumi}
%\renewcommand{\theequation}{\theenumi}

%\begin{abstract}
%%\boldmath
%In this letter, an algorithm for evaluating the exact analytical bit error rate  (BER)  for the piecewise linear (PL) combiner for  multiple relays is presented. Previous results were available only for upto three relays. The algorithm is unique in the sense that  the actual mathematical expressions, that are prohibitively large, need not be explicitly obtained. The diversity gain due to multiple relays is shown through plots of the analytical BER, well supported by simulations. 
%
%\end{abstract}
% IEEEtran.cls defaults to using nonbold math in the Abstract.
% This preserves the distinction between vectors and scalars. However,
% if the journal you are submitting to favors bold math in the abstract,
% then you can use LaTeX's standard command \boldmath at the very start
% of the abstract to achieve this. Many IEEE journals frown on math
% in the abstract anyway.

% Note that keywords are not normally used for peerreview papers.
%\begin{IEEEkeywords}
%Cooperative diversity, decode and forward, piecewise linear
%\end{IEEEkeywords}



% For peer review papers, you can put extra information on the cover
% page as needed:
% \ifCLASSOPTIONpeerreview
% \begin{center} \bfseries EDICS Category: 3-BBND \end{center}
% \fi
%
% For peerreview papers, this IEEEtran command inserts a page break and
% creates the second title. It will be ignored for other modes.
%\IEEEpeerreviewmaketitle




  \item
  Cards with numbers 2 to 101 are placed in a box. A card is selected at random.Find the probability that the card has
\begin{enumerate}[label=(\roman*)]
	\item an even number 
	\item a square number
\end{enumerate}
\solution
%\begin{table}[H]
	\centering
\begin{tabular}{|c|c|c|}
\hline
Random variable &Value &Definition\\ \hline
\multirow{3}{*}{X} &0 &Slips of Rs 1\\
&1 &Slips of Rs 5\\
&2 &Slips of Rs 13\\ \hline
\multirow{2}{*}{Y} &0 &Box A\\
&1 &Box B\\\hline
\end{tabular}
\caption{}
\label{tab:Distribution}
\end{table}
See \tabref{tab:Distribution}.
\begin{align}
p_{Y}\brak{k}= \begin{cases} 
      \frac{1}{3} & {k=0} \\
      \frac{2}{3 }& {k=1} 
   \end{cases}
   \\
p_{Y|X}\brak{0|0} = \frac{19}{25}\, 
p_{Y|X}\brak{0|1} = \frac{6}{25}\,
p_{Y|X}\brak{1|0} = \frac{45}{50}\,
p_{Y|X}\brak{1|2} = \frac{5}{50}
\end{align}
The desired probability is the probability that a slip drawn at random is marked other than Rs 1,
\begin{align}
&=1-p_X\brak{0}\\
&= p_X(1) + p_X(2)
\end{align}
Using Bayes theorem,
\begin{align}
&= p_Y\brak{0} \times \pr{Y=0 | X=1} + p_Y\brak{1} \times \pr{Y=1|X=2}\\
&=\frac{1}{3} \times \frac{6}{25} + \frac{2}{3} \times \frac{5}{50}\\
&=\frac{11}{75}
\end{align}

\newpage

%\tableofcontents

\bigskip

\renewcommand{\thefigure}{\theenumi}
\renewcommand{\thetable}{\theenumi}
%\renewcommand{\theequation}{\theenumi}

%\begin{abstract}
%%\boldmath
%In this letter, an algorithm for evaluating the exact analytical bit error rate  (BER)  for the piecewise linear (PL) combiner for  multiple relays is presented. Previous results were available only for upto three relays. The algorithm is unique in the sense that  the actual mathematical expressions, that are prohibitively large, need not be explicitly obtained. The diversity gain due to multiple relays is shown through plots of the analytical BER, well supported by simulations. 
%
%\end{abstract}
% IEEEtran.cls defaults to using nonbold math in the Abstract.
% This preserves the distinction between vectors and scalars. However,
% if the journal you are submitting to favors bold math in the abstract,
% then you can use LaTeX's standard command \boldmath at the very start
% of the abstract to achieve this. Many IEEE journals frown on math
% in the abstract anyway.

% Note that keywords are not normally used for peerreview papers.
%\begin{IEEEkeywords}
%Cooperative diversity, decode and forward, piecewise linear
%\end{IEEEkeywords}



% For peer review papers, you can put extra information on the cover
% page as needed:
% \ifCLASSOPTIONpeerreview
% \begin{center} \bfseries EDICS Category: 3-BBND \end{center}
% \fi
%
% For peerreview papers, this IEEEtran command inserts a page break and
% creates the second title. It will be ignored for other modes.
%\IEEEpeerreviewmaketitle




\item
The king, queen and jack of clubs are removed from a deck of 52 playing cards and then well shuffled. Now one card is drawn at random from the remaining cards.  Determine the probability that the card is
\begin{enumerate}[label=(\roman*)]
\item a club
\item 10 of hearts
\end{enumerate}
\solution
%\begin{table}[H]
	\centering
\begin{tabular}{|c|c|c|}
\hline
Random variable &Value &Definition\\ \hline
\multirow{3}{*}{X} &0 &Slips of Rs 1\\
&1 &Slips of Rs 5\\
&2 &Slips of Rs 13\\ \hline
\multirow{2}{*}{Y} &0 &Box A\\
&1 &Box B\\\hline
\end{tabular}
\caption{}
\label{tab:Distribution}
\end{table}
See \tabref{tab:Distribution}.
\begin{align}
p_{Y}\brak{k}= \begin{cases} 
      \frac{1}{3} & {k=0} \\
      \frac{2}{3 }& {k=1} 
   \end{cases}
   \\
p_{Y|X}\brak{0|0} = \frac{19}{25}\, 
p_{Y|X}\brak{0|1} = \frac{6}{25}\,
p_{Y|X}\brak{1|0} = \frac{45}{50}\,
p_{Y|X}\brak{1|2} = \frac{5}{50}
\end{align}
The desired probability is the probability that a slip drawn at random is marked other than Rs 1,
\begin{align}
&=1-p_X\brak{0}\\
&= p_X(1) + p_X(2)
\end{align}
Using Bayes theorem,
\begin{align}
&= p_Y\brak{0} \times \pr{Y=0 | X=1} + p_Y\brak{1} \times \pr{Y=1|X=2}\\
&=\frac{1}{3} \times \frac{6}{25} + \frac{2}{3} \times \frac{5}{50}\\
&=\frac{11}{75}
\end{align}

\newpage

%\tableofcontents

\bigskip

\renewcommand{\thefigure}{\theenumi}
\renewcommand{\thetable}{\theenumi}
%\renewcommand{\theequation}{\theenumi}

%\begin{abstract}
%%\boldmath
%In this letter, an algorithm for evaluating the exact analytical bit error rate  (BER)  for the piecewise linear (PL) combiner for  multiple relays is presented. Previous results were available only for upto three relays. The algorithm is unique in the sense that  the actual mathematical expressions, that are prohibitively large, need not be explicitly obtained. The diversity gain due to multiple relays is shown through plots of the analytical BER, well supported by simulations. 
%
%\end{abstract}
% IEEEtran.cls defaults to using nonbold math in the Abstract.
% This preserves the distinction between vectors and scalars. However,
% if the journal you are submitting to favors bold math in the abstract,
% then you can use LaTeX's standard command \boldmath at the very start
% of the abstract to achieve this. Many IEEE journals frown on math
% in the abstract anyway.

% Note that keywords are not normally used for peerreview papers.
%\begin{IEEEkeywords}
%Cooperative diversity, decode and forward, piecewise linear
%\end{IEEEkeywords}



% For peer review papers, you can put extra information on the cover
% page as needed:
% \ifCLASSOPTIONpeerreview
% \begin{center} \bfseries EDICS Category: 3-BBND \end{center}
% \fi
%
% For peerreview papers, this IEEEtran command inserts a page break and
% creates the second title. It will be ignored for other modes.
%\IEEEpeerreviewmaketitle




\item A team of medical students doing their internship have to assist during surgeries
at a city hospital. The probabilities of surgeries rated as very complex, complex,
routine, simple or very simple are respectively, 0.15, 0.20, 0.31, 0.26, .08. Find
the probabilities that a particular surgery will be rated
\begin{enumerate}
	\item complex or very complex;
	\item neither very complex nor very simple;
	\item routine or complex
	\item routine or simple
\end{enumerate}
\solution
%\begin{table}[H]
	\centering
\begin{tabular}{|c|c|c|}
\hline
Random variable &Value &Definition\\ \hline
\multirow{3}{*}{X} &0 &Slips of Rs 1\\
&1 &Slips of Rs 5\\
&2 &Slips of Rs 13\\ \hline
\multirow{2}{*}{Y} &0 &Box A\\
&1 &Box B\\\hline
\end{tabular}
\caption{}
\label{tab:Distribution}
\end{table}
See \tabref{tab:Distribution}.
\begin{align}
p_{Y}\brak{k}= \begin{cases} 
      \frac{1}{3} & {k=0} \\
      \frac{2}{3 }& {k=1} 
   \end{cases}
   \\
p_{Y|X}\brak{0|0} = \frac{19}{25}\, 
p_{Y|X}\brak{0|1} = \frac{6}{25}\,
p_{Y|X}\brak{1|0} = \frac{45}{50}\,
p_{Y|X}\brak{1|2} = \frac{5}{50}
\end{align}
The desired probability is the probability that a slip drawn at random is marked other than Rs 1,
\begin{align}
&=1-p_X\brak{0}\\
&= p_X(1) + p_X(2)
\end{align}
Using Bayes theorem,
\begin{align}
&= p_Y\brak{0} \times \pr{Y=0 | X=1} + p_Y\brak{1} \times \pr{Y=1|X=2}\\
&=\frac{1}{3} \times \frac{6}{25} + \frac{2}{3} \times \frac{5}{50}\\
&=\frac{11}{75}
\end{align}

\newpage

%\tableofcontents

\bigskip

\renewcommand{\thefigure}{\theenumi}
\renewcommand{\thetable}{\theenumi}
%\renewcommand{\theequation}{\theenumi}

%\begin{abstract}
%%\boldmath
%In this letter, an algorithm for evaluating the exact analytical bit error rate  (BER)  for the piecewise linear (PL) combiner for  multiple relays is presented. Previous results were available only for upto three relays. The algorithm is unique in the sense that  the actual mathematical expressions, that are prohibitively large, need not be explicitly obtained. The diversity gain due to multiple relays is shown through plots of the analytical BER, well supported by simulations. 
%
%\end{abstract}
% IEEEtran.cls defaults to using nonbold math in the Abstract.
% This preserves the distinction between vectors and scalars. However,
% if the journal you are submitting to favors bold math in the abstract,
% then you can use LaTeX's standard command \boldmath at the very start
% of the abstract to achieve this. Many IEEE journals frown on math
% in the abstract anyway.

% Note that keywords are not normally used for peerreview papers.
%\begin{IEEEkeywords}
%Cooperative diversity, decode and forward, piecewise linear
%\end{IEEEkeywords}



% For peer review papers, you can put extra information on the cover
% page as needed:
% \ifCLASSOPTIONpeerreview
% \begin{center} \bfseries EDICS Category: 3-BBND \end{center}
% \fi
%
% For peerreview papers, this IEEEtran command inserts a page break and
% creates the second title. It will be ignored for other modes.
%\IEEEpeerreviewmaketitle




\item A card is selected from a pack of 52 cards.
\begin{enumerate}[label=(\alph*)]
    \item How many points are there in the sample space?
    \item Calculate the probability that the card is an ace of spades.
    \item Calculate the probability that the card is (i) an ace and (ii) black card.
\end{enumerate}
\solution
%Let $X$ be an bernoulli rv defined as in \tabref{tab:exemplar/11/16/3/26}.  Then, 
\begin{equation}
    p =
        \frac{4}{11} 
\end{equation}
\begin{table}[H]
	\centering
	\input{exemplar/11/16/3/26/tables/Table2.tex}
	\caption{}
        \label{tab:exemplar/11/16/3/26}
\end{table}

\item The probability that a non leap year selected at random will contain 53 sundays.
\\
\solution
%\begin{table}[H]
	\centering
\begin{tabular}{|c|c|c|}
\hline
Random variable &Value &Definition\\ \hline
\multirow{3}{*}{X} &0 &Slips of Rs 1\\
&1 &Slips of Rs 5\\
&2 &Slips of Rs 13\\ \hline
\multirow{2}{*}{Y} &0 &Box A\\
&1 &Box B\\\hline
\end{tabular}
\caption{}
\label{tab:Distribution}
\end{table}
See \tabref{tab:Distribution}.
\begin{align}
p_{Y}\brak{k}= \begin{cases} 
      \frac{1}{3} & {k=0} \\
      \frac{2}{3 }& {k=1} 
   \end{cases}
   \\
p_{Y|X}\brak{0|0} = \frac{19}{25}\, 
p_{Y|X}\brak{0|1} = \frac{6}{25}\,
p_{Y|X}\brak{1|0} = \frac{45}{50}\,
p_{Y|X}\brak{1|2} = \frac{5}{50}
\end{align}
The desired probability is the probability that a slip drawn at random is marked other than Rs 1,
\begin{align}
&=1-p_X\brak{0}\\
&= p_X(1) + p_X(2)
\end{align}
Using Bayes theorem,
\begin{align}
&= p_Y\brak{0} \times \pr{Y=0 | X=1} + p_Y\brak{1} \times \pr{Y=1|X=2}\\
&=\frac{1}{3} \times \frac{6}{25} + \frac{2}{3} \times \frac{5}{50}\\
&=\frac{11}{75}
\end{align}

\newpage

%\tableofcontents

\bigskip

\renewcommand{\thefigure}{\theenumi}
\renewcommand{\thetable}{\theenumi}
%\renewcommand{\theequation}{\theenumi}

%\begin{abstract}
%%\boldmath
%In this letter, an algorithm for evaluating the exact analytical bit error rate  (BER)  for the piecewise linear (PL) combiner for  multiple relays is presented. Previous results were available only for upto three relays. The algorithm is unique in the sense that  the actual mathematical expressions, that are prohibitively large, need not be explicitly obtained. The diversity gain due to multiple relays is shown through plots of the analytical BER, well supported by simulations. 
%
%\end{abstract}
% IEEEtran.cls defaults to using nonbold math in the Abstract.
% This preserves the distinction between vectors and scalars. However,
% if the journal you are submitting to favors bold math in the abstract,
% then you can use LaTeX's standard command \boldmath at the very start
% of the abstract to achieve this. Many IEEE journals frown on math
% in the abstract anyway.

% Note that keywords are not normally used for peerreview papers.
%\begin{IEEEkeywords}
%Cooperative diversity, decode and forward, piecewise linear
%\end{IEEEkeywords}



% For peer review papers, you can put extra information on the cover
% page as needed:
% \ifCLASSOPTIONpeerreview
% \begin{center} \bfseries EDICS Category: 3-BBND \end{center}
% \fi
%
% For peerreview papers, this IEEEtran command inserts a page break and
% creates the second title. It will be ignored for other modes.
%\IEEEpeerreviewmaketitle




\item One of the four persons John, Rita, Aslam or Gurpreet will be promoted next
month. Consequently the sample space consists of four elementary outcomes
S = {John promoted, Rita promoted, Aslam promoted, Gurpreet promoted}
You are told that the chances of John’s promotion is same as that of Gurpreet,
Rita’s chances of promotion are twice as likely as Johns. Aslam’s chances are
four times that of John.
\begin{enumerate}
	\item Determine
	\begin{enumerate}
		\item P (John promoted)
		\item P (Rita promoted)
		\item P (Aslam promoted)
		\item P (Gurpreet promoted)
	\end{enumerate}
	\item If A = {John promoted or Gurpreet promoted}, find P (A).
\end{enumerate}
\solution
%\begin{table}[H]
	\centering
\begin{tabular}{|c|c|c|}
\hline
Random variable &Value &Definition\\ \hline
\multirow{3}{*}{X} &0 &Slips of Rs 1\\
&1 &Slips of Rs 5\\
&2 &Slips of Rs 13\\ \hline
\multirow{2}{*}{Y} &0 &Box A\\
&1 &Box B\\\hline
\end{tabular}
\caption{}
\label{tab:Distribution}
\end{table}
See \tabref{tab:Distribution}.
\begin{align}
p_{Y}\brak{k}= \begin{cases} 
      \frac{1}{3} & {k=0} \\
      \frac{2}{3 }& {k=1} 
   \end{cases}
   \\
p_{Y|X}\brak{0|0} = \frac{19}{25}\, 
p_{Y|X}\brak{0|1} = \frac{6}{25}\,
p_{Y|X}\brak{1|0} = \frac{45}{50}\,
p_{Y|X}\brak{1|2} = \frac{5}{50}
\end{align}
The desired probability is the probability that a slip drawn at random is marked other than Rs 1,
\begin{align}
&=1-p_X\brak{0}\\
&= p_X(1) + p_X(2)
\end{align}
Using Bayes theorem,
\begin{align}
&= p_Y\brak{0} \times \pr{Y=0 | X=1} + p_Y\brak{1} \times \pr{Y=1|X=2}\\
&=\frac{1}{3} \times \frac{6}{25} + \frac{2}{3} \times \frac{5}{50}\\
&=\frac{11}{75}
\end{align}

\newpage

%\tableofcontents

\bigskip

\renewcommand{\thefigure}{\theenumi}
\renewcommand{\thetable}{\theenumi}
%\renewcommand{\theequation}{\theenumi}

%\begin{abstract}
%%\boldmath
%In this letter, an algorithm for evaluating the exact analytical bit error rate  (BER)  for the piecewise linear (PL) combiner for  multiple relays is presented. Previous results were available only for upto three relays. The algorithm is unique in the sense that  the actual mathematical expressions, that are prohibitively large, need not be explicitly obtained. The diversity gain due to multiple relays is shown through plots of the analytical BER, well supported by simulations. 
%
%\end{abstract}
% IEEEtran.cls defaults to using nonbold math in the Abstract.
% This preserves the distinction between vectors and scalars. However,
% if the journal you are submitting to favors bold math in the abstract,
% then you can use LaTeX's standard command \boldmath at the very start
% of the abstract to achieve this. Many IEEE journals frown on math
% in the abstract anyway.

% Note that keywords are not normally used for peerreview papers.
%\begin{IEEEkeywords}
%Cooperative diversity, decode and forward, piecewise linear
%\end{IEEEkeywords}



% For peer review papers, you can put extra information on the cover
% page as needed:
% \ifCLASSOPTIONpeerreview
% \begin{center} \bfseries EDICS Category: 3-BBND \end{center}
% \fi
%
% For peerreview papers, this IEEEtran command inserts a page break and
% creates the second title. It will be ignored for other modes.
%\IEEEpeerreviewmaketitle




\item A card is drawn from a deck of 52 cards. Find the probability of getting a king or a heart or a red card.\\
\solution
%\begin{table}[H]
	\centering
\begin{tabular}{|c|c|c|}
\hline
Random variable &Value &Definition\\ \hline
\multirow{3}{*}{X} &0 &Slips of Rs 1\\
&1 &Slips of Rs 5\\
&2 &Slips of Rs 13\\ \hline
\multirow{2}{*}{Y} &0 &Box A\\
&1 &Box B\\\hline
\end{tabular}
\caption{}
\label{tab:Distribution}
\end{table}
See \tabref{tab:Distribution}.
\begin{align}
p_{Y}\brak{k}= \begin{cases} 
      \frac{1}{3} & {k=0} \\
      \frac{2}{3 }& {k=1} 
   \end{cases}
   \\
p_{Y|X}\brak{0|0} = \frac{19}{25}\, 
p_{Y|X}\brak{0|1} = \frac{6}{25}\,
p_{Y|X}\brak{1|0} = \frac{45}{50}\,
p_{Y|X}\brak{1|2} = \frac{5}{50}
\end{align}
The desired probability is the probability that a slip drawn at random is marked other than Rs 1,
\begin{align}
&=1-p_X\brak{0}\\
&= p_X(1) + p_X(2)
\end{align}
Using Bayes theorem,
\begin{align}
&= p_Y\brak{0} \times \pr{Y=0 | X=1} + p_Y\brak{1} \times \pr{Y=1|X=2}\\
&=\frac{1}{3} \times \frac{6}{25} + \frac{2}{3} \times \frac{5}{50}\\
&=\frac{11}{75}
\end{align}

\newpage

%\tableofcontents

\bigskip

\renewcommand{\thefigure}{\theenumi}
\renewcommand{\thetable}{\theenumi}
%\renewcommand{\theequation}{\theenumi}

%\begin{abstract}
%%\boldmath
%In this letter, an algorithm for evaluating the exact analytical bit error rate  (BER)  for the piecewise linear (PL) combiner for  multiple relays is presented. Previous results were available only for upto three relays. The algorithm is unique in the sense that  the actual mathematical expressions, that are prohibitively large, need not be explicitly obtained. The diversity gain due to multiple relays is shown through plots of the analytical BER, well supported by simulations. 
%
%\end{abstract}
% IEEEtran.cls defaults to using nonbold math in the Abstract.
% This preserves the distinction between vectors and scalars. However,
% if the journal you are submitting to favors bold math in the abstract,
% then you can use LaTeX's standard command \boldmath at the very start
% of the abstract to achieve this. Many IEEE journals frown on math
% in the abstract anyway.

% Note that keywords are not normally used for peerreview papers.
%\begin{IEEEkeywords}
%Cooperative diversity, decode and forward, piecewise linear
%\end{IEEEkeywords}



% For peer review papers, you can put extra information on the cover
% page as needed:
% \ifCLASSOPTIONpeerreview
% \begin{center} \bfseries EDICS Category: 3-BBND \end{center}
% \fi
%
% For peerreview papers, this IEEEtran command inserts a page break and
% creates the second title. It will be ignored for other modes.
%\IEEEpeerreviewmaketitle




\item The probability that a student will pass his examination is 0.73, the probability of
the student getting a compartment is 0.13, and the probability that the student will
either pass or get compartment is 0.96. State True or False.\\
\solution
%\begin{table}[H]
	\centering
\begin{tabular}{|c|c|c|}
\hline
Random variable &Value &Definition\\ \hline
\multirow{3}{*}{X} &0 &Slips of Rs 1\\
&1 &Slips of Rs 5\\
&2 &Slips of Rs 13\\ \hline
\multirow{2}{*}{Y} &0 &Box A\\
&1 &Box B\\\hline
\end{tabular}
\caption{}
\label{tab:Distribution}
\end{table}
See \tabref{tab:Distribution}.
\begin{align}
p_{Y}\brak{k}= \begin{cases} 
      \frac{1}{3} & {k=0} \\
      \frac{2}{3 }& {k=1} 
   \end{cases}
   \\
p_{Y|X}\brak{0|0} = \frac{19}{25}\, 
p_{Y|X}\brak{0|1} = \frac{6}{25}\,
p_{Y|X}\brak{1|0} = \frac{45}{50}\,
p_{Y|X}\brak{1|2} = \frac{5}{50}
\end{align}
The desired probability is the probability that a slip drawn at random is marked other than Rs 1,
\begin{align}
&=1-p_X\brak{0}\\
&= p_X(1) + p_X(2)
\end{align}
Using Bayes theorem,
\begin{align}
&= p_Y\brak{0} \times \pr{Y=0 | X=1} + p_Y\brak{1} \times \pr{Y=1|X=2}\\
&=\frac{1}{3} \times \frac{6}{25} + \frac{2}{3} \times \frac{5}{50}\\
&=\frac{11}{75}
\end{align}

\newpage

%\tableofcontents

\bigskip

\renewcommand{\thefigure}{\theenumi}
\renewcommand{\thetable}{\theenumi}
%\renewcommand{\theequation}{\theenumi}

%\begin{abstract}
%%\boldmath
%In this letter, an algorithm for evaluating the exact analytical bit error rate  (BER)  for the piecewise linear (PL) combiner for  multiple relays is presented. Previous results were available only for upto three relays. The algorithm is unique in the sense that  the actual mathematical expressions, that are prohibitively large, need not be explicitly obtained. The diversity gain due to multiple relays is shown through plots of the analytical BER, well supported by simulations. 
%
%\end{abstract}
% IEEEtran.cls defaults to using nonbold math in the Abstract.
% This preserves the distinction between vectors and scalars. However,
% if the journal you are submitting to favors bold math in the abstract,
% then you can use LaTeX's standard command \boldmath at the very start
% of the abstract to achieve this. Many IEEE journals frown on math
% in the abstract anyway.

% Note that keywords are not normally used for peerreview papers.
%\begin{IEEEkeywords}
%Cooperative diversity, decode and forward, piecewise linear
%\end{IEEEkeywords}



% For peer review papers, you can put extra information on the cover
% page as needed:
% \ifCLASSOPTIONpeerreview
% \begin{center} \bfseries EDICS Category: 3-BBND \end{center}
% \fi
%
% For peerreview papers, this IEEEtran command inserts a page break and
% creates the second title. It will be ignored for other modes.
%\IEEEpeerreviewmaketitle




\item A card is selected from a pack of 52 cards\\
\begin{enumerate}[label=(\alph*)]
\item How many points are there in the sample space?
\item Calculate the probability that the cards is an ace of spades.
\item Calculate the probability that the card is (i) an ace (ii)black card.\\
\end{enumerate}
%\input{ncert/11/16/3/4_1/Prob_4.tex}
\item In a non-leap year, the probability of having 53 tuesdays or 53 wednesdays is\\
\solution
%A non-leap year has a total of 365 days, and a week has 7 days.\\
So it can be expressed as 
\begin{align}
365\text{days} &=52\times 7+1 \text{day}
\end{align}
$\implies$ 52 tuesdays or wednesdays\\
Random variable X denotes the days of a week
\begin{align}
p_X\brak{k}&=\frac{1}{7}; \quad \brak{1<k<7}
\end{align}
So the probability of extra day being tuesday or wednesday is
\begin{align}
p_X\brak{3}+p_X\brak{4}&=\frac{1}{7}+\frac{1}{7}=\frac{2}{7}
\end{align}



\item There are 1000 sealed envelopes in a box, 10 of them contain a cash prize of
Rs 100 each, 100 of them contain a cash prize of Rs 50 each and 200 of them
contain a cash prize of Rs 10 each and rest do not contain any cash prize. If they
are well shuffled and an envelope is picked up out, what is the probability that it
contains no cash prize?\\
\solution
%\begin{table}[H]
	\centering
\begin{tabular}{|c|c|c|}
\hline
Random variable &Value &Definition\\ \hline
\multirow{3}{*}{X} &0 &Slips of Rs 1\\
&1 &Slips of Rs 5\\
&2 &Slips of Rs 13\\ \hline
\multirow{2}{*}{Y} &0 &Box A\\
&1 &Box B\\\hline
\end{tabular}
\caption{}
\label{tab:Distribution}
\end{table}
See \tabref{tab:Distribution}.
\begin{align}
p_{Y}\brak{k}= \begin{cases} 
      \frac{1}{3} & {k=0} \\
      \frac{2}{3 }& {k=1} 
   \end{cases}
   \\
p_{Y|X}\brak{0|0} = \frac{19}{25}\, 
p_{Y|X}\brak{0|1} = \frac{6}{25}\,
p_{Y|X}\brak{1|0} = \frac{45}{50}\,
p_{Y|X}\brak{1|2} = \frac{5}{50}
\end{align}
The desired probability is the probability that a slip drawn at random is marked other than Rs 1,
\begin{align}
&=1-p_X\brak{0}\\
&= p_X(1) + p_X(2)
\end{align}
Using Bayes theorem,
\begin{align}
&= p_Y\brak{0} \times \pr{Y=0 | X=1} + p_Y\brak{1} \times \pr{Y=1|X=2}\\
&=\frac{1}{3} \times \frac{6}{25} + \frac{2}{3} \times \frac{5}{50}\\
&=\frac{11}{75}
\end{align}

\newpage

%\tableofcontents

\bigskip

\renewcommand{\thefigure}{\theenumi}
\renewcommand{\thetable}{\theenumi}
%\renewcommand{\theequation}{\theenumi}

%\begin{abstract}
%%\boldmath
%In this letter, an algorithm for evaluating the exact analytical bit error rate  (BER)  for the piecewise linear (PL) combiner for  multiple relays is presented. Previous results were available only for upto three relays. The algorithm is unique in the sense that  the actual mathematical expressions, that are prohibitively large, need not be explicitly obtained. The diversity gain due to multiple relays is shown through plots of the analytical BER, well supported by simulations. 
%
%\end{abstract}
% IEEEtran.cls defaults to using nonbold math in the Abstract.
% This preserves the distinction between vectors and scalars. However,
% if the journal you are submitting to favors bold math in the abstract,
% then you can use LaTeX's standard command \boldmath at the very start
% of the abstract to achieve this. Many IEEE journals frown on math
% in the abstract anyway.

% Note that keywords are not normally used for peerreview papers.
%\begin{IEEEkeywords}
%Cooperative diversity, decode and forward, piecewise linear
%\end{IEEEkeywords}



% For peer review papers, you can put extra information on the cover
% page as needed:
% \ifCLASSOPTIONpeerreview
% \begin{center} \bfseries EDICS Category: 3-BBND \end{center}
% \fi
%
% For peerreview papers, this IEEEtran command inserts a page break and
% creates the second title. It will be ignored for other modes.
%\IEEEpeerreviewmaketitle




\item 
A die is thrown and a card is selected at random from a deck of 52 playing cards. The probability of getting an even number on the die and a spade card.\\
\solution
%\begin{table}[H]
	\centering
\begin{tabular}{|c|c|c|}
\hline
Random variable &Value &Definition\\ \hline
\multirow{3}{*}{X} &0 &Slips of Rs 1\\
&1 &Slips of Rs 5\\
&2 &Slips of Rs 13\\ \hline
\multirow{2}{*}{Y} &0 &Box A\\
&1 &Box B\\\hline
\end{tabular}
\caption{}
\label{tab:Distribution}
\end{table}
See \tabref{tab:Distribution}.
\begin{align}
p_{Y}\brak{k}= \begin{cases} 
      \frac{1}{3} & {k=0} \\
      \frac{2}{3 }& {k=1} 
   \end{cases}
   \\
p_{Y|X}\brak{0|0} = \frac{19}{25}\, 
p_{Y|X}\brak{0|1} = \frac{6}{25}\,
p_{Y|X}\brak{1|0} = \frac{45}{50}\,
p_{Y|X}\brak{1|2} = \frac{5}{50}
\end{align}
The desired probability is the probability that a slip drawn at random is marked other than Rs 1,
\begin{align}
&=1-p_X\brak{0}\\
&= p_X(1) + p_X(2)
\end{align}
Using Bayes theorem,
\begin{align}
&= p_Y\brak{0} \times \pr{Y=0 | X=1} + p_Y\brak{1} \times \pr{Y=1|X=2}\\
&=\frac{1}{3} \times \frac{6}{25} + \frac{2}{3} \times \frac{5}{50}\\
&=\frac{11}{75}
\end{align}

\newpage

%\tableofcontents

\bigskip

\renewcommand{\thefigure}{\theenumi}
\renewcommand{\thetable}{\theenumi}
%\renewcommand{\theequation}{\theenumi}

%\begin{abstract}
%%\boldmath
%In this letter, an algorithm for evaluating the exact analytical bit error rate  (BER)  for the piecewise linear (PL) combiner for  multiple relays is presented. Previous results were available only for upto three relays. The algorithm is unique in the sense that  the actual mathematical expressions, that are prohibitively large, need not be explicitly obtained. The diversity gain due to multiple relays is shown through plots of the analytical BER, well supported by simulations. 
%
%\end{abstract}
% IEEEtran.cls defaults to using nonbold math in the Abstract.
% This preserves the distinction between vectors and scalars. However,
% if the journal you are submitting to favors bold math in the abstract,
% then you can use LaTeX's standard command \boldmath at the very start
% of the abstract to achieve this. Many IEEE journals frown on math
% in the abstract anyway.

% Note that keywords are not normally used for peerreview papers.
%\begin{IEEEkeywords}
%Cooperative diversity, decode and forward, piecewise linear
%\end{IEEEkeywords}



% For peer review papers, you can put extra information on the cover
% page as needed:
% \ifCLASSOPTIONpeerreview
% \begin{center} \bfseries EDICS Category: 3-BBND \end{center}
% \fi
%
% For peerreview papers, this IEEEtran command inserts a page break and
% creates the second title. It will be ignored for other modes.
%\IEEEpeerreviewmaketitle




\item
If 4-digit numbers greater than 5,000 are randomly formed from the digits 0, 1, 3, 5, and 7, what is the probability of forming a number divisible by 5 when:
\begin{enumerate}
    \item The digits are repeated?
    \item The repetition of digits is not allowed?
\end{enumerate}
\solution
%\begin{table}[H]
	\centering
\begin{tabular}{|c|c|c|}
\hline
Random variable &Value &Definition\\ \hline
\multirow{3}{*}{X} &0 &Slips of Rs 1\\
&1 &Slips of Rs 5\\
&2 &Slips of Rs 13\\ \hline
\multirow{2}{*}{Y} &0 &Box A\\
&1 &Box B\\\hline
\end{tabular}
\caption{}
\label{tab:Distribution}
\end{table}
See \tabref{tab:Distribution}.
\begin{align}
p_{Y}\brak{k}= \begin{cases} 
      \frac{1}{3} & {k=0} \\
      \frac{2}{3 }& {k=1} 
   \end{cases}
   \\
p_{Y|X}\brak{0|0} = \frac{19}{25}\, 
p_{Y|X}\brak{0|1} = \frac{6}{25}\,
p_{Y|X}\brak{1|0} = \frac{45}{50}\,
p_{Y|X}\brak{1|2} = \frac{5}{50}
\end{align}
The desired probability is the probability that a slip drawn at random is marked other than Rs 1,
\begin{align}
&=1-p_X\brak{0}\\
&= p_X(1) + p_X(2)
\end{align}
Using Bayes theorem,
\begin{align}
&= p_Y\brak{0} \times \pr{Y=0 | X=1} + p_Y\brak{1} \times \pr{Y=1|X=2}\\
&=\frac{1}{3} \times \frac{6}{25} + \frac{2}{3} \times \frac{5}{50}\\
&=\frac{11}{75}
\end{align}

\newpage

%\tableofcontents

\bigskip

\renewcommand{\thefigure}{\theenumi}
\renewcommand{\thetable}{\theenumi}
%\renewcommand{\theequation}{\theenumi}

%\begin{abstract}
%%\boldmath
%In this letter, an algorithm for evaluating the exact analytical bit error rate  (BER)  for the piecewise linear (PL) combiner for  multiple relays is presented. Previous results were available only for upto three relays. The algorithm is unique in the sense that  the actual mathematical expressions, that are prohibitively large, need not be explicitly obtained. The diversity gain due to multiple relays is shown through plots of the analytical BER, well supported by simulations. 
%
%\end{abstract}
% IEEEtran.cls defaults to using nonbold math in the Abstract.
% This preserves the distinction between vectors and scalars. However,
% if the journal you are submitting to favors bold math in the abstract,
% then you can use LaTeX's standard command \boldmath at the very start
% of the abstract to achieve this. Many IEEE journals frown on math
% in the abstract anyway.

% Note that keywords are not normally used for peerreview papers.
%\begin{IEEEkeywords}
%Cooperative diversity, decode and forward, piecewise linear
%\end{IEEEkeywords}



% For peer review papers, you can put extra information on the cover
% page as needed:
% \ifCLASSOPTIONpeerreview
% \begin{center} \bfseries EDICS Category: 3-BBND \end{center}
% \fi
%
% For peerreview papers, this IEEEtran command inserts a page break and
% creates the second title. It will be ignored for other modes.
%\IEEEpeerreviewmaketitle




\item Consider the probability space $\brak{\Omega, \mathcal{G}, P}$ where $\Omega = [0,2]$ and $\mathcal{G} = \cbrak{\phi, \Omega, [0,1], (1,2]}$. Let $X$ and $Y$ be two functions on $\Omega$ defined as
\begin{align*}
    X(\omega) = 
    \begin{cases}
        1 & \text{if }\omega \in [0, 1]\\
        2 & \text{if }\omega \in (1, 2]
    \end{cases}
\end{align*}
and
\begin{align*}
    Y(\omega) = 
    \begin{cases}
        2 & \text{if }\omega \in [0, 1.5]\\
        3 & \text{if }\omega \in (1.5, 2].
    \end{cases}
\end{align*}
Then which one of the following statements is true?
\begin{enumerate}
    \item [(A)] $X$ is a random variable with respect to $\mathcal{G}$, but $Y$ is not a random variable with respect to $\mathcal{G}$.
    \item [(B)] $Y$ is a random variable with respect to $\mathcal{G}$, but $X$ is not a random variable with respect to $\mathcal{G}$.
    \item [(C)] Neither $X$ nor $Y$ is a random variable with respect to $\mathcal{G}$.
    \item [(D)] Both $X$ and $Y$ are random variables with respect to $\mathcal{G}$.
\end{enumerate} \hfill (GATE ST 2023)\\
\solution
%\begin{table}[H]
	\centering
\begin{tabular}{|c|c|c|}
\hline
Random variable &Value &Definition\\ \hline
\multirow{3}{*}{X} &0 &Slips of Rs 1\\
&1 &Slips of Rs 5\\
&2 &Slips of Rs 13\\ \hline
\multirow{2}{*}{Y} &0 &Box A\\
&1 &Box B\\\hline
\end{tabular}
\caption{}
\label{tab:Distribution}
\end{table}
See \tabref{tab:Distribution}.
\begin{align}
p_{Y}\brak{k}= \begin{cases} 
      \frac{1}{3} & {k=0} \\
      \frac{2}{3 }& {k=1} 
   \end{cases}
   \\
p_{Y|X}\brak{0|0} = \frac{19}{25}\, 
p_{Y|X}\brak{0|1} = \frac{6}{25}\,
p_{Y|X}\brak{1|0} = \frac{45}{50}\,
p_{Y|X}\brak{1|2} = \frac{5}{50}
\end{align}
The desired probability is the probability that a slip drawn at random is marked other than Rs 1,
\begin{align}
&=1-p_X\brak{0}\\
&= p_X(1) + p_X(2)
\end{align}
Using Bayes theorem,
\begin{align}
&= p_Y\brak{0} \times \pr{Y=0 | X=1} + p_Y\brak{1} \times \pr{Y=1|X=2}\\
&=\frac{1}{3} \times \frac{6}{25} + \frac{2}{3} \times \frac{5}{50}\\
&=\frac{11}{75}
\end{align}

\newpage

%\tableofcontents

\bigskip

\renewcommand{\thefigure}{\theenumi}
\renewcommand{\thetable}{\theenumi}
%\renewcommand{\theequation}{\theenumi}

%\begin{abstract}
%%\boldmath
%In this letter, an algorithm for evaluating the exact analytical bit error rate  (BER)  for the piecewise linear (PL) combiner for  multiple relays is presented. Previous results were available only for upto three relays. The algorithm is unique in the sense that  the actual mathematical expressions, that are prohibitively large, need not be explicitly obtained. The diversity gain due to multiple relays is shown through plots of the analytical BER, well supported by simulations. 
%
%\end{abstract}
% IEEEtran.cls defaults to using nonbold math in the Abstract.
% This preserves the distinction between vectors and scalars. However,
% if the journal you are submitting to favors bold math in the abstract,
% then you can use LaTeX's standard command \boldmath at the very start
% of the abstract to achieve this. Many IEEE journals frown on math
% in the abstract anyway.

% Note that keywords are not normally used for peerreview papers.
%\begin{IEEEkeywords}
%Cooperative diversity, decode and forward, piecewise linear
%\end{IEEEkeywords}



% For peer review papers, you can put extra information on the cover
% page as needed:
% \ifCLASSOPTIONpeerreview
% \begin{center} \bfseries EDICS Category: 3-BBND \end{center}
% \fi
%
% For peerreview papers, this IEEEtran command inserts a page break and
% creates the second title. It will be ignored for other modes.
%\IEEEpeerreviewmaketitle




	\item  A die is loaded in such a way that each odd number is twice as likely to occur as
each even number. Find $P(G)$, where $G$ is the event that a number greater than
3 occurs on a single roll of the die.
\\
\solution
		%\begin{table}[H]
	\centering
\begin{tabular}{|c|c|c|}
\hline
Random variable &Value &Definition\\ \hline
\multirow{3}{*}{X} &0 &Slips of Rs 1\\
&1 &Slips of Rs 5\\
&2 &Slips of Rs 13\\ \hline
\multirow{2}{*}{Y} &0 &Box A\\
&1 &Box B\\\hline
\end{tabular}
\caption{}
\label{tab:Distribution}
\end{table}
See \tabref{tab:Distribution}.
\begin{align}
p_{Y}\brak{k}= \begin{cases} 
      \frac{1}{3} & {k=0} \\
      \frac{2}{3 }& {k=1} 
   \end{cases}
   \\
p_{Y|X}\brak{0|0} = \frac{19}{25}\, 
p_{Y|X}\brak{0|1} = \frac{6}{25}\,
p_{Y|X}\brak{1|0} = \frac{45}{50}\,
p_{Y|X}\brak{1|2} = \frac{5}{50}
\end{align}
The desired probability is the probability that a slip drawn at random is marked other than Rs 1,
\begin{align}
&=1-p_X\brak{0}\\
&= p_X(1) + p_X(2)
\end{align}
Using Bayes theorem,
\begin{align}
&= p_Y\brak{0} \times \pr{Y=0 | X=1} + p_Y\brak{1} \times \pr{Y=1|X=2}\\
&=\frac{1}{3} \times \frac{6}{25} + \frac{2}{3} \times \frac{5}{50}\\
&=\frac{11}{75}
\end{align}

\newpage

%\tableofcontents

\bigskip

\renewcommand{\thefigure}{\theenumi}
\renewcommand{\thetable}{\theenumi}
%\renewcommand{\theequation}{\theenumi}

%\begin{abstract}
%%\boldmath
%In this letter, an algorithm for evaluating the exact analytical bit error rate  (BER)  for the piecewise linear (PL) combiner for  multiple relays is presented. Previous results were available only for upto three relays. The algorithm is unique in the sense that  the actual mathematical expressions, that are prohibitively large, need not be explicitly obtained. The diversity gain due to multiple relays is shown through plots of the analytical BER, well supported by simulations. 
%
%\end{abstract}
% IEEEtran.cls defaults to using nonbold math in the Abstract.
% This preserves the distinction between vectors and scalars. However,
% if the journal you are submitting to favors bold math in the abstract,
% then you can use LaTeX's standard command \boldmath at the very start
% of the abstract to achieve this. Many IEEE journals frown on math
% in the abstract anyway.

% Note that keywords are not normally used for peerreview papers.
%\begin{IEEEkeywords}
%Cooperative diversity, decode and forward, piecewise linear
%\end{IEEEkeywords}



% For peer review papers, you can put extra information on the cover
% page as needed:
% \ifCLASSOPTIONpeerreview
% \begin{center} \bfseries EDICS Category: 3-BBND \end{center}
% \fi
%
% For peerreview papers, this IEEEtran command inserts a page break and
% creates the second title. It will be ignored for other modes.
%\IEEEpeerreviewmaketitle




	\item All the jacks, queens and kings are removed from a deck of 52 playing cards. The remaining cards are well shuffled and then one card is drawn at random. Giving ace a value 1 similar value for other cards, find the probability that the card has a value 
		\begin{enumerate}
			\item 7
			\item greater than 7
			\item less than 7
		\end{enumerate}
		%Number of cards left after removing all jacks, queens and kings 
\begin{align}
N	= 52 - 4\times 3
	= 40
\end{align}
%\begin{table}[H]
%\def\arraystretch{1.2}
%\begin{tabular}{|c|c|c|}
%\hline
%	\textbf{Parameter} &\textbf{Value} &\textbf{Description}\\ \hline
%	$X$ &1-10 &Represents the value of the card picked \\ \hline
%\end{tabular}
%\end{table}
Let $1 \le X \le 10$ be the value of the card picked.  Then,
\begin{align}
	p_X(k) &= \Pr(X=k)\ \forall\ 1 \leq k \leq 10\\
	&= \frac{4\times 1}{40}\\
	&= \frac{1}{10}\\
	\therefore p_X(k) &= 
	\begin{cases}
		\frac{1}{10} & 1 \leq k \leq 10\\
		0 & \text{otherwise}
	\end{cases}
\end{align}
and
\begin{align}
	F_{X}(k) &= \sum_{m=0}^{k}p_{X}(m) \quad 1 \leq k \leq 10\\
	&= \frac{k}{10}\\
	\therefore F_{X}(k) &= 
	\begin{cases}
		0 & k \leq 0\\
		\frac{k}{10} & 1\leq k \leq 10\\
		1 & k > 10 
	\end{cases}
\end{align}
\begin{enumerate}
	\item Probability that card has value equal to 7 is
		\begin{align}
			 p_{X}(7)
			= \frac{1}{10}
		\end{align}
	\item Probability that card has value greater than 7 is
		\begin{align}
			1 - F_X(7)
			&= 1 - \frac{7}{10}
			\\
			&= \frac{3}{10}
		\end{align}
	\item Probability that card has value less than 7 is
		\begin{align}
			 F_{X}(6)
			=\frac{6}{10}
		\end{align}
\end{enumerate}

  \item A Lot consists of 48 mobile phones of which 42 are good, 3 have only minor defects and 3 have major defects.Varnika will buy a phone if it is good but the trader will only buy a mobile if it has no major defects. One phone is selected at random from the lot. What is the probability that it is
\begin{enumerate}
	\item acceptable to Varnika?
            \item acceptable to the trader?
\end{enumerate}
\solution
	%\begin{table}[H]
	\centering
\begin{tabular}{|c|c|c|}
\hline
Random variable &Value &Definition\\ \hline
\multirow{3}{*}{X} &0 &Slips of Rs 1\\
&1 &Slips of Rs 5\\
&2 &Slips of Rs 13\\ \hline
\multirow{2}{*}{Y} &0 &Box A\\
&1 &Box B\\\hline
\end{tabular}
\caption{}
\label{tab:Distribution}
\end{table}
See \tabref{tab:Distribution}.
\begin{align}
p_{Y}\brak{k}= \begin{cases} 
      \frac{1}{3} & {k=0} \\
      \frac{2}{3 }& {k=1} 
   \end{cases}
   \\
p_{Y|X}\brak{0|0} = \frac{19}{25}\, 
p_{Y|X}\brak{0|1} = \frac{6}{25}\,
p_{Y|X}\brak{1|0} = \frac{45}{50}\,
p_{Y|X}\brak{1|2} = \frac{5}{50}
\end{align}
The desired probability is the probability that a slip drawn at random is marked other than Rs 1,
\begin{align}
&=1-p_X\brak{0}\\
&= p_X(1) + p_X(2)
\end{align}
Using Bayes theorem,
\begin{align}
&= p_Y\brak{0} \times \pr{Y=0 | X=1} + p_Y\brak{1} \times \pr{Y=1|X=2}\\
&=\frac{1}{3} \times \frac{6}{25} + \frac{2}{3} \times \frac{5}{50}\\
&=\frac{11}{75}
\end{align}

\newpage

%\tableofcontents

\bigskip

\renewcommand{\thefigure}{\theenumi}
\renewcommand{\thetable}{\theenumi}
%\renewcommand{\theequation}{\theenumi}

%\begin{abstract}
%%\boldmath
%In this letter, an algorithm for evaluating the exact analytical bit error rate  (BER)  for the piecewise linear (PL) combiner for  multiple relays is presented. Previous results were available only for upto three relays. The algorithm is unique in the sense that  the actual mathematical expressions, that are prohibitively large, need not be explicitly obtained. The diversity gain due to multiple relays is shown through plots of the analytical BER, well supported by simulations. 
%
%\end{abstract}
% IEEEtran.cls defaults to using nonbold math in the Abstract.
% This preserves the distinction between vectors and scalars. However,
% if the journal you are submitting to favors bold math in the abstract,
% then you can use LaTeX's standard command \boldmath at the very start
% of the abstract to achieve this. Many IEEE journals frown on math
% in the abstract anyway.

% Note that keywords are not normally used for peerreview papers.
%\begin{IEEEkeywords}
%Cooperative diversity, decode and forward, piecewise linear
%\end{IEEEkeywords}



% For peer review papers, you can put extra information on the cover
% page as needed:
% \ifCLASSOPTIONpeerreview
% \begin{center} \bfseries EDICS Category: 3-BBND \end{center}
% \fi
%
% For peerreview papers, this IEEEtran command inserts a page break and
% creates the second title. It will be ignored for other modes.
%\IEEEpeerreviewmaketitle




 \item A student says that if you throw a die, it will show up 1 or not 1. Therefore, the probability of getting 1 and the probability of getting 'not 1' each is equal to $\frac{1}{2}$. Is this correct? Give reasons.\\
 \solution
        %\begin{table}[H]
	\centering
\begin{tabular}{|c|c|c|}
\hline
Random variable &Value &Definition\\ \hline
\multirow{3}{*}{X} &0 &Slips of Rs 1\\
&1 &Slips of Rs 5\\
&2 &Slips of Rs 13\\ \hline
\multirow{2}{*}{Y} &0 &Box A\\
&1 &Box B\\\hline
\end{tabular}
\caption{}
\label{tab:Distribution}
\end{table}
See \tabref{tab:Distribution}.
\begin{align}
p_{Y}\brak{k}= \begin{cases} 
      \frac{1}{3} & {k=0} \\
      \frac{2}{3 }& {k=1} 
   \end{cases}
   \\
p_{Y|X}\brak{0|0} = \frac{19}{25}\, 
p_{Y|X}\brak{0|1} = \frac{6}{25}\,
p_{Y|X}\brak{1|0} = \frac{45}{50}\,
p_{Y|X}\brak{1|2} = \frac{5}{50}
\end{align}
The desired probability is the probability that a slip drawn at random is marked other than Rs 1,
\begin{align}
&=1-p_X\brak{0}\\
&= p_X(1) + p_X(2)
\end{align}
Using Bayes theorem,
\begin{align}
&= p_Y\brak{0} \times \pr{Y=0 | X=1} + p_Y\brak{1} \times \pr{Y=1|X=2}\\
&=\frac{1}{3} \times \frac{6}{25} + \frac{2}{3} \times \frac{5}{50}\\
&=\frac{11}{75}
\end{align}

\newpage

%\tableofcontents

\bigskip

\renewcommand{\thefigure}{\theenumi}
\renewcommand{\thetable}{\theenumi}
%\renewcommand{\theequation}{\theenumi}

%\begin{abstract}
%%\boldmath
%In this letter, an algorithm for evaluating the exact analytical bit error rate  (BER)  for the piecewise linear (PL) combiner for  multiple relays is presented. Previous results were available only for upto three relays. The algorithm is unique in the sense that  the actual mathematical expressions, that are prohibitively large, need not be explicitly obtained. The diversity gain due to multiple relays is shown through plots of the analytical BER, well supported by simulations. 
%
%\end{abstract}
% IEEEtran.cls defaults to using nonbold math in the Abstract.
% This preserves the distinction between vectors and scalars. However,
% if the journal you are submitting to favors bold math in the abstract,
% then you can use LaTeX's standard command \boldmath at the very start
% of the abstract to achieve this. Many IEEE journals frown on math
% in the abstract anyway.

% Note that keywords are not normally used for peerreview papers.
%\begin{IEEEkeywords}
%Cooperative diversity, decode and forward, piecewise linear
%\end{IEEEkeywords}



% For peer review papers, you can put extra information on the cover
% page as needed:
% \ifCLASSOPTIONpeerreview
% \begin{center} \bfseries EDICS Category: 3-BBND \end{center}
% \fi
%
% For peerreview papers, this IEEEtran command inserts a page break and
% creates the second title. It will be ignored for other modes.
%\IEEEpeerreviewmaketitle




   \item Four candidates A, B, C, D have ap-
plied for the assignment to coach a school cricket
team. If A is twice as likely to be selected as B, and
B and C are given about the same chance of being
selected, while C is twice as likely to be selected
as D, what are the probabilities that
\begin{enumerate}
\item C will be selected?
\item A will not be selected?
\end{enumerate}
	%\begin{table}[H]
	\centering
\begin{tabular}{|c|c|c|}
\hline
Random variable &Value &Definition\\ \hline
\multirow{3}{*}{X} &0 &Slips of Rs 1\\
&1 &Slips of Rs 5\\
&2 &Slips of Rs 13\\ \hline
\multirow{2}{*}{Y} &0 &Box A\\
&1 &Box B\\\hline
\end{tabular}
\caption{}
\label{tab:Distribution}
\end{table}
See \tabref{tab:Distribution}.
\begin{align}
p_{Y}\brak{k}= \begin{cases} 
      \frac{1}{3} & {k=0} \\
      \frac{2}{3 }& {k=1} 
   \end{cases}
   \\
p_{Y|X}\brak{0|0} = \frac{19}{25}\, 
p_{Y|X}\brak{0|1} = \frac{6}{25}\,
p_{Y|X}\brak{1|0} = \frac{45}{50}\,
p_{Y|X}\brak{1|2} = \frac{5}{50}
\end{align}
The desired probability is the probability that a slip drawn at random is marked other than Rs 1,
\begin{align}
&=1-p_X\brak{0}\\
&= p_X(1) + p_X(2)
\end{align}
Using Bayes theorem,
\begin{align}
&= p_Y\brak{0} \times \pr{Y=0 | X=1} + p_Y\brak{1} \times \pr{Y=1|X=2}\\
&=\frac{1}{3} \times \frac{6}{25} + \frac{2}{3} \times \frac{5}{50}\\
&=\frac{11}{75}
\end{align}

\newpage

%\tableofcontents

\bigskip

\renewcommand{\thefigure}{\theenumi}
\renewcommand{\thetable}{\theenumi}
%\renewcommand{\theequation}{\theenumi}

%\begin{abstract}
%%\boldmath
%In this letter, an algorithm for evaluating the exact analytical bit error rate  (BER)  for the piecewise linear (PL) combiner for  multiple relays is presented. Previous results were available only for upto three relays. The algorithm is unique in the sense that  the actual mathematical expressions, that are prohibitively large, need not be explicitly obtained. The diversity gain due to multiple relays is shown through plots of the analytical BER, well supported by simulations. 
%
%\end{abstract}
% IEEEtran.cls defaults to using nonbold math in the Abstract.
% This preserves the distinction between vectors and scalars. However,
% if the journal you are submitting to favors bold math in the abstract,
% then you can use LaTeX's standard command \boldmath at the very start
% of the abstract to achieve this. Many IEEE journals frown on math
% in the abstract anyway.

% Note that keywords are not normally used for peerreview papers.
%\begin{IEEEkeywords}
%Cooperative diversity, decode and forward, piecewise linear
%\end{IEEEkeywords}



% For peer review papers, you can put extra information on the cover
% page as needed:
% \ifCLASSOPTIONpeerreview
% \begin{center} \bfseries EDICS Category: 3-BBND \end{center}
% \fi
%
% For peerreview papers, this IEEEtran command inserts a page break and
% creates the second title. It will be ignored for other modes.
%\IEEEpeerreviewmaketitle




 \item A bag contain 24 balls of which $x$ balls are red, $2x$ are white and $3x$ are blue. A ball is selected at random, What is the probability that it is
\begin{enumerate}[label=\alph*)]
\item not red ?
\item white ?
\end{enumerate}
%\begin{table}[H]
	\centering
\begin{tabular}{|c|c|c|}
\hline
Random variable &Value &Definition\\ \hline
\multirow{3}{*}{X} &0 &Slips of Rs 1\\
&1 &Slips of Rs 5\\
&2 &Slips of Rs 13\\ \hline
\multirow{2}{*}{Y} &0 &Box A\\
&1 &Box B\\\hline
\end{tabular}
\caption{}
\label{tab:Distribution}
\end{table}
See \tabref{tab:Distribution}.
\begin{align}
p_{Y}\brak{k}= \begin{cases} 
      \frac{1}{3} & {k=0} \\
      \frac{2}{3 }& {k=1} 
   \end{cases}
   \\
p_{Y|X}\brak{0|0} = \frac{19}{25}\, 
p_{Y|X}\brak{0|1} = \frac{6}{25}\,
p_{Y|X}\brak{1|0} = \frac{45}{50}\,
p_{Y|X}\brak{1|2} = \frac{5}{50}
\end{align}
The desired probability is the probability that a slip drawn at random is marked other than Rs 1,
\begin{align}
&=1-p_X\brak{0}\\
&= p_X(1) + p_X(2)
\end{align}
Using Bayes theorem,
\begin{align}
&= p_Y\brak{0} \times \pr{Y=0 | X=1} + p_Y\brak{1} \times \pr{Y=1|X=2}\\
&=\frac{1}{3} \times \frac{6}{25} + \frac{2}{3} \times \frac{5}{50}\\
&=\frac{11}{75}
\end{align}

\newpage

%\tableofcontents

\bigskip

\renewcommand{\thefigure}{\theenumi}
\renewcommand{\thetable}{\theenumi}
%\renewcommand{\theequation}{\theenumi}

%\begin{abstract}
%%\boldmath
%In this letter, an algorithm for evaluating the exact analytical bit error rate  (BER)  for the piecewise linear (PL) combiner for  multiple relays is presented. Previous results were available only for upto three relays. The algorithm is unique in the sense that  the actual mathematical expressions, that are prohibitively large, need not be explicitly obtained. The diversity gain due to multiple relays is shown through plots of the analytical BER, well supported by simulations. 
%
%\end{abstract}
% IEEEtran.cls defaults to using nonbold math in the Abstract.
% This preserves the distinction between vectors and scalars. However,
% if the journal you are submitting to favors bold math in the abstract,
% then you can use LaTeX's standard command \boldmath at the very start
% of the abstract to achieve this. Many IEEE journals frown on math
% in the abstract anyway.

% Note that keywords are not normally used for peerreview papers.
%\begin{IEEEkeywords}
%Cooperative diversity, decode and forward, piecewise linear
%\end{IEEEkeywords}



% For peer review papers, you can put extra information on the cover
% page as needed:
% \ifCLASSOPTIONpeerreview
% \begin{center} \bfseries EDICS Category: 3-BBND \end{center}
% \fi
%
% For peerreview papers, this IEEEtran command inserts a page break and
% creates the second title. It will be ignored for other modes.
%\IEEEpeerreviewmaketitle




If the letters of the word ASSASSINATION are arranged at random. Find the Probability that
\begin{enumerate}[label=(\alph*)]
\item Four $S's$ come consecutively in the word
\item Two  $I's$ and two $N's$ come together
\item All $A's$ are not coming together
\item No two $A's$ are coming together
\end{enumerate}
%\begin{table}[H]
	\centering
\begin{tabular}{|c|c|c|}
\hline
Random variable &Value &Definition\\ \hline
\multirow{3}{*}{X} &0 &Slips of Rs 1\\
&1 &Slips of Rs 5\\
&2 &Slips of Rs 13\\ \hline
\multirow{2}{*}{Y} &0 &Box A\\
&1 &Box B\\\hline
\end{tabular}
\caption{}
\label{tab:Distribution}
\end{table}
See \tabref{tab:Distribution}.
\begin{align}
p_{Y}\brak{k}= \begin{cases} 
      \frac{1}{3} & {k=0} \\
      \frac{2}{3 }& {k=1} 
   \end{cases}
   \\
p_{Y|X}\brak{0|0} = \frac{19}{25}\, 
p_{Y|X}\brak{0|1} = \frac{6}{25}\,
p_{Y|X}\brak{1|0} = \frac{45}{50}\,
p_{Y|X}\brak{1|2} = \frac{5}{50}
\end{align}
The desired probability is the probability that a slip drawn at random is marked other than Rs 1,
\begin{align}
&=1-p_X\brak{0}\\
&= p_X(1) + p_X(2)
\end{align}
Using Bayes theorem,
\begin{align}
&= p_Y\brak{0} \times \pr{Y=0 | X=1} + p_Y\brak{1} \times \pr{Y=1|X=2}\\
&=\frac{1}{3} \times \frac{6}{25} + \frac{2}{3} \times \frac{5}{50}\\
&=\frac{11}{75}
\end{align}

\newpage

%\tableofcontents

\bigskip

\renewcommand{\thefigure}{\theenumi}
\renewcommand{\thetable}{\theenumi}
%\renewcommand{\theequation}{\theenumi}

%\begin{abstract}
%%\boldmath
%In this letter, an algorithm for evaluating the exact analytical bit error rate  (BER)  for the piecewise linear (PL) combiner for  multiple relays is presented. Previous results were available only for upto three relays. The algorithm is unique in the sense that  the actual mathematical expressions, that are prohibitively large, need not be explicitly obtained. The diversity gain due to multiple relays is shown through plots of the analytical BER, well supported by simulations. 
%
%\end{abstract}
% IEEEtran.cls defaults to using nonbold math in the Abstract.
% This preserves the distinction between vectors and scalars. However,
% if the journal you are submitting to favors bold math in the abstract,
% then you can use LaTeX's standard command \boldmath at the very start
% of the abstract to achieve this. Many IEEE journals frown on math
% in the abstract anyway.

% Note that keywords are not normally used for peerreview papers.
%\begin{IEEEkeywords}
%Cooperative diversity, decode and forward, piecewise linear
%\end{IEEEkeywords}



% For peer review papers, you can put extra information on the cover
% page as needed:
% \ifCLASSOPTIONpeerreview
% \begin{center} \bfseries EDICS Category: 3-BBND \end{center}
% \fi
%
% For peerreview papers, this IEEEtran command inserts a page break and
% creates the second title. It will be ignored for other modes.
%\IEEEpeerreviewmaketitle




	\item One urn contains two black balls (labelled B1 and B2) and one white ball. A
	second urn contains one black ball and two white balls (labelled W1 and W2).
	Suppose the following experiment is performed. One of the two urns is chosen
	at random. Next a ball is randomly chosen from the urn. Then a second ball is
	chosen at random from the same urn without replacing the first ball.
	
	\begin{enumerate}
	\item What is the probability that two black balls are chosen?
	
	\item What is the probability that two balls of opposite colour are chosen?
	\end{enumerate}
	\solution
	%\begin{align}
    \label{eq:12.13.6.18.1}
	\because	\pr{A|B} &> \pr{A},\
\frac{\pr{AB}}{\pr{B}} > \pr{A}
\\
    \label{eq:12.13.6.18.2}
	\implies \pr{AB} &> \pr{A}\pr{B}
	\\
	\text{or, } \frac{\pr{AB}}{\pr{A}} &=\pr{B|A} > \pr{A}
\end{align}

\end{enumerate}

	\item 
The number lock of a suitcase has 4 wheels each labelled with ten digits i.e. from 0 to 9.The lock opens with a sequence of four digits with no repeats.What is the probability of a person getting the right sequence to open the suitcase.
\\
\solution
		%\begin{enumerate}[label=\thesection.\arabic*,ref=\thesection.\theenumi]
	\item One card is drawn from a well-shuffled deck of 52 cards. Find the probability of getting
\begin{enumerate}
\item A king of red colour 
\item A face card 
\item A red face card
\item The jack of hearts
\item A spade
\item The queen of diamonds

\end{enumerate}
\solution
		%\begin{table}[H]
	\centering
\begin{tabular}{|c|c|c|}
\hline
Random variable &Value &Definition\\ \hline
\multirow{3}{*}{X} &0 &Slips of Rs 1\\
&1 &Slips of Rs 5\\
&2 &Slips of Rs 13\\ \hline
\multirow{2}{*}{Y} &0 &Box A\\
&1 &Box B\\\hline
\end{tabular}
\caption{}
\label{tab:Distribution}
\end{table}
See \tabref{tab:Distribution}.
\begin{align}
p_{Y}\brak{k}= \begin{cases} 
      \frac{1}{3} & {k=0} \\
      \frac{2}{3 }& {k=1} 
   \end{cases}
   \\
p_{Y|X}\brak{0|0} = \frac{19}{25}\, 
p_{Y|X}\brak{0|1} = \frac{6}{25}\,
p_{Y|X}\brak{1|0} = \frac{45}{50}\,
p_{Y|X}\brak{1|2} = \frac{5}{50}
\end{align}
The desired probability is the probability that a slip drawn at random is marked other than Rs 1,
\begin{align}
&=1-p_X\brak{0}\\
&= p_X(1) + p_X(2)
\end{align}
Using Bayes theorem,
\begin{align}
&= p_Y\brak{0} \times \pr{Y=0 | X=1} + p_Y\brak{1} \times \pr{Y=1|X=2}\\
&=\frac{1}{3} \times \frac{6}{25} + \frac{2}{3} \times \frac{5}{50}\\
&=\frac{11}{75}
\end{align}

\newpage

%\tableofcontents

\bigskip

\renewcommand{\thefigure}{\theenumi}
\renewcommand{\thetable}{\theenumi}
%\renewcommand{\theequation}{\theenumi}

%\begin{abstract}
%%\boldmath
%In this letter, an algorithm for evaluating the exact analytical bit error rate  (BER)  for the piecewise linear (PL) combiner for  multiple relays is presented. Previous results were available only for upto three relays. The algorithm is unique in the sense that  the actual mathematical expressions, that are prohibitively large, need not be explicitly obtained. The diversity gain due to multiple relays is shown through plots of the analytical BER, well supported by simulations. 
%
%\end{abstract}
% IEEEtran.cls defaults to using nonbold math in the Abstract.
% This preserves the distinction between vectors and scalars. However,
% if the journal you are submitting to favors bold math in the abstract,
% then you can use LaTeX's standard command \boldmath at the very start
% of the abstract to achieve this. Many IEEE journals frown on math
% in the abstract anyway.

% Note that keywords are not normally used for peerreview papers.
%\begin{IEEEkeywords}
%Cooperative diversity, decode and forward, piecewise linear
%\end{IEEEkeywords}



% For peer review papers, you can put extra information on the cover
% page as needed:
% \ifCLASSOPTIONpeerreview
% \begin{center} \bfseries EDICS Category: 3-BBND \end{center}
% \fi
%
% For peerreview papers, this IEEEtran command inserts a page break and
% creates the second title. It will be ignored for other modes.
%\IEEEpeerreviewmaketitle




	\item Five cards—the ten, jack, queen, king and ace of diamonds, are well-shuffled with their face downwards. One card is then picked up at random.
\begin{enumerate}
\item
What is the probability that the card is the queen? 
\item
If the queen is drawn and put aside, what is the probability that the second card picked up is (a) an ace? (b) a queen?\\
\end{enumerate}
\solution
		%\begin{enumerate}[label=\thesection.\arabic*,ref=\thesection.\theenumi]
	\item One card is drawn from a well-shuffled deck of 52 cards. Find the probability of getting
\begin{enumerate}
\item A king of red colour 
\item A face card 
\item A red face card
\item The jack of hearts
\item A spade
\item The queen of diamonds

\end{enumerate}
\solution
		%\input{ncert/10/15/1/14/main.tex}
	\item Five cards—the ten, jack, queen, king and ace of diamonds, are well-shuffled with their face downwards. One card is then picked up at random.
\begin{enumerate}
\item
What is the probability that the card is the queen? 
\item
If the queen is drawn and put aside, what is the probability that the second card picked up is (a) an ace? (b) a queen?\\
\end{enumerate}
\solution
		%\input{ncert/10/15/1/15/defs.tex}
	\item A bag contains $5$ red balls and some blue balls. If the probability of drawing a blue ball is double that if a red ball, determine the number of blue balls in the bag. 
		\\
\solution
		%\input{ncert/10/15/2/3/defs.tex}
	\item A card is selected from a pack of 52 cards.
 \begin{enumerate}[label=(\alph*)] 
                 \item How many points are there in the sample space?
                 \item Calculate the probability that the card is an ace of spades.
                 \item Calculate the probability that the card is (i) an ace and (ii) black card.
 \end{enumerate}
\solution
		%\input{ncert/11/16/3/4/main.tex}
\item Four cards are drawn from a well-shuffled deck of 52 cards. What is the probability of obtaining 3 diamonds and one spade.
\\
\solution
		%\input{ncert/11/16/4/2/defs.tex}
\item In a certain lottery 10,000 tickets are sold and ten equal prizes are awarded. What is the probability of not getting a prize if you buy (a) one ticket (b) two tickets (c) 10 tickets ?	
\\
\solution
		%\input{ncert/11/16/4/4/defs.tex}
		%
\item 
Out of 100 students, two sections of 40 and 60 are formed. If you and your friend are among the 100 students, what is the probability that
\begin{enumerate}
\item you both enter the same section?
\item you both enter the different sections?
\end{enumerate}
\solution
		%\input{ncert/11/16/4/5/defs.tex}
	\item 
The number lock of a suitcase has 4 wheels each labelled with ten digits i.e. from 0 to 9.The lock opens with a sequence of four digits with no repeats.What is the probability of a person getting the right sequence to open the suitcase.
\\
\solution
		%\input{ncert/11/16/4/10/defs.tex}
		%
\item 
Two cards are drawn at random and without replacement from a pack of 52 playing cards. Find the probability that both the cards are black.
\\
\solution
		%\input{ncert/12/13/2/2/defs.tex}
		\item A box of oranges is inspected by examining three randomly selected oranges drawn without replacement. If all the three oranges are good, the box is approved for sale, otherwise, it is rejected. Find the probability that a box containing 15 oranges out of which 12 are good and 3 are bad ones will be approved for sale.
		\label{ncert/12/13/2/3/defs.tex}
		\item Two balls are drawn at random with replacement from a box containing 10 black and 8 red balls. Find the probability that
		\label{ncert/12/13/2/12}
\begin{enumerate}
\item both balls are red.
\item first ball is black and second is red.
\item one of them is black and other is red.
\end{enumerate}

\item In a hostel, 60\% of the students read Hindi newspaper, 40\% read English newspaper and 20\% read both Hindi and English newspapers. A student is selected at random.
		\label{ncert/12/13/2/15}
\begin{enumerate}
\item Find the probability that she reads neither Hindi nor English newspapers.
\item If she reads Hindi newspaper, find the probability that she reads English newspaper.
\item If she reads English newspaper, find the probability that she reads Hindi newspaper.\\
\end{enumerate}
\item The probability of obtaining an even prime number on each die, when a pair of dice is rolled is 
\begin{enumerate}
    \item $0$ 
    
    \item $\frac{1}{3}$ 
    
    \item $\frac{1}{12}$ 
    
    \item $\frac{1}{36}$ 
\end{enumerate}
\solution
		%\input{ncert/12/13/2/17/defs.tex}
	\item A bag contains 4 red and 4 black balls, another bag contains 2 red and 6 black balls. One of the two bags is selected at random and a ball is drawn from the bag which is found to be red. Find the probability that the ball is drawn from the first bag.
\\
\solution
		%\input{ncert/12/13/3/2/main.tex}
  \item
  Cards with numbers 2 to 101 are placed in a box. A card is selected at random.Find the probability that the card has
\begin{enumerate}[label=(\roman*)]
	\item an even number 
	\item a square number
\end{enumerate}
\solution
%\input{exemplar/10/13/3/32/main.tex}
\item
The king, queen and jack of clubs are removed from a deck of 52 playing cards and then well shuffled. Now one card is drawn at random from the remaining cards.  Determine the probability that the card is
\begin{enumerate}[label=(\roman*)]
\item a club
\item 10 of hearts
\end{enumerate}
\solution
%\input{exemplar/10/13/3/29/main.tex}
\item A team of medical students doing their internship have to assist during surgeries
at a city hospital. The probabilities of surgeries rated as very complex, complex,
routine, simple or very simple are respectively, 0.15, 0.20, 0.31, 0.26, .08. Find
the probabilities that a particular surgery will be rated
\begin{enumerate}
	\item complex or very complex;
	\item neither very complex nor very simple;
	\item routine or complex
	\item routine or simple
\end{enumerate}
\solution
%\input{exemplar/11/16/3/8(1)/main.tex}
\item A card is selected from a pack of 52 cards.
\begin{enumerate}[label=(\alph*)]
    \item How many points are there in the sample space?
    \item Calculate the probability that the card is an ace of spades.
    \item Calculate the probability that the card is (i) an ace and (ii) black card.
\end{enumerate}
\solution
%\input{exemplar/11/16/3/4/main2.tex}
\item The probability that a non leap year selected at random will contain 53 sundays.
\\
\solution
%\input{exemplar/10/13/1/19/main.tex}
\item One of the four persons John, Rita, Aslam or Gurpreet will be promoted next
month. Consequently the sample space consists of four elementary outcomes
S = {John promoted, Rita promoted, Aslam promoted, Gurpreet promoted}
You are told that the chances of John’s promotion is same as that of Gurpreet,
Rita’s chances of promotion are twice as likely as Johns. Aslam’s chances are
four times that of John.
\begin{enumerate}
	\item Determine
	\begin{enumerate}
		\item P (John promoted)
		\item P (Rita promoted)
		\item P (Aslam promoted)
		\item P (Gurpreet promoted)
	\end{enumerate}
	\item If A = {John promoted or Gurpreet promoted}, find P (A).
\end{enumerate}
\solution
%\input{exemplar/11/16/3/10/main.tex}
\item A card is drawn from a deck of 52 cards. Find the probability of getting a king or a heart or a red card.\\
\solution
%\input{exemplar/11/16/3/15/main.tex}
\item The probability that a student will pass his examination is 0.73, the probability of
the student getting a compartment is 0.13, and the probability that the student will
either pass or get compartment is 0.96. State True or False.\\
\solution
%\input{exemplar/11/16/3/31/main.tex}
\item A card is selected from a pack of 52 cards\\
\begin{enumerate}[label=(\alph*)]
\item How many points are there in the sample space?
\item Calculate the probability that the cards is an ace of spades.
\item Calculate the probability that the card is (i) an ace (ii)black card.\\
\end{enumerate}
%\input{ncert/11/16/3/4_1/Prob_4.tex}
\item In a non-leap year, the probability of having 53 tuesdays or 53 wednesdays is\\
\solution
%\input{exemplar/11/16/3/18/main.tex}
\item There are 1000 sealed envelopes in a box, 10 of them contain a cash prize of
Rs 100 each, 100 of them contain a cash prize of Rs 50 each and 200 of them
contain a cash prize of Rs 10 each and rest do not contain any cash prize. If they
are well shuffled and an envelope is picked up out, what is the probability that it
contains no cash prize?\\
\solution
%\input{exemplar/10/13/3/34/main.tex}
\item 
A die is thrown and a card is selected at random from a deck of 52 playing cards. The probability of getting an even number on the die and a spade card.\\
\solution
%\input{exemplar/12/13/3/78/main.tex}
\item
If 4-digit numbers greater than 5,000 are randomly formed from the digits 0, 1, 3, 5, and 7, what is the probability of forming a number divisible by 5 when:
\begin{enumerate}
    \item The digits are repeated?
    \item The repetition of digits is not allowed?
\end{enumerate}
\solution
%\input{ncert/11/16/4/9/main.tex}
\item Consider the probability space $\brak{\Omega, \mathcal{G}, P}$ where $\Omega = [0,2]$ and $\mathcal{G} = \cbrak{\phi, \Omega, [0,1], (1,2]}$. Let $X$ and $Y$ be two functions on $\Omega$ defined as
\begin{align*}
    X(\omega) = 
    \begin{cases}
        1 & \text{if }\omega \in [0, 1]\\
        2 & \text{if }\omega \in (1, 2]
    \end{cases}
\end{align*}
and
\begin{align*}
    Y(\omega) = 
    \begin{cases}
        2 & \text{if }\omega \in [0, 1.5]\\
        3 & \text{if }\omega \in (1.5, 2].
    \end{cases}
\end{align*}
Then which one of the following statements is true?
\begin{enumerate}
    \item [(A)] $X$ is a random variable with respect to $\mathcal{G}$, but $Y$ is not a random variable with respect to $\mathcal{G}$.
    \item [(B)] $Y$ is a random variable with respect to $\mathcal{G}$, but $X$ is not a random variable with respect to $\mathcal{G}$.
    \item [(C)] Neither $X$ nor $Y$ is a random variable with respect to $\mathcal{G}$.
    \item [(D)] Both $X$ and $Y$ are random variables with respect to $\mathcal{G}$.
\end{enumerate} \hfill (GATE ST 2023)\\
\solution
%\input{gate/ST/2023/14/main.tex}
	\item  A die is loaded in such a way that each odd number is twice as likely to occur as
each even number. Find $P(G)$, where $G$ is the event that a number greater than
3 occurs on a single roll of the die.
\\
\solution
		%\input{exemplar/11/16/3/5/main.tex}
	\item All the jacks, queens and kings are removed from a deck of 52 playing cards. The remaining cards are well shuffled and then one card is drawn at random. Giving ace a value 1 similar value for other cards, find the probability that the card has a value 
		\begin{enumerate}
			\item 7
			\item greater than 7
			\item less than 7
		\end{enumerate}
		%\input{exemplar/10/13/3/30/main.tex}
  \item A Lot consists of 48 mobile phones of which 42 are good, 3 have only minor defects and 3 have major defects.Varnika will buy a phone if it is good but the trader will only buy a mobile if it has no major defects. One phone is selected at random from the lot. What is the probability that it is
\begin{enumerate}
	\item acceptable to Varnika?
            \item acceptable to the trader?
\end{enumerate}
\solution
	%\input{exemplar/10/13/3/40/main.tex}
 \item A student says that if you throw a die, it will show up 1 or not 1. Therefore, the probability of getting 1 and the probability of getting 'not 1' each is equal to $\frac{1}{2}$. Is this correct? Give reasons.\\
 \solution
        %\input{exemplar/10/13/2/9/main.tex}
   \item Four candidates A, B, C, D have ap-
plied for the assignment to coach a school cricket
team. If A is twice as likely to be selected as B, and
B and C are given about the same chance of being
selected, while C is twice as likely to be selected
as D, what are the probabilities that
\begin{enumerate}
\item C will be selected?
\item A will not be selected?
\end{enumerate}
	%\input{exemplar/11/16/3/9/main.tex}
 \item A bag contain 24 balls of which $x$ balls are red, $2x$ are white and $3x$ are blue. A ball is selected at random, What is the probability that it is
\begin{enumerate}[label=\alph*)]
\item not red ?
\item white ?
\end{enumerate}
%\input{exemplar/10/13/3/41/main.tex}
If the letters of the word ASSASSINATION are arranged at random. Find the Probability that
\begin{enumerate}[label=(\alph*)]
\item Four $S's$ come consecutively in the word
\item Two  $I's$ and two $N's$ come together
\item All $A's$ are not coming together
\item No two $A's$ are coming together
\end{enumerate}
%\input{exemplar/11/16/3/14/main.tex}
	\item One urn contains two black balls (labelled B1 and B2) and one white ball. A
	second urn contains one black ball and two white balls (labelled W1 and W2).
	Suppose the following experiment is performed. One of the two urns is chosen
	at random. Next a ball is randomly chosen from the urn. Then a second ball is
	chosen at random from the same urn without replacing the first ball.
	
	\begin{enumerate}
	\item What is the probability that two black balls are chosen?
	
	\item What is the probability that two balls of opposite colour are chosen?
	\end{enumerate}
	\solution
	%\input{exemplar/11/16/3/12/main1.tex}
\end{enumerate}

	\item A bag contains $5$ red balls and some blue balls. If the probability of drawing a blue ball is double that if a red ball, determine the number of blue balls in the bag. 
		\\
\solution
		%\begin{enumerate}[label=\thesection.\arabic*,ref=\thesection.\theenumi]
	\item One card is drawn from a well-shuffled deck of 52 cards. Find the probability of getting
\begin{enumerate}
\item A king of red colour 
\item A face card 
\item A red face card
\item The jack of hearts
\item A spade
\item The queen of diamonds

\end{enumerate}
\solution
		%\input{ncert/10/15/1/14/main.tex}
	\item Five cards—the ten, jack, queen, king and ace of diamonds, are well-shuffled with their face downwards. One card is then picked up at random.
\begin{enumerate}
\item
What is the probability that the card is the queen? 
\item
If the queen is drawn and put aside, what is the probability that the second card picked up is (a) an ace? (b) a queen?\\
\end{enumerate}
\solution
		%\input{ncert/10/15/1/15/defs.tex}
	\item A bag contains $5$ red balls and some blue balls. If the probability of drawing a blue ball is double that if a red ball, determine the number of blue balls in the bag. 
		\\
\solution
		%\input{ncert/10/15/2/3/defs.tex}
	\item A card is selected from a pack of 52 cards.
 \begin{enumerate}[label=(\alph*)] 
                 \item How many points are there in the sample space?
                 \item Calculate the probability that the card is an ace of spades.
                 \item Calculate the probability that the card is (i) an ace and (ii) black card.
 \end{enumerate}
\solution
		%\input{ncert/11/16/3/4/main.tex}
\item Four cards are drawn from a well-shuffled deck of 52 cards. What is the probability of obtaining 3 diamonds and one spade.
\\
\solution
		%\input{ncert/11/16/4/2/defs.tex}
\item In a certain lottery 10,000 tickets are sold and ten equal prizes are awarded. What is the probability of not getting a prize if you buy (a) one ticket (b) two tickets (c) 10 tickets ?	
\\
\solution
		%\input{ncert/11/16/4/4/defs.tex}
		%
\item 
Out of 100 students, two sections of 40 and 60 are formed. If you and your friend are among the 100 students, what is the probability that
\begin{enumerate}
\item you both enter the same section?
\item you both enter the different sections?
\end{enumerate}
\solution
		%\input{ncert/11/16/4/5/defs.tex}
	\item 
The number lock of a suitcase has 4 wheels each labelled with ten digits i.e. from 0 to 9.The lock opens with a sequence of four digits with no repeats.What is the probability of a person getting the right sequence to open the suitcase.
\\
\solution
		%\input{ncert/11/16/4/10/defs.tex}
		%
\item 
Two cards are drawn at random and without replacement from a pack of 52 playing cards. Find the probability that both the cards are black.
\\
\solution
		%\input{ncert/12/13/2/2/defs.tex}
		\item A box of oranges is inspected by examining three randomly selected oranges drawn without replacement. If all the three oranges are good, the box is approved for sale, otherwise, it is rejected. Find the probability that a box containing 15 oranges out of which 12 are good and 3 are bad ones will be approved for sale.
		\label{ncert/12/13/2/3/defs.tex}
		\item Two balls are drawn at random with replacement from a box containing 10 black and 8 red balls. Find the probability that
		\label{ncert/12/13/2/12}
\begin{enumerate}
\item both balls are red.
\item first ball is black and second is red.
\item one of them is black and other is red.
\end{enumerate}

\item In a hostel, 60\% of the students read Hindi newspaper, 40\% read English newspaper and 20\% read both Hindi and English newspapers. A student is selected at random.
		\label{ncert/12/13/2/15}
\begin{enumerate}
\item Find the probability that she reads neither Hindi nor English newspapers.
\item If she reads Hindi newspaper, find the probability that she reads English newspaper.
\item If she reads English newspaper, find the probability that she reads Hindi newspaper.\\
\end{enumerate}
\item The probability of obtaining an even prime number on each die, when a pair of dice is rolled is 
\begin{enumerate}
    \item $0$ 
    
    \item $\frac{1}{3}$ 
    
    \item $\frac{1}{12}$ 
    
    \item $\frac{1}{36}$ 
\end{enumerate}
\solution
		%\input{ncert/12/13/2/17/defs.tex}
	\item A bag contains 4 red and 4 black balls, another bag contains 2 red and 6 black balls. One of the two bags is selected at random and a ball is drawn from the bag which is found to be red. Find the probability that the ball is drawn from the first bag.
\\
\solution
		%\input{ncert/12/13/3/2/main.tex}
  \item
  Cards with numbers 2 to 101 are placed in a box. A card is selected at random.Find the probability that the card has
\begin{enumerate}[label=(\roman*)]
	\item an even number 
	\item a square number
\end{enumerate}
\solution
%\input{exemplar/10/13/3/32/main.tex}
\item
The king, queen and jack of clubs are removed from a deck of 52 playing cards and then well shuffled. Now one card is drawn at random from the remaining cards.  Determine the probability that the card is
\begin{enumerate}[label=(\roman*)]
\item a club
\item 10 of hearts
\end{enumerate}
\solution
%\input{exemplar/10/13/3/29/main.tex}
\item A team of medical students doing their internship have to assist during surgeries
at a city hospital. The probabilities of surgeries rated as very complex, complex,
routine, simple or very simple are respectively, 0.15, 0.20, 0.31, 0.26, .08. Find
the probabilities that a particular surgery will be rated
\begin{enumerate}
	\item complex or very complex;
	\item neither very complex nor very simple;
	\item routine or complex
	\item routine or simple
\end{enumerate}
\solution
%\input{exemplar/11/16/3/8(1)/main.tex}
\item A card is selected from a pack of 52 cards.
\begin{enumerate}[label=(\alph*)]
    \item How many points are there in the sample space?
    \item Calculate the probability that the card is an ace of spades.
    \item Calculate the probability that the card is (i) an ace and (ii) black card.
\end{enumerate}
\solution
%\input{exemplar/11/16/3/4/main2.tex}
\item The probability that a non leap year selected at random will contain 53 sundays.
\\
\solution
%\input{exemplar/10/13/1/19/main.tex}
\item One of the four persons John, Rita, Aslam or Gurpreet will be promoted next
month. Consequently the sample space consists of four elementary outcomes
S = {John promoted, Rita promoted, Aslam promoted, Gurpreet promoted}
You are told that the chances of John’s promotion is same as that of Gurpreet,
Rita’s chances of promotion are twice as likely as Johns. Aslam’s chances are
four times that of John.
\begin{enumerate}
	\item Determine
	\begin{enumerate}
		\item P (John promoted)
		\item P (Rita promoted)
		\item P (Aslam promoted)
		\item P (Gurpreet promoted)
	\end{enumerate}
	\item If A = {John promoted or Gurpreet promoted}, find P (A).
\end{enumerate}
\solution
%\input{exemplar/11/16/3/10/main.tex}
\item A card is drawn from a deck of 52 cards. Find the probability of getting a king or a heart or a red card.\\
\solution
%\input{exemplar/11/16/3/15/main.tex}
\item The probability that a student will pass his examination is 0.73, the probability of
the student getting a compartment is 0.13, and the probability that the student will
either pass or get compartment is 0.96. State True or False.\\
\solution
%\input{exemplar/11/16/3/31/main.tex}
\item A card is selected from a pack of 52 cards\\
\begin{enumerate}[label=(\alph*)]
\item How many points are there in the sample space?
\item Calculate the probability that the cards is an ace of spades.
\item Calculate the probability that the card is (i) an ace (ii)black card.\\
\end{enumerate}
%\input{ncert/11/16/3/4_1/Prob_4.tex}
\item In a non-leap year, the probability of having 53 tuesdays or 53 wednesdays is\\
\solution
%\input{exemplar/11/16/3/18/main.tex}
\item There are 1000 sealed envelopes in a box, 10 of them contain a cash prize of
Rs 100 each, 100 of them contain a cash prize of Rs 50 each and 200 of them
contain a cash prize of Rs 10 each and rest do not contain any cash prize. If they
are well shuffled and an envelope is picked up out, what is the probability that it
contains no cash prize?\\
\solution
%\input{exemplar/10/13/3/34/main.tex}
\item 
A die is thrown and a card is selected at random from a deck of 52 playing cards. The probability of getting an even number on the die and a spade card.\\
\solution
%\input{exemplar/12/13/3/78/main.tex}
\item
If 4-digit numbers greater than 5,000 are randomly formed from the digits 0, 1, 3, 5, and 7, what is the probability of forming a number divisible by 5 when:
\begin{enumerate}
    \item The digits are repeated?
    \item The repetition of digits is not allowed?
\end{enumerate}
\solution
%\input{ncert/11/16/4/9/main.tex}
\item Consider the probability space $\brak{\Omega, \mathcal{G}, P}$ where $\Omega = [0,2]$ and $\mathcal{G} = \cbrak{\phi, \Omega, [0,1], (1,2]}$. Let $X$ and $Y$ be two functions on $\Omega$ defined as
\begin{align*}
    X(\omega) = 
    \begin{cases}
        1 & \text{if }\omega \in [0, 1]\\
        2 & \text{if }\omega \in (1, 2]
    \end{cases}
\end{align*}
and
\begin{align*}
    Y(\omega) = 
    \begin{cases}
        2 & \text{if }\omega \in [0, 1.5]\\
        3 & \text{if }\omega \in (1.5, 2].
    \end{cases}
\end{align*}
Then which one of the following statements is true?
\begin{enumerate}
    \item [(A)] $X$ is a random variable with respect to $\mathcal{G}$, but $Y$ is not a random variable with respect to $\mathcal{G}$.
    \item [(B)] $Y$ is a random variable with respect to $\mathcal{G}$, but $X$ is not a random variable with respect to $\mathcal{G}$.
    \item [(C)] Neither $X$ nor $Y$ is a random variable with respect to $\mathcal{G}$.
    \item [(D)] Both $X$ and $Y$ are random variables with respect to $\mathcal{G}$.
\end{enumerate} \hfill (GATE ST 2023)\\
\solution
%\input{gate/ST/2023/14/main.tex}
	\item  A die is loaded in such a way that each odd number is twice as likely to occur as
each even number. Find $P(G)$, where $G$ is the event that a number greater than
3 occurs on a single roll of the die.
\\
\solution
		%\input{exemplar/11/16/3/5/main.tex}
	\item All the jacks, queens and kings are removed from a deck of 52 playing cards. The remaining cards are well shuffled and then one card is drawn at random. Giving ace a value 1 similar value for other cards, find the probability that the card has a value 
		\begin{enumerate}
			\item 7
			\item greater than 7
			\item less than 7
		\end{enumerate}
		%\input{exemplar/10/13/3/30/main.tex}
  \item A Lot consists of 48 mobile phones of which 42 are good, 3 have only minor defects and 3 have major defects.Varnika will buy a phone if it is good but the trader will only buy a mobile if it has no major defects. One phone is selected at random from the lot. What is the probability that it is
\begin{enumerate}
	\item acceptable to Varnika?
            \item acceptable to the trader?
\end{enumerate}
\solution
	%\input{exemplar/10/13/3/40/main.tex}
 \item A student says that if you throw a die, it will show up 1 or not 1. Therefore, the probability of getting 1 and the probability of getting 'not 1' each is equal to $\frac{1}{2}$. Is this correct? Give reasons.\\
 \solution
        %\input{exemplar/10/13/2/9/main.tex}
   \item Four candidates A, B, C, D have ap-
plied for the assignment to coach a school cricket
team. If A is twice as likely to be selected as B, and
B and C are given about the same chance of being
selected, while C is twice as likely to be selected
as D, what are the probabilities that
\begin{enumerate}
\item C will be selected?
\item A will not be selected?
\end{enumerate}
	%\input{exemplar/11/16/3/9/main.tex}
 \item A bag contain 24 balls of which $x$ balls are red, $2x$ are white and $3x$ are blue. A ball is selected at random, What is the probability that it is
\begin{enumerate}[label=\alph*)]
\item not red ?
\item white ?
\end{enumerate}
%\input{exemplar/10/13/3/41/main.tex}
If the letters of the word ASSASSINATION are arranged at random. Find the Probability that
\begin{enumerate}[label=(\alph*)]
\item Four $S's$ come consecutively in the word
\item Two  $I's$ and two $N's$ come together
\item All $A's$ are not coming together
\item No two $A's$ are coming together
\end{enumerate}
%\input{exemplar/11/16/3/14/main.tex}
	\item One urn contains two black balls (labelled B1 and B2) and one white ball. A
	second urn contains one black ball and two white balls (labelled W1 and W2).
	Suppose the following experiment is performed. One of the two urns is chosen
	at random. Next a ball is randomly chosen from the urn. Then a second ball is
	chosen at random from the same urn without replacing the first ball.
	
	\begin{enumerate}
	\item What is the probability that two black balls are chosen?
	
	\item What is the probability that two balls of opposite colour are chosen?
	\end{enumerate}
	\solution
	%\input{exemplar/11/16/3/12/main1.tex}
\end{enumerate}

	\item A card is selected from a pack of 52 cards.
 \begin{enumerate}[label=(\alph*)] 
                 \item How many points are there in the sample space?
                 \item Calculate the probability that the card is an ace of spades.
                 \item Calculate the probability that the card is (i) an ace and (ii) black card.
 \end{enumerate}
\solution
		%\begin{table}[H]
	\centering
\begin{tabular}{|c|c|c|}
\hline
Random variable &Value &Definition\\ \hline
\multirow{3}{*}{X} &0 &Slips of Rs 1\\
&1 &Slips of Rs 5\\
&2 &Slips of Rs 13\\ \hline
\multirow{2}{*}{Y} &0 &Box A\\
&1 &Box B\\\hline
\end{tabular}
\caption{}
\label{tab:Distribution}
\end{table}
See \tabref{tab:Distribution}.
\begin{align}
p_{Y}\brak{k}= \begin{cases} 
      \frac{1}{3} & {k=0} \\
      \frac{2}{3 }& {k=1} 
   \end{cases}
   \\
p_{Y|X}\brak{0|0} = \frac{19}{25}\, 
p_{Y|X}\brak{0|1} = \frac{6}{25}\,
p_{Y|X}\brak{1|0} = \frac{45}{50}\,
p_{Y|X}\brak{1|2} = \frac{5}{50}
\end{align}
The desired probability is the probability that a slip drawn at random is marked other than Rs 1,
\begin{align}
&=1-p_X\brak{0}\\
&= p_X(1) + p_X(2)
\end{align}
Using Bayes theorem,
\begin{align}
&= p_Y\brak{0} \times \pr{Y=0 | X=1} + p_Y\brak{1} \times \pr{Y=1|X=2}\\
&=\frac{1}{3} \times \frac{6}{25} + \frac{2}{3} \times \frac{5}{50}\\
&=\frac{11}{75}
\end{align}

\newpage

%\tableofcontents

\bigskip

\renewcommand{\thefigure}{\theenumi}
\renewcommand{\thetable}{\theenumi}
%\renewcommand{\theequation}{\theenumi}

%\begin{abstract}
%%\boldmath
%In this letter, an algorithm for evaluating the exact analytical bit error rate  (BER)  for the piecewise linear (PL) combiner for  multiple relays is presented. Previous results were available only for upto three relays. The algorithm is unique in the sense that  the actual mathematical expressions, that are prohibitively large, need not be explicitly obtained. The diversity gain due to multiple relays is shown through plots of the analytical BER, well supported by simulations. 
%
%\end{abstract}
% IEEEtran.cls defaults to using nonbold math in the Abstract.
% This preserves the distinction between vectors and scalars. However,
% if the journal you are submitting to favors bold math in the abstract,
% then you can use LaTeX's standard command \boldmath at the very start
% of the abstract to achieve this. Many IEEE journals frown on math
% in the abstract anyway.

% Note that keywords are not normally used for peerreview papers.
%\begin{IEEEkeywords}
%Cooperative diversity, decode and forward, piecewise linear
%\end{IEEEkeywords}



% For peer review papers, you can put extra information on the cover
% page as needed:
% \ifCLASSOPTIONpeerreview
% \begin{center} \bfseries EDICS Category: 3-BBND \end{center}
% \fi
%
% For peerreview papers, this IEEEtran command inserts a page break and
% creates the second title. It will be ignored for other modes.
%\IEEEpeerreviewmaketitle




\item Four cards are drawn from a well-shuffled deck of 52 cards. What is the probability of obtaining 3 diamonds and one spade.
\\
\solution
		%\begin{enumerate}[label=\thesection.\arabic*,ref=\thesection.\theenumi]
	\item One card is drawn from a well-shuffled deck of 52 cards. Find the probability of getting
\begin{enumerate}
\item A king of red colour 
\item A face card 
\item A red face card
\item The jack of hearts
\item A spade
\item The queen of diamonds

\end{enumerate}
\solution
		%\input{ncert/10/15/1/14/main.tex}
	\item Five cards—the ten, jack, queen, king and ace of diamonds, are well-shuffled with their face downwards. One card is then picked up at random.
\begin{enumerate}
\item
What is the probability that the card is the queen? 
\item
If the queen is drawn and put aside, what is the probability that the second card picked up is (a) an ace? (b) a queen?\\
\end{enumerate}
\solution
		%\input{ncert/10/15/1/15/defs.tex}
	\item A bag contains $5$ red balls and some blue balls. If the probability of drawing a blue ball is double that if a red ball, determine the number of blue balls in the bag. 
		\\
\solution
		%\input{ncert/10/15/2/3/defs.tex}
	\item A card is selected from a pack of 52 cards.
 \begin{enumerate}[label=(\alph*)] 
                 \item How many points are there in the sample space?
                 \item Calculate the probability that the card is an ace of spades.
                 \item Calculate the probability that the card is (i) an ace and (ii) black card.
 \end{enumerate}
\solution
		%\input{ncert/11/16/3/4/main.tex}
\item Four cards are drawn from a well-shuffled deck of 52 cards. What is the probability of obtaining 3 diamonds and one spade.
\\
\solution
		%\input{ncert/11/16/4/2/defs.tex}
\item In a certain lottery 10,000 tickets are sold and ten equal prizes are awarded. What is the probability of not getting a prize if you buy (a) one ticket (b) two tickets (c) 10 tickets ?	
\\
\solution
		%\input{ncert/11/16/4/4/defs.tex}
		%
\item 
Out of 100 students, two sections of 40 and 60 are formed. If you and your friend are among the 100 students, what is the probability that
\begin{enumerate}
\item you both enter the same section?
\item you both enter the different sections?
\end{enumerate}
\solution
		%\input{ncert/11/16/4/5/defs.tex}
	\item 
The number lock of a suitcase has 4 wheels each labelled with ten digits i.e. from 0 to 9.The lock opens with a sequence of four digits with no repeats.What is the probability of a person getting the right sequence to open the suitcase.
\\
\solution
		%\input{ncert/11/16/4/10/defs.tex}
		%
\item 
Two cards are drawn at random and without replacement from a pack of 52 playing cards. Find the probability that both the cards are black.
\\
\solution
		%\input{ncert/12/13/2/2/defs.tex}
		\item A box of oranges is inspected by examining three randomly selected oranges drawn without replacement. If all the three oranges are good, the box is approved for sale, otherwise, it is rejected. Find the probability that a box containing 15 oranges out of which 12 are good and 3 are bad ones will be approved for sale.
		\label{ncert/12/13/2/3/defs.tex}
		\item Two balls are drawn at random with replacement from a box containing 10 black and 8 red balls. Find the probability that
		\label{ncert/12/13/2/12}
\begin{enumerate}
\item both balls are red.
\item first ball is black and second is red.
\item one of them is black and other is red.
\end{enumerate}

\item In a hostel, 60\% of the students read Hindi newspaper, 40\% read English newspaper and 20\% read both Hindi and English newspapers. A student is selected at random.
		\label{ncert/12/13/2/15}
\begin{enumerate}
\item Find the probability that she reads neither Hindi nor English newspapers.
\item If she reads Hindi newspaper, find the probability that she reads English newspaper.
\item If she reads English newspaper, find the probability that she reads Hindi newspaper.\\
\end{enumerate}
\item The probability of obtaining an even prime number on each die, when a pair of dice is rolled is 
\begin{enumerate}
    \item $0$ 
    
    \item $\frac{1}{3}$ 
    
    \item $\frac{1}{12}$ 
    
    \item $\frac{1}{36}$ 
\end{enumerate}
\solution
		%\input{ncert/12/13/2/17/defs.tex}
	\item A bag contains 4 red and 4 black balls, another bag contains 2 red and 6 black balls. One of the two bags is selected at random and a ball is drawn from the bag which is found to be red. Find the probability that the ball is drawn from the first bag.
\\
\solution
		%\input{ncert/12/13/3/2/main.tex}
  \item
  Cards with numbers 2 to 101 are placed in a box. A card is selected at random.Find the probability that the card has
\begin{enumerate}[label=(\roman*)]
	\item an even number 
	\item a square number
\end{enumerate}
\solution
%\input{exemplar/10/13/3/32/main.tex}
\item
The king, queen and jack of clubs are removed from a deck of 52 playing cards and then well shuffled. Now one card is drawn at random from the remaining cards.  Determine the probability that the card is
\begin{enumerate}[label=(\roman*)]
\item a club
\item 10 of hearts
\end{enumerate}
\solution
%\input{exemplar/10/13/3/29/main.tex}
\item A team of medical students doing their internship have to assist during surgeries
at a city hospital. The probabilities of surgeries rated as very complex, complex,
routine, simple or very simple are respectively, 0.15, 0.20, 0.31, 0.26, .08. Find
the probabilities that a particular surgery will be rated
\begin{enumerate}
	\item complex or very complex;
	\item neither very complex nor very simple;
	\item routine or complex
	\item routine or simple
\end{enumerate}
\solution
%\input{exemplar/11/16/3/8(1)/main.tex}
\item A card is selected from a pack of 52 cards.
\begin{enumerate}[label=(\alph*)]
    \item How many points are there in the sample space?
    \item Calculate the probability that the card is an ace of spades.
    \item Calculate the probability that the card is (i) an ace and (ii) black card.
\end{enumerate}
\solution
%\input{exemplar/11/16/3/4/main2.tex}
\item The probability that a non leap year selected at random will contain 53 sundays.
\\
\solution
%\input{exemplar/10/13/1/19/main.tex}
\item One of the four persons John, Rita, Aslam or Gurpreet will be promoted next
month. Consequently the sample space consists of four elementary outcomes
S = {John promoted, Rita promoted, Aslam promoted, Gurpreet promoted}
You are told that the chances of John’s promotion is same as that of Gurpreet,
Rita’s chances of promotion are twice as likely as Johns. Aslam’s chances are
four times that of John.
\begin{enumerate}
	\item Determine
	\begin{enumerate}
		\item P (John promoted)
		\item P (Rita promoted)
		\item P (Aslam promoted)
		\item P (Gurpreet promoted)
	\end{enumerate}
	\item If A = {John promoted or Gurpreet promoted}, find P (A).
\end{enumerate}
\solution
%\input{exemplar/11/16/3/10/main.tex}
\item A card is drawn from a deck of 52 cards. Find the probability of getting a king or a heart or a red card.\\
\solution
%\input{exemplar/11/16/3/15/main.tex}
\item The probability that a student will pass his examination is 0.73, the probability of
the student getting a compartment is 0.13, and the probability that the student will
either pass or get compartment is 0.96. State True or False.\\
\solution
%\input{exemplar/11/16/3/31/main.tex}
\item A card is selected from a pack of 52 cards\\
\begin{enumerate}[label=(\alph*)]
\item How many points are there in the sample space?
\item Calculate the probability that the cards is an ace of spades.
\item Calculate the probability that the card is (i) an ace (ii)black card.\\
\end{enumerate}
%\input{ncert/11/16/3/4_1/Prob_4.tex}
\item In a non-leap year, the probability of having 53 tuesdays or 53 wednesdays is\\
\solution
%\input{exemplar/11/16/3/18/main.tex}
\item There are 1000 sealed envelopes in a box, 10 of them contain a cash prize of
Rs 100 each, 100 of them contain a cash prize of Rs 50 each and 200 of them
contain a cash prize of Rs 10 each and rest do not contain any cash prize. If they
are well shuffled and an envelope is picked up out, what is the probability that it
contains no cash prize?\\
\solution
%\input{exemplar/10/13/3/34/main.tex}
\item 
A die is thrown and a card is selected at random from a deck of 52 playing cards. The probability of getting an even number on the die and a spade card.\\
\solution
%\input{exemplar/12/13/3/78/main.tex}
\item
If 4-digit numbers greater than 5,000 are randomly formed from the digits 0, 1, 3, 5, and 7, what is the probability of forming a number divisible by 5 when:
\begin{enumerate}
    \item The digits are repeated?
    \item The repetition of digits is not allowed?
\end{enumerate}
\solution
%\input{ncert/11/16/4/9/main.tex}
\item Consider the probability space $\brak{\Omega, \mathcal{G}, P}$ where $\Omega = [0,2]$ and $\mathcal{G} = \cbrak{\phi, \Omega, [0,1], (1,2]}$. Let $X$ and $Y$ be two functions on $\Omega$ defined as
\begin{align*}
    X(\omega) = 
    \begin{cases}
        1 & \text{if }\omega \in [0, 1]\\
        2 & \text{if }\omega \in (1, 2]
    \end{cases}
\end{align*}
and
\begin{align*}
    Y(\omega) = 
    \begin{cases}
        2 & \text{if }\omega \in [0, 1.5]\\
        3 & \text{if }\omega \in (1.5, 2].
    \end{cases}
\end{align*}
Then which one of the following statements is true?
\begin{enumerate}
    \item [(A)] $X$ is a random variable with respect to $\mathcal{G}$, but $Y$ is not a random variable with respect to $\mathcal{G}$.
    \item [(B)] $Y$ is a random variable with respect to $\mathcal{G}$, but $X$ is not a random variable with respect to $\mathcal{G}$.
    \item [(C)] Neither $X$ nor $Y$ is a random variable with respect to $\mathcal{G}$.
    \item [(D)] Both $X$ and $Y$ are random variables with respect to $\mathcal{G}$.
\end{enumerate} \hfill (GATE ST 2023)\\
\solution
%\input{gate/ST/2023/14/main.tex}
	\item  A die is loaded in such a way that each odd number is twice as likely to occur as
each even number. Find $P(G)$, where $G$ is the event that a number greater than
3 occurs on a single roll of the die.
\\
\solution
		%\input{exemplar/11/16/3/5/main.tex}
	\item All the jacks, queens and kings are removed from a deck of 52 playing cards. The remaining cards are well shuffled and then one card is drawn at random. Giving ace a value 1 similar value for other cards, find the probability that the card has a value 
		\begin{enumerate}
			\item 7
			\item greater than 7
			\item less than 7
		\end{enumerate}
		%\input{exemplar/10/13/3/30/main.tex}
  \item A Lot consists of 48 mobile phones of which 42 are good, 3 have only minor defects and 3 have major defects.Varnika will buy a phone if it is good but the trader will only buy a mobile if it has no major defects. One phone is selected at random from the lot. What is the probability that it is
\begin{enumerate}
	\item acceptable to Varnika?
            \item acceptable to the trader?
\end{enumerate}
\solution
	%\input{exemplar/10/13/3/40/main.tex}
 \item A student says that if you throw a die, it will show up 1 or not 1. Therefore, the probability of getting 1 and the probability of getting 'not 1' each is equal to $\frac{1}{2}$. Is this correct? Give reasons.\\
 \solution
        %\input{exemplar/10/13/2/9/main.tex}
   \item Four candidates A, B, C, D have ap-
plied for the assignment to coach a school cricket
team. If A is twice as likely to be selected as B, and
B and C are given about the same chance of being
selected, while C is twice as likely to be selected
as D, what are the probabilities that
\begin{enumerate}
\item C will be selected?
\item A will not be selected?
\end{enumerate}
	%\input{exemplar/11/16/3/9/main.tex}
 \item A bag contain 24 balls of which $x$ balls are red, $2x$ are white and $3x$ are blue. A ball is selected at random, What is the probability that it is
\begin{enumerate}[label=\alph*)]
\item not red ?
\item white ?
\end{enumerate}
%\input{exemplar/10/13/3/41/main.tex}
If the letters of the word ASSASSINATION are arranged at random. Find the Probability that
\begin{enumerate}[label=(\alph*)]
\item Four $S's$ come consecutively in the word
\item Two  $I's$ and two $N's$ come together
\item All $A's$ are not coming together
\item No two $A's$ are coming together
\end{enumerate}
%\input{exemplar/11/16/3/14/main.tex}
	\item One urn contains two black balls (labelled B1 and B2) and one white ball. A
	second urn contains one black ball and two white balls (labelled W1 and W2).
	Suppose the following experiment is performed. One of the two urns is chosen
	at random. Next a ball is randomly chosen from the urn. Then a second ball is
	chosen at random from the same urn without replacing the first ball.
	
	\begin{enumerate}
	\item What is the probability that two black balls are chosen?
	
	\item What is the probability that two balls of opposite colour are chosen?
	\end{enumerate}
	\solution
	%\input{exemplar/11/16/3/12/main1.tex}
\end{enumerate}

\item In a certain lottery 10,000 tickets are sold and ten equal prizes are awarded. What is the probability of not getting a prize if you buy (a) one ticket (b) two tickets (c) 10 tickets ?	
\\
\solution
		%\begin{enumerate}[label=\thesection.\arabic*,ref=\thesection.\theenumi]
	\item One card is drawn from a well-shuffled deck of 52 cards. Find the probability of getting
\begin{enumerate}
\item A king of red colour 
\item A face card 
\item A red face card
\item The jack of hearts
\item A spade
\item The queen of diamonds

\end{enumerate}
\solution
		%\input{ncert/10/15/1/14/main.tex}
	\item Five cards—the ten, jack, queen, king and ace of diamonds, are well-shuffled with their face downwards. One card is then picked up at random.
\begin{enumerate}
\item
What is the probability that the card is the queen? 
\item
If the queen is drawn and put aside, what is the probability that the second card picked up is (a) an ace? (b) a queen?\\
\end{enumerate}
\solution
		%\input{ncert/10/15/1/15/defs.tex}
	\item A bag contains $5$ red balls and some blue balls. If the probability of drawing a blue ball is double that if a red ball, determine the number of blue balls in the bag. 
		\\
\solution
		%\input{ncert/10/15/2/3/defs.tex}
	\item A card is selected from a pack of 52 cards.
 \begin{enumerate}[label=(\alph*)] 
                 \item How many points are there in the sample space?
                 \item Calculate the probability that the card is an ace of spades.
                 \item Calculate the probability that the card is (i) an ace and (ii) black card.
 \end{enumerate}
\solution
		%\input{ncert/11/16/3/4/main.tex}
\item Four cards are drawn from a well-shuffled deck of 52 cards. What is the probability of obtaining 3 diamonds and one spade.
\\
\solution
		%\input{ncert/11/16/4/2/defs.tex}
\item In a certain lottery 10,000 tickets are sold and ten equal prizes are awarded. What is the probability of not getting a prize if you buy (a) one ticket (b) two tickets (c) 10 tickets ?	
\\
\solution
		%\input{ncert/11/16/4/4/defs.tex}
		%
\item 
Out of 100 students, two sections of 40 and 60 are formed. If you and your friend are among the 100 students, what is the probability that
\begin{enumerate}
\item you both enter the same section?
\item you both enter the different sections?
\end{enumerate}
\solution
		%\input{ncert/11/16/4/5/defs.tex}
	\item 
The number lock of a suitcase has 4 wheels each labelled with ten digits i.e. from 0 to 9.The lock opens with a sequence of four digits with no repeats.What is the probability of a person getting the right sequence to open the suitcase.
\\
\solution
		%\input{ncert/11/16/4/10/defs.tex}
		%
\item 
Two cards are drawn at random and without replacement from a pack of 52 playing cards. Find the probability that both the cards are black.
\\
\solution
		%\input{ncert/12/13/2/2/defs.tex}
		\item A box of oranges is inspected by examining three randomly selected oranges drawn without replacement. If all the three oranges are good, the box is approved for sale, otherwise, it is rejected. Find the probability that a box containing 15 oranges out of which 12 are good and 3 are bad ones will be approved for sale.
		\label{ncert/12/13/2/3/defs.tex}
		\item Two balls are drawn at random with replacement from a box containing 10 black and 8 red balls. Find the probability that
		\label{ncert/12/13/2/12}
\begin{enumerate}
\item both balls are red.
\item first ball is black and second is red.
\item one of them is black and other is red.
\end{enumerate}

\item In a hostel, 60\% of the students read Hindi newspaper, 40\% read English newspaper and 20\% read both Hindi and English newspapers. A student is selected at random.
		\label{ncert/12/13/2/15}
\begin{enumerate}
\item Find the probability that she reads neither Hindi nor English newspapers.
\item If she reads Hindi newspaper, find the probability that she reads English newspaper.
\item If she reads English newspaper, find the probability that she reads Hindi newspaper.\\
\end{enumerate}
\item The probability of obtaining an even prime number on each die, when a pair of dice is rolled is 
\begin{enumerate}
    \item $0$ 
    
    \item $\frac{1}{3}$ 
    
    \item $\frac{1}{12}$ 
    
    \item $\frac{1}{36}$ 
\end{enumerate}
\solution
		%\input{ncert/12/13/2/17/defs.tex}
	\item A bag contains 4 red and 4 black balls, another bag contains 2 red and 6 black balls. One of the two bags is selected at random and a ball is drawn from the bag which is found to be red. Find the probability that the ball is drawn from the first bag.
\\
\solution
		%\input{ncert/12/13/3/2/main.tex}
  \item
  Cards with numbers 2 to 101 are placed in a box. A card is selected at random.Find the probability that the card has
\begin{enumerate}[label=(\roman*)]
	\item an even number 
	\item a square number
\end{enumerate}
\solution
%\input{exemplar/10/13/3/32/main.tex}
\item
The king, queen and jack of clubs are removed from a deck of 52 playing cards and then well shuffled. Now one card is drawn at random from the remaining cards.  Determine the probability that the card is
\begin{enumerate}[label=(\roman*)]
\item a club
\item 10 of hearts
\end{enumerate}
\solution
%\input{exemplar/10/13/3/29/main.tex}
\item A team of medical students doing their internship have to assist during surgeries
at a city hospital. The probabilities of surgeries rated as very complex, complex,
routine, simple or very simple are respectively, 0.15, 0.20, 0.31, 0.26, .08. Find
the probabilities that a particular surgery will be rated
\begin{enumerate}
	\item complex or very complex;
	\item neither very complex nor very simple;
	\item routine or complex
	\item routine or simple
\end{enumerate}
\solution
%\input{exemplar/11/16/3/8(1)/main.tex}
\item A card is selected from a pack of 52 cards.
\begin{enumerate}[label=(\alph*)]
    \item How many points are there in the sample space?
    \item Calculate the probability that the card is an ace of spades.
    \item Calculate the probability that the card is (i) an ace and (ii) black card.
\end{enumerate}
\solution
%\input{exemplar/11/16/3/4/main2.tex}
\item The probability that a non leap year selected at random will contain 53 sundays.
\\
\solution
%\input{exemplar/10/13/1/19/main.tex}
\item One of the four persons John, Rita, Aslam or Gurpreet will be promoted next
month. Consequently the sample space consists of four elementary outcomes
S = {John promoted, Rita promoted, Aslam promoted, Gurpreet promoted}
You are told that the chances of John’s promotion is same as that of Gurpreet,
Rita’s chances of promotion are twice as likely as Johns. Aslam’s chances are
four times that of John.
\begin{enumerate}
	\item Determine
	\begin{enumerate}
		\item P (John promoted)
		\item P (Rita promoted)
		\item P (Aslam promoted)
		\item P (Gurpreet promoted)
	\end{enumerate}
	\item If A = {John promoted or Gurpreet promoted}, find P (A).
\end{enumerate}
\solution
%\input{exemplar/11/16/3/10/main.tex}
\item A card is drawn from a deck of 52 cards. Find the probability of getting a king or a heart or a red card.\\
\solution
%\input{exemplar/11/16/3/15/main.tex}
\item The probability that a student will pass his examination is 0.73, the probability of
the student getting a compartment is 0.13, and the probability that the student will
either pass or get compartment is 0.96. State True or False.\\
\solution
%\input{exemplar/11/16/3/31/main.tex}
\item A card is selected from a pack of 52 cards\\
\begin{enumerate}[label=(\alph*)]
\item How many points are there in the sample space?
\item Calculate the probability that the cards is an ace of spades.
\item Calculate the probability that the card is (i) an ace (ii)black card.\\
\end{enumerate}
%\input{ncert/11/16/3/4_1/Prob_4.tex}
\item In a non-leap year, the probability of having 53 tuesdays or 53 wednesdays is\\
\solution
%\input{exemplar/11/16/3/18/main.tex}
\item There are 1000 sealed envelopes in a box, 10 of them contain a cash prize of
Rs 100 each, 100 of them contain a cash prize of Rs 50 each and 200 of them
contain a cash prize of Rs 10 each and rest do not contain any cash prize. If they
are well shuffled and an envelope is picked up out, what is the probability that it
contains no cash prize?\\
\solution
%\input{exemplar/10/13/3/34/main.tex}
\item 
A die is thrown and a card is selected at random from a deck of 52 playing cards. The probability of getting an even number on the die and a spade card.\\
\solution
%\input{exemplar/12/13/3/78/main.tex}
\item
If 4-digit numbers greater than 5,000 are randomly formed from the digits 0, 1, 3, 5, and 7, what is the probability of forming a number divisible by 5 when:
\begin{enumerate}
    \item The digits are repeated?
    \item The repetition of digits is not allowed?
\end{enumerate}
\solution
%\input{ncert/11/16/4/9/main.tex}
\item Consider the probability space $\brak{\Omega, \mathcal{G}, P}$ where $\Omega = [0,2]$ and $\mathcal{G} = \cbrak{\phi, \Omega, [0,1], (1,2]}$. Let $X$ and $Y$ be two functions on $\Omega$ defined as
\begin{align*}
    X(\omega) = 
    \begin{cases}
        1 & \text{if }\omega \in [0, 1]\\
        2 & \text{if }\omega \in (1, 2]
    \end{cases}
\end{align*}
and
\begin{align*}
    Y(\omega) = 
    \begin{cases}
        2 & \text{if }\omega \in [0, 1.5]\\
        3 & \text{if }\omega \in (1.5, 2].
    \end{cases}
\end{align*}
Then which one of the following statements is true?
\begin{enumerate}
    \item [(A)] $X$ is a random variable with respect to $\mathcal{G}$, but $Y$ is not a random variable with respect to $\mathcal{G}$.
    \item [(B)] $Y$ is a random variable with respect to $\mathcal{G}$, but $X$ is not a random variable with respect to $\mathcal{G}$.
    \item [(C)] Neither $X$ nor $Y$ is a random variable with respect to $\mathcal{G}$.
    \item [(D)] Both $X$ and $Y$ are random variables with respect to $\mathcal{G}$.
\end{enumerate} \hfill (GATE ST 2023)\\
\solution
%\input{gate/ST/2023/14/main.tex}
	\item  A die is loaded in such a way that each odd number is twice as likely to occur as
each even number. Find $P(G)$, where $G$ is the event that a number greater than
3 occurs on a single roll of the die.
\\
\solution
		%\input{exemplar/11/16/3/5/main.tex}
	\item All the jacks, queens and kings are removed from a deck of 52 playing cards. The remaining cards are well shuffled and then one card is drawn at random. Giving ace a value 1 similar value for other cards, find the probability that the card has a value 
		\begin{enumerate}
			\item 7
			\item greater than 7
			\item less than 7
		\end{enumerate}
		%\input{exemplar/10/13/3/30/main.tex}
  \item A Lot consists of 48 mobile phones of which 42 are good, 3 have only minor defects and 3 have major defects.Varnika will buy a phone if it is good but the trader will only buy a mobile if it has no major defects. One phone is selected at random from the lot. What is the probability that it is
\begin{enumerate}
	\item acceptable to Varnika?
            \item acceptable to the trader?
\end{enumerate}
\solution
	%\input{exemplar/10/13/3/40/main.tex}
 \item A student says that if you throw a die, it will show up 1 or not 1. Therefore, the probability of getting 1 and the probability of getting 'not 1' each is equal to $\frac{1}{2}$. Is this correct? Give reasons.\\
 \solution
        %\input{exemplar/10/13/2/9/main.tex}
   \item Four candidates A, B, C, D have ap-
plied for the assignment to coach a school cricket
team. If A is twice as likely to be selected as B, and
B and C are given about the same chance of being
selected, while C is twice as likely to be selected
as D, what are the probabilities that
\begin{enumerate}
\item C will be selected?
\item A will not be selected?
\end{enumerate}
	%\input{exemplar/11/16/3/9/main.tex}
 \item A bag contain 24 balls of which $x$ balls are red, $2x$ are white and $3x$ are blue. A ball is selected at random, What is the probability that it is
\begin{enumerate}[label=\alph*)]
\item not red ?
\item white ?
\end{enumerate}
%\input{exemplar/10/13/3/41/main.tex}
If the letters of the word ASSASSINATION are arranged at random. Find the Probability that
\begin{enumerate}[label=(\alph*)]
\item Four $S's$ come consecutively in the word
\item Two  $I's$ and two $N's$ come together
\item All $A's$ are not coming together
\item No two $A's$ are coming together
\end{enumerate}
%\input{exemplar/11/16/3/14/main.tex}
	\item One urn contains two black balls (labelled B1 and B2) and one white ball. A
	second urn contains one black ball and two white balls (labelled W1 and W2).
	Suppose the following experiment is performed. One of the two urns is chosen
	at random. Next a ball is randomly chosen from the urn. Then a second ball is
	chosen at random from the same urn without replacing the first ball.
	
	\begin{enumerate}
	\item What is the probability that two black balls are chosen?
	
	\item What is the probability that two balls of opposite colour are chosen?
	\end{enumerate}
	\solution
	%\input{exemplar/11/16/3/12/main1.tex}
\end{enumerate}

		%
\item 
Out of 100 students, two sections of 40 and 60 are formed. If you and your friend are among the 100 students, what is the probability that
\begin{enumerate}
\item you both enter the same section?
\item you both enter the different sections?
\end{enumerate}
\solution
		%\begin{enumerate}[label=\thesection.\arabic*,ref=\thesection.\theenumi]
	\item One card is drawn from a well-shuffled deck of 52 cards. Find the probability of getting
\begin{enumerate}
\item A king of red colour 
\item A face card 
\item A red face card
\item The jack of hearts
\item A spade
\item The queen of diamonds

\end{enumerate}
\solution
		%\input{ncert/10/15/1/14/main.tex}
	\item Five cards—the ten, jack, queen, king and ace of diamonds, are well-shuffled with their face downwards. One card is then picked up at random.
\begin{enumerate}
\item
What is the probability that the card is the queen? 
\item
If the queen is drawn and put aside, what is the probability that the second card picked up is (a) an ace? (b) a queen?\\
\end{enumerate}
\solution
		%\input{ncert/10/15/1/15/defs.tex}
	\item A bag contains $5$ red balls and some blue balls. If the probability of drawing a blue ball is double that if a red ball, determine the number of blue balls in the bag. 
		\\
\solution
		%\input{ncert/10/15/2/3/defs.tex}
	\item A card is selected from a pack of 52 cards.
 \begin{enumerate}[label=(\alph*)] 
                 \item How many points are there in the sample space?
                 \item Calculate the probability that the card is an ace of spades.
                 \item Calculate the probability that the card is (i) an ace and (ii) black card.
 \end{enumerate}
\solution
		%\input{ncert/11/16/3/4/main.tex}
\item Four cards are drawn from a well-shuffled deck of 52 cards. What is the probability of obtaining 3 diamonds and one spade.
\\
\solution
		%\input{ncert/11/16/4/2/defs.tex}
\item In a certain lottery 10,000 tickets are sold and ten equal prizes are awarded. What is the probability of not getting a prize if you buy (a) one ticket (b) two tickets (c) 10 tickets ?	
\\
\solution
		%\input{ncert/11/16/4/4/defs.tex}
		%
\item 
Out of 100 students, two sections of 40 and 60 are formed. If you and your friend are among the 100 students, what is the probability that
\begin{enumerate}
\item you both enter the same section?
\item you both enter the different sections?
\end{enumerate}
\solution
		%\input{ncert/11/16/4/5/defs.tex}
	\item 
The number lock of a suitcase has 4 wheels each labelled with ten digits i.e. from 0 to 9.The lock opens with a sequence of four digits with no repeats.What is the probability of a person getting the right sequence to open the suitcase.
\\
\solution
		%\input{ncert/11/16/4/10/defs.tex}
		%
\item 
Two cards are drawn at random and without replacement from a pack of 52 playing cards. Find the probability that both the cards are black.
\\
\solution
		%\input{ncert/12/13/2/2/defs.tex}
		\item A box of oranges is inspected by examining three randomly selected oranges drawn without replacement. If all the three oranges are good, the box is approved for sale, otherwise, it is rejected. Find the probability that a box containing 15 oranges out of which 12 are good and 3 are bad ones will be approved for sale.
		\label{ncert/12/13/2/3/defs.tex}
		\item Two balls are drawn at random with replacement from a box containing 10 black and 8 red balls. Find the probability that
		\label{ncert/12/13/2/12}
\begin{enumerate}
\item both balls are red.
\item first ball is black and second is red.
\item one of them is black and other is red.
\end{enumerate}

\item In a hostel, 60\% of the students read Hindi newspaper, 40\% read English newspaper and 20\% read both Hindi and English newspapers. A student is selected at random.
		\label{ncert/12/13/2/15}
\begin{enumerate}
\item Find the probability that she reads neither Hindi nor English newspapers.
\item If she reads Hindi newspaper, find the probability that she reads English newspaper.
\item If she reads English newspaper, find the probability that she reads Hindi newspaper.\\
\end{enumerate}
\item The probability of obtaining an even prime number on each die, when a pair of dice is rolled is 
\begin{enumerate}
    \item $0$ 
    
    \item $\frac{1}{3}$ 
    
    \item $\frac{1}{12}$ 
    
    \item $\frac{1}{36}$ 
\end{enumerate}
\solution
		%\input{ncert/12/13/2/17/defs.tex}
	\item A bag contains 4 red and 4 black balls, another bag contains 2 red and 6 black balls. One of the two bags is selected at random and a ball is drawn from the bag which is found to be red. Find the probability that the ball is drawn from the first bag.
\\
\solution
		%\input{ncert/12/13/3/2/main.tex}
  \item
  Cards with numbers 2 to 101 are placed in a box. A card is selected at random.Find the probability that the card has
\begin{enumerate}[label=(\roman*)]
	\item an even number 
	\item a square number
\end{enumerate}
\solution
%\input{exemplar/10/13/3/32/main.tex}
\item
The king, queen and jack of clubs are removed from a deck of 52 playing cards and then well shuffled. Now one card is drawn at random from the remaining cards.  Determine the probability that the card is
\begin{enumerate}[label=(\roman*)]
\item a club
\item 10 of hearts
\end{enumerate}
\solution
%\input{exemplar/10/13/3/29/main.tex}
\item A team of medical students doing their internship have to assist during surgeries
at a city hospital. The probabilities of surgeries rated as very complex, complex,
routine, simple or very simple are respectively, 0.15, 0.20, 0.31, 0.26, .08. Find
the probabilities that a particular surgery will be rated
\begin{enumerate}
	\item complex or very complex;
	\item neither very complex nor very simple;
	\item routine or complex
	\item routine or simple
\end{enumerate}
\solution
%\input{exemplar/11/16/3/8(1)/main.tex}
\item A card is selected from a pack of 52 cards.
\begin{enumerate}[label=(\alph*)]
    \item How many points are there in the sample space?
    \item Calculate the probability that the card is an ace of spades.
    \item Calculate the probability that the card is (i) an ace and (ii) black card.
\end{enumerate}
\solution
%\input{exemplar/11/16/3/4/main2.tex}
\item The probability that a non leap year selected at random will contain 53 sundays.
\\
\solution
%\input{exemplar/10/13/1/19/main.tex}
\item One of the four persons John, Rita, Aslam or Gurpreet will be promoted next
month. Consequently the sample space consists of four elementary outcomes
S = {John promoted, Rita promoted, Aslam promoted, Gurpreet promoted}
You are told that the chances of John’s promotion is same as that of Gurpreet,
Rita’s chances of promotion are twice as likely as Johns. Aslam’s chances are
four times that of John.
\begin{enumerate}
	\item Determine
	\begin{enumerate}
		\item P (John promoted)
		\item P (Rita promoted)
		\item P (Aslam promoted)
		\item P (Gurpreet promoted)
	\end{enumerate}
	\item If A = {John promoted or Gurpreet promoted}, find P (A).
\end{enumerate}
\solution
%\input{exemplar/11/16/3/10/main.tex}
\item A card is drawn from a deck of 52 cards. Find the probability of getting a king or a heart or a red card.\\
\solution
%\input{exemplar/11/16/3/15/main.tex}
\item The probability that a student will pass his examination is 0.73, the probability of
the student getting a compartment is 0.13, and the probability that the student will
either pass or get compartment is 0.96. State True or False.\\
\solution
%\input{exemplar/11/16/3/31/main.tex}
\item A card is selected from a pack of 52 cards\\
\begin{enumerate}[label=(\alph*)]
\item How many points are there in the sample space?
\item Calculate the probability that the cards is an ace of spades.
\item Calculate the probability that the card is (i) an ace (ii)black card.\\
\end{enumerate}
%\input{ncert/11/16/3/4_1/Prob_4.tex}
\item In a non-leap year, the probability of having 53 tuesdays or 53 wednesdays is\\
\solution
%\input{exemplar/11/16/3/18/main.tex}
\item There are 1000 sealed envelopes in a box, 10 of them contain a cash prize of
Rs 100 each, 100 of them contain a cash prize of Rs 50 each and 200 of them
contain a cash prize of Rs 10 each and rest do not contain any cash prize. If they
are well shuffled and an envelope is picked up out, what is the probability that it
contains no cash prize?\\
\solution
%\input{exemplar/10/13/3/34/main.tex}
\item 
A die is thrown and a card is selected at random from a deck of 52 playing cards. The probability of getting an even number on the die and a spade card.\\
\solution
%\input{exemplar/12/13/3/78/main.tex}
\item
If 4-digit numbers greater than 5,000 are randomly formed from the digits 0, 1, 3, 5, and 7, what is the probability of forming a number divisible by 5 when:
\begin{enumerate}
    \item The digits are repeated?
    \item The repetition of digits is not allowed?
\end{enumerate}
\solution
%\input{ncert/11/16/4/9/main.tex}
\item Consider the probability space $\brak{\Omega, \mathcal{G}, P}$ where $\Omega = [0,2]$ and $\mathcal{G} = \cbrak{\phi, \Omega, [0,1], (1,2]}$. Let $X$ and $Y$ be two functions on $\Omega$ defined as
\begin{align*}
    X(\omega) = 
    \begin{cases}
        1 & \text{if }\omega \in [0, 1]\\
        2 & \text{if }\omega \in (1, 2]
    \end{cases}
\end{align*}
and
\begin{align*}
    Y(\omega) = 
    \begin{cases}
        2 & \text{if }\omega \in [0, 1.5]\\
        3 & \text{if }\omega \in (1.5, 2].
    \end{cases}
\end{align*}
Then which one of the following statements is true?
\begin{enumerate}
    \item [(A)] $X$ is a random variable with respect to $\mathcal{G}$, but $Y$ is not a random variable with respect to $\mathcal{G}$.
    \item [(B)] $Y$ is a random variable with respect to $\mathcal{G}$, but $X$ is not a random variable with respect to $\mathcal{G}$.
    \item [(C)] Neither $X$ nor $Y$ is a random variable with respect to $\mathcal{G}$.
    \item [(D)] Both $X$ and $Y$ are random variables with respect to $\mathcal{G}$.
\end{enumerate} \hfill (GATE ST 2023)\\
\solution
%\input{gate/ST/2023/14/main.tex}
	\item  A die is loaded in such a way that each odd number is twice as likely to occur as
each even number. Find $P(G)$, where $G$ is the event that a number greater than
3 occurs on a single roll of the die.
\\
\solution
		%\input{exemplar/11/16/3/5/main.tex}
	\item All the jacks, queens and kings are removed from a deck of 52 playing cards. The remaining cards are well shuffled and then one card is drawn at random. Giving ace a value 1 similar value for other cards, find the probability that the card has a value 
		\begin{enumerate}
			\item 7
			\item greater than 7
			\item less than 7
		\end{enumerate}
		%\input{exemplar/10/13/3/30/main.tex}
  \item A Lot consists of 48 mobile phones of which 42 are good, 3 have only minor defects and 3 have major defects.Varnika will buy a phone if it is good but the trader will only buy a mobile if it has no major defects. One phone is selected at random from the lot. What is the probability that it is
\begin{enumerate}
	\item acceptable to Varnika?
            \item acceptable to the trader?
\end{enumerate}
\solution
	%\input{exemplar/10/13/3/40/main.tex}
 \item A student says that if you throw a die, it will show up 1 or not 1. Therefore, the probability of getting 1 and the probability of getting 'not 1' each is equal to $\frac{1}{2}$. Is this correct? Give reasons.\\
 \solution
        %\input{exemplar/10/13/2/9/main.tex}
   \item Four candidates A, B, C, D have ap-
plied for the assignment to coach a school cricket
team. If A is twice as likely to be selected as B, and
B and C are given about the same chance of being
selected, while C is twice as likely to be selected
as D, what are the probabilities that
\begin{enumerate}
\item C will be selected?
\item A will not be selected?
\end{enumerate}
	%\input{exemplar/11/16/3/9/main.tex}
 \item A bag contain 24 balls of which $x$ balls are red, $2x$ are white and $3x$ are blue. A ball is selected at random, What is the probability that it is
\begin{enumerate}[label=\alph*)]
\item not red ?
\item white ?
\end{enumerate}
%\input{exemplar/10/13/3/41/main.tex}
If the letters of the word ASSASSINATION are arranged at random. Find the Probability that
\begin{enumerate}[label=(\alph*)]
\item Four $S's$ come consecutively in the word
\item Two  $I's$ and two $N's$ come together
\item All $A's$ are not coming together
\item No two $A's$ are coming together
\end{enumerate}
%\input{exemplar/11/16/3/14/main.tex}
	\item One urn contains two black balls (labelled B1 and B2) and one white ball. A
	second urn contains one black ball and two white balls (labelled W1 and W2).
	Suppose the following experiment is performed. One of the two urns is chosen
	at random. Next a ball is randomly chosen from the urn. Then a second ball is
	chosen at random from the same urn without replacing the first ball.
	
	\begin{enumerate}
	\item What is the probability that two black balls are chosen?
	
	\item What is the probability that two balls of opposite colour are chosen?
	\end{enumerate}
	\solution
	%\input{exemplar/11/16/3/12/main1.tex}
\end{enumerate}

	\item 
The number lock of a suitcase has 4 wheels each labelled with ten digits i.e. from 0 to 9.The lock opens with a sequence of four digits with no repeats.What is the probability of a person getting the right sequence to open the suitcase.
\\
\solution
		%\begin{enumerate}[label=\thesection.\arabic*,ref=\thesection.\theenumi]
	\item One card is drawn from a well-shuffled deck of 52 cards. Find the probability of getting
\begin{enumerate}
\item A king of red colour 
\item A face card 
\item A red face card
\item The jack of hearts
\item A spade
\item The queen of diamonds

\end{enumerate}
\solution
		%\input{ncert/10/15/1/14/main.tex}
	\item Five cards—the ten, jack, queen, king and ace of diamonds, are well-shuffled with their face downwards. One card is then picked up at random.
\begin{enumerate}
\item
What is the probability that the card is the queen? 
\item
If the queen is drawn and put aside, what is the probability that the second card picked up is (a) an ace? (b) a queen?\\
\end{enumerate}
\solution
		%\input{ncert/10/15/1/15/defs.tex}
	\item A bag contains $5$ red balls and some blue balls. If the probability of drawing a blue ball is double that if a red ball, determine the number of blue balls in the bag. 
		\\
\solution
		%\input{ncert/10/15/2/3/defs.tex}
	\item A card is selected from a pack of 52 cards.
 \begin{enumerate}[label=(\alph*)] 
                 \item How many points are there in the sample space?
                 \item Calculate the probability that the card is an ace of spades.
                 \item Calculate the probability that the card is (i) an ace and (ii) black card.
 \end{enumerate}
\solution
		%\input{ncert/11/16/3/4/main.tex}
\item Four cards are drawn from a well-shuffled deck of 52 cards. What is the probability of obtaining 3 diamonds and one spade.
\\
\solution
		%\input{ncert/11/16/4/2/defs.tex}
\item In a certain lottery 10,000 tickets are sold and ten equal prizes are awarded. What is the probability of not getting a prize if you buy (a) one ticket (b) two tickets (c) 10 tickets ?	
\\
\solution
		%\input{ncert/11/16/4/4/defs.tex}
		%
\item 
Out of 100 students, two sections of 40 and 60 are formed. If you and your friend are among the 100 students, what is the probability that
\begin{enumerate}
\item you both enter the same section?
\item you both enter the different sections?
\end{enumerate}
\solution
		%\input{ncert/11/16/4/5/defs.tex}
	\item 
The number lock of a suitcase has 4 wheels each labelled with ten digits i.e. from 0 to 9.The lock opens with a sequence of four digits with no repeats.What is the probability of a person getting the right sequence to open the suitcase.
\\
\solution
		%\input{ncert/11/16/4/10/defs.tex}
		%
\item 
Two cards are drawn at random and without replacement from a pack of 52 playing cards. Find the probability that both the cards are black.
\\
\solution
		%\input{ncert/12/13/2/2/defs.tex}
		\item A box of oranges is inspected by examining three randomly selected oranges drawn without replacement. If all the three oranges are good, the box is approved for sale, otherwise, it is rejected. Find the probability that a box containing 15 oranges out of which 12 are good and 3 are bad ones will be approved for sale.
		\label{ncert/12/13/2/3/defs.tex}
		\item Two balls are drawn at random with replacement from a box containing 10 black and 8 red balls. Find the probability that
		\label{ncert/12/13/2/12}
\begin{enumerate}
\item both balls are red.
\item first ball is black and second is red.
\item one of them is black and other is red.
\end{enumerate}

\item In a hostel, 60\% of the students read Hindi newspaper, 40\% read English newspaper and 20\% read both Hindi and English newspapers. A student is selected at random.
		\label{ncert/12/13/2/15}
\begin{enumerate}
\item Find the probability that she reads neither Hindi nor English newspapers.
\item If she reads Hindi newspaper, find the probability that she reads English newspaper.
\item If she reads English newspaper, find the probability that she reads Hindi newspaper.\\
\end{enumerate}
\item The probability of obtaining an even prime number on each die, when a pair of dice is rolled is 
\begin{enumerate}
    \item $0$ 
    
    \item $\frac{1}{3}$ 
    
    \item $\frac{1}{12}$ 
    
    \item $\frac{1}{36}$ 
\end{enumerate}
\solution
		%\input{ncert/12/13/2/17/defs.tex}
	\item A bag contains 4 red and 4 black balls, another bag contains 2 red and 6 black balls. One of the two bags is selected at random and a ball is drawn from the bag which is found to be red. Find the probability that the ball is drawn from the first bag.
\\
\solution
		%\input{ncert/12/13/3/2/main.tex}
  \item
  Cards with numbers 2 to 101 are placed in a box. A card is selected at random.Find the probability that the card has
\begin{enumerate}[label=(\roman*)]
	\item an even number 
	\item a square number
\end{enumerate}
\solution
%\input{exemplar/10/13/3/32/main.tex}
\item
The king, queen and jack of clubs are removed from a deck of 52 playing cards and then well shuffled. Now one card is drawn at random from the remaining cards.  Determine the probability that the card is
\begin{enumerate}[label=(\roman*)]
\item a club
\item 10 of hearts
\end{enumerate}
\solution
%\input{exemplar/10/13/3/29/main.tex}
\item A team of medical students doing their internship have to assist during surgeries
at a city hospital. The probabilities of surgeries rated as very complex, complex,
routine, simple or very simple are respectively, 0.15, 0.20, 0.31, 0.26, .08. Find
the probabilities that a particular surgery will be rated
\begin{enumerate}
	\item complex or very complex;
	\item neither very complex nor very simple;
	\item routine or complex
	\item routine or simple
\end{enumerate}
\solution
%\input{exemplar/11/16/3/8(1)/main.tex}
\item A card is selected from a pack of 52 cards.
\begin{enumerate}[label=(\alph*)]
    \item How many points are there in the sample space?
    \item Calculate the probability that the card is an ace of spades.
    \item Calculate the probability that the card is (i) an ace and (ii) black card.
\end{enumerate}
\solution
%\input{exemplar/11/16/3/4/main2.tex}
\item The probability that a non leap year selected at random will contain 53 sundays.
\\
\solution
%\input{exemplar/10/13/1/19/main.tex}
\item One of the four persons John, Rita, Aslam or Gurpreet will be promoted next
month. Consequently the sample space consists of four elementary outcomes
S = {John promoted, Rita promoted, Aslam promoted, Gurpreet promoted}
You are told that the chances of John’s promotion is same as that of Gurpreet,
Rita’s chances of promotion are twice as likely as Johns. Aslam’s chances are
four times that of John.
\begin{enumerate}
	\item Determine
	\begin{enumerate}
		\item P (John promoted)
		\item P (Rita promoted)
		\item P (Aslam promoted)
		\item P (Gurpreet promoted)
	\end{enumerate}
	\item If A = {John promoted or Gurpreet promoted}, find P (A).
\end{enumerate}
\solution
%\input{exemplar/11/16/3/10/main.tex}
\item A card is drawn from a deck of 52 cards. Find the probability of getting a king or a heart or a red card.\\
\solution
%\input{exemplar/11/16/3/15/main.tex}
\item The probability that a student will pass his examination is 0.73, the probability of
the student getting a compartment is 0.13, and the probability that the student will
either pass or get compartment is 0.96. State True or False.\\
\solution
%\input{exemplar/11/16/3/31/main.tex}
\item A card is selected from a pack of 52 cards\\
\begin{enumerate}[label=(\alph*)]
\item How many points are there in the sample space?
\item Calculate the probability that the cards is an ace of spades.
\item Calculate the probability that the card is (i) an ace (ii)black card.\\
\end{enumerate}
%\input{ncert/11/16/3/4_1/Prob_4.tex}
\item In a non-leap year, the probability of having 53 tuesdays or 53 wednesdays is\\
\solution
%\input{exemplar/11/16/3/18/main.tex}
\item There are 1000 sealed envelopes in a box, 10 of them contain a cash prize of
Rs 100 each, 100 of them contain a cash prize of Rs 50 each and 200 of them
contain a cash prize of Rs 10 each and rest do not contain any cash prize. If they
are well shuffled and an envelope is picked up out, what is the probability that it
contains no cash prize?\\
\solution
%\input{exemplar/10/13/3/34/main.tex}
\item 
A die is thrown and a card is selected at random from a deck of 52 playing cards. The probability of getting an even number on the die and a spade card.\\
\solution
%\input{exemplar/12/13/3/78/main.tex}
\item
If 4-digit numbers greater than 5,000 are randomly formed from the digits 0, 1, 3, 5, and 7, what is the probability of forming a number divisible by 5 when:
\begin{enumerate}
    \item The digits are repeated?
    \item The repetition of digits is not allowed?
\end{enumerate}
\solution
%\input{ncert/11/16/4/9/main.tex}
\item Consider the probability space $\brak{\Omega, \mathcal{G}, P}$ where $\Omega = [0,2]$ and $\mathcal{G} = \cbrak{\phi, \Omega, [0,1], (1,2]}$. Let $X$ and $Y$ be two functions on $\Omega$ defined as
\begin{align*}
    X(\omega) = 
    \begin{cases}
        1 & \text{if }\omega \in [0, 1]\\
        2 & \text{if }\omega \in (1, 2]
    \end{cases}
\end{align*}
and
\begin{align*}
    Y(\omega) = 
    \begin{cases}
        2 & \text{if }\omega \in [0, 1.5]\\
        3 & \text{if }\omega \in (1.5, 2].
    \end{cases}
\end{align*}
Then which one of the following statements is true?
\begin{enumerate}
    \item [(A)] $X$ is a random variable with respect to $\mathcal{G}$, but $Y$ is not a random variable with respect to $\mathcal{G}$.
    \item [(B)] $Y$ is a random variable with respect to $\mathcal{G}$, but $X$ is not a random variable with respect to $\mathcal{G}$.
    \item [(C)] Neither $X$ nor $Y$ is a random variable with respect to $\mathcal{G}$.
    \item [(D)] Both $X$ and $Y$ are random variables with respect to $\mathcal{G}$.
\end{enumerate} \hfill (GATE ST 2023)\\
\solution
%\input{gate/ST/2023/14/main.tex}
	\item  A die is loaded in such a way that each odd number is twice as likely to occur as
each even number. Find $P(G)$, where $G$ is the event that a number greater than
3 occurs on a single roll of the die.
\\
\solution
		%\input{exemplar/11/16/3/5/main.tex}
	\item All the jacks, queens and kings are removed from a deck of 52 playing cards. The remaining cards are well shuffled and then one card is drawn at random. Giving ace a value 1 similar value for other cards, find the probability that the card has a value 
		\begin{enumerate}
			\item 7
			\item greater than 7
			\item less than 7
		\end{enumerate}
		%\input{exemplar/10/13/3/30/main.tex}
  \item A Lot consists of 48 mobile phones of which 42 are good, 3 have only minor defects and 3 have major defects.Varnika will buy a phone if it is good but the trader will only buy a mobile if it has no major defects. One phone is selected at random from the lot. What is the probability that it is
\begin{enumerate}
	\item acceptable to Varnika?
            \item acceptable to the trader?
\end{enumerate}
\solution
	%\input{exemplar/10/13/3/40/main.tex}
 \item A student says that if you throw a die, it will show up 1 or not 1. Therefore, the probability of getting 1 and the probability of getting 'not 1' each is equal to $\frac{1}{2}$. Is this correct? Give reasons.\\
 \solution
        %\input{exemplar/10/13/2/9/main.tex}
   \item Four candidates A, B, C, D have ap-
plied for the assignment to coach a school cricket
team. If A is twice as likely to be selected as B, and
B and C are given about the same chance of being
selected, while C is twice as likely to be selected
as D, what are the probabilities that
\begin{enumerate}
\item C will be selected?
\item A will not be selected?
\end{enumerate}
	%\input{exemplar/11/16/3/9/main.tex}
 \item A bag contain 24 balls of which $x$ balls are red, $2x$ are white and $3x$ are blue. A ball is selected at random, What is the probability that it is
\begin{enumerate}[label=\alph*)]
\item not red ?
\item white ?
\end{enumerate}
%\input{exemplar/10/13/3/41/main.tex}
If the letters of the word ASSASSINATION are arranged at random. Find the Probability that
\begin{enumerate}[label=(\alph*)]
\item Four $S's$ come consecutively in the word
\item Two  $I's$ and two $N's$ come together
\item All $A's$ are not coming together
\item No two $A's$ are coming together
\end{enumerate}
%\input{exemplar/11/16/3/14/main.tex}
	\item One urn contains two black balls (labelled B1 and B2) and one white ball. A
	second urn contains one black ball and two white balls (labelled W1 and W2).
	Suppose the following experiment is performed. One of the two urns is chosen
	at random. Next a ball is randomly chosen from the urn. Then a second ball is
	chosen at random from the same urn without replacing the first ball.
	
	\begin{enumerate}
	\item What is the probability that two black balls are chosen?
	
	\item What is the probability that two balls of opposite colour are chosen?
	\end{enumerate}
	\solution
	%\input{exemplar/11/16/3/12/main1.tex}
\end{enumerate}

		%
\item 
Two cards are drawn at random and without replacement from a pack of 52 playing cards. Find the probability that both the cards are black.
\\
\solution
		%\begin{enumerate}[label=\thesection.\arabic*,ref=\thesection.\theenumi]
	\item One card is drawn from a well-shuffled deck of 52 cards. Find the probability of getting
\begin{enumerate}
\item A king of red colour 
\item A face card 
\item A red face card
\item The jack of hearts
\item A spade
\item The queen of diamonds

\end{enumerate}
\solution
		%\input{ncert/10/15/1/14/main.tex}
	\item Five cards—the ten, jack, queen, king and ace of diamonds, are well-shuffled with their face downwards. One card is then picked up at random.
\begin{enumerate}
\item
What is the probability that the card is the queen? 
\item
If the queen is drawn and put aside, what is the probability that the second card picked up is (a) an ace? (b) a queen?\\
\end{enumerate}
\solution
		%\input{ncert/10/15/1/15/defs.tex}
	\item A bag contains $5$ red balls and some blue balls. If the probability of drawing a blue ball is double that if a red ball, determine the number of blue balls in the bag. 
		\\
\solution
		%\input{ncert/10/15/2/3/defs.tex}
	\item A card is selected from a pack of 52 cards.
 \begin{enumerate}[label=(\alph*)] 
                 \item How many points are there in the sample space?
                 \item Calculate the probability that the card is an ace of spades.
                 \item Calculate the probability that the card is (i) an ace and (ii) black card.
 \end{enumerate}
\solution
		%\input{ncert/11/16/3/4/main.tex}
\item Four cards are drawn from a well-shuffled deck of 52 cards. What is the probability of obtaining 3 diamonds and one spade.
\\
\solution
		%\input{ncert/11/16/4/2/defs.tex}
\item In a certain lottery 10,000 tickets are sold and ten equal prizes are awarded. What is the probability of not getting a prize if you buy (a) one ticket (b) two tickets (c) 10 tickets ?	
\\
\solution
		%\input{ncert/11/16/4/4/defs.tex}
		%
\item 
Out of 100 students, two sections of 40 and 60 are formed. If you and your friend are among the 100 students, what is the probability that
\begin{enumerate}
\item you both enter the same section?
\item you both enter the different sections?
\end{enumerate}
\solution
		%\input{ncert/11/16/4/5/defs.tex}
	\item 
The number lock of a suitcase has 4 wheels each labelled with ten digits i.e. from 0 to 9.The lock opens with a sequence of four digits with no repeats.What is the probability of a person getting the right sequence to open the suitcase.
\\
\solution
		%\input{ncert/11/16/4/10/defs.tex}
		%
\item 
Two cards are drawn at random and without replacement from a pack of 52 playing cards. Find the probability that both the cards are black.
\\
\solution
		%\input{ncert/12/13/2/2/defs.tex}
		\item A box of oranges is inspected by examining three randomly selected oranges drawn without replacement. If all the three oranges are good, the box is approved for sale, otherwise, it is rejected. Find the probability that a box containing 15 oranges out of which 12 are good and 3 are bad ones will be approved for sale.
		\label{ncert/12/13/2/3/defs.tex}
		\item Two balls are drawn at random with replacement from a box containing 10 black and 8 red balls. Find the probability that
		\label{ncert/12/13/2/12}
\begin{enumerate}
\item both balls are red.
\item first ball is black and second is red.
\item one of them is black and other is red.
\end{enumerate}

\item In a hostel, 60\% of the students read Hindi newspaper, 40\% read English newspaper and 20\% read both Hindi and English newspapers. A student is selected at random.
		\label{ncert/12/13/2/15}
\begin{enumerate}
\item Find the probability that she reads neither Hindi nor English newspapers.
\item If she reads Hindi newspaper, find the probability that she reads English newspaper.
\item If she reads English newspaper, find the probability that she reads Hindi newspaper.\\
\end{enumerate}
\item The probability of obtaining an even prime number on each die, when a pair of dice is rolled is 
\begin{enumerate}
    \item $0$ 
    
    \item $\frac{1}{3}$ 
    
    \item $\frac{1}{12}$ 
    
    \item $\frac{1}{36}$ 
\end{enumerate}
\solution
		%\input{ncert/12/13/2/17/defs.tex}
	\item A bag contains 4 red and 4 black balls, another bag contains 2 red and 6 black balls. One of the two bags is selected at random and a ball is drawn from the bag which is found to be red. Find the probability that the ball is drawn from the first bag.
\\
\solution
		%\input{ncert/12/13/3/2/main.tex}
  \item
  Cards with numbers 2 to 101 are placed in a box. A card is selected at random.Find the probability that the card has
\begin{enumerate}[label=(\roman*)]
	\item an even number 
	\item a square number
\end{enumerate}
\solution
%\input{exemplar/10/13/3/32/main.tex}
\item
The king, queen and jack of clubs are removed from a deck of 52 playing cards and then well shuffled. Now one card is drawn at random from the remaining cards.  Determine the probability that the card is
\begin{enumerate}[label=(\roman*)]
\item a club
\item 10 of hearts
\end{enumerate}
\solution
%\input{exemplar/10/13/3/29/main.tex}
\item A team of medical students doing their internship have to assist during surgeries
at a city hospital. The probabilities of surgeries rated as very complex, complex,
routine, simple or very simple are respectively, 0.15, 0.20, 0.31, 0.26, .08. Find
the probabilities that a particular surgery will be rated
\begin{enumerate}
	\item complex or very complex;
	\item neither very complex nor very simple;
	\item routine or complex
	\item routine or simple
\end{enumerate}
\solution
%\input{exemplar/11/16/3/8(1)/main.tex}
\item A card is selected from a pack of 52 cards.
\begin{enumerate}[label=(\alph*)]
    \item How many points are there in the sample space?
    \item Calculate the probability that the card is an ace of spades.
    \item Calculate the probability that the card is (i) an ace and (ii) black card.
\end{enumerate}
\solution
%\input{exemplar/11/16/3/4/main2.tex}
\item The probability that a non leap year selected at random will contain 53 sundays.
\\
\solution
%\input{exemplar/10/13/1/19/main.tex}
\item One of the four persons John, Rita, Aslam or Gurpreet will be promoted next
month. Consequently the sample space consists of four elementary outcomes
S = {John promoted, Rita promoted, Aslam promoted, Gurpreet promoted}
You are told that the chances of John’s promotion is same as that of Gurpreet,
Rita’s chances of promotion are twice as likely as Johns. Aslam’s chances are
four times that of John.
\begin{enumerate}
	\item Determine
	\begin{enumerate}
		\item P (John promoted)
		\item P (Rita promoted)
		\item P (Aslam promoted)
		\item P (Gurpreet promoted)
	\end{enumerate}
	\item If A = {John promoted or Gurpreet promoted}, find P (A).
\end{enumerate}
\solution
%\input{exemplar/11/16/3/10/main.tex}
\item A card is drawn from a deck of 52 cards. Find the probability of getting a king or a heart or a red card.\\
\solution
%\input{exemplar/11/16/3/15/main.tex}
\item The probability that a student will pass his examination is 0.73, the probability of
the student getting a compartment is 0.13, and the probability that the student will
either pass or get compartment is 0.96. State True or False.\\
\solution
%\input{exemplar/11/16/3/31/main.tex}
\item A card is selected from a pack of 52 cards\\
\begin{enumerate}[label=(\alph*)]
\item How many points are there in the sample space?
\item Calculate the probability that the cards is an ace of spades.
\item Calculate the probability that the card is (i) an ace (ii)black card.\\
\end{enumerate}
%\input{ncert/11/16/3/4_1/Prob_4.tex}
\item In a non-leap year, the probability of having 53 tuesdays or 53 wednesdays is\\
\solution
%\input{exemplar/11/16/3/18/main.tex}
\item There are 1000 sealed envelopes in a box, 10 of them contain a cash prize of
Rs 100 each, 100 of them contain a cash prize of Rs 50 each and 200 of them
contain a cash prize of Rs 10 each and rest do not contain any cash prize. If they
are well shuffled and an envelope is picked up out, what is the probability that it
contains no cash prize?\\
\solution
%\input{exemplar/10/13/3/34/main.tex}
\item 
A die is thrown and a card is selected at random from a deck of 52 playing cards. The probability of getting an even number on the die and a spade card.\\
\solution
%\input{exemplar/12/13/3/78/main.tex}
\item
If 4-digit numbers greater than 5,000 are randomly formed from the digits 0, 1, 3, 5, and 7, what is the probability of forming a number divisible by 5 when:
\begin{enumerate}
    \item The digits are repeated?
    \item The repetition of digits is not allowed?
\end{enumerate}
\solution
%\input{ncert/11/16/4/9/main.tex}
\item Consider the probability space $\brak{\Omega, \mathcal{G}, P}$ where $\Omega = [0,2]$ and $\mathcal{G} = \cbrak{\phi, \Omega, [0,1], (1,2]}$. Let $X$ and $Y$ be two functions on $\Omega$ defined as
\begin{align*}
    X(\omega) = 
    \begin{cases}
        1 & \text{if }\omega \in [0, 1]\\
        2 & \text{if }\omega \in (1, 2]
    \end{cases}
\end{align*}
and
\begin{align*}
    Y(\omega) = 
    \begin{cases}
        2 & \text{if }\omega \in [0, 1.5]\\
        3 & \text{if }\omega \in (1.5, 2].
    \end{cases}
\end{align*}
Then which one of the following statements is true?
\begin{enumerate}
    \item [(A)] $X$ is a random variable with respect to $\mathcal{G}$, but $Y$ is not a random variable with respect to $\mathcal{G}$.
    \item [(B)] $Y$ is a random variable with respect to $\mathcal{G}$, but $X$ is not a random variable with respect to $\mathcal{G}$.
    \item [(C)] Neither $X$ nor $Y$ is a random variable with respect to $\mathcal{G}$.
    \item [(D)] Both $X$ and $Y$ are random variables with respect to $\mathcal{G}$.
\end{enumerate} \hfill (GATE ST 2023)\\
\solution
%\input{gate/ST/2023/14/main.tex}
	\item  A die is loaded in such a way that each odd number is twice as likely to occur as
each even number. Find $P(G)$, where $G$ is the event that a number greater than
3 occurs on a single roll of the die.
\\
\solution
		%\input{exemplar/11/16/3/5/main.tex}
	\item All the jacks, queens and kings are removed from a deck of 52 playing cards. The remaining cards are well shuffled and then one card is drawn at random. Giving ace a value 1 similar value for other cards, find the probability that the card has a value 
		\begin{enumerate}
			\item 7
			\item greater than 7
			\item less than 7
		\end{enumerate}
		%\input{exemplar/10/13/3/30/main.tex}
  \item A Lot consists of 48 mobile phones of which 42 are good, 3 have only minor defects and 3 have major defects.Varnika will buy a phone if it is good but the trader will only buy a mobile if it has no major defects. One phone is selected at random from the lot. What is the probability that it is
\begin{enumerate}
	\item acceptable to Varnika?
            \item acceptable to the trader?
\end{enumerate}
\solution
	%\input{exemplar/10/13/3/40/main.tex}
 \item A student says that if you throw a die, it will show up 1 or not 1. Therefore, the probability of getting 1 and the probability of getting 'not 1' each is equal to $\frac{1}{2}$. Is this correct? Give reasons.\\
 \solution
        %\input{exemplar/10/13/2/9/main.tex}
   \item Four candidates A, B, C, D have ap-
plied for the assignment to coach a school cricket
team. If A is twice as likely to be selected as B, and
B and C are given about the same chance of being
selected, while C is twice as likely to be selected
as D, what are the probabilities that
\begin{enumerate}
\item C will be selected?
\item A will not be selected?
\end{enumerate}
	%\input{exemplar/11/16/3/9/main.tex}
 \item A bag contain 24 balls of which $x$ balls are red, $2x$ are white and $3x$ are blue. A ball is selected at random, What is the probability that it is
\begin{enumerate}[label=\alph*)]
\item not red ?
\item white ?
\end{enumerate}
%\input{exemplar/10/13/3/41/main.tex}
If the letters of the word ASSASSINATION are arranged at random. Find the Probability that
\begin{enumerate}[label=(\alph*)]
\item Four $S's$ come consecutively in the word
\item Two  $I's$ and two $N's$ come together
\item All $A's$ are not coming together
\item No two $A's$ are coming together
\end{enumerate}
%\input{exemplar/11/16/3/14/main.tex}
	\item One urn contains two black balls (labelled B1 and B2) and one white ball. A
	second urn contains one black ball and two white balls (labelled W1 and W2).
	Suppose the following experiment is performed. One of the two urns is chosen
	at random. Next a ball is randomly chosen from the urn. Then a second ball is
	chosen at random from the same urn without replacing the first ball.
	
	\begin{enumerate}
	\item What is the probability that two black balls are chosen?
	
	\item What is the probability that two balls of opposite colour are chosen?
	\end{enumerate}
	\solution
	%\input{exemplar/11/16/3/12/main1.tex}
\end{enumerate}

		\item A box of oranges is inspected by examining three randomly selected oranges drawn without replacement. If all the three oranges are good, the box is approved for sale, otherwise, it is rejected. Find the probability that a box containing 15 oranges out of which 12 are good and 3 are bad ones will be approved for sale.
		\label{ncert/12/13/2/3/defs.tex}
		\item Two balls are drawn at random with replacement from a box containing 10 black and 8 red balls. Find the probability that
		\label{ncert/12/13/2/12}
\begin{enumerate}
\item both balls are red.
\item first ball is black and second is red.
\item one of them is black and other is red.
\end{enumerate}

\item In a hostel, 60\% of the students read Hindi newspaper, 40\% read English newspaper and 20\% read both Hindi and English newspapers. A student is selected at random.
		\label{ncert/12/13/2/15}
\begin{enumerate}
\item Find the probability that she reads neither Hindi nor English newspapers.
\item If she reads Hindi newspaper, find the probability that she reads English newspaper.
\item If she reads English newspaper, find the probability that she reads Hindi newspaper.\\
\end{enumerate}
\item The probability of obtaining an even prime number on each die, when a pair of dice is rolled is 
\begin{enumerate}
    \item $0$ 
    
    \item $\frac{1}{3}$ 
    
    \item $\frac{1}{12}$ 
    
    \item $\frac{1}{36}$ 
\end{enumerate}
\solution
		%\begin{enumerate}[label=\thesection.\arabic*,ref=\thesection.\theenumi]
	\item One card is drawn from a well-shuffled deck of 52 cards. Find the probability of getting
\begin{enumerate}
\item A king of red colour 
\item A face card 
\item A red face card
\item The jack of hearts
\item A spade
\item The queen of diamonds

\end{enumerate}
\solution
		%\input{ncert/10/15/1/14/main.tex}
	\item Five cards—the ten, jack, queen, king and ace of diamonds, are well-shuffled with their face downwards. One card is then picked up at random.
\begin{enumerate}
\item
What is the probability that the card is the queen? 
\item
If the queen is drawn and put aside, what is the probability that the second card picked up is (a) an ace? (b) a queen?\\
\end{enumerate}
\solution
		%\input{ncert/10/15/1/15/defs.tex}
	\item A bag contains $5$ red balls and some blue balls. If the probability of drawing a blue ball is double that if a red ball, determine the number of blue balls in the bag. 
		\\
\solution
		%\input{ncert/10/15/2/3/defs.tex}
	\item A card is selected from a pack of 52 cards.
 \begin{enumerate}[label=(\alph*)] 
                 \item How many points are there in the sample space?
                 \item Calculate the probability that the card is an ace of spades.
                 \item Calculate the probability that the card is (i) an ace and (ii) black card.
 \end{enumerate}
\solution
		%\input{ncert/11/16/3/4/main.tex}
\item Four cards are drawn from a well-shuffled deck of 52 cards. What is the probability of obtaining 3 diamonds and one spade.
\\
\solution
		%\input{ncert/11/16/4/2/defs.tex}
\item In a certain lottery 10,000 tickets are sold and ten equal prizes are awarded. What is the probability of not getting a prize if you buy (a) one ticket (b) two tickets (c) 10 tickets ?	
\\
\solution
		%\input{ncert/11/16/4/4/defs.tex}
		%
\item 
Out of 100 students, two sections of 40 and 60 are formed. If you and your friend are among the 100 students, what is the probability that
\begin{enumerate}
\item you both enter the same section?
\item you both enter the different sections?
\end{enumerate}
\solution
		%\input{ncert/11/16/4/5/defs.tex}
	\item 
The number lock of a suitcase has 4 wheels each labelled with ten digits i.e. from 0 to 9.The lock opens with a sequence of four digits with no repeats.What is the probability of a person getting the right sequence to open the suitcase.
\\
\solution
		%\input{ncert/11/16/4/10/defs.tex}
		%
\item 
Two cards are drawn at random and without replacement from a pack of 52 playing cards. Find the probability that both the cards are black.
\\
\solution
		%\input{ncert/12/13/2/2/defs.tex}
		\item A box of oranges is inspected by examining three randomly selected oranges drawn without replacement. If all the three oranges are good, the box is approved for sale, otherwise, it is rejected. Find the probability that a box containing 15 oranges out of which 12 are good and 3 are bad ones will be approved for sale.
		\label{ncert/12/13/2/3/defs.tex}
		\item Two balls are drawn at random with replacement from a box containing 10 black and 8 red balls. Find the probability that
		\label{ncert/12/13/2/12}
\begin{enumerate}
\item both balls are red.
\item first ball is black and second is red.
\item one of them is black and other is red.
\end{enumerate}

\item In a hostel, 60\% of the students read Hindi newspaper, 40\% read English newspaper and 20\% read both Hindi and English newspapers. A student is selected at random.
		\label{ncert/12/13/2/15}
\begin{enumerate}
\item Find the probability that she reads neither Hindi nor English newspapers.
\item If she reads Hindi newspaper, find the probability that she reads English newspaper.
\item If she reads English newspaper, find the probability that she reads Hindi newspaper.\\
\end{enumerate}
\item The probability of obtaining an even prime number on each die, when a pair of dice is rolled is 
\begin{enumerate}
    \item $0$ 
    
    \item $\frac{1}{3}$ 
    
    \item $\frac{1}{12}$ 
    
    \item $\frac{1}{36}$ 
\end{enumerate}
\solution
		%\input{ncert/12/13/2/17/defs.tex}
	\item A bag contains 4 red and 4 black balls, another bag contains 2 red and 6 black balls. One of the two bags is selected at random and a ball is drawn from the bag which is found to be red. Find the probability that the ball is drawn from the first bag.
\\
\solution
		%\input{ncert/12/13/3/2/main.tex}
  \item
  Cards with numbers 2 to 101 are placed in a box. A card is selected at random.Find the probability that the card has
\begin{enumerate}[label=(\roman*)]
	\item an even number 
	\item a square number
\end{enumerate}
\solution
%\input{exemplar/10/13/3/32/main.tex}
\item
The king, queen and jack of clubs are removed from a deck of 52 playing cards and then well shuffled. Now one card is drawn at random from the remaining cards.  Determine the probability that the card is
\begin{enumerate}[label=(\roman*)]
\item a club
\item 10 of hearts
\end{enumerate}
\solution
%\input{exemplar/10/13/3/29/main.tex}
\item A team of medical students doing their internship have to assist during surgeries
at a city hospital. The probabilities of surgeries rated as very complex, complex,
routine, simple or very simple are respectively, 0.15, 0.20, 0.31, 0.26, .08. Find
the probabilities that a particular surgery will be rated
\begin{enumerate}
	\item complex or very complex;
	\item neither very complex nor very simple;
	\item routine or complex
	\item routine or simple
\end{enumerate}
\solution
%\input{exemplar/11/16/3/8(1)/main.tex}
\item A card is selected from a pack of 52 cards.
\begin{enumerate}[label=(\alph*)]
    \item How many points are there in the sample space?
    \item Calculate the probability that the card is an ace of spades.
    \item Calculate the probability that the card is (i) an ace and (ii) black card.
\end{enumerate}
\solution
%\input{exemplar/11/16/3/4/main2.tex}
\item The probability that a non leap year selected at random will contain 53 sundays.
\\
\solution
%\input{exemplar/10/13/1/19/main.tex}
\item One of the four persons John, Rita, Aslam or Gurpreet will be promoted next
month. Consequently the sample space consists of four elementary outcomes
S = {John promoted, Rita promoted, Aslam promoted, Gurpreet promoted}
You are told that the chances of John’s promotion is same as that of Gurpreet,
Rita’s chances of promotion are twice as likely as Johns. Aslam’s chances are
four times that of John.
\begin{enumerate}
	\item Determine
	\begin{enumerate}
		\item P (John promoted)
		\item P (Rita promoted)
		\item P (Aslam promoted)
		\item P (Gurpreet promoted)
	\end{enumerate}
	\item If A = {John promoted or Gurpreet promoted}, find P (A).
\end{enumerate}
\solution
%\input{exemplar/11/16/3/10/main.tex}
\item A card is drawn from a deck of 52 cards. Find the probability of getting a king or a heart or a red card.\\
\solution
%\input{exemplar/11/16/3/15/main.tex}
\item The probability that a student will pass his examination is 0.73, the probability of
the student getting a compartment is 0.13, and the probability that the student will
either pass or get compartment is 0.96. State True or False.\\
\solution
%\input{exemplar/11/16/3/31/main.tex}
\item A card is selected from a pack of 52 cards\\
\begin{enumerate}[label=(\alph*)]
\item How many points are there in the sample space?
\item Calculate the probability that the cards is an ace of spades.
\item Calculate the probability that the card is (i) an ace (ii)black card.\\
\end{enumerate}
%\input{ncert/11/16/3/4_1/Prob_4.tex}
\item In a non-leap year, the probability of having 53 tuesdays or 53 wednesdays is\\
\solution
%\input{exemplar/11/16/3/18/main.tex}
\item There are 1000 sealed envelopes in a box, 10 of them contain a cash prize of
Rs 100 each, 100 of them contain a cash prize of Rs 50 each and 200 of them
contain a cash prize of Rs 10 each and rest do not contain any cash prize. If they
are well shuffled and an envelope is picked up out, what is the probability that it
contains no cash prize?\\
\solution
%\input{exemplar/10/13/3/34/main.tex}
\item 
A die is thrown and a card is selected at random from a deck of 52 playing cards. The probability of getting an even number on the die and a spade card.\\
\solution
%\input{exemplar/12/13/3/78/main.tex}
\item
If 4-digit numbers greater than 5,000 are randomly formed from the digits 0, 1, 3, 5, and 7, what is the probability of forming a number divisible by 5 when:
\begin{enumerate}
    \item The digits are repeated?
    \item The repetition of digits is not allowed?
\end{enumerate}
\solution
%\input{ncert/11/16/4/9/main.tex}
\item Consider the probability space $\brak{\Omega, \mathcal{G}, P}$ where $\Omega = [0,2]$ and $\mathcal{G} = \cbrak{\phi, \Omega, [0,1], (1,2]}$. Let $X$ and $Y$ be two functions on $\Omega$ defined as
\begin{align*}
    X(\omega) = 
    \begin{cases}
        1 & \text{if }\omega \in [0, 1]\\
        2 & \text{if }\omega \in (1, 2]
    \end{cases}
\end{align*}
and
\begin{align*}
    Y(\omega) = 
    \begin{cases}
        2 & \text{if }\omega \in [0, 1.5]\\
        3 & \text{if }\omega \in (1.5, 2].
    \end{cases}
\end{align*}
Then which one of the following statements is true?
\begin{enumerate}
    \item [(A)] $X$ is a random variable with respect to $\mathcal{G}$, but $Y$ is not a random variable with respect to $\mathcal{G}$.
    \item [(B)] $Y$ is a random variable with respect to $\mathcal{G}$, but $X$ is not a random variable with respect to $\mathcal{G}$.
    \item [(C)] Neither $X$ nor $Y$ is a random variable with respect to $\mathcal{G}$.
    \item [(D)] Both $X$ and $Y$ are random variables with respect to $\mathcal{G}$.
\end{enumerate} \hfill (GATE ST 2023)\\
\solution
%\input{gate/ST/2023/14/main.tex}
	\item  A die is loaded in such a way that each odd number is twice as likely to occur as
each even number. Find $P(G)$, where $G$ is the event that a number greater than
3 occurs on a single roll of the die.
\\
\solution
		%\input{exemplar/11/16/3/5/main.tex}
	\item All the jacks, queens and kings are removed from a deck of 52 playing cards. The remaining cards are well shuffled and then one card is drawn at random. Giving ace a value 1 similar value for other cards, find the probability that the card has a value 
		\begin{enumerate}
			\item 7
			\item greater than 7
			\item less than 7
		\end{enumerate}
		%\input{exemplar/10/13/3/30/main.tex}
  \item A Lot consists of 48 mobile phones of which 42 are good, 3 have only minor defects and 3 have major defects.Varnika will buy a phone if it is good but the trader will only buy a mobile if it has no major defects. One phone is selected at random from the lot. What is the probability that it is
\begin{enumerate}
	\item acceptable to Varnika?
            \item acceptable to the trader?
\end{enumerate}
\solution
	%\input{exemplar/10/13/3/40/main.tex}
 \item A student says that if you throw a die, it will show up 1 or not 1. Therefore, the probability of getting 1 and the probability of getting 'not 1' each is equal to $\frac{1}{2}$. Is this correct? Give reasons.\\
 \solution
        %\input{exemplar/10/13/2/9/main.tex}
   \item Four candidates A, B, C, D have ap-
plied for the assignment to coach a school cricket
team. If A is twice as likely to be selected as B, and
B and C are given about the same chance of being
selected, while C is twice as likely to be selected
as D, what are the probabilities that
\begin{enumerate}
\item C will be selected?
\item A will not be selected?
\end{enumerate}
	%\input{exemplar/11/16/3/9/main.tex}
 \item A bag contain 24 balls of which $x$ balls are red, $2x$ are white and $3x$ are blue. A ball is selected at random, What is the probability that it is
\begin{enumerate}[label=\alph*)]
\item not red ?
\item white ?
\end{enumerate}
%\input{exemplar/10/13/3/41/main.tex}
If the letters of the word ASSASSINATION are arranged at random. Find the Probability that
\begin{enumerate}[label=(\alph*)]
\item Four $S's$ come consecutively in the word
\item Two  $I's$ and two $N's$ come together
\item All $A's$ are not coming together
\item No two $A's$ are coming together
\end{enumerate}
%\input{exemplar/11/16/3/14/main.tex}
	\item One urn contains two black balls (labelled B1 and B2) and one white ball. A
	second urn contains one black ball and two white balls (labelled W1 and W2).
	Suppose the following experiment is performed. One of the two urns is chosen
	at random. Next a ball is randomly chosen from the urn. Then a second ball is
	chosen at random from the same urn without replacing the first ball.
	
	\begin{enumerate}
	\item What is the probability that two black balls are chosen?
	
	\item What is the probability that two balls of opposite colour are chosen?
	\end{enumerate}
	\solution
	%\input{exemplar/11/16/3/12/main1.tex}
\end{enumerate}

	\item A bag contains 4 red and 4 black balls, another bag contains 2 red and 6 black balls. One of the two bags is selected at random and a ball is drawn from the bag which is found to be red. Find the probability that the ball is drawn from the first bag.
\\
\solution
		%\begin{table}[H]
	\centering
\begin{tabular}{|c|c|c|}
\hline
Random variable &Value &Definition\\ \hline
\multirow{3}{*}{X} &0 &Slips of Rs 1\\
&1 &Slips of Rs 5\\
&2 &Slips of Rs 13\\ \hline
\multirow{2}{*}{Y} &0 &Box A\\
&1 &Box B\\\hline
\end{tabular}
\caption{}
\label{tab:Distribution}
\end{table}
See \tabref{tab:Distribution}.
\begin{align}
p_{Y}\brak{k}= \begin{cases} 
      \frac{1}{3} & {k=0} \\
      \frac{2}{3 }& {k=1} 
   \end{cases}
   \\
p_{Y|X}\brak{0|0} = \frac{19}{25}\, 
p_{Y|X}\brak{0|1} = \frac{6}{25}\,
p_{Y|X}\brak{1|0} = \frac{45}{50}\,
p_{Y|X}\brak{1|2} = \frac{5}{50}
\end{align}
The desired probability is the probability that a slip drawn at random is marked other than Rs 1,
\begin{align}
&=1-p_X\brak{0}\\
&= p_X(1) + p_X(2)
\end{align}
Using Bayes theorem,
\begin{align}
&= p_Y\brak{0} \times \pr{Y=0 | X=1} + p_Y\brak{1} \times \pr{Y=1|X=2}\\
&=\frac{1}{3} \times \frac{6}{25} + \frac{2}{3} \times \frac{5}{50}\\
&=\frac{11}{75}
\end{align}

\newpage

%\tableofcontents

\bigskip

\renewcommand{\thefigure}{\theenumi}
\renewcommand{\thetable}{\theenumi}
%\renewcommand{\theequation}{\theenumi}

%\begin{abstract}
%%\boldmath
%In this letter, an algorithm for evaluating the exact analytical bit error rate  (BER)  for the piecewise linear (PL) combiner for  multiple relays is presented. Previous results were available only for upto three relays. The algorithm is unique in the sense that  the actual mathematical expressions, that are prohibitively large, need not be explicitly obtained. The diversity gain due to multiple relays is shown through plots of the analytical BER, well supported by simulations. 
%
%\end{abstract}
% IEEEtran.cls defaults to using nonbold math in the Abstract.
% This preserves the distinction between vectors and scalars. However,
% if the journal you are submitting to favors bold math in the abstract,
% then you can use LaTeX's standard command \boldmath at the very start
% of the abstract to achieve this. Many IEEE journals frown on math
% in the abstract anyway.

% Note that keywords are not normally used for peerreview papers.
%\begin{IEEEkeywords}
%Cooperative diversity, decode and forward, piecewise linear
%\end{IEEEkeywords}



% For peer review papers, you can put extra information on the cover
% page as needed:
% \ifCLASSOPTIONpeerreview
% \begin{center} \bfseries EDICS Category: 3-BBND \end{center}
% \fi
%
% For peerreview papers, this IEEEtran command inserts a page break and
% creates the second title. It will be ignored for other modes.
%\IEEEpeerreviewmaketitle




  \item
  Cards with numbers 2 to 101 are placed in a box. A card is selected at random.Find the probability that the card has
\begin{enumerate}[label=(\roman*)]
	\item an even number 
	\item a square number
\end{enumerate}
\solution
%\begin{table}[H]
	\centering
\begin{tabular}{|c|c|c|}
\hline
Random variable &Value &Definition\\ \hline
\multirow{3}{*}{X} &0 &Slips of Rs 1\\
&1 &Slips of Rs 5\\
&2 &Slips of Rs 13\\ \hline
\multirow{2}{*}{Y} &0 &Box A\\
&1 &Box B\\\hline
\end{tabular}
\caption{}
\label{tab:Distribution}
\end{table}
See \tabref{tab:Distribution}.
\begin{align}
p_{Y}\brak{k}= \begin{cases} 
      \frac{1}{3} & {k=0} \\
      \frac{2}{3 }& {k=1} 
   \end{cases}
   \\
p_{Y|X}\brak{0|0} = \frac{19}{25}\, 
p_{Y|X}\brak{0|1} = \frac{6}{25}\,
p_{Y|X}\brak{1|0} = \frac{45}{50}\,
p_{Y|X}\brak{1|2} = \frac{5}{50}
\end{align}
The desired probability is the probability that a slip drawn at random is marked other than Rs 1,
\begin{align}
&=1-p_X\brak{0}\\
&= p_X(1) + p_X(2)
\end{align}
Using Bayes theorem,
\begin{align}
&= p_Y\brak{0} \times \pr{Y=0 | X=1} + p_Y\brak{1} \times \pr{Y=1|X=2}\\
&=\frac{1}{3} \times \frac{6}{25} + \frac{2}{3} \times \frac{5}{50}\\
&=\frac{11}{75}
\end{align}

\newpage

%\tableofcontents

\bigskip

\renewcommand{\thefigure}{\theenumi}
\renewcommand{\thetable}{\theenumi}
%\renewcommand{\theequation}{\theenumi}

%\begin{abstract}
%%\boldmath
%In this letter, an algorithm for evaluating the exact analytical bit error rate  (BER)  for the piecewise linear (PL) combiner for  multiple relays is presented. Previous results were available only for upto three relays. The algorithm is unique in the sense that  the actual mathematical expressions, that are prohibitively large, need not be explicitly obtained. The diversity gain due to multiple relays is shown through plots of the analytical BER, well supported by simulations. 
%
%\end{abstract}
% IEEEtran.cls defaults to using nonbold math in the Abstract.
% This preserves the distinction between vectors and scalars. However,
% if the journal you are submitting to favors bold math in the abstract,
% then you can use LaTeX's standard command \boldmath at the very start
% of the abstract to achieve this. Many IEEE journals frown on math
% in the abstract anyway.

% Note that keywords are not normally used for peerreview papers.
%\begin{IEEEkeywords}
%Cooperative diversity, decode and forward, piecewise linear
%\end{IEEEkeywords}



% For peer review papers, you can put extra information on the cover
% page as needed:
% \ifCLASSOPTIONpeerreview
% \begin{center} \bfseries EDICS Category: 3-BBND \end{center}
% \fi
%
% For peerreview papers, this IEEEtran command inserts a page break and
% creates the second title. It will be ignored for other modes.
%\IEEEpeerreviewmaketitle




\item
The king, queen and jack of clubs are removed from a deck of 52 playing cards and then well shuffled. Now one card is drawn at random from the remaining cards.  Determine the probability that the card is
\begin{enumerate}[label=(\roman*)]
\item a club
\item 10 of hearts
\end{enumerate}
\solution
%\begin{table}[H]
	\centering
\begin{tabular}{|c|c|c|}
\hline
Random variable &Value &Definition\\ \hline
\multirow{3}{*}{X} &0 &Slips of Rs 1\\
&1 &Slips of Rs 5\\
&2 &Slips of Rs 13\\ \hline
\multirow{2}{*}{Y} &0 &Box A\\
&1 &Box B\\\hline
\end{tabular}
\caption{}
\label{tab:Distribution}
\end{table}
See \tabref{tab:Distribution}.
\begin{align}
p_{Y}\brak{k}= \begin{cases} 
      \frac{1}{3} & {k=0} \\
      \frac{2}{3 }& {k=1} 
   \end{cases}
   \\
p_{Y|X}\brak{0|0} = \frac{19}{25}\, 
p_{Y|X}\brak{0|1} = \frac{6}{25}\,
p_{Y|X}\brak{1|0} = \frac{45}{50}\,
p_{Y|X}\brak{1|2} = \frac{5}{50}
\end{align}
The desired probability is the probability that a slip drawn at random is marked other than Rs 1,
\begin{align}
&=1-p_X\brak{0}\\
&= p_X(1) + p_X(2)
\end{align}
Using Bayes theorem,
\begin{align}
&= p_Y\brak{0} \times \pr{Y=0 | X=1} + p_Y\brak{1} \times \pr{Y=1|X=2}\\
&=\frac{1}{3} \times \frac{6}{25} + \frac{2}{3} \times \frac{5}{50}\\
&=\frac{11}{75}
\end{align}

\newpage

%\tableofcontents

\bigskip

\renewcommand{\thefigure}{\theenumi}
\renewcommand{\thetable}{\theenumi}
%\renewcommand{\theequation}{\theenumi}

%\begin{abstract}
%%\boldmath
%In this letter, an algorithm for evaluating the exact analytical bit error rate  (BER)  for the piecewise linear (PL) combiner for  multiple relays is presented. Previous results were available only for upto three relays. The algorithm is unique in the sense that  the actual mathematical expressions, that are prohibitively large, need not be explicitly obtained. The diversity gain due to multiple relays is shown through plots of the analytical BER, well supported by simulations. 
%
%\end{abstract}
% IEEEtran.cls defaults to using nonbold math in the Abstract.
% This preserves the distinction between vectors and scalars. However,
% if the journal you are submitting to favors bold math in the abstract,
% then you can use LaTeX's standard command \boldmath at the very start
% of the abstract to achieve this. Many IEEE journals frown on math
% in the abstract anyway.

% Note that keywords are not normally used for peerreview papers.
%\begin{IEEEkeywords}
%Cooperative diversity, decode and forward, piecewise linear
%\end{IEEEkeywords}



% For peer review papers, you can put extra information on the cover
% page as needed:
% \ifCLASSOPTIONpeerreview
% \begin{center} \bfseries EDICS Category: 3-BBND \end{center}
% \fi
%
% For peerreview papers, this IEEEtran command inserts a page break and
% creates the second title. It will be ignored for other modes.
%\IEEEpeerreviewmaketitle




\item A team of medical students doing their internship have to assist during surgeries
at a city hospital. The probabilities of surgeries rated as very complex, complex,
routine, simple or very simple are respectively, 0.15, 0.20, 0.31, 0.26, .08. Find
the probabilities that a particular surgery will be rated
\begin{enumerate}
	\item complex or very complex;
	\item neither very complex nor very simple;
	\item routine or complex
	\item routine or simple
\end{enumerate}
\solution
%\begin{table}[H]
	\centering
\begin{tabular}{|c|c|c|}
\hline
Random variable &Value &Definition\\ \hline
\multirow{3}{*}{X} &0 &Slips of Rs 1\\
&1 &Slips of Rs 5\\
&2 &Slips of Rs 13\\ \hline
\multirow{2}{*}{Y} &0 &Box A\\
&1 &Box B\\\hline
\end{tabular}
\caption{}
\label{tab:Distribution}
\end{table}
See \tabref{tab:Distribution}.
\begin{align}
p_{Y}\brak{k}= \begin{cases} 
      \frac{1}{3} & {k=0} \\
      \frac{2}{3 }& {k=1} 
   \end{cases}
   \\
p_{Y|X}\brak{0|0} = \frac{19}{25}\, 
p_{Y|X}\brak{0|1} = \frac{6}{25}\,
p_{Y|X}\brak{1|0} = \frac{45}{50}\,
p_{Y|X}\brak{1|2} = \frac{5}{50}
\end{align}
The desired probability is the probability that a slip drawn at random is marked other than Rs 1,
\begin{align}
&=1-p_X\brak{0}\\
&= p_X(1) + p_X(2)
\end{align}
Using Bayes theorem,
\begin{align}
&= p_Y\brak{0} \times \pr{Y=0 | X=1} + p_Y\brak{1} \times \pr{Y=1|X=2}\\
&=\frac{1}{3} \times \frac{6}{25} + \frac{2}{3} \times \frac{5}{50}\\
&=\frac{11}{75}
\end{align}

\newpage

%\tableofcontents

\bigskip

\renewcommand{\thefigure}{\theenumi}
\renewcommand{\thetable}{\theenumi}
%\renewcommand{\theequation}{\theenumi}

%\begin{abstract}
%%\boldmath
%In this letter, an algorithm for evaluating the exact analytical bit error rate  (BER)  for the piecewise linear (PL) combiner for  multiple relays is presented. Previous results were available only for upto three relays. The algorithm is unique in the sense that  the actual mathematical expressions, that are prohibitively large, need not be explicitly obtained. The diversity gain due to multiple relays is shown through plots of the analytical BER, well supported by simulations. 
%
%\end{abstract}
% IEEEtran.cls defaults to using nonbold math in the Abstract.
% This preserves the distinction between vectors and scalars. However,
% if the journal you are submitting to favors bold math in the abstract,
% then you can use LaTeX's standard command \boldmath at the very start
% of the abstract to achieve this. Many IEEE journals frown on math
% in the abstract anyway.

% Note that keywords are not normally used for peerreview papers.
%\begin{IEEEkeywords}
%Cooperative diversity, decode and forward, piecewise linear
%\end{IEEEkeywords}



% For peer review papers, you can put extra information on the cover
% page as needed:
% \ifCLASSOPTIONpeerreview
% \begin{center} \bfseries EDICS Category: 3-BBND \end{center}
% \fi
%
% For peerreview papers, this IEEEtran command inserts a page break and
% creates the second title. It will be ignored for other modes.
%\IEEEpeerreviewmaketitle




\item A card is selected from a pack of 52 cards.
\begin{enumerate}[label=(\alph*)]
    \item How many points are there in the sample space?
    \item Calculate the probability that the card is an ace of spades.
    \item Calculate the probability that the card is (i) an ace and (ii) black card.
\end{enumerate}
\solution
%Let $X$ be an bernoulli rv defined as in \tabref{tab:exemplar/11/16/3/26}.  Then, 
\begin{equation}
    p =
        \frac{4}{11} 
\end{equation}
\begin{table}[H]
	\centering
	\input{exemplar/11/16/3/26/tables/Table2.tex}
	\caption{}
        \label{tab:exemplar/11/16/3/26}
\end{table}

\item The probability that a non leap year selected at random will contain 53 sundays.
\\
\solution
%\begin{table}[H]
	\centering
\begin{tabular}{|c|c|c|}
\hline
Random variable &Value &Definition\\ \hline
\multirow{3}{*}{X} &0 &Slips of Rs 1\\
&1 &Slips of Rs 5\\
&2 &Slips of Rs 13\\ \hline
\multirow{2}{*}{Y} &0 &Box A\\
&1 &Box B\\\hline
\end{tabular}
\caption{}
\label{tab:Distribution}
\end{table}
See \tabref{tab:Distribution}.
\begin{align}
p_{Y}\brak{k}= \begin{cases} 
      \frac{1}{3} & {k=0} \\
      \frac{2}{3 }& {k=1} 
   \end{cases}
   \\
p_{Y|X}\brak{0|0} = \frac{19}{25}\, 
p_{Y|X}\brak{0|1} = \frac{6}{25}\,
p_{Y|X}\brak{1|0} = \frac{45}{50}\,
p_{Y|X}\brak{1|2} = \frac{5}{50}
\end{align}
The desired probability is the probability that a slip drawn at random is marked other than Rs 1,
\begin{align}
&=1-p_X\brak{0}\\
&= p_X(1) + p_X(2)
\end{align}
Using Bayes theorem,
\begin{align}
&= p_Y\brak{0} \times \pr{Y=0 | X=1} + p_Y\brak{1} \times \pr{Y=1|X=2}\\
&=\frac{1}{3} \times \frac{6}{25} + \frac{2}{3} \times \frac{5}{50}\\
&=\frac{11}{75}
\end{align}

\newpage

%\tableofcontents

\bigskip

\renewcommand{\thefigure}{\theenumi}
\renewcommand{\thetable}{\theenumi}
%\renewcommand{\theequation}{\theenumi}

%\begin{abstract}
%%\boldmath
%In this letter, an algorithm for evaluating the exact analytical bit error rate  (BER)  for the piecewise linear (PL) combiner for  multiple relays is presented. Previous results were available only for upto three relays. The algorithm is unique in the sense that  the actual mathematical expressions, that are prohibitively large, need not be explicitly obtained. The diversity gain due to multiple relays is shown through plots of the analytical BER, well supported by simulations. 
%
%\end{abstract}
% IEEEtran.cls defaults to using nonbold math in the Abstract.
% This preserves the distinction between vectors and scalars. However,
% if the journal you are submitting to favors bold math in the abstract,
% then you can use LaTeX's standard command \boldmath at the very start
% of the abstract to achieve this. Many IEEE journals frown on math
% in the abstract anyway.

% Note that keywords are not normally used for peerreview papers.
%\begin{IEEEkeywords}
%Cooperative diversity, decode and forward, piecewise linear
%\end{IEEEkeywords}



% For peer review papers, you can put extra information on the cover
% page as needed:
% \ifCLASSOPTIONpeerreview
% \begin{center} \bfseries EDICS Category: 3-BBND \end{center}
% \fi
%
% For peerreview papers, this IEEEtran command inserts a page break and
% creates the second title. It will be ignored for other modes.
%\IEEEpeerreviewmaketitle




\item One of the four persons John, Rita, Aslam or Gurpreet will be promoted next
month. Consequently the sample space consists of four elementary outcomes
S = {John promoted, Rita promoted, Aslam promoted, Gurpreet promoted}
You are told that the chances of John’s promotion is same as that of Gurpreet,
Rita’s chances of promotion are twice as likely as Johns. Aslam’s chances are
four times that of John.
\begin{enumerate}
	\item Determine
	\begin{enumerate}
		\item P (John promoted)
		\item P (Rita promoted)
		\item P (Aslam promoted)
		\item P (Gurpreet promoted)
	\end{enumerate}
	\item If A = {John promoted or Gurpreet promoted}, find P (A).
\end{enumerate}
\solution
%\begin{table}[H]
	\centering
\begin{tabular}{|c|c|c|}
\hline
Random variable &Value &Definition\\ \hline
\multirow{3}{*}{X} &0 &Slips of Rs 1\\
&1 &Slips of Rs 5\\
&2 &Slips of Rs 13\\ \hline
\multirow{2}{*}{Y} &0 &Box A\\
&1 &Box B\\\hline
\end{tabular}
\caption{}
\label{tab:Distribution}
\end{table}
See \tabref{tab:Distribution}.
\begin{align}
p_{Y}\brak{k}= \begin{cases} 
      \frac{1}{3} & {k=0} \\
      \frac{2}{3 }& {k=1} 
   \end{cases}
   \\
p_{Y|X}\brak{0|0} = \frac{19}{25}\, 
p_{Y|X}\brak{0|1} = \frac{6}{25}\,
p_{Y|X}\brak{1|0} = \frac{45}{50}\,
p_{Y|X}\brak{1|2} = \frac{5}{50}
\end{align}
The desired probability is the probability that a slip drawn at random is marked other than Rs 1,
\begin{align}
&=1-p_X\brak{0}\\
&= p_X(1) + p_X(2)
\end{align}
Using Bayes theorem,
\begin{align}
&= p_Y\brak{0} \times \pr{Y=0 | X=1} + p_Y\brak{1} \times \pr{Y=1|X=2}\\
&=\frac{1}{3} \times \frac{6}{25} + \frac{2}{3} \times \frac{5}{50}\\
&=\frac{11}{75}
\end{align}

\newpage

%\tableofcontents

\bigskip

\renewcommand{\thefigure}{\theenumi}
\renewcommand{\thetable}{\theenumi}
%\renewcommand{\theequation}{\theenumi}

%\begin{abstract}
%%\boldmath
%In this letter, an algorithm for evaluating the exact analytical bit error rate  (BER)  for the piecewise linear (PL) combiner for  multiple relays is presented. Previous results were available only for upto three relays. The algorithm is unique in the sense that  the actual mathematical expressions, that are prohibitively large, need not be explicitly obtained. The diversity gain due to multiple relays is shown through plots of the analytical BER, well supported by simulations. 
%
%\end{abstract}
% IEEEtran.cls defaults to using nonbold math in the Abstract.
% This preserves the distinction between vectors and scalars. However,
% if the journal you are submitting to favors bold math in the abstract,
% then you can use LaTeX's standard command \boldmath at the very start
% of the abstract to achieve this. Many IEEE journals frown on math
% in the abstract anyway.

% Note that keywords are not normally used for peerreview papers.
%\begin{IEEEkeywords}
%Cooperative diversity, decode and forward, piecewise linear
%\end{IEEEkeywords}



% For peer review papers, you can put extra information on the cover
% page as needed:
% \ifCLASSOPTIONpeerreview
% \begin{center} \bfseries EDICS Category: 3-BBND \end{center}
% \fi
%
% For peerreview papers, this IEEEtran command inserts a page break and
% creates the second title. It will be ignored for other modes.
%\IEEEpeerreviewmaketitle




\item A card is drawn from a deck of 52 cards. Find the probability of getting a king or a heart or a red card.\\
\solution
%\begin{table}[H]
	\centering
\begin{tabular}{|c|c|c|}
\hline
Random variable &Value &Definition\\ \hline
\multirow{3}{*}{X} &0 &Slips of Rs 1\\
&1 &Slips of Rs 5\\
&2 &Slips of Rs 13\\ \hline
\multirow{2}{*}{Y} &0 &Box A\\
&1 &Box B\\\hline
\end{tabular}
\caption{}
\label{tab:Distribution}
\end{table}
See \tabref{tab:Distribution}.
\begin{align}
p_{Y}\brak{k}= \begin{cases} 
      \frac{1}{3} & {k=0} \\
      \frac{2}{3 }& {k=1} 
   \end{cases}
   \\
p_{Y|X}\brak{0|0} = \frac{19}{25}\, 
p_{Y|X}\brak{0|1} = \frac{6}{25}\,
p_{Y|X}\brak{1|0} = \frac{45}{50}\,
p_{Y|X}\brak{1|2} = \frac{5}{50}
\end{align}
The desired probability is the probability that a slip drawn at random is marked other than Rs 1,
\begin{align}
&=1-p_X\brak{0}\\
&= p_X(1) + p_X(2)
\end{align}
Using Bayes theorem,
\begin{align}
&= p_Y\brak{0} \times \pr{Y=0 | X=1} + p_Y\brak{1} \times \pr{Y=1|X=2}\\
&=\frac{1}{3} \times \frac{6}{25} + \frac{2}{3} \times \frac{5}{50}\\
&=\frac{11}{75}
\end{align}

\newpage

%\tableofcontents

\bigskip

\renewcommand{\thefigure}{\theenumi}
\renewcommand{\thetable}{\theenumi}
%\renewcommand{\theequation}{\theenumi}

%\begin{abstract}
%%\boldmath
%In this letter, an algorithm for evaluating the exact analytical bit error rate  (BER)  for the piecewise linear (PL) combiner for  multiple relays is presented. Previous results were available only for upto three relays. The algorithm is unique in the sense that  the actual mathematical expressions, that are prohibitively large, need not be explicitly obtained. The diversity gain due to multiple relays is shown through plots of the analytical BER, well supported by simulations. 
%
%\end{abstract}
% IEEEtran.cls defaults to using nonbold math in the Abstract.
% This preserves the distinction between vectors and scalars. However,
% if the journal you are submitting to favors bold math in the abstract,
% then you can use LaTeX's standard command \boldmath at the very start
% of the abstract to achieve this. Many IEEE journals frown on math
% in the abstract anyway.

% Note that keywords are not normally used for peerreview papers.
%\begin{IEEEkeywords}
%Cooperative diversity, decode and forward, piecewise linear
%\end{IEEEkeywords}



% For peer review papers, you can put extra information on the cover
% page as needed:
% \ifCLASSOPTIONpeerreview
% \begin{center} \bfseries EDICS Category: 3-BBND \end{center}
% \fi
%
% For peerreview papers, this IEEEtran command inserts a page break and
% creates the second title. It will be ignored for other modes.
%\IEEEpeerreviewmaketitle




\item The probability that a student will pass his examination is 0.73, the probability of
the student getting a compartment is 0.13, and the probability that the student will
either pass or get compartment is 0.96. State True or False.\\
\solution
%\begin{table}[H]
	\centering
\begin{tabular}{|c|c|c|}
\hline
Random variable &Value &Definition\\ \hline
\multirow{3}{*}{X} &0 &Slips of Rs 1\\
&1 &Slips of Rs 5\\
&2 &Slips of Rs 13\\ \hline
\multirow{2}{*}{Y} &0 &Box A\\
&1 &Box B\\\hline
\end{tabular}
\caption{}
\label{tab:Distribution}
\end{table}
See \tabref{tab:Distribution}.
\begin{align}
p_{Y}\brak{k}= \begin{cases} 
      \frac{1}{3} & {k=0} \\
      \frac{2}{3 }& {k=1} 
   \end{cases}
   \\
p_{Y|X}\brak{0|0} = \frac{19}{25}\, 
p_{Y|X}\brak{0|1} = \frac{6}{25}\,
p_{Y|X}\brak{1|0} = \frac{45}{50}\,
p_{Y|X}\brak{1|2} = \frac{5}{50}
\end{align}
The desired probability is the probability that a slip drawn at random is marked other than Rs 1,
\begin{align}
&=1-p_X\brak{0}\\
&= p_X(1) + p_X(2)
\end{align}
Using Bayes theorem,
\begin{align}
&= p_Y\brak{0} \times \pr{Y=0 | X=1} + p_Y\brak{1} \times \pr{Y=1|X=2}\\
&=\frac{1}{3} \times \frac{6}{25} + \frac{2}{3} \times \frac{5}{50}\\
&=\frac{11}{75}
\end{align}

\newpage

%\tableofcontents

\bigskip

\renewcommand{\thefigure}{\theenumi}
\renewcommand{\thetable}{\theenumi}
%\renewcommand{\theequation}{\theenumi}

%\begin{abstract}
%%\boldmath
%In this letter, an algorithm for evaluating the exact analytical bit error rate  (BER)  for the piecewise linear (PL) combiner for  multiple relays is presented. Previous results were available only for upto three relays. The algorithm is unique in the sense that  the actual mathematical expressions, that are prohibitively large, need not be explicitly obtained. The diversity gain due to multiple relays is shown through plots of the analytical BER, well supported by simulations. 
%
%\end{abstract}
% IEEEtran.cls defaults to using nonbold math in the Abstract.
% This preserves the distinction between vectors and scalars. However,
% if the journal you are submitting to favors bold math in the abstract,
% then you can use LaTeX's standard command \boldmath at the very start
% of the abstract to achieve this. Many IEEE journals frown on math
% in the abstract anyway.

% Note that keywords are not normally used for peerreview papers.
%\begin{IEEEkeywords}
%Cooperative diversity, decode and forward, piecewise linear
%\end{IEEEkeywords}



% For peer review papers, you can put extra information on the cover
% page as needed:
% \ifCLASSOPTIONpeerreview
% \begin{center} \bfseries EDICS Category: 3-BBND \end{center}
% \fi
%
% For peerreview papers, this IEEEtran command inserts a page break and
% creates the second title. It will be ignored for other modes.
%\IEEEpeerreviewmaketitle




\item A card is selected from a pack of 52 cards\\
\begin{enumerate}[label=(\alph*)]
\item How many points are there in the sample space?
\item Calculate the probability that the cards is an ace of spades.
\item Calculate the probability that the card is (i) an ace (ii)black card.\\
\end{enumerate}
%\input{ncert/11/16/3/4_1/Prob_4.tex}
\item In a non-leap year, the probability of having 53 tuesdays or 53 wednesdays is\\
\solution
%A non-leap year has a total of 365 days, and a week has 7 days.\\
So it can be expressed as 
\begin{align}
365\text{days} &=52\times 7+1 \text{day}
\end{align}
$\implies$ 52 tuesdays or wednesdays\\
Random variable X denotes the days of a week
\begin{align}
p_X\brak{k}&=\frac{1}{7}; \quad \brak{1<k<7}
\end{align}
So the probability of extra day being tuesday or wednesday is
\begin{align}
p_X\brak{3}+p_X\brak{4}&=\frac{1}{7}+\frac{1}{7}=\frac{2}{7}
\end{align}



\item There are 1000 sealed envelopes in a box, 10 of them contain a cash prize of
Rs 100 each, 100 of them contain a cash prize of Rs 50 each and 200 of them
contain a cash prize of Rs 10 each and rest do not contain any cash prize. If they
are well shuffled and an envelope is picked up out, what is the probability that it
contains no cash prize?\\
\solution
%\begin{table}[H]
	\centering
\begin{tabular}{|c|c|c|}
\hline
Random variable &Value &Definition\\ \hline
\multirow{3}{*}{X} &0 &Slips of Rs 1\\
&1 &Slips of Rs 5\\
&2 &Slips of Rs 13\\ \hline
\multirow{2}{*}{Y} &0 &Box A\\
&1 &Box B\\\hline
\end{tabular}
\caption{}
\label{tab:Distribution}
\end{table}
See \tabref{tab:Distribution}.
\begin{align}
p_{Y}\brak{k}= \begin{cases} 
      \frac{1}{3} & {k=0} \\
      \frac{2}{3 }& {k=1} 
   \end{cases}
   \\
p_{Y|X}\brak{0|0} = \frac{19}{25}\, 
p_{Y|X}\brak{0|1} = \frac{6}{25}\,
p_{Y|X}\brak{1|0} = \frac{45}{50}\,
p_{Y|X}\brak{1|2} = \frac{5}{50}
\end{align}
The desired probability is the probability that a slip drawn at random is marked other than Rs 1,
\begin{align}
&=1-p_X\brak{0}\\
&= p_X(1) + p_X(2)
\end{align}
Using Bayes theorem,
\begin{align}
&= p_Y\brak{0} \times \pr{Y=0 | X=1} + p_Y\brak{1} \times \pr{Y=1|X=2}\\
&=\frac{1}{3} \times \frac{6}{25} + \frac{2}{3} \times \frac{5}{50}\\
&=\frac{11}{75}
\end{align}

\newpage

%\tableofcontents

\bigskip

\renewcommand{\thefigure}{\theenumi}
\renewcommand{\thetable}{\theenumi}
%\renewcommand{\theequation}{\theenumi}

%\begin{abstract}
%%\boldmath
%In this letter, an algorithm for evaluating the exact analytical bit error rate  (BER)  for the piecewise linear (PL) combiner for  multiple relays is presented. Previous results were available only for upto three relays. The algorithm is unique in the sense that  the actual mathematical expressions, that are prohibitively large, need not be explicitly obtained. The diversity gain due to multiple relays is shown through plots of the analytical BER, well supported by simulations. 
%
%\end{abstract}
% IEEEtran.cls defaults to using nonbold math in the Abstract.
% This preserves the distinction between vectors and scalars. However,
% if the journal you are submitting to favors bold math in the abstract,
% then you can use LaTeX's standard command \boldmath at the very start
% of the abstract to achieve this. Many IEEE journals frown on math
% in the abstract anyway.

% Note that keywords are not normally used for peerreview papers.
%\begin{IEEEkeywords}
%Cooperative diversity, decode and forward, piecewise linear
%\end{IEEEkeywords}



% For peer review papers, you can put extra information on the cover
% page as needed:
% \ifCLASSOPTIONpeerreview
% \begin{center} \bfseries EDICS Category: 3-BBND \end{center}
% \fi
%
% For peerreview papers, this IEEEtran command inserts a page break and
% creates the second title. It will be ignored for other modes.
%\IEEEpeerreviewmaketitle




\item 
A die is thrown and a card is selected at random from a deck of 52 playing cards. The probability of getting an even number on the die and a spade card.\\
\solution
%\begin{table}[H]
	\centering
\begin{tabular}{|c|c|c|}
\hline
Random variable &Value &Definition\\ \hline
\multirow{3}{*}{X} &0 &Slips of Rs 1\\
&1 &Slips of Rs 5\\
&2 &Slips of Rs 13\\ \hline
\multirow{2}{*}{Y} &0 &Box A\\
&1 &Box B\\\hline
\end{tabular}
\caption{}
\label{tab:Distribution}
\end{table}
See \tabref{tab:Distribution}.
\begin{align}
p_{Y}\brak{k}= \begin{cases} 
      \frac{1}{3} & {k=0} \\
      \frac{2}{3 }& {k=1} 
   \end{cases}
   \\
p_{Y|X}\brak{0|0} = \frac{19}{25}\, 
p_{Y|X}\brak{0|1} = \frac{6}{25}\,
p_{Y|X}\brak{1|0} = \frac{45}{50}\,
p_{Y|X}\brak{1|2} = \frac{5}{50}
\end{align}
The desired probability is the probability that a slip drawn at random is marked other than Rs 1,
\begin{align}
&=1-p_X\brak{0}\\
&= p_X(1) + p_X(2)
\end{align}
Using Bayes theorem,
\begin{align}
&= p_Y\brak{0} \times \pr{Y=0 | X=1} + p_Y\brak{1} \times \pr{Y=1|X=2}\\
&=\frac{1}{3} \times \frac{6}{25} + \frac{2}{3} \times \frac{5}{50}\\
&=\frac{11}{75}
\end{align}

\newpage

%\tableofcontents

\bigskip

\renewcommand{\thefigure}{\theenumi}
\renewcommand{\thetable}{\theenumi}
%\renewcommand{\theequation}{\theenumi}

%\begin{abstract}
%%\boldmath
%In this letter, an algorithm for evaluating the exact analytical bit error rate  (BER)  for the piecewise linear (PL) combiner for  multiple relays is presented. Previous results were available only for upto three relays. The algorithm is unique in the sense that  the actual mathematical expressions, that are prohibitively large, need not be explicitly obtained. The diversity gain due to multiple relays is shown through plots of the analytical BER, well supported by simulations. 
%
%\end{abstract}
% IEEEtran.cls defaults to using nonbold math in the Abstract.
% This preserves the distinction between vectors and scalars. However,
% if the journal you are submitting to favors bold math in the abstract,
% then you can use LaTeX's standard command \boldmath at the very start
% of the abstract to achieve this. Many IEEE journals frown on math
% in the abstract anyway.

% Note that keywords are not normally used for peerreview papers.
%\begin{IEEEkeywords}
%Cooperative diversity, decode and forward, piecewise linear
%\end{IEEEkeywords}



% For peer review papers, you can put extra information on the cover
% page as needed:
% \ifCLASSOPTIONpeerreview
% \begin{center} \bfseries EDICS Category: 3-BBND \end{center}
% \fi
%
% For peerreview papers, this IEEEtran command inserts a page break and
% creates the second title. It will be ignored for other modes.
%\IEEEpeerreviewmaketitle




\item
If 4-digit numbers greater than 5,000 are randomly formed from the digits 0, 1, 3, 5, and 7, what is the probability of forming a number divisible by 5 when:
\begin{enumerate}
    \item The digits are repeated?
    \item The repetition of digits is not allowed?
\end{enumerate}
\solution
%\begin{table}[H]
	\centering
\begin{tabular}{|c|c|c|}
\hline
Random variable &Value &Definition\\ \hline
\multirow{3}{*}{X} &0 &Slips of Rs 1\\
&1 &Slips of Rs 5\\
&2 &Slips of Rs 13\\ \hline
\multirow{2}{*}{Y} &0 &Box A\\
&1 &Box B\\\hline
\end{tabular}
\caption{}
\label{tab:Distribution}
\end{table}
See \tabref{tab:Distribution}.
\begin{align}
p_{Y}\brak{k}= \begin{cases} 
      \frac{1}{3} & {k=0} \\
      \frac{2}{3 }& {k=1} 
   \end{cases}
   \\
p_{Y|X}\brak{0|0} = \frac{19}{25}\, 
p_{Y|X}\brak{0|1} = \frac{6}{25}\,
p_{Y|X}\brak{1|0} = \frac{45}{50}\,
p_{Y|X}\brak{1|2} = \frac{5}{50}
\end{align}
The desired probability is the probability that a slip drawn at random is marked other than Rs 1,
\begin{align}
&=1-p_X\brak{0}\\
&= p_X(1) + p_X(2)
\end{align}
Using Bayes theorem,
\begin{align}
&= p_Y\brak{0} \times \pr{Y=0 | X=1} + p_Y\brak{1} \times \pr{Y=1|X=2}\\
&=\frac{1}{3} \times \frac{6}{25} + \frac{2}{3} \times \frac{5}{50}\\
&=\frac{11}{75}
\end{align}

\newpage

%\tableofcontents

\bigskip

\renewcommand{\thefigure}{\theenumi}
\renewcommand{\thetable}{\theenumi}
%\renewcommand{\theequation}{\theenumi}

%\begin{abstract}
%%\boldmath
%In this letter, an algorithm for evaluating the exact analytical bit error rate  (BER)  for the piecewise linear (PL) combiner for  multiple relays is presented. Previous results were available only for upto three relays. The algorithm is unique in the sense that  the actual mathematical expressions, that are prohibitively large, need not be explicitly obtained. The diversity gain due to multiple relays is shown through plots of the analytical BER, well supported by simulations. 
%
%\end{abstract}
% IEEEtran.cls defaults to using nonbold math in the Abstract.
% This preserves the distinction between vectors and scalars. However,
% if the journal you are submitting to favors bold math in the abstract,
% then you can use LaTeX's standard command \boldmath at the very start
% of the abstract to achieve this. Many IEEE journals frown on math
% in the abstract anyway.

% Note that keywords are not normally used for peerreview papers.
%\begin{IEEEkeywords}
%Cooperative diversity, decode and forward, piecewise linear
%\end{IEEEkeywords}



% For peer review papers, you can put extra information on the cover
% page as needed:
% \ifCLASSOPTIONpeerreview
% \begin{center} \bfseries EDICS Category: 3-BBND \end{center}
% \fi
%
% For peerreview papers, this IEEEtran command inserts a page break and
% creates the second title. It will be ignored for other modes.
%\IEEEpeerreviewmaketitle




\item Consider the probability space $\brak{\Omega, \mathcal{G}, P}$ where $\Omega = [0,2]$ and $\mathcal{G} = \cbrak{\phi, \Omega, [0,1], (1,2]}$. Let $X$ and $Y$ be two functions on $\Omega$ defined as
\begin{align*}
    X(\omega) = 
    \begin{cases}
        1 & \text{if }\omega \in [0, 1]\\
        2 & \text{if }\omega \in (1, 2]
    \end{cases}
\end{align*}
and
\begin{align*}
    Y(\omega) = 
    \begin{cases}
        2 & \text{if }\omega \in [0, 1.5]\\
        3 & \text{if }\omega \in (1.5, 2].
    \end{cases}
\end{align*}
Then which one of the following statements is true?
\begin{enumerate}
    \item [(A)] $X$ is a random variable with respect to $\mathcal{G}$, but $Y$ is not a random variable with respect to $\mathcal{G}$.
    \item [(B)] $Y$ is a random variable with respect to $\mathcal{G}$, but $X$ is not a random variable with respect to $\mathcal{G}$.
    \item [(C)] Neither $X$ nor $Y$ is a random variable with respect to $\mathcal{G}$.
    \item [(D)] Both $X$ and $Y$ are random variables with respect to $\mathcal{G}$.
\end{enumerate} \hfill (GATE ST 2023)\\
\solution
%\begin{table}[H]
	\centering
\begin{tabular}{|c|c|c|}
\hline
Random variable &Value &Definition\\ \hline
\multirow{3}{*}{X} &0 &Slips of Rs 1\\
&1 &Slips of Rs 5\\
&2 &Slips of Rs 13\\ \hline
\multirow{2}{*}{Y} &0 &Box A\\
&1 &Box B\\\hline
\end{tabular}
\caption{}
\label{tab:Distribution}
\end{table}
See \tabref{tab:Distribution}.
\begin{align}
p_{Y}\brak{k}= \begin{cases} 
      \frac{1}{3} & {k=0} \\
      \frac{2}{3 }& {k=1} 
   \end{cases}
   \\
p_{Y|X}\brak{0|0} = \frac{19}{25}\, 
p_{Y|X}\brak{0|1} = \frac{6}{25}\,
p_{Y|X}\brak{1|0} = \frac{45}{50}\,
p_{Y|X}\brak{1|2} = \frac{5}{50}
\end{align}
The desired probability is the probability that a slip drawn at random is marked other than Rs 1,
\begin{align}
&=1-p_X\brak{0}\\
&= p_X(1) + p_X(2)
\end{align}
Using Bayes theorem,
\begin{align}
&= p_Y\brak{0} \times \pr{Y=0 | X=1} + p_Y\brak{1} \times \pr{Y=1|X=2}\\
&=\frac{1}{3} \times \frac{6}{25} + \frac{2}{3} \times \frac{5}{50}\\
&=\frac{11}{75}
\end{align}

\newpage

%\tableofcontents

\bigskip

\renewcommand{\thefigure}{\theenumi}
\renewcommand{\thetable}{\theenumi}
%\renewcommand{\theequation}{\theenumi}

%\begin{abstract}
%%\boldmath
%In this letter, an algorithm for evaluating the exact analytical bit error rate  (BER)  for the piecewise linear (PL) combiner for  multiple relays is presented. Previous results were available only for upto three relays. The algorithm is unique in the sense that  the actual mathematical expressions, that are prohibitively large, need not be explicitly obtained. The diversity gain due to multiple relays is shown through plots of the analytical BER, well supported by simulations. 
%
%\end{abstract}
% IEEEtran.cls defaults to using nonbold math in the Abstract.
% This preserves the distinction between vectors and scalars. However,
% if the journal you are submitting to favors bold math in the abstract,
% then you can use LaTeX's standard command \boldmath at the very start
% of the abstract to achieve this. Many IEEE journals frown on math
% in the abstract anyway.

% Note that keywords are not normally used for peerreview papers.
%\begin{IEEEkeywords}
%Cooperative diversity, decode and forward, piecewise linear
%\end{IEEEkeywords}



% For peer review papers, you can put extra information on the cover
% page as needed:
% \ifCLASSOPTIONpeerreview
% \begin{center} \bfseries EDICS Category: 3-BBND \end{center}
% \fi
%
% For peerreview papers, this IEEEtran command inserts a page break and
% creates the second title. It will be ignored for other modes.
%\IEEEpeerreviewmaketitle




	\item  A die is loaded in such a way that each odd number is twice as likely to occur as
each even number. Find $P(G)$, where $G$ is the event that a number greater than
3 occurs on a single roll of the die.
\\
\solution
		%\begin{table}[H]
	\centering
\begin{tabular}{|c|c|c|}
\hline
Random variable &Value &Definition\\ \hline
\multirow{3}{*}{X} &0 &Slips of Rs 1\\
&1 &Slips of Rs 5\\
&2 &Slips of Rs 13\\ \hline
\multirow{2}{*}{Y} &0 &Box A\\
&1 &Box B\\\hline
\end{tabular}
\caption{}
\label{tab:Distribution}
\end{table}
See \tabref{tab:Distribution}.
\begin{align}
p_{Y}\brak{k}= \begin{cases} 
      \frac{1}{3} & {k=0} \\
      \frac{2}{3 }& {k=1} 
   \end{cases}
   \\
p_{Y|X}\brak{0|0} = \frac{19}{25}\, 
p_{Y|X}\brak{0|1} = \frac{6}{25}\,
p_{Y|X}\brak{1|0} = \frac{45}{50}\,
p_{Y|X}\brak{1|2} = \frac{5}{50}
\end{align}
The desired probability is the probability that a slip drawn at random is marked other than Rs 1,
\begin{align}
&=1-p_X\brak{0}\\
&= p_X(1) + p_X(2)
\end{align}
Using Bayes theorem,
\begin{align}
&= p_Y\brak{0} \times \pr{Y=0 | X=1} + p_Y\brak{1} \times \pr{Y=1|X=2}\\
&=\frac{1}{3} \times \frac{6}{25} + \frac{2}{3} \times \frac{5}{50}\\
&=\frac{11}{75}
\end{align}

\newpage

%\tableofcontents

\bigskip

\renewcommand{\thefigure}{\theenumi}
\renewcommand{\thetable}{\theenumi}
%\renewcommand{\theequation}{\theenumi}

%\begin{abstract}
%%\boldmath
%In this letter, an algorithm for evaluating the exact analytical bit error rate  (BER)  for the piecewise linear (PL) combiner for  multiple relays is presented. Previous results were available only for upto three relays. The algorithm is unique in the sense that  the actual mathematical expressions, that are prohibitively large, need not be explicitly obtained. The diversity gain due to multiple relays is shown through plots of the analytical BER, well supported by simulations. 
%
%\end{abstract}
% IEEEtran.cls defaults to using nonbold math in the Abstract.
% This preserves the distinction between vectors and scalars. However,
% if the journal you are submitting to favors bold math in the abstract,
% then you can use LaTeX's standard command \boldmath at the very start
% of the abstract to achieve this. Many IEEE journals frown on math
% in the abstract anyway.

% Note that keywords are not normally used for peerreview papers.
%\begin{IEEEkeywords}
%Cooperative diversity, decode and forward, piecewise linear
%\end{IEEEkeywords}



% For peer review papers, you can put extra information on the cover
% page as needed:
% \ifCLASSOPTIONpeerreview
% \begin{center} \bfseries EDICS Category: 3-BBND \end{center}
% \fi
%
% For peerreview papers, this IEEEtran command inserts a page break and
% creates the second title. It will be ignored for other modes.
%\IEEEpeerreviewmaketitle




	\item All the jacks, queens and kings are removed from a deck of 52 playing cards. The remaining cards are well shuffled and then one card is drawn at random. Giving ace a value 1 similar value for other cards, find the probability that the card has a value 
		\begin{enumerate}
			\item 7
			\item greater than 7
			\item less than 7
		\end{enumerate}
		%Number of cards left after removing all jacks, queens and kings 
\begin{align}
N	= 52 - 4\times 3
	= 40
\end{align}
%\begin{table}[H]
%\def\arraystretch{1.2}
%\begin{tabular}{|c|c|c|}
%\hline
%	\textbf{Parameter} &\textbf{Value} &\textbf{Description}\\ \hline
%	$X$ &1-10 &Represents the value of the card picked \\ \hline
%\end{tabular}
%\end{table}
Let $1 \le X \le 10$ be the value of the card picked.  Then,
\begin{align}
	p_X(k) &= \Pr(X=k)\ \forall\ 1 \leq k \leq 10\\
	&= \frac{4\times 1}{40}\\
	&= \frac{1}{10}\\
	\therefore p_X(k) &= 
	\begin{cases}
		\frac{1}{10} & 1 \leq k \leq 10\\
		0 & \text{otherwise}
	\end{cases}
\end{align}
and
\begin{align}
	F_{X}(k) &= \sum_{m=0}^{k}p_{X}(m) \quad 1 \leq k \leq 10\\
	&= \frac{k}{10}\\
	\therefore F_{X}(k) &= 
	\begin{cases}
		0 & k \leq 0\\
		\frac{k}{10} & 1\leq k \leq 10\\
		1 & k > 10 
	\end{cases}
\end{align}
\begin{enumerate}
	\item Probability that card has value equal to 7 is
		\begin{align}
			 p_{X}(7)
			= \frac{1}{10}
		\end{align}
	\item Probability that card has value greater than 7 is
		\begin{align}
			1 - F_X(7)
			&= 1 - \frac{7}{10}
			\\
			&= \frac{3}{10}
		\end{align}
	\item Probability that card has value less than 7 is
		\begin{align}
			 F_{X}(6)
			=\frac{6}{10}
		\end{align}
\end{enumerate}

  \item A Lot consists of 48 mobile phones of which 42 are good, 3 have only minor defects and 3 have major defects.Varnika will buy a phone if it is good but the trader will only buy a mobile if it has no major defects. One phone is selected at random from the lot. What is the probability that it is
\begin{enumerate}
	\item acceptable to Varnika?
            \item acceptable to the trader?
\end{enumerate}
\solution
	%\begin{table}[H]
	\centering
\begin{tabular}{|c|c|c|}
\hline
Random variable &Value &Definition\\ \hline
\multirow{3}{*}{X} &0 &Slips of Rs 1\\
&1 &Slips of Rs 5\\
&2 &Slips of Rs 13\\ \hline
\multirow{2}{*}{Y} &0 &Box A\\
&1 &Box B\\\hline
\end{tabular}
\caption{}
\label{tab:Distribution}
\end{table}
See \tabref{tab:Distribution}.
\begin{align}
p_{Y}\brak{k}= \begin{cases} 
      \frac{1}{3} & {k=0} \\
      \frac{2}{3 }& {k=1} 
   \end{cases}
   \\
p_{Y|X}\brak{0|0} = \frac{19}{25}\, 
p_{Y|X}\brak{0|1} = \frac{6}{25}\,
p_{Y|X}\brak{1|0} = \frac{45}{50}\,
p_{Y|X}\brak{1|2} = \frac{5}{50}
\end{align}
The desired probability is the probability that a slip drawn at random is marked other than Rs 1,
\begin{align}
&=1-p_X\brak{0}\\
&= p_X(1) + p_X(2)
\end{align}
Using Bayes theorem,
\begin{align}
&= p_Y\brak{0} \times \pr{Y=0 | X=1} + p_Y\brak{1} \times \pr{Y=1|X=2}\\
&=\frac{1}{3} \times \frac{6}{25} + \frac{2}{3} \times \frac{5}{50}\\
&=\frac{11}{75}
\end{align}

\newpage

%\tableofcontents

\bigskip

\renewcommand{\thefigure}{\theenumi}
\renewcommand{\thetable}{\theenumi}
%\renewcommand{\theequation}{\theenumi}

%\begin{abstract}
%%\boldmath
%In this letter, an algorithm for evaluating the exact analytical bit error rate  (BER)  for the piecewise linear (PL) combiner for  multiple relays is presented. Previous results were available only for upto three relays. The algorithm is unique in the sense that  the actual mathematical expressions, that are prohibitively large, need not be explicitly obtained. The diversity gain due to multiple relays is shown through plots of the analytical BER, well supported by simulations. 
%
%\end{abstract}
% IEEEtran.cls defaults to using nonbold math in the Abstract.
% This preserves the distinction between vectors and scalars. However,
% if the journal you are submitting to favors bold math in the abstract,
% then you can use LaTeX's standard command \boldmath at the very start
% of the abstract to achieve this. Many IEEE journals frown on math
% in the abstract anyway.

% Note that keywords are not normally used for peerreview papers.
%\begin{IEEEkeywords}
%Cooperative diversity, decode and forward, piecewise linear
%\end{IEEEkeywords}



% For peer review papers, you can put extra information on the cover
% page as needed:
% \ifCLASSOPTIONpeerreview
% \begin{center} \bfseries EDICS Category: 3-BBND \end{center}
% \fi
%
% For peerreview papers, this IEEEtran command inserts a page break and
% creates the second title. It will be ignored for other modes.
%\IEEEpeerreviewmaketitle




 \item A student says that if you throw a die, it will show up 1 or not 1. Therefore, the probability of getting 1 and the probability of getting 'not 1' each is equal to $\frac{1}{2}$. Is this correct? Give reasons.\\
 \solution
        %\begin{table}[H]
	\centering
\begin{tabular}{|c|c|c|}
\hline
Random variable &Value &Definition\\ \hline
\multirow{3}{*}{X} &0 &Slips of Rs 1\\
&1 &Slips of Rs 5\\
&2 &Slips of Rs 13\\ \hline
\multirow{2}{*}{Y} &0 &Box A\\
&1 &Box B\\\hline
\end{tabular}
\caption{}
\label{tab:Distribution}
\end{table}
See \tabref{tab:Distribution}.
\begin{align}
p_{Y}\brak{k}= \begin{cases} 
      \frac{1}{3} & {k=0} \\
      \frac{2}{3 }& {k=1} 
   \end{cases}
   \\
p_{Y|X}\brak{0|0} = \frac{19}{25}\, 
p_{Y|X}\brak{0|1} = \frac{6}{25}\,
p_{Y|X}\brak{1|0} = \frac{45}{50}\,
p_{Y|X}\brak{1|2} = \frac{5}{50}
\end{align}
The desired probability is the probability that a slip drawn at random is marked other than Rs 1,
\begin{align}
&=1-p_X\brak{0}\\
&= p_X(1) + p_X(2)
\end{align}
Using Bayes theorem,
\begin{align}
&= p_Y\brak{0} \times \pr{Y=0 | X=1} + p_Y\brak{1} \times \pr{Y=1|X=2}\\
&=\frac{1}{3} \times \frac{6}{25} + \frac{2}{3} \times \frac{5}{50}\\
&=\frac{11}{75}
\end{align}

\newpage

%\tableofcontents

\bigskip

\renewcommand{\thefigure}{\theenumi}
\renewcommand{\thetable}{\theenumi}
%\renewcommand{\theequation}{\theenumi}

%\begin{abstract}
%%\boldmath
%In this letter, an algorithm for evaluating the exact analytical bit error rate  (BER)  for the piecewise linear (PL) combiner for  multiple relays is presented. Previous results were available only for upto three relays. The algorithm is unique in the sense that  the actual mathematical expressions, that are prohibitively large, need not be explicitly obtained. The diversity gain due to multiple relays is shown through plots of the analytical BER, well supported by simulations. 
%
%\end{abstract}
% IEEEtran.cls defaults to using nonbold math in the Abstract.
% This preserves the distinction between vectors and scalars. However,
% if the journal you are submitting to favors bold math in the abstract,
% then you can use LaTeX's standard command \boldmath at the very start
% of the abstract to achieve this. Many IEEE journals frown on math
% in the abstract anyway.

% Note that keywords are not normally used for peerreview papers.
%\begin{IEEEkeywords}
%Cooperative diversity, decode and forward, piecewise linear
%\end{IEEEkeywords}



% For peer review papers, you can put extra information on the cover
% page as needed:
% \ifCLASSOPTIONpeerreview
% \begin{center} \bfseries EDICS Category: 3-BBND \end{center}
% \fi
%
% For peerreview papers, this IEEEtran command inserts a page break and
% creates the second title. It will be ignored for other modes.
%\IEEEpeerreviewmaketitle




   \item Four candidates A, B, C, D have ap-
plied for the assignment to coach a school cricket
team. If A is twice as likely to be selected as B, and
B and C are given about the same chance of being
selected, while C is twice as likely to be selected
as D, what are the probabilities that
\begin{enumerate}
\item C will be selected?
\item A will not be selected?
\end{enumerate}
	%\begin{table}[H]
	\centering
\begin{tabular}{|c|c|c|}
\hline
Random variable &Value &Definition\\ \hline
\multirow{3}{*}{X} &0 &Slips of Rs 1\\
&1 &Slips of Rs 5\\
&2 &Slips of Rs 13\\ \hline
\multirow{2}{*}{Y} &0 &Box A\\
&1 &Box B\\\hline
\end{tabular}
\caption{}
\label{tab:Distribution}
\end{table}
See \tabref{tab:Distribution}.
\begin{align}
p_{Y}\brak{k}= \begin{cases} 
      \frac{1}{3} & {k=0} \\
      \frac{2}{3 }& {k=1} 
   \end{cases}
   \\
p_{Y|X}\brak{0|0} = \frac{19}{25}\, 
p_{Y|X}\brak{0|1} = \frac{6}{25}\,
p_{Y|X}\brak{1|0} = \frac{45}{50}\,
p_{Y|X}\brak{1|2} = \frac{5}{50}
\end{align}
The desired probability is the probability that a slip drawn at random is marked other than Rs 1,
\begin{align}
&=1-p_X\brak{0}\\
&= p_X(1) + p_X(2)
\end{align}
Using Bayes theorem,
\begin{align}
&= p_Y\brak{0} \times \pr{Y=0 | X=1} + p_Y\brak{1} \times \pr{Y=1|X=2}\\
&=\frac{1}{3} \times \frac{6}{25} + \frac{2}{3} \times \frac{5}{50}\\
&=\frac{11}{75}
\end{align}

\newpage

%\tableofcontents

\bigskip

\renewcommand{\thefigure}{\theenumi}
\renewcommand{\thetable}{\theenumi}
%\renewcommand{\theequation}{\theenumi}

%\begin{abstract}
%%\boldmath
%In this letter, an algorithm for evaluating the exact analytical bit error rate  (BER)  for the piecewise linear (PL) combiner for  multiple relays is presented. Previous results were available only for upto three relays. The algorithm is unique in the sense that  the actual mathematical expressions, that are prohibitively large, need not be explicitly obtained. The diversity gain due to multiple relays is shown through plots of the analytical BER, well supported by simulations. 
%
%\end{abstract}
% IEEEtran.cls defaults to using nonbold math in the Abstract.
% This preserves the distinction between vectors and scalars. However,
% if the journal you are submitting to favors bold math in the abstract,
% then you can use LaTeX's standard command \boldmath at the very start
% of the abstract to achieve this. Many IEEE journals frown on math
% in the abstract anyway.

% Note that keywords are not normally used for peerreview papers.
%\begin{IEEEkeywords}
%Cooperative diversity, decode and forward, piecewise linear
%\end{IEEEkeywords}



% For peer review papers, you can put extra information on the cover
% page as needed:
% \ifCLASSOPTIONpeerreview
% \begin{center} \bfseries EDICS Category: 3-BBND \end{center}
% \fi
%
% For peerreview papers, this IEEEtran command inserts a page break and
% creates the second title. It will be ignored for other modes.
%\IEEEpeerreviewmaketitle




 \item A bag contain 24 balls of which $x$ balls are red, $2x$ are white and $3x$ are blue. A ball is selected at random, What is the probability that it is
\begin{enumerate}[label=\alph*)]
\item not red ?
\item white ?
\end{enumerate}
%\begin{table}[H]
	\centering
\begin{tabular}{|c|c|c|}
\hline
Random variable &Value &Definition\\ \hline
\multirow{3}{*}{X} &0 &Slips of Rs 1\\
&1 &Slips of Rs 5\\
&2 &Slips of Rs 13\\ \hline
\multirow{2}{*}{Y} &0 &Box A\\
&1 &Box B\\\hline
\end{tabular}
\caption{}
\label{tab:Distribution}
\end{table}
See \tabref{tab:Distribution}.
\begin{align}
p_{Y}\brak{k}= \begin{cases} 
      \frac{1}{3} & {k=0} \\
      \frac{2}{3 }& {k=1} 
   \end{cases}
   \\
p_{Y|X}\brak{0|0} = \frac{19}{25}\, 
p_{Y|X}\brak{0|1} = \frac{6}{25}\,
p_{Y|X}\brak{1|0} = \frac{45}{50}\,
p_{Y|X}\brak{1|2} = \frac{5}{50}
\end{align}
The desired probability is the probability that a slip drawn at random is marked other than Rs 1,
\begin{align}
&=1-p_X\brak{0}\\
&= p_X(1) + p_X(2)
\end{align}
Using Bayes theorem,
\begin{align}
&= p_Y\brak{0} \times \pr{Y=0 | X=1} + p_Y\brak{1} \times \pr{Y=1|X=2}\\
&=\frac{1}{3} \times \frac{6}{25} + \frac{2}{3} \times \frac{5}{50}\\
&=\frac{11}{75}
\end{align}

\newpage

%\tableofcontents

\bigskip

\renewcommand{\thefigure}{\theenumi}
\renewcommand{\thetable}{\theenumi}
%\renewcommand{\theequation}{\theenumi}

%\begin{abstract}
%%\boldmath
%In this letter, an algorithm for evaluating the exact analytical bit error rate  (BER)  for the piecewise linear (PL) combiner for  multiple relays is presented. Previous results were available only for upto three relays. The algorithm is unique in the sense that  the actual mathematical expressions, that are prohibitively large, need not be explicitly obtained. The diversity gain due to multiple relays is shown through plots of the analytical BER, well supported by simulations. 
%
%\end{abstract}
% IEEEtran.cls defaults to using nonbold math in the Abstract.
% This preserves the distinction between vectors and scalars. However,
% if the journal you are submitting to favors bold math in the abstract,
% then you can use LaTeX's standard command \boldmath at the very start
% of the abstract to achieve this. Many IEEE journals frown on math
% in the abstract anyway.

% Note that keywords are not normally used for peerreview papers.
%\begin{IEEEkeywords}
%Cooperative diversity, decode and forward, piecewise linear
%\end{IEEEkeywords}



% For peer review papers, you can put extra information on the cover
% page as needed:
% \ifCLASSOPTIONpeerreview
% \begin{center} \bfseries EDICS Category: 3-BBND \end{center}
% \fi
%
% For peerreview papers, this IEEEtran command inserts a page break and
% creates the second title. It will be ignored for other modes.
%\IEEEpeerreviewmaketitle




If the letters of the word ASSASSINATION are arranged at random. Find the Probability that
\begin{enumerate}[label=(\alph*)]
\item Four $S's$ come consecutively in the word
\item Two  $I's$ and two $N's$ come together
\item All $A's$ are not coming together
\item No two $A's$ are coming together
\end{enumerate}
%\begin{table}[H]
	\centering
\begin{tabular}{|c|c|c|}
\hline
Random variable &Value &Definition\\ \hline
\multirow{3}{*}{X} &0 &Slips of Rs 1\\
&1 &Slips of Rs 5\\
&2 &Slips of Rs 13\\ \hline
\multirow{2}{*}{Y} &0 &Box A\\
&1 &Box B\\\hline
\end{tabular}
\caption{}
\label{tab:Distribution}
\end{table}
See \tabref{tab:Distribution}.
\begin{align}
p_{Y}\brak{k}= \begin{cases} 
      \frac{1}{3} & {k=0} \\
      \frac{2}{3 }& {k=1} 
   \end{cases}
   \\
p_{Y|X}\brak{0|0} = \frac{19}{25}\, 
p_{Y|X}\brak{0|1} = \frac{6}{25}\,
p_{Y|X}\brak{1|0} = \frac{45}{50}\,
p_{Y|X}\brak{1|2} = \frac{5}{50}
\end{align}
The desired probability is the probability that a slip drawn at random is marked other than Rs 1,
\begin{align}
&=1-p_X\brak{0}\\
&= p_X(1) + p_X(2)
\end{align}
Using Bayes theorem,
\begin{align}
&= p_Y\brak{0} \times \pr{Y=0 | X=1} + p_Y\brak{1} \times \pr{Y=1|X=2}\\
&=\frac{1}{3} \times \frac{6}{25} + \frac{2}{3} \times \frac{5}{50}\\
&=\frac{11}{75}
\end{align}

\newpage

%\tableofcontents

\bigskip

\renewcommand{\thefigure}{\theenumi}
\renewcommand{\thetable}{\theenumi}
%\renewcommand{\theequation}{\theenumi}

%\begin{abstract}
%%\boldmath
%In this letter, an algorithm for evaluating the exact analytical bit error rate  (BER)  for the piecewise linear (PL) combiner for  multiple relays is presented. Previous results were available only for upto three relays. The algorithm is unique in the sense that  the actual mathematical expressions, that are prohibitively large, need not be explicitly obtained. The diversity gain due to multiple relays is shown through plots of the analytical BER, well supported by simulations. 
%
%\end{abstract}
% IEEEtran.cls defaults to using nonbold math in the Abstract.
% This preserves the distinction between vectors and scalars. However,
% if the journal you are submitting to favors bold math in the abstract,
% then you can use LaTeX's standard command \boldmath at the very start
% of the abstract to achieve this. Many IEEE journals frown on math
% in the abstract anyway.

% Note that keywords are not normally used for peerreview papers.
%\begin{IEEEkeywords}
%Cooperative diversity, decode and forward, piecewise linear
%\end{IEEEkeywords}



% For peer review papers, you can put extra information on the cover
% page as needed:
% \ifCLASSOPTIONpeerreview
% \begin{center} \bfseries EDICS Category: 3-BBND \end{center}
% \fi
%
% For peerreview papers, this IEEEtran command inserts a page break and
% creates the second title. It will be ignored for other modes.
%\IEEEpeerreviewmaketitle




	\item One urn contains two black balls (labelled B1 and B2) and one white ball. A
	second urn contains one black ball and two white balls (labelled W1 and W2).
	Suppose the following experiment is performed. One of the two urns is chosen
	at random. Next a ball is randomly chosen from the urn. Then a second ball is
	chosen at random from the same urn without replacing the first ball.
	
	\begin{enumerate}
	\item What is the probability that two black balls are chosen?
	
	\item What is the probability that two balls of opposite colour are chosen?
	\end{enumerate}
	\solution
	%\begin{align}
    \label{eq:12.13.6.18.1}
	\because	\pr{A|B} &> \pr{A},\
\frac{\pr{AB}}{\pr{B}} > \pr{A}
\\
    \label{eq:12.13.6.18.2}
	\implies \pr{AB} &> \pr{A}\pr{B}
	\\
	\text{or, } \frac{\pr{AB}}{\pr{A}} &=\pr{B|A} > \pr{A}
\end{align}

\end{enumerate}

		%
\item 
Two cards are drawn at random and without replacement from a pack of 52 playing cards. Find the probability that both the cards are black.
\\
\solution
		%\begin{enumerate}[label=\thesection.\arabic*,ref=\thesection.\theenumi]
	\item One card is drawn from a well-shuffled deck of 52 cards. Find the probability of getting
\begin{enumerate}
\item A king of red colour 
\item A face card 
\item A red face card
\item The jack of hearts
\item A spade
\item The queen of diamonds

\end{enumerate}
\solution
		%\begin{table}[H]
	\centering
\begin{tabular}{|c|c|c|}
\hline
Random variable &Value &Definition\\ \hline
\multirow{3}{*}{X} &0 &Slips of Rs 1\\
&1 &Slips of Rs 5\\
&2 &Slips of Rs 13\\ \hline
\multirow{2}{*}{Y} &0 &Box A\\
&1 &Box B\\\hline
\end{tabular}
\caption{}
\label{tab:Distribution}
\end{table}
See \tabref{tab:Distribution}.
\begin{align}
p_{Y}\brak{k}= \begin{cases} 
      \frac{1}{3} & {k=0} \\
      \frac{2}{3 }& {k=1} 
   \end{cases}
   \\
p_{Y|X}\brak{0|0} = \frac{19}{25}\, 
p_{Y|X}\brak{0|1} = \frac{6}{25}\,
p_{Y|X}\brak{1|0} = \frac{45}{50}\,
p_{Y|X}\brak{1|2} = \frac{5}{50}
\end{align}
The desired probability is the probability that a slip drawn at random is marked other than Rs 1,
\begin{align}
&=1-p_X\brak{0}\\
&= p_X(1) + p_X(2)
\end{align}
Using Bayes theorem,
\begin{align}
&= p_Y\brak{0} \times \pr{Y=0 | X=1} + p_Y\brak{1} \times \pr{Y=1|X=2}\\
&=\frac{1}{3} \times \frac{6}{25} + \frac{2}{3} \times \frac{5}{50}\\
&=\frac{11}{75}
\end{align}

\newpage

%\tableofcontents

\bigskip

\renewcommand{\thefigure}{\theenumi}
\renewcommand{\thetable}{\theenumi}
%\renewcommand{\theequation}{\theenumi}

%\begin{abstract}
%%\boldmath
%In this letter, an algorithm for evaluating the exact analytical bit error rate  (BER)  for the piecewise linear (PL) combiner for  multiple relays is presented. Previous results were available only for upto three relays. The algorithm is unique in the sense that  the actual mathematical expressions, that are prohibitively large, need not be explicitly obtained. The diversity gain due to multiple relays is shown through plots of the analytical BER, well supported by simulations. 
%
%\end{abstract}
% IEEEtran.cls defaults to using nonbold math in the Abstract.
% This preserves the distinction between vectors and scalars. However,
% if the journal you are submitting to favors bold math in the abstract,
% then you can use LaTeX's standard command \boldmath at the very start
% of the abstract to achieve this. Many IEEE journals frown on math
% in the abstract anyway.

% Note that keywords are not normally used for peerreview papers.
%\begin{IEEEkeywords}
%Cooperative diversity, decode and forward, piecewise linear
%\end{IEEEkeywords}



% For peer review papers, you can put extra information on the cover
% page as needed:
% \ifCLASSOPTIONpeerreview
% \begin{center} \bfseries EDICS Category: 3-BBND \end{center}
% \fi
%
% For peerreview papers, this IEEEtran command inserts a page break and
% creates the second title. It will be ignored for other modes.
%\IEEEpeerreviewmaketitle




	\item Five cards—the ten, jack, queen, king and ace of diamonds, are well-shuffled with their face downwards. One card is then picked up at random.
\begin{enumerate}
\item
What is the probability that the card is the queen? 
\item
If the queen is drawn and put aside, what is the probability that the second card picked up is (a) an ace? (b) a queen?\\
\end{enumerate}
\solution
		%\begin{enumerate}[label=\thesection.\arabic*,ref=\thesection.\theenumi]
	\item One card is drawn from a well-shuffled deck of 52 cards. Find the probability of getting
\begin{enumerate}
\item A king of red colour 
\item A face card 
\item A red face card
\item The jack of hearts
\item A spade
\item The queen of diamonds

\end{enumerate}
\solution
		%\input{ncert/10/15/1/14/main.tex}
	\item Five cards—the ten, jack, queen, king and ace of diamonds, are well-shuffled with their face downwards. One card is then picked up at random.
\begin{enumerate}
\item
What is the probability that the card is the queen? 
\item
If the queen is drawn and put aside, what is the probability that the second card picked up is (a) an ace? (b) a queen?\\
\end{enumerate}
\solution
		%\input{ncert/10/15/1/15/defs.tex}
	\item A bag contains $5$ red balls and some blue balls. If the probability of drawing a blue ball is double that if a red ball, determine the number of blue balls in the bag. 
		\\
\solution
		%\input{ncert/10/15/2/3/defs.tex}
	\item A card is selected from a pack of 52 cards.
 \begin{enumerate}[label=(\alph*)] 
                 \item How many points are there in the sample space?
                 \item Calculate the probability that the card is an ace of spades.
                 \item Calculate the probability that the card is (i) an ace and (ii) black card.
 \end{enumerate}
\solution
		%\input{ncert/11/16/3/4/main.tex}
\item Four cards are drawn from a well-shuffled deck of 52 cards. What is the probability of obtaining 3 diamonds and one spade.
\\
\solution
		%\input{ncert/11/16/4/2/defs.tex}
\item In a certain lottery 10,000 tickets are sold and ten equal prizes are awarded. What is the probability of not getting a prize if you buy (a) one ticket (b) two tickets (c) 10 tickets ?	
\\
\solution
		%\input{ncert/11/16/4/4/defs.tex}
		%
\item 
Out of 100 students, two sections of 40 and 60 are formed. If you and your friend are among the 100 students, what is the probability that
\begin{enumerate}
\item you both enter the same section?
\item you both enter the different sections?
\end{enumerate}
\solution
		%\input{ncert/11/16/4/5/defs.tex}
	\item 
The number lock of a suitcase has 4 wheels each labelled with ten digits i.e. from 0 to 9.The lock opens with a sequence of four digits with no repeats.What is the probability of a person getting the right sequence to open the suitcase.
\\
\solution
		%\input{ncert/11/16/4/10/defs.tex}
		%
\item 
Two cards are drawn at random and without replacement from a pack of 52 playing cards. Find the probability that both the cards are black.
\\
\solution
		%\input{ncert/12/13/2/2/defs.tex}
		\item A box of oranges is inspected by examining three randomly selected oranges drawn without replacement. If all the three oranges are good, the box is approved for sale, otherwise, it is rejected. Find the probability that a box containing 15 oranges out of which 12 are good and 3 are bad ones will be approved for sale.
		\label{ncert/12/13/2/3/defs.tex}
		\item Two balls are drawn at random with replacement from a box containing 10 black and 8 red balls. Find the probability that
		\label{ncert/12/13/2/12}
\begin{enumerate}
\item both balls are red.
\item first ball is black and second is red.
\item one of them is black and other is red.
\end{enumerate}

\item In a hostel, 60\% of the students read Hindi newspaper, 40\% read English newspaper and 20\% read both Hindi and English newspapers. A student is selected at random.
		\label{ncert/12/13/2/15}
\begin{enumerate}
\item Find the probability that she reads neither Hindi nor English newspapers.
\item If she reads Hindi newspaper, find the probability that she reads English newspaper.
\item If she reads English newspaper, find the probability that she reads Hindi newspaper.\\
\end{enumerate}
\item The probability of obtaining an even prime number on each die, when a pair of dice is rolled is 
\begin{enumerate}
    \item $0$ 
    
    \item $\frac{1}{3}$ 
    
    \item $\frac{1}{12}$ 
    
    \item $\frac{1}{36}$ 
\end{enumerate}
\solution
		%\input{ncert/12/13/2/17/defs.tex}
	\item A bag contains 4 red and 4 black balls, another bag contains 2 red and 6 black balls. One of the two bags is selected at random and a ball is drawn from the bag which is found to be red. Find the probability that the ball is drawn from the first bag.
\\
\solution
		%\input{ncert/12/13/3/2/main.tex}
  \item
  Cards with numbers 2 to 101 are placed in a box. A card is selected at random.Find the probability that the card has
\begin{enumerate}[label=(\roman*)]
	\item an even number 
	\item a square number
\end{enumerate}
\solution
%\input{exemplar/10/13/3/32/main.tex}
\item
The king, queen and jack of clubs are removed from a deck of 52 playing cards and then well shuffled. Now one card is drawn at random from the remaining cards.  Determine the probability that the card is
\begin{enumerate}[label=(\roman*)]
\item a club
\item 10 of hearts
\end{enumerate}
\solution
%\input{exemplar/10/13/3/29/main.tex}
\item A team of medical students doing their internship have to assist during surgeries
at a city hospital. The probabilities of surgeries rated as very complex, complex,
routine, simple or very simple are respectively, 0.15, 0.20, 0.31, 0.26, .08. Find
the probabilities that a particular surgery will be rated
\begin{enumerate}
	\item complex or very complex;
	\item neither very complex nor very simple;
	\item routine or complex
	\item routine or simple
\end{enumerate}
\solution
%\input{exemplar/11/16/3/8(1)/main.tex}
\item A card is selected from a pack of 52 cards.
\begin{enumerate}[label=(\alph*)]
    \item How many points are there in the sample space?
    \item Calculate the probability that the card is an ace of spades.
    \item Calculate the probability that the card is (i) an ace and (ii) black card.
\end{enumerate}
\solution
%\input{exemplar/11/16/3/4/main2.tex}
\item The probability that a non leap year selected at random will contain 53 sundays.
\\
\solution
%\input{exemplar/10/13/1/19/main.tex}
\item One of the four persons John, Rita, Aslam or Gurpreet will be promoted next
month. Consequently the sample space consists of four elementary outcomes
S = {John promoted, Rita promoted, Aslam promoted, Gurpreet promoted}
You are told that the chances of John’s promotion is same as that of Gurpreet,
Rita’s chances of promotion are twice as likely as Johns. Aslam’s chances are
four times that of John.
\begin{enumerate}
	\item Determine
	\begin{enumerate}
		\item P (John promoted)
		\item P (Rita promoted)
		\item P (Aslam promoted)
		\item P (Gurpreet promoted)
	\end{enumerate}
	\item If A = {John promoted or Gurpreet promoted}, find P (A).
\end{enumerate}
\solution
%\input{exemplar/11/16/3/10/main.tex}
\item A card is drawn from a deck of 52 cards. Find the probability of getting a king or a heart or a red card.\\
\solution
%\input{exemplar/11/16/3/15/main.tex}
\item The probability that a student will pass his examination is 0.73, the probability of
the student getting a compartment is 0.13, and the probability that the student will
either pass or get compartment is 0.96. State True or False.\\
\solution
%\input{exemplar/11/16/3/31/main.tex}
\item A card is selected from a pack of 52 cards\\
\begin{enumerate}[label=(\alph*)]
\item How many points are there in the sample space?
\item Calculate the probability that the cards is an ace of spades.
\item Calculate the probability that the card is (i) an ace (ii)black card.\\
\end{enumerate}
%\input{ncert/11/16/3/4_1/Prob_4.tex}
\item In a non-leap year, the probability of having 53 tuesdays or 53 wednesdays is\\
\solution
%\input{exemplar/11/16/3/18/main.tex}
\item There are 1000 sealed envelopes in a box, 10 of them contain a cash prize of
Rs 100 each, 100 of them contain a cash prize of Rs 50 each and 200 of them
contain a cash prize of Rs 10 each and rest do not contain any cash prize. If they
are well shuffled and an envelope is picked up out, what is the probability that it
contains no cash prize?\\
\solution
%\input{exemplar/10/13/3/34/main.tex}
\item 
A die is thrown and a card is selected at random from a deck of 52 playing cards. The probability of getting an even number on the die and a spade card.\\
\solution
%\input{exemplar/12/13/3/78/main.tex}
\item
If 4-digit numbers greater than 5,000 are randomly formed from the digits 0, 1, 3, 5, and 7, what is the probability of forming a number divisible by 5 when:
\begin{enumerate}
    \item The digits are repeated?
    \item The repetition of digits is not allowed?
\end{enumerate}
\solution
%\input{ncert/11/16/4/9/main.tex}
\item Consider the probability space $\brak{\Omega, \mathcal{G}, P}$ where $\Omega = [0,2]$ and $\mathcal{G} = \cbrak{\phi, \Omega, [0,1], (1,2]}$. Let $X$ and $Y$ be two functions on $\Omega$ defined as
\begin{align*}
    X(\omega) = 
    \begin{cases}
        1 & \text{if }\omega \in [0, 1]\\
        2 & \text{if }\omega \in (1, 2]
    \end{cases}
\end{align*}
and
\begin{align*}
    Y(\omega) = 
    \begin{cases}
        2 & \text{if }\omega \in [0, 1.5]\\
        3 & \text{if }\omega \in (1.5, 2].
    \end{cases}
\end{align*}
Then which one of the following statements is true?
\begin{enumerate}
    \item [(A)] $X$ is a random variable with respect to $\mathcal{G}$, but $Y$ is not a random variable with respect to $\mathcal{G}$.
    \item [(B)] $Y$ is a random variable with respect to $\mathcal{G}$, but $X$ is not a random variable with respect to $\mathcal{G}$.
    \item [(C)] Neither $X$ nor $Y$ is a random variable with respect to $\mathcal{G}$.
    \item [(D)] Both $X$ and $Y$ are random variables with respect to $\mathcal{G}$.
\end{enumerate} \hfill (GATE ST 2023)\\
\solution
%\input{gate/ST/2023/14/main.tex}
	\item  A die is loaded in such a way that each odd number is twice as likely to occur as
each even number. Find $P(G)$, where $G$ is the event that a number greater than
3 occurs on a single roll of the die.
\\
\solution
		%\input{exemplar/11/16/3/5/main.tex}
	\item All the jacks, queens and kings are removed from a deck of 52 playing cards. The remaining cards are well shuffled and then one card is drawn at random. Giving ace a value 1 similar value for other cards, find the probability that the card has a value 
		\begin{enumerate}
			\item 7
			\item greater than 7
			\item less than 7
		\end{enumerate}
		%\input{exemplar/10/13/3/30/main.tex}
  \item A Lot consists of 48 mobile phones of which 42 are good, 3 have only minor defects and 3 have major defects.Varnika will buy a phone if it is good but the trader will only buy a mobile if it has no major defects. One phone is selected at random from the lot. What is the probability that it is
\begin{enumerate}
	\item acceptable to Varnika?
            \item acceptable to the trader?
\end{enumerate}
\solution
	%\input{exemplar/10/13/3/40/main.tex}
 \item A student says that if you throw a die, it will show up 1 or not 1. Therefore, the probability of getting 1 and the probability of getting 'not 1' each is equal to $\frac{1}{2}$. Is this correct? Give reasons.\\
 \solution
        %\input{exemplar/10/13/2/9/main.tex}
   \item Four candidates A, B, C, D have ap-
plied for the assignment to coach a school cricket
team. If A is twice as likely to be selected as B, and
B and C are given about the same chance of being
selected, while C is twice as likely to be selected
as D, what are the probabilities that
\begin{enumerate}
\item C will be selected?
\item A will not be selected?
\end{enumerate}
	%\input{exemplar/11/16/3/9/main.tex}
 \item A bag contain 24 balls of which $x$ balls are red, $2x$ are white and $3x$ are blue. A ball is selected at random, What is the probability that it is
\begin{enumerate}[label=\alph*)]
\item not red ?
\item white ?
\end{enumerate}
%\input{exemplar/10/13/3/41/main.tex}
If the letters of the word ASSASSINATION are arranged at random. Find the Probability that
\begin{enumerate}[label=(\alph*)]
\item Four $S's$ come consecutively in the word
\item Two  $I's$ and two $N's$ come together
\item All $A's$ are not coming together
\item No two $A's$ are coming together
\end{enumerate}
%\input{exemplar/11/16/3/14/main.tex}
	\item One urn contains two black balls (labelled B1 and B2) and one white ball. A
	second urn contains one black ball and two white balls (labelled W1 and W2).
	Suppose the following experiment is performed. One of the two urns is chosen
	at random. Next a ball is randomly chosen from the urn. Then a second ball is
	chosen at random from the same urn without replacing the first ball.
	
	\begin{enumerate}
	\item What is the probability that two black balls are chosen?
	
	\item What is the probability that two balls of opposite colour are chosen?
	\end{enumerate}
	\solution
	%\input{exemplar/11/16/3/12/main1.tex}
\end{enumerate}

	\item A bag contains $5$ red balls and some blue balls. If the probability of drawing a blue ball is double that if a red ball, determine the number of blue balls in the bag. 
		\\
\solution
		%\begin{enumerate}[label=\thesection.\arabic*,ref=\thesection.\theenumi]
	\item One card is drawn from a well-shuffled deck of 52 cards. Find the probability of getting
\begin{enumerate}
\item A king of red colour 
\item A face card 
\item A red face card
\item The jack of hearts
\item A spade
\item The queen of diamonds

\end{enumerate}
\solution
		%\input{ncert/10/15/1/14/main.tex}
	\item Five cards—the ten, jack, queen, king and ace of diamonds, are well-shuffled with their face downwards. One card is then picked up at random.
\begin{enumerate}
\item
What is the probability that the card is the queen? 
\item
If the queen is drawn and put aside, what is the probability that the second card picked up is (a) an ace? (b) a queen?\\
\end{enumerate}
\solution
		%\input{ncert/10/15/1/15/defs.tex}
	\item A bag contains $5$ red balls and some blue balls. If the probability of drawing a blue ball is double that if a red ball, determine the number of blue balls in the bag. 
		\\
\solution
		%\input{ncert/10/15/2/3/defs.tex}
	\item A card is selected from a pack of 52 cards.
 \begin{enumerate}[label=(\alph*)] 
                 \item How many points are there in the sample space?
                 \item Calculate the probability that the card is an ace of spades.
                 \item Calculate the probability that the card is (i) an ace and (ii) black card.
 \end{enumerate}
\solution
		%\input{ncert/11/16/3/4/main.tex}
\item Four cards are drawn from a well-shuffled deck of 52 cards. What is the probability of obtaining 3 diamonds and one spade.
\\
\solution
		%\input{ncert/11/16/4/2/defs.tex}
\item In a certain lottery 10,000 tickets are sold and ten equal prizes are awarded. What is the probability of not getting a prize if you buy (a) one ticket (b) two tickets (c) 10 tickets ?	
\\
\solution
		%\input{ncert/11/16/4/4/defs.tex}
		%
\item 
Out of 100 students, two sections of 40 and 60 are formed. If you and your friend are among the 100 students, what is the probability that
\begin{enumerate}
\item you both enter the same section?
\item you both enter the different sections?
\end{enumerate}
\solution
		%\input{ncert/11/16/4/5/defs.tex}
	\item 
The number lock of a suitcase has 4 wheels each labelled with ten digits i.e. from 0 to 9.The lock opens with a sequence of four digits with no repeats.What is the probability of a person getting the right sequence to open the suitcase.
\\
\solution
		%\input{ncert/11/16/4/10/defs.tex}
		%
\item 
Two cards are drawn at random and without replacement from a pack of 52 playing cards. Find the probability that both the cards are black.
\\
\solution
		%\input{ncert/12/13/2/2/defs.tex}
		\item A box of oranges is inspected by examining three randomly selected oranges drawn without replacement. If all the three oranges are good, the box is approved for sale, otherwise, it is rejected. Find the probability that a box containing 15 oranges out of which 12 are good and 3 are bad ones will be approved for sale.
		\label{ncert/12/13/2/3/defs.tex}
		\item Two balls are drawn at random with replacement from a box containing 10 black and 8 red balls. Find the probability that
		\label{ncert/12/13/2/12}
\begin{enumerate}
\item both balls are red.
\item first ball is black and second is red.
\item one of them is black and other is red.
\end{enumerate}

\item In a hostel, 60\% of the students read Hindi newspaper, 40\% read English newspaper and 20\% read both Hindi and English newspapers. A student is selected at random.
		\label{ncert/12/13/2/15}
\begin{enumerate}
\item Find the probability that she reads neither Hindi nor English newspapers.
\item If she reads Hindi newspaper, find the probability that she reads English newspaper.
\item If she reads English newspaper, find the probability that she reads Hindi newspaper.\\
\end{enumerate}
\item The probability of obtaining an even prime number on each die, when a pair of dice is rolled is 
\begin{enumerate}
    \item $0$ 
    
    \item $\frac{1}{3}$ 
    
    \item $\frac{1}{12}$ 
    
    \item $\frac{1}{36}$ 
\end{enumerate}
\solution
		%\input{ncert/12/13/2/17/defs.tex}
	\item A bag contains 4 red and 4 black balls, another bag contains 2 red and 6 black balls. One of the two bags is selected at random and a ball is drawn from the bag which is found to be red. Find the probability that the ball is drawn from the first bag.
\\
\solution
		%\input{ncert/12/13/3/2/main.tex}
  \item
  Cards with numbers 2 to 101 are placed in a box. A card is selected at random.Find the probability that the card has
\begin{enumerate}[label=(\roman*)]
	\item an even number 
	\item a square number
\end{enumerate}
\solution
%\input{exemplar/10/13/3/32/main.tex}
\item
The king, queen and jack of clubs are removed from a deck of 52 playing cards and then well shuffled. Now one card is drawn at random from the remaining cards.  Determine the probability that the card is
\begin{enumerate}[label=(\roman*)]
\item a club
\item 10 of hearts
\end{enumerate}
\solution
%\input{exemplar/10/13/3/29/main.tex}
\item A team of medical students doing their internship have to assist during surgeries
at a city hospital. The probabilities of surgeries rated as very complex, complex,
routine, simple or very simple are respectively, 0.15, 0.20, 0.31, 0.26, .08. Find
the probabilities that a particular surgery will be rated
\begin{enumerate}
	\item complex or very complex;
	\item neither very complex nor very simple;
	\item routine or complex
	\item routine or simple
\end{enumerate}
\solution
%\input{exemplar/11/16/3/8(1)/main.tex}
\item A card is selected from a pack of 52 cards.
\begin{enumerate}[label=(\alph*)]
    \item How many points are there in the sample space?
    \item Calculate the probability that the card is an ace of spades.
    \item Calculate the probability that the card is (i) an ace and (ii) black card.
\end{enumerate}
\solution
%\input{exemplar/11/16/3/4/main2.tex}
\item The probability that a non leap year selected at random will contain 53 sundays.
\\
\solution
%\input{exemplar/10/13/1/19/main.tex}
\item One of the four persons John, Rita, Aslam or Gurpreet will be promoted next
month. Consequently the sample space consists of four elementary outcomes
S = {John promoted, Rita promoted, Aslam promoted, Gurpreet promoted}
You are told that the chances of John’s promotion is same as that of Gurpreet,
Rita’s chances of promotion are twice as likely as Johns. Aslam’s chances are
four times that of John.
\begin{enumerate}
	\item Determine
	\begin{enumerate}
		\item P (John promoted)
		\item P (Rita promoted)
		\item P (Aslam promoted)
		\item P (Gurpreet promoted)
	\end{enumerate}
	\item If A = {John promoted or Gurpreet promoted}, find P (A).
\end{enumerate}
\solution
%\input{exemplar/11/16/3/10/main.tex}
\item A card is drawn from a deck of 52 cards. Find the probability of getting a king or a heart or a red card.\\
\solution
%\input{exemplar/11/16/3/15/main.tex}
\item The probability that a student will pass his examination is 0.73, the probability of
the student getting a compartment is 0.13, and the probability that the student will
either pass or get compartment is 0.96. State True or False.\\
\solution
%\input{exemplar/11/16/3/31/main.tex}
\item A card is selected from a pack of 52 cards\\
\begin{enumerate}[label=(\alph*)]
\item How many points are there in the sample space?
\item Calculate the probability that the cards is an ace of spades.
\item Calculate the probability that the card is (i) an ace (ii)black card.\\
\end{enumerate}
%\input{ncert/11/16/3/4_1/Prob_4.tex}
\item In a non-leap year, the probability of having 53 tuesdays or 53 wednesdays is\\
\solution
%\input{exemplar/11/16/3/18/main.tex}
\item There are 1000 sealed envelopes in a box, 10 of them contain a cash prize of
Rs 100 each, 100 of them contain a cash prize of Rs 50 each and 200 of them
contain a cash prize of Rs 10 each and rest do not contain any cash prize. If they
are well shuffled and an envelope is picked up out, what is the probability that it
contains no cash prize?\\
\solution
%\input{exemplar/10/13/3/34/main.tex}
\item 
A die is thrown and a card is selected at random from a deck of 52 playing cards. The probability of getting an even number on the die and a spade card.\\
\solution
%\input{exemplar/12/13/3/78/main.tex}
\item
If 4-digit numbers greater than 5,000 are randomly formed from the digits 0, 1, 3, 5, and 7, what is the probability of forming a number divisible by 5 when:
\begin{enumerate}
    \item The digits are repeated?
    \item The repetition of digits is not allowed?
\end{enumerate}
\solution
%\input{ncert/11/16/4/9/main.tex}
\item Consider the probability space $\brak{\Omega, \mathcal{G}, P}$ where $\Omega = [0,2]$ and $\mathcal{G} = \cbrak{\phi, \Omega, [0,1], (1,2]}$. Let $X$ and $Y$ be two functions on $\Omega$ defined as
\begin{align*}
    X(\omega) = 
    \begin{cases}
        1 & \text{if }\omega \in [0, 1]\\
        2 & \text{if }\omega \in (1, 2]
    \end{cases}
\end{align*}
and
\begin{align*}
    Y(\omega) = 
    \begin{cases}
        2 & \text{if }\omega \in [0, 1.5]\\
        3 & \text{if }\omega \in (1.5, 2].
    \end{cases}
\end{align*}
Then which one of the following statements is true?
\begin{enumerate}
    \item [(A)] $X$ is a random variable with respect to $\mathcal{G}$, but $Y$ is not a random variable with respect to $\mathcal{G}$.
    \item [(B)] $Y$ is a random variable with respect to $\mathcal{G}$, but $X$ is not a random variable with respect to $\mathcal{G}$.
    \item [(C)] Neither $X$ nor $Y$ is a random variable with respect to $\mathcal{G}$.
    \item [(D)] Both $X$ and $Y$ are random variables with respect to $\mathcal{G}$.
\end{enumerate} \hfill (GATE ST 2023)\\
\solution
%\input{gate/ST/2023/14/main.tex}
	\item  A die is loaded in such a way that each odd number is twice as likely to occur as
each even number. Find $P(G)$, where $G$ is the event that a number greater than
3 occurs on a single roll of the die.
\\
\solution
		%\input{exemplar/11/16/3/5/main.tex}
	\item All the jacks, queens and kings are removed from a deck of 52 playing cards. The remaining cards are well shuffled and then one card is drawn at random. Giving ace a value 1 similar value for other cards, find the probability that the card has a value 
		\begin{enumerate}
			\item 7
			\item greater than 7
			\item less than 7
		\end{enumerate}
		%\input{exemplar/10/13/3/30/main.tex}
  \item A Lot consists of 48 mobile phones of which 42 are good, 3 have only minor defects and 3 have major defects.Varnika will buy a phone if it is good but the trader will only buy a mobile if it has no major defects. One phone is selected at random from the lot. What is the probability that it is
\begin{enumerate}
	\item acceptable to Varnika?
            \item acceptable to the trader?
\end{enumerate}
\solution
	%\input{exemplar/10/13/3/40/main.tex}
 \item A student says that if you throw a die, it will show up 1 or not 1. Therefore, the probability of getting 1 and the probability of getting 'not 1' each is equal to $\frac{1}{2}$. Is this correct? Give reasons.\\
 \solution
        %\input{exemplar/10/13/2/9/main.tex}
   \item Four candidates A, B, C, D have ap-
plied for the assignment to coach a school cricket
team. If A is twice as likely to be selected as B, and
B and C are given about the same chance of being
selected, while C is twice as likely to be selected
as D, what are the probabilities that
\begin{enumerate}
\item C will be selected?
\item A will not be selected?
\end{enumerate}
	%\input{exemplar/11/16/3/9/main.tex}
 \item A bag contain 24 balls of which $x$ balls are red, $2x$ are white and $3x$ are blue. A ball is selected at random, What is the probability that it is
\begin{enumerate}[label=\alph*)]
\item not red ?
\item white ?
\end{enumerate}
%\input{exemplar/10/13/3/41/main.tex}
If the letters of the word ASSASSINATION are arranged at random. Find the Probability that
\begin{enumerate}[label=(\alph*)]
\item Four $S's$ come consecutively in the word
\item Two  $I's$ and two $N's$ come together
\item All $A's$ are not coming together
\item No two $A's$ are coming together
\end{enumerate}
%\input{exemplar/11/16/3/14/main.tex}
	\item One urn contains two black balls (labelled B1 and B2) and one white ball. A
	second urn contains one black ball and two white balls (labelled W1 and W2).
	Suppose the following experiment is performed. One of the two urns is chosen
	at random. Next a ball is randomly chosen from the urn. Then a second ball is
	chosen at random from the same urn without replacing the first ball.
	
	\begin{enumerate}
	\item What is the probability that two black balls are chosen?
	
	\item What is the probability that two balls of opposite colour are chosen?
	\end{enumerate}
	\solution
	%\input{exemplar/11/16/3/12/main1.tex}
\end{enumerate}

	\item A card is selected from a pack of 52 cards.
 \begin{enumerate}[label=(\alph*)] 
                 \item How many points are there in the sample space?
                 \item Calculate the probability that the card is an ace of spades.
                 \item Calculate the probability that the card is (i) an ace and (ii) black card.
 \end{enumerate}
\solution
		%\begin{table}[H]
	\centering
\begin{tabular}{|c|c|c|}
\hline
Random variable &Value &Definition\\ \hline
\multirow{3}{*}{X} &0 &Slips of Rs 1\\
&1 &Slips of Rs 5\\
&2 &Slips of Rs 13\\ \hline
\multirow{2}{*}{Y} &0 &Box A\\
&1 &Box B\\\hline
\end{tabular}
\caption{}
\label{tab:Distribution}
\end{table}
See \tabref{tab:Distribution}.
\begin{align}
p_{Y}\brak{k}= \begin{cases} 
      \frac{1}{3} & {k=0} \\
      \frac{2}{3 }& {k=1} 
   \end{cases}
   \\
p_{Y|X}\brak{0|0} = \frac{19}{25}\, 
p_{Y|X}\brak{0|1} = \frac{6}{25}\,
p_{Y|X}\brak{1|0} = \frac{45}{50}\,
p_{Y|X}\brak{1|2} = \frac{5}{50}
\end{align}
The desired probability is the probability that a slip drawn at random is marked other than Rs 1,
\begin{align}
&=1-p_X\brak{0}\\
&= p_X(1) + p_X(2)
\end{align}
Using Bayes theorem,
\begin{align}
&= p_Y\brak{0} \times \pr{Y=0 | X=1} + p_Y\brak{1} \times \pr{Y=1|X=2}\\
&=\frac{1}{3} \times \frac{6}{25} + \frac{2}{3} \times \frac{5}{50}\\
&=\frac{11}{75}
\end{align}

\newpage

%\tableofcontents

\bigskip

\renewcommand{\thefigure}{\theenumi}
\renewcommand{\thetable}{\theenumi}
%\renewcommand{\theequation}{\theenumi}

%\begin{abstract}
%%\boldmath
%In this letter, an algorithm for evaluating the exact analytical bit error rate  (BER)  for the piecewise linear (PL) combiner for  multiple relays is presented. Previous results were available only for upto three relays. The algorithm is unique in the sense that  the actual mathematical expressions, that are prohibitively large, need not be explicitly obtained. The diversity gain due to multiple relays is shown through plots of the analytical BER, well supported by simulations. 
%
%\end{abstract}
% IEEEtran.cls defaults to using nonbold math in the Abstract.
% This preserves the distinction between vectors and scalars. However,
% if the journal you are submitting to favors bold math in the abstract,
% then you can use LaTeX's standard command \boldmath at the very start
% of the abstract to achieve this. Many IEEE journals frown on math
% in the abstract anyway.

% Note that keywords are not normally used for peerreview papers.
%\begin{IEEEkeywords}
%Cooperative diversity, decode and forward, piecewise linear
%\end{IEEEkeywords}



% For peer review papers, you can put extra information on the cover
% page as needed:
% \ifCLASSOPTIONpeerreview
% \begin{center} \bfseries EDICS Category: 3-BBND \end{center}
% \fi
%
% For peerreview papers, this IEEEtran command inserts a page break and
% creates the second title. It will be ignored for other modes.
%\IEEEpeerreviewmaketitle




\item Four cards are drawn from a well-shuffled deck of 52 cards. What is the probability of obtaining 3 diamonds and one spade.
\\
\solution
		%\begin{enumerate}[label=\thesection.\arabic*,ref=\thesection.\theenumi]
	\item One card is drawn from a well-shuffled deck of 52 cards. Find the probability of getting
\begin{enumerate}
\item A king of red colour 
\item A face card 
\item A red face card
\item The jack of hearts
\item A spade
\item The queen of diamonds

\end{enumerate}
\solution
		%\input{ncert/10/15/1/14/main.tex}
	\item Five cards—the ten, jack, queen, king and ace of diamonds, are well-shuffled with their face downwards. One card is then picked up at random.
\begin{enumerate}
\item
What is the probability that the card is the queen? 
\item
If the queen is drawn and put aside, what is the probability that the second card picked up is (a) an ace? (b) a queen?\\
\end{enumerate}
\solution
		%\input{ncert/10/15/1/15/defs.tex}
	\item A bag contains $5$ red balls and some blue balls. If the probability of drawing a blue ball is double that if a red ball, determine the number of blue balls in the bag. 
		\\
\solution
		%\input{ncert/10/15/2/3/defs.tex}
	\item A card is selected from a pack of 52 cards.
 \begin{enumerate}[label=(\alph*)] 
                 \item How many points are there in the sample space?
                 \item Calculate the probability that the card is an ace of spades.
                 \item Calculate the probability that the card is (i) an ace and (ii) black card.
 \end{enumerate}
\solution
		%\input{ncert/11/16/3/4/main.tex}
\item Four cards are drawn from a well-shuffled deck of 52 cards. What is the probability of obtaining 3 diamonds and one spade.
\\
\solution
		%\input{ncert/11/16/4/2/defs.tex}
\item In a certain lottery 10,000 tickets are sold and ten equal prizes are awarded. What is the probability of not getting a prize if you buy (a) one ticket (b) two tickets (c) 10 tickets ?	
\\
\solution
		%\input{ncert/11/16/4/4/defs.tex}
		%
\item 
Out of 100 students, two sections of 40 and 60 are formed. If you and your friend are among the 100 students, what is the probability that
\begin{enumerate}
\item you both enter the same section?
\item you both enter the different sections?
\end{enumerate}
\solution
		%\input{ncert/11/16/4/5/defs.tex}
	\item 
The number lock of a suitcase has 4 wheels each labelled with ten digits i.e. from 0 to 9.The lock opens with a sequence of four digits with no repeats.What is the probability of a person getting the right sequence to open the suitcase.
\\
\solution
		%\input{ncert/11/16/4/10/defs.tex}
		%
\item 
Two cards are drawn at random and without replacement from a pack of 52 playing cards. Find the probability that both the cards are black.
\\
\solution
		%\input{ncert/12/13/2/2/defs.tex}
		\item A box of oranges is inspected by examining three randomly selected oranges drawn without replacement. If all the three oranges are good, the box is approved for sale, otherwise, it is rejected. Find the probability that a box containing 15 oranges out of which 12 are good and 3 are bad ones will be approved for sale.
		\label{ncert/12/13/2/3/defs.tex}
		\item Two balls are drawn at random with replacement from a box containing 10 black and 8 red balls. Find the probability that
		\label{ncert/12/13/2/12}
\begin{enumerate}
\item both balls are red.
\item first ball is black and second is red.
\item one of them is black and other is red.
\end{enumerate}

\item In a hostel, 60\% of the students read Hindi newspaper, 40\% read English newspaper and 20\% read both Hindi and English newspapers. A student is selected at random.
		\label{ncert/12/13/2/15}
\begin{enumerate}
\item Find the probability that she reads neither Hindi nor English newspapers.
\item If she reads Hindi newspaper, find the probability that she reads English newspaper.
\item If she reads English newspaper, find the probability that she reads Hindi newspaper.\\
\end{enumerate}
\item The probability of obtaining an even prime number on each die, when a pair of dice is rolled is 
\begin{enumerate}
    \item $0$ 
    
    \item $\frac{1}{3}$ 
    
    \item $\frac{1}{12}$ 
    
    \item $\frac{1}{36}$ 
\end{enumerate}
\solution
		%\input{ncert/12/13/2/17/defs.tex}
	\item A bag contains 4 red and 4 black balls, another bag contains 2 red and 6 black balls. One of the two bags is selected at random and a ball is drawn from the bag which is found to be red. Find the probability that the ball is drawn from the first bag.
\\
\solution
		%\input{ncert/12/13/3/2/main.tex}
  \item
  Cards with numbers 2 to 101 are placed in a box. A card is selected at random.Find the probability that the card has
\begin{enumerate}[label=(\roman*)]
	\item an even number 
	\item a square number
\end{enumerate}
\solution
%\input{exemplar/10/13/3/32/main.tex}
\item
The king, queen and jack of clubs are removed from a deck of 52 playing cards and then well shuffled. Now one card is drawn at random from the remaining cards.  Determine the probability that the card is
\begin{enumerate}[label=(\roman*)]
\item a club
\item 10 of hearts
\end{enumerate}
\solution
%\input{exemplar/10/13/3/29/main.tex}
\item A team of medical students doing their internship have to assist during surgeries
at a city hospital. The probabilities of surgeries rated as very complex, complex,
routine, simple or very simple are respectively, 0.15, 0.20, 0.31, 0.26, .08. Find
the probabilities that a particular surgery will be rated
\begin{enumerate}
	\item complex or very complex;
	\item neither very complex nor very simple;
	\item routine or complex
	\item routine or simple
\end{enumerate}
\solution
%\input{exemplar/11/16/3/8(1)/main.tex}
\item A card is selected from a pack of 52 cards.
\begin{enumerate}[label=(\alph*)]
    \item How many points are there in the sample space?
    \item Calculate the probability that the card is an ace of spades.
    \item Calculate the probability that the card is (i) an ace and (ii) black card.
\end{enumerate}
\solution
%\input{exemplar/11/16/3/4/main2.tex}
\item The probability that a non leap year selected at random will contain 53 sundays.
\\
\solution
%\input{exemplar/10/13/1/19/main.tex}
\item One of the four persons John, Rita, Aslam or Gurpreet will be promoted next
month. Consequently the sample space consists of four elementary outcomes
S = {John promoted, Rita promoted, Aslam promoted, Gurpreet promoted}
You are told that the chances of John’s promotion is same as that of Gurpreet,
Rita’s chances of promotion are twice as likely as Johns. Aslam’s chances are
four times that of John.
\begin{enumerate}
	\item Determine
	\begin{enumerate}
		\item P (John promoted)
		\item P (Rita promoted)
		\item P (Aslam promoted)
		\item P (Gurpreet promoted)
	\end{enumerate}
	\item If A = {John promoted or Gurpreet promoted}, find P (A).
\end{enumerate}
\solution
%\input{exemplar/11/16/3/10/main.tex}
\item A card is drawn from a deck of 52 cards. Find the probability of getting a king or a heart or a red card.\\
\solution
%\input{exemplar/11/16/3/15/main.tex}
\item The probability that a student will pass his examination is 0.73, the probability of
the student getting a compartment is 0.13, and the probability that the student will
either pass or get compartment is 0.96. State True or False.\\
\solution
%\input{exemplar/11/16/3/31/main.tex}
\item A card is selected from a pack of 52 cards\\
\begin{enumerate}[label=(\alph*)]
\item How many points are there in the sample space?
\item Calculate the probability that the cards is an ace of spades.
\item Calculate the probability that the card is (i) an ace (ii)black card.\\
\end{enumerate}
%\input{ncert/11/16/3/4_1/Prob_4.tex}
\item In a non-leap year, the probability of having 53 tuesdays or 53 wednesdays is\\
\solution
%\input{exemplar/11/16/3/18/main.tex}
\item There are 1000 sealed envelopes in a box, 10 of them contain a cash prize of
Rs 100 each, 100 of them contain a cash prize of Rs 50 each and 200 of them
contain a cash prize of Rs 10 each and rest do not contain any cash prize. If they
are well shuffled and an envelope is picked up out, what is the probability that it
contains no cash prize?\\
\solution
%\input{exemplar/10/13/3/34/main.tex}
\item 
A die is thrown and a card is selected at random from a deck of 52 playing cards. The probability of getting an even number on the die and a spade card.\\
\solution
%\input{exemplar/12/13/3/78/main.tex}
\item
If 4-digit numbers greater than 5,000 are randomly formed from the digits 0, 1, 3, 5, and 7, what is the probability of forming a number divisible by 5 when:
\begin{enumerate}
    \item The digits are repeated?
    \item The repetition of digits is not allowed?
\end{enumerate}
\solution
%\input{ncert/11/16/4/9/main.tex}
\item Consider the probability space $\brak{\Omega, \mathcal{G}, P}$ where $\Omega = [0,2]$ and $\mathcal{G} = \cbrak{\phi, \Omega, [0,1], (1,2]}$. Let $X$ and $Y$ be two functions on $\Omega$ defined as
\begin{align*}
    X(\omega) = 
    \begin{cases}
        1 & \text{if }\omega \in [0, 1]\\
        2 & \text{if }\omega \in (1, 2]
    \end{cases}
\end{align*}
and
\begin{align*}
    Y(\omega) = 
    \begin{cases}
        2 & \text{if }\omega \in [0, 1.5]\\
        3 & \text{if }\omega \in (1.5, 2].
    \end{cases}
\end{align*}
Then which one of the following statements is true?
\begin{enumerate}
    \item [(A)] $X$ is a random variable with respect to $\mathcal{G}$, but $Y$ is not a random variable with respect to $\mathcal{G}$.
    \item [(B)] $Y$ is a random variable with respect to $\mathcal{G}$, but $X$ is not a random variable with respect to $\mathcal{G}$.
    \item [(C)] Neither $X$ nor $Y$ is a random variable with respect to $\mathcal{G}$.
    \item [(D)] Both $X$ and $Y$ are random variables with respect to $\mathcal{G}$.
\end{enumerate} \hfill (GATE ST 2023)\\
\solution
%\input{gate/ST/2023/14/main.tex}
	\item  A die is loaded in such a way that each odd number is twice as likely to occur as
each even number. Find $P(G)$, where $G$ is the event that a number greater than
3 occurs on a single roll of the die.
\\
\solution
		%\input{exemplar/11/16/3/5/main.tex}
	\item All the jacks, queens and kings are removed from a deck of 52 playing cards. The remaining cards are well shuffled and then one card is drawn at random. Giving ace a value 1 similar value for other cards, find the probability that the card has a value 
		\begin{enumerate}
			\item 7
			\item greater than 7
			\item less than 7
		\end{enumerate}
		%\input{exemplar/10/13/3/30/main.tex}
  \item A Lot consists of 48 mobile phones of which 42 are good, 3 have only minor defects and 3 have major defects.Varnika will buy a phone if it is good but the trader will only buy a mobile if it has no major defects. One phone is selected at random from the lot. What is the probability that it is
\begin{enumerate}
	\item acceptable to Varnika?
            \item acceptable to the trader?
\end{enumerate}
\solution
	%\input{exemplar/10/13/3/40/main.tex}
 \item A student says that if you throw a die, it will show up 1 or not 1. Therefore, the probability of getting 1 and the probability of getting 'not 1' each is equal to $\frac{1}{2}$. Is this correct? Give reasons.\\
 \solution
        %\input{exemplar/10/13/2/9/main.tex}
   \item Four candidates A, B, C, D have ap-
plied for the assignment to coach a school cricket
team. If A is twice as likely to be selected as B, and
B and C are given about the same chance of being
selected, while C is twice as likely to be selected
as D, what are the probabilities that
\begin{enumerate}
\item C will be selected?
\item A will not be selected?
\end{enumerate}
	%\input{exemplar/11/16/3/9/main.tex}
 \item A bag contain 24 balls of which $x$ balls are red, $2x$ are white and $3x$ are blue. A ball is selected at random, What is the probability that it is
\begin{enumerate}[label=\alph*)]
\item not red ?
\item white ?
\end{enumerate}
%\input{exemplar/10/13/3/41/main.tex}
If the letters of the word ASSASSINATION are arranged at random. Find the Probability that
\begin{enumerate}[label=(\alph*)]
\item Four $S's$ come consecutively in the word
\item Two  $I's$ and two $N's$ come together
\item All $A's$ are not coming together
\item No two $A's$ are coming together
\end{enumerate}
%\input{exemplar/11/16/3/14/main.tex}
	\item One urn contains two black balls (labelled B1 and B2) and one white ball. A
	second urn contains one black ball and two white balls (labelled W1 and W2).
	Suppose the following experiment is performed. One of the two urns is chosen
	at random. Next a ball is randomly chosen from the urn. Then a second ball is
	chosen at random from the same urn without replacing the first ball.
	
	\begin{enumerate}
	\item What is the probability that two black balls are chosen?
	
	\item What is the probability that two balls of opposite colour are chosen?
	\end{enumerate}
	\solution
	%\input{exemplar/11/16/3/12/main1.tex}
\end{enumerate}

\item In a certain lottery 10,000 tickets are sold and ten equal prizes are awarded. What is the probability of not getting a prize if you buy (a) one ticket (b) two tickets (c) 10 tickets ?	
\\
\solution
		%\begin{enumerate}[label=\thesection.\arabic*,ref=\thesection.\theenumi]
	\item One card is drawn from a well-shuffled deck of 52 cards. Find the probability of getting
\begin{enumerate}
\item A king of red colour 
\item A face card 
\item A red face card
\item The jack of hearts
\item A spade
\item The queen of diamonds

\end{enumerate}
\solution
		%\input{ncert/10/15/1/14/main.tex}
	\item Five cards—the ten, jack, queen, king and ace of diamonds, are well-shuffled with their face downwards. One card is then picked up at random.
\begin{enumerate}
\item
What is the probability that the card is the queen? 
\item
If the queen is drawn and put aside, what is the probability that the second card picked up is (a) an ace? (b) a queen?\\
\end{enumerate}
\solution
		%\input{ncert/10/15/1/15/defs.tex}
	\item A bag contains $5$ red balls and some blue balls. If the probability of drawing a blue ball is double that if a red ball, determine the number of blue balls in the bag. 
		\\
\solution
		%\input{ncert/10/15/2/3/defs.tex}
	\item A card is selected from a pack of 52 cards.
 \begin{enumerate}[label=(\alph*)] 
                 \item How many points are there in the sample space?
                 \item Calculate the probability that the card is an ace of spades.
                 \item Calculate the probability that the card is (i) an ace and (ii) black card.
 \end{enumerate}
\solution
		%\input{ncert/11/16/3/4/main.tex}
\item Four cards are drawn from a well-shuffled deck of 52 cards. What is the probability of obtaining 3 diamonds and one spade.
\\
\solution
		%\input{ncert/11/16/4/2/defs.tex}
\item In a certain lottery 10,000 tickets are sold and ten equal prizes are awarded. What is the probability of not getting a prize if you buy (a) one ticket (b) two tickets (c) 10 tickets ?	
\\
\solution
		%\input{ncert/11/16/4/4/defs.tex}
		%
\item 
Out of 100 students, two sections of 40 and 60 are formed. If you and your friend are among the 100 students, what is the probability that
\begin{enumerate}
\item you both enter the same section?
\item you both enter the different sections?
\end{enumerate}
\solution
		%\input{ncert/11/16/4/5/defs.tex}
	\item 
The number lock of a suitcase has 4 wheels each labelled with ten digits i.e. from 0 to 9.The lock opens with a sequence of four digits with no repeats.What is the probability of a person getting the right sequence to open the suitcase.
\\
\solution
		%\input{ncert/11/16/4/10/defs.tex}
		%
\item 
Two cards are drawn at random and without replacement from a pack of 52 playing cards. Find the probability that both the cards are black.
\\
\solution
		%\input{ncert/12/13/2/2/defs.tex}
		\item A box of oranges is inspected by examining three randomly selected oranges drawn without replacement. If all the three oranges are good, the box is approved for sale, otherwise, it is rejected. Find the probability that a box containing 15 oranges out of which 12 are good and 3 are bad ones will be approved for sale.
		\label{ncert/12/13/2/3/defs.tex}
		\item Two balls are drawn at random with replacement from a box containing 10 black and 8 red balls. Find the probability that
		\label{ncert/12/13/2/12}
\begin{enumerate}
\item both balls are red.
\item first ball is black and second is red.
\item one of them is black and other is red.
\end{enumerate}

\item In a hostel, 60\% of the students read Hindi newspaper, 40\% read English newspaper and 20\% read both Hindi and English newspapers. A student is selected at random.
		\label{ncert/12/13/2/15}
\begin{enumerate}
\item Find the probability that she reads neither Hindi nor English newspapers.
\item If she reads Hindi newspaper, find the probability that she reads English newspaper.
\item If she reads English newspaper, find the probability that she reads Hindi newspaper.\\
\end{enumerate}
\item The probability of obtaining an even prime number on each die, when a pair of dice is rolled is 
\begin{enumerate}
    \item $0$ 
    
    \item $\frac{1}{3}$ 
    
    \item $\frac{1}{12}$ 
    
    \item $\frac{1}{36}$ 
\end{enumerate}
\solution
		%\input{ncert/12/13/2/17/defs.tex}
	\item A bag contains 4 red and 4 black balls, another bag contains 2 red and 6 black balls. One of the two bags is selected at random and a ball is drawn from the bag which is found to be red. Find the probability that the ball is drawn from the first bag.
\\
\solution
		%\input{ncert/12/13/3/2/main.tex}
  \item
  Cards with numbers 2 to 101 are placed in a box. A card is selected at random.Find the probability that the card has
\begin{enumerate}[label=(\roman*)]
	\item an even number 
	\item a square number
\end{enumerate}
\solution
%\input{exemplar/10/13/3/32/main.tex}
\item
The king, queen and jack of clubs are removed from a deck of 52 playing cards and then well shuffled. Now one card is drawn at random from the remaining cards.  Determine the probability that the card is
\begin{enumerate}[label=(\roman*)]
\item a club
\item 10 of hearts
\end{enumerate}
\solution
%\input{exemplar/10/13/3/29/main.tex}
\item A team of medical students doing their internship have to assist during surgeries
at a city hospital. The probabilities of surgeries rated as very complex, complex,
routine, simple or very simple are respectively, 0.15, 0.20, 0.31, 0.26, .08. Find
the probabilities that a particular surgery will be rated
\begin{enumerate}
	\item complex or very complex;
	\item neither very complex nor very simple;
	\item routine or complex
	\item routine or simple
\end{enumerate}
\solution
%\input{exemplar/11/16/3/8(1)/main.tex}
\item A card is selected from a pack of 52 cards.
\begin{enumerate}[label=(\alph*)]
    \item How many points are there in the sample space?
    \item Calculate the probability that the card is an ace of spades.
    \item Calculate the probability that the card is (i) an ace and (ii) black card.
\end{enumerate}
\solution
%\input{exemplar/11/16/3/4/main2.tex}
\item The probability that a non leap year selected at random will contain 53 sundays.
\\
\solution
%\input{exemplar/10/13/1/19/main.tex}
\item One of the four persons John, Rita, Aslam or Gurpreet will be promoted next
month. Consequently the sample space consists of four elementary outcomes
S = {John promoted, Rita promoted, Aslam promoted, Gurpreet promoted}
You are told that the chances of John’s promotion is same as that of Gurpreet,
Rita’s chances of promotion are twice as likely as Johns. Aslam’s chances are
four times that of John.
\begin{enumerate}
	\item Determine
	\begin{enumerate}
		\item P (John promoted)
		\item P (Rita promoted)
		\item P (Aslam promoted)
		\item P (Gurpreet promoted)
	\end{enumerate}
	\item If A = {John promoted or Gurpreet promoted}, find P (A).
\end{enumerate}
\solution
%\input{exemplar/11/16/3/10/main.tex}
\item A card is drawn from a deck of 52 cards. Find the probability of getting a king or a heart or a red card.\\
\solution
%\input{exemplar/11/16/3/15/main.tex}
\item The probability that a student will pass his examination is 0.73, the probability of
the student getting a compartment is 0.13, and the probability that the student will
either pass or get compartment is 0.96. State True or False.\\
\solution
%\input{exemplar/11/16/3/31/main.tex}
\item A card is selected from a pack of 52 cards\\
\begin{enumerate}[label=(\alph*)]
\item How many points are there in the sample space?
\item Calculate the probability that the cards is an ace of spades.
\item Calculate the probability that the card is (i) an ace (ii)black card.\\
\end{enumerate}
%\input{ncert/11/16/3/4_1/Prob_4.tex}
\item In a non-leap year, the probability of having 53 tuesdays or 53 wednesdays is\\
\solution
%\input{exemplar/11/16/3/18/main.tex}
\item There are 1000 sealed envelopes in a box, 10 of them contain a cash prize of
Rs 100 each, 100 of them contain a cash prize of Rs 50 each and 200 of them
contain a cash prize of Rs 10 each and rest do not contain any cash prize. If they
are well shuffled and an envelope is picked up out, what is the probability that it
contains no cash prize?\\
\solution
%\input{exemplar/10/13/3/34/main.tex}
\item 
A die is thrown and a card is selected at random from a deck of 52 playing cards. The probability of getting an even number on the die and a spade card.\\
\solution
%\input{exemplar/12/13/3/78/main.tex}
\item
If 4-digit numbers greater than 5,000 are randomly formed from the digits 0, 1, 3, 5, and 7, what is the probability of forming a number divisible by 5 when:
\begin{enumerate}
    \item The digits are repeated?
    \item The repetition of digits is not allowed?
\end{enumerate}
\solution
%\input{ncert/11/16/4/9/main.tex}
\item Consider the probability space $\brak{\Omega, \mathcal{G}, P}$ where $\Omega = [0,2]$ and $\mathcal{G} = \cbrak{\phi, \Omega, [0,1], (1,2]}$. Let $X$ and $Y$ be two functions on $\Omega$ defined as
\begin{align*}
    X(\omega) = 
    \begin{cases}
        1 & \text{if }\omega \in [0, 1]\\
        2 & \text{if }\omega \in (1, 2]
    \end{cases}
\end{align*}
and
\begin{align*}
    Y(\omega) = 
    \begin{cases}
        2 & \text{if }\omega \in [0, 1.5]\\
        3 & \text{if }\omega \in (1.5, 2].
    \end{cases}
\end{align*}
Then which one of the following statements is true?
\begin{enumerate}
    \item [(A)] $X$ is a random variable with respect to $\mathcal{G}$, but $Y$ is not a random variable with respect to $\mathcal{G}$.
    \item [(B)] $Y$ is a random variable with respect to $\mathcal{G}$, but $X$ is not a random variable with respect to $\mathcal{G}$.
    \item [(C)] Neither $X$ nor $Y$ is a random variable with respect to $\mathcal{G}$.
    \item [(D)] Both $X$ and $Y$ are random variables with respect to $\mathcal{G}$.
\end{enumerate} \hfill (GATE ST 2023)\\
\solution
%\input{gate/ST/2023/14/main.tex}
	\item  A die is loaded in such a way that each odd number is twice as likely to occur as
each even number. Find $P(G)$, where $G$ is the event that a number greater than
3 occurs on a single roll of the die.
\\
\solution
		%\input{exemplar/11/16/3/5/main.tex}
	\item All the jacks, queens and kings are removed from a deck of 52 playing cards. The remaining cards are well shuffled and then one card is drawn at random. Giving ace a value 1 similar value for other cards, find the probability that the card has a value 
		\begin{enumerate}
			\item 7
			\item greater than 7
			\item less than 7
		\end{enumerate}
		%\input{exemplar/10/13/3/30/main.tex}
  \item A Lot consists of 48 mobile phones of which 42 are good, 3 have only minor defects and 3 have major defects.Varnika will buy a phone if it is good but the trader will only buy a mobile if it has no major defects. One phone is selected at random from the lot. What is the probability that it is
\begin{enumerate}
	\item acceptable to Varnika?
            \item acceptable to the trader?
\end{enumerate}
\solution
	%\input{exemplar/10/13/3/40/main.tex}
 \item A student says that if you throw a die, it will show up 1 or not 1. Therefore, the probability of getting 1 and the probability of getting 'not 1' each is equal to $\frac{1}{2}$. Is this correct? Give reasons.\\
 \solution
        %\input{exemplar/10/13/2/9/main.tex}
   \item Four candidates A, B, C, D have ap-
plied for the assignment to coach a school cricket
team. If A is twice as likely to be selected as B, and
B and C are given about the same chance of being
selected, while C is twice as likely to be selected
as D, what are the probabilities that
\begin{enumerate}
\item C will be selected?
\item A will not be selected?
\end{enumerate}
	%\input{exemplar/11/16/3/9/main.tex}
 \item A bag contain 24 balls of which $x$ balls are red, $2x$ are white and $3x$ are blue. A ball is selected at random, What is the probability that it is
\begin{enumerate}[label=\alph*)]
\item not red ?
\item white ?
\end{enumerate}
%\input{exemplar/10/13/3/41/main.tex}
If the letters of the word ASSASSINATION are arranged at random. Find the Probability that
\begin{enumerate}[label=(\alph*)]
\item Four $S's$ come consecutively in the word
\item Two  $I's$ and two $N's$ come together
\item All $A's$ are not coming together
\item No two $A's$ are coming together
\end{enumerate}
%\input{exemplar/11/16/3/14/main.tex}
	\item One urn contains two black balls (labelled B1 and B2) and one white ball. A
	second urn contains one black ball and two white balls (labelled W1 and W2).
	Suppose the following experiment is performed. One of the two urns is chosen
	at random. Next a ball is randomly chosen from the urn. Then a second ball is
	chosen at random from the same urn without replacing the first ball.
	
	\begin{enumerate}
	\item What is the probability that two black balls are chosen?
	
	\item What is the probability that two balls of opposite colour are chosen?
	\end{enumerate}
	\solution
	%\input{exemplar/11/16/3/12/main1.tex}
\end{enumerate}

		%
\item 
Out of 100 students, two sections of 40 and 60 are formed. If you and your friend are among the 100 students, what is the probability that
\begin{enumerate}
\item you both enter the same section?
\item you both enter the different sections?
\end{enumerate}
\solution
		%\begin{enumerate}[label=\thesection.\arabic*,ref=\thesection.\theenumi]
	\item One card is drawn from a well-shuffled deck of 52 cards. Find the probability of getting
\begin{enumerate}
\item A king of red colour 
\item A face card 
\item A red face card
\item The jack of hearts
\item A spade
\item The queen of diamonds

\end{enumerate}
\solution
		%\input{ncert/10/15/1/14/main.tex}
	\item Five cards—the ten, jack, queen, king and ace of diamonds, are well-shuffled with their face downwards. One card is then picked up at random.
\begin{enumerate}
\item
What is the probability that the card is the queen? 
\item
If the queen is drawn and put aside, what is the probability that the second card picked up is (a) an ace? (b) a queen?\\
\end{enumerate}
\solution
		%\input{ncert/10/15/1/15/defs.tex}
	\item A bag contains $5$ red balls and some blue balls. If the probability of drawing a blue ball is double that if a red ball, determine the number of blue balls in the bag. 
		\\
\solution
		%\input{ncert/10/15/2/3/defs.tex}
	\item A card is selected from a pack of 52 cards.
 \begin{enumerate}[label=(\alph*)] 
                 \item How many points are there in the sample space?
                 \item Calculate the probability that the card is an ace of spades.
                 \item Calculate the probability that the card is (i) an ace and (ii) black card.
 \end{enumerate}
\solution
		%\input{ncert/11/16/3/4/main.tex}
\item Four cards are drawn from a well-shuffled deck of 52 cards. What is the probability of obtaining 3 diamonds and one spade.
\\
\solution
		%\input{ncert/11/16/4/2/defs.tex}
\item In a certain lottery 10,000 tickets are sold and ten equal prizes are awarded. What is the probability of not getting a prize if you buy (a) one ticket (b) two tickets (c) 10 tickets ?	
\\
\solution
		%\input{ncert/11/16/4/4/defs.tex}
		%
\item 
Out of 100 students, two sections of 40 and 60 are formed. If you and your friend are among the 100 students, what is the probability that
\begin{enumerate}
\item you both enter the same section?
\item you both enter the different sections?
\end{enumerate}
\solution
		%\input{ncert/11/16/4/5/defs.tex}
	\item 
The number lock of a suitcase has 4 wheels each labelled with ten digits i.e. from 0 to 9.The lock opens with a sequence of four digits with no repeats.What is the probability of a person getting the right sequence to open the suitcase.
\\
\solution
		%\input{ncert/11/16/4/10/defs.tex}
		%
\item 
Two cards are drawn at random and without replacement from a pack of 52 playing cards. Find the probability that both the cards are black.
\\
\solution
		%\input{ncert/12/13/2/2/defs.tex}
		\item A box of oranges is inspected by examining three randomly selected oranges drawn without replacement. If all the three oranges are good, the box is approved for sale, otherwise, it is rejected. Find the probability that a box containing 15 oranges out of which 12 are good and 3 are bad ones will be approved for sale.
		\label{ncert/12/13/2/3/defs.tex}
		\item Two balls are drawn at random with replacement from a box containing 10 black and 8 red balls. Find the probability that
		\label{ncert/12/13/2/12}
\begin{enumerate}
\item both balls are red.
\item first ball is black and second is red.
\item one of them is black and other is red.
\end{enumerate}

\item In a hostel, 60\% of the students read Hindi newspaper, 40\% read English newspaper and 20\% read both Hindi and English newspapers. A student is selected at random.
		\label{ncert/12/13/2/15}
\begin{enumerate}
\item Find the probability that she reads neither Hindi nor English newspapers.
\item If she reads Hindi newspaper, find the probability that she reads English newspaper.
\item If she reads English newspaper, find the probability that she reads Hindi newspaper.\\
\end{enumerate}
\item The probability of obtaining an even prime number on each die, when a pair of dice is rolled is 
\begin{enumerate}
    \item $0$ 
    
    \item $\frac{1}{3}$ 
    
    \item $\frac{1}{12}$ 
    
    \item $\frac{1}{36}$ 
\end{enumerate}
\solution
		%\input{ncert/12/13/2/17/defs.tex}
	\item A bag contains 4 red and 4 black balls, another bag contains 2 red and 6 black balls. One of the two bags is selected at random and a ball is drawn from the bag which is found to be red. Find the probability that the ball is drawn from the first bag.
\\
\solution
		%\input{ncert/12/13/3/2/main.tex}
  \item
  Cards with numbers 2 to 101 are placed in a box. A card is selected at random.Find the probability that the card has
\begin{enumerate}[label=(\roman*)]
	\item an even number 
	\item a square number
\end{enumerate}
\solution
%\input{exemplar/10/13/3/32/main.tex}
\item
The king, queen and jack of clubs are removed from a deck of 52 playing cards and then well shuffled. Now one card is drawn at random from the remaining cards.  Determine the probability that the card is
\begin{enumerate}[label=(\roman*)]
\item a club
\item 10 of hearts
\end{enumerate}
\solution
%\input{exemplar/10/13/3/29/main.tex}
\item A team of medical students doing their internship have to assist during surgeries
at a city hospital. The probabilities of surgeries rated as very complex, complex,
routine, simple or very simple are respectively, 0.15, 0.20, 0.31, 0.26, .08. Find
the probabilities that a particular surgery will be rated
\begin{enumerate}
	\item complex or very complex;
	\item neither very complex nor very simple;
	\item routine or complex
	\item routine or simple
\end{enumerate}
\solution
%\input{exemplar/11/16/3/8(1)/main.tex}
\item A card is selected from a pack of 52 cards.
\begin{enumerate}[label=(\alph*)]
    \item How many points are there in the sample space?
    \item Calculate the probability that the card is an ace of spades.
    \item Calculate the probability that the card is (i) an ace and (ii) black card.
\end{enumerate}
\solution
%\input{exemplar/11/16/3/4/main2.tex}
\item The probability that a non leap year selected at random will contain 53 sundays.
\\
\solution
%\input{exemplar/10/13/1/19/main.tex}
\item One of the four persons John, Rita, Aslam or Gurpreet will be promoted next
month. Consequently the sample space consists of four elementary outcomes
S = {John promoted, Rita promoted, Aslam promoted, Gurpreet promoted}
You are told that the chances of John’s promotion is same as that of Gurpreet,
Rita’s chances of promotion are twice as likely as Johns. Aslam’s chances are
four times that of John.
\begin{enumerate}
	\item Determine
	\begin{enumerate}
		\item P (John promoted)
		\item P (Rita promoted)
		\item P (Aslam promoted)
		\item P (Gurpreet promoted)
	\end{enumerate}
	\item If A = {John promoted or Gurpreet promoted}, find P (A).
\end{enumerate}
\solution
%\input{exemplar/11/16/3/10/main.tex}
\item A card is drawn from a deck of 52 cards. Find the probability of getting a king or a heart or a red card.\\
\solution
%\input{exemplar/11/16/3/15/main.tex}
\item The probability that a student will pass his examination is 0.73, the probability of
the student getting a compartment is 0.13, and the probability that the student will
either pass or get compartment is 0.96. State True or False.\\
\solution
%\input{exemplar/11/16/3/31/main.tex}
\item A card is selected from a pack of 52 cards\\
\begin{enumerate}[label=(\alph*)]
\item How many points are there in the sample space?
\item Calculate the probability that the cards is an ace of spades.
\item Calculate the probability that the card is (i) an ace (ii)black card.\\
\end{enumerate}
%\input{ncert/11/16/3/4_1/Prob_4.tex}
\item In a non-leap year, the probability of having 53 tuesdays or 53 wednesdays is\\
\solution
%\input{exemplar/11/16/3/18/main.tex}
\item There are 1000 sealed envelopes in a box, 10 of them contain a cash prize of
Rs 100 each, 100 of them contain a cash prize of Rs 50 each and 200 of them
contain a cash prize of Rs 10 each and rest do not contain any cash prize. If they
are well shuffled and an envelope is picked up out, what is the probability that it
contains no cash prize?\\
\solution
%\input{exemplar/10/13/3/34/main.tex}
\item 
A die is thrown and a card is selected at random from a deck of 52 playing cards. The probability of getting an even number on the die and a spade card.\\
\solution
%\input{exemplar/12/13/3/78/main.tex}
\item
If 4-digit numbers greater than 5,000 are randomly formed from the digits 0, 1, 3, 5, and 7, what is the probability of forming a number divisible by 5 when:
\begin{enumerate}
    \item The digits are repeated?
    \item The repetition of digits is not allowed?
\end{enumerate}
\solution
%\input{ncert/11/16/4/9/main.tex}
\item Consider the probability space $\brak{\Omega, \mathcal{G}, P}$ where $\Omega = [0,2]$ and $\mathcal{G} = \cbrak{\phi, \Omega, [0,1], (1,2]}$. Let $X$ and $Y$ be two functions on $\Omega$ defined as
\begin{align*}
    X(\omega) = 
    \begin{cases}
        1 & \text{if }\omega \in [0, 1]\\
        2 & \text{if }\omega \in (1, 2]
    \end{cases}
\end{align*}
and
\begin{align*}
    Y(\omega) = 
    \begin{cases}
        2 & \text{if }\omega \in [0, 1.5]\\
        3 & \text{if }\omega \in (1.5, 2].
    \end{cases}
\end{align*}
Then which one of the following statements is true?
\begin{enumerate}
    \item [(A)] $X$ is a random variable with respect to $\mathcal{G}$, but $Y$ is not a random variable with respect to $\mathcal{G}$.
    \item [(B)] $Y$ is a random variable with respect to $\mathcal{G}$, but $X$ is not a random variable with respect to $\mathcal{G}$.
    \item [(C)] Neither $X$ nor $Y$ is a random variable with respect to $\mathcal{G}$.
    \item [(D)] Both $X$ and $Y$ are random variables with respect to $\mathcal{G}$.
\end{enumerate} \hfill (GATE ST 2023)\\
\solution
%\input{gate/ST/2023/14/main.tex}
	\item  A die is loaded in such a way that each odd number is twice as likely to occur as
each even number. Find $P(G)$, where $G$ is the event that a number greater than
3 occurs on a single roll of the die.
\\
\solution
		%\input{exemplar/11/16/3/5/main.tex}
	\item All the jacks, queens and kings are removed from a deck of 52 playing cards. The remaining cards are well shuffled and then one card is drawn at random. Giving ace a value 1 similar value for other cards, find the probability that the card has a value 
		\begin{enumerate}
			\item 7
			\item greater than 7
			\item less than 7
		\end{enumerate}
		%\input{exemplar/10/13/3/30/main.tex}
  \item A Lot consists of 48 mobile phones of which 42 are good, 3 have only minor defects and 3 have major defects.Varnika will buy a phone if it is good but the trader will only buy a mobile if it has no major defects. One phone is selected at random from the lot. What is the probability that it is
\begin{enumerate}
	\item acceptable to Varnika?
            \item acceptable to the trader?
\end{enumerate}
\solution
	%\input{exemplar/10/13/3/40/main.tex}
 \item A student says that if you throw a die, it will show up 1 or not 1. Therefore, the probability of getting 1 and the probability of getting 'not 1' each is equal to $\frac{1}{2}$. Is this correct? Give reasons.\\
 \solution
        %\input{exemplar/10/13/2/9/main.tex}
   \item Four candidates A, B, C, D have ap-
plied for the assignment to coach a school cricket
team. If A is twice as likely to be selected as B, and
B and C are given about the same chance of being
selected, while C is twice as likely to be selected
as D, what are the probabilities that
\begin{enumerate}
\item C will be selected?
\item A will not be selected?
\end{enumerate}
	%\input{exemplar/11/16/3/9/main.tex}
 \item A bag contain 24 balls of which $x$ balls are red, $2x$ are white and $3x$ are blue. A ball is selected at random, What is the probability that it is
\begin{enumerate}[label=\alph*)]
\item not red ?
\item white ?
\end{enumerate}
%\input{exemplar/10/13/3/41/main.tex}
If the letters of the word ASSASSINATION are arranged at random. Find the Probability that
\begin{enumerate}[label=(\alph*)]
\item Four $S's$ come consecutively in the word
\item Two  $I's$ and two $N's$ come together
\item All $A's$ are not coming together
\item No two $A's$ are coming together
\end{enumerate}
%\input{exemplar/11/16/3/14/main.tex}
	\item One urn contains two black balls (labelled B1 and B2) and one white ball. A
	second urn contains one black ball and two white balls (labelled W1 and W2).
	Suppose the following experiment is performed. One of the two urns is chosen
	at random. Next a ball is randomly chosen from the urn. Then a second ball is
	chosen at random from the same urn without replacing the first ball.
	
	\begin{enumerate}
	\item What is the probability that two black balls are chosen?
	
	\item What is the probability that two balls of opposite colour are chosen?
	\end{enumerate}
	\solution
	%\input{exemplar/11/16/3/12/main1.tex}
\end{enumerate}

	\item 
The number lock of a suitcase has 4 wheels each labelled with ten digits i.e. from 0 to 9.The lock opens with a sequence of four digits with no repeats.What is the probability of a person getting the right sequence to open the suitcase.
\\
\solution
		%\begin{enumerate}[label=\thesection.\arabic*,ref=\thesection.\theenumi]
	\item One card is drawn from a well-shuffled deck of 52 cards. Find the probability of getting
\begin{enumerate}
\item A king of red colour 
\item A face card 
\item A red face card
\item The jack of hearts
\item A spade
\item The queen of diamonds

\end{enumerate}
\solution
		%\input{ncert/10/15/1/14/main.tex}
	\item Five cards—the ten, jack, queen, king and ace of diamonds, are well-shuffled with their face downwards. One card is then picked up at random.
\begin{enumerate}
\item
What is the probability that the card is the queen? 
\item
If the queen is drawn and put aside, what is the probability that the second card picked up is (a) an ace? (b) a queen?\\
\end{enumerate}
\solution
		%\input{ncert/10/15/1/15/defs.tex}
	\item A bag contains $5$ red balls and some blue balls. If the probability of drawing a blue ball is double that if a red ball, determine the number of blue balls in the bag. 
		\\
\solution
		%\input{ncert/10/15/2/3/defs.tex}
	\item A card is selected from a pack of 52 cards.
 \begin{enumerate}[label=(\alph*)] 
                 \item How many points are there in the sample space?
                 \item Calculate the probability that the card is an ace of spades.
                 \item Calculate the probability that the card is (i) an ace and (ii) black card.
 \end{enumerate}
\solution
		%\input{ncert/11/16/3/4/main.tex}
\item Four cards are drawn from a well-shuffled deck of 52 cards. What is the probability of obtaining 3 diamonds and one spade.
\\
\solution
		%\input{ncert/11/16/4/2/defs.tex}
\item In a certain lottery 10,000 tickets are sold and ten equal prizes are awarded. What is the probability of not getting a prize if you buy (a) one ticket (b) two tickets (c) 10 tickets ?	
\\
\solution
		%\input{ncert/11/16/4/4/defs.tex}
		%
\item 
Out of 100 students, two sections of 40 and 60 are formed. If you and your friend are among the 100 students, what is the probability that
\begin{enumerate}
\item you both enter the same section?
\item you both enter the different sections?
\end{enumerate}
\solution
		%\input{ncert/11/16/4/5/defs.tex}
	\item 
The number lock of a suitcase has 4 wheels each labelled with ten digits i.e. from 0 to 9.The lock opens with a sequence of four digits with no repeats.What is the probability of a person getting the right sequence to open the suitcase.
\\
\solution
		%\input{ncert/11/16/4/10/defs.tex}
		%
\item 
Two cards are drawn at random and without replacement from a pack of 52 playing cards. Find the probability that both the cards are black.
\\
\solution
		%\input{ncert/12/13/2/2/defs.tex}
		\item A box of oranges is inspected by examining three randomly selected oranges drawn without replacement. If all the three oranges are good, the box is approved for sale, otherwise, it is rejected. Find the probability that a box containing 15 oranges out of which 12 are good and 3 are bad ones will be approved for sale.
		\label{ncert/12/13/2/3/defs.tex}
		\item Two balls are drawn at random with replacement from a box containing 10 black and 8 red balls. Find the probability that
		\label{ncert/12/13/2/12}
\begin{enumerate}
\item both balls are red.
\item first ball is black and second is red.
\item one of them is black and other is red.
\end{enumerate}

\item In a hostel, 60\% of the students read Hindi newspaper, 40\% read English newspaper and 20\% read both Hindi and English newspapers. A student is selected at random.
		\label{ncert/12/13/2/15}
\begin{enumerate}
\item Find the probability that she reads neither Hindi nor English newspapers.
\item If she reads Hindi newspaper, find the probability that she reads English newspaper.
\item If she reads English newspaper, find the probability that she reads Hindi newspaper.\\
\end{enumerate}
\item The probability of obtaining an even prime number on each die, when a pair of dice is rolled is 
\begin{enumerate}
    \item $0$ 
    
    \item $\frac{1}{3}$ 
    
    \item $\frac{1}{12}$ 
    
    \item $\frac{1}{36}$ 
\end{enumerate}
\solution
		%\input{ncert/12/13/2/17/defs.tex}
	\item A bag contains 4 red and 4 black balls, another bag contains 2 red and 6 black balls. One of the two bags is selected at random and a ball is drawn from the bag which is found to be red. Find the probability that the ball is drawn from the first bag.
\\
\solution
		%\input{ncert/12/13/3/2/main.tex}
  \item
  Cards with numbers 2 to 101 are placed in a box. A card is selected at random.Find the probability that the card has
\begin{enumerate}[label=(\roman*)]
	\item an even number 
	\item a square number
\end{enumerate}
\solution
%\input{exemplar/10/13/3/32/main.tex}
\item
The king, queen and jack of clubs are removed from a deck of 52 playing cards and then well shuffled. Now one card is drawn at random from the remaining cards.  Determine the probability that the card is
\begin{enumerate}[label=(\roman*)]
\item a club
\item 10 of hearts
\end{enumerate}
\solution
%\input{exemplar/10/13/3/29/main.tex}
\item A team of medical students doing their internship have to assist during surgeries
at a city hospital. The probabilities of surgeries rated as very complex, complex,
routine, simple or very simple are respectively, 0.15, 0.20, 0.31, 0.26, .08. Find
the probabilities that a particular surgery will be rated
\begin{enumerate}
	\item complex or very complex;
	\item neither very complex nor very simple;
	\item routine or complex
	\item routine or simple
\end{enumerate}
\solution
%\input{exemplar/11/16/3/8(1)/main.tex}
\item A card is selected from a pack of 52 cards.
\begin{enumerate}[label=(\alph*)]
    \item How many points are there in the sample space?
    \item Calculate the probability that the card is an ace of spades.
    \item Calculate the probability that the card is (i) an ace and (ii) black card.
\end{enumerate}
\solution
%\input{exemplar/11/16/3/4/main2.tex}
\item The probability that a non leap year selected at random will contain 53 sundays.
\\
\solution
%\input{exemplar/10/13/1/19/main.tex}
\item One of the four persons John, Rita, Aslam or Gurpreet will be promoted next
month. Consequently the sample space consists of four elementary outcomes
S = {John promoted, Rita promoted, Aslam promoted, Gurpreet promoted}
You are told that the chances of John’s promotion is same as that of Gurpreet,
Rita’s chances of promotion are twice as likely as Johns. Aslam’s chances are
four times that of John.
\begin{enumerate}
	\item Determine
	\begin{enumerate}
		\item P (John promoted)
		\item P (Rita promoted)
		\item P (Aslam promoted)
		\item P (Gurpreet promoted)
	\end{enumerate}
	\item If A = {John promoted or Gurpreet promoted}, find P (A).
\end{enumerate}
\solution
%\input{exemplar/11/16/3/10/main.tex}
\item A card is drawn from a deck of 52 cards. Find the probability of getting a king or a heart or a red card.\\
\solution
%\input{exemplar/11/16/3/15/main.tex}
\item The probability that a student will pass his examination is 0.73, the probability of
the student getting a compartment is 0.13, and the probability that the student will
either pass or get compartment is 0.96. State True or False.\\
\solution
%\input{exemplar/11/16/3/31/main.tex}
\item A card is selected from a pack of 52 cards\\
\begin{enumerate}[label=(\alph*)]
\item How many points are there in the sample space?
\item Calculate the probability that the cards is an ace of spades.
\item Calculate the probability that the card is (i) an ace (ii)black card.\\
\end{enumerate}
%\input{ncert/11/16/3/4_1/Prob_4.tex}
\item In a non-leap year, the probability of having 53 tuesdays or 53 wednesdays is\\
\solution
%\input{exemplar/11/16/3/18/main.tex}
\item There are 1000 sealed envelopes in a box, 10 of them contain a cash prize of
Rs 100 each, 100 of them contain a cash prize of Rs 50 each and 200 of them
contain a cash prize of Rs 10 each and rest do not contain any cash prize. If they
are well shuffled and an envelope is picked up out, what is the probability that it
contains no cash prize?\\
\solution
%\input{exemplar/10/13/3/34/main.tex}
\item 
A die is thrown and a card is selected at random from a deck of 52 playing cards. The probability of getting an even number on the die and a spade card.\\
\solution
%\input{exemplar/12/13/3/78/main.tex}
\item
If 4-digit numbers greater than 5,000 are randomly formed from the digits 0, 1, 3, 5, and 7, what is the probability of forming a number divisible by 5 when:
\begin{enumerate}
    \item The digits are repeated?
    \item The repetition of digits is not allowed?
\end{enumerate}
\solution
%\input{ncert/11/16/4/9/main.tex}
\item Consider the probability space $\brak{\Omega, \mathcal{G}, P}$ where $\Omega = [0,2]$ and $\mathcal{G} = \cbrak{\phi, \Omega, [0,1], (1,2]}$. Let $X$ and $Y$ be two functions on $\Omega$ defined as
\begin{align*}
    X(\omega) = 
    \begin{cases}
        1 & \text{if }\omega \in [0, 1]\\
        2 & \text{if }\omega \in (1, 2]
    \end{cases}
\end{align*}
and
\begin{align*}
    Y(\omega) = 
    \begin{cases}
        2 & \text{if }\omega \in [0, 1.5]\\
        3 & \text{if }\omega \in (1.5, 2].
    \end{cases}
\end{align*}
Then which one of the following statements is true?
\begin{enumerate}
    \item [(A)] $X$ is a random variable with respect to $\mathcal{G}$, but $Y$ is not a random variable with respect to $\mathcal{G}$.
    \item [(B)] $Y$ is a random variable with respect to $\mathcal{G}$, but $X$ is not a random variable with respect to $\mathcal{G}$.
    \item [(C)] Neither $X$ nor $Y$ is a random variable with respect to $\mathcal{G}$.
    \item [(D)] Both $X$ and $Y$ are random variables with respect to $\mathcal{G}$.
\end{enumerate} \hfill (GATE ST 2023)\\
\solution
%\input{gate/ST/2023/14/main.tex}
	\item  A die is loaded in such a way that each odd number is twice as likely to occur as
each even number. Find $P(G)$, where $G$ is the event that a number greater than
3 occurs on a single roll of the die.
\\
\solution
		%\input{exemplar/11/16/3/5/main.tex}
	\item All the jacks, queens and kings are removed from a deck of 52 playing cards. The remaining cards are well shuffled and then one card is drawn at random. Giving ace a value 1 similar value for other cards, find the probability that the card has a value 
		\begin{enumerate}
			\item 7
			\item greater than 7
			\item less than 7
		\end{enumerate}
		%\input{exemplar/10/13/3/30/main.tex}
  \item A Lot consists of 48 mobile phones of which 42 are good, 3 have only minor defects and 3 have major defects.Varnika will buy a phone if it is good but the trader will only buy a mobile if it has no major defects. One phone is selected at random from the lot. What is the probability that it is
\begin{enumerate}
	\item acceptable to Varnika?
            \item acceptable to the trader?
\end{enumerate}
\solution
	%\input{exemplar/10/13/3/40/main.tex}
 \item A student says that if you throw a die, it will show up 1 or not 1. Therefore, the probability of getting 1 and the probability of getting 'not 1' each is equal to $\frac{1}{2}$. Is this correct? Give reasons.\\
 \solution
        %\input{exemplar/10/13/2/9/main.tex}
   \item Four candidates A, B, C, D have ap-
plied for the assignment to coach a school cricket
team. If A is twice as likely to be selected as B, and
B and C are given about the same chance of being
selected, while C is twice as likely to be selected
as D, what are the probabilities that
\begin{enumerate}
\item C will be selected?
\item A will not be selected?
\end{enumerate}
	%\input{exemplar/11/16/3/9/main.tex}
 \item A bag contain 24 balls of which $x$ balls are red, $2x$ are white and $3x$ are blue. A ball is selected at random, What is the probability that it is
\begin{enumerate}[label=\alph*)]
\item not red ?
\item white ?
\end{enumerate}
%\input{exemplar/10/13/3/41/main.tex}
If the letters of the word ASSASSINATION are arranged at random. Find the Probability that
\begin{enumerate}[label=(\alph*)]
\item Four $S's$ come consecutively in the word
\item Two  $I's$ and two $N's$ come together
\item All $A's$ are not coming together
\item No two $A's$ are coming together
\end{enumerate}
%\input{exemplar/11/16/3/14/main.tex}
	\item One urn contains two black balls (labelled B1 and B2) and one white ball. A
	second urn contains one black ball and two white balls (labelled W1 and W2).
	Suppose the following experiment is performed. One of the two urns is chosen
	at random. Next a ball is randomly chosen from the urn. Then a second ball is
	chosen at random from the same urn without replacing the first ball.
	
	\begin{enumerate}
	\item What is the probability that two black balls are chosen?
	
	\item What is the probability that two balls of opposite colour are chosen?
	\end{enumerate}
	\solution
	%\input{exemplar/11/16/3/12/main1.tex}
\end{enumerate}

		%
\item 
Two cards are drawn at random and without replacement from a pack of 52 playing cards. Find the probability that both the cards are black.
\\
\solution
		%\begin{enumerate}[label=\thesection.\arabic*,ref=\thesection.\theenumi]
	\item One card is drawn from a well-shuffled deck of 52 cards. Find the probability of getting
\begin{enumerate}
\item A king of red colour 
\item A face card 
\item A red face card
\item The jack of hearts
\item A spade
\item The queen of diamonds

\end{enumerate}
\solution
		%\input{ncert/10/15/1/14/main.tex}
	\item Five cards—the ten, jack, queen, king and ace of diamonds, are well-shuffled with their face downwards. One card is then picked up at random.
\begin{enumerate}
\item
What is the probability that the card is the queen? 
\item
If the queen is drawn and put aside, what is the probability that the second card picked up is (a) an ace? (b) a queen?\\
\end{enumerate}
\solution
		%\input{ncert/10/15/1/15/defs.tex}
	\item A bag contains $5$ red balls and some blue balls. If the probability of drawing a blue ball is double that if a red ball, determine the number of blue balls in the bag. 
		\\
\solution
		%\input{ncert/10/15/2/3/defs.tex}
	\item A card is selected from a pack of 52 cards.
 \begin{enumerate}[label=(\alph*)] 
                 \item How many points are there in the sample space?
                 \item Calculate the probability that the card is an ace of spades.
                 \item Calculate the probability that the card is (i) an ace and (ii) black card.
 \end{enumerate}
\solution
		%\input{ncert/11/16/3/4/main.tex}
\item Four cards are drawn from a well-shuffled deck of 52 cards. What is the probability of obtaining 3 diamonds and one spade.
\\
\solution
		%\input{ncert/11/16/4/2/defs.tex}
\item In a certain lottery 10,000 tickets are sold and ten equal prizes are awarded. What is the probability of not getting a prize if you buy (a) one ticket (b) two tickets (c) 10 tickets ?	
\\
\solution
		%\input{ncert/11/16/4/4/defs.tex}
		%
\item 
Out of 100 students, two sections of 40 and 60 are formed. If you and your friend are among the 100 students, what is the probability that
\begin{enumerate}
\item you both enter the same section?
\item you both enter the different sections?
\end{enumerate}
\solution
		%\input{ncert/11/16/4/5/defs.tex}
	\item 
The number lock of a suitcase has 4 wheels each labelled with ten digits i.e. from 0 to 9.The lock opens with a sequence of four digits with no repeats.What is the probability of a person getting the right sequence to open the suitcase.
\\
\solution
		%\input{ncert/11/16/4/10/defs.tex}
		%
\item 
Two cards are drawn at random and without replacement from a pack of 52 playing cards. Find the probability that both the cards are black.
\\
\solution
		%\input{ncert/12/13/2/2/defs.tex}
		\item A box of oranges is inspected by examining three randomly selected oranges drawn without replacement. If all the three oranges are good, the box is approved for sale, otherwise, it is rejected. Find the probability that a box containing 15 oranges out of which 12 are good and 3 are bad ones will be approved for sale.
		\label{ncert/12/13/2/3/defs.tex}
		\item Two balls are drawn at random with replacement from a box containing 10 black and 8 red balls. Find the probability that
		\label{ncert/12/13/2/12}
\begin{enumerate}
\item both balls are red.
\item first ball is black and second is red.
\item one of them is black and other is red.
\end{enumerate}

\item In a hostel, 60\% of the students read Hindi newspaper, 40\% read English newspaper and 20\% read both Hindi and English newspapers. A student is selected at random.
		\label{ncert/12/13/2/15}
\begin{enumerate}
\item Find the probability that she reads neither Hindi nor English newspapers.
\item If she reads Hindi newspaper, find the probability that she reads English newspaper.
\item If she reads English newspaper, find the probability that she reads Hindi newspaper.\\
\end{enumerate}
\item The probability of obtaining an even prime number on each die, when a pair of dice is rolled is 
\begin{enumerate}
    \item $0$ 
    
    \item $\frac{1}{3}$ 
    
    \item $\frac{1}{12}$ 
    
    \item $\frac{1}{36}$ 
\end{enumerate}
\solution
		%\input{ncert/12/13/2/17/defs.tex}
	\item A bag contains 4 red and 4 black balls, another bag contains 2 red and 6 black balls. One of the two bags is selected at random and a ball is drawn from the bag which is found to be red. Find the probability that the ball is drawn from the first bag.
\\
\solution
		%\input{ncert/12/13/3/2/main.tex}
  \item
  Cards with numbers 2 to 101 are placed in a box. A card is selected at random.Find the probability that the card has
\begin{enumerate}[label=(\roman*)]
	\item an even number 
	\item a square number
\end{enumerate}
\solution
%\input{exemplar/10/13/3/32/main.tex}
\item
The king, queen and jack of clubs are removed from a deck of 52 playing cards and then well shuffled. Now one card is drawn at random from the remaining cards.  Determine the probability that the card is
\begin{enumerate}[label=(\roman*)]
\item a club
\item 10 of hearts
\end{enumerate}
\solution
%\input{exemplar/10/13/3/29/main.tex}
\item A team of medical students doing their internship have to assist during surgeries
at a city hospital. The probabilities of surgeries rated as very complex, complex,
routine, simple or very simple are respectively, 0.15, 0.20, 0.31, 0.26, .08. Find
the probabilities that a particular surgery will be rated
\begin{enumerate}
	\item complex or very complex;
	\item neither very complex nor very simple;
	\item routine or complex
	\item routine or simple
\end{enumerate}
\solution
%\input{exemplar/11/16/3/8(1)/main.tex}
\item A card is selected from a pack of 52 cards.
\begin{enumerate}[label=(\alph*)]
    \item How many points are there in the sample space?
    \item Calculate the probability that the card is an ace of spades.
    \item Calculate the probability that the card is (i) an ace and (ii) black card.
\end{enumerate}
\solution
%\input{exemplar/11/16/3/4/main2.tex}
\item The probability that a non leap year selected at random will contain 53 sundays.
\\
\solution
%\input{exemplar/10/13/1/19/main.tex}
\item One of the four persons John, Rita, Aslam or Gurpreet will be promoted next
month. Consequently the sample space consists of four elementary outcomes
S = {John promoted, Rita promoted, Aslam promoted, Gurpreet promoted}
You are told that the chances of John’s promotion is same as that of Gurpreet,
Rita’s chances of promotion are twice as likely as Johns. Aslam’s chances are
four times that of John.
\begin{enumerate}
	\item Determine
	\begin{enumerate}
		\item P (John promoted)
		\item P (Rita promoted)
		\item P (Aslam promoted)
		\item P (Gurpreet promoted)
	\end{enumerate}
	\item If A = {John promoted or Gurpreet promoted}, find P (A).
\end{enumerate}
\solution
%\input{exemplar/11/16/3/10/main.tex}
\item A card is drawn from a deck of 52 cards. Find the probability of getting a king or a heart or a red card.\\
\solution
%\input{exemplar/11/16/3/15/main.tex}
\item The probability that a student will pass his examination is 0.73, the probability of
the student getting a compartment is 0.13, and the probability that the student will
either pass or get compartment is 0.96. State True or False.\\
\solution
%\input{exemplar/11/16/3/31/main.tex}
\item A card is selected from a pack of 52 cards\\
\begin{enumerate}[label=(\alph*)]
\item How many points are there in the sample space?
\item Calculate the probability that the cards is an ace of spades.
\item Calculate the probability that the card is (i) an ace (ii)black card.\\
\end{enumerate}
%\input{ncert/11/16/3/4_1/Prob_4.tex}
\item In a non-leap year, the probability of having 53 tuesdays or 53 wednesdays is\\
\solution
%\input{exemplar/11/16/3/18/main.tex}
\item There are 1000 sealed envelopes in a box, 10 of them contain a cash prize of
Rs 100 each, 100 of them contain a cash prize of Rs 50 each and 200 of them
contain a cash prize of Rs 10 each and rest do not contain any cash prize. If they
are well shuffled and an envelope is picked up out, what is the probability that it
contains no cash prize?\\
\solution
%\input{exemplar/10/13/3/34/main.tex}
\item 
A die is thrown and a card is selected at random from a deck of 52 playing cards. The probability of getting an even number on the die and a spade card.\\
\solution
%\input{exemplar/12/13/3/78/main.tex}
\item
If 4-digit numbers greater than 5,000 are randomly formed from the digits 0, 1, 3, 5, and 7, what is the probability of forming a number divisible by 5 when:
\begin{enumerate}
    \item The digits are repeated?
    \item The repetition of digits is not allowed?
\end{enumerate}
\solution
%\input{ncert/11/16/4/9/main.tex}
\item Consider the probability space $\brak{\Omega, \mathcal{G}, P}$ where $\Omega = [0,2]$ and $\mathcal{G} = \cbrak{\phi, \Omega, [0,1], (1,2]}$. Let $X$ and $Y$ be two functions on $\Omega$ defined as
\begin{align*}
    X(\omega) = 
    \begin{cases}
        1 & \text{if }\omega \in [0, 1]\\
        2 & \text{if }\omega \in (1, 2]
    \end{cases}
\end{align*}
and
\begin{align*}
    Y(\omega) = 
    \begin{cases}
        2 & \text{if }\omega \in [0, 1.5]\\
        3 & \text{if }\omega \in (1.5, 2].
    \end{cases}
\end{align*}
Then which one of the following statements is true?
\begin{enumerate}
    \item [(A)] $X$ is a random variable with respect to $\mathcal{G}$, but $Y$ is not a random variable with respect to $\mathcal{G}$.
    \item [(B)] $Y$ is a random variable with respect to $\mathcal{G}$, but $X$ is not a random variable with respect to $\mathcal{G}$.
    \item [(C)] Neither $X$ nor $Y$ is a random variable with respect to $\mathcal{G}$.
    \item [(D)] Both $X$ and $Y$ are random variables with respect to $\mathcal{G}$.
\end{enumerate} \hfill (GATE ST 2023)\\
\solution
%\input{gate/ST/2023/14/main.tex}
	\item  A die is loaded in such a way that each odd number is twice as likely to occur as
each even number. Find $P(G)$, where $G$ is the event that a number greater than
3 occurs on a single roll of the die.
\\
\solution
		%\input{exemplar/11/16/3/5/main.tex}
	\item All the jacks, queens and kings are removed from a deck of 52 playing cards. The remaining cards are well shuffled and then one card is drawn at random. Giving ace a value 1 similar value for other cards, find the probability that the card has a value 
		\begin{enumerate}
			\item 7
			\item greater than 7
			\item less than 7
		\end{enumerate}
		%\input{exemplar/10/13/3/30/main.tex}
  \item A Lot consists of 48 mobile phones of which 42 are good, 3 have only minor defects and 3 have major defects.Varnika will buy a phone if it is good but the trader will only buy a mobile if it has no major defects. One phone is selected at random from the lot. What is the probability that it is
\begin{enumerate}
	\item acceptable to Varnika?
            \item acceptable to the trader?
\end{enumerate}
\solution
	%\input{exemplar/10/13/3/40/main.tex}
 \item A student says that if you throw a die, it will show up 1 or not 1. Therefore, the probability of getting 1 and the probability of getting 'not 1' each is equal to $\frac{1}{2}$. Is this correct? Give reasons.\\
 \solution
        %\input{exemplar/10/13/2/9/main.tex}
   \item Four candidates A, B, C, D have ap-
plied for the assignment to coach a school cricket
team. If A is twice as likely to be selected as B, and
B and C are given about the same chance of being
selected, while C is twice as likely to be selected
as D, what are the probabilities that
\begin{enumerate}
\item C will be selected?
\item A will not be selected?
\end{enumerate}
	%\input{exemplar/11/16/3/9/main.tex}
 \item A bag contain 24 balls of which $x$ balls are red, $2x$ are white and $3x$ are blue. A ball is selected at random, What is the probability that it is
\begin{enumerate}[label=\alph*)]
\item not red ?
\item white ?
\end{enumerate}
%\input{exemplar/10/13/3/41/main.tex}
If the letters of the word ASSASSINATION are arranged at random. Find the Probability that
\begin{enumerate}[label=(\alph*)]
\item Four $S's$ come consecutively in the word
\item Two  $I's$ and two $N's$ come together
\item All $A's$ are not coming together
\item No two $A's$ are coming together
\end{enumerate}
%\input{exemplar/11/16/3/14/main.tex}
	\item One urn contains two black balls (labelled B1 and B2) and one white ball. A
	second urn contains one black ball and two white balls (labelled W1 and W2).
	Suppose the following experiment is performed. One of the two urns is chosen
	at random. Next a ball is randomly chosen from the urn. Then a second ball is
	chosen at random from the same urn without replacing the first ball.
	
	\begin{enumerate}
	\item What is the probability that two black balls are chosen?
	
	\item What is the probability that two balls of opposite colour are chosen?
	\end{enumerate}
	\solution
	%\input{exemplar/11/16/3/12/main1.tex}
\end{enumerate}

		\item A box of oranges is inspected by examining three randomly selected oranges drawn without replacement. If all the three oranges are good, the box is approved for sale, otherwise, it is rejected. Find the probability that a box containing 15 oranges out of which 12 are good and 3 are bad ones will be approved for sale.
		\label{ncert/12/13/2/3/defs.tex}
		\item Two balls are drawn at random with replacement from a box containing 10 black and 8 red balls. Find the probability that
		\label{ncert/12/13/2/12}
\begin{enumerate}
\item both balls are red.
\item first ball is black and second is red.
\item one of them is black and other is red.
\end{enumerate}

\item In a hostel, 60\% of the students read Hindi newspaper, 40\% read English newspaper and 20\% read both Hindi and English newspapers. A student is selected at random.
		\label{ncert/12/13/2/15}
\begin{enumerate}
\item Find the probability that she reads neither Hindi nor English newspapers.
\item If she reads Hindi newspaper, find the probability that she reads English newspaper.
\item If she reads English newspaper, find the probability that she reads Hindi newspaper.\\
\end{enumerate}
\item The probability of obtaining an even prime number on each die, when a pair of dice is rolled is 
\begin{enumerate}
    \item $0$ 
    
    \item $\frac{1}{3}$ 
    
    \item $\frac{1}{12}$ 
    
    \item $\frac{1}{36}$ 
\end{enumerate}
\solution
		%\begin{enumerate}[label=\thesection.\arabic*,ref=\thesection.\theenumi]
	\item One card is drawn from a well-shuffled deck of 52 cards. Find the probability of getting
\begin{enumerate}
\item A king of red colour 
\item A face card 
\item A red face card
\item The jack of hearts
\item A spade
\item The queen of diamonds

\end{enumerate}
\solution
		%\input{ncert/10/15/1/14/main.tex}
	\item Five cards—the ten, jack, queen, king and ace of diamonds, are well-shuffled with their face downwards. One card is then picked up at random.
\begin{enumerate}
\item
What is the probability that the card is the queen? 
\item
If the queen is drawn and put aside, what is the probability that the second card picked up is (a) an ace? (b) a queen?\\
\end{enumerate}
\solution
		%\input{ncert/10/15/1/15/defs.tex}
	\item A bag contains $5$ red balls and some blue balls. If the probability of drawing a blue ball is double that if a red ball, determine the number of blue balls in the bag. 
		\\
\solution
		%\input{ncert/10/15/2/3/defs.tex}
	\item A card is selected from a pack of 52 cards.
 \begin{enumerate}[label=(\alph*)] 
                 \item How many points are there in the sample space?
                 \item Calculate the probability that the card is an ace of spades.
                 \item Calculate the probability that the card is (i) an ace and (ii) black card.
 \end{enumerate}
\solution
		%\input{ncert/11/16/3/4/main.tex}
\item Four cards are drawn from a well-shuffled deck of 52 cards. What is the probability of obtaining 3 diamonds and one spade.
\\
\solution
		%\input{ncert/11/16/4/2/defs.tex}
\item In a certain lottery 10,000 tickets are sold and ten equal prizes are awarded. What is the probability of not getting a prize if you buy (a) one ticket (b) two tickets (c) 10 tickets ?	
\\
\solution
		%\input{ncert/11/16/4/4/defs.tex}
		%
\item 
Out of 100 students, two sections of 40 and 60 are formed. If you and your friend are among the 100 students, what is the probability that
\begin{enumerate}
\item you both enter the same section?
\item you both enter the different sections?
\end{enumerate}
\solution
		%\input{ncert/11/16/4/5/defs.tex}
	\item 
The number lock of a suitcase has 4 wheels each labelled with ten digits i.e. from 0 to 9.The lock opens with a sequence of four digits with no repeats.What is the probability of a person getting the right sequence to open the suitcase.
\\
\solution
		%\input{ncert/11/16/4/10/defs.tex}
		%
\item 
Two cards are drawn at random and without replacement from a pack of 52 playing cards. Find the probability that both the cards are black.
\\
\solution
		%\input{ncert/12/13/2/2/defs.tex}
		\item A box of oranges is inspected by examining three randomly selected oranges drawn without replacement. If all the three oranges are good, the box is approved for sale, otherwise, it is rejected. Find the probability that a box containing 15 oranges out of which 12 are good and 3 are bad ones will be approved for sale.
		\label{ncert/12/13/2/3/defs.tex}
		\item Two balls are drawn at random with replacement from a box containing 10 black and 8 red balls. Find the probability that
		\label{ncert/12/13/2/12}
\begin{enumerate}
\item both balls are red.
\item first ball is black and second is red.
\item one of them is black and other is red.
\end{enumerate}

\item In a hostel, 60\% of the students read Hindi newspaper, 40\% read English newspaper and 20\% read both Hindi and English newspapers. A student is selected at random.
		\label{ncert/12/13/2/15}
\begin{enumerate}
\item Find the probability that she reads neither Hindi nor English newspapers.
\item If she reads Hindi newspaper, find the probability that she reads English newspaper.
\item If she reads English newspaper, find the probability that she reads Hindi newspaper.\\
\end{enumerate}
\item The probability of obtaining an even prime number on each die, when a pair of dice is rolled is 
\begin{enumerate}
    \item $0$ 
    
    \item $\frac{1}{3}$ 
    
    \item $\frac{1}{12}$ 
    
    \item $\frac{1}{36}$ 
\end{enumerate}
\solution
		%\input{ncert/12/13/2/17/defs.tex}
	\item A bag contains 4 red and 4 black balls, another bag contains 2 red and 6 black balls. One of the two bags is selected at random and a ball is drawn from the bag which is found to be red. Find the probability that the ball is drawn from the first bag.
\\
\solution
		%\input{ncert/12/13/3/2/main.tex}
  \item
  Cards with numbers 2 to 101 are placed in a box. A card is selected at random.Find the probability that the card has
\begin{enumerate}[label=(\roman*)]
	\item an even number 
	\item a square number
\end{enumerate}
\solution
%\input{exemplar/10/13/3/32/main.tex}
\item
The king, queen and jack of clubs are removed from a deck of 52 playing cards and then well shuffled. Now one card is drawn at random from the remaining cards.  Determine the probability that the card is
\begin{enumerate}[label=(\roman*)]
\item a club
\item 10 of hearts
\end{enumerate}
\solution
%\input{exemplar/10/13/3/29/main.tex}
\item A team of medical students doing their internship have to assist during surgeries
at a city hospital. The probabilities of surgeries rated as very complex, complex,
routine, simple or very simple are respectively, 0.15, 0.20, 0.31, 0.26, .08. Find
the probabilities that a particular surgery will be rated
\begin{enumerate}
	\item complex or very complex;
	\item neither very complex nor very simple;
	\item routine or complex
	\item routine or simple
\end{enumerate}
\solution
%\input{exemplar/11/16/3/8(1)/main.tex}
\item A card is selected from a pack of 52 cards.
\begin{enumerate}[label=(\alph*)]
    \item How many points are there in the sample space?
    \item Calculate the probability that the card is an ace of spades.
    \item Calculate the probability that the card is (i) an ace and (ii) black card.
\end{enumerate}
\solution
%\input{exemplar/11/16/3/4/main2.tex}
\item The probability that a non leap year selected at random will contain 53 sundays.
\\
\solution
%\input{exemplar/10/13/1/19/main.tex}
\item One of the four persons John, Rita, Aslam or Gurpreet will be promoted next
month. Consequently the sample space consists of four elementary outcomes
S = {John promoted, Rita promoted, Aslam promoted, Gurpreet promoted}
You are told that the chances of John’s promotion is same as that of Gurpreet,
Rita’s chances of promotion are twice as likely as Johns. Aslam’s chances are
four times that of John.
\begin{enumerate}
	\item Determine
	\begin{enumerate}
		\item P (John promoted)
		\item P (Rita promoted)
		\item P (Aslam promoted)
		\item P (Gurpreet promoted)
	\end{enumerate}
	\item If A = {John promoted or Gurpreet promoted}, find P (A).
\end{enumerate}
\solution
%\input{exemplar/11/16/3/10/main.tex}
\item A card is drawn from a deck of 52 cards. Find the probability of getting a king or a heart or a red card.\\
\solution
%\input{exemplar/11/16/3/15/main.tex}
\item The probability that a student will pass his examination is 0.73, the probability of
the student getting a compartment is 0.13, and the probability that the student will
either pass or get compartment is 0.96. State True or False.\\
\solution
%\input{exemplar/11/16/3/31/main.tex}
\item A card is selected from a pack of 52 cards\\
\begin{enumerate}[label=(\alph*)]
\item How many points are there in the sample space?
\item Calculate the probability that the cards is an ace of spades.
\item Calculate the probability that the card is (i) an ace (ii)black card.\\
\end{enumerate}
%\input{ncert/11/16/3/4_1/Prob_4.tex}
\item In a non-leap year, the probability of having 53 tuesdays or 53 wednesdays is\\
\solution
%\input{exemplar/11/16/3/18/main.tex}
\item There are 1000 sealed envelopes in a box, 10 of them contain a cash prize of
Rs 100 each, 100 of them contain a cash prize of Rs 50 each and 200 of them
contain a cash prize of Rs 10 each and rest do not contain any cash prize. If they
are well shuffled and an envelope is picked up out, what is the probability that it
contains no cash prize?\\
\solution
%\input{exemplar/10/13/3/34/main.tex}
\item 
A die is thrown and a card is selected at random from a deck of 52 playing cards. The probability of getting an even number on the die and a spade card.\\
\solution
%\input{exemplar/12/13/3/78/main.tex}
\item
If 4-digit numbers greater than 5,000 are randomly formed from the digits 0, 1, 3, 5, and 7, what is the probability of forming a number divisible by 5 when:
\begin{enumerate}
    \item The digits are repeated?
    \item The repetition of digits is not allowed?
\end{enumerate}
\solution
%\input{ncert/11/16/4/9/main.tex}
\item Consider the probability space $\brak{\Omega, \mathcal{G}, P}$ where $\Omega = [0,2]$ and $\mathcal{G} = \cbrak{\phi, \Omega, [0,1], (1,2]}$. Let $X$ and $Y$ be two functions on $\Omega$ defined as
\begin{align*}
    X(\omega) = 
    \begin{cases}
        1 & \text{if }\omega \in [0, 1]\\
        2 & \text{if }\omega \in (1, 2]
    \end{cases}
\end{align*}
and
\begin{align*}
    Y(\omega) = 
    \begin{cases}
        2 & \text{if }\omega \in [0, 1.5]\\
        3 & \text{if }\omega \in (1.5, 2].
    \end{cases}
\end{align*}
Then which one of the following statements is true?
\begin{enumerate}
    \item [(A)] $X$ is a random variable with respect to $\mathcal{G}$, but $Y$ is not a random variable with respect to $\mathcal{G}$.
    \item [(B)] $Y$ is a random variable with respect to $\mathcal{G}$, but $X$ is not a random variable with respect to $\mathcal{G}$.
    \item [(C)] Neither $X$ nor $Y$ is a random variable with respect to $\mathcal{G}$.
    \item [(D)] Both $X$ and $Y$ are random variables with respect to $\mathcal{G}$.
\end{enumerate} \hfill (GATE ST 2023)\\
\solution
%\input{gate/ST/2023/14/main.tex}
	\item  A die is loaded in such a way that each odd number is twice as likely to occur as
each even number. Find $P(G)$, where $G$ is the event that a number greater than
3 occurs on a single roll of the die.
\\
\solution
		%\input{exemplar/11/16/3/5/main.tex}
	\item All the jacks, queens and kings are removed from a deck of 52 playing cards. The remaining cards are well shuffled and then one card is drawn at random. Giving ace a value 1 similar value for other cards, find the probability that the card has a value 
		\begin{enumerate}
			\item 7
			\item greater than 7
			\item less than 7
		\end{enumerate}
		%\input{exemplar/10/13/3/30/main.tex}
  \item A Lot consists of 48 mobile phones of which 42 are good, 3 have only minor defects and 3 have major defects.Varnika will buy a phone if it is good but the trader will only buy a mobile if it has no major defects. One phone is selected at random from the lot. What is the probability that it is
\begin{enumerate}
	\item acceptable to Varnika?
            \item acceptable to the trader?
\end{enumerate}
\solution
	%\input{exemplar/10/13/3/40/main.tex}
 \item A student says that if you throw a die, it will show up 1 or not 1. Therefore, the probability of getting 1 and the probability of getting 'not 1' each is equal to $\frac{1}{2}$. Is this correct? Give reasons.\\
 \solution
        %\input{exemplar/10/13/2/9/main.tex}
   \item Four candidates A, B, C, D have ap-
plied for the assignment to coach a school cricket
team. If A is twice as likely to be selected as B, and
B and C are given about the same chance of being
selected, while C is twice as likely to be selected
as D, what are the probabilities that
\begin{enumerate}
\item C will be selected?
\item A will not be selected?
\end{enumerate}
	%\input{exemplar/11/16/3/9/main.tex}
 \item A bag contain 24 balls of which $x$ balls are red, $2x$ are white and $3x$ are blue. A ball is selected at random, What is the probability that it is
\begin{enumerate}[label=\alph*)]
\item not red ?
\item white ?
\end{enumerate}
%\input{exemplar/10/13/3/41/main.tex}
If the letters of the word ASSASSINATION are arranged at random. Find the Probability that
\begin{enumerate}[label=(\alph*)]
\item Four $S's$ come consecutively in the word
\item Two  $I's$ and two $N's$ come together
\item All $A's$ are not coming together
\item No two $A's$ are coming together
\end{enumerate}
%\input{exemplar/11/16/3/14/main.tex}
	\item One urn contains two black balls (labelled B1 and B2) and one white ball. A
	second urn contains one black ball and two white balls (labelled W1 and W2).
	Suppose the following experiment is performed. One of the two urns is chosen
	at random. Next a ball is randomly chosen from the urn. Then a second ball is
	chosen at random from the same urn without replacing the first ball.
	
	\begin{enumerate}
	\item What is the probability that two black balls are chosen?
	
	\item What is the probability that two balls of opposite colour are chosen?
	\end{enumerate}
	\solution
	%\input{exemplar/11/16/3/12/main1.tex}
\end{enumerate}

	\item A bag contains 4 red and 4 black balls, another bag contains 2 red and 6 black balls. One of the two bags is selected at random and a ball is drawn from the bag which is found to be red. Find the probability that the ball is drawn from the first bag.
\\
\solution
		%\begin{table}[H]
	\centering
\begin{tabular}{|c|c|c|}
\hline
Random variable &Value &Definition\\ \hline
\multirow{3}{*}{X} &0 &Slips of Rs 1\\
&1 &Slips of Rs 5\\
&2 &Slips of Rs 13\\ \hline
\multirow{2}{*}{Y} &0 &Box A\\
&1 &Box B\\\hline
\end{tabular}
\caption{}
\label{tab:Distribution}
\end{table}
See \tabref{tab:Distribution}.
\begin{align}
p_{Y}\brak{k}= \begin{cases} 
      \frac{1}{3} & {k=0} \\
      \frac{2}{3 }& {k=1} 
   \end{cases}
   \\
p_{Y|X}\brak{0|0} = \frac{19}{25}\, 
p_{Y|X}\brak{0|1} = \frac{6}{25}\,
p_{Y|X}\brak{1|0} = \frac{45}{50}\,
p_{Y|X}\brak{1|2} = \frac{5}{50}
\end{align}
The desired probability is the probability that a slip drawn at random is marked other than Rs 1,
\begin{align}
&=1-p_X\brak{0}\\
&= p_X(1) + p_X(2)
\end{align}
Using Bayes theorem,
\begin{align}
&= p_Y\brak{0} \times \pr{Y=0 | X=1} + p_Y\brak{1} \times \pr{Y=1|X=2}\\
&=\frac{1}{3} \times \frac{6}{25} + \frac{2}{3} \times \frac{5}{50}\\
&=\frac{11}{75}
\end{align}

\newpage

%\tableofcontents

\bigskip

\renewcommand{\thefigure}{\theenumi}
\renewcommand{\thetable}{\theenumi}
%\renewcommand{\theequation}{\theenumi}

%\begin{abstract}
%%\boldmath
%In this letter, an algorithm for evaluating the exact analytical bit error rate  (BER)  for the piecewise linear (PL) combiner for  multiple relays is presented. Previous results were available only for upto three relays. The algorithm is unique in the sense that  the actual mathematical expressions, that are prohibitively large, need not be explicitly obtained. The diversity gain due to multiple relays is shown through plots of the analytical BER, well supported by simulations. 
%
%\end{abstract}
% IEEEtran.cls defaults to using nonbold math in the Abstract.
% This preserves the distinction between vectors and scalars. However,
% if the journal you are submitting to favors bold math in the abstract,
% then you can use LaTeX's standard command \boldmath at the very start
% of the abstract to achieve this. Many IEEE journals frown on math
% in the abstract anyway.

% Note that keywords are not normally used for peerreview papers.
%\begin{IEEEkeywords}
%Cooperative diversity, decode and forward, piecewise linear
%\end{IEEEkeywords}



% For peer review papers, you can put extra information on the cover
% page as needed:
% \ifCLASSOPTIONpeerreview
% \begin{center} \bfseries EDICS Category: 3-BBND \end{center}
% \fi
%
% For peerreview papers, this IEEEtran command inserts a page break and
% creates the second title. It will be ignored for other modes.
%\IEEEpeerreviewmaketitle




  \item
  Cards with numbers 2 to 101 are placed in a box. A card is selected at random.Find the probability that the card has
\begin{enumerate}[label=(\roman*)]
	\item an even number 
	\item a square number
\end{enumerate}
\solution
%\begin{table}[H]
	\centering
\begin{tabular}{|c|c|c|}
\hline
Random variable &Value &Definition\\ \hline
\multirow{3}{*}{X} &0 &Slips of Rs 1\\
&1 &Slips of Rs 5\\
&2 &Slips of Rs 13\\ \hline
\multirow{2}{*}{Y} &0 &Box A\\
&1 &Box B\\\hline
\end{tabular}
\caption{}
\label{tab:Distribution}
\end{table}
See \tabref{tab:Distribution}.
\begin{align}
p_{Y}\brak{k}= \begin{cases} 
      \frac{1}{3} & {k=0} \\
      \frac{2}{3 }& {k=1} 
   \end{cases}
   \\
p_{Y|X}\brak{0|0} = \frac{19}{25}\, 
p_{Y|X}\brak{0|1} = \frac{6}{25}\,
p_{Y|X}\brak{1|0} = \frac{45}{50}\,
p_{Y|X}\brak{1|2} = \frac{5}{50}
\end{align}
The desired probability is the probability that a slip drawn at random is marked other than Rs 1,
\begin{align}
&=1-p_X\brak{0}\\
&= p_X(1) + p_X(2)
\end{align}
Using Bayes theorem,
\begin{align}
&= p_Y\brak{0} \times \pr{Y=0 | X=1} + p_Y\brak{1} \times \pr{Y=1|X=2}\\
&=\frac{1}{3} \times \frac{6}{25} + \frac{2}{3} \times \frac{5}{50}\\
&=\frac{11}{75}
\end{align}

\newpage

%\tableofcontents

\bigskip

\renewcommand{\thefigure}{\theenumi}
\renewcommand{\thetable}{\theenumi}
%\renewcommand{\theequation}{\theenumi}

%\begin{abstract}
%%\boldmath
%In this letter, an algorithm for evaluating the exact analytical bit error rate  (BER)  for the piecewise linear (PL) combiner for  multiple relays is presented. Previous results were available only for upto three relays. The algorithm is unique in the sense that  the actual mathematical expressions, that are prohibitively large, need not be explicitly obtained. The diversity gain due to multiple relays is shown through plots of the analytical BER, well supported by simulations. 
%
%\end{abstract}
% IEEEtran.cls defaults to using nonbold math in the Abstract.
% This preserves the distinction between vectors and scalars. However,
% if the journal you are submitting to favors bold math in the abstract,
% then you can use LaTeX's standard command \boldmath at the very start
% of the abstract to achieve this. Many IEEE journals frown on math
% in the abstract anyway.

% Note that keywords are not normally used for peerreview papers.
%\begin{IEEEkeywords}
%Cooperative diversity, decode and forward, piecewise linear
%\end{IEEEkeywords}



% For peer review papers, you can put extra information on the cover
% page as needed:
% \ifCLASSOPTIONpeerreview
% \begin{center} \bfseries EDICS Category: 3-BBND \end{center}
% \fi
%
% For peerreview papers, this IEEEtran command inserts a page break and
% creates the second title. It will be ignored for other modes.
%\IEEEpeerreviewmaketitle




\item
The king, queen and jack of clubs are removed from a deck of 52 playing cards and then well shuffled. Now one card is drawn at random from the remaining cards.  Determine the probability that the card is
\begin{enumerate}[label=(\roman*)]
\item a club
\item 10 of hearts
\end{enumerate}
\solution
%\begin{table}[H]
	\centering
\begin{tabular}{|c|c|c|}
\hline
Random variable &Value &Definition\\ \hline
\multirow{3}{*}{X} &0 &Slips of Rs 1\\
&1 &Slips of Rs 5\\
&2 &Slips of Rs 13\\ \hline
\multirow{2}{*}{Y} &0 &Box A\\
&1 &Box B\\\hline
\end{tabular}
\caption{}
\label{tab:Distribution}
\end{table}
See \tabref{tab:Distribution}.
\begin{align}
p_{Y}\brak{k}= \begin{cases} 
      \frac{1}{3} & {k=0} \\
      \frac{2}{3 }& {k=1} 
   \end{cases}
   \\
p_{Y|X}\brak{0|0} = \frac{19}{25}\, 
p_{Y|X}\brak{0|1} = \frac{6}{25}\,
p_{Y|X}\brak{1|0} = \frac{45}{50}\,
p_{Y|X}\brak{1|2} = \frac{5}{50}
\end{align}
The desired probability is the probability that a slip drawn at random is marked other than Rs 1,
\begin{align}
&=1-p_X\brak{0}\\
&= p_X(1) + p_X(2)
\end{align}
Using Bayes theorem,
\begin{align}
&= p_Y\brak{0} \times \pr{Y=0 | X=1} + p_Y\brak{1} \times \pr{Y=1|X=2}\\
&=\frac{1}{3} \times \frac{6}{25} + \frac{2}{3} \times \frac{5}{50}\\
&=\frac{11}{75}
\end{align}

\newpage

%\tableofcontents

\bigskip

\renewcommand{\thefigure}{\theenumi}
\renewcommand{\thetable}{\theenumi}
%\renewcommand{\theequation}{\theenumi}

%\begin{abstract}
%%\boldmath
%In this letter, an algorithm for evaluating the exact analytical bit error rate  (BER)  for the piecewise linear (PL) combiner for  multiple relays is presented. Previous results were available only for upto three relays. The algorithm is unique in the sense that  the actual mathematical expressions, that are prohibitively large, need not be explicitly obtained. The diversity gain due to multiple relays is shown through plots of the analytical BER, well supported by simulations. 
%
%\end{abstract}
% IEEEtran.cls defaults to using nonbold math in the Abstract.
% This preserves the distinction between vectors and scalars. However,
% if the journal you are submitting to favors bold math in the abstract,
% then you can use LaTeX's standard command \boldmath at the very start
% of the abstract to achieve this. Many IEEE journals frown on math
% in the abstract anyway.

% Note that keywords are not normally used for peerreview papers.
%\begin{IEEEkeywords}
%Cooperative diversity, decode and forward, piecewise linear
%\end{IEEEkeywords}



% For peer review papers, you can put extra information on the cover
% page as needed:
% \ifCLASSOPTIONpeerreview
% \begin{center} \bfseries EDICS Category: 3-BBND \end{center}
% \fi
%
% For peerreview papers, this IEEEtran command inserts a page break and
% creates the second title. It will be ignored for other modes.
%\IEEEpeerreviewmaketitle




\item A team of medical students doing their internship have to assist during surgeries
at a city hospital. The probabilities of surgeries rated as very complex, complex,
routine, simple or very simple are respectively, 0.15, 0.20, 0.31, 0.26, .08. Find
the probabilities that a particular surgery will be rated
\begin{enumerate}
	\item complex or very complex;
	\item neither very complex nor very simple;
	\item routine or complex
	\item routine or simple
\end{enumerate}
\solution
%\begin{table}[H]
	\centering
\begin{tabular}{|c|c|c|}
\hline
Random variable &Value &Definition\\ \hline
\multirow{3}{*}{X} &0 &Slips of Rs 1\\
&1 &Slips of Rs 5\\
&2 &Slips of Rs 13\\ \hline
\multirow{2}{*}{Y} &0 &Box A\\
&1 &Box B\\\hline
\end{tabular}
\caption{}
\label{tab:Distribution}
\end{table}
See \tabref{tab:Distribution}.
\begin{align}
p_{Y}\brak{k}= \begin{cases} 
      \frac{1}{3} & {k=0} \\
      \frac{2}{3 }& {k=1} 
   \end{cases}
   \\
p_{Y|X}\brak{0|0} = \frac{19}{25}\, 
p_{Y|X}\brak{0|1} = \frac{6}{25}\,
p_{Y|X}\brak{1|0} = \frac{45}{50}\,
p_{Y|X}\brak{1|2} = \frac{5}{50}
\end{align}
The desired probability is the probability that a slip drawn at random is marked other than Rs 1,
\begin{align}
&=1-p_X\brak{0}\\
&= p_X(1) + p_X(2)
\end{align}
Using Bayes theorem,
\begin{align}
&= p_Y\brak{0} \times \pr{Y=0 | X=1} + p_Y\brak{1} \times \pr{Y=1|X=2}\\
&=\frac{1}{3} \times \frac{6}{25} + \frac{2}{3} \times \frac{5}{50}\\
&=\frac{11}{75}
\end{align}

\newpage

%\tableofcontents

\bigskip

\renewcommand{\thefigure}{\theenumi}
\renewcommand{\thetable}{\theenumi}
%\renewcommand{\theequation}{\theenumi}

%\begin{abstract}
%%\boldmath
%In this letter, an algorithm for evaluating the exact analytical bit error rate  (BER)  for the piecewise linear (PL) combiner for  multiple relays is presented. Previous results were available only for upto three relays. The algorithm is unique in the sense that  the actual mathematical expressions, that are prohibitively large, need not be explicitly obtained. The diversity gain due to multiple relays is shown through plots of the analytical BER, well supported by simulations. 
%
%\end{abstract}
% IEEEtran.cls defaults to using nonbold math in the Abstract.
% This preserves the distinction between vectors and scalars. However,
% if the journal you are submitting to favors bold math in the abstract,
% then you can use LaTeX's standard command \boldmath at the very start
% of the abstract to achieve this. Many IEEE journals frown on math
% in the abstract anyway.

% Note that keywords are not normally used for peerreview papers.
%\begin{IEEEkeywords}
%Cooperative diversity, decode and forward, piecewise linear
%\end{IEEEkeywords}



% For peer review papers, you can put extra information on the cover
% page as needed:
% \ifCLASSOPTIONpeerreview
% \begin{center} \bfseries EDICS Category: 3-BBND \end{center}
% \fi
%
% For peerreview papers, this IEEEtran command inserts a page break and
% creates the second title. It will be ignored for other modes.
%\IEEEpeerreviewmaketitle




\item A card is selected from a pack of 52 cards.
\begin{enumerate}[label=(\alph*)]
    \item How many points are there in the sample space?
    \item Calculate the probability that the card is an ace of spades.
    \item Calculate the probability that the card is (i) an ace and (ii) black card.
\end{enumerate}
\solution
%Let $X$ be an bernoulli rv defined as in \tabref{tab:exemplar/11/16/3/26}.  Then, 
\begin{equation}
    p =
        \frac{4}{11} 
\end{equation}
\begin{table}[H]
	\centering
	\input{exemplar/11/16/3/26/tables/Table2.tex}
	\caption{}
        \label{tab:exemplar/11/16/3/26}
\end{table}

\item The probability that a non leap year selected at random will contain 53 sundays.
\\
\solution
%\begin{table}[H]
	\centering
\begin{tabular}{|c|c|c|}
\hline
Random variable &Value &Definition\\ \hline
\multirow{3}{*}{X} &0 &Slips of Rs 1\\
&1 &Slips of Rs 5\\
&2 &Slips of Rs 13\\ \hline
\multirow{2}{*}{Y} &0 &Box A\\
&1 &Box B\\\hline
\end{tabular}
\caption{}
\label{tab:Distribution}
\end{table}
See \tabref{tab:Distribution}.
\begin{align}
p_{Y}\brak{k}= \begin{cases} 
      \frac{1}{3} & {k=0} \\
      \frac{2}{3 }& {k=1} 
   \end{cases}
   \\
p_{Y|X}\brak{0|0} = \frac{19}{25}\, 
p_{Y|X}\brak{0|1} = \frac{6}{25}\,
p_{Y|X}\brak{1|0} = \frac{45}{50}\,
p_{Y|X}\brak{1|2} = \frac{5}{50}
\end{align}
The desired probability is the probability that a slip drawn at random is marked other than Rs 1,
\begin{align}
&=1-p_X\brak{0}\\
&= p_X(1) + p_X(2)
\end{align}
Using Bayes theorem,
\begin{align}
&= p_Y\brak{0} \times \pr{Y=0 | X=1} + p_Y\brak{1} \times \pr{Y=1|X=2}\\
&=\frac{1}{3} \times \frac{6}{25} + \frac{2}{3} \times \frac{5}{50}\\
&=\frac{11}{75}
\end{align}

\newpage

%\tableofcontents

\bigskip

\renewcommand{\thefigure}{\theenumi}
\renewcommand{\thetable}{\theenumi}
%\renewcommand{\theequation}{\theenumi}

%\begin{abstract}
%%\boldmath
%In this letter, an algorithm for evaluating the exact analytical bit error rate  (BER)  for the piecewise linear (PL) combiner for  multiple relays is presented. Previous results were available only for upto three relays. The algorithm is unique in the sense that  the actual mathematical expressions, that are prohibitively large, need not be explicitly obtained. The diversity gain due to multiple relays is shown through plots of the analytical BER, well supported by simulations. 
%
%\end{abstract}
% IEEEtran.cls defaults to using nonbold math in the Abstract.
% This preserves the distinction between vectors and scalars. However,
% if the journal you are submitting to favors bold math in the abstract,
% then you can use LaTeX's standard command \boldmath at the very start
% of the abstract to achieve this. Many IEEE journals frown on math
% in the abstract anyway.

% Note that keywords are not normally used for peerreview papers.
%\begin{IEEEkeywords}
%Cooperative diversity, decode and forward, piecewise linear
%\end{IEEEkeywords}



% For peer review papers, you can put extra information on the cover
% page as needed:
% \ifCLASSOPTIONpeerreview
% \begin{center} \bfseries EDICS Category: 3-BBND \end{center}
% \fi
%
% For peerreview papers, this IEEEtran command inserts a page break and
% creates the second title. It will be ignored for other modes.
%\IEEEpeerreviewmaketitle




\item One of the four persons John, Rita, Aslam or Gurpreet will be promoted next
month. Consequently the sample space consists of four elementary outcomes
S = {John promoted, Rita promoted, Aslam promoted, Gurpreet promoted}
You are told that the chances of John’s promotion is same as that of Gurpreet,
Rita’s chances of promotion are twice as likely as Johns. Aslam’s chances are
four times that of John.
\begin{enumerate}
	\item Determine
	\begin{enumerate}
		\item P (John promoted)
		\item P (Rita promoted)
		\item P (Aslam promoted)
		\item P (Gurpreet promoted)
	\end{enumerate}
	\item If A = {John promoted or Gurpreet promoted}, find P (A).
\end{enumerate}
\solution
%\begin{table}[H]
	\centering
\begin{tabular}{|c|c|c|}
\hline
Random variable &Value &Definition\\ \hline
\multirow{3}{*}{X} &0 &Slips of Rs 1\\
&1 &Slips of Rs 5\\
&2 &Slips of Rs 13\\ \hline
\multirow{2}{*}{Y} &0 &Box A\\
&1 &Box B\\\hline
\end{tabular}
\caption{}
\label{tab:Distribution}
\end{table}
See \tabref{tab:Distribution}.
\begin{align}
p_{Y}\brak{k}= \begin{cases} 
      \frac{1}{3} & {k=0} \\
      \frac{2}{3 }& {k=1} 
   \end{cases}
   \\
p_{Y|X}\brak{0|0} = \frac{19}{25}\, 
p_{Y|X}\brak{0|1} = \frac{6}{25}\,
p_{Y|X}\brak{1|0} = \frac{45}{50}\,
p_{Y|X}\brak{1|2} = \frac{5}{50}
\end{align}
The desired probability is the probability that a slip drawn at random is marked other than Rs 1,
\begin{align}
&=1-p_X\brak{0}\\
&= p_X(1) + p_X(2)
\end{align}
Using Bayes theorem,
\begin{align}
&= p_Y\brak{0} \times \pr{Y=0 | X=1} + p_Y\brak{1} \times \pr{Y=1|X=2}\\
&=\frac{1}{3} \times \frac{6}{25} + \frac{2}{3} \times \frac{5}{50}\\
&=\frac{11}{75}
\end{align}

\newpage

%\tableofcontents

\bigskip

\renewcommand{\thefigure}{\theenumi}
\renewcommand{\thetable}{\theenumi}
%\renewcommand{\theequation}{\theenumi}

%\begin{abstract}
%%\boldmath
%In this letter, an algorithm for evaluating the exact analytical bit error rate  (BER)  for the piecewise linear (PL) combiner for  multiple relays is presented. Previous results were available only for upto three relays. The algorithm is unique in the sense that  the actual mathematical expressions, that are prohibitively large, need not be explicitly obtained. The diversity gain due to multiple relays is shown through plots of the analytical BER, well supported by simulations. 
%
%\end{abstract}
% IEEEtran.cls defaults to using nonbold math in the Abstract.
% This preserves the distinction between vectors and scalars. However,
% if the journal you are submitting to favors bold math in the abstract,
% then you can use LaTeX's standard command \boldmath at the very start
% of the abstract to achieve this. Many IEEE journals frown on math
% in the abstract anyway.

% Note that keywords are not normally used for peerreview papers.
%\begin{IEEEkeywords}
%Cooperative diversity, decode and forward, piecewise linear
%\end{IEEEkeywords}



% For peer review papers, you can put extra information on the cover
% page as needed:
% \ifCLASSOPTIONpeerreview
% \begin{center} \bfseries EDICS Category: 3-BBND \end{center}
% \fi
%
% For peerreview papers, this IEEEtran command inserts a page break and
% creates the second title. It will be ignored for other modes.
%\IEEEpeerreviewmaketitle




\item A card is drawn from a deck of 52 cards. Find the probability of getting a king or a heart or a red card.\\
\solution
%\begin{table}[H]
	\centering
\begin{tabular}{|c|c|c|}
\hline
Random variable &Value &Definition\\ \hline
\multirow{3}{*}{X} &0 &Slips of Rs 1\\
&1 &Slips of Rs 5\\
&2 &Slips of Rs 13\\ \hline
\multirow{2}{*}{Y} &0 &Box A\\
&1 &Box B\\\hline
\end{tabular}
\caption{}
\label{tab:Distribution}
\end{table}
See \tabref{tab:Distribution}.
\begin{align}
p_{Y}\brak{k}= \begin{cases} 
      \frac{1}{3} & {k=0} \\
      \frac{2}{3 }& {k=1} 
   \end{cases}
   \\
p_{Y|X}\brak{0|0} = \frac{19}{25}\, 
p_{Y|X}\brak{0|1} = \frac{6}{25}\,
p_{Y|X}\brak{1|0} = \frac{45}{50}\,
p_{Y|X}\brak{1|2} = \frac{5}{50}
\end{align}
The desired probability is the probability that a slip drawn at random is marked other than Rs 1,
\begin{align}
&=1-p_X\brak{0}\\
&= p_X(1) + p_X(2)
\end{align}
Using Bayes theorem,
\begin{align}
&= p_Y\brak{0} \times \pr{Y=0 | X=1} + p_Y\brak{1} \times \pr{Y=1|X=2}\\
&=\frac{1}{3} \times \frac{6}{25} + \frac{2}{3} \times \frac{5}{50}\\
&=\frac{11}{75}
\end{align}

\newpage

%\tableofcontents

\bigskip

\renewcommand{\thefigure}{\theenumi}
\renewcommand{\thetable}{\theenumi}
%\renewcommand{\theequation}{\theenumi}

%\begin{abstract}
%%\boldmath
%In this letter, an algorithm for evaluating the exact analytical bit error rate  (BER)  for the piecewise linear (PL) combiner for  multiple relays is presented. Previous results were available only for upto three relays. The algorithm is unique in the sense that  the actual mathematical expressions, that are prohibitively large, need not be explicitly obtained. The diversity gain due to multiple relays is shown through plots of the analytical BER, well supported by simulations. 
%
%\end{abstract}
% IEEEtran.cls defaults to using nonbold math in the Abstract.
% This preserves the distinction between vectors and scalars. However,
% if the journal you are submitting to favors bold math in the abstract,
% then you can use LaTeX's standard command \boldmath at the very start
% of the abstract to achieve this. Many IEEE journals frown on math
% in the abstract anyway.

% Note that keywords are not normally used for peerreview papers.
%\begin{IEEEkeywords}
%Cooperative diversity, decode and forward, piecewise linear
%\end{IEEEkeywords}



% For peer review papers, you can put extra information on the cover
% page as needed:
% \ifCLASSOPTIONpeerreview
% \begin{center} \bfseries EDICS Category: 3-BBND \end{center}
% \fi
%
% For peerreview papers, this IEEEtran command inserts a page break and
% creates the second title. It will be ignored for other modes.
%\IEEEpeerreviewmaketitle




\item The probability that a student will pass his examination is 0.73, the probability of
the student getting a compartment is 0.13, and the probability that the student will
either pass or get compartment is 0.96. State True or False.\\
\solution
%\begin{table}[H]
	\centering
\begin{tabular}{|c|c|c|}
\hline
Random variable &Value &Definition\\ \hline
\multirow{3}{*}{X} &0 &Slips of Rs 1\\
&1 &Slips of Rs 5\\
&2 &Slips of Rs 13\\ \hline
\multirow{2}{*}{Y} &0 &Box A\\
&1 &Box B\\\hline
\end{tabular}
\caption{}
\label{tab:Distribution}
\end{table}
See \tabref{tab:Distribution}.
\begin{align}
p_{Y}\brak{k}= \begin{cases} 
      \frac{1}{3} & {k=0} \\
      \frac{2}{3 }& {k=1} 
   \end{cases}
   \\
p_{Y|X}\brak{0|0} = \frac{19}{25}\, 
p_{Y|X}\brak{0|1} = \frac{6}{25}\,
p_{Y|X}\brak{1|0} = \frac{45}{50}\,
p_{Y|X}\brak{1|2} = \frac{5}{50}
\end{align}
The desired probability is the probability that a slip drawn at random is marked other than Rs 1,
\begin{align}
&=1-p_X\brak{0}\\
&= p_X(1) + p_X(2)
\end{align}
Using Bayes theorem,
\begin{align}
&= p_Y\brak{0} \times \pr{Y=0 | X=1} + p_Y\brak{1} \times \pr{Y=1|X=2}\\
&=\frac{1}{3} \times \frac{6}{25} + \frac{2}{3} \times \frac{5}{50}\\
&=\frac{11}{75}
\end{align}

\newpage

%\tableofcontents

\bigskip

\renewcommand{\thefigure}{\theenumi}
\renewcommand{\thetable}{\theenumi}
%\renewcommand{\theequation}{\theenumi}

%\begin{abstract}
%%\boldmath
%In this letter, an algorithm for evaluating the exact analytical bit error rate  (BER)  for the piecewise linear (PL) combiner for  multiple relays is presented. Previous results were available only for upto three relays. The algorithm is unique in the sense that  the actual mathematical expressions, that are prohibitively large, need not be explicitly obtained. The diversity gain due to multiple relays is shown through plots of the analytical BER, well supported by simulations. 
%
%\end{abstract}
% IEEEtran.cls defaults to using nonbold math in the Abstract.
% This preserves the distinction between vectors and scalars. However,
% if the journal you are submitting to favors bold math in the abstract,
% then you can use LaTeX's standard command \boldmath at the very start
% of the abstract to achieve this. Many IEEE journals frown on math
% in the abstract anyway.

% Note that keywords are not normally used for peerreview papers.
%\begin{IEEEkeywords}
%Cooperative diversity, decode and forward, piecewise linear
%\end{IEEEkeywords}



% For peer review papers, you can put extra information on the cover
% page as needed:
% \ifCLASSOPTIONpeerreview
% \begin{center} \bfseries EDICS Category: 3-BBND \end{center}
% \fi
%
% For peerreview papers, this IEEEtran command inserts a page break and
% creates the second title. It will be ignored for other modes.
%\IEEEpeerreviewmaketitle




\item A card is selected from a pack of 52 cards\\
\begin{enumerate}[label=(\alph*)]
\item How many points are there in the sample space?
\item Calculate the probability that the cards is an ace of spades.
\item Calculate the probability that the card is (i) an ace (ii)black card.\\
\end{enumerate}
%\input{ncert/11/16/3/4_1/Prob_4.tex}
\item In a non-leap year, the probability of having 53 tuesdays or 53 wednesdays is\\
\solution
%A non-leap year has a total of 365 days, and a week has 7 days.\\
So it can be expressed as 
\begin{align}
365\text{days} &=52\times 7+1 \text{day}
\end{align}
$\implies$ 52 tuesdays or wednesdays\\
Random variable X denotes the days of a week
\begin{align}
p_X\brak{k}&=\frac{1}{7}; \quad \brak{1<k<7}
\end{align}
So the probability of extra day being tuesday or wednesday is
\begin{align}
p_X\brak{3}+p_X\brak{4}&=\frac{1}{7}+\frac{1}{7}=\frac{2}{7}
\end{align}



\item There are 1000 sealed envelopes in a box, 10 of them contain a cash prize of
Rs 100 each, 100 of them contain a cash prize of Rs 50 each and 200 of them
contain a cash prize of Rs 10 each and rest do not contain any cash prize. If they
are well shuffled and an envelope is picked up out, what is the probability that it
contains no cash prize?\\
\solution
%\begin{table}[H]
	\centering
\begin{tabular}{|c|c|c|}
\hline
Random variable &Value &Definition\\ \hline
\multirow{3}{*}{X} &0 &Slips of Rs 1\\
&1 &Slips of Rs 5\\
&2 &Slips of Rs 13\\ \hline
\multirow{2}{*}{Y} &0 &Box A\\
&1 &Box B\\\hline
\end{tabular}
\caption{}
\label{tab:Distribution}
\end{table}
See \tabref{tab:Distribution}.
\begin{align}
p_{Y}\brak{k}= \begin{cases} 
      \frac{1}{3} & {k=0} \\
      \frac{2}{3 }& {k=1} 
   \end{cases}
   \\
p_{Y|X}\brak{0|0} = \frac{19}{25}\, 
p_{Y|X}\brak{0|1} = \frac{6}{25}\,
p_{Y|X}\brak{1|0} = \frac{45}{50}\,
p_{Y|X}\brak{1|2} = \frac{5}{50}
\end{align}
The desired probability is the probability that a slip drawn at random is marked other than Rs 1,
\begin{align}
&=1-p_X\brak{0}\\
&= p_X(1) + p_X(2)
\end{align}
Using Bayes theorem,
\begin{align}
&= p_Y\brak{0} \times \pr{Y=0 | X=1} + p_Y\brak{1} \times \pr{Y=1|X=2}\\
&=\frac{1}{3} \times \frac{6}{25} + \frac{2}{3} \times \frac{5}{50}\\
&=\frac{11}{75}
\end{align}

\newpage

%\tableofcontents

\bigskip

\renewcommand{\thefigure}{\theenumi}
\renewcommand{\thetable}{\theenumi}
%\renewcommand{\theequation}{\theenumi}

%\begin{abstract}
%%\boldmath
%In this letter, an algorithm for evaluating the exact analytical bit error rate  (BER)  for the piecewise linear (PL) combiner for  multiple relays is presented. Previous results were available only for upto three relays. The algorithm is unique in the sense that  the actual mathematical expressions, that are prohibitively large, need not be explicitly obtained. The diversity gain due to multiple relays is shown through plots of the analytical BER, well supported by simulations. 
%
%\end{abstract}
% IEEEtran.cls defaults to using nonbold math in the Abstract.
% This preserves the distinction between vectors and scalars. However,
% if the journal you are submitting to favors bold math in the abstract,
% then you can use LaTeX's standard command \boldmath at the very start
% of the abstract to achieve this. Many IEEE journals frown on math
% in the abstract anyway.

% Note that keywords are not normally used for peerreview papers.
%\begin{IEEEkeywords}
%Cooperative diversity, decode and forward, piecewise linear
%\end{IEEEkeywords}



% For peer review papers, you can put extra information on the cover
% page as needed:
% \ifCLASSOPTIONpeerreview
% \begin{center} \bfseries EDICS Category: 3-BBND \end{center}
% \fi
%
% For peerreview papers, this IEEEtran command inserts a page break and
% creates the second title. It will be ignored for other modes.
%\IEEEpeerreviewmaketitle




\item 
A die is thrown and a card is selected at random from a deck of 52 playing cards. The probability of getting an even number on the die and a spade card.\\
\solution
%\begin{table}[H]
	\centering
\begin{tabular}{|c|c|c|}
\hline
Random variable &Value &Definition\\ \hline
\multirow{3}{*}{X} &0 &Slips of Rs 1\\
&1 &Slips of Rs 5\\
&2 &Slips of Rs 13\\ \hline
\multirow{2}{*}{Y} &0 &Box A\\
&1 &Box B\\\hline
\end{tabular}
\caption{}
\label{tab:Distribution}
\end{table}
See \tabref{tab:Distribution}.
\begin{align}
p_{Y}\brak{k}= \begin{cases} 
      \frac{1}{3} & {k=0} \\
      \frac{2}{3 }& {k=1} 
   \end{cases}
   \\
p_{Y|X}\brak{0|0} = \frac{19}{25}\, 
p_{Y|X}\brak{0|1} = \frac{6}{25}\,
p_{Y|X}\brak{1|0} = \frac{45}{50}\,
p_{Y|X}\brak{1|2} = \frac{5}{50}
\end{align}
The desired probability is the probability that a slip drawn at random is marked other than Rs 1,
\begin{align}
&=1-p_X\brak{0}\\
&= p_X(1) + p_X(2)
\end{align}
Using Bayes theorem,
\begin{align}
&= p_Y\brak{0} \times \pr{Y=0 | X=1} + p_Y\brak{1} \times \pr{Y=1|X=2}\\
&=\frac{1}{3} \times \frac{6}{25} + \frac{2}{3} \times \frac{5}{50}\\
&=\frac{11}{75}
\end{align}

\newpage

%\tableofcontents

\bigskip

\renewcommand{\thefigure}{\theenumi}
\renewcommand{\thetable}{\theenumi}
%\renewcommand{\theequation}{\theenumi}

%\begin{abstract}
%%\boldmath
%In this letter, an algorithm for evaluating the exact analytical bit error rate  (BER)  for the piecewise linear (PL) combiner for  multiple relays is presented. Previous results were available only for upto three relays. The algorithm is unique in the sense that  the actual mathematical expressions, that are prohibitively large, need not be explicitly obtained. The diversity gain due to multiple relays is shown through plots of the analytical BER, well supported by simulations. 
%
%\end{abstract}
% IEEEtran.cls defaults to using nonbold math in the Abstract.
% This preserves the distinction between vectors and scalars. However,
% if the journal you are submitting to favors bold math in the abstract,
% then you can use LaTeX's standard command \boldmath at the very start
% of the abstract to achieve this. Many IEEE journals frown on math
% in the abstract anyway.

% Note that keywords are not normally used for peerreview papers.
%\begin{IEEEkeywords}
%Cooperative diversity, decode and forward, piecewise linear
%\end{IEEEkeywords}



% For peer review papers, you can put extra information on the cover
% page as needed:
% \ifCLASSOPTIONpeerreview
% \begin{center} \bfseries EDICS Category: 3-BBND \end{center}
% \fi
%
% For peerreview papers, this IEEEtran command inserts a page break and
% creates the second title. It will be ignored for other modes.
%\IEEEpeerreviewmaketitle




\item
If 4-digit numbers greater than 5,000 are randomly formed from the digits 0, 1, 3, 5, and 7, what is the probability of forming a number divisible by 5 when:
\begin{enumerate}
    \item The digits are repeated?
    \item The repetition of digits is not allowed?
\end{enumerate}
\solution
%\begin{table}[H]
	\centering
\begin{tabular}{|c|c|c|}
\hline
Random variable &Value &Definition\\ \hline
\multirow{3}{*}{X} &0 &Slips of Rs 1\\
&1 &Slips of Rs 5\\
&2 &Slips of Rs 13\\ \hline
\multirow{2}{*}{Y} &0 &Box A\\
&1 &Box B\\\hline
\end{tabular}
\caption{}
\label{tab:Distribution}
\end{table}
See \tabref{tab:Distribution}.
\begin{align}
p_{Y}\brak{k}= \begin{cases} 
      \frac{1}{3} & {k=0} \\
      \frac{2}{3 }& {k=1} 
   \end{cases}
   \\
p_{Y|X}\brak{0|0} = \frac{19}{25}\, 
p_{Y|X}\brak{0|1} = \frac{6}{25}\,
p_{Y|X}\brak{1|0} = \frac{45}{50}\,
p_{Y|X}\brak{1|2} = \frac{5}{50}
\end{align}
The desired probability is the probability that a slip drawn at random is marked other than Rs 1,
\begin{align}
&=1-p_X\brak{0}\\
&= p_X(1) + p_X(2)
\end{align}
Using Bayes theorem,
\begin{align}
&= p_Y\brak{0} \times \pr{Y=0 | X=1} + p_Y\brak{1} \times \pr{Y=1|X=2}\\
&=\frac{1}{3} \times \frac{6}{25} + \frac{2}{3} \times \frac{5}{50}\\
&=\frac{11}{75}
\end{align}

\newpage

%\tableofcontents

\bigskip

\renewcommand{\thefigure}{\theenumi}
\renewcommand{\thetable}{\theenumi}
%\renewcommand{\theequation}{\theenumi}

%\begin{abstract}
%%\boldmath
%In this letter, an algorithm for evaluating the exact analytical bit error rate  (BER)  for the piecewise linear (PL) combiner for  multiple relays is presented. Previous results were available only for upto three relays. The algorithm is unique in the sense that  the actual mathematical expressions, that are prohibitively large, need not be explicitly obtained. The diversity gain due to multiple relays is shown through plots of the analytical BER, well supported by simulations. 
%
%\end{abstract}
% IEEEtran.cls defaults to using nonbold math in the Abstract.
% This preserves the distinction between vectors and scalars. However,
% if the journal you are submitting to favors bold math in the abstract,
% then you can use LaTeX's standard command \boldmath at the very start
% of the abstract to achieve this. Many IEEE journals frown on math
% in the abstract anyway.

% Note that keywords are not normally used for peerreview papers.
%\begin{IEEEkeywords}
%Cooperative diversity, decode and forward, piecewise linear
%\end{IEEEkeywords}



% For peer review papers, you can put extra information on the cover
% page as needed:
% \ifCLASSOPTIONpeerreview
% \begin{center} \bfseries EDICS Category: 3-BBND \end{center}
% \fi
%
% For peerreview papers, this IEEEtran command inserts a page break and
% creates the second title. It will be ignored for other modes.
%\IEEEpeerreviewmaketitle




\item Consider the probability space $\brak{\Omega, \mathcal{G}, P}$ where $\Omega = [0,2]$ and $\mathcal{G} = \cbrak{\phi, \Omega, [0,1], (1,2]}$. Let $X$ and $Y$ be two functions on $\Omega$ defined as
\begin{align*}
    X(\omega) = 
    \begin{cases}
        1 & \text{if }\omega \in [0, 1]\\
        2 & \text{if }\omega \in (1, 2]
    \end{cases}
\end{align*}
and
\begin{align*}
    Y(\omega) = 
    \begin{cases}
        2 & \text{if }\omega \in [0, 1.5]\\
        3 & \text{if }\omega \in (1.5, 2].
    \end{cases}
\end{align*}
Then which one of the following statements is true?
\begin{enumerate}
    \item [(A)] $X$ is a random variable with respect to $\mathcal{G}$, but $Y$ is not a random variable with respect to $\mathcal{G}$.
    \item [(B)] $Y$ is a random variable with respect to $\mathcal{G}$, but $X$ is not a random variable with respect to $\mathcal{G}$.
    \item [(C)] Neither $X$ nor $Y$ is a random variable with respect to $\mathcal{G}$.
    \item [(D)] Both $X$ and $Y$ are random variables with respect to $\mathcal{G}$.
\end{enumerate} \hfill (GATE ST 2023)\\
\solution
%\begin{table}[H]
	\centering
\begin{tabular}{|c|c|c|}
\hline
Random variable &Value &Definition\\ \hline
\multirow{3}{*}{X} &0 &Slips of Rs 1\\
&1 &Slips of Rs 5\\
&2 &Slips of Rs 13\\ \hline
\multirow{2}{*}{Y} &0 &Box A\\
&1 &Box B\\\hline
\end{tabular}
\caption{}
\label{tab:Distribution}
\end{table}
See \tabref{tab:Distribution}.
\begin{align}
p_{Y}\brak{k}= \begin{cases} 
      \frac{1}{3} & {k=0} \\
      \frac{2}{3 }& {k=1} 
   \end{cases}
   \\
p_{Y|X}\brak{0|0} = \frac{19}{25}\, 
p_{Y|X}\brak{0|1} = \frac{6}{25}\,
p_{Y|X}\brak{1|0} = \frac{45}{50}\,
p_{Y|X}\brak{1|2} = \frac{5}{50}
\end{align}
The desired probability is the probability that a slip drawn at random is marked other than Rs 1,
\begin{align}
&=1-p_X\brak{0}\\
&= p_X(1) + p_X(2)
\end{align}
Using Bayes theorem,
\begin{align}
&= p_Y\brak{0} \times \pr{Y=0 | X=1} + p_Y\brak{1} \times \pr{Y=1|X=2}\\
&=\frac{1}{3} \times \frac{6}{25} + \frac{2}{3} \times \frac{5}{50}\\
&=\frac{11}{75}
\end{align}

\newpage

%\tableofcontents

\bigskip

\renewcommand{\thefigure}{\theenumi}
\renewcommand{\thetable}{\theenumi}
%\renewcommand{\theequation}{\theenumi}

%\begin{abstract}
%%\boldmath
%In this letter, an algorithm for evaluating the exact analytical bit error rate  (BER)  for the piecewise linear (PL) combiner for  multiple relays is presented. Previous results were available only for upto three relays. The algorithm is unique in the sense that  the actual mathematical expressions, that are prohibitively large, need not be explicitly obtained. The diversity gain due to multiple relays is shown through plots of the analytical BER, well supported by simulations. 
%
%\end{abstract}
% IEEEtran.cls defaults to using nonbold math in the Abstract.
% This preserves the distinction between vectors and scalars. However,
% if the journal you are submitting to favors bold math in the abstract,
% then you can use LaTeX's standard command \boldmath at the very start
% of the abstract to achieve this. Many IEEE journals frown on math
% in the abstract anyway.

% Note that keywords are not normally used for peerreview papers.
%\begin{IEEEkeywords}
%Cooperative diversity, decode and forward, piecewise linear
%\end{IEEEkeywords}



% For peer review papers, you can put extra information on the cover
% page as needed:
% \ifCLASSOPTIONpeerreview
% \begin{center} \bfseries EDICS Category: 3-BBND \end{center}
% \fi
%
% For peerreview papers, this IEEEtran command inserts a page break and
% creates the second title. It will be ignored for other modes.
%\IEEEpeerreviewmaketitle




	\item  A die is loaded in such a way that each odd number is twice as likely to occur as
each even number. Find $P(G)$, where $G$ is the event that a number greater than
3 occurs on a single roll of the die.
\\
\solution
		%\begin{table}[H]
	\centering
\begin{tabular}{|c|c|c|}
\hline
Random variable &Value &Definition\\ \hline
\multirow{3}{*}{X} &0 &Slips of Rs 1\\
&1 &Slips of Rs 5\\
&2 &Slips of Rs 13\\ \hline
\multirow{2}{*}{Y} &0 &Box A\\
&1 &Box B\\\hline
\end{tabular}
\caption{}
\label{tab:Distribution}
\end{table}
See \tabref{tab:Distribution}.
\begin{align}
p_{Y}\brak{k}= \begin{cases} 
      \frac{1}{3} & {k=0} \\
      \frac{2}{3 }& {k=1} 
   \end{cases}
   \\
p_{Y|X}\brak{0|0} = \frac{19}{25}\, 
p_{Y|X}\brak{0|1} = \frac{6}{25}\,
p_{Y|X}\brak{1|0} = \frac{45}{50}\,
p_{Y|X}\brak{1|2} = \frac{5}{50}
\end{align}
The desired probability is the probability that a slip drawn at random is marked other than Rs 1,
\begin{align}
&=1-p_X\brak{0}\\
&= p_X(1) + p_X(2)
\end{align}
Using Bayes theorem,
\begin{align}
&= p_Y\brak{0} \times \pr{Y=0 | X=1} + p_Y\brak{1} \times \pr{Y=1|X=2}\\
&=\frac{1}{3} \times \frac{6}{25} + \frac{2}{3} \times \frac{5}{50}\\
&=\frac{11}{75}
\end{align}

\newpage

%\tableofcontents

\bigskip

\renewcommand{\thefigure}{\theenumi}
\renewcommand{\thetable}{\theenumi}
%\renewcommand{\theequation}{\theenumi}

%\begin{abstract}
%%\boldmath
%In this letter, an algorithm for evaluating the exact analytical bit error rate  (BER)  for the piecewise linear (PL) combiner for  multiple relays is presented. Previous results were available only for upto three relays. The algorithm is unique in the sense that  the actual mathematical expressions, that are prohibitively large, need not be explicitly obtained. The diversity gain due to multiple relays is shown through plots of the analytical BER, well supported by simulations. 
%
%\end{abstract}
% IEEEtran.cls defaults to using nonbold math in the Abstract.
% This preserves the distinction between vectors and scalars. However,
% if the journal you are submitting to favors bold math in the abstract,
% then you can use LaTeX's standard command \boldmath at the very start
% of the abstract to achieve this. Many IEEE journals frown on math
% in the abstract anyway.

% Note that keywords are not normally used for peerreview papers.
%\begin{IEEEkeywords}
%Cooperative diversity, decode and forward, piecewise linear
%\end{IEEEkeywords}



% For peer review papers, you can put extra information on the cover
% page as needed:
% \ifCLASSOPTIONpeerreview
% \begin{center} \bfseries EDICS Category: 3-BBND \end{center}
% \fi
%
% For peerreview papers, this IEEEtran command inserts a page break and
% creates the second title. It will be ignored for other modes.
%\IEEEpeerreviewmaketitle




	\item All the jacks, queens and kings are removed from a deck of 52 playing cards. The remaining cards are well shuffled and then one card is drawn at random. Giving ace a value 1 similar value for other cards, find the probability that the card has a value 
		\begin{enumerate}
			\item 7
			\item greater than 7
			\item less than 7
		\end{enumerate}
		%Number of cards left after removing all jacks, queens and kings 
\begin{align}
N	= 52 - 4\times 3
	= 40
\end{align}
%\begin{table}[H]
%\def\arraystretch{1.2}
%\begin{tabular}{|c|c|c|}
%\hline
%	\textbf{Parameter} &\textbf{Value} &\textbf{Description}\\ \hline
%	$X$ &1-10 &Represents the value of the card picked \\ \hline
%\end{tabular}
%\end{table}
Let $1 \le X \le 10$ be the value of the card picked.  Then,
\begin{align}
	p_X(k) &= \Pr(X=k)\ \forall\ 1 \leq k \leq 10\\
	&= \frac{4\times 1}{40}\\
	&= \frac{1}{10}\\
	\therefore p_X(k) &= 
	\begin{cases}
		\frac{1}{10} & 1 \leq k \leq 10\\
		0 & \text{otherwise}
	\end{cases}
\end{align}
and
\begin{align}
	F_{X}(k) &= \sum_{m=0}^{k}p_{X}(m) \quad 1 \leq k \leq 10\\
	&= \frac{k}{10}\\
	\therefore F_{X}(k) &= 
	\begin{cases}
		0 & k \leq 0\\
		\frac{k}{10} & 1\leq k \leq 10\\
		1 & k > 10 
	\end{cases}
\end{align}
\begin{enumerate}
	\item Probability that card has value equal to 7 is
		\begin{align}
			 p_{X}(7)
			= \frac{1}{10}
		\end{align}
	\item Probability that card has value greater than 7 is
		\begin{align}
			1 - F_X(7)
			&= 1 - \frac{7}{10}
			\\
			&= \frac{3}{10}
		\end{align}
	\item Probability that card has value less than 7 is
		\begin{align}
			 F_{X}(6)
			=\frac{6}{10}
		\end{align}
\end{enumerate}

  \item A Lot consists of 48 mobile phones of which 42 are good, 3 have only minor defects and 3 have major defects.Varnika will buy a phone if it is good but the trader will only buy a mobile if it has no major defects. One phone is selected at random from the lot. What is the probability that it is
\begin{enumerate}
	\item acceptable to Varnika?
            \item acceptable to the trader?
\end{enumerate}
\solution
	%\begin{table}[H]
	\centering
\begin{tabular}{|c|c|c|}
\hline
Random variable &Value &Definition\\ \hline
\multirow{3}{*}{X} &0 &Slips of Rs 1\\
&1 &Slips of Rs 5\\
&2 &Slips of Rs 13\\ \hline
\multirow{2}{*}{Y} &0 &Box A\\
&1 &Box B\\\hline
\end{tabular}
\caption{}
\label{tab:Distribution}
\end{table}
See \tabref{tab:Distribution}.
\begin{align}
p_{Y}\brak{k}= \begin{cases} 
      \frac{1}{3} & {k=0} \\
      \frac{2}{3 }& {k=1} 
   \end{cases}
   \\
p_{Y|X}\brak{0|0} = \frac{19}{25}\, 
p_{Y|X}\brak{0|1} = \frac{6}{25}\,
p_{Y|X}\brak{1|0} = \frac{45}{50}\,
p_{Y|X}\brak{1|2} = \frac{5}{50}
\end{align}
The desired probability is the probability that a slip drawn at random is marked other than Rs 1,
\begin{align}
&=1-p_X\brak{0}\\
&= p_X(1) + p_X(2)
\end{align}
Using Bayes theorem,
\begin{align}
&= p_Y\brak{0} \times \pr{Y=0 | X=1} + p_Y\brak{1} \times \pr{Y=1|X=2}\\
&=\frac{1}{3} \times \frac{6}{25} + \frac{2}{3} \times \frac{5}{50}\\
&=\frac{11}{75}
\end{align}

\newpage

%\tableofcontents

\bigskip

\renewcommand{\thefigure}{\theenumi}
\renewcommand{\thetable}{\theenumi}
%\renewcommand{\theequation}{\theenumi}

%\begin{abstract}
%%\boldmath
%In this letter, an algorithm for evaluating the exact analytical bit error rate  (BER)  for the piecewise linear (PL) combiner for  multiple relays is presented. Previous results were available only for upto three relays. The algorithm is unique in the sense that  the actual mathematical expressions, that are prohibitively large, need not be explicitly obtained. The diversity gain due to multiple relays is shown through plots of the analytical BER, well supported by simulations. 
%
%\end{abstract}
% IEEEtran.cls defaults to using nonbold math in the Abstract.
% This preserves the distinction between vectors and scalars. However,
% if the journal you are submitting to favors bold math in the abstract,
% then you can use LaTeX's standard command \boldmath at the very start
% of the abstract to achieve this. Many IEEE journals frown on math
% in the abstract anyway.

% Note that keywords are not normally used for peerreview papers.
%\begin{IEEEkeywords}
%Cooperative diversity, decode and forward, piecewise linear
%\end{IEEEkeywords}



% For peer review papers, you can put extra information on the cover
% page as needed:
% \ifCLASSOPTIONpeerreview
% \begin{center} \bfseries EDICS Category: 3-BBND \end{center}
% \fi
%
% For peerreview papers, this IEEEtran command inserts a page break and
% creates the second title. It will be ignored for other modes.
%\IEEEpeerreviewmaketitle




 \item A student says that if you throw a die, it will show up 1 or not 1. Therefore, the probability of getting 1 and the probability of getting 'not 1' each is equal to $\frac{1}{2}$. Is this correct? Give reasons.\\
 \solution
        %\begin{table}[H]
	\centering
\begin{tabular}{|c|c|c|}
\hline
Random variable &Value &Definition\\ \hline
\multirow{3}{*}{X} &0 &Slips of Rs 1\\
&1 &Slips of Rs 5\\
&2 &Slips of Rs 13\\ \hline
\multirow{2}{*}{Y} &0 &Box A\\
&1 &Box B\\\hline
\end{tabular}
\caption{}
\label{tab:Distribution}
\end{table}
See \tabref{tab:Distribution}.
\begin{align}
p_{Y}\brak{k}= \begin{cases} 
      \frac{1}{3} & {k=0} \\
      \frac{2}{3 }& {k=1} 
   \end{cases}
   \\
p_{Y|X}\brak{0|0} = \frac{19}{25}\, 
p_{Y|X}\brak{0|1} = \frac{6}{25}\,
p_{Y|X}\brak{1|0} = \frac{45}{50}\,
p_{Y|X}\brak{1|2} = \frac{5}{50}
\end{align}
The desired probability is the probability that a slip drawn at random is marked other than Rs 1,
\begin{align}
&=1-p_X\brak{0}\\
&= p_X(1) + p_X(2)
\end{align}
Using Bayes theorem,
\begin{align}
&= p_Y\brak{0} \times \pr{Y=0 | X=1} + p_Y\brak{1} \times \pr{Y=1|X=2}\\
&=\frac{1}{3} \times \frac{6}{25} + \frac{2}{3} \times \frac{5}{50}\\
&=\frac{11}{75}
\end{align}

\newpage

%\tableofcontents

\bigskip

\renewcommand{\thefigure}{\theenumi}
\renewcommand{\thetable}{\theenumi}
%\renewcommand{\theequation}{\theenumi}

%\begin{abstract}
%%\boldmath
%In this letter, an algorithm for evaluating the exact analytical bit error rate  (BER)  for the piecewise linear (PL) combiner for  multiple relays is presented. Previous results were available only for upto three relays. The algorithm is unique in the sense that  the actual mathematical expressions, that are prohibitively large, need not be explicitly obtained. The diversity gain due to multiple relays is shown through plots of the analytical BER, well supported by simulations. 
%
%\end{abstract}
% IEEEtran.cls defaults to using nonbold math in the Abstract.
% This preserves the distinction between vectors and scalars. However,
% if the journal you are submitting to favors bold math in the abstract,
% then you can use LaTeX's standard command \boldmath at the very start
% of the abstract to achieve this. Many IEEE journals frown on math
% in the abstract anyway.

% Note that keywords are not normally used for peerreview papers.
%\begin{IEEEkeywords}
%Cooperative diversity, decode and forward, piecewise linear
%\end{IEEEkeywords}



% For peer review papers, you can put extra information on the cover
% page as needed:
% \ifCLASSOPTIONpeerreview
% \begin{center} \bfseries EDICS Category: 3-BBND \end{center}
% \fi
%
% For peerreview papers, this IEEEtran command inserts a page break and
% creates the second title. It will be ignored for other modes.
%\IEEEpeerreviewmaketitle




   \item Four candidates A, B, C, D have ap-
plied for the assignment to coach a school cricket
team. If A is twice as likely to be selected as B, and
B and C are given about the same chance of being
selected, while C is twice as likely to be selected
as D, what are the probabilities that
\begin{enumerate}
\item C will be selected?
\item A will not be selected?
\end{enumerate}
	%\begin{table}[H]
	\centering
\begin{tabular}{|c|c|c|}
\hline
Random variable &Value &Definition\\ \hline
\multirow{3}{*}{X} &0 &Slips of Rs 1\\
&1 &Slips of Rs 5\\
&2 &Slips of Rs 13\\ \hline
\multirow{2}{*}{Y} &0 &Box A\\
&1 &Box B\\\hline
\end{tabular}
\caption{}
\label{tab:Distribution}
\end{table}
See \tabref{tab:Distribution}.
\begin{align}
p_{Y}\brak{k}= \begin{cases} 
      \frac{1}{3} & {k=0} \\
      \frac{2}{3 }& {k=1} 
   \end{cases}
   \\
p_{Y|X}\brak{0|0} = \frac{19}{25}\, 
p_{Y|X}\brak{0|1} = \frac{6}{25}\,
p_{Y|X}\brak{1|0} = \frac{45}{50}\,
p_{Y|X}\brak{1|2} = \frac{5}{50}
\end{align}
The desired probability is the probability that a slip drawn at random is marked other than Rs 1,
\begin{align}
&=1-p_X\brak{0}\\
&= p_X(1) + p_X(2)
\end{align}
Using Bayes theorem,
\begin{align}
&= p_Y\brak{0} \times \pr{Y=0 | X=1} + p_Y\brak{1} \times \pr{Y=1|X=2}\\
&=\frac{1}{3} \times \frac{6}{25} + \frac{2}{3} \times \frac{5}{50}\\
&=\frac{11}{75}
\end{align}

\newpage

%\tableofcontents

\bigskip

\renewcommand{\thefigure}{\theenumi}
\renewcommand{\thetable}{\theenumi}
%\renewcommand{\theequation}{\theenumi}

%\begin{abstract}
%%\boldmath
%In this letter, an algorithm for evaluating the exact analytical bit error rate  (BER)  for the piecewise linear (PL) combiner for  multiple relays is presented. Previous results were available only for upto three relays. The algorithm is unique in the sense that  the actual mathematical expressions, that are prohibitively large, need not be explicitly obtained. The diversity gain due to multiple relays is shown through plots of the analytical BER, well supported by simulations. 
%
%\end{abstract}
% IEEEtran.cls defaults to using nonbold math in the Abstract.
% This preserves the distinction between vectors and scalars. However,
% if the journal you are submitting to favors bold math in the abstract,
% then you can use LaTeX's standard command \boldmath at the very start
% of the abstract to achieve this. Many IEEE journals frown on math
% in the abstract anyway.

% Note that keywords are not normally used for peerreview papers.
%\begin{IEEEkeywords}
%Cooperative diversity, decode and forward, piecewise linear
%\end{IEEEkeywords}



% For peer review papers, you can put extra information on the cover
% page as needed:
% \ifCLASSOPTIONpeerreview
% \begin{center} \bfseries EDICS Category: 3-BBND \end{center}
% \fi
%
% For peerreview papers, this IEEEtran command inserts a page break and
% creates the second title. It will be ignored for other modes.
%\IEEEpeerreviewmaketitle




 \item A bag contain 24 balls of which $x$ balls are red, $2x$ are white and $3x$ are blue. A ball is selected at random, What is the probability that it is
\begin{enumerate}[label=\alph*)]
\item not red ?
\item white ?
\end{enumerate}
%\begin{table}[H]
	\centering
\begin{tabular}{|c|c|c|}
\hline
Random variable &Value &Definition\\ \hline
\multirow{3}{*}{X} &0 &Slips of Rs 1\\
&1 &Slips of Rs 5\\
&2 &Slips of Rs 13\\ \hline
\multirow{2}{*}{Y} &0 &Box A\\
&1 &Box B\\\hline
\end{tabular}
\caption{}
\label{tab:Distribution}
\end{table}
See \tabref{tab:Distribution}.
\begin{align}
p_{Y}\brak{k}= \begin{cases} 
      \frac{1}{3} & {k=0} \\
      \frac{2}{3 }& {k=1} 
   \end{cases}
   \\
p_{Y|X}\brak{0|0} = \frac{19}{25}\, 
p_{Y|X}\brak{0|1} = \frac{6}{25}\,
p_{Y|X}\brak{1|0} = \frac{45}{50}\,
p_{Y|X}\brak{1|2} = \frac{5}{50}
\end{align}
The desired probability is the probability that a slip drawn at random is marked other than Rs 1,
\begin{align}
&=1-p_X\brak{0}\\
&= p_X(1) + p_X(2)
\end{align}
Using Bayes theorem,
\begin{align}
&= p_Y\brak{0} \times \pr{Y=0 | X=1} + p_Y\brak{1} \times \pr{Y=1|X=2}\\
&=\frac{1}{3} \times \frac{6}{25} + \frac{2}{3} \times \frac{5}{50}\\
&=\frac{11}{75}
\end{align}

\newpage

%\tableofcontents

\bigskip

\renewcommand{\thefigure}{\theenumi}
\renewcommand{\thetable}{\theenumi}
%\renewcommand{\theequation}{\theenumi}

%\begin{abstract}
%%\boldmath
%In this letter, an algorithm for evaluating the exact analytical bit error rate  (BER)  for the piecewise linear (PL) combiner for  multiple relays is presented. Previous results were available only for upto three relays. The algorithm is unique in the sense that  the actual mathematical expressions, that are prohibitively large, need not be explicitly obtained. The diversity gain due to multiple relays is shown through plots of the analytical BER, well supported by simulations. 
%
%\end{abstract}
% IEEEtran.cls defaults to using nonbold math in the Abstract.
% This preserves the distinction between vectors and scalars. However,
% if the journal you are submitting to favors bold math in the abstract,
% then you can use LaTeX's standard command \boldmath at the very start
% of the abstract to achieve this. Many IEEE journals frown on math
% in the abstract anyway.

% Note that keywords are not normally used for peerreview papers.
%\begin{IEEEkeywords}
%Cooperative diversity, decode and forward, piecewise linear
%\end{IEEEkeywords}



% For peer review papers, you can put extra information on the cover
% page as needed:
% \ifCLASSOPTIONpeerreview
% \begin{center} \bfseries EDICS Category: 3-BBND \end{center}
% \fi
%
% For peerreview papers, this IEEEtran command inserts a page break and
% creates the second title. It will be ignored for other modes.
%\IEEEpeerreviewmaketitle




If the letters of the word ASSASSINATION are arranged at random. Find the Probability that
\begin{enumerate}[label=(\alph*)]
\item Four $S's$ come consecutively in the word
\item Two  $I's$ and two $N's$ come together
\item All $A's$ are not coming together
\item No two $A's$ are coming together
\end{enumerate}
%\begin{table}[H]
	\centering
\begin{tabular}{|c|c|c|}
\hline
Random variable &Value &Definition\\ \hline
\multirow{3}{*}{X} &0 &Slips of Rs 1\\
&1 &Slips of Rs 5\\
&2 &Slips of Rs 13\\ \hline
\multirow{2}{*}{Y} &0 &Box A\\
&1 &Box B\\\hline
\end{tabular}
\caption{}
\label{tab:Distribution}
\end{table}
See \tabref{tab:Distribution}.
\begin{align}
p_{Y}\brak{k}= \begin{cases} 
      \frac{1}{3} & {k=0} \\
      \frac{2}{3 }& {k=1} 
   \end{cases}
   \\
p_{Y|X}\brak{0|0} = \frac{19}{25}\, 
p_{Y|X}\brak{0|1} = \frac{6}{25}\,
p_{Y|X}\brak{1|0} = \frac{45}{50}\,
p_{Y|X}\brak{1|2} = \frac{5}{50}
\end{align}
The desired probability is the probability that a slip drawn at random is marked other than Rs 1,
\begin{align}
&=1-p_X\brak{0}\\
&= p_X(1) + p_X(2)
\end{align}
Using Bayes theorem,
\begin{align}
&= p_Y\brak{0} \times \pr{Y=0 | X=1} + p_Y\brak{1} \times \pr{Y=1|X=2}\\
&=\frac{1}{3} \times \frac{6}{25} + \frac{2}{3} \times \frac{5}{50}\\
&=\frac{11}{75}
\end{align}

\newpage

%\tableofcontents

\bigskip

\renewcommand{\thefigure}{\theenumi}
\renewcommand{\thetable}{\theenumi}
%\renewcommand{\theequation}{\theenumi}

%\begin{abstract}
%%\boldmath
%In this letter, an algorithm for evaluating the exact analytical bit error rate  (BER)  for the piecewise linear (PL) combiner for  multiple relays is presented. Previous results were available only for upto three relays. The algorithm is unique in the sense that  the actual mathematical expressions, that are prohibitively large, need not be explicitly obtained. The diversity gain due to multiple relays is shown through plots of the analytical BER, well supported by simulations. 
%
%\end{abstract}
% IEEEtran.cls defaults to using nonbold math in the Abstract.
% This preserves the distinction between vectors and scalars. However,
% if the journal you are submitting to favors bold math in the abstract,
% then you can use LaTeX's standard command \boldmath at the very start
% of the abstract to achieve this. Many IEEE journals frown on math
% in the abstract anyway.

% Note that keywords are not normally used for peerreview papers.
%\begin{IEEEkeywords}
%Cooperative diversity, decode and forward, piecewise linear
%\end{IEEEkeywords}



% For peer review papers, you can put extra information on the cover
% page as needed:
% \ifCLASSOPTIONpeerreview
% \begin{center} \bfseries EDICS Category: 3-BBND \end{center}
% \fi
%
% For peerreview papers, this IEEEtran command inserts a page break and
% creates the second title. It will be ignored for other modes.
%\IEEEpeerreviewmaketitle




	\item One urn contains two black balls (labelled B1 and B2) and one white ball. A
	second urn contains one black ball and two white balls (labelled W1 and W2).
	Suppose the following experiment is performed. One of the two urns is chosen
	at random. Next a ball is randomly chosen from the urn. Then a second ball is
	chosen at random from the same urn without replacing the first ball.
	
	\begin{enumerate}
	\item What is the probability that two black balls are chosen?
	
	\item What is the probability that two balls of opposite colour are chosen?
	\end{enumerate}
	\solution
	%\begin{align}
    \label{eq:12.13.6.18.1}
	\because	\pr{A|B} &> \pr{A},\
\frac{\pr{AB}}{\pr{B}} > \pr{A}
\\
    \label{eq:12.13.6.18.2}
	\implies \pr{AB} &> \pr{A}\pr{B}
	\\
	\text{or, } \frac{\pr{AB}}{\pr{A}} &=\pr{B|A} > \pr{A}
\end{align}

\end{enumerate}

		\item A box of oranges is inspected by examining three randomly selected oranges drawn without replacement. If all the three oranges are good, the box is approved for sale, otherwise, it is rejected. Find the probability that a box containing 15 oranges out of which 12 are good and 3 are bad ones will be approved for sale.
		\label{ncert/12/13/2/3/defs.tex}
		\item Two balls are drawn at random with replacement from a box containing 10 black and 8 red balls. Find the probability that
		\label{ncert/12/13/2/12}
\begin{enumerate}
\item both balls are red.
\item first ball is black and second is red.
\item one of them is black and other is red.
\end{enumerate}

\item In a hostel, 60\% of the students read Hindi newspaper, 40\% read English newspaper and 20\% read both Hindi and English newspapers. A student is selected at random.
		\label{ncert/12/13/2/15}
\begin{enumerate}
\item Find the probability that she reads neither Hindi nor English newspapers.
\item If she reads Hindi newspaper, find the probability that she reads English newspaper.
\item If she reads English newspaper, find the probability that she reads Hindi newspaper.\\
\end{enumerate}
\item The probability of obtaining an even prime number on each die, when a pair of dice is rolled is 
\begin{enumerate}
    \item $0$ 
    
    \item $\frac{1}{3}$ 
    
    \item $\frac{1}{12}$ 
    
    \item $\frac{1}{36}$ 
\end{enumerate}
\solution
		%\begin{enumerate}[label=\thesection.\arabic*,ref=\thesection.\theenumi]
	\item One card is drawn from a well-shuffled deck of 52 cards. Find the probability of getting
\begin{enumerate}
\item A king of red colour 
\item A face card 
\item A red face card
\item The jack of hearts
\item A spade
\item The queen of diamonds

\end{enumerate}
\solution
		%\begin{table}[H]
	\centering
\begin{tabular}{|c|c|c|}
\hline
Random variable &Value &Definition\\ \hline
\multirow{3}{*}{X} &0 &Slips of Rs 1\\
&1 &Slips of Rs 5\\
&2 &Slips of Rs 13\\ \hline
\multirow{2}{*}{Y} &0 &Box A\\
&1 &Box B\\\hline
\end{tabular}
\caption{}
\label{tab:Distribution}
\end{table}
See \tabref{tab:Distribution}.
\begin{align}
p_{Y}\brak{k}= \begin{cases} 
      \frac{1}{3} & {k=0} \\
      \frac{2}{3 }& {k=1} 
   \end{cases}
   \\
p_{Y|X}\brak{0|0} = \frac{19}{25}\, 
p_{Y|X}\brak{0|1} = \frac{6}{25}\,
p_{Y|X}\brak{1|0} = \frac{45}{50}\,
p_{Y|X}\brak{1|2} = \frac{5}{50}
\end{align}
The desired probability is the probability that a slip drawn at random is marked other than Rs 1,
\begin{align}
&=1-p_X\brak{0}\\
&= p_X(1) + p_X(2)
\end{align}
Using Bayes theorem,
\begin{align}
&= p_Y\brak{0} \times \pr{Y=0 | X=1} + p_Y\brak{1} \times \pr{Y=1|X=2}\\
&=\frac{1}{3} \times \frac{6}{25} + \frac{2}{3} \times \frac{5}{50}\\
&=\frac{11}{75}
\end{align}

\newpage

%\tableofcontents

\bigskip

\renewcommand{\thefigure}{\theenumi}
\renewcommand{\thetable}{\theenumi}
%\renewcommand{\theequation}{\theenumi}

%\begin{abstract}
%%\boldmath
%In this letter, an algorithm for evaluating the exact analytical bit error rate  (BER)  for the piecewise linear (PL) combiner for  multiple relays is presented. Previous results were available only for upto three relays. The algorithm is unique in the sense that  the actual mathematical expressions, that are prohibitively large, need not be explicitly obtained. The diversity gain due to multiple relays is shown through plots of the analytical BER, well supported by simulations. 
%
%\end{abstract}
% IEEEtran.cls defaults to using nonbold math in the Abstract.
% This preserves the distinction between vectors and scalars. However,
% if the journal you are submitting to favors bold math in the abstract,
% then you can use LaTeX's standard command \boldmath at the very start
% of the abstract to achieve this. Many IEEE journals frown on math
% in the abstract anyway.

% Note that keywords are not normally used for peerreview papers.
%\begin{IEEEkeywords}
%Cooperative diversity, decode and forward, piecewise linear
%\end{IEEEkeywords}



% For peer review papers, you can put extra information on the cover
% page as needed:
% \ifCLASSOPTIONpeerreview
% \begin{center} \bfseries EDICS Category: 3-BBND \end{center}
% \fi
%
% For peerreview papers, this IEEEtran command inserts a page break and
% creates the second title. It will be ignored for other modes.
%\IEEEpeerreviewmaketitle




	\item Five cards—the ten, jack, queen, king and ace of diamonds, are well-shuffled with their face downwards. One card is then picked up at random.
\begin{enumerate}
\item
What is the probability that the card is the queen? 
\item
If the queen is drawn and put aside, what is the probability that the second card picked up is (a) an ace? (b) a queen?\\
\end{enumerate}
\solution
		%\begin{enumerate}[label=\thesection.\arabic*,ref=\thesection.\theenumi]
	\item One card is drawn from a well-shuffled deck of 52 cards. Find the probability of getting
\begin{enumerate}
\item A king of red colour 
\item A face card 
\item A red face card
\item The jack of hearts
\item A spade
\item The queen of diamonds

\end{enumerate}
\solution
		%\input{ncert/10/15/1/14/main.tex}
	\item Five cards—the ten, jack, queen, king and ace of diamonds, are well-shuffled with their face downwards. One card is then picked up at random.
\begin{enumerate}
\item
What is the probability that the card is the queen? 
\item
If the queen is drawn and put aside, what is the probability that the second card picked up is (a) an ace? (b) a queen?\\
\end{enumerate}
\solution
		%\input{ncert/10/15/1/15/defs.tex}
	\item A bag contains $5$ red balls and some blue balls. If the probability of drawing a blue ball is double that if a red ball, determine the number of blue balls in the bag. 
		\\
\solution
		%\input{ncert/10/15/2/3/defs.tex}
	\item A card is selected from a pack of 52 cards.
 \begin{enumerate}[label=(\alph*)] 
                 \item How many points are there in the sample space?
                 \item Calculate the probability that the card is an ace of spades.
                 \item Calculate the probability that the card is (i) an ace and (ii) black card.
 \end{enumerate}
\solution
		%\input{ncert/11/16/3/4/main.tex}
\item Four cards are drawn from a well-shuffled deck of 52 cards. What is the probability of obtaining 3 diamonds and one spade.
\\
\solution
		%\input{ncert/11/16/4/2/defs.tex}
\item In a certain lottery 10,000 tickets are sold and ten equal prizes are awarded. What is the probability of not getting a prize if you buy (a) one ticket (b) two tickets (c) 10 tickets ?	
\\
\solution
		%\input{ncert/11/16/4/4/defs.tex}
		%
\item 
Out of 100 students, two sections of 40 and 60 are formed. If you and your friend are among the 100 students, what is the probability that
\begin{enumerate}
\item you both enter the same section?
\item you both enter the different sections?
\end{enumerate}
\solution
		%\input{ncert/11/16/4/5/defs.tex}
	\item 
The number lock of a suitcase has 4 wheels each labelled with ten digits i.e. from 0 to 9.The lock opens with a sequence of four digits with no repeats.What is the probability of a person getting the right sequence to open the suitcase.
\\
\solution
		%\input{ncert/11/16/4/10/defs.tex}
		%
\item 
Two cards are drawn at random and without replacement from a pack of 52 playing cards. Find the probability that both the cards are black.
\\
\solution
		%\input{ncert/12/13/2/2/defs.tex}
		\item A box of oranges is inspected by examining three randomly selected oranges drawn without replacement. If all the three oranges are good, the box is approved for sale, otherwise, it is rejected. Find the probability that a box containing 15 oranges out of which 12 are good and 3 are bad ones will be approved for sale.
		\label{ncert/12/13/2/3/defs.tex}
		\item Two balls are drawn at random with replacement from a box containing 10 black and 8 red balls. Find the probability that
		\label{ncert/12/13/2/12}
\begin{enumerate}
\item both balls are red.
\item first ball is black and second is red.
\item one of them is black and other is red.
\end{enumerate}

\item In a hostel, 60\% of the students read Hindi newspaper, 40\% read English newspaper and 20\% read both Hindi and English newspapers. A student is selected at random.
		\label{ncert/12/13/2/15}
\begin{enumerate}
\item Find the probability that she reads neither Hindi nor English newspapers.
\item If she reads Hindi newspaper, find the probability that she reads English newspaper.
\item If she reads English newspaper, find the probability that she reads Hindi newspaper.\\
\end{enumerate}
\item The probability of obtaining an even prime number on each die, when a pair of dice is rolled is 
\begin{enumerate}
    \item $0$ 
    
    \item $\frac{1}{3}$ 
    
    \item $\frac{1}{12}$ 
    
    \item $\frac{1}{36}$ 
\end{enumerate}
\solution
		%\input{ncert/12/13/2/17/defs.tex}
	\item A bag contains 4 red and 4 black balls, another bag contains 2 red and 6 black balls. One of the two bags is selected at random and a ball is drawn from the bag which is found to be red. Find the probability that the ball is drawn from the first bag.
\\
\solution
		%\input{ncert/12/13/3/2/main.tex}
  \item
  Cards with numbers 2 to 101 are placed in a box. A card is selected at random.Find the probability that the card has
\begin{enumerate}[label=(\roman*)]
	\item an even number 
	\item a square number
\end{enumerate}
\solution
%\input{exemplar/10/13/3/32/main.tex}
\item
The king, queen and jack of clubs are removed from a deck of 52 playing cards and then well shuffled. Now one card is drawn at random from the remaining cards.  Determine the probability that the card is
\begin{enumerate}[label=(\roman*)]
\item a club
\item 10 of hearts
\end{enumerate}
\solution
%\input{exemplar/10/13/3/29/main.tex}
\item A team of medical students doing their internship have to assist during surgeries
at a city hospital. The probabilities of surgeries rated as very complex, complex,
routine, simple or very simple are respectively, 0.15, 0.20, 0.31, 0.26, .08. Find
the probabilities that a particular surgery will be rated
\begin{enumerate}
	\item complex or very complex;
	\item neither very complex nor very simple;
	\item routine or complex
	\item routine or simple
\end{enumerate}
\solution
%\input{exemplar/11/16/3/8(1)/main.tex}
\item A card is selected from a pack of 52 cards.
\begin{enumerate}[label=(\alph*)]
    \item How many points are there in the sample space?
    \item Calculate the probability that the card is an ace of spades.
    \item Calculate the probability that the card is (i) an ace and (ii) black card.
\end{enumerate}
\solution
%\input{exemplar/11/16/3/4/main2.tex}
\item The probability that a non leap year selected at random will contain 53 sundays.
\\
\solution
%\input{exemplar/10/13/1/19/main.tex}
\item One of the four persons John, Rita, Aslam or Gurpreet will be promoted next
month. Consequently the sample space consists of four elementary outcomes
S = {John promoted, Rita promoted, Aslam promoted, Gurpreet promoted}
You are told that the chances of John’s promotion is same as that of Gurpreet,
Rita’s chances of promotion are twice as likely as Johns. Aslam’s chances are
four times that of John.
\begin{enumerate}
	\item Determine
	\begin{enumerate}
		\item P (John promoted)
		\item P (Rita promoted)
		\item P (Aslam promoted)
		\item P (Gurpreet promoted)
	\end{enumerate}
	\item If A = {John promoted or Gurpreet promoted}, find P (A).
\end{enumerate}
\solution
%\input{exemplar/11/16/3/10/main.tex}
\item A card is drawn from a deck of 52 cards. Find the probability of getting a king or a heart or a red card.\\
\solution
%\input{exemplar/11/16/3/15/main.tex}
\item The probability that a student will pass his examination is 0.73, the probability of
the student getting a compartment is 0.13, and the probability that the student will
either pass or get compartment is 0.96. State True or False.\\
\solution
%\input{exemplar/11/16/3/31/main.tex}
\item A card is selected from a pack of 52 cards\\
\begin{enumerate}[label=(\alph*)]
\item How many points are there in the sample space?
\item Calculate the probability that the cards is an ace of spades.
\item Calculate the probability that the card is (i) an ace (ii)black card.\\
\end{enumerate}
%\input{ncert/11/16/3/4_1/Prob_4.tex}
\item In a non-leap year, the probability of having 53 tuesdays or 53 wednesdays is\\
\solution
%\input{exemplar/11/16/3/18/main.tex}
\item There are 1000 sealed envelopes in a box, 10 of them contain a cash prize of
Rs 100 each, 100 of them contain a cash prize of Rs 50 each and 200 of them
contain a cash prize of Rs 10 each and rest do not contain any cash prize. If they
are well shuffled and an envelope is picked up out, what is the probability that it
contains no cash prize?\\
\solution
%\input{exemplar/10/13/3/34/main.tex}
\item 
A die is thrown and a card is selected at random from a deck of 52 playing cards. The probability of getting an even number on the die and a spade card.\\
\solution
%\input{exemplar/12/13/3/78/main.tex}
\item
If 4-digit numbers greater than 5,000 are randomly formed from the digits 0, 1, 3, 5, and 7, what is the probability of forming a number divisible by 5 when:
\begin{enumerate}
    \item The digits are repeated?
    \item The repetition of digits is not allowed?
\end{enumerate}
\solution
%\input{ncert/11/16/4/9/main.tex}
\item Consider the probability space $\brak{\Omega, \mathcal{G}, P}$ where $\Omega = [0,2]$ and $\mathcal{G} = \cbrak{\phi, \Omega, [0,1], (1,2]}$. Let $X$ and $Y$ be two functions on $\Omega$ defined as
\begin{align*}
    X(\omega) = 
    \begin{cases}
        1 & \text{if }\omega \in [0, 1]\\
        2 & \text{if }\omega \in (1, 2]
    \end{cases}
\end{align*}
and
\begin{align*}
    Y(\omega) = 
    \begin{cases}
        2 & \text{if }\omega \in [0, 1.5]\\
        3 & \text{if }\omega \in (1.5, 2].
    \end{cases}
\end{align*}
Then which one of the following statements is true?
\begin{enumerate}
    \item [(A)] $X$ is a random variable with respect to $\mathcal{G}$, but $Y$ is not a random variable with respect to $\mathcal{G}$.
    \item [(B)] $Y$ is a random variable with respect to $\mathcal{G}$, but $X$ is not a random variable with respect to $\mathcal{G}$.
    \item [(C)] Neither $X$ nor $Y$ is a random variable with respect to $\mathcal{G}$.
    \item [(D)] Both $X$ and $Y$ are random variables with respect to $\mathcal{G}$.
\end{enumerate} \hfill (GATE ST 2023)\\
\solution
%\input{gate/ST/2023/14/main.tex}
	\item  A die is loaded in such a way that each odd number is twice as likely to occur as
each even number. Find $P(G)$, where $G$ is the event that a number greater than
3 occurs on a single roll of the die.
\\
\solution
		%\input{exemplar/11/16/3/5/main.tex}
	\item All the jacks, queens and kings are removed from a deck of 52 playing cards. The remaining cards are well shuffled and then one card is drawn at random. Giving ace a value 1 similar value for other cards, find the probability that the card has a value 
		\begin{enumerate}
			\item 7
			\item greater than 7
			\item less than 7
		\end{enumerate}
		%\input{exemplar/10/13/3/30/main.tex}
  \item A Lot consists of 48 mobile phones of which 42 are good, 3 have only minor defects and 3 have major defects.Varnika will buy a phone if it is good but the trader will only buy a mobile if it has no major defects. One phone is selected at random from the lot. What is the probability that it is
\begin{enumerate}
	\item acceptable to Varnika?
            \item acceptable to the trader?
\end{enumerate}
\solution
	%\input{exemplar/10/13/3/40/main.tex}
 \item A student says that if you throw a die, it will show up 1 or not 1. Therefore, the probability of getting 1 and the probability of getting 'not 1' each is equal to $\frac{1}{2}$. Is this correct? Give reasons.\\
 \solution
        %\input{exemplar/10/13/2/9/main.tex}
   \item Four candidates A, B, C, D have ap-
plied for the assignment to coach a school cricket
team. If A is twice as likely to be selected as B, and
B and C are given about the same chance of being
selected, while C is twice as likely to be selected
as D, what are the probabilities that
\begin{enumerate}
\item C will be selected?
\item A will not be selected?
\end{enumerate}
	%\input{exemplar/11/16/3/9/main.tex}
 \item A bag contain 24 balls of which $x$ balls are red, $2x$ are white and $3x$ are blue. A ball is selected at random, What is the probability that it is
\begin{enumerate}[label=\alph*)]
\item not red ?
\item white ?
\end{enumerate}
%\input{exemplar/10/13/3/41/main.tex}
If the letters of the word ASSASSINATION are arranged at random. Find the Probability that
\begin{enumerate}[label=(\alph*)]
\item Four $S's$ come consecutively in the word
\item Two  $I's$ and two $N's$ come together
\item All $A's$ are not coming together
\item No two $A's$ are coming together
\end{enumerate}
%\input{exemplar/11/16/3/14/main.tex}
	\item One urn contains two black balls (labelled B1 and B2) and one white ball. A
	second urn contains one black ball and two white balls (labelled W1 and W2).
	Suppose the following experiment is performed. One of the two urns is chosen
	at random. Next a ball is randomly chosen from the urn. Then a second ball is
	chosen at random from the same urn without replacing the first ball.
	
	\begin{enumerate}
	\item What is the probability that two black balls are chosen?
	
	\item What is the probability that two balls of opposite colour are chosen?
	\end{enumerate}
	\solution
	%\input{exemplar/11/16/3/12/main1.tex}
\end{enumerate}

	\item A bag contains $5$ red balls and some blue balls. If the probability of drawing a blue ball is double that if a red ball, determine the number of blue balls in the bag. 
		\\
\solution
		%\begin{enumerate}[label=\thesection.\arabic*,ref=\thesection.\theenumi]
	\item One card is drawn from a well-shuffled deck of 52 cards. Find the probability of getting
\begin{enumerate}
\item A king of red colour 
\item A face card 
\item A red face card
\item The jack of hearts
\item A spade
\item The queen of diamonds

\end{enumerate}
\solution
		%\input{ncert/10/15/1/14/main.tex}
	\item Five cards—the ten, jack, queen, king and ace of diamonds, are well-shuffled with their face downwards. One card is then picked up at random.
\begin{enumerate}
\item
What is the probability that the card is the queen? 
\item
If the queen is drawn and put aside, what is the probability that the second card picked up is (a) an ace? (b) a queen?\\
\end{enumerate}
\solution
		%\input{ncert/10/15/1/15/defs.tex}
	\item A bag contains $5$ red balls and some blue balls. If the probability of drawing a blue ball is double that if a red ball, determine the number of blue balls in the bag. 
		\\
\solution
		%\input{ncert/10/15/2/3/defs.tex}
	\item A card is selected from a pack of 52 cards.
 \begin{enumerate}[label=(\alph*)] 
                 \item How many points are there in the sample space?
                 \item Calculate the probability that the card is an ace of spades.
                 \item Calculate the probability that the card is (i) an ace and (ii) black card.
 \end{enumerate}
\solution
		%\input{ncert/11/16/3/4/main.tex}
\item Four cards are drawn from a well-shuffled deck of 52 cards. What is the probability of obtaining 3 diamonds and one spade.
\\
\solution
		%\input{ncert/11/16/4/2/defs.tex}
\item In a certain lottery 10,000 tickets are sold and ten equal prizes are awarded. What is the probability of not getting a prize if you buy (a) one ticket (b) two tickets (c) 10 tickets ?	
\\
\solution
		%\input{ncert/11/16/4/4/defs.tex}
		%
\item 
Out of 100 students, two sections of 40 and 60 are formed. If you and your friend are among the 100 students, what is the probability that
\begin{enumerate}
\item you both enter the same section?
\item you both enter the different sections?
\end{enumerate}
\solution
		%\input{ncert/11/16/4/5/defs.tex}
	\item 
The number lock of a suitcase has 4 wheels each labelled with ten digits i.e. from 0 to 9.The lock opens with a sequence of four digits with no repeats.What is the probability of a person getting the right sequence to open the suitcase.
\\
\solution
		%\input{ncert/11/16/4/10/defs.tex}
		%
\item 
Two cards are drawn at random and without replacement from a pack of 52 playing cards. Find the probability that both the cards are black.
\\
\solution
		%\input{ncert/12/13/2/2/defs.tex}
		\item A box of oranges is inspected by examining three randomly selected oranges drawn without replacement. If all the three oranges are good, the box is approved for sale, otherwise, it is rejected. Find the probability that a box containing 15 oranges out of which 12 are good and 3 are bad ones will be approved for sale.
		\label{ncert/12/13/2/3/defs.tex}
		\item Two balls are drawn at random with replacement from a box containing 10 black and 8 red balls. Find the probability that
		\label{ncert/12/13/2/12}
\begin{enumerate}
\item both balls are red.
\item first ball is black and second is red.
\item one of them is black and other is red.
\end{enumerate}

\item In a hostel, 60\% of the students read Hindi newspaper, 40\% read English newspaper and 20\% read both Hindi and English newspapers. A student is selected at random.
		\label{ncert/12/13/2/15}
\begin{enumerate}
\item Find the probability that she reads neither Hindi nor English newspapers.
\item If she reads Hindi newspaper, find the probability that she reads English newspaper.
\item If she reads English newspaper, find the probability that she reads Hindi newspaper.\\
\end{enumerate}
\item The probability of obtaining an even prime number on each die, when a pair of dice is rolled is 
\begin{enumerate}
    \item $0$ 
    
    \item $\frac{1}{3}$ 
    
    \item $\frac{1}{12}$ 
    
    \item $\frac{1}{36}$ 
\end{enumerate}
\solution
		%\input{ncert/12/13/2/17/defs.tex}
	\item A bag contains 4 red and 4 black balls, another bag contains 2 red and 6 black balls. One of the two bags is selected at random and a ball is drawn from the bag which is found to be red. Find the probability that the ball is drawn from the first bag.
\\
\solution
		%\input{ncert/12/13/3/2/main.tex}
  \item
  Cards with numbers 2 to 101 are placed in a box. A card is selected at random.Find the probability that the card has
\begin{enumerate}[label=(\roman*)]
	\item an even number 
	\item a square number
\end{enumerate}
\solution
%\input{exemplar/10/13/3/32/main.tex}
\item
The king, queen and jack of clubs are removed from a deck of 52 playing cards and then well shuffled. Now one card is drawn at random from the remaining cards.  Determine the probability that the card is
\begin{enumerate}[label=(\roman*)]
\item a club
\item 10 of hearts
\end{enumerate}
\solution
%\input{exemplar/10/13/3/29/main.tex}
\item A team of medical students doing their internship have to assist during surgeries
at a city hospital. The probabilities of surgeries rated as very complex, complex,
routine, simple or very simple are respectively, 0.15, 0.20, 0.31, 0.26, .08. Find
the probabilities that a particular surgery will be rated
\begin{enumerate}
	\item complex or very complex;
	\item neither very complex nor very simple;
	\item routine or complex
	\item routine or simple
\end{enumerate}
\solution
%\input{exemplar/11/16/3/8(1)/main.tex}
\item A card is selected from a pack of 52 cards.
\begin{enumerate}[label=(\alph*)]
    \item How many points are there in the sample space?
    \item Calculate the probability that the card is an ace of spades.
    \item Calculate the probability that the card is (i) an ace and (ii) black card.
\end{enumerate}
\solution
%\input{exemplar/11/16/3/4/main2.tex}
\item The probability that a non leap year selected at random will contain 53 sundays.
\\
\solution
%\input{exemplar/10/13/1/19/main.tex}
\item One of the four persons John, Rita, Aslam or Gurpreet will be promoted next
month. Consequently the sample space consists of four elementary outcomes
S = {John promoted, Rita promoted, Aslam promoted, Gurpreet promoted}
You are told that the chances of John’s promotion is same as that of Gurpreet,
Rita’s chances of promotion are twice as likely as Johns. Aslam’s chances are
four times that of John.
\begin{enumerate}
	\item Determine
	\begin{enumerate}
		\item P (John promoted)
		\item P (Rita promoted)
		\item P (Aslam promoted)
		\item P (Gurpreet promoted)
	\end{enumerate}
	\item If A = {John promoted or Gurpreet promoted}, find P (A).
\end{enumerate}
\solution
%\input{exemplar/11/16/3/10/main.tex}
\item A card is drawn from a deck of 52 cards. Find the probability of getting a king or a heart or a red card.\\
\solution
%\input{exemplar/11/16/3/15/main.tex}
\item The probability that a student will pass his examination is 0.73, the probability of
the student getting a compartment is 0.13, and the probability that the student will
either pass or get compartment is 0.96. State True or False.\\
\solution
%\input{exemplar/11/16/3/31/main.tex}
\item A card is selected from a pack of 52 cards\\
\begin{enumerate}[label=(\alph*)]
\item How many points are there in the sample space?
\item Calculate the probability that the cards is an ace of spades.
\item Calculate the probability that the card is (i) an ace (ii)black card.\\
\end{enumerate}
%\input{ncert/11/16/3/4_1/Prob_4.tex}
\item In a non-leap year, the probability of having 53 tuesdays or 53 wednesdays is\\
\solution
%\input{exemplar/11/16/3/18/main.tex}
\item There are 1000 sealed envelopes in a box, 10 of them contain a cash prize of
Rs 100 each, 100 of them contain a cash prize of Rs 50 each and 200 of them
contain a cash prize of Rs 10 each and rest do not contain any cash prize. If they
are well shuffled and an envelope is picked up out, what is the probability that it
contains no cash prize?\\
\solution
%\input{exemplar/10/13/3/34/main.tex}
\item 
A die is thrown and a card is selected at random from a deck of 52 playing cards. The probability of getting an even number on the die and a spade card.\\
\solution
%\input{exemplar/12/13/3/78/main.tex}
\item
If 4-digit numbers greater than 5,000 are randomly formed from the digits 0, 1, 3, 5, and 7, what is the probability of forming a number divisible by 5 when:
\begin{enumerate}
    \item The digits are repeated?
    \item The repetition of digits is not allowed?
\end{enumerate}
\solution
%\input{ncert/11/16/4/9/main.tex}
\item Consider the probability space $\brak{\Omega, \mathcal{G}, P}$ where $\Omega = [0,2]$ and $\mathcal{G} = \cbrak{\phi, \Omega, [0,1], (1,2]}$. Let $X$ and $Y$ be two functions on $\Omega$ defined as
\begin{align*}
    X(\omega) = 
    \begin{cases}
        1 & \text{if }\omega \in [0, 1]\\
        2 & \text{if }\omega \in (1, 2]
    \end{cases}
\end{align*}
and
\begin{align*}
    Y(\omega) = 
    \begin{cases}
        2 & \text{if }\omega \in [0, 1.5]\\
        3 & \text{if }\omega \in (1.5, 2].
    \end{cases}
\end{align*}
Then which one of the following statements is true?
\begin{enumerate}
    \item [(A)] $X$ is a random variable with respect to $\mathcal{G}$, but $Y$ is not a random variable with respect to $\mathcal{G}$.
    \item [(B)] $Y$ is a random variable with respect to $\mathcal{G}$, but $X$ is not a random variable with respect to $\mathcal{G}$.
    \item [(C)] Neither $X$ nor $Y$ is a random variable with respect to $\mathcal{G}$.
    \item [(D)] Both $X$ and $Y$ are random variables with respect to $\mathcal{G}$.
\end{enumerate} \hfill (GATE ST 2023)\\
\solution
%\input{gate/ST/2023/14/main.tex}
	\item  A die is loaded in such a way that each odd number is twice as likely to occur as
each even number. Find $P(G)$, where $G$ is the event that a number greater than
3 occurs on a single roll of the die.
\\
\solution
		%\input{exemplar/11/16/3/5/main.tex}
	\item All the jacks, queens and kings are removed from a deck of 52 playing cards. The remaining cards are well shuffled and then one card is drawn at random. Giving ace a value 1 similar value for other cards, find the probability that the card has a value 
		\begin{enumerate}
			\item 7
			\item greater than 7
			\item less than 7
		\end{enumerate}
		%\input{exemplar/10/13/3/30/main.tex}
  \item A Lot consists of 48 mobile phones of which 42 are good, 3 have only minor defects and 3 have major defects.Varnika will buy a phone if it is good but the trader will only buy a mobile if it has no major defects. One phone is selected at random from the lot. What is the probability that it is
\begin{enumerate}
	\item acceptable to Varnika?
            \item acceptable to the trader?
\end{enumerate}
\solution
	%\input{exemplar/10/13/3/40/main.tex}
 \item A student says that if you throw a die, it will show up 1 or not 1. Therefore, the probability of getting 1 and the probability of getting 'not 1' each is equal to $\frac{1}{2}$. Is this correct? Give reasons.\\
 \solution
        %\input{exemplar/10/13/2/9/main.tex}
   \item Four candidates A, B, C, D have ap-
plied for the assignment to coach a school cricket
team. If A is twice as likely to be selected as B, and
B and C are given about the same chance of being
selected, while C is twice as likely to be selected
as D, what are the probabilities that
\begin{enumerate}
\item C will be selected?
\item A will not be selected?
\end{enumerate}
	%\input{exemplar/11/16/3/9/main.tex}
 \item A bag contain 24 balls of which $x$ balls are red, $2x$ are white and $3x$ are blue. A ball is selected at random, What is the probability that it is
\begin{enumerate}[label=\alph*)]
\item not red ?
\item white ?
\end{enumerate}
%\input{exemplar/10/13/3/41/main.tex}
If the letters of the word ASSASSINATION are arranged at random. Find the Probability that
\begin{enumerate}[label=(\alph*)]
\item Four $S's$ come consecutively in the word
\item Two  $I's$ and two $N's$ come together
\item All $A's$ are not coming together
\item No two $A's$ are coming together
\end{enumerate}
%\input{exemplar/11/16/3/14/main.tex}
	\item One urn contains two black balls (labelled B1 and B2) and one white ball. A
	second urn contains one black ball and two white balls (labelled W1 and W2).
	Suppose the following experiment is performed. One of the two urns is chosen
	at random. Next a ball is randomly chosen from the urn. Then a second ball is
	chosen at random from the same urn without replacing the first ball.
	
	\begin{enumerate}
	\item What is the probability that two black balls are chosen?
	
	\item What is the probability that two balls of opposite colour are chosen?
	\end{enumerate}
	\solution
	%\input{exemplar/11/16/3/12/main1.tex}
\end{enumerate}

	\item A card is selected from a pack of 52 cards.
 \begin{enumerate}[label=(\alph*)] 
                 \item How many points are there in the sample space?
                 \item Calculate the probability that the card is an ace of spades.
                 \item Calculate the probability that the card is (i) an ace and (ii) black card.
 \end{enumerate}
\solution
		%\begin{table}[H]
	\centering
\begin{tabular}{|c|c|c|}
\hline
Random variable &Value &Definition\\ \hline
\multirow{3}{*}{X} &0 &Slips of Rs 1\\
&1 &Slips of Rs 5\\
&2 &Slips of Rs 13\\ \hline
\multirow{2}{*}{Y} &0 &Box A\\
&1 &Box B\\\hline
\end{tabular}
\caption{}
\label{tab:Distribution}
\end{table}
See \tabref{tab:Distribution}.
\begin{align}
p_{Y}\brak{k}= \begin{cases} 
      \frac{1}{3} & {k=0} \\
      \frac{2}{3 }& {k=1} 
   \end{cases}
   \\
p_{Y|X}\brak{0|0} = \frac{19}{25}\, 
p_{Y|X}\brak{0|1} = \frac{6}{25}\,
p_{Y|X}\brak{1|0} = \frac{45}{50}\,
p_{Y|X}\brak{1|2} = \frac{5}{50}
\end{align}
The desired probability is the probability that a slip drawn at random is marked other than Rs 1,
\begin{align}
&=1-p_X\brak{0}\\
&= p_X(1) + p_X(2)
\end{align}
Using Bayes theorem,
\begin{align}
&= p_Y\brak{0} \times \pr{Y=0 | X=1} + p_Y\brak{1} \times \pr{Y=1|X=2}\\
&=\frac{1}{3} \times \frac{6}{25} + \frac{2}{3} \times \frac{5}{50}\\
&=\frac{11}{75}
\end{align}

\newpage

%\tableofcontents

\bigskip

\renewcommand{\thefigure}{\theenumi}
\renewcommand{\thetable}{\theenumi}
%\renewcommand{\theequation}{\theenumi}

%\begin{abstract}
%%\boldmath
%In this letter, an algorithm for evaluating the exact analytical bit error rate  (BER)  for the piecewise linear (PL) combiner for  multiple relays is presented. Previous results were available only for upto three relays. The algorithm is unique in the sense that  the actual mathematical expressions, that are prohibitively large, need not be explicitly obtained. The diversity gain due to multiple relays is shown through plots of the analytical BER, well supported by simulations. 
%
%\end{abstract}
% IEEEtran.cls defaults to using nonbold math in the Abstract.
% This preserves the distinction between vectors and scalars. However,
% if the journal you are submitting to favors bold math in the abstract,
% then you can use LaTeX's standard command \boldmath at the very start
% of the abstract to achieve this. Many IEEE journals frown on math
% in the abstract anyway.

% Note that keywords are not normally used for peerreview papers.
%\begin{IEEEkeywords}
%Cooperative diversity, decode and forward, piecewise linear
%\end{IEEEkeywords}



% For peer review papers, you can put extra information on the cover
% page as needed:
% \ifCLASSOPTIONpeerreview
% \begin{center} \bfseries EDICS Category: 3-BBND \end{center}
% \fi
%
% For peerreview papers, this IEEEtran command inserts a page break and
% creates the second title. It will be ignored for other modes.
%\IEEEpeerreviewmaketitle




\item Four cards are drawn from a well-shuffled deck of 52 cards. What is the probability of obtaining 3 diamonds and one spade.
\\
\solution
		%\begin{enumerate}[label=\thesection.\arabic*,ref=\thesection.\theenumi]
	\item One card is drawn from a well-shuffled deck of 52 cards. Find the probability of getting
\begin{enumerate}
\item A king of red colour 
\item A face card 
\item A red face card
\item The jack of hearts
\item A spade
\item The queen of diamonds

\end{enumerate}
\solution
		%\input{ncert/10/15/1/14/main.tex}
	\item Five cards—the ten, jack, queen, king and ace of diamonds, are well-shuffled with their face downwards. One card is then picked up at random.
\begin{enumerate}
\item
What is the probability that the card is the queen? 
\item
If the queen is drawn and put aside, what is the probability that the second card picked up is (a) an ace? (b) a queen?\\
\end{enumerate}
\solution
		%\input{ncert/10/15/1/15/defs.tex}
	\item A bag contains $5$ red balls and some blue balls. If the probability of drawing a blue ball is double that if a red ball, determine the number of blue balls in the bag. 
		\\
\solution
		%\input{ncert/10/15/2/3/defs.tex}
	\item A card is selected from a pack of 52 cards.
 \begin{enumerate}[label=(\alph*)] 
                 \item How many points are there in the sample space?
                 \item Calculate the probability that the card is an ace of spades.
                 \item Calculate the probability that the card is (i) an ace and (ii) black card.
 \end{enumerate}
\solution
		%\input{ncert/11/16/3/4/main.tex}
\item Four cards are drawn from a well-shuffled deck of 52 cards. What is the probability of obtaining 3 diamonds and one spade.
\\
\solution
		%\input{ncert/11/16/4/2/defs.tex}
\item In a certain lottery 10,000 tickets are sold and ten equal prizes are awarded. What is the probability of not getting a prize if you buy (a) one ticket (b) two tickets (c) 10 tickets ?	
\\
\solution
		%\input{ncert/11/16/4/4/defs.tex}
		%
\item 
Out of 100 students, two sections of 40 and 60 are formed. If you and your friend are among the 100 students, what is the probability that
\begin{enumerate}
\item you both enter the same section?
\item you both enter the different sections?
\end{enumerate}
\solution
		%\input{ncert/11/16/4/5/defs.tex}
	\item 
The number lock of a suitcase has 4 wheels each labelled with ten digits i.e. from 0 to 9.The lock opens with a sequence of four digits with no repeats.What is the probability of a person getting the right sequence to open the suitcase.
\\
\solution
		%\input{ncert/11/16/4/10/defs.tex}
		%
\item 
Two cards are drawn at random and without replacement from a pack of 52 playing cards. Find the probability that both the cards are black.
\\
\solution
		%\input{ncert/12/13/2/2/defs.tex}
		\item A box of oranges is inspected by examining three randomly selected oranges drawn without replacement. If all the three oranges are good, the box is approved for sale, otherwise, it is rejected. Find the probability that a box containing 15 oranges out of which 12 are good and 3 are bad ones will be approved for sale.
		\label{ncert/12/13/2/3/defs.tex}
		\item Two balls are drawn at random with replacement from a box containing 10 black and 8 red balls. Find the probability that
		\label{ncert/12/13/2/12}
\begin{enumerate}
\item both balls are red.
\item first ball is black and second is red.
\item one of them is black and other is red.
\end{enumerate}

\item In a hostel, 60\% of the students read Hindi newspaper, 40\% read English newspaper and 20\% read both Hindi and English newspapers. A student is selected at random.
		\label{ncert/12/13/2/15}
\begin{enumerate}
\item Find the probability that she reads neither Hindi nor English newspapers.
\item If she reads Hindi newspaper, find the probability that she reads English newspaper.
\item If she reads English newspaper, find the probability that she reads Hindi newspaper.\\
\end{enumerate}
\item The probability of obtaining an even prime number on each die, when a pair of dice is rolled is 
\begin{enumerate}
    \item $0$ 
    
    \item $\frac{1}{3}$ 
    
    \item $\frac{1}{12}$ 
    
    \item $\frac{1}{36}$ 
\end{enumerate}
\solution
		%\input{ncert/12/13/2/17/defs.tex}
	\item A bag contains 4 red and 4 black balls, another bag contains 2 red and 6 black balls. One of the two bags is selected at random and a ball is drawn from the bag which is found to be red. Find the probability that the ball is drawn from the first bag.
\\
\solution
		%\input{ncert/12/13/3/2/main.tex}
  \item
  Cards with numbers 2 to 101 are placed in a box. A card is selected at random.Find the probability that the card has
\begin{enumerate}[label=(\roman*)]
	\item an even number 
	\item a square number
\end{enumerate}
\solution
%\input{exemplar/10/13/3/32/main.tex}
\item
The king, queen and jack of clubs are removed from a deck of 52 playing cards and then well shuffled. Now one card is drawn at random from the remaining cards.  Determine the probability that the card is
\begin{enumerate}[label=(\roman*)]
\item a club
\item 10 of hearts
\end{enumerate}
\solution
%\input{exemplar/10/13/3/29/main.tex}
\item A team of medical students doing their internship have to assist during surgeries
at a city hospital. The probabilities of surgeries rated as very complex, complex,
routine, simple or very simple are respectively, 0.15, 0.20, 0.31, 0.26, .08. Find
the probabilities that a particular surgery will be rated
\begin{enumerate}
	\item complex or very complex;
	\item neither very complex nor very simple;
	\item routine or complex
	\item routine or simple
\end{enumerate}
\solution
%\input{exemplar/11/16/3/8(1)/main.tex}
\item A card is selected from a pack of 52 cards.
\begin{enumerate}[label=(\alph*)]
    \item How many points are there in the sample space?
    \item Calculate the probability that the card is an ace of spades.
    \item Calculate the probability that the card is (i) an ace and (ii) black card.
\end{enumerate}
\solution
%\input{exemplar/11/16/3/4/main2.tex}
\item The probability that a non leap year selected at random will contain 53 sundays.
\\
\solution
%\input{exemplar/10/13/1/19/main.tex}
\item One of the four persons John, Rita, Aslam or Gurpreet will be promoted next
month. Consequently the sample space consists of four elementary outcomes
S = {John promoted, Rita promoted, Aslam promoted, Gurpreet promoted}
You are told that the chances of John’s promotion is same as that of Gurpreet,
Rita’s chances of promotion are twice as likely as Johns. Aslam’s chances are
four times that of John.
\begin{enumerate}
	\item Determine
	\begin{enumerate}
		\item P (John promoted)
		\item P (Rita promoted)
		\item P (Aslam promoted)
		\item P (Gurpreet promoted)
	\end{enumerate}
	\item If A = {John promoted or Gurpreet promoted}, find P (A).
\end{enumerate}
\solution
%\input{exemplar/11/16/3/10/main.tex}
\item A card is drawn from a deck of 52 cards. Find the probability of getting a king or a heart or a red card.\\
\solution
%\input{exemplar/11/16/3/15/main.tex}
\item The probability that a student will pass his examination is 0.73, the probability of
the student getting a compartment is 0.13, and the probability that the student will
either pass or get compartment is 0.96. State True or False.\\
\solution
%\input{exemplar/11/16/3/31/main.tex}
\item A card is selected from a pack of 52 cards\\
\begin{enumerate}[label=(\alph*)]
\item How many points are there in the sample space?
\item Calculate the probability that the cards is an ace of spades.
\item Calculate the probability that the card is (i) an ace (ii)black card.\\
\end{enumerate}
%\input{ncert/11/16/3/4_1/Prob_4.tex}
\item In a non-leap year, the probability of having 53 tuesdays or 53 wednesdays is\\
\solution
%\input{exemplar/11/16/3/18/main.tex}
\item There are 1000 sealed envelopes in a box, 10 of them contain a cash prize of
Rs 100 each, 100 of them contain a cash prize of Rs 50 each and 200 of them
contain a cash prize of Rs 10 each and rest do not contain any cash prize. If they
are well shuffled and an envelope is picked up out, what is the probability that it
contains no cash prize?\\
\solution
%\input{exemplar/10/13/3/34/main.tex}
\item 
A die is thrown and a card is selected at random from a deck of 52 playing cards. The probability of getting an even number on the die and a spade card.\\
\solution
%\input{exemplar/12/13/3/78/main.tex}
\item
If 4-digit numbers greater than 5,000 are randomly formed from the digits 0, 1, 3, 5, and 7, what is the probability of forming a number divisible by 5 when:
\begin{enumerate}
    \item The digits are repeated?
    \item The repetition of digits is not allowed?
\end{enumerate}
\solution
%\input{ncert/11/16/4/9/main.tex}
\item Consider the probability space $\brak{\Omega, \mathcal{G}, P}$ where $\Omega = [0,2]$ and $\mathcal{G} = \cbrak{\phi, \Omega, [0,1], (1,2]}$. Let $X$ and $Y$ be two functions on $\Omega$ defined as
\begin{align*}
    X(\omega) = 
    \begin{cases}
        1 & \text{if }\omega \in [0, 1]\\
        2 & \text{if }\omega \in (1, 2]
    \end{cases}
\end{align*}
and
\begin{align*}
    Y(\omega) = 
    \begin{cases}
        2 & \text{if }\omega \in [0, 1.5]\\
        3 & \text{if }\omega \in (1.5, 2].
    \end{cases}
\end{align*}
Then which one of the following statements is true?
\begin{enumerate}
    \item [(A)] $X$ is a random variable with respect to $\mathcal{G}$, but $Y$ is not a random variable with respect to $\mathcal{G}$.
    \item [(B)] $Y$ is a random variable with respect to $\mathcal{G}$, but $X$ is not a random variable with respect to $\mathcal{G}$.
    \item [(C)] Neither $X$ nor $Y$ is a random variable with respect to $\mathcal{G}$.
    \item [(D)] Both $X$ and $Y$ are random variables with respect to $\mathcal{G}$.
\end{enumerate} \hfill (GATE ST 2023)\\
\solution
%\input{gate/ST/2023/14/main.tex}
	\item  A die is loaded in such a way that each odd number is twice as likely to occur as
each even number. Find $P(G)$, where $G$ is the event that a number greater than
3 occurs on a single roll of the die.
\\
\solution
		%\input{exemplar/11/16/3/5/main.tex}
	\item All the jacks, queens and kings are removed from a deck of 52 playing cards. The remaining cards are well shuffled and then one card is drawn at random. Giving ace a value 1 similar value for other cards, find the probability that the card has a value 
		\begin{enumerate}
			\item 7
			\item greater than 7
			\item less than 7
		\end{enumerate}
		%\input{exemplar/10/13/3/30/main.tex}
  \item A Lot consists of 48 mobile phones of which 42 are good, 3 have only minor defects and 3 have major defects.Varnika will buy a phone if it is good but the trader will only buy a mobile if it has no major defects. One phone is selected at random from the lot. What is the probability that it is
\begin{enumerate}
	\item acceptable to Varnika?
            \item acceptable to the trader?
\end{enumerate}
\solution
	%\input{exemplar/10/13/3/40/main.tex}
 \item A student says that if you throw a die, it will show up 1 or not 1. Therefore, the probability of getting 1 and the probability of getting 'not 1' each is equal to $\frac{1}{2}$. Is this correct? Give reasons.\\
 \solution
        %\input{exemplar/10/13/2/9/main.tex}
   \item Four candidates A, B, C, D have ap-
plied for the assignment to coach a school cricket
team. If A is twice as likely to be selected as B, and
B and C are given about the same chance of being
selected, while C is twice as likely to be selected
as D, what are the probabilities that
\begin{enumerate}
\item C will be selected?
\item A will not be selected?
\end{enumerate}
	%\input{exemplar/11/16/3/9/main.tex}
 \item A bag contain 24 balls of which $x$ balls are red, $2x$ are white and $3x$ are blue. A ball is selected at random, What is the probability that it is
\begin{enumerate}[label=\alph*)]
\item not red ?
\item white ?
\end{enumerate}
%\input{exemplar/10/13/3/41/main.tex}
If the letters of the word ASSASSINATION are arranged at random. Find the Probability that
\begin{enumerate}[label=(\alph*)]
\item Four $S's$ come consecutively in the word
\item Two  $I's$ and two $N's$ come together
\item All $A's$ are not coming together
\item No two $A's$ are coming together
\end{enumerate}
%\input{exemplar/11/16/3/14/main.tex}
	\item One urn contains two black balls (labelled B1 and B2) and one white ball. A
	second urn contains one black ball and two white balls (labelled W1 and W2).
	Suppose the following experiment is performed. One of the two urns is chosen
	at random. Next a ball is randomly chosen from the urn. Then a second ball is
	chosen at random from the same urn without replacing the first ball.
	
	\begin{enumerate}
	\item What is the probability that two black balls are chosen?
	
	\item What is the probability that two balls of opposite colour are chosen?
	\end{enumerate}
	\solution
	%\input{exemplar/11/16/3/12/main1.tex}
\end{enumerate}

\item In a certain lottery 10,000 tickets are sold and ten equal prizes are awarded. What is the probability of not getting a prize if you buy (a) one ticket (b) two tickets (c) 10 tickets ?	
\\
\solution
		%\begin{enumerate}[label=\thesection.\arabic*,ref=\thesection.\theenumi]
	\item One card is drawn from a well-shuffled deck of 52 cards. Find the probability of getting
\begin{enumerate}
\item A king of red colour 
\item A face card 
\item A red face card
\item The jack of hearts
\item A spade
\item The queen of diamonds

\end{enumerate}
\solution
		%\input{ncert/10/15/1/14/main.tex}
	\item Five cards—the ten, jack, queen, king and ace of diamonds, are well-shuffled with their face downwards. One card is then picked up at random.
\begin{enumerate}
\item
What is the probability that the card is the queen? 
\item
If the queen is drawn and put aside, what is the probability that the second card picked up is (a) an ace? (b) a queen?\\
\end{enumerate}
\solution
		%\input{ncert/10/15/1/15/defs.tex}
	\item A bag contains $5$ red balls and some blue balls. If the probability of drawing a blue ball is double that if a red ball, determine the number of blue balls in the bag. 
		\\
\solution
		%\input{ncert/10/15/2/3/defs.tex}
	\item A card is selected from a pack of 52 cards.
 \begin{enumerate}[label=(\alph*)] 
                 \item How many points are there in the sample space?
                 \item Calculate the probability that the card is an ace of spades.
                 \item Calculate the probability that the card is (i) an ace and (ii) black card.
 \end{enumerate}
\solution
		%\input{ncert/11/16/3/4/main.tex}
\item Four cards are drawn from a well-shuffled deck of 52 cards. What is the probability of obtaining 3 diamonds and one spade.
\\
\solution
		%\input{ncert/11/16/4/2/defs.tex}
\item In a certain lottery 10,000 tickets are sold and ten equal prizes are awarded. What is the probability of not getting a prize if you buy (a) one ticket (b) two tickets (c) 10 tickets ?	
\\
\solution
		%\input{ncert/11/16/4/4/defs.tex}
		%
\item 
Out of 100 students, two sections of 40 and 60 are formed. If you and your friend are among the 100 students, what is the probability that
\begin{enumerate}
\item you both enter the same section?
\item you both enter the different sections?
\end{enumerate}
\solution
		%\input{ncert/11/16/4/5/defs.tex}
	\item 
The number lock of a suitcase has 4 wheels each labelled with ten digits i.e. from 0 to 9.The lock opens with a sequence of four digits with no repeats.What is the probability of a person getting the right sequence to open the suitcase.
\\
\solution
		%\input{ncert/11/16/4/10/defs.tex}
		%
\item 
Two cards are drawn at random and without replacement from a pack of 52 playing cards. Find the probability that both the cards are black.
\\
\solution
		%\input{ncert/12/13/2/2/defs.tex}
		\item A box of oranges is inspected by examining three randomly selected oranges drawn without replacement. If all the three oranges are good, the box is approved for sale, otherwise, it is rejected. Find the probability that a box containing 15 oranges out of which 12 are good and 3 are bad ones will be approved for sale.
		\label{ncert/12/13/2/3/defs.tex}
		\item Two balls are drawn at random with replacement from a box containing 10 black and 8 red balls. Find the probability that
		\label{ncert/12/13/2/12}
\begin{enumerate}
\item both balls are red.
\item first ball is black and second is red.
\item one of them is black and other is red.
\end{enumerate}

\item In a hostel, 60\% of the students read Hindi newspaper, 40\% read English newspaper and 20\% read both Hindi and English newspapers. A student is selected at random.
		\label{ncert/12/13/2/15}
\begin{enumerate}
\item Find the probability that she reads neither Hindi nor English newspapers.
\item If she reads Hindi newspaper, find the probability that she reads English newspaper.
\item If she reads English newspaper, find the probability that she reads Hindi newspaper.\\
\end{enumerate}
\item The probability of obtaining an even prime number on each die, when a pair of dice is rolled is 
\begin{enumerate}
    \item $0$ 
    
    \item $\frac{1}{3}$ 
    
    \item $\frac{1}{12}$ 
    
    \item $\frac{1}{36}$ 
\end{enumerate}
\solution
		%\input{ncert/12/13/2/17/defs.tex}
	\item A bag contains 4 red and 4 black balls, another bag contains 2 red and 6 black balls. One of the two bags is selected at random and a ball is drawn from the bag which is found to be red. Find the probability that the ball is drawn from the first bag.
\\
\solution
		%\input{ncert/12/13/3/2/main.tex}
  \item
  Cards with numbers 2 to 101 are placed in a box. A card is selected at random.Find the probability that the card has
\begin{enumerate}[label=(\roman*)]
	\item an even number 
	\item a square number
\end{enumerate}
\solution
%\input{exemplar/10/13/3/32/main.tex}
\item
The king, queen and jack of clubs are removed from a deck of 52 playing cards and then well shuffled. Now one card is drawn at random from the remaining cards.  Determine the probability that the card is
\begin{enumerate}[label=(\roman*)]
\item a club
\item 10 of hearts
\end{enumerate}
\solution
%\input{exemplar/10/13/3/29/main.tex}
\item A team of medical students doing their internship have to assist during surgeries
at a city hospital. The probabilities of surgeries rated as very complex, complex,
routine, simple or very simple are respectively, 0.15, 0.20, 0.31, 0.26, .08. Find
the probabilities that a particular surgery will be rated
\begin{enumerate}
	\item complex or very complex;
	\item neither very complex nor very simple;
	\item routine or complex
	\item routine or simple
\end{enumerate}
\solution
%\input{exemplar/11/16/3/8(1)/main.tex}
\item A card is selected from a pack of 52 cards.
\begin{enumerate}[label=(\alph*)]
    \item How many points are there in the sample space?
    \item Calculate the probability that the card is an ace of spades.
    \item Calculate the probability that the card is (i) an ace and (ii) black card.
\end{enumerate}
\solution
%\input{exemplar/11/16/3/4/main2.tex}
\item The probability that a non leap year selected at random will contain 53 sundays.
\\
\solution
%\input{exemplar/10/13/1/19/main.tex}
\item One of the four persons John, Rita, Aslam or Gurpreet will be promoted next
month. Consequently the sample space consists of four elementary outcomes
S = {John promoted, Rita promoted, Aslam promoted, Gurpreet promoted}
You are told that the chances of John’s promotion is same as that of Gurpreet,
Rita’s chances of promotion are twice as likely as Johns. Aslam’s chances are
four times that of John.
\begin{enumerate}
	\item Determine
	\begin{enumerate}
		\item P (John promoted)
		\item P (Rita promoted)
		\item P (Aslam promoted)
		\item P (Gurpreet promoted)
	\end{enumerate}
	\item If A = {John promoted or Gurpreet promoted}, find P (A).
\end{enumerate}
\solution
%\input{exemplar/11/16/3/10/main.tex}
\item A card is drawn from a deck of 52 cards. Find the probability of getting a king or a heart or a red card.\\
\solution
%\input{exemplar/11/16/3/15/main.tex}
\item The probability that a student will pass his examination is 0.73, the probability of
the student getting a compartment is 0.13, and the probability that the student will
either pass or get compartment is 0.96. State True or False.\\
\solution
%\input{exemplar/11/16/3/31/main.tex}
\item A card is selected from a pack of 52 cards\\
\begin{enumerate}[label=(\alph*)]
\item How many points are there in the sample space?
\item Calculate the probability that the cards is an ace of spades.
\item Calculate the probability that the card is (i) an ace (ii)black card.\\
\end{enumerate}
%\input{ncert/11/16/3/4_1/Prob_4.tex}
\item In a non-leap year, the probability of having 53 tuesdays or 53 wednesdays is\\
\solution
%\input{exemplar/11/16/3/18/main.tex}
\item There are 1000 sealed envelopes in a box, 10 of them contain a cash prize of
Rs 100 each, 100 of them contain a cash prize of Rs 50 each and 200 of them
contain a cash prize of Rs 10 each and rest do not contain any cash prize. If they
are well shuffled and an envelope is picked up out, what is the probability that it
contains no cash prize?\\
\solution
%\input{exemplar/10/13/3/34/main.tex}
\item 
A die is thrown and a card is selected at random from a deck of 52 playing cards. The probability of getting an even number on the die and a spade card.\\
\solution
%\input{exemplar/12/13/3/78/main.tex}
\item
If 4-digit numbers greater than 5,000 are randomly formed from the digits 0, 1, 3, 5, and 7, what is the probability of forming a number divisible by 5 when:
\begin{enumerate}
    \item The digits are repeated?
    \item The repetition of digits is not allowed?
\end{enumerate}
\solution
%\input{ncert/11/16/4/9/main.tex}
\item Consider the probability space $\brak{\Omega, \mathcal{G}, P}$ where $\Omega = [0,2]$ and $\mathcal{G} = \cbrak{\phi, \Omega, [0,1], (1,2]}$. Let $X$ and $Y$ be two functions on $\Omega$ defined as
\begin{align*}
    X(\omega) = 
    \begin{cases}
        1 & \text{if }\omega \in [0, 1]\\
        2 & \text{if }\omega \in (1, 2]
    \end{cases}
\end{align*}
and
\begin{align*}
    Y(\omega) = 
    \begin{cases}
        2 & \text{if }\omega \in [0, 1.5]\\
        3 & \text{if }\omega \in (1.5, 2].
    \end{cases}
\end{align*}
Then which one of the following statements is true?
\begin{enumerate}
    \item [(A)] $X$ is a random variable with respect to $\mathcal{G}$, but $Y$ is not a random variable with respect to $\mathcal{G}$.
    \item [(B)] $Y$ is a random variable with respect to $\mathcal{G}$, but $X$ is not a random variable with respect to $\mathcal{G}$.
    \item [(C)] Neither $X$ nor $Y$ is a random variable with respect to $\mathcal{G}$.
    \item [(D)] Both $X$ and $Y$ are random variables with respect to $\mathcal{G}$.
\end{enumerate} \hfill (GATE ST 2023)\\
\solution
%\input{gate/ST/2023/14/main.tex}
	\item  A die is loaded in such a way that each odd number is twice as likely to occur as
each even number. Find $P(G)$, where $G$ is the event that a number greater than
3 occurs on a single roll of the die.
\\
\solution
		%\input{exemplar/11/16/3/5/main.tex}
	\item All the jacks, queens and kings are removed from a deck of 52 playing cards. The remaining cards are well shuffled and then one card is drawn at random. Giving ace a value 1 similar value for other cards, find the probability that the card has a value 
		\begin{enumerate}
			\item 7
			\item greater than 7
			\item less than 7
		\end{enumerate}
		%\input{exemplar/10/13/3/30/main.tex}
  \item A Lot consists of 48 mobile phones of which 42 are good, 3 have only minor defects and 3 have major defects.Varnika will buy a phone if it is good but the trader will only buy a mobile if it has no major defects. One phone is selected at random from the lot. What is the probability that it is
\begin{enumerate}
	\item acceptable to Varnika?
            \item acceptable to the trader?
\end{enumerate}
\solution
	%\input{exemplar/10/13/3/40/main.tex}
 \item A student says that if you throw a die, it will show up 1 or not 1. Therefore, the probability of getting 1 and the probability of getting 'not 1' each is equal to $\frac{1}{2}$. Is this correct? Give reasons.\\
 \solution
        %\input{exemplar/10/13/2/9/main.tex}
   \item Four candidates A, B, C, D have ap-
plied for the assignment to coach a school cricket
team. If A is twice as likely to be selected as B, and
B and C are given about the same chance of being
selected, while C is twice as likely to be selected
as D, what are the probabilities that
\begin{enumerate}
\item C will be selected?
\item A will not be selected?
\end{enumerate}
	%\input{exemplar/11/16/3/9/main.tex}
 \item A bag contain 24 balls of which $x$ balls are red, $2x$ are white and $3x$ are blue. A ball is selected at random, What is the probability that it is
\begin{enumerate}[label=\alph*)]
\item not red ?
\item white ?
\end{enumerate}
%\input{exemplar/10/13/3/41/main.tex}
If the letters of the word ASSASSINATION are arranged at random. Find the Probability that
\begin{enumerate}[label=(\alph*)]
\item Four $S's$ come consecutively in the word
\item Two  $I's$ and two $N's$ come together
\item All $A's$ are not coming together
\item No two $A's$ are coming together
\end{enumerate}
%\input{exemplar/11/16/3/14/main.tex}
	\item One urn contains two black balls (labelled B1 and B2) and one white ball. A
	second urn contains one black ball and two white balls (labelled W1 and W2).
	Suppose the following experiment is performed. One of the two urns is chosen
	at random. Next a ball is randomly chosen from the urn. Then a second ball is
	chosen at random from the same urn without replacing the first ball.
	
	\begin{enumerate}
	\item What is the probability that two black balls are chosen?
	
	\item What is the probability that two balls of opposite colour are chosen?
	\end{enumerate}
	\solution
	%\input{exemplar/11/16/3/12/main1.tex}
\end{enumerate}

		%
\item 
Out of 100 students, two sections of 40 and 60 are formed. If you and your friend are among the 100 students, what is the probability that
\begin{enumerate}
\item you both enter the same section?
\item you both enter the different sections?
\end{enumerate}
\solution
		%\begin{enumerate}[label=\thesection.\arabic*,ref=\thesection.\theenumi]
	\item One card is drawn from a well-shuffled deck of 52 cards. Find the probability of getting
\begin{enumerate}
\item A king of red colour 
\item A face card 
\item A red face card
\item The jack of hearts
\item A spade
\item The queen of diamonds

\end{enumerate}
\solution
		%\input{ncert/10/15/1/14/main.tex}
	\item Five cards—the ten, jack, queen, king and ace of diamonds, are well-shuffled with their face downwards. One card is then picked up at random.
\begin{enumerate}
\item
What is the probability that the card is the queen? 
\item
If the queen is drawn and put aside, what is the probability that the second card picked up is (a) an ace? (b) a queen?\\
\end{enumerate}
\solution
		%\input{ncert/10/15/1/15/defs.tex}
	\item A bag contains $5$ red balls and some blue balls. If the probability of drawing a blue ball is double that if a red ball, determine the number of blue balls in the bag. 
		\\
\solution
		%\input{ncert/10/15/2/3/defs.tex}
	\item A card is selected from a pack of 52 cards.
 \begin{enumerate}[label=(\alph*)] 
                 \item How many points are there in the sample space?
                 \item Calculate the probability that the card is an ace of spades.
                 \item Calculate the probability that the card is (i) an ace and (ii) black card.
 \end{enumerate}
\solution
		%\input{ncert/11/16/3/4/main.tex}
\item Four cards are drawn from a well-shuffled deck of 52 cards. What is the probability of obtaining 3 diamonds and one spade.
\\
\solution
		%\input{ncert/11/16/4/2/defs.tex}
\item In a certain lottery 10,000 tickets are sold and ten equal prizes are awarded. What is the probability of not getting a prize if you buy (a) one ticket (b) two tickets (c) 10 tickets ?	
\\
\solution
		%\input{ncert/11/16/4/4/defs.tex}
		%
\item 
Out of 100 students, two sections of 40 and 60 are formed. If you and your friend are among the 100 students, what is the probability that
\begin{enumerate}
\item you both enter the same section?
\item you both enter the different sections?
\end{enumerate}
\solution
		%\input{ncert/11/16/4/5/defs.tex}
	\item 
The number lock of a suitcase has 4 wheels each labelled with ten digits i.e. from 0 to 9.The lock opens with a sequence of four digits with no repeats.What is the probability of a person getting the right sequence to open the suitcase.
\\
\solution
		%\input{ncert/11/16/4/10/defs.tex}
		%
\item 
Two cards are drawn at random and without replacement from a pack of 52 playing cards. Find the probability that both the cards are black.
\\
\solution
		%\input{ncert/12/13/2/2/defs.tex}
		\item A box of oranges is inspected by examining three randomly selected oranges drawn without replacement. If all the three oranges are good, the box is approved for sale, otherwise, it is rejected. Find the probability that a box containing 15 oranges out of which 12 are good and 3 are bad ones will be approved for sale.
		\label{ncert/12/13/2/3/defs.tex}
		\item Two balls are drawn at random with replacement from a box containing 10 black and 8 red balls. Find the probability that
		\label{ncert/12/13/2/12}
\begin{enumerate}
\item both balls are red.
\item first ball is black and second is red.
\item one of them is black and other is red.
\end{enumerate}

\item In a hostel, 60\% of the students read Hindi newspaper, 40\% read English newspaper and 20\% read both Hindi and English newspapers. A student is selected at random.
		\label{ncert/12/13/2/15}
\begin{enumerate}
\item Find the probability that she reads neither Hindi nor English newspapers.
\item If she reads Hindi newspaper, find the probability that she reads English newspaper.
\item If she reads English newspaper, find the probability that she reads Hindi newspaper.\\
\end{enumerate}
\item The probability of obtaining an even prime number on each die, when a pair of dice is rolled is 
\begin{enumerate}
    \item $0$ 
    
    \item $\frac{1}{3}$ 
    
    \item $\frac{1}{12}$ 
    
    \item $\frac{1}{36}$ 
\end{enumerate}
\solution
		%\input{ncert/12/13/2/17/defs.tex}
	\item A bag contains 4 red and 4 black balls, another bag contains 2 red and 6 black balls. One of the two bags is selected at random and a ball is drawn from the bag which is found to be red. Find the probability that the ball is drawn from the first bag.
\\
\solution
		%\input{ncert/12/13/3/2/main.tex}
  \item
  Cards with numbers 2 to 101 are placed in a box. A card is selected at random.Find the probability that the card has
\begin{enumerate}[label=(\roman*)]
	\item an even number 
	\item a square number
\end{enumerate}
\solution
%\input{exemplar/10/13/3/32/main.tex}
\item
The king, queen and jack of clubs are removed from a deck of 52 playing cards and then well shuffled. Now one card is drawn at random from the remaining cards.  Determine the probability that the card is
\begin{enumerate}[label=(\roman*)]
\item a club
\item 10 of hearts
\end{enumerate}
\solution
%\input{exemplar/10/13/3/29/main.tex}
\item A team of medical students doing their internship have to assist during surgeries
at a city hospital. The probabilities of surgeries rated as very complex, complex,
routine, simple or very simple are respectively, 0.15, 0.20, 0.31, 0.26, .08. Find
the probabilities that a particular surgery will be rated
\begin{enumerate}
	\item complex or very complex;
	\item neither very complex nor very simple;
	\item routine or complex
	\item routine or simple
\end{enumerate}
\solution
%\input{exemplar/11/16/3/8(1)/main.tex}
\item A card is selected from a pack of 52 cards.
\begin{enumerate}[label=(\alph*)]
    \item How many points are there in the sample space?
    \item Calculate the probability that the card is an ace of spades.
    \item Calculate the probability that the card is (i) an ace and (ii) black card.
\end{enumerate}
\solution
%\input{exemplar/11/16/3/4/main2.tex}
\item The probability that a non leap year selected at random will contain 53 sundays.
\\
\solution
%\input{exemplar/10/13/1/19/main.tex}
\item One of the four persons John, Rita, Aslam or Gurpreet will be promoted next
month. Consequently the sample space consists of four elementary outcomes
S = {John promoted, Rita promoted, Aslam promoted, Gurpreet promoted}
You are told that the chances of John’s promotion is same as that of Gurpreet,
Rita’s chances of promotion are twice as likely as Johns. Aslam’s chances are
four times that of John.
\begin{enumerate}
	\item Determine
	\begin{enumerate}
		\item P (John promoted)
		\item P (Rita promoted)
		\item P (Aslam promoted)
		\item P (Gurpreet promoted)
	\end{enumerate}
	\item If A = {John promoted or Gurpreet promoted}, find P (A).
\end{enumerate}
\solution
%\input{exemplar/11/16/3/10/main.tex}
\item A card is drawn from a deck of 52 cards. Find the probability of getting a king or a heart or a red card.\\
\solution
%\input{exemplar/11/16/3/15/main.tex}
\item The probability that a student will pass his examination is 0.73, the probability of
the student getting a compartment is 0.13, and the probability that the student will
either pass or get compartment is 0.96. State True or False.\\
\solution
%\input{exemplar/11/16/3/31/main.tex}
\item A card is selected from a pack of 52 cards\\
\begin{enumerate}[label=(\alph*)]
\item How many points are there in the sample space?
\item Calculate the probability that the cards is an ace of spades.
\item Calculate the probability that the card is (i) an ace (ii)black card.\\
\end{enumerate}
%\input{ncert/11/16/3/4_1/Prob_4.tex}
\item In a non-leap year, the probability of having 53 tuesdays or 53 wednesdays is\\
\solution
%\input{exemplar/11/16/3/18/main.tex}
\item There are 1000 sealed envelopes in a box, 10 of them contain a cash prize of
Rs 100 each, 100 of them contain a cash prize of Rs 50 each and 200 of them
contain a cash prize of Rs 10 each and rest do not contain any cash prize. If they
are well shuffled and an envelope is picked up out, what is the probability that it
contains no cash prize?\\
\solution
%\input{exemplar/10/13/3/34/main.tex}
\item 
A die is thrown and a card is selected at random from a deck of 52 playing cards. The probability of getting an even number on the die and a spade card.\\
\solution
%\input{exemplar/12/13/3/78/main.tex}
\item
If 4-digit numbers greater than 5,000 are randomly formed from the digits 0, 1, 3, 5, and 7, what is the probability of forming a number divisible by 5 when:
\begin{enumerate}
    \item The digits are repeated?
    \item The repetition of digits is not allowed?
\end{enumerate}
\solution
%\input{ncert/11/16/4/9/main.tex}
\item Consider the probability space $\brak{\Omega, \mathcal{G}, P}$ where $\Omega = [0,2]$ and $\mathcal{G} = \cbrak{\phi, \Omega, [0,1], (1,2]}$. Let $X$ and $Y$ be two functions on $\Omega$ defined as
\begin{align*}
    X(\omega) = 
    \begin{cases}
        1 & \text{if }\omega \in [0, 1]\\
        2 & \text{if }\omega \in (1, 2]
    \end{cases}
\end{align*}
and
\begin{align*}
    Y(\omega) = 
    \begin{cases}
        2 & \text{if }\omega \in [0, 1.5]\\
        3 & \text{if }\omega \in (1.5, 2].
    \end{cases}
\end{align*}
Then which one of the following statements is true?
\begin{enumerate}
    \item [(A)] $X$ is a random variable with respect to $\mathcal{G}$, but $Y$ is not a random variable with respect to $\mathcal{G}$.
    \item [(B)] $Y$ is a random variable with respect to $\mathcal{G}$, but $X$ is not a random variable with respect to $\mathcal{G}$.
    \item [(C)] Neither $X$ nor $Y$ is a random variable with respect to $\mathcal{G}$.
    \item [(D)] Both $X$ and $Y$ are random variables with respect to $\mathcal{G}$.
\end{enumerate} \hfill (GATE ST 2023)\\
\solution
%\input{gate/ST/2023/14/main.tex}
	\item  A die is loaded in such a way that each odd number is twice as likely to occur as
each even number. Find $P(G)$, where $G$ is the event that a number greater than
3 occurs on a single roll of the die.
\\
\solution
		%\input{exemplar/11/16/3/5/main.tex}
	\item All the jacks, queens and kings are removed from a deck of 52 playing cards. The remaining cards are well shuffled and then one card is drawn at random. Giving ace a value 1 similar value for other cards, find the probability that the card has a value 
		\begin{enumerate}
			\item 7
			\item greater than 7
			\item less than 7
		\end{enumerate}
		%\input{exemplar/10/13/3/30/main.tex}
  \item A Lot consists of 48 mobile phones of which 42 are good, 3 have only minor defects and 3 have major defects.Varnika will buy a phone if it is good but the trader will only buy a mobile if it has no major defects. One phone is selected at random from the lot. What is the probability that it is
\begin{enumerate}
	\item acceptable to Varnika?
            \item acceptable to the trader?
\end{enumerate}
\solution
	%\input{exemplar/10/13/3/40/main.tex}
 \item A student says that if you throw a die, it will show up 1 or not 1. Therefore, the probability of getting 1 and the probability of getting 'not 1' each is equal to $\frac{1}{2}$. Is this correct? Give reasons.\\
 \solution
        %\input{exemplar/10/13/2/9/main.tex}
   \item Four candidates A, B, C, D have ap-
plied for the assignment to coach a school cricket
team. If A is twice as likely to be selected as B, and
B and C are given about the same chance of being
selected, while C is twice as likely to be selected
as D, what are the probabilities that
\begin{enumerate}
\item C will be selected?
\item A will not be selected?
\end{enumerate}
	%\input{exemplar/11/16/3/9/main.tex}
 \item A bag contain 24 balls of which $x$ balls are red, $2x$ are white and $3x$ are blue. A ball is selected at random, What is the probability that it is
\begin{enumerate}[label=\alph*)]
\item not red ?
\item white ?
\end{enumerate}
%\input{exemplar/10/13/3/41/main.tex}
If the letters of the word ASSASSINATION are arranged at random. Find the Probability that
\begin{enumerate}[label=(\alph*)]
\item Four $S's$ come consecutively in the word
\item Two  $I's$ and two $N's$ come together
\item All $A's$ are not coming together
\item No two $A's$ are coming together
\end{enumerate}
%\input{exemplar/11/16/3/14/main.tex}
	\item One urn contains two black balls (labelled B1 and B2) and one white ball. A
	second urn contains one black ball and two white balls (labelled W1 and W2).
	Suppose the following experiment is performed. One of the two urns is chosen
	at random. Next a ball is randomly chosen from the urn. Then a second ball is
	chosen at random from the same urn without replacing the first ball.
	
	\begin{enumerate}
	\item What is the probability that two black balls are chosen?
	
	\item What is the probability that two balls of opposite colour are chosen?
	\end{enumerate}
	\solution
	%\input{exemplar/11/16/3/12/main1.tex}
\end{enumerate}

	\item 
The number lock of a suitcase has 4 wheels each labelled with ten digits i.e. from 0 to 9.The lock opens with a sequence of four digits with no repeats.What is the probability of a person getting the right sequence to open the suitcase.
\\
\solution
		%\begin{enumerate}[label=\thesection.\arabic*,ref=\thesection.\theenumi]
	\item One card is drawn from a well-shuffled deck of 52 cards. Find the probability of getting
\begin{enumerate}
\item A king of red colour 
\item A face card 
\item A red face card
\item The jack of hearts
\item A spade
\item The queen of diamonds

\end{enumerate}
\solution
		%\input{ncert/10/15/1/14/main.tex}
	\item Five cards—the ten, jack, queen, king and ace of diamonds, are well-shuffled with their face downwards. One card is then picked up at random.
\begin{enumerate}
\item
What is the probability that the card is the queen? 
\item
If the queen is drawn and put aside, what is the probability that the second card picked up is (a) an ace? (b) a queen?\\
\end{enumerate}
\solution
		%\input{ncert/10/15/1/15/defs.tex}
	\item A bag contains $5$ red balls and some blue balls. If the probability of drawing a blue ball is double that if a red ball, determine the number of blue balls in the bag. 
		\\
\solution
		%\input{ncert/10/15/2/3/defs.tex}
	\item A card is selected from a pack of 52 cards.
 \begin{enumerate}[label=(\alph*)] 
                 \item How many points are there in the sample space?
                 \item Calculate the probability that the card is an ace of spades.
                 \item Calculate the probability that the card is (i) an ace and (ii) black card.
 \end{enumerate}
\solution
		%\input{ncert/11/16/3/4/main.tex}
\item Four cards are drawn from a well-shuffled deck of 52 cards. What is the probability of obtaining 3 diamonds and one spade.
\\
\solution
		%\input{ncert/11/16/4/2/defs.tex}
\item In a certain lottery 10,000 tickets are sold and ten equal prizes are awarded. What is the probability of not getting a prize if you buy (a) one ticket (b) two tickets (c) 10 tickets ?	
\\
\solution
		%\input{ncert/11/16/4/4/defs.tex}
		%
\item 
Out of 100 students, two sections of 40 and 60 are formed. If you and your friend are among the 100 students, what is the probability that
\begin{enumerate}
\item you both enter the same section?
\item you both enter the different sections?
\end{enumerate}
\solution
		%\input{ncert/11/16/4/5/defs.tex}
	\item 
The number lock of a suitcase has 4 wheels each labelled with ten digits i.e. from 0 to 9.The lock opens with a sequence of four digits with no repeats.What is the probability of a person getting the right sequence to open the suitcase.
\\
\solution
		%\input{ncert/11/16/4/10/defs.tex}
		%
\item 
Two cards are drawn at random and without replacement from a pack of 52 playing cards. Find the probability that both the cards are black.
\\
\solution
		%\input{ncert/12/13/2/2/defs.tex}
		\item A box of oranges is inspected by examining three randomly selected oranges drawn without replacement. If all the three oranges are good, the box is approved for sale, otherwise, it is rejected. Find the probability that a box containing 15 oranges out of which 12 are good and 3 are bad ones will be approved for sale.
		\label{ncert/12/13/2/3/defs.tex}
		\item Two balls are drawn at random with replacement from a box containing 10 black and 8 red balls. Find the probability that
		\label{ncert/12/13/2/12}
\begin{enumerate}
\item both balls are red.
\item first ball is black and second is red.
\item one of them is black and other is red.
\end{enumerate}

\item In a hostel, 60\% of the students read Hindi newspaper, 40\% read English newspaper and 20\% read both Hindi and English newspapers. A student is selected at random.
		\label{ncert/12/13/2/15}
\begin{enumerate}
\item Find the probability that she reads neither Hindi nor English newspapers.
\item If she reads Hindi newspaper, find the probability that she reads English newspaper.
\item If she reads English newspaper, find the probability that she reads Hindi newspaper.\\
\end{enumerate}
\item The probability of obtaining an even prime number on each die, when a pair of dice is rolled is 
\begin{enumerate}
    \item $0$ 
    
    \item $\frac{1}{3}$ 
    
    \item $\frac{1}{12}$ 
    
    \item $\frac{1}{36}$ 
\end{enumerate}
\solution
		%\input{ncert/12/13/2/17/defs.tex}
	\item A bag contains 4 red and 4 black balls, another bag contains 2 red and 6 black balls. One of the two bags is selected at random and a ball is drawn from the bag which is found to be red. Find the probability that the ball is drawn from the first bag.
\\
\solution
		%\input{ncert/12/13/3/2/main.tex}
  \item
  Cards with numbers 2 to 101 are placed in a box. A card is selected at random.Find the probability that the card has
\begin{enumerate}[label=(\roman*)]
	\item an even number 
	\item a square number
\end{enumerate}
\solution
%\input{exemplar/10/13/3/32/main.tex}
\item
The king, queen and jack of clubs are removed from a deck of 52 playing cards and then well shuffled. Now one card is drawn at random from the remaining cards.  Determine the probability that the card is
\begin{enumerate}[label=(\roman*)]
\item a club
\item 10 of hearts
\end{enumerate}
\solution
%\input{exemplar/10/13/3/29/main.tex}
\item A team of medical students doing their internship have to assist during surgeries
at a city hospital. The probabilities of surgeries rated as very complex, complex,
routine, simple or very simple are respectively, 0.15, 0.20, 0.31, 0.26, .08. Find
the probabilities that a particular surgery will be rated
\begin{enumerate}
	\item complex or very complex;
	\item neither very complex nor very simple;
	\item routine or complex
	\item routine or simple
\end{enumerate}
\solution
%\input{exemplar/11/16/3/8(1)/main.tex}
\item A card is selected from a pack of 52 cards.
\begin{enumerate}[label=(\alph*)]
    \item How many points are there in the sample space?
    \item Calculate the probability that the card is an ace of spades.
    \item Calculate the probability that the card is (i) an ace and (ii) black card.
\end{enumerate}
\solution
%\input{exemplar/11/16/3/4/main2.tex}
\item The probability that a non leap year selected at random will contain 53 sundays.
\\
\solution
%\input{exemplar/10/13/1/19/main.tex}
\item One of the four persons John, Rita, Aslam or Gurpreet will be promoted next
month. Consequently the sample space consists of four elementary outcomes
S = {John promoted, Rita promoted, Aslam promoted, Gurpreet promoted}
You are told that the chances of John’s promotion is same as that of Gurpreet,
Rita’s chances of promotion are twice as likely as Johns. Aslam’s chances are
four times that of John.
\begin{enumerate}
	\item Determine
	\begin{enumerate}
		\item P (John promoted)
		\item P (Rita promoted)
		\item P (Aslam promoted)
		\item P (Gurpreet promoted)
	\end{enumerate}
	\item If A = {John promoted or Gurpreet promoted}, find P (A).
\end{enumerate}
\solution
%\input{exemplar/11/16/3/10/main.tex}
\item A card is drawn from a deck of 52 cards. Find the probability of getting a king or a heart or a red card.\\
\solution
%\input{exemplar/11/16/3/15/main.tex}
\item The probability that a student will pass his examination is 0.73, the probability of
the student getting a compartment is 0.13, and the probability that the student will
either pass or get compartment is 0.96. State True or False.\\
\solution
%\input{exemplar/11/16/3/31/main.tex}
\item A card is selected from a pack of 52 cards\\
\begin{enumerate}[label=(\alph*)]
\item How many points are there in the sample space?
\item Calculate the probability that the cards is an ace of spades.
\item Calculate the probability that the card is (i) an ace (ii)black card.\\
\end{enumerate}
%\input{ncert/11/16/3/4_1/Prob_4.tex}
\item In a non-leap year, the probability of having 53 tuesdays or 53 wednesdays is\\
\solution
%\input{exemplar/11/16/3/18/main.tex}
\item There are 1000 sealed envelopes in a box, 10 of them contain a cash prize of
Rs 100 each, 100 of them contain a cash prize of Rs 50 each and 200 of them
contain a cash prize of Rs 10 each and rest do not contain any cash prize. If they
are well shuffled and an envelope is picked up out, what is the probability that it
contains no cash prize?\\
\solution
%\input{exemplar/10/13/3/34/main.tex}
\item 
A die is thrown and a card is selected at random from a deck of 52 playing cards. The probability of getting an even number on the die and a spade card.\\
\solution
%\input{exemplar/12/13/3/78/main.tex}
\item
If 4-digit numbers greater than 5,000 are randomly formed from the digits 0, 1, 3, 5, and 7, what is the probability of forming a number divisible by 5 when:
\begin{enumerate}
    \item The digits are repeated?
    \item The repetition of digits is not allowed?
\end{enumerate}
\solution
%\input{ncert/11/16/4/9/main.tex}
\item Consider the probability space $\brak{\Omega, \mathcal{G}, P}$ where $\Omega = [0,2]$ and $\mathcal{G} = \cbrak{\phi, \Omega, [0,1], (1,2]}$. Let $X$ and $Y$ be two functions on $\Omega$ defined as
\begin{align*}
    X(\omega) = 
    \begin{cases}
        1 & \text{if }\omega \in [0, 1]\\
        2 & \text{if }\omega \in (1, 2]
    \end{cases}
\end{align*}
and
\begin{align*}
    Y(\omega) = 
    \begin{cases}
        2 & \text{if }\omega \in [0, 1.5]\\
        3 & \text{if }\omega \in (1.5, 2].
    \end{cases}
\end{align*}
Then which one of the following statements is true?
\begin{enumerate}
    \item [(A)] $X$ is a random variable with respect to $\mathcal{G}$, but $Y$ is not a random variable with respect to $\mathcal{G}$.
    \item [(B)] $Y$ is a random variable with respect to $\mathcal{G}$, but $X$ is not a random variable with respect to $\mathcal{G}$.
    \item [(C)] Neither $X$ nor $Y$ is a random variable with respect to $\mathcal{G}$.
    \item [(D)] Both $X$ and $Y$ are random variables with respect to $\mathcal{G}$.
\end{enumerate} \hfill (GATE ST 2023)\\
\solution
%\input{gate/ST/2023/14/main.tex}
	\item  A die is loaded in such a way that each odd number is twice as likely to occur as
each even number. Find $P(G)$, where $G$ is the event that a number greater than
3 occurs on a single roll of the die.
\\
\solution
		%\input{exemplar/11/16/3/5/main.tex}
	\item All the jacks, queens and kings are removed from a deck of 52 playing cards. The remaining cards are well shuffled and then one card is drawn at random. Giving ace a value 1 similar value for other cards, find the probability that the card has a value 
		\begin{enumerate}
			\item 7
			\item greater than 7
			\item less than 7
		\end{enumerate}
		%\input{exemplar/10/13/3/30/main.tex}
  \item A Lot consists of 48 mobile phones of which 42 are good, 3 have only minor defects and 3 have major defects.Varnika will buy a phone if it is good but the trader will only buy a mobile if it has no major defects. One phone is selected at random from the lot. What is the probability that it is
\begin{enumerate}
	\item acceptable to Varnika?
            \item acceptable to the trader?
\end{enumerate}
\solution
	%\input{exemplar/10/13/3/40/main.tex}
 \item A student says that if you throw a die, it will show up 1 or not 1. Therefore, the probability of getting 1 and the probability of getting 'not 1' each is equal to $\frac{1}{2}$. Is this correct? Give reasons.\\
 \solution
        %\input{exemplar/10/13/2/9/main.tex}
   \item Four candidates A, B, C, D have ap-
plied for the assignment to coach a school cricket
team. If A is twice as likely to be selected as B, and
B and C are given about the same chance of being
selected, while C is twice as likely to be selected
as D, what are the probabilities that
\begin{enumerate}
\item C will be selected?
\item A will not be selected?
\end{enumerate}
	%\input{exemplar/11/16/3/9/main.tex}
 \item A bag contain 24 balls of which $x$ balls are red, $2x$ are white and $3x$ are blue. A ball is selected at random, What is the probability that it is
\begin{enumerate}[label=\alph*)]
\item not red ?
\item white ?
\end{enumerate}
%\input{exemplar/10/13/3/41/main.tex}
If the letters of the word ASSASSINATION are arranged at random. Find the Probability that
\begin{enumerate}[label=(\alph*)]
\item Four $S's$ come consecutively in the word
\item Two  $I's$ and two $N's$ come together
\item All $A's$ are not coming together
\item No two $A's$ are coming together
\end{enumerate}
%\input{exemplar/11/16/3/14/main.tex}
	\item One urn contains two black balls (labelled B1 and B2) and one white ball. A
	second urn contains one black ball and two white balls (labelled W1 and W2).
	Suppose the following experiment is performed. One of the two urns is chosen
	at random. Next a ball is randomly chosen from the urn. Then a second ball is
	chosen at random from the same urn without replacing the first ball.
	
	\begin{enumerate}
	\item What is the probability that two black balls are chosen?
	
	\item What is the probability that two balls of opposite colour are chosen?
	\end{enumerate}
	\solution
	%\input{exemplar/11/16/3/12/main1.tex}
\end{enumerate}

		%
\item 
Two cards are drawn at random and without replacement from a pack of 52 playing cards. Find the probability that both the cards are black.
\\
\solution
		%\begin{enumerate}[label=\thesection.\arabic*,ref=\thesection.\theenumi]
	\item One card is drawn from a well-shuffled deck of 52 cards. Find the probability of getting
\begin{enumerate}
\item A king of red colour 
\item A face card 
\item A red face card
\item The jack of hearts
\item A spade
\item The queen of diamonds

\end{enumerate}
\solution
		%\input{ncert/10/15/1/14/main.tex}
	\item Five cards—the ten, jack, queen, king and ace of diamonds, are well-shuffled with their face downwards. One card is then picked up at random.
\begin{enumerate}
\item
What is the probability that the card is the queen? 
\item
If the queen is drawn and put aside, what is the probability that the second card picked up is (a) an ace? (b) a queen?\\
\end{enumerate}
\solution
		%\input{ncert/10/15/1/15/defs.tex}
	\item A bag contains $5$ red balls and some blue balls. If the probability of drawing a blue ball is double that if a red ball, determine the number of blue balls in the bag. 
		\\
\solution
		%\input{ncert/10/15/2/3/defs.tex}
	\item A card is selected from a pack of 52 cards.
 \begin{enumerate}[label=(\alph*)] 
                 \item How many points are there in the sample space?
                 \item Calculate the probability that the card is an ace of spades.
                 \item Calculate the probability that the card is (i) an ace and (ii) black card.
 \end{enumerate}
\solution
		%\input{ncert/11/16/3/4/main.tex}
\item Four cards are drawn from a well-shuffled deck of 52 cards. What is the probability of obtaining 3 diamonds and one spade.
\\
\solution
		%\input{ncert/11/16/4/2/defs.tex}
\item In a certain lottery 10,000 tickets are sold and ten equal prizes are awarded. What is the probability of not getting a prize if you buy (a) one ticket (b) two tickets (c) 10 tickets ?	
\\
\solution
		%\input{ncert/11/16/4/4/defs.tex}
		%
\item 
Out of 100 students, two sections of 40 and 60 are formed. If you and your friend are among the 100 students, what is the probability that
\begin{enumerate}
\item you both enter the same section?
\item you both enter the different sections?
\end{enumerate}
\solution
		%\input{ncert/11/16/4/5/defs.tex}
	\item 
The number lock of a suitcase has 4 wheels each labelled with ten digits i.e. from 0 to 9.The lock opens with a sequence of four digits with no repeats.What is the probability of a person getting the right sequence to open the suitcase.
\\
\solution
		%\input{ncert/11/16/4/10/defs.tex}
		%
\item 
Two cards are drawn at random and without replacement from a pack of 52 playing cards. Find the probability that both the cards are black.
\\
\solution
		%\input{ncert/12/13/2/2/defs.tex}
		\item A box of oranges is inspected by examining three randomly selected oranges drawn without replacement. If all the three oranges are good, the box is approved for sale, otherwise, it is rejected. Find the probability that a box containing 15 oranges out of which 12 are good and 3 are bad ones will be approved for sale.
		\label{ncert/12/13/2/3/defs.tex}
		\item Two balls are drawn at random with replacement from a box containing 10 black and 8 red balls. Find the probability that
		\label{ncert/12/13/2/12}
\begin{enumerate}
\item both balls are red.
\item first ball is black and second is red.
\item one of them is black and other is red.
\end{enumerate}

\item In a hostel, 60\% of the students read Hindi newspaper, 40\% read English newspaper and 20\% read both Hindi and English newspapers. A student is selected at random.
		\label{ncert/12/13/2/15}
\begin{enumerate}
\item Find the probability that she reads neither Hindi nor English newspapers.
\item If she reads Hindi newspaper, find the probability that she reads English newspaper.
\item If she reads English newspaper, find the probability that she reads Hindi newspaper.\\
\end{enumerate}
\item The probability of obtaining an even prime number on each die, when a pair of dice is rolled is 
\begin{enumerate}
    \item $0$ 
    
    \item $\frac{1}{3}$ 
    
    \item $\frac{1}{12}$ 
    
    \item $\frac{1}{36}$ 
\end{enumerate}
\solution
		%\input{ncert/12/13/2/17/defs.tex}
	\item A bag contains 4 red and 4 black balls, another bag contains 2 red and 6 black balls. One of the two bags is selected at random and a ball is drawn from the bag which is found to be red. Find the probability that the ball is drawn from the first bag.
\\
\solution
		%\input{ncert/12/13/3/2/main.tex}
  \item
  Cards with numbers 2 to 101 are placed in a box. A card is selected at random.Find the probability that the card has
\begin{enumerate}[label=(\roman*)]
	\item an even number 
	\item a square number
\end{enumerate}
\solution
%\input{exemplar/10/13/3/32/main.tex}
\item
The king, queen and jack of clubs are removed from a deck of 52 playing cards and then well shuffled. Now one card is drawn at random from the remaining cards.  Determine the probability that the card is
\begin{enumerate}[label=(\roman*)]
\item a club
\item 10 of hearts
\end{enumerate}
\solution
%\input{exemplar/10/13/3/29/main.tex}
\item A team of medical students doing their internship have to assist during surgeries
at a city hospital. The probabilities of surgeries rated as very complex, complex,
routine, simple or very simple are respectively, 0.15, 0.20, 0.31, 0.26, .08. Find
the probabilities that a particular surgery will be rated
\begin{enumerate}
	\item complex or very complex;
	\item neither very complex nor very simple;
	\item routine or complex
	\item routine or simple
\end{enumerate}
\solution
%\input{exemplar/11/16/3/8(1)/main.tex}
\item A card is selected from a pack of 52 cards.
\begin{enumerate}[label=(\alph*)]
    \item How many points are there in the sample space?
    \item Calculate the probability that the card is an ace of spades.
    \item Calculate the probability that the card is (i) an ace and (ii) black card.
\end{enumerate}
\solution
%\input{exemplar/11/16/3/4/main2.tex}
\item The probability that a non leap year selected at random will contain 53 sundays.
\\
\solution
%\input{exemplar/10/13/1/19/main.tex}
\item One of the four persons John, Rita, Aslam or Gurpreet will be promoted next
month. Consequently the sample space consists of four elementary outcomes
S = {John promoted, Rita promoted, Aslam promoted, Gurpreet promoted}
You are told that the chances of John’s promotion is same as that of Gurpreet,
Rita’s chances of promotion are twice as likely as Johns. Aslam’s chances are
four times that of John.
\begin{enumerate}
	\item Determine
	\begin{enumerate}
		\item P (John promoted)
		\item P (Rita promoted)
		\item P (Aslam promoted)
		\item P (Gurpreet promoted)
	\end{enumerate}
	\item If A = {John promoted or Gurpreet promoted}, find P (A).
\end{enumerate}
\solution
%\input{exemplar/11/16/3/10/main.tex}
\item A card is drawn from a deck of 52 cards. Find the probability of getting a king or a heart or a red card.\\
\solution
%\input{exemplar/11/16/3/15/main.tex}
\item The probability that a student will pass his examination is 0.73, the probability of
the student getting a compartment is 0.13, and the probability that the student will
either pass or get compartment is 0.96. State True or False.\\
\solution
%\input{exemplar/11/16/3/31/main.tex}
\item A card is selected from a pack of 52 cards\\
\begin{enumerate}[label=(\alph*)]
\item How many points are there in the sample space?
\item Calculate the probability that the cards is an ace of spades.
\item Calculate the probability that the card is (i) an ace (ii)black card.\\
\end{enumerate}
%\input{ncert/11/16/3/4_1/Prob_4.tex}
\item In a non-leap year, the probability of having 53 tuesdays or 53 wednesdays is\\
\solution
%\input{exemplar/11/16/3/18/main.tex}
\item There are 1000 sealed envelopes in a box, 10 of them contain a cash prize of
Rs 100 each, 100 of them contain a cash prize of Rs 50 each and 200 of them
contain a cash prize of Rs 10 each and rest do not contain any cash prize. If they
are well shuffled and an envelope is picked up out, what is the probability that it
contains no cash prize?\\
\solution
%\input{exemplar/10/13/3/34/main.tex}
\item 
A die is thrown and a card is selected at random from a deck of 52 playing cards. The probability of getting an even number on the die and a spade card.\\
\solution
%\input{exemplar/12/13/3/78/main.tex}
\item
If 4-digit numbers greater than 5,000 are randomly formed from the digits 0, 1, 3, 5, and 7, what is the probability of forming a number divisible by 5 when:
\begin{enumerate}
    \item The digits are repeated?
    \item The repetition of digits is not allowed?
\end{enumerate}
\solution
%\input{ncert/11/16/4/9/main.tex}
\item Consider the probability space $\brak{\Omega, \mathcal{G}, P}$ where $\Omega = [0,2]$ and $\mathcal{G} = \cbrak{\phi, \Omega, [0,1], (1,2]}$. Let $X$ and $Y$ be two functions on $\Omega$ defined as
\begin{align*}
    X(\omega) = 
    \begin{cases}
        1 & \text{if }\omega \in [0, 1]\\
        2 & \text{if }\omega \in (1, 2]
    \end{cases}
\end{align*}
and
\begin{align*}
    Y(\omega) = 
    \begin{cases}
        2 & \text{if }\omega \in [0, 1.5]\\
        3 & \text{if }\omega \in (1.5, 2].
    \end{cases}
\end{align*}
Then which one of the following statements is true?
\begin{enumerate}
    \item [(A)] $X$ is a random variable with respect to $\mathcal{G}$, but $Y$ is not a random variable with respect to $\mathcal{G}$.
    \item [(B)] $Y$ is a random variable with respect to $\mathcal{G}$, but $X$ is not a random variable with respect to $\mathcal{G}$.
    \item [(C)] Neither $X$ nor $Y$ is a random variable with respect to $\mathcal{G}$.
    \item [(D)] Both $X$ and $Y$ are random variables with respect to $\mathcal{G}$.
\end{enumerate} \hfill (GATE ST 2023)\\
\solution
%\input{gate/ST/2023/14/main.tex}
	\item  A die is loaded in such a way that each odd number is twice as likely to occur as
each even number. Find $P(G)$, where $G$ is the event that a number greater than
3 occurs on a single roll of the die.
\\
\solution
		%\input{exemplar/11/16/3/5/main.tex}
	\item All the jacks, queens and kings are removed from a deck of 52 playing cards. The remaining cards are well shuffled and then one card is drawn at random. Giving ace a value 1 similar value for other cards, find the probability that the card has a value 
		\begin{enumerate}
			\item 7
			\item greater than 7
			\item less than 7
		\end{enumerate}
		%\input{exemplar/10/13/3/30/main.tex}
  \item A Lot consists of 48 mobile phones of which 42 are good, 3 have only minor defects and 3 have major defects.Varnika will buy a phone if it is good but the trader will only buy a mobile if it has no major defects. One phone is selected at random from the lot. What is the probability that it is
\begin{enumerate}
	\item acceptable to Varnika?
            \item acceptable to the trader?
\end{enumerate}
\solution
	%\input{exemplar/10/13/3/40/main.tex}
 \item A student says that if you throw a die, it will show up 1 or not 1. Therefore, the probability of getting 1 and the probability of getting 'not 1' each is equal to $\frac{1}{2}$. Is this correct? Give reasons.\\
 \solution
        %\input{exemplar/10/13/2/9/main.tex}
   \item Four candidates A, B, C, D have ap-
plied for the assignment to coach a school cricket
team. If A is twice as likely to be selected as B, and
B and C are given about the same chance of being
selected, while C is twice as likely to be selected
as D, what are the probabilities that
\begin{enumerate}
\item C will be selected?
\item A will not be selected?
\end{enumerate}
	%\input{exemplar/11/16/3/9/main.tex}
 \item A bag contain 24 balls of which $x$ balls are red, $2x$ are white and $3x$ are blue. A ball is selected at random, What is the probability that it is
\begin{enumerate}[label=\alph*)]
\item not red ?
\item white ?
\end{enumerate}
%\input{exemplar/10/13/3/41/main.tex}
If the letters of the word ASSASSINATION are arranged at random. Find the Probability that
\begin{enumerate}[label=(\alph*)]
\item Four $S's$ come consecutively in the word
\item Two  $I's$ and two $N's$ come together
\item All $A's$ are not coming together
\item No two $A's$ are coming together
\end{enumerate}
%\input{exemplar/11/16/3/14/main.tex}
	\item One urn contains two black balls (labelled B1 and B2) and one white ball. A
	second urn contains one black ball and two white balls (labelled W1 and W2).
	Suppose the following experiment is performed. One of the two urns is chosen
	at random. Next a ball is randomly chosen from the urn. Then a second ball is
	chosen at random from the same urn without replacing the first ball.
	
	\begin{enumerate}
	\item What is the probability that two black balls are chosen?
	
	\item What is the probability that two balls of opposite colour are chosen?
	\end{enumerate}
	\solution
	%\input{exemplar/11/16/3/12/main1.tex}
\end{enumerate}

		\item A box of oranges is inspected by examining three randomly selected oranges drawn without replacement. If all the three oranges are good, the box is approved for sale, otherwise, it is rejected. Find the probability that a box containing 15 oranges out of which 12 are good and 3 are bad ones will be approved for sale.
		\label{ncert/12/13/2/3/defs.tex}
		\item Two balls are drawn at random with replacement from a box containing 10 black and 8 red balls. Find the probability that
		\label{ncert/12/13/2/12}
\begin{enumerate}
\item both balls are red.
\item first ball is black and second is red.
\item one of them is black and other is red.
\end{enumerate}

\item In a hostel, 60\% of the students read Hindi newspaper, 40\% read English newspaper and 20\% read both Hindi and English newspapers. A student is selected at random.
		\label{ncert/12/13/2/15}
\begin{enumerate}
\item Find the probability that she reads neither Hindi nor English newspapers.
\item If she reads Hindi newspaper, find the probability that she reads English newspaper.
\item If she reads English newspaper, find the probability that she reads Hindi newspaper.\\
\end{enumerate}
\item The probability of obtaining an even prime number on each die, when a pair of dice is rolled is 
\begin{enumerate}
    \item $0$ 
    
    \item $\frac{1}{3}$ 
    
    \item $\frac{1}{12}$ 
    
    \item $\frac{1}{36}$ 
\end{enumerate}
\solution
		%\begin{enumerate}[label=\thesection.\arabic*,ref=\thesection.\theenumi]
	\item One card is drawn from a well-shuffled deck of 52 cards. Find the probability of getting
\begin{enumerate}
\item A king of red colour 
\item A face card 
\item A red face card
\item The jack of hearts
\item A spade
\item The queen of diamonds

\end{enumerate}
\solution
		%\input{ncert/10/15/1/14/main.tex}
	\item Five cards—the ten, jack, queen, king and ace of diamonds, are well-shuffled with their face downwards. One card is then picked up at random.
\begin{enumerate}
\item
What is the probability that the card is the queen? 
\item
If the queen is drawn and put aside, what is the probability that the second card picked up is (a) an ace? (b) a queen?\\
\end{enumerate}
\solution
		%\input{ncert/10/15/1/15/defs.tex}
	\item A bag contains $5$ red balls and some blue balls. If the probability of drawing a blue ball is double that if a red ball, determine the number of blue balls in the bag. 
		\\
\solution
		%\input{ncert/10/15/2/3/defs.tex}
	\item A card is selected from a pack of 52 cards.
 \begin{enumerate}[label=(\alph*)] 
                 \item How many points are there in the sample space?
                 \item Calculate the probability that the card is an ace of spades.
                 \item Calculate the probability that the card is (i) an ace and (ii) black card.
 \end{enumerate}
\solution
		%\input{ncert/11/16/3/4/main.tex}
\item Four cards are drawn from a well-shuffled deck of 52 cards. What is the probability of obtaining 3 diamonds and one spade.
\\
\solution
		%\input{ncert/11/16/4/2/defs.tex}
\item In a certain lottery 10,000 tickets are sold and ten equal prizes are awarded. What is the probability of not getting a prize if you buy (a) one ticket (b) two tickets (c) 10 tickets ?	
\\
\solution
		%\input{ncert/11/16/4/4/defs.tex}
		%
\item 
Out of 100 students, two sections of 40 and 60 are formed. If you and your friend are among the 100 students, what is the probability that
\begin{enumerate}
\item you both enter the same section?
\item you both enter the different sections?
\end{enumerate}
\solution
		%\input{ncert/11/16/4/5/defs.tex}
	\item 
The number lock of a suitcase has 4 wheels each labelled with ten digits i.e. from 0 to 9.The lock opens with a sequence of four digits with no repeats.What is the probability of a person getting the right sequence to open the suitcase.
\\
\solution
		%\input{ncert/11/16/4/10/defs.tex}
		%
\item 
Two cards are drawn at random and without replacement from a pack of 52 playing cards. Find the probability that both the cards are black.
\\
\solution
		%\input{ncert/12/13/2/2/defs.tex}
		\item A box of oranges is inspected by examining three randomly selected oranges drawn without replacement. If all the three oranges are good, the box is approved for sale, otherwise, it is rejected. Find the probability that a box containing 15 oranges out of which 12 are good and 3 are bad ones will be approved for sale.
		\label{ncert/12/13/2/3/defs.tex}
		\item Two balls are drawn at random with replacement from a box containing 10 black and 8 red balls. Find the probability that
		\label{ncert/12/13/2/12}
\begin{enumerate}
\item both balls are red.
\item first ball is black and second is red.
\item one of them is black and other is red.
\end{enumerate}

\item In a hostel, 60\% of the students read Hindi newspaper, 40\% read English newspaper and 20\% read both Hindi and English newspapers. A student is selected at random.
		\label{ncert/12/13/2/15}
\begin{enumerate}
\item Find the probability that she reads neither Hindi nor English newspapers.
\item If she reads Hindi newspaper, find the probability that she reads English newspaper.
\item If she reads English newspaper, find the probability that she reads Hindi newspaper.\\
\end{enumerate}
\item The probability of obtaining an even prime number on each die, when a pair of dice is rolled is 
\begin{enumerate}
    \item $0$ 
    
    \item $\frac{1}{3}$ 
    
    \item $\frac{1}{12}$ 
    
    \item $\frac{1}{36}$ 
\end{enumerate}
\solution
		%\input{ncert/12/13/2/17/defs.tex}
	\item A bag contains 4 red and 4 black balls, another bag contains 2 red and 6 black balls. One of the two bags is selected at random and a ball is drawn from the bag which is found to be red. Find the probability that the ball is drawn from the first bag.
\\
\solution
		%\input{ncert/12/13/3/2/main.tex}
  \item
  Cards with numbers 2 to 101 are placed in a box. A card is selected at random.Find the probability that the card has
\begin{enumerate}[label=(\roman*)]
	\item an even number 
	\item a square number
\end{enumerate}
\solution
%\input{exemplar/10/13/3/32/main.tex}
\item
The king, queen and jack of clubs are removed from a deck of 52 playing cards and then well shuffled. Now one card is drawn at random from the remaining cards.  Determine the probability that the card is
\begin{enumerate}[label=(\roman*)]
\item a club
\item 10 of hearts
\end{enumerate}
\solution
%\input{exemplar/10/13/3/29/main.tex}
\item A team of medical students doing their internship have to assist during surgeries
at a city hospital. The probabilities of surgeries rated as very complex, complex,
routine, simple or very simple are respectively, 0.15, 0.20, 0.31, 0.26, .08. Find
the probabilities that a particular surgery will be rated
\begin{enumerate}
	\item complex or very complex;
	\item neither very complex nor very simple;
	\item routine or complex
	\item routine or simple
\end{enumerate}
\solution
%\input{exemplar/11/16/3/8(1)/main.tex}
\item A card is selected from a pack of 52 cards.
\begin{enumerate}[label=(\alph*)]
    \item How many points are there in the sample space?
    \item Calculate the probability that the card is an ace of spades.
    \item Calculate the probability that the card is (i) an ace and (ii) black card.
\end{enumerate}
\solution
%\input{exemplar/11/16/3/4/main2.tex}
\item The probability that a non leap year selected at random will contain 53 sundays.
\\
\solution
%\input{exemplar/10/13/1/19/main.tex}
\item One of the four persons John, Rita, Aslam or Gurpreet will be promoted next
month. Consequently the sample space consists of four elementary outcomes
S = {John promoted, Rita promoted, Aslam promoted, Gurpreet promoted}
You are told that the chances of John’s promotion is same as that of Gurpreet,
Rita’s chances of promotion are twice as likely as Johns. Aslam’s chances are
four times that of John.
\begin{enumerate}
	\item Determine
	\begin{enumerate}
		\item P (John promoted)
		\item P (Rita promoted)
		\item P (Aslam promoted)
		\item P (Gurpreet promoted)
	\end{enumerate}
	\item If A = {John promoted or Gurpreet promoted}, find P (A).
\end{enumerate}
\solution
%\input{exemplar/11/16/3/10/main.tex}
\item A card is drawn from a deck of 52 cards. Find the probability of getting a king or a heart or a red card.\\
\solution
%\input{exemplar/11/16/3/15/main.tex}
\item The probability that a student will pass his examination is 0.73, the probability of
the student getting a compartment is 0.13, and the probability that the student will
either pass or get compartment is 0.96. State True or False.\\
\solution
%\input{exemplar/11/16/3/31/main.tex}
\item A card is selected from a pack of 52 cards\\
\begin{enumerate}[label=(\alph*)]
\item How many points are there in the sample space?
\item Calculate the probability that the cards is an ace of spades.
\item Calculate the probability that the card is (i) an ace (ii)black card.\\
\end{enumerate}
%\input{ncert/11/16/3/4_1/Prob_4.tex}
\item In a non-leap year, the probability of having 53 tuesdays or 53 wednesdays is\\
\solution
%\input{exemplar/11/16/3/18/main.tex}
\item There are 1000 sealed envelopes in a box, 10 of them contain a cash prize of
Rs 100 each, 100 of them contain a cash prize of Rs 50 each and 200 of them
contain a cash prize of Rs 10 each and rest do not contain any cash prize. If they
are well shuffled and an envelope is picked up out, what is the probability that it
contains no cash prize?\\
\solution
%\input{exemplar/10/13/3/34/main.tex}
\item 
A die is thrown and a card is selected at random from a deck of 52 playing cards. The probability of getting an even number on the die and a spade card.\\
\solution
%\input{exemplar/12/13/3/78/main.tex}
\item
If 4-digit numbers greater than 5,000 are randomly formed from the digits 0, 1, 3, 5, and 7, what is the probability of forming a number divisible by 5 when:
\begin{enumerate}
    \item The digits are repeated?
    \item The repetition of digits is not allowed?
\end{enumerate}
\solution
%\input{ncert/11/16/4/9/main.tex}
\item Consider the probability space $\brak{\Omega, \mathcal{G}, P}$ where $\Omega = [0,2]$ and $\mathcal{G} = \cbrak{\phi, \Omega, [0,1], (1,2]}$. Let $X$ and $Y$ be two functions on $\Omega$ defined as
\begin{align*}
    X(\omega) = 
    \begin{cases}
        1 & \text{if }\omega \in [0, 1]\\
        2 & \text{if }\omega \in (1, 2]
    \end{cases}
\end{align*}
and
\begin{align*}
    Y(\omega) = 
    \begin{cases}
        2 & \text{if }\omega \in [0, 1.5]\\
        3 & \text{if }\omega \in (1.5, 2].
    \end{cases}
\end{align*}
Then which one of the following statements is true?
\begin{enumerate}
    \item [(A)] $X$ is a random variable with respect to $\mathcal{G}$, but $Y$ is not a random variable with respect to $\mathcal{G}$.
    \item [(B)] $Y$ is a random variable with respect to $\mathcal{G}$, but $X$ is not a random variable with respect to $\mathcal{G}$.
    \item [(C)] Neither $X$ nor $Y$ is a random variable with respect to $\mathcal{G}$.
    \item [(D)] Both $X$ and $Y$ are random variables with respect to $\mathcal{G}$.
\end{enumerate} \hfill (GATE ST 2023)\\
\solution
%\input{gate/ST/2023/14/main.tex}
	\item  A die is loaded in such a way that each odd number is twice as likely to occur as
each even number. Find $P(G)$, where $G$ is the event that a number greater than
3 occurs on a single roll of the die.
\\
\solution
		%\input{exemplar/11/16/3/5/main.tex}
	\item All the jacks, queens and kings are removed from a deck of 52 playing cards. The remaining cards are well shuffled and then one card is drawn at random. Giving ace a value 1 similar value for other cards, find the probability that the card has a value 
		\begin{enumerate}
			\item 7
			\item greater than 7
			\item less than 7
		\end{enumerate}
		%\input{exemplar/10/13/3/30/main.tex}
  \item A Lot consists of 48 mobile phones of which 42 are good, 3 have only minor defects and 3 have major defects.Varnika will buy a phone if it is good but the trader will only buy a mobile if it has no major defects. One phone is selected at random from the lot. What is the probability that it is
\begin{enumerate}
	\item acceptable to Varnika?
            \item acceptable to the trader?
\end{enumerate}
\solution
	%\input{exemplar/10/13/3/40/main.tex}
 \item A student says that if you throw a die, it will show up 1 or not 1. Therefore, the probability of getting 1 and the probability of getting 'not 1' each is equal to $\frac{1}{2}$. Is this correct? Give reasons.\\
 \solution
        %\input{exemplar/10/13/2/9/main.tex}
   \item Four candidates A, B, C, D have ap-
plied for the assignment to coach a school cricket
team. If A is twice as likely to be selected as B, and
B and C are given about the same chance of being
selected, while C is twice as likely to be selected
as D, what are the probabilities that
\begin{enumerate}
\item C will be selected?
\item A will not be selected?
\end{enumerate}
	%\input{exemplar/11/16/3/9/main.tex}
 \item A bag contain 24 balls of which $x$ balls are red, $2x$ are white and $3x$ are blue. A ball is selected at random, What is the probability that it is
\begin{enumerate}[label=\alph*)]
\item not red ?
\item white ?
\end{enumerate}
%\input{exemplar/10/13/3/41/main.tex}
If the letters of the word ASSASSINATION are arranged at random. Find the Probability that
\begin{enumerate}[label=(\alph*)]
\item Four $S's$ come consecutively in the word
\item Two  $I's$ and two $N's$ come together
\item All $A's$ are not coming together
\item No two $A's$ are coming together
\end{enumerate}
%\input{exemplar/11/16/3/14/main.tex}
	\item One urn contains two black balls (labelled B1 and B2) and one white ball. A
	second urn contains one black ball and two white balls (labelled W1 and W2).
	Suppose the following experiment is performed. One of the two urns is chosen
	at random. Next a ball is randomly chosen from the urn. Then a second ball is
	chosen at random from the same urn without replacing the first ball.
	
	\begin{enumerate}
	\item What is the probability that two black balls are chosen?
	
	\item What is the probability that two balls of opposite colour are chosen?
	\end{enumerate}
	\solution
	%\input{exemplar/11/16/3/12/main1.tex}
\end{enumerate}

	\item A bag contains 4 red and 4 black balls, another bag contains 2 red and 6 black balls. One of the two bags is selected at random and a ball is drawn from the bag which is found to be red. Find the probability that the ball is drawn from the first bag.
\\
\solution
		%\begin{table}[H]
	\centering
\begin{tabular}{|c|c|c|}
\hline
Random variable &Value &Definition\\ \hline
\multirow{3}{*}{X} &0 &Slips of Rs 1\\
&1 &Slips of Rs 5\\
&2 &Slips of Rs 13\\ \hline
\multirow{2}{*}{Y} &0 &Box A\\
&1 &Box B\\\hline
\end{tabular}
\caption{}
\label{tab:Distribution}
\end{table}
See \tabref{tab:Distribution}.
\begin{align}
p_{Y}\brak{k}= \begin{cases} 
      \frac{1}{3} & {k=0} \\
      \frac{2}{3 }& {k=1} 
   \end{cases}
   \\
p_{Y|X}\brak{0|0} = \frac{19}{25}\, 
p_{Y|X}\brak{0|1} = \frac{6}{25}\,
p_{Y|X}\brak{1|0} = \frac{45}{50}\,
p_{Y|X}\brak{1|2} = \frac{5}{50}
\end{align}
The desired probability is the probability that a slip drawn at random is marked other than Rs 1,
\begin{align}
&=1-p_X\brak{0}\\
&= p_X(1) + p_X(2)
\end{align}
Using Bayes theorem,
\begin{align}
&= p_Y\brak{0} \times \pr{Y=0 | X=1} + p_Y\brak{1} \times \pr{Y=1|X=2}\\
&=\frac{1}{3} \times \frac{6}{25} + \frac{2}{3} \times \frac{5}{50}\\
&=\frac{11}{75}
\end{align}

\newpage

%\tableofcontents

\bigskip

\renewcommand{\thefigure}{\theenumi}
\renewcommand{\thetable}{\theenumi}
%\renewcommand{\theequation}{\theenumi}

%\begin{abstract}
%%\boldmath
%In this letter, an algorithm for evaluating the exact analytical bit error rate  (BER)  for the piecewise linear (PL) combiner for  multiple relays is presented. Previous results were available only for upto three relays. The algorithm is unique in the sense that  the actual mathematical expressions, that are prohibitively large, need not be explicitly obtained. The diversity gain due to multiple relays is shown through plots of the analytical BER, well supported by simulations. 
%
%\end{abstract}
% IEEEtran.cls defaults to using nonbold math in the Abstract.
% This preserves the distinction between vectors and scalars. However,
% if the journal you are submitting to favors bold math in the abstract,
% then you can use LaTeX's standard command \boldmath at the very start
% of the abstract to achieve this. Many IEEE journals frown on math
% in the abstract anyway.

% Note that keywords are not normally used for peerreview papers.
%\begin{IEEEkeywords}
%Cooperative diversity, decode and forward, piecewise linear
%\end{IEEEkeywords}



% For peer review papers, you can put extra information on the cover
% page as needed:
% \ifCLASSOPTIONpeerreview
% \begin{center} \bfseries EDICS Category: 3-BBND \end{center}
% \fi
%
% For peerreview papers, this IEEEtran command inserts a page break and
% creates the second title. It will be ignored for other modes.
%\IEEEpeerreviewmaketitle




  \item
  Cards with numbers 2 to 101 are placed in a box. A card is selected at random.Find the probability that the card has
\begin{enumerate}[label=(\roman*)]
	\item an even number 
	\item a square number
\end{enumerate}
\solution
%\begin{table}[H]
	\centering
\begin{tabular}{|c|c|c|}
\hline
Random variable &Value &Definition\\ \hline
\multirow{3}{*}{X} &0 &Slips of Rs 1\\
&1 &Slips of Rs 5\\
&2 &Slips of Rs 13\\ \hline
\multirow{2}{*}{Y} &0 &Box A\\
&1 &Box B\\\hline
\end{tabular}
\caption{}
\label{tab:Distribution}
\end{table}
See \tabref{tab:Distribution}.
\begin{align}
p_{Y}\brak{k}= \begin{cases} 
      \frac{1}{3} & {k=0} \\
      \frac{2}{3 }& {k=1} 
   \end{cases}
   \\
p_{Y|X}\brak{0|0} = \frac{19}{25}\, 
p_{Y|X}\brak{0|1} = \frac{6}{25}\,
p_{Y|X}\brak{1|0} = \frac{45}{50}\,
p_{Y|X}\brak{1|2} = \frac{5}{50}
\end{align}
The desired probability is the probability that a slip drawn at random is marked other than Rs 1,
\begin{align}
&=1-p_X\brak{0}\\
&= p_X(1) + p_X(2)
\end{align}
Using Bayes theorem,
\begin{align}
&= p_Y\brak{0} \times \pr{Y=0 | X=1} + p_Y\brak{1} \times \pr{Y=1|X=2}\\
&=\frac{1}{3} \times \frac{6}{25} + \frac{2}{3} \times \frac{5}{50}\\
&=\frac{11}{75}
\end{align}

\newpage

%\tableofcontents

\bigskip

\renewcommand{\thefigure}{\theenumi}
\renewcommand{\thetable}{\theenumi}
%\renewcommand{\theequation}{\theenumi}

%\begin{abstract}
%%\boldmath
%In this letter, an algorithm for evaluating the exact analytical bit error rate  (BER)  for the piecewise linear (PL) combiner for  multiple relays is presented. Previous results were available only for upto three relays. The algorithm is unique in the sense that  the actual mathematical expressions, that are prohibitively large, need not be explicitly obtained. The diversity gain due to multiple relays is shown through plots of the analytical BER, well supported by simulations. 
%
%\end{abstract}
% IEEEtran.cls defaults to using nonbold math in the Abstract.
% This preserves the distinction between vectors and scalars. However,
% if the journal you are submitting to favors bold math in the abstract,
% then you can use LaTeX's standard command \boldmath at the very start
% of the abstract to achieve this. Many IEEE journals frown on math
% in the abstract anyway.

% Note that keywords are not normally used for peerreview papers.
%\begin{IEEEkeywords}
%Cooperative diversity, decode and forward, piecewise linear
%\end{IEEEkeywords}



% For peer review papers, you can put extra information on the cover
% page as needed:
% \ifCLASSOPTIONpeerreview
% \begin{center} \bfseries EDICS Category: 3-BBND \end{center}
% \fi
%
% For peerreview papers, this IEEEtran command inserts a page break and
% creates the second title. It will be ignored for other modes.
%\IEEEpeerreviewmaketitle




\item
The king, queen and jack of clubs are removed from a deck of 52 playing cards and then well shuffled. Now one card is drawn at random from the remaining cards.  Determine the probability that the card is
\begin{enumerate}[label=(\roman*)]
\item a club
\item 10 of hearts
\end{enumerate}
\solution
%\begin{table}[H]
	\centering
\begin{tabular}{|c|c|c|}
\hline
Random variable &Value &Definition\\ \hline
\multirow{3}{*}{X} &0 &Slips of Rs 1\\
&1 &Slips of Rs 5\\
&2 &Slips of Rs 13\\ \hline
\multirow{2}{*}{Y} &0 &Box A\\
&1 &Box B\\\hline
\end{tabular}
\caption{}
\label{tab:Distribution}
\end{table}
See \tabref{tab:Distribution}.
\begin{align}
p_{Y}\brak{k}= \begin{cases} 
      \frac{1}{3} & {k=0} \\
      \frac{2}{3 }& {k=1} 
   \end{cases}
   \\
p_{Y|X}\brak{0|0} = \frac{19}{25}\, 
p_{Y|X}\brak{0|1} = \frac{6}{25}\,
p_{Y|X}\brak{1|0} = \frac{45}{50}\,
p_{Y|X}\brak{1|2} = \frac{5}{50}
\end{align}
The desired probability is the probability that a slip drawn at random is marked other than Rs 1,
\begin{align}
&=1-p_X\brak{0}\\
&= p_X(1) + p_X(2)
\end{align}
Using Bayes theorem,
\begin{align}
&= p_Y\brak{0} \times \pr{Y=0 | X=1} + p_Y\brak{1} \times \pr{Y=1|X=2}\\
&=\frac{1}{3} \times \frac{6}{25} + \frac{2}{3} \times \frac{5}{50}\\
&=\frac{11}{75}
\end{align}

\newpage

%\tableofcontents

\bigskip

\renewcommand{\thefigure}{\theenumi}
\renewcommand{\thetable}{\theenumi}
%\renewcommand{\theequation}{\theenumi}

%\begin{abstract}
%%\boldmath
%In this letter, an algorithm for evaluating the exact analytical bit error rate  (BER)  for the piecewise linear (PL) combiner for  multiple relays is presented. Previous results were available only for upto three relays. The algorithm is unique in the sense that  the actual mathematical expressions, that are prohibitively large, need not be explicitly obtained. The diversity gain due to multiple relays is shown through plots of the analytical BER, well supported by simulations. 
%
%\end{abstract}
% IEEEtran.cls defaults to using nonbold math in the Abstract.
% This preserves the distinction between vectors and scalars. However,
% if the journal you are submitting to favors bold math in the abstract,
% then you can use LaTeX's standard command \boldmath at the very start
% of the abstract to achieve this. Many IEEE journals frown on math
% in the abstract anyway.

% Note that keywords are not normally used for peerreview papers.
%\begin{IEEEkeywords}
%Cooperative diversity, decode and forward, piecewise linear
%\end{IEEEkeywords}



% For peer review papers, you can put extra information on the cover
% page as needed:
% \ifCLASSOPTIONpeerreview
% \begin{center} \bfseries EDICS Category: 3-BBND \end{center}
% \fi
%
% For peerreview papers, this IEEEtran command inserts a page break and
% creates the second title. It will be ignored for other modes.
%\IEEEpeerreviewmaketitle




\item A team of medical students doing their internship have to assist during surgeries
at a city hospital. The probabilities of surgeries rated as very complex, complex,
routine, simple or very simple are respectively, 0.15, 0.20, 0.31, 0.26, .08. Find
the probabilities that a particular surgery will be rated
\begin{enumerate}
	\item complex or very complex;
	\item neither very complex nor very simple;
	\item routine or complex
	\item routine or simple
\end{enumerate}
\solution
%\begin{table}[H]
	\centering
\begin{tabular}{|c|c|c|}
\hline
Random variable &Value &Definition\\ \hline
\multirow{3}{*}{X} &0 &Slips of Rs 1\\
&1 &Slips of Rs 5\\
&2 &Slips of Rs 13\\ \hline
\multirow{2}{*}{Y} &0 &Box A\\
&1 &Box B\\\hline
\end{tabular}
\caption{}
\label{tab:Distribution}
\end{table}
See \tabref{tab:Distribution}.
\begin{align}
p_{Y}\brak{k}= \begin{cases} 
      \frac{1}{3} & {k=0} \\
      \frac{2}{3 }& {k=1} 
   \end{cases}
   \\
p_{Y|X}\brak{0|0} = \frac{19}{25}\, 
p_{Y|X}\brak{0|1} = \frac{6}{25}\,
p_{Y|X}\brak{1|0} = \frac{45}{50}\,
p_{Y|X}\brak{1|2} = \frac{5}{50}
\end{align}
The desired probability is the probability that a slip drawn at random is marked other than Rs 1,
\begin{align}
&=1-p_X\brak{0}\\
&= p_X(1) + p_X(2)
\end{align}
Using Bayes theorem,
\begin{align}
&= p_Y\brak{0} \times \pr{Y=0 | X=1} + p_Y\brak{1} \times \pr{Y=1|X=2}\\
&=\frac{1}{3} \times \frac{6}{25} + \frac{2}{3} \times \frac{5}{50}\\
&=\frac{11}{75}
\end{align}

\newpage

%\tableofcontents

\bigskip

\renewcommand{\thefigure}{\theenumi}
\renewcommand{\thetable}{\theenumi}
%\renewcommand{\theequation}{\theenumi}

%\begin{abstract}
%%\boldmath
%In this letter, an algorithm for evaluating the exact analytical bit error rate  (BER)  for the piecewise linear (PL) combiner for  multiple relays is presented. Previous results were available only for upto three relays. The algorithm is unique in the sense that  the actual mathematical expressions, that are prohibitively large, need not be explicitly obtained. The diversity gain due to multiple relays is shown through plots of the analytical BER, well supported by simulations. 
%
%\end{abstract}
% IEEEtran.cls defaults to using nonbold math in the Abstract.
% This preserves the distinction between vectors and scalars. However,
% if the journal you are submitting to favors bold math in the abstract,
% then you can use LaTeX's standard command \boldmath at the very start
% of the abstract to achieve this. Many IEEE journals frown on math
% in the abstract anyway.

% Note that keywords are not normally used for peerreview papers.
%\begin{IEEEkeywords}
%Cooperative diversity, decode and forward, piecewise linear
%\end{IEEEkeywords}



% For peer review papers, you can put extra information on the cover
% page as needed:
% \ifCLASSOPTIONpeerreview
% \begin{center} \bfseries EDICS Category: 3-BBND \end{center}
% \fi
%
% For peerreview papers, this IEEEtran command inserts a page break and
% creates the second title. It will be ignored for other modes.
%\IEEEpeerreviewmaketitle




\item A card is selected from a pack of 52 cards.
\begin{enumerate}[label=(\alph*)]
    \item How many points are there in the sample space?
    \item Calculate the probability that the card is an ace of spades.
    \item Calculate the probability that the card is (i) an ace and (ii) black card.
\end{enumerate}
\solution
%Let $X$ be an bernoulli rv defined as in \tabref{tab:exemplar/11/16/3/26}.  Then, 
\begin{equation}
    p =
        \frac{4}{11} 
\end{equation}
\begin{table}[H]
	\centering
	\input{exemplar/11/16/3/26/tables/Table2.tex}
	\caption{}
        \label{tab:exemplar/11/16/3/26}
\end{table}

\item The probability that a non leap year selected at random will contain 53 sundays.
\\
\solution
%\begin{table}[H]
	\centering
\begin{tabular}{|c|c|c|}
\hline
Random variable &Value &Definition\\ \hline
\multirow{3}{*}{X} &0 &Slips of Rs 1\\
&1 &Slips of Rs 5\\
&2 &Slips of Rs 13\\ \hline
\multirow{2}{*}{Y} &0 &Box A\\
&1 &Box B\\\hline
\end{tabular}
\caption{}
\label{tab:Distribution}
\end{table}
See \tabref{tab:Distribution}.
\begin{align}
p_{Y}\brak{k}= \begin{cases} 
      \frac{1}{3} & {k=0} \\
      \frac{2}{3 }& {k=1} 
   \end{cases}
   \\
p_{Y|X}\brak{0|0} = \frac{19}{25}\, 
p_{Y|X}\brak{0|1} = \frac{6}{25}\,
p_{Y|X}\brak{1|0} = \frac{45}{50}\,
p_{Y|X}\brak{1|2} = \frac{5}{50}
\end{align}
The desired probability is the probability that a slip drawn at random is marked other than Rs 1,
\begin{align}
&=1-p_X\brak{0}\\
&= p_X(1) + p_X(2)
\end{align}
Using Bayes theorem,
\begin{align}
&= p_Y\brak{0} \times \pr{Y=0 | X=1} + p_Y\brak{1} \times \pr{Y=1|X=2}\\
&=\frac{1}{3} \times \frac{6}{25} + \frac{2}{3} \times \frac{5}{50}\\
&=\frac{11}{75}
\end{align}

\newpage

%\tableofcontents

\bigskip

\renewcommand{\thefigure}{\theenumi}
\renewcommand{\thetable}{\theenumi}
%\renewcommand{\theequation}{\theenumi}

%\begin{abstract}
%%\boldmath
%In this letter, an algorithm for evaluating the exact analytical bit error rate  (BER)  for the piecewise linear (PL) combiner for  multiple relays is presented. Previous results were available only for upto three relays. The algorithm is unique in the sense that  the actual mathematical expressions, that are prohibitively large, need not be explicitly obtained. The diversity gain due to multiple relays is shown through plots of the analytical BER, well supported by simulations. 
%
%\end{abstract}
% IEEEtran.cls defaults to using nonbold math in the Abstract.
% This preserves the distinction between vectors and scalars. However,
% if the journal you are submitting to favors bold math in the abstract,
% then you can use LaTeX's standard command \boldmath at the very start
% of the abstract to achieve this. Many IEEE journals frown on math
% in the abstract anyway.

% Note that keywords are not normally used for peerreview papers.
%\begin{IEEEkeywords}
%Cooperative diversity, decode and forward, piecewise linear
%\end{IEEEkeywords}



% For peer review papers, you can put extra information on the cover
% page as needed:
% \ifCLASSOPTIONpeerreview
% \begin{center} \bfseries EDICS Category: 3-BBND \end{center}
% \fi
%
% For peerreview papers, this IEEEtran command inserts a page break and
% creates the second title. It will be ignored for other modes.
%\IEEEpeerreviewmaketitle




\item One of the four persons John, Rita, Aslam or Gurpreet will be promoted next
month. Consequently the sample space consists of four elementary outcomes
S = {John promoted, Rita promoted, Aslam promoted, Gurpreet promoted}
You are told that the chances of John’s promotion is same as that of Gurpreet,
Rita’s chances of promotion are twice as likely as Johns. Aslam’s chances are
four times that of John.
\begin{enumerate}
	\item Determine
	\begin{enumerate}
		\item P (John promoted)
		\item P (Rita promoted)
		\item P (Aslam promoted)
		\item P (Gurpreet promoted)
	\end{enumerate}
	\item If A = {John promoted or Gurpreet promoted}, find P (A).
\end{enumerate}
\solution
%\begin{table}[H]
	\centering
\begin{tabular}{|c|c|c|}
\hline
Random variable &Value &Definition\\ \hline
\multirow{3}{*}{X} &0 &Slips of Rs 1\\
&1 &Slips of Rs 5\\
&2 &Slips of Rs 13\\ \hline
\multirow{2}{*}{Y} &0 &Box A\\
&1 &Box B\\\hline
\end{tabular}
\caption{}
\label{tab:Distribution}
\end{table}
See \tabref{tab:Distribution}.
\begin{align}
p_{Y}\brak{k}= \begin{cases} 
      \frac{1}{3} & {k=0} \\
      \frac{2}{3 }& {k=1} 
   \end{cases}
   \\
p_{Y|X}\brak{0|0} = \frac{19}{25}\, 
p_{Y|X}\brak{0|1} = \frac{6}{25}\,
p_{Y|X}\brak{1|0} = \frac{45}{50}\,
p_{Y|X}\brak{1|2} = \frac{5}{50}
\end{align}
The desired probability is the probability that a slip drawn at random is marked other than Rs 1,
\begin{align}
&=1-p_X\brak{0}\\
&= p_X(1) + p_X(2)
\end{align}
Using Bayes theorem,
\begin{align}
&= p_Y\brak{0} \times \pr{Y=0 | X=1} + p_Y\brak{1} \times \pr{Y=1|X=2}\\
&=\frac{1}{3} \times \frac{6}{25} + \frac{2}{3} \times \frac{5}{50}\\
&=\frac{11}{75}
\end{align}

\newpage

%\tableofcontents

\bigskip

\renewcommand{\thefigure}{\theenumi}
\renewcommand{\thetable}{\theenumi}
%\renewcommand{\theequation}{\theenumi}

%\begin{abstract}
%%\boldmath
%In this letter, an algorithm for evaluating the exact analytical bit error rate  (BER)  for the piecewise linear (PL) combiner for  multiple relays is presented. Previous results were available only for upto three relays. The algorithm is unique in the sense that  the actual mathematical expressions, that are prohibitively large, need not be explicitly obtained. The diversity gain due to multiple relays is shown through plots of the analytical BER, well supported by simulations. 
%
%\end{abstract}
% IEEEtran.cls defaults to using nonbold math in the Abstract.
% This preserves the distinction between vectors and scalars. However,
% if the journal you are submitting to favors bold math in the abstract,
% then you can use LaTeX's standard command \boldmath at the very start
% of the abstract to achieve this. Many IEEE journals frown on math
% in the abstract anyway.

% Note that keywords are not normally used for peerreview papers.
%\begin{IEEEkeywords}
%Cooperative diversity, decode and forward, piecewise linear
%\end{IEEEkeywords}



% For peer review papers, you can put extra information on the cover
% page as needed:
% \ifCLASSOPTIONpeerreview
% \begin{center} \bfseries EDICS Category: 3-BBND \end{center}
% \fi
%
% For peerreview papers, this IEEEtran command inserts a page break and
% creates the second title. It will be ignored for other modes.
%\IEEEpeerreviewmaketitle




\item A card is drawn from a deck of 52 cards. Find the probability of getting a king or a heart or a red card.\\
\solution
%\begin{table}[H]
	\centering
\begin{tabular}{|c|c|c|}
\hline
Random variable &Value &Definition\\ \hline
\multirow{3}{*}{X} &0 &Slips of Rs 1\\
&1 &Slips of Rs 5\\
&2 &Slips of Rs 13\\ \hline
\multirow{2}{*}{Y} &0 &Box A\\
&1 &Box B\\\hline
\end{tabular}
\caption{}
\label{tab:Distribution}
\end{table}
See \tabref{tab:Distribution}.
\begin{align}
p_{Y}\brak{k}= \begin{cases} 
      \frac{1}{3} & {k=0} \\
      \frac{2}{3 }& {k=1} 
   \end{cases}
   \\
p_{Y|X}\brak{0|0} = \frac{19}{25}\, 
p_{Y|X}\brak{0|1} = \frac{6}{25}\,
p_{Y|X}\brak{1|0} = \frac{45}{50}\,
p_{Y|X}\brak{1|2} = \frac{5}{50}
\end{align}
The desired probability is the probability that a slip drawn at random is marked other than Rs 1,
\begin{align}
&=1-p_X\brak{0}\\
&= p_X(1) + p_X(2)
\end{align}
Using Bayes theorem,
\begin{align}
&= p_Y\brak{0} \times \pr{Y=0 | X=1} + p_Y\brak{1} \times \pr{Y=1|X=2}\\
&=\frac{1}{3} \times \frac{6}{25} + \frac{2}{3} \times \frac{5}{50}\\
&=\frac{11}{75}
\end{align}

\newpage

%\tableofcontents

\bigskip

\renewcommand{\thefigure}{\theenumi}
\renewcommand{\thetable}{\theenumi}
%\renewcommand{\theequation}{\theenumi}

%\begin{abstract}
%%\boldmath
%In this letter, an algorithm for evaluating the exact analytical bit error rate  (BER)  for the piecewise linear (PL) combiner for  multiple relays is presented. Previous results were available only for upto three relays. The algorithm is unique in the sense that  the actual mathematical expressions, that are prohibitively large, need not be explicitly obtained. The diversity gain due to multiple relays is shown through plots of the analytical BER, well supported by simulations. 
%
%\end{abstract}
% IEEEtran.cls defaults to using nonbold math in the Abstract.
% This preserves the distinction between vectors and scalars. However,
% if the journal you are submitting to favors bold math in the abstract,
% then you can use LaTeX's standard command \boldmath at the very start
% of the abstract to achieve this. Many IEEE journals frown on math
% in the abstract anyway.

% Note that keywords are not normally used for peerreview papers.
%\begin{IEEEkeywords}
%Cooperative diversity, decode and forward, piecewise linear
%\end{IEEEkeywords}



% For peer review papers, you can put extra information on the cover
% page as needed:
% \ifCLASSOPTIONpeerreview
% \begin{center} \bfseries EDICS Category: 3-BBND \end{center}
% \fi
%
% For peerreview papers, this IEEEtran command inserts a page break and
% creates the second title. It will be ignored for other modes.
%\IEEEpeerreviewmaketitle




\item The probability that a student will pass his examination is 0.73, the probability of
the student getting a compartment is 0.13, and the probability that the student will
either pass or get compartment is 0.96. State True or False.\\
\solution
%\begin{table}[H]
	\centering
\begin{tabular}{|c|c|c|}
\hline
Random variable &Value &Definition\\ \hline
\multirow{3}{*}{X} &0 &Slips of Rs 1\\
&1 &Slips of Rs 5\\
&2 &Slips of Rs 13\\ \hline
\multirow{2}{*}{Y} &0 &Box A\\
&1 &Box B\\\hline
\end{tabular}
\caption{}
\label{tab:Distribution}
\end{table}
See \tabref{tab:Distribution}.
\begin{align}
p_{Y}\brak{k}= \begin{cases} 
      \frac{1}{3} & {k=0} \\
      \frac{2}{3 }& {k=1} 
   \end{cases}
   \\
p_{Y|X}\brak{0|0} = \frac{19}{25}\, 
p_{Y|X}\brak{0|1} = \frac{6}{25}\,
p_{Y|X}\brak{1|0} = \frac{45}{50}\,
p_{Y|X}\brak{1|2} = \frac{5}{50}
\end{align}
The desired probability is the probability that a slip drawn at random is marked other than Rs 1,
\begin{align}
&=1-p_X\brak{0}\\
&= p_X(1) + p_X(2)
\end{align}
Using Bayes theorem,
\begin{align}
&= p_Y\brak{0} \times \pr{Y=0 | X=1} + p_Y\brak{1} \times \pr{Y=1|X=2}\\
&=\frac{1}{3} \times \frac{6}{25} + \frac{2}{3} \times \frac{5}{50}\\
&=\frac{11}{75}
\end{align}

\newpage

%\tableofcontents

\bigskip

\renewcommand{\thefigure}{\theenumi}
\renewcommand{\thetable}{\theenumi}
%\renewcommand{\theequation}{\theenumi}

%\begin{abstract}
%%\boldmath
%In this letter, an algorithm for evaluating the exact analytical bit error rate  (BER)  for the piecewise linear (PL) combiner for  multiple relays is presented. Previous results were available only for upto three relays. The algorithm is unique in the sense that  the actual mathematical expressions, that are prohibitively large, need not be explicitly obtained. The diversity gain due to multiple relays is shown through plots of the analytical BER, well supported by simulations. 
%
%\end{abstract}
% IEEEtran.cls defaults to using nonbold math in the Abstract.
% This preserves the distinction between vectors and scalars. However,
% if the journal you are submitting to favors bold math in the abstract,
% then you can use LaTeX's standard command \boldmath at the very start
% of the abstract to achieve this. Many IEEE journals frown on math
% in the abstract anyway.

% Note that keywords are not normally used for peerreview papers.
%\begin{IEEEkeywords}
%Cooperative diversity, decode and forward, piecewise linear
%\end{IEEEkeywords}



% For peer review papers, you can put extra information on the cover
% page as needed:
% \ifCLASSOPTIONpeerreview
% \begin{center} \bfseries EDICS Category: 3-BBND \end{center}
% \fi
%
% For peerreview papers, this IEEEtran command inserts a page break and
% creates the second title. It will be ignored for other modes.
%\IEEEpeerreviewmaketitle




\item A card is selected from a pack of 52 cards\\
\begin{enumerate}[label=(\alph*)]
\item How many points are there in the sample space?
\item Calculate the probability that the cards is an ace of spades.
\item Calculate the probability that the card is (i) an ace (ii)black card.\\
\end{enumerate}
%\input{ncert/11/16/3/4_1/Prob_4.tex}
\item In a non-leap year, the probability of having 53 tuesdays or 53 wednesdays is\\
\solution
%A non-leap year has a total of 365 days, and a week has 7 days.\\
So it can be expressed as 
\begin{align}
365\text{days} &=52\times 7+1 \text{day}
\end{align}
$\implies$ 52 tuesdays or wednesdays\\
Random variable X denotes the days of a week
\begin{align}
p_X\brak{k}&=\frac{1}{7}; \quad \brak{1<k<7}
\end{align}
So the probability of extra day being tuesday or wednesday is
\begin{align}
p_X\brak{3}+p_X\brak{4}&=\frac{1}{7}+\frac{1}{7}=\frac{2}{7}
\end{align}



\item There are 1000 sealed envelopes in a box, 10 of them contain a cash prize of
Rs 100 each, 100 of them contain a cash prize of Rs 50 each and 200 of them
contain a cash prize of Rs 10 each and rest do not contain any cash prize. If they
are well shuffled and an envelope is picked up out, what is the probability that it
contains no cash prize?\\
\solution
%\begin{table}[H]
	\centering
\begin{tabular}{|c|c|c|}
\hline
Random variable &Value &Definition\\ \hline
\multirow{3}{*}{X} &0 &Slips of Rs 1\\
&1 &Slips of Rs 5\\
&2 &Slips of Rs 13\\ \hline
\multirow{2}{*}{Y} &0 &Box A\\
&1 &Box B\\\hline
\end{tabular}
\caption{}
\label{tab:Distribution}
\end{table}
See \tabref{tab:Distribution}.
\begin{align}
p_{Y}\brak{k}= \begin{cases} 
      \frac{1}{3} & {k=0} \\
      \frac{2}{3 }& {k=1} 
   \end{cases}
   \\
p_{Y|X}\brak{0|0} = \frac{19}{25}\, 
p_{Y|X}\brak{0|1} = \frac{6}{25}\,
p_{Y|X}\brak{1|0} = \frac{45}{50}\,
p_{Y|X}\brak{1|2} = \frac{5}{50}
\end{align}
The desired probability is the probability that a slip drawn at random is marked other than Rs 1,
\begin{align}
&=1-p_X\brak{0}\\
&= p_X(1) + p_X(2)
\end{align}
Using Bayes theorem,
\begin{align}
&= p_Y\brak{0} \times \pr{Y=0 | X=1} + p_Y\brak{1} \times \pr{Y=1|X=2}\\
&=\frac{1}{3} \times \frac{6}{25} + \frac{2}{3} \times \frac{5}{50}\\
&=\frac{11}{75}
\end{align}

\newpage

%\tableofcontents

\bigskip

\renewcommand{\thefigure}{\theenumi}
\renewcommand{\thetable}{\theenumi}
%\renewcommand{\theequation}{\theenumi}

%\begin{abstract}
%%\boldmath
%In this letter, an algorithm for evaluating the exact analytical bit error rate  (BER)  for the piecewise linear (PL) combiner for  multiple relays is presented. Previous results were available only for upto three relays. The algorithm is unique in the sense that  the actual mathematical expressions, that are prohibitively large, need not be explicitly obtained. The diversity gain due to multiple relays is shown through plots of the analytical BER, well supported by simulations. 
%
%\end{abstract}
% IEEEtran.cls defaults to using nonbold math in the Abstract.
% This preserves the distinction between vectors and scalars. However,
% if the journal you are submitting to favors bold math in the abstract,
% then you can use LaTeX's standard command \boldmath at the very start
% of the abstract to achieve this. Many IEEE journals frown on math
% in the abstract anyway.

% Note that keywords are not normally used for peerreview papers.
%\begin{IEEEkeywords}
%Cooperative diversity, decode and forward, piecewise linear
%\end{IEEEkeywords}



% For peer review papers, you can put extra information on the cover
% page as needed:
% \ifCLASSOPTIONpeerreview
% \begin{center} \bfseries EDICS Category: 3-BBND \end{center}
% \fi
%
% For peerreview papers, this IEEEtran command inserts a page break and
% creates the second title. It will be ignored for other modes.
%\IEEEpeerreviewmaketitle




\item 
A die is thrown and a card is selected at random from a deck of 52 playing cards. The probability of getting an even number on the die and a spade card.\\
\solution
%\begin{table}[H]
	\centering
\begin{tabular}{|c|c|c|}
\hline
Random variable &Value &Definition\\ \hline
\multirow{3}{*}{X} &0 &Slips of Rs 1\\
&1 &Slips of Rs 5\\
&2 &Slips of Rs 13\\ \hline
\multirow{2}{*}{Y} &0 &Box A\\
&1 &Box B\\\hline
\end{tabular}
\caption{}
\label{tab:Distribution}
\end{table}
See \tabref{tab:Distribution}.
\begin{align}
p_{Y}\brak{k}= \begin{cases} 
      \frac{1}{3} & {k=0} \\
      \frac{2}{3 }& {k=1} 
   \end{cases}
   \\
p_{Y|X}\brak{0|0} = \frac{19}{25}\, 
p_{Y|X}\brak{0|1} = \frac{6}{25}\,
p_{Y|X}\brak{1|0} = \frac{45}{50}\,
p_{Y|X}\brak{1|2} = \frac{5}{50}
\end{align}
The desired probability is the probability that a slip drawn at random is marked other than Rs 1,
\begin{align}
&=1-p_X\brak{0}\\
&= p_X(1) + p_X(2)
\end{align}
Using Bayes theorem,
\begin{align}
&= p_Y\brak{0} \times \pr{Y=0 | X=1} + p_Y\brak{1} \times \pr{Y=1|X=2}\\
&=\frac{1}{3} \times \frac{6}{25} + \frac{2}{3} \times \frac{5}{50}\\
&=\frac{11}{75}
\end{align}

\newpage

%\tableofcontents

\bigskip

\renewcommand{\thefigure}{\theenumi}
\renewcommand{\thetable}{\theenumi}
%\renewcommand{\theequation}{\theenumi}

%\begin{abstract}
%%\boldmath
%In this letter, an algorithm for evaluating the exact analytical bit error rate  (BER)  for the piecewise linear (PL) combiner for  multiple relays is presented. Previous results were available only for upto three relays. The algorithm is unique in the sense that  the actual mathematical expressions, that are prohibitively large, need not be explicitly obtained. The diversity gain due to multiple relays is shown through plots of the analytical BER, well supported by simulations. 
%
%\end{abstract}
% IEEEtran.cls defaults to using nonbold math in the Abstract.
% This preserves the distinction between vectors and scalars. However,
% if the journal you are submitting to favors bold math in the abstract,
% then you can use LaTeX's standard command \boldmath at the very start
% of the abstract to achieve this. Many IEEE journals frown on math
% in the abstract anyway.

% Note that keywords are not normally used for peerreview papers.
%\begin{IEEEkeywords}
%Cooperative diversity, decode and forward, piecewise linear
%\end{IEEEkeywords}



% For peer review papers, you can put extra information on the cover
% page as needed:
% \ifCLASSOPTIONpeerreview
% \begin{center} \bfseries EDICS Category: 3-BBND \end{center}
% \fi
%
% For peerreview papers, this IEEEtran command inserts a page break and
% creates the second title. It will be ignored for other modes.
%\IEEEpeerreviewmaketitle




\item
If 4-digit numbers greater than 5,000 are randomly formed from the digits 0, 1, 3, 5, and 7, what is the probability of forming a number divisible by 5 when:
\begin{enumerate}
    \item The digits are repeated?
    \item The repetition of digits is not allowed?
\end{enumerate}
\solution
%\begin{table}[H]
	\centering
\begin{tabular}{|c|c|c|}
\hline
Random variable &Value &Definition\\ \hline
\multirow{3}{*}{X} &0 &Slips of Rs 1\\
&1 &Slips of Rs 5\\
&2 &Slips of Rs 13\\ \hline
\multirow{2}{*}{Y} &0 &Box A\\
&1 &Box B\\\hline
\end{tabular}
\caption{}
\label{tab:Distribution}
\end{table}
See \tabref{tab:Distribution}.
\begin{align}
p_{Y}\brak{k}= \begin{cases} 
      \frac{1}{3} & {k=0} \\
      \frac{2}{3 }& {k=1} 
   \end{cases}
   \\
p_{Y|X}\brak{0|0} = \frac{19}{25}\, 
p_{Y|X}\brak{0|1} = \frac{6}{25}\,
p_{Y|X}\brak{1|0} = \frac{45}{50}\,
p_{Y|X}\brak{1|2} = \frac{5}{50}
\end{align}
The desired probability is the probability that a slip drawn at random is marked other than Rs 1,
\begin{align}
&=1-p_X\brak{0}\\
&= p_X(1) + p_X(2)
\end{align}
Using Bayes theorem,
\begin{align}
&= p_Y\brak{0} \times \pr{Y=0 | X=1} + p_Y\brak{1} \times \pr{Y=1|X=2}\\
&=\frac{1}{3} \times \frac{6}{25} + \frac{2}{3} \times \frac{5}{50}\\
&=\frac{11}{75}
\end{align}

\newpage

%\tableofcontents

\bigskip

\renewcommand{\thefigure}{\theenumi}
\renewcommand{\thetable}{\theenumi}
%\renewcommand{\theequation}{\theenumi}

%\begin{abstract}
%%\boldmath
%In this letter, an algorithm for evaluating the exact analytical bit error rate  (BER)  for the piecewise linear (PL) combiner for  multiple relays is presented. Previous results were available only for upto three relays. The algorithm is unique in the sense that  the actual mathematical expressions, that are prohibitively large, need not be explicitly obtained. The diversity gain due to multiple relays is shown through plots of the analytical BER, well supported by simulations. 
%
%\end{abstract}
% IEEEtran.cls defaults to using nonbold math in the Abstract.
% This preserves the distinction between vectors and scalars. However,
% if the journal you are submitting to favors bold math in the abstract,
% then you can use LaTeX's standard command \boldmath at the very start
% of the abstract to achieve this. Many IEEE journals frown on math
% in the abstract anyway.

% Note that keywords are not normally used for peerreview papers.
%\begin{IEEEkeywords}
%Cooperative diversity, decode and forward, piecewise linear
%\end{IEEEkeywords}



% For peer review papers, you can put extra information on the cover
% page as needed:
% \ifCLASSOPTIONpeerreview
% \begin{center} \bfseries EDICS Category: 3-BBND \end{center}
% \fi
%
% For peerreview papers, this IEEEtran command inserts a page break and
% creates the second title. It will be ignored for other modes.
%\IEEEpeerreviewmaketitle




\item Consider the probability space $\brak{\Omega, \mathcal{G}, P}$ where $\Omega = [0,2]$ and $\mathcal{G} = \cbrak{\phi, \Omega, [0,1], (1,2]}$. Let $X$ and $Y$ be two functions on $\Omega$ defined as
\begin{align*}
    X(\omega) = 
    \begin{cases}
        1 & \text{if }\omega \in [0, 1]\\
        2 & \text{if }\omega \in (1, 2]
    \end{cases}
\end{align*}
and
\begin{align*}
    Y(\omega) = 
    \begin{cases}
        2 & \text{if }\omega \in [0, 1.5]\\
        3 & \text{if }\omega \in (1.5, 2].
    \end{cases}
\end{align*}
Then which one of the following statements is true?
\begin{enumerate}
    \item [(A)] $X$ is a random variable with respect to $\mathcal{G}$, but $Y$ is not a random variable with respect to $\mathcal{G}$.
    \item [(B)] $Y$ is a random variable with respect to $\mathcal{G}$, but $X$ is not a random variable with respect to $\mathcal{G}$.
    \item [(C)] Neither $X$ nor $Y$ is a random variable with respect to $\mathcal{G}$.
    \item [(D)] Both $X$ and $Y$ are random variables with respect to $\mathcal{G}$.
\end{enumerate} \hfill (GATE ST 2023)\\
\solution
%\begin{table}[H]
	\centering
\begin{tabular}{|c|c|c|}
\hline
Random variable &Value &Definition\\ \hline
\multirow{3}{*}{X} &0 &Slips of Rs 1\\
&1 &Slips of Rs 5\\
&2 &Slips of Rs 13\\ \hline
\multirow{2}{*}{Y} &0 &Box A\\
&1 &Box B\\\hline
\end{tabular}
\caption{}
\label{tab:Distribution}
\end{table}
See \tabref{tab:Distribution}.
\begin{align}
p_{Y}\brak{k}= \begin{cases} 
      \frac{1}{3} & {k=0} \\
      \frac{2}{3 }& {k=1} 
   \end{cases}
   \\
p_{Y|X}\brak{0|0} = \frac{19}{25}\, 
p_{Y|X}\brak{0|1} = \frac{6}{25}\,
p_{Y|X}\brak{1|0} = \frac{45}{50}\,
p_{Y|X}\brak{1|2} = \frac{5}{50}
\end{align}
The desired probability is the probability that a slip drawn at random is marked other than Rs 1,
\begin{align}
&=1-p_X\brak{0}\\
&= p_X(1) + p_X(2)
\end{align}
Using Bayes theorem,
\begin{align}
&= p_Y\brak{0} \times \pr{Y=0 | X=1} + p_Y\brak{1} \times \pr{Y=1|X=2}\\
&=\frac{1}{3} \times \frac{6}{25} + \frac{2}{3} \times \frac{5}{50}\\
&=\frac{11}{75}
\end{align}

\newpage

%\tableofcontents

\bigskip

\renewcommand{\thefigure}{\theenumi}
\renewcommand{\thetable}{\theenumi}
%\renewcommand{\theequation}{\theenumi}

%\begin{abstract}
%%\boldmath
%In this letter, an algorithm for evaluating the exact analytical bit error rate  (BER)  for the piecewise linear (PL) combiner for  multiple relays is presented. Previous results were available only for upto three relays. The algorithm is unique in the sense that  the actual mathematical expressions, that are prohibitively large, need not be explicitly obtained. The diversity gain due to multiple relays is shown through plots of the analytical BER, well supported by simulations. 
%
%\end{abstract}
% IEEEtran.cls defaults to using nonbold math in the Abstract.
% This preserves the distinction between vectors and scalars. However,
% if the journal you are submitting to favors bold math in the abstract,
% then you can use LaTeX's standard command \boldmath at the very start
% of the abstract to achieve this. Many IEEE journals frown on math
% in the abstract anyway.

% Note that keywords are not normally used for peerreview papers.
%\begin{IEEEkeywords}
%Cooperative diversity, decode and forward, piecewise linear
%\end{IEEEkeywords}



% For peer review papers, you can put extra information on the cover
% page as needed:
% \ifCLASSOPTIONpeerreview
% \begin{center} \bfseries EDICS Category: 3-BBND \end{center}
% \fi
%
% For peerreview papers, this IEEEtran command inserts a page break and
% creates the second title. It will be ignored for other modes.
%\IEEEpeerreviewmaketitle




	\item  A die is loaded in such a way that each odd number is twice as likely to occur as
each even number. Find $P(G)$, where $G$ is the event that a number greater than
3 occurs on a single roll of the die.
\\
\solution
		%\begin{table}[H]
	\centering
\begin{tabular}{|c|c|c|}
\hline
Random variable &Value &Definition\\ \hline
\multirow{3}{*}{X} &0 &Slips of Rs 1\\
&1 &Slips of Rs 5\\
&2 &Slips of Rs 13\\ \hline
\multirow{2}{*}{Y} &0 &Box A\\
&1 &Box B\\\hline
\end{tabular}
\caption{}
\label{tab:Distribution}
\end{table}
See \tabref{tab:Distribution}.
\begin{align}
p_{Y}\brak{k}= \begin{cases} 
      \frac{1}{3} & {k=0} \\
      \frac{2}{3 }& {k=1} 
   \end{cases}
   \\
p_{Y|X}\brak{0|0} = \frac{19}{25}\, 
p_{Y|X}\brak{0|1} = \frac{6}{25}\,
p_{Y|X}\brak{1|0} = \frac{45}{50}\,
p_{Y|X}\brak{1|2} = \frac{5}{50}
\end{align}
The desired probability is the probability that a slip drawn at random is marked other than Rs 1,
\begin{align}
&=1-p_X\brak{0}\\
&= p_X(1) + p_X(2)
\end{align}
Using Bayes theorem,
\begin{align}
&= p_Y\brak{0} \times \pr{Y=0 | X=1} + p_Y\brak{1} \times \pr{Y=1|X=2}\\
&=\frac{1}{3} \times \frac{6}{25} + \frac{2}{3} \times \frac{5}{50}\\
&=\frac{11}{75}
\end{align}

\newpage

%\tableofcontents

\bigskip

\renewcommand{\thefigure}{\theenumi}
\renewcommand{\thetable}{\theenumi}
%\renewcommand{\theequation}{\theenumi}

%\begin{abstract}
%%\boldmath
%In this letter, an algorithm for evaluating the exact analytical bit error rate  (BER)  for the piecewise linear (PL) combiner for  multiple relays is presented. Previous results were available only for upto three relays. The algorithm is unique in the sense that  the actual mathematical expressions, that are prohibitively large, need not be explicitly obtained. The diversity gain due to multiple relays is shown through plots of the analytical BER, well supported by simulations. 
%
%\end{abstract}
% IEEEtran.cls defaults to using nonbold math in the Abstract.
% This preserves the distinction between vectors and scalars. However,
% if the journal you are submitting to favors bold math in the abstract,
% then you can use LaTeX's standard command \boldmath at the very start
% of the abstract to achieve this. Many IEEE journals frown on math
% in the abstract anyway.

% Note that keywords are not normally used for peerreview papers.
%\begin{IEEEkeywords}
%Cooperative diversity, decode and forward, piecewise linear
%\end{IEEEkeywords}



% For peer review papers, you can put extra information on the cover
% page as needed:
% \ifCLASSOPTIONpeerreview
% \begin{center} \bfseries EDICS Category: 3-BBND \end{center}
% \fi
%
% For peerreview papers, this IEEEtran command inserts a page break and
% creates the second title. It will be ignored for other modes.
%\IEEEpeerreviewmaketitle




	\item All the jacks, queens and kings are removed from a deck of 52 playing cards. The remaining cards are well shuffled and then one card is drawn at random. Giving ace a value 1 similar value for other cards, find the probability that the card has a value 
		\begin{enumerate}
			\item 7
			\item greater than 7
			\item less than 7
		\end{enumerate}
		%Number of cards left after removing all jacks, queens and kings 
\begin{align}
N	= 52 - 4\times 3
	= 40
\end{align}
%\begin{table}[H]
%\def\arraystretch{1.2}
%\begin{tabular}{|c|c|c|}
%\hline
%	\textbf{Parameter} &\textbf{Value} &\textbf{Description}\\ \hline
%	$X$ &1-10 &Represents the value of the card picked \\ \hline
%\end{tabular}
%\end{table}
Let $1 \le X \le 10$ be the value of the card picked.  Then,
\begin{align}
	p_X(k) &= \Pr(X=k)\ \forall\ 1 \leq k \leq 10\\
	&= \frac{4\times 1}{40}\\
	&= \frac{1}{10}\\
	\therefore p_X(k) &= 
	\begin{cases}
		\frac{1}{10} & 1 \leq k \leq 10\\
		0 & \text{otherwise}
	\end{cases}
\end{align}
and
\begin{align}
	F_{X}(k) &= \sum_{m=0}^{k}p_{X}(m) \quad 1 \leq k \leq 10\\
	&= \frac{k}{10}\\
	\therefore F_{X}(k) &= 
	\begin{cases}
		0 & k \leq 0\\
		\frac{k}{10} & 1\leq k \leq 10\\
		1 & k > 10 
	\end{cases}
\end{align}
\begin{enumerate}
	\item Probability that card has value equal to 7 is
		\begin{align}
			 p_{X}(7)
			= \frac{1}{10}
		\end{align}
	\item Probability that card has value greater than 7 is
		\begin{align}
			1 - F_X(7)
			&= 1 - \frac{7}{10}
			\\
			&= \frac{3}{10}
		\end{align}
	\item Probability that card has value less than 7 is
		\begin{align}
			 F_{X}(6)
			=\frac{6}{10}
		\end{align}
\end{enumerate}

  \item A Lot consists of 48 mobile phones of which 42 are good, 3 have only minor defects and 3 have major defects.Varnika will buy a phone if it is good but the trader will only buy a mobile if it has no major defects. One phone is selected at random from the lot. What is the probability that it is
\begin{enumerate}
	\item acceptable to Varnika?
            \item acceptable to the trader?
\end{enumerate}
\solution
	%\begin{table}[H]
	\centering
\begin{tabular}{|c|c|c|}
\hline
Random variable &Value &Definition\\ \hline
\multirow{3}{*}{X} &0 &Slips of Rs 1\\
&1 &Slips of Rs 5\\
&2 &Slips of Rs 13\\ \hline
\multirow{2}{*}{Y} &0 &Box A\\
&1 &Box B\\\hline
\end{tabular}
\caption{}
\label{tab:Distribution}
\end{table}
See \tabref{tab:Distribution}.
\begin{align}
p_{Y}\brak{k}= \begin{cases} 
      \frac{1}{3} & {k=0} \\
      \frac{2}{3 }& {k=1} 
   \end{cases}
   \\
p_{Y|X}\brak{0|0} = \frac{19}{25}\, 
p_{Y|X}\brak{0|1} = \frac{6}{25}\,
p_{Y|X}\brak{1|0} = \frac{45}{50}\,
p_{Y|X}\brak{1|2} = \frac{5}{50}
\end{align}
The desired probability is the probability that a slip drawn at random is marked other than Rs 1,
\begin{align}
&=1-p_X\brak{0}\\
&= p_X(1) + p_X(2)
\end{align}
Using Bayes theorem,
\begin{align}
&= p_Y\brak{0} \times \pr{Y=0 | X=1} + p_Y\brak{1} \times \pr{Y=1|X=2}\\
&=\frac{1}{3} \times \frac{6}{25} + \frac{2}{3} \times \frac{5}{50}\\
&=\frac{11}{75}
\end{align}

\newpage

%\tableofcontents

\bigskip

\renewcommand{\thefigure}{\theenumi}
\renewcommand{\thetable}{\theenumi}
%\renewcommand{\theequation}{\theenumi}

%\begin{abstract}
%%\boldmath
%In this letter, an algorithm for evaluating the exact analytical bit error rate  (BER)  for the piecewise linear (PL) combiner for  multiple relays is presented. Previous results were available only for upto three relays. The algorithm is unique in the sense that  the actual mathematical expressions, that are prohibitively large, need not be explicitly obtained. The diversity gain due to multiple relays is shown through plots of the analytical BER, well supported by simulations. 
%
%\end{abstract}
% IEEEtran.cls defaults to using nonbold math in the Abstract.
% This preserves the distinction between vectors and scalars. However,
% if the journal you are submitting to favors bold math in the abstract,
% then you can use LaTeX's standard command \boldmath at the very start
% of the abstract to achieve this. Many IEEE journals frown on math
% in the abstract anyway.

% Note that keywords are not normally used for peerreview papers.
%\begin{IEEEkeywords}
%Cooperative diversity, decode and forward, piecewise linear
%\end{IEEEkeywords}



% For peer review papers, you can put extra information on the cover
% page as needed:
% \ifCLASSOPTIONpeerreview
% \begin{center} \bfseries EDICS Category: 3-BBND \end{center}
% \fi
%
% For peerreview papers, this IEEEtran command inserts a page break and
% creates the second title. It will be ignored for other modes.
%\IEEEpeerreviewmaketitle




 \item A student says that if you throw a die, it will show up 1 or not 1. Therefore, the probability of getting 1 and the probability of getting 'not 1' each is equal to $\frac{1}{2}$. Is this correct? Give reasons.\\
 \solution
        %\begin{table}[H]
	\centering
\begin{tabular}{|c|c|c|}
\hline
Random variable &Value &Definition\\ \hline
\multirow{3}{*}{X} &0 &Slips of Rs 1\\
&1 &Slips of Rs 5\\
&2 &Slips of Rs 13\\ \hline
\multirow{2}{*}{Y} &0 &Box A\\
&1 &Box B\\\hline
\end{tabular}
\caption{}
\label{tab:Distribution}
\end{table}
See \tabref{tab:Distribution}.
\begin{align}
p_{Y}\brak{k}= \begin{cases} 
      \frac{1}{3} & {k=0} \\
      \frac{2}{3 }& {k=1} 
   \end{cases}
   \\
p_{Y|X}\brak{0|0} = \frac{19}{25}\, 
p_{Y|X}\brak{0|1} = \frac{6}{25}\,
p_{Y|X}\brak{1|0} = \frac{45}{50}\,
p_{Y|X}\brak{1|2} = \frac{5}{50}
\end{align}
The desired probability is the probability that a slip drawn at random is marked other than Rs 1,
\begin{align}
&=1-p_X\brak{0}\\
&= p_X(1) + p_X(2)
\end{align}
Using Bayes theorem,
\begin{align}
&= p_Y\brak{0} \times \pr{Y=0 | X=1} + p_Y\brak{1} \times \pr{Y=1|X=2}\\
&=\frac{1}{3} \times \frac{6}{25} + \frac{2}{3} \times \frac{5}{50}\\
&=\frac{11}{75}
\end{align}

\newpage

%\tableofcontents

\bigskip

\renewcommand{\thefigure}{\theenumi}
\renewcommand{\thetable}{\theenumi}
%\renewcommand{\theequation}{\theenumi}

%\begin{abstract}
%%\boldmath
%In this letter, an algorithm for evaluating the exact analytical bit error rate  (BER)  for the piecewise linear (PL) combiner for  multiple relays is presented. Previous results were available only for upto three relays. The algorithm is unique in the sense that  the actual mathematical expressions, that are prohibitively large, need not be explicitly obtained. The diversity gain due to multiple relays is shown through plots of the analytical BER, well supported by simulations. 
%
%\end{abstract}
% IEEEtran.cls defaults to using nonbold math in the Abstract.
% This preserves the distinction between vectors and scalars. However,
% if the journal you are submitting to favors bold math in the abstract,
% then you can use LaTeX's standard command \boldmath at the very start
% of the abstract to achieve this. Many IEEE journals frown on math
% in the abstract anyway.

% Note that keywords are not normally used for peerreview papers.
%\begin{IEEEkeywords}
%Cooperative diversity, decode and forward, piecewise linear
%\end{IEEEkeywords}



% For peer review papers, you can put extra information on the cover
% page as needed:
% \ifCLASSOPTIONpeerreview
% \begin{center} \bfseries EDICS Category: 3-BBND \end{center}
% \fi
%
% For peerreview papers, this IEEEtran command inserts a page break and
% creates the second title. It will be ignored for other modes.
%\IEEEpeerreviewmaketitle




   \item Four candidates A, B, C, D have ap-
plied for the assignment to coach a school cricket
team. If A is twice as likely to be selected as B, and
B and C are given about the same chance of being
selected, while C is twice as likely to be selected
as D, what are the probabilities that
\begin{enumerate}
\item C will be selected?
\item A will not be selected?
\end{enumerate}
	%\begin{table}[H]
	\centering
\begin{tabular}{|c|c|c|}
\hline
Random variable &Value &Definition\\ \hline
\multirow{3}{*}{X} &0 &Slips of Rs 1\\
&1 &Slips of Rs 5\\
&2 &Slips of Rs 13\\ \hline
\multirow{2}{*}{Y} &0 &Box A\\
&1 &Box B\\\hline
\end{tabular}
\caption{}
\label{tab:Distribution}
\end{table}
See \tabref{tab:Distribution}.
\begin{align}
p_{Y}\brak{k}= \begin{cases} 
      \frac{1}{3} & {k=0} \\
      \frac{2}{3 }& {k=1} 
   \end{cases}
   \\
p_{Y|X}\brak{0|0} = \frac{19}{25}\, 
p_{Y|X}\brak{0|1} = \frac{6}{25}\,
p_{Y|X}\brak{1|0} = \frac{45}{50}\,
p_{Y|X}\brak{1|2} = \frac{5}{50}
\end{align}
The desired probability is the probability that a slip drawn at random is marked other than Rs 1,
\begin{align}
&=1-p_X\brak{0}\\
&= p_X(1) + p_X(2)
\end{align}
Using Bayes theorem,
\begin{align}
&= p_Y\brak{0} \times \pr{Y=0 | X=1} + p_Y\brak{1} \times \pr{Y=1|X=2}\\
&=\frac{1}{3} \times \frac{6}{25} + \frac{2}{3} \times \frac{5}{50}\\
&=\frac{11}{75}
\end{align}

\newpage

%\tableofcontents

\bigskip

\renewcommand{\thefigure}{\theenumi}
\renewcommand{\thetable}{\theenumi}
%\renewcommand{\theequation}{\theenumi}

%\begin{abstract}
%%\boldmath
%In this letter, an algorithm for evaluating the exact analytical bit error rate  (BER)  for the piecewise linear (PL) combiner for  multiple relays is presented. Previous results were available only for upto three relays. The algorithm is unique in the sense that  the actual mathematical expressions, that are prohibitively large, need not be explicitly obtained. The diversity gain due to multiple relays is shown through plots of the analytical BER, well supported by simulations. 
%
%\end{abstract}
% IEEEtran.cls defaults to using nonbold math in the Abstract.
% This preserves the distinction between vectors and scalars. However,
% if the journal you are submitting to favors bold math in the abstract,
% then you can use LaTeX's standard command \boldmath at the very start
% of the abstract to achieve this. Many IEEE journals frown on math
% in the abstract anyway.

% Note that keywords are not normally used for peerreview papers.
%\begin{IEEEkeywords}
%Cooperative diversity, decode and forward, piecewise linear
%\end{IEEEkeywords}



% For peer review papers, you can put extra information on the cover
% page as needed:
% \ifCLASSOPTIONpeerreview
% \begin{center} \bfseries EDICS Category: 3-BBND \end{center}
% \fi
%
% For peerreview papers, this IEEEtran command inserts a page break and
% creates the second title. It will be ignored for other modes.
%\IEEEpeerreviewmaketitle




 \item A bag contain 24 balls of which $x$ balls are red, $2x$ are white and $3x$ are blue. A ball is selected at random, What is the probability that it is
\begin{enumerate}[label=\alph*)]
\item not red ?
\item white ?
\end{enumerate}
%\begin{table}[H]
	\centering
\begin{tabular}{|c|c|c|}
\hline
Random variable &Value &Definition\\ \hline
\multirow{3}{*}{X} &0 &Slips of Rs 1\\
&1 &Slips of Rs 5\\
&2 &Slips of Rs 13\\ \hline
\multirow{2}{*}{Y} &0 &Box A\\
&1 &Box B\\\hline
\end{tabular}
\caption{}
\label{tab:Distribution}
\end{table}
See \tabref{tab:Distribution}.
\begin{align}
p_{Y}\brak{k}= \begin{cases} 
      \frac{1}{3} & {k=0} \\
      \frac{2}{3 }& {k=1} 
   \end{cases}
   \\
p_{Y|X}\brak{0|0} = \frac{19}{25}\, 
p_{Y|X}\brak{0|1} = \frac{6}{25}\,
p_{Y|X}\brak{1|0} = \frac{45}{50}\,
p_{Y|X}\brak{1|2} = \frac{5}{50}
\end{align}
The desired probability is the probability that a slip drawn at random is marked other than Rs 1,
\begin{align}
&=1-p_X\brak{0}\\
&= p_X(1) + p_X(2)
\end{align}
Using Bayes theorem,
\begin{align}
&= p_Y\brak{0} \times \pr{Y=0 | X=1} + p_Y\brak{1} \times \pr{Y=1|X=2}\\
&=\frac{1}{3} \times \frac{6}{25} + \frac{2}{3} \times \frac{5}{50}\\
&=\frac{11}{75}
\end{align}

\newpage

%\tableofcontents

\bigskip

\renewcommand{\thefigure}{\theenumi}
\renewcommand{\thetable}{\theenumi}
%\renewcommand{\theequation}{\theenumi}

%\begin{abstract}
%%\boldmath
%In this letter, an algorithm for evaluating the exact analytical bit error rate  (BER)  for the piecewise linear (PL) combiner for  multiple relays is presented. Previous results were available only for upto three relays. The algorithm is unique in the sense that  the actual mathematical expressions, that are prohibitively large, need not be explicitly obtained. The diversity gain due to multiple relays is shown through plots of the analytical BER, well supported by simulations. 
%
%\end{abstract}
% IEEEtran.cls defaults to using nonbold math in the Abstract.
% This preserves the distinction between vectors and scalars. However,
% if the journal you are submitting to favors bold math in the abstract,
% then you can use LaTeX's standard command \boldmath at the very start
% of the abstract to achieve this. Many IEEE journals frown on math
% in the abstract anyway.

% Note that keywords are not normally used for peerreview papers.
%\begin{IEEEkeywords}
%Cooperative diversity, decode and forward, piecewise linear
%\end{IEEEkeywords}



% For peer review papers, you can put extra information on the cover
% page as needed:
% \ifCLASSOPTIONpeerreview
% \begin{center} \bfseries EDICS Category: 3-BBND \end{center}
% \fi
%
% For peerreview papers, this IEEEtran command inserts a page break and
% creates the second title. It will be ignored for other modes.
%\IEEEpeerreviewmaketitle




If the letters of the word ASSASSINATION are arranged at random. Find the Probability that
\begin{enumerate}[label=(\alph*)]
\item Four $S's$ come consecutively in the word
\item Two  $I's$ and two $N's$ come together
\item All $A's$ are not coming together
\item No two $A's$ are coming together
\end{enumerate}
%\begin{table}[H]
	\centering
\begin{tabular}{|c|c|c|}
\hline
Random variable &Value &Definition\\ \hline
\multirow{3}{*}{X} &0 &Slips of Rs 1\\
&1 &Slips of Rs 5\\
&2 &Slips of Rs 13\\ \hline
\multirow{2}{*}{Y} &0 &Box A\\
&1 &Box B\\\hline
\end{tabular}
\caption{}
\label{tab:Distribution}
\end{table}
See \tabref{tab:Distribution}.
\begin{align}
p_{Y}\brak{k}= \begin{cases} 
      \frac{1}{3} & {k=0} \\
      \frac{2}{3 }& {k=1} 
   \end{cases}
   \\
p_{Y|X}\brak{0|0} = \frac{19}{25}\, 
p_{Y|X}\brak{0|1} = \frac{6}{25}\,
p_{Y|X}\brak{1|0} = \frac{45}{50}\,
p_{Y|X}\brak{1|2} = \frac{5}{50}
\end{align}
The desired probability is the probability that a slip drawn at random is marked other than Rs 1,
\begin{align}
&=1-p_X\brak{0}\\
&= p_X(1) + p_X(2)
\end{align}
Using Bayes theorem,
\begin{align}
&= p_Y\brak{0} \times \pr{Y=0 | X=1} + p_Y\brak{1} \times \pr{Y=1|X=2}\\
&=\frac{1}{3} \times \frac{6}{25} + \frac{2}{3} \times \frac{5}{50}\\
&=\frac{11}{75}
\end{align}

\newpage

%\tableofcontents

\bigskip

\renewcommand{\thefigure}{\theenumi}
\renewcommand{\thetable}{\theenumi}
%\renewcommand{\theequation}{\theenumi}

%\begin{abstract}
%%\boldmath
%In this letter, an algorithm for evaluating the exact analytical bit error rate  (BER)  for the piecewise linear (PL) combiner for  multiple relays is presented. Previous results were available only for upto three relays. The algorithm is unique in the sense that  the actual mathematical expressions, that are prohibitively large, need not be explicitly obtained. The diversity gain due to multiple relays is shown through plots of the analytical BER, well supported by simulations. 
%
%\end{abstract}
% IEEEtran.cls defaults to using nonbold math in the Abstract.
% This preserves the distinction between vectors and scalars. However,
% if the journal you are submitting to favors bold math in the abstract,
% then you can use LaTeX's standard command \boldmath at the very start
% of the abstract to achieve this. Many IEEE journals frown on math
% in the abstract anyway.

% Note that keywords are not normally used for peerreview papers.
%\begin{IEEEkeywords}
%Cooperative diversity, decode and forward, piecewise linear
%\end{IEEEkeywords}



% For peer review papers, you can put extra information on the cover
% page as needed:
% \ifCLASSOPTIONpeerreview
% \begin{center} \bfseries EDICS Category: 3-BBND \end{center}
% \fi
%
% For peerreview papers, this IEEEtran command inserts a page break and
% creates the second title. It will be ignored for other modes.
%\IEEEpeerreviewmaketitle




	\item One urn contains two black balls (labelled B1 and B2) and one white ball. A
	second urn contains one black ball and two white balls (labelled W1 and W2).
	Suppose the following experiment is performed. One of the two urns is chosen
	at random. Next a ball is randomly chosen from the urn. Then a second ball is
	chosen at random from the same urn without replacing the first ball.
	
	\begin{enumerate}
	\item What is the probability that two black balls are chosen?
	
	\item What is the probability that two balls of opposite colour are chosen?
	\end{enumerate}
	\solution
	%\begin{align}
    \label{eq:12.13.6.18.1}
	\because	\pr{A|B} &> \pr{A},\
\frac{\pr{AB}}{\pr{B}} > \pr{A}
\\
    \label{eq:12.13.6.18.2}
	\implies \pr{AB} &> \pr{A}\pr{B}
	\\
	\text{or, } \frac{\pr{AB}}{\pr{A}} &=\pr{B|A} > \pr{A}
\end{align}

\end{enumerate}

	\item A bag contains 4 red and 4 black balls, another bag contains 2 red and 6 black balls. One of the two bags is selected at random and a ball is drawn from the bag which is found to be red. Find the probability that the ball is drawn from the first bag.
\\
\solution
		%\begin{table}[H]
	\centering
\begin{tabular}{|c|c|c|}
\hline
Random variable &Value &Definition\\ \hline
\multirow{3}{*}{X} &0 &Slips of Rs 1\\
&1 &Slips of Rs 5\\
&2 &Slips of Rs 13\\ \hline
\multirow{2}{*}{Y} &0 &Box A\\
&1 &Box B\\\hline
\end{tabular}
\caption{}
\label{tab:Distribution}
\end{table}
See \tabref{tab:Distribution}.
\begin{align}
p_{Y}\brak{k}= \begin{cases} 
      \frac{1}{3} & {k=0} \\
      \frac{2}{3 }& {k=1} 
   \end{cases}
   \\
p_{Y|X}\brak{0|0} = \frac{19}{25}\, 
p_{Y|X}\brak{0|1} = \frac{6}{25}\,
p_{Y|X}\brak{1|0} = \frac{45}{50}\,
p_{Y|X}\brak{1|2} = \frac{5}{50}
\end{align}
The desired probability is the probability that a slip drawn at random is marked other than Rs 1,
\begin{align}
&=1-p_X\brak{0}\\
&= p_X(1) + p_X(2)
\end{align}
Using Bayes theorem,
\begin{align}
&= p_Y\brak{0} \times \pr{Y=0 | X=1} + p_Y\brak{1} \times \pr{Y=1|X=2}\\
&=\frac{1}{3} \times \frac{6}{25} + \frac{2}{3} \times \frac{5}{50}\\
&=\frac{11}{75}
\end{align}

\newpage

%\tableofcontents

\bigskip

\renewcommand{\thefigure}{\theenumi}
\renewcommand{\thetable}{\theenumi}
%\renewcommand{\theequation}{\theenumi}

%\begin{abstract}
%%\boldmath
%In this letter, an algorithm for evaluating the exact analytical bit error rate  (BER)  for the piecewise linear (PL) combiner for  multiple relays is presented. Previous results were available only for upto three relays. The algorithm is unique in the sense that  the actual mathematical expressions, that are prohibitively large, need not be explicitly obtained. The diversity gain due to multiple relays is shown through plots of the analytical BER, well supported by simulations. 
%
%\end{abstract}
% IEEEtran.cls defaults to using nonbold math in the Abstract.
% This preserves the distinction between vectors and scalars. However,
% if the journal you are submitting to favors bold math in the abstract,
% then you can use LaTeX's standard command \boldmath at the very start
% of the abstract to achieve this. Many IEEE journals frown on math
% in the abstract anyway.

% Note that keywords are not normally used for peerreview papers.
%\begin{IEEEkeywords}
%Cooperative diversity, decode and forward, piecewise linear
%\end{IEEEkeywords}



% For peer review papers, you can put extra information on the cover
% page as needed:
% \ifCLASSOPTIONpeerreview
% \begin{center} \bfseries EDICS Category: 3-BBND \end{center}
% \fi
%
% For peerreview papers, this IEEEtran command inserts a page break and
% creates the second title. It will be ignored for other modes.
%\IEEEpeerreviewmaketitle




  \item
  Cards with numbers 2 to 101 are placed in a box. A card is selected at random.Find the probability that the card has
\begin{enumerate}[label=(\roman*)]
	\item an even number 
	\item a square number
\end{enumerate}
\solution
%\begin{table}[H]
	\centering
\begin{tabular}{|c|c|c|}
\hline
Random variable &Value &Definition\\ \hline
\multirow{3}{*}{X} &0 &Slips of Rs 1\\
&1 &Slips of Rs 5\\
&2 &Slips of Rs 13\\ \hline
\multirow{2}{*}{Y} &0 &Box A\\
&1 &Box B\\\hline
\end{tabular}
\caption{}
\label{tab:Distribution}
\end{table}
See \tabref{tab:Distribution}.
\begin{align}
p_{Y}\brak{k}= \begin{cases} 
      \frac{1}{3} & {k=0} \\
      \frac{2}{3 }& {k=1} 
   \end{cases}
   \\
p_{Y|X}\brak{0|0} = \frac{19}{25}\, 
p_{Y|X}\brak{0|1} = \frac{6}{25}\,
p_{Y|X}\brak{1|0} = \frac{45}{50}\,
p_{Y|X}\brak{1|2} = \frac{5}{50}
\end{align}
The desired probability is the probability that a slip drawn at random is marked other than Rs 1,
\begin{align}
&=1-p_X\brak{0}\\
&= p_X(1) + p_X(2)
\end{align}
Using Bayes theorem,
\begin{align}
&= p_Y\brak{0} \times \pr{Y=0 | X=1} + p_Y\brak{1} \times \pr{Y=1|X=2}\\
&=\frac{1}{3} \times \frac{6}{25} + \frac{2}{3} \times \frac{5}{50}\\
&=\frac{11}{75}
\end{align}

\newpage

%\tableofcontents

\bigskip

\renewcommand{\thefigure}{\theenumi}
\renewcommand{\thetable}{\theenumi}
%\renewcommand{\theequation}{\theenumi}

%\begin{abstract}
%%\boldmath
%In this letter, an algorithm for evaluating the exact analytical bit error rate  (BER)  for the piecewise linear (PL) combiner for  multiple relays is presented. Previous results were available only for upto three relays. The algorithm is unique in the sense that  the actual mathematical expressions, that are prohibitively large, need not be explicitly obtained. The diversity gain due to multiple relays is shown through plots of the analytical BER, well supported by simulations. 
%
%\end{abstract}
% IEEEtran.cls defaults to using nonbold math in the Abstract.
% This preserves the distinction between vectors and scalars. However,
% if the journal you are submitting to favors bold math in the abstract,
% then you can use LaTeX's standard command \boldmath at the very start
% of the abstract to achieve this. Many IEEE journals frown on math
% in the abstract anyway.

% Note that keywords are not normally used for peerreview papers.
%\begin{IEEEkeywords}
%Cooperative diversity, decode and forward, piecewise linear
%\end{IEEEkeywords}



% For peer review papers, you can put extra information on the cover
% page as needed:
% \ifCLASSOPTIONpeerreview
% \begin{center} \bfseries EDICS Category: 3-BBND \end{center}
% \fi
%
% For peerreview papers, this IEEEtran command inserts a page break and
% creates the second title. It will be ignored for other modes.
%\IEEEpeerreviewmaketitle




\item
The king, queen and jack of clubs are removed from a deck of 52 playing cards and then well shuffled. Now one card is drawn at random from the remaining cards.  Determine the probability that the card is
\begin{enumerate}[label=(\roman*)]
\item a club
\item 10 of hearts
\end{enumerate}
\solution
%\begin{table}[H]
	\centering
\begin{tabular}{|c|c|c|}
\hline
Random variable &Value &Definition\\ \hline
\multirow{3}{*}{X} &0 &Slips of Rs 1\\
&1 &Slips of Rs 5\\
&2 &Slips of Rs 13\\ \hline
\multirow{2}{*}{Y} &0 &Box A\\
&1 &Box B\\\hline
\end{tabular}
\caption{}
\label{tab:Distribution}
\end{table}
See \tabref{tab:Distribution}.
\begin{align}
p_{Y}\brak{k}= \begin{cases} 
      \frac{1}{3} & {k=0} \\
      \frac{2}{3 }& {k=1} 
   \end{cases}
   \\
p_{Y|X}\brak{0|0} = \frac{19}{25}\, 
p_{Y|X}\brak{0|1} = \frac{6}{25}\,
p_{Y|X}\brak{1|0} = \frac{45}{50}\,
p_{Y|X}\brak{1|2} = \frac{5}{50}
\end{align}
The desired probability is the probability that a slip drawn at random is marked other than Rs 1,
\begin{align}
&=1-p_X\brak{0}\\
&= p_X(1) + p_X(2)
\end{align}
Using Bayes theorem,
\begin{align}
&= p_Y\brak{0} \times \pr{Y=0 | X=1} + p_Y\brak{1} \times \pr{Y=1|X=2}\\
&=\frac{1}{3} \times \frac{6}{25} + \frac{2}{3} \times \frac{5}{50}\\
&=\frac{11}{75}
\end{align}

\newpage

%\tableofcontents

\bigskip

\renewcommand{\thefigure}{\theenumi}
\renewcommand{\thetable}{\theenumi}
%\renewcommand{\theequation}{\theenumi}

%\begin{abstract}
%%\boldmath
%In this letter, an algorithm for evaluating the exact analytical bit error rate  (BER)  for the piecewise linear (PL) combiner for  multiple relays is presented. Previous results were available only for upto three relays. The algorithm is unique in the sense that  the actual mathematical expressions, that are prohibitively large, need not be explicitly obtained. The diversity gain due to multiple relays is shown through plots of the analytical BER, well supported by simulations. 
%
%\end{abstract}
% IEEEtran.cls defaults to using nonbold math in the Abstract.
% This preserves the distinction between vectors and scalars. However,
% if the journal you are submitting to favors bold math in the abstract,
% then you can use LaTeX's standard command \boldmath at the very start
% of the abstract to achieve this. Many IEEE journals frown on math
% in the abstract anyway.

% Note that keywords are not normally used for peerreview papers.
%\begin{IEEEkeywords}
%Cooperative diversity, decode and forward, piecewise linear
%\end{IEEEkeywords}



% For peer review papers, you can put extra information on the cover
% page as needed:
% \ifCLASSOPTIONpeerreview
% \begin{center} \bfseries EDICS Category: 3-BBND \end{center}
% \fi
%
% For peerreview papers, this IEEEtran command inserts a page break and
% creates the second title. It will be ignored for other modes.
%\IEEEpeerreviewmaketitle




\item A team of medical students doing their internship have to assist during surgeries
at a city hospital. The probabilities of surgeries rated as very complex, complex,
routine, simple or very simple are respectively, 0.15, 0.20, 0.31, 0.26, .08. Find
the probabilities that a particular surgery will be rated
\begin{enumerate}
	\item complex or very complex;
	\item neither very complex nor very simple;
	\item routine or complex
	\item routine or simple
\end{enumerate}
\solution
%\begin{table}[H]
	\centering
\begin{tabular}{|c|c|c|}
\hline
Random variable &Value &Definition\\ \hline
\multirow{3}{*}{X} &0 &Slips of Rs 1\\
&1 &Slips of Rs 5\\
&2 &Slips of Rs 13\\ \hline
\multirow{2}{*}{Y} &0 &Box A\\
&1 &Box B\\\hline
\end{tabular}
\caption{}
\label{tab:Distribution}
\end{table}
See \tabref{tab:Distribution}.
\begin{align}
p_{Y}\brak{k}= \begin{cases} 
      \frac{1}{3} & {k=0} \\
      \frac{2}{3 }& {k=1} 
   \end{cases}
   \\
p_{Y|X}\brak{0|0} = \frac{19}{25}\, 
p_{Y|X}\brak{0|1} = \frac{6}{25}\,
p_{Y|X}\brak{1|0} = \frac{45}{50}\,
p_{Y|X}\brak{1|2} = \frac{5}{50}
\end{align}
The desired probability is the probability that a slip drawn at random is marked other than Rs 1,
\begin{align}
&=1-p_X\brak{0}\\
&= p_X(1) + p_X(2)
\end{align}
Using Bayes theorem,
\begin{align}
&= p_Y\brak{0} \times \pr{Y=0 | X=1} + p_Y\brak{1} \times \pr{Y=1|X=2}\\
&=\frac{1}{3} \times \frac{6}{25} + \frac{2}{3} \times \frac{5}{50}\\
&=\frac{11}{75}
\end{align}

\newpage

%\tableofcontents

\bigskip

\renewcommand{\thefigure}{\theenumi}
\renewcommand{\thetable}{\theenumi}
%\renewcommand{\theequation}{\theenumi}

%\begin{abstract}
%%\boldmath
%In this letter, an algorithm for evaluating the exact analytical bit error rate  (BER)  for the piecewise linear (PL) combiner for  multiple relays is presented. Previous results were available only for upto three relays. The algorithm is unique in the sense that  the actual mathematical expressions, that are prohibitively large, need not be explicitly obtained. The diversity gain due to multiple relays is shown through plots of the analytical BER, well supported by simulations. 
%
%\end{abstract}
% IEEEtran.cls defaults to using nonbold math in the Abstract.
% This preserves the distinction between vectors and scalars. However,
% if the journal you are submitting to favors bold math in the abstract,
% then you can use LaTeX's standard command \boldmath at the very start
% of the abstract to achieve this. Many IEEE journals frown on math
% in the abstract anyway.

% Note that keywords are not normally used for peerreview papers.
%\begin{IEEEkeywords}
%Cooperative diversity, decode and forward, piecewise linear
%\end{IEEEkeywords}



% For peer review papers, you can put extra information on the cover
% page as needed:
% \ifCLASSOPTIONpeerreview
% \begin{center} \bfseries EDICS Category: 3-BBND \end{center}
% \fi
%
% For peerreview papers, this IEEEtran command inserts a page break and
% creates the second title. It will be ignored for other modes.
%\IEEEpeerreviewmaketitle




\item A card is selected from a pack of 52 cards.
\begin{enumerate}[label=(\alph*)]
    \item How many points are there in the sample space?
    \item Calculate the probability that the card is an ace of spades.
    \item Calculate the probability that the card is (i) an ace and (ii) black card.
\end{enumerate}
\solution
%Let $X$ be an bernoulli rv defined as in \tabref{tab:exemplar/11/16/3/26}.  Then, 
\begin{equation}
    p =
        \frac{4}{11} 
\end{equation}
\begin{table}[H]
	\centering
	\input{exemplar/11/16/3/26/tables/Table2.tex}
	\caption{}
        \label{tab:exemplar/11/16/3/26}
\end{table}

\item The probability that a non leap year selected at random will contain 53 sundays.
\\
\solution
%\begin{table}[H]
	\centering
\begin{tabular}{|c|c|c|}
\hline
Random variable &Value &Definition\\ \hline
\multirow{3}{*}{X} &0 &Slips of Rs 1\\
&1 &Slips of Rs 5\\
&2 &Slips of Rs 13\\ \hline
\multirow{2}{*}{Y} &0 &Box A\\
&1 &Box B\\\hline
\end{tabular}
\caption{}
\label{tab:Distribution}
\end{table}
See \tabref{tab:Distribution}.
\begin{align}
p_{Y}\brak{k}= \begin{cases} 
      \frac{1}{3} & {k=0} \\
      \frac{2}{3 }& {k=1} 
   \end{cases}
   \\
p_{Y|X}\brak{0|0} = \frac{19}{25}\, 
p_{Y|X}\brak{0|1} = \frac{6}{25}\,
p_{Y|X}\brak{1|0} = \frac{45}{50}\,
p_{Y|X}\brak{1|2} = \frac{5}{50}
\end{align}
The desired probability is the probability that a slip drawn at random is marked other than Rs 1,
\begin{align}
&=1-p_X\brak{0}\\
&= p_X(1) + p_X(2)
\end{align}
Using Bayes theorem,
\begin{align}
&= p_Y\brak{0} \times \pr{Y=0 | X=1} + p_Y\brak{1} \times \pr{Y=1|X=2}\\
&=\frac{1}{3} \times \frac{6}{25} + \frac{2}{3} \times \frac{5}{50}\\
&=\frac{11}{75}
\end{align}

\newpage

%\tableofcontents

\bigskip

\renewcommand{\thefigure}{\theenumi}
\renewcommand{\thetable}{\theenumi}
%\renewcommand{\theequation}{\theenumi}

%\begin{abstract}
%%\boldmath
%In this letter, an algorithm for evaluating the exact analytical bit error rate  (BER)  for the piecewise linear (PL) combiner for  multiple relays is presented. Previous results were available only for upto three relays. The algorithm is unique in the sense that  the actual mathematical expressions, that are prohibitively large, need not be explicitly obtained. The diversity gain due to multiple relays is shown through plots of the analytical BER, well supported by simulations. 
%
%\end{abstract}
% IEEEtran.cls defaults to using nonbold math in the Abstract.
% This preserves the distinction between vectors and scalars. However,
% if the journal you are submitting to favors bold math in the abstract,
% then you can use LaTeX's standard command \boldmath at the very start
% of the abstract to achieve this. Many IEEE journals frown on math
% in the abstract anyway.

% Note that keywords are not normally used for peerreview papers.
%\begin{IEEEkeywords}
%Cooperative diversity, decode and forward, piecewise linear
%\end{IEEEkeywords}



% For peer review papers, you can put extra information on the cover
% page as needed:
% \ifCLASSOPTIONpeerreview
% \begin{center} \bfseries EDICS Category: 3-BBND \end{center}
% \fi
%
% For peerreview papers, this IEEEtran command inserts a page break and
% creates the second title. It will be ignored for other modes.
%\IEEEpeerreviewmaketitle




\item One of the four persons John, Rita, Aslam or Gurpreet will be promoted next
month. Consequently the sample space consists of four elementary outcomes
S = {John promoted, Rita promoted, Aslam promoted, Gurpreet promoted}
You are told that the chances of John’s promotion is same as that of Gurpreet,
Rita’s chances of promotion are twice as likely as Johns. Aslam’s chances are
four times that of John.
\begin{enumerate}
	\item Determine
	\begin{enumerate}
		\item P (John promoted)
		\item P (Rita promoted)
		\item P (Aslam promoted)
		\item P (Gurpreet promoted)
	\end{enumerate}
	\item If A = {John promoted or Gurpreet promoted}, find P (A).
\end{enumerate}
\solution
%\begin{table}[H]
	\centering
\begin{tabular}{|c|c|c|}
\hline
Random variable &Value &Definition\\ \hline
\multirow{3}{*}{X} &0 &Slips of Rs 1\\
&1 &Slips of Rs 5\\
&2 &Slips of Rs 13\\ \hline
\multirow{2}{*}{Y} &0 &Box A\\
&1 &Box B\\\hline
\end{tabular}
\caption{}
\label{tab:Distribution}
\end{table}
See \tabref{tab:Distribution}.
\begin{align}
p_{Y}\brak{k}= \begin{cases} 
      \frac{1}{3} & {k=0} \\
      \frac{2}{3 }& {k=1} 
   \end{cases}
   \\
p_{Y|X}\brak{0|0} = \frac{19}{25}\, 
p_{Y|X}\brak{0|1} = \frac{6}{25}\,
p_{Y|X}\brak{1|0} = \frac{45}{50}\,
p_{Y|X}\brak{1|2} = \frac{5}{50}
\end{align}
The desired probability is the probability that a slip drawn at random is marked other than Rs 1,
\begin{align}
&=1-p_X\brak{0}\\
&= p_X(1) + p_X(2)
\end{align}
Using Bayes theorem,
\begin{align}
&= p_Y\brak{0} \times \pr{Y=0 | X=1} + p_Y\brak{1} \times \pr{Y=1|X=2}\\
&=\frac{1}{3} \times \frac{6}{25} + \frac{2}{3} \times \frac{5}{50}\\
&=\frac{11}{75}
\end{align}

\newpage

%\tableofcontents

\bigskip

\renewcommand{\thefigure}{\theenumi}
\renewcommand{\thetable}{\theenumi}
%\renewcommand{\theequation}{\theenumi}

%\begin{abstract}
%%\boldmath
%In this letter, an algorithm for evaluating the exact analytical bit error rate  (BER)  for the piecewise linear (PL) combiner for  multiple relays is presented. Previous results were available only for upto three relays. The algorithm is unique in the sense that  the actual mathematical expressions, that are prohibitively large, need not be explicitly obtained. The diversity gain due to multiple relays is shown through plots of the analytical BER, well supported by simulations. 
%
%\end{abstract}
% IEEEtran.cls defaults to using nonbold math in the Abstract.
% This preserves the distinction between vectors and scalars. However,
% if the journal you are submitting to favors bold math in the abstract,
% then you can use LaTeX's standard command \boldmath at the very start
% of the abstract to achieve this. Many IEEE journals frown on math
% in the abstract anyway.

% Note that keywords are not normally used for peerreview papers.
%\begin{IEEEkeywords}
%Cooperative diversity, decode and forward, piecewise linear
%\end{IEEEkeywords}



% For peer review papers, you can put extra information on the cover
% page as needed:
% \ifCLASSOPTIONpeerreview
% \begin{center} \bfseries EDICS Category: 3-BBND \end{center}
% \fi
%
% For peerreview papers, this IEEEtran command inserts a page break and
% creates the second title. It will be ignored for other modes.
%\IEEEpeerreviewmaketitle




\item A card is drawn from a deck of 52 cards. Find the probability of getting a king or a heart or a red card.\\
\solution
%\begin{table}[H]
	\centering
\begin{tabular}{|c|c|c|}
\hline
Random variable &Value &Definition\\ \hline
\multirow{3}{*}{X} &0 &Slips of Rs 1\\
&1 &Slips of Rs 5\\
&2 &Slips of Rs 13\\ \hline
\multirow{2}{*}{Y} &0 &Box A\\
&1 &Box B\\\hline
\end{tabular}
\caption{}
\label{tab:Distribution}
\end{table}
See \tabref{tab:Distribution}.
\begin{align}
p_{Y}\brak{k}= \begin{cases} 
      \frac{1}{3} & {k=0} \\
      \frac{2}{3 }& {k=1} 
   \end{cases}
   \\
p_{Y|X}\brak{0|0} = \frac{19}{25}\, 
p_{Y|X}\brak{0|1} = \frac{6}{25}\,
p_{Y|X}\brak{1|0} = \frac{45}{50}\,
p_{Y|X}\brak{1|2} = \frac{5}{50}
\end{align}
The desired probability is the probability that a slip drawn at random is marked other than Rs 1,
\begin{align}
&=1-p_X\brak{0}\\
&= p_X(1) + p_X(2)
\end{align}
Using Bayes theorem,
\begin{align}
&= p_Y\brak{0} \times \pr{Y=0 | X=1} + p_Y\brak{1} \times \pr{Y=1|X=2}\\
&=\frac{1}{3} \times \frac{6}{25} + \frac{2}{3} \times \frac{5}{50}\\
&=\frac{11}{75}
\end{align}

\newpage

%\tableofcontents

\bigskip

\renewcommand{\thefigure}{\theenumi}
\renewcommand{\thetable}{\theenumi}
%\renewcommand{\theequation}{\theenumi}

%\begin{abstract}
%%\boldmath
%In this letter, an algorithm for evaluating the exact analytical bit error rate  (BER)  for the piecewise linear (PL) combiner for  multiple relays is presented. Previous results were available only for upto three relays. The algorithm is unique in the sense that  the actual mathematical expressions, that are prohibitively large, need not be explicitly obtained. The diversity gain due to multiple relays is shown through plots of the analytical BER, well supported by simulations. 
%
%\end{abstract}
% IEEEtran.cls defaults to using nonbold math in the Abstract.
% This preserves the distinction between vectors and scalars. However,
% if the journal you are submitting to favors bold math in the abstract,
% then you can use LaTeX's standard command \boldmath at the very start
% of the abstract to achieve this. Many IEEE journals frown on math
% in the abstract anyway.

% Note that keywords are not normally used for peerreview papers.
%\begin{IEEEkeywords}
%Cooperative diversity, decode and forward, piecewise linear
%\end{IEEEkeywords}



% For peer review papers, you can put extra information on the cover
% page as needed:
% \ifCLASSOPTIONpeerreview
% \begin{center} \bfseries EDICS Category: 3-BBND \end{center}
% \fi
%
% For peerreview papers, this IEEEtran command inserts a page break and
% creates the second title. It will be ignored for other modes.
%\IEEEpeerreviewmaketitle




\item The probability that a student will pass his examination is 0.73, the probability of
the student getting a compartment is 0.13, and the probability that the student will
either pass or get compartment is 0.96. State True or False.\\
\solution
%\begin{table}[H]
	\centering
\begin{tabular}{|c|c|c|}
\hline
Random variable &Value &Definition\\ \hline
\multirow{3}{*}{X} &0 &Slips of Rs 1\\
&1 &Slips of Rs 5\\
&2 &Slips of Rs 13\\ \hline
\multirow{2}{*}{Y} &0 &Box A\\
&1 &Box B\\\hline
\end{tabular}
\caption{}
\label{tab:Distribution}
\end{table}
See \tabref{tab:Distribution}.
\begin{align}
p_{Y}\brak{k}= \begin{cases} 
      \frac{1}{3} & {k=0} \\
      \frac{2}{3 }& {k=1} 
   \end{cases}
   \\
p_{Y|X}\brak{0|0} = \frac{19}{25}\, 
p_{Y|X}\brak{0|1} = \frac{6}{25}\,
p_{Y|X}\brak{1|0} = \frac{45}{50}\,
p_{Y|X}\brak{1|2} = \frac{5}{50}
\end{align}
The desired probability is the probability that a slip drawn at random is marked other than Rs 1,
\begin{align}
&=1-p_X\brak{0}\\
&= p_X(1) + p_X(2)
\end{align}
Using Bayes theorem,
\begin{align}
&= p_Y\brak{0} \times \pr{Y=0 | X=1} + p_Y\brak{1} \times \pr{Y=1|X=2}\\
&=\frac{1}{3} \times \frac{6}{25} + \frac{2}{3} \times \frac{5}{50}\\
&=\frac{11}{75}
\end{align}

\newpage

%\tableofcontents

\bigskip

\renewcommand{\thefigure}{\theenumi}
\renewcommand{\thetable}{\theenumi}
%\renewcommand{\theequation}{\theenumi}

%\begin{abstract}
%%\boldmath
%In this letter, an algorithm for evaluating the exact analytical bit error rate  (BER)  for the piecewise linear (PL) combiner for  multiple relays is presented. Previous results were available only for upto three relays. The algorithm is unique in the sense that  the actual mathematical expressions, that are prohibitively large, need not be explicitly obtained. The diversity gain due to multiple relays is shown through plots of the analytical BER, well supported by simulations. 
%
%\end{abstract}
% IEEEtran.cls defaults to using nonbold math in the Abstract.
% This preserves the distinction between vectors and scalars. However,
% if the journal you are submitting to favors bold math in the abstract,
% then you can use LaTeX's standard command \boldmath at the very start
% of the abstract to achieve this. Many IEEE journals frown on math
% in the abstract anyway.

% Note that keywords are not normally used for peerreview papers.
%\begin{IEEEkeywords}
%Cooperative diversity, decode and forward, piecewise linear
%\end{IEEEkeywords}



% For peer review papers, you can put extra information on the cover
% page as needed:
% \ifCLASSOPTIONpeerreview
% \begin{center} \bfseries EDICS Category: 3-BBND \end{center}
% \fi
%
% For peerreview papers, this IEEEtran command inserts a page break and
% creates the second title. It will be ignored for other modes.
%\IEEEpeerreviewmaketitle




\item A card is selected from a pack of 52 cards\\
\begin{enumerate}[label=(\alph*)]
\item How many points are there in the sample space?
\item Calculate the probability that the cards is an ace of spades.
\item Calculate the probability that the card is (i) an ace (ii)black card.\\
\end{enumerate}
%\input{ncert/11/16/3/4_1/Prob_4.tex}
\item In a non-leap year, the probability of having 53 tuesdays or 53 wednesdays is\\
\solution
%A non-leap year has a total of 365 days, and a week has 7 days.\\
So it can be expressed as 
\begin{align}
365\text{days} &=52\times 7+1 \text{day}
\end{align}
$\implies$ 52 tuesdays or wednesdays\\
Random variable X denotes the days of a week
\begin{align}
p_X\brak{k}&=\frac{1}{7}; \quad \brak{1<k<7}
\end{align}
So the probability of extra day being tuesday or wednesday is
\begin{align}
p_X\brak{3}+p_X\brak{4}&=\frac{1}{7}+\frac{1}{7}=\frac{2}{7}
\end{align}



\item There are 1000 sealed envelopes in a box, 10 of them contain a cash prize of
Rs 100 each, 100 of them contain a cash prize of Rs 50 each and 200 of them
contain a cash prize of Rs 10 each and rest do not contain any cash prize. If they
are well shuffled and an envelope is picked up out, what is the probability that it
contains no cash prize?\\
\solution
%\begin{table}[H]
	\centering
\begin{tabular}{|c|c|c|}
\hline
Random variable &Value &Definition\\ \hline
\multirow{3}{*}{X} &0 &Slips of Rs 1\\
&1 &Slips of Rs 5\\
&2 &Slips of Rs 13\\ \hline
\multirow{2}{*}{Y} &0 &Box A\\
&1 &Box B\\\hline
\end{tabular}
\caption{}
\label{tab:Distribution}
\end{table}
See \tabref{tab:Distribution}.
\begin{align}
p_{Y}\brak{k}= \begin{cases} 
      \frac{1}{3} & {k=0} \\
      \frac{2}{3 }& {k=1} 
   \end{cases}
   \\
p_{Y|X}\brak{0|0} = \frac{19}{25}\, 
p_{Y|X}\brak{0|1} = \frac{6}{25}\,
p_{Y|X}\brak{1|0} = \frac{45}{50}\,
p_{Y|X}\brak{1|2} = \frac{5}{50}
\end{align}
The desired probability is the probability that a slip drawn at random is marked other than Rs 1,
\begin{align}
&=1-p_X\brak{0}\\
&= p_X(1) + p_X(2)
\end{align}
Using Bayes theorem,
\begin{align}
&= p_Y\brak{0} \times \pr{Y=0 | X=1} + p_Y\brak{1} \times \pr{Y=1|X=2}\\
&=\frac{1}{3} \times \frac{6}{25} + \frac{2}{3} \times \frac{5}{50}\\
&=\frac{11}{75}
\end{align}

\newpage

%\tableofcontents

\bigskip

\renewcommand{\thefigure}{\theenumi}
\renewcommand{\thetable}{\theenumi}
%\renewcommand{\theequation}{\theenumi}

%\begin{abstract}
%%\boldmath
%In this letter, an algorithm for evaluating the exact analytical bit error rate  (BER)  for the piecewise linear (PL) combiner for  multiple relays is presented. Previous results were available only for upto three relays. The algorithm is unique in the sense that  the actual mathematical expressions, that are prohibitively large, need not be explicitly obtained. The diversity gain due to multiple relays is shown through plots of the analytical BER, well supported by simulations. 
%
%\end{abstract}
% IEEEtran.cls defaults to using nonbold math in the Abstract.
% This preserves the distinction between vectors and scalars. However,
% if the journal you are submitting to favors bold math in the abstract,
% then you can use LaTeX's standard command \boldmath at the very start
% of the abstract to achieve this. Many IEEE journals frown on math
% in the abstract anyway.

% Note that keywords are not normally used for peerreview papers.
%\begin{IEEEkeywords}
%Cooperative diversity, decode and forward, piecewise linear
%\end{IEEEkeywords}



% For peer review papers, you can put extra information on the cover
% page as needed:
% \ifCLASSOPTIONpeerreview
% \begin{center} \bfseries EDICS Category: 3-BBND \end{center}
% \fi
%
% For peerreview papers, this IEEEtran command inserts a page break and
% creates the second title. It will be ignored for other modes.
%\IEEEpeerreviewmaketitle




\item 
A die is thrown and a card is selected at random from a deck of 52 playing cards. The probability of getting an even number on the die and a spade card.\\
\solution
%\begin{table}[H]
	\centering
\begin{tabular}{|c|c|c|}
\hline
Random variable &Value &Definition\\ \hline
\multirow{3}{*}{X} &0 &Slips of Rs 1\\
&1 &Slips of Rs 5\\
&2 &Slips of Rs 13\\ \hline
\multirow{2}{*}{Y} &0 &Box A\\
&1 &Box B\\\hline
\end{tabular}
\caption{}
\label{tab:Distribution}
\end{table}
See \tabref{tab:Distribution}.
\begin{align}
p_{Y}\brak{k}= \begin{cases} 
      \frac{1}{3} & {k=0} \\
      \frac{2}{3 }& {k=1} 
   \end{cases}
   \\
p_{Y|X}\brak{0|0} = \frac{19}{25}\, 
p_{Y|X}\brak{0|1} = \frac{6}{25}\,
p_{Y|X}\brak{1|0} = \frac{45}{50}\,
p_{Y|X}\brak{1|2} = \frac{5}{50}
\end{align}
The desired probability is the probability that a slip drawn at random is marked other than Rs 1,
\begin{align}
&=1-p_X\brak{0}\\
&= p_X(1) + p_X(2)
\end{align}
Using Bayes theorem,
\begin{align}
&= p_Y\brak{0} \times \pr{Y=0 | X=1} + p_Y\brak{1} \times \pr{Y=1|X=2}\\
&=\frac{1}{3} \times \frac{6}{25} + \frac{2}{3} \times \frac{5}{50}\\
&=\frac{11}{75}
\end{align}

\newpage

%\tableofcontents

\bigskip

\renewcommand{\thefigure}{\theenumi}
\renewcommand{\thetable}{\theenumi}
%\renewcommand{\theequation}{\theenumi}

%\begin{abstract}
%%\boldmath
%In this letter, an algorithm for evaluating the exact analytical bit error rate  (BER)  for the piecewise linear (PL) combiner for  multiple relays is presented. Previous results were available only for upto three relays. The algorithm is unique in the sense that  the actual mathematical expressions, that are prohibitively large, need not be explicitly obtained. The diversity gain due to multiple relays is shown through plots of the analytical BER, well supported by simulations. 
%
%\end{abstract}
% IEEEtran.cls defaults to using nonbold math in the Abstract.
% This preserves the distinction between vectors and scalars. However,
% if the journal you are submitting to favors bold math in the abstract,
% then you can use LaTeX's standard command \boldmath at the very start
% of the abstract to achieve this. Many IEEE journals frown on math
% in the abstract anyway.

% Note that keywords are not normally used for peerreview papers.
%\begin{IEEEkeywords}
%Cooperative diversity, decode and forward, piecewise linear
%\end{IEEEkeywords}



% For peer review papers, you can put extra information on the cover
% page as needed:
% \ifCLASSOPTIONpeerreview
% \begin{center} \bfseries EDICS Category: 3-BBND \end{center}
% \fi
%
% For peerreview papers, this IEEEtran command inserts a page break and
% creates the second title. It will be ignored for other modes.
%\IEEEpeerreviewmaketitle




\item
If 4-digit numbers greater than 5,000 are randomly formed from the digits 0, 1, 3, 5, and 7, what is the probability of forming a number divisible by 5 when:
\begin{enumerate}
    \item The digits are repeated?
    \item The repetition of digits is not allowed?
\end{enumerate}
\solution
%\begin{table}[H]
	\centering
\begin{tabular}{|c|c|c|}
\hline
Random variable &Value &Definition\\ \hline
\multirow{3}{*}{X} &0 &Slips of Rs 1\\
&1 &Slips of Rs 5\\
&2 &Slips of Rs 13\\ \hline
\multirow{2}{*}{Y} &0 &Box A\\
&1 &Box B\\\hline
\end{tabular}
\caption{}
\label{tab:Distribution}
\end{table}
See \tabref{tab:Distribution}.
\begin{align}
p_{Y}\brak{k}= \begin{cases} 
      \frac{1}{3} & {k=0} \\
      \frac{2}{3 }& {k=1} 
   \end{cases}
   \\
p_{Y|X}\brak{0|0} = \frac{19}{25}\, 
p_{Y|X}\brak{0|1} = \frac{6}{25}\,
p_{Y|X}\brak{1|0} = \frac{45}{50}\,
p_{Y|X}\brak{1|2} = \frac{5}{50}
\end{align}
The desired probability is the probability that a slip drawn at random is marked other than Rs 1,
\begin{align}
&=1-p_X\brak{0}\\
&= p_X(1) + p_X(2)
\end{align}
Using Bayes theorem,
\begin{align}
&= p_Y\brak{0} \times \pr{Y=0 | X=1} + p_Y\brak{1} \times \pr{Y=1|X=2}\\
&=\frac{1}{3} \times \frac{6}{25} + \frac{2}{3} \times \frac{5}{50}\\
&=\frac{11}{75}
\end{align}

\newpage

%\tableofcontents

\bigskip

\renewcommand{\thefigure}{\theenumi}
\renewcommand{\thetable}{\theenumi}
%\renewcommand{\theequation}{\theenumi}

%\begin{abstract}
%%\boldmath
%In this letter, an algorithm for evaluating the exact analytical bit error rate  (BER)  for the piecewise linear (PL) combiner for  multiple relays is presented. Previous results were available only for upto three relays. The algorithm is unique in the sense that  the actual mathematical expressions, that are prohibitively large, need not be explicitly obtained. The diversity gain due to multiple relays is shown through plots of the analytical BER, well supported by simulations. 
%
%\end{abstract}
% IEEEtran.cls defaults to using nonbold math in the Abstract.
% This preserves the distinction between vectors and scalars. However,
% if the journal you are submitting to favors bold math in the abstract,
% then you can use LaTeX's standard command \boldmath at the very start
% of the abstract to achieve this. Many IEEE journals frown on math
% in the abstract anyway.

% Note that keywords are not normally used for peerreview papers.
%\begin{IEEEkeywords}
%Cooperative diversity, decode and forward, piecewise linear
%\end{IEEEkeywords}



% For peer review papers, you can put extra information on the cover
% page as needed:
% \ifCLASSOPTIONpeerreview
% \begin{center} \bfseries EDICS Category: 3-BBND \end{center}
% \fi
%
% For peerreview papers, this IEEEtran command inserts a page break and
% creates the second title. It will be ignored for other modes.
%\IEEEpeerreviewmaketitle




\item Consider the probability space $\brak{\Omega, \mathcal{G}, P}$ where $\Omega = [0,2]$ and $\mathcal{G} = \cbrak{\phi, \Omega, [0,1], (1,2]}$. Let $X$ and $Y$ be two functions on $\Omega$ defined as
\begin{align*}
    X(\omega) = 
    \begin{cases}
        1 & \text{if }\omega \in [0, 1]\\
        2 & \text{if }\omega \in (1, 2]
    \end{cases}
\end{align*}
and
\begin{align*}
    Y(\omega) = 
    \begin{cases}
        2 & \text{if }\omega \in [0, 1.5]\\
        3 & \text{if }\omega \in (1.5, 2].
    \end{cases}
\end{align*}
Then which one of the following statements is true?
\begin{enumerate}
    \item [(A)] $X$ is a random variable with respect to $\mathcal{G}$, but $Y$ is not a random variable with respect to $\mathcal{G}$.
    \item [(B)] $Y$ is a random variable with respect to $\mathcal{G}$, but $X$ is not a random variable with respect to $\mathcal{G}$.
    \item [(C)] Neither $X$ nor $Y$ is a random variable with respect to $\mathcal{G}$.
    \item [(D)] Both $X$ and $Y$ are random variables with respect to $\mathcal{G}$.
\end{enumerate} \hfill (GATE ST 2023)\\
\solution
%\begin{table}[H]
	\centering
\begin{tabular}{|c|c|c|}
\hline
Random variable &Value &Definition\\ \hline
\multirow{3}{*}{X} &0 &Slips of Rs 1\\
&1 &Slips of Rs 5\\
&2 &Slips of Rs 13\\ \hline
\multirow{2}{*}{Y} &0 &Box A\\
&1 &Box B\\\hline
\end{tabular}
\caption{}
\label{tab:Distribution}
\end{table}
See \tabref{tab:Distribution}.
\begin{align}
p_{Y}\brak{k}= \begin{cases} 
      \frac{1}{3} & {k=0} \\
      \frac{2}{3 }& {k=1} 
   \end{cases}
   \\
p_{Y|X}\brak{0|0} = \frac{19}{25}\, 
p_{Y|X}\brak{0|1} = \frac{6}{25}\,
p_{Y|X}\brak{1|0} = \frac{45}{50}\,
p_{Y|X}\brak{1|2} = \frac{5}{50}
\end{align}
The desired probability is the probability that a slip drawn at random is marked other than Rs 1,
\begin{align}
&=1-p_X\brak{0}\\
&= p_X(1) + p_X(2)
\end{align}
Using Bayes theorem,
\begin{align}
&= p_Y\brak{0} \times \pr{Y=0 | X=1} + p_Y\brak{1} \times \pr{Y=1|X=2}\\
&=\frac{1}{3} \times \frac{6}{25} + \frac{2}{3} \times \frac{5}{50}\\
&=\frac{11}{75}
\end{align}

\newpage

%\tableofcontents

\bigskip

\renewcommand{\thefigure}{\theenumi}
\renewcommand{\thetable}{\theenumi}
%\renewcommand{\theequation}{\theenumi}

%\begin{abstract}
%%\boldmath
%In this letter, an algorithm for evaluating the exact analytical bit error rate  (BER)  for the piecewise linear (PL) combiner for  multiple relays is presented. Previous results were available only for upto three relays. The algorithm is unique in the sense that  the actual mathematical expressions, that are prohibitively large, need not be explicitly obtained. The diversity gain due to multiple relays is shown through plots of the analytical BER, well supported by simulations. 
%
%\end{abstract}
% IEEEtran.cls defaults to using nonbold math in the Abstract.
% This preserves the distinction between vectors and scalars. However,
% if the journal you are submitting to favors bold math in the abstract,
% then you can use LaTeX's standard command \boldmath at the very start
% of the abstract to achieve this. Many IEEE journals frown on math
% in the abstract anyway.

% Note that keywords are not normally used for peerreview papers.
%\begin{IEEEkeywords}
%Cooperative diversity, decode and forward, piecewise linear
%\end{IEEEkeywords}



% For peer review papers, you can put extra information on the cover
% page as needed:
% \ifCLASSOPTIONpeerreview
% \begin{center} \bfseries EDICS Category: 3-BBND \end{center}
% \fi
%
% For peerreview papers, this IEEEtran command inserts a page break and
% creates the second title. It will be ignored for other modes.
%\IEEEpeerreviewmaketitle




	\item  A die is loaded in such a way that each odd number is twice as likely to occur as
each even number. Find $P(G)$, where $G$ is the event that a number greater than
3 occurs on a single roll of the die.
\\
\solution
		%\begin{table}[H]
	\centering
\begin{tabular}{|c|c|c|}
\hline
Random variable &Value &Definition\\ \hline
\multirow{3}{*}{X} &0 &Slips of Rs 1\\
&1 &Slips of Rs 5\\
&2 &Slips of Rs 13\\ \hline
\multirow{2}{*}{Y} &0 &Box A\\
&1 &Box B\\\hline
\end{tabular}
\caption{}
\label{tab:Distribution}
\end{table}
See \tabref{tab:Distribution}.
\begin{align}
p_{Y}\brak{k}= \begin{cases} 
      \frac{1}{3} & {k=0} \\
      \frac{2}{3 }& {k=1} 
   \end{cases}
   \\
p_{Y|X}\brak{0|0} = \frac{19}{25}\, 
p_{Y|X}\brak{0|1} = \frac{6}{25}\,
p_{Y|X}\brak{1|0} = \frac{45}{50}\,
p_{Y|X}\brak{1|2} = \frac{5}{50}
\end{align}
The desired probability is the probability that a slip drawn at random is marked other than Rs 1,
\begin{align}
&=1-p_X\brak{0}\\
&= p_X(1) + p_X(2)
\end{align}
Using Bayes theorem,
\begin{align}
&= p_Y\brak{0} \times \pr{Y=0 | X=1} + p_Y\brak{1} \times \pr{Y=1|X=2}\\
&=\frac{1}{3} \times \frac{6}{25} + \frac{2}{3} \times \frac{5}{50}\\
&=\frac{11}{75}
\end{align}

\newpage

%\tableofcontents

\bigskip

\renewcommand{\thefigure}{\theenumi}
\renewcommand{\thetable}{\theenumi}
%\renewcommand{\theequation}{\theenumi}

%\begin{abstract}
%%\boldmath
%In this letter, an algorithm for evaluating the exact analytical bit error rate  (BER)  for the piecewise linear (PL) combiner for  multiple relays is presented. Previous results were available only for upto three relays. The algorithm is unique in the sense that  the actual mathematical expressions, that are prohibitively large, need not be explicitly obtained. The diversity gain due to multiple relays is shown through plots of the analytical BER, well supported by simulations. 
%
%\end{abstract}
% IEEEtran.cls defaults to using nonbold math in the Abstract.
% This preserves the distinction between vectors and scalars. However,
% if the journal you are submitting to favors bold math in the abstract,
% then you can use LaTeX's standard command \boldmath at the very start
% of the abstract to achieve this. Many IEEE journals frown on math
% in the abstract anyway.

% Note that keywords are not normally used for peerreview papers.
%\begin{IEEEkeywords}
%Cooperative diversity, decode and forward, piecewise linear
%\end{IEEEkeywords}



% For peer review papers, you can put extra information on the cover
% page as needed:
% \ifCLASSOPTIONpeerreview
% \begin{center} \bfseries EDICS Category: 3-BBND \end{center}
% \fi
%
% For peerreview papers, this IEEEtran command inserts a page break and
% creates the second title. It will be ignored for other modes.
%\IEEEpeerreviewmaketitle




	\item All the jacks, queens and kings are removed from a deck of 52 playing cards. The remaining cards are well shuffled and then one card is drawn at random. Giving ace a value 1 similar value for other cards, find the probability that the card has a value 
		\begin{enumerate}
			\item 7
			\item greater than 7
			\item less than 7
		\end{enumerate}
		%Number of cards left after removing all jacks, queens and kings 
\begin{align}
N	= 52 - 4\times 3
	= 40
\end{align}
%\begin{table}[H]
%\def\arraystretch{1.2}
%\begin{tabular}{|c|c|c|}
%\hline
%	\textbf{Parameter} &\textbf{Value} &\textbf{Description}\\ \hline
%	$X$ &1-10 &Represents the value of the card picked \\ \hline
%\end{tabular}
%\end{table}
Let $1 \le X \le 10$ be the value of the card picked.  Then,
\begin{align}
	p_X(k) &= \Pr(X=k)\ \forall\ 1 \leq k \leq 10\\
	&= \frac{4\times 1}{40}\\
	&= \frac{1}{10}\\
	\therefore p_X(k) &= 
	\begin{cases}
		\frac{1}{10} & 1 \leq k \leq 10\\
		0 & \text{otherwise}
	\end{cases}
\end{align}
and
\begin{align}
	F_{X}(k) &= \sum_{m=0}^{k}p_{X}(m) \quad 1 \leq k \leq 10\\
	&= \frac{k}{10}\\
	\therefore F_{X}(k) &= 
	\begin{cases}
		0 & k \leq 0\\
		\frac{k}{10} & 1\leq k \leq 10\\
		1 & k > 10 
	\end{cases}
\end{align}
\begin{enumerate}
	\item Probability that card has value equal to 7 is
		\begin{align}
			 p_{X}(7)
			= \frac{1}{10}
		\end{align}
	\item Probability that card has value greater than 7 is
		\begin{align}
			1 - F_X(7)
			&= 1 - \frac{7}{10}
			\\
			&= \frac{3}{10}
		\end{align}
	\item Probability that card has value less than 7 is
		\begin{align}
			 F_{X}(6)
			=\frac{6}{10}
		\end{align}
\end{enumerate}

  \item A Lot consists of 48 mobile phones of which 42 are good, 3 have only minor defects and 3 have major defects.Varnika will buy a phone if it is good but the trader will only buy a mobile if it has no major defects. One phone is selected at random from the lot. What is the probability that it is
\begin{enumerate}
	\item acceptable to Varnika?
            \item acceptable to the trader?
\end{enumerate}
\solution
	%\begin{table}[H]
	\centering
\begin{tabular}{|c|c|c|}
\hline
Random variable &Value &Definition\\ \hline
\multirow{3}{*}{X} &0 &Slips of Rs 1\\
&1 &Slips of Rs 5\\
&2 &Slips of Rs 13\\ \hline
\multirow{2}{*}{Y} &0 &Box A\\
&1 &Box B\\\hline
\end{tabular}
\caption{}
\label{tab:Distribution}
\end{table}
See \tabref{tab:Distribution}.
\begin{align}
p_{Y}\brak{k}= \begin{cases} 
      \frac{1}{3} & {k=0} \\
      \frac{2}{3 }& {k=1} 
   \end{cases}
   \\
p_{Y|X}\brak{0|0} = \frac{19}{25}\, 
p_{Y|X}\brak{0|1} = \frac{6}{25}\,
p_{Y|X}\brak{1|0} = \frac{45}{50}\,
p_{Y|X}\brak{1|2} = \frac{5}{50}
\end{align}
The desired probability is the probability that a slip drawn at random is marked other than Rs 1,
\begin{align}
&=1-p_X\brak{0}\\
&= p_X(1) + p_X(2)
\end{align}
Using Bayes theorem,
\begin{align}
&= p_Y\brak{0} \times \pr{Y=0 | X=1} + p_Y\brak{1} \times \pr{Y=1|X=2}\\
&=\frac{1}{3} \times \frac{6}{25} + \frac{2}{3} \times \frac{5}{50}\\
&=\frac{11}{75}
\end{align}

\newpage

%\tableofcontents

\bigskip

\renewcommand{\thefigure}{\theenumi}
\renewcommand{\thetable}{\theenumi}
%\renewcommand{\theequation}{\theenumi}

%\begin{abstract}
%%\boldmath
%In this letter, an algorithm for evaluating the exact analytical bit error rate  (BER)  for the piecewise linear (PL) combiner for  multiple relays is presented. Previous results were available only for upto three relays. The algorithm is unique in the sense that  the actual mathematical expressions, that are prohibitively large, need not be explicitly obtained. The diversity gain due to multiple relays is shown through plots of the analytical BER, well supported by simulations. 
%
%\end{abstract}
% IEEEtran.cls defaults to using nonbold math in the Abstract.
% This preserves the distinction between vectors and scalars. However,
% if the journal you are submitting to favors bold math in the abstract,
% then you can use LaTeX's standard command \boldmath at the very start
% of the abstract to achieve this. Many IEEE journals frown on math
% in the abstract anyway.

% Note that keywords are not normally used for peerreview papers.
%\begin{IEEEkeywords}
%Cooperative diversity, decode and forward, piecewise linear
%\end{IEEEkeywords}



% For peer review papers, you can put extra information on the cover
% page as needed:
% \ifCLASSOPTIONpeerreview
% \begin{center} \bfseries EDICS Category: 3-BBND \end{center}
% \fi
%
% For peerreview papers, this IEEEtran command inserts a page break and
% creates the second title. It will be ignored for other modes.
%\IEEEpeerreviewmaketitle




 \item A student says that if you throw a die, it will show up 1 or not 1. Therefore, the probability of getting 1 and the probability of getting 'not 1' each is equal to $\frac{1}{2}$. Is this correct? Give reasons.\\
 \solution
        %\begin{table}[H]
	\centering
\begin{tabular}{|c|c|c|}
\hline
Random variable &Value &Definition\\ \hline
\multirow{3}{*}{X} &0 &Slips of Rs 1\\
&1 &Slips of Rs 5\\
&2 &Slips of Rs 13\\ \hline
\multirow{2}{*}{Y} &0 &Box A\\
&1 &Box B\\\hline
\end{tabular}
\caption{}
\label{tab:Distribution}
\end{table}
See \tabref{tab:Distribution}.
\begin{align}
p_{Y}\brak{k}= \begin{cases} 
      \frac{1}{3} & {k=0} \\
      \frac{2}{3 }& {k=1} 
   \end{cases}
   \\
p_{Y|X}\brak{0|0} = \frac{19}{25}\, 
p_{Y|X}\brak{0|1} = \frac{6}{25}\,
p_{Y|X}\brak{1|0} = \frac{45}{50}\,
p_{Y|X}\brak{1|2} = \frac{5}{50}
\end{align}
The desired probability is the probability that a slip drawn at random is marked other than Rs 1,
\begin{align}
&=1-p_X\brak{0}\\
&= p_X(1) + p_X(2)
\end{align}
Using Bayes theorem,
\begin{align}
&= p_Y\brak{0} \times \pr{Y=0 | X=1} + p_Y\brak{1} \times \pr{Y=1|X=2}\\
&=\frac{1}{3} \times \frac{6}{25} + \frac{2}{3} \times \frac{5}{50}\\
&=\frac{11}{75}
\end{align}

\newpage

%\tableofcontents

\bigskip

\renewcommand{\thefigure}{\theenumi}
\renewcommand{\thetable}{\theenumi}
%\renewcommand{\theequation}{\theenumi}

%\begin{abstract}
%%\boldmath
%In this letter, an algorithm for evaluating the exact analytical bit error rate  (BER)  for the piecewise linear (PL) combiner for  multiple relays is presented. Previous results were available only for upto three relays. The algorithm is unique in the sense that  the actual mathematical expressions, that are prohibitively large, need not be explicitly obtained. The diversity gain due to multiple relays is shown through plots of the analytical BER, well supported by simulations. 
%
%\end{abstract}
% IEEEtran.cls defaults to using nonbold math in the Abstract.
% This preserves the distinction between vectors and scalars. However,
% if the journal you are submitting to favors bold math in the abstract,
% then you can use LaTeX's standard command \boldmath at the very start
% of the abstract to achieve this. Many IEEE journals frown on math
% in the abstract anyway.

% Note that keywords are not normally used for peerreview papers.
%\begin{IEEEkeywords}
%Cooperative diversity, decode and forward, piecewise linear
%\end{IEEEkeywords}



% For peer review papers, you can put extra information on the cover
% page as needed:
% \ifCLASSOPTIONpeerreview
% \begin{center} \bfseries EDICS Category: 3-BBND \end{center}
% \fi
%
% For peerreview papers, this IEEEtran command inserts a page break and
% creates the second title. It will be ignored for other modes.
%\IEEEpeerreviewmaketitle




   \item Four candidates A, B, C, D have ap-
plied for the assignment to coach a school cricket
team. If A is twice as likely to be selected as B, and
B and C are given about the same chance of being
selected, while C is twice as likely to be selected
as D, what are the probabilities that
\begin{enumerate}
\item C will be selected?
\item A will not be selected?
\end{enumerate}
	%\begin{table}[H]
	\centering
\begin{tabular}{|c|c|c|}
\hline
Random variable &Value &Definition\\ \hline
\multirow{3}{*}{X} &0 &Slips of Rs 1\\
&1 &Slips of Rs 5\\
&2 &Slips of Rs 13\\ \hline
\multirow{2}{*}{Y} &0 &Box A\\
&1 &Box B\\\hline
\end{tabular}
\caption{}
\label{tab:Distribution}
\end{table}
See \tabref{tab:Distribution}.
\begin{align}
p_{Y}\brak{k}= \begin{cases} 
      \frac{1}{3} & {k=0} \\
      \frac{2}{3 }& {k=1} 
   \end{cases}
   \\
p_{Y|X}\brak{0|0} = \frac{19}{25}\, 
p_{Y|X}\brak{0|1} = \frac{6}{25}\,
p_{Y|X}\brak{1|0} = \frac{45}{50}\,
p_{Y|X}\brak{1|2} = \frac{5}{50}
\end{align}
The desired probability is the probability that a slip drawn at random is marked other than Rs 1,
\begin{align}
&=1-p_X\brak{0}\\
&= p_X(1) + p_X(2)
\end{align}
Using Bayes theorem,
\begin{align}
&= p_Y\brak{0} \times \pr{Y=0 | X=1} + p_Y\brak{1} \times \pr{Y=1|X=2}\\
&=\frac{1}{3} \times \frac{6}{25} + \frac{2}{3} \times \frac{5}{50}\\
&=\frac{11}{75}
\end{align}

\newpage

%\tableofcontents

\bigskip

\renewcommand{\thefigure}{\theenumi}
\renewcommand{\thetable}{\theenumi}
%\renewcommand{\theequation}{\theenumi}

%\begin{abstract}
%%\boldmath
%In this letter, an algorithm for evaluating the exact analytical bit error rate  (BER)  for the piecewise linear (PL) combiner for  multiple relays is presented. Previous results were available only for upto three relays. The algorithm is unique in the sense that  the actual mathematical expressions, that are prohibitively large, need not be explicitly obtained. The diversity gain due to multiple relays is shown through plots of the analytical BER, well supported by simulations. 
%
%\end{abstract}
% IEEEtran.cls defaults to using nonbold math in the Abstract.
% This preserves the distinction between vectors and scalars. However,
% if the journal you are submitting to favors bold math in the abstract,
% then you can use LaTeX's standard command \boldmath at the very start
% of the abstract to achieve this. Many IEEE journals frown on math
% in the abstract anyway.

% Note that keywords are not normally used for peerreview papers.
%\begin{IEEEkeywords}
%Cooperative diversity, decode and forward, piecewise linear
%\end{IEEEkeywords}



% For peer review papers, you can put extra information on the cover
% page as needed:
% \ifCLASSOPTIONpeerreview
% \begin{center} \bfseries EDICS Category: 3-BBND \end{center}
% \fi
%
% For peerreview papers, this IEEEtran command inserts a page break and
% creates the second title. It will be ignored for other modes.
%\IEEEpeerreviewmaketitle




 \item A bag contain 24 balls of which $x$ balls are red, $2x$ are white and $3x$ are blue. A ball is selected at random, What is the probability that it is
\begin{enumerate}[label=\alph*)]
\item not red ?
\item white ?
\end{enumerate}
%\begin{table}[H]
	\centering
\begin{tabular}{|c|c|c|}
\hline
Random variable &Value &Definition\\ \hline
\multirow{3}{*}{X} &0 &Slips of Rs 1\\
&1 &Slips of Rs 5\\
&2 &Slips of Rs 13\\ \hline
\multirow{2}{*}{Y} &0 &Box A\\
&1 &Box B\\\hline
\end{tabular}
\caption{}
\label{tab:Distribution}
\end{table}
See \tabref{tab:Distribution}.
\begin{align}
p_{Y}\brak{k}= \begin{cases} 
      \frac{1}{3} & {k=0} \\
      \frac{2}{3 }& {k=1} 
   \end{cases}
   \\
p_{Y|X}\brak{0|0} = \frac{19}{25}\, 
p_{Y|X}\brak{0|1} = \frac{6}{25}\,
p_{Y|X}\brak{1|0} = \frac{45}{50}\,
p_{Y|X}\brak{1|2} = \frac{5}{50}
\end{align}
The desired probability is the probability that a slip drawn at random is marked other than Rs 1,
\begin{align}
&=1-p_X\brak{0}\\
&= p_X(1) + p_X(2)
\end{align}
Using Bayes theorem,
\begin{align}
&= p_Y\brak{0} \times \pr{Y=0 | X=1} + p_Y\brak{1} \times \pr{Y=1|X=2}\\
&=\frac{1}{3} \times \frac{6}{25} + \frac{2}{3} \times \frac{5}{50}\\
&=\frac{11}{75}
\end{align}

\newpage

%\tableofcontents

\bigskip

\renewcommand{\thefigure}{\theenumi}
\renewcommand{\thetable}{\theenumi}
%\renewcommand{\theequation}{\theenumi}

%\begin{abstract}
%%\boldmath
%In this letter, an algorithm for evaluating the exact analytical bit error rate  (BER)  for the piecewise linear (PL) combiner for  multiple relays is presented. Previous results were available only for upto three relays. The algorithm is unique in the sense that  the actual mathematical expressions, that are prohibitively large, need not be explicitly obtained. The diversity gain due to multiple relays is shown through plots of the analytical BER, well supported by simulations. 
%
%\end{abstract}
% IEEEtran.cls defaults to using nonbold math in the Abstract.
% This preserves the distinction between vectors and scalars. However,
% if the journal you are submitting to favors bold math in the abstract,
% then you can use LaTeX's standard command \boldmath at the very start
% of the abstract to achieve this. Many IEEE journals frown on math
% in the abstract anyway.

% Note that keywords are not normally used for peerreview papers.
%\begin{IEEEkeywords}
%Cooperative diversity, decode and forward, piecewise linear
%\end{IEEEkeywords}



% For peer review papers, you can put extra information on the cover
% page as needed:
% \ifCLASSOPTIONpeerreview
% \begin{center} \bfseries EDICS Category: 3-BBND \end{center}
% \fi
%
% For peerreview papers, this IEEEtran command inserts a page break and
% creates the second title. It will be ignored for other modes.
%\IEEEpeerreviewmaketitle




If the letters of the word ASSASSINATION are arranged at random. Find the Probability that
\begin{enumerate}[label=(\alph*)]
\item Four $S's$ come consecutively in the word
\item Two  $I's$ and two $N's$ come together
\item All $A's$ are not coming together
\item No two $A's$ are coming together
\end{enumerate}
%\begin{table}[H]
	\centering
\begin{tabular}{|c|c|c|}
\hline
Random variable &Value &Definition\\ \hline
\multirow{3}{*}{X} &0 &Slips of Rs 1\\
&1 &Slips of Rs 5\\
&2 &Slips of Rs 13\\ \hline
\multirow{2}{*}{Y} &0 &Box A\\
&1 &Box B\\\hline
\end{tabular}
\caption{}
\label{tab:Distribution}
\end{table}
See \tabref{tab:Distribution}.
\begin{align}
p_{Y}\brak{k}= \begin{cases} 
      \frac{1}{3} & {k=0} \\
      \frac{2}{3 }& {k=1} 
   \end{cases}
   \\
p_{Y|X}\brak{0|0} = \frac{19}{25}\, 
p_{Y|X}\brak{0|1} = \frac{6}{25}\,
p_{Y|X}\brak{1|0} = \frac{45}{50}\,
p_{Y|X}\brak{1|2} = \frac{5}{50}
\end{align}
The desired probability is the probability that a slip drawn at random is marked other than Rs 1,
\begin{align}
&=1-p_X\brak{0}\\
&= p_X(1) + p_X(2)
\end{align}
Using Bayes theorem,
\begin{align}
&= p_Y\brak{0} \times \pr{Y=0 | X=1} + p_Y\brak{1} \times \pr{Y=1|X=2}\\
&=\frac{1}{3} \times \frac{6}{25} + \frac{2}{3} \times \frac{5}{50}\\
&=\frac{11}{75}
\end{align}

\newpage

%\tableofcontents

\bigskip

\renewcommand{\thefigure}{\theenumi}
\renewcommand{\thetable}{\theenumi}
%\renewcommand{\theequation}{\theenumi}

%\begin{abstract}
%%\boldmath
%In this letter, an algorithm for evaluating the exact analytical bit error rate  (BER)  for the piecewise linear (PL) combiner for  multiple relays is presented. Previous results were available only for upto three relays. The algorithm is unique in the sense that  the actual mathematical expressions, that are prohibitively large, need not be explicitly obtained. The diversity gain due to multiple relays is shown through plots of the analytical BER, well supported by simulations. 
%
%\end{abstract}
% IEEEtran.cls defaults to using nonbold math in the Abstract.
% This preserves the distinction between vectors and scalars. However,
% if the journal you are submitting to favors bold math in the abstract,
% then you can use LaTeX's standard command \boldmath at the very start
% of the abstract to achieve this. Many IEEE journals frown on math
% in the abstract anyway.

% Note that keywords are not normally used for peerreview papers.
%\begin{IEEEkeywords}
%Cooperative diversity, decode and forward, piecewise linear
%\end{IEEEkeywords}



% For peer review papers, you can put extra information on the cover
% page as needed:
% \ifCLASSOPTIONpeerreview
% \begin{center} \bfseries EDICS Category: 3-BBND \end{center}
% \fi
%
% For peerreview papers, this IEEEtran command inserts a page break and
% creates the second title. It will be ignored for other modes.
%\IEEEpeerreviewmaketitle




	\item One urn contains two black balls (labelled B1 and B2) and one white ball. A
	second urn contains one black ball and two white balls (labelled W1 and W2).
	Suppose the following experiment is performed. One of the two urns is chosen
	at random. Next a ball is randomly chosen from the urn. Then a second ball is
	chosen at random from the same urn without replacing the first ball.
	
	\begin{enumerate}
	\item What is the probability that two black balls are chosen?
	
	\item What is the probability that two balls of opposite colour are chosen?
	\end{enumerate}
	\solution
	%\begin{align}
    \label{eq:12.13.6.18.1}
	\because	\pr{A|B} &> \pr{A},\
\frac{\pr{AB}}{\pr{B}} > \pr{A}
\\
    \label{eq:12.13.6.18.2}
	\implies \pr{AB} &> \pr{A}\pr{B}
	\\
	\text{or, } \frac{\pr{AB}}{\pr{A}} &=\pr{B|A} > \pr{A}
\end{align}

\end{enumerate}

	\item A card is selected from a pack of 52 cards.
 \begin{enumerate}[label=(\alph*)] 
                 \item How many points are there in the sample space?
                 \item Calculate the probability that the card is an ace of spades.
                 \item Calculate the probability that the card is (i) an ace and (ii) black card.
 \end{enumerate}
\solution
		%\begin{table}[H]
	\centering
\begin{tabular}{|c|c|c|}
\hline
Random variable &Value &Definition\\ \hline
\multirow{3}{*}{X} &0 &Slips of Rs 1\\
&1 &Slips of Rs 5\\
&2 &Slips of Rs 13\\ \hline
\multirow{2}{*}{Y} &0 &Box A\\
&1 &Box B\\\hline
\end{tabular}
\caption{}
\label{tab:Distribution}
\end{table}
See \tabref{tab:Distribution}.
\begin{align}
p_{Y}\brak{k}= \begin{cases} 
      \frac{1}{3} & {k=0} \\
      \frac{2}{3 }& {k=1} 
   \end{cases}
   \\
p_{Y|X}\brak{0|0} = \frac{19}{25}\, 
p_{Y|X}\brak{0|1} = \frac{6}{25}\,
p_{Y|X}\brak{1|0} = \frac{45}{50}\,
p_{Y|X}\brak{1|2} = \frac{5}{50}
\end{align}
The desired probability is the probability that a slip drawn at random is marked other than Rs 1,
\begin{align}
&=1-p_X\brak{0}\\
&= p_X(1) + p_X(2)
\end{align}
Using Bayes theorem,
\begin{align}
&= p_Y\brak{0} \times \pr{Y=0 | X=1} + p_Y\brak{1} \times \pr{Y=1|X=2}\\
&=\frac{1}{3} \times \frac{6}{25} + \frac{2}{3} \times \frac{5}{50}\\
&=\frac{11}{75}
\end{align}

\newpage

%\tableofcontents

\bigskip

\renewcommand{\thefigure}{\theenumi}
\renewcommand{\thetable}{\theenumi}
%\renewcommand{\theequation}{\theenumi}

%\begin{abstract}
%%\boldmath
%In this letter, an algorithm for evaluating the exact analytical bit error rate  (BER)  for the piecewise linear (PL) combiner for  multiple relays is presented. Previous results were available only for upto three relays. The algorithm is unique in the sense that  the actual mathematical expressions, that are prohibitively large, need not be explicitly obtained. The diversity gain due to multiple relays is shown through plots of the analytical BER, well supported by simulations. 
%
%\end{abstract}
% IEEEtran.cls defaults to using nonbold math in the Abstract.
% This preserves the distinction between vectors and scalars. However,
% if the journal you are submitting to favors bold math in the abstract,
% then you can use LaTeX's standard command \boldmath at the very start
% of the abstract to achieve this. Many IEEE journals frown on math
% in the abstract anyway.

% Note that keywords are not normally used for peerreview papers.
%\begin{IEEEkeywords}
%Cooperative diversity, decode and forward, piecewise linear
%\end{IEEEkeywords}



% For peer review papers, you can put extra information on the cover
% page as needed:
% \ifCLASSOPTIONpeerreview
% \begin{center} \bfseries EDICS Category: 3-BBND \end{center}
% \fi
%
% For peerreview papers, this IEEEtran command inserts a page break and
% creates the second title. It will be ignored for other modes.
%\IEEEpeerreviewmaketitle




\item Four cards are drawn from a well-shuffled deck of 52 cards. What is the probability of obtaining 3 diamonds and one spade.
\\
\solution
		%\begin{enumerate}[label=\thesection.\arabic*,ref=\thesection.\theenumi]
	\item One card is drawn from a well-shuffled deck of 52 cards. Find the probability of getting
\begin{enumerate}
\item A king of red colour 
\item A face card 
\item A red face card
\item The jack of hearts
\item A spade
\item The queen of diamonds

\end{enumerate}
\solution
		%\begin{table}[H]
	\centering
\begin{tabular}{|c|c|c|}
\hline
Random variable &Value &Definition\\ \hline
\multirow{3}{*}{X} &0 &Slips of Rs 1\\
&1 &Slips of Rs 5\\
&2 &Slips of Rs 13\\ \hline
\multirow{2}{*}{Y} &0 &Box A\\
&1 &Box B\\\hline
\end{tabular}
\caption{}
\label{tab:Distribution}
\end{table}
See \tabref{tab:Distribution}.
\begin{align}
p_{Y}\brak{k}= \begin{cases} 
      \frac{1}{3} & {k=0} \\
      \frac{2}{3 }& {k=1} 
   \end{cases}
   \\
p_{Y|X}\brak{0|0} = \frac{19}{25}\, 
p_{Y|X}\brak{0|1} = \frac{6}{25}\,
p_{Y|X}\brak{1|0} = \frac{45}{50}\,
p_{Y|X}\brak{1|2} = \frac{5}{50}
\end{align}
The desired probability is the probability that a slip drawn at random is marked other than Rs 1,
\begin{align}
&=1-p_X\brak{0}\\
&= p_X(1) + p_X(2)
\end{align}
Using Bayes theorem,
\begin{align}
&= p_Y\brak{0} \times \pr{Y=0 | X=1} + p_Y\brak{1} \times \pr{Y=1|X=2}\\
&=\frac{1}{3} \times \frac{6}{25} + \frac{2}{3} \times \frac{5}{50}\\
&=\frac{11}{75}
\end{align}

\newpage

%\tableofcontents

\bigskip

\renewcommand{\thefigure}{\theenumi}
\renewcommand{\thetable}{\theenumi}
%\renewcommand{\theequation}{\theenumi}

%\begin{abstract}
%%\boldmath
%In this letter, an algorithm for evaluating the exact analytical bit error rate  (BER)  for the piecewise linear (PL) combiner for  multiple relays is presented. Previous results were available only for upto three relays. The algorithm is unique in the sense that  the actual mathematical expressions, that are prohibitively large, need not be explicitly obtained. The diversity gain due to multiple relays is shown through plots of the analytical BER, well supported by simulations. 
%
%\end{abstract}
% IEEEtran.cls defaults to using nonbold math in the Abstract.
% This preserves the distinction between vectors and scalars. However,
% if the journal you are submitting to favors bold math in the abstract,
% then you can use LaTeX's standard command \boldmath at the very start
% of the abstract to achieve this. Many IEEE journals frown on math
% in the abstract anyway.

% Note that keywords are not normally used for peerreview papers.
%\begin{IEEEkeywords}
%Cooperative diversity, decode and forward, piecewise linear
%\end{IEEEkeywords}



% For peer review papers, you can put extra information on the cover
% page as needed:
% \ifCLASSOPTIONpeerreview
% \begin{center} \bfseries EDICS Category: 3-BBND \end{center}
% \fi
%
% For peerreview papers, this IEEEtran command inserts a page break and
% creates the second title. It will be ignored for other modes.
%\IEEEpeerreviewmaketitle




	\item Five cards—the ten, jack, queen, king and ace of diamonds, are well-shuffled with their face downwards. One card is then picked up at random.
\begin{enumerate}
\item
What is the probability that the card is the queen? 
\item
If the queen is drawn and put aside, what is the probability that the second card picked up is (a) an ace? (b) a queen?\\
\end{enumerate}
\solution
		%\begin{enumerate}[label=\thesection.\arabic*,ref=\thesection.\theenumi]
	\item One card is drawn from a well-shuffled deck of 52 cards. Find the probability of getting
\begin{enumerate}
\item A king of red colour 
\item A face card 
\item A red face card
\item The jack of hearts
\item A spade
\item The queen of diamonds

\end{enumerate}
\solution
		%\begin{table}[H]
	\centering
\begin{tabular}{|c|c|c|}
\hline
Random variable &Value &Definition\\ \hline
\multirow{3}{*}{X} &0 &Slips of Rs 1\\
&1 &Slips of Rs 5\\
&2 &Slips of Rs 13\\ \hline
\multirow{2}{*}{Y} &0 &Box A\\
&1 &Box B\\\hline
\end{tabular}
\caption{}
\label{tab:Distribution}
\end{table}
See \tabref{tab:Distribution}.
\begin{align}
p_{Y}\brak{k}= \begin{cases} 
      \frac{1}{3} & {k=0} \\
      \frac{2}{3 }& {k=1} 
   \end{cases}
   \\
p_{Y|X}\brak{0|0} = \frac{19}{25}\, 
p_{Y|X}\brak{0|1} = \frac{6}{25}\,
p_{Y|X}\brak{1|0} = \frac{45}{50}\,
p_{Y|X}\brak{1|2} = \frac{5}{50}
\end{align}
The desired probability is the probability that a slip drawn at random is marked other than Rs 1,
\begin{align}
&=1-p_X\brak{0}\\
&= p_X(1) + p_X(2)
\end{align}
Using Bayes theorem,
\begin{align}
&= p_Y\brak{0} \times \pr{Y=0 | X=1} + p_Y\brak{1} \times \pr{Y=1|X=2}\\
&=\frac{1}{3} \times \frac{6}{25} + \frac{2}{3} \times \frac{5}{50}\\
&=\frac{11}{75}
\end{align}

\newpage

%\tableofcontents

\bigskip

\renewcommand{\thefigure}{\theenumi}
\renewcommand{\thetable}{\theenumi}
%\renewcommand{\theequation}{\theenumi}

%\begin{abstract}
%%\boldmath
%In this letter, an algorithm for evaluating the exact analytical bit error rate  (BER)  for the piecewise linear (PL) combiner for  multiple relays is presented. Previous results were available only for upto three relays. The algorithm is unique in the sense that  the actual mathematical expressions, that are prohibitively large, need not be explicitly obtained. The diversity gain due to multiple relays is shown through plots of the analytical BER, well supported by simulations. 
%
%\end{abstract}
% IEEEtran.cls defaults to using nonbold math in the Abstract.
% This preserves the distinction between vectors and scalars. However,
% if the journal you are submitting to favors bold math in the abstract,
% then you can use LaTeX's standard command \boldmath at the very start
% of the abstract to achieve this. Many IEEE journals frown on math
% in the abstract anyway.

% Note that keywords are not normally used for peerreview papers.
%\begin{IEEEkeywords}
%Cooperative diversity, decode and forward, piecewise linear
%\end{IEEEkeywords}



% For peer review papers, you can put extra information on the cover
% page as needed:
% \ifCLASSOPTIONpeerreview
% \begin{center} \bfseries EDICS Category: 3-BBND \end{center}
% \fi
%
% For peerreview papers, this IEEEtran command inserts a page break and
% creates the second title. It will be ignored for other modes.
%\IEEEpeerreviewmaketitle




	\item Five cards—the ten, jack, queen, king and ace of diamonds, are well-shuffled with their face downwards. One card is then picked up at random.
\begin{enumerate}
\item
What is the probability that the card is the queen? 
\item
If the queen is drawn and put aside, what is the probability that the second card picked up is (a) an ace? (b) a queen?\\
\end{enumerate}
\solution
		%\begin{enumerate}[label=\thesection.\arabic*,ref=\thesection.\theenumi]
	\item One card is drawn from a well-shuffled deck of 52 cards. Find the probability of getting
\begin{enumerate}
\item A king of red colour 
\item A face card 
\item A red face card
\item The jack of hearts
\item A spade
\item The queen of diamonds

\end{enumerate}
\solution
		%\input{ncert/10/15/1/14/main.tex}
	\item Five cards—the ten, jack, queen, king and ace of diamonds, are well-shuffled with their face downwards. One card is then picked up at random.
\begin{enumerate}
\item
What is the probability that the card is the queen? 
\item
If the queen is drawn and put aside, what is the probability that the second card picked up is (a) an ace? (b) a queen?\\
\end{enumerate}
\solution
		%\input{ncert/10/15/1/15/defs.tex}
	\item A bag contains $5$ red balls and some blue balls. If the probability of drawing a blue ball is double that if a red ball, determine the number of blue balls in the bag. 
		\\
\solution
		%\input{ncert/10/15/2/3/defs.tex}
	\item A card is selected from a pack of 52 cards.
 \begin{enumerate}[label=(\alph*)] 
                 \item How many points are there in the sample space?
                 \item Calculate the probability that the card is an ace of spades.
                 \item Calculate the probability that the card is (i) an ace and (ii) black card.
 \end{enumerate}
\solution
		%\input{ncert/11/16/3/4/main.tex}
\item Four cards are drawn from a well-shuffled deck of 52 cards. What is the probability of obtaining 3 diamonds and one spade.
\\
\solution
		%\input{ncert/11/16/4/2/defs.tex}
\item In a certain lottery 10,000 tickets are sold and ten equal prizes are awarded. What is the probability of not getting a prize if you buy (a) one ticket (b) two tickets (c) 10 tickets ?	
\\
\solution
		%\input{ncert/11/16/4/4/defs.tex}
		%
\item 
Out of 100 students, two sections of 40 and 60 are formed. If you and your friend are among the 100 students, what is the probability that
\begin{enumerate}
\item you both enter the same section?
\item you both enter the different sections?
\end{enumerate}
\solution
		%\input{ncert/11/16/4/5/defs.tex}
	\item 
The number lock of a suitcase has 4 wheels each labelled with ten digits i.e. from 0 to 9.The lock opens with a sequence of four digits with no repeats.What is the probability of a person getting the right sequence to open the suitcase.
\\
\solution
		%\input{ncert/11/16/4/10/defs.tex}
		%
\item 
Two cards are drawn at random and without replacement from a pack of 52 playing cards. Find the probability that both the cards are black.
\\
\solution
		%\input{ncert/12/13/2/2/defs.tex}
		\item A box of oranges is inspected by examining three randomly selected oranges drawn without replacement. If all the three oranges are good, the box is approved for sale, otherwise, it is rejected. Find the probability that a box containing 15 oranges out of which 12 are good and 3 are bad ones will be approved for sale.
		\label{ncert/12/13/2/3/defs.tex}
		\item Two balls are drawn at random with replacement from a box containing 10 black and 8 red balls. Find the probability that
		\label{ncert/12/13/2/12}
\begin{enumerate}
\item both balls are red.
\item first ball is black and second is red.
\item one of them is black and other is red.
\end{enumerate}

\item In a hostel, 60\% of the students read Hindi newspaper, 40\% read English newspaper and 20\% read both Hindi and English newspapers. A student is selected at random.
		\label{ncert/12/13/2/15}
\begin{enumerate}
\item Find the probability that she reads neither Hindi nor English newspapers.
\item If she reads Hindi newspaper, find the probability that she reads English newspaper.
\item If she reads English newspaper, find the probability that she reads Hindi newspaper.\\
\end{enumerate}
\item The probability of obtaining an even prime number on each die, when a pair of dice is rolled is 
\begin{enumerate}
    \item $0$ 
    
    \item $\frac{1}{3}$ 
    
    \item $\frac{1}{12}$ 
    
    \item $\frac{1}{36}$ 
\end{enumerate}
\solution
		%\input{ncert/12/13/2/17/defs.tex}
	\item A bag contains 4 red and 4 black balls, another bag contains 2 red and 6 black balls. One of the two bags is selected at random and a ball is drawn from the bag which is found to be red. Find the probability that the ball is drawn from the first bag.
\\
\solution
		%\input{ncert/12/13/3/2/main.tex}
  \item
  Cards with numbers 2 to 101 are placed in a box. A card is selected at random.Find the probability that the card has
\begin{enumerate}[label=(\roman*)]
	\item an even number 
	\item a square number
\end{enumerate}
\solution
%\input{exemplar/10/13/3/32/main.tex}
\item
The king, queen and jack of clubs are removed from a deck of 52 playing cards and then well shuffled. Now one card is drawn at random from the remaining cards.  Determine the probability that the card is
\begin{enumerate}[label=(\roman*)]
\item a club
\item 10 of hearts
\end{enumerate}
\solution
%\input{exemplar/10/13/3/29/main.tex}
\item A team of medical students doing their internship have to assist during surgeries
at a city hospital. The probabilities of surgeries rated as very complex, complex,
routine, simple or very simple are respectively, 0.15, 0.20, 0.31, 0.26, .08. Find
the probabilities that a particular surgery will be rated
\begin{enumerate}
	\item complex or very complex;
	\item neither very complex nor very simple;
	\item routine or complex
	\item routine or simple
\end{enumerate}
\solution
%\input{exemplar/11/16/3/8(1)/main.tex}
\item A card is selected from a pack of 52 cards.
\begin{enumerate}[label=(\alph*)]
    \item How many points are there in the sample space?
    \item Calculate the probability that the card is an ace of spades.
    \item Calculate the probability that the card is (i) an ace and (ii) black card.
\end{enumerate}
\solution
%\input{exemplar/11/16/3/4/main2.tex}
\item The probability that a non leap year selected at random will contain 53 sundays.
\\
\solution
%\input{exemplar/10/13/1/19/main.tex}
\item One of the four persons John, Rita, Aslam or Gurpreet will be promoted next
month. Consequently the sample space consists of four elementary outcomes
S = {John promoted, Rita promoted, Aslam promoted, Gurpreet promoted}
You are told that the chances of John’s promotion is same as that of Gurpreet,
Rita’s chances of promotion are twice as likely as Johns. Aslam’s chances are
four times that of John.
\begin{enumerate}
	\item Determine
	\begin{enumerate}
		\item P (John promoted)
		\item P (Rita promoted)
		\item P (Aslam promoted)
		\item P (Gurpreet promoted)
	\end{enumerate}
	\item If A = {John promoted or Gurpreet promoted}, find P (A).
\end{enumerate}
\solution
%\input{exemplar/11/16/3/10/main.tex}
\item A card is drawn from a deck of 52 cards. Find the probability of getting a king or a heart or a red card.\\
\solution
%\input{exemplar/11/16/3/15/main.tex}
\item The probability that a student will pass his examination is 0.73, the probability of
the student getting a compartment is 0.13, and the probability that the student will
either pass or get compartment is 0.96. State True or False.\\
\solution
%\input{exemplar/11/16/3/31/main.tex}
\item A card is selected from a pack of 52 cards\\
\begin{enumerate}[label=(\alph*)]
\item How many points are there in the sample space?
\item Calculate the probability that the cards is an ace of spades.
\item Calculate the probability that the card is (i) an ace (ii)black card.\\
\end{enumerate}
%\input{ncert/11/16/3/4_1/Prob_4.tex}
\item In a non-leap year, the probability of having 53 tuesdays or 53 wednesdays is\\
\solution
%\input{exemplar/11/16/3/18/main.tex}
\item There are 1000 sealed envelopes in a box, 10 of them contain a cash prize of
Rs 100 each, 100 of them contain a cash prize of Rs 50 each and 200 of them
contain a cash prize of Rs 10 each and rest do not contain any cash prize. If they
are well shuffled and an envelope is picked up out, what is the probability that it
contains no cash prize?\\
\solution
%\input{exemplar/10/13/3/34/main.tex}
\item 
A die is thrown and a card is selected at random from a deck of 52 playing cards. The probability of getting an even number on the die and a spade card.\\
\solution
%\input{exemplar/12/13/3/78/main.tex}
\item
If 4-digit numbers greater than 5,000 are randomly formed from the digits 0, 1, 3, 5, and 7, what is the probability of forming a number divisible by 5 when:
\begin{enumerate}
    \item The digits are repeated?
    \item The repetition of digits is not allowed?
\end{enumerate}
\solution
%\input{ncert/11/16/4/9/main.tex}
\item Consider the probability space $\brak{\Omega, \mathcal{G}, P}$ where $\Omega = [0,2]$ and $\mathcal{G} = \cbrak{\phi, \Omega, [0,1], (1,2]}$. Let $X$ and $Y$ be two functions on $\Omega$ defined as
\begin{align*}
    X(\omega) = 
    \begin{cases}
        1 & \text{if }\omega \in [0, 1]\\
        2 & \text{if }\omega \in (1, 2]
    \end{cases}
\end{align*}
and
\begin{align*}
    Y(\omega) = 
    \begin{cases}
        2 & \text{if }\omega \in [0, 1.5]\\
        3 & \text{if }\omega \in (1.5, 2].
    \end{cases}
\end{align*}
Then which one of the following statements is true?
\begin{enumerate}
    \item [(A)] $X$ is a random variable with respect to $\mathcal{G}$, but $Y$ is not a random variable with respect to $\mathcal{G}$.
    \item [(B)] $Y$ is a random variable with respect to $\mathcal{G}$, but $X$ is not a random variable with respect to $\mathcal{G}$.
    \item [(C)] Neither $X$ nor $Y$ is a random variable with respect to $\mathcal{G}$.
    \item [(D)] Both $X$ and $Y$ are random variables with respect to $\mathcal{G}$.
\end{enumerate} \hfill (GATE ST 2023)\\
\solution
%\input{gate/ST/2023/14/main.tex}
	\item  A die is loaded in such a way that each odd number is twice as likely to occur as
each even number. Find $P(G)$, where $G$ is the event that a number greater than
3 occurs on a single roll of the die.
\\
\solution
		%\input{exemplar/11/16/3/5/main.tex}
	\item All the jacks, queens and kings are removed from a deck of 52 playing cards. The remaining cards are well shuffled and then one card is drawn at random. Giving ace a value 1 similar value for other cards, find the probability that the card has a value 
		\begin{enumerate}
			\item 7
			\item greater than 7
			\item less than 7
		\end{enumerate}
		%\input{exemplar/10/13/3/30/main.tex}
  \item A Lot consists of 48 mobile phones of which 42 are good, 3 have only minor defects and 3 have major defects.Varnika will buy a phone if it is good but the trader will only buy a mobile if it has no major defects. One phone is selected at random from the lot. What is the probability that it is
\begin{enumerate}
	\item acceptable to Varnika?
            \item acceptable to the trader?
\end{enumerate}
\solution
	%\input{exemplar/10/13/3/40/main.tex}
 \item A student says that if you throw a die, it will show up 1 or not 1. Therefore, the probability of getting 1 and the probability of getting 'not 1' each is equal to $\frac{1}{2}$. Is this correct? Give reasons.\\
 \solution
        %\input{exemplar/10/13/2/9/main.tex}
   \item Four candidates A, B, C, D have ap-
plied for the assignment to coach a school cricket
team. If A is twice as likely to be selected as B, and
B and C are given about the same chance of being
selected, while C is twice as likely to be selected
as D, what are the probabilities that
\begin{enumerate}
\item C will be selected?
\item A will not be selected?
\end{enumerate}
	%\input{exemplar/11/16/3/9/main.tex}
 \item A bag contain 24 balls of which $x$ balls are red, $2x$ are white and $3x$ are blue. A ball is selected at random, What is the probability that it is
\begin{enumerate}[label=\alph*)]
\item not red ?
\item white ?
\end{enumerate}
%\input{exemplar/10/13/3/41/main.tex}
If the letters of the word ASSASSINATION are arranged at random. Find the Probability that
\begin{enumerate}[label=(\alph*)]
\item Four $S's$ come consecutively in the word
\item Two  $I's$ and two $N's$ come together
\item All $A's$ are not coming together
\item No two $A's$ are coming together
\end{enumerate}
%\input{exemplar/11/16/3/14/main.tex}
	\item One urn contains two black balls (labelled B1 and B2) and one white ball. A
	second urn contains one black ball and two white balls (labelled W1 and W2).
	Suppose the following experiment is performed. One of the two urns is chosen
	at random. Next a ball is randomly chosen from the urn. Then a second ball is
	chosen at random from the same urn without replacing the first ball.
	
	\begin{enumerate}
	\item What is the probability that two black balls are chosen?
	
	\item What is the probability that two balls of opposite colour are chosen?
	\end{enumerate}
	\solution
	%\input{exemplar/11/16/3/12/main1.tex}
\end{enumerate}

	\item A bag contains $5$ red balls and some blue balls. If the probability of drawing a blue ball is double that if a red ball, determine the number of blue balls in the bag. 
		\\
\solution
		%\begin{enumerate}[label=\thesection.\arabic*,ref=\thesection.\theenumi]
	\item One card is drawn from a well-shuffled deck of 52 cards. Find the probability of getting
\begin{enumerate}
\item A king of red colour 
\item A face card 
\item A red face card
\item The jack of hearts
\item A spade
\item The queen of diamonds

\end{enumerate}
\solution
		%\input{ncert/10/15/1/14/main.tex}
	\item Five cards—the ten, jack, queen, king and ace of diamonds, are well-shuffled with their face downwards. One card is then picked up at random.
\begin{enumerate}
\item
What is the probability that the card is the queen? 
\item
If the queen is drawn and put aside, what is the probability that the second card picked up is (a) an ace? (b) a queen?\\
\end{enumerate}
\solution
		%\input{ncert/10/15/1/15/defs.tex}
	\item A bag contains $5$ red balls and some blue balls. If the probability of drawing a blue ball is double that if a red ball, determine the number of blue balls in the bag. 
		\\
\solution
		%\input{ncert/10/15/2/3/defs.tex}
	\item A card is selected from a pack of 52 cards.
 \begin{enumerate}[label=(\alph*)] 
                 \item How many points are there in the sample space?
                 \item Calculate the probability that the card is an ace of spades.
                 \item Calculate the probability that the card is (i) an ace and (ii) black card.
 \end{enumerate}
\solution
		%\input{ncert/11/16/3/4/main.tex}
\item Four cards are drawn from a well-shuffled deck of 52 cards. What is the probability of obtaining 3 diamonds and one spade.
\\
\solution
		%\input{ncert/11/16/4/2/defs.tex}
\item In a certain lottery 10,000 tickets are sold and ten equal prizes are awarded. What is the probability of not getting a prize if you buy (a) one ticket (b) two tickets (c) 10 tickets ?	
\\
\solution
		%\input{ncert/11/16/4/4/defs.tex}
		%
\item 
Out of 100 students, two sections of 40 and 60 are formed. If you and your friend are among the 100 students, what is the probability that
\begin{enumerate}
\item you both enter the same section?
\item you both enter the different sections?
\end{enumerate}
\solution
		%\input{ncert/11/16/4/5/defs.tex}
	\item 
The number lock of a suitcase has 4 wheels each labelled with ten digits i.e. from 0 to 9.The lock opens with a sequence of four digits with no repeats.What is the probability of a person getting the right sequence to open the suitcase.
\\
\solution
		%\input{ncert/11/16/4/10/defs.tex}
		%
\item 
Two cards are drawn at random and without replacement from a pack of 52 playing cards. Find the probability that both the cards are black.
\\
\solution
		%\input{ncert/12/13/2/2/defs.tex}
		\item A box of oranges is inspected by examining three randomly selected oranges drawn without replacement. If all the three oranges are good, the box is approved for sale, otherwise, it is rejected. Find the probability that a box containing 15 oranges out of which 12 are good and 3 are bad ones will be approved for sale.
		\label{ncert/12/13/2/3/defs.tex}
		\item Two balls are drawn at random with replacement from a box containing 10 black and 8 red balls. Find the probability that
		\label{ncert/12/13/2/12}
\begin{enumerate}
\item both balls are red.
\item first ball is black and second is red.
\item one of them is black and other is red.
\end{enumerate}

\item In a hostel, 60\% of the students read Hindi newspaper, 40\% read English newspaper and 20\% read both Hindi and English newspapers. A student is selected at random.
		\label{ncert/12/13/2/15}
\begin{enumerate}
\item Find the probability that she reads neither Hindi nor English newspapers.
\item If she reads Hindi newspaper, find the probability that she reads English newspaper.
\item If she reads English newspaper, find the probability that she reads Hindi newspaper.\\
\end{enumerate}
\item The probability of obtaining an even prime number on each die, when a pair of dice is rolled is 
\begin{enumerate}
    \item $0$ 
    
    \item $\frac{1}{3}$ 
    
    \item $\frac{1}{12}$ 
    
    \item $\frac{1}{36}$ 
\end{enumerate}
\solution
		%\input{ncert/12/13/2/17/defs.tex}
	\item A bag contains 4 red and 4 black balls, another bag contains 2 red and 6 black balls. One of the two bags is selected at random and a ball is drawn from the bag which is found to be red. Find the probability that the ball is drawn from the first bag.
\\
\solution
		%\input{ncert/12/13/3/2/main.tex}
  \item
  Cards with numbers 2 to 101 are placed in a box. A card is selected at random.Find the probability that the card has
\begin{enumerate}[label=(\roman*)]
	\item an even number 
	\item a square number
\end{enumerate}
\solution
%\input{exemplar/10/13/3/32/main.tex}
\item
The king, queen and jack of clubs are removed from a deck of 52 playing cards and then well shuffled. Now one card is drawn at random from the remaining cards.  Determine the probability that the card is
\begin{enumerate}[label=(\roman*)]
\item a club
\item 10 of hearts
\end{enumerate}
\solution
%\input{exemplar/10/13/3/29/main.tex}
\item A team of medical students doing their internship have to assist during surgeries
at a city hospital. The probabilities of surgeries rated as very complex, complex,
routine, simple or very simple are respectively, 0.15, 0.20, 0.31, 0.26, .08. Find
the probabilities that a particular surgery will be rated
\begin{enumerate}
	\item complex or very complex;
	\item neither very complex nor very simple;
	\item routine or complex
	\item routine or simple
\end{enumerate}
\solution
%\input{exemplar/11/16/3/8(1)/main.tex}
\item A card is selected from a pack of 52 cards.
\begin{enumerate}[label=(\alph*)]
    \item How many points are there in the sample space?
    \item Calculate the probability that the card is an ace of spades.
    \item Calculate the probability that the card is (i) an ace and (ii) black card.
\end{enumerate}
\solution
%\input{exemplar/11/16/3/4/main2.tex}
\item The probability that a non leap year selected at random will contain 53 sundays.
\\
\solution
%\input{exemplar/10/13/1/19/main.tex}
\item One of the four persons John, Rita, Aslam or Gurpreet will be promoted next
month. Consequently the sample space consists of four elementary outcomes
S = {John promoted, Rita promoted, Aslam promoted, Gurpreet promoted}
You are told that the chances of John’s promotion is same as that of Gurpreet,
Rita’s chances of promotion are twice as likely as Johns. Aslam’s chances are
four times that of John.
\begin{enumerate}
	\item Determine
	\begin{enumerate}
		\item P (John promoted)
		\item P (Rita promoted)
		\item P (Aslam promoted)
		\item P (Gurpreet promoted)
	\end{enumerate}
	\item If A = {John promoted or Gurpreet promoted}, find P (A).
\end{enumerate}
\solution
%\input{exemplar/11/16/3/10/main.tex}
\item A card is drawn from a deck of 52 cards. Find the probability of getting a king or a heart or a red card.\\
\solution
%\input{exemplar/11/16/3/15/main.tex}
\item The probability that a student will pass his examination is 0.73, the probability of
the student getting a compartment is 0.13, and the probability that the student will
either pass or get compartment is 0.96. State True or False.\\
\solution
%\input{exemplar/11/16/3/31/main.tex}
\item A card is selected from a pack of 52 cards\\
\begin{enumerate}[label=(\alph*)]
\item How many points are there in the sample space?
\item Calculate the probability that the cards is an ace of spades.
\item Calculate the probability that the card is (i) an ace (ii)black card.\\
\end{enumerate}
%\input{ncert/11/16/3/4_1/Prob_4.tex}
\item In a non-leap year, the probability of having 53 tuesdays or 53 wednesdays is\\
\solution
%\input{exemplar/11/16/3/18/main.tex}
\item There are 1000 sealed envelopes in a box, 10 of them contain a cash prize of
Rs 100 each, 100 of them contain a cash prize of Rs 50 each and 200 of them
contain a cash prize of Rs 10 each and rest do not contain any cash prize. If they
are well shuffled and an envelope is picked up out, what is the probability that it
contains no cash prize?\\
\solution
%\input{exemplar/10/13/3/34/main.tex}
\item 
A die is thrown and a card is selected at random from a deck of 52 playing cards. The probability of getting an even number on the die and a spade card.\\
\solution
%\input{exemplar/12/13/3/78/main.tex}
\item
If 4-digit numbers greater than 5,000 are randomly formed from the digits 0, 1, 3, 5, and 7, what is the probability of forming a number divisible by 5 when:
\begin{enumerate}
    \item The digits are repeated?
    \item The repetition of digits is not allowed?
\end{enumerate}
\solution
%\input{ncert/11/16/4/9/main.tex}
\item Consider the probability space $\brak{\Omega, \mathcal{G}, P}$ where $\Omega = [0,2]$ and $\mathcal{G} = \cbrak{\phi, \Omega, [0,1], (1,2]}$. Let $X$ and $Y$ be two functions on $\Omega$ defined as
\begin{align*}
    X(\omega) = 
    \begin{cases}
        1 & \text{if }\omega \in [0, 1]\\
        2 & \text{if }\omega \in (1, 2]
    \end{cases}
\end{align*}
and
\begin{align*}
    Y(\omega) = 
    \begin{cases}
        2 & \text{if }\omega \in [0, 1.5]\\
        3 & \text{if }\omega \in (1.5, 2].
    \end{cases}
\end{align*}
Then which one of the following statements is true?
\begin{enumerate}
    \item [(A)] $X$ is a random variable with respect to $\mathcal{G}$, but $Y$ is not a random variable with respect to $\mathcal{G}$.
    \item [(B)] $Y$ is a random variable with respect to $\mathcal{G}$, but $X$ is not a random variable with respect to $\mathcal{G}$.
    \item [(C)] Neither $X$ nor $Y$ is a random variable with respect to $\mathcal{G}$.
    \item [(D)] Both $X$ and $Y$ are random variables with respect to $\mathcal{G}$.
\end{enumerate} \hfill (GATE ST 2023)\\
\solution
%\input{gate/ST/2023/14/main.tex}
	\item  A die is loaded in such a way that each odd number is twice as likely to occur as
each even number. Find $P(G)$, where $G$ is the event that a number greater than
3 occurs on a single roll of the die.
\\
\solution
		%\input{exemplar/11/16/3/5/main.tex}
	\item All the jacks, queens and kings are removed from a deck of 52 playing cards. The remaining cards are well shuffled and then one card is drawn at random. Giving ace a value 1 similar value for other cards, find the probability that the card has a value 
		\begin{enumerate}
			\item 7
			\item greater than 7
			\item less than 7
		\end{enumerate}
		%\input{exemplar/10/13/3/30/main.tex}
  \item A Lot consists of 48 mobile phones of which 42 are good, 3 have only minor defects and 3 have major defects.Varnika will buy a phone if it is good but the trader will only buy a mobile if it has no major defects. One phone is selected at random from the lot. What is the probability that it is
\begin{enumerate}
	\item acceptable to Varnika?
            \item acceptable to the trader?
\end{enumerate}
\solution
	%\input{exemplar/10/13/3/40/main.tex}
 \item A student says that if you throw a die, it will show up 1 or not 1. Therefore, the probability of getting 1 and the probability of getting 'not 1' each is equal to $\frac{1}{2}$. Is this correct? Give reasons.\\
 \solution
        %\input{exemplar/10/13/2/9/main.tex}
   \item Four candidates A, B, C, D have ap-
plied for the assignment to coach a school cricket
team. If A is twice as likely to be selected as B, and
B and C are given about the same chance of being
selected, while C is twice as likely to be selected
as D, what are the probabilities that
\begin{enumerate}
\item C will be selected?
\item A will not be selected?
\end{enumerate}
	%\input{exemplar/11/16/3/9/main.tex}
 \item A bag contain 24 balls of which $x$ balls are red, $2x$ are white and $3x$ are blue. A ball is selected at random, What is the probability that it is
\begin{enumerate}[label=\alph*)]
\item not red ?
\item white ?
\end{enumerate}
%\input{exemplar/10/13/3/41/main.tex}
If the letters of the word ASSASSINATION are arranged at random. Find the Probability that
\begin{enumerate}[label=(\alph*)]
\item Four $S's$ come consecutively in the word
\item Two  $I's$ and two $N's$ come together
\item All $A's$ are not coming together
\item No two $A's$ are coming together
\end{enumerate}
%\input{exemplar/11/16/3/14/main.tex}
	\item One urn contains two black balls (labelled B1 and B2) and one white ball. A
	second urn contains one black ball and two white balls (labelled W1 and W2).
	Suppose the following experiment is performed. One of the two urns is chosen
	at random. Next a ball is randomly chosen from the urn. Then a second ball is
	chosen at random from the same urn without replacing the first ball.
	
	\begin{enumerate}
	\item What is the probability that two black balls are chosen?
	
	\item What is the probability that two balls of opposite colour are chosen?
	\end{enumerate}
	\solution
	%\input{exemplar/11/16/3/12/main1.tex}
\end{enumerate}

	\item A card is selected from a pack of 52 cards.
 \begin{enumerate}[label=(\alph*)] 
                 \item How many points are there in the sample space?
                 \item Calculate the probability that the card is an ace of spades.
                 \item Calculate the probability that the card is (i) an ace and (ii) black card.
 \end{enumerate}
\solution
		%\begin{table}[H]
	\centering
\begin{tabular}{|c|c|c|}
\hline
Random variable &Value &Definition\\ \hline
\multirow{3}{*}{X} &0 &Slips of Rs 1\\
&1 &Slips of Rs 5\\
&2 &Slips of Rs 13\\ \hline
\multirow{2}{*}{Y} &0 &Box A\\
&1 &Box B\\\hline
\end{tabular}
\caption{}
\label{tab:Distribution}
\end{table}
See \tabref{tab:Distribution}.
\begin{align}
p_{Y}\brak{k}= \begin{cases} 
      \frac{1}{3} & {k=0} \\
      \frac{2}{3 }& {k=1} 
   \end{cases}
   \\
p_{Y|X}\brak{0|0} = \frac{19}{25}\, 
p_{Y|X}\brak{0|1} = \frac{6}{25}\,
p_{Y|X}\brak{1|0} = \frac{45}{50}\,
p_{Y|X}\brak{1|2} = \frac{5}{50}
\end{align}
The desired probability is the probability that a slip drawn at random is marked other than Rs 1,
\begin{align}
&=1-p_X\brak{0}\\
&= p_X(1) + p_X(2)
\end{align}
Using Bayes theorem,
\begin{align}
&= p_Y\brak{0} \times \pr{Y=0 | X=1} + p_Y\brak{1} \times \pr{Y=1|X=2}\\
&=\frac{1}{3} \times \frac{6}{25} + \frac{2}{3} \times \frac{5}{50}\\
&=\frac{11}{75}
\end{align}

\newpage

%\tableofcontents

\bigskip

\renewcommand{\thefigure}{\theenumi}
\renewcommand{\thetable}{\theenumi}
%\renewcommand{\theequation}{\theenumi}

%\begin{abstract}
%%\boldmath
%In this letter, an algorithm for evaluating the exact analytical bit error rate  (BER)  for the piecewise linear (PL) combiner for  multiple relays is presented. Previous results were available only for upto three relays. The algorithm is unique in the sense that  the actual mathematical expressions, that are prohibitively large, need not be explicitly obtained. The diversity gain due to multiple relays is shown through plots of the analytical BER, well supported by simulations. 
%
%\end{abstract}
% IEEEtran.cls defaults to using nonbold math in the Abstract.
% This preserves the distinction between vectors and scalars. However,
% if the journal you are submitting to favors bold math in the abstract,
% then you can use LaTeX's standard command \boldmath at the very start
% of the abstract to achieve this. Many IEEE journals frown on math
% in the abstract anyway.

% Note that keywords are not normally used for peerreview papers.
%\begin{IEEEkeywords}
%Cooperative diversity, decode and forward, piecewise linear
%\end{IEEEkeywords}



% For peer review papers, you can put extra information on the cover
% page as needed:
% \ifCLASSOPTIONpeerreview
% \begin{center} \bfseries EDICS Category: 3-BBND \end{center}
% \fi
%
% For peerreview papers, this IEEEtran command inserts a page break and
% creates the second title. It will be ignored for other modes.
%\IEEEpeerreviewmaketitle




\item Four cards are drawn from a well-shuffled deck of 52 cards. What is the probability of obtaining 3 diamonds and one spade.
\\
\solution
		%\begin{enumerate}[label=\thesection.\arabic*,ref=\thesection.\theenumi]
	\item One card is drawn from a well-shuffled deck of 52 cards. Find the probability of getting
\begin{enumerate}
\item A king of red colour 
\item A face card 
\item A red face card
\item The jack of hearts
\item A spade
\item The queen of diamonds

\end{enumerate}
\solution
		%\input{ncert/10/15/1/14/main.tex}
	\item Five cards—the ten, jack, queen, king and ace of diamonds, are well-shuffled with their face downwards. One card is then picked up at random.
\begin{enumerate}
\item
What is the probability that the card is the queen? 
\item
If the queen is drawn and put aside, what is the probability that the second card picked up is (a) an ace? (b) a queen?\\
\end{enumerate}
\solution
		%\input{ncert/10/15/1/15/defs.tex}
	\item A bag contains $5$ red balls and some blue balls. If the probability of drawing a blue ball is double that if a red ball, determine the number of blue balls in the bag. 
		\\
\solution
		%\input{ncert/10/15/2/3/defs.tex}
	\item A card is selected from a pack of 52 cards.
 \begin{enumerate}[label=(\alph*)] 
                 \item How many points are there in the sample space?
                 \item Calculate the probability that the card is an ace of spades.
                 \item Calculate the probability that the card is (i) an ace and (ii) black card.
 \end{enumerate}
\solution
		%\input{ncert/11/16/3/4/main.tex}
\item Four cards are drawn from a well-shuffled deck of 52 cards. What is the probability of obtaining 3 diamonds and one spade.
\\
\solution
		%\input{ncert/11/16/4/2/defs.tex}
\item In a certain lottery 10,000 tickets are sold and ten equal prizes are awarded. What is the probability of not getting a prize if you buy (a) one ticket (b) two tickets (c) 10 tickets ?	
\\
\solution
		%\input{ncert/11/16/4/4/defs.tex}
		%
\item 
Out of 100 students, two sections of 40 and 60 are formed. If you and your friend are among the 100 students, what is the probability that
\begin{enumerate}
\item you both enter the same section?
\item you both enter the different sections?
\end{enumerate}
\solution
		%\input{ncert/11/16/4/5/defs.tex}
	\item 
The number lock of a suitcase has 4 wheels each labelled with ten digits i.e. from 0 to 9.The lock opens with a sequence of four digits with no repeats.What is the probability of a person getting the right sequence to open the suitcase.
\\
\solution
		%\input{ncert/11/16/4/10/defs.tex}
		%
\item 
Two cards are drawn at random and without replacement from a pack of 52 playing cards. Find the probability that both the cards are black.
\\
\solution
		%\input{ncert/12/13/2/2/defs.tex}
		\item A box of oranges is inspected by examining three randomly selected oranges drawn without replacement. If all the three oranges are good, the box is approved for sale, otherwise, it is rejected. Find the probability that a box containing 15 oranges out of which 12 are good and 3 are bad ones will be approved for sale.
		\label{ncert/12/13/2/3/defs.tex}
		\item Two balls are drawn at random with replacement from a box containing 10 black and 8 red balls. Find the probability that
		\label{ncert/12/13/2/12}
\begin{enumerate}
\item both balls are red.
\item first ball is black and second is red.
\item one of them is black and other is red.
\end{enumerate}

\item In a hostel, 60\% of the students read Hindi newspaper, 40\% read English newspaper and 20\% read both Hindi and English newspapers. A student is selected at random.
		\label{ncert/12/13/2/15}
\begin{enumerate}
\item Find the probability that she reads neither Hindi nor English newspapers.
\item If she reads Hindi newspaper, find the probability that she reads English newspaper.
\item If she reads English newspaper, find the probability that she reads Hindi newspaper.\\
\end{enumerate}
\item The probability of obtaining an even prime number on each die, when a pair of dice is rolled is 
\begin{enumerate}
    \item $0$ 
    
    \item $\frac{1}{3}$ 
    
    \item $\frac{1}{12}$ 
    
    \item $\frac{1}{36}$ 
\end{enumerate}
\solution
		%\input{ncert/12/13/2/17/defs.tex}
	\item A bag contains 4 red and 4 black balls, another bag contains 2 red and 6 black balls. One of the two bags is selected at random and a ball is drawn from the bag which is found to be red. Find the probability that the ball is drawn from the first bag.
\\
\solution
		%\input{ncert/12/13/3/2/main.tex}
  \item
  Cards with numbers 2 to 101 are placed in a box. A card is selected at random.Find the probability that the card has
\begin{enumerate}[label=(\roman*)]
	\item an even number 
	\item a square number
\end{enumerate}
\solution
%\input{exemplar/10/13/3/32/main.tex}
\item
The king, queen and jack of clubs are removed from a deck of 52 playing cards and then well shuffled. Now one card is drawn at random from the remaining cards.  Determine the probability that the card is
\begin{enumerate}[label=(\roman*)]
\item a club
\item 10 of hearts
\end{enumerate}
\solution
%\input{exemplar/10/13/3/29/main.tex}
\item A team of medical students doing their internship have to assist during surgeries
at a city hospital. The probabilities of surgeries rated as very complex, complex,
routine, simple or very simple are respectively, 0.15, 0.20, 0.31, 0.26, .08. Find
the probabilities that a particular surgery will be rated
\begin{enumerate}
	\item complex or very complex;
	\item neither very complex nor very simple;
	\item routine or complex
	\item routine or simple
\end{enumerate}
\solution
%\input{exemplar/11/16/3/8(1)/main.tex}
\item A card is selected from a pack of 52 cards.
\begin{enumerate}[label=(\alph*)]
    \item How many points are there in the sample space?
    \item Calculate the probability that the card is an ace of spades.
    \item Calculate the probability that the card is (i) an ace and (ii) black card.
\end{enumerate}
\solution
%\input{exemplar/11/16/3/4/main2.tex}
\item The probability that a non leap year selected at random will contain 53 sundays.
\\
\solution
%\input{exemplar/10/13/1/19/main.tex}
\item One of the four persons John, Rita, Aslam or Gurpreet will be promoted next
month. Consequently the sample space consists of four elementary outcomes
S = {John promoted, Rita promoted, Aslam promoted, Gurpreet promoted}
You are told that the chances of John’s promotion is same as that of Gurpreet,
Rita’s chances of promotion are twice as likely as Johns. Aslam’s chances are
four times that of John.
\begin{enumerate}
	\item Determine
	\begin{enumerate}
		\item P (John promoted)
		\item P (Rita promoted)
		\item P (Aslam promoted)
		\item P (Gurpreet promoted)
	\end{enumerate}
	\item If A = {John promoted or Gurpreet promoted}, find P (A).
\end{enumerate}
\solution
%\input{exemplar/11/16/3/10/main.tex}
\item A card is drawn from a deck of 52 cards. Find the probability of getting a king or a heart or a red card.\\
\solution
%\input{exemplar/11/16/3/15/main.tex}
\item The probability that a student will pass his examination is 0.73, the probability of
the student getting a compartment is 0.13, and the probability that the student will
either pass or get compartment is 0.96. State True or False.\\
\solution
%\input{exemplar/11/16/3/31/main.tex}
\item A card is selected from a pack of 52 cards\\
\begin{enumerate}[label=(\alph*)]
\item How many points are there in the sample space?
\item Calculate the probability that the cards is an ace of spades.
\item Calculate the probability that the card is (i) an ace (ii)black card.\\
\end{enumerate}
%\input{ncert/11/16/3/4_1/Prob_4.tex}
\item In a non-leap year, the probability of having 53 tuesdays or 53 wednesdays is\\
\solution
%\input{exemplar/11/16/3/18/main.tex}
\item There are 1000 sealed envelopes in a box, 10 of them contain a cash prize of
Rs 100 each, 100 of them contain a cash prize of Rs 50 each and 200 of them
contain a cash prize of Rs 10 each and rest do not contain any cash prize. If they
are well shuffled and an envelope is picked up out, what is the probability that it
contains no cash prize?\\
\solution
%\input{exemplar/10/13/3/34/main.tex}
\item 
A die is thrown and a card is selected at random from a deck of 52 playing cards. The probability of getting an even number on the die and a spade card.\\
\solution
%\input{exemplar/12/13/3/78/main.tex}
\item
If 4-digit numbers greater than 5,000 are randomly formed from the digits 0, 1, 3, 5, and 7, what is the probability of forming a number divisible by 5 when:
\begin{enumerate}
    \item The digits are repeated?
    \item The repetition of digits is not allowed?
\end{enumerate}
\solution
%\input{ncert/11/16/4/9/main.tex}
\item Consider the probability space $\brak{\Omega, \mathcal{G}, P}$ where $\Omega = [0,2]$ and $\mathcal{G} = \cbrak{\phi, \Omega, [0,1], (1,2]}$. Let $X$ and $Y$ be two functions on $\Omega$ defined as
\begin{align*}
    X(\omega) = 
    \begin{cases}
        1 & \text{if }\omega \in [0, 1]\\
        2 & \text{if }\omega \in (1, 2]
    \end{cases}
\end{align*}
and
\begin{align*}
    Y(\omega) = 
    \begin{cases}
        2 & \text{if }\omega \in [0, 1.5]\\
        3 & \text{if }\omega \in (1.5, 2].
    \end{cases}
\end{align*}
Then which one of the following statements is true?
\begin{enumerate}
    \item [(A)] $X$ is a random variable with respect to $\mathcal{G}$, but $Y$ is not a random variable with respect to $\mathcal{G}$.
    \item [(B)] $Y$ is a random variable with respect to $\mathcal{G}$, but $X$ is not a random variable with respect to $\mathcal{G}$.
    \item [(C)] Neither $X$ nor $Y$ is a random variable with respect to $\mathcal{G}$.
    \item [(D)] Both $X$ and $Y$ are random variables with respect to $\mathcal{G}$.
\end{enumerate} \hfill (GATE ST 2023)\\
\solution
%\input{gate/ST/2023/14/main.tex}
	\item  A die is loaded in such a way that each odd number is twice as likely to occur as
each even number. Find $P(G)$, where $G$ is the event that a number greater than
3 occurs on a single roll of the die.
\\
\solution
		%\input{exemplar/11/16/3/5/main.tex}
	\item All the jacks, queens and kings are removed from a deck of 52 playing cards. The remaining cards are well shuffled and then one card is drawn at random. Giving ace a value 1 similar value for other cards, find the probability that the card has a value 
		\begin{enumerate}
			\item 7
			\item greater than 7
			\item less than 7
		\end{enumerate}
		%\input{exemplar/10/13/3/30/main.tex}
  \item A Lot consists of 48 mobile phones of which 42 are good, 3 have only minor defects and 3 have major defects.Varnika will buy a phone if it is good but the trader will only buy a mobile if it has no major defects. One phone is selected at random from the lot. What is the probability that it is
\begin{enumerate}
	\item acceptable to Varnika?
            \item acceptable to the trader?
\end{enumerate}
\solution
	%\input{exemplar/10/13/3/40/main.tex}
 \item A student says that if you throw a die, it will show up 1 or not 1. Therefore, the probability of getting 1 and the probability of getting 'not 1' each is equal to $\frac{1}{2}$. Is this correct? Give reasons.\\
 \solution
        %\input{exemplar/10/13/2/9/main.tex}
   \item Four candidates A, B, C, D have ap-
plied for the assignment to coach a school cricket
team. If A is twice as likely to be selected as B, and
B and C are given about the same chance of being
selected, while C is twice as likely to be selected
as D, what are the probabilities that
\begin{enumerate}
\item C will be selected?
\item A will not be selected?
\end{enumerate}
	%\input{exemplar/11/16/3/9/main.tex}
 \item A bag contain 24 balls of which $x$ balls are red, $2x$ are white and $3x$ are blue. A ball is selected at random, What is the probability that it is
\begin{enumerate}[label=\alph*)]
\item not red ?
\item white ?
\end{enumerate}
%\input{exemplar/10/13/3/41/main.tex}
If the letters of the word ASSASSINATION are arranged at random. Find the Probability that
\begin{enumerate}[label=(\alph*)]
\item Four $S's$ come consecutively in the word
\item Two  $I's$ and two $N's$ come together
\item All $A's$ are not coming together
\item No two $A's$ are coming together
\end{enumerate}
%\input{exemplar/11/16/3/14/main.tex}
	\item One urn contains two black balls (labelled B1 and B2) and one white ball. A
	second urn contains one black ball and two white balls (labelled W1 and W2).
	Suppose the following experiment is performed. One of the two urns is chosen
	at random. Next a ball is randomly chosen from the urn. Then a second ball is
	chosen at random from the same urn without replacing the first ball.
	
	\begin{enumerate}
	\item What is the probability that two black balls are chosen?
	
	\item What is the probability that two balls of opposite colour are chosen?
	\end{enumerate}
	\solution
	%\input{exemplar/11/16/3/12/main1.tex}
\end{enumerate}

\item In a certain lottery 10,000 tickets are sold and ten equal prizes are awarded. What is the probability of not getting a prize if you buy (a) one ticket (b) two tickets (c) 10 tickets ?	
\\
\solution
		%\begin{enumerate}[label=\thesection.\arabic*,ref=\thesection.\theenumi]
	\item One card is drawn from a well-shuffled deck of 52 cards. Find the probability of getting
\begin{enumerate}
\item A king of red colour 
\item A face card 
\item A red face card
\item The jack of hearts
\item A spade
\item The queen of diamonds

\end{enumerate}
\solution
		%\input{ncert/10/15/1/14/main.tex}
	\item Five cards—the ten, jack, queen, king and ace of diamonds, are well-shuffled with their face downwards. One card is then picked up at random.
\begin{enumerate}
\item
What is the probability that the card is the queen? 
\item
If the queen is drawn and put aside, what is the probability that the second card picked up is (a) an ace? (b) a queen?\\
\end{enumerate}
\solution
		%\input{ncert/10/15/1/15/defs.tex}
	\item A bag contains $5$ red balls and some blue balls. If the probability of drawing a blue ball is double that if a red ball, determine the number of blue balls in the bag. 
		\\
\solution
		%\input{ncert/10/15/2/3/defs.tex}
	\item A card is selected from a pack of 52 cards.
 \begin{enumerate}[label=(\alph*)] 
                 \item How many points are there in the sample space?
                 \item Calculate the probability that the card is an ace of spades.
                 \item Calculate the probability that the card is (i) an ace and (ii) black card.
 \end{enumerate}
\solution
		%\input{ncert/11/16/3/4/main.tex}
\item Four cards are drawn from a well-shuffled deck of 52 cards. What is the probability of obtaining 3 diamonds and one spade.
\\
\solution
		%\input{ncert/11/16/4/2/defs.tex}
\item In a certain lottery 10,000 tickets are sold and ten equal prizes are awarded. What is the probability of not getting a prize if you buy (a) one ticket (b) two tickets (c) 10 tickets ?	
\\
\solution
		%\input{ncert/11/16/4/4/defs.tex}
		%
\item 
Out of 100 students, two sections of 40 and 60 are formed. If you and your friend are among the 100 students, what is the probability that
\begin{enumerate}
\item you both enter the same section?
\item you both enter the different sections?
\end{enumerate}
\solution
		%\input{ncert/11/16/4/5/defs.tex}
	\item 
The number lock of a suitcase has 4 wheels each labelled with ten digits i.e. from 0 to 9.The lock opens with a sequence of four digits with no repeats.What is the probability of a person getting the right sequence to open the suitcase.
\\
\solution
		%\input{ncert/11/16/4/10/defs.tex}
		%
\item 
Two cards are drawn at random and without replacement from a pack of 52 playing cards. Find the probability that both the cards are black.
\\
\solution
		%\input{ncert/12/13/2/2/defs.tex}
		\item A box of oranges is inspected by examining three randomly selected oranges drawn without replacement. If all the three oranges are good, the box is approved for sale, otherwise, it is rejected. Find the probability that a box containing 15 oranges out of which 12 are good and 3 are bad ones will be approved for sale.
		\label{ncert/12/13/2/3/defs.tex}
		\item Two balls are drawn at random with replacement from a box containing 10 black and 8 red balls. Find the probability that
		\label{ncert/12/13/2/12}
\begin{enumerate}
\item both balls are red.
\item first ball is black and second is red.
\item one of them is black and other is red.
\end{enumerate}

\item In a hostel, 60\% of the students read Hindi newspaper, 40\% read English newspaper and 20\% read both Hindi and English newspapers. A student is selected at random.
		\label{ncert/12/13/2/15}
\begin{enumerate}
\item Find the probability that she reads neither Hindi nor English newspapers.
\item If she reads Hindi newspaper, find the probability that she reads English newspaper.
\item If she reads English newspaper, find the probability that she reads Hindi newspaper.\\
\end{enumerate}
\item The probability of obtaining an even prime number on each die, when a pair of dice is rolled is 
\begin{enumerate}
    \item $0$ 
    
    \item $\frac{1}{3}$ 
    
    \item $\frac{1}{12}$ 
    
    \item $\frac{1}{36}$ 
\end{enumerate}
\solution
		%\input{ncert/12/13/2/17/defs.tex}
	\item A bag contains 4 red and 4 black balls, another bag contains 2 red and 6 black balls. One of the two bags is selected at random and a ball is drawn from the bag which is found to be red. Find the probability that the ball is drawn from the first bag.
\\
\solution
		%\input{ncert/12/13/3/2/main.tex}
  \item
  Cards with numbers 2 to 101 are placed in a box. A card is selected at random.Find the probability that the card has
\begin{enumerate}[label=(\roman*)]
	\item an even number 
	\item a square number
\end{enumerate}
\solution
%\input{exemplar/10/13/3/32/main.tex}
\item
The king, queen and jack of clubs are removed from a deck of 52 playing cards and then well shuffled. Now one card is drawn at random from the remaining cards.  Determine the probability that the card is
\begin{enumerate}[label=(\roman*)]
\item a club
\item 10 of hearts
\end{enumerate}
\solution
%\input{exemplar/10/13/3/29/main.tex}
\item A team of medical students doing their internship have to assist during surgeries
at a city hospital. The probabilities of surgeries rated as very complex, complex,
routine, simple or very simple are respectively, 0.15, 0.20, 0.31, 0.26, .08. Find
the probabilities that a particular surgery will be rated
\begin{enumerate}
	\item complex or very complex;
	\item neither very complex nor very simple;
	\item routine or complex
	\item routine or simple
\end{enumerate}
\solution
%\input{exemplar/11/16/3/8(1)/main.tex}
\item A card is selected from a pack of 52 cards.
\begin{enumerate}[label=(\alph*)]
    \item How many points are there in the sample space?
    \item Calculate the probability that the card is an ace of spades.
    \item Calculate the probability that the card is (i) an ace and (ii) black card.
\end{enumerate}
\solution
%\input{exemplar/11/16/3/4/main2.tex}
\item The probability that a non leap year selected at random will contain 53 sundays.
\\
\solution
%\input{exemplar/10/13/1/19/main.tex}
\item One of the four persons John, Rita, Aslam or Gurpreet will be promoted next
month. Consequently the sample space consists of four elementary outcomes
S = {John promoted, Rita promoted, Aslam promoted, Gurpreet promoted}
You are told that the chances of John’s promotion is same as that of Gurpreet,
Rita’s chances of promotion are twice as likely as Johns. Aslam’s chances are
four times that of John.
\begin{enumerate}
	\item Determine
	\begin{enumerate}
		\item P (John promoted)
		\item P (Rita promoted)
		\item P (Aslam promoted)
		\item P (Gurpreet promoted)
	\end{enumerate}
	\item If A = {John promoted or Gurpreet promoted}, find P (A).
\end{enumerate}
\solution
%\input{exemplar/11/16/3/10/main.tex}
\item A card is drawn from a deck of 52 cards. Find the probability of getting a king or a heart or a red card.\\
\solution
%\input{exemplar/11/16/3/15/main.tex}
\item The probability that a student will pass his examination is 0.73, the probability of
the student getting a compartment is 0.13, and the probability that the student will
either pass or get compartment is 0.96. State True or False.\\
\solution
%\input{exemplar/11/16/3/31/main.tex}
\item A card is selected from a pack of 52 cards\\
\begin{enumerate}[label=(\alph*)]
\item How many points are there in the sample space?
\item Calculate the probability that the cards is an ace of spades.
\item Calculate the probability that the card is (i) an ace (ii)black card.\\
\end{enumerate}
%\input{ncert/11/16/3/4_1/Prob_4.tex}
\item In a non-leap year, the probability of having 53 tuesdays or 53 wednesdays is\\
\solution
%\input{exemplar/11/16/3/18/main.tex}
\item There are 1000 sealed envelopes in a box, 10 of them contain a cash prize of
Rs 100 each, 100 of them contain a cash prize of Rs 50 each and 200 of them
contain a cash prize of Rs 10 each and rest do not contain any cash prize. If they
are well shuffled and an envelope is picked up out, what is the probability that it
contains no cash prize?\\
\solution
%\input{exemplar/10/13/3/34/main.tex}
\item 
A die is thrown and a card is selected at random from a deck of 52 playing cards. The probability of getting an even number on the die and a spade card.\\
\solution
%\input{exemplar/12/13/3/78/main.tex}
\item
If 4-digit numbers greater than 5,000 are randomly formed from the digits 0, 1, 3, 5, and 7, what is the probability of forming a number divisible by 5 when:
\begin{enumerate}
    \item The digits are repeated?
    \item The repetition of digits is not allowed?
\end{enumerate}
\solution
%\input{ncert/11/16/4/9/main.tex}
\item Consider the probability space $\brak{\Omega, \mathcal{G}, P}$ where $\Omega = [0,2]$ and $\mathcal{G} = \cbrak{\phi, \Omega, [0,1], (1,2]}$. Let $X$ and $Y$ be two functions on $\Omega$ defined as
\begin{align*}
    X(\omega) = 
    \begin{cases}
        1 & \text{if }\omega \in [0, 1]\\
        2 & \text{if }\omega \in (1, 2]
    \end{cases}
\end{align*}
and
\begin{align*}
    Y(\omega) = 
    \begin{cases}
        2 & \text{if }\omega \in [0, 1.5]\\
        3 & \text{if }\omega \in (1.5, 2].
    \end{cases}
\end{align*}
Then which one of the following statements is true?
\begin{enumerate}
    \item [(A)] $X$ is a random variable with respect to $\mathcal{G}$, but $Y$ is not a random variable with respect to $\mathcal{G}$.
    \item [(B)] $Y$ is a random variable with respect to $\mathcal{G}$, but $X$ is not a random variable with respect to $\mathcal{G}$.
    \item [(C)] Neither $X$ nor $Y$ is a random variable with respect to $\mathcal{G}$.
    \item [(D)] Both $X$ and $Y$ are random variables with respect to $\mathcal{G}$.
\end{enumerate} \hfill (GATE ST 2023)\\
\solution
%\input{gate/ST/2023/14/main.tex}
	\item  A die is loaded in such a way that each odd number is twice as likely to occur as
each even number. Find $P(G)$, where $G$ is the event that a number greater than
3 occurs on a single roll of the die.
\\
\solution
		%\input{exemplar/11/16/3/5/main.tex}
	\item All the jacks, queens and kings are removed from a deck of 52 playing cards. The remaining cards are well shuffled and then one card is drawn at random. Giving ace a value 1 similar value for other cards, find the probability that the card has a value 
		\begin{enumerate}
			\item 7
			\item greater than 7
			\item less than 7
		\end{enumerate}
		%\input{exemplar/10/13/3/30/main.tex}
  \item A Lot consists of 48 mobile phones of which 42 are good, 3 have only minor defects and 3 have major defects.Varnika will buy a phone if it is good but the trader will only buy a mobile if it has no major defects. One phone is selected at random from the lot. What is the probability that it is
\begin{enumerate}
	\item acceptable to Varnika?
            \item acceptable to the trader?
\end{enumerate}
\solution
	%\input{exemplar/10/13/3/40/main.tex}
 \item A student says that if you throw a die, it will show up 1 or not 1. Therefore, the probability of getting 1 and the probability of getting 'not 1' each is equal to $\frac{1}{2}$. Is this correct? Give reasons.\\
 \solution
        %\input{exemplar/10/13/2/9/main.tex}
   \item Four candidates A, B, C, D have ap-
plied for the assignment to coach a school cricket
team. If A is twice as likely to be selected as B, and
B and C are given about the same chance of being
selected, while C is twice as likely to be selected
as D, what are the probabilities that
\begin{enumerate}
\item C will be selected?
\item A will not be selected?
\end{enumerate}
	%\input{exemplar/11/16/3/9/main.tex}
 \item A bag contain 24 balls of which $x$ balls are red, $2x$ are white and $3x$ are blue. A ball is selected at random, What is the probability that it is
\begin{enumerate}[label=\alph*)]
\item not red ?
\item white ?
\end{enumerate}
%\input{exemplar/10/13/3/41/main.tex}
If the letters of the word ASSASSINATION are arranged at random. Find the Probability that
\begin{enumerate}[label=(\alph*)]
\item Four $S's$ come consecutively in the word
\item Two  $I's$ and two $N's$ come together
\item All $A's$ are not coming together
\item No two $A's$ are coming together
\end{enumerate}
%\input{exemplar/11/16/3/14/main.tex}
	\item One urn contains two black balls (labelled B1 and B2) and one white ball. A
	second urn contains one black ball and two white balls (labelled W1 and W2).
	Suppose the following experiment is performed. One of the two urns is chosen
	at random. Next a ball is randomly chosen from the urn. Then a second ball is
	chosen at random from the same urn without replacing the first ball.
	
	\begin{enumerate}
	\item What is the probability that two black balls are chosen?
	
	\item What is the probability that two balls of opposite colour are chosen?
	\end{enumerate}
	\solution
	%\input{exemplar/11/16/3/12/main1.tex}
\end{enumerate}

		%
\item 
Out of 100 students, two sections of 40 and 60 are formed. If you and your friend are among the 100 students, what is the probability that
\begin{enumerate}
\item you both enter the same section?
\item you both enter the different sections?
\end{enumerate}
\solution
		%\begin{enumerate}[label=\thesection.\arabic*,ref=\thesection.\theenumi]
	\item One card is drawn from a well-shuffled deck of 52 cards. Find the probability of getting
\begin{enumerate}
\item A king of red colour 
\item A face card 
\item A red face card
\item The jack of hearts
\item A spade
\item The queen of diamonds

\end{enumerate}
\solution
		%\input{ncert/10/15/1/14/main.tex}
	\item Five cards—the ten, jack, queen, king and ace of diamonds, are well-shuffled with their face downwards. One card is then picked up at random.
\begin{enumerate}
\item
What is the probability that the card is the queen? 
\item
If the queen is drawn and put aside, what is the probability that the second card picked up is (a) an ace? (b) a queen?\\
\end{enumerate}
\solution
		%\input{ncert/10/15/1/15/defs.tex}
	\item A bag contains $5$ red balls and some blue balls. If the probability of drawing a blue ball is double that if a red ball, determine the number of blue balls in the bag. 
		\\
\solution
		%\input{ncert/10/15/2/3/defs.tex}
	\item A card is selected from a pack of 52 cards.
 \begin{enumerate}[label=(\alph*)] 
                 \item How many points are there in the sample space?
                 \item Calculate the probability that the card is an ace of spades.
                 \item Calculate the probability that the card is (i) an ace and (ii) black card.
 \end{enumerate}
\solution
		%\input{ncert/11/16/3/4/main.tex}
\item Four cards are drawn from a well-shuffled deck of 52 cards. What is the probability of obtaining 3 diamonds and one spade.
\\
\solution
		%\input{ncert/11/16/4/2/defs.tex}
\item In a certain lottery 10,000 tickets are sold and ten equal prizes are awarded. What is the probability of not getting a prize if you buy (a) one ticket (b) two tickets (c) 10 tickets ?	
\\
\solution
		%\input{ncert/11/16/4/4/defs.tex}
		%
\item 
Out of 100 students, two sections of 40 and 60 are formed. If you and your friend are among the 100 students, what is the probability that
\begin{enumerate}
\item you both enter the same section?
\item you both enter the different sections?
\end{enumerate}
\solution
		%\input{ncert/11/16/4/5/defs.tex}
	\item 
The number lock of a suitcase has 4 wheels each labelled with ten digits i.e. from 0 to 9.The lock opens with a sequence of four digits with no repeats.What is the probability of a person getting the right sequence to open the suitcase.
\\
\solution
		%\input{ncert/11/16/4/10/defs.tex}
		%
\item 
Two cards are drawn at random and without replacement from a pack of 52 playing cards. Find the probability that both the cards are black.
\\
\solution
		%\input{ncert/12/13/2/2/defs.tex}
		\item A box of oranges is inspected by examining three randomly selected oranges drawn without replacement. If all the three oranges are good, the box is approved for sale, otherwise, it is rejected. Find the probability that a box containing 15 oranges out of which 12 are good and 3 are bad ones will be approved for sale.
		\label{ncert/12/13/2/3/defs.tex}
		\item Two balls are drawn at random with replacement from a box containing 10 black and 8 red balls. Find the probability that
		\label{ncert/12/13/2/12}
\begin{enumerate}
\item both balls are red.
\item first ball is black and second is red.
\item one of them is black and other is red.
\end{enumerate}

\item In a hostel, 60\% of the students read Hindi newspaper, 40\% read English newspaper and 20\% read both Hindi and English newspapers. A student is selected at random.
		\label{ncert/12/13/2/15}
\begin{enumerate}
\item Find the probability that she reads neither Hindi nor English newspapers.
\item If she reads Hindi newspaper, find the probability that she reads English newspaper.
\item If she reads English newspaper, find the probability that she reads Hindi newspaper.\\
\end{enumerate}
\item The probability of obtaining an even prime number on each die, when a pair of dice is rolled is 
\begin{enumerate}
    \item $0$ 
    
    \item $\frac{1}{3}$ 
    
    \item $\frac{1}{12}$ 
    
    \item $\frac{1}{36}$ 
\end{enumerate}
\solution
		%\input{ncert/12/13/2/17/defs.tex}
	\item A bag contains 4 red and 4 black balls, another bag contains 2 red and 6 black balls. One of the two bags is selected at random and a ball is drawn from the bag which is found to be red. Find the probability that the ball is drawn from the first bag.
\\
\solution
		%\input{ncert/12/13/3/2/main.tex}
  \item
  Cards with numbers 2 to 101 are placed in a box. A card is selected at random.Find the probability that the card has
\begin{enumerate}[label=(\roman*)]
	\item an even number 
	\item a square number
\end{enumerate}
\solution
%\input{exemplar/10/13/3/32/main.tex}
\item
The king, queen and jack of clubs are removed from a deck of 52 playing cards and then well shuffled. Now one card is drawn at random from the remaining cards.  Determine the probability that the card is
\begin{enumerate}[label=(\roman*)]
\item a club
\item 10 of hearts
\end{enumerate}
\solution
%\input{exemplar/10/13/3/29/main.tex}
\item A team of medical students doing their internship have to assist during surgeries
at a city hospital. The probabilities of surgeries rated as very complex, complex,
routine, simple or very simple are respectively, 0.15, 0.20, 0.31, 0.26, .08. Find
the probabilities that a particular surgery will be rated
\begin{enumerate}
	\item complex or very complex;
	\item neither very complex nor very simple;
	\item routine or complex
	\item routine or simple
\end{enumerate}
\solution
%\input{exemplar/11/16/3/8(1)/main.tex}
\item A card is selected from a pack of 52 cards.
\begin{enumerate}[label=(\alph*)]
    \item How many points are there in the sample space?
    \item Calculate the probability that the card is an ace of spades.
    \item Calculate the probability that the card is (i) an ace and (ii) black card.
\end{enumerate}
\solution
%\input{exemplar/11/16/3/4/main2.tex}
\item The probability that a non leap year selected at random will contain 53 sundays.
\\
\solution
%\input{exemplar/10/13/1/19/main.tex}
\item One of the four persons John, Rita, Aslam or Gurpreet will be promoted next
month. Consequently the sample space consists of four elementary outcomes
S = {John promoted, Rita promoted, Aslam promoted, Gurpreet promoted}
You are told that the chances of John’s promotion is same as that of Gurpreet,
Rita’s chances of promotion are twice as likely as Johns. Aslam’s chances are
four times that of John.
\begin{enumerate}
	\item Determine
	\begin{enumerate}
		\item P (John promoted)
		\item P (Rita promoted)
		\item P (Aslam promoted)
		\item P (Gurpreet promoted)
	\end{enumerate}
	\item If A = {John promoted or Gurpreet promoted}, find P (A).
\end{enumerate}
\solution
%\input{exemplar/11/16/3/10/main.tex}
\item A card is drawn from a deck of 52 cards. Find the probability of getting a king or a heart or a red card.\\
\solution
%\input{exemplar/11/16/3/15/main.tex}
\item The probability that a student will pass his examination is 0.73, the probability of
the student getting a compartment is 0.13, and the probability that the student will
either pass or get compartment is 0.96. State True or False.\\
\solution
%\input{exemplar/11/16/3/31/main.tex}
\item A card is selected from a pack of 52 cards\\
\begin{enumerate}[label=(\alph*)]
\item How many points are there in the sample space?
\item Calculate the probability that the cards is an ace of spades.
\item Calculate the probability that the card is (i) an ace (ii)black card.\\
\end{enumerate}
%\input{ncert/11/16/3/4_1/Prob_4.tex}
\item In a non-leap year, the probability of having 53 tuesdays or 53 wednesdays is\\
\solution
%\input{exemplar/11/16/3/18/main.tex}
\item There are 1000 sealed envelopes in a box, 10 of them contain a cash prize of
Rs 100 each, 100 of them contain a cash prize of Rs 50 each and 200 of them
contain a cash prize of Rs 10 each and rest do not contain any cash prize. If they
are well shuffled and an envelope is picked up out, what is the probability that it
contains no cash prize?\\
\solution
%\input{exemplar/10/13/3/34/main.tex}
\item 
A die is thrown and a card is selected at random from a deck of 52 playing cards. The probability of getting an even number on the die and a spade card.\\
\solution
%\input{exemplar/12/13/3/78/main.tex}
\item
If 4-digit numbers greater than 5,000 are randomly formed from the digits 0, 1, 3, 5, and 7, what is the probability of forming a number divisible by 5 when:
\begin{enumerate}
    \item The digits are repeated?
    \item The repetition of digits is not allowed?
\end{enumerate}
\solution
%\input{ncert/11/16/4/9/main.tex}
\item Consider the probability space $\brak{\Omega, \mathcal{G}, P}$ where $\Omega = [0,2]$ and $\mathcal{G} = \cbrak{\phi, \Omega, [0,1], (1,2]}$. Let $X$ and $Y$ be two functions on $\Omega$ defined as
\begin{align*}
    X(\omega) = 
    \begin{cases}
        1 & \text{if }\omega \in [0, 1]\\
        2 & \text{if }\omega \in (1, 2]
    \end{cases}
\end{align*}
and
\begin{align*}
    Y(\omega) = 
    \begin{cases}
        2 & \text{if }\omega \in [0, 1.5]\\
        3 & \text{if }\omega \in (1.5, 2].
    \end{cases}
\end{align*}
Then which one of the following statements is true?
\begin{enumerate}
    \item [(A)] $X$ is a random variable with respect to $\mathcal{G}$, but $Y$ is not a random variable with respect to $\mathcal{G}$.
    \item [(B)] $Y$ is a random variable with respect to $\mathcal{G}$, but $X$ is not a random variable with respect to $\mathcal{G}$.
    \item [(C)] Neither $X$ nor $Y$ is a random variable with respect to $\mathcal{G}$.
    \item [(D)] Both $X$ and $Y$ are random variables with respect to $\mathcal{G}$.
\end{enumerate} \hfill (GATE ST 2023)\\
\solution
%\input{gate/ST/2023/14/main.tex}
	\item  A die is loaded in such a way that each odd number is twice as likely to occur as
each even number. Find $P(G)$, where $G$ is the event that a number greater than
3 occurs on a single roll of the die.
\\
\solution
		%\input{exemplar/11/16/3/5/main.tex}
	\item All the jacks, queens and kings are removed from a deck of 52 playing cards. The remaining cards are well shuffled and then one card is drawn at random. Giving ace a value 1 similar value for other cards, find the probability that the card has a value 
		\begin{enumerate}
			\item 7
			\item greater than 7
			\item less than 7
		\end{enumerate}
		%\input{exemplar/10/13/3/30/main.tex}
  \item A Lot consists of 48 mobile phones of which 42 are good, 3 have only minor defects and 3 have major defects.Varnika will buy a phone if it is good but the trader will only buy a mobile if it has no major defects. One phone is selected at random from the lot. What is the probability that it is
\begin{enumerate}
	\item acceptable to Varnika?
            \item acceptable to the trader?
\end{enumerate}
\solution
	%\input{exemplar/10/13/3/40/main.tex}
 \item A student says that if you throw a die, it will show up 1 or not 1. Therefore, the probability of getting 1 and the probability of getting 'not 1' each is equal to $\frac{1}{2}$. Is this correct? Give reasons.\\
 \solution
        %\input{exemplar/10/13/2/9/main.tex}
   \item Four candidates A, B, C, D have ap-
plied for the assignment to coach a school cricket
team. If A is twice as likely to be selected as B, and
B and C are given about the same chance of being
selected, while C is twice as likely to be selected
as D, what are the probabilities that
\begin{enumerate}
\item C will be selected?
\item A will not be selected?
\end{enumerate}
	%\input{exemplar/11/16/3/9/main.tex}
 \item A bag contain 24 balls of which $x$ balls are red, $2x$ are white and $3x$ are blue. A ball is selected at random, What is the probability that it is
\begin{enumerate}[label=\alph*)]
\item not red ?
\item white ?
\end{enumerate}
%\input{exemplar/10/13/3/41/main.tex}
If the letters of the word ASSASSINATION are arranged at random. Find the Probability that
\begin{enumerate}[label=(\alph*)]
\item Four $S's$ come consecutively in the word
\item Two  $I's$ and two $N's$ come together
\item All $A's$ are not coming together
\item No two $A's$ are coming together
\end{enumerate}
%\input{exemplar/11/16/3/14/main.tex}
	\item One urn contains two black balls (labelled B1 and B2) and one white ball. A
	second urn contains one black ball and two white balls (labelled W1 and W2).
	Suppose the following experiment is performed. One of the two urns is chosen
	at random. Next a ball is randomly chosen from the urn. Then a second ball is
	chosen at random from the same urn without replacing the first ball.
	
	\begin{enumerate}
	\item What is the probability that two black balls are chosen?
	
	\item What is the probability that two balls of opposite colour are chosen?
	\end{enumerate}
	\solution
	%\input{exemplar/11/16/3/12/main1.tex}
\end{enumerate}

	\item 
The number lock of a suitcase has 4 wheels each labelled with ten digits i.e. from 0 to 9.The lock opens with a sequence of four digits with no repeats.What is the probability of a person getting the right sequence to open the suitcase.
\\
\solution
		%\begin{enumerate}[label=\thesection.\arabic*,ref=\thesection.\theenumi]
	\item One card is drawn from a well-shuffled deck of 52 cards. Find the probability of getting
\begin{enumerate}
\item A king of red colour 
\item A face card 
\item A red face card
\item The jack of hearts
\item A spade
\item The queen of diamonds

\end{enumerate}
\solution
		%\input{ncert/10/15/1/14/main.tex}
	\item Five cards—the ten, jack, queen, king and ace of diamonds, are well-shuffled with their face downwards. One card is then picked up at random.
\begin{enumerate}
\item
What is the probability that the card is the queen? 
\item
If the queen is drawn and put aside, what is the probability that the second card picked up is (a) an ace? (b) a queen?\\
\end{enumerate}
\solution
		%\input{ncert/10/15/1/15/defs.tex}
	\item A bag contains $5$ red balls and some blue balls. If the probability of drawing a blue ball is double that if a red ball, determine the number of blue balls in the bag. 
		\\
\solution
		%\input{ncert/10/15/2/3/defs.tex}
	\item A card is selected from a pack of 52 cards.
 \begin{enumerate}[label=(\alph*)] 
                 \item How many points are there in the sample space?
                 \item Calculate the probability that the card is an ace of spades.
                 \item Calculate the probability that the card is (i) an ace and (ii) black card.
 \end{enumerate}
\solution
		%\input{ncert/11/16/3/4/main.tex}
\item Four cards are drawn from a well-shuffled deck of 52 cards. What is the probability of obtaining 3 diamonds and one spade.
\\
\solution
		%\input{ncert/11/16/4/2/defs.tex}
\item In a certain lottery 10,000 tickets are sold and ten equal prizes are awarded. What is the probability of not getting a prize if you buy (a) one ticket (b) two tickets (c) 10 tickets ?	
\\
\solution
		%\input{ncert/11/16/4/4/defs.tex}
		%
\item 
Out of 100 students, two sections of 40 and 60 are formed. If you and your friend are among the 100 students, what is the probability that
\begin{enumerate}
\item you both enter the same section?
\item you both enter the different sections?
\end{enumerate}
\solution
		%\input{ncert/11/16/4/5/defs.tex}
	\item 
The number lock of a suitcase has 4 wheels each labelled with ten digits i.e. from 0 to 9.The lock opens with a sequence of four digits with no repeats.What is the probability of a person getting the right sequence to open the suitcase.
\\
\solution
		%\input{ncert/11/16/4/10/defs.tex}
		%
\item 
Two cards are drawn at random and without replacement from a pack of 52 playing cards. Find the probability that both the cards are black.
\\
\solution
		%\input{ncert/12/13/2/2/defs.tex}
		\item A box of oranges is inspected by examining three randomly selected oranges drawn without replacement. If all the three oranges are good, the box is approved for sale, otherwise, it is rejected. Find the probability that a box containing 15 oranges out of which 12 are good and 3 are bad ones will be approved for sale.
		\label{ncert/12/13/2/3/defs.tex}
		\item Two balls are drawn at random with replacement from a box containing 10 black and 8 red balls. Find the probability that
		\label{ncert/12/13/2/12}
\begin{enumerate}
\item both balls are red.
\item first ball is black and second is red.
\item one of them is black and other is red.
\end{enumerate}

\item In a hostel, 60\% of the students read Hindi newspaper, 40\% read English newspaper and 20\% read both Hindi and English newspapers. A student is selected at random.
		\label{ncert/12/13/2/15}
\begin{enumerate}
\item Find the probability that she reads neither Hindi nor English newspapers.
\item If she reads Hindi newspaper, find the probability that she reads English newspaper.
\item If she reads English newspaper, find the probability that she reads Hindi newspaper.\\
\end{enumerate}
\item The probability of obtaining an even prime number on each die, when a pair of dice is rolled is 
\begin{enumerate}
    \item $0$ 
    
    \item $\frac{1}{3}$ 
    
    \item $\frac{1}{12}$ 
    
    \item $\frac{1}{36}$ 
\end{enumerate}
\solution
		%\input{ncert/12/13/2/17/defs.tex}
	\item A bag contains 4 red and 4 black balls, another bag contains 2 red and 6 black balls. One of the two bags is selected at random and a ball is drawn from the bag which is found to be red. Find the probability that the ball is drawn from the first bag.
\\
\solution
		%\input{ncert/12/13/3/2/main.tex}
  \item
  Cards with numbers 2 to 101 are placed in a box. A card is selected at random.Find the probability that the card has
\begin{enumerate}[label=(\roman*)]
	\item an even number 
	\item a square number
\end{enumerate}
\solution
%\input{exemplar/10/13/3/32/main.tex}
\item
The king, queen and jack of clubs are removed from a deck of 52 playing cards and then well shuffled. Now one card is drawn at random from the remaining cards.  Determine the probability that the card is
\begin{enumerate}[label=(\roman*)]
\item a club
\item 10 of hearts
\end{enumerate}
\solution
%\input{exemplar/10/13/3/29/main.tex}
\item A team of medical students doing their internship have to assist during surgeries
at a city hospital. The probabilities of surgeries rated as very complex, complex,
routine, simple or very simple are respectively, 0.15, 0.20, 0.31, 0.26, .08. Find
the probabilities that a particular surgery will be rated
\begin{enumerate}
	\item complex or very complex;
	\item neither very complex nor very simple;
	\item routine or complex
	\item routine or simple
\end{enumerate}
\solution
%\input{exemplar/11/16/3/8(1)/main.tex}
\item A card is selected from a pack of 52 cards.
\begin{enumerate}[label=(\alph*)]
    \item How many points are there in the sample space?
    \item Calculate the probability that the card is an ace of spades.
    \item Calculate the probability that the card is (i) an ace and (ii) black card.
\end{enumerate}
\solution
%\input{exemplar/11/16/3/4/main2.tex}
\item The probability that a non leap year selected at random will contain 53 sundays.
\\
\solution
%\input{exemplar/10/13/1/19/main.tex}
\item One of the four persons John, Rita, Aslam or Gurpreet will be promoted next
month. Consequently the sample space consists of four elementary outcomes
S = {John promoted, Rita promoted, Aslam promoted, Gurpreet promoted}
You are told that the chances of John’s promotion is same as that of Gurpreet,
Rita’s chances of promotion are twice as likely as Johns. Aslam’s chances are
four times that of John.
\begin{enumerate}
	\item Determine
	\begin{enumerate}
		\item P (John promoted)
		\item P (Rita promoted)
		\item P (Aslam promoted)
		\item P (Gurpreet promoted)
	\end{enumerate}
	\item If A = {John promoted or Gurpreet promoted}, find P (A).
\end{enumerate}
\solution
%\input{exemplar/11/16/3/10/main.tex}
\item A card is drawn from a deck of 52 cards. Find the probability of getting a king or a heart or a red card.\\
\solution
%\input{exemplar/11/16/3/15/main.tex}
\item The probability that a student will pass his examination is 0.73, the probability of
the student getting a compartment is 0.13, and the probability that the student will
either pass or get compartment is 0.96. State True or False.\\
\solution
%\input{exemplar/11/16/3/31/main.tex}
\item A card is selected from a pack of 52 cards\\
\begin{enumerate}[label=(\alph*)]
\item How many points are there in the sample space?
\item Calculate the probability that the cards is an ace of spades.
\item Calculate the probability that the card is (i) an ace (ii)black card.\\
\end{enumerate}
%\input{ncert/11/16/3/4_1/Prob_4.tex}
\item In a non-leap year, the probability of having 53 tuesdays or 53 wednesdays is\\
\solution
%\input{exemplar/11/16/3/18/main.tex}
\item There are 1000 sealed envelopes in a box, 10 of them contain a cash prize of
Rs 100 each, 100 of them contain a cash prize of Rs 50 each and 200 of them
contain a cash prize of Rs 10 each and rest do not contain any cash prize. If they
are well shuffled and an envelope is picked up out, what is the probability that it
contains no cash prize?\\
\solution
%\input{exemplar/10/13/3/34/main.tex}
\item 
A die is thrown and a card is selected at random from a deck of 52 playing cards. The probability of getting an even number on the die and a spade card.\\
\solution
%\input{exemplar/12/13/3/78/main.tex}
\item
If 4-digit numbers greater than 5,000 are randomly formed from the digits 0, 1, 3, 5, and 7, what is the probability of forming a number divisible by 5 when:
\begin{enumerate}
    \item The digits are repeated?
    \item The repetition of digits is not allowed?
\end{enumerate}
\solution
%\input{ncert/11/16/4/9/main.tex}
\item Consider the probability space $\brak{\Omega, \mathcal{G}, P}$ where $\Omega = [0,2]$ and $\mathcal{G} = \cbrak{\phi, \Omega, [0,1], (1,2]}$. Let $X$ and $Y$ be two functions on $\Omega$ defined as
\begin{align*}
    X(\omega) = 
    \begin{cases}
        1 & \text{if }\omega \in [0, 1]\\
        2 & \text{if }\omega \in (1, 2]
    \end{cases}
\end{align*}
and
\begin{align*}
    Y(\omega) = 
    \begin{cases}
        2 & \text{if }\omega \in [0, 1.5]\\
        3 & \text{if }\omega \in (1.5, 2].
    \end{cases}
\end{align*}
Then which one of the following statements is true?
\begin{enumerate}
    \item [(A)] $X$ is a random variable with respect to $\mathcal{G}$, but $Y$ is not a random variable with respect to $\mathcal{G}$.
    \item [(B)] $Y$ is a random variable with respect to $\mathcal{G}$, but $X$ is not a random variable with respect to $\mathcal{G}$.
    \item [(C)] Neither $X$ nor $Y$ is a random variable with respect to $\mathcal{G}$.
    \item [(D)] Both $X$ and $Y$ are random variables with respect to $\mathcal{G}$.
\end{enumerate} \hfill (GATE ST 2023)\\
\solution
%\input{gate/ST/2023/14/main.tex}
	\item  A die is loaded in such a way that each odd number is twice as likely to occur as
each even number. Find $P(G)$, where $G$ is the event that a number greater than
3 occurs on a single roll of the die.
\\
\solution
		%\input{exemplar/11/16/3/5/main.tex}
	\item All the jacks, queens and kings are removed from a deck of 52 playing cards. The remaining cards are well shuffled and then one card is drawn at random. Giving ace a value 1 similar value for other cards, find the probability that the card has a value 
		\begin{enumerate}
			\item 7
			\item greater than 7
			\item less than 7
		\end{enumerate}
		%\input{exemplar/10/13/3/30/main.tex}
  \item A Lot consists of 48 mobile phones of which 42 are good, 3 have only minor defects and 3 have major defects.Varnika will buy a phone if it is good but the trader will only buy a mobile if it has no major defects. One phone is selected at random from the lot. What is the probability that it is
\begin{enumerate}
	\item acceptable to Varnika?
            \item acceptable to the trader?
\end{enumerate}
\solution
	%\input{exemplar/10/13/3/40/main.tex}
 \item A student says that if you throw a die, it will show up 1 or not 1. Therefore, the probability of getting 1 and the probability of getting 'not 1' each is equal to $\frac{1}{2}$. Is this correct? Give reasons.\\
 \solution
        %\input{exemplar/10/13/2/9/main.tex}
   \item Four candidates A, B, C, D have ap-
plied for the assignment to coach a school cricket
team. If A is twice as likely to be selected as B, and
B and C are given about the same chance of being
selected, while C is twice as likely to be selected
as D, what are the probabilities that
\begin{enumerate}
\item C will be selected?
\item A will not be selected?
\end{enumerate}
	%\input{exemplar/11/16/3/9/main.tex}
 \item A bag contain 24 balls of which $x$ balls are red, $2x$ are white and $3x$ are blue. A ball is selected at random, What is the probability that it is
\begin{enumerate}[label=\alph*)]
\item not red ?
\item white ?
\end{enumerate}
%\input{exemplar/10/13/3/41/main.tex}
If the letters of the word ASSASSINATION are arranged at random. Find the Probability that
\begin{enumerate}[label=(\alph*)]
\item Four $S's$ come consecutively in the word
\item Two  $I's$ and two $N's$ come together
\item All $A's$ are not coming together
\item No two $A's$ are coming together
\end{enumerate}
%\input{exemplar/11/16/3/14/main.tex}
	\item One urn contains two black balls (labelled B1 and B2) and one white ball. A
	second urn contains one black ball and two white balls (labelled W1 and W2).
	Suppose the following experiment is performed. One of the two urns is chosen
	at random. Next a ball is randomly chosen from the urn. Then a second ball is
	chosen at random from the same urn without replacing the first ball.
	
	\begin{enumerate}
	\item What is the probability that two black balls are chosen?
	
	\item What is the probability that two balls of opposite colour are chosen?
	\end{enumerate}
	\solution
	%\input{exemplar/11/16/3/12/main1.tex}
\end{enumerate}

		%
\item 
Two cards are drawn at random and without replacement from a pack of 52 playing cards. Find the probability that both the cards are black.
\\
\solution
		%\begin{enumerate}[label=\thesection.\arabic*,ref=\thesection.\theenumi]
	\item One card is drawn from a well-shuffled deck of 52 cards. Find the probability of getting
\begin{enumerate}
\item A king of red colour 
\item A face card 
\item A red face card
\item The jack of hearts
\item A spade
\item The queen of diamonds

\end{enumerate}
\solution
		%\input{ncert/10/15/1/14/main.tex}
	\item Five cards—the ten, jack, queen, king and ace of diamonds, are well-shuffled with their face downwards. One card is then picked up at random.
\begin{enumerate}
\item
What is the probability that the card is the queen? 
\item
If the queen is drawn and put aside, what is the probability that the second card picked up is (a) an ace? (b) a queen?\\
\end{enumerate}
\solution
		%\input{ncert/10/15/1/15/defs.tex}
	\item A bag contains $5$ red balls and some blue balls. If the probability of drawing a blue ball is double that if a red ball, determine the number of blue balls in the bag. 
		\\
\solution
		%\input{ncert/10/15/2/3/defs.tex}
	\item A card is selected from a pack of 52 cards.
 \begin{enumerate}[label=(\alph*)] 
                 \item How many points are there in the sample space?
                 \item Calculate the probability that the card is an ace of spades.
                 \item Calculate the probability that the card is (i) an ace and (ii) black card.
 \end{enumerate}
\solution
		%\input{ncert/11/16/3/4/main.tex}
\item Four cards are drawn from a well-shuffled deck of 52 cards. What is the probability of obtaining 3 diamonds and one spade.
\\
\solution
		%\input{ncert/11/16/4/2/defs.tex}
\item In a certain lottery 10,000 tickets are sold and ten equal prizes are awarded. What is the probability of not getting a prize if you buy (a) one ticket (b) two tickets (c) 10 tickets ?	
\\
\solution
		%\input{ncert/11/16/4/4/defs.tex}
		%
\item 
Out of 100 students, two sections of 40 and 60 are formed. If you and your friend are among the 100 students, what is the probability that
\begin{enumerate}
\item you both enter the same section?
\item you both enter the different sections?
\end{enumerate}
\solution
		%\input{ncert/11/16/4/5/defs.tex}
	\item 
The number lock of a suitcase has 4 wheels each labelled with ten digits i.e. from 0 to 9.The lock opens with a sequence of four digits with no repeats.What is the probability of a person getting the right sequence to open the suitcase.
\\
\solution
		%\input{ncert/11/16/4/10/defs.tex}
		%
\item 
Two cards are drawn at random and without replacement from a pack of 52 playing cards. Find the probability that both the cards are black.
\\
\solution
		%\input{ncert/12/13/2/2/defs.tex}
		\item A box of oranges is inspected by examining three randomly selected oranges drawn without replacement. If all the three oranges are good, the box is approved for sale, otherwise, it is rejected. Find the probability that a box containing 15 oranges out of which 12 are good and 3 are bad ones will be approved for sale.
		\label{ncert/12/13/2/3/defs.tex}
		\item Two balls are drawn at random with replacement from a box containing 10 black and 8 red balls. Find the probability that
		\label{ncert/12/13/2/12}
\begin{enumerate}
\item both balls are red.
\item first ball is black and second is red.
\item one of them is black and other is red.
\end{enumerate}

\item In a hostel, 60\% of the students read Hindi newspaper, 40\% read English newspaper and 20\% read both Hindi and English newspapers. A student is selected at random.
		\label{ncert/12/13/2/15}
\begin{enumerate}
\item Find the probability that she reads neither Hindi nor English newspapers.
\item If she reads Hindi newspaper, find the probability that she reads English newspaper.
\item If she reads English newspaper, find the probability that she reads Hindi newspaper.\\
\end{enumerate}
\item The probability of obtaining an even prime number on each die, when a pair of dice is rolled is 
\begin{enumerate}
    \item $0$ 
    
    \item $\frac{1}{3}$ 
    
    \item $\frac{1}{12}$ 
    
    \item $\frac{1}{36}$ 
\end{enumerate}
\solution
		%\input{ncert/12/13/2/17/defs.tex}
	\item A bag contains 4 red and 4 black balls, another bag contains 2 red and 6 black balls. One of the two bags is selected at random and a ball is drawn from the bag which is found to be red. Find the probability that the ball is drawn from the first bag.
\\
\solution
		%\input{ncert/12/13/3/2/main.tex}
  \item
  Cards with numbers 2 to 101 are placed in a box. A card is selected at random.Find the probability that the card has
\begin{enumerate}[label=(\roman*)]
	\item an even number 
	\item a square number
\end{enumerate}
\solution
%\input{exemplar/10/13/3/32/main.tex}
\item
The king, queen and jack of clubs are removed from a deck of 52 playing cards and then well shuffled. Now one card is drawn at random from the remaining cards.  Determine the probability that the card is
\begin{enumerate}[label=(\roman*)]
\item a club
\item 10 of hearts
\end{enumerate}
\solution
%\input{exemplar/10/13/3/29/main.tex}
\item A team of medical students doing their internship have to assist during surgeries
at a city hospital. The probabilities of surgeries rated as very complex, complex,
routine, simple or very simple are respectively, 0.15, 0.20, 0.31, 0.26, .08. Find
the probabilities that a particular surgery will be rated
\begin{enumerate}
	\item complex or very complex;
	\item neither very complex nor very simple;
	\item routine or complex
	\item routine or simple
\end{enumerate}
\solution
%\input{exemplar/11/16/3/8(1)/main.tex}
\item A card is selected from a pack of 52 cards.
\begin{enumerate}[label=(\alph*)]
    \item How many points are there in the sample space?
    \item Calculate the probability that the card is an ace of spades.
    \item Calculate the probability that the card is (i) an ace and (ii) black card.
\end{enumerate}
\solution
%\input{exemplar/11/16/3/4/main2.tex}
\item The probability that a non leap year selected at random will contain 53 sundays.
\\
\solution
%\input{exemplar/10/13/1/19/main.tex}
\item One of the four persons John, Rita, Aslam or Gurpreet will be promoted next
month. Consequently the sample space consists of four elementary outcomes
S = {John promoted, Rita promoted, Aslam promoted, Gurpreet promoted}
You are told that the chances of John’s promotion is same as that of Gurpreet,
Rita’s chances of promotion are twice as likely as Johns. Aslam’s chances are
four times that of John.
\begin{enumerate}
	\item Determine
	\begin{enumerate}
		\item P (John promoted)
		\item P (Rita promoted)
		\item P (Aslam promoted)
		\item P (Gurpreet promoted)
	\end{enumerate}
	\item If A = {John promoted or Gurpreet promoted}, find P (A).
\end{enumerate}
\solution
%\input{exemplar/11/16/3/10/main.tex}
\item A card is drawn from a deck of 52 cards. Find the probability of getting a king or a heart or a red card.\\
\solution
%\input{exemplar/11/16/3/15/main.tex}
\item The probability that a student will pass his examination is 0.73, the probability of
the student getting a compartment is 0.13, and the probability that the student will
either pass or get compartment is 0.96. State True or False.\\
\solution
%\input{exemplar/11/16/3/31/main.tex}
\item A card is selected from a pack of 52 cards\\
\begin{enumerate}[label=(\alph*)]
\item How many points are there in the sample space?
\item Calculate the probability that the cards is an ace of spades.
\item Calculate the probability that the card is (i) an ace (ii)black card.\\
\end{enumerate}
%\input{ncert/11/16/3/4_1/Prob_4.tex}
\item In a non-leap year, the probability of having 53 tuesdays or 53 wednesdays is\\
\solution
%\input{exemplar/11/16/3/18/main.tex}
\item There are 1000 sealed envelopes in a box, 10 of them contain a cash prize of
Rs 100 each, 100 of them contain a cash prize of Rs 50 each and 200 of them
contain a cash prize of Rs 10 each and rest do not contain any cash prize. If they
are well shuffled and an envelope is picked up out, what is the probability that it
contains no cash prize?\\
\solution
%\input{exemplar/10/13/3/34/main.tex}
\item 
A die is thrown and a card is selected at random from a deck of 52 playing cards. The probability of getting an even number on the die and a spade card.\\
\solution
%\input{exemplar/12/13/3/78/main.tex}
\item
If 4-digit numbers greater than 5,000 are randomly formed from the digits 0, 1, 3, 5, and 7, what is the probability of forming a number divisible by 5 when:
\begin{enumerate}
    \item The digits are repeated?
    \item The repetition of digits is not allowed?
\end{enumerate}
\solution
%\input{ncert/11/16/4/9/main.tex}
\item Consider the probability space $\brak{\Omega, \mathcal{G}, P}$ where $\Omega = [0,2]$ and $\mathcal{G} = \cbrak{\phi, \Omega, [0,1], (1,2]}$. Let $X$ and $Y$ be two functions on $\Omega$ defined as
\begin{align*}
    X(\omega) = 
    \begin{cases}
        1 & \text{if }\omega \in [0, 1]\\
        2 & \text{if }\omega \in (1, 2]
    \end{cases}
\end{align*}
and
\begin{align*}
    Y(\omega) = 
    \begin{cases}
        2 & \text{if }\omega \in [0, 1.5]\\
        3 & \text{if }\omega \in (1.5, 2].
    \end{cases}
\end{align*}
Then which one of the following statements is true?
\begin{enumerate}
    \item [(A)] $X$ is a random variable with respect to $\mathcal{G}$, but $Y$ is not a random variable with respect to $\mathcal{G}$.
    \item [(B)] $Y$ is a random variable with respect to $\mathcal{G}$, but $X$ is not a random variable with respect to $\mathcal{G}$.
    \item [(C)] Neither $X$ nor $Y$ is a random variable with respect to $\mathcal{G}$.
    \item [(D)] Both $X$ and $Y$ are random variables with respect to $\mathcal{G}$.
\end{enumerate} \hfill (GATE ST 2023)\\
\solution
%\input{gate/ST/2023/14/main.tex}
	\item  A die is loaded in such a way that each odd number is twice as likely to occur as
each even number. Find $P(G)$, where $G$ is the event that a number greater than
3 occurs on a single roll of the die.
\\
\solution
		%\input{exemplar/11/16/3/5/main.tex}
	\item All the jacks, queens and kings are removed from a deck of 52 playing cards. The remaining cards are well shuffled and then one card is drawn at random. Giving ace a value 1 similar value for other cards, find the probability that the card has a value 
		\begin{enumerate}
			\item 7
			\item greater than 7
			\item less than 7
		\end{enumerate}
		%\input{exemplar/10/13/3/30/main.tex}
  \item A Lot consists of 48 mobile phones of which 42 are good, 3 have only minor defects and 3 have major defects.Varnika will buy a phone if it is good but the trader will only buy a mobile if it has no major defects. One phone is selected at random from the lot. What is the probability that it is
\begin{enumerate}
	\item acceptable to Varnika?
            \item acceptable to the trader?
\end{enumerate}
\solution
	%\input{exemplar/10/13/3/40/main.tex}
 \item A student says that if you throw a die, it will show up 1 or not 1. Therefore, the probability of getting 1 and the probability of getting 'not 1' each is equal to $\frac{1}{2}$. Is this correct? Give reasons.\\
 \solution
        %\input{exemplar/10/13/2/9/main.tex}
   \item Four candidates A, B, C, D have ap-
plied for the assignment to coach a school cricket
team. If A is twice as likely to be selected as B, and
B and C are given about the same chance of being
selected, while C is twice as likely to be selected
as D, what are the probabilities that
\begin{enumerate}
\item C will be selected?
\item A will not be selected?
\end{enumerate}
	%\input{exemplar/11/16/3/9/main.tex}
 \item A bag contain 24 balls of which $x$ balls are red, $2x$ are white and $3x$ are blue. A ball is selected at random, What is the probability that it is
\begin{enumerate}[label=\alph*)]
\item not red ?
\item white ?
\end{enumerate}
%\input{exemplar/10/13/3/41/main.tex}
If the letters of the word ASSASSINATION are arranged at random. Find the Probability that
\begin{enumerate}[label=(\alph*)]
\item Four $S's$ come consecutively in the word
\item Two  $I's$ and two $N's$ come together
\item All $A's$ are not coming together
\item No two $A's$ are coming together
\end{enumerate}
%\input{exemplar/11/16/3/14/main.tex}
	\item One urn contains two black balls (labelled B1 and B2) and one white ball. A
	second urn contains one black ball and two white balls (labelled W1 and W2).
	Suppose the following experiment is performed. One of the two urns is chosen
	at random. Next a ball is randomly chosen from the urn. Then a second ball is
	chosen at random from the same urn without replacing the first ball.
	
	\begin{enumerate}
	\item What is the probability that two black balls are chosen?
	
	\item What is the probability that two balls of opposite colour are chosen?
	\end{enumerate}
	\solution
	%\input{exemplar/11/16/3/12/main1.tex}
\end{enumerate}

		\item A box of oranges is inspected by examining three randomly selected oranges drawn without replacement. If all the three oranges are good, the box is approved for sale, otherwise, it is rejected. Find the probability that a box containing 15 oranges out of which 12 are good and 3 are bad ones will be approved for sale.
		\label{ncert/12/13/2/3/defs.tex}
		\item Two balls are drawn at random with replacement from a box containing 10 black and 8 red balls. Find the probability that
		\label{ncert/12/13/2/12}
\begin{enumerate}
\item both balls are red.
\item first ball is black and second is red.
\item one of them is black and other is red.
\end{enumerate}

\item In a hostel, 60\% of the students read Hindi newspaper, 40\% read English newspaper and 20\% read both Hindi and English newspapers. A student is selected at random.
		\label{ncert/12/13/2/15}
\begin{enumerate}
\item Find the probability that she reads neither Hindi nor English newspapers.
\item If she reads Hindi newspaper, find the probability that she reads English newspaper.
\item If she reads English newspaper, find the probability that she reads Hindi newspaper.\\
\end{enumerate}
\item The probability of obtaining an even prime number on each die, when a pair of dice is rolled is 
\begin{enumerate}
    \item $0$ 
    
    \item $\frac{1}{3}$ 
    
    \item $\frac{1}{12}$ 
    
    \item $\frac{1}{36}$ 
\end{enumerate}
\solution
		%\begin{enumerate}[label=\thesection.\arabic*,ref=\thesection.\theenumi]
	\item One card is drawn from a well-shuffled deck of 52 cards. Find the probability of getting
\begin{enumerate}
\item A king of red colour 
\item A face card 
\item A red face card
\item The jack of hearts
\item A spade
\item The queen of diamonds

\end{enumerate}
\solution
		%\input{ncert/10/15/1/14/main.tex}
	\item Five cards—the ten, jack, queen, king and ace of diamonds, are well-shuffled with their face downwards. One card is then picked up at random.
\begin{enumerate}
\item
What is the probability that the card is the queen? 
\item
If the queen is drawn and put aside, what is the probability that the second card picked up is (a) an ace? (b) a queen?\\
\end{enumerate}
\solution
		%\input{ncert/10/15/1/15/defs.tex}
	\item A bag contains $5$ red balls and some blue balls. If the probability of drawing a blue ball is double that if a red ball, determine the number of blue balls in the bag. 
		\\
\solution
		%\input{ncert/10/15/2/3/defs.tex}
	\item A card is selected from a pack of 52 cards.
 \begin{enumerate}[label=(\alph*)] 
                 \item How many points are there in the sample space?
                 \item Calculate the probability that the card is an ace of spades.
                 \item Calculate the probability that the card is (i) an ace and (ii) black card.
 \end{enumerate}
\solution
		%\input{ncert/11/16/3/4/main.tex}
\item Four cards are drawn from a well-shuffled deck of 52 cards. What is the probability of obtaining 3 diamonds and one spade.
\\
\solution
		%\input{ncert/11/16/4/2/defs.tex}
\item In a certain lottery 10,000 tickets are sold and ten equal prizes are awarded. What is the probability of not getting a prize if you buy (a) one ticket (b) two tickets (c) 10 tickets ?	
\\
\solution
		%\input{ncert/11/16/4/4/defs.tex}
		%
\item 
Out of 100 students, two sections of 40 and 60 are formed. If you and your friend are among the 100 students, what is the probability that
\begin{enumerate}
\item you both enter the same section?
\item you both enter the different sections?
\end{enumerate}
\solution
		%\input{ncert/11/16/4/5/defs.tex}
	\item 
The number lock of a suitcase has 4 wheels each labelled with ten digits i.e. from 0 to 9.The lock opens with a sequence of four digits with no repeats.What is the probability of a person getting the right sequence to open the suitcase.
\\
\solution
		%\input{ncert/11/16/4/10/defs.tex}
		%
\item 
Two cards are drawn at random and without replacement from a pack of 52 playing cards. Find the probability that both the cards are black.
\\
\solution
		%\input{ncert/12/13/2/2/defs.tex}
		\item A box of oranges is inspected by examining three randomly selected oranges drawn without replacement. If all the three oranges are good, the box is approved for sale, otherwise, it is rejected. Find the probability that a box containing 15 oranges out of which 12 are good and 3 are bad ones will be approved for sale.
		\label{ncert/12/13/2/3/defs.tex}
		\item Two balls are drawn at random with replacement from a box containing 10 black and 8 red balls. Find the probability that
		\label{ncert/12/13/2/12}
\begin{enumerate}
\item both balls are red.
\item first ball is black and second is red.
\item one of them is black and other is red.
\end{enumerate}

\item In a hostel, 60\% of the students read Hindi newspaper, 40\% read English newspaper and 20\% read both Hindi and English newspapers. A student is selected at random.
		\label{ncert/12/13/2/15}
\begin{enumerate}
\item Find the probability that she reads neither Hindi nor English newspapers.
\item If she reads Hindi newspaper, find the probability that she reads English newspaper.
\item If she reads English newspaper, find the probability that she reads Hindi newspaper.\\
\end{enumerate}
\item The probability of obtaining an even prime number on each die, when a pair of dice is rolled is 
\begin{enumerate}
    \item $0$ 
    
    \item $\frac{1}{3}$ 
    
    \item $\frac{1}{12}$ 
    
    \item $\frac{1}{36}$ 
\end{enumerate}
\solution
		%\input{ncert/12/13/2/17/defs.tex}
	\item A bag contains 4 red and 4 black balls, another bag contains 2 red and 6 black balls. One of the two bags is selected at random and a ball is drawn from the bag which is found to be red. Find the probability that the ball is drawn from the first bag.
\\
\solution
		%\input{ncert/12/13/3/2/main.tex}
  \item
  Cards with numbers 2 to 101 are placed in a box. A card is selected at random.Find the probability that the card has
\begin{enumerate}[label=(\roman*)]
	\item an even number 
	\item a square number
\end{enumerate}
\solution
%\input{exemplar/10/13/3/32/main.tex}
\item
The king, queen and jack of clubs are removed from a deck of 52 playing cards and then well shuffled. Now one card is drawn at random from the remaining cards.  Determine the probability that the card is
\begin{enumerate}[label=(\roman*)]
\item a club
\item 10 of hearts
\end{enumerate}
\solution
%\input{exemplar/10/13/3/29/main.tex}
\item A team of medical students doing their internship have to assist during surgeries
at a city hospital. The probabilities of surgeries rated as very complex, complex,
routine, simple or very simple are respectively, 0.15, 0.20, 0.31, 0.26, .08. Find
the probabilities that a particular surgery will be rated
\begin{enumerate}
	\item complex or very complex;
	\item neither very complex nor very simple;
	\item routine or complex
	\item routine or simple
\end{enumerate}
\solution
%\input{exemplar/11/16/3/8(1)/main.tex}
\item A card is selected from a pack of 52 cards.
\begin{enumerate}[label=(\alph*)]
    \item How many points are there in the sample space?
    \item Calculate the probability that the card is an ace of spades.
    \item Calculate the probability that the card is (i) an ace and (ii) black card.
\end{enumerate}
\solution
%\input{exemplar/11/16/3/4/main2.tex}
\item The probability that a non leap year selected at random will contain 53 sundays.
\\
\solution
%\input{exemplar/10/13/1/19/main.tex}
\item One of the four persons John, Rita, Aslam or Gurpreet will be promoted next
month. Consequently the sample space consists of four elementary outcomes
S = {John promoted, Rita promoted, Aslam promoted, Gurpreet promoted}
You are told that the chances of John’s promotion is same as that of Gurpreet,
Rita’s chances of promotion are twice as likely as Johns. Aslam’s chances are
four times that of John.
\begin{enumerate}
	\item Determine
	\begin{enumerate}
		\item P (John promoted)
		\item P (Rita promoted)
		\item P (Aslam promoted)
		\item P (Gurpreet promoted)
	\end{enumerate}
	\item If A = {John promoted or Gurpreet promoted}, find P (A).
\end{enumerate}
\solution
%\input{exemplar/11/16/3/10/main.tex}
\item A card is drawn from a deck of 52 cards. Find the probability of getting a king or a heart or a red card.\\
\solution
%\input{exemplar/11/16/3/15/main.tex}
\item The probability that a student will pass his examination is 0.73, the probability of
the student getting a compartment is 0.13, and the probability that the student will
either pass or get compartment is 0.96. State True or False.\\
\solution
%\input{exemplar/11/16/3/31/main.tex}
\item A card is selected from a pack of 52 cards\\
\begin{enumerate}[label=(\alph*)]
\item How many points are there in the sample space?
\item Calculate the probability that the cards is an ace of spades.
\item Calculate the probability that the card is (i) an ace (ii)black card.\\
\end{enumerate}
%\input{ncert/11/16/3/4_1/Prob_4.tex}
\item In a non-leap year, the probability of having 53 tuesdays or 53 wednesdays is\\
\solution
%\input{exemplar/11/16/3/18/main.tex}
\item There are 1000 sealed envelopes in a box, 10 of them contain a cash prize of
Rs 100 each, 100 of them contain a cash prize of Rs 50 each and 200 of them
contain a cash prize of Rs 10 each and rest do not contain any cash prize. If they
are well shuffled and an envelope is picked up out, what is the probability that it
contains no cash prize?\\
\solution
%\input{exemplar/10/13/3/34/main.tex}
\item 
A die is thrown and a card is selected at random from a deck of 52 playing cards. The probability of getting an even number on the die and a spade card.\\
\solution
%\input{exemplar/12/13/3/78/main.tex}
\item
If 4-digit numbers greater than 5,000 are randomly formed from the digits 0, 1, 3, 5, and 7, what is the probability of forming a number divisible by 5 when:
\begin{enumerate}
    \item The digits are repeated?
    \item The repetition of digits is not allowed?
\end{enumerate}
\solution
%\input{ncert/11/16/4/9/main.tex}
\item Consider the probability space $\brak{\Omega, \mathcal{G}, P}$ where $\Omega = [0,2]$ and $\mathcal{G} = \cbrak{\phi, \Omega, [0,1], (1,2]}$. Let $X$ and $Y$ be two functions on $\Omega$ defined as
\begin{align*}
    X(\omega) = 
    \begin{cases}
        1 & \text{if }\omega \in [0, 1]\\
        2 & \text{if }\omega \in (1, 2]
    \end{cases}
\end{align*}
and
\begin{align*}
    Y(\omega) = 
    \begin{cases}
        2 & \text{if }\omega \in [0, 1.5]\\
        3 & \text{if }\omega \in (1.5, 2].
    \end{cases}
\end{align*}
Then which one of the following statements is true?
\begin{enumerate}
    \item [(A)] $X$ is a random variable with respect to $\mathcal{G}$, but $Y$ is not a random variable with respect to $\mathcal{G}$.
    \item [(B)] $Y$ is a random variable with respect to $\mathcal{G}$, but $X$ is not a random variable with respect to $\mathcal{G}$.
    \item [(C)] Neither $X$ nor $Y$ is a random variable with respect to $\mathcal{G}$.
    \item [(D)] Both $X$ and $Y$ are random variables with respect to $\mathcal{G}$.
\end{enumerate} \hfill (GATE ST 2023)\\
\solution
%\input{gate/ST/2023/14/main.tex}
	\item  A die is loaded in such a way that each odd number is twice as likely to occur as
each even number. Find $P(G)$, where $G$ is the event that a number greater than
3 occurs on a single roll of the die.
\\
\solution
		%\input{exemplar/11/16/3/5/main.tex}
	\item All the jacks, queens and kings are removed from a deck of 52 playing cards. The remaining cards are well shuffled and then one card is drawn at random. Giving ace a value 1 similar value for other cards, find the probability that the card has a value 
		\begin{enumerate}
			\item 7
			\item greater than 7
			\item less than 7
		\end{enumerate}
		%\input{exemplar/10/13/3/30/main.tex}
  \item A Lot consists of 48 mobile phones of which 42 are good, 3 have only minor defects and 3 have major defects.Varnika will buy a phone if it is good but the trader will only buy a mobile if it has no major defects. One phone is selected at random from the lot. What is the probability that it is
\begin{enumerate}
	\item acceptable to Varnika?
            \item acceptable to the trader?
\end{enumerate}
\solution
	%\input{exemplar/10/13/3/40/main.tex}
 \item A student says that if you throw a die, it will show up 1 or not 1. Therefore, the probability of getting 1 and the probability of getting 'not 1' each is equal to $\frac{1}{2}$. Is this correct? Give reasons.\\
 \solution
        %\input{exemplar/10/13/2/9/main.tex}
   \item Four candidates A, B, C, D have ap-
plied for the assignment to coach a school cricket
team. If A is twice as likely to be selected as B, and
B and C are given about the same chance of being
selected, while C is twice as likely to be selected
as D, what are the probabilities that
\begin{enumerate}
\item C will be selected?
\item A will not be selected?
\end{enumerate}
	%\input{exemplar/11/16/3/9/main.tex}
 \item A bag contain 24 balls of which $x$ balls are red, $2x$ are white and $3x$ are blue. A ball is selected at random, What is the probability that it is
\begin{enumerate}[label=\alph*)]
\item not red ?
\item white ?
\end{enumerate}
%\input{exemplar/10/13/3/41/main.tex}
If the letters of the word ASSASSINATION are arranged at random. Find the Probability that
\begin{enumerate}[label=(\alph*)]
\item Four $S's$ come consecutively in the word
\item Two  $I's$ and two $N's$ come together
\item All $A's$ are not coming together
\item No two $A's$ are coming together
\end{enumerate}
%\input{exemplar/11/16/3/14/main.tex}
	\item One urn contains two black balls (labelled B1 and B2) and one white ball. A
	second urn contains one black ball and two white balls (labelled W1 and W2).
	Suppose the following experiment is performed. One of the two urns is chosen
	at random. Next a ball is randomly chosen from the urn. Then a second ball is
	chosen at random from the same urn without replacing the first ball.
	
	\begin{enumerate}
	\item What is the probability that two black balls are chosen?
	
	\item What is the probability that two balls of opposite colour are chosen?
	\end{enumerate}
	\solution
	%\input{exemplar/11/16/3/12/main1.tex}
\end{enumerate}

	\item A bag contains 4 red and 4 black balls, another bag contains 2 red and 6 black balls. One of the two bags is selected at random and a ball is drawn from the bag which is found to be red. Find the probability that the ball is drawn from the first bag.
\\
\solution
		%\begin{table}[H]
	\centering
\begin{tabular}{|c|c|c|}
\hline
Random variable &Value &Definition\\ \hline
\multirow{3}{*}{X} &0 &Slips of Rs 1\\
&1 &Slips of Rs 5\\
&2 &Slips of Rs 13\\ \hline
\multirow{2}{*}{Y} &0 &Box A\\
&1 &Box B\\\hline
\end{tabular}
\caption{}
\label{tab:Distribution}
\end{table}
See \tabref{tab:Distribution}.
\begin{align}
p_{Y}\brak{k}= \begin{cases} 
      \frac{1}{3} & {k=0} \\
      \frac{2}{3 }& {k=1} 
   \end{cases}
   \\
p_{Y|X}\brak{0|0} = \frac{19}{25}\, 
p_{Y|X}\brak{0|1} = \frac{6}{25}\,
p_{Y|X}\brak{1|0} = \frac{45}{50}\,
p_{Y|X}\brak{1|2} = \frac{5}{50}
\end{align}
The desired probability is the probability that a slip drawn at random is marked other than Rs 1,
\begin{align}
&=1-p_X\brak{0}\\
&= p_X(1) + p_X(2)
\end{align}
Using Bayes theorem,
\begin{align}
&= p_Y\brak{0} \times \pr{Y=0 | X=1} + p_Y\brak{1} \times \pr{Y=1|X=2}\\
&=\frac{1}{3} \times \frac{6}{25} + \frac{2}{3} \times \frac{5}{50}\\
&=\frac{11}{75}
\end{align}

\newpage

%\tableofcontents

\bigskip

\renewcommand{\thefigure}{\theenumi}
\renewcommand{\thetable}{\theenumi}
%\renewcommand{\theequation}{\theenumi}

%\begin{abstract}
%%\boldmath
%In this letter, an algorithm for evaluating the exact analytical bit error rate  (BER)  for the piecewise linear (PL) combiner for  multiple relays is presented. Previous results were available only for upto three relays. The algorithm is unique in the sense that  the actual mathematical expressions, that are prohibitively large, need not be explicitly obtained. The diversity gain due to multiple relays is shown through plots of the analytical BER, well supported by simulations. 
%
%\end{abstract}
% IEEEtran.cls defaults to using nonbold math in the Abstract.
% This preserves the distinction between vectors and scalars. However,
% if the journal you are submitting to favors bold math in the abstract,
% then you can use LaTeX's standard command \boldmath at the very start
% of the abstract to achieve this. Many IEEE journals frown on math
% in the abstract anyway.

% Note that keywords are not normally used for peerreview papers.
%\begin{IEEEkeywords}
%Cooperative diversity, decode and forward, piecewise linear
%\end{IEEEkeywords}



% For peer review papers, you can put extra information on the cover
% page as needed:
% \ifCLASSOPTIONpeerreview
% \begin{center} \bfseries EDICS Category: 3-BBND \end{center}
% \fi
%
% For peerreview papers, this IEEEtran command inserts a page break and
% creates the second title. It will be ignored for other modes.
%\IEEEpeerreviewmaketitle




  \item
  Cards with numbers 2 to 101 are placed in a box. A card is selected at random.Find the probability that the card has
\begin{enumerate}[label=(\roman*)]
	\item an even number 
	\item a square number
\end{enumerate}
\solution
%\begin{table}[H]
	\centering
\begin{tabular}{|c|c|c|}
\hline
Random variable &Value &Definition\\ \hline
\multirow{3}{*}{X} &0 &Slips of Rs 1\\
&1 &Slips of Rs 5\\
&2 &Slips of Rs 13\\ \hline
\multirow{2}{*}{Y} &0 &Box A\\
&1 &Box B\\\hline
\end{tabular}
\caption{}
\label{tab:Distribution}
\end{table}
See \tabref{tab:Distribution}.
\begin{align}
p_{Y}\brak{k}= \begin{cases} 
      \frac{1}{3} & {k=0} \\
      \frac{2}{3 }& {k=1} 
   \end{cases}
   \\
p_{Y|X}\brak{0|0} = \frac{19}{25}\, 
p_{Y|X}\brak{0|1} = \frac{6}{25}\,
p_{Y|X}\brak{1|0} = \frac{45}{50}\,
p_{Y|X}\brak{1|2} = \frac{5}{50}
\end{align}
The desired probability is the probability that a slip drawn at random is marked other than Rs 1,
\begin{align}
&=1-p_X\brak{0}\\
&= p_X(1) + p_X(2)
\end{align}
Using Bayes theorem,
\begin{align}
&= p_Y\brak{0} \times \pr{Y=0 | X=1} + p_Y\brak{1} \times \pr{Y=1|X=2}\\
&=\frac{1}{3} \times \frac{6}{25} + \frac{2}{3} \times \frac{5}{50}\\
&=\frac{11}{75}
\end{align}

\newpage

%\tableofcontents

\bigskip

\renewcommand{\thefigure}{\theenumi}
\renewcommand{\thetable}{\theenumi}
%\renewcommand{\theequation}{\theenumi}

%\begin{abstract}
%%\boldmath
%In this letter, an algorithm for evaluating the exact analytical bit error rate  (BER)  for the piecewise linear (PL) combiner for  multiple relays is presented. Previous results were available only for upto three relays. The algorithm is unique in the sense that  the actual mathematical expressions, that are prohibitively large, need not be explicitly obtained. The diversity gain due to multiple relays is shown through plots of the analytical BER, well supported by simulations. 
%
%\end{abstract}
% IEEEtran.cls defaults to using nonbold math in the Abstract.
% This preserves the distinction between vectors and scalars. However,
% if the journal you are submitting to favors bold math in the abstract,
% then you can use LaTeX's standard command \boldmath at the very start
% of the abstract to achieve this. Many IEEE journals frown on math
% in the abstract anyway.

% Note that keywords are not normally used for peerreview papers.
%\begin{IEEEkeywords}
%Cooperative diversity, decode and forward, piecewise linear
%\end{IEEEkeywords}



% For peer review papers, you can put extra information on the cover
% page as needed:
% \ifCLASSOPTIONpeerreview
% \begin{center} \bfseries EDICS Category: 3-BBND \end{center}
% \fi
%
% For peerreview papers, this IEEEtran command inserts a page break and
% creates the second title. It will be ignored for other modes.
%\IEEEpeerreviewmaketitle




\item
The king, queen and jack of clubs are removed from a deck of 52 playing cards and then well shuffled. Now one card is drawn at random from the remaining cards.  Determine the probability that the card is
\begin{enumerate}[label=(\roman*)]
\item a club
\item 10 of hearts
\end{enumerate}
\solution
%\begin{table}[H]
	\centering
\begin{tabular}{|c|c|c|}
\hline
Random variable &Value &Definition\\ \hline
\multirow{3}{*}{X} &0 &Slips of Rs 1\\
&1 &Slips of Rs 5\\
&2 &Slips of Rs 13\\ \hline
\multirow{2}{*}{Y} &0 &Box A\\
&1 &Box B\\\hline
\end{tabular}
\caption{}
\label{tab:Distribution}
\end{table}
See \tabref{tab:Distribution}.
\begin{align}
p_{Y}\brak{k}= \begin{cases} 
      \frac{1}{3} & {k=0} \\
      \frac{2}{3 }& {k=1} 
   \end{cases}
   \\
p_{Y|X}\brak{0|0} = \frac{19}{25}\, 
p_{Y|X}\brak{0|1} = \frac{6}{25}\,
p_{Y|X}\brak{1|0} = \frac{45}{50}\,
p_{Y|X}\brak{1|2} = \frac{5}{50}
\end{align}
The desired probability is the probability that a slip drawn at random is marked other than Rs 1,
\begin{align}
&=1-p_X\brak{0}\\
&= p_X(1) + p_X(2)
\end{align}
Using Bayes theorem,
\begin{align}
&= p_Y\brak{0} \times \pr{Y=0 | X=1} + p_Y\brak{1} \times \pr{Y=1|X=2}\\
&=\frac{1}{3} \times \frac{6}{25} + \frac{2}{3} \times \frac{5}{50}\\
&=\frac{11}{75}
\end{align}

\newpage

%\tableofcontents

\bigskip

\renewcommand{\thefigure}{\theenumi}
\renewcommand{\thetable}{\theenumi}
%\renewcommand{\theequation}{\theenumi}

%\begin{abstract}
%%\boldmath
%In this letter, an algorithm for evaluating the exact analytical bit error rate  (BER)  for the piecewise linear (PL) combiner for  multiple relays is presented. Previous results were available only for upto three relays. The algorithm is unique in the sense that  the actual mathematical expressions, that are prohibitively large, need not be explicitly obtained. The diversity gain due to multiple relays is shown through plots of the analytical BER, well supported by simulations. 
%
%\end{abstract}
% IEEEtran.cls defaults to using nonbold math in the Abstract.
% This preserves the distinction between vectors and scalars. However,
% if the journal you are submitting to favors bold math in the abstract,
% then you can use LaTeX's standard command \boldmath at the very start
% of the abstract to achieve this. Many IEEE journals frown on math
% in the abstract anyway.

% Note that keywords are not normally used for peerreview papers.
%\begin{IEEEkeywords}
%Cooperative diversity, decode and forward, piecewise linear
%\end{IEEEkeywords}



% For peer review papers, you can put extra information on the cover
% page as needed:
% \ifCLASSOPTIONpeerreview
% \begin{center} \bfseries EDICS Category: 3-BBND \end{center}
% \fi
%
% For peerreview papers, this IEEEtran command inserts a page break and
% creates the second title. It will be ignored for other modes.
%\IEEEpeerreviewmaketitle




\item A team of medical students doing their internship have to assist during surgeries
at a city hospital. The probabilities of surgeries rated as very complex, complex,
routine, simple or very simple are respectively, 0.15, 0.20, 0.31, 0.26, .08. Find
the probabilities that a particular surgery will be rated
\begin{enumerate}
	\item complex or very complex;
	\item neither very complex nor very simple;
	\item routine or complex
	\item routine or simple
\end{enumerate}
\solution
%\begin{table}[H]
	\centering
\begin{tabular}{|c|c|c|}
\hline
Random variable &Value &Definition\\ \hline
\multirow{3}{*}{X} &0 &Slips of Rs 1\\
&1 &Slips of Rs 5\\
&2 &Slips of Rs 13\\ \hline
\multirow{2}{*}{Y} &0 &Box A\\
&1 &Box B\\\hline
\end{tabular}
\caption{}
\label{tab:Distribution}
\end{table}
See \tabref{tab:Distribution}.
\begin{align}
p_{Y}\brak{k}= \begin{cases} 
      \frac{1}{3} & {k=0} \\
      \frac{2}{3 }& {k=1} 
   \end{cases}
   \\
p_{Y|X}\brak{0|0} = \frac{19}{25}\, 
p_{Y|X}\brak{0|1} = \frac{6}{25}\,
p_{Y|X}\brak{1|0} = \frac{45}{50}\,
p_{Y|X}\brak{1|2} = \frac{5}{50}
\end{align}
The desired probability is the probability that a slip drawn at random is marked other than Rs 1,
\begin{align}
&=1-p_X\brak{0}\\
&= p_X(1) + p_X(2)
\end{align}
Using Bayes theorem,
\begin{align}
&= p_Y\brak{0} \times \pr{Y=0 | X=1} + p_Y\brak{1} \times \pr{Y=1|X=2}\\
&=\frac{1}{3} \times \frac{6}{25} + \frac{2}{3} \times \frac{5}{50}\\
&=\frac{11}{75}
\end{align}

\newpage

%\tableofcontents

\bigskip

\renewcommand{\thefigure}{\theenumi}
\renewcommand{\thetable}{\theenumi}
%\renewcommand{\theequation}{\theenumi}

%\begin{abstract}
%%\boldmath
%In this letter, an algorithm for evaluating the exact analytical bit error rate  (BER)  for the piecewise linear (PL) combiner for  multiple relays is presented. Previous results were available only for upto three relays. The algorithm is unique in the sense that  the actual mathematical expressions, that are prohibitively large, need not be explicitly obtained. The diversity gain due to multiple relays is shown through plots of the analytical BER, well supported by simulations. 
%
%\end{abstract}
% IEEEtran.cls defaults to using nonbold math in the Abstract.
% This preserves the distinction between vectors and scalars. However,
% if the journal you are submitting to favors bold math in the abstract,
% then you can use LaTeX's standard command \boldmath at the very start
% of the abstract to achieve this. Many IEEE journals frown on math
% in the abstract anyway.

% Note that keywords are not normally used for peerreview papers.
%\begin{IEEEkeywords}
%Cooperative diversity, decode and forward, piecewise linear
%\end{IEEEkeywords}



% For peer review papers, you can put extra information on the cover
% page as needed:
% \ifCLASSOPTIONpeerreview
% \begin{center} \bfseries EDICS Category: 3-BBND \end{center}
% \fi
%
% For peerreview papers, this IEEEtran command inserts a page break and
% creates the second title. It will be ignored for other modes.
%\IEEEpeerreviewmaketitle




\item A card is selected from a pack of 52 cards.
\begin{enumerate}[label=(\alph*)]
    \item How many points are there in the sample space?
    \item Calculate the probability that the card is an ace of spades.
    \item Calculate the probability that the card is (i) an ace and (ii) black card.
\end{enumerate}
\solution
%Let $X$ be an bernoulli rv defined as in \tabref{tab:exemplar/11/16/3/26}.  Then, 
\begin{equation}
    p =
        \frac{4}{11} 
\end{equation}
\begin{table}[H]
	\centering
	\input{exemplar/11/16/3/26/tables/Table2.tex}
	\caption{}
        \label{tab:exemplar/11/16/3/26}
\end{table}

\item The probability that a non leap year selected at random will contain 53 sundays.
\\
\solution
%\begin{table}[H]
	\centering
\begin{tabular}{|c|c|c|}
\hline
Random variable &Value &Definition\\ \hline
\multirow{3}{*}{X} &0 &Slips of Rs 1\\
&1 &Slips of Rs 5\\
&2 &Slips of Rs 13\\ \hline
\multirow{2}{*}{Y} &0 &Box A\\
&1 &Box B\\\hline
\end{tabular}
\caption{}
\label{tab:Distribution}
\end{table}
See \tabref{tab:Distribution}.
\begin{align}
p_{Y}\brak{k}= \begin{cases} 
      \frac{1}{3} & {k=0} \\
      \frac{2}{3 }& {k=1} 
   \end{cases}
   \\
p_{Y|X}\brak{0|0} = \frac{19}{25}\, 
p_{Y|X}\brak{0|1} = \frac{6}{25}\,
p_{Y|X}\brak{1|0} = \frac{45}{50}\,
p_{Y|X}\brak{1|2} = \frac{5}{50}
\end{align}
The desired probability is the probability that a slip drawn at random is marked other than Rs 1,
\begin{align}
&=1-p_X\brak{0}\\
&= p_X(1) + p_X(2)
\end{align}
Using Bayes theorem,
\begin{align}
&= p_Y\brak{0} \times \pr{Y=0 | X=1} + p_Y\brak{1} \times \pr{Y=1|X=2}\\
&=\frac{1}{3} \times \frac{6}{25} + \frac{2}{3} \times \frac{5}{50}\\
&=\frac{11}{75}
\end{align}

\newpage

%\tableofcontents

\bigskip

\renewcommand{\thefigure}{\theenumi}
\renewcommand{\thetable}{\theenumi}
%\renewcommand{\theequation}{\theenumi}

%\begin{abstract}
%%\boldmath
%In this letter, an algorithm for evaluating the exact analytical bit error rate  (BER)  for the piecewise linear (PL) combiner for  multiple relays is presented. Previous results were available only for upto three relays. The algorithm is unique in the sense that  the actual mathematical expressions, that are prohibitively large, need not be explicitly obtained. The diversity gain due to multiple relays is shown through plots of the analytical BER, well supported by simulations. 
%
%\end{abstract}
% IEEEtran.cls defaults to using nonbold math in the Abstract.
% This preserves the distinction between vectors and scalars. However,
% if the journal you are submitting to favors bold math in the abstract,
% then you can use LaTeX's standard command \boldmath at the very start
% of the abstract to achieve this. Many IEEE journals frown on math
% in the abstract anyway.

% Note that keywords are not normally used for peerreview papers.
%\begin{IEEEkeywords}
%Cooperative diversity, decode and forward, piecewise linear
%\end{IEEEkeywords}



% For peer review papers, you can put extra information on the cover
% page as needed:
% \ifCLASSOPTIONpeerreview
% \begin{center} \bfseries EDICS Category: 3-BBND \end{center}
% \fi
%
% For peerreview papers, this IEEEtran command inserts a page break and
% creates the second title. It will be ignored for other modes.
%\IEEEpeerreviewmaketitle




\item One of the four persons John, Rita, Aslam or Gurpreet will be promoted next
month. Consequently the sample space consists of four elementary outcomes
S = {John promoted, Rita promoted, Aslam promoted, Gurpreet promoted}
You are told that the chances of John’s promotion is same as that of Gurpreet,
Rita’s chances of promotion are twice as likely as Johns. Aslam’s chances are
four times that of John.
\begin{enumerate}
	\item Determine
	\begin{enumerate}
		\item P (John promoted)
		\item P (Rita promoted)
		\item P (Aslam promoted)
		\item P (Gurpreet promoted)
	\end{enumerate}
	\item If A = {John promoted or Gurpreet promoted}, find P (A).
\end{enumerate}
\solution
%\begin{table}[H]
	\centering
\begin{tabular}{|c|c|c|}
\hline
Random variable &Value &Definition\\ \hline
\multirow{3}{*}{X} &0 &Slips of Rs 1\\
&1 &Slips of Rs 5\\
&2 &Slips of Rs 13\\ \hline
\multirow{2}{*}{Y} &0 &Box A\\
&1 &Box B\\\hline
\end{tabular}
\caption{}
\label{tab:Distribution}
\end{table}
See \tabref{tab:Distribution}.
\begin{align}
p_{Y}\brak{k}= \begin{cases} 
      \frac{1}{3} & {k=0} \\
      \frac{2}{3 }& {k=1} 
   \end{cases}
   \\
p_{Y|X}\brak{0|0} = \frac{19}{25}\, 
p_{Y|X}\brak{0|1} = \frac{6}{25}\,
p_{Y|X}\brak{1|0} = \frac{45}{50}\,
p_{Y|X}\brak{1|2} = \frac{5}{50}
\end{align}
The desired probability is the probability that a slip drawn at random is marked other than Rs 1,
\begin{align}
&=1-p_X\brak{0}\\
&= p_X(1) + p_X(2)
\end{align}
Using Bayes theorem,
\begin{align}
&= p_Y\brak{0} \times \pr{Y=0 | X=1} + p_Y\brak{1} \times \pr{Y=1|X=2}\\
&=\frac{1}{3} \times \frac{6}{25} + \frac{2}{3} \times \frac{5}{50}\\
&=\frac{11}{75}
\end{align}

\newpage

%\tableofcontents

\bigskip

\renewcommand{\thefigure}{\theenumi}
\renewcommand{\thetable}{\theenumi}
%\renewcommand{\theequation}{\theenumi}

%\begin{abstract}
%%\boldmath
%In this letter, an algorithm for evaluating the exact analytical bit error rate  (BER)  for the piecewise linear (PL) combiner for  multiple relays is presented. Previous results were available only for upto three relays. The algorithm is unique in the sense that  the actual mathematical expressions, that are prohibitively large, need not be explicitly obtained. The diversity gain due to multiple relays is shown through plots of the analytical BER, well supported by simulations. 
%
%\end{abstract}
% IEEEtran.cls defaults to using nonbold math in the Abstract.
% This preserves the distinction between vectors and scalars. However,
% if the journal you are submitting to favors bold math in the abstract,
% then you can use LaTeX's standard command \boldmath at the very start
% of the abstract to achieve this. Many IEEE journals frown on math
% in the abstract anyway.

% Note that keywords are not normally used for peerreview papers.
%\begin{IEEEkeywords}
%Cooperative diversity, decode and forward, piecewise linear
%\end{IEEEkeywords}



% For peer review papers, you can put extra information on the cover
% page as needed:
% \ifCLASSOPTIONpeerreview
% \begin{center} \bfseries EDICS Category: 3-BBND \end{center}
% \fi
%
% For peerreview papers, this IEEEtran command inserts a page break and
% creates the second title. It will be ignored for other modes.
%\IEEEpeerreviewmaketitle




\item A card is drawn from a deck of 52 cards. Find the probability of getting a king or a heart or a red card.\\
\solution
%\begin{table}[H]
	\centering
\begin{tabular}{|c|c|c|}
\hline
Random variable &Value &Definition\\ \hline
\multirow{3}{*}{X} &0 &Slips of Rs 1\\
&1 &Slips of Rs 5\\
&2 &Slips of Rs 13\\ \hline
\multirow{2}{*}{Y} &0 &Box A\\
&1 &Box B\\\hline
\end{tabular}
\caption{}
\label{tab:Distribution}
\end{table}
See \tabref{tab:Distribution}.
\begin{align}
p_{Y}\brak{k}= \begin{cases} 
      \frac{1}{3} & {k=0} \\
      \frac{2}{3 }& {k=1} 
   \end{cases}
   \\
p_{Y|X}\brak{0|0} = \frac{19}{25}\, 
p_{Y|X}\brak{0|1} = \frac{6}{25}\,
p_{Y|X}\brak{1|0} = \frac{45}{50}\,
p_{Y|X}\brak{1|2} = \frac{5}{50}
\end{align}
The desired probability is the probability that a slip drawn at random is marked other than Rs 1,
\begin{align}
&=1-p_X\brak{0}\\
&= p_X(1) + p_X(2)
\end{align}
Using Bayes theorem,
\begin{align}
&= p_Y\brak{0} \times \pr{Y=0 | X=1} + p_Y\brak{1} \times \pr{Y=1|X=2}\\
&=\frac{1}{3} \times \frac{6}{25} + \frac{2}{3} \times \frac{5}{50}\\
&=\frac{11}{75}
\end{align}

\newpage

%\tableofcontents

\bigskip

\renewcommand{\thefigure}{\theenumi}
\renewcommand{\thetable}{\theenumi}
%\renewcommand{\theequation}{\theenumi}

%\begin{abstract}
%%\boldmath
%In this letter, an algorithm for evaluating the exact analytical bit error rate  (BER)  for the piecewise linear (PL) combiner for  multiple relays is presented. Previous results were available only for upto three relays. The algorithm is unique in the sense that  the actual mathematical expressions, that are prohibitively large, need not be explicitly obtained. The diversity gain due to multiple relays is shown through plots of the analytical BER, well supported by simulations. 
%
%\end{abstract}
% IEEEtran.cls defaults to using nonbold math in the Abstract.
% This preserves the distinction between vectors and scalars. However,
% if the journal you are submitting to favors bold math in the abstract,
% then you can use LaTeX's standard command \boldmath at the very start
% of the abstract to achieve this. Many IEEE journals frown on math
% in the abstract anyway.

% Note that keywords are not normally used for peerreview papers.
%\begin{IEEEkeywords}
%Cooperative diversity, decode and forward, piecewise linear
%\end{IEEEkeywords}



% For peer review papers, you can put extra information on the cover
% page as needed:
% \ifCLASSOPTIONpeerreview
% \begin{center} \bfseries EDICS Category: 3-BBND \end{center}
% \fi
%
% For peerreview papers, this IEEEtran command inserts a page break and
% creates the second title. It will be ignored for other modes.
%\IEEEpeerreviewmaketitle




\item The probability that a student will pass his examination is 0.73, the probability of
the student getting a compartment is 0.13, and the probability that the student will
either pass or get compartment is 0.96. State True or False.\\
\solution
%\begin{table}[H]
	\centering
\begin{tabular}{|c|c|c|}
\hline
Random variable &Value &Definition\\ \hline
\multirow{3}{*}{X} &0 &Slips of Rs 1\\
&1 &Slips of Rs 5\\
&2 &Slips of Rs 13\\ \hline
\multirow{2}{*}{Y} &0 &Box A\\
&1 &Box B\\\hline
\end{tabular}
\caption{}
\label{tab:Distribution}
\end{table}
See \tabref{tab:Distribution}.
\begin{align}
p_{Y}\brak{k}= \begin{cases} 
      \frac{1}{3} & {k=0} \\
      \frac{2}{3 }& {k=1} 
   \end{cases}
   \\
p_{Y|X}\brak{0|0} = \frac{19}{25}\, 
p_{Y|X}\brak{0|1} = \frac{6}{25}\,
p_{Y|X}\brak{1|0} = \frac{45}{50}\,
p_{Y|X}\brak{1|2} = \frac{5}{50}
\end{align}
The desired probability is the probability that a slip drawn at random is marked other than Rs 1,
\begin{align}
&=1-p_X\brak{0}\\
&= p_X(1) + p_X(2)
\end{align}
Using Bayes theorem,
\begin{align}
&= p_Y\brak{0} \times \pr{Y=0 | X=1} + p_Y\brak{1} \times \pr{Y=1|X=2}\\
&=\frac{1}{3} \times \frac{6}{25} + \frac{2}{3} \times \frac{5}{50}\\
&=\frac{11}{75}
\end{align}

\newpage

%\tableofcontents

\bigskip

\renewcommand{\thefigure}{\theenumi}
\renewcommand{\thetable}{\theenumi}
%\renewcommand{\theequation}{\theenumi}

%\begin{abstract}
%%\boldmath
%In this letter, an algorithm for evaluating the exact analytical bit error rate  (BER)  for the piecewise linear (PL) combiner for  multiple relays is presented. Previous results were available only for upto three relays. The algorithm is unique in the sense that  the actual mathematical expressions, that are prohibitively large, need not be explicitly obtained. The diversity gain due to multiple relays is shown through plots of the analytical BER, well supported by simulations. 
%
%\end{abstract}
% IEEEtran.cls defaults to using nonbold math in the Abstract.
% This preserves the distinction between vectors and scalars. However,
% if the journal you are submitting to favors bold math in the abstract,
% then you can use LaTeX's standard command \boldmath at the very start
% of the abstract to achieve this. Many IEEE journals frown on math
% in the abstract anyway.

% Note that keywords are not normally used for peerreview papers.
%\begin{IEEEkeywords}
%Cooperative diversity, decode and forward, piecewise linear
%\end{IEEEkeywords}



% For peer review papers, you can put extra information on the cover
% page as needed:
% \ifCLASSOPTIONpeerreview
% \begin{center} \bfseries EDICS Category: 3-BBND \end{center}
% \fi
%
% For peerreview papers, this IEEEtran command inserts a page break and
% creates the second title. It will be ignored for other modes.
%\IEEEpeerreviewmaketitle




\item A card is selected from a pack of 52 cards\\
\begin{enumerate}[label=(\alph*)]
\item How many points are there in the sample space?
\item Calculate the probability that the cards is an ace of spades.
\item Calculate the probability that the card is (i) an ace (ii)black card.\\
\end{enumerate}
%\input{ncert/11/16/3/4_1/Prob_4.tex}
\item In a non-leap year, the probability of having 53 tuesdays or 53 wednesdays is\\
\solution
%A non-leap year has a total of 365 days, and a week has 7 days.\\
So it can be expressed as 
\begin{align}
365\text{days} &=52\times 7+1 \text{day}
\end{align}
$\implies$ 52 tuesdays or wednesdays\\
Random variable X denotes the days of a week
\begin{align}
p_X\brak{k}&=\frac{1}{7}; \quad \brak{1<k<7}
\end{align}
So the probability of extra day being tuesday or wednesday is
\begin{align}
p_X\brak{3}+p_X\brak{4}&=\frac{1}{7}+\frac{1}{7}=\frac{2}{7}
\end{align}



\item There are 1000 sealed envelopes in a box, 10 of them contain a cash prize of
Rs 100 each, 100 of them contain a cash prize of Rs 50 each and 200 of them
contain a cash prize of Rs 10 each and rest do not contain any cash prize. If they
are well shuffled and an envelope is picked up out, what is the probability that it
contains no cash prize?\\
\solution
%\begin{table}[H]
	\centering
\begin{tabular}{|c|c|c|}
\hline
Random variable &Value &Definition\\ \hline
\multirow{3}{*}{X} &0 &Slips of Rs 1\\
&1 &Slips of Rs 5\\
&2 &Slips of Rs 13\\ \hline
\multirow{2}{*}{Y} &0 &Box A\\
&1 &Box B\\\hline
\end{tabular}
\caption{}
\label{tab:Distribution}
\end{table}
See \tabref{tab:Distribution}.
\begin{align}
p_{Y}\brak{k}= \begin{cases} 
      \frac{1}{3} & {k=0} \\
      \frac{2}{3 }& {k=1} 
   \end{cases}
   \\
p_{Y|X}\brak{0|0} = \frac{19}{25}\, 
p_{Y|X}\brak{0|1} = \frac{6}{25}\,
p_{Y|X}\brak{1|0} = \frac{45}{50}\,
p_{Y|X}\brak{1|2} = \frac{5}{50}
\end{align}
The desired probability is the probability that a slip drawn at random is marked other than Rs 1,
\begin{align}
&=1-p_X\brak{0}\\
&= p_X(1) + p_X(2)
\end{align}
Using Bayes theorem,
\begin{align}
&= p_Y\brak{0} \times \pr{Y=0 | X=1} + p_Y\brak{1} \times \pr{Y=1|X=2}\\
&=\frac{1}{3} \times \frac{6}{25} + \frac{2}{3} \times \frac{5}{50}\\
&=\frac{11}{75}
\end{align}

\newpage

%\tableofcontents

\bigskip

\renewcommand{\thefigure}{\theenumi}
\renewcommand{\thetable}{\theenumi}
%\renewcommand{\theequation}{\theenumi}

%\begin{abstract}
%%\boldmath
%In this letter, an algorithm for evaluating the exact analytical bit error rate  (BER)  for the piecewise linear (PL) combiner for  multiple relays is presented. Previous results were available only for upto three relays. The algorithm is unique in the sense that  the actual mathematical expressions, that are prohibitively large, need not be explicitly obtained. The diversity gain due to multiple relays is shown through plots of the analytical BER, well supported by simulations. 
%
%\end{abstract}
% IEEEtran.cls defaults to using nonbold math in the Abstract.
% This preserves the distinction between vectors and scalars. However,
% if the journal you are submitting to favors bold math in the abstract,
% then you can use LaTeX's standard command \boldmath at the very start
% of the abstract to achieve this. Many IEEE journals frown on math
% in the abstract anyway.

% Note that keywords are not normally used for peerreview papers.
%\begin{IEEEkeywords}
%Cooperative diversity, decode and forward, piecewise linear
%\end{IEEEkeywords}



% For peer review papers, you can put extra information on the cover
% page as needed:
% \ifCLASSOPTIONpeerreview
% \begin{center} \bfseries EDICS Category: 3-BBND \end{center}
% \fi
%
% For peerreview papers, this IEEEtran command inserts a page break and
% creates the second title. It will be ignored for other modes.
%\IEEEpeerreviewmaketitle




\item 
A die is thrown and a card is selected at random from a deck of 52 playing cards. The probability of getting an even number on the die and a spade card.\\
\solution
%\begin{table}[H]
	\centering
\begin{tabular}{|c|c|c|}
\hline
Random variable &Value &Definition\\ \hline
\multirow{3}{*}{X} &0 &Slips of Rs 1\\
&1 &Slips of Rs 5\\
&2 &Slips of Rs 13\\ \hline
\multirow{2}{*}{Y} &0 &Box A\\
&1 &Box B\\\hline
\end{tabular}
\caption{}
\label{tab:Distribution}
\end{table}
See \tabref{tab:Distribution}.
\begin{align}
p_{Y}\brak{k}= \begin{cases} 
      \frac{1}{3} & {k=0} \\
      \frac{2}{3 }& {k=1} 
   \end{cases}
   \\
p_{Y|X}\brak{0|0} = \frac{19}{25}\, 
p_{Y|X}\brak{0|1} = \frac{6}{25}\,
p_{Y|X}\brak{1|0} = \frac{45}{50}\,
p_{Y|X}\brak{1|2} = \frac{5}{50}
\end{align}
The desired probability is the probability that a slip drawn at random is marked other than Rs 1,
\begin{align}
&=1-p_X\brak{0}\\
&= p_X(1) + p_X(2)
\end{align}
Using Bayes theorem,
\begin{align}
&= p_Y\brak{0} \times \pr{Y=0 | X=1} + p_Y\brak{1} \times \pr{Y=1|X=2}\\
&=\frac{1}{3} \times \frac{6}{25} + \frac{2}{3} \times \frac{5}{50}\\
&=\frac{11}{75}
\end{align}

\newpage

%\tableofcontents

\bigskip

\renewcommand{\thefigure}{\theenumi}
\renewcommand{\thetable}{\theenumi}
%\renewcommand{\theequation}{\theenumi}

%\begin{abstract}
%%\boldmath
%In this letter, an algorithm for evaluating the exact analytical bit error rate  (BER)  for the piecewise linear (PL) combiner for  multiple relays is presented. Previous results were available only for upto three relays. The algorithm is unique in the sense that  the actual mathematical expressions, that are prohibitively large, need not be explicitly obtained. The diversity gain due to multiple relays is shown through plots of the analytical BER, well supported by simulations. 
%
%\end{abstract}
% IEEEtran.cls defaults to using nonbold math in the Abstract.
% This preserves the distinction between vectors and scalars. However,
% if the journal you are submitting to favors bold math in the abstract,
% then you can use LaTeX's standard command \boldmath at the very start
% of the abstract to achieve this. Many IEEE journals frown on math
% in the abstract anyway.

% Note that keywords are not normally used for peerreview papers.
%\begin{IEEEkeywords}
%Cooperative diversity, decode and forward, piecewise linear
%\end{IEEEkeywords}



% For peer review papers, you can put extra information on the cover
% page as needed:
% \ifCLASSOPTIONpeerreview
% \begin{center} \bfseries EDICS Category: 3-BBND \end{center}
% \fi
%
% For peerreview papers, this IEEEtran command inserts a page break and
% creates the second title. It will be ignored for other modes.
%\IEEEpeerreviewmaketitle




\item
If 4-digit numbers greater than 5,000 are randomly formed from the digits 0, 1, 3, 5, and 7, what is the probability of forming a number divisible by 5 when:
\begin{enumerate}
    \item The digits are repeated?
    \item The repetition of digits is not allowed?
\end{enumerate}
\solution
%\begin{table}[H]
	\centering
\begin{tabular}{|c|c|c|}
\hline
Random variable &Value &Definition\\ \hline
\multirow{3}{*}{X} &0 &Slips of Rs 1\\
&1 &Slips of Rs 5\\
&2 &Slips of Rs 13\\ \hline
\multirow{2}{*}{Y} &0 &Box A\\
&1 &Box B\\\hline
\end{tabular}
\caption{}
\label{tab:Distribution}
\end{table}
See \tabref{tab:Distribution}.
\begin{align}
p_{Y}\brak{k}= \begin{cases} 
      \frac{1}{3} & {k=0} \\
      \frac{2}{3 }& {k=1} 
   \end{cases}
   \\
p_{Y|X}\brak{0|0} = \frac{19}{25}\, 
p_{Y|X}\brak{0|1} = \frac{6}{25}\,
p_{Y|X}\brak{1|0} = \frac{45}{50}\,
p_{Y|X}\brak{1|2} = \frac{5}{50}
\end{align}
The desired probability is the probability that a slip drawn at random is marked other than Rs 1,
\begin{align}
&=1-p_X\brak{0}\\
&= p_X(1) + p_X(2)
\end{align}
Using Bayes theorem,
\begin{align}
&= p_Y\brak{0} \times \pr{Y=0 | X=1} + p_Y\brak{1} \times \pr{Y=1|X=2}\\
&=\frac{1}{3} \times \frac{6}{25} + \frac{2}{3} \times \frac{5}{50}\\
&=\frac{11}{75}
\end{align}

\newpage

%\tableofcontents

\bigskip

\renewcommand{\thefigure}{\theenumi}
\renewcommand{\thetable}{\theenumi}
%\renewcommand{\theequation}{\theenumi}

%\begin{abstract}
%%\boldmath
%In this letter, an algorithm for evaluating the exact analytical bit error rate  (BER)  for the piecewise linear (PL) combiner for  multiple relays is presented. Previous results were available only for upto three relays. The algorithm is unique in the sense that  the actual mathematical expressions, that are prohibitively large, need not be explicitly obtained. The diversity gain due to multiple relays is shown through plots of the analytical BER, well supported by simulations. 
%
%\end{abstract}
% IEEEtran.cls defaults to using nonbold math in the Abstract.
% This preserves the distinction between vectors and scalars. However,
% if the journal you are submitting to favors bold math in the abstract,
% then you can use LaTeX's standard command \boldmath at the very start
% of the abstract to achieve this. Many IEEE journals frown on math
% in the abstract anyway.

% Note that keywords are not normally used for peerreview papers.
%\begin{IEEEkeywords}
%Cooperative diversity, decode and forward, piecewise linear
%\end{IEEEkeywords}



% For peer review papers, you can put extra information on the cover
% page as needed:
% \ifCLASSOPTIONpeerreview
% \begin{center} \bfseries EDICS Category: 3-BBND \end{center}
% \fi
%
% For peerreview papers, this IEEEtran command inserts a page break and
% creates the second title. It will be ignored for other modes.
%\IEEEpeerreviewmaketitle




\item Consider the probability space $\brak{\Omega, \mathcal{G}, P}$ where $\Omega = [0,2]$ and $\mathcal{G} = \cbrak{\phi, \Omega, [0,1], (1,2]}$. Let $X$ and $Y$ be two functions on $\Omega$ defined as
\begin{align*}
    X(\omega) = 
    \begin{cases}
        1 & \text{if }\omega \in [0, 1]\\
        2 & \text{if }\omega \in (1, 2]
    \end{cases}
\end{align*}
and
\begin{align*}
    Y(\omega) = 
    \begin{cases}
        2 & \text{if }\omega \in [0, 1.5]\\
        3 & \text{if }\omega \in (1.5, 2].
    \end{cases}
\end{align*}
Then which one of the following statements is true?
\begin{enumerate}
    \item [(A)] $X$ is a random variable with respect to $\mathcal{G}$, but $Y$ is not a random variable with respect to $\mathcal{G}$.
    \item [(B)] $Y$ is a random variable with respect to $\mathcal{G}$, but $X$ is not a random variable with respect to $\mathcal{G}$.
    \item [(C)] Neither $X$ nor $Y$ is a random variable with respect to $\mathcal{G}$.
    \item [(D)] Both $X$ and $Y$ are random variables with respect to $\mathcal{G}$.
\end{enumerate} \hfill (GATE ST 2023)\\
\solution
%\begin{table}[H]
	\centering
\begin{tabular}{|c|c|c|}
\hline
Random variable &Value &Definition\\ \hline
\multirow{3}{*}{X} &0 &Slips of Rs 1\\
&1 &Slips of Rs 5\\
&2 &Slips of Rs 13\\ \hline
\multirow{2}{*}{Y} &0 &Box A\\
&1 &Box B\\\hline
\end{tabular}
\caption{}
\label{tab:Distribution}
\end{table}
See \tabref{tab:Distribution}.
\begin{align}
p_{Y}\brak{k}= \begin{cases} 
      \frac{1}{3} & {k=0} \\
      \frac{2}{3 }& {k=1} 
   \end{cases}
   \\
p_{Y|X}\brak{0|0} = \frac{19}{25}\, 
p_{Y|X}\brak{0|1} = \frac{6}{25}\,
p_{Y|X}\brak{1|0} = \frac{45}{50}\,
p_{Y|X}\brak{1|2} = \frac{5}{50}
\end{align}
The desired probability is the probability that a slip drawn at random is marked other than Rs 1,
\begin{align}
&=1-p_X\brak{0}\\
&= p_X(1) + p_X(2)
\end{align}
Using Bayes theorem,
\begin{align}
&= p_Y\brak{0} \times \pr{Y=0 | X=1} + p_Y\brak{1} \times \pr{Y=1|X=2}\\
&=\frac{1}{3} \times \frac{6}{25} + \frac{2}{3} \times \frac{5}{50}\\
&=\frac{11}{75}
\end{align}

\newpage

%\tableofcontents

\bigskip

\renewcommand{\thefigure}{\theenumi}
\renewcommand{\thetable}{\theenumi}
%\renewcommand{\theequation}{\theenumi}

%\begin{abstract}
%%\boldmath
%In this letter, an algorithm for evaluating the exact analytical bit error rate  (BER)  for the piecewise linear (PL) combiner for  multiple relays is presented. Previous results were available only for upto three relays. The algorithm is unique in the sense that  the actual mathematical expressions, that are prohibitively large, need not be explicitly obtained. The diversity gain due to multiple relays is shown through plots of the analytical BER, well supported by simulations. 
%
%\end{abstract}
% IEEEtran.cls defaults to using nonbold math in the Abstract.
% This preserves the distinction between vectors and scalars. However,
% if the journal you are submitting to favors bold math in the abstract,
% then you can use LaTeX's standard command \boldmath at the very start
% of the abstract to achieve this. Many IEEE journals frown on math
% in the abstract anyway.

% Note that keywords are not normally used for peerreview papers.
%\begin{IEEEkeywords}
%Cooperative diversity, decode and forward, piecewise linear
%\end{IEEEkeywords}



% For peer review papers, you can put extra information on the cover
% page as needed:
% \ifCLASSOPTIONpeerreview
% \begin{center} \bfseries EDICS Category: 3-BBND \end{center}
% \fi
%
% For peerreview papers, this IEEEtran command inserts a page break and
% creates the second title. It will be ignored for other modes.
%\IEEEpeerreviewmaketitle




	\item  A die is loaded in such a way that each odd number is twice as likely to occur as
each even number. Find $P(G)$, where $G$ is the event that a number greater than
3 occurs on a single roll of the die.
\\
\solution
		%\begin{table}[H]
	\centering
\begin{tabular}{|c|c|c|}
\hline
Random variable &Value &Definition\\ \hline
\multirow{3}{*}{X} &0 &Slips of Rs 1\\
&1 &Slips of Rs 5\\
&2 &Slips of Rs 13\\ \hline
\multirow{2}{*}{Y} &0 &Box A\\
&1 &Box B\\\hline
\end{tabular}
\caption{}
\label{tab:Distribution}
\end{table}
See \tabref{tab:Distribution}.
\begin{align}
p_{Y}\brak{k}= \begin{cases} 
      \frac{1}{3} & {k=0} \\
      \frac{2}{3 }& {k=1} 
   \end{cases}
   \\
p_{Y|X}\brak{0|0} = \frac{19}{25}\, 
p_{Y|X}\brak{0|1} = \frac{6}{25}\,
p_{Y|X}\brak{1|0} = \frac{45}{50}\,
p_{Y|X}\brak{1|2} = \frac{5}{50}
\end{align}
The desired probability is the probability that a slip drawn at random is marked other than Rs 1,
\begin{align}
&=1-p_X\brak{0}\\
&= p_X(1) + p_X(2)
\end{align}
Using Bayes theorem,
\begin{align}
&= p_Y\brak{0} \times \pr{Y=0 | X=1} + p_Y\brak{1} \times \pr{Y=1|X=2}\\
&=\frac{1}{3} \times \frac{6}{25} + \frac{2}{3} \times \frac{5}{50}\\
&=\frac{11}{75}
\end{align}

\newpage

%\tableofcontents

\bigskip

\renewcommand{\thefigure}{\theenumi}
\renewcommand{\thetable}{\theenumi}
%\renewcommand{\theequation}{\theenumi}

%\begin{abstract}
%%\boldmath
%In this letter, an algorithm for evaluating the exact analytical bit error rate  (BER)  for the piecewise linear (PL) combiner for  multiple relays is presented. Previous results were available only for upto three relays. The algorithm is unique in the sense that  the actual mathematical expressions, that are prohibitively large, need not be explicitly obtained. The diversity gain due to multiple relays is shown through plots of the analytical BER, well supported by simulations. 
%
%\end{abstract}
% IEEEtran.cls defaults to using nonbold math in the Abstract.
% This preserves the distinction between vectors and scalars. However,
% if the journal you are submitting to favors bold math in the abstract,
% then you can use LaTeX's standard command \boldmath at the very start
% of the abstract to achieve this. Many IEEE journals frown on math
% in the abstract anyway.

% Note that keywords are not normally used for peerreview papers.
%\begin{IEEEkeywords}
%Cooperative diversity, decode and forward, piecewise linear
%\end{IEEEkeywords}



% For peer review papers, you can put extra information on the cover
% page as needed:
% \ifCLASSOPTIONpeerreview
% \begin{center} \bfseries EDICS Category: 3-BBND \end{center}
% \fi
%
% For peerreview papers, this IEEEtran command inserts a page break and
% creates the second title. It will be ignored for other modes.
%\IEEEpeerreviewmaketitle




	\item All the jacks, queens and kings are removed from a deck of 52 playing cards. The remaining cards are well shuffled and then one card is drawn at random. Giving ace a value 1 similar value for other cards, find the probability that the card has a value 
		\begin{enumerate}
			\item 7
			\item greater than 7
			\item less than 7
		\end{enumerate}
		%Number of cards left after removing all jacks, queens and kings 
\begin{align}
N	= 52 - 4\times 3
	= 40
\end{align}
%\begin{table}[H]
%\def\arraystretch{1.2}
%\begin{tabular}{|c|c|c|}
%\hline
%	\textbf{Parameter} &\textbf{Value} &\textbf{Description}\\ \hline
%	$X$ &1-10 &Represents the value of the card picked \\ \hline
%\end{tabular}
%\end{table}
Let $1 \le X \le 10$ be the value of the card picked.  Then,
\begin{align}
	p_X(k) &= \Pr(X=k)\ \forall\ 1 \leq k \leq 10\\
	&= \frac{4\times 1}{40}\\
	&= \frac{1}{10}\\
	\therefore p_X(k) &= 
	\begin{cases}
		\frac{1}{10} & 1 \leq k \leq 10\\
		0 & \text{otherwise}
	\end{cases}
\end{align}
and
\begin{align}
	F_{X}(k) &= \sum_{m=0}^{k}p_{X}(m) \quad 1 \leq k \leq 10\\
	&= \frac{k}{10}\\
	\therefore F_{X}(k) &= 
	\begin{cases}
		0 & k \leq 0\\
		\frac{k}{10} & 1\leq k \leq 10\\
		1 & k > 10 
	\end{cases}
\end{align}
\begin{enumerate}
	\item Probability that card has value equal to 7 is
		\begin{align}
			 p_{X}(7)
			= \frac{1}{10}
		\end{align}
	\item Probability that card has value greater than 7 is
		\begin{align}
			1 - F_X(7)
			&= 1 - \frac{7}{10}
			\\
			&= \frac{3}{10}
		\end{align}
	\item Probability that card has value less than 7 is
		\begin{align}
			 F_{X}(6)
			=\frac{6}{10}
		\end{align}
\end{enumerate}

  \item A Lot consists of 48 mobile phones of which 42 are good, 3 have only minor defects and 3 have major defects.Varnika will buy a phone if it is good but the trader will only buy a mobile if it has no major defects. One phone is selected at random from the lot. What is the probability that it is
\begin{enumerate}
	\item acceptable to Varnika?
            \item acceptable to the trader?
\end{enumerate}
\solution
	%\begin{table}[H]
	\centering
\begin{tabular}{|c|c|c|}
\hline
Random variable &Value &Definition\\ \hline
\multirow{3}{*}{X} &0 &Slips of Rs 1\\
&1 &Slips of Rs 5\\
&2 &Slips of Rs 13\\ \hline
\multirow{2}{*}{Y} &0 &Box A\\
&1 &Box B\\\hline
\end{tabular}
\caption{}
\label{tab:Distribution}
\end{table}
See \tabref{tab:Distribution}.
\begin{align}
p_{Y}\brak{k}= \begin{cases} 
      \frac{1}{3} & {k=0} \\
      \frac{2}{3 }& {k=1} 
   \end{cases}
   \\
p_{Y|X}\brak{0|0} = \frac{19}{25}\, 
p_{Y|X}\brak{0|1} = \frac{6}{25}\,
p_{Y|X}\brak{1|0} = \frac{45}{50}\,
p_{Y|X}\brak{1|2} = \frac{5}{50}
\end{align}
The desired probability is the probability that a slip drawn at random is marked other than Rs 1,
\begin{align}
&=1-p_X\brak{0}\\
&= p_X(1) + p_X(2)
\end{align}
Using Bayes theorem,
\begin{align}
&= p_Y\brak{0} \times \pr{Y=0 | X=1} + p_Y\brak{1} \times \pr{Y=1|X=2}\\
&=\frac{1}{3} \times \frac{6}{25} + \frac{2}{3} \times \frac{5}{50}\\
&=\frac{11}{75}
\end{align}

\newpage

%\tableofcontents

\bigskip

\renewcommand{\thefigure}{\theenumi}
\renewcommand{\thetable}{\theenumi}
%\renewcommand{\theequation}{\theenumi}

%\begin{abstract}
%%\boldmath
%In this letter, an algorithm for evaluating the exact analytical bit error rate  (BER)  for the piecewise linear (PL) combiner for  multiple relays is presented. Previous results were available only for upto three relays. The algorithm is unique in the sense that  the actual mathematical expressions, that are prohibitively large, need not be explicitly obtained. The diversity gain due to multiple relays is shown through plots of the analytical BER, well supported by simulations. 
%
%\end{abstract}
% IEEEtran.cls defaults to using nonbold math in the Abstract.
% This preserves the distinction between vectors and scalars. However,
% if the journal you are submitting to favors bold math in the abstract,
% then you can use LaTeX's standard command \boldmath at the very start
% of the abstract to achieve this. Many IEEE journals frown on math
% in the abstract anyway.

% Note that keywords are not normally used for peerreview papers.
%\begin{IEEEkeywords}
%Cooperative diversity, decode and forward, piecewise linear
%\end{IEEEkeywords}



% For peer review papers, you can put extra information on the cover
% page as needed:
% \ifCLASSOPTIONpeerreview
% \begin{center} \bfseries EDICS Category: 3-BBND \end{center}
% \fi
%
% For peerreview papers, this IEEEtran command inserts a page break and
% creates the second title. It will be ignored for other modes.
%\IEEEpeerreviewmaketitle




 \item A student says that if you throw a die, it will show up 1 or not 1. Therefore, the probability of getting 1 and the probability of getting 'not 1' each is equal to $\frac{1}{2}$. Is this correct? Give reasons.\\
 \solution
        %\begin{table}[H]
	\centering
\begin{tabular}{|c|c|c|}
\hline
Random variable &Value &Definition\\ \hline
\multirow{3}{*}{X} &0 &Slips of Rs 1\\
&1 &Slips of Rs 5\\
&2 &Slips of Rs 13\\ \hline
\multirow{2}{*}{Y} &0 &Box A\\
&1 &Box B\\\hline
\end{tabular}
\caption{}
\label{tab:Distribution}
\end{table}
See \tabref{tab:Distribution}.
\begin{align}
p_{Y}\brak{k}= \begin{cases} 
      \frac{1}{3} & {k=0} \\
      \frac{2}{3 }& {k=1} 
   \end{cases}
   \\
p_{Y|X}\brak{0|0} = \frac{19}{25}\, 
p_{Y|X}\brak{0|1} = \frac{6}{25}\,
p_{Y|X}\brak{1|0} = \frac{45}{50}\,
p_{Y|X}\brak{1|2} = \frac{5}{50}
\end{align}
The desired probability is the probability that a slip drawn at random is marked other than Rs 1,
\begin{align}
&=1-p_X\brak{0}\\
&= p_X(1) + p_X(2)
\end{align}
Using Bayes theorem,
\begin{align}
&= p_Y\brak{0} \times \pr{Y=0 | X=1} + p_Y\brak{1} \times \pr{Y=1|X=2}\\
&=\frac{1}{3} \times \frac{6}{25} + \frac{2}{3} \times \frac{5}{50}\\
&=\frac{11}{75}
\end{align}

\newpage

%\tableofcontents

\bigskip

\renewcommand{\thefigure}{\theenumi}
\renewcommand{\thetable}{\theenumi}
%\renewcommand{\theequation}{\theenumi}

%\begin{abstract}
%%\boldmath
%In this letter, an algorithm for evaluating the exact analytical bit error rate  (BER)  for the piecewise linear (PL) combiner for  multiple relays is presented. Previous results were available only for upto three relays. The algorithm is unique in the sense that  the actual mathematical expressions, that are prohibitively large, need not be explicitly obtained. The diversity gain due to multiple relays is shown through plots of the analytical BER, well supported by simulations. 
%
%\end{abstract}
% IEEEtran.cls defaults to using nonbold math in the Abstract.
% This preserves the distinction between vectors and scalars. However,
% if the journal you are submitting to favors bold math in the abstract,
% then you can use LaTeX's standard command \boldmath at the very start
% of the abstract to achieve this. Many IEEE journals frown on math
% in the abstract anyway.

% Note that keywords are not normally used for peerreview papers.
%\begin{IEEEkeywords}
%Cooperative diversity, decode and forward, piecewise linear
%\end{IEEEkeywords}



% For peer review papers, you can put extra information on the cover
% page as needed:
% \ifCLASSOPTIONpeerreview
% \begin{center} \bfseries EDICS Category: 3-BBND \end{center}
% \fi
%
% For peerreview papers, this IEEEtran command inserts a page break and
% creates the second title. It will be ignored for other modes.
%\IEEEpeerreviewmaketitle




   \item Four candidates A, B, C, D have ap-
plied for the assignment to coach a school cricket
team. If A is twice as likely to be selected as B, and
B and C are given about the same chance of being
selected, while C is twice as likely to be selected
as D, what are the probabilities that
\begin{enumerate}
\item C will be selected?
\item A will not be selected?
\end{enumerate}
	%\begin{table}[H]
	\centering
\begin{tabular}{|c|c|c|}
\hline
Random variable &Value &Definition\\ \hline
\multirow{3}{*}{X} &0 &Slips of Rs 1\\
&1 &Slips of Rs 5\\
&2 &Slips of Rs 13\\ \hline
\multirow{2}{*}{Y} &0 &Box A\\
&1 &Box B\\\hline
\end{tabular}
\caption{}
\label{tab:Distribution}
\end{table}
See \tabref{tab:Distribution}.
\begin{align}
p_{Y}\brak{k}= \begin{cases} 
      \frac{1}{3} & {k=0} \\
      \frac{2}{3 }& {k=1} 
   \end{cases}
   \\
p_{Y|X}\brak{0|0} = \frac{19}{25}\, 
p_{Y|X}\brak{0|1} = \frac{6}{25}\,
p_{Y|X}\brak{1|0} = \frac{45}{50}\,
p_{Y|X}\brak{1|2} = \frac{5}{50}
\end{align}
The desired probability is the probability that a slip drawn at random is marked other than Rs 1,
\begin{align}
&=1-p_X\brak{0}\\
&= p_X(1) + p_X(2)
\end{align}
Using Bayes theorem,
\begin{align}
&= p_Y\brak{0} \times \pr{Y=0 | X=1} + p_Y\brak{1} \times \pr{Y=1|X=2}\\
&=\frac{1}{3} \times \frac{6}{25} + \frac{2}{3} \times \frac{5}{50}\\
&=\frac{11}{75}
\end{align}

\newpage

%\tableofcontents

\bigskip

\renewcommand{\thefigure}{\theenumi}
\renewcommand{\thetable}{\theenumi}
%\renewcommand{\theequation}{\theenumi}

%\begin{abstract}
%%\boldmath
%In this letter, an algorithm for evaluating the exact analytical bit error rate  (BER)  for the piecewise linear (PL) combiner for  multiple relays is presented. Previous results were available only for upto three relays. The algorithm is unique in the sense that  the actual mathematical expressions, that are prohibitively large, need not be explicitly obtained. The diversity gain due to multiple relays is shown through plots of the analytical BER, well supported by simulations. 
%
%\end{abstract}
% IEEEtran.cls defaults to using nonbold math in the Abstract.
% This preserves the distinction between vectors and scalars. However,
% if the journal you are submitting to favors bold math in the abstract,
% then you can use LaTeX's standard command \boldmath at the very start
% of the abstract to achieve this. Many IEEE journals frown on math
% in the abstract anyway.

% Note that keywords are not normally used for peerreview papers.
%\begin{IEEEkeywords}
%Cooperative diversity, decode and forward, piecewise linear
%\end{IEEEkeywords}



% For peer review papers, you can put extra information on the cover
% page as needed:
% \ifCLASSOPTIONpeerreview
% \begin{center} \bfseries EDICS Category: 3-BBND \end{center}
% \fi
%
% For peerreview papers, this IEEEtran command inserts a page break and
% creates the second title. It will be ignored for other modes.
%\IEEEpeerreviewmaketitle




 \item A bag contain 24 balls of which $x$ balls are red, $2x$ are white and $3x$ are blue. A ball is selected at random, What is the probability that it is
\begin{enumerate}[label=\alph*)]
\item not red ?
\item white ?
\end{enumerate}
%\begin{table}[H]
	\centering
\begin{tabular}{|c|c|c|}
\hline
Random variable &Value &Definition\\ \hline
\multirow{3}{*}{X} &0 &Slips of Rs 1\\
&1 &Slips of Rs 5\\
&2 &Slips of Rs 13\\ \hline
\multirow{2}{*}{Y} &0 &Box A\\
&1 &Box B\\\hline
\end{tabular}
\caption{}
\label{tab:Distribution}
\end{table}
See \tabref{tab:Distribution}.
\begin{align}
p_{Y}\brak{k}= \begin{cases} 
      \frac{1}{3} & {k=0} \\
      \frac{2}{3 }& {k=1} 
   \end{cases}
   \\
p_{Y|X}\brak{0|0} = \frac{19}{25}\, 
p_{Y|X}\brak{0|1} = \frac{6}{25}\,
p_{Y|X}\brak{1|0} = \frac{45}{50}\,
p_{Y|X}\brak{1|2} = \frac{5}{50}
\end{align}
The desired probability is the probability that a slip drawn at random is marked other than Rs 1,
\begin{align}
&=1-p_X\brak{0}\\
&= p_X(1) + p_X(2)
\end{align}
Using Bayes theorem,
\begin{align}
&= p_Y\brak{0} \times \pr{Y=0 | X=1} + p_Y\brak{1} \times \pr{Y=1|X=2}\\
&=\frac{1}{3} \times \frac{6}{25} + \frac{2}{3} \times \frac{5}{50}\\
&=\frac{11}{75}
\end{align}

\newpage

%\tableofcontents

\bigskip

\renewcommand{\thefigure}{\theenumi}
\renewcommand{\thetable}{\theenumi}
%\renewcommand{\theequation}{\theenumi}

%\begin{abstract}
%%\boldmath
%In this letter, an algorithm for evaluating the exact analytical bit error rate  (BER)  for the piecewise linear (PL) combiner for  multiple relays is presented. Previous results were available only for upto three relays. The algorithm is unique in the sense that  the actual mathematical expressions, that are prohibitively large, need not be explicitly obtained. The diversity gain due to multiple relays is shown through plots of the analytical BER, well supported by simulations. 
%
%\end{abstract}
% IEEEtran.cls defaults to using nonbold math in the Abstract.
% This preserves the distinction between vectors and scalars. However,
% if the journal you are submitting to favors bold math in the abstract,
% then you can use LaTeX's standard command \boldmath at the very start
% of the abstract to achieve this. Many IEEE journals frown on math
% in the abstract anyway.

% Note that keywords are not normally used for peerreview papers.
%\begin{IEEEkeywords}
%Cooperative diversity, decode and forward, piecewise linear
%\end{IEEEkeywords}



% For peer review papers, you can put extra information on the cover
% page as needed:
% \ifCLASSOPTIONpeerreview
% \begin{center} \bfseries EDICS Category: 3-BBND \end{center}
% \fi
%
% For peerreview papers, this IEEEtran command inserts a page break and
% creates the second title. It will be ignored for other modes.
%\IEEEpeerreviewmaketitle




If the letters of the word ASSASSINATION are arranged at random. Find the Probability that
\begin{enumerate}[label=(\alph*)]
\item Four $S's$ come consecutively in the word
\item Two  $I's$ and two $N's$ come together
\item All $A's$ are not coming together
\item No two $A's$ are coming together
\end{enumerate}
%\begin{table}[H]
	\centering
\begin{tabular}{|c|c|c|}
\hline
Random variable &Value &Definition\\ \hline
\multirow{3}{*}{X} &0 &Slips of Rs 1\\
&1 &Slips of Rs 5\\
&2 &Slips of Rs 13\\ \hline
\multirow{2}{*}{Y} &0 &Box A\\
&1 &Box B\\\hline
\end{tabular}
\caption{}
\label{tab:Distribution}
\end{table}
See \tabref{tab:Distribution}.
\begin{align}
p_{Y}\brak{k}= \begin{cases} 
      \frac{1}{3} & {k=0} \\
      \frac{2}{3 }& {k=1} 
   \end{cases}
   \\
p_{Y|X}\brak{0|0} = \frac{19}{25}\, 
p_{Y|X}\brak{0|1} = \frac{6}{25}\,
p_{Y|X}\brak{1|0} = \frac{45}{50}\,
p_{Y|X}\brak{1|2} = \frac{5}{50}
\end{align}
The desired probability is the probability that a slip drawn at random is marked other than Rs 1,
\begin{align}
&=1-p_X\brak{0}\\
&= p_X(1) + p_X(2)
\end{align}
Using Bayes theorem,
\begin{align}
&= p_Y\brak{0} \times \pr{Y=0 | X=1} + p_Y\brak{1} \times \pr{Y=1|X=2}\\
&=\frac{1}{3} \times \frac{6}{25} + \frac{2}{3} \times \frac{5}{50}\\
&=\frac{11}{75}
\end{align}

\newpage

%\tableofcontents

\bigskip

\renewcommand{\thefigure}{\theenumi}
\renewcommand{\thetable}{\theenumi}
%\renewcommand{\theequation}{\theenumi}

%\begin{abstract}
%%\boldmath
%In this letter, an algorithm for evaluating the exact analytical bit error rate  (BER)  for the piecewise linear (PL) combiner for  multiple relays is presented. Previous results were available only for upto three relays. The algorithm is unique in the sense that  the actual mathematical expressions, that are prohibitively large, need not be explicitly obtained. The diversity gain due to multiple relays is shown through plots of the analytical BER, well supported by simulations. 
%
%\end{abstract}
% IEEEtran.cls defaults to using nonbold math in the Abstract.
% This preserves the distinction between vectors and scalars. However,
% if the journal you are submitting to favors bold math in the abstract,
% then you can use LaTeX's standard command \boldmath at the very start
% of the abstract to achieve this. Many IEEE journals frown on math
% in the abstract anyway.

% Note that keywords are not normally used for peerreview papers.
%\begin{IEEEkeywords}
%Cooperative diversity, decode and forward, piecewise linear
%\end{IEEEkeywords}



% For peer review papers, you can put extra information on the cover
% page as needed:
% \ifCLASSOPTIONpeerreview
% \begin{center} \bfseries EDICS Category: 3-BBND \end{center}
% \fi
%
% For peerreview papers, this IEEEtran command inserts a page break and
% creates the second title. It will be ignored for other modes.
%\IEEEpeerreviewmaketitle




	\item One urn contains two black balls (labelled B1 and B2) and one white ball. A
	second urn contains one black ball and two white balls (labelled W1 and W2).
	Suppose the following experiment is performed. One of the two urns is chosen
	at random. Next a ball is randomly chosen from the urn. Then a second ball is
	chosen at random from the same urn without replacing the first ball.
	
	\begin{enumerate}
	\item What is the probability that two black balls are chosen?
	
	\item What is the probability that two balls of opposite colour are chosen?
	\end{enumerate}
	\solution
	%\begin{align}
    \label{eq:12.13.6.18.1}
	\because	\pr{A|B} &> \pr{A},\
\frac{\pr{AB}}{\pr{B}} > \pr{A}
\\
    \label{eq:12.13.6.18.2}
	\implies \pr{AB} &> \pr{A}\pr{B}
	\\
	\text{or, } \frac{\pr{AB}}{\pr{A}} &=\pr{B|A} > \pr{A}
\end{align}

\end{enumerate}

	\item A bag contains $5$ red balls and some blue balls. If the probability of drawing a blue ball is double that if a red ball, determine the number of blue balls in the bag. 
		\\
\solution
		%\begin{enumerate}[label=\thesection.\arabic*,ref=\thesection.\theenumi]
	\item One card is drawn from a well-shuffled deck of 52 cards. Find the probability of getting
\begin{enumerate}
\item A king of red colour 
\item A face card 
\item A red face card
\item The jack of hearts
\item A spade
\item The queen of diamonds

\end{enumerate}
\solution
		%\begin{table}[H]
	\centering
\begin{tabular}{|c|c|c|}
\hline
Random variable &Value &Definition\\ \hline
\multirow{3}{*}{X} &0 &Slips of Rs 1\\
&1 &Slips of Rs 5\\
&2 &Slips of Rs 13\\ \hline
\multirow{2}{*}{Y} &0 &Box A\\
&1 &Box B\\\hline
\end{tabular}
\caption{}
\label{tab:Distribution}
\end{table}
See \tabref{tab:Distribution}.
\begin{align}
p_{Y}\brak{k}= \begin{cases} 
      \frac{1}{3} & {k=0} \\
      \frac{2}{3 }& {k=1} 
   \end{cases}
   \\
p_{Y|X}\brak{0|0} = \frac{19}{25}\, 
p_{Y|X}\brak{0|1} = \frac{6}{25}\,
p_{Y|X}\brak{1|0} = \frac{45}{50}\,
p_{Y|X}\brak{1|2} = \frac{5}{50}
\end{align}
The desired probability is the probability that a slip drawn at random is marked other than Rs 1,
\begin{align}
&=1-p_X\brak{0}\\
&= p_X(1) + p_X(2)
\end{align}
Using Bayes theorem,
\begin{align}
&= p_Y\brak{0} \times \pr{Y=0 | X=1} + p_Y\brak{1} \times \pr{Y=1|X=2}\\
&=\frac{1}{3} \times \frac{6}{25} + \frac{2}{3} \times \frac{5}{50}\\
&=\frac{11}{75}
\end{align}

\newpage

%\tableofcontents

\bigskip

\renewcommand{\thefigure}{\theenumi}
\renewcommand{\thetable}{\theenumi}
%\renewcommand{\theequation}{\theenumi}

%\begin{abstract}
%%\boldmath
%In this letter, an algorithm for evaluating the exact analytical bit error rate  (BER)  for the piecewise linear (PL) combiner for  multiple relays is presented. Previous results were available only for upto three relays. The algorithm is unique in the sense that  the actual mathematical expressions, that are prohibitively large, need not be explicitly obtained. The diversity gain due to multiple relays is shown through plots of the analytical BER, well supported by simulations. 
%
%\end{abstract}
% IEEEtran.cls defaults to using nonbold math in the Abstract.
% This preserves the distinction between vectors and scalars. However,
% if the journal you are submitting to favors bold math in the abstract,
% then you can use LaTeX's standard command \boldmath at the very start
% of the abstract to achieve this. Many IEEE journals frown on math
% in the abstract anyway.

% Note that keywords are not normally used for peerreview papers.
%\begin{IEEEkeywords}
%Cooperative diversity, decode and forward, piecewise linear
%\end{IEEEkeywords}



% For peer review papers, you can put extra information on the cover
% page as needed:
% \ifCLASSOPTIONpeerreview
% \begin{center} \bfseries EDICS Category: 3-BBND \end{center}
% \fi
%
% For peerreview papers, this IEEEtran command inserts a page break and
% creates the second title. It will be ignored for other modes.
%\IEEEpeerreviewmaketitle




	\item Five cards—the ten, jack, queen, king and ace of diamonds, are well-shuffled with their face downwards. One card is then picked up at random.
\begin{enumerate}
\item
What is the probability that the card is the queen? 
\item
If the queen is drawn and put aside, what is the probability that the second card picked up is (a) an ace? (b) a queen?\\
\end{enumerate}
\solution
		%\begin{enumerate}[label=\thesection.\arabic*,ref=\thesection.\theenumi]
	\item One card is drawn from a well-shuffled deck of 52 cards. Find the probability of getting
\begin{enumerate}
\item A king of red colour 
\item A face card 
\item A red face card
\item The jack of hearts
\item A spade
\item The queen of diamonds

\end{enumerate}
\solution
		%\input{ncert/10/15/1/14/main.tex}
	\item Five cards—the ten, jack, queen, king and ace of diamonds, are well-shuffled with their face downwards. One card is then picked up at random.
\begin{enumerate}
\item
What is the probability that the card is the queen? 
\item
If the queen is drawn and put aside, what is the probability that the second card picked up is (a) an ace? (b) a queen?\\
\end{enumerate}
\solution
		%\input{ncert/10/15/1/15/defs.tex}
	\item A bag contains $5$ red balls and some blue balls. If the probability of drawing a blue ball is double that if a red ball, determine the number of blue balls in the bag. 
		\\
\solution
		%\input{ncert/10/15/2/3/defs.tex}
	\item A card is selected from a pack of 52 cards.
 \begin{enumerate}[label=(\alph*)] 
                 \item How many points are there in the sample space?
                 \item Calculate the probability that the card is an ace of spades.
                 \item Calculate the probability that the card is (i) an ace and (ii) black card.
 \end{enumerate}
\solution
		%\input{ncert/11/16/3/4/main.tex}
\item Four cards are drawn from a well-shuffled deck of 52 cards. What is the probability of obtaining 3 diamonds and one spade.
\\
\solution
		%\input{ncert/11/16/4/2/defs.tex}
\item In a certain lottery 10,000 tickets are sold and ten equal prizes are awarded. What is the probability of not getting a prize if you buy (a) one ticket (b) two tickets (c) 10 tickets ?	
\\
\solution
		%\input{ncert/11/16/4/4/defs.tex}
		%
\item 
Out of 100 students, two sections of 40 and 60 are formed. If you and your friend are among the 100 students, what is the probability that
\begin{enumerate}
\item you both enter the same section?
\item you both enter the different sections?
\end{enumerate}
\solution
		%\input{ncert/11/16/4/5/defs.tex}
	\item 
The number lock of a suitcase has 4 wheels each labelled with ten digits i.e. from 0 to 9.The lock opens with a sequence of four digits with no repeats.What is the probability of a person getting the right sequence to open the suitcase.
\\
\solution
		%\input{ncert/11/16/4/10/defs.tex}
		%
\item 
Two cards are drawn at random and without replacement from a pack of 52 playing cards. Find the probability that both the cards are black.
\\
\solution
		%\input{ncert/12/13/2/2/defs.tex}
		\item A box of oranges is inspected by examining three randomly selected oranges drawn without replacement. If all the three oranges are good, the box is approved for sale, otherwise, it is rejected. Find the probability that a box containing 15 oranges out of which 12 are good and 3 are bad ones will be approved for sale.
		\label{ncert/12/13/2/3/defs.tex}
		\item Two balls are drawn at random with replacement from a box containing 10 black and 8 red balls. Find the probability that
		\label{ncert/12/13/2/12}
\begin{enumerate}
\item both balls are red.
\item first ball is black and second is red.
\item one of them is black and other is red.
\end{enumerate}

\item In a hostel, 60\% of the students read Hindi newspaper, 40\% read English newspaper and 20\% read both Hindi and English newspapers. A student is selected at random.
		\label{ncert/12/13/2/15}
\begin{enumerate}
\item Find the probability that she reads neither Hindi nor English newspapers.
\item If she reads Hindi newspaper, find the probability that she reads English newspaper.
\item If she reads English newspaper, find the probability that she reads Hindi newspaper.\\
\end{enumerate}
\item The probability of obtaining an even prime number on each die, when a pair of dice is rolled is 
\begin{enumerate}
    \item $0$ 
    
    \item $\frac{1}{3}$ 
    
    \item $\frac{1}{12}$ 
    
    \item $\frac{1}{36}$ 
\end{enumerate}
\solution
		%\input{ncert/12/13/2/17/defs.tex}
	\item A bag contains 4 red and 4 black balls, another bag contains 2 red and 6 black balls. One of the two bags is selected at random and a ball is drawn from the bag which is found to be red. Find the probability that the ball is drawn from the first bag.
\\
\solution
		%\input{ncert/12/13/3/2/main.tex}
  \item
  Cards with numbers 2 to 101 are placed in a box. A card is selected at random.Find the probability that the card has
\begin{enumerate}[label=(\roman*)]
	\item an even number 
	\item a square number
\end{enumerate}
\solution
%\input{exemplar/10/13/3/32/main.tex}
\item
The king, queen and jack of clubs are removed from a deck of 52 playing cards and then well shuffled. Now one card is drawn at random from the remaining cards.  Determine the probability that the card is
\begin{enumerate}[label=(\roman*)]
\item a club
\item 10 of hearts
\end{enumerate}
\solution
%\input{exemplar/10/13/3/29/main.tex}
\item A team of medical students doing their internship have to assist during surgeries
at a city hospital. The probabilities of surgeries rated as very complex, complex,
routine, simple or very simple are respectively, 0.15, 0.20, 0.31, 0.26, .08. Find
the probabilities that a particular surgery will be rated
\begin{enumerate}
	\item complex or very complex;
	\item neither very complex nor very simple;
	\item routine or complex
	\item routine or simple
\end{enumerate}
\solution
%\input{exemplar/11/16/3/8(1)/main.tex}
\item A card is selected from a pack of 52 cards.
\begin{enumerate}[label=(\alph*)]
    \item How many points are there in the sample space?
    \item Calculate the probability that the card is an ace of spades.
    \item Calculate the probability that the card is (i) an ace and (ii) black card.
\end{enumerate}
\solution
%\input{exemplar/11/16/3/4/main2.tex}
\item The probability that a non leap year selected at random will contain 53 sundays.
\\
\solution
%\input{exemplar/10/13/1/19/main.tex}
\item One of the four persons John, Rita, Aslam or Gurpreet will be promoted next
month. Consequently the sample space consists of four elementary outcomes
S = {John promoted, Rita promoted, Aslam promoted, Gurpreet promoted}
You are told that the chances of John’s promotion is same as that of Gurpreet,
Rita’s chances of promotion are twice as likely as Johns. Aslam’s chances are
four times that of John.
\begin{enumerate}
	\item Determine
	\begin{enumerate}
		\item P (John promoted)
		\item P (Rita promoted)
		\item P (Aslam promoted)
		\item P (Gurpreet promoted)
	\end{enumerate}
	\item If A = {John promoted or Gurpreet promoted}, find P (A).
\end{enumerate}
\solution
%\input{exemplar/11/16/3/10/main.tex}
\item A card is drawn from a deck of 52 cards. Find the probability of getting a king or a heart or a red card.\\
\solution
%\input{exemplar/11/16/3/15/main.tex}
\item The probability that a student will pass his examination is 0.73, the probability of
the student getting a compartment is 0.13, and the probability that the student will
either pass or get compartment is 0.96. State True or False.\\
\solution
%\input{exemplar/11/16/3/31/main.tex}
\item A card is selected from a pack of 52 cards\\
\begin{enumerate}[label=(\alph*)]
\item How many points are there in the sample space?
\item Calculate the probability that the cards is an ace of spades.
\item Calculate the probability that the card is (i) an ace (ii)black card.\\
\end{enumerate}
%\input{ncert/11/16/3/4_1/Prob_4.tex}
\item In a non-leap year, the probability of having 53 tuesdays or 53 wednesdays is\\
\solution
%\input{exemplar/11/16/3/18/main.tex}
\item There are 1000 sealed envelopes in a box, 10 of them contain a cash prize of
Rs 100 each, 100 of them contain a cash prize of Rs 50 each and 200 of them
contain a cash prize of Rs 10 each and rest do not contain any cash prize. If they
are well shuffled and an envelope is picked up out, what is the probability that it
contains no cash prize?\\
\solution
%\input{exemplar/10/13/3/34/main.tex}
\item 
A die is thrown and a card is selected at random from a deck of 52 playing cards. The probability of getting an even number on the die and a spade card.\\
\solution
%\input{exemplar/12/13/3/78/main.tex}
\item
If 4-digit numbers greater than 5,000 are randomly formed from the digits 0, 1, 3, 5, and 7, what is the probability of forming a number divisible by 5 when:
\begin{enumerate}
    \item The digits are repeated?
    \item The repetition of digits is not allowed?
\end{enumerate}
\solution
%\input{ncert/11/16/4/9/main.tex}
\item Consider the probability space $\brak{\Omega, \mathcal{G}, P}$ where $\Omega = [0,2]$ and $\mathcal{G} = \cbrak{\phi, \Omega, [0,1], (1,2]}$. Let $X$ and $Y$ be two functions on $\Omega$ defined as
\begin{align*}
    X(\omega) = 
    \begin{cases}
        1 & \text{if }\omega \in [0, 1]\\
        2 & \text{if }\omega \in (1, 2]
    \end{cases}
\end{align*}
and
\begin{align*}
    Y(\omega) = 
    \begin{cases}
        2 & \text{if }\omega \in [0, 1.5]\\
        3 & \text{if }\omega \in (1.5, 2].
    \end{cases}
\end{align*}
Then which one of the following statements is true?
\begin{enumerate}
    \item [(A)] $X$ is a random variable with respect to $\mathcal{G}$, but $Y$ is not a random variable with respect to $\mathcal{G}$.
    \item [(B)] $Y$ is a random variable with respect to $\mathcal{G}$, but $X$ is not a random variable with respect to $\mathcal{G}$.
    \item [(C)] Neither $X$ nor $Y$ is a random variable with respect to $\mathcal{G}$.
    \item [(D)] Both $X$ and $Y$ are random variables with respect to $\mathcal{G}$.
\end{enumerate} \hfill (GATE ST 2023)\\
\solution
%\input{gate/ST/2023/14/main.tex}
	\item  A die is loaded in such a way that each odd number is twice as likely to occur as
each even number. Find $P(G)$, where $G$ is the event that a number greater than
3 occurs on a single roll of the die.
\\
\solution
		%\input{exemplar/11/16/3/5/main.tex}
	\item All the jacks, queens and kings are removed from a deck of 52 playing cards. The remaining cards are well shuffled and then one card is drawn at random. Giving ace a value 1 similar value for other cards, find the probability that the card has a value 
		\begin{enumerate}
			\item 7
			\item greater than 7
			\item less than 7
		\end{enumerate}
		%\input{exemplar/10/13/3/30/main.tex}
  \item A Lot consists of 48 mobile phones of which 42 are good, 3 have only minor defects and 3 have major defects.Varnika will buy a phone if it is good but the trader will only buy a mobile if it has no major defects. One phone is selected at random from the lot. What is the probability that it is
\begin{enumerate}
	\item acceptable to Varnika?
            \item acceptable to the trader?
\end{enumerate}
\solution
	%\input{exemplar/10/13/3/40/main.tex}
 \item A student says that if you throw a die, it will show up 1 or not 1. Therefore, the probability of getting 1 and the probability of getting 'not 1' each is equal to $\frac{1}{2}$. Is this correct? Give reasons.\\
 \solution
        %\input{exemplar/10/13/2/9/main.tex}
   \item Four candidates A, B, C, D have ap-
plied for the assignment to coach a school cricket
team. If A is twice as likely to be selected as B, and
B and C are given about the same chance of being
selected, while C is twice as likely to be selected
as D, what are the probabilities that
\begin{enumerate}
\item C will be selected?
\item A will not be selected?
\end{enumerate}
	%\input{exemplar/11/16/3/9/main.tex}
 \item A bag contain 24 balls of which $x$ balls are red, $2x$ are white and $3x$ are blue. A ball is selected at random, What is the probability that it is
\begin{enumerate}[label=\alph*)]
\item not red ?
\item white ?
\end{enumerate}
%\input{exemplar/10/13/3/41/main.tex}
If the letters of the word ASSASSINATION are arranged at random. Find the Probability that
\begin{enumerate}[label=(\alph*)]
\item Four $S's$ come consecutively in the word
\item Two  $I's$ and two $N's$ come together
\item All $A's$ are not coming together
\item No two $A's$ are coming together
\end{enumerate}
%\input{exemplar/11/16/3/14/main.tex}
	\item One urn contains two black balls (labelled B1 and B2) and one white ball. A
	second urn contains one black ball and two white balls (labelled W1 and W2).
	Suppose the following experiment is performed. One of the two urns is chosen
	at random. Next a ball is randomly chosen from the urn. Then a second ball is
	chosen at random from the same urn without replacing the first ball.
	
	\begin{enumerate}
	\item What is the probability that two black balls are chosen?
	
	\item What is the probability that two balls of opposite colour are chosen?
	\end{enumerate}
	\solution
	%\input{exemplar/11/16/3/12/main1.tex}
\end{enumerate}

	\item A bag contains $5$ red balls and some blue balls. If the probability of drawing a blue ball is double that if a red ball, determine the number of blue balls in the bag. 
		\\
\solution
		%\begin{enumerate}[label=\thesection.\arabic*,ref=\thesection.\theenumi]
	\item One card is drawn from a well-shuffled deck of 52 cards. Find the probability of getting
\begin{enumerate}
\item A king of red colour 
\item A face card 
\item A red face card
\item The jack of hearts
\item A spade
\item The queen of diamonds

\end{enumerate}
\solution
		%\input{ncert/10/15/1/14/main.tex}
	\item Five cards—the ten, jack, queen, king and ace of diamonds, are well-shuffled with their face downwards. One card is then picked up at random.
\begin{enumerate}
\item
What is the probability that the card is the queen? 
\item
If the queen is drawn and put aside, what is the probability that the second card picked up is (a) an ace? (b) a queen?\\
\end{enumerate}
\solution
		%\input{ncert/10/15/1/15/defs.tex}
	\item A bag contains $5$ red balls and some blue balls. If the probability of drawing a blue ball is double that if a red ball, determine the number of blue balls in the bag. 
		\\
\solution
		%\input{ncert/10/15/2/3/defs.tex}
	\item A card is selected from a pack of 52 cards.
 \begin{enumerate}[label=(\alph*)] 
                 \item How many points are there in the sample space?
                 \item Calculate the probability that the card is an ace of spades.
                 \item Calculate the probability that the card is (i) an ace and (ii) black card.
 \end{enumerate}
\solution
		%\input{ncert/11/16/3/4/main.tex}
\item Four cards are drawn from a well-shuffled deck of 52 cards. What is the probability of obtaining 3 diamonds and one spade.
\\
\solution
		%\input{ncert/11/16/4/2/defs.tex}
\item In a certain lottery 10,000 tickets are sold and ten equal prizes are awarded. What is the probability of not getting a prize if you buy (a) one ticket (b) two tickets (c) 10 tickets ?	
\\
\solution
		%\input{ncert/11/16/4/4/defs.tex}
		%
\item 
Out of 100 students, two sections of 40 and 60 are formed. If you and your friend are among the 100 students, what is the probability that
\begin{enumerate}
\item you both enter the same section?
\item you both enter the different sections?
\end{enumerate}
\solution
		%\input{ncert/11/16/4/5/defs.tex}
	\item 
The number lock of a suitcase has 4 wheels each labelled with ten digits i.e. from 0 to 9.The lock opens with a sequence of four digits with no repeats.What is the probability of a person getting the right sequence to open the suitcase.
\\
\solution
		%\input{ncert/11/16/4/10/defs.tex}
		%
\item 
Two cards are drawn at random and without replacement from a pack of 52 playing cards. Find the probability that both the cards are black.
\\
\solution
		%\input{ncert/12/13/2/2/defs.tex}
		\item A box of oranges is inspected by examining three randomly selected oranges drawn without replacement. If all the three oranges are good, the box is approved for sale, otherwise, it is rejected. Find the probability that a box containing 15 oranges out of which 12 are good and 3 are bad ones will be approved for sale.
		\label{ncert/12/13/2/3/defs.tex}
		\item Two balls are drawn at random with replacement from a box containing 10 black and 8 red balls. Find the probability that
		\label{ncert/12/13/2/12}
\begin{enumerate}
\item both balls are red.
\item first ball is black and second is red.
\item one of them is black and other is red.
\end{enumerate}

\item In a hostel, 60\% of the students read Hindi newspaper, 40\% read English newspaper and 20\% read both Hindi and English newspapers. A student is selected at random.
		\label{ncert/12/13/2/15}
\begin{enumerate}
\item Find the probability that she reads neither Hindi nor English newspapers.
\item If she reads Hindi newspaper, find the probability that she reads English newspaper.
\item If she reads English newspaper, find the probability that she reads Hindi newspaper.\\
\end{enumerate}
\item The probability of obtaining an even prime number on each die, when a pair of dice is rolled is 
\begin{enumerate}
    \item $0$ 
    
    \item $\frac{1}{3}$ 
    
    \item $\frac{1}{12}$ 
    
    \item $\frac{1}{36}$ 
\end{enumerate}
\solution
		%\input{ncert/12/13/2/17/defs.tex}
	\item A bag contains 4 red and 4 black balls, another bag contains 2 red and 6 black balls. One of the two bags is selected at random and a ball is drawn from the bag which is found to be red. Find the probability that the ball is drawn from the first bag.
\\
\solution
		%\input{ncert/12/13/3/2/main.tex}
  \item
  Cards with numbers 2 to 101 are placed in a box. A card is selected at random.Find the probability that the card has
\begin{enumerate}[label=(\roman*)]
	\item an even number 
	\item a square number
\end{enumerate}
\solution
%\input{exemplar/10/13/3/32/main.tex}
\item
The king, queen and jack of clubs are removed from a deck of 52 playing cards and then well shuffled. Now one card is drawn at random from the remaining cards.  Determine the probability that the card is
\begin{enumerate}[label=(\roman*)]
\item a club
\item 10 of hearts
\end{enumerate}
\solution
%\input{exemplar/10/13/3/29/main.tex}
\item A team of medical students doing their internship have to assist during surgeries
at a city hospital. The probabilities of surgeries rated as very complex, complex,
routine, simple or very simple are respectively, 0.15, 0.20, 0.31, 0.26, .08. Find
the probabilities that a particular surgery will be rated
\begin{enumerate}
	\item complex or very complex;
	\item neither very complex nor very simple;
	\item routine or complex
	\item routine or simple
\end{enumerate}
\solution
%\input{exemplar/11/16/3/8(1)/main.tex}
\item A card is selected from a pack of 52 cards.
\begin{enumerate}[label=(\alph*)]
    \item How many points are there in the sample space?
    \item Calculate the probability that the card is an ace of spades.
    \item Calculate the probability that the card is (i) an ace and (ii) black card.
\end{enumerate}
\solution
%\input{exemplar/11/16/3/4/main2.tex}
\item The probability that a non leap year selected at random will contain 53 sundays.
\\
\solution
%\input{exemplar/10/13/1/19/main.tex}
\item One of the four persons John, Rita, Aslam or Gurpreet will be promoted next
month. Consequently the sample space consists of four elementary outcomes
S = {John promoted, Rita promoted, Aslam promoted, Gurpreet promoted}
You are told that the chances of John’s promotion is same as that of Gurpreet,
Rita’s chances of promotion are twice as likely as Johns. Aslam’s chances are
four times that of John.
\begin{enumerate}
	\item Determine
	\begin{enumerate}
		\item P (John promoted)
		\item P (Rita promoted)
		\item P (Aslam promoted)
		\item P (Gurpreet promoted)
	\end{enumerate}
	\item If A = {John promoted or Gurpreet promoted}, find P (A).
\end{enumerate}
\solution
%\input{exemplar/11/16/3/10/main.tex}
\item A card is drawn from a deck of 52 cards. Find the probability of getting a king or a heart or a red card.\\
\solution
%\input{exemplar/11/16/3/15/main.tex}
\item The probability that a student will pass his examination is 0.73, the probability of
the student getting a compartment is 0.13, and the probability that the student will
either pass or get compartment is 0.96. State True or False.\\
\solution
%\input{exemplar/11/16/3/31/main.tex}
\item A card is selected from a pack of 52 cards\\
\begin{enumerate}[label=(\alph*)]
\item How many points are there in the sample space?
\item Calculate the probability that the cards is an ace of spades.
\item Calculate the probability that the card is (i) an ace (ii)black card.\\
\end{enumerate}
%\input{ncert/11/16/3/4_1/Prob_4.tex}
\item In a non-leap year, the probability of having 53 tuesdays or 53 wednesdays is\\
\solution
%\input{exemplar/11/16/3/18/main.tex}
\item There are 1000 sealed envelopes in a box, 10 of them contain a cash prize of
Rs 100 each, 100 of them contain a cash prize of Rs 50 each and 200 of them
contain a cash prize of Rs 10 each and rest do not contain any cash prize. If they
are well shuffled and an envelope is picked up out, what is the probability that it
contains no cash prize?\\
\solution
%\input{exemplar/10/13/3/34/main.tex}
\item 
A die is thrown and a card is selected at random from a deck of 52 playing cards. The probability of getting an even number on the die and a spade card.\\
\solution
%\input{exemplar/12/13/3/78/main.tex}
\item
If 4-digit numbers greater than 5,000 are randomly formed from the digits 0, 1, 3, 5, and 7, what is the probability of forming a number divisible by 5 when:
\begin{enumerate}
    \item The digits are repeated?
    \item The repetition of digits is not allowed?
\end{enumerate}
\solution
%\input{ncert/11/16/4/9/main.tex}
\item Consider the probability space $\brak{\Omega, \mathcal{G}, P}$ where $\Omega = [0,2]$ and $\mathcal{G} = \cbrak{\phi, \Omega, [0,1], (1,2]}$. Let $X$ and $Y$ be two functions on $\Omega$ defined as
\begin{align*}
    X(\omega) = 
    \begin{cases}
        1 & \text{if }\omega \in [0, 1]\\
        2 & \text{if }\omega \in (1, 2]
    \end{cases}
\end{align*}
and
\begin{align*}
    Y(\omega) = 
    \begin{cases}
        2 & \text{if }\omega \in [0, 1.5]\\
        3 & \text{if }\omega \in (1.5, 2].
    \end{cases}
\end{align*}
Then which one of the following statements is true?
\begin{enumerate}
    \item [(A)] $X$ is a random variable with respect to $\mathcal{G}$, but $Y$ is not a random variable with respect to $\mathcal{G}$.
    \item [(B)] $Y$ is a random variable with respect to $\mathcal{G}$, but $X$ is not a random variable with respect to $\mathcal{G}$.
    \item [(C)] Neither $X$ nor $Y$ is a random variable with respect to $\mathcal{G}$.
    \item [(D)] Both $X$ and $Y$ are random variables with respect to $\mathcal{G}$.
\end{enumerate} \hfill (GATE ST 2023)\\
\solution
%\input{gate/ST/2023/14/main.tex}
	\item  A die is loaded in such a way that each odd number is twice as likely to occur as
each even number. Find $P(G)$, where $G$ is the event that a number greater than
3 occurs on a single roll of the die.
\\
\solution
		%\input{exemplar/11/16/3/5/main.tex}
	\item All the jacks, queens and kings are removed from a deck of 52 playing cards. The remaining cards are well shuffled and then one card is drawn at random. Giving ace a value 1 similar value for other cards, find the probability that the card has a value 
		\begin{enumerate}
			\item 7
			\item greater than 7
			\item less than 7
		\end{enumerate}
		%\input{exemplar/10/13/3/30/main.tex}
  \item A Lot consists of 48 mobile phones of which 42 are good, 3 have only minor defects and 3 have major defects.Varnika will buy a phone if it is good but the trader will only buy a mobile if it has no major defects. One phone is selected at random from the lot. What is the probability that it is
\begin{enumerate}
	\item acceptable to Varnika?
            \item acceptable to the trader?
\end{enumerate}
\solution
	%\input{exemplar/10/13/3/40/main.tex}
 \item A student says that if you throw a die, it will show up 1 or not 1. Therefore, the probability of getting 1 and the probability of getting 'not 1' each is equal to $\frac{1}{2}$. Is this correct? Give reasons.\\
 \solution
        %\input{exemplar/10/13/2/9/main.tex}
   \item Four candidates A, B, C, D have ap-
plied for the assignment to coach a school cricket
team. If A is twice as likely to be selected as B, and
B and C are given about the same chance of being
selected, while C is twice as likely to be selected
as D, what are the probabilities that
\begin{enumerate}
\item C will be selected?
\item A will not be selected?
\end{enumerate}
	%\input{exemplar/11/16/3/9/main.tex}
 \item A bag contain 24 balls of which $x$ balls are red, $2x$ are white and $3x$ are blue. A ball is selected at random, What is the probability that it is
\begin{enumerate}[label=\alph*)]
\item not red ?
\item white ?
\end{enumerate}
%\input{exemplar/10/13/3/41/main.tex}
If the letters of the word ASSASSINATION are arranged at random. Find the Probability that
\begin{enumerate}[label=(\alph*)]
\item Four $S's$ come consecutively in the word
\item Two  $I's$ and two $N's$ come together
\item All $A's$ are not coming together
\item No two $A's$ are coming together
\end{enumerate}
%\input{exemplar/11/16/3/14/main.tex}
	\item One urn contains two black balls (labelled B1 and B2) and one white ball. A
	second urn contains one black ball and two white balls (labelled W1 and W2).
	Suppose the following experiment is performed. One of the two urns is chosen
	at random. Next a ball is randomly chosen from the urn. Then a second ball is
	chosen at random from the same urn without replacing the first ball.
	
	\begin{enumerate}
	\item What is the probability that two black balls are chosen?
	
	\item What is the probability that two balls of opposite colour are chosen?
	\end{enumerate}
	\solution
	%\input{exemplar/11/16/3/12/main1.tex}
\end{enumerate}

	\item A card is selected from a pack of 52 cards.
 \begin{enumerate}[label=(\alph*)] 
                 \item How many points are there in the sample space?
                 \item Calculate the probability that the card is an ace of spades.
                 \item Calculate the probability that the card is (i) an ace and (ii) black card.
 \end{enumerate}
\solution
		%\begin{table}[H]
	\centering
\begin{tabular}{|c|c|c|}
\hline
Random variable &Value &Definition\\ \hline
\multirow{3}{*}{X} &0 &Slips of Rs 1\\
&1 &Slips of Rs 5\\
&2 &Slips of Rs 13\\ \hline
\multirow{2}{*}{Y} &0 &Box A\\
&1 &Box B\\\hline
\end{tabular}
\caption{}
\label{tab:Distribution}
\end{table}
See \tabref{tab:Distribution}.
\begin{align}
p_{Y}\brak{k}= \begin{cases} 
      \frac{1}{3} & {k=0} \\
      \frac{2}{3 }& {k=1} 
   \end{cases}
   \\
p_{Y|X}\brak{0|0} = \frac{19}{25}\, 
p_{Y|X}\brak{0|1} = \frac{6}{25}\,
p_{Y|X}\brak{1|0} = \frac{45}{50}\,
p_{Y|X}\brak{1|2} = \frac{5}{50}
\end{align}
The desired probability is the probability that a slip drawn at random is marked other than Rs 1,
\begin{align}
&=1-p_X\brak{0}\\
&= p_X(1) + p_X(2)
\end{align}
Using Bayes theorem,
\begin{align}
&= p_Y\brak{0} \times \pr{Y=0 | X=1} + p_Y\brak{1} \times \pr{Y=1|X=2}\\
&=\frac{1}{3} \times \frac{6}{25} + \frac{2}{3} \times \frac{5}{50}\\
&=\frac{11}{75}
\end{align}

\newpage

%\tableofcontents

\bigskip

\renewcommand{\thefigure}{\theenumi}
\renewcommand{\thetable}{\theenumi}
%\renewcommand{\theequation}{\theenumi}

%\begin{abstract}
%%\boldmath
%In this letter, an algorithm for evaluating the exact analytical bit error rate  (BER)  for the piecewise linear (PL) combiner for  multiple relays is presented. Previous results were available only for upto three relays. The algorithm is unique in the sense that  the actual mathematical expressions, that are prohibitively large, need not be explicitly obtained. The diversity gain due to multiple relays is shown through plots of the analytical BER, well supported by simulations. 
%
%\end{abstract}
% IEEEtran.cls defaults to using nonbold math in the Abstract.
% This preserves the distinction between vectors and scalars. However,
% if the journal you are submitting to favors bold math in the abstract,
% then you can use LaTeX's standard command \boldmath at the very start
% of the abstract to achieve this. Many IEEE journals frown on math
% in the abstract anyway.

% Note that keywords are not normally used for peerreview papers.
%\begin{IEEEkeywords}
%Cooperative diversity, decode and forward, piecewise linear
%\end{IEEEkeywords}



% For peer review papers, you can put extra information on the cover
% page as needed:
% \ifCLASSOPTIONpeerreview
% \begin{center} \bfseries EDICS Category: 3-BBND \end{center}
% \fi
%
% For peerreview papers, this IEEEtran command inserts a page break and
% creates the second title. It will be ignored for other modes.
%\IEEEpeerreviewmaketitle




\item Four cards are drawn from a well-shuffled deck of 52 cards. What is the probability of obtaining 3 diamonds and one spade.
\\
\solution
		%\begin{enumerate}[label=\thesection.\arabic*,ref=\thesection.\theenumi]
	\item One card is drawn from a well-shuffled deck of 52 cards. Find the probability of getting
\begin{enumerate}
\item A king of red colour 
\item A face card 
\item A red face card
\item The jack of hearts
\item A spade
\item The queen of diamonds

\end{enumerate}
\solution
		%\input{ncert/10/15/1/14/main.tex}
	\item Five cards—the ten, jack, queen, king and ace of diamonds, are well-shuffled with their face downwards. One card is then picked up at random.
\begin{enumerate}
\item
What is the probability that the card is the queen? 
\item
If the queen is drawn and put aside, what is the probability that the second card picked up is (a) an ace? (b) a queen?\\
\end{enumerate}
\solution
		%\input{ncert/10/15/1/15/defs.tex}
	\item A bag contains $5$ red balls and some blue balls. If the probability of drawing a blue ball is double that if a red ball, determine the number of blue balls in the bag. 
		\\
\solution
		%\input{ncert/10/15/2/3/defs.tex}
	\item A card is selected from a pack of 52 cards.
 \begin{enumerate}[label=(\alph*)] 
                 \item How many points are there in the sample space?
                 \item Calculate the probability that the card is an ace of spades.
                 \item Calculate the probability that the card is (i) an ace and (ii) black card.
 \end{enumerate}
\solution
		%\input{ncert/11/16/3/4/main.tex}
\item Four cards are drawn from a well-shuffled deck of 52 cards. What is the probability of obtaining 3 diamonds and one spade.
\\
\solution
		%\input{ncert/11/16/4/2/defs.tex}
\item In a certain lottery 10,000 tickets are sold and ten equal prizes are awarded. What is the probability of not getting a prize if you buy (a) one ticket (b) two tickets (c) 10 tickets ?	
\\
\solution
		%\input{ncert/11/16/4/4/defs.tex}
		%
\item 
Out of 100 students, two sections of 40 and 60 are formed. If you and your friend are among the 100 students, what is the probability that
\begin{enumerate}
\item you both enter the same section?
\item you both enter the different sections?
\end{enumerate}
\solution
		%\input{ncert/11/16/4/5/defs.tex}
	\item 
The number lock of a suitcase has 4 wheels each labelled with ten digits i.e. from 0 to 9.The lock opens with a sequence of four digits with no repeats.What is the probability of a person getting the right sequence to open the suitcase.
\\
\solution
		%\input{ncert/11/16/4/10/defs.tex}
		%
\item 
Two cards are drawn at random and without replacement from a pack of 52 playing cards. Find the probability that both the cards are black.
\\
\solution
		%\input{ncert/12/13/2/2/defs.tex}
		\item A box of oranges is inspected by examining three randomly selected oranges drawn without replacement. If all the three oranges are good, the box is approved for sale, otherwise, it is rejected. Find the probability that a box containing 15 oranges out of which 12 are good and 3 are bad ones will be approved for sale.
		\label{ncert/12/13/2/3/defs.tex}
		\item Two balls are drawn at random with replacement from a box containing 10 black and 8 red balls. Find the probability that
		\label{ncert/12/13/2/12}
\begin{enumerate}
\item both balls are red.
\item first ball is black and second is red.
\item one of them is black and other is red.
\end{enumerate}

\item In a hostel, 60\% of the students read Hindi newspaper, 40\% read English newspaper and 20\% read both Hindi and English newspapers. A student is selected at random.
		\label{ncert/12/13/2/15}
\begin{enumerate}
\item Find the probability that she reads neither Hindi nor English newspapers.
\item If she reads Hindi newspaper, find the probability that she reads English newspaper.
\item If she reads English newspaper, find the probability that she reads Hindi newspaper.\\
\end{enumerate}
\item The probability of obtaining an even prime number on each die, when a pair of dice is rolled is 
\begin{enumerate}
    \item $0$ 
    
    \item $\frac{1}{3}$ 
    
    \item $\frac{1}{12}$ 
    
    \item $\frac{1}{36}$ 
\end{enumerate}
\solution
		%\input{ncert/12/13/2/17/defs.tex}
	\item A bag contains 4 red and 4 black balls, another bag contains 2 red and 6 black balls. One of the two bags is selected at random and a ball is drawn from the bag which is found to be red. Find the probability that the ball is drawn from the first bag.
\\
\solution
		%\input{ncert/12/13/3/2/main.tex}
  \item
  Cards with numbers 2 to 101 are placed in a box. A card is selected at random.Find the probability that the card has
\begin{enumerate}[label=(\roman*)]
	\item an even number 
	\item a square number
\end{enumerate}
\solution
%\input{exemplar/10/13/3/32/main.tex}
\item
The king, queen and jack of clubs are removed from a deck of 52 playing cards and then well shuffled. Now one card is drawn at random from the remaining cards.  Determine the probability that the card is
\begin{enumerate}[label=(\roman*)]
\item a club
\item 10 of hearts
\end{enumerate}
\solution
%\input{exemplar/10/13/3/29/main.tex}
\item A team of medical students doing their internship have to assist during surgeries
at a city hospital. The probabilities of surgeries rated as very complex, complex,
routine, simple or very simple are respectively, 0.15, 0.20, 0.31, 0.26, .08. Find
the probabilities that a particular surgery will be rated
\begin{enumerate}
	\item complex or very complex;
	\item neither very complex nor very simple;
	\item routine or complex
	\item routine or simple
\end{enumerate}
\solution
%\input{exemplar/11/16/3/8(1)/main.tex}
\item A card is selected from a pack of 52 cards.
\begin{enumerate}[label=(\alph*)]
    \item How many points are there in the sample space?
    \item Calculate the probability that the card is an ace of spades.
    \item Calculate the probability that the card is (i) an ace and (ii) black card.
\end{enumerate}
\solution
%\input{exemplar/11/16/3/4/main2.tex}
\item The probability that a non leap year selected at random will contain 53 sundays.
\\
\solution
%\input{exemplar/10/13/1/19/main.tex}
\item One of the four persons John, Rita, Aslam or Gurpreet will be promoted next
month. Consequently the sample space consists of four elementary outcomes
S = {John promoted, Rita promoted, Aslam promoted, Gurpreet promoted}
You are told that the chances of John’s promotion is same as that of Gurpreet,
Rita’s chances of promotion are twice as likely as Johns. Aslam’s chances are
four times that of John.
\begin{enumerate}
	\item Determine
	\begin{enumerate}
		\item P (John promoted)
		\item P (Rita promoted)
		\item P (Aslam promoted)
		\item P (Gurpreet promoted)
	\end{enumerate}
	\item If A = {John promoted or Gurpreet promoted}, find P (A).
\end{enumerate}
\solution
%\input{exemplar/11/16/3/10/main.tex}
\item A card is drawn from a deck of 52 cards. Find the probability of getting a king or a heart or a red card.\\
\solution
%\input{exemplar/11/16/3/15/main.tex}
\item The probability that a student will pass his examination is 0.73, the probability of
the student getting a compartment is 0.13, and the probability that the student will
either pass or get compartment is 0.96. State True or False.\\
\solution
%\input{exemplar/11/16/3/31/main.tex}
\item A card is selected from a pack of 52 cards\\
\begin{enumerate}[label=(\alph*)]
\item How many points are there in the sample space?
\item Calculate the probability that the cards is an ace of spades.
\item Calculate the probability that the card is (i) an ace (ii)black card.\\
\end{enumerate}
%\input{ncert/11/16/3/4_1/Prob_4.tex}
\item In a non-leap year, the probability of having 53 tuesdays or 53 wednesdays is\\
\solution
%\input{exemplar/11/16/3/18/main.tex}
\item There are 1000 sealed envelopes in a box, 10 of them contain a cash prize of
Rs 100 each, 100 of them contain a cash prize of Rs 50 each and 200 of them
contain a cash prize of Rs 10 each and rest do not contain any cash prize. If they
are well shuffled and an envelope is picked up out, what is the probability that it
contains no cash prize?\\
\solution
%\input{exemplar/10/13/3/34/main.tex}
\item 
A die is thrown and a card is selected at random from a deck of 52 playing cards. The probability of getting an even number on the die and a spade card.\\
\solution
%\input{exemplar/12/13/3/78/main.tex}
\item
If 4-digit numbers greater than 5,000 are randomly formed from the digits 0, 1, 3, 5, and 7, what is the probability of forming a number divisible by 5 when:
\begin{enumerate}
    \item The digits are repeated?
    \item The repetition of digits is not allowed?
\end{enumerate}
\solution
%\input{ncert/11/16/4/9/main.tex}
\item Consider the probability space $\brak{\Omega, \mathcal{G}, P}$ where $\Omega = [0,2]$ and $\mathcal{G} = \cbrak{\phi, \Omega, [0,1], (1,2]}$. Let $X$ and $Y$ be two functions on $\Omega$ defined as
\begin{align*}
    X(\omega) = 
    \begin{cases}
        1 & \text{if }\omega \in [0, 1]\\
        2 & \text{if }\omega \in (1, 2]
    \end{cases}
\end{align*}
and
\begin{align*}
    Y(\omega) = 
    \begin{cases}
        2 & \text{if }\omega \in [0, 1.5]\\
        3 & \text{if }\omega \in (1.5, 2].
    \end{cases}
\end{align*}
Then which one of the following statements is true?
\begin{enumerate}
    \item [(A)] $X$ is a random variable with respect to $\mathcal{G}$, but $Y$ is not a random variable with respect to $\mathcal{G}$.
    \item [(B)] $Y$ is a random variable with respect to $\mathcal{G}$, but $X$ is not a random variable with respect to $\mathcal{G}$.
    \item [(C)] Neither $X$ nor $Y$ is a random variable with respect to $\mathcal{G}$.
    \item [(D)] Both $X$ and $Y$ are random variables with respect to $\mathcal{G}$.
\end{enumerate} \hfill (GATE ST 2023)\\
\solution
%\input{gate/ST/2023/14/main.tex}
	\item  A die is loaded in such a way that each odd number is twice as likely to occur as
each even number. Find $P(G)$, where $G$ is the event that a number greater than
3 occurs on a single roll of the die.
\\
\solution
		%\input{exemplar/11/16/3/5/main.tex}
	\item All the jacks, queens and kings are removed from a deck of 52 playing cards. The remaining cards are well shuffled and then one card is drawn at random. Giving ace a value 1 similar value for other cards, find the probability that the card has a value 
		\begin{enumerate}
			\item 7
			\item greater than 7
			\item less than 7
		\end{enumerate}
		%\input{exemplar/10/13/3/30/main.tex}
  \item A Lot consists of 48 mobile phones of which 42 are good, 3 have only minor defects and 3 have major defects.Varnika will buy a phone if it is good but the trader will only buy a mobile if it has no major defects. One phone is selected at random from the lot. What is the probability that it is
\begin{enumerate}
	\item acceptable to Varnika?
            \item acceptable to the trader?
\end{enumerate}
\solution
	%\input{exemplar/10/13/3/40/main.tex}
 \item A student says that if you throw a die, it will show up 1 or not 1. Therefore, the probability of getting 1 and the probability of getting 'not 1' each is equal to $\frac{1}{2}$. Is this correct? Give reasons.\\
 \solution
        %\input{exemplar/10/13/2/9/main.tex}
   \item Four candidates A, B, C, D have ap-
plied for the assignment to coach a school cricket
team. If A is twice as likely to be selected as B, and
B and C are given about the same chance of being
selected, while C is twice as likely to be selected
as D, what are the probabilities that
\begin{enumerate}
\item C will be selected?
\item A will not be selected?
\end{enumerate}
	%\input{exemplar/11/16/3/9/main.tex}
 \item A bag contain 24 balls of which $x$ balls are red, $2x$ are white and $3x$ are blue. A ball is selected at random, What is the probability that it is
\begin{enumerate}[label=\alph*)]
\item not red ?
\item white ?
\end{enumerate}
%\input{exemplar/10/13/3/41/main.tex}
If the letters of the word ASSASSINATION are arranged at random. Find the Probability that
\begin{enumerate}[label=(\alph*)]
\item Four $S's$ come consecutively in the word
\item Two  $I's$ and two $N's$ come together
\item All $A's$ are not coming together
\item No two $A's$ are coming together
\end{enumerate}
%\input{exemplar/11/16/3/14/main.tex}
	\item One urn contains two black balls (labelled B1 and B2) and one white ball. A
	second urn contains one black ball and two white balls (labelled W1 and W2).
	Suppose the following experiment is performed. One of the two urns is chosen
	at random. Next a ball is randomly chosen from the urn. Then a second ball is
	chosen at random from the same urn without replacing the first ball.
	
	\begin{enumerate}
	\item What is the probability that two black balls are chosen?
	
	\item What is the probability that two balls of opposite colour are chosen?
	\end{enumerate}
	\solution
	%\input{exemplar/11/16/3/12/main1.tex}
\end{enumerate}

\item In a certain lottery 10,000 tickets are sold and ten equal prizes are awarded. What is the probability of not getting a prize if you buy (a) one ticket (b) two tickets (c) 10 tickets ?	
\\
\solution
		%\begin{enumerate}[label=\thesection.\arabic*,ref=\thesection.\theenumi]
	\item One card is drawn from a well-shuffled deck of 52 cards. Find the probability of getting
\begin{enumerate}
\item A king of red colour 
\item A face card 
\item A red face card
\item The jack of hearts
\item A spade
\item The queen of diamonds

\end{enumerate}
\solution
		%\input{ncert/10/15/1/14/main.tex}
	\item Five cards—the ten, jack, queen, king and ace of diamonds, are well-shuffled with their face downwards. One card is then picked up at random.
\begin{enumerate}
\item
What is the probability that the card is the queen? 
\item
If the queen is drawn and put aside, what is the probability that the second card picked up is (a) an ace? (b) a queen?\\
\end{enumerate}
\solution
		%\input{ncert/10/15/1/15/defs.tex}
	\item A bag contains $5$ red balls and some blue balls. If the probability of drawing a blue ball is double that if a red ball, determine the number of blue balls in the bag. 
		\\
\solution
		%\input{ncert/10/15/2/3/defs.tex}
	\item A card is selected from a pack of 52 cards.
 \begin{enumerate}[label=(\alph*)] 
                 \item How many points are there in the sample space?
                 \item Calculate the probability that the card is an ace of spades.
                 \item Calculate the probability that the card is (i) an ace and (ii) black card.
 \end{enumerate}
\solution
		%\input{ncert/11/16/3/4/main.tex}
\item Four cards are drawn from a well-shuffled deck of 52 cards. What is the probability of obtaining 3 diamonds and one spade.
\\
\solution
		%\input{ncert/11/16/4/2/defs.tex}
\item In a certain lottery 10,000 tickets are sold and ten equal prizes are awarded. What is the probability of not getting a prize if you buy (a) one ticket (b) two tickets (c) 10 tickets ?	
\\
\solution
		%\input{ncert/11/16/4/4/defs.tex}
		%
\item 
Out of 100 students, two sections of 40 and 60 are formed. If you and your friend are among the 100 students, what is the probability that
\begin{enumerate}
\item you both enter the same section?
\item you both enter the different sections?
\end{enumerate}
\solution
		%\input{ncert/11/16/4/5/defs.tex}
	\item 
The number lock of a suitcase has 4 wheels each labelled with ten digits i.e. from 0 to 9.The lock opens with a sequence of four digits with no repeats.What is the probability of a person getting the right sequence to open the suitcase.
\\
\solution
		%\input{ncert/11/16/4/10/defs.tex}
		%
\item 
Two cards are drawn at random and without replacement from a pack of 52 playing cards. Find the probability that both the cards are black.
\\
\solution
		%\input{ncert/12/13/2/2/defs.tex}
		\item A box of oranges is inspected by examining three randomly selected oranges drawn without replacement. If all the three oranges are good, the box is approved for sale, otherwise, it is rejected. Find the probability that a box containing 15 oranges out of which 12 are good and 3 are bad ones will be approved for sale.
		\label{ncert/12/13/2/3/defs.tex}
		\item Two balls are drawn at random with replacement from a box containing 10 black and 8 red balls. Find the probability that
		\label{ncert/12/13/2/12}
\begin{enumerate}
\item both balls are red.
\item first ball is black and second is red.
\item one of them is black and other is red.
\end{enumerate}

\item In a hostel, 60\% of the students read Hindi newspaper, 40\% read English newspaper and 20\% read both Hindi and English newspapers. A student is selected at random.
		\label{ncert/12/13/2/15}
\begin{enumerate}
\item Find the probability that she reads neither Hindi nor English newspapers.
\item If she reads Hindi newspaper, find the probability that she reads English newspaper.
\item If she reads English newspaper, find the probability that she reads Hindi newspaper.\\
\end{enumerate}
\item The probability of obtaining an even prime number on each die, when a pair of dice is rolled is 
\begin{enumerate}
    \item $0$ 
    
    \item $\frac{1}{3}$ 
    
    \item $\frac{1}{12}$ 
    
    \item $\frac{1}{36}$ 
\end{enumerate}
\solution
		%\input{ncert/12/13/2/17/defs.tex}
	\item A bag contains 4 red and 4 black balls, another bag contains 2 red and 6 black balls. One of the two bags is selected at random and a ball is drawn from the bag which is found to be red. Find the probability that the ball is drawn from the first bag.
\\
\solution
		%\input{ncert/12/13/3/2/main.tex}
  \item
  Cards with numbers 2 to 101 are placed in a box. A card is selected at random.Find the probability that the card has
\begin{enumerate}[label=(\roman*)]
	\item an even number 
	\item a square number
\end{enumerate}
\solution
%\input{exemplar/10/13/3/32/main.tex}
\item
The king, queen and jack of clubs are removed from a deck of 52 playing cards and then well shuffled. Now one card is drawn at random from the remaining cards.  Determine the probability that the card is
\begin{enumerate}[label=(\roman*)]
\item a club
\item 10 of hearts
\end{enumerate}
\solution
%\input{exemplar/10/13/3/29/main.tex}
\item A team of medical students doing their internship have to assist during surgeries
at a city hospital. The probabilities of surgeries rated as very complex, complex,
routine, simple or very simple are respectively, 0.15, 0.20, 0.31, 0.26, .08. Find
the probabilities that a particular surgery will be rated
\begin{enumerate}
	\item complex or very complex;
	\item neither very complex nor very simple;
	\item routine or complex
	\item routine or simple
\end{enumerate}
\solution
%\input{exemplar/11/16/3/8(1)/main.tex}
\item A card is selected from a pack of 52 cards.
\begin{enumerate}[label=(\alph*)]
    \item How many points are there in the sample space?
    \item Calculate the probability that the card is an ace of spades.
    \item Calculate the probability that the card is (i) an ace and (ii) black card.
\end{enumerate}
\solution
%\input{exemplar/11/16/3/4/main2.tex}
\item The probability that a non leap year selected at random will contain 53 sundays.
\\
\solution
%\input{exemplar/10/13/1/19/main.tex}
\item One of the four persons John, Rita, Aslam or Gurpreet will be promoted next
month. Consequently the sample space consists of four elementary outcomes
S = {John promoted, Rita promoted, Aslam promoted, Gurpreet promoted}
You are told that the chances of John’s promotion is same as that of Gurpreet,
Rita’s chances of promotion are twice as likely as Johns. Aslam’s chances are
four times that of John.
\begin{enumerate}
	\item Determine
	\begin{enumerate}
		\item P (John promoted)
		\item P (Rita promoted)
		\item P (Aslam promoted)
		\item P (Gurpreet promoted)
	\end{enumerate}
	\item If A = {John promoted or Gurpreet promoted}, find P (A).
\end{enumerate}
\solution
%\input{exemplar/11/16/3/10/main.tex}
\item A card is drawn from a deck of 52 cards. Find the probability of getting a king or a heart or a red card.\\
\solution
%\input{exemplar/11/16/3/15/main.tex}
\item The probability that a student will pass his examination is 0.73, the probability of
the student getting a compartment is 0.13, and the probability that the student will
either pass or get compartment is 0.96. State True or False.\\
\solution
%\input{exemplar/11/16/3/31/main.tex}
\item A card is selected from a pack of 52 cards\\
\begin{enumerate}[label=(\alph*)]
\item How many points are there in the sample space?
\item Calculate the probability that the cards is an ace of spades.
\item Calculate the probability that the card is (i) an ace (ii)black card.\\
\end{enumerate}
%\input{ncert/11/16/3/4_1/Prob_4.tex}
\item In a non-leap year, the probability of having 53 tuesdays or 53 wednesdays is\\
\solution
%\input{exemplar/11/16/3/18/main.tex}
\item There are 1000 sealed envelopes in a box, 10 of them contain a cash prize of
Rs 100 each, 100 of them contain a cash prize of Rs 50 each and 200 of them
contain a cash prize of Rs 10 each and rest do not contain any cash prize. If they
are well shuffled and an envelope is picked up out, what is the probability that it
contains no cash prize?\\
\solution
%\input{exemplar/10/13/3/34/main.tex}
\item 
A die is thrown and a card is selected at random from a deck of 52 playing cards. The probability of getting an even number on the die and a spade card.\\
\solution
%\input{exemplar/12/13/3/78/main.tex}
\item
If 4-digit numbers greater than 5,000 are randomly formed from the digits 0, 1, 3, 5, and 7, what is the probability of forming a number divisible by 5 when:
\begin{enumerate}
    \item The digits are repeated?
    \item The repetition of digits is not allowed?
\end{enumerate}
\solution
%\input{ncert/11/16/4/9/main.tex}
\item Consider the probability space $\brak{\Omega, \mathcal{G}, P}$ where $\Omega = [0,2]$ and $\mathcal{G} = \cbrak{\phi, \Omega, [0,1], (1,2]}$. Let $X$ and $Y$ be two functions on $\Omega$ defined as
\begin{align*}
    X(\omega) = 
    \begin{cases}
        1 & \text{if }\omega \in [0, 1]\\
        2 & \text{if }\omega \in (1, 2]
    \end{cases}
\end{align*}
and
\begin{align*}
    Y(\omega) = 
    \begin{cases}
        2 & \text{if }\omega \in [0, 1.5]\\
        3 & \text{if }\omega \in (1.5, 2].
    \end{cases}
\end{align*}
Then which one of the following statements is true?
\begin{enumerate}
    \item [(A)] $X$ is a random variable with respect to $\mathcal{G}$, but $Y$ is not a random variable with respect to $\mathcal{G}$.
    \item [(B)] $Y$ is a random variable with respect to $\mathcal{G}$, but $X$ is not a random variable with respect to $\mathcal{G}$.
    \item [(C)] Neither $X$ nor $Y$ is a random variable with respect to $\mathcal{G}$.
    \item [(D)] Both $X$ and $Y$ are random variables with respect to $\mathcal{G}$.
\end{enumerate} \hfill (GATE ST 2023)\\
\solution
%\input{gate/ST/2023/14/main.tex}
	\item  A die is loaded in such a way that each odd number is twice as likely to occur as
each even number. Find $P(G)$, where $G$ is the event that a number greater than
3 occurs on a single roll of the die.
\\
\solution
		%\input{exemplar/11/16/3/5/main.tex}
	\item All the jacks, queens and kings are removed from a deck of 52 playing cards. The remaining cards are well shuffled and then one card is drawn at random. Giving ace a value 1 similar value for other cards, find the probability that the card has a value 
		\begin{enumerate}
			\item 7
			\item greater than 7
			\item less than 7
		\end{enumerate}
		%\input{exemplar/10/13/3/30/main.tex}
  \item A Lot consists of 48 mobile phones of which 42 are good, 3 have only minor defects and 3 have major defects.Varnika will buy a phone if it is good but the trader will only buy a mobile if it has no major defects. One phone is selected at random from the lot. What is the probability that it is
\begin{enumerate}
	\item acceptable to Varnika?
            \item acceptable to the trader?
\end{enumerate}
\solution
	%\input{exemplar/10/13/3/40/main.tex}
 \item A student says that if you throw a die, it will show up 1 or not 1. Therefore, the probability of getting 1 and the probability of getting 'not 1' each is equal to $\frac{1}{2}$. Is this correct? Give reasons.\\
 \solution
        %\input{exemplar/10/13/2/9/main.tex}
   \item Four candidates A, B, C, D have ap-
plied for the assignment to coach a school cricket
team. If A is twice as likely to be selected as B, and
B and C are given about the same chance of being
selected, while C is twice as likely to be selected
as D, what are the probabilities that
\begin{enumerate}
\item C will be selected?
\item A will not be selected?
\end{enumerate}
	%\input{exemplar/11/16/3/9/main.tex}
 \item A bag contain 24 balls of which $x$ balls are red, $2x$ are white and $3x$ are blue. A ball is selected at random, What is the probability that it is
\begin{enumerate}[label=\alph*)]
\item not red ?
\item white ?
\end{enumerate}
%\input{exemplar/10/13/3/41/main.tex}
If the letters of the word ASSASSINATION are arranged at random. Find the Probability that
\begin{enumerate}[label=(\alph*)]
\item Four $S's$ come consecutively in the word
\item Two  $I's$ and two $N's$ come together
\item All $A's$ are not coming together
\item No two $A's$ are coming together
\end{enumerate}
%\input{exemplar/11/16/3/14/main.tex}
	\item One urn contains two black balls (labelled B1 and B2) and one white ball. A
	second urn contains one black ball and two white balls (labelled W1 and W2).
	Suppose the following experiment is performed. One of the two urns is chosen
	at random. Next a ball is randomly chosen from the urn. Then a second ball is
	chosen at random from the same urn without replacing the first ball.
	
	\begin{enumerate}
	\item What is the probability that two black balls are chosen?
	
	\item What is the probability that two balls of opposite colour are chosen?
	\end{enumerate}
	\solution
	%\input{exemplar/11/16/3/12/main1.tex}
\end{enumerate}

		%
\item 
Out of 100 students, two sections of 40 and 60 are formed. If you and your friend are among the 100 students, what is the probability that
\begin{enumerate}
\item you both enter the same section?
\item you both enter the different sections?
\end{enumerate}
\solution
		%\begin{enumerate}[label=\thesection.\arabic*,ref=\thesection.\theenumi]
	\item One card is drawn from a well-shuffled deck of 52 cards. Find the probability of getting
\begin{enumerate}
\item A king of red colour 
\item A face card 
\item A red face card
\item The jack of hearts
\item A spade
\item The queen of diamonds

\end{enumerate}
\solution
		%\input{ncert/10/15/1/14/main.tex}
	\item Five cards—the ten, jack, queen, king and ace of diamonds, are well-shuffled with their face downwards. One card is then picked up at random.
\begin{enumerate}
\item
What is the probability that the card is the queen? 
\item
If the queen is drawn and put aside, what is the probability that the second card picked up is (a) an ace? (b) a queen?\\
\end{enumerate}
\solution
		%\input{ncert/10/15/1/15/defs.tex}
	\item A bag contains $5$ red balls and some blue balls. If the probability of drawing a blue ball is double that if a red ball, determine the number of blue balls in the bag. 
		\\
\solution
		%\input{ncert/10/15/2/3/defs.tex}
	\item A card is selected from a pack of 52 cards.
 \begin{enumerate}[label=(\alph*)] 
                 \item How many points are there in the sample space?
                 \item Calculate the probability that the card is an ace of spades.
                 \item Calculate the probability that the card is (i) an ace and (ii) black card.
 \end{enumerate}
\solution
		%\input{ncert/11/16/3/4/main.tex}
\item Four cards are drawn from a well-shuffled deck of 52 cards. What is the probability of obtaining 3 diamonds and one spade.
\\
\solution
		%\input{ncert/11/16/4/2/defs.tex}
\item In a certain lottery 10,000 tickets are sold and ten equal prizes are awarded. What is the probability of not getting a prize if you buy (a) one ticket (b) two tickets (c) 10 tickets ?	
\\
\solution
		%\input{ncert/11/16/4/4/defs.tex}
		%
\item 
Out of 100 students, two sections of 40 and 60 are formed. If you and your friend are among the 100 students, what is the probability that
\begin{enumerate}
\item you both enter the same section?
\item you both enter the different sections?
\end{enumerate}
\solution
		%\input{ncert/11/16/4/5/defs.tex}
	\item 
The number lock of a suitcase has 4 wheels each labelled with ten digits i.e. from 0 to 9.The lock opens with a sequence of four digits with no repeats.What is the probability of a person getting the right sequence to open the suitcase.
\\
\solution
		%\input{ncert/11/16/4/10/defs.tex}
		%
\item 
Two cards are drawn at random and without replacement from a pack of 52 playing cards. Find the probability that both the cards are black.
\\
\solution
		%\input{ncert/12/13/2/2/defs.tex}
		\item A box of oranges is inspected by examining three randomly selected oranges drawn without replacement. If all the three oranges are good, the box is approved for sale, otherwise, it is rejected. Find the probability that a box containing 15 oranges out of which 12 are good and 3 are bad ones will be approved for sale.
		\label{ncert/12/13/2/3/defs.tex}
		\item Two balls are drawn at random with replacement from a box containing 10 black and 8 red balls. Find the probability that
		\label{ncert/12/13/2/12}
\begin{enumerate}
\item both balls are red.
\item first ball is black and second is red.
\item one of them is black and other is red.
\end{enumerate}

\item In a hostel, 60\% of the students read Hindi newspaper, 40\% read English newspaper and 20\% read both Hindi and English newspapers. A student is selected at random.
		\label{ncert/12/13/2/15}
\begin{enumerate}
\item Find the probability that she reads neither Hindi nor English newspapers.
\item If she reads Hindi newspaper, find the probability that she reads English newspaper.
\item If she reads English newspaper, find the probability that she reads Hindi newspaper.\\
\end{enumerate}
\item The probability of obtaining an even prime number on each die, when a pair of dice is rolled is 
\begin{enumerate}
    \item $0$ 
    
    \item $\frac{1}{3}$ 
    
    \item $\frac{1}{12}$ 
    
    \item $\frac{1}{36}$ 
\end{enumerate}
\solution
		%\input{ncert/12/13/2/17/defs.tex}
	\item A bag contains 4 red and 4 black balls, another bag contains 2 red and 6 black balls. One of the two bags is selected at random and a ball is drawn from the bag which is found to be red. Find the probability that the ball is drawn from the first bag.
\\
\solution
		%\input{ncert/12/13/3/2/main.tex}
  \item
  Cards with numbers 2 to 101 are placed in a box. A card is selected at random.Find the probability that the card has
\begin{enumerate}[label=(\roman*)]
	\item an even number 
	\item a square number
\end{enumerate}
\solution
%\input{exemplar/10/13/3/32/main.tex}
\item
The king, queen and jack of clubs are removed from a deck of 52 playing cards and then well shuffled. Now one card is drawn at random from the remaining cards.  Determine the probability that the card is
\begin{enumerate}[label=(\roman*)]
\item a club
\item 10 of hearts
\end{enumerate}
\solution
%\input{exemplar/10/13/3/29/main.tex}
\item A team of medical students doing their internship have to assist during surgeries
at a city hospital. The probabilities of surgeries rated as very complex, complex,
routine, simple or very simple are respectively, 0.15, 0.20, 0.31, 0.26, .08. Find
the probabilities that a particular surgery will be rated
\begin{enumerate}
	\item complex or very complex;
	\item neither very complex nor very simple;
	\item routine or complex
	\item routine or simple
\end{enumerate}
\solution
%\input{exemplar/11/16/3/8(1)/main.tex}
\item A card is selected from a pack of 52 cards.
\begin{enumerate}[label=(\alph*)]
    \item How many points are there in the sample space?
    \item Calculate the probability that the card is an ace of spades.
    \item Calculate the probability that the card is (i) an ace and (ii) black card.
\end{enumerate}
\solution
%\input{exemplar/11/16/3/4/main2.tex}
\item The probability that a non leap year selected at random will contain 53 sundays.
\\
\solution
%\input{exemplar/10/13/1/19/main.tex}
\item One of the four persons John, Rita, Aslam or Gurpreet will be promoted next
month. Consequently the sample space consists of four elementary outcomes
S = {John promoted, Rita promoted, Aslam promoted, Gurpreet promoted}
You are told that the chances of John’s promotion is same as that of Gurpreet,
Rita’s chances of promotion are twice as likely as Johns. Aslam’s chances are
four times that of John.
\begin{enumerate}
	\item Determine
	\begin{enumerate}
		\item P (John promoted)
		\item P (Rita promoted)
		\item P (Aslam promoted)
		\item P (Gurpreet promoted)
	\end{enumerate}
	\item If A = {John promoted or Gurpreet promoted}, find P (A).
\end{enumerate}
\solution
%\input{exemplar/11/16/3/10/main.tex}
\item A card is drawn from a deck of 52 cards. Find the probability of getting a king or a heart or a red card.\\
\solution
%\input{exemplar/11/16/3/15/main.tex}
\item The probability that a student will pass his examination is 0.73, the probability of
the student getting a compartment is 0.13, and the probability that the student will
either pass or get compartment is 0.96. State True or False.\\
\solution
%\input{exemplar/11/16/3/31/main.tex}
\item A card is selected from a pack of 52 cards\\
\begin{enumerate}[label=(\alph*)]
\item How many points are there in the sample space?
\item Calculate the probability that the cards is an ace of spades.
\item Calculate the probability that the card is (i) an ace (ii)black card.\\
\end{enumerate}
%\input{ncert/11/16/3/4_1/Prob_4.tex}
\item In a non-leap year, the probability of having 53 tuesdays or 53 wednesdays is\\
\solution
%\input{exemplar/11/16/3/18/main.tex}
\item There are 1000 sealed envelopes in a box, 10 of them contain a cash prize of
Rs 100 each, 100 of them contain a cash prize of Rs 50 each and 200 of them
contain a cash prize of Rs 10 each and rest do not contain any cash prize. If they
are well shuffled and an envelope is picked up out, what is the probability that it
contains no cash prize?\\
\solution
%\input{exemplar/10/13/3/34/main.tex}
\item 
A die is thrown and a card is selected at random from a deck of 52 playing cards. The probability of getting an even number on the die and a spade card.\\
\solution
%\input{exemplar/12/13/3/78/main.tex}
\item
If 4-digit numbers greater than 5,000 are randomly formed from the digits 0, 1, 3, 5, and 7, what is the probability of forming a number divisible by 5 when:
\begin{enumerate}
    \item The digits are repeated?
    \item The repetition of digits is not allowed?
\end{enumerate}
\solution
%\input{ncert/11/16/4/9/main.tex}
\item Consider the probability space $\brak{\Omega, \mathcal{G}, P}$ where $\Omega = [0,2]$ and $\mathcal{G} = \cbrak{\phi, \Omega, [0,1], (1,2]}$. Let $X$ and $Y$ be two functions on $\Omega$ defined as
\begin{align*}
    X(\omega) = 
    \begin{cases}
        1 & \text{if }\omega \in [0, 1]\\
        2 & \text{if }\omega \in (1, 2]
    \end{cases}
\end{align*}
and
\begin{align*}
    Y(\omega) = 
    \begin{cases}
        2 & \text{if }\omega \in [0, 1.5]\\
        3 & \text{if }\omega \in (1.5, 2].
    \end{cases}
\end{align*}
Then which one of the following statements is true?
\begin{enumerate}
    \item [(A)] $X$ is a random variable with respect to $\mathcal{G}$, but $Y$ is not a random variable with respect to $\mathcal{G}$.
    \item [(B)] $Y$ is a random variable with respect to $\mathcal{G}$, but $X$ is not a random variable with respect to $\mathcal{G}$.
    \item [(C)] Neither $X$ nor $Y$ is a random variable with respect to $\mathcal{G}$.
    \item [(D)] Both $X$ and $Y$ are random variables with respect to $\mathcal{G}$.
\end{enumerate} \hfill (GATE ST 2023)\\
\solution
%\input{gate/ST/2023/14/main.tex}
	\item  A die is loaded in such a way that each odd number is twice as likely to occur as
each even number. Find $P(G)$, where $G$ is the event that a number greater than
3 occurs on a single roll of the die.
\\
\solution
		%\input{exemplar/11/16/3/5/main.tex}
	\item All the jacks, queens and kings are removed from a deck of 52 playing cards. The remaining cards are well shuffled and then one card is drawn at random. Giving ace a value 1 similar value for other cards, find the probability that the card has a value 
		\begin{enumerate}
			\item 7
			\item greater than 7
			\item less than 7
		\end{enumerate}
		%\input{exemplar/10/13/3/30/main.tex}
  \item A Lot consists of 48 mobile phones of which 42 are good, 3 have only minor defects and 3 have major defects.Varnika will buy a phone if it is good but the trader will only buy a mobile if it has no major defects. One phone is selected at random from the lot. What is the probability that it is
\begin{enumerate}
	\item acceptable to Varnika?
            \item acceptable to the trader?
\end{enumerate}
\solution
	%\input{exemplar/10/13/3/40/main.tex}
 \item A student says that if you throw a die, it will show up 1 or not 1. Therefore, the probability of getting 1 and the probability of getting 'not 1' each is equal to $\frac{1}{2}$. Is this correct? Give reasons.\\
 \solution
        %\input{exemplar/10/13/2/9/main.tex}
   \item Four candidates A, B, C, D have ap-
plied for the assignment to coach a school cricket
team. If A is twice as likely to be selected as B, and
B and C are given about the same chance of being
selected, while C is twice as likely to be selected
as D, what are the probabilities that
\begin{enumerate}
\item C will be selected?
\item A will not be selected?
\end{enumerate}
	%\input{exemplar/11/16/3/9/main.tex}
 \item A bag contain 24 balls of which $x$ balls are red, $2x$ are white and $3x$ are blue. A ball is selected at random, What is the probability that it is
\begin{enumerate}[label=\alph*)]
\item not red ?
\item white ?
\end{enumerate}
%\input{exemplar/10/13/3/41/main.tex}
If the letters of the word ASSASSINATION are arranged at random. Find the Probability that
\begin{enumerate}[label=(\alph*)]
\item Four $S's$ come consecutively in the word
\item Two  $I's$ and two $N's$ come together
\item All $A's$ are not coming together
\item No two $A's$ are coming together
\end{enumerate}
%\input{exemplar/11/16/3/14/main.tex}
	\item One urn contains two black balls (labelled B1 and B2) and one white ball. A
	second urn contains one black ball and two white balls (labelled W1 and W2).
	Suppose the following experiment is performed. One of the two urns is chosen
	at random. Next a ball is randomly chosen from the urn. Then a second ball is
	chosen at random from the same urn without replacing the first ball.
	
	\begin{enumerate}
	\item What is the probability that two black balls are chosen?
	
	\item What is the probability that two balls of opposite colour are chosen?
	\end{enumerate}
	\solution
	%\input{exemplar/11/16/3/12/main1.tex}
\end{enumerate}

	\item 
The number lock of a suitcase has 4 wheels each labelled with ten digits i.e. from 0 to 9.The lock opens with a sequence of four digits with no repeats.What is the probability of a person getting the right sequence to open the suitcase.
\\
\solution
		%\begin{enumerate}[label=\thesection.\arabic*,ref=\thesection.\theenumi]
	\item One card is drawn from a well-shuffled deck of 52 cards. Find the probability of getting
\begin{enumerate}
\item A king of red colour 
\item A face card 
\item A red face card
\item The jack of hearts
\item A spade
\item The queen of diamonds

\end{enumerate}
\solution
		%\input{ncert/10/15/1/14/main.tex}
	\item Five cards—the ten, jack, queen, king and ace of diamonds, are well-shuffled with their face downwards. One card is then picked up at random.
\begin{enumerate}
\item
What is the probability that the card is the queen? 
\item
If the queen is drawn and put aside, what is the probability that the second card picked up is (a) an ace? (b) a queen?\\
\end{enumerate}
\solution
		%\input{ncert/10/15/1/15/defs.tex}
	\item A bag contains $5$ red balls and some blue balls. If the probability of drawing a blue ball is double that if a red ball, determine the number of blue balls in the bag. 
		\\
\solution
		%\input{ncert/10/15/2/3/defs.tex}
	\item A card is selected from a pack of 52 cards.
 \begin{enumerate}[label=(\alph*)] 
                 \item How many points are there in the sample space?
                 \item Calculate the probability that the card is an ace of spades.
                 \item Calculate the probability that the card is (i) an ace and (ii) black card.
 \end{enumerate}
\solution
		%\input{ncert/11/16/3/4/main.tex}
\item Four cards are drawn from a well-shuffled deck of 52 cards. What is the probability of obtaining 3 diamonds and one spade.
\\
\solution
		%\input{ncert/11/16/4/2/defs.tex}
\item In a certain lottery 10,000 tickets are sold and ten equal prizes are awarded. What is the probability of not getting a prize if you buy (a) one ticket (b) two tickets (c) 10 tickets ?	
\\
\solution
		%\input{ncert/11/16/4/4/defs.tex}
		%
\item 
Out of 100 students, two sections of 40 and 60 are formed. If you and your friend are among the 100 students, what is the probability that
\begin{enumerate}
\item you both enter the same section?
\item you both enter the different sections?
\end{enumerate}
\solution
		%\input{ncert/11/16/4/5/defs.tex}
	\item 
The number lock of a suitcase has 4 wheels each labelled with ten digits i.e. from 0 to 9.The lock opens with a sequence of four digits with no repeats.What is the probability of a person getting the right sequence to open the suitcase.
\\
\solution
		%\input{ncert/11/16/4/10/defs.tex}
		%
\item 
Two cards are drawn at random and without replacement from a pack of 52 playing cards. Find the probability that both the cards are black.
\\
\solution
		%\input{ncert/12/13/2/2/defs.tex}
		\item A box of oranges is inspected by examining three randomly selected oranges drawn without replacement. If all the three oranges are good, the box is approved for sale, otherwise, it is rejected. Find the probability that a box containing 15 oranges out of which 12 are good and 3 are bad ones will be approved for sale.
		\label{ncert/12/13/2/3/defs.tex}
		\item Two balls are drawn at random with replacement from a box containing 10 black and 8 red balls. Find the probability that
		\label{ncert/12/13/2/12}
\begin{enumerate}
\item both balls are red.
\item first ball is black and second is red.
\item one of them is black and other is red.
\end{enumerate}

\item In a hostel, 60\% of the students read Hindi newspaper, 40\% read English newspaper and 20\% read both Hindi and English newspapers. A student is selected at random.
		\label{ncert/12/13/2/15}
\begin{enumerate}
\item Find the probability that she reads neither Hindi nor English newspapers.
\item If she reads Hindi newspaper, find the probability that she reads English newspaper.
\item If she reads English newspaper, find the probability that she reads Hindi newspaper.\\
\end{enumerate}
\item The probability of obtaining an even prime number on each die, when a pair of dice is rolled is 
\begin{enumerate}
    \item $0$ 
    
    \item $\frac{1}{3}$ 
    
    \item $\frac{1}{12}$ 
    
    \item $\frac{1}{36}$ 
\end{enumerate}
\solution
		%\input{ncert/12/13/2/17/defs.tex}
	\item A bag contains 4 red and 4 black balls, another bag contains 2 red and 6 black balls. One of the two bags is selected at random and a ball is drawn from the bag which is found to be red. Find the probability that the ball is drawn from the first bag.
\\
\solution
		%\input{ncert/12/13/3/2/main.tex}
  \item
  Cards with numbers 2 to 101 are placed in a box. A card is selected at random.Find the probability that the card has
\begin{enumerate}[label=(\roman*)]
	\item an even number 
	\item a square number
\end{enumerate}
\solution
%\input{exemplar/10/13/3/32/main.tex}
\item
The king, queen and jack of clubs are removed from a deck of 52 playing cards and then well shuffled. Now one card is drawn at random from the remaining cards.  Determine the probability that the card is
\begin{enumerate}[label=(\roman*)]
\item a club
\item 10 of hearts
\end{enumerate}
\solution
%\input{exemplar/10/13/3/29/main.tex}
\item A team of medical students doing their internship have to assist during surgeries
at a city hospital. The probabilities of surgeries rated as very complex, complex,
routine, simple or very simple are respectively, 0.15, 0.20, 0.31, 0.26, .08. Find
the probabilities that a particular surgery will be rated
\begin{enumerate}
	\item complex or very complex;
	\item neither very complex nor very simple;
	\item routine or complex
	\item routine or simple
\end{enumerate}
\solution
%\input{exemplar/11/16/3/8(1)/main.tex}
\item A card is selected from a pack of 52 cards.
\begin{enumerate}[label=(\alph*)]
    \item How many points are there in the sample space?
    \item Calculate the probability that the card is an ace of spades.
    \item Calculate the probability that the card is (i) an ace and (ii) black card.
\end{enumerate}
\solution
%\input{exemplar/11/16/3/4/main2.tex}
\item The probability that a non leap year selected at random will contain 53 sundays.
\\
\solution
%\input{exemplar/10/13/1/19/main.tex}
\item One of the four persons John, Rita, Aslam or Gurpreet will be promoted next
month. Consequently the sample space consists of four elementary outcomes
S = {John promoted, Rita promoted, Aslam promoted, Gurpreet promoted}
You are told that the chances of John’s promotion is same as that of Gurpreet,
Rita’s chances of promotion are twice as likely as Johns. Aslam’s chances are
four times that of John.
\begin{enumerate}
	\item Determine
	\begin{enumerate}
		\item P (John promoted)
		\item P (Rita promoted)
		\item P (Aslam promoted)
		\item P (Gurpreet promoted)
	\end{enumerate}
	\item If A = {John promoted or Gurpreet promoted}, find P (A).
\end{enumerate}
\solution
%\input{exemplar/11/16/3/10/main.tex}
\item A card is drawn from a deck of 52 cards. Find the probability of getting a king or a heart or a red card.\\
\solution
%\input{exemplar/11/16/3/15/main.tex}
\item The probability that a student will pass his examination is 0.73, the probability of
the student getting a compartment is 0.13, and the probability that the student will
either pass or get compartment is 0.96. State True or False.\\
\solution
%\input{exemplar/11/16/3/31/main.tex}
\item A card is selected from a pack of 52 cards\\
\begin{enumerate}[label=(\alph*)]
\item How many points are there in the sample space?
\item Calculate the probability that the cards is an ace of spades.
\item Calculate the probability that the card is (i) an ace (ii)black card.\\
\end{enumerate}
%\input{ncert/11/16/3/4_1/Prob_4.tex}
\item In a non-leap year, the probability of having 53 tuesdays or 53 wednesdays is\\
\solution
%\input{exemplar/11/16/3/18/main.tex}
\item There are 1000 sealed envelopes in a box, 10 of them contain a cash prize of
Rs 100 each, 100 of them contain a cash prize of Rs 50 each and 200 of them
contain a cash prize of Rs 10 each and rest do not contain any cash prize. If they
are well shuffled and an envelope is picked up out, what is the probability that it
contains no cash prize?\\
\solution
%\input{exemplar/10/13/3/34/main.tex}
\item 
A die is thrown and a card is selected at random from a deck of 52 playing cards. The probability of getting an even number on the die and a spade card.\\
\solution
%\input{exemplar/12/13/3/78/main.tex}
\item
If 4-digit numbers greater than 5,000 are randomly formed from the digits 0, 1, 3, 5, and 7, what is the probability of forming a number divisible by 5 when:
\begin{enumerate}
    \item The digits are repeated?
    \item The repetition of digits is not allowed?
\end{enumerate}
\solution
%\input{ncert/11/16/4/9/main.tex}
\item Consider the probability space $\brak{\Omega, \mathcal{G}, P}$ where $\Omega = [0,2]$ and $\mathcal{G} = \cbrak{\phi, \Omega, [0,1], (1,2]}$. Let $X$ and $Y$ be two functions on $\Omega$ defined as
\begin{align*}
    X(\omega) = 
    \begin{cases}
        1 & \text{if }\omega \in [0, 1]\\
        2 & \text{if }\omega \in (1, 2]
    \end{cases}
\end{align*}
and
\begin{align*}
    Y(\omega) = 
    \begin{cases}
        2 & \text{if }\omega \in [0, 1.5]\\
        3 & \text{if }\omega \in (1.5, 2].
    \end{cases}
\end{align*}
Then which one of the following statements is true?
\begin{enumerate}
    \item [(A)] $X$ is a random variable with respect to $\mathcal{G}$, but $Y$ is not a random variable with respect to $\mathcal{G}$.
    \item [(B)] $Y$ is a random variable with respect to $\mathcal{G}$, but $X$ is not a random variable with respect to $\mathcal{G}$.
    \item [(C)] Neither $X$ nor $Y$ is a random variable with respect to $\mathcal{G}$.
    \item [(D)] Both $X$ and $Y$ are random variables with respect to $\mathcal{G}$.
\end{enumerate} \hfill (GATE ST 2023)\\
\solution
%\input{gate/ST/2023/14/main.tex}
	\item  A die is loaded in such a way that each odd number is twice as likely to occur as
each even number. Find $P(G)$, where $G$ is the event that a number greater than
3 occurs on a single roll of the die.
\\
\solution
		%\input{exemplar/11/16/3/5/main.tex}
	\item All the jacks, queens and kings are removed from a deck of 52 playing cards. The remaining cards are well shuffled and then one card is drawn at random. Giving ace a value 1 similar value for other cards, find the probability that the card has a value 
		\begin{enumerate}
			\item 7
			\item greater than 7
			\item less than 7
		\end{enumerate}
		%\input{exemplar/10/13/3/30/main.tex}
  \item A Lot consists of 48 mobile phones of which 42 are good, 3 have only minor defects and 3 have major defects.Varnika will buy a phone if it is good but the trader will only buy a mobile if it has no major defects. One phone is selected at random from the lot. What is the probability that it is
\begin{enumerate}
	\item acceptable to Varnika?
            \item acceptable to the trader?
\end{enumerate}
\solution
	%\input{exemplar/10/13/3/40/main.tex}
 \item A student says that if you throw a die, it will show up 1 or not 1. Therefore, the probability of getting 1 and the probability of getting 'not 1' each is equal to $\frac{1}{2}$. Is this correct? Give reasons.\\
 \solution
        %\input{exemplar/10/13/2/9/main.tex}
   \item Four candidates A, B, C, D have ap-
plied for the assignment to coach a school cricket
team. If A is twice as likely to be selected as B, and
B and C are given about the same chance of being
selected, while C is twice as likely to be selected
as D, what are the probabilities that
\begin{enumerate}
\item C will be selected?
\item A will not be selected?
\end{enumerate}
	%\input{exemplar/11/16/3/9/main.tex}
 \item A bag contain 24 balls of which $x$ balls are red, $2x$ are white and $3x$ are blue. A ball is selected at random, What is the probability that it is
\begin{enumerate}[label=\alph*)]
\item not red ?
\item white ?
\end{enumerate}
%\input{exemplar/10/13/3/41/main.tex}
If the letters of the word ASSASSINATION are arranged at random. Find the Probability that
\begin{enumerate}[label=(\alph*)]
\item Four $S's$ come consecutively in the word
\item Two  $I's$ and two $N's$ come together
\item All $A's$ are not coming together
\item No two $A's$ are coming together
\end{enumerate}
%\input{exemplar/11/16/3/14/main.tex}
	\item One urn contains two black balls (labelled B1 and B2) and one white ball. A
	second urn contains one black ball and two white balls (labelled W1 and W2).
	Suppose the following experiment is performed. One of the two urns is chosen
	at random. Next a ball is randomly chosen from the urn. Then a second ball is
	chosen at random from the same urn without replacing the first ball.
	
	\begin{enumerate}
	\item What is the probability that two black balls are chosen?
	
	\item What is the probability that two balls of opposite colour are chosen?
	\end{enumerate}
	\solution
	%\input{exemplar/11/16/3/12/main1.tex}
\end{enumerate}

		%
\item 
Two cards are drawn at random and without replacement from a pack of 52 playing cards. Find the probability that both the cards are black.
\\
\solution
		%\begin{enumerate}[label=\thesection.\arabic*,ref=\thesection.\theenumi]
	\item One card is drawn from a well-shuffled deck of 52 cards. Find the probability of getting
\begin{enumerate}
\item A king of red colour 
\item A face card 
\item A red face card
\item The jack of hearts
\item A spade
\item The queen of diamonds

\end{enumerate}
\solution
		%\input{ncert/10/15/1/14/main.tex}
	\item Five cards—the ten, jack, queen, king and ace of diamonds, are well-shuffled with their face downwards. One card is then picked up at random.
\begin{enumerate}
\item
What is the probability that the card is the queen? 
\item
If the queen is drawn and put aside, what is the probability that the second card picked up is (a) an ace? (b) a queen?\\
\end{enumerate}
\solution
		%\input{ncert/10/15/1/15/defs.tex}
	\item A bag contains $5$ red balls and some blue balls. If the probability of drawing a blue ball is double that if a red ball, determine the number of blue balls in the bag. 
		\\
\solution
		%\input{ncert/10/15/2/3/defs.tex}
	\item A card is selected from a pack of 52 cards.
 \begin{enumerate}[label=(\alph*)] 
                 \item How many points are there in the sample space?
                 \item Calculate the probability that the card is an ace of spades.
                 \item Calculate the probability that the card is (i) an ace and (ii) black card.
 \end{enumerate}
\solution
		%\input{ncert/11/16/3/4/main.tex}
\item Four cards are drawn from a well-shuffled deck of 52 cards. What is the probability of obtaining 3 diamonds and one spade.
\\
\solution
		%\input{ncert/11/16/4/2/defs.tex}
\item In a certain lottery 10,000 tickets are sold and ten equal prizes are awarded. What is the probability of not getting a prize if you buy (a) one ticket (b) two tickets (c) 10 tickets ?	
\\
\solution
		%\input{ncert/11/16/4/4/defs.tex}
		%
\item 
Out of 100 students, two sections of 40 and 60 are formed. If you and your friend are among the 100 students, what is the probability that
\begin{enumerate}
\item you both enter the same section?
\item you both enter the different sections?
\end{enumerate}
\solution
		%\input{ncert/11/16/4/5/defs.tex}
	\item 
The number lock of a suitcase has 4 wheels each labelled with ten digits i.e. from 0 to 9.The lock opens with a sequence of four digits with no repeats.What is the probability of a person getting the right sequence to open the suitcase.
\\
\solution
		%\input{ncert/11/16/4/10/defs.tex}
		%
\item 
Two cards are drawn at random and without replacement from a pack of 52 playing cards. Find the probability that both the cards are black.
\\
\solution
		%\input{ncert/12/13/2/2/defs.tex}
		\item A box of oranges is inspected by examining three randomly selected oranges drawn without replacement. If all the three oranges are good, the box is approved for sale, otherwise, it is rejected. Find the probability that a box containing 15 oranges out of which 12 are good and 3 are bad ones will be approved for sale.
		\label{ncert/12/13/2/3/defs.tex}
		\item Two balls are drawn at random with replacement from a box containing 10 black and 8 red balls. Find the probability that
		\label{ncert/12/13/2/12}
\begin{enumerate}
\item both balls are red.
\item first ball is black and second is red.
\item one of them is black and other is red.
\end{enumerate}

\item In a hostel, 60\% of the students read Hindi newspaper, 40\% read English newspaper and 20\% read both Hindi and English newspapers. A student is selected at random.
		\label{ncert/12/13/2/15}
\begin{enumerate}
\item Find the probability that she reads neither Hindi nor English newspapers.
\item If she reads Hindi newspaper, find the probability that she reads English newspaper.
\item If she reads English newspaper, find the probability that she reads Hindi newspaper.\\
\end{enumerate}
\item The probability of obtaining an even prime number on each die, when a pair of dice is rolled is 
\begin{enumerate}
    \item $0$ 
    
    \item $\frac{1}{3}$ 
    
    \item $\frac{1}{12}$ 
    
    \item $\frac{1}{36}$ 
\end{enumerate}
\solution
		%\input{ncert/12/13/2/17/defs.tex}
	\item A bag contains 4 red and 4 black balls, another bag contains 2 red and 6 black balls. One of the two bags is selected at random and a ball is drawn from the bag which is found to be red. Find the probability that the ball is drawn from the first bag.
\\
\solution
		%\input{ncert/12/13/3/2/main.tex}
  \item
  Cards with numbers 2 to 101 are placed in a box. A card is selected at random.Find the probability that the card has
\begin{enumerate}[label=(\roman*)]
	\item an even number 
	\item a square number
\end{enumerate}
\solution
%\input{exemplar/10/13/3/32/main.tex}
\item
The king, queen and jack of clubs are removed from a deck of 52 playing cards and then well shuffled. Now one card is drawn at random from the remaining cards.  Determine the probability that the card is
\begin{enumerate}[label=(\roman*)]
\item a club
\item 10 of hearts
\end{enumerate}
\solution
%\input{exemplar/10/13/3/29/main.tex}
\item A team of medical students doing their internship have to assist during surgeries
at a city hospital. The probabilities of surgeries rated as very complex, complex,
routine, simple or very simple are respectively, 0.15, 0.20, 0.31, 0.26, .08. Find
the probabilities that a particular surgery will be rated
\begin{enumerate}
	\item complex or very complex;
	\item neither very complex nor very simple;
	\item routine or complex
	\item routine or simple
\end{enumerate}
\solution
%\input{exemplar/11/16/3/8(1)/main.tex}
\item A card is selected from a pack of 52 cards.
\begin{enumerate}[label=(\alph*)]
    \item How many points are there in the sample space?
    \item Calculate the probability that the card is an ace of spades.
    \item Calculate the probability that the card is (i) an ace and (ii) black card.
\end{enumerate}
\solution
%\input{exemplar/11/16/3/4/main2.tex}
\item The probability that a non leap year selected at random will contain 53 sundays.
\\
\solution
%\input{exemplar/10/13/1/19/main.tex}
\item One of the four persons John, Rita, Aslam or Gurpreet will be promoted next
month. Consequently the sample space consists of four elementary outcomes
S = {John promoted, Rita promoted, Aslam promoted, Gurpreet promoted}
You are told that the chances of John’s promotion is same as that of Gurpreet,
Rita’s chances of promotion are twice as likely as Johns. Aslam’s chances are
four times that of John.
\begin{enumerate}
	\item Determine
	\begin{enumerate}
		\item P (John promoted)
		\item P (Rita promoted)
		\item P (Aslam promoted)
		\item P (Gurpreet promoted)
	\end{enumerate}
	\item If A = {John promoted or Gurpreet promoted}, find P (A).
\end{enumerate}
\solution
%\input{exemplar/11/16/3/10/main.tex}
\item A card is drawn from a deck of 52 cards. Find the probability of getting a king or a heart or a red card.\\
\solution
%\input{exemplar/11/16/3/15/main.tex}
\item The probability that a student will pass his examination is 0.73, the probability of
the student getting a compartment is 0.13, and the probability that the student will
either pass or get compartment is 0.96. State True or False.\\
\solution
%\input{exemplar/11/16/3/31/main.tex}
\item A card is selected from a pack of 52 cards\\
\begin{enumerate}[label=(\alph*)]
\item How many points are there in the sample space?
\item Calculate the probability that the cards is an ace of spades.
\item Calculate the probability that the card is (i) an ace (ii)black card.\\
\end{enumerate}
%\input{ncert/11/16/3/4_1/Prob_4.tex}
\item In a non-leap year, the probability of having 53 tuesdays or 53 wednesdays is\\
\solution
%\input{exemplar/11/16/3/18/main.tex}
\item There are 1000 sealed envelopes in a box, 10 of them contain a cash prize of
Rs 100 each, 100 of them contain a cash prize of Rs 50 each and 200 of them
contain a cash prize of Rs 10 each and rest do not contain any cash prize. If they
are well shuffled and an envelope is picked up out, what is the probability that it
contains no cash prize?\\
\solution
%\input{exemplar/10/13/3/34/main.tex}
\item 
A die is thrown and a card is selected at random from a deck of 52 playing cards. The probability of getting an even number on the die and a spade card.\\
\solution
%\input{exemplar/12/13/3/78/main.tex}
\item
If 4-digit numbers greater than 5,000 are randomly formed from the digits 0, 1, 3, 5, and 7, what is the probability of forming a number divisible by 5 when:
\begin{enumerate}
    \item The digits are repeated?
    \item The repetition of digits is not allowed?
\end{enumerate}
\solution
%\input{ncert/11/16/4/9/main.tex}
\item Consider the probability space $\brak{\Omega, \mathcal{G}, P}$ where $\Omega = [0,2]$ and $\mathcal{G} = \cbrak{\phi, \Omega, [0,1], (1,2]}$. Let $X$ and $Y$ be two functions on $\Omega$ defined as
\begin{align*}
    X(\omega) = 
    \begin{cases}
        1 & \text{if }\omega \in [0, 1]\\
        2 & \text{if }\omega \in (1, 2]
    \end{cases}
\end{align*}
and
\begin{align*}
    Y(\omega) = 
    \begin{cases}
        2 & \text{if }\omega \in [0, 1.5]\\
        3 & \text{if }\omega \in (1.5, 2].
    \end{cases}
\end{align*}
Then which one of the following statements is true?
\begin{enumerate}
    \item [(A)] $X$ is a random variable with respect to $\mathcal{G}$, but $Y$ is not a random variable with respect to $\mathcal{G}$.
    \item [(B)] $Y$ is a random variable with respect to $\mathcal{G}$, but $X$ is not a random variable with respect to $\mathcal{G}$.
    \item [(C)] Neither $X$ nor $Y$ is a random variable with respect to $\mathcal{G}$.
    \item [(D)] Both $X$ and $Y$ are random variables with respect to $\mathcal{G}$.
\end{enumerate} \hfill (GATE ST 2023)\\
\solution
%\input{gate/ST/2023/14/main.tex}
	\item  A die is loaded in such a way that each odd number is twice as likely to occur as
each even number. Find $P(G)$, where $G$ is the event that a number greater than
3 occurs on a single roll of the die.
\\
\solution
		%\input{exemplar/11/16/3/5/main.tex}
	\item All the jacks, queens and kings are removed from a deck of 52 playing cards. The remaining cards are well shuffled and then one card is drawn at random. Giving ace a value 1 similar value for other cards, find the probability that the card has a value 
		\begin{enumerate}
			\item 7
			\item greater than 7
			\item less than 7
		\end{enumerate}
		%\input{exemplar/10/13/3/30/main.tex}
  \item A Lot consists of 48 mobile phones of which 42 are good, 3 have only minor defects and 3 have major defects.Varnika will buy a phone if it is good but the trader will only buy a mobile if it has no major defects. One phone is selected at random from the lot. What is the probability that it is
\begin{enumerate}
	\item acceptable to Varnika?
            \item acceptable to the trader?
\end{enumerate}
\solution
	%\input{exemplar/10/13/3/40/main.tex}
 \item A student says that if you throw a die, it will show up 1 or not 1. Therefore, the probability of getting 1 and the probability of getting 'not 1' each is equal to $\frac{1}{2}$. Is this correct? Give reasons.\\
 \solution
        %\input{exemplar/10/13/2/9/main.tex}
   \item Four candidates A, B, C, D have ap-
plied for the assignment to coach a school cricket
team. If A is twice as likely to be selected as B, and
B and C are given about the same chance of being
selected, while C is twice as likely to be selected
as D, what are the probabilities that
\begin{enumerate}
\item C will be selected?
\item A will not be selected?
\end{enumerate}
	%\input{exemplar/11/16/3/9/main.tex}
 \item A bag contain 24 balls of which $x$ balls are red, $2x$ are white and $3x$ are blue. A ball is selected at random, What is the probability that it is
\begin{enumerate}[label=\alph*)]
\item not red ?
\item white ?
\end{enumerate}
%\input{exemplar/10/13/3/41/main.tex}
If the letters of the word ASSASSINATION are arranged at random. Find the Probability that
\begin{enumerate}[label=(\alph*)]
\item Four $S's$ come consecutively in the word
\item Two  $I's$ and two $N's$ come together
\item All $A's$ are not coming together
\item No two $A's$ are coming together
\end{enumerate}
%\input{exemplar/11/16/3/14/main.tex}
	\item One urn contains two black balls (labelled B1 and B2) and one white ball. A
	second urn contains one black ball and two white balls (labelled W1 and W2).
	Suppose the following experiment is performed. One of the two urns is chosen
	at random. Next a ball is randomly chosen from the urn. Then a second ball is
	chosen at random from the same urn without replacing the first ball.
	
	\begin{enumerate}
	\item What is the probability that two black balls are chosen?
	
	\item What is the probability that two balls of opposite colour are chosen?
	\end{enumerate}
	\solution
	%\input{exemplar/11/16/3/12/main1.tex}
\end{enumerate}

		\item A box of oranges is inspected by examining three randomly selected oranges drawn without replacement. If all the three oranges are good, the box is approved for sale, otherwise, it is rejected. Find the probability that a box containing 15 oranges out of which 12 are good and 3 are bad ones will be approved for sale.
		\label{ncert/12/13/2/3/defs.tex}
		\item Two balls are drawn at random with replacement from a box containing 10 black and 8 red balls. Find the probability that
		\label{ncert/12/13/2/12}
\begin{enumerate}
\item both balls are red.
\item first ball is black and second is red.
\item one of them is black and other is red.
\end{enumerate}

\item In a hostel, 60\% of the students read Hindi newspaper, 40\% read English newspaper and 20\% read both Hindi and English newspapers. A student is selected at random.
		\label{ncert/12/13/2/15}
\begin{enumerate}
\item Find the probability that she reads neither Hindi nor English newspapers.
\item If she reads Hindi newspaper, find the probability that she reads English newspaper.
\item If she reads English newspaper, find the probability that she reads Hindi newspaper.\\
\end{enumerate}
\item The probability of obtaining an even prime number on each die, when a pair of dice is rolled is 
\begin{enumerate}
    \item $0$ 
    
    \item $\frac{1}{3}$ 
    
    \item $\frac{1}{12}$ 
    
    \item $\frac{1}{36}$ 
\end{enumerate}
\solution
		%\begin{enumerate}[label=\thesection.\arabic*,ref=\thesection.\theenumi]
	\item One card is drawn from a well-shuffled deck of 52 cards. Find the probability of getting
\begin{enumerate}
\item A king of red colour 
\item A face card 
\item A red face card
\item The jack of hearts
\item A spade
\item The queen of diamonds

\end{enumerate}
\solution
		%\input{ncert/10/15/1/14/main.tex}
	\item Five cards—the ten, jack, queen, king and ace of diamonds, are well-shuffled with their face downwards. One card is then picked up at random.
\begin{enumerate}
\item
What is the probability that the card is the queen? 
\item
If the queen is drawn and put aside, what is the probability that the second card picked up is (a) an ace? (b) a queen?\\
\end{enumerate}
\solution
		%\input{ncert/10/15/1/15/defs.tex}
	\item A bag contains $5$ red balls and some blue balls. If the probability of drawing a blue ball is double that if a red ball, determine the number of blue balls in the bag. 
		\\
\solution
		%\input{ncert/10/15/2/3/defs.tex}
	\item A card is selected from a pack of 52 cards.
 \begin{enumerate}[label=(\alph*)] 
                 \item How many points are there in the sample space?
                 \item Calculate the probability that the card is an ace of spades.
                 \item Calculate the probability that the card is (i) an ace and (ii) black card.
 \end{enumerate}
\solution
		%\input{ncert/11/16/3/4/main.tex}
\item Four cards are drawn from a well-shuffled deck of 52 cards. What is the probability of obtaining 3 diamonds and one spade.
\\
\solution
		%\input{ncert/11/16/4/2/defs.tex}
\item In a certain lottery 10,000 tickets are sold and ten equal prizes are awarded. What is the probability of not getting a prize if you buy (a) one ticket (b) two tickets (c) 10 tickets ?	
\\
\solution
		%\input{ncert/11/16/4/4/defs.tex}
		%
\item 
Out of 100 students, two sections of 40 and 60 are formed. If you and your friend are among the 100 students, what is the probability that
\begin{enumerate}
\item you both enter the same section?
\item you both enter the different sections?
\end{enumerate}
\solution
		%\input{ncert/11/16/4/5/defs.tex}
	\item 
The number lock of a suitcase has 4 wheels each labelled with ten digits i.e. from 0 to 9.The lock opens with a sequence of four digits with no repeats.What is the probability of a person getting the right sequence to open the suitcase.
\\
\solution
		%\input{ncert/11/16/4/10/defs.tex}
		%
\item 
Two cards are drawn at random and without replacement from a pack of 52 playing cards. Find the probability that both the cards are black.
\\
\solution
		%\input{ncert/12/13/2/2/defs.tex}
		\item A box of oranges is inspected by examining three randomly selected oranges drawn without replacement. If all the three oranges are good, the box is approved for sale, otherwise, it is rejected. Find the probability that a box containing 15 oranges out of which 12 are good and 3 are bad ones will be approved for sale.
		\label{ncert/12/13/2/3/defs.tex}
		\item Two balls are drawn at random with replacement from a box containing 10 black and 8 red balls. Find the probability that
		\label{ncert/12/13/2/12}
\begin{enumerate}
\item both balls are red.
\item first ball is black and second is red.
\item one of them is black and other is red.
\end{enumerate}

\item In a hostel, 60\% of the students read Hindi newspaper, 40\% read English newspaper and 20\% read both Hindi and English newspapers. A student is selected at random.
		\label{ncert/12/13/2/15}
\begin{enumerate}
\item Find the probability that she reads neither Hindi nor English newspapers.
\item If she reads Hindi newspaper, find the probability that she reads English newspaper.
\item If she reads English newspaper, find the probability that she reads Hindi newspaper.\\
\end{enumerate}
\item The probability of obtaining an even prime number on each die, when a pair of dice is rolled is 
\begin{enumerate}
    \item $0$ 
    
    \item $\frac{1}{3}$ 
    
    \item $\frac{1}{12}$ 
    
    \item $\frac{1}{36}$ 
\end{enumerate}
\solution
		%\input{ncert/12/13/2/17/defs.tex}
	\item A bag contains 4 red and 4 black balls, another bag contains 2 red and 6 black balls. One of the two bags is selected at random and a ball is drawn from the bag which is found to be red. Find the probability that the ball is drawn from the first bag.
\\
\solution
		%\input{ncert/12/13/3/2/main.tex}
  \item
  Cards with numbers 2 to 101 are placed in a box. A card is selected at random.Find the probability that the card has
\begin{enumerate}[label=(\roman*)]
	\item an even number 
	\item a square number
\end{enumerate}
\solution
%\input{exemplar/10/13/3/32/main.tex}
\item
The king, queen and jack of clubs are removed from a deck of 52 playing cards and then well shuffled. Now one card is drawn at random from the remaining cards.  Determine the probability that the card is
\begin{enumerate}[label=(\roman*)]
\item a club
\item 10 of hearts
\end{enumerate}
\solution
%\input{exemplar/10/13/3/29/main.tex}
\item A team of medical students doing their internship have to assist during surgeries
at a city hospital. The probabilities of surgeries rated as very complex, complex,
routine, simple or very simple are respectively, 0.15, 0.20, 0.31, 0.26, .08. Find
the probabilities that a particular surgery will be rated
\begin{enumerate}
	\item complex or very complex;
	\item neither very complex nor very simple;
	\item routine or complex
	\item routine or simple
\end{enumerate}
\solution
%\input{exemplar/11/16/3/8(1)/main.tex}
\item A card is selected from a pack of 52 cards.
\begin{enumerate}[label=(\alph*)]
    \item How many points are there in the sample space?
    \item Calculate the probability that the card is an ace of spades.
    \item Calculate the probability that the card is (i) an ace and (ii) black card.
\end{enumerate}
\solution
%\input{exemplar/11/16/3/4/main2.tex}
\item The probability that a non leap year selected at random will contain 53 sundays.
\\
\solution
%\input{exemplar/10/13/1/19/main.tex}
\item One of the four persons John, Rita, Aslam or Gurpreet will be promoted next
month. Consequently the sample space consists of four elementary outcomes
S = {John promoted, Rita promoted, Aslam promoted, Gurpreet promoted}
You are told that the chances of John’s promotion is same as that of Gurpreet,
Rita’s chances of promotion are twice as likely as Johns. Aslam’s chances are
four times that of John.
\begin{enumerate}
	\item Determine
	\begin{enumerate}
		\item P (John promoted)
		\item P (Rita promoted)
		\item P (Aslam promoted)
		\item P (Gurpreet promoted)
	\end{enumerate}
	\item If A = {John promoted or Gurpreet promoted}, find P (A).
\end{enumerate}
\solution
%\input{exemplar/11/16/3/10/main.tex}
\item A card is drawn from a deck of 52 cards. Find the probability of getting a king or a heart or a red card.\\
\solution
%\input{exemplar/11/16/3/15/main.tex}
\item The probability that a student will pass his examination is 0.73, the probability of
the student getting a compartment is 0.13, and the probability that the student will
either pass or get compartment is 0.96. State True or False.\\
\solution
%\input{exemplar/11/16/3/31/main.tex}
\item A card is selected from a pack of 52 cards\\
\begin{enumerate}[label=(\alph*)]
\item How many points are there in the sample space?
\item Calculate the probability that the cards is an ace of spades.
\item Calculate the probability that the card is (i) an ace (ii)black card.\\
\end{enumerate}
%\input{ncert/11/16/3/4_1/Prob_4.tex}
\item In a non-leap year, the probability of having 53 tuesdays or 53 wednesdays is\\
\solution
%\input{exemplar/11/16/3/18/main.tex}
\item There are 1000 sealed envelopes in a box, 10 of them contain a cash prize of
Rs 100 each, 100 of them contain a cash prize of Rs 50 each and 200 of them
contain a cash prize of Rs 10 each and rest do not contain any cash prize. If they
are well shuffled and an envelope is picked up out, what is the probability that it
contains no cash prize?\\
\solution
%\input{exemplar/10/13/3/34/main.tex}
\item 
A die is thrown and a card is selected at random from a deck of 52 playing cards. The probability of getting an even number on the die and a spade card.\\
\solution
%\input{exemplar/12/13/3/78/main.tex}
\item
If 4-digit numbers greater than 5,000 are randomly formed from the digits 0, 1, 3, 5, and 7, what is the probability of forming a number divisible by 5 when:
\begin{enumerate}
    \item The digits are repeated?
    \item The repetition of digits is not allowed?
\end{enumerate}
\solution
%\input{ncert/11/16/4/9/main.tex}
\item Consider the probability space $\brak{\Omega, \mathcal{G}, P}$ where $\Omega = [0,2]$ and $\mathcal{G} = \cbrak{\phi, \Omega, [0,1], (1,2]}$. Let $X$ and $Y$ be two functions on $\Omega$ defined as
\begin{align*}
    X(\omega) = 
    \begin{cases}
        1 & \text{if }\omega \in [0, 1]\\
        2 & \text{if }\omega \in (1, 2]
    \end{cases}
\end{align*}
and
\begin{align*}
    Y(\omega) = 
    \begin{cases}
        2 & \text{if }\omega \in [0, 1.5]\\
        3 & \text{if }\omega \in (1.5, 2].
    \end{cases}
\end{align*}
Then which one of the following statements is true?
\begin{enumerate}
    \item [(A)] $X$ is a random variable with respect to $\mathcal{G}$, but $Y$ is not a random variable with respect to $\mathcal{G}$.
    \item [(B)] $Y$ is a random variable with respect to $\mathcal{G}$, but $X$ is not a random variable with respect to $\mathcal{G}$.
    \item [(C)] Neither $X$ nor $Y$ is a random variable with respect to $\mathcal{G}$.
    \item [(D)] Both $X$ and $Y$ are random variables with respect to $\mathcal{G}$.
\end{enumerate} \hfill (GATE ST 2023)\\
\solution
%\input{gate/ST/2023/14/main.tex}
	\item  A die is loaded in such a way that each odd number is twice as likely to occur as
each even number. Find $P(G)$, where $G$ is the event that a number greater than
3 occurs on a single roll of the die.
\\
\solution
		%\input{exemplar/11/16/3/5/main.tex}
	\item All the jacks, queens and kings are removed from a deck of 52 playing cards. The remaining cards are well shuffled and then one card is drawn at random. Giving ace a value 1 similar value for other cards, find the probability that the card has a value 
		\begin{enumerate}
			\item 7
			\item greater than 7
			\item less than 7
		\end{enumerate}
		%\input{exemplar/10/13/3/30/main.tex}
  \item A Lot consists of 48 mobile phones of which 42 are good, 3 have only minor defects and 3 have major defects.Varnika will buy a phone if it is good but the trader will only buy a mobile if it has no major defects. One phone is selected at random from the lot. What is the probability that it is
\begin{enumerate}
	\item acceptable to Varnika?
            \item acceptable to the trader?
\end{enumerate}
\solution
	%\input{exemplar/10/13/3/40/main.tex}
 \item A student says that if you throw a die, it will show up 1 or not 1. Therefore, the probability of getting 1 and the probability of getting 'not 1' each is equal to $\frac{1}{2}$. Is this correct? Give reasons.\\
 \solution
        %\input{exemplar/10/13/2/9/main.tex}
   \item Four candidates A, B, C, D have ap-
plied for the assignment to coach a school cricket
team. If A is twice as likely to be selected as B, and
B and C are given about the same chance of being
selected, while C is twice as likely to be selected
as D, what are the probabilities that
\begin{enumerate}
\item C will be selected?
\item A will not be selected?
\end{enumerate}
	%\input{exemplar/11/16/3/9/main.tex}
 \item A bag contain 24 balls of which $x$ balls are red, $2x$ are white and $3x$ are blue. A ball is selected at random, What is the probability that it is
\begin{enumerate}[label=\alph*)]
\item not red ?
\item white ?
\end{enumerate}
%\input{exemplar/10/13/3/41/main.tex}
If the letters of the word ASSASSINATION are arranged at random. Find the Probability that
\begin{enumerate}[label=(\alph*)]
\item Four $S's$ come consecutively in the word
\item Two  $I's$ and two $N's$ come together
\item All $A's$ are not coming together
\item No two $A's$ are coming together
\end{enumerate}
%\input{exemplar/11/16/3/14/main.tex}
	\item One urn contains two black balls (labelled B1 and B2) and one white ball. A
	second urn contains one black ball and two white balls (labelled W1 and W2).
	Suppose the following experiment is performed. One of the two urns is chosen
	at random. Next a ball is randomly chosen from the urn. Then a second ball is
	chosen at random from the same urn without replacing the first ball.
	
	\begin{enumerate}
	\item What is the probability that two black balls are chosen?
	
	\item What is the probability that two balls of opposite colour are chosen?
	\end{enumerate}
	\solution
	%\input{exemplar/11/16/3/12/main1.tex}
\end{enumerate}

	\item A bag contains 4 red and 4 black balls, another bag contains 2 red and 6 black balls. One of the two bags is selected at random and a ball is drawn from the bag which is found to be red. Find the probability that the ball is drawn from the first bag.
\\
\solution
		%\begin{table}[H]
	\centering
\begin{tabular}{|c|c|c|}
\hline
Random variable &Value &Definition\\ \hline
\multirow{3}{*}{X} &0 &Slips of Rs 1\\
&1 &Slips of Rs 5\\
&2 &Slips of Rs 13\\ \hline
\multirow{2}{*}{Y} &0 &Box A\\
&1 &Box B\\\hline
\end{tabular}
\caption{}
\label{tab:Distribution}
\end{table}
See \tabref{tab:Distribution}.
\begin{align}
p_{Y}\brak{k}= \begin{cases} 
      \frac{1}{3} & {k=0} \\
      \frac{2}{3 }& {k=1} 
   \end{cases}
   \\
p_{Y|X}\brak{0|0} = \frac{19}{25}\, 
p_{Y|X}\brak{0|1} = \frac{6}{25}\,
p_{Y|X}\brak{1|0} = \frac{45}{50}\,
p_{Y|X}\brak{1|2} = \frac{5}{50}
\end{align}
The desired probability is the probability that a slip drawn at random is marked other than Rs 1,
\begin{align}
&=1-p_X\brak{0}\\
&= p_X(1) + p_X(2)
\end{align}
Using Bayes theorem,
\begin{align}
&= p_Y\brak{0} \times \pr{Y=0 | X=1} + p_Y\brak{1} \times \pr{Y=1|X=2}\\
&=\frac{1}{3} \times \frac{6}{25} + \frac{2}{3} \times \frac{5}{50}\\
&=\frac{11}{75}
\end{align}

\newpage

%\tableofcontents

\bigskip

\renewcommand{\thefigure}{\theenumi}
\renewcommand{\thetable}{\theenumi}
%\renewcommand{\theequation}{\theenumi}

%\begin{abstract}
%%\boldmath
%In this letter, an algorithm for evaluating the exact analytical bit error rate  (BER)  for the piecewise linear (PL) combiner for  multiple relays is presented. Previous results were available only for upto three relays. The algorithm is unique in the sense that  the actual mathematical expressions, that are prohibitively large, need not be explicitly obtained. The diversity gain due to multiple relays is shown through plots of the analytical BER, well supported by simulations. 
%
%\end{abstract}
% IEEEtran.cls defaults to using nonbold math in the Abstract.
% This preserves the distinction between vectors and scalars. However,
% if the journal you are submitting to favors bold math in the abstract,
% then you can use LaTeX's standard command \boldmath at the very start
% of the abstract to achieve this. Many IEEE journals frown on math
% in the abstract anyway.

% Note that keywords are not normally used for peerreview papers.
%\begin{IEEEkeywords}
%Cooperative diversity, decode and forward, piecewise linear
%\end{IEEEkeywords}



% For peer review papers, you can put extra information on the cover
% page as needed:
% \ifCLASSOPTIONpeerreview
% \begin{center} \bfseries EDICS Category: 3-BBND \end{center}
% \fi
%
% For peerreview papers, this IEEEtran command inserts a page break and
% creates the second title. It will be ignored for other modes.
%\IEEEpeerreviewmaketitle




  \item
  Cards with numbers 2 to 101 are placed in a box. A card is selected at random.Find the probability that the card has
\begin{enumerate}[label=(\roman*)]
	\item an even number 
	\item a square number
\end{enumerate}
\solution
%\begin{table}[H]
	\centering
\begin{tabular}{|c|c|c|}
\hline
Random variable &Value &Definition\\ \hline
\multirow{3}{*}{X} &0 &Slips of Rs 1\\
&1 &Slips of Rs 5\\
&2 &Slips of Rs 13\\ \hline
\multirow{2}{*}{Y} &0 &Box A\\
&1 &Box B\\\hline
\end{tabular}
\caption{}
\label{tab:Distribution}
\end{table}
See \tabref{tab:Distribution}.
\begin{align}
p_{Y}\brak{k}= \begin{cases} 
      \frac{1}{3} & {k=0} \\
      \frac{2}{3 }& {k=1} 
   \end{cases}
   \\
p_{Y|X}\brak{0|0} = \frac{19}{25}\, 
p_{Y|X}\brak{0|1} = \frac{6}{25}\,
p_{Y|X}\brak{1|0} = \frac{45}{50}\,
p_{Y|X}\brak{1|2} = \frac{5}{50}
\end{align}
The desired probability is the probability that a slip drawn at random is marked other than Rs 1,
\begin{align}
&=1-p_X\brak{0}\\
&= p_X(1) + p_X(2)
\end{align}
Using Bayes theorem,
\begin{align}
&= p_Y\brak{0} \times \pr{Y=0 | X=1} + p_Y\brak{1} \times \pr{Y=1|X=2}\\
&=\frac{1}{3} \times \frac{6}{25} + \frac{2}{3} \times \frac{5}{50}\\
&=\frac{11}{75}
\end{align}

\newpage

%\tableofcontents

\bigskip

\renewcommand{\thefigure}{\theenumi}
\renewcommand{\thetable}{\theenumi}
%\renewcommand{\theequation}{\theenumi}

%\begin{abstract}
%%\boldmath
%In this letter, an algorithm for evaluating the exact analytical bit error rate  (BER)  for the piecewise linear (PL) combiner for  multiple relays is presented. Previous results were available only for upto three relays. The algorithm is unique in the sense that  the actual mathematical expressions, that are prohibitively large, need not be explicitly obtained. The diversity gain due to multiple relays is shown through plots of the analytical BER, well supported by simulations. 
%
%\end{abstract}
% IEEEtran.cls defaults to using nonbold math in the Abstract.
% This preserves the distinction between vectors and scalars. However,
% if the journal you are submitting to favors bold math in the abstract,
% then you can use LaTeX's standard command \boldmath at the very start
% of the abstract to achieve this. Many IEEE journals frown on math
% in the abstract anyway.

% Note that keywords are not normally used for peerreview papers.
%\begin{IEEEkeywords}
%Cooperative diversity, decode and forward, piecewise linear
%\end{IEEEkeywords}



% For peer review papers, you can put extra information on the cover
% page as needed:
% \ifCLASSOPTIONpeerreview
% \begin{center} \bfseries EDICS Category: 3-BBND \end{center}
% \fi
%
% For peerreview papers, this IEEEtran command inserts a page break and
% creates the second title. It will be ignored for other modes.
%\IEEEpeerreviewmaketitle




\item
The king, queen and jack of clubs are removed from a deck of 52 playing cards and then well shuffled. Now one card is drawn at random from the remaining cards.  Determine the probability that the card is
\begin{enumerate}[label=(\roman*)]
\item a club
\item 10 of hearts
\end{enumerate}
\solution
%\begin{table}[H]
	\centering
\begin{tabular}{|c|c|c|}
\hline
Random variable &Value &Definition\\ \hline
\multirow{3}{*}{X} &0 &Slips of Rs 1\\
&1 &Slips of Rs 5\\
&2 &Slips of Rs 13\\ \hline
\multirow{2}{*}{Y} &0 &Box A\\
&1 &Box B\\\hline
\end{tabular}
\caption{}
\label{tab:Distribution}
\end{table}
See \tabref{tab:Distribution}.
\begin{align}
p_{Y}\brak{k}= \begin{cases} 
      \frac{1}{3} & {k=0} \\
      \frac{2}{3 }& {k=1} 
   \end{cases}
   \\
p_{Y|X}\brak{0|0} = \frac{19}{25}\, 
p_{Y|X}\brak{0|1} = \frac{6}{25}\,
p_{Y|X}\brak{1|0} = \frac{45}{50}\,
p_{Y|X}\brak{1|2} = \frac{5}{50}
\end{align}
The desired probability is the probability that a slip drawn at random is marked other than Rs 1,
\begin{align}
&=1-p_X\brak{0}\\
&= p_X(1) + p_X(2)
\end{align}
Using Bayes theorem,
\begin{align}
&= p_Y\brak{0} \times \pr{Y=0 | X=1} + p_Y\brak{1} \times \pr{Y=1|X=2}\\
&=\frac{1}{3} \times \frac{6}{25} + \frac{2}{3} \times \frac{5}{50}\\
&=\frac{11}{75}
\end{align}

\newpage

%\tableofcontents

\bigskip

\renewcommand{\thefigure}{\theenumi}
\renewcommand{\thetable}{\theenumi}
%\renewcommand{\theequation}{\theenumi}

%\begin{abstract}
%%\boldmath
%In this letter, an algorithm for evaluating the exact analytical bit error rate  (BER)  for the piecewise linear (PL) combiner for  multiple relays is presented. Previous results were available only for upto three relays. The algorithm is unique in the sense that  the actual mathematical expressions, that are prohibitively large, need not be explicitly obtained. The diversity gain due to multiple relays is shown through plots of the analytical BER, well supported by simulations. 
%
%\end{abstract}
% IEEEtran.cls defaults to using nonbold math in the Abstract.
% This preserves the distinction between vectors and scalars. However,
% if the journal you are submitting to favors bold math in the abstract,
% then you can use LaTeX's standard command \boldmath at the very start
% of the abstract to achieve this. Many IEEE journals frown on math
% in the abstract anyway.

% Note that keywords are not normally used for peerreview papers.
%\begin{IEEEkeywords}
%Cooperative diversity, decode and forward, piecewise linear
%\end{IEEEkeywords}



% For peer review papers, you can put extra information on the cover
% page as needed:
% \ifCLASSOPTIONpeerreview
% \begin{center} \bfseries EDICS Category: 3-BBND \end{center}
% \fi
%
% For peerreview papers, this IEEEtran command inserts a page break and
% creates the second title. It will be ignored for other modes.
%\IEEEpeerreviewmaketitle




\item A team of medical students doing their internship have to assist during surgeries
at a city hospital. The probabilities of surgeries rated as very complex, complex,
routine, simple or very simple are respectively, 0.15, 0.20, 0.31, 0.26, .08. Find
the probabilities that a particular surgery will be rated
\begin{enumerate}
	\item complex or very complex;
	\item neither very complex nor very simple;
	\item routine or complex
	\item routine or simple
\end{enumerate}
\solution
%\begin{table}[H]
	\centering
\begin{tabular}{|c|c|c|}
\hline
Random variable &Value &Definition\\ \hline
\multirow{3}{*}{X} &0 &Slips of Rs 1\\
&1 &Slips of Rs 5\\
&2 &Slips of Rs 13\\ \hline
\multirow{2}{*}{Y} &0 &Box A\\
&1 &Box B\\\hline
\end{tabular}
\caption{}
\label{tab:Distribution}
\end{table}
See \tabref{tab:Distribution}.
\begin{align}
p_{Y}\brak{k}= \begin{cases} 
      \frac{1}{3} & {k=0} \\
      \frac{2}{3 }& {k=1} 
   \end{cases}
   \\
p_{Y|X}\brak{0|0} = \frac{19}{25}\, 
p_{Y|X}\brak{0|1} = \frac{6}{25}\,
p_{Y|X}\brak{1|0} = \frac{45}{50}\,
p_{Y|X}\brak{1|2} = \frac{5}{50}
\end{align}
The desired probability is the probability that a slip drawn at random is marked other than Rs 1,
\begin{align}
&=1-p_X\brak{0}\\
&= p_X(1) + p_X(2)
\end{align}
Using Bayes theorem,
\begin{align}
&= p_Y\brak{0} \times \pr{Y=0 | X=1} + p_Y\brak{1} \times \pr{Y=1|X=2}\\
&=\frac{1}{3} \times \frac{6}{25} + \frac{2}{3} \times \frac{5}{50}\\
&=\frac{11}{75}
\end{align}

\newpage

%\tableofcontents

\bigskip

\renewcommand{\thefigure}{\theenumi}
\renewcommand{\thetable}{\theenumi}
%\renewcommand{\theequation}{\theenumi}

%\begin{abstract}
%%\boldmath
%In this letter, an algorithm for evaluating the exact analytical bit error rate  (BER)  for the piecewise linear (PL) combiner for  multiple relays is presented. Previous results were available only for upto three relays. The algorithm is unique in the sense that  the actual mathematical expressions, that are prohibitively large, need not be explicitly obtained. The diversity gain due to multiple relays is shown through plots of the analytical BER, well supported by simulations. 
%
%\end{abstract}
% IEEEtran.cls defaults to using nonbold math in the Abstract.
% This preserves the distinction between vectors and scalars. However,
% if the journal you are submitting to favors bold math in the abstract,
% then you can use LaTeX's standard command \boldmath at the very start
% of the abstract to achieve this. Many IEEE journals frown on math
% in the abstract anyway.

% Note that keywords are not normally used for peerreview papers.
%\begin{IEEEkeywords}
%Cooperative diversity, decode and forward, piecewise linear
%\end{IEEEkeywords}



% For peer review papers, you can put extra information on the cover
% page as needed:
% \ifCLASSOPTIONpeerreview
% \begin{center} \bfseries EDICS Category: 3-BBND \end{center}
% \fi
%
% For peerreview papers, this IEEEtran command inserts a page break and
% creates the second title. It will be ignored for other modes.
%\IEEEpeerreviewmaketitle




\item A card is selected from a pack of 52 cards.
\begin{enumerate}[label=(\alph*)]
    \item How many points are there in the sample space?
    \item Calculate the probability that the card is an ace of spades.
    \item Calculate the probability that the card is (i) an ace and (ii) black card.
\end{enumerate}
\solution
%Let $X$ be an bernoulli rv defined as in \tabref{tab:exemplar/11/16/3/26}.  Then, 
\begin{equation}
    p =
        \frac{4}{11} 
\end{equation}
\begin{table}[H]
	\centering
	\input{exemplar/11/16/3/26/tables/Table2.tex}
	\caption{}
        \label{tab:exemplar/11/16/3/26}
\end{table}

\item The probability that a non leap year selected at random will contain 53 sundays.
\\
\solution
%\begin{table}[H]
	\centering
\begin{tabular}{|c|c|c|}
\hline
Random variable &Value &Definition\\ \hline
\multirow{3}{*}{X} &0 &Slips of Rs 1\\
&1 &Slips of Rs 5\\
&2 &Slips of Rs 13\\ \hline
\multirow{2}{*}{Y} &0 &Box A\\
&1 &Box B\\\hline
\end{tabular}
\caption{}
\label{tab:Distribution}
\end{table}
See \tabref{tab:Distribution}.
\begin{align}
p_{Y}\brak{k}= \begin{cases} 
      \frac{1}{3} & {k=0} \\
      \frac{2}{3 }& {k=1} 
   \end{cases}
   \\
p_{Y|X}\brak{0|0} = \frac{19}{25}\, 
p_{Y|X}\brak{0|1} = \frac{6}{25}\,
p_{Y|X}\brak{1|0} = \frac{45}{50}\,
p_{Y|X}\brak{1|2} = \frac{5}{50}
\end{align}
The desired probability is the probability that a slip drawn at random is marked other than Rs 1,
\begin{align}
&=1-p_X\brak{0}\\
&= p_X(1) + p_X(2)
\end{align}
Using Bayes theorem,
\begin{align}
&= p_Y\brak{0} \times \pr{Y=0 | X=1} + p_Y\brak{1} \times \pr{Y=1|X=2}\\
&=\frac{1}{3} \times \frac{6}{25} + \frac{2}{3} \times \frac{5}{50}\\
&=\frac{11}{75}
\end{align}

\newpage

%\tableofcontents

\bigskip

\renewcommand{\thefigure}{\theenumi}
\renewcommand{\thetable}{\theenumi}
%\renewcommand{\theequation}{\theenumi}

%\begin{abstract}
%%\boldmath
%In this letter, an algorithm for evaluating the exact analytical bit error rate  (BER)  for the piecewise linear (PL) combiner for  multiple relays is presented. Previous results were available only for upto three relays. The algorithm is unique in the sense that  the actual mathematical expressions, that are prohibitively large, need not be explicitly obtained. The diversity gain due to multiple relays is shown through plots of the analytical BER, well supported by simulations. 
%
%\end{abstract}
% IEEEtran.cls defaults to using nonbold math in the Abstract.
% This preserves the distinction between vectors and scalars. However,
% if the journal you are submitting to favors bold math in the abstract,
% then you can use LaTeX's standard command \boldmath at the very start
% of the abstract to achieve this. Many IEEE journals frown on math
% in the abstract anyway.

% Note that keywords are not normally used for peerreview papers.
%\begin{IEEEkeywords}
%Cooperative diversity, decode and forward, piecewise linear
%\end{IEEEkeywords}



% For peer review papers, you can put extra information on the cover
% page as needed:
% \ifCLASSOPTIONpeerreview
% \begin{center} \bfseries EDICS Category: 3-BBND \end{center}
% \fi
%
% For peerreview papers, this IEEEtran command inserts a page break and
% creates the second title. It will be ignored for other modes.
%\IEEEpeerreviewmaketitle




\item One of the four persons John, Rita, Aslam or Gurpreet will be promoted next
month. Consequently the sample space consists of four elementary outcomes
S = {John promoted, Rita promoted, Aslam promoted, Gurpreet promoted}
You are told that the chances of John’s promotion is same as that of Gurpreet,
Rita’s chances of promotion are twice as likely as Johns. Aslam’s chances are
four times that of John.
\begin{enumerate}
	\item Determine
	\begin{enumerate}
		\item P (John promoted)
		\item P (Rita promoted)
		\item P (Aslam promoted)
		\item P (Gurpreet promoted)
	\end{enumerate}
	\item If A = {John promoted or Gurpreet promoted}, find P (A).
\end{enumerate}
\solution
%\begin{table}[H]
	\centering
\begin{tabular}{|c|c|c|}
\hline
Random variable &Value &Definition\\ \hline
\multirow{3}{*}{X} &0 &Slips of Rs 1\\
&1 &Slips of Rs 5\\
&2 &Slips of Rs 13\\ \hline
\multirow{2}{*}{Y} &0 &Box A\\
&1 &Box B\\\hline
\end{tabular}
\caption{}
\label{tab:Distribution}
\end{table}
See \tabref{tab:Distribution}.
\begin{align}
p_{Y}\brak{k}= \begin{cases} 
      \frac{1}{3} & {k=0} \\
      \frac{2}{3 }& {k=1} 
   \end{cases}
   \\
p_{Y|X}\brak{0|0} = \frac{19}{25}\, 
p_{Y|X}\brak{0|1} = \frac{6}{25}\,
p_{Y|X}\brak{1|0} = \frac{45}{50}\,
p_{Y|X}\brak{1|2} = \frac{5}{50}
\end{align}
The desired probability is the probability that a slip drawn at random is marked other than Rs 1,
\begin{align}
&=1-p_X\brak{0}\\
&= p_X(1) + p_X(2)
\end{align}
Using Bayes theorem,
\begin{align}
&= p_Y\brak{0} \times \pr{Y=0 | X=1} + p_Y\brak{1} \times \pr{Y=1|X=2}\\
&=\frac{1}{3} \times \frac{6}{25} + \frac{2}{3} \times \frac{5}{50}\\
&=\frac{11}{75}
\end{align}

\newpage

%\tableofcontents

\bigskip

\renewcommand{\thefigure}{\theenumi}
\renewcommand{\thetable}{\theenumi}
%\renewcommand{\theequation}{\theenumi}

%\begin{abstract}
%%\boldmath
%In this letter, an algorithm for evaluating the exact analytical bit error rate  (BER)  for the piecewise linear (PL) combiner for  multiple relays is presented. Previous results were available only for upto three relays. The algorithm is unique in the sense that  the actual mathematical expressions, that are prohibitively large, need not be explicitly obtained. The diversity gain due to multiple relays is shown through plots of the analytical BER, well supported by simulations. 
%
%\end{abstract}
% IEEEtran.cls defaults to using nonbold math in the Abstract.
% This preserves the distinction between vectors and scalars. However,
% if the journal you are submitting to favors bold math in the abstract,
% then you can use LaTeX's standard command \boldmath at the very start
% of the abstract to achieve this. Many IEEE journals frown on math
% in the abstract anyway.

% Note that keywords are not normally used for peerreview papers.
%\begin{IEEEkeywords}
%Cooperative diversity, decode and forward, piecewise linear
%\end{IEEEkeywords}



% For peer review papers, you can put extra information on the cover
% page as needed:
% \ifCLASSOPTIONpeerreview
% \begin{center} \bfseries EDICS Category: 3-BBND \end{center}
% \fi
%
% For peerreview papers, this IEEEtran command inserts a page break and
% creates the second title. It will be ignored for other modes.
%\IEEEpeerreviewmaketitle




\item A card is drawn from a deck of 52 cards. Find the probability of getting a king or a heart or a red card.\\
\solution
%\begin{table}[H]
	\centering
\begin{tabular}{|c|c|c|}
\hline
Random variable &Value &Definition\\ \hline
\multirow{3}{*}{X} &0 &Slips of Rs 1\\
&1 &Slips of Rs 5\\
&2 &Slips of Rs 13\\ \hline
\multirow{2}{*}{Y} &0 &Box A\\
&1 &Box B\\\hline
\end{tabular}
\caption{}
\label{tab:Distribution}
\end{table}
See \tabref{tab:Distribution}.
\begin{align}
p_{Y}\brak{k}= \begin{cases} 
      \frac{1}{3} & {k=0} \\
      \frac{2}{3 }& {k=1} 
   \end{cases}
   \\
p_{Y|X}\brak{0|0} = \frac{19}{25}\, 
p_{Y|X}\brak{0|1} = \frac{6}{25}\,
p_{Y|X}\brak{1|0} = \frac{45}{50}\,
p_{Y|X}\brak{1|2} = \frac{5}{50}
\end{align}
The desired probability is the probability that a slip drawn at random is marked other than Rs 1,
\begin{align}
&=1-p_X\brak{0}\\
&= p_X(1) + p_X(2)
\end{align}
Using Bayes theorem,
\begin{align}
&= p_Y\brak{0} \times \pr{Y=0 | X=1} + p_Y\brak{1} \times \pr{Y=1|X=2}\\
&=\frac{1}{3} \times \frac{6}{25} + \frac{2}{3} \times \frac{5}{50}\\
&=\frac{11}{75}
\end{align}

\newpage

%\tableofcontents

\bigskip

\renewcommand{\thefigure}{\theenumi}
\renewcommand{\thetable}{\theenumi}
%\renewcommand{\theequation}{\theenumi}

%\begin{abstract}
%%\boldmath
%In this letter, an algorithm for evaluating the exact analytical bit error rate  (BER)  for the piecewise linear (PL) combiner for  multiple relays is presented. Previous results were available only for upto three relays. The algorithm is unique in the sense that  the actual mathematical expressions, that are prohibitively large, need not be explicitly obtained. The diversity gain due to multiple relays is shown through plots of the analytical BER, well supported by simulations. 
%
%\end{abstract}
% IEEEtran.cls defaults to using nonbold math in the Abstract.
% This preserves the distinction between vectors and scalars. However,
% if the journal you are submitting to favors bold math in the abstract,
% then you can use LaTeX's standard command \boldmath at the very start
% of the abstract to achieve this. Many IEEE journals frown on math
% in the abstract anyway.

% Note that keywords are not normally used for peerreview papers.
%\begin{IEEEkeywords}
%Cooperative diversity, decode and forward, piecewise linear
%\end{IEEEkeywords}



% For peer review papers, you can put extra information on the cover
% page as needed:
% \ifCLASSOPTIONpeerreview
% \begin{center} \bfseries EDICS Category: 3-BBND \end{center}
% \fi
%
% For peerreview papers, this IEEEtran command inserts a page break and
% creates the second title. It will be ignored for other modes.
%\IEEEpeerreviewmaketitle




\item The probability that a student will pass his examination is 0.73, the probability of
the student getting a compartment is 0.13, and the probability that the student will
either pass or get compartment is 0.96. State True or False.\\
\solution
%\begin{table}[H]
	\centering
\begin{tabular}{|c|c|c|}
\hline
Random variable &Value &Definition\\ \hline
\multirow{3}{*}{X} &0 &Slips of Rs 1\\
&1 &Slips of Rs 5\\
&2 &Slips of Rs 13\\ \hline
\multirow{2}{*}{Y} &0 &Box A\\
&1 &Box B\\\hline
\end{tabular}
\caption{}
\label{tab:Distribution}
\end{table}
See \tabref{tab:Distribution}.
\begin{align}
p_{Y}\brak{k}= \begin{cases} 
      \frac{1}{3} & {k=0} \\
      \frac{2}{3 }& {k=1} 
   \end{cases}
   \\
p_{Y|X}\brak{0|0} = \frac{19}{25}\, 
p_{Y|X}\brak{0|1} = \frac{6}{25}\,
p_{Y|X}\brak{1|0} = \frac{45}{50}\,
p_{Y|X}\brak{1|2} = \frac{5}{50}
\end{align}
The desired probability is the probability that a slip drawn at random is marked other than Rs 1,
\begin{align}
&=1-p_X\brak{0}\\
&= p_X(1) + p_X(2)
\end{align}
Using Bayes theorem,
\begin{align}
&= p_Y\brak{0} \times \pr{Y=0 | X=1} + p_Y\brak{1} \times \pr{Y=1|X=2}\\
&=\frac{1}{3} \times \frac{6}{25} + \frac{2}{3} \times \frac{5}{50}\\
&=\frac{11}{75}
\end{align}

\newpage

%\tableofcontents

\bigskip

\renewcommand{\thefigure}{\theenumi}
\renewcommand{\thetable}{\theenumi}
%\renewcommand{\theequation}{\theenumi}

%\begin{abstract}
%%\boldmath
%In this letter, an algorithm for evaluating the exact analytical bit error rate  (BER)  for the piecewise linear (PL) combiner for  multiple relays is presented. Previous results were available only for upto three relays. The algorithm is unique in the sense that  the actual mathematical expressions, that are prohibitively large, need not be explicitly obtained. The diversity gain due to multiple relays is shown through plots of the analytical BER, well supported by simulations. 
%
%\end{abstract}
% IEEEtran.cls defaults to using nonbold math in the Abstract.
% This preserves the distinction between vectors and scalars. However,
% if the journal you are submitting to favors bold math in the abstract,
% then you can use LaTeX's standard command \boldmath at the very start
% of the abstract to achieve this. Many IEEE journals frown on math
% in the abstract anyway.

% Note that keywords are not normally used for peerreview papers.
%\begin{IEEEkeywords}
%Cooperative diversity, decode and forward, piecewise linear
%\end{IEEEkeywords}



% For peer review papers, you can put extra information on the cover
% page as needed:
% \ifCLASSOPTIONpeerreview
% \begin{center} \bfseries EDICS Category: 3-BBND \end{center}
% \fi
%
% For peerreview papers, this IEEEtran command inserts a page break and
% creates the second title. It will be ignored for other modes.
%\IEEEpeerreviewmaketitle




\item A card is selected from a pack of 52 cards\\
\begin{enumerate}[label=(\alph*)]
\item How many points are there in the sample space?
\item Calculate the probability that the cards is an ace of spades.
\item Calculate the probability that the card is (i) an ace (ii)black card.\\
\end{enumerate}
%\input{ncert/11/16/3/4_1/Prob_4.tex}
\item In a non-leap year, the probability of having 53 tuesdays or 53 wednesdays is\\
\solution
%A non-leap year has a total of 365 days, and a week has 7 days.\\
So it can be expressed as 
\begin{align}
365\text{days} &=52\times 7+1 \text{day}
\end{align}
$\implies$ 52 tuesdays or wednesdays\\
Random variable X denotes the days of a week
\begin{align}
p_X\brak{k}&=\frac{1}{7}; \quad \brak{1<k<7}
\end{align}
So the probability of extra day being tuesday or wednesday is
\begin{align}
p_X\brak{3}+p_X\brak{4}&=\frac{1}{7}+\frac{1}{7}=\frac{2}{7}
\end{align}



\item There are 1000 sealed envelopes in a box, 10 of them contain a cash prize of
Rs 100 each, 100 of them contain a cash prize of Rs 50 each and 200 of them
contain a cash prize of Rs 10 each and rest do not contain any cash prize. If they
are well shuffled and an envelope is picked up out, what is the probability that it
contains no cash prize?\\
\solution
%\begin{table}[H]
	\centering
\begin{tabular}{|c|c|c|}
\hline
Random variable &Value &Definition\\ \hline
\multirow{3}{*}{X} &0 &Slips of Rs 1\\
&1 &Slips of Rs 5\\
&2 &Slips of Rs 13\\ \hline
\multirow{2}{*}{Y} &0 &Box A\\
&1 &Box B\\\hline
\end{tabular}
\caption{}
\label{tab:Distribution}
\end{table}
See \tabref{tab:Distribution}.
\begin{align}
p_{Y}\brak{k}= \begin{cases} 
      \frac{1}{3} & {k=0} \\
      \frac{2}{3 }& {k=1} 
   \end{cases}
   \\
p_{Y|X}\brak{0|0} = \frac{19}{25}\, 
p_{Y|X}\brak{0|1} = \frac{6}{25}\,
p_{Y|X}\brak{1|0} = \frac{45}{50}\,
p_{Y|X}\brak{1|2} = \frac{5}{50}
\end{align}
The desired probability is the probability that a slip drawn at random is marked other than Rs 1,
\begin{align}
&=1-p_X\brak{0}\\
&= p_X(1) + p_X(2)
\end{align}
Using Bayes theorem,
\begin{align}
&= p_Y\brak{0} \times \pr{Y=0 | X=1} + p_Y\brak{1} \times \pr{Y=1|X=2}\\
&=\frac{1}{3} \times \frac{6}{25} + \frac{2}{3} \times \frac{5}{50}\\
&=\frac{11}{75}
\end{align}

\newpage

%\tableofcontents

\bigskip

\renewcommand{\thefigure}{\theenumi}
\renewcommand{\thetable}{\theenumi}
%\renewcommand{\theequation}{\theenumi}

%\begin{abstract}
%%\boldmath
%In this letter, an algorithm for evaluating the exact analytical bit error rate  (BER)  for the piecewise linear (PL) combiner for  multiple relays is presented. Previous results were available only for upto three relays. The algorithm is unique in the sense that  the actual mathematical expressions, that are prohibitively large, need not be explicitly obtained. The diversity gain due to multiple relays is shown through plots of the analytical BER, well supported by simulations. 
%
%\end{abstract}
% IEEEtran.cls defaults to using nonbold math in the Abstract.
% This preserves the distinction between vectors and scalars. However,
% if the journal you are submitting to favors bold math in the abstract,
% then you can use LaTeX's standard command \boldmath at the very start
% of the abstract to achieve this. Many IEEE journals frown on math
% in the abstract anyway.

% Note that keywords are not normally used for peerreview papers.
%\begin{IEEEkeywords}
%Cooperative diversity, decode and forward, piecewise linear
%\end{IEEEkeywords}



% For peer review papers, you can put extra information on the cover
% page as needed:
% \ifCLASSOPTIONpeerreview
% \begin{center} \bfseries EDICS Category: 3-BBND \end{center}
% \fi
%
% For peerreview papers, this IEEEtran command inserts a page break and
% creates the second title. It will be ignored for other modes.
%\IEEEpeerreviewmaketitle




\item 
A die is thrown and a card is selected at random from a deck of 52 playing cards. The probability of getting an even number on the die and a spade card.\\
\solution
%\begin{table}[H]
	\centering
\begin{tabular}{|c|c|c|}
\hline
Random variable &Value &Definition\\ \hline
\multirow{3}{*}{X} &0 &Slips of Rs 1\\
&1 &Slips of Rs 5\\
&2 &Slips of Rs 13\\ \hline
\multirow{2}{*}{Y} &0 &Box A\\
&1 &Box B\\\hline
\end{tabular}
\caption{}
\label{tab:Distribution}
\end{table}
See \tabref{tab:Distribution}.
\begin{align}
p_{Y}\brak{k}= \begin{cases} 
      \frac{1}{3} & {k=0} \\
      \frac{2}{3 }& {k=1} 
   \end{cases}
   \\
p_{Y|X}\brak{0|0} = \frac{19}{25}\, 
p_{Y|X}\brak{0|1} = \frac{6}{25}\,
p_{Y|X}\brak{1|0} = \frac{45}{50}\,
p_{Y|X}\brak{1|2} = \frac{5}{50}
\end{align}
The desired probability is the probability that a slip drawn at random is marked other than Rs 1,
\begin{align}
&=1-p_X\brak{0}\\
&= p_X(1) + p_X(2)
\end{align}
Using Bayes theorem,
\begin{align}
&= p_Y\brak{0} \times \pr{Y=0 | X=1} + p_Y\brak{1} \times \pr{Y=1|X=2}\\
&=\frac{1}{3} \times \frac{6}{25} + \frac{2}{3} \times \frac{5}{50}\\
&=\frac{11}{75}
\end{align}

\newpage

%\tableofcontents

\bigskip

\renewcommand{\thefigure}{\theenumi}
\renewcommand{\thetable}{\theenumi}
%\renewcommand{\theequation}{\theenumi}

%\begin{abstract}
%%\boldmath
%In this letter, an algorithm for evaluating the exact analytical bit error rate  (BER)  for the piecewise linear (PL) combiner for  multiple relays is presented. Previous results were available only for upto three relays. The algorithm is unique in the sense that  the actual mathematical expressions, that are prohibitively large, need not be explicitly obtained. The diversity gain due to multiple relays is shown through plots of the analytical BER, well supported by simulations. 
%
%\end{abstract}
% IEEEtran.cls defaults to using nonbold math in the Abstract.
% This preserves the distinction between vectors and scalars. However,
% if the journal you are submitting to favors bold math in the abstract,
% then you can use LaTeX's standard command \boldmath at the very start
% of the abstract to achieve this. Many IEEE journals frown on math
% in the abstract anyway.

% Note that keywords are not normally used for peerreview papers.
%\begin{IEEEkeywords}
%Cooperative diversity, decode and forward, piecewise linear
%\end{IEEEkeywords}



% For peer review papers, you can put extra information on the cover
% page as needed:
% \ifCLASSOPTIONpeerreview
% \begin{center} \bfseries EDICS Category: 3-BBND \end{center}
% \fi
%
% For peerreview papers, this IEEEtran command inserts a page break and
% creates the second title. It will be ignored for other modes.
%\IEEEpeerreviewmaketitle




\item
If 4-digit numbers greater than 5,000 are randomly formed from the digits 0, 1, 3, 5, and 7, what is the probability of forming a number divisible by 5 when:
\begin{enumerate}
    \item The digits are repeated?
    \item The repetition of digits is not allowed?
\end{enumerate}
\solution
%\begin{table}[H]
	\centering
\begin{tabular}{|c|c|c|}
\hline
Random variable &Value &Definition\\ \hline
\multirow{3}{*}{X} &0 &Slips of Rs 1\\
&1 &Slips of Rs 5\\
&2 &Slips of Rs 13\\ \hline
\multirow{2}{*}{Y} &0 &Box A\\
&1 &Box B\\\hline
\end{tabular}
\caption{}
\label{tab:Distribution}
\end{table}
See \tabref{tab:Distribution}.
\begin{align}
p_{Y}\brak{k}= \begin{cases} 
      \frac{1}{3} & {k=0} \\
      \frac{2}{3 }& {k=1} 
   \end{cases}
   \\
p_{Y|X}\brak{0|0} = \frac{19}{25}\, 
p_{Y|X}\brak{0|1} = \frac{6}{25}\,
p_{Y|X}\brak{1|0} = \frac{45}{50}\,
p_{Y|X}\brak{1|2} = \frac{5}{50}
\end{align}
The desired probability is the probability that a slip drawn at random is marked other than Rs 1,
\begin{align}
&=1-p_X\brak{0}\\
&= p_X(1) + p_X(2)
\end{align}
Using Bayes theorem,
\begin{align}
&= p_Y\brak{0} \times \pr{Y=0 | X=1} + p_Y\brak{1} \times \pr{Y=1|X=2}\\
&=\frac{1}{3} \times \frac{6}{25} + \frac{2}{3} \times \frac{5}{50}\\
&=\frac{11}{75}
\end{align}

\newpage

%\tableofcontents

\bigskip

\renewcommand{\thefigure}{\theenumi}
\renewcommand{\thetable}{\theenumi}
%\renewcommand{\theequation}{\theenumi}

%\begin{abstract}
%%\boldmath
%In this letter, an algorithm for evaluating the exact analytical bit error rate  (BER)  for the piecewise linear (PL) combiner for  multiple relays is presented. Previous results were available only for upto three relays. The algorithm is unique in the sense that  the actual mathematical expressions, that are prohibitively large, need not be explicitly obtained. The diversity gain due to multiple relays is shown through plots of the analytical BER, well supported by simulations. 
%
%\end{abstract}
% IEEEtran.cls defaults to using nonbold math in the Abstract.
% This preserves the distinction between vectors and scalars. However,
% if the journal you are submitting to favors bold math in the abstract,
% then you can use LaTeX's standard command \boldmath at the very start
% of the abstract to achieve this. Many IEEE journals frown on math
% in the abstract anyway.

% Note that keywords are not normally used for peerreview papers.
%\begin{IEEEkeywords}
%Cooperative diversity, decode and forward, piecewise linear
%\end{IEEEkeywords}



% For peer review papers, you can put extra information on the cover
% page as needed:
% \ifCLASSOPTIONpeerreview
% \begin{center} \bfseries EDICS Category: 3-BBND \end{center}
% \fi
%
% For peerreview papers, this IEEEtran command inserts a page break and
% creates the second title. It will be ignored for other modes.
%\IEEEpeerreviewmaketitle




\item Consider the probability space $\brak{\Omega, \mathcal{G}, P}$ where $\Omega = [0,2]$ and $\mathcal{G} = \cbrak{\phi, \Omega, [0,1], (1,2]}$. Let $X$ and $Y$ be two functions on $\Omega$ defined as
\begin{align*}
    X(\omega) = 
    \begin{cases}
        1 & \text{if }\omega \in [0, 1]\\
        2 & \text{if }\omega \in (1, 2]
    \end{cases}
\end{align*}
and
\begin{align*}
    Y(\omega) = 
    \begin{cases}
        2 & \text{if }\omega \in [0, 1.5]\\
        3 & \text{if }\omega \in (1.5, 2].
    \end{cases}
\end{align*}
Then which one of the following statements is true?
\begin{enumerate}
    \item [(A)] $X$ is a random variable with respect to $\mathcal{G}$, but $Y$ is not a random variable with respect to $\mathcal{G}$.
    \item [(B)] $Y$ is a random variable with respect to $\mathcal{G}$, but $X$ is not a random variable with respect to $\mathcal{G}$.
    \item [(C)] Neither $X$ nor $Y$ is a random variable with respect to $\mathcal{G}$.
    \item [(D)] Both $X$ and $Y$ are random variables with respect to $\mathcal{G}$.
\end{enumerate} \hfill (GATE ST 2023)\\
\solution
%\begin{table}[H]
	\centering
\begin{tabular}{|c|c|c|}
\hline
Random variable &Value &Definition\\ \hline
\multirow{3}{*}{X} &0 &Slips of Rs 1\\
&1 &Slips of Rs 5\\
&2 &Slips of Rs 13\\ \hline
\multirow{2}{*}{Y} &0 &Box A\\
&1 &Box B\\\hline
\end{tabular}
\caption{}
\label{tab:Distribution}
\end{table}
See \tabref{tab:Distribution}.
\begin{align}
p_{Y}\brak{k}= \begin{cases} 
      \frac{1}{3} & {k=0} \\
      \frac{2}{3 }& {k=1} 
   \end{cases}
   \\
p_{Y|X}\brak{0|0} = \frac{19}{25}\, 
p_{Y|X}\brak{0|1} = \frac{6}{25}\,
p_{Y|X}\brak{1|0} = \frac{45}{50}\,
p_{Y|X}\brak{1|2} = \frac{5}{50}
\end{align}
The desired probability is the probability that a slip drawn at random is marked other than Rs 1,
\begin{align}
&=1-p_X\brak{0}\\
&= p_X(1) + p_X(2)
\end{align}
Using Bayes theorem,
\begin{align}
&= p_Y\brak{0} \times \pr{Y=0 | X=1} + p_Y\brak{1} \times \pr{Y=1|X=2}\\
&=\frac{1}{3} \times \frac{6}{25} + \frac{2}{3} \times \frac{5}{50}\\
&=\frac{11}{75}
\end{align}

\newpage

%\tableofcontents

\bigskip

\renewcommand{\thefigure}{\theenumi}
\renewcommand{\thetable}{\theenumi}
%\renewcommand{\theequation}{\theenumi}

%\begin{abstract}
%%\boldmath
%In this letter, an algorithm for evaluating the exact analytical bit error rate  (BER)  for the piecewise linear (PL) combiner for  multiple relays is presented. Previous results were available only for upto three relays. The algorithm is unique in the sense that  the actual mathematical expressions, that are prohibitively large, need not be explicitly obtained. The diversity gain due to multiple relays is shown through plots of the analytical BER, well supported by simulations. 
%
%\end{abstract}
% IEEEtran.cls defaults to using nonbold math in the Abstract.
% This preserves the distinction between vectors and scalars. However,
% if the journal you are submitting to favors bold math in the abstract,
% then you can use LaTeX's standard command \boldmath at the very start
% of the abstract to achieve this. Many IEEE journals frown on math
% in the abstract anyway.

% Note that keywords are not normally used for peerreview papers.
%\begin{IEEEkeywords}
%Cooperative diversity, decode and forward, piecewise linear
%\end{IEEEkeywords}



% For peer review papers, you can put extra information on the cover
% page as needed:
% \ifCLASSOPTIONpeerreview
% \begin{center} \bfseries EDICS Category: 3-BBND \end{center}
% \fi
%
% For peerreview papers, this IEEEtran command inserts a page break and
% creates the second title. It will be ignored for other modes.
%\IEEEpeerreviewmaketitle




	\item  A die is loaded in such a way that each odd number is twice as likely to occur as
each even number. Find $P(G)$, where $G$ is the event that a number greater than
3 occurs on a single roll of the die.
\\
\solution
		%\begin{table}[H]
	\centering
\begin{tabular}{|c|c|c|}
\hline
Random variable &Value &Definition\\ \hline
\multirow{3}{*}{X} &0 &Slips of Rs 1\\
&1 &Slips of Rs 5\\
&2 &Slips of Rs 13\\ \hline
\multirow{2}{*}{Y} &0 &Box A\\
&1 &Box B\\\hline
\end{tabular}
\caption{}
\label{tab:Distribution}
\end{table}
See \tabref{tab:Distribution}.
\begin{align}
p_{Y}\brak{k}= \begin{cases} 
      \frac{1}{3} & {k=0} \\
      \frac{2}{3 }& {k=1} 
   \end{cases}
   \\
p_{Y|X}\brak{0|0} = \frac{19}{25}\, 
p_{Y|X}\brak{0|1} = \frac{6}{25}\,
p_{Y|X}\brak{1|0} = \frac{45}{50}\,
p_{Y|X}\brak{1|2} = \frac{5}{50}
\end{align}
The desired probability is the probability that a slip drawn at random is marked other than Rs 1,
\begin{align}
&=1-p_X\brak{0}\\
&= p_X(1) + p_X(2)
\end{align}
Using Bayes theorem,
\begin{align}
&= p_Y\brak{0} \times \pr{Y=0 | X=1} + p_Y\brak{1} \times \pr{Y=1|X=2}\\
&=\frac{1}{3} \times \frac{6}{25} + \frac{2}{3} \times \frac{5}{50}\\
&=\frac{11}{75}
\end{align}

\newpage

%\tableofcontents

\bigskip

\renewcommand{\thefigure}{\theenumi}
\renewcommand{\thetable}{\theenumi}
%\renewcommand{\theequation}{\theenumi}

%\begin{abstract}
%%\boldmath
%In this letter, an algorithm for evaluating the exact analytical bit error rate  (BER)  for the piecewise linear (PL) combiner for  multiple relays is presented. Previous results were available only for upto three relays. The algorithm is unique in the sense that  the actual mathematical expressions, that are prohibitively large, need not be explicitly obtained. The diversity gain due to multiple relays is shown through plots of the analytical BER, well supported by simulations. 
%
%\end{abstract}
% IEEEtran.cls defaults to using nonbold math in the Abstract.
% This preserves the distinction between vectors and scalars. However,
% if the journal you are submitting to favors bold math in the abstract,
% then you can use LaTeX's standard command \boldmath at the very start
% of the abstract to achieve this. Many IEEE journals frown on math
% in the abstract anyway.

% Note that keywords are not normally used for peerreview papers.
%\begin{IEEEkeywords}
%Cooperative diversity, decode and forward, piecewise linear
%\end{IEEEkeywords}



% For peer review papers, you can put extra information on the cover
% page as needed:
% \ifCLASSOPTIONpeerreview
% \begin{center} \bfseries EDICS Category: 3-BBND \end{center}
% \fi
%
% For peerreview papers, this IEEEtran command inserts a page break and
% creates the second title. It will be ignored for other modes.
%\IEEEpeerreviewmaketitle




	\item All the jacks, queens and kings are removed from a deck of 52 playing cards. The remaining cards are well shuffled and then one card is drawn at random. Giving ace a value 1 similar value for other cards, find the probability that the card has a value 
		\begin{enumerate}
			\item 7
			\item greater than 7
			\item less than 7
		\end{enumerate}
		%Number of cards left after removing all jacks, queens and kings 
\begin{align}
N	= 52 - 4\times 3
	= 40
\end{align}
%\begin{table}[H]
%\def\arraystretch{1.2}
%\begin{tabular}{|c|c|c|}
%\hline
%	\textbf{Parameter} &\textbf{Value} &\textbf{Description}\\ \hline
%	$X$ &1-10 &Represents the value of the card picked \\ \hline
%\end{tabular}
%\end{table}
Let $1 \le X \le 10$ be the value of the card picked.  Then,
\begin{align}
	p_X(k) &= \Pr(X=k)\ \forall\ 1 \leq k \leq 10\\
	&= \frac{4\times 1}{40}\\
	&= \frac{1}{10}\\
	\therefore p_X(k) &= 
	\begin{cases}
		\frac{1}{10} & 1 \leq k \leq 10\\
		0 & \text{otherwise}
	\end{cases}
\end{align}
and
\begin{align}
	F_{X}(k) &= \sum_{m=0}^{k}p_{X}(m) \quad 1 \leq k \leq 10\\
	&= \frac{k}{10}\\
	\therefore F_{X}(k) &= 
	\begin{cases}
		0 & k \leq 0\\
		\frac{k}{10} & 1\leq k \leq 10\\
		1 & k > 10 
	\end{cases}
\end{align}
\begin{enumerate}
	\item Probability that card has value equal to 7 is
		\begin{align}
			 p_{X}(7)
			= \frac{1}{10}
		\end{align}
	\item Probability that card has value greater than 7 is
		\begin{align}
			1 - F_X(7)
			&= 1 - \frac{7}{10}
			\\
			&= \frac{3}{10}
		\end{align}
	\item Probability that card has value less than 7 is
		\begin{align}
			 F_{X}(6)
			=\frac{6}{10}
		\end{align}
\end{enumerate}

  \item A Lot consists of 48 mobile phones of which 42 are good, 3 have only minor defects and 3 have major defects.Varnika will buy a phone if it is good but the trader will only buy a mobile if it has no major defects. One phone is selected at random from the lot. What is the probability that it is
\begin{enumerate}
	\item acceptable to Varnika?
            \item acceptable to the trader?
\end{enumerate}
\solution
	%\begin{table}[H]
	\centering
\begin{tabular}{|c|c|c|}
\hline
Random variable &Value &Definition\\ \hline
\multirow{3}{*}{X} &0 &Slips of Rs 1\\
&1 &Slips of Rs 5\\
&2 &Slips of Rs 13\\ \hline
\multirow{2}{*}{Y} &0 &Box A\\
&1 &Box B\\\hline
\end{tabular}
\caption{}
\label{tab:Distribution}
\end{table}
See \tabref{tab:Distribution}.
\begin{align}
p_{Y}\brak{k}= \begin{cases} 
      \frac{1}{3} & {k=0} \\
      \frac{2}{3 }& {k=1} 
   \end{cases}
   \\
p_{Y|X}\brak{0|0} = \frac{19}{25}\, 
p_{Y|X}\brak{0|1} = \frac{6}{25}\,
p_{Y|X}\brak{1|0} = \frac{45}{50}\,
p_{Y|X}\brak{1|2} = \frac{5}{50}
\end{align}
The desired probability is the probability that a slip drawn at random is marked other than Rs 1,
\begin{align}
&=1-p_X\brak{0}\\
&= p_X(1) + p_X(2)
\end{align}
Using Bayes theorem,
\begin{align}
&= p_Y\brak{0} \times \pr{Y=0 | X=1} + p_Y\brak{1} \times \pr{Y=1|X=2}\\
&=\frac{1}{3} \times \frac{6}{25} + \frac{2}{3} \times \frac{5}{50}\\
&=\frac{11}{75}
\end{align}

\newpage

%\tableofcontents

\bigskip

\renewcommand{\thefigure}{\theenumi}
\renewcommand{\thetable}{\theenumi}
%\renewcommand{\theequation}{\theenumi}

%\begin{abstract}
%%\boldmath
%In this letter, an algorithm for evaluating the exact analytical bit error rate  (BER)  for the piecewise linear (PL) combiner for  multiple relays is presented. Previous results were available only for upto three relays. The algorithm is unique in the sense that  the actual mathematical expressions, that are prohibitively large, need not be explicitly obtained. The diversity gain due to multiple relays is shown through plots of the analytical BER, well supported by simulations. 
%
%\end{abstract}
% IEEEtran.cls defaults to using nonbold math in the Abstract.
% This preserves the distinction between vectors and scalars. However,
% if the journal you are submitting to favors bold math in the abstract,
% then you can use LaTeX's standard command \boldmath at the very start
% of the abstract to achieve this. Many IEEE journals frown on math
% in the abstract anyway.

% Note that keywords are not normally used for peerreview papers.
%\begin{IEEEkeywords}
%Cooperative diversity, decode and forward, piecewise linear
%\end{IEEEkeywords}



% For peer review papers, you can put extra information on the cover
% page as needed:
% \ifCLASSOPTIONpeerreview
% \begin{center} \bfseries EDICS Category: 3-BBND \end{center}
% \fi
%
% For peerreview papers, this IEEEtran command inserts a page break and
% creates the second title. It will be ignored for other modes.
%\IEEEpeerreviewmaketitle




 \item A student says that if you throw a die, it will show up 1 or not 1. Therefore, the probability of getting 1 and the probability of getting 'not 1' each is equal to $\frac{1}{2}$. Is this correct? Give reasons.\\
 \solution
        %\begin{table}[H]
	\centering
\begin{tabular}{|c|c|c|}
\hline
Random variable &Value &Definition\\ \hline
\multirow{3}{*}{X} &0 &Slips of Rs 1\\
&1 &Slips of Rs 5\\
&2 &Slips of Rs 13\\ \hline
\multirow{2}{*}{Y} &0 &Box A\\
&1 &Box B\\\hline
\end{tabular}
\caption{}
\label{tab:Distribution}
\end{table}
See \tabref{tab:Distribution}.
\begin{align}
p_{Y}\brak{k}= \begin{cases} 
      \frac{1}{3} & {k=0} \\
      \frac{2}{3 }& {k=1} 
   \end{cases}
   \\
p_{Y|X}\brak{0|0} = \frac{19}{25}\, 
p_{Y|X}\brak{0|1} = \frac{6}{25}\,
p_{Y|X}\brak{1|0} = \frac{45}{50}\,
p_{Y|X}\brak{1|2} = \frac{5}{50}
\end{align}
The desired probability is the probability that a slip drawn at random is marked other than Rs 1,
\begin{align}
&=1-p_X\brak{0}\\
&= p_X(1) + p_X(2)
\end{align}
Using Bayes theorem,
\begin{align}
&= p_Y\brak{0} \times \pr{Y=0 | X=1} + p_Y\brak{1} \times \pr{Y=1|X=2}\\
&=\frac{1}{3} \times \frac{6}{25} + \frac{2}{3} \times \frac{5}{50}\\
&=\frac{11}{75}
\end{align}

\newpage

%\tableofcontents

\bigskip

\renewcommand{\thefigure}{\theenumi}
\renewcommand{\thetable}{\theenumi}
%\renewcommand{\theequation}{\theenumi}

%\begin{abstract}
%%\boldmath
%In this letter, an algorithm for evaluating the exact analytical bit error rate  (BER)  for the piecewise linear (PL) combiner for  multiple relays is presented. Previous results were available only for upto three relays. The algorithm is unique in the sense that  the actual mathematical expressions, that are prohibitively large, need not be explicitly obtained. The diversity gain due to multiple relays is shown through plots of the analytical BER, well supported by simulations. 
%
%\end{abstract}
% IEEEtran.cls defaults to using nonbold math in the Abstract.
% This preserves the distinction between vectors and scalars. However,
% if the journal you are submitting to favors bold math in the abstract,
% then you can use LaTeX's standard command \boldmath at the very start
% of the abstract to achieve this. Many IEEE journals frown on math
% in the abstract anyway.

% Note that keywords are not normally used for peerreview papers.
%\begin{IEEEkeywords}
%Cooperative diversity, decode and forward, piecewise linear
%\end{IEEEkeywords}



% For peer review papers, you can put extra information on the cover
% page as needed:
% \ifCLASSOPTIONpeerreview
% \begin{center} \bfseries EDICS Category: 3-BBND \end{center}
% \fi
%
% For peerreview papers, this IEEEtran command inserts a page break and
% creates the second title. It will be ignored for other modes.
%\IEEEpeerreviewmaketitle




   \item Four candidates A, B, C, D have ap-
plied for the assignment to coach a school cricket
team. If A is twice as likely to be selected as B, and
B and C are given about the same chance of being
selected, while C is twice as likely to be selected
as D, what are the probabilities that
\begin{enumerate}
\item C will be selected?
\item A will not be selected?
\end{enumerate}
	%\begin{table}[H]
	\centering
\begin{tabular}{|c|c|c|}
\hline
Random variable &Value &Definition\\ \hline
\multirow{3}{*}{X} &0 &Slips of Rs 1\\
&1 &Slips of Rs 5\\
&2 &Slips of Rs 13\\ \hline
\multirow{2}{*}{Y} &0 &Box A\\
&1 &Box B\\\hline
\end{tabular}
\caption{}
\label{tab:Distribution}
\end{table}
See \tabref{tab:Distribution}.
\begin{align}
p_{Y}\brak{k}= \begin{cases} 
      \frac{1}{3} & {k=0} \\
      \frac{2}{3 }& {k=1} 
   \end{cases}
   \\
p_{Y|X}\brak{0|0} = \frac{19}{25}\, 
p_{Y|X}\brak{0|1} = \frac{6}{25}\,
p_{Y|X}\brak{1|0} = \frac{45}{50}\,
p_{Y|X}\brak{1|2} = \frac{5}{50}
\end{align}
The desired probability is the probability that a slip drawn at random is marked other than Rs 1,
\begin{align}
&=1-p_X\brak{0}\\
&= p_X(1) + p_X(2)
\end{align}
Using Bayes theorem,
\begin{align}
&= p_Y\brak{0} \times \pr{Y=0 | X=1} + p_Y\brak{1} \times \pr{Y=1|X=2}\\
&=\frac{1}{3} \times \frac{6}{25} + \frac{2}{3} \times \frac{5}{50}\\
&=\frac{11}{75}
\end{align}

\newpage

%\tableofcontents

\bigskip

\renewcommand{\thefigure}{\theenumi}
\renewcommand{\thetable}{\theenumi}
%\renewcommand{\theequation}{\theenumi}

%\begin{abstract}
%%\boldmath
%In this letter, an algorithm for evaluating the exact analytical bit error rate  (BER)  for the piecewise linear (PL) combiner for  multiple relays is presented. Previous results were available only for upto three relays. The algorithm is unique in the sense that  the actual mathematical expressions, that are prohibitively large, need not be explicitly obtained. The diversity gain due to multiple relays is shown through plots of the analytical BER, well supported by simulations. 
%
%\end{abstract}
% IEEEtran.cls defaults to using nonbold math in the Abstract.
% This preserves the distinction between vectors and scalars. However,
% if the journal you are submitting to favors bold math in the abstract,
% then you can use LaTeX's standard command \boldmath at the very start
% of the abstract to achieve this. Many IEEE journals frown on math
% in the abstract anyway.

% Note that keywords are not normally used for peerreview papers.
%\begin{IEEEkeywords}
%Cooperative diversity, decode and forward, piecewise linear
%\end{IEEEkeywords}



% For peer review papers, you can put extra information on the cover
% page as needed:
% \ifCLASSOPTIONpeerreview
% \begin{center} \bfseries EDICS Category: 3-BBND \end{center}
% \fi
%
% For peerreview papers, this IEEEtran command inserts a page break and
% creates the second title. It will be ignored for other modes.
%\IEEEpeerreviewmaketitle




 \item A bag contain 24 balls of which $x$ balls are red, $2x$ are white and $3x$ are blue. A ball is selected at random, What is the probability that it is
\begin{enumerate}[label=\alph*)]
\item not red ?
\item white ?
\end{enumerate}
%\begin{table}[H]
	\centering
\begin{tabular}{|c|c|c|}
\hline
Random variable &Value &Definition\\ \hline
\multirow{3}{*}{X} &0 &Slips of Rs 1\\
&1 &Slips of Rs 5\\
&2 &Slips of Rs 13\\ \hline
\multirow{2}{*}{Y} &0 &Box A\\
&1 &Box B\\\hline
\end{tabular}
\caption{}
\label{tab:Distribution}
\end{table}
See \tabref{tab:Distribution}.
\begin{align}
p_{Y}\brak{k}= \begin{cases} 
      \frac{1}{3} & {k=0} \\
      \frac{2}{3 }& {k=1} 
   \end{cases}
   \\
p_{Y|X}\brak{0|0} = \frac{19}{25}\, 
p_{Y|X}\brak{0|1} = \frac{6}{25}\,
p_{Y|X}\brak{1|0} = \frac{45}{50}\,
p_{Y|X}\brak{1|2} = \frac{5}{50}
\end{align}
The desired probability is the probability that a slip drawn at random is marked other than Rs 1,
\begin{align}
&=1-p_X\brak{0}\\
&= p_X(1) + p_X(2)
\end{align}
Using Bayes theorem,
\begin{align}
&= p_Y\brak{0} \times \pr{Y=0 | X=1} + p_Y\brak{1} \times \pr{Y=1|X=2}\\
&=\frac{1}{3} \times \frac{6}{25} + \frac{2}{3} \times \frac{5}{50}\\
&=\frac{11}{75}
\end{align}

\newpage

%\tableofcontents

\bigskip

\renewcommand{\thefigure}{\theenumi}
\renewcommand{\thetable}{\theenumi}
%\renewcommand{\theequation}{\theenumi}

%\begin{abstract}
%%\boldmath
%In this letter, an algorithm for evaluating the exact analytical bit error rate  (BER)  for the piecewise linear (PL) combiner for  multiple relays is presented. Previous results were available only for upto three relays. The algorithm is unique in the sense that  the actual mathematical expressions, that are prohibitively large, need not be explicitly obtained. The diversity gain due to multiple relays is shown through plots of the analytical BER, well supported by simulations. 
%
%\end{abstract}
% IEEEtran.cls defaults to using nonbold math in the Abstract.
% This preserves the distinction between vectors and scalars. However,
% if the journal you are submitting to favors bold math in the abstract,
% then you can use LaTeX's standard command \boldmath at the very start
% of the abstract to achieve this. Many IEEE journals frown on math
% in the abstract anyway.

% Note that keywords are not normally used for peerreview papers.
%\begin{IEEEkeywords}
%Cooperative diversity, decode and forward, piecewise linear
%\end{IEEEkeywords}



% For peer review papers, you can put extra information on the cover
% page as needed:
% \ifCLASSOPTIONpeerreview
% \begin{center} \bfseries EDICS Category: 3-BBND \end{center}
% \fi
%
% For peerreview papers, this IEEEtran command inserts a page break and
% creates the second title. It will be ignored for other modes.
%\IEEEpeerreviewmaketitle




If the letters of the word ASSASSINATION are arranged at random. Find the Probability that
\begin{enumerate}[label=(\alph*)]
\item Four $S's$ come consecutively in the word
\item Two  $I's$ and two $N's$ come together
\item All $A's$ are not coming together
\item No two $A's$ are coming together
\end{enumerate}
%\begin{table}[H]
	\centering
\begin{tabular}{|c|c|c|}
\hline
Random variable &Value &Definition\\ \hline
\multirow{3}{*}{X} &0 &Slips of Rs 1\\
&1 &Slips of Rs 5\\
&2 &Slips of Rs 13\\ \hline
\multirow{2}{*}{Y} &0 &Box A\\
&1 &Box B\\\hline
\end{tabular}
\caption{}
\label{tab:Distribution}
\end{table}
See \tabref{tab:Distribution}.
\begin{align}
p_{Y}\brak{k}= \begin{cases} 
      \frac{1}{3} & {k=0} \\
      \frac{2}{3 }& {k=1} 
   \end{cases}
   \\
p_{Y|X}\brak{0|0} = \frac{19}{25}\, 
p_{Y|X}\brak{0|1} = \frac{6}{25}\,
p_{Y|X}\brak{1|0} = \frac{45}{50}\,
p_{Y|X}\brak{1|2} = \frac{5}{50}
\end{align}
The desired probability is the probability that a slip drawn at random is marked other than Rs 1,
\begin{align}
&=1-p_X\brak{0}\\
&= p_X(1) + p_X(2)
\end{align}
Using Bayes theorem,
\begin{align}
&= p_Y\brak{0} \times \pr{Y=0 | X=1} + p_Y\brak{1} \times \pr{Y=1|X=2}\\
&=\frac{1}{3} \times \frac{6}{25} + \frac{2}{3} \times \frac{5}{50}\\
&=\frac{11}{75}
\end{align}

\newpage

%\tableofcontents

\bigskip

\renewcommand{\thefigure}{\theenumi}
\renewcommand{\thetable}{\theenumi}
%\renewcommand{\theequation}{\theenumi}

%\begin{abstract}
%%\boldmath
%In this letter, an algorithm for evaluating the exact analytical bit error rate  (BER)  for the piecewise linear (PL) combiner for  multiple relays is presented. Previous results were available only for upto three relays. The algorithm is unique in the sense that  the actual mathematical expressions, that are prohibitively large, need not be explicitly obtained. The diversity gain due to multiple relays is shown through plots of the analytical BER, well supported by simulations. 
%
%\end{abstract}
% IEEEtran.cls defaults to using nonbold math in the Abstract.
% This preserves the distinction between vectors and scalars. However,
% if the journal you are submitting to favors bold math in the abstract,
% then you can use LaTeX's standard command \boldmath at the very start
% of the abstract to achieve this. Many IEEE journals frown on math
% in the abstract anyway.

% Note that keywords are not normally used for peerreview papers.
%\begin{IEEEkeywords}
%Cooperative diversity, decode and forward, piecewise linear
%\end{IEEEkeywords}



% For peer review papers, you can put extra information on the cover
% page as needed:
% \ifCLASSOPTIONpeerreview
% \begin{center} \bfseries EDICS Category: 3-BBND \end{center}
% \fi
%
% For peerreview papers, this IEEEtran command inserts a page break and
% creates the second title. It will be ignored for other modes.
%\IEEEpeerreviewmaketitle




	\item One urn contains two black balls (labelled B1 and B2) and one white ball. A
	second urn contains one black ball and two white balls (labelled W1 and W2).
	Suppose the following experiment is performed. One of the two urns is chosen
	at random. Next a ball is randomly chosen from the urn. Then a second ball is
	chosen at random from the same urn without replacing the first ball.
	
	\begin{enumerate}
	\item What is the probability that two black balls are chosen?
	
	\item What is the probability that two balls of opposite colour are chosen?
	\end{enumerate}
	\solution
	%\begin{align}
    \label{eq:12.13.6.18.1}
	\because	\pr{A|B} &> \pr{A},\
\frac{\pr{AB}}{\pr{B}} > \pr{A}
\\
    \label{eq:12.13.6.18.2}
	\implies \pr{AB} &> \pr{A}\pr{B}
	\\
	\text{or, } \frac{\pr{AB}}{\pr{A}} &=\pr{B|A} > \pr{A}
\end{align}

\end{enumerate}

	\item A card is selected from a pack of 52 cards.
 \begin{enumerate}[label=(\alph*)] 
                 \item How many points are there in the sample space?
                 \item Calculate the probability that the card is an ace of spades.
                 \item Calculate the probability that the card is (i) an ace and (ii) black card.
 \end{enumerate}
\solution
		%\begin{table}[H]
	\centering
\begin{tabular}{|c|c|c|}
\hline
Random variable &Value &Definition\\ \hline
\multirow{3}{*}{X} &0 &Slips of Rs 1\\
&1 &Slips of Rs 5\\
&2 &Slips of Rs 13\\ \hline
\multirow{2}{*}{Y} &0 &Box A\\
&1 &Box B\\\hline
\end{tabular}
\caption{}
\label{tab:Distribution}
\end{table}
See \tabref{tab:Distribution}.
\begin{align}
p_{Y}\brak{k}= \begin{cases} 
      \frac{1}{3} & {k=0} \\
      \frac{2}{3 }& {k=1} 
   \end{cases}
   \\
p_{Y|X}\brak{0|0} = \frac{19}{25}\, 
p_{Y|X}\brak{0|1} = \frac{6}{25}\,
p_{Y|X}\brak{1|0} = \frac{45}{50}\,
p_{Y|X}\brak{1|2} = \frac{5}{50}
\end{align}
The desired probability is the probability that a slip drawn at random is marked other than Rs 1,
\begin{align}
&=1-p_X\brak{0}\\
&= p_X(1) + p_X(2)
\end{align}
Using Bayes theorem,
\begin{align}
&= p_Y\brak{0} \times \pr{Y=0 | X=1} + p_Y\brak{1} \times \pr{Y=1|X=2}\\
&=\frac{1}{3} \times \frac{6}{25} + \frac{2}{3} \times \frac{5}{50}\\
&=\frac{11}{75}
\end{align}

\newpage

%\tableofcontents

\bigskip

\renewcommand{\thefigure}{\theenumi}
\renewcommand{\thetable}{\theenumi}
%\renewcommand{\theequation}{\theenumi}

%\begin{abstract}
%%\boldmath
%In this letter, an algorithm for evaluating the exact analytical bit error rate  (BER)  for the piecewise linear (PL) combiner for  multiple relays is presented. Previous results were available only for upto three relays. The algorithm is unique in the sense that  the actual mathematical expressions, that are prohibitively large, need not be explicitly obtained. The diversity gain due to multiple relays is shown through plots of the analytical BER, well supported by simulations. 
%
%\end{abstract}
% IEEEtran.cls defaults to using nonbold math in the Abstract.
% This preserves the distinction between vectors and scalars. However,
% if the journal you are submitting to favors bold math in the abstract,
% then you can use LaTeX's standard command \boldmath at the very start
% of the abstract to achieve this. Many IEEE journals frown on math
% in the abstract anyway.

% Note that keywords are not normally used for peerreview papers.
%\begin{IEEEkeywords}
%Cooperative diversity, decode and forward, piecewise linear
%\end{IEEEkeywords}



% For peer review papers, you can put extra information on the cover
% page as needed:
% \ifCLASSOPTIONpeerreview
% \begin{center} \bfseries EDICS Category: 3-BBND \end{center}
% \fi
%
% For peerreview papers, this IEEEtran command inserts a page break and
% creates the second title. It will be ignored for other modes.
%\IEEEpeerreviewmaketitle




\item Four cards are drawn from a well-shuffled deck of 52 cards. What is the probability of obtaining 3 diamonds and one spade.
\\
\solution
		%\begin{enumerate}[label=\thesection.\arabic*,ref=\thesection.\theenumi]
	\item One card is drawn from a well-shuffled deck of 52 cards. Find the probability of getting
\begin{enumerate}
\item A king of red colour 
\item A face card 
\item A red face card
\item The jack of hearts
\item A spade
\item The queen of diamonds

\end{enumerate}
\solution
		%\begin{table}[H]
	\centering
\begin{tabular}{|c|c|c|}
\hline
Random variable &Value &Definition\\ \hline
\multirow{3}{*}{X} &0 &Slips of Rs 1\\
&1 &Slips of Rs 5\\
&2 &Slips of Rs 13\\ \hline
\multirow{2}{*}{Y} &0 &Box A\\
&1 &Box B\\\hline
\end{tabular}
\caption{}
\label{tab:Distribution}
\end{table}
See \tabref{tab:Distribution}.
\begin{align}
p_{Y}\brak{k}= \begin{cases} 
      \frac{1}{3} & {k=0} \\
      \frac{2}{3 }& {k=1} 
   \end{cases}
   \\
p_{Y|X}\brak{0|0} = \frac{19}{25}\, 
p_{Y|X}\brak{0|1} = \frac{6}{25}\,
p_{Y|X}\brak{1|0} = \frac{45}{50}\,
p_{Y|X}\brak{1|2} = \frac{5}{50}
\end{align}
The desired probability is the probability that a slip drawn at random is marked other than Rs 1,
\begin{align}
&=1-p_X\brak{0}\\
&= p_X(1) + p_X(2)
\end{align}
Using Bayes theorem,
\begin{align}
&= p_Y\brak{0} \times \pr{Y=0 | X=1} + p_Y\brak{1} \times \pr{Y=1|X=2}\\
&=\frac{1}{3} \times \frac{6}{25} + \frac{2}{3} \times \frac{5}{50}\\
&=\frac{11}{75}
\end{align}

\newpage

%\tableofcontents

\bigskip

\renewcommand{\thefigure}{\theenumi}
\renewcommand{\thetable}{\theenumi}
%\renewcommand{\theequation}{\theenumi}

%\begin{abstract}
%%\boldmath
%In this letter, an algorithm for evaluating the exact analytical bit error rate  (BER)  for the piecewise linear (PL) combiner for  multiple relays is presented. Previous results were available only for upto three relays. The algorithm is unique in the sense that  the actual mathematical expressions, that are prohibitively large, need not be explicitly obtained. The diversity gain due to multiple relays is shown through plots of the analytical BER, well supported by simulations. 
%
%\end{abstract}
% IEEEtran.cls defaults to using nonbold math in the Abstract.
% This preserves the distinction between vectors and scalars. However,
% if the journal you are submitting to favors bold math in the abstract,
% then you can use LaTeX's standard command \boldmath at the very start
% of the abstract to achieve this. Many IEEE journals frown on math
% in the abstract anyway.

% Note that keywords are not normally used for peerreview papers.
%\begin{IEEEkeywords}
%Cooperative diversity, decode and forward, piecewise linear
%\end{IEEEkeywords}



% For peer review papers, you can put extra information on the cover
% page as needed:
% \ifCLASSOPTIONpeerreview
% \begin{center} \bfseries EDICS Category: 3-BBND \end{center}
% \fi
%
% For peerreview papers, this IEEEtran command inserts a page break and
% creates the second title. It will be ignored for other modes.
%\IEEEpeerreviewmaketitle




	\item Five cards—the ten, jack, queen, king and ace of diamonds, are well-shuffled with their face downwards. One card is then picked up at random.
\begin{enumerate}
\item
What is the probability that the card is the queen? 
\item
If the queen is drawn and put aside, what is the probability that the second card picked up is (a) an ace? (b) a queen?\\
\end{enumerate}
\solution
		%\begin{enumerate}[label=\thesection.\arabic*,ref=\thesection.\theenumi]
	\item One card is drawn from a well-shuffled deck of 52 cards. Find the probability of getting
\begin{enumerate}
\item A king of red colour 
\item A face card 
\item A red face card
\item The jack of hearts
\item A spade
\item The queen of diamonds

\end{enumerate}
\solution
		%\input{ncert/10/15/1/14/main.tex}
	\item Five cards—the ten, jack, queen, king and ace of diamonds, are well-shuffled with their face downwards. One card is then picked up at random.
\begin{enumerate}
\item
What is the probability that the card is the queen? 
\item
If the queen is drawn and put aside, what is the probability that the second card picked up is (a) an ace? (b) a queen?\\
\end{enumerate}
\solution
		%\input{ncert/10/15/1/15/defs.tex}
	\item A bag contains $5$ red balls and some blue balls. If the probability of drawing a blue ball is double that if a red ball, determine the number of blue balls in the bag. 
		\\
\solution
		%\input{ncert/10/15/2/3/defs.tex}
	\item A card is selected from a pack of 52 cards.
 \begin{enumerate}[label=(\alph*)] 
                 \item How many points are there in the sample space?
                 \item Calculate the probability that the card is an ace of spades.
                 \item Calculate the probability that the card is (i) an ace and (ii) black card.
 \end{enumerate}
\solution
		%\input{ncert/11/16/3/4/main.tex}
\item Four cards are drawn from a well-shuffled deck of 52 cards. What is the probability of obtaining 3 diamonds and one spade.
\\
\solution
		%\input{ncert/11/16/4/2/defs.tex}
\item In a certain lottery 10,000 tickets are sold and ten equal prizes are awarded. What is the probability of not getting a prize if you buy (a) one ticket (b) two tickets (c) 10 tickets ?	
\\
\solution
		%\input{ncert/11/16/4/4/defs.tex}
		%
\item 
Out of 100 students, two sections of 40 and 60 are formed. If you and your friend are among the 100 students, what is the probability that
\begin{enumerate}
\item you both enter the same section?
\item you both enter the different sections?
\end{enumerate}
\solution
		%\input{ncert/11/16/4/5/defs.tex}
	\item 
The number lock of a suitcase has 4 wheels each labelled with ten digits i.e. from 0 to 9.The lock opens with a sequence of four digits with no repeats.What is the probability of a person getting the right sequence to open the suitcase.
\\
\solution
		%\input{ncert/11/16/4/10/defs.tex}
		%
\item 
Two cards are drawn at random and without replacement from a pack of 52 playing cards. Find the probability that both the cards are black.
\\
\solution
		%\input{ncert/12/13/2/2/defs.tex}
		\item A box of oranges is inspected by examining three randomly selected oranges drawn without replacement. If all the three oranges are good, the box is approved for sale, otherwise, it is rejected. Find the probability that a box containing 15 oranges out of which 12 are good and 3 are bad ones will be approved for sale.
		\label{ncert/12/13/2/3/defs.tex}
		\item Two balls are drawn at random with replacement from a box containing 10 black and 8 red balls. Find the probability that
		\label{ncert/12/13/2/12}
\begin{enumerate}
\item both balls are red.
\item first ball is black and second is red.
\item one of them is black and other is red.
\end{enumerate}

\item In a hostel, 60\% of the students read Hindi newspaper, 40\% read English newspaper and 20\% read both Hindi and English newspapers. A student is selected at random.
		\label{ncert/12/13/2/15}
\begin{enumerate}
\item Find the probability that she reads neither Hindi nor English newspapers.
\item If she reads Hindi newspaper, find the probability that she reads English newspaper.
\item If she reads English newspaper, find the probability that she reads Hindi newspaper.\\
\end{enumerate}
\item The probability of obtaining an even prime number on each die, when a pair of dice is rolled is 
\begin{enumerate}
    \item $0$ 
    
    \item $\frac{1}{3}$ 
    
    \item $\frac{1}{12}$ 
    
    \item $\frac{1}{36}$ 
\end{enumerate}
\solution
		%\input{ncert/12/13/2/17/defs.tex}
	\item A bag contains 4 red and 4 black balls, another bag contains 2 red and 6 black balls. One of the two bags is selected at random and a ball is drawn from the bag which is found to be red. Find the probability that the ball is drawn from the first bag.
\\
\solution
		%\input{ncert/12/13/3/2/main.tex}
  \item
  Cards with numbers 2 to 101 are placed in a box. A card is selected at random.Find the probability that the card has
\begin{enumerate}[label=(\roman*)]
	\item an even number 
	\item a square number
\end{enumerate}
\solution
%\input{exemplar/10/13/3/32/main.tex}
\item
The king, queen and jack of clubs are removed from a deck of 52 playing cards and then well shuffled. Now one card is drawn at random from the remaining cards.  Determine the probability that the card is
\begin{enumerate}[label=(\roman*)]
\item a club
\item 10 of hearts
\end{enumerate}
\solution
%\input{exemplar/10/13/3/29/main.tex}
\item A team of medical students doing their internship have to assist during surgeries
at a city hospital. The probabilities of surgeries rated as very complex, complex,
routine, simple or very simple are respectively, 0.15, 0.20, 0.31, 0.26, .08. Find
the probabilities that a particular surgery will be rated
\begin{enumerate}
	\item complex or very complex;
	\item neither very complex nor very simple;
	\item routine or complex
	\item routine or simple
\end{enumerate}
\solution
%\input{exemplar/11/16/3/8(1)/main.tex}
\item A card is selected from a pack of 52 cards.
\begin{enumerate}[label=(\alph*)]
    \item How many points are there in the sample space?
    \item Calculate the probability that the card is an ace of spades.
    \item Calculate the probability that the card is (i) an ace and (ii) black card.
\end{enumerate}
\solution
%\input{exemplar/11/16/3/4/main2.tex}
\item The probability that a non leap year selected at random will contain 53 sundays.
\\
\solution
%\input{exemplar/10/13/1/19/main.tex}
\item One of the four persons John, Rita, Aslam or Gurpreet will be promoted next
month. Consequently the sample space consists of four elementary outcomes
S = {John promoted, Rita promoted, Aslam promoted, Gurpreet promoted}
You are told that the chances of John’s promotion is same as that of Gurpreet,
Rita’s chances of promotion are twice as likely as Johns. Aslam’s chances are
four times that of John.
\begin{enumerate}
	\item Determine
	\begin{enumerate}
		\item P (John promoted)
		\item P (Rita promoted)
		\item P (Aslam promoted)
		\item P (Gurpreet promoted)
	\end{enumerate}
	\item If A = {John promoted or Gurpreet promoted}, find P (A).
\end{enumerate}
\solution
%\input{exemplar/11/16/3/10/main.tex}
\item A card is drawn from a deck of 52 cards. Find the probability of getting a king or a heart or a red card.\\
\solution
%\input{exemplar/11/16/3/15/main.tex}
\item The probability that a student will pass his examination is 0.73, the probability of
the student getting a compartment is 0.13, and the probability that the student will
either pass or get compartment is 0.96. State True or False.\\
\solution
%\input{exemplar/11/16/3/31/main.tex}
\item A card is selected from a pack of 52 cards\\
\begin{enumerate}[label=(\alph*)]
\item How many points are there in the sample space?
\item Calculate the probability that the cards is an ace of spades.
\item Calculate the probability that the card is (i) an ace (ii)black card.\\
\end{enumerate}
%\input{ncert/11/16/3/4_1/Prob_4.tex}
\item In a non-leap year, the probability of having 53 tuesdays or 53 wednesdays is\\
\solution
%\input{exemplar/11/16/3/18/main.tex}
\item There are 1000 sealed envelopes in a box, 10 of them contain a cash prize of
Rs 100 each, 100 of them contain a cash prize of Rs 50 each and 200 of them
contain a cash prize of Rs 10 each and rest do not contain any cash prize. If they
are well shuffled and an envelope is picked up out, what is the probability that it
contains no cash prize?\\
\solution
%\input{exemplar/10/13/3/34/main.tex}
\item 
A die is thrown and a card is selected at random from a deck of 52 playing cards. The probability of getting an even number on the die and a spade card.\\
\solution
%\input{exemplar/12/13/3/78/main.tex}
\item
If 4-digit numbers greater than 5,000 are randomly formed from the digits 0, 1, 3, 5, and 7, what is the probability of forming a number divisible by 5 when:
\begin{enumerate}
    \item The digits are repeated?
    \item The repetition of digits is not allowed?
\end{enumerate}
\solution
%\input{ncert/11/16/4/9/main.tex}
\item Consider the probability space $\brak{\Omega, \mathcal{G}, P}$ where $\Omega = [0,2]$ and $\mathcal{G} = \cbrak{\phi, \Omega, [0,1], (1,2]}$. Let $X$ and $Y$ be two functions on $\Omega$ defined as
\begin{align*}
    X(\omega) = 
    \begin{cases}
        1 & \text{if }\omega \in [0, 1]\\
        2 & \text{if }\omega \in (1, 2]
    \end{cases}
\end{align*}
and
\begin{align*}
    Y(\omega) = 
    \begin{cases}
        2 & \text{if }\omega \in [0, 1.5]\\
        3 & \text{if }\omega \in (1.5, 2].
    \end{cases}
\end{align*}
Then which one of the following statements is true?
\begin{enumerate}
    \item [(A)] $X$ is a random variable with respect to $\mathcal{G}$, but $Y$ is not a random variable with respect to $\mathcal{G}$.
    \item [(B)] $Y$ is a random variable with respect to $\mathcal{G}$, but $X$ is not a random variable with respect to $\mathcal{G}$.
    \item [(C)] Neither $X$ nor $Y$ is a random variable with respect to $\mathcal{G}$.
    \item [(D)] Both $X$ and $Y$ are random variables with respect to $\mathcal{G}$.
\end{enumerate} \hfill (GATE ST 2023)\\
\solution
%\input{gate/ST/2023/14/main.tex}
	\item  A die is loaded in such a way that each odd number is twice as likely to occur as
each even number. Find $P(G)$, where $G$ is the event that a number greater than
3 occurs on a single roll of the die.
\\
\solution
		%\input{exemplar/11/16/3/5/main.tex}
	\item All the jacks, queens and kings are removed from a deck of 52 playing cards. The remaining cards are well shuffled and then one card is drawn at random. Giving ace a value 1 similar value for other cards, find the probability that the card has a value 
		\begin{enumerate}
			\item 7
			\item greater than 7
			\item less than 7
		\end{enumerate}
		%\input{exemplar/10/13/3/30/main.tex}
  \item A Lot consists of 48 mobile phones of which 42 are good, 3 have only minor defects and 3 have major defects.Varnika will buy a phone if it is good but the trader will only buy a mobile if it has no major defects. One phone is selected at random from the lot. What is the probability that it is
\begin{enumerate}
	\item acceptable to Varnika?
            \item acceptable to the trader?
\end{enumerate}
\solution
	%\input{exemplar/10/13/3/40/main.tex}
 \item A student says that if you throw a die, it will show up 1 or not 1. Therefore, the probability of getting 1 and the probability of getting 'not 1' each is equal to $\frac{1}{2}$. Is this correct? Give reasons.\\
 \solution
        %\input{exemplar/10/13/2/9/main.tex}
   \item Four candidates A, B, C, D have ap-
plied for the assignment to coach a school cricket
team. If A is twice as likely to be selected as B, and
B and C are given about the same chance of being
selected, while C is twice as likely to be selected
as D, what are the probabilities that
\begin{enumerate}
\item C will be selected?
\item A will not be selected?
\end{enumerate}
	%\input{exemplar/11/16/3/9/main.tex}
 \item A bag contain 24 balls of which $x$ balls are red, $2x$ are white and $3x$ are blue. A ball is selected at random, What is the probability that it is
\begin{enumerate}[label=\alph*)]
\item not red ?
\item white ?
\end{enumerate}
%\input{exemplar/10/13/3/41/main.tex}
If the letters of the word ASSASSINATION are arranged at random. Find the Probability that
\begin{enumerate}[label=(\alph*)]
\item Four $S's$ come consecutively in the word
\item Two  $I's$ and two $N's$ come together
\item All $A's$ are not coming together
\item No two $A's$ are coming together
\end{enumerate}
%\input{exemplar/11/16/3/14/main.tex}
	\item One urn contains two black balls (labelled B1 and B2) and one white ball. A
	second urn contains one black ball and two white balls (labelled W1 and W2).
	Suppose the following experiment is performed. One of the two urns is chosen
	at random. Next a ball is randomly chosen from the urn. Then a second ball is
	chosen at random from the same urn without replacing the first ball.
	
	\begin{enumerate}
	\item What is the probability that two black balls are chosen?
	
	\item What is the probability that two balls of opposite colour are chosen?
	\end{enumerate}
	\solution
	%\input{exemplar/11/16/3/12/main1.tex}
\end{enumerate}

	\item A bag contains $5$ red balls and some blue balls. If the probability of drawing a blue ball is double that if a red ball, determine the number of blue balls in the bag. 
		\\
\solution
		%\begin{enumerate}[label=\thesection.\arabic*,ref=\thesection.\theenumi]
	\item One card is drawn from a well-shuffled deck of 52 cards. Find the probability of getting
\begin{enumerate}
\item A king of red colour 
\item A face card 
\item A red face card
\item The jack of hearts
\item A spade
\item The queen of diamonds

\end{enumerate}
\solution
		%\input{ncert/10/15/1/14/main.tex}
	\item Five cards—the ten, jack, queen, king and ace of diamonds, are well-shuffled with their face downwards. One card is then picked up at random.
\begin{enumerate}
\item
What is the probability that the card is the queen? 
\item
If the queen is drawn and put aside, what is the probability that the second card picked up is (a) an ace? (b) a queen?\\
\end{enumerate}
\solution
		%\input{ncert/10/15/1/15/defs.tex}
	\item A bag contains $5$ red balls and some blue balls. If the probability of drawing a blue ball is double that if a red ball, determine the number of blue balls in the bag. 
		\\
\solution
		%\input{ncert/10/15/2/3/defs.tex}
	\item A card is selected from a pack of 52 cards.
 \begin{enumerate}[label=(\alph*)] 
                 \item How many points are there in the sample space?
                 \item Calculate the probability that the card is an ace of spades.
                 \item Calculate the probability that the card is (i) an ace and (ii) black card.
 \end{enumerate}
\solution
		%\input{ncert/11/16/3/4/main.tex}
\item Four cards are drawn from a well-shuffled deck of 52 cards. What is the probability of obtaining 3 diamonds and one spade.
\\
\solution
		%\input{ncert/11/16/4/2/defs.tex}
\item In a certain lottery 10,000 tickets are sold and ten equal prizes are awarded. What is the probability of not getting a prize if you buy (a) one ticket (b) two tickets (c) 10 tickets ?	
\\
\solution
		%\input{ncert/11/16/4/4/defs.tex}
		%
\item 
Out of 100 students, two sections of 40 and 60 are formed. If you and your friend are among the 100 students, what is the probability that
\begin{enumerate}
\item you both enter the same section?
\item you both enter the different sections?
\end{enumerate}
\solution
		%\input{ncert/11/16/4/5/defs.tex}
	\item 
The number lock of a suitcase has 4 wheels each labelled with ten digits i.e. from 0 to 9.The lock opens with a sequence of four digits with no repeats.What is the probability of a person getting the right sequence to open the suitcase.
\\
\solution
		%\input{ncert/11/16/4/10/defs.tex}
		%
\item 
Two cards are drawn at random and without replacement from a pack of 52 playing cards. Find the probability that both the cards are black.
\\
\solution
		%\input{ncert/12/13/2/2/defs.tex}
		\item A box of oranges is inspected by examining three randomly selected oranges drawn without replacement. If all the three oranges are good, the box is approved for sale, otherwise, it is rejected. Find the probability that a box containing 15 oranges out of which 12 are good and 3 are bad ones will be approved for sale.
		\label{ncert/12/13/2/3/defs.tex}
		\item Two balls are drawn at random with replacement from a box containing 10 black and 8 red balls. Find the probability that
		\label{ncert/12/13/2/12}
\begin{enumerate}
\item both balls are red.
\item first ball is black and second is red.
\item one of them is black and other is red.
\end{enumerate}

\item In a hostel, 60\% of the students read Hindi newspaper, 40\% read English newspaper and 20\% read both Hindi and English newspapers. A student is selected at random.
		\label{ncert/12/13/2/15}
\begin{enumerate}
\item Find the probability that she reads neither Hindi nor English newspapers.
\item If she reads Hindi newspaper, find the probability that she reads English newspaper.
\item If she reads English newspaper, find the probability that she reads Hindi newspaper.\\
\end{enumerate}
\item The probability of obtaining an even prime number on each die, when a pair of dice is rolled is 
\begin{enumerate}
    \item $0$ 
    
    \item $\frac{1}{3}$ 
    
    \item $\frac{1}{12}$ 
    
    \item $\frac{1}{36}$ 
\end{enumerate}
\solution
		%\input{ncert/12/13/2/17/defs.tex}
	\item A bag contains 4 red and 4 black balls, another bag contains 2 red and 6 black balls. One of the two bags is selected at random and a ball is drawn from the bag which is found to be red. Find the probability that the ball is drawn from the first bag.
\\
\solution
		%\input{ncert/12/13/3/2/main.tex}
  \item
  Cards with numbers 2 to 101 are placed in a box. A card is selected at random.Find the probability that the card has
\begin{enumerate}[label=(\roman*)]
	\item an even number 
	\item a square number
\end{enumerate}
\solution
%\input{exemplar/10/13/3/32/main.tex}
\item
The king, queen and jack of clubs are removed from a deck of 52 playing cards and then well shuffled. Now one card is drawn at random from the remaining cards.  Determine the probability that the card is
\begin{enumerate}[label=(\roman*)]
\item a club
\item 10 of hearts
\end{enumerate}
\solution
%\input{exemplar/10/13/3/29/main.tex}
\item A team of medical students doing their internship have to assist during surgeries
at a city hospital. The probabilities of surgeries rated as very complex, complex,
routine, simple or very simple are respectively, 0.15, 0.20, 0.31, 0.26, .08. Find
the probabilities that a particular surgery will be rated
\begin{enumerate}
	\item complex or very complex;
	\item neither very complex nor very simple;
	\item routine or complex
	\item routine or simple
\end{enumerate}
\solution
%\input{exemplar/11/16/3/8(1)/main.tex}
\item A card is selected from a pack of 52 cards.
\begin{enumerate}[label=(\alph*)]
    \item How many points are there in the sample space?
    \item Calculate the probability that the card is an ace of spades.
    \item Calculate the probability that the card is (i) an ace and (ii) black card.
\end{enumerate}
\solution
%\input{exemplar/11/16/3/4/main2.tex}
\item The probability that a non leap year selected at random will contain 53 sundays.
\\
\solution
%\input{exemplar/10/13/1/19/main.tex}
\item One of the four persons John, Rita, Aslam or Gurpreet will be promoted next
month. Consequently the sample space consists of four elementary outcomes
S = {John promoted, Rita promoted, Aslam promoted, Gurpreet promoted}
You are told that the chances of John’s promotion is same as that of Gurpreet,
Rita’s chances of promotion are twice as likely as Johns. Aslam’s chances are
four times that of John.
\begin{enumerate}
	\item Determine
	\begin{enumerate}
		\item P (John promoted)
		\item P (Rita promoted)
		\item P (Aslam promoted)
		\item P (Gurpreet promoted)
	\end{enumerate}
	\item If A = {John promoted or Gurpreet promoted}, find P (A).
\end{enumerate}
\solution
%\input{exemplar/11/16/3/10/main.tex}
\item A card is drawn from a deck of 52 cards. Find the probability of getting a king or a heart or a red card.\\
\solution
%\input{exemplar/11/16/3/15/main.tex}
\item The probability that a student will pass his examination is 0.73, the probability of
the student getting a compartment is 0.13, and the probability that the student will
either pass or get compartment is 0.96. State True or False.\\
\solution
%\input{exemplar/11/16/3/31/main.tex}
\item A card is selected from a pack of 52 cards\\
\begin{enumerate}[label=(\alph*)]
\item How many points are there in the sample space?
\item Calculate the probability that the cards is an ace of spades.
\item Calculate the probability that the card is (i) an ace (ii)black card.\\
\end{enumerate}
%\input{ncert/11/16/3/4_1/Prob_4.tex}
\item In a non-leap year, the probability of having 53 tuesdays or 53 wednesdays is\\
\solution
%\input{exemplar/11/16/3/18/main.tex}
\item There are 1000 sealed envelopes in a box, 10 of them contain a cash prize of
Rs 100 each, 100 of them contain a cash prize of Rs 50 each and 200 of them
contain a cash prize of Rs 10 each and rest do not contain any cash prize. If they
are well shuffled and an envelope is picked up out, what is the probability that it
contains no cash prize?\\
\solution
%\input{exemplar/10/13/3/34/main.tex}
\item 
A die is thrown and a card is selected at random from a deck of 52 playing cards. The probability of getting an even number on the die and a spade card.\\
\solution
%\input{exemplar/12/13/3/78/main.tex}
\item
If 4-digit numbers greater than 5,000 are randomly formed from the digits 0, 1, 3, 5, and 7, what is the probability of forming a number divisible by 5 when:
\begin{enumerate}
    \item The digits are repeated?
    \item The repetition of digits is not allowed?
\end{enumerate}
\solution
%\input{ncert/11/16/4/9/main.tex}
\item Consider the probability space $\brak{\Omega, \mathcal{G}, P}$ where $\Omega = [0,2]$ and $\mathcal{G} = \cbrak{\phi, \Omega, [0,1], (1,2]}$. Let $X$ and $Y$ be two functions on $\Omega$ defined as
\begin{align*}
    X(\omega) = 
    \begin{cases}
        1 & \text{if }\omega \in [0, 1]\\
        2 & \text{if }\omega \in (1, 2]
    \end{cases}
\end{align*}
and
\begin{align*}
    Y(\omega) = 
    \begin{cases}
        2 & \text{if }\omega \in [0, 1.5]\\
        3 & \text{if }\omega \in (1.5, 2].
    \end{cases}
\end{align*}
Then which one of the following statements is true?
\begin{enumerate}
    \item [(A)] $X$ is a random variable with respect to $\mathcal{G}$, but $Y$ is not a random variable with respect to $\mathcal{G}$.
    \item [(B)] $Y$ is a random variable with respect to $\mathcal{G}$, but $X$ is not a random variable with respect to $\mathcal{G}$.
    \item [(C)] Neither $X$ nor $Y$ is a random variable with respect to $\mathcal{G}$.
    \item [(D)] Both $X$ and $Y$ are random variables with respect to $\mathcal{G}$.
\end{enumerate} \hfill (GATE ST 2023)\\
\solution
%\input{gate/ST/2023/14/main.tex}
	\item  A die is loaded in such a way that each odd number is twice as likely to occur as
each even number. Find $P(G)$, where $G$ is the event that a number greater than
3 occurs on a single roll of the die.
\\
\solution
		%\input{exemplar/11/16/3/5/main.tex}
	\item All the jacks, queens and kings are removed from a deck of 52 playing cards. The remaining cards are well shuffled and then one card is drawn at random. Giving ace a value 1 similar value for other cards, find the probability that the card has a value 
		\begin{enumerate}
			\item 7
			\item greater than 7
			\item less than 7
		\end{enumerate}
		%\input{exemplar/10/13/3/30/main.tex}
  \item A Lot consists of 48 mobile phones of which 42 are good, 3 have only minor defects and 3 have major defects.Varnika will buy a phone if it is good but the trader will only buy a mobile if it has no major defects. One phone is selected at random from the lot. What is the probability that it is
\begin{enumerate}
	\item acceptable to Varnika?
            \item acceptable to the trader?
\end{enumerate}
\solution
	%\input{exemplar/10/13/3/40/main.tex}
 \item A student says that if you throw a die, it will show up 1 or not 1. Therefore, the probability of getting 1 and the probability of getting 'not 1' each is equal to $\frac{1}{2}$. Is this correct? Give reasons.\\
 \solution
        %\input{exemplar/10/13/2/9/main.tex}
   \item Four candidates A, B, C, D have ap-
plied for the assignment to coach a school cricket
team. If A is twice as likely to be selected as B, and
B and C are given about the same chance of being
selected, while C is twice as likely to be selected
as D, what are the probabilities that
\begin{enumerate}
\item C will be selected?
\item A will not be selected?
\end{enumerate}
	%\input{exemplar/11/16/3/9/main.tex}
 \item A bag contain 24 balls of which $x$ balls are red, $2x$ are white and $3x$ are blue. A ball is selected at random, What is the probability that it is
\begin{enumerate}[label=\alph*)]
\item not red ?
\item white ?
\end{enumerate}
%\input{exemplar/10/13/3/41/main.tex}
If the letters of the word ASSASSINATION are arranged at random. Find the Probability that
\begin{enumerate}[label=(\alph*)]
\item Four $S's$ come consecutively in the word
\item Two  $I's$ and two $N's$ come together
\item All $A's$ are not coming together
\item No two $A's$ are coming together
\end{enumerate}
%\input{exemplar/11/16/3/14/main.tex}
	\item One urn contains two black balls (labelled B1 and B2) and one white ball. A
	second urn contains one black ball and two white balls (labelled W1 and W2).
	Suppose the following experiment is performed. One of the two urns is chosen
	at random. Next a ball is randomly chosen from the urn. Then a second ball is
	chosen at random from the same urn without replacing the first ball.
	
	\begin{enumerate}
	\item What is the probability that two black balls are chosen?
	
	\item What is the probability that two balls of opposite colour are chosen?
	\end{enumerate}
	\solution
	%\input{exemplar/11/16/3/12/main1.tex}
\end{enumerate}

	\item A card is selected from a pack of 52 cards.
 \begin{enumerate}[label=(\alph*)] 
                 \item How many points are there in the sample space?
                 \item Calculate the probability that the card is an ace of spades.
                 \item Calculate the probability that the card is (i) an ace and (ii) black card.
 \end{enumerate}
\solution
		%\begin{table}[H]
	\centering
\begin{tabular}{|c|c|c|}
\hline
Random variable &Value &Definition\\ \hline
\multirow{3}{*}{X} &0 &Slips of Rs 1\\
&1 &Slips of Rs 5\\
&2 &Slips of Rs 13\\ \hline
\multirow{2}{*}{Y} &0 &Box A\\
&1 &Box B\\\hline
\end{tabular}
\caption{}
\label{tab:Distribution}
\end{table}
See \tabref{tab:Distribution}.
\begin{align}
p_{Y}\brak{k}= \begin{cases} 
      \frac{1}{3} & {k=0} \\
      \frac{2}{3 }& {k=1} 
   \end{cases}
   \\
p_{Y|X}\brak{0|0} = \frac{19}{25}\, 
p_{Y|X}\brak{0|1} = \frac{6}{25}\,
p_{Y|X}\brak{1|0} = \frac{45}{50}\,
p_{Y|X}\brak{1|2} = \frac{5}{50}
\end{align}
The desired probability is the probability that a slip drawn at random is marked other than Rs 1,
\begin{align}
&=1-p_X\brak{0}\\
&= p_X(1) + p_X(2)
\end{align}
Using Bayes theorem,
\begin{align}
&= p_Y\brak{0} \times \pr{Y=0 | X=1} + p_Y\brak{1} \times \pr{Y=1|X=2}\\
&=\frac{1}{3} \times \frac{6}{25} + \frac{2}{3} \times \frac{5}{50}\\
&=\frac{11}{75}
\end{align}

\newpage

%\tableofcontents

\bigskip

\renewcommand{\thefigure}{\theenumi}
\renewcommand{\thetable}{\theenumi}
%\renewcommand{\theequation}{\theenumi}

%\begin{abstract}
%%\boldmath
%In this letter, an algorithm for evaluating the exact analytical bit error rate  (BER)  for the piecewise linear (PL) combiner for  multiple relays is presented. Previous results were available only for upto three relays. The algorithm is unique in the sense that  the actual mathematical expressions, that are prohibitively large, need not be explicitly obtained. The diversity gain due to multiple relays is shown through plots of the analytical BER, well supported by simulations. 
%
%\end{abstract}
% IEEEtran.cls defaults to using nonbold math in the Abstract.
% This preserves the distinction between vectors and scalars. However,
% if the journal you are submitting to favors bold math in the abstract,
% then you can use LaTeX's standard command \boldmath at the very start
% of the abstract to achieve this. Many IEEE journals frown on math
% in the abstract anyway.

% Note that keywords are not normally used for peerreview papers.
%\begin{IEEEkeywords}
%Cooperative diversity, decode and forward, piecewise linear
%\end{IEEEkeywords}



% For peer review papers, you can put extra information on the cover
% page as needed:
% \ifCLASSOPTIONpeerreview
% \begin{center} \bfseries EDICS Category: 3-BBND \end{center}
% \fi
%
% For peerreview papers, this IEEEtran command inserts a page break and
% creates the second title. It will be ignored for other modes.
%\IEEEpeerreviewmaketitle




\item Four cards are drawn from a well-shuffled deck of 52 cards. What is the probability of obtaining 3 diamonds and one spade.
\\
\solution
		%\begin{enumerate}[label=\thesection.\arabic*,ref=\thesection.\theenumi]
	\item One card is drawn from a well-shuffled deck of 52 cards. Find the probability of getting
\begin{enumerate}
\item A king of red colour 
\item A face card 
\item A red face card
\item The jack of hearts
\item A spade
\item The queen of diamonds

\end{enumerate}
\solution
		%\input{ncert/10/15/1/14/main.tex}
	\item Five cards—the ten, jack, queen, king and ace of diamonds, are well-shuffled with their face downwards. One card is then picked up at random.
\begin{enumerate}
\item
What is the probability that the card is the queen? 
\item
If the queen is drawn and put aside, what is the probability that the second card picked up is (a) an ace? (b) a queen?\\
\end{enumerate}
\solution
		%\input{ncert/10/15/1/15/defs.tex}
	\item A bag contains $5$ red balls and some blue balls. If the probability of drawing a blue ball is double that if a red ball, determine the number of blue balls in the bag. 
		\\
\solution
		%\input{ncert/10/15/2/3/defs.tex}
	\item A card is selected from a pack of 52 cards.
 \begin{enumerate}[label=(\alph*)] 
                 \item How many points are there in the sample space?
                 \item Calculate the probability that the card is an ace of spades.
                 \item Calculate the probability that the card is (i) an ace and (ii) black card.
 \end{enumerate}
\solution
		%\input{ncert/11/16/3/4/main.tex}
\item Four cards are drawn from a well-shuffled deck of 52 cards. What is the probability of obtaining 3 diamonds and one spade.
\\
\solution
		%\input{ncert/11/16/4/2/defs.tex}
\item In a certain lottery 10,000 tickets are sold and ten equal prizes are awarded. What is the probability of not getting a prize if you buy (a) one ticket (b) two tickets (c) 10 tickets ?	
\\
\solution
		%\input{ncert/11/16/4/4/defs.tex}
		%
\item 
Out of 100 students, two sections of 40 and 60 are formed. If you and your friend are among the 100 students, what is the probability that
\begin{enumerate}
\item you both enter the same section?
\item you both enter the different sections?
\end{enumerate}
\solution
		%\input{ncert/11/16/4/5/defs.tex}
	\item 
The number lock of a suitcase has 4 wheels each labelled with ten digits i.e. from 0 to 9.The lock opens with a sequence of four digits with no repeats.What is the probability of a person getting the right sequence to open the suitcase.
\\
\solution
		%\input{ncert/11/16/4/10/defs.tex}
		%
\item 
Two cards are drawn at random and without replacement from a pack of 52 playing cards. Find the probability that both the cards are black.
\\
\solution
		%\input{ncert/12/13/2/2/defs.tex}
		\item A box of oranges is inspected by examining three randomly selected oranges drawn without replacement. If all the three oranges are good, the box is approved for sale, otherwise, it is rejected. Find the probability that a box containing 15 oranges out of which 12 are good and 3 are bad ones will be approved for sale.
		\label{ncert/12/13/2/3/defs.tex}
		\item Two balls are drawn at random with replacement from a box containing 10 black and 8 red balls. Find the probability that
		\label{ncert/12/13/2/12}
\begin{enumerate}
\item both balls are red.
\item first ball is black and second is red.
\item one of them is black and other is red.
\end{enumerate}

\item In a hostel, 60\% of the students read Hindi newspaper, 40\% read English newspaper and 20\% read both Hindi and English newspapers. A student is selected at random.
		\label{ncert/12/13/2/15}
\begin{enumerate}
\item Find the probability that she reads neither Hindi nor English newspapers.
\item If she reads Hindi newspaper, find the probability that she reads English newspaper.
\item If she reads English newspaper, find the probability that she reads Hindi newspaper.\\
\end{enumerate}
\item The probability of obtaining an even prime number on each die, when a pair of dice is rolled is 
\begin{enumerate}
    \item $0$ 
    
    \item $\frac{1}{3}$ 
    
    \item $\frac{1}{12}$ 
    
    \item $\frac{1}{36}$ 
\end{enumerate}
\solution
		%\input{ncert/12/13/2/17/defs.tex}
	\item A bag contains 4 red and 4 black balls, another bag contains 2 red and 6 black balls. One of the two bags is selected at random and a ball is drawn from the bag which is found to be red. Find the probability that the ball is drawn from the first bag.
\\
\solution
		%\input{ncert/12/13/3/2/main.tex}
  \item
  Cards with numbers 2 to 101 are placed in a box. A card is selected at random.Find the probability that the card has
\begin{enumerate}[label=(\roman*)]
	\item an even number 
	\item a square number
\end{enumerate}
\solution
%\input{exemplar/10/13/3/32/main.tex}
\item
The king, queen and jack of clubs are removed from a deck of 52 playing cards and then well shuffled. Now one card is drawn at random from the remaining cards.  Determine the probability that the card is
\begin{enumerate}[label=(\roman*)]
\item a club
\item 10 of hearts
\end{enumerate}
\solution
%\input{exemplar/10/13/3/29/main.tex}
\item A team of medical students doing their internship have to assist during surgeries
at a city hospital. The probabilities of surgeries rated as very complex, complex,
routine, simple or very simple are respectively, 0.15, 0.20, 0.31, 0.26, .08. Find
the probabilities that a particular surgery will be rated
\begin{enumerate}
	\item complex or very complex;
	\item neither very complex nor very simple;
	\item routine or complex
	\item routine or simple
\end{enumerate}
\solution
%\input{exemplar/11/16/3/8(1)/main.tex}
\item A card is selected from a pack of 52 cards.
\begin{enumerate}[label=(\alph*)]
    \item How many points are there in the sample space?
    \item Calculate the probability that the card is an ace of spades.
    \item Calculate the probability that the card is (i) an ace and (ii) black card.
\end{enumerate}
\solution
%\input{exemplar/11/16/3/4/main2.tex}
\item The probability that a non leap year selected at random will contain 53 sundays.
\\
\solution
%\input{exemplar/10/13/1/19/main.tex}
\item One of the four persons John, Rita, Aslam or Gurpreet will be promoted next
month. Consequently the sample space consists of four elementary outcomes
S = {John promoted, Rita promoted, Aslam promoted, Gurpreet promoted}
You are told that the chances of John’s promotion is same as that of Gurpreet,
Rita’s chances of promotion are twice as likely as Johns. Aslam’s chances are
four times that of John.
\begin{enumerate}
	\item Determine
	\begin{enumerate}
		\item P (John promoted)
		\item P (Rita promoted)
		\item P (Aslam promoted)
		\item P (Gurpreet promoted)
	\end{enumerate}
	\item If A = {John promoted or Gurpreet promoted}, find P (A).
\end{enumerate}
\solution
%\input{exemplar/11/16/3/10/main.tex}
\item A card is drawn from a deck of 52 cards. Find the probability of getting a king or a heart or a red card.\\
\solution
%\input{exemplar/11/16/3/15/main.tex}
\item The probability that a student will pass his examination is 0.73, the probability of
the student getting a compartment is 0.13, and the probability that the student will
either pass or get compartment is 0.96. State True or False.\\
\solution
%\input{exemplar/11/16/3/31/main.tex}
\item A card is selected from a pack of 52 cards\\
\begin{enumerate}[label=(\alph*)]
\item How many points are there in the sample space?
\item Calculate the probability that the cards is an ace of spades.
\item Calculate the probability that the card is (i) an ace (ii)black card.\\
\end{enumerate}
%\input{ncert/11/16/3/4_1/Prob_4.tex}
\item In a non-leap year, the probability of having 53 tuesdays or 53 wednesdays is\\
\solution
%\input{exemplar/11/16/3/18/main.tex}
\item There are 1000 sealed envelopes in a box, 10 of them contain a cash prize of
Rs 100 each, 100 of them contain a cash prize of Rs 50 each and 200 of them
contain a cash prize of Rs 10 each and rest do not contain any cash prize. If they
are well shuffled and an envelope is picked up out, what is the probability that it
contains no cash prize?\\
\solution
%\input{exemplar/10/13/3/34/main.tex}
\item 
A die is thrown and a card is selected at random from a deck of 52 playing cards. The probability of getting an even number on the die and a spade card.\\
\solution
%\input{exemplar/12/13/3/78/main.tex}
\item
If 4-digit numbers greater than 5,000 are randomly formed from the digits 0, 1, 3, 5, and 7, what is the probability of forming a number divisible by 5 when:
\begin{enumerate}
    \item The digits are repeated?
    \item The repetition of digits is not allowed?
\end{enumerate}
\solution
%\input{ncert/11/16/4/9/main.tex}
\item Consider the probability space $\brak{\Omega, \mathcal{G}, P}$ where $\Omega = [0,2]$ and $\mathcal{G} = \cbrak{\phi, \Omega, [0,1], (1,2]}$. Let $X$ and $Y$ be two functions on $\Omega$ defined as
\begin{align*}
    X(\omega) = 
    \begin{cases}
        1 & \text{if }\omega \in [0, 1]\\
        2 & \text{if }\omega \in (1, 2]
    \end{cases}
\end{align*}
and
\begin{align*}
    Y(\omega) = 
    \begin{cases}
        2 & \text{if }\omega \in [0, 1.5]\\
        3 & \text{if }\omega \in (1.5, 2].
    \end{cases}
\end{align*}
Then which one of the following statements is true?
\begin{enumerate}
    \item [(A)] $X$ is a random variable with respect to $\mathcal{G}$, but $Y$ is not a random variable with respect to $\mathcal{G}$.
    \item [(B)] $Y$ is a random variable with respect to $\mathcal{G}$, but $X$ is not a random variable with respect to $\mathcal{G}$.
    \item [(C)] Neither $X$ nor $Y$ is a random variable with respect to $\mathcal{G}$.
    \item [(D)] Both $X$ and $Y$ are random variables with respect to $\mathcal{G}$.
\end{enumerate} \hfill (GATE ST 2023)\\
\solution
%\input{gate/ST/2023/14/main.tex}
	\item  A die is loaded in such a way that each odd number is twice as likely to occur as
each even number. Find $P(G)$, where $G$ is the event that a number greater than
3 occurs on a single roll of the die.
\\
\solution
		%\input{exemplar/11/16/3/5/main.tex}
	\item All the jacks, queens and kings are removed from a deck of 52 playing cards. The remaining cards are well shuffled and then one card is drawn at random. Giving ace a value 1 similar value for other cards, find the probability that the card has a value 
		\begin{enumerate}
			\item 7
			\item greater than 7
			\item less than 7
		\end{enumerate}
		%\input{exemplar/10/13/3/30/main.tex}
  \item A Lot consists of 48 mobile phones of which 42 are good, 3 have only minor defects and 3 have major defects.Varnika will buy a phone if it is good but the trader will only buy a mobile if it has no major defects. One phone is selected at random from the lot. What is the probability that it is
\begin{enumerate}
	\item acceptable to Varnika?
            \item acceptable to the trader?
\end{enumerate}
\solution
	%\input{exemplar/10/13/3/40/main.tex}
 \item A student says that if you throw a die, it will show up 1 or not 1. Therefore, the probability of getting 1 and the probability of getting 'not 1' each is equal to $\frac{1}{2}$. Is this correct? Give reasons.\\
 \solution
        %\input{exemplar/10/13/2/9/main.tex}
   \item Four candidates A, B, C, D have ap-
plied for the assignment to coach a school cricket
team. If A is twice as likely to be selected as B, and
B and C are given about the same chance of being
selected, while C is twice as likely to be selected
as D, what are the probabilities that
\begin{enumerate}
\item C will be selected?
\item A will not be selected?
\end{enumerate}
	%\input{exemplar/11/16/3/9/main.tex}
 \item A bag contain 24 balls of which $x$ balls are red, $2x$ are white and $3x$ are blue. A ball is selected at random, What is the probability that it is
\begin{enumerate}[label=\alph*)]
\item not red ?
\item white ?
\end{enumerate}
%\input{exemplar/10/13/3/41/main.tex}
If the letters of the word ASSASSINATION are arranged at random. Find the Probability that
\begin{enumerate}[label=(\alph*)]
\item Four $S's$ come consecutively in the word
\item Two  $I's$ and two $N's$ come together
\item All $A's$ are not coming together
\item No two $A's$ are coming together
\end{enumerate}
%\input{exemplar/11/16/3/14/main.tex}
	\item One urn contains two black balls (labelled B1 and B2) and one white ball. A
	second urn contains one black ball and two white balls (labelled W1 and W2).
	Suppose the following experiment is performed. One of the two urns is chosen
	at random. Next a ball is randomly chosen from the urn. Then a second ball is
	chosen at random from the same urn without replacing the first ball.
	
	\begin{enumerate}
	\item What is the probability that two black balls are chosen?
	
	\item What is the probability that two balls of opposite colour are chosen?
	\end{enumerate}
	\solution
	%\input{exemplar/11/16/3/12/main1.tex}
\end{enumerate}

\item In a certain lottery 10,000 tickets are sold and ten equal prizes are awarded. What is the probability of not getting a prize if you buy (a) one ticket (b) two tickets (c) 10 tickets ?	
\\
\solution
		%\begin{enumerate}[label=\thesection.\arabic*,ref=\thesection.\theenumi]
	\item One card is drawn from a well-shuffled deck of 52 cards. Find the probability of getting
\begin{enumerate}
\item A king of red colour 
\item A face card 
\item A red face card
\item The jack of hearts
\item A spade
\item The queen of diamonds

\end{enumerate}
\solution
		%\input{ncert/10/15/1/14/main.tex}
	\item Five cards—the ten, jack, queen, king and ace of diamonds, are well-shuffled with their face downwards. One card is then picked up at random.
\begin{enumerate}
\item
What is the probability that the card is the queen? 
\item
If the queen is drawn and put aside, what is the probability that the second card picked up is (a) an ace? (b) a queen?\\
\end{enumerate}
\solution
		%\input{ncert/10/15/1/15/defs.tex}
	\item A bag contains $5$ red balls and some blue balls. If the probability of drawing a blue ball is double that if a red ball, determine the number of blue balls in the bag. 
		\\
\solution
		%\input{ncert/10/15/2/3/defs.tex}
	\item A card is selected from a pack of 52 cards.
 \begin{enumerate}[label=(\alph*)] 
                 \item How many points are there in the sample space?
                 \item Calculate the probability that the card is an ace of spades.
                 \item Calculate the probability that the card is (i) an ace and (ii) black card.
 \end{enumerate}
\solution
		%\input{ncert/11/16/3/4/main.tex}
\item Four cards are drawn from a well-shuffled deck of 52 cards. What is the probability of obtaining 3 diamonds and one spade.
\\
\solution
		%\input{ncert/11/16/4/2/defs.tex}
\item In a certain lottery 10,000 tickets are sold and ten equal prizes are awarded. What is the probability of not getting a prize if you buy (a) one ticket (b) two tickets (c) 10 tickets ?	
\\
\solution
		%\input{ncert/11/16/4/4/defs.tex}
		%
\item 
Out of 100 students, two sections of 40 and 60 are formed. If you and your friend are among the 100 students, what is the probability that
\begin{enumerate}
\item you both enter the same section?
\item you both enter the different sections?
\end{enumerate}
\solution
		%\input{ncert/11/16/4/5/defs.tex}
	\item 
The number lock of a suitcase has 4 wheels each labelled with ten digits i.e. from 0 to 9.The lock opens with a sequence of four digits with no repeats.What is the probability of a person getting the right sequence to open the suitcase.
\\
\solution
		%\input{ncert/11/16/4/10/defs.tex}
		%
\item 
Two cards are drawn at random and without replacement from a pack of 52 playing cards. Find the probability that both the cards are black.
\\
\solution
		%\input{ncert/12/13/2/2/defs.tex}
		\item A box of oranges is inspected by examining three randomly selected oranges drawn without replacement. If all the three oranges are good, the box is approved for sale, otherwise, it is rejected. Find the probability that a box containing 15 oranges out of which 12 are good and 3 are bad ones will be approved for sale.
		\label{ncert/12/13/2/3/defs.tex}
		\item Two balls are drawn at random with replacement from a box containing 10 black and 8 red balls. Find the probability that
		\label{ncert/12/13/2/12}
\begin{enumerate}
\item both balls are red.
\item first ball is black and second is red.
\item one of them is black and other is red.
\end{enumerate}

\item In a hostel, 60\% of the students read Hindi newspaper, 40\% read English newspaper and 20\% read both Hindi and English newspapers. A student is selected at random.
		\label{ncert/12/13/2/15}
\begin{enumerate}
\item Find the probability that she reads neither Hindi nor English newspapers.
\item If she reads Hindi newspaper, find the probability that she reads English newspaper.
\item If she reads English newspaper, find the probability that she reads Hindi newspaper.\\
\end{enumerate}
\item The probability of obtaining an even prime number on each die, when a pair of dice is rolled is 
\begin{enumerate}
    \item $0$ 
    
    \item $\frac{1}{3}$ 
    
    \item $\frac{1}{12}$ 
    
    \item $\frac{1}{36}$ 
\end{enumerate}
\solution
		%\input{ncert/12/13/2/17/defs.tex}
	\item A bag contains 4 red and 4 black balls, another bag contains 2 red and 6 black balls. One of the two bags is selected at random and a ball is drawn from the bag which is found to be red. Find the probability that the ball is drawn from the first bag.
\\
\solution
		%\input{ncert/12/13/3/2/main.tex}
  \item
  Cards with numbers 2 to 101 are placed in a box. A card is selected at random.Find the probability that the card has
\begin{enumerate}[label=(\roman*)]
	\item an even number 
	\item a square number
\end{enumerate}
\solution
%\input{exemplar/10/13/3/32/main.tex}
\item
The king, queen and jack of clubs are removed from a deck of 52 playing cards and then well shuffled. Now one card is drawn at random from the remaining cards.  Determine the probability that the card is
\begin{enumerate}[label=(\roman*)]
\item a club
\item 10 of hearts
\end{enumerate}
\solution
%\input{exemplar/10/13/3/29/main.tex}
\item A team of medical students doing their internship have to assist during surgeries
at a city hospital. The probabilities of surgeries rated as very complex, complex,
routine, simple or very simple are respectively, 0.15, 0.20, 0.31, 0.26, .08. Find
the probabilities that a particular surgery will be rated
\begin{enumerate}
	\item complex or very complex;
	\item neither very complex nor very simple;
	\item routine or complex
	\item routine or simple
\end{enumerate}
\solution
%\input{exemplar/11/16/3/8(1)/main.tex}
\item A card is selected from a pack of 52 cards.
\begin{enumerate}[label=(\alph*)]
    \item How many points are there in the sample space?
    \item Calculate the probability that the card is an ace of spades.
    \item Calculate the probability that the card is (i) an ace and (ii) black card.
\end{enumerate}
\solution
%\input{exemplar/11/16/3/4/main2.tex}
\item The probability that a non leap year selected at random will contain 53 sundays.
\\
\solution
%\input{exemplar/10/13/1/19/main.tex}
\item One of the four persons John, Rita, Aslam or Gurpreet will be promoted next
month. Consequently the sample space consists of four elementary outcomes
S = {John promoted, Rita promoted, Aslam promoted, Gurpreet promoted}
You are told that the chances of John’s promotion is same as that of Gurpreet,
Rita’s chances of promotion are twice as likely as Johns. Aslam’s chances are
four times that of John.
\begin{enumerate}
	\item Determine
	\begin{enumerate}
		\item P (John promoted)
		\item P (Rita promoted)
		\item P (Aslam promoted)
		\item P (Gurpreet promoted)
	\end{enumerate}
	\item If A = {John promoted or Gurpreet promoted}, find P (A).
\end{enumerate}
\solution
%\input{exemplar/11/16/3/10/main.tex}
\item A card is drawn from a deck of 52 cards. Find the probability of getting a king or a heart or a red card.\\
\solution
%\input{exemplar/11/16/3/15/main.tex}
\item The probability that a student will pass his examination is 0.73, the probability of
the student getting a compartment is 0.13, and the probability that the student will
either pass or get compartment is 0.96. State True or False.\\
\solution
%\input{exemplar/11/16/3/31/main.tex}
\item A card is selected from a pack of 52 cards\\
\begin{enumerate}[label=(\alph*)]
\item How many points are there in the sample space?
\item Calculate the probability that the cards is an ace of spades.
\item Calculate the probability that the card is (i) an ace (ii)black card.\\
\end{enumerate}
%\input{ncert/11/16/3/4_1/Prob_4.tex}
\item In a non-leap year, the probability of having 53 tuesdays or 53 wednesdays is\\
\solution
%\input{exemplar/11/16/3/18/main.tex}
\item There are 1000 sealed envelopes in a box, 10 of them contain a cash prize of
Rs 100 each, 100 of them contain a cash prize of Rs 50 each and 200 of them
contain a cash prize of Rs 10 each and rest do not contain any cash prize. If they
are well shuffled and an envelope is picked up out, what is the probability that it
contains no cash prize?\\
\solution
%\input{exemplar/10/13/3/34/main.tex}
\item 
A die is thrown and a card is selected at random from a deck of 52 playing cards. The probability of getting an even number on the die and a spade card.\\
\solution
%\input{exemplar/12/13/3/78/main.tex}
\item
If 4-digit numbers greater than 5,000 are randomly formed from the digits 0, 1, 3, 5, and 7, what is the probability of forming a number divisible by 5 when:
\begin{enumerate}
    \item The digits are repeated?
    \item The repetition of digits is not allowed?
\end{enumerate}
\solution
%\input{ncert/11/16/4/9/main.tex}
\item Consider the probability space $\brak{\Omega, \mathcal{G}, P}$ where $\Omega = [0,2]$ and $\mathcal{G} = \cbrak{\phi, \Omega, [0,1], (1,2]}$. Let $X$ and $Y$ be two functions on $\Omega$ defined as
\begin{align*}
    X(\omega) = 
    \begin{cases}
        1 & \text{if }\omega \in [0, 1]\\
        2 & \text{if }\omega \in (1, 2]
    \end{cases}
\end{align*}
and
\begin{align*}
    Y(\omega) = 
    \begin{cases}
        2 & \text{if }\omega \in [0, 1.5]\\
        3 & \text{if }\omega \in (1.5, 2].
    \end{cases}
\end{align*}
Then which one of the following statements is true?
\begin{enumerate}
    \item [(A)] $X$ is a random variable with respect to $\mathcal{G}$, but $Y$ is not a random variable with respect to $\mathcal{G}$.
    \item [(B)] $Y$ is a random variable with respect to $\mathcal{G}$, but $X$ is not a random variable with respect to $\mathcal{G}$.
    \item [(C)] Neither $X$ nor $Y$ is a random variable with respect to $\mathcal{G}$.
    \item [(D)] Both $X$ and $Y$ are random variables with respect to $\mathcal{G}$.
\end{enumerate} \hfill (GATE ST 2023)\\
\solution
%\input{gate/ST/2023/14/main.tex}
	\item  A die is loaded in such a way that each odd number is twice as likely to occur as
each even number. Find $P(G)$, where $G$ is the event that a number greater than
3 occurs on a single roll of the die.
\\
\solution
		%\input{exemplar/11/16/3/5/main.tex}
	\item All the jacks, queens and kings are removed from a deck of 52 playing cards. The remaining cards are well shuffled and then one card is drawn at random. Giving ace a value 1 similar value for other cards, find the probability that the card has a value 
		\begin{enumerate}
			\item 7
			\item greater than 7
			\item less than 7
		\end{enumerate}
		%\input{exemplar/10/13/3/30/main.tex}
  \item A Lot consists of 48 mobile phones of which 42 are good, 3 have only minor defects and 3 have major defects.Varnika will buy a phone if it is good but the trader will only buy a mobile if it has no major defects. One phone is selected at random from the lot. What is the probability that it is
\begin{enumerate}
	\item acceptable to Varnika?
            \item acceptable to the trader?
\end{enumerate}
\solution
	%\input{exemplar/10/13/3/40/main.tex}
 \item A student says that if you throw a die, it will show up 1 or not 1. Therefore, the probability of getting 1 and the probability of getting 'not 1' each is equal to $\frac{1}{2}$. Is this correct? Give reasons.\\
 \solution
        %\input{exemplar/10/13/2/9/main.tex}
   \item Four candidates A, B, C, D have ap-
plied for the assignment to coach a school cricket
team. If A is twice as likely to be selected as B, and
B and C are given about the same chance of being
selected, while C is twice as likely to be selected
as D, what are the probabilities that
\begin{enumerate}
\item C will be selected?
\item A will not be selected?
\end{enumerate}
	%\input{exemplar/11/16/3/9/main.tex}
 \item A bag contain 24 balls of which $x$ balls are red, $2x$ are white and $3x$ are blue. A ball is selected at random, What is the probability that it is
\begin{enumerate}[label=\alph*)]
\item not red ?
\item white ?
\end{enumerate}
%\input{exemplar/10/13/3/41/main.tex}
If the letters of the word ASSASSINATION are arranged at random. Find the Probability that
\begin{enumerate}[label=(\alph*)]
\item Four $S's$ come consecutively in the word
\item Two  $I's$ and two $N's$ come together
\item All $A's$ are not coming together
\item No two $A's$ are coming together
\end{enumerate}
%\input{exemplar/11/16/3/14/main.tex}
	\item One urn contains two black balls (labelled B1 and B2) and one white ball. A
	second urn contains one black ball and two white balls (labelled W1 and W2).
	Suppose the following experiment is performed. One of the two urns is chosen
	at random. Next a ball is randomly chosen from the urn. Then a second ball is
	chosen at random from the same urn without replacing the first ball.
	
	\begin{enumerate}
	\item What is the probability that two black balls are chosen?
	
	\item What is the probability that two balls of opposite colour are chosen?
	\end{enumerate}
	\solution
	%\input{exemplar/11/16/3/12/main1.tex}
\end{enumerate}

		%
\item 
Out of 100 students, two sections of 40 and 60 are formed. If you and your friend are among the 100 students, what is the probability that
\begin{enumerate}
\item you both enter the same section?
\item you both enter the different sections?
\end{enumerate}
\solution
		%\begin{enumerate}[label=\thesection.\arabic*,ref=\thesection.\theenumi]
	\item One card is drawn from a well-shuffled deck of 52 cards. Find the probability of getting
\begin{enumerate}
\item A king of red colour 
\item A face card 
\item A red face card
\item The jack of hearts
\item A spade
\item The queen of diamonds

\end{enumerate}
\solution
		%\input{ncert/10/15/1/14/main.tex}
	\item Five cards—the ten, jack, queen, king and ace of diamonds, are well-shuffled with their face downwards. One card is then picked up at random.
\begin{enumerate}
\item
What is the probability that the card is the queen? 
\item
If the queen is drawn and put aside, what is the probability that the second card picked up is (a) an ace? (b) a queen?\\
\end{enumerate}
\solution
		%\input{ncert/10/15/1/15/defs.tex}
	\item A bag contains $5$ red balls and some blue balls. If the probability of drawing a blue ball is double that if a red ball, determine the number of blue balls in the bag. 
		\\
\solution
		%\input{ncert/10/15/2/3/defs.tex}
	\item A card is selected from a pack of 52 cards.
 \begin{enumerate}[label=(\alph*)] 
                 \item How many points are there in the sample space?
                 \item Calculate the probability that the card is an ace of spades.
                 \item Calculate the probability that the card is (i) an ace and (ii) black card.
 \end{enumerate}
\solution
		%\input{ncert/11/16/3/4/main.tex}
\item Four cards are drawn from a well-shuffled deck of 52 cards. What is the probability of obtaining 3 diamonds and one spade.
\\
\solution
		%\input{ncert/11/16/4/2/defs.tex}
\item In a certain lottery 10,000 tickets are sold and ten equal prizes are awarded. What is the probability of not getting a prize if you buy (a) one ticket (b) two tickets (c) 10 tickets ?	
\\
\solution
		%\input{ncert/11/16/4/4/defs.tex}
		%
\item 
Out of 100 students, two sections of 40 and 60 are formed. If you and your friend are among the 100 students, what is the probability that
\begin{enumerate}
\item you both enter the same section?
\item you both enter the different sections?
\end{enumerate}
\solution
		%\input{ncert/11/16/4/5/defs.tex}
	\item 
The number lock of a suitcase has 4 wheels each labelled with ten digits i.e. from 0 to 9.The lock opens with a sequence of four digits with no repeats.What is the probability of a person getting the right sequence to open the suitcase.
\\
\solution
		%\input{ncert/11/16/4/10/defs.tex}
		%
\item 
Two cards are drawn at random and without replacement from a pack of 52 playing cards. Find the probability that both the cards are black.
\\
\solution
		%\input{ncert/12/13/2/2/defs.tex}
		\item A box of oranges is inspected by examining three randomly selected oranges drawn without replacement. If all the three oranges are good, the box is approved for sale, otherwise, it is rejected. Find the probability that a box containing 15 oranges out of which 12 are good and 3 are bad ones will be approved for sale.
		\label{ncert/12/13/2/3/defs.tex}
		\item Two balls are drawn at random with replacement from a box containing 10 black and 8 red balls. Find the probability that
		\label{ncert/12/13/2/12}
\begin{enumerate}
\item both balls are red.
\item first ball is black and second is red.
\item one of them is black and other is red.
\end{enumerate}

\item In a hostel, 60\% of the students read Hindi newspaper, 40\% read English newspaper and 20\% read both Hindi and English newspapers. A student is selected at random.
		\label{ncert/12/13/2/15}
\begin{enumerate}
\item Find the probability that she reads neither Hindi nor English newspapers.
\item If she reads Hindi newspaper, find the probability that she reads English newspaper.
\item If she reads English newspaper, find the probability that she reads Hindi newspaper.\\
\end{enumerate}
\item The probability of obtaining an even prime number on each die, when a pair of dice is rolled is 
\begin{enumerate}
    \item $0$ 
    
    \item $\frac{1}{3}$ 
    
    \item $\frac{1}{12}$ 
    
    \item $\frac{1}{36}$ 
\end{enumerate}
\solution
		%\input{ncert/12/13/2/17/defs.tex}
	\item A bag contains 4 red and 4 black balls, another bag contains 2 red and 6 black balls. One of the two bags is selected at random and a ball is drawn from the bag which is found to be red. Find the probability that the ball is drawn from the first bag.
\\
\solution
		%\input{ncert/12/13/3/2/main.tex}
  \item
  Cards with numbers 2 to 101 are placed in a box. A card is selected at random.Find the probability that the card has
\begin{enumerate}[label=(\roman*)]
	\item an even number 
	\item a square number
\end{enumerate}
\solution
%\input{exemplar/10/13/3/32/main.tex}
\item
The king, queen and jack of clubs are removed from a deck of 52 playing cards and then well shuffled. Now one card is drawn at random from the remaining cards.  Determine the probability that the card is
\begin{enumerate}[label=(\roman*)]
\item a club
\item 10 of hearts
\end{enumerate}
\solution
%\input{exemplar/10/13/3/29/main.tex}
\item A team of medical students doing their internship have to assist during surgeries
at a city hospital. The probabilities of surgeries rated as very complex, complex,
routine, simple or very simple are respectively, 0.15, 0.20, 0.31, 0.26, .08. Find
the probabilities that a particular surgery will be rated
\begin{enumerate}
	\item complex or very complex;
	\item neither very complex nor very simple;
	\item routine or complex
	\item routine or simple
\end{enumerate}
\solution
%\input{exemplar/11/16/3/8(1)/main.tex}
\item A card is selected from a pack of 52 cards.
\begin{enumerate}[label=(\alph*)]
    \item How many points are there in the sample space?
    \item Calculate the probability that the card is an ace of spades.
    \item Calculate the probability that the card is (i) an ace and (ii) black card.
\end{enumerate}
\solution
%\input{exemplar/11/16/3/4/main2.tex}
\item The probability that a non leap year selected at random will contain 53 sundays.
\\
\solution
%\input{exemplar/10/13/1/19/main.tex}
\item One of the four persons John, Rita, Aslam or Gurpreet will be promoted next
month. Consequently the sample space consists of four elementary outcomes
S = {John promoted, Rita promoted, Aslam promoted, Gurpreet promoted}
You are told that the chances of John’s promotion is same as that of Gurpreet,
Rita’s chances of promotion are twice as likely as Johns. Aslam’s chances are
four times that of John.
\begin{enumerate}
	\item Determine
	\begin{enumerate}
		\item P (John promoted)
		\item P (Rita promoted)
		\item P (Aslam promoted)
		\item P (Gurpreet promoted)
	\end{enumerate}
	\item If A = {John promoted or Gurpreet promoted}, find P (A).
\end{enumerate}
\solution
%\input{exemplar/11/16/3/10/main.tex}
\item A card is drawn from a deck of 52 cards. Find the probability of getting a king or a heart or a red card.\\
\solution
%\input{exemplar/11/16/3/15/main.tex}
\item The probability that a student will pass his examination is 0.73, the probability of
the student getting a compartment is 0.13, and the probability that the student will
either pass or get compartment is 0.96. State True or False.\\
\solution
%\input{exemplar/11/16/3/31/main.tex}
\item A card is selected from a pack of 52 cards\\
\begin{enumerate}[label=(\alph*)]
\item How many points are there in the sample space?
\item Calculate the probability that the cards is an ace of spades.
\item Calculate the probability that the card is (i) an ace (ii)black card.\\
\end{enumerate}
%\input{ncert/11/16/3/4_1/Prob_4.tex}
\item In a non-leap year, the probability of having 53 tuesdays or 53 wednesdays is\\
\solution
%\input{exemplar/11/16/3/18/main.tex}
\item There are 1000 sealed envelopes in a box, 10 of them contain a cash prize of
Rs 100 each, 100 of them contain a cash prize of Rs 50 each and 200 of them
contain a cash prize of Rs 10 each and rest do not contain any cash prize. If they
are well shuffled and an envelope is picked up out, what is the probability that it
contains no cash prize?\\
\solution
%\input{exemplar/10/13/3/34/main.tex}
\item 
A die is thrown and a card is selected at random from a deck of 52 playing cards. The probability of getting an even number on the die and a spade card.\\
\solution
%\input{exemplar/12/13/3/78/main.tex}
\item
If 4-digit numbers greater than 5,000 are randomly formed from the digits 0, 1, 3, 5, and 7, what is the probability of forming a number divisible by 5 when:
\begin{enumerate}
    \item The digits are repeated?
    \item The repetition of digits is not allowed?
\end{enumerate}
\solution
%\input{ncert/11/16/4/9/main.tex}
\item Consider the probability space $\brak{\Omega, \mathcal{G}, P}$ where $\Omega = [0,2]$ and $\mathcal{G} = \cbrak{\phi, \Omega, [0,1], (1,2]}$. Let $X$ and $Y$ be two functions on $\Omega$ defined as
\begin{align*}
    X(\omega) = 
    \begin{cases}
        1 & \text{if }\omega \in [0, 1]\\
        2 & \text{if }\omega \in (1, 2]
    \end{cases}
\end{align*}
and
\begin{align*}
    Y(\omega) = 
    \begin{cases}
        2 & \text{if }\omega \in [0, 1.5]\\
        3 & \text{if }\omega \in (1.5, 2].
    \end{cases}
\end{align*}
Then which one of the following statements is true?
\begin{enumerate}
    \item [(A)] $X$ is a random variable with respect to $\mathcal{G}$, but $Y$ is not a random variable with respect to $\mathcal{G}$.
    \item [(B)] $Y$ is a random variable with respect to $\mathcal{G}$, but $X$ is not a random variable with respect to $\mathcal{G}$.
    \item [(C)] Neither $X$ nor $Y$ is a random variable with respect to $\mathcal{G}$.
    \item [(D)] Both $X$ and $Y$ are random variables with respect to $\mathcal{G}$.
\end{enumerate} \hfill (GATE ST 2023)\\
\solution
%\input{gate/ST/2023/14/main.tex}
	\item  A die is loaded in such a way that each odd number is twice as likely to occur as
each even number. Find $P(G)$, where $G$ is the event that a number greater than
3 occurs on a single roll of the die.
\\
\solution
		%\input{exemplar/11/16/3/5/main.tex}
	\item All the jacks, queens and kings are removed from a deck of 52 playing cards. The remaining cards are well shuffled and then one card is drawn at random. Giving ace a value 1 similar value for other cards, find the probability that the card has a value 
		\begin{enumerate}
			\item 7
			\item greater than 7
			\item less than 7
		\end{enumerate}
		%\input{exemplar/10/13/3/30/main.tex}
  \item A Lot consists of 48 mobile phones of which 42 are good, 3 have only minor defects and 3 have major defects.Varnika will buy a phone if it is good but the trader will only buy a mobile if it has no major defects. One phone is selected at random from the lot. What is the probability that it is
\begin{enumerate}
	\item acceptable to Varnika?
            \item acceptable to the trader?
\end{enumerate}
\solution
	%\input{exemplar/10/13/3/40/main.tex}
 \item A student says that if you throw a die, it will show up 1 or not 1. Therefore, the probability of getting 1 and the probability of getting 'not 1' each is equal to $\frac{1}{2}$. Is this correct? Give reasons.\\
 \solution
        %\input{exemplar/10/13/2/9/main.tex}
   \item Four candidates A, B, C, D have ap-
plied for the assignment to coach a school cricket
team. If A is twice as likely to be selected as B, and
B and C are given about the same chance of being
selected, while C is twice as likely to be selected
as D, what are the probabilities that
\begin{enumerate}
\item C will be selected?
\item A will not be selected?
\end{enumerate}
	%\input{exemplar/11/16/3/9/main.tex}
 \item A bag contain 24 balls of which $x$ balls are red, $2x$ are white and $3x$ are blue. A ball is selected at random, What is the probability that it is
\begin{enumerate}[label=\alph*)]
\item not red ?
\item white ?
\end{enumerate}
%\input{exemplar/10/13/3/41/main.tex}
If the letters of the word ASSASSINATION are arranged at random. Find the Probability that
\begin{enumerate}[label=(\alph*)]
\item Four $S's$ come consecutively in the word
\item Two  $I's$ and two $N's$ come together
\item All $A's$ are not coming together
\item No two $A's$ are coming together
\end{enumerate}
%\input{exemplar/11/16/3/14/main.tex}
	\item One urn contains two black balls (labelled B1 and B2) and one white ball. A
	second urn contains one black ball and two white balls (labelled W1 and W2).
	Suppose the following experiment is performed. One of the two urns is chosen
	at random. Next a ball is randomly chosen from the urn. Then a second ball is
	chosen at random from the same urn without replacing the first ball.
	
	\begin{enumerate}
	\item What is the probability that two black balls are chosen?
	
	\item What is the probability that two balls of opposite colour are chosen?
	\end{enumerate}
	\solution
	%\input{exemplar/11/16/3/12/main1.tex}
\end{enumerate}

	\item 
The number lock of a suitcase has 4 wheels each labelled with ten digits i.e. from 0 to 9.The lock opens with a sequence of four digits with no repeats.What is the probability of a person getting the right sequence to open the suitcase.
\\
\solution
		%\begin{enumerate}[label=\thesection.\arabic*,ref=\thesection.\theenumi]
	\item One card is drawn from a well-shuffled deck of 52 cards. Find the probability of getting
\begin{enumerate}
\item A king of red colour 
\item A face card 
\item A red face card
\item The jack of hearts
\item A spade
\item The queen of diamonds

\end{enumerate}
\solution
		%\input{ncert/10/15/1/14/main.tex}
	\item Five cards—the ten, jack, queen, king and ace of diamonds, are well-shuffled with their face downwards. One card is then picked up at random.
\begin{enumerate}
\item
What is the probability that the card is the queen? 
\item
If the queen is drawn and put aside, what is the probability that the second card picked up is (a) an ace? (b) a queen?\\
\end{enumerate}
\solution
		%\input{ncert/10/15/1/15/defs.tex}
	\item A bag contains $5$ red balls and some blue balls. If the probability of drawing a blue ball is double that if a red ball, determine the number of blue balls in the bag. 
		\\
\solution
		%\input{ncert/10/15/2/3/defs.tex}
	\item A card is selected from a pack of 52 cards.
 \begin{enumerate}[label=(\alph*)] 
                 \item How many points are there in the sample space?
                 \item Calculate the probability that the card is an ace of spades.
                 \item Calculate the probability that the card is (i) an ace and (ii) black card.
 \end{enumerate}
\solution
		%\input{ncert/11/16/3/4/main.tex}
\item Four cards are drawn from a well-shuffled deck of 52 cards. What is the probability of obtaining 3 diamonds and one spade.
\\
\solution
		%\input{ncert/11/16/4/2/defs.tex}
\item In a certain lottery 10,000 tickets are sold and ten equal prizes are awarded. What is the probability of not getting a prize if you buy (a) one ticket (b) two tickets (c) 10 tickets ?	
\\
\solution
		%\input{ncert/11/16/4/4/defs.tex}
		%
\item 
Out of 100 students, two sections of 40 and 60 are formed. If you and your friend are among the 100 students, what is the probability that
\begin{enumerate}
\item you both enter the same section?
\item you both enter the different sections?
\end{enumerate}
\solution
		%\input{ncert/11/16/4/5/defs.tex}
	\item 
The number lock of a suitcase has 4 wheels each labelled with ten digits i.e. from 0 to 9.The lock opens with a sequence of four digits with no repeats.What is the probability of a person getting the right sequence to open the suitcase.
\\
\solution
		%\input{ncert/11/16/4/10/defs.tex}
		%
\item 
Two cards are drawn at random and without replacement from a pack of 52 playing cards. Find the probability that both the cards are black.
\\
\solution
		%\input{ncert/12/13/2/2/defs.tex}
		\item A box of oranges is inspected by examining three randomly selected oranges drawn without replacement. If all the three oranges are good, the box is approved for sale, otherwise, it is rejected. Find the probability that a box containing 15 oranges out of which 12 are good and 3 are bad ones will be approved for sale.
		\label{ncert/12/13/2/3/defs.tex}
		\item Two balls are drawn at random with replacement from a box containing 10 black and 8 red balls. Find the probability that
		\label{ncert/12/13/2/12}
\begin{enumerate}
\item both balls are red.
\item first ball is black and second is red.
\item one of them is black and other is red.
\end{enumerate}

\item In a hostel, 60\% of the students read Hindi newspaper, 40\% read English newspaper and 20\% read both Hindi and English newspapers. A student is selected at random.
		\label{ncert/12/13/2/15}
\begin{enumerate}
\item Find the probability that she reads neither Hindi nor English newspapers.
\item If she reads Hindi newspaper, find the probability that she reads English newspaper.
\item If she reads English newspaper, find the probability that she reads Hindi newspaper.\\
\end{enumerate}
\item The probability of obtaining an even prime number on each die, when a pair of dice is rolled is 
\begin{enumerate}
    \item $0$ 
    
    \item $\frac{1}{3}$ 
    
    \item $\frac{1}{12}$ 
    
    \item $\frac{1}{36}$ 
\end{enumerate}
\solution
		%\input{ncert/12/13/2/17/defs.tex}
	\item A bag contains 4 red and 4 black balls, another bag contains 2 red and 6 black balls. One of the two bags is selected at random and a ball is drawn from the bag which is found to be red. Find the probability that the ball is drawn from the first bag.
\\
\solution
		%\input{ncert/12/13/3/2/main.tex}
  \item
  Cards with numbers 2 to 101 are placed in a box. A card is selected at random.Find the probability that the card has
\begin{enumerate}[label=(\roman*)]
	\item an even number 
	\item a square number
\end{enumerate}
\solution
%\input{exemplar/10/13/3/32/main.tex}
\item
The king, queen and jack of clubs are removed from a deck of 52 playing cards and then well shuffled. Now one card is drawn at random from the remaining cards.  Determine the probability that the card is
\begin{enumerate}[label=(\roman*)]
\item a club
\item 10 of hearts
\end{enumerate}
\solution
%\input{exemplar/10/13/3/29/main.tex}
\item A team of medical students doing their internship have to assist during surgeries
at a city hospital. The probabilities of surgeries rated as very complex, complex,
routine, simple or very simple are respectively, 0.15, 0.20, 0.31, 0.26, .08. Find
the probabilities that a particular surgery will be rated
\begin{enumerate}
	\item complex or very complex;
	\item neither very complex nor very simple;
	\item routine or complex
	\item routine or simple
\end{enumerate}
\solution
%\input{exemplar/11/16/3/8(1)/main.tex}
\item A card is selected from a pack of 52 cards.
\begin{enumerate}[label=(\alph*)]
    \item How many points are there in the sample space?
    \item Calculate the probability that the card is an ace of spades.
    \item Calculate the probability that the card is (i) an ace and (ii) black card.
\end{enumerate}
\solution
%\input{exemplar/11/16/3/4/main2.tex}
\item The probability that a non leap year selected at random will contain 53 sundays.
\\
\solution
%\input{exemplar/10/13/1/19/main.tex}
\item One of the four persons John, Rita, Aslam or Gurpreet will be promoted next
month. Consequently the sample space consists of four elementary outcomes
S = {John promoted, Rita promoted, Aslam promoted, Gurpreet promoted}
You are told that the chances of John’s promotion is same as that of Gurpreet,
Rita’s chances of promotion are twice as likely as Johns. Aslam’s chances are
four times that of John.
\begin{enumerate}
	\item Determine
	\begin{enumerate}
		\item P (John promoted)
		\item P (Rita promoted)
		\item P (Aslam promoted)
		\item P (Gurpreet promoted)
	\end{enumerate}
	\item If A = {John promoted or Gurpreet promoted}, find P (A).
\end{enumerate}
\solution
%\input{exemplar/11/16/3/10/main.tex}
\item A card is drawn from a deck of 52 cards. Find the probability of getting a king or a heart or a red card.\\
\solution
%\input{exemplar/11/16/3/15/main.tex}
\item The probability that a student will pass his examination is 0.73, the probability of
the student getting a compartment is 0.13, and the probability that the student will
either pass or get compartment is 0.96. State True or False.\\
\solution
%\input{exemplar/11/16/3/31/main.tex}
\item A card is selected from a pack of 52 cards\\
\begin{enumerate}[label=(\alph*)]
\item How many points are there in the sample space?
\item Calculate the probability that the cards is an ace of spades.
\item Calculate the probability that the card is (i) an ace (ii)black card.\\
\end{enumerate}
%\input{ncert/11/16/3/4_1/Prob_4.tex}
\item In a non-leap year, the probability of having 53 tuesdays or 53 wednesdays is\\
\solution
%\input{exemplar/11/16/3/18/main.tex}
\item There are 1000 sealed envelopes in a box, 10 of them contain a cash prize of
Rs 100 each, 100 of them contain a cash prize of Rs 50 each and 200 of them
contain a cash prize of Rs 10 each and rest do not contain any cash prize. If they
are well shuffled and an envelope is picked up out, what is the probability that it
contains no cash prize?\\
\solution
%\input{exemplar/10/13/3/34/main.tex}
\item 
A die is thrown and a card is selected at random from a deck of 52 playing cards. The probability of getting an even number on the die and a spade card.\\
\solution
%\input{exemplar/12/13/3/78/main.tex}
\item
If 4-digit numbers greater than 5,000 are randomly formed from the digits 0, 1, 3, 5, and 7, what is the probability of forming a number divisible by 5 when:
\begin{enumerate}
    \item The digits are repeated?
    \item The repetition of digits is not allowed?
\end{enumerate}
\solution
%\input{ncert/11/16/4/9/main.tex}
\item Consider the probability space $\brak{\Omega, \mathcal{G}, P}$ where $\Omega = [0,2]$ and $\mathcal{G} = \cbrak{\phi, \Omega, [0,1], (1,2]}$. Let $X$ and $Y$ be two functions on $\Omega$ defined as
\begin{align*}
    X(\omega) = 
    \begin{cases}
        1 & \text{if }\omega \in [0, 1]\\
        2 & \text{if }\omega \in (1, 2]
    \end{cases}
\end{align*}
and
\begin{align*}
    Y(\omega) = 
    \begin{cases}
        2 & \text{if }\omega \in [0, 1.5]\\
        3 & \text{if }\omega \in (1.5, 2].
    \end{cases}
\end{align*}
Then which one of the following statements is true?
\begin{enumerate}
    \item [(A)] $X$ is a random variable with respect to $\mathcal{G}$, but $Y$ is not a random variable with respect to $\mathcal{G}$.
    \item [(B)] $Y$ is a random variable with respect to $\mathcal{G}$, but $X$ is not a random variable with respect to $\mathcal{G}$.
    \item [(C)] Neither $X$ nor $Y$ is a random variable with respect to $\mathcal{G}$.
    \item [(D)] Both $X$ and $Y$ are random variables with respect to $\mathcal{G}$.
\end{enumerate} \hfill (GATE ST 2023)\\
\solution
%\input{gate/ST/2023/14/main.tex}
	\item  A die is loaded in such a way that each odd number is twice as likely to occur as
each even number. Find $P(G)$, where $G$ is the event that a number greater than
3 occurs on a single roll of the die.
\\
\solution
		%\input{exemplar/11/16/3/5/main.tex}
	\item All the jacks, queens and kings are removed from a deck of 52 playing cards. The remaining cards are well shuffled and then one card is drawn at random. Giving ace a value 1 similar value for other cards, find the probability that the card has a value 
		\begin{enumerate}
			\item 7
			\item greater than 7
			\item less than 7
		\end{enumerate}
		%\input{exemplar/10/13/3/30/main.tex}
  \item A Lot consists of 48 mobile phones of which 42 are good, 3 have only minor defects and 3 have major defects.Varnika will buy a phone if it is good but the trader will only buy a mobile if it has no major defects. One phone is selected at random from the lot. What is the probability that it is
\begin{enumerate}
	\item acceptable to Varnika?
            \item acceptable to the trader?
\end{enumerate}
\solution
	%\input{exemplar/10/13/3/40/main.tex}
 \item A student says that if you throw a die, it will show up 1 or not 1. Therefore, the probability of getting 1 and the probability of getting 'not 1' each is equal to $\frac{1}{2}$. Is this correct? Give reasons.\\
 \solution
        %\input{exemplar/10/13/2/9/main.tex}
   \item Four candidates A, B, C, D have ap-
plied for the assignment to coach a school cricket
team. If A is twice as likely to be selected as B, and
B and C are given about the same chance of being
selected, while C is twice as likely to be selected
as D, what are the probabilities that
\begin{enumerate}
\item C will be selected?
\item A will not be selected?
\end{enumerate}
	%\input{exemplar/11/16/3/9/main.tex}
 \item A bag contain 24 balls of which $x$ balls are red, $2x$ are white and $3x$ are blue. A ball is selected at random, What is the probability that it is
\begin{enumerate}[label=\alph*)]
\item not red ?
\item white ?
\end{enumerate}
%\input{exemplar/10/13/3/41/main.tex}
If the letters of the word ASSASSINATION are arranged at random. Find the Probability that
\begin{enumerate}[label=(\alph*)]
\item Four $S's$ come consecutively in the word
\item Two  $I's$ and two $N's$ come together
\item All $A's$ are not coming together
\item No two $A's$ are coming together
\end{enumerate}
%\input{exemplar/11/16/3/14/main.tex}
	\item One urn contains two black balls (labelled B1 and B2) and one white ball. A
	second urn contains one black ball and two white balls (labelled W1 and W2).
	Suppose the following experiment is performed. One of the two urns is chosen
	at random. Next a ball is randomly chosen from the urn. Then a second ball is
	chosen at random from the same urn without replacing the first ball.
	
	\begin{enumerate}
	\item What is the probability that two black balls are chosen?
	
	\item What is the probability that two balls of opposite colour are chosen?
	\end{enumerate}
	\solution
	%\input{exemplar/11/16/3/12/main1.tex}
\end{enumerate}

		%
\item 
Two cards are drawn at random and without replacement from a pack of 52 playing cards. Find the probability that both the cards are black.
\\
\solution
		%\begin{enumerate}[label=\thesection.\arabic*,ref=\thesection.\theenumi]
	\item One card is drawn from a well-shuffled deck of 52 cards. Find the probability of getting
\begin{enumerate}
\item A king of red colour 
\item A face card 
\item A red face card
\item The jack of hearts
\item A spade
\item The queen of diamonds

\end{enumerate}
\solution
		%\input{ncert/10/15/1/14/main.tex}
	\item Five cards—the ten, jack, queen, king and ace of diamonds, are well-shuffled with their face downwards. One card is then picked up at random.
\begin{enumerate}
\item
What is the probability that the card is the queen? 
\item
If the queen is drawn and put aside, what is the probability that the second card picked up is (a) an ace? (b) a queen?\\
\end{enumerate}
\solution
		%\input{ncert/10/15/1/15/defs.tex}
	\item A bag contains $5$ red balls and some blue balls. If the probability of drawing a blue ball is double that if a red ball, determine the number of blue balls in the bag. 
		\\
\solution
		%\input{ncert/10/15/2/3/defs.tex}
	\item A card is selected from a pack of 52 cards.
 \begin{enumerate}[label=(\alph*)] 
                 \item How many points are there in the sample space?
                 \item Calculate the probability that the card is an ace of spades.
                 \item Calculate the probability that the card is (i) an ace and (ii) black card.
 \end{enumerate}
\solution
		%\input{ncert/11/16/3/4/main.tex}
\item Four cards are drawn from a well-shuffled deck of 52 cards. What is the probability of obtaining 3 diamonds and one spade.
\\
\solution
		%\input{ncert/11/16/4/2/defs.tex}
\item In a certain lottery 10,000 tickets are sold and ten equal prizes are awarded. What is the probability of not getting a prize if you buy (a) one ticket (b) two tickets (c) 10 tickets ?	
\\
\solution
		%\input{ncert/11/16/4/4/defs.tex}
		%
\item 
Out of 100 students, two sections of 40 and 60 are formed. If you and your friend are among the 100 students, what is the probability that
\begin{enumerate}
\item you both enter the same section?
\item you both enter the different sections?
\end{enumerate}
\solution
		%\input{ncert/11/16/4/5/defs.tex}
	\item 
The number lock of a suitcase has 4 wheels each labelled with ten digits i.e. from 0 to 9.The lock opens with a sequence of four digits with no repeats.What is the probability of a person getting the right sequence to open the suitcase.
\\
\solution
		%\input{ncert/11/16/4/10/defs.tex}
		%
\item 
Two cards are drawn at random and without replacement from a pack of 52 playing cards. Find the probability that both the cards are black.
\\
\solution
		%\input{ncert/12/13/2/2/defs.tex}
		\item A box of oranges is inspected by examining three randomly selected oranges drawn without replacement. If all the three oranges are good, the box is approved for sale, otherwise, it is rejected. Find the probability that a box containing 15 oranges out of which 12 are good and 3 are bad ones will be approved for sale.
		\label{ncert/12/13/2/3/defs.tex}
		\item Two balls are drawn at random with replacement from a box containing 10 black and 8 red balls. Find the probability that
		\label{ncert/12/13/2/12}
\begin{enumerate}
\item both balls are red.
\item first ball is black and second is red.
\item one of them is black and other is red.
\end{enumerate}

\item In a hostel, 60\% of the students read Hindi newspaper, 40\% read English newspaper and 20\% read both Hindi and English newspapers. A student is selected at random.
		\label{ncert/12/13/2/15}
\begin{enumerate}
\item Find the probability that she reads neither Hindi nor English newspapers.
\item If she reads Hindi newspaper, find the probability that she reads English newspaper.
\item If she reads English newspaper, find the probability that she reads Hindi newspaper.\\
\end{enumerate}
\item The probability of obtaining an even prime number on each die, when a pair of dice is rolled is 
\begin{enumerate}
    \item $0$ 
    
    \item $\frac{1}{3}$ 
    
    \item $\frac{1}{12}$ 
    
    \item $\frac{1}{36}$ 
\end{enumerate}
\solution
		%\input{ncert/12/13/2/17/defs.tex}
	\item A bag contains 4 red and 4 black balls, another bag contains 2 red and 6 black balls. One of the two bags is selected at random and a ball is drawn from the bag which is found to be red. Find the probability that the ball is drawn from the first bag.
\\
\solution
		%\input{ncert/12/13/3/2/main.tex}
  \item
  Cards with numbers 2 to 101 are placed in a box. A card is selected at random.Find the probability that the card has
\begin{enumerate}[label=(\roman*)]
	\item an even number 
	\item a square number
\end{enumerate}
\solution
%\input{exemplar/10/13/3/32/main.tex}
\item
The king, queen and jack of clubs are removed from a deck of 52 playing cards and then well shuffled. Now one card is drawn at random from the remaining cards.  Determine the probability that the card is
\begin{enumerate}[label=(\roman*)]
\item a club
\item 10 of hearts
\end{enumerate}
\solution
%\input{exemplar/10/13/3/29/main.tex}
\item A team of medical students doing their internship have to assist during surgeries
at a city hospital. The probabilities of surgeries rated as very complex, complex,
routine, simple or very simple are respectively, 0.15, 0.20, 0.31, 0.26, .08. Find
the probabilities that a particular surgery will be rated
\begin{enumerate}
	\item complex or very complex;
	\item neither very complex nor very simple;
	\item routine or complex
	\item routine or simple
\end{enumerate}
\solution
%\input{exemplar/11/16/3/8(1)/main.tex}
\item A card is selected from a pack of 52 cards.
\begin{enumerate}[label=(\alph*)]
    \item How many points are there in the sample space?
    \item Calculate the probability that the card is an ace of spades.
    \item Calculate the probability that the card is (i) an ace and (ii) black card.
\end{enumerate}
\solution
%\input{exemplar/11/16/3/4/main2.tex}
\item The probability that a non leap year selected at random will contain 53 sundays.
\\
\solution
%\input{exemplar/10/13/1/19/main.tex}
\item One of the four persons John, Rita, Aslam or Gurpreet will be promoted next
month. Consequently the sample space consists of four elementary outcomes
S = {John promoted, Rita promoted, Aslam promoted, Gurpreet promoted}
You are told that the chances of John’s promotion is same as that of Gurpreet,
Rita’s chances of promotion are twice as likely as Johns. Aslam’s chances are
four times that of John.
\begin{enumerate}
	\item Determine
	\begin{enumerate}
		\item P (John promoted)
		\item P (Rita promoted)
		\item P (Aslam promoted)
		\item P (Gurpreet promoted)
	\end{enumerate}
	\item If A = {John promoted or Gurpreet promoted}, find P (A).
\end{enumerate}
\solution
%\input{exemplar/11/16/3/10/main.tex}
\item A card is drawn from a deck of 52 cards. Find the probability of getting a king or a heart or a red card.\\
\solution
%\input{exemplar/11/16/3/15/main.tex}
\item The probability that a student will pass his examination is 0.73, the probability of
the student getting a compartment is 0.13, and the probability that the student will
either pass or get compartment is 0.96. State True or False.\\
\solution
%\input{exemplar/11/16/3/31/main.tex}
\item A card is selected from a pack of 52 cards\\
\begin{enumerate}[label=(\alph*)]
\item How many points are there in the sample space?
\item Calculate the probability that the cards is an ace of spades.
\item Calculate the probability that the card is (i) an ace (ii)black card.\\
\end{enumerate}
%\input{ncert/11/16/3/4_1/Prob_4.tex}
\item In a non-leap year, the probability of having 53 tuesdays or 53 wednesdays is\\
\solution
%\input{exemplar/11/16/3/18/main.tex}
\item There are 1000 sealed envelopes in a box, 10 of them contain a cash prize of
Rs 100 each, 100 of them contain a cash prize of Rs 50 each and 200 of them
contain a cash prize of Rs 10 each and rest do not contain any cash prize. If they
are well shuffled and an envelope is picked up out, what is the probability that it
contains no cash prize?\\
\solution
%\input{exemplar/10/13/3/34/main.tex}
\item 
A die is thrown and a card is selected at random from a deck of 52 playing cards. The probability of getting an even number on the die and a spade card.\\
\solution
%\input{exemplar/12/13/3/78/main.tex}
\item
If 4-digit numbers greater than 5,000 are randomly formed from the digits 0, 1, 3, 5, and 7, what is the probability of forming a number divisible by 5 when:
\begin{enumerate}
    \item The digits are repeated?
    \item The repetition of digits is not allowed?
\end{enumerate}
\solution
%\input{ncert/11/16/4/9/main.tex}
\item Consider the probability space $\brak{\Omega, \mathcal{G}, P}$ where $\Omega = [0,2]$ and $\mathcal{G} = \cbrak{\phi, \Omega, [0,1], (1,2]}$. Let $X$ and $Y$ be two functions on $\Omega$ defined as
\begin{align*}
    X(\omega) = 
    \begin{cases}
        1 & \text{if }\omega \in [0, 1]\\
        2 & \text{if }\omega \in (1, 2]
    \end{cases}
\end{align*}
and
\begin{align*}
    Y(\omega) = 
    \begin{cases}
        2 & \text{if }\omega \in [0, 1.5]\\
        3 & \text{if }\omega \in (1.5, 2].
    \end{cases}
\end{align*}
Then which one of the following statements is true?
\begin{enumerate}
    \item [(A)] $X$ is a random variable with respect to $\mathcal{G}$, but $Y$ is not a random variable with respect to $\mathcal{G}$.
    \item [(B)] $Y$ is a random variable with respect to $\mathcal{G}$, but $X$ is not a random variable with respect to $\mathcal{G}$.
    \item [(C)] Neither $X$ nor $Y$ is a random variable with respect to $\mathcal{G}$.
    \item [(D)] Both $X$ and $Y$ are random variables with respect to $\mathcal{G}$.
\end{enumerate} \hfill (GATE ST 2023)\\
\solution
%\input{gate/ST/2023/14/main.tex}
	\item  A die is loaded in such a way that each odd number is twice as likely to occur as
each even number. Find $P(G)$, where $G$ is the event that a number greater than
3 occurs on a single roll of the die.
\\
\solution
		%\input{exemplar/11/16/3/5/main.tex}
	\item All the jacks, queens and kings are removed from a deck of 52 playing cards. The remaining cards are well shuffled and then one card is drawn at random. Giving ace a value 1 similar value for other cards, find the probability that the card has a value 
		\begin{enumerate}
			\item 7
			\item greater than 7
			\item less than 7
		\end{enumerate}
		%\input{exemplar/10/13/3/30/main.tex}
  \item A Lot consists of 48 mobile phones of which 42 are good, 3 have only minor defects and 3 have major defects.Varnika will buy a phone if it is good but the trader will only buy a mobile if it has no major defects. One phone is selected at random from the lot. What is the probability that it is
\begin{enumerate}
	\item acceptable to Varnika?
            \item acceptable to the trader?
\end{enumerate}
\solution
	%\input{exemplar/10/13/3/40/main.tex}
 \item A student says that if you throw a die, it will show up 1 or not 1. Therefore, the probability of getting 1 and the probability of getting 'not 1' each is equal to $\frac{1}{2}$. Is this correct? Give reasons.\\
 \solution
        %\input{exemplar/10/13/2/9/main.tex}
   \item Four candidates A, B, C, D have ap-
plied for the assignment to coach a school cricket
team. If A is twice as likely to be selected as B, and
B and C are given about the same chance of being
selected, while C is twice as likely to be selected
as D, what are the probabilities that
\begin{enumerate}
\item C will be selected?
\item A will not be selected?
\end{enumerate}
	%\input{exemplar/11/16/3/9/main.tex}
 \item A bag contain 24 balls of which $x$ balls are red, $2x$ are white and $3x$ are blue. A ball is selected at random, What is the probability that it is
\begin{enumerate}[label=\alph*)]
\item not red ?
\item white ?
\end{enumerate}
%\input{exemplar/10/13/3/41/main.tex}
If the letters of the word ASSASSINATION are arranged at random. Find the Probability that
\begin{enumerate}[label=(\alph*)]
\item Four $S's$ come consecutively in the word
\item Two  $I's$ and two $N's$ come together
\item All $A's$ are not coming together
\item No two $A's$ are coming together
\end{enumerate}
%\input{exemplar/11/16/3/14/main.tex}
	\item One urn contains two black balls (labelled B1 and B2) and one white ball. A
	second urn contains one black ball and two white balls (labelled W1 and W2).
	Suppose the following experiment is performed. One of the two urns is chosen
	at random. Next a ball is randomly chosen from the urn. Then a second ball is
	chosen at random from the same urn without replacing the first ball.
	
	\begin{enumerate}
	\item What is the probability that two black balls are chosen?
	
	\item What is the probability that two balls of opposite colour are chosen?
	\end{enumerate}
	\solution
	%\input{exemplar/11/16/3/12/main1.tex}
\end{enumerate}

		\item A box of oranges is inspected by examining three randomly selected oranges drawn without replacement. If all the three oranges are good, the box is approved for sale, otherwise, it is rejected. Find the probability that a box containing 15 oranges out of which 12 are good and 3 are bad ones will be approved for sale.
		\label{ncert/12/13/2/3/defs.tex}
		\item Two balls are drawn at random with replacement from a box containing 10 black and 8 red balls. Find the probability that
		\label{ncert/12/13/2/12}
\begin{enumerate}
\item both balls are red.
\item first ball is black and second is red.
\item one of them is black and other is red.
\end{enumerate}

\item In a hostel, 60\% of the students read Hindi newspaper, 40\% read English newspaper and 20\% read both Hindi and English newspapers. A student is selected at random.
		\label{ncert/12/13/2/15}
\begin{enumerate}
\item Find the probability that she reads neither Hindi nor English newspapers.
\item If she reads Hindi newspaper, find the probability that she reads English newspaper.
\item If she reads English newspaper, find the probability that she reads Hindi newspaper.\\
\end{enumerate}
\item The probability of obtaining an even prime number on each die, when a pair of dice is rolled is 
\begin{enumerate}
    \item $0$ 
    
    \item $\frac{1}{3}$ 
    
    \item $\frac{1}{12}$ 
    
    \item $\frac{1}{36}$ 
\end{enumerate}
\solution
		%\begin{enumerate}[label=\thesection.\arabic*,ref=\thesection.\theenumi]
	\item One card is drawn from a well-shuffled deck of 52 cards. Find the probability of getting
\begin{enumerate}
\item A king of red colour 
\item A face card 
\item A red face card
\item The jack of hearts
\item A spade
\item The queen of diamonds

\end{enumerate}
\solution
		%\input{ncert/10/15/1/14/main.tex}
	\item Five cards—the ten, jack, queen, king and ace of diamonds, are well-shuffled with their face downwards. One card is then picked up at random.
\begin{enumerate}
\item
What is the probability that the card is the queen? 
\item
If the queen is drawn and put aside, what is the probability that the second card picked up is (a) an ace? (b) a queen?\\
\end{enumerate}
\solution
		%\input{ncert/10/15/1/15/defs.tex}
	\item A bag contains $5$ red balls and some blue balls. If the probability of drawing a blue ball is double that if a red ball, determine the number of blue balls in the bag. 
		\\
\solution
		%\input{ncert/10/15/2/3/defs.tex}
	\item A card is selected from a pack of 52 cards.
 \begin{enumerate}[label=(\alph*)] 
                 \item How many points are there in the sample space?
                 \item Calculate the probability that the card is an ace of spades.
                 \item Calculate the probability that the card is (i) an ace and (ii) black card.
 \end{enumerate}
\solution
		%\input{ncert/11/16/3/4/main.tex}
\item Four cards are drawn from a well-shuffled deck of 52 cards. What is the probability of obtaining 3 diamonds and one spade.
\\
\solution
		%\input{ncert/11/16/4/2/defs.tex}
\item In a certain lottery 10,000 tickets are sold and ten equal prizes are awarded. What is the probability of not getting a prize if you buy (a) one ticket (b) two tickets (c) 10 tickets ?	
\\
\solution
		%\input{ncert/11/16/4/4/defs.tex}
		%
\item 
Out of 100 students, two sections of 40 and 60 are formed. If you and your friend are among the 100 students, what is the probability that
\begin{enumerate}
\item you both enter the same section?
\item you both enter the different sections?
\end{enumerate}
\solution
		%\input{ncert/11/16/4/5/defs.tex}
	\item 
The number lock of a suitcase has 4 wheels each labelled with ten digits i.e. from 0 to 9.The lock opens with a sequence of four digits with no repeats.What is the probability of a person getting the right sequence to open the suitcase.
\\
\solution
		%\input{ncert/11/16/4/10/defs.tex}
		%
\item 
Two cards are drawn at random and without replacement from a pack of 52 playing cards. Find the probability that both the cards are black.
\\
\solution
		%\input{ncert/12/13/2/2/defs.tex}
		\item A box of oranges is inspected by examining three randomly selected oranges drawn without replacement. If all the three oranges are good, the box is approved for sale, otherwise, it is rejected. Find the probability that a box containing 15 oranges out of which 12 are good and 3 are bad ones will be approved for sale.
		\label{ncert/12/13/2/3/defs.tex}
		\item Two balls are drawn at random with replacement from a box containing 10 black and 8 red balls. Find the probability that
		\label{ncert/12/13/2/12}
\begin{enumerate}
\item both balls are red.
\item first ball is black and second is red.
\item one of them is black and other is red.
\end{enumerate}

\item In a hostel, 60\% of the students read Hindi newspaper, 40\% read English newspaper and 20\% read both Hindi and English newspapers. A student is selected at random.
		\label{ncert/12/13/2/15}
\begin{enumerate}
\item Find the probability that she reads neither Hindi nor English newspapers.
\item If she reads Hindi newspaper, find the probability that she reads English newspaper.
\item If she reads English newspaper, find the probability that she reads Hindi newspaper.\\
\end{enumerate}
\item The probability of obtaining an even prime number on each die, when a pair of dice is rolled is 
\begin{enumerate}
    \item $0$ 
    
    \item $\frac{1}{3}$ 
    
    \item $\frac{1}{12}$ 
    
    \item $\frac{1}{36}$ 
\end{enumerate}
\solution
		%\input{ncert/12/13/2/17/defs.tex}
	\item A bag contains 4 red and 4 black balls, another bag contains 2 red and 6 black balls. One of the two bags is selected at random and a ball is drawn from the bag which is found to be red. Find the probability that the ball is drawn from the first bag.
\\
\solution
		%\input{ncert/12/13/3/2/main.tex}
  \item
  Cards with numbers 2 to 101 are placed in a box. A card is selected at random.Find the probability that the card has
\begin{enumerate}[label=(\roman*)]
	\item an even number 
	\item a square number
\end{enumerate}
\solution
%\input{exemplar/10/13/3/32/main.tex}
\item
The king, queen and jack of clubs are removed from a deck of 52 playing cards and then well shuffled. Now one card is drawn at random from the remaining cards.  Determine the probability that the card is
\begin{enumerate}[label=(\roman*)]
\item a club
\item 10 of hearts
\end{enumerate}
\solution
%\input{exemplar/10/13/3/29/main.tex}
\item A team of medical students doing their internship have to assist during surgeries
at a city hospital. The probabilities of surgeries rated as very complex, complex,
routine, simple or very simple are respectively, 0.15, 0.20, 0.31, 0.26, .08. Find
the probabilities that a particular surgery will be rated
\begin{enumerate}
	\item complex or very complex;
	\item neither very complex nor very simple;
	\item routine or complex
	\item routine or simple
\end{enumerate}
\solution
%\input{exemplar/11/16/3/8(1)/main.tex}
\item A card is selected from a pack of 52 cards.
\begin{enumerate}[label=(\alph*)]
    \item How many points are there in the sample space?
    \item Calculate the probability that the card is an ace of spades.
    \item Calculate the probability that the card is (i) an ace and (ii) black card.
\end{enumerate}
\solution
%\input{exemplar/11/16/3/4/main2.tex}
\item The probability that a non leap year selected at random will contain 53 sundays.
\\
\solution
%\input{exemplar/10/13/1/19/main.tex}
\item One of the four persons John, Rita, Aslam or Gurpreet will be promoted next
month. Consequently the sample space consists of four elementary outcomes
S = {John promoted, Rita promoted, Aslam promoted, Gurpreet promoted}
You are told that the chances of John’s promotion is same as that of Gurpreet,
Rita’s chances of promotion are twice as likely as Johns. Aslam’s chances are
four times that of John.
\begin{enumerate}
	\item Determine
	\begin{enumerate}
		\item P (John promoted)
		\item P (Rita promoted)
		\item P (Aslam promoted)
		\item P (Gurpreet promoted)
	\end{enumerate}
	\item If A = {John promoted or Gurpreet promoted}, find P (A).
\end{enumerate}
\solution
%\input{exemplar/11/16/3/10/main.tex}
\item A card is drawn from a deck of 52 cards. Find the probability of getting a king or a heart or a red card.\\
\solution
%\input{exemplar/11/16/3/15/main.tex}
\item The probability that a student will pass his examination is 0.73, the probability of
the student getting a compartment is 0.13, and the probability that the student will
either pass or get compartment is 0.96. State True or False.\\
\solution
%\input{exemplar/11/16/3/31/main.tex}
\item A card is selected from a pack of 52 cards\\
\begin{enumerate}[label=(\alph*)]
\item How many points are there in the sample space?
\item Calculate the probability that the cards is an ace of spades.
\item Calculate the probability that the card is (i) an ace (ii)black card.\\
\end{enumerate}
%\input{ncert/11/16/3/4_1/Prob_4.tex}
\item In a non-leap year, the probability of having 53 tuesdays or 53 wednesdays is\\
\solution
%\input{exemplar/11/16/3/18/main.tex}
\item There are 1000 sealed envelopes in a box, 10 of them contain a cash prize of
Rs 100 each, 100 of them contain a cash prize of Rs 50 each and 200 of them
contain a cash prize of Rs 10 each and rest do not contain any cash prize. If they
are well shuffled and an envelope is picked up out, what is the probability that it
contains no cash prize?\\
\solution
%\input{exemplar/10/13/3/34/main.tex}
\item 
A die is thrown and a card is selected at random from a deck of 52 playing cards. The probability of getting an even number on the die and a spade card.\\
\solution
%\input{exemplar/12/13/3/78/main.tex}
\item
If 4-digit numbers greater than 5,000 are randomly formed from the digits 0, 1, 3, 5, and 7, what is the probability of forming a number divisible by 5 when:
\begin{enumerate}
    \item The digits are repeated?
    \item The repetition of digits is not allowed?
\end{enumerate}
\solution
%\input{ncert/11/16/4/9/main.tex}
\item Consider the probability space $\brak{\Omega, \mathcal{G}, P}$ where $\Omega = [0,2]$ and $\mathcal{G} = \cbrak{\phi, \Omega, [0,1], (1,2]}$. Let $X$ and $Y$ be two functions on $\Omega$ defined as
\begin{align*}
    X(\omega) = 
    \begin{cases}
        1 & \text{if }\omega \in [0, 1]\\
        2 & \text{if }\omega \in (1, 2]
    \end{cases}
\end{align*}
and
\begin{align*}
    Y(\omega) = 
    \begin{cases}
        2 & \text{if }\omega \in [0, 1.5]\\
        3 & \text{if }\omega \in (1.5, 2].
    \end{cases}
\end{align*}
Then which one of the following statements is true?
\begin{enumerate}
    \item [(A)] $X$ is a random variable with respect to $\mathcal{G}$, but $Y$ is not a random variable with respect to $\mathcal{G}$.
    \item [(B)] $Y$ is a random variable with respect to $\mathcal{G}$, but $X$ is not a random variable with respect to $\mathcal{G}$.
    \item [(C)] Neither $X$ nor $Y$ is a random variable with respect to $\mathcal{G}$.
    \item [(D)] Both $X$ and $Y$ are random variables with respect to $\mathcal{G}$.
\end{enumerate} \hfill (GATE ST 2023)\\
\solution
%\input{gate/ST/2023/14/main.tex}
	\item  A die is loaded in such a way that each odd number is twice as likely to occur as
each even number. Find $P(G)$, where $G$ is the event that a number greater than
3 occurs on a single roll of the die.
\\
\solution
		%\input{exemplar/11/16/3/5/main.tex}
	\item All the jacks, queens and kings are removed from a deck of 52 playing cards. The remaining cards are well shuffled and then one card is drawn at random. Giving ace a value 1 similar value for other cards, find the probability that the card has a value 
		\begin{enumerate}
			\item 7
			\item greater than 7
			\item less than 7
		\end{enumerate}
		%\input{exemplar/10/13/3/30/main.tex}
  \item A Lot consists of 48 mobile phones of which 42 are good, 3 have only minor defects and 3 have major defects.Varnika will buy a phone if it is good but the trader will only buy a mobile if it has no major defects. One phone is selected at random from the lot. What is the probability that it is
\begin{enumerate}
	\item acceptable to Varnika?
            \item acceptable to the trader?
\end{enumerate}
\solution
	%\input{exemplar/10/13/3/40/main.tex}
 \item A student says that if you throw a die, it will show up 1 or not 1. Therefore, the probability of getting 1 and the probability of getting 'not 1' each is equal to $\frac{1}{2}$. Is this correct? Give reasons.\\
 \solution
        %\input{exemplar/10/13/2/9/main.tex}
   \item Four candidates A, B, C, D have ap-
plied for the assignment to coach a school cricket
team. If A is twice as likely to be selected as B, and
B and C are given about the same chance of being
selected, while C is twice as likely to be selected
as D, what are the probabilities that
\begin{enumerate}
\item C will be selected?
\item A will not be selected?
\end{enumerate}
	%\input{exemplar/11/16/3/9/main.tex}
 \item A bag contain 24 balls of which $x$ balls are red, $2x$ are white and $3x$ are blue. A ball is selected at random, What is the probability that it is
\begin{enumerate}[label=\alph*)]
\item not red ?
\item white ?
\end{enumerate}
%\input{exemplar/10/13/3/41/main.tex}
If the letters of the word ASSASSINATION are arranged at random. Find the Probability that
\begin{enumerate}[label=(\alph*)]
\item Four $S's$ come consecutively in the word
\item Two  $I's$ and two $N's$ come together
\item All $A's$ are not coming together
\item No two $A's$ are coming together
\end{enumerate}
%\input{exemplar/11/16/3/14/main.tex}
	\item One urn contains two black balls (labelled B1 and B2) and one white ball. A
	second urn contains one black ball and two white balls (labelled W1 and W2).
	Suppose the following experiment is performed. One of the two urns is chosen
	at random. Next a ball is randomly chosen from the urn. Then a second ball is
	chosen at random from the same urn without replacing the first ball.
	
	\begin{enumerate}
	\item What is the probability that two black balls are chosen?
	
	\item What is the probability that two balls of opposite colour are chosen?
	\end{enumerate}
	\solution
	%\input{exemplar/11/16/3/12/main1.tex}
\end{enumerate}

	\item A bag contains 4 red and 4 black balls, another bag contains 2 red and 6 black balls. One of the two bags is selected at random and a ball is drawn from the bag which is found to be red. Find the probability that the ball is drawn from the first bag.
\\
\solution
		%\begin{table}[H]
	\centering
\begin{tabular}{|c|c|c|}
\hline
Random variable &Value &Definition\\ \hline
\multirow{3}{*}{X} &0 &Slips of Rs 1\\
&1 &Slips of Rs 5\\
&2 &Slips of Rs 13\\ \hline
\multirow{2}{*}{Y} &0 &Box A\\
&1 &Box B\\\hline
\end{tabular}
\caption{}
\label{tab:Distribution}
\end{table}
See \tabref{tab:Distribution}.
\begin{align}
p_{Y}\brak{k}= \begin{cases} 
      \frac{1}{3} & {k=0} \\
      \frac{2}{3 }& {k=1} 
   \end{cases}
   \\
p_{Y|X}\brak{0|0} = \frac{19}{25}\, 
p_{Y|X}\brak{0|1} = \frac{6}{25}\,
p_{Y|X}\brak{1|0} = \frac{45}{50}\,
p_{Y|X}\brak{1|2} = \frac{5}{50}
\end{align}
The desired probability is the probability that a slip drawn at random is marked other than Rs 1,
\begin{align}
&=1-p_X\brak{0}\\
&= p_X(1) + p_X(2)
\end{align}
Using Bayes theorem,
\begin{align}
&= p_Y\brak{0} \times \pr{Y=0 | X=1} + p_Y\brak{1} \times \pr{Y=1|X=2}\\
&=\frac{1}{3} \times \frac{6}{25} + \frac{2}{3} \times \frac{5}{50}\\
&=\frac{11}{75}
\end{align}

\newpage

%\tableofcontents

\bigskip

\renewcommand{\thefigure}{\theenumi}
\renewcommand{\thetable}{\theenumi}
%\renewcommand{\theequation}{\theenumi}

%\begin{abstract}
%%\boldmath
%In this letter, an algorithm for evaluating the exact analytical bit error rate  (BER)  for the piecewise linear (PL) combiner for  multiple relays is presented. Previous results were available only for upto three relays. The algorithm is unique in the sense that  the actual mathematical expressions, that are prohibitively large, need not be explicitly obtained. The diversity gain due to multiple relays is shown through plots of the analytical BER, well supported by simulations. 
%
%\end{abstract}
% IEEEtran.cls defaults to using nonbold math in the Abstract.
% This preserves the distinction between vectors and scalars. However,
% if the journal you are submitting to favors bold math in the abstract,
% then you can use LaTeX's standard command \boldmath at the very start
% of the abstract to achieve this. Many IEEE journals frown on math
% in the abstract anyway.

% Note that keywords are not normally used for peerreview papers.
%\begin{IEEEkeywords}
%Cooperative diversity, decode and forward, piecewise linear
%\end{IEEEkeywords}



% For peer review papers, you can put extra information on the cover
% page as needed:
% \ifCLASSOPTIONpeerreview
% \begin{center} \bfseries EDICS Category: 3-BBND \end{center}
% \fi
%
% For peerreview papers, this IEEEtran command inserts a page break and
% creates the second title. It will be ignored for other modes.
%\IEEEpeerreviewmaketitle




  \item
  Cards with numbers 2 to 101 are placed in a box. A card is selected at random.Find the probability that the card has
\begin{enumerate}[label=(\roman*)]
	\item an even number 
	\item a square number
\end{enumerate}
\solution
%\begin{table}[H]
	\centering
\begin{tabular}{|c|c|c|}
\hline
Random variable &Value &Definition\\ \hline
\multirow{3}{*}{X} &0 &Slips of Rs 1\\
&1 &Slips of Rs 5\\
&2 &Slips of Rs 13\\ \hline
\multirow{2}{*}{Y} &0 &Box A\\
&1 &Box B\\\hline
\end{tabular}
\caption{}
\label{tab:Distribution}
\end{table}
See \tabref{tab:Distribution}.
\begin{align}
p_{Y}\brak{k}= \begin{cases} 
      \frac{1}{3} & {k=0} \\
      \frac{2}{3 }& {k=1} 
   \end{cases}
   \\
p_{Y|X}\brak{0|0} = \frac{19}{25}\, 
p_{Y|X}\brak{0|1} = \frac{6}{25}\,
p_{Y|X}\brak{1|0} = \frac{45}{50}\,
p_{Y|X}\brak{1|2} = \frac{5}{50}
\end{align}
The desired probability is the probability that a slip drawn at random is marked other than Rs 1,
\begin{align}
&=1-p_X\brak{0}\\
&= p_X(1) + p_X(2)
\end{align}
Using Bayes theorem,
\begin{align}
&= p_Y\brak{0} \times \pr{Y=0 | X=1} + p_Y\brak{1} \times \pr{Y=1|X=2}\\
&=\frac{1}{3} \times \frac{6}{25} + \frac{2}{3} \times \frac{5}{50}\\
&=\frac{11}{75}
\end{align}

\newpage

%\tableofcontents

\bigskip

\renewcommand{\thefigure}{\theenumi}
\renewcommand{\thetable}{\theenumi}
%\renewcommand{\theequation}{\theenumi}

%\begin{abstract}
%%\boldmath
%In this letter, an algorithm for evaluating the exact analytical bit error rate  (BER)  for the piecewise linear (PL) combiner for  multiple relays is presented. Previous results were available only for upto three relays. The algorithm is unique in the sense that  the actual mathematical expressions, that are prohibitively large, need not be explicitly obtained. The diversity gain due to multiple relays is shown through plots of the analytical BER, well supported by simulations. 
%
%\end{abstract}
% IEEEtran.cls defaults to using nonbold math in the Abstract.
% This preserves the distinction between vectors and scalars. However,
% if the journal you are submitting to favors bold math in the abstract,
% then you can use LaTeX's standard command \boldmath at the very start
% of the abstract to achieve this. Many IEEE journals frown on math
% in the abstract anyway.

% Note that keywords are not normally used for peerreview papers.
%\begin{IEEEkeywords}
%Cooperative diversity, decode and forward, piecewise linear
%\end{IEEEkeywords}



% For peer review papers, you can put extra information on the cover
% page as needed:
% \ifCLASSOPTIONpeerreview
% \begin{center} \bfseries EDICS Category: 3-BBND \end{center}
% \fi
%
% For peerreview papers, this IEEEtran command inserts a page break and
% creates the second title. It will be ignored for other modes.
%\IEEEpeerreviewmaketitle




\item
The king, queen and jack of clubs are removed from a deck of 52 playing cards and then well shuffled. Now one card is drawn at random from the remaining cards.  Determine the probability that the card is
\begin{enumerate}[label=(\roman*)]
\item a club
\item 10 of hearts
\end{enumerate}
\solution
%\begin{table}[H]
	\centering
\begin{tabular}{|c|c|c|}
\hline
Random variable &Value &Definition\\ \hline
\multirow{3}{*}{X} &0 &Slips of Rs 1\\
&1 &Slips of Rs 5\\
&2 &Slips of Rs 13\\ \hline
\multirow{2}{*}{Y} &0 &Box A\\
&1 &Box B\\\hline
\end{tabular}
\caption{}
\label{tab:Distribution}
\end{table}
See \tabref{tab:Distribution}.
\begin{align}
p_{Y}\brak{k}= \begin{cases} 
      \frac{1}{3} & {k=0} \\
      \frac{2}{3 }& {k=1} 
   \end{cases}
   \\
p_{Y|X}\brak{0|0} = \frac{19}{25}\, 
p_{Y|X}\brak{0|1} = \frac{6}{25}\,
p_{Y|X}\brak{1|0} = \frac{45}{50}\,
p_{Y|X}\brak{1|2} = \frac{5}{50}
\end{align}
The desired probability is the probability that a slip drawn at random is marked other than Rs 1,
\begin{align}
&=1-p_X\brak{0}\\
&= p_X(1) + p_X(2)
\end{align}
Using Bayes theorem,
\begin{align}
&= p_Y\brak{0} \times \pr{Y=0 | X=1} + p_Y\brak{1} \times \pr{Y=1|X=2}\\
&=\frac{1}{3} \times \frac{6}{25} + \frac{2}{3} \times \frac{5}{50}\\
&=\frac{11}{75}
\end{align}

\newpage

%\tableofcontents

\bigskip

\renewcommand{\thefigure}{\theenumi}
\renewcommand{\thetable}{\theenumi}
%\renewcommand{\theequation}{\theenumi}

%\begin{abstract}
%%\boldmath
%In this letter, an algorithm for evaluating the exact analytical bit error rate  (BER)  for the piecewise linear (PL) combiner for  multiple relays is presented. Previous results were available only for upto three relays. The algorithm is unique in the sense that  the actual mathematical expressions, that are prohibitively large, need not be explicitly obtained. The diversity gain due to multiple relays is shown through plots of the analytical BER, well supported by simulations. 
%
%\end{abstract}
% IEEEtran.cls defaults to using nonbold math in the Abstract.
% This preserves the distinction between vectors and scalars. However,
% if the journal you are submitting to favors bold math in the abstract,
% then you can use LaTeX's standard command \boldmath at the very start
% of the abstract to achieve this. Many IEEE journals frown on math
% in the abstract anyway.

% Note that keywords are not normally used for peerreview papers.
%\begin{IEEEkeywords}
%Cooperative diversity, decode and forward, piecewise linear
%\end{IEEEkeywords}



% For peer review papers, you can put extra information on the cover
% page as needed:
% \ifCLASSOPTIONpeerreview
% \begin{center} \bfseries EDICS Category: 3-BBND \end{center}
% \fi
%
% For peerreview papers, this IEEEtran command inserts a page break and
% creates the second title. It will be ignored for other modes.
%\IEEEpeerreviewmaketitle




\item A team of medical students doing their internship have to assist during surgeries
at a city hospital. The probabilities of surgeries rated as very complex, complex,
routine, simple or very simple are respectively, 0.15, 0.20, 0.31, 0.26, .08. Find
the probabilities that a particular surgery will be rated
\begin{enumerate}
	\item complex or very complex;
	\item neither very complex nor very simple;
	\item routine or complex
	\item routine or simple
\end{enumerate}
\solution
%\begin{table}[H]
	\centering
\begin{tabular}{|c|c|c|}
\hline
Random variable &Value &Definition\\ \hline
\multirow{3}{*}{X} &0 &Slips of Rs 1\\
&1 &Slips of Rs 5\\
&2 &Slips of Rs 13\\ \hline
\multirow{2}{*}{Y} &0 &Box A\\
&1 &Box B\\\hline
\end{tabular}
\caption{}
\label{tab:Distribution}
\end{table}
See \tabref{tab:Distribution}.
\begin{align}
p_{Y}\brak{k}= \begin{cases} 
      \frac{1}{3} & {k=0} \\
      \frac{2}{3 }& {k=1} 
   \end{cases}
   \\
p_{Y|X}\brak{0|0} = \frac{19}{25}\, 
p_{Y|X}\brak{0|1} = \frac{6}{25}\,
p_{Y|X}\brak{1|0} = \frac{45}{50}\,
p_{Y|X}\brak{1|2} = \frac{5}{50}
\end{align}
The desired probability is the probability that a slip drawn at random is marked other than Rs 1,
\begin{align}
&=1-p_X\brak{0}\\
&= p_X(1) + p_X(2)
\end{align}
Using Bayes theorem,
\begin{align}
&= p_Y\brak{0} \times \pr{Y=0 | X=1} + p_Y\brak{1} \times \pr{Y=1|X=2}\\
&=\frac{1}{3} \times \frac{6}{25} + \frac{2}{3} \times \frac{5}{50}\\
&=\frac{11}{75}
\end{align}

\newpage

%\tableofcontents

\bigskip

\renewcommand{\thefigure}{\theenumi}
\renewcommand{\thetable}{\theenumi}
%\renewcommand{\theequation}{\theenumi}

%\begin{abstract}
%%\boldmath
%In this letter, an algorithm for evaluating the exact analytical bit error rate  (BER)  for the piecewise linear (PL) combiner for  multiple relays is presented. Previous results were available only for upto three relays. The algorithm is unique in the sense that  the actual mathematical expressions, that are prohibitively large, need not be explicitly obtained. The diversity gain due to multiple relays is shown through plots of the analytical BER, well supported by simulations. 
%
%\end{abstract}
% IEEEtran.cls defaults to using nonbold math in the Abstract.
% This preserves the distinction between vectors and scalars. However,
% if the journal you are submitting to favors bold math in the abstract,
% then you can use LaTeX's standard command \boldmath at the very start
% of the abstract to achieve this. Many IEEE journals frown on math
% in the abstract anyway.

% Note that keywords are not normally used for peerreview papers.
%\begin{IEEEkeywords}
%Cooperative diversity, decode and forward, piecewise linear
%\end{IEEEkeywords}



% For peer review papers, you can put extra information on the cover
% page as needed:
% \ifCLASSOPTIONpeerreview
% \begin{center} \bfseries EDICS Category: 3-BBND \end{center}
% \fi
%
% For peerreview papers, this IEEEtran command inserts a page break and
% creates the second title. It will be ignored for other modes.
%\IEEEpeerreviewmaketitle




\item A card is selected from a pack of 52 cards.
\begin{enumerate}[label=(\alph*)]
    \item How many points are there in the sample space?
    \item Calculate the probability that the card is an ace of spades.
    \item Calculate the probability that the card is (i) an ace and (ii) black card.
\end{enumerate}
\solution
%Let $X$ be an bernoulli rv defined as in \tabref{tab:exemplar/11/16/3/26}.  Then, 
\begin{equation}
    p =
        \frac{4}{11} 
\end{equation}
\begin{table}[H]
	\centering
	\input{exemplar/11/16/3/26/tables/Table2.tex}
	\caption{}
        \label{tab:exemplar/11/16/3/26}
\end{table}

\item The probability that a non leap year selected at random will contain 53 sundays.
\\
\solution
%\begin{table}[H]
	\centering
\begin{tabular}{|c|c|c|}
\hline
Random variable &Value &Definition\\ \hline
\multirow{3}{*}{X} &0 &Slips of Rs 1\\
&1 &Slips of Rs 5\\
&2 &Slips of Rs 13\\ \hline
\multirow{2}{*}{Y} &0 &Box A\\
&1 &Box B\\\hline
\end{tabular}
\caption{}
\label{tab:Distribution}
\end{table}
See \tabref{tab:Distribution}.
\begin{align}
p_{Y}\brak{k}= \begin{cases} 
      \frac{1}{3} & {k=0} \\
      \frac{2}{3 }& {k=1} 
   \end{cases}
   \\
p_{Y|X}\brak{0|0} = \frac{19}{25}\, 
p_{Y|X}\brak{0|1} = \frac{6}{25}\,
p_{Y|X}\brak{1|0} = \frac{45}{50}\,
p_{Y|X}\brak{1|2} = \frac{5}{50}
\end{align}
The desired probability is the probability that a slip drawn at random is marked other than Rs 1,
\begin{align}
&=1-p_X\brak{0}\\
&= p_X(1) + p_X(2)
\end{align}
Using Bayes theorem,
\begin{align}
&= p_Y\brak{0} \times \pr{Y=0 | X=1} + p_Y\brak{1} \times \pr{Y=1|X=2}\\
&=\frac{1}{3} \times \frac{6}{25} + \frac{2}{3} \times \frac{5}{50}\\
&=\frac{11}{75}
\end{align}

\newpage

%\tableofcontents

\bigskip

\renewcommand{\thefigure}{\theenumi}
\renewcommand{\thetable}{\theenumi}
%\renewcommand{\theequation}{\theenumi}

%\begin{abstract}
%%\boldmath
%In this letter, an algorithm for evaluating the exact analytical bit error rate  (BER)  for the piecewise linear (PL) combiner for  multiple relays is presented. Previous results were available only for upto three relays. The algorithm is unique in the sense that  the actual mathematical expressions, that are prohibitively large, need not be explicitly obtained. The diversity gain due to multiple relays is shown through plots of the analytical BER, well supported by simulations. 
%
%\end{abstract}
% IEEEtran.cls defaults to using nonbold math in the Abstract.
% This preserves the distinction between vectors and scalars. However,
% if the journal you are submitting to favors bold math in the abstract,
% then you can use LaTeX's standard command \boldmath at the very start
% of the abstract to achieve this. Many IEEE journals frown on math
% in the abstract anyway.

% Note that keywords are not normally used for peerreview papers.
%\begin{IEEEkeywords}
%Cooperative diversity, decode and forward, piecewise linear
%\end{IEEEkeywords}



% For peer review papers, you can put extra information on the cover
% page as needed:
% \ifCLASSOPTIONpeerreview
% \begin{center} \bfseries EDICS Category: 3-BBND \end{center}
% \fi
%
% For peerreview papers, this IEEEtran command inserts a page break and
% creates the second title. It will be ignored for other modes.
%\IEEEpeerreviewmaketitle




\item One of the four persons John, Rita, Aslam or Gurpreet will be promoted next
month. Consequently the sample space consists of four elementary outcomes
S = {John promoted, Rita promoted, Aslam promoted, Gurpreet promoted}
You are told that the chances of John’s promotion is same as that of Gurpreet,
Rita’s chances of promotion are twice as likely as Johns. Aslam’s chances are
four times that of John.
\begin{enumerate}
	\item Determine
	\begin{enumerate}
		\item P (John promoted)
		\item P (Rita promoted)
		\item P (Aslam promoted)
		\item P (Gurpreet promoted)
	\end{enumerate}
	\item If A = {John promoted or Gurpreet promoted}, find P (A).
\end{enumerate}
\solution
%\begin{table}[H]
	\centering
\begin{tabular}{|c|c|c|}
\hline
Random variable &Value &Definition\\ \hline
\multirow{3}{*}{X} &0 &Slips of Rs 1\\
&1 &Slips of Rs 5\\
&2 &Slips of Rs 13\\ \hline
\multirow{2}{*}{Y} &0 &Box A\\
&1 &Box B\\\hline
\end{tabular}
\caption{}
\label{tab:Distribution}
\end{table}
See \tabref{tab:Distribution}.
\begin{align}
p_{Y}\brak{k}= \begin{cases} 
      \frac{1}{3} & {k=0} \\
      \frac{2}{3 }& {k=1} 
   \end{cases}
   \\
p_{Y|X}\brak{0|0} = \frac{19}{25}\, 
p_{Y|X}\brak{0|1} = \frac{6}{25}\,
p_{Y|X}\brak{1|0} = \frac{45}{50}\,
p_{Y|X}\brak{1|2} = \frac{5}{50}
\end{align}
The desired probability is the probability that a slip drawn at random is marked other than Rs 1,
\begin{align}
&=1-p_X\brak{0}\\
&= p_X(1) + p_X(2)
\end{align}
Using Bayes theorem,
\begin{align}
&= p_Y\brak{0} \times \pr{Y=0 | X=1} + p_Y\brak{1} \times \pr{Y=1|X=2}\\
&=\frac{1}{3} \times \frac{6}{25} + \frac{2}{3} \times \frac{5}{50}\\
&=\frac{11}{75}
\end{align}

\newpage

%\tableofcontents

\bigskip

\renewcommand{\thefigure}{\theenumi}
\renewcommand{\thetable}{\theenumi}
%\renewcommand{\theequation}{\theenumi}

%\begin{abstract}
%%\boldmath
%In this letter, an algorithm for evaluating the exact analytical bit error rate  (BER)  for the piecewise linear (PL) combiner for  multiple relays is presented. Previous results were available only for upto three relays. The algorithm is unique in the sense that  the actual mathematical expressions, that are prohibitively large, need not be explicitly obtained. The diversity gain due to multiple relays is shown through plots of the analytical BER, well supported by simulations. 
%
%\end{abstract}
% IEEEtran.cls defaults to using nonbold math in the Abstract.
% This preserves the distinction between vectors and scalars. However,
% if the journal you are submitting to favors bold math in the abstract,
% then you can use LaTeX's standard command \boldmath at the very start
% of the abstract to achieve this. Many IEEE journals frown on math
% in the abstract anyway.

% Note that keywords are not normally used for peerreview papers.
%\begin{IEEEkeywords}
%Cooperative diversity, decode and forward, piecewise linear
%\end{IEEEkeywords}



% For peer review papers, you can put extra information on the cover
% page as needed:
% \ifCLASSOPTIONpeerreview
% \begin{center} \bfseries EDICS Category: 3-BBND \end{center}
% \fi
%
% For peerreview papers, this IEEEtran command inserts a page break and
% creates the second title. It will be ignored for other modes.
%\IEEEpeerreviewmaketitle




\item A card is drawn from a deck of 52 cards. Find the probability of getting a king or a heart or a red card.\\
\solution
%\begin{table}[H]
	\centering
\begin{tabular}{|c|c|c|}
\hline
Random variable &Value &Definition\\ \hline
\multirow{3}{*}{X} &0 &Slips of Rs 1\\
&1 &Slips of Rs 5\\
&2 &Slips of Rs 13\\ \hline
\multirow{2}{*}{Y} &0 &Box A\\
&1 &Box B\\\hline
\end{tabular}
\caption{}
\label{tab:Distribution}
\end{table}
See \tabref{tab:Distribution}.
\begin{align}
p_{Y}\brak{k}= \begin{cases} 
      \frac{1}{3} & {k=0} \\
      \frac{2}{3 }& {k=1} 
   \end{cases}
   \\
p_{Y|X}\brak{0|0} = \frac{19}{25}\, 
p_{Y|X}\brak{0|1} = \frac{6}{25}\,
p_{Y|X}\brak{1|0} = \frac{45}{50}\,
p_{Y|X}\brak{1|2} = \frac{5}{50}
\end{align}
The desired probability is the probability that a slip drawn at random is marked other than Rs 1,
\begin{align}
&=1-p_X\brak{0}\\
&= p_X(1) + p_X(2)
\end{align}
Using Bayes theorem,
\begin{align}
&= p_Y\brak{0} \times \pr{Y=0 | X=1} + p_Y\brak{1} \times \pr{Y=1|X=2}\\
&=\frac{1}{3} \times \frac{6}{25} + \frac{2}{3} \times \frac{5}{50}\\
&=\frac{11}{75}
\end{align}

\newpage

%\tableofcontents

\bigskip

\renewcommand{\thefigure}{\theenumi}
\renewcommand{\thetable}{\theenumi}
%\renewcommand{\theequation}{\theenumi}

%\begin{abstract}
%%\boldmath
%In this letter, an algorithm for evaluating the exact analytical bit error rate  (BER)  for the piecewise linear (PL) combiner for  multiple relays is presented. Previous results were available only for upto three relays. The algorithm is unique in the sense that  the actual mathematical expressions, that are prohibitively large, need not be explicitly obtained. The diversity gain due to multiple relays is shown through plots of the analytical BER, well supported by simulations. 
%
%\end{abstract}
% IEEEtran.cls defaults to using nonbold math in the Abstract.
% This preserves the distinction between vectors and scalars. However,
% if the journal you are submitting to favors bold math in the abstract,
% then you can use LaTeX's standard command \boldmath at the very start
% of the abstract to achieve this. Many IEEE journals frown on math
% in the abstract anyway.

% Note that keywords are not normally used for peerreview papers.
%\begin{IEEEkeywords}
%Cooperative diversity, decode and forward, piecewise linear
%\end{IEEEkeywords}



% For peer review papers, you can put extra information on the cover
% page as needed:
% \ifCLASSOPTIONpeerreview
% \begin{center} \bfseries EDICS Category: 3-BBND \end{center}
% \fi
%
% For peerreview papers, this IEEEtran command inserts a page break and
% creates the second title. It will be ignored for other modes.
%\IEEEpeerreviewmaketitle




\item The probability that a student will pass his examination is 0.73, the probability of
the student getting a compartment is 0.13, and the probability that the student will
either pass or get compartment is 0.96. State True or False.\\
\solution
%\begin{table}[H]
	\centering
\begin{tabular}{|c|c|c|}
\hline
Random variable &Value &Definition\\ \hline
\multirow{3}{*}{X} &0 &Slips of Rs 1\\
&1 &Slips of Rs 5\\
&2 &Slips of Rs 13\\ \hline
\multirow{2}{*}{Y} &0 &Box A\\
&1 &Box B\\\hline
\end{tabular}
\caption{}
\label{tab:Distribution}
\end{table}
See \tabref{tab:Distribution}.
\begin{align}
p_{Y}\brak{k}= \begin{cases} 
      \frac{1}{3} & {k=0} \\
      \frac{2}{3 }& {k=1} 
   \end{cases}
   \\
p_{Y|X}\brak{0|0} = \frac{19}{25}\, 
p_{Y|X}\brak{0|1} = \frac{6}{25}\,
p_{Y|X}\brak{1|0} = \frac{45}{50}\,
p_{Y|X}\brak{1|2} = \frac{5}{50}
\end{align}
The desired probability is the probability that a slip drawn at random is marked other than Rs 1,
\begin{align}
&=1-p_X\brak{0}\\
&= p_X(1) + p_X(2)
\end{align}
Using Bayes theorem,
\begin{align}
&= p_Y\brak{0} \times \pr{Y=0 | X=1} + p_Y\brak{1} \times \pr{Y=1|X=2}\\
&=\frac{1}{3} \times \frac{6}{25} + \frac{2}{3} \times \frac{5}{50}\\
&=\frac{11}{75}
\end{align}

\newpage

%\tableofcontents

\bigskip

\renewcommand{\thefigure}{\theenumi}
\renewcommand{\thetable}{\theenumi}
%\renewcommand{\theequation}{\theenumi}

%\begin{abstract}
%%\boldmath
%In this letter, an algorithm for evaluating the exact analytical bit error rate  (BER)  for the piecewise linear (PL) combiner for  multiple relays is presented. Previous results were available only for upto three relays. The algorithm is unique in the sense that  the actual mathematical expressions, that are prohibitively large, need not be explicitly obtained. The diversity gain due to multiple relays is shown through plots of the analytical BER, well supported by simulations. 
%
%\end{abstract}
% IEEEtran.cls defaults to using nonbold math in the Abstract.
% This preserves the distinction between vectors and scalars. However,
% if the journal you are submitting to favors bold math in the abstract,
% then you can use LaTeX's standard command \boldmath at the very start
% of the abstract to achieve this. Many IEEE journals frown on math
% in the abstract anyway.

% Note that keywords are not normally used for peerreview papers.
%\begin{IEEEkeywords}
%Cooperative diversity, decode and forward, piecewise linear
%\end{IEEEkeywords}



% For peer review papers, you can put extra information on the cover
% page as needed:
% \ifCLASSOPTIONpeerreview
% \begin{center} \bfseries EDICS Category: 3-BBND \end{center}
% \fi
%
% For peerreview papers, this IEEEtran command inserts a page break and
% creates the second title. It will be ignored for other modes.
%\IEEEpeerreviewmaketitle




\item A card is selected from a pack of 52 cards\\
\begin{enumerate}[label=(\alph*)]
\item How many points are there in the sample space?
\item Calculate the probability that the cards is an ace of spades.
\item Calculate the probability that the card is (i) an ace (ii)black card.\\
\end{enumerate}
%\input{ncert/11/16/3/4_1/Prob_4.tex}
\item In a non-leap year, the probability of having 53 tuesdays or 53 wednesdays is\\
\solution
%A non-leap year has a total of 365 days, and a week has 7 days.\\
So it can be expressed as 
\begin{align}
365\text{days} &=52\times 7+1 \text{day}
\end{align}
$\implies$ 52 tuesdays or wednesdays\\
Random variable X denotes the days of a week
\begin{align}
p_X\brak{k}&=\frac{1}{7}; \quad \brak{1<k<7}
\end{align}
So the probability of extra day being tuesday or wednesday is
\begin{align}
p_X\brak{3}+p_X\brak{4}&=\frac{1}{7}+\frac{1}{7}=\frac{2}{7}
\end{align}



\item There are 1000 sealed envelopes in a box, 10 of them contain a cash prize of
Rs 100 each, 100 of them contain a cash prize of Rs 50 each and 200 of them
contain a cash prize of Rs 10 each and rest do not contain any cash prize. If they
are well shuffled and an envelope is picked up out, what is the probability that it
contains no cash prize?\\
\solution
%\begin{table}[H]
	\centering
\begin{tabular}{|c|c|c|}
\hline
Random variable &Value &Definition\\ \hline
\multirow{3}{*}{X} &0 &Slips of Rs 1\\
&1 &Slips of Rs 5\\
&2 &Slips of Rs 13\\ \hline
\multirow{2}{*}{Y} &0 &Box A\\
&1 &Box B\\\hline
\end{tabular}
\caption{}
\label{tab:Distribution}
\end{table}
See \tabref{tab:Distribution}.
\begin{align}
p_{Y}\brak{k}= \begin{cases} 
      \frac{1}{3} & {k=0} \\
      \frac{2}{3 }& {k=1} 
   \end{cases}
   \\
p_{Y|X}\brak{0|0} = \frac{19}{25}\, 
p_{Y|X}\brak{0|1} = \frac{6}{25}\,
p_{Y|X}\brak{1|0} = \frac{45}{50}\,
p_{Y|X}\brak{1|2} = \frac{5}{50}
\end{align}
The desired probability is the probability that a slip drawn at random is marked other than Rs 1,
\begin{align}
&=1-p_X\brak{0}\\
&= p_X(1) + p_X(2)
\end{align}
Using Bayes theorem,
\begin{align}
&= p_Y\brak{0} \times \pr{Y=0 | X=1} + p_Y\brak{1} \times \pr{Y=1|X=2}\\
&=\frac{1}{3} \times \frac{6}{25} + \frac{2}{3} \times \frac{5}{50}\\
&=\frac{11}{75}
\end{align}

\newpage

%\tableofcontents

\bigskip

\renewcommand{\thefigure}{\theenumi}
\renewcommand{\thetable}{\theenumi}
%\renewcommand{\theequation}{\theenumi}

%\begin{abstract}
%%\boldmath
%In this letter, an algorithm for evaluating the exact analytical bit error rate  (BER)  for the piecewise linear (PL) combiner for  multiple relays is presented. Previous results were available only for upto three relays. The algorithm is unique in the sense that  the actual mathematical expressions, that are prohibitively large, need not be explicitly obtained. The diversity gain due to multiple relays is shown through plots of the analytical BER, well supported by simulations. 
%
%\end{abstract}
% IEEEtran.cls defaults to using nonbold math in the Abstract.
% This preserves the distinction between vectors and scalars. However,
% if the journal you are submitting to favors bold math in the abstract,
% then you can use LaTeX's standard command \boldmath at the very start
% of the abstract to achieve this. Many IEEE journals frown on math
% in the abstract anyway.

% Note that keywords are not normally used for peerreview papers.
%\begin{IEEEkeywords}
%Cooperative diversity, decode and forward, piecewise linear
%\end{IEEEkeywords}



% For peer review papers, you can put extra information on the cover
% page as needed:
% \ifCLASSOPTIONpeerreview
% \begin{center} \bfseries EDICS Category: 3-BBND \end{center}
% \fi
%
% For peerreview papers, this IEEEtran command inserts a page break and
% creates the second title. It will be ignored for other modes.
%\IEEEpeerreviewmaketitle




\item 
A die is thrown and a card is selected at random from a deck of 52 playing cards. The probability of getting an even number on the die and a spade card.\\
\solution
%\begin{table}[H]
	\centering
\begin{tabular}{|c|c|c|}
\hline
Random variable &Value &Definition\\ \hline
\multirow{3}{*}{X} &0 &Slips of Rs 1\\
&1 &Slips of Rs 5\\
&2 &Slips of Rs 13\\ \hline
\multirow{2}{*}{Y} &0 &Box A\\
&1 &Box B\\\hline
\end{tabular}
\caption{}
\label{tab:Distribution}
\end{table}
See \tabref{tab:Distribution}.
\begin{align}
p_{Y}\brak{k}= \begin{cases} 
      \frac{1}{3} & {k=0} \\
      \frac{2}{3 }& {k=1} 
   \end{cases}
   \\
p_{Y|X}\brak{0|0} = \frac{19}{25}\, 
p_{Y|X}\brak{0|1} = \frac{6}{25}\,
p_{Y|X}\brak{1|0} = \frac{45}{50}\,
p_{Y|X}\brak{1|2} = \frac{5}{50}
\end{align}
The desired probability is the probability that a slip drawn at random is marked other than Rs 1,
\begin{align}
&=1-p_X\brak{0}\\
&= p_X(1) + p_X(2)
\end{align}
Using Bayes theorem,
\begin{align}
&= p_Y\brak{0} \times \pr{Y=0 | X=1} + p_Y\brak{1} \times \pr{Y=1|X=2}\\
&=\frac{1}{3} \times \frac{6}{25} + \frac{2}{3} \times \frac{5}{50}\\
&=\frac{11}{75}
\end{align}

\newpage

%\tableofcontents

\bigskip

\renewcommand{\thefigure}{\theenumi}
\renewcommand{\thetable}{\theenumi}
%\renewcommand{\theequation}{\theenumi}

%\begin{abstract}
%%\boldmath
%In this letter, an algorithm for evaluating the exact analytical bit error rate  (BER)  for the piecewise linear (PL) combiner for  multiple relays is presented. Previous results were available only for upto three relays. The algorithm is unique in the sense that  the actual mathematical expressions, that are prohibitively large, need not be explicitly obtained. The diversity gain due to multiple relays is shown through plots of the analytical BER, well supported by simulations. 
%
%\end{abstract}
% IEEEtran.cls defaults to using nonbold math in the Abstract.
% This preserves the distinction between vectors and scalars. However,
% if the journal you are submitting to favors bold math in the abstract,
% then you can use LaTeX's standard command \boldmath at the very start
% of the abstract to achieve this. Many IEEE journals frown on math
% in the abstract anyway.

% Note that keywords are not normally used for peerreview papers.
%\begin{IEEEkeywords}
%Cooperative diversity, decode and forward, piecewise linear
%\end{IEEEkeywords}



% For peer review papers, you can put extra information on the cover
% page as needed:
% \ifCLASSOPTIONpeerreview
% \begin{center} \bfseries EDICS Category: 3-BBND \end{center}
% \fi
%
% For peerreview papers, this IEEEtran command inserts a page break and
% creates the second title. It will be ignored for other modes.
%\IEEEpeerreviewmaketitle




\item
If 4-digit numbers greater than 5,000 are randomly formed from the digits 0, 1, 3, 5, and 7, what is the probability of forming a number divisible by 5 when:
\begin{enumerate}
    \item The digits are repeated?
    \item The repetition of digits is not allowed?
\end{enumerate}
\solution
%\begin{table}[H]
	\centering
\begin{tabular}{|c|c|c|}
\hline
Random variable &Value &Definition\\ \hline
\multirow{3}{*}{X} &0 &Slips of Rs 1\\
&1 &Slips of Rs 5\\
&2 &Slips of Rs 13\\ \hline
\multirow{2}{*}{Y} &0 &Box A\\
&1 &Box B\\\hline
\end{tabular}
\caption{}
\label{tab:Distribution}
\end{table}
See \tabref{tab:Distribution}.
\begin{align}
p_{Y}\brak{k}= \begin{cases} 
      \frac{1}{3} & {k=0} \\
      \frac{2}{3 }& {k=1} 
   \end{cases}
   \\
p_{Y|X}\brak{0|0} = \frac{19}{25}\, 
p_{Y|X}\brak{0|1} = \frac{6}{25}\,
p_{Y|X}\brak{1|0} = \frac{45}{50}\,
p_{Y|X}\brak{1|2} = \frac{5}{50}
\end{align}
The desired probability is the probability that a slip drawn at random is marked other than Rs 1,
\begin{align}
&=1-p_X\brak{0}\\
&= p_X(1) + p_X(2)
\end{align}
Using Bayes theorem,
\begin{align}
&= p_Y\brak{0} \times \pr{Y=0 | X=1} + p_Y\brak{1} \times \pr{Y=1|X=2}\\
&=\frac{1}{3} \times \frac{6}{25} + \frac{2}{3} \times \frac{5}{50}\\
&=\frac{11}{75}
\end{align}

\newpage

%\tableofcontents

\bigskip

\renewcommand{\thefigure}{\theenumi}
\renewcommand{\thetable}{\theenumi}
%\renewcommand{\theequation}{\theenumi}

%\begin{abstract}
%%\boldmath
%In this letter, an algorithm for evaluating the exact analytical bit error rate  (BER)  for the piecewise linear (PL) combiner for  multiple relays is presented. Previous results were available only for upto three relays. The algorithm is unique in the sense that  the actual mathematical expressions, that are prohibitively large, need not be explicitly obtained. The diversity gain due to multiple relays is shown through plots of the analytical BER, well supported by simulations. 
%
%\end{abstract}
% IEEEtran.cls defaults to using nonbold math in the Abstract.
% This preserves the distinction between vectors and scalars. However,
% if the journal you are submitting to favors bold math in the abstract,
% then you can use LaTeX's standard command \boldmath at the very start
% of the abstract to achieve this. Many IEEE journals frown on math
% in the abstract anyway.

% Note that keywords are not normally used for peerreview papers.
%\begin{IEEEkeywords}
%Cooperative diversity, decode and forward, piecewise linear
%\end{IEEEkeywords}



% For peer review papers, you can put extra information on the cover
% page as needed:
% \ifCLASSOPTIONpeerreview
% \begin{center} \bfseries EDICS Category: 3-BBND \end{center}
% \fi
%
% For peerreview papers, this IEEEtran command inserts a page break and
% creates the second title. It will be ignored for other modes.
%\IEEEpeerreviewmaketitle




\item Consider the probability space $\brak{\Omega, \mathcal{G}, P}$ where $\Omega = [0,2]$ and $\mathcal{G} = \cbrak{\phi, \Omega, [0,1], (1,2]}$. Let $X$ and $Y$ be two functions on $\Omega$ defined as
\begin{align*}
    X(\omega) = 
    \begin{cases}
        1 & \text{if }\omega \in [0, 1]\\
        2 & \text{if }\omega \in (1, 2]
    \end{cases}
\end{align*}
and
\begin{align*}
    Y(\omega) = 
    \begin{cases}
        2 & \text{if }\omega \in [0, 1.5]\\
        3 & \text{if }\omega \in (1.5, 2].
    \end{cases}
\end{align*}
Then which one of the following statements is true?
\begin{enumerate}
    \item [(A)] $X$ is a random variable with respect to $\mathcal{G}$, but $Y$ is not a random variable with respect to $\mathcal{G}$.
    \item [(B)] $Y$ is a random variable with respect to $\mathcal{G}$, but $X$ is not a random variable with respect to $\mathcal{G}$.
    \item [(C)] Neither $X$ nor $Y$ is a random variable with respect to $\mathcal{G}$.
    \item [(D)] Both $X$ and $Y$ are random variables with respect to $\mathcal{G}$.
\end{enumerate} \hfill (GATE ST 2023)\\
\solution
%\begin{table}[H]
	\centering
\begin{tabular}{|c|c|c|}
\hline
Random variable &Value &Definition\\ \hline
\multirow{3}{*}{X} &0 &Slips of Rs 1\\
&1 &Slips of Rs 5\\
&2 &Slips of Rs 13\\ \hline
\multirow{2}{*}{Y} &0 &Box A\\
&1 &Box B\\\hline
\end{tabular}
\caption{}
\label{tab:Distribution}
\end{table}
See \tabref{tab:Distribution}.
\begin{align}
p_{Y}\brak{k}= \begin{cases} 
      \frac{1}{3} & {k=0} \\
      \frac{2}{3 }& {k=1} 
   \end{cases}
   \\
p_{Y|X}\brak{0|0} = \frac{19}{25}\, 
p_{Y|X}\brak{0|1} = \frac{6}{25}\,
p_{Y|X}\brak{1|0} = \frac{45}{50}\,
p_{Y|X}\brak{1|2} = \frac{5}{50}
\end{align}
The desired probability is the probability that a slip drawn at random is marked other than Rs 1,
\begin{align}
&=1-p_X\brak{0}\\
&= p_X(1) + p_X(2)
\end{align}
Using Bayes theorem,
\begin{align}
&= p_Y\brak{0} \times \pr{Y=0 | X=1} + p_Y\brak{1} \times \pr{Y=1|X=2}\\
&=\frac{1}{3} \times \frac{6}{25} + \frac{2}{3} \times \frac{5}{50}\\
&=\frac{11}{75}
\end{align}

\newpage

%\tableofcontents

\bigskip

\renewcommand{\thefigure}{\theenumi}
\renewcommand{\thetable}{\theenumi}
%\renewcommand{\theequation}{\theenumi}

%\begin{abstract}
%%\boldmath
%In this letter, an algorithm for evaluating the exact analytical bit error rate  (BER)  for the piecewise linear (PL) combiner for  multiple relays is presented. Previous results were available only for upto three relays. The algorithm is unique in the sense that  the actual mathematical expressions, that are prohibitively large, need not be explicitly obtained. The diversity gain due to multiple relays is shown through plots of the analytical BER, well supported by simulations. 
%
%\end{abstract}
% IEEEtran.cls defaults to using nonbold math in the Abstract.
% This preserves the distinction between vectors and scalars. However,
% if the journal you are submitting to favors bold math in the abstract,
% then you can use LaTeX's standard command \boldmath at the very start
% of the abstract to achieve this. Many IEEE journals frown on math
% in the abstract anyway.

% Note that keywords are not normally used for peerreview papers.
%\begin{IEEEkeywords}
%Cooperative diversity, decode and forward, piecewise linear
%\end{IEEEkeywords}



% For peer review papers, you can put extra information on the cover
% page as needed:
% \ifCLASSOPTIONpeerreview
% \begin{center} \bfseries EDICS Category: 3-BBND \end{center}
% \fi
%
% For peerreview papers, this IEEEtran command inserts a page break and
% creates the second title. It will be ignored for other modes.
%\IEEEpeerreviewmaketitle




	\item  A die is loaded in such a way that each odd number is twice as likely to occur as
each even number. Find $P(G)$, where $G$ is the event that a number greater than
3 occurs on a single roll of the die.
\\
\solution
		%\begin{table}[H]
	\centering
\begin{tabular}{|c|c|c|}
\hline
Random variable &Value &Definition\\ \hline
\multirow{3}{*}{X} &0 &Slips of Rs 1\\
&1 &Slips of Rs 5\\
&2 &Slips of Rs 13\\ \hline
\multirow{2}{*}{Y} &0 &Box A\\
&1 &Box B\\\hline
\end{tabular}
\caption{}
\label{tab:Distribution}
\end{table}
See \tabref{tab:Distribution}.
\begin{align}
p_{Y}\brak{k}= \begin{cases} 
      \frac{1}{3} & {k=0} \\
      \frac{2}{3 }& {k=1} 
   \end{cases}
   \\
p_{Y|X}\brak{0|0} = \frac{19}{25}\, 
p_{Y|X}\brak{0|1} = \frac{6}{25}\,
p_{Y|X}\brak{1|0} = \frac{45}{50}\,
p_{Y|X}\brak{1|2} = \frac{5}{50}
\end{align}
The desired probability is the probability that a slip drawn at random is marked other than Rs 1,
\begin{align}
&=1-p_X\brak{0}\\
&= p_X(1) + p_X(2)
\end{align}
Using Bayes theorem,
\begin{align}
&= p_Y\brak{0} \times \pr{Y=0 | X=1} + p_Y\brak{1} \times \pr{Y=1|X=2}\\
&=\frac{1}{3} \times \frac{6}{25} + \frac{2}{3} \times \frac{5}{50}\\
&=\frac{11}{75}
\end{align}

\newpage

%\tableofcontents

\bigskip

\renewcommand{\thefigure}{\theenumi}
\renewcommand{\thetable}{\theenumi}
%\renewcommand{\theequation}{\theenumi}

%\begin{abstract}
%%\boldmath
%In this letter, an algorithm for evaluating the exact analytical bit error rate  (BER)  for the piecewise linear (PL) combiner for  multiple relays is presented. Previous results were available only for upto three relays. The algorithm is unique in the sense that  the actual mathematical expressions, that are prohibitively large, need not be explicitly obtained. The diversity gain due to multiple relays is shown through plots of the analytical BER, well supported by simulations. 
%
%\end{abstract}
% IEEEtran.cls defaults to using nonbold math in the Abstract.
% This preserves the distinction between vectors and scalars. However,
% if the journal you are submitting to favors bold math in the abstract,
% then you can use LaTeX's standard command \boldmath at the very start
% of the abstract to achieve this. Many IEEE journals frown on math
% in the abstract anyway.

% Note that keywords are not normally used for peerreview papers.
%\begin{IEEEkeywords}
%Cooperative diversity, decode and forward, piecewise linear
%\end{IEEEkeywords}



% For peer review papers, you can put extra information on the cover
% page as needed:
% \ifCLASSOPTIONpeerreview
% \begin{center} \bfseries EDICS Category: 3-BBND \end{center}
% \fi
%
% For peerreview papers, this IEEEtran command inserts a page break and
% creates the second title. It will be ignored for other modes.
%\IEEEpeerreviewmaketitle




	\item All the jacks, queens and kings are removed from a deck of 52 playing cards. The remaining cards are well shuffled and then one card is drawn at random. Giving ace a value 1 similar value for other cards, find the probability that the card has a value 
		\begin{enumerate}
			\item 7
			\item greater than 7
			\item less than 7
		\end{enumerate}
		%Number of cards left after removing all jacks, queens and kings 
\begin{align}
N	= 52 - 4\times 3
	= 40
\end{align}
%\begin{table}[H]
%\def\arraystretch{1.2}
%\begin{tabular}{|c|c|c|}
%\hline
%	\textbf{Parameter} &\textbf{Value} &\textbf{Description}\\ \hline
%	$X$ &1-10 &Represents the value of the card picked \\ \hline
%\end{tabular}
%\end{table}
Let $1 \le X \le 10$ be the value of the card picked.  Then,
\begin{align}
	p_X(k) &= \Pr(X=k)\ \forall\ 1 \leq k \leq 10\\
	&= \frac{4\times 1}{40}\\
	&= \frac{1}{10}\\
	\therefore p_X(k) &= 
	\begin{cases}
		\frac{1}{10} & 1 \leq k \leq 10\\
		0 & \text{otherwise}
	\end{cases}
\end{align}
and
\begin{align}
	F_{X}(k) &= \sum_{m=0}^{k}p_{X}(m) \quad 1 \leq k \leq 10\\
	&= \frac{k}{10}\\
	\therefore F_{X}(k) &= 
	\begin{cases}
		0 & k \leq 0\\
		\frac{k}{10} & 1\leq k \leq 10\\
		1 & k > 10 
	\end{cases}
\end{align}
\begin{enumerate}
	\item Probability that card has value equal to 7 is
		\begin{align}
			 p_{X}(7)
			= \frac{1}{10}
		\end{align}
	\item Probability that card has value greater than 7 is
		\begin{align}
			1 - F_X(7)
			&= 1 - \frac{7}{10}
			\\
			&= \frac{3}{10}
		\end{align}
	\item Probability that card has value less than 7 is
		\begin{align}
			 F_{X}(6)
			=\frac{6}{10}
		\end{align}
\end{enumerate}

  \item A Lot consists of 48 mobile phones of which 42 are good, 3 have only minor defects and 3 have major defects.Varnika will buy a phone if it is good but the trader will only buy a mobile if it has no major defects. One phone is selected at random from the lot. What is the probability that it is
\begin{enumerate}
	\item acceptable to Varnika?
            \item acceptable to the trader?
\end{enumerate}
\solution
	%\begin{table}[H]
	\centering
\begin{tabular}{|c|c|c|}
\hline
Random variable &Value &Definition\\ \hline
\multirow{3}{*}{X} &0 &Slips of Rs 1\\
&1 &Slips of Rs 5\\
&2 &Slips of Rs 13\\ \hline
\multirow{2}{*}{Y} &0 &Box A\\
&1 &Box B\\\hline
\end{tabular}
\caption{}
\label{tab:Distribution}
\end{table}
See \tabref{tab:Distribution}.
\begin{align}
p_{Y}\brak{k}= \begin{cases} 
      \frac{1}{3} & {k=0} \\
      \frac{2}{3 }& {k=1} 
   \end{cases}
   \\
p_{Y|X}\brak{0|0} = \frac{19}{25}\, 
p_{Y|X}\brak{0|1} = \frac{6}{25}\,
p_{Y|X}\brak{1|0} = \frac{45}{50}\,
p_{Y|X}\brak{1|2} = \frac{5}{50}
\end{align}
The desired probability is the probability that a slip drawn at random is marked other than Rs 1,
\begin{align}
&=1-p_X\brak{0}\\
&= p_X(1) + p_X(2)
\end{align}
Using Bayes theorem,
\begin{align}
&= p_Y\brak{0} \times \pr{Y=0 | X=1} + p_Y\brak{1} \times \pr{Y=1|X=2}\\
&=\frac{1}{3} \times \frac{6}{25} + \frac{2}{3} \times \frac{5}{50}\\
&=\frac{11}{75}
\end{align}

\newpage

%\tableofcontents

\bigskip

\renewcommand{\thefigure}{\theenumi}
\renewcommand{\thetable}{\theenumi}
%\renewcommand{\theequation}{\theenumi}

%\begin{abstract}
%%\boldmath
%In this letter, an algorithm for evaluating the exact analytical bit error rate  (BER)  for the piecewise linear (PL) combiner for  multiple relays is presented. Previous results were available only for upto three relays. The algorithm is unique in the sense that  the actual mathematical expressions, that are prohibitively large, need not be explicitly obtained. The diversity gain due to multiple relays is shown through plots of the analytical BER, well supported by simulations. 
%
%\end{abstract}
% IEEEtran.cls defaults to using nonbold math in the Abstract.
% This preserves the distinction between vectors and scalars. However,
% if the journal you are submitting to favors bold math in the abstract,
% then you can use LaTeX's standard command \boldmath at the very start
% of the abstract to achieve this. Many IEEE journals frown on math
% in the abstract anyway.

% Note that keywords are not normally used for peerreview papers.
%\begin{IEEEkeywords}
%Cooperative diversity, decode and forward, piecewise linear
%\end{IEEEkeywords}



% For peer review papers, you can put extra information on the cover
% page as needed:
% \ifCLASSOPTIONpeerreview
% \begin{center} \bfseries EDICS Category: 3-BBND \end{center}
% \fi
%
% For peerreview papers, this IEEEtran command inserts a page break and
% creates the second title. It will be ignored for other modes.
%\IEEEpeerreviewmaketitle




 \item A student says that if you throw a die, it will show up 1 or not 1. Therefore, the probability of getting 1 and the probability of getting 'not 1' each is equal to $\frac{1}{2}$. Is this correct? Give reasons.\\
 \solution
        %\begin{table}[H]
	\centering
\begin{tabular}{|c|c|c|}
\hline
Random variable &Value &Definition\\ \hline
\multirow{3}{*}{X} &0 &Slips of Rs 1\\
&1 &Slips of Rs 5\\
&2 &Slips of Rs 13\\ \hline
\multirow{2}{*}{Y} &0 &Box A\\
&1 &Box B\\\hline
\end{tabular}
\caption{}
\label{tab:Distribution}
\end{table}
See \tabref{tab:Distribution}.
\begin{align}
p_{Y}\brak{k}= \begin{cases} 
      \frac{1}{3} & {k=0} \\
      \frac{2}{3 }& {k=1} 
   \end{cases}
   \\
p_{Y|X}\brak{0|0} = \frac{19}{25}\, 
p_{Y|X}\brak{0|1} = \frac{6}{25}\,
p_{Y|X}\brak{1|0} = \frac{45}{50}\,
p_{Y|X}\brak{1|2} = \frac{5}{50}
\end{align}
The desired probability is the probability that a slip drawn at random is marked other than Rs 1,
\begin{align}
&=1-p_X\brak{0}\\
&= p_X(1) + p_X(2)
\end{align}
Using Bayes theorem,
\begin{align}
&= p_Y\brak{0} \times \pr{Y=0 | X=1} + p_Y\brak{1} \times \pr{Y=1|X=2}\\
&=\frac{1}{3} \times \frac{6}{25} + \frac{2}{3} \times \frac{5}{50}\\
&=\frac{11}{75}
\end{align}

\newpage

%\tableofcontents

\bigskip

\renewcommand{\thefigure}{\theenumi}
\renewcommand{\thetable}{\theenumi}
%\renewcommand{\theequation}{\theenumi}

%\begin{abstract}
%%\boldmath
%In this letter, an algorithm for evaluating the exact analytical bit error rate  (BER)  for the piecewise linear (PL) combiner for  multiple relays is presented. Previous results were available only for upto three relays. The algorithm is unique in the sense that  the actual mathematical expressions, that are prohibitively large, need not be explicitly obtained. The diversity gain due to multiple relays is shown through plots of the analytical BER, well supported by simulations. 
%
%\end{abstract}
% IEEEtran.cls defaults to using nonbold math in the Abstract.
% This preserves the distinction between vectors and scalars. However,
% if the journal you are submitting to favors bold math in the abstract,
% then you can use LaTeX's standard command \boldmath at the very start
% of the abstract to achieve this. Many IEEE journals frown on math
% in the abstract anyway.

% Note that keywords are not normally used for peerreview papers.
%\begin{IEEEkeywords}
%Cooperative diversity, decode and forward, piecewise linear
%\end{IEEEkeywords}



% For peer review papers, you can put extra information on the cover
% page as needed:
% \ifCLASSOPTIONpeerreview
% \begin{center} \bfseries EDICS Category: 3-BBND \end{center}
% \fi
%
% For peerreview papers, this IEEEtran command inserts a page break and
% creates the second title. It will be ignored for other modes.
%\IEEEpeerreviewmaketitle




   \item Four candidates A, B, C, D have ap-
plied for the assignment to coach a school cricket
team. If A is twice as likely to be selected as B, and
B and C are given about the same chance of being
selected, while C is twice as likely to be selected
as D, what are the probabilities that
\begin{enumerate}
\item C will be selected?
\item A will not be selected?
\end{enumerate}
	%\begin{table}[H]
	\centering
\begin{tabular}{|c|c|c|}
\hline
Random variable &Value &Definition\\ \hline
\multirow{3}{*}{X} &0 &Slips of Rs 1\\
&1 &Slips of Rs 5\\
&2 &Slips of Rs 13\\ \hline
\multirow{2}{*}{Y} &0 &Box A\\
&1 &Box B\\\hline
\end{tabular}
\caption{}
\label{tab:Distribution}
\end{table}
See \tabref{tab:Distribution}.
\begin{align}
p_{Y}\brak{k}= \begin{cases} 
      \frac{1}{3} & {k=0} \\
      \frac{2}{3 }& {k=1} 
   \end{cases}
   \\
p_{Y|X}\brak{0|0} = \frac{19}{25}\, 
p_{Y|X}\brak{0|1} = \frac{6}{25}\,
p_{Y|X}\brak{1|0} = \frac{45}{50}\,
p_{Y|X}\brak{1|2} = \frac{5}{50}
\end{align}
The desired probability is the probability that a slip drawn at random is marked other than Rs 1,
\begin{align}
&=1-p_X\brak{0}\\
&= p_X(1) + p_X(2)
\end{align}
Using Bayes theorem,
\begin{align}
&= p_Y\brak{0} \times \pr{Y=0 | X=1} + p_Y\brak{1} \times \pr{Y=1|X=2}\\
&=\frac{1}{3} \times \frac{6}{25} + \frac{2}{3} \times \frac{5}{50}\\
&=\frac{11}{75}
\end{align}

\newpage

%\tableofcontents

\bigskip

\renewcommand{\thefigure}{\theenumi}
\renewcommand{\thetable}{\theenumi}
%\renewcommand{\theequation}{\theenumi}

%\begin{abstract}
%%\boldmath
%In this letter, an algorithm for evaluating the exact analytical bit error rate  (BER)  for the piecewise linear (PL) combiner for  multiple relays is presented. Previous results were available only for upto three relays. The algorithm is unique in the sense that  the actual mathematical expressions, that are prohibitively large, need not be explicitly obtained. The diversity gain due to multiple relays is shown through plots of the analytical BER, well supported by simulations. 
%
%\end{abstract}
% IEEEtran.cls defaults to using nonbold math in the Abstract.
% This preserves the distinction between vectors and scalars. However,
% if the journal you are submitting to favors bold math in the abstract,
% then you can use LaTeX's standard command \boldmath at the very start
% of the abstract to achieve this. Many IEEE journals frown on math
% in the abstract anyway.

% Note that keywords are not normally used for peerreview papers.
%\begin{IEEEkeywords}
%Cooperative diversity, decode and forward, piecewise linear
%\end{IEEEkeywords}



% For peer review papers, you can put extra information on the cover
% page as needed:
% \ifCLASSOPTIONpeerreview
% \begin{center} \bfseries EDICS Category: 3-BBND \end{center}
% \fi
%
% For peerreview papers, this IEEEtran command inserts a page break and
% creates the second title. It will be ignored for other modes.
%\IEEEpeerreviewmaketitle




 \item A bag contain 24 balls of which $x$ balls are red, $2x$ are white and $3x$ are blue. A ball is selected at random, What is the probability that it is
\begin{enumerate}[label=\alph*)]
\item not red ?
\item white ?
\end{enumerate}
%\begin{table}[H]
	\centering
\begin{tabular}{|c|c|c|}
\hline
Random variable &Value &Definition\\ \hline
\multirow{3}{*}{X} &0 &Slips of Rs 1\\
&1 &Slips of Rs 5\\
&2 &Slips of Rs 13\\ \hline
\multirow{2}{*}{Y} &0 &Box A\\
&1 &Box B\\\hline
\end{tabular}
\caption{}
\label{tab:Distribution}
\end{table}
See \tabref{tab:Distribution}.
\begin{align}
p_{Y}\brak{k}= \begin{cases} 
      \frac{1}{3} & {k=0} \\
      \frac{2}{3 }& {k=1} 
   \end{cases}
   \\
p_{Y|X}\brak{0|0} = \frac{19}{25}\, 
p_{Y|X}\brak{0|1} = \frac{6}{25}\,
p_{Y|X}\brak{1|0} = \frac{45}{50}\,
p_{Y|X}\brak{1|2} = \frac{5}{50}
\end{align}
The desired probability is the probability that a slip drawn at random is marked other than Rs 1,
\begin{align}
&=1-p_X\brak{0}\\
&= p_X(1) + p_X(2)
\end{align}
Using Bayes theorem,
\begin{align}
&= p_Y\brak{0} \times \pr{Y=0 | X=1} + p_Y\brak{1} \times \pr{Y=1|X=2}\\
&=\frac{1}{3} \times \frac{6}{25} + \frac{2}{3} \times \frac{5}{50}\\
&=\frac{11}{75}
\end{align}

\newpage

%\tableofcontents

\bigskip

\renewcommand{\thefigure}{\theenumi}
\renewcommand{\thetable}{\theenumi}
%\renewcommand{\theequation}{\theenumi}

%\begin{abstract}
%%\boldmath
%In this letter, an algorithm for evaluating the exact analytical bit error rate  (BER)  for the piecewise linear (PL) combiner for  multiple relays is presented. Previous results were available only for upto three relays. The algorithm is unique in the sense that  the actual mathematical expressions, that are prohibitively large, need not be explicitly obtained. The diversity gain due to multiple relays is shown through plots of the analytical BER, well supported by simulations. 
%
%\end{abstract}
% IEEEtran.cls defaults to using nonbold math in the Abstract.
% This preserves the distinction between vectors and scalars. However,
% if the journal you are submitting to favors bold math in the abstract,
% then you can use LaTeX's standard command \boldmath at the very start
% of the abstract to achieve this. Many IEEE journals frown on math
% in the abstract anyway.

% Note that keywords are not normally used for peerreview papers.
%\begin{IEEEkeywords}
%Cooperative diversity, decode and forward, piecewise linear
%\end{IEEEkeywords}



% For peer review papers, you can put extra information on the cover
% page as needed:
% \ifCLASSOPTIONpeerreview
% \begin{center} \bfseries EDICS Category: 3-BBND \end{center}
% \fi
%
% For peerreview papers, this IEEEtran command inserts a page break and
% creates the second title. It will be ignored for other modes.
%\IEEEpeerreviewmaketitle




If the letters of the word ASSASSINATION are arranged at random. Find the Probability that
\begin{enumerate}[label=(\alph*)]
\item Four $S's$ come consecutively in the word
\item Two  $I's$ and two $N's$ come together
\item All $A's$ are not coming together
\item No two $A's$ are coming together
\end{enumerate}
%\begin{table}[H]
	\centering
\begin{tabular}{|c|c|c|}
\hline
Random variable &Value &Definition\\ \hline
\multirow{3}{*}{X} &0 &Slips of Rs 1\\
&1 &Slips of Rs 5\\
&2 &Slips of Rs 13\\ \hline
\multirow{2}{*}{Y} &0 &Box A\\
&1 &Box B\\\hline
\end{tabular}
\caption{}
\label{tab:Distribution}
\end{table}
See \tabref{tab:Distribution}.
\begin{align}
p_{Y}\brak{k}= \begin{cases} 
      \frac{1}{3} & {k=0} \\
      \frac{2}{3 }& {k=1} 
   \end{cases}
   \\
p_{Y|X}\brak{0|0} = \frac{19}{25}\, 
p_{Y|X}\brak{0|1} = \frac{6}{25}\,
p_{Y|X}\brak{1|0} = \frac{45}{50}\,
p_{Y|X}\brak{1|2} = \frac{5}{50}
\end{align}
The desired probability is the probability that a slip drawn at random is marked other than Rs 1,
\begin{align}
&=1-p_X\brak{0}\\
&= p_X(1) + p_X(2)
\end{align}
Using Bayes theorem,
\begin{align}
&= p_Y\brak{0} \times \pr{Y=0 | X=1} + p_Y\brak{1} \times \pr{Y=1|X=2}\\
&=\frac{1}{3} \times \frac{6}{25} + \frac{2}{3} \times \frac{5}{50}\\
&=\frac{11}{75}
\end{align}

\newpage

%\tableofcontents

\bigskip

\renewcommand{\thefigure}{\theenumi}
\renewcommand{\thetable}{\theenumi}
%\renewcommand{\theequation}{\theenumi}

%\begin{abstract}
%%\boldmath
%In this letter, an algorithm for evaluating the exact analytical bit error rate  (BER)  for the piecewise linear (PL) combiner for  multiple relays is presented. Previous results were available only for upto three relays. The algorithm is unique in the sense that  the actual mathematical expressions, that are prohibitively large, need not be explicitly obtained. The diversity gain due to multiple relays is shown through plots of the analytical BER, well supported by simulations. 
%
%\end{abstract}
% IEEEtran.cls defaults to using nonbold math in the Abstract.
% This preserves the distinction between vectors and scalars. However,
% if the journal you are submitting to favors bold math in the abstract,
% then you can use LaTeX's standard command \boldmath at the very start
% of the abstract to achieve this. Many IEEE journals frown on math
% in the abstract anyway.

% Note that keywords are not normally used for peerreview papers.
%\begin{IEEEkeywords}
%Cooperative diversity, decode and forward, piecewise linear
%\end{IEEEkeywords}



% For peer review papers, you can put extra information on the cover
% page as needed:
% \ifCLASSOPTIONpeerreview
% \begin{center} \bfseries EDICS Category: 3-BBND \end{center}
% \fi
%
% For peerreview papers, this IEEEtran command inserts a page break and
% creates the second title. It will be ignored for other modes.
%\IEEEpeerreviewmaketitle




	\item One urn contains two black balls (labelled B1 and B2) and one white ball. A
	second urn contains one black ball and two white balls (labelled W1 and W2).
	Suppose the following experiment is performed. One of the two urns is chosen
	at random. Next a ball is randomly chosen from the urn. Then a second ball is
	chosen at random from the same urn without replacing the first ball.
	
	\begin{enumerate}
	\item What is the probability that two black balls are chosen?
	
	\item What is the probability that two balls of opposite colour are chosen?
	\end{enumerate}
	\solution
	%\begin{align}
    \label{eq:12.13.6.18.1}
	\because	\pr{A|B} &> \pr{A},\
\frac{\pr{AB}}{\pr{B}} > \pr{A}
\\
    \label{eq:12.13.6.18.2}
	\implies \pr{AB} &> \pr{A}\pr{B}
	\\
	\text{or, } \frac{\pr{AB}}{\pr{A}} &=\pr{B|A} > \pr{A}
\end{align}

\end{enumerate}

\item In a certain lottery 10,000 tickets are sold and ten equal prizes are awarded. What is the probability of not getting a prize if you buy (a) one ticket (b) two tickets (c) 10 tickets ?	
\\
\solution
		%\begin{enumerate}[label=\thesection.\arabic*,ref=\thesection.\theenumi]
	\item One card is drawn from a well-shuffled deck of 52 cards. Find the probability of getting
\begin{enumerate}
\item A king of red colour 
\item A face card 
\item A red face card
\item The jack of hearts
\item A spade
\item The queen of diamonds

\end{enumerate}
\solution
		%\begin{table}[H]
	\centering
\begin{tabular}{|c|c|c|}
\hline
Random variable &Value &Definition\\ \hline
\multirow{3}{*}{X} &0 &Slips of Rs 1\\
&1 &Slips of Rs 5\\
&2 &Slips of Rs 13\\ \hline
\multirow{2}{*}{Y} &0 &Box A\\
&1 &Box B\\\hline
\end{tabular}
\caption{}
\label{tab:Distribution}
\end{table}
See \tabref{tab:Distribution}.
\begin{align}
p_{Y}\brak{k}= \begin{cases} 
      \frac{1}{3} & {k=0} \\
      \frac{2}{3 }& {k=1} 
   \end{cases}
   \\
p_{Y|X}\brak{0|0} = \frac{19}{25}\, 
p_{Y|X}\brak{0|1} = \frac{6}{25}\,
p_{Y|X}\brak{1|0} = \frac{45}{50}\,
p_{Y|X}\brak{1|2} = \frac{5}{50}
\end{align}
The desired probability is the probability that a slip drawn at random is marked other than Rs 1,
\begin{align}
&=1-p_X\brak{0}\\
&= p_X(1) + p_X(2)
\end{align}
Using Bayes theorem,
\begin{align}
&= p_Y\brak{0} \times \pr{Y=0 | X=1} + p_Y\brak{1} \times \pr{Y=1|X=2}\\
&=\frac{1}{3} \times \frac{6}{25} + \frac{2}{3} \times \frac{5}{50}\\
&=\frac{11}{75}
\end{align}

\newpage

%\tableofcontents

\bigskip

\renewcommand{\thefigure}{\theenumi}
\renewcommand{\thetable}{\theenumi}
%\renewcommand{\theequation}{\theenumi}

%\begin{abstract}
%%\boldmath
%In this letter, an algorithm for evaluating the exact analytical bit error rate  (BER)  for the piecewise linear (PL) combiner for  multiple relays is presented. Previous results were available only for upto three relays. The algorithm is unique in the sense that  the actual mathematical expressions, that are prohibitively large, need not be explicitly obtained. The diversity gain due to multiple relays is shown through plots of the analytical BER, well supported by simulations. 
%
%\end{abstract}
% IEEEtran.cls defaults to using nonbold math in the Abstract.
% This preserves the distinction between vectors and scalars. However,
% if the journal you are submitting to favors bold math in the abstract,
% then you can use LaTeX's standard command \boldmath at the very start
% of the abstract to achieve this. Many IEEE journals frown on math
% in the abstract anyway.

% Note that keywords are not normally used for peerreview papers.
%\begin{IEEEkeywords}
%Cooperative diversity, decode and forward, piecewise linear
%\end{IEEEkeywords}



% For peer review papers, you can put extra information on the cover
% page as needed:
% \ifCLASSOPTIONpeerreview
% \begin{center} \bfseries EDICS Category: 3-BBND \end{center}
% \fi
%
% For peerreview papers, this IEEEtran command inserts a page break and
% creates the second title. It will be ignored for other modes.
%\IEEEpeerreviewmaketitle




	\item Five cards—the ten, jack, queen, king and ace of diamonds, are well-shuffled with their face downwards. One card is then picked up at random.
\begin{enumerate}
\item
What is the probability that the card is the queen? 
\item
If the queen is drawn and put aside, what is the probability that the second card picked up is (a) an ace? (b) a queen?\\
\end{enumerate}
\solution
		%\begin{enumerate}[label=\thesection.\arabic*,ref=\thesection.\theenumi]
	\item One card is drawn from a well-shuffled deck of 52 cards. Find the probability of getting
\begin{enumerate}
\item A king of red colour 
\item A face card 
\item A red face card
\item The jack of hearts
\item A spade
\item The queen of diamonds

\end{enumerate}
\solution
		%\input{ncert/10/15/1/14/main.tex}
	\item Five cards—the ten, jack, queen, king and ace of diamonds, are well-shuffled with their face downwards. One card is then picked up at random.
\begin{enumerate}
\item
What is the probability that the card is the queen? 
\item
If the queen is drawn and put aside, what is the probability that the second card picked up is (a) an ace? (b) a queen?\\
\end{enumerate}
\solution
		%\input{ncert/10/15/1/15/defs.tex}
	\item A bag contains $5$ red balls and some blue balls. If the probability of drawing a blue ball is double that if a red ball, determine the number of blue balls in the bag. 
		\\
\solution
		%\input{ncert/10/15/2/3/defs.tex}
	\item A card is selected from a pack of 52 cards.
 \begin{enumerate}[label=(\alph*)] 
                 \item How many points are there in the sample space?
                 \item Calculate the probability that the card is an ace of spades.
                 \item Calculate the probability that the card is (i) an ace and (ii) black card.
 \end{enumerate}
\solution
		%\input{ncert/11/16/3/4/main.tex}
\item Four cards are drawn from a well-shuffled deck of 52 cards. What is the probability of obtaining 3 diamonds and one spade.
\\
\solution
		%\input{ncert/11/16/4/2/defs.tex}
\item In a certain lottery 10,000 tickets are sold and ten equal prizes are awarded. What is the probability of not getting a prize if you buy (a) one ticket (b) two tickets (c) 10 tickets ?	
\\
\solution
		%\input{ncert/11/16/4/4/defs.tex}
		%
\item 
Out of 100 students, two sections of 40 and 60 are formed. If you and your friend are among the 100 students, what is the probability that
\begin{enumerate}
\item you both enter the same section?
\item you both enter the different sections?
\end{enumerate}
\solution
		%\input{ncert/11/16/4/5/defs.tex}
	\item 
The number lock of a suitcase has 4 wheels each labelled with ten digits i.e. from 0 to 9.The lock opens with a sequence of four digits with no repeats.What is the probability of a person getting the right sequence to open the suitcase.
\\
\solution
		%\input{ncert/11/16/4/10/defs.tex}
		%
\item 
Two cards are drawn at random and without replacement from a pack of 52 playing cards. Find the probability that both the cards are black.
\\
\solution
		%\input{ncert/12/13/2/2/defs.tex}
		\item A box of oranges is inspected by examining three randomly selected oranges drawn without replacement. If all the three oranges are good, the box is approved for sale, otherwise, it is rejected. Find the probability that a box containing 15 oranges out of which 12 are good and 3 are bad ones will be approved for sale.
		\label{ncert/12/13/2/3/defs.tex}
		\item Two balls are drawn at random with replacement from a box containing 10 black and 8 red balls. Find the probability that
		\label{ncert/12/13/2/12}
\begin{enumerate}
\item both balls are red.
\item first ball is black and second is red.
\item one of them is black and other is red.
\end{enumerate}

\item In a hostel, 60\% of the students read Hindi newspaper, 40\% read English newspaper and 20\% read both Hindi and English newspapers. A student is selected at random.
		\label{ncert/12/13/2/15}
\begin{enumerate}
\item Find the probability that she reads neither Hindi nor English newspapers.
\item If she reads Hindi newspaper, find the probability that she reads English newspaper.
\item If she reads English newspaper, find the probability that she reads Hindi newspaper.\\
\end{enumerate}
\item The probability of obtaining an even prime number on each die, when a pair of dice is rolled is 
\begin{enumerate}
    \item $0$ 
    
    \item $\frac{1}{3}$ 
    
    \item $\frac{1}{12}$ 
    
    \item $\frac{1}{36}$ 
\end{enumerate}
\solution
		%\input{ncert/12/13/2/17/defs.tex}
	\item A bag contains 4 red and 4 black balls, another bag contains 2 red and 6 black balls. One of the two bags is selected at random and a ball is drawn from the bag which is found to be red. Find the probability that the ball is drawn from the first bag.
\\
\solution
		%\input{ncert/12/13/3/2/main.tex}
  \item
  Cards with numbers 2 to 101 are placed in a box. A card is selected at random.Find the probability that the card has
\begin{enumerate}[label=(\roman*)]
	\item an even number 
	\item a square number
\end{enumerate}
\solution
%\input{exemplar/10/13/3/32/main.tex}
\item
The king, queen and jack of clubs are removed from a deck of 52 playing cards and then well shuffled. Now one card is drawn at random from the remaining cards.  Determine the probability that the card is
\begin{enumerate}[label=(\roman*)]
\item a club
\item 10 of hearts
\end{enumerate}
\solution
%\input{exemplar/10/13/3/29/main.tex}
\item A team of medical students doing their internship have to assist during surgeries
at a city hospital. The probabilities of surgeries rated as very complex, complex,
routine, simple or very simple are respectively, 0.15, 0.20, 0.31, 0.26, .08. Find
the probabilities that a particular surgery will be rated
\begin{enumerate}
	\item complex or very complex;
	\item neither very complex nor very simple;
	\item routine or complex
	\item routine or simple
\end{enumerate}
\solution
%\input{exemplar/11/16/3/8(1)/main.tex}
\item A card is selected from a pack of 52 cards.
\begin{enumerate}[label=(\alph*)]
    \item How many points are there in the sample space?
    \item Calculate the probability that the card is an ace of spades.
    \item Calculate the probability that the card is (i) an ace and (ii) black card.
\end{enumerate}
\solution
%\input{exemplar/11/16/3/4/main2.tex}
\item The probability that a non leap year selected at random will contain 53 sundays.
\\
\solution
%\input{exemplar/10/13/1/19/main.tex}
\item One of the four persons John, Rita, Aslam or Gurpreet will be promoted next
month. Consequently the sample space consists of four elementary outcomes
S = {John promoted, Rita promoted, Aslam promoted, Gurpreet promoted}
You are told that the chances of John’s promotion is same as that of Gurpreet,
Rita’s chances of promotion are twice as likely as Johns. Aslam’s chances are
four times that of John.
\begin{enumerate}
	\item Determine
	\begin{enumerate}
		\item P (John promoted)
		\item P (Rita promoted)
		\item P (Aslam promoted)
		\item P (Gurpreet promoted)
	\end{enumerate}
	\item If A = {John promoted or Gurpreet promoted}, find P (A).
\end{enumerate}
\solution
%\input{exemplar/11/16/3/10/main.tex}
\item A card is drawn from a deck of 52 cards. Find the probability of getting a king or a heart or a red card.\\
\solution
%\input{exemplar/11/16/3/15/main.tex}
\item The probability that a student will pass his examination is 0.73, the probability of
the student getting a compartment is 0.13, and the probability that the student will
either pass or get compartment is 0.96. State True or False.\\
\solution
%\input{exemplar/11/16/3/31/main.tex}
\item A card is selected from a pack of 52 cards\\
\begin{enumerate}[label=(\alph*)]
\item How many points are there in the sample space?
\item Calculate the probability that the cards is an ace of spades.
\item Calculate the probability that the card is (i) an ace (ii)black card.\\
\end{enumerate}
%\input{ncert/11/16/3/4_1/Prob_4.tex}
\item In a non-leap year, the probability of having 53 tuesdays or 53 wednesdays is\\
\solution
%\input{exemplar/11/16/3/18/main.tex}
\item There are 1000 sealed envelopes in a box, 10 of them contain a cash prize of
Rs 100 each, 100 of them contain a cash prize of Rs 50 each and 200 of them
contain a cash prize of Rs 10 each and rest do not contain any cash prize. If they
are well shuffled and an envelope is picked up out, what is the probability that it
contains no cash prize?\\
\solution
%\input{exemplar/10/13/3/34/main.tex}
\item 
A die is thrown and a card is selected at random from a deck of 52 playing cards. The probability of getting an even number on the die and a spade card.\\
\solution
%\input{exemplar/12/13/3/78/main.tex}
\item
If 4-digit numbers greater than 5,000 are randomly formed from the digits 0, 1, 3, 5, and 7, what is the probability of forming a number divisible by 5 when:
\begin{enumerate}
    \item The digits are repeated?
    \item The repetition of digits is not allowed?
\end{enumerate}
\solution
%\input{ncert/11/16/4/9/main.tex}
\item Consider the probability space $\brak{\Omega, \mathcal{G}, P}$ where $\Omega = [0,2]$ and $\mathcal{G} = \cbrak{\phi, \Omega, [0,1], (1,2]}$. Let $X$ and $Y$ be two functions on $\Omega$ defined as
\begin{align*}
    X(\omega) = 
    \begin{cases}
        1 & \text{if }\omega \in [0, 1]\\
        2 & \text{if }\omega \in (1, 2]
    \end{cases}
\end{align*}
and
\begin{align*}
    Y(\omega) = 
    \begin{cases}
        2 & \text{if }\omega \in [0, 1.5]\\
        3 & \text{if }\omega \in (1.5, 2].
    \end{cases}
\end{align*}
Then which one of the following statements is true?
\begin{enumerate}
    \item [(A)] $X$ is a random variable with respect to $\mathcal{G}$, but $Y$ is not a random variable with respect to $\mathcal{G}$.
    \item [(B)] $Y$ is a random variable with respect to $\mathcal{G}$, but $X$ is not a random variable with respect to $\mathcal{G}$.
    \item [(C)] Neither $X$ nor $Y$ is a random variable with respect to $\mathcal{G}$.
    \item [(D)] Both $X$ and $Y$ are random variables with respect to $\mathcal{G}$.
\end{enumerate} \hfill (GATE ST 2023)\\
\solution
%\input{gate/ST/2023/14/main.tex}
	\item  A die is loaded in such a way that each odd number is twice as likely to occur as
each even number. Find $P(G)$, where $G$ is the event that a number greater than
3 occurs on a single roll of the die.
\\
\solution
		%\input{exemplar/11/16/3/5/main.tex}
	\item All the jacks, queens and kings are removed from a deck of 52 playing cards. The remaining cards are well shuffled and then one card is drawn at random. Giving ace a value 1 similar value for other cards, find the probability that the card has a value 
		\begin{enumerate}
			\item 7
			\item greater than 7
			\item less than 7
		\end{enumerate}
		%\input{exemplar/10/13/3/30/main.tex}
  \item A Lot consists of 48 mobile phones of which 42 are good, 3 have only minor defects and 3 have major defects.Varnika will buy a phone if it is good but the trader will only buy a mobile if it has no major defects. One phone is selected at random from the lot. What is the probability that it is
\begin{enumerate}
	\item acceptable to Varnika?
            \item acceptable to the trader?
\end{enumerate}
\solution
	%\input{exemplar/10/13/3/40/main.tex}
 \item A student says that if you throw a die, it will show up 1 or not 1. Therefore, the probability of getting 1 and the probability of getting 'not 1' each is equal to $\frac{1}{2}$. Is this correct? Give reasons.\\
 \solution
        %\input{exemplar/10/13/2/9/main.tex}
   \item Four candidates A, B, C, D have ap-
plied for the assignment to coach a school cricket
team. If A is twice as likely to be selected as B, and
B and C are given about the same chance of being
selected, while C is twice as likely to be selected
as D, what are the probabilities that
\begin{enumerate}
\item C will be selected?
\item A will not be selected?
\end{enumerate}
	%\input{exemplar/11/16/3/9/main.tex}
 \item A bag contain 24 balls of which $x$ balls are red, $2x$ are white and $3x$ are blue. A ball is selected at random, What is the probability that it is
\begin{enumerate}[label=\alph*)]
\item not red ?
\item white ?
\end{enumerate}
%\input{exemplar/10/13/3/41/main.tex}
If the letters of the word ASSASSINATION are arranged at random. Find the Probability that
\begin{enumerate}[label=(\alph*)]
\item Four $S's$ come consecutively in the word
\item Two  $I's$ and two $N's$ come together
\item All $A's$ are not coming together
\item No two $A's$ are coming together
\end{enumerate}
%\input{exemplar/11/16/3/14/main.tex}
	\item One urn contains two black balls (labelled B1 and B2) and one white ball. A
	second urn contains one black ball and two white balls (labelled W1 and W2).
	Suppose the following experiment is performed. One of the two urns is chosen
	at random. Next a ball is randomly chosen from the urn. Then a second ball is
	chosen at random from the same urn without replacing the first ball.
	
	\begin{enumerate}
	\item What is the probability that two black balls are chosen?
	
	\item What is the probability that two balls of opposite colour are chosen?
	\end{enumerate}
	\solution
	%\input{exemplar/11/16/3/12/main1.tex}
\end{enumerate}

	\item A bag contains $5$ red balls and some blue balls. If the probability of drawing a blue ball is double that if a red ball, determine the number of blue balls in the bag. 
		\\
\solution
		%\begin{enumerate}[label=\thesection.\arabic*,ref=\thesection.\theenumi]
	\item One card is drawn from a well-shuffled deck of 52 cards. Find the probability of getting
\begin{enumerate}
\item A king of red colour 
\item A face card 
\item A red face card
\item The jack of hearts
\item A spade
\item The queen of diamonds

\end{enumerate}
\solution
		%\input{ncert/10/15/1/14/main.tex}
	\item Five cards—the ten, jack, queen, king and ace of diamonds, are well-shuffled with their face downwards. One card is then picked up at random.
\begin{enumerate}
\item
What is the probability that the card is the queen? 
\item
If the queen is drawn and put aside, what is the probability that the second card picked up is (a) an ace? (b) a queen?\\
\end{enumerate}
\solution
		%\input{ncert/10/15/1/15/defs.tex}
	\item A bag contains $5$ red balls and some blue balls. If the probability of drawing a blue ball is double that if a red ball, determine the number of blue balls in the bag. 
		\\
\solution
		%\input{ncert/10/15/2/3/defs.tex}
	\item A card is selected from a pack of 52 cards.
 \begin{enumerate}[label=(\alph*)] 
                 \item How many points are there in the sample space?
                 \item Calculate the probability that the card is an ace of spades.
                 \item Calculate the probability that the card is (i) an ace and (ii) black card.
 \end{enumerate}
\solution
		%\input{ncert/11/16/3/4/main.tex}
\item Four cards are drawn from a well-shuffled deck of 52 cards. What is the probability of obtaining 3 diamonds and one spade.
\\
\solution
		%\input{ncert/11/16/4/2/defs.tex}
\item In a certain lottery 10,000 tickets are sold and ten equal prizes are awarded. What is the probability of not getting a prize if you buy (a) one ticket (b) two tickets (c) 10 tickets ?	
\\
\solution
		%\input{ncert/11/16/4/4/defs.tex}
		%
\item 
Out of 100 students, two sections of 40 and 60 are formed. If you and your friend are among the 100 students, what is the probability that
\begin{enumerate}
\item you both enter the same section?
\item you both enter the different sections?
\end{enumerate}
\solution
		%\input{ncert/11/16/4/5/defs.tex}
	\item 
The number lock of a suitcase has 4 wheels each labelled with ten digits i.e. from 0 to 9.The lock opens with a sequence of four digits with no repeats.What is the probability of a person getting the right sequence to open the suitcase.
\\
\solution
		%\input{ncert/11/16/4/10/defs.tex}
		%
\item 
Two cards are drawn at random and without replacement from a pack of 52 playing cards. Find the probability that both the cards are black.
\\
\solution
		%\input{ncert/12/13/2/2/defs.tex}
		\item A box of oranges is inspected by examining three randomly selected oranges drawn without replacement. If all the three oranges are good, the box is approved for sale, otherwise, it is rejected. Find the probability that a box containing 15 oranges out of which 12 are good and 3 are bad ones will be approved for sale.
		\label{ncert/12/13/2/3/defs.tex}
		\item Two balls are drawn at random with replacement from a box containing 10 black and 8 red balls. Find the probability that
		\label{ncert/12/13/2/12}
\begin{enumerate}
\item both balls are red.
\item first ball is black and second is red.
\item one of them is black and other is red.
\end{enumerate}

\item In a hostel, 60\% of the students read Hindi newspaper, 40\% read English newspaper and 20\% read both Hindi and English newspapers. A student is selected at random.
		\label{ncert/12/13/2/15}
\begin{enumerate}
\item Find the probability that she reads neither Hindi nor English newspapers.
\item If she reads Hindi newspaper, find the probability that she reads English newspaper.
\item If she reads English newspaper, find the probability that she reads Hindi newspaper.\\
\end{enumerate}
\item The probability of obtaining an even prime number on each die, when a pair of dice is rolled is 
\begin{enumerate}
    \item $0$ 
    
    \item $\frac{1}{3}$ 
    
    \item $\frac{1}{12}$ 
    
    \item $\frac{1}{36}$ 
\end{enumerate}
\solution
		%\input{ncert/12/13/2/17/defs.tex}
	\item A bag contains 4 red and 4 black balls, another bag contains 2 red and 6 black balls. One of the two bags is selected at random and a ball is drawn from the bag which is found to be red. Find the probability that the ball is drawn from the first bag.
\\
\solution
		%\input{ncert/12/13/3/2/main.tex}
  \item
  Cards with numbers 2 to 101 are placed in a box. A card is selected at random.Find the probability that the card has
\begin{enumerate}[label=(\roman*)]
	\item an even number 
	\item a square number
\end{enumerate}
\solution
%\input{exemplar/10/13/3/32/main.tex}
\item
The king, queen and jack of clubs are removed from a deck of 52 playing cards and then well shuffled. Now one card is drawn at random from the remaining cards.  Determine the probability that the card is
\begin{enumerate}[label=(\roman*)]
\item a club
\item 10 of hearts
\end{enumerate}
\solution
%\input{exemplar/10/13/3/29/main.tex}
\item A team of medical students doing their internship have to assist during surgeries
at a city hospital. The probabilities of surgeries rated as very complex, complex,
routine, simple or very simple are respectively, 0.15, 0.20, 0.31, 0.26, .08. Find
the probabilities that a particular surgery will be rated
\begin{enumerate}
	\item complex or very complex;
	\item neither very complex nor very simple;
	\item routine or complex
	\item routine or simple
\end{enumerate}
\solution
%\input{exemplar/11/16/3/8(1)/main.tex}
\item A card is selected from a pack of 52 cards.
\begin{enumerate}[label=(\alph*)]
    \item How many points are there in the sample space?
    \item Calculate the probability that the card is an ace of spades.
    \item Calculate the probability that the card is (i) an ace and (ii) black card.
\end{enumerate}
\solution
%\input{exemplar/11/16/3/4/main2.tex}
\item The probability that a non leap year selected at random will contain 53 sundays.
\\
\solution
%\input{exemplar/10/13/1/19/main.tex}
\item One of the four persons John, Rita, Aslam or Gurpreet will be promoted next
month. Consequently the sample space consists of four elementary outcomes
S = {John promoted, Rita promoted, Aslam promoted, Gurpreet promoted}
You are told that the chances of John’s promotion is same as that of Gurpreet,
Rita’s chances of promotion are twice as likely as Johns. Aslam’s chances are
four times that of John.
\begin{enumerate}
	\item Determine
	\begin{enumerate}
		\item P (John promoted)
		\item P (Rita promoted)
		\item P (Aslam promoted)
		\item P (Gurpreet promoted)
	\end{enumerate}
	\item If A = {John promoted or Gurpreet promoted}, find P (A).
\end{enumerate}
\solution
%\input{exemplar/11/16/3/10/main.tex}
\item A card is drawn from a deck of 52 cards. Find the probability of getting a king or a heart or a red card.\\
\solution
%\input{exemplar/11/16/3/15/main.tex}
\item The probability that a student will pass his examination is 0.73, the probability of
the student getting a compartment is 0.13, and the probability that the student will
either pass or get compartment is 0.96. State True or False.\\
\solution
%\input{exemplar/11/16/3/31/main.tex}
\item A card is selected from a pack of 52 cards\\
\begin{enumerate}[label=(\alph*)]
\item How many points are there in the sample space?
\item Calculate the probability that the cards is an ace of spades.
\item Calculate the probability that the card is (i) an ace (ii)black card.\\
\end{enumerate}
%\input{ncert/11/16/3/4_1/Prob_4.tex}
\item In a non-leap year, the probability of having 53 tuesdays or 53 wednesdays is\\
\solution
%\input{exemplar/11/16/3/18/main.tex}
\item There are 1000 sealed envelopes in a box, 10 of them contain a cash prize of
Rs 100 each, 100 of them contain a cash prize of Rs 50 each and 200 of them
contain a cash prize of Rs 10 each and rest do not contain any cash prize. If they
are well shuffled and an envelope is picked up out, what is the probability that it
contains no cash prize?\\
\solution
%\input{exemplar/10/13/3/34/main.tex}
\item 
A die is thrown and a card is selected at random from a deck of 52 playing cards. The probability of getting an even number on the die and a spade card.\\
\solution
%\input{exemplar/12/13/3/78/main.tex}
\item
If 4-digit numbers greater than 5,000 are randomly formed from the digits 0, 1, 3, 5, and 7, what is the probability of forming a number divisible by 5 when:
\begin{enumerate}
    \item The digits are repeated?
    \item The repetition of digits is not allowed?
\end{enumerate}
\solution
%\input{ncert/11/16/4/9/main.tex}
\item Consider the probability space $\brak{\Omega, \mathcal{G}, P}$ where $\Omega = [0,2]$ and $\mathcal{G} = \cbrak{\phi, \Omega, [0,1], (1,2]}$. Let $X$ and $Y$ be two functions on $\Omega$ defined as
\begin{align*}
    X(\omega) = 
    \begin{cases}
        1 & \text{if }\omega \in [0, 1]\\
        2 & \text{if }\omega \in (1, 2]
    \end{cases}
\end{align*}
and
\begin{align*}
    Y(\omega) = 
    \begin{cases}
        2 & \text{if }\omega \in [0, 1.5]\\
        3 & \text{if }\omega \in (1.5, 2].
    \end{cases}
\end{align*}
Then which one of the following statements is true?
\begin{enumerate}
    \item [(A)] $X$ is a random variable with respect to $\mathcal{G}$, but $Y$ is not a random variable with respect to $\mathcal{G}$.
    \item [(B)] $Y$ is a random variable with respect to $\mathcal{G}$, but $X$ is not a random variable with respect to $\mathcal{G}$.
    \item [(C)] Neither $X$ nor $Y$ is a random variable with respect to $\mathcal{G}$.
    \item [(D)] Both $X$ and $Y$ are random variables with respect to $\mathcal{G}$.
\end{enumerate} \hfill (GATE ST 2023)\\
\solution
%\input{gate/ST/2023/14/main.tex}
	\item  A die is loaded in such a way that each odd number is twice as likely to occur as
each even number. Find $P(G)$, where $G$ is the event that a number greater than
3 occurs on a single roll of the die.
\\
\solution
		%\input{exemplar/11/16/3/5/main.tex}
	\item All the jacks, queens and kings are removed from a deck of 52 playing cards. The remaining cards are well shuffled and then one card is drawn at random. Giving ace a value 1 similar value for other cards, find the probability that the card has a value 
		\begin{enumerate}
			\item 7
			\item greater than 7
			\item less than 7
		\end{enumerate}
		%\input{exemplar/10/13/3/30/main.tex}
  \item A Lot consists of 48 mobile phones of which 42 are good, 3 have only minor defects and 3 have major defects.Varnika will buy a phone if it is good but the trader will only buy a mobile if it has no major defects. One phone is selected at random from the lot. What is the probability that it is
\begin{enumerate}
	\item acceptable to Varnika?
            \item acceptable to the trader?
\end{enumerate}
\solution
	%\input{exemplar/10/13/3/40/main.tex}
 \item A student says that if you throw a die, it will show up 1 or not 1. Therefore, the probability of getting 1 and the probability of getting 'not 1' each is equal to $\frac{1}{2}$. Is this correct? Give reasons.\\
 \solution
        %\input{exemplar/10/13/2/9/main.tex}
   \item Four candidates A, B, C, D have ap-
plied for the assignment to coach a school cricket
team. If A is twice as likely to be selected as B, and
B and C are given about the same chance of being
selected, while C is twice as likely to be selected
as D, what are the probabilities that
\begin{enumerate}
\item C will be selected?
\item A will not be selected?
\end{enumerate}
	%\input{exemplar/11/16/3/9/main.tex}
 \item A bag contain 24 balls of which $x$ balls are red, $2x$ are white and $3x$ are blue. A ball is selected at random, What is the probability that it is
\begin{enumerate}[label=\alph*)]
\item not red ?
\item white ?
\end{enumerate}
%\input{exemplar/10/13/3/41/main.tex}
If the letters of the word ASSASSINATION are arranged at random. Find the Probability that
\begin{enumerate}[label=(\alph*)]
\item Four $S's$ come consecutively in the word
\item Two  $I's$ and two $N's$ come together
\item All $A's$ are not coming together
\item No two $A's$ are coming together
\end{enumerate}
%\input{exemplar/11/16/3/14/main.tex}
	\item One urn contains two black balls (labelled B1 and B2) and one white ball. A
	second urn contains one black ball and two white balls (labelled W1 and W2).
	Suppose the following experiment is performed. One of the two urns is chosen
	at random. Next a ball is randomly chosen from the urn. Then a second ball is
	chosen at random from the same urn without replacing the first ball.
	
	\begin{enumerate}
	\item What is the probability that two black balls are chosen?
	
	\item What is the probability that two balls of opposite colour are chosen?
	\end{enumerate}
	\solution
	%\input{exemplar/11/16/3/12/main1.tex}
\end{enumerate}

	\item A card is selected from a pack of 52 cards.
 \begin{enumerate}[label=(\alph*)] 
                 \item How many points are there in the sample space?
                 \item Calculate the probability that the card is an ace of spades.
                 \item Calculate the probability that the card is (i) an ace and (ii) black card.
 \end{enumerate}
\solution
		%\begin{table}[H]
	\centering
\begin{tabular}{|c|c|c|}
\hline
Random variable &Value &Definition\\ \hline
\multirow{3}{*}{X} &0 &Slips of Rs 1\\
&1 &Slips of Rs 5\\
&2 &Slips of Rs 13\\ \hline
\multirow{2}{*}{Y} &0 &Box A\\
&1 &Box B\\\hline
\end{tabular}
\caption{}
\label{tab:Distribution}
\end{table}
See \tabref{tab:Distribution}.
\begin{align}
p_{Y}\brak{k}= \begin{cases} 
      \frac{1}{3} & {k=0} \\
      \frac{2}{3 }& {k=1} 
   \end{cases}
   \\
p_{Y|X}\brak{0|0} = \frac{19}{25}\, 
p_{Y|X}\brak{0|1} = \frac{6}{25}\,
p_{Y|X}\brak{1|0} = \frac{45}{50}\,
p_{Y|X}\brak{1|2} = \frac{5}{50}
\end{align}
The desired probability is the probability that a slip drawn at random is marked other than Rs 1,
\begin{align}
&=1-p_X\brak{0}\\
&= p_X(1) + p_X(2)
\end{align}
Using Bayes theorem,
\begin{align}
&= p_Y\brak{0} \times \pr{Y=0 | X=1} + p_Y\brak{1} \times \pr{Y=1|X=2}\\
&=\frac{1}{3} \times \frac{6}{25} + \frac{2}{3} \times \frac{5}{50}\\
&=\frac{11}{75}
\end{align}

\newpage

%\tableofcontents

\bigskip

\renewcommand{\thefigure}{\theenumi}
\renewcommand{\thetable}{\theenumi}
%\renewcommand{\theequation}{\theenumi}

%\begin{abstract}
%%\boldmath
%In this letter, an algorithm for evaluating the exact analytical bit error rate  (BER)  for the piecewise linear (PL) combiner for  multiple relays is presented. Previous results were available only for upto three relays. The algorithm is unique in the sense that  the actual mathematical expressions, that are prohibitively large, need not be explicitly obtained. The diversity gain due to multiple relays is shown through plots of the analytical BER, well supported by simulations. 
%
%\end{abstract}
% IEEEtran.cls defaults to using nonbold math in the Abstract.
% This preserves the distinction between vectors and scalars. However,
% if the journal you are submitting to favors bold math in the abstract,
% then you can use LaTeX's standard command \boldmath at the very start
% of the abstract to achieve this. Many IEEE journals frown on math
% in the abstract anyway.

% Note that keywords are not normally used for peerreview papers.
%\begin{IEEEkeywords}
%Cooperative diversity, decode and forward, piecewise linear
%\end{IEEEkeywords}



% For peer review papers, you can put extra information on the cover
% page as needed:
% \ifCLASSOPTIONpeerreview
% \begin{center} \bfseries EDICS Category: 3-BBND \end{center}
% \fi
%
% For peerreview papers, this IEEEtran command inserts a page break and
% creates the second title. It will be ignored for other modes.
%\IEEEpeerreviewmaketitle




\item Four cards are drawn from a well-shuffled deck of 52 cards. What is the probability of obtaining 3 diamonds and one spade.
\\
\solution
		%\begin{enumerate}[label=\thesection.\arabic*,ref=\thesection.\theenumi]
	\item One card is drawn from a well-shuffled deck of 52 cards. Find the probability of getting
\begin{enumerate}
\item A king of red colour 
\item A face card 
\item A red face card
\item The jack of hearts
\item A spade
\item The queen of diamonds

\end{enumerate}
\solution
		%\input{ncert/10/15/1/14/main.tex}
	\item Five cards—the ten, jack, queen, king and ace of diamonds, are well-shuffled with their face downwards. One card is then picked up at random.
\begin{enumerate}
\item
What is the probability that the card is the queen? 
\item
If the queen is drawn and put aside, what is the probability that the second card picked up is (a) an ace? (b) a queen?\\
\end{enumerate}
\solution
		%\input{ncert/10/15/1/15/defs.tex}
	\item A bag contains $5$ red balls and some blue balls. If the probability of drawing a blue ball is double that if a red ball, determine the number of blue balls in the bag. 
		\\
\solution
		%\input{ncert/10/15/2/3/defs.tex}
	\item A card is selected from a pack of 52 cards.
 \begin{enumerate}[label=(\alph*)] 
                 \item How many points are there in the sample space?
                 \item Calculate the probability that the card is an ace of spades.
                 \item Calculate the probability that the card is (i) an ace and (ii) black card.
 \end{enumerate}
\solution
		%\input{ncert/11/16/3/4/main.tex}
\item Four cards are drawn from a well-shuffled deck of 52 cards. What is the probability of obtaining 3 diamonds and one spade.
\\
\solution
		%\input{ncert/11/16/4/2/defs.tex}
\item In a certain lottery 10,000 tickets are sold and ten equal prizes are awarded. What is the probability of not getting a prize if you buy (a) one ticket (b) two tickets (c) 10 tickets ?	
\\
\solution
		%\input{ncert/11/16/4/4/defs.tex}
		%
\item 
Out of 100 students, two sections of 40 and 60 are formed. If you and your friend are among the 100 students, what is the probability that
\begin{enumerate}
\item you both enter the same section?
\item you both enter the different sections?
\end{enumerate}
\solution
		%\input{ncert/11/16/4/5/defs.tex}
	\item 
The number lock of a suitcase has 4 wheels each labelled with ten digits i.e. from 0 to 9.The lock opens with a sequence of four digits with no repeats.What is the probability of a person getting the right sequence to open the suitcase.
\\
\solution
		%\input{ncert/11/16/4/10/defs.tex}
		%
\item 
Two cards are drawn at random and without replacement from a pack of 52 playing cards. Find the probability that both the cards are black.
\\
\solution
		%\input{ncert/12/13/2/2/defs.tex}
		\item A box of oranges is inspected by examining three randomly selected oranges drawn without replacement. If all the three oranges are good, the box is approved for sale, otherwise, it is rejected. Find the probability that a box containing 15 oranges out of which 12 are good and 3 are bad ones will be approved for sale.
		\label{ncert/12/13/2/3/defs.tex}
		\item Two balls are drawn at random with replacement from a box containing 10 black and 8 red balls. Find the probability that
		\label{ncert/12/13/2/12}
\begin{enumerate}
\item both balls are red.
\item first ball is black and second is red.
\item one of them is black and other is red.
\end{enumerate}

\item In a hostel, 60\% of the students read Hindi newspaper, 40\% read English newspaper and 20\% read both Hindi and English newspapers. A student is selected at random.
		\label{ncert/12/13/2/15}
\begin{enumerate}
\item Find the probability that she reads neither Hindi nor English newspapers.
\item If she reads Hindi newspaper, find the probability that she reads English newspaper.
\item If she reads English newspaper, find the probability that she reads Hindi newspaper.\\
\end{enumerate}
\item The probability of obtaining an even prime number on each die, when a pair of dice is rolled is 
\begin{enumerate}
    \item $0$ 
    
    \item $\frac{1}{3}$ 
    
    \item $\frac{1}{12}$ 
    
    \item $\frac{1}{36}$ 
\end{enumerate}
\solution
		%\input{ncert/12/13/2/17/defs.tex}
	\item A bag contains 4 red and 4 black balls, another bag contains 2 red and 6 black balls. One of the two bags is selected at random and a ball is drawn from the bag which is found to be red. Find the probability that the ball is drawn from the first bag.
\\
\solution
		%\input{ncert/12/13/3/2/main.tex}
  \item
  Cards with numbers 2 to 101 are placed in a box. A card is selected at random.Find the probability that the card has
\begin{enumerate}[label=(\roman*)]
	\item an even number 
	\item a square number
\end{enumerate}
\solution
%\input{exemplar/10/13/3/32/main.tex}
\item
The king, queen and jack of clubs are removed from a deck of 52 playing cards and then well shuffled. Now one card is drawn at random from the remaining cards.  Determine the probability that the card is
\begin{enumerate}[label=(\roman*)]
\item a club
\item 10 of hearts
\end{enumerate}
\solution
%\input{exemplar/10/13/3/29/main.tex}
\item A team of medical students doing their internship have to assist during surgeries
at a city hospital. The probabilities of surgeries rated as very complex, complex,
routine, simple or very simple are respectively, 0.15, 0.20, 0.31, 0.26, .08. Find
the probabilities that a particular surgery will be rated
\begin{enumerate}
	\item complex or very complex;
	\item neither very complex nor very simple;
	\item routine or complex
	\item routine or simple
\end{enumerate}
\solution
%\input{exemplar/11/16/3/8(1)/main.tex}
\item A card is selected from a pack of 52 cards.
\begin{enumerate}[label=(\alph*)]
    \item How many points are there in the sample space?
    \item Calculate the probability that the card is an ace of spades.
    \item Calculate the probability that the card is (i) an ace and (ii) black card.
\end{enumerate}
\solution
%\input{exemplar/11/16/3/4/main2.tex}
\item The probability that a non leap year selected at random will contain 53 sundays.
\\
\solution
%\input{exemplar/10/13/1/19/main.tex}
\item One of the four persons John, Rita, Aslam or Gurpreet will be promoted next
month. Consequently the sample space consists of four elementary outcomes
S = {John promoted, Rita promoted, Aslam promoted, Gurpreet promoted}
You are told that the chances of John’s promotion is same as that of Gurpreet,
Rita’s chances of promotion are twice as likely as Johns. Aslam’s chances are
four times that of John.
\begin{enumerate}
	\item Determine
	\begin{enumerate}
		\item P (John promoted)
		\item P (Rita promoted)
		\item P (Aslam promoted)
		\item P (Gurpreet promoted)
	\end{enumerate}
	\item If A = {John promoted or Gurpreet promoted}, find P (A).
\end{enumerate}
\solution
%\input{exemplar/11/16/3/10/main.tex}
\item A card is drawn from a deck of 52 cards. Find the probability of getting a king or a heart or a red card.\\
\solution
%\input{exemplar/11/16/3/15/main.tex}
\item The probability that a student will pass his examination is 0.73, the probability of
the student getting a compartment is 0.13, and the probability that the student will
either pass or get compartment is 0.96. State True or False.\\
\solution
%\input{exemplar/11/16/3/31/main.tex}
\item A card is selected from a pack of 52 cards\\
\begin{enumerate}[label=(\alph*)]
\item How many points are there in the sample space?
\item Calculate the probability that the cards is an ace of spades.
\item Calculate the probability that the card is (i) an ace (ii)black card.\\
\end{enumerate}
%\input{ncert/11/16/3/4_1/Prob_4.tex}
\item In a non-leap year, the probability of having 53 tuesdays or 53 wednesdays is\\
\solution
%\input{exemplar/11/16/3/18/main.tex}
\item There are 1000 sealed envelopes in a box, 10 of them contain a cash prize of
Rs 100 each, 100 of them contain a cash prize of Rs 50 each and 200 of them
contain a cash prize of Rs 10 each and rest do not contain any cash prize. If they
are well shuffled and an envelope is picked up out, what is the probability that it
contains no cash prize?\\
\solution
%\input{exemplar/10/13/3/34/main.tex}
\item 
A die is thrown and a card is selected at random from a deck of 52 playing cards. The probability of getting an even number on the die and a spade card.\\
\solution
%\input{exemplar/12/13/3/78/main.tex}
\item
If 4-digit numbers greater than 5,000 are randomly formed from the digits 0, 1, 3, 5, and 7, what is the probability of forming a number divisible by 5 when:
\begin{enumerate}
    \item The digits are repeated?
    \item The repetition of digits is not allowed?
\end{enumerate}
\solution
%\input{ncert/11/16/4/9/main.tex}
\item Consider the probability space $\brak{\Omega, \mathcal{G}, P}$ where $\Omega = [0,2]$ and $\mathcal{G} = \cbrak{\phi, \Omega, [0,1], (1,2]}$. Let $X$ and $Y$ be two functions on $\Omega$ defined as
\begin{align*}
    X(\omega) = 
    \begin{cases}
        1 & \text{if }\omega \in [0, 1]\\
        2 & \text{if }\omega \in (1, 2]
    \end{cases}
\end{align*}
and
\begin{align*}
    Y(\omega) = 
    \begin{cases}
        2 & \text{if }\omega \in [0, 1.5]\\
        3 & \text{if }\omega \in (1.5, 2].
    \end{cases}
\end{align*}
Then which one of the following statements is true?
\begin{enumerate}
    \item [(A)] $X$ is a random variable with respect to $\mathcal{G}$, but $Y$ is not a random variable with respect to $\mathcal{G}$.
    \item [(B)] $Y$ is a random variable with respect to $\mathcal{G}$, but $X$ is not a random variable with respect to $\mathcal{G}$.
    \item [(C)] Neither $X$ nor $Y$ is a random variable with respect to $\mathcal{G}$.
    \item [(D)] Both $X$ and $Y$ are random variables with respect to $\mathcal{G}$.
\end{enumerate} \hfill (GATE ST 2023)\\
\solution
%\input{gate/ST/2023/14/main.tex}
	\item  A die is loaded in such a way that each odd number is twice as likely to occur as
each even number. Find $P(G)$, where $G$ is the event that a number greater than
3 occurs on a single roll of the die.
\\
\solution
		%\input{exemplar/11/16/3/5/main.tex}
	\item All the jacks, queens and kings are removed from a deck of 52 playing cards. The remaining cards are well shuffled and then one card is drawn at random. Giving ace a value 1 similar value for other cards, find the probability that the card has a value 
		\begin{enumerate}
			\item 7
			\item greater than 7
			\item less than 7
		\end{enumerate}
		%\input{exemplar/10/13/3/30/main.tex}
  \item A Lot consists of 48 mobile phones of which 42 are good, 3 have only minor defects and 3 have major defects.Varnika will buy a phone if it is good but the trader will only buy a mobile if it has no major defects. One phone is selected at random from the lot. What is the probability that it is
\begin{enumerate}
	\item acceptable to Varnika?
            \item acceptable to the trader?
\end{enumerate}
\solution
	%\input{exemplar/10/13/3/40/main.tex}
 \item A student says that if you throw a die, it will show up 1 or not 1. Therefore, the probability of getting 1 and the probability of getting 'not 1' each is equal to $\frac{1}{2}$. Is this correct? Give reasons.\\
 \solution
        %\input{exemplar/10/13/2/9/main.tex}
   \item Four candidates A, B, C, D have ap-
plied for the assignment to coach a school cricket
team. If A is twice as likely to be selected as B, and
B and C are given about the same chance of being
selected, while C is twice as likely to be selected
as D, what are the probabilities that
\begin{enumerate}
\item C will be selected?
\item A will not be selected?
\end{enumerate}
	%\input{exemplar/11/16/3/9/main.tex}
 \item A bag contain 24 balls of which $x$ balls are red, $2x$ are white and $3x$ are blue. A ball is selected at random, What is the probability that it is
\begin{enumerate}[label=\alph*)]
\item not red ?
\item white ?
\end{enumerate}
%\input{exemplar/10/13/3/41/main.tex}
If the letters of the word ASSASSINATION are arranged at random. Find the Probability that
\begin{enumerate}[label=(\alph*)]
\item Four $S's$ come consecutively in the word
\item Two  $I's$ and two $N's$ come together
\item All $A's$ are not coming together
\item No two $A's$ are coming together
\end{enumerate}
%\input{exemplar/11/16/3/14/main.tex}
	\item One urn contains two black balls (labelled B1 and B2) and one white ball. A
	second urn contains one black ball and two white balls (labelled W1 and W2).
	Suppose the following experiment is performed. One of the two urns is chosen
	at random. Next a ball is randomly chosen from the urn. Then a second ball is
	chosen at random from the same urn without replacing the first ball.
	
	\begin{enumerate}
	\item What is the probability that two black balls are chosen?
	
	\item What is the probability that two balls of opposite colour are chosen?
	\end{enumerate}
	\solution
	%\input{exemplar/11/16/3/12/main1.tex}
\end{enumerate}

\item In a certain lottery 10,000 tickets are sold and ten equal prizes are awarded. What is the probability of not getting a prize if you buy (a) one ticket (b) two tickets (c) 10 tickets ?	
\\
\solution
		%\begin{enumerate}[label=\thesection.\arabic*,ref=\thesection.\theenumi]
	\item One card is drawn from a well-shuffled deck of 52 cards. Find the probability of getting
\begin{enumerate}
\item A king of red colour 
\item A face card 
\item A red face card
\item The jack of hearts
\item A spade
\item The queen of diamonds

\end{enumerate}
\solution
		%\input{ncert/10/15/1/14/main.tex}
	\item Five cards—the ten, jack, queen, king and ace of diamonds, are well-shuffled with their face downwards. One card is then picked up at random.
\begin{enumerate}
\item
What is the probability that the card is the queen? 
\item
If the queen is drawn and put aside, what is the probability that the second card picked up is (a) an ace? (b) a queen?\\
\end{enumerate}
\solution
		%\input{ncert/10/15/1/15/defs.tex}
	\item A bag contains $5$ red balls and some blue balls. If the probability of drawing a blue ball is double that if a red ball, determine the number of blue balls in the bag. 
		\\
\solution
		%\input{ncert/10/15/2/3/defs.tex}
	\item A card is selected from a pack of 52 cards.
 \begin{enumerate}[label=(\alph*)] 
                 \item How many points are there in the sample space?
                 \item Calculate the probability that the card is an ace of spades.
                 \item Calculate the probability that the card is (i) an ace and (ii) black card.
 \end{enumerate}
\solution
		%\input{ncert/11/16/3/4/main.tex}
\item Four cards are drawn from a well-shuffled deck of 52 cards. What is the probability of obtaining 3 diamonds and one spade.
\\
\solution
		%\input{ncert/11/16/4/2/defs.tex}
\item In a certain lottery 10,000 tickets are sold and ten equal prizes are awarded. What is the probability of not getting a prize if you buy (a) one ticket (b) two tickets (c) 10 tickets ?	
\\
\solution
		%\input{ncert/11/16/4/4/defs.tex}
		%
\item 
Out of 100 students, two sections of 40 and 60 are formed. If you and your friend are among the 100 students, what is the probability that
\begin{enumerate}
\item you both enter the same section?
\item you both enter the different sections?
\end{enumerate}
\solution
		%\input{ncert/11/16/4/5/defs.tex}
	\item 
The number lock of a suitcase has 4 wheels each labelled with ten digits i.e. from 0 to 9.The lock opens with a sequence of four digits with no repeats.What is the probability of a person getting the right sequence to open the suitcase.
\\
\solution
		%\input{ncert/11/16/4/10/defs.tex}
		%
\item 
Two cards are drawn at random and without replacement from a pack of 52 playing cards. Find the probability that both the cards are black.
\\
\solution
		%\input{ncert/12/13/2/2/defs.tex}
		\item A box of oranges is inspected by examining three randomly selected oranges drawn without replacement. If all the three oranges are good, the box is approved for sale, otherwise, it is rejected. Find the probability that a box containing 15 oranges out of which 12 are good and 3 are bad ones will be approved for sale.
		\label{ncert/12/13/2/3/defs.tex}
		\item Two balls are drawn at random with replacement from a box containing 10 black and 8 red balls. Find the probability that
		\label{ncert/12/13/2/12}
\begin{enumerate}
\item both balls are red.
\item first ball is black and second is red.
\item one of them is black and other is red.
\end{enumerate}

\item In a hostel, 60\% of the students read Hindi newspaper, 40\% read English newspaper and 20\% read both Hindi and English newspapers. A student is selected at random.
		\label{ncert/12/13/2/15}
\begin{enumerate}
\item Find the probability that she reads neither Hindi nor English newspapers.
\item If she reads Hindi newspaper, find the probability that she reads English newspaper.
\item If she reads English newspaper, find the probability that she reads Hindi newspaper.\\
\end{enumerate}
\item The probability of obtaining an even prime number on each die, when a pair of dice is rolled is 
\begin{enumerate}
    \item $0$ 
    
    \item $\frac{1}{3}$ 
    
    \item $\frac{1}{12}$ 
    
    \item $\frac{1}{36}$ 
\end{enumerate}
\solution
		%\input{ncert/12/13/2/17/defs.tex}
	\item A bag contains 4 red and 4 black balls, another bag contains 2 red and 6 black balls. One of the two bags is selected at random and a ball is drawn from the bag which is found to be red. Find the probability that the ball is drawn from the first bag.
\\
\solution
		%\input{ncert/12/13/3/2/main.tex}
  \item
  Cards with numbers 2 to 101 are placed in a box. A card is selected at random.Find the probability that the card has
\begin{enumerate}[label=(\roman*)]
	\item an even number 
	\item a square number
\end{enumerate}
\solution
%\input{exemplar/10/13/3/32/main.tex}
\item
The king, queen and jack of clubs are removed from a deck of 52 playing cards and then well shuffled. Now one card is drawn at random from the remaining cards.  Determine the probability that the card is
\begin{enumerate}[label=(\roman*)]
\item a club
\item 10 of hearts
\end{enumerate}
\solution
%\input{exemplar/10/13/3/29/main.tex}
\item A team of medical students doing their internship have to assist during surgeries
at a city hospital. The probabilities of surgeries rated as very complex, complex,
routine, simple or very simple are respectively, 0.15, 0.20, 0.31, 0.26, .08. Find
the probabilities that a particular surgery will be rated
\begin{enumerate}
	\item complex or very complex;
	\item neither very complex nor very simple;
	\item routine or complex
	\item routine or simple
\end{enumerate}
\solution
%\input{exemplar/11/16/3/8(1)/main.tex}
\item A card is selected from a pack of 52 cards.
\begin{enumerate}[label=(\alph*)]
    \item How many points are there in the sample space?
    \item Calculate the probability that the card is an ace of spades.
    \item Calculate the probability that the card is (i) an ace and (ii) black card.
\end{enumerate}
\solution
%\input{exemplar/11/16/3/4/main2.tex}
\item The probability that a non leap year selected at random will contain 53 sundays.
\\
\solution
%\input{exemplar/10/13/1/19/main.tex}
\item One of the four persons John, Rita, Aslam or Gurpreet will be promoted next
month. Consequently the sample space consists of four elementary outcomes
S = {John promoted, Rita promoted, Aslam promoted, Gurpreet promoted}
You are told that the chances of John’s promotion is same as that of Gurpreet,
Rita’s chances of promotion are twice as likely as Johns. Aslam’s chances are
four times that of John.
\begin{enumerate}
	\item Determine
	\begin{enumerate}
		\item P (John promoted)
		\item P (Rita promoted)
		\item P (Aslam promoted)
		\item P (Gurpreet promoted)
	\end{enumerate}
	\item If A = {John promoted or Gurpreet promoted}, find P (A).
\end{enumerate}
\solution
%\input{exemplar/11/16/3/10/main.tex}
\item A card is drawn from a deck of 52 cards. Find the probability of getting a king or a heart or a red card.\\
\solution
%\input{exemplar/11/16/3/15/main.tex}
\item The probability that a student will pass his examination is 0.73, the probability of
the student getting a compartment is 0.13, and the probability that the student will
either pass or get compartment is 0.96. State True or False.\\
\solution
%\input{exemplar/11/16/3/31/main.tex}
\item A card is selected from a pack of 52 cards\\
\begin{enumerate}[label=(\alph*)]
\item How many points are there in the sample space?
\item Calculate the probability that the cards is an ace of spades.
\item Calculate the probability that the card is (i) an ace (ii)black card.\\
\end{enumerate}
%\input{ncert/11/16/3/4_1/Prob_4.tex}
\item In a non-leap year, the probability of having 53 tuesdays or 53 wednesdays is\\
\solution
%\input{exemplar/11/16/3/18/main.tex}
\item There are 1000 sealed envelopes in a box, 10 of them contain a cash prize of
Rs 100 each, 100 of them contain a cash prize of Rs 50 each and 200 of them
contain a cash prize of Rs 10 each and rest do not contain any cash prize. If they
are well shuffled and an envelope is picked up out, what is the probability that it
contains no cash prize?\\
\solution
%\input{exemplar/10/13/3/34/main.tex}
\item 
A die is thrown and a card is selected at random from a deck of 52 playing cards. The probability of getting an even number on the die and a spade card.\\
\solution
%\input{exemplar/12/13/3/78/main.tex}
\item
If 4-digit numbers greater than 5,000 are randomly formed from the digits 0, 1, 3, 5, and 7, what is the probability of forming a number divisible by 5 when:
\begin{enumerate}
    \item The digits are repeated?
    \item The repetition of digits is not allowed?
\end{enumerate}
\solution
%\input{ncert/11/16/4/9/main.tex}
\item Consider the probability space $\brak{\Omega, \mathcal{G}, P}$ where $\Omega = [0,2]$ and $\mathcal{G} = \cbrak{\phi, \Omega, [0,1], (1,2]}$. Let $X$ and $Y$ be two functions on $\Omega$ defined as
\begin{align*}
    X(\omega) = 
    \begin{cases}
        1 & \text{if }\omega \in [0, 1]\\
        2 & \text{if }\omega \in (1, 2]
    \end{cases}
\end{align*}
and
\begin{align*}
    Y(\omega) = 
    \begin{cases}
        2 & \text{if }\omega \in [0, 1.5]\\
        3 & \text{if }\omega \in (1.5, 2].
    \end{cases}
\end{align*}
Then which one of the following statements is true?
\begin{enumerate}
    \item [(A)] $X$ is a random variable with respect to $\mathcal{G}$, but $Y$ is not a random variable with respect to $\mathcal{G}$.
    \item [(B)] $Y$ is a random variable with respect to $\mathcal{G}$, but $X$ is not a random variable with respect to $\mathcal{G}$.
    \item [(C)] Neither $X$ nor $Y$ is a random variable with respect to $\mathcal{G}$.
    \item [(D)] Both $X$ and $Y$ are random variables with respect to $\mathcal{G}$.
\end{enumerate} \hfill (GATE ST 2023)\\
\solution
%\input{gate/ST/2023/14/main.tex}
	\item  A die is loaded in such a way that each odd number is twice as likely to occur as
each even number. Find $P(G)$, where $G$ is the event that a number greater than
3 occurs on a single roll of the die.
\\
\solution
		%\input{exemplar/11/16/3/5/main.tex}
	\item All the jacks, queens and kings are removed from a deck of 52 playing cards. The remaining cards are well shuffled and then one card is drawn at random. Giving ace a value 1 similar value for other cards, find the probability that the card has a value 
		\begin{enumerate}
			\item 7
			\item greater than 7
			\item less than 7
		\end{enumerate}
		%\input{exemplar/10/13/3/30/main.tex}
  \item A Lot consists of 48 mobile phones of which 42 are good, 3 have only minor defects and 3 have major defects.Varnika will buy a phone if it is good but the trader will only buy a mobile if it has no major defects. One phone is selected at random from the lot. What is the probability that it is
\begin{enumerate}
	\item acceptable to Varnika?
            \item acceptable to the trader?
\end{enumerate}
\solution
	%\input{exemplar/10/13/3/40/main.tex}
 \item A student says that if you throw a die, it will show up 1 or not 1. Therefore, the probability of getting 1 and the probability of getting 'not 1' each is equal to $\frac{1}{2}$. Is this correct? Give reasons.\\
 \solution
        %\input{exemplar/10/13/2/9/main.tex}
   \item Four candidates A, B, C, D have ap-
plied for the assignment to coach a school cricket
team. If A is twice as likely to be selected as B, and
B and C are given about the same chance of being
selected, while C is twice as likely to be selected
as D, what are the probabilities that
\begin{enumerate}
\item C will be selected?
\item A will not be selected?
\end{enumerate}
	%\input{exemplar/11/16/3/9/main.tex}
 \item A bag contain 24 balls of which $x$ balls are red, $2x$ are white and $3x$ are blue. A ball is selected at random, What is the probability that it is
\begin{enumerate}[label=\alph*)]
\item not red ?
\item white ?
\end{enumerate}
%\input{exemplar/10/13/3/41/main.tex}
If the letters of the word ASSASSINATION are arranged at random. Find the Probability that
\begin{enumerate}[label=(\alph*)]
\item Four $S's$ come consecutively in the word
\item Two  $I's$ and two $N's$ come together
\item All $A's$ are not coming together
\item No two $A's$ are coming together
\end{enumerate}
%\input{exemplar/11/16/3/14/main.tex}
	\item One urn contains two black balls (labelled B1 and B2) and one white ball. A
	second urn contains one black ball and two white balls (labelled W1 and W2).
	Suppose the following experiment is performed. One of the two urns is chosen
	at random. Next a ball is randomly chosen from the urn. Then a second ball is
	chosen at random from the same urn without replacing the first ball.
	
	\begin{enumerate}
	\item What is the probability that two black balls are chosen?
	
	\item What is the probability that two balls of opposite colour are chosen?
	\end{enumerate}
	\solution
	%\input{exemplar/11/16/3/12/main1.tex}
\end{enumerate}

		%
\item 
Out of 100 students, two sections of 40 and 60 are formed. If you and your friend are among the 100 students, what is the probability that
\begin{enumerate}
\item you both enter the same section?
\item you both enter the different sections?
\end{enumerate}
\solution
		%\begin{enumerate}[label=\thesection.\arabic*,ref=\thesection.\theenumi]
	\item One card is drawn from a well-shuffled deck of 52 cards. Find the probability of getting
\begin{enumerate}
\item A king of red colour 
\item A face card 
\item A red face card
\item The jack of hearts
\item A spade
\item The queen of diamonds

\end{enumerate}
\solution
		%\input{ncert/10/15/1/14/main.tex}
	\item Five cards—the ten, jack, queen, king and ace of diamonds, are well-shuffled with their face downwards. One card is then picked up at random.
\begin{enumerate}
\item
What is the probability that the card is the queen? 
\item
If the queen is drawn and put aside, what is the probability that the second card picked up is (a) an ace? (b) a queen?\\
\end{enumerate}
\solution
		%\input{ncert/10/15/1/15/defs.tex}
	\item A bag contains $5$ red balls and some blue balls. If the probability of drawing a blue ball is double that if a red ball, determine the number of blue balls in the bag. 
		\\
\solution
		%\input{ncert/10/15/2/3/defs.tex}
	\item A card is selected from a pack of 52 cards.
 \begin{enumerate}[label=(\alph*)] 
                 \item How many points are there in the sample space?
                 \item Calculate the probability that the card is an ace of spades.
                 \item Calculate the probability that the card is (i) an ace and (ii) black card.
 \end{enumerate}
\solution
		%\input{ncert/11/16/3/4/main.tex}
\item Four cards are drawn from a well-shuffled deck of 52 cards. What is the probability of obtaining 3 diamonds and one spade.
\\
\solution
		%\input{ncert/11/16/4/2/defs.tex}
\item In a certain lottery 10,000 tickets are sold and ten equal prizes are awarded. What is the probability of not getting a prize if you buy (a) one ticket (b) two tickets (c) 10 tickets ?	
\\
\solution
		%\input{ncert/11/16/4/4/defs.tex}
		%
\item 
Out of 100 students, two sections of 40 and 60 are formed. If you and your friend are among the 100 students, what is the probability that
\begin{enumerate}
\item you both enter the same section?
\item you both enter the different sections?
\end{enumerate}
\solution
		%\input{ncert/11/16/4/5/defs.tex}
	\item 
The number lock of a suitcase has 4 wheels each labelled with ten digits i.e. from 0 to 9.The lock opens with a sequence of four digits with no repeats.What is the probability of a person getting the right sequence to open the suitcase.
\\
\solution
		%\input{ncert/11/16/4/10/defs.tex}
		%
\item 
Two cards are drawn at random and without replacement from a pack of 52 playing cards. Find the probability that both the cards are black.
\\
\solution
		%\input{ncert/12/13/2/2/defs.tex}
		\item A box of oranges is inspected by examining three randomly selected oranges drawn without replacement. If all the three oranges are good, the box is approved for sale, otherwise, it is rejected. Find the probability that a box containing 15 oranges out of which 12 are good and 3 are bad ones will be approved for sale.
		\label{ncert/12/13/2/3/defs.tex}
		\item Two balls are drawn at random with replacement from a box containing 10 black and 8 red balls. Find the probability that
		\label{ncert/12/13/2/12}
\begin{enumerate}
\item both balls are red.
\item first ball is black and second is red.
\item one of them is black and other is red.
\end{enumerate}

\item In a hostel, 60\% of the students read Hindi newspaper, 40\% read English newspaper and 20\% read both Hindi and English newspapers. A student is selected at random.
		\label{ncert/12/13/2/15}
\begin{enumerate}
\item Find the probability that she reads neither Hindi nor English newspapers.
\item If she reads Hindi newspaper, find the probability that she reads English newspaper.
\item If she reads English newspaper, find the probability that she reads Hindi newspaper.\\
\end{enumerate}
\item The probability of obtaining an even prime number on each die, when a pair of dice is rolled is 
\begin{enumerate}
    \item $0$ 
    
    \item $\frac{1}{3}$ 
    
    \item $\frac{1}{12}$ 
    
    \item $\frac{1}{36}$ 
\end{enumerate}
\solution
		%\input{ncert/12/13/2/17/defs.tex}
	\item A bag contains 4 red and 4 black balls, another bag contains 2 red and 6 black balls. One of the two bags is selected at random and a ball is drawn from the bag which is found to be red. Find the probability that the ball is drawn from the first bag.
\\
\solution
		%\input{ncert/12/13/3/2/main.tex}
  \item
  Cards with numbers 2 to 101 are placed in a box. A card is selected at random.Find the probability that the card has
\begin{enumerate}[label=(\roman*)]
	\item an even number 
	\item a square number
\end{enumerate}
\solution
%\input{exemplar/10/13/3/32/main.tex}
\item
The king, queen and jack of clubs are removed from a deck of 52 playing cards and then well shuffled. Now one card is drawn at random from the remaining cards.  Determine the probability that the card is
\begin{enumerate}[label=(\roman*)]
\item a club
\item 10 of hearts
\end{enumerate}
\solution
%\input{exemplar/10/13/3/29/main.tex}
\item A team of medical students doing their internship have to assist during surgeries
at a city hospital. The probabilities of surgeries rated as very complex, complex,
routine, simple or very simple are respectively, 0.15, 0.20, 0.31, 0.26, .08. Find
the probabilities that a particular surgery will be rated
\begin{enumerate}
	\item complex or very complex;
	\item neither very complex nor very simple;
	\item routine or complex
	\item routine or simple
\end{enumerate}
\solution
%\input{exemplar/11/16/3/8(1)/main.tex}
\item A card is selected from a pack of 52 cards.
\begin{enumerate}[label=(\alph*)]
    \item How many points are there in the sample space?
    \item Calculate the probability that the card is an ace of spades.
    \item Calculate the probability that the card is (i) an ace and (ii) black card.
\end{enumerate}
\solution
%\input{exemplar/11/16/3/4/main2.tex}
\item The probability that a non leap year selected at random will contain 53 sundays.
\\
\solution
%\input{exemplar/10/13/1/19/main.tex}
\item One of the four persons John, Rita, Aslam or Gurpreet will be promoted next
month. Consequently the sample space consists of four elementary outcomes
S = {John promoted, Rita promoted, Aslam promoted, Gurpreet promoted}
You are told that the chances of John’s promotion is same as that of Gurpreet,
Rita’s chances of promotion are twice as likely as Johns. Aslam’s chances are
four times that of John.
\begin{enumerate}
	\item Determine
	\begin{enumerate}
		\item P (John promoted)
		\item P (Rita promoted)
		\item P (Aslam promoted)
		\item P (Gurpreet promoted)
	\end{enumerate}
	\item If A = {John promoted or Gurpreet promoted}, find P (A).
\end{enumerate}
\solution
%\input{exemplar/11/16/3/10/main.tex}
\item A card is drawn from a deck of 52 cards. Find the probability of getting a king or a heart or a red card.\\
\solution
%\input{exemplar/11/16/3/15/main.tex}
\item The probability that a student will pass his examination is 0.73, the probability of
the student getting a compartment is 0.13, and the probability that the student will
either pass or get compartment is 0.96. State True or False.\\
\solution
%\input{exemplar/11/16/3/31/main.tex}
\item A card is selected from a pack of 52 cards\\
\begin{enumerate}[label=(\alph*)]
\item How many points are there in the sample space?
\item Calculate the probability that the cards is an ace of spades.
\item Calculate the probability that the card is (i) an ace (ii)black card.\\
\end{enumerate}
%\input{ncert/11/16/3/4_1/Prob_4.tex}
\item In a non-leap year, the probability of having 53 tuesdays or 53 wednesdays is\\
\solution
%\input{exemplar/11/16/3/18/main.tex}
\item There are 1000 sealed envelopes in a box, 10 of them contain a cash prize of
Rs 100 each, 100 of them contain a cash prize of Rs 50 each and 200 of them
contain a cash prize of Rs 10 each and rest do not contain any cash prize. If they
are well shuffled and an envelope is picked up out, what is the probability that it
contains no cash prize?\\
\solution
%\input{exemplar/10/13/3/34/main.tex}
\item 
A die is thrown and a card is selected at random from a deck of 52 playing cards. The probability of getting an even number on the die and a spade card.\\
\solution
%\input{exemplar/12/13/3/78/main.tex}
\item
If 4-digit numbers greater than 5,000 are randomly formed from the digits 0, 1, 3, 5, and 7, what is the probability of forming a number divisible by 5 when:
\begin{enumerate}
    \item The digits are repeated?
    \item The repetition of digits is not allowed?
\end{enumerate}
\solution
%\input{ncert/11/16/4/9/main.tex}
\item Consider the probability space $\brak{\Omega, \mathcal{G}, P}$ where $\Omega = [0,2]$ and $\mathcal{G} = \cbrak{\phi, \Omega, [0,1], (1,2]}$. Let $X$ and $Y$ be two functions on $\Omega$ defined as
\begin{align*}
    X(\omega) = 
    \begin{cases}
        1 & \text{if }\omega \in [0, 1]\\
        2 & \text{if }\omega \in (1, 2]
    \end{cases}
\end{align*}
and
\begin{align*}
    Y(\omega) = 
    \begin{cases}
        2 & \text{if }\omega \in [0, 1.5]\\
        3 & \text{if }\omega \in (1.5, 2].
    \end{cases}
\end{align*}
Then which one of the following statements is true?
\begin{enumerate}
    \item [(A)] $X$ is a random variable with respect to $\mathcal{G}$, but $Y$ is not a random variable with respect to $\mathcal{G}$.
    \item [(B)] $Y$ is a random variable with respect to $\mathcal{G}$, but $X$ is not a random variable with respect to $\mathcal{G}$.
    \item [(C)] Neither $X$ nor $Y$ is a random variable with respect to $\mathcal{G}$.
    \item [(D)] Both $X$ and $Y$ are random variables with respect to $\mathcal{G}$.
\end{enumerate} \hfill (GATE ST 2023)\\
\solution
%\input{gate/ST/2023/14/main.tex}
	\item  A die is loaded in such a way that each odd number is twice as likely to occur as
each even number. Find $P(G)$, where $G$ is the event that a number greater than
3 occurs on a single roll of the die.
\\
\solution
		%\input{exemplar/11/16/3/5/main.tex}
	\item All the jacks, queens and kings are removed from a deck of 52 playing cards. The remaining cards are well shuffled and then one card is drawn at random. Giving ace a value 1 similar value for other cards, find the probability that the card has a value 
		\begin{enumerate}
			\item 7
			\item greater than 7
			\item less than 7
		\end{enumerate}
		%\input{exemplar/10/13/3/30/main.tex}
  \item A Lot consists of 48 mobile phones of which 42 are good, 3 have only minor defects and 3 have major defects.Varnika will buy a phone if it is good but the trader will only buy a mobile if it has no major defects. One phone is selected at random from the lot. What is the probability that it is
\begin{enumerate}
	\item acceptable to Varnika?
            \item acceptable to the trader?
\end{enumerate}
\solution
	%\input{exemplar/10/13/3/40/main.tex}
 \item A student says that if you throw a die, it will show up 1 or not 1. Therefore, the probability of getting 1 and the probability of getting 'not 1' each is equal to $\frac{1}{2}$. Is this correct? Give reasons.\\
 \solution
        %\input{exemplar/10/13/2/9/main.tex}
   \item Four candidates A, B, C, D have ap-
plied for the assignment to coach a school cricket
team. If A is twice as likely to be selected as B, and
B and C are given about the same chance of being
selected, while C is twice as likely to be selected
as D, what are the probabilities that
\begin{enumerate}
\item C will be selected?
\item A will not be selected?
\end{enumerate}
	%\input{exemplar/11/16/3/9/main.tex}
 \item A bag contain 24 balls of which $x$ balls are red, $2x$ are white and $3x$ are blue. A ball is selected at random, What is the probability that it is
\begin{enumerate}[label=\alph*)]
\item not red ?
\item white ?
\end{enumerate}
%\input{exemplar/10/13/3/41/main.tex}
If the letters of the word ASSASSINATION are arranged at random. Find the Probability that
\begin{enumerate}[label=(\alph*)]
\item Four $S's$ come consecutively in the word
\item Two  $I's$ and two $N's$ come together
\item All $A's$ are not coming together
\item No two $A's$ are coming together
\end{enumerate}
%\input{exemplar/11/16/3/14/main.tex}
	\item One urn contains two black balls (labelled B1 and B2) and one white ball. A
	second urn contains one black ball and two white balls (labelled W1 and W2).
	Suppose the following experiment is performed. One of the two urns is chosen
	at random. Next a ball is randomly chosen from the urn. Then a second ball is
	chosen at random from the same urn without replacing the first ball.
	
	\begin{enumerate}
	\item What is the probability that two black balls are chosen?
	
	\item What is the probability that two balls of opposite colour are chosen?
	\end{enumerate}
	\solution
	%\input{exemplar/11/16/3/12/main1.tex}
\end{enumerate}

	\item 
The number lock of a suitcase has 4 wheels each labelled with ten digits i.e. from 0 to 9.The lock opens with a sequence of four digits with no repeats.What is the probability of a person getting the right sequence to open the suitcase.
\\
\solution
		%\begin{enumerate}[label=\thesection.\arabic*,ref=\thesection.\theenumi]
	\item One card is drawn from a well-shuffled deck of 52 cards. Find the probability of getting
\begin{enumerate}
\item A king of red colour 
\item A face card 
\item A red face card
\item The jack of hearts
\item A spade
\item The queen of diamonds

\end{enumerate}
\solution
		%\input{ncert/10/15/1/14/main.tex}
	\item Five cards—the ten, jack, queen, king and ace of diamonds, are well-shuffled with their face downwards. One card is then picked up at random.
\begin{enumerate}
\item
What is the probability that the card is the queen? 
\item
If the queen is drawn and put aside, what is the probability that the second card picked up is (a) an ace? (b) a queen?\\
\end{enumerate}
\solution
		%\input{ncert/10/15/1/15/defs.tex}
	\item A bag contains $5$ red balls and some blue balls. If the probability of drawing a blue ball is double that if a red ball, determine the number of blue balls in the bag. 
		\\
\solution
		%\input{ncert/10/15/2/3/defs.tex}
	\item A card is selected from a pack of 52 cards.
 \begin{enumerate}[label=(\alph*)] 
                 \item How many points are there in the sample space?
                 \item Calculate the probability that the card is an ace of spades.
                 \item Calculate the probability that the card is (i) an ace and (ii) black card.
 \end{enumerate}
\solution
		%\input{ncert/11/16/3/4/main.tex}
\item Four cards are drawn from a well-shuffled deck of 52 cards. What is the probability of obtaining 3 diamonds and one spade.
\\
\solution
		%\input{ncert/11/16/4/2/defs.tex}
\item In a certain lottery 10,000 tickets are sold and ten equal prizes are awarded. What is the probability of not getting a prize if you buy (a) one ticket (b) two tickets (c) 10 tickets ?	
\\
\solution
		%\input{ncert/11/16/4/4/defs.tex}
		%
\item 
Out of 100 students, two sections of 40 and 60 are formed. If you and your friend are among the 100 students, what is the probability that
\begin{enumerate}
\item you both enter the same section?
\item you both enter the different sections?
\end{enumerate}
\solution
		%\input{ncert/11/16/4/5/defs.tex}
	\item 
The number lock of a suitcase has 4 wheels each labelled with ten digits i.e. from 0 to 9.The lock opens with a sequence of four digits with no repeats.What is the probability of a person getting the right sequence to open the suitcase.
\\
\solution
		%\input{ncert/11/16/4/10/defs.tex}
		%
\item 
Two cards are drawn at random and without replacement from a pack of 52 playing cards. Find the probability that both the cards are black.
\\
\solution
		%\input{ncert/12/13/2/2/defs.tex}
		\item A box of oranges is inspected by examining three randomly selected oranges drawn without replacement. If all the three oranges are good, the box is approved for sale, otherwise, it is rejected. Find the probability that a box containing 15 oranges out of which 12 are good and 3 are bad ones will be approved for sale.
		\label{ncert/12/13/2/3/defs.tex}
		\item Two balls are drawn at random with replacement from a box containing 10 black and 8 red balls. Find the probability that
		\label{ncert/12/13/2/12}
\begin{enumerate}
\item both balls are red.
\item first ball is black and second is red.
\item one of them is black and other is red.
\end{enumerate}

\item In a hostel, 60\% of the students read Hindi newspaper, 40\% read English newspaper and 20\% read both Hindi and English newspapers. A student is selected at random.
		\label{ncert/12/13/2/15}
\begin{enumerate}
\item Find the probability that she reads neither Hindi nor English newspapers.
\item If she reads Hindi newspaper, find the probability that she reads English newspaper.
\item If she reads English newspaper, find the probability that she reads Hindi newspaper.\\
\end{enumerate}
\item The probability of obtaining an even prime number on each die, when a pair of dice is rolled is 
\begin{enumerate}
    \item $0$ 
    
    \item $\frac{1}{3}$ 
    
    \item $\frac{1}{12}$ 
    
    \item $\frac{1}{36}$ 
\end{enumerate}
\solution
		%\input{ncert/12/13/2/17/defs.tex}
	\item A bag contains 4 red and 4 black balls, another bag contains 2 red and 6 black balls. One of the two bags is selected at random and a ball is drawn from the bag which is found to be red. Find the probability that the ball is drawn from the first bag.
\\
\solution
		%\input{ncert/12/13/3/2/main.tex}
  \item
  Cards with numbers 2 to 101 are placed in a box. A card is selected at random.Find the probability that the card has
\begin{enumerate}[label=(\roman*)]
	\item an even number 
	\item a square number
\end{enumerate}
\solution
%\input{exemplar/10/13/3/32/main.tex}
\item
The king, queen and jack of clubs are removed from a deck of 52 playing cards and then well shuffled. Now one card is drawn at random from the remaining cards.  Determine the probability that the card is
\begin{enumerate}[label=(\roman*)]
\item a club
\item 10 of hearts
\end{enumerate}
\solution
%\input{exemplar/10/13/3/29/main.tex}
\item A team of medical students doing their internship have to assist during surgeries
at a city hospital. The probabilities of surgeries rated as very complex, complex,
routine, simple or very simple are respectively, 0.15, 0.20, 0.31, 0.26, .08. Find
the probabilities that a particular surgery will be rated
\begin{enumerate}
	\item complex or very complex;
	\item neither very complex nor very simple;
	\item routine or complex
	\item routine or simple
\end{enumerate}
\solution
%\input{exemplar/11/16/3/8(1)/main.tex}
\item A card is selected from a pack of 52 cards.
\begin{enumerate}[label=(\alph*)]
    \item How many points are there in the sample space?
    \item Calculate the probability that the card is an ace of spades.
    \item Calculate the probability that the card is (i) an ace and (ii) black card.
\end{enumerate}
\solution
%\input{exemplar/11/16/3/4/main2.tex}
\item The probability that a non leap year selected at random will contain 53 sundays.
\\
\solution
%\input{exemplar/10/13/1/19/main.tex}
\item One of the four persons John, Rita, Aslam or Gurpreet will be promoted next
month. Consequently the sample space consists of four elementary outcomes
S = {John promoted, Rita promoted, Aslam promoted, Gurpreet promoted}
You are told that the chances of John’s promotion is same as that of Gurpreet,
Rita’s chances of promotion are twice as likely as Johns. Aslam’s chances are
four times that of John.
\begin{enumerate}
	\item Determine
	\begin{enumerate}
		\item P (John promoted)
		\item P (Rita promoted)
		\item P (Aslam promoted)
		\item P (Gurpreet promoted)
	\end{enumerate}
	\item If A = {John promoted or Gurpreet promoted}, find P (A).
\end{enumerate}
\solution
%\input{exemplar/11/16/3/10/main.tex}
\item A card is drawn from a deck of 52 cards. Find the probability of getting a king or a heart or a red card.\\
\solution
%\input{exemplar/11/16/3/15/main.tex}
\item The probability that a student will pass his examination is 0.73, the probability of
the student getting a compartment is 0.13, and the probability that the student will
either pass or get compartment is 0.96. State True or False.\\
\solution
%\input{exemplar/11/16/3/31/main.tex}
\item A card is selected from a pack of 52 cards\\
\begin{enumerate}[label=(\alph*)]
\item How many points are there in the sample space?
\item Calculate the probability that the cards is an ace of spades.
\item Calculate the probability that the card is (i) an ace (ii)black card.\\
\end{enumerate}
%\input{ncert/11/16/3/4_1/Prob_4.tex}
\item In a non-leap year, the probability of having 53 tuesdays or 53 wednesdays is\\
\solution
%\input{exemplar/11/16/3/18/main.tex}
\item There are 1000 sealed envelopes in a box, 10 of them contain a cash prize of
Rs 100 each, 100 of them contain a cash prize of Rs 50 each and 200 of them
contain a cash prize of Rs 10 each and rest do not contain any cash prize. If they
are well shuffled and an envelope is picked up out, what is the probability that it
contains no cash prize?\\
\solution
%\input{exemplar/10/13/3/34/main.tex}
\item 
A die is thrown and a card is selected at random from a deck of 52 playing cards. The probability of getting an even number on the die and a spade card.\\
\solution
%\input{exemplar/12/13/3/78/main.tex}
\item
If 4-digit numbers greater than 5,000 are randomly formed from the digits 0, 1, 3, 5, and 7, what is the probability of forming a number divisible by 5 when:
\begin{enumerate}
    \item The digits are repeated?
    \item The repetition of digits is not allowed?
\end{enumerate}
\solution
%\input{ncert/11/16/4/9/main.tex}
\item Consider the probability space $\brak{\Omega, \mathcal{G}, P}$ where $\Omega = [0,2]$ and $\mathcal{G} = \cbrak{\phi, \Omega, [0,1], (1,2]}$. Let $X$ and $Y$ be two functions on $\Omega$ defined as
\begin{align*}
    X(\omega) = 
    \begin{cases}
        1 & \text{if }\omega \in [0, 1]\\
        2 & \text{if }\omega \in (1, 2]
    \end{cases}
\end{align*}
and
\begin{align*}
    Y(\omega) = 
    \begin{cases}
        2 & \text{if }\omega \in [0, 1.5]\\
        3 & \text{if }\omega \in (1.5, 2].
    \end{cases}
\end{align*}
Then which one of the following statements is true?
\begin{enumerate}
    \item [(A)] $X$ is a random variable with respect to $\mathcal{G}$, but $Y$ is not a random variable with respect to $\mathcal{G}$.
    \item [(B)] $Y$ is a random variable with respect to $\mathcal{G}$, but $X$ is not a random variable with respect to $\mathcal{G}$.
    \item [(C)] Neither $X$ nor $Y$ is a random variable with respect to $\mathcal{G}$.
    \item [(D)] Both $X$ and $Y$ are random variables with respect to $\mathcal{G}$.
\end{enumerate} \hfill (GATE ST 2023)\\
\solution
%\input{gate/ST/2023/14/main.tex}
	\item  A die is loaded in such a way that each odd number is twice as likely to occur as
each even number. Find $P(G)$, where $G$ is the event that a number greater than
3 occurs on a single roll of the die.
\\
\solution
		%\input{exemplar/11/16/3/5/main.tex}
	\item All the jacks, queens and kings are removed from a deck of 52 playing cards. The remaining cards are well shuffled and then one card is drawn at random. Giving ace a value 1 similar value for other cards, find the probability that the card has a value 
		\begin{enumerate}
			\item 7
			\item greater than 7
			\item less than 7
		\end{enumerate}
		%\input{exemplar/10/13/3/30/main.tex}
  \item A Lot consists of 48 mobile phones of which 42 are good, 3 have only minor defects and 3 have major defects.Varnika will buy a phone if it is good but the trader will only buy a mobile if it has no major defects. One phone is selected at random from the lot. What is the probability that it is
\begin{enumerate}
	\item acceptable to Varnika?
            \item acceptable to the trader?
\end{enumerate}
\solution
	%\input{exemplar/10/13/3/40/main.tex}
 \item A student says that if you throw a die, it will show up 1 or not 1. Therefore, the probability of getting 1 and the probability of getting 'not 1' each is equal to $\frac{1}{2}$. Is this correct? Give reasons.\\
 \solution
        %\input{exemplar/10/13/2/9/main.tex}
   \item Four candidates A, B, C, D have ap-
plied for the assignment to coach a school cricket
team. If A is twice as likely to be selected as B, and
B and C are given about the same chance of being
selected, while C is twice as likely to be selected
as D, what are the probabilities that
\begin{enumerate}
\item C will be selected?
\item A will not be selected?
\end{enumerate}
	%\input{exemplar/11/16/3/9/main.tex}
 \item A bag contain 24 balls of which $x$ balls are red, $2x$ are white and $3x$ are blue. A ball is selected at random, What is the probability that it is
\begin{enumerate}[label=\alph*)]
\item not red ?
\item white ?
\end{enumerate}
%\input{exemplar/10/13/3/41/main.tex}
If the letters of the word ASSASSINATION are arranged at random. Find the Probability that
\begin{enumerate}[label=(\alph*)]
\item Four $S's$ come consecutively in the word
\item Two  $I's$ and two $N's$ come together
\item All $A's$ are not coming together
\item No two $A's$ are coming together
\end{enumerate}
%\input{exemplar/11/16/3/14/main.tex}
	\item One urn contains two black balls (labelled B1 and B2) and one white ball. A
	second urn contains one black ball and two white balls (labelled W1 and W2).
	Suppose the following experiment is performed. One of the two urns is chosen
	at random. Next a ball is randomly chosen from the urn. Then a second ball is
	chosen at random from the same urn without replacing the first ball.
	
	\begin{enumerate}
	\item What is the probability that two black balls are chosen?
	
	\item What is the probability that two balls of opposite colour are chosen?
	\end{enumerate}
	\solution
	%\input{exemplar/11/16/3/12/main1.tex}
\end{enumerate}

		%
\item 
Two cards are drawn at random and without replacement from a pack of 52 playing cards. Find the probability that both the cards are black.
\\
\solution
		%\begin{enumerate}[label=\thesection.\arabic*,ref=\thesection.\theenumi]
	\item One card is drawn from a well-shuffled deck of 52 cards. Find the probability of getting
\begin{enumerate}
\item A king of red colour 
\item A face card 
\item A red face card
\item The jack of hearts
\item A spade
\item The queen of diamonds

\end{enumerate}
\solution
		%\input{ncert/10/15/1/14/main.tex}
	\item Five cards—the ten, jack, queen, king and ace of diamonds, are well-shuffled with their face downwards. One card is then picked up at random.
\begin{enumerate}
\item
What is the probability that the card is the queen? 
\item
If the queen is drawn and put aside, what is the probability that the second card picked up is (a) an ace? (b) a queen?\\
\end{enumerate}
\solution
		%\input{ncert/10/15/1/15/defs.tex}
	\item A bag contains $5$ red balls and some blue balls. If the probability of drawing a blue ball is double that if a red ball, determine the number of blue balls in the bag. 
		\\
\solution
		%\input{ncert/10/15/2/3/defs.tex}
	\item A card is selected from a pack of 52 cards.
 \begin{enumerate}[label=(\alph*)] 
                 \item How many points are there in the sample space?
                 \item Calculate the probability that the card is an ace of spades.
                 \item Calculate the probability that the card is (i) an ace and (ii) black card.
 \end{enumerate}
\solution
		%\input{ncert/11/16/3/4/main.tex}
\item Four cards are drawn from a well-shuffled deck of 52 cards. What is the probability of obtaining 3 diamonds and one spade.
\\
\solution
		%\input{ncert/11/16/4/2/defs.tex}
\item In a certain lottery 10,000 tickets are sold and ten equal prizes are awarded. What is the probability of not getting a prize if you buy (a) one ticket (b) two tickets (c) 10 tickets ?	
\\
\solution
		%\input{ncert/11/16/4/4/defs.tex}
		%
\item 
Out of 100 students, two sections of 40 and 60 are formed. If you and your friend are among the 100 students, what is the probability that
\begin{enumerate}
\item you both enter the same section?
\item you both enter the different sections?
\end{enumerate}
\solution
		%\input{ncert/11/16/4/5/defs.tex}
	\item 
The number lock of a suitcase has 4 wheels each labelled with ten digits i.e. from 0 to 9.The lock opens with a sequence of four digits with no repeats.What is the probability of a person getting the right sequence to open the suitcase.
\\
\solution
		%\input{ncert/11/16/4/10/defs.tex}
		%
\item 
Two cards are drawn at random and without replacement from a pack of 52 playing cards. Find the probability that both the cards are black.
\\
\solution
		%\input{ncert/12/13/2/2/defs.tex}
		\item A box of oranges is inspected by examining three randomly selected oranges drawn without replacement. If all the three oranges are good, the box is approved for sale, otherwise, it is rejected. Find the probability that a box containing 15 oranges out of which 12 are good and 3 are bad ones will be approved for sale.
		\label{ncert/12/13/2/3/defs.tex}
		\item Two balls are drawn at random with replacement from a box containing 10 black and 8 red balls. Find the probability that
		\label{ncert/12/13/2/12}
\begin{enumerate}
\item both balls are red.
\item first ball is black and second is red.
\item one of them is black and other is red.
\end{enumerate}

\item In a hostel, 60\% of the students read Hindi newspaper, 40\% read English newspaper and 20\% read both Hindi and English newspapers. A student is selected at random.
		\label{ncert/12/13/2/15}
\begin{enumerate}
\item Find the probability that she reads neither Hindi nor English newspapers.
\item If she reads Hindi newspaper, find the probability that she reads English newspaper.
\item If she reads English newspaper, find the probability that she reads Hindi newspaper.\\
\end{enumerate}
\item The probability of obtaining an even prime number on each die, when a pair of dice is rolled is 
\begin{enumerate}
    \item $0$ 
    
    \item $\frac{1}{3}$ 
    
    \item $\frac{1}{12}$ 
    
    \item $\frac{1}{36}$ 
\end{enumerate}
\solution
		%\input{ncert/12/13/2/17/defs.tex}
	\item A bag contains 4 red and 4 black balls, another bag contains 2 red and 6 black balls. One of the two bags is selected at random and a ball is drawn from the bag which is found to be red. Find the probability that the ball is drawn from the first bag.
\\
\solution
		%\input{ncert/12/13/3/2/main.tex}
  \item
  Cards with numbers 2 to 101 are placed in a box. A card is selected at random.Find the probability that the card has
\begin{enumerate}[label=(\roman*)]
	\item an even number 
	\item a square number
\end{enumerate}
\solution
%\input{exemplar/10/13/3/32/main.tex}
\item
The king, queen and jack of clubs are removed from a deck of 52 playing cards and then well shuffled. Now one card is drawn at random from the remaining cards.  Determine the probability that the card is
\begin{enumerate}[label=(\roman*)]
\item a club
\item 10 of hearts
\end{enumerate}
\solution
%\input{exemplar/10/13/3/29/main.tex}
\item A team of medical students doing their internship have to assist during surgeries
at a city hospital. The probabilities of surgeries rated as very complex, complex,
routine, simple or very simple are respectively, 0.15, 0.20, 0.31, 0.26, .08. Find
the probabilities that a particular surgery will be rated
\begin{enumerate}
	\item complex or very complex;
	\item neither very complex nor very simple;
	\item routine or complex
	\item routine or simple
\end{enumerate}
\solution
%\input{exemplar/11/16/3/8(1)/main.tex}
\item A card is selected from a pack of 52 cards.
\begin{enumerate}[label=(\alph*)]
    \item How many points are there in the sample space?
    \item Calculate the probability that the card is an ace of spades.
    \item Calculate the probability that the card is (i) an ace and (ii) black card.
\end{enumerate}
\solution
%\input{exemplar/11/16/3/4/main2.tex}
\item The probability that a non leap year selected at random will contain 53 sundays.
\\
\solution
%\input{exemplar/10/13/1/19/main.tex}
\item One of the four persons John, Rita, Aslam or Gurpreet will be promoted next
month. Consequently the sample space consists of four elementary outcomes
S = {John promoted, Rita promoted, Aslam promoted, Gurpreet promoted}
You are told that the chances of John’s promotion is same as that of Gurpreet,
Rita’s chances of promotion are twice as likely as Johns. Aslam’s chances are
four times that of John.
\begin{enumerate}
	\item Determine
	\begin{enumerate}
		\item P (John promoted)
		\item P (Rita promoted)
		\item P (Aslam promoted)
		\item P (Gurpreet promoted)
	\end{enumerate}
	\item If A = {John promoted or Gurpreet promoted}, find P (A).
\end{enumerate}
\solution
%\input{exemplar/11/16/3/10/main.tex}
\item A card is drawn from a deck of 52 cards. Find the probability of getting a king or a heart or a red card.\\
\solution
%\input{exemplar/11/16/3/15/main.tex}
\item The probability that a student will pass his examination is 0.73, the probability of
the student getting a compartment is 0.13, and the probability that the student will
either pass or get compartment is 0.96. State True or False.\\
\solution
%\input{exemplar/11/16/3/31/main.tex}
\item A card is selected from a pack of 52 cards\\
\begin{enumerate}[label=(\alph*)]
\item How many points are there in the sample space?
\item Calculate the probability that the cards is an ace of spades.
\item Calculate the probability that the card is (i) an ace (ii)black card.\\
\end{enumerate}
%\input{ncert/11/16/3/4_1/Prob_4.tex}
\item In a non-leap year, the probability of having 53 tuesdays or 53 wednesdays is\\
\solution
%\input{exemplar/11/16/3/18/main.tex}
\item There are 1000 sealed envelopes in a box, 10 of them contain a cash prize of
Rs 100 each, 100 of them contain a cash prize of Rs 50 each and 200 of them
contain a cash prize of Rs 10 each and rest do not contain any cash prize. If they
are well shuffled and an envelope is picked up out, what is the probability that it
contains no cash prize?\\
\solution
%\input{exemplar/10/13/3/34/main.tex}
\item 
A die is thrown and a card is selected at random from a deck of 52 playing cards. The probability of getting an even number on the die and a spade card.\\
\solution
%\input{exemplar/12/13/3/78/main.tex}
\item
If 4-digit numbers greater than 5,000 are randomly formed from the digits 0, 1, 3, 5, and 7, what is the probability of forming a number divisible by 5 when:
\begin{enumerate}
    \item The digits are repeated?
    \item The repetition of digits is not allowed?
\end{enumerate}
\solution
%\input{ncert/11/16/4/9/main.tex}
\item Consider the probability space $\brak{\Omega, \mathcal{G}, P}$ where $\Omega = [0,2]$ and $\mathcal{G} = \cbrak{\phi, \Omega, [0,1], (1,2]}$. Let $X$ and $Y$ be two functions on $\Omega$ defined as
\begin{align*}
    X(\omega) = 
    \begin{cases}
        1 & \text{if }\omega \in [0, 1]\\
        2 & \text{if }\omega \in (1, 2]
    \end{cases}
\end{align*}
and
\begin{align*}
    Y(\omega) = 
    \begin{cases}
        2 & \text{if }\omega \in [0, 1.5]\\
        3 & \text{if }\omega \in (1.5, 2].
    \end{cases}
\end{align*}
Then which one of the following statements is true?
\begin{enumerate}
    \item [(A)] $X$ is a random variable with respect to $\mathcal{G}$, but $Y$ is not a random variable with respect to $\mathcal{G}$.
    \item [(B)] $Y$ is a random variable with respect to $\mathcal{G}$, but $X$ is not a random variable with respect to $\mathcal{G}$.
    \item [(C)] Neither $X$ nor $Y$ is a random variable with respect to $\mathcal{G}$.
    \item [(D)] Both $X$ and $Y$ are random variables with respect to $\mathcal{G}$.
\end{enumerate} \hfill (GATE ST 2023)\\
\solution
%\input{gate/ST/2023/14/main.tex}
	\item  A die is loaded in such a way that each odd number is twice as likely to occur as
each even number. Find $P(G)$, where $G$ is the event that a number greater than
3 occurs on a single roll of the die.
\\
\solution
		%\input{exemplar/11/16/3/5/main.tex}
	\item All the jacks, queens and kings are removed from a deck of 52 playing cards. The remaining cards are well shuffled and then one card is drawn at random. Giving ace a value 1 similar value for other cards, find the probability that the card has a value 
		\begin{enumerate}
			\item 7
			\item greater than 7
			\item less than 7
		\end{enumerate}
		%\input{exemplar/10/13/3/30/main.tex}
  \item A Lot consists of 48 mobile phones of which 42 are good, 3 have only minor defects and 3 have major defects.Varnika will buy a phone if it is good but the trader will only buy a mobile if it has no major defects. One phone is selected at random from the lot. What is the probability that it is
\begin{enumerate}
	\item acceptable to Varnika?
            \item acceptable to the trader?
\end{enumerate}
\solution
	%\input{exemplar/10/13/3/40/main.tex}
 \item A student says that if you throw a die, it will show up 1 or not 1. Therefore, the probability of getting 1 and the probability of getting 'not 1' each is equal to $\frac{1}{2}$. Is this correct? Give reasons.\\
 \solution
        %\input{exemplar/10/13/2/9/main.tex}
   \item Four candidates A, B, C, D have ap-
plied for the assignment to coach a school cricket
team. If A is twice as likely to be selected as B, and
B and C are given about the same chance of being
selected, while C is twice as likely to be selected
as D, what are the probabilities that
\begin{enumerate}
\item C will be selected?
\item A will not be selected?
\end{enumerate}
	%\input{exemplar/11/16/3/9/main.tex}
 \item A bag contain 24 balls of which $x$ balls are red, $2x$ are white and $3x$ are blue. A ball is selected at random, What is the probability that it is
\begin{enumerate}[label=\alph*)]
\item not red ?
\item white ?
\end{enumerate}
%\input{exemplar/10/13/3/41/main.tex}
If the letters of the word ASSASSINATION are arranged at random. Find the Probability that
\begin{enumerate}[label=(\alph*)]
\item Four $S's$ come consecutively in the word
\item Two  $I's$ and two $N's$ come together
\item All $A's$ are not coming together
\item No two $A's$ are coming together
\end{enumerate}
%\input{exemplar/11/16/3/14/main.tex}
	\item One urn contains two black balls (labelled B1 and B2) and one white ball. A
	second urn contains one black ball and two white balls (labelled W1 and W2).
	Suppose the following experiment is performed. One of the two urns is chosen
	at random. Next a ball is randomly chosen from the urn. Then a second ball is
	chosen at random from the same urn without replacing the first ball.
	
	\begin{enumerate}
	\item What is the probability that two black balls are chosen?
	
	\item What is the probability that two balls of opposite colour are chosen?
	\end{enumerate}
	\solution
	%\input{exemplar/11/16/3/12/main1.tex}
\end{enumerate}

		\item A box of oranges is inspected by examining three randomly selected oranges drawn without replacement. If all the three oranges are good, the box is approved for sale, otherwise, it is rejected. Find the probability that a box containing 15 oranges out of which 12 are good and 3 are bad ones will be approved for sale.
		\label{ncert/12/13/2/3/defs.tex}
		\item Two balls are drawn at random with replacement from a box containing 10 black and 8 red balls. Find the probability that
		\label{ncert/12/13/2/12}
\begin{enumerate}
\item both balls are red.
\item first ball is black and second is red.
\item one of them is black and other is red.
\end{enumerate}

\item In a hostel, 60\% of the students read Hindi newspaper, 40\% read English newspaper and 20\% read both Hindi and English newspapers. A student is selected at random.
		\label{ncert/12/13/2/15}
\begin{enumerate}
\item Find the probability that she reads neither Hindi nor English newspapers.
\item If she reads Hindi newspaper, find the probability that she reads English newspaper.
\item If she reads English newspaper, find the probability that she reads Hindi newspaper.\\
\end{enumerate}
\item The probability of obtaining an even prime number on each die, when a pair of dice is rolled is 
\begin{enumerate}
    \item $0$ 
    
    \item $\frac{1}{3}$ 
    
    \item $\frac{1}{12}$ 
    
    \item $\frac{1}{36}$ 
\end{enumerate}
\solution
		%\begin{enumerate}[label=\thesection.\arabic*,ref=\thesection.\theenumi]
	\item One card is drawn from a well-shuffled deck of 52 cards. Find the probability of getting
\begin{enumerate}
\item A king of red colour 
\item A face card 
\item A red face card
\item The jack of hearts
\item A spade
\item The queen of diamonds

\end{enumerate}
\solution
		%\input{ncert/10/15/1/14/main.tex}
	\item Five cards—the ten, jack, queen, king and ace of diamonds, are well-shuffled with their face downwards. One card is then picked up at random.
\begin{enumerate}
\item
What is the probability that the card is the queen? 
\item
If the queen is drawn and put aside, what is the probability that the second card picked up is (a) an ace? (b) a queen?\\
\end{enumerate}
\solution
		%\input{ncert/10/15/1/15/defs.tex}
	\item A bag contains $5$ red balls and some blue balls. If the probability of drawing a blue ball is double that if a red ball, determine the number of blue balls in the bag. 
		\\
\solution
		%\input{ncert/10/15/2/3/defs.tex}
	\item A card is selected from a pack of 52 cards.
 \begin{enumerate}[label=(\alph*)] 
                 \item How many points are there in the sample space?
                 \item Calculate the probability that the card is an ace of spades.
                 \item Calculate the probability that the card is (i) an ace and (ii) black card.
 \end{enumerate}
\solution
		%\input{ncert/11/16/3/4/main.tex}
\item Four cards are drawn from a well-shuffled deck of 52 cards. What is the probability of obtaining 3 diamonds and one spade.
\\
\solution
		%\input{ncert/11/16/4/2/defs.tex}
\item In a certain lottery 10,000 tickets are sold and ten equal prizes are awarded. What is the probability of not getting a prize if you buy (a) one ticket (b) two tickets (c) 10 tickets ?	
\\
\solution
		%\input{ncert/11/16/4/4/defs.tex}
		%
\item 
Out of 100 students, two sections of 40 and 60 are formed. If you and your friend are among the 100 students, what is the probability that
\begin{enumerate}
\item you both enter the same section?
\item you both enter the different sections?
\end{enumerate}
\solution
		%\input{ncert/11/16/4/5/defs.tex}
	\item 
The number lock of a suitcase has 4 wheels each labelled with ten digits i.e. from 0 to 9.The lock opens with a sequence of four digits with no repeats.What is the probability of a person getting the right sequence to open the suitcase.
\\
\solution
		%\input{ncert/11/16/4/10/defs.tex}
		%
\item 
Two cards are drawn at random and without replacement from a pack of 52 playing cards. Find the probability that both the cards are black.
\\
\solution
		%\input{ncert/12/13/2/2/defs.tex}
		\item A box of oranges is inspected by examining three randomly selected oranges drawn without replacement. If all the three oranges are good, the box is approved for sale, otherwise, it is rejected. Find the probability that a box containing 15 oranges out of which 12 are good and 3 are bad ones will be approved for sale.
		\label{ncert/12/13/2/3/defs.tex}
		\item Two balls are drawn at random with replacement from a box containing 10 black and 8 red balls. Find the probability that
		\label{ncert/12/13/2/12}
\begin{enumerate}
\item both balls are red.
\item first ball is black and second is red.
\item one of them is black and other is red.
\end{enumerate}

\item In a hostel, 60\% of the students read Hindi newspaper, 40\% read English newspaper and 20\% read both Hindi and English newspapers. A student is selected at random.
		\label{ncert/12/13/2/15}
\begin{enumerate}
\item Find the probability that she reads neither Hindi nor English newspapers.
\item If she reads Hindi newspaper, find the probability that she reads English newspaper.
\item If she reads English newspaper, find the probability that she reads Hindi newspaper.\\
\end{enumerate}
\item The probability of obtaining an even prime number on each die, when a pair of dice is rolled is 
\begin{enumerate}
    \item $0$ 
    
    \item $\frac{1}{3}$ 
    
    \item $\frac{1}{12}$ 
    
    \item $\frac{1}{36}$ 
\end{enumerate}
\solution
		%\input{ncert/12/13/2/17/defs.tex}
	\item A bag contains 4 red and 4 black balls, another bag contains 2 red and 6 black balls. One of the two bags is selected at random and a ball is drawn from the bag which is found to be red. Find the probability that the ball is drawn from the first bag.
\\
\solution
		%\input{ncert/12/13/3/2/main.tex}
  \item
  Cards with numbers 2 to 101 are placed in a box. A card is selected at random.Find the probability that the card has
\begin{enumerate}[label=(\roman*)]
	\item an even number 
	\item a square number
\end{enumerate}
\solution
%\input{exemplar/10/13/3/32/main.tex}
\item
The king, queen and jack of clubs are removed from a deck of 52 playing cards and then well shuffled. Now one card is drawn at random from the remaining cards.  Determine the probability that the card is
\begin{enumerate}[label=(\roman*)]
\item a club
\item 10 of hearts
\end{enumerate}
\solution
%\input{exemplar/10/13/3/29/main.tex}
\item A team of medical students doing their internship have to assist during surgeries
at a city hospital. The probabilities of surgeries rated as very complex, complex,
routine, simple or very simple are respectively, 0.15, 0.20, 0.31, 0.26, .08. Find
the probabilities that a particular surgery will be rated
\begin{enumerate}
	\item complex or very complex;
	\item neither very complex nor very simple;
	\item routine or complex
	\item routine or simple
\end{enumerate}
\solution
%\input{exemplar/11/16/3/8(1)/main.tex}
\item A card is selected from a pack of 52 cards.
\begin{enumerate}[label=(\alph*)]
    \item How many points are there in the sample space?
    \item Calculate the probability that the card is an ace of spades.
    \item Calculate the probability that the card is (i) an ace and (ii) black card.
\end{enumerate}
\solution
%\input{exemplar/11/16/3/4/main2.tex}
\item The probability that a non leap year selected at random will contain 53 sundays.
\\
\solution
%\input{exemplar/10/13/1/19/main.tex}
\item One of the four persons John, Rita, Aslam or Gurpreet will be promoted next
month. Consequently the sample space consists of four elementary outcomes
S = {John promoted, Rita promoted, Aslam promoted, Gurpreet promoted}
You are told that the chances of John’s promotion is same as that of Gurpreet,
Rita’s chances of promotion are twice as likely as Johns. Aslam’s chances are
four times that of John.
\begin{enumerate}
	\item Determine
	\begin{enumerate}
		\item P (John promoted)
		\item P (Rita promoted)
		\item P (Aslam promoted)
		\item P (Gurpreet promoted)
	\end{enumerate}
	\item If A = {John promoted or Gurpreet promoted}, find P (A).
\end{enumerate}
\solution
%\input{exemplar/11/16/3/10/main.tex}
\item A card is drawn from a deck of 52 cards. Find the probability of getting a king or a heart or a red card.\\
\solution
%\input{exemplar/11/16/3/15/main.tex}
\item The probability that a student will pass his examination is 0.73, the probability of
the student getting a compartment is 0.13, and the probability that the student will
either pass or get compartment is 0.96. State True or False.\\
\solution
%\input{exemplar/11/16/3/31/main.tex}
\item A card is selected from a pack of 52 cards\\
\begin{enumerate}[label=(\alph*)]
\item How many points are there in the sample space?
\item Calculate the probability that the cards is an ace of spades.
\item Calculate the probability that the card is (i) an ace (ii)black card.\\
\end{enumerate}
%\input{ncert/11/16/3/4_1/Prob_4.tex}
\item In a non-leap year, the probability of having 53 tuesdays or 53 wednesdays is\\
\solution
%\input{exemplar/11/16/3/18/main.tex}
\item There are 1000 sealed envelopes in a box, 10 of them contain a cash prize of
Rs 100 each, 100 of them contain a cash prize of Rs 50 each and 200 of them
contain a cash prize of Rs 10 each and rest do not contain any cash prize. If they
are well shuffled and an envelope is picked up out, what is the probability that it
contains no cash prize?\\
\solution
%\input{exemplar/10/13/3/34/main.tex}
\item 
A die is thrown and a card is selected at random from a deck of 52 playing cards. The probability of getting an even number on the die and a spade card.\\
\solution
%\input{exemplar/12/13/3/78/main.tex}
\item
If 4-digit numbers greater than 5,000 are randomly formed from the digits 0, 1, 3, 5, and 7, what is the probability of forming a number divisible by 5 when:
\begin{enumerate}
    \item The digits are repeated?
    \item The repetition of digits is not allowed?
\end{enumerate}
\solution
%\input{ncert/11/16/4/9/main.tex}
\item Consider the probability space $\brak{\Omega, \mathcal{G}, P}$ where $\Omega = [0,2]$ and $\mathcal{G} = \cbrak{\phi, \Omega, [0,1], (1,2]}$. Let $X$ and $Y$ be two functions on $\Omega$ defined as
\begin{align*}
    X(\omega) = 
    \begin{cases}
        1 & \text{if }\omega \in [0, 1]\\
        2 & \text{if }\omega \in (1, 2]
    \end{cases}
\end{align*}
and
\begin{align*}
    Y(\omega) = 
    \begin{cases}
        2 & \text{if }\omega \in [0, 1.5]\\
        3 & \text{if }\omega \in (1.5, 2].
    \end{cases}
\end{align*}
Then which one of the following statements is true?
\begin{enumerate}
    \item [(A)] $X$ is a random variable with respect to $\mathcal{G}$, but $Y$ is not a random variable with respect to $\mathcal{G}$.
    \item [(B)] $Y$ is a random variable with respect to $\mathcal{G}$, but $X$ is not a random variable with respect to $\mathcal{G}$.
    \item [(C)] Neither $X$ nor $Y$ is a random variable with respect to $\mathcal{G}$.
    \item [(D)] Both $X$ and $Y$ are random variables with respect to $\mathcal{G}$.
\end{enumerate} \hfill (GATE ST 2023)\\
\solution
%\input{gate/ST/2023/14/main.tex}
	\item  A die is loaded in such a way that each odd number is twice as likely to occur as
each even number. Find $P(G)$, where $G$ is the event that a number greater than
3 occurs on a single roll of the die.
\\
\solution
		%\input{exemplar/11/16/3/5/main.tex}
	\item All the jacks, queens and kings are removed from a deck of 52 playing cards. The remaining cards are well shuffled and then one card is drawn at random. Giving ace a value 1 similar value for other cards, find the probability that the card has a value 
		\begin{enumerate}
			\item 7
			\item greater than 7
			\item less than 7
		\end{enumerate}
		%\input{exemplar/10/13/3/30/main.tex}
  \item A Lot consists of 48 mobile phones of which 42 are good, 3 have only minor defects and 3 have major defects.Varnika will buy a phone if it is good but the trader will only buy a mobile if it has no major defects. One phone is selected at random from the lot. What is the probability that it is
\begin{enumerate}
	\item acceptable to Varnika?
            \item acceptable to the trader?
\end{enumerate}
\solution
	%\input{exemplar/10/13/3/40/main.tex}
 \item A student says that if you throw a die, it will show up 1 or not 1. Therefore, the probability of getting 1 and the probability of getting 'not 1' each is equal to $\frac{1}{2}$. Is this correct? Give reasons.\\
 \solution
        %\input{exemplar/10/13/2/9/main.tex}
   \item Four candidates A, B, C, D have ap-
plied for the assignment to coach a school cricket
team. If A is twice as likely to be selected as B, and
B and C are given about the same chance of being
selected, while C is twice as likely to be selected
as D, what are the probabilities that
\begin{enumerate}
\item C will be selected?
\item A will not be selected?
\end{enumerate}
	%\input{exemplar/11/16/3/9/main.tex}
 \item A bag contain 24 balls of which $x$ balls are red, $2x$ are white and $3x$ are blue. A ball is selected at random, What is the probability that it is
\begin{enumerate}[label=\alph*)]
\item not red ?
\item white ?
\end{enumerate}
%\input{exemplar/10/13/3/41/main.tex}
If the letters of the word ASSASSINATION are arranged at random. Find the Probability that
\begin{enumerate}[label=(\alph*)]
\item Four $S's$ come consecutively in the word
\item Two  $I's$ and two $N's$ come together
\item All $A's$ are not coming together
\item No two $A's$ are coming together
\end{enumerate}
%\input{exemplar/11/16/3/14/main.tex}
	\item One urn contains two black balls (labelled B1 and B2) and one white ball. A
	second urn contains one black ball and two white balls (labelled W1 and W2).
	Suppose the following experiment is performed. One of the two urns is chosen
	at random. Next a ball is randomly chosen from the urn. Then a second ball is
	chosen at random from the same urn without replacing the first ball.
	
	\begin{enumerate}
	\item What is the probability that two black balls are chosen?
	
	\item What is the probability that two balls of opposite colour are chosen?
	\end{enumerate}
	\solution
	%\input{exemplar/11/16/3/12/main1.tex}
\end{enumerate}

	\item A bag contains 4 red and 4 black balls, another bag contains 2 red and 6 black balls. One of the two bags is selected at random and a ball is drawn from the bag which is found to be red. Find the probability that the ball is drawn from the first bag.
\\
\solution
		%\begin{table}[H]
	\centering
\begin{tabular}{|c|c|c|}
\hline
Random variable &Value &Definition\\ \hline
\multirow{3}{*}{X} &0 &Slips of Rs 1\\
&1 &Slips of Rs 5\\
&2 &Slips of Rs 13\\ \hline
\multirow{2}{*}{Y} &0 &Box A\\
&1 &Box B\\\hline
\end{tabular}
\caption{}
\label{tab:Distribution}
\end{table}
See \tabref{tab:Distribution}.
\begin{align}
p_{Y}\brak{k}= \begin{cases} 
      \frac{1}{3} & {k=0} \\
      \frac{2}{3 }& {k=1} 
   \end{cases}
   \\
p_{Y|X}\brak{0|0} = \frac{19}{25}\, 
p_{Y|X}\brak{0|1} = \frac{6}{25}\,
p_{Y|X}\brak{1|0} = \frac{45}{50}\,
p_{Y|X}\brak{1|2} = \frac{5}{50}
\end{align}
The desired probability is the probability that a slip drawn at random is marked other than Rs 1,
\begin{align}
&=1-p_X\brak{0}\\
&= p_X(1) + p_X(2)
\end{align}
Using Bayes theorem,
\begin{align}
&= p_Y\brak{0} \times \pr{Y=0 | X=1} + p_Y\brak{1} \times \pr{Y=1|X=2}\\
&=\frac{1}{3} \times \frac{6}{25} + \frac{2}{3} \times \frac{5}{50}\\
&=\frac{11}{75}
\end{align}

\newpage

%\tableofcontents

\bigskip

\renewcommand{\thefigure}{\theenumi}
\renewcommand{\thetable}{\theenumi}
%\renewcommand{\theequation}{\theenumi}

%\begin{abstract}
%%\boldmath
%In this letter, an algorithm for evaluating the exact analytical bit error rate  (BER)  for the piecewise linear (PL) combiner for  multiple relays is presented. Previous results were available only for upto three relays. The algorithm is unique in the sense that  the actual mathematical expressions, that are prohibitively large, need not be explicitly obtained. The diversity gain due to multiple relays is shown through plots of the analytical BER, well supported by simulations. 
%
%\end{abstract}
% IEEEtran.cls defaults to using nonbold math in the Abstract.
% This preserves the distinction between vectors and scalars. However,
% if the journal you are submitting to favors bold math in the abstract,
% then you can use LaTeX's standard command \boldmath at the very start
% of the abstract to achieve this. Many IEEE journals frown on math
% in the abstract anyway.

% Note that keywords are not normally used for peerreview papers.
%\begin{IEEEkeywords}
%Cooperative diversity, decode and forward, piecewise linear
%\end{IEEEkeywords}



% For peer review papers, you can put extra information on the cover
% page as needed:
% \ifCLASSOPTIONpeerreview
% \begin{center} \bfseries EDICS Category: 3-BBND \end{center}
% \fi
%
% For peerreview papers, this IEEEtran command inserts a page break and
% creates the second title. It will be ignored for other modes.
%\IEEEpeerreviewmaketitle




  \item
  Cards with numbers 2 to 101 are placed in a box. A card is selected at random.Find the probability that the card has
\begin{enumerate}[label=(\roman*)]
	\item an even number 
	\item a square number
\end{enumerate}
\solution
%\begin{table}[H]
	\centering
\begin{tabular}{|c|c|c|}
\hline
Random variable &Value &Definition\\ \hline
\multirow{3}{*}{X} &0 &Slips of Rs 1\\
&1 &Slips of Rs 5\\
&2 &Slips of Rs 13\\ \hline
\multirow{2}{*}{Y} &0 &Box A\\
&1 &Box B\\\hline
\end{tabular}
\caption{}
\label{tab:Distribution}
\end{table}
See \tabref{tab:Distribution}.
\begin{align}
p_{Y}\brak{k}= \begin{cases} 
      \frac{1}{3} & {k=0} \\
      \frac{2}{3 }& {k=1} 
   \end{cases}
   \\
p_{Y|X}\brak{0|0} = \frac{19}{25}\, 
p_{Y|X}\brak{0|1} = \frac{6}{25}\,
p_{Y|X}\brak{1|0} = \frac{45}{50}\,
p_{Y|X}\brak{1|2} = \frac{5}{50}
\end{align}
The desired probability is the probability that a slip drawn at random is marked other than Rs 1,
\begin{align}
&=1-p_X\brak{0}\\
&= p_X(1) + p_X(2)
\end{align}
Using Bayes theorem,
\begin{align}
&= p_Y\brak{0} \times \pr{Y=0 | X=1} + p_Y\brak{1} \times \pr{Y=1|X=2}\\
&=\frac{1}{3} \times \frac{6}{25} + \frac{2}{3} \times \frac{5}{50}\\
&=\frac{11}{75}
\end{align}

\newpage

%\tableofcontents

\bigskip

\renewcommand{\thefigure}{\theenumi}
\renewcommand{\thetable}{\theenumi}
%\renewcommand{\theequation}{\theenumi}

%\begin{abstract}
%%\boldmath
%In this letter, an algorithm for evaluating the exact analytical bit error rate  (BER)  for the piecewise linear (PL) combiner for  multiple relays is presented. Previous results were available only for upto three relays. The algorithm is unique in the sense that  the actual mathematical expressions, that are prohibitively large, need not be explicitly obtained. The diversity gain due to multiple relays is shown through plots of the analytical BER, well supported by simulations. 
%
%\end{abstract}
% IEEEtran.cls defaults to using nonbold math in the Abstract.
% This preserves the distinction between vectors and scalars. However,
% if the journal you are submitting to favors bold math in the abstract,
% then you can use LaTeX's standard command \boldmath at the very start
% of the abstract to achieve this. Many IEEE journals frown on math
% in the abstract anyway.

% Note that keywords are not normally used for peerreview papers.
%\begin{IEEEkeywords}
%Cooperative diversity, decode and forward, piecewise linear
%\end{IEEEkeywords}



% For peer review papers, you can put extra information on the cover
% page as needed:
% \ifCLASSOPTIONpeerreview
% \begin{center} \bfseries EDICS Category: 3-BBND \end{center}
% \fi
%
% For peerreview papers, this IEEEtran command inserts a page break and
% creates the second title. It will be ignored for other modes.
%\IEEEpeerreviewmaketitle




\item
The king, queen and jack of clubs are removed from a deck of 52 playing cards and then well shuffled. Now one card is drawn at random from the remaining cards.  Determine the probability that the card is
\begin{enumerate}[label=(\roman*)]
\item a club
\item 10 of hearts
\end{enumerate}
\solution
%\begin{table}[H]
	\centering
\begin{tabular}{|c|c|c|}
\hline
Random variable &Value &Definition\\ \hline
\multirow{3}{*}{X} &0 &Slips of Rs 1\\
&1 &Slips of Rs 5\\
&2 &Slips of Rs 13\\ \hline
\multirow{2}{*}{Y} &0 &Box A\\
&1 &Box B\\\hline
\end{tabular}
\caption{}
\label{tab:Distribution}
\end{table}
See \tabref{tab:Distribution}.
\begin{align}
p_{Y}\brak{k}= \begin{cases} 
      \frac{1}{3} & {k=0} \\
      \frac{2}{3 }& {k=1} 
   \end{cases}
   \\
p_{Y|X}\brak{0|0} = \frac{19}{25}\, 
p_{Y|X}\brak{0|1} = \frac{6}{25}\,
p_{Y|X}\brak{1|0} = \frac{45}{50}\,
p_{Y|X}\brak{1|2} = \frac{5}{50}
\end{align}
The desired probability is the probability that a slip drawn at random is marked other than Rs 1,
\begin{align}
&=1-p_X\brak{0}\\
&= p_X(1) + p_X(2)
\end{align}
Using Bayes theorem,
\begin{align}
&= p_Y\brak{0} \times \pr{Y=0 | X=1} + p_Y\brak{1} \times \pr{Y=1|X=2}\\
&=\frac{1}{3} \times \frac{6}{25} + \frac{2}{3} \times \frac{5}{50}\\
&=\frac{11}{75}
\end{align}

\newpage

%\tableofcontents

\bigskip

\renewcommand{\thefigure}{\theenumi}
\renewcommand{\thetable}{\theenumi}
%\renewcommand{\theequation}{\theenumi}

%\begin{abstract}
%%\boldmath
%In this letter, an algorithm for evaluating the exact analytical bit error rate  (BER)  for the piecewise linear (PL) combiner for  multiple relays is presented. Previous results were available only for upto three relays. The algorithm is unique in the sense that  the actual mathematical expressions, that are prohibitively large, need not be explicitly obtained. The diversity gain due to multiple relays is shown through plots of the analytical BER, well supported by simulations. 
%
%\end{abstract}
% IEEEtran.cls defaults to using nonbold math in the Abstract.
% This preserves the distinction between vectors and scalars. However,
% if the journal you are submitting to favors bold math in the abstract,
% then you can use LaTeX's standard command \boldmath at the very start
% of the abstract to achieve this. Many IEEE journals frown on math
% in the abstract anyway.

% Note that keywords are not normally used for peerreview papers.
%\begin{IEEEkeywords}
%Cooperative diversity, decode and forward, piecewise linear
%\end{IEEEkeywords}



% For peer review papers, you can put extra information on the cover
% page as needed:
% \ifCLASSOPTIONpeerreview
% \begin{center} \bfseries EDICS Category: 3-BBND \end{center}
% \fi
%
% For peerreview papers, this IEEEtran command inserts a page break and
% creates the second title. It will be ignored for other modes.
%\IEEEpeerreviewmaketitle




\item A team of medical students doing their internship have to assist during surgeries
at a city hospital. The probabilities of surgeries rated as very complex, complex,
routine, simple or very simple are respectively, 0.15, 0.20, 0.31, 0.26, .08. Find
the probabilities that a particular surgery will be rated
\begin{enumerate}
	\item complex or very complex;
	\item neither very complex nor very simple;
	\item routine or complex
	\item routine or simple
\end{enumerate}
\solution
%\begin{table}[H]
	\centering
\begin{tabular}{|c|c|c|}
\hline
Random variable &Value &Definition\\ \hline
\multirow{3}{*}{X} &0 &Slips of Rs 1\\
&1 &Slips of Rs 5\\
&2 &Slips of Rs 13\\ \hline
\multirow{2}{*}{Y} &0 &Box A\\
&1 &Box B\\\hline
\end{tabular}
\caption{}
\label{tab:Distribution}
\end{table}
See \tabref{tab:Distribution}.
\begin{align}
p_{Y}\brak{k}= \begin{cases} 
      \frac{1}{3} & {k=0} \\
      \frac{2}{3 }& {k=1} 
   \end{cases}
   \\
p_{Y|X}\brak{0|0} = \frac{19}{25}\, 
p_{Y|X}\brak{0|1} = \frac{6}{25}\,
p_{Y|X}\brak{1|0} = \frac{45}{50}\,
p_{Y|X}\brak{1|2} = \frac{5}{50}
\end{align}
The desired probability is the probability that a slip drawn at random is marked other than Rs 1,
\begin{align}
&=1-p_X\brak{0}\\
&= p_X(1) + p_X(2)
\end{align}
Using Bayes theorem,
\begin{align}
&= p_Y\brak{0} \times \pr{Y=0 | X=1} + p_Y\brak{1} \times \pr{Y=1|X=2}\\
&=\frac{1}{3} \times \frac{6}{25} + \frac{2}{3} \times \frac{5}{50}\\
&=\frac{11}{75}
\end{align}

\newpage

%\tableofcontents

\bigskip

\renewcommand{\thefigure}{\theenumi}
\renewcommand{\thetable}{\theenumi}
%\renewcommand{\theequation}{\theenumi}

%\begin{abstract}
%%\boldmath
%In this letter, an algorithm for evaluating the exact analytical bit error rate  (BER)  for the piecewise linear (PL) combiner for  multiple relays is presented. Previous results were available only for upto three relays. The algorithm is unique in the sense that  the actual mathematical expressions, that are prohibitively large, need not be explicitly obtained. The diversity gain due to multiple relays is shown through plots of the analytical BER, well supported by simulations. 
%
%\end{abstract}
% IEEEtran.cls defaults to using nonbold math in the Abstract.
% This preserves the distinction between vectors and scalars. However,
% if the journal you are submitting to favors bold math in the abstract,
% then you can use LaTeX's standard command \boldmath at the very start
% of the abstract to achieve this. Many IEEE journals frown on math
% in the abstract anyway.

% Note that keywords are not normally used for peerreview papers.
%\begin{IEEEkeywords}
%Cooperative diversity, decode and forward, piecewise linear
%\end{IEEEkeywords}



% For peer review papers, you can put extra information on the cover
% page as needed:
% \ifCLASSOPTIONpeerreview
% \begin{center} \bfseries EDICS Category: 3-BBND \end{center}
% \fi
%
% For peerreview papers, this IEEEtran command inserts a page break and
% creates the second title. It will be ignored for other modes.
%\IEEEpeerreviewmaketitle




\item A card is selected from a pack of 52 cards.
\begin{enumerate}[label=(\alph*)]
    \item How many points are there in the sample space?
    \item Calculate the probability that the card is an ace of spades.
    \item Calculate the probability that the card is (i) an ace and (ii) black card.
\end{enumerate}
\solution
%Let $X$ be an bernoulli rv defined as in \tabref{tab:exemplar/11/16/3/26}.  Then, 
\begin{equation}
    p =
        \frac{4}{11} 
\end{equation}
\begin{table}[H]
	\centering
	\input{exemplar/11/16/3/26/tables/Table2.tex}
	\caption{}
        \label{tab:exemplar/11/16/3/26}
\end{table}

\item The probability that a non leap year selected at random will contain 53 sundays.
\\
\solution
%\begin{table}[H]
	\centering
\begin{tabular}{|c|c|c|}
\hline
Random variable &Value &Definition\\ \hline
\multirow{3}{*}{X} &0 &Slips of Rs 1\\
&1 &Slips of Rs 5\\
&2 &Slips of Rs 13\\ \hline
\multirow{2}{*}{Y} &0 &Box A\\
&1 &Box B\\\hline
\end{tabular}
\caption{}
\label{tab:Distribution}
\end{table}
See \tabref{tab:Distribution}.
\begin{align}
p_{Y}\brak{k}= \begin{cases} 
      \frac{1}{3} & {k=0} \\
      \frac{2}{3 }& {k=1} 
   \end{cases}
   \\
p_{Y|X}\brak{0|0} = \frac{19}{25}\, 
p_{Y|X}\brak{0|1} = \frac{6}{25}\,
p_{Y|X}\brak{1|0} = \frac{45}{50}\,
p_{Y|X}\brak{1|2} = \frac{5}{50}
\end{align}
The desired probability is the probability that a slip drawn at random is marked other than Rs 1,
\begin{align}
&=1-p_X\brak{0}\\
&= p_X(1) + p_X(2)
\end{align}
Using Bayes theorem,
\begin{align}
&= p_Y\brak{0} \times \pr{Y=0 | X=1} + p_Y\brak{1} \times \pr{Y=1|X=2}\\
&=\frac{1}{3} \times \frac{6}{25} + \frac{2}{3} \times \frac{5}{50}\\
&=\frac{11}{75}
\end{align}

\newpage

%\tableofcontents

\bigskip

\renewcommand{\thefigure}{\theenumi}
\renewcommand{\thetable}{\theenumi}
%\renewcommand{\theequation}{\theenumi}

%\begin{abstract}
%%\boldmath
%In this letter, an algorithm for evaluating the exact analytical bit error rate  (BER)  for the piecewise linear (PL) combiner for  multiple relays is presented. Previous results were available only for upto three relays. The algorithm is unique in the sense that  the actual mathematical expressions, that are prohibitively large, need not be explicitly obtained. The diversity gain due to multiple relays is shown through plots of the analytical BER, well supported by simulations. 
%
%\end{abstract}
% IEEEtran.cls defaults to using nonbold math in the Abstract.
% This preserves the distinction between vectors and scalars. However,
% if the journal you are submitting to favors bold math in the abstract,
% then you can use LaTeX's standard command \boldmath at the very start
% of the abstract to achieve this. Many IEEE journals frown on math
% in the abstract anyway.

% Note that keywords are not normally used for peerreview papers.
%\begin{IEEEkeywords}
%Cooperative diversity, decode and forward, piecewise linear
%\end{IEEEkeywords}



% For peer review papers, you can put extra information on the cover
% page as needed:
% \ifCLASSOPTIONpeerreview
% \begin{center} \bfseries EDICS Category: 3-BBND \end{center}
% \fi
%
% For peerreview papers, this IEEEtran command inserts a page break and
% creates the second title. It will be ignored for other modes.
%\IEEEpeerreviewmaketitle




\item One of the four persons John, Rita, Aslam or Gurpreet will be promoted next
month. Consequently the sample space consists of four elementary outcomes
S = {John promoted, Rita promoted, Aslam promoted, Gurpreet promoted}
You are told that the chances of John’s promotion is same as that of Gurpreet,
Rita’s chances of promotion are twice as likely as Johns. Aslam’s chances are
four times that of John.
\begin{enumerate}
	\item Determine
	\begin{enumerate}
		\item P (John promoted)
		\item P (Rita promoted)
		\item P (Aslam promoted)
		\item P (Gurpreet promoted)
	\end{enumerate}
	\item If A = {John promoted or Gurpreet promoted}, find P (A).
\end{enumerate}
\solution
%\begin{table}[H]
	\centering
\begin{tabular}{|c|c|c|}
\hline
Random variable &Value &Definition\\ \hline
\multirow{3}{*}{X} &0 &Slips of Rs 1\\
&1 &Slips of Rs 5\\
&2 &Slips of Rs 13\\ \hline
\multirow{2}{*}{Y} &0 &Box A\\
&1 &Box B\\\hline
\end{tabular}
\caption{}
\label{tab:Distribution}
\end{table}
See \tabref{tab:Distribution}.
\begin{align}
p_{Y}\brak{k}= \begin{cases} 
      \frac{1}{3} & {k=0} \\
      \frac{2}{3 }& {k=1} 
   \end{cases}
   \\
p_{Y|X}\brak{0|0} = \frac{19}{25}\, 
p_{Y|X}\brak{0|1} = \frac{6}{25}\,
p_{Y|X}\brak{1|0} = \frac{45}{50}\,
p_{Y|X}\brak{1|2} = \frac{5}{50}
\end{align}
The desired probability is the probability that a slip drawn at random is marked other than Rs 1,
\begin{align}
&=1-p_X\brak{0}\\
&= p_X(1) + p_X(2)
\end{align}
Using Bayes theorem,
\begin{align}
&= p_Y\brak{0} \times \pr{Y=0 | X=1} + p_Y\brak{1} \times \pr{Y=1|X=2}\\
&=\frac{1}{3} \times \frac{6}{25} + \frac{2}{3} \times \frac{5}{50}\\
&=\frac{11}{75}
\end{align}

\newpage

%\tableofcontents

\bigskip

\renewcommand{\thefigure}{\theenumi}
\renewcommand{\thetable}{\theenumi}
%\renewcommand{\theequation}{\theenumi}

%\begin{abstract}
%%\boldmath
%In this letter, an algorithm for evaluating the exact analytical bit error rate  (BER)  for the piecewise linear (PL) combiner for  multiple relays is presented. Previous results were available only for upto three relays. The algorithm is unique in the sense that  the actual mathematical expressions, that are prohibitively large, need not be explicitly obtained. The diversity gain due to multiple relays is shown through plots of the analytical BER, well supported by simulations. 
%
%\end{abstract}
% IEEEtran.cls defaults to using nonbold math in the Abstract.
% This preserves the distinction between vectors and scalars. However,
% if the journal you are submitting to favors bold math in the abstract,
% then you can use LaTeX's standard command \boldmath at the very start
% of the abstract to achieve this. Many IEEE journals frown on math
% in the abstract anyway.

% Note that keywords are not normally used for peerreview papers.
%\begin{IEEEkeywords}
%Cooperative diversity, decode and forward, piecewise linear
%\end{IEEEkeywords}



% For peer review papers, you can put extra information on the cover
% page as needed:
% \ifCLASSOPTIONpeerreview
% \begin{center} \bfseries EDICS Category: 3-BBND \end{center}
% \fi
%
% For peerreview papers, this IEEEtran command inserts a page break and
% creates the second title. It will be ignored for other modes.
%\IEEEpeerreviewmaketitle




\item A card is drawn from a deck of 52 cards. Find the probability of getting a king or a heart or a red card.\\
\solution
%\begin{table}[H]
	\centering
\begin{tabular}{|c|c|c|}
\hline
Random variable &Value &Definition\\ \hline
\multirow{3}{*}{X} &0 &Slips of Rs 1\\
&1 &Slips of Rs 5\\
&2 &Slips of Rs 13\\ \hline
\multirow{2}{*}{Y} &0 &Box A\\
&1 &Box B\\\hline
\end{tabular}
\caption{}
\label{tab:Distribution}
\end{table}
See \tabref{tab:Distribution}.
\begin{align}
p_{Y}\brak{k}= \begin{cases} 
      \frac{1}{3} & {k=0} \\
      \frac{2}{3 }& {k=1} 
   \end{cases}
   \\
p_{Y|X}\brak{0|0} = \frac{19}{25}\, 
p_{Y|X}\brak{0|1} = \frac{6}{25}\,
p_{Y|X}\brak{1|0} = \frac{45}{50}\,
p_{Y|X}\brak{1|2} = \frac{5}{50}
\end{align}
The desired probability is the probability that a slip drawn at random is marked other than Rs 1,
\begin{align}
&=1-p_X\brak{0}\\
&= p_X(1) + p_X(2)
\end{align}
Using Bayes theorem,
\begin{align}
&= p_Y\brak{0} \times \pr{Y=0 | X=1} + p_Y\brak{1} \times \pr{Y=1|X=2}\\
&=\frac{1}{3} \times \frac{6}{25} + \frac{2}{3} \times \frac{5}{50}\\
&=\frac{11}{75}
\end{align}

\newpage

%\tableofcontents

\bigskip

\renewcommand{\thefigure}{\theenumi}
\renewcommand{\thetable}{\theenumi}
%\renewcommand{\theequation}{\theenumi}

%\begin{abstract}
%%\boldmath
%In this letter, an algorithm for evaluating the exact analytical bit error rate  (BER)  for the piecewise linear (PL) combiner for  multiple relays is presented. Previous results were available only for upto three relays. The algorithm is unique in the sense that  the actual mathematical expressions, that are prohibitively large, need not be explicitly obtained. The diversity gain due to multiple relays is shown through plots of the analytical BER, well supported by simulations. 
%
%\end{abstract}
% IEEEtran.cls defaults to using nonbold math in the Abstract.
% This preserves the distinction between vectors and scalars. However,
% if the journal you are submitting to favors bold math in the abstract,
% then you can use LaTeX's standard command \boldmath at the very start
% of the abstract to achieve this. Many IEEE journals frown on math
% in the abstract anyway.

% Note that keywords are not normally used for peerreview papers.
%\begin{IEEEkeywords}
%Cooperative diversity, decode and forward, piecewise linear
%\end{IEEEkeywords}



% For peer review papers, you can put extra information on the cover
% page as needed:
% \ifCLASSOPTIONpeerreview
% \begin{center} \bfseries EDICS Category: 3-BBND \end{center}
% \fi
%
% For peerreview papers, this IEEEtran command inserts a page break and
% creates the second title. It will be ignored for other modes.
%\IEEEpeerreviewmaketitle




\item The probability that a student will pass his examination is 0.73, the probability of
the student getting a compartment is 0.13, and the probability that the student will
either pass or get compartment is 0.96. State True or False.\\
\solution
%\begin{table}[H]
	\centering
\begin{tabular}{|c|c|c|}
\hline
Random variable &Value &Definition\\ \hline
\multirow{3}{*}{X} &0 &Slips of Rs 1\\
&1 &Slips of Rs 5\\
&2 &Slips of Rs 13\\ \hline
\multirow{2}{*}{Y} &0 &Box A\\
&1 &Box B\\\hline
\end{tabular}
\caption{}
\label{tab:Distribution}
\end{table}
See \tabref{tab:Distribution}.
\begin{align}
p_{Y}\brak{k}= \begin{cases} 
      \frac{1}{3} & {k=0} \\
      \frac{2}{3 }& {k=1} 
   \end{cases}
   \\
p_{Y|X}\brak{0|0} = \frac{19}{25}\, 
p_{Y|X}\brak{0|1} = \frac{6}{25}\,
p_{Y|X}\brak{1|0} = \frac{45}{50}\,
p_{Y|X}\brak{1|2} = \frac{5}{50}
\end{align}
The desired probability is the probability that a slip drawn at random is marked other than Rs 1,
\begin{align}
&=1-p_X\brak{0}\\
&= p_X(1) + p_X(2)
\end{align}
Using Bayes theorem,
\begin{align}
&= p_Y\brak{0} \times \pr{Y=0 | X=1} + p_Y\brak{1} \times \pr{Y=1|X=2}\\
&=\frac{1}{3} \times \frac{6}{25} + \frac{2}{3} \times \frac{5}{50}\\
&=\frac{11}{75}
\end{align}

\newpage

%\tableofcontents

\bigskip

\renewcommand{\thefigure}{\theenumi}
\renewcommand{\thetable}{\theenumi}
%\renewcommand{\theequation}{\theenumi}

%\begin{abstract}
%%\boldmath
%In this letter, an algorithm for evaluating the exact analytical bit error rate  (BER)  for the piecewise linear (PL) combiner for  multiple relays is presented. Previous results were available only for upto three relays. The algorithm is unique in the sense that  the actual mathematical expressions, that are prohibitively large, need not be explicitly obtained. The diversity gain due to multiple relays is shown through plots of the analytical BER, well supported by simulations. 
%
%\end{abstract}
% IEEEtran.cls defaults to using nonbold math in the Abstract.
% This preserves the distinction between vectors and scalars. However,
% if the journal you are submitting to favors bold math in the abstract,
% then you can use LaTeX's standard command \boldmath at the very start
% of the abstract to achieve this. Many IEEE journals frown on math
% in the abstract anyway.

% Note that keywords are not normally used for peerreview papers.
%\begin{IEEEkeywords}
%Cooperative diversity, decode and forward, piecewise linear
%\end{IEEEkeywords}



% For peer review papers, you can put extra information on the cover
% page as needed:
% \ifCLASSOPTIONpeerreview
% \begin{center} \bfseries EDICS Category: 3-BBND \end{center}
% \fi
%
% For peerreview papers, this IEEEtran command inserts a page break and
% creates the second title. It will be ignored for other modes.
%\IEEEpeerreviewmaketitle




\item A card is selected from a pack of 52 cards\\
\begin{enumerate}[label=(\alph*)]
\item How many points are there in the sample space?
\item Calculate the probability that the cards is an ace of spades.
\item Calculate the probability that the card is (i) an ace (ii)black card.\\
\end{enumerate}
%\input{ncert/11/16/3/4_1/Prob_4.tex}
\item In a non-leap year, the probability of having 53 tuesdays or 53 wednesdays is\\
\solution
%A non-leap year has a total of 365 days, and a week has 7 days.\\
So it can be expressed as 
\begin{align}
365\text{days} &=52\times 7+1 \text{day}
\end{align}
$\implies$ 52 tuesdays or wednesdays\\
Random variable X denotes the days of a week
\begin{align}
p_X\brak{k}&=\frac{1}{7}; \quad \brak{1<k<7}
\end{align}
So the probability of extra day being tuesday or wednesday is
\begin{align}
p_X\brak{3}+p_X\brak{4}&=\frac{1}{7}+\frac{1}{7}=\frac{2}{7}
\end{align}



\item There are 1000 sealed envelopes in a box, 10 of them contain a cash prize of
Rs 100 each, 100 of them contain a cash prize of Rs 50 each and 200 of them
contain a cash prize of Rs 10 each and rest do not contain any cash prize. If they
are well shuffled and an envelope is picked up out, what is the probability that it
contains no cash prize?\\
\solution
%\begin{table}[H]
	\centering
\begin{tabular}{|c|c|c|}
\hline
Random variable &Value &Definition\\ \hline
\multirow{3}{*}{X} &0 &Slips of Rs 1\\
&1 &Slips of Rs 5\\
&2 &Slips of Rs 13\\ \hline
\multirow{2}{*}{Y} &0 &Box A\\
&1 &Box B\\\hline
\end{tabular}
\caption{}
\label{tab:Distribution}
\end{table}
See \tabref{tab:Distribution}.
\begin{align}
p_{Y}\brak{k}= \begin{cases} 
      \frac{1}{3} & {k=0} \\
      \frac{2}{3 }& {k=1} 
   \end{cases}
   \\
p_{Y|X}\brak{0|0} = \frac{19}{25}\, 
p_{Y|X}\brak{0|1} = \frac{6}{25}\,
p_{Y|X}\brak{1|0} = \frac{45}{50}\,
p_{Y|X}\brak{1|2} = \frac{5}{50}
\end{align}
The desired probability is the probability that a slip drawn at random is marked other than Rs 1,
\begin{align}
&=1-p_X\brak{0}\\
&= p_X(1) + p_X(2)
\end{align}
Using Bayes theorem,
\begin{align}
&= p_Y\brak{0} \times \pr{Y=0 | X=1} + p_Y\brak{1} \times \pr{Y=1|X=2}\\
&=\frac{1}{3} \times \frac{6}{25} + \frac{2}{3} \times \frac{5}{50}\\
&=\frac{11}{75}
\end{align}

\newpage

%\tableofcontents

\bigskip

\renewcommand{\thefigure}{\theenumi}
\renewcommand{\thetable}{\theenumi}
%\renewcommand{\theequation}{\theenumi}

%\begin{abstract}
%%\boldmath
%In this letter, an algorithm for evaluating the exact analytical bit error rate  (BER)  for the piecewise linear (PL) combiner for  multiple relays is presented. Previous results were available only for upto three relays. The algorithm is unique in the sense that  the actual mathematical expressions, that are prohibitively large, need not be explicitly obtained. The diversity gain due to multiple relays is shown through plots of the analytical BER, well supported by simulations. 
%
%\end{abstract}
% IEEEtran.cls defaults to using nonbold math in the Abstract.
% This preserves the distinction between vectors and scalars. However,
% if the journal you are submitting to favors bold math in the abstract,
% then you can use LaTeX's standard command \boldmath at the very start
% of the abstract to achieve this. Many IEEE journals frown on math
% in the abstract anyway.

% Note that keywords are not normally used for peerreview papers.
%\begin{IEEEkeywords}
%Cooperative diversity, decode and forward, piecewise linear
%\end{IEEEkeywords}



% For peer review papers, you can put extra information on the cover
% page as needed:
% \ifCLASSOPTIONpeerreview
% \begin{center} \bfseries EDICS Category: 3-BBND \end{center}
% \fi
%
% For peerreview papers, this IEEEtran command inserts a page break and
% creates the second title. It will be ignored for other modes.
%\IEEEpeerreviewmaketitle




\item 
A die is thrown and a card is selected at random from a deck of 52 playing cards. The probability of getting an even number on the die and a spade card.\\
\solution
%\begin{table}[H]
	\centering
\begin{tabular}{|c|c|c|}
\hline
Random variable &Value &Definition\\ \hline
\multirow{3}{*}{X} &0 &Slips of Rs 1\\
&1 &Slips of Rs 5\\
&2 &Slips of Rs 13\\ \hline
\multirow{2}{*}{Y} &0 &Box A\\
&1 &Box B\\\hline
\end{tabular}
\caption{}
\label{tab:Distribution}
\end{table}
See \tabref{tab:Distribution}.
\begin{align}
p_{Y}\brak{k}= \begin{cases} 
      \frac{1}{3} & {k=0} \\
      \frac{2}{3 }& {k=1} 
   \end{cases}
   \\
p_{Y|X}\brak{0|0} = \frac{19}{25}\, 
p_{Y|X}\brak{0|1} = \frac{6}{25}\,
p_{Y|X}\brak{1|0} = \frac{45}{50}\,
p_{Y|X}\brak{1|2} = \frac{5}{50}
\end{align}
The desired probability is the probability that a slip drawn at random is marked other than Rs 1,
\begin{align}
&=1-p_X\brak{0}\\
&= p_X(1) + p_X(2)
\end{align}
Using Bayes theorem,
\begin{align}
&= p_Y\brak{0} \times \pr{Y=0 | X=1} + p_Y\brak{1} \times \pr{Y=1|X=2}\\
&=\frac{1}{3} \times \frac{6}{25} + \frac{2}{3} \times \frac{5}{50}\\
&=\frac{11}{75}
\end{align}

\newpage

%\tableofcontents

\bigskip

\renewcommand{\thefigure}{\theenumi}
\renewcommand{\thetable}{\theenumi}
%\renewcommand{\theequation}{\theenumi}

%\begin{abstract}
%%\boldmath
%In this letter, an algorithm for evaluating the exact analytical bit error rate  (BER)  for the piecewise linear (PL) combiner for  multiple relays is presented. Previous results were available only for upto three relays. The algorithm is unique in the sense that  the actual mathematical expressions, that are prohibitively large, need not be explicitly obtained. The diversity gain due to multiple relays is shown through plots of the analytical BER, well supported by simulations. 
%
%\end{abstract}
% IEEEtran.cls defaults to using nonbold math in the Abstract.
% This preserves the distinction between vectors and scalars. However,
% if the journal you are submitting to favors bold math in the abstract,
% then you can use LaTeX's standard command \boldmath at the very start
% of the abstract to achieve this. Many IEEE journals frown on math
% in the abstract anyway.

% Note that keywords are not normally used for peerreview papers.
%\begin{IEEEkeywords}
%Cooperative diversity, decode and forward, piecewise linear
%\end{IEEEkeywords}



% For peer review papers, you can put extra information on the cover
% page as needed:
% \ifCLASSOPTIONpeerreview
% \begin{center} \bfseries EDICS Category: 3-BBND \end{center}
% \fi
%
% For peerreview papers, this IEEEtran command inserts a page break and
% creates the second title. It will be ignored for other modes.
%\IEEEpeerreviewmaketitle




\item
If 4-digit numbers greater than 5,000 are randomly formed from the digits 0, 1, 3, 5, and 7, what is the probability of forming a number divisible by 5 when:
\begin{enumerate}
    \item The digits are repeated?
    \item The repetition of digits is not allowed?
\end{enumerate}
\solution
%\begin{table}[H]
	\centering
\begin{tabular}{|c|c|c|}
\hline
Random variable &Value &Definition\\ \hline
\multirow{3}{*}{X} &0 &Slips of Rs 1\\
&1 &Slips of Rs 5\\
&2 &Slips of Rs 13\\ \hline
\multirow{2}{*}{Y} &0 &Box A\\
&1 &Box B\\\hline
\end{tabular}
\caption{}
\label{tab:Distribution}
\end{table}
See \tabref{tab:Distribution}.
\begin{align}
p_{Y}\brak{k}= \begin{cases} 
      \frac{1}{3} & {k=0} \\
      \frac{2}{3 }& {k=1} 
   \end{cases}
   \\
p_{Y|X}\brak{0|0} = \frac{19}{25}\, 
p_{Y|X}\brak{0|1} = \frac{6}{25}\,
p_{Y|X}\brak{1|0} = \frac{45}{50}\,
p_{Y|X}\brak{1|2} = \frac{5}{50}
\end{align}
The desired probability is the probability that a slip drawn at random is marked other than Rs 1,
\begin{align}
&=1-p_X\brak{0}\\
&= p_X(1) + p_X(2)
\end{align}
Using Bayes theorem,
\begin{align}
&= p_Y\brak{0} \times \pr{Y=0 | X=1} + p_Y\brak{1} \times \pr{Y=1|X=2}\\
&=\frac{1}{3} \times \frac{6}{25} + \frac{2}{3} \times \frac{5}{50}\\
&=\frac{11}{75}
\end{align}

\newpage

%\tableofcontents

\bigskip

\renewcommand{\thefigure}{\theenumi}
\renewcommand{\thetable}{\theenumi}
%\renewcommand{\theequation}{\theenumi}

%\begin{abstract}
%%\boldmath
%In this letter, an algorithm for evaluating the exact analytical bit error rate  (BER)  for the piecewise linear (PL) combiner for  multiple relays is presented. Previous results were available only for upto three relays. The algorithm is unique in the sense that  the actual mathematical expressions, that are prohibitively large, need not be explicitly obtained. The diversity gain due to multiple relays is shown through plots of the analytical BER, well supported by simulations. 
%
%\end{abstract}
% IEEEtran.cls defaults to using nonbold math in the Abstract.
% This preserves the distinction between vectors and scalars. However,
% if the journal you are submitting to favors bold math in the abstract,
% then you can use LaTeX's standard command \boldmath at the very start
% of the abstract to achieve this. Many IEEE journals frown on math
% in the abstract anyway.

% Note that keywords are not normally used for peerreview papers.
%\begin{IEEEkeywords}
%Cooperative diversity, decode and forward, piecewise linear
%\end{IEEEkeywords}



% For peer review papers, you can put extra information on the cover
% page as needed:
% \ifCLASSOPTIONpeerreview
% \begin{center} \bfseries EDICS Category: 3-BBND \end{center}
% \fi
%
% For peerreview papers, this IEEEtran command inserts a page break and
% creates the second title. It will be ignored for other modes.
%\IEEEpeerreviewmaketitle




\item Consider the probability space $\brak{\Omega, \mathcal{G}, P}$ where $\Omega = [0,2]$ and $\mathcal{G} = \cbrak{\phi, \Omega, [0,1], (1,2]}$. Let $X$ and $Y$ be two functions on $\Omega$ defined as
\begin{align*}
    X(\omega) = 
    \begin{cases}
        1 & \text{if }\omega \in [0, 1]\\
        2 & \text{if }\omega \in (1, 2]
    \end{cases}
\end{align*}
and
\begin{align*}
    Y(\omega) = 
    \begin{cases}
        2 & \text{if }\omega \in [0, 1.5]\\
        3 & \text{if }\omega \in (1.5, 2].
    \end{cases}
\end{align*}
Then which one of the following statements is true?
\begin{enumerate}
    \item [(A)] $X$ is a random variable with respect to $\mathcal{G}$, but $Y$ is not a random variable with respect to $\mathcal{G}$.
    \item [(B)] $Y$ is a random variable with respect to $\mathcal{G}$, but $X$ is not a random variable with respect to $\mathcal{G}$.
    \item [(C)] Neither $X$ nor $Y$ is a random variable with respect to $\mathcal{G}$.
    \item [(D)] Both $X$ and $Y$ are random variables with respect to $\mathcal{G}$.
\end{enumerate} \hfill (GATE ST 2023)\\
\solution
%\begin{table}[H]
	\centering
\begin{tabular}{|c|c|c|}
\hline
Random variable &Value &Definition\\ \hline
\multirow{3}{*}{X} &0 &Slips of Rs 1\\
&1 &Slips of Rs 5\\
&2 &Slips of Rs 13\\ \hline
\multirow{2}{*}{Y} &0 &Box A\\
&1 &Box B\\\hline
\end{tabular}
\caption{}
\label{tab:Distribution}
\end{table}
See \tabref{tab:Distribution}.
\begin{align}
p_{Y}\brak{k}= \begin{cases} 
      \frac{1}{3} & {k=0} \\
      \frac{2}{3 }& {k=1} 
   \end{cases}
   \\
p_{Y|X}\brak{0|0} = \frac{19}{25}\, 
p_{Y|X}\brak{0|1} = \frac{6}{25}\,
p_{Y|X}\brak{1|0} = \frac{45}{50}\,
p_{Y|X}\brak{1|2} = \frac{5}{50}
\end{align}
The desired probability is the probability that a slip drawn at random is marked other than Rs 1,
\begin{align}
&=1-p_X\brak{0}\\
&= p_X(1) + p_X(2)
\end{align}
Using Bayes theorem,
\begin{align}
&= p_Y\brak{0} \times \pr{Y=0 | X=1} + p_Y\brak{1} \times \pr{Y=1|X=2}\\
&=\frac{1}{3} \times \frac{6}{25} + \frac{2}{3} \times \frac{5}{50}\\
&=\frac{11}{75}
\end{align}

\newpage

%\tableofcontents

\bigskip

\renewcommand{\thefigure}{\theenumi}
\renewcommand{\thetable}{\theenumi}
%\renewcommand{\theequation}{\theenumi}

%\begin{abstract}
%%\boldmath
%In this letter, an algorithm for evaluating the exact analytical bit error rate  (BER)  for the piecewise linear (PL) combiner for  multiple relays is presented. Previous results were available only for upto three relays. The algorithm is unique in the sense that  the actual mathematical expressions, that are prohibitively large, need not be explicitly obtained. The diversity gain due to multiple relays is shown through plots of the analytical BER, well supported by simulations. 
%
%\end{abstract}
% IEEEtran.cls defaults to using nonbold math in the Abstract.
% This preserves the distinction between vectors and scalars. However,
% if the journal you are submitting to favors bold math in the abstract,
% then you can use LaTeX's standard command \boldmath at the very start
% of the abstract to achieve this. Many IEEE journals frown on math
% in the abstract anyway.

% Note that keywords are not normally used for peerreview papers.
%\begin{IEEEkeywords}
%Cooperative diversity, decode and forward, piecewise linear
%\end{IEEEkeywords}



% For peer review papers, you can put extra information on the cover
% page as needed:
% \ifCLASSOPTIONpeerreview
% \begin{center} \bfseries EDICS Category: 3-BBND \end{center}
% \fi
%
% For peerreview papers, this IEEEtran command inserts a page break and
% creates the second title. It will be ignored for other modes.
%\IEEEpeerreviewmaketitle




	\item  A die is loaded in such a way that each odd number is twice as likely to occur as
each even number. Find $P(G)$, where $G$ is the event that a number greater than
3 occurs on a single roll of the die.
\\
\solution
		%\begin{table}[H]
	\centering
\begin{tabular}{|c|c|c|}
\hline
Random variable &Value &Definition\\ \hline
\multirow{3}{*}{X} &0 &Slips of Rs 1\\
&1 &Slips of Rs 5\\
&2 &Slips of Rs 13\\ \hline
\multirow{2}{*}{Y} &0 &Box A\\
&1 &Box B\\\hline
\end{tabular}
\caption{}
\label{tab:Distribution}
\end{table}
See \tabref{tab:Distribution}.
\begin{align}
p_{Y}\brak{k}= \begin{cases} 
      \frac{1}{3} & {k=0} \\
      \frac{2}{3 }& {k=1} 
   \end{cases}
   \\
p_{Y|X}\brak{0|0} = \frac{19}{25}\, 
p_{Y|X}\brak{0|1} = \frac{6}{25}\,
p_{Y|X}\brak{1|0} = \frac{45}{50}\,
p_{Y|X}\brak{1|2} = \frac{5}{50}
\end{align}
The desired probability is the probability that a slip drawn at random is marked other than Rs 1,
\begin{align}
&=1-p_X\brak{0}\\
&= p_X(1) + p_X(2)
\end{align}
Using Bayes theorem,
\begin{align}
&= p_Y\brak{0} \times \pr{Y=0 | X=1} + p_Y\brak{1} \times \pr{Y=1|X=2}\\
&=\frac{1}{3} \times \frac{6}{25} + \frac{2}{3} \times \frac{5}{50}\\
&=\frac{11}{75}
\end{align}

\newpage

%\tableofcontents

\bigskip

\renewcommand{\thefigure}{\theenumi}
\renewcommand{\thetable}{\theenumi}
%\renewcommand{\theequation}{\theenumi}

%\begin{abstract}
%%\boldmath
%In this letter, an algorithm for evaluating the exact analytical bit error rate  (BER)  for the piecewise linear (PL) combiner for  multiple relays is presented. Previous results were available only for upto three relays. The algorithm is unique in the sense that  the actual mathematical expressions, that are prohibitively large, need not be explicitly obtained. The diversity gain due to multiple relays is shown through plots of the analytical BER, well supported by simulations. 
%
%\end{abstract}
% IEEEtran.cls defaults to using nonbold math in the Abstract.
% This preserves the distinction between vectors and scalars. However,
% if the journal you are submitting to favors bold math in the abstract,
% then you can use LaTeX's standard command \boldmath at the very start
% of the abstract to achieve this. Many IEEE journals frown on math
% in the abstract anyway.

% Note that keywords are not normally used for peerreview papers.
%\begin{IEEEkeywords}
%Cooperative diversity, decode and forward, piecewise linear
%\end{IEEEkeywords}



% For peer review papers, you can put extra information on the cover
% page as needed:
% \ifCLASSOPTIONpeerreview
% \begin{center} \bfseries EDICS Category: 3-BBND \end{center}
% \fi
%
% For peerreview papers, this IEEEtran command inserts a page break and
% creates the second title. It will be ignored for other modes.
%\IEEEpeerreviewmaketitle




	\item All the jacks, queens and kings are removed from a deck of 52 playing cards. The remaining cards are well shuffled and then one card is drawn at random. Giving ace a value 1 similar value for other cards, find the probability that the card has a value 
		\begin{enumerate}
			\item 7
			\item greater than 7
			\item less than 7
		\end{enumerate}
		%Number of cards left after removing all jacks, queens and kings 
\begin{align}
N	= 52 - 4\times 3
	= 40
\end{align}
%\begin{table}[H]
%\def\arraystretch{1.2}
%\begin{tabular}{|c|c|c|}
%\hline
%	\textbf{Parameter} &\textbf{Value} &\textbf{Description}\\ \hline
%	$X$ &1-10 &Represents the value of the card picked \\ \hline
%\end{tabular}
%\end{table}
Let $1 \le X \le 10$ be the value of the card picked.  Then,
\begin{align}
	p_X(k) &= \Pr(X=k)\ \forall\ 1 \leq k \leq 10\\
	&= \frac{4\times 1}{40}\\
	&= \frac{1}{10}\\
	\therefore p_X(k) &= 
	\begin{cases}
		\frac{1}{10} & 1 \leq k \leq 10\\
		0 & \text{otherwise}
	\end{cases}
\end{align}
and
\begin{align}
	F_{X}(k) &= \sum_{m=0}^{k}p_{X}(m) \quad 1 \leq k \leq 10\\
	&= \frac{k}{10}\\
	\therefore F_{X}(k) &= 
	\begin{cases}
		0 & k \leq 0\\
		\frac{k}{10} & 1\leq k \leq 10\\
		1 & k > 10 
	\end{cases}
\end{align}
\begin{enumerate}
	\item Probability that card has value equal to 7 is
		\begin{align}
			 p_{X}(7)
			= \frac{1}{10}
		\end{align}
	\item Probability that card has value greater than 7 is
		\begin{align}
			1 - F_X(7)
			&= 1 - \frac{7}{10}
			\\
			&= \frac{3}{10}
		\end{align}
	\item Probability that card has value less than 7 is
		\begin{align}
			 F_{X}(6)
			=\frac{6}{10}
		\end{align}
\end{enumerate}

  \item A Lot consists of 48 mobile phones of which 42 are good, 3 have only minor defects and 3 have major defects.Varnika will buy a phone if it is good but the trader will only buy a mobile if it has no major defects. One phone is selected at random from the lot. What is the probability that it is
\begin{enumerate}
	\item acceptable to Varnika?
            \item acceptable to the trader?
\end{enumerate}
\solution
	%\begin{table}[H]
	\centering
\begin{tabular}{|c|c|c|}
\hline
Random variable &Value &Definition\\ \hline
\multirow{3}{*}{X} &0 &Slips of Rs 1\\
&1 &Slips of Rs 5\\
&2 &Slips of Rs 13\\ \hline
\multirow{2}{*}{Y} &0 &Box A\\
&1 &Box B\\\hline
\end{tabular}
\caption{}
\label{tab:Distribution}
\end{table}
See \tabref{tab:Distribution}.
\begin{align}
p_{Y}\brak{k}= \begin{cases} 
      \frac{1}{3} & {k=0} \\
      \frac{2}{3 }& {k=1} 
   \end{cases}
   \\
p_{Y|X}\brak{0|0} = \frac{19}{25}\, 
p_{Y|X}\brak{0|1} = \frac{6}{25}\,
p_{Y|X}\brak{1|0} = \frac{45}{50}\,
p_{Y|X}\brak{1|2} = \frac{5}{50}
\end{align}
The desired probability is the probability that a slip drawn at random is marked other than Rs 1,
\begin{align}
&=1-p_X\brak{0}\\
&= p_X(1) + p_X(2)
\end{align}
Using Bayes theorem,
\begin{align}
&= p_Y\brak{0} \times \pr{Y=0 | X=1} + p_Y\brak{1} \times \pr{Y=1|X=2}\\
&=\frac{1}{3} \times \frac{6}{25} + \frac{2}{3} \times \frac{5}{50}\\
&=\frac{11}{75}
\end{align}

\newpage

%\tableofcontents

\bigskip

\renewcommand{\thefigure}{\theenumi}
\renewcommand{\thetable}{\theenumi}
%\renewcommand{\theequation}{\theenumi}

%\begin{abstract}
%%\boldmath
%In this letter, an algorithm for evaluating the exact analytical bit error rate  (BER)  for the piecewise linear (PL) combiner for  multiple relays is presented. Previous results were available only for upto three relays. The algorithm is unique in the sense that  the actual mathematical expressions, that are prohibitively large, need not be explicitly obtained. The diversity gain due to multiple relays is shown through plots of the analytical BER, well supported by simulations. 
%
%\end{abstract}
% IEEEtran.cls defaults to using nonbold math in the Abstract.
% This preserves the distinction between vectors and scalars. However,
% if the journal you are submitting to favors bold math in the abstract,
% then you can use LaTeX's standard command \boldmath at the very start
% of the abstract to achieve this. Many IEEE journals frown on math
% in the abstract anyway.

% Note that keywords are not normally used for peerreview papers.
%\begin{IEEEkeywords}
%Cooperative diversity, decode and forward, piecewise linear
%\end{IEEEkeywords}



% For peer review papers, you can put extra information on the cover
% page as needed:
% \ifCLASSOPTIONpeerreview
% \begin{center} \bfseries EDICS Category: 3-BBND \end{center}
% \fi
%
% For peerreview papers, this IEEEtran command inserts a page break and
% creates the second title. It will be ignored for other modes.
%\IEEEpeerreviewmaketitle




 \item A student says that if you throw a die, it will show up 1 or not 1. Therefore, the probability of getting 1 and the probability of getting 'not 1' each is equal to $\frac{1}{2}$. Is this correct? Give reasons.\\
 \solution
        %\begin{table}[H]
	\centering
\begin{tabular}{|c|c|c|}
\hline
Random variable &Value &Definition\\ \hline
\multirow{3}{*}{X} &0 &Slips of Rs 1\\
&1 &Slips of Rs 5\\
&2 &Slips of Rs 13\\ \hline
\multirow{2}{*}{Y} &0 &Box A\\
&1 &Box B\\\hline
\end{tabular}
\caption{}
\label{tab:Distribution}
\end{table}
See \tabref{tab:Distribution}.
\begin{align}
p_{Y}\brak{k}= \begin{cases} 
      \frac{1}{3} & {k=0} \\
      \frac{2}{3 }& {k=1} 
   \end{cases}
   \\
p_{Y|X}\brak{0|0} = \frac{19}{25}\, 
p_{Y|X}\brak{0|1} = \frac{6}{25}\,
p_{Y|X}\brak{1|0} = \frac{45}{50}\,
p_{Y|X}\brak{1|2} = \frac{5}{50}
\end{align}
The desired probability is the probability that a slip drawn at random is marked other than Rs 1,
\begin{align}
&=1-p_X\brak{0}\\
&= p_X(1) + p_X(2)
\end{align}
Using Bayes theorem,
\begin{align}
&= p_Y\brak{0} \times \pr{Y=0 | X=1} + p_Y\brak{1} \times \pr{Y=1|X=2}\\
&=\frac{1}{3} \times \frac{6}{25} + \frac{2}{3} \times \frac{5}{50}\\
&=\frac{11}{75}
\end{align}

\newpage

%\tableofcontents

\bigskip

\renewcommand{\thefigure}{\theenumi}
\renewcommand{\thetable}{\theenumi}
%\renewcommand{\theequation}{\theenumi}

%\begin{abstract}
%%\boldmath
%In this letter, an algorithm for evaluating the exact analytical bit error rate  (BER)  for the piecewise linear (PL) combiner for  multiple relays is presented. Previous results were available only for upto three relays. The algorithm is unique in the sense that  the actual mathematical expressions, that are prohibitively large, need not be explicitly obtained. The diversity gain due to multiple relays is shown through plots of the analytical BER, well supported by simulations. 
%
%\end{abstract}
% IEEEtran.cls defaults to using nonbold math in the Abstract.
% This preserves the distinction between vectors and scalars. However,
% if the journal you are submitting to favors bold math in the abstract,
% then you can use LaTeX's standard command \boldmath at the very start
% of the abstract to achieve this. Many IEEE journals frown on math
% in the abstract anyway.

% Note that keywords are not normally used for peerreview papers.
%\begin{IEEEkeywords}
%Cooperative diversity, decode and forward, piecewise linear
%\end{IEEEkeywords}



% For peer review papers, you can put extra information on the cover
% page as needed:
% \ifCLASSOPTIONpeerreview
% \begin{center} \bfseries EDICS Category: 3-BBND \end{center}
% \fi
%
% For peerreview papers, this IEEEtran command inserts a page break and
% creates the second title. It will be ignored for other modes.
%\IEEEpeerreviewmaketitle




   \item Four candidates A, B, C, D have ap-
plied for the assignment to coach a school cricket
team. If A is twice as likely to be selected as B, and
B and C are given about the same chance of being
selected, while C is twice as likely to be selected
as D, what are the probabilities that
\begin{enumerate}
\item C will be selected?
\item A will not be selected?
\end{enumerate}
	%\begin{table}[H]
	\centering
\begin{tabular}{|c|c|c|}
\hline
Random variable &Value &Definition\\ \hline
\multirow{3}{*}{X} &0 &Slips of Rs 1\\
&1 &Slips of Rs 5\\
&2 &Slips of Rs 13\\ \hline
\multirow{2}{*}{Y} &0 &Box A\\
&1 &Box B\\\hline
\end{tabular}
\caption{}
\label{tab:Distribution}
\end{table}
See \tabref{tab:Distribution}.
\begin{align}
p_{Y}\brak{k}= \begin{cases} 
      \frac{1}{3} & {k=0} \\
      \frac{2}{3 }& {k=1} 
   \end{cases}
   \\
p_{Y|X}\brak{0|0} = \frac{19}{25}\, 
p_{Y|X}\brak{0|1} = \frac{6}{25}\,
p_{Y|X}\brak{1|0} = \frac{45}{50}\,
p_{Y|X}\brak{1|2} = \frac{5}{50}
\end{align}
The desired probability is the probability that a slip drawn at random is marked other than Rs 1,
\begin{align}
&=1-p_X\brak{0}\\
&= p_X(1) + p_X(2)
\end{align}
Using Bayes theorem,
\begin{align}
&= p_Y\brak{0} \times \pr{Y=0 | X=1} + p_Y\brak{1} \times \pr{Y=1|X=2}\\
&=\frac{1}{3} \times \frac{6}{25} + \frac{2}{3} \times \frac{5}{50}\\
&=\frac{11}{75}
\end{align}

\newpage

%\tableofcontents

\bigskip

\renewcommand{\thefigure}{\theenumi}
\renewcommand{\thetable}{\theenumi}
%\renewcommand{\theequation}{\theenumi}

%\begin{abstract}
%%\boldmath
%In this letter, an algorithm for evaluating the exact analytical bit error rate  (BER)  for the piecewise linear (PL) combiner for  multiple relays is presented. Previous results were available only for upto three relays. The algorithm is unique in the sense that  the actual mathematical expressions, that are prohibitively large, need not be explicitly obtained. The diversity gain due to multiple relays is shown through plots of the analytical BER, well supported by simulations. 
%
%\end{abstract}
% IEEEtran.cls defaults to using nonbold math in the Abstract.
% This preserves the distinction between vectors and scalars. However,
% if the journal you are submitting to favors bold math in the abstract,
% then you can use LaTeX's standard command \boldmath at the very start
% of the abstract to achieve this. Many IEEE journals frown on math
% in the abstract anyway.

% Note that keywords are not normally used for peerreview papers.
%\begin{IEEEkeywords}
%Cooperative diversity, decode and forward, piecewise linear
%\end{IEEEkeywords}



% For peer review papers, you can put extra information on the cover
% page as needed:
% \ifCLASSOPTIONpeerreview
% \begin{center} \bfseries EDICS Category: 3-BBND \end{center}
% \fi
%
% For peerreview papers, this IEEEtran command inserts a page break and
% creates the second title. It will be ignored for other modes.
%\IEEEpeerreviewmaketitle




 \item A bag contain 24 balls of which $x$ balls are red, $2x$ are white and $3x$ are blue. A ball is selected at random, What is the probability that it is
\begin{enumerate}[label=\alph*)]
\item not red ?
\item white ?
\end{enumerate}
%\begin{table}[H]
	\centering
\begin{tabular}{|c|c|c|}
\hline
Random variable &Value &Definition\\ \hline
\multirow{3}{*}{X} &0 &Slips of Rs 1\\
&1 &Slips of Rs 5\\
&2 &Slips of Rs 13\\ \hline
\multirow{2}{*}{Y} &0 &Box A\\
&1 &Box B\\\hline
\end{tabular}
\caption{}
\label{tab:Distribution}
\end{table}
See \tabref{tab:Distribution}.
\begin{align}
p_{Y}\brak{k}= \begin{cases} 
      \frac{1}{3} & {k=0} \\
      \frac{2}{3 }& {k=1} 
   \end{cases}
   \\
p_{Y|X}\brak{0|0} = \frac{19}{25}\, 
p_{Y|X}\brak{0|1} = \frac{6}{25}\,
p_{Y|X}\brak{1|0} = \frac{45}{50}\,
p_{Y|X}\brak{1|2} = \frac{5}{50}
\end{align}
The desired probability is the probability that a slip drawn at random is marked other than Rs 1,
\begin{align}
&=1-p_X\brak{0}\\
&= p_X(1) + p_X(2)
\end{align}
Using Bayes theorem,
\begin{align}
&= p_Y\brak{0} \times \pr{Y=0 | X=1} + p_Y\brak{1} \times \pr{Y=1|X=2}\\
&=\frac{1}{3} \times \frac{6}{25} + \frac{2}{3} \times \frac{5}{50}\\
&=\frac{11}{75}
\end{align}

\newpage

%\tableofcontents

\bigskip

\renewcommand{\thefigure}{\theenumi}
\renewcommand{\thetable}{\theenumi}
%\renewcommand{\theequation}{\theenumi}

%\begin{abstract}
%%\boldmath
%In this letter, an algorithm for evaluating the exact analytical bit error rate  (BER)  for the piecewise linear (PL) combiner for  multiple relays is presented. Previous results were available only for upto three relays. The algorithm is unique in the sense that  the actual mathematical expressions, that are prohibitively large, need not be explicitly obtained. The diversity gain due to multiple relays is shown through plots of the analytical BER, well supported by simulations. 
%
%\end{abstract}
% IEEEtran.cls defaults to using nonbold math in the Abstract.
% This preserves the distinction between vectors and scalars. However,
% if the journal you are submitting to favors bold math in the abstract,
% then you can use LaTeX's standard command \boldmath at the very start
% of the abstract to achieve this. Many IEEE journals frown on math
% in the abstract anyway.

% Note that keywords are not normally used for peerreview papers.
%\begin{IEEEkeywords}
%Cooperative diversity, decode and forward, piecewise linear
%\end{IEEEkeywords}



% For peer review papers, you can put extra information on the cover
% page as needed:
% \ifCLASSOPTIONpeerreview
% \begin{center} \bfseries EDICS Category: 3-BBND \end{center}
% \fi
%
% For peerreview papers, this IEEEtran command inserts a page break and
% creates the second title. It will be ignored for other modes.
%\IEEEpeerreviewmaketitle




If the letters of the word ASSASSINATION are arranged at random. Find the Probability that
\begin{enumerate}[label=(\alph*)]
\item Four $S's$ come consecutively in the word
\item Two  $I's$ and two $N's$ come together
\item All $A's$ are not coming together
\item No two $A's$ are coming together
\end{enumerate}
%\begin{table}[H]
	\centering
\begin{tabular}{|c|c|c|}
\hline
Random variable &Value &Definition\\ \hline
\multirow{3}{*}{X} &0 &Slips of Rs 1\\
&1 &Slips of Rs 5\\
&2 &Slips of Rs 13\\ \hline
\multirow{2}{*}{Y} &0 &Box A\\
&1 &Box B\\\hline
\end{tabular}
\caption{}
\label{tab:Distribution}
\end{table}
See \tabref{tab:Distribution}.
\begin{align}
p_{Y}\brak{k}= \begin{cases} 
      \frac{1}{3} & {k=0} \\
      \frac{2}{3 }& {k=1} 
   \end{cases}
   \\
p_{Y|X}\brak{0|0} = \frac{19}{25}\, 
p_{Y|X}\brak{0|1} = \frac{6}{25}\,
p_{Y|X}\brak{1|0} = \frac{45}{50}\,
p_{Y|X}\brak{1|2} = \frac{5}{50}
\end{align}
The desired probability is the probability that a slip drawn at random is marked other than Rs 1,
\begin{align}
&=1-p_X\brak{0}\\
&= p_X(1) + p_X(2)
\end{align}
Using Bayes theorem,
\begin{align}
&= p_Y\brak{0} \times \pr{Y=0 | X=1} + p_Y\brak{1} \times \pr{Y=1|X=2}\\
&=\frac{1}{3} \times \frac{6}{25} + \frac{2}{3} \times \frac{5}{50}\\
&=\frac{11}{75}
\end{align}

\newpage

%\tableofcontents

\bigskip

\renewcommand{\thefigure}{\theenumi}
\renewcommand{\thetable}{\theenumi}
%\renewcommand{\theequation}{\theenumi}

%\begin{abstract}
%%\boldmath
%In this letter, an algorithm for evaluating the exact analytical bit error rate  (BER)  for the piecewise linear (PL) combiner for  multiple relays is presented. Previous results were available only for upto three relays. The algorithm is unique in the sense that  the actual mathematical expressions, that are prohibitively large, need not be explicitly obtained. The diversity gain due to multiple relays is shown through plots of the analytical BER, well supported by simulations. 
%
%\end{abstract}
% IEEEtran.cls defaults to using nonbold math in the Abstract.
% This preserves the distinction between vectors and scalars. However,
% if the journal you are submitting to favors bold math in the abstract,
% then you can use LaTeX's standard command \boldmath at the very start
% of the abstract to achieve this. Many IEEE journals frown on math
% in the abstract anyway.

% Note that keywords are not normally used for peerreview papers.
%\begin{IEEEkeywords}
%Cooperative diversity, decode and forward, piecewise linear
%\end{IEEEkeywords}



% For peer review papers, you can put extra information on the cover
% page as needed:
% \ifCLASSOPTIONpeerreview
% \begin{center} \bfseries EDICS Category: 3-BBND \end{center}
% \fi
%
% For peerreview papers, this IEEEtran command inserts a page break and
% creates the second title. It will be ignored for other modes.
%\IEEEpeerreviewmaketitle




	\item One urn contains two black balls (labelled B1 and B2) and one white ball. A
	second urn contains one black ball and two white balls (labelled W1 and W2).
	Suppose the following experiment is performed. One of the two urns is chosen
	at random. Next a ball is randomly chosen from the urn. Then a second ball is
	chosen at random from the same urn without replacing the first ball.
	
	\begin{enumerate}
	\item What is the probability that two black balls are chosen?
	
	\item What is the probability that two balls of opposite colour are chosen?
	\end{enumerate}
	\solution
	%\begin{align}
    \label{eq:12.13.6.18.1}
	\because	\pr{A|B} &> \pr{A},\
\frac{\pr{AB}}{\pr{B}} > \pr{A}
\\
    \label{eq:12.13.6.18.2}
	\implies \pr{AB} &> \pr{A}\pr{B}
	\\
	\text{or, } \frac{\pr{AB}}{\pr{A}} &=\pr{B|A} > \pr{A}
\end{align}

\end{enumerate}

		%
\item 
Out of 100 students, two sections of 40 and 60 are formed. If you and your friend are among the 100 students, what is the probability that
\begin{enumerate}
\item you both enter the same section?
\item you both enter the different sections?
\end{enumerate}
\solution
		%\begin{enumerate}[label=\thesection.\arabic*,ref=\thesection.\theenumi]
	\item One card is drawn from a well-shuffled deck of 52 cards. Find the probability of getting
\begin{enumerate}
\item A king of red colour 
\item A face card 
\item A red face card
\item The jack of hearts
\item A spade
\item The queen of diamonds

\end{enumerate}
\solution
		%\begin{table}[H]
	\centering
\begin{tabular}{|c|c|c|}
\hline
Random variable &Value &Definition\\ \hline
\multirow{3}{*}{X} &0 &Slips of Rs 1\\
&1 &Slips of Rs 5\\
&2 &Slips of Rs 13\\ \hline
\multirow{2}{*}{Y} &0 &Box A\\
&1 &Box B\\\hline
\end{tabular}
\caption{}
\label{tab:Distribution}
\end{table}
See \tabref{tab:Distribution}.
\begin{align}
p_{Y}\brak{k}= \begin{cases} 
      \frac{1}{3} & {k=0} \\
      \frac{2}{3 }& {k=1} 
   \end{cases}
   \\
p_{Y|X}\brak{0|0} = \frac{19}{25}\, 
p_{Y|X}\brak{0|1} = \frac{6}{25}\,
p_{Y|X}\brak{1|0} = \frac{45}{50}\,
p_{Y|X}\brak{1|2} = \frac{5}{50}
\end{align}
The desired probability is the probability that a slip drawn at random is marked other than Rs 1,
\begin{align}
&=1-p_X\brak{0}\\
&= p_X(1) + p_X(2)
\end{align}
Using Bayes theorem,
\begin{align}
&= p_Y\brak{0} \times \pr{Y=0 | X=1} + p_Y\brak{1} \times \pr{Y=1|X=2}\\
&=\frac{1}{3} \times \frac{6}{25} + \frac{2}{3} \times \frac{5}{50}\\
&=\frac{11}{75}
\end{align}

\newpage

%\tableofcontents

\bigskip

\renewcommand{\thefigure}{\theenumi}
\renewcommand{\thetable}{\theenumi}
%\renewcommand{\theequation}{\theenumi}

%\begin{abstract}
%%\boldmath
%In this letter, an algorithm for evaluating the exact analytical bit error rate  (BER)  for the piecewise linear (PL) combiner for  multiple relays is presented. Previous results were available only for upto three relays. The algorithm is unique in the sense that  the actual mathematical expressions, that are prohibitively large, need not be explicitly obtained. The diversity gain due to multiple relays is shown through plots of the analytical BER, well supported by simulations. 
%
%\end{abstract}
% IEEEtran.cls defaults to using nonbold math in the Abstract.
% This preserves the distinction between vectors and scalars. However,
% if the journal you are submitting to favors bold math in the abstract,
% then you can use LaTeX's standard command \boldmath at the very start
% of the abstract to achieve this. Many IEEE journals frown on math
% in the abstract anyway.

% Note that keywords are not normally used for peerreview papers.
%\begin{IEEEkeywords}
%Cooperative diversity, decode and forward, piecewise linear
%\end{IEEEkeywords}



% For peer review papers, you can put extra information on the cover
% page as needed:
% \ifCLASSOPTIONpeerreview
% \begin{center} \bfseries EDICS Category: 3-BBND \end{center}
% \fi
%
% For peerreview papers, this IEEEtran command inserts a page break and
% creates the second title. It will be ignored for other modes.
%\IEEEpeerreviewmaketitle




	\item Five cards—the ten, jack, queen, king and ace of diamonds, are well-shuffled with their face downwards. One card is then picked up at random.
\begin{enumerate}
\item
What is the probability that the card is the queen? 
\item
If the queen is drawn and put aside, what is the probability that the second card picked up is (a) an ace? (b) a queen?\\
\end{enumerate}
\solution
		%\begin{enumerate}[label=\thesection.\arabic*,ref=\thesection.\theenumi]
	\item One card is drawn from a well-shuffled deck of 52 cards. Find the probability of getting
\begin{enumerate}
\item A king of red colour 
\item A face card 
\item A red face card
\item The jack of hearts
\item A spade
\item The queen of diamonds

\end{enumerate}
\solution
		%\input{ncert/10/15/1/14/main.tex}
	\item Five cards—the ten, jack, queen, king and ace of diamonds, are well-shuffled with their face downwards. One card is then picked up at random.
\begin{enumerate}
\item
What is the probability that the card is the queen? 
\item
If the queen is drawn and put aside, what is the probability that the second card picked up is (a) an ace? (b) a queen?\\
\end{enumerate}
\solution
		%\input{ncert/10/15/1/15/defs.tex}
	\item A bag contains $5$ red balls and some blue balls. If the probability of drawing a blue ball is double that if a red ball, determine the number of blue balls in the bag. 
		\\
\solution
		%\input{ncert/10/15/2/3/defs.tex}
	\item A card is selected from a pack of 52 cards.
 \begin{enumerate}[label=(\alph*)] 
                 \item How many points are there in the sample space?
                 \item Calculate the probability that the card is an ace of spades.
                 \item Calculate the probability that the card is (i) an ace and (ii) black card.
 \end{enumerate}
\solution
		%\input{ncert/11/16/3/4/main.tex}
\item Four cards are drawn from a well-shuffled deck of 52 cards. What is the probability of obtaining 3 diamonds and one spade.
\\
\solution
		%\input{ncert/11/16/4/2/defs.tex}
\item In a certain lottery 10,000 tickets are sold and ten equal prizes are awarded. What is the probability of not getting a prize if you buy (a) one ticket (b) two tickets (c) 10 tickets ?	
\\
\solution
		%\input{ncert/11/16/4/4/defs.tex}
		%
\item 
Out of 100 students, two sections of 40 and 60 are formed. If you and your friend are among the 100 students, what is the probability that
\begin{enumerate}
\item you both enter the same section?
\item you both enter the different sections?
\end{enumerate}
\solution
		%\input{ncert/11/16/4/5/defs.tex}
	\item 
The number lock of a suitcase has 4 wheels each labelled with ten digits i.e. from 0 to 9.The lock opens with a sequence of four digits with no repeats.What is the probability of a person getting the right sequence to open the suitcase.
\\
\solution
		%\input{ncert/11/16/4/10/defs.tex}
		%
\item 
Two cards are drawn at random and without replacement from a pack of 52 playing cards. Find the probability that both the cards are black.
\\
\solution
		%\input{ncert/12/13/2/2/defs.tex}
		\item A box of oranges is inspected by examining three randomly selected oranges drawn without replacement. If all the three oranges are good, the box is approved for sale, otherwise, it is rejected. Find the probability that a box containing 15 oranges out of which 12 are good and 3 are bad ones will be approved for sale.
		\label{ncert/12/13/2/3/defs.tex}
		\item Two balls are drawn at random with replacement from a box containing 10 black and 8 red balls. Find the probability that
		\label{ncert/12/13/2/12}
\begin{enumerate}
\item both balls are red.
\item first ball is black and second is red.
\item one of them is black and other is red.
\end{enumerate}

\item In a hostel, 60\% of the students read Hindi newspaper, 40\% read English newspaper and 20\% read both Hindi and English newspapers. A student is selected at random.
		\label{ncert/12/13/2/15}
\begin{enumerate}
\item Find the probability that she reads neither Hindi nor English newspapers.
\item If she reads Hindi newspaper, find the probability that she reads English newspaper.
\item If she reads English newspaper, find the probability that she reads Hindi newspaper.\\
\end{enumerate}
\item The probability of obtaining an even prime number on each die, when a pair of dice is rolled is 
\begin{enumerate}
    \item $0$ 
    
    \item $\frac{1}{3}$ 
    
    \item $\frac{1}{12}$ 
    
    \item $\frac{1}{36}$ 
\end{enumerate}
\solution
		%\input{ncert/12/13/2/17/defs.tex}
	\item A bag contains 4 red and 4 black balls, another bag contains 2 red and 6 black balls. One of the two bags is selected at random and a ball is drawn from the bag which is found to be red. Find the probability that the ball is drawn from the first bag.
\\
\solution
		%\input{ncert/12/13/3/2/main.tex}
  \item
  Cards with numbers 2 to 101 are placed in a box. A card is selected at random.Find the probability that the card has
\begin{enumerate}[label=(\roman*)]
	\item an even number 
	\item a square number
\end{enumerate}
\solution
%\input{exemplar/10/13/3/32/main.tex}
\item
The king, queen and jack of clubs are removed from a deck of 52 playing cards and then well shuffled. Now one card is drawn at random from the remaining cards.  Determine the probability that the card is
\begin{enumerate}[label=(\roman*)]
\item a club
\item 10 of hearts
\end{enumerate}
\solution
%\input{exemplar/10/13/3/29/main.tex}
\item A team of medical students doing their internship have to assist during surgeries
at a city hospital. The probabilities of surgeries rated as very complex, complex,
routine, simple or very simple are respectively, 0.15, 0.20, 0.31, 0.26, .08. Find
the probabilities that a particular surgery will be rated
\begin{enumerate}
	\item complex or very complex;
	\item neither very complex nor very simple;
	\item routine or complex
	\item routine or simple
\end{enumerate}
\solution
%\input{exemplar/11/16/3/8(1)/main.tex}
\item A card is selected from a pack of 52 cards.
\begin{enumerate}[label=(\alph*)]
    \item How many points are there in the sample space?
    \item Calculate the probability that the card is an ace of spades.
    \item Calculate the probability that the card is (i) an ace and (ii) black card.
\end{enumerate}
\solution
%\input{exemplar/11/16/3/4/main2.tex}
\item The probability that a non leap year selected at random will contain 53 sundays.
\\
\solution
%\input{exemplar/10/13/1/19/main.tex}
\item One of the four persons John, Rita, Aslam or Gurpreet will be promoted next
month. Consequently the sample space consists of four elementary outcomes
S = {John promoted, Rita promoted, Aslam promoted, Gurpreet promoted}
You are told that the chances of John’s promotion is same as that of Gurpreet,
Rita’s chances of promotion are twice as likely as Johns. Aslam’s chances are
four times that of John.
\begin{enumerate}
	\item Determine
	\begin{enumerate}
		\item P (John promoted)
		\item P (Rita promoted)
		\item P (Aslam promoted)
		\item P (Gurpreet promoted)
	\end{enumerate}
	\item If A = {John promoted or Gurpreet promoted}, find P (A).
\end{enumerate}
\solution
%\input{exemplar/11/16/3/10/main.tex}
\item A card is drawn from a deck of 52 cards. Find the probability of getting a king or a heart or a red card.\\
\solution
%\input{exemplar/11/16/3/15/main.tex}
\item The probability that a student will pass his examination is 0.73, the probability of
the student getting a compartment is 0.13, and the probability that the student will
either pass or get compartment is 0.96. State True or False.\\
\solution
%\input{exemplar/11/16/3/31/main.tex}
\item A card is selected from a pack of 52 cards\\
\begin{enumerate}[label=(\alph*)]
\item How many points are there in the sample space?
\item Calculate the probability that the cards is an ace of spades.
\item Calculate the probability that the card is (i) an ace (ii)black card.\\
\end{enumerate}
%\input{ncert/11/16/3/4_1/Prob_4.tex}
\item In a non-leap year, the probability of having 53 tuesdays or 53 wednesdays is\\
\solution
%\input{exemplar/11/16/3/18/main.tex}
\item There are 1000 sealed envelopes in a box, 10 of them contain a cash prize of
Rs 100 each, 100 of them contain a cash prize of Rs 50 each and 200 of them
contain a cash prize of Rs 10 each and rest do not contain any cash prize. If they
are well shuffled and an envelope is picked up out, what is the probability that it
contains no cash prize?\\
\solution
%\input{exemplar/10/13/3/34/main.tex}
\item 
A die is thrown and a card is selected at random from a deck of 52 playing cards. The probability of getting an even number on the die and a spade card.\\
\solution
%\input{exemplar/12/13/3/78/main.tex}
\item
If 4-digit numbers greater than 5,000 are randomly formed from the digits 0, 1, 3, 5, and 7, what is the probability of forming a number divisible by 5 when:
\begin{enumerate}
    \item The digits are repeated?
    \item The repetition of digits is not allowed?
\end{enumerate}
\solution
%\input{ncert/11/16/4/9/main.tex}
\item Consider the probability space $\brak{\Omega, \mathcal{G}, P}$ where $\Omega = [0,2]$ and $\mathcal{G} = \cbrak{\phi, \Omega, [0,1], (1,2]}$. Let $X$ and $Y$ be two functions on $\Omega$ defined as
\begin{align*}
    X(\omega) = 
    \begin{cases}
        1 & \text{if }\omega \in [0, 1]\\
        2 & \text{if }\omega \in (1, 2]
    \end{cases}
\end{align*}
and
\begin{align*}
    Y(\omega) = 
    \begin{cases}
        2 & \text{if }\omega \in [0, 1.5]\\
        3 & \text{if }\omega \in (1.5, 2].
    \end{cases}
\end{align*}
Then which one of the following statements is true?
\begin{enumerate}
    \item [(A)] $X$ is a random variable with respect to $\mathcal{G}$, but $Y$ is not a random variable with respect to $\mathcal{G}$.
    \item [(B)] $Y$ is a random variable with respect to $\mathcal{G}$, but $X$ is not a random variable with respect to $\mathcal{G}$.
    \item [(C)] Neither $X$ nor $Y$ is a random variable with respect to $\mathcal{G}$.
    \item [(D)] Both $X$ and $Y$ are random variables with respect to $\mathcal{G}$.
\end{enumerate} \hfill (GATE ST 2023)\\
\solution
%\input{gate/ST/2023/14/main.tex}
	\item  A die is loaded in such a way that each odd number is twice as likely to occur as
each even number. Find $P(G)$, where $G$ is the event that a number greater than
3 occurs on a single roll of the die.
\\
\solution
		%\input{exemplar/11/16/3/5/main.tex}
	\item All the jacks, queens and kings are removed from a deck of 52 playing cards. The remaining cards are well shuffled and then one card is drawn at random. Giving ace a value 1 similar value for other cards, find the probability that the card has a value 
		\begin{enumerate}
			\item 7
			\item greater than 7
			\item less than 7
		\end{enumerate}
		%\input{exemplar/10/13/3/30/main.tex}
  \item A Lot consists of 48 mobile phones of which 42 are good, 3 have only minor defects and 3 have major defects.Varnika will buy a phone if it is good but the trader will only buy a mobile if it has no major defects. One phone is selected at random from the lot. What is the probability that it is
\begin{enumerate}
	\item acceptable to Varnika?
            \item acceptable to the trader?
\end{enumerate}
\solution
	%\input{exemplar/10/13/3/40/main.tex}
 \item A student says that if you throw a die, it will show up 1 or not 1. Therefore, the probability of getting 1 and the probability of getting 'not 1' each is equal to $\frac{1}{2}$. Is this correct? Give reasons.\\
 \solution
        %\input{exemplar/10/13/2/9/main.tex}
   \item Four candidates A, B, C, D have ap-
plied for the assignment to coach a school cricket
team. If A is twice as likely to be selected as B, and
B and C are given about the same chance of being
selected, while C is twice as likely to be selected
as D, what are the probabilities that
\begin{enumerate}
\item C will be selected?
\item A will not be selected?
\end{enumerate}
	%\input{exemplar/11/16/3/9/main.tex}
 \item A bag contain 24 balls of which $x$ balls are red, $2x$ are white and $3x$ are blue. A ball is selected at random, What is the probability that it is
\begin{enumerate}[label=\alph*)]
\item not red ?
\item white ?
\end{enumerate}
%\input{exemplar/10/13/3/41/main.tex}
If the letters of the word ASSASSINATION are arranged at random. Find the Probability that
\begin{enumerate}[label=(\alph*)]
\item Four $S's$ come consecutively in the word
\item Two  $I's$ and two $N's$ come together
\item All $A's$ are not coming together
\item No two $A's$ are coming together
\end{enumerate}
%\input{exemplar/11/16/3/14/main.tex}
	\item One urn contains two black balls (labelled B1 and B2) and one white ball. A
	second urn contains one black ball and two white balls (labelled W1 and W2).
	Suppose the following experiment is performed. One of the two urns is chosen
	at random. Next a ball is randomly chosen from the urn. Then a second ball is
	chosen at random from the same urn without replacing the first ball.
	
	\begin{enumerate}
	\item What is the probability that two black balls are chosen?
	
	\item What is the probability that two balls of opposite colour are chosen?
	\end{enumerate}
	\solution
	%\input{exemplar/11/16/3/12/main1.tex}
\end{enumerate}

	\item A bag contains $5$ red balls and some blue balls. If the probability of drawing a blue ball is double that if a red ball, determine the number of blue balls in the bag. 
		\\
\solution
		%\begin{enumerate}[label=\thesection.\arabic*,ref=\thesection.\theenumi]
	\item One card is drawn from a well-shuffled deck of 52 cards. Find the probability of getting
\begin{enumerate}
\item A king of red colour 
\item A face card 
\item A red face card
\item The jack of hearts
\item A spade
\item The queen of diamonds

\end{enumerate}
\solution
		%\input{ncert/10/15/1/14/main.tex}
	\item Five cards—the ten, jack, queen, king and ace of diamonds, are well-shuffled with their face downwards. One card is then picked up at random.
\begin{enumerate}
\item
What is the probability that the card is the queen? 
\item
If the queen is drawn and put aside, what is the probability that the second card picked up is (a) an ace? (b) a queen?\\
\end{enumerate}
\solution
		%\input{ncert/10/15/1/15/defs.tex}
	\item A bag contains $5$ red balls and some blue balls. If the probability of drawing a blue ball is double that if a red ball, determine the number of blue balls in the bag. 
		\\
\solution
		%\input{ncert/10/15/2/3/defs.tex}
	\item A card is selected from a pack of 52 cards.
 \begin{enumerate}[label=(\alph*)] 
                 \item How many points are there in the sample space?
                 \item Calculate the probability that the card is an ace of spades.
                 \item Calculate the probability that the card is (i) an ace and (ii) black card.
 \end{enumerate}
\solution
		%\input{ncert/11/16/3/4/main.tex}
\item Four cards are drawn from a well-shuffled deck of 52 cards. What is the probability of obtaining 3 diamonds and one spade.
\\
\solution
		%\input{ncert/11/16/4/2/defs.tex}
\item In a certain lottery 10,000 tickets are sold and ten equal prizes are awarded. What is the probability of not getting a prize if you buy (a) one ticket (b) two tickets (c) 10 tickets ?	
\\
\solution
		%\input{ncert/11/16/4/4/defs.tex}
		%
\item 
Out of 100 students, two sections of 40 and 60 are formed. If you and your friend are among the 100 students, what is the probability that
\begin{enumerate}
\item you both enter the same section?
\item you both enter the different sections?
\end{enumerate}
\solution
		%\input{ncert/11/16/4/5/defs.tex}
	\item 
The number lock of a suitcase has 4 wheels each labelled with ten digits i.e. from 0 to 9.The lock opens with a sequence of four digits with no repeats.What is the probability of a person getting the right sequence to open the suitcase.
\\
\solution
		%\input{ncert/11/16/4/10/defs.tex}
		%
\item 
Two cards are drawn at random and without replacement from a pack of 52 playing cards. Find the probability that both the cards are black.
\\
\solution
		%\input{ncert/12/13/2/2/defs.tex}
		\item A box of oranges is inspected by examining three randomly selected oranges drawn without replacement. If all the three oranges are good, the box is approved for sale, otherwise, it is rejected. Find the probability that a box containing 15 oranges out of which 12 are good and 3 are bad ones will be approved for sale.
		\label{ncert/12/13/2/3/defs.tex}
		\item Two balls are drawn at random with replacement from a box containing 10 black and 8 red balls. Find the probability that
		\label{ncert/12/13/2/12}
\begin{enumerate}
\item both balls are red.
\item first ball is black and second is red.
\item one of them is black and other is red.
\end{enumerate}

\item In a hostel, 60\% of the students read Hindi newspaper, 40\% read English newspaper and 20\% read both Hindi and English newspapers. A student is selected at random.
		\label{ncert/12/13/2/15}
\begin{enumerate}
\item Find the probability that she reads neither Hindi nor English newspapers.
\item If she reads Hindi newspaper, find the probability that she reads English newspaper.
\item If she reads English newspaper, find the probability that she reads Hindi newspaper.\\
\end{enumerate}
\item The probability of obtaining an even prime number on each die, when a pair of dice is rolled is 
\begin{enumerate}
    \item $0$ 
    
    \item $\frac{1}{3}$ 
    
    \item $\frac{1}{12}$ 
    
    \item $\frac{1}{36}$ 
\end{enumerate}
\solution
		%\input{ncert/12/13/2/17/defs.tex}
	\item A bag contains 4 red and 4 black balls, another bag contains 2 red and 6 black balls. One of the two bags is selected at random and a ball is drawn from the bag which is found to be red. Find the probability that the ball is drawn from the first bag.
\\
\solution
		%\input{ncert/12/13/3/2/main.tex}
  \item
  Cards with numbers 2 to 101 are placed in a box. A card is selected at random.Find the probability that the card has
\begin{enumerate}[label=(\roman*)]
	\item an even number 
	\item a square number
\end{enumerate}
\solution
%\input{exemplar/10/13/3/32/main.tex}
\item
The king, queen and jack of clubs are removed from a deck of 52 playing cards and then well shuffled. Now one card is drawn at random from the remaining cards.  Determine the probability that the card is
\begin{enumerate}[label=(\roman*)]
\item a club
\item 10 of hearts
\end{enumerate}
\solution
%\input{exemplar/10/13/3/29/main.tex}
\item A team of medical students doing their internship have to assist during surgeries
at a city hospital. The probabilities of surgeries rated as very complex, complex,
routine, simple or very simple are respectively, 0.15, 0.20, 0.31, 0.26, .08. Find
the probabilities that a particular surgery will be rated
\begin{enumerate}
	\item complex or very complex;
	\item neither very complex nor very simple;
	\item routine or complex
	\item routine or simple
\end{enumerate}
\solution
%\input{exemplar/11/16/3/8(1)/main.tex}
\item A card is selected from a pack of 52 cards.
\begin{enumerate}[label=(\alph*)]
    \item How many points are there in the sample space?
    \item Calculate the probability that the card is an ace of spades.
    \item Calculate the probability that the card is (i) an ace and (ii) black card.
\end{enumerate}
\solution
%\input{exemplar/11/16/3/4/main2.tex}
\item The probability that a non leap year selected at random will contain 53 sundays.
\\
\solution
%\input{exemplar/10/13/1/19/main.tex}
\item One of the four persons John, Rita, Aslam or Gurpreet will be promoted next
month. Consequently the sample space consists of four elementary outcomes
S = {John promoted, Rita promoted, Aslam promoted, Gurpreet promoted}
You are told that the chances of John’s promotion is same as that of Gurpreet,
Rita’s chances of promotion are twice as likely as Johns. Aslam’s chances are
four times that of John.
\begin{enumerate}
	\item Determine
	\begin{enumerate}
		\item P (John promoted)
		\item P (Rita promoted)
		\item P (Aslam promoted)
		\item P (Gurpreet promoted)
	\end{enumerate}
	\item If A = {John promoted or Gurpreet promoted}, find P (A).
\end{enumerate}
\solution
%\input{exemplar/11/16/3/10/main.tex}
\item A card is drawn from a deck of 52 cards. Find the probability of getting a king or a heart or a red card.\\
\solution
%\input{exemplar/11/16/3/15/main.tex}
\item The probability that a student will pass his examination is 0.73, the probability of
the student getting a compartment is 0.13, and the probability that the student will
either pass or get compartment is 0.96. State True or False.\\
\solution
%\input{exemplar/11/16/3/31/main.tex}
\item A card is selected from a pack of 52 cards\\
\begin{enumerate}[label=(\alph*)]
\item How many points are there in the sample space?
\item Calculate the probability that the cards is an ace of spades.
\item Calculate the probability that the card is (i) an ace (ii)black card.\\
\end{enumerate}
%\input{ncert/11/16/3/4_1/Prob_4.tex}
\item In a non-leap year, the probability of having 53 tuesdays or 53 wednesdays is\\
\solution
%\input{exemplar/11/16/3/18/main.tex}
\item There are 1000 sealed envelopes in a box, 10 of them contain a cash prize of
Rs 100 each, 100 of them contain a cash prize of Rs 50 each and 200 of them
contain a cash prize of Rs 10 each and rest do not contain any cash prize. If they
are well shuffled and an envelope is picked up out, what is the probability that it
contains no cash prize?\\
\solution
%\input{exemplar/10/13/3/34/main.tex}
\item 
A die is thrown and a card is selected at random from a deck of 52 playing cards. The probability of getting an even number on the die and a spade card.\\
\solution
%\input{exemplar/12/13/3/78/main.tex}
\item
If 4-digit numbers greater than 5,000 are randomly formed from the digits 0, 1, 3, 5, and 7, what is the probability of forming a number divisible by 5 when:
\begin{enumerate}
    \item The digits are repeated?
    \item The repetition of digits is not allowed?
\end{enumerate}
\solution
%\input{ncert/11/16/4/9/main.tex}
\item Consider the probability space $\brak{\Omega, \mathcal{G}, P}$ where $\Omega = [0,2]$ and $\mathcal{G} = \cbrak{\phi, \Omega, [0,1], (1,2]}$. Let $X$ and $Y$ be two functions on $\Omega$ defined as
\begin{align*}
    X(\omega) = 
    \begin{cases}
        1 & \text{if }\omega \in [0, 1]\\
        2 & \text{if }\omega \in (1, 2]
    \end{cases}
\end{align*}
and
\begin{align*}
    Y(\omega) = 
    \begin{cases}
        2 & \text{if }\omega \in [0, 1.5]\\
        3 & \text{if }\omega \in (1.5, 2].
    \end{cases}
\end{align*}
Then which one of the following statements is true?
\begin{enumerate}
    \item [(A)] $X$ is a random variable with respect to $\mathcal{G}$, but $Y$ is not a random variable with respect to $\mathcal{G}$.
    \item [(B)] $Y$ is a random variable with respect to $\mathcal{G}$, but $X$ is not a random variable with respect to $\mathcal{G}$.
    \item [(C)] Neither $X$ nor $Y$ is a random variable with respect to $\mathcal{G}$.
    \item [(D)] Both $X$ and $Y$ are random variables with respect to $\mathcal{G}$.
\end{enumerate} \hfill (GATE ST 2023)\\
\solution
%\input{gate/ST/2023/14/main.tex}
	\item  A die is loaded in such a way that each odd number is twice as likely to occur as
each even number. Find $P(G)$, where $G$ is the event that a number greater than
3 occurs on a single roll of the die.
\\
\solution
		%\input{exemplar/11/16/3/5/main.tex}
	\item All the jacks, queens and kings are removed from a deck of 52 playing cards. The remaining cards are well shuffled and then one card is drawn at random. Giving ace a value 1 similar value for other cards, find the probability that the card has a value 
		\begin{enumerate}
			\item 7
			\item greater than 7
			\item less than 7
		\end{enumerate}
		%\input{exemplar/10/13/3/30/main.tex}
  \item A Lot consists of 48 mobile phones of which 42 are good, 3 have only minor defects and 3 have major defects.Varnika will buy a phone if it is good but the trader will only buy a mobile if it has no major defects. One phone is selected at random from the lot. What is the probability that it is
\begin{enumerate}
	\item acceptable to Varnika?
            \item acceptable to the trader?
\end{enumerate}
\solution
	%\input{exemplar/10/13/3/40/main.tex}
 \item A student says that if you throw a die, it will show up 1 or not 1. Therefore, the probability of getting 1 and the probability of getting 'not 1' each is equal to $\frac{1}{2}$. Is this correct? Give reasons.\\
 \solution
        %\input{exemplar/10/13/2/9/main.tex}
   \item Four candidates A, B, C, D have ap-
plied for the assignment to coach a school cricket
team. If A is twice as likely to be selected as B, and
B and C are given about the same chance of being
selected, while C is twice as likely to be selected
as D, what are the probabilities that
\begin{enumerate}
\item C will be selected?
\item A will not be selected?
\end{enumerate}
	%\input{exemplar/11/16/3/9/main.tex}
 \item A bag contain 24 balls of which $x$ balls are red, $2x$ are white and $3x$ are blue. A ball is selected at random, What is the probability that it is
\begin{enumerate}[label=\alph*)]
\item not red ?
\item white ?
\end{enumerate}
%\input{exemplar/10/13/3/41/main.tex}
If the letters of the word ASSASSINATION are arranged at random. Find the Probability that
\begin{enumerate}[label=(\alph*)]
\item Four $S's$ come consecutively in the word
\item Two  $I's$ and two $N's$ come together
\item All $A's$ are not coming together
\item No two $A's$ are coming together
\end{enumerate}
%\input{exemplar/11/16/3/14/main.tex}
	\item One urn contains two black balls (labelled B1 and B2) and one white ball. A
	second urn contains one black ball and two white balls (labelled W1 and W2).
	Suppose the following experiment is performed. One of the two urns is chosen
	at random. Next a ball is randomly chosen from the urn. Then a second ball is
	chosen at random from the same urn without replacing the first ball.
	
	\begin{enumerate}
	\item What is the probability that two black balls are chosen?
	
	\item What is the probability that two balls of opposite colour are chosen?
	\end{enumerate}
	\solution
	%\input{exemplar/11/16/3/12/main1.tex}
\end{enumerate}

	\item A card is selected from a pack of 52 cards.
 \begin{enumerate}[label=(\alph*)] 
                 \item How many points are there in the sample space?
                 \item Calculate the probability that the card is an ace of spades.
                 \item Calculate the probability that the card is (i) an ace and (ii) black card.
 \end{enumerate}
\solution
		%\begin{table}[H]
	\centering
\begin{tabular}{|c|c|c|}
\hline
Random variable &Value &Definition\\ \hline
\multirow{3}{*}{X} &0 &Slips of Rs 1\\
&1 &Slips of Rs 5\\
&2 &Slips of Rs 13\\ \hline
\multirow{2}{*}{Y} &0 &Box A\\
&1 &Box B\\\hline
\end{tabular}
\caption{}
\label{tab:Distribution}
\end{table}
See \tabref{tab:Distribution}.
\begin{align}
p_{Y}\brak{k}= \begin{cases} 
      \frac{1}{3} & {k=0} \\
      \frac{2}{3 }& {k=1} 
   \end{cases}
   \\
p_{Y|X}\brak{0|0} = \frac{19}{25}\, 
p_{Y|X}\brak{0|1} = \frac{6}{25}\,
p_{Y|X}\brak{1|0} = \frac{45}{50}\,
p_{Y|X}\brak{1|2} = \frac{5}{50}
\end{align}
The desired probability is the probability that a slip drawn at random is marked other than Rs 1,
\begin{align}
&=1-p_X\brak{0}\\
&= p_X(1) + p_X(2)
\end{align}
Using Bayes theorem,
\begin{align}
&= p_Y\brak{0} \times \pr{Y=0 | X=1} + p_Y\brak{1} \times \pr{Y=1|X=2}\\
&=\frac{1}{3} \times \frac{6}{25} + \frac{2}{3} \times \frac{5}{50}\\
&=\frac{11}{75}
\end{align}

\newpage

%\tableofcontents

\bigskip

\renewcommand{\thefigure}{\theenumi}
\renewcommand{\thetable}{\theenumi}
%\renewcommand{\theequation}{\theenumi}

%\begin{abstract}
%%\boldmath
%In this letter, an algorithm for evaluating the exact analytical bit error rate  (BER)  for the piecewise linear (PL) combiner for  multiple relays is presented. Previous results were available only for upto three relays. The algorithm is unique in the sense that  the actual mathematical expressions, that are prohibitively large, need not be explicitly obtained. The diversity gain due to multiple relays is shown through plots of the analytical BER, well supported by simulations. 
%
%\end{abstract}
% IEEEtran.cls defaults to using nonbold math in the Abstract.
% This preserves the distinction between vectors and scalars. However,
% if the journal you are submitting to favors bold math in the abstract,
% then you can use LaTeX's standard command \boldmath at the very start
% of the abstract to achieve this. Many IEEE journals frown on math
% in the abstract anyway.

% Note that keywords are not normally used for peerreview papers.
%\begin{IEEEkeywords}
%Cooperative diversity, decode and forward, piecewise linear
%\end{IEEEkeywords}



% For peer review papers, you can put extra information on the cover
% page as needed:
% \ifCLASSOPTIONpeerreview
% \begin{center} \bfseries EDICS Category: 3-BBND \end{center}
% \fi
%
% For peerreview papers, this IEEEtran command inserts a page break and
% creates the second title. It will be ignored for other modes.
%\IEEEpeerreviewmaketitle




\item Four cards are drawn from a well-shuffled deck of 52 cards. What is the probability of obtaining 3 diamonds and one spade.
\\
\solution
		%\begin{enumerate}[label=\thesection.\arabic*,ref=\thesection.\theenumi]
	\item One card is drawn from a well-shuffled deck of 52 cards. Find the probability of getting
\begin{enumerate}
\item A king of red colour 
\item A face card 
\item A red face card
\item The jack of hearts
\item A spade
\item The queen of diamonds

\end{enumerate}
\solution
		%\input{ncert/10/15/1/14/main.tex}
	\item Five cards—the ten, jack, queen, king and ace of diamonds, are well-shuffled with their face downwards. One card is then picked up at random.
\begin{enumerate}
\item
What is the probability that the card is the queen? 
\item
If the queen is drawn and put aside, what is the probability that the second card picked up is (a) an ace? (b) a queen?\\
\end{enumerate}
\solution
		%\input{ncert/10/15/1/15/defs.tex}
	\item A bag contains $5$ red balls and some blue balls. If the probability of drawing a blue ball is double that if a red ball, determine the number of blue balls in the bag. 
		\\
\solution
		%\input{ncert/10/15/2/3/defs.tex}
	\item A card is selected from a pack of 52 cards.
 \begin{enumerate}[label=(\alph*)] 
                 \item How many points are there in the sample space?
                 \item Calculate the probability that the card is an ace of spades.
                 \item Calculate the probability that the card is (i) an ace and (ii) black card.
 \end{enumerate}
\solution
		%\input{ncert/11/16/3/4/main.tex}
\item Four cards are drawn from a well-shuffled deck of 52 cards. What is the probability of obtaining 3 diamonds and one spade.
\\
\solution
		%\input{ncert/11/16/4/2/defs.tex}
\item In a certain lottery 10,000 tickets are sold and ten equal prizes are awarded. What is the probability of not getting a prize if you buy (a) one ticket (b) two tickets (c) 10 tickets ?	
\\
\solution
		%\input{ncert/11/16/4/4/defs.tex}
		%
\item 
Out of 100 students, two sections of 40 and 60 are formed. If you and your friend are among the 100 students, what is the probability that
\begin{enumerate}
\item you both enter the same section?
\item you both enter the different sections?
\end{enumerate}
\solution
		%\input{ncert/11/16/4/5/defs.tex}
	\item 
The number lock of a suitcase has 4 wheels each labelled with ten digits i.e. from 0 to 9.The lock opens with a sequence of four digits with no repeats.What is the probability of a person getting the right sequence to open the suitcase.
\\
\solution
		%\input{ncert/11/16/4/10/defs.tex}
		%
\item 
Two cards are drawn at random and without replacement from a pack of 52 playing cards. Find the probability that both the cards are black.
\\
\solution
		%\input{ncert/12/13/2/2/defs.tex}
		\item A box of oranges is inspected by examining three randomly selected oranges drawn without replacement. If all the three oranges are good, the box is approved for sale, otherwise, it is rejected. Find the probability that a box containing 15 oranges out of which 12 are good and 3 are bad ones will be approved for sale.
		\label{ncert/12/13/2/3/defs.tex}
		\item Two balls are drawn at random with replacement from a box containing 10 black and 8 red balls. Find the probability that
		\label{ncert/12/13/2/12}
\begin{enumerate}
\item both balls are red.
\item first ball is black and second is red.
\item one of them is black and other is red.
\end{enumerate}

\item In a hostel, 60\% of the students read Hindi newspaper, 40\% read English newspaper and 20\% read both Hindi and English newspapers. A student is selected at random.
		\label{ncert/12/13/2/15}
\begin{enumerate}
\item Find the probability that she reads neither Hindi nor English newspapers.
\item If she reads Hindi newspaper, find the probability that she reads English newspaper.
\item If she reads English newspaper, find the probability that she reads Hindi newspaper.\\
\end{enumerate}
\item The probability of obtaining an even prime number on each die, when a pair of dice is rolled is 
\begin{enumerate}
    \item $0$ 
    
    \item $\frac{1}{3}$ 
    
    \item $\frac{1}{12}$ 
    
    \item $\frac{1}{36}$ 
\end{enumerate}
\solution
		%\input{ncert/12/13/2/17/defs.tex}
	\item A bag contains 4 red and 4 black balls, another bag contains 2 red and 6 black balls. One of the two bags is selected at random and a ball is drawn from the bag which is found to be red. Find the probability that the ball is drawn from the first bag.
\\
\solution
		%\input{ncert/12/13/3/2/main.tex}
  \item
  Cards with numbers 2 to 101 are placed in a box. A card is selected at random.Find the probability that the card has
\begin{enumerate}[label=(\roman*)]
	\item an even number 
	\item a square number
\end{enumerate}
\solution
%\input{exemplar/10/13/3/32/main.tex}
\item
The king, queen and jack of clubs are removed from a deck of 52 playing cards and then well shuffled. Now one card is drawn at random from the remaining cards.  Determine the probability that the card is
\begin{enumerate}[label=(\roman*)]
\item a club
\item 10 of hearts
\end{enumerate}
\solution
%\input{exemplar/10/13/3/29/main.tex}
\item A team of medical students doing their internship have to assist during surgeries
at a city hospital. The probabilities of surgeries rated as very complex, complex,
routine, simple or very simple are respectively, 0.15, 0.20, 0.31, 0.26, .08. Find
the probabilities that a particular surgery will be rated
\begin{enumerate}
	\item complex or very complex;
	\item neither very complex nor very simple;
	\item routine or complex
	\item routine or simple
\end{enumerate}
\solution
%\input{exemplar/11/16/3/8(1)/main.tex}
\item A card is selected from a pack of 52 cards.
\begin{enumerate}[label=(\alph*)]
    \item How many points are there in the sample space?
    \item Calculate the probability that the card is an ace of spades.
    \item Calculate the probability that the card is (i) an ace and (ii) black card.
\end{enumerate}
\solution
%\input{exemplar/11/16/3/4/main2.tex}
\item The probability that a non leap year selected at random will contain 53 sundays.
\\
\solution
%\input{exemplar/10/13/1/19/main.tex}
\item One of the four persons John, Rita, Aslam or Gurpreet will be promoted next
month. Consequently the sample space consists of four elementary outcomes
S = {John promoted, Rita promoted, Aslam promoted, Gurpreet promoted}
You are told that the chances of John’s promotion is same as that of Gurpreet,
Rita’s chances of promotion are twice as likely as Johns. Aslam’s chances are
four times that of John.
\begin{enumerate}
	\item Determine
	\begin{enumerate}
		\item P (John promoted)
		\item P (Rita promoted)
		\item P (Aslam promoted)
		\item P (Gurpreet promoted)
	\end{enumerate}
	\item If A = {John promoted or Gurpreet promoted}, find P (A).
\end{enumerate}
\solution
%\input{exemplar/11/16/3/10/main.tex}
\item A card is drawn from a deck of 52 cards. Find the probability of getting a king or a heart or a red card.\\
\solution
%\input{exemplar/11/16/3/15/main.tex}
\item The probability that a student will pass his examination is 0.73, the probability of
the student getting a compartment is 0.13, and the probability that the student will
either pass or get compartment is 0.96. State True or False.\\
\solution
%\input{exemplar/11/16/3/31/main.tex}
\item A card is selected from a pack of 52 cards\\
\begin{enumerate}[label=(\alph*)]
\item How many points are there in the sample space?
\item Calculate the probability that the cards is an ace of spades.
\item Calculate the probability that the card is (i) an ace (ii)black card.\\
\end{enumerate}
%\input{ncert/11/16/3/4_1/Prob_4.tex}
\item In a non-leap year, the probability of having 53 tuesdays or 53 wednesdays is\\
\solution
%\input{exemplar/11/16/3/18/main.tex}
\item There are 1000 sealed envelopes in a box, 10 of them contain a cash prize of
Rs 100 each, 100 of them contain a cash prize of Rs 50 each and 200 of them
contain a cash prize of Rs 10 each and rest do not contain any cash prize. If they
are well shuffled and an envelope is picked up out, what is the probability that it
contains no cash prize?\\
\solution
%\input{exemplar/10/13/3/34/main.tex}
\item 
A die is thrown and a card is selected at random from a deck of 52 playing cards. The probability of getting an even number on the die and a spade card.\\
\solution
%\input{exemplar/12/13/3/78/main.tex}
\item
If 4-digit numbers greater than 5,000 are randomly formed from the digits 0, 1, 3, 5, and 7, what is the probability of forming a number divisible by 5 when:
\begin{enumerate}
    \item The digits are repeated?
    \item The repetition of digits is not allowed?
\end{enumerate}
\solution
%\input{ncert/11/16/4/9/main.tex}
\item Consider the probability space $\brak{\Omega, \mathcal{G}, P}$ where $\Omega = [0,2]$ and $\mathcal{G} = \cbrak{\phi, \Omega, [0,1], (1,2]}$. Let $X$ and $Y$ be two functions on $\Omega$ defined as
\begin{align*}
    X(\omega) = 
    \begin{cases}
        1 & \text{if }\omega \in [0, 1]\\
        2 & \text{if }\omega \in (1, 2]
    \end{cases}
\end{align*}
and
\begin{align*}
    Y(\omega) = 
    \begin{cases}
        2 & \text{if }\omega \in [0, 1.5]\\
        3 & \text{if }\omega \in (1.5, 2].
    \end{cases}
\end{align*}
Then which one of the following statements is true?
\begin{enumerate}
    \item [(A)] $X$ is a random variable with respect to $\mathcal{G}$, but $Y$ is not a random variable with respect to $\mathcal{G}$.
    \item [(B)] $Y$ is a random variable with respect to $\mathcal{G}$, but $X$ is not a random variable with respect to $\mathcal{G}$.
    \item [(C)] Neither $X$ nor $Y$ is a random variable with respect to $\mathcal{G}$.
    \item [(D)] Both $X$ and $Y$ are random variables with respect to $\mathcal{G}$.
\end{enumerate} \hfill (GATE ST 2023)\\
\solution
%\input{gate/ST/2023/14/main.tex}
	\item  A die is loaded in such a way that each odd number is twice as likely to occur as
each even number. Find $P(G)$, where $G$ is the event that a number greater than
3 occurs on a single roll of the die.
\\
\solution
		%\input{exemplar/11/16/3/5/main.tex}
	\item All the jacks, queens and kings are removed from a deck of 52 playing cards. The remaining cards are well shuffled and then one card is drawn at random. Giving ace a value 1 similar value for other cards, find the probability that the card has a value 
		\begin{enumerate}
			\item 7
			\item greater than 7
			\item less than 7
		\end{enumerate}
		%\input{exemplar/10/13/3/30/main.tex}
  \item A Lot consists of 48 mobile phones of which 42 are good, 3 have only minor defects and 3 have major defects.Varnika will buy a phone if it is good but the trader will only buy a mobile if it has no major defects. One phone is selected at random from the lot. What is the probability that it is
\begin{enumerate}
	\item acceptable to Varnika?
            \item acceptable to the trader?
\end{enumerate}
\solution
	%\input{exemplar/10/13/3/40/main.tex}
 \item A student says that if you throw a die, it will show up 1 or not 1. Therefore, the probability of getting 1 and the probability of getting 'not 1' each is equal to $\frac{1}{2}$. Is this correct? Give reasons.\\
 \solution
        %\input{exemplar/10/13/2/9/main.tex}
   \item Four candidates A, B, C, D have ap-
plied for the assignment to coach a school cricket
team. If A is twice as likely to be selected as B, and
B and C are given about the same chance of being
selected, while C is twice as likely to be selected
as D, what are the probabilities that
\begin{enumerate}
\item C will be selected?
\item A will not be selected?
\end{enumerate}
	%\input{exemplar/11/16/3/9/main.tex}
 \item A bag contain 24 balls of which $x$ balls are red, $2x$ are white and $3x$ are blue. A ball is selected at random, What is the probability that it is
\begin{enumerate}[label=\alph*)]
\item not red ?
\item white ?
\end{enumerate}
%\input{exemplar/10/13/3/41/main.tex}
If the letters of the word ASSASSINATION are arranged at random. Find the Probability that
\begin{enumerate}[label=(\alph*)]
\item Four $S's$ come consecutively in the word
\item Two  $I's$ and two $N's$ come together
\item All $A's$ are not coming together
\item No two $A's$ are coming together
\end{enumerate}
%\input{exemplar/11/16/3/14/main.tex}
	\item One urn contains two black balls (labelled B1 and B2) and one white ball. A
	second urn contains one black ball and two white balls (labelled W1 and W2).
	Suppose the following experiment is performed. One of the two urns is chosen
	at random. Next a ball is randomly chosen from the urn. Then a second ball is
	chosen at random from the same urn without replacing the first ball.
	
	\begin{enumerate}
	\item What is the probability that two black balls are chosen?
	
	\item What is the probability that two balls of opposite colour are chosen?
	\end{enumerate}
	\solution
	%\input{exemplar/11/16/3/12/main1.tex}
\end{enumerate}

\item In a certain lottery 10,000 tickets are sold and ten equal prizes are awarded. What is the probability of not getting a prize if you buy (a) one ticket (b) two tickets (c) 10 tickets ?	
\\
\solution
		%\begin{enumerate}[label=\thesection.\arabic*,ref=\thesection.\theenumi]
	\item One card is drawn from a well-shuffled deck of 52 cards. Find the probability of getting
\begin{enumerate}
\item A king of red colour 
\item A face card 
\item A red face card
\item The jack of hearts
\item A spade
\item The queen of diamonds

\end{enumerate}
\solution
		%\input{ncert/10/15/1/14/main.tex}
	\item Five cards—the ten, jack, queen, king and ace of diamonds, are well-shuffled with their face downwards. One card is then picked up at random.
\begin{enumerate}
\item
What is the probability that the card is the queen? 
\item
If the queen is drawn and put aside, what is the probability that the second card picked up is (a) an ace? (b) a queen?\\
\end{enumerate}
\solution
		%\input{ncert/10/15/1/15/defs.tex}
	\item A bag contains $5$ red balls and some blue balls. If the probability of drawing a blue ball is double that if a red ball, determine the number of blue balls in the bag. 
		\\
\solution
		%\input{ncert/10/15/2/3/defs.tex}
	\item A card is selected from a pack of 52 cards.
 \begin{enumerate}[label=(\alph*)] 
                 \item How many points are there in the sample space?
                 \item Calculate the probability that the card is an ace of spades.
                 \item Calculate the probability that the card is (i) an ace and (ii) black card.
 \end{enumerate}
\solution
		%\input{ncert/11/16/3/4/main.tex}
\item Four cards are drawn from a well-shuffled deck of 52 cards. What is the probability of obtaining 3 diamonds and one spade.
\\
\solution
		%\input{ncert/11/16/4/2/defs.tex}
\item In a certain lottery 10,000 tickets are sold and ten equal prizes are awarded. What is the probability of not getting a prize if you buy (a) one ticket (b) two tickets (c) 10 tickets ?	
\\
\solution
		%\input{ncert/11/16/4/4/defs.tex}
		%
\item 
Out of 100 students, two sections of 40 and 60 are formed. If you and your friend are among the 100 students, what is the probability that
\begin{enumerate}
\item you both enter the same section?
\item you both enter the different sections?
\end{enumerate}
\solution
		%\input{ncert/11/16/4/5/defs.tex}
	\item 
The number lock of a suitcase has 4 wheels each labelled with ten digits i.e. from 0 to 9.The lock opens with a sequence of four digits with no repeats.What is the probability of a person getting the right sequence to open the suitcase.
\\
\solution
		%\input{ncert/11/16/4/10/defs.tex}
		%
\item 
Two cards are drawn at random and without replacement from a pack of 52 playing cards. Find the probability that both the cards are black.
\\
\solution
		%\input{ncert/12/13/2/2/defs.tex}
		\item A box of oranges is inspected by examining three randomly selected oranges drawn without replacement. If all the three oranges are good, the box is approved for sale, otherwise, it is rejected. Find the probability that a box containing 15 oranges out of which 12 are good and 3 are bad ones will be approved for sale.
		\label{ncert/12/13/2/3/defs.tex}
		\item Two balls are drawn at random with replacement from a box containing 10 black and 8 red balls. Find the probability that
		\label{ncert/12/13/2/12}
\begin{enumerate}
\item both balls are red.
\item first ball is black and second is red.
\item one of them is black and other is red.
\end{enumerate}

\item In a hostel, 60\% of the students read Hindi newspaper, 40\% read English newspaper and 20\% read both Hindi and English newspapers. A student is selected at random.
		\label{ncert/12/13/2/15}
\begin{enumerate}
\item Find the probability that she reads neither Hindi nor English newspapers.
\item If she reads Hindi newspaper, find the probability that she reads English newspaper.
\item If she reads English newspaper, find the probability that she reads Hindi newspaper.\\
\end{enumerate}
\item The probability of obtaining an even prime number on each die, when a pair of dice is rolled is 
\begin{enumerate}
    \item $0$ 
    
    \item $\frac{1}{3}$ 
    
    \item $\frac{1}{12}$ 
    
    \item $\frac{1}{36}$ 
\end{enumerate}
\solution
		%\input{ncert/12/13/2/17/defs.tex}
	\item A bag contains 4 red and 4 black balls, another bag contains 2 red and 6 black balls. One of the two bags is selected at random and a ball is drawn from the bag which is found to be red. Find the probability that the ball is drawn from the first bag.
\\
\solution
		%\input{ncert/12/13/3/2/main.tex}
  \item
  Cards with numbers 2 to 101 are placed in a box. A card is selected at random.Find the probability that the card has
\begin{enumerate}[label=(\roman*)]
	\item an even number 
	\item a square number
\end{enumerate}
\solution
%\input{exemplar/10/13/3/32/main.tex}
\item
The king, queen and jack of clubs are removed from a deck of 52 playing cards and then well shuffled. Now one card is drawn at random from the remaining cards.  Determine the probability that the card is
\begin{enumerate}[label=(\roman*)]
\item a club
\item 10 of hearts
\end{enumerate}
\solution
%\input{exemplar/10/13/3/29/main.tex}
\item A team of medical students doing their internship have to assist during surgeries
at a city hospital. The probabilities of surgeries rated as very complex, complex,
routine, simple or very simple are respectively, 0.15, 0.20, 0.31, 0.26, .08. Find
the probabilities that a particular surgery will be rated
\begin{enumerate}
	\item complex or very complex;
	\item neither very complex nor very simple;
	\item routine or complex
	\item routine or simple
\end{enumerate}
\solution
%\input{exemplar/11/16/3/8(1)/main.tex}
\item A card is selected from a pack of 52 cards.
\begin{enumerate}[label=(\alph*)]
    \item How many points are there in the sample space?
    \item Calculate the probability that the card is an ace of spades.
    \item Calculate the probability that the card is (i) an ace and (ii) black card.
\end{enumerate}
\solution
%\input{exemplar/11/16/3/4/main2.tex}
\item The probability that a non leap year selected at random will contain 53 sundays.
\\
\solution
%\input{exemplar/10/13/1/19/main.tex}
\item One of the four persons John, Rita, Aslam or Gurpreet will be promoted next
month. Consequently the sample space consists of four elementary outcomes
S = {John promoted, Rita promoted, Aslam promoted, Gurpreet promoted}
You are told that the chances of John’s promotion is same as that of Gurpreet,
Rita’s chances of promotion are twice as likely as Johns. Aslam’s chances are
four times that of John.
\begin{enumerate}
	\item Determine
	\begin{enumerate}
		\item P (John promoted)
		\item P (Rita promoted)
		\item P (Aslam promoted)
		\item P (Gurpreet promoted)
	\end{enumerate}
	\item If A = {John promoted or Gurpreet promoted}, find P (A).
\end{enumerate}
\solution
%\input{exemplar/11/16/3/10/main.tex}
\item A card is drawn from a deck of 52 cards. Find the probability of getting a king or a heart or a red card.\\
\solution
%\input{exemplar/11/16/3/15/main.tex}
\item The probability that a student will pass his examination is 0.73, the probability of
the student getting a compartment is 0.13, and the probability that the student will
either pass or get compartment is 0.96. State True or False.\\
\solution
%\input{exemplar/11/16/3/31/main.tex}
\item A card is selected from a pack of 52 cards\\
\begin{enumerate}[label=(\alph*)]
\item How many points are there in the sample space?
\item Calculate the probability that the cards is an ace of spades.
\item Calculate the probability that the card is (i) an ace (ii)black card.\\
\end{enumerate}
%\input{ncert/11/16/3/4_1/Prob_4.tex}
\item In a non-leap year, the probability of having 53 tuesdays or 53 wednesdays is\\
\solution
%\input{exemplar/11/16/3/18/main.tex}
\item There are 1000 sealed envelopes in a box, 10 of them contain a cash prize of
Rs 100 each, 100 of them contain a cash prize of Rs 50 each and 200 of them
contain a cash prize of Rs 10 each and rest do not contain any cash prize. If they
are well shuffled and an envelope is picked up out, what is the probability that it
contains no cash prize?\\
\solution
%\input{exemplar/10/13/3/34/main.tex}
\item 
A die is thrown and a card is selected at random from a deck of 52 playing cards. The probability of getting an even number on the die and a spade card.\\
\solution
%\input{exemplar/12/13/3/78/main.tex}
\item
If 4-digit numbers greater than 5,000 are randomly formed from the digits 0, 1, 3, 5, and 7, what is the probability of forming a number divisible by 5 when:
\begin{enumerate}
    \item The digits are repeated?
    \item The repetition of digits is not allowed?
\end{enumerate}
\solution
%\input{ncert/11/16/4/9/main.tex}
\item Consider the probability space $\brak{\Omega, \mathcal{G}, P}$ where $\Omega = [0,2]$ and $\mathcal{G} = \cbrak{\phi, \Omega, [0,1], (1,2]}$. Let $X$ and $Y$ be two functions on $\Omega$ defined as
\begin{align*}
    X(\omega) = 
    \begin{cases}
        1 & \text{if }\omega \in [0, 1]\\
        2 & \text{if }\omega \in (1, 2]
    \end{cases}
\end{align*}
and
\begin{align*}
    Y(\omega) = 
    \begin{cases}
        2 & \text{if }\omega \in [0, 1.5]\\
        3 & \text{if }\omega \in (1.5, 2].
    \end{cases}
\end{align*}
Then which one of the following statements is true?
\begin{enumerate}
    \item [(A)] $X$ is a random variable with respect to $\mathcal{G}$, but $Y$ is not a random variable with respect to $\mathcal{G}$.
    \item [(B)] $Y$ is a random variable with respect to $\mathcal{G}$, but $X$ is not a random variable with respect to $\mathcal{G}$.
    \item [(C)] Neither $X$ nor $Y$ is a random variable with respect to $\mathcal{G}$.
    \item [(D)] Both $X$ and $Y$ are random variables with respect to $\mathcal{G}$.
\end{enumerate} \hfill (GATE ST 2023)\\
\solution
%\input{gate/ST/2023/14/main.tex}
	\item  A die is loaded in such a way that each odd number is twice as likely to occur as
each even number. Find $P(G)$, where $G$ is the event that a number greater than
3 occurs on a single roll of the die.
\\
\solution
		%\input{exemplar/11/16/3/5/main.tex}
	\item All the jacks, queens and kings are removed from a deck of 52 playing cards. The remaining cards are well shuffled and then one card is drawn at random. Giving ace a value 1 similar value for other cards, find the probability that the card has a value 
		\begin{enumerate}
			\item 7
			\item greater than 7
			\item less than 7
		\end{enumerate}
		%\input{exemplar/10/13/3/30/main.tex}
  \item A Lot consists of 48 mobile phones of which 42 are good, 3 have only minor defects and 3 have major defects.Varnika will buy a phone if it is good but the trader will only buy a mobile if it has no major defects. One phone is selected at random from the lot. What is the probability that it is
\begin{enumerate}
	\item acceptable to Varnika?
            \item acceptable to the trader?
\end{enumerate}
\solution
	%\input{exemplar/10/13/3/40/main.tex}
 \item A student says that if you throw a die, it will show up 1 or not 1. Therefore, the probability of getting 1 and the probability of getting 'not 1' each is equal to $\frac{1}{2}$. Is this correct? Give reasons.\\
 \solution
        %\input{exemplar/10/13/2/9/main.tex}
   \item Four candidates A, B, C, D have ap-
plied for the assignment to coach a school cricket
team. If A is twice as likely to be selected as B, and
B and C are given about the same chance of being
selected, while C is twice as likely to be selected
as D, what are the probabilities that
\begin{enumerate}
\item C will be selected?
\item A will not be selected?
\end{enumerate}
	%\input{exemplar/11/16/3/9/main.tex}
 \item A bag contain 24 balls of which $x$ balls are red, $2x$ are white and $3x$ are blue. A ball is selected at random, What is the probability that it is
\begin{enumerate}[label=\alph*)]
\item not red ?
\item white ?
\end{enumerate}
%\input{exemplar/10/13/3/41/main.tex}
If the letters of the word ASSASSINATION are arranged at random. Find the Probability that
\begin{enumerate}[label=(\alph*)]
\item Four $S's$ come consecutively in the word
\item Two  $I's$ and two $N's$ come together
\item All $A's$ are not coming together
\item No two $A's$ are coming together
\end{enumerate}
%\input{exemplar/11/16/3/14/main.tex}
	\item One urn contains two black balls (labelled B1 and B2) and one white ball. A
	second urn contains one black ball and two white balls (labelled W1 and W2).
	Suppose the following experiment is performed. One of the two urns is chosen
	at random. Next a ball is randomly chosen from the urn. Then a second ball is
	chosen at random from the same urn without replacing the first ball.
	
	\begin{enumerate}
	\item What is the probability that two black balls are chosen?
	
	\item What is the probability that two balls of opposite colour are chosen?
	\end{enumerate}
	\solution
	%\input{exemplar/11/16/3/12/main1.tex}
\end{enumerate}

		%
\item 
Out of 100 students, two sections of 40 and 60 are formed. If you and your friend are among the 100 students, what is the probability that
\begin{enumerate}
\item you both enter the same section?
\item you both enter the different sections?
\end{enumerate}
\solution
		%\begin{enumerate}[label=\thesection.\arabic*,ref=\thesection.\theenumi]
	\item One card is drawn from a well-shuffled deck of 52 cards. Find the probability of getting
\begin{enumerate}
\item A king of red colour 
\item A face card 
\item A red face card
\item The jack of hearts
\item A spade
\item The queen of diamonds

\end{enumerate}
\solution
		%\input{ncert/10/15/1/14/main.tex}
	\item Five cards—the ten, jack, queen, king and ace of diamonds, are well-shuffled with their face downwards. One card is then picked up at random.
\begin{enumerate}
\item
What is the probability that the card is the queen? 
\item
If the queen is drawn and put aside, what is the probability that the second card picked up is (a) an ace? (b) a queen?\\
\end{enumerate}
\solution
		%\input{ncert/10/15/1/15/defs.tex}
	\item A bag contains $5$ red balls and some blue balls. If the probability of drawing a blue ball is double that if a red ball, determine the number of blue balls in the bag. 
		\\
\solution
		%\input{ncert/10/15/2/3/defs.tex}
	\item A card is selected from a pack of 52 cards.
 \begin{enumerate}[label=(\alph*)] 
                 \item How many points are there in the sample space?
                 \item Calculate the probability that the card is an ace of spades.
                 \item Calculate the probability that the card is (i) an ace and (ii) black card.
 \end{enumerate}
\solution
		%\input{ncert/11/16/3/4/main.tex}
\item Four cards are drawn from a well-shuffled deck of 52 cards. What is the probability of obtaining 3 diamonds and one spade.
\\
\solution
		%\input{ncert/11/16/4/2/defs.tex}
\item In a certain lottery 10,000 tickets are sold and ten equal prizes are awarded. What is the probability of not getting a prize if you buy (a) one ticket (b) two tickets (c) 10 tickets ?	
\\
\solution
		%\input{ncert/11/16/4/4/defs.tex}
		%
\item 
Out of 100 students, two sections of 40 and 60 are formed. If you and your friend are among the 100 students, what is the probability that
\begin{enumerate}
\item you both enter the same section?
\item you both enter the different sections?
\end{enumerate}
\solution
		%\input{ncert/11/16/4/5/defs.tex}
	\item 
The number lock of a suitcase has 4 wheels each labelled with ten digits i.e. from 0 to 9.The lock opens with a sequence of four digits with no repeats.What is the probability of a person getting the right sequence to open the suitcase.
\\
\solution
		%\input{ncert/11/16/4/10/defs.tex}
		%
\item 
Two cards are drawn at random and without replacement from a pack of 52 playing cards. Find the probability that both the cards are black.
\\
\solution
		%\input{ncert/12/13/2/2/defs.tex}
		\item A box of oranges is inspected by examining three randomly selected oranges drawn without replacement. If all the three oranges are good, the box is approved for sale, otherwise, it is rejected. Find the probability that a box containing 15 oranges out of which 12 are good and 3 are bad ones will be approved for sale.
		\label{ncert/12/13/2/3/defs.tex}
		\item Two balls are drawn at random with replacement from a box containing 10 black and 8 red balls. Find the probability that
		\label{ncert/12/13/2/12}
\begin{enumerate}
\item both balls are red.
\item first ball is black and second is red.
\item one of them is black and other is red.
\end{enumerate}

\item In a hostel, 60\% of the students read Hindi newspaper, 40\% read English newspaper and 20\% read both Hindi and English newspapers. A student is selected at random.
		\label{ncert/12/13/2/15}
\begin{enumerate}
\item Find the probability that she reads neither Hindi nor English newspapers.
\item If she reads Hindi newspaper, find the probability that she reads English newspaper.
\item If she reads English newspaper, find the probability that she reads Hindi newspaper.\\
\end{enumerate}
\item The probability of obtaining an even prime number on each die, when a pair of dice is rolled is 
\begin{enumerate}
    \item $0$ 
    
    \item $\frac{1}{3}$ 
    
    \item $\frac{1}{12}$ 
    
    \item $\frac{1}{36}$ 
\end{enumerate}
\solution
		%\input{ncert/12/13/2/17/defs.tex}
	\item A bag contains 4 red and 4 black balls, another bag contains 2 red and 6 black balls. One of the two bags is selected at random and a ball is drawn from the bag which is found to be red. Find the probability that the ball is drawn from the first bag.
\\
\solution
		%\input{ncert/12/13/3/2/main.tex}
  \item
  Cards with numbers 2 to 101 are placed in a box. A card is selected at random.Find the probability that the card has
\begin{enumerate}[label=(\roman*)]
	\item an even number 
	\item a square number
\end{enumerate}
\solution
%\input{exemplar/10/13/3/32/main.tex}
\item
The king, queen and jack of clubs are removed from a deck of 52 playing cards and then well shuffled. Now one card is drawn at random from the remaining cards.  Determine the probability that the card is
\begin{enumerate}[label=(\roman*)]
\item a club
\item 10 of hearts
\end{enumerate}
\solution
%\input{exemplar/10/13/3/29/main.tex}
\item A team of medical students doing their internship have to assist during surgeries
at a city hospital. The probabilities of surgeries rated as very complex, complex,
routine, simple or very simple are respectively, 0.15, 0.20, 0.31, 0.26, .08. Find
the probabilities that a particular surgery will be rated
\begin{enumerate}
	\item complex or very complex;
	\item neither very complex nor very simple;
	\item routine or complex
	\item routine or simple
\end{enumerate}
\solution
%\input{exemplar/11/16/3/8(1)/main.tex}
\item A card is selected from a pack of 52 cards.
\begin{enumerate}[label=(\alph*)]
    \item How many points are there in the sample space?
    \item Calculate the probability that the card is an ace of spades.
    \item Calculate the probability that the card is (i) an ace and (ii) black card.
\end{enumerate}
\solution
%\input{exemplar/11/16/3/4/main2.tex}
\item The probability that a non leap year selected at random will contain 53 sundays.
\\
\solution
%\input{exemplar/10/13/1/19/main.tex}
\item One of the four persons John, Rita, Aslam or Gurpreet will be promoted next
month. Consequently the sample space consists of four elementary outcomes
S = {John promoted, Rita promoted, Aslam promoted, Gurpreet promoted}
You are told that the chances of John’s promotion is same as that of Gurpreet,
Rita’s chances of promotion are twice as likely as Johns. Aslam’s chances are
four times that of John.
\begin{enumerate}
	\item Determine
	\begin{enumerate}
		\item P (John promoted)
		\item P (Rita promoted)
		\item P (Aslam promoted)
		\item P (Gurpreet promoted)
	\end{enumerate}
	\item If A = {John promoted or Gurpreet promoted}, find P (A).
\end{enumerate}
\solution
%\input{exemplar/11/16/3/10/main.tex}
\item A card is drawn from a deck of 52 cards. Find the probability of getting a king or a heart or a red card.\\
\solution
%\input{exemplar/11/16/3/15/main.tex}
\item The probability that a student will pass his examination is 0.73, the probability of
the student getting a compartment is 0.13, and the probability that the student will
either pass or get compartment is 0.96. State True or False.\\
\solution
%\input{exemplar/11/16/3/31/main.tex}
\item A card is selected from a pack of 52 cards\\
\begin{enumerate}[label=(\alph*)]
\item How many points are there in the sample space?
\item Calculate the probability that the cards is an ace of spades.
\item Calculate the probability that the card is (i) an ace (ii)black card.\\
\end{enumerate}
%\input{ncert/11/16/3/4_1/Prob_4.tex}
\item In a non-leap year, the probability of having 53 tuesdays or 53 wednesdays is\\
\solution
%\input{exemplar/11/16/3/18/main.tex}
\item There are 1000 sealed envelopes in a box, 10 of them contain a cash prize of
Rs 100 each, 100 of them contain a cash prize of Rs 50 each and 200 of them
contain a cash prize of Rs 10 each and rest do not contain any cash prize. If they
are well shuffled and an envelope is picked up out, what is the probability that it
contains no cash prize?\\
\solution
%\input{exemplar/10/13/3/34/main.tex}
\item 
A die is thrown and a card is selected at random from a deck of 52 playing cards. The probability of getting an even number on the die and a spade card.\\
\solution
%\input{exemplar/12/13/3/78/main.tex}
\item
If 4-digit numbers greater than 5,000 are randomly formed from the digits 0, 1, 3, 5, and 7, what is the probability of forming a number divisible by 5 when:
\begin{enumerate}
    \item The digits are repeated?
    \item The repetition of digits is not allowed?
\end{enumerate}
\solution
%\input{ncert/11/16/4/9/main.tex}
\item Consider the probability space $\brak{\Omega, \mathcal{G}, P}$ where $\Omega = [0,2]$ and $\mathcal{G} = \cbrak{\phi, \Omega, [0,1], (1,2]}$. Let $X$ and $Y$ be two functions on $\Omega$ defined as
\begin{align*}
    X(\omega) = 
    \begin{cases}
        1 & \text{if }\omega \in [0, 1]\\
        2 & \text{if }\omega \in (1, 2]
    \end{cases}
\end{align*}
and
\begin{align*}
    Y(\omega) = 
    \begin{cases}
        2 & \text{if }\omega \in [0, 1.5]\\
        3 & \text{if }\omega \in (1.5, 2].
    \end{cases}
\end{align*}
Then which one of the following statements is true?
\begin{enumerate}
    \item [(A)] $X$ is a random variable with respect to $\mathcal{G}$, but $Y$ is not a random variable with respect to $\mathcal{G}$.
    \item [(B)] $Y$ is a random variable with respect to $\mathcal{G}$, but $X$ is not a random variable with respect to $\mathcal{G}$.
    \item [(C)] Neither $X$ nor $Y$ is a random variable with respect to $\mathcal{G}$.
    \item [(D)] Both $X$ and $Y$ are random variables with respect to $\mathcal{G}$.
\end{enumerate} \hfill (GATE ST 2023)\\
\solution
%\input{gate/ST/2023/14/main.tex}
	\item  A die is loaded in such a way that each odd number is twice as likely to occur as
each even number. Find $P(G)$, where $G$ is the event that a number greater than
3 occurs on a single roll of the die.
\\
\solution
		%\input{exemplar/11/16/3/5/main.tex}
	\item All the jacks, queens and kings are removed from a deck of 52 playing cards. The remaining cards are well shuffled and then one card is drawn at random. Giving ace a value 1 similar value for other cards, find the probability that the card has a value 
		\begin{enumerate}
			\item 7
			\item greater than 7
			\item less than 7
		\end{enumerate}
		%\input{exemplar/10/13/3/30/main.tex}
  \item A Lot consists of 48 mobile phones of which 42 are good, 3 have only minor defects and 3 have major defects.Varnika will buy a phone if it is good but the trader will only buy a mobile if it has no major defects. One phone is selected at random from the lot. What is the probability that it is
\begin{enumerate}
	\item acceptable to Varnika?
            \item acceptable to the trader?
\end{enumerate}
\solution
	%\input{exemplar/10/13/3/40/main.tex}
 \item A student says that if you throw a die, it will show up 1 or not 1. Therefore, the probability of getting 1 and the probability of getting 'not 1' each is equal to $\frac{1}{2}$. Is this correct? Give reasons.\\
 \solution
        %\input{exemplar/10/13/2/9/main.tex}
   \item Four candidates A, B, C, D have ap-
plied for the assignment to coach a school cricket
team. If A is twice as likely to be selected as B, and
B and C are given about the same chance of being
selected, while C is twice as likely to be selected
as D, what are the probabilities that
\begin{enumerate}
\item C will be selected?
\item A will not be selected?
\end{enumerate}
	%\input{exemplar/11/16/3/9/main.tex}
 \item A bag contain 24 balls of which $x$ balls are red, $2x$ are white and $3x$ are blue. A ball is selected at random, What is the probability that it is
\begin{enumerate}[label=\alph*)]
\item not red ?
\item white ?
\end{enumerate}
%\input{exemplar/10/13/3/41/main.tex}
If the letters of the word ASSASSINATION are arranged at random. Find the Probability that
\begin{enumerate}[label=(\alph*)]
\item Four $S's$ come consecutively in the word
\item Two  $I's$ and two $N's$ come together
\item All $A's$ are not coming together
\item No two $A's$ are coming together
\end{enumerate}
%\input{exemplar/11/16/3/14/main.tex}
	\item One urn contains two black balls (labelled B1 and B2) and one white ball. A
	second urn contains one black ball and two white balls (labelled W1 and W2).
	Suppose the following experiment is performed. One of the two urns is chosen
	at random. Next a ball is randomly chosen from the urn. Then a second ball is
	chosen at random from the same urn without replacing the first ball.
	
	\begin{enumerate}
	\item What is the probability that two black balls are chosen?
	
	\item What is the probability that two balls of opposite colour are chosen?
	\end{enumerate}
	\solution
	%\input{exemplar/11/16/3/12/main1.tex}
\end{enumerate}

	\item 
The number lock of a suitcase has 4 wheels each labelled with ten digits i.e. from 0 to 9.The lock opens with a sequence of four digits with no repeats.What is the probability of a person getting the right sequence to open the suitcase.
\\
\solution
		%\begin{enumerate}[label=\thesection.\arabic*,ref=\thesection.\theenumi]
	\item One card is drawn from a well-shuffled deck of 52 cards. Find the probability of getting
\begin{enumerate}
\item A king of red colour 
\item A face card 
\item A red face card
\item The jack of hearts
\item A spade
\item The queen of diamonds

\end{enumerate}
\solution
		%\input{ncert/10/15/1/14/main.tex}
	\item Five cards—the ten, jack, queen, king and ace of diamonds, are well-shuffled with their face downwards. One card is then picked up at random.
\begin{enumerate}
\item
What is the probability that the card is the queen? 
\item
If the queen is drawn and put aside, what is the probability that the second card picked up is (a) an ace? (b) a queen?\\
\end{enumerate}
\solution
		%\input{ncert/10/15/1/15/defs.tex}
	\item A bag contains $5$ red balls and some blue balls. If the probability of drawing a blue ball is double that if a red ball, determine the number of blue balls in the bag. 
		\\
\solution
		%\input{ncert/10/15/2/3/defs.tex}
	\item A card is selected from a pack of 52 cards.
 \begin{enumerate}[label=(\alph*)] 
                 \item How many points are there in the sample space?
                 \item Calculate the probability that the card is an ace of spades.
                 \item Calculate the probability that the card is (i) an ace and (ii) black card.
 \end{enumerate}
\solution
		%\input{ncert/11/16/3/4/main.tex}
\item Four cards are drawn from a well-shuffled deck of 52 cards. What is the probability of obtaining 3 diamonds and one spade.
\\
\solution
		%\input{ncert/11/16/4/2/defs.tex}
\item In a certain lottery 10,000 tickets are sold and ten equal prizes are awarded. What is the probability of not getting a prize if you buy (a) one ticket (b) two tickets (c) 10 tickets ?	
\\
\solution
		%\input{ncert/11/16/4/4/defs.tex}
		%
\item 
Out of 100 students, two sections of 40 and 60 are formed. If you and your friend are among the 100 students, what is the probability that
\begin{enumerate}
\item you both enter the same section?
\item you both enter the different sections?
\end{enumerate}
\solution
		%\input{ncert/11/16/4/5/defs.tex}
	\item 
The number lock of a suitcase has 4 wheels each labelled with ten digits i.e. from 0 to 9.The lock opens with a sequence of four digits with no repeats.What is the probability of a person getting the right sequence to open the suitcase.
\\
\solution
		%\input{ncert/11/16/4/10/defs.tex}
		%
\item 
Two cards are drawn at random and without replacement from a pack of 52 playing cards. Find the probability that both the cards are black.
\\
\solution
		%\input{ncert/12/13/2/2/defs.tex}
		\item A box of oranges is inspected by examining three randomly selected oranges drawn without replacement. If all the three oranges are good, the box is approved for sale, otherwise, it is rejected. Find the probability that a box containing 15 oranges out of which 12 are good and 3 are bad ones will be approved for sale.
		\label{ncert/12/13/2/3/defs.tex}
		\item Two balls are drawn at random with replacement from a box containing 10 black and 8 red balls. Find the probability that
		\label{ncert/12/13/2/12}
\begin{enumerate}
\item both balls are red.
\item first ball is black and second is red.
\item one of them is black and other is red.
\end{enumerate}

\item In a hostel, 60\% of the students read Hindi newspaper, 40\% read English newspaper and 20\% read both Hindi and English newspapers. A student is selected at random.
		\label{ncert/12/13/2/15}
\begin{enumerate}
\item Find the probability that she reads neither Hindi nor English newspapers.
\item If she reads Hindi newspaper, find the probability that she reads English newspaper.
\item If she reads English newspaper, find the probability that she reads Hindi newspaper.\\
\end{enumerate}
\item The probability of obtaining an even prime number on each die, when a pair of dice is rolled is 
\begin{enumerate}
    \item $0$ 
    
    \item $\frac{1}{3}$ 
    
    \item $\frac{1}{12}$ 
    
    \item $\frac{1}{36}$ 
\end{enumerate}
\solution
		%\input{ncert/12/13/2/17/defs.tex}
	\item A bag contains 4 red and 4 black balls, another bag contains 2 red and 6 black balls. One of the two bags is selected at random and a ball is drawn from the bag which is found to be red. Find the probability that the ball is drawn from the first bag.
\\
\solution
		%\input{ncert/12/13/3/2/main.tex}
  \item
  Cards with numbers 2 to 101 are placed in a box. A card is selected at random.Find the probability that the card has
\begin{enumerate}[label=(\roman*)]
	\item an even number 
	\item a square number
\end{enumerate}
\solution
%\input{exemplar/10/13/3/32/main.tex}
\item
The king, queen and jack of clubs are removed from a deck of 52 playing cards and then well shuffled. Now one card is drawn at random from the remaining cards.  Determine the probability that the card is
\begin{enumerate}[label=(\roman*)]
\item a club
\item 10 of hearts
\end{enumerate}
\solution
%\input{exemplar/10/13/3/29/main.tex}
\item A team of medical students doing their internship have to assist during surgeries
at a city hospital. The probabilities of surgeries rated as very complex, complex,
routine, simple or very simple are respectively, 0.15, 0.20, 0.31, 0.26, .08. Find
the probabilities that a particular surgery will be rated
\begin{enumerate}
	\item complex or very complex;
	\item neither very complex nor very simple;
	\item routine or complex
	\item routine or simple
\end{enumerate}
\solution
%\input{exemplar/11/16/3/8(1)/main.tex}
\item A card is selected from a pack of 52 cards.
\begin{enumerate}[label=(\alph*)]
    \item How many points are there in the sample space?
    \item Calculate the probability that the card is an ace of spades.
    \item Calculate the probability that the card is (i) an ace and (ii) black card.
\end{enumerate}
\solution
%\input{exemplar/11/16/3/4/main2.tex}
\item The probability that a non leap year selected at random will contain 53 sundays.
\\
\solution
%\input{exemplar/10/13/1/19/main.tex}
\item One of the four persons John, Rita, Aslam or Gurpreet will be promoted next
month. Consequently the sample space consists of four elementary outcomes
S = {John promoted, Rita promoted, Aslam promoted, Gurpreet promoted}
You are told that the chances of John’s promotion is same as that of Gurpreet,
Rita’s chances of promotion are twice as likely as Johns. Aslam’s chances are
four times that of John.
\begin{enumerate}
	\item Determine
	\begin{enumerate}
		\item P (John promoted)
		\item P (Rita promoted)
		\item P (Aslam promoted)
		\item P (Gurpreet promoted)
	\end{enumerate}
	\item If A = {John promoted or Gurpreet promoted}, find P (A).
\end{enumerate}
\solution
%\input{exemplar/11/16/3/10/main.tex}
\item A card is drawn from a deck of 52 cards. Find the probability of getting a king or a heart or a red card.\\
\solution
%\input{exemplar/11/16/3/15/main.tex}
\item The probability that a student will pass his examination is 0.73, the probability of
the student getting a compartment is 0.13, and the probability that the student will
either pass or get compartment is 0.96. State True or False.\\
\solution
%\input{exemplar/11/16/3/31/main.tex}
\item A card is selected from a pack of 52 cards\\
\begin{enumerate}[label=(\alph*)]
\item How many points are there in the sample space?
\item Calculate the probability that the cards is an ace of spades.
\item Calculate the probability that the card is (i) an ace (ii)black card.\\
\end{enumerate}
%\input{ncert/11/16/3/4_1/Prob_4.tex}
\item In a non-leap year, the probability of having 53 tuesdays or 53 wednesdays is\\
\solution
%\input{exemplar/11/16/3/18/main.tex}
\item There are 1000 sealed envelopes in a box, 10 of them contain a cash prize of
Rs 100 each, 100 of them contain a cash prize of Rs 50 each and 200 of them
contain a cash prize of Rs 10 each and rest do not contain any cash prize. If they
are well shuffled and an envelope is picked up out, what is the probability that it
contains no cash prize?\\
\solution
%\input{exemplar/10/13/3/34/main.tex}
\item 
A die is thrown and a card is selected at random from a deck of 52 playing cards. The probability of getting an even number on the die and a spade card.\\
\solution
%\input{exemplar/12/13/3/78/main.tex}
\item
If 4-digit numbers greater than 5,000 are randomly formed from the digits 0, 1, 3, 5, and 7, what is the probability of forming a number divisible by 5 when:
\begin{enumerate}
    \item The digits are repeated?
    \item The repetition of digits is not allowed?
\end{enumerate}
\solution
%\input{ncert/11/16/4/9/main.tex}
\item Consider the probability space $\brak{\Omega, \mathcal{G}, P}$ where $\Omega = [0,2]$ and $\mathcal{G} = \cbrak{\phi, \Omega, [0,1], (1,2]}$. Let $X$ and $Y$ be two functions on $\Omega$ defined as
\begin{align*}
    X(\omega) = 
    \begin{cases}
        1 & \text{if }\omega \in [0, 1]\\
        2 & \text{if }\omega \in (1, 2]
    \end{cases}
\end{align*}
and
\begin{align*}
    Y(\omega) = 
    \begin{cases}
        2 & \text{if }\omega \in [0, 1.5]\\
        3 & \text{if }\omega \in (1.5, 2].
    \end{cases}
\end{align*}
Then which one of the following statements is true?
\begin{enumerate}
    \item [(A)] $X$ is a random variable with respect to $\mathcal{G}$, but $Y$ is not a random variable with respect to $\mathcal{G}$.
    \item [(B)] $Y$ is a random variable with respect to $\mathcal{G}$, but $X$ is not a random variable with respect to $\mathcal{G}$.
    \item [(C)] Neither $X$ nor $Y$ is a random variable with respect to $\mathcal{G}$.
    \item [(D)] Both $X$ and $Y$ are random variables with respect to $\mathcal{G}$.
\end{enumerate} \hfill (GATE ST 2023)\\
\solution
%\input{gate/ST/2023/14/main.tex}
	\item  A die is loaded in such a way that each odd number is twice as likely to occur as
each even number. Find $P(G)$, where $G$ is the event that a number greater than
3 occurs on a single roll of the die.
\\
\solution
		%\input{exemplar/11/16/3/5/main.tex}
	\item All the jacks, queens and kings are removed from a deck of 52 playing cards. The remaining cards are well shuffled and then one card is drawn at random. Giving ace a value 1 similar value for other cards, find the probability that the card has a value 
		\begin{enumerate}
			\item 7
			\item greater than 7
			\item less than 7
		\end{enumerate}
		%\input{exemplar/10/13/3/30/main.tex}
  \item A Lot consists of 48 mobile phones of which 42 are good, 3 have only minor defects and 3 have major defects.Varnika will buy a phone if it is good but the trader will only buy a mobile if it has no major defects. One phone is selected at random from the lot. What is the probability that it is
\begin{enumerate}
	\item acceptable to Varnika?
            \item acceptable to the trader?
\end{enumerate}
\solution
	%\input{exemplar/10/13/3/40/main.tex}
 \item A student says that if you throw a die, it will show up 1 or not 1. Therefore, the probability of getting 1 and the probability of getting 'not 1' each is equal to $\frac{1}{2}$. Is this correct? Give reasons.\\
 \solution
        %\input{exemplar/10/13/2/9/main.tex}
   \item Four candidates A, B, C, D have ap-
plied for the assignment to coach a school cricket
team. If A is twice as likely to be selected as B, and
B and C are given about the same chance of being
selected, while C is twice as likely to be selected
as D, what are the probabilities that
\begin{enumerate}
\item C will be selected?
\item A will not be selected?
\end{enumerate}
	%\input{exemplar/11/16/3/9/main.tex}
 \item A bag contain 24 balls of which $x$ balls are red, $2x$ are white and $3x$ are blue. A ball is selected at random, What is the probability that it is
\begin{enumerate}[label=\alph*)]
\item not red ?
\item white ?
\end{enumerate}
%\input{exemplar/10/13/3/41/main.tex}
If the letters of the word ASSASSINATION are arranged at random. Find the Probability that
\begin{enumerate}[label=(\alph*)]
\item Four $S's$ come consecutively in the word
\item Two  $I's$ and two $N's$ come together
\item All $A's$ are not coming together
\item No two $A's$ are coming together
\end{enumerate}
%\input{exemplar/11/16/3/14/main.tex}
	\item One urn contains two black balls (labelled B1 and B2) and one white ball. A
	second urn contains one black ball and two white balls (labelled W1 and W2).
	Suppose the following experiment is performed. One of the two urns is chosen
	at random. Next a ball is randomly chosen from the urn. Then a second ball is
	chosen at random from the same urn without replacing the first ball.
	
	\begin{enumerate}
	\item What is the probability that two black balls are chosen?
	
	\item What is the probability that two balls of opposite colour are chosen?
	\end{enumerate}
	\solution
	%\input{exemplar/11/16/3/12/main1.tex}
\end{enumerate}

		%
\item 
Two cards are drawn at random and without replacement from a pack of 52 playing cards. Find the probability that both the cards are black.
\\
\solution
		%\begin{enumerate}[label=\thesection.\arabic*,ref=\thesection.\theenumi]
	\item One card is drawn from a well-shuffled deck of 52 cards. Find the probability of getting
\begin{enumerate}
\item A king of red colour 
\item A face card 
\item A red face card
\item The jack of hearts
\item A spade
\item The queen of diamonds

\end{enumerate}
\solution
		%\input{ncert/10/15/1/14/main.tex}
	\item Five cards—the ten, jack, queen, king and ace of diamonds, are well-shuffled with their face downwards. One card is then picked up at random.
\begin{enumerate}
\item
What is the probability that the card is the queen? 
\item
If the queen is drawn and put aside, what is the probability that the second card picked up is (a) an ace? (b) a queen?\\
\end{enumerate}
\solution
		%\input{ncert/10/15/1/15/defs.tex}
	\item A bag contains $5$ red balls and some blue balls. If the probability of drawing a blue ball is double that if a red ball, determine the number of blue balls in the bag. 
		\\
\solution
		%\input{ncert/10/15/2/3/defs.tex}
	\item A card is selected from a pack of 52 cards.
 \begin{enumerate}[label=(\alph*)] 
                 \item How many points are there in the sample space?
                 \item Calculate the probability that the card is an ace of spades.
                 \item Calculate the probability that the card is (i) an ace and (ii) black card.
 \end{enumerate}
\solution
		%\input{ncert/11/16/3/4/main.tex}
\item Four cards are drawn from a well-shuffled deck of 52 cards. What is the probability of obtaining 3 diamonds and one spade.
\\
\solution
		%\input{ncert/11/16/4/2/defs.tex}
\item In a certain lottery 10,000 tickets are sold and ten equal prizes are awarded. What is the probability of not getting a prize if you buy (a) one ticket (b) two tickets (c) 10 tickets ?	
\\
\solution
		%\input{ncert/11/16/4/4/defs.tex}
		%
\item 
Out of 100 students, two sections of 40 and 60 are formed. If you and your friend are among the 100 students, what is the probability that
\begin{enumerate}
\item you both enter the same section?
\item you both enter the different sections?
\end{enumerate}
\solution
		%\input{ncert/11/16/4/5/defs.tex}
	\item 
The number lock of a suitcase has 4 wheels each labelled with ten digits i.e. from 0 to 9.The lock opens with a sequence of four digits with no repeats.What is the probability of a person getting the right sequence to open the suitcase.
\\
\solution
		%\input{ncert/11/16/4/10/defs.tex}
		%
\item 
Two cards are drawn at random and without replacement from a pack of 52 playing cards. Find the probability that both the cards are black.
\\
\solution
		%\input{ncert/12/13/2/2/defs.tex}
		\item A box of oranges is inspected by examining three randomly selected oranges drawn without replacement. If all the three oranges are good, the box is approved for sale, otherwise, it is rejected. Find the probability that a box containing 15 oranges out of which 12 are good and 3 are bad ones will be approved for sale.
		\label{ncert/12/13/2/3/defs.tex}
		\item Two balls are drawn at random with replacement from a box containing 10 black and 8 red balls. Find the probability that
		\label{ncert/12/13/2/12}
\begin{enumerate}
\item both balls are red.
\item first ball is black and second is red.
\item one of them is black and other is red.
\end{enumerate}

\item In a hostel, 60\% of the students read Hindi newspaper, 40\% read English newspaper and 20\% read both Hindi and English newspapers. A student is selected at random.
		\label{ncert/12/13/2/15}
\begin{enumerate}
\item Find the probability that she reads neither Hindi nor English newspapers.
\item If she reads Hindi newspaper, find the probability that she reads English newspaper.
\item If she reads English newspaper, find the probability that she reads Hindi newspaper.\\
\end{enumerate}
\item The probability of obtaining an even prime number on each die, when a pair of dice is rolled is 
\begin{enumerate}
    \item $0$ 
    
    \item $\frac{1}{3}$ 
    
    \item $\frac{1}{12}$ 
    
    \item $\frac{1}{36}$ 
\end{enumerate}
\solution
		%\input{ncert/12/13/2/17/defs.tex}
	\item A bag contains 4 red and 4 black balls, another bag contains 2 red and 6 black balls. One of the two bags is selected at random and a ball is drawn from the bag which is found to be red. Find the probability that the ball is drawn from the first bag.
\\
\solution
		%\input{ncert/12/13/3/2/main.tex}
  \item
  Cards with numbers 2 to 101 are placed in a box. A card is selected at random.Find the probability that the card has
\begin{enumerate}[label=(\roman*)]
	\item an even number 
	\item a square number
\end{enumerate}
\solution
%\input{exemplar/10/13/3/32/main.tex}
\item
The king, queen and jack of clubs are removed from a deck of 52 playing cards and then well shuffled. Now one card is drawn at random from the remaining cards.  Determine the probability that the card is
\begin{enumerate}[label=(\roman*)]
\item a club
\item 10 of hearts
\end{enumerate}
\solution
%\input{exemplar/10/13/3/29/main.tex}
\item A team of medical students doing their internship have to assist during surgeries
at a city hospital. The probabilities of surgeries rated as very complex, complex,
routine, simple or very simple are respectively, 0.15, 0.20, 0.31, 0.26, .08. Find
the probabilities that a particular surgery will be rated
\begin{enumerate}
	\item complex or very complex;
	\item neither very complex nor very simple;
	\item routine or complex
	\item routine or simple
\end{enumerate}
\solution
%\input{exemplar/11/16/3/8(1)/main.tex}
\item A card is selected from a pack of 52 cards.
\begin{enumerate}[label=(\alph*)]
    \item How many points are there in the sample space?
    \item Calculate the probability that the card is an ace of spades.
    \item Calculate the probability that the card is (i) an ace and (ii) black card.
\end{enumerate}
\solution
%\input{exemplar/11/16/3/4/main2.tex}
\item The probability that a non leap year selected at random will contain 53 sundays.
\\
\solution
%\input{exemplar/10/13/1/19/main.tex}
\item One of the four persons John, Rita, Aslam or Gurpreet will be promoted next
month. Consequently the sample space consists of four elementary outcomes
S = {John promoted, Rita promoted, Aslam promoted, Gurpreet promoted}
You are told that the chances of John’s promotion is same as that of Gurpreet,
Rita’s chances of promotion are twice as likely as Johns. Aslam’s chances are
four times that of John.
\begin{enumerate}
	\item Determine
	\begin{enumerate}
		\item P (John promoted)
		\item P (Rita promoted)
		\item P (Aslam promoted)
		\item P (Gurpreet promoted)
	\end{enumerate}
	\item If A = {John promoted or Gurpreet promoted}, find P (A).
\end{enumerate}
\solution
%\input{exemplar/11/16/3/10/main.tex}
\item A card is drawn from a deck of 52 cards. Find the probability of getting a king or a heart or a red card.\\
\solution
%\input{exemplar/11/16/3/15/main.tex}
\item The probability that a student will pass his examination is 0.73, the probability of
the student getting a compartment is 0.13, and the probability that the student will
either pass or get compartment is 0.96. State True or False.\\
\solution
%\input{exemplar/11/16/3/31/main.tex}
\item A card is selected from a pack of 52 cards\\
\begin{enumerate}[label=(\alph*)]
\item How many points are there in the sample space?
\item Calculate the probability that the cards is an ace of spades.
\item Calculate the probability that the card is (i) an ace (ii)black card.\\
\end{enumerate}
%\input{ncert/11/16/3/4_1/Prob_4.tex}
\item In a non-leap year, the probability of having 53 tuesdays or 53 wednesdays is\\
\solution
%\input{exemplar/11/16/3/18/main.tex}
\item There are 1000 sealed envelopes in a box, 10 of them contain a cash prize of
Rs 100 each, 100 of them contain a cash prize of Rs 50 each and 200 of them
contain a cash prize of Rs 10 each and rest do not contain any cash prize. If they
are well shuffled and an envelope is picked up out, what is the probability that it
contains no cash prize?\\
\solution
%\input{exemplar/10/13/3/34/main.tex}
\item 
A die is thrown and a card is selected at random from a deck of 52 playing cards. The probability of getting an even number on the die and a spade card.\\
\solution
%\input{exemplar/12/13/3/78/main.tex}
\item
If 4-digit numbers greater than 5,000 are randomly formed from the digits 0, 1, 3, 5, and 7, what is the probability of forming a number divisible by 5 when:
\begin{enumerate}
    \item The digits are repeated?
    \item The repetition of digits is not allowed?
\end{enumerate}
\solution
%\input{ncert/11/16/4/9/main.tex}
\item Consider the probability space $\brak{\Omega, \mathcal{G}, P}$ where $\Omega = [0,2]$ and $\mathcal{G} = \cbrak{\phi, \Omega, [0,1], (1,2]}$. Let $X$ and $Y$ be two functions on $\Omega$ defined as
\begin{align*}
    X(\omega) = 
    \begin{cases}
        1 & \text{if }\omega \in [0, 1]\\
        2 & \text{if }\omega \in (1, 2]
    \end{cases}
\end{align*}
and
\begin{align*}
    Y(\omega) = 
    \begin{cases}
        2 & \text{if }\omega \in [0, 1.5]\\
        3 & \text{if }\omega \in (1.5, 2].
    \end{cases}
\end{align*}
Then which one of the following statements is true?
\begin{enumerate}
    \item [(A)] $X$ is a random variable with respect to $\mathcal{G}$, but $Y$ is not a random variable with respect to $\mathcal{G}$.
    \item [(B)] $Y$ is a random variable with respect to $\mathcal{G}$, but $X$ is not a random variable with respect to $\mathcal{G}$.
    \item [(C)] Neither $X$ nor $Y$ is a random variable with respect to $\mathcal{G}$.
    \item [(D)] Both $X$ and $Y$ are random variables with respect to $\mathcal{G}$.
\end{enumerate} \hfill (GATE ST 2023)\\
\solution
%\input{gate/ST/2023/14/main.tex}
	\item  A die is loaded in such a way that each odd number is twice as likely to occur as
each even number. Find $P(G)$, where $G$ is the event that a number greater than
3 occurs on a single roll of the die.
\\
\solution
		%\input{exemplar/11/16/3/5/main.tex}
	\item All the jacks, queens and kings are removed from a deck of 52 playing cards. The remaining cards are well shuffled and then one card is drawn at random. Giving ace a value 1 similar value for other cards, find the probability that the card has a value 
		\begin{enumerate}
			\item 7
			\item greater than 7
			\item less than 7
		\end{enumerate}
		%\input{exemplar/10/13/3/30/main.tex}
  \item A Lot consists of 48 mobile phones of which 42 are good, 3 have only minor defects and 3 have major defects.Varnika will buy a phone if it is good but the trader will only buy a mobile if it has no major defects. One phone is selected at random from the lot. What is the probability that it is
\begin{enumerate}
	\item acceptable to Varnika?
            \item acceptable to the trader?
\end{enumerate}
\solution
	%\input{exemplar/10/13/3/40/main.tex}
 \item A student says that if you throw a die, it will show up 1 or not 1. Therefore, the probability of getting 1 and the probability of getting 'not 1' each is equal to $\frac{1}{2}$. Is this correct? Give reasons.\\
 \solution
        %\input{exemplar/10/13/2/9/main.tex}
   \item Four candidates A, B, C, D have ap-
plied for the assignment to coach a school cricket
team. If A is twice as likely to be selected as B, and
B and C are given about the same chance of being
selected, while C is twice as likely to be selected
as D, what are the probabilities that
\begin{enumerate}
\item C will be selected?
\item A will not be selected?
\end{enumerate}
	%\input{exemplar/11/16/3/9/main.tex}
 \item A bag contain 24 balls of which $x$ balls are red, $2x$ are white and $3x$ are blue. A ball is selected at random, What is the probability that it is
\begin{enumerate}[label=\alph*)]
\item not red ?
\item white ?
\end{enumerate}
%\input{exemplar/10/13/3/41/main.tex}
If the letters of the word ASSASSINATION are arranged at random. Find the Probability that
\begin{enumerate}[label=(\alph*)]
\item Four $S's$ come consecutively in the word
\item Two  $I's$ and two $N's$ come together
\item All $A's$ are not coming together
\item No two $A's$ are coming together
\end{enumerate}
%\input{exemplar/11/16/3/14/main.tex}
	\item One urn contains two black balls (labelled B1 and B2) and one white ball. A
	second urn contains one black ball and two white balls (labelled W1 and W2).
	Suppose the following experiment is performed. One of the two urns is chosen
	at random. Next a ball is randomly chosen from the urn. Then a second ball is
	chosen at random from the same urn without replacing the first ball.
	
	\begin{enumerate}
	\item What is the probability that two black balls are chosen?
	
	\item What is the probability that two balls of opposite colour are chosen?
	\end{enumerate}
	\solution
	%\input{exemplar/11/16/3/12/main1.tex}
\end{enumerate}

		\item A box of oranges is inspected by examining three randomly selected oranges drawn without replacement. If all the three oranges are good, the box is approved for sale, otherwise, it is rejected. Find the probability that a box containing 15 oranges out of which 12 are good and 3 are bad ones will be approved for sale.
		\label{ncert/12/13/2/3/defs.tex}
		\item Two balls are drawn at random with replacement from a box containing 10 black and 8 red balls. Find the probability that
		\label{ncert/12/13/2/12}
\begin{enumerate}
\item both balls are red.
\item first ball is black and second is red.
\item one of them is black and other is red.
\end{enumerate}

\item In a hostel, 60\% of the students read Hindi newspaper, 40\% read English newspaper and 20\% read both Hindi and English newspapers. A student is selected at random.
		\label{ncert/12/13/2/15}
\begin{enumerate}
\item Find the probability that she reads neither Hindi nor English newspapers.
\item If she reads Hindi newspaper, find the probability that she reads English newspaper.
\item If she reads English newspaper, find the probability that she reads Hindi newspaper.\\
\end{enumerate}
\item The probability of obtaining an even prime number on each die, when a pair of dice is rolled is 
\begin{enumerate}
    \item $0$ 
    
    \item $\frac{1}{3}$ 
    
    \item $\frac{1}{12}$ 
    
    \item $\frac{1}{36}$ 
\end{enumerate}
\solution
		%\begin{enumerate}[label=\thesection.\arabic*,ref=\thesection.\theenumi]
	\item One card is drawn from a well-shuffled deck of 52 cards. Find the probability of getting
\begin{enumerate}
\item A king of red colour 
\item A face card 
\item A red face card
\item The jack of hearts
\item A spade
\item The queen of diamonds

\end{enumerate}
\solution
		%\input{ncert/10/15/1/14/main.tex}
	\item Five cards—the ten, jack, queen, king and ace of diamonds, are well-shuffled with their face downwards. One card is then picked up at random.
\begin{enumerate}
\item
What is the probability that the card is the queen? 
\item
If the queen is drawn and put aside, what is the probability that the second card picked up is (a) an ace? (b) a queen?\\
\end{enumerate}
\solution
		%\input{ncert/10/15/1/15/defs.tex}
	\item A bag contains $5$ red balls and some blue balls. If the probability of drawing a blue ball is double that if a red ball, determine the number of blue balls in the bag. 
		\\
\solution
		%\input{ncert/10/15/2/3/defs.tex}
	\item A card is selected from a pack of 52 cards.
 \begin{enumerate}[label=(\alph*)] 
                 \item How many points are there in the sample space?
                 \item Calculate the probability that the card is an ace of spades.
                 \item Calculate the probability that the card is (i) an ace and (ii) black card.
 \end{enumerate}
\solution
		%\input{ncert/11/16/3/4/main.tex}
\item Four cards are drawn from a well-shuffled deck of 52 cards. What is the probability of obtaining 3 diamonds and one spade.
\\
\solution
		%\input{ncert/11/16/4/2/defs.tex}
\item In a certain lottery 10,000 tickets are sold and ten equal prizes are awarded. What is the probability of not getting a prize if you buy (a) one ticket (b) two tickets (c) 10 tickets ?	
\\
\solution
		%\input{ncert/11/16/4/4/defs.tex}
		%
\item 
Out of 100 students, two sections of 40 and 60 are formed. If you and your friend are among the 100 students, what is the probability that
\begin{enumerate}
\item you both enter the same section?
\item you both enter the different sections?
\end{enumerate}
\solution
		%\input{ncert/11/16/4/5/defs.tex}
	\item 
The number lock of a suitcase has 4 wheels each labelled with ten digits i.e. from 0 to 9.The lock opens with a sequence of four digits with no repeats.What is the probability of a person getting the right sequence to open the suitcase.
\\
\solution
		%\input{ncert/11/16/4/10/defs.tex}
		%
\item 
Two cards are drawn at random and without replacement from a pack of 52 playing cards. Find the probability that both the cards are black.
\\
\solution
		%\input{ncert/12/13/2/2/defs.tex}
		\item A box of oranges is inspected by examining three randomly selected oranges drawn without replacement. If all the three oranges are good, the box is approved for sale, otherwise, it is rejected. Find the probability that a box containing 15 oranges out of which 12 are good and 3 are bad ones will be approved for sale.
		\label{ncert/12/13/2/3/defs.tex}
		\item Two balls are drawn at random with replacement from a box containing 10 black and 8 red balls. Find the probability that
		\label{ncert/12/13/2/12}
\begin{enumerate}
\item both balls are red.
\item first ball is black and second is red.
\item one of them is black and other is red.
\end{enumerate}

\item In a hostel, 60\% of the students read Hindi newspaper, 40\% read English newspaper and 20\% read both Hindi and English newspapers. A student is selected at random.
		\label{ncert/12/13/2/15}
\begin{enumerate}
\item Find the probability that she reads neither Hindi nor English newspapers.
\item If she reads Hindi newspaper, find the probability that she reads English newspaper.
\item If she reads English newspaper, find the probability that she reads Hindi newspaper.\\
\end{enumerate}
\item The probability of obtaining an even prime number on each die, when a pair of dice is rolled is 
\begin{enumerate}
    \item $0$ 
    
    \item $\frac{1}{3}$ 
    
    \item $\frac{1}{12}$ 
    
    \item $\frac{1}{36}$ 
\end{enumerate}
\solution
		%\input{ncert/12/13/2/17/defs.tex}
	\item A bag contains 4 red and 4 black balls, another bag contains 2 red and 6 black balls. One of the two bags is selected at random and a ball is drawn from the bag which is found to be red. Find the probability that the ball is drawn from the first bag.
\\
\solution
		%\input{ncert/12/13/3/2/main.tex}
  \item
  Cards with numbers 2 to 101 are placed in a box. A card is selected at random.Find the probability that the card has
\begin{enumerate}[label=(\roman*)]
	\item an even number 
	\item a square number
\end{enumerate}
\solution
%\input{exemplar/10/13/3/32/main.tex}
\item
The king, queen and jack of clubs are removed from a deck of 52 playing cards and then well shuffled. Now one card is drawn at random from the remaining cards.  Determine the probability that the card is
\begin{enumerate}[label=(\roman*)]
\item a club
\item 10 of hearts
\end{enumerate}
\solution
%\input{exemplar/10/13/3/29/main.tex}
\item A team of medical students doing their internship have to assist during surgeries
at a city hospital. The probabilities of surgeries rated as very complex, complex,
routine, simple or very simple are respectively, 0.15, 0.20, 0.31, 0.26, .08. Find
the probabilities that a particular surgery will be rated
\begin{enumerate}
	\item complex or very complex;
	\item neither very complex nor very simple;
	\item routine or complex
	\item routine or simple
\end{enumerate}
\solution
%\input{exemplar/11/16/3/8(1)/main.tex}
\item A card is selected from a pack of 52 cards.
\begin{enumerate}[label=(\alph*)]
    \item How many points are there in the sample space?
    \item Calculate the probability that the card is an ace of spades.
    \item Calculate the probability that the card is (i) an ace and (ii) black card.
\end{enumerate}
\solution
%\input{exemplar/11/16/3/4/main2.tex}
\item The probability that a non leap year selected at random will contain 53 sundays.
\\
\solution
%\input{exemplar/10/13/1/19/main.tex}
\item One of the four persons John, Rita, Aslam or Gurpreet will be promoted next
month. Consequently the sample space consists of four elementary outcomes
S = {John promoted, Rita promoted, Aslam promoted, Gurpreet promoted}
You are told that the chances of John’s promotion is same as that of Gurpreet,
Rita’s chances of promotion are twice as likely as Johns. Aslam’s chances are
four times that of John.
\begin{enumerate}
	\item Determine
	\begin{enumerate}
		\item P (John promoted)
		\item P (Rita promoted)
		\item P (Aslam promoted)
		\item P (Gurpreet promoted)
	\end{enumerate}
	\item If A = {John promoted or Gurpreet promoted}, find P (A).
\end{enumerate}
\solution
%\input{exemplar/11/16/3/10/main.tex}
\item A card is drawn from a deck of 52 cards. Find the probability of getting a king or a heart or a red card.\\
\solution
%\input{exemplar/11/16/3/15/main.tex}
\item The probability that a student will pass his examination is 0.73, the probability of
the student getting a compartment is 0.13, and the probability that the student will
either pass or get compartment is 0.96. State True or False.\\
\solution
%\input{exemplar/11/16/3/31/main.tex}
\item A card is selected from a pack of 52 cards\\
\begin{enumerate}[label=(\alph*)]
\item How many points are there in the sample space?
\item Calculate the probability that the cards is an ace of spades.
\item Calculate the probability that the card is (i) an ace (ii)black card.\\
\end{enumerate}
%\input{ncert/11/16/3/4_1/Prob_4.tex}
\item In a non-leap year, the probability of having 53 tuesdays or 53 wednesdays is\\
\solution
%\input{exemplar/11/16/3/18/main.tex}
\item There are 1000 sealed envelopes in a box, 10 of them contain a cash prize of
Rs 100 each, 100 of them contain a cash prize of Rs 50 each and 200 of them
contain a cash prize of Rs 10 each and rest do not contain any cash prize. If they
are well shuffled and an envelope is picked up out, what is the probability that it
contains no cash prize?\\
\solution
%\input{exemplar/10/13/3/34/main.tex}
\item 
A die is thrown and a card is selected at random from a deck of 52 playing cards. The probability of getting an even number on the die and a spade card.\\
\solution
%\input{exemplar/12/13/3/78/main.tex}
\item
If 4-digit numbers greater than 5,000 are randomly formed from the digits 0, 1, 3, 5, and 7, what is the probability of forming a number divisible by 5 when:
\begin{enumerate}
    \item The digits are repeated?
    \item The repetition of digits is not allowed?
\end{enumerate}
\solution
%\input{ncert/11/16/4/9/main.tex}
\item Consider the probability space $\brak{\Omega, \mathcal{G}, P}$ where $\Omega = [0,2]$ and $\mathcal{G} = \cbrak{\phi, \Omega, [0,1], (1,2]}$. Let $X$ and $Y$ be two functions on $\Omega$ defined as
\begin{align*}
    X(\omega) = 
    \begin{cases}
        1 & \text{if }\omega \in [0, 1]\\
        2 & \text{if }\omega \in (1, 2]
    \end{cases}
\end{align*}
and
\begin{align*}
    Y(\omega) = 
    \begin{cases}
        2 & \text{if }\omega \in [0, 1.5]\\
        3 & \text{if }\omega \in (1.5, 2].
    \end{cases}
\end{align*}
Then which one of the following statements is true?
\begin{enumerate}
    \item [(A)] $X$ is a random variable with respect to $\mathcal{G}$, but $Y$ is not a random variable with respect to $\mathcal{G}$.
    \item [(B)] $Y$ is a random variable with respect to $\mathcal{G}$, but $X$ is not a random variable with respect to $\mathcal{G}$.
    \item [(C)] Neither $X$ nor $Y$ is a random variable with respect to $\mathcal{G}$.
    \item [(D)] Both $X$ and $Y$ are random variables with respect to $\mathcal{G}$.
\end{enumerate} \hfill (GATE ST 2023)\\
\solution
%\input{gate/ST/2023/14/main.tex}
	\item  A die is loaded in such a way that each odd number is twice as likely to occur as
each even number. Find $P(G)$, where $G$ is the event that a number greater than
3 occurs on a single roll of the die.
\\
\solution
		%\input{exemplar/11/16/3/5/main.tex}
	\item All the jacks, queens and kings are removed from a deck of 52 playing cards. The remaining cards are well shuffled and then one card is drawn at random. Giving ace a value 1 similar value for other cards, find the probability that the card has a value 
		\begin{enumerate}
			\item 7
			\item greater than 7
			\item less than 7
		\end{enumerate}
		%\input{exemplar/10/13/3/30/main.tex}
  \item A Lot consists of 48 mobile phones of which 42 are good, 3 have only minor defects and 3 have major defects.Varnika will buy a phone if it is good but the trader will only buy a mobile if it has no major defects. One phone is selected at random from the lot. What is the probability that it is
\begin{enumerate}
	\item acceptable to Varnika?
            \item acceptable to the trader?
\end{enumerate}
\solution
	%\input{exemplar/10/13/3/40/main.tex}
 \item A student says that if you throw a die, it will show up 1 or not 1. Therefore, the probability of getting 1 and the probability of getting 'not 1' each is equal to $\frac{1}{2}$. Is this correct? Give reasons.\\
 \solution
        %\input{exemplar/10/13/2/9/main.tex}
   \item Four candidates A, B, C, D have ap-
plied for the assignment to coach a school cricket
team. If A is twice as likely to be selected as B, and
B and C are given about the same chance of being
selected, while C is twice as likely to be selected
as D, what are the probabilities that
\begin{enumerate}
\item C will be selected?
\item A will not be selected?
\end{enumerate}
	%\input{exemplar/11/16/3/9/main.tex}
 \item A bag contain 24 balls of which $x$ balls are red, $2x$ are white and $3x$ are blue. A ball is selected at random, What is the probability that it is
\begin{enumerate}[label=\alph*)]
\item not red ?
\item white ?
\end{enumerate}
%\input{exemplar/10/13/3/41/main.tex}
If the letters of the word ASSASSINATION are arranged at random. Find the Probability that
\begin{enumerate}[label=(\alph*)]
\item Four $S's$ come consecutively in the word
\item Two  $I's$ and two $N's$ come together
\item All $A's$ are not coming together
\item No two $A's$ are coming together
\end{enumerate}
%\input{exemplar/11/16/3/14/main.tex}
	\item One urn contains two black balls (labelled B1 and B2) and one white ball. A
	second urn contains one black ball and two white balls (labelled W1 and W2).
	Suppose the following experiment is performed. One of the two urns is chosen
	at random. Next a ball is randomly chosen from the urn. Then a second ball is
	chosen at random from the same urn without replacing the first ball.
	
	\begin{enumerate}
	\item What is the probability that two black balls are chosen?
	
	\item What is the probability that two balls of opposite colour are chosen?
	\end{enumerate}
	\solution
	%\input{exemplar/11/16/3/12/main1.tex}
\end{enumerate}

	\item A bag contains 4 red and 4 black balls, another bag contains 2 red and 6 black balls. One of the two bags is selected at random and a ball is drawn from the bag which is found to be red. Find the probability that the ball is drawn from the first bag.
\\
\solution
		%\begin{table}[H]
	\centering
\begin{tabular}{|c|c|c|}
\hline
Random variable &Value &Definition\\ \hline
\multirow{3}{*}{X} &0 &Slips of Rs 1\\
&1 &Slips of Rs 5\\
&2 &Slips of Rs 13\\ \hline
\multirow{2}{*}{Y} &0 &Box A\\
&1 &Box B\\\hline
\end{tabular}
\caption{}
\label{tab:Distribution}
\end{table}
See \tabref{tab:Distribution}.
\begin{align}
p_{Y}\brak{k}= \begin{cases} 
      \frac{1}{3} & {k=0} \\
      \frac{2}{3 }& {k=1} 
   \end{cases}
   \\
p_{Y|X}\brak{0|0} = \frac{19}{25}\, 
p_{Y|X}\brak{0|1} = \frac{6}{25}\,
p_{Y|X}\brak{1|0} = \frac{45}{50}\,
p_{Y|X}\brak{1|2} = \frac{5}{50}
\end{align}
The desired probability is the probability that a slip drawn at random is marked other than Rs 1,
\begin{align}
&=1-p_X\brak{0}\\
&= p_X(1) + p_X(2)
\end{align}
Using Bayes theorem,
\begin{align}
&= p_Y\brak{0} \times \pr{Y=0 | X=1} + p_Y\brak{1} \times \pr{Y=1|X=2}\\
&=\frac{1}{3} \times \frac{6}{25} + \frac{2}{3} \times \frac{5}{50}\\
&=\frac{11}{75}
\end{align}

\newpage

%\tableofcontents

\bigskip

\renewcommand{\thefigure}{\theenumi}
\renewcommand{\thetable}{\theenumi}
%\renewcommand{\theequation}{\theenumi}

%\begin{abstract}
%%\boldmath
%In this letter, an algorithm for evaluating the exact analytical bit error rate  (BER)  for the piecewise linear (PL) combiner for  multiple relays is presented. Previous results were available only for upto three relays. The algorithm is unique in the sense that  the actual mathematical expressions, that are prohibitively large, need not be explicitly obtained. The diversity gain due to multiple relays is shown through plots of the analytical BER, well supported by simulations. 
%
%\end{abstract}
% IEEEtran.cls defaults to using nonbold math in the Abstract.
% This preserves the distinction between vectors and scalars. However,
% if the journal you are submitting to favors bold math in the abstract,
% then you can use LaTeX's standard command \boldmath at the very start
% of the abstract to achieve this. Many IEEE journals frown on math
% in the abstract anyway.

% Note that keywords are not normally used for peerreview papers.
%\begin{IEEEkeywords}
%Cooperative diversity, decode and forward, piecewise linear
%\end{IEEEkeywords}



% For peer review papers, you can put extra information on the cover
% page as needed:
% \ifCLASSOPTIONpeerreview
% \begin{center} \bfseries EDICS Category: 3-BBND \end{center}
% \fi
%
% For peerreview papers, this IEEEtran command inserts a page break and
% creates the second title. It will be ignored for other modes.
%\IEEEpeerreviewmaketitle




  \item
  Cards with numbers 2 to 101 are placed in a box. A card is selected at random.Find the probability that the card has
\begin{enumerate}[label=(\roman*)]
	\item an even number 
	\item a square number
\end{enumerate}
\solution
%\begin{table}[H]
	\centering
\begin{tabular}{|c|c|c|}
\hline
Random variable &Value &Definition\\ \hline
\multirow{3}{*}{X} &0 &Slips of Rs 1\\
&1 &Slips of Rs 5\\
&2 &Slips of Rs 13\\ \hline
\multirow{2}{*}{Y} &0 &Box A\\
&1 &Box B\\\hline
\end{tabular}
\caption{}
\label{tab:Distribution}
\end{table}
See \tabref{tab:Distribution}.
\begin{align}
p_{Y}\brak{k}= \begin{cases} 
      \frac{1}{3} & {k=0} \\
      \frac{2}{3 }& {k=1} 
   \end{cases}
   \\
p_{Y|X}\brak{0|0} = \frac{19}{25}\, 
p_{Y|X}\brak{0|1} = \frac{6}{25}\,
p_{Y|X}\brak{1|0} = \frac{45}{50}\,
p_{Y|X}\brak{1|2} = \frac{5}{50}
\end{align}
The desired probability is the probability that a slip drawn at random is marked other than Rs 1,
\begin{align}
&=1-p_X\brak{0}\\
&= p_X(1) + p_X(2)
\end{align}
Using Bayes theorem,
\begin{align}
&= p_Y\brak{0} \times \pr{Y=0 | X=1} + p_Y\brak{1} \times \pr{Y=1|X=2}\\
&=\frac{1}{3} \times \frac{6}{25} + \frac{2}{3} \times \frac{5}{50}\\
&=\frac{11}{75}
\end{align}

\newpage

%\tableofcontents

\bigskip

\renewcommand{\thefigure}{\theenumi}
\renewcommand{\thetable}{\theenumi}
%\renewcommand{\theequation}{\theenumi}

%\begin{abstract}
%%\boldmath
%In this letter, an algorithm for evaluating the exact analytical bit error rate  (BER)  for the piecewise linear (PL) combiner for  multiple relays is presented. Previous results were available only for upto three relays. The algorithm is unique in the sense that  the actual mathematical expressions, that are prohibitively large, need not be explicitly obtained. The diversity gain due to multiple relays is shown through plots of the analytical BER, well supported by simulations. 
%
%\end{abstract}
% IEEEtran.cls defaults to using nonbold math in the Abstract.
% This preserves the distinction between vectors and scalars. However,
% if the journal you are submitting to favors bold math in the abstract,
% then you can use LaTeX's standard command \boldmath at the very start
% of the abstract to achieve this. Many IEEE journals frown on math
% in the abstract anyway.

% Note that keywords are not normally used for peerreview papers.
%\begin{IEEEkeywords}
%Cooperative diversity, decode and forward, piecewise linear
%\end{IEEEkeywords}



% For peer review papers, you can put extra information on the cover
% page as needed:
% \ifCLASSOPTIONpeerreview
% \begin{center} \bfseries EDICS Category: 3-BBND \end{center}
% \fi
%
% For peerreview papers, this IEEEtran command inserts a page break and
% creates the second title. It will be ignored for other modes.
%\IEEEpeerreviewmaketitle




\item
The king, queen and jack of clubs are removed from a deck of 52 playing cards and then well shuffled. Now one card is drawn at random from the remaining cards.  Determine the probability that the card is
\begin{enumerate}[label=(\roman*)]
\item a club
\item 10 of hearts
\end{enumerate}
\solution
%\begin{table}[H]
	\centering
\begin{tabular}{|c|c|c|}
\hline
Random variable &Value &Definition\\ \hline
\multirow{3}{*}{X} &0 &Slips of Rs 1\\
&1 &Slips of Rs 5\\
&2 &Slips of Rs 13\\ \hline
\multirow{2}{*}{Y} &0 &Box A\\
&1 &Box B\\\hline
\end{tabular}
\caption{}
\label{tab:Distribution}
\end{table}
See \tabref{tab:Distribution}.
\begin{align}
p_{Y}\brak{k}= \begin{cases} 
      \frac{1}{3} & {k=0} \\
      \frac{2}{3 }& {k=1} 
   \end{cases}
   \\
p_{Y|X}\brak{0|0} = \frac{19}{25}\, 
p_{Y|X}\brak{0|1} = \frac{6}{25}\,
p_{Y|X}\brak{1|0} = \frac{45}{50}\,
p_{Y|X}\brak{1|2} = \frac{5}{50}
\end{align}
The desired probability is the probability that a slip drawn at random is marked other than Rs 1,
\begin{align}
&=1-p_X\brak{0}\\
&= p_X(1) + p_X(2)
\end{align}
Using Bayes theorem,
\begin{align}
&= p_Y\brak{0} \times \pr{Y=0 | X=1} + p_Y\brak{1} \times \pr{Y=1|X=2}\\
&=\frac{1}{3} \times \frac{6}{25} + \frac{2}{3} \times \frac{5}{50}\\
&=\frac{11}{75}
\end{align}

\newpage

%\tableofcontents

\bigskip

\renewcommand{\thefigure}{\theenumi}
\renewcommand{\thetable}{\theenumi}
%\renewcommand{\theequation}{\theenumi}

%\begin{abstract}
%%\boldmath
%In this letter, an algorithm for evaluating the exact analytical bit error rate  (BER)  for the piecewise linear (PL) combiner for  multiple relays is presented. Previous results were available only for upto three relays. The algorithm is unique in the sense that  the actual mathematical expressions, that are prohibitively large, need not be explicitly obtained. The diversity gain due to multiple relays is shown through plots of the analytical BER, well supported by simulations. 
%
%\end{abstract}
% IEEEtran.cls defaults to using nonbold math in the Abstract.
% This preserves the distinction between vectors and scalars. However,
% if the journal you are submitting to favors bold math in the abstract,
% then you can use LaTeX's standard command \boldmath at the very start
% of the abstract to achieve this. Many IEEE journals frown on math
% in the abstract anyway.

% Note that keywords are not normally used for peerreview papers.
%\begin{IEEEkeywords}
%Cooperative diversity, decode and forward, piecewise linear
%\end{IEEEkeywords}



% For peer review papers, you can put extra information on the cover
% page as needed:
% \ifCLASSOPTIONpeerreview
% \begin{center} \bfseries EDICS Category: 3-BBND \end{center}
% \fi
%
% For peerreview papers, this IEEEtran command inserts a page break and
% creates the second title. It will be ignored for other modes.
%\IEEEpeerreviewmaketitle




\item A team of medical students doing their internship have to assist during surgeries
at a city hospital. The probabilities of surgeries rated as very complex, complex,
routine, simple or very simple are respectively, 0.15, 0.20, 0.31, 0.26, .08. Find
the probabilities that a particular surgery will be rated
\begin{enumerate}
	\item complex or very complex;
	\item neither very complex nor very simple;
	\item routine or complex
	\item routine or simple
\end{enumerate}
\solution
%\begin{table}[H]
	\centering
\begin{tabular}{|c|c|c|}
\hline
Random variable &Value &Definition\\ \hline
\multirow{3}{*}{X} &0 &Slips of Rs 1\\
&1 &Slips of Rs 5\\
&2 &Slips of Rs 13\\ \hline
\multirow{2}{*}{Y} &0 &Box A\\
&1 &Box B\\\hline
\end{tabular}
\caption{}
\label{tab:Distribution}
\end{table}
See \tabref{tab:Distribution}.
\begin{align}
p_{Y}\brak{k}= \begin{cases} 
      \frac{1}{3} & {k=0} \\
      \frac{2}{3 }& {k=1} 
   \end{cases}
   \\
p_{Y|X}\brak{0|0} = \frac{19}{25}\, 
p_{Y|X}\brak{0|1} = \frac{6}{25}\,
p_{Y|X}\brak{1|0} = \frac{45}{50}\,
p_{Y|X}\brak{1|2} = \frac{5}{50}
\end{align}
The desired probability is the probability that a slip drawn at random is marked other than Rs 1,
\begin{align}
&=1-p_X\brak{0}\\
&= p_X(1) + p_X(2)
\end{align}
Using Bayes theorem,
\begin{align}
&= p_Y\brak{0} \times \pr{Y=0 | X=1} + p_Y\brak{1} \times \pr{Y=1|X=2}\\
&=\frac{1}{3} \times \frac{6}{25} + \frac{2}{3} \times \frac{5}{50}\\
&=\frac{11}{75}
\end{align}

\newpage

%\tableofcontents

\bigskip

\renewcommand{\thefigure}{\theenumi}
\renewcommand{\thetable}{\theenumi}
%\renewcommand{\theequation}{\theenumi}

%\begin{abstract}
%%\boldmath
%In this letter, an algorithm for evaluating the exact analytical bit error rate  (BER)  for the piecewise linear (PL) combiner for  multiple relays is presented. Previous results were available only for upto three relays. The algorithm is unique in the sense that  the actual mathematical expressions, that are prohibitively large, need not be explicitly obtained. The diversity gain due to multiple relays is shown through plots of the analytical BER, well supported by simulations. 
%
%\end{abstract}
% IEEEtran.cls defaults to using nonbold math in the Abstract.
% This preserves the distinction between vectors and scalars. However,
% if the journal you are submitting to favors bold math in the abstract,
% then you can use LaTeX's standard command \boldmath at the very start
% of the abstract to achieve this. Many IEEE journals frown on math
% in the abstract anyway.

% Note that keywords are not normally used for peerreview papers.
%\begin{IEEEkeywords}
%Cooperative diversity, decode and forward, piecewise linear
%\end{IEEEkeywords}



% For peer review papers, you can put extra information on the cover
% page as needed:
% \ifCLASSOPTIONpeerreview
% \begin{center} \bfseries EDICS Category: 3-BBND \end{center}
% \fi
%
% For peerreview papers, this IEEEtran command inserts a page break and
% creates the second title. It will be ignored for other modes.
%\IEEEpeerreviewmaketitle




\item A card is selected from a pack of 52 cards.
\begin{enumerate}[label=(\alph*)]
    \item How many points are there in the sample space?
    \item Calculate the probability that the card is an ace of spades.
    \item Calculate the probability that the card is (i) an ace and (ii) black card.
\end{enumerate}
\solution
%Let $X$ be an bernoulli rv defined as in \tabref{tab:exemplar/11/16/3/26}.  Then, 
\begin{equation}
    p =
        \frac{4}{11} 
\end{equation}
\begin{table}[H]
	\centering
	\input{exemplar/11/16/3/26/tables/Table2.tex}
	\caption{}
        \label{tab:exemplar/11/16/3/26}
\end{table}

\item The probability that a non leap year selected at random will contain 53 sundays.
\\
\solution
%\begin{table}[H]
	\centering
\begin{tabular}{|c|c|c|}
\hline
Random variable &Value &Definition\\ \hline
\multirow{3}{*}{X} &0 &Slips of Rs 1\\
&1 &Slips of Rs 5\\
&2 &Slips of Rs 13\\ \hline
\multirow{2}{*}{Y} &0 &Box A\\
&1 &Box B\\\hline
\end{tabular}
\caption{}
\label{tab:Distribution}
\end{table}
See \tabref{tab:Distribution}.
\begin{align}
p_{Y}\brak{k}= \begin{cases} 
      \frac{1}{3} & {k=0} \\
      \frac{2}{3 }& {k=1} 
   \end{cases}
   \\
p_{Y|X}\brak{0|0} = \frac{19}{25}\, 
p_{Y|X}\brak{0|1} = \frac{6}{25}\,
p_{Y|X}\brak{1|0} = \frac{45}{50}\,
p_{Y|X}\brak{1|2} = \frac{5}{50}
\end{align}
The desired probability is the probability that a slip drawn at random is marked other than Rs 1,
\begin{align}
&=1-p_X\brak{0}\\
&= p_X(1) + p_X(2)
\end{align}
Using Bayes theorem,
\begin{align}
&= p_Y\brak{0} \times \pr{Y=0 | X=1} + p_Y\brak{1} \times \pr{Y=1|X=2}\\
&=\frac{1}{3} \times \frac{6}{25} + \frac{2}{3} \times \frac{5}{50}\\
&=\frac{11}{75}
\end{align}

\newpage

%\tableofcontents

\bigskip

\renewcommand{\thefigure}{\theenumi}
\renewcommand{\thetable}{\theenumi}
%\renewcommand{\theequation}{\theenumi}

%\begin{abstract}
%%\boldmath
%In this letter, an algorithm for evaluating the exact analytical bit error rate  (BER)  for the piecewise linear (PL) combiner for  multiple relays is presented. Previous results were available only for upto three relays. The algorithm is unique in the sense that  the actual mathematical expressions, that are prohibitively large, need not be explicitly obtained. The diversity gain due to multiple relays is shown through plots of the analytical BER, well supported by simulations. 
%
%\end{abstract}
% IEEEtran.cls defaults to using nonbold math in the Abstract.
% This preserves the distinction between vectors and scalars. However,
% if the journal you are submitting to favors bold math in the abstract,
% then you can use LaTeX's standard command \boldmath at the very start
% of the abstract to achieve this. Many IEEE journals frown on math
% in the abstract anyway.

% Note that keywords are not normally used for peerreview papers.
%\begin{IEEEkeywords}
%Cooperative diversity, decode and forward, piecewise linear
%\end{IEEEkeywords}



% For peer review papers, you can put extra information on the cover
% page as needed:
% \ifCLASSOPTIONpeerreview
% \begin{center} \bfseries EDICS Category: 3-BBND \end{center}
% \fi
%
% For peerreview papers, this IEEEtran command inserts a page break and
% creates the second title. It will be ignored for other modes.
%\IEEEpeerreviewmaketitle




\item One of the four persons John, Rita, Aslam or Gurpreet will be promoted next
month. Consequently the sample space consists of four elementary outcomes
S = {John promoted, Rita promoted, Aslam promoted, Gurpreet promoted}
You are told that the chances of John’s promotion is same as that of Gurpreet,
Rita’s chances of promotion are twice as likely as Johns. Aslam’s chances are
four times that of John.
\begin{enumerate}
	\item Determine
	\begin{enumerate}
		\item P (John promoted)
		\item P (Rita promoted)
		\item P (Aslam promoted)
		\item P (Gurpreet promoted)
	\end{enumerate}
	\item If A = {John promoted or Gurpreet promoted}, find P (A).
\end{enumerate}
\solution
%\begin{table}[H]
	\centering
\begin{tabular}{|c|c|c|}
\hline
Random variable &Value &Definition\\ \hline
\multirow{3}{*}{X} &0 &Slips of Rs 1\\
&1 &Slips of Rs 5\\
&2 &Slips of Rs 13\\ \hline
\multirow{2}{*}{Y} &0 &Box A\\
&1 &Box B\\\hline
\end{tabular}
\caption{}
\label{tab:Distribution}
\end{table}
See \tabref{tab:Distribution}.
\begin{align}
p_{Y}\brak{k}= \begin{cases} 
      \frac{1}{3} & {k=0} \\
      \frac{2}{3 }& {k=1} 
   \end{cases}
   \\
p_{Y|X}\brak{0|0} = \frac{19}{25}\, 
p_{Y|X}\brak{0|1} = \frac{6}{25}\,
p_{Y|X}\brak{1|0} = \frac{45}{50}\,
p_{Y|X}\brak{1|2} = \frac{5}{50}
\end{align}
The desired probability is the probability that a slip drawn at random is marked other than Rs 1,
\begin{align}
&=1-p_X\brak{0}\\
&= p_X(1) + p_X(2)
\end{align}
Using Bayes theorem,
\begin{align}
&= p_Y\brak{0} \times \pr{Y=0 | X=1} + p_Y\brak{1} \times \pr{Y=1|X=2}\\
&=\frac{1}{3} \times \frac{6}{25} + \frac{2}{3} \times \frac{5}{50}\\
&=\frac{11}{75}
\end{align}

\newpage

%\tableofcontents

\bigskip

\renewcommand{\thefigure}{\theenumi}
\renewcommand{\thetable}{\theenumi}
%\renewcommand{\theequation}{\theenumi}

%\begin{abstract}
%%\boldmath
%In this letter, an algorithm for evaluating the exact analytical bit error rate  (BER)  for the piecewise linear (PL) combiner for  multiple relays is presented. Previous results were available only for upto three relays. The algorithm is unique in the sense that  the actual mathematical expressions, that are prohibitively large, need not be explicitly obtained. The diversity gain due to multiple relays is shown through plots of the analytical BER, well supported by simulations. 
%
%\end{abstract}
% IEEEtran.cls defaults to using nonbold math in the Abstract.
% This preserves the distinction between vectors and scalars. However,
% if the journal you are submitting to favors bold math in the abstract,
% then you can use LaTeX's standard command \boldmath at the very start
% of the abstract to achieve this. Many IEEE journals frown on math
% in the abstract anyway.

% Note that keywords are not normally used for peerreview papers.
%\begin{IEEEkeywords}
%Cooperative diversity, decode and forward, piecewise linear
%\end{IEEEkeywords}



% For peer review papers, you can put extra information on the cover
% page as needed:
% \ifCLASSOPTIONpeerreview
% \begin{center} \bfseries EDICS Category: 3-BBND \end{center}
% \fi
%
% For peerreview papers, this IEEEtran command inserts a page break and
% creates the second title. It will be ignored for other modes.
%\IEEEpeerreviewmaketitle




\item A card is drawn from a deck of 52 cards. Find the probability of getting a king or a heart or a red card.\\
\solution
%\begin{table}[H]
	\centering
\begin{tabular}{|c|c|c|}
\hline
Random variable &Value &Definition\\ \hline
\multirow{3}{*}{X} &0 &Slips of Rs 1\\
&1 &Slips of Rs 5\\
&2 &Slips of Rs 13\\ \hline
\multirow{2}{*}{Y} &0 &Box A\\
&1 &Box B\\\hline
\end{tabular}
\caption{}
\label{tab:Distribution}
\end{table}
See \tabref{tab:Distribution}.
\begin{align}
p_{Y}\brak{k}= \begin{cases} 
      \frac{1}{3} & {k=0} \\
      \frac{2}{3 }& {k=1} 
   \end{cases}
   \\
p_{Y|X}\brak{0|0} = \frac{19}{25}\, 
p_{Y|X}\brak{0|1} = \frac{6}{25}\,
p_{Y|X}\brak{1|0} = \frac{45}{50}\,
p_{Y|X}\brak{1|2} = \frac{5}{50}
\end{align}
The desired probability is the probability that a slip drawn at random is marked other than Rs 1,
\begin{align}
&=1-p_X\brak{0}\\
&= p_X(1) + p_X(2)
\end{align}
Using Bayes theorem,
\begin{align}
&= p_Y\brak{0} \times \pr{Y=0 | X=1} + p_Y\brak{1} \times \pr{Y=1|X=2}\\
&=\frac{1}{3} \times \frac{6}{25} + \frac{2}{3} \times \frac{5}{50}\\
&=\frac{11}{75}
\end{align}

\newpage

%\tableofcontents

\bigskip

\renewcommand{\thefigure}{\theenumi}
\renewcommand{\thetable}{\theenumi}
%\renewcommand{\theequation}{\theenumi}

%\begin{abstract}
%%\boldmath
%In this letter, an algorithm for evaluating the exact analytical bit error rate  (BER)  for the piecewise linear (PL) combiner for  multiple relays is presented. Previous results were available only for upto three relays. The algorithm is unique in the sense that  the actual mathematical expressions, that are prohibitively large, need not be explicitly obtained. The diversity gain due to multiple relays is shown through plots of the analytical BER, well supported by simulations. 
%
%\end{abstract}
% IEEEtran.cls defaults to using nonbold math in the Abstract.
% This preserves the distinction between vectors and scalars. However,
% if the journal you are submitting to favors bold math in the abstract,
% then you can use LaTeX's standard command \boldmath at the very start
% of the abstract to achieve this. Many IEEE journals frown on math
% in the abstract anyway.

% Note that keywords are not normally used for peerreview papers.
%\begin{IEEEkeywords}
%Cooperative diversity, decode and forward, piecewise linear
%\end{IEEEkeywords}



% For peer review papers, you can put extra information on the cover
% page as needed:
% \ifCLASSOPTIONpeerreview
% \begin{center} \bfseries EDICS Category: 3-BBND \end{center}
% \fi
%
% For peerreview papers, this IEEEtran command inserts a page break and
% creates the second title. It will be ignored for other modes.
%\IEEEpeerreviewmaketitle




\item The probability that a student will pass his examination is 0.73, the probability of
the student getting a compartment is 0.13, and the probability that the student will
either pass or get compartment is 0.96. State True or False.\\
\solution
%\begin{table}[H]
	\centering
\begin{tabular}{|c|c|c|}
\hline
Random variable &Value &Definition\\ \hline
\multirow{3}{*}{X} &0 &Slips of Rs 1\\
&1 &Slips of Rs 5\\
&2 &Slips of Rs 13\\ \hline
\multirow{2}{*}{Y} &0 &Box A\\
&1 &Box B\\\hline
\end{tabular}
\caption{}
\label{tab:Distribution}
\end{table}
See \tabref{tab:Distribution}.
\begin{align}
p_{Y}\brak{k}= \begin{cases} 
      \frac{1}{3} & {k=0} \\
      \frac{2}{3 }& {k=1} 
   \end{cases}
   \\
p_{Y|X}\brak{0|0} = \frac{19}{25}\, 
p_{Y|X}\brak{0|1} = \frac{6}{25}\,
p_{Y|X}\brak{1|0} = \frac{45}{50}\,
p_{Y|X}\brak{1|2} = \frac{5}{50}
\end{align}
The desired probability is the probability that a slip drawn at random is marked other than Rs 1,
\begin{align}
&=1-p_X\brak{0}\\
&= p_X(1) + p_X(2)
\end{align}
Using Bayes theorem,
\begin{align}
&= p_Y\brak{0} \times \pr{Y=0 | X=1} + p_Y\brak{1} \times \pr{Y=1|X=2}\\
&=\frac{1}{3} \times \frac{6}{25} + \frac{2}{3} \times \frac{5}{50}\\
&=\frac{11}{75}
\end{align}

\newpage

%\tableofcontents

\bigskip

\renewcommand{\thefigure}{\theenumi}
\renewcommand{\thetable}{\theenumi}
%\renewcommand{\theequation}{\theenumi}

%\begin{abstract}
%%\boldmath
%In this letter, an algorithm for evaluating the exact analytical bit error rate  (BER)  for the piecewise linear (PL) combiner for  multiple relays is presented. Previous results were available only for upto three relays. The algorithm is unique in the sense that  the actual mathematical expressions, that are prohibitively large, need not be explicitly obtained. The diversity gain due to multiple relays is shown through plots of the analytical BER, well supported by simulations. 
%
%\end{abstract}
% IEEEtran.cls defaults to using nonbold math in the Abstract.
% This preserves the distinction between vectors and scalars. However,
% if the journal you are submitting to favors bold math in the abstract,
% then you can use LaTeX's standard command \boldmath at the very start
% of the abstract to achieve this. Many IEEE journals frown on math
% in the abstract anyway.

% Note that keywords are not normally used for peerreview papers.
%\begin{IEEEkeywords}
%Cooperative diversity, decode and forward, piecewise linear
%\end{IEEEkeywords}



% For peer review papers, you can put extra information on the cover
% page as needed:
% \ifCLASSOPTIONpeerreview
% \begin{center} \bfseries EDICS Category: 3-BBND \end{center}
% \fi
%
% For peerreview papers, this IEEEtran command inserts a page break and
% creates the second title. It will be ignored for other modes.
%\IEEEpeerreviewmaketitle




\item A card is selected from a pack of 52 cards\\
\begin{enumerate}[label=(\alph*)]
\item How many points are there in the sample space?
\item Calculate the probability that the cards is an ace of spades.
\item Calculate the probability that the card is (i) an ace (ii)black card.\\
\end{enumerate}
%\input{ncert/11/16/3/4_1/Prob_4.tex}
\item In a non-leap year, the probability of having 53 tuesdays or 53 wednesdays is\\
\solution
%A non-leap year has a total of 365 days, and a week has 7 days.\\
So it can be expressed as 
\begin{align}
365\text{days} &=52\times 7+1 \text{day}
\end{align}
$\implies$ 52 tuesdays or wednesdays\\
Random variable X denotes the days of a week
\begin{align}
p_X\brak{k}&=\frac{1}{7}; \quad \brak{1<k<7}
\end{align}
So the probability of extra day being tuesday or wednesday is
\begin{align}
p_X\brak{3}+p_X\brak{4}&=\frac{1}{7}+\frac{1}{7}=\frac{2}{7}
\end{align}



\item There are 1000 sealed envelopes in a box, 10 of them contain a cash prize of
Rs 100 each, 100 of them contain a cash prize of Rs 50 each and 200 of them
contain a cash prize of Rs 10 each and rest do not contain any cash prize. If they
are well shuffled and an envelope is picked up out, what is the probability that it
contains no cash prize?\\
\solution
%\begin{table}[H]
	\centering
\begin{tabular}{|c|c|c|}
\hline
Random variable &Value &Definition\\ \hline
\multirow{3}{*}{X} &0 &Slips of Rs 1\\
&1 &Slips of Rs 5\\
&2 &Slips of Rs 13\\ \hline
\multirow{2}{*}{Y} &0 &Box A\\
&1 &Box B\\\hline
\end{tabular}
\caption{}
\label{tab:Distribution}
\end{table}
See \tabref{tab:Distribution}.
\begin{align}
p_{Y}\brak{k}= \begin{cases} 
      \frac{1}{3} & {k=0} \\
      \frac{2}{3 }& {k=1} 
   \end{cases}
   \\
p_{Y|X}\brak{0|0} = \frac{19}{25}\, 
p_{Y|X}\brak{0|1} = \frac{6}{25}\,
p_{Y|X}\brak{1|0} = \frac{45}{50}\,
p_{Y|X}\brak{1|2} = \frac{5}{50}
\end{align}
The desired probability is the probability that a slip drawn at random is marked other than Rs 1,
\begin{align}
&=1-p_X\brak{0}\\
&= p_X(1) + p_X(2)
\end{align}
Using Bayes theorem,
\begin{align}
&= p_Y\brak{0} \times \pr{Y=0 | X=1} + p_Y\brak{1} \times \pr{Y=1|X=2}\\
&=\frac{1}{3} \times \frac{6}{25} + \frac{2}{3} \times \frac{5}{50}\\
&=\frac{11}{75}
\end{align}

\newpage

%\tableofcontents

\bigskip

\renewcommand{\thefigure}{\theenumi}
\renewcommand{\thetable}{\theenumi}
%\renewcommand{\theequation}{\theenumi}

%\begin{abstract}
%%\boldmath
%In this letter, an algorithm for evaluating the exact analytical bit error rate  (BER)  for the piecewise linear (PL) combiner for  multiple relays is presented. Previous results were available only for upto three relays. The algorithm is unique in the sense that  the actual mathematical expressions, that are prohibitively large, need not be explicitly obtained. The diversity gain due to multiple relays is shown through plots of the analytical BER, well supported by simulations. 
%
%\end{abstract}
% IEEEtran.cls defaults to using nonbold math in the Abstract.
% This preserves the distinction between vectors and scalars. However,
% if the journal you are submitting to favors bold math in the abstract,
% then you can use LaTeX's standard command \boldmath at the very start
% of the abstract to achieve this. Many IEEE journals frown on math
% in the abstract anyway.

% Note that keywords are not normally used for peerreview papers.
%\begin{IEEEkeywords}
%Cooperative diversity, decode and forward, piecewise linear
%\end{IEEEkeywords}



% For peer review papers, you can put extra information on the cover
% page as needed:
% \ifCLASSOPTIONpeerreview
% \begin{center} \bfseries EDICS Category: 3-BBND \end{center}
% \fi
%
% For peerreview papers, this IEEEtran command inserts a page break and
% creates the second title. It will be ignored for other modes.
%\IEEEpeerreviewmaketitle




\item 
A die is thrown and a card is selected at random from a deck of 52 playing cards. The probability of getting an even number on the die and a spade card.\\
\solution
%\begin{table}[H]
	\centering
\begin{tabular}{|c|c|c|}
\hline
Random variable &Value &Definition\\ \hline
\multirow{3}{*}{X} &0 &Slips of Rs 1\\
&1 &Slips of Rs 5\\
&2 &Slips of Rs 13\\ \hline
\multirow{2}{*}{Y} &0 &Box A\\
&1 &Box B\\\hline
\end{tabular}
\caption{}
\label{tab:Distribution}
\end{table}
See \tabref{tab:Distribution}.
\begin{align}
p_{Y}\brak{k}= \begin{cases} 
      \frac{1}{3} & {k=0} \\
      \frac{2}{3 }& {k=1} 
   \end{cases}
   \\
p_{Y|X}\brak{0|0} = \frac{19}{25}\, 
p_{Y|X}\brak{0|1} = \frac{6}{25}\,
p_{Y|X}\brak{1|0} = \frac{45}{50}\,
p_{Y|X}\brak{1|2} = \frac{5}{50}
\end{align}
The desired probability is the probability that a slip drawn at random is marked other than Rs 1,
\begin{align}
&=1-p_X\brak{0}\\
&= p_X(1) + p_X(2)
\end{align}
Using Bayes theorem,
\begin{align}
&= p_Y\brak{0} \times \pr{Y=0 | X=1} + p_Y\brak{1} \times \pr{Y=1|X=2}\\
&=\frac{1}{3} \times \frac{6}{25} + \frac{2}{3} \times \frac{5}{50}\\
&=\frac{11}{75}
\end{align}

\newpage

%\tableofcontents

\bigskip

\renewcommand{\thefigure}{\theenumi}
\renewcommand{\thetable}{\theenumi}
%\renewcommand{\theequation}{\theenumi}

%\begin{abstract}
%%\boldmath
%In this letter, an algorithm for evaluating the exact analytical bit error rate  (BER)  for the piecewise linear (PL) combiner for  multiple relays is presented. Previous results were available only for upto three relays. The algorithm is unique in the sense that  the actual mathematical expressions, that are prohibitively large, need not be explicitly obtained. The diversity gain due to multiple relays is shown through plots of the analytical BER, well supported by simulations. 
%
%\end{abstract}
% IEEEtran.cls defaults to using nonbold math in the Abstract.
% This preserves the distinction between vectors and scalars. However,
% if the journal you are submitting to favors bold math in the abstract,
% then you can use LaTeX's standard command \boldmath at the very start
% of the abstract to achieve this. Many IEEE journals frown on math
% in the abstract anyway.

% Note that keywords are not normally used for peerreview papers.
%\begin{IEEEkeywords}
%Cooperative diversity, decode and forward, piecewise linear
%\end{IEEEkeywords}



% For peer review papers, you can put extra information on the cover
% page as needed:
% \ifCLASSOPTIONpeerreview
% \begin{center} \bfseries EDICS Category: 3-BBND \end{center}
% \fi
%
% For peerreview papers, this IEEEtran command inserts a page break and
% creates the second title. It will be ignored for other modes.
%\IEEEpeerreviewmaketitle




\item
If 4-digit numbers greater than 5,000 are randomly formed from the digits 0, 1, 3, 5, and 7, what is the probability of forming a number divisible by 5 when:
\begin{enumerate}
    \item The digits are repeated?
    \item The repetition of digits is not allowed?
\end{enumerate}
\solution
%\begin{table}[H]
	\centering
\begin{tabular}{|c|c|c|}
\hline
Random variable &Value &Definition\\ \hline
\multirow{3}{*}{X} &0 &Slips of Rs 1\\
&1 &Slips of Rs 5\\
&2 &Slips of Rs 13\\ \hline
\multirow{2}{*}{Y} &0 &Box A\\
&1 &Box B\\\hline
\end{tabular}
\caption{}
\label{tab:Distribution}
\end{table}
See \tabref{tab:Distribution}.
\begin{align}
p_{Y}\brak{k}= \begin{cases} 
      \frac{1}{3} & {k=0} \\
      \frac{2}{3 }& {k=1} 
   \end{cases}
   \\
p_{Y|X}\brak{0|0} = \frac{19}{25}\, 
p_{Y|X}\brak{0|1} = \frac{6}{25}\,
p_{Y|X}\brak{1|0} = \frac{45}{50}\,
p_{Y|X}\brak{1|2} = \frac{5}{50}
\end{align}
The desired probability is the probability that a slip drawn at random is marked other than Rs 1,
\begin{align}
&=1-p_X\brak{0}\\
&= p_X(1) + p_X(2)
\end{align}
Using Bayes theorem,
\begin{align}
&= p_Y\brak{0} \times \pr{Y=0 | X=1} + p_Y\brak{1} \times \pr{Y=1|X=2}\\
&=\frac{1}{3} \times \frac{6}{25} + \frac{2}{3} \times \frac{5}{50}\\
&=\frac{11}{75}
\end{align}

\newpage

%\tableofcontents

\bigskip

\renewcommand{\thefigure}{\theenumi}
\renewcommand{\thetable}{\theenumi}
%\renewcommand{\theequation}{\theenumi}

%\begin{abstract}
%%\boldmath
%In this letter, an algorithm for evaluating the exact analytical bit error rate  (BER)  for the piecewise linear (PL) combiner for  multiple relays is presented. Previous results were available only for upto three relays. The algorithm is unique in the sense that  the actual mathematical expressions, that are prohibitively large, need not be explicitly obtained. The diversity gain due to multiple relays is shown through plots of the analytical BER, well supported by simulations. 
%
%\end{abstract}
% IEEEtran.cls defaults to using nonbold math in the Abstract.
% This preserves the distinction between vectors and scalars. However,
% if the journal you are submitting to favors bold math in the abstract,
% then you can use LaTeX's standard command \boldmath at the very start
% of the abstract to achieve this. Many IEEE journals frown on math
% in the abstract anyway.

% Note that keywords are not normally used for peerreview papers.
%\begin{IEEEkeywords}
%Cooperative diversity, decode and forward, piecewise linear
%\end{IEEEkeywords}



% For peer review papers, you can put extra information on the cover
% page as needed:
% \ifCLASSOPTIONpeerreview
% \begin{center} \bfseries EDICS Category: 3-BBND \end{center}
% \fi
%
% For peerreview papers, this IEEEtran command inserts a page break and
% creates the second title. It will be ignored for other modes.
%\IEEEpeerreviewmaketitle




\item Consider the probability space $\brak{\Omega, \mathcal{G}, P}$ where $\Omega = [0,2]$ and $\mathcal{G} = \cbrak{\phi, \Omega, [0,1], (1,2]}$. Let $X$ and $Y$ be two functions on $\Omega$ defined as
\begin{align*}
    X(\omega) = 
    \begin{cases}
        1 & \text{if }\omega \in [0, 1]\\
        2 & \text{if }\omega \in (1, 2]
    \end{cases}
\end{align*}
and
\begin{align*}
    Y(\omega) = 
    \begin{cases}
        2 & \text{if }\omega \in [0, 1.5]\\
        3 & \text{if }\omega \in (1.5, 2].
    \end{cases}
\end{align*}
Then which one of the following statements is true?
\begin{enumerate}
    \item [(A)] $X$ is a random variable with respect to $\mathcal{G}$, but $Y$ is not a random variable with respect to $\mathcal{G}$.
    \item [(B)] $Y$ is a random variable with respect to $\mathcal{G}$, but $X$ is not a random variable with respect to $\mathcal{G}$.
    \item [(C)] Neither $X$ nor $Y$ is a random variable with respect to $\mathcal{G}$.
    \item [(D)] Both $X$ and $Y$ are random variables with respect to $\mathcal{G}$.
\end{enumerate} \hfill (GATE ST 2023)\\
\solution
%\begin{table}[H]
	\centering
\begin{tabular}{|c|c|c|}
\hline
Random variable &Value &Definition\\ \hline
\multirow{3}{*}{X} &0 &Slips of Rs 1\\
&1 &Slips of Rs 5\\
&2 &Slips of Rs 13\\ \hline
\multirow{2}{*}{Y} &0 &Box A\\
&1 &Box B\\\hline
\end{tabular}
\caption{}
\label{tab:Distribution}
\end{table}
See \tabref{tab:Distribution}.
\begin{align}
p_{Y}\brak{k}= \begin{cases} 
      \frac{1}{3} & {k=0} \\
      \frac{2}{3 }& {k=1} 
   \end{cases}
   \\
p_{Y|X}\brak{0|0} = \frac{19}{25}\, 
p_{Y|X}\brak{0|1} = \frac{6}{25}\,
p_{Y|X}\brak{1|0} = \frac{45}{50}\,
p_{Y|X}\brak{1|2} = \frac{5}{50}
\end{align}
The desired probability is the probability that a slip drawn at random is marked other than Rs 1,
\begin{align}
&=1-p_X\brak{0}\\
&= p_X(1) + p_X(2)
\end{align}
Using Bayes theorem,
\begin{align}
&= p_Y\brak{0} \times \pr{Y=0 | X=1} + p_Y\brak{1} \times \pr{Y=1|X=2}\\
&=\frac{1}{3} \times \frac{6}{25} + \frac{2}{3} \times \frac{5}{50}\\
&=\frac{11}{75}
\end{align}

\newpage

%\tableofcontents

\bigskip

\renewcommand{\thefigure}{\theenumi}
\renewcommand{\thetable}{\theenumi}
%\renewcommand{\theequation}{\theenumi}

%\begin{abstract}
%%\boldmath
%In this letter, an algorithm for evaluating the exact analytical bit error rate  (BER)  for the piecewise linear (PL) combiner for  multiple relays is presented. Previous results were available only for upto three relays. The algorithm is unique in the sense that  the actual mathematical expressions, that are prohibitively large, need not be explicitly obtained. The diversity gain due to multiple relays is shown through plots of the analytical BER, well supported by simulations. 
%
%\end{abstract}
% IEEEtran.cls defaults to using nonbold math in the Abstract.
% This preserves the distinction between vectors and scalars. However,
% if the journal you are submitting to favors bold math in the abstract,
% then you can use LaTeX's standard command \boldmath at the very start
% of the abstract to achieve this. Many IEEE journals frown on math
% in the abstract anyway.

% Note that keywords are not normally used for peerreview papers.
%\begin{IEEEkeywords}
%Cooperative diversity, decode and forward, piecewise linear
%\end{IEEEkeywords}



% For peer review papers, you can put extra information on the cover
% page as needed:
% \ifCLASSOPTIONpeerreview
% \begin{center} \bfseries EDICS Category: 3-BBND \end{center}
% \fi
%
% For peerreview papers, this IEEEtran command inserts a page break and
% creates the second title. It will be ignored for other modes.
%\IEEEpeerreviewmaketitle




	\item  A die is loaded in such a way that each odd number is twice as likely to occur as
each even number. Find $P(G)$, where $G$ is the event that a number greater than
3 occurs on a single roll of the die.
\\
\solution
		%\begin{table}[H]
	\centering
\begin{tabular}{|c|c|c|}
\hline
Random variable &Value &Definition\\ \hline
\multirow{3}{*}{X} &0 &Slips of Rs 1\\
&1 &Slips of Rs 5\\
&2 &Slips of Rs 13\\ \hline
\multirow{2}{*}{Y} &0 &Box A\\
&1 &Box B\\\hline
\end{tabular}
\caption{}
\label{tab:Distribution}
\end{table}
See \tabref{tab:Distribution}.
\begin{align}
p_{Y}\brak{k}= \begin{cases} 
      \frac{1}{3} & {k=0} \\
      \frac{2}{3 }& {k=1} 
   \end{cases}
   \\
p_{Y|X}\brak{0|0} = \frac{19}{25}\, 
p_{Y|X}\brak{0|1} = \frac{6}{25}\,
p_{Y|X}\brak{1|0} = \frac{45}{50}\,
p_{Y|X}\brak{1|2} = \frac{5}{50}
\end{align}
The desired probability is the probability that a slip drawn at random is marked other than Rs 1,
\begin{align}
&=1-p_X\brak{0}\\
&= p_X(1) + p_X(2)
\end{align}
Using Bayes theorem,
\begin{align}
&= p_Y\brak{0} \times \pr{Y=0 | X=1} + p_Y\brak{1} \times \pr{Y=1|X=2}\\
&=\frac{1}{3} \times \frac{6}{25} + \frac{2}{3} \times \frac{5}{50}\\
&=\frac{11}{75}
\end{align}

\newpage

%\tableofcontents

\bigskip

\renewcommand{\thefigure}{\theenumi}
\renewcommand{\thetable}{\theenumi}
%\renewcommand{\theequation}{\theenumi}

%\begin{abstract}
%%\boldmath
%In this letter, an algorithm for evaluating the exact analytical bit error rate  (BER)  for the piecewise linear (PL) combiner for  multiple relays is presented. Previous results were available only for upto three relays. The algorithm is unique in the sense that  the actual mathematical expressions, that are prohibitively large, need not be explicitly obtained. The diversity gain due to multiple relays is shown through plots of the analytical BER, well supported by simulations. 
%
%\end{abstract}
% IEEEtran.cls defaults to using nonbold math in the Abstract.
% This preserves the distinction between vectors and scalars. However,
% if the journal you are submitting to favors bold math in the abstract,
% then you can use LaTeX's standard command \boldmath at the very start
% of the abstract to achieve this. Many IEEE journals frown on math
% in the abstract anyway.

% Note that keywords are not normally used for peerreview papers.
%\begin{IEEEkeywords}
%Cooperative diversity, decode and forward, piecewise linear
%\end{IEEEkeywords}



% For peer review papers, you can put extra information on the cover
% page as needed:
% \ifCLASSOPTIONpeerreview
% \begin{center} \bfseries EDICS Category: 3-BBND \end{center}
% \fi
%
% For peerreview papers, this IEEEtran command inserts a page break and
% creates the second title. It will be ignored for other modes.
%\IEEEpeerreviewmaketitle




	\item All the jacks, queens and kings are removed from a deck of 52 playing cards. The remaining cards are well shuffled and then one card is drawn at random. Giving ace a value 1 similar value for other cards, find the probability that the card has a value 
		\begin{enumerate}
			\item 7
			\item greater than 7
			\item less than 7
		\end{enumerate}
		%Number of cards left after removing all jacks, queens and kings 
\begin{align}
N	= 52 - 4\times 3
	= 40
\end{align}
%\begin{table}[H]
%\def\arraystretch{1.2}
%\begin{tabular}{|c|c|c|}
%\hline
%	\textbf{Parameter} &\textbf{Value} &\textbf{Description}\\ \hline
%	$X$ &1-10 &Represents the value of the card picked \\ \hline
%\end{tabular}
%\end{table}
Let $1 \le X \le 10$ be the value of the card picked.  Then,
\begin{align}
	p_X(k) &= \Pr(X=k)\ \forall\ 1 \leq k \leq 10\\
	&= \frac{4\times 1}{40}\\
	&= \frac{1}{10}\\
	\therefore p_X(k) &= 
	\begin{cases}
		\frac{1}{10} & 1 \leq k \leq 10\\
		0 & \text{otherwise}
	\end{cases}
\end{align}
and
\begin{align}
	F_{X}(k) &= \sum_{m=0}^{k}p_{X}(m) \quad 1 \leq k \leq 10\\
	&= \frac{k}{10}\\
	\therefore F_{X}(k) &= 
	\begin{cases}
		0 & k \leq 0\\
		\frac{k}{10} & 1\leq k \leq 10\\
		1 & k > 10 
	\end{cases}
\end{align}
\begin{enumerate}
	\item Probability that card has value equal to 7 is
		\begin{align}
			 p_{X}(7)
			= \frac{1}{10}
		\end{align}
	\item Probability that card has value greater than 7 is
		\begin{align}
			1 - F_X(7)
			&= 1 - \frac{7}{10}
			\\
			&= \frac{3}{10}
		\end{align}
	\item Probability that card has value less than 7 is
		\begin{align}
			 F_{X}(6)
			=\frac{6}{10}
		\end{align}
\end{enumerate}

  \item A Lot consists of 48 mobile phones of which 42 are good, 3 have only minor defects and 3 have major defects.Varnika will buy a phone if it is good but the trader will only buy a mobile if it has no major defects. One phone is selected at random from the lot. What is the probability that it is
\begin{enumerate}
	\item acceptable to Varnika?
            \item acceptable to the trader?
\end{enumerate}
\solution
	%\begin{table}[H]
	\centering
\begin{tabular}{|c|c|c|}
\hline
Random variable &Value &Definition\\ \hline
\multirow{3}{*}{X} &0 &Slips of Rs 1\\
&1 &Slips of Rs 5\\
&2 &Slips of Rs 13\\ \hline
\multirow{2}{*}{Y} &0 &Box A\\
&1 &Box B\\\hline
\end{tabular}
\caption{}
\label{tab:Distribution}
\end{table}
See \tabref{tab:Distribution}.
\begin{align}
p_{Y}\brak{k}= \begin{cases} 
      \frac{1}{3} & {k=0} \\
      \frac{2}{3 }& {k=1} 
   \end{cases}
   \\
p_{Y|X}\brak{0|0} = \frac{19}{25}\, 
p_{Y|X}\brak{0|1} = \frac{6}{25}\,
p_{Y|X}\brak{1|0} = \frac{45}{50}\,
p_{Y|X}\brak{1|2} = \frac{5}{50}
\end{align}
The desired probability is the probability that a slip drawn at random is marked other than Rs 1,
\begin{align}
&=1-p_X\brak{0}\\
&= p_X(1) + p_X(2)
\end{align}
Using Bayes theorem,
\begin{align}
&= p_Y\brak{0} \times \pr{Y=0 | X=1} + p_Y\brak{1} \times \pr{Y=1|X=2}\\
&=\frac{1}{3} \times \frac{6}{25} + \frac{2}{3} \times \frac{5}{50}\\
&=\frac{11}{75}
\end{align}

\newpage

%\tableofcontents

\bigskip

\renewcommand{\thefigure}{\theenumi}
\renewcommand{\thetable}{\theenumi}
%\renewcommand{\theequation}{\theenumi}

%\begin{abstract}
%%\boldmath
%In this letter, an algorithm for evaluating the exact analytical bit error rate  (BER)  for the piecewise linear (PL) combiner for  multiple relays is presented. Previous results were available only for upto three relays. The algorithm is unique in the sense that  the actual mathematical expressions, that are prohibitively large, need not be explicitly obtained. The diversity gain due to multiple relays is shown through plots of the analytical BER, well supported by simulations. 
%
%\end{abstract}
% IEEEtran.cls defaults to using nonbold math in the Abstract.
% This preserves the distinction between vectors and scalars. However,
% if the journal you are submitting to favors bold math in the abstract,
% then you can use LaTeX's standard command \boldmath at the very start
% of the abstract to achieve this. Many IEEE journals frown on math
% in the abstract anyway.

% Note that keywords are not normally used for peerreview papers.
%\begin{IEEEkeywords}
%Cooperative diversity, decode and forward, piecewise linear
%\end{IEEEkeywords}



% For peer review papers, you can put extra information on the cover
% page as needed:
% \ifCLASSOPTIONpeerreview
% \begin{center} \bfseries EDICS Category: 3-BBND \end{center}
% \fi
%
% For peerreview papers, this IEEEtran command inserts a page break and
% creates the second title. It will be ignored for other modes.
%\IEEEpeerreviewmaketitle




 \item A student says that if you throw a die, it will show up 1 or not 1. Therefore, the probability of getting 1 and the probability of getting 'not 1' each is equal to $\frac{1}{2}$. Is this correct? Give reasons.\\
 \solution
        %\begin{table}[H]
	\centering
\begin{tabular}{|c|c|c|}
\hline
Random variable &Value &Definition\\ \hline
\multirow{3}{*}{X} &0 &Slips of Rs 1\\
&1 &Slips of Rs 5\\
&2 &Slips of Rs 13\\ \hline
\multirow{2}{*}{Y} &0 &Box A\\
&1 &Box B\\\hline
\end{tabular}
\caption{}
\label{tab:Distribution}
\end{table}
See \tabref{tab:Distribution}.
\begin{align}
p_{Y}\brak{k}= \begin{cases} 
      \frac{1}{3} & {k=0} \\
      \frac{2}{3 }& {k=1} 
   \end{cases}
   \\
p_{Y|X}\brak{0|0} = \frac{19}{25}\, 
p_{Y|X}\brak{0|1} = \frac{6}{25}\,
p_{Y|X}\brak{1|0} = \frac{45}{50}\,
p_{Y|X}\brak{1|2} = \frac{5}{50}
\end{align}
The desired probability is the probability that a slip drawn at random is marked other than Rs 1,
\begin{align}
&=1-p_X\brak{0}\\
&= p_X(1) + p_X(2)
\end{align}
Using Bayes theorem,
\begin{align}
&= p_Y\brak{0} \times \pr{Y=0 | X=1} + p_Y\brak{1} \times \pr{Y=1|X=2}\\
&=\frac{1}{3} \times \frac{6}{25} + \frac{2}{3} \times \frac{5}{50}\\
&=\frac{11}{75}
\end{align}

\newpage

%\tableofcontents

\bigskip

\renewcommand{\thefigure}{\theenumi}
\renewcommand{\thetable}{\theenumi}
%\renewcommand{\theequation}{\theenumi}

%\begin{abstract}
%%\boldmath
%In this letter, an algorithm for evaluating the exact analytical bit error rate  (BER)  for the piecewise linear (PL) combiner for  multiple relays is presented. Previous results were available only for upto three relays. The algorithm is unique in the sense that  the actual mathematical expressions, that are prohibitively large, need not be explicitly obtained. The diversity gain due to multiple relays is shown through plots of the analytical BER, well supported by simulations. 
%
%\end{abstract}
% IEEEtran.cls defaults to using nonbold math in the Abstract.
% This preserves the distinction between vectors and scalars. However,
% if the journal you are submitting to favors bold math in the abstract,
% then you can use LaTeX's standard command \boldmath at the very start
% of the abstract to achieve this. Many IEEE journals frown on math
% in the abstract anyway.

% Note that keywords are not normally used for peerreview papers.
%\begin{IEEEkeywords}
%Cooperative diversity, decode and forward, piecewise linear
%\end{IEEEkeywords}



% For peer review papers, you can put extra information on the cover
% page as needed:
% \ifCLASSOPTIONpeerreview
% \begin{center} \bfseries EDICS Category: 3-BBND \end{center}
% \fi
%
% For peerreview papers, this IEEEtran command inserts a page break and
% creates the second title. It will be ignored for other modes.
%\IEEEpeerreviewmaketitle




   \item Four candidates A, B, C, D have ap-
plied for the assignment to coach a school cricket
team. If A is twice as likely to be selected as B, and
B and C are given about the same chance of being
selected, while C is twice as likely to be selected
as D, what are the probabilities that
\begin{enumerate}
\item C will be selected?
\item A will not be selected?
\end{enumerate}
	%\begin{table}[H]
	\centering
\begin{tabular}{|c|c|c|}
\hline
Random variable &Value &Definition\\ \hline
\multirow{3}{*}{X} &0 &Slips of Rs 1\\
&1 &Slips of Rs 5\\
&2 &Slips of Rs 13\\ \hline
\multirow{2}{*}{Y} &0 &Box A\\
&1 &Box B\\\hline
\end{tabular}
\caption{}
\label{tab:Distribution}
\end{table}
See \tabref{tab:Distribution}.
\begin{align}
p_{Y}\brak{k}= \begin{cases} 
      \frac{1}{3} & {k=0} \\
      \frac{2}{3 }& {k=1} 
   \end{cases}
   \\
p_{Y|X}\brak{0|0} = \frac{19}{25}\, 
p_{Y|X}\brak{0|1} = \frac{6}{25}\,
p_{Y|X}\brak{1|0} = \frac{45}{50}\,
p_{Y|X}\brak{1|2} = \frac{5}{50}
\end{align}
The desired probability is the probability that a slip drawn at random is marked other than Rs 1,
\begin{align}
&=1-p_X\brak{0}\\
&= p_X(1) + p_X(2)
\end{align}
Using Bayes theorem,
\begin{align}
&= p_Y\brak{0} \times \pr{Y=0 | X=1} + p_Y\brak{1} \times \pr{Y=1|X=2}\\
&=\frac{1}{3} \times \frac{6}{25} + \frac{2}{3} \times \frac{5}{50}\\
&=\frac{11}{75}
\end{align}

\newpage

%\tableofcontents

\bigskip

\renewcommand{\thefigure}{\theenumi}
\renewcommand{\thetable}{\theenumi}
%\renewcommand{\theequation}{\theenumi}

%\begin{abstract}
%%\boldmath
%In this letter, an algorithm for evaluating the exact analytical bit error rate  (BER)  for the piecewise linear (PL) combiner for  multiple relays is presented. Previous results were available only for upto three relays. The algorithm is unique in the sense that  the actual mathematical expressions, that are prohibitively large, need not be explicitly obtained. The diversity gain due to multiple relays is shown through plots of the analytical BER, well supported by simulations. 
%
%\end{abstract}
% IEEEtran.cls defaults to using nonbold math in the Abstract.
% This preserves the distinction between vectors and scalars. However,
% if the journal you are submitting to favors bold math in the abstract,
% then you can use LaTeX's standard command \boldmath at the very start
% of the abstract to achieve this. Many IEEE journals frown on math
% in the abstract anyway.

% Note that keywords are not normally used for peerreview papers.
%\begin{IEEEkeywords}
%Cooperative diversity, decode and forward, piecewise linear
%\end{IEEEkeywords}



% For peer review papers, you can put extra information on the cover
% page as needed:
% \ifCLASSOPTIONpeerreview
% \begin{center} \bfseries EDICS Category: 3-BBND \end{center}
% \fi
%
% For peerreview papers, this IEEEtran command inserts a page break and
% creates the second title. It will be ignored for other modes.
%\IEEEpeerreviewmaketitle




 \item A bag contain 24 balls of which $x$ balls are red, $2x$ are white and $3x$ are blue. A ball is selected at random, What is the probability that it is
\begin{enumerate}[label=\alph*)]
\item not red ?
\item white ?
\end{enumerate}
%\begin{table}[H]
	\centering
\begin{tabular}{|c|c|c|}
\hline
Random variable &Value &Definition\\ \hline
\multirow{3}{*}{X} &0 &Slips of Rs 1\\
&1 &Slips of Rs 5\\
&2 &Slips of Rs 13\\ \hline
\multirow{2}{*}{Y} &0 &Box A\\
&1 &Box B\\\hline
\end{tabular}
\caption{}
\label{tab:Distribution}
\end{table}
See \tabref{tab:Distribution}.
\begin{align}
p_{Y}\brak{k}= \begin{cases} 
      \frac{1}{3} & {k=0} \\
      \frac{2}{3 }& {k=1} 
   \end{cases}
   \\
p_{Y|X}\brak{0|0} = \frac{19}{25}\, 
p_{Y|X}\brak{0|1} = \frac{6}{25}\,
p_{Y|X}\brak{1|0} = \frac{45}{50}\,
p_{Y|X}\brak{1|2} = \frac{5}{50}
\end{align}
The desired probability is the probability that a slip drawn at random is marked other than Rs 1,
\begin{align}
&=1-p_X\brak{0}\\
&= p_X(1) + p_X(2)
\end{align}
Using Bayes theorem,
\begin{align}
&= p_Y\brak{0} \times \pr{Y=0 | X=1} + p_Y\brak{1} \times \pr{Y=1|X=2}\\
&=\frac{1}{3} \times \frac{6}{25} + \frac{2}{3} \times \frac{5}{50}\\
&=\frac{11}{75}
\end{align}

\newpage

%\tableofcontents

\bigskip

\renewcommand{\thefigure}{\theenumi}
\renewcommand{\thetable}{\theenumi}
%\renewcommand{\theequation}{\theenumi}

%\begin{abstract}
%%\boldmath
%In this letter, an algorithm for evaluating the exact analytical bit error rate  (BER)  for the piecewise linear (PL) combiner for  multiple relays is presented. Previous results were available only for upto three relays. The algorithm is unique in the sense that  the actual mathematical expressions, that are prohibitively large, need not be explicitly obtained. The diversity gain due to multiple relays is shown through plots of the analytical BER, well supported by simulations. 
%
%\end{abstract}
% IEEEtran.cls defaults to using nonbold math in the Abstract.
% This preserves the distinction between vectors and scalars. However,
% if the journal you are submitting to favors bold math in the abstract,
% then you can use LaTeX's standard command \boldmath at the very start
% of the abstract to achieve this. Many IEEE journals frown on math
% in the abstract anyway.

% Note that keywords are not normally used for peerreview papers.
%\begin{IEEEkeywords}
%Cooperative diversity, decode and forward, piecewise linear
%\end{IEEEkeywords}



% For peer review papers, you can put extra information on the cover
% page as needed:
% \ifCLASSOPTIONpeerreview
% \begin{center} \bfseries EDICS Category: 3-BBND \end{center}
% \fi
%
% For peerreview papers, this IEEEtran command inserts a page break and
% creates the second title. It will be ignored for other modes.
%\IEEEpeerreviewmaketitle




If the letters of the word ASSASSINATION are arranged at random. Find the Probability that
\begin{enumerate}[label=(\alph*)]
\item Four $S's$ come consecutively in the word
\item Two  $I's$ and two $N's$ come together
\item All $A's$ are not coming together
\item No two $A's$ are coming together
\end{enumerate}
%\begin{table}[H]
	\centering
\begin{tabular}{|c|c|c|}
\hline
Random variable &Value &Definition\\ \hline
\multirow{3}{*}{X} &0 &Slips of Rs 1\\
&1 &Slips of Rs 5\\
&2 &Slips of Rs 13\\ \hline
\multirow{2}{*}{Y} &0 &Box A\\
&1 &Box B\\\hline
\end{tabular}
\caption{}
\label{tab:Distribution}
\end{table}
See \tabref{tab:Distribution}.
\begin{align}
p_{Y}\brak{k}= \begin{cases} 
      \frac{1}{3} & {k=0} \\
      \frac{2}{3 }& {k=1} 
   \end{cases}
   \\
p_{Y|X}\brak{0|0} = \frac{19}{25}\, 
p_{Y|X}\brak{0|1} = \frac{6}{25}\,
p_{Y|X}\brak{1|0} = \frac{45}{50}\,
p_{Y|X}\brak{1|2} = \frac{5}{50}
\end{align}
The desired probability is the probability that a slip drawn at random is marked other than Rs 1,
\begin{align}
&=1-p_X\brak{0}\\
&= p_X(1) + p_X(2)
\end{align}
Using Bayes theorem,
\begin{align}
&= p_Y\brak{0} \times \pr{Y=0 | X=1} + p_Y\brak{1} \times \pr{Y=1|X=2}\\
&=\frac{1}{3} \times \frac{6}{25} + \frac{2}{3} \times \frac{5}{50}\\
&=\frac{11}{75}
\end{align}

\newpage

%\tableofcontents

\bigskip

\renewcommand{\thefigure}{\theenumi}
\renewcommand{\thetable}{\theenumi}
%\renewcommand{\theequation}{\theenumi}

%\begin{abstract}
%%\boldmath
%In this letter, an algorithm for evaluating the exact analytical bit error rate  (BER)  for the piecewise linear (PL) combiner for  multiple relays is presented. Previous results were available only for upto three relays. The algorithm is unique in the sense that  the actual mathematical expressions, that are prohibitively large, need not be explicitly obtained. The diversity gain due to multiple relays is shown through plots of the analytical BER, well supported by simulations. 
%
%\end{abstract}
% IEEEtran.cls defaults to using nonbold math in the Abstract.
% This preserves the distinction between vectors and scalars. However,
% if the journal you are submitting to favors bold math in the abstract,
% then you can use LaTeX's standard command \boldmath at the very start
% of the abstract to achieve this. Many IEEE journals frown on math
% in the abstract anyway.

% Note that keywords are not normally used for peerreview papers.
%\begin{IEEEkeywords}
%Cooperative diversity, decode and forward, piecewise linear
%\end{IEEEkeywords}



% For peer review papers, you can put extra information on the cover
% page as needed:
% \ifCLASSOPTIONpeerreview
% \begin{center} \bfseries EDICS Category: 3-BBND \end{center}
% \fi
%
% For peerreview papers, this IEEEtran command inserts a page break and
% creates the second title. It will be ignored for other modes.
%\IEEEpeerreviewmaketitle




	\item One urn contains two black balls (labelled B1 and B2) and one white ball. A
	second urn contains one black ball and two white balls (labelled W1 and W2).
	Suppose the following experiment is performed. One of the two urns is chosen
	at random. Next a ball is randomly chosen from the urn. Then a second ball is
	chosen at random from the same urn without replacing the first ball.
	
	\begin{enumerate}
	\item What is the probability that two black balls are chosen?
	
	\item What is the probability that two balls of opposite colour are chosen?
	\end{enumerate}
	\solution
	%\begin{align}
    \label{eq:12.13.6.18.1}
	\because	\pr{A|B} &> \pr{A},\
\frac{\pr{AB}}{\pr{B}} > \pr{A}
\\
    \label{eq:12.13.6.18.2}
	\implies \pr{AB} &> \pr{A}\pr{B}
	\\
	\text{or, } \frac{\pr{AB}}{\pr{A}} &=\pr{B|A} > \pr{A}
\end{align}

\end{enumerate}

	\item 
The number lock of a suitcase has 4 wheels each labelled with ten digits i.e. from 0 to 9.The lock opens with a sequence of four digits with no repeats.What is the probability of a person getting the right sequence to open the suitcase.
\\
\solution
		%\begin{enumerate}[label=\thesection.\arabic*,ref=\thesection.\theenumi]
	\item One card is drawn from a well-shuffled deck of 52 cards. Find the probability of getting
\begin{enumerate}
\item A king of red colour 
\item A face card 
\item A red face card
\item The jack of hearts
\item A spade
\item The queen of diamonds

\end{enumerate}
\solution
		%\begin{table}[H]
	\centering
\begin{tabular}{|c|c|c|}
\hline
Random variable &Value &Definition\\ \hline
\multirow{3}{*}{X} &0 &Slips of Rs 1\\
&1 &Slips of Rs 5\\
&2 &Slips of Rs 13\\ \hline
\multirow{2}{*}{Y} &0 &Box A\\
&1 &Box B\\\hline
\end{tabular}
\caption{}
\label{tab:Distribution}
\end{table}
See \tabref{tab:Distribution}.
\begin{align}
p_{Y}\brak{k}= \begin{cases} 
      \frac{1}{3} & {k=0} \\
      \frac{2}{3 }& {k=1} 
   \end{cases}
   \\
p_{Y|X}\brak{0|0} = \frac{19}{25}\, 
p_{Y|X}\brak{0|1} = \frac{6}{25}\,
p_{Y|X}\brak{1|0} = \frac{45}{50}\,
p_{Y|X}\brak{1|2} = \frac{5}{50}
\end{align}
The desired probability is the probability that a slip drawn at random is marked other than Rs 1,
\begin{align}
&=1-p_X\brak{0}\\
&= p_X(1) + p_X(2)
\end{align}
Using Bayes theorem,
\begin{align}
&= p_Y\brak{0} \times \pr{Y=0 | X=1} + p_Y\brak{1} \times \pr{Y=1|X=2}\\
&=\frac{1}{3} \times \frac{6}{25} + \frac{2}{3} \times \frac{5}{50}\\
&=\frac{11}{75}
\end{align}

\newpage

%\tableofcontents

\bigskip

\renewcommand{\thefigure}{\theenumi}
\renewcommand{\thetable}{\theenumi}
%\renewcommand{\theequation}{\theenumi}

%\begin{abstract}
%%\boldmath
%In this letter, an algorithm for evaluating the exact analytical bit error rate  (BER)  for the piecewise linear (PL) combiner for  multiple relays is presented. Previous results were available only for upto three relays. The algorithm is unique in the sense that  the actual mathematical expressions, that are prohibitively large, need not be explicitly obtained. The diversity gain due to multiple relays is shown through plots of the analytical BER, well supported by simulations. 
%
%\end{abstract}
% IEEEtran.cls defaults to using nonbold math in the Abstract.
% This preserves the distinction between vectors and scalars. However,
% if the journal you are submitting to favors bold math in the abstract,
% then you can use LaTeX's standard command \boldmath at the very start
% of the abstract to achieve this. Many IEEE journals frown on math
% in the abstract anyway.

% Note that keywords are not normally used for peerreview papers.
%\begin{IEEEkeywords}
%Cooperative diversity, decode and forward, piecewise linear
%\end{IEEEkeywords}



% For peer review papers, you can put extra information on the cover
% page as needed:
% \ifCLASSOPTIONpeerreview
% \begin{center} \bfseries EDICS Category: 3-BBND \end{center}
% \fi
%
% For peerreview papers, this IEEEtran command inserts a page break and
% creates the second title. It will be ignored for other modes.
%\IEEEpeerreviewmaketitle




	\item Five cards—the ten, jack, queen, king and ace of diamonds, are well-shuffled with their face downwards. One card is then picked up at random.
\begin{enumerate}
\item
What is the probability that the card is the queen? 
\item
If the queen is drawn and put aside, what is the probability that the second card picked up is (a) an ace? (b) a queen?\\
\end{enumerate}
\solution
		%\begin{enumerate}[label=\thesection.\arabic*,ref=\thesection.\theenumi]
	\item One card is drawn from a well-shuffled deck of 52 cards. Find the probability of getting
\begin{enumerate}
\item A king of red colour 
\item A face card 
\item A red face card
\item The jack of hearts
\item A spade
\item The queen of diamonds

\end{enumerate}
\solution
		%\input{ncert/10/15/1/14/main.tex}
	\item Five cards—the ten, jack, queen, king and ace of diamonds, are well-shuffled with their face downwards. One card is then picked up at random.
\begin{enumerate}
\item
What is the probability that the card is the queen? 
\item
If the queen is drawn and put aside, what is the probability that the second card picked up is (a) an ace? (b) a queen?\\
\end{enumerate}
\solution
		%\input{ncert/10/15/1/15/defs.tex}
	\item A bag contains $5$ red balls and some blue balls. If the probability of drawing a blue ball is double that if a red ball, determine the number of blue balls in the bag. 
		\\
\solution
		%\input{ncert/10/15/2/3/defs.tex}
	\item A card is selected from a pack of 52 cards.
 \begin{enumerate}[label=(\alph*)] 
                 \item How many points are there in the sample space?
                 \item Calculate the probability that the card is an ace of spades.
                 \item Calculate the probability that the card is (i) an ace and (ii) black card.
 \end{enumerate}
\solution
		%\input{ncert/11/16/3/4/main.tex}
\item Four cards are drawn from a well-shuffled deck of 52 cards. What is the probability of obtaining 3 diamonds and one spade.
\\
\solution
		%\input{ncert/11/16/4/2/defs.tex}
\item In a certain lottery 10,000 tickets are sold and ten equal prizes are awarded. What is the probability of not getting a prize if you buy (a) one ticket (b) two tickets (c) 10 tickets ?	
\\
\solution
		%\input{ncert/11/16/4/4/defs.tex}
		%
\item 
Out of 100 students, two sections of 40 and 60 are formed. If you and your friend are among the 100 students, what is the probability that
\begin{enumerate}
\item you both enter the same section?
\item you both enter the different sections?
\end{enumerate}
\solution
		%\input{ncert/11/16/4/5/defs.tex}
	\item 
The number lock of a suitcase has 4 wheels each labelled with ten digits i.e. from 0 to 9.The lock opens with a sequence of four digits with no repeats.What is the probability of a person getting the right sequence to open the suitcase.
\\
\solution
		%\input{ncert/11/16/4/10/defs.tex}
		%
\item 
Two cards are drawn at random and without replacement from a pack of 52 playing cards. Find the probability that both the cards are black.
\\
\solution
		%\input{ncert/12/13/2/2/defs.tex}
		\item A box of oranges is inspected by examining three randomly selected oranges drawn without replacement. If all the three oranges are good, the box is approved for sale, otherwise, it is rejected. Find the probability that a box containing 15 oranges out of which 12 are good and 3 are bad ones will be approved for sale.
		\label{ncert/12/13/2/3/defs.tex}
		\item Two balls are drawn at random with replacement from a box containing 10 black and 8 red balls. Find the probability that
		\label{ncert/12/13/2/12}
\begin{enumerate}
\item both balls are red.
\item first ball is black and second is red.
\item one of them is black and other is red.
\end{enumerate}

\item In a hostel, 60\% of the students read Hindi newspaper, 40\% read English newspaper and 20\% read both Hindi and English newspapers. A student is selected at random.
		\label{ncert/12/13/2/15}
\begin{enumerate}
\item Find the probability that she reads neither Hindi nor English newspapers.
\item If she reads Hindi newspaper, find the probability that she reads English newspaper.
\item If she reads English newspaper, find the probability that she reads Hindi newspaper.\\
\end{enumerate}
\item The probability of obtaining an even prime number on each die, when a pair of dice is rolled is 
\begin{enumerate}
    \item $0$ 
    
    \item $\frac{1}{3}$ 
    
    \item $\frac{1}{12}$ 
    
    \item $\frac{1}{36}$ 
\end{enumerate}
\solution
		%\input{ncert/12/13/2/17/defs.tex}
	\item A bag contains 4 red and 4 black balls, another bag contains 2 red and 6 black balls. One of the two bags is selected at random and a ball is drawn from the bag which is found to be red. Find the probability that the ball is drawn from the first bag.
\\
\solution
		%\input{ncert/12/13/3/2/main.tex}
  \item
  Cards with numbers 2 to 101 are placed in a box. A card is selected at random.Find the probability that the card has
\begin{enumerate}[label=(\roman*)]
	\item an even number 
	\item a square number
\end{enumerate}
\solution
%\input{exemplar/10/13/3/32/main.tex}
\item
The king, queen and jack of clubs are removed from a deck of 52 playing cards and then well shuffled. Now one card is drawn at random from the remaining cards.  Determine the probability that the card is
\begin{enumerate}[label=(\roman*)]
\item a club
\item 10 of hearts
\end{enumerate}
\solution
%\input{exemplar/10/13/3/29/main.tex}
\item A team of medical students doing their internship have to assist during surgeries
at a city hospital. The probabilities of surgeries rated as very complex, complex,
routine, simple or very simple are respectively, 0.15, 0.20, 0.31, 0.26, .08. Find
the probabilities that a particular surgery will be rated
\begin{enumerate}
	\item complex or very complex;
	\item neither very complex nor very simple;
	\item routine or complex
	\item routine or simple
\end{enumerate}
\solution
%\input{exemplar/11/16/3/8(1)/main.tex}
\item A card is selected from a pack of 52 cards.
\begin{enumerate}[label=(\alph*)]
    \item How many points are there in the sample space?
    \item Calculate the probability that the card is an ace of spades.
    \item Calculate the probability that the card is (i) an ace and (ii) black card.
\end{enumerate}
\solution
%\input{exemplar/11/16/3/4/main2.tex}
\item The probability that a non leap year selected at random will contain 53 sundays.
\\
\solution
%\input{exemplar/10/13/1/19/main.tex}
\item One of the four persons John, Rita, Aslam or Gurpreet will be promoted next
month. Consequently the sample space consists of four elementary outcomes
S = {John promoted, Rita promoted, Aslam promoted, Gurpreet promoted}
You are told that the chances of John’s promotion is same as that of Gurpreet,
Rita’s chances of promotion are twice as likely as Johns. Aslam’s chances are
four times that of John.
\begin{enumerate}
	\item Determine
	\begin{enumerate}
		\item P (John promoted)
		\item P (Rita promoted)
		\item P (Aslam promoted)
		\item P (Gurpreet promoted)
	\end{enumerate}
	\item If A = {John promoted or Gurpreet promoted}, find P (A).
\end{enumerate}
\solution
%\input{exemplar/11/16/3/10/main.tex}
\item A card is drawn from a deck of 52 cards. Find the probability of getting a king or a heart or a red card.\\
\solution
%\input{exemplar/11/16/3/15/main.tex}
\item The probability that a student will pass his examination is 0.73, the probability of
the student getting a compartment is 0.13, and the probability that the student will
either pass or get compartment is 0.96. State True or False.\\
\solution
%\input{exemplar/11/16/3/31/main.tex}
\item A card is selected from a pack of 52 cards\\
\begin{enumerate}[label=(\alph*)]
\item How many points are there in the sample space?
\item Calculate the probability that the cards is an ace of spades.
\item Calculate the probability that the card is (i) an ace (ii)black card.\\
\end{enumerate}
%\input{ncert/11/16/3/4_1/Prob_4.tex}
\item In a non-leap year, the probability of having 53 tuesdays or 53 wednesdays is\\
\solution
%\input{exemplar/11/16/3/18/main.tex}
\item There are 1000 sealed envelopes in a box, 10 of them contain a cash prize of
Rs 100 each, 100 of them contain a cash prize of Rs 50 each and 200 of them
contain a cash prize of Rs 10 each and rest do not contain any cash prize. If they
are well shuffled and an envelope is picked up out, what is the probability that it
contains no cash prize?\\
\solution
%\input{exemplar/10/13/3/34/main.tex}
\item 
A die is thrown and a card is selected at random from a deck of 52 playing cards. The probability of getting an even number on the die and a spade card.\\
\solution
%\input{exemplar/12/13/3/78/main.tex}
\item
If 4-digit numbers greater than 5,000 are randomly formed from the digits 0, 1, 3, 5, and 7, what is the probability of forming a number divisible by 5 when:
\begin{enumerate}
    \item The digits are repeated?
    \item The repetition of digits is not allowed?
\end{enumerate}
\solution
%\input{ncert/11/16/4/9/main.tex}
\item Consider the probability space $\brak{\Omega, \mathcal{G}, P}$ where $\Omega = [0,2]$ and $\mathcal{G} = \cbrak{\phi, \Omega, [0,1], (1,2]}$. Let $X$ and $Y$ be two functions on $\Omega$ defined as
\begin{align*}
    X(\omega) = 
    \begin{cases}
        1 & \text{if }\omega \in [0, 1]\\
        2 & \text{if }\omega \in (1, 2]
    \end{cases}
\end{align*}
and
\begin{align*}
    Y(\omega) = 
    \begin{cases}
        2 & \text{if }\omega \in [0, 1.5]\\
        3 & \text{if }\omega \in (1.5, 2].
    \end{cases}
\end{align*}
Then which one of the following statements is true?
\begin{enumerate}
    \item [(A)] $X$ is a random variable with respect to $\mathcal{G}$, but $Y$ is not a random variable with respect to $\mathcal{G}$.
    \item [(B)] $Y$ is a random variable with respect to $\mathcal{G}$, but $X$ is not a random variable with respect to $\mathcal{G}$.
    \item [(C)] Neither $X$ nor $Y$ is a random variable with respect to $\mathcal{G}$.
    \item [(D)] Both $X$ and $Y$ are random variables with respect to $\mathcal{G}$.
\end{enumerate} \hfill (GATE ST 2023)\\
\solution
%\input{gate/ST/2023/14/main.tex}
	\item  A die is loaded in such a way that each odd number is twice as likely to occur as
each even number. Find $P(G)$, where $G$ is the event that a number greater than
3 occurs on a single roll of the die.
\\
\solution
		%\input{exemplar/11/16/3/5/main.tex}
	\item All the jacks, queens and kings are removed from a deck of 52 playing cards. The remaining cards are well shuffled and then one card is drawn at random. Giving ace a value 1 similar value for other cards, find the probability that the card has a value 
		\begin{enumerate}
			\item 7
			\item greater than 7
			\item less than 7
		\end{enumerate}
		%\input{exemplar/10/13/3/30/main.tex}
  \item A Lot consists of 48 mobile phones of which 42 are good, 3 have only minor defects and 3 have major defects.Varnika will buy a phone if it is good but the trader will only buy a mobile if it has no major defects. One phone is selected at random from the lot. What is the probability that it is
\begin{enumerate}
	\item acceptable to Varnika?
            \item acceptable to the trader?
\end{enumerate}
\solution
	%\input{exemplar/10/13/3/40/main.tex}
 \item A student says that if you throw a die, it will show up 1 or not 1. Therefore, the probability of getting 1 and the probability of getting 'not 1' each is equal to $\frac{1}{2}$. Is this correct? Give reasons.\\
 \solution
        %\input{exemplar/10/13/2/9/main.tex}
   \item Four candidates A, B, C, D have ap-
plied for the assignment to coach a school cricket
team. If A is twice as likely to be selected as B, and
B and C are given about the same chance of being
selected, while C is twice as likely to be selected
as D, what are the probabilities that
\begin{enumerate}
\item C will be selected?
\item A will not be selected?
\end{enumerate}
	%\input{exemplar/11/16/3/9/main.tex}
 \item A bag contain 24 balls of which $x$ balls are red, $2x$ are white and $3x$ are blue. A ball is selected at random, What is the probability that it is
\begin{enumerate}[label=\alph*)]
\item not red ?
\item white ?
\end{enumerate}
%\input{exemplar/10/13/3/41/main.tex}
If the letters of the word ASSASSINATION are arranged at random. Find the Probability that
\begin{enumerate}[label=(\alph*)]
\item Four $S's$ come consecutively in the word
\item Two  $I's$ and two $N's$ come together
\item All $A's$ are not coming together
\item No two $A's$ are coming together
\end{enumerate}
%\input{exemplar/11/16/3/14/main.tex}
	\item One urn contains two black balls (labelled B1 and B2) and one white ball. A
	second urn contains one black ball and two white balls (labelled W1 and W2).
	Suppose the following experiment is performed. One of the two urns is chosen
	at random. Next a ball is randomly chosen from the urn. Then a second ball is
	chosen at random from the same urn without replacing the first ball.
	
	\begin{enumerate}
	\item What is the probability that two black balls are chosen?
	
	\item What is the probability that two balls of opposite colour are chosen?
	\end{enumerate}
	\solution
	%\input{exemplar/11/16/3/12/main1.tex}
\end{enumerate}

	\item A bag contains $5$ red balls and some blue balls. If the probability of drawing a blue ball is double that if a red ball, determine the number of blue balls in the bag. 
		\\
\solution
		%\begin{enumerate}[label=\thesection.\arabic*,ref=\thesection.\theenumi]
	\item One card is drawn from a well-shuffled deck of 52 cards. Find the probability of getting
\begin{enumerate}
\item A king of red colour 
\item A face card 
\item A red face card
\item The jack of hearts
\item A spade
\item The queen of diamonds

\end{enumerate}
\solution
		%\input{ncert/10/15/1/14/main.tex}
	\item Five cards—the ten, jack, queen, king and ace of diamonds, are well-shuffled with their face downwards. One card is then picked up at random.
\begin{enumerate}
\item
What is the probability that the card is the queen? 
\item
If the queen is drawn and put aside, what is the probability that the second card picked up is (a) an ace? (b) a queen?\\
\end{enumerate}
\solution
		%\input{ncert/10/15/1/15/defs.tex}
	\item A bag contains $5$ red balls and some blue balls. If the probability of drawing a blue ball is double that if a red ball, determine the number of blue balls in the bag. 
		\\
\solution
		%\input{ncert/10/15/2/3/defs.tex}
	\item A card is selected from a pack of 52 cards.
 \begin{enumerate}[label=(\alph*)] 
                 \item How many points are there in the sample space?
                 \item Calculate the probability that the card is an ace of spades.
                 \item Calculate the probability that the card is (i) an ace and (ii) black card.
 \end{enumerate}
\solution
		%\input{ncert/11/16/3/4/main.tex}
\item Four cards are drawn from a well-shuffled deck of 52 cards. What is the probability of obtaining 3 diamonds and one spade.
\\
\solution
		%\input{ncert/11/16/4/2/defs.tex}
\item In a certain lottery 10,000 tickets are sold and ten equal prizes are awarded. What is the probability of not getting a prize if you buy (a) one ticket (b) two tickets (c) 10 tickets ?	
\\
\solution
		%\input{ncert/11/16/4/4/defs.tex}
		%
\item 
Out of 100 students, two sections of 40 and 60 are formed. If you and your friend are among the 100 students, what is the probability that
\begin{enumerate}
\item you both enter the same section?
\item you both enter the different sections?
\end{enumerate}
\solution
		%\input{ncert/11/16/4/5/defs.tex}
	\item 
The number lock of a suitcase has 4 wheels each labelled with ten digits i.e. from 0 to 9.The lock opens with a sequence of four digits with no repeats.What is the probability of a person getting the right sequence to open the suitcase.
\\
\solution
		%\input{ncert/11/16/4/10/defs.tex}
		%
\item 
Two cards are drawn at random and without replacement from a pack of 52 playing cards. Find the probability that both the cards are black.
\\
\solution
		%\input{ncert/12/13/2/2/defs.tex}
		\item A box of oranges is inspected by examining three randomly selected oranges drawn without replacement. If all the three oranges are good, the box is approved for sale, otherwise, it is rejected. Find the probability that a box containing 15 oranges out of which 12 are good and 3 are bad ones will be approved for sale.
		\label{ncert/12/13/2/3/defs.tex}
		\item Two balls are drawn at random with replacement from a box containing 10 black and 8 red balls. Find the probability that
		\label{ncert/12/13/2/12}
\begin{enumerate}
\item both balls are red.
\item first ball is black and second is red.
\item one of them is black and other is red.
\end{enumerate}

\item In a hostel, 60\% of the students read Hindi newspaper, 40\% read English newspaper and 20\% read both Hindi and English newspapers. A student is selected at random.
		\label{ncert/12/13/2/15}
\begin{enumerate}
\item Find the probability that she reads neither Hindi nor English newspapers.
\item If she reads Hindi newspaper, find the probability that she reads English newspaper.
\item If she reads English newspaper, find the probability that she reads Hindi newspaper.\\
\end{enumerate}
\item The probability of obtaining an even prime number on each die, when a pair of dice is rolled is 
\begin{enumerate}
    \item $0$ 
    
    \item $\frac{1}{3}$ 
    
    \item $\frac{1}{12}$ 
    
    \item $\frac{1}{36}$ 
\end{enumerate}
\solution
		%\input{ncert/12/13/2/17/defs.tex}
	\item A bag contains 4 red and 4 black balls, another bag contains 2 red and 6 black balls. One of the two bags is selected at random and a ball is drawn from the bag which is found to be red. Find the probability that the ball is drawn from the first bag.
\\
\solution
		%\input{ncert/12/13/3/2/main.tex}
  \item
  Cards with numbers 2 to 101 are placed in a box. A card is selected at random.Find the probability that the card has
\begin{enumerate}[label=(\roman*)]
	\item an even number 
	\item a square number
\end{enumerate}
\solution
%\input{exemplar/10/13/3/32/main.tex}
\item
The king, queen and jack of clubs are removed from a deck of 52 playing cards and then well shuffled. Now one card is drawn at random from the remaining cards.  Determine the probability that the card is
\begin{enumerate}[label=(\roman*)]
\item a club
\item 10 of hearts
\end{enumerate}
\solution
%\input{exemplar/10/13/3/29/main.tex}
\item A team of medical students doing their internship have to assist during surgeries
at a city hospital. The probabilities of surgeries rated as very complex, complex,
routine, simple or very simple are respectively, 0.15, 0.20, 0.31, 0.26, .08. Find
the probabilities that a particular surgery will be rated
\begin{enumerate}
	\item complex or very complex;
	\item neither very complex nor very simple;
	\item routine or complex
	\item routine or simple
\end{enumerate}
\solution
%\input{exemplar/11/16/3/8(1)/main.tex}
\item A card is selected from a pack of 52 cards.
\begin{enumerate}[label=(\alph*)]
    \item How many points are there in the sample space?
    \item Calculate the probability that the card is an ace of spades.
    \item Calculate the probability that the card is (i) an ace and (ii) black card.
\end{enumerate}
\solution
%\input{exemplar/11/16/3/4/main2.tex}
\item The probability that a non leap year selected at random will contain 53 sundays.
\\
\solution
%\input{exemplar/10/13/1/19/main.tex}
\item One of the four persons John, Rita, Aslam or Gurpreet will be promoted next
month. Consequently the sample space consists of four elementary outcomes
S = {John promoted, Rita promoted, Aslam promoted, Gurpreet promoted}
You are told that the chances of John’s promotion is same as that of Gurpreet,
Rita’s chances of promotion are twice as likely as Johns. Aslam’s chances are
four times that of John.
\begin{enumerate}
	\item Determine
	\begin{enumerate}
		\item P (John promoted)
		\item P (Rita promoted)
		\item P (Aslam promoted)
		\item P (Gurpreet promoted)
	\end{enumerate}
	\item If A = {John promoted or Gurpreet promoted}, find P (A).
\end{enumerate}
\solution
%\input{exemplar/11/16/3/10/main.tex}
\item A card is drawn from a deck of 52 cards. Find the probability of getting a king or a heart or a red card.\\
\solution
%\input{exemplar/11/16/3/15/main.tex}
\item The probability that a student will pass his examination is 0.73, the probability of
the student getting a compartment is 0.13, and the probability that the student will
either pass or get compartment is 0.96. State True or False.\\
\solution
%\input{exemplar/11/16/3/31/main.tex}
\item A card is selected from a pack of 52 cards\\
\begin{enumerate}[label=(\alph*)]
\item How many points are there in the sample space?
\item Calculate the probability that the cards is an ace of spades.
\item Calculate the probability that the card is (i) an ace (ii)black card.\\
\end{enumerate}
%\input{ncert/11/16/3/4_1/Prob_4.tex}
\item In a non-leap year, the probability of having 53 tuesdays or 53 wednesdays is\\
\solution
%\input{exemplar/11/16/3/18/main.tex}
\item There are 1000 sealed envelopes in a box, 10 of them contain a cash prize of
Rs 100 each, 100 of them contain a cash prize of Rs 50 each and 200 of them
contain a cash prize of Rs 10 each and rest do not contain any cash prize. If they
are well shuffled and an envelope is picked up out, what is the probability that it
contains no cash prize?\\
\solution
%\input{exemplar/10/13/3/34/main.tex}
\item 
A die is thrown and a card is selected at random from a deck of 52 playing cards. The probability of getting an even number on the die and a spade card.\\
\solution
%\input{exemplar/12/13/3/78/main.tex}
\item
If 4-digit numbers greater than 5,000 are randomly formed from the digits 0, 1, 3, 5, and 7, what is the probability of forming a number divisible by 5 when:
\begin{enumerate}
    \item The digits are repeated?
    \item The repetition of digits is not allowed?
\end{enumerate}
\solution
%\input{ncert/11/16/4/9/main.tex}
\item Consider the probability space $\brak{\Omega, \mathcal{G}, P}$ where $\Omega = [0,2]$ and $\mathcal{G} = \cbrak{\phi, \Omega, [0,1], (1,2]}$. Let $X$ and $Y$ be two functions on $\Omega$ defined as
\begin{align*}
    X(\omega) = 
    \begin{cases}
        1 & \text{if }\omega \in [0, 1]\\
        2 & \text{if }\omega \in (1, 2]
    \end{cases}
\end{align*}
and
\begin{align*}
    Y(\omega) = 
    \begin{cases}
        2 & \text{if }\omega \in [0, 1.5]\\
        3 & \text{if }\omega \in (1.5, 2].
    \end{cases}
\end{align*}
Then which one of the following statements is true?
\begin{enumerate}
    \item [(A)] $X$ is a random variable with respect to $\mathcal{G}$, but $Y$ is not a random variable with respect to $\mathcal{G}$.
    \item [(B)] $Y$ is a random variable with respect to $\mathcal{G}$, but $X$ is not a random variable with respect to $\mathcal{G}$.
    \item [(C)] Neither $X$ nor $Y$ is a random variable with respect to $\mathcal{G}$.
    \item [(D)] Both $X$ and $Y$ are random variables with respect to $\mathcal{G}$.
\end{enumerate} \hfill (GATE ST 2023)\\
\solution
%\input{gate/ST/2023/14/main.tex}
	\item  A die is loaded in such a way that each odd number is twice as likely to occur as
each even number. Find $P(G)$, where $G$ is the event that a number greater than
3 occurs on a single roll of the die.
\\
\solution
		%\input{exemplar/11/16/3/5/main.tex}
	\item All the jacks, queens and kings are removed from a deck of 52 playing cards. The remaining cards are well shuffled and then one card is drawn at random. Giving ace a value 1 similar value for other cards, find the probability that the card has a value 
		\begin{enumerate}
			\item 7
			\item greater than 7
			\item less than 7
		\end{enumerate}
		%\input{exemplar/10/13/3/30/main.tex}
  \item A Lot consists of 48 mobile phones of which 42 are good, 3 have only minor defects and 3 have major defects.Varnika will buy a phone if it is good but the trader will only buy a mobile if it has no major defects. One phone is selected at random from the lot. What is the probability that it is
\begin{enumerate}
	\item acceptable to Varnika?
            \item acceptable to the trader?
\end{enumerate}
\solution
	%\input{exemplar/10/13/3/40/main.tex}
 \item A student says that if you throw a die, it will show up 1 or not 1. Therefore, the probability of getting 1 and the probability of getting 'not 1' each is equal to $\frac{1}{2}$. Is this correct? Give reasons.\\
 \solution
        %\input{exemplar/10/13/2/9/main.tex}
   \item Four candidates A, B, C, D have ap-
plied for the assignment to coach a school cricket
team. If A is twice as likely to be selected as B, and
B and C are given about the same chance of being
selected, while C is twice as likely to be selected
as D, what are the probabilities that
\begin{enumerate}
\item C will be selected?
\item A will not be selected?
\end{enumerate}
	%\input{exemplar/11/16/3/9/main.tex}
 \item A bag contain 24 balls of which $x$ balls are red, $2x$ are white and $3x$ are blue. A ball is selected at random, What is the probability that it is
\begin{enumerate}[label=\alph*)]
\item not red ?
\item white ?
\end{enumerate}
%\input{exemplar/10/13/3/41/main.tex}
If the letters of the word ASSASSINATION are arranged at random. Find the Probability that
\begin{enumerate}[label=(\alph*)]
\item Four $S's$ come consecutively in the word
\item Two  $I's$ and two $N's$ come together
\item All $A's$ are not coming together
\item No two $A's$ are coming together
\end{enumerate}
%\input{exemplar/11/16/3/14/main.tex}
	\item One urn contains two black balls (labelled B1 and B2) and one white ball. A
	second urn contains one black ball and two white balls (labelled W1 and W2).
	Suppose the following experiment is performed. One of the two urns is chosen
	at random. Next a ball is randomly chosen from the urn. Then a second ball is
	chosen at random from the same urn without replacing the first ball.
	
	\begin{enumerate}
	\item What is the probability that two black balls are chosen?
	
	\item What is the probability that two balls of opposite colour are chosen?
	\end{enumerate}
	\solution
	%\input{exemplar/11/16/3/12/main1.tex}
\end{enumerate}

	\item A card is selected from a pack of 52 cards.
 \begin{enumerate}[label=(\alph*)] 
                 \item How many points are there in the sample space?
                 \item Calculate the probability that the card is an ace of spades.
                 \item Calculate the probability that the card is (i) an ace and (ii) black card.
 \end{enumerate}
\solution
		%\begin{table}[H]
	\centering
\begin{tabular}{|c|c|c|}
\hline
Random variable &Value &Definition\\ \hline
\multirow{3}{*}{X} &0 &Slips of Rs 1\\
&1 &Slips of Rs 5\\
&2 &Slips of Rs 13\\ \hline
\multirow{2}{*}{Y} &0 &Box A\\
&1 &Box B\\\hline
\end{tabular}
\caption{}
\label{tab:Distribution}
\end{table}
See \tabref{tab:Distribution}.
\begin{align}
p_{Y}\brak{k}= \begin{cases} 
      \frac{1}{3} & {k=0} \\
      \frac{2}{3 }& {k=1} 
   \end{cases}
   \\
p_{Y|X}\brak{0|0} = \frac{19}{25}\, 
p_{Y|X}\brak{0|1} = \frac{6}{25}\,
p_{Y|X}\brak{1|0} = \frac{45}{50}\,
p_{Y|X}\brak{1|2} = \frac{5}{50}
\end{align}
The desired probability is the probability that a slip drawn at random is marked other than Rs 1,
\begin{align}
&=1-p_X\brak{0}\\
&= p_X(1) + p_X(2)
\end{align}
Using Bayes theorem,
\begin{align}
&= p_Y\brak{0} \times \pr{Y=0 | X=1} + p_Y\brak{1} \times \pr{Y=1|X=2}\\
&=\frac{1}{3} \times \frac{6}{25} + \frac{2}{3} \times \frac{5}{50}\\
&=\frac{11}{75}
\end{align}

\newpage

%\tableofcontents

\bigskip

\renewcommand{\thefigure}{\theenumi}
\renewcommand{\thetable}{\theenumi}
%\renewcommand{\theequation}{\theenumi}

%\begin{abstract}
%%\boldmath
%In this letter, an algorithm for evaluating the exact analytical bit error rate  (BER)  for the piecewise linear (PL) combiner for  multiple relays is presented. Previous results were available only for upto three relays. The algorithm is unique in the sense that  the actual mathematical expressions, that are prohibitively large, need not be explicitly obtained. The diversity gain due to multiple relays is shown through plots of the analytical BER, well supported by simulations. 
%
%\end{abstract}
% IEEEtran.cls defaults to using nonbold math in the Abstract.
% This preserves the distinction between vectors and scalars. However,
% if the journal you are submitting to favors bold math in the abstract,
% then you can use LaTeX's standard command \boldmath at the very start
% of the abstract to achieve this. Many IEEE journals frown on math
% in the abstract anyway.

% Note that keywords are not normally used for peerreview papers.
%\begin{IEEEkeywords}
%Cooperative diversity, decode and forward, piecewise linear
%\end{IEEEkeywords}



% For peer review papers, you can put extra information on the cover
% page as needed:
% \ifCLASSOPTIONpeerreview
% \begin{center} \bfseries EDICS Category: 3-BBND \end{center}
% \fi
%
% For peerreview papers, this IEEEtran command inserts a page break and
% creates the second title. It will be ignored for other modes.
%\IEEEpeerreviewmaketitle




\item Four cards are drawn from a well-shuffled deck of 52 cards. What is the probability of obtaining 3 diamonds and one spade.
\\
\solution
		%\begin{enumerate}[label=\thesection.\arabic*,ref=\thesection.\theenumi]
	\item One card is drawn from a well-shuffled deck of 52 cards. Find the probability of getting
\begin{enumerate}
\item A king of red colour 
\item A face card 
\item A red face card
\item The jack of hearts
\item A spade
\item The queen of diamonds

\end{enumerate}
\solution
		%\input{ncert/10/15/1/14/main.tex}
	\item Five cards—the ten, jack, queen, king and ace of diamonds, are well-shuffled with their face downwards. One card is then picked up at random.
\begin{enumerate}
\item
What is the probability that the card is the queen? 
\item
If the queen is drawn and put aside, what is the probability that the second card picked up is (a) an ace? (b) a queen?\\
\end{enumerate}
\solution
		%\input{ncert/10/15/1/15/defs.tex}
	\item A bag contains $5$ red balls and some blue balls. If the probability of drawing a blue ball is double that if a red ball, determine the number of blue balls in the bag. 
		\\
\solution
		%\input{ncert/10/15/2/3/defs.tex}
	\item A card is selected from a pack of 52 cards.
 \begin{enumerate}[label=(\alph*)] 
                 \item How many points are there in the sample space?
                 \item Calculate the probability that the card is an ace of spades.
                 \item Calculate the probability that the card is (i) an ace and (ii) black card.
 \end{enumerate}
\solution
		%\input{ncert/11/16/3/4/main.tex}
\item Four cards are drawn from a well-shuffled deck of 52 cards. What is the probability of obtaining 3 diamonds and one spade.
\\
\solution
		%\input{ncert/11/16/4/2/defs.tex}
\item In a certain lottery 10,000 tickets are sold and ten equal prizes are awarded. What is the probability of not getting a prize if you buy (a) one ticket (b) two tickets (c) 10 tickets ?	
\\
\solution
		%\input{ncert/11/16/4/4/defs.tex}
		%
\item 
Out of 100 students, two sections of 40 and 60 are formed. If you and your friend are among the 100 students, what is the probability that
\begin{enumerate}
\item you both enter the same section?
\item you both enter the different sections?
\end{enumerate}
\solution
		%\input{ncert/11/16/4/5/defs.tex}
	\item 
The number lock of a suitcase has 4 wheels each labelled with ten digits i.e. from 0 to 9.The lock opens with a sequence of four digits with no repeats.What is the probability of a person getting the right sequence to open the suitcase.
\\
\solution
		%\input{ncert/11/16/4/10/defs.tex}
		%
\item 
Two cards are drawn at random and without replacement from a pack of 52 playing cards. Find the probability that both the cards are black.
\\
\solution
		%\input{ncert/12/13/2/2/defs.tex}
		\item A box of oranges is inspected by examining three randomly selected oranges drawn without replacement. If all the three oranges are good, the box is approved for sale, otherwise, it is rejected. Find the probability that a box containing 15 oranges out of which 12 are good and 3 are bad ones will be approved for sale.
		\label{ncert/12/13/2/3/defs.tex}
		\item Two balls are drawn at random with replacement from a box containing 10 black and 8 red balls. Find the probability that
		\label{ncert/12/13/2/12}
\begin{enumerate}
\item both balls are red.
\item first ball is black and second is red.
\item one of them is black and other is red.
\end{enumerate}

\item In a hostel, 60\% of the students read Hindi newspaper, 40\% read English newspaper and 20\% read both Hindi and English newspapers. A student is selected at random.
		\label{ncert/12/13/2/15}
\begin{enumerate}
\item Find the probability that she reads neither Hindi nor English newspapers.
\item If she reads Hindi newspaper, find the probability that she reads English newspaper.
\item If she reads English newspaper, find the probability that she reads Hindi newspaper.\\
\end{enumerate}
\item The probability of obtaining an even prime number on each die, when a pair of dice is rolled is 
\begin{enumerate}
    \item $0$ 
    
    \item $\frac{1}{3}$ 
    
    \item $\frac{1}{12}$ 
    
    \item $\frac{1}{36}$ 
\end{enumerate}
\solution
		%\input{ncert/12/13/2/17/defs.tex}
	\item A bag contains 4 red and 4 black balls, another bag contains 2 red and 6 black balls. One of the two bags is selected at random and a ball is drawn from the bag which is found to be red. Find the probability that the ball is drawn from the first bag.
\\
\solution
		%\input{ncert/12/13/3/2/main.tex}
  \item
  Cards with numbers 2 to 101 are placed in a box. A card is selected at random.Find the probability that the card has
\begin{enumerate}[label=(\roman*)]
	\item an even number 
	\item a square number
\end{enumerate}
\solution
%\input{exemplar/10/13/3/32/main.tex}
\item
The king, queen and jack of clubs are removed from a deck of 52 playing cards and then well shuffled. Now one card is drawn at random from the remaining cards.  Determine the probability that the card is
\begin{enumerate}[label=(\roman*)]
\item a club
\item 10 of hearts
\end{enumerate}
\solution
%\input{exemplar/10/13/3/29/main.tex}
\item A team of medical students doing their internship have to assist during surgeries
at a city hospital. The probabilities of surgeries rated as very complex, complex,
routine, simple or very simple are respectively, 0.15, 0.20, 0.31, 0.26, .08. Find
the probabilities that a particular surgery will be rated
\begin{enumerate}
	\item complex or very complex;
	\item neither very complex nor very simple;
	\item routine or complex
	\item routine or simple
\end{enumerate}
\solution
%\input{exemplar/11/16/3/8(1)/main.tex}
\item A card is selected from a pack of 52 cards.
\begin{enumerate}[label=(\alph*)]
    \item How many points are there in the sample space?
    \item Calculate the probability that the card is an ace of spades.
    \item Calculate the probability that the card is (i) an ace and (ii) black card.
\end{enumerate}
\solution
%\input{exemplar/11/16/3/4/main2.tex}
\item The probability that a non leap year selected at random will contain 53 sundays.
\\
\solution
%\input{exemplar/10/13/1/19/main.tex}
\item One of the four persons John, Rita, Aslam or Gurpreet will be promoted next
month. Consequently the sample space consists of four elementary outcomes
S = {John promoted, Rita promoted, Aslam promoted, Gurpreet promoted}
You are told that the chances of John’s promotion is same as that of Gurpreet,
Rita’s chances of promotion are twice as likely as Johns. Aslam’s chances are
four times that of John.
\begin{enumerate}
	\item Determine
	\begin{enumerate}
		\item P (John promoted)
		\item P (Rita promoted)
		\item P (Aslam promoted)
		\item P (Gurpreet promoted)
	\end{enumerate}
	\item If A = {John promoted or Gurpreet promoted}, find P (A).
\end{enumerate}
\solution
%\input{exemplar/11/16/3/10/main.tex}
\item A card is drawn from a deck of 52 cards. Find the probability of getting a king or a heart or a red card.\\
\solution
%\input{exemplar/11/16/3/15/main.tex}
\item The probability that a student will pass his examination is 0.73, the probability of
the student getting a compartment is 0.13, and the probability that the student will
either pass or get compartment is 0.96. State True or False.\\
\solution
%\input{exemplar/11/16/3/31/main.tex}
\item A card is selected from a pack of 52 cards\\
\begin{enumerate}[label=(\alph*)]
\item How many points are there in the sample space?
\item Calculate the probability that the cards is an ace of spades.
\item Calculate the probability that the card is (i) an ace (ii)black card.\\
\end{enumerate}
%\input{ncert/11/16/3/4_1/Prob_4.tex}
\item In a non-leap year, the probability of having 53 tuesdays or 53 wednesdays is\\
\solution
%\input{exemplar/11/16/3/18/main.tex}
\item There are 1000 sealed envelopes in a box, 10 of them contain a cash prize of
Rs 100 each, 100 of them contain a cash prize of Rs 50 each and 200 of them
contain a cash prize of Rs 10 each and rest do not contain any cash prize. If they
are well shuffled and an envelope is picked up out, what is the probability that it
contains no cash prize?\\
\solution
%\input{exemplar/10/13/3/34/main.tex}
\item 
A die is thrown and a card is selected at random from a deck of 52 playing cards. The probability of getting an even number on the die and a spade card.\\
\solution
%\input{exemplar/12/13/3/78/main.tex}
\item
If 4-digit numbers greater than 5,000 are randomly formed from the digits 0, 1, 3, 5, and 7, what is the probability of forming a number divisible by 5 when:
\begin{enumerate}
    \item The digits are repeated?
    \item The repetition of digits is not allowed?
\end{enumerate}
\solution
%\input{ncert/11/16/4/9/main.tex}
\item Consider the probability space $\brak{\Omega, \mathcal{G}, P}$ where $\Omega = [0,2]$ and $\mathcal{G} = \cbrak{\phi, \Omega, [0,1], (1,2]}$. Let $X$ and $Y$ be two functions on $\Omega$ defined as
\begin{align*}
    X(\omega) = 
    \begin{cases}
        1 & \text{if }\omega \in [0, 1]\\
        2 & \text{if }\omega \in (1, 2]
    \end{cases}
\end{align*}
and
\begin{align*}
    Y(\omega) = 
    \begin{cases}
        2 & \text{if }\omega \in [0, 1.5]\\
        3 & \text{if }\omega \in (1.5, 2].
    \end{cases}
\end{align*}
Then which one of the following statements is true?
\begin{enumerate}
    \item [(A)] $X$ is a random variable with respect to $\mathcal{G}$, but $Y$ is not a random variable with respect to $\mathcal{G}$.
    \item [(B)] $Y$ is a random variable with respect to $\mathcal{G}$, but $X$ is not a random variable with respect to $\mathcal{G}$.
    \item [(C)] Neither $X$ nor $Y$ is a random variable with respect to $\mathcal{G}$.
    \item [(D)] Both $X$ and $Y$ are random variables with respect to $\mathcal{G}$.
\end{enumerate} \hfill (GATE ST 2023)\\
\solution
%\input{gate/ST/2023/14/main.tex}
	\item  A die is loaded in such a way that each odd number is twice as likely to occur as
each even number. Find $P(G)$, where $G$ is the event that a number greater than
3 occurs on a single roll of the die.
\\
\solution
		%\input{exemplar/11/16/3/5/main.tex}
	\item All the jacks, queens and kings are removed from a deck of 52 playing cards. The remaining cards are well shuffled and then one card is drawn at random. Giving ace a value 1 similar value for other cards, find the probability that the card has a value 
		\begin{enumerate}
			\item 7
			\item greater than 7
			\item less than 7
		\end{enumerate}
		%\input{exemplar/10/13/3/30/main.tex}
  \item A Lot consists of 48 mobile phones of which 42 are good, 3 have only minor defects and 3 have major defects.Varnika will buy a phone if it is good but the trader will only buy a mobile if it has no major defects. One phone is selected at random from the lot. What is the probability that it is
\begin{enumerate}
	\item acceptable to Varnika?
            \item acceptable to the trader?
\end{enumerate}
\solution
	%\input{exemplar/10/13/3/40/main.tex}
 \item A student says that if you throw a die, it will show up 1 or not 1. Therefore, the probability of getting 1 and the probability of getting 'not 1' each is equal to $\frac{1}{2}$. Is this correct? Give reasons.\\
 \solution
        %\input{exemplar/10/13/2/9/main.tex}
   \item Four candidates A, B, C, D have ap-
plied for the assignment to coach a school cricket
team. If A is twice as likely to be selected as B, and
B and C are given about the same chance of being
selected, while C is twice as likely to be selected
as D, what are the probabilities that
\begin{enumerate}
\item C will be selected?
\item A will not be selected?
\end{enumerate}
	%\input{exemplar/11/16/3/9/main.tex}
 \item A bag contain 24 balls of which $x$ balls are red, $2x$ are white and $3x$ are blue. A ball is selected at random, What is the probability that it is
\begin{enumerate}[label=\alph*)]
\item not red ?
\item white ?
\end{enumerate}
%\input{exemplar/10/13/3/41/main.tex}
If the letters of the word ASSASSINATION are arranged at random. Find the Probability that
\begin{enumerate}[label=(\alph*)]
\item Four $S's$ come consecutively in the word
\item Two  $I's$ and two $N's$ come together
\item All $A's$ are not coming together
\item No two $A's$ are coming together
\end{enumerate}
%\input{exemplar/11/16/3/14/main.tex}
	\item One urn contains two black balls (labelled B1 and B2) and one white ball. A
	second urn contains one black ball and two white balls (labelled W1 and W2).
	Suppose the following experiment is performed. One of the two urns is chosen
	at random. Next a ball is randomly chosen from the urn. Then a second ball is
	chosen at random from the same urn without replacing the first ball.
	
	\begin{enumerate}
	\item What is the probability that two black balls are chosen?
	
	\item What is the probability that two balls of opposite colour are chosen?
	\end{enumerate}
	\solution
	%\input{exemplar/11/16/3/12/main1.tex}
\end{enumerate}

\item In a certain lottery 10,000 tickets are sold and ten equal prizes are awarded. What is the probability of not getting a prize if you buy (a) one ticket (b) two tickets (c) 10 tickets ?	
\\
\solution
		%\begin{enumerate}[label=\thesection.\arabic*,ref=\thesection.\theenumi]
	\item One card is drawn from a well-shuffled deck of 52 cards. Find the probability of getting
\begin{enumerate}
\item A king of red colour 
\item A face card 
\item A red face card
\item The jack of hearts
\item A spade
\item The queen of diamonds

\end{enumerate}
\solution
		%\input{ncert/10/15/1/14/main.tex}
	\item Five cards—the ten, jack, queen, king and ace of diamonds, are well-shuffled with their face downwards. One card is then picked up at random.
\begin{enumerate}
\item
What is the probability that the card is the queen? 
\item
If the queen is drawn and put aside, what is the probability that the second card picked up is (a) an ace? (b) a queen?\\
\end{enumerate}
\solution
		%\input{ncert/10/15/1/15/defs.tex}
	\item A bag contains $5$ red balls and some blue balls. If the probability of drawing a blue ball is double that if a red ball, determine the number of blue balls in the bag. 
		\\
\solution
		%\input{ncert/10/15/2/3/defs.tex}
	\item A card is selected from a pack of 52 cards.
 \begin{enumerate}[label=(\alph*)] 
                 \item How many points are there in the sample space?
                 \item Calculate the probability that the card is an ace of spades.
                 \item Calculate the probability that the card is (i) an ace and (ii) black card.
 \end{enumerate}
\solution
		%\input{ncert/11/16/3/4/main.tex}
\item Four cards are drawn from a well-shuffled deck of 52 cards. What is the probability of obtaining 3 diamonds and one spade.
\\
\solution
		%\input{ncert/11/16/4/2/defs.tex}
\item In a certain lottery 10,000 tickets are sold and ten equal prizes are awarded. What is the probability of not getting a prize if you buy (a) one ticket (b) two tickets (c) 10 tickets ?	
\\
\solution
		%\input{ncert/11/16/4/4/defs.tex}
		%
\item 
Out of 100 students, two sections of 40 and 60 are formed. If you and your friend are among the 100 students, what is the probability that
\begin{enumerate}
\item you both enter the same section?
\item you both enter the different sections?
\end{enumerate}
\solution
		%\input{ncert/11/16/4/5/defs.tex}
	\item 
The number lock of a suitcase has 4 wheels each labelled with ten digits i.e. from 0 to 9.The lock opens with a sequence of four digits with no repeats.What is the probability of a person getting the right sequence to open the suitcase.
\\
\solution
		%\input{ncert/11/16/4/10/defs.tex}
		%
\item 
Two cards are drawn at random and without replacement from a pack of 52 playing cards. Find the probability that both the cards are black.
\\
\solution
		%\input{ncert/12/13/2/2/defs.tex}
		\item A box of oranges is inspected by examining three randomly selected oranges drawn without replacement. If all the three oranges are good, the box is approved for sale, otherwise, it is rejected. Find the probability that a box containing 15 oranges out of which 12 are good and 3 are bad ones will be approved for sale.
		\label{ncert/12/13/2/3/defs.tex}
		\item Two balls are drawn at random with replacement from a box containing 10 black and 8 red balls. Find the probability that
		\label{ncert/12/13/2/12}
\begin{enumerate}
\item both balls are red.
\item first ball is black and second is red.
\item one of them is black and other is red.
\end{enumerate}

\item In a hostel, 60\% of the students read Hindi newspaper, 40\% read English newspaper and 20\% read both Hindi and English newspapers. A student is selected at random.
		\label{ncert/12/13/2/15}
\begin{enumerate}
\item Find the probability that she reads neither Hindi nor English newspapers.
\item If she reads Hindi newspaper, find the probability that she reads English newspaper.
\item If she reads English newspaper, find the probability that she reads Hindi newspaper.\\
\end{enumerate}
\item The probability of obtaining an even prime number on each die, when a pair of dice is rolled is 
\begin{enumerate}
    \item $0$ 
    
    \item $\frac{1}{3}$ 
    
    \item $\frac{1}{12}$ 
    
    \item $\frac{1}{36}$ 
\end{enumerate}
\solution
		%\input{ncert/12/13/2/17/defs.tex}
	\item A bag contains 4 red and 4 black balls, another bag contains 2 red and 6 black balls. One of the two bags is selected at random and a ball is drawn from the bag which is found to be red. Find the probability that the ball is drawn from the first bag.
\\
\solution
		%\input{ncert/12/13/3/2/main.tex}
  \item
  Cards with numbers 2 to 101 are placed in a box. A card is selected at random.Find the probability that the card has
\begin{enumerate}[label=(\roman*)]
	\item an even number 
	\item a square number
\end{enumerate}
\solution
%\input{exemplar/10/13/3/32/main.tex}
\item
The king, queen and jack of clubs are removed from a deck of 52 playing cards and then well shuffled. Now one card is drawn at random from the remaining cards.  Determine the probability that the card is
\begin{enumerate}[label=(\roman*)]
\item a club
\item 10 of hearts
\end{enumerate}
\solution
%\input{exemplar/10/13/3/29/main.tex}
\item A team of medical students doing their internship have to assist during surgeries
at a city hospital. The probabilities of surgeries rated as very complex, complex,
routine, simple or very simple are respectively, 0.15, 0.20, 0.31, 0.26, .08. Find
the probabilities that a particular surgery will be rated
\begin{enumerate}
	\item complex or very complex;
	\item neither very complex nor very simple;
	\item routine or complex
	\item routine or simple
\end{enumerate}
\solution
%\input{exemplar/11/16/3/8(1)/main.tex}
\item A card is selected from a pack of 52 cards.
\begin{enumerate}[label=(\alph*)]
    \item How many points are there in the sample space?
    \item Calculate the probability that the card is an ace of spades.
    \item Calculate the probability that the card is (i) an ace and (ii) black card.
\end{enumerate}
\solution
%\input{exemplar/11/16/3/4/main2.tex}
\item The probability that a non leap year selected at random will contain 53 sundays.
\\
\solution
%\input{exemplar/10/13/1/19/main.tex}
\item One of the four persons John, Rita, Aslam or Gurpreet will be promoted next
month. Consequently the sample space consists of four elementary outcomes
S = {John promoted, Rita promoted, Aslam promoted, Gurpreet promoted}
You are told that the chances of John’s promotion is same as that of Gurpreet,
Rita’s chances of promotion are twice as likely as Johns. Aslam’s chances are
four times that of John.
\begin{enumerate}
	\item Determine
	\begin{enumerate}
		\item P (John promoted)
		\item P (Rita promoted)
		\item P (Aslam promoted)
		\item P (Gurpreet promoted)
	\end{enumerate}
	\item If A = {John promoted or Gurpreet promoted}, find P (A).
\end{enumerate}
\solution
%\input{exemplar/11/16/3/10/main.tex}
\item A card is drawn from a deck of 52 cards. Find the probability of getting a king or a heart or a red card.\\
\solution
%\input{exemplar/11/16/3/15/main.tex}
\item The probability that a student will pass his examination is 0.73, the probability of
the student getting a compartment is 0.13, and the probability that the student will
either pass or get compartment is 0.96. State True or False.\\
\solution
%\input{exemplar/11/16/3/31/main.tex}
\item A card is selected from a pack of 52 cards\\
\begin{enumerate}[label=(\alph*)]
\item How many points are there in the sample space?
\item Calculate the probability that the cards is an ace of spades.
\item Calculate the probability that the card is (i) an ace (ii)black card.\\
\end{enumerate}
%\input{ncert/11/16/3/4_1/Prob_4.tex}
\item In a non-leap year, the probability of having 53 tuesdays or 53 wednesdays is\\
\solution
%\input{exemplar/11/16/3/18/main.tex}
\item There are 1000 sealed envelopes in a box, 10 of them contain a cash prize of
Rs 100 each, 100 of them contain a cash prize of Rs 50 each and 200 of them
contain a cash prize of Rs 10 each and rest do not contain any cash prize. If they
are well shuffled and an envelope is picked up out, what is the probability that it
contains no cash prize?\\
\solution
%\input{exemplar/10/13/3/34/main.tex}
\item 
A die is thrown and a card is selected at random from a deck of 52 playing cards. The probability of getting an even number on the die and a spade card.\\
\solution
%\input{exemplar/12/13/3/78/main.tex}
\item
If 4-digit numbers greater than 5,000 are randomly formed from the digits 0, 1, 3, 5, and 7, what is the probability of forming a number divisible by 5 when:
\begin{enumerate}
    \item The digits are repeated?
    \item The repetition of digits is not allowed?
\end{enumerate}
\solution
%\input{ncert/11/16/4/9/main.tex}
\item Consider the probability space $\brak{\Omega, \mathcal{G}, P}$ where $\Omega = [0,2]$ and $\mathcal{G} = \cbrak{\phi, \Omega, [0,1], (1,2]}$. Let $X$ and $Y$ be two functions on $\Omega$ defined as
\begin{align*}
    X(\omega) = 
    \begin{cases}
        1 & \text{if }\omega \in [0, 1]\\
        2 & \text{if }\omega \in (1, 2]
    \end{cases}
\end{align*}
and
\begin{align*}
    Y(\omega) = 
    \begin{cases}
        2 & \text{if }\omega \in [0, 1.5]\\
        3 & \text{if }\omega \in (1.5, 2].
    \end{cases}
\end{align*}
Then which one of the following statements is true?
\begin{enumerate}
    \item [(A)] $X$ is a random variable with respect to $\mathcal{G}$, but $Y$ is not a random variable with respect to $\mathcal{G}$.
    \item [(B)] $Y$ is a random variable with respect to $\mathcal{G}$, but $X$ is not a random variable with respect to $\mathcal{G}$.
    \item [(C)] Neither $X$ nor $Y$ is a random variable with respect to $\mathcal{G}$.
    \item [(D)] Both $X$ and $Y$ are random variables with respect to $\mathcal{G}$.
\end{enumerate} \hfill (GATE ST 2023)\\
\solution
%\input{gate/ST/2023/14/main.tex}
	\item  A die is loaded in such a way that each odd number is twice as likely to occur as
each even number. Find $P(G)$, where $G$ is the event that a number greater than
3 occurs on a single roll of the die.
\\
\solution
		%\input{exemplar/11/16/3/5/main.tex}
	\item All the jacks, queens and kings are removed from a deck of 52 playing cards. The remaining cards are well shuffled and then one card is drawn at random. Giving ace a value 1 similar value for other cards, find the probability that the card has a value 
		\begin{enumerate}
			\item 7
			\item greater than 7
			\item less than 7
		\end{enumerate}
		%\input{exemplar/10/13/3/30/main.tex}
  \item A Lot consists of 48 mobile phones of which 42 are good, 3 have only minor defects and 3 have major defects.Varnika will buy a phone if it is good but the trader will only buy a mobile if it has no major defects. One phone is selected at random from the lot. What is the probability that it is
\begin{enumerate}
	\item acceptable to Varnika?
            \item acceptable to the trader?
\end{enumerate}
\solution
	%\input{exemplar/10/13/3/40/main.tex}
 \item A student says that if you throw a die, it will show up 1 or not 1. Therefore, the probability of getting 1 and the probability of getting 'not 1' each is equal to $\frac{1}{2}$. Is this correct? Give reasons.\\
 \solution
        %\input{exemplar/10/13/2/9/main.tex}
   \item Four candidates A, B, C, D have ap-
plied for the assignment to coach a school cricket
team. If A is twice as likely to be selected as B, and
B and C are given about the same chance of being
selected, while C is twice as likely to be selected
as D, what are the probabilities that
\begin{enumerate}
\item C will be selected?
\item A will not be selected?
\end{enumerate}
	%\input{exemplar/11/16/3/9/main.tex}
 \item A bag contain 24 balls of which $x$ balls are red, $2x$ are white and $3x$ are blue. A ball is selected at random, What is the probability that it is
\begin{enumerate}[label=\alph*)]
\item not red ?
\item white ?
\end{enumerate}
%\input{exemplar/10/13/3/41/main.tex}
If the letters of the word ASSASSINATION are arranged at random. Find the Probability that
\begin{enumerate}[label=(\alph*)]
\item Four $S's$ come consecutively in the word
\item Two  $I's$ and two $N's$ come together
\item All $A's$ are not coming together
\item No two $A's$ are coming together
\end{enumerate}
%\input{exemplar/11/16/3/14/main.tex}
	\item One urn contains two black balls (labelled B1 and B2) and one white ball. A
	second urn contains one black ball and two white balls (labelled W1 and W2).
	Suppose the following experiment is performed. One of the two urns is chosen
	at random. Next a ball is randomly chosen from the urn. Then a second ball is
	chosen at random from the same urn without replacing the first ball.
	
	\begin{enumerate}
	\item What is the probability that two black balls are chosen?
	
	\item What is the probability that two balls of opposite colour are chosen?
	\end{enumerate}
	\solution
	%\input{exemplar/11/16/3/12/main1.tex}
\end{enumerate}

		%
\item 
Out of 100 students, two sections of 40 and 60 are formed. If you and your friend are among the 100 students, what is the probability that
\begin{enumerate}
\item you both enter the same section?
\item you both enter the different sections?
\end{enumerate}
\solution
		%\begin{enumerate}[label=\thesection.\arabic*,ref=\thesection.\theenumi]
	\item One card is drawn from a well-shuffled deck of 52 cards. Find the probability of getting
\begin{enumerate}
\item A king of red colour 
\item A face card 
\item A red face card
\item The jack of hearts
\item A spade
\item The queen of diamonds

\end{enumerate}
\solution
		%\input{ncert/10/15/1/14/main.tex}
	\item Five cards—the ten, jack, queen, king and ace of diamonds, are well-shuffled with their face downwards. One card is then picked up at random.
\begin{enumerate}
\item
What is the probability that the card is the queen? 
\item
If the queen is drawn and put aside, what is the probability that the second card picked up is (a) an ace? (b) a queen?\\
\end{enumerate}
\solution
		%\input{ncert/10/15/1/15/defs.tex}
	\item A bag contains $5$ red balls and some blue balls. If the probability of drawing a blue ball is double that if a red ball, determine the number of blue balls in the bag. 
		\\
\solution
		%\input{ncert/10/15/2/3/defs.tex}
	\item A card is selected from a pack of 52 cards.
 \begin{enumerate}[label=(\alph*)] 
                 \item How many points are there in the sample space?
                 \item Calculate the probability that the card is an ace of spades.
                 \item Calculate the probability that the card is (i) an ace and (ii) black card.
 \end{enumerate}
\solution
		%\input{ncert/11/16/3/4/main.tex}
\item Four cards are drawn from a well-shuffled deck of 52 cards. What is the probability of obtaining 3 diamonds and one spade.
\\
\solution
		%\input{ncert/11/16/4/2/defs.tex}
\item In a certain lottery 10,000 tickets are sold and ten equal prizes are awarded. What is the probability of not getting a prize if you buy (a) one ticket (b) two tickets (c) 10 tickets ?	
\\
\solution
		%\input{ncert/11/16/4/4/defs.tex}
		%
\item 
Out of 100 students, two sections of 40 and 60 are formed. If you and your friend are among the 100 students, what is the probability that
\begin{enumerate}
\item you both enter the same section?
\item you both enter the different sections?
\end{enumerate}
\solution
		%\input{ncert/11/16/4/5/defs.tex}
	\item 
The number lock of a suitcase has 4 wheels each labelled with ten digits i.e. from 0 to 9.The lock opens with a sequence of four digits with no repeats.What is the probability of a person getting the right sequence to open the suitcase.
\\
\solution
		%\input{ncert/11/16/4/10/defs.tex}
		%
\item 
Two cards are drawn at random and without replacement from a pack of 52 playing cards. Find the probability that both the cards are black.
\\
\solution
		%\input{ncert/12/13/2/2/defs.tex}
		\item A box of oranges is inspected by examining three randomly selected oranges drawn without replacement. If all the three oranges are good, the box is approved for sale, otherwise, it is rejected. Find the probability that a box containing 15 oranges out of which 12 are good and 3 are bad ones will be approved for sale.
		\label{ncert/12/13/2/3/defs.tex}
		\item Two balls are drawn at random with replacement from a box containing 10 black and 8 red balls. Find the probability that
		\label{ncert/12/13/2/12}
\begin{enumerate}
\item both balls are red.
\item first ball is black and second is red.
\item one of them is black and other is red.
\end{enumerate}

\item In a hostel, 60\% of the students read Hindi newspaper, 40\% read English newspaper and 20\% read both Hindi and English newspapers. A student is selected at random.
		\label{ncert/12/13/2/15}
\begin{enumerate}
\item Find the probability that she reads neither Hindi nor English newspapers.
\item If she reads Hindi newspaper, find the probability that she reads English newspaper.
\item If she reads English newspaper, find the probability that she reads Hindi newspaper.\\
\end{enumerate}
\item The probability of obtaining an even prime number on each die, when a pair of dice is rolled is 
\begin{enumerate}
    \item $0$ 
    
    \item $\frac{1}{3}$ 
    
    \item $\frac{1}{12}$ 
    
    \item $\frac{1}{36}$ 
\end{enumerate}
\solution
		%\input{ncert/12/13/2/17/defs.tex}
	\item A bag contains 4 red and 4 black balls, another bag contains 2 red and 6 black balls. One of the two bags is selected at random and a ball is drawn from the bag which is found to be red. Find the probability that the ball is drawn from the first bag.
\\
\solution
		%\input{ncert/12/13/3/2/main.tex}
  \item
  Cards with numbers 2 to 101 are placed in a box. A card is selected at random.Find the probability that the card has
\begin{enumerate}[label=(\roman*)]
	\item an even number 
	\item a square number
\end{enumerate}
\solution
%\input{exemplar/10/13/3/32/main.tex}
\item
The king, queen and jack of clubs are removed from a deck of 52 playing cards and then well shuffled. Now one card is drawn at random from the remaining cards.  Determine the probability that the card is
\begin{enumerate}[label=(\roman*)]
\item a club
\item 10 of hearts
\end{enumerate}
\solution
%\input{exemplar/10/13/3/29/main.tex}
\item A team of medical students doing their internship have to assist during surgeries
at a city hospital. The probabilities of surgeries rated as very complex, complex,
routine, simple or very simple are respectively, 0.15, 0.20, 0.31, 0.26, .08. Find
the probabilities that a particular surgery will be rated
\begin{enumerate}
	\item complex or very complex;
	\item neither very complex nor very simple;
	\item routine or complex
	\item routine or simple
\end{enumerate}
\solution
%\input{exemplar/11/16/3/8(1)/main.tex}
\item A card is selected from a pack of 52 cards.
\begin{enumerate}[label=(\alph*)]
    \item How many points are there in the sample space?
    \item Calculate the probability that the card is an ace of spades.
    \item Calculate the probability that the card is (i) an ace and (ii) black card.
\end{enumerate}
\solution
%\input{exemplar/11/16/3/4/main2.tex}
\item The probability that a non leap year selected at random will contain 53 sundays.
\\
\solution
%\input{exemplar/10/13/1/19/main.tex}
\item One of the four persons John, Rita, Aslam or Gurpreet will be promoted next
month. Consequently the sample space consists of four elementary outcomes
S = {John promoted, Rita promoted, Aslam promoted, Gurpreet promoted}
You are told that the chances of John’s promotion is same as that of Gurpreet,
Rita’s chances of promotion are twice as likely as Johns. Aslam’s chances are
four times that of John.
\begin{enumerate}
	\item Determine
	\begin{enumerate}
		\item P (John promoted)
		\item P (Rita promoted)
		\item P (Aslam promoted)
		\item P (Gurpreet promoted)
	\end{enumerate}
	\item If A = {John promoted or Gurpreet promoted}, find P (A).
\end{enumerate}
\solution
%\input{exemplar/11/16/3/10/main.tex}
\item A card is drawn from a deck of 52 cards. Find the probability of getting a king or a heart or a red card.\\
\solution
%\input{exemplar/11/16/3/15/main.tex}
\item The probability that a student will pass his examination is 0.73, the probability of
the student getting a compartment is 0.13, and the probability that the student will
either pass or get compartment is 0.96. State True or False.\\
\solution
%\input{exemplar/11/16/3/31/main.tex}
\item A card is selected from a pack of 52 cards\\
\begin{enumerate}[label=(\alph*)]
\item How many points are there in the sample space?
\item Calculate the probability that the cards is an ace of spades.
\item Calculate the probability that the card is (i) an ace (ii)black card.\\
\end{enumerate}
%\input{ncert/11/16/3/4_1/Prob_4.tex}
\item In a non-leap year, the probability of having 53 tuesdays or 53 wednesdays is\\
\solution
%\input{exemplar/11/16/3/18/main.tex}
\item There are 1000 sealed envelopes in a box, 10 of them contain a cash prize of
Rs 100 each, 100 of them contain a cash prize of Rs 50 each and 200 of them
contain a cash prize of Rs 10 each and rest do not contain any cash prize. If they
are well shuffled and an envelope is picked up out, what is the probability that it
contains no cash prize?\\
\solution
%\input{exemplar/10/13/3/34/main.tex}
\item 
A die is thrown and a card is selected at random from a deck of 52 playing cards. The probability of getting an even number on the die and a spade card.\\
\solution
%\input{exemplar/12/13/3/78/main.tex}
\item
If 4-digit numbers greater than 5,000 are randomly formed from the digits 0, 1, 3, 5, and 7, what is the probability of forming a number divisible by 5 when:
\begin{enumerate}
    \item The digits are repeated?
    \item The repetition of digits is not allowed?
\end{enumerate}
\solution
%\input{ncert/11/16/4/9/main.tex}
\item Consider the probability space $\brak{\Omega, \mathcal{G}, P}$ where $\Omega = [0,2]$ and $\mathcal{G} = \cbrak{\phi, \Omega, [0,1], (1,2]}$. Let $X$ and $Y$ be two functions on $\Omega$ defined as
\begin{align*}
    X(\omega) = 
    \begin{cases}
        1 & \text{if }\omega \in [0, 1]\\
        2 & \text{if }\omega \in (1, 2]
    \end{cases}
\end{align*}
and
\begin{align*}
    Y(\omega) = 
    \begin{cases}
        2 & \text{if }\omega \in [0, 1.5]\\
        3 & \text{if }\omega \in (1.5, 2].
    \end{cases}
\end{align*}
Then which one of the following statements is true?
\begin{enumerate}
    \item [(A)] $X$ is a random variable with respect to $\mathcal{G}$, but $Y$ is not a random variable with respect to $\mathcal{G}$.
    \item [(B)] $Y$ is a random variable with respect to $\mathcal{G}$, but $X$ is not a random variable with respect to $\mathcal{G}$.
    \item [(C)] Neither $X$ nor $Y$ is a random variable with respect to $\mathcal{G}$.
    \item [(D)] Both $X$ and $Y$ are random variables with respect to $\mathcal{G}$.
\end{enumerate} \hfill (GATE ST 2023)\\
\solution
%\input{gate/ST/2023/14/main.tex}
	\item  A die is loaded in such a way that each odd number is twice as likely to occur as
each even number. Find $P(G)$, where $G$ is the event that a number greater than
3 occurs on a single roll of the die.
\\
\solution
		%\input{exemplar/11/16/3/5/main.tex}
	\item All the jacks, queens and kings are removed from a deck of 52 playing cards. The remaining cards are well shuffled and then one card is drawn at random. Giving ace a value 1 similar value for other cards, find the probability that the card has a value 
		\begin{enumerate}
			\item 7
			\item greater than 7
			\item less than 7
		\end{enumerate}
		%\input{exemplar/10/13/3/30/main.tex}
  \item A Lot consists of 48 mobile phones of which 42 are good, 3 have only minor defects and 3 have major defects.Varnika will buy a phone if it is good but the trader will only buy a mobile if it has no major defects. One phone is selected at random from the lot. What is the probability that it is
\begin{enumerate}
	\item acceptable to Varnika?
            \item acceptable to the trader?
\end{enumerate}
\solution
	%\input{exemplar/10/13/3/40/main.tex}
 \item A student says that if you throw a die, it will show up 1 or not 1. Therefore, the probability of getting 1 and the probability of getting 'not 1' each is equal to $\frac{1}{2}$. Is this correct? Give reasons.\\
 \solution
        %\input{exemplar/10/13/2/9/main.tex}
   \item Four candidates A, B, C, D have ap-
plied for the assignment to coach a school cricket
team. If A is twice as likely to be selected as B, and
B and C are given about the same chance of being
selected, while C is twice as likely to be selected
as D, what are the probabilities that
\begin{enumerate}
\item C will be selected?
\item A will not be selected?
\end{enumerate}
	%\input{exemplar/11/16/3/9/main.tex}
 \item A bag contain 24 balls of which $x$ balls are red, $2x$ are white and $3x$ are blue. A ball is selected at random, What is the probability that it is
\begin{enumerate}[label=\alph*)]
\item not red ?
\item white ?
\end{enumerate}
%\input{exemplar/10/13/3/41/main.tex}
If the letters of the word ASSASSINATION are arranged at random. Find the Probability that
\begin{enumerate}[label=(\alph*)]
\item Four $S's$ come consecutively in the word
\item Two  $I's$ and two $N's$ come together
\item All $A's$ are not coming together
\item No two $A's$ are coming together
\end{enumerate}
%\input{exemplar/11/16/3/14/main.tex}
	\item One urn contains two black balls (labelled B1 and B2) and one white ball. A
	second urn contains one black ball and two white balls (labelled W1 and W2).
	Suppose the following experiment is performed. One of the two urns is chosen
	at random. Next a ball is randomly chosen from the urn. Then a second ball is
	chosen at random from the same urn without replacing the first ball.
	
	\begin{enumerate}
	\item What is the probability that two black balls are chosen?
	
	\item What is the probability that two balls of opposite colour are chosen?
	\end{enumerate}
	\solution
	%\input{exemplar/11/16/3/12/main1.tex}
\end{enumerate}

	\item 
The number lock of a suitcase has 4 wheels each labelled with ten digits i.e. from 0 to 9.The lock opens with a sequence of four digits with no repeats.What is the probability of a person getting the right sequence to open the suitcase.
\\
\solution
		%\begin{enumerate}[label=\thesection.\arabic*,ref=\thesection.\theenumi]
	\item One card is drawn from a well-shuffled deck of 52 cards. Find the probability of getting
\begin{enumerate}
\item A king of red colour 
\item A face card 
\item A red face card
\item The jack of hearts
\item A spade
\item The queen of diamonds

\end{enumerate}
\solution
		%\input{ncert/10/15/1/14/main.tex}
	\item Five cards—the ten, jack, queen, king and ace of diamonds, are well-shuffled with their face downwards. One card is then picked up at random.
\begin{enumerate}
\item
What is the probability that the card is the queen? 
\item
If the queen is drawn and put aside, what is the probability that the second card picked up is (a) an ace? (b) a queen?\\
\end{enumerate}
\solution
		%\input{ncert/10/15/1/15/defs.tex}
	\item A bag contains $5$ red balls and some blue balls. If the probability of drawing a blue ball is double that if a red ball, determine the number of blue balls in the bag. 
		\\
\solution
		%\input{ncert/10/15/2/3/defs.tex}
	\item A card is selected from a pack of 52 cards.
 \begin{enumerate}[label=(\alph*)] 
                 \item How many points are there in the sample space?
                 \item Calculate the probability that the card is an ace of spades.
                 \item Calculate the probability that the card is (i) an ace and (ii) black card.
 \end{enumerate}
\solution
		%\input{ncert/11/16/3/4/main.tex}
\item Four cards are drawn from a well-shuffled deck of 52 cards. What is the probability of obtaining 3 diamonds and one spade.
\\
\solution
		%\input{ncert/11/16/4/2/defs.tex}
\item In a certain lottery 10,000 tickets are sold and ten equal prizes are awarded. What is the probability of not getting a prize if you buy (a) one ticket (b) two tickets (c) 10 tickets ?	
\\
\solution
		%\input{ncert/11/16/4/4/defs.tex}
		%
\item 
Out of 100 students, two sections of 40 and 60 are formed. If you and your friend are among the 100 students, what is the probability that
\begin{enumerate}
\item you both enter the same section?
\item you both enter the different sections?
\end{enumerate}
\solution
		%\input{ncert/11/16/4/5/defs.tex}
	\item 
The number lock of a suitcase has 4 wheels each labelled with ten digits i.e. from 0 to 9.The lock opens with a sequence of four digits with no repeats.What is the probability of a person getting the right sequence to open the suitcase.
\\
\solution
		%\input{ncert/11/16/4/10/defs.tex}
		%
\item 
Two cards are drawn at random and without replacement from a pack of 52 playing cards. Find the probability that both the cards are black.
\\
\solution
		%\input{ncert/12/13/2/2/defs.tex}
		\item A box of oranges is inspected by examining three randomly selected oranges drawn without replacement. If all the three oranges are good, the box is approved for sale, otherwise, it is rejected. Find the probability that a box containing 15 oranges out of which 12 are good and 3 are bad ones will be approved for sale.
		\label{ncert/12/13/2/3/defs.tex}
		\item Two balls are drawn at random with replacement from a box containing 10 black and 8 red balls. Find the probability that
		\label{ncert/12/13/2/12}
\begin{enumerate}
\item both balls are red.
\item first ball is black and second is red.
\item one of them is black and other is red.
\end{enumerate}

\item In a hostel, 60\% of the students read Hindi newspaper, 40\% read English newspaper and 20\% read both Hindi and English newspapers. A student is selected at random.
		\label{ncert/12/13/2/15}
\begin{enumerate}
\item Find the probability that she reads neither Hindi nor English newspapers.
\item If she reads Hindi newspaper, find the probability that she reads English newspaper.
\item If she reads English newspaper, find the probability that she reads Hindi newspaper.\\
\end{enumerate}
\item The probability of obtaining an even prime number on each die, when a pair of dice is rolled is 
\begin{enumerate}
    \item $0$ 
    
    \item $\frac{1}{3}$ 
    
    \item $\frac{1}{12}$ 
    
    \item $\frac{1}{36}$ 
\end{enumerate}
\solution
		%\input{ncert/12/13/2/17/defs.tex}
	\item A bag contains 4 red and 4 black balls, another bag contains 2 red and 6 black balls. One of the two bags is selected at random and a ball is drawn from the bag which is found to be red. Find the probability that the ball is drawn from the first bag.
\\
\solution
		%\input{ncert/12/13/3/2/main.tex}
  \item
  Cards with numbers 2 to 101 are placed in a box. A card is selected at random.Find the probability that the card has
\begin{enumerate}[label=(\roman*)]
	\item an even number 
	\item a square number
\end{enumerate}
\solution
%\input{exemplar/10/13/3/32/main.tex}
\item
The king, queen and jack of clubs are removed from a deck of 52 playing cards and then well shuffled. Now one card is drawn at random from the remaining cards.  Determine the probability that the card is
\begin{enumerate}[label=(\roman*)]
\item a club
\item 10 of hearts
\end{enumerate}
\solution
%\input{exemplar/10/13/3/29/main.tex}
\item A team of medical students doing their internship have to assist during surgeries
at a city hospital. The probabilities of surgeries rated as very complex, complex,
routine, simple or very simple are respectively, 0.15, 0.20, 0.31, 0.26, .08. Find
the probabilities that a particular surgery will be rated
\begin{enumerate}
	\item complex or very complex;
	\item neither very complex nor very simple;
	\item routine or complex
	\item routine or simple
\end{enumerate}
\solution
%\input{exemplar/11/16/3/8(1)/main.tex}
\item A card is selected from a pack of 52 cards.
\begin{enumerate}[label=(\alph*)]
    \item How many points are there in the sample space?
    \item Calculate the probability that the card is an ace of spades.
    \item Calculate the probability that the card is (i) an ace and (ii) black card.
\end{enumerate}
\solution
%\input{exemplar/11/16/3/4/main2.tex}
\item The probability that a non leap year selected at random will contain 53 sundays.
\\
\solution
%\input{exemplar/10/13/1/19/main.tex}
\item One of the four persons John, Rita, Aslam or Gurpreet will be promoted next
month. Consequently the sample space consists of four elementary outcomes
S = {John promoted, Rita promoted, Aslam promoted, Gurpreet promoted}
You are told that the chances of John’s promotion is same as that of Gurpreet,
Rita’s chances of promotion are twice as likely as Johns. Aslam’s chances are
four times that of John.
\begin{enumerate}
	\item Determine
	\begin{enumerate}
		\item P (John promoted)
		\item P (Rita promoted)
		\item P (Aslam promoted)
		\item P (Gurpreet promoted)
	\end{enumerate}
	\item If A = {John promoted or Gurpreet promoted}, find P (A).
\end{enumerate}
\solution
%\input{exemplar/11/16/3/10/main.tex}
\item A card is drawn from a deck of 52 cards. Find the probability of getting a king or a heart or a red card.\\
\solution
%\input{exemplar/11/16/3/15/main.tex}
\item The probability that a student will pass his examination is 0.73, the probability of
the student getting a compartment is 0.13, and the probability that the student will
either pass or get compartment is 0.96. State True or False.\\
\solution
%\input{exemplar/11/16/3/31/main.tex}
\item A card is selected from a pack of 52 cards\\
\begin{enumerate}[label=(\alph*)]
\item How many points are there in the sample space?
\item Calculate the probability that the cards is an ace of spades.
\item Calculate the probability that the card is (i) an ace (ii)black card.\\
\end{enumerate}
%\input{ncert/11/16/3/4_1/Prob_4.tex}
\item In a non-leap year, the probability of having 53 tuesdays or 53 wednesdays is\\
\solution
%\input{exemplar/11/16/3/18/main.tex}
\item There are 1000 sealed envelopes in a box, 10 of them contain a cash prize of
Rs 100 each, 100 of them contain a cash prize of Rs 50 each and 200 of them
contain a cash prize of Rs 10 each and rest do not contain any cash prize. If they
are well shuffled and an envelope is picked up out, what is the probability that it
contains no cash prize?\\
\solution
%\input{exemplar/10/13/3/34/main.tex}
\item 
A die is thrown and a card is selected at random from a deck of 52 playing cards. The probability of getting an even number on the die and a spade card.\\
\solution
%\input{exemplar/12/13/3/78/main.tex}
\item
If 4-digit numbers greater than 5,000 are randomly formed from the digits 0, 1, 3, 5, and 7, what is the probability of forming a number divisible by 5 when:
\begin{enumerate}
    \item The digits are repeated?
    \item The repetition of digits is not allowed?
\end{enumerate}
\solution
%\input{ncert/11/16/4/9/main.tex}
\item Consider the probability space $\brak{\Omega, \mathcal{G}, P}$ where $\Omega = [0,2]$ and $\mathcal{G} = \cbrak{\phi, \Omega, [0,1], (1,2]}$. Let $X$ and $Y$ be two functions on $\Omega$ defined as
\begin{align*}
    X(\omega) = 
    \begin{cases}
        1 & \text{if }\omega \in [0, 1]\\
        2 & \text{if }\omega \in (1, 2]
    \end{cases}
\end{align*}
and
\begin{align*}
    Y(\omega) = 
    \begin{cases}
        2 & \text{if }\omega \in [0, 1.5]\\
        3 & \text{if }\omega \in (1.5, 2].
    \end{cases}
\end{align*}
Then which one of the following statements is true?
\begin{enumerate}
    \item [(A)] $X$ is a random variable with respect to $\mathcal{G}$, but $Y$ is not a random variable with respect to $\mathcal{G}$.
    \item [(B)] $Y$ is a random variable with respect to $\mathcal{G}$, but $X$ is not a random variable with respect to $\mathcal{G}$.
    \item [(C)] Neither $X$ nor $Y$ is a random variable with respect to $\mathcal{G}$.
    \item [(D)] Both $X$ and $Y$ are random variables with respect to $\mathcal{G}$.
\end{enumerate} \hfill (GATE ST 2023)\\
\solution
%\input{gate/ST/2023/14/main.tex}
	\item  A die is loaded in such a way that each odd number is twice as likely to occur as
each even number. Find $P(G)$, where $G$ is the event that a number greater than
3 occurs on a single roll of the die.
\\
\solution
		%\input{exemplar/11/16/3/5/main.tex}
	\item All the jacks, queens and kings are removed from a deck of 52 playing cards. The remaining cards are well shuffled and then one card is drawn at random. Giving ace a value 1 similar value for other cards, find the probability that the card has a value 
		\begin{enumerate}
			\item 7
			\item greater than 7
			\item less than 7
		\end{enumerate}
		%\input{exemplar/10/13/3/30/main.tex}
  \item A Lot consists of 48 mobile phones of which 42 are good, 3 have only minor defects and 3 have major defects.Varnika will buy a phone if it is good but the trader will only buy a mobile if it has no major defects. One phone is selected at random from the lot. What is the probability that it is
\begin{enumerate}
	\item acceptable to Varnika?
            \item acceptable to the trader?
\end{enumerate}
\solution
	%\input{exemplar/10/13/3/40/main.tex}
 \item A student says that if you throw a die, it will show up 1 or not 1. Therefore, the probability of getting 1 and the probability of getting 'not 1' each is equal to $\frac{1}{2}$. Is this correct? Give reasons.\\
 \solution
        %\input{exemplar/10/13/2/9/main.tex}
   \item Four candidates A, B, C, D have ap-
plied for the assignment to coach a school cricket
team. If A is twice as likely to be selected as B, and
B and C are given about the same chance of being
selected, while C is twice as likely to be selected
as D, what are the probabilities that
\begin{enumerate}
\item C will be selected?
\item A will not be selected?
\end{enumerate}
	%\input{exemplar/11/16/3/9/main.tex}
 \item A bag contain 24 balls of which $x$ balls are red, $2x$ are white and $3x$ are blue. A ball is selected at random, What is the probability that it is
\begin{enumerate}[label=\alph*)]
\item not red ?
\item white ?
\end{enumerate}
%\input{exemplar/10/13/3/41/main.tex}
If the letters of the word ASSASSINATION are arranged at random. Find the Probability that
\begin{enumerate}[label=(\alph*)]
\item Four $S's$ come consecutively in the word
\item Two  $I's$ and two $N's$ come together
\item All $A's$ are not coming together
\item No two $A's$ are coming together
\end{enumerate}
%\input{exemplar/11/16/3/14/main.tex}
	\item One urn contains two black balls (labelled B1 and B2) and one white ball. A
	second urn contains one black ball and two white balls (labelled W1 and W2).
	Suppose the following experiment is performed. One of the two urns is chosen
	at random. Next a ball is randomly chosen from the urn. Then a second ball is
	chosen at random from the same urn without replacing the first ball.
	
	\begin{enumerate}
	\item What is the probability that two black balls are chosen?
	
	\item What is the probability that two balls of opposite colour are chosen?
	\end{enumerate}
	\solution
	%\input{exemplar/11/16/3/12/main1.tex}
\end{enumerate}

		%
\item 
Two cards are drawn at random and without replacement from a pack of 52 playing cards. Find the probability that both the cards are black.
\\
\solution
		%\begin{enumerate}[label=\thesection.\arabic*,ref=\thesection.\theenumi]
	\item One card is drawn from a well-shuffled deck of 52 cards. Find the probability of getting
\begin{enumerate}
\item A king of red colour 
\item A face card 
\item A red face card
\item The jack of hearts
\item A spade
\item The queen of diamonds

\end{enumerate}
\solution
		%\input{ncert/10/15/1/14/main.tex}
	\item Five cards—the ten, jack, queen, king and ace of diamonds, are well-shuffled with their face downwards. One card is then picked up at random.
\begin{enumerate}
\item
What is the probability that the card is the queen? 
\item
If the queen is drawn and put aside, what is the probability that the second card picked up is (a) an ace? (b) a queen?\\
\end{enumerate}
\solution
		%\input{ncert/10/15/1/15/defs.tex}
	\item A bag contains $5$ red balls and some blue balls. If the probability of drawing a blue ball is double that if a red ball, determine the number of blue balls in the bag. 
		\\
\solution
		%\input{ncert/10/15/2/3/defs.tex}
	\item A card is selected from a pack of 52 cards.
 \begin{enumerate}[label=(\alph*)] 
                 \item How many points are there in the sample space?
                 \item Calculate the probability that the card is an ace of spades.
                 \item Calculate the probability that the card is (i) an ace and (ii) black card.
 \end{enumerate}
\solution
		%\input{ncert/11/16/3/4/main.tex}
\item Four cards are drawn from a well-shuffled deck of 52 cards. What is the probability of obtaining 3 diamonds and one spade.
\\
\solution
		%\input{ncert/11/16/4/2/defs.tex}
\item In a certain lottery 10,000 tickets are sold and ten equal prizes are awarded. What is the probability of not getting a prize if you buy (a) one ticket (b) two tickets (c) 10 tickets ?	
\\
\solution
		%\input{ncert/11/16/4/4/defs.tex}
		%
\item 
Out of 100 students, two sections of 40 and 60 are formed. If you and your friend are among the 100 students, what is the probability that
\begin{enumerate}
\item you both enter the same section?
\item you both enter the different sections?
\end{enumerate}
\solution
		%\input{ncert/11/16/4/5/defs.tex}
	\item 
The number lock of a suitcase has 4 wheels each labelled with ten digits i.e. from 0 to 9.The lock opens with a sequence of four digits with no repeats.What is the probability of a person getting the right sequence to open the suitcase.
\\
\solution
		%\input{ncert/11/16/4/10/defs.tex}
		%
\item 
Two cards are drawn at random and without replacement from a pack of 52 playing cards. Find the probability that both the cards are black.
\\
\solution
		%\input{ncert/12/13/2/2/defs.tex}
		\item A box of oranges is inspected by examining three randomly selected oranges drawn without replacement. If all the three oranges are good, the box is approved for sale, otherwise, it is rejected. Find the probability that a box containing 15 oranges out of which 12 are good and 3 are bad ones will be approved for sale.
		\label{ncert/12/13/2/3/defs.tex}
		\item Two balls are drawn at random with replacement from a box containing 10 black and 8 red balls. Find the probability that
		\label{ncert/12/13/2/12}
\begin{enumerate}
\item both balls are red.
\item first ball is black and second is red.
\item one of them is black and other is red.
\end{enumerate}

\item In a hostel, 60\% of the students read Hindi newspaper, 40\% read English newspaper and 20\% read both Hindi and English newspapers. A student is selected at random.
		\label{ncert/12/13/2/15}
\begin{enumerate}
\item Find the probability that she reads neither Hindi nor English newspapers.
\item If she reads Hindi newspaper, find the probability that she reads English newspaper.
\item If she reads English newspaper, find the probability that she reads Hindi newspaper.\\
\end{enumerate}
\item The probability of obtaining an even prime number on each die, when a pair of dice is rolled is 
\begin{enumerate}
    \item $0$ 
    
    \item $\frac{1}{3}$ 
    
    \item $\frac{1}{12}$ 
    
    \item $\frac{1}{36}$ 
\end{enumerate}
\solution
		%\input{ncert/12/13/2/17/defs.tex}
	\item A bag contains 4 red and 4 black balls, another bag contains 2 red and 6 black balls. One of the two bags is selected at random and a ball is drawn from the bag which is found to be red. Find the probability that the ball is drawn from the first bag.
\\
\solution
		%\input{ncert/12/13/3/2/main.tex}
  \item
  Cards with numbers 2 to 101 are placed in a box. A card is selected at random.Find the probability that the card has
\begin{enumerate}[label=(\roman*)]
	\item an even number 
	\item a square number
\end{enumerate}
\solution
%\input{exemplar/10/13/3/32/main.tex}
\item
The king, queen and jack of clubs are removed from a deck of 52 playing cards and then well shuffled. Now one card is drawn at random from the remaining cards.  Determine the probability that the card is
\begin{enumerate}[label=(\roman*)]
\item a club
\item 10 of hearts
\end{enumerate}
\solution
%\input{exemplar/10/13/3/29/main.tex}
\item A team of medical students doing their internship have to assist during surgeries
at a city hospital. The probabilities of surgeries rated as very complex, complex,
routine, simple or very simple are respectively, 0.15, 0.20, 0.31, 0.26, .08. Find
the probabilities that a particular surgery will be rated
\begin{enumerate}
	\item complex or very complex;
	\item neither very complex nor very simple;
	\item routine or complex
	\item routine or simple
\end{enumerate}
\solution
%\input{exemplar/11/16/3/8(1)/main.tex}
\item A card is selected from a pack of 52 cards.
\begin{enumerate}[label=(\alph*)]
    \item How many points are there in the sample space?
    \item Calculate the probability that the card is an ace of spades.
    \item Calculate the probability that the card is (i) an ace and (ii) black card.
\end{enumerate}
\solution
%\input{exemplar/11/16/3/4/main2.tex}
\item The probability that a non leap year selected at random will contain 53 sundays.
\\
\solution
%\input{exemplar/10/13/1/19/main.tex}
\item One of the four persons John, Rita, Aslam or Gurpreet will be promoted next
month. Consequently the sample space consists of four elementary outcomes
S = {John promoted, Rita promoted, Aslam promoted, Gurpreet promoted}
You are told that the chances of John’s promotion is same as that of Gurpreet,
Rita’s chances of promotion are twice as likely as Johns. Aslam’s chances are
four times that of John.
\begin{enumerate}
	\item Determine
	\begin{enumerate}
		\item P (John promoted)
		\item P (Rita promoted)
		\item P (Aslam promoted)
		\item P (Gurpreet promoted)
	\end{enumerate}
	\item If A = {John promoted or Gurpreet promoted}, find P (A).
\end{enumerate}
\solution
%\input{exemplar/11/16/3/10/main.tex}
\item A card is drawn from a deck of 52 cards. Find the probability of getting a king or a heart or a red card.\\
\solution
%\input{exemplar/11/16/3/15/main.tex}
\item The probability that a student will pass his examination is 0.73, the probability of
the student getting a compartment is 0.13, and the probability that the student will
either pass or get compartment is 0.96. State True or False.\\
\solution
%\input{exemplar/11/16/3/31/main.tex}
\item A card is selected from a pack of 52 cards\\
\begin{enumerate}[label=(\alph*)]
\item How many points are there in the sample space?
\item Calculate the probability that the cards is an ace of spades.
\item Calculate the probability that the card is (i) an ace (ii)black card.\\
\end{enumerate}
%\input{ncert/11/16/3/4_1/Prob_4.tex}
\item In a non-leap year, the probability of having 53 tuesdays or 53 wednesdays is\\
\solution
%\input{exemplar/11/16/3/18/main.tex}
\item There are 1000 sealed envelopes in a box, 10 of them contain a cash prize of
Rs 100 each, 100 of them contain a cash prize of Rs 50 each and 200 of them
contain a cash prize of Rs 10 each and rest do not contain any cash prize. If they
are well shuffled and an envelope is picked up out, what is the probability that it
contains no cash prize?\\
\solution
%\input{exemplar/10/13/3/34/main.tex}
\item 
A die is thrown and a card is selected at random from a deck of 52 playing cards. The probability of getting an even number on the die and a spade card.\\
\solution
%\input{exemplar/12/13/3/78/main.tex}
\item
If 4-digit numbers greater than 5,000 are randomly formed from the digits 0, 1, 3, 5, and 7, what is the probability of forming a number divisible by 5 when:
\begin{enumerate}
    \item The digits are repeated?
    \item The repetition of digits is not allowed?
\end{enumerate}
\solution
%\input{ncert/11/16/4/9/main.tex}
\item Consider the probability space $\brak{\Omega, \mathcal{G}, P}$ where $\Omega = [0,2]$ and $\mathcal{G} = \cbrak{\phi, \Omega, [0,1], (1,2]}$. Let $X$ and $Y$ be two functions on $\Omega$ defined as
\begin{align*}
    X(\omega) = 
    \begin{cases}
        1 & \text{if }\omega \in [0, 1]\\
        2 & \text{if }\omega \in (1, 2]
    \end{cases}
\end{align*}
and
\begin{align*}
    Y(\omega) = 
    \begin{cases}
        2 & \text{if }\omega \in [0, 1.5]\\
        3 & \text{if }\omega \in (1.5, 2].
    \end{cases}
\end{align*}
Then which one of the following statements is true?
\begin{enumerate}
    \item [(A)] $X$ is a random variable with respect to $\mathcal{G}$, but $Y$ is not a random variable with respect to $\mathcal{G}$.
    \item [(B)] $Y$ is a random variable with respect to $\mathcal{G}$, but $X$ is not a random variable with respect to $\mathcal{G}$.
    \item [(C)] Neither $X$ nor $Y$ is a random variable with respect to $\mathcal{G}$.
    \item [(D)] Both $X$ and $Y$ are random variables with respect to $\mathcal{G}$.
\end{enumerate} \hfill (GATE ST 2023)\\
\solution
%\input{gate/ST/2023/14/main.tex}
	\item  A die is loaded in such a way that each odd number is twice as likely to occur as
each even number. Find $P(G)$, where $G$ is the event that a number greater than
3 occurs on a single roll of the die.
\\
\solution
		%\input{exemplar/11/16/3/5/main.tex}
	\item All the jacks, queens and kings are removed from a deck of 52 playing cards. The remaining cards are well shuffled and then one card is drawn at random. Giving ace a value 1 similar value for other cards, find the probability that the card has a value 
		\begin{enumerate}
			\item 7
			\item greater than 7
			\item less than 7
		\end{enumerate}
		%\input{exemplar/10/13/3/30/main.tex}
  \item A Lot consists of 48 mobile phones of which 42 are good, 3 have only minor defects and 3 have major defects.Varnika will buy a phone if it is good but the trader will only buy a mobile if it has no major defects. One phone is selected at random from the lot. What is the probability that it is
\begin{enumerate}
	\item acceptable to Varnika?
            \item acceptable to the trader?
\end{enumerate}
\solution
	%\input{exemplar/10/13/3/40/main.tex}
 \item A student says that if you throw a die, it will show up 1 or not 1. Therefore, the probability of getting 1 and the probability of getting 'not 1' each is equal to $\frac{1}{2}$. Is this correct? Give reasons.\\
 \solution
        %\input{exemplar/10/13/2/9/main.tex}
   \item Four candidates A, B, C, D have ap-
plied for the assignment to coach a school cricket
team. If A is twice as likely to be selected as B, and
B and C are given about the same chance of being
selected, while C is twice as likely to be selected
as D, what are the probabilities that
\begin{enumerate}
\item C will be selected?
\item A will not be selected?
\end{enumerate}
	%\input{exemplar/11/16/3/9/main.tex}
 \item A bag contain 24 balls of which $x$ balls are red, $2x$ are white and $3x$ are blue. A ball is selected at random, What is the probability that it is
\begin{enumerate}[label=\alph*)]
\item not red ?
\item white ?
\end{enumerate}
%\input{exemplar/10/13/3/41/main.tex}
If the letters of the word ASSASSINATION are arranged at random. Find the Probability that
\begin{enumerate}[label=(\alph*)]
\item Four $S's$ come consecutively in the word
\item Two  $I's$ and two $N's$ come together
\item All $A's$ are not coming together
\item No two $A's$ are coming together
\end{enumerate}
%\input{exemplar/11/16/3/14/main.tex}
	\item One urn contains two black balls (labelled B1 and B2) and one white ball. A
	second urn contains one black ball and two white balls (labelled W1 and W2).
	Suppose the following experiment is performed. One of the two urns is chosen
	at random. Next a ball is randomly chosen from the urn. Then a second ball is
	chosen at random from the same urn without replacing the first ball.
	
	\begin{enumerate}
	\item What is the probability that two black balls are chosen?
	
	\item What is the probability that two balls of opposite colour are chosen?
	\end{enumerate}
	\solution
	%\input{exemplar/11/16/3/12/main1.tex}
\end{enumerate}

		\item A box of oranges is inspected by examining three randomly selected oranges drawn without replacement. If all the three oranges are good, the box is approved for sale, otherwise, it is rejected. Find the probability that a box containing 15 oranges out of which 12 are good and 3 are bad ones will be approved for sale.
		\label{ncert/12/13/2/3/defs.tex}
		\item Two balls are drawn at random with replacement from a box containing 10 black and 8 red balls. Find the probability that
		\label{ncert/12/13/2/12}
\begin{enumerate}
\item both balls are red.
\item first ball is black and second is red.
\item one of them is black and other is red.
\end{enumerate}

\item In a hostel, 60\% of the students read Hindi newspaper, 40\% read English newspaper and 20\% read both Hindi and English newspapers. A student is selected at random.
		\label{ncert/12/13/2/15}
\begin{enumerate}
\item Find the probability that she reads neither Hindi nor English newspapers.
\item If she reads Hindi newspaper, find the probability that she reads English newspaper.
\item If she reads English newspaper, find the probability that she reads Hindi newspaper.\\
\end{enumerate}
\item The probability of obtaining an even prime number on each die, when a pair of dice is rolled is 
\begin{enumerate}
    \item $0$ 
    
    \item $\frac{1}{3}$ 
    
    \item $\frac{1}{12}$ 
    
    \item $\frac{1}{36}$ 
\end{enumerate}
\solution
		%\begin{enumerate}[label=\thesection.\arabic*,ref=\thesection.\theenumi]
	\item One card is drawn from a well-shuffled deck of 52 cards. Find the probability of getting
\begin{enumerate}
\item A king of red colour 
\item A face card 
\item A red face card
\item The jack of hearts
\item A spade
\item The queen of diamonds

\end{enumerate}
\solution
		%\input{ncert/10/15/1/14/main.tex}
	\item Five cards—the ten, jack, queen, king and ace of diamonds, are well-shuffled with their face downwards. One card is then picked up at random.
\begin{enumerate}
\item
What is the probability that the card is the queen? 
\item
If the queen is drawn and put aside, what is the probability that the second card picked up is (a) an ace? (b) a queen?\\
\end{enumerate}
\solution
		%\input{ncert/10/15/1/15/defs.tex}
	\item A bag contains $5$ red balls and some blue balls. If the probability of drawing a blue ball is double that if a red ball, determine the number of blue balls in the bag. 
		\\
\solution
		%\input{ncert/10/15/2/3/defs.tex}
	\item A card is selected from a pack of 52 cards.
 \begin{enumerate}[label=(\alph*)] 
                 \item How many points are there in the sample space?
                 \item Calculate the probability that the card is an ace of spades.
                 \item Calculate the probability that the card is (i) an ace and (ii) black card.
 \end{enumerate}
\solution
		%\input{ncert/11/16/3/4/main.tex}
\item Four cards are drawn from a well-shuffled deck of 52 cards. What is the probability of obtaining 3 diamonds and one spade.
\\
\solution
		%\input{ncert/11/16/4/2/defs.tex}
\item In a certain lottery 10,000 tickets are sold and ten equal prizes are awarded. What is the probability of not getting a prize if you buy (a) one ticket (b) two tickets (c) 10 tickets ?	
\\
\solution
		%\input{ncert/11/16/4/4/defs.tex}
		%
\item 
Out of 100 students, two sections of 40 and 60 are formed. If you and your friend are among the 100 students, what is the probability that
\begin{enumerate}
\item you both enter the same section?
\item you both enter the different sections?
\end{enumerate}
\solution
		%\input{ncert/11/16/4/5/defs.tex}
	\item 
The number lock of a suitcase has 4 wheels each labelled with ten digits i.e. from 0 to 9.The lock opens with a sequence of four digits with no repeats.What is the probability of a person getting the right sequence to open the suitcase.
\\
\solution
		%\input{ncert/11/16/4/10/defs.tex}
		%
\item 
Two cards are drawn at random and without replacement from a pack of 52 playing cards. Find the probability that both the cards are black.
\\
\solution
		%\input{ncert/12/13/2/2/defs.tex}
		\item A box of oranges is inspected by examining three randomly selected oranges drawn without replacement. If all the three oranges are good, the box is approved for sale, otherwise, it is rejected. Find the probability that a box containing 15 oranges out of which 12 are good and 3 are bad ones will be approved for sale.
		\label{ncert/12/13/2/3/defs.tex}
		\item Two balls are drawn at random with replacement from a box containing 10 black and 8 red balls. Find the probability that
		\label{ncert/12/13/2/12}
\begin{enumerate}
\item both balls are red.
\item first ball is black and second is red.
\item one of them is black and other is red.
\end{enumerate}

\item In a hostel, 60\% of the students read Hindi newspaper, 40\% read English newspaper and 20\% read both Hindi and English newspapers. A student is selected at random.
		\label{ncert/12/13/2/15}
\begin{enumerate}
\item Find the probability that she reads neither Hindi nor English newspapers.
\item If she reads Hindi newspaper, find the probability that she reads English newspaper.
\item If she reads English newspaper, find the probability that she reads Hindi newspaper.\\
\end{enumerate}
\item The probability of obtaining an even prime number on each die, when a pair of dice is rolled is 
\begin{enumerate}
    \item $0$ 
    
    \item $\frac{1}{3}$ 
    
    \item $\frac{1}{12}$ 
    
    \item $\frac{1}{36}$ 
\end{enumerate}
\solution
		%\input{ncert/12/13/2/17/defs.tex}
	\item A bag contains 4 red and 4 black balls, another bag contains 2 red and 6 black balls. One of the two bags is selected at random and a ball is drawn from the bag which is found to be red. Find the probability that the ball is drawn from the first bag.
\\
\solution
		%\input{ncert/12/13/3/2/main.tex}
  \item
  Cards with numbers 2 to 101 are placed in a box. A card is selected at random.Find the probability that the card has
\begin{enumerate}[label=(\roman*)]
	\item an even number 
	\item a square number
\end{enumerate}
\solution
%\input{exemplar/10/13/3/32/main.tex}
\item
The king, queen and jack of clubs are removed from a deck of 52 playing cards and then well shuffled. Now one card is drawn at random from the remaining cards.  Determine the probability that the card is
\begin{enumerate}[label=(\roman*)]
\item a club
\item 10 of hearts
\end{enumerate}
\solution
%\input{exemplar/10/13/3/29/main.tex}
\item A team of medical students doing their internship have to assist during surgeries
at a city hospital. The probabilities of surgeries rated as very complex, complex,
routine, simple or very simple are respectively, 0.15, 0.20, 0.31, 0.26, .08. Find
the probabilities that a particular surgery will be rated
\begin{enumerate}
	\item complex or very complex;
	\item neither very complex nor very simple;
	\item routine or complex
	\item routine or simple
\end{enumerate}
\solution
%\input{exemplar/11/16/3/8(1)/main.tex}
\item A card is selected from a pack of 52 cards.
\begin{enumerate}[label=(\alph*)]
    \item How many points are there in the sample space?
    \item Calculate the probability that the card is an ace of spades.
    \item Calculate the probability that the card is (i) an ace and (ii) black card.
\end{enumerate}
\solution
%\input{exemplar/11/16/3/4/main2.tex}
\item The probability that a non leap year selected at random will contain 53 sundays.
\\
\solution
%\input{exemplar/10/13/1/19/main.tex}
\item One of the four persons John, Rita, Aslam or Gurpreet will be promoted next
month. Consequently the sample space consists of four elementary outcomes
S = {John promoted, Rita promoted, Aslam promoted, Gurpreet promoted}
You are told that the chances of John’s promotion is same as that of Gurpreet,
Rita’s chances of promotion are twice as likely as Johns. Aslam’s chances are
four times that of John.
\begin{enumerate}
	\item Determine
	\begin{enumerate}
		\item P (John promoted)
		\item P (Rita promoted)
		\item P (Aslam promoted)
		\item P (Gurpreet promoted)
	\end{enumerate}
	\item If A = {John promoted or Gurpreet promoted}, find P (A).
\end{enumerate}
\solution
%\input{exemplar/11/16/3/10/main.tex}
\item A card is drawn from a deck of 52 cards. Find the probability of getting a king or a heart or a red card.\\
\solution
%\input{exemplar/11/16/3/15/main.tex}
\item The probability that a student will pass his examination is 0.73, the probability of
the student getting a compartment is 0.13, and the probability that the student will
either pass or get compartment is 0.96. State True or False.\\
\solution
%\input{exemplar/11/16/3/31/main.tex}
\item A card is selected from a pack of 52 cards\\
\begin{enumerate}[label=(\alph*)]
\item How many points are there in the sample space?
\item Calculate the probability that the cards is an ace of spades.
\item Calculate the probability that the card is (i) an ace (ii)black card.\\
\end{enumerate}
%\input{ncert/11/16/3/4_1/Prob_4.tex}
\item In a non-leap year, the probability of having 53 tuesdays or 53 wednesdays is\\
\solution
%\input{exemplar/11/16/3/18/main.tex}
\item There are 1000 sealed envelopes in a box, 10 of them contain a cash prize of
Rs 100 each, 100 of them contain a cash prize of Rs 50 each and 200 of them
contain a cash prize of Rs 10 each and rest do not contain any cash prize. If they
are well shuffled and an envelope is picked up out, what is the probability that it
contains no cash prize?\\
\solution
%\input{exemplar/10/13/3/34/main.tex}
\item 
A die is thrown and a card is selected at random from a deck of 52 playing cards. The probability of getting an even number on the die and a spade card.\\
\solution
%\input{exemplar/12/13/3/78/main.tex}
\item
If 4-digit numbers greater than 5,000 are randomly formed from the digits 0, 1, 3, 5, and 7, what is the probability of forming a number divisible by 5 when:
\begin{enumerate}
    \item The digits are repeated?
    \item The repetition of digits is not allowed?
\end{enumerate}
\solution
%\input{ncert/11/16/4/9/main.tex}
\item Consider the probability space $\brak{\Omega, \mathcal{G}, P}$ where $\Omega = [0,2]$ and $\mathcal{G} = \cbrak{\phi, \Omega, [0,1], (1,2]}$. Let $X$ and $Y$ be two functions on $\Omega$ defined as
\begin{align*}
    X(\omega) = 
    \begin{cases}
        1 & \text{if }\omega \in [0, 1]\\
        2 & \text{if }\omega \in (1, 2]
    \end{cases}
\end{align*}
and
\begin{align*}
    Y(\omega) = 
    \begin{cases}
        2 & \text{if }\omega \in [0, 1.5]\\
        3 & \text{if }\omega \in (1.5, 2].
    \end{cases}
\end{align*}
Then which one of the following statements is true?
\begin{enumerate}
    \item [(A)] $X$ is a random variable with respect to $\mathcal{G}$, but $Y$ is not a random variable with respect to $\mathcal{G}$.
    \item [(B)] $Y$ is a random variable with respect to $\mathcal{G}$, but $X$ is not a random variable with respect to $\mathcal{G}$.
    \item [(C)] Neither $X$ nor $Y$ is a random variable with respect to $\mathcal{G}$.
    \item [(D)] Both $X$ and $Y$ are random variables with respect to $\mathcal{G}$.
\end{enumerate} \hfill (GATE ST 2023)\\
\solution
%\input{gate/ST/2023/14/main.tex}
	\item  A die is loaded in such a way that each odd number is twice as likely to occur as
each even number. Find $P(G)$, where $G$ is the event that a number greater than
3 occurs on a single roll of the die.
\\
\solution
		%\input{exemplar/11/16/3/5/main.tex}
	\item All the jacks, queens and kings are removed from a deck of 52 playing cards. The remaining cards are well shuffled and then one card is drawn at random. Giving ace a value 1 similar value for other cards, find the probability that the card has a value 
		\begin{enumerate}
			\item 7
			\item greater than 7
			\item less than 7
		\end{enumerate}
		%\input{exemplar/10/13/3/30/main.tex}
  \item A Lot consists of 48 mobile phones of which 42 are good, 3 have only minor defects and 3 have major defects.Varnika will buy a phone if it is good but the trader will only buy a mobile if it has no major defects. One phone is selected at random from the lot. What is the probability that it is
\begin{enumerate}
	\item acceptable to Varnika?
            \item acceptable to the trader?
\end{enumerate}
\solution
	%\input{exemplar/10/13/3/40/main.tex}
 \item A student says that if you throw a die, it will show up 1 or not 1. Therefore, the probability of getting 1 and the probability of getting 'not 1' each is equal to $\frac{1}{2}$. Is this correct? Give reasons.\\
 \solution
        %\input{exemplar/10/13/2/9/main.tex}
   \item Four candidates A, B, C, D have ap-
plied for the assignment to coach a school cricket
team. If A is twice as likely to be selected as B, and
B and C are given about the same chance of being
selected, while C is twice as likely to be selected
as D, what are the probabilities that
\begin{enumerate}
\item C will be selected?
\item A will not be selected?
\end{enumerate}
	%\input{exemplar/11/16/3/9/main.tex}
 \item A bag contain 24 balls of which $x$ balls are red, $2x$ are white and $3x$ are blue. A ball is selected at random, What is the probability that it is
\begin{enumerate}[label=\alph*)]
\item not red ?
\item white ?
\end{enumerate}
%\input{exemplar/10/13/3/41/main.tex}
If the letters of the word ASSASSINATION are arranged at random. Find the Probability that
\begin{enumerate}[label=(\alph*)]
\item Four $S's$ come consecutively in the word
\item Two  $I's$ and two $N's$ come together
\item All $A's$ are not coming together
\item No two $A's$ are coming together
\end{enumerate}
%\input{exemplar/11/16/3/14/main.tex}
	\item One urn contains two black balls (labelled B1 and B2) and one white ball. A
	second urn contains one black ball and two white balls (labelled W1 and W2).
	Suppose the following experiment is performed. One of the two urns is chosen
	at random. Next a ball is randomly chosen from the urn. Then a second ball is
	chosen at random from the same urn without replacing the first ball.
	
	\begin{enumerate}
	\item What is the probability that two black balls are chosen?
	
	\item What is the probability that two balls of opposite colour are chosen?
	\end{enumerate}
	\solution
	%\input{exemplar/11/16/3/12/main1.tex}
\end{enumerate}

	\item A bag contains 4 red and 4 black balls, another bag contains 2 red and 6 black balls. One of the two bags is selected at random and a ball is drawn from the bag which is found to be red. Find the probability that the ball is drawn from the first bag.
\\
\solution
		%\begin{table}[H]
	\centering
\begin{tabular}{|c|c|c|}
\hline
Random variable &Value &Definition\\ \hline
\multirow{3}{*}{X} &0 &Slips of Rs 1\\
&1 &Slips of Rs 5\\
&2 &Slips of Rs 13\\ \hline
\multirow{2}{*}{Y} &0 &Box A\\
&1 &Box B\\\hline
\end{tabular}
\caption{}
\label{tab:Distribution}
\end{table}
See \tabref{tab:Distribution}.
\begin{align}
p_{Y}\brak{k}= \begin{cases} 
      \frac{1}{3} & {k=0} \\
      \frac{2}{3 }& {k=1} 
   \end{cases}
   \\
p_{Y|X}\brak{0|0} = \frac{19}{25}\, 
p_{Y|X}\brak{0|1} = \frac{6}{25}\,
p_{Y|X}\brak{1|0} = \frac{45}{50}\,
p_{Y|X}\brak{1|2} = \frac{5}{50}
\end{align}
The desired probability is the probability that a slip drawn at random is marked other than Rs 1,
\begin{align}
&=1-p_X\brak{0}\\
&= p_X(1) + p_X(2)
\end{align}
Using Bayes theorem,
\begin{align}
&= p_Y\brak{0} \times \pr{Y=0 | X=1} + p_Y\brak{1} \times \pr{Y=1|X=2}\\
&=\frac{1}{3} \times \frac{6}{25} + \frac{2}{3} \times \frac{5}{50}\\
&=\frac{11}{75}
\end{align}

\newpage

%\tableofcontents

\bigskip

\renewcommand{\thefigure}{\theenumi}
\renewcommand{\thetable}{\theenumi}
%\renewcommand{\theequation}{\theenumi}

%\begin{abstract}
%%\boldmath
%In this letter, an algorithm for evaluating the exact analytical bit error rate  (BER)  for the piecewise linear (PL) combiner for  multiple relays is presented. Previous results were available only for upto three relays. The algorithm is unique in the sense that  the actual mathematical expressions, that are prohibitively large, need not be explicitly obtained. The diversity gain due to multiple relays is shown through plots of the analytical BER, well supported by simulations. 
%
%\end{abstract}
% IEEEtran.cls defaults to using nonbold math in the Abstract.
% This preserves the distinction between vectors and scalars. However,
% if the journal you are submitting to favors bold math in the abstract,
% then you can use LaTeX's standard command \boldmath at the very start
% of the abstract to achieve this. Many IEEE journals frown on math
% in the abstract anyway.

% Note that keywords are not normally used for peerreview papers.
%\begin{IEEEkeywords}
%Cooperative diversity, decode and forward, piecewise linear
%\end{IEEEkeywords}



% For peer review papers, you can put extra information on the cover
% page as needed:
% \ifCLASSOPTIONpeerreview
% \begin{center} \bfseries EDICS Category: 3-BBND \end{center}
% \fi
%
% For peerreview papers, this IEEEtran command inserts a page break and
% creates the second title. It will be ignored for other modes.
%\IEEEpeerreviewmaketitle




  \item
  Cards with numbers 2 to 101 are placed in a box. A card is selected at random.Find the probability that the card has
\begin{enumerate}[label=(\roman*)]
	\item an even number 
	\item a square number
\end{enumerate}
\solution
%\begin{table}[H]
	\centering
\begin{tabular}{|c|c|c|}
\hline
Random variable &Value &Definition\\ \hline
\multirow{3}{*}{X} &0 &Slips of Rs 1\\
&1 &Slips of Rs 5\\
&2 &Slips of Rs 13\\ \hline
\multirow{2}{*}{Y} &0 &Box A\\
&1 &Box B\\\hline
\end{tabular}
\caption{}
\label{tab:Distribution}
\end{table}
See \tabref{tab:Distribution}.
\begin{align}
p_{Y}\brak{k}= \begin{cases} 
      \frac{1}{3} & {k=0} \\
      \frac{2}{3 }& {k=1} 
   \end{cases}
   \\
p_{Y|X}\brak{0|0} = \frac{19}{25}\, 
p_{Y|X}\brak{0|1} = \frac{6}{25}\,
p_{Y|X}\brak{1|0} = \frac{45}{50}\,
p_{Y|X}\brak{1|2} = \frac{5}{50}
\end{align}
The desired probability is the probability that a slip drawn at random is marked other than Rs 1,
\begin{align}
&=1-p_X\brak{0}\\
&= p_X(1) + p_X(2)
\end{align}
Using Bayes theorem,
\begin{align}
&= p_Y\brak{0} \times \pr{Y=0 | X=1} + p_Y\brak{1} \times \pr{Y=1|X=2}\\
&=\frac{1}{3} \times \frac{6}{25} + \frac{2}{3} \times \frac{5}{50}\\
&=\frac{11}{75}
\end{align}

\newpage

%\tableofcontents

\bigskip

\renewcommand{\thefigure}{\theenumi}
\renewcommand{\thetable}{\theenumi}
%\renewcommand{\theequation}{\theenumi}

%\begin{abstract}
%%\boldmath
%In this letter, an algorithm for evaluating the exact analytical bit error rate  (BER)  for the piecewise linear (PL) combiner for  multiple relays is presented. Previous results were available only for upto three relays. The algorithm is unique in the sense that  the actual mathematical expressions, that are prohibitively large, need not be explicitly obtained. The diversity gain due to multiple relays is shown through plots of the analytical BER, well supported by simulations. 
%
%\end{abstract}
% IEEEtran.cls defaults to using nonbold math in the Abstract.
% This preserves the distinction between vectors and scalars. However,
% if the journal you are submitting to favors bold math in the abstract,
% then you can use LaTeX's standard command \boldmath at the very start
% of the abstract to achieve this. Many IEEE journals frown on math
% in the abstract anyway.

% Note that keywords are not normally used for peerreview papers.
%\begin{IEEEkeywords}
%Cooperative diversity, decode and forward, piecewise linear
%\end{IEEEkeywords}



% For peer review papers, you can put extra information on the cover
% page as needed:
% \ifCLASSOPTIONpeerreview
% \begin{center} \bfseries EDICS Category: 3-BBND \end{center}
% \fi
%
% For peerreview papers, this IEEEtran command inserts a page break and
% creates the second title. It will be ignored for other modes.
%\IEEEpeerreviewmaketitle




\item
The king, queen and jack of clubs are removed from a deck of 52 playing cards and then well shuffled. Now one card is drawn at random from the remaining cards.  Determine the probability that the card is
\begin{enumerate}[label=(\roman*)]
\item a club
\item 10 of hearts
\end{enumerate}
\solution
%\begin{table}[H]
	\centering
\begin{tabular}{|c|c|c|}
\hline
Random variable &Value &Definition\\ \hline
\multirow{3}{*}{X} &0 &Slips of Rs 1\\
&1 &Slips of Rs 5\\
&2 &Slips of Rs 13\\ \hline
\multirow{2}{*}{Y} &0 &Box A\\
&1 &Box B\\\hline
\end{tabular}
\caption{}
\label{tab:Distribution}
\end{table}
See \tabref{tab:Distribution}.
\begin{align}
p_{Y}\brak{k}= \begin{cases} 
      \frac{1}{3} & {k=0} \\
      \frac{2}{3 }& {k=1} 
   \end{cases}
   \\
p_{Y|X}\brak{0|0} = \frac{19}{25}\, 
p_{Y|X}\brak{0|1} = \frac{6}{25}\,
p_{Y|X}\brak{1|0} = \frac{45}{50}\,
p_{Y|X}\brak{1|2} = \frac{5}{50}
\end{align}
The desired probability is the probability that a slip drawn at random is marked other than Rs 1,
\begin{align}
&=1-p_X\brak{0}\\
&= p_X(1) + p_X(2)
\end{align}
Using Bayes theorem,
\begin{align}
&= p_Y\brak{0} \times \pr{Y=0 | X=1} + p_Y\brak{1} \times \pr{Y=1|X=2}\\
&=\frac{1}{3} \times \frac{6}{25} + \frac{2}{3} \times \frac{5}{50}\\
&=\frac{11}{75}
\end{align}

\newpage

%\tableofcontents

\bigskip

\renewcommand{\thefigure}{\theenumi}
\renewcommand{\thetable}{\theenumi}
%\renewcommand{\theequation}{\theenumi}

%\begin{abstract}
%%\boldmath
%In this letter, an algorithm for evaluating the exact analytical bit error rate  (BER)  for the piecewise linear (PL) combiner for  multiple relays is presented. Previous results were available only for upto three relays. The algorithm is unique in the sense that  the actual mathematical expressions, that are prohibitively large, need not be explicitly obtained. The diversity gain due to multiple relays is shown through plots of the analytical BER, well supported by simulations. 
%
%\end{abstract}
% IEEEtran.cls defaults to using nonbold math in the Abstract.
% This preserves the distinction between vectors and scalars. However,
% if the journal you are submitting to favors bold math in the abstract,
% then you can use LaTeX's standard command \boldmath at the very start
% of the abstract to achieve this. Many IEEE journals frown on math
% in the abstract anyway.

% Note that keywords are not normally used for peerreview papers.
%\begin{IEEEkeywords}
%Cooperative diversity, decode and forward, piecewise linear
%\end{IEEEkeywords}



% For peer review papers, you can put extra information on the cover
% page as needed:
% \ifCLASSOPTIONpeerreview
% \begin{center} \bfseries EDICS Category: 3-BBND \end{center}
% \fi
%
% For peerreview papers, this IEEEtran command inserts a page break and
% creates the second title. It will be ignored for other modes.
%\IEEEpeerreviewmaketitle




\item A team of medical students doing their internship have to assist during surgeries
at a city hospital. The probabilities of surgeries rated as very complex, complex,
routine, simple or very simple are respectively, 0.15, 0.20, 0.31, 0.26, .08. Find
the probabilities that a particular surgery will be rated
\begin{enumerate}
	\item complex or very complex;
	\item neither very complex nor very simple;
	\item routine or complex
	\item routine or simple
\end{enumerate}
\solution
%\begin{table}[H]
	\centering
\begin{tabular}{|c|c|c|}
\hline
Random variable &Value &Definition\\ \hline
\multirow{3}{*}{X} &0 &Slips of Rs 1\\
&1 &Slips of Rs 5\\
&2 &Slips of Rs 13\\ \hline
\multirow{2}{*}{Y} &0 &Box A\\
&1 &Box B\\\hline
\end{tabular}
\caption{}
\label{tab:Distribution}
\end{table}
See \tabref{tab:Distribution}.
\begin{align}
p_{Y}\brak{k}= \begin{cases} 
      \frac{1}{3} & {k=0} \\
      \frac{2}{3 }& {k=1} 
   \end{cases}
   \\
p_{Y|X}\brak{0|0} = \frac{19}{25}\, 
p_{Y|X}\brak{0|1} = \frac{6}{25}\,
p_{Y|X}\brak{1|0} = \frac{45}{50}\,
p_{Y|X}\brak{1|2} = \frac{5}{50}
\end{align}
The desired probability is the probability that a slip drawn at random is marked other than Rs 1,
\begin{align}
&=1-p_X\brak{0}\\
&= p_X(1) + p_X(2)
\end{align}
Using Bayes theorem,
\begin{align}
&= p_Y\brak{0} \times \pr{Y=0 | X=1} + p_Y\brak{1} \times \pr{Y=1|X=2}\\
&=\frac{1}{3} \times \frac{6}{25} + \frac{2}{3} \times \frac{5}{50}\\
&=\frac{11}{75}
\end{align}

\newpage

%\tableofcontents

\bigskip

\renewcommand{\thefigure}{\theenumi}
\renewcommand{\thetable}{\theenumi}
%\renewcommand{\theequation}{\theenumi}

%\begin{abstract}
%%\boldmath
%In this letter, an algorithm for evaluating the exact analytical bit error rate  (BER)  for the piecewise linear (PL) combiner for  multiple relays is presented. Previous results were available only for upto three relays. The algorithm is unique in the sense that  the actual mathematical expressions, that are prohibitively large, need not be explicitly obtained. The diversity gain due to multiple relays is shown through plots of the analytical BER, well supported by simulations. 
%
%\end{abstract}
% IEEEtran.cls defaults to using nonbold math in the Abstract.
% This preserves the distinction between vectors and scalars. However,
% if the journal you are submitting to favors bold math in the abstract,
% then you can use LaTeX's standard command \boldmath at the very start
% of the abstract to achieve this. Many IEEE journals frown on math
% in the abstract anyway.

% Note that keywords are not normally used for peerreview papers.
%\begin{IEEEkeywords}
%Cooperative diversity, decode and forward, piecewise linear
%\end{IEEEkeywords}



% For peer review papers, you can put extra information on the cover
% page as needed:
% \ifCLASSOPTIONpeerreview
% \begin{center} \bfseries EDICS Category: 3-BBND \end{center}
% \fi
%
% For peerreview papers, this IEEEtran command inserts a page break and
% creates the second title. It will be ignored for other modes.
%\IEEEpeerreviewmaketitle




\item A card is selected from a pack of 52 cards.
\begin{enumerate}[label=(\alph*)]
    \item How many points are there in the sample space?
    \item Calculate the probability that the card is an ace of spades.
    \item Calculate the probability that the card is (i) an ace and (ii) black card.
\end{enumerate}
\solution
%Let $X$ be an bernoulli rv defined as in \tabref{tab:exemplar/11/16/3/26}.  Then, 
\begin{equation}
    p =
        \frac{4}{11} 
\end{equation}
\begin{table}[H]
	\centering
	\input{exemplar/11/16/3/26/tables/Table2.tex}
	\caption{}
        \label{tab:exemplar/11/16/3/26}
\end{table}

\item The probability that a non leap year selected at random will contain 53 sundays.
\\
\solution
%\begin{table}[H]
	\centering
\begin{tabular}{|c|c|c|}
\hline
Random variable &Value &Definition\\ \hline
\multirow{3}{*}{X} &0 &Slips of Rs 1\\
&1 &Slips of Rs 5\\
&2 &Slips of Rs 13\\ \hline
\multirow{2}{*}{Y} &0 &Box A\\
&1 &Box B\\\hline
\end{tabular}
\caption{}
\label{tab:Distribution}
\end{table}
See \tabref{tab:Distribution}.
\begin{align}
p_{Y}\brak{k}= \begin{cases} 
      \frac{1}{3} & {k=0} \\
      \frac{2}{3 }& {k=1} 
   \end{cases}
   \\
p_{Y|X}\brak{0|0} = \frac{19}{25}\, 
p_{Y|X}\brak{0|1} = \frac{6}{25}\,
p_{Y|X}\brak{1|0} = \frac{45}{50}\,
p_{Y|X}\brak{1|2} = \frac{5}{50}
\end{align}
The desired probability is the probability that a slip drawn at random is marked other than Rs 1,
\begin{align}
&=1-p_X\brak{0}\\
&= p_X(1) + p_X(2)
\end{align}
Using Bayes theorem,
\begin{align}
&= p_Y\brak{0} \times \pr{Y=0 | X=1} + p_Y\brak{1} \times \pr{Y=1|X=2}\\
&=\frac{1}{3} \times \frac{6}{25} + \frac{2}{3} \times \frac{5}{50}\\
&=\frac{11}{75}
\end{align}

\newpage

%\tableofcontents

\bigskip

\renewcommand{\thefigure}{\theenumi}
\renewcommand{\thetable}{\theenumi}
%\renewcommand{\theequation}{\theenumi}

%\begin{abstract}
%%\boldmath
%In this letter, an algorithm for evaluating the exact analytical bit error rate  (BER)  for the piecewise linear (PL) combiner for  multiple relays is presented. Previous results were available only for upto three relays. The algorithm is unique in the sense that  the actual mathematical expressions, that are prohibitively large, need not be explicitly obtained. The diversity gain due to multiple relays is shown through plots of the analytical BER, well supported by simulations. 
%
%\end{abstract}
% IEEEtran.cls defaults to using nonbold math in the Abstract.
% This preserves the distinction between vectors and scalars. However,
% if the journal you are submitting to favors bold math in the abstract,
% then you can use LaTeX's standard command \boldmath at the very start
% of the abstract to achieve this. Many IEEE journals frown on math
% in the abstract anyway.

% Note that keywords are not normally used for peerreview papers.
%\begin{IEEEkeywords}
%Cooperative diversity, decode and forward, piecewise linear
%\end{IEEEkeywords}



% For peer review papers, you can put extra information on the cover
% page as needed:
% \ifCLASSOPTIONpeerreview
% \begin{center} \bfseries EDICS Category: 3-BBND \end{center}
% \fi
%
% For peerreview papers, this IEEEtran command inserts a page break and
% creates the second title. It will be ignored for other modes.
%\IEEEpeerreviewmaketitle




\item One of the four persons John, Rita, Aslam or Gurpreet will be promoted next
month. Consequently the sample space consists of four elementary outcomes
S = {John promoted, Rita promoted, Aslam promoted, Gurpreet promoted}
You are told that the chances of John’s promotion is same as that of Gurpreet,
Rita’s chances of promotion are twice as likely as Johns. Aslam’s chances are
four times that of John.
\begin{enumerate}
	\item Determine
	\begin{enumerate}
		\item P (John promoted)
		\item P (Rita promoted)
		\item P (Aslam promoted)
		\item P (Gurpreet promoted)
	\end{enumerate}
	\item If A = {John promoted or Gurpreet promoted}, find P (A).
\end{enumerate}
\solution
%\begin{table}[H]
	\centering
\begin{tabular}{|c|c|c|}
\hline
Random variable &Value &Definition\\ \hline
\multirow{3}{*}{X} &0 &Slips of Rs 1\\
&1 &Slips of Rs 5\\
&2 &Slips of Rs 13\\ \hline
\multirow{2}{*}{Y} &0 &Box A\\
&1 &Box B\\\hline
\end{tabular}
\caption{}
\label{tab:Distribution}
\end{table}
See \tabref{tab:Distribution}.
\begin{align}
p_{Y}\brak{k}= \begin{cases} 
      \frac{1}{3} & {k=0} \\
      \frac{2}{3 }& {k=1} 
   \end{cases}
   \\
p_{Y|X}\brak{0|0} = \frac{19}{25}\, 
p_{Y|X}\brak{0|1} = \frac{6}{25}\,
p_{Y|X}\brak{1|0} = \frac{45}{50}\,
p_{Y|X}\brak{1|2} = \frac{5}{50}
\end{align}
The desired probability is the probability that a slip drawn at random is marked other than Rs 1,
\begin{align}
&=1-p_X\brak{0}\\
&= p_X(1) + p_X(2)
\end{align}
Using Bayes theorem,
\begin{align}
&= p_Y\brak{0} \times \pr{Y=0 | X=1} + p_Y\brak{1} \times \pr{Y=1|X=2}\\
&=\frac{1}{3} \times \frac{6}{25} + \frac{2}{3} \times \frac{5}{50}\\
&=\frac{11}{75}
\end{align}

\newpage

%\tableofcontents

\bigskip

\renewcommand{\thefigure}{\theenumi}
\renewcommand{\thetable}{\theenumi}
%\renewcommand{\theequation}{\theenumi}

%\begin{abstract}
%%\boldmath
%In this letter, an algorithm for evaluating the exact analytical bit error rate  (BER)  for the piecewise linear (PL) combiner for  multiple relays is presented. Previous results were available only for upto three relays. The algorithm is unique in the sense that  the actual mathematical expressions, that are prohibitively large, need not be explicitly obtained. The diversity gain due to multiple relays is shown through plots of the analytical BER, well supported by simulations. 
%
%\end{abstract}
% IEEEtran.cls defaults to using nonbold math in the Abstract.
% This preserves the distinction between vectors and scalars. However,
% if the journal you are submitting to favors bold math in the abstract,
% then you can use LaTeX's standard command \boldmath at the very start
% of the abstract to achieve this. Many IEEE journals frown on math
% in the abstract anyway.

% Note that keywords are not normally used for peerreview papers.
%\begin{IEEEkeywords}
%Cooperative diversity, decode and forward, piecewise linear
%\end{IEEEkeywords}



% For peer review papers, you can put extra information on the cover
% page as needed:
% \ifCLASSOPTIONpeerreview
% \begin{center} \bfseries EDICS Category: 3-BBND \end{center}
% \fi
%
% For peerreview papers, this IEEEtran command inserts a page break and
% creates the second title. It will be ignored for other modes.
%\IEEEpeerreviewmaketitle




\item A card is drawn from a deck of 52 cards. Find the probability of getting a king or a heart or a red card.\\
\solution
%\begin{table}[H]
	\centering
\begin{tabular}{|c|c|c|}
\hline
Random variable &Value &Definition\\ \hline
\multirow{3}{*}{X} &0 &Slips of Rs 1\\
&1 &Slips of Rs 5\\
&2 &Slips of Rs 13\\ \hline
\multirow{2}{*}{Y} &0 &Box A\\
&1 &Box B\\\hline
\end{tabular}
\caption{}
\label{tab:Distribution}
\end{table}
See \tabref{tab:Distribution}.
\begin{align}
p_{Y}\brak{k}= \begin{cases} 
      \frac{1}{3} & {k=0} \\
      \frac{2}{3 }& {k=1} 
   \end{cases}
   \\
p_{Y|X}\brak{0|0} = \frac{19}{25}\, 
p_{Y|X}\brak{0|1} = \frac{6}{25}\,
p_{Y|X}\brak{1|0} = \frac{45}{50}\,
p_{Y|X}\brak{1|2} = \frac{5}{50}
\end{align}
The desired probability is the probability that a slip drawn at random is marked other than Rs 1,
\begin{align}
&=1-p_X\brak{0}\\
&= p_X(1) + p_X(2)
\end{align}
Using Bayes theorem,
\begin{align}
&= p_Y\brak{0} \times \pr{Y=0 | X=1} + p_Y\brak{1} \times \pr{Y=1|X=2}\\
&=\frac{1}{3} \times \frac{6}{25} + \frac{2}{3} \times \frac{5}{50}\\
&=\frac{11}{75}
\end{align}

\newpage

%\tableofcontents

\bigskip

\renewcommand{\thefigure}{\theenumi}
\renewcommand{\thetable}{\theenumi}
%\renewcommand{\theequation}{\theenumi}

%\begin{abstract}
%%\boldmath
%In this letter, an algorithm for evaluating the exact analytical bit error rate  (BER)  for the piecewise linear (PL) combiner for  multiple relays is presented. Previous results were available only for upto three relays. The algorithm is unique in the sense that  the actual mathematical expressions, that are prohibitively large, need not be explicitly obtained. The diversity gain due to multiple relays is shown through plots of the analytical BER, well supported by simulations. 
%
%\end{abstract}
% IEEEtran.cls defaults to using nonbold math in the Abstract.
% This preserves the distinction between vectors and scalars. However,
% if the journal you are submitting to favors bold math in the abstract,
% then you can use LaTeX's standard command \boldmath at the very start
% of the abstract to achieve this. Many IEEE journals frown on math
% in the abstract anyway.

% Note that keywords are not normally used for peerreview papers.
%\begin{IEEEkeywords}
%Cooperative diversity, decode and forward, piecewise linear
%\end{IEEEkeywords}



% For peer review papers, you can put extra information on the cover
% page as needed:
% \ifCLASSOPTIONpeerreview
% \begin{center} \bfseries EDICS Category: 3-BBND \end{center}
% \fi
%
% For peerreview papers, this IEEEtran command inserts a page break and
% creates the second title. It will be ignored for other modes.
%\IEEEpeerreviewmaketitle




\item The probability that a student will pass his examination is 0.73, the probability of
the student getting a compartment is 0.13, and the probability that the student will
either pass or get compartment is 0.96. State True or False.\\
\solution
%\begin{table}[H]
	\centering
\begin{tabular}{|c|c|c|}
\hline
Random variable &Value &Definition\\ \hline
\multirow{3}{*}{X} &0 &Slips of Rs 1\\
&1 &Slips of Rs 5\\
&2 &Slips of Rs 13\\ \hline
\multirow{2}{*}{Y} &0 &Box A\\
&1 &Box B\\\hline
\end{tabular}
\caption{}
\label{tab:Distribution}
\end{table}
See \tabref{tab:Distribution}.
\begin{align}
p_{Y}\brak{k}= \begin{cases} 
      \frac{1}{3} & {k=0} \\
      \frac{2}{3 }& {k=1} 
   \end{cases}
   \\
p_{Y|X}\brak{0|0} = \frac{19}{25}\, 
p_{Y|X}\brak{0|1} = \frac{6}{25}\,
p_{Y|X}\brak{1|0} = \frac{45}{50}\,
p_{Y|X}\brak{1|2} = \frac{5}{50}
\end{align}
The desired probability is the probability that a slip drawn at random is marked other than Rs 1,
\begin{align}
&=1-p_X\brak{0}\\
&= p_X(1) + p_X(2)
\end{align}
Using Bayes theorem,
\begin{align}
&= p_Y\brak{0} \times \pr{Y=0 | X=1} + p_Y\brak{1} \times \pr{Y=1|X=2}\\
&=\frac{1}{3} \times \frac{6}{25} + \frac{2}{3} \times \frac{5}{50}\\
&=\frac{11}{75}
\end{align}

\newpage

%\tableofcontents

\bigskip

\renewcommand{\thefigure}{\theenumi}
\renewcommand{\thetable}{\theenumi}
%\renewcommand{\theequation}{\theenumi}

%\begin{abstract}
%%\boldmath
%In this letter, an algorithm for evaluating the exact analytical bit error rate  (BER)  for the piecewise linear (PL) combiner for  multiple relays is presented. Previous results were available only for upto three relays. The algorithm is unique in the sense that  the actual mathematical expressions, that are prohibitively large, need not be explicitly obtained. The diversity gain due to multiple relays is shown through plots of the analytical BER, well supported by simulations. 
%
%\end{abstract}
% IEEEtran.cls defaults to using nonbold math in the Abstract.
% This preserves the distinction between vectors and scalars. However,
% if the journal you are submitting to favors bold math in the abstract,
% then you can use LaTeX's standard command \boldmath at the very start
% of the abstract to achieve this. Many IEEE journals frown on math
% in the abstract anyway.

% Note that keywords are not normally used for peerreview papers.
%\begin{IEEEkeywords}
%Cooperative diversity, decode and forward, piecewise linear
%\end{IEEEkeywords}



% For peer review papers, you can put extra information on the cover
% page as needed:
% \ifCLASSOPTIONpeerreview
% \begin{center} \bfseries EDICS Category: 3-BBND \end{center}
% \fi
%
% For peerreview papers, this IEEEtran command inserts a page break and
% creates the second title. It will be ignored for other modes.
%\IEEEpeerreviewmaketitle




\item A card is selected from a pack of 52 cards\\
\begin{enumerate}[label=(\alph*)]
\item How many points are there in the sample space?
\item Calculate the probability that the cards is an ace of spades.
\item Calculate the probability that the card is (i) an ace (ii)black card.\\
\end{enumerate}
%\input{ncert/11/16/3/4_1/Prob_4.tex}
\item In a non-leap year, the probability of having 53 tuesdays or 53 wednesdays is\\
\solution
%A non-leap year has a total of 365 days, and a week has 7 days.\\
So it can be expressed as 
\begin{align}
365\text{days} &=52\times 7+1 \text{day}
\end{align}
$\implies$ 52 tuesdays or wednesdays\\
Random variable X denotes the days of a week
\begin{align}
p_X\brak{k}&=\frac{1}{7}; \quad \brak{1<k<7}
\end{align}
So the probability of extra day being tuesday or wednesday is
\begin{align}
p_X\brak{3}+p_X\brak{4}&=\frac{1}{7}+\frac{1}{7}=\frac{2}{7}
\end{align}



\item There are 1000 sealed envelopes in a box, 10 of them contain a cash prize of
Rs 100 each, 100 of them contain a cash prize of Rs 50 each and 200 of them
contain a cash prize of Rs 10 each and rest do not contain any cash prize. If they
are well shuffled and an envelope is picked up out, what is the probability that it
contains no cash prize?\\
\solution
%\begin{table}[H]
	\centering
\begin{tabular}{|c|c|c|}
\hline
Random variable &Value &Definition\\ \hline
\multirow{3}{*}{X} &0 &Slips of Rs 1\\
&1 &Slips of Rs 5\\
&2 &Slips of Rs 13\\ \hline
\multirow{2}{*}{Y} &0 &Box A\\
&1 &Box B\\\hline
\end{tabular}
\caption{}
\label{tab:Distribution}
\end{table}
See \tabref{tab:Distribution}.
\begin{align}
p_{Y}\brak{k}= \begin{cases} 
      \frac{1}{3} & {k=0} \\
      \frac{2}{3 }& {k=1} 
   \end{cases}
   \\
p_{Y|X}\brak{0|0} = \frac{19}{25}\, 
p_{Y|X}\brak{0|1} = \frac{6}{25}\,
p_{Y|X}\brak{1|0} = \frac{45}{50}\,
p_{Y|X}\brak{1|2} = \frac{5}{50}
\end{align}
The desired probability is the probability that a slip drawn at random is marked other than Rs 1,
\begin{align}
&=1-p_X\brak{0}\\
&= p_X(1) + p_X(2)
\end{align}
Using Bayes theorem,
\begin{align}
&= p_Y\brak{0} \times \pr{Y=0 | X=1} + p_Y\brak{1} \times \pr{Y=1|X=2}\\
&=\frac{1}{3} \times \frac{6}{25} + \frac{2}{3} \times \frac{5}{50}\\
&=\frac{11}{75}
\end{align}

\newpage

%\tableofcontents

\bigskip

\renewcommand{\thefigure}{\theenumi}
\renewcommand{\thetable}{\theenumi}
%\renewcommand{\theequation}{\theenumi}

%\begin{abstract}
%%\boldmath
%In this letter, an algorithm for evaluating the exact analytical bit error rate  (BER)  for the piecewise linear (PL) combiner for  multiple relays is presented. Previous results were available only for upto three relays. The algorithm is unique in the sense that  the actual mathematical expressions, that are prohibitively large, need not be explicitly obtained. The diversity gain due to multiple relays is shown through plots of the analytical BER, well supported by simulations. 
%
%\end{abstract}
% IEEEtran.cls defaults to using nonbold math in the Abstract.
% This preserves the distinction between vectors and scalars. However,
% if the journal you are submitting to favors bold math in the abstract,
% then you can use LaTeX's standard command \boldmath at the very start
% of the abstract to achieve this. Many IEEE journals frown on math
% in the abstract anyway.

% Note that keywords are not normally used for peerreview papers.
%\begin{IEEEkeywords}
%Cooperative diversity, decode and forward, piecewise linear
%\end{IEEEkeywords}



% For peer review papers, you can put extra information on the cover
% page as needed:
% \ifCLASSOPTIONpeerreview
% \begin{center} \bfseries EDICS Category: 3-BBND \end{center}
% \fi
%
% For peerreview papers, this IEEEtran command inserts a page break and
% creates the second title. It will be ignored for other modes.
%\IEEEpeerreviewmaketitle




\item 
A die is thrown and a card is selected at random from a deck of 52 playing cards. The probability of getting an even number on the die and a spade card.\\
\solution
%\begin{table}[H]
	\centering
\begin{tabular}{|c|c|c|}
\hline
Random variable &Value &Definition\\ \hline
\multirow{3}{*}{X} &0 &Slips of Rs 1\\
&1 &Slips of Rs 5\\
&2 &Slips of Rs 13\\ \hline
\multirow{2}{*}{Y} &0 &Box A\\
&1 &Box B\\\hline
\end{tabular}
\caption{}
\label{tab:Distribution}
\end{table}
See \tabref{tab:Distribution}.
\begin{align}
p_{Y}\brak{k}= \begin{cases} 
      \frac{1}{3} & {k=0} \\
      \frac{2}{3 }& {k=1} 
   \end{cases}
   \\
p_{Y|X}\brak{0|0} = \frac{19}{25}\, 
p_{Y|X}\brak{0|1} = \frac{6}{25}\,
p_{Y|X}\brak{1|0} = \frac{45}{50}\,
p_{Y|X}\brak{1|2} = \frac{5}{50}
\end{align}
The desired probability is the probability that a slip drawn at random is marked other than Rs 1,
\begin{align}
&=1-p_X\brak{0}\\
&= p_X(1) + p_X(2)
\end{align}
Using Bayes theorem,
\begin{align}
&= p_Y\brak{0} \times \pr{Y=0 | X=1} + p_Y\brak{1} \times \pr{Y=1|X=2}\\
&=\frac{1}{3} \times \frac{6}{25} + \frac{2}{3} \times \frac{5}{50}\\
&=\frac{11}{75}
\end{align}

\newpage

%\tableofcontents

\bigskip

\renewcommand{\thefigure}{\theenumi}
\renewcommand{\thetable}{\theenumi}
%\renewcommand{\theequation}{\theenumi}

%\begin{abstract}
%%\boldmath
%In this letter, an algorithm for evaluating the exact analytical bit error rate  (BER)  for the piecewise linear (PL) combiner for  multiple relays is presented. Previous results were available only for upto three relays. The algorithm is unique in the sense that  the actual mathematical expressions, that are prohibitively large, need not be explicitly obtained. The diversity gain due to multiple relays is shown through plots of the analytical BER, well supported by simulations. 
%
%\end{abstract}
% IEEEtran.cls defaults to using nonbold math in the Abstract.
% This preserves the distinction between vectors and scalars. However,
% if the journal you are submitting to favors bold math in the abstract,
% then you can use LaTeX's standard command \boldmath at the very start
% of the abstract to achieve this. Many IEEE journals frown on math
% in the abstract anyway.

% Note that keywords are not normally used for peerreview papers.
%\begin{IEEEkeywords}
%Cooperative diversity, decode and forward, piecewise linear
%\end{IEEEkeywords}



% For peer review papers, you can put extra information on the cover
% page as needed:
% \ifCLASSOPTIONpeerreview
% \begin{center} \bfseries EDICS Category: 3-BBND \end{center}
% \fi
%
% For peerreview papers, this IEEEtran command inserts a page break and
% creates the second title. It will be ignored for other modes.
%\IEEEpeerreviewmaketitle




\item
If 4-digit numbers greater than 5,000 are randomly formed from the digits 0, 1, 3, 5, and 7, what is the probability of forming a number divisible by 5 when:
\begin{enumerate}
    \item The digits are repeated?
    \item The repetition of digits is not allowed?
\end{enumerate}
\solution
%\begin{table}[H]
	\centering
\begin{tabular}{|c|c|c|}
\hline
Random variable &Value &Definition\\ \hline
\multirow{3}{*}{X} &0 &Slips of Rs 1\\
&1 &Slips of Rs 5\\
&2 &Slips of Rs 13\\ \hline
\multirow{2}{*}{Y} &0 &Box A\\
&1 &Box B\\\hline
\end{tabular}
\caption{}
\label{tab:Distribution}
\end{table}
See \tabref{tab:Distribution}.
\begin{align}
p_{Y}\brak{k}= \begin{cases} 
      \frac{1}{3} & {k=0} \\
      \frac{2}{3 }& {k=1} 
   \end{cases}
   \\
p_{Y|X}\brak{0|0} = \frac{19}{25}\, 
p_{Y|X}\brak{0|1} = \frac{6}{25}\,
p_{Y|X}\brak{1|0} = \frac{45}{50}\,
p_{Y|X}\brak{1|2} = \frac{5}{50}
\end{align}
The desired probability is the probability that a slip drawn at random is marked other than Rs 1,
\begin{align}
&=1-p_X\brak{0}\\
&= p_X(1) + p_X(2)
\end{align}
Using Bayes theorem,
\begin{align}
&= p_Y\brak{0} \times \pr{Y=0 | X=1} + p_Y\brak{1} \times \pr{Y=1|X=2}\\
&=\frac{1}{3} \times \frac{6}{25} + \frac{2}{3} \times \frac{5}{50}\\
&=\frac{11}{75}
\end{align}

\newpage

%\tableofcontents

\bigskip

\renewcommand{\thefigure}{\theenumi}
\renewcommand{\thetable}{\theenumi}
%\renewcommand{\theequation}{\theenumi}

%\begin{abstract}
%%\boldmath
%In this letter, an algorithm for evaluating the exact analytical bit error rate  (BER)  for the piecewise linear (PL) combiner for  multiple relays is presented. Previous results were available only for upto three relays. The algorithm is unique in the sense that  the actual mathematical expressions, that are prohibitively large, need not be explicitly obtained. The diversity gain due to multiple relays is shown through plots of the analytical BER, well supported by simulations. 
%
%\end{abstract}
% IEEEtran.cls defaults to using nonbold math in the Abstract.
% This preserves the distinction between vectors and scalars. However,
% if the journal you are submitting to favors bold math in the abstract,
% then you can use LaTeX's standard command \boldmath at the very start
% of the abstract to achieve this. Many IEEE journals frown on math
% in the abstract anyway.

% Note that keywords are not normally used for peerreview papers.
%\begin{IEEEkeywords}
%Cooperative diversity, decode and forward, piecewise linear
%\end{IEEEkeywords}



% For peer review papers, you can put extra information on the cover
% page as needed:
% \ifCLASSOPTIONpeerreview
% \begin{center} \bfseries EDICS Category: 3-BBND \end{center}
% \fi
%
% For peerreview papers, this IEEEtran command inserts a page break and
% creates the second title. It will be ignored for other modes.
%\IEEEpeerreviewmaketitle




\item Consider the probability space $\brak{\Omega, \mathcal{G}, P}$ where $\Omega = [0,2]$ and $\mathcal{G} = \cbrak{\phi, \Omega, [0,1], (1,2]}$. Let $X$ and $Y$ be two functions on $\Omega$ defined as
\begin{align*}
    X(\omega) = 
    \begin{cases}
        1 & \text{if }\omega \in [0, 1]\\
        2 & \text{if }\omega \in (1, 2]
    \end{cases}
\end{align*}
and
\begin{align*}
    Y(\omega) = 
    \begin{cases}
        2 & \text{if }\omega \in [0, 1.5]\\
        3 & \text{if }\omega \in (1.5, 2].
    \end{cases}
\end{align*}
Then which one of the following statements is true?
\begin{enumerate}
    \item [(A)] $X$ is a random variable with respect to $\mathcal{G}$, but $Y$ is not a random variable with respect to $\mathcal{G}$.
    \item [(B)] $Y$ is a random variable with respect to $\mathcal{G}$, but $X$ is not a random variable with respect to $\mathcal{G}$.
    \item [(C)] Neither $X$ nor $Y$ is a random variable with respect to $\mathcal{G}$.
    \item [(D)] Both $X$ and $Y$ are random variables with respect to $\mathcal{G}$.
\end{enumerate} \hfill (GATE ST 2023)\\
\solution
%\begin{table}[H]
	\centering
\begin{tabular}{|c|c|c|}
\hline
Random variable &Value &Definition\\ \hline
\multirow{3}{*}{X} &0 &Slips of Rs 1\\
&1 &Slips of Rs 5\\
&2 &Slips of Rs 13\\ \hline
\multirow{2}{*}{Y} &0 &Box A\\
&1 &Box B\\\hline
\end{tabular}
\caption{}
\label{tab:Distribution}
\end{table}
See \tabref{tab:Distribution}.
\begin{align}
p_{Y}\brak{k}= \begin{cases} 
      \frac{1}{3} & {k=0} \\
      \frac{2}{3 }& {k=1} 
   \end{cases}
   \\
p_{Y|X}\brak{0|0} = \frac{19}{25}\, 
p_{Y|X}\brak{0|1} = \frac{6}{25}\,
p_{Y|X}\brak{1|0} = \frac{45}{50}\,
p_{Y|X}\brak{1|2} = \frac{5}{50}
\end{align}
The desired probability is the probability that a slip drawn at random is marked other than Rs 1,
\begin{align}
&=1-p_X\brak{0}\\
&= p_X(1) + p_X(2)
\end{align}
Using Bayes theorem,
\begin{align}
&= p_Y\brak{0} \times \pr{Y=0 | X=1} + p_Y\brak{1} \times \pr{Y=1|X=2}\\
&=\frac{1}{3} \times \frac{6}{25} + \frac{2}{3} \times \frac{5}{50}\\
&=\frac{11}{75}
\end{align}

\newpage

%\tableofcontents

\bigskip

\renewcommand{\thefigure}{\theenumi}
\renewcommand{\thetable}{\theenumi}
%\renewcommand{\theequation}{\theenumi}

%\begin{abstract}
%%\boldmath
%In this letter, an algorithm for evaluating the exact analytical bit error rate  (BER)  for the piecewise linear (PL) combiner for  multiple relays is presented. Previous results were available only for upto three relays. The algorithm is unique in the sense that  the actual mathematical expressions, that are prohibitively large, need not be explicitly obtained. The diversity gain due to multiple relays is shown through plots of the analytical BER, well supported by simulations. 
%
%\end{abstract}
% IEEEtran.cls defaults to using nonbold math in the Abstract.
% This preserves the distinction between vectors and scalars. However,
% if the journal you are submitting to favors bold math in the abstract,
% then you can use LaTeX's standard command \boldmath at the very start
% of the abstract to achieve this. Many IEEE journals frown on math
% in the abstract anyway.

% Note that keywords are not normally used for peerreview papers.
%\begin{IEEEkeywords}
%Cooperative diversity, decode and forward, piecewise linear
%\end{IEEEkeywords}



% For peer review papers, you can put extra information on the cover
% page as needed:
% \ifCLASSOPTIONpeerreview
% \begin{center} \bfseries EDICS Category: 3-BBND \end{center}
% \fi
%
% For peerreview papers, this IEEEtran command inserts a page break and
% creates the second title. It will be ignored for other modes.
%\IEEEpeerreviewmaketitle




	\item  A die is loaded in such a way that each odd number is twice as likely to occur as
each even number. Find $P(G)$, where $G$ is the event that a number greater than
3 occurs on a single roll of the die.
\\
\solution
		%\begin{table}[H]
	\centering
\begin{tabular}{|c|c|c|}
\hline
Random variable &Value &Definition\\ \hline
\multirow{3}{*}{X} &0 &Slips of Rs 1\\
&1 &Slips of Rs 5\\
&2 &Slips of Rs 13\\ \hline
\multirow{2}{*}{Y} &0 &Box A\\
&1 &Box B\\\hline
\end{tabular}
\caption{}
\label{tab:Distribution}
\end{table}
See \tabref{tab:Distribution}.
\begin{align}
p_{Y}\brak{k}= \begin{cases} 
      \frac{1}{3} & {k=0} \\
      \frac{2}{3 }& {k=1} 
   \end{cases}
   \\
p_{Y|X}\brak{0|0} = \frac{19}{25}\, 
p_{Y|X}\brak{0|1} = \frac{6}{25}\,
p_{Y|X}\brak{1|0} = \frac{45}{50}\,
p_{Y|X}\brak{1|2} = \frac{5}{50}
\end{align}
The desired probability is the probability that a slip drawn at random is marked other than Rs 1,
\begin{align}
&=1-p_X\brak{0}\\
&= p_X(1) + p_X(2)
\end{align}
Using Bayes theorem,
\begin{align}
&= p_Y\brak{0} \times \pr{Y=0 | X=1} + p_Y\brak{1} \times \pr{Y=1|X=2}\\
&=\frac{1}{3} \times \frac{6}{25} + \frac{2}{3} \times \frac{5}{50}\\
&=\frac{11}{75}
\end{align}

\newpage

%\tableofcontents

\bigskip

\renewcommand{\thefigure}{\theenumi}
\renewcommand{\thetable}{\theenumi}
%\renewcommand{\theequation}{\theenumi}

%\begin{abstract}
%%\boldmath
%In this letter, an algorithm for evaluating the exact analytical bit error rate  (BER)  for the piecewise linear (PL) combiner for  multiple relays is presented. Previous results were available only for upto three relays. The algorithm is unique in the sense that  the actual mathematical expressions, that are prohibitively large, need not be explicitly obtained. The diversity gain due to multiple relays is shown through plots of the analytical BER, well supported by simulations. 
%
%\end{abstract}
% IEEEtran.cls defaults to using nonbold math in the Abstract.
% This preserves the distinction between vectors and scalars. However,
% if the journal you are submitting to favors bold math in the abstract,
% then you can use LaTeX's standard command \boldmath at the very start
% of the abstract to achieve this. Many IEEE journals frown on math
% in the abstract anyway.

% Note that keywords are not normally used for peerreview papers.
%\begin{IEEEkeywords}
%Cooperative diversity, decode and forward, piecewise linear
%\end{IEEEkeywords}



% For peer review papers, you can put extra information on the cover
% page as needed:
% \ifCLASSOPTIONpeerreview
% \begin{center} \bfseries EDICS Category: 3-BBND \end{center}
% \fi
%
% For peerreview papers, this IEEEtran command inserts a page break and
% creates the second title. It will be ignored for other modes.
%\IEEEpeerreviewmaketitle




	\item All the jacks, queens and kings are removed from a deck of 52 playing cards. The remaining cards are well shuffled and then one card is drawn at random. Giving ace a value 1 similar value for other cards, find the probability that the card has a value 
		\begin{enumerate}
			\item 7
			\item greater than 7
			\item less than 7
		\end{enumerate}
		%Number of cards left after removing all jacks, queens and kings 
\begin{align}
N	= 52 - 4\times 3
	= 40
\end{align}
%\begin{table}[H]
%\def\arraystretch{1.2}
%\begin{tabular}{|c|c|c|}
%\hline
%	\textbf{Parameter} &\textbf{Value} &\textbf{Description}\\ \hline
%	$X$ &1-10 &Represents the value of the card picked \\ \hline
%\end{tabular}
%\end{table}
Let $1 \le X \le 10$ be the value of the card picked.  Then,
\begin{align}
	p_X(k) &= \Pr(X=k)\ \forall\ 1 \leq k \leq 10\\
	&= \frac{4\times 1}{40}\\
	&= \frac{1}{10}\\
	\therefore p_X(k) &= 
	\begin{cases}
		\frac{1}{10} & 1 \leq k \leq 10\\
		0 & \text{otherwise}
	\end{cases}
\end{align}
and
\begin{align}
	F_{X}(k) &= \sum_{m=0}^{k}p_{X}(m) \quad 1 \leq k \leq 10\\
	&= \frac{k}{10}\\
	\therefore F_{X}(k) &= 
	\begin{cases}
		0 & k \leq 0\\
		\frac{k}{10} & 1\leq k \leq 10\\
		1 & k > 10 
	\end{cases}
\end{align}
\begin{enumerate}
	\item Probability that card has value equal to 7 is
		\begin{align}
			 p_{X}(7)
			= \frac{1}{10}
		\end{align}
	\item Probability that card has value greater than 7 is
		\begin{align}
			1 - F_X(7)
			&= 1 - \frac{7}{10}
			\\
			&= \frac{3}{10}
		\end{align}
	\item Probability that card has value less than 7 is
		\begin{align}
			 F_{X}(6)
			=\frac{6}{10}
		\end{align}
\end{enumerate}

  \item A Lot consists of 48 mobile phones of which 42 are good, 3 have only minor defects and 3 have major defects.Varnika will buy a phone if it is good but the trader will only buy a mobile if it has no major defects. One phone is selected at random from the lot. What is the probability that it is
\begin{enumerate}
	\item acceptable to Varnika?
            \item acceptable to the trader?
\end{enumerate}
\solution
	%\begin{table}[H]
	\centering
\begin{tabular}{|c|c|c|}
\hline
Random variable &Value &Definition\\ \hline
\multirow{3}{*}{X} &0 &Slips of Rs 1\\
&1 &Slips of Rs 5\\
&2 &Slips of Rs 13\\ \hline
\multirow{2}{*}{Y} &0 &Box A\\
&1 &Box B\\\hline
\end{tabular}
\caption{}
\label{tab:Distribution}
\end{table}
See \tabref{tab:Distribution}.
\begin{align}
p_{Y}\brak{k}= \begin{cases} 
      \frac{1}{3} & {k=0} \\
      \frac{2}{3 }& {k=1} 
   \end{cases}
   \\
p_{Y|X}\brak{0|0} = \frac{19}{25}\, 
p_{Y|X}\brak{0|1} = \frac{6}{25}\,
p_{Y|X}\brak{1|0} = \frac{45}{50}\,
p_{Y|X}\brak{1|2} = \frac{5}{50}
\end{align}
The desired probability is the probability that a slip drawn at random is marked other than Rs 1,
\begin{align}
&=1-p_X\brak{0}\\
&= p_X(1) + p_X(2)
\end{align}
Using Bayes theorem,
\begin{align}
&= p_Y\brak{0} \times \pr{Y=0 | X=1} + p_Y\brak{1} \times \pr{Y=1|X=2}\\
&=\frac{1}{3} \times \frac{6}{25} + \frac{2}{3} \times \frac{5}{50}\\
&=\frac{11}{75}
\end{align}

\newpage

%\tableofcontents

\bigskip

\renewcommand{\thefigure}{\theenumi}
\renewcommand{\thetable}{\theenumi}
%\renewcommand{\theequation}{\theenumi}

%\begin{abstract}
%%\boldmath
%In this letter, an algorithm for evaluating the exact analytical bit error rate  (BER)  for the piecewise linear (PL) combiner for  multiple relays is presented. Previous results were available only for upto three relays. The algorithm is unique in the sense that  the actual mathematical expressions, that are prohibitively large, need not be explicitly obtained. The diversity gain due to multiple relays is shown through plots of the analytical BER, well supported by simulations. 
%
%\end{abstract}
% IEEEtran.cls defaults to using nonbold math in the Abstract.
% This preserves the distinction between vectors and scalars. However,
% if the journal you are submitting to favors bold math in the abstract,
% then you can use LaTeX's standard command \boldmath at the very start
% of the abstract to achieve this. Many IEEE journals frown on math
% in the abstract anyway.

% Note that keywords are not normally used for peerreview papers.
%\begin{IEEEkeywords}
%Cooperative diversity, decode and forward, piecewise linear
%\end{IEEEkeywords}



% For peer review papers, you can put extra information on the cover
% page as needed:
% \ifCLASSOPTIONpeerreview
% \begin{center} \bfseries EDICS Category: 3-BBND \end{center}
% \fi
%
% For peerreview papers, this IEEEtran command inserts a page break and
% creates the second title. It will be ignored for other modes.
%\IEEEpeerreviewmaketitle




 \item A student says that if you throw a die, it will show up 1 or not 1. Therefore, the probability of getting 1 and the probability of getting 'not 1' each is equal to $\frac{1}{2}$. Is this correct? Give reasons.\\
 \solution
        %\begin{table}[H]
	\centering
\begin{tabular}{|c|c|c|}
\hline
Random variable &Value &Definition\\ \hline
\multirow{3}{*}{X} &0 &Slips of Rs 1\\
&1 &Slips of Rs 5\\
&2 &Slips of Rs 13\\ \hline
\multirow{2}{*}{Y} &0 &Box A\\
&1 &Box B\\\hline
\end{tabular}
\caption{}
\label{tab:Distribution}
\end{table}
See \tabref{tab:Distribution}.
\begin{align}
p_{Y}\brak{k}= \begin{cases} 
      \frac{1}{3} & {k=0} \\
      \frac{2}{3 }& {k=1} 
   \end{cases}
   \\
p_{Y|X}\brak{0|0} = \frac{19}{25}\, 
p_{Y|X}\brak{0|1} = \frac{6}{25}\,
p_{Y|X}\brak{1|0} = \frac{45}{50}\,
p_{Y|X}\brak{1|2} = \frac{5}{50}
\end{align}
The desired probability is the probability that a slip drawn at random is marked other than Rs 1,
\begin{align}
&=1-p_X\brak{0}\\
&= p_X(1) + p_X(2)
\end{align}
Using Bayes theorem,
\begin{align}
&= p_Y\brak{0} \times \pr{Y=0 | X=1} + p_Y\brak{1} \times \pr{Y=1|X=2}\\
&=\frac{1}{3} \times \frac{6}{25} + \frac{2}{3} \times \frac{5}{50}\\
&=\frac{11}{75}
\end{align}

\newpage

%\tableofcontents

\bigskip

\renewcommand{\thefigure}{\theenumi}
\renewcommand{\thetable}{\theenumi}
%\renewcommand{\theequation}{\theenumi}

%\begin{abstract}
%%\boldmath
%In this letter, an algorithm for evaluating the exact analytical bit error rate  (BER)  for the piecewise linear (PL) combiner for  multiple relays is presented. Previous results were available only for upto three relays. The algorithm is unique in the sense that  the actual mathematical expressions, that are prohibitively large, need not be explicitly obtained. The diversity gain due to multiple relays is shown through plots of the analytical BER, well supported by simulations. 
%
%\end{abstract}
% IEEEtran.cls defaults to using nonbold math in the Abstract.
% This preserves the distinction between vectors and scalars. However,
% if the journal you are submitting to favors bold math in the abstract,
% then you can use LaTeX's standard command \boldmath at the very start
% of the abstract to achieve this. Many IEEE journals frown on math
% in the abstract anyway.

% Note that keywords are not normally used for peerreview papers.
%\begin{IEEEkeywords}
%Cooperative diversity, decode and forward, piecewise linear
%\end{IEEEkeywords}



% For peer review papers, you can put extra information on the cover
% page as needed:
% \ifCLASSOPTIONpeerreview
% \begin{center} \bfseries EDICS Category: 3-BBND \end{center}
% \fi
%
% For peerreview papers, this IEEEtran command inserts a page break and
% creates the second title. It will be ignored for other modes.
%\IEEEpeerreviewmaketitle




   \item Four candidates A, B, C, D have ap-
plied for the assignment to coach a school cricket
team. If A is twice as likely to be selected as B, and
B and C are given about the same chance of being
selected, while C is twice as likely to be selected
as D, what are the probabilities that
\begin{enumerate}
\item C will be selected?
\item A will not be selected?
\end{enumerate}
	%\begin{table}[H]
	\centering
\begin{tabular}{|c|c|c|}
\hline
Random variable &Value &Definition\\ \hline
\multirow{3}{*}{X} &0 &Slips of Rs 1\\
&1 &Slips of Rs 5\\
&2 &Slips of Rs 13\\ \hline
\multirow{2}{*}{Y} &0 &Box A\\
&1 &Box B\\\hline
\end{tabular}
\caption{}
\label{tab:Distribution}
\end{table}
See \tabref{tab:Distribution}.
\begin{align}
p_{Y}\brak{k}= \begin{cases} 
      \frac{1}{3} & {k=0} \\
      \frac{2}{3 }& {k=1} 
   \end{cases}
   \\
p_{Y|X}\brak{0|0} = \frac{19}{25}\, 
p_{Y|X}\brak{0|1} = \frac{6}{25}\,
p_{Y|X}\brak{1|0} = \frac{45}{50}\,
p_{Y|X}\brak{1|2} = \frac{5}{50}
\end{align}
The desired probability is the probability that a slip drawn at random is marked other than Rs 1,
\begin{align}
&=1-p_X\brak{0}\\
&= p_X(1) + p_X(2)
\end{align}
Using Bayes theorem,
\begin{align}
&= p_Y\brak{0} \times \pr{Y=0 | X=1} + p_Y\brak{1} \times \pr{Y=1|X=2}\\
&=\frac{1}{3} \times \frac{6}{25} + \frac{2}{3} \times \frac{5}{50}\\
&=\frac{11}{75}
\end{align}

\newpage

%\tableofcontents

\bigskip

\renewcommand{\thefigure}{\theenumi}
\renewcommand{\thetable}{\theenumi}
%\renewcommand{\theequation}{\theenumi}

%\begin{abstract}
%%\boldmath
%In this letter, an algorithm for evaluating the exact analytical bit error rate  (BER)  for the piecewise linear (PL) combiner for  multiple relays is presented. Previous results were available only for upto three relays. The algorithm is unique in the sense that  the actual mathematical expressions, that are prohibitively large, need not be explicitly obtained. The diversity gain due to multiple relays is shown through plots of the analytical BER, well supported by simulations. 
%
%\end{abstract}
% IEEEtran.cls defaults to using nonbold math in the Abstract.
% This preserves the distinction between vectors and scalars. However,
% if the journal you are submitting to favors bold math in the abstract,
% then you can use LaTeX's standard command \boldmath at the very start
% of the abstract to achieve this. Many IEEE journals frown on math
% in the abstract anyway.

% Note that keywords are not normally used for peerreview papers.
%\begin{IEEEkeywords}
%Cooperative diversity, decode and forward, piecewise linear
%\end{IEEEkeywords}



% For peer review papers, you can put extra information on the cover
% page as needed:
% \ifCLASSOPTIONpeerreview
% \begin{center} \bfseries EDICS Category: 3-BBND \end{center}
% \fi
%
% For peerreview papers, this IEEEtran command inserts a page break and
% creates the second title. It will be ignored for other modes.
%\IEEEpeerreviewmaketitle




 \item A bag contain 24 balls of which $x$ balls are red, $2x$ are white and $3x$ are blue. A ball is selected at random, What is the probability that it is
\begin{enumerate}[label=\alph*)]
\item not red ?
\item white ?
\end{enumerate}
%\begin{table}[H]
	\centering
\begin{tabular}{|c|c|c|}
\hline
Random variable &Value &Definition\\ \hline
\multirow{3}{*}{X} &0 &Slips of Rs 1\\
&1 &Slips of Rs 5\\
&2 &Slips of Rs 13\\ \hline
\multirow{2}{*}{Y} &0 &Box A\\
&1 &Box B\\\hline
\end{tabular}
\caption{}
\label{tab:Distribution}
\end{table}
See \tabref{tab:Distribution}.
\begin{align}
p_{Y}\brak{k}= \begin{cases} 
      \frac{1}{3} & {k=0} \\
      \frac{2}{3 }& {k=1} 
   \end{cases}
   \\
p_{Y|X}\brak{0|0} = \frac{19}{25}\, 
p_{Y|X}\brak{0|1} = \frac{6}{25}\,
p_{Y|X}\brak{1|0} = \frac{45}{50}\,
p_{Y|X}\brak{1|2} = \frac{5}{50}
\end{align}
The desired probability is the probability that a slip drawn at random is marked other than Rs 1,
\begin{align}
&=1-p_X\brak{0}\\
&= p_X(1) + p_X(2)
\end{align}
Using Bayes theorem,
\begin{align}
&= p_Y\brak{0} \times \pr{Y=0 | X=1} + p_Y\brak{1} \times \pr{Y=1|X=2}\\
&=\frac{1}{3} \times \frac{6}{25} + \frac{2}{3} \times \frac{5}{50}\\
&=\frac{11}{75}
\end{align}

\newpage

%\tableofcontents

\bigskip

\renewcommand{\thefigure}{\theenumi}
\renewcommand{\thetable}{\theenumi}
%\renewcommand{\theequation}{\theenumi}

%\begin{abstract}
%%\boldmath
%In this letter, an algorithm for evaluating the exact analytical bit error rate  (BER)  for the piecewise linear (PL) combiner for  multiple relays is presented. Previous results were available only for upto three relays. The algorithm is unique in the sense that  the actual mathematical expressions, that are prohibitively large, need not be explicitly obtained. The diversity gain due to multiple relays is shown through plots of the analytical BER, well supported by simulations. 
%
%\end{abstract}
% IEEEtran.cls defaults to using nonbold math in the Abstract.
% This preserves the distinction between vectors and scalars. However,
% if the journal you are submitting to favors bold math in the abstract,
% then you can use LaTeX's standard command \boldmath at the very start
% of the abstract to achieve this. Many IEEE journals frown on math
% in the abstract anyway.

% Note that keywords are not normally used for peerreview papers.
%\begin{IEEEkeywords}
%Cooperative diversity, decode and forward, piecewise linear
%\end{IEEEkeywords}



% For peer review papers, you can put extra information on the cover
% page as needed:
% \ifCLASSOPTIONpeerreview
% \begin{center} \bfseries EDICS Category: 3-BBND \end{center}
% \fi
%
% For peerreview papers, this IEEEtran command inserts a page break and
% creates the second title. It will be ignored for other modes.
%\IEEEpeerreviewmaketitle




If the letters of the word ASSASSINATION are arranged at random. Find the Probability that
\begin{enumerate}[label=(\alph*)]
\item Four $S's$ come consecutively in the word
\item Two  $I's$ and two $N's$ come together
\item All $A's$ are not coming together
\item No two $A's$ are coming together
\end{enumerate}
%\begin{table}[H]
	\centering
\begin{tabular}{|c|c|c|}
\hline
Random variable &Value &Definition\\ \hline
\multirow{3}{*}{X} &0 &Slips of Rs 1\\
&1 &Slips of Rs 5\\
&2 &Slips of Rs 13\\ \hline
\multirow{2}{*}{Y} &0 &Box A\\
&1 &Box B\\\hline
\end{tabular}
\caption{}
\label{tab:Distribution}
\end{table}
See \tabref{tab:Distribution}.
\begin{align}
p_{Y}\brak{k}= \begin{cases} 
      \frac{1}{3} & {k=0} \\
      \frac{2}{3 }& {k=1} 
   \end{cases}
   \\
p_{Y|X}\brak{0|0} = \frac{19}{25}\, 
p_{Y|X}\brak{0|1} = \frac{6}{25}\,
p_{Y|X}\brak{1|0} = \frac{45}{50}\,
p_{Y|X}\brak{1|2} = \frac{5}{50}
\end{align}
The desired probability is the probability that a slip drawn at random is marked other than Rs 1,
\begin{align}
&=1-p_X\brak{0}\\
&= p_X(1) + p_X(2)
\end{align}
Using Bayes theorem,
\begin{align}
&= p_Y\brak{0} \times \pr{Y=0 | X=1} + p_Y\brak{1} \times \pr{Y=1|X=2}\\
&=\frac{1}{3} \times \frac{6}{25} + \frac{2}{3} \times \frac{5}{50}\\
&=\frac{11}{75}
\end{align}

\newpage

%\tableofcontents

\bigskip

\renewcommand{\thefigure}{\theenumi}
\renewcommand{\thetable}{\theenumi}
%\renewcommand{\theequation}{\theenumi}

%\begin{abstract}
%%\boldmath
%In this letter, an algorithm for evaluating the exact analytical bit error rate  (BER)  for the piecewise linear (PL) combiner for  multiple relays is presented. Previous results were available only for upto three relays. The algorithm is unique in the sense that  the actual mathematical expressions, that are prohibitively large, need not be explicitly obtained. The diversity gain due to multiple relays is shown through plots of the analytical BER, well supported by simulations. 
%
%\end{abstract}
% IEEEtran.cls defaults to using nonbold math in the Abstract.
% This preserves the distinction between vectors and scalars. However,
% if the journal you are submitting to favors bold math in the abstract,
% then you can use LaTeX's standard command \boldmath at the very start
% of the abstract to achieve this. Many IEEE journals frown on math
% in the abstract anyway.

% Note that keywords are not normally used for peerreview papers.
%\begin{IEEEkeywords}
%Cooperative diversity, decode and forward, piecewise linear
%\end{IEEEkeywords}



% For peer review papers, you can put extra information on the cover
% page as needed:
% \ifCLASSOPTIONpeerreview
% \begin{center} \bfseries EDICS Category: 3-BBND \end{center}
% \fi
%
% For peerreview papers, this IEEEtran command inserts a page break and
% creates the second title. It will be ignored for other modes.
%\IEEEpeerreviewmaketitle




	\item One urn contains two black balls (labelled B1 and B2) and one white ball. A
	second urn contains one black ball and two white balls (labelled W1 and W2).
	Suppose the following experiment is performed. One of the two urns is chosen
	at random. Next a ball is randomly chosen from the urn. Then a second ball is
	chosen at random from the same urn without replacing the first ball.
	
	\begin{enumerate}
	\item What is the probability that two black balls are chosen?
	
	\item What is the probability that two balls of opposite colour are chosen?
	\end{enumerate}
	\solution
	%\begin{align}
    \label{eq:12.13.6.18.1}
	\because	\pr{A|B} &> \pr{A},\
\frac{\pr{AB}}{\pr{B}} > \pr{A}
\\
    \label{eq:12.13.6.18.2}
	\implies \pr{AB} &> \pr{A}\pr{B}
	\\
	\text{or, } \frac{\pr{AB}}{\pr{A}} &=\pr{B|A} > \pr{A}
\end{align}

\end{enumerate}

		%
\item 
Two cards are drawn at random and without replacement from a pack of 52 playing cards. Find the probability that both the cards are black.
\\
\solution
		%\begin{enumerate}[label=\thesection.\arabic*,ref=\thesection.\theenumi]
	\item One card is drawn from a well-shuffled deck of 52 cards. Find the probability of getting
\begin{enumerate}
\item A king of red colour 
\item A face card 
\item A red face card
\item The jack of hearts
\item A spade
\item The queen of diamonds

\end{enumerate}
\solution
		%\begin{table}[H]
	\centering
\begin{tabular}{|c|c|c|}
\hline
Random variable &Value &Definition\\ \hline
\multirow{3}{*}{X} &0 &Slips of Rs 1\\
&1 &Slips of Rs 5\\
&2 &Slips of Rs 13\\ \hline
\multirow{2}{*}{Y} &0 &Box A\\
&1 &Box B\\\hline
\end{tabular}
\caption{}
\label{tab:Distribution}
\end{table}
See \tabref{tab:Distribution}.
\begin{align}
p_{Y}\brak{k}= \begin{cases} 
      \frac{1}{3} & {k=0} \\
      \frac{2}{3 }& {k=1} 
   \end{cases}
   \\
p_{Y|X}\brak{0|0} = \frac{19}{25}\, 
p_{Y|X}\brak{0|1} = \frac{6}{25}\,
p_{Y|X}\brak{1|0} = \frac{45}{50}\,
p_{Y|X}\brak{1|2} = \frac{5}{50}
\end{align}
The desired probability is the probability that a slip drawn at random is marked other than Rs 1,
\begin{align}
&=1-p_X\brak{0}\\
&= p_X(1) + p_X(2)
\end{align}
Using Bayes theorem,
\begin{align}
&= p_Y\brak{0} \times \pr{Y=0 | X=1} + p_Y\brak{1} \times \pr{Y=1|X=2}\\
&=\frac{1}{3} \times \frac{6}{25} + \frac{2}{3} \times \frac{5}{50}\\
&=\frac{11}{75}
\end{align}

\newpage

%\tableofcontents

\bigskip

\renewcommand{\thefigure}{\theenumi}
\renewcommand{\thetable}{\theenumi}
%\renewcommand{\theequation}{\theenumi}

%\begin{abstract}
%%\boldmath
%In this letter, an algorithm for evaluating the exact analytical bit error rate  (BER)  for the piecewise linear (PL) combiner for  multiple relays is presented. Previous results were available only for upto three relays. The algorithm is unique in the sense that  the actual mathematical expressions, that are prohibitively large, need not be explicitly obtained. The diversity gain due to multiple relays is shown through plots of the analytical BER, well supported by simulations. 
%
%\end{abstract}
% IEEEtran.cls defaults to using nonbold math in the Abstract.
% This preserves the distinction between vectors and scalars. However,
% if the journal you are submitting to favors bold math in the abstract,
% then you can use LaTeX's standard command \boldmath at the very start
% of the abstract to achieve this. Many IEEE journals frown on math
% in the abstract anyway.

% Note that keywords are not normally used for peerreview papers.
%\begin{IEEEkeywords}
%Cooperative diversity, decode and forward, piecewise linear
%\end{IEEEkeywords}



% For peer review papers, you can put extra information on the cover
% page as needed:
% \ifCLASSOPTIONpeerreview
% \begin{center} \bfseries EDICS Category: 3-BBND \end{center}
% \fi
%
% For peerreview papers, this IEEEtran command inserts a page break and
% creates the second title. It will be ignored for other modes.
%\IEEEpeerreviewmaketitle




	\item Five cards—the ten, jack, queen, king and ace of diamonds, are well-shuffled with their face downwards. One card is then picked up at random.
\begin{enumerate}
\item
What is the probability that the card is the queen? 
\item
If the queen is drawn and put aside, what is the probability that the second card picked up is (a) an ace? (b) a queen?\\
\end{enumerate}
\solution
		%\begin{enumerate}[label=\thesection.\arabic*,ref=\thesection.\theenumi]
	\item One card is drawn from a well-shuffled deck of 52 cards. Find the probability of getting
\begin{enumerate}
\item A king of red colour 
\item A face card 
\item A red face card
\item The jack of hearts
\item A spade
\item The queen of diamonds

\end{enumerate}
\solution
		%\input{ncert/10/15/1/14/main.tex}
	\item Five cards—the ten, jack, queen, king and ace of diamonds, are well-shuffled with their face downwards. One card is then picked up at random.
\begin{enumerate}
\item
What is the probability that the card is the queen? 
\item
If the queen is drawn and put aside, what is the probability that the second card picked up is (a) an ace? (b) a queen?\\
\end{enumerate}
\solution
		%\input{ncert/10/15/1/15/defs.tex}
	\item A bag contains $5$ red balls and some blue balls. If the probability of drawing a blue ball is double that if a red ball, determine the number of blue balls in the bag. 
		\\
\solution
		%\input{ncert/10/15/2/3/defs.tex}
	\item A card is selected from a pack of 52 cards.
 \begin{enumerate}[label=(\alph*)] 
                 \item How many points are there in the sample space?
                 \item Calculate the probability that the card is an ace of spades.
                 \item Calculate the probability that the card is (i) an ace and (ii) black card.
 \end{enumerate}
\solution
		%\input{ncert/11/16/3/4/main.tex}
\item Four cards are drawn from a well-shuffled deck of 52 cards. What is the probability of obtaining 3 diamonds and one spade.
\\
\solution
		%\input{ncert/11/16/4/2/defs.tex}
\item In a certain lottery 10,000 tickets are sold and ten equal prizes are awarded. What is the probability of not getting a prize if you buy (a) one ticket (b) two tickets (c) 10 tickets ?	
\\
\solution
		%\input{ncert/11/16/4/4/defs.tex}
		%
\item 
Out of 100 students, two sections of 40 and 60 are formed. If you and your friend are among the 100 students, what is the probability that
\begin{enumerate}
\item you both enter the same section?
\item you both enter the different sections?
\end{enumerate}
\solution
		%\input{ncert/11/16/4/5/defs.tex}
	\item 
The number lock of a suitcase has 4 wheels each labelled with ten digits i.e. from 0 to 9.The lock opens with a sequence of four digits with no repeats.What is the probability of a person getting the right sequence to open the suitcase.
\\
\solution
		%\input{ncert/11/16/4/10/defs.tex}
		%
\item 
Two cards are drawn at random and without replacement from a pack of 52 playing cards. Find the probability that both the cards are black.
\\
\solution
		%\input{ncert/12/13/2/2/defs.tex}
		\item A box of oranges is inspected by examining three randomly selected oranges drawn without replacement. If all the three oranges are good, the box is approved for sale, otherwise, it is rejected. Find the probability that a box containing 15 oranges out of which 12 are good and 3 are bad ones will be approved for sale.
		\label{ncert/12/13/2/3/defs.tex}
		\item Two balls are drawn at random with replacement from a box containing 10 black and 8 red balls. Find the probability that
		\label{ncert/12/13/2/12}
\begin{enumerate}
\item both balls are red.
\item first ball is black and second is red.
\item one of them is black and other is red.
\end{enumerate}

\item In a hostel, 60\% of the students read Hindi newspaper, 40\% read English newspaper and 20\% read both Hindi and English newspapers. A student is selected at random.
		\label{ncert/12/13/2/15}
\begin{enumerate}
\item Find the probability that she reads neither Hindi nor English newspapers.
\item If she reads Hindi newspaper, find the probability that she reads English newspaper.
\item If she reads English newspaper, find the probability that she reads Hindi newspaper.\\
\end{enumerate}
\item The probability of obtaining an even prime number on each die, when a pair of dice is rolled is 
\begin{enumerate}
    \item $0$ 
    
    \item $\frac{1}{3}$ 
    
    \item $\frac{1}{12}$ 
    
    \item $\frac{1}{36}$ 
\end{enumerate}
\solution
		%\input{ncert/12/13/2/17/defs.tex}
	\item A bag contains 4 red and 4 black balls, another bag contains 2 red and 6 black balls. One of the two bags is selected at random and a ball is drawn from the bag which is found to be red. Find the probability that the ball is drawn from the first bag.
\\
\solution
		%\input{ncert/12/13/3/2/main.tex}
  \item
  Cards with numbers 2 to 101 are placed in a box. A card is selected at random.Find the probability that the card has
\begin{enumerate}[label=(\roman*)]
	\item an even number 
	\item a square number
\end{enumerate}
\solution
%\input{exemplar/10/13/3/32/main.tex}
\item
The king, queen and jack of clubs are removed from a deck of 52 playing cards and then well shuffled. Now one card is drawn at random from the remaining cards.  Determine the probability that the card is
\begin{enumerate}[label=(\roman*)]
\item a club
\item 10 of hearts
\end{enumerate}
\solution
%\input{exemplar/10/13/3/29/main.tex}
\item A team of medical students doing their internship have to assist during surgeries
at a city hospital. The probabilities of surgeries rated as very complex, complex,
routine, simple or very simple are respectively, 0.15, 0.20, 0.31, 0.26, .08. Find
the probabilities that a particular surgery will be rated
\begin{enumerate}
	\item complex or very complex;
	\item neither very complex nor very simple;
	\item routine or complex
	\item routine or simple
\end{enumerate}
\solution
%\input{exemplar/11/16/3/8(1)/main.tex}
\item A card is selected from a pack of 52 cards.
\begin{enumerate}[label=(\alph*)]
    \item How many points are there in the sample space?
    \item Calculate the probability that the card is an ace of spades.
    \item Calculate the probability that the card is (i) an ace and (ii) black card.
\end{enumerate}
\solution
%\input{exemplar/11/16/3/4/main2.tex}
\item The probability that a non leap year selected at random will contain 53 sundays.
\\
\solution
%\input{exemplar/10/13/1/19/main.tex}
\item One of the four persons John, Rita, Aslam or Gurpreet will be promoted next
month. Consequently the sample space consists of four elementary outcomes
S = {John promoted, Rita promoted, Aslam promoted, Gurpreet promoted}
You are told that the chances of John’s promotion is same as that of Gurpreet,
Rita’s chances of promotion are twice as likely as Johns. Aslam’s chances are
four times that of John.
\begin{enumerate}
	\item Determine
	\begin{enumerate}
		\item P (John promoted)
		\item P (Rita promoted)
		\item P (Aslam promoted)
		\item P (Gurpreet promoted)
	\end{enumerate}
	\item If A = {John promoted or Gurpreet promoted}, find P (A).
\end{enumerate}
\solution
%\input{exemplar/11/16/3/10/main.tex}
\item A card is drawn from a deck of 52 cards. Find the probability of getting a king or a heart or a red card.\\
\solution
%\input{exemplar/11/16/3/15/main.tex}
\item The probability that a student will pass his examination is 0.73, the probability of
the student getting a compartment is 0.13, and the probability that the student will
either pass or get compartment is 0.96. State True or False.\\
\solution
%\input{exemplar/11/16/3/31/main.tex}
\item A card is selected from a pack of 52 cards\\
\begin{enumerate}[label=(\alph*)]
\item How many points are there in the sample space?
\item Calculate the probability that the cards is an ace of spades.
\item Calculate the probability that the card is (i) an ace (ii)black card.\\
\end{enumerate}
%\input{ncert/11/16/3/4_1/Prob_4.tex}
\item In a non-leap year, the probability of having 53 tuesdays or 53 wednesdays is\\
\solution
%\input{exemplar/11/16/3/18/main.tex}
\item There are 1000 sealed envelopes in a box, 10 of them contain a cash prize of
Rs 100 each, 100 of them contain a cash prize of Rs 50 each and 200 of them
contain a cash prize of Rs 10 each and rest do not contain any cash prize. If they
are well shuffled and an envelope is picked up out, what is the probability that it
contains no cash prize?\\
\solution
%\input{exemplar/10/13/3/34/main.tex}
\item 
A die is thrown and a card is selected at random from a deck of 52 playing cards. The probability of getting an even number on the die and a spade card.\\
\solution
%\input{exemplar/12/13/3/78/main.tex}
\item
If 4-digit numbers greater than 5,000 are randomly formed from the digits 0, 1, 3, 5, and 7, what is the probability of forming a number divisible by 5 when:
\begin{enumerate}
    \item The digits are repeated?
    \item The repetition of digits is not allowed?
\end{enumerate}
\solution
%\input{ncert/11/16/4/9/main.tex}
\item Consider the probability space $\brak{\Omega, \mathcal{G}, P}$ where $\Omega = [0,2]$ and $\mathcal{G} = \cbrak{\phi, \Omega, [0,1], (1,2]}$. Let $X$ and $Y$ be two functions on $\Omega$ defined as
\begin{align*}
    X(\omega) = 
    \begin{cases}
        1 & \text{if }\omega \in [0, 1]\\
        2 & \text{if }\omega \in (1, 2]
    \end{cases}
\end{align*}
and
\begin{align*}
    Y(\omega) = 
    \begin{cases}
        2 & \text{if }\omega \in [0, 1.5]\\
        3 & \text{if }\omega \in (1.5, 2].
    \end{cases}
\end{align*}
Then which one of the following statements is true?
\begin{enumerate}
    \item [(A)] $X$ is a random variable with respect to $\mathcal{G}$, but $Y$ is not a random variable with respect to $\mathcal{G}$.
    \item [(B)] $Y$ is a random variable with respect to $\mathcal{G}$, but $X$ is not a random variable with respect to $\mathcal{G}$.
    \item [(C)] Neither $X$ nor $Y$ is a random variable with respect to $\mathcal{G}$.
    \item [(D)] Both $X$ and $Y$ are random variables with respect to $\mathcal{G}$.
\end{enumerate} \hfill (GATE ST 2023)\\
\solution
%\input{gate/ST/2023/14/main.tex}
	\item  A die is loaded in such a way that each odd number is twice as likely to occur as
each even number. Find $P(G)$, where $G$ is the event that a number greater than
3 occurs on a single roll of the die.
\\
\solution
		%\input{exemplar/11/16/3/5/main.tex}
	\item All the jacks, queens and kings are removed from a deck of 52 playing cards. The remaining cards are well shuffled and then one card is drawn at random. Giving ace a value 1 similar value for other cards, find the probability that the card has a value 
		\begin{enumerate}
			\item 7
			\item greater than 7
			\item less than 7
		\end{enumerate}
		%\input{exemplar/10/13/3/30/main.tex}
  \item A Lot consists of 48 mobile phones of which 42 are good, 3 have only minor defects and 3 have major defects.Varnika will buy a phone if it is good but the trader will only buy a mobile if it has no major defects. One phone is selected at random from the lot. What is the probability that it is
\begin{enumerate}
	\item acceptable to Varnika?
            \item acceptable to the trader?
\end{enumerate}
\solution
	%\input{exemplar/10/13/3/40/main.tex}
 \item A student says that if you throw a die, it will show up 1 or not 1. Therefore, the probability of getting 1 and the probability of getting 'not 1' each is equal to $\frac{1}{2}$. Is this correct? Give reasons.\\
 \solution
        %\input{exemplar/10/13/2/9/main.tex}
   \item Four candidates A, B, C, D have ap-
plied for the assignment to coach a school cricket
team. If A is twice as likely to be selected as B, and
B and C are given about the same chance of being
selected, while C is twice as likely to be selected
as D, what are the probabilities that
\begin{enumerate}
\item C will be selected?
\item A will not be selected?
\end{enumerate}
	%\input{exemplar/11/16/3/9/main.tex}
 \item A bag contain 24 balls of which $x$ balls are red, $2x$ are white and $3x$ are blue. A ball is selected at random, What is the probability that it is
\begin{enumerate}[label=\alph*)]
\item not red ?
\item white ?
\end{enumerate}
%\input{exemplar/10/13/3/41/main.tex}
If the letters of the word ASSASSINATION are arranged at random. Find the Probability that
\begin{enumerate}[label=(\alph*)]
\item Four $S's$ come consecutively in the word
\item Two  $I's$ and two $N's$ come together
\item All $A's$ are not coming together
\item No two $A's$ are coming together
\end{enumerate}
%\input{exemplar/11/16/3/14/main.tex}
	\item One urn contains two black balls (labelled B1 and B2) and one white ball. A
	second urn contains one black ball and two white balls (labelled W1 and W2).
	Suppose the following experiment is performed. One of the two urns is chosen
	at random. Next a ball is randomly chosen from the urn. Then a second ball is
	chosen at random from the same urn without replacing the first ball.
	
	\begin{enumerate}
	\item What is the probability that two black balls are chosen?
	
	\item What is the probability that two balls of opposite colour are chosen?
	\end{enumerate}
	\solution
	%\input{exemplar/11/16/3/12/main1.tex}
\end{enumerate}

	\item A bag contains $5$ red balls and some blue balls. If the probability of drawing a blue ball is double that if a red ball, determine the number of blue balls in the bag. 
		\\
\solution
		%\begin{enumerate}[label=\thesection.\arabic*,ref=\thesection.\theenumi]
	\item One card is drawn from a well-shuffled deck of 52 cards. Find the probability of getting
\begin{enumerate}
\item A king of red colour 
\item A face card 
\item A red face card
\item The jack of hearts
\item A spade
\item The queen of diamonds

\end{enumerate}
\solution
		%\input{ncert/10/15/1/14/main.tex}
	\item Five cards—the ten, jack, queen, king and ace of diamonds, are well-shuffled with their face downwards. One card is then picked up at random.
\begin{enumerate}
\item
What is the probability that the card is the queen? 
\item
If the queen is drawn and put aside, what is the probability that the second card picked up is (a) an ace? (b) a queen?\\
\end{enumerate}
\solution
		%\input{ncert/10/15/1/15/defs.tex}
	\item A bag contains $5$ red balls and some blue balls. If the probability of drawing a blue ball is double that if a red ball, determine the number of blue balls in the bag. 
		\\
\solution
		%\input{ncert/10/15/2/3/defs.tex}
	\item A card is selected from a pack of 52 cards.
 \begin{enumerate}[label=(\alph*)] 
                 \item How many points are there in the sample space?
                 \item Calculate the probability that the card is an ace of spades.
                 \item Calculate the probability that the card is (i) an ace and (ii) black card.
 \end{enumerate}
\solution
		%\input{ncert/11/16/3/4/main.tex}
\item Four cards are drawn from a well-shuffled deck of 52 cards. What is the probability of obtaining 3 diamonds and one spade.
\\
\solution
		%\input{ncert/11/16/4/2/defs.tex}
\item In a certain lottery 10,000 tickets are sold and ten equal prizes are awarded. What is the probability of not getting a prize if you buy (a) one ticket (b) two tickets (c) 10 tickets ?	
\\
\solution
		%\input{ncert/11/16/4/4/defs.tex}
		%
\item 
Out of 100 students, two sections of 40 and 60 are formed. If you and your friend are among the 100 students, what is the probability that
\begin{enumerate}
\item you both enter the same section?
\item you both enter the different sections?
\end{enumerate}
\solution
		%\input{ncert/11/16/4/5/defs.tex}
	\item 
The number lock of a suitcase has 4 wheels each labelled with ten digits i.e. from 0 to 9.The lock opens with a sequence of four digits with no repeats.What is the probability of a person getting the right sequence to open the suitcase.
\\
\solution
		%\input{ncert/11/16/4/10/defs.tex}
		%
\item 
Two cards are drawn at random and without replacement from a pack of 52 playing cards. Find the probability that both the cards are black.
\\
\solution
		%\input{ncert/12/13/2/2/defs.tex}
		\item A box of oranges is inspected by examining three randomly selected oranges drawn without replacement. If all the three oranges are good, the box is approved for sale, otherwise, it is rejected. Find the probability that a box containing 15 oranges out of which 12 are good and 3 are bad ones will be approved for sale.
		\label{ncert/12/13/2/3/defs.tex}
		\item Two balls are drawn at random with replacement from a box containing 10 black and 8 red balls. Find the probability that
		\label{ncert/12/13/2/12}
\begin{enumerate}
\item both balls are red.
\item first ball is black and second is red.
\item one of them is black and other is red.
\end{enumerate}

\item In a hostel, 60\% of the students read Hindi newspaper, 40\% read English newspaper and 20\% read both Hindi and English newspapers. A student is selected at random.
		\label{ncert/12/13/2/15}
\begin{enumerate}
\item Find the probability that she reads neither Hindi nor English newspapers.
\item If she reads Hindi newspaper, find the probability that she reads English newspaper.
\item If she reads English newspaper, find the probability that she reads Hindi newspaper.\\
\end{enumerate}
\item The probability of obtaining an even prime number on each die, when a pair of dice is rolled is 
\begin{enumerate}
    \item $0$ 
    
    \item $\frac{1}{3}$ 
    
    \item $\frac{1}{12}$ 
    
    \item $\frac{1}{36}$ 
\end{enumerate}
\solution
		%\input{ncert/12/13/2/17/defs.tex}
	\item A bag contains 4 red and 4 black balls, another bag contains 2 red and 6 black balls. One of the two bags is selected at random and a ball is drawn from the bag which is found to be red. Find the probability that the ball is drawn from the first bag.
\\
\solution
		%\input{ncert/12/13/3/2/main.tex}
  \item
  Cards with numbers 2 to 101 are placed in a box. A card is selected at random.Find the probability that the card has
\begin{enumerate}[label=(\roman*)]
	\item an even number 
	\item a square number
\end{enumerate}
\solution
%\input{exemplar/10/13/3/32/main.tex}
\item
The king, queen and jack of clubs are removed from a deck of 52 playing cards and then well shuffled. Now one card is drawn at random from the remaining cards.  Determine the probability that the card is
\begin{enumerate}[label=(\roman*)]
\item a club
\item 10 of hearts
\end{enumerate}
\solution
%\input{exemplar/10/13/3/29/main.tex}
\item A team of medical students doing their internship have to assist during surgeries
at a city hospital. The probabilities of surgeries rated as very complex, complex,
routine, simple or very simple are respectively, 0.15, 0.20, 0.31, 0.26, .08. Find
the probabilities that a particular surgery will be rated
\begin{enumerate}
	\item complex or very complex;
	\item neither very complex nor very simple;
	\item routine or complex
	\item routine or simple
\end{enumerate}
\solution
%\input{exemplar/11/16/3/8(1)/main.tex}
\item A card is selected from a pack of 52 cards.
\begin{enumerate}[label=(\alph*)]
    \item How many points are there in the sample space?
    \item Calculate the probability that the card is an ace of spades.
    \item Calculate the probability that the card is (i) an ace and (ii) black card.
\end{enumerate}
\solution
%\input{exemplar/11/16/3/4/main2.tex}
\item The probability that a non leap year selected at random will contain 53 sundays.
\\
\solution
%\input{exemplar/10/13/1/19/main.tex}
\item One of the four persons John, Rita, Aslam or Gurpreet will be promoted next
month. Consequently the sample space consists of four elementary outcomes
S = {John promoted, Rita promoted, Aslam promoted, Gurpreet promoted}
You are told that the chances of John’s promotion is same as that of Gurpreet,
Rita’s chances of promotion are twice as likely as Johns. Aslam’s chances are
four times that of John.
\begin{enumerate}
	\item Determine
	\begin{enumerate}
		\item P (John promoted)
		\item P (Rita promoted)
		\item P (Aslam promoted)
		\item P (Gurpreet promoted)
	\end{enumerate}
	\item If A = {John promoted or Gurpreet promoted}, find P (A).
\end{enumerate}
\solution
%\input{exemplar/11/16/3/10/main.tex}
\item A card is drawn from a deck of 52 cards. Find the probability of getting a king or a heart or a red card.\\
\solution
%\input{exemplar/11/16/3/15/main.tex}
\item The probability that a student will pass his examination is 0.73, the probability of
the student getting a compartment is 0.13, and the probability that the student will
either pass or get compartment is 0.96. State True or False.\\
\solution
%\input{exemplar/11/16/3/31/main.tex}
\item A card is selected from a pack of 52 cards\\
\begin{enumerate}[label=(\alph*)]
\item How many points are there in the sample space?
\item Calculate the probability that the cards is an ace of spades.
\item Calculate the probability that the card is (i) an ace (ii)black card.\\
\end{enumerate}
%\input{ncert/11/16/3/4_1/Prob_4.tex}
\item In a non-leap year, the probability of having 53 tuesdays or 53 wednesdays is\\
\solution
%\input{exemplar/11/16/3/18/main.tex}
\item There are 1000 sealed envelopes in a box, 10 of them contain a cash prize of
Rs 100 each, 100 of them contain a cash prize of Rs 50 each and 200 of them
contain a cash prize of Rs 10 each and rest do not contain any cash prize. If they
are well shuffled and an envelope is picked up out, what is the probability that it
contains no cash prize?\\
\solution
%\input{exemplar/10/13/3/34/main.tex}
\item 
A die is thrown and a card is selected at random from a deck of 52 playing cards. The probability of getting an even number on the die and a spade card.\\
\solution
%\input{exemplar/12/13/3/78/main.tex}
\item
If 4-digit numbers greater than 5,000 are randomly formed from the digits 0, 1, 3, 5, and 7, what is the probability of forming a number divisible by 5 when:
\begin{enumerate}
    \item The digits are repeated?
    \item The repetition of digits is not allowed?
\end{enumerate}
\solution
%\input{ncert/11/16/4/9/main.tex}
\item Consider the probability space $\brak{\Omega, \mathcal{G}, P}$ where $\Omega = [0,2]$ and $\mathcal{G} = \cbrak{\phi, \Omega, [0,1], (1,2]}$. Let $X$ and $Y$ be two functions on $\Omega$ defined as
\begin{align*}
    X(\omega) = 
    \begin{cases}
        1 & \text{if }\omega \in [0, 1]\\
        2 & \text{if }\omega \in (1, 2]
    \end{cases}
\end{align*}
and
\begin{align*}
    Y(\omega) = 
    \begin{cases}
        2 & \text{if }\omega \in [0, 1.5]\\
        3 & \text{if }\omega \in (1.5, 2].
    \end{cases}
\end{align*}
Then which one of the following statements is true?
\begin{enumerate}
    \item [(A)] $X$ is a random variable with respect to $\mathcal{G}$, but $Y$ is not a random variable with respect to $\mathcal{G}$.
    \item [(B)] $Y$ is a random variable with respect to $\mathcal{G}$, but $X$ is not a random variable with respect to $\mathcal{G}$.
    \item [(C)] Neither $X$ nor $Y$ is a random variable with respect to $\mathcal{G}$.
    \item [(D)] Both $X$ and $Y$ are random variables with respect to $\mathcal{G}$.
\end{enumerate} \hfill (GATE ST 2023)\\
\solution
%\input{gate/ST/2023/14/main.tex}
	\item  A die is loaded in such a way that each odd number is twice as likely to occur as
each even number. Find $P(G)$, where $G$ is the event that a number greater than
3 occurs on a single roll of the die.
\\
\solution
		%\input{exemplar/11/16/3/5/main.tex}
	\item All the jacks, queens and kings are removed from a deck of 52 playing cards. The remaining cards are well shuffled and then one card is drawn at random. Giving ace a value 1 similar value for other cards, find the probability that the card has a value 
		\begin{enumerate}
			\item 7
			\item greater than 7
			\item less than 7
		\end{enumerate}
		%\input{exemplar/10/13/3/30/main.tex}
  \item A Lot consists of 48 mobile phones of which 42 are good, 3 have only minor defects and 3 have major defects.Varnika will buy a phone if it is good but the trader will only buy a mobile if it has no major defects. One phone is selected at random from the lot. What is the probability that it is
\begin{enumerate}
	\item acceptable to Varnika?
            \item acceptable to the trader?
\end{enumerate}
\solution
	%\input{exemplar/10/13/3/40/main.tex}
 \item A student says that if you throw a die, it will show up 1 or not 1. Therefore, the probability of getting 1 and the probability of getting 'not 1' each is equal to $\frac{1}{2}$. Is this correct? Give reasons.\\
 \solution
        %\input{exemplar/10/13/2/9/main.tex}
   \item Four candidates A, B, C, D have ap-
plied for the assignment to coach a school cricket
team. If A is twice as likely to be selected as B, and
B and C are given about the same chance of being
selected, while C is twice as likely to be selected
as D, what are the probabilities that
\begin{enumerate}
\item C will be selected?
\item A will not be selected?
\end{enumerate}
	%\input{exemplar/11/16/3/9/main.tex}
 \item A bag contain 24 balls of which $x$ balls are red, $2x$ are white and $3x$ are blue. A ball is selected at random, What is the probability that it is
\begin{enumerate}[label=\alph*)]
\item not red ?
\item white ?
\end{enumerate}
%\input{exemplar/10/13/3/41/main.tex}
If the letters of the word ASSASSINATION are arranged at random. Find the Probability that
\begin{enumerate}[label=(\alph*)]
\item Four $S's$ come consecutively in the word
\item Two  $I's$ and two $N's$ come together
\item All $A's$ are not coming together
\item No two $A's$ are coming together
\end{enumerate}
%\input{exemplar/11/16/3/14/main.tex}
	\item One urn contains two black balls (labelled B1 and B2) and one white ball. A
	second urn contains one black ball and two white balls (labelled W1 and W2).
	Suppose the following experiment is performed. One of the two urns is chosen
	at random. Next a ball is randomly chosen from the urn. Then a second ball is
	chosen at random from the same urn without replacing the first ball.
	
	\begin{enumerate}
	\item What is the probability that two black balls are chosen?
	
	\item What is the probability that two balls of opposite colour are chosen?
	\end{enumerate}
	\solution
	%\input{exemplar/11/16/3/12/main1.tex}
\end{enumerate}

	\item A card is selected from a pack of 52 cards.
 \begin{enumerate}[label=(\alph*)] 
                 \item How many points are there in the sample space?
                 \item Calculate the probability that the card is an ace of spades.
                 \item Calculate the probability that the card is (i) an ace and (ii) black card.
 \end{enumerate}
\solution
		%\begin{table}[H]
	\centering
\begin{tabular}{|c|c|c|}
\hline
Random variable &Value &Definition\\ \hline
\multirow{3}{*}{X} &0 &Slips of Rs 1\\
&1 &Slips of Rs 5\\
&2 &Slips of Rs 13\\ \hline
\multirow{2}{*}{Y} &0 &Box A\\
&1 &Box B\\\hline
\end{tabular}
\caption{}
\label{tab:Distribution}
\end{table}
See \tabref{tab:Distribution}.
\begin{align}
p_{Y}\brak{k}= \begin{cases} 
      \frac{1}{3} & {k=0} \\
      \frac{2}{3 }& {k=1} 
   \end{cases}
   \\
p_{Y|X}\brak{0|0} = \frac{19}{25}\, 
p_{Y|X}\brak{0|1} = \frac{6}{25}\,
p_{Y|X}\brak{1|0} = \frac{45}{50}\,
p_{Y|X}\brak{1|2} = \frac{5}{50}
\end{align}
The desired probability is the probability that a slip drawn at random is marked other than Rs 1,
\begin{align}
&=1-p_X\brak{0}\\
&= p_X(1) + p_X(2)
\end{align}
Using Bayes theorem,
\begin{align}
&= p_Y\brak{0} \times \pr{Y=0 | X=1} + p_Y\brak{1} \times \pr{Y=1|X=2}\\
&=\frac{1}{3} \times \frac{6}{25} + \frac{2}{3} \times \frac{5}{50}\\
&=\frac{11}{75}
\end{align}

\newpage

%\tableofcontents

\bigskip

\renewcommand{\thefigure}{\theenumi}
\renewcommand{\thetable}{\theenumi}
%\renewcommand{\theequation}{\theenumi}

%\begin{abstract}
%%\boldmath
%In this letter, an algorithm for evaluating the exact analytical bit error rate  (BER)  for the piecewise linear (PL) combiner for  multiple relays is presented. Previous results were available only for upto three relays. The algorithm is unique in the sense that  the actual mathematical expressions, that are prohibitively large, need not be explicitly obtained. The diversity gain due to multiple relays is shown through plots of the analytical BER, well supported by simulations. 
%
%\end{abstract}
% IEEEtran.cls defaults to using nonbold math in the Abstract.
% This preserves the distinction between vectors and scalars. However,
% if the journal you are submitting to favors bold math in the abstract,
% then you can use LaTeX's standard command \boldmath at the very start
% of the abstract to achieve this. Many IEEE journals frown on math
% in the abstract anyway.

% Note that keywords are not normally used for peerreview papers.
%\begin{IEEEkeywords}
%Cooperative diversity, decode and forward, piecewise linear
%\end{IEEEkeywords}



% For peer review papers, you can put extra information on the cover
% page as needed:
% \ifCLASSOPTIONpeerreview
% \begin{center} \bfseries EDICS Category: 3-BBND \end{center}
% \fi
%
% For peerreview papers, this IEEEtran command inserts a page break and
% creates the second title. It will be ignored for other modes.
%\IEEEpeerreviewmaketitle




\item Four cards are drawn from a well-shuffled deck of 52 cards. What is the probability of obtaining 3 diamonds and one spade.
\\
\solution
		%\begin{enumerate}[label=\thesection.\arabic*,ref=\thesection.\theenumi]
	\item One card is drawn from a well-shuffled deck of 52 cards. Find the probability of getting
\begin{enumerate}
\item A king of red colour 
\item A face card 
\item A red face card
\item The jack of hearts
\item A spade
\item The queen of diamonds

\end{enumerate}
\solution
		%\input{ncert/10/15/1/14/main.tex}
	\item Five cards—the ten, jack, queen, king and ace of diamonds, are well-shuffled with their face downwards. One card is then picked up at random.
\begin{enumerate}
\item
What is the probability that the card is the queen? 
\item
If the queen is drawn and put aside, what is the probability that the second card picked up is (a) an ace? (b) a queen?\\
\end{enumerate}
\solution
		%\input{ncert/10/15/1/15/defs.tex}
	\item A bag contains $5$ red balls and some blue balls. If the probability of drawing a blue ball is double that if a red ball, determine the number of blue balls in the bag. 
		\\
\solution
		%\input{ncert/10/15/2/3/defs.tex}
	\item A card is selected from a pack of 52 cards.
 \begin{enumerate}[label=(\alph*)] 
                 \item How many points are there in the sample space?
                 \item Calculate the probability that the card is an ace of spades.
                 \item Calculate the probability that the card is (i) an ace and (ii) black card.
 \end{enumerate}
\solution
		%\input{ncert/11/16/3/4/main.tex}
\item Four cards are drawn from a well-shuffled deck of 52 cards. What is the probability of obtaining 3 diamonds and one spade.
\\
\solution
		%\input{ncert/11/16/4/2/defs.tex}
\item In a certain lottery 10,000 tickets are sold and ten equal prizes are awarded. What is the probability of not getting a prize if you buy (a) one ticket (b) two tickets (c) 10 tickets ?	
\\
\solution
		%\input{ncert/11/16/4/4/defs.tex}
		%
\item 
Out of 100 students, two sections of 40 and 60 are formed. If you and your friend are among the 100 students, what is the probability that
\begin{enumerate}
\item you both enter the same section?
\item you both enter the different sections?
\end{enumerate}
\solution
		%\input{ncert/11/16/4/5/defs.tex}
	\item 
The number lock of a suitcase has 4 wheels each labelled with ten digits i.e. from 0 to 9.The lock opens with a sequence of four digits with no repeats.What is the probability of a person getting the right sequence to open the suitcase.
\\
\solution
		%\input{ncert/11/16/4/10/defs.tex}
		%
\item 
Two cards are drawn at random and without replacement from a pack of 52 playing cards. Find the probability that both the cards are black.
\\
\solution
		%\input{ncert/12/13/2/2/defs.tex}
		\item A box of oranges is inspected by examining three randomly selected oranges drawn without replacement. If all the three oranges are good, the box is approved for sale, otherwise, it is rejected. Find the probability that a box containing 15 oranges out of which 12 are good and 3 are bad ones will be approved for sale.
		\label{ncert/12/13/2/3/defs.tex}
		\item Two balls are drawn at random with replacement from a box containing 10 black and 8 red balls. Find the probability that
		\label{ncert/12/13/2/12}
\begin{enumerate}
\item both balls are red.
\item first ball is black and second is red.
\item one of them is black and other is red.
\end{enumerate}

\item In a hostel, 60\% of the students read Hindi newspaper, 40\% read English newspaper and 20\% read both Hindi and English newspapers. A student is selected at random.
		\label{ncert/12/13/2/15}
\begin{enumerate}
\item Find the probability that she reads neither Hindi nor English newspapers.
\item If she reads Hindi newspaper, find the probability that she reads English newspaper.
\item If she reads English newspaper, find the probability that she reads Hindi newspaper.\\
\end{enumerate}
\item The probability of obtaining an even prime number on each die, when a pair of dice is rolled is 
\begin{enumerate}
    \item $0$ 
    
    \item $\frac{1}{3}$ 
    
    \item $\frac{1}{12}$ 
    
    \item $\frac{1}{36}$ 
\end{enumerate}
\solution
		%\input{ncert/12/13/2/17/defs.tex}
	\item A bag contains 4 red and 4 black balls, another bag contains 2 red and 6 black balls. One of the two bags is selected at random and a ball is drawn from the bag which is found to be red. Find the probability that the ball is drawn from the first bag.
\\
\solution
		%\input{ncert/12/13/3/2/main.tex}
  \item
  Cards with numbers 2 to 101 are placed in a box. A card is selected at random.Find the probability that the card has
\begin{enumerate}[label=(\roman*)]
	\item an even number 
	\item a square number
\end{enumerate}
\solution
%\input{exemplar/10/13/3/32/main.tex}
\item
The king, queen and jack of clubs are removed from a deck of 52 playing cards and then well shuffled. Now one card is drawn at random from the remaining cards.  Determine the probability that the card is
\begin{enumerate}[label=(\roman*)]
\item a club
\item 10 of hearts
\end{enumerate}
\solution
%\input{exemplar/10/13/3/29/main.tex}
\item A team of medical students doing their internship have to assist during surgeries
at a city hospital. The probabilities of surgeries rated as very complex, complex,
routine, simple or very simple are respectively, 0.15, 0.20, 0.31, 0.26, .08. Find
the probabilities that a particular surgery will be rated
\begin{enumerate}
	\item complex or very complex;
	\item neither very complex nor very simple;
	\item routine or complex
	\item routine or simple
\end{enumerate}
\solution
%\input{exemplar/11/16/3/8(1)/main.tex}
\item A card is selected from a pack of 52 cards.
\begin{enumerate}[label=(\alph*)]
    \item How many points are there in the sample space?
    \item Calculate the probability that the card is an ace of spades.
    \item Calculate the probability that the card is (i) an ace and (ii) black card.
\end{enumerate}
\solution
%\input{exemplar/11/16/3/4/main2.tex}
\item The probability that a non leap year selected at random will contain 53 sundays.
\\
\solution
%\input{exemplar/10/13/1/19/main.tex}
\item One of the four persons John, Rita, Aslam or Gurpreet will be promoted next
month. Consequently the sample space consists of four elementary outcomes
S = {John promoted, Rita promoted, Aslam promoted, Gurpreet promoted}
You are told that the chances of John’s promotion is same as that of Gurpreet,
Rita’s chances of promotion are twice as likely as Johns. Aslam’s chances are
four times that of John.
\begin{enumerate}
	\item Determine
	\begin{enumerate}
		\item P (John promoted)
		\item P (Rita promoted)
		\item P (Aslam promoted)
		\item P (Gurpreet promoted)
	\end{enumerate}
	\item If A = {John promoted or Gurpreet promoted}, find P (A).
\end{enumerate}
\solution
%\input{exemplar/11/16/3/10/main.tex}
\item A card is drawn from a deck of 52 cards. Find the probability of getting a king or a heart or a red card.\\
\solution
%\input{exemplar/11/16/3/15/main.tex}
\item The probability that a student will pass his examination is 0.73, the probability of
the student getting a compartment is 0.13, and the probability that the student will
either pass or get compartment is 0.96. State True or False.\\
\solution
%\input{exemplar/11/16/3/31/main.tex}
\item A card is selected from a pack of 52 cards\\
\begin{enumerate}[label=(\alph*)]
\item How many points are there in the sample space?
\item Calculate the probability that the cards is an ace of spades.
\item Calculate the probability that the card is (i) an ace (ii)black card.\\
\end{enumerate}
%\input{ncert/11/16/3/4_1/Prob_4.tex}
\item In a non-leap year, the probability of having 53 tuesdays or 53 wednesdays is\\
\solution
%\input{exemplar/11/16/3/18/main.tex}
\item There are 1000 sealed envelopes in a box, 10 of them contain a cash prize of
Rs 100 each, 100 of them contain a cash prize of Rs 50 each and 200 of them
contain a cash prize of Rs 10 each and rest do not contain any cash prize. If they
are well shuffled and an envelope is picked up out, what is the probability that it
contains no cash prize?\\
\solution
%\input{exemplar/10/13/3/34/main.tex}
\item 
A die is thrown and a card is selected at random from a deck of 52 playing cards. The probability of getting an even number on the die and a spade card.\\
\solution
%\input{exemplar/12/13/3/78/main.tex}
\item
If 4-digit numbers greater than 5,000 are randomly formed from the digits 0, 1, 3, 5, and 7, what is the probability of forming a number divisible by 5 when:
\begin{enumerate}
    \item The digits are repeated?
    \item The repetition of digits is not allowed?
\end{enumerate}
\solution
%\input{ncert/11/16/4/9/main.tex}
\item Consider the probability space $\brak{\Omega, \mathcal{G}, P}$ where $\Omega = [0,2]$ and $\mathcal{G} = \cbrak{\phi, \Omega, [0,1], (1,2]}$. Let $X$ and $Y$ be two functions on $\Omega$ defined as
\begin{align*}
    X(\omega) = 
    \begin{cases}
        1 & \text{if }\omega \in [0, 1]\\
        2 & \text{if }\omega \in (1, 2]
    \end{cases}
\end{align*}
and
\begin{align*}
    Y(\omega) = 
    \begin{cases}
        2 & \text{if }\omega \in [0, 1.5]\\
        3 & \text{if }\omega \in (1.5, 2].
    \end{cases}
\end{align*}
Then which one of the following statements is true?
\begin{enumerate}
    \item [(A)] $X$ is a random variable with respect to $\mathcal{G}$, but $Y$ is not a random variable with respect to $\mathcal{G}$.
    \item [(B)] $Y$ is a random variable with respect to $\mathcal{G}$, but $X$ is not a random variable with respect to $\mathcal{G}$.
    \item [(C)] Neither $X$ nor $Y$ is a random variable with respect to $\mathcal{G}$.
    \item [(D)] Both $X$ and $Y$ are random variables with respect to $\mathcal{G}$.
\end{enumerate} \hfill (GATE ST 2023)\\
\solution
%\input{gate/ST/2023/14/main.tex}
	\item  A die is loaded in such a way that each odd number is twice as likely to occur as
each even number. Find $P(G)$, where $G$ is the event that a number greater than
3 occurs on a single roll of the die.
\\
\solution
		%\input{exemplar/11/16/3/5/main.tex}
	\item All the jacks, queens and kings are removed from a deck of 52 playing cards. The remaining cards are well shuffled and then one card is drawn at random. Giving ace a value 1 similar value for other cards, find the probability that the card has a value 
		\begin{enumerate}
			\item 7
			\item greater than 7
			\item less than 7
		\end{enumerate}
		%\input{exemplar/10/13/3/30/main.tex}
  \item A Lot consists of 48 mobile phones of which 42 are good, 3 have only minor defects and 3 have major defects.Varnika will buy a phone if it is good but the trader will only buy a mobile if it has no major defects. One phone is selected at random from the lot. What is the probability that it is
\begin{enumerate}
	\item acceptable to Varnika?
            \item acceptable to the trader?
\end{enumerate}
\solution
	%\input{exemplar/10/13/3/40/main.tex}
 \item A student says that if you throw a die, it will show up 1 or not 1. Therefore, the probability of getting 1 and the probability of getting 'not 1' each is equal to $\frac{1}{2}$. Is this correct? Give reasons.\\
 \solution
        %\input{exemplar/10/13/2/9/main.tex}
   \item Four candidates A, B, C, D have ap-
plied for the assignment to coach a school cricket
team. If A is twice as likely to be selected as B, and
B and C are given about the same chance of being
selected, while C is twice as likely to be selected
as D, what are the probabilities that
\begin{enumerate}
\item C will be selected?
\item A will not be selected?
\end{enumerate}
	%\input{exemplar/11/16/3/9/main.tex}
 \item A bag contain 24 balls of which $x$ balls are red, $2x$ are white and $3x$ are blue. A ball is selected at random, What is the probability that it is
\begin{enumerate}[label=\alph*)]
\item not red ?
\item white ?
\end{enumerate}
%\input{exemplar/10/13/3/41/main.tex}
If the letters of the word ASSASSINATION are arranged at random. Find the Probability that
\begin{enumerate}[label=(\alph*)]
\item Four $S's$ come consecutively in the word
\item Two  $I's$ and two $N's$ come together
\item All $A's$ are not coming together
\item No two $A's$ are coming together
\end{enumerate}
%\input{exemplar/11/16/3/14/main.tex}
	\item One urn contains two black balls (labelled B1 and B2) and one white ball. A
	second urn contains one black ball and two white balls (labelled W1 and W2).
	Suppose the following experiment is performed. One of the two urns is chosen
	at random. Next a ball is randomly chosen from the urn. Then a second ball is
	chosen at random from the same urn without replacing the first ball.
	
	\begin{enumerate}
	\item What is the probability that two black balls are chosen?
	
	\item What is the probability that two balls of opposite colour are chosen?
	\end{enumerate}
	\solution
	%\input{exemplar/11/16/3/12/main1.tex}
\end{enumerate}

\item In a certain lottery 10,000 tickets are sold and ten equal prizes are awarded. What is the probability of not getting a prize if you buy (a) one ticket (b) two tickets (c) 10 tickets ?	
\\
\solution
		%\begin{enumerate}[label=\thesection.\arabic*,ref=\thesection.\theenumi]
	\item One card is drawn from a well-shuffled deck of 52 cards. Find the probability of getting
\begin{enumerate}
\item A king of red colour 
\item A face card 
\item A red face card
\item The jack of hearts
\item A spade
\item The queen of diamonds

\end{enumerate}
\solution
		%\input{ncert/10/15/1/14/main.tex}
	\item Five cards—the ten, jack, queen, king and ace of diamonds, are well-shuffled with their face downwards. One card is then picked up at random.
\begin{enumerate}
\item
What is the probability that the card is the queen? 
\item
If the queen is drawn and put aside, what is the probability that the second card picked up is (a) an ace? (b) a queen?\\
\end{enumerate}
\solution
		%\input{ncert/10/15/1/15/defs.tex}
	\item A bag contains $5$ red balls and some blue balls. If the probability of drawing a blue ball is double that if a red ball, determine the number of blue balls in the bag. 
		\\
\solution
		%\input{ncert/10/15/2/3/defs.tex}
	\item A card is selected from a pack of 52 cards.
 \begin{enumerate}[label=(\alph*)] 
                 \item How many points are there in the sample space?
                 \item Calculate the probability that the card is an ace of spades.
                 \item Calculate the probability that the card is (i) an ace and (ii) black card.
 \end{enumerate}
\solution
		%\input{ncert/11/16/3/4/main.tex}
\item Four cards are drawn from a well-shuffled deck of 52 cards. What is the probability of obtaining 3 diamonds and one spade.
\\
\solution
		%\input{ncert/11/16/4/2/defs.tex}
\item In a certain lottery 10,000 tickets are sold and ten equal prizes are awarded. What is the probability of not getting a prize if you buy (a) one ticket (b) two tickets (c) 10 tickets ?	
\\
\solution
		%\input{ncert/11/16/4/4/defs.tex}
		%
\item 
Out of 100 students, two sections of 40 and 60 are formed. If you and your friend are among the 100 students, what is the probability that
\begin{enumerate}
\item you both enter the same section?
\item you both enter the different sections?
\end{enumerate}
\solution
		%\input{ncert/11/16/4/5/defs.tex}
	\item 
The number lock of a suitcase has 4 wheels each labelled with ten digits i.e. from 0 to 9.The lock opens with a sequence of four digits with no repeats.What is the probability of a person getting the right sequence to open the suitcase.
\\
\solution
		%\input{ncert/11/16/4/10/defs.tex}
		%
\item 
Two cards are drawn at random and without replacement from a pack of 52 playing cards. Find the probability that both the cards are black.
\\
\solution
		%\input{ncert/12/13/2/2/defs.tex}
		\item A box of oranges is inspected by examining three randomly selected oranges drawn without replacement. If all the three oranges are good, the box is approved for sale, otherwise, it is rejected. Find the probability that a box containing 15 oranges out of which 12 are good and 3 are bad ones will be approved for sale.
		\label{ncert/12/13/2/3/defs.tex}
		\item Two balls are drawn at random with replacement from a box containing 10 black and 8 red balls. Find the probability that
		\label{ncert/12/13/2/12}
\begin{enumerate}
\item both balls are red.
\item first ball is black and second is red.
\item one of them is black and other is red.
\end{enumerate}

\item In a hostel, 60\% of the students read Hindi newspaper, 40\% read English newspaper and 20\% read both Hindi and English newspapers. A student is selected at random.
		\label{ncert/12/13/2/15}
\begin{enumerate}
\item Find the probability that she reads neither Hindi nor English newspapers.
\item If she reads Hindi newspaper, find the probability that she reads English newspaper.
\item If she reads English newspaper, find the probability that she reads Hindi newspaper.\\
\end{enumerate}
\item The probability of obtaining an even prime number on each die, when a pair of dice is rolled is 
\begin{enumerate}
    \item $0$ 
    
    \item $\frac{1}{3}$ 
    
    \item $\frac{1}{12}$ 
    
    \item $\frac{1}{36}$ 
\end{enumerate}
\solution
		%\input{ncert/12/13/2/17/defs.tex}
	\item A bag contains 4 red and 4 black balls, another bag contains 2 red and 6 black balls. One of the two bags is selected at random and a ball is drawn from the bag which is found to be red. Find the probability that the ball is drawn from the first bag.
\\
\solution
		%\input{ncert/12/13/3/2/main.tex}
  \item
  Cards with numbers 2 to 101 are placed in a box. A card is selected at random.Find the probability that the card has
\begin{enumerate}[label=(\roman*)]
	\item an even number 
	\item a square number
\end{enumerate}
\solution
%\input{exemplar/10/13/3/32/main.tex}
\item
The king, queen and jack of clubs are removed from a deck of 52 playing cards and then well shuffled. Now one card is drawn at random from the remaining cards.  Determine the probability that the card is
\begin{enumerate}[label=(\roman*)]
\item a club
\item 10 of hearts
\end{enumerate}
\solution
%\input{exemplar/10/13/3/29/main.tex}
\item A team of medical students doing their internship have to assist during surgeries
at a city hospital. The probabilities of surgeries rated as very complex, complex,
routine, simple or very simple are respectively, 0.15, 0.20, 0.31, 0.26, .08. Find
the probabilities that a particular surgery will be rated
\begin{enumerate}
	\item complex or very complex;
	\item neither very complex nor very simple;
	\item routine or complex
	\item routine or simple
\end{enumerate}
\solution
%\input{exemplar/11/16/3/8(1)/main.tex}
\item A card is selected from a pack of 52 cards.
\begin{enumerate}[label=(\alph*)]
    \item How many points are there in the sample space?
    \item Calculate the probability that the card is an ace of spades.
    \item Calculate the probability that the card is (i) an ace and (ii) black card.
\end{enumerate}
\solution
%\input{exemplar/11/16/3/4/main2.tex}
\item The probability that a non leap year selected at random will contain 53 sundays.
\\
\solution
%\input{exemplar/10/13/1/19/main.tex}
\item One of the four persons John, Rita, Aslam or Gurpreet will be promoted next
month. Consequently the sample space consists of four elementary outcomes
S = {John promoted, Rita promoted, Aslam promoted, Gurpreet promoted}
You are told that the chances of John’s promotion is same as that of Gurpreet,
Rita’s chances of promotion are twice as likely as Johns. Aslam’s chances are
four times that of John.
\begin{enumerate}
	\item Determine
	\begin{enumerate}
		\item P (John promoted)
		\item P (Rita promoted)
		\item P (Aslam promoted)
		\item P (Gurpreet promoted)
	\end{enumerate}
	\item If A = {John promoted or Gurpreet promoted}, find P (A).
\end{enumerate}
\solution
%\input{exemplar/11/16/3/10/main.tex}
\item A card is drawn from a deck of 52 cards. Find the probability of getting a king or a heart or a red card.\\
\solution
%\input{exemplar/11/16/3/15/main.tex}
\item The probability that a student will pass his examination is 0.73, the probability of
the student getting a compartment is 0.13, and the probability that the student will
either pass or get compartment is 0.96. State True or False.\\
\solution
%\input{exemplar/11/16/3/31/main.tex}
\item A card is selected from a pack of 52 cards\\
\begin{enumerate}[label=(\alph*)]
\item How many points are there in the sample space?
\item Calculate the probability that the cards is an ace of spades.
\item Calculate the probability that the card is (i) an ace (ii)black card.\\
\end{enumerate}
%\input{ncert/11/16/3/4_1/Prob_4.tex}
\item In a non-leap year, the probability of having 53 tuesdays or 53 wednesdays is\\
\solution
%\input{exemplar/11/16/3/18/main.tex}
\item There are 1000 sealed envelopes in a box, 10 of them contain a cash prize of
Rs 100 each, 100 of them contain a cash prize of Rs 50 each and 200 of them
contain a cash prize of Rs 10 each and rest do not contain any cash prize. If they
are well shuffled and an envelope is picked up out, what is the probability that it
contains no cash prize?\\
\solution
%\input{exemplar/10/13/3/34/main.tex}
\item 
A die is thrown and a card is selected at random from a deck of 52 playing cards. The probability of getting an even number on the die and a spade card.\\
\solution
%\input{exemplar/12/13/3/78/main.tex}
\item
If 4-digit numbers greater than 5,000 are randomly formed from the digits 0, 1, 3, 5, and 7, what is the probability of forming a number divisible by 5 when:
\begin{enumerate}
    \item The digits are repeated?
    \item The repetition of digits is not allowed?
\end{enumerate}
\solution
%\input{ncert/11/16/4/9/main.tex}
\item Consider the probability space $\brak{\Omega, \mathcal{G}, P}$ where $\Omega = [0,2]$ and $\mathcal{G} = \cbrak{\phi, \Omega, [0,1], (1,2]}$. Let $X$ and $Y$ be two functions on $\Omega$ defined as
\begin{align*}
    X(\omega) = 
    \begin{cases}
        1 & \text{if }\omega \in [0, 1]\\
        2 & \text{if }\omega \in (1, 2]
    \end{cases}
\end{align*}
and
\begin{align*}
    Y(\omega) = 
    \begin{cases}
        2 & \text{if }\omega \in [0, 1.5]\\
        3 & \text{if }\omega \in (1.5, 2].
    \end{cases}
\end{align*}
Then which one of the following statements is true?
\begin{enumerate}
    \item [(A)] $X$ is a random variable with respect to $\mathcal{G}$, but $Y$ is not a random variable with respect to $\mathcal{G}$.
    \item [(B)] $Y$ is a random variable with respect to $\mathcal{G}$, but $X$ is not a random variable with respect to $\mathcal{G}$.
    \item [(C)] Neither $X$ nor $Y$ is a random variable with respect to $\mathcal{G}$.
    \item [(D)] Both $X$ and $Y$ are random variables with respect to $\mathcal{G}$.
\end{enumerate} \hfill (GATE ST 2023)\\
\solution
%\input{gate/ST/2023/14/main.tex}
	\item  A die is loaded in such a way that each odd number is twice as likely to occur as
each even number. Find $P(G)$, where $G$ is the event that a number greater than
3 occurs on a single roll of the die.
\\
\solution
		%\input{exemplar/11/16/3/5/main.tex}
	\item All the jacks, queens and kings are removed from a deck of 52 playing cards. The remaining cards are well shuffled and then one card is drawn at random. Giving ace a value 1 similar value for other cards, find the probability that the card has a value 
		\begin{enumerate}
			\item 7
			\item greater than 7
			\item less than 7
		\end{enumerate}
		%\input{exemplar/10/13/3/30/main.tex}
  \item A Lot consists of 48 mobile phones of which 42 are good, 3 have only minor defects and 3 have major defects.Varnika will buy a phone if it is good but the trader will only buy a mobile if it has no major defects. One phone is selected at random from the lot. What is the probability that it is
\begin{enumerate}
	\item acceptable to Varnika?
            \item acceptable to the trader?
\end{enumerate}
\solution
	%\input{exemplar/10/13/3/40/main.tex}
 \item A student says that if you throw a die, it will show up 1 or not 1. Therefore, the probability of getting 1 and the probability of getting 'not 1' each is equal to $\frac{1}{2}$. Is this correct? Give reasons.\\
 \solution
        %\input{exemplar/10/13/2/9/main.tex}
   \item Four candidates A, B, C, D have ap-
plied for the assignment to coach a school cricket
team. If A is twice as likely to be selected as B, and
B and C are given about the same chance of being
selected, while C is twice as likely to be selected
as D, what are the probabilities that
\begin{enumerate}
\item C will be selected?
\item A will not be selected?
\end{enumerate}
	%\input{exemplar/11/16/3/9/main.tex}
 \item A bag contain 24 balls of which $x$ balls are red, $2x$ are white and $3x$ are blue. A ball is selected at random, What is the probability that it is
\begin{enumerate}[label=\alph*)]
\item not red ?
\item white ?
\end{enumerate}
%\input{exemplar/10/13/3/41/main.tex}
If the letters of the word ASSASSINATION are arranged at random. Find the Probability that
\begin{enumerate}[label=(\alph*)]
\item Four $S's$ come consecutively in the word
\item Two  $I's$ and two $N's$ come together
\item All $A's$ are not coming together
\item No two $A's$ are coming together
\end{enumerate}
%\input{exemplar/11/16/3/14/main.tex}
	\item One urn contains two black balls (labelled B1 and B2) and one white ball. A
	second urn contains one black ball and two white balls (labelled W1 and W2).
	Suppose the following experiment is performed. One of the two urns is chosen
	at random. Next a ball is randomly chosen from the urn. Then a second ball is
	chosen at random from the same urn without replacing the first ball.
	
	\begin{enumerate}
	\item What is the probability that two black balls are chosen?
	
	\item What is the probability that two balls of opposite colour are chosen?
	\end{enumerate}
	\solution
	%\input{exemplar/11/16/3/12/main1.tex}
\end{enumerate}

		%
\item 
Out of 100 students, two sections of 40 and 60 are formed. If you and your friend are among the 100 students, what is the probability that
\begin{enumerate}
\item you both enter the same section?
\item you both enter the different sections?
\end{enumerate}
\solution
		%\begin{enumerate}[label=\thesection.\arabic*,ref=\thesection.\theenumi]
	\item One card is drawn from a well-shuffled deck of 52 cards. Find the probability of getting
\begin{enumerate}
\item A king of red colour 
\item A face card 
\item A red face card
\item The jack of hearts
\item A spade
\item The queen of diamonds

\end{enumerate}
\solution
		%\input{ncert/10/15/1/14/main.tex}
	\item Five cards—the ten, jack, queen, king and ace of diamonds, are well-shuffled with their face downwards. One card is then picked up at random.
\begin{enumerate}
\item
What is the probability that the card is the queen? 
\item
If the queen is drawn and put aside, what is the probability that the second card picked up is (a) an ace? (b) a queen?\\
\end{enumerate}
\solution
		%\input{ncert/10/15/1/15/defs.tex}
	\item A bag contains $5$ red balls and some blue balls. If the probability of drawing a blue ball is double that if a red ball, determine the number of blue balls in the bag. 
		\\
\solution
		%\input{ncert/10/15/2/3/defs.tex}
	\item A card is selected from a pack of 52 cards.
 \begin{enumerate}[label=(\alph*)] 
                 \item How many points are there in the sample space?
                 \item Calculate the probability that the card is an ace of spades.
                 \item Calculate the probability that the card is (i) an ace and (ii) black card.
 \end{enumerate}
\solution
		%\input{ncert/11/16/3/4/main.tex}
\item Four cards are drawn from a well-shuffled deck of 52 cards. What is the probability of obtaining 3 diamonds and one spade.
\\
\solution
		%\input{ncert/11/16/4/2/defs.tex}
\item In a certain lottery 10,000 tickets are sold and ten equal prizes are awarded. What is the probability of not getting a prize if you buy (a) one ticket (b) two tickets (c) 10 tickets ?	
\\
\solution
		%\input{ncert/11/16/4/4/defs.tex}
		%
\item 
Out of 100 students, two sections of 40 and 60 are formed. If you and your friend are among the 100 students, what is the probability that
\begin{enumerate}
\item you both enter the same section?
\item you both enter the different sections?
\end{enumerate}
\solution
		%\input{ncert/11/16/4/5/defs.tex}
	\item 
The number lock of a suitcase has 4 wheels each labelled with ten digits i.e. from 0 to 9.The lock opens with a sequence of four digits with no repeats.What is the probability of a person getting the right sequence to open the suitcase.
\\
\solution
		%\input{ncert/11/16/4/10/defs.tex}
		%
\item 
Two cards are drawn at random and without replacement from a pack of 52 playing cards. Find the probability that both the cards are black.
\\
\solution
		%\input{ncert/12/13/2/2/defs.tex}
		\item A box of oranges is inspected by examining three randomly selected oranges drawn without replacement. If all the three oranges are good, the box is approved for sale, otherwise, it is rejected. Find the probability that a box containing 15 oranges out of which 12 are good and 3 are bad ones will be approved for sale.
		\label{ncert/12/13/2/3/defs.tex}
		\item Two balls are drawn at random with replacement from a box containing 10 black and 8 red balls. Find the probability that
		\label{ncert/12/13/2/12}
\begin{enumerate}
\item both balls are red.
\item first ball is black and second is red.
\item one of them is black and other is red.
\end{enumerate}

\item In a hostel, 60\% of the students read Hindi newspaper, 40\% read English newspaper and 20\% read both Hindi and English newspapers. A student is selected at random.
		\label{ncert/12/13/2/15}
\begin{enumerate}
\item Find the probability that she reads neither Hindi nor English newspapers.
\item If she reads Hindi newspaper, find the probability that she reads English newspaper.
\item If she reads English newspaper, find the probability that she reads Hindi newspaper.\\
\end{enumerate}
\item The probability of obtaining an even prime number on each die, when a pair of dice is rolled is 
\begin{enumerate}
    \item $0$ 
    
    \item $\frac{1}{3}$ 
    
    \item $\frac{1}{12}$ 
    
    \item $\frac{1}{36}$ 
\end{enumerate}
\solution
		%\input{ncert/12/13/2/17/defs.tex}
	\item A bag contains 4 red and 4 black balls, another bag contains 2 red and 6 black balls. One of the two bags is selected at random and a ball is drawn from the bag which is found to be red. Find the probability that the ball is drawn from the first bag.
\\
\solution
		%\input{ncert/12/13/3/2/main.tex}
  \item
  Cards with numbers 2 to 101 are placed in a box. A card is selected at random.Find the probability that the card has
\begin{enumerate}[label=(\roman*)]
	\item an even number 
	\item a square number
\end{enumerate}
\solution
%\input{exemplar/10/13/3/32/main.tex}
\item
The king, queen and jack of clubs are removed from a deck of 52 playing cards and then well shuffled. Now one card is drawn at random from the remaining cards.  Determine the probability that the card is
\begin{enumerate}[label=(\roman*)]
\item a club
\item 10 of hearts
\end{enumerate}
\solution
%\input{exemplar/10/13/3/29/main.tex}
\item A team of medical students doing their internship have to assist during surgeries
at a city hospital. The probabilities of surgeries rated as very complex, complex,
routine, simple or very simple are respectively, 0.15, 0.20, 0.31, 0.26, .08. Find
the probabilities that a particular surgery will be rated
\begin{enumerate}
	\item complex or very complex;
	\item neither very complex nor very simple;
	\item routine or complex
	\item routine or simple
\end{enumerate}
\solution
%\input{exemplar/11/16/3/8(1)/main.tex}
\item A card is selected from a pack of 52 cards.
\begin{enumerate}[label=(\alph*)]
    \item How many points are there in the sample space?
    \item Calculate the probability that the card is an ace of spades.
    \item Calculate the probability that the card is (i) an ace and (ii) black card.
\end{enumerate}
\solution
%\input{exemplar/11/16/3/4/main2.tex}
\item The probability that a non leap year selected at random will contain 53 sundays.
\\
\solution
%\input{exemplar/10/13/1/19/main.tex}
\item One of the four persons John, Rita, Aslam or Gurpreet will be promoted next
month. Consequently the sample space consists of four elementary outcomes
S = {John promoted, Rita promoted, Aslam promoted, Gurpreet promoted}
You are told that the chances of John’s promotion is same as that of Gurpreet,
Rita’s chances of promotion are twice as likely as Johns. Aslam’s chances are
four times that of John.
\begin{enumerate}
	\item Determine
	\begin{enumerate}
		\item P (John promoted)
		\item P (Rita promoted)
		\item P (Aslam promoted)
		\item P (Gurpreet promoted)
	\end{enumerate}
	\item If A = {John promoted or Gurpreet promoted}, find P (A).
\end{enumerate}
\solution
%\input{exemplar/11/16/3/10/main.tex}
\item A card is drawn from a deck of 52 cards. Find the probability of getting a king or a heart or a red card.\\
\solution
%\input{exemplar/11/16/3/15/main.tex}
\item The probability that a student will pass his examination is 0.73, the probability of
the student getting a compartment is 0.13, and the probability that the student will
either pass or get compartment is 0.96. State True or False.\\
\solution
%\input{exemplar/11/16/3/31/main.tex}
\item A card is selected from a pack of 52 cards\\
\begin{enumerate}[label=(\alph*)]
\item How many points are there in the sample space?
\item Calculate the probability that the cards is an ace of spades.
\item Calculate the probability that the card is (i) an ace (ii)black card.\\
\end{enumerate}
%\input{ncert/11/16/3/4_1/Prob_4.tex}
\item In a non-leap year, the probability of having 53 tuesdays or 53 wednesdays is\\
\solution
%\input{exemplar/11/16/3/18/main.tex}
\item There are 1000 sealed envelopes in a box, 10 of them contain a cash prize of
Rs 100 each, 100 of them contain a cash prize of Rs 50 each and 200 of them
contain a cash prize of Rs 10 each and rest do not contain any cash prize. If they
are well shuffled and an envelope is picked up out, what is the probability that it
contains no cash prize?\\
\solution
%\input{exemplar/10/13/3/34/main.tex}
\item 
A die is thrown and a card is selected at random from a deck of 52 playing cards. The probability of getting an even number on the die and a spade card.\\
\solution
%\input{exemplar/12/13/3/78/main.tex}
\item
If 4-digit numbers greater than 5,000 are randomly formed from the digits 0, 1, 3, 5, and 7, what is the probability of forming a number divisible by 5 when:
\begin{enumerate}
    \item The digits are repeated?
    \item The repetition of digits is not allowed?
\end{enumerate}
\solution
%\input{ncert/11/16/4/9/main.tex}
\item Consider the probability space $\brak{\Omega, \mathcal{G}, P}$ where $\Omega = [0,2]$ and $\mathcal{G} = \cbrak{\phi, \Omega, [0,1], (1,2]}$. Let $X$ and $Y$ be two functions on $\Omega$ defined as
\begin{align*}
    X(\omega) = 
    \begin{cases}
        1 & \text{if }\omega \in [0, 1]\\
        2 & \text{if }\omega \in (1, 2]
    \end{cases}
\end{align*}
and
\begin{align*}
    Y(\omega) = 
    \begin{cases}
        2 & \text{if }\omega \in [0, 1.5]\\
        3 & \text{if }\omega \in (1.5, 2].
    \end{cases}
\end{align*}
Then which one of the following statements is true?
\begin{enumerate}
    \item [(A)] $X$ is a random variable with respect to $\mathcal{G}$, but $Y$ is not a random variable with respect to $\mathcal{G}$.
    \item [(B)] $Y$ is a random variable with respect to $\mathcal{G}$, but $X$ is not a random variable with respect to $\mathcal{G}$.
    \item [(C)] Neither $X$ nor $Y$ is a random variable with respect to $\mathcal{G}$.
    \item [(D)] Both $X$ and $Y$ are random variables with respect to $\mathcal{G}$.
\end{enumerate} \hfill (GATE ST 2023)\\
\solution
%\input{gate/ST/2023/14/main.tex}
	\item  A die is loaded in such a way that each odd number is twice as likely to occur as
each even number. Find $P(G)$, where $G$ is the event that a number greater than
3 occurs on a single roll of the die.
\\
\solution
		%\input{exemplar/11/16/3/5/main.tex}
	\item All the jacks, queens and kings are removed from a deck of 52 playing cards. The remaining cards are well shuffled and then one card is drawn at random. Giving ace a value 1 similar value for other cards, find the probability that the card has a value 
		\begin{enumerate}
			\item 7
			\item greater than 7
			\item less than 7
		\end{enumerate}
		%\input{exemplar/10/13/3/30/main.tex}
  \item A Lot consists of 48 mobile phones of which 42 are good, 3 have only minor defects and 3 have major defects.Varnika will buy a phone if it is good but the trader will only buy a mobile if it has no major defects. One phone is selected at random from the lot. What is the probability that it is
\begin{enumerate}
	\item acceptable to Varnika?
            \item acceptable to the trader?
\end{enumerate}
\solution
	%\input{exemplar/10/13/3/40/main.tex}
 \item A student says that if you throw a die, it will show up 1 or not 1. Therefore, the probability of getting 1 and the probability of getting 'not 1' each is equal to $\frac{1}{2}$. Is this correct? Give reasons.\\
 \solution
        %\input{exemplar/10/13/2/9/main.tex}
   \item Four candidates A, B, C, D have ap-
plied for the assignment to coach a school cricket
team. If A is twice as likely to be selected as B, and
B and C are given about the same chance of being
selected, while C is twice as likely to be selected
as D, what are the probabilities that
\begin{enumerate}
\item C will be selected?
\item A will not be selected?
\end{enumerate}
	%\input{exemplar/11/16/3/9/main.tex}
 \item A bag contain 24 balls of which $x$ balls are red, $2x$ are white and $3x$ are blue. A ball is selected at random, What is the probability that it is
\begin{enumerate}[label=\alph*)]
\item not red ?
\item white ?
\end{enumerate}
%\input{exemplar/10/13/3/41/main.tex}
If the letters of the word ASSASSINATION are arranged at random. Find the Probability that
\begin{enumerate}[label=(\alph*)]
\item Four $S's$ come consecutively in the word
\item Two  $I's$ and two $N's$ come together
\item All $A's$ are not coming together
\item No two $A's$ are coming together
\end{enumerate}
%\input{exemplar/11/16/3/14/main.tex}
	\item One urn contains two black balls (labelled B1 and B2) and one white ball. A
	second urn contains one black ball and two white balls (labelled W1 and W2).
	Suppose the following experiment is performed. One of the two urns is chosen
	at random. Next a ball is randomly chosen from the urn. Then a second ball is
	chosen at random from the same urn without replacing the first ball.
	
	\begin{enumerate}
	\item What is the probability that two black balls are chosen?
	
	\item What is the probability that two balls of opposite colour are chosen?
	\end{enumerate}
	\solution
	%\input{exemplar/11/16/3/12/main1.tex}
\end{enumerate}

	\item 
The number lock of a suitcase has 4 wheels each labelled with ten digits i.e. from 0 to 9.The lock opens with a sequence of four digits with no repeats.What is the probability of a person getting the right sequence to open the suitcase.
\\
\solution
		%\begin{enumerate}[label=\thesection.\arabic*,ref=\thesection.\theenumi]
	\item One card is drawn from a well-shuffled deck of 52 cards. Find the probability of getting
\begin{enumerate}
\item A king of red colour 
\item A face card 
\item A red face card
\item The jack of hearts
\item A spade
\item The queen of diamonds

\end{enumerate}
\solution
		%\input{ncert/10/15/1/14/main.tex}
	\item Five cards—the ten, jack, queen, king and ace of diamonds, are well-shuffled with their face downwards. One card is then picked up at random.
\begin{enumerate}
\item
What is the probability that the card is the queen? 
\item
If the queen is drawn and put aside, what is the probability that the second card picked up is (a) an ace? (b) a queen?\\
\end{enumerate}
\solution
		%\input{ncert/10/15/1/15/defs.tex}
	\item A bag contains $5$ red balls and some blue balls. If the probability of drawing a blue ball is double that if a red ball, determine the number of blue balls in the bag. 
		\\
\solution
		%\input{ncert/10/15/2/3/defs.tex}
	\item A card is selected from a pack of 52 cards.
 \begin{enumerate}[label=(\alph*)] 
                 \item How many points are there in the sample space?
                 \item Calculate the probability that the card is an ace of spades.
                 \item Calculate the probability that the card is (i) an ace and (ii) black card.
 \end{enumerate}
\solution
		%\input{ncert/11/16/3/4/main.tex}
\item Four cards are drawn from a well-shuffled deck of 52 cards. What is the probability of obtaining 3 diamonds and one spade.
\\
\solution
		%\input{ncert/11/16/4/2/defs.tex}
\item In a certain lottery 10,000 tickets are sold and ten equal prizes are awarded. What is the probability of not getting a prize if you buy (a) one ticket (b) two tickets (c) 10 tickets ?	
\\
\solution
		%\input{ncert/11/16/4/4/defs.tex}
		%
\item 
Out of 100 students, two sections of 40 and 60 are formed. If you and your friend are among the 100 students, what is the probability that
\begin{enumerate}
\item you both enter the same section?
\item you both enter the different sections?
\end{enumerate}
\solution
		%\input{ncert/11/16/4/5/defs.tex}
	\item 
The number lock of a suitcase has 4 wheels each labelled with ten digits i.e. from 0 to 9.The lock opens with a sequence of four digits with no repeats.What is the probability of a person getting the right sequence to open the suitcase.
\\
\solution
		%\input{ncert/11/16/4/10/defs.tex}
		%
\item 
Two cards are drawn at random and without replacement from a pack of 52 playing cards. Find the probability that both the cards are black.
\\
\solution
		%\input{ncert/12/13/2/2/defs.tex}
		\item A box of oranges is inspected by examining three randomly selected oranges drawn without replacement. If all the three oranges are good, the box is approved for sale, otherwise, it is rejected. Find the probability that a box containing 15 oranges out of which 12 are good and 3 are bad ones will be approved for sale.
		\label{ncert/12/13/2/3/defs.tex}
		\item Two balls are drawn at random with replacement from a box containing 10 black and 8 red balls. Find the probability that
		\label{ncert/12/13/2/12}
\begin{enumerate}
\item both balls are red.
\item first ball is black and second is red.
\item one of them is black and other is red.
\end{enumerate}

\item In a hostel, 60\% of the students read Hindi newspaper, 40\% read English newspaper and 20\% read both Hindi and English newspapers. A student is selected at random.
		\label{ncert/12/13/2/15}
\begin{enumerate}
\item Find the probability that she reads neither Hindi nor English newspapers.
\item If she reads Hindi newspaper, find the probability that she reads English newspaper.
\item If she reads English newspaper, find the probability that she reads Hindi newspaper.\\
\end{enumerate}
\item The probability of obtaining an even prime number on each die, when a pair of dice is rolled is 
\begin{enumerate}
    \item $0$ 
    
    \item $\frac{1}{3}$ 
    
    \item $\frac{1}{12}$ 
    
    \item $\frac{1}{36}$ 
\end{enumerate}
\solution
		%\input{ncert/12/13/2/17/defs.tex}
	\item A bag contains 4 red and 4 black balls, another bag contains 2 red and 6 black balls. One of the two bags is selected at random and a ball is drawn from the bag which is found to be red. Find the probability that the ball is drawn from the first bag.
\\
\solution
		%\input{ncert/12/13/3/2/main.tex}
  \item
  Cards with numbers 2 to 101 are placed in a box. A card is selected at random.Find the probability that the card has
\begin{enumerate}[label=(\roman*)]
	\item an even number 
	\item a square number
\end{enumerate}
\solution
%\input{exemplar/10/13/3/32/main.tex}
\item
The king, queen and jack of clubs are removed from a deck of 52 playing cards and then well shuffled. Now one card is drawn at random from the remaining cards.  Determine the probability that the card is
\begin{enumerate}[label=(\roman*)]
\item a club
\item 10 of hearts
\end{enumerate}
\solution
%\input{exemplar/10/13/3/29/main.tex}
\item A team of medical students doing their internship have to assist during surgeries
at a city hospital. The probabilities of surgeries rated as very complex, complex,
routine, simple or very simple are respectively, 0.15, 0.20, 0.31, 0.26, .08. Find
the probabilities that a particular surgery will be rated
\begin{enumerate}
	\item complex or very complex;
	\item neither very complex nor very simple;
	\item routine or complex
	\item routine or simple
\end{enumerate}
\solution
%\input{exemplar/11/16/3/8(1)/main.tex}
\item A card is selected from a pack of 52 cards.
\begin{enumerate}[label=(\alph*)]
    \item How many points are there in the sample space?
    \item Calculate the probability that the card is an ace of spades.
    \item Calculate the probability that the card is (i) an ace and (ii) black card.
\end{enumerate}
\solution
%\input{exemplar/11/16/3/4/main2.tex}
\item The probability that a non leap year selected at random will contain 53 sundays.
\\
\solution
%\input{exemplar/10/13/1/19/main.tex}
\item One of the four persons John, Rita, Aslam or Gurpreet will be promoted next
month. Consequently the sample space consists of four elementary outcomes
S = {John promoted, Rita promoted, Aslam promoted, Gurpreet promoted}
You are told that the chances of John’s promotion is same as that of Gurpreet,
Rita’s chances of promotion are twice as likely as Johns. Aslam’s chances are
four times that of John.
\begin{enumerate}
	\item Determine
	\begin{enumerate}
		\item P (John promoted)
		\item P (Rita promoted)
		\item P (Aslam promoted)
		\item P (Gurpreet promoted)
	\end{enumerate}
	\item If A = {John promoted or Gurpreet promoted}, find P (A).
\end{enumerate}
\solution
%\input{exemplar/11/16/3/10/main.tex}
\item A card is drawn from a deck of 52 cards. Find the probability of getting a king or a heart or a red card.\\
\solution
%\input{exemplar/11/16/3/15/main.tex}
\item The probability that a student will pass his examination is 0.73, the probability of
the student getting a compartment is 0.13, and the probability that the student will
either pass or get compartment is 0.96. State True or False.\\
\solution
%\input{exemplar/11/16/3/31/main.tex}
\item A card is selected from a pack of 52 cards\\
\begin{enumerate}[label=(\alph*)]
\item How many points are there in the sample space?
\item Calculate the probability that the cards is an ace of spades.
\item Calculate the probability that the card is (i) an ace (ii)black card.\\
\end{enumerate}
%\input{ncert/11/16/3/4_1/Prob_4.tex}
\item In a non-leap year, the probability of having 53 tuesdays or 53 wednesdays is\\
\solution
%\input{exemplar/11/16/3/18/main.tex}
\item There are 1000 sealed envelopes in a box, 10 of them contain a cash prize of
Rs 100 each, 100 of them contain a cash prize of Rs 50 each and 200 of them
contain a cash prize of Rs 10 each and rest do not contain any cash prize. If they
are well shuffled and an envelope is picked up out, what is the probability that it
contains no cash prize?\\
\solution
%\input{exemplar/10/13/3/34/main.tex}
\item 
A die is thrown and a card is selected at random from a deck of 52 playing cards. The probability of getting an even number on the die and a spade card.\\
\solution
%\input{exemplar/12/13/3/78/main.tex}
\item
If 4-digit numbers greater than 5,000 are randomly formed from the digits 0, 1, 3, 5, and 7, what is the probability of forming a number divisible by 5 when:
\begin{enumerate}
    \item The digits are repeated?
    \item The repetition of digits is not allowed?
\end{enumerate}
\solution
%\input{ncert/11/16/4/9/main.tex}
\item Consider the probability space $\brak{\Omega, \mathcal{G}, P}$ where $\Omega = [0,2]$ and $\mathcal{G} = \cbrak{\phi, \Omega, [0,1], (1,2]}$. Let $X$ and $Y$ be two functions on $\Omega$ defined as
\begin{align*}
    X(\omega) = 
    \begin{cases}
        1 & \text{if }\omega \in [0, 1]\\
        2 & \text{if }\omega \in (1, 2]
    \end{cases}
\end{align*}
and
\begin{align*}
    Y(\omega) = 
    \begin{cases}
        2 & \text{if }\omega \in [0, 1.5]\\
        3 & \text{if }\omega \in (1.5, 2].
    \end{cases}
\end{align*}
Then which one of the following statements is true?
\begin{enumerate}
    \item [(A)] $X$ is a random variable with respect to $\mathcal{G}$, but $Y$ is not a random variable with respect to $\mathcal{G}$.
    \item [(B)] $Y$ is a random variable with respect to $\mathcal{G}$, but $X$ is not a random variable with respect to $\mathcal{G}$.
    \item [(C)] Neither $X$ nor $Y$ is a random variable with respect to $\mathcal{G}$.
    \item [(D)] Both $X$ and $Y$ are random variables with respect to $\mathcal{G}$.
\end{enumerate} \hfill (GATE ST 2023)\\
\solution
%\input{gate/ST/2023/14/main.tex}
	\item  A die is loaded in such a way that each odd number is twice as likely to occur as
each even number. Find $P(G)$, where $G$ is the event that a number greater than
3 occurs on a single roll of the die.
\\
\solution
		%\input{exemplar/11/16/3/5/main.tex}
	\item All the jacks, queens and kings are removed from a deck of 52 playing cards. The remaining cards are well shuffled and then one card is drawn at random. Giving ace a value 1 similar value for other cards, find the probability that the card has a value 
		\begin{enumerate}
			\item 7
			\item greater than 7
			\item less than 7
		\end{enumerate}
		%\input{exemplar/10/13/3/30/main.tex}
  \item A Lot consists of 48 mobile phones of which 42 are good, 3 have only minor defects and 3 have major defects.Varnika will buy a phone if it is good but the trader will only buy a mobile if it has no major defects. One phone is selected at random from the lot. What is the probability that it is
\begin{enumerate}
	\item acceptable to Varnika?
            \item acceptable to the trader?
\end{enumerate}
\solution
	%\input{exemplar/10/13/3/40/main.tex}
 \item A student says that if you throw a die, it will show up 1 or not 1. Therefore, the probability of getting 1 and the probability of getting 'not 1' each is equal to $\frac{1}{2}$. Is this correct? Give reasons.\\
 \solution
        %\input{exemplar/10/13/2/9/main.tex}
   \item Four candidates A, B, C, D have ap-
plied for the assignment to coach a school cricket
team. If A is twice as likely to be selected as B, and
B and C are given about the same chance of being
selected, while C is twice as likely to be selected
as D, what are the probabilities that
\begin{enumerate}
\item C will be selected?
\item A will not be selected?
\end{enumerate}
	%\input{exemplar/11/16/3/9/main.tex}
 \item A bag contain 24 balls of which $x$ balls are red, $2x$ are white and $3x$ are blue. A ball is selected at random, What is the probability that it is
\begin{enumerate}[label=\alph*)]
\item not red ?
\item white ?
\end{enumerate}
%\input{exemplar/10/13/3/41/main.tex}
If the letters of the word ASSASSINATION are arranged at random. Find the Probability that
\begin{enumerate}[label=(\alph*)]
\item Four $S's$ come consecutively in the word
\item Two  $I's$ and two $N's$ come together
\item All $A's$ are not coming together
\item No two $A's$ are coming together
\end{enumerate}
%\input{exemplar/11/16/3/14/main.tex}
	\item One urn contains two black balls (labelled B1 and B2) and one white ball. A
	second urn contains one black ball and two white balls (labelled W1 and W2).
	Suppose the following experiment is performed. One of the two urns is chosen
	at random. Next a ball is randomly chosen from the urn. Then a second ball is
	chosen at random from the same urn without replacing the first ball.
	
	\begin{enumerate}
	\item What is the probability that two black balls are chosen?
	
	\item What is the probability that two balls of opposite colour are chosen?
	\end{enumerate}
	\solution
	%\input{exemplar/11/16/3/12/main1.tex}
\end{enumerate}

		%
\item 
Two cards are drawn at random and without replacement from a pack of 52 playing cards. Find the probability that both the cards are black.
\\
\solution
		%\begin{enumerate}[label=\thesection.\arabic*,ref=\thesection.\theenumi]
	\item One card is drawn from a well-shuffled deck of 52 cards. Find the probability of getting
\begin{enumerate}
\item A king of red colour 
\item A face card 
\item A red face card
\item The jack of hearts
\item A spade
\item The queen of diamonds

\end{enumerate}
\solution
		%\input{ncert/10/15/1/14/main.tex}
	\item Five cards—the ten, jack, queen, king and ace of diamonds, are well-shuffled with their face downwards. One card is then picked up at random.
\begin{enumerate}
\item
What is the probability that the card is the queen? 
\item
If the queen is drawn and put aside, what is the probability that the second card picked up is (a) an ace? (b) a queen?\\
\end{enumerate}
\solution
		%\input{ncert/10/15/1/15/defs.tex}
	\item A bag contains $5$ red balls and some blue balls. If the probability of drawing a blue ball is double that if a red ball, determine the number of blue balls in the bag. 
		\\
\solution
		%\input{ncert/10/15/2/3/defs.tex}
	\item A card is selected from a pack of 52 cards.
 \begin{enumerate}[label=(\alph*)] 
                 \item How many points are there in the sample space?
                 \item Calculate the probability that the card is an ace of spades.
                 \item Calculate the probability that the card is (i) an ace and (ii) black card.
 \end{enumerate}
\solution
		%\input{ncert/11/16/3/4/main.tex}
\item Four cards are drawn from a well-shuffled deck of 52 cards. What is the probability of obtaining 3 diamonds and one spade.
\\
\solution
		%\input{ncert/11/16/4/2/defs.tex}
\item In a certain lottery 10,000 tickets are sold and ten equal prizes are awarded. What is the probability of not getting a prize if you buy (a) one ticket (b) two tickets (c) 10 tickets ?	
\\
\solution
		%\input{ncert/11/16/4/4/defs.tex}
		%
\item 
Out of 100 students, two sections of 40 and 60 are formed. If you and your friend are among the 100 students, what is the probability that
\begin{enumerate}
\item you both enter the same section?
\item you both enter the different sections?
\end{enumerate}
\solution
		%\input{ncert/11/16/4/5/defs.tex}
	\item 
The number lock of a suitcase has 4 wheels each labelled with ten digits i.e. from 0 to 9.The lock opens with a sequence of four digits with no repeats.What is the probability of a person getting the right sequence to open the suitcase.
\\
\solution
		%\input{ncert/11/16/4/10/defs.tex}
		%
\item 
Two cards are drawn at random and without replacement from a pack of 52 playing cards. Find the probability that both the cards are black.
\\
\solution
		%\input{ncert/12/13/2/2/defs.tex}
		\item A box of oranges is inspected by examining three randomly selected oranges drawn without replacement. If all the three oranges are good, the box is approved for sale, otherwise, it is rejected. Find the probability that a box containing 15 oranges out of which 12 are good and 3 are bad ones will be approved for sale.
		\label{ncert/12/13/2/3/defs.tex}
		\item Two balls are drawn at random with replacement from a box containing 10 black and 8 red balls. Find the probability that
		\label{ncert/12/13/2/12}
\begin{enumerate}
\item both balls are red.
\item first ball is black and second is red.
\item one of them is black and other is red.
\end{enumerate}

\item In a hostel, 60\% of the students read Hindi newspaper, 40\% read English newspaper and 20\% read both Hindi and English newspapers. A student is selected at random.
		\label{ncert/12/13/2/15}
\begin{enumerate}
\item Find the probability that she reads neither Hindi nor English newspapers.
\item If she reads Hindi newspaper, find the probability that she reads English newspaper.
\item If she reads English newspaper, find the probability that she reads Hindi newspaper.\\
\end{enumerate}
\item The probability of obtaining an even prime number on each die, when a pair of dice is rolled is 
\begin{enumerate}
    \item $0$ 
    
    \item $\frac{1}{3}$ 
    
    \item $\frac{1}{12}$ 
    
    \item $\frac{1}{36}$ 
\end{enumerate}
\solution
		%\input{ncert/12/13/2/17/defs.tex}
	\item A bag contains 4 red and 4 black balls, another bag contains 2 red and 6 black balls. One of the two bags is selected at random and a ball is drawn from the bag which is found to be red. Find the probability that the ball is drawn from the first bag.
\\
\solution
		%\input{ncert/12/13/3/2/main.tex}
  \item
  Cards with numbers 2 to 101 are placed in a box. A card is selected at random.Find the probability that the card has
\begin{enumerate}[label=(\roman*)]
	\item an even number 
	\item a square number
\end{enumerate}
\solution
%\input{exemplar/10/13/3/32/main.tex}
\item
The king, queen and jack of clubs are removed from a deck of 52 playing cards and then well shuffled. Now one card is drawn at random from the remaining cards.  Determine the probability that the card is
\begin{enumerate}[label=(\roman*)]
\item a club
\item 10 of hearts
\end{enumerate}
\solution
%\input{exemplar/10/13/3/29/main.tex}
\item A team of medical students doing their internship have to assist during surgeries
at a city hospital. The probabilities of surgeries rated as very complex, complex,
routine, simple or very simple are respectively, 0.15, 0.20, 0.31, 0.26, .08. Find
the probabilities that a particular surgery will be rated
\begin{enumerate}
	\item complex or very complex;
	\item neither very complex nor very simple;
	\item routine or complex
	\item routine or simple
\end{enumerate}
\solution
%\input{exemplar/11/16/3/8(1)/main.tex}
\item A card is selected from a pack of 52 cards.
\begin{enumerate}[label=(\alph*)]
    \item How many points are there in the sample space?
    \item Calculate the probability that the card is an ace of spades.
    \item Calculate the probability that the card is (i) an ace and (ii) black card.
\end{enumerate}
\solution
%\input{exemplar/11/16/3/4/main2.tex}
\item The probability that a non leap year selected at random will contain 53 sundays.
\\
\solution
%\input{exemplar/10/13/1/19/main.tex}
\item One of the four persons John, Rita, Aslam or Gurpreet will be promoted next
month. Consequently the sample space consists of four elementary outcomes
S = {John promoted, Rita promoted, Aslam promoted, Gurpreet promoted}
You are told that the chances of John’s promotion is same as that of Gurpreet,
Rita’s chances of promotion are twice as likely as Johns. Aslam’s chances are
four times that of John.
\begin{enumerate}
	\item Determine
	\begin{enumerate}
		\item P (John promoted)
		\item P (Rita promoted)
		\item P (Aslam promoted)
		\item P (Gurpreet promoted)
	\end{enumerate}
	\item If A = {John promoted or Gurpreet promoted}, find P (A).
\end{enumerate}
\solution
%\input{exemplar/11/16/3/10/main.tex}
\item A card is drawn from a deck of 52 cards. Find the probability of getting a king or a heart or a red card.\\
\solution
%\input{exemplar/11/16/3/15/main.tex}
\item The probability that a student will pass his examination is 0.73, the probability of
the student getting a compartment is 0.13, and the probability that the student will
either pass or get compartment is 0.96. State True or False.\\
\solution
%\input{exemplar/11/16/3/31/main.tex}
\item A card is selected from a pack of 52 cards\\
\begin{enumerate}[label=(\alph*)]
\item How many points are there in the sample space?
\item Calculate the probability that the cards is an ace of spades.
\item Calculate the probability that the card is (i) an ace (ii)black card.\\
\end{enumerate}
%\input{ncert/11/16/3/4_1/Prob_4.tex}
\item In a non-leap year, the probability of having 53 tuesdays or 53 wednesdays is\\
\solution
%\input{exemplar/11/16/3/18/main.tex}
\item There are 1000 sealed envelopes in a box, 10 of them contain a cash prize of
Rs 100 each, 100 of them contain a cash prize of Rs 50 each and 200 of them
contain a cash prize of Rs 10 each and rest do not contain any cash prize. If they
are well shuffled and an envelope is picked up out, what is the probability that it
contains no cash prize?\\
\solution
%\input{exemplar/10/13/3/34/main.tex}
\item 
A die is thrown and a card is selected at random from a deck of 52 playing cards. The probability of getting an even number on the die and a spade card.\\
\solution
%\input{exemplar/12/13/3/78/main.tex}
\item
If 4-digit numbers greater than 5,000 are randomly formed from the digits 0, 1, 3, 5, and 7, what is the probability of forming a number divisible by 5 when:
\begin{enumerate}
    \item The digits are repeated?
    \item The repetition of digits is not allowed?
\end{enumerate}
\solution
%\input{ncert/11/16/4/9/main.tex}
\item Consider the probability space $\brak{\Omega, \mathcal{G}, P}$ where $\Omega = [0,2]$ and $\mathcal{G} = \cbrak{\phi, \Omega, [0,1], (1,2]}$. Let $X$ and $Y$ be two functions on $\Omega$ defined as
\begin{align*}
    X(\omega) = 
    \begin{cases}
        1 & \text{if }\omega \in [0, 1]\\
        2 & \text{if }\omega \in (1, 2]
    \end{cases}
\end{align*}
and
\begin{align*}
    Y(\omega) = 
    \begin{cases}
        2 & \text{if }\omega \in [0, 1.5]\\
        3 & \text{if }\omega \in (1.5, 2].
    \end{cases}
\end{align*}
Then which one of the following statements is true?
\begin{enumerate}
    \item [(A)] $X$ is a random variable with respect to $\mathcal{G}$, but $Y$ is not a random variable with respect to $\mathcal{G}$.
    \item [(B)] $Y$ is a random variable with respect to $\mathcal{G}$, but $X$ is not a random variable with respect to $\mathcal{G}$.
    \item [(C)] Neither $X$ nor $Y$ is a random variable with respect to $\mathcal{G}$.
    \item [(D)] Both $X$ and $Y$ are random variables with respect to $\mathcal{G}$.
\end{enumerate} \hfill (GATE ST 2023)\\
\solution
%\input{gate/ST/2023/14/main.tex}
	\item  A die is loaded in such a way that each odd number is twice as likely to occur as
each even number. Find $P(G)$, where $G$ is the event that a number greater than
3 occurs on a single roll of the die.
\\
\solution
		%\input{exemplar/11/16/3/5/main.tex}
	\item All the jacks, queens and kings are removed from a deck of 52 playing cards. The remaining cards are well shuffled and then one card is drawn at random. Giving ace a value 1 similar value for other cards, find the probability that the card has a value 
		\begin{enumerate}
			\item 7
			\item greater than 7
			\item less than 7
		\end{enumerate}
		%\input{exemplar/10/13/3/30/main.tex}
  \item A Lot consists of 48 mobile phones of which 42 are good, 3 have only minor defects and 3 have major defects.Varnika will buy a phone if it is good but the trader will only buy a mobile if it has no major defects. One phone is selected at random from the lot. What is the probability that it is
\begin{enumerate}
	\item acceptable to Varnika?
            \item acceptable to the trader?
\end{enumerate}
\solution
	%\input{exemplar/10/13/3/40/main.tex}
 \item A student says that if you throw a die, it will show up 1 or not 1. Therefore, the probability of getting 1 and the probability of getting 'not 1' each is equal to $\frac{1}{2}$. Is this correct? Give reasons.\\
 \solution
        %\input{exemplar/10/13/2/9/main.tex}
   \item Four candidates A, B, C, D have ap-
plied for the assignment to coach a school cricket
team. If A is twice as likely to be selected as B, and
B and C are given about the same chance of being
selected, while C is twice as likely to be selected
as D, what are the probabilities that
\begin{enumerate}
\item C will be selected?
\item A will not be selected?
\end{enumerate}
	%\input{exemplar/11/16/3/9/main.tex}
 \item A bag contain 24 balls of which $x$ balls are red, $2x$ are white and $3x$ are blue. A ball is selected at random, What is the probability that it is
\begin{enumerate}[label=\alph*)]
\item not red ?
\item white ?
\end{enumerate}
%\input{exemplar/10/13/3/41/main.tex}
If the letters of the word ASSASSINATION are arranged at random. Find the Probability that
\begin{enumerate}[label=(\alph*)]
\item Four $S's$ come consecutively in the word
\item Two  $I's$ and two $N's$ come together
\item All $A's$ are not coming together
\item No two $A's$ are coming together
\end{enumerate}
%\input{exemplar/11/16/3/14/main.tex}
	\item One urn contains two black balls (labelled B1 and B2) and one white ball. A
	second urn contains one black ball and two white balls (labelled W1 and W2).
	Suppose the following experiment is performed. One of the two urns is chosen
	at random. Next a ball is randomly chosen from the urn. Then a second ball is
	chosen at random from the same urn without replacing the first ball.
	
	\begin{enumerate}
	\item What is the probability that two black balls are chosen?
	
	\item What is the probability that two balls of opposite colour are chosen?
	\end{enumerate}
	\solution
	%\input{exemplar/11/16/3/12/main1.tex}
\end{enumerate}

		\item A box of oranges is inspected by examining three randomly selected oranges drawn without replacement. If all the three oranges are good, the box is approved for sale, otherwise, it is rejected. Find the probability that a box containing 15 oranges out of which 12 are good and 3 are bad ones will be approved for sale.
		\label{ncert/12/13/2/3/defs.tex}
		\item Two balls are drawn at random with replacement from a box containing 10 black and 8 red balls. Find the probability that
		\label{ncert/12/13/2/12}
\begin{enumerate}
\item both balls are red.
\item first ball is black and second is red.
\item one of them is black and other is red.
\end{enumerate}

\item In a hostel, 60\% of the students read Hindi newspaper, 40\% read English newspaper and 20\% read both Hindi and English newspapers. A student is selected at random.
		\label{ncert/12/13/2/15}
\begin{enumerate}
\item Find the probability that she reads neither Hindi nor English newspapers.
\item If she reads Hindi newspaper, find the probability that she reads English newspaper.
\item If she reads English newspaper, find the probability that she reads Hindi newspaper.\\
\end{enumerate}
\item The probability of obtaining an even prime number on each die, when a pair of dice is rolled is 
\begin{enumerate}
    \item $0$ 
    
    \item $\frac{1}{3}$ 
    
    \item $\frac{1}{12}$ 
    
    \item $\frac{1}{36}$ 
\end{enumerate}
\solution
		%\begin{enumerate}[label=\thesection.\arabic*,ref=\thesection.\theenumi]
	\item One card is drawn from a well-shuffled deck of 52 cards. Find the probability of getting
\begin{enumerate}
\item A king of red colour 
\item A face card 
\item A red face card
\item The jack of hearts
\item A spade
\item The queen of diamonds

\end{enumerate}
\solution
		%\input{ncert/10/15/1/14/main.tex}
	\item Five cards—the ten, jack, queen, king and ace of diamonds, are well-shuffled with their face downwards. One card is then picked up at random.
\begin{enumerate}
\item
What is the probability that the card is the queen? 
\item
If the queen is drawn and put aside, what is the probability that the second card picked up is (a) an ace? (b) a queen?\\
\end{enumerate}
\solution
		%\input{ncert/10/15/1/15/defs.tex}
	\item A bag contains $5$ red balls and some blue balls. If the probability of drawing a blue ball is double that if a red ball, determine the number of blue balls in the bag. 
		\\
\solution
		%\input{ncert/10/15/2/3/defs.tex}
	\item A card is selected from a pack of 52 cards.
 \begin{enumerate}[label=(\alph*)] 
                 \item How many points are there in the sample space?
                 \item Calculate the probability that the card is an ace of spades.
                 \item Calculate the probability that the card is (i) an ace and (ii) black card.
 \end{enumerate}
\solution
		%\input{ncert/11/16/3/4/main.tex}
\item Four cards are drawn from a well-shuffled deck of 52 cards. What is the probability of obtaining 3 diamonds and one spade.
\\
\solution
		%\input{ncert/11/16/4/2/defs.tex}
\item In a certain lottery 10,000 tickets are sold and ten equal prizes are awarded. What is the probability of not getting a prize if you buy (a) one ticket (b) two tickets (c) 10 tickets ?	
\\
\solution
		%\input{ncert/11/16/4/4/defs.tex}
		%
\item 
Out of 100 students, two sections of 40 and 60 are formed. If you and your friend are among the 100 students, what is the probability that
\begin{enumerate}
\item you both enter the same section?
\item you both enter the different sections?
\end{enumerate}
\solution
		%\input{ncert/11/16/4/5/defs.tex}
	\item 
The number lock of a suitcase has 4 wheels each labelled with ten digits i.e. from 0 to 9.The lock opens with a sequence of four digits with no repeats.What is the probability of a person getting the right sequence to open the suitcase.
\\
\solution
		%\input{ncert/11/16/4/10/defs.tex}
		%
\item 
Two cards are drawn at random and without replacement from a pack of 52 playing cards. Find the probability that both the cards are black.
\\
\solution
		%\input{ncert/12/13/2/2/defs.tex}
		\item A box of oranges is inspected by examining three randomly selected oranges drawn without replacement. If all the three oranges are good, the box is approved for sale, otherwise, it is rejected. Find the probability that a box containing 15 oranges out of which 12 are good and 3 are bad ones will be approved for sale.
		\label{ncert/12/13/2/3/defs.tex}
		\item Two balls are drawn at random with replacement from a box containing 10 black and 8 red balls. Find the probability that
		\label{ncert/12/13/2/12}
\begin{enumerate}
\item both balls are red.
\item first ball is black and second is red.
\item one of them is black and other is red.
\end{enumerate}

\item In a hostel, 60\% of the students read Hindi newspaper, 40\% read English newspaper and 20\% read both Hindi and English newspapers. A student is selected at random.
		\label{ncert/12/13/2/15}
\begin{enumerate}
\item Find the probability that she reads neither Hindi nor English newspapers.
\item If she reads Hindi newspaper, find the probability that she reads English newspaper.
\item If she reads English newspaper, find the probability that she reads Hindi newspaper.\\
\end{enumerate}
\item The probability of obtaining an even prime number on each die, when a pair of dice is rolled is 
\begin{enumerate}
    \item $0$ 
    
    \item $\frac{1}{3}$ 
    
    \item $\frac{1}{12}$ 
    
    \item $\frac{1}{36}$ 
\end{enumerate}
\solution
		%\input{ncert/12/13/2/17/defs.tex}
	\item A bag contains 4 red and 4 black balls, another bag contains 2 red and 6 black balls. One of the two bags is selected at random and a ball is drawn from the bag which is found to be red. Find the probability that the ball is drawn from the first bag.
\\
\solution
		%\input{ncert/12/13/3/2/main.tex}
  \item
  Cards with numbers 2 to 101 are placed in a box. A card is selected at random.Find the probability that the card has
\begin{enumerate}[label=(\roman*)]
	\item an even number 
	\item a square number
\end{enumerate}
\solution
%\input{exemplar/10/13/3/32/main.tex}
\item
The king, queen and jack of clubs are removed from a deck of 52 playing cards and then well shuffled. Now one card is drawn at random from the remaining cards.  Determine the probability that the card is
\begin{enumerate}[label=(\roman*)]
\item a club
\item 10 of hearts
\end{enumerate}
\solution
%\input{exemplar/10/13/3/29/main.tex}
\item A team of medical students doing their internship have to assist during surgeries
at a city hospital. The probabilities of surgeries rated as very complex, complex,
routine, simple or very simple are respectively, 0.15, 0.20, 0.31, 0.26, .08. Find
the probabilities that a particular surgery will be rated
\begin{enumerate}
	\item complex or very complex;
	\item neither very complex nor very simple;
	\item routine or complex
	\item routine or simple
\end{enumerate}
\solution
%\input{exemplar/11/16/3/8(1)/main.tex}
\item A card is selected from a pack of 52 cards.
\begin{enumerate}[label=(\alph*)]
    \item How many points are there in the sample space?
    \item Calculate the probability that the card is an ace of spades.
    \item Calculate the probability that the card is (i) an ace and (ii) black card.
\end{enumerate}
\solution
%\input{exemplar/11/16/3/4/main2.tex}
\item The probability that a non leap year selected at random will contain 53 sundays.
\\
\solution
%\input{exemplar/10/13/1/19/main.tex}
\item One of the four persons John, Rita, Aslam or Gurpreet will be promoted next
month. Consequently the sample space consists of four elementary outcomes
S = {John promoted, Rita promoted, Aslam promoted, Gurpreet promoted}
You are told that the chances of John’s promotion is same as that of Gurpreet,
Rita’s chances of promotion are twice as likely as Johns. Aslam’s chances are
four times that of John.
\begin{enumerate}
	\item Determine
	\begin{enumerate}
		\item P (John promoted)
		\item P (Rita promoted)
		\item P (Aslam promoted)
		\item P (Gurpreet promoted)
	\end{enumerate}
	\item If A = {John promoted or Gurpreet promoted}, find P (A).
\end{enumerate}
\solution
%\input{exemplar/11/16/3/10/main.tex}
\item A card is drawn from a deck of 52 cards. Find the probability of getting a king or a heart or a red card.\\
\solution
%\input{exemplar/11/16/3/15/main.tex}
\item The probability that a student will pass his examination is 0.73, the probability of
the student getting a compartment is 0.13, and the probability that the student will
either pass or get compartment is 0.96. State True or False.\\
\solution
%\input{exemplar/11/16/3/31/main.tex}
\item A card is selected from a pack of 52 cards\\
\begin{enumerate}[label=(\alph*)]
\item How many points are there in the sample space?
\item Calculate the probability that the cards is an ace of spades.
\item Calculate the probability that the card is (i) an ace (ii)black card.\\
\end{enumerate}
%\input{ncert/11/16/3/4_1/Prob_4.tex}
\item In a non-leap year, the probability of having 53 tuesdays or 53 wednesdays is\\
\solution
%\input{exemplar/11/16/3/18/main.tex}
\item There are 1000 sealed envelopes in a box, 10 of them contain a cash prize of
Rs 100 each, 100 of them contain a cash prize of Rs 50 each and 200 of them
contain a cash prize of Rs 10 each and rest do not contain any cash prize. If they
are well shuffled and an envelope is picked up out, what is the probability that it
contains no cash prize?\\
\solution
%\input{exemplar/10/13/3/34/main.tex}
\item 
A die is thrown and a card is selected at random from a deck of 52 playing cards. The probability of getting an even number on the die and a spade card.\\
\solution
%\input{exemplar/12/13/3/78/main.tex}
\item
If 4-digit numbers greater than 5,000 are randomly formed from the digits 0, 1, 3, 5, and 7, what is the probability of forming a number divisible by 5 when:
\begin{enumerate}
    \item The digits are repeated?
    \item The repetition of digits is not allowed?
\end{enumerate}
\solution
%\input{ncert/11/16/4/9/main.tex}
\item Consider the probability space $\brak{\Omega, \mathcal{G}, P}$ where $\Omega = [0,2]$ and $\mathcal{G} = \cbrak{\phi, \Omega, [0,1], (1,2]}$. Let $X$ and $Y$ be two functions on $\Omega$ defined as
\begin{align*}
    X(\omega) = 
    \begin{cases}
        1 & \text{if }\omega \in [0, 1]\\
        2 & \text{if }\omega \in (1, 2]
    \end{cases}
\end{align*}
and
\begin{align*}
    Y(\omega) = 
    \begin{cases}
        2 & \text{if }\omega \in [0, 1.5]\\
        3 & \text{if }\omega \in (1.5, 2].
    \end{cases}
\end{align*}
Then which one of the following statements is true?
\begin{enumerate}
    \item [(A)] $X$ is a random variable with respect to $\mathcal{G}$, but $Y$ is not a random variable with respect to $\mathcal{G}$.
    \item [(B)] $Y$ is a random variable with respect to $\mathcal{G}$, but $X$ is not a random variable with respect to $\mathcal{G}$.
    \item [(C)] Neither $X$ nor $Y$ is a random variable with respect to $\mathcal{G}$.
    \item [(D)] Both $X$ and $Y$ are random variables with respect to $\mathcal{G}$.
\end{enumerate} \hfill (GATE ST 2023)\\
\solution
%\input{gate/ST/2023/14/main.tex}
	\item  A die is loaded in such a way that each odd number is twice as likely to occur as
each even number. Find $P(G)$, where $G$ is the event that a number greater than
3 occurs on a single roll of the die.
\\
\solution
		%\input{exemplar/11/16/3/5/main.tex}
	\item All the jacks, queens and kings are removed from a deck of 52 playing cards. The remaining cards are well shuffled and then one card is drawn at random. Giving ace a value 1 similar value for other cards, find the probability that the card has a value 
		\begin{enumerate}
			\item 7
			\item greater than 7
			\item less than 7
		\end{enumerate}
		%\input{exemplar/10/13/3/30/main.tex}
  \item A Lot consists of 48 mobile phones of which 42 are good, 3 have only minor defects and 3 have major defects.Varnika will buy a phone if it is good but the trader will only buy a mobile if it has no major defects. One phone is selected at random from the lot. What is the probability that it is
\begin{enumerate}
	\item acceptable to Varnika?
            \item acceptable to the trader?
\end{enumerate}
\solution
	%\input{exemplar/10/13/3/40/main.tex}
 \item A student says that if you throw a die, it will show up 1 or not 1. Therefore, the probability of getting 1 and the probability of getting 'not 1' each is equal to $\frac{1}{2}$. Is this correct? Give reasons.\\
 \solution
        %\input{exemplar/10/13/2/9/main.tex}
   \item Four candidates A, B, C, D have ap-
plied for the assignment to coach a school cricket
team. If A is twice as likely to be selected as B, and
B and C are given about the same chance of being
selected, while C is twice as likely to be selected
as D, what are the probabilities that
\begin{enumerate}
\item C will be selected?
\item A will not be selected?
\end{enumerate}
	%\input{exemplar/11/16/3/9/main.tex}
 \item A bag contain 24 balls of which $x$ balls are red, $2x$ are white and $3x$ are blue. A ball is selected at random, What is the probability that it is
\begin{enumerate}[label=\alph*)]
\item not red ?
\item white ?
\end{enumerate}
%\input{exemplar/10/13/3/41/main.tex}
If the letters of the word ASSASSINATION are arranged at random. Find the Probability that
\begin{enumerate}[label=(\alph*)]
\item Four $S's$ come consecutively in the word
\item Two  $I's$ and two $N's$ come together
\item All $A's$ are not coming together
\item No two $A's$ are coming together
\end{enumerate}
%\input{exemplar/11/16/3/14/main.tex}
	\item One urn contains two black balls (labelled B1 and B2) and one white ball. A
	second urn contains one black ball and two white balls (labelled W1 and W2).
	Suppose the following experiment is performed. One of the two urns is chosen
	at random. Next a ball is randomly chosen from the urn. Then a second ball is
	chosen at random from the same urn without replacing the first ball.
	
	\begin{enumerate}
	\item What is the probability that two black balls are chosen?
	
	\item What is the probability that two balls of opposite colour are chosen?
	\end{enumerate}
	\solution
	%\input{exemplar/11/16/3/12/main1.tex}
\end{enumerate}

	\item A bag contains 4 red and 4 black balls, another bag contains 2 red and 6 black balls. One of the two bags is selected at random and a ball is drawn from the bag which is found to be red. Find the probability that the ball is drawn from the first bag.
\\
\solution
		%\begin{table}[H]
	\centering
\begin{tabular}{|c|c|c|}
\hline
Random variable &Value &Definition\\ \hline
\multirow{3}{*}{X} &0 &Slips of Rs 1\\
&1 &Slips of Rs 5\\
&2 &Slips of Rs 13\\ \hline
\multirow{2}{*}{Y} &0 &Box A\\
&1 &Box B\\\hline
\end{tabular}
\caption{}
\label{tab:Distribution}
\end{table}
See \tabref{tab:Distribution}.
\begin{align}
p_{Y}\brak{k}= \begin{cases} 
      \frac{1}{3} & {k=0} \\
      \frac{2}{3 }& {k=1} 
   \end{cases}
   \\
p_{Y|X}\brak{0|0} = \frac{19}{25}\, 
p_{Y|X}\brak{0|1} = \frac{6}{25}\,
p_{Y|X}\brak{1|0} = \frac{45}{50}\,
p_{Y|X}\brak{1|2} = \frac{5}{50}
\end{align}
The desired probability is the probability that a slip drawn at random is marked other than Rs 1,
\begin{align}
&=1-p_X\brak{0}\\
&= p_X(1) + p_X(2)
\end{align}
Using Bayes theorem,
\begin{align}
&= p_Y\brak{0} \times \pr{Y=0 | X=1} + p_Y\brak{1} \times \pr{Y=1|X=2}\\
&=\frac{1}{3} \times \frac{6}{25} + \frac{2}{3} \times \frac{5}{50}\\
&=\frac{11}{75}
\end{align}

\newpage

%\tableofcontents

\bigskip

\renewcommand{\thefigure}{\theenumi}
\renewcommand{\thetable}{\theenumi}
%\renewcommand{\theequation}{\theenumi}

%\begin{abstract}
%%\boldmath
%In this letter, an algorithm for evaluating the exact analytical bit error rate  (BER)  for the piecewise linear (PL) combiner for  multiple relays is presented. Previous results were available only for upto three relays. The algorithm is unique in the sense that  the actual mathematical expressions, that are prohibitively large, need not be explicitly obtained. The diversity gain due to multiple relays is shown through plots of the analytical BER, well supported by simulations. 
%
%\end{abstract}
% IEEEtran.cls defaults to using nonbold math in the Abstract.
% This preserves the distinction between vectors and scalars. However,
% if the journal you are submitting to favors bold math in the abstract,
% then you can use LaTeX's standard command \boldmath at the very start
% of the abstract to achieve this. Many IEEE journals frown on math
% in the abstract anyway.

% Note that keywords are not normally used for peerreview papers.
%\begin{IEEEkeywords}
%Cooperative diversity, decode and forward, piecewise linear
%\end{IEEEkeywords}



% For peer review papers, you can put extra information on the cover
% page as needed:
% \ifCLASSOPTIONpeerreview
% \begin{center} \bfseries EDICS Category: 3-BBND \end{center}
% \fi
%
% For peerreview papers, this IEEEtran command inserts a page break and
% creates the second title. It will be ignored for other modes.
%\IEEEpeerreviewmaketitle




  \item
  Cards with numbers 2 to 101 are placed in a box. A card is selected at random.Find the probability that the card has
\begin{enumerate}[label=(\roman*)]
	\item an even number 
	\item a square number
\end{enumerate}
\solution
%\begin{table}[H]
	\centering
\begin{tabular}{|c|c|c|}
\hline
Random variable &Value &Definition\\ \hline
\multirow{3}{*}{X} &0 &Slips of Rs 1\\
&1 &Slips of Rs 5\\
&2 &Slips of Rs 13\\ \hline
\multirow{2}{*}{Y} &0 &Box A\\
&1 &Box B\\\hline
\end{tabular}
\caption{}
\label{tab:Distribution}
\end{table}
See \tabref{tab:Distribution}.
\begin{align}
p_{Y}\brak{k}= \begin{cases} 
      \frac{1}{3} & {k=0} \\
      \frac{2}{3 }& {k=1} 
   \end{cases}
   \\
p_{Y|X}\brak{0|0} = \frac{19}{25}\, 
p_{Y|X}\brak{0|1} = \frac{6}{25}\,
p_{Y|X}\brak{1|0} = \frac{45}{50}\,
p_{Y|X}\brak{1|2} = \frac{5}{50}
\end{align}
The desired probability is the probability that a slip drawn at random is marked other than Rs 1,
\begin{align}
&=1-p_X\brak{0}\\
&= p_X(1) + p_X(2)
\end{align}
Using Bayes theorem,
\begin{align}
&= p_Y\brak{0} \times \pr{Y=0 | X=1} + p_Y\brak{1} \times \pr{Y=1|X=2}\\
&=\frac{1}{3} \times \frac{6}{25} + \frac{2}{3} \times \frac{5}{50}\\
&=\frac{11}{75}
\end{align}

\newpage

%\tableofcontents

\bigskip

\renewcommand{\thefigure}{\theenumi}
\renewcommand{\thetable}{\theenumi}
%\renewcommand{\theequation}{\theenumi}

%\begin{abstract}
%%\boldmath
%In this letter, an algorithm for evaluating the exact analytical bit error rate  (BER)  for the piecewise linear (PL) combiner for  multiple relays is presented. Previous results were available only for upto three relays. The algorithm is unique in the sense that  the actual mathematical expressions, that are prohibitively large, need not be explicitly obtained. The diversity gain due to multiple relays is shown through plots of the analytical BER, well supported by simulations. 
%
%\end{abstract}
% IEEEtran.cls defaults to using nonbold math in the Abstract.
% This preserves the distinction between vectors and scalars. However,
% if the journal you are submitting to favors bold math in the abstract,
% then you can use LaTeX's standard command \boldmath at the very start
% of the abstract to achieve this. Many IEEE journals frown on math
% in the abstract anyway.

% Note that keywords are not normally used for peerreview papers.
%\begin{IEEEkeywords}
%Cooperative diversity, decode and forward, piecewise linear
%\end{IEEEkeywords}



% For peer review papers, you can put extra information on the cover
% page as needed:
% \ifCLASSOPTIONpeerreview
% \begin{center} \bfseries EDICS Category: 3-BBND \end{center}
% \fi
%
% For peerreview papers, this IEEEtran command inserts a page break and
% creates the second title. It will be ignored for other modes.
%\IEEEpeerreviewmaketitle




\item
The king, queen and jack of clubs are removed from a deck of 52 playing cards and then well shuffled. Now one card is drawn at random from the remaining cards.  Determine the probability that the card is
\begin{enumerate}[label=(\roman*)]
\item a club
\item 10 of hearts
\end{enumerate}
\solution
%\begin{table}[H]
	\centering
\begin{tabular}{|c|c|c|}
\hline
Random variable &Value &Definition\\ \hline
\multirow{3}{*}{X} &0 &Slips of Rs 1\\
&1 &Slips of Rs 5\\
&2 &Slips of Rs 13\\ \hline
\multirow{2}{*}{Y} &0 &Box A\\
&1 &Box B\\\hline
\end{tabular}
\caption{}
\label{tab:Distribution}
\end{table}
See \tabref{tab:Distribution}.
\begin{align}
p_{Y}\brak{k}= \begin{cases} 
      \frac{1}{3} & {k=0} \\
      \frac{2}{3 }& {k=1} 
   \end{cases}
   \\
p_{Y|X}\brak{0|0} = \frac{19}{25}\, 
p_{Y|X}\brak{0|1} = \frac{6}{25}\,
p_{Y|X}\brak{1|0} = \frac{45}{50}\,
p_{Y|X}\brak{1|2} = \frac{5}{50}
\end{align}
The desired probability is the probability that a slip drawn at random is marked other than Rs 1,
\begin{align}
&=1-p_X\brak{0}\\
&= p_X(1) + p_X(2)
\end{align}
Using Bayes theorem,
\begin{align}
&= p_Y\brak{0} \times \pr{Y=0 | X=1} + p_Y\brak{1} \times \pr{Y=1|X=2}\\
&=\frac{1}{3} \times \frac{6}{25} + \frac{2}{3} \times \frac{5}{50}\\
&=\frac{11}{75}
\end{align}

\newpage

%\tableofcontents

\bigskip

\renewcommand{\thefigure}{\theenumi}
\renewcommand{\thetable}{\theenumi}
%\renewcommand{\theequation}{\theenumi}

%\begin{abstract}
%%\boldmath
%In this letter, an algorithm for evaluating the exact analytical bit error rate  (BER)  for the piecewise linear (PL) combiner for  multiple relays is presented. Previous results were available only for upto three relays. The algorithm is unique in the sense that  the actual mathematical expressions, that are prohibitively large, need not be explicitly obtained. The diversity gain due to multiple relays is shown through plots of the analytical BER, well supported by simulations. 
%
%\end{abstract}
% IEEEtran.cls defaults to using nonbold math in the Abstract.
% This preserves the distinction between vectors and scalars. However,
% if the journal you are submitting to favors bold math in the abstract,
% then you can use LaTeX's standard command \boldmath at the very start
% of the abstract to achieve this. Many IEEE journals frown on math
% in the abstract anyway.

% Note that keywords are not normally used for peerreview papers.
%\begin{IEEEkeywords}
%Cooperative diversity, decode and forward, piecewise linear
%\end{IEEEkeywords}



% For peer review papers, you can put extra information on the cover
% page as needed:
% \ifCLASSOPTIONpeerreview
% \begin{center} \bfseries EDICS Category: 3-BBND \end{center}
% \fi
%
% For peerreview papers, this IEEEtran command inserts a page break and
% creates the second title. It will be ignored for other modes.
%\IEEEpeerreviewmaketitle




\item A team of medical students doing their internship have to assist during surgeries
at a city hospital. The probabilities of surgeries rated as very complex, complex,
routine, simple or very simple are respectively, 0.15, 0.20, 0.31, 0.26, .08. Find
the probabilities that a particular surgery will be rated
\begin{enumerate}
	\item complex or very complex;
	\item neither very complex nor very simple;
	\item routine or complex
	\item routine or simple
\end{enumerate}
\solution
%\begin{table}[H]
	\centering
\begin{tabular}{|c|c|c|}
\hline
Random variable &Value &Definition\\ \hline
\multirow{3}{*}{X} &0 &Slips of Rs 1\\
&1 &Slips of Rs 5\\
&2 &Slips of Rs 13\\ \hline
\multirow{2}{*}{Y} &0 &Box A\\
&1 &Box B\\\hline
\end{tabular}
\caption{}
\label{tab:Distribution}
\end{table}
See \tabref{tab:Distribution}.
\begin{align}
p_{Y}\brak{k}= \begin{cases} 
      \frac{1}{3} & {k=0} \\
      \frac{2}{3 }& {k=1} 
   \end{cases}
   \\
p_{Y|X}\brak{0|0} = \frac{19}{25}\, 
p_{Y|X}\brak{0|1} = \frac{6}{25}\,
p_{Y|X}\brak{1|0} = \frac{45}{50}\,
p_{Y|X}\brak{1|2} = \frac{5}{50}
\end{align}
The desired probability is the probability that a slip drawn at random is marked other than Rs 1,
\begin{align}
&=1-p_X\brak{0}\\
&= p_X(1) + p_X(2)
\end{align}
Using Bayes theorem,
\begin{align}
&= p_Y\brak{0} \times \pr{Y=0 | X=1} + p_Y\brak{1} \times \pr{Y=1|X=2}\\
&=\frac{1}{3} \times \frac{6}{25} + \frac{2}{3} \times \frac{5}{50}\\
&=\frac{11}{75}
\end{align}

\newpage

%\tableofcontents

\bigskip

\renewcommand{\thefigure}{\theenumi}
\renewcommand{\thetable}{\theenumi}
%\renewcommand{\theequation}{\theenumi}

%\begin{abstract}
%%\boldmath
%In this letter, an algorithm for evaluating the exact analytical bit error rate  (BER)  for the piecewise linear (PL) combiner for  multiple relays is presented. Previous results were available only for upto three relays. The algorithm is unique in the sense that  the actual mathematical expressions, that are prohibitively large, need not be explicitly obtained. The diversity gain due to multiple relays is shown through plots of the analytical BER, well supported by simulations. 
%
%\end{abstract}
% IEEEtran.cls defaults to using nonbold math in the Abstract.
% This preserves the distinction between vectors and scalars. However,
% if the journal you are submitting to favors bold math in the abstract,
% then you can use LaTeX's standard command \boldmath at the very start
% of the abstract to achieve this. Many IEEE journals frown on math
% in the abstract anyway.

% Note that keywords are not normally used for peerreview papers.
%\begin{IEEEkeywords}
%Cooperative diversity, decode and forward, piecewise linear
%\end{IEEEkeywords}



% For peer review papers, you can put extra information on the cover
% page as needed:
% \ifCLASSOPTIONpeerreview
% \begin{center} \bfseries EDICS Category: 3-BBND \end{center}
% \fi
%
% For peerreview papers, this IEEEtran command inserts a page break and
% creates the second title. It will be ignored for other modes.
%\IEEEpeerreviewmaketitle




\item A card is selected from a pack of 52 cards.
\begin{enumerate}[label=(\alph*)]
    \item How many points are there in the sample space?
    \item Calculate the probability that the card is an ace of spades.
    \item Calculate the probability that the card is (i) an ace and (ii) black card.
\end{enumerate}
\solution
%Let $X$ be an bernoulli rv defined as in \tabref{tab:exemplar/11/16/3/26}.  Then, 
\begin{equation}
    p =
        \frac{4}{11} 
\end{equation}
\begin{table}[H]
	\centering
	\input{exemplar/11/16/3/26/tables/Table2.tex}
	\caption{}
        \label{tab:exemplar/11/16/3/26}
\end{table}

\item The probability that a non leap year selected at random will contain 53 sundays.
\\
\solution
%\begin{table}[H]
	\centering
\begin{tabular}{|c|c|c|}
\hline
Random variable &Value &Definition\\ \hline
\multirow{3}{*}{X} &0 &Slips of Rs 1\\
&1 &Slips of Rs 5\\
&2 &Slips of Rs 13\\ \hline
\multirow{2}{*}{Y} &0 &Box A\\
&1 &Box B\\\hline
\end{tabular}
\caption{}
\label{tab:Distribution}
\end{table}
See \tabref{tab:Distribution}.
\begin{align}
p_{Y}\brak{k}= \begin{cases} 
      \frac{1}{3} & {k=0} \\
      \frac{2}{3 }& {k=1} 
   \end{cases}
   \\
p_{Y|X}\brak{0|0} = \frac{19}{25}\, 
p_{Y|X}\brak{0|1} = \frac{6}{25}\,
p_{Y|X}\brak{1|0} = \frac{45}{50}\,
p_{Y|X}\brak{1|2} = \frac{5}{50}
\end{align}
The desired probability is the probability that a slip drawn at random is marked other than Rs 1,
\begin{align}
&=1-p_X\brak{0}\\
&= p_X(1) + p_X(2)
\end{align}
Using Bayes theorem,
\begin{align}
&= p_Y\brak{0} \times \pr{Y=0 | X=1} + p_Y\brak{1} \times \pr{Y=1|X=2}\\
&=\frac{1}{3} \times \frac{6}{25} + \frac{2}{3} \times \frac{5}{50}\\
&=\frac{11}{75}
\end{align}

\newpage

%\tableofcontents

\bigskip

\renewcommand{\thefigure}{\theenumi}
\renewcommand{\thetable}{\theenumi}
%\renewcommand{\theequation}{\theenumi}

%\begin{abstract}
%%\boldmath
%In this letter, an algorithm for evaluating the exact analytical bit error rate  (BER)  for the piecewise linear (PL) combiner for  multiple relays is presented. Previous results were available only for upto three relays. The algorithm is unique in the sense that  the actual mathematical expressions, that are prohibitively large, need not be explicitly obtained. The diversity gain due to multiple relays is shown through plots of the analytical BER, well supported by simulations. 
%
%\end{abstract}
% IEEEtran.cls defaults to using nonbold math in the Abstract.
% This preserves the distinction between vectors and scalars. However,
% if the journal you are submitting to favors bold math in the abstract,
% then you can use LaTeX's standard command \boldmath at the very start
% of the abstract to achieve this. Many IEEE journals frown on math
% in the abstract anyway.

% Note that keywords are not normally used for peerreview papers.
%\begin{IEEEkeywords}
%Cooperative diversity, decode and forward, piecewise linear
%\end{IEEEkeywords}



% For peer review papers, you can put extra information on the cover
% page as needed:
% \ifCLASSOPTIONpeerreview
% \begin{center} \bfseries EDICS Category: 3-BBND \end{center}
% \fi
%
% For peerreview papers, this IEEEtran command inserts a page break and
% creates the second title. It will be ignored for other modes.
%\IEEEpeerreviewmaketitle




\item One of the four persons John, Rita, Aslam or Gurpreet will be promoted next
month. Consequently the sample space consists of four elementary outcomes
S = {John promoted, Rita promoted, Aslam promoted, Gurpreet promoted}
You are told that the chances of John’s promotion is same as that of Gurpreet,
Rita’s chances of promotion are twice as likely as Johns. Aslam’s chances are
four times that of John.
\begin{enumerate}
	\item Determine
	\begin{enumerate}
		\item P (John promoted)
		\item P (Rita promoted)
		\item P (Aslam promoted)
		\item P (Gurpreet promoted)
	\end{enumerate}
	\item If A = {John promoted or Gurpreet promoted}, find P (A).
\end{enumerate}
\solution
%\begin{table}[H]
	\centering
\begin{tabular}{|c|c|c|}
\hline
Random variable &Value &Definition\\ \hline
\multirow{3}{*}{X} &0 &Slips of Rs 1\\
&1 &Slips of Rs 5\\
&2 &Slips of Rs 13\\ \hline
\multirow{2}{*}{Y} &0 &Box A\\
&1 &Box B\\\hline
\end{tabular}
\caption{}
\label{tab:Distribution}
\end{table}
See \tabref{tab:Distribution}.
\begin{align}
p_{Y}\brak{k}= \begin{cases} 
      \frac{1}{3} & {k=0} \\
      \frac{2}{3 }& {k=1} 
   \end{cases}
   \\
p_{Y|X}\brak{0|0} = \frac{19}{25}\, 
p_{Y|X}\brak{0|1} = \frac{6}{25}\,
p_{Y|X}\brak{1|0} = \frac{45}{50}\,
p_{Y|X}\brak{1|2} = \frac{5}{50}
\end{align}
The desired probability is the probability that a slip drawn at random is marked other than Rs 1,
\begin{align}
&=1-p_X\brak{0}\\
&= p_X(1) + p_X(2)
\end{align}
Using Bayes theorem,
\begin{align}
&= p_Y\brak{0} \times \pr{Y=0 | X=1} + p_Y\brak{1} \times \pr{Y=1|X=2}\\
&=\frac{1}{3} \times \frac{6}{25} + \frac{2}{3} \times \frac{5}{50}\\
&=\frac{11}{75}
\end{align}

\newpage

%\tableofcontents

\bigskip

\renewcommand{\thefigure}{\theenumi}
\renewcommand{\thetable}{\theenumi}
%\renewcommand{\theequation}{\theenumi}

%\begin{abstract}
%%\boldmath
%In this letter, an algorithm for evaluating the exact analytical bit error rate  (BER)  for the piecewise linear (PL) combiner for  multiple relays is presented. Previous results were available only for upto three relays. The algorithm is unique in the sense that  the actual mathematical expressions, that are prohibitively large, need not be explicitly obtained. The diversity gain due to multiple relays is shown through plots of the analytical BER, well supported by simulations. 
%
%\end{abstract}
% IEEEtran.cls defaults to using nonbold math in the Abstract.
% This preserves the distinction between vectors and scalars. However,
% if the journal you are submitting to favors bold math in the abstract,
% then you can use LaTeX's standard command \boldmath at the very start
% of the abstract to achieve this. Many IEEE journals frown on math
% in the abstract anyway.

% Note that keywords are not normally used for peerreview papers.
%\begin{IEEEkeywords}
%Cooperative diversity, decode and forward, piecewise linear
%\end{IEEEkeywords}



% For peer review papers, you can put extra information on the cover
% page as needed:
% \ifCLASSOPTIONpeerreview
% \begin{center} \bfseries EDICS Category: 3-BBND \end{center}
% \fi
%
% For peerreview papers, this IEEEtran command inserts a page break and
% creates the second title. It will be ignored for other modes.
%\IEEEpeerreviewmaketitle




\item A card is drawn from a deck of 52 cards. Find the probability of getting a king or a heart or a red card.\\
\solution
%\begin{table}[H]
	\centering
\begin{tabular}{|c|c|c|}
\hline
Random variable &Value &Definition\\ \hline
\multirow{3}{*}{X} &0 &Slips of Rs 1\\
&1 &Slips of Rs 5\\
&2 &Slips of Rs 13\\ \hline
\multirow{2}{*}{Y} &0 &Box A\\
&1 &Box B\\\hline
\end{tabular}
\caption{}
\label{tab:Distribution}
\end{table}
See \tabref{tab:Distribution}.
\begin{align}
p_{Y}\brak{k}= \begin{cases} 
      \frac{1}{3} & {k=0} \\
      \frac{2}{3 }& {k=1} 
   \end{cases}
   \\
p_{Y|X}\brak{0|0} = \frac{19}{25}\, 
p_{Y|X}\brak{0|1} = \frac{6}{25}\,
p_{Y|X}\brak{1|0} = \frac{45}{50}\,
p_{Y|X}\brak{1|2} = \frac{5}{50}
\end{align}
The desired probability is the probability that a slip drawn at random is marked other than Rs 1,
\begin{align}
&=1-p_X\brak{0}\\
&= p_X(1) + p_X(2)
\end{align}
Using Bayes theorem,
\begin{align}
&= p_Y\brak{0} \times \pr{Y=0 | X=1} + p_Y\brak{1} \times \pr{Y=1|X=2}\\
&=\frac{1}{3} \times \frac{6}{25} + \frac{2}{3} \times \frac{5}{50}\\
&=\frac{11}{75}
\end{align}

\newpage

%\tableofcontents

\bigskip

\renewcommand{\thefigure}{\theenumi}
\renewcommand{\thetable}{\theenumi}
%\renewcommand{\theequation}{\theenumi}

%\begin{abstract}
%%\boldmath
%In this letter, an algorithm for evaluating the exact analytical bit error rate  (BER)  for the piecewise linear (PL) combiner for  multiple relays is presented. Previous results were available only for upto three relays. The algorithm is unique in the sense that  the actual mathematical expressions, that are prohibitively large, need not be explicitly obtained. The diversity gain due to multiple relays is shown through plots of the analytical BER, well supported by simulations. 
%
%\end{abstract}
% IEEEtran.cls defaults to using nonbold math in the Abstract.
% This preserves the distinction between vectors and scalars. However,
% if the journal you are submitting to favors bold math in the abstract,
% then you can use LaTeX's standard command \boldmath at the very start
% of the abstract to achieve this. Many IEEE journals frown on math
% in the abstract anyway.

% Note that keywords are not normally used for peerreview papers.
%\begin{IEEEkeywords}
%Cooperative diversity, decode and forward, piecewise linear
%\end{IEEEkeywords}



% For peer review papers, you can put extra information on the cover
% page as needed:
% \ifCLASSOPTIONpeerreview
% \begin{center} \bfseries EDICS Category: 3-BBND \end{center}
% \fi
%
% For peerreview papers, this IEEEtran command inserts a page break and
% creates the second title. It will be ignored for other modes.
%\IEEEpeerreviewmaketitle




\item The probability that a student will pass his examination is 0.73, the probability of
the student getting a compartment is 0.13, and the probability that the student will
either pass or get compartment is 0.96. State True or False.\\
\solution
%\begin{table}[H]
	\centering
\begin{tabular}{|c|c|c|}
\hline
Random variable &Value &Definition\\ \hline
\multirow{3}{*}{X} &0 &Slips of Rs 1\\
&1 &Slips of Rs 5\\
&2 &Slips of Rs 13\\ \hline
\multirow{2}{*}{Y} &0 &Box A\\
&1 &Box B\\\hline
\end{tabular}
\caption{}
\label{tab:Distribution}
\end{table}
See \tabref{tab:Distribution}.
\begin{align}
p_{Y}\brak{k}= \begin{cases} 
      \frac{1}{3} & {k=0} \\
      \frac{2}{3 }& {k=1} 
   \end{cases}
   \\
p_{Y|X}\brak{0|0} = \frac{19}{25}\, 
p_{Y|X}\brak{0|1} = \frac{6}{25}\,
p_{Y|X}\brak{1|0} = \frac{45}{50}\,
p_{Y|X}\brak{1|2} = \frac{5}{50}
\end{align}
The desired probability is the probability that a slip drawn at random is marked other than Rs 1,
\begin{align}
&=1-p_X\brak{0}\\
&= p_X(1) + p_X(2)
\end{align}
Using Bayes theorem,
\begin{align}
&= p_Y\brak{0} \times \pr{Y=0 | X=1} + p_Y\brak{1} \times \pr{Y=1|X=2}\\
&=\frac{1}{3} \times \frac{6}{25} + \frac{2}{3} \times \frac{5}{50}\\
&=\frac{11}{75}
\end{align}

\newpage

%\tableofcontents

\bigskip

\renewcommand{\thefigure}{\theenumi}
\renewcommand{\thetable}{\theenumi}
%\renewcommand{\theequation}{\theenumi}

%\begin{abstract}
%%\boldmath
%In this letter, an algorithm for evaluating the exact analytical bit error rate  (BER)  for the piecewise linear (PL) combiner for  multiple relays is presented. Previous results were available only for upto three relays. The algorithm is unique in the sense that  the actual mathematical expressions, that are prohibitively large, need not be explicitly obtained. The diversity gain due to multiple relays is shown through plots of the analytical BER, well supported by simulations. 
%
%\end{abstract}
% IEEEtran.cls defaults to using nonbold math in the Abstract.
% This preserves the distinction between vectors and scalars. However,
% if the journal you are submitting to favors bold math in the abstract,
% then you can use LaTeX's standard command \boldmath at the very start
% of the abstract to achieve this. Many IEEE journals frown on math
% in the abstract anyway.

% Note that keywords are not normally used for peerreview papers.
%\begin{IEEEkeywords}
%Cooperative diversity, decode and forward, piecewise linear
%\end{IEEEkeywords}



% For peer review papers, you can put extra information on the cover
% page as needed:
% \ifCLASSOPTIONpeerreview
% \begin{center} \bfseries EDICS Category: 3-BBND \end{center}
% \fi
%
% For peerreview papers, this IEEEtran command inserts a page break and
% creates the second title. It will be ignored for other modes.
%\IEEEpeerreviewmaketitle




\item A card is selected from a pack of 52 cards\\
\begin{enumerate}[label=(\alph*)]
\item How many points are there in the sample space?
\item Calculate the probability that the cards is an ace of spades.
\item Calculate the probability that the card is (i) an ace (ii)black card.\\
\end{enumerate}
%\input{ncert/11/16/3/4_1/Prob_4.tex}
\item In a non-leap year, the probability of having 53 tuesdays or 53 wednesdays is\\
\solution
%A non-leap year has a total of 365 days, and a week has 7 days.\\
So it can be expressed as 
\begin{align}
365\text{days} &=52\times 7+1 \text{day}
\end{align}
$\implies$ 52 tuesdays or wednesdays\\
Random variable X denotes the days of a week
\begin{align}
p_X\brak{k}&=\frac{1}{7}; \quad \brak{1<k<7}
\end{align}
So the probability of extra day being tuesday or wednesday is
\begin{align}
p_X\brak{3}+p_X\brak{4}&=\frac{1}{7}+\frac{1}{7}=\frac{2}{7}
\end{align}



\item There are 1000 sealed envelopes in a box, 10 of them contain a cash prize of
Rs 100 each, 100 of them contain a cash prize of Rs 50 each and 200 of them
contain a cash prize of Rs 10 each and rest do not contain any cash prize. If they
are well shuffled and an envelope is picked up out, what is the probability that it
contains no cash prize?\\
\solution
%\begin{table}[H]
	\centering
\begin{tabular}{|c|c|c|}
\hline
Random variable &Value &Definition\\ \hline
\multirow{3}{*}{X} &0 &Slips of Rs 1\\
&1 &Slips of Rs 5\\
&2 &Slips of Rs 13\\ \hline
\multirow{2}{*}{Y} &0 &Box A\\
&1 &Box B\\\hline
\end{tabular}
\caption{}
\label{tab:Distribution}
\end{table}
See \tabref{tab:Distribution}.
\begin{align}
p_{Y}\brak{k}= \begin{cases} 
      \frac{1}{3} & {k=0} \\
      \frac{2}{3 }& {k=1} 
   \end{cases}
   \\
p_{Y|X}\brak{0|0} = \frac{19}{25}\, 
p_{Y|X}\brak{0|1} = \frac{6}{25}\,
p_{Y|X}\brak{1|0} = \frac{45}{50}\,
p_{Y|X}\brak{1|2} = \frac{5}{50}
\end{align}
The desired probability is the probability that a slip drawn at random is marked other than Rs 1,
\begin{align}
&=1-p_X\brak{0}\\
&= p_X(1) + p_X(2)
\end{align}
Using Bayes theorem,
\begin{align}
&= p_Y\brak{0} \times \pr{Y=0 | X=1} + p_Y\brak{1} \times \pr{Y=1|X=2}\\
&=\frac{1}{3} \times \frac{6}{25} + \frac{2}{3} \times \frac{5}{50}\\
&=\frac{11}{75}
\end{align}

\newpage

%\tableofcontents

\bigskip

\renewcommand{\thefigure}{\theenumi}
\renewcommand{\thetable}{\theenumi}
%\renewcommand{\theequation}{\theenumi}

%\begin{abstract}
%%\boldmath
%In this letter, an algorithm for evaluating the exact analytical bit error rate  (BER)  for the piecewise linear (PL) combiner for  multiple relays is presented. Previous results were available only for upto three relays. The algorithm is unique in the sense that  the actual mathematical expressions, that are prohibitively large, need not be explicitly obtained. The diversity gain due to multiple relays is shown through plots of the analytical BER, well supported by simulations. 
%
%\end{abstract}
% IEEEtran.cls defaults to using nonbold math in the Abstract.
% This preserves the distinction between vectors and scalars. However,
% if the journal you are submitting to favors bold math in the abstract,
% then you can use LaTeX's standard command \boldmath at the very start
% of the abstract to achieve this. Many IEEE journals frown on math
% in the abstract anyway.

% Note that keywords are not normally used for peerreview papers.
%\begin{IEEEkeywords}
%Cooperative diversity, decode and forward, piecewise linear
%\end{IEEEkeywords}



% For peer review papers, you can put extra information on the cover
% page as needed:
% \ifCLASSOPTIONpeerreview
% \begin{center} \bfseries EDICS Category: 3-BBND \end{center}
% \fi
%
% For peerreview papers, this IEEEtran command inserts a page break and
% creates the second title. It will be ignored for other modes.
%\IEEEpeerreviewmaketitle




\item 
A die is thrown and a card is selected at random from a deck of 52 playing cards. The probability of getting an even number on the die and a spade card.\\
\solution
%\begin{table}[H]
	\centering
\begin{tabular}{|c|c|c|}
\hline
Random variable &Value &Definition\\ \hline
\multirow{3}{*}{X} &0 &Slips of Rs 1\\
&1 &Slips of Rs 5\\
&2 &Slips of Rs 13\\ \hline
\multirow{2}{*}{Y} &0 &Box A\\
&1 &Box B\\\hline
\end{tabular}
\caption{}
\label{tab:Distribution}
\end{table}
See \tabref{tab:Distribution}.
\begin{align}
p_{Y}\brak{k}= \begin{cases} 
      \frac{1}{3} & {k=0} \\
      \frac{2}{3 }& {k=1} 
   \end{cases}
   \\
p_{Y|X}\brak{0|0} = \frac{19}{25}\, 
p_{Y|X}\brak{0|1} = \frac{6}{25}\,
p_{Y|X}\brak{1|0} = \frac{45}{50}\,
p_{Y|X}\brak{1|2} = \frac{5}{50}
\end{align}
The desired probability is the probability that a slip drawn at random is marked other than Rs 1,
\begin{align}
&=1-p_X\brak{0}\\
&= p_X(1) + p_X(2)
\end{align}
Using Bayes theorem,
\begin{align}
&= p_Y\brak{0} \times \pr{Y=0 | X=1} + p_Y\brak{1} \times \pr{Y=1|X=2}\\
&=\frac{1}{3} \times \frac{6}{25} + \frac{2}{3} \times \frac{5}{50}\\
&=\frac{11}{75}
\end{align}

\newpage

%\tableofcontents

\bigskip

\renewcommand{\thefigure}{\theenumi}
\renewcommand{\thetable}{\theenumi}
%\renewcommand{\theequation}{\theenumi}

%\begin{abstract}
%%\boldmath
%In this letter, an algorithm for evaluating the exact analytical bit error rate  (BER)  for the piecewise linear (PL) combiner for  multiple relays is presented. Previous results were available only for upto three relays. The algorithm is unique in the sense that  the actual mathematical expressions, that are prohibitively large, need not be explicitly obtained. The diversity gain due to multiple relays is shown through plots of the analytical BER, well supported by simulations. 
%
%\end{abstract}
% IEEEtran.cls defaults to using nonbold math in the Abstract.
% This preserves the distinction between vectors and scalars. However,
% if the journal you are submitting to favors bold math in the abstract,
% then you can use LaTeX's standard command \boldmath at the very start
% of the abstract to achieve this. Many IEEE journals frown on math
% in the abstract anyway.

% Note that keywords are not normally used for peerreview papers.
%\begin{IEEEkeywords}
%Cooperative diversity, decode and forward, piecewise linear
%\end{IEEEkeywords}



% For peer review papers, you can put extra information on the cover
% page as needed:
% \ifCLASSOPTIONpeerreview
% \begin{center} \bfseries EDICS Category: 3-BBND \end{center}
% \fi
%
% For peerreview papers, this IEEEtran command inserts a page break and
% creates the second title. It will be ignored for other modes.
%\IEEEpeerreviewmaketitle




\item
If 4-digit numbers greater than 5,000 are randomly formed from the digits 0, 1, 3, 5, and 7, what is the probability of forming a number divisible by 5 when:
\begin{enumerate}
    \item The digits are repeated?
    \item The repetition of digits is not allowed?
\end{enumerate}
\solution
%\begin{table}[H]
	\centering
\begin{tabular}{|c|c|c|}
\hline
Random variable &Value &Definition\\ \hline
\multirow{3}{*}{X} &0 &Slips of Rs 1\\
&1 &Slips of Rs 5\\
&2 &Slips of Rs 13\\ \hline
\multirow{2}{*}{Y} &0 &Box A\\
&1 &Box B\\\hline
\end{tabular}
\caption{}
\label{tab:Distribution}
\end{table}
See \tabref{tab:Distribution}.
\begin{align}
p_{Y}\brak{k}= \begin{cases} 
      \frac{1}{3} & {k=0} \\
      \frac{2}{3 }& {k=1} 
   \end{cases}
   \\
p_{Y|X}\brak{0|0} = \frac{19}{25}\, 
p_{Y|X}\brak{0|1} = \frac{6}{25}\,
p_{Y|X}\brak{1|0} = \frac{45}{50}\,
p_{Y|X}\brak{1|2} = \frac{5}{50}
\end{align}
The desired probability is the probability that a slip drawn at random is marked other than Rs 1,
\begin{align}
&=1-p_X\brak{0}\\
&= p_X(1) + p_X(2)
\end{align}
Using Bayes theorem,
\begin{align}
&= p_Y\brak{0} \times \pr{Y=0 | X=1} + p_Y\brak{1} \times \pr{Y=1|X=2}\\
&=\frac{1}{3} \times \frac{6}{25} + \frac{2}{3} \times \frac{5}{50}\\
&=\frac{11}{75}
\end{align}

\newpage

%\tableofcontents

\bigskip

\renewcommand{\thefigure}{\theenumi}
\renewcommand{\thetable}{\theenumi}
%\renewcommand{\theequation}{\theenumi}

%\begin{abstract}
%%\boldmath
%In this letter, an algorithm for evaluating the exact analytical bit error rate  (BER)  for the piecewise linear (PL) combiner for  multiple relays is presented. Previous results were available only for upto three relays. The algorithm is unique in the sense that  the actual mathematical expressions, that are prohibitively large, need not be explicitly obtained. The diversity gain due to multiple relays is shown through plots of the analytical BER, well supported by simulations. 
%
%\end{abstract}
% IEEEtran.cls defaults to using nonbold math in the Abstract.
% This preserves the distinction between vectors and scalars. However,
% if the journal you are submitting to favors bold math in the abstract,
% then you can use LaTeX's standard command \boldmath at the very start
% of the abstract to achieve this. Many IEEE journals frown on math
% in the abstract anyway.

% Note that keywords are not normally used for peerreview papers.
%\begin{IEEEkeywords}
%Cooperative diversity, decode and forward, piecewise linear
%\end{IEEEkeywords}



% For peer review papers, you can put extra information on the cover
% page as needed:
% \ifCLASSOPTIONpeerreview
% \begin{center} \bfseries EDICS Category: 3-BBND \end{center}
% \fi
%
% For peerreview papers, this IEEEtran command inserts a page break and
% creates the second title. It will be ignored for other modes.
%\IEEEpeerreviewmaketitle




\item Consider the probability space $\brak{\Omega, \mathcal{G}, P}$ where $\Omega = [0,2]$ and $\mathcal{G} = \cbrak{\phi, \Omega, [0,1], (1,2]}$. Let $X$ and $Y$ be two functions on $\Omega$ defined as
\begin{align*}
    X(\omega) = 
    \begin{cases}
        1 & \text{if }\omega \in [0, 1]\\
        2 & \text{if }\omega \in (1, 2]
    \end{cases}
\end{align*}
and
\begin{align*}
    Y(\omega) = 
    \begin{cases}
        2 & \text{if }\omega \in [0, 1.5]\\
        3 & \text{if }\omega \in (1.5, 2].
    \end{cases}
\end{align*}
Then which one of the following statements is true?
\begin{enumerate}
    \item [(A)] $X$ is a random variable with respect to $\mathcal{G}$, but $Y$ is not a random variable with respect to $\mathcal{G}$.
    \item [(B)] $Y$ is a random variable with respect to $\mathcal{G}$, but $X$ is not a random variable with respect to $\mathcal{G}$.
    \item [(C)] Neither $X$ nor $Y$ is a random variable with respect to $\mathcal{G}$.
    \item [(D)] Both $X$ and $Y$ are random variables with respect to $\mathcal{G}$.
\end{enumerate} \hfill (GATE ST 2023)\\
\solution
%\begin{table}[H]
	\centering
\begin{tabular}{|c|c|c|}
\hline
Random variable &Value &Definition\\ \hline
\multirow{3}{*}{X} &0 &Slips of Rs 1\\
&1 &Slips of Rs 5\\
&2 &Slips of Rs 13\\ \hline
\multirow{2}{*}{Y} &0 &Box A\\
&1 &Box B\\\hline
\end{tabular}
\caption{}
\label{tab:Distribution}
\end{table}
See \tabref{tab:Distribution}.
\begin{align}
p_{Y}\brak{k}= \begin{cases} 
      \frac{1}{3} & {k=0} \\
      \frac{2}{3 }& {k=1} 
   \end{cases}
   \\
p_{Y|X}\brak{0|0} = \frac{19}{25}\, 
p_{Y|X}\brak{0|1} = \frac{6}{25}\,
p_{Y|X}\brak{1|0} = \frac{45}{50}\,
p_{Y|X}\brak{1|2} = \frac{5}{50}
\end{align}
The desired probability is the probability that a slip drawn at random is marked other than Rs 1,
\begin{align}
&=1-p_X\brak{0}\\
&= p_X(1) + p_X(2)
\end{align}
Using Bayes theorem,
\begin{align}
&= p_Y\brak{0} \times \pr{Y=0 | X=1} + p_Y\brak{1} \times \pr{Y=1|X=2}\\
&=\frac{1}{3} \times \frac{6}{25} + \frac{2}{3} \times \frac{5}{50}\\
&=\frac{11}{75}
\end{align}

\newpage

%\tableofcontents

\bigskip

\renewcommand{\thefigure}{\theenumi}
\renewcommand{\thetable}{\theenumi}
%\renewcommand{\theequation}{\theenumi}

%\begin{abstract}
%%\boldmath
%In this letter, an algorithm for evaluating the exact analytical bit error rate  (BER)  for the piecewise linear (PL) combiner for  multiple relays is presented. Previous results were available only for upto three relays. The algorithm is unique in the sense that  the actual mathematical expressions, that are prohibitively large, need not be explicitly obtained. The diversity gain due to multiple relays is shown through plots of the analytical BER, well supported by simulations. 
%
%\end{abstract}
% IEEEtran.cls defaults to using nonbold math in the Abstract.
% This preserves the distinction between vectors and scalars. However,
% if the journal you are submitting to favors bold math in the abstract,
% then you can use LaTeX's standard command \boldmath at the very start
% of the abstract to achieve this. Many IEEE journals frown on math
% in the abstract anyway.

% Note that keywords are not normally used for peerreview papers.
%\begin{IEEEkeywords}
%Cooperative diversity, decode and forward, piecewise linear
%\end{IEEEkeywords}



% For peer review papers, you can put extra information on the cover
% page as needed:
% \ifCLASSOPTIONpeerreview
% \begin{center} \bfseries EDICS Category: 3-BBND \end{center}
% \fi
%
% For peerreview papers, this IEEEtran command inserts a page break and
% creates the second title. It will be ignored for other modes.
%\IEEEpeerreviewmaketitle




	\item  A die is loaded in such a way that each odd number is twice as likely to occur as
each even number. Find $P(G)$, where $G$ is the event that a number greater than
3 occurs on a single roll of the die.
\\
\solution
		%\begin{table}[H]
	\centering
\begin{tabular}{|c|c|c|}
\hline
Random variable &Value &Definition\\ \hline
\multirow{3}{*}{X} &0 &Slips of Rs 1\\
&1 &Slips of Rs 5\\
&2 &Slips of Rs 13\\ \hline
\multirow{2}{*}{Y} &0 &Box A\\
&1 &Box B\\\hline
\end{tabular}
\caption{}
\label{tab:Distribution}
\end{table}
See \tabref{tab:Distribution}.
\begin{align}
p_{Y}\brak{k}= \begin{cases} 
      \frac{1}{3} & {k=0} \\
      \frac{2}{3 }& {k=1} 
   \end{cases}
   \\
p_{Y|X}\brak{0|0} = \frac{19}{25}\, 
p_{Y|X}\brak{0|1} = \frac{6}{25}\,
p_{Y|X}\brak{1|0} = \frac{45}{50}\,
p_{Y|X}\brak{1|2} = \frac{5}{50}
\end{align}
The desired probability is the probability that a slip drawn at random is marked other than Rs 1,
\begin{align}
&=1-p_X\brak{0}\\
&= p_X(1) + p_X(2)
\end{align}
Using Bayes theorem,
\begin{align}
&= p_Y\brak{0} \times \pr{Y=0 | X=1} + p_Y\brak{1} \times \pr{Y=1|X=2}\\
&=\frac{1}{3} \times \frac{6}{25} + \frac{2}{3} \times \frac{5}{50}\\
&=\frac{11}{75}
\end{align}

\newpage

%\tableofcontents

\bigskip

\renewcommand{\thefigure}{\theenumi}
\renewcommand{\thetable}{\theenumi}
%\renewcommand{\theequation}{\theenumi}

%\begin{abstract}
%%\boldmath
%In this letter, an algorithm for evaluating the exact analytical bit error rate  (BER)  for the piecewise linear (PL) combiner for  multiple relays is presented. Previous results were available only for upto three relays. The algorithm is unique in the sense that  the actual mathematical expressions, that are prohibitively large, need not be explicitly obtained. The diversity gain due to multiple relays is shown through plots of the analytical BER, well supported by simulations. 
%
%\end{abstract}
% IEEEtran.cls defaults to using nonbold math in the Abstract.
% This preserves the distinction between vectors and scalars. However,
% if the journal you are submitting to favors bold math in the abstract,
% then you can use LaTeX's standard command \boldmath at the very start
% of the abstract to achieve this. Many IEEE journals frown on math
% in the abstract anyway.

% Note that keywords are not normally used for peerreview papers.
%\begin{IEEEkeywords}
%Cooperative diversity, decode and forward, piecewise linear
%\end{IEEEkeywords}



% For peer review papers, you can put extra information on the cover
% page as needed:
% \ifCLASSOPTIONpeerreview
% \begin{center} \bfseries EDICS Category: 3-BBND \end{center}
% \fi
%
% For peerreview papers, this IEEEtran command inserts a page break and
% creates the second title. It will be ignored for other modes.
%\IEEEpeerreviewmaketitle




	\item All the jacks, queens and kings are removed from a deck of 52 playing cards. The remaining cards are well shuffled and then one card is drawn at random. Giving ace a value 1 similar value for other cards, find the probability that the card has a value 
		\begin{enumerate}
			\item 7
			\item greater than 7
			\item less than 7
		\end{enumerate}
		%Number of cards left after removing all jacks, queens and kings 
\begin{align}
N	= 52 - 4\times 3
	= 40
\end{align}
%\begin{table}[H]
%\def\arraystretch{1.2}
%\begin{tabular}{|c|c|c|}
%\hline
%	\textbf{Parameter} &\textbf{Value} &\textbf{Description}\\ \hline
%	$X$ &1-10 &Represents the value of the card picked \\ \hline
%\end{tabular}
%\end{table}
Let $1 \le X \le 10$ be the value of the card picked.  Then,
\begin{align}
	p_X(k) &= \Pr(X=k)\ \forall\ 1 \leq k \leq 10\\
	&= \frac{4\times 1}{40}\\
	&= \frac{1}{10}\\
	\therefore p_X(k) &= 
	\begin{cases}
		\frac{1}{10} & 1 \leq k \leq 10\\
		0 & \text{otherwise}
	\end{cases}
\end{align}
and
\begin{align}
	F_{X}(k) &= \sum_{m=0}^{k}p_{X}(m) \quad 1 \leq k \leq 10\\
	&= \frac{k}{10}\\
	\therefore F_{X}(k) &= 
	\begin{cases}
		0 & k \leq 0\\
		\frac{k}{10} & 1\leq k \leq 10\\
		1 & k > 10 
	\end{cases}
\end{align}
\begin{enumerate}
	\item Probability that card has value equal to 7 is
		\begin{align}
			 p_{X}(7)
			= \frac{1}{10}
		\end{align}
	\item Probability that card has value greater than 7 is
		\begin{align}
			1 - F_X(7)
			&= 1 - \frac{7}{10}
			\\
			&= \frac{3}{10}
		\end{align}
	\item Probability that card has value less than 7 is
		\begin{align}
			 F_{X}(6)
			=\frac{6}{10}
		\end{align}
\end{enumerate}

  \item A Lot consists of 48 mobile phones of which 42 are good, 3 have only minor defects and 3 have major defects.Varnika will buy a phone if it is good but the trader will only buy a mobile if it has no major defects. One phone is selected at random from the lot. What is the probability that it is
\begin{enumerate}
	\item acceptable to Varnika?
            \item acceptable to the trader?
\end{enumerate}
\solution
	%\begin{table}[H]
	\centering
\begin{tabular}{|c|c|c|}
\hline
Random variable &Value &Definition\\ \hline
\multirow{3}{*}{X} &0 &Slips of Rs 1\\
&1 &Slips of Rs 5\\
&2 &Slips of Rs 13\\ \hline
\multirow{2}{*}{Y} &0 &Box A\\
&1 &Box B\\\hline
\end{tabular}
\caption{}
\label{tab:Distribution}
\end{table}
See \tabref{tab:Distribution}.
\begin{align}
p_{Y}\brak{k}= \begin{cases} 
      \frac{1}{3} & {k=0} \\
      \frac{2}{3 }& {k=1} 
   \end{cases}
   \\
p_{Y|X}\brak{0|0} = \frac{19}{25}\, 
p_{Y|X}\brak{0|1} = \frac{6}{25}\,
p_{Y|X}\brak{1|0} = \frac{45}{50}\,
p_{Y|X}\brak{1|2} = \frac{5}{50}
\end{align}
The desired probability is the probability that a slip drawn at random is marked other than Rs 1,
\begin{align}
&=1-p_X\brak{0}\\
&= p_X(1) + p_X(2)
\end{align}
Using Bayes theorem,
\begin{align}
&= p_Y\brak{0} \times \pr{Y=0 | X=1} + p_Y\brak{1} \times \pr{Y=1|X=2}\\
&=\frac{1}{3} \times \frac{6}{25} + \frac{2}{3} \times \frac{5}{50}\\
&=\frac{11}{75}
\end{align}

\newpage

%\tableofcontents

\bigskip

\renewcommand{\thefigure}{\theenumi}
\renewcommand{\thetable}{\theenumi}
%\renewcommand{\theequation}{\theenumi}

%\begin{abstract}
%%\boldmath
%In this letter, an algorithm for evaluating the exact analytical bit error rate  (BER)  for the piecewise linear (PL) combiner for  multiple relays is presented. Previous results were available only for upto three relays. The algorithm is unique in the sense that  the actual mathematical expressions, that are prohibitively large, need not be explicitly obtained. The diversity gain due to multiple relays is shown through plots of the analytical BER, well supported by simulations. 
%
%\end{abstract}
% IEEEtran.cls defaults to using nonbold math in the Abstract.
% This preserves the distinction between vectors and scalars. However,
% if the journal you are submitting to favors bold math in the abstract,
% then you can use LaTeX's standard command \boldmath at the very start
% of the abstract to achieve this. Many IEEE journals frown on math
% in the abstract anyway.

% Note that keywords are not normally used for peerreview papers.
%\begin{IEEEkeywords}
%Cooperative diversity, decode and forward, piecewise linear
%\end{IEEEkeywords}



% For peer review papers, you can put extra information on the cover
% page as needed:
% \ifCLASSOPTIONpeerreview
% \begin{center} \bfseries EDICS Category: 3-BBND \end{center}
% \fi
%
% For peerreview papers, this IEEEtran command inserts a page break and
% creates the second title. It will be ignored for other modes.
%\IEEEpeerreviewmaketitle




 \item A student says that if you throw a die, it will show up 1 or not 1. Therefore, the probability of getting 1 and the probability of getting 'not 1' each is equal to $\frac{1}{2}$. Is this correct? Give reasons.\\
 \solution
        %\begin{table}[H]
	\centering
\begin{tabular}{|c|c|c|}
\hline
Random variable &Value &Definition\\ \hline
\multirow{3}{*}{X} &0 &Slips of Rs 1\\
&1 &Slips of Rs 5\\
&2 &Slips of Rs 13\\ \hline
\multirow{2}{*}{Y} &0 &Box A\\
&1 &Box B\\\hline
\end{tabular}
\caption{}
\label{tab:Distribution}
\end{table}
See \tabref{tab:Distribution}.
\begin{align}
p_{Y}\brak{k}= \begin{cases} 
      \frac{1}{3} & {k=0} \\
      \frac{2}{3 }& {k=1} 
   \end{cases}
   \\
p_{Y|X}\brak{0|0} = \frac{19}{25}\, 
p_{Y|X}\brak{0|1} = \frac{6}{25}\,
p_{Y|X}\brak{1|0} = \frac{45}{50}\,
p_{Y|X}\brak{1|2} = \frac{5}{50}
\end{align}
The desired probability is the probability that a slip drawn at random is marked other than Rs 1,
\begin{align}
&=1-p_X\brak{0}\\
&= p_X(1) + p_X(2)
\end{align}
Using Bayes theorem,
\begin{align}
&= p_Y\brak{0} \times \pr{Y=0 | X=1} + p_Y\brak{1} \times \pr{Y=1|X=2}\\
&=\frac{1}{3} \times \frac{6}{25} + \frac{2}{3} \times \frac{5}{50}\\
&=\frac{11}{75}
\end{align}

\newpage

%\tableofcontents

\bigskip

\renewcommand{\thefigure}{\theenumi}
\renewcommand{\thetable}{\theenumi}
%\renewcommand{\theequation}{\theenumi}

%\begin{abstract}
%%\boldmath
%In this letter, an algorithm for evaluating the exact analytical bit error rate  (BER)  for the piecewise linear (PL) combiner for  multiple relays is presented. Previous results were available only for upto three relays. The algorithm is unique in the sense that  the actual mathematical expressions, that are prohibitively large, need not be explicitly obtained. The diversity gain due to multiple relays is shown through plots of the analytical BER, well supported by simulations. 
%
%\end{abstract}
% IEEEtran.cls defaults to using nonbold math in the Abstract.
% This preserves the distinction between vectors and scalars. However,
% if the journal you are submitting to favors bold math in the abstract,
% then you can use LaTeX's standard command \boldmath at the very start
% of the abstract to achieve this. Many IEEE journals frown on math
% in the abstract anyway.

% Note that keywords are not normally used for peerreview papers.
%\begin{IEEEkeywords}
%Cooperative diversity, decode and forward, piecewise linear
%\end{IEEEkeywords}



% For peer review papers, you can put extra information on the cover
% page as needed:
% \ifCLASSOPTIONpeerreview
% \begin{center} \bfseries EDICS Category: 3-BBND \end{center}
% \fi
%
% For peerreview papers, this IEEEtran command inserts a page break and
% creates the second title. It will be ignored for other modes.
%\IEEEpeerreviewmaketitle




   \item Four candidates A, B, C, D have ap-
plied for the assignment to coach a school cricket
team. If A is twice as likely to be selected as B, and
B and C are given about the same chance of being
selected, while C is twice as likely to be selected
as D, what are the probabilities that
\begin{enumerate}
\item C will be selected?
\item A will not be selected?
\end{enumerate}
	%\begin{table}[H]
	\centering
\begin{tabular}{|c|c|c|}
\hline
Random variable &Value &Definition\\ \hline
\multirow{3}{*}{X} &0 &Slips of Rs 1\\
&1 &Slips of Rs 5\\
&2 &Slips of Rs 13\\ \hline
\multirow{2}{*}{Y} &0 &Box A\\
&1 &Box B\\\hline
\end{tabular}
\caption{}
\label{tab:Distribution}
\end{table}
See \tabref{tab:Distribution}.
\begin{align}
p_{Y}\brak{k}= \begin{cases} 
      \frac{1}{3} & {k=0} \\
      \frac{2}{3 }& {k=1} 
   \end{cases}
   \\
p_{Y|X}\brak{0|0} = \frac{19}{25}\, 
p_{Y|X}\brak{0|1} = \frac{6}{25}\,
p_{Y|X}\brak{1|0} = \frac{45}{50}\,
p_{Y|X}\brak{1|2} = \frac{5}{50}
\end{align}
The desired probability is the probability that a slip drawn at random is marked other than Rs 1,
\begin{align}
&=1-p_X\brak{0}\\
&= p_X(1) + p_X(2)
\end{align}
Using Bayes theorem,
\begin{align}
&= p_Y\brak{0} \times \pr{Y=0 | X=1} + p_Y\brak{1} \times \pr{Y=1|X=2}\\
&=\frac{1}{3} \times \frac{6}{25} + \frac{2}{3} \times \frac{5}{50}\\
&=\frac{11}{75}
\end{align}

\newpage

%\tableofcontents

\bigskip

\renewcommand{\thefigure}{\theenumi}
\renewcommand{\thetable}{\theenumi}
%\renewcommand{\theequation}{\theenumi}

%\begin{abstract}
%%\boldmath
%In this letter, an algorithm for evaluating the exact analytical bit error rate  (BER)  for the piecewise linear (PL) combiner for  multiple relays is presented. Previous results were available only for upto three relays. The algorithm is unique in the sense that  the actual mathematical expressions, that are prohibitively large, need not be explicitly obtained. The diversity gain due to multiple relays is shown through plots of the analytical BER, well supported by simulations. 
%
%\end{abstract}
% IEEEtran.cls defaults to using nonbold math in the Abstract.
% This preserves the distinction between vectors and scalars. However,
% if the journal you are submitting to favors bold math in the abstract,
% then you can use LaTeX's standard command \boldmath at the very start
% of the abstract to achieve this. Many IEEE journals frown on math
% in the abstract anyway.

% Note that keywords are not normally used for peerreview papers.
%\begin{IEEEkeywords}
%Cooperative diversity, decode and forward, piecewise linear
%\end{IEEEkeywords}



% For peer review papers, you can put extra information on the cover
% page as needed:
% \ifCLASSOPTIONpeerreview
% \begin{center} \bfseries EDICS Category: 3-BBND \end{center}
% \fi
%
% For peerreview papers, this IEEEtran command inserts a page break and
% creates the second title. It will be ignored for other modes.
%\IEEEpeerreviewmaketitle




 \item A bag contain 24 balls of which $x$ balls are red, $2x$ are white and $3x$ are blue. A ball is selected at random, What is the probability that it is
\begin{enumerate}[label=\alph*)]
\item not red ?
\item white ?
\end{enumerate}
%\begin{table}[H]
	\centering
\begin{tabular}{|c|c|c|}
\hline
Random variable &Value &Definition\\ \hline
\multirow{3}{*}{X} &0 &Slips of Rs 1\\
&1 &Slips of Rs 5\\
&2 &Slips of Rs 13\\ \hline
\multirow{2}{*}{Y} &0 &Box A\\
&1 &Box B\\\hline
\end{tabular}
\caption{}
\label{tab:Distribution}
\end{table}
See \tabref{tab:Distribution}.
\begin{align}
p_{Y}\brak{k}= \begin{cases} 
      \frac{1}{3} & {k=0} \\
      \frac{2}{3 }& {k=1} 
   \end{cases}
   \\
p_{Y|X}\brak{0|0} = \frac{19}{25}\, 
p_{Y|X}\brak{0|1} = \frac{6}{25}\,
p_{Y|X}\brak{1|0} = \frac{45}{50}\,
p_{Y|X}\brak{1|2} = \frac{5}{50}
\end{align}
The desired probability is the probability that a slip drawn at random is marked other than Rs 1,
\begin{align}
&=1-p_X\brak{0}\\
&= p_X(1) + p_X(2)
\end{align}
Using Bayes theorem,
\begin{align}
&= p_Y\brak{0} \times \pr{Y=0 | X=1} + p_Y\brak{1} \times \pr{Y=1|X=2}\\
&=\frac{1}{3} \times \frac{6}{25} + \frac{2}{3} \times \frac{5}{50}\\
&=\frac{11}{75}
\end{align}

\newpage

%\tableofcontents

\bigskip

\renewcommand{\thefigure}{\theenumi}
\renewcommand{\thetable}{\theenumi}
%\renewcommand{\theequation}{\theenumi}

%\begin{abstract}
%%\boldmath
%In this letter, an algorithm for evaluating the exact analytical bit error rate  (BER)  for the piecewise linear (PL) combiner for  multiple relays is presented. Previous results were available only for upto three relays. The algorithm is unique in the sense that  the actual mathematical expressions, that are prohibitively large, need not be explicitly obtained. The diversity gain due to multiple relays is shown through plots of the analytical BER, well supported by simulations. 
%
%\end{abstract}
% IEEEtran.cls defaults to using nonbold math in the Abstract.
% This preserves the distinction between vectors and scalars. However,
% if the journal you are submitting to favors bold math in the abstract,
% then you can use LaTeX's standard command \boldmath at the very start
% of the abstract to achieve this. Many IEEE journals frown on math
% in the abstract anyway.

% Note that keywords are not normally used for peerreview papers.
%\begin{IEEEkeywords}
%Cooperative diversity, decode and forward, piecewise linear
%\end{IEEEkeywords}



% For peer review papers, you can put extra information on the cover
% page as needed:
% \ifCLASSOPTIONpeerreview
% \begin{center} \bfseries EDICS Category: 3-BBND \end{center}
% \fi
%
% For peerreview papers, this IEEEtran command inserts a page break and
% creates the second title. It will be ignored for other modes.
%\IEEEpeerreviewmaketitle




If the letters of the word ASSASSINATION are arranged at random. Find the Probability that
\begin{enumerate}[label=(\alph*)]
\item Four $S's$ come consecutively in the word
\item Two  $I's$ and two $N's$ come together
\item All $A's$ are not coming together
\item No two $A's$ are coming together
\end{enumerate}
%\begin{table}[H]
	\centering
\begin{tabular}{|c|c|c|}
\hline
Random variable &Value &Definition\\ \hline
\multirow{3}{*}{X} &0 &Slips of Rs 1\\
&1 &Slips of Rs 5\\
&2 &Slips of Rs 13\\ \hline
\multirow{2}{*}{Y} &0 &Box A\\
&1 &Box B\\\hline
\end{tabular}
\caption{}
\label{tab:Distribution}
\end{table}
See \tabref{tab:Distribution}.
\begin{align}
p_{Y}\brak{k}= \begin{cases} 
      \frac{1}{3} & {k=0} \\
      \frac{2}{3 }& {k=1} 
   \end{cases}
   \\
p_{Y|X}\brak{0|0} = \frac{19}{25}\, 
p_{Y|X}\brak{0|1} = \frac{6}{25}\,
p_{Y|X}\brak{1|0} = \frac{45}{50}\,
p_{Y|X}\brak{1|2} = \frac{5}{50}
\end{align}
The desired probability is the probability that a slip drawn at random is marked other than Rs 1,
\begin{align}
&=1-p_X\brak{0}\\
&= p_X(1) + p_X(2)
\end{align}
Using Bayes theorem,
\begin{align}
&= p_Y\brak{0} \times \pr{Y=0 | X=1} + p_Y\brak{1} \times \pr{Y=1|X=2}\\
&=\frac{1}{3} \times \frac{6}{25} + \frac{2}{3} \times \frac{5}{50}\\
&=\frac{11}{75}
\end{align}

\newpage

%\tableofcontents

\bigskip

\renewcommand{\thefigure}{\theenumi}
\renewcommand{\thetable}{\theenumi}
%\renewcommand{\theequation}{\theenumi}

%\begin{abstract}
%%\boldmath
%In this letter, an algorithm for evaluating the exact analytical bit error rate  (BER)  for the piecewise linear (PL) combiner for  multiple relays is presented. Previous results were available only for upto three relays. The algorithm is unique in the sense that  the actual mathematical expressions, that are prohibitively large, need not be explicitly obtained. The diversity gain due to multiple relays is shown through plots of the analytical BER, well supported by simulations. 
%
%\end{abstract}
% IEEEtran.cls defaults to using nonbold math in the Abstract.
% This preserves the distinction between vectors and scalars. However,
% if the journal you are submitting to favors bold math in the abstract,
% then you can use LaTeX's standard command \boldmath at the very start
% of the abstract to achieve this. Many IEEE journals frown on math
% in the abstract anyway.

% Note that keywords are not normally used for peerreview papers.
%\begin{IEEEkeywords}
%Cooperative diversity, decode and forward, piecewise linear
%\end{IEEEkeywords}



% For peer review papers, you can put extra information on the cover
% page as needed:
% \ifCLASSOPTIONpeerreview
% \begin{center} \bfseries EDICS Category: 3-BBND \end{center}
% \fi
%
% For peerreview papers, this IEEEtran command inserts a page break and
% creates the second title. It will be ignored for other modes.
%\IEEEpeerreviewmaketitle




	\item One urn contains two black balls (labelled B1 and B2) and one white ball. A
	second urn contains one black ball and two white balls (labelled W1 and W2).
	Suppose the following experiment is performed. One of the two urns is chosen
	at random. Next a ball is randomly chosen from the urn. Then a second ball is
	chosen at random from the same urn without replacing the first ball.
	
	\begin{enumerate}
	\item What is the probability that two black balls are chosen?
	
	\item What is the probability that two balls of opposite colour are chosen?
	\end{enumerate}
	\solution
	%\begin{align}
    \label{eq:12.13.6.18.1}
	\because	\pr{A|B} &> \pr{A},\
\frac{\pr{AB}}{\pr{B}} > \pr{A}
\\
    \label{eq:12.13.6.18.2}
	\implies \pr{AB} &> \pr{A}\pr{B}
	\\
	\text{or, } \frac{\pr{AB}}{\pr{A}} &=\pr{B|A} > \pr{A}
\end{align}

\end{enumerate}

		\item A box of oranges is inspected by examining three randomly selected oranges drawn without replacement. If all the three oranges are good, the box is approved for sale, otherwise, it is rejected. Find the probability that a box containing 15 oranges out of which 12 are good and 3 are bad ones will be approved for sale.
		\label{ncert/12/13/2/3/defs.tex}
		\item Two balls are drawn at random with replacement from a box containing 10 black and 8 red balls. Find the probability that
		\label{ncert/12/13/2/12}
\begin{enumerate}
\item both balls are red.
\item first ball is black and second is red.
\item one of them is black and other is red.
\end{enumerate}

\item In a hostel, 60\% of the students read Hindi newspaper, 40\% read English newspaper and 20\% read both Hindi and English newspapers. A student is selected at random.
		\label{ncert/12/13/2/15}
\begin{enumerate}
\item Find the probability that she reads neither Hindi nor English newspapers.
\item If she reads Hindi newspaper, find the probability that she reads English newspaper.
\item If she reads English newspaper, find the probability that she reads Hindi newspaper.\\
\end{enumerate}
\item The probability of obtaining an even prime number on each die, when a pair of dice is rolled is 
\begin{enumerate}
    \item $0$ 
    
    \item $\frac{1}{3}$ 
    
    \item $\frac{1}{12}$ 
    
    \item $\frac{1}{36}$ 
\end{enumerate}
\solution
		%\begin{enumerate}[label=\thesection.\arabic*,ref=\thesection.\theenumi]
	\item One card is drawn from a well-shuffled deck of 52 cards. Find the probability of getting
\begin{enumerate}
\item A king of red colour 
\item A face card 
\item A red face card
\item The jack of hearts
\item A spade
\item The queen of diamonds

\end{enumerate}
\solution
		%\begin{table}[H]
	\centering
\begin{tabular}{|c|c|c|}
\hline
Random variable &Value &Definition\\ \hline
\multirow{3}{*}{X} &0 &Slips of Rs 1\\
&1 &Slips of Rs 5\\
&2 &Slips of Rs 13\\ \hline
\multirow{2}{*}{Y} &0 &Box A\\
&1 &Box B\\\hline
\end{tabular}
\caption{}
\label{tab:Distribution}
\end{table}
See \tabref{tab:Distribution}.
\begin{align}
p_{Y}\brak{k}= \begin{cases} 
      \frac{1}{3} & {k=0} \\
      \frac{2}{3 }& {k=1} 
   \end{cases}
   \\
p_{Y|X}\brak{0|0} = \frac{19}{25}\, 
p_{Y|X}\brak{0|1} = \frac{6}{25}\,
p_{Y|X}\brak{1|0} = \frac{45}{50}\,
p_{Y|X}\brak{1|2} = \frac{5}{50}
\end{align}
The desired probability is the probability that a slip drawn at random is marked other than Rs 1,
\begin{align}
&=1-p_X\brak{0}\\
&= p_X(1) + p_X(2)
\end{align}
Using Bayes theorem,
\begin{align}
&= p_Y\brak{0} \times \pr{Y=0 | X=1} + p_Y\brak{1} \times \pr{Y=1|X=2}\\
&=\frac{1}{3} \times \frac{6}{25} + \frac{2}{3} \times \frac{5}{50}\\
&=\frac{11}{75}
\end{align}

\newpage

%\tableofcontents

\bigskip

\renewcommand{\thefigure}{\theenumi}
\renewcommand{\thetable}{\theenumi}
%\renewcommand{\theequation}{\theenumi}

%\begin{abstract}
%%\boldmath
%In this letter, an algorithm for evaluating the exact analytical bit error rate  (BER)  for the piecewise linear (PL) combiner for  multiple relays is presented. Previous results were available only for upto three relays. The algorithm is unique in the sense that  the actual mathematical expressions, that are prohibitively large, need not be explicitly obtained. The diversity gain due to multiple relays is shown through plots of the analytical BER, well supported by simulations. 
%
%\end{abstract}
% IEEEtran.cls defaults to using nonbold math in the Abstract.
% This preserves the distinction between vectors and scalars. However,
% if the journal you are submitting to favors bold math in the abstract,
% then you can use LaTeX's standard command \boldmath at the very start
% of the abstract to achieve this. Many IEEE journals frown on math
% in the abstract anyway.

% Note that keywords are not normally used for peerreview papers.
%\begin{IEEEkeywords}
%Cooperative diversity, decode and forward, piecewise linear
%\end{IEEEkeywords}



% For peer review papers, you can put extra information on the cover
% page as needed:
% \ifCLASSOPTIONpeerreview
% \begin{center} \bfseries EDICS Category: 3-BBND \end{center}
% \fi
%
% For peerreview papers, this IEEEtran command inserts a page break and
% creates the second title. It will be ignored for other modes.
%\IEEEpeerreviewmaketitle




	\item Five cards—the ten, jack, queen, king and ace of diamonds, are well-shuffled with their face downwards. One card is then picked up at random.
\begin{enumerate}
\item
What is the probability that the card is the queen? 
\item
If the queen is drawn and put aside, what is the probability that the second card picked up is (a) an ace? (b) a queen?\\
\end{enumerate}
\solution
		%\begin{enumerate}[label=\thesection.\arabic*,ref=\thesection.\theenumi]
	\item One card is drawn from a well-shuffled deck of 52 cards. Find the probability of getting
\begin{enumerate}
\item A king of red colour 
\item A face card 
\item A red face card
\item The jack of hearts
\item A spade
\item The queen of diamonds

\end{enumerate}
\solution
		%\input{ncert/10/15/1/14/main.tex}
	\item Five cards—the ten, jack, queen, king and ace of diamonds, are well-shuffled with their face downwards. One card is then picked up at random.
\begin{enumerate}
\item
What is the probability that the card is the queen? 
\item
If the queen is drawn and put aside, what is the probability that the second card picked up is (a) an ace? (b) a queen?\\
\end{enumerate}
\solution
		%\input{ncert/10/15/1/15/defs.tex}
	\item A bag contains $5$ red balls and some blue balls. If the probability of drawing a blue ball is double that if a red ball, determine the number of blue balls in the bag. 
		\\
\solution
		%\input{ncert/10/15/2/3/defs.tex}
	\item A card is selected from a pack of 52 cards.
 \begin{enumerate}[label=(\alph*)] 
                 \item How many points are there in the sample space?
                 \item Calculate the probability that the card is an ace of spades.
                 \item Calculate the probability that the card is (i) an ace and (ii) black card.
 \end{enumerate}
\solution
		%\input{ncert/11/16/3/4/main.tex}
\item Four cards are drawn from a well-shuffled deck of 52 cards. What is the probability of obtaining 3 diamonds and one spade.
\\
\solution
		%\input{ncert/11/16/4/2/defs.tex}
\item In a certain lottery 10,000 tickets are sold and ten equal prizes are awarded. What is the probability of not getting a prize if you buy (a) one ticket (b) two tickets (c) 10 tickets ?	
\\
\solution
		%\input{ncert/11/16/4/4/defs.tex}
		%
\item 
Out of 100 students, two sections of 40 and 60 are formed. If you and your friend are among the 100 students, what is the probability that
\begin{enumerate}
\item you both enter the same section?
\item you both enter the different sections?
\end{enumerate}
\solution
		%\input{ncert/11/16/4/5/defs.tex}
	\item 
The number lock of a suitcase has 4 wheels each labelled with ten digits i.e. from 0 to 9.The lock opens with a sequence of four digits with no repeats.What is the probability of a person getting the right sequence to open the suitcase.
\\
\solution
		%\input{ncert/11/16/4/10/defs.tex}
		%
\item 
Two cards are drawn at random and without replacement from a pack of 52 playing cards. Find the probability that both the cards are black.
\\
\solution
		%\input{ncert/12/13/2/2/defs.tex}
		\item A box of oranges is inspected by examining three randomly selected oranges drawn without replacement. If all the three oranges are good, the box is approved for sale, otherwise, it is rejected. Find the probability that a box containing 15 oranges out of which 12 are good and 3 are bad ones will be approved for sale.
		\label{ncert/12/13/2/3/defs.tex}
		\item Two balls are drawn at random with replacement from a box containing 10 black and 8 red balls. Find the probability that
		\label{ncert/12/13/2/12}
\begin{enumerate}
\item both balls are red.
\item first ball is black and second is red.
\item one of them is black and other is red.
\end{enumerate}

\item In a hostel, 60\% of the students read Hindi newspaper, 40\% read English newspaper and 20\% read both Hindi and English newspapers. A student is selected at random.
		\label{ncert/12/13/2/15}
\begin{enumerate}
\item Find the probability that she reads neither Hindi nor English newspapers.
\item If she reads Hindi newspaper, find the probability that she reads English newspaper.
\item If she reads English newspaper, find the probability that she reads Hindi newspaper.\\
\end{enumerate}
\item The probability of obtaining an even prime number on each die, when a pair of dice is rolled is 
\begin{enumerate}
    \item $0$ 
    
    \item $\frac{1}{3}$ 
    
    \item $\frac{1}{12}$ 
    
    \item $\frac{1}{36}$ 
\end{enumerate}
\solution
		%\input{ncert/12/13/2/17/defs.tex}
	\item A bag contains 4 red and 4 black balls, another bag contains 2 red and 6 black balls. One of the two bags is selected at random and a ball is drawn from the bag which is found to be red. Find the probability that the ball is drawn from the first bag.
\\
\solution
		%\input{ncert/12/13/3/2/main.tex}
  \item
  Cards with numbers 2 to 101 are placed in a box. A card is selected at random.Find the probability that the card has
\begin{enumerate}[label=(\roman*)]
	\item an even number 
	\item a square number
\end{enumerate}
\solution
%\input{exemplar/10/13/3/32/main.tex}
\item
The king, queen and jack of clubs are removed from a deck of 52 playing cards and then well shuffled. Now one card is drawn at random from the remaining cards.  Determine the probability that the card is
\begin{enumerate}[label=(\roman*)]
\item a club
\item 10 of hearts
\end{enumerate}
\solution
%\input{exemplar/10/13/3/29/main.tex}
\item A team of medical students doing their internship have to assist during surgeries
at a city hospital. The probabilities of surgeries rated as very complex, complex,
routine, simple or very simple are respectively, 0.15, 0.20, 0.31, 0.26, .08. Find
the probabilities that a particular surgery will be rated
\begin{enumerate}
	\item complex or very complex;
	\item neither very complex nor very simple;
	\item routine or complex
	\item routine or simple
\end{enumerate}
\solution
%\input{exemplar/11/16/3/8(1)/main.tex}
\item A card is selected from a pack of 52 cards.
\begin{enumerate}[label=(\alph*)]
    \item How many points are there in the sample space?
    \item Calculate the probability that the card is an ace of spades.
    \item Calculate the probability that the card is (i) an ace and (ii) black card.
\end{enumerate}
\solution
%\input{exemplar/11/16/3/4/main2.tex}
\item The probability that a non leap year selected at random will contain 53 sundays.
\\
\solution
%\input{exemplar/10/13/1/19/main.tex}
\item One of the four persons John, Rita, Aslam or Gurpreet will be promoted next
month. Consequently the sample space consists of four elementary outcomes
S = {John promoted, Rita promoted, Aslam promoted, Gurpreet promoted}
You are told that the chances of John’s promotion is same as that of Gurpreet,
Rita’s chances of promotion are twice as likely as Johns. Aslam’s chances are
four times that of John.
\begin{enumerate}
	\item Determine
	\begin{enumerate}
		\item P (John promoted)
		\item P (Rita promoted)
		\item P (Aslam promoted)
		\item P (Gurpreet promoted)
	\end{enumerate}
	\item If A = {John promoted or Gurpreet promoted}, find P (A).
\end{enumerate}
\solution
%\input{exemplar/11/16/3/10/main.tex}
\item A card is drawn from a deck of 52 cards. Find the probability of getting a king or a heart or a red card.\\
\solution
%\input{exemplar/11/16/3/15/main.tex}
\item The probability that a student will pass his examination is 0.73, the probability of
the student getting a compartment is 0.13, and the probability that the student will
either pass or get compartment is 0.96. State True or False.\\
\solution
%\input{exemplar/11/16/3/31/main.tex}
\item A card is selected from a pack of 52 cards\\
\begin{enumerate}[label=(\alph*)]
\item How many points are there in the sample space?
\item Calculate the probability that the cards is an ace of spades.
\item Calculate the probability that the card is (i) an ace (ii)black card.\\
\end{enumerate}
%\input{ncert/11/16/3/4_1/Prob_4.tex}
\item In a non-leap year, the probability of having 53 tuesdays or 53 wednesdays is\\
\solution
%\input{exemplar/11/16/3/18/main.tex}
\item There are 1000 sealed envelopes in a box, 10 of them contain a cash prize of
Rs 100 each, 100 of them contain a cash prize of Rs 50 each and 200 of them
contain a cash prize of Rs 10 each and rest do not contain any cash prize. If they
are well shuffled and an envelope is picked up out, what is the probability that it
contains no cash prize?\\
\solution
%\input{exemplar/10/13/3/34/main.tex}
\item 
A die is thrown and a card is selected at random from a deck of 52 playing cards. The probability of getting an even number on the die and a spade card.\\
\solution
%\input{exemplar/12/13/3/78/main.tex}
\item
If 4-digit numbers greater than 5,000 are randomly formed from the digits 0, 1, 3, 5, and 7, what is the probability of forming a number divisible by 5 when:
\begin{enumerate}
    \item The digits are repeated?
    \item The repetition of digits is not allowed?
\end{enumerate}
\solution
%\input{ncert/11/16/4/9/main.tex}
\item Consider the probability space $\brak{\Omega, \mathcal{G}, P}$ where $\Omega = [0,2]$ and $\mathcal{G} = \cbrak{\phi, \Omega, [0,1], (1,2]}$. Let $X$ and $Y$ be two functions on $\Omega$ defined as
\begin{align*}
    X(\omega) = 
    \begin{cases}
        1 & \text{if }\omega \in [0, 1]\\
        2 & \text{if }\omega \in (1, 2]
    \end{cases}
\end{align*}
and
\begin{align*}
    Y(\omega) = 
    \begin{cases}
        2 & \text{if }\omega \in [0, 1.5]\\
        3 & \text{if }\omega \in (1.5, 2].
    \end{cases}
\end{align*}
Then which one of the following statements is true?
\begin{enumerate}
    \item [(A)] $X$ is a random variable with respect to $\mathcal{G}$, but $Y$ is not a random variable with respect to $\mathcal{G}$.
    \item [(B)] $Y$ is a random variable with respect to $\mathcal{G}$, but $X$ is not a random variable with respect to $\mathcal{G}$.
    \item [(C)] Neither $X$ nor $Y$ is a random variable with respect to $\mathcal{G}$.
    \item [(D)] Both $X$ and $Y$ are random variables with respect to $\mathcal{G}$.
\end{enumerate} \hfill (GATE ST 2023)\\
\solution
%\input{gate/ST/2023/14/main.tex}
	\item  A die is loaded in such a way that each odd number is twice as likely to occur as
each even number. Find $P(G)$, where $G$ is the event that a number greater than
3 occurs on a single roll of the die.
\\
\solution
		%\input{exemplar/11/16/3/5/main.tex}
	\item All the jacks, queens and kings are removed from a deck of 52 playing cards. The remaining cards are well shuffled and then one card is drawn at random. Giving ace a value 1 similar value for other cards, find the probability that the card has a value 
		\begin{enumerate}
			\item 7
			\item greater than 7
			\item less than 7
		\end{enumerate}
		%\input{exemplar/10/13/3/30/main.tex}
  \item A Lot consists of 48 mobile phones of which 42 are good, 3 have only minor defects and 3 have major defects.Varnika will buy a phone if it is good but the trader will only buy a mobile if it has no major defects. One phone is selected at random from the lot. What is the probability that it is
\begin{enumerate}
	\item acceptable to Varnika?
            \item acceptable to the trader?
\end{enumerate}
\solution
	%\input{exemplar/10/13/3/40/main.tex}
 \item A student says that if you throw a die, it will show up 1 or not 1. Therefore, the probability of getting 1 and the probability of getting 'not 1' each is equal to $\frac{1}{2}$. Is this correct? Give reasons.\\
 \solution
        %\input{exemplar/10/13/2/9/main.tex}
   \item Four candidates A, B, C, D have ap-
plied for the assignment to coach a school cricket
team. If A is twice as likely to be selected as B, and
B and C are given about the same chance of being
selected, while C is twice as likely to be selected
as D, what are the probabilities that
\begin{enumerate}
\item C will be selected?
\item A will not be selected?
\end{enumerate}
	%\input{exemplar/11/16/3/9/main.tex}
 \item A bag contain 24 balls of which $x$ balls are red, $2x$ are white and $3x$ are blue. A ball is selected at random, What is the probability that it is
\begin{enumerate}[label=\alph*)]
\item not red ?
\item white ?
\end{enumerate}
%\input{exemplar/10/13/3/41/main.tex}
If the letters of the word ASSASSINATION are arranged at random. Find the Probability that
\begin{enumerate}[label=(\alph*)]
\item Four $S's$ come consecutively in the word
\item Two  $I's$ and two $N's$ come together
\item All $A's$ are not coming together
\item No two $A's$ are coming together
\end{enumerate}
%\input{exemplar/11/16/3/14/main.tex}
	\item One urn contains two black balls (labelled B1 and B2) and one white ball. A
	second urn contains one black ball and two white balls (labelled W1 and W2).
	Suppose the following experiment is performed. One of the two urns is chosen
	at random. Next a ball is randomly chosen from the urn. Then a second ball is
	chosen at random from the same urn without replacing the first ball.
	
	\begin{enumerate}
	\item What is the probability that two black balls are chosen?
	
	\item What is the probability that two balls of opposite colour are chosen?
	\end{enumerate}
	\solution
	%\input{exemplar/11/16/3/12/main1.tex}
\end{enumerate}

	\item A bag contains $5$ red balls and some blue balls. If the probability of drawing a blue ball is double that if a red ball, determine the number of blue balls in the bag. 
		\\
\solution
		%\begin{enumerate}[label=\thesection.\arabic*,ref=\thesection.\theenumi]
	\item One card is drawn from a well-shuffled deck of 52 cards. Find the probability of getting
\begin{enumerate}
\item A king of red colour 
\item A face card 
\item A red face card
\item The jack of hearts
\item A spade
\item The queen of diamonds

\end{enumerate}
\solution
		%\input{ncert/10/15/1/14/main.tex}
	\item Five cards—the ten, jack, queen, king and ace of diamonds, are well-shuffled with their face downwards. One card is then picked up at random.
\begin{enumerate}
\item
What is the probability that the card is the queen? 
\item
If the queen is drawn and put aside, what is the probability that the second card picked up is (a) an ace? (b) a queen?\\
\end{enumerate}
\solution
		%\input{ncert/10/15/1/15/defs.tex}
	\item A bag contains $5$ red balls and some blue balls. If the probability of drawing a blue ball is double that if a red ball, determine the number of blue balls in the bag. 
		\\
\solution
		%\input{ncert/10/15/2/3/defs.tex}
	\item A card is selected from a pack of 52 cards.
 \begin{enumerate}[label=(\alph*)] 
                 \item How many points are there in the sample space?
                 \item Calculate the probability that the card is an ace of spades.
                 \item Calculate the probability that the card is (i) an ace and (ii) black card.
 \end{enumerate}
\solution
		%\input{ncert/11/16/3/4/main.tex}
\item Four cards are drawn from a well-shuffled deck of 52 cards. What is the probability of obtaining 3 diamonds and one spade.
\\
\solution
		%\input{ncert/11/16/4/2/defs.tex}
\item In a certain lottery 10,000 tickets are sold and ten equal prizes are awarded. What is the probability of not getting a prize if you buy (a) one ticket (b) two tickets (c) 10 tickets ?	
\\
\solution
		%\input{ncert/11/16/4/4/defs.tex}
		%
\item 
Out of 100 students, two sections of 40 and 60 are formed. If you and your friend are among the 100 students, what is the probability that
\begin{enumerate}
\item you both enter the same section?
\item you both enter the different sections?
\end{enumerate}
\solution
		%\input{ncert/11/16/4/5/defs.tex}
	\item 
The number lock of a suitcase has 4 wheels each labelled with ten digits i.e. from 0 to 9.The lock opens with a sequence of four digits with no repeats.What is the probability of a person getting the right sequence to open the suitcase.
\\
\solution
		%\input{ncert/11/16/4/10/defs.tex}
		%
\item 
Two cards are drawn at random and without replacement from a pack of 52 playing cards. Find the probability that both the cards are black.
\\
\solution
		%\input{ncert/12/13/2/2/defs.tex}
		\item A box of oranges is inspected by examining three randomly selected oranges drawn without replacement. If all the three oranges are good, the box is approved for sale, otherwise, it is rejected. Find the probability that a box containing 15 oranges out of which 12 are good and 3 are bad ones will be approved for sale.
		\label{ncert/12/13/2/3/defs.tex}
		\item Two balls are drawn at random with replacement from a box containing 10 black and 8 red balls. Find the probability that
		\label{ncert/12/13/2/12}
\begin{enumerate}
\item both balls are red.
\item first ball is black and second is red.
\item one of them is black and other is red.
\end{enumerate}

\item In a hostel, 60\% of the students read Hindi newspaper, 40\% read English newspaper and 20\% read both Hindi and English newspapers. A student is selected at random.
		\label{ncert/12/13/2/15}
\begin{enumerate}
\item Find the probability that she reads neither Hindi nor English newspapers.
\item If she reads Hindi newspaper, find the probability that she reads English newspaper.
\item If she reads English newspaper, find the probability that she reads Hindi newspaper.\\
\end{enumerate}
\item The probability of obtaining an even prime number on each die, when a pair of dice is rolled is 
\begin{enumerate}
    \item $0$ 
    
    \item $\frac{1}{3}$ 
    
    \item $\frac{1}{12}$ 
    
    \item $\frac{1}{36}$ 
\end{enumerate}
\solution
		%\input{ncert/12/13/2/17/defs.tex}
	\item A bag contains 4 red and 4 black balls, another bag contains 2 red and 6 black balls. One of the two bags is selected at random and a ball is drawn from the bag which is found to be red. Find the probability that the ball is drawn from the first bag.
\\
\solution
		%\input{ncert/12/13/3/2/main.tex}
  \item
  Cards with numbers 2 to 101 are placed in a box. A card is selected at random.Find the probability that the card has
\begin{enumerate}[label=(\roman*)]
	\item an even number 
	\item a square number
\end{enumerate}
\solution
%\input{exemplar/10/13/3/32/main.tex}
\item
The king, queen and jack of clubs are removed from a deck of 52 playing cards and then well shuffled. Now one card is drawn at random from the remaining cards.  Determine the probability that the card is
\begin{enumerate}[label=(\roman*)]
\item a club
\item 10 of hearts
\end{enumerate}
\solution
%\input{exemplar/10/13/3/29/main.tex}
\item A team of medical students doing their internship have to assist during surgeries
at a city hospital. The probabilities of surgeries rated as very complex, complex,
routine, simple or very simple are respectively, 0.15, 0.20, 0.31, 0.26, .08. Find
the probabilities that a particular surgery will be rated
\begin{enumerate}
	\item complex or very complex;
	\item neither very complex nor very simple;
	\item routine or complex
	\item routine or simple
\end{enumerate}
\solution
%\input{exemplar/11/16/3/8(1)/main.tex}
\item A card is selected from a pack of 52 cards.
\begin{enumerate}[label=(\alph*)]
    \item How many points are there in the sample space?
    \item Calculate the probability that the card is an ace of spades.
    \item Calculate the probability that the card is (i) an ace and (ii) black card.
\end{enumerate}
\solution
%\input{exemplar/11/16/3/4/main2.tex}
\item The probability that a non leap year selected at random will contain 53 sundays.
\\
\solution
%\input{exemplar/10/13/1/19/main.tex}
\item One of the four persons John, Rita, Aslam or Gurpreet will be promoted next
month. Consequently the sample space consists of four elementary outcomes
S = {John promoted, Rita promoted, Aslam promoted, Gurpreet promoted}
You are told that the chances of John’s promotion is same as that of Gurpreet,
Rita’s chances of promotion are twice as likely as Johns. Aslam’s chances are
four times that of John.
\begin{enumerate}
	\item Determine
	\begin{enumerate}
		\item P (John promoted)
		\item P (Rita promoted)
		\item P (Aslam promoted)
		\item P (Gurpreet promoted)
	\end{enumerate}
	\item If A = {John promoted or Gurpreet promoted}, find P (A).
\end{enumerate}
\solution
%\input{exemplar/11/16/3/10/main.tex}
\item A card is drawn from a deck of 52 cards. Find the probability of getting a king or a heart or a red card.\\
\solution
%\input{exemplar/11/16/3/15/main.tex}
\item The probability that a student will pass his examination is 0.73, the probability of
the student getting a compartment is 0.13, and the probability that the student will
either pass or get compartment is 0.96. State True or False.\\
\solution
%\input{exemplar/11/16/3/31/main.tex}
\item A card is selected from a pack of 52 cards\\
\begin{enumerate}[label=(\alph*)]
\item How many points are there in the sample space?
\item Calculate the probability that the cards is an ace of spades.
\item Calculate the probability that the card is (i) an ace (ii)black card.\\
\end{enumerate}
%\input{ncert/11/16/3/4_1/Prob_4.tex}
\item In a non-leap year, the probability of having 53 tuesdays or 53 wednesdays is\\
\solution
%\input{exemplar/11/16/3/18/main.tex}
\item There are 1000 sealed envelopes in a box, 10 of them contain a cash prize of
Rs 100 each, 100 of them contain a cash prize of Rs 50 each and 200 of them
contain a cash prize of Rs 10 each and rest do not contain any cash prize. If they
are well shuffled and an envelope is picked up out, what is the probability that it
contains no cash prize?\\
\solution
%\input{exemplar/10/13/3/34/main.tex}
\item 
A die is thrown and a card is selected at random from a deck of 52 playing cards. The probability of getting an even number on the die and a spade card.\\
\solution
%\input{exemplar/12/13/3/78/main.tex}
\item
If 4-digit numbers greater than 5,000 are randomly formed from the digits 0, 1, 3, 5, and 7, what is the probability of forming a number divisible by 5 when:
\begin{enumerate}
    \item The digits are repeated?
    \item The repetition of digits is not allowed?
\end{enumerate}
\solution
%\input{ncert/11/16/4/9/main.tex}
\item Consider the probability space $\brak{\Omega, \mathcal{G}, P}$ where $\Omega = [0,2]$ and $\mathcal{G} = \cbrak{\phi, \Omega, [0,1], (1,2]}$. Let $X$ and $Y$ be two functions on $\Omega$ defined as
\begin{align*}
    X(\omega) = 
    \begin{cases}
        1 & \text{if }\omega \in [0, 1]\\
        2 & \text{if }\omega \in (1, 2]
    \end{cases}
\end{align*}
and
\begin{align*}
    Y(\omega) = 
    \begin{cases}
        2 & \text{if }\omega \in [0, 1.5]\\
        3 & \text{if }\omega \in (1.5, 2].
    \end{cases}
\end{align*}
Then which one of the following statements is true?
\begin{enumerate}
    \item [(A)] $X$ is a random variable with respect to $\mathcal{G}$, but $Y$ is not a random variable with respect to $\mathcal{G}$.
    \item [(B)] $Y$ is a random variable with respect to $\mathcal{G}$, but $X$ is not a random variable with respect to $\mathcal{G}$.
    \item [(C)] Neither $X$ nor $Y$ is a random variable with respect to $\mathcal{G}$.
    \item [(D)] Both $X$ and $Y$ are random variables with respect to $\mathcal{G}$.
\end{enumerate} \hfill (GATE ST 2023)\\
\solution
%\input{gate/ST/2023/14/main.tex}
	\item  A die is loaded in such a way that each odd number is twice as likely to occur as
each even number. Find $P(G)$, where $G$ is the event that a number greater than
3 occurs on a single roll of the die.
\\
\solution
		%\input{exemplar/11/16/3/5/main.tex}
	\item All the jacks, queens and kings are removed from a deck of 52 playing cards. The remaining cards are well shuffled and then one card is drawn at random. Giving ace a value 1 similar value for other cards, find the probability that the card has a value 
		\begin{enumerate}
			\item 7
			\item greater than 7
			\item less than 7
		\end{enumerate}
		%\input{exemplar/10/13/3/30/main.tex}
  \item A Lot consists of 48 mobile phones of which 42 are good, 3 have only minor defects and 3 have major defects.Varnika will buy a phone if it is good but the trader will only buy a mobile if it has no major defects. One phone is selected at random from the lot. What is the probability that it is
\begin{enumerate}
	\item acceptable to Varnika?
            \item acceptable to the trader?
\end{enumerate}
\solution
	%\input{exemplar/10/13/3/40/main.tex}
 \item A student says that if you throw a die, it will show up 1 or not 1. Therefore, the probability of getting 1 and the probability of getting 'not 1' each is equal to $\frac{1}{2}$. Is this correct? Give reasons.\\
 \solution
        %\input{exemplar/10/13/2/9/main.tex}
   \item Four candidates A, B, C, D have ap-
plied for the assignment to coach a school cricket
team. If A is twice as likely to be selected as B, and
B and C are given about the same chance of being
selected, while C is twice as likely to be selected
as D, what are the probabilities that
\begin{enumerate}
\item C will be selected?
\item A will not be selected?
\end{enumerate}
	%\input{exemplar/11/16/3/9/main.tex}
 \item A bag contain 24 balls of which $x$ balls are red, $2x$ are white and $3x$ are blue. A ball is selected at random, What is the probability that it is
\begin{enumerate}[label=\alph*)]
\item not red ?
\item white ?
\end{enumerate}
%\input{exemplar/10/13/3/41/main.tex}
If the letters of the word ASSASSINATION are arranged at random. Find the Probability that
\begin{enumerate}[label=(\alph*)]
\item Four $S's$ come consecutively in the word
\item Two  $I's$ and two $N's$ come together
\item All $A's$ are not coming together
\item No two $A's$ are coming together
\end{enumerate}
%\input{exemplar/11/16/3/14/main.tex}
	\item One urn contains two black balls (labelled B1 and B2) and one white ball. A
	second urn contains one black ball and two white balls (labelled W1 and W2).
	Suppose the following experiment is performed. One of the two urns is chosen
	at random. Next a ball is randomly chosen from the urn. Then a second ball is
	chosen at random from the same urn without replacing the first ball.
	
	\begin{enumerate}
	\item What is the probability that two black balls are chosen?
	
	\item What is the probability that two balls of opposite colour are chosen?
	\end{enumerate}
	\solution
	%\input{exemplar/11/16/3/12/main1.tex}
\end{enumerate}

	\item A card is selected from a pack of 52 cards.
 \begin{enumerate}[label=(\alph*)] 
                 \item How many points are there in the sample space?
                 \item Calculate the probability that the card is an ace of spades.
                 \item Calculate the probability that the card is (i) an ace and (ii) black card.
 \end{enumerate}
\solution
		%\begin{table}[H]
	\centering
\begin{tabular}{|c|c|c|}
\hline
Random variable &Value &Definition\\ \hline
\multirow{3}{*}{X} &0 &Slips of Rs 1\\
&1 &Slips of Rs 5\\
&2 &Slips of Rs 13\\ \hline
\multirow{2}{*}{Y} &0 &Box A\\
&1 &Box B\\\hline
\end{tabular}
\caption{}
\label{tab:Distribution}
\end{table}
See \tabref{tab:Distribution}.
\begin{align}
p_{Y}\brak{k}= \begin{cases} 
      \frac{1}{3} & {k=0} \\
      \frac{2}{3 }& {k=1} 
   \end{cases}
   \\
p_{Y|X}\brak{0|0} = \frac{19}{25}\, 
p_{Y|X}\brak{0|1} = \frac{6}{25}\,
p_{Y|X}\brak{1|0} = \frac{45}{50}\,
p_{Y|X}\brak{1|2} = \frac{5}{50}
\end{align}
The desired probability is the probability that a slip drawn at random is marked other than Rs 1,
\begin{align}
&=1-p_X\brak{0}\\
&= p_X(1) + p_X(2)
\end{align}
Using Bayes theorem,
\begin{align}
&= p_Y\brak{0} \times \pr{Y=0 | X=1} + p_Y\brak{1} \times \pr{Y=1|X=2}\\
&=\frac{1}{3} \times \frac{6}{25} + \frac{2}{3} \times \frac{5}{50}\\
&=\frac{11}{75}
\end{align}

\newpage

%\tableofcontents

\bigskip

\renewcommand{\thefigure}{\theenumi}
\renewcommand{\thetable}{\theenumi}
%\renewcommand{\theequation}{\theenumi}

%\begin{abstract}
%%\boldmath
%In this letter, an algorithm for evaluating the exact analytical bit error rate  (BER)  for the piecewise linear (PL) combiner for  multiple relays is presented. Previous results were available only for upto three relays. The algorithm is unique in the sense that  the actual mathematical expressions, that are prohibitively large, need not be explicitly obtained. The diversity gain due to multiple relays is shown through plots of the analytical BER, well supported by simulations. 
%
%\end{abstract}
% IEEEtran.cls defaults to using nonbold math in the Abstract.
% This preserves the distinction between vectors and scalars. However,
% if the journal you are submitting to favors bold math in the abstract,
% then you can use LaTeX's standard command \boldmath at the very start
% of the abstract to achieve this. Many IEEE journals frown on math
% in the abstract anyway.

% Note that keywords are not normally used for peerreview papers.
%\begin{IEEEkeywords}
%Cooperative diversity, decode and forward, piecewise linear
%\end{IEEEkeywords}



% For peer review papers, you can put extra information on the cover
% page as needed:
% \ifCLASSOPTIONpeerreview
% \begin{center} \bfseries EDICS Category: 3-BBND \end{center}
% \fi
%
% For peerreview papers, this IEEEtran command inserts a page break and
% creates the second title. It will be ignored for other modes.
%\IEEEpeerreviewmaketitle




\item Four cards are drawn from a well-shuffled deck of 52 cards. What is the probability of obtaining 3 diamonds and one spade.
\\
\solution
		%\begin{enumerate}[label=\thesection.\arabic*,ref=\thesection.\theenumi]
	\item One card is drawn from a well-shuffled deck of 52 cards. Find the probability of getting
\begin{enumerate}
\item A king of red colour 
\item A face card 
\item A red face card
\item The jack of hearts
\item A spade
\item The queen of diamonds

\end{enumerate}
\solution
		%\input{ncert/10/15/1/14/main.tex}
	\item Five cards—the ten, jack, queen, king and ace of diamonds, are well-shuffled with their face downwards. One card is then picked up at random.
\begin{enumerate}
\item
What is the probability that the card is the queen? 
\item
If the queen is drawn and put aside, what is the probability that the second card picked up is (a) an ace? (b) a queen?\\
\end{enumerate}
\solution
		%\input{ncert/10/15/1/15/defs.tex}
	\item A bag contains $5$ red balls and some blue balls. If the probability of drawing a blue ball is double that if a red ball, determine the number of blue balls in the bag. 
		\\
\solution
		%\input{ncert/10/15/2/3/defs.tex}
	\item A card is selected from a pack of 52 cards.
 \begin{enumerate}[label=(\alph*)] 
                 \item How many points are there in the sample space?
                 \item Calculate the probability that the card is an ace of spades.
                 \item Calculate the probability that the card is (i) an ace and (ii) black card.
 \end{enumerate}
\solution
		%\input{ncert/11/16/3/4/main.tex}
\item Four cards are drawn from a well-shuffled deck of 52 cards. What is the probability of obtaining 3 diamonds and one spade.
\\
\solution
		%\input{ncert/11/16/4/2/defs.tex}
\item In a certain lottery 10,000 tickets are sold and ten equal prizes are awarded. What is the probability of not getting a prize if you buy (a) one ticket (b) two tickets (c) 10 tickets ?	
\\
\solution
		%\input{ncert/11/16/4/4/defs.tex}
		%
\item 
Out of 100 students, two sections of 40 and 60 are formed. If you and your friend are among the 100 students, what is the probability that
\begin{enumerate}
\item you both enter the same section?
\item you both enter the different sections?
\end{enumerate}
\solution
		%\input{ncert/11/16/4/5/defs.tex}
	\item 
The number lock of a suitcase has 4 wheels each labelled with ten digits i.e. from 0 to 9.The lock opens with a sequence of four digits with no repeats.What is the probability of a person getting the right sequence to open the suitcase.
\\
\solution
		%\input{ncert/11/16/4/10/defs.tex}
		%
\item 
Two cards are drawn at random and without replacement from a pack of 52 playing cards. Find the probability that both the cards are black.
\\
\solution
		%\input{ncert/12/13/2/2/defs.tex}
		\item A box of oranges is inspected by examining three randomly selected oranges drawn without replacement. If all the three oranges are good, the box is approved for sale, otherwise, it is rejected. Find the probability that a box containing 15 oranges out of which 12 are good and 3 are bad ones will be approved for sale.
		\label{ncert/12/13/2/3/defs.tex}
		\item Two balls are drawn at random with replacement from a box containing 10 black and 8 red balls. Find the probability that
		\label{ncert/12/13/2/12}
\begin{enumerate}
\item both balls are red.
\item first ball is black and second is red.
\item one of them is black and other is red.
\end{enumerate}

\item In a hostel, 60\% of the students read Hindi newspaper, 40\% read English newspaper and 20\% read both Hindi and English newspapers. A student is selected at random.
		\label{ncert/12/13/2/15}
\begin{enumerate}
\item Find the probability that she reads neither Hindi nor English newspapers.
\item If she reads Hindi newspaper, find the probability that she reads English newspaper.
\item If she reads English newspaper, find the probability that she reads Hindi newspaper.\\
\end{enumerate}
\item The probability of obtaining an even prime number on each die, when a pair of dice is rolled is 
\begin{enumerate}
    \item $0$ 
    
    \item $\frac{1}{3}$ 
    
    \item $\frac{1}{12}$ 
    
    \item $\frac{1}{36}$ 
\end{enumerate}
\solution
		%\input{ncert/12/13/2/17/defs.tex}
	\item A bag contains 4 red and 4 black balls, another bag contains 2 red and 6 black balls. One of the two bags is selected at random and a ball is drawn from the bag which is found to be red. Find the probability that the ball is drawn from the first bag.
\\
\solution
		%\input{ncert/12/13/3/2/main.tex}
  \item
  Cards with numbers 2 to 101 are placed in a box. A card is selected at random.Find the probability that the card has
\begin{enumerate}[label=(\roman*)]
	\item an even number 
	\item a square number
\end{enumerate}
\solution
%\input{exemplar/10/13/3/32/main.tex}
\item
The king, queen and jack of clubs are removed from a deck of 52 playing cards and then well shuffled. Now one card is drawn at random from the remaining cards.  Determine the probability that the card is
\begin{enumerate}[label=(\roman*)]
\item a club
\item 10 of hearts
\end{enumerate}
\solution
%\input{exemplar/10/13/3/29/main.tex}
\item A team of medical students doing their internship have to assist during surgeries
at a city hospital. The probabilities of surgeries rated as very complex, complex,
routine, simple or very simple are respectively, 0.15, 0.20, 0.31, 0.26, .08. Find
the probabilities that a particular surgery will be rated
\begin{enumerate}
	\item complex or very complex;
	\item neither very complex nor very simple;
	\item routine or complex
	\item routine or simple
\end{enumerate}
\solution
%\input{exemplar/11/16/3/8(1)/main.tex}
\item A card is selected from a pack of 52 cards.
\begin{enumerate}[label=(\alph*)]
    \item How many points are there in the sample space?
    \item Calculate the probability that the card is an ace of spades.
    \item Calculate the probability that the card is (i) an ace and (ii) black card.
\end{enumerate}
\solution
%\input{exemplar/11/16/3/4/main2.tex}
\item The probability that a non leap year selected at random will contain 53 sundays.
\\
\solution
%\input{exemplar/10/13/1/19/main.tex}
\item One of the four persons John, Rita, Aslam or Gurpreet will be promoted next
month. Consequently the sample space consists of four elementary outcomes
S = {John promoted, Rita promoted, Aslam promoted, Gurpreet promoted}
You are told that the chances of John’s promotion is same as that of Gurpreet,
Rita’s chances of promotion are twice as likely as Johns. Aslam’s chances are
four times that of John.
\begin{enumerate}
	\item Determine
	\begin{enumerate}
		\item P (John promoted)
		\item P (Rita promoted)
		\item P (Aslam promoted)
		\item P (Gurpreet promoted)
	\end{enumerate}
	\item If A = {John promoted or Gurpreet promoted}, find P (A).
\end{enumerate}
\solution
%\input{exemplar/11/16/3/10/main.tex}
\item A card is drawn from a deck of 52 cards. Find the probability of getting a king or a heart or a red card.\\
\solution
%\input{exemplar/11/16/3/15/main.tex}
\item The probability that a student will pass his examination is 0.73, the probability of
the student getting a compartment is 0.13, and the probability that the student will
either pass or get compartment is 0.96. State True or False.\\
\solution
%\input{exemplar/11/16/3/31/main.tex}
\item A card is selected from a pack of 52 cards\\
\begin{enumerate}[label=(\alph*)]
\item How many points are there in the sample space?
\item Calculate the probability that the cards is an ace of spades.
\item Calculate the probability that the card is (i) an ace (ii)black card.\\
\end{enumerate}
%\input{ncert/11/16/3/4_1/Prob_4.tex}
\item In a non-leap year, the probability of having 53 tuesdays or 53 wednesdays is\\
\solution
%\input{exemplar/11/16/3/18/main.tex}
\item There are 1000 sealed envelopes in a box, 10 of them contain a cash prize of
Rs 100 each, 100 of them contain a cash prize of Rs 50 each and 200 of them
contain a cash prize of Rs 10 each and rest do not contain any cash prize. If they
are well shuffled and an envelope is picked up out, what is the probability that it
contains no cash prize?\\
\solution
%\input{exemplar/10/13/3/34/main.tex}
\item 
A die is thrown and a card is selected at random from a deck of 52 playing cards. The probability of getting an even number on the die and a spade card.\\
\solution
%\input{exemplar/12/13/3/78/main.tex}
\item
If 4-digit numbers greater than 5,000 are randomly formed from the digits 0, 1, 3, 5, and 7, what is the probability of forming a number divisible by 5 when:
\begin{enumerate}
    \item The digits are repeated?
    \item The repetition of digits is not allowed?
\end{enumerate}
\solution
%\input{ncert/11/16/4/9/main.tex}
\item Consider the probability space $\brak{\Omega, \mathcal{G}, P}$ where $\Omega = [0,2]$ and $\mathcal{G} = \cbrak{\phi, \Omega, [0,1], (1,2]}$. Let $X$ and $Y$ be two functions on $\Omega$ defined as
\begin{align*}
    X(\omega) = 
    \begin{cases}
        1 & \text{if }\omega \in [0, 1]\\
        2 & \text{if }\omega \in (1, 2]
    \end{cases}
\end{align*}
and
\begin{align*}
    Y(\omega) = 
    \begin{cases}
        2 & \text{if }\omega \in [0, 1.5]\\
        3 & \text{if }\omega \in (1.5, 2].
    \end{cases}
\end{align*}
Then which one of the following statements is true?
\begin{enumerate}
    \item [(A)] $X$ is a random variable with respect to $\mathcal{G}$, but $Y$ is not a random variable with respect to $\mathcal{G}$.
    \item [(B)] $Y$ is a random variable with respect to $\mathcal{G}$, but $X$ is not a random variable with respect to $\mathcal{G}$.
    \item [(C)] Neither $X$ nor $Y$ is a random variable with respect to $\mathcal{G}$.
    \item [(D)] Both $X$ and $Y$ are random variables with respect to $\mathcal{G}$.
\end{enumerate} \hfill (GATE ST 2023)\\
\solution
%\input{gate/ST/2023/14/main.tex}
	\item  A die is loaded in such a way that each odd number is twice as likely to occur as
each even number. Find $P(G)$, where $G$ is the event that a number greater than
3 occurs on a single roll of the die.
\\
\solution
		%\input{exemplar/11/16/3/5/main.tex}
	\item All the jacks, queens and kings are removed from a deck of 52 playing cards. The remaining cards are well shuffled and then one card is drawn at random. Giving ace a value 1 similar value for other cards, find the probability that the card has a value 
		\begin{enumerate}
			\item 7
			\item greater than 7
			\item less than 7
		\end{enumerate}
		%\input{exemplar/10/13/3/30/main.tex}
  \item A Lot consists of 48 mobile phones of which 42 are good, 3 have only minor defects and 3 have major defects.Varnika will buy a phone if it is good but the trader will only buy a mobile if it has no major defects. One phone is selected at random from the lot. What is the probability that it is
\begin{enumerate}
	\item acceptable to Varnika?
            \item acceptable to the trader?
\end{enumerate}
\solution
	%\input{exemplar/10/13/3/40/main.tex}
 \item A student says that if you throw a die, it will show up 1 or not 1. Therefore, the probability of getting 1 and the probability of getting 'not 1' each is equal to $\frac{1}{2}$. Is this correct? Give reasons.\\
 \solution
        %\input{exemplar/10/13/2/9/main.tex}
   \item Four candidates A, B, C, D have ap-
plied for the assignment to coach a school cricket
team. If A is twice as likely to be selected as B, and
B and C are given about the same chance of being
selected, while C is twice as likely to be selected
as D, what are the probabilities that
\begin{enumerate}
\item C will be selected?
\item A will not be selected?
\end{enumerate}
	%\input{exemplar/11/16/3/9/main.tex}
 \item A bag contain 24 balls of which $x$ balls are red, $2x$ are white and $3x$ are blue. A ball is selected at random, What is the probability that it is
\begin{enumerate}[label=\alph*)]
\item not red ?
\item white ?
\end{enumerate}
%\input{exemplar/10/13/3/41/main.tex}
If the letters of the word ASSASSINATION are arranged at random. Find the Probability that
\begin{enumerate}[label=(\alph*)]
\item Four $S's$ come consecutively in the word
\item Two  $I's$ and two $N's$ come together
\item All $A's$ are not coming together
\item No two $A's$ are coming together
\end{enumerate}
%\input{exemplar/11/16/3/14/main.tex}
	\item One urn contains two black balls (labelled B1 and B2) and one white ball. A
	second urn contains one black ball and two white balls (labelled W1 and W2).
	Suppose the following experiment is performed. One of the two urns is chosen
	at random. Next a ball is randomly chosen from the urn. Then a second ball is
	chosen at random from the same urn without replacing the first ball.
	
	\begin{enumerate}
	\item What is the probability that two black balls are chosen?
	
	\item What is the probability that two balls of opposite colour are chosen?
	\end{enumerate}
	\solution
	%\input{exemplar/11/16/3/12/main1.tex}
\end{enumerate}

\item In a certain lottery 10,000 tickets are sold and ten equal prizes are awarded. What is the probability of not getting a prize if you buy (a) one ticket (b) two tickets (c) 10 tickets ?	
\\
\solution
		%\begin{enumerate}[label=\thesection.\arabic*,ref=\thesection.\theenumi]
	\item One card is drawn from a well-shuffled deck of 52 cards. Find the probability of getting
\begin{enumerate}
\item A king of red colour 
\item A face card 
\item A red face card
\item The jack of hearts
\item A spade
\item The queen of diamonds

\end{enumerate}
\solution
		%\input{ncert/10/15/1/14/main.tex}
	\item Five cards—the ten, jack, queen, king and ace of diamonds, are well-shuffled with their face downwards. One card is then picked up at random.
\begin{enumerate}
\item
What is the probability that the card is the queen? 
\item
If the queen is drawn and put aside, what is the probability that the second card picked up is (a) an ace? (b) a queen?\\
\end{enumerate}
\solution
		%\input{ncert/10/15/1/15/defs.tex}
	\item A bag contains $5$ red balls and some blue balls. If the probability of drawing a blue ball is double that if a red ball, determine the number of blue balls in the bag. 
		\\
\solution
		%\input{ncert/10/15/2/3/defs.tex}
	\item A card is selected from a pack of 52 cards.
 \begin{enumerate}[label=(\alph*)] 
                 \item How many points are there in the sample space?
                 \item Calculate the probability that the card is an ace of spades.
                 \item Calculate the probability that the card is (i) an ace and (ii) black card.
 \end{enumerate}
\solution
		%\input{ncert/11/16/3/4/main.tex}
\item Four cards are drawn from a well-shuffled deck of 52 cards. What is the probability of obtaining 3 diamonds and one spade.
\\
\solution
		%\input{ncert/11/16/4/2/defs.tex}
\item In a certain lottery 10,000 tickets are sold and ten equal prizes are awarded. What is the probability of not getting a prize if you buy (a) one ticket (b) two tickets (c) 10 tickets ?	
\\
\solution
		%\input{ncert/11/16/4/4/defs.tex}
		%
\item 
Out of 100 students, two sections of 40 and 60 are formed. If you and your friend are among the 100 students, what is the probability that
\begin{enumerate}
\item you both enter the same section?
\item you both enter the different sections?
\end{enumerate}
\solution
		%\input{ncert/11/16/4/5/defs.tex}
	\item 
The number lock of a suitcase has 4 wheels each labelled with ten digits i.e. from 0 to 9.The lock opens with a sequence of four digits with no repeats.What is the probability of a person getting the right sequence to open the suitcase.
\\
\solution
		%\input{ncert/11/16/4/10/defs.tex}
		%
\item 
Two cards are drawn at random and without replacement from a pack of 52 playing cards. Find the probability that both the cards are black.
\\
\solution
		%\input{ncert/12/13/2/2/defs.tex}
		\item A box of oranges is inspected by examining three randomly selected oranges drawn without replacement. If all the three oranges are good, the box is approved for sale, otherwise, it is rejected. Find the probability that a box containing 15 oranges out of which 12 are good and 3 are bad ones will be approved for sale.
		\label{ncert/12/13/2/3/defs.tex}
		\item Two balls are drawn at random with replacement from a box containing 10 black and 8 red balls. Find the probability that
		\label{ncert/12/13/2/12}
\begin{enumerate}
\item both balls are red.
\item first ball is black and second is red.
\item one of them is black and other is red.
\end{enumerate}

\item In a hostel, 60\% of the students read Hindi newspaper, 40\% read English newspaper and 20\% read both Hindi and English newspapers. A student is selected at random.
		\label{ncert/12/13/2/15}
\begin{enumerate}
\item Find the probability that she reads neither Hindi nor English newspapers.
\item If she reads Hindi newspaper, find the probability that she reads English newspaper.
\item If she reads English newspaper, find the probability that she reads Hindi newspaper.\\
\end{enumerate}
\item The probability of obtaining an even prime number on each die, when a pair of dice is rolled is 
\begin{enumerate}
    \item $0$ 
    
    \item $\frac{1}{3}$ 
    
    \item $\frac{1}{12}$ 
    
    \item $\frac{1}{36}$ 
\end{enumerate}
\solution
		%\input{ncert/12/13/2/17/defs.tex}
	\item A bag contains 4 red and 4 black balls, another bag contains 2 red and 6 black balls. One of the two bags is selected at random and a ball is drawn from the bag which is found to be red. Find the probability that the ball is drawn from the first bag.
\\
\solution
		%\input{ncert/12/13/3/2/main.tex}
  \item
  Cards with numbers 2 to 101 are placed in a box. A card is selected at random.Find the probability that the card has
\begin{enumerate}[label=(\roman*)]
	\item an even number 
	\item a square number
\end{enumerate}
\solution
%\input{exemplar/10/13/3/32/main.tex}
\item
The king, queen and jack of clubs are removed from a deck of 52 playing cards and then well shuffled. Now one card is drawn at random from the remaining cards.  Determine the probability that the card is
\begin{enumerate}[label=(\roman*)]
\item a club
\item 10 of hearts
\end{enumerate}
\solution
%\input{exemplar/10/13/3/29/main.tex}
\item A team of medical students doing their internship have to assist during surgeries
at a city hospital. The probabilities of surgeries rated as very complex, complex,
routine, simple or very simple are respectively, 0.15, 0.20, 0.31, 0.26, .08. Find
the probabilities that a particular surgery will be rated
\begin{enumerate}
	\item complex or very complex;
	\item neither very complex nor very simple;
	\item routine or complex
	\item routine or simple
\end{enumerate}
\solution
%\input{exemplar/11/16/3/8(1)/main.tex}
\item A card is selected from a pack of 52 cards.
\begin{enumerate}[label=(\alph*)]
    \item How many points are there in the sample space?
    \item Calculate the probability that the card is an ace of spades.
    \item Calculate the probability that the card is (i) an ace and (ii) black card.
\end{enumerate}
\solution
%\input{exemplar/11/16/3/4/main2.tex}
\item The probability that a non leap year selected at random will contain 53 sundays.
\\
\solution
%\input{exemplar/10/13/1/19/main.tex}
\item One of the four persons John, Rita, Aslam or Gurpreet will be promoted next
month. Consequently the sample space consists of four elementary outcomes
S = {John promoted, Rita promoted, Aslam promoted, Gurpreet promoted}
You are told that the chances of John’s promotion is same as that of Gurpreet,
Rita’s chances of promotion are twice as likely as Johns. Aslam’s chances are
four times that of John.
\begin{enumerate}
	\item Determine
	\begin{enumerate}
		\item P (John promoted)
		\item P (Rita promoted)
		\item P (Aslam promoted)
		\item P (Gurpreet promoted)
	\end{enumerate}
	\item If A = {John promoted or Gurpreet promoted}, find P (A).
\end{enumerate}
\solution
%\input{exemplar/11/16/3/10/main.tex}
\item A card is drawn from a deck of 52 cards. Find the probability of getting a king or a heart or a red card.\\
\solution
%\input{exemplar/11/16/3/15/main.tex}
\item The probability that a student will pass his examination is 0.73, the probability of
the student getting a compartment is 0.13, and the probability that the student will
either pass or get compartment is 0.96. State True or False.\\
\solution
%\input{exemplar/11/16/3/31/main.tex}
\item A card is selected from a pack of 52 cards\\
\begin{enumerate}[label=(\alph*)]
\item How many points are there in the sample space?
\item Calculate the probability that the cards is an ace of spades.
\item Calculate the probability that the card is (i) an ace (ii)black card.\\
\end{enumerate}
%\input{ncert/11/16/3/4_1/Prob_4.tex}
\item In a non-leap year, the probability of having 53 tuesdays or 53 wednesdays is\\
\solution
%\input{exemplar/11/16/3/18/main.tex}
\item There are 1000 sealed envelopes in a box, 10 of them contain a cash prize of
Rs 100 each, 100 of them contain a cash prize of Rs 50 each and 200 of them
contain a cash prize of Rs 10 each and rest do not contain any cash prize. If they
are well shuffled and an envelope is picked up out, what is the probability that it
contains no cash prize?\\
\solution
%\input{exemplar/10/13/3/34/main.tex}
\item 
A die is thrown and a card is selected at random from a deck of 52 playing cards. The probability of getting an even number on the die and a spade card.\\
\solution
%\input{exemplar/12/13/3/78/main.tex}
\item
If 4-digit numbers greater than 5,000 are randomly formed from the digits 0, 1, 3, 5, and 7, what is the probability of forming a number divisible by 5 when:
\begin{enumerate}
    \item The digits are repeated?
    \item The repetition of digits is not allowed?
\end{enumerate}
\solution
%\input{ncert/11/16/4/9/main.tex}
\item Consider the probability space $\brak{\Omega, \mathcal{G}, P}$ where $\Omega = [0,2]$ and $\mathcal{G} = \cbrak{\phi, \Omega, [0,1], (1,2]}$. Let $X$ and $Y$ be two functions on $\Omega$ defined as
\begin{align*}
    X(\omega) = 
    \begin{cases}
        1 & \text{if }\omega \in [0, 1]\\
        2 & \text{if }\omega \in (1, 2]
    \end{cases}
\end{align*}
and
\begin{align*}
    Y(\omega) = 
    \begin{cases}
        2 & \text{if }\omega \in [0, 1.5]\\
        3 & \text{if }\omega \in (1.5, 2].
    \end{cases}
\end{align*}
Then which one of the following statements is true?
\begin{enumerate}
    \item [(A)] $X$ is a random variable with respect to $\mathcal{G}$, but $Y$ is not a random variable with respect to $\mathcal{G}$.
    \item [(B)] $Y$ is a random variable with respect to $\mathcal{G}$, but $X$ is not a random variable with respect to $\mathcal{G}$.
    \item [(C)] Neither $X$ nor $Y$ is a random variable with respect to $\mathcal{G}$.
    \item [(D)] Both $X$ and $Y$ are random variables with respect to $\mathcal{G}$.
\end{enumerate} \hfill (GATE ST 2023)\\
\solution
%\input{gate/ST/2023/14/main.tex}
	\item  A die is loaded in such a way that each odd number is twice as likely to occur as
each even number. Find $P(G)$, where $G$ is the event that a number greater than
3 occurs on a single roll of the die.
\\
\solution
		%\input{exemplar/11/16/3/5/main.tex}
	\item All the jacks, queens and kings are removed from a deck of 52 playing cards. The remaining cards are well shuffled and then one card is drawn at random. Giving ace a value 1 similar value for other cards, find the probability that the card has a value 
		\begin{enumerate}
			\item 7
			\item greater than 7
			\item less than 7
		\end{enumerate}
		%\input{exemplar/10/13/3/30/main.tex}
  \item A Lot consists of 48 mobile phones of which 42 are good, 3 have only minor defects and 3 have major defects.Varnika will buy a phone if it is good but the trader will only buy a mobile if it has no major defects. One phone is selected at random from the lot. What is the probability that it is
\begin{enumerate}
	\item acceptable to Varnika?
            \item acceptable to the trader?
\end{enumerate}
\solution
	%\input{exemplar/10/13/3/40/main.tex}
 \item A student says that if you throw a die, it will show up 1 or not 1. Therefore, the probability of getting 1 and the probability of getting 'not 1' each is equal to $\frac{1}{2}$. Is this correct? Give reasons.\\
 \solution
        %\input{exemplar/10/13/2/9/main.tex}
   \item Four candidates A, B, C, D have ap-
plied for the assignment to coach a school cricket
team. If A is twice as likely to be selected as B, and
B and C are given about the same chance of being
selected, while C is twice as likely to be selected
as D, what are the probabilities that
\begin{enumerate}
\item C will be selected?
\item A will not be selected?
\end{enumerate}
	%\input{exemplar/11/16/3/9/main.tex}
 \item A bag contain 24 balls of which $x$ balls are red, $2x$ are white and $3x$ are blue. A ball is selected at random, What is the probability that it is
\begin{enumerate}[label=\alph*)]
\item not red ?
\item white ?
\end{enumerate}
%\input{exemplar/10/13/3/41/main.tex}
If the letters of the word ASSASSINATION are arranged at random. Find the Probability that
\begin{enumerate}[label=(\alph*)]
\item Four $S's$ come consecutively in the word
\item Two  $I's$ and two $N's$ come together
\item All $A's$ are not coming together
\item No two $A's$ are coming together
\end{enumerate}
%\input{exemplar/11/16/3/14/main.tex}
	\item One urn contains two black balls (labelled B1 and B2) and one white ball. A
	second urn contains one black ball and two white balls (labelled W1 and W2).
	Suppose the following experiment is performed. One of the two urns is chosen
	at random. Next a ball is randomly chosen from the urn. Then a second ball is
	chosen at random from the same urn without replacing the first ball.
	
	\begin{enumerate}
	\item What is the probability that two black balls are chosen?
	
	\item What is the probability that two balls of opposite colour are chosen?
	\end{enumerate}
	\solution
	%\input{exemplar/11/16/3/12/main1.tex}
\end{enumerate}

		%
\item 
Out of 100 students, two sections of 40 and 60 are formed. If you and your friend are among the 100 students, what is the probability that
\begin{enumerate}
\item you both enter the same section?
\item you both enter the different sections?
\end{enumerate}
\solution
		%\begin{enumerate}[label=\thesection.\arabic*,ref=\thesection.\theenumi]
	\item One card is drawn from a well-shuffled deck of 52 cards. Find the probability of getting
\begin{enumerate}
\item A king of red colour 
\item A face card 
\item A red face card
\item The jack of hearts
\item A spade
\item The queen of diamonds

\end{enumerate}
\solution
		%\input{ncert/10/15/1/14/main.tex}
	\item Five cards—the ten, jack, queen, king and ace of diamonds, are well-shuffled with their face downwards. One card is then picked up at random.
\begin{enumerate}
\item
What is the probability that the card is the queen? 
\item
If the queen is drawn and put aside, what is the probability that the second card picked up is (a) an ace? (b) a queen?\\
\end{enumerate}
\solution
		%\input{ncert/10/15/1/15/defs.tex}
	\item A bag contains $5$ red balls and some blue balls. If the probability of drawing a blue ball is double that if a red ball, determine the number of blue balls in the bag. 
		\\
\solution
		%\input{ncert/10/15/2/3/defs.tex}
	\item A card is selected from a pack of 52 cards.
 \begin{enumerate}[label=(\alph*)] 
                 \item How many points are there in the sample space?
                 \item Calculate the probability that the card is an ace of spades.
                 \item Calculate the probability that the card is (i) an ace and (ii) black card.
 \end{enumerate}
\solution
		%\input{ncert/11/16/3/4/main.tex}
\item Four cards are drawn from a well-shuffled deck of 52 cards. What is the probability of obtaining 3 diamonds and one spade.
\\
\solution
		%\input{ncert/11/16/4/2/defs.tex}
\item In a certain lottery 10,000 tickets are sold and ten equal prizes are awarded. What is the probability of not getting a prize if you buy (a) one ticket (b) two tickets (c) 10 tickets ?	
\\
\solution
		%\input{ncert/11/16/4/4/defs.tex}
		%
\item 
Out of 100 students, two sections of 40 and 60 are formed. If you and your friend are among the 100 students, what is the probability that
\begin{enumerate}
\item you both enter the same section?
\item you both enter the different sections?
\end{enumerate}
\solution
		%\input{ncert/11/16/4/5/defs.tex}
	\item 
The number lock of a suitcase has 4 wheels each labelled with ten digits i.e. from 0 to 9.The lock opens with a sequence of four digits with no repeats.What is the probability of a person getting the right sequence to open the suitcase.
\\
\solution
		%\input{ncert/11/16/4/10/defs.tex}
		%
\item 
Two cards are drawn at random and without replacement from a pack of 52 playing cards. Find the probability that both the cards are black.
\\
\solution
		%\input{ncert/12/13/2/2/defs.tex}
		\item A box of oranges is inspected by examining three randomly selected oranges drawn without replacement. If all the three oranges are good, the box is approved for sale, otherwise, it is rejected. Find the probability that a box containing 15 oranges out of which 12 are good and 3 are bad ones will be approved for sale.
		\label{ncert/12/13/2/3/defs.tex}
		\item Two balls are drawn at random with replacement from a box containing 10 black and 8 red balls. Find the probability that
		\label{ncert/12/13/2/12}
\begin{enumerate}
\item both balls are red.
\item first ball is black and second is red.
\item one of them is black and other is red.
\end{enumerate}

\item In a hostel, 60\% of the students read Hindi newspaper, 40\% read English newspaper and 20\% read both Hindi and English newspapers. A student is selected at random.
		\label{ncert/12/13/2/15}
\begin{enumerate}
\item Find the probability that she reads neither Hindi nor English newspapers.
\item If she reads Hindi newspaper, find the probability that she reads English newspaper.
\item If she reads English newspaper, find the probability that she reads Hindi newspaper.\\
\end{enumerate}
\item The probability of obtaining an even prime number on each die, when a pair of dice is rolled is 
\begin{enumerate}
    \item $0$ 
    
    \item $\frac{1}{3}$ 
    
    \item $\frac{1}{12}$ 
    
    \item $\frac{1}{36}$ 
\end{enumerate}
\solution
		%\input{ncert/12/13/2/17/defs.tex}
	\item A bag contains 4 red and 4 black balls, another bag contains 2 red and 6 black balls. One of the two bags is selected at random and a ball is drawn from the bag which is found to be red. Find the probability that the ball is drawn from the first bag.
\\
\solution
		%\input{ncert/12/13/3/2/main.tex}
  \item
  Cards with numbers 2 to 101 are placed in a box. A card is selected at random.Find the probability that the card has
\begin{enumerate}[label=(\roman*)]
	\item an even number 
	\item a square number
\end{enumerate}
\solution
%\input{exemplar/10/13/3/32/main.tex}
\item
The king, queen and jack of clubs are removed from a deck of 52 playing cards and then well shuffled. Now one card is drawn at random from the remaining cards.  Determine the probability that the card is
\begin{enumerate}[label=(\roman*)]
\item a club
\item 10 of hearts
\end{enumerate}
\solution
%\input{exemplar/10/13/3/29/main.tex}
\item A team of medical students doing their internship have to assist during surgeries
at a city hospital. The probabilities of surgeries rated as very complex, complex,
routine, simple or very simple are respectively, 0.15, 0.20, 0.31, 0.26, .08. Find
the probabilities that a particular surgery will be rated
\begin{enumerate}
	\item complex or very complex;
	\item neither very complex nor very simple;
	\item routine or complex
	\item routine or simple
\end{enumerate}
\solution
%\input{exemplar/11/16/3/8(1)/main.tex}
\item A card is selected from a pack of 52 cards.
\begin{enumerate}[label=(\alph*)]
    \item How many points are there in the sample space?
    \item Calculate the probability that the card is an ace of spades.
    \item Calculate the probability that the card is (i) an ace and (ii) black card.
\end{enumerate}
\solution
%\input{exemplar/11/16/3/4/main2.tex}
\item The probability that a non leap year selected at random will contain 53 sundays.
\\
\solution
%\input{exemplar/10/13/1/19/main.tex}
\item One of the four persons John, Rita, Aslam or Gurpreet will be promoted next
month. Consequently the sample space consists of four elementary outcomes
S = {John promoted, Rita promoted, Aslam promoted, Gurpreet promoted}
You are told that the chances of John’s promotion is same as that of Gurpreet,
Rita’s chances of promotion are twice as likely as Johns. Aslam’s chances are
four times that of John.
\begin{enumerate}
	\item Determine
	\begin{enumerate}
		\item P (John promoted)
		\item P (Rita promoted)
		\item P (Aslam promoted)
		\item P (Gurpreet promoted)
	\end{enumerate}
	\item If A = {John promoted or Gurpreet promoted}, find P (A).
\end{enumerate}
\solution
%\input{exemplar/11/16/3/10/main.tex}
\item A card is drawn from a deck of 52 cards. Find the probability of getting a king or a heart or a red card.\\
\solution
%\input{exemplar/11/16/3/15/main.tex}
\item The probability that a student will pass his examination is 0.73, the probability of
the student getting a compartment is 0.13, and the probability that the student will
either pass or get compartment is 0.96. State True or False.\\
\solution
%\input{exemplar/11/16/3/31/main.tex}
\item A card is selected from a pack of 52 cards\\
\begin{enumerate}[label=(\alph*)]
\item How many points are there in the sample space?
\item Calculate the probability that the cards is an ace of spades.
\item Calculate the probability that the card is (i) an ace (ii)black card.\\
\end{enumerate}
%\input{ncert/11/16/3/4_1/Prob_4.tex}
\item In a non-leap year, the probability of having 53 tuesdays or 53 wednesdays is\\
\solution
%\input{exemplar/11/16/3/18/main.tex}
\item There are 1000 sealed envelopes in a box, 10 of them contain a cash prize of
Rs 100 each, 100 of them contain a cash prize of Rs 50 each and 200 of them
contain a cash prize of Rs 10 each and rest do not contain any cash prize. If they
are well shuffled and an envelope is picked up out, what is the probability that it
contains no cash prize?\\
\solution
%\input{exemplar/10/13/3/34/main.tex}
\item 
A die is thrown and a card is selected at random from a deck of 52 playing cards. The probability of getting an even number on the die and a spade card.\\
\solution
%\input{exemplar/12/13/3/78/main.tex}
\item
If 4-digit numbers greater than 5,000 are randomly formed from the digits 0, 1, 3, 5, and 7, what is the probability of forming a number divisible by 5 when:
\begin{enumerate}
    \item The digits are repeated?
    \item The repetition of digits is not allowed?
\end{enumerate}
\solution
%\input{ncert/11/16/4/9/main.tex}
\item Consider the probability space $\brak{\Omega, \mathcal{G}, P}$ where $\Omega = [0,2]$ and $\mathcal{G} = \cbrak{\phi, \Omega, [0,1], (1,2]}$. Let $X$ and $Y$ be two functions on $\Omega$ defined as
\begin{align*}
    X(\omega) = 
    \begin{cases}
        1 & \text{if }\omega \in [0, 1]\\
        2 & \text{if }\omega \in (1, 2]
    \end{cases}
\end{align*}
and
\begin{align*}
    Y(\omega) = 
    \begin{cases}
        2 & \text{if }\omega \in [0, 1.5]\\
        3 & \text{if }\omega \in (1.5, 2].
    \end{cases}
\end{align*}
Then which one of the following statements is true?
\begin{enumerate}
    \item [(A)] $X$ is a random variable with respect to $\mathcal{G}$, but $Y$ is not a random variable with respect to $\mathcal{G}$.
    \item [(B)] $Y$ is a random variable with respect to $\mathcal{G}$, but $X$ is not a random variable with respect to $\mathcal{G}$.
    \item [(C)] Neither $X$ nor $Y$ is a random variable with respect to $\mathcal{G}$.
    \item [(D)] Both $X$ and $Y$ are random variables with respect to $\mathcal{G}$.
\end{enumerate} \hfill (GATE ST 2023)\\
\solution
%\input{gate/ST/2023/14/main.tex}
	\item  A die is loaded in such a way that each odd number is twice as likely to occur as
each even number. Find $P(G)$, where $G$ is the event that a number greater than
3 occurs on a single roll of the die.
\\
\solution
		%\input{exemplar/11/16/3/5/main.tex}
	\item All the jacks, queens and kings are removed from a deck of 52 playing cards. The remaining cards are well shuffled and then one card is drawn at random. Giving ace a value 1 similar value for other cards, find the probability that the card has a value 
		\begin{enumerate}
			\item 7
			\item greater than 7
			\item less than 7
		\end{enumerate}
		%\input{exemplar/10/13/3/30/main.tex}
  \item A Lot consists of 48 mobile phones of which 42 are good, 3 have only minor defects and 3 have major defects.Varnika will buy a phone if it is good but the trader will only buy a mobile if it has no major defects. One phone is selected at random from the lot. What is the probability that it is
\begin{enumerate}
	\item acceptable to Varnika?
            \item acceptable to the trader?
\end{enumerate}
\solution
	%\input{exemplar/10/13/3/40/main.tex}
 \item A student says that if you throw a die, it will show up 1 or not 1. Therefore, the probability of getting 1 and the probability of getting 'not 1' each is equal to $\frac{1}{2}$. Is this correct? Give reasons.\\
 \solution
        %\input{exemplar/10/13/2/9/main.tex}
   \item Four candidates A, B, C, D have ap-
plied for the assignment to coach a school cricket
team. If A is twice as likely to be selected as B, and
B and C are given about the same chance of being
selected, while C is twice as likely to be selected
as D, what are the probabilities that
\begin{enumerate}
\item C will be selected?
\item A will not be selected?
\end{enumerate}
	%\input{exemplar/11/16/3/9/main.tex}
 \item A bag contain 24 balls of which $x$ balls are red, $2x$ are white and $3x$ are blue. A ball is selected at random, What is the probability that it is
\begin{enumerate}[label=\alph*)]
\item not red ?
\item white ?
\end{enumerate}
%\input{exemplar/10/13/3/41/main.tex}
If the letters of the word ASSASSINATION are arranged at random. Find the Probability that
\begin{enumerate}[label=(\alph*)]
\item Four $S's$ come consecutively in the word
\item Two  $I's$ and two $N's$ come together
\item All $A's$ are not coming together
\item No two $A's$ are coming together
\end{enumerate}
%\input{exemplar/11/16/3/14/main.tex}
	\item One urn contains two black balls (labelled B1 and B2) and one white ball. A
	second urn contains one black ball and two white balls (labelled W1 and W2).
	Suppose the following experiment is performed. One of the two urns is chosen
	at random. Next a ball is randomly chosen from the urn. Then a second ball is
	chosen at random from the same urn without replacing the first ball.
	
	\begin{enumerate}
	\item What is the probability that two black balls are chosen?
	
	\item What is the probability that two balls of opposite colour are chosen?
	\end{enumerate}
	\solution
	%\input{exemplar/11/16/3/12/main1.tex}
\end{enumerate}

	\item 
The number lock of a suitcase has 4 wheels each labelled with ten digits i.e. from 0 to 9.The lock opens with a sequence of four digits with no repeats.What is the probability of a person getting the right sequence to open the suitcase.
\\
\solution
		%\begin{enumerate}[label=\thesection.\arabic*,ref=\thesection.\theenumi]
	\item One card is drawn from a well-shuffled deck of 52 cards. Find the probability of getting
\begin{enumerate}
\item A king of red colour 
\item A face card 
\item A red face card
\item The jack of hearts
\item A spade
\item The queen of diamonds

\end{enumerate}
\solution
		%\input{ncert/10/15/1/14/main.tex}
	\item Five cards—the ten, jack, queen, king and ace of diamonds, are well-shuffled with their face downwards. One card is then picked up at random.
\begin{enumerate}
\item
What is the probability that the card is the queen? 
\item
If the queen is drawn and put aside, what is the probability that the second card picked up is (a) an ace? (b) a queen?\\
\end{enumerate}
\solution
		%\input{ncert/10/15/1/15/defs.tex}
	\item A bag contains $5$ red balls and some blue balls. If the probability of drawing a blue ball is double that if a red ball, determine the number of blue balls in the bag. 
		\\
\solution
		%\input{ncert/10/15/2/3/defs.tex}
	\item A card is selected from a pack of 52 cards.
 \begin{enumerate}[label=(\alph*)] 
                 \item How many points are there in the sample space?
                 \item Calculate the probability that the card is an ace of spades.
                 \item Calculate the probability that the card is (i) an ace and (ii) black card.
 \end{enumerate}
\solution
		%\input{ncert/11/16/3/4/main.tex}
\item Four cards are drawn from a well-shuffled deck of 52 cards. What is the probability of obtaining 3 diamonds and one spade.
\\
\solution
		%\input{ncert/11/16/4/2/defs.tex}
\item In a certain lottery 10,000 tickets are sold and ten equal prizes are awarded. What is the probability of not getting a prize if you buy (a) one ticket (b) two tickets (c) 10 tickets ?	
\\
\solution
		%\input{ncert/11/16/4/4/defs.tex}
		%
\item 
Out of 100 students, two sections of 40 and 60 are formed. If you and your friend are among the 100 students, what is the probability that
\begin{enumerate}
\item you both enter the same section?
\item you both enter the different sections?
\end{enumerate}
\solution
		%\input{ncert/11/16/4/5/defs.tex}
	\item 
The number lock of a suitcase has 4 wheels each labelled with ten digits i.e. from 0 to 9.The lock opens with a sequence of four digits with no repeats.What is the probability of a person getting the right sequence to open the suitcase.
\\
\solution
		%\input{ncert/11/16/4/10/defs.tex}
		%
\item 
Two cards are drawn at random and without replacement from a pack of 52 playing cards. Find the probability that both the cards are black.
\\
\solution
		%\input{ncert/12/13/2/2/defs.tex}
		\item A box of oranges is inspected by examining three randomly selected oranges drawn without replacement. If all the three oranges are good, the box is approved for sale, otherwise, it is rejected. Find the probability that a box containing 15 oranges out of which 12 are good and 3 are bad ones will be approved for sale.
		\label{ncert/12/13/2/3/defs.tex}
		\item Two balls are drawn at random with replacement from a box containing 10 black and 8 red balls. Find the probability that
		\label{ncert/12/13/2/12}
\begin{enumerate}
\item both balls are red.
\item first ball is black and second is red.
\item one of them is black and other is red.
\end{enumerate}

\item In a hostel, 60\% of the students read Hindi newspaper, 40\% read English newspaper and 20\% read both Hindi and English newspapers. A student is selected at random.
		\label{ncert/12/13/2/15}
\begin{enumerate}
\item Find the probability that she reads neither Hindi nor English newspapers.
\item If she reads Hindi newspaper, find the probability that she reads English newspaper.
\item If she reads English newspaper, find the probability that she reads Hindi newspaper.\\
\end{enumerate}
\item The probability of obtaining an even prime number on each die, when a pair of dice is rolled is 
\begin{enumerate}
    \item $0$ 
    
    \item $\frac{1}{3}$ 
    
    \item $\frac{1}{12}$ 
    
    \item $\frac{1}{36}$ 
\end{enumerate}
\solution
		%\input{ncert/12/13/2/17/defs.tex}
	\item A bag contains 4 red and 4 black balls, another bag contains 2 red and 6 black balls. One of the two bags is selected at random and a ball is drawn from the bag which is found to be red. Find the probability that the ball is drawn from the first bag.
\\
\solution
		%\input{ncert/12/13/3/2/main.tex}
  \item
  Cards with numbers 2 to 101 are placed in a box. A card is selected at random.Find the probability that the card has
\begin{enumerate}[label=(\roman*)]
	\item an even number 
	\item a square number
\end{enumerate}
\solution
%\input{exemplar/10/13/3/32/main.tex}
\item
The king, queen and jack of clubs are removed from a deck of 52 playing cards and then well shuffled. Now one card is drawn at random from the remaining cards.  Determine the probability that the card is
\begin{enumerate}[label=(\roman*)]
\item a club
\item 10 of hearts
\end{enumerate}
\solution
%\input{exemplar/10/13/3/29/main.tex}
\item A team of medical students doing their internship have to assist during surgeries
at a city hospital. The probabilities of surgeries rated as very complex, complex,
routine, simple or very simple are respectively, 0.15, 0.20, 0.31, 0.26, .08. Find
the probabilities that a particular surgery will be rated
\begin{enumerate}
	\item complex or very complex;
	\item neither very complex nor very simple;
	\item routine or complex
	\item routine or simple
\end{enumerate}
\solution
%\input{exemplar/11/16/3/8(1)/main.tex}
\item A card is selected from a pack of 52 cards.
\begin{enumerate}[label=(\alph*)]
    \item How many points are there in the sample space?
    \item Calculate the probability that the card is an ace of spades.
    \item Calculate the probability that the card is (i) an ace and (ii) black card.
\end{enumerate}
\solution
%\input{exemplar/11/16/3/4/main2.tex}
\item The probability that a non leap year selected at random will contain 53 sundays.
\\
\solution
%\input{exemplar/10/13/1/19/main.tex}
\item One of the four persons John, Rita, Aslam or Gurpreet will be promoted next
month. Consequently the sample space consists of four elementary outcomes
S = {John promoted, Rita promoted, Aslam promoted, Gurpreet promoted}
You are told that the chances of John’s promotion is same as that of Gurpreet,
Rita’s chances of promotion are twice as likely as Johns. Aslam’s chances are
four times that of John.
\begin{enumerate}
	\item Determine
	\begin{enumerate}
		\item P (John promoted)
		\item P (Rita promoted)
		\item P (Aslam promoted)
		\item P (Gurpreet promoted)
	\end{enumerate}
	\item If A = {John promoted or Gurpreet promoted}, find P (A).
\end{enumerate}
\solution
%\input{exemplar/11/16/3/10/main.tex}
\item A card is drawn from a deck of 52 cards. Find the probability of getting a king or a heart or a red card.\\
\solution
%\input{exemplar/11/16/3/15/main.tex}
\item The probability that a student will pass his examination is 0.73, the probability of
the student getting a compartment is 0.13, and the probability that the student will
either pass or get compartment is 0.96. State True or False.\\
\solution
%\input{exemplar/11/16/3/31/main.tex}
\item A card is selected from a pack of 52 cards\\
\begin{enumerate}[label=(\alph*)]
\item How many points are there in the sample space?
\item Calculate the probability that the cards is an ace of spades.
\item Calculate the probability that the card is (i) an ace (ii)black card.\\
\end{enumerate}
%\input{ncert/11/16/3/4_1/Prob_4.tex}
\item In a non-leap year, the probability of having 53 tuesdays or 53 wednesdays is\\
\solution
%\input{exemplar/11/16/3/18/main.tex}
\item There are 1000 sealed envelopes in a box, 10 of them contain a cash prize of
Rs 100 each, 100 of them contain a cash prize of Rs 50 each and 200 of them
contain a cash prize of Rs 10 each and rest do not contain any cash prize. If they
are well shuffled and an envelope is picked up out, what is the probability that it
contains no cash prize?\\
\solution
%\input{exemplar/10/13/3/34/main.tex}
\item 
A die is thrown and a card is selected at random from a deck of 52 playing cards. The probability of getting an even number on the die and a spade card.\\
\solution
%\input{exemplar/12/13/3/78/main.tex}
\item
If 4-digit numbers greater than 5,000 are randomly formed from the digits 0, 1, 3, 5, and 7, what is the probability of forming a number divisible by 5 when:
\begin{enumerate}
    \item The digits are repeated?
    \item The repetition of digits is not allowed?
\end{enumerate}
\solution
%\input{ncert/11/16/4/9/main.tex}
\item Consider the probability space $\brak{\Omega, \mathcal{G}, P}$ where $\Omega = [0,2]$ and $\mathcal{G} = \cbrak{\phi, \Omega, [0,1], (1,2]}$. Let $X$ and $Y$ be two functions on $\Omega$ defined as
\begin{align*}
    X(\omega) = 
    \begin{cases}
        1 & \text{if }\omega \in [0, 1]\\
        2 & \text{if }\omega \in (1, 2]
    \end{cases}
\end{align*}
and
\begin{align*}
    Y(\omega) = 
    \begin{cases}
        2 & \text{if }\omega \in [0, 1.5]\\
        3 & \text{if }\omega \in (1.5, 2].
    \end{cases}
\end{align*}
Then which one of the following statements is true?
\begin{enumerate}
    \item [(A)] $X$ is a random variable with respect to $\mathcal{G}$, but $Y$ is not a random variable with respect to $\mathcal{G}$.
    \item [(B)] $Y$ is a random variable with respect to $\mathcal{G}$, but $X$ is not a random variable with respect to $\mathcal{G}$.
    \item [(C)] Neither $X$ nor $Y$ is a random variable with respect to $\mathcal{G}$.
    \item [(D)] Both $X$ and $Y$ are random variables with respect to $\mathcal{G}$.
\end{enumerate} \hfill (GATE ST 2023)\\
\solution
%\input{gate/ST/2023/14/main.tex}
	\item  A die is loaded in such a way that each odd number is twice as likely to occur as
each even number. Find $P(G)$, where $G$ is the event that a number greater than
3 occurs on a single roll of the die.
\\
\solution
		%\input{exemplar/11/16/3/5/main.tex}
	\item All the jacks, queens and kings are removed from a deck of 52 playing cards. The remaining cards are well shuffled and then one card is drawn at random. Giving ace a value 1 similar value for other cards, find the probability that the card has a value 
		\begin{enumerate}
			\item 7
			\item greater than 7
			\item less than 7
		\end{enumerate}
		%\input{exemplar/10/13/3/30/main.tex}
  \item A Lot consists of 48 mobile phones of which 42 are good, 3 have only minor defects and 3 have major defects.Varnika will buy a phone if it is good but the trader will only buy a mobile if it has no major defects. One phone is selected at random from the lot. What is the probability that it is
\begin{enumerate}
	\item acceptable to Varnika?
            \item acceptable to the trader?
\end{enumerate}
\solution
	%\input{exemplar/10/13/3/40/main.tex}
 \item A student says that if you throw a die, it will show up 1 or not 1. Therefore, the probability of getting 1 and the probability of getting 'not 1' each is equal to $\frac{1}{2}$. Is this correct? Give reasons.\\
 \solution
        %\input{exemplar/10/13/2/9/main.tex}
   \item Four candidates A, B, C, D have ap-
plied for the assignment to coach a school cricket
team. If A is twice as likely to be selected as B, and
B and C are given about the same chance of being
selected, while C is twice as likely to be selected
as D, what are the probabilities that
\begin{enumerate}
\item C will be selected?
\item A will not be selected?
\end{enumerate}
	%\input{exemplar/11/16/3/9/main.tex}
 \item A bag contain 24 balls of which $x$ balls are red, $2x$ are white and $3x$ are blue. A ball is selected at random, What is the probability that it is
\begin{enumerate}[label=\alph*)]
\item not red ?
\item white ?
\end{enumerate}
%\input{exemplar/10/13/3/41/main.tex}
If the letters of the word ASSASSINATION are arranged at random. Find the Probability that
\begin{enumerate}[label=(\alph*)]
\item Four $S's$ come consecutively in the word
\item Two  $I's$ and two $N's$ come together
\item All $A's$ are not coming together
\item No two $A's$ are coming together
\end{enumerate}
%\input{exemplar/11/16/3/14/main.tex}
	\item One urn contains two black balls (labelled B1 and B2) and one white ball. A
	second urn contains one black ball and two white balls (labelled W1 and W2).
	Suppose the following experiment is performed. One of the two urns is chosen
	at random. Next a ball is randomly chosen from the urn. Then a second ball is
	chosen at random from the same urn without replacing the first ball.
	
	\begin{enumerate}
	\item What is the probability that two black balls are chosen?
	
	\item What is the probability that two balls of opposite colour are chosen?
	\end{enumerate}
	\solution
	%\input{exemplar/11/16/3/12/main1.tex}
\end{enumerate}

		%
\item 
Two cards are drawn at random and without replacement from a pack of 52 playing cards. Find the probability that both the cards are black.
\\
\solution
		%\begin{enumerate}[label=\thesection.\arabic*,ref=\thesection.\theenumi]
	\item One card is drawn from a well-shuffled deck of 52 cards. Find the probability of getting
\begin{enumerate}
\item A king of red colour 
\item A face card 
\item A red face card
\item The jack of hearts
\item A spade
\item The queen of diamonds

\end{enumerate}
\solution
		%\input{ncert/10/15/1/14/main.tex}
	\item Five cards—the ten, jack, queen, king and ace of diamonds, are well-shuffled with their face downwards. One card is then picked up at random.
\begin{enumerate}
\item
What is the probability that the card is the queen? 
\item
If the queen is drawn and put aside, what is the probability that the second card picked up is (a) an ace? (b) a queen?\\
\end{enumerate}
\solution
		%\input{ncert/10/15/1/15/defs.tex}
	\item A bag contains $5$ red balls and some blue balls. If the probability of drawing a blue ball is double that if a red ball, determine the number of blue balls in the bag. 
		\\
\solution
		%\input{ncert/10/15/2/3/defs.tex}
	\item A card is selected from a pack of 52 cards.
 \begin{enumerate}[label=(\alph*)] 
                 \item How many points are there in the sample space?
                 \item Calculate the probability that the card is an ace of spades.
                 \item Calculate the probability that the card is (i) an ace and (ii) black card.
 \end{enumerate}
\solution
		%\input{ncert/11/16/3/4/main.tex}
\item Four cards are drawn from a well-shuffled deck of 52 cards. What is the probability of obtaining 3 diamonds and one spade.
\\
\solution
		%\input{ncert/11/16/4/2/defs.tex}
\item In a certain lottery 10,000 tickets are sold and ten equal prizes are awarded. What is the probability of not getting a prize if you buy (a) one ticket (b) two tickets (c) 10 tickets ?	
\\
\solution
		%\input{ncert/11/16/4/4/defs.tex}
		%
\item 
Out of 100 students, two sections of 40 and 60 are formed. If you and your friend are among the 100 students, what is the probability that
\begin{enumerate}
\item you both enter the same section?
\item you both enter the different sections?
\end{enumerate}
\solution
		%\input{ncert/11/16/4/5/defs.tex}
	\item 
The number lock of a suitcase has 4 wheels each labelled with ten digits i.e. from 0 to 9.The lock opens with a sequence of four digits with no repeats.What is the probability of a person getting the right sequence to open the suitcase.
\\
\solution
		%\input{ncert/11/16/4/10/defs.tex}
		%
\item 
Two cards are drawn at random and without replacement from a pack of 52 playing cards. Find the probability that both the cards are black.
\\
\solution
		%\input{ncert/12/13/2/2/defs.tex}
		\item A box of oranges is inspected by examining three randomly selected oranges drawn without replacement. If all the three oranges are good, the box is approved for sale, otherwise, it is rejected. Find the probability that a box containing 15 oranges out of which 12 are good and 3 are bad ones will be approved for sale.
		\label{ncert/12/13/2/3/defs.tex}
		\item Two balls are drawn at random with replacement from a box containing 10 black and 8 red balls. Find the probability that
		\label{ncert/12/13/2/12}
\begin{enumerate}
\item both balls are red.
\item first ball is black and second is red.
\item one of them is black and other is red.
\end{enumerate}

\item In a hostel, 60\% of the students read Hindi newspaper, 40\% read English newspaper and 20\% read both Hindi and English newspapers. A student is selected at random.
		\label{ncert/12/13/2/15}
\begin{enumerate}
\item Find the probability that she reads neither Hindi nor English newspapers.
\item If she reads Hindi newspaper, find the probability that she reads English newspaper.
\item If she reads English newspaper, find the probability that she reads Hindi newspaper.\\
\end{enumerate}
\item The probability of obtaining an even prime number on each die, when a pair of dice is rolled is 
\begin{enumerate}
    \item $0$ 
    
    \item $\frac{1}{3}$ 
    
    \item $\frac{1}{12}$ 
    
    \item $\frac{1}{36}$ 
\end{enumerate}
\solution
		%\input{ncert/12/13/2/17/defs.tex}
	\item A bag contains 4 red and 4 black balls, another bag contains 2 red and 6 black balls. One of the two bags is selected at random and a ball is drawn from the bag which is found to be red. Find the probability that the ball is drawn from the first bag.
\\
\solution
		%\input{ncert/12/13/3/2/main.tex}
  \item
  Cards with numbers 2 to 101 are placed in a box. A card is selected at random.Find the probability that the card has
\begin{enumerate}[label=(\roman*)]
	\item an even number 
	\item a square number
\end{enumerate}
\solution
%\input{exemplar/10/13/3/32/main.tex}
\item
The king, queen and jack of clubs are removed from a deck of 52 playing cards and then well shuffled. Now one card is drawn at random from the remaining cards.  Determine the probability that the card is
\begin{enumerate}[label=(\roman*)]
\item a club
\item 10 of hearts
\end{enumerate}
\solution
%\input{exemplar/10/13/3/29/main.tex}
\item A team of medical students doing their internship have to assist during surgeries
at a city hospital. The probabilities of surgeries rated as very complex, complex,
routine, simple or very simple are respectively, 0.15, 0.20, 0.31, 0.26, .08. Find
the probabilities that a particular surgery will be rated
\begin{enumerate}
	\item complex or very complex;
	\item neither very complex nor very simple;
	\item routine or complex
	\item routine or simple
\end{enumerate}
\solution
%\input{exemplar/11/16/3/8(1)/main.tex}
\item A card is selected from a pack of 52 cards.
\begin{enumerate}[label=(\alph*)]
    \item How many points are there in the sample space?
    \item Calculate the probability that the card is an ace of spades.
    \item Calculate the probability that the card is (i) an ace and (ii) black card.
\end{enumerate}
\solution
%\input{exemplar/11/16/3/4/main2.tex}
\item The probability that a non leap year selected at random will contain 53 sundays.
\\
\solution
%\input{exemplar/10/13/1/19/main.tex}
\item One of the four persons John, Rita, Aslam or Gurpreet will be promoted next
month. Consequently the sample space consists of four elementary outcomes
S = {John promoted, Rita promoted, Aslam promoted, Gurpreet promoted}
You are told that the chances of John’s promotion is same as that of Gurpreet,
Rita’s chances of promotion are twice as likely as Johns. Aslam’s chances are
four times that of John.
\begin{enumerate}
	\item Determine
	\begin{enumerate}
		\item P (John promoted)
		\item P (Rita promoted)
		\item P (Aslam promoted)
		\item P (Gurpreet promoted)
	\end{enumerate}
	\item If A = {John promoted or Gurpreet promoted}, find P (A).
\end{enumerate}
\solution
%\input{exemplar/11/16/3/10/main.tex}
\item A card is drawn from a deck of 52 cards. Find the probability of getting a king or a heart or a red card.\\
\solution
%\input{exemplar/11/16/3/15/main.tex}
\item The probability that a student will pass his examination is 0.73, the probability of
the student getting a compartment is 0.13, and the probability that the student will
either pass or get compartment is 0.96. State True or False.\\
\solution
%\input{exemplar/11/16/3/31/main.tex}
\item A card is selected from a pack of 52 cards\\
\begin{enumerate}[label=(\alph*)]
\item How many points are there in the sample space?
\item Calculate the probability that the cards is an ace of spades.
\item Calculate the probability that the card is (i) an ace (ii)black card.\\
\end{enumerate}
%\input{ncert/11/16/3/4_1/Prob_4.tex}
\item In a non-leap year, the probability of having 53 tuesdays or 53 wednesdays is\\
\solution
%\input{exemplar/11/16/3/18/main.tex}
\item There are 1000 sealed envelopes in a box, 10 of them contain a cash prize of
Rs 100 each, 100 of them contain a cash prize of Rs 50 each and 200 of them
contain a cash prize of Rs 10 each and rest do not contain any cash prize. If they
are well shuffled and an envelope is picked up out, what is the probability that it
contains no cash prize?\\
\solution
%\input{exemplar/10/13/3/34/main.tex}
\item 
A die is thrown and a card is selected at random from a deck of 52 playing cards. The probability of getting an even number on the die and a spade card.\\
\solution
%\input{exemplar/12/13/3/78/main.tex}
\item
If 4-digit numbers greater than 5,000 are randomly formed from the digits 0, 1, 3, 5, and 7, what is the probability of forming a number divisible by 5 when:
\begin{enumerate}
    \item The digits are repeated?
    \item The repetition of digits is not allowed?
\end{enumerate}
\solution
%\input{ncert/11/16/4/9/main.tex}
\item Consider the probability space $\brak{\Omega, \mathcal{G}, P}$ where $\Omega = [0,2]$ and $\mathcal{G} = \cbrak{\phi, \Omega, [0,1], (1,2]}$. Let $X$ and $Y$ be two functions on $\Omega$ defined as
\begin{align*}
    X(\omega) = 
    \begin{cases}
        1 & \text{if }\omega \in [0, 1]\\
        2 & \text{if }\omega \in (1, 2]
    \end{cases}
\end{align*}
and
\begin{align*}
    Y(\omega) = 
    \begin{cases}
        2 & \text{if }\omega \in [0, 1.5]\\
        3 & \text{if }\omega \in (1.5, 2].
    \end{cases}
\end{align*}
Then which one of the following statements is true?
\begin{enumerate}
    \item [(A)] $X$ is a random variable with respect to $\mathcal{G}$, but $Y$ is not a random variable with respect to $\mathcal{G}$.
    \item [(B)] $Y$ is a random variable with respect to $\mathcal{G}$, but $X$ is not a random variable with respect to $\mathcal{G}$.
    \item [(C)] Neither $X$ nor $Y$ is a random variable with respect to $\mathcal{G}$.
    \item [(D)] Both $X$ and $Y$ are random variables with respect to $\mathcal{G}$.
\end{enumerate} \hfill (GATE ST 2023)\\
\solution
%\input{gate/ST/2023/14/main.tex}
	\item  A die is loaded in such a way that each odd number is twice as likely to occur as
each even number. Find $P(G)$, where $G$ is the event that a number greater than
3 occurs on a single roll of the die.
\\
\solution
		%\input{exemplar/11/16/3/5/main.tex}
	\item All the jacks, queens and kings are removed from a deck of 52 playing cards. The remaining cards are well shuffled and then one card is drawn at random. Giving ace a value 1 similar value for other cards, find the probability that the card has a value 
		\begin{enumerate}
			\item 7
			\item greater than 7
			\item less than 7
		\end{enumerate}
		%\input{exemplar/10/13/3/30/main.tex}
  \item A Lot consists of 48 mobile phones of which 42 are good, 3 have only minor defects and 3 have major defects.Varnika will buy a phone if it is good but the trader will only buy a mobile if it has no major defects. One phone is selected at random from the lot. What is the probability that it is
\begin{enumerate}
	\item acceptable to Varnika?
            \item acceptable to the trader?
\end{enumerate}
\solution
	%\input{exemplar/10/13/3/40/main.tex}
 \item A student says that if you throw a die, it will show up 1 or not 1. Therefore, the probability of getting 1 and the probability of getting 'not 1' each is equal to $\frac{1}{2}$. Is this correct? Give reasons.\\
 \solution
        %\input{exemplar/10/13/2/9/main.tex}
   \item Four candidates A, B, C, D have ap-
plied for the assignment to coach a school cricket
team. If A is twice as likely to be selected as B, and
B and C are given about the same chance of being
selected, while C is twice as likely to be selected
as D, what are the probabilities that
\begin{enumerate}
\item C will be selected?
\item A will not be selected?
\end{enumerate}
	%\input{exemplar/11/16/3/9/main.tex}
 \item A bag contain 24 balls of which $x$ balls are red, $2x$ are white and $3x$ are blue. A ball is selected at random, What is the probability that it is
\begin{enumerate}[label=\alph*)]
\item not red ?
\item white ?
\end{enumerate}
%\input{exemplar/10/13/3/41/main.tex}
If the letters of the word ASSASSINATION are arranged at random. Find the Probability that
\begin{enumerate}[label=(\alph*)]
\item Four $S's$ come consecutively in the word
\item Two  $I's$ and two $N's$ come together
\item All $A's$ are not coming together
\item No two $A's$ are coming together
\end{enumerate}
%\input{exemplar/11/16/3/14/main.tex}
	\item One urn contains two black balls (labelled B1 and B2) and one white ball. A
	second urn contains one black ball and two white balls (labelled W1 and W2).
	Suppose the following experiment is performed. One of the two urns is chosen
	at random. Next a ball is randomly chosen from the urn. Then a second ball is
	chosen at random from the same urn without replacing the first ball.
	
	\begin{enumerate}
	\item What is the probability that two black balls are chosen?
	
	\item What is the probability that two balls of opposite colour are chosen?
	\end{enumerate}
	\solution
	%\input{exemplar/11/16/3/12/main1.tex}
\end{enumerate}

		\item A box of oranges is inspected by examining three randomly selected oranges drawn without replacement. If all the three oranges are good, the box is approved for sale, otherwise, it is rejected. Find the probability that a box containing 15 oranges out of which 12 are good and 3 are bad ones will be approved for sale.
		\label{ncert/12/13/2/3/defs.tex}
		\item Two balls are drawn at random with replacement from a box containing 10 black and 8 red balls. Find the probability that
		\label{ncert/12/13/2/12}
\begin{enumerate}
\item both balls are red.
\item first ball is black and second is red.
\item one of them is black and other is red.
\end{enumerate}

\item In a hostel, 60\% of the students read Hindi newspaper, 40\% read English newspaper and 20\% read both Hindi and English newspapers. A student is selected at random.
		\label{ncert/12/13/2/15}
\begin{enumerate}
\item Find the probability that she reads neither Hindi nor English newspapers.
\item If she reads Hindi newspaper, find the probability that she reads English newspaper.
\item If she reads English newspaper, find the probability that she reads Hindi newspaper.\\
\end{enumerate}
\item The probability of obtaining an even prime number on each die, when a pair of dice is rolled is 
\begin{enumerate}
    \item $0$ 
    
    \item $\frac{1}{3}$ 
    
    \item $\frac{1}{12}$ 
    
    \item $\frac{1}{36}$ 
\end{enumerate}
\solution
		%\begin{enumerate}[label=\thesection.\arabic*,ref=\thesection.\theenumi]
	\item One card is drawn from a well-shuffled deck of 52 cards. Find the probability of getting
\begin{enumerate}
\item A king of red colour 
\item A face card 
\item A red face card
\item The jack of hearts
\item A spade
\item The queen of diamonds

\end{enumerate}
\solution
		%\input{ncert/10/15/1/14/main.tex}
	\item Five cards—the ten, jack, queen, king and ace of diamonds, are well-shuffled with their face downwards. One card is then picked up at random.
\begin{enumerate}
\item
What is the probability that the card is the queen? 
\item
If the queen is drawn and put aside, what is the probability that the second card picked up is (a) an ace? (b) a queen?\\
\end{enumerate}
\solution
		%\input{ncert/10/15/1/15/defs.tex}
	\item A bag contains $5$ red balls and some blue balls. If the probability of drawing a blue ball is double that if a red ball, determine the number of blue balls in the bag. 
		\\
\solution
		%\input{ncert/10/15/2/3/defs.tex}
	\item A card is selected from a pack of 52 cards.
 \begin{enumerate}[label=(\alph*)] 
                 \item How many points are there in the sample space?
                 \item Calculate the probability that the card is an ace of spades.
                 \item Calculate the probability that the card is (i) an ace and (ii) black card.
 \end{enumerate}
\solution
		%\input{ncert/11/16/3/4/main.tex}
\item Four cards are drawn from a well-shuffled deck of 52 cards. What is the probability of obtaining 3 diamonds and one spade.
\\
\solution
		%\input{ncert/11/16/4/2/defs.tex}
\item In a certain lottery 10,000 tickets are sold and ten equal prizes are awarded. What is the probability of not getting a prize if you buy (a) one ticket (b) two tickets (c) 10 tickets ?	
\\
\solution
		%\input{ncert/11/16/4/4/defs.tex}
		%
\item 
Out of 100 students, two sections of 40 and 60 are formed. If you and your friend are among the 100 students, what is the probability that
\begin{enumerate}
\item you both enter the same section?
\item you both enter the different sections?
\end{enumerate}
\solution
		%\input{ncert/11/16/4/5/defs.tex}
	\item 
The number lock of a suitcase has 4 wheels each labelled with ten digits i.e. from 0 to 9.The lock opens with a sequence of four digits with no repeats.What is the probability of a person getting the right sequence to open the suitcase.
\\
\solution
		%\input{ncert/11/16/4/10/defs.tex}
		%
\item 
Two cards are drawn at random and without replacement from a pack of 52 playing cards. Find the probability that both the cards are black.
\\
\solution
		%\input{ncert/12/13/2/2/defs.tex}
		\item A box of oranges is inspected by examining three randomly selected oranges drawn without replacement. If all the three oranges are good, the box is approved for sale, otherwise, it is rejected. Find the probability that a box containing 15 oranges out of which 12 are good and 3 are bad ones will be approved for sale.
		\label{ncert/12/13/2/3/defs.tex}
		\item Two balls are drawn at random with replacement from a box containing 10 black and 8 red balls. Find the probability that
		\label{ncert/12/13/2/12}
\begin{enumerate}
\item both balls are red.
\item first ball is black and second is red.
\item one of them is black and other is red.
\end{enumerate}

\item In a hostel, 60\% of the students read Hindi newspaper, 40\% read English newspaper and 20\% read both Hindi and English newspapers. A student is selected at random.
		\label{ncert/12/13/2/15}
\begin{enumerate}
\item Find the probability that she reads neither Hindi nor English newspapers.
\item If she reads Hindi newspaper, find the probability that she reads English newspaper.
\item If she reads English newspaper, find the probability that she reads Hindi newspaper.\\
\end{enumerate}
\item The probability of obtaining an even prime number on each die, when a pair of dice is rolled is 
\begin{enumerate}
    \item $0$ 
    
    \item $\frac{1}{3}$ 
    
    \item $\frac{1}{12}$ 
    
    \item $\frac{1}{36}$ 
\end{enumerate}
\solution
		%\input{ncert/12/13/2/17/defs.tex}
	\item A bag contains 4 red and 4 black balls, another bag contains 2 red and 6 black balls. One of the two bags is selected at random and a ball is drawn from the bag which is found to be red. Find the probability that the ball is drawn from the first bag.
\\
\solution
		%\input{ncert/12/13/3/2/main.tex}
  \item
  Cards with numbers 2 to 101 are placed in a box. A card is selected at random.Find the probability that the card has
\begin{enumerate}[label=(\roman*)]
	\item an even number 
	\item a square number
\end{enumerate}
\solution
%\input{exemplar/10/13/3/32/main.tex}
\item
The king, queen and jack of clubs are removed from a deck of 52 playing cards and then well shuffled. Now one card is drawn at random from the remaining cards.  Determine the probability that the card is
\begin{enumerate}[label=(\roman*)]
\item a club
\item 10 of hearts
\end{enumerate}
\solution
%\input{exemplar/10/13/3/29/main.tex}
\item A team of medical students doing their internship have to assist during surgeries
at a city hospital. The probabilities of surgeries rated as very complex, complex,
routine, simple or very simple are respectively, 0.15, 0.20, 0.31, 0.26, .08. Find
the probabilities that a particular surgery will be rated
\begin{enumerate}
	\item complex or very complex;
	\item neither very complex nor very simple;
	\item routine or complex
	\item routine or simple
\end{enumerate}
\solution
%\input{exemplar/11/16/3/8(1)/main.tex}
\item A card is selected from a pack of 52 cards.
\begin{enumerate}[label=(\alph*)]
    \item How many points are there in the sample space?
    \item Calculate the probability that the card is an ace of spades.
    \item Calculate the probability that the card is (i) an ace and (ii) black card.
\end{enumerate}
\solution
%\input{exemplar/11/16/3/4/main2.tex}
\item The probability that a non leap year selected at random will contain 53 sundays.
\\
\solution
%\input{exemplar/10/13/1/19/main.tex}
\item One of the four persons John, Rita, Aslam or Gurpreet will be promoted next
month. Consequently the sample space consists of four elementary outcomes
S = {John promoted, Rita promoted, Aslam promoted, Gurpreet promoted}
You are told that the chances of John’s promotion is same as that of Gurpreet,
Rita’s chances of promotion are twice as likely as Johns. Aslam’s chances are
four times that of John.
\begin{enumerate}
	\item Determine
	\begin{enumerate}
		\item P (John promoted)
		\item P (Rita promoted)
		\item P (Aslam promoted)
		\item P (Gurpreet promoted)
	\end{enumerate}
	\item If A = {John promoted or Gurpreet promoted}, find P (A).
\end{enumerate}
\solution
%\input{exemplar/11/16/3/10/main.tex}
\item A card is drawn from a deck of 52 cards. Find the probability of getting a king or a heart or a red card.\\
\solution
%\input{exemplar/11/16/3/15/main.tex}
\item The probability that a student will pass his examination is 0.73, the probability of
the student getting a compartment is 0.13, and the probability that the student will
either pass or get compartment is 0.96. State True or False.\\
\solution
%\input{exemplar/11/16/3/31/main.tex}
\item A card is selected from a pack of 52 cards\\
\begin{enumerate}[label=(\alph*)]
\item How many points are there in the sample space?
\item Calculate the probability that the cards is an ace of spades.
\item Calculate the probability that the card is (i) an ace (ii)black card.\\
\end{enumerate}
%\input{ncert/11/16/3/4_1/Prob_4.tex}
\item In a non-leap year, the probability of having 53 tuesdays or 53 wednesdays is\\
\solution
%\input{exemplar/11/16/3/18/main.tex}
\item There are 1000 sealed envelopes in a box, 10 of them contain a cash prize of
Rs 100 each, 100 of them contain a cash prize of Rs 50 each and 200 of them
contain a cash prize of Rs 10 each and rest do not contain any cash prize. If they
are well shuffled and an envelope is picked up out, what is the probability that it
contains no cash prize?\\
\solution
%\input{exemplar/10/13/3/34/main.tex}
\item 
A die is thrown and a card is selected at random from a deck of 52 playing cards. The probability of getting an even number on the die and a spade card.\\
\solution
%\input{exemplar/12/13/3/78/main.tex}
\item
If 4-digit numbers greater than 5,000 are randomly formed from the digits 0, 1, 3, 5, and 7, what is the probability of forming a number divisible by 5 when:
\begin{enumerate}
    \item The digits are repeated?
    \item The repetition of digits is not allowed?
\end{enumerate}
\solution
%\input{ncert/11/16/4/9/main.tex}
\item Consider the probability space $\brak{\Omega, \mathcal{G}, P}$ where $\Omega = [0,2]$ and $\mathcal{G} = \cbrak{\phi, \Omega, [0,1], (1,2]}$. Let $X$ and $Y$ be two functions on $\Omega$ defined as
\begin{align*}
    X(\omega) = 
    \begin{cases}
        1 & \text{if }\omega \in [0, 1]\\
        2 & \text{if }\omega \in (1, 2]
    \end{cases}
\end{align*}
and
\begin{align*}
    Y(\omega) = 
    \begin{cases}
        2 & \text{if }\omega \in [0, 1.5]\\
        3 & \text{if }\omega \in (1.5, 2].
    \end{cases}
\end{align*}
Then which one of the following statements is true?
\begin{enumerate}
    \item [(A)] $X$ is a random variable with respect to $\mathcal{G}$, but $Y$ is not a random variable with respect to $\mathcal{G}$.
    \item [(B)] $Y$ is a random variable with respect to $\mathcal{G}$, but $X$ is not a random variable with respect to $\mathcal{G}$.
    \item [(C)] Neither $X$ nor $Y$ is a random variable with respect to $\mathcal{G}$.
    \item [(D)] Both $X$ and $Y$ are random variables with respect to $\mathcal{G}$.
\end{enumerate} \hfill (GATE ST 2023)\\
\solution
%\input{gate/ST/2023/14/main.tex}
	\item  A die is loaded in such a way that each odd number is twice as likely to occur as
each even number. Find $P(G)$, where $G$ is the event that a number greater than
3 occurs on a single roll of the die.
\\
\solution
		%\input{exemplar/11/16/3/5/main.tex}
	\item All the jacks, queens and kings are removed from a deck of 52 playing cards. The remaining cards are well shuffled and then one card is drawn at random. Giving ace a value 1 similar value for other cards, find the probability that the card has a value 
		\begin{enumerate}
			\item 7
			\item greater than 7
			\item less than 7
		\end{enumerate}
		%\input{exemplar/10/13/3/30/main.tex}
  \item A Lot consists of 48 mobile phones of which 42 are good, 3 have only minor defects and 3 have major defects.Varnika will buy a phone if it is good but the trader will only buy a mobile if it has no major defects. One phone is selected at random from the lot. What is the probability that it is
\begin{enumerate}
	\item acceptable to Varnika?
            \item acceptable to the trader?
\end{enumerate}
\solution
	%\input{exemplar/10/13/3/40/main.tex}
 \item A student says that if you throw a die, it will show up 1 or not 1. Therefore, the probability of getting 1 and the probability of getting 'not 1' each is equal to $\frac{1}{2}$. Is this correct? Give reasons.\\
 \solution
        %\input{exemplar/10/13/2/9/main.tex}
   \item Four candidates A, B, C, D have ap-
plied for the assignment to coach a school cricket
team. If A is twice as likely to be selected as B, and
B and C are given about the same chance of being
selected, while C is twice as likely to be selected
as D, what are the probabilities that
\begin{enumerate}
\item C will be selected?
\item A will not be selected?
\end{enumerate}
	%\input{exemplar/11/16/3/9/main.tex}
 \item A bag contain 24 balls of which $x$ balls are red, $2x$ are white and $3x$ are blue. A ball is selected at random, What is the probability that it is
\begin{enumerate}[label=\alph*)]
\item not red ?
\item white ?
\end{enumerate}
%\input{exemplar/10/13/3/41/main.tex}
If the letters of the word ASSASSINATION are arranged at random. Find the Probability that
\begin{enumerate}[label=(\alph*)]
\item Four $S's$ come consecutively in the word
\item Two  $I's$ and two $N's$ come together
\item All $A's$ are not coming together
\item No two $A's$ are coming together
\end{enumerate}
%\input{exemplar/11/16/3/14/main.tex}
	\item One urn contains two black balls (labelled B1 and B2) and one white ball. A
	second urn contains one black ball and two white balls (labelled W1 and W2).
	Suppose the following experiment is performed. One of the two urns is chosen
	at random. Next a ball is randomly chosen from the urn. Then a second ball is
	chosen at random from the same urn without replacing the first ball.
	
	\begin{enumerate}
	\item What is the probability that two black balls are chosen?
	
	\item What is the probability that two balls of opposite colour are chosen?
	\end{enumerate}
	\solution
	%\input{exemplar/11/16/3/12/main1.tex}
\end{enumerate}

	\item A bag contains 4 red and 4 black balls, another bag contains 2 red and 6 black balls. One of the two bags is selected at random and a ball is drawn from the bag which is found to be red. Find the probability that the ball is drawn from the first bag.
\\
\solution
		%\begin{table}[H]
	\centering
\begin{tabular}{|c|c|c|}
\hline
Random variable &Value &Definition\\ \hline
\multirow{3}{*}{X} &0 &Slips of Rs 1\\
&1 &Slips of Rs 5\\
&2 &Slips of Rs 13\\ \hline
\multirow{2}{*}{Y} &0 &Box A\\
&1 &Box B\\\hline
\end{tabular}
\caption{}
\label{tab:Distribution}
\end{table}
See \tabref{tab:Distribution}.
\begin{align}
p_{Y}\brak{k}= \begin{cases} 
      \frac{1}{3} & {k=0} \\
      \frac{2}{3 }& {k=1} 
   \end{cases}
   \\
p_{Y|X}\brak{0|0} = \frac{19}{25}\, 
p_{Y|X}\brak{0|1} = \frac{6}{25}\,
p_{Y|X}\brak{1|0} = \frac{45}{50}\,
p_{Y|X}\brak{1|2} = \frac{5}{50}
\end{align}
The desired probability is the probability that a slip drawn at random is marked other than Rs 1,
\begin{align}
&=1-p_X\brak{0}\\
&= p_X(1) + p_X(2)
\end{align}
Using Bayes theorem,
\begin{align}
&= p_Y\brak{0} \times \pr{Y=0 | X=1} + p_Y\brak{1} \times \pr{Y=1|X=2}\\
&=\frac{1}{3} \times \frac{6}{25} + \frac{2}{3} \times \frac{5}{50}\\
&=\frac{11}{75}
\end{align}

\newpage

%\tableofcontents

\bigskip

\renewcommand{\thefigure}{\theenumi}
\renewcommand{\thetable}{\theenumi}
%\renewcommand{\theequation}{\theenumi}

%\begin{abstract}
%%\boldmath
%In this letter, an algorithm for evaluating the exact analytical bit error rate  (BER)  for the piecewise linear (PL) combiner for  multiple relays is presented. Previous results were available only for upto three relays. The algorithm is unique in the sense that  the actual mathematical expressions, that are prohibitively large, need not be explicitly obtained. The diversity gain due to multiple relays is shown through plots of the analytical BER, well supported by simulations. 
%
%\end{abstract}
% IEEEtran.cls defaults to using nonbold math in the Abstract.
% This preserves the distinction between vectors and scalars. However,
% if the journal you are submitting to favors bold math in the abstract,
% then you can use LaTeX's standard command \boldmath at the very start
% of the abstract to achieve this. Many IEEE journals frown on math
% in the abstract anyway.

% Note that keywords are not normally used for peerreview papers.
%\begin{IEEEkeywords}
%Cooperative diversity, decode and forward, piecewise linear
%\end{IEEEkeywords}



% For peer review papers, you can put extra information on the cover
% page as needed:
% \ifCLASSOPTIONpeerreview
% \begin{center} \bfseries EDICS Category: 3-BBND \end{center}
% \fi
%
% For peerreview papers, this IEEEtran command inserts a page break and
% creates the second title. It will be ignored for other modes.
%\IEEEpeerreviewmaketitle




  \item
  Cards with numbers 2 to 101 are placed in a box. A card is selected at random.Find the probability that the card has
\begin{enumerate}[label=(\roman*)]
	\item an even number 
	\item a square number
\end{enumerate}
\solution
%\begin{table}[H]
	\centering
\begin{tabular}{|c|c|c|}
\hline
Random variable &Value &Definition\\ \hline
\multirow{3}{*}{X} &0 &Slips of Rs 1\\
&1 &Slips of Rs 5\\
&2 &Slips of Rs 13\\ \hline
\multirow{2}{*}{Y} &0 &Box A\\
&1 &Box B\\\hline
\end{tabular}
\caption{}
\label{tab:Distribution}
\end{table}
See \tabref{tab:Distribution}.
\begin{align}
p_{Y}\brak{k}= \begin{cases} 
      \frac{1}{3} & {k=0} \\
      \frac{2}{3 }& {k=1} 
   \end{cases}
   \\
p_{Y|X}\brak{0|0} = \frac{19}{25}\, 
p_{Y|X}\brak{0|1} = \frac{6}{25}\,
p_{Y|X}\brak{1|0} = \frac{45}{50}\,
p_{Y|X}\brak{1|2} = \frac{5}{50}
\end{align}
The desired probability is the probability that a slip drawn at random is marked other than Rs 1,
\begin{align}
&=1-p_X\brak{0}\\
&= p_X(1) + p_X(2)
\end{align}
Using Bayes theorem,
\begin{align}
&= p_Y\brak{0} \times \pr{Y=0 | X=1} + p_Y\brak{1} \times \pr{Y=1|X=2}\\
&=\frac{1}{3} \times \frac{6}{25} + \frac{2}{3} \times \frac{5}{50}\\
&=\frac{11}{75}
\end{align}

\newpage

%\tableofcontents

\bigskip

\renewcommand{\thefigure}{\theenumi}
\renewcommand{\thetable}{\theenumi}
%\renewcommand{\theequation}{\theenumi}

%\begin{abstract}
%%\boldmath
%In this letter, an algorithm for evaluating the exact analytical bit error rate  (BER)  for the piecewise linear (PL) combiner for  multiple relays is presented. Previous results were available only for upto three relays. The algorithm is unique in the sense that  the actual mathematical expressions, that are prohibitively large, need not be explicitly obtained. The diversity gain due to multiple relays is shown through plots of the analytical BER, well supported by simulations. 
%
%\end{abstract}
% IEEEtran.cls defaults to using nonbold math in the Abstract.
% This preserves the distinction between vectors and scalars. However,
% if the journal you are submitting to favors bold math in the abstract,
% then you can use LaTeX's standard command \boldmath at the very start
% of the abstract to achieve this. Many IEEE journals frown on math
% in the abstract anyway.

% Note that keywords are not normally used for peerreview papers.
%\begin{IEEEkeywords}
%Cooperative diversity, decode and forward, piecewise linear
%\end{IEEEkeywords}



% For peer review papers, you can put extra information on the cover
% page as needed:
% \ifCLASSOPTIONpeerreview
% \begin{center} \bfseries EDICS Category: 3-BBND \end{center}
% \fi
%
% For peerreview papers, this IEEEtran command inserts a page break and
% creates the second title. It will be ignored for other modes.
%\IEEEpeerreviewmaketitle




\item
The king, queen and jack of clubs are removed from a deck of 52 playing cards and then well shuffled. Now one card is drawn at random from the remaining cards.  Determine the probability that the card is
\begin{enumerate}[label=(\roman*)]
\item a club
\item 10 of hearts
\end{enumerate}
\solution
%\begin{table}[H]
	\centering
\begin{tabular}{|c|c|c|}
\hline
Random variable &Value &Definition\\ \hline
\multirow{3}{*}{X} &0 &Slips of Rs 1\\
&1 &Slips of Rs 5\\
&2 &Slips of Rs 13\\ \hline
\multirow{2}{*}{Y} &0 &Box A\\
&1 &Box B\\\hline
\end{tabular}
\caption{}
\label{tab:Distribution}
\end{table}
See \tabref{tab:Distribution}.
\begin{align}
p_{Y}\brak{k}= \begin{cases} 
      \frac{1}{3} & {k=0} \\
      \frac{2}{3 }& {k=1} 
   \end{cases}
   \\
p_{Y|X}\brak{0|0} = \frac{19}{25}\, 
p_{Y|X}\brak{0|1} = \frac{6}{25}\,
p_{Y|X}\brak{1|0} = \frac{45}{50}\,
p_{Y|X}\brak{1|2} = \frac{5}{50}
\end{align}
The desired probability is the probability that a slip drawn at random is marked other than Rs 1,
\begin{align}
&=1-p_X\brak{0}\\
&= p_X(1) + p_X(2)
\end{align}
Using Bayes theorem,
\begin{align}
&= p_Y\brak{0} \times \pr{Y=0 | X=1} + p_Y\brak{1} \times \pr{Y=1|X=2}\\
&=\frac{1}{3} \times \frac{6}{25} + \frac{2}{3} \times \frac{5}{50}\\
&=\frac{11}{75}
\end{align}

\newpage

%\tableofcontents

\bigskip

\renewcommand{\thefigure}{\theenumi}
\renewcommand{\thetable}{\theenumi}
%\renewcommand{\theequation}{\theenumi}

%\begin{abstract}
%%\boldmath
%In this letter, an algorithm for evaluating the exact analytical bit error rate  (BER)  for the piecewise linear (PL) combiner for  multiple relays is presented. Previous results were available only for upto three relays. The algorithm is unique in the sense that  the actual mathematical expressions, that are prohibitively large, need not be explicitly obtained. The diversity gain due to multiple relays is shown through plots of the analytical BER, well supported by simulations. 
%
%\end{abstract}
% IEEEtran.cls defaults to using nonbold math in the Abstract.
% This preserves the distinction between vectors and scalars. However,
% if the journal you are submitting to favors bold math in the abstract,
% then you can use LaTeX's standard command \boldmath at the very start
% of the abstract to achieve this. Many IEEE journals frown on math
% in the abstract anyway.

% Note that keywords are not normally used for peerreview papers.
%\begin{IEEEkeywords}
%Cooperative diversity, decode and forward, piecewise linear
%\end{IEEEkeywords}



% For peer review papers, you can put extra information on the cover
% page as needed:
% \ifCLASSOPTIONpeerreview
% \begin{center} \bfseries EDICS Category: 3-BBND \end{center}
% \fi
%
% For peerreview papers, this IEEEtran command inserts a page break and
% creates the second title. It will be ignored for other modes.
%\IEEEpeerreviewmaketitle




\item A team of medical students doing their internship have to assist during surgeries
at a city hospital. The probabilities of surgeries rated as very complex, complex,
routine, simple or very simple are respectively, 0.15, 0.20, 0.31, 0.26, .08. Find
the probabilities that a particular surgery will be rated
\begin{enumerate}
	\item complex or very complex;
	\item neither very complex nor very simple;
	\item routine or complex
	\item routine or simple
\end{enumerate}
\solution
%\begin{table}[H]
	\centering
\begin{tabular}{|c|c|c|}
\hline
Random variable &Value &Definition\\ \hline
\multirow{3}{*}{X} &0 &Slips of Rs 1\\
&1 &Slips of Rs 5\\
&2 &Slips of Rs 13\\ \hline
\multirow{2}{*}{Y} &0 &Box A\\
&1 &Box B\\\hline
\end{tabular}
\caption{}
\label{tab:Distribution}
\end{table}
See \tabref{tab:Distribution}.
\begin{align}
p_{Y}\brak{k}= \begin{cases} 
      \frac{1}{3} & {k=0} \\
      \frac{2}{3 }& {k=1} 
   \end{cases}
   \\
p_{Y|X}\brak{0|0} = \frac{19}{25}\, 
p_{Y|X}\brak{0|1} = \frac{6}{25}\,
p_{Y|X}\brak{1|0} = \frac{45}{50}\,
p_{Y|X}\brak{1|2} = \frac{5}{50}
\end{align}
The desired probability is the probability that a slip drawn at random is marked other than Rs 1,
\begin{align}
&=1-p_X\brak{0}\\
&= p_X(1) + p_X(2)
\end{align}
Using Bayes theorem,
\begin{align}
&= p_Y\brak{0} \times \pr{Y=0 | X=1} + p_Y\brak{1} \times \pr{Y=1|X=2}\\
&=\frac{1}{3} \times \frac{6}{25} + \frac{2}{3} \times \frac{5}{50}\\
&=\frac{11}{75}
\end{align}

\newpage

%\tableofcontents

\bigskip

\renewcommand{\thefigure}{\theenumi}
\renewcommand{\thetable}{\theenumi}
%\renewcommand{\theequation}{\theenumi}

%\begin{abstract}
%%\boldmath
%In this letter, an algorithm for evaluating the exact analytical bit error rate  (BER)  for the piecewise linear (PL) combiner for  multiple relays is presented. Previous results were available only for upto three relays. The algorithm is unique in the sense that  the actual mathematical expressions, that are prohibitively large, need not be explicitly obtained. The diversity gain due to multiple relays is shown through plots of the analytical BER, well supported by simulations. 
%
%\end{abstract}
% IEEEtran.cls defaults to using nonbold math in the Abstract.
% This preserves the distinction between vectors and scalars. However,
% if the journal you are submitting to favors bold math in the abstract,
% then you can use LaTeX's standard command \boldmath at the very start
% of the abstract to achieve this. Many IEEE journals frown on math
% in the abstract anyway.

% Note that keywords are not normally used for peerreview papers.
%\begin{IEEEkeywords}
%Cooperative diversity, decode and forward, piecewise linear
%\end{IEEEkeywords}



% For peer review papers, you can put extra information on the cover
% page as needed:
% \ifCLASSOPTIONpeerreview
% \begin{center} \bfseries EDICS Category: 3-BBND \end{center}
% \fi
%
% For peerreview papers, this IEEEtran command inserts a page break and
% creates the second title. It will be ignored for other modes.
%\IEEEpeerreviewmaketitle




\item A card is selected from a pack of 52 cards.
\begin{enumerate}[label=(\alph*)]
    \item How many points are there in the sample space?
    \item Calculate the probability that the card is an ace of spades.
    \item Calculate the probability that the card is (i) an ace and (ii) black card.
\end{enumerate}
\solution
%Let $X$ be an bernoulli rv defined as in \tabref{tab:exemplar/11/16/3/26}.  Then, 
\begin{equation}
    p =
        \frac{4}{11} 
\end{equation}
\begin{table}[H]
	\centering
	\input{exemplar/11/16/3/26/tables/Table2.tex}
	\caption{}
        \label{tab:exemplar/11/16/3/26}
\end{table}

\item The probability that a non leap year selected at random will contain 53 sundays.
\\
\solution
%\begin{table}[H]
	\centering
\begin{tabular}{|c|c|c|}
\hline
Random variable &Value &Definition\\ \hline
\multirow{3}{*}{X} &0 &Slips of Rs 1\\
&1 &Slips of Rs 5\\
&2 &Slips of Rs 13\\ \hline
\multirow{2}{*}{Y} &0 &Box A\\
&1 &Box B\\\hline
\end{tabular}
\caption{}
\label{tab:Distribution}
\end{table}
See \tabref{tab:Distribution}.
\begin{align}
p_{Y}\brak{k}= \begin{cases} 
      \frac{1}{3} & {k=0} \\
      \frac{2}{3 }& {k=1} 
   \end{cases}
   \\
p_{Y|X}\brak{0|0} = \frac{19}{25}\, 
p_{Y|X}\brak{0|1} = \frac{6}{25}\,
p_{Y|X}\brak{1|0} = \frac{45}{50}\,
p_{Y|X}\brak{1|2} = \frac{5}{50}
\end{align}
The desired probability is the probability that a slip drawn at random is marked other than Rs 1,
\begin{align}
&=1-p_X\brak{0}\\
&= p_X(1) + p_X(2)
\end{align}
Using Bayes theorem,
\begin{align}
&= p_Y\brak{0} \times \pr{Y=0 | X=1} + p_Y\brak{1} \times \pr{Y=1|X=2}\\
&=\frac{1}{3} \times \frac{6}{25} + \frac{2}{3} \times \frac{5}{50}\\
&=\frac{11}{75}
\end{align}

\newpage

%\tableofcontents

\bigskip

\renewcommand{\thefigure}{\theenumi}
\renewcommand{\thetable}{\theenumi}
%\renewcommand{\theequation}{\theenumi}

%\begin{abstract}
%%\boldmath
%In this letter, an algorithm for evaluating the exact analytical bit error rate  (BER)  for the piecewise linear (PL) combiner for  multiple relays is presented. Previous results were available only for upto three relays. The algorithm is unique in the sense that  the actual mathematical expressions, that are prohibitively large, need not be explicitly obtained. The diversity gain due to multiple relays is shown through plots of the analytical BER, well supported by simulations. 
%
%\end{abstract}
% IEEEtran.cls defaults to using nonbold math in the Abstract.
% This preserves the distinction between vectors and scalars. However,
% if the journal you are submitting to favors bold math in the abstract,
% then you can use LaTeX's standard command \boldmath at the very start
% of the abstract to achieve this. Many IEEE journals frown on math
% in the abstract anyway.

% Note that keywords are not normally used for peerreview papers.
%\begin{IEEEkeywords}
%Cooperative diversity, decode and forward, piecewise linear
%\end{IEEEkeywords}



% For peer review papers, you can put extra information on the cover
% page as needed:
% \ifCLASSOPTIONpeerreview
% \begin{center} \bfseries EDICS Category: 3-BBND \end{center}
% \fi
%
% For peerreview papers, this IEEEtran command inserts a page break and
% creates the second title. It will be ignored for other modes.
%\IEEEpeerreviewmaketitle




\item One of the four persons John, Rita, Aslam or Gurpreet will be promoted next
month. Consequently the sample space consists of four elementary outcomes
S = {John promoted, Rita promoted, Aslam promoted, Gurpreet promoted}
You are told that the chances of John’s promotion is same as that of Gurpreet,
Rita’s chances of promotion are twice as likely as Johns. Aslam’s chances are
four times that of John.
\begin{enumerate}
	\item Determine
	\begin{enumerate}
		\item P (John promoted)
		\item P (Rita promoted)
		\item P (Aslam promoted)
		\item P (Gurpreet promoted)
	\end{enumerate}
	\item If A = {John promoted or Gurpreet promoted}, find P (A).
\end{enumerate}
\solution
%\begin{table}[H]
	\centering
\begin{tabular}{|c|c|c|}
\hline
Random variable &Value &Definition\\ \hline
\multirow{3}{*}{X} &0 &Slips of Rs 1\\
&1 &Slips of Rs 5\\
&2 &Slips of Rs 13\\ \hline
\multirow{2}{*}{Y} &0 &Box A\\
&1 &Box B\\\hline
\end{tabular}
\caption{}
\label{tab:Distribution}
\end{table}
See \tabref{tab:Distribution}.
\begin{align}
p_{Y}\brak{k}= \begin{cases} 
      \frac{1}{3} & {k=0} \\
      \frac{2}{3 }& {k=1} 
   \end{cases}
   \\
p_{Y|X}\brak{0|0} = \frac{19}{25}\, 
p_{Y|X}\brak{0|1} = \frac{6}{25}\,
p_{Y|X}\brak{1|0} = \frac{45}{50}\,
p_{Y|X}\brak{1|2} = \frac{5}{50}
\end{align}
The desired probability is the probability that a slip drawn at random is marked other than Rs 1,
\begin{align}
&=1-p_X\brak{0}\\
&= p_X(1) + p_X(2)
\end{align}
Using Bayes theorem,
\begin{align}
&= p_Y\brak{0} \times \pr{Y=0 | X=1} + p_Y\brak{1} \times \pr{Y=1|X=2}\\
&=\frac{1}{3} \times \frac{6}{25} + \frac{2}{3} \times \frac{5}{50}\\
&=\frac{11}{75}
\end{align}

\newpage

%\tableofcontents

\bigskip

\renewcommand{\thefigure}{\theenumi}
\renewcommand{\thetable}{\theenumi}
%\renewcommand{\theequation}{\theenumi}

%\begin{abstract}
%%\boldmath
%In this letter, an algorithm for evaluating the exact analytical bit error rate  (BER)  for the piecewise linear (PL) combiner for  multiple relays is presented. Previous results were available only for upto three relays. The algorithm is unique in the sense that  the actual mathematical expressions, that are prohibitively large, need not be explicitly obtained. The diversity gain due to multiple relays is shown through plots of the analytical BER, well supported by simulations. 
%
%\end{abstract}
% IEEEtran.cls defaults to using nonbold math in the Abstract.
% This preserves the distinction between vectors and scalars. However,
% if the journal you are submitting to favors bold math in the abstract,
% then you can use LaTeX's standard command \boldmath at the very start
% of the abstract to achieve this. Many IEEE journals frown on math
% in the abstract anyway.

% Note that keywords are not normally used for peerreview papers.
%\begin{IEEEkeywords}
%Cooperative diversity, decode and forward, piecewise linear
%\end{IEEEkeywords}



% For peer review papers, you can put extra information on the cover
% page as needed:
% \ifCLASSOPTIONpeerreview
% \begin{center} \bfseries EDICS Category: 3-BBND \end{center}
% \fi
%
% For peerreview papers, this IEEEtran command inserts a page break and
% creates the second title. It will be ignored for other modes.
%\IEEEpeerreviewmaketitle




\item A card is drawn from a deck of 52 cards. Find the probability of getting a king or a heart or a red card.\\
\solution
%\begin{table}[H]
	\centering
\begin{tabular}{|c|c|c|}
\hline
Random variable &Value &Definition\\ \hline
\multirow{3}{*}{X} &0 &Slips of Rs 1\\
&1 &Slips of Rs 5\\
&2 &Slips of Rs 13\\ \hline
\multirow{2}{*}{Y} &0 &Box A\\
&1 &Box B\\\hline
\end{tabular}
\caption{}
\label{tab:Distribution}
\end{table}
See \tabref{tab:Distribution}.
\begin{align}
p_{Y}\brak{k}= \begin{cases} 
      \frac{1}{3} & {k=0} \\
      \frac{2}{3 }& {k=1} 
   \end{cases}
   \\
p_{Y|X}\brak{0|0} = \frac{19}{25}\, 
p_{Y|X}\brak{0|1} = \frac{6}{25}\,
p_{Y|X}\brak{1|0} = \frac{45}{50}\,
p_{Y|X}\brak{1|2} = \frac{5}{50}
\end{align}
The desired probability is the probability that a slip drawn at random is marked other than Rs 1,
\begin{align}
&=1-p_X\brak{0}\\
&= p_X(1) + p_X(2)
\end{align}
Using Bayes theorem,
\begin{align}
&= p_Y\brak{0} \times \pr{Y=0 | X=1} + p_Y\brak{1} \times \pr{Y=1|X=2}\\
&=\frac{1}{3} \times \frac{6}{25} + \frac{2}{3} \times \frac{5}{50}\\
&=\frac{11}{75}
\end{align}

\newpage

%\tableofcontents

\bigskip

\renewcommand{\thefigure}{\theenumi}
\renewcommand{\thetable}{\theenumi}
%\renewcommand{\theequation}{\theenumi}

%\begin{abstract}
%%\boldmath
%In this letter, an algorithm for evaluating the exact analytical bit error rate  (BER)  for the piecewise linear (PL) combiner for  multiple relays is presented. Previous results were available only for upto three relays. The algorithm is unique in the sense that  the actual mathematical expressions, that are prohibitively large, need not be explicitly obtained. The diversity gain due to multiple relays is shown through plots of the analytical BER, well supported by simulations. 
%
%\end{abstract}
% IEEEtran.cls defaults to using nonbold math in the Abstract.
% This preserves the distinction between vectors and scalars. However,
% if the journal you are submitting to favors bold math in the abstract,
% then you can use LaTeX's standard command \boldmath at the very start
% of the abstract to achieve this. Many IEEE journals frown on math
% in the abstract anyway.

% Note that keywords are not normally used for peerreview papers.
%\begin{IEEEkeywords}
%Cooperative diversity, decode and forward, piecewise linear
%\end{IEEEkeywords}



% For peer review papers, you can put extra information on the cover
% page as needed:
% \ifCLASSOPTIONpeerreview
% \begin{center} \bfseries EDICS Category: 3-BBND \end{center}
% \fi
%
% For peerreview papers, this IEEEtran command inserts a page break and
% creates the second title. It will be ignored for other modes.
%\IEEEpeerreviewmaketitle




\item The probability that a student will pass his examination is 0.73, the probability of
the student getting a compartment is 0.13, and the probability that the student will
either pass or get compartment is 0.96. State True or False.\\
\solution
%\begin{table}[H]
	\centering
\begin{tabular}{|c|c|c|}
\hline
Random variable &Value &Definition\\ \hline
\multirow{3}{*}{X} &0 &Slips of Rs 1\\
&1 &Slips of Rs 5\\
&2 &Slips of Rs 13\\ \hline
\multirow{2}{*}{Y} &0 &Box A\\
&1 &Box B\\\hline
\end{tabular}
\caption{}
\label{tab:Distribution}
\end{table}
See \tabref{tab:Distribution}.
\begin{align}
p_{Y}\brak{k}= \begin{cases} 
      \frac{1}{3} & {k=0} \\
      \frac{2}{3 }& {k=1} 
   \end{cases}
   \\
p_{Y|X}\brak{0|0} = \frac{19}{25}\, 
p_{Y|X}\brak{0|1} = \frac{6}{25}\,
p_{Y|X}\brak{1|0} = \frac{45}{50}\,
p_{Y|X}\brak{1|2} = \frac{5}{50}
\end{align}
The desired probability is the probability that a slip drawn at random is marked other than Rs 1,
\begin{align}
&=1-p_X\brak{0}\\
&= p_X(1) + p_X(2)
\end{align}
Using Bayes theorem,
\begin{align}
&= p_Y\brak{0} \times \pr{Y=0 | X=1} + p_Y\brak{1} \times \pr{Y=1|X=2}\\
&=\frac{1}{3} \times \frac{6}{25} + \frac{2}{3} \times \frac{5}{50}\\
&=\frac{11}{75}
\end{align}

\newpage

%\tableofcontents

\bigskip

\renewcommand{\thefigure}{\theenumi}
\renewcommand{\thetable}{\theenumi}
%\renewcommand{\theequation}{\theenumi}

%\begin{abstract}
%%\boldmath
%In this letter, an algorithm for evaluating the exact analytical bit error rate  (BER)  for the piecewise linear (PL) combiner for  multiple relays is presented. Previous results were available only for upto three relays. The algorithm is unique in the sense that  the actual mathematical expressions, that are prohibitively large, need not be explicitly obtained. The diversity gain due to multiple relays is shown through plots of the analytical BER, well supported by simulations. 
%
%\end{abstract}
% IEEEtran.cls defaults to using nonbold math in the Abstract.
% This preserves the distinction between vectors and scalars. However,
% if the journal you are submitting to favors bold math in the abstract,
% then you can use LaTeX's standard command \boldmath at the very start
% of the abstract to achieve this. Many IEEE journals frown on math
% in the abstract anyway.

% Note that keywords are not normally used for peerreview papers.
%\begin{IEEEkeywords}
%Cooperative diversity, decode and forward, piecewise linear
%\end{IEEEkeywords}



% For peer review papers, you can put extra information on the cover
% page as needed:
% \ifCLASSOPTIONpeerreview
% \begin{center} \bfseries EDICS Category: 3-BBND \end{center}
% \fi
%
% For peerreview papers, this IEEEtran command inserts a page break and
% creates the second title. It will be ignored for other modes.
%\IEEEpeerreviewmaketitle




\item A card is selected from a pack of 52 cards\\
\begin{enumerate}[label=(\alph*)]
\item How many points are there in the sample space?
\item Calculate the probability that the cards is an ace of spades.
\item Calculate the probability that the card is (i) an ace (ii)black card.\\
\end{enumerate}
%\input{ncert/11/16/3/4_1/Prob_4.tex}
\item In a non-leap year, the probability of having 53 tuesdays or 53 wednesdays is\\
\solution
%A non-leap year has a total of 365 days, and a week has 7 days.\\
So it can be expressed as 
\begin{align}
365\text{days} &=52\times 7+1 \text{day}
\end{align}
$\implies$ 52 tuesdays or wednesdays\\
Random variable X denotes the days of a week
\begin{align}
p_X\brak{k}&=\frac{1}{7}; \quad \brak{1<k<7}
\end{align}
So the probability of extra day being tuesday or wednesday is
\begin{align}
p_X\brak{3}+p_X\brak{4}&=\frac{1}{7}+\frac{1}{7}=\frac{2}{7}
\end{align}



\item There are 1000 sealed envelopes in a box, 10 of them contain a cash prize of
Rs 100 each, 100 of them contain a cash prize of Rs 50 each and 200 of them
contain a cash prize of Rs 10 each and rest do not contain any cash prize. If they
are well shuffled and an envelope is picked up out, what is the probability that it
contains no cash prize?\\
\solution
%\begin{table}[H]
	\centering
\begin{tabular}{|c|c|c|}
\hline
Random variable &Value &Definition\\ \hline
\multirow{3}{*}{X} &0 &Slips of Rs 1\\
&1 &Slips of Rs 5\\
&2 &Slips of Rs 13\\ \hline
\multirow{2}{*}{Y} &0 &Box A\\
&1 &Box B\\\hline
\end{tabular}
\caption{}
\label{tab:Distribution}
\end{table}
See \tabref{tab:Distribution}.
\begin{align}
p_{Y}\brak{k}= \begin{cases} 
      \frac{1}{3} & {k=0} \\
      \frac{2}{3 }& {k=1} 
   \end{cases}
   \\
p_{Y|X}\brak{0|0} = \frac{19}{25}\, 
p_{Y|X}\brak{0|1} = \frac{6}{25}\,
p_{Y|X}\brak{1|0} = \frac{45}{50}\,
p_{Y|X}\brak{1|2} = \frac{5}{50}
\end{align}
The desired probability is the probability that a slip drawn at random is marked other than Rs 1,
\begin{align}
&=1-p_X\brak{0}\\
&= p_X(1) + p_X(2)
\end{align}
Using Bayes theorem,
\begin{align}
&= p_Y\brak{0} \times \pr{Y=0 | X=1} + p_Y\brak{1} \times \pr{Y=1|X=2}\\
&=\frac{1}{3} \times \frac{6}{25} + \frac{2}{3} \times \frac{5}{50}\\
&=\frac{11}{75}
\end{align}

\newpage

%\tableofcontents

\bigskip

\renewcommand{\thefigure}{\theenumi}
\renewcommand{\thetable}{\theenumi}
%\renewcommand{\theequation}{\theenumi}

%\begin{abstract}
%%\boldmath
%In this letter, an algorithm for evaluating the exact analytical bit error rate  (BER)  for the piecewise linear (PL) combiner for  multiple relays is presented. Previous results were available only for upto three relays. The algorithm is unique in the sense that  the actual mathematical expressions, that are prohibitively large, need not be explicitly obtained. The diversity gain due to multiple relays is shown through plots of the analytical BER, well supported by simulations. 
%
%\end{abstract}
% IEEEtran.cls defaults to using nonbold math in the Abstract.
% This preserves the distinction between vectors and scalars. However,
% if the journal you are submitting to favors bold math in the abstract,
% then you can use LaTeX's standard command \boldmath at the very start
% of the abstract to achieve this. Many IEEE journals frown on math
% in the abstract anyway.

% Note that keywords are not normally used for peerreview papers.
%\begin{IEEEkeywords}
%Cooperative diversity, decode and forward, piecewise linear
%\end{IEEEkeywords}



% For peer review papers, you can put extra information on the cover
% page as needed:
% \ifCLASSOPTIONpeerreview
% \begin{center} \bfseries EDICS Category: 3-BBND \end{center}
% \fi
%
% For peerreview papers, this IEEEtran command inserts a page break and
% creates the second title. It will be ignored for other modes.
%\IEEEpeerreviewmaketitle




\item 
A die is thrown and a card is selected at random from a deck of 52 playing cards. The probability of getting an even number on the die and a spade card.\\
\solution
%\begin{table}[H]
	\centering
\begin{tabular}{|c|c|c|}
\hline
Random variable &Value &Definition\\ \hline
\multirow{3}{*}{X} &0 &Slips of Rs 1\\
&1 &Slips of Rs 5\\
&2 &Slips of Rs 13\\ \hline
\multirow{2}{*}{Y} &0 &Box A\\
&1 &Box B\\\hline
\end{tabular}
\caption{}
\label{tab:Distribution}
\end{table}
See \tabref{tab:Distribution}.
\begin{align}
p_{Y}\brak{k}= \begin{cases} 
      \frac{1}{3} & {k=0} \\
      \frac{2}{3 }& {k=1} 
   \end{cases}
   \\
p_{Y|X}\brak{0|0} = \frac{19}{25}\, 
p_{Y|X}\brak{0|1} = \frac{6}{25}\,
p_{Y|X}\brak{1|0} = \frac{45}{50}\,
p_{Y|X}\brak{1|2} = \frac{5}{50}
\end{align}
The desired probability is the probability that a slip drawn at random is marked other than Rs 1,
\begin{align}
&=1-p_X\brak{0}\\
&= p_X(1) + p_X(2)
\end{align}
Using Bayes theorem,
\begin{align}
&= p_Y\brak{0} \times \pr{Y=0 | X=1} + p_Y\brak{1} \times \pr{Y=1|X=2}\\
&=\frac{1}{3} \times \frac{6}{25} + \frac{2}{3} \times \frac{5}{50}\\
&=\frac{11}{75}
\end{align}

\newpage

%\tableofcontents

\bigskip

\renewcommand{\thefigure}{\theenumi}
\renewcommand{\thetable}{\theenumi}
%\renewcommand{\theequation}{\theenumi}

%\begin{abstract}
%%\boldmath
%In this letter, an algorithm for evaluating the exact analytical bit error rate  (BER)  for the piecewise linear (PL) combiner for  multiple relays is presented. Previous results were available only for upto three relays. The algorithm is unique in the sense that  the actual mathematical expressions, that are prohibitively large, need not be explicitly obtained. The diversity gain due to multiple relays is shown through plots of the analytical BER, well supported by simulations. 
%
%\end{abstract}
% IEEEtran.cls defaults to using nonbold math in the Abstract.
% This preserves the distinction between vectors and scalars. However,
% if the journal you are submitting to favors bold math in the abstract,
% then you can use LaTeX's standard command \boldmath at the very start
% of the abstract to achieve this. Many IEEE journals frown on math
% in the abstract anyway.

% Note that keywords are not normally used for peerreview papers.
%\begin{IEEEkeywords}
%Cooperative diversity, decode and forward, piecewise linear
%\end{IEEEkeywords}



% For peer review papers, you can put extra information on the cover
% page as needed:
% \ifCLASSOPTIONpeerreview
% \begin{center} \bfseries EDICS Category: 3-BBND \end{center}
% \fi
%
% For peerreview papers, this IEEEtran command inserts a page break and
% creates the second title. It will be ignored for other modes.
%\IEEEpeerreviewmaketitle




\item
If 4-digit numbers greater than 5,000 are randomly formed from the digits 0, 1, 3, 5, and 7, what is the probability of forming a number divisible by 5 when:
\begin{enumerate}
    \item The digits are repeated?
    \item The repetition of digits is not allowed?
\end{enumerate}
\solution
%\begin{table}[H]
	\centering
\begin{tabular}{|c|c|c|}
\hline
Random variable &Value &Definition\\ \hline
\multirow{3}{*}{X} &0 &Slips of Rs 1\\
&1 &Slips of Rs 5\\
&2 &Slips of Rs 13\\ \hline
\multirow{2}{*}{Y} &0 &Box A\\
&1 &Box B\\\hline
\end{tabular}
\caption{}
\label{tab:Distribution}
\end{table}
See \tabref{tab:Distribution}.
\begin{align}
p_{Y}\brak{k}= \begin{cases} 
      \frac{1}{3} & {k=0} \\
      \frac{2}{3 }& {k=1} 
   \end{cases}
   \\
p_{Y|X}\brak{0|0} = \frac{19}{25}\, 
p_{Y|X}\brak{0|1} = \frac{6}{25}\,
p_{Y|X}\brak{1|0} = \frac{45}{50}\,
p_{Y|X}\brak{1|2} = \frac{5}{50}
\end{align}
The desired probability is the probability that a slip drawn at random is marked other than Rs 1,
\begin{align}
&=1-p_X\brak{0}\\
&= p_X(1) + p_X(2)
\end{align}
Using Bayes theorem,
\begin{align}
&= p_Y\brak{0} \times \pr{Y=0 | X=1} + p_Y\brak{1} \times \pr{Y=1|X=2}\\
&=\frac{1}{3} \times \frac{6}{25} + \frac{2}{3} \times \frac{5}{50}\\
&=\frac{11}{75}
\end{align}

\newpage

%\tableofcontents

\bigskip

\renewcommand{\thefigure}{\theenumi}
\renewcommand{\thetable}{\theenumi}
%\renewcommand{\theequation}{\theenumi}

%\begin{abstract}
%%\boldmath
%In this letter, an algorithm for evaluating the exact analytical bit error rate  (BER)  for the piecewise linear (PL) combiner for  multiple relays is presented. Previous results were available only for upto three relays. The algorithm is unique in the sense that  the actual mathematical expressions, that are prohibitively large, need not be explicitly obtained. The diversity gain due to multiple relays is shown through plots of the analytical BER, well supported by simulations. 
%
%\end{abstract}
% IEEEtran.cls defaults to using nonbold math in the Abstract.
% This preserves the distinction between vectors and scalars. However,
% if the journal you are submitting to favors bold math in the abstract,
% then you can use LaTeX's standard command \boldmath at the very start
% of the abstract to achieve this. Many IEEE journals frown on math
% in the abstract anyway.

% Note that keywords are not normally used for peerreview papers.
%\begin{IEEEkeywords}
%Cooperative diversity, decode and forward, piecewise linear
%\end{IEEEkeywords}



% For peer review papers, you can put extra information on the cover
% page as needed:
% \ifCLASSOPTIONpeerreview
% \begin{center} \bfseries EDICS Category: 3-BBND \end{center}
% \fi
%
% For peerreview papers, this IEEEtran command inserts a page break and
% creates the second title. It will be ignored for other modes.
%\IEEEpeerreviewmaketitle




\item Consider the probability space $\brak{\Omega, \mathcal{G}, P}$ where $\Omega = [0,2]$ and $\mathcal{G} = \cbrak{\phi, \Omega, [0,1], (1,2]}$. Let $X$ and $Y$ be two functions on $\Omega$ defined as
\begin{align*}
    X(\omega) = 
    \begin{cases}
        1 & \text{if }\omega \in [0, 1]\\
        2 & \text{if }\omega \in (1, 2]
    \end{cases}
\end{align*}
and
\begin{align*}
    Y(\omega) = 
    \begin{cases}
        2 & \text{if }\omega \in [0, 1.5]\\
        3 & \text{if }\omega \in (1.5, 2].
    \end{cases}
\end{align*}
Then which one of the following statements is true?
\begin{enumerate}
    \item [(A)] $X$ is a random variable with respect to $\mathcal{G}$, but $Y$ is not a random variable with respect to $\mathcal{G}$.
    \item [(B)] $Y$ is a random variable with respect to $\mathcal{G}$, but $X$ is not a random variable with respect to $\mathcal{G}$.
    \item [(C)] Neither $X$ nor $Y$ is a random variable with respect to $\mathcal{G}$.
    \item [(D)] Both $X$ and $Y$ are random variables with respect to $\mathcal{G}$.
\end{enumerate} \hfill (GATE ST 2023)\\
\solution
%\begin{table}[H]
	\centering
\begin{tabular}{|c|c|c|}
\hline
Random variable &Value &Definition\\ \hline
\multirow{3}{*}{X} &0 &Slips of Rs 1\\
&1 &Slips of Rs 5\\
&2 &Slips of Rs 13\\ \hline
\multirow{2}{*}{Y} &0 &Box A\\
&1 &Box B\\\hline
\end{tabular}
\caption{}
\label{tab:Distribution}
\end{table}
See \tabref{tab:Distribution}.
\begin{align}
p_{Y}\brak{k}= \begin{cases} 
      \frac{1}{3} & {k=0} \\
      \frac{2}{3 }& {k=1} 
   \end{cases}
   \\
p_{Y|X}\brak{0|0} = \frac{19}{25}\, 
p_{Y|X}\brak{0|1} = \frac{6}{25}\,
p_{Y|X}\brak{1|0} = \frac{45}{50}\,
p_{Y|X}\brak{1|2} = \frac{5}{50}
\end{align}
The desired probability is the probability that a slip drawn at random is marked other than Rs 1,
\begin{align}
&=1-p_X\brak{0}\\
&= p_X(1) + p_X(2)
\end{align}
Using Bayes theorem,
\begin{align}
&= p_Y\brak{0} \times \pr{Y=0 | X=1} + p_Y\brak{1} \times \pr{Y=1|X=2}\\
&=\frac{1}{3} \times \frac{6}{25} + \frac{2}{3} \times \frac{5}{50}\\
&=\frac{11}{75}
\end{align}

\newpage

%\tableofcontents

\bigskip

\renewcommand{\thefigure}{\theenumi}
\renewcommand{\thetable}{\theenumi}
%\renewcommand{\theequation}{\theenumi}

%\begin{abstract}
%%\boldmath
%In this letter, an algorithm for evaluating the exact analytical bit error rate  (BER)  for the piecewise linear (PL) combiner for  multiple relays is presented. Previous results were available only for upto three relays. The algorithm is unique in the sense that  the actual mathematical expressions, that are prohibitively large, need not be explicitly obtained. The diversity gain due to multiple relays is shown through plots of the analytical BER, well supported by simulations. 
%
%\end{abstract}
% IEEEtran.cls defaults to using nonbold math in the Abstract.
% This preserves the distinction between vectors and scalars. However,
% if the journal you are submitting to favors bold math in the abstract,
% then you can use LaTeX's standard command \boldmath at the very start
% of the abstract to achieve this. Many IEEE journals frown on math
% in the abstract anyway.

% Note that keywords are not normally used for peerreview papers.
%\begin{IEEEkeywords}
%Cooperative diversity, decode and forward, piecewise linear
%\end{IEEEkeywords}



% For peer review papers, you can put extra information on the cover
% page as needed:
% \ifCLASSOPTIONpeerreview
% \begin{center} \bfseries EDICS Category: 3-BBND \end{center}
% \fi
%
% For peerreview papers, this IEEEtran command inserts a page break and
% creates the second title. It will be ignored for other modes.
%\IEEEpeerreviewmaketitle




	\item  A die is loaded in such a way that each odd number is twice as likely to occur as
each even number. Find $P(G)$, where $G$ is the event that a number greater than
3 occurs on a single roll of the die.
\\
\solution
		%\begin{table}[H]
	\centering
\begin{tabular}{|c|c|c|}
\hline
Random variable &Value &Definition\\ \hline
\multirow{3}{*}{X} &0 &Slips of Rs 1\\
&1 &Slips of Rs 5\\
&2 &Slips of Rs 13\\ \hline
\multirow{2}{*}{Y} &0 &Box A\\
&1 &Box B\\\hline
\end{tabular}
\caption{}
\label{tab:Distribution}
\end{table}
See \tabref{tab:Distribution}.
\begin{align}
p_{Y}\brak{k}= \begin{cases} 
      \frac{1}{3} & {k=0} \\
      \frac{2}{3 }& {k=1} 
   \end{cases}
   \\
p_{Y|X}\brak{0|0} = \frac{19}{25}\, 
p_{Y|X}\brak{0|1} = \frac{6}{25}\,
p_{Y|X}\brak{1|0} = \frac{45}{50}\,
p_{Y|X}\brak{1|2} = \frac{5}{50}
\end{align}
The desired probability is the probability that a slip drawn at random is marked other than Rs 1,
\begin{align}
&=1-p_X\brak{0}\\
&= p_X(1) + p_X(2)
\end{align}
Using Bayes theorem,
\begin{align}
&= p_Y\brak{0} \times \pr{Y=0 | X=1} + p_Y\brak{1} \times \pr{Y=1|X=2}\\
&=\frac{1}{3} \times \frac{6}{25} + \frac{2}{3} \times \frac{5}{50}\\
&=\frac{11}{75}
\end{align}

\newpage

%\tableofcontents

\bigskip

\renewcommand{\thefigure}{\theenumi}
\renewcommand{\thetable}{\theenumi}
%\renewcommand{\theequation}{\theenumi}

%\begin{abstract}
%%\boldmath
%In this letter, an algorithm for evaluating the exact analytical bit error rate  (BER)  for the piecewise linear (PL) combiner for  multiple relays is presented. Previous results were available only for upto three relays. The algorithm is unique in the sense that  the actual mathematical expressions, that are prohibitively large, need not be explicitly obtained. The diversity gain due to multiple relays is shown through plots of the analytical BER, well supported by simulations. 
%
%\end{abstract}
% IEEEtran.cls defaults to using nonbold math in the Abstract.
% This preserves the distinction between vectors and scalars. However,
% if the journal you are submitting to favors bold math in the abstract,
% then you can use LaTeX's standard command \boldmath at the very start
% of the abstract to achieve this. Many IEEE journals frown on math
% in the abstract anyway.

% Note that keywords are not normally used for peerreview papers.
%\begin{IEEEkeywords}
%Cooperative diversity, decode and forward, piecewise linear
%\end{IEEEkeywords}



% For peer review papers, you can put extra information on the cover
% page as needed:
% \ifCLASSOPTIONpeerreview
% \begin{center} \bfseries EDICS Category: 3-BBND \end{center}
% \fi
%
% For peerreview papers, this IEEEtran command inserts a page break and
% creates the second title. It will be ignored for other modes.
%\IEEEpeerreviewmaketitle




	\item All the jacks, queens and kings are removed from a deck of 52 playing cards. The remaining cards are well shuffled and then one card is drawn at random. Giving ace a value 1 similar value for other cards, find the probability that the card has a value 
		\begin{enumerate}
			\item 7
			\item greater than 7
			\item less than 7
		\end{enumerate}
		%Number of cards left after removing all jacks, queens and kings 
\begin{align}
N	= 52 - 4\times 3
	= 40
\end{align}
%\begin{table}[H]
%\def\arraystretch{1.2}
%\begin{tabular}{|c|c|c|}
%\hline
%	\textbf{Parameter} &\textbf{Value} &\textbf{Description}\\ \hline
%	$X$ &1-10 &Represents the value of the card picked \\ \hline
%\end{tabular}
%\end{table}
Let $1 \le X \le 10$ be the value of the card picked.  Then,
\begin{align}
	p_X(k) &= \Pr(X=k)\ \forall\ 1 \leq k \leq 10\\
	&= \frac{4\times 1}{40}\\
	&= \frac{1}{10}\\
	\therefore p_X(k) &= 
	\begin{cases}
		\frac{1}{10} & 1 \leq k \leq 10\\
		0 & \text{otherwise}
	\end{cases}
\end{align}
and
\begin{align}
	F_{X}(k) &= \sum_{m=0}^{k}p_{X}(m) \quad 1 \leq k \leq 10\\
	&= \frac{k}{10}\\
	\therefore F_{X}(k) &= 
	\begin{cases}
		0 & k \leq 0\\
		\frac{k}{10} & 1\leq k \leq 10\\
		1 & k > 10 
	\end{cases}
\end{align}
\begin{enumerate}
	\item Probability that card has value equal to 7 is
		\begin{align}
			 p_{X}(7)
			= \frac{1}{10}
		\end{align}
	\item Probability that card has value greater than 7 is
		\begin{align}
			1 - F_X(7)
			&= 1 - \frac{7}{10}
			\\
			&= \frac{3}{10}
		\end{align}
	\item Probability that card has value less than 7 is
		\begin{align}
			 F_{X}(6)
			=\frac{6}{10}
		\end{align}
\end{enumerate}

  \item A Lot consists of 48 mobile phones of which 42 are good, 3 have only minor defects and 3 have major defects.Varnika will buy a phone if it is good but the trader will only buy a mobile if it has no major defects. One phone is selected at random from the lot. What is the probability that it is
\begin{enumerate}
	\item acceptable to Varnika?
            \item acceptable to the trader?
\end{enumerate}
\solution
	%\begin{table}[H]
	\centering
\begin{tabular}{|c|c|c|}
\hline
Random variable &Value &Definition\\ \hline
\multirow{3}{*}{X} &0 &Slips of Rs 1\\
&1 &Slips of Rs 5\\
&2 &Slips of Rs 13\\ \hline
\multirow{2}{*}{Y} &0 &Box A\\
&1 &Box B\\\hline
\end{tabular}
\caption{}
\label{tab:Distribution}
\end{table}
See \tabref{tab:Distribution}.
\begin{align}
p_{Y}\brak{k}= \begin{cases} 
      \frac{1}{3} & {k=0} \\
      \frac{2}{3 }& {k=1} 
   \end{cases}
   \\
p_{Y|X}\brak{0|0} = \frac{19}{25}\, 
p_{Y|X}\brak{0|1} = \frac{6}{25}\,
p_{Y|X}\brak{1|0} = \frac{45}{50}\,
p_{Y|X}\brak{1|2} = \frac{5}{50}
\end{align}
The desired probability is the probability that a slip drawn at random is marked other than Rs 1,
\begin{align}
&=1-p_X\brak{0}\\
&= p_X(1) + p_X(2)
\end{align}
Using Bayes theorem,
\begin{align}
&= p_Y\brak{0} \times \pr{Y=0 | X=1} + p_Y\brak{1} \times \pr{Y=1|X=2}\\
&=\frac{1}{3} \times \frac{6}{25} + \frac{2}{3} \times \frac{5}{50}\\
&=\frac{11}{75}
\end{align}

\newpage

%\tableofcontents

\bigskip

\renewcommand{\thefigure}{\theenumi}
\renewcommand{\thetable}{\theenumi}
%\renewcommand{\theequation}{\theenumi}

%\begin{abstract}
%%\boldmath
%In this letter, an algorithm for evaluating the exact analytical bit error rate  (BER)  for the piecewise linear (PL) combiner for  multiple relays is presented. Previous results were available only for upto three relays. The algorithm is unique in the sense that  the actual mathematical expressions, that are prohibitively large, need not be explicitly obtained. The diversity gain due to multiple relays is shown through plots of the analytical BER, well supported by simulations. 
%
%\end{abstract}
% IEEEtran.cls defaults to using nonbold math in the Abstract.
% This preserves the distinction between vectors and scalars. However,
% if the journal you are submitting to favors bold math in the abstract,
% then you can use LaTeX's standard command \boldmath at the very start
% of the abstract to achieve this. Many IEEE journals frown on math
% in the abstract anyway.

% Note that keywords are not normally used for peerreview papers.
%\begin{IEEEkeywords}
%Cooperative diversity, decode and forward, piecewise linear
%\end{IEEEkeywords}



% For peer review papers, you can put extra information on the cover
% page as needed:
% \ifCLASSOPTIONpeerreview
% \begin{center} \bfseries EDICS Category: 3-BBND \end{center}
% \fi
%
% For peerreview papers, this IEEEtran command inserts a page break and
% creates the second title. It will be ignored for other modes.
%\IEEEpeerreviewmaketitle




 \item A student says that if you throw a die, it will show up 1 or not 1. Therefore, the probability of getting 1 and the probability of getting 'not 1' each is equal to $\frac{1}{2}$. Is this correct? Give reasons.\\
 \solution
        %\begin{table}[H]
	\centering
\begin{tabular}{|c|c|c|}
\hline
Random variable &Value &Definition\\ \hline
\multirow{3}{*}{X} &0 &Slips of Rs 1\\
&1 &Slips of Rs 5\\
&2 &Slips of Rs 13\\ \hline
\multirow{2}{*}{Y} &0 &Box A\\
&1 &Box B\\\hline
\end{tabular}
\caption{}
\label{tab:Distribution}
\end{table}
See \tabref{tab:Distribution}.
\begin{align}
p_{Y}\brak{k}= \begin{cases} 
      \frac{1}{3} & {k=0} \\
      \frac{2}{3 }& {k=1} 
   \end{cases}
   \\
p_{Y|X}\brak{0|0} = \frac{19}{25}\, 
p_{Y|X}\brak{0|1} = \frac{6}{25}\,
p_{Y|X}\brak{1|0} = \frac{45}{50}\,
p_{Y|X}\brak{1|2} = \frac{5}{50}
\end{align}
The desired probability is the probability that a slip drawn at random is marked other than Rs 1,
\begin{align}
&=1-p_X\brak{0}\\
&= p_X(1) + p_X(2)
\end{align}
Using Bayes theorem,
\begin{align}
&= p_Y\brak{0} \times \pr{Y=0 | X=1} + p_Y\brak{1} \times \pr{Y=1|X=2}\\
&=\frac{1}{3} \times \frac{6}{25} + \frac{2}{3} \times \frac{5}{50}\\
&=\frac{11}{75}
\end{align}

\newpage

%\tableofcontents

\bigskip

\renewcommand{\thefigure}{\theenumi}
\renewcommand{\thetable}{\theenumi}
%\renewcommand{\theequation}{\theenumi}

%\begin{abstract}
%%\boldmath
%In this letter, an algorithm for evaluating the exact analytical bit error rate  (BER)  for the piecewise linear (PL) combiner for  multiple relays is presented. Previous results were available only for upto three relays. The algorithm is unique in the sense that  the actual mathematical expressions, that are prohibitively large, need not be explicitly obtained. The diversity gain due to multiple relays is shown through plots of the analytical BER, well supported by simulations. 
%
%\end{abstract}
% IEEEtran.cls defaults to using nonbold math in the Abstract.
% This preserves the distinction between vectors and scalars. However,
% if the journal you are submitting to favors bold math in the abstract,
% then you can use LaTeX's standard command \boldmath at the very start
% of the abstract to achieve this. Many IEEE journals frown on math
% in the abstract anyway.

% Note that keywords are not normally used for peerreview papers.
%\begin{IEEEkeywords}
%Cooperative diversity, decode and forward, piecewise linear
%\end{IEEEkeywords}



% For peer review papers, you can put extra information on the cover
% page as needed:
% \ifCLASSOPTIONpeerreview
% \begin{center} \bfseries EDICS Category: 3-BBND \end{center}
% \fi
%
% For peerreview papers, this IEEEtran command inserts a page break and
% creates the second title. It will be ignored for other modes.
%\IEEEpeerreviewmaketitle




   \item Four candidates A, B, C, D have ap-
plied for the assignment to coach a school cricket
team. If A is twice as likely to be selected as B, and
B and C are given about the same chance of being
selected, while C is twice as likely to be selected
as D, what are the probabilities that
\begin{enumerate}
\item C will be selected?
\item A will not be selected?
\end{enumerate}
	%\begin{table}[H]
	\centering
\begin{tabular}{|c|c|c|}
\hline
Random variable &Value &Definition\\ \hline
\multirow{3}{*}{X} &0 &Slips of Rs 1\\
&1 &Slips of Rs 5\\
&2 &Slips of Rs 13\\ \hline
\multirow{2}{*}{Y} &0 &Box A\\
&1 &Box B\\\hline
\end{tabular}
\caption{}
\label{tab:Distribution}
\end{table}
See \tabref{tab:Distribution}.
\begin{align}
p_{Y}\brak{k}= \begin{cases} 
      \frac{1}{3} & {k=0} \\
      \frac{2}{3 }& {k=1} 
   \end{cases}
   \\
p_{Y|X}\brak{0|0} = \frac{19}{25}\, 
p_{Y|X}\brak{0|1} = \frac{6}{25}\,
p_{Y|X}\brak{1|0} = \frac{45}{50}\,
p_{Y|X}\brak{1|2} = \frac{5}{50}
\end{align}
The desired probability is the probability that a slip drawn at random is marked other than Rs 1,
\begin{align}
&=1-p_X\brak{0}\\
&= p_X(1) + p_X(2)
\end{align}
Using Bayes theorem,
\begin{align}
&= p_Y\brak{0} \times \pr{Y=0 | X=1} + p_Y\brak{1} \times \pr{Y=1|X=2}\\
&=\frac{1}{3} \times \frac{6}{25} + \frac{2}{3} \times \frac{5}{50}\\
&=\frac{11}{75}
\end{align}

\newpage

%\tableofcontents

\bigskip

\renewcommand{\thefigure}{\theenumi}
\renewcommand{\thetable}{\theenumi}
%\renewcommand{\theequation}{\theenumi}

%\begin{abstract}
%%\boldmath
%In this letter, an algorithm for evaluating the exact analytical bit error rate  (BER)  for the piecewise linear (PL) combiner for  multiple relays is presented. Previous results were available only for upto three relays. The algorithm is unique in the sense that  the actual mathematical expressions, that are prohibitively large, need not be explicitly obtained. The diversity gain due to multiple relays is shown through plots of the analytical BER, well supported by simulations. 
%
%\end{abstract}
% IEEEtran.cls defaults to using nonbold math in the Abstract.
% This preserves the distinction between vectors and scalars. However,
% if the journal you are submitting to favors bold math in the abstract,
% then you can use LaTeX's standard command \boldmath at the very start
% of the abstract to achieve this. Many IEEE journals frown on math
% in the abstract anyway.

% Note that keywords are not normally used for peerreview papers.
%\begin{IEEEkeywords}
%Cooperative diversity, decode and forward, piecewise linear
%\end{IEEEkeywords}



% For peer review papers, you can put extra information on the cover
% page as needed:
% \ifCLASSOPTIONpeerreview
% \begin{center} \bfseries EDICS Category: 3-BBND \end{center}
% \fi
%
% For peerreview papers, this IEEEtran command inserts a page break and
% creates the second title. It will be ignored for other modes.
%\IEEEpeerreviewmaketitle




 \item A bag contain 24 balls of which $x$ balls are red, $2x$ are white and $3x$ are blue. A ball is selected at random, What is the probability that it is
\begin{enumerate}[label=\alph*)]
\item not red ?
\item white ?
\end{enumerate}
%\begin{table}[H]
	\centering
\begin{tabular}{|c|c|c|}
\hline
Random variable &Value &Definition\\ \hline
\multirow{3}{*}{X} &0 &Slips of Rs 1\\
&1 &Slips of Rs 5\\
&2 &Slips of Rs 13\\ \hline
\multirow{2}{*}{Y} &0 &Box A\\
&1 &Box B\\\hline
\end{tabular}
\caption{}
\label{tab:Distribution}
\end{table}
See \tabref{tab:Distribution}.
\begin{align}
p_{Y}\brak{k}= \begin{cases} 
      \frac{1}{3} & {k=0} \\
      \frac{2}{3 }& {k=1} 
   \end{cases}
   \\
p_{Y|X}\brak{0|0} = \frac{19}{25}\, 
p_{Y|X}\brak{0|1} = \frac{6}{25}\,
p_{Y|X}\brak{1|0} = \frac{45}{50}\,
p_{Y|X}\brak{1|2} = \frac{5}{50}
\end{align}
The desired probability is the probability that a slip drawn at random is marked other than Rs 1,
\begin{align}
&=1-p_X\brak{0}\\
&= p_X(1) + p_X(2)
\end{align}
Using Bayes theorem,
\begin{align}
&= p_Y\brak{0} \times \pr{Y=0 | X=1} + p_Y\brak{1} \times \pr{Y=1|X=2}\\
&=\frac{1}{3} \times \frac{6}{25} + \frac{2}{3} \times \frac{5}{50}\\
&=\frac{11}{75}
\end{align}

\newpage

%\tableofcontents

\bigskip

\renewcommand{\thefigure}{\theenumi}
\renewcommand{\thetable}{\theenumi}
%\renewcommand{\theequation}{\theenumi}

%\begin{abstract}
%%\boldmath
%In this letter, an algorithm for evaluating the exact analytical bit error rate  (BER)  for the piecewise linear (PL) combiner for  multiple relays is presented. Previous results were available only for upto three relays. The algorithm is unique in the sense that  the actual mathematical expressions, that are prohibitively large, need not be explicitly obtained. The diversity gain due to multiple relays is shown through plots of the analytical BER, well supported by simulations. 
%
%\end{abstract}
% IEEEtran.cls defaults to using nonbold math in the Abstract.
% This preserves the distinction between vectors and scalars. However,
% if the journal you are submitting to favors bold math in the abstract,
% then you can use LaTeX's standard command \boldmath at the very start
% of the abstract to achieve this. Many IEEE journals frown on math
% in the abstract anyway.

% Note that keywords are not normally used for peerreview papers.
%\begin{IEEEkeywords}
%Cooperative diversity, decode and forward, piecewise linear
%\end{IEEEkeywords}



% For peer review papers, you can put extra information on the cover
% page as needed:
% \ifCLASSOPTIONpeerreview
% \begin{center} \bfseries EDICS Category: 3-BBND \end{center}
% \fi
%
% For peerreview papers, this IEEEtran command inserts a page break and
% creates the second title. It will be ignored for other modes.
%\IEEEpeerreviewmaketitle




If the letters of the word ASSASSINATION are arranged at random. Find the Probability that
\begin{enumerate}[label=(\alph*)]
\item Four $S's$ come consecutively in the word
\item Two  $I's$ and two $N's$ come together
\item All $A's$ are not coming together
\item No two $A's$ are coming together
\end{enumerate}
%\begin{table}[H]
	\centering
\begin{tabular}{|c|c|c|}
\hline
Random variable &Value &Definition\\ \hline
\multirow{3}{*}{X} &0 &Slips of Rs 1\\
&1 &Slips of Rs 5\\
&2 &Slips of Rs 13\\ \hline
\multirow{2}{*}{Y} &0 &Box A\\
&1 &Box B\\\hline
\end{tabular}
\caption{}
\label{tab:Distribution}
\end{table}
See \tabref{tab:Distribution}.
\begin{align}
p_{Y}\brak{k}= \begin{cases} 
      \frac{1}{3} & {k=0} \\
      \frac{2}{3 }& {k=1} 
   \end{cases}
   \\
p_{Y|X}\brak{0|0} = \frac{19}{25}\, 
p_{Y|X}\brak{0|1} = \frac{6}{25}\,
p_{Y|X}\brak{1|0} = \frac{45}{50}\,
p_{Y|X}\brak{1|2} = \frac{5}{50}
\end{align}
The desired probability is the probability that a slip drawn at random is marked other than Rs 1,
\begin{align}
&=1-p_X\brak{0}\\
&= p_X(1) + p_X(2)
\end{align}
Using Bayes theorem,
\begin{align}
&= p_Y\brak{0} \times \pr{Y=0 | X=1} + p_Y\brak{1} \times \pr{Y=1|X=2}\\
&=\frac{1}{3} \times \frac{6}{25} + \frac{2}{3} \times \frac{5}{50}\\
&=\frac{11}{75}
\end{align}

\newpage

%\tableofcontents

\bigskip

\renewcommand{\thefigure}{\theenumi}
\renewcommand{\thetable}{\theenumi}
%\renewcommand{\theequation}{\theenumi}

%\begin{abstract}
%%\boldmath
%In this letter, an algorithm for evaluating the exact analytical bit error rate  (BER)  for the piecewise linear (PL) combiner for  multiple relays is presented. Previous results were available only for upto three relays. The algorithm is unique in the sense that  the actual mathematical expressions, that are prohibitively large, need not be explicitly obtained. The diversity gain due to multiple relays is shown through plots of the analytical BER, well supported by simulations. 
%
%\end{abstract}
% IEEEtran.cls defaults to using nonbold math in the Abstract.
% This preserves the distinction between vectors and scalars. However,
% if the journal you are submitting to favors bold math in the abstract,
% then you can use LaTeX's standard command \boldmath at the very start
% of the abstract to achieve this. Many IEEE journals frown on math
% in the abstract anyway.

% Note that keywords are not normally used for peerreview papers.
%\begin{IEEEkeywords}
%Cooperative diversity, decode and forward, piecewise linear
%\end{IEEEkeywords}



% For peer review papers, you can put extra information on the cover
% page as needed:
% \ifCLASSOPTIONpeerreview
% \begin{center} \bfseries EDICS Category: 3-BBND \end{center}
% \fi
%
% For peerreview papers, this IEEEtran command inserts a page break and
% creates the second title. It will be ignored for other modes.
%\IEEEpeerreviewmaketitle




	\item One urn contains two black balls (labelled B1 and B2) and one white ball. A
	second urn contains one black ball and two white balls (labelled W1 and W2).
	Suppose the following experiment is performed. One of the two urns is chosen
	at random. Next a ball is randomly chosen from the urn. Then a second ball is
	chosen at random from the same urn without replacing the first ball.
	
	\begin{enumerate}
	\item What is the probability that two black balls are chosen?
	
	\item What is the probability that two balls of opposite colour are chosen?
	\end{enumerate}
	\solution
	%\begin{align}
    \label{eq:12.13.6.18.1}
	\because	\pr{A|B} &> \pr{A},\
\frac{\pr{AB}}{\pr{B}} > \pr{A}
\\
    \label{eq:12.13.6.18.2}
	\implies \pr{AB} &> \pr{A}\pr{B}
	\\
	\text{or, } \frac{\pr{AB}}{\pr{A}} &=\pr{B|A} > \pr{A}
\end{align}

\end{enumerate}

	\item A bag contains 4 red and 4 black balls, another bag contains 2 red and 6 black balls. One of the two bags is selected at random and a ball is drawn from the bag which is found to be red. Find the probability that the ball is drawn from the first bag.
\\
\solution
		%\begin{table}[H]
	\centering
\begin{tabular}{|c|c|c|}
\hline
Random variable &Value &Definition\\ \hline
\multirow{3}{*}{X} &0 &Slips of Rs 1\\
&1 &Slips of Rs 5\\
&2 &Slips of Rs 13\\ \hline
\multirow{2}{*}{Y} &0 &Box A\\
&1 &Box B\\\hline
\end{tabular}
\caption{}
\label{tab:Distribution}
\end{table}
See \tabref{tab:Distribution}.
\begin{align}
p_{Y}\brak{k}= \begin{cases} 
      \frac{1}{3} & {k=0} \\
      \frac{2}{3 }& {k=1} 
   \end{cases}
   \\
p_{Y|X}\brak{0|0} = \frac{19}{25}\, 
p_{Y|X}\brak{0|1} = \frac{6}{25}\,
p_{Y|X}\brak{1|0} = \frac{45}{50}\,
p_{Y|X}\brak{1|2} = \frac{5}{50}
\end{align}
The desired probability is the probability that a slip drawn at random is marked other than Rs 1,
\begin{align}
&=1-p_X\brak{0}\\
&= p_X(1) + p_X(2)
\end{align}
Using Bayes theorem,
\begin{align}
&= p_Y\brak{0} \times \pr{Y=0 | X=1} + p_Y\brak{1} \times \pr{Y=1|X=2}\\
&=\frac{1}{3} \times \frac{6}{25} + \frac{2}{3} \times \frac{5}{50}\\
&=\frac{11}{75}
\end{align}

\newpage

%\tableofcontents

\bigskip

\renewcommand{\thefigure}{\theenumi}
\renewcommand{\thetable}{\theenumi}
%\renewcommand{\theequation}{\theenumi}

%\begin{abstract}
%%\boldmath
%In this letter, an algorithm for evaluating the exact analytical bit error rate  (BER)  for the piecewise linear (PL) combiner for  multiple relays is presented. Previous results were available only for upto three relays. The algorithm is unique in the sense that  the actual mathematical expressions, that are prohibitively large, need not be explicitly obtained. The diversity gain due to multiple relays is shown through plots of the analytical BER, well supported by simulations. 
%
%\end{abstract}
% IEEEtran.cls defaults to using nonbold math in the Abstract.
% This preserves the distinction between vectors and scalars. However,
% if the journal you are submitting to favors bold math in the abstract,
% then you can use LaTeX's standard command \boldmath at the very start
% of the abstract to achieve this. Many IEEE journals frown on math
% in the abstract anyway.

% Note that keywords are not normally used for peerreview papers.
%\begin{IEEEkeywords}
%Cooperative diversity, decode and forward, piecewise linear
%\end{IEEEkeywords}



% For peer review papers, you can put extra information on the cover
% page as needed:
% \ifCLASSOPTIONpeerreview
% \begin{center} \bfseries EDICS Category: 3-BBND \end{center}
% \fi
%
% For peerreview papers, this IEEEtran command inserts a page break and
% creates the second title. It will be ignored for other modes.
%\IEEEpeerreviewmaketitle




  \item
  Cards with numbers 2 to 101 are placed in a box. A card is selected at random.Find the probability that the card has
\begin{enumerate}[label=(\roman*)]
	\item an even number 
	\item a square number
\end{enumerate}
\solution
%\begin{table}[H]
	\centering
\begin{tabular}{|c|c|c|}
\hline
Random variable &Value &Definition\\ \hline
\multirow{3}{*}{X} &0 &Slips of Rs 1\\
&1 &Slips of Rs 5\\
&2 &Slips of Rs 13\\ \hline
\multirow{2}{*}{Y} &0 &Box A\\
&1 &Box B\\\hline
\end{tabular}
\caption{}
\label{tab:Distribution}
\end{table}
See \tabref{tab:Distribution}.
\begin{align}
p_{Y}\brak{k}= \begin{cases} 
      \frac{1}{3} & {k=0} \\
      \frac{2}{3 }& {k=1} 
   \end{cases}
   \\
p_{Y|X}\brak{0|0} = \frac{19}{25}\, 
p_{Y|X}\brak{0|1} = \frac{6}{25}\,
p_{Y|X}\brak{1|0} = \frac{45}{50}\,
p_{Y|X}\brak{1|2} = \frac{5}{50}
\end{align}
The desired probability is the probability that a slip drawn at random is marked other than Rs 1,
\begin{align}
&=1-p_X\brak{0}\\
&= p_X(1) + p_X(2)
\end{align}
Using Bayes theorem,
\begin{align}
&= p_Y\brak{0} \times \pr{Y=0 | X=1} + p_Y\brak{1} \times \pr{Y=1|X=2}\\
&=\frac{1}{3} \times \frac{6}{25} + \frac{2}{3} \times \frac{5}{50}\\
&=\frac{11}{75}
\end{align}

\newpage

%\tableofcontents

\bigskip

\renewcommand{\thefigure}{\theenumi}
\renewcommand{\thetable}{\theenumi}
%\renewcommand{\theequation}{\theenumi}

%\begin{abstract}
%%\boldmath
%In this letter, an algorithm for evaluating the exact analytical bit error rate  (BER)  for the piecewise linear (PL) combiner for  multiple relays is presented. Previous results were available only for upto three relays. The algorithm is unique in the sense that  the actual mathematical expressions, that are prohibitively large, need not be explicitly obtained. The diversity gain due to multiple relays is shown through plots of the analytical BER, well supported by simulations. 
%
%\end{abstract}
% IEEEtran.cls defaults to using nonbold math in the Abstract.
% This preserves the distinction between vectors and scalars. However,
% if the journal you are submitting to favors bold math in the abstract,
% then you can use LaTeX's standard command \boldmath at the very start
% of the abstract to achieve this. Many IEEE journals frown on math
% in the abstract anyway.

% Note that keywords are not normally used for peerreview papers.
%\begin{IEEEkeywords}
%Cooperative diversity, decode and forward, piecewise linear
%\end{IEEEkeywords}



% For peer review papers, you can put extra information on the cover
% page as needed:
% \ifCLASSOPTIONpeerreview
% \begin{center} \bfseries EDICS Category: 3-BBND \end{center}
% \fi
%
% For peerreview papers, this IEEEtran command inserts a page break and
% creates the second title. It will be ignored for other modes.
%\IEEEpeerreviewmaketitle




\item
The king, queen and jack of clubs are removed from a deck of 52 playing cards and then well shuffled. Now one card is drawn at random from the remaining cards.  Determine the probability that the card is
\begin{enumerate}[label=(\roman*)]
\item a club
\item 10 of hearts
\end{enumerate}
\solution
%\begin{table}[H]
	\centering
\begin{tabular}{|c|c|c|}
\hline
Random variable &Value &Definition\\ \hline
\multirow{3}{*}{X} &0 &Slips of Rs 1\\
&1 &Slips of Rs 5\\
&2 &Slips of Rs 13\\ \hline
\multirow{2}{*}{Y} &0 &Box A\\
&1 &Box B\\\hline
\end{tabular}
\caption{}
\label{tab:Distribution}
\end{table}
See \tabref{tab:Distribution}.
\begin{align}
p_{Y}\brak{k}= \begin{cases} 
      \frac{1}{3} & {k=0} \\
      \frac{2}{3 }& {k=1} 
   \end{cases}
   \\
p_{Y|X}\brak{0|0} = \frac{19}{25}\, 
p_{Y|X}\brak{0|1} = \frac{6}{25}\,
p_{Y|X}\brak{1|0} = \frac{45}{50}\,
p_{Y|X}\brak{1|2} = \frac{5}{50}
\end{align}
The desired probability is the probability that a slip drawn at random is marked other than Rs 1,
\begin{align}
&=1-p_X\brak{0}\\
&= p_X(1) + p_X(2)
\end{align}
Using Bayes theorem,
\begin{align}
&= p_Y\brak{0} \times \pr{Y=0 | X=1} + p_Y\brak{1} \times \pr{Y=1|X=2}\\
&=\frac{1}{3} \times \frac{6}{25} + \frac{2}{3} \times \frac{5}{50}\\
&=\frac{11}{75}
\end{align}

\newpage

%\tableofcontents

\bigskip

\renewcommand{\thefigure}{\theenumi}
\renewcommand{\thetable}{\theenumi}
%\renewcommand{\theequation}{\theenumi}

%\begin{abstract}
%%\boldmath
%In this letter, an algorithm for evaluating the exact analytical bit error rate  (BER)  for the piecewise linear (PL) combiner for  multiple relays is presented. Previous results were available only for upto three relays. The algorithm is unique in the sense that  the actual mathematical expressions, that are prohibitively large, need not be explicitly obtained. The diversity gain due to multiple relays is shown through plots of the analytical BER, well supported by simulations. 
%
%\end{abstract}
% IEEEtran.cls defaults to using nonbold math in the Abstract.
% This preserves the distinction between vectors and scalars. However,
% if the journal you are submitting to favors bold math in the abstract,
% then you can use LaTeX's standard command \boldmath at the very start
% of the abstract to achieve this. Many IEEE journals frown on math
% in the abstract anyway.

% Note that keywords are not normally used for peerreview papers.
%\begin{IEEEkeywords}
%Cooperative diversity, decode and forward, piecewise linear
%\end{IEEEkeywords}



% For peer review papers, you can put extra information on the cover
% page as needed:
% \ifCLASSOPTIONpeerreview
% \begin{center} \bfseries EDICS Category: 3-BBND \end{center}
% \fi
%
% For peerreview papers, this IEEEtran command inserts a page break and
% creates the second title. It will be ignored for other modes.
%\IEEEpeerreviewmaketitle




\item A team of medical students doing their internship have to assist during surgeries
at a city hospital. The probabilities of surgeries rated as very complex, complex,
routine, simple or very simple are respectively, 0.15, 0.20, 0.31, 0.26, .08. Find
the probabilities that a particular surgery will be rated
\begin{enumerate}
	\item complex or very complex;
	\item neither very complex nor very simple;
	\item routine or complex
	\item routine or simple
\end{enumerate}
\solution
%\begin{table}[H]
	\centering
\begin{tabular}{|c|c|c|}
\hline
Random variable &Value &Definition\\ \hline
\multirow{3}{*}{X} &0 &Slips of Rs 1\\
&1 &Slips of Rs 5\\
&2 &Slips of Rs 13\\ \hline
\multirow{2}{*}{Y} &0 &Box A\\
&1 &Box B\\\hline
\end{tabular}
\caption{}
\label{tab:Distribution}
\end{table}
See \tabref{tab:Distribution}.
\begin{align}
p_{Y}\brak{k}= \begin{cases} 
      \frac{1}{3} & {k=0} \\
      \frac{2}{3 }& {k=1} 
   \end{cases}
   \\
p_{Y|X}\brak{0|0} = \frac{19}{25}\, 
p_{Y|X}\brak{0|1} = \frac{6}{25}\,
p_{Y|X}\brak{1|0} = \frac{45}{50}\,
p_{Y|X}\brak{1|2} = \frac{5}{50}
\end{align}
The desired probability is the probability that a slip drawn at random is marked other than Rs 1,
\begin{align}
&=1-p_X\brak{0}\\
&= p_X(1) + p_X(2)
\end{align}
Using Bayes theorem,
\begin{align}
&= p_Y\brak{0} \times \pr{Y=0 | X=1} + p_Y\brak{1} \times \pr{Y=1|X=2}\\
&=\frac{1}{3} \times \frac{6}{25} + \frac{2}{3} \times \frac{5}{50}\\
&=\frac{11}{75}
\end{align}

\newpage

%\tableofcontents

\bigskip

\renewcommand{\thefigure}{\theenumi}
\renewcommand{\thetable}{\theenumi}
%\renewcommand{\theequation}{\theenumi}

%\begin{abstract}
%%\boldmath
%In this letter, an algorithm for evaluating the exact analytical bit error rate  (BER)  for the piecewise linear (PL) combiner for  multiple relays is presented. Previous results were available only for upto three relays. The algorithm is unique in the sense that  the actual mathematical expressions, that are prohibitively large, need not be explicitly obtained. The diversity gain due to multiple relays is shown through plots of the analytical BER, well supported by simulations. 
%
%\end{abstract}
% IEEEtran.cls defaults to using nonbold math in the Abstract.
% This preserves the distinction between vectors and scalars. However,
% if the journal you are submitting to favors bold math in the abstract,
% then you can use LaTeX's standard command \boldmath at the very start
% of the abstract to achieve this. Many IEEE journals frown on math
% in the abstract anyway.

% Note that keywords are not normally used for peerreview papers.
%\begin{IEEEkeywords}
%Cooperative diversity, decode and forward, piecewise linear
%\end{IEEEkeywords}



% For peer review papers, you can put extra information on the cover
% page as needed:
% \ifCLASSOPTIONpeerreview
% \begin{center} \bfseries EDICS Category: 3-BBND \end{center}
% \fi
%
% For peerreview papers, this IEEEtran command inserts a page break and
% creates the second title. It will be ignored for other modes.
%\IEEEpeerreviewmaketitle




\item A card is selected from a pack of 52 cards.
\begin{enumerate}[label=(\alph*)]
    \item How many points are there in the sample space?
    \item Calculate the probability that the card is an ace of spades.
    \item Calculate the probability that the card is (i) an ace and (ii) black card.
\end{enumerate}
\solution
%Let $X$ be an bernoulli rv defined as in \tabref{tab:exemplar/11/16/3/26}.  Then, 
\begin{equation}
    p =
        \frac{4}{11} 
\end{equation}
\begin{table}[H]
	\centering
	\input{exemplar/11/16/3/26/tables/Table2.tex}
	\caption{}
        \label{tab:exemplar/11/16/3/26}
\end{table}

\item The probability that a non leap year selected at random will contain 53 sundays.
\\
\solution
%\begin{table}[H]
	\centering
\begin{tabular}{|c|c|c|}
\hline
Random variable &Value &Definition\\ \hline
\multirow{3}{*}{X} &0 &Slips of Rs 1\\
&1 &Slips of Rs 5\\
&2 &Slips of Rs 13\\ \hline
\multirow{2}{*}{Y} &0 &Box A\\
&1 &Box B\\\hline
\end{tabular}
\caption{}
\label{tab:Distribution}
\end{table}
See \tabref{tab:Distribution}.
\begin{align}
p_{Y}\brak{k}= \begin{cases} 
      \frac{1}{3} & {k=0} \\
      \frac{2}{3 }& {k=1} 
   \end{cases}
   \\
p_{Y|X}\brak{0|0} = \frac{19}{25}\, 
p_{Y|X}\brak{0|1} = \frac{6}{25}\,
p_{Y|X}\brak{1|0} = \frac{45}{50}\,
p_{Y|X}\brak{1|2} = \frac{5}{50}
\end{align}
The desired probability is the probability that a slip drawn at random is marked other than Rs 1,
\begin{align}
&=1-p_X\brak{0}\\
&= p_X(1) + p_X(2)
\end{align}
Using Bayes theorem,
\begin{align}
&= p_Y\brak{0} \times \pr{Y=0 | X=1} + p_Y\brak{1} \times \pr{Y=1|X=2}\\
&=\frac{1}{3} \times \frac{6}{25} + \frac{2}{3} \times \frac{5}{50}\\
&=\frac{11}{75}
\end{align}

\newpage

%\tableofcontents

\bigskip

\renewcommand{\thefigure}{\theenumi}
\renewcommand{\thetable}{\theenumi}
%\renewcommand{\theequation}{\theenumi}

%\begin{abstract}
%%\boldmath
%In this letter, an algorithm for evaluating the exact analytical bit error rate  (BER)  for the piecewise linear (PL) combiner for  multiple relays is presented. Previous results were available only for upto three relays. The algorithm is unique in the sense that  the actual mathematical expressions, that are prohibitively large, need not be explicitly obtained. The diversity gain due to multiple relays is shown through plots of the analytical BER, well supported by simulations. 
%
%\end{abstract}
% IEEEtran.cls defaults to using nonbold math in the Abstract.
% This preserves the distinction between vectors and scalars. However,
% if the journal you are submitting to favors bold math in the abstract,
% then you can use LaTeX's standard command \boldmath at the very start
% of the abstract to achieve this. Many IEEE journals frown on math
% in the abstract anyway.

% Note that keywords are not normally used for peerreview papers.
%\begin{IEEEkeywords}
%Cooperative diversity, decode and forward, piecewise linear
%\end{IEEEkeywords}



% For peer review papers, you can put extra information on the cover
% page as needed:
% \ifCLASSOPTIONpeerreview
% \begin{center} \bfseries EDICS Category: 3-BBND \end{center}
% \fi
%
% For peerreview papers, this IEEEtran command inserts a page break and
% creates the second title. It will be ignored for other modes.
%\IEEEpeerreviewmaketitle




\item One of the four persons John, Rita, Aslam or Gurpreet will be promoted next
month. Consequently the sample space consists of four elementary outcomes
S = {John promoted, Rita promoted, Aslam promoted, Gurpreet promoted}
You are told that the chances of John’s promotion is same as that of Gurpreet,
Rita’s chances of promotion are twice as likely as Johns. Aslam’s chances are
four times that of John.
\begin{enumerate}
	\item Determine
	\begin{enumerate}
		\item P (John promoted)
		\item P (Rita promoted)
		\item P (Aslam promoted)
		\item P (Gurpreet promoted)
	\end{enumerate}
	\item If A = {John promoted or Gurpreet promoted}, find P (A).
\end{enumerate}
\solution
%\begin{table}[H]
	\centering
\begin{tabular}{|c|c|c|}
\hline
Random variable &Value &Definition\\ \hline
\multirow{3}{*}{X} &0 &Slips of Rs 1\\
&1 &Slips of Rs 5\\
&2 &Slips of Rs 13\\ \hline
\multirow{2}{*}{Y} &0 &Box A\\
&1 &Box B\\\hline
\end{tabular}
\caption{}
\label{tab:Distribution}
\end{table}
See \tabref{tab:Distribution}.
\begin{align}
p_{Y}\brak{k}= \begin{cases} 
      \frac{1}{3} & {k=0} \\
      \frac{2}{3 }& {k=1} 
   \end{cases}
   \\
p_{Y|X}\brak{0|0} = \frac{19}{25}\, 
p_{Y|X}\brak{0|1} = \frac{6}{25}\,
p_{Y|X}\brak{1|0} = \frac{45}{50}\,
p_{Y|X}\brak{1|2} = \frac{5}{50}
\end{align}
The desired probability is the probability that a slip drawn at random is marked other than Rs 1,
\begin{align}
&=1-p_X\brak{0}\\
&= p_X(1) + p_X(2)
\end{align}
Using Bayes theorem,
\begin{align}
&= p_Y\brak{0} \times \pr{Y=0 | X=1} + p_Y\brak{1} \times \pr{Y=1|X=2}\\
&=\frac{1}{3} \times \frac{6}{25} + \frac{2}{3} \times \frac{5}{50}\\
&=\frac{11}{75}
\end{align}

\newpage

%\tableofcontents

\bigskip

\renewcommand{\thefigure}{\theenumi}
\renewcommand{\thetable}{\theenumi}
%\renewcommand{\theequation}{\theenumi}

%\begin{abstract}
%%\boldmath
%In this letter, an algorithm for evaluating the exact analytical bit error rate  (BER)  for the piecewise linear (PL) combiner for  multiple relays is presented. Previous results were available only for upto three relays. The algorithm is unique in the sense that  the actual mathematical expressions, that are prohibitively large, need not be explicitly obtained. The diversity gain due to multiple relays is shown through plots of the analytical BER, well supported by simulations. 
%
%\end{abstract}
% IEEEtran.cls defaults to using nonbold math in the Abstract.
% This preserves the distinction between vectors and scalars. However,
% if the journal you are submitting to favors bold math in the abstract,
% then you can use LaTeX's standard command \boldmath at the very start
% of the abstract to achieve this. Many IEEE journals frown on math
% in the abstract anyway.

% Note that keywords are not normally used for peerreview papers.
%\begin{IEEEkeywords}
%Cooperative diversity, decode and forward, piecewise linear
%\end{IEEEkeywords}



% For peer review papers, you can put extra information on the cover
% page as needed:
% \ifCLASSOPTIONpeerreview
% \begin{center} \bfseries EDICS Category: 3-BBND \end{center}
% \fi
%
% For peerreview papers, this IEEEtran command inserts a page break and
% creates the second title. It will be ignored for other modes.
%\IEEEpeerreviewmaketitle




\item A card is drawn from a deck of 52 cards. Find the probability of getting a king or a heart or a red card.\\
\solution
%\begin{table}[H]
	\centering
\begin{tabular}{|c|c|c|}
\hline
Random variable &Value &Definition\\ \hline
\multirow{3}{*}{X} &0 &Slips of Rs 1\\
&1 &Slips of Rs 5\\
&2 &Slips of Rs 13\\ \hline
\multirow{2}{*}{Y} &0 &Box A\\
&1 &Box B\\\hline
\end{tabular}
\caption{}
\label{tab:Distribution}
\end{table}
See \tabref{tab:Distribution}.
\begin{align}
p_{Y}\brak{k}= \begin{cases} 
      \frac{1}{3} & {k=0} \\
      \frac{2}{3 }& {k=1} 
   \end{cases}
   \\
p_{Y|X}\brak{0|0} = \frac{19}{25}\, 
p_{Y|X}\brak{0|1} = \frac{6}{25}\,
p_{Y|X}\brak{1|0} = \frac{45}{50}\,
p_{Y|X}\brak{1|2} = \frac{5}{50}
\end{align}
The desired probability is the probability that a slip drawn at random is marked other than Rs 1,
\begin{align}
&=1-p_X\brak{0}\\
&= p_X(1) + p_X(2)
\end{align}
Using Bayes theorem,
\begin{align}
&= p_Y\brak{0} \times \pr{Y=0 | X=1} + p_Y\brak{1} \times \pr{Y=1|X=2}\\
&=\frac{1}{3} \times \frac{6}{25} + \frac{2}{3} \times \frac{5}{50}\\
&=\frac{11}{75}
\end{align}

\newpage

%\tableofcontents

\bigskip

\renewcommand{\thefigure}{\theenumi}
\renewcommand{\thetable}{\theenumi}
%\renewcommand{\theequation}{\theenumi}

%\begin{abstract}
%%\boldmath
%In this letter, an algorithm for evaluating the exact analytical bit error rate  (BER)  for the piecewise linear (PL) combiner for  multiple relays is presented. Previous results were available only for upto three relays. The algorithm is unique in the sense that  the actual mathematical expressions, that are prohibitively large, need not be explicitly obtained. The diversity gain due to multiple relays is shown through plots of the analytical BER, well supported by simulations. 
%
%\end{abstract}
% IEEEtran.cls defaults to using nonbold math in the Abstract.
% This preserves the distinction between vectors and scalars. However,
% if the journal you are submitting to favors bold math in the abstract,
% then you can use LaTeX's standard command \boldmath at the very start
% of the abstract to achieve this. Many IEEE journals frown on math
% in the abstract anyway.

% Note that keywords are not normally used for peerreview papers.
%\begin{IEEEkeywords}
%Cooperative diversity, decode and forward, piecewise linear
%\end{IEEEkeywords}



% For peer review papers, you can put extra information on the cover
% page as needed:
% \ifCLASSOPTIONpeerreview
% \begin{center} \bfseries EDICS Category: 3-BBND \end{center}
% \fi
%
% For peerreview papers, this IEEEtran command inserts a page break and
% creates the second title. It will be ignored for other modes.
%\IEEEpeerreviewmaketitle




\item The probability that a student will pass his examination is 0.73, the probability of
the student getting a compartment is 0.13, and the probability that the student will
either pass or get compartment is 0.96. State True or False.\\
\solution
%\begin{table}[H]
	\centering
\begin{tabular}{|c|c|c|}
\hline
Random variable &Value &Definition\\ \hline
\multirow{3}{*}{X} &0 &Slips of Rs 1\\
&1 &Slips of Rs 5\\
&2 &Slips of Rs 13\\ \hline
\multirow{2}{*}{Y} &0 &Box A\\
&1 &Box B\\\hline
\end{tabular}
\caption{}
\label{tab:Distribution}
\end{table}
See \tabref{tab:Distribution}.
\begin{align}
p_{Y}\brak{k}= \begin{cases} 
      \frac{1}{3} & {k=0} \\
      \frac{2}{3 }& {k=1} 
   \end{cases}
   \\
p_{Y|X}\brak{0|0} = \frac{19}{25}\, 
p_{Y|X}\brak{0|1} = \frac{6}{25}\,
p_{Y|X}\brak{1|0} = \frac{45}{50}\,
p_{Y|X}\brak{1|2} = \frac{5}{50}
\end{align}
The desired probability is the probability that a slip drawn at random is marked other than Rs 1,
\begin{align}
&=1-p_X\brak{0}\\
&= p_X(1) + p_X(2)
\end{align}
Using Bayes theorem,
\begin{align}
&= p_Y\brak{0} \times \pr{Y=0 | X=1} + p_Y\brak{1} \times \pr{Y=1|X=2}\\
&=\frac{1}{3} \times \frac{6}{25} + \frac{2}{3} \times \frac{5}{50}\\
&=\frac{11}{75}
\end{align}

\newpage

%\tableofcontents

\bigskip

\renewcommand{\thefigure}{\theenumi}
\renewcommand{\thetable}{\theenumi}
%\renewcommand{\theequation}{\theenumi}

%\begin{abstract}
%%\boldmath
%In this letter, an algorithm for evaluating the exact analytical bit error rate  (BER)  for the piecewise linear (PL) combiner for  multiple relays is presented. Previous results were available only for upto three relays. The algorithm is unique in the sense that  the actual mathematical expressions, that are prohibitively large, need not be explicitly obtained. The diversity gain due to multiple relays is shown through plots of the analytical BER, well supported by simulations. 
%
%\end{abstract}
% IEEEtran.cls defaults to using nonbold math in the Abstract.
% This preserves the distinction between vectors and scalars. However,
% if the journal you are submitting to favors bold math in the abstract,
% then you can use LaTeX's standard command \boldmath at the very start
% of the abstract to achieve this. Many IEEE journals frown on math
% in the abstract anyway.

% Note that keywords are not normally used for peerreview papers.
%\begin{IEEEkeywords}
%Cooperative diversity, decode and forward, piecewise linear
%\end{IEEEkeywords}



% For peer review papers, you can put extra information on the cover
% page as needed:
% \ifCLASSOPTIONpeerreview
% \begin{center} \bfseries EDICS Category: 3-BBND \end{center}
% \fi
%
% For peerreview papers, this IEEEtran command inserts a page break and
% creates the second title. It will be ignored for other modes.
%\IEEEpeerreviewmaketitle




\item A card is selected from a pack of 52 cards\\
\begin{enumerate}[label=(\alph*)]
\item How many points are there in the sample space?
\item Calculate the probability that the cards is an ace of spades.
\item Calculate the probability that the card is (i) an ace (ii)black card.\\
\end{enumerate}
%\input{ncert/11/16/3/4_1/Prob_4.tex}
\item In a non-leap year, the probability of having 53 tuesdays or 53 wednesdays is\\
\solution
%A non-leap year has a total of 365 days, and a week has 7 days.\\
So it can be expressed as 
\begin{align}
365\text{days} &=52\times 7+1 \text{day}
\end{align}
$\implies$ 52 tuesdays or wednesdays\\
Random variable X denotes the days of a week
\begin{align}
p_X\brak{k}&=\frac{1}{7}; \quad \brak{1<k<7}
\end{align}
So the probability of extra day being tuesday or wednesday is
\begin{align}
p_X\brak{3}+p_X\brak{4}&=\frac{1}{7}+\frac{1}{7}=\frac{2}{7}
\end{align}



\item There are 1000 sealed envelopes in a box, 10 of them contain a cash prize of
Rs 100 each, 100 of them contain a cash prize of Rs 50 each and 200 of them
contain a cash prize of Rs 10 each and rest do not contain any cash prize. If they
are well shuffled and an envelope is picked up out, what is the probability that it
contains no cash prize?\\
\solution
%\begin{table}[H]
	\centering
\begin{tabular}{|c|c|c|}
\hline
Random variable &Value &Definition\\ \hline
\multirow{3}{*}{X} &0 &Slips of Rs 1\\
&1 &Slips of Rs 5\\
&2 &Slips of Rs 13\\ \hline
\multirow{2}{*}{Y} &0 &Box A\\
&1 &Box B\\\hline
\end{tabular}
\caption{}
\label{tab:Distribution}
\end{table}
See \tabref{tab:Distribution}.
\begin{align}
p_{Y}\brak{k}= \begin{cases} 
      \frac{1}{3} & {k=0} \\
      \frac{2}{3 }& {k=1} 
   \end{cases}
   \\
p_{Y|X}\brak{0|0} = \frac{19}{25}\, 
p_{Y|X}\brak{0|1} = \frac{6}{25}\,
p_{Y|X}\brak{1|0} = \frac{45}{50}\,
p_{Y|X}\brak{1|2} = \frac{5}{50}
\end{align}
The desired probability is the probability that a slip drawn at random is marked other than Rs 1,
\begin{align}
&=1-p_X\brak{0}\\
&= p_X(1) + p_X(2)
\end{align}
Using Bayes theorem,
\begin{align}
&= p_Y\brak{0} \times \pr{Y=0 | X=1} + p_Y\brak{1} \times \pr{Y=1|X=2}\\
&=\frac{1}{3} \times \frac{6}{25} + \frac{2}{3} \times \frac{5}{50}\\
&=\frac{11}{75}
\end{align}

\newpage

%\tableofcontents

\bigskip

\renewcommand{\thefigure}{\theenumi}
\renewcommand{\thetable}{\theenumi}
%\renewcommand{\theequation}{\theenumi}

%\begin{abstract}
%%\boldmath
%In this letter, an algorithm for evaluating the exact analytical bit error rate  (BER)  for the piecewise linear (PL) combiner for  multiple relays is presented. Previous results were available only for upto three relays. The algorithm is unique in the sense that  the actual mathematical expressions, that are prohibitively large, need not be explicitly obtained. The diversity gain due to multiple relays is shown through plots of the analytical BER, well supported by simulations. 
%
%\end{abstract}
% IEEEtran.cls defaults to using nonbold math in the Abstract.
% This preserves the distinction between vectors and scalars. However,
% if the journal you are submitting to favors bold math in the abstract,
% then you can use LaTeX's standard command \boldmath at the very start
% of the abstract to achieve this. Many IEEE journals frown on math
% in the abstract anyway.

% Note that keywords are not normally used for peerreview papers.
%\begin{IEEEkeywords}
%Cooperative diversity, decode and forward, piecewise linear
%\end{IEEEkeywords}



% For peer review papers, you can put extra information on the cover
% page as needed:
% \ifCLASSOPTIONpeerreview
% \begin{center} \bfseries EDICS Category: 3-BBND \end{center}
% \fi
%
% For peerreview papers, this IEEEtran command inserts a page break and
% creates the second title. It will be ignored for other modes.
%\IEEEpeerreviewmaketitle




\item 
A die is thrown and a card is selected at random from a deck of 52 playing cards. The probability of getting an even number on the die and a spade card.\\
\solution
%\begin{table}[H]
	\centering
\begin{tabular}{|c|c|c|}
\hline
Random variable &Value &Definition\\ \hline
\multirow{3}{*}{X} &0 &Slips of Rs 1\\
&1 &Slips of Rs 5\\
&2 &Slips of Rs 13\\ \hline
\multirow{2}{*}{Y} &0 &Box A\\
&1 &Box B\\\hline
\end{tabular}
\caption{}
\label{tab:Distribution}
\end{table}
See \tabref{tab:Distribution}.
\begin{align}
p_{Y}\brak{k}= \begin{cases} 
      \frac{1}{3} & {k=0} \\
      \frac{2}{3 }& {k=1} 
   \end{cases}
   \\
p_{Y|X}\brak{0|0} = \frac{19}{25}\, 
p_{Y|X}\brak{0|1} = \frac{6}{25}\,
p_{Y|X}\brak{1|0} = \frac{45}{50}\,
p_{Y|X}\brak{1|2} = \frac{5}{50}
\end{align}
The desired probability is the probability that a slip drawn at random is marked other than Rs 1,
\begin{align}
&=1-p_X\brak{0}\\
&= p_X(1) + p_X(2)
\end{align}
Using Bayes theorem,
\begin{align}
&= p_Y\brak{0} \times \pr{Y=0 | X=1} + p_Y\brak{1} \times \pr{Y=1|X=2}\\
&=\frac{1}{3} \times \frac{6}{25} + \frac{2}{3} \times \frac{5}{50}\\
&=\frac{11}{75}
\end{align}

\newpage

%\tableofcontents

\bigskip

\renewcommand{\thefigure}{\theenumi}
\renewcommand{\thetable}{\theenumi}
%\renewcommand{\theequation}{\theenumi}

%\begin{abstract}
%%\boldmath
%In this letter, an algorithm for evaluating the exact analytical bit error rate  (BER)  for the piecewise linear (PL) combiner for  multiple relays is presented. Previous results were available only for upto three relays. The algorithm is unique in the sense that  the actual mathematical expressions, that are prohibitively large, need not be explicitly obtained. The diversity gain due to multiple relays is shown through plots of the analytical BER, well supported by simulations. 
%
%\end{abstract}
% IEEEtran.cls defaults to using nonbold math in the Abstract.
% This preserves the distinction between vectors and scalars. However,
% if the journal you are submitting to favors bold math in the abstract,
% then you can use LaTeX's standard command \boldmath at the very start
% of the abstract to achieve this. Many IEEE journals frown on math
% in the abstract anyway.

% Note that keywords are not normally used for peerreview papers.
%\begin{IEEEkeywords}
%Cooperative diversity, decode and forward, piecewise linear
%\end{IEEEkeywords}



% For peer review papers, you can put extra information on the cover
% page as needed:
% \ifCLASSOPTIONpeerreview
% \begin{center} \bfseries EDICS Category: 3-BBND \end{center}
% \fi
%
% For peerreview papers, this IEEEtran command inserts a page break and
% creates the second title. It will be ignored for other modes.
%\IEEEpeerreviewmaketitle




\item
If 4-digit numbers greater than 5,000 are randomly formed from the digits 0, 1, 3, 5, and 7, what is the probability of forming a number divisible by 5 when:
\begin{enumerate}
    \item The digits are repeated?
    \item The repetition of digits is not allowed?
\end{enumerate}
\solution
%\begin{table}[H]
	\centering
\begin{tabular}{|c|c|c|}
\hline
Random variable &Value &Definition\\ \hline
\multirow{3}{*}{X} &0 &Slips of Rs 1\\
&1 &Slips of Rs 5\\
&2 &Slips of Rs 13\\ \hline
\multirow{2}{*}{Y} &0 &Box A\\
&1 &Box B\\\hline
\end{tabular}
\caption{}
\label{tab:Distribution}
\end{table}
See \tabref{tab:Distribution}.
\begin{align}
p_{Y}\brak{k}= \begin{cases} 
      \frac{1}{3} & {k=0} \\
      \frac{2}{3 }& {k=1} 
   \end{cases}
   \\
p_{Y|X}\brak{0|0} = \frac{19}{25}\, 
p_{Y|X}\brak{0|1} = \frac{6}{25}\,
p_{Y|X}\brak{1|0} = \frac{45}{50}\,
p_{Y|X}\brak{1|2} = \frac{5}{50}
\end{align}
The desired probability is the probability that a slip drawn at random is marked other than Rs 1,
\begin{align}
&=1-p_X\brak{0}\\
&= p_X(1) + p_X(2)
\end{align}
Using Bayes theorem,
\begin{align}
&= p_Y\brak{0} \times \pr{Y=0 | X=1} + p_Y\brak{1} \times \pr{Y=1|X=2}\\
&=\frac{1}{3} \times \frac{6}{25} + \frac{2}{3} \times \frac{5}{50}\\
&=\frac{11}{75}
\end{align}

\newpage

%\tableofcontents

\bigskip

\renewcommand{\thefigure}{\theenumi}
\renewcommand{\thetable}{\theenumi}
%\renewcommand{\theequation}{\theenumi}

%\begin{abstract}
%%\boldmath
%In this letter, an algorithm for evaluating the exact analytical bit error rate  (BER)  for the piecewise linear (PL) combiner for  multiple relays is presented. Previous results were available only for upto three relays. The algorithm is unique in the sense that  the actual mathematical expressions, that are prohibitively large, need not be explicitly obtained. The diversity gain due to multiple relays is shown through plots of the analytical BER, well supported by simulations. 
%
%\end{abstract}
% IEEEtran.cls defaults to using nonbold math in the Abstract.
% This preserves the distinction between vectors and scalars. However,
% if the journal you are submitting to favors bold math in the abstract,
% then you can use LaTeX's standard command \boldmath at the very start
% of the abstract to achieve this. Many IEEE journals frown on math
% in the abstract anyway.

% Note that keywords are not normally used for peerreview papers.
%\begin{IEEEkeywords}
%Cooperative diversity, decode and forward, piecewise linear
%\end{IEEEkeywords}



% For peer review papers, you can put extra information on the cover
% page as needed:
% \ifCLASSOPTIONpeerreview
% \begin{center} \bfseries EDICS Category: 3-BBND \end{center}
% \fi
%
% For peerreview papers, this IEEEtran command inserts a page break and
% creates the second title. It will be ignored for other modes.
%\IEEEpeerreviewmaketitle




\item Consider the probability space $\brak{\Omega, \mathcal{G}, P}$ where $\Omega = [0,2]$ and $\mathcal{G} = \cbrak{\phi, \Omega, [0,1], (1,2]}$. Let $X$ and $Y$ be two functions on $\Omega$ defined as
\begin{align*}
    X(\omega) = 
    \begin{cases}
        1 & \text{if }\omega \in [0, 1]\\
        2 & \text{if }\omega \in (1, 2]
    \end{cases}
\end{align*}
and
\begin{align*}
    Y(\omega) = 
    \begin{cases}
        2 & \text{if }\omega \in [0, 1.5]\\
        3 & \text{if }\omega \in (1.5, 2].
    \end{cases}
\end{align*}
Then which one of the following statements is true?
\begin{enumerate}
    \item [(A)] $X$ is a random variable with respect to $\mathcal{G}$, but $Y$ is not a random variable with respect to $\mathcal{G}$.
    \item [(B)] $Y$ is a random variable with respect to $\mathcal{G}$, but $X$ is not a random variable with respect to $\mathcal{G}$.
    \item [(C)] Neither $X$ nor $Y$ is a random variable with respect to $\mathcal{G}$.
    \item [(D)] Both $X$ and $Y$ are random variables with respect to $\mathcal{G}$.
\end{enumerate} \hfill (GATE ST 2023)\\
\solution
%\begin{table}[H]
	\centering
\begin{tabular}{|c|c|c|}
\hline
Random variable &Value &Definition\\ \hline
\multirow{3}{*}{X} &0 &Slips of Rs 1\\
&1 &Slips of Rs 5\\
&2 &Slips of Rs 13\\ \hline
\multirow{2}{*}{Y} &0 &Box A\\
&1 &Box B\\\hline
\end{tabular}
\caption{}
\label{tab:Distribution}
\end{table}
See \tabref{tab:Distribution}.
\begin{align}
p_{Y}\brak{k}= \begin{cases} 
      \frac{1}{3} & {k=0} \\
      \frac{2}{3 }& {k=1} 
   \end{cases}
   \\
p_{Y|X}\brak{0|0} = \frac{19}{25}\, 
p_{Y|X}\brak{0|1} = \frac{6}{25}\,
p_{Y|X}\brak{1|0} = \frac{45}{50}\,
p_{Y|X}\brak{1|2} = \frac{5}{50}
\end{align}
The desired probability is the probability that a slip drawn at random is marked other than Rs 1,
\begin{align}
&=1-p_X\brak{0}\\
&= p_X(1) + p_X(2)
\end{align}
Using Bayes theorem,
\begin{align}
&= p_Y\brak{0} \times \pr{Y=0 | X=1} + p_Y\brak{1} \times \pr{Y=1|X=2}\\
&=\frac{1}{3} \times \frac{6}{25} + \frac{2}{3} \times \frac{5}{50}\\
&=\frac{11}{75}
\end{align}

\newpage

%\tableofcontents

\bigskip

\renewcommand{\thefigure}{\theenumi}
\renewcommand{\thetable}{\theenumi}
%\renewcommand{\theequation}{\theenumi}

%\begin{abstract}
%%\boldmath
%In this letter, an algorithm for evaluating the exact analytical bit error rate  (BER)  for the piecewise linear (PL) combiner for  multiple relays is presented. Previous results were available only for upto three relays. The algorithm is unique in the sense that  the actual mathematical expressions, that are prohibitively large, need not be explicitly obtained. The diversity gain due to multiple relays is shown through plots of the analytical BER, well supported by simulations. 
%
%\end{abstract}
% IEEEtran.cls defaults to using nonbold math in the Abstract.
% This preserves the distinction between vectors and scalars. However,
% if the journal you are submitting to favors bold math in the abstract,
% then you can use LaTeX's standard command \boldmath at the very start
% of the abstract to achieve this. Many IEEE journals frown on math
% in the abstract anyway.

% Note that keywords are not normally used for peerreview papers.
%\begin{IEEEkeywords}
%Cooperative diversity, decode and forward, piecewise linear
%\end{IEEEkeywords}



% For peer review papers, you can put extra information on the cover
% page as needed:
% \ifCLASSOPTIONpeerreview
% \begin{center} \bfseries EDICS Category: 3-BBND \end{center}
% \fi
%
% For peerreview papers, this IEEEtran command inserts a page break and
% creates the second title. It will be ignored for other modes.
%\IEEEpeerreviewmaketitle




	\item  A die is loaded in such a way that each odd number is twice as likely to occur as
each even number. Find $P(G)$, where $G$ is the event that a number greater than
3 occurs on a single roll of the die.
\\
\solution
		%\begin{table}[H]
	\centering
\begin{tabular}{|c|c|c|}
\hline
Random variable &Value &Definition\\ \hline
\multirow{3}{*}{X} &0 &Slips of Rs 1\\
&1 &Slips of Rs 5\\
&2 &Slips of Rs 13\\ \hline
\multirow{2}{*}{Y} &0 &Box A\\
&1 &Box B\\\hline
\end{tabular}
\caption{}
\label{tab:Distribution}
\end{table}
See \tabref{tab:Distribution}.
\begin{align}
p_{Y}\brak{k}= \begin{cases} 
      \frac{1}{3} & {k=0} \\
      \frac{2}{3 }& {k=1} 
   \end{cases}
   \\
p_{Y|X}\brak{0|0} = \frac{19}{25}\, 
p_{Y|X}\brak{0|1} = \frac{6}{25}\,
p_{Y|X}\brak{1|0} = \frac{45}{50}\,
p_{Y|X}\brak{1|2} = \frac{5}{50}
\end{align}
The desired probability is the probability that a slip drawn at random is marked other than Rs 1,
\begin{align}
&=1-p_X\brak{0}\\
&= p_X(1) + p_X(2)
\end{align}
Using Bayes theorem,
\begin{align}
&= p_Y\brak{0} \times \pr{Y=0 | X=1} + p_Y\brak{1} \times \pr{Y=1|X=2}\\
&=\frac{1}{3} \times \frac{6}{25} + \frac{2}{3} \times \frac{5}{50}\\
&=\frac{11}{75}
\end{align}

\newpage

%\tableofcontents

\bigskip

\renewcommand{\thefigure}{\theenumi}
\renewcommand{\thetable}{\theenumi}
%\renewcommand{\theequation}{\theenumi}

%\begin{abstract}
%%\boldmath
%In this letter, an algorithm for evaluating the exact analytical bit error rate  (BER)  for the piecewise linear (PL) combiner for  multiple relays is presented. Previous results were available only for upto three relays. The algorithm is unique in the sense that  the actual mathematical expressions, that are prohibitively large, need not be explicitly obtained. The diversity gain due to multiple relays is shown through plots of the analytical BER, well supported by simulations. 
%
%\end{abstract}
% IEEEtran.cls defaults to using nonbold math in the Abstract.
% This preserves the distinction between vectors and scalars. However,
% if the journal you are submitting to favors bold math in the abstract,
% then you can use LaTeX's standard command \boldmath at the very start
% of the abstract to achieve this. Many IEEE journals frown on math
% in the abstract anyway.

% Note that keywords are not normally used for peerreview papers.
%\begin{IEEEkeywords}
%Cooperative diversity, decode and forward, piecewise linear
%\end{IEEEkeywords}



% For peer review papers, you can put extra information on the cover
% page as needed:
% \ifCLASSOPTIONpeerreview
% \begin{center} \bfseries EDICS Category: 3-BBND \end{center}
% \fi
%
% For peerreview papers, this IEEEtran command inserts a page break and
% creates the second title. It will be ignored for other modes.
%\IEEEpeerreviewmaketitle




	\item All the jacks, queens and kings are removed from a deck of 52 playing cards. The remaining cards are well shuffled and then one card is drawn at random. Giving ace a value 1 similar value for other cards, find the probability that the card has a value 
		\begin{enumerate}
			\item 7
			\item greater than 7
			\item less than 7
		\end{enumerate}
		%Number of cards left after removing all jacks, queens and kings 
\begin{align}
N	= 52 - 4\times 3
	= 40
\end{align}
%\begin{table}[H]
%\def\arraystretch{1.2}
%\begin{tabular}{|c|c|c|}
%\hline
%	\textbf{Parameter} &\textbf{Value} &\textbf{Description}\\ \hline
%	$X$ &1-10 &Represents the value of the card picked \\ \hline
%\end{tabular}
%\end{table}
Let $1 \le X \le 10$ be the value of the card picked.  Then,
\begin{align}
	p_X(k) &= \Pr(X=k)\ \forall\ 1 \leq k \leq 10\\
	&= \frac{4\times 1}{40}\\
	&= \frac{1}{10}\\
	\therefore p_X(k) &= 
	\begin{cases}
		\frac{1}{10} & 1 \leq k \leq 10\\
		0 & \text{otherwise}
	\end{cases}
\end{align}
and
\begin{align}
	F_{X}(k) &= \sum_{m=0}^{k}p_{X}(m) \quad 1 \leq k \leq 10\\
	&= \frac{k}{10}\\
	\therefore F_{X}(k) &= 
	\begin{cases}
		0 & k \leq 0\\
		\frac{k}{10} & 1\leq k \leq 10\\
		1 & k > 10 
	\end{cases}
\end{align}
\begin{enumerate}
	\item Probability that card has value equal to 7 is
		\begin{align}
			 p_{X}(7)
			= \frac{1}{10}
		\end{align}
	\item Probability that card has value greater than 7 is
		\begin{align}
			1 - F_X(7)
			&= 1 - \frac{7}{10}
			\\
			&= \frac{3}{10}
		\end{align}
	\item Probability that card has value less than 7 is
		\begin{align}
			 F_{X}(6)
			=\frac{6}{10}
		\end{align}
\end{enumerate}

  \item A Lot consists of 48 mobile phones of which 42 are good, 3 have only minor defects and 3 have major defects.Varnika will buy a phone if it is good but the trader will only buy a mobile if it has no major defects. One phone is selected at random from the lot. What is the probability that it is
\begin{enumerate}
	\item acceptable to Varnika?
            \item acceptable to the trader?
\end{enumerate}
\solution
	%\begin{table}[H]
	\centering
\begin{tabular}{|c|c|c|}
\hline
Random variable &Value &Definition\\ \hline
\multirow{3}{*}{X} &0 &Slips of Rs 1\\
&1 &Slips of Rs 5\\
&2 &Slips of Rs 13\\ \hline
\multirow{2}{*}{Y} &0 &Box A\\
&1 &Box B\\\hline
\end{tabular}
\caption{}
\label{tab:Distribution}
\end{table}
See \tabref{tab:Distribution}.
\begin{align}
p_{Y}\brak{k}= \begin{cases} 
      \frac{1}{3} & {k=0} \\
      \frac{2}{3 }& {k=1} 
   \end{cases}
   \\
p_{Y|X}\brak{0|0} = \frac{19}{25}\, 
p_{Y|X}\brak{0|1} = \frac{6}{25}\,
p_{Y|X}\brak{1|0} = \frac{45}{50}\,
p_{Y|X}\brak{1|2} = \frac{5}{50}
\end{align}
The desired probability is the probability that a slip drawn at random is marked other than Rs 1,
\begin{align}
&=1-p_X\brak{0}\\
&= p_X(1) + p_X(2)
\end{align}
Using Bayes theorem,
\begin{align}
&= p_Y\brak{0} \times \pr{Y=0 | X=1} + p_Y\brak{1} \times \pr{Y=1|X=2}\\
&=\frac{1}{3} \times \frac{6}{25} + \frac{2}{3} \times \frac{5}{50}\\
&=\frac{11}{75}
\end{align}

\newpage

%\tableofcontents

\bigskip

\renewcommand{\thefigure}{\theenumi}
\renewcommand{\thetable}{\theenumi}
%\renewcommand{\theequation}{\theenumi}

%\begin{abstract}
%%\boldmath
%In this letter, an algorithm for evaluating the exact analytical bit error rate  (BER)  for the piecewise linear (PL) combiner for  multiple relays is presented. Previous results were available only for upto three relays. The algorithm is unique in the sense that  the actual mathematical expressions, that are prohibitively large, need not be explicitly obtained. The diversity gain due to multiple relays is shown through plots of the analytical BER, well supported by simulations. 
%
%\end{abstract}
% IEEEtran.cls defaults to using nonbold math in the Abstract.
% This preserves the distinction between vectors and scalars. However,
% if the journal you are submitting to favors bold math in the abstract,
% then you can use LaTeX's standard command \boldmath at the very start
% of the abstract to achieve this. Many IEEE journals frown on math
% in the abstract anyway.

% Note that keywords are not normally used for peerreview papers.
%\begin{IEEEkeywords}
%Cooperative diversity, decode and forward, piecewise linear
%\end{IEEEkeywords}



% For peer review papers, you can put extra information on the cover
% page as needed:
% \ifCLASSOPTIONpeerreview
% \begin{center} \bfseries EDICS Category: 3-BBND \end{center}
% \fi
%
% For peerreview papers, this IEEEtran command inserts a page break and
% creates the second title. It will be ignored for other modes.
%\IEEEpeerreviewmaketitle




 \item A student says that if you throw a die, it will show up 1 or not 1. Therefore, the probability of getting 1 and the probability of getting 'not 1' each is equal to $\frac{1}{2}$. Is this correct? Give reasons.\\
 \solution
        %\begin{table}[H]
	\centering
\begin{tabular}{|c|c|c|}
\hline
Random variable &Value &Definition\\ \hline
\multirow{3}{*}{X} &0 &Slips of Rs 1\\
&1 &Slips of Rs 5\\
&2 &Slips of Rs 13\\ \hline
\multirow{2}{*}{Y} &0 &Box A\\
&1 &Box B\\\hline
\end{tabular}
\caption{}
\label{tab:Distribution}
\end{table}
See \tabref{tab:Distribution}.
\begin{align}
p_{Y}\brak{k}= \begin{cases} 
      \frac{1}{3} & {k=0} \\
      \frac{2}{3 }& {k=1} 
   \end{cases}
   \\
p_{Y|X}\brak{0|0} = \frac{19}{25}\, 
p_{Y|X}\brak{0|1} = \frac{6}{25}\,
p_{Y|X}\brak{1|0} = \frac{45}{50}\,
p_{Y|X}\brak{1|2} = \frac{5}{50}
\end{align}
The desired probability is the probability that a slip drawn at random is marked other than Rs 1,
\begin{align}
&=1-p_X\brak{0}\\
&= p_X(1) + p_X(2)
\end{align}
Using Bayes theorem,
\begin{align}
&= p_Y\brak{0} \times \pr{Y=0 | X=1} + p_Y\brak{1} \times \pr{Y=1|X=2}\\
&=\frac{1}{3} \times \frac{6}{25} + \frac{2}{3} \times \frac{5}{50}\\
&=\frac{11}{75}
\end{align}

\newpage

%\tableofcontents

\bigskip

\renewcommand{\thefigure}{\theenumi}
\renewcommand{\thetable}{\theenumi}
%\renewcommand{\theequation}{\theenumi}

%\begin{abstract}
%%\boldmath
%In this letter, an algorithm for evaluating the exact analytical bit error rate  (BER)  for the piecewise linear (PL) combiner for  multiple relays is presented. Previous results were available only for upto three relays. The algorithm is unique in the sense that  the actual mathematical expressions, that are prohibitively large, need not be explicitly obtained. The diversity gain due to multiple relays is shown through plots of the analytical BER, well supported by simulations. 
%
%\end{abstract}
% IEEEtran.cls defaults to using nonbold math in the Abstract.
% This preserves the distinction between vectors and scalars. However,
% if the journal you are submitting to favors bold math in the abstract,
% then you can use LaTeX's standard command \boldmath at the very start
% of the abstract to achieve this. Many IEEE journals frown on math
% in the abstract anyway.

% Note that keywords are not normally used for peerreview papers.
%\begin{IEEEkeywords}
%Cooperative diversity, decode and forward, piecewise linear
%\end{IEEEkeywords}



% For peer review papers, you can put extra information on the cover
% page as needed:
% \ifCLASSOPTIONpeerreview
% \begin{center} \bfseries EDICS Category: 3-BBND \end{center}
% \fi
%
% For peerreview papers, this IEEEtran command inserts a page break and
% creates the second title. It will be ignored for other modes.
%\IEEEpeerreviewmaketitle




   \item Four candidates A, B, C, D have ap-
plied for the assignment to coach a school cricket
team. If A is twice as likely to be selected as B, and
B and C are given about the same chance of being
selected, while C is twice as likely to be selected
as D, what are the probabilities that
\begin{enumerate}
\item C will be selected?
\item A will not be selected?
\end{enumerate}
	%\begin{table}[H]
	\centering
\begin{tabular}{|c|c|c|}
\hline
Random variable &Value &Definition\\ \hline
\multirow{3}{*}{X} &0 &Slips of Rs 1\\
&1 &Slips of Rs 5\\
&2 &Slips of Rs 13\\ \hline
\multirow{2}{*}{Y} &0 &Box A\\
&1 &Box B\\\hline
\end{tabular}
\caption{}
\label{tab:Distribution}
\end{table}
See \tabref{tab:Distribution}.
\begin{align}
p_{Y}\brak{k}= \begin{cases} 
      \frac{1}{3} & {k=0} \\
      \frac{2}{3 }& {k=1} 
   \end{cases}
   \\
p_{Y|X}\brak{0|0} = \frac{19}{25}\, 
p_{Y|X}\brak{0|1} = \frac{6}{25}\,
p_{Y|X}\brak{1|0} = \frac{45}{50}\,
p_{Y|X}\brak{1|2} = \frac{5}{50}
\end{align}
The desired probability is the probability that a slip drawn at random is marked other than Rs 1,
\begin{align}
&=1-p_X\brak{0}\\
&= p_X(1) + p_X(2)
\end{align}
Using Bayes theorem,
\begin{align}
&= p_Y\brak{0} \times \pr{Y=0 | X=1} + p_Y\brak{1} \times \pr{Y=1|X=2}\\
&=\frac{1}{3} \times \frac{6}{25} + \frac{2}{3} \times \frac{5}{50}\\
&=\frac{11}{75}
\end{align}

\newpage

%\tableofcontents

\bigskip

\renewcommand{\thefigure}{\theenumi}
\renewcommand{\thetable}{\theenumi}
%\renewcommand{\theequation}{\theenumi}

%\begin{abstract}
%%\boldmath
%In this letter, an algorithm for evaluating the exact analytical bit error rate  (BER)  for the piecewise linear (PL) combiner for  multiple relays is presented. Previous results were available only for upto three relays. The algorithm is unique in the sense that  the actual mathematical expressions, that are prohibitively large, need not be explicitly obtained. The diversity gain due to multiple relays is shown through plots of the analytical BER, well supported by simulations. 
%
%\end{abstract}
% IEEEtran.cls defaults to using nonbold math in the Abstract.
% This preserves the distinction between vectors and scalars. However,
% if the journal you are submitting to favors bold math in the abstract,
% then you can use LaTeX's standard command \boldmath at the very start
% of the abstract to achieve this. Many IEEE journals frown on math
% in the abstract anyway.

% Note that keywords are not normally used for peerreview papers.
%\begin{IEEEkeywords}
%Cooperative diversity, decode and forward, piecewise linear
%\end{IEEEkeywords}



% For peer review papers, you can put extra information on the cover
% page as needed:
% \ifCLASSOPTIONpeerreview
% \begin{center} \bfseries EDICS Category: 3-BBND \end{center}
% \fi
%
% For peerreview papers, this IEEEtran command inserts a page break and
% creates the second title. It will be ignored for other modes.
%\IEEEpeerreviewmaketitle




 \item A bag contain 24 balls of which $x$ balls are red, $2x$ are white and $3x$ are blue. A ball is selected at random, What is the probability that it is
\begin{enumerate}[label=\alph*)]
\item not red ?
\item white ?
\end{enumerate}
%\begin{table}[H]
	\centering
\begin{tabular}{|c|c|c|}
\hline
Random variable &Value &Definition\\ \hline
\multirow{3}{*}{X} &0 &Slips of Rs 1\\
&1 &Slips of Rs 5\\
&2 &Slips of Rs 13\\ \hline
\multirow{2}{*}{Y} &0 &Box A\\
&1 &Box B\\\hline
\end{tabular}
\caption{}
\label{tab:Distribution}
\end{table}
See \tabref{tab:Distribution}.
\begin{align}
p_{Y}\brak{k}= \begin{cases} 
      \frac{1}{3} & {k=0} \\
      \frac{2}{3 }& {k=1} 
   \end{cases}
   \\
p_{Y|X}\brak{0|0} = \frac{19}{25}\, 
p_{Y|X}\brak{0|1} = \frac{6}{25}\,
p_{Y|X}\brak{1|0} = \frac{45}{50}\,
p_{Y|X}\brak{1|2} = \frac{5}{50}
\end{align}
The desired probability is the probability that a slip drawn at random is marked other than Rs 1,
\begin{align}
&=1-p_X\brak{0}\\
&= p_X(1) + p_X(2)
\end{align}
Using Bayes theorem,
\begin{align}
&= p_Y\brak{0} \times \pr{Y=0 | X=1} + p_Y\brak{1} \times \pr{Y=1|X=2}\\
&=\frac{1}{3} \times \frac{6}{25} + \frac{2}{3} \times \frac{5}{50}\\
&=\frac{11}{75}
\end{align}

\newpage

%\tableofcontents

\bigskip

\renewcommand{\thefigure}{\theenumi}
\renewcommand{\thetable}{\theenumi}
%\renewcommand{\theequation}{\theenumi}

%\begin{abstract}
%%\boldmath
%In this letter, an algorithm for evaluating the exact analytical bit error rate  (BER)  for the piecewise linear (PL) combiner for  multiple relays is presented. Previous results were available only for upto three relays. The algorithm is unique in the sense that  the actual mathematical expressions, that are prohibitively large, need not be explicitly obtained. The diversity gain due to multiple relays is shown through plots of the analytical BER, well supported by simulations. 
%
%\end{abstract}
% IEEEtran.cls defaults to using nonbold math in the Abstract.
% This preserves the distinction between vectors and scalars. However,
% if the journal you are submitting to favors bold math in the abstract,
% then you can use LaTeX's standard command \boldmath at the very start
% of the abstract to achieve this. Many IEEE journals frown on math
% in the abstract anyway.

% Note that keywords are not normally used for peerreview papers.
%\begin{IEEEkeywords}
%Cooperative diversity, decode and forward, piecewise linear
%\end{IEEEkeywords}



% For peer review papers, you can put extra information on the cover
% page as needed:
% \ifCLASSOPTIONpeerreview
% \begin{center} \bfseries EDICS Category: 3-BBND \end{center}
% \fi
%
% For peerreview papers, this IEEEtran command inserts a page break and
% creates the second title. It will be ignored for other modes.
%\IEEEpeerreviewmaketitle




If the letters of the word ASSASSINATION are arranged at random. Find the Probability that
\begin{enumerate}[label=(\alph*)]
\item Four $S's$ come consecutively in the word
\item Two  $I's$ and two $N's$ come together
\item All $A's$ are not coming together
\item No two $A's$ are coming together
\end{enumerate}
%\begin{table}[H]
	\centering
\begin{tabular}{|c|c|c|}
\hline
Random variable &Value &Definition\\ \hline
\multirow{3}{*}{X} &0 &Slips of Rs 1\\
&1 &Slips of Rs 5\\
&2 &Slips of Rs 13\\ \hline
\multirow{2}{*}{Y} &0 &Box A\\
&1 &Box B\\\hline
\end{tabular}
\caption{}
\label{tab:Distribution}
\end{table}
See \tabref{tab:Distribution}.
\begin{align}
p_{Y}\brak{k}= \begin{cases} 
      \frac{1}{3} & {k=0} \\
      \frac{2}{3 }& {k=1} 
   \end{cases}
   \\
p_{Y|X}\brak{0|0} = \frac{19}{25}\, 
p_{Y|X}\brak{0|1} = \frac{6}{25}\,
p_{Y|X}\brak{1|0} = \frac{45}{50}\,
p_{Y|X}\brak{1|2} = \frac{5}{50}
\end{align}
The desired probability is the probability that a slip drawn at random is marked other than Rs 1,
\begin{align}
&=1-p_X\brak{0}\\
&= p_X(1) + p_X(2)
\end{align}
Using Bayes theorem,
\begin{align}
&= p_Y\brak{0} \times \pr{Y=0 | X=1} + p_Y\brak{1} \times \pr{Y=1|X=2}\\
&=\frac{1}{3} \times \frac{6}{25} + \frac{2}{3} \times \frac{5}{50}\\
&=\frac{11}{75}
\end{align}

\newpage

%\tableofcontents

\bigskip

\renewcommand{\thefigure}{\theenumi}
\renewcommand{\thetable}{\theenumi}
%\renewcommand{\theequation}{\theenumi}

%\begin{abstract}
%%\boldmath
%In this letter, an algorithm for evaluating the exact analytical bit error rate  (BER)  for the piecewise linear (PL) combiner for  multiple relays is presented. Previous results were available only for upto three relays. The algorithm is unique in the sense that  the actual mathematical expressions, that are prohibitively large, need not be explicitly obtained. The diversity gain due to multiple relays is shown through plots of the analytical BER, well supported by simulations. 
%
%\end{abstract}
% IEEEtran.cls defaults to using nonbold math in the Abstract.
% This preserves the distinction between vectors and scalars. However,
% if the journal you are submitting to favors bold math in the abstract,
% then you can use LaTeX's standard command \boldmath at the very start
% of the abstract to achieve this. Many IEEE journals frown on math
% in the abstract anyway.

% Note that keywords are not normally used for peerreview papers.
%\begin{IEEEkeywords}
%Cooperative diversity, decode and forward, piecewise linear
%\end{IEEEkeywords}



% For peer review papers, you can put extra information on the cover
% page as needed:
% \ifCLASSOPTIONpeerreview
% \begin{center} \bfseries EDICS Category: 3-BBND \end{center}
% \fi
%
% For peerreview papers, this IEEEtran command inserts a page break and
% creates the second title. It will be ignored for other modes.
%\IEEEpeerreviewmaketitle




	\item One urn contains two black balls (labelled B1 and B2) and one white ball. A
	second urn contains one black ball and two white balls (labelled W1 and W2).
	Suppose the following experiment is performed. One of the two urns is chosen
	at random. Next a ball is randomly chosen from the urn. Then a second ball is
	chosen at random from the same urn without replacing the first ball.
	
	\begin{enumerate}
	\item What is the probability that two black balls are chosen?
	
	\item What is the probability that two balls of opposite colour are chosen?
	\end{enumerate}
	\solution
	%\begin{align}
    \label{eq:12.13.6.18.1}
	\because	\pr{A|B} &> \pr{A},\
\frac{\pr{AB}}{\pr{B}} > \pr{A}
\\
    \label{eq:12.13.6.18.2}
	\implies \pr{AB} &> \pr{A}\pr{B}
	\\
	\text{or, } \frac{\pr{AB}}{\pr{A}} &=\pr{B|A} > \pr{A}
\end{align}

\end{enumerate}

\item In a certain lottery 10,000 tickets are sold and ten equal prizes are awarded. What is the probability of not getting a prize if you buy (a) one ticket (b) two tickets (c) 10 tickets ?	
\\
\solution
		%\begin{enumerate}[label=\thesection.\arabic*,ref=\thesection.\theenumi]
	\item One card is drawn from a well-shuffled deck of 52 cards. Find the probability of getting
\begin{enumerate}
\item A king of red colour 
\item A face card 
\item A red face card
\item The jack of hearts
\item A spade
\item The queen of diamonds

\end{enumerate}
\solution
		%\begin{table}[H]
	\centering
\begin{tabular}{|c|c|c|}
\hline
Random variable &Value &Definition\\ \hline
\multirow{3}{*}{X} &0 &Slips of Rs 1\\
&1 &Slips of Rs 5\\
&2 &Slips of Rs 13\\ \hline
\multirow{2}{*}{Y} &0 &Box A\\
&1 &Box B\\\hline
\end{tabular}
\caption{}
\label{tab:Distribution}
\end{table}
See \tabref{tab:Distribution}.
\begin{align}
p_{Y}\brak{k}= \begin{cases} 
      \frac{1}{3} & {k=0} \\
      \frac{2}{3 }& {k=1} 
   \end{cases}
   \\
p_{Y|X}\brak{0|0} = \frac{19}{25}\, 
p_{Y|X}\brak{0|1} = \frac{6}{25}\,
p_{Y|X}\brak{1|0} = \frac{45}{50}\,
p_{Y|X}\brak{1|2} = \frac{5}{50}
\end{align}
The desired probability is the probability that a slip drawn at random is marked other than Rs 1,
\begin{align}
&=1-p_X\brak{0}\\
&= p_X(1) + p_X(2)
\end{align}
Using Bayes theorem,
\begin{align}
&= p_Y\brak{0} \times \pr{Y=0 | X=1} + p_Y\brak{1} \times \pr{Y=1|X=2}\\
&=\frac{1}{3} \times \frac{6}{25} + \frac{2}{3} \times \frac{5}{50}\\
&=\frac{11}{75}
\end{align}

\newpage

%\tableofcontents

\bigskip

\renewcommand{\thefigure}{\theenumi}
\renewcommand{\thetable}{\theenumi}
%\renewcommand{\theequation}{\theenumi}

%\begin{abstract}
%%\boldmath
%In this letter, an algorithm for evaluating the exact analytical bit error rate  (BER)  for the piecewise linear (PL) combiner for  multiple relays is presented. Previous results were available only for upto three relays. The algorithm is unique in the sense that  the actual mathematical expressions, that are prohibitively large, need not be explicitly obtained. The diversity gain due to multiple relays is shown through plots of the analytical BER, well supported by simulations. 
%
%\end{abstract}
% IEEEtran.cls defaults to using nonbold math in the Abstract.
% This preserves the distinction between vectors and scalars. However,
% if the journal you are submitting to favors bold math in the abstract,
% then you can use LaTeX's standard command \boldmath at the very start
% of the abstract to achieve this. Many IEEE journals frown on math
% in the abstract anyway.

% Note that keywords are not normally used for peerreview papers.
%\begin{IEEEkeywords}
%Cooperative diversity, decode and forward, piecewise linear
%\end{IEEEkeywords}



% For peer review papers, you can put extra information on the cover
% page as needed:
% \ifCLASSOPTIONpeerreview
% \begin{center} \bfseries EDICS Category: 3-BBND \end{center}
% \fi
%
% For peerreview papers, this IEEEtran command inserts a page break and
% creates the second title. It will be ignored for other modes.
%\IEEEpeerreviewmaketitle




	\item Five cards—the ten, jack, queen, king and ace of diamonds, are well-shuffled with their face downwards. One card is then picked up at random.
\begin{enumerate}
\item
What is the probability that the card is the queen? 
\item
If the queen is drawn and put aside, what is the probability that the second card picked up is (a) an ace? (b) a queen?\\
\end{enumerate}
\solution
		%\begin{enumerate}[label=\thesection.\arabic*,ref=\thesection.\theenumi]
	\item One card is drawn from a well-shuffled deck of 52 cards. Find the probability of getting
\begin{enumerate}
\item A king of red colour 
\item A face card 
\item A red face card
\item The jack of hearts
\item A spade
\item The queen of diamonds

\end{enumerate}
\solution
		%\begin{table}[H]
	\centering
\begin{tabular}{|c|c|c|}
\hline
Random variable &Value &Definition\\ \hline
\multirow{3}{*}{X} &0 &Slips of Rs 1\\
&1 &Slips of Rs 5\\
&2 &Slips of Rs 13\\ \hline
\multirow{2}{*}{Y} &0 &Box A\\
&1 &Box B\\\hline
\end{tabular}
\caption{}
\label{tab:Distribution}
\end{table}
See \tabref{tab:Distribution}.
\begin{align}
p_{Y}\brak{k}= \begin{cases} 
      \frac{1}{3} & {k=0} \\
      \frac{2}{3 }& {k=1} 
   \end{cases}
   \\
p_{Y|X}\brak{0|0} = \frac{19}{25}\, 
p_{Y|X}\brak{0|1} = \frac{6}{25}\,
p_{Y|X}\brak{1|0} = \frac{45}{50}\,
p_{Y|X}\brak{1|2} = \frac{5}{50}
\end{align}
The desired probability is the probability that a slip drawn at random is marked other than Rs 1,
\begin{align}
&=1-p_X\brak{0}\\
&= p_X(1) + p_X(2)
\end{align}
Using Bayes theorem,
\begin{align}
&= p_Y\brak{0} \times \pr{Y=0 | X=1} + p_Y\brak{1} \times \pr{Y=1|X=2}\\
&=\frac{1}{3} \times \frac{6}{25} + \frac{2}{3} \times \frac{5}{50}\\
&=\frac{11}{75}
\end{align}

\newpage

%\tableofcontents

\bigskip

\renewcommand{\thefigure}{\theenumi}
\renewcommand{\thetable}{\theenumi}
%\renewcommand{\theequation}{\theenumi}

%\begin{abstract}
%%\boldmath
%In this letter, an algorithm for evaluating the exact analytical bit error rate  (BER)  for the piecewise linear (PL) combiner for  multiple relays is presented. Previous results were available only for upto three relays. The algorithm is unique in the sense that  the actual mathematical expressions, that are prohibitively large, need not be explicitly obtained. The diversity gain due to multiple relays is shown through plots of the analytical BER, well supported by simulations. 
%
%\end{abstract}
% IEEEtran.cls defaults to using nonbold math in the Abstract.
% This preserves the distinction between vectors and scalars. However,
% if the journal you are submitting to favors bold math in the abstract,
% then you can use LaTeX's standard command \boldmath at the very start
% of the abstract to achieve this. Many IEEE journals frown on math
% in the abstract anyway.

% Note that keywords are not normally used for peerreview papers.
%\begin{IEEEkeywords}
%Cooperative diversity, decode and forward, piecewise linear
%\end{IEEEkeywords}



% For peer review papers, you can put extra information on the cover
% page as needed:
% \ifCLASSOPTIONpeerreview
% \begin{center} \bfseries EDICS Category: 3-BBND \end{center}
% \fi
%
% For peerreview papers, this IEEEtran command inserts a page break and
% creates the second title. It will be ignored for other modes.
%\IEEEpeerreviewmaketitle




	\item Five cards—the ten, jack, queen, king and ace of diamonds, are well-shuffled with their face downwards. One card is then picked up at random.
\begin{enumerate}
\item
What is the probability that the card is the queen? 
\item
If the queen is drawn and put aside, what is the probability that the second card picked up is (a) an ace? (b) a queen?\\
\end{enumerate}
\solution
		%\begin{enumerate}[label=\thesection.\arabic*,ref=\thesection.\theenumi]
	\item One card is drawn from a well-shuffled deck of 52 cards. Find the probability of getting
\begin{enumerate}
\item A king of red colour 
\item A face card 
\item A red face card
\item The jack of hearts
\item A spade
\item The queen of diamonds

\end{enumerate}
\solution
		%\input{ncert/10/15/1/14/main.tex}
	\item Five cards—the ten, jack, queen, king and ace of diamonds, are well-shuffled with their face downwards. One card is then picked up at random.
\begin{enumerate}
\item
What is the probability that the card is the queen? 
\item
If the queen is drawn and put aside, what is the probability that the second card picked up is (a) an ace? (b) a queen?\\
\end{enumerate}
\solution
		%\input{ncert/10/15/1/15/defs.tex}
	\item A bag contains $5$ red balls and some blue balls. If the probability of drawing a blue ball is double that if a red ball, determine the number of blue balls in the bag. 
		\\
\solution
		%\input{ncert/10/15/2/3/defs.tex}
	\item A card is selected from a pack of 52 cards.
 \begin{enumerate}[label=(\alph*)] 
                 \item How many points are there in the sample space?
                 \item Calculate the probability that the card is an ace of spades.
                 \item Calculate the probability that the card is (i) an ace and (ii) black card.
 \end{enumerate}
\solution
		%\input{ncert/11/16/3/4/main.tex}
\item Four cards are drawn from a well-shuffled deck of 52 cards. What is the probability of obtaining 3 diamonds and one spade.
\\
\solution
		%\input{ncert/11/16/4/2/defs.tex}
\item In a certain lottery 10,000 tickets are sold and ten equal prizes are awarded. What is the probability of not getting a prize if you buy (a) one ticket (b) two tickets (c) 10 tickets ?	
\\
\solution
		%\input{ncert/11/16/4/4/defs.tex}
		%
\item 
Out of 100 students, two sections of 40 and 60 are formed. If you and your friend are among the 100 students, what is the probability that
\begin{enumerate}
\item you both enter the same section?
\item you both enter the different sections?
\end{enumerate}
\solution
		%\input{ncert/11/16/4/5/defs.tex}
	\item 
The number lock of a suitcase has 4 wheels each labelled with ten digits i.e. from 0 to 9.The lock opens with a sequence of four digits with no repeats.What is the probability of a person getting the right sequence to open the suitcase.
\\
\solution
		%\input{ncert/11/16/4/10/defs.tex}
		%
\item 
Two cards are drawn at random and without replacement from a pack of 52 playing cards. Find the probability that both the cards are black.
\\
\solution
		%\input{ncert/12/13/2/2/defs.tex}
		\item A box of oranges is inspected by examining three randomly selected oranges drawn without replacement. If all the three oranges are good, the box is approved for sale, otherwise, it is rejected. Find the probability that a box containing 15 oranges out of which 12 are good and 3 are bad ones will be approved for sale.
		\label{ncert/12/13/2/3/defs.tex}
		\item Two balls are drawn at random with replacement from a box containing 10 black and 8 red balls. Find the probability that
		\label{ncert/12/13/2/12}
\begin{enumerate}
\item both balls are red.
\item first ball is black and second is red.
\item one of them is black and other is red.
\end{enumerate}

\item In a hostel, 60\% of the students read Hindi newspaper, 40\% read English newspaper and 20\% read both Hindi and English newspapers. A student is selected at random.
		\label{ncert/12/13/2/15}
\begin{enumerate}
\item Find the probability that she reads neither Hindi nor English newspapers.
\item If she reads Hindi newspaper, find the probability that she reads English newspaper.
\item If she reads English newspaper, find the probability that she reads Hindi newspaper.\\
\end{enumerate}
\item The probability of obtaining an even prime number on each die, when a pair of dice is rolled is 
\begin{enumerate}
    \item $0$ 
    
    \item $\frac{1}{3}$ 
    
    \item $\frac{1}{12}$ 
    
    \item $\frac{1}{36}$ 
\end{enumerate}
\solution
		%\input{ncert/12/13/2/17/defs.tex}
	\item A bag contains 4 red and 4 black balls, another bag contains 2 red and 6 black balls. One of the two bags is selected at random and a ball is drawn from the bag which is found to be red. Find the probability that the ball is drawn from the first bag.
\\
\solution
		%\input{ncert/12/13/3/2/main.tex}
  \item
  Cards with numbers 2 to 101 are placed in a box. A card is selected at random.Find the probability that the card has
\begin{enumerate}[label=(\roman*)]
	\item an even number 
	\item a square number
\end{enumerate}
\solution
%\input{exemplar/10/13/3/32/main.tex}
\item
The king, queen and jack of clubs are removed from a deck of 52 playing cards and then well shuffled. Now one card is drawn at random from the remaining cards.  Determine the probability that the card is
\begin{enumerate}[label=(\roman*)]
\item a club
\item 10 of hearts
\end{enumerate}
\solution
%\input{exemplar/10/13/3/29/main.tex}
\item A team of medical students doing their internship have to assist during surgeries
at a city hospital. The probabilities of surgeries rated as very complex, complex,
routine, simple or very simple are respectively, 0.15, 0.20, 0.31, 0.26, .08. Find
the probabilities that a particular surgery will be rated
\begin{enumerate}
	\item complex or very complex;
	\item neither very complex nor very simple;
	\item routine or complex
	\item routine or simple
\end{enumerate}
\solution
%\input{exemplar/11/16/3/8(1)/main.tex}
\item A card is selected from a pack of 52 cards.
\begin{enumerate}[label=(\alph*)]
    \item How many points are there in the sample space?
    \item Calculate the probability that the card is an ace of spades.
    \item Calculate the probability that the card is (i) an ace and (ii) black card.
\end{enumerate}
\solution
%\input{exemplar/11/16/3/4/main2.tex}
\item The probability that a non leap year selected at random will contain 53 sundays.
\\
\solution
%\input{exemplar/10/13/1/19/main.tex}
\item One of the four persons John, Rita, Aslam or Gurpreet will be promoted next
month. Consequently the sample space consists of four elementary outcomes
S = {John promoted, Rita promoted, Aslam promoted, Gurpreet promoted}
You are told that the chances of John’s promotion is same as that of Gurpreet,
Rita’s chances of promotion are twice as likely as Johns. Aslam’s chances are
four times that of John.
\begin{enumerate}
	\item Determine
	\begin{enumerate}
		\item P (John promoted)
		\item P (Rita promoted)
		\item P (Aslam promoted)
		\item P (Gurpreet promoted)
	\end{enumerate}
	\item If A = {John promoted or Gurpreet promoted}, find P (A).
\end{enumerate}
\solution
%\input{exemplar/11/16/3/10/main.tex}
\item A card is drawn from a deck of 52 cards. Find the probability of getting a king or a heart or a red card.\\
\solution
%\input{exemplar/11/16/3/15/main.tex}
\item The probability that a student will pass his examination is 0.73, the probability of
the student getting a compartment is 0.13, and the probability that the student will
either pass or get compartment is 0.96. State True or False.\\
\solution
%\input{exemplar/11/16/3/31/main.tex}
\item A card is selected from a pack of 52 cards\\
\begin{enumerate}[label=(\alph*)]
\item How many points are there in the sample space?
\item Calculate the probability that the cards is an ace of spades.
\item Calculate the probability that the card is (i) an ace (ii)black card.\\
\end{enumerate}
%\input{ncert/11/16/3/4_1/Prob_4.tex}
\item In a non-leap year, the probability of having 53 tuesdays or 53 wednesdays is\\
\solution
%\input{exemplar/11/16/3/18/main.tex}
\item There are 1000 sealed envelopes in a box, 10 of them contain a cash prize of
Rs 100 each, 100 of them contain a cash prize of Rs 50 each and 200 of them
contain a cash prize of Rs 10 each and rest do not contain any cash prize. If they
are well shuffled and an envelope is picked up out, what is the probability that it
contains no cash prize?\\
\solution
%\input{exemplar/10/13/3/34/main.tex}
\item 
A die is thrown and a card is selected at random from a deck of 52 playing cards. The probability of getting an even number on the die and a spade card.\\
\solution
%\input{exemplar/12/13/3/78/main.tex}
\item
If 4-digit numbers greater than 5,000 are randomly formed from the digits 0, 1, 3, 5, and 7, what is the probability of forming a number divisible by 5 when:
\begin{enumerate}
    \item The digits are repeated?
    \item The repetition of digits is not allowed?
\end{enumerate}
\solution
%\input{ncert/11/16/4/9/main.tex}
\item Consider the probability space $\brak{\Omega, \mathcal{G}, P}$ where $\Omega = [0,2]$ and $\mathcal{G} = \cbrak{\phi, \Omega, [0,1], (1,2]}$. Let $X$ and $Y$ be two functions on $\Omega$ defined as
\begin{align*}
    X(\omega) = 
    \begin{cases}
        1 & \text{if }\omega \in [0, 1]\\
        2 & \text{if }\omega \in (1, 2]
    \end{cases}
\end{align*}
and
\begin{align*}
    Y(\omega) = 
    \begin{cases}
        2 & \text{if }\omega \in [0, 1.5]\\
        3 & \text{if }\omega \in (1.5, 2].
    \end{cases}
\end{align*}
Then which one of the following statements is true?
\begin{enumerate}
    \item [(A)] $X$ is a random variable with respect to $\mathcal{G}$, but $Y$ is not a random variable with respect to $\mathcal{G}$.
    \item [(B)] $Y$ is a random variable with respect to $\mathcal{G}$, but $X$ is not a random variable with respect to $\mathcal{G}$.
    \item [(C)] Neither $X$ nor $Y$ is a random variable with respect to $\mathcal{G}$.
    \item [(D)] Both $X$ and $Y$ are random variables with respect to $\mathcal{G}$.
\end{enumerate} \hfill (GATE ST 2023)\\
\solution
%\input{gate/ST/2023/14/main.tex}
	\item  A die is loaded in such a way that each odd number is twice as likely to occur as
each even number. Find $P(G)$, where $G$ is the event that a number greater than
3 occurs on a single roll of the die.
\\
\solution
		%\input{exemplar/11/16/3/5/main.tex}
	\item All the jacks, queens and kings are removed from a deck of 52 playing cards. The remaining cards are well shuffled and then one card is drawn at random. Giving ace a value 1 similar value for other cards, find the probability that the card has a value 
		\begin{enumerate}
			\item 7
			\item greater than 7
			\item less than 7
		\end{enumerate}
		%\input{exemplar/10/13/3/30/main.tex}
  \item A Lot consists of 48 mobile phones of which 42 are good, 3 have only minor defects and 3 have major defects.Varnika will buy a phone if it is good but the trader will only buy a mobile if it has no major defects. One phone is selected at random from the lot. What is the probability that it is
\begin{enumerate}
	\item acceptable to Varnika?
            \item acceptable to the trader?
\end{enumerate}
\solution
	%\input{exemplar/10/13/3/40/main.tex}
 \item A student says that if you throw a die, it will show up 1 or not 1. Therefore, the probability of getting 1 and the probability of getting 'not 1' each is equal to $\frac{1}{2}$. Is this correct? Give reasons.\\
 \solution
        %\input{exemplar/10/13/2/9/main.tex}
   \item Four candidates A, B, C, D have ap-
plied for the assignment to coach a school cricket
team. If A is twice as likely to be selected as B, and
B and C are given about the same chance of being
selected, while C is twice as likely to be selected
as D, what are the probabilities that
\begin{enumerate}
\item C will be selected?
\item A will not be selected?
\end{enumerate}
	%\input{exemplar/11/16/3/9/main.tex}
 \item A bag contain 24 balls of which $x$ balls are red, $2x$ are white and $3x$ are blue. A ball is selected at random, What is the probability that it is
\begin{enumerate}[label=\alph*)]
\item not red ?
\item white ?
\end{enumerate}
%\input{exemplar/10/13/3/41/main.tex}
If the letters of the word ASSASSINATION are arranged at random. Find the Probability that
\begin{enumerate}[label=(\alph*)]
\item Four $S's$ come consecutively in the word
\item Two  $I's$ and two $N's$ come together
\item All $A's$ are not coming together
\item No two $A's$ are coming together
\end{enumerate}
%\input{exemplar/11/16/3/14/main.tex}
	\item One urn contains two black balls (labelled B1 and B2) and one white ball. A
	second urn contains one black ball and two white balls (labelled W1 and W2).
	Suppose the following experiment is performed. One of the two urns is chosen
	at random. Next a ball is randomly chosen from the urn. Then a second ball is
	chosen at random from the same urn without replacing the first ball.
	
	\begin{enumerate}
	\item What is the probability that two black balls are chosen?
	
	\item What is the probability that two balls of opposite colour are chosen?
	\end{enumerate}
	\solution
	%\input{exemplar/11/16/3/12/main1.tex}
\end{enumerate}

	\item A bag contains $5$ red balls and some blue balls. If the probability of drawing a blue ball is double that if a red ball, determine the number of blue balls in the bag. 
		\\
\solution
		%\begin{enumerate}[label=\thesection.\arabic*,ref=\thesection.\theenumi]
	\item One card is drawn from a well-shuffled deck of 52 cards. Find the probability of getting
\begin{enumerate}
\item A king of red colour 
\item A face card 
\item A red face card
\item The jack of hearts
\item A spade
\item The queen of diamonds

\end{enumerate}
\solution
		%\input{ncert/10/15/1/14/main.tex}
	\item Five cards—the ten, jack, queen, king and ace of diamonds, are well-shuffled with their face downwards. One card is then picked up at random.
\begin{enumerate}
\item
What is the probability that the card is the queen? 
\item
If the queen is drawn and put aside, what is the probability that the second card picked up is (a) an ace? (b) a queen?\\
\end{enumerate}
\solution
		%\input{ncert/10/15/1/15/defs.tex}
	\item A bag contains $5$ red balls and some blue balls. If the probability of drawing a blue ball is double that if a red ball, determine the number of blue balls in the bag. 
		\\
\solution
		%\input{ncert/10/15/2/3/defs.tex}
	\item A card is selected from a pack of 52 cards.
 \begin{enumerate}[label=(\alph*)] 
                 \item How many points are there in the sample space?
                 \item Calculate the probability that the card is an ace of spades.
                 \item Calculate the probability that the card is (i) an ace and (ii) black card.
 \end{enumerate}
\solution
		%\input{ncert/11/16/3/4/main.tex}
\item Four cards are drawn from a well-shuffled deck of 52 cards. What is the probability of obtaining 3 diamonds and one spade.
\\
\solution
		%\input{ncert/11/16/4/2/defs.tex}
\item In a certain lottery 10,000 tickets are sold and ten equal prizes are awarded. What is the probability of not getting a prize if you buy (a) one ticket (b) two tickets (c) 10 tickets ?	
\\
\solution
		%\input{ncert/11/16/4/4/defs.tex}
		%
\item 
Out of 100 students, two sections of 40 and 60 are formed. If you and your friend are among the 100 students, what is the probability that
\begin{enumerate}
\item you both enter the same section?
\item you both enter the different sections?
\end{enumerate}
\solution
		%\input{ncert/11/16/4/5/defs.tex}
	\item 
The number lock of a suitcase has 4 wheels each labelled with ten digits i.e. from 0 to 9.The lock opens with a sequence of four digits with no repeats.What is the probability of a person getting the right sequence to open the suitcase.
\\
\solution
		%\input{ncert/11/16/4/10/defs.tex}
		%
\item 
Two cards are drawn at random and without replacement from a pack of 52 playing cards. Find the probability that both the cards are black.
\\
\solution
		%\input{ncert/12/13/2/2/defs.tex}
		\item A box of oranges is inspected by examining three randomly selected oranges drawn without replacement. If all the three oranges are good, the box is approved for sale, otherwise, it is rejected. Find the probability that a box containing 15 oranges out of which 12 are good and 3 are bad ones will be approved for sale.
		\label{ncert/12/13/2/3/defs.tex}
		\item Two balls are drawn at random with replacement from a box containing 10 black and 8 red balls. Find the probability that
		\label{ncert/12/13/2/12}
\begin{enumerate}
\item both balls are red.
\item first ball is black and second is red.
\item one of them is black and other is red.
\end{enumerate}

\item In a hostel, 60\% of the students read Hindi newspaper, 40\% read English newspaper and 20\% read both Hindi and English newspapers. A student is selected at random.
		\label{ncert/12/13/2/15}
\begin{enumerate}
\item Find the probability that she reads neither Hindi nor English newspapers.
\item If she reads Hindi newspaper, find the probability that she reads English newspaper.
\item If she reads English newspaper, find the probability that she reads Hindi newspaper.\\
\end{enumerate}
\item The probability of obtaining an even prime number on each die, when a pair of dice is rolled is 
\begin{enumerate}
    \item $0$ 
    
    \item $\frac{1}{3}$ 
    
    \item $\frac{1}{12}$ 
    
    \item $\frac{1}{36}$ 
\end{enumerate}
\solution
		%\input{ncert/12/13/2/17/defs.tex}
	\item A bag contains 4 red and 4 black balls, another bag contains 2 red and 6 black balls. One of the two bags is selected at random and a ball is drawn from the bag which is found to be red. Find the probability that the ball is drawn from the first bag.
\\
\solution
		%\input{ncert/12/13/3/2/main.tex}
  \item
  Cards with numbers 2 to 101 are placed in a box. A card is selected at random.Find the probability that the card has
\begin{enumerate}[label=(\roman*)]
	\item an even number 
	\item a square number
\end{enumerate}
\solution
%\input{exemplar/10/13/3/32/main.tex}
\item
The king, queen and jack of clubs are removed from a deck of 52 playing cards and then well shuffled. Now one card is drawn at random from the remaining cards.  Determine the probability that the card is
\begin{enumerate}[label=(\roman*)]
\item a club
\item 10 of hearts
\end{enumerate}
\solution
%\input{exemplar/10/13/3/29/main.tex}
\item A team of medical students doing their internship have to assist during surgeries
at a city hospital. The probabilities of surgeries rated as very complex, complex,
routine, simple or very simple are respectively, 0.15, 0.20, 0.31, 0.26, .08. Find
the probabilities that a particular surgery will be rated
\begin{enumerate}
	\item complex or very complex;
	\item neither very complex nor very simple;
	\item routine or complex
	\item routine or simple
\end{enumerate}
\solution
%\input{exemplar/11/16/3/8(1)/main.tex}
\item A card is selected from a pack of 52 cards.
\begin{enumerate}[label=(\alph*)]
    \item How many points are there in the sample space?
    \item Calculate the probability that the card is an ace of spades.
    \item Calculate the probability that the card is (i) an ace and (ii) black card.
\end{enumerate}
\solution
%\input{exemplar/11/16/3/4/main2.tex}
\item The probability that a non leap year selected at random will contain 53 sundays.
\\
\solution
%\input{exemplar/10/13/1/19/main.tex}
\item One of the four persons John, Rita, Aslam or Gurpreet will be promoted next
month. Consequently the sample space consists of four elementary outcomes
S = {John promoted, Rita promoted, Aslam promoted, Gurpreet promoted}
You are told that the chances of John’s promotion is same as that of Gurpreet,
Rita’s chances of promotion are twice as likely as Johns. Aslam’s chances are
four times that of John.
\begin{enumerate}
	\item Determine
	\begin{enumerate}
		\item P (John promoted)
		\item P (Rita promoted)
		\item P (Aslam promoted)
		\item P (Gurpreet promoted)
	\end{enumerate}
	\item If A = {John promoted or Gurpreet promoted}, find P (A).
\end{enumerate}
\solution
%\input{exemplar/11/16/3/10/main.tex}
\item A card is drawn from a deck of 52 cards. Find the probability of getting a king or a heart or a red card.\\
\solution
%\input{exemplar/11/16/3/15/main.tex}
\item The probability that a student will pass his examination is 0.73, the probability of
the student getting a compartment is 0.13, and the probability that the student will
either pass or get compartment is 0.96. State True or False.\\
\solution
%\input{exemplar/11/16/3/31/main.tex}
\item A card is selected from a pack of 52 cards\\
\begin{enumerate}[label=(\alph*)]
\item How many points are there in the sample space?
\item Calculate the probability that the cards is an ace of spades.
\item Calculate the probability that the card is (i) an ace (ii)black card.\\
\end{enumerate}
%\input{ncert/11/16/3/4_1/Prob_4.tex}
\item In a non-leap year, the probability of having 53 tuesdays or 53 wednesdays is\\
\solution
%\input{exemplar/11/16/3/18/main.tex}
\item There are 1000 sealed envelopes in a box, 10 of them contain a cash prize of
Rs 100 each, 100 of them contain a cash prize of Rs 50 each and 200 of them
contain a cash prize of Rs 10 each and rest do not contain any cash prize. If they
are well shuffled and an envelope is picked up out, what is the probability that it
contains no cash prize?\\
\solution
%\input{exemplar/10/13/3/34/main.tex}
\item 
A die is thrown and a card is selected at random from a deck of 52 playing cards. The probability of getting an even number on the die and a spade card.\\
\solution
%\input{exemplar/12/13/3/78/main.tex}
\item
If 4-digit numbers greater than 5,000 are randomly formed from the digits 0, 1, 3, 5, and 7, what is the probability of forming a number divisible by 5 when:
\begin{enumerate}
    \item The digits are repeated?
    \item The repetition of digits is not allowed?
\end{enumerate}
\solution
%\input{ncert/11/16/4/9/main.tex}
\item Consider the probability space $\brak{\Omega, \mathcal{G}, P}$ where $\Omega = [0,2]$ and $\mathcal{G} = \cbrak{\phi, \Omega, [0,1], (1,2]}$. Let $X$ and $Y$ be two functions on $\Omega$ defined as
\begin{align*}
    X(\omega) = 
    \begin{cases}
        1 & \text{if }\omega \in [0, 1]\\
        2 & \text{if }\omega \in (1, 2]
    \end{cases}
\end{align*}
and
\begin{align*}
    Y(\omega) = 
    \begin{cases}
        2 & \text{if }\omega \in [0, 1.5]\\
        3 & \text{if }\omega \in (1.5, 2].
    \end{cases}
\end{align*}
Then which one of the following statements is true?
\begin{enumerate}
    \item [(A)] $X$ is a random variable with respect to $\mathcal{G}$, but $Y$ is not a random variable with respect to $\mathcal{G}$.
    \item [(B)] $Y$ is a random variable with respect to $\mathcal{G}$, but $X$ is not a random variable with respect to $\mathcal{G}$.
    \item [(C)] Neither $X$ nor $Y$ is a random variable with respect to $\mathcal{G}$.
    \item [(D)] Both $X$ and $Y$ are random variables with respect to $\mathcal{G}$.
\end{enumerate} \hfill (GATE ST 2023)\\
\solution
%\input{gate/ST/2023/14/main.tex}
	\item  A die is loaded in such a way that each odd number is twice as likely to occur as
each even number. Find $P(G)$, where $G$ is the event that a number greater than
3 occurs on a single roll of the die.
\\
\solution
		%\input{exemplar/11/16/3/5/main.tex}
	\item All the jacks, queens and kings are removed from a deck of 52 playing cards. The remaining cards are well shuffled and then one card is drawn at random. Giving ace a value 1 similar value for other cards, find the probability that the card has a value 
		\begin{enumerate}
			\item 7
			\item greater than 7
			\item less than 7
		\end{enumerate}
		%\input{exemplar/10/13/3/30/main.tex}
  \item A Lot consists of 48 mobile phones of which 42 are good, 3 have only minor defects and 3 have major defects.Varnika will buy a phone if it is good but the trader will only buy a mobile if it has no major defects. One phone is selected at random from the lot. What is the probability that it is
\begin{enumerate}
	\item acceptable to Varnika?
            \item acceptable to the trader?
\end{enumerate}
\solution
	%\input{exemplar/10/13/3/40/main.tex}
 \item A student says that if you throw a die, it will show up 1 or not 1. Therefore, the probability of getting 1 and the probability of getting 'not 1' each is equal to $\frac{1}{2}$. Is this correct? Give reasons.\\
 \solution
        %\input{exemplar/10/13/2/9/main.tex}
   \item Four candidates A, B, C, D have ap-
plied for the assignment to coach a school cricket
team. If A is twice as likely to be selected as B, and
B and C are given about the same chance of being
selected, while C is twice as likely to be selected
as D, what are the probabilities that
\begin{enumerate}
\item C will be selected?
\item A will not be selected?
\end{enumerate}
	%\input{exemplar/11/16/3/9/main.tex}
 \item A bag contain 24 balls of which $x$ balls are red, $2x$ are white and $3x$ are blue. A ball is selected at random, What is the probability that it is
\begin{enumerate}[label=\alph*)]
\item not red ?
\item white ?
\end{enumerate}
%\input{exemplar/10/13/3/41/main.tex}
If the letters of the word ASSASSINATION are arranged at random. Find the Probability that
\begin{enumerate}[label=(\alph*)]
\item Four $S's$ come consecutively in the word
\item Two  $I's$ and two $N's$ come together
\item All $A's$ are not coming together
\item No two $A's$ are coming together
\end{enumerate}
%\input{exemplar/11/16/3/14/main.tex}
	\item One urn contains two black balls (labelled B1 and B2) and one white ball. A
	second urn contains one black ball and two white balls (labelled W1 and W2).
	Suppose the following experiment is performed. One of the two urns is chosen
	at random. Next a ball is randomly chosen from the urn. Then a second ball is
	chosen at random from the same urn without replacing the first ball.
	
	\begin{enumerate}
	\item What is the probability that two black balls are chosen?
	
	\item What is the probability that two balls of opposite colour are chosen?
	\end{enumerate}
	\solution
	%\input{exemplar/11/16/3/12/main1.tex}
\end{enumerate}

	\item A card is selected from a pack of 52 cards.
 \begin{enumerate}[label=(\alph*)] 
                 \item How many points are there in the sample space?
                 \item Calculate the probability that the card is an ace of spades.
                 \item Calculate the probability that the card is (i) an ace and (ii) black card.
 \end{enumerate}
\solution
		%\begin{table}[H]
	\centering
\begin{tabular}{|c|c|c|}
\hline
Random variable &Value &Definition\\ \hline
\multirow{3}{*}{X} &0 &Slips of Rs 1\\
&1 &Slips of Rs 5\\
&2 &Slips of Rs 13\\ \hline
\multirow{2}{*}{Y} &0 &Box A\\
&1 &Box B\\\hline
\end{tabular}
\caption{}
\label{tab:Distribution}
\end{table}
See \tabref{tab:Distribution}.
\begin{align}
p_{Y}\brak{k}= \begin{cases} 
      \frac{1}{3} & {k=0} \\
      \frac{2}{3 }& {k=1} 
   \end{cases}
   \\
p_{Y|X}\brak{0|0} = \frac{19}{25}\, 
p_{Y|X}\brak{0|1} = \frac{6}{25}\,
p_{Y|X}\brak{1|0} = \frac{45}{50}\,
p_{Y|X}\brak{1|2} = \frac{5}{50}
\end{align}
The desired probability is the probability that a slip drawn at random is marked other than Rs 1,
\begin{align}
&=1-p_X\brak{0}\\
&= p_X(1) + p_X(2)
\end{align}
Using Bayes theorem,
\begin{align}
&= p_Y\brak{0} \times \pr{Y=0 | X=1} + p_Y\brak{1} \times \pr{Y=1|X=2}\\
&=\frac{1}{3} \times \frac{6}{25} + \frac{2}{3} \times \frac{5}{50}\\
&=\frac{11}{75}
\end{align}

\newpage

%\tableofcontents

\bigskip

\renewcommand{\thefigure}{\theenumi}
\renewcommand{\thetable}{\theenumi}
%\renewcommand{\theequation}{\theenumi}

%\begin{abstract}
%%\boldmath
%In this letter, an algorithm for evaluating the exact analytical bit error rate  (BER)  for the piecewise linear (PL) combiner for  multiple relays is presented. Previous results were available only for upto three relays. The algorithm is unique in the sense that  the actual mathematical expressions, that are prohibitively large, need not be explicitly obtained. The diversity gain due to multiple relays is shown through plots of the analytical BER, well supported by simulations. 
%
%\end{abstract}
% IEEEtran.cls defaults to using nonbold math in the Abstract.
% This preserves the distinction between vectors and scalars. However,
% if the journal you are submitting to favors bold math in the abstract,
% then you can use LaTeX's standard command \boldmath at the very start
% of the abstract to achieve this. Many IEEE journals frown on math
% in the abstract anyway.

% Note that keywords are not normally used for peerreview papers.
%\begin{IEEEkeywords}
%Cooperative diversity, decode and forward, piecewise linear
%\end{IEEEkeywords}



% For peer review papers, you can put extra information on the cover
% page as needed:
% \ifCLASSOPTIONpeerreview
% \begin{center} \bfseries EDICS Category: 3-BBND \end{center}
% \fi
%
% For peerreview papers, this IEEEtran command inserts a page break and
% creates the second title. It will be ignored for other modes.
%\IEEEpeerreviewmaketitle




\item Four cards are drawn from a well-shuffled deck of 52 cards. What is the probability of obtaining 3 diamonds and one spade.
\\
\solution
		%\begin{enumerate}[label=\thesection.\arabic*,ref=\thesection.\theenumi]
	\item One card is drawn from a well-shuffled deck of 52 cards. Find the probability of getting
\begin{enumerate}
\item A king of red colour 
\item A face card 
\item A red face card
\item The jack of hearts
\item A spade
\item The queen of diamonds

\end{enumerate}
\solution
		%\input{ncert/10/15/1/14/main.tex}
	\item Five cards—the ten, jack, queen, king and ace of diamonds, are well-shuffled with their face downwards. One card is then picked up at random.
\begin{enumerate}
\item
What is the probability that the card is the queen? 
\item
If the queen is drawn and put aside, what is the probability that the second card picked up is (a) an ace? (b) a queen?\\
\end{enumerate}
\solution
		%\input{ncert/10/15/1/15/defs.tex}
	\item A bag contains $5$ red balls and some blue balls. If the probability of drawing a blue ball is double that if a red ball, determine the number of blue balls in the bag. 
		\\
\solution
		%\input{ncert/10/15/2/3/defs.tex}
	\item A card is selected from a pack of 52 cards.
 \begin{enumerate}[label=(\alph*)] 
                 \item How many points are there in the sample space?
                 \item Calculate the probability that the card is an ace of spades.
                 \item Calculate the probability that the card is (i) an ace and (ii) black card.
 \end{enumerate}
\solution
		%\input{ncert/11/16/3/4/main.tex}
\item Four cards are drawn from a well-shuffled deck of 52 cards. What is the probability of obtaining 3 diamonds and one spade.
\\
\solution
		%\input{ncert/11/16/4/2/defs.tex}
\item In a certain lottery 10,000 tickets are sold and ten equal prizes are awarded. What is the probability of not getting a prize if you buy (a) one ticket (b) two tickets (c) 10 tickets ?	
\\
\solution
		%\input{ncert/11/16/4/4/defs.tex}
		%
\item 
Out of 100 students, two sections of 40 and 60 are formed. If you and your friend are among the 100 students, what is the probability that
\begin{enumerate}
\item you both enter the same section?
\item you both enter the different sections?
\end{enumerate}
\solution
		%\input{ncert/11/16/4/5/defs.tex}
	\item 
The number lock of a suitcase has 4 wheels each labelled with ten digits i.e. from 0 to 9.The lock opens with a sequence of four digits with no repeats.What is the probability of a person getting the right sequence to open the suitcase.
\\
\solution
		%\input{ncert/11/16/4/10/defs.tex}
		%
\item 
Two cards are drawn at random and without replacement from a pack of 52 playing cards. Find the probability that both the cards are black.
\\
\solution
		%\input{ncert/12/13/2/2/defs.tex}
		\item A box of oranges is inspected by examining three randomly selected oranges drawn without replacement. If all the three oranges are good, the box is approved for sale, otherwise, it is rejected. Find the probability that a box containing 15 oranges out of which 12 are good and 3 are bad ones will be approved for sale.
		\label{ncert/12/13/2/3/defs.tex}
		\item Two balls are drawn at random with replacement from a box containing 10 black and 8 red balls. Find the probability that
		\label{ncert/12/13/2/12}
\begin{enumerate}
\item both balls are red.
\item first ball is black and second is red.
\item one of them is black and other is red.
\end{enumerate}

\item In a hostel, 60\% of the students read Hindi newspaper, 40\% read English newspaper and 20\% read both Hindi and English newspapers. A student is selected at random.
		\label{ncert/12/13/2/15}
\begin{enumerate}
\item Find the probability that she reads neither Hindi nor English newspapers.
\item If she reads Hindi newspaper, find the probability that she reads English newspaper.
\item If she reads English newspaper, find the probability that she reads Hindi newspaper.\\
\end{enumerate}
\item The probability of obtaining an even prime number on each die, when a pair of dice is rolled is 
\begin{enumerate}
    \item $0$ 
    
    \item $\frac{1}{3}$ 
    
    \item $\frac{1}{12}$ 
    
    \item $\frac{1}{36}$ 
\end{enumerate}
\solution
		%\input{ncert/12/13/2/17/defs.tex}
	\item A bag contains 4 red and 4 black balls, another bag contains 2 red and 6 black balls. One of the two bags is selected at random and a ball is drawn from the bag which is found to be red. Find the probability that the ball is drawn from the first bag.
\\
\solution
		%\input{ncert/12/13/3/2/main.tex}
  \item
  Cards with numbers 2 to 101 are placed in a box. A card is selected at random.Find the probability that the card has
\begin{enumerate}[label=(\roman*)]
	\item an even number 
	\item a square number
\end{enumerate}
\solution
%\input{exemplar/10/13/3/32/main.tex}
\item
The king, queen and jack of clubs are removed from a deck of 52 playing cards and then well shuffled. Now one card is drawn at random from the remaining cards.  Determine the probability that the card is
\begin{enumerate}[label=(\roman*)]
\item a club
\item 10 of hearts
\end{enumerate}
\solution
%\input{exemplar/10/13/3/29/main.tex}
\item A team of medical students doing their internship have to assist during surgeries
at a city hospital. The probabilities of surgeries rated as very complex, complex,
routine, simple or very simple are respectively, 0.15, 0.20, 0.31, 0.26, .08. Find
the probabilities that a particular surgery will be rated
\begin{enumerate}
	\item complex or very complex;
	\item neither very complex nor very simple;
	\item routine or complex
	\item routine or simple
\end{enumerate}
\solution
%\input{exemplar/11/16/3/8(1)/main.tex}
\item A card is selected from a pack of 52 cards.
\begin{enumerate}[label=(\alph*)]
    \item How many points are there in the sample space?
    \item Calculate the probability that the card is an ace of spades.
    \item Calculate the probability that the card is (i) an ace and (ii) black card.
\end{enumerate}
\solution
%\input{exemplar/11/16/3/4/main2.tex}
\item The probability that a non leap year selected at random will contain 53 sundays.
\\
\solution
%\input{exemplar/10/13/1/19/main.tex}
\item One of the four persons John, Rita, Aslam or Gurpreet will be promoted next
month. Consequently the sample space consists of four elementary outcomes
S = {John promoted, Rita promoted, Aslam promoted, Gurpreet promoted}
You are told that the chances of John’s promotion is same as that of Gurpreet,
Rita’s chances of promotion are twice as likely as Johns. Aslam’s chances are
four times that of John.
\begin{enumerate}
	\item Determine
	\begin{enumerate}
		\item P (John promoted)
		\item P (Rita promoted)
		\item P (Aslam promoted)
		\item P (Gurpreet promoted)
	\end{enumerate}
	\item If A = {John promoted or Gurpreet promoted}, find P (A).
\end{enumerate}
\solution
%\input{exemplar/11/16/3/10/main.tex}
\item A card is drawn from a deck of 52 cards. Find the probability of getting a king or a heart or a red card.\\
\solution
%\input{exemplar/11/16/3/15/main.tex}
\item The probability that a student will pass his examination is 0.73, the probability of
the student getting a compartment is 0.13, and the probability that the student will
either pass or get compartment is 0.96. State True or False.\\
\solution
%\input{exemplar/11/16/3/31/main.tex}
\item A card is selected from a pack of 52 cards\\
\begin{enumerate}[label=(\alph*)]
\item How many points are there in the sample space?
\item Calculate the probability that the cards is an ace of spades.
\item Calculate the probability that the card is (i) an ace (ii)black card.\\
\end{enumerate}
%\input{ncert/11/16/3/4_1/Prob_4.tex}
\item In a non-leap year, the probability of having 53 tuesdays or 53 wednesdays is\\
\solution
%\input{exemplar/11/16/3/18/main.tex}
\item There are 1000 sealed envelopes in a box, 10 of them contain a cash prize of
Rs 100 each, 100 of them contain a cash prize of Rs 50 each and 200 of them
contain a cash prize of Rs 10 each and rest do not contain any cash prize. If they
are well shuffled and an envelope is picked up out, what is the probability that it
contains no cash prize?\\
\solution
%\input{exemplar/10/13/3/34/main.tex}
\item 
A die is thrown and a card is selected at random from a deck of 52 playing cards. The probability of getting an even number on the die and a spade card.\\
\solution
%\input{exemplar/12/13/3/78/main.tex}
\item
If 4-digit numbers greater than 5,000 are randomly formed from the digits 0, 1, 3, 5, and 7, what is the probability of forming a number divisible by 5 when:
\begin{enumerate}
    \item The digits are repeated?
    \item The repetition of digits is not allowed?
\end{enumerate}
\solution
%\input{ncert/11/16/4/9/main.tex}
\item Consider the probability space $\brak{\Omega, \mathcal{G}, P}$ where $\Omega = [0,2]$ and $\mathcal{G} = \cbrak{\phi, \Omega, [0,1], (1,2]}$. Let $X$ and $Y$ be two functions on $\Omega$ defined as
\begin{align*}
    X(\omega) = 
    \begin{cases}
        1 & \text{if }\omega \in [0, 1]\\
        2 & \text{if }\omega \in (1, 2]
    \end{cases}
\end{align*}
and
\begin{align*}
    Y(\omega) = 
    \begin{cases}
        2 & \text{if }\omega \in [0, 1.5]\\
        3 & \text{if }\omega \in (1.5, 2].
    \end{cases}
\end{align*}
Then which one of the following statements is true?
\begin{enumerate}
    \item [(A)] $X$ is a random variable with respect to $\mathcal{G}$, but $Y$ is not a random variable with respect to $\mathcal{G}$.
    \item [(B)] $Y$ is a random variable with respect to $\mathcal{G}$, but $X$ is not a random variable with respect to $\mathcal{G}$.
    \item [(C)] Neither $X$ nor $Y$ is a random variable with respect to $\mathcal{G}$.
    \item [(D)] Both $X$ and $Y$ are random variables with respect to $\mathcal{G}$.
\end{enumerate} \hfill (GATE ST 2023)\\
\solution
%\input{gate/ST/2023/14/main.tex}
	\item  A die is loaded in such a way that each odd number is twice as likely to occur as
each even number. Find $P(G)$, where $G$ is the event that a number greater than
3 occurs on a single roll of the die.
\\
\solution
		%\input{exemplar/11/16/3/5/main.tex}
	\item All the jacks, queens and kings are removed from a deck of 52 playing cards. The remaining cards are well shuffled and then one card is drawn at random. Giving ace a value 1 similar value for other cards, find the probability that the card has a value 
		\begin{enumerate}
			\item 7
			\item greater than 7
			\item less than 7
		\end{enumerate}
		%\input{exemplar/10/13/3/30/main.tex}
  \item A Lot consists of 48 mobile phones of which 42 are good, 3 have only minor defects and 3 have major defects.Varnika will buy a phone if it is good but the trader will only buy a mobile if it has no major defects. One phone is selected at random from the lot. What is the probability that it is
\begin{enumerate}
	\item acceptable to Varnika?
            \item acceptable to the trader?
\end{enumerate}
\solution
	%\input{exemplar/10/13/3/40/main.tex}
 \item A student says that if you throw a die, it will show up 1 or not 1. Therefore, the probability of getting 1 and the probability of getting 'not 1' each is equal to $\frac{1}{2}$. Is this correct? Give reasons.\\
 \solution
        %\input{exemplar/10/13/2/9/main.tex}
   \item Four candidates A, B, C, D have ap-
plied for the assignment to coach a school cricket
team. If A is twice as likely to be selected as B, and
B and C are given about the same chance of being
selected, while C is twice as likely to be selected
as D, what are the probabilities that
\begin{enumerate}
\item C will be selected?
\item A will not be selected?
\end{enumerate}
	%\input{exemplar/11/16/3/9/main.tex}
 \item A bag contain 24 balls of which $x$ balls are red, $2x$ are white and $3x$ are blue. A ball is selected at random, What is the probability that it is
\begin{enumerate}[label=\alph*)]
\item not red ?
\item white ?
\end{enumerate}
%\input{exemplar/10/13/3/41/main.tex}
If the letters of the word ASSASSINATION are arranged at random. Find the Probability that
\begin{enumerate}[label=(\alph*)]
\item Four $S's$ come consecutively in the word
\item Two  $I's$ and two $N's$ come together
\item All $A's$ are not coming together
\item No two $A's$ are coming together
\end{enumerate}
%\input{exemplar/11/16/3/14/main.tex}
	\item One urn contains two black balls (labelled B1 and B2) and one white ball. A
	second urn contains one black ball and two white balls (labelled W1 and W2).
	Suppose the following experiment is performed. One of the two urns is chosen
	at random. Next a ball is randomly chosen from the urn. Then a second ball is
	chosen at random from the same urn without replacing the first ball.
	
	\begin{enumerate}
	\item What is the probability that two black balls are chosen?
	
	\item What is the probability that two balls of opposite colour are chosen?
	\end{enumerate}
	\solution
	%\input{exemplar/11/16/3/12/main1.tex}
\end{enumerate}

\item In a certain lottery 10,000 tickets are sold and ten equal prizes are awarded. What is the probability of not getting a prize if you buy (a) one ticket (b) two tickets (c) 10 tickets ?	
\\
\solution
		%\begin{enumerate}[label=\thesection.\arabic*,ref=\thesection.\theenumi]
	\item One card is drawn from a well-shuffled deck of 52 cards. Find the probability of getting
\begin{enumerate}
\item A king of red colour 
\item A face card 
\item A red face card
\item The jack of hearts
\item A spade
\item The queen of diamonds

\end{enumerate}
\solution
		%\input{ncert/10/15/1/14/main.tex}
	\item Five cards—the ten, jack, queen, king and ace of diamonds, are well-shuffled with their face downwards. One card is then picked up at random.
\begin{enumerate}
\item
What is the probability that the card is the queen? 
\item
If the queen is drawn and put aside, what is the probability that the second card picked up is (a) an ace? (b) a queen?\\
\end{enumerate}
\solution
		%\input{ncert/10/15/1/15/defs.tex}
	\item A bag contains $5$ red balls and some blue balls. If the probability of drawing a blue ball is double that if a red ball, determine the number of blue balls in the bag. 
		\\
\solution
		%\input{ncert/10/15/2/3/defs.tex}
	\item A card is selected from a pack of 52 cards.
 \begin{enumerate}[label=(\alph*)] 
                 \item How many points are there in the sample space?
                 \item Calculate the probability that the card is an ace of spades.
                 \item Calculate the probability that the card is (i) an ace and (ii) black card.
 \end{enumerate}
\solution
		%\input{ncert/11/16/3/4/main.tex}
\item Four cards are drawn from a well-shuffled deck of 52 cards. What is the probability of obtaining 3 diamonds and one spade.
\\
\solution
		%\input{ncert/11/16/4/2/defs.tex}
\item In a certain lottery 10,000 tickets are sold and ten equal prizes are awarded. What is the probability of not getting a prize if you buy (a) one ticket (b) two tickets (c) 10 tickets ?	
\\
\solution
		%\input{ncert/11/16/4/4/defs.tex}
		%
\item 
Out of 100 students, two sections of 40 and 60 are formed. If you and your friend are among the 100 students, what is the probability that
\begin{enumerate}
\item you both enter the same section?
\item you both enter the different sections?
\end{enumerate}
\solution
		%\input{ncert/11/16/4/5/defs.tex}
	\item 
The number lock of a suitcase has 4 wheels each labelled with ten digits i.e. from 0 to 9.The lock opens with a sequence of four digits with no repeats.What is the probability of a person getting the right sequence to open the suitcase.
\\
\solution
		%\input{ncert/11/16/4/10/defs.tex}
		%
\item 
Two cards are drawn at random and without replacement from a pack of 52 playing cards. Find the probability that both the cards are black.
\\
\solution
		%\input{ncert/12/13/2/2/defs.tex}
		\item A box of oranges is inspected by examining three randomly selected oranges drawn without replacement. If all the three oranges are good, the box is approved for sale, otherwise, it is rejected. Find the probability that a box containing 15 oranges out of which 12 are good and 3 are bad ones will be approved for sale.
		\label{ncert/12/13/2/3/defs.tex}
		\item Two balls are drawn at random with replacement from a box containing 10 black and 8 red balls. Find the probability that
		\label{ncert/12/13/2/12}
\begin{enumerate}
\item both balls are red.
\item first ball is black and second is red.
\item one of them is black and other is red.
\end{enumerate}

\item In a hostel, 60\% of the students read Hindi newspaper, 40\% read English newspaper and 20\% read both Hindi and English newspapers. A student is selected at random.
		\label{ncert/12/13/2/15}
\begin{enumerate}
\item Find the probability that she reads neither Hindi nor English newspapers.
\item If she reads Hindi newspaper, find the probability that she reads English newspaper.
\item If she reads English newspaper, find the probability that she reads Hindi newspaper.\\
\end{enumerate}
\item The probability of obtaining an even prime number on each die, when a pair of dice is rolled is 
\begin{enumerate}
    \item $0$ 
    
    \item $\frac{1}{3}$ 
    
    \item $\frac{1}{12}$ 
    
    \item $\frac{1}{36}$ 
\end{enumerate}
\solution
		%\input{ncert/12/13/2/17/defs.tex}
	\item A bag contains 4 red and 4 black balls, another bag contains 2 red and 6 black balls. One of the two bags is selected at random and a ball is drawn from the bag which is found to be red. Find the probability that the ball is drawn from the first bag.
\\
\solution
		%\input{ncert/12/13/3/2/main.tex}
  \item
  Cards with numbers 2 to 101 are placed in a box. A card is selected at random.Find the probability that the card has
\begin{enumerate}[label=(\roman*)]
	\item an even number 
	\item a square number
\end{enumerate}
\solution
%\input{exemplar/10/13/3/32/main.tex}
\item
The king, queen and jack of clubs are removed from a deck of 52 playing cards and then well shuffled. Now one card is drawn at random from the remaining cards.  Determine the probability that the card is
\begin{enumerate}[label=(\roman*)]
\item a club
\item 10 of hearts
\end{enumerate}
\solution
%\input{exemplar/10/13/3/29/main.tex}
\item A team of medical students doing their internship have to assist during surgeries
at a city hospital. The probabilities of surgeries rated as very complex, complex,
routine, simple or very simple are respectively, 0.15, 0.20, 0.31, 0.26, .08. Find
the probabilities that a particular surgery will be rated
\begin{enumerate}
	\item complex or very complex;
	\item neither very complex nor very simple;
	\item routine or complex
	\item routine or simple
\end{enumerate}
\solution
%\input{exemplar/11/16/3/8(1)/main.tex}
\item A card is selected from a pack of 52 cards.
\begin{enumerate}[label=(\alph*)]
    \item How many points are there in the sample space?
    \item Calculate the probability that the card is an ace of spades.
    \item Calculate the probability that the card is (i) an ace and (ii) black card.
\end{enumerate}
\solution
%\input{exemplar/11/16/3/4/main2.tex}
\item The probability that a non leap year selected at random will contain 53 sundays.
\\
\solution
%\input{exemplar/10/13/1/19/main.tex}
\item One of the four persons John, Rita, Aslam or Gurpreet will be promoted next
month. Consequently the sample space consists of four elementary outcomes
S = {John promoted, Rita promoted, Aslam promoted, Gurpreet promoted}
You are told that the chances of John’s promotion is same as that of Gurpreet,
Rita’s chances of promotion are twice as likely as Johns. Aslam’s chances are
four times that of John.
\begin{enumerate}
	\item Determine
	\begin{enumerate}
		\item P (John promoted)
		\item P (Rita promoted)
		\item P (Aslam promoted)
		\item P (Gurpreet promoted)
	\end{enumerate}
	\item If A = {John promoted or Gurpreet promoted}, find P (A).
\end{enumerate}
\solution
%\input{exemplar/11/16/3/10/main.tex}
\item A card is drawn from a deck of 52 cards. Find the probability of getting a king or a heart or a red card.\\
\solution
%\input{exemplar/11/16/3/15/main.tex}
\item The probability that a student will pass his examination is 0.73, the probability of
the student getting a compartment is 0.13, and the probability that the student will
either pass or get compartment is 0.96. State True or False.\\
\solution
%\input{exemplar/11/16/3/31/main.tex}
\item A card is selected from a pack of 52 cards\\
\begin{enumerate}[label=(\alph*)]
\item How many points are there in the sample space?
\item Calculate the probability that the cards is an ace of spades.
\item Calculate the probability that the card is (i) an ace (ii)black card.\\
\end{enumerate}
%\input{ncert/11/16/3/4_1/Prob_4.tex}
\item In a non-leap year, the probability of having 53 tuesdays or 53 wednesdays is\\
\solution
%\input{exemplar/11/16/3/18/main.tex}
\item There are 1000 sealed envelopes in a box, 10 of them contain a cash prize of
Rs 100 each, 100 of them contain a cash prize of Rs 50 each and 200 of them
contain a cash prize of Rs 10 each and rest do not contain any cash prize. If they
are well shuffled and an envelope is picked up out, what is the probability that it
contains no cash prize?\\
\solution
%\input{exemplar/10/13/3/34/main.tex}
\item 
A die is thrown and a card is selected at random from a deck of 52 playing cards. The probability of getting an even number on the die and a spade card.\\
\solution
%\input{exemplar/12/13/3/78/main.tex}
\item
If 4-digit numbers greater than 5,000 are randomly formed from the digits 0, 1, 3, 5, and 7, what is the probability of forming a number divisible by 5 when:
\begin{enumerate}
    \item The digits are repeated?
    \item The repetition of digits is not allowed?
\end{enumerate}
\solution
%\input{ncert/11/16/4/9/main.tex}
\item Consider the probability space $\brak{\Omega, \mathcal{G}, P}$ where $\Omega = [0,2]$ and $\mathcal{G} = \cbrak{\phi, \Omega, [0,1], (1,2]}$. Let $X$ and $Y$ be two functions on $\Omega$ defined as
\begin{align*}
    X(\omega) = 
    \begin{cases}
        1 & \text{if }\omega \in [0, 1]\\
        2 & \text{if }\omega \in (1, 2]
    \end{cases}
\end{align*}
and
\begin{align*}
    Y(\omega) = 
    \begin{cases}
        2 & \text{if }\omega \in [0, 1.5]\\
        3 & \text{if }\omega \in (1.5, 2].
    \end{cases}
\end{align*}
Then which one of the following statements is true?
\begin{enumerate}
    \item [(A)] $X$ is a random variable with respect to $\mathcal{G}$, but $Y$ is not a random variable with respect to $\mathcal{G}$.
    \item [(B)] $Y$ is a random variable with respect to $\mathcal{G}$, but $X$ is not a random variable with respect to $\mathcal{G}$.
    \item [(C)] Neither $X$ nor $Y$ is a random variable with respect to $\mathcal{G}$.
    \item [(D)] Both $X$ and $Y$ are random variables with respect to $\mathcal{G}$.
\end{enumerate} \hfill (GATE ST 2023)\\
\solution
%\input{gate/ST/2023/14/main.tex}
	\item  A die is loaded in such a way that each odd number is twice as likely to occur as
each even number. Find $P(G)$, where $G$ is the event that a number greater than
3 occurs on a single roll of the die.
\\
\solution
		%\input{exemplar/11/16/3/5/main.tex}
	\item All the jacks, queens and kings are removed from a deck of 52 playing cards. The remaining cards are well shuffled and then one card is drawn at random. Giving ace a value 1 similar value for other cards, find the probability that the card has a value 
		\begin{enumerate}
			\item 7
			\item greater than 7
			\item less than 7
		\end{enumerate}
		%\input{exemplar/10/13/3/30/main.tex}
  \item A Lot consists of 48 mobile phones of which 42 are good, 3 have only minor defects and 3 have major defects.Varnika will buy a phone if it is good but the trader will only buy a mobile if it has no major defects. One phone is selected at random from the lot. What is the probability that it is
\begin{enumerate}
	\item acceptable to Varnika?
            \item acceptable to the trader?
\end{enumerate}
\solution
	%\input{exemplar/10/13/3/40/main.tex}
 \item A student says that if you throw a die, it will show up 1 or not 1. Therefore, the probability of getting 1 and the probability of getting 'not 1' each is equal to $\frac{1}{2}$. Is this correct? Give reasons.\\
 \solution
        %\input{exemplar/10/13/2/9/main.tex}
   \item Four candidates A, B, C, D have ap-
plied for the assignment to coach a school cricket
team. If A is twice as likely to be selected as B, and
B and C are given about the same chance of being
selected, while C is twice as likely to be selected
as D, what are the probabilities that
\begin{enumerate}
\item C will be selected?
\item A will not be selected?
\end{enumerate}
	%\input{exemplar/11/16/3/9/main.tex}
 \item A bag contain 24 balls of which $x$ balls are red, $2x$ are white and $3x$ are blue. A ball is selected at random, What is the probability that it is
\begin{enumerate}[label=\alph*)]
\item not red ?
\item white ?
\end{enumerate}
%\input{exemplar/10/13/3/41/main.tex}
If the letters of the word ASSASSINATION are arranged at random. Find the Probability that
\begin{enumerate}[label=(\alph*)]
\item Four $S's$ come consecutively in the word
\item Two  $I's$ and two $N's$ come together
\item All $A's$ are not coming together
\item No two $A's$ are coming together
\end{enumerate}
%\input{exemplar/11/16/3/14/main.tex}
	\item One urn contains two black balls (labelled B1 and B2) and one white ball. A
	second urn contains one black ball and two white balls (labelled W1 and W2).
	Suppose the following experiment is performed. One of the two urns is chosen
	at random. Next a ball is randomly chosen from the urn. Then a second ball is
	chosen at random from the same urn without replacing the first ball.
	
	\begin{enumerate}
	\item What is the probability that two black balls are chosen?
	
	\item What is the probability that two balls of opposite colour are chosen?
	\end{enumerate}
	\solution
	%\input{exemplar/11/16/3/12/main1.tex}
\end{enumerate}

		%
\item 
Out of 100 students, two sections of 40 and 60 are formed. If you and your friend are among the 100 students, what is the probability that
\begin{enumerate}
\item you both enter the same section?
\item you both enter the different sections?
\end{enumerate}
\solution
		%\begin{enumerate}[label=\thesection.\arabic*,ref=\thesection.\theenumi]
	\item One card is drawn from a well-shuffled deck of 52 cards. Find the probability of getting
\begin{enumerate}
\item A king of red colour 
\item A face card 
\item A red face card
\item The jack of hearts
\item A spade
\item The queen of diamonds

\end{enumerate}
\solution
		%\input{ncert/10/15/1/14/main.tex}
	\item Five cards—the ten, jack, queen, king and ace of diamonds, are well-shuffled with their face downwards. One card is then picked up at random.
\begin{enumerate}
\item
What is the probability that the card is the queen? 
\item
If the queen is drawn and put aside, what is the probability that the second card picked up is (a) an ace? (b) a queen?\\
\end{enumerate}
\solution
		%\input{ncert/10/15/1/15/defs.tex}
	\item A bag contains $5$ red balls and some blue balls. If the probability of drawing a blue ball is double that if a red ball, determine the number of blue balls in the bag. 
		\\
\solution
		%\input{ncert/10/15/2/3/defs.tex}
	\item A card is selected from a pack of 52 cards.
 \begin{enumerate}[label=(\alph*)] 
                 \item How many points are there in the sample space?
                 \item Calculate the probability that the card is an ace of spades.
                 \item Calculate the probability that the card is (i) an ace and (ii) black card.
 \end{enumerate}
\solution
		%\input{ncert/11/16/3/4/main.tex}
\item Four cards are drawn from a well-shuffled deck of 52 cards. What is the probability of obtaining 3 diamonds and one spade.
\\
\solution
		%\input{ncert/11/16/4/2/defs.tex}
\item In a certain lottery 10,000 tickets are sold and ten equal prizes are awarded. What is the probability of not getting a prize if you buy (a) one ticket (b) two tickets (c) 10 tickets ?	
\\
\solution
		%\input{ncert/11/16/4/4/defs.tex}
		%
\item 
Out of 100 students, two sections of 40 and 60 are formed. If you and your friend are among the 100 students, what is the probability that
\begin{enumerate}
\item you both enter the same section?
\item you both enter the different sections?
\end{enumerate}
\solution
		%\input{ncert/11/16/4/5/defs.tex}
	\item 
The number lock of a suitcase has 4 wheels each labelled with ten digits i.e. from 0 to 9.The lock opens with a sequence of four digits with no repeats.What is the probability of a person getting the right sequence to open the suitcase.
\\
\solution
		%\input{ncert/11/16/4/10/defs.tex}
		%
\item 
Two cards are drawn at random and without replacement from a pack of 52 playing cards. Find the probability that both the cards are black.
\\
\solution
		%\input{ncert/12/13/2/2/defs.tex}
		\item A box of oranges is inspected by examining three randomly selected oranges drawn without replacement. If all the three oranges are good, the box is approved for sale, otherwise, it is rejected. Find the probability that a box containing 15 oranges out of which 12 are good and 3 are bad ones will be approved for sale.
		\label{ncert/12/13/2/3/defs.tex}
		\item Two balls are drawn at random with replacement from a box containing 10 black and 8 red balls. Find the probability that
		\label{ncert/12/13/2/12}
\begin{enumerate}
\item both balls are red.
\item first ball is black and second is red.
\item one of them is black and other is red.
\end{enumerate}

\item In a hostel, 60\% of the students read Hindi newspaper, 40\% read English newspaper and 20\% read both Hindi and English newspapers. A student is selected at random.
		\label{ncert/12/13/2/15}
\begin{enumerate}
\item Find the probability that she reads neither Hindi nor English newspapers.
\item If she reads Hindi newspaper, find the probability that she reads English newspaper.
\item If she reads English newspaper, find the probability that she reads Hindi newspaper.\\
\end{enumerate}
\item The probability of obtaining an even prime number on each die, when a pair of dice is rolled is 
\begin{enumerate}
    \item $0$ 
    
    \item $\frac{1}{3}$ 
    
    \item $\frac{1}{12}$ 
    
    \item $\frac{1}{36}$ 
\end{enumerate}
\solution
		%\input{ncert/12/13/2/17/defs.tex}
	\item A bag contains 4 red and 4 black balls, another bag contains 2 red and 6 black balls. One of the two bags is selected at random and a ball is drawn from the bag which is found to be red. Find the probability that the ball is drawn from the first bag.
\\
\solution
		%\input{ncert/12/13/3/2/main.tex}
  \item
  Cards with numbers 2 to 101 are placed in a box. A card is selected at random.Find the probability that the card has
\begin{enumerate}[label=(\roman*)]
	\item an even number 
	\item a square number
\end{enumerate}
\solution
%\input{exemplar/10/13/3/32/main.tex}
\item
The king, queen and jack of clubs are removed from a deck of 52 playing cards and then well shuffled. Now one card is drawn at random from the remaining cards.  Determine the probability that the card is
\begin{enumerate}[label=(\roman*)]
\item a club
\item 10 of hearts
\end{enumerate}
\solution
%\input{exemplar/10/13/3/29/main.tex}
\item A team of medical students doing their internship have to assist during surgeries
at a city hospital. The probabilities of surgeries rated as very complex, complex,
routine, simple or very simple are respectively, 0.15, 0.20, 0.31, 0.26, .08. Find
the probabilities that a particular surgery will be rated
\begin{enumerate}
	\item complex or very complex;
	\item neither very complex nor very simple;
	\item routine or complex
	\item routine or simple
\end{enumerate}
\solution
%\input{exemplar/11/16/3/8(1)/main.tex}
\item A card is selected from a pack of 52 cards.
\begin{enumerate}[label=(\alph*)]
    \item How many points are there in the sample space?
    \item Calculate the probability that the card is an ace of spades.
    \item Calculate the probability that the card is (i) an ace and (ii) black card.
\end{enumerate}
\solution
%\input{exemplar/11/16/3/4/main2.tex}
\item The probability that a non leap year selected at random will contain 53 sundays.
\\
\solution
%\input{exemplar/10/13/1/19/main.tex}
\item One of the four persons John, Rita, Aslam or Gurpreet will be promoted next
month. Consequently the sample space consists of four elementary outcomes
S = {John promoted, Rita promoted, Aslam promoted, Gurpreet promoted}
You are told that the chances of John’s promotion is same as that of Gurpreet,
Rita’s chances of promotion are twice as likely as Johns. Aslam’s chances are
four times that of John.
\begin{enumerate}
	\item Determine
	\begin{enumerate}
		\item P (John promoted)
		\item P (Rita promoted)
		\item P (Aslam promoted)
		\item P (Gurpreet promoted)
	\end{enumerate}
	\item If A = {John promoted or Gurpreet promoted}, find P (A).
\end{enumerate}
\solution
%\input{exemplar/11/16/3/10/main.tex}
\item A card is drawn from a deck of 52 cards. Find the probability of getting a king or a heart or a red card.\\
\solution
%\input{exemplar/11/16/3/15/main.tex}
\item The probability that a student will pass his examination is 0.73, the probability of
the student getting a compartment is 0.13, and the probability that the student will
either pass or get compartment is 0.96. State True or False.\\
\solution
%\input{exemplar/11/16/3/31/main.tex}
\item A card is selected from a pack of 52 cards\\
\begin{enumerate}[label=(\alph*)]
\item How many points are there in the sample space?
\item Calculate the probability that the cards is an ace of spades.
\item Calculate the probability that the card is (i) an ace (ii)black card.\\
\end{enumerate}
%\input{ncert/11/16/3/4_1/Prob_4.tex}
\item In a non-leap year, the probability of having 53 tuesdays or 53 wednesdays is\\
\solution
%\input{exemplar/11/16/3/18/main.tex}
\item There are 1000 sealed envelopes in a box, 10 of them contain a cash prize of
Rs 100 each, 100 of them contain a cash prize of Rs 50 each and 200 of them
contain a cash prize of Rs 10 each and rest do not contain any cash prize. If they
are well shuffled and an envelope is picked up out, what is the probability that it
contains no cash prize?\\
\solution
%\input{exemplar/10/13/3/34/main.tex}
\item 
A die is thrown and a card is selected at random from a deck of 52 playing cards. The probability of getting an even number on the die and a spade card.\\
\solution
%\input{exemplar/12/13/3/78/main.tex}
\item
If 4-digit numbers greater than 5,000 are randomly formed from the digits 0, 1, 3, 5, and 7, what is the probability of forming a number divisible by 5 when:
\begin{enumerate}
    \item The digits are repeated?
    \item The repetition of digits is not allowed?
\end{enumerate}
\solution
%\input{ncert/11/16/4/9/main.tex}
\item Consider the probability space $\brak{\Omega, \mathcal{G}, P}$ where $\Omega = [0,2]$ and $\mathcal{G} = \cbrak{\phi, \Omega, [0,1], (1,2]}$. Let $X$ and $Y$ be two functions on $\Omega$ defined as
\begin{align*}
    X(\omega) = 
    \begin{cases}
        1 & \text{if }\omega \in [0, 1]\\
        2 & \text{if }\omega \in (1, 2]
    \end{cases}
\end{align*}
and
\begin{align*}
    Y(\omega) = 
    \begin{cases}
        2 & \text{if }\omega \in [0, 1.5]\\
        3 & \text{if }\omega \in (1.5, 2].
    \end{cases}
\end{align*}
Then which one of the following statements is true?
\begin{enumerate}
    \item [(A)] $X$ is a random variable with respect to $\mathcal{G}$, but $Y$ is not a random variable with respect to $\mathcal{G}$.
    \item [(B)] $Y$ is a random variable with respect to $\mathcal{G}$, but $X$ is not a random variable with respect to $\mathcal{G}$.
    \item [(C)] Neither $X$ nor $Y$ is a random variable with respect to $\mathcal{G}$.
    \item [(D)] Both $X$ and $Y$ are random variables with respect to $\mathcal{G}$.
\end{enumerate} \hfill (GATE ST 2023)\\
\solution
%\input{gate/ST/2023/14/main.tex}
	\item  A die is loaded in such a way that each odd number is twice as likely to occur as
each even number. Find $P(G)$, where $G$ is the event that a number greater than
3 occurs on a single roll of the die.
\\
\solution
		%\input{exemplar/11/16/3/5/main.tex}
	\item All the jacks, queens and kings are removed from a deck of 52 playing cards. The remaining cards are well shuffled and then one card is drawn at random. Giving ace a value 1 similar value for other cards, find the probability that the card has a value 
		\begin{enumerate}
			\item 7
			\item greater than 7
			\item less than 7
		\end{enumerate}
		%\input{exemplar/10/13/3/30/main.tex}
  \item A Lot consists of 48 mobile phones of which 42 are good, 3 have only minor defects and 3 have major defects.Varnika will buy a phone if it is good but the trader will only buy a mobile if it has no major defects. One phone is selected at random from the lot. What is the probability that it is
\begin{enumerate}
	\item acceptable to Varnika?
            \item acceptable to the trader?
\end{enumerate}
\solution
	%\input{exemplar/10/13/3/40/main.tex}
 \item A student says that if you throw a die, it will show up 1 or not 1. Therefore, the probability of getting 1 and the probability of getting 'not 1' each is equal to $\frac{1}{2}$. Is this correct? Give reasons.\\
 \solution
        %\input{exemplar/10/13/2/9/main.tex}
   \item Four candidates A, B, C, D have ap-
plied for the assignment to coach a school cricket
team. If A is twice as likely to be selected as B, and
B and C are given about the same chance of being
selected, while C is twice as likely to be selected
as D, what are the probabilities that
\begin{enumerate}
\item C will be selected?
\item A will not be selected?
\end{enumerate}
	%\input{exemplar/11/16/3/9/main.tex}
 \item A bag contain 24 balls of which $x$ balls are red, $2x$ are white and $3x$ are blue. A ball is selected at random, What is the probability that it is
\begin{enumerate}[label=\alph*)]
\item not red ?
\item white ?
\end{enumerate}
%\input{exemplar/10/13/3/41/main.tex}
If the letters of the word ASSASSINATION are arranged at random. Find the Probability that
\begin{enumerate}[label=(\alph*)]
\item Four $S's$ come consecutively in the word
\item Two  $I's$ and two $N's$ come together
\item All $A's$ are not coming together
\item No two $A's$ are coming together
\end{enumerate}
%\input{exemplar/11/16/3/14/main.tex}
	\item One urn contains two black balls (labelled B1 and B2) and one white ball. A
	second urn contains one black ball and two white balls (labelled W1 and W2).
	Suppose the following experiment is performed. One of the two urns is chosen
	at random. Next a ball is randomly chosen from the urn. Then a second ball is
	chosen at random from the same urn without replacing the first ball.
	
	\begin{enumerate}
	\item What is the probability that two black balls are chosen?
	
	\item What is the probability that two balls of opposite colour are chosen?
	\end{enumerate}
	\solution
	%\input{exemplar/11/16/3/12/main1.tex}
\end{enumerate}

	\item 
The number lock of a suitcase has 4 wheels each labelled with ten digits i.e. from 0 to 9.The lock opens with a sequence of four digits with no repeats.What is the probability of a person getting the right sequence to open the suitcase.
\\
\solution
		%\begin{enumerate}[label=\thesection.\arabic*,ref=\thesection.\theenumi]
	\item One card is drawn from a well-shuffled deck of 52 cards. Find the probability of getting
\begin{enumerate}
\item A king of red colour 
\item A face card 
\item A red face card
\item The jack of hearts
\item A spade
\item The queen of diamonds

\end{enumerate}
\solution
		%\input{ncert/10/15/1/14/main.tex}
	\item Five cards—the ten, jack, queen, king and ace of diamonds, are well-shuffled with their face downwards. One card is then picked up at random.
\begin{enumerate}
\item
What is the probability that the card is the queen? 
\item
If the queen is drawn and put aside, what is the probability that the second card picked up is (a) an ace? (b) a queen?\\
\end{enumerate}
\solution
		%\input{ncert/10/15/1/15/defs.tex}
	\item A bag contains $5$ red balls and some blue balls. If the probability of drawing a blue ball is double that if a red ball, determine the number of blue balls in the bag. 
		\\
\solution
		%\input{ncert/10/15/2/3/defs.tex}
	\item A card is selected from a pack of 52 cards.
 \begin{enumerate}[label=(\alph*)] 
                 \item How many points are there in the sample space?
                 \item Calculate the probability that the card is an ace of spades.
                 \item Calculate the probability that the card is (i) an ace and (ii) black card.
 \end{enumerate}
\solution
		%\input{ncert/11/16/3/4/main.tex}
\item Four cards are drawn from a well-shuffled deck of 52 cards. What is the probability of obtaining 3 diamonds and one spade.
\\
\solution
		%\input{ncert/11/16/4/2/defs.tex}
\item In a certain lottery 10,000 tickets are sold and ten equal prizes are awarded. What is the probability of not getting a prize if you buy (a) one ticket (b) two tickets (c) 10 tickets ?	
\\
\solution
		%\input{ncert/11/16/4/4/defs.tex}
		%
\item 
Out of 100 students, two sections of 40 and 60 are formed. If you and your friend are among the 100 students, what is the probability that
\begin{enumerate}
\item you both enter the same section?
\item you both enter the different sections?
\end{enumerate}
\solution
		%\input{ncert/11/16/4/5/defs.tex}
	\item 
The number lock of a suitcase has 4 wheels each labelled with ten digits i.e. from 0 to 9.The lock opens with a sequence of four digits with no repeats.What is the probability of a person getting the right sequence to open the suitcase.
\\
\solution
		%\input{ncert/11/16/4/10/defs.tex}
		%
\item 
Two cards are drawn at random and without replacement from a pack of 52 playing cards. Find the probability that both the cards are black.
\\
\solution
		%\input{ncert/12/13/2/2/defs.tex}
		\item A box of oranges is inspected by examining three randomly selected oranges drawn without replacement. If all the three oranges are good, the box is approved for sale, otherwise, it is rejected. Find the probability that a box containing 15 oranges out of which 12 are good and 3 are bad ones will be approved for sale.
		\label{ncert/12/13/2/3/defs.tex}
		\item Two balls are drawn at random with replacement from a box containing 10 black and 8 red balls. Find the probability that
		\label{ncert/12/13/2/12}
\begin{enumerate}
\item both balls are red.
\item first ball is black and second is red.
\item one of them is black and other is red.
\end{enumerate}

\item In a hostel, 60\% of the students read Hindi newspaper, 40\% read English newspaper and 20\% read both Hindi and English newspapers. A student is selected at random.
		\label{ncert/12/13/2/15}
\begin{enumerate}
\item Find the probability that she reads neither Hindi nor English newspapers.
\item If she reads Hindi newspaper, find the probability that she reads English newspaper.
\item If she reads English newspaper, find the probability that she reads Hindi newspaper.\\
\end{enumerate}
\item The probability of obtaining an even prime number on each die, when a pair of dice is rolled is 
\begin{enumerate}
    \item $0$ 
    
    \item $\frac{1}{3}$ 
    
    \item $\frac{1}{12}$ 
    
    \item $\frac{1}{36}$ 
\end{enumerate}
\solution
		%\input{ncert/12/13/2/17/defs.tex}
	\item A bag contains 4 red and 4 black balls, another bag contains 2 red and 6 black balls. One of the two bags is selected at random and a ball is drawn from the bag which is found to be red. Find the probability that the ball is drawn from the first bag.
\\
\solution
		%\input{ncert/12/13/3/2/main.tex}
  \item
  Cards with numbers 2 to 101 are placed in a box. A card is selected at random.Find the probability that the card has
\begin{enumerate}[label=(\roman*)]
	\item an even number 
	\item a square number
\end{enumerate}
\solution
%\input{exemplar/10/13/3/32/main.tex}
\item
The king, queen and jack of clubs are removed from a deck of 52 playing cards and then well shuffled. Now one card is drawn at random from the remaining cards.  Determine the probability that the card is
\begin{enumerate}[label=(\roman*)]
\item a club
\item 10 of hearts
\end{enumerate}
\solution
%\input{exemplar/10/13/3/29/main.tex}
\item A team of medical students doing their internship have to assist during surgeries
at a city hospital. The probabilities of surgeries rated as very complex, complex,
routine, simple or very simple are respectively, 0.15, 0.20, 0.31, 0.26, .08. Find
the probabilities that a particular surgery will be rated
\begin{enumerate}
	\item complex or very complex;
	\item neither very complex nor very simple;
	\item routine or complex
	\item routine or simple
\end{enumerate}
\solution
%\input{exemplar/11/16/3/8(1)/main.tex}
\item A card is selected from a pack of 52 cards.
\begin{enumerate}[label=(\alph*)]
    \item How many points are there in the sample space?
    \item Calculate the probability that the card is an ace of spades.
    \item Calculate the probability that the card is (i) an ace and (ii) black card.
\end{enumerate}
\solution
%\input{exemplar/11/16/3/4/main2.tex}
\item The probability that a non leap year selected at random will contain 53 sundays.
\\
\solution
%\input{exemplar/10/13/1/19/main.tex}
\item One of the four persons John, Rita, Aslam or Gurpreet will be promoted next
month. Consequently the sample space consists of four elementary outcomes
S = {John promoted, Rita promoted, Aslam promoted, Gurpreet promoted}
You are told that the chances of John’s promotion is same as that of Gurpreet,
Rita’s chances of promotion are twice as likely as Johns. Aslam’s chances are
four times that of John.
\begin{enumerate}
	\item Determine
	\begin{enumerate}
		\item P (John promoted)
		\item P (Rita promoted)
		\item P (Aslam promoted)
		\item P (Gurpreet promoted)
	\end{enumerate}
	\item If A = {John promoted or Gurpreet promoted}, find P (A).
\end{enumerate}
\solution
%\input{exemplar/11/16/3/10/main.tex}
\item A card is drawn from a deck of 52 cards. Find the probability of getting a king or a heart or a red card.\\
\solution
%\input{exemplar/11/16/3/15/main.tex}
\item The probability that a student will pass his examination is 0.73, the probability of
the student getting a compartment is 0.13, and the probability that the student will
either pass or get compartment is 0.96. State True or False.\\
\solution
%\input{exemplar/11/16/3/31/main.tex}
\item A card is selected from a pack of 52 cards\\
\begin{enumerate}[label=(\alph*)]
\item How many points are there in the sample space?
\item Calculate the probability that the cards is an ace of spades.
\item Calculate the probability that the card is (i) an ace (ii)black card.\\
\end{enumerate}
%\input{ncert/11/16/3/4_1/Prob_4.tex}
\item In a non-leap year, the probability of having 53 tuesdays or 53 wednesdays is\\
\solution
%\input{exemplar/11/16/3/18/main.tex}
\item There are 1000 sealed envelopes in a box, 10 of them contain a cash prize of
Rs 100 each, 100 of them contain a cash prize of Rs 50 each and 200 of them
contain a cash prize of Rs 10 each and rest do not contain any cash prize. If they
are well shuffled and an envelope is picked up out, what is the probability that it
contains no cash prize?\\
\solution
%\input{exemplar/10/13/3/34/main.tex}
\item 
A die is thrown and a card is selected at random from a deck of 52 playing cards. The probability of getting an even number on the die and a spade card.\\
\solution
%\input{exemplar/12/13/3/78/main.tex}
\item
If 4-digit numbers greater than 5,000 are randomly formed from the digits 0, 1, 3, 5, and 7, what is the probability of forming a number divisible by 5 when:
\begin{enumerate}
    \item The digits are repeated?
    \item The repetition of digits is not allowed?
\end{enumerate}
\solution
%\input{ncert/11/16/4/9/main.tex}
\item Consider the probability space $\brak{\Omega, \mathcal{G}, P}$ where $\Omega = [0,2]$ and $\mathcal{G} = \cbrak{\phi, \Omega, [0,1], (1,2]}$. Let $X$ and $Y$ be two functions on $\Omega$ defined as
\begin{align*}
    X(\omega) = 
    \begin{cases}
        1 & \text{if }\omega \in [0, 1]\\
        2 & \text{if }\omega \in (1, 2]
    \end{cases}
\end{align*}
and
\begin{align*}
    Y(\omega) = 
    \begin{cases}
        2 & \text{if }\omega \in [0, 1.5]\\
        3 & \text{if }\omega \in (1.5, 2].
    \end{cases}
\end{align*}
Then which one of the following statements is true?
\begin{enumerate}
    \item [(A)] $X$ is a random variable with respect to $\mathcal{G}$, but $Y$ is not a random variable with respect to $\mathcal{G}$.
    \item [(B)] $Y$ is a random variable with respect to $\mathcal{G}$, but $X$ is not a random variable with respect to $\mathcal{G}$.
    \item [(C)] Neither $X$ nor $Y$ is a random variable with respect to $\mathcal{G}$.
    \item [(D)] Both $X$ and $Y$ are random variables with respect to $\mathcal{G}$.
\end{enumerate} \hfill (GATE ST 2023)\\
\solution
%\input{gate/ST/2023/14/main.tex}
	\item  A die is loaded in such a way that each odd number is twice as likely to occur as
each even number. Find $P(G)$, where $G$ is the event that a number greater than
3 occurs on a single roll of the die.
\\
\solution
		%\input{exemplar/11/16/3/5/main.tex}
	\item All the jacks, queens and kings are removed from a deck of 52 playing cards. The remaining cards are well shuffled and then one card is drawn at random. Giving ace a value 1 similar value for other cards, find the probability that the card has a value 
		\begin{enumerate}
			\item 7
			\item greater than 7
			\item less than 7
		\end{enumerate}
		%\input{exemplar/10/13/3/30/main.tex}
  \item A Lot consists of 48 mobile phones of which 42 are good, 3 have only minor defects and 3 have major defects.Varnika will buy a phone if it is good but the trader will only buy a mobile if it has no major defects. One phone is selected at random from the lot. What is the probability that it is
\begin{enumerate}
	\item acceptable to Varnika?
            \item acceptable to the trader?
\end{enumerate}
\solution
	%\input{exemplar/10/13/3/40/main.tex}
 \item A student says that if you throw a die, it will show up 1 or not 1. Therefore, the probability of getting 1 and the probability of getting 'not 1' each is equal to $\frac{1}{2}$. Is this correct? Give reasons.\\
 \solution
        %\input{exemplar/10/13/2/9/main.tex}
   \item Four candidates A, B, C, D have ap-
plied for the assignment to coach a school cricket
team. If A is twice as likely to be selected as B, and
B and C are given about the same chance of being
selected, while C is twice as likely to be selected
as D, what are the probabilities that
\begin{enumerate}
\item C will be selected?
\item A will not be selected?
\end{enumerate}
	%\input{exemplar/11/16/3/9/main.tex}
 \item A bag contain 24 balls of which $x$ balls are red, $2x$ are white and $3x$ are blue. A ball is selected at random, What is the probability that it is
\begin{enumerate}[label=\alph*)]
\item not red ?
\item white ?
\end{enumerate}
%\input{exemplar/10/13/3/41/main.tex}
If the letters of the word ASSASSINATION are arranged at random. Find the Probability that
\begin{enumerate}[label=(\alph*)]
\item Four $S's$ come consecutively in the word
\item Two  $I's$ and two $N's$ come together
\item All $A's$ are not coming together
\item No two $A's$ are coming together
\end{enumerate}
%\input{exemplar/11/16/3/14/main.tex}
	\item One urn contains two black balls (labelled B1 and B2) and one white ball. A
	second urn contains one black ball and two white balls (labelled W1 and W2).
	Suppose the following experiment is performed. One of the two urns is chosen
	at random. Next a ball is randomly chosen from the urn. Then a second ball is
	chosen at random from the same urn without replacing the first ball.
	
	\begin{enumerate}
	\item What is the probability that two black balls are chosen?
	
	\item What is the probability that two balls of opposite colour are chosen?
	\end{enumerate}
	\solution
	%\input{exemplar/11/16/3/12/main1.tex}
\end{enumerate}

		%
\item 
Two cards are drawn at random and without replacement from a pack of 52 playing cards. Find the probability that both the cards are black.
\\
\solution
		%\begin{enumerate}[label=\thesection.\arabic*,ref=\thesection.\theenumi]
	\item One card is drawn from a well-shuffled deck of 52 cards. Find the probability of getting
\begin{enumerate}
\item A king of red colour 
\item A face card 
\item A red face card
\item The jack of hearts
\item A spade
\item The queen of diamonds

\end{enumerate}
\solution
		%\input{ncert/10/15/1/14/main.tex}
	\item Five cards—the ten, jack, queen, king and ace of diamonds, are well-shuffled with their face downwards. One card is then picked up at random.
\begin{enumerate}
\item
What is the probability that the card is the queen? 
\item
If the queen is drawn and put aside, what is the probability that the second card picked up is (a) an ace? (b) a queen?\\
\end{enumerate}
\solution
		%\input{ncert/10/15/1/15/defs.tex}
	\item A bag contains $5$ red balls and some blue balls. If the probability of drawing a blue ball is double that if a red ball, determine the number of blue balls in the bag. 
		\\
\solution
		%\input{ncert/10/15/2/3/defs.tex}
	\item A card is selected from a pack of 52 cards.
 \begin{enumerate}[label=(\alph*)] 
                 \item How many points are there in the sample space?
                 \item Calculate the probability that the card is an ace of spades.
                 \item Calculate the probability that the card is (i) an ace and (ii) black card.
 \end{enumerate}
\solution
		%\input{ncert/11/16/3/4/main.tex}
\item Four cards are drawn from a well-shuffled deck of 52 cards. What is the probability of obtaining 3 diamonds and one spade.
\\
\solution
		%\input{ncert/11/16/4/2/defs.tex}
\item In a certain lottery 10,000 tickets are sold and ten equal prizes are awarded. What is the probability of not getting a prize if you buy (a) one ticket (b) two tickets (c) 10 tickets ?	
\\
\solution
		%\input{ncert/11/16/4/4/defs.tex}
		%
\item 
Out of 100 students, two sections of 40 and 60 are formed. If you and your friend are among the 100 students, what is the probability that
\begin{enumerate}
\item you both enter the same section?
\item you both enter the different sections?
\end{enumerate}
\solution
		%\input{ncert/11/16/4/5/defs.tex}
	\item 
The number lock of a suitcase has 4 wheels each labelled with ten digits i.e. from 0 to 9.The lock opens with a sequence of four digits with no repeats.What is the probability of a person getting the right sequence to open the suitcase.
\\
\solution
		%\input{ncert/11/16/4/10/defs.tex}
		%
\item 
Two cards are drawn at random and without replacement from a pack of 52 playing cards. Find the probability that both the cards are black.
\\
\solution
		%\input{ncert/12/13/2/2/defs.tex}
		\item A box of oranges is inspected by examining three randomly selected oranges drawn without replacement. If all the three oranges are good, the box is approved for sale, otherwise, it is rejected. Find the probability that a box containing 15 oranges out of which 12 are good and 3 are bad ones will be approved for sale.
		\label{ncert/12/13/2/3/defs.tex}
		\item Two balls are drawn at random with replacement from a box containing 10 black and 8 red balls. Find the probability that
		\label{ncert/12/13/2/12}
\begin{enumerate}
\item both balls are red.
\item first ball is black and second is red.
\item one of them is black and other is red.
\end{enumerate}

\item In a hostel, 60\% of the students read Hindi newspaper, 40\% read English newspaper and 20\% read both Hindi and English newspapers. A student is selected at random.
		\label{ncert/12/13/2/15}
\begin{enumerate}
\item Find the probability that she reads neither Hindi nor English newspapers.
\item If she reads Hindi newspaper, find the probability that she reads English newspaper.
\item If she reads English newspaper, find the probability that she reads Hindi newspaper.\\
\end{enumerate}
\item The probability of obtaining an even prime number on each die, when a pair of dice is rolled is 
\begin{enumerate}
    \item $0$ 
    
    \item $\frac{1}{3}$ 
    
    \item $\frac{1}{12}$ 
    
    \item $\frac{1}{36}$ 
\end{enumerate}
\solution
		%\input{ncert/12/13/2/17/defs.tex}
	\item A bag contains 4 red and 4 black balls, another bag contains 2 red and 6 black balls. One of the two bags is selected at random and a ball is drawn from the bag which is found to be red. Find the probability that the ball is drawn from the first bag.
\\
\solution
		%\input{ncert/12/13/3/2/main.tex}
  \item
  Cards with numbers 2 to 101 are placed in a box. A card is selected at random.Find the probability that the card has
\begin{enumerate}[label=(\roman*)]
	\item an even number 
	\item a square number
\end{enumerate}
\solution
%\input{exemplar/10/13/3/32/main.tex}
\item
The king, queen and jack of clubs are removed from a deck of 52 playing cards and then well shuffled. Now one card is drawn at random from the remaining cards.  Determine the probability that the card is
\begin{enumerate}[label=(\roman*)]
\item a club
\item 10 of hearts
\end{enumerate}
\solution
%\input{exemplar/10/13/3/29/main.tex}
\item A team of medical students doing their internship have to assist during surgeries
at a city hospital. The probabilities of surgeries rated as very complex, complex,
routine, simple or very simple are respectively, 0.15, 0.20, 0.31, 0.26, .08. Find
the probabilities that a particular surgery will be rated
\begin{enumerate}
	\item complex or very complex;
	\item neither very complex nor very simple;
	\item routine or complex
	\item routine or simple
\end{enumerate}
\solution
%\input{exemplar/11/16/3/8(1)/main.tex}
\item A card is selected from a pack of 52 cards.
\begin{enumerate}[label=(\alph*)]
    \item How many points are there in the sample space?
    \item Calculate the probability that the card is an ace of spades.
    \item Calculate the probability that the card is (i) an ace and (ii) black card.
\end{enumerate}
\solution
%\input{exemplar/11/16/3/4/main2.tex}
\item The probability that a non leap year selected at random will contain 53 sundays.
\\
\solution
%\input{exemplar/10/13/1/19/main.tex}
\item One of the four persons John, Rita, Aslam or Gurpreet will be promoted next
month. Consequently the sample space consists of four elementary outcomes
S = {John promoted, Rita promoted, Aslam promoted, Gurpreet promoted}
You are told that the chances of John’s promotion is same as that of Gurpreet,
Rita’s chances of promotion are twice as likely as Johns. Aslam’s chances are
four times that of John.
\begin{enumerate}
	\item Determine
	\begin{enumerate}
		\item P (John promoted)
		\item P (Rita promoted)
		\item P (Aslam promoted)
		\item P (Gurpreet promoted)
	\end{enumerate}
	\item If A = {John promoted or Gurpreet promoted}, find P (A).
\end{enumerate}
\solution
%\input{exemplar/11/16/3/10/main.tex}
\item A card is drawn from a deck of 52 cards. Find the probability of getting a king or a heart or a red card.\\
\solution
%\input{exemplar/11/16/3/15/main.tex}
\item The probability that a student will pass his examination is 0.73, the probability of
the student getting a compartment is 0.13, and the probability that the student will
either pass or get compartment is 0.96. State True or False.\\
\solution
%\input{exemplar/11/16/3/31/main.tex}
\item A card is selected from a pack of 52 cards\\
\begin{enumerate}[label=(\alph*)]
\item How many points are there in the sample space?
\item Calculate the probability that the cards is an ace of spades.
\item Calculate the probability that the card is (i) an ace (ii)black card.\\
\end{enumerate}
%\input{ncert/11/16/3/4_1/Prob_4.tex}
\item In a non-leap year, the probability of having 53 tuesdays or 53 wednesdays is\\
\solution
%\input{exemplar/11/16/3/18/main.tex}
\item There are 1000 sealed envelopes in a box, 10 of them contain a cash prize of
Rs 100 each, 100 of them contain a cash prize of Rs 50 each and 200 of them
contain a cash prize of Rs 10 each and rest do not contain any cash prize. If they
are well shuffled and an envelope is picked up out, what is the probability that it
contains no cash prize?\\
\solution
%\input{exemplar/10/13/3/34/main.tex}
\item 
A die is thrown and a card is selected at random from a deck of 52 playing cards. The probability of getting an even number on the die and a spade card.\\
\solution
%\input{exemplar/12/13/3/78/main.tex}
\item
If 4-digit numbers greater than 5,000 are randomly formed from the digits 0, 1, 3, 5, and 7, what is the probability of forming a number divisible by 5 when:
\begin{enumerate}
    \item The digits are repeated?
    \item The repetition of digits is not allowed?
\end{enumerate}
\solution
%\input{ncert/11/16/4/9/main.tex}
\item Consider the probability space $\brak{\Omega, \mathcal{G}, P}$ where $\Omega = [0,2]$ and $\mathcal{G} = \cbrak{\phi, \Omega, [0,1], (1,2]}$. Let $X$ and $Y$ be two functions on $\Omega$ defined as
\begin{align*}
    X(\omega) = 
    \begin{cases}
        1 & \text{if }\omega \in [0, 1]\\
        2 & \text{if }\omega \in (1, 2]
    \end{cases}
\end{align*}
and
\begin{align*}
    Y(\omega) = 
    \begin{cases}
        2 & \text{if }\omega \in [0, 1.5]\\
        3 & \text{if }\omega \in (1.5, 2].
    \end{cases}
\end{align*}
Then which one of the following statements is true?
\begin{enumerate}
    \item [(A)] $X$ is a random variable with respect to $\mathcal{G}$, but $Y$ is not a random variable with respect to $\mathcal{G}$.
    \item [(B)] $Y$ is a random variable with respect to $\mathcal{G}$, but $X$ is not a random variable with respect to $\mathcal{G}$.
    \item [(C)] Neither $X$ nor $Y$ is a random variable with respect to $\mathcal{G}$.
    \item [(D)] Both $X$ and $Y$ are random variables with respect to $\mathcal{G}$.
\end{enumerate} \hfill (GATE ST 2023)\\
\solution
%\input{gate/ST/2023/14/main.tex}
	\item  A die is loaded in such a way that each odd number is twice as likely to occur as
each even number. Find $P(G)$, where $G$ is the event that a number greater than
3 occurs on a single roll of the die.
\\
\solution
		%\input{exemplar/11/16/3/5/main.tex}
	\item All the jacks, queens and kings are removed from a deck of 52 playing cards. The remaining cards are well shuffled and then one card is drawn at random. Giving ace a value 1 similar value for other cards, find the probability that the card has a value 
		\begin{enumerate}
			\item 7
			\item greater than 7
			\item less than 7
		\end{enumerate}
		%\input{exemplar/10/13/3/30/main.tex}
  \item A Lot consists of 48 mobile phones of which 42 are good, 3 have only minor defects and 3 have major defects.Varnika will buy a phone if it is good but the trader will only buy a mobile if it has no major defects. One phone is selected at random from the lot. What is the probability that it is
\begin{enumerate}
	\item acceptable to Varnika?
            \item acceptable to the trader?
\end{enumerate}
\solution
	%\input{exemplar/10/13/3/40/main.tex}
 \item A student says that if you throw a die, it will show up 1 or not 1. Therefore, the probability of getting 1 and the probability of getting 'not 1' each is equal to $\frac{1}{2}$. Is this correct? Give reasons.\\
 \solution
        %\input{exemplar/10/13/2/9/main.tex}
   \item Four candidates A, B, C, D have ap-
plied for the assignment to coach a school cricket
team. If A is twice as likely to be selected as B, and
B and C are given about the same chance of being
selected, while C is twice as likely to be selected
as D, what are the probabilities that
\begin{enumerate}
\item C will be selected?
\item A will not be selected?
\end{enumerate}
	%\input{exemplar/11/16/3/9/main.tex}
 \item A bag contain 24 balls of which $x$ balls are red, $2x$ are white and $3x$ are blue. A ball is selected at random, What is the probability that it is
\begin{enumerate}[label=\alph*)]
\item not red ?
\item white ?
\end{enumerate}
%\input{exemplar/10/13/3/41/main.tex}
If the letters of the word ASSASSINATION are arranged at random. Find the Probability that
\begin{enumerate}[label=(\alph*)]
\item Four $S's$ come consecutively in the word
\item Two  $I's$ and two $N's$ come together
\item All $A's$ are not coming together
\item No two $A's$ are coming together
\end{enumerate}
%\input{exemplar/11/16/3/14/main.tex}
	\item One urn contains two black balls (labelled B1 and B2) and one white ball. A
	second urn contains one black ball and two white balls (labelled W1 and W2).
	Suppose the following experiment is performed. One of the two urns is chosen
	at random. Next a ball is randomly chosen from the urn. Then a second ball is
	chosen at random from the same urn without replacing the first ball.
	
	\begin{enumerate}
	\item What is the probability that two black balls are chosen?
	
	\item What is the probability that two balls of opposite colour are chosen?
	\end{enumerate}
	\solution
	%\input{exemplar/11/16/3/12/main1.tex}
\end{enumerate}

		\item A box of oranges is inspected by examining three randomly selected oranges drawn without replacement. If all the three oranges are good, the box is approved for sale, otherwise, it is rejected. Find the probability that a box containing 15 oranges out of which 12 are good and 3 are bad ones will be approved for sale.
		\label{ncert/12/13/2/3/defs.tex}
		\item Two balls are drawn at random with replacement from a box containing 10 black and 8 red balls. Find the probability that
		\label{ncert/12/13/2/12}
\begin{enumerate}
\item both balls are red.
\item first ball is black and second is red.
\item one of them is black and other is red.
\end{enumerate}

\item In a hostel, 60\% of the students read Hindi newspaper, 40\% read English newspaper and 20\% read both Hindi and English newspapers. A student is selected at random.
		\label{ncert/12/13/2/15}
\begin{enumerate}
\item Find the probability that she reads neither Hindi nor English newspapers.
\item If she reads Hindi newspaper, find the probability that she reads English newspaper.
\item If she reads English newspaper, find the probability that she reads Hindi newspaper.\\
\end{enumerate}
\item The probability of obtaining an even prime number on each die, when a pair of dice is rolled is 
\begin{enumerate}
    \item $0$ 
    
    \item $\frac{1}{3}$ 
    
    \item $\frac{1}{12}$ 
    
    \item $\frac{1}{36}$ 
\end{enumerate}
\solution
		%\begin{enumerate}[label=\thesection.\arabic*,ref=\thesection.\theenumi]
	\item One card is drawn from a well-shuffled deck of 52 cards. Find the probability of getting
\begin{enumerate}
\item A king of red colour 
\item A face card 
\item A red face card
\item The jack of hearts
\item A spade
\item The queen of diamonds

\end{enumerate}
\solution
		%\input{ncert/10/15/1/14/main.tex}
	\item Five cards—the ten, jack, queen, king and ace of diamonds, are well-shuffled with their face downwards. One card is then picked up at random.
\begin{enumerate}
\item
What is the probability that the card is the queen? 
\item
If the queen is drawn and put aside, what is the probability that the second card picked up is (a) an ace? (b) a queen?\\
\end{enumerate}
\solution
		%\input{ncert/10/15/1/15/defs.tex}
	\item A bag contains $5$ red balls and some blue balls. If the probability of drawing a blue ball is double that if a red ball, determine the number of blue balls in the bag. 
		\\
\solution
		%\input{ncert/10/15/2/3/defs.tex}
	\item A card is selected from a pack of 52 cards.
 \begin{enumerate}[label=(\alph*)] 
                 \item How many points are there in the sample space?
                 \item Calculate the probability that the card is an ace of spades.
                 \item Calculate the probability that the card is (i) an ace and (ii) black card.
 \end{enumerate}
\solution
		%\input{ncert/11/16/3/4/main.tex}
\item Four cards are drawn from a well-shuffled deck of 52 cards. What is the probability of obtaining 3 diamonds and one spade.
\\
\solution
		%\input{ncert/11/16/4/2/defs.tex}
\item In a certain lottery 10,000 tickets are sold and ten equal prizes are awarded. What is the probability of not getting a prize if you buy (a) one ticket (b) two tickets (c) 10 tickets ?	
\\
\solution
		%\input{ncert/11/16/4/4/defs.tex}
		%
\item 
Out of 100 students, two sections of 40 and 60 are formed. If you and your friend are among the 100 students, what is the probability that
\begin{enumerate}
\item you both enter the same section?
\item you both enter the different sections?
\end{enumerate}
\solution
		%\input{ncert/11/16/4/5/defs.tex}
	\item 
The number lock of a suitcase has 4 wheels each labelled with ten digits i.e. from 0 to 9.The lock opens with a sequence of four digits with no repeats.What is the probability of a person getting the right sequence to open the suitcase.
\\
\solution
		%\input{ncert/11/16/4/10/defs.tex}
		%
\item 
Two cards are drawn at random and without replacement from a pack of 52 playing cards. Find the probability that both the cards are black.
\\
\solution
		%\input{ncert/12/13/2/2/defs.tex}
		\item A box of oranges is inspected by examining three randomly selected oranges drawn without replacement. If all the three oranges are good, the box is approved for sale, otherwise, it is rejected. Find the probability that a box containing 15 oranges out of which 12 are good and 3 are bad ones will be approved for sale.
		\label{ncert/12/13/2/3/defs.tex}
		\item Two balls are drawn at random with replacement from a box containing 10 black and 8 red balls. Find the probability that
		\label{ncert/12/13/2/12}
\begin{enumerate}
\item both balls are red.
\item first ball is black and second is red.
\item one of them is black and other is red.
\end{enumerate}

\item In a hostel, 60\% of the students read Hindi newspaper, 40\% read English newspaper and 20\% read both Hindi and English newspapers. A student is selected at random.
		\label{ncert/12/13/2/15}
\begin{enumerate}
\item Find the probability that she reads neither Hindi nor English newspapers.
\item If she reads Hindi newspaper, find the probability that she reads English newspaper.
\item If she reads English newspaper, find the probability that she reads Hindi newspaper.\\
\end{enumerate}
\item The probability of obtaining an even prime number on each die, when a pair of dice is rolled is 
\begin{enumerate}
    \item $0$ 
    
    \item $\frac{1}{3}$ 
    
    \item $\frac{1}{12}$ 
    
    \item $\frac{1}{36}$ 
\end{enumerate}
\solution
		%\input{ncert/12/13/2/17/defs.tex}
	\item A bag contains 4 red and 4 black balls, another bag contains 2 red and 6 black balls. One of the two bags is selected at random and a ball is drawn from the bag which is found to be red. Find the probability that the ball is drawn from the first bag.
\\
\solution
		%\input{ncert/12/13/3/2/main.tex}
  \item
  Cards with numbers 2 to 101 are placed in a box. A card is selected at random.Find the probability that the card has
\begin{enumerate}[label=(\roman*)]
	\item an even number 
	\item a square number
\end{enumerate}
\solution
%\input{exemplar/10/13/3/32/main.tex}
\item
The king, queen and jack of clubs are removed from a deck of 52 playing cards and then well shuffled. Now one card is drawn at random from the remaining cards.  Determine the probability that the card is
\begin{enumerate}[label=(\roman*)]
\item a club
\item 10 of hearts
\end{enumerate}
\solution
%\input{exemplar/10/13/3/29/main.tex}
\item A team of medical students doing their internship have to assist during surgeries
at a city hospital. The probabilities of surgeries rated as very complex, complex,
routine, simple or very simple are respectively, 0.15, 0.20, 0.31, 0.26, .08. Find
the probabilities that a particular surgery will be rated
\begin{enumerate}
	\item complex or very complex;
	\item neither very complex nor very simple;
	\item routine or complex
	\item routine or simple
\end{enumerate}
\solution
%\input{exemplar/11/16/3/8(1)/main.tex}
\item A card is selected from a pack of 52 cards.
\begin{enumerate}[label=(\alph*)]
    \item How many points are there in the sample space?
    \item Calculate the probability that the card is an ace of spades.
    \item Calculate the probability that the card is (i) an ace and (ii) black card.
\end{enumerate}
\solution
%\input{exemplar/11/16/3/4/main2.tex}
\item The probability that a non leap year selected at random will contain 53 sundays.
\\
\solution
%\input{exemplar/10/13/1/19/main.tex}
\item One of the four persons John, Rita, Aslam or Gurpreet will be promoted next
month. Consequently the sample space consists of four elementary outcomes
S = {John promoted, Rita promoted, Aslam promoted, Gurpreet promoted}
You are told that the chances of John’s promotion is same as that of Gurpreet,
Rita’s chances of promotion are twice as likely as Johns. Aslam’s chances are
four times that of John.
\begin{enumerate}
	\item Determine
	\begin{enumerate}
		\item P (John promoted)
		\item P (Rita promoted)
		\item P (Aslam promoted)
		\item P (Gurpreet promoted)
	\end{enumerate}
	\item If A = {John promoted or Gurpreet promoted}, find P (A).
\end{enumerate}
\solution
%\input{exemplar/11/16/3/10/main.tex}
\item A card is drawn from a deck of 52 cards. Find the probability of getting a king or a heart or a red card.\\
\solution
%\input{exemplar/11/16/3/15/main.tex}
\item The probability that a student will pass his examination is 0.73, the probability of
the student getting a compartment is 0.13, and the probability that the student will
either pass or get compartment is 0.96. State True or False.\\
\solution
%\input{exemplar/11/16/3/31/main.tex}
\item A card is selected from a pack of 52 cards\\
\begin{enumerate}[label=(\alph*)]
\item How many points are there in the sample space?
\item Calculate the probability that the cards is an ace of spades.
\item Calculate the probability that the card is (i) an ace (ii)black card.\\
\end{enumerate}
%\input{ncert/11/16/3/4_1/Prob_4.tex}
\item In a non-leap year, the probability of having 53 tuesdays or 53 wednesdays is\\
\solution
%\input{exemplar/11/16/3/18/main.tex}
\item There are 1000 sealed envelopes in a box, 10 of them contain a cash prize of
Rs 100 each, 100 of them contain a cash prize of Rs 50 each and 200 of them
contain a cash prize of Rs 10 each and rest do not contain any cash prize. If they
are well shuffled and an envelope is picked up out, what is the probability that it
contains no cash prize?\\
\solution
%\input{exemplar/10/13/3/34/main.tex}
\item 
A die is thrown and a card is selected at random from a deck of 52 playing cards. The probability of getting an even number on the die and a spade card.\\
\solution
%\input{exemplar/12/13/3/78/main.tex}
\item
If 4-digit numbers greater than 5,000 are randomly formed from the digits 0, 1, 3, 5, and 7, what is the probability of forming a number divisible by 5 when:
\begin{enumerate}
    \item The digits are repeated?
    \item The repetition of digits is not allowed?
\end{enumerate}
\solution
%\input{ncert/11/16/4/9/main.tex}
\item Consider the probability space $\brak{\Omega, \mathcal{G}, P}$ where $\Omega = [0,2]$ and $\mathcal{G} = \cbrak{\phi, \Omega, [0,1], (1,2]}$. Let $X$ and $Y$ be two functions on $\Omega$ defined as
\begin{align*}
    X(\omega) = 
    \begin{cases}
        1 & \text{if }\omega \in [0, 1]\\
        2 & \text{if }\omega \in (1, 2]
    \end{cases}
\end{align*}
and
\begin{align*}
    Y(\omega) = 
    \begin{cases}
        2 & \text{if }\omega \in [0, 1.5]\\
        3 & \text{if }\omega \in (1.5, 2].
    \end{cases}
\end{align*}
Then which one of the following statements is true?
\begin{enumerate}
    \item [(A)] $X$ is a random variable with respect to $\mathcal{G}$, but $Y$ is not a random variable with respect to $\mathcal{G}$.
    \item [(B)] $Y$ is a random variable with respect to $\mathcal{G}$, but $X$ is not a random variable with respect to $\mathcal{G}$.
    \item [(C)] Neither $X$ nor $Y$ is a random variable with respect to $\mathcal{G}$.
    \item [(D)] Both $X$ and $Y$ are random variables with respect to $\mathcal{G}$.
\end{enumerate} \hfill (GATE ST 2023)\\
\solution
%\input{gate/ST/2023/14/main.tex}
	\item  A die is loaded in such a way that each odd number is twice as likely to occur as
each even number. Find $P(G)$, where $G$ is the event that a number greater than
3 occurs on a single roll of the die.
\\
\solution
		%\input{exemplar/11/16/3/5/main.tex}
	\item All the jacks, queens and kings are removed from a deck of 52 playing cards. The remaining cards are well shuffled and then one card is drawn at random. Giving ace a value 1 similar value for other cards, find the probability that the card has a value 
		\begin{enumerate}
			\item 7
			\item greater than 7
			\item less than 7
		\end{enumerate}
		%\input{exemplar/10/13/3/30/main.tex}
  \item A Lot consists of 48 mobile phones of which 42 are good, 3 have only minor defects and 3 have major defects.Varnika will buy a phone if it is good but the trader will only buy a mobile if it has no major defects. One phone is selected at random from the lot. What is the probability that it is
\begin{enumerate}
	\item acceptable to Varnika?
            \item acceptable to the trader?
\end{enumerate}
\solution
	%\input{exemplar/10/13/3/40/main.tex}
 \item A student says that if you throw a die, it will show up 1 or not 1. Therefore, the probability of getting 1 and the probability of getting 'not 1' each is equal to $\frac{1}{2}$. Is this correct? Give reasons.\\
 \solution
        %\input{exemplar/10/13/2/9/main.tex}
   \item Four candidates A, B, C, D have ap-
plied for the assignment to coach a school cricket
team. If A is twice as likely to be selected as B, and
B and C are given about the same chance of being
selected, while C is twice as likely to be selected
as D, what are the probabilities that
\begin{enumerate}
\item C will be selected?
\item A will not be selected?
\end{enumerate}
	%\input{exemplar/11/16/3/9/main.tex}
 \item A bag contain 24 balls of which $x$ balls are red, $2x$ are white and $3x$ are blue. A ball is selected at random, What is the probability that it is
\begin{enumerate}[label=\alph*)]
\item not red ?
\item white ?
\end{enumerate}
%\input{exemplar/10/13/3/41/main.tex}
If the letters of the word ASSASSINATION are arranged at random. Find the Probability that
\begin{enumerate}[label=(\alph*)]
\item Four $S's$ come consecutively in the word
\item Two  $I's$ and two $N's$ come together
\item All $A's$ are not coming together
\item No two $A's$ are coming together
\end{enumerate}
%\input{exemplar/11/16/3/14/main.tex}
	\item One urn contains two black balls (labelled B1 and B2) and one white ball. A
	second urn contains one black ball and two white balls (labelled W1 and W2).
	Suppose the following experiment is performed. One of the two urns is chosen
	at random. Next a ball is randomly chosen from the urn. Then a second ball is
	chosen at random from the same urn without replacing the first ball.
	
	\begin{enumerate}
	\item What is the probability that two black balls are chosen?
	
	\item What is the probability that two balls of opposite colour are chosen?
	\end{enumerate}
	\solution
	%\input{exemplar/11/16/3/12/main1.tex}
\end{enumerate}

	\item A bag contains 4 red and 4 black balls, another bag contains 2 red and 6 black balls. One of the two bags is selected at random and a ball is drawn from the bag which is found to be red. Find the probability that the ball is drawn from the first bag.
\\
\solution
		%\begin{table}[H]
	\centering
\begin{tabular}{|c|c|c|}
\hline
Random variable &Value &Definition\\ \hline
\multirow{3}{*}{X} &0 &Slips of Rs 1\\
&1 &Slips of Rs 5\\
&2 &Slips of Rs 13\\ \hline
\multirow{2}{*}{Y} &0 &Box A\\
&1 &Box B\\\hline
\end{tabular}
\caption{}
\label{tab:Distribution}
\end{table}
See \tabref{tab:Distribution}.
\begin{align}
p_{Y}\brak{k}= \begin{cases} 
      \frac{1}{3} & {k=0} \\
      \frac{2}{3 }& {k=1} 
   \end{cases}
   \\
p_{Y|X}\brak{0|0} = \frac{19}{25}\, 
p_{Y|X}\brak{0|1} = \frac{6}{25}\,
p_{Y|X}\brak{1|0} = \frac{45}{50}\,
p_{Y|X}\brak{1|2} = \frac{5}{50}
\end{align}
The desired probability is the probability that a slip drawn at random is marked other than Rs 1,
\begin{align}
&=1-p_X\brak{0}\\
&= p_X(1) + p_X(2)
\end{align}
Using Bayes theorem,
\begin{align}
&= p_Y\brak{0} \times \pr{Y=0 | X=1} + p_Y\brak{1} \times \pr{Y=1|X=2}\\
&=\frac{1}{3} \times \frac{6}{25} + \frac{2}{3} \times \frac{5}{50}\\
&=\frac{11}{75}
\end{align}

\newpage

%\tableofcontents

\bigskip

\renewcommand{\thefigure}{\theenumi}
\renewcommand{\thetable}{\theenumi}
%\renewcommand{\theequation}{\theenumi}

%\begin{abstract}
%%\boldmath
%In this letter, an algorithm for evaluating the exact analytical bit error rate  (BER)  for the piecewise linear (PL) combiner for  multiple relays is presented. Previous results were available only for upto three relays. The algorithm is unique in the sense that  the actual mathematical expressions, that are prohibitively large, need not be explicitly obtained. The diversity gain due to multiple relays is shown through plots of the analytical BER, well supported by simulations. 
%
%\end{abstract}
% IEEEtran.cls defaults to using nonbold math in the Abstract.
% This preserves the distinction between vectors and scalars. However,
% if the journal you are submitting to favors bold math in the abstract,
% then you can use LaTeX's standard command \boldmath at the very start
% of the abstract to achieve this. Many IEEE journals frown on math
% in the abstract anyway.

% Note that keywords are not normally used for peerreview papers.
%\begin{IEEEkeywords}
%Cooperative diversity, decode and forward, piecewise linear
%\end{IEEEkeywords}



% For peer review papers, you can put extra information on the cover
% page as needed:
% \ifCLASSOPTIONpeerreview
% \begin{center} \bfseries EDICS Category: 3-BBND \end{center}
% \fi
%
% For peerreview papers, this IEEEtran command inserts a page break and
% creates the second title. It will be ignored for other modes.
%\IEEEpeerreviewmaketitle




  \item
  Cards with numbers 2 to 101 are placed in a box. A card is selected at random.Find the probability that the card has
\begin{enumerate}[label=(\roman*)]
	\item an even number 
	\item a square number
\end{enumerate}
\solution
%\begin{table}[H]
	\centering
\begin{tabular}{|c|c|c|}
\hline
Random variable &Value &Definition\\ \hline
\multirow{3}{*}{X} &0 &Slips of Rs 1\\
&1 &Slips of Rs 5\\
&2 &Slips of Rs 13\\ \hline
\multirow{2}{*}{Y} &0 &Box A\\
&1 &Box B\\\hline
\end{tabular}
\caption{}
\label{tab:Distribution}
\end{table}
See \tabref{tab:Distribution}.
\begin{align}
p_{Y}\brak{k}= \begin{cases} 
      \frac{1}{3} & {k=0} \\
      \frac{2}{3 }& {k=1} 
   \end{cases}
   \\
p_{Y|X}\brak{0|0} = \frac{19}{25}\, 
p_{Y|X}\brak{0|1} = \frac{6}{25}\,
p_{Y|X}\brak{1|0} = \frac{45}{50}\,
p_{Y|X}\brak{1|2} = \frac{5}{50}
\end{align}
The desired probability is the probability that a slip drawn at random is marked other than Rs 1,
\begin{align}
&=1-p_X\brak{0}\\
&= p_X(1) + p_X(2)
\end{align}
Using Bayes theorem,
\begin{align}
&= p_Y\brak{0} \times \pr{Y=0 | X=1} + p_Y\brak{1} \times \pr{Y=1|X=2}\\
&=\frac{1}{3} \times \frac{6}{25} + \frac{2}{3} \times \frac{5}{50}\\
&=\frac{11}{75}
\end{align}

\newpage

%\tableofcontents

\bigskip

\renewcommand{\thefigure}{\theenumi}
\renewcommand{\thetable}{\theenumi}
%\renewcommand{\theequation}{\theenumi}

%\begin{abstract}
%%\boldmath
%In this letter, an algorithm for evaluating the exact analytical bit error rate  (BER)  for the piecewise linear (PL) combiner for  multiple relays is presented. Previous results were available only for upto three relays. The algorithm is unique in the sense that  the actual mathematical expressions, that are prohibitively large, need not be explicitly obtained. The diversity gain due to multiple relays is shown through plots of the analytical BER, well supported by simulations. 
%
%\end{abstract}
% IEEEtran.cls defaults to using nonbold math in the Abstract.
% This preserves the distinction between vectors and scalars. However,
% if the journal you are submitting to favors bold math in the abstract,
% then you can use LaTeX's standard command \boldmath at the very start
% of the abstract to achieve this. Many IEEE journals frown on math
% in the abstract anyway.

% Note that keywords are not normally used for peerreview papers.
%\begin{IEEEkeywords}
%Cooperative diversity, decode and forward, piecewise linear
%\end{IEEEkeywords}



% For peer review papers, you can put extra information on the cover
% page as needed:
% \ifCLASSOPTIONpeerreview
% \begin{center} \bfseries EDICS Category: 3-BBND \end{center}
% \fi
%
% For peerreview papers, this IEEEtran command inserts a page break and
% creates the second title. It will be ignored for other modes.
%\IEEEpeerreviewmaketitle




\item
The king, queen and jack of clubs are removed from a deck of 52 playing cards and then well shuffled. Now one card is drawn at random from the remaining cards.  Determine the probability that the card is
\begin{enumerate}[label=(\roman*)]
\item a club
\item 10 of hearts
\end{enumerate}
\solution
%\begin{table}[H]
	\centering
\begin{tabular}{|c|c|c|}
\hline
Random variable &Value &Definition\\ \hline
\multirow{3}{*}{X} &0 &Slips of Rs 1\\
&1 &Slips of Rs 5\\
&2 &Slips of Rs 13\\ \hline
\multirow{2}{*}{Y} &0 &Box A\\
&1 &Box B\\\hline
\end{tabular}
\caption{}
\label{tab:Distribution}
\end{table}
See \tabref{tab:Distribution}.
\begin{align}
p_{Y}\brak{k}= \begin{cases} 
      \frac{1}{3} & {k=0} \\
      \frac{2}{3 }& {k=1} 
   \end{cases}
   \\
p_{Y|X}\brak{0|0} = \frac{19}{25}\, 
p_{Y|X}\brak{0|1} = \frac{6}{25}\,
p_{Y|X}\brak{1|0} = \frac{45}{50}\,
p_{Y|X}\brak{1|2} = \frac{5}{50}
\end{align}
The desired probability is the probability that a slip drawn at random is marked other than Rs 1,
\begin{align}
&=1-p_X\brak{0}\\
&= p_X(1) + p_X(2)
\end{align}
Using Bayes theorem,
\begin{align}
&= p_Y\brak{0} \times \pr{Y=0 | X=1} + p_Y\brak{1} \times \pr{Y=1|X=2}\\
&=\frac{1}{3} \times \frac{6}{25} + \frac{2}{3} \times \frac{5}{50}\\
&=\frac{11}{75}
\end{align}

\newpage

%\tableofcontents

\bigskip

\renewcommand{\thefigure}{\theenumi}
\renewcommand{\thetable}{\theenumi}
%\renewcommand{\theequation}{\theenumi}

%\begin{abstract}
%%\boldmath
%In this letter, an algorithm for evaluating the exact analytical bit error rate  (BER)  for the piecewise linear (PL) combiner for  multiple relays is presented. Previous results were available only for upto three relays. The algorithm is unique in the sense that  the actual mathematical expressions, that are prohibitively large, need not be explicitly obtained. The diversity gain due to multiple relays is shown through plots of the analytical BER, well supported by simulations. 
%
%\end{abstract}
% IEEEtran.cls defaults to using nonbold math in the Abstract.
% This preserves the distinction between vectors and scalars. However,
% if the journal you are submitting to favors bold math in the abstract,
% then you can use LaTeX's standard command \boldmath at the very start
% of the abstract to achieve this. Many IEEE journals frown on math
% in the abstract anyway.

% Note that keywords are not normally used for peerreview papers.
%\begin{IEEEkeywords}
%Cooperative diversity, decode and forward, piecewise linear
%\end{IEEEkeywords}



% For peer review papers, you can put extra information on the cover
% page as needed:
% \ifCLASSOPTIONpeerreview
% \begin{center} \bfseries EDICS Category: 3-BBND \end{center}
% \fi
%
% For peerreview papers, this IEEEtran command inserts a page break and
% creates the second title. It will be ignored for other modes.
%\IEEEpeerreviewmaketitle




\item A team of medical students doing their internship have to assist during surgeries
at a city hospital. The probabilities of surgeries rated as very complex, complex,
routine, simple or very simple are respectively, 0.15, 0.20, 0.31, 0.26, .08. Find
the probabilities that a particular surgery will be rated
\begin{enumerate}
	\item complex or very complex;
	\item neither very complex nor very simple;
	\item routine or complex
	\item routine or simple
\end{enumerate}
\solution
%\begin{table}[H]
	\centering
\begin{tabular}{|c|c|c|}
\hline
Random variable &Value &Definition\\ \hline
\multirow{3}{*}{X} &0 &Slips of Rs 1\\
&1 &Slips of Rs 5\\
&2 &Slips of Rs 13\\ \hline
\multirow{2}{*}{Y} &0 &Box A\\
&1 &Box B\\\hline
\end{tabular}
\caption{}
\label{tab:Distribution}
\end{table}
See \tabref{tab:Distribution}.
\begin{align}
p_{Y}\brak{k}= \begin{cases} 
      \frac{1}{3} & {k=0} \\
      \frac{2}{3 }& {k=1} 
   \end{cases}
   \\
p_{Y|X}\brak{0|0} = \frac{19}{25}\, 
p_{Y|X}\brak{0|1} = \frac{6}{25}\,
p_{Y|X}\brak{1|0} = \frac{45}{50}\,
p_{Y|X}\brak{1|2} = \frac{5}{50}
\end{align}
The desired probability is the probability that a slip drawn at random is marked other than Rs 1,
\begin{align}
&=1-p_X\brak{0}\\
&= p_X(1) + p_X(2)
\end{align}
Using Bayes theorem,
\begin{align}
&= p_Y\brak{0} \times \pr{Y=0 | X=1} + p_Y\brak{1} \times \pr{Y=1|X=2}\\
&=\frac{1}{3} \times \frac{6}{25} + \frac{2}{3} \times \frac{5}{50}\\
&=\frac{11}{75}
\end{align}

\newpage

%\tableofcontents

\bigskip

\renewcommand{\thefigure}{\theenumi}
\renewcommand{\thetable}{\theenumi}
%\renewcommand{\theequation}{\theenumi}

%\begin{abstract}
%%\boldmath
%In this letter, an algorithm for evaluating the exact analytical bit error rate  (BER)  for the piecewise linear (PL) combiner for  multiple relays is presented. Previous results were available only for upto three relays. The algorithm is unique in the sense that  the actual mathematical expressions, that are prohibitively large, need not be explicitly obtained. The diversity gain due to multiple relays is shown through plots of the analytical BER, well supported by simulations. 
%
%\end{abstract}
% IEEEtran.cls defaults to using nonbold math in the Abstract.
% This preserves the distinction between vectors and scalars. However,
% if the journal you are submitting to favors bold math in the abstract,
% then you can use LaTeX's standard command \boldmath at the very start
% of the abstract to achieve this. Many IEEE journals frown on math
% in the abstract anyway.

% Note that keywords are not normally used for peerreview papers.
%\begin{IEEEkeywords}
%Cooperative diversity, decode and forward, piecewise linear
%\end{IEEEkeywords}



% For peer review papers, you can put extra information on the cover
% page as needed:
% \ifCLASSOPTIONpeerreview
% \begin{center} \bfseries EDICS Category: 3-BBND \end{center}
% \fi
%
% For peerreview papers, this IEEEtran command inserts a page break and
% creates the second title. It will be ignored for other modes.
%\IEEEpeerreviewmaketitle




\item A card is selected from a pack of 52 cards.
\begin{enumerate}[label=(\alph*)]
    \item How many points are there in the sample space?
    \item Calculate the probability that the card is an ace of spades.
    \item Calculate the probability that the card is (i) an ace and (ii) black card.
\end{enumerate}
\solution
%Let $X$ be an bernoulli rv defined as in \tabref{tab:exemplar/11/16/3/26}.  Then, 
\begin{equation}
    p =
        \frac{4}{11} 
\end{equation}
\begin{table}[H]
	\centering
	\input{exemplar/11/16/3/26/tables/Table2.tex}
	\caption{}
        \label{tab:exemplar/11/16/3/26}
\end{table}

\item The probability that a non leap year selected at random will contain 53 sundays.
\\
\solution
%\begin{table}[H]
	\centering
\begin{tabular}{|c|c|c|}
\hline
Random variable &Value &Definition\\ \hline
\multirow{3}{*}{X} &0 &Slips of Rs 1\\
&1 &Slips of Rs 5\\
&2 &Slips of Rs 13\\ \hline
\multirow{2}{*}{Y} &0 &Box A\\
&1 &Box B\\\hline
\end{tabular}
\caption{}
\label{tab:Distribution}
\end{table}
See \tabref{tab:Distribution}.
\begin{align}
p_{Y}\brak{k}= \begin{cases} 
      \frac{1}{3} & {k=0} \\
      \frac{2}{3 }& {k=1} 
   \end{cases}
   \\
p_{Y|X}\brak{0|0} = \frac{19}{25}\, 
p_{Y|X}\brak{0|1} = \frac{6}{25}\,
p_{Y|X}\brak{1|0} = \frac{45}{50}\,
p_{Y|X}\brak{1|2} = \frac{5}{50}
\end{align}
The desired probability is the probability that a slip drawn at random is marked other than Rs 1,
\begin{align}
&=1-p_X\brak{0}\\
&= p_X(1) + p_X(2)
\end{align}
Using Bayes theorem,
\begin{align}
&= p_Y\brak{0} \times \pr{Y=0 | X=1} + p_Y\brak{1} \times \pr{Y=1|X=2}\\
&=\frac{1}{3} \times \frac{6}{25} + \frac{2}{3} \times \frac{5}{50}\\
&=\frac{11}{75}
\end{align}

\newpage

%\tableofcontents

\bigskip

\renewcommand{\thefigure}{\theenumi}
\renewcommand{\thetable}{\theenumi}
%\renewcommand{\theequation}{\theenumi}

%\begin{abstract}
%%\boldmath
%In this letter, an algorithm for evaluating the exact analytical bit error rate  (BER)  for the piecewise linear (PL) combiner for  multiple relays is presented. Previous results were available only for upto three relays. The algorithm is unique in the sense that  the actual mathematical expressions, that are prohibitively large, need not be explicitly obtained. The diversity gain due to multiple relays is shown through plots of the analytical BER, well supported by simulations. 
%
%\end{abstract}
% IEEEtran.cls defaults to using nonbold math in the Abstract.
% This preserves the distinction between vectors and scalars. However,
% if the journal you are submitting to favors bold math in the abstract,
% then you can use LaTeX's standard command \boldmath at the very start
% of the abstract to achieve this. Many IEEE journals frown on math
% in the abstract anyway.

% Note that keywords are not normally used for peerreview papers.
%\begin{IEEEkeywords}
%Cooperative diversity, decode and forward, piecewise linear
%\end{IEEEkeywords}



% For peer review papers, you can put extra information on the cover
% page as needed:
% \ifCLASSOPTIONpeerreview
% \begin{center} \bfseries EDICS Category: 3-BBND \end{center}
% \fi
%
% For peerreview papers, this IEEEtran command inserts a page break and
% creates the second title. It will be ignored for other modes.
%\IEEEpeerreviewmaketitle




\item One of the four persons John, Rita, Aslam or Gurpreet will be promoted next
month. Consequently the sample space consists of four elementary outcomes
S = {John promoted, Rita promoted, Aslam promoted, Gurpreet promoted}
You are told that the chances of John’s promotion is same as that of Gurpreet,
Rita’s chances of promotion are twice as likely as Johns. Aslam’s chances are
four times that of John.
\begin{enumerate}
	\item Determine
	\begin{enumerate}
		\item P (John promoted)
		\item P (Rita promoted)
		\item P (Aslam promoted)
		\item P (Gurpreet promoted)
	\end{enumerate}
	\item If A = {John promoted or Gurpreet promoted}, find P (A).
\end{enumerate}
\solution
%\begin{table}[H]
	\centering
\begin{tabular}{|c|c|c|}
\hline
Random variable &Value &Definition\\ \hline
\multirow{3}{*}{X} &0 &Slips of Rs 1\\
&1 &Slips of Rs 5\\
&2 &Slips of Rs 13\\ \hline
\multirow{2}{*}{Y} &0 &Box A\\
&1 &Box B\\\hline
\end{tabular}
\caption{}
\label{tab:Distribution}
\end{table}
See \tabref{tab:Distribution}.
\begin{align}
p_{Y}\brak{k}= \begin{cases} 
      \frac{1}{3} & {k=0} \\
      \frac{2}{3 }& {k=1} 
   \end{cases}
   \\
p_{Y|X}\brak{0|0} = \frac{19}{25}\, 
p_{Y|X}\brak{0|1} = \frac{6}{25}\,
p_{Y|X}\brak{1|0} = \frac{45}{50}\,
p_{Y|X}\brak{1|2} = \frac{5}{50}
\end{align}
The desired probability is the probability that a slip drawn at random is marked other than Rs 1,
\begin{align}
&=1-p_X\brak{0}\\
&= p_X(1) + p_X(2)
\end{align}
Using Bayes theorem,
\begin{align}
&= p_Y\brak{0} \times \pr{Y=0 | X=1} + p_Y\brak{1} \times \pr{Y=1|X=2}\\
&=\frac{1}{3} \times \frac{6}{25} + \frac{2}{3} \times \frac{5}{50}\\
&=\frac{11}{75}
\end{align}

\newpage

%\tableofcontents

\bigskip

\renewcommand{\thefigure}{\theenumi}
\renewcommand{\thetable}{\theenumi}
%\renewcommand{\theequation}{\theenumi}

%\begin{abstract}
%%\boldmath
%In this letter, an algorithm for evaluating the exact analytical bit error rate  (BER)  for the piecewise linear (PL) combiner for  multiple relays is presented. Previous results were available only for upto three relays. The algorithm is unique in the sense that  the actual mathematical expressions, that are prohibitively large, need not be explicitly obtained. The diversity gain due to multiple relays is shown through plots of the analytical BER, well supported by simulations. 
%
%\end{abstract}
% IEEEtran.cls defaults to using nonbold math in the Abstract.
% This preserves the distinction between vectors and scalars. However,
% if the journal you are submitting to favors bold math in the abstract,
% then you can use LaTeX's standard command \boldmath at the very start
% of the abstract to achieve this. Many IEEE journals frown on math
% in the abstract anyway.

% Note that keywords are not normally used for peerreview papers.
%\begin{IEEEkeywords}
%Cooperative diversity, decode and forward, piecewise linear
%\end{IEEEkeywords}



% For peer review papers, you can put extra information on the cover
% page as needed:
% \ifCLASSOPTIONpeerreview
% \begin{center} \bfseries EDICS Category: 3-BBND \end{center}
% \fi
%
% For peerreview papers, this IEEEtran command inserts a page break and
% creates the second title. It will be ignored for other modes.
%\IEEEpeerreviewmaketitle




\item A card is drawn from a deck of 52 cards. Find the probability of getting a king or a heart or a red card.\\
\solution
%\begin{table}[H]
	\centering
\begin{tabular}{|c|c|c|}
\hline
Random variable &Value &Definition\\ \hline
\multirow{3}{*}{X} &0 &Slips of Rs 1\\
&1 &Slips of Rs 5\\
&2 &Slips of Rs 13\\ \hline
\multirow{2}{*}{Y} &0 &Box A\\
&1 &Box B\\\hline
\end{tabular}
\caption{}
\label{tab:Distribution}
\end{table}
See \tabref{tab:Distribution}.
\begin{align}
p_{Y}\brak{k}= \begin{cases} 
      \frac{1}{3} & {k=0} \\
      \frac{2}{3 }& {k=1} 
   \end{cases}
   \\
p_{Y|X}\brak{0|0} = \frac{19}{25}\, 
p_{Y|X}\brak{0|1} = \frac{6}{25}\,
p_{Y|X}\brak{1|0} = \frac{45}{50}\,
p_{Y|X}\brak{1|2} = \frac{5}{50}
\end{align}
The desired probability is the probability that a slip drawn at random is marked other than Rs 1,
\begin{align}
&=1-p_X\brak{0}\\
&= p_X(1) + p_X(2)
\end{align}
Using Bayes theorem,
\begin{align}
&= p_Y\brak{0} \times \pr{Y=0 | X=1} + p_Y\brak{1} \times \pr{Y=1|X=2}\\
&=\frac{1}{3} \times \frac{6}{25} + \frac{2}{3} \times \frac{5}{50}\\
&=\frac{11}{75}
\end{align}

\newpage

%\tableofcontents

\bigskip

\renewcommand{\thefigure}{\theenumi}
\renewcommand{\thetable}{\theenumi}
%\renewcommand{\theequation}{\theenumi}

%\begin{abstract}
%%\boldmath
%In this letter, an algorithm for evaluating the exact analytical bit error rate  (BER)  for the piecewise linear (PL) combiner for  multiple relays is presented. Previous results were available only for upto three relays. The algorithm is unique in the sense that  the actual mathematical expressions, that are prohibitively large, need not be explicitly obtained. The diversity gain due to multiple relays is shown through plots of the analytical BER, well supported by simulations. 
%
%\end{abstract}
% IEEEtran.cls defaults to using nonbold math in the Abstract.
% This preserves the distinction between vectors and scalars. However,
% if the journal you are submitting to favors bold math in the abstract,
% then you can use LaTeX's standard command \boldmath at the very start
% of the abstract to achieve this. Many IEEE journals frown on math
% in the abstract anyway.

% Note that keywords are not normally used for peerreview papers.
%\begin{IEEEkeywords}
%Cooperative diversity, decode and forward, piecewise linear
%\end{IEEEkeywords}



% For peer review papers, you can put extra information on the cover
% page as needed:
% \ifCLASSOPTIONpeerreview
% \begin{center} \bfseries EDICS Category: 3-BBND \end{center}
% \fi
%
% For peerreview papers, this IEEEtran command inserts a page break and
% creates the second title. It will be ignored for other modes.
%\IEEEpeerreviewmaketitle




\item The probability that a student will pass his examination is 0.73, the probability of
the student getting a compartment is 0.13, and the probability that the student will
either pass or get compartment is 0.96. State True or False.\\
\solution
%\begin{table}[H]
	\centering
\begin{tabular}{|c|c|c|}
\hline
Random variable &Value &Definition\\ \hline
\multirow{3}{*}{X} &0 &Slips of Rs 1\\
&1 &Slips of Rs 5\\
&2 &Slips of Rs 13\\ \hline
\multirow{2}{*}{Y} &0 &Box A\\
&1 &Box B\\\hline
\end{tabular}
\caption{}
\label{tab:Distribution}
\end{table}
See \tabref{tab:Distribution}.
\begin{align}
p_{Y}\brak{k}= \begin{cases} 
      \frac{1}{3} & {k=0} \\
      \frac{2}{3 }& {k=1} 
   \end{cases}
   \\
p_{Y|X}\brak{0|0} = \frac{19}{25}\, 
p_{Y|X}\brak{0|1} = \frac{6}{25}\,
p_{Y|X}\brak{1|0} = \frac{45}{50}\,
p_{Y|X}\brak{1|2} = \frac{5}{50}
\end{align}
The desired probability is the probability that a slip drawn at random is marked other than Rs 1,
\begin{align}
&=1-p_X\brak{0}\\
&= p_X(1) + p_X(2)
\end{align}
Using Bayes theorem,
\begin{align}
&= p_Y\brak{0} \times \pr{Y=0 | X=1} + p_Y\brak{1} \times \pr{Y=1|X=2}\\
&=\frac{1}{3} \times \frac{6}{25} + \frac{2}{3} \times \frac{5}{50}\\
&=\frac{11}{75}
\end{align}

\newpage

%\tableofcontents

\bigskip

\renewcommand{\thefigure}{\theenumi}
\renewcommand{\thetable}{\theenumi}
%\renewcommand{\theequation}{\theenumi}

%\begin{abstract}
%%\boldmath
%In this letter, an algorithm for evaluating the exact analytical bit error rate  (BER)  for the piecewise linear (PL) combiner for  multiple relays is presented. Previous results were available only for upto three relays. The algorithm is unique in the sense that  the actual mathematical expressions, that are prohibitively large, need not be explicitly obtained. The diversity gain due to multiple relays is shown through plots of the analytical BER, well supported by simulations. 
%
%\end{abstract}
% IEEEtran.cls defaults to using nonbold math in the Abstract.
% This preserves the distinction between vectors and scalars. However,
% if the journal you are submitting to favors bold math in the abstract,
% then you can use LaTeX's standard command \boldmath at the very start
% of the abstract to achieve this. Many IEEE journals frown on math
% in the abstract anyway.

% Note that keywords are not normally used for peerreview papers.
%\begin{IEEEkeywords}
%Cooperative diversity, decode and forward, piecewise linear
%\end{IEEEkeywords}



% For peer review papers, you can put extra information on the cover
% page as needed:
% \ifCLASSOPTIONpeerreview
% \begin{center} \bfseries EDICS Category: 3-BBND \end{center}
% \fi
%
% For peerreview papers, this IEEEtran command inserts a page break and
% creates the second title. It will be ignored for other modes.
%\IEEEpeerreviewmaketitle




\item A card is selected from a pack of 52 cards\\
\begin{enumerate}[label=(\alph*)]
\item How many points are there in the sample space?
\item Calculate the probability that the cards is an ace of spades.
\item Calculate the probability that the card is (i) an ace (ii)black card.\\
\end{enumerate}
%\input{ncert/11/16/3/4_1/Prob_4.tex}
\item In a non-leap year, the probability of having 53 tuesdays or 53 wednesdays is\\
\solution
%A non-leap year has a total of 365 days, and a week has 7 days.\\
So it can be expressed as 
\begin{align}
365\text{days} &=52\times 7+1 \text{day}
\end{align}
$\implies$ 52 tuesdays or wednesdays\\
Random variable X denotes the days of a week
\begin{align}
p_X\brak{k}&=\frac{1}{7}; \quad \brak{1<k<7}
\end{align}
So the probability of extra day being tuesday or wednesday is
\begin{align}
p_X\brak{3}+p_X\brak{4}&=\frac{1}{7}+\frac{1}{7}=\frac{2}{7}
\end{align}



\item There are 1000 sealed envelopes in a box, 10 of them contain a cash prize of
Rs 100 each, 100 of them contain a cash prize of Rs 50 each and 200 of them
contain a cash prize of Rs 10 each and rest do not contain any cash prize. If they
are well shuffled and an envelope is picked up out, what is the probability that it
contains no cash prize?\\
\solution
%\begin{table}[H]
	\centering
\begin{tabular}{|c|c|c|}
\hline
Random variable &Value &Definition\\ \hline
\multirow{3}{*}{X} &0 &Slips of Rs 1\\
&1 &Slips of Rs 5\\
&2 &Slips of Rs 13\\ \hline
\multirow{2}{*}{Y} &0 &Box A\\
&1 &Box B\\\hline
\end{tabular}
\caption{}
\label{tab:Distribution}
\end{table}
See \tabref{tab:Distribution}.
\begin{align}
p_{Y}\brak{k}= \begin{cases} 
      \frac{1}{3} & {k=0} \\
      \frac{2}{3 }& {k=1} 
   \end{cases}
   \\
p_{Y|X}\brak{0|0} = \frac{19}{25}\, 
p_{Y|X}\brak{0|1} = \frac{6}{25}\,
p_{Y|X}\brak{1|0} = \frac{45}{50}\,
p_{Y|X}\brak{1|2} = \frac{5}{50}
\end{align}
The desired probability is the probability that a slip drawn at random is marked other than Rs 1,
\begin{align}
&=1-p_X\brak{0}\\
&= p_X(1) + p_X(2)
\end{align}
Using Bayes theorem,
\begin{align}
&= p_Y\brak{0} \times \pr{Y=0 | X=1} + p_Y\brak{1} \times \pr{Y=1|X=2}\\
&=\frac{1}{3} \times \frac{6}{25} + \frac{2}{3} \times \frac{5}{50}\\
&=\frac{11}{75}
\end{align}

\newpage

%\tableofcontents

\bigskip

\renewcommand{\thefigure}{\theenumi}
\renewcommand{\thetable}{\theenumi}
%\renewcommand{\theequation}{\theenumi}

%\begin{abstract}
%%\boldmath
%In this letter, an algorithm for evaluating the exact analytical bit error rate  (BER)  for the piecewise linear (PL) combiner for  multiple relays is presented. Previous results were available only for upto three relays. The algorithm is unique in the sense that  the actual mathematical expressions, that are prohibitively large, need not be explicitly obtained. The diversity gain due to multiple relays is shown through plots of the analytical BER, well supported by simulations. 
%
%\end{abstract}
% IEEEtran.cls defaults to using nonbold math in the Abstract.
% This preserves the distinction between vectors and scalars. However,
% if the journal you are submitting to favors bold math in the abstract,
% then you can use LaTeX's standard command \boldmath at the very start
% of the abstract to achieve this. Many IEEE journals frown on math
% in the abstract anyway.

% Note that keywords are not normally used for peerreview papers.
%\begin{IEEEkeywords}
%Cooperative diversity, decode and forward, piecewise linear
%\end{IEEEkeywords}



% For peer review papers, you can put extra information on the cover
% page as needed:
% \ifCLASSOPTIONpeerreview
% \begin{center} \bfseries EDICS Category: 3-BBND \end{center}
% \fi
%
% For peerreview papers, this IEEEtran command inserts a page break and
% creates the second title. It will be ignored for other modes.
%\IEEEpeerreviewmaketitle




\item 
A die is thrown and a card is selected at random from a deck of 52 playing cards. The probability of getting an even number on the die and a spade card.\\
\solution
%\begin{table}[H]
	\centering
\begin{tabular}{|c|c|c|}
\hline
Random variable &Value &Definition\\ \hline
\multirow{3}{*}{X} &0 &Slips of Rs 1\\
&1 &Slips of Rs 5\\
&2 &Slips of Rs 13\\ \hline
\multirow{2}{*}{Y} &0 &Box A\\
&1 &Box B\\\hline
\end{tabular}
\caption{}
\label{tab:Distribution}
\end{table}
See \tabref{tab:Distribution}.
\begin{align}
p_{Y}\brak{k}= \begin{cases} 
      \frac{1}{3} & {k=0} \\
      \frac{2}{3 }& {k=1} 
   \end{cases}
   \\
p_{Y|X}\brak{0|0} = \frac{19}{25}\, 
p_{Y|X}\brak{0|1} = \frac{6}{25}\,
p_{Y|X}\brak{1|0} = \frac{45}{50}\,
p_{Y|X}\brak{1|2} = \frac{5}{50}
\end{align}
The desired probability is the probability that a slip drawn at random is marked other than Rs 1,
\begin{align}
&=1-p_X\brak{0}\\
&= p_X(1) + p_X(2)
\end{align}
Using Bayes theorem,
\begin{align}
&= p_Y\brak{0} \times \pr{Y=0 | X=1} + p_Y\brak{1} \times \pr{Y=1|X=2}\\
&=\frac{1}{3} \times \frac{6}{25} + \frac{2}{3} \times \frac{5}{50}\\
&=\frac{11}{75}
\end{align}

\newpage

%\tableofcontents

\bigskip

\renewcommand{\thefigure}{\theenumi}
\renewcommand{\thetable}{\theenumi}
%\renewcommand{\theequation}{\theenumi}

%\begin{abstract}
%%\boldmath
%In this letter, an algorithm for evaluating the exact analytical bit error rate  (BER)  for the piecewise linear (PL) combiner for  multiple relays is presented. Previous results were available only for upto three relays. The algorithm is unique in the sense that  the actual mathematical expressions, that are prohibitively large, need not be explicitly obtained. The diversity gain due to multiple relays is shown through plots of the analytical BER, well supported by simulations. 
%
%\end{abstract}
% IEEEtran.cls defaults to using nonbold math in the Abstract.
% This preserves the distinction between vectors and scalars. However,
% if the journal you are submitting to favors bold math in the abstract,
% then you can use LaTeX's standard command \boldmath at the very start
% of the abstract to achieve this. Many IEEE journals frown on math
% in the abstract anyway.

% Note that keywords are not normally used for peerreview papers.
%\begin{IEEEkeywords}
%Cooperative diversity, decode and forward, piecewise linear
%\end{IEEEkeywords}



% For peer review papers, you can put extra information on the cover
% page as needed:
% \ifCLASSOPTIONpeerreview
% \begin{center} \bfseries EDICS Category: 3-BBND \end{center}
% \fi
%
% For peerreview papers, this IEEEtran command inserts a page break and
% creates the second title. It will be ignored for other modes.
%\IEEEpeerreviewmaketitle




\item
If 4-digit numbers greater than 5,000 are randomly formed from the digits 0, 1, 3, 5, and 7, what is the probability of forming a number divisible by 5 when:
\begin{enumerate}
    \item The digits are repeated?
    \item The repetition of digits is not allowed?
\end{enumerate}
\solution
%\begin{table}[H]
	\centering
\begin{tabular}{|c|c|c|}
\hline
Random variable &Value &Definition\\ \hline
\multirow{3}{*}{X} &0 &Slips of Rs 1\\
&1 &Slips of Rs 5\\
&2 &Slips of Rs 13\\ \hline
\multirow{2}{*}{Y} &0 &Box A\\
&1 &Box B\\\hline
\end{tabular}
\caption{}
\label{tab:Distribution}
\end{table}
See \tabref{tab:Distribution}.
\begin{align}
p_{Y}\brak{k}= \begin{cases} 
      \frac{1}{3} & {k=0} \\
      \frac{2}{3 }& {k=1} 
   \end{cases}
   \\
p_{Y|X}\brak{0|0} = \frac{19}{25}\, 
p_{Y|X}\brak{0|1} = \frac{6}{25}\,
p_{Y|X}\brak{1|0} = \frac{45}{50}\,
p_{Y|X}\brak{1|2} = \frac{5}{50}
\end{align}
The desired probability is the probability that a slip drawn at random is marked other than Rs 1,
\begin{align}
&=1-p_X\brak{0}\\
&= p_X(1) + p_X(2)
\end{align}
Using Bayes theorem,
\begin{align}
&= p_Y\brak{0} \times \pr{Y=0 | X=1} + p_Y\brak{1} \times \pr{Y=1|X=2}\\
&=\frac{1}{3} \times \frac{6}{25} + \frac{2}{3} \times \frac{5}{50}\\
&=\frac{11}{75}
\end{align}

\newpage

%\tableofcontents

\bigskip

\renewcommand{\thefigure}{\theenumi}
\renewcommand{\thetable}{\theenumi}
%\renewcommand{\theequation}{\theenumi}

%\begin{abstract}
%%\boldmath
%In this letter, an algorithm for evaluating the exact analytical bit error rate  (BER)  for the piecewise linear (PL) combiner for  multiple relays is presented. Previous results were available only for upto three relays. The algorithm is unique in the sense that  the actual mathematical expressions, that are prohibitively large, need not be explicitly obtained. The diversity gain due to multiple relays is shown through plots of the analytical BER, well supported by simulations. 
%
%\end{abstract}
% IEEEtran.cls defaults to using nonbold math in the Abstract.
% This preserves the distinction between vectors and scalars. However,
% if the journal you are submitting to favors bold math in the abstract,
% then you can use LaTeX's standard command \boldmath at the very start
% of the abstract to achieve this. Many IEEE journals frown on math
% in the abstract anyway.

% Note that keywords are not normally used for peerreview papers.
%\begin{IEEEkeywords}
%Cooperative diversity, decode and forward, piecewise linear
%\end{IEEEkeywords}



% For peer review papers, you can put extra information on the cover
% page as needed:
% \ifCLASSOPTIONpeerreview
% \begin{center} \bfseries EDICS Category: 3-BBND \end{center}
% \fi
%
% For peerreview papers, this IEEEtran command inserts a page break and
% creates the second title. It will be ignored for other modes.
%\IEEEpeerreviewmaketitle




\item Consider the probability space $\brak{\Omega, \mathcal{G}, P}$ where $\Omega = [0,2]$ and $\mathcal{G} = \cbrak{\phi, \Omega, [0,1], (1,2]}$. Let $X$ and $Y$ be two functions on $\Omega$ defined as
\begin{align*}
    X(\omega) = 
    \begin{cases}
        1 & \text{if }\omega \in [0, 1]\\
        2 & \text{if }\omega \in (1, 2]
    \end{cases}
\end{align*}
and
\begin{align*}
    Y(\omega) = 
    \begin{cases}
        2 & \text{if }\omega \in [0, 1.5]\\
        3 & \text{if }\omega \in (1.5, 2].
    \end{cases}
\end{align*}
Then which one of the following statements is true?
\begin{enumerate}
    \item [(A)] $X$ is a random variable with respect to $\mathcal{G}$, but $Y$ is not a random variable with respect to $\mathcal{G}$.
    \item [(B)] $Y$ is a random variable with respect to $\mathcal{G}$, but $X$ is not a random variable with respect to $\mathcal{G}$.
    \item [(C)] Neither $X$ nor $Y$ is a random variable with respect to $\mathcal{G}$.
    \item [(D)] Both $X$ and $Y$ are random variables with respect to $\mathcal{G}$.
\end{enumerate} \hfill (GATE ST 2023)\\
\solution
%\begin{table}[H]
	\centering
\begin{tabular}{|c|c|c|}
\hline
Random variable &Value &Definition\\ \hline
\multirow{3}{*}{X} &0 &Slips of Rs 1\\
&1 &Slips of Rs 5\\
&2 &Slips of Rs 13\\ \hline
\multirow{2}{*}{Y} &0 &Box A\\
&1 &Box B\\\hline
\end{tabular}
\caption{}
\label{tab:Distribution}
\end{table}
See \tabref{tab:Distribution}.
\begin{align}
p_{Y}\brak{k}= \begin{cases} 
      \frac{1}{3} & {k=0} \\
      \frac{2}{3 }& {k=1} 
   \end{cases}
   \\
p_{Y|X}\brak{0|0} = \frac{19}{25}\, 
p_{Y|X}\brak{0|1} = \frac{6}{25}\,
p_{Y|X}\brak{1|0} = \frac{45}{50}\,
p_{Y|X}\brak{1|2} = \frac{5}{50}
\end{align}
The desired probability is the probability that a slip drawn at random is marked other than Rs 1,
\begin{align}
&=1-p_X\brak{0}\\
&= p_X(1) + p_X(2)
\end{align}
Using Bayes theorem,
\begin{align}
&= p_Y\brak{0} \times \pr{Y=0 | X=1} + p_Y\brak{1} \times \pr{Y=1|X=2}\\
&=\frac{1}{3} \times \frac{6}{25} + \frac{2}{3} \times \frac{5}{50}\\
&=\frac{11}{75}
\end{align}

\newpage

%\tableofcontents

\bigskip

\renewcommand{\thefigure}{\theenumi}
\renewcommand{\thetable}{\theenumi}
%\renewcommand{\theequation}{\theenumi}

%\begin{abstract}
%%\boldmath
%In this letter, an algorithm for evaluating the exact analytical bit error rate  (BER)  for the piecewise linear (PL) combiner for  multiple relays is presented. Previous results were available only for upto three relays. The algorithm is unique in the sense that  the actual mathematical expressions, that are prohibitively large, need not be explicitly obtained. The diversity gain due to multiple relays is shown through plots of the analytical BER, well supported by simulations. 
%
%\end{abstract}
% IEEEtran.cls defaults to using nonbold math in the Abstract.
% This preserves the distinction between vectors and scalars. However,
% if the journal you are submitting to favors bold math in the abstract,
% then you can use LaTeX's standard command \boldmath at the very start
% of the abstract to achieve this. Many IEEE journals frown on math
% in the abstract anyway.

% Note that keywords are not normally used for peerreview papers.
%\begin{IEEEkeywords}
%Cooperative diversity, decode and forward, piecewise linear
%\end{IEEEkeywords}



% For peer review papers, you can put extra information on the cover
% page as needed:
% \ifCLASSOPTIONpeerreview
% \begin{center} \bfseries EDICS Category: 3-BBND \end{center}
% \fi
%
% For peerreview papers, this IEEEtran command inserts a page break and
% creates the second title. It will be ignored for other modes.
%\IEEEpeerreviewmaketitle




	\item  A die is loaded in such a way that each odd number is twice as likely to occur as
each even number. Find $P(G)$, where $G$ is the event that a number greater than
3 occurs on a single roll of the die.
\\
\solution
		%\begin{table}[H]
	\centering
\begin{tabular}{|c|c|c|}
\hline
Random variable &Value &Definition\\ \hline
\multirow{3}{*}{X} &0 &Slips of Rs 1\\
&1 &Slips of Rs 5\\
&2 &Slips of Rs 13\\ \hline
\multirow{2}{*}{Y} &0 &Box A\\
&1 &Box B\\\hline
\end{tabular}
\caption{}
\label{tab:Distribution}
\end{table}
See \tabref{tab:Distribution}.
\begin{align}
p_{Y}\brak{k}= \begin{cases} 
      \frac{1}{3} & {k=0} \\
      \frac{2}{3 }& {k=1} 
   \end{cases}
   \\
p_{Y|X}\brak{0|0} = \frac{19}{25}\, 
p_{Y|X}\brak{0|1} = \frac{6}{25}\,
p_{Y|X}\brak{1|0} = \frac{45}{50}\,
p_{Y|X}\brak{1|2} = \frac{5}{50}
\end{align}
The desired probability is the probability that a slip drawn at random is marked other than Rs 1,
\begin{align}
&=1-p_X\brak{0}\\
&= p_X(1) + p_X(2)
\end{align}
Using Bayes theorem,
\begin{align}
&= p_Y\brak{0} \times \pr{Y=0 | X=1} + p_Y\brak{1} \times \pr{Y=1|X=2}\\
&=\frac{1}{3} \times \frac{6}{25} + \frac{2}{3} \times \frac{5}{50}\\
&=\frac{11}{75}
\end{align}

\newpage

%\tableofcontents

\bigskip

\renewcommand{\thefigure}{\theenumi}
\renewcommand{\thetable}{\theenumi}
%\renewcommand{\theequation}{\theenumi}

%\begin{abstract}
%%\boldmath
%In this letter, an algorithm for evaluating the exact analytical bit error rate  (BER)  for the piecewise linear (PL) combiner for  multiple relays is presented. Previous results were available only for upto three relays. The algorithm is unique in the sense that  the actual mathematical expressions, that are prohibitively large, need not be explicitly obtained. The diversity gain due to multiple relays is shown through plots of the analytical BER, well supported by simulations. 
%
%\end{abstract}
% IEEEtran.cls defaults to using nonbold math in the Abstract.
% This preserves the distinction between vectors and scalars. However,
% if the journal you are submitting to favors bold math in the abstract,
% then you can use LaTeX's standard command \boldmath at the very start
% of the abstract to achieve this. Many IEEE journals frown on math
% in the abstract anyway.

% Note that keywords are not normally used for peerreview papers.
%\begin{IEEEkeywords}
%Cooperative diversity, decode and forward, piecewise linear
%\end{IEEEkeywords}



% For peer review papers, you can put extra information on the cover
% page as needed:
% \ifCLASSOPTIONpeerreview
% \begin{center} \bfseries EDICS Category: 3-BBND \end{center}
% \fi
%
% For peerreview papers, this IEEEtran command inserts a page break and
% creates the second title. It will be ignored for other modes.
%\IEEEpeerreviewmaketitle




	\item All the jacks, queens and kings are removed from a deck of 52 playing cards. The remaining cards are well shuffled and then one card is drawn at random. Giving ace a value 1 similar value for other cards, find the probability that the card has a value 
		\begin{enumerate}
			\item 7
			\item greater than 7
			\item less than 7
		\end{enumerate}
		%Number of cards left after removing all jacks, queens and kings 
\begin{align}
N	= 52 - 4\times 3
	= 40
\end{align}
%\begin{table}[H]
%\def\arraystretch{1.2}
%\begin{tabular}{|c|c|c|}
%\hline
%	\textbf{Parameter} &\textbf{Value} &\textbf{Description}\\ \hline
%	$X$ &1-10 &Represents the value of the card picked \\ \hline
%\end{tabular}
%\end{table}
Let $1 \le X \le 10$ be the value of the card picked.  Then,
\begin{align}
	p_X(k) &= \Pr(X=k)\ \forall\ 1 \leq k \leq 10\\
	&= \frac{4\times 1}{40}\\
	&= \frac{1}{10}\\
	\therefore p_X(k) &= 
	\begin{cases}
		\frac{1}{10} & 1 \leq k \leq 10\\
		0 & \text{otherwise}
	\end{cases}
\end{align}
and
\begin{align}
	F_{X}(k) &= \sum_{m=0}^{k}p_{X}(m) \quad 1 \leq k \leq 10\\
	&= \frac{k}{10}\\
	\therefore F_{X}(k) &= 
	\begin{cases}
		0 & k \leq 0\\
		\frac{k}{10} & 1\leq k \leq 10\\
		1 & k > 10 
	\end{cases}
\end{align}
\begin{enumerate}
	\item Probability that card has value equal to 7 is
		\begin{align}
			 p_{X}(7)
			= \frac{1}{10}
		\end{align}
	\item Probability that card has value greater than 7 is
		\begin{align}
			1 - F_X(7)
			&= 1 - \frac{7}{10}
			\\
			&= \frac{3}{10}
		\end{align}
	\item Probability that card has value less than 7 is
		\begin{align}
			 F_{X}(6)
			=\frac{6}{10}
		\end{align}
\end{enumerate}

  \item A Lot consists of 48 mobile phones of which 42 are good, 3 have only minor defects and 3 have major defects.Varnika will buy a phone if it is good but the trader will only buy a mobile if it has no major defects. One phone is selected at random from the lot. What is the probability that it is
\begin{enumerate}
	\item acceptable to Varnika?
            \item acceptable to the trader?
\end{enumerate}
\solution
	%\begin{table}[H]
	\centering
\begin{tabular}{|c|c|c|}
\hline
Random variable &Value &Definition\\ \hline
\multirow{3}{*}{X} &0 &Slips of Rs 1\\
&1 &Slips of Rs 5\\
&2 &Slips of Rs 13\\ \hline
\multirow{2}{*}{Y} &0 &Box A\\
&1 &Box B\\\hline
\end{tabular}
\caption{}
\label{tab:Distribution}
\end{table}
See \tabref{tab:Distribution}.
\begin{align}
p_{Y}\brak{k}= \begin{cases} 
      \frac{1}{3} & {k=0} \\
      \frac{2}{3 }& {k=1} 
   \end{cases}
   \\
p_{Y|X}\brak{0|0} = \frac{19}{25}\, 
p_{Y|X}\brak{0|1} = \frac{6}{25}\,
p_{Y|X}\brak{1|0} = \frac{45}{50}\,
p_{Y|X}\brak{1|2} = \frac{5}{50}
\end{align}
The desired probability is the probability that a slip drawn at random is marked other than Rs 1,
\begin{align}
&=1-p_X\brak{0}\\
&= p_X(1) + p_X(2)
\end{align}
Using Bayes theorem,
\begin{align}
&= p_Y\brak{0} \times \pr{Y=0 | X=1} + p_Y\brak{1} \times \pr{Y=1|X=2}\\
&=\frac{1}{3} \times \frac{6}{25} + \frac{2}{3} \times \frac{5}{50}\\
&=\frac{11}{75}
\end{align}

\newpage

%\tableofcontents

\bigskip

\renewcommand{\thefigure}{\theenumi}
\renewcommand{\thetable}{\theenumi}
%\renewcommand{\theequation}{\theenumi}

%\begin{abstract}
%%\boldmath
%In this letter, an algorithm for evaluating the exact analytical bit error rate  (BER)  for the piecewise linear (PL) combiner for  multiple relays is presented. Previous results were available only for upto three relays. The algorithm is unique in the sense that  the actual mathematical expressions, that are prohibitively large, need not be explicitly obtained. The diversity gain due to multiple relays is shown through plots of the analytical BER, well supported by simulations. 
%
%\end{abstract}
% IEEEtran.cls defaults to using nonbold math in the Abstract.
% This preserves the distinction between vectors and scalars. However,
% if the journal you are submitting to favors bold math in the abstract,
% then you can use LaTeX's standard command \boldmath at the very start
% of the abstract to achieve this. Many IEEE journals frown on math
% in the abstract anyway.

% Note that keywords are not normally used for peerreview papers.
%\begin{IEEEkeywords}
%Cooperative diversity, decode and forward, piecewise linear
%\end{IEEEkeywords}



% For peer review papers, you can put extra information on the cover
% page as needed:
% \ifCLASSOPTIONpeerreview
% \begin{center} \bfseries EDICS Category: 3-BBND \end{center}
% \fi
%
% For peerreview papers, this IEEEtran command inserts a page break and
% creates the second title. It will be ignored for other modes.
%\IEEEpeerreviewmaketitle




 \item A student says that if you throw a die, it will show up 1 or not 1. Therefore, the probability of getting 1 and the probability of getting 'not 1' each is equal to $\frac{1}{2}$. Is this correct? Give reasons.\\
 \solution
        %\begin{table}[H]
	\centering
\begin{tabular}{|c|c|c|}
\hline
Random variable &Value &Definition\\ \hline
\multirow{3}{*}{X} &0 &Slips of Rs 1\\
&1 &Slips of Rs 5\\
&2 &Slips of Rs 13\\ \hline
\multirow{2}{*}{Y} &0 &Box A\\
&1 &Box B\\\hline
\end{tabular}
\caption{}
\label{tab:Distribution}
\end{table}
See \tabref{tab:Distribution}.
\begin{align}
p_{Y}\brak{k}= \begin{cases} 
      \frac{1}{3} & {k=0} \\
      \frac{2}{3 }& {k=1} 
   \end{cases}
   \\
p_{Y|X}\brak{0|0} = \frac{19}{25}\, 
p_{Y|X}\brak{0|1} = \frac{6}{25}\,
p_{Y|X}\brak{1|0} = \frac{45}{50}\,
p_{Y|X}\brak{1|2} = \frac{5}{50}
\end{align}
The desired probability is the probability that a slip drawn at random is marked other than Rs 1,
\begin{align}
&=1-p_X\brak{0}\\
&= p_X(1) + p_X(2)
\end{align}
Using Bayes theorem,
\begin{align}
&= p_Y\brak{0} \times \pr{Y=0 | X=1} + p_Y\brak{1} \times \pr{Y=1|X=2}\\
&=\frac{1}{3} \times \frac{6}{25} + \frac{2}{3} \times \frac{5}{50}\\
&=\frac{11}{75}
\end{align}

\newpage

%\tableofcontents

\bigskip

\renewcommand{\thefigure}{\theenumi}
\renewcommand{\thetable}{\theenumi}
%\renewcommand{\theequation}{\theenumi}

%\begin{abstract}
%%\boldmath
%In this letter, an algorithm for evaluating the exact analytical bit error rate  (BER)  for the piecewise linear (PL) combiner for  multiple relays is presented. Previous results were available only for upto three relays. The algorithm is unique in the sense that  the actual mathematical expressions, that are prohibitively large, need not be explicitly obtained. The diversity gain due to multiple relays is shown through plots of the analytical BER, well supported by simulations. 
%
%\end{abstract}
% IEEEtran.cls defaults to using nonbold math in the Abstract.
% This preserves the distinction between vectors and scalars. However,
% if the journal you are submitting to favors bold math in the abstract,
% then you can use LaTeX's standard command \boldmath at the very start
% of the abstract to achieve this. Many IEEE journals frown on math
% in the abstract anyway.

% Note that keywords are not normally used for peerreview papers.
%\begin{IEEEkeywords}
%Cooperative diversity, decode and forward, piecewise linear
%\end{IEEEkeywords}



% For peer review papers, you can put extra information on the cover
% page as needed:
% \ifCLASSOPTIONpeerreview
% \begin{center} \bfseries EDICS Category: 3-BBND \end{center}
% \fi
%
% For peerreview papers, this IEEEtran command inserts a page break and
% creates the second title. It will be ignored for other modes.
%\IEEEpeerreviewmaketitle




   \item Four candidates A, B, C, D have ap-
plied for the assignment to coach a school cricket
team. If A is twice as likely to be selected as B, and
B and C are given about the same chance of being
selected, while C is twice as likely to be selected
as D, what are the probabilities that
\begin{enumerate}
\item C will be selected?
\item A will not be selected?
\end{enumerate}
	%\begin{table}[H]
	\centering
\begin{tabular}{|c|c|c|}
\hline
Random variable &Value &Definition\\ \hline
\multirow{3}{*}{X} &0 &Slips of Rs 1\\
&1 &Slips of Rs 5\\
&2 &Slips of Rs 13\\ \hline
\multirow{2}{*}{Y} &0 &Box A\\
&1 &Box B\\\hline
\end{tabular}
\caption{}
\label{tab:Distribution}
\end{table}
See \tabref{tab:Distribution}.
\begin{align}
p_{Y}\brak{k}= \begin{cases} 
      \frac{1}{3} & {k=0} \\
      \frac{2}{3 }& {k=1} 
   \end{cases}
   \\
p_{Y|X}\brak{0|0} = \frac{19}{25}\, 
p_{Y|X}\brak{0|1} = \frac{6}{25}\,
p_{Y|X}\brak{1|0} = \frac{45}{50}\,
p_{Y|X}\brak{1|2} = \frac{5}{50}
\end{align}
The desired probability is the probability that a slip drawn at random is marked other than Rs 1,
\begin{align}
&=1-p_X\brak{0}\\
&= p_X(1) + p_X(2)
\end{align}
Using Bayes theorem,
\begin{align}
&= p_Y\brak{0} \times \pr{Y=0 | X=1} + p_Y\brak{1} \times \pr{Y=1|X=2}\\
&=\frac{1}{3} \times \frac{6}{25} + \frac{2}{3} \times \frac{5}{50}\\
&=\frac{11}{75}
\end{align}

\newpage

%\tableofcontents

\bigskip

\renewcommand{\thefigure}{\theenumi}
\renewcommand{\thetable}{\theenumi}
%\renewcommand{\theequation}{\theenumi}

%\begin{abstract}
%%\boldmath
%In this letter, an algorithm for evaluating the exact analytical bit error rate  (BER)  for the piecewise linear (PL) combiner for  multiple relays is presented. Previous results were available only for upto three relays. The algorithm is unique in the sense that  the actual mathematical expressions, that are prohibitively large, need not be explicitly obtained. The diversity gain due to multiple relays is shown through plots of the analytical BER, well supported by simulations. 
%
%\end{abstract}
% IEEEtran.cls defaults to using nonbold math in the Abstract.
% This preserves the distinction between vectors and scalars. However,
% if the journal you are submitting to favors bold math in the abstract,
% then you can use LaTeX's standard command \boldmath at the very start
% of the abstract to achieve this. Many IEEE journals frown on math
% in the abstract anyway.

% Note that keywords are not normally used for peerreview papers.
%\begin{IEEEkeywords}
%Cooperative diversity, decode and forward, piecewise linear
%\end{IEEEkeywords}



% For peer review papers, you can put extra information on the cover
% page as needed:
% \ifCLASSOPTIONpeerreview
% \begin{center} \bfseries EDICS Category: 3-BBND \end{center}
% \fi
%
% For peerreview papers, this IEEEtran command inserts a page break and
% creates the second title. It will be ignored for other modes.
%\IEEEpeerreviewmaketitle




 \item A bag contain 24 balls of which $x$ balls are red, $2x$ are white and $3x$ are blue. A ball is selected at random, What is the probability that it is
\begin{enumerate}[label=\alph*)]
\item not red ?
\item white ?
\end{enumerate}
%\begin{table}[H]
	\centering
\begin{tabular}{|c|c|c|}
\hline
Random variable &Value &Definition\\ \hline
\multirow{3}{*}{X} &0 &Slips of Rs 1\\
&1 &Slips of Rs 5\\
&2 &Slips of Rs 13\\ \hline
\multirow{2}{*}{Y} &0 &Box A\\
&1 &Box B\\\hline
\end{tabular}
\caption{}
\label{tab:Distribution}
\end{table}
See \tabref{tab:Distribution}.
\begin{align}
p_{Y}\brak{k}= \begin{cases} 
      \frac{1}{3} & {k=0} \\
      \frac{2}{3 }& {k=1} 
   \end{cases}
   \\
p_{Y|X}\brak{0|0} = \frac{19}{25}\, 
p_{Y|X}\brak{0|1} = \frac{6}{25}\,
p_{Y|X}\brak{1|0} = \frac{45}{50}\,
p_{Y|X}\brak{1|2} = \frac{5}{50}
\end{align}
The desired probability is the probability that a slip drawn at random is marked other than Rs 1,
\begin{align}
&=1-p_X\brak{0}\\
&= p_X(1) + p_X(2)
\end{align}
Using Bayes theorem,
\begin{align}
&= p_Y\brak{0} \times \pr{Y=0 | X=1} + p_Y\brak{1} \times \pr{Y=1|X=2}\\
&=\frac{1}{3} \times \frac{6}{25} + \frac{2}{3} \times \frac{5}{50}\\
&=\frac{11}{75}
\end{align}

\newpage

%\tableofcontents

\bigskip

\renewcommand{\thefigure}{\theenumi}
\renewcommand{\thetable}{\theenumi}
%\renewcommand{\theequation}{\theenumi}

%\begin{abstract}
%%\boldmath
%In this letter, an algorithm for evaluating the exact analytical bit error rate  (BER)  for the piecewise linear (PL) combiner for  multiple relays is presented. Previous results were available only for upto three relays. The algorithm is unique in the sense that  the actual mathematical expressions, that are prohibitively large, need not be explicitly obtained. The diversity gain due to multiple relays is shown through plots of the analytical BER, well supported by simulations. 
%
%\end{abstract}
% IEEEtran.cls defaults to using nonbold math in the Abstract.
% This preserves the distinction between vectors and scalars. However,
% if the journal you are submitting to favors bold math in the abstract,
% then you can use LaTeX's standard command \boldmath at the very start
% of the abstract to achieve this. Many IEEE journals frown on math
% in the abstract anyway.

% Note that keywords are not normally used for peerreview papers.
%\begin{IEEEkeywords}
%Cooperative diversity, decode and forward, piecewise linear
%\end{IEEEkeywords}



% For peer review papers, you can put extra information on the cover
% page as needed:
% \ifCLASSOPTIONpeerreview
% \begin{center} \bfseries EDICS Category: 3-BBND \end{center}
% \fi
%
% For peerreview papers, this IEEEtran command inserts a page break and
% creates the second title. It will be ignored for other modes.
%\IEEEpeerreviewmaketitle




If the letters of the word ASSASSINATION are arranged at random. Find the Probability that
\begin{enumerate}[label=(\alph*)]
\item Four $S's$ come consecutively in the word
\item Two  $I's$ and two $N's$ come together
\item All $A's$ are not coming together
\item No two $A's$ are coming together
\end{enumerate}
%\begin{table}[H]
	\centering
\begin{tabular}{|c|c|c|}
\hline
Random variable &Value &Definition\\ \hline
\multirow{3}{*}{X} &0 &Slips of Rs 1\\
&1 &Slips of Rs 5\\
&2 &Slips of Rs 13\\ \hline
\multirow{2}{*}{Y} &0 &Box A\\
&1 &Box B\\\hline
\end{tabular}
\caption{}
\label{tab:Distribution}
\end{table}
See \tabref{tab:Distribution}.
\begin{align}
p_{Y}\brak{k}= \begin{cases} 
      \frac{1}{3} & {k=0} \\
      \frac{2}{3 }& {k=1} 
   \end{cases}
   \\
p_{Y|X}\brak{0|0} = \frac{19}{25}\, 
p_{Y|X}\brak{0|1} = \frac{6}{25}\,
p_{Y|X}\brak{1|0} = \frac{45}{50}\,
p_{Y|X}\brak{1|2} = \frac{5}{50}
\end{align}
The desired probability is the probability that a slip drawn at random is marked other than Rs 1,
\begin{align}
&=1-p_X\brak{0}\\
&= p_X(1) + p_X(2)
\end{align}
Using Bayes theorem,
\begin{align}
&= p_Y\brak{0} \times \pr{Y=0 | X=1} + p_Y\brak{1} \times \pr{Y=1|X=2}\\
&=\frac{1}{3} \times \frac{6}{25} + \frac{2}{3} \times \frac{5}{50}\\
&=\frac{11}{75}
\end{align}

\newpage

%\tableofcontents

\bigskip

\renewcommand{\thefigure}{\theenumi}
\renewcommand{\thetable}{\theenumi}
%\renewcommand{\theequation}{\theenumi}

%\begin{abstract}
%%\boldmath
%In this letter, an algorithm for evaluating the exact analytical bit error rate  (BER)  for the piecewise linear (PL) combiner for  multiple relays is presented. Previous results were available only for upto three relays. The algorithm is unique in the sense that  the actual mathematical expressions, that are prohibitively large, need not be explicitly obtained. The diversity gain due to multiple relays is shown through plots of the analytical BER, well supported by simulations. 
%
%\end{abstract}
% IEEEtran.cls defaults to using nonbold math in the Abstract.
% This preserves the distinction between vectors and scalars. However,
% if the journal you are submitting to favors bold math in the abstract,
% then you can use LaTeX's standard command \boldmath at the very start
% of the abstract to achieve this. Many IEEE journals frown on math
% in the abstract anyway.

% Note that keywords are not normally used for peerreview papers.
%\begin{IEEEkeywords}
%Cooperative diversity, decode and forward, piecewise linear
%\end{IEEEkeywords}



% For peer review papers, you can put extra information on the cover
% page as needed:
% \ifCLASSOPTIONpeerreview
% \begin{center} \bfseries EDICS Category: 3-BBND \end{center}
% \fi
%
% For peerreview papers, this IEEEtran command inserts a page break and
% creates the second title. It will be ignored for other modes.
%\IEEEpeerreviewmaketitle




	\item One urn contains two black balls (labelled B1 and B2) and one white ball. A
	second urn contains one black ball and two white balls (labelled W1 and W2).
	Suppose the following experiment is performed. One of the two urns is chosen
	at random. Next a ball is randomly chosen from the urn. Then a second ball is
	chosen at random from the same urn without replacing the first ball.
	
	\begin{enumerate}
	\item What is the probability that two black balls are chosen?
	
	\item What is the probability that two balls of opposite colour are chosen?
	\end{enumerate}
	\solution
	%\begin{align}
    \label{eq:12.13.6.18.1}
	\because	\pr{A|B} &> \pr{A},\
\frac{\pr{AB}}{\pr{B}} > \pr{A}
\\
    \label{eq:12.13.6.18.2}
	\implies \pr{AB} &> \pr{A}\pr{B}
	\\
	\text{or, } \frac{\pr{AB}}{\pr{A}} &=\pr{B|A} > \pr{A}
\end{align}

\end{enumerate}

	\item A bag contains $5$ red balls and some blue balls. If the probability of drawing a blue ball is double that if a red ball, determine the number of blue balls in the bag. 
		\\
\solution
		%\begin{enumerate}[label=\thesection.\arabic*,ref=\thesection.\theenumi]
	\item One card is drawn from a well-shuffled deck of 52 cards. Find the probability of getting
\begin{enumerate}
\item A king of red colour 
\item A face card 
\item A red face card
\item The jack of hearts
\item A spade
\item The queen of diamonds

\end{enumerate}
\solution
		%\begin{table}[H]
	\centering
\begin{tabular}{|c|c|c|}
\hline
Random variable &Value &Definition\\ \hline
\multirow{3}{*}{X} &0 &Slips of Rs 1\\
&1 &Slips of Rs 5\\
&2 &Slips of Rs 13\\ \hline
\multirow{2}{*}{Y} &0 &Box A\\
&1 &Box B\\\hline
\end{tabular}
\caption{}
\label{tab:Distribution}
\end{table}
See \tabref{tab:Distribution}.
\begin{align}
p_{Y}\brak{k}= \begin{cases} 
      \frac{1}{3} & {k=0} \\
      \frac{2}{3 }& {k=1} 
   \end{cases}
   \\
p_{Y|X}\brak{0|0} = \frac{19}{25}\, 
p_{Y|X}\brak{0|1} = \frac{6}{25}\,
p_{Y|X}\brak{1|0} = \frac{45}{50}\,
p_{Y|X}\brak{1|2} = \frac{5}{50}
\end{align}
The desired probability is the probability that a slip drawn at random is marked other than Rs 1,
\begin{align}
&=1-p_X\brak{0}\\
&= p_X(1) + p_X(2)
\end{align}
Using Bayes theorem,
\begin{align}
&= p_Y\brak{0} \times \pr{Y=0 | X=1} + p_Y\brak{1} \times \pr{Y=1|X=2}\\
&=\frac{1}{3} \times \frac{6}{25} + \frac{2}{3} \times \frac{5}{50}\\
&=\frac{11}{75}
\end{align}

\newpage

%\tableofcontents

\bigskip

\renewcommand{\thefigure}{\theenumi}
\renewcommand{\thetable}{\theenumi}
%\renewcommand{\theequation}{\theenumi}

%\begin{abstract}
%%\boldmath
%In this letter, an algorithm for evaluating the exact analytical bit error rate  (BER)  for the piecewise linear (PL) combiner for  multiple relays is presented. Previous results were available only for upto three relays. The algorithm is unique in the sense that  the actual mathematical expressions, that are prohibitively large, need not be explicitly obtained. The diversity gain due to multiple relays is shown through plots of the analytical BER, well supported by simulations. 
%
%\end{abstract}
% IEEEtran.cls defaults to using nonbold math in the Abstract.
% This preserves the distinction between vectors and scalars. However,
% if the journal you are submitting to favors bold math in the abstract,
% then you can use LaTeX's standard command \boldmath at the very start
% of the abstract to achieve this. Many IEEE journals frown on math
% in the abstract anyway.

% Note that keywords are not normally used for peerreview papers.
%\begin{IEEEkeywords}
%Cooperative diversity, decode and forward, piecewise linear
%\end{IEEEkeywords}



% For peer review papers, you can put extra information on the cover
% page as needed:
% \ifCLASSOPTIONpeerreview
% \begin{center} \bfseries EDICS Category: 3-BBND \end{center}
% \fi
%
% For peerreview papers, this IEEEtran command inserts a page break and
% creates the second title. It will be ignored for other modes.
%\IEEEpeerreviewmaketitle




	\item Five cards—the ten, jack, queen, king and ace of diamonds, are well-shuffled with their face downwards. One card is then picked up at random.
\begin{enumerate}
\item
What is the probability that the card is the queen? 
\item
If the queen is drawn and put aside, what is the probability that the second card picked up is (a) an ace? (b) a queen?\\
\end{enumerate}
\solution
		%\begin{enumerate}[label=\thesection.\arabic*,ref=\thesection.\theenumi]
	\item One card is drawn from a well-shuffled deck of 52 cards. Find the probability of getting
\begin{enumerate}
\item A king of red colour 
\item A face card 
\item A red face card
\item The jack of hearts
\item A spade
\item The queen of diamonds

\end{enumerate}
\solution
		%\input{ncert/10/15/1/14/main.tex}
	\item Five cards—the ten, jack, queen, king and ace of diamonds, are well-shuffled with their face downwards. One card is then picked up at random.
\begin{enumerate}
\item
What is the probability that the card is the queen? 
\item
If the queen is drawn and put aside, what is the probability that the second card picked up is (a) an ace? (b) a queen?\\
\end{enumerate}
\solution
		%\input{ncert/10/15/1/15/defs.tex}
	\item A bag contains $5$ red balls and some blue balls. If the probability of drawing a blue ball is double that if a red ball, determine the number of blue balls in the bag. 
		\\
\solution
		%\input{ncert/10/15/2/3/defs.tex}
	\item A card is selected from a pack of 52 cards.
 \begin{enumerate}[label=(\alph*)] 
                 \item How many points are there in the sample space?
                 \item Calculate the probability that the card is an ace of spades.
                 \item Calculate the probability that the card is (i) an ace and (ii) black card.
 \end{enumerate}
\solution
		%\input{ncert/11/16/3/4/main.tex}
\item Four cards are drawn from a well-shuffled deck of 52 cards. What is the probability of obtaining 3 diamonds and one spade.
\\
\solution
		%\input{ncert/11/16/4/2/defs.tex}
\item In a certain lottery 10,000 tickets are sold and ten equal prizes are awarded. What is the probability of not getting a prize if you buy (a) one ticket (b) two tickets (c) 10 tickets ?	
\\
\solution
		%\input{ncert/11/16/4/4/defs.tex}
		%
\item 
Out of 100 students, two sections of 40 and 60 are formed. If you and your friend are among the 100 students, what is the probability that
\begin{enumerate}
\item you both enter the same section?
\item you both enter the different sections?
\end{enumerate}
\solution
		%\input{ncert/11/16/4/5/defs.tex}
	\item 
The number lock of a suitcase has 4 wheels each labelled with ten digits i.e. from 0 to 9.The lock opens with a sequence of four digits with no repeats.What is the probability of a person getting the right sequence to open the suitcase.
\\
\solution
		%\input{ncert/11/16/4/10/defs.tex}
		%
\item 
Two cards are drawn at random and without replacement from a pack of 52 playing cards. Find the probability that both the cards are black.
\\
\solution
		%\input{ncert/12/13/2/2/defs.tex}
		\item A box of oranges is inspected by examining three randomly selected oranges drawn without replacement. If all the three oranges are good, the box is approved for sale, otherwise, it is rejected. Find the probability that a box containing 15 oranges out of which 12 are good and 3 are bad ones will be approved for sale.
		\label{ncert/12/13/2/3/defs.tex}
		\item Two balls are drawn at random with replacement from a box containing 10 black and 8 red balls. Find the probability that
		\label{ncert/12/13/2/12}
\begin{enumerate}
\item both balls are red.
\item first ball is black and second is red.
\item one of them is black and other is red.
\end{enumerate}

\item In a hostel, 60\% of the students read Hindi newspaper, 40\% read English newspaper and 20\% read both Hindi and English newspapers. A student is selected at random.
		\label{ncert/12/13/2/15}
\begin{enumerate}
\item Find the probability that she reads neither Hindi nor English newspapers.
\item If she reads Hindi newspaper, find the probability that she reads English newspaper.
\item If she reads English newspaper, find the probability that she reads Hindi newspaper.\\
\end{enumerate}
\item The probability of obtaining an even prime number on each die, when a pair of dice is rolled is 
\begin{enumerate}
    \item $0$ 
    
    \item $\frac{1}{3}$ 
    
    \item $\frac{1}{12}$ 
    
    \item $\frac{1}{36}$ 
\end{enumerate}
\solution
		%\input{ncert/12/13/2/17/defs.tex}
	\item A bag contains 4 red and 4 black balls, another bag contains 2 red and 6 black balls. One of the two bags is selected at random and a ball is drawn from the bag which is found to be red. Find the probability that the ball is drawn from the first bag.
\\
\solution
		%\input{ncert/12/13/3/2/main.tex}
  \item
  Cards with numbers 2 to 101 are placed in a box. A card is selected at random.Find the probability that the card has
\begin{enumerate}[label=(\roman*)]
	\item an even number 
	\item a square number
\end{enumerate}
\solution
%\input{exemplar/10/13/3/32/main.tex}
\item
The king, queen and jack of clubs are removed from a deck of 52 playing cards and then well shuffled. Now one card is drawn at random from the remaining cards.  Determine the probability that the card is
\begin{enumerate}[label=(\roman*)]
\item a club
\item 10 of hearts
\end{enumerate}
\solution
%\input{exemplar/10/13/3/29/main.tex}
\item A team of medical students doing their internship have to assist during surgeries
at a city hospital. The probabilities of surgeries rated as very complex, complex,
routine, simple or very simple are respectively, 0.15, 0.20, 0.31, 0.26, .08. Find
the probabilities that a particular surgery will be rated
\begin{enumerate}
	\item complex or very complex;
	\item neither very complex nor very simple;
	\item routine or complex
	\item routine or simple
\end{enumerate}
\solution
%\input{exemplar/11/16/3/8(1)/main.tex}
\item A card is selected from a pack of 52 cards.
\begin{enumerate}[label=(\alph*)]
    \item How many points are there in the sample space?
    \item Calculate the probability that the card is an ace of spades.
    \item Calculate the probability that the card is (i) an ace and (ii) black card.
\end{enumerate}
\solution
%\input{exemplar/11/16/3/4/main2.tex}
\item The probability that a non leap year selected at random will contain 53 sundays.
\\
\solution
%\input{exemplar/10/13/1/19/main.tex}
\item One of the four persons John, Rita, Aslam or Gurpreet will be promoted next
month. Consequently the sample space consists of four elementary outcomes
S = {John promoted, Rita promoted, Aslam promoted, Gurpreet promoted}
You are told that the chances of John’s promotion is same as that of Gurpreet,
Rita’s chances of promotion are twice as likely as Johns. Aslam’s chances are
four times that of John.
\begin{enumerate}
	\item Determine
	\begin{enumerate}
		\item P (John promoted)
		\item P (Rita promoted)
		\item P (Aslam promoted)
		\item P (Gurpreet promoted)
	\end{enumerate}
	\item If A = {John promoted or Gurpreet promoted}, find P (A).
\end{enumerate}
\solution
%\input{exemplar/11/16/3/10/main.tex}
\item A card is drawn from a deck of 52 cards. Find the probability of getting a king or a heart or a red card.\\
\solution
%\input{exemplar/11/16/3/15/main.tex}
\item The probability that a student will pass his examination is 0.73, the probability of
the student getting a compartment is 0.13, and the probability that the student will
either pass or get compartment is 0.96. State True or False.\\
\solution
%\input{exemplar/11/16/3/31/main.tex}
\item A card is selected from a pack of 52 cards\\
\begin{enumerate}[label=(\alph*)]
\item How many points are there in the sample space?
\item Calculate the probability that the cards is an ace of spades.
\item Calculate the probability that the card is (i) an ace (ii)black card.\\
\end{enumerate}
%\input{ncert/11/16/3/4_1/Prob_4.tex}
\item In a non-leap year, the probability of having 53 tuesdays or 53 wednesdays is\\
\solution
%\input{exemplar/11/16/3/18/main.tex}
\item There are 1000 sealed envelopes in a box, 10 of them contain a cash prize of
Rs 100 each, 100 of them contain a cash prize of Rs 50 each and 200 of them
contain a cash prize of Rs 10 each and rest do not contain any cash prize. If they
are well shuffled and an envelope is picked up out, what is the probability that it
contains no cash prize?\\
\solution
%\input{exemplar/10/13/3/34/main.tex}
\item 
A die is thrown and a card is selected at random from a deck of 52 playing cards. The probability of getting an even number on the die and a spade card.\\
\solution
%\input{exemplar/12/13/3/78/main.tex}
\item
If 4-digit numbers greater than 5,000 are randomly formed from the digits 0, 1, 3, 5, and 7, what is the probability of forming a number divisible by 5 when:
\begin{enumerate}
    \item The digits are repeated?
    \item The repetition of digits is not allowed?
\end{enumerate}
\solution
%\input{ncert/11/16/4/9/main.tex}
\item Consider the probability space $\brak{\Omega, \mathcal{G}, P}$ where $\Omega = [0,2]$ and $\mathcal{G} = \cbrak{\phi, \Omega, [0,1], (1,2]}$. Let $X$ and $Y$ be two functions on $\Omega$ defined as
\begin{align*}
    X(\omega) = 
    \begin{cases}
        1 & \text{if }\omega \in [0, 1]\\
        2 & \text{if }\omega \in (1, 2]
    \end{cases}
\end{align*}
and
\begin{align*}
    Y(\omega) = 
    \begin{cases}
        2 & \text{if }\omega \in [0, 1.5]\\
        3 & \text{if }\omega \in (1.5, 2].
    \end{cases}
\end{align*}
Then which one of the following statements is true?
\begin{enumerate}
    \item [(A)] $X$ is a random variable with respect to $\mathcal{G}$, but $Y$ is not a random variable with respect to $\mathcal{G}$.
    \item [(B)] $Y$ is a random variable with respect to $\mathcal{G}$, but $X$ is not a random variable with respect to $\mathcal{G}$.
    \item [(C)] Neither $X$ nor $Y$ is a random variable with respect to $\mathcal{G}$.
    \item [(D)] Both $X$ and $Y$ are random variables with respect to $\mathcal{G}$.
\end{enumerate} \hfill (GATE ST 2023)\\
\solution
%\input{gate/ST/2023/14/main.tex}
	\item  A die is loaded in such a way that each odd number is twice as likely to occur as
each even number. Find $P(G)$, where $G$ is the event that a number greater than
3 occurs on a single roll of the die.
\\
\solution
		%\input{exemplar/11/16/3/5/main.tex}
	\item All the jacks, queens and kings are removed from a deck of 52 playing cards. The remaining cards are well shuffled and then one card is drawn at random. Giving ace a value 1 similar value for other cards, find the probability that the card has a value 
		\begin{enumerate}
			\item 7
			\item greater than 7
			\item less than 7
		\end{enumerate}
		%\input{exemplar/10/13/3/30/main.tex}
  \item A Lot consists of 48 mobile phones of which 42 are good, 3 have only minor defects and 3 have major defects.Varnika will buy a phone if it is good but the trader will only buy a mobile if it has no major defects. One phone is selected at random from the lot. What is the probability that it is
\begin{enumerate}
	\item acceptable to Varnika?
            \item acceptable to the trader?
\end{enumerate}
\solution
	%\input{exemplar/10/13/3/40/main.tex}
 \item A student says that if you throw a die, it will show up 1 or not 1. Therefore, the probability of getting 1 and the probability of getting 'not 1' each is equal to $\frac{1}{2}$. Is this correct? Give reasons.\\
 \solution
        %\input{exemplar/10/13/2/9/main.tex}
   \item Four candidates A, B, C, D have ap-
plied for the assignment to coach a school cricket
team. If A is twice as likely to be selected as B, and
B and C are given about the same chance of being
selected, while C is twice as likely to be selected
as D, what are the probabilities that
\begin{enumerate}
\item C will be selected?
\item A will not be selected?
\end{enumerate}
	%\input{exemplar/11/16/3/9/main.tex}
 \item A bag contain 24 balls of which $x$ balls are red, $2x$ are white and $3x$ are blue. A ball is selected at random, What is the probability that it is
\begin{enumerate}[label=\alph*)]
\item not red ?
\item white ?
\end{enumerate}
%\input{exemplar/10/13/3/41/main.tex}
If the letters of the word ASSASSINATION are arranged at random. Find the Probability that
\begin{enumerate}[label=(\alph*)]
\item Four $S's$ come consecutively in the word
\item Two  $I's$ and two $N's$ come together
\item All $A's$ are not coming together
\item No two $A's$ are coming together
\end{enumerate}
%\input{exemplar/11/16/3/14/main.tex}
	\item One urn contains two black balls (labelled B1 and B2) and one white ball. A
	second urn contains one black ball and two white balls (labelled W1 and W2).
	Suppose the following experiment is performed. One of the two urns is chosen
	at random. Next a ball is randomly chosen from the urn. Then a second ball is
	chosen at random from the same urn without replacing the first ball.
	
	\begin{enumerate}
	\item What is the probability that two black balls are chosen?
	
	\item What is the probability that two balls of opposite colour are chosen?
	\end{enumerate}
	\solution
	%\input{exemplar/11/16/3/12/main1.tex}
\end{enumerate}

	\item A bag contains $5$ red balls and some blue balls. If the probability of drawing a blue ball is double that if a red ball, determine the number of blue balls in the bag. 
		\\
\solution
		%\begin{enumerate}[label=\thesection.\arabic*,ref=\thesection.\theenumi]
	\item One card is drawn from a well-shuffled deck of 52 cards. Find the probability of getting
\begin{enumerate}
\item A king of red colour 
\item A face card 
\item A red face card
\item The jack of hearts
\item A spade
\item The queen of diamonds

\end{enumerate}
\solution
		%\input{ncert/10/15/1/14/main.tex}
	\item Five cards—the ten, jack, queen, king and ace of diamonds, are well-shuffled with their face downwards. One card is then picked up at random.
\begin{enumerate}
\item
What is the probability that the card is the queen? 
\item
If the queen is drawn and put aside, what is the probability that the second card picked up is (a) an ace? (b) a queen?\\
\end{enumerate}
\solution
		%\input{ncert/10/15/1/15/defs.tex}
	\item A bag contains $5$ red balls and some blue balls. If the probability of drawing a blue ball is double that if a red ball, determine the number of blue balls in the bag. 
		\\
\solution
		%\input{ncert/10/15/2/3/defs.tex}
	\item A card is selected from a pack of 52 cards.
 \begin{enumerate}[label=(\alph*)] 
                 \item How many points are there in the sample space?
                 \item Calculate the probability that the card is an ace of spades.
                 \item Calculate the probability that the card is (i) an ace and (ii) black card.
 \end{enumerate}
\solution
		%\input{ncert/11/16/3/4/main.tex}
\item Four cards are drawn from a well-shuffled deck of 52 cards. What is the probability of obtaining 3 diamonds and one spade.
\\
\solution
		%\input{ncert/11/16/4/2/defs.tex}
\item In a certain lottery 10,000 tickets are sold and ten equal prizes are awarded. What is the probability of not getting a prize if you buy (a) one ticket (b) two tickets (c) 10 tickets ?	
\\
\solution
		%\input{ncert/11/16/4/4/defs.tex}
		%
\item 
Out of 100 students, two sections of 40 and 60 are formed. If you and your friend are among the 100 students, what is the probability that
\begin{enumerate}
\item you both enter the same section?
\item you both enter the different sections?
\end{enumerate}
\solution
		%\input{ncert/11/16/4/5/defs.tex}
	\item 
The number lock of a suitcase has 4 wheels each labelled with ten digits i.e. from 0 to 9.The lock opens with a sequence of four digits with no repeats.What is the probability of a person getting the right sequence to open the suitcase.
\\
\solution
		%\input{ncert/11/16/4/10/defs.tex}
		%
\item 
Two cards are drawn at random and without replacement from a pack of 52 playing cards. Find the probability that both the cards are black.
\\
\solution
		%\input{ncert/12/13/2/2/defs.tex}
		\item A box of oranges is inspected by examining three randomly selected oranges drawn without replacement. If all the three oranges are good, the box is approved for sale, otherwise, it is rejected. Find the probability that a box containing 15 oranges out of which 12 are good and 3 are bad ones will be approved for sale.
		\label{ncert/12/13/2/3/defs.tex}
		\item Two balls are drawn at random with replacement from a box containing 10 black and 8 red balls. Find the probability that
		\label{ncert/12/13/2/12}
\begin{enumerate}
\item both balls are red.
\item first ball is black and second is red.
\item one of them is black and other is red.
\end{enumerate}

\item In a hostel, 60\% of the students read Hindi newspaper, 40\% read English newspaper and 20\% read both Hindi and English newspapers. A student is selected at random.
		\label{ncert/12/13/2/15}
\begin{enumerate}
\item Find the probability that she reads neither Hindi nor English newspapers.
\item If she reads Hindi newspaper, find the probability that she reads English newspaper.
\item If she reads English newspaper, find the probability that she reads Hindi newspaper.\\
\end{enumerate}
\item The probability of obtaining an even prime number on each die, when a pair of dice is rolled is 
\begin{enumerate}
    \item $0$ 
    
    \item $\frac{1}{3}$ 
    
    \item $\frac{1}{12}$ 
    
    \item $\frac{1}{36}$ 
\end{enumerate}
\solution
		%\input{ncert/12/13/2/17/defs.tex}
	\item A bag contains 4 red and 4 black balls, another bag contains 2 red and 6 black balls. One of the two bags is selected at random and a ball is drawn from the bag which is found to be red. Find the probability that the ball is drawn from the first bag.
\\
\solution
		%\input{ncert/12/13/3/2/main.tex}
  \item
  Cards with numbers 2 to 101 are placed in a box. A card is selected at random.Find the probability that the card has
\begin{enumerate}[label=(\roman*)]
	\item an even number 
	\item a square number
\end{enumerate}
\solution
%\input{exemplar/10/13/3/32/main.tex}
\item
The king, queen and jack of clubs are removed from a deck of 52 playing cards and then well shuffled. Now one card is drawn at random from the remaining cards.  Determine the probability that the card is
\begin{enumerate}[label=(\roman*)]
\item a club
\item 10 of hearts
\end{enumerate}
\solution
%\input{exemplar/10/13/3/29/main.tex}
\item A team of medical students doing their internship have to assist during surgeries
at a city hospital. The probabilities of surgeries rated as very complex, complex,
routine, simple or very simple are respectively, 0.15, 0.20, 0.31, 0.26, .08. Find
the probabilities that a particular surgery will be rated
\begin{enumerate}
	\item complex or very complex;
	\item neither very complex nor very simple;
	\item routine or complex
	\item routine or simple
\end{enumerate}
\solution
%\input{exemplar/11/16/3/8(1)/main.tex}
\item A card is selected from a pack of 52 cards.
\begin{enumerate}[label=(\alph*)]
    \item How many points are there in the sample space?
    \item Calculate the probability that the card is an ace of spades.
    \item Calculate the probability that the card is (i) an ace and (ii) black card.
\end{enumerate}
\solution
%\input{exemplar/11/16/3/4/main2.tex}
\item The probability that a non leap year selected at random will contain 53 sundays.
\\
\solution
%\input{exemplar/10/13/1/19/main.tex}
\item One of the four persons John, Rita, Aslam or Gurpreet will be promoted next
month. Consequently the sample space consists of four elementary outcomes
S = {John promoted, Rita promoted, Aslam promoted, Gurpreet promoted}
You are told that the chances of John’s promotion is same as that of Gurpreet,
Rita’s chances of promotion are twice as likely as Johns. Aslam’s chances are
four times that of John.
\begin{enumerate}
	\item Determine
	\begin{enumerate}
		\item P (John promoted)
		\item P (Rita promoted)
		\item P (Aslam promoted)
		\item P (Gurpreet promoted)
	\end{enumerate}
	\item If A = {John promoted or Gurpreet promoted}, find P (A).
\end{enumerate}
\solution
%\input{exemplar/11/16/3/10/main.tex}
\item A card is drawn from a deck of 52 cards. Find the probability of getting a king or a heart or a red card.\\
\solution
%\input{exemplar/11/16/3/15/main.tex}
\item The probability that a student will pass his examination is 0.73, the probability of
the student getting a compartment is 0.13, and the probability that the student will
either pass or get compartment is 0.96. State True or False.\\
\solution
%\input{exemplar/11/16/3/31/main.tex}
\item A card is selected from a pack of 52 cards\\
\begin{enumerate}[label=(\alph*)]
\item How many points are there in the sample space?
\item Calculate the probability that the cards is an ace of spades.
\item Calculate the probability that the card is (i) an ace (ii)black card.\\
\end{enumerate}
%\input{ncert/11/16/3/4_1/Prob_4.tex}
\item In a non-leap year, the probability of having 53 tuesdays or 53 wednesdays is\\
\solution
%\input{exemplar/11/16/3/18/main.tex}
\item There are 1000 sealed envelopes in a box, 10 of them contain a cash prize of
Rs 100 each, 100 of them contain a cash prize of Rs 50 each and 200 of them
contain a cash prize of Rs 10 each and rest do not contain any cash prize. If they
are well shuffled and an envelope is picked up out, what is the probability that it
contains no cash prize?\\
\solution
%\input{exemplar/10/13/3/34/main.tex}
\item 
A die is thrown and a card is selected at random from a deck of 52 playing cards. The probability of getting an even number on the die and a spade card.\\
\solution
%\input{exemplar/12/13/3/78/main.tex}
\item
If 4-digit numbers greater than 5,000 are randomly formed from the digits 0, 1, 3, 5, and 7, what is the probability of forming a number divisible by 5 when:
\begin{enumerate}
    \item The digits are repeated?
    \item The repetition of digits is not allowed?
\end{enumerate}
\solution
%\input{ncert/11/16/4/9/main.tex}
\item Consider the probability space $\brak{\Omega, \mathcal{G}, P}$ where $\Omega = [0,2]$ and $\mathcal{G} = \cbrak{\phi, \Omega, [0,1], (1,2]}$. Let $X$ and $Y$ be two functions on $\Omega$ defined as
\begin{align*}
    X(\omega) = 
    \begin{cases}
        1 & \text{if }\omega \in [0, 1]\\
        2 & \text{if }\omega \in (1, 2]
    \end{cases}
\end{align*}
and
\begin{align*}
    Y(\omega) = 
    \begin{cases}
        2 & \text{if }\omega \in [0, 1.5]\\
        3 & \text{if }\omega \in (1.5, 2].
    \end{cases}
\end{align*}
Then which one of the following statements is true?
\begin{enumerate}
    \item [(A)] $X$ is a random variable with respect to $\mathcal{G}$, but $Y$ is not a random variable with respect to $\mathcal{G}$.
    \item [(B)] $Y$ is a random variable with respect to $\mathcal{G}$, but $X$ is not a random variable with respect to $\mathcal{G}$.
    \item [(C)] Neither $X$ nor $Y$ is a random variable with respect to $\mathcal{G}$.
    \item [(D)] Both $X$ and $Y$ are random variables with respect to $\mathcal{G}$.
\end{enumerate} \hfill (GATE ST 2023)\\
\solution
%\input{gate/ST/2023/14/main.tex}
	\item  A die is loaded in such a way that each odd number is twice as likely to occur as
each even number. Find $P(G)$, where $G$ is the event that a number greater than
3 occurs on a single roll of the die.
\\
\solution
		%\input{exemplar/11/16/3/5/main.tex}
	\item All the jacks, queens and kings are removed from a deck of 52 playing cards. The remaining cards are well shuffled and then one card is drawn at random. Giving ace a value 1 similar value for other cards, find the probability that the card has a value 
		\begin{enumerate}
			\item 7
			\item greater than 7
			\item less than 7
		\end{enumerate}
		%\input{exemplar/10/13/3/30/main.tex}
  \item A Lot consists of 48 mobile phones of which 42 are good, 3 have only minor defects and 3 have major defects.Varnika will buy a phone if it is good but the trader will only buy a mobile if it has no major defects. One phone is selected at random from the lot. What is the probability that it is
\begin{enumerate}
	\item acceptable to Varnika?
            \item acceptable to the trader?
\end{enumerate}
\solution
	%\input{exemplar/10/13/3/40/main.tex}
 \item A student says that if you throw a die, it will show up 1 or not 1. Therefore, the probability of getting 1 and the probability of getting 'not 1' each is equal to $\frac{1}{2}$. Is this correct? Give reasons.\\
 \solution
        %\input{exemplar/10/13/2/9/main.tex}
   \item Four candidates A, B, C, D have ap-
plied for the assignment to coach a school cricket
team. If A is twice as likely to be selected as B, and
B and C are given about the same chance of being
selected, while C is twice as likely to be selected
as D, what are the probabilities that
\begin{enumerate}
\item C will be selected?
\item A will not be selected?
\end{enumerate}
	%\input{exemplar/11/16/3/9/main.tex}
 \item A bag contain 24 balls of which $x$ balls are red, $2x$ are white and $3x$ are blue. A ball is selected at random, What is the probability that it is
\begin{enumerate}[label=\alph*)]
\item not red ?
\item white ?
\end{enumerate}
%\input{exemplar/10/13/3/41/main.tex}
If the letters of the word ASSASSINATION are arranged at random. Find the Probability that
\begin{enumerate}[label=(\alph*)]
\item Four $S's$ come consecutively in the word
\item Two  $I's$ and two $N's$ come together
\item All $A's$ are not coming together
\item No two $A's$ are coming together
\end{enumerate}
%\input{exemplar/11/16/3/14/main.tex}
	\item One urn contains two black balls (labelled B1 and B2) and one white ball. A
	second urn contains one black ball and two white balls (labelled W1 and W2).
	Suppose the following experiment is performed. One of the two urns is chosen
	at random. Next a ball is randomly chosen from the urn. Then a second ball is
	chosen at random from the same urn without replacing the first ball.
	
	\begin{enumerate}
	\item What is the probability that two black balls are chosen?
	
	\item What is the probability that two balls of opposite colour are chosen?
	\end{enumerate}
	\solution
	%\input{exemplar/11/16/3/12/main1.tex}
\end{enumerate}

	\item A card is selected from a pack of 52 cards.
 \begin{enumerate}[label=(\alph*)] 
                 \item How many points are there in the sample space?
                 \item Calculate the probability that the card is an ace of spades.
                 \item Calculate the probability that the card is (i) an ace and (ii) black card.
 \end{enumerate}
\solution
		%\begin{table}[H]
	\centering
\begin{tabular}{|c|c|c|}
\hline
Random variable &Value &Definition\\ \hline
\multirow{3}{*}{X} &0 &Slips of Rs 1\\
&1 &Slips of Rs 5\\
&2 &Slips of Rs 13\\ \hline
\multirow{2}{*}{Y} &0 &Box A\\
&1 &Box B\\\hline
\end{tabular}
\caption{}
\label{tab:Distribution}
\end{table}
See \tabref{tab:Distribution}.
\begin{align}
p_{Y}\brak{k}= \begin{cases} 
      \frac{1}{3} & {k=0} \\
      \frac{2}{3 }& {k=1} 
   \end{cases}
   \\
p_{Y|X}\brak{0|0} = \frac{19}{25}\, 
p_{Y|X}\brak{0|1} = \frac{6}{25}\,
p_{Y|X}\brak{1|0} = \frac{45}{50}\,
p_{Y|X}\brak{1|2} = \frac{5}{50}
\end{align}
The desired probability is the probability that a slip drawn at random is marked other than Rs 1,
\begin{align}
&=1-p_X\brak{0}\\
&= p_X(1) + p_X(2)
\end{align}
Using Bayes theorem,
\begin{align}
&= p_Y\brak{0} \times \pr{Y=0 | X=1} + p_Y\brak{1} \times \pr{Y=1|X=2}\\
&=\frac{1}{3} \times \frac{6}{25} + \frac{2}{3} \times \frac{5}{50}\\
&=\frac{11}{75}
\end{align}

\newpage

%\tableofcontents

\bigskip

\renewcommand{\thefigure}{\theenumi}
\renewcommand{\thetable}{\theenumi}
%\renewcommand{\theequation}{\theenumi}

%\begin{abstract}
%%\boldmath
%In this letter, an algorithm for evaluating the exact analytical bit error rate  (BER)  for the piecewise linear (PL) combiner for  multiple relays is presented. Previous results were available only for upto three relays. The algorithm is unique in the sense that  the actual mathematical expressions, that are prohibitively large, need not be explicitly obtained. The diversity gain due to multiple relays is shown through plots of the analytical BER, well supported by simulations. 
%
%\end{abstract}
% IEEEtran.cls defaults to using nonbold math in the Abstract.
% This preserves the distinction between vectors and scalars. However,
% if the journal you are submitting to favors bold math in the abstract,
% then you can use LaTeX's standard command \boldmath at the very start
% of the abstract to achieve this. Many IEEE journals frown on math
% in the abstract anyway.

% Note that keywords are not normally used for peerreview papers.
%\begin{IEEEkeywords}
%Cooperative diversity, decode and forward, piecewise linear
%\end{IEEEkeywords}



% For peer review papers, you can put extra information on the cover
% page as needed:
% \ifCLASSOPTIONpeerreview
% \begin{center} \bfseries EDICS Category: 3-BBND \end{center}
% \fi
%
% For peerreview papers, this IEEEtran command inserts a page break and
% creates the second title. It will be ignored for other modes.
%\IEEEpeerreviewmaketitle




\item Four cards are drawn from a well-shuffled deck of 52 cards. What is the probability of obtaining 3 diamonds and one spade.
\\
\solution
		%\begin{enumerate}[label=\thesection.\arabic*,ref=\thesection.\theenumi]
	\item One card is drawn from a well-shuffled deck of 52 cards. Find the probability of getting
\begin{enumerate}
\item A king of red colour 
\item A face card 
\item A red face card
\item The jack of hearts
\item A spade
\item The queen of diamonds

\end{enumerate}
\solution
		%\input{ncert/10/15/1/14/main.tex}
	\item Five cards—the ten, jack, queen, king and ace of diamonds, are well-shuffled with their face downwards. One card is then picked up at random.
\begin{enumerate}
\item
What is the probability that the card is the queen? 
\item
If the queen is drawn and put aside, what is the probability that the second card picked up is (a) an ace? (b) a queen?\\
\end{enumerate}
\solution
		%\input{ncert/10/15/1/15/defs.tex}
	\item A bag contains $5$ red balls and some blue balls. If the probability of drawing a blue ball is double that if a red ball, determine the number of blue balls in the bag. 
		\\
\solution
		%\input{ncert/10/15/2/3/defs.tex}
	\item A card is selected from a pack of 52 cards.
 \begin{enumerate}[label=(\alph*)] 
                 \item How many points are there in the sample space?
                 \item Calculate the probability that the card is an ace of spades.
                 \item Calculate the probability that the card is (i) an ace and (ii) black card.
 \end{enumerate}
\solution
		%\input{ncert/11/16/3/4/main.tex}
\item Four cards are drawn from a well-shuffled deck of 52 cards. What is the probability of obtaining 3 diamonds and one spade.
\\
\solution
		%\input{ncert/11/16/4/2/defs.tex}
\item In a certain lottery 10,000 tickets are sold and ten equal prizes are awarded. What is the probability of not getting a prize if you buy (a) one ticket (b) two tickets (c) 10 tickets ?	
\\
\solution
		%\input{ncert/11/16/4/4/defs.tex}
		%
\item 
Out of 100 students, two sections of 40 and 60 are formed. If you and your friend are among the 100 students, what is the probability that
\begin{enumerate}
\item you both enter the same section?
\item you both enter the different sections?
\end{enumerate}
\solution
		%\input{ncert/11/16/4/5/defs.tex}
	\item 
The number lock of a suitcase has 4 wheels each labelled with ten digits i.e. from 0 to 9.The lock opens with a sequence of four digits with no repeats.What is the probability of a person getting the right sequence to open the suitcase.
\\
\solution
		%\input{ncert/11/16/4/10/defs.tex}
		%
\item 
Two cards are drawn at random and without replacement from a pack of 52 playing cards. Find the probability that both the cards are black.
\\
\solution
		%\input{ncert/12/13/2/2/defs.tex}
		\item A box of oranges is inspected by examining three randomly selected oranges drawn without replacement. If all the three oranges are good, the box is approved for sale, otherwise, it is rejected. Find the probability that a box containing 15 oranges out of which 12 are good and 3 are bad ones will be approved for sale.
		\label{ncert/12/13/2/3/defs.tex}
		\item Two balls are drawn at random with replacement from a box containing 10 black and 8 red balls. Find the probability that
		\label{ncert/12/13/2/12}
\begin{enumerate}
\item both balls are red.
\item first ball is black and second is red.
\item one of them is black and other is red.
\end{enumerate}

\item In a hostel, 60\% of the students read Hindi newspaper, 40\% read English newspaper and 20\% read both Hindi and English newspapers. A student is selected at random.
		\label{ncert/12/13/2/15}
\begin{enumerate}
\item Find the probability that she reads neither Hindi nor English newspapers.
\item If she reads Hindi newspaper, find the probability that she reads English newspaper.
\item If she reads English newspaper, find the probability that she reads Hindi newspaper.\\
\end{enumerate}
\item The probability of obtaining an even prime number on each die, when a pair of dice is rolled is 
\begin{enumerate}
    \item $0$ 
    
    \item $\frac{1}{3}$ 
    
    \item $\frac{1}{12}$ 
    
    \item $\frac{1}{36}$ 
\end{enumerate}
\solution
		%\input{ncert/12/13/2/17/defs.tex}
	\item A bag contains 4 red and 4 black balls, another bag contains 2 red and 6 black balls. One of the two bags is selected at random and a ball is drawn from the bag which is found to be red. Find the probability that the ball is drawn from the first bag.
\\
\solution
		%\input{ncert/12/13/3/2/main.tex}
  \item
  Cards with numbers 2 to 101 are placed in a box. A card is selected at random.Find the probability that the card has
\begin{enumerate}[label=(\roman*)]
	\item an even number 
	\item a square number
\end{enumerate}
\solution
%\input{exemplar/10/13/3/32/main.tex}
\item
The king, queen and jack of clubs are removed from a deck of 52 playing cards and then well shuffled. Now one card is drawn at random from the remaining cards.  Determine the probability that the card is
\begin{enumerate}[label=(\roman*)]
\item a club
\item 10 of hearts
\end{enumerate}
\solution
%\input{exemplar/10/13/3/29/main.tex}
\item A team of medical students doing their internship have to assist during surgeries
at a city hospital. The probabilities of surgeries rated as very complex, complex,
routine, simple or very simple are respectively, 0.15, 0.20, 0.31, 0.26, .08. Find
the probabilities that a particular surgery will be rated
\begin{enumerate}
	\item complex or very complex;
	\item neither very complex nor very simple;
	\item routine or complex
	\item routine or simple
\end{enumerate}
\solution
%\input{exemplar/11/16/3/8(1)/main.tex}
\item A card is selected from a pack of 52 cards.
\begin{enumerate}[label=(\alph*)]
    \item How many points are there in the sample space?
    \item Calculate the probability that the card is an ace of spades.
    \item Calculate the probability that the card is (i) an ace and (ii) black card.
\end{enumerate}
\solution
%\input{exemplar/11/16/3/4/main2.tex}
\item The probability that a non leap year selected at random will contain 53 sundays.
\\
\solution
%\input{exemplar/10/13/1/19/main.tex}
\item One of the four persons John, Rita, Aslam or Gurpreet will be promoted next
month. Consequently the sample space consists of four elementary outcomes
S = {John promoted, Rita promoted, Aslam promoted, Gurpreet promoted}
You are told that the chances of John’s promotion is same as that of Gurpreet,
Rita’s chances of promotion are twice as likely as Johns. Aslam’s chances are
four times that of John.
\begin{enumerate}
	\item Determine
	\begin{enumerate}
		\item P (John promoted)
		\item P (Rita promoted)
		\item P (Aslam promoted)
		\item P (Gurpreet promoted)
	\end{enumerate}
	\item If A = {John promoted or Gurpreet promoted}, find P (A).
\end{enumerate}
\solution
%\input{exemplar/11/16/3/10/main.tex}
\item A card is drawn from a deck of 52 cards. Find the probability of getting a king or a heart or a red card.\\
\solution
%\input{exemplar/11/16/3/15/main.tex}
\item The probability that a student will pass his examination is 0.73, the probability of
the student getting a compartment is 0.13, and the probability that the student will
either pass or get compartment is 0.96. State True or False.\\
\solution
%\input{exemplar/11/16/3/31/main.tex}
\item A card is selected from a pack of 52 cards\\
\begin{enumerate}[label=(\alph*)]
\item How many points are there in the sample space?
\item Calculate the probability that the cards is an ace of spades.
\item Calculate the probability that the card is (i) an ace (ii)black card.\\
\end{enumerate}
%\input{ncert/11/16/3/4_1/Prob_4.tex}
\item In a non-leap year, the probability of having 53 tuesdays or 53 wednesdays is\\
\solution
%\input{exemplar/11/16/3/18/main.tex}
\item There are 1000 sealed envelopes in a box, 10 of them contain a cash prize of
Rs 100 each, 100 of them contain a cash prize of Rs 50 each and 200 of them
contain a cash prize of Rs 10 each and rest do not contain any cash prize. If they
are well shuffled and an envelope is picked up out, what is the probability that it
contains no cash prize?\\
\solution
%\input{exemplar/10/13/3/34/main.tex}
\item 
A die is thrown and a card is selected at random from a deck of 52 playing cards. The probability of getting an even number on the die and a spade card.\\
\solution
%\input{exemplar/12/13/3/78/main.tex}
\item
If 4-digit numbers greater than 5,000 are randomly formed from the digits 0, 1, 3, 5, and 7, what is the probability of forming a number divisible by 5 when:
\begin{enumerate}
    \item The digits are repeated?
    \item The repetition of digits is not allowed?
\end{enumerate}
\solution
%\input{ncert/11/16/4/9/main.tex}
\item Consider the probability space $\brak{\Omega, \mathcal{G}, P}$ where $\Omega = [0,2]$ and $\mathcal{G} = \cbrak{\phi, \Omega, [0,1], (1,2]}$. Let $X$ and $Y$ be two functions on $\Omega$ defined as
\begin{align*}
    X(\omega) = 
    \begin{cases}
        1 & \text{if }\omega \in [0, 1]\\
        2 & \text{if }\omega \in (1, 2]
    \end{cases}
\end{align*}
and
\begin{align*}
    Y(\omega) = 
    \begin{cases}
        2 & \text{if }\omega \in [0, 1.5]\\
        3 & \text{if }\omega \in (1.5, 2].
    \end{cases}
\end{align*}
Then which one of the following statements is true?
\begin{enumerate}
    \item [(A)] $X$ is a random variable with respect to $\mathcal{G}$, but $Y$ is not a random variable with respect to $\mathcal{G}$.
    \item [(B)] $Y$ is a random variable with respect to $\mathcal{G}$, but $X$ is not a random variable with respect to $\mathcal{G}$.
    \item [(C)] Neither $X$ nor $Y$ is a random variable with respect to $\mathcal{G}$.
    \item [(D)] Both $X$ and $Y$ are random variables with respect to $\mathcal{G}$.
\end{enumerate} \hfill (GATE ST 2023)\\
\solution
%\input{gate/ST/2023/14/main.tex}
	\item  A die is loaded in such a way that each odd number is twice as likely to occur as
each even number. Find $P(G)$, where $G$ is the event that a number greater than
3 occurs on a single roll of the die.
\\
\solution
		%\input{exemplar/11/16/3/5/main.tex}
	\item All the jacks, queens and kings are removed from a deck of 52 playing cards. The remaining cards are well shuffled and then one card is drawn at random. Giving ace a value 1 similar value for other cards, find the probability that the card has a value 
		\begin{enumerate}
			\item 7
			\item greater than 7
			\item less than 7
		\end{enumerate}
		%\input{exemplar/10/13/3/30/main.tex}
  \item A Lot consists of 48 mobile phones of which 42 are good, 3 have only minor defects and 3 have major defects.Varnika will buy a phone if it is good but the trader will only buy a mobile if it has no major defects. One phone is selected at random from the lot. What is the probability that it is
\begin{enumerate}
	\item acceptable to Varnika?
            \item acceptable to the trader?
\end{enumerate}
\solution
	%\input{exemplar/10/13/3/40/main.tex}
 \item A student says that if you throw a die, it will show up 1 or not 1. Therefore, the probability of getting 1 and the probability of getting 'not 1' each is equal to $\frac{1}{2}$. Is this correct? Give reasons.\\
 \solution
        %\input{exemplar/10/13/2/9/main.tex}
   \item Four candidates A, B, C, D have ap-
plied for the assignment to coach a school cricket
team. If A is twice as likely to be selected as B, and
B and C are given about the same chance of being
selected, while C is twice as likely to be selected
as D, what are the probabilities that
\begin{enumerate}
\item C will be selected?
\item A will not be selected?
\end{enumerate}
	%\input{exemplar/11/16/3/9/main.tex}
 \item A bag contain 24 balls of which $x$ balls are red, $2x$ are white and $3x$ are blue. A ball is selected at random, What is the probability that it is
\begin{enumerate}[label=\alph*)]
\item not red ?
\item white ?
\end{enumerate}
%\input{exemplar/10/13/3/41/main.tex}
If the letters of the word ASSASSINATION are arranged at random. Find the Probability that
\begin{enumerate}[label=(\alph*)]
\item Four $S's$ come consecutively in the word
\item Two  $I's$ and two $N's$ come together
\item All $A's$ are not coming together
\item No two $A's$ are coming together
\end{enumerate}
%\input{exemplar/11/16/3/14/main.tex}
	\item One urn contains two black balls (labelled B1 and B2) and one white ball. A
	second urn contains one black ball and two white balls (labelled W1 and W2).
	Suppose the following experiment is performed. One of the two urns is chosen
	at random. Next a ball is randomly chosen from the urn. Then a second ball is
	chosen at random from the same urn without replacing the first ball.
	
	\begin{enumerate}
	\item What is the probability that two black balls are chosen?
	
	\item What is the probability that two balls of opposite colour are chosen?
	\end{enumerate}
	\solution
	%\input{exemplar/11/16/3/12/main1.tex}
\end{enumerate}

\item In a certain lottery 10,000 tickets are sold and ten equal prizes are awarded. What is the probability of not getting a prize if you buy (a) one ticket (b) two tickets (c) 10 tickets ?	
\\
\solution
		%\begin{enumerate}[label=\thesection.\arabic*,ref=\thesection.\theenumi]
	\item One card is drawn from a well-shuffled deck of 52 cards. Find the probability of getting
\begin{enumerate}
\item A king of red colour 
\item A face card 
\item A red face card
\item The jack of hearts
\item A spade
\item The queen of diamonds

\end{enumerate}
\solution
		%\input{ncert/10/15/1/14/main.tex}
	\item Five cards—the ten, jack, queen, king and ace of diamonds, are well-shuffled with their face downwards. One card is then picked up at random.
\begin{enumerate}
\item
What is the probability that the card is the queen? 
\item
If the queen is drawn and put aside, what is the probability that the second card picked up is (a) an ace? (b) a queen?\\
\end{enumerate}
\solution
		%\input{ncert/10/15/1/15/defs.tex}
	\item A bag contains $5$ red balls and some blue balls. If the probability of drawing a blue ball is double that if a red ball, determine the number of blue balls in the bag. 
		\\
\solution
		%\input{ncert/10/15/2/3/defs.tex}
	\item A card is selected from a pack of 52 cards.
 \begin{enumerate}[label=(\alph*)] 
                 \item How many points are there in the sample space?
                 \item Calculate the probability that the card is an ace of spades.
                 \item Calculate the probability that the card is (i) an ace and (ii) black card.
 \end{enumerate}
\solution
		%\input{ncert/11/16/3/4/main.tex}
\item Four cards are drawn from a well-shuffled deck of 52 cards. What is the probability of obtaining 3 diamonds and one spade.
\\
\solution
		%\input{ncert/11/16/4/2/defs.tex}
\item In a certain lottery 10,000 tickets are sold and ten equal prizes are awarded. What is the probability of not getting a prize if you buy (a) one ticket (b) two tickets (c) 10 tickets ?	
\\
\solution
		%\input{ncert/11/16/4/4/defs.tex}
		%
\item 
Out of 100 students, two sections of 40 and 60 are formed. If you and your friend are among the 100 students, what is the probability that
\begin{enumerate}
\item you both enter the same section?
\item you both enter the different sections?
\end{enumerate}
\solution
		%\input{ncert/11/16/4/5/defs.tex}
	\item 
The number lock of a suitcase has 4 wheels each labelled with ten digits i.e. from 0 to 9.The lock opens with a sequence of four digits with no repeats.What is the probability of a person getting the right sequence to open the suitcase.
\\
\solution
		%\input{ncert/11/16/4/10/defs.tex}
		%
\item 
Two cards are drawn at random and without replacement from a pack of 52 playing cards. Find the probability that both the cards are black.
\\
\solution
		%\input{ncert/12/13/2/2/defs.tex}
		\item A box of oranges is inspected by examining three randomly selected oranges drawn without replacement. If all the three oranges are good, the box is approved for sale, otherwise, it is rejected. Find the probability that a box containing 15 oranges out of which 12 are good and 3 are bad ones will be approved for sale.
		\label{ncert/12/13/2/3/defs.tex}
		\item Two balls are drawn at random with replacement from a box containing 10 black and 8 red balls. Find the probability that
		\label{ncert/12/13/2/12}
\begin{enumerate}
\item both balls are red.
\item first ball is black and second is red.
\item one of them is black and other is red.
\end{enumerate}

\item In a hostel, 60\% of the students read Hindi newspaper, 40\% read English newspaper and 20\% read both Hindi and English newspapers. A student is selected at random.
		\label{ncert/12/13/2/15}
\begin{enumerate}
\item Find the probability that she reads neither Hindi nor English newspapers.
\item If she reads Hindi newspaper, find the probability that she reads English newspaper.
\item If she reads English newspaper, find the probability that she reads Hindi newspaper.\\
\end{enumerate}
\item The probability of obtaining an even prime number on each die, when a pair of dice is rolled is 
\begin{enumerate}
    \item $0$ 
    
    \item $\frac{1}{3}$ 
    
    \item $\frac{1}{12}$ 
    
    \item $\frac{1}{36}$ 
\end{enumerate}
\solution
		%\input{ncert/12/13/2/17/defs.tex}
	\item A bag contains 4 red and 4 black balls, another bag contains 2 red and 6 black balls. One of the two bags is selected at random and a ball is drawn from the bag which is found to be red. Find the probability that the ball is drawn from the first bag.
\\
\solution
		%\input{ncert/12/13/3/2/main.tex}
  \item
  Cards with numbers 2 to 101 are placed in a box. A card is selected at random.Find the probability that the card has
\begin{enumerate}[label=(\roman*)]
	\item an even number 
	\item a square number
\end{enumerate}
\solution
%\input{exemplar/10/13/3/32/main.tex}
\item
The king, queen and jack of clubs are removed from a deck of 52 playing cards and then well shuffled. Now one card is drawn at random from the remaining cards.  Determine the probability that the card is
\begin{enumerate}[label=(\roman*)]
\item a club
\item 10 of hearts
\end{enumerate}
\solution
%\input{exemplar/10/13/3/29/main.tex}
\item A team of medical students doing their internship have to assist during surgeries
at a city hospital. The probabilities of surgeries rated as very complex, complex,
routine, simple or very simple are respectively, 0.15, 0.20, 0.31, 0.26, .08. Find
the probabilities that a particular surgery will be rated
\begin{enumerate}
	\item complex or very complex;
	\item neither very complex nor very simple;
	\item routine or complex
	\item routine or simple
\end{enumerate}
\solution
%\input{exemplar/11/16/3/8(1)/main.tex}
\item A card is selected from a pack of 52 cards.
\begin{enumerate}[label=(\alph*)]
    \item How many points are there in the sample space?
    \item Calculate the probability that the card is an ace of spades.
    \item Calculate the probability that the card is (i) an ace and (ii) black card.
\end{enumerate}
\solution
%\input{exemplar/11/16/3/4/main2.tex}
\item The probability that a non leap year selected at random will contain 53 sundays.
\\
\solution
%\input{exemplar/10/13/1/19/main.tex}
\item One of the four persons John, Rita, Aslam or Gurpreet will be promoted next
month. Consequently the sample space consists of four elementary outcomes
S = {John promoted, Rita promoted, Aslam promoted, Gurpreet promoted}
You are told that the chances of John’s promotion is same as that of Gurpreet,
Rita’s chances of promotion are twice as likely as Johns. Aslam’s chances are
four times that of John.
\begin{enumerate}
	\item Determine
	\begin{enumerate}
		\item P (John promoted)
		\item P (Rita promoted)
		\item P (Aslam promoted)
		\item P (Gurpreet promoted)
	\end{enumerate}
	\item If A = {John promoted or Gurpreet promoted}, find P (A).
\end{enumerate}
\solution
%\input{exemplar/11/16/3/10/main.tex}
\item A card is drawn from a deck of 52 cards. Find the probability of getting a king or a heart or a red card.\\
\solution
%\input{exemplar/11/16/3/15/main.tex}
\item The probability that a student will pass his examination is 0.73, the probability of
the student getting a compartment is 0.13, and the probability that the student will
either pass or get compartment is 0.96. State True or False.\\
\solution
%\input{exemplar/11/16/3/31/main.tex}
\item A card is selected from a pack of 52 cards\\
\begin{enumerate}[label=(\alph*)]
\item How many points are there in the sample space?
\item Calculate the probability that the cards is an ace of spades.
\item Calculate the probability that the card is (i) an ace (ii)black card.\\
\end{enumerate}
%\input{ncert/11/16/3/4_1/Prob_4.tex}
\item In a non-leap year, the probability of having 53 tuesdays or 53 wednesdays is\\
\solution
%\input{exemplar/11/16/3/18/main.tex}
\item There are 1000 sealed envelopes in a box, 10 of them contain a cash prize of
Rs 100 each, 100 of them contain a cash prize of Rs 50 each and 200 of them
contain a cash prize of Rs 10 each and rest do not contain any cash prize. If they
are well shuffled and an envelope is picked up out, what is the probability that it
contains no cash prize?\\
\solution
%\input{exemplar/10/13/3/34/main.tex}
\item 
A die is thrown and a card is selected at random from a deck of 52 playing cards. The probability of getting an even number on the die and a spade card.\\
\solution
%\input{exemplar/12/13/3/78/main.tex}
\item
If 4-digit numbers greater than 5,000 are randomly formed from the digits 0, 1, 3, 5, and 7, what is the probability of forming a number divisible by 5 when:
\begin{enumerate}
    \item The digits are repeated?
    \item The repetition of digits is not allowed?
\end{enumerate}
\solution
%\input{ncert/11/16/4/9/main.tex}
\item Consider the probability space $\brak{\Omega, \mathcal{G}, P}$ where $\Omega = [0,2]$ and $\mathcal{G} = \cbrak{\phi, \Omega, [0,1], (1,2]}$. Let $X$ and $Y$ be two functions on $\Omega$ defined as
\begin{align*}
    X(\omega) = 
    \begin{cases}
        1 & \text{if }\omega \in [0, 1]\\
        2 & \text{if }\omega \in (1, 2]
    \end{cases}
\end{align*}
and
\begin{align*}
    Y(\omega) = 
    \begin{cases}
        2 & \text{if }\omega \in [0, 1.5]\\
        3 & \text{if }\omega \in (1.5, 2].
    \end{cases}
\end{align*}
Then which one of the following statements is true?
\begin{enumerate}
    \item [(A)] $X$ is a random variable with respect to $\mathcal{G}$, but $Y$ is not a random variable with respect to $\mathcal{G}$.
    \item [(B)] $Y$ is a random variable with respect to $\mathcal{G}$, but $X$ is not a random variable with respect to $\mathcal{G}$.
    \item [(C)] Neither $X$ nor $Y$ is a random variable with respect to $\mathcal{G}$.
    \item [(D)] Both $X$ and $Y$ are random variables with respect to $\mathcal{G}$.
\end{enumerate} \hfill (GATE ST 2023)\\
\solution
%\input{gate/ST/2023/14/main.tex}
	\item  A die is loaded in such a way that each odd number is twice as likely to occur as
each even number. Find $P(G)$, where $G$ is the event that a number greater than
3 occurs on a single roll of the die.
\\
\solution
		%\input{exemplar/11/16/3/5/main.tex}
	\item All the jacks, queens and kings are removed from a deck of 52 playing cards. The remaining cards are well shuffled and then one card is drawn at random. Giving ace a value 1 similar value for other cards, find the probability that the card has a value 
		\begin{enumerate}
			\item 7
			\item greater than 7
			\item less than 7
		\end{enumerate}
		%\input{exemplar/10/13/3/30/main.tex}
  \item A Lot consists of 48 mobile phones of which 42 are good, 3 have only minor defects and 3 have major defects.Varnika will buy a phone if it is good but the trader will only buy a mobile if it has no major defects. One phone is selected at random from the lot. What is the probability that it is
\begin{enumerate}
	\item acceptable to Varnika?
            \item acceptable to the trader?
\end{enumerate}
\solution
	%\input{exemplar/10/13/3/40/main.tex}
 \item A student says that if you throw a die, it will show up 1 or not 1. Therefore, the probability of getting 1 and the probability of getting 'not 1' each is equal to $\frac{1}{2}$. Is this correct? Give reasons.\\
 \solution
        %\input{exemplar/10/13/2/9/main.tex}
   \item Four candidates A, B, C, D have ap-
plied for the assignment to coach a school cricket
team. If A is twice as likely to be selected as B, and
B and C are given about the same chance of being
selected, while C is twice as likely to be selected
as D, what are the probabilities that
\begin{enumerate}
\item C will be selected?
\item A will not be selected?
\end{enumerate}
	%\input{exemplar/11/16/3/9/main.tex}
 \item A bag contain 24 balls of which $x$ balls are red, $2x$ are white and $3x$ are blue. A ball is selected at random, What is the probability that it is
\begin{enumerate}[label=\alph*)]
\item not red ?
\item white ?
\end{enumerate}
%\input{exemplar/10/13/3/41/main.tex}
If the letters of the word ASSASSINATION are arranged at random. Find the Probability that
\begin{enumerate}[label=(\alph*)]
\item Four $S's$ come consecutively in the word
\item Two  $I's$ and two $N's$ come together
\item All $A's$ are not coming together
\item No two $A's$ are coming together
\end{enumerate}
%\input{exemplar/11/16/3/14/main.tex}
	\item One urn contains two black balls (labelled B1 and B2) and one white ball. A
	second urn contains one black ball and two white balls (labelled W1 and W2).
	Suppose the following experiment is performed. One of the two urns is chosen
	at random. Next a ball is randomly chosen from the urn. Then a second ball is
	chosen at random from the same urn without replacing the first ball.
	
	\begin{enumerate}
	\item What is the probability that two black balls are chosen?
	
	\item What is the probability that two balls of opposite colour are chosen?
	\end{enumerate}
	\solution
	%\input{exemplar/11/16/3/12/main1.tex}
\end{enumerate}

		%
\item 
Out of 100 students, two sections of 40 and 60 are formed. If you and your friend are among the 100 students, what is the probability that
\begin{enumerate}
\item you both enter the same section?
\item you both enter the different sections?
\end{enumerate}
\solution
		%\begin{enumerate}[label=\thesection.\arabic*,ref=\thesection.\theenumi]
	\item One card is drawn from a well-shuffled deck of 52 cards. Find the probability of getting
\begin{enumerate}
\item A king of red colour 
\item A face card 
\item A red face card
\item The jack of hearts
\item A spade
\item The queen of diamonds

\end{enumerate}
\solution
		%\input{ncert/10/15/1/14/main.tex}
	\item Five cards—the ten, jack, queen, king and ace of diamonds, are well-shuffled with their face downwards. One card is then picked up at random.
\begin{enumerate}
\item
What is the probability that the card is the queen? 
\item
If the queen is drawn and put aside, what is the probability that the second card picked up is (a) an ace? (b) a queen?\\
\end{enumerate}
\solution
		%\input{ncert/10/15/1/15/defs.tex}
	\item A bag contains $5$ red balls and some blue balls. If the probability of drawing a blue ball is double that if a red ball, determine the number of blue balls in the bag. 
		\\
\solution
		%\input{ncert/10/15/2/3/defs.tex}
	\item A card is selected from a pack of 52 cards.
 \begin{enumerate}[label=(\alph*)] 
                 \item How many points are there in the sample space?
                 \item Calculate the probability that the card is an ace of spades.
                 \item Calculate the probability that the card is (i) an ace and (ii) black card.
 \end{enumerate}
\solution
		%\input{ncert/11/16/3/4/main.tex}
\item Four cards are drawn from a well-shuffled deck of 52 cards. What is the probability of obtaining 3 diamonds and one spade.
\\
\solution
		%\input{ncert/11/16/4/2/defs.tex}
\item In a certain lottery 10,000 tickets are sold and ten equal prizes are awarded. What is the probability of not getting a prize if you buy (a) one ticket (b) two tickets (c) 10 tickets ?	
\\
\solution
		%\input{ncert/11/16/4/4/defs.tex}
		%
\item 
Out of 100 students, two sections of 40 and 60 are formed. If you and your friend are among the 100 students, what is the probability that
\begin{enumerate}
\item you both enter the same section?
\item you both enter the different sections?
\end{enumerate}
\solution
		%\input{ncert/11/16/4/5/defs.tex}
	\item 
The number lock of a suitcase has 4 wheels each labelled with ten digits i.e. from 0 to 9.The lock opens with a sequence of four digits with no repeats.What is the probability of a person getting the right sequence to open the suitcase.
\\
\solution
		%\input{ncert/11/16/4/10/defs.tex}
		%
\item 
Two cards are drawn at random and without replacement from a pack of 52 playing cards. Find the probability that both the cards are black.
\\
\solution
		%\input{ncert/12/13/2/2/defs.tex}
		\item A box of oranges is inspected by examining three randomly selected oranges drawn without replacement. If all the three oranges are good, the box is approved for sale, otherwise, it is rejected. Find the probability that a box containing 15 oranges out of which 12 are good and 3 are bad ones will be approved for sale.
		\label{ncert/12/13/2/3/defs.tex}
		\item Two balls are drawn at random with replacement from a box containing 10 black and 8 red balls. Find the probability that
		\label{ncert/12/13/2/12}
\begin{enumerate}
\item both balls are red.
\item first ball is black and second is red.
\item one of them is black and other is red.
\end{enumerate}

\item In a hostel, 60\% of the students read Hindi newspaper, 40\% read English newspaper and 20\% read both Hindi and English newspapers. A student is selected at random.
		\label{ncert/12/13/2/15}
\begin{enumerate}
\item Find the probability that she reads neither Hindi nor English newspapers.
\item If she reads Hindi newspaper, find the probability that she reads English newspaper.
\item If she reads English newspaper, find the probability that she reads Hindi newspaper.\\
\end{enumerate}
\item The probability of obtaining an even prime number on each die, when a pair of dice is rolled is 
\begin{enumerate}
    \item $0$ 
    
    \item $\frac{1}{3}$ 
    
    \item $\frac{1}{12}$ 
    
    \item $\frac{1}{36}$ 
\end{enumerate}
\solution
		%\input{ncert/12/13/2/17/defs.tex}
	\item A bag contains 4 red and 4 black balls, another bag contains 2 red and 6 black balls. One of the two bags is selected at random and a ball is drawn from the bag which is found to be red. Find the probability that the ball is drawn from the first bag.
\\
\solution
		%\input{ncert/12/13/3/2/main.tex}
  \item
  Cards with numbers 2 to 101 are placed in a box. A card is selected at random.Find the probability that the card has
\begin{enumerate}[label=(\roman*)]
	\item an even number 
	\item a square number
\end{enumerate}
\solution
%\input{exemplar/10/13/3/32/main.tex}
\item
The king, queen and jack of clubs are removed from a deck of 52 playing cards and then well shuffled. Now one card is drawn at random from the remaining cards.  Determine the probability that the card is
\begin{enumerate}[label=(\roman*)]
\item a club
\item 10 of hearts
\end{enumerate}
\solution
%\input{exemplar/10/13/3/29/main.tex}
\item A team of medical students doing their internship have to assist during surgeries
at a city hospital. The probabilities of surgeries rated as very complex, complex,
routine, simple or very simple are respectively, 0.15, 0.20, 0.31, 0.26, .08. Find
the probabilities that a particular surgery will be rated
\begin{enumerate}
	\item complex or very complex;
	\item neither very complex nor very simple;
	\item routine or complex
	\item routine or simple
\end{enumerate}
\solution
%\input{exemplar/11/16/3/8(1)/main.tex}
\item A card is selected from a pack of 52 cards.
\begin{enumerate}[label=(\alph*)]
    \item How many points are there in the sample space?
    \item Calculate the probability that the card is an ace of spades.
    \item Calculate the probability that the card is (i) an ace and (ii) black card.
\end{enumerate}
\solution
%\input{exemplar/11/16/3/4/main2.tex}
\item The probability that a non leap year selected at random will contain 53 sundays.
\\
\solution
%\input{exemplar/10/13/1/19/main.tex}
\item One of the four persons John, Rita, Aslam or Gurpreet will be promoted next
month. Consequently the sample space consists of four elementary outcomes
S = {John promoted, Rita promoted, Aslam promoted, Gurpreet promoted}
You are told that the chances of John’s promotion is same as that of Gurpreet,
Rita’s chances of promotion are twice as likely as Johns. Aslam’s chances are
four times that of John.
\begin{enumerate}
	\item Determine
	\begin{enumerate}
		\item P (John promoted)
		\item P (Rita promoted)
		\item P (Aslam promoted)
		\item P (Gurpreet promoted)
	\end{enumerate}
	\item If A = {John promoted or Gurpreet promoted}, find P (A).
\end{enumerate}
\solution
%\input{exemplar/11/16/3/10/main.tex}
\item A card is drawn from a deck of 52 cards. Find the probability of getting a king or a heart or a red card.\\
\solution
%\input{exemplar/11/16/3/15/main.tex}
\item The probability that a student will pass his examination is 0.73, the probability of
the student getting a compartment is 0.13, and the probability that the student will
either pass or get compartment is 0.96. State True or False.\\
\solution
%\input{exemplar/11/16/3/31/main.tex}
\item A card is selected from a pack of 52 cards\\
\begin{enumerate}[label=(\alph*)]
\item How many points are there in the sample space?
\item Calculate the probability that the cards is an ace of spades.
\item Calculate the probability that the card is (i) an ace (ii)black card.\\
\end{enumerate}
%\input{ncert/11/16/3/4_1/Prob_4.tex}
\item In a non-leap year, the probability of having 53 tuesdays or 53 wednesdays is\\
\solution
%\input{exemplar/11/16/3/18/main.tex}
\item There are 1000 sealed envelopes in a box, 10 of them contain a cash prize of
Rs 100 each, 100 of them contain a cash prize of Rs 50 each and 200 of them
contain a cash prize of Rs 10 each and rest do not contain any cash prize. If they
are well shuffled and an envelope is picked up out, what is the probability that it
contains no cash prize?\\
\solution
%\input{exemplar/10/13/3/34/main.tex}
\item 
A die is thrown and a card is selected at random from a deck of 52 playing cards. The probability of getting an even number on the die and a spade card.\\
\solution
%\input{exemplar/12/13/3/78/main.tex}
\item
If 4-digit numbers greater than 5,000 are randomly formed from the digits 0, 1, 3, 5, and 7, what is the probability of forming a number divisible by 5 when:
\begin{enumerate}
    \item The digits are repeated?
    \item The repetition of digits is not allowed?
\end{enumerate}
\solution
%\input{ncert/11/16/4/9/main.tex}
\item Consider the probability space $\brak{\Omega, \mathcal{G}, P}$ where $\Omega = [0,2]$ and $\mathcal{G} = \cbrak{\phi, \Omega, [0,1], (1,2]}$. Let $X$ and $Y$ be two functions on $\Omega$ defined as
\begin{align*}
    X(\omega) = 
    \begin{cases}
        1 & \text{if }\omega \in [0, 1]\\
        2 & \text{if }\omega \in (1, 2]
    \end{cases}
\end{align*}
and
\begin{align*}
    Y(\omega) = 
    \begin{cases}
        2 & \text{if }\omega \in [0, 1.5]\\
        3 & \text{if }\omega \in (1.5, 2].
    \end{cases}
\end{align*}
Then which one of the following statements is true?
\begin{enumerate}
    \item [(A)] $X$ is a random variable with respect to $\mathcal{G}$, but $Y$ is not a random variable with respect to $\mathcal{G}$.
    \item [(B)] $Y$ is a random variable with respect to $\mathcal{G}$, but $X$ is not a random variable with respect to $\mathcal{G}$.
    \item [(C)] Neither $X$ nor $Y$ is a random variable with respect to $\mathcal{G}$.
    \item [(D)] Both $X$ and $Y$ are random variables with respect to $\mathcal{G}$.
\end{enumerate} \hfill (GATE ST 2023)\\
\solution
%\input{gate/ST/2023/14/main.tex}
	\item  A die is loaded in such a way that each odd number is twice as likely to occur as
each even number. Find $P(G)$, where $G$ is the event that a number greater than
3 occurs on a single roll of the die.
\\
\solution
		%\input{exemplar/11/16/3/5/main.tex}
	\item All the jacks, queens and kings are removed from a deck of 52 playing cards. The remaining cards are well shuffled and then one card is drawn at random. Giving ace a value 1 similar value for other cards, find the probability that the card has a value 
		\begin{enumerate}
			\item 7
			\item greater than 7
			\item less than 7
		\end{enumerate}
		%\input{exemplar/10/13/3/30/main.tex}
  \item A Lot consists of 48 mobile phones of which 42 are good, 3 have only minor defects and 3 have major defects.Varnika will buy a phone if it is good but the trader will only buy a mobile if it has no major defects. One phone is selected at random from the lot. What is the probability that it is
\begin{enumerate}
	\item acceptable to Varnika?
            \item acceptable to the trader?
\end{enumerate}
\solution
	%\input{exemplar/10/13/3/40/main.tex}
 \item A student says that if you throw a die, it will show up 1 or not 1. Therefore, the probability of getting 1 and the probability of getting 'not 1' each is equal to $\frac{1}{2}$. Is this correct? Give reasons.\\
 \solution
        %\input{exemplar/10/13/2/9/main.tex}
   \item Four candidates A, B, C, D have ap-
plied for the assignment to coach a school cricket
team. If A is twice as likely to be selected as B, and
B and C are given about the same chance of being
selected, while C is twice as likely to be selected
as D, what are the probabilities that
\begin{enumerate}
\item C will be selected?
\item A will not be selected?
\end{enumerate}
	%\input{exemplar/11/16/3/9/main.tex}
 \item A bag contain 24 balls of which $x$ balls are red, $2x$ are white and $3x$ are blue. A ball is selected at random, What is the probability that it is
\begin{enumerate}[label=\alph*)]
\item not red ?
\item white ?
\end{enumerate}
%\input{exemplar/10/13/3/41/main.tex}
If the letters of the word ASSASSINATION are arranged at random. Find the Probability that
\begin{enumerate}[label=(\alph*)]
\item Four $S's$ come consecutively in the word
\item Two  $I's$ and two $N's$ come together
\item All $A's$ are not coming together
\item No two $A's$ are coming together
\end{enumerate}
%\input{exemplar/11/16/3/14/main.tex}
	\item One urn contains two black balls (labelled B1 and B2) and one white ball. A
	second urn contains one black ball and two white balls (labelled W1 and W2).
	Suppose the following experiment is performed. One of the two urns is chosen
	at random. Next a ball is randomly chosen from the urn. Then a second ball is
	chosen at random from the same urn without replacing the first ball.
	
	\begin{enumerate}
	\item What is the probability that two black balls are chosen?
	
	\item What is the probability that two balls of opposite colour are chosen?
	\end{enumerate}
	\solution
	%\input{exemplar/11/16/3/12/main1.tex}
\end{enumerate}

	\item 
The number lock of a suitcase has 4 wheels each labelled with ten digits i.e. from 0 to 9.The lock opens with a sequence of four digits with no repeats.What is the probability of a person getting the right sequence to open the suitcase.
\\
\solution
		%\begin{enumerate}[label=\thesection.\arabic*,ref=\thesection.\theenumi]
	\item One card is drawn from a well-shuffled deck of 52 cards. Find the probability of getting
\begin{enumerate}
\item A king of red colour 
\item A face card 
\item A red face card
\item The jack of hearts
\item A spade
\item The queen of diamonds

\end{enumerate}
\solution
		%\input{ncert/10/15/1/14/main.tex}
	\item Five cards—the ten, jack, queen, king and ace of diamonds, are well-shuffled with their face downwards. One card is then picked up at random.
\begin{enumerate}
\item
What is the probability that the card is the queen? 
\item
If the queen is drawn and put aside, what is the probability that the second card picked up is (a) an ace? (b) a queen?\\
\end{enumerate}
\solution
		%\input{ncert/10/15/1/15/defs.tex}
	\item A bag contains $5$ red balls and some blue balls. If the probability of drawing a blue ball is double that if a red ball, determine the number of blue balls in the bag. 
		\\
\solution
		%\input{ncert/10/15/2/3/defs.tex}
	\item A card is selected from a pack of 52 cards.
 \begin{enumerate}[label=(\alph*)] 
                 \item How many points are there in the sample space?
                 \item Calculate the probability that the card is an ace of spades.
                 \item Calculate the probability that the card is (i) an ace and (ii) black card.
 \end{enumerate}
\solution
		%\input{ncert/11/16/3/4/main.tex}
\item Four cards are drawn from a well-shuffled deck of 52 cards. What is the probability of obtaining 3 diamonds and one spade.
\\
\solution
		%\input{ncert/11/16/4/2/defs.tex}
\item In a certain lottery 10,000 tickets are sold and ten equal prizes are awarded. What is the probability of not getting a prize if you buy (a) one ticket (b) two tickets (c) 10 tickets ?	
\\
\solution
		%\input{ncert/11/16/4/4/defs.tex}
		%
\item 
Out of 100 students, two sections of 40 and 60 are formed. If you and your friend are among the 100 students, what is the probability that
\begin{enumerate}
\item you both enter the same section?
\item you both enter the different sections?
\end{enumerate}
\solution
		%\input{ncert/11/16/4/5/defs.tex}
	\item 
The number lock of a suitcase has 4 wheels each labelled with ten digits i.e. from 0 to 9.The lock opens with a sequence of four digits with no repeats.What is the probability of a person getting the right sequence to open the suitcase.
\\
\solution
		%\input{ncert/11/16/4/10/defs.tex}
		%
\item 
Two cards are drawn at random and without replacement from a pack of 52 playing cards. Find the probability that both the cards are black.
\\
\solution
		%\input{ncert/12/13/2/2/defs.tex}
		\item A box of oranges is inspected by examining three randomly selected oranges drawn without replacement. If all the three oranges are good, the box is approved for sale, otherwise, it is rejected. Find the probability that a box containing 15 oranges out of which 12 are good and 3 are bad ones will be approved for sale.
		\label{ncert/12/13/2/3/defs.tex}
		\item Two balls are drawn at random with replacement from a box containing 10 black and 8 red balls. Find the probability that
		\label{ncert/12/13/2/12}
\begin{enumerate}
\item both balls are red.
\item first ball is black and second is red.
\item one of them is black and other is red.
\end{enumerate}

\item In a hostel, 60\% of the students read Hindi newspaper, 40\% read English newspaper and 20\% read both Hindi and English newspapers. A student is selected at random.
		\label{ncert/12/13/2/15}
\begin{enumerate}
\item Find the probability that she reads neither Hindi nor English newspapers.
\item If she reads Hindi newspaper, find the probability that she reads English newspaper.
\item If she reads English newspaper, find the probability that she reads Hindi newspaper.\\
\end{enumerate}
\item The probability of obtaining an even prime number on each die, when a pair of dice is rolled is 
\begin{enumerate}
    \item $0$ 
    
    \item $\frac{1}{3}$ 
    
    \item $\frac{1}{12}$ 
    
    \item $\frac{1}{36}$ 
\end{enumerate}
\solution
		%\input{ncert/12/13/2/17/defs.tex}
	\item A bag contains 4 red and 4 black balls, another bag contains 2 red and 6 black balls. One of the two bags is selected at random and a ball is drawn from the bag which is found to be red. Find the probability that the ball is drawn from the first bag.
\\
\solution
		%\input{ncert/12/13/3/2/main.tex}
  \item
  Cards with numbers 2 to 101 are placed in a box. A card is selected at random.Find the probability that the card has
\begin{enumerate}[label=(\roman*)]
	\item an even number 
	\item a square number
\end{enumerate}
\solution
%\input{exemplar/10/13/3/32/main.tex}
\item
The king, queen and jack of clubs are removed from a deck of 52 playing cards and then well shuffled. Now one card is drawn at random from the remaining cards.  Determine the probability that the card is
\begin{enumerate}[label=(\roman*)]
\item a club
\item 10 of hearts
\end{enumerate}
\solution
%\input{exemplar/10/13/3/29/main.tex}
\item A team of medical students doing their internship have to assist during surgeries
at a city hospital. The probabilities of surgeries rated as very complex, complex,
routine, simple or very simple are respectively, 0.15, 0.20, 0.31, 0.26, .08. Find
the probabilities that a particular surgery will be rated
\begin{enumerate}
	\item complex or very complex;
	\item neither very complex nor very simple;
	\item routine or complex
	\item routine or simple
\end{enumerate}
\solution
%\input{exemplar/11/16/3/8(1)/main.tex}
\item A card is selected from a pack of 52 cards.
\begin{enumerate}[label=(\alph*)]
    \item How many points are there in the sample space?
    \item Calculate the probability that the card is an ace of spades.
    \item Calculate the probability that the card is (i) an ace and (ii) black card.
\end{enumerate}
\solution
%\input{exemplar/11/16/3/4/main2.tex}
\item The probability that a non leap year selected at random will contain 53 sundays.
\\
\solution
%\input{exemplar/10/13/1/19/main.tex}
\item One of the four persons John, Rita, Aslam or Gurpreet will be promoted next
month. Consequently the sample space consists of four elementary outcomes
S = {John promoted, Rita promoted, Aslam promoted, Gurpreet promoted}
You are told that the chances of John’s promotion is same as that of Gurpreet,
Rita’s chances of promotion are twice as likely as Johns. Aslam’s chances are
four times that of John.
\begin{enumerate}
	\item Determine
	\begin{enumerate}
		\item P (John promoted)
		\item P (Rita promoted)
		\item P (Aslam promoted)
		\item P (Gurpreet promoted)
	\end{enumerate}
	\item If A = {John promoted or Gurpreet promoted}, find P (A).
\end{enumerate}
\solution
%\input{exemplar/11/16/3/10/main.tex}
\item A card is drawn from a deck of 52 cards. Find the probability of getting a king or a heart or a red card.\\
\solution
%\input{exemplar/11/16/3/15/main.tex}
\item The probability that a student will pass his examination is 0.73, the probability of
the student getting a compartment is 0.13, and the probability that the student will
either pass or get compartment is 0.96. State True or False.\\
\solution
%\input{exemplar/11/16/3/31/main.tex}
\item A card is selected from a pack of 52 cards\\
\begin{enumerate}[label=(\alph*)]
\item How many points are there in the sample space?
\item Calculate the probability that the cards is an ace of spades.
\item Calculate the probability that the card is (i) an ace (ii)black card.\\
\end{enumerate}
%\input{ncert/11/16/3/4_1/Prob_4.tex}
\item In a non-leap year, the probability of having 53 tuesdays or 53 wednesdays is\\
\solution
%\input{exemplar/11/16/3/18/main.tex}
\item There are 1000 sealed envelopes in a box, 10 of them contain a cash prize of
Rs 100 each, 100 of them contain a cash prize of Rs 50 each and 200 of them
contain a cash prize of Rs 10 each and rest do not contain any cash prize. If they
are well shuffled and an envelope is picked up out, what is the probability that it
contains no cash prize?\\
\solution
%\input{exemplar/10/13/3/34/main.tex}
\item 
A die is thrown and a card is selected at random from a deck of 52 playing cards. The probability of getting an even number on the die and a spade card.\\
\solution
%\input{exemplar/12/13/3/78/main.tex}
\item
If 4-digit numbers greater than 5,000 are randomly formed from the digits 0, 1, 3, 5, and 7, what is the probability of forming a number divisible by 5 when:
\begin{enumerate}
    \item The digits are repeated?
    \item The repetition of digits is not allowed?
\end{enumerate}
\solution
%\input{ncert/11/16/4/9/main.tex}
\item Consider the probability space $\brak{\Omega, \mathcal{G}, P}$ where $\Omega = [0,2]$ and $\mathcal{G} = \cbrak{\phi, \Omega, [0,1], (1,2]}$. Let $X$ and $Y$ be two functions on $\Omega$ defined as
\begin{align*}
    X(\omega) = 
    \begin{cases}
        1 & \text{if }\omega \in [0, 1]\\
        2 & \text{if }\omega \in (1, 2]
    \end{cases}
\end{align*}
and
\begin{align*}
    Y(\omega) = 
    \begin{cases}
        2 & \text{if }\omega \in [0, 1.5]\\
        3 & \text{if }\omega \in (1.5, 2].
    \end{cases}
\end{align*}
Then which one of the following statements is true?
\begin{enumerate}
    \item [(A)] $X$ is a random variable with respect to $\mathcal{G}$, but $Y$ is not a random variable with respect to $\mathcal{G}$.
    \item [(B)] $Y$ is a random variable with respect to $\mathcal{G}$, but $X$ is not a random variable with respect to $\mathcal{G}$.
    \item [(C)] Neither $X$ nor $Y$ is a random variable with respect to $\mathcal{G}$.
    \item [(D)] Both $X$ and $Y$ are random variables with respect to $\mathcal{G}$.
\end{enumerate} \hfill (GATE ST 2023)\\
\solution
%\input{gate/ST/2023/14/main.tex}
	\item  A die is loaded in such a way that each odd number is twice as likely to occur as
each even number. Find $P(G)$, where $G$ is the event that a number greater than
3 occurs on a single roll of the die.
\\
\solution
		%\input{exemplar/11/16/3/5/main.tex}
	\item All the jacks, queens and kings are removed from a deck of 52 playing cards. The remaining cards are well shuffled and then one card is drawn at random. Giving ace a value 1 similar value for other cards, find the probability that the card has a value 
		\begin{enumerate}
			\item 7
			\item greater than 7
			\item less than 7
		\end{enumerate}
		%\input{exemplar/10/13/3/30/main.tex}
  \item A Lot consists of 48 mobile phones of which 42 are good, 3 have only minor defects and 3 have major defects.Varnika will buy a phone if it is good but the trader will only buy a mobile if it has no major defects. One phone is selected at random from the lot. What is the probability that it is
\begin{enumerate}
	\item acceptable to Varnika?
            \item acceptable to the trader?
\end{enumerate}
\solution
	%\input{exemplar/10/13/3/40/main.tex}
 \item A student says that if you throw a die, it will show up 1 or not 1. Therefore, the probability of getting 1 and the probability of getting 'not 1' each is equal to $\frac{1}{2}$. Is this correct? Give reasons.\\
 \solution
        %\input{exemplar/10/13/2/9/main.tex}
   \item Four candidates A, B, C, D have ap-
plied for the assignment to coach a school cricket
team. If A is twice as likely to be selected as B, and
B and C are given about the same chance of being
selected, while C is twice as likely to be selected
as D, what are the probabilities that
\begin{enumerate}
\item C will be selected?
\item A will not be selected?
\end{enumerate}
	%\input{exemplar/11/16/3/9/main.tex}
 \item A bag contain 24 balls of which $x$ balls are red, $2x$ are white and $3x$ are blue. A ball is selected at random, What is the probability that it is
\begin{enumerate}[label=\alph*)]
\item not red ?
\item white ?
\end{enumerate}
%\input{exemplar/10/13/3/41/main.tex}
If the letters of the word ASSASSINATION are arranged at random. Find the Probability that
\begin{enumerate}[label=(\alph*)]
\item Four $S's$ come consecutively in the word
\item Two  $I's$ and two $N's$ come together
\item All $A's$ are not coming together
\item No two $A's$ are coming together
\end{enumerate}
%\input{exemplar/11/16/3/14/main.tex}
	\item One urn contains two black balls (labelled B1 and B2) and one white ball. A
	second urn contains one black ball and two white balls (labelled W1 and W2).
	Suppose the following experiment is performed. One of the two urns is chosen
	at random. Next a ball is randomly chosen from the urn. Then a second ball is
	chosen at random from the same urn without replacing the first ball.
	
	\begin{enumerate}
	\item What is the probability that two black balls are chosen?
	
	\item What is the probability that two balls of opposite colour are chosen?
	\end{enumerate}
	\solution
	%\input{exemplar/11/16/3/12/main1.tex}
\end{enumerate}

		%
\item 
Two cards are drawn at random and without replacement from a pack of 52 playing cards. Find the probability that both the cards are black.
\\
\solution
		%\begin{enumerate}[label=\thesection.\arabic*,ref=\thesection.\theenumi]
	\item One card is drawn from a well-shuffled deck of 52 cards. Find the probability of getting
\begin{enumerate}
\item A king of red colour 
\item A face card 
\item A red face card
\item The jack of hearts
\item A spade
\item The queen of diamonds

\end{enumerate}
\solution
		%\input{ncert/10/15/1/14/main.tex}
	\item Five cards—the ten, jack, queen, king and ace of diamonds, are well-shuffled with their face downwards. One card is then picked up at random.
\begin{enumerate}
\item
What is the probability that the card is the queen? 
\item
If the queen is drawn and put aside, what is the probability that the second card picked up is (a) an ace? (b) a queen?\\
\end{enumerate}
\solution
		%\input{ncert/10/15/1/15/defs.tex}
	\item A bag contains $5$ red balls and some blue balls. If the probability of drawing a blue ball is double that if a red ball, determine the number of blue balls in the bag. 
		\\
\solution
		%\input{ncert/10/15/2/3/defs.tex}
	\item A card is selected from a pack of 52 cards.
 \begin{enumerate}[label=(\alph*)] 
                 \item How many points are there in the sample space?
                 \item Calculate the probability that the card is an ace of spades.
                 \item Calculate the probability that the card is (i) an ace and (ii) black card.
 \end{enumerate}
\solution
		%\input{ncert/11/16/3/4/main.tex}
\item Four cards are drawn from a well-shuffled deck of 52 cards. What is the probability of obtaining 3 diamonds and one spade.
\\
\solution
		%\input{ncert/11/16/4/2/defs.tex}
\item In a certain lottery 10,000 tickets are sold and ten equal prizes are awarded. What is the probability of not getting a prize if you buy (a) one ticket (b) two tickets (c) 10 tickets ?	
\\
\solution
		%\input{ncert/11/16/4/4/defs.tex}
		%
\item 
Out of 100 students, two sections of 40 and 60 are formed. If you and your friend are among the 100 students, what is the probability that
\begin{enumerate}
\item you both enter the same section?
\item you both enter the different sections?
\end{enumerate}
\solution
		%\input{ncert/11/16/4/5/defs.tex}
	\item 
The number lock of a suitcase has 4 wheels each labelled with ten digits i.e. from 0 to 9.The lock opens with a sequence of four digits with no repeats.What is the probability of a person getting the right sequence to open the suitcase.
\\
\solution
		%\input{ncert/11/16/4/10/defs.tex}
		%
\item 
Two cards are drawn at random and without replacement from a pack of 52 playing cards. Find the probability that both the cards are black.
\\
\solution
		%\input{ncert/12/13/2/2/defs.tex}
		\item A box of oranges is inspected by examining three randomly selected oranges drawn without replacement. If all the three oranges are good, the box is approved for sale, otherwise, it is rejected. Find the probability that a box containing 15 oranges out of which 12 are good and 3 are bad ones will be approved for sale.
		\label{ncert/12/13/2/3/defs.tex}
		\item Two balls are drawn at random with replacement from a box containing 10 black and 8 red balls. Find the probability that
		\label{ncert/12/13/2/12}
\begin{enumerate}
\item both balls are red.
\item first ball is black and second is red.
\item one of them is black and other is red.
\end{enumerate}

\item In a hostel, 60\% of the students read Hindi newspaper, 40\% read English newspaper and 20\% read both Hindi and English newspapers. A student is selected at random.
		\label{ncert/12/13/2/15}
\begin{enumerate}
\item Find the probability that she reads neither Hindi nor English newspapers.
\item If she reads Hindi newspaper, find the probability that she reads English newspaper.
\item If she reads English newspaper, find the probability that she reads Hindi newspaper.\\
\end{enumerate}
\item The probability of obtaining an even prime number on each die, when a pair of dice is rolled is 
\begin{enumerate}
    \item $0$ 
    
    \item $\frac{1}{3}$ 
    
    \item $\frac{1}{12}$ 
    
    \item $\frac{1}{36}$ 
\end{enumerate}
\solution
		%\input{ncert/12/13/2/17/defs.tex}
	\item A bag contains 4 red and 4 black balls, another bag contains 2 red and 6 black balls. One of the two bags is selected at random and a ball is drawn from the bag which is found to be red. Find the probability that the ball is drawn from the first bag.
\\
\solution
		%\input{ncert/12/13/3/2/main.tex}
  \item
  Cards with numbers 2 to 101 are placed in a box. A card is selected at random.Find the probability that the card has
\begin{enumerate}[label=(\roman*)]
	\item an even number 
	\item a square number
\end{enumerate}
\solution
%\input{exemplar/10/13/3/32/main.tex}
\item
The king, queen and jack of clubs are removed from a deck of 52 playing cards and then well shuffled. Now one card is drawn at random from the remaining cards.  Determine the probability that the card is
\begin{enumerate}[label=(\roman*)]
\item a club
\item 10 of hearts
\end{enumerate}
\solution
%\input{exemplar/10/13/3/29/main.tex}
\item A team of medical students doing their internship have to assist during surgeries
at a city hospital. The probabilities of surgeries rated as very complex, complex,
routine, simple or very simple are respectively, 0.15, 0.20, 0.31, 0.26, .08. Find
the probabilities that a particular surgery will be rated
\begin{enumerate}
	\item complex or very complex;
	\item neither very complex nor very simple;
	\item routine or complex
	\item routine or simple
\end{enumerate}
\solution
%\input{exemplar/11/16/3/8(1)/main.tex}
\item A card is selected from a pack of 52 cards.
\begin{enumerate}[label=(\alph*)]
    \item How many points are there in the sample space?
    \item Calculate the probability that the card is an ace of spades.
    \item Calculate the probability that the card is (i) an ace and (ii) black card.
\end{enumerate}
\solution
%\input{exemplar/11/16/3/4/main2.tex}
\item The probability that a non leap year selected at random will contain 53 sundays.
\\
\solution
%\input{exemplar/10/13/1/19/main.tex}
\item One of the four persons John, Rita, Aslam or Gurpreet will be promoted next
month. Consequently the sample space consists of four elementary outcomes
S = {John promoted, Rita promoted, Aslam promoted, Gurpreet promoted}
You are told that the chances of John’s promotion is same as that of Gurpreet,
Rita’s chances of promotion are twice as likely as Johns. Aslam’s chances are
four times that of John.
\begin{enumerate}
	\item Determine
	\begin{enumerate}
		\item P (John promoted)
		\item P (Rita promoted)
		\item P (Aslam promoted)
		\item P (Gurpreet promoted)
	\end{enumerate}
	\item If A = {John promoted or Gurpreet promoted}, find P (A).
\end{enumerate}
\solution
%\input{exemplar/11/16/3/10/main.tex}
\item A card is drawn from a deck of 52 cards. Find the probability of getting a king or a heart or a red card.\\
\solution
%\input{exemplar/11/16/3/15/main.tex}
\item The probability that a student will pass his examination is 0.73, the probability of
the student getting a compartment is 0.13, and the probability that the student will
either pass or get compartment is 0.96. State True or False.\\
\solution
%\input{exemplar/11/16/3/31/main.tex}
\item A card is selected from a pack of 52 cards\\
\begin{enumerate}[label=(\alph*)]
\item How many points are there in the sample space?
\item Calculate the probability that the cards is an ace of spades.
\item Calculate the probability that the card is (i) an ace (ii)black card.\\
\end{enumerate}
%\input{ncert/11/16/3/4_1/Prob_4.tex}
\item In a non-leap year, the probability of having 53 tuesdays or 53 wednesdays is\\
\solution
%\input{exemplar/11/16/3/18/main.tex}
\item There are 1000 sealed envelopes in a box, 10 of them contain a cash prize of
Rs 100 each, 100 of them contain a cash prize of Rs 50 each and 200 of them
contain a cash prize of Rs 10 each and rest do not contain any cash prize. If they
are well shuffled and an envelope is picked up out, what is the probability that it
contains no cash prize?\\
\solution
%\input{exemplar/10/13/3/34/main.tex}
\item 
A die is thrown and a card is selected at random from a deck of 52 playing cards. The probability of getting an even number on the die and a spade card.\\
\solution
%\input{exemplar/12/13/3/78/main.tex}
\item
If 4-digit numbers greater than 5,000 are randomly formed from the digits 0, 1, 3, 5, and 7, what is the probability of forming a number divisible by 5 when:
\begin{enumerate}
    \item The digits are repeated?
    \item The repetition of digits is not allowed?
\end{enumerate}
\solution
%\input{ncert/11/16/4/9/main.tex}
\item Consider the probability space $\brak{\Omega, \mathcal{G}, P}$ where $\Omega = [0,2]$ and $\mathcal{G} = \cbrak{\phi, \Omega, [0,1], (1,2]}$. Let $X$ and $Y$ be two functions on $\Omega$ defined as
\begin{align*}
    X(\omega) = 
    \begin{cases}
        1 & \text{if }\omega \in [0, 1]\\
        2 & \text{if }\omega \in (1, 2]
    \end{cases}
\end{align*}
and
\begin{align*}
    Y(\omega) = 
    \begin{cases}
        2 & \text{if }\omega \in [0, 1.5]\\
        3 & \text{if }\omega \in (1.5, 2].
    \end{cases}
\end{align*}
Then which one of the following statements is true?
\begin{enumerate}
    \item [(A)] $X$ is a random variable with respect to $\mathcal{G}$, but $Y$ is not a random variable with respect to $\mathcal{G}$.
    \item [(B)] $Y$ is a random variable with respect to $\mathcal{G}$, but $X$ is not a random variable with respect to $\mathcal{G}$.
    \item [(C)] Neither $X$ nor $Y$ is a random variable with respect to $\mathcal{G}$.
    \item [(D)] Both $X$ and $Y$ are random variables with respect to $\mathcal{G}$.
\end{enumerate} \hfill (GATE ST 2023)\\
\solution
%\input{gate/ST/2023/14/main.tex}
	\item  A die is loaded in such a way that each odd number is twice as likely to occur as
each even number. Find $P(G)$, where $G$ is the event that a number greater than
3 occurs on a single roll of the die.
\\
\solution
		%\input{exemplar/11/16/3/5/main.tex}
	\item All the jacks, queens and kings are removed from a deck of 52 playing cards. The remaining cards are well shuffled and then one card is drawn at random. Giving ace a value 1 similar value for other cards, find the probability that the card has a value 
		\begin{enumerate}
			\item 7
			\item greater than 7
			\item less than 7
		\end{enumerate}
		%\input{exemplar/10/13/3/30/main.tex}
  \item A Lot consists of 48 mobile phones of which 42 are good, 3 have only minor defects and 3 have major defects.Varnika will buy a phone if it is good but the trader will only buy a mobile if it has no major defects. One phone is selected at random from the lot. What is the probability that it is
\begin{enumerate}
	\item acceptable to Varnika?
            \item acceptable to the trader?
\end{enumerate}
\solution
	%\input{exemplar/10/13/3/40/main.tex}
 \item A student says that if you throw a die, it will show up 1 or not 1. Therefore, the probability of getting 1 and the probability of getting 'not 1' each is equal to $\frac{1}{2}$. Is this correct? Give reasons.\\
 \solution
        %\input{exemplar/10/13/2/9/main.tex}
   \item Four candidates A, B, C, D have ap-
plied for the assignment to coach a school cricket
team. If A is twice as likely to be selected as B, and
B and C are given about the same chance of being
selected, while C is twice as likely to be selected
as D, what are the probabilities that
\begin{enumerate}
\item C will be selected?
\item A will not be selected?
\end{enumerate}
	%\input{exemplar/11/16/3/9/main.tex}
 \item A bag contain 24 balls of which $x$ balls are red, $2x$ are white and $3x$ are blue. A ball is selected at random, What is the probability that it is
\begin{enumerate}[label=\alph*)]
\item not red ?
\item white ?
\end{enumerate}
%\input{exemplar/10/13/3/41/main.tex}
If the letters of the word ASSASSINATION are arranged at random. Find the Probability that
\begin{enumerate}[label=(\alph*)]
\item Four $S's$ come consecutively in the word
\item Two  $I's$ and two $N's$ come together
\item All $A's$ are not coming together
\item No two $A's$ are coming together
\end{enumerate}
%\input{exemplar/11/16/3/14/main.tex}
	\item One urn contains two black balls (labelled B1 and B2) and one white ball. A
	second urn contains one black ball and two white balls (labelled W1 and W2).
	Suppose the following experiment is performed. One of the two urns is chosen
	at random. Next a ball is randomly chosen from the urn. Then a second ball is
	chosen at random from the same urn without replacing the first ball.
	
	\begin{enumerate}
	\item What is the probability that two black balls are chosen?
	
	\item What is the probability that two balls of opposite colour are chosen?
	\end{enumerate}
	\solution
	%\input{exemplar/11/16/3/12/main1.tex}
\end{enumerate}

		\item A box of oranges is inspected by examining three randomly selected oranges drawn without replacement. If all the three oranges are good, the box is approved for sale, otherwise, it is rejected. Find the probability that a box containing 15 oranges out of which 12 are good and 3 are bad ones will be approved for sale.
		\label{ncert/12/13/2/3/defs.tex}
		\item Two balls are drawn at random with replacement from a box containing 10 black and 8 red balls. Find the probability that
		\label{ncert/12/13/2/12}
\begin{enumerate}
\item both balls are red.
\item first ball is black and second is red.
\item one of them is black and other is red.
\end{enumerate}

\item In a hostel, 60\% of the students read Hindi newspaper, 40\% read English newspaper and 20\% read both Hindi and English newspapers. A student is selected at random.
		\label{ncert/12/13/2/15}
\begin{enumerate}
\item Find the probability that she reads neither Hindi nor English newspapers.
\item If she reads Hindi newspaper, find the probability that she reads English newspaper.
\item If she reads English newspaper, find the probability that she reads Hindi newspaper.\\
\end{enumerate}
\item The probability of obtaining an even prime number on each die, when a pair of dice is rolled is 
\begin{enumerate}
    \item $0$ 
    
    \item $\frac{1}{3}$ 
    
    \item $\frac{1}{12}$ 
    
    \item $\frac{1}{36}$ 
\end{enumerate}
\solution
		%\begin{enumerate}[label=\thesection.\arabic*,ref=\thesection.\theenumi]
	\item One card is drawn from a well-shuffled deck of 52 cards. Find the probability of getting
\begin{enumerate}
\item A king of red colour 
\item A face card 
\item A red face card
\item The jack of hearts
\item A spade
\item The queen of diamonds

\end{enumerate}
\solution
		%\input{ncert/10/15/1/14/main.tex}
	\item Five cards—the ten, jack, queen, king and ace of diamonds, are well-shuffled with their face downwards. One card is then picked up at random.
\begin{enumerate}
\item
What is the probability that the card is the queen? 
\item
If the queen is drawn and put aside, what is the probability that the second card picked up is (a) an ace? (b) a queen?\\
\end{enumerate}
\solution
		%\input{ncert/10/15/1/15/defs.tex}
	\item A bag contains $5$ red balls and some blue balls. If the probability of drawing a blue ball is double that if a red ball, determine the number of blue balls in the bag. 
		\\
\solution
		%\input{ncert/10/15/2/3/defs.tex}
	\item A card is selected from a pack of 52 cards.
 \begin{enumerate}[label=(\alph*)] 
                 \item How many points are there in the sample space?
                 \item Calculate the probability that the card is an ace of spades.
                 \item Calculate the probability that the card is (i) an ace and (ii) black card.
 \end{enumerate}
\solution
		%\input{ncert/11/16/3/4/main.tex}
\item Four cards are drawn from a well-shuffled deck of 52 cards. What is the probability of obtaining 3 diamonds and one spade.
\\
\solution
		%\input{ncert/11/16/4/2/defs.tex}
\item In a certain lottery 10,000 tickets are sold and ten equal prizes are awarded. What is the probability of not getting a prize if you buy (a) one ticket (b) two tickets (c) 10 tickets ?	
\\
\solution
		%\input{ncert/11/16/4/4/defs.tex}
		%
\item 
Out of 100 students, two sections of 40 and 60 are formed. If you and your friend are among the 100 students, what is the probability that
\begin{enumerate}
\item you both enter the same section?
\item you both enter the different sections?
\end{enumerate}
\solution
		%\input{ncert/11/16/4/5/defs.tex}
	\item 
The number lock of a suitcase has 4 wheels each labelled with ten digits i.e. from 0 to 9.The lock opens with a sequence of four digits with no repeats.What is the probability of a person getting the right sequence to open the suitcase.
\\
\solution
		%\input{ncert/11/16/4/10/defs.tex}
		%
\item 
Two cards are drawn at random and without replacement from a pack of 52 playing cards. Find the probability that both the cards are black.
\\
\solution
		%\input{ncert/12/13/2/2/defs.tex}
		\item A box of oranges is inspected by examining three randomly selected oranges drawn without replacement. If all the three oranges are good, the box is approved for sale, otherwise, it is rejected. Find the probability that a box containing 15 oranges out of which 12 are good and 3 are bad ones will be approved for sale.
		\label{ncert/12/13/2/3/defs.tex}
		\item Two balls are drawn at random with replacement from a box containing 10 black and 8 red balls. Find the probability that
		\label{ncert/12/13/2/12}
\begin{enumerate}
\item both balls are red.
\item first ball is black and second is red.
\item one of them is black and other is red.
\end{enumerate}

\item In a hostel, 60\% of the students read Hindi newspaper, 40\% read English newspaper and 20\% read both Hindi and English newspapers. A student is selected at random.
		\label{ncert/12/13/2/15}
\begin{enumerate}
\item Find the probability that she reads neither Hindi nor English newspapers.
\item If she reads Hindi newspaper, find the probability that she reads English newspaper.
\item If she reads English newspaper, find the probability that she reads Hindi newspaper.\\
\end{enumerate}
\item The probability of obtaining an even prime number on each die, when a pair of dice is rolled is 
\begin{enumerate}
    \item $0$ 
    
    \item $\frac{1}{3}$ 
    
    \item $\frac{1}{12}$ 
    
    \item $\frac{1}{36}$ 
\end{enumerate}
\solution
		%\input{ncert/12/13/2/17/defs.tex}
	\item A bag contains 4 red and 4 black balls, another bag contains 2 red and 6 black balls. One of the two bags is selected at random and a ball is drawn from the bag which is found to be red. Find the probability that the ball is drawn from the first bag.
\\
\solution
		%\input{ncert/12/13/3/2/main.tex}
  \item
  Cards with numbers 2 to 101 are placed in a box. A card is selected at random.Find the probability that the card has
\begin{enumerate}[label=(\roman*)]
	\item an even number 
	\item a square number
\end{enumerate}
\solution
%\input{exemplar/10/13/3/32/main.tex}
\item
The king, queen and jack of clubs are removed from a deck of 52 playing cards and then well shuffled. Now one card is drawn at random from the remaining cards.  Determine the probability that the card is
\begin{enumerate}[label=(\roman*)]
\item a club
\item 10 of hearts
\end{enumerate}
\solution
%\input{exemplar/10/13/3/29/main.tex}
\item A team of medical students doing their internship have to assist during surgeries
at a city hospital. The probabilities of surgeries rated as very complex, complex,
routine, simple or very simple are respectively, 0.15, 0.20, 0.31, 0.26, .08. Find
the probabilities that a particular surgery will be rated
\begin{enumerate}
	\item complex or very complex;
	\item neither very complex nor very simple;
	\item routine or complex
	\item routine or simple
\end{enumerate}
\solution
%\input{exemplar/11/16/3/8(1)/main.tex}
\item A card is selected from a pack of 52 cards.
\begin{enumerate}[label=(\alph*)]
    \item How many points are there in the sample space?
    \item Calculate the probability that the card is an ace of spades.
    \item Calculate the probability that the card is (i) an ace and (ii) black card.
\end{enumerate}
\solution
%\input{exemplar/11/16/3/4/main2.tex}
\item The probability that a non leap year selected at random will contain 53 sundays.
\\
\solution
%\input{exemplar/10/13/1/19/main.tex}
\item One of the four persons John, Rita, Aslam or Gurpreet will be promoted next
month. Consequently the sample space consists of four elementary outcomes
S = {John promoted, Rita promoted, Aslam promoted, Gurpreet promoted}
You are told that the chances of John’s promotion is same as that of Gurpreet,
Rita’s chances of promotion are twice as likely as Johns. Aslam’s chances are
four times that of John.
\begin{enumerate}
	\item Determine
	\begin{enumerate}
		\item P (John promoted)
		\item P (Rita promoted)
		\item P (Aslam promoted)
		\item P (Gurpreet promoted)
	\end{enumerate}
	\item If A = {John promoted or Gurpreet promoted}, find P (A).
\end{enumerate}
\solution
%\input{exemplar/11/16/3/10/main.tex}
\item A card is drawn from a deck of 52 cards. Find the probability of getting a king or a heart or a red card.\\
\solution
%\input{exemplar/11/16/3/15/main.tex}
\item The probability that a student will pass his examination is 0.73, the probability of
the student getting a compartment is 0.13, and the probability that the student will
either pass or get compartment is 0.96. State True or False.\\
\solution
%\input{exemplar/11/16/3/31/main.tex}
\item A card is selected from a pack of 52 cards\\
\begin{enumerate}[label=(\alph*)]
\item How many points are there in the sample space?
\item Calculate the probability that the cards is an ace of spades.
\item Calculate the probability that the card is (i) an ace (ii)black card.\\
\end{enumerate}
%\input{ncert/11/16/3/4_1/Prob_4.tex}
\item In a non-leap year, the probability of having 53 tuesdays or 53 wednesdays is\\
\solution
%\input{exemplar/11/16/3/18/main.tex}
\item There are 1000 sealed envelopes in a box, 10 of them contain a cash prize of
Rs 100 each, 100 of them contain a cash prize of Rs 50 each and 200 of them
contain a cash prize of Rs 10 each and rest do not contain any cash prize. If they
are well shuffled and an envelope is picked up out, what is the probability that it
contains no cash prize?\\
\solution
%\input{exemplar/10/13/3/34/main.tex}
\item 
A die is thrown and a card is selected at random from a deck of 52 playing cards. The probability of getting an even number on the die and a spade card.\\
\solution
%\input{exemplar/12/13/3/78/main.tex}
\item
If 4-digit numbers greater than 5,000 are randomly formed from the digits 0, 1, 3, 5, and 7, what is the probability of forming a number divisible by 5 when:
\begin{enumerate}
    \item The digits are repeated?
    \item The repetition of digits is not allowed?
\end{enumerate}
\solution
%\input{ncert/11/16/4/9/main.tex}
\item Consider the probability space $\brak{\Omega, \mathcal{G}, P}$ where $\Omega = [0,2]$ and $\mathcal{G} = \cbrak{\phi, \Omega, [0,1], (1,2]}$. Let $X$ and $Y$ be two functions on $\Omega$ defined as
\begin{align*}
    X(\omega) = 
    \begin{cases}
        1 & \text{if }\omega \in [0, 1]\\
        2 & \text{if }\omega \in (1, 2]
    \end{cases}
\end{align*}
and
\begin{align*}
    Y(\omega) = 
    \begin{cases}
        2 & \text{if }\omega \in [0, 1.5]\\
        3 & \text{if }\omega \in (1.5, 2].
    \end{cases}
\end{align*}
Then which one of the following statements is true?
\begin{enumerate}
    \item [(A)] $X$ is a random variable with respect to $\mathcal{G}$, but $Y$ is not a random variable with respect to $\mathcal{G}$.
    \item [(B)] $Y$ is a random variable with respect to $\mathcal{G}$, but $X$ is not a random variable with respect to $\mathcal{G}$.
    \item [(C)] Neither $X$ nor $Y$ is a random variable with respect to $\mathcal{G}$.
    \item [(D)] Both $X$ and $Y$ are random variables with respect to $\mathcal{G}$.
\end{enumerate} \hfill (GATE ST 2023)\\
\solution
%\input{gate/ST/2023/14/main.tex}
	\item  A die is loaded in such a way that each odd number is twice as likely to occur as
each even number. Find $P(G)$, where $G$ is the event that a number greater than
3 occurs on a single roll of the die.
\\
\solution
		%\input{exemplar/11/16/3/5/main.tex}
	\item All the jacks, queens and kings are removed from a deck of 52 playing cards. The remaining cards are well shuffled and then one card is drawn at random. Giving ace a value 1 similar value for other cards, find the probability that the card has a value 
		\begin{enumerate}
			\item 7
			\item greater than 7
			\item less than 7
		\end{enumerate}
		%\input{exemplar/10/13/3/30/main.tex}
  \item A Lot consists of 48 mobile phones of which 42 are good, 3 have only minor defects and 3 have major defects.Varnika will buy a phone if it is good but the trader will only buy a mobile if it has no major defects. One phone is selected at random from the lot. What is the probability that it is
\begin{enumerate}
	\item acceptable to Varnika?
            \item acceptable to the trader?
\end{enumerate}
\solution
	%\input{exemplar/10/13/3/40/main.tex}
 \item A student says that if you throw a die, it will show up 1 or not 1. Therefore, the probability of getting 1 and the probability of getting 'not 1' each is equal to $\frac{1}{2}$. Is this correct? Give reasons.\\
 \solution
        %\input{exemplar/10/13/2/9/main.tex}
   \item Four candidates A, B, C, D have ap-
plied for the assignment to coach a school cricket
team. If A is twice as likely to be selected as B, and
B and C are given about the same chance of being
selected, while C is twice as likely to be selected
as D, what are the probabilities that
\begin{enumerate}
\item C will be selected?
\item A will not be selected?
\end{enumerate}
	%\input{exemplar/11/16/3/9/main.tex}
 \item A bag contain 24 balls of which $x$ balls are red, $2x$ are white and $3x$ are blue. A ball is selected at random, What is the probability that it is
\begin{enumerate}[label=\alph*)]
\item not red ?
\item white ?
\end{enumerate}
%\input{exemplar/10/13/3/41/main.tex}
If the letters of the word ASSASSINATION are arranged at random. Find the Probability that
\begin{enumerate}[label=(\alph*)]
\item Four $S's$ come consecutively in the word
\item Two  $I's$ and two $N's$ come together
\item All $A's$ are not coming together
\item No two $A's$ are coming together
\end{enumerate}
%\input{exemplar/11/16/3/14/main.tex}
	\item One urn contains two black balls (labelled B1 and B2) and one white ball. A
	second urn contains one black ball and two white balls (labelled W1 and W2).
	Suppose the following experiment is performed. One of the two urns is chosen
	at random. Next a ball is randomly chosen from the urn. Then a second ball is
	chosen at random from the same urn without replacing the first ball.
	
	\begin{enumerate}
	\item What is the probability that two black balls are chosen?
	
	\item What is the probability that two balls of opposite colour are chosen?
	\end{enumerate}
	\solution
	%\input{exemplar/11/16/3/12/main1.tex}
\end{enumerate}

	\item A bag contains 4 red and 4 black balls, another bag contains 2 red and 6 black balls. One of the two bags is selected at random and a ball is drawn from the bag which is found to be red. Find the probability that the ball is drawn from the first bag.
\\
\solution
		%\begin{table}[H]
	\centering
\begin{tabular}{|c|c|c|}
\hline
Random variable &Value &Definition\\ \hline
\multirow{3}{*}{X} &0 &Slips of Rs 1\\
&1 &Slips of Rs 5\\
&2 &Slips of Rs 13\\ \hline
\multirow{2}{*}{Y} &0 &Box A\\
&1 &Box B\\\hline
\end{tabular}
\caption{}
\label{tab:Distribution}
\end{table}
See \tabref{tab:Distribution}.
\begin{align}
p_{Y}\brak{k}= \begin{cases} 
      \frac{1}{3} & {k=0} \\
      \frac{2}{3 }& {k=1} 
   \end{cases}
   \\
p_{Y|X}\brak{0|0} = \frac{19}{25}\, 
p_{Y|X}\brak{0|1} = \frac{6}{25}\,
p_{Y|X}\brak{1|0} = \frac{45}{50}\,
p_{Y|X}\brak{1|2} = \frac{5}{50}
\end{align}
The desired probability is the probability that a slip drawn at random is marked other than Rs 1,
\begin{align}
&=1-p_X\brak{0}\\
&= p_X(1) + p_X(2)
\end{align}
Using Bayes theorem,
\begin{align}
&= p_Y\brak{0} \times \pr{Y=0 | X=1} + p_Y\brak{1} \times \pr{Y=1|X=2}\\
&=\frac{1}{3} \times \frac{6}{25} + \frac{2}{3} \times \frac{5}{50}\\
&=\frac{11}{75}
\end{align}

\newpage

%\tableofcontents

\bigskip

\renewcommand{\thefigure}{\theenumi}
\renewcommand{\thetable}{\theenumi}
%\renewcommand{\theequation}{\theenumi}

%\begin{abstract}
%%\boldmath
%In this letter, an algorithm for evaluating the exact analytical bit error rate  (BER)  for the piecewise linear (PL) combiner for  multiple relays is presented. Previous results were available only for upto three relays. The algorithm is unique in the sense that  the actual mathematical expressions, that are prohibitively large, need not be explicitly obtained. The diversity gain due to multiple relays is shown through plots of the analytical BER, well supported by simulations. 
%
%\end{abstract}
% IEEEtran.cls defaults to using nonbold math in the Abstract.
% This preserves the distinction between vectors and scalars. However,
% if the journal you are submitting to favors bold math in the abstract,
% then you can use LaTeX's standard command \boldmath at the very start
% of the abstract to achieve this. Many IEEE journals frown on math
% in the abstract anyway.

% Note that keywords are not normally used for peerreview papers.
%\begin{IEEEkeywords}
%Cooperative diversity, decode and forward, piecewise linear
%\end{IEEEkeywords}



% For peer review papers, you can put extra information on the cover
% page as needed:
% \ifCLASSOPTIONpeerreview
% \begin{center} \bfseries EDICS Category: 3-BBND \end{center}
% \fi
%
% For peerreview papers, this IEEEtran command inserts a page break and
% creates the second title. It will be ignored for other modes.
%\IEEEpeerreviewmaketitle




  \item
  Cards with numbers 2 to 101 are placed in a box. A card is selected at random.Find the probability that the card has
\begin{enumerate}[label=(\roman*)]
	\item an even number 
	\item a square number
\end{enumerate}
\solution
%\begin{table}[H]
	\centering
\begin{tabular}{|c|c|c|}
\hline
Random variable &Value &Definition\\ \hline
\multirow{3}{*}{X} &0 &Slips of Rs 1\\
&1 &Slips of Rs 5\\
&2 &Slips of Rs 13\\ \hline
\multirow{2}{*}{Y} &0 &Box A\\
&1 &Box B\\\hline
\end{tabular}
\caption{}
\label{tab:Distribution}
\end{table}
See \tabref{tab:Distribution}.
\begin{align}
p_{Y}\brak{k}= \begin{cases} 
      \frac{1}{3} & {k=0} \\
      \frac{2}{3 }& {k=1} 
   \end{cases}
   \\
p_{Y|X}\brak{0|0} = \frac{19}{25}\, 
p_{Y|X}\brak{0|1} = \frac{6}{25}\,
p_{Y|X}\brak{1|0} = \frac{45}{50}\,
p_{Y|X}\brak{1|2} = \frac{5}{50}
\end{align}
The desired probability is the probability that a slip drawn at random is marked other than Rs 1,
\begin{align}
&=1-p_X\brak{0}\\
&= p_X(1) + p_X(2)
\end{align}
Using Bayes theorem,
\begin{align}
&= p_Y\brak{0} \times \pr{Y=0 | X=1} + p_Y\brak{1} \times \pr{Y=1|X=2}\\
&=\frac{1}{3} \times \frac{6}{25} + \frac{2}{3} \times \frac{5}{50}\\
&=\frac{11}{75}
\end{align}

\newpage

%\tableofcontents

\bigskip

\renewcommand{\thefigure}{\theenumi}
\renewcommand{\thetable}{\theenumi}
%\renewcommand{\theequation}{\theenumi}

%\begin{abstract}
%%\boldmath
%In this letter, an algorithm for evaluating the exact analytical bit error rate  (BER)  for the piecewise linear (PL) combiner for  multiple relays is presented. Previous results were available only for upto three relays. The algorithm is unique in the sense that  the actual mathematical expressions, that are prohibitively large, need not be explicitly obtained. The diversity gain due to multiple relays is shown through plots of the analytical BER, well supported by simulations. 
%
%\end{abstract}
% IEEEtran.cls defaults to using nonbold math in the Abstract.
% This preserves the distinction between vectors and scalars. However,
% if the journal you are submitting to favors bold math in the abstract,
% then you can use LaTeX's standard command \boldmath at the very start
% of the abstract to achieve this. Many IEEE journals frown on math
% in the abstract anyway.

% Note that keywords are not normally used for peerreview papers.
%\begin{IEEEkeywords}
%Cooperative diversity, decode and forward, piecewise linear
%\end{IEEEkeywords}



% For peer review papers, you can put extra information on the cover
% page as needed:
% \ifCLASSOPTIONpeerreview
% \begin{center} \bfseries EDICS Category: 3-BBND \end{center}
% \fi
%
% For peerreview papers, this IEEEtran command inserts a page break and
% creates the second title. It will be ignored for other modes.
%\IEEEpeerreviewmaketitle




\item
The king, queen and jack of clubs are removed from a deck of 52 playing cards and then well shuffled. Now one card is drawn at random from the remaining cards.  Determine the probability that the card is
\begin{enumerate}[label=(\roman*)]
\item a club
\item 10 of hearts
\end{enumerate}
\solution
%\begin{table}[H]
	\centering
\begin{tabular}{|c|c|c|}
\hline
Random variable &Value &Definition\\ \hline
\multirow{3}{*}{X} &0 &Slips of Rs 1\\
&1 &Slips of Rs 5\\
&2 &Slips of Rs 13\\ \hline
\multirow{2}{*}{Y} &0 &Box A\\
&1 &Box B\\\hline
\end{tabular}
\caption{}
\label{tab:Distribution}
\end{table}
See \tabref{tab:Distribution}.
\begin{align}
p_{Y}\brak{k}= \begin{cases} 
      \frac{1}{3} & {k=0} \\
      \frac{2}{3 }& {k=1} 
   \end{cases}
   \\
p_{Y|X}\brak{0|0} = \frac{19}{25}\, 
p_{Y|X}\brak{0|1} = \frac{6}{25}\,
p_{Y|X}\brak{1|0} = \frac{45}{50}\,
p_{Y|X}\brak{1|2} = \frac{5}{50}
\end{align}
The desired probability is the probability that a slip drawn at random is marked other than Rs 1,
\begin{align}
&=1-p_X\brak{0}\\
&= p_X(1) + p_X(2)
\end{align}
Using Bayes theorem,
\begin{align}
&= p_Y\brak{0} \times \pr{Y=0 | X=1} + p_Y\brak{1} \times \pr{Y=1|X=2}\\
&=\frac{1}{3} \times \frac{6}{25} + \frac{2}{3} \times \frac{5}{50}\\
&=\frac{11}{75}
\end{align}

\newpage

%\tableofcontents

\bigskip

\renewcommand{\thefigure}{\theenumi}
\renewcommand{\thetable}{\theenumi}
%\renewcommand{\theequation}{\theenumi}

%\begin{abstract}
%%\boldmath
%In this letter, an algorithm for evaluating the exact analytical bit error rate  (BER)  for the piecewise linear (PL) combiner for  multiple relays is presented. Previous results were available only for upto three relays. The algorithm is unique in the sense that  the actual mathematical expressions, that are prohibitively large, need not be explicitly obtained. The diversity gain due to multiple relays is shown through plots of the analytical BER, well supported by simulations. 
%
%\end{abstract}
% IEEEtran.cls defaults to using nonbold math in the Abstract.
% This preserves the distinction between vectors and scalars. However,
% if the journal you are submitting to favors bold math in the abstract,
% then you can use LaTeX's standard command \boldmath at the very start
% of the abstract to achieve this. Many IEEE journals frown on math
% in the abstract anyway.

% Note that keywords are not normally used for peerreview papers.
%\begin{IEEEkeywords}
%Cooperative diversity, decode and forward, piecewise linear
%\end{IEEEkeywords}



% For peer review papers, you can put extra information on the cover
% page as needed:
% \ifCLASSOPTIONpeerreview
% \begin{center} \bfseries EDICS Category: 3-BBND \end{center}
% \fi
%
% For peerreview papers, this IEEEtran command inserts a page break and
% creates the second title. It will be ignored for other modes.
%\IEEEpeerreviewmaketitle




\item A team of medical students doing their internship have to assist during surgeries
at a city hospital. The probabilities of surgeries rated as very complex, complex,
routine, simple or very simple are respectively, 0.15, 0.20, 0.31, 0.26, .08. Find
the probabilities that a particular surgery will be rated
\begin{enumerate}
	\item complex or very complex;
	\item neither very complex nor very simple;
	\item routine or complex
	\item routine or simple
\end{enumerate}
\solution
%\begin{table}[H]
	\centering
\begin{tabular}{|c|c|c|}
\hline
Random variable &Value &Definition\\ \hline
\multirow{3}{*}{X} &0 &Slips of Rs 1\\
&1 &Slips of Rs 5\\
&2 &Slips of Rs 13\\ \hline
\multirow{2}{*}{Y} &0 &Box A\\
&1 &Box B\\\hline
\end{tabular}
\caption{}
\label{tab:Distribution}
\end{table}
See \tabref{tab:Distribution}.
\begin{align}
p_{Y}\brak{k}= \begin{cases} 
      \frac{1}{3} & {k=0} \\
      \frac{2}{3 }& {k=1} 
   \end{cases}
   \\
p_{Y|X}\brak{0|0} = \frac{19}{25}\, 
p_{Y|X}\brak{0|1} = \frac{6}{25}\,
p_{Y|X}\brak{1|0} = \frac{45}{50}\,
p_{Y|X}\brak{1|2} = \frac{5}{50}
\end{align}
The desired probability is the probability that a slip drawn at random is marked other than Rs 1,
\begin{align}
&=1-p_X\brak{0}\\
&= p_X(1) + p_X(2)
\end{align}
Using Bayes theorem,
\begin{align}
&= p_Y\brak{0} \times \pr{Y=0 | X=1} + p_Y\brak{1} \times \pr{Y=1|X=2}\\
&=\frac{1}{3} \times \frac{6}{25} + \frac{2}{3} \times \frac{5}{50}\\
&=\frac{11}{75}
\end{align}

\newpage

%\tableofcontents

\bigskip

\renewcommand{\thefigure}{\theenumi}
\renewcommand{\thetable}{\theenumi}
%\renewcommand{\theequation}{\theenumi}

%\begin{abstract}
%%\boldmath
%In this letter, an algorithm for evaluating the exact analytical bit error rate  (BER)  for the piecewise linear (PL) combiner for  multiple relays is presented. Previous results were available only for upto three relays. The algorithm is unique in the sense that  the actual mathematical expressions, that are prohibitively large, need not be explicitly obtained. The diversity gain due to multiple relays is shown through plots of the analytical BER, well supported by simulations. 
%
%\end{abstract}
% IEEEtran.cls defaults to using nonbold math in the Abstract.
% This preserves the distinction between vectors and scalars. However,
% if the journal you are submitting to favors bold math in the abstract,
% then you can use LaTeX's standard command \boldmath at the very start
% of the abstract to achieve this. Many IEEE journals frown on math
% in the abstract anyway.

% Note that keywords are not normally used for peerreview papers.
%\begin{IEEEkeywords}
%Cooperative diversity, decode and forward, piecewise linear
%\end{IEEEkeywords}



% For peer review papers, you can put extra information on the cover
% page as needed:
% \ifCLASSOPTIONpeerreview
% \begin{center} \bfseries EDICS Category: 3-BBND \end{center}
% \fi
%
% For peerreview papers, this IEEEtran command inserts a page break and
% creates the second title. It will be ignored for other modes.
%\IEEEpeerreviewmaketitle




\item A card is selected from a pack of 52 cards.
\begin{enumerate}[label=(\alph*)]
    \item How many points are there in the sample space?
    \item Calculate the probability that the card is an ace of spades.
    \item Calculate the probability that the card is (i) an ace and (ii) black card.
\end{enumerate}
\solution
%Let $X$ be an bernoulli rv defined as in \tabref{tab:exemplar/11/16/3/26}.  Then, 
\begin{equation}
    p =
        \frac{4}{11} 
\end{equation}
\begin{table}[H]
	\centering
	\input{exemplar/11/16/3/26/tables/Table2.tex}
	\caption{}
        \label{tab:exemplar/11/16/3/26}
\end{table}

\item The probability that a non leap year selected at random will contain 53 sundays.
\\
\solution
%\begin{table}[H]
	\centering
\begin{tabular}{|c|c|c|}
\hline
Random variable &Value &Definition\\ \hline
\multirow{3}{*}{X} &0 &Slips of Rs 1\\
&1 &Slips of Rs 5\\
&2 &Slips of Rs 13\\ \hline
\multirow{2}{*}{Y} &0 &Box A\\
&1 &Box B\\\hline
\end{tabular}
\caption{}
\label{tab:Distribution}
\end{table}
See \tabref{tab:Distribution}.
\begin{align}
p_{Y}\brak{k}= \begin{cases} 
      \frac{1}{3} & {k=0} \\
      \frac{2}{3 }& {k=1} 
   \end{cases}
   \\
p_{Y|X}\brak{0|0} = \frac{19}{25}\, 
p_{Y|X}\brak{0|1} = \frac{6}{25}\,
p_{Y|X}\brak{1|0} = \frac{45}{50}\,
p_{Y|X}\brak{1|2} = \frac{5}{50}
\end{align}
The desired probability is the probability that a slip drawn at random is marked other than Rs 1,
\begin{align}
&=1-p_X\brak{0}\\
&= p_X(1) + p_X(2)
\end{align}
Using Bayes theorem,
\begin{align}
&= p_Y\brak{0} \times \pr{Y=0 | X=1} + p_Y\brak{1} \times \pr{Y=1|X=2}\\
&=\frac{1}{3} \times \frac{6}{25} + \frac{2}{3} \times \frac{5}{50}\\
&=\frac{11}{75}
\end{align}

\newpage

%\tableofcontents

\bigskip

\renewcommand{\thefigure}{\theenumi}
\renewcommand{\thetable}{\theenumi}
%\renewcommand{\theequation}{\theenumi}

%\begin{abstract}
%%\boldmath
%In this letter, an algorithm for evaluating the exact analytical bit error rate  (BER)  for the piecewise linear (PL) combiner for  multiple relays is presented. Previous results were available only for upto three relays. The algorithm is unique in the sense that  the actual mathematical expressions, that are prohibitively large, need not be explicitly obtained. The diversity gain due to multiple relays is shown through plots of the analytical BER, well supported by simulations. 
%
%\end{abstract}
% IEEEtran.cls defaults to using nonbold math in the Abstract.
% This preserves the distinction between vectors and scalars. However,
% if the journal you are submitting to favors bold math in the abstract,
% then you can use LaTeX's standard command \boldmath at the very start
% of the abstract to achieve this. Many IEEE journals frown on math
% in the abstract anyway.

% Note that keywords are not normally used for peerreview papers.
%\begin{IEEEkeywords}
%Cooperative diversity, decode and forward, piecewise linear
%\end{IEEEkeywords}



% For peer review papers, you can put extra information on the cover
% page as needed:
% \ifCLASSOPTIONpeerreview
% \begin{center} \bfseries EDICS Category: 3-BBND \end{center}
% \fi
%
% For peerreview papers, this IEEEtran command inserts a page break and
% creates the second title. It will be ignored for other modes.
%\IEEEpeerreviewmaketitle




\item One of the four persons John, Rita, Aslam or Gurpreet will be promoted next
month. Consequently the sample space consists of four elementary outcomes
S = {John promoted, Rita promoted, Aslam promoted, Gurpreet promoted}
You are told that the chances of John’s promotion is same as that of Gurpreet,
Rita’s chances of promotion are twice as likely as Johns. Aslam’s chances are
four times that of John.
\begin{enumerate}
	\item Determine
	\begin{enumerate}
		\item P (John promoted)
		\item P (Rita promoted)
		\item P (Aslam promoted)
		\item P (Gurpreet promoted)
	\end{enumerate}
	\item If A = {John promoted or Gurpreet promoted}, find P (A).
\end{enumerate}
\solution
%\begin{table}[H]
	\centering
\begin{tabular}{|c|c|c|}
\hline
Random variable &Value &Definition\\ \hline
\multirow{3}{*}{X} &0 &Slips of Rs 1\\
&1 &Slips of Rs 5\\
&2 &Slips of Rs 13\\ \hline
\multirow{2}{*}{Y} &0 &Box A\\
&1 &Box B\\\hline
\end{tabular}
\caption{}
\label{tab:Distribution}
\end{table}
See \tabref{tab:Distribution}.
\begin{align}
p_{Y}\brak{k}= \begin{cases} 
      \frac{1}{3} & {k=0} \\
      \frac{2}{3 }& {k=1} 
   \end{cases}
   \\
p_{Y|X}\brak{0|0} = \frac{19}{25}\, 
p_{Y|X}\brak{0|1} = \frac{6}{25}\,
p_{Y|X}\brak{1|0} = \frac{45}{50}\,
p_{Y|X}\brak{1|2} = \frac{5}{50}
\end{align}
The desired probability is the probability that a slip drawn at random is marked other than Rs 1,
\begin{align}
&=1-p_X\brak{0}\\
&= p_X(1) + p_X(2)
\end{align}
Using Bayes theorem,
\begin{align}
&= p_Y\brak{0} \times \pr{Y=0 | X=1} + p_Y\brak{1} \times \pr{Y=1|X=2}\\
&=\frac{1}{3} \times \frac{6}{25} + \frac{2}{3} \times \frac{5}{50}\\
&=\frac{11}{75}
\end{align}

\newpage

%\tableofcontents

\bigskip

\renewcommand{\thefigure}{\theenumi}
\renewcommand{\thetable}{\theenumi}
%\renewcommand{\theequation}{\theenumi}

%\begin{abstract}
%%\boldmath
%In this letter, an algorithm for evaluating the exact analytical bit error rate  (BER)  for the piecewise linear (PL) combiner for  multiple relays is presented. Previous results were available only for upto three relays. The algorithm is unique in the sense that  the actual mathematical expressions, that are prohibitively large, need not be explicitly obtained. The diversity gain due to multiple relays is shown through plots of the analytical BER, well supported by simulations. 
%
%\end{abstract}
% IEEEtran.cls defaults to using nonbold math in the Abstract.
% This preserves the distinction between vectors and scalars. However,
% if the journal you are submitting to favors bold math in the abstract,
% then you can use LaTeX's standard command \boldmath at the very start
% of the abstract to achieve this. Many IEEE journals frown on math
% in the abstract anyway.

% Note that keywords are not normally used for peerreview papers.
%\begin{IEEEkeywords}
%Cooperative diversity, decode and forward, piecewise linear
%\end{IEEEkeywords}



% For peer review papers, you can put extra information on the cover
% page as needed:
% \ifCLASSOPTIONpeerreview
% \begin{center} \bfseries EDICS Category: 3-BBND \end{center}
% \fi
%
% For peerreview papers, this IEEEtran command inserts a page break and
% creates the second title. It will be ignored for other modes.
%\IEEEpeerreviewmaketitle




\item A card is drawn from a deck of 52 cards. Find the probability of getting a king or a heart or a red card.\\
\solution
%\begin{table}[H]
	\centering
\begin{tabular}{|c|c|c|}
\hline
Random variable &Value &Definition\\ \hline
\multirow{3}{*}{X} &0 &Slips of Rs 1\\
&1 &Slips of Rs 5\\
&2 &Slips of Rs 13\\ \hline
\multirow{2}{*}{Y} &0 &Box A\\
&1 &Box B\\\hline
\end{tabular}
\caption{}
\label{tab:Distribution}
\end{table}
See \tabref{tab:Distribution}.
\begin{align}
p_{Y}\brak{k}= \begin{cases} 
      \frac{1}{3} & {k=0} \\
      \frac{2}{3 }& {k=1} 
   \end{cases}
   \\
p_{Y|X}\brak{0|0} = \frac{19}{25}\, 
p_{Y|X}\brak{0|1} = \frac{6}{25}\,
p_{Y|X}\brak{1|0} = \frac{45}{50}\,
p_{Y|X}\brak{1|2} = \frac{5}{50}
\end{align}
The desired probability is the probability that a slip drawn at random is marked other than Rs 1,
\begin{align}
&=1-p_X\brak{0}\\
&= p_X(1) + p_X(2)
\end{align}
Using Bayes theorem,
\begin{align}
&= p_Y\brak{0} \times \pr{Y=0 | X=1} + p_Y\brak{1} \times \pr{Y=1|X=2}\\
&=\frac{1}{3} \times \frac{6}{25} + \frac{2}{3} \times \frac{5}{50}\\
&=\frac{11}{75}
\end{align}

\newpage

%\tableofcontents

\bigskip

\renewcommand{\thefigure}{\theenumi}
\renewcommand{\thetable}{\theenumi}
%\renewcommand{\theequation}{\theenumi}

%\begin{abstract}
%%\boldmath
%In this letter, an algorithm for evaluating the exact analytical bit error rate  (BER)  for the piecewise linear (PL) combiner for  multiple relays is presented. Previous results were available only for upto three relays. The algorithm is unique in the sense that  the actual mathematical expressions, that are prohibitively large, need not be explicitly obtained. The diversity gain due to multiple relays is shown through plots of the analytical BER, well supported by simulations. 
%
%\end{abstract}
% IEEEtran.cls defaults to using nonbold math in the Abstract.
% This preserves the distinction between vectors and scalars. However,
% if the journal you are submitting to favors bold math in the abstract,
% then you can use LaTeX's standard command \boldmath at the very start
% of the abstract to achieve this. Many IEEE journals frown on math
% in the abstract anyway.

% Note that keywords are not normally used for peerreview papers.
%\begin{IEEEkeywords}
%Cooperative diversity, decode and forward, piecewise linear
%\end{IEEEkeywords}



% For peer review papers, you can put extra information on the cover
% page as needed:
% \ifCLASSOPTIONpeerreview
% \begin{center} \bfseries EDICS Category: 3-BBND \end{center}
% \fi
%
% For peerreview papers, this IEEEtran command inserts a page break and
% creates the second title. It will be ignored for other modes.
%\IEEEpeerreviewmaketitle




\item The probability that a student will pass his examination is 0.73, the probability of
the student getting a compartment is 0.13, and the probability that the student will
either pass or get compartment is 0.96. State True or False.\\
\solution
%\begin{table}[H]
	\centering
\begin{tabular}{|c|c|c|}
\hline
Random variable &Value &Definition\\ \hline
\multirow{3}{*}{X} &0 &Slips of Rs 1\\
&1 &Slips of Rs 5\\
&2 &Slips of Rs 13\\ \hline
\multirow{2}{*}{Y} &0 &Box A\\
&1 &Box B\\\hline
\end{tabular}
\caption{}
\label{tab:Distribution}
\end{table}
See \tabref{tab:Distribution}.
\begin{align}
p_{Y}\brak{k}= \begin{cases} 
      \frac{1}{3} & {k=0} \\
      \frac{2}{3 }& {k=1} 
   \end{cases}
   \\
p_{Y|X}\brak{0|0} = \frac{19}{25}\, 
p_{Y|X}\brak{0|1} = \frac{6}{25}\,
p_{Y|X}\brak{1|0} = \frac{45}{50}\,
p_{Y|X}\brak{1|2} = \frac{5}{50}
\end{align}
The desired probability is the probability that a slip drawn at random is marked other than Rs 1,
\begin{align}
&=1-p_X\brak{0}\\
&= p_X(1) + p_X(2)
\end{align}
Using Bayes theorem,
\begin{align}
&= p_Y\brak{0} \times \pr{Y=0 | X=1} + p_Y\brak{1} \times \pr{Y=1|X=2}\\
&=\frac{1}{3} \times \frac{6}{25} + \frac{2}{3} \times \frac{5}{50}\\
&=\frac{11}{75}
\end{align}

\newpage

%\tableofcontents

\bigskip

\renewcommand{\thefigure}{\theenumi}
\renewcommand{\thetable}{\theenumi}
%\renewcommand{\theequation}{\theenumi}

%\begin{abstract}
%%\boldmath
%In this letter, an algorithm for evaluating the exact analytical bit error rate  (BER)  for the piecewise linear (PL) combiner for  multiple relays is presented. Previous results were available only for upto three relays. The algorithm is unique in the sense that  the actual mathematical expressions, that are prohibitively large, need not be explicitly obtained. The diversity gain due to multiple relays is shown through plots of the analytical BER, well supported by simulations. 
%
%\end{abstract}
% IEEEtran.cls defaults to using nonbold math in the Abstract.
% This preserves the distinction between vectors and scalars. However,
% if the journal you are submitting to favors bold math in the abstract,
% then you can use LaTeX's standard command \boldmath at the very start
% of the abstract to achieve this. Many IEEE journals frown on math
% in the abstract anyway.

% Note that keywords are not normally used for peerreview papers.
%\begin{IEEEkeywords}
%Cooperative diversity, decode and forward, piecewise linear
%\end{IEEEkeywords}



% For peer review papers, you can put extra information on the cover
% page as needed:
% \ifCLASSOPTIONpeerreview
% \begin{center} \bfseries EDICS Category: 3-BBND \end{center}
% \fi
%
% For peerreview papers, this IEEEtran command inserts a page break and
% creates the second title. It will be ignored for other modes.
%\IEEEpeerreviewmaketitle




\item A card is selected from a pack of 52 cards\\
\begin{enumerate}[label=(\alph*)]
\item How many points are there in the sample space?
\item Calculate the probability that the cards is an ace of spades.
\item Calculate the probability that the card is (i) an ace (ii)black card.\\
\end{enumerate}
%\input{ncert/11/16/3/4_1/Prob_4.tex}
\item In a non-leap year, the probability of having 53 tuesdays or 53 wednesdays is\\
\solution
%A non-leap year has a total of 365 days, and a week has 7 days.\\
So it can be expressed as 
\begin{align}
365\text{days} &=52\times 7+1 \text{day}
\end{align}
$\implies$ 52 tuesdays or wednesdays\\
Random variable X denotes the days of a week
\begin{align}
p_X\brak{k}&=\frac{1}{7}; \quad \brak{1<k<7}
\end{align}
So the probability of extra day being tuesday or wednesday is
\begin{align}
p_X\brak{3}+p_X\brak{4}&=\frac{1}{7}+\frac{1}{7}=\frac{2}{7}
\end{align}



\item There are 1000 sealed envelopes in a box, 10 of them contain a cash prize of
Rs 100 each, 100 of them contain a cash prize of Rs 50 each and 200 of them
contain a cash prize of Rs 10 each and rest do not contain any cash prize. If they
are well shuffled and an envelope is picked up out, what is the probability that it
contains no cash prize?\\
\solution
%\begin{table}[H]
	\centering
\begin{tabular}{|c|c|c|}
\hline
Random variable &Value &Definition\\ \hline
\multirow{3}{*}{X} &0 &Slips of Rs 1\\
&1 &Slips of Rs 5\\
&2 &Slips of Rs 13\\ \hline
\multirow{2}{*}{Y} &0 &Box A\\
&1 &Box B\\\hline
\end{tabular}
\caption{}
\label{tab:Distribution}
\end{table}
See \tabref{tab:Distribution}.
\begin{align}
p_{Y}\brak{k}= \begin{cases} 
      \frac{1}{3} & {k=0} \\
      \frac{2}{3 }& {k=1} 
   \end{cases}
   \\
p_{Y|X}\brak{0|0} = \frac{19}{25}\, 
p_{Y|X}\brak{0|1} = \frac{6}{25}\,
p_{Y|X}\brak{1|0} = \frac{45}{50}\,
p_{Y|X}\brak{1|2} = \frac{5}{50}
\end{align}
The desired probability is the probability that a slip drawn at random is marked other than Rs 1,
\begin{align}
&=1-p_X\brak{0}\\
&= p_X(1) + p_X(2)
\end{align}
Using Bayes theorem,
\begin{align}
&= p_Y\brak{0} \times \pr{Y=0 | X=1} + p_Y\brak{1} \times \pr{Y=1|X=2}\\
&=\frac{1}{3} \times \frac{6}{25} + \frac{2}{3} \times \frac{5}{50}\\
&=\frac{11}{75}
\end{align}

\newpage

%\tableofcontents

\bigskip

\renewcommand{\thefigure}{\theenumi}
\renewcommand{\thetable}{\theenumi}
%\renewcommand{\theequation}{\theenumi}

%\begin{abstract}
%%\boldmath
%In this letter, an algorithm for evaluating the exact analytical bit error rate  (BER)  for the piecewise linear (PL) combiner for  multiple relays is presented. Previous results were available only for upto three relays. The algorithm is unique in the sense that  the actual mathematical expressions, that are prohibitively large, need not be explicitly obtained. The diversity gain due to multiple relays is shown through plots of the analytical BER, well supported by simulations. 
%
%\end{abstract}
% IEEEtran.cls defaults to using nonbold math in the Abstract.
% This preserves the distinction between vectors and scalars. However,
% if the journal you are submitting to favors bold math in the abstract,
% then you can use LaTeX's standard command \boldmath at the very start
% of the abstract to achieve this. Many IEEE journals frown on math
% in the abstract anyway.

% Note that keywords are not normally used for peerreview papers.
%\begin{IEEEkeywords}
%Cooperative diversity, decode and forward, piecewise linear
%\end{IEEEkeywords}



% For peer review papers, you can put extra information on the cover
% page as needed:
% \ifCLASSOPTIONpeerreview
% \begin{center} \bfseries EDICS Category: 3-BBND \end{center}
% \fi
%
% For peerreview papers, this IEEEtran command inserts a page break and
% creates the second title. It will be ignored for other modes.
%\IEEEpeerreviewmaketitle




\item 
A die is thrown and a card is selected at random from a deck of 52 playing cards. The probability of getting an even number on the die and a spade card.\\
\solution
%\begin{table}[H]
	\centering
\begin{tabular}{|c|c|c|}
\hline
Random variable &Value &Definition\\ \hline
\multirow{3}{*}{X} &0 &Slips of Rs 1\\
&1 &Slips of Rs 5\\
&2 &Slips of Rs 13\\ \hline
\multirow{2}{*}{Y} &0 &Box A\\
&1 &Box B\\\hline
\end{tabular}
\caption{}
\label{tab:Distribution}
\end{table}
See \tabref{tab:Distribution}.
\begin{align}
p_{Y}\brak{k}= \begin{cases} 
      \frac{1}{3} & {k=0} \\
      \frac{2}{3 }& {k=1} 
   \end{cases}
   \\
p_{Y|X}\brak{0|0} = \frac{19}{25}\, 
p_{Y|X}\brak{0|1} = \frac{6}{25}\,
p_{Y|X}\brak{1|0} = \frac{45}{50}\,
p_{Y|X}\brak{1|2} = \frac{5}{50}
\end{align}
The desired probability is the probability that a slip drawn at random is marked other than Rs 1,
\begin{align}
&=1-p_X\brak{0}\\
&= p_X(1) + p_X(2)
\end{align}
Using Bayes theorem,
\begin{align}
&= p_Y\brak{0} \times \pr{Y=0 | X=1} + p_Y\brak{1} \times \pr{Y=1|X=2}\\
&=\frac{1}{3} \times \frac{6}{25} + \frac{2}{3} \times \frac{5}{50}\\
&=\frac{11}{75}
\end{align}

\newpage

%\tableofcontents

\bigskip

\renewcommand{\thefigure}{\theenumi}
\renewcommand{\thetable}{\theenumi}
%\renewcommand{\theequation}{\theenumi}

%\begin{abstract}
%%\boldmath
%In this letter, an algorithm for evaluating the exact analytical bit error rate  (BER)  for the piecewise linear (PL) combiner for  multiple relays is presented. Previous results were available only for upto three relays. The algorithm is unique in the sense that  the actual mathematical expressions, that are prohibitively large, need not be explicitly obtained. The diversity gain due to multiple relays is shown through plots of the analytical BER, well supported by simulations. 
%
%\end{abstract}
% IEEEtran.cls defaults to using nonbold math in the Abstract.
% This preserves the distinction between vectors and scalars. However,
% if the journal you are submitting to favors bold math in the abstract,
% then you can use LaTeX's standard command \boldmath at the very start
% of the abstract to achieve this. Many IEEE journals frown on math
% in the abstract anyway.

% Note that keywords are not normally used for peerreview papers.
%\begin{IEEEkeywords}
%Cooperative diversity, decode and forward, piecewise linear
%\end{IEEEkeywords}



% For peer review papers, you can put extra information on the cover
% page as needed:
% \ifCLASSOPTIONpeerreview
% \begin{center} \bfseries EDICS Category: 3-BBND \end{center}
% \fi
%
% For peerreview papers, this IEEEtran command inserts a page break and
% creates the second title. It will be ignored for other modes.
%\IEEEpeerreviewmaketitle




\item
If 4-digit numbers greater than 5,000 are randomly formed from the digits 0, 1, 3, 5, and 7, what is the probability of forming a number divisible by 5 when:
\begin{enumerate}
    \item The digits are repeated?
    \item The repetition of digits is not allowed?
\end{enumerate}
\solution
%\begin{table}[H]
	\centering
\begin{tabular}{|c|c|c|}
\hline
Random variable &Value &Definition\\ \hline
\multirow{3}{*}{X} &0 &Slips of Rs 1\\
&1 &Slips of Rs 5\\
&2 &Slips of Rs 13\\ \hline
\multirow{2}{*}{Y} &0 &Box A\\
&1 &Box B\\\hline
\end{tabular}
\caption{}
\label{tab:Distribution}
\end{table}
See \tabref{tab:Distribution}.
\begin{align}
p_{Y}\brak{k}= \begin{cases} 
      \frac{1}{3} & {k=0} \\
      \frac{2}{3 }& {k=1} 
   \end{cases}
   \\
p_{Y|X}\brak{0|0} = \frac{19}{25}\, 
p_{Y|X}\brak{0|1} = \frac{6}{25}\,
p_{Y|X}\brak{1|0} = \frac{45}{50}\,
p_{Y|X}\brak{1|2} = \frac{5}{50}
\end{align}
The desired probability is the probability that a slip drawn at random is marked other than Rs 1,
\begin{align}
&=1-p_X\brak{0}\\
&= p_X(1) + p_X(2)
\end{align}
Using Bayes theorem,
\begin{align}
&= p_Y\brak{0} \times \pr{Y=0 | X=1} + p_Y\brak{1} \times \pr{Y=1|X=2}\\
&=\frac{1}{3} \times \frac{6}{25} + \frac{2}{3} \times \frac{5}{50}\\
&=\frac{11}{75}
\end{align}

\newpage

%\tableofcontents

\bigskip

\renewcommand{\thefigure}{\theenumi}
\renewcommand{\thetable}{\theenumi}
%\renewcommand{\theequation}{\theenumi}

%\begin{abstract}
%%\boldmath
%In this letter, an algorithm for evaluating the exact analytical bit error rate  (BER)  for the piecewise linear (PL) combiner for  multiple relays is presented. Previous results were available only for upto three relays. The algorithm is unique in the sense that  the actual mathematical expressions, that are prohibitively large, need not be explicitly obtained. The diversity gain due to multiple relays is shown through plots of the analytical BER, well supported by simulations. 
%
%\end{abstract}
% IEEEtran.cls defaults to using nonbold math in the Abstract.
% This preserves the distinction between vectors and scalars. However,
% if the journal you are submitting to favors bold math in the abstract,
% then you can use LaTeX's standard command \boldmath at the very start
% of the abstract to achieve this. Many IEEE journals frown on math
% in the abstract anyway.

% Note that keywords are not normally used for peerreview papers.
%\begin{IEEEkeywords}
%Cooperative diversity, decode and forward, piecewise linear
%\end{IEEEkeywords}



% For peer review papers, you can put extra information on the cover
% page as needed:
% \ifCLASSOPTIONpeerreview
% \begin{center} \bfseries EDICS Category: 3-BBND \end{center}
% \fi
%
% For peerreview papers, this IEEEtran command inserts a page break and
% creates the second title. It will be ignored for other modes.
%\IEEEpeerreviewmaketitle




\item Consider the probability space $\brak{\Omega, \mathcal{G}, P}$ where $\Omega = [0,2]$ and $\mathcal{G} = \cbrak{\phi, \Omega, [0,1], (1,2]}$. Let $X$ and $Y$ be two functions on $\Omega$ defined as
\begin{align*}
    X(\omega) = 
    \begin{cases}
        1 & \text{if }\omega \in [0, 1]\\
        2 & \text{if }\omega \in (1, 2]
    \end{cases}
\end{align*}
and
\begin{align*}
    Y(\omega) = 
    \begin{cases}
        2 & \text{if }\omega \in [0, 1.5]\\
        3 & \text{if }\omega \in (1.5, 2].
    \end{cases}
\end{align*}
Then which one of the following statements is true?
\begin{enumerate}
    \item [(A)] $X$ is a random variable with respect to $\mathcal{G}$, but $Y$ is not a random variable with respect to $\mathcal{G}$.
    \item [(B)] $Y$ is a random variable with respect to $\mathcal{G}$, but $X$ is not a random variable with respect to $\mathcal{G}$.
    \item [(C)] Neither $X$ nor $Y$ is a random variable with respect to $\mathcal{G}$.
    \item [(D)] Both $X$ and $Y$ are random variables with respect to $\mathcal{G}$.
\end{enumerate} \hfill (GATE ST 2023)\\
\solution
%\begin{table}[H]
	\centering
\begin{tabular}{|c|c|c|}
\hline
Random variable &Value &Definition\\ \hline
\multirow{3}{*}{X} &0 &Slips of Rs 1\\
&1 &Slips of Rs 5\\
&2 &Slips of Rs 13\\ \hline
\multirow{2}{*}{Y} &0 &Box A\\
&1 &Box B\\\hline
\end{tabular}
\caption{}
\label{tab:Distribution}
\end{table}
See \tabref{tab:Distribution}.
\begin{align}
p_{Y}\brak{k}= \begin{cases} 
      \frac{1}{3} & {k=0} \\
      \frac{2}{3 }& {k=1} 
   \end{cases}
   \\
p_{Y|X}\brak{0|0} = \frac{19}{25}\, 
p_{Y|X}\brak{0|1} = \frac{6}{25}\,
p_{Y|X}\brak{1|0} = \frac{45}{50}\,
p_{Y|X}\brak{1|2} = \frac{5}{50}
\end{align}
The desired probability is the probability that a slip drawn at random is marked other than Rs 1,
\begin{align}
&=1-p_X\brak{0}\\
&= p_X(1) + p_X(2)
\end{align}
Using Bayes theorem,
\begin{align}
&= p_Y\brak{0} \times \pr{Y=0 | X=1} + p_Y\brak{1} \times \pr{Y=1|X=2}\\
&=\frac{1}{3} \times \frac{6}{25} + \frac{2}{3} \times \frac{5}{50}\\
&=\frac{11}{75}
\end{align}

\newpage

%\tableofcontents

\bigskip

\renewcommand{\thefigure}{\theenumi}
\renewcommand{\thetable}{\theenumi}
%\renewcommand{\theequation}{\theenumi}

%\begin{abstract}
%%\boldmath
%In this letter, an algorithm for evaluating the exact analytical bit error rate  (BER)  for the piecewise linear (PL) combiner for  multiple relays is presented. Previous results were available only for upto three relays. The algorithm is unique in the sense that  the actual mathematical expressions, that are prohibitively large, need not be explicitly obtained. The diversity gain due to multiple relays is shown through plots of the analytical BER, well supported by simulations. 
%
%\end{abstract}
% IEEEtran.cls defaults to using nonbold math in the Abstract.
% This preserves the distinction between vectors and scalars. However,
% if the journal you are submitting to favors bold math in the abstract,
% then you can use LaTeX's standard command \boldmath at the very start
% of the abstract to achieve this. Many IEEE journals frown on math
% in the abstract anyway.

% Note that keywords are not normally used for peerreview papers.
%\begin{IEEEkeywords}
%Cooperative diversity, decode and forward, piecewise linear
%\end{IEEEkeywords}



% For peer review papers, you can put extra information on the cover
% page as needed:
% \ifCLASSOPTIONpeerreview
% \begin{center} \bfseries EDICS Category: 3-BBND \end{center}
% \fi
%
% For peerreview papers, this IEEEtran command inserts a page break and
% creates the second title. It will be ignored for other modes.
%\IEEEpeerreviewmaketitle




	\item  A die is loaded in such a way that each odd number is twice as likely to occur as
each even number. Find $P(G)$, where $G$ is the event that a number greater than
3 occurs on a single roll of the die.
\\
\solution
		%\begin{table}[H]
	\centering
\begin{tabular}{|c|c|c|}
\hline
Random variable &Value &Definition\\ \hline
\multirow{3}{*}{X} &0 &Slips of Rs 1\\
&1 &Slips of Rs 5\\
&2 &Slips of Rs 13\\ \hline
\multirow{2}{*}{Y} &0 &Box A\\
&1 &Box B\\\hline
\end{tabular}
\caption{}
\label{tab:Distribution}
\end{table}
See \tabref{tab:Distribution}.
\begin{align}
p_{Y}\brak{k}= \begin{cases} 
      \frac{1}{3} & {k=0} \\
      \frac{2}{3 }& {k=1} 
   \end{cases}
   \\
p_{Y|X}\brak{0|0} = \frac{19}{25}\, 
p_{Y|X}\brak{0|1} = \frac{6}{25}\,
p_{Y|X}\brak{1|0} = \frac{45}{50}\,
p_{Y|X}\brak{1|2} = \frac{5}{50}
\end{align}
The desired probability is the probability that a slip drawn at random is marked other than Rs 1,
\begin{align}
&=1-p_X\brak{0}\\
&= p_X(1) + p_X(2)
\end{align}
Using Bayes theorem,
\begin{align}
&= p_Y\brak{0} \times \pr{Y=0 | X=1} + p_Y\brak{1} \times \pr{Y=1|X=2}\\
&=\frac{1}{3} \times \frac{6}{25} + \frac{2}{3} \times \frac{5}{50}\\
&=\frac{11}{75}
\end{align}

\newpage

%\tableofcontents

\bigskip

\renewcommand{\thefigure}{\theenumi}
\renewcommand{\thetable}{\theenumi}
%\renewcommand{\theequation}{\theenumi}

%\begin{abstract}
%%\boldmath
%In this letter, an algorithm for evaluating the exact analytical bit error rate  (BER)  for the piecewise linear (PL) combiner for  multiple relays is presented. Previous results were available only for upto three relays. The algorithm is unique in the sense that  the actual mathematical expressions, that are prohibitively large, need not be explicitly obtained. The diversity gain due to multiple relays is shown through plots of the analytical BER, well supported by simulations. 
%
%\end{abstract}
% IEEEtran.cls defaults to using nonbold math in the Abstract.
% This preserves the distinction between vectors and scalars. However,
% if the journal you are submitting to favors bold math in the abstract,
% then you can use LaTeX's standard command \boldmath at the very start
% of the abstract to achieve this. Many IEEE journals frown on math
% in the abstract anyway.

% Note that keywords are not normally used for peerreview papers.
%\begin{IEEEkeywords}
%Cooperative diversity, decode and forward, piecewise linear
%\end{IEEEkeywords}



% For peer review papers, you can put extra information on the cover
% page as needed:
% \ifCLASSOPTIONpeerreview
% \begin{center} \bfseries EDICS Category: 3-BBND \end{center}
% \fi
%
% For peerreview papers, this IEEEtran command inserts a page break and
% creates the second title. It will be ignored for other modes.
%\IEEEpeerreviewmaketitle




	\item All the jacks, queens and kings are removed from a deck of 52 playing cards. The remaining cards are well shuffled and then one card is drawn at random. Giving ace a value 1 similar value for other cards, find the probability that the card has a value 
		\begin{enumerate}
			\item 7
			\item greater than 7
			\item less than 7
		\end{enumerate}
		%Number of cards left after removing all jacks, queens and kings 
\begin{align}
N	= 52 - 4\times 3
	= 40
\end{align}
%\begin{table}[H]
%\def\arraystretch{1.2}
%\begin{tabular}{|c|c|c|}
%\hline
%	\textbf{Parameter} &\textbf{Value} &\textbf{Description}\\ \hline
%	$X$ &1-10 &Represents the value of the card picked \\ \hline
%\end{tabular}
%\end{table}
Let $1 \le X \le 10$ be the value of the card picked.  Then,
\begin{align}
	p_X(k) &= \Pr(X=k)\ \forall\ 1 \leq k \leq 10\\
	&= \frac{4\times 1}{40}\\
	&= \frac{1}{10}\\
	\therefore p_X(k) &= 
	\begin{cases}
		\frac{1}{10} & 1 \leq k \leq 10\\
		0 & \text{otherwise}
	\end{cases}
\end{align}
and
\begin{align}
	F_{X}(k) &= \sum_{m=0}^{k}p_{X}(m) \quad 1 \leq k \leq 10\\
	&= \frac{k}{10}\\
	\therefore F_{X}(k) &= 
	\begin{cases}
		0 & k \leq 0\\
		\frac{k}{10} & 1\leq k \leq 10\\
		1 & k > 10 
	\end{cases}
\end{align}
\begin{enumerate}
	\item Probability that card has value equal to 7 is
		\begin{align}
			 p_{X}(7)
			= \frac{1}{10}
		\end{align}
	\item Probability that card has value greater than 7 is
		\begin{align}
			1 - F_X(7)
			&= 1 - \frac{7}{10}
			\\
			&= \frac{3}{10}
		\end{align}
	\item Probability that card has value less than 7 is
		\begin{align}
			 F_{X}(6)
			=\frac{6}{10}
		\end{align}
\end{enumerate}

  \item A Lot consists of 48 mobile phones of which 42 are good, 3 have only minor defects and 3 have major defects.Varnika will buy a phone if it is good but the trader will only buy a mobile if it has no major defects. One phone is selected at random from the lot. What is the probability that it is
\begin{enumerate}
	\item acceptable to Varnika?
            \item acceptable to the trader?
\end{enumerate}
\solution
	%\begin{table}[H]
	\centering
\begin{tabular}{|c|c|c|}
\hline
Random variable &Value &Definition\\ \hline
\multirow{3}{*}{X} &0 &Slips of Rs 1\\
&1 &Slips of Rs 5\\
&2 &Slips of Rs 13\\ \hline
\multirow{2}{*}{Y} &0 &Box A\\
&1 &Box B\\\hline
\end{tabular}
\caption{}
\label{tab:Distribution}
\end{table}
See \tabref{tab:Distribution}.
\begin{align}
p_{Y}\brak{k}= \begin{cases} 
      \frac{1}{3} & {k=0} \\
      \frac{2}{3 }& {k=1} 
   \end{cases}
   \\
p_{Y|X}\brak{0|0} = \frac{19}{25}\, 
p_{Y|X}\brak{0|1} = \frac{6}{25}\,
p_{Y|X}\brak{1|0} = \frac{45}{50}\,
p_{Y|X}\brak{1|2} = \frac{5}{50}
\end{align}
The desired probability is the probability that a slip drawn at random is marked other than Rs 1,
\begin{align}
&=1-p_X\brak{0}\\
&= p_X(1) + p_X(2)
\end{align}
Using Bayes theorem,
\begin{align}
&= p_Y\brak{0} \times \pr{Y=0 | X=1} + p_Y\brak{1} \times \pr{Y=1|X=2}\\
&=\frac{1}{3} \times \frac{6}{25} + \frac{2}{3} \times \frac{5}{50}\\
&=\frac{11}{75}
\end{align}

\newpage

%\tableofcontents

\bigskip

\renewcommand{\thefigure}{\theenumi}
\renewcommand{\thetable}{\theenumi}
%\renewcommand{\theequation}{\theenumi}

%\begin{abstract}
%%\boldmath
%In this letter, an algorithm for evaluating the exact analytical bit error rate  (BER)  for the piecewise linear (PL) combiner for  multiple relays is presented. Previous results were available only for upto three relays. The algorithm is unique in the sense that  the actual mathematical expressions, that are prohibitively large, need not be explicitly obtained. The diversity gain due to multiple relays is shown through plots of the analytical BER, well supported by simulations. 
%
%\end{abstract}
% IEEEtran.cls defaults to using nonbold math in the Abstract.
% This preserves the distinction between vectors and scalars. However,
% if the journal you are submitting to favors bold math in the abstract,
% then you can use LaTeX's standard command \boldmath at the very start
% of the abstract to achieve this. Many IEEE journals frown on math
% in the abstract anyway.

% Note that keywords are not normally used for peerreview papers.
%\begin{IEEEkeywords}
%Cooperative diversity, decode and forward, piecewise linear
%\end{IEEEkeywords}



% For peer review papers, you can put extra information on the cover
% page as needed:
% \ifCLASSOPTIONpeerreview
% \begin{center} \bfseries EDICS Category: 3-BBND \end{center}
% \fi
%
% For peerreview papers, this IEEEtran command inserts a page break and
% creates the second title. It will be ignored for other modes.
%\IEEEpeerreviewmaketitle




 \item A student says that if you throw a die, it will show up 1 or not 1. Therefore, the probability of getting 1 and the probability of getting 'not 1' each is equal to $\frac{1}{2}$. Is this correct? Give reasons.\\
 \solution
        %\begin{table}[H]
	\centering
\begin{tabular}{|c|c|c|}
\hline
Random variable &Value &Definition\\ \hline
\multirow{3}{*}{X} &0 &Slips of Rs 1\\
&1 &Slips of Rs 5\\
&2 &Slips of Rs 13\\ \hline
\multirow{2}{*}{Y} &0 &Box A\\
&1 &Box B\\\hline
\end{tabular}
\caption{}
\label{tab:Distribution}
\end{table}
See \tabref{tab:Distribution}.
\begin{align}
p_{Y}\brak{k}= \begin{cases} 
      \frac{1}{3} & {k=0} \\
      \frac{2}{3 }& {k=1} 
   \end{cases}
   \\
p_{Y|X}\brak{0|0} = \frac{19}{25}\, 
p_{Y|X}\brak{0|1} = \frac{6}{25}\,
p_{Y|X}\brak{1|0} = \frac{45}{50}\,
p_{Y|X}\brak{1|2} = \frac{5}{50}
\end{align}
The desired probability is the probability that a slip drawn at random is marked other than Rs 1,
\begin{align}
&=1-p_X\brak{0}\\
&= p_X(1) + p_X(2)
\end{align}
Using Bayes theorem,
\begin{align}
&= p_Y\brak{0} \times \pr{Y=0 | X=1} + p_Y\brak{1} \times \pr{Y=1|X=2}\\
&=\frac{1}{3} \times \frac{6}{25} + \frac{2}{3} \times \frac{5}{50}\\
&=\frac{11}{75}
\end{align}

\newpage

%\tableofcontents

\bigskip

\renewcommand{\thefigure}{\theenumi}
\renewcommand{\thetable}{\theenumi}
%\renewcommand{\theequation}{\theenumi}

%\begin{abstract}
%%\boldmath
%In this letter, an algorithm for evaluating the exact analytical bit error rate  (BER)  for the piecewise linear (PL) combiner for  multiple relays is presented. Previous results were available only for upto three relays. The algorithm is unique in the sense that  the actual mathematical expressions, that are prohibitively large, need not be explicitly obtained. The diversity gain due to multiple relays is shown through plots of the analytical BER, well supported by simulations. 
%
%\end{abstract}
% IEEEtran.cls defaults to using nonbold math in the Abstract.
% This preserves the distinction between vectors and scalars. However,
% if the journal you are submitting to favors bold math in the abstract,
% then you can use LaTeX's standard command \boldmath at the very start
% of the abstract to achieve this. Many IEEE journals frown on math
% in the abstract anyway.

% Note that keywords are not normally used for peerreview papers.
%\begin{IEEEkeywords}
%Cooperative diversity, decode and forward, piecewise linear
%\end{IEEEkeywords}



% For peer review papers, you can put extra information on the cover
% page as needed:
% \ifCLASSOPTIONpeerreview
% \begin{center} \bfseries EDICS Category: 3-BBND \end{center}
% \fi
%
% For peerreview papers, this IEEEtran command inserts a page break and
% creates the second title. It will be ignored for other modes.
%\IEEEpeerreviewmaketitle




   \item Four candidates A, B, C, D have ap-
plied for the assignment to coach a school cricket
team. If A is twice as likely to be selected as B, and
B and C are given about the same chance of being
selected, while C is twice as likely to be selected
as D, what are the probabilities that
\begin{enumerate}
\item C will be selected?
\item A will not be selected?
\end{enumerate}
	%\begin{table}[H]
	\centering
\begin{tabular}{|c|c|c|}
\hline
Random variable &Value &Definition\\ \hline
\multirow{3}{*}{X} &0 &Slips of Rs 1\\
&1 &Slips of Rs 5\\
&2 &Slips of Rs 13\\ \hline
\multirow{2}{*}{Y} &0 &Box A\\
&1 &Box B\\\hline
\end{tabular}
\caption{}
\label{tab:Distribution}
\end{table}
See \tabref{tab:Distribution}.
\begin{align}
p_{Y}\brak{k}= \begin{cases} 
      \frac{1}{3} & {k=0} \\
      \frac{2}{3 }& {k=1} 
   \end{cases}
   \\
p_{Y|X}\brak{0|0} = \frac{19}{25}\, 
p_{Y|X}\brak{0|1} = \frac{6}{25}\,
p_{Y|X}\brak{1|0} = \frac{45}{50}\,
p_{Y|X}\brak{1|2} = \frac{5}{50}
\end{align}
The desired probability is the probability that a slip drawn at random is marked other than Rs 1,
\begin{align}
&=1-p_X\brak{0}\\
&= p_X(1) + p_X(2)
\end{align}
Using Bayes theorem,
\begin{align}
&= p_Y\brak{0} \times \pr{Y=0 | X=1} + p_Y\brak{1} \times \pr{Y=1|X=2}\\
&=\frac{1}{3} \times \frac{6}{25} + \frac{2}{3} \times \frac{5}{50}\\
&=\frac{11}{75}
\end{align}

\newpage

%\tableofcontents

\bigskip

\renewcommand{\thefigure}{\theenumi}
\renewcommand{\thetable}{\theenumi}
%\renewcommand{\theequation}{\theenumi}

%\begin{abstract}
%%\boldmath
%In this letter, an algorithm for evaluating the exact analytical bit error rate  (BER)  for the piecewise linear (PL) combiner for  multiple relays is presented. Previous results were available only for upto three relays. The algorithm is unique in the sense that  the actual mathematical expressions, that are prohibitively large, need not be explicitly obtained. The diversity gain due to multiple relays is shown through plots of the analytical BER, well supported by simulations. 
%
%\end{abstract}
% IEEEtran.cls defaults to using nonbold math in the Abstract.
% This preserves the distinction between vectors and scalars. However,
% if the journal you are submitting to favors bold math in the abstract,
% then you can use LaTeX's standard command \boldmath at the very start
% of the abstract to achieve this. Many IEEE journals frown on math
% in the abstract anyway.

% Note that keywords are not normally used for peerreview papers.
%\begin{IEEEkeywords}
%Cooperative diversity, decode and forward, piecewise linear
%\end{IEEEkeywords}



% For peer review papers, you can put extra information on the cover
% page as needed:
% \ifCLASSOPTIONpeerreview
% \begin{center} \bfseries EDICS Category: 3-BBND \end{center}
% \fi
%
% For peerreview papers, this IEEEtran command inserts a page break and
% creates the second title. It will be ignored for other modes.
%\IEEEpeerreviewmaketitle




 \item A bag contain 24 balls of which $x$ balls are red, $2x$ are white and $3x$ are blue. A ball is selected at random, What is the probability that it is
\begin{enumerate}[label=\alph*)]
\item not red ?
\item white ?
\end{enumerate}
%\begin{table}[H]
	\centering
\begin{tabular}{|c|c|c|}
\hline
Random variable &Value &Definition\\ \hline
\multirow{3}{*}{X} &0 &Slips of Rs 1\\
&1 &Slips of Rs 5\\
&2 &Slips of Rs 13\\ \hline
\multirow{2}{*}{Y} &0 &Box A\\
&1 &Box B\\\hline
\end{tabular}
\caption{}
\label{tab:Distribution}
\end{table}
See \tabref{tab:Distribution}.
\begin{align}
p_{Y}\brak{k}= \begin{cases} 
      \frac{1}{3} & {k=0} \\
      \frac{2}{3 }& {k=1} 
   \end{cases}
   \\
p_{Y|X}\brak{0|0} = \frac{19}{25}\, 
p_{Y|X}\brak{0|1} = \frac{6}{25}\,
p_{Y|X}\brak{1|0} = \frac{45}{50}\,
p_{Y|X}\brak{1|2} = \frac{5}{50}
\end{align}
The desired probability is the probability that a slip drawn at random is marked other than Rs 1,
\begin{align}
&=1-p_X\brak{0}\\
&= p_X(1) + p_X(2)
\end{align}
Using Bayes theorem,
\begin{align}
&= p_Y\brak{0} \times \pr{Y=0 | X=1} + p_Y\brak{1} \times \pr{Y=1|X=2}\\
&=\frac{1}{3} \times \frac{6}{25} + \frac{2}{3} \times \frac{5}{50}\\
&=\frac{11}{75}
\end{align}

\newpage

%\tableofcontents

\bigskip

\renewcommand{\thefigure}{\theenumi}
\renewcommand{\thetable}{\theenumi}
%\renewcommand{\theequation}{\theenumi}

%\begin{abstract}
%%\boldmath
%In this letter, an algorithm for evaluating the exact analytical bit error rate  (BER)  for the piecewise linear (PL) combiner for  multiple relays is presented. Previous results were available only for upto three relays. The algorithm is unique in the sense that  the actual mathematical expressions, that are prohibitively large, need not be explicitly obtained. The diversity gain due to multiple relays is shown through plots of the analytical BER, well supported by simulations. 
%
%\end{abstract}
% IEEEtran.cls defaults to using nonbold math in the Abstract.
% This preserves the distinction between vectors and scalars. However,
% if the journal you are submitting to favors bold math in the abstract,
% then you can use LaTeX's standard command \boldmath at the very start
% of the abstract to achieve this. Many IEEE journals frown on math
% in the abstract anyway.

% Note that keywords are not normally used for peerreview papers.
%\begin{IEEEkeywords}
%Cooperative diversity, decode and forward, piecewise linear
%\end{IEEEkeywords}



% For peer review papers, you can put extra information on the cover
% page as needed:
% \ifCLASSOPTIONpeerreview
% \begin{center} \bfseries EDICS Category: 3-BBND \end{center}
% \fi
%
% For peerreview papers, this IEEEtran command inserts a page break and
% creates the second title. It will be ignored for other modes.
%\IEEEpeerreviewmaketitle




If the letters of the word ASSASSINATION are arranged at random. Find the Probability that
\begin{enumerate}[label=(\alph*)]
\item Four $S's$ come consecutively in the word
\item Two  $I's$ and two $N's$ come together
\item All $A's$ are not coming together
\item No two $A's$ are coming together
\end{enumerate}
%\begin{table}[H]
	\centering
\begin{tabular}{|c|c|c|}
\hline
Random variable &Value &Definition\\ \hline
\multirow{3}{*}{X} &0 &Slips of Rs 1\\
&1 &Slips of Rs 5\\
&2 &Slips of Rs 13\\ \hline
\multirow{2}{*}{Y} &0 &Box A\\
&1 &Box B\\\hline
\end{tabular}
\caption{}
\label{tab:Distribution}
\end{table}
See \tabref{tab:Distribution}.
\begin{align}
p_{Y}\brak{k}= \begin{cases} 
      \frac{1}{3} & {k=0} \\
      \frac{2}{3 }& {k=1} 
   \end{cases}
   \\
p_{Y|X}\brak{0|0} = \frac{19}{25}\, 
p_{Y|X}\brak{0|1} = \frac{6}{25}\,
p_{Y|X}\brak{1|0} = \frac{45}{50}\,
p_{Y|X}\brak{1|2} = \frac{5}{50}
\end{align}
The desired probability is the probability that a slip drawn at random is marked other than Rs 1,
\begin{align}
&=1-p_X\brak{0}\\
&= p_X(1) + p_X(2)
\end{align}
Using Bayes theorem,
\begin{align}
&= p_Y\brak{0} \times \pr{Y=0 | X=1} + p_Y\brak{1} \times \pr{Y=1|X=2}\\
&=\frac{1}{3} \times \frac{6}{25} + \frac{2}{3} \times \frac{5}{50}\\
&=\frac{11}{75}
\end{align}

\newpage

%\tableofcontents

\bigskip

\renewcommand{\thefigure}{\theenumi}
\renewcommand{\thetable}{\theenumi}
%\renewcommand{\theequation}{\theenumi}

%\begin{abstract}
%%\boldmath
%In this letter, an algorithm for evaluating the exact analytical bit error rate  (BER)  for the piecewise linear (PL) combiner for  multiple relays is presented. Previous results were available only for upto three relays. The algorithm is unique in the sense that  the actual mathematical expressions, that are prohibitively large, need not be explicitly obtained. The diversity gain due to multiple relays is shown through plots of the analytical BER, well supported by simulations. 
%
%\end{abstract}
% IEEEtran.cls defaults to using nonbold math in the Abstract.
% This preserves the distinction between vectors and scalars. However,
% if the journal you are submitting to favors bold math in the abstract,
% then you can use LaTeX's standard command \boldmath at the very start
% of the abstract to achieve this. Many IEEE journals frown on math
% in the abstract anyway.

% Note that keywords are not normally used for peerreview papers.
%\begin{IEEEkeywords}
%Cooperative diversity, decode and forward, piecewise linear
%\end{IEEEkeywords}



% For peer review papers, you can put extra information on the cover
% page as needed:
% \ifCLASSOPTIONpeerreview
% \begin{center} \bfseries EDICS Category: 3-BBND \end{center}
% \fi
%
% For peerreview papers, this IEEEtran command inserts a page break and
% creates the second title. It will be ignored for other modes.
%\IEEEpeerreviewmaketitle




	\item One urn contains two black balls (labelled B1 and B2) and one white ball. A
	second urn contains one black ball and two white balls (labelled W1 and W2).
	Suppose the following experiment is performed. One of the two urns is chosen
	at random. Next a ball is randomly chosen from the urn. Then a second ball is
	chosen at random from the same urn without replacing the first ball.
	
	\begin{enumerate}
	\item What is the probability that two black balls are chosen?
	
	\item What is the probability that two balls of opposite colour are chosen?
	\end{enumerate}
	\solution
	%\begin{align}
    \label{eq:12.13.6.18.1}
	\because	\pr{A|B} &> \pr{A},\
\frac{\pr{AB}}{\pr{B}} > \pr{A}
\\
    \label{eq:12.13.6.18.2}
	\implies \pr{AB} &> \pr{A}\pr{B}
	\\
	\text{or, } \frac{\pr{AB}}{\pr{A}} &=\pr{B|A} > \pr{A}
\end{align}

\end{enumerate}

	\item A card is selected from a pack of 52 cards.
 \begin{enumerate}[label=(\alph*)] 
                 \item How many points are there in the sample space?
                 \item Calculate the probability that the card is an ace of spades.
                 \item Calculate the probability that the card is (i) an ace and (ii) black card.
 \end{enumerate}
\solution
		%\begin{table}[H]
	\centering
\begin{tabular}{|c|c|c|}
\hline
Random variable &Value &Definition\\ \hline
\multirow{3}{*}{X} &0 &Slips of Rs 1\\
&1 &Slips of Rs 5\\
&2 &Slips of Rs 13\\ \hline
\multirow{2}{*}{Y} &0 &Box A\\
&1 &Box B\\\hline
\end{tabular}
\caption{}
\label{tab:Distribution}
\end{table}
See \tabref{tab:Distribution}.
\begin{align}
p_{Y}\brak{k}= \begin{cases} 
      \frac{1}{3} & {k=0} \\
      \frac{2}{3 }& {k=1} 
   \end{cases}
   \\
p_{Y|X}\brak{0|0} = \frac{19}{25}\, 
p_{Y|X}\brak{0|1} = \frac{6}{25}\,
p_{Y|X}\brak{1|0} = \frac{45}{50}\,
p_{Y|X}\brak{1|2} = \frac{5}{50}
\end{align}
The desired probability is the probability that a slip drawn at random is marked other than Rs 1,
\begin{align}
&=1-p_X\brak{0}\\
&= p_X(1) + p_X(2)
\end{align}
Using Bayes theorem,
\begin{align}
&= p_Y\brak{0} \times \pr{Y=0 | X=1} + p_Y\brak{1} \times \pr{Y=1|X=2}\\
&=\frac{1}{3} \times \frac{6}{25} + \frac{2}{3} \times \frac{5}{50}\\
&=\frac{11}{75}
\end{align}

\newpage

%\tableofcontents

\bigskip

\renewcommand{\thefigure}{\theenumi}
\renewcommand{\thetable}{\theenumi}
%\renewcommand{\theequation}{\theenumi}

%\begin{abstract}
%%\boldmath
%In this letter, an algorithm for evaluating the exact analytical bit error rate  (BER)  for the piecewise linear (PL) combiner for  multiple relays is presented. Previous results were available only for upto three relays. The algorithm is unique in the sense that  the actual mathematical expressions, that are prohibitively large, need not be explicitly obtained. The diversity gain due to multiple relays is shown through plots of the analytical BER, well supported by simulations. 
%
%\end{abstract}
% IEEEtran.cls defaults to using nonbold math in the Abstract.
% This preserves the distinction between vectors and scalars. However,
% if the journal you are submitting to favors bold math in the abstract,
% then you can use LaTeX's standard command \boldmath at the very start
% of the abstract to achieve this. Many IEEE journals frown on math
% in the abstract anyway.

% Note that keywords are not normally used for peerreview papers.
%\begin{IEEEkeywords}
%Cooperative diversity, decode and forward, piecewise linear
%\end{IEEEkeywords}



% For peer review papers, you can put extra information on the cover
% page as needed:
% \ifCLASSOPTIONpeerreview
% \begin{center} \bfseries EDICS Category: 3-BBND \end{center}
% \fi
%
% For peerreview papers, this IEEEtran command inserts a page break and
% creates the second title. It will be ignored for other modes.
%\IEEEpeerreviewmaketitle




\item Four cards are drawn from a well-shuffled deck of 52 cards. What is the probability of obtaining 3 diamonds and one spade.
\\
\solution
		%\begin{enumerate}[label=\thesection.\arabic*,ref=\thesection.\theenumi]
	\item One card is drawn from a well-shuffled deck of 52 cards. Find the probability of getting
\begin{enumerate}
\item A king of red colour 
\item A face card 
\item A red face card
\item The jack of hearts
\item A spade
\item The queen of diamonds

\end{enumerate}
\solution
		%\begin{table}[H]
	\centering
\begin{tabular}{|c|c|c|}
\hline
Random variable &Value &Definition\\ \hline
\multirow{3}{*}{X} &0 &Slips of Rs 1\\
&1 &Slips of Rs 5\\
&2 &Slips of Rs 13\\ \hline
\multirow{2}{*}{Y} &0 &Box A\\
&1 &Box B\\\hline
\end{tabular}
\caption{}
\label{tab:Distribution}
\end{table}
See \tabref{tab:Distribution}.
\begin{align}
p_{Y}\brak{k}= \begin{cases} 
      \frac{1}{3} & {k=0} \\
      \frac{2}{3 }& {k=1} 
   \end{cases}
   \\
p_{Y|X}\brak{0|0} = \frac{19}{25}\, 
p_{Y|X}\brak{0|1} = \frac{6}{25}\,
p_{Y|X}\brak{1|0} = \frac{45}{50}\,
p_{Y|X}\brak{1|2} = \frac{5}{50}
\end{align}
The desired probability is the probability that a slip drawn at random is marked other than Rs 1,
\begin{align}
&=1-p_X\brak{0}\\
&= p_X(1) + p_X(2)
\end{align}
Using Bayes theorem,
\begin{align}
&= p_Y\brak{0} \times \pr{Y=0 | X=1} + p_Y\brak{1} \times \pr{Y=1|X=2}\\
&=\frac{1}{3} \times \frac{6}{25} + \frac{2}{3} \times \frac{5}{50}\\
&=\frac{11}{75}
\end{align}

\newpage

%\tableofcontents

\bigskip

\renewcommand{\thefigure}{\theenumi}
\renewcommand{\thetable}{\theenumi}
%\renewcommand{\theequation}{\theenumi}

%\begin{abstract}
%%\boldmath
%In this letter, an algorithm for evaluating the exact analytical bit error rate  (BER)  for the piecewise linear (PL) combiner for  multiple relays is presented. Previous results were available only for upto three relays. The algorithm is unique in the sense that  the actual mathematical expressions, that are prohibitively large, need not be explicitly obtained. The diversity gain due to multiple relays is shown through plots of the analytical BER, well supported by simulations. 
%
%\end{abstract}
% IEEEtran.cls defaults to using nonbold math in the Abstract.
% This preserves the distinction between vectors and scalars. However,
% if the journal you are submitting to favors bold math in the abstract,
% then you can use LaTeX's standard command \boldmath at the very start
% of the abstract to achieve this. Many IEEE journals frown on math
% in the abstract anyway.

% Note that keywords are not normally used for peerreview papers.
%\begin{IEEEkeywords}
%Cooperative diversity, decode and forward, piecewise linear
%\end{IEEEkeywords}



% For peer review papers, you can put extra information on the cover
% page as needed:
% \ifCLASSOPTIONpeerreview
% \begin{center} \bfseries EDICS Category: 3-BBND \end{center}
% \fi
%
% For peerreview papers, this IEEEtran command inserts a page break and
% creates the second title. It will be ignored for other modes.
%\IEEEpeerreviewmaketitle




	\item Five cards—the ten, jack, queen, king and ace of diamonds, are well-shuffled with their face downwards. One card is then picked up at random.
\begin{enumerate}
\item
What is the probability that the card is the queen? 
\item
If the queen is drawn and put aside, what is the probability that the second card picked up is (a) an ace? (b) a queen?\\
\end{enumerate}
\solution
		%\begin{enumerate}[label=\thesection.\arabic*,ref=\thesection.\theenumi]
	\item One card is drawn from a well-shuffled deck of 52 cards. Find the probability of getting
\begin{enumerate}
\item A king of red colour 
\item A face card 
\item A red face card
\item The jack of hearts
\item A spade
\item The queen of diamonds

\end{enumerate}
\solution
		%\input{ncert/10/15/1/14/main.tex}
	\item Five cards—the ten, jack, queen, king and ace of diamonds, are well-shuffled with their face downwards. One card is then picked up at random.
\begin{enumerate}
\item
What is the probability that the card is the queen? 
\item
If the queen is drawn and put aside, what is the probability that the second card picked up is (a) an ace? (b) a queen?\\
\end{enumerate}
\solution
		%\input{ncert/10/15/1/15/defs.tex}
	\item A bag contains $5$ red balls and some blue balls. If the probability of drawing a blue ball is double that if a red ball, determine the number of blue balls in the bag. 
		\\
\solution
		%\input{ncert/10/15/2/3/defs.tex}
	\item A card is selected from a pack of 52 cards.
 \begin{enumerate}[label=(\alph*)] 
                 \item How many points are there in the sample space?
                 \item Calculate the probability that the card is an ace of spades.
                 \item Calculate the probability that the card is (i) an ace and (ii) black card.
 \end{enumerate}
\solution
		%\input{ncert/11/16/3/4/main.tex}
\item Four cards are drawn from a well-shuffled deck of 52 cards. What is the probability of obtaining 3 diamonds and one spade.
\\
\solution
		%\input{ncert/11/16/4/2/defs.tex}
\item In a certain lottery 10,000 tickets are sold and ten equal prizes are awarded. What is the probability of not getting a prize if you buy (a) one ticket (b) two tickets (c) 10 tickets ?	
\\
\solution
		%\input{ncert/11/16/4/4/defs.tex}
		%
\item 
Out of 100 students, two sections of 40 and 60 are formed. If you and your friend are among the 100 students, what is the probability that
\begin{enumerate}
\item you both enter the same section?
\item you both enter the different sections?
\end{enumerate}
\solution
		%\input{ncert/11/16/4/5/defs.tex}
	\item 
The number lock of a suitcase has 4 wheels each labelled with ten digits i.e. from 0 to 9.The lock opens with a sequence of four digits with no repeats.What is the probability of a person getting the right sequence to open the suitcase.
\\
\solution
		%\input{ncert/11/16/4/10/defs.tex}
		%
\item 
Two cards are drawn at random and without replacement from a pack of 52 playing cards. Find the probability that both the cards are black.
\\
\solution
		%\input{ncert/12/13/2/2/defs.tex}
		\item A box of oranges is inspected by examining three randomly selected oranges drawn without replacement. If all the three oranges are good, the box is approved for sale, otherwise, it is rejected. Find the probability that a box containing 15 oranges out of which 12 are good and 3 are bad ones will be approved for sale.
		\label{ncert/12/13/2/3/defs.tex}
		\item Two balls are drawn at random with replacement from a box containing 10 black and 8 red balls. Find the probability that
		\label{ncert/12/13/2/12}
\begin{enumerate}
\item both balls are red.
\item first ball is black and second is red.
\item one of them is black and other is red.
\end{enumerate}

\item In a hostel, 60\% of the students read Hindi newspaper, 40\% read English newspaper and 20\% read both Hindi and English newspapers. A student is selected at random.
		\label{ncert/12/13/2/15}
\begin{enumerate}
\item Find the probability that she reads neither Hindi nor English newspapers.
\item If she reads Hindi newspaper, find the probability that she reads English newspaper.
\item If she reads English newspaper, find the probability that she reads Hindi newspaper.\\
\end{enumerate}
\item The probability of obtaining an even prime number on each die, when a pair of dice is rolled is 
\begin{enumerate}
    \item $0$ 
    
    \item $\frac{1}{3}$ 
    
    \item $\frac{1}{12}$ 
    
    \item $\frac{1}{36}$ 
\end{enumerate}
\solution
		%\input{ncert/12/13/2/17/defs.tex}
	\item A bag contains 4 red and 4 black balls, another bag contains 2 red and 6 black balls. One of the two bags is selected at random and a ball is drawn from the bag which is found to be red. Find the probability that the ball is drawn from the first bag.
\\
\solution
		%\input{ncert/12/13/3/2/main.tex}
  \item
  Cards with numbers 2 to 101 are placed in a box. A card is selected at random.Find the probability that the card has
\begin{enumerate}[label=(\roman*)]
	\item an even number 
	\item a square number
\end{enumerate}
\solution
%\input{exemplar/10/13/3/32/main.tex}
\item
The king, queen and jack of clubs are removed from a deck of 52 playing cards and then well shuffled. Now one card is drawn at random from the remaining cards.  Determine the probability that the card is
\begin{enumerate}[label=(\roman*)]
\item a club
\item 10 of hearts
\end{enumerate}
\solution
%\input{exemplar/10/13/3/29/main.tex}
\item A team of medical students doing their internship have to assist during surgeries
at a city hospital. The probabilities of surgeries rated as very complex, complex,
routine, simple or very simple are respectively, 0.15, 0.20, 0.31, 0.26, .08. Find
the probabilities that a particular surgery will be rated
\begin{enumerate}
	\item complex or very complex;
	\item neither very complex nor very simple;
	\item routine or complex
	\item routine or simple
\end{enumerate}
\solution
%\input{exemplar/11/16/3/8(1)/main.tex}
\item A card is selected from a pack of 52 cards.
\begin{enumerate}[label=(\alph*)]
    \item How many points are there in the sample space?
    \item Calculate the probability that the card is an ace of spades.
    \item Calculate the probability that the card is (i) an ace and (ii) black card.
\end{enumerate}
\solution
%\input{exemplar/11/16/3/4/main2.tex}
\item The probability that a non leap year selected at random will contain 53 sundays.
\\
\solution
%\input{exemplar/10/13/1/19/main.tex}
\item One of the four persons John, Rita, Aslam or Gurpreet will be promoted next
month. Consequently the sample space consists of four elementary outcomes
S = {John promoted, Rita promoted, Aslam promoted, Gurpreet promoted}
You are told that the chances of John’s promotion is same as that of Gurpreet,
Rita’s chances of promotion are twice as likely as Johns. Aslam’s chances are
four times that of John.
\begin{enumerate}
	\item Determine
	\begin{enumerate}
		\item P (John promoted)
		\item P (Rita promoted)
		\item P (Aslam promoted)
		\item P (Gurpreet promoted)
	\end{enumerate}
	\item If A = {John promoted or Gurpreet promoted}, find P (A).
\end{enumerate}
\solution
%\input{exemplar/11/16/3/10/main.tex}
\item A card is drawn from a deck of 52 cards. Find the probability of getting a king or a heart or a red card.\\
\solution
%\input{exemplar/11/16/3/15/main.tex}
\item The probability that a student will pass his examination is 0.73, the probability of
the student getting a compartment is 0.13, and the probability that the student will
either pass or get compartment is 0.96. State True or False.\\
\solution
%\input{exemplar/11/16/3/31/main.tex}
\item A card is selected from a pack of 52 cards\\
\begin{enumerate}[label=(\alph*)]
\item How many points are there in the sample space?
\item Calculate the probability that the cards is an ace of spades.
\item Calculate the probability that the card is (i) an ace (ii)black card.\\
\end{enumerate}
%\input{ncert/11/16/3/4_1/Prob_4.tex}
\item In a non-leap year, the probability of having 53 tuesdays or 53 wednesdays is\\
\solution
%\input{exemplar/11/16/3/18/main.tex}
\item There are 1000 sealed envelopes in a box, 10 of them contain a cash prize of
Rs 100 each, 100 of them contain a cash prize of Rs 50 each and 200 of them
contain a cash prize of Rs 10 each and rest do not contain any cash prize. If they
are well shuffled and an envelope is picked up out, what is the probability that it
contains no cash prize?\\
\solution
%\input{exemplar/10/13/3/34/main.tex}
\item 
A die is thrown and a card is selected at random from a deck of 52 playing cards. The probability of getting an even number on the die and a spade card.\\
\solution
%\input{exemplar/12/13/3/78/main.tex}
\item
If 4-digit numbers greater than 5,000 are randomly formed from the digits 0, 1, 3, 5, and 7, what is the probability of forming a number divisible by 5 when:
\begin{enumerate}
    \item The digits are repeated?
    \item The repetition of digits is not allowed?
\end{enumerate}
\solution
%\input{ncert/11/16/4/9/main.tex}
\item Consider the probability space $\brak{\Omega, \mathcal{G}, P}$ where $\Omega = [0,2]$ and $\mathcal{G} = \cbrak{\phi, \Omega, [0,1], (1,2]}$. Let $X$ and $Y$ be two functions on $\Omega$ defined as
\begin{align*}
    X(\omega) = 
    \begin{cases}
        1 & \text{if }\omega \in [0, 1]\\
        2 & \text{if }\omega \in (1, 2]
    \end{cases}
\end{align*}
and
\begin{align*}
    Y(\omega) = 
    \begin{cases}
        2 & \text{if }\omega \in [0, 1.5]\\
        3 & \text{if }\omega \in (1.5, 2].
    \end{cases}
\end{align*}
Then which one of the following statements is true?
\begin{enumerate}
    \item [(A)] $X$ is a random variable with respect to $\mathcal{G}$, but $Y$ is not a random variable with respect to $\mathcal{G}$.
    \item [(B)] $Y$ is a random variable with respect to $\mathcal{G}$, but $X$ is not a random variable with respect to $\mathcal{G}$.
    \item [(C)] Neither $X$ nor $Y$ is a random variable with respect to $\mathcal{G}$.
    \item [(D)] Both $X$ and $Y$ are random variables with respect to $\mathcal{G}$.
\end{enumerate} \hfill (GATE ST 2023)\\
\solution
%\input{gate/ST/2023/14/main.tex}
	\item  A die is loaded in such a way that each odd number is twice as likely to occur as
each even number. Find $P(G)$, where $G$ is the event that a number greater than
3 occurs on a single roll of the die.
\\
\solution
		%\input{exemplar/11/16/3/5/main.tex}
	\item All the jacks, queens and kings are removed from a deck of 52 playing cards. The remaining cards are well shuffled and then one card is drawn at random. Giving ace a value 1 similar value for other cards, find the probability that the card has a value 
		\begin{enumerate}
			\item 7
			\item greater than 7
			\item less than 7
		\end{enumerate}
		%\input{exemplar/10/13/3/30/main.tex}
  \item A Lot consists of 48 mobile phones of which 42 are good, 3 have only minor defects and 3 have major defects.Varnika will buy a phone if it is good but the trader will only buy a mobile if it has no major defects. One phone is selected at random from the lot. What is the probability that it is
\begin{enumerate}
	\item acceptable to Varnika?
            \item acceptable to the trader?
\end{enumerate}
\solution
	%\input{exemplar/10/13/3/40/main.tex}
 \item A student says that if you throw a die, it will show up 1 or not 1. Therefore, the probability of getting 1 and the probability of getting 'not 1' each is equal to $\frac{1}{2}$. Is this correct? Give reasons.\\
 \solution
        %\input{exemplar/10/13/2/9/main.tex}
   \item Four candidates A, B, C, D have ap-
plied for the assignment to coach a school cricket
team. If A is twice as likely to be selected as B, and
B and C are given about the same chance of being
selected, while C is twice as likely to be selected
as D, what are the probabilities that
\begin{enumerate}
\item C will be selected?
\item A will not be selected?
\end{enumerate}
	%\input{exemplar/11/16/3/9/main.tex}
 \item A bag contain 24 balls of which $x$ balls are red, $2x$ are white and $3x$ are blue. A ball is selected at random, What is the probability that it is
\begin{enumerate}[label=\alph*)]
\item not red ?
\item white ?
\end{enumerate}
%\input{exemplar/10/13/3/41/main.tex}
If the letters of the word ASSASSINATION are arranged at random. Find the Probability that
\begin{enumerate}[label=(\alph*)]
\item Four $S's$ come consecutively in the word
\item Two  $I's$ and two $N's$ come together
\item All $A's$ are not coming together
\item No two $A's$ are coming together
\end{enumerate}
%\input{exemplar/11/16/3/14/main.tex}
	\item One urn contains two black balls (labelled B1 and B2) and one white ball. A
	second urn contains one black ball and two white balls (labelled W1 and W2).
	Suppose the following experiment is performed. One of the two urns is chosen
	at random. Next a ball is randomly chosen from the urn. Then a second ball is
	chosen at random from the same urn without replacing the first ball.
	
	\begin{enumerate}
	\item What is the probability that two black balls are chosen?
	
	\item What is the probability that two balls of opposite colour are chosen?
	\end{enumerate}
	\solution
	%\input{exemplar/11/16/3/12/main1.tex}
\end{enumerate}

	\item A bag contains $5$ red balls and some blue balls. If the probability of drawing a blue ball is double that if a red ball, determine the number of blue balls in the bag. 
		\\
\solution
		%\begin{enumerate}[label=\thesection.\arabic*,ref=\thesection.\theenumi]
	\item One card is drawn from a well-shuffled deck of 52 cards. Find the probability of getting
\begin{enumerate}
\item A king of red colour 
\item A face card 
\item A red face card
\item The jack of hearts
\item A spade
\item The queen of diamonds

\end{enumerate}
\solution
		%\input{ncert/10/15/1/14/main.tex}
	\item Five cards—the ten, jack, queen, king and ace of diamonds, are well-shuffled with their face downwards. One card is then picked up at random.
\begin{enumerate}
\item
What is the probability that the card is the queen? 
\item
If the queen is drawn and put aside, what is the probability that the second card picked up is (a) an ace? (b) a queen?\\
\end{enumerate}
\solution
		%\input{ncert/10/15/1/15/defs.tex}
	\item A bag contains $5$ red balls and some blue balls. If the probability of drawing a blue ball is double that if a red ball, determine the number of blue balls in the bag. 
		\\
\solution
		%\input{ncert/10/15/2/3/defs.tex}
	\item A card is selected from a pack of 52 cards.
 \begin{enumerate}[label=(\alph*)] 
                 \item How many points are there in the sample space?
                 \item Calculate the probability that the card is an ace of spades.
                 \item Calculate the probability that the card is (i) an ace and (ii) black card.
 \end{enumerate}
\solution
		%\input{ncert/11/16/3/4/main.tex}
\item Four cards are drawn from a well-shuffled deck of 52 cards. What is the probability of obtaining 3 diamonds and one spade.
\\
\solution
		%\input{ncert/11/16/4/2/defs.tex}
\item In a certain lottery 10,000 tickets are sold and ten equal prizes are awarded. What is the probability of not getting a prize if you buy (a) one ticket (b) two tickets (c) 10 tickets ?	
\\
\solution
		%\input{ncert/11/16/4/4/defs.tex}
		%
\item 
Out of 100 students, two sections of 40 and 60 are formed. If you and your friend are among the 100 students, what is the probability that
\begin{enumerate}
\item you both enter the same section?
\item you both enter the different sections?
\end{enumerate}
\solution
		%\input{ncert/11/16/4/5/defs.tex}
	\item 
The number lock of a suitcase has 4 wheels each labelled with ten digits i.e. from 0 to 9.The lock opens with a sequence of four digits with no repeats.What is the probability of a person getting the right sequence to open the suitcase.
\\
\solution
		%\input{ncert/11/16/4/10/defs.tex}
		%
\item 
Two cards are drawn at random and without replacement from a pack of 52 playing cards. Find the probability that both the cards are black.
\\
\solution
		%\input{ncert/12/13/2/2/defs.tex}
		\item A box of oranges is inspected by examining three randomly selected oranges drawn without replacement. If all the three oranges are good, the box is approved for sale, otherwise, it is rejected. Find the probability that a box containing 15 oranges out of which 12 are good and 3 are bad ones will be approved for sale.
		\label{ncert/12/13/2/3/defs.tex}
		\item Two balls are drawn at random with replacement from a box containing 10 black and 8 red balls. Find the probability that
		\label{ncert/12/13/2/12}
\begin{enumerate}
\item both balls are red.
\item first ball is black and second is red.
\item one of them is black and other is red.
\end{enumerate}

\item In a hostel, 60\% of the students read Hindi newspaper, 40\% read English newspaper and 20\% read both Hindi and English newspapers. A student is selected at random.
		\label{ncert/12/13/2/15}
\begin{enumerate}
\item Find the probability that she reads neither Hindi nor English newspapers.
\item If she reads Hindi newspaper, find the probability that she reads English newspaper.
\item If she reads English newspaper, find the probability that she reads Hindi newspaper.\\
\end{enumerate}
\item The probability of obtaining an even prime number on each die, when a pair of dice is rolled is 
\begin{enumerate}
    \item $0$ 
    
    \item $\frac{1}{3}$ 
    
    \item $\frac{1}{12}$ 
    
    \item $\frac{1}{36}$ 
\end{enumerate}
\solution
		%\input{ncert/12/13/2/17/defs.tex}
	\item A bag contains 4 red and 4 black balls, another bag contains 2 red and 6 black balls. One of the two bags is selected at random and a ball is drawn from the bag which is found to be red. Find the probability that the ball is drawn from the first bag.
\\
\solution
		%\input{ncert/12/13/3/2/main.tex}
  \item
  Cards with numbers 2 to 101 are placed in a box. A card is selected at random.Find the probability that the card has
\begin{enumerate}[label=(\roman*)]
	\item an even number 
	\item a square number
\end{enumerate}
\solution
%\input{exemplar/10/13/3/32/main.tex}
\item
The king, queen and jack of clubs are removed from a deck of 52 playing cards and then well shuffled. Now one card is drawn at random from the remaining cards.  Determine the probability that the card is
\begin{enumerate}[label=(\roman*)]
\item a club
\item 10 of hearts
\end{enumerate}
\solution
%\input{exemplar/10/13/3/29/main.tex}
\item A team of medical students doing their internship have to assist during surgeries
at a city hospital. The probabilities of surgeries rated as very complex, complex,
routine, simple or very simple are respectively, 0.15, 0.20, 0.31, 0.26, .08. Find
the probabilities that a particular surgery will be rated
\begin{enumerate}
	\item complex or very complex;
	\item neither very complex nor very simple;
	\item routine or complex
	\item routine or simple
\end{enumerate}
\solution
%\input{exemplar/11/16/3/8(1)/main.tex}
\item A card is selected from a pack of 52 cards.
\begin{enumerate}[label=(\alph*)]
    \item How many points are there in the sample space?
    \item Calculate the probability that the card is an ace of spades.
    \item Calculate the probability that the card is (i) an ace and (ii) black card.
\end{enumerate}
\solution
%\input{exemplar/11/16/3/4/main2.tex}
\item The probability that a non leap year selected at random will contain 53 sundays.
\\
\solution
%\input{exemplar/10/13/1/19/main.tex}
\item One of the four persons John, Rita, Aslam or Gurpreet will be promoted next
month. Consequently the sample space consists of four elementary outcomes
S = {John promoted, Rita promoted, Aslam promoted, Gurpreet promoted}
You are told that the chances of John’s promotion is same as that of Gurpreet,
Rita’s chances of promotion are twice as likely as Johns. Aslam’s chances are
four times that of John.
\begin{enumerate}
	\item Determine
	\begin{enumerate}
		\item P (John promoted)
		\item P (Rita promoted)
		\item P (Aslam promoted)
		\item P (Gurpreet promoted)
	\end{enumerate}
	\item If A = {John promoted or Gurpreet promoted}, find P (A).
\end{enumerate}
\solution
%\input{exemplar/11/16/3/10/main.tex}
\item A card is drawn from a deck of 52 cards. Find the probability of getting a king or a heart or a red card.\\
\solution
%\input{exemplar/11/16/3/15/main.tex}
\item The probability that a student will pass his examination is 0.73, the probability of
the student getting a compartment is 0.13, and the probability that the student will
either pass or get compartment is 0.96. State True or False.\\
\solution
%\input{exemplar/11/16/3/31/main.tex}
\item A card is selected from a pack of 52 cards\\
\begin{enumerate}[label=(\alph*)]
\item How many points are there in the sample space?
\item Calculate the probability that the cards is an ace of spades.
\item Calculate the probability that the card is (i) an ace (ii)black card.\\
\end{enumerate}
%\input{ncert/11/16/3/4_1/Prob_4.tex}
\item In a non-leap year, the probability of having 53 tuesdays or 53 wednesdays is\\
\solution
%\input{exemplar/11/16/3/18/main.tex}
\item There are 1000 sealed envelopes in a box, 10 of them contain a cash prize of
Rs 100 each, 100 of them contain a cash prize of Rs 50 each and 200 of them
contain a cash prize of Rs 10 each and rest do not contain any cash prize. If they
are well shuffled and an envelope is picked up out, what is the probability that it
contains no cash prize?\\
\solution
%\input{exemplar/10/13/3/34/main.tex}
\item 
A die is thrown and a card is selected at random from a deck of 52 playing cards. The probability of getting an even number on the die and a spade card.\\
\solution
%\input{exemplar/12/13/3/78/main.tex}
\item
If 4-digit numbers greater than 5,000 are randomly formed from the digits 0, 1, 3, 5, and 7, what is the probability of forming a number divisible by 5 when:
\begin{enumerate}
    \item The digits are repeated?
    \item The repetition of digits is not allowed?
\end{enumerate}
\solution
%\input{ncert/11/16/4/9/main.tex}
\item Consider the probability space $\brak{\Omega, \mathcal{G}, P}$ where $\Omega = [0,2]$ and $\mathcal{G} = \cbrak{\phi, \Omega, [0,1], (1,2]}$. Let $X$ and $Y$ be two functions on $\Omega$ defined as
\begin{align*}
    X(\omega) = 
    \begin{cases}
        1 & \text{if }\omega \in [0, 1]\\
        2 & \text{if }\omega \in (1, 2]
    \end{cases}
\end{align*}
and
\begin{align*}
    Y(\omega) = 
    \begin{cases}
        2 & \text{if }\omega \in [0, 1.5]\\
        3 & \text{if }\omega \in (1.5, 2].
    \end{cases}
\end{align*}
Then which one of the following statements is true?
\begin{enumerate}
    \item [(A)] $X$ is a random variable with respect to $\mathcal{G}$, but $Y$ is not a random variable with respect to $\mathcal{G}$.
    \item [(B)] $Y$ is a random variable with respect to $\mathcal{G}$, but $X$ is not a random variable with respect to $\mathcal{G}$.
    \item [(C)] Neither $X$ nor $Y$ is a random variable with respect to $\mathcal{G}$.
    \item [(D)] Both $X$ and $Y$ are random variables with respect to $\mathcal{G}$.
\end{enumerate} \hfill (GATE ST 2023)\\
\solution
%\input{gate/ST/2023/14/main.tex}
	\item  A die is loaded in such a way that each odd number is twice as likely to occur as
each even number. Find $P(G)$, where $G$ is the event that a number greater than
3 occurs on a single roll of the die.
\\
\solution
		%\input{exemplar/11/16/3/5/main.tex}
	\item All the jacks, queens and kings are removed from a deck of 52 playing cards. The remaining cards are well shuffled and then one card is drawn at random. Giving ace a value 1 similar value for other cards, find the probability that the card has a value 
		\begin{enumerate}
			\item 7
			\item greater than 7
			\item less than 7
		\end{enumerate}
		%\input{exemplar/10/13/3/30/main.tex}
  \item A Lot consists of 48 mobile phones of which 42 are good, 3 have only minor defects and 3 have major defects.Varnika will buy a phone if it is good but the trader will only buy a mobile if it has no major defects. One phone is selected at random from the lot. What is the probability that it is
\begin{enumerate}
	\item acceptable to Varnika?
            \item acceptable to the trader?
\end{enumerate}
\solution
	%\input{exemplar/10/13/3/40/main.tex}
 \item A student says that if you throw a die, it will show up 1 or not 1. Therefore, the probability of getting 1 and the probability of getting 'not 1' each is equal to $\frac{1}{2}$. Is this correct? Give reasons.\\
 \solution
        %\input{exemplar/10/13/2/9/main.tex}
   \item Four candidates A, B, C, D have ap-
plied for the assignment to coach a school cricket
team. If A is twice as likely to be selected as B, and
B and C are given about the same chance of being
selected, while C is twice as likely to be selected
as D, what are the probabilities that
\begin{enumerate}
\item C will be selected?
\item A will not be selected?
\end{enumerate}
	%\input{exemplar/11/16/3/9/main.tex}
 \item A bag contain 24 balls of which $x$ balls are red, $2x$ are white and $3x$ are blue. A ball is selected at random, What is the probability that it is
\begin{enumerate}[label=\alph*)]
\item not red ?
\item white ?
\end{enumerate}
%\input{exemplar/10/13/3/41/main.tex}
If the letters of the word ASSASSINATION are arranged at random. Find the Probability that
\begin{enumerate}[label=(\alph*)]
\item Four $S's$ come consecutively in the word
\item Two  $I's$ and two $N's$ come together
\item All $A's$ are not coming together
\item No two $A's$ are coming together
\end{enumerate}
%\input{exemplar/11/16/3/14/main.tex}
	\item One urn contains two black balls (labelled B1 and B2) and one white ball. A
	second urn contains one black ball and two white balls (labelled W1 and W2).
	Suppose the following experiment is performed. One of the two urns is chosen
	at random. Next a ball is randomly chosen from the urn. Then a second ball is
	chosen at random from the same urn without replacing the first ball.
	
	\begin{enumerate}
	\item What is the probability that two black balls are chosen?
	
	\item What is the probability that two balls of opposite colour are chosen?
	\end{enumerate}
	\solution
	%\input{exemplar/11/16/3/12/main1.tex}
\end{enumerate}

	\item A card is selected from a pack of 52 cards.
 \begin{enumerate}[label=(\alph*)] 
                 \item How many points are there in the sample space?
                 \item Calculate the probability that the card is an ace of spades.
                 \item Calculate the probability that the card is (i) an ace and (ii) black card.
 \end{enumerate}
\solution
		%\begin{table}[H]
	\centering
\begin{tabular}{|c|c|c|}
\hline
Random variable &Value &Definition\\ \hline
\multirow{3}{*}{X} &0 &Slips of Rs 1\\
&1 &Slips of Rs 5\\
&2 &Slips of Rs 13\\ \hline
\multirow{2}{*}{Y} &0 &Box A\\
&1 &Box B\\\hline
\end{tabular}
\caption{}
\label{tab:Distribution}
\end{table}
See \tabref{tab:Distribution}.
\begin{align}
p_{Y}\brak{k}= \begin{cases} 
      \frac{1}{3} & {k=0} \\
      \frac{2}{3 }& {k=1} 
   \end{cases}
   \\
p_{Y|X}\brak{0|0} = \frac{19}{25}\, 
p_{Y|X}\brak{0|1} = \frac{6}{25}\,
p_{Y|X}\brak{1|0} = \frac{45}{50}\,
p_{Y|X}\brak{1|2} = \frac{5}{50}
\end{align}
The desired probability is the probability that a slip drawn at random is marked other than Rs 1,
\begin{align}
&=1-p_X\brak{0}\\
&= p_X(1) + p_X(2)
\end{align}
Using Bayes theorem,
\begin{align}
&= p_Y\brak{0} \times \pr{Y=0 | X=1} + p_Y\brak{1} \times \pr{Y=1|X=2}\\
&=\frac{1}{3} \times \frac{6}{25} + \frac{2}{3} \times \frac{5}{50}\\
&=\frac{11}{75}
\end{align}

\newpage

%\tableofcontents

\bigskip

\renewcommand{\thefigure}{\theenumi}
\renewcommand{\thetable}{\theenumi}
%\renewcommand{\theequation}{\theenumi}

%\begin{abstract}
%%\boldmath
%In this letter, an algorithm for evaluating the exact analytical bit error rate  (BER)  for the piecewise linear (PL) combiner for  multiple relays is presented. Previous results were available only for upto three relays. The algorithm is unique in the sense that  the actual mathematical expressions, that are prohibitively large, need not be explicitly obtained. The diversity gain due to multiple relays is shown through plots of the analytical BER, well supported by simulations. 
%
%\end{abstract}
% IEEEtran.cls defaults to using nonbold math in the Abstract.
% This preserves the distinction between vectors and scalars. However,
% if the journal you are submitting to favors bold math in the abstract,
% then you can use LaTeX's standard command \boldmath at the very start
% of the abstract to achieve this. Many IEEE journals frown on math
% in the abstract anyway.

% Note that keywords are not normally used for peerreview papers.
%\begin{IEEEkeywords}
%Cooperative diversity, decode and forward, piecewise linear
%\end{IEEEkeywords}



% For peer review papers, you can put extra information on the cover
% page as needed:
% \ifCLASSOPTIONpeerreview
% \begin{center} \bfseries EDICS Category: 3-BBND \end{center}
% \fi
%
% For peerreview papers, this IEEEtran command inserts a page break and
% creates the second title. It will be ignored for other modes.
%\IEEEpeerreviewmaketitle




\item Four cards are drawn from a well-shuffled deck of 52 cards. What is the probability of obtaining 3 diamonds and one spade.
\\
\solution
		%\begin{enumerate}[label=\thesection.\arabic*,ref=\thesection.\theenumi]
	\item One card is drawn from a well-shuffled deck of 52 cards. Find the probability of getting
\begin{enumerate}
\item A king of red colour 
\item A face card 
\item A red face card
\item The jack of hearts
\item A spade
\item The queen of diamonds

\end{enumerate}
\solution
		%\input{ncert/10/15/1/14/main.tex}
	\item Five cards—the ten, jack, queen, king and ace of diamonds, are well-shuffled with their face downwards. One card is then picked up at random.
\begin{enumerate}
\item
What is the probability that the card is the queen? 
\item
If the queen is drawn and put aside, what is the probability that the second card picked up is (a) an ace? (b) a queen?\\
\end{enumerate}
\solution
		%\input{ncert/10/15/1/15/defs.tex}
	\item A bag contains $5$ red balls and some blue balls. If the probability of drawing a blue ball is double that if a red ball, determine the number of blue balls in the bag. 
		\\
\solution
		%\input{ncert/10/15/2/3/defs.tex}
	\item A card is selected from a pack of 52 cards.
 \begin{enumerate}[label=(\alph*)] 
                 \item How many points are there in the sample space?
                 \item Calculate the probability that the card is an ace of spades.
                 \item Calculate the probability that the card is (i) an ace and (ii) black card.
 \end{enumerate}
\solution
		%\input{ncert/11/16/3/4/main.tex}
\item Four cards are drawn from a well-shuffled deck of 52 cards. What is the probability of obtaining 3 diamonds and one spade.
\\
\solution
		%\input{ncert/11/16/4/2/defs.tex}
\item In a certain lottery 10,000 tickets are sold and ten equal prizes are awarded. What is the probability of not getting a prize if you buy (a) one ticket (b) two tickets (c) 10 tickets ?	
\\
\solution
		%\input{ncert/11/16/4/4/defs.tex}
		%
\item 
Out of 100 students, two sections of 40 and 60 are formed. If you and your friend are among the 100 students, what is the probability that
\begin{enumerate}
\item you both enter the same section?
\item you both enter the different sections?
\end{enumerate}
\solution
		%\input{ncert/11/16/4/5/defs.tex}
	\item 
The number lock of a suitcase has 4 wheels each labelled with ten digits i.e. from 0 to 9.The lock opens with a sequence of four digits with no repeats.What is the probability of a person getting the right sequence to open the suitcase.
\\
\solution
		%\input{ncert/11/16/4/10/defs.tex}
		%
\item 
Two cards are drawn at random and without replacement from a pack of 52 playing cards. Find the probability that both the cards are black.
\\
\solution
		%\input{ncert/12/13/2/2/defs.tex}
		\item A box of oranges is inspected by examining three randomly selected oranges drawn without replacement. If all the three oranges are good, the box is approved for sale, otherwise, it is rejected. Find the probability that a box containing 15 oranges out of which 12 are good and 3 are bad ones will be approved for sale.
		\label{ncert/12/13/2/3/defs.tex}
		\item Two balls are drawn at random with replacement from a box containing 10 black and 8 red balls. Find the probability that
		\label{ncert/12/13/2/12}
\begin{enumerate}
\item both balls are red.
\item first ball is black and second is red.
\item one of them is black and other is red.
\end{enumerate}

\item In a hostel, 60\% of the students read Hindi newspaper, 40\% read English newspaper and 20\% read both Hindi and English newspapers. A student is selected at random.
		\label{ncert/12/13/2/15}
\begin{enumerate}
\item Find the probability that she reads neither Hindi nor English newspapers.
\item If she reads Hindi newspaper, find the probability that she reads English newspaper.
\item If she reads English newspaper, find the probability that she reads Hindi newspaper.\\
\end{enumerate}
\item The probability of obtaining an even prime number on each die, when a pair of dice is rolled is 
\begin{enumerate}
    \item $0$ 
    
    \item $\frac{1}{3}$ 
    
    \item $\frac{1}{12}$ 
    
    \item $\frac{1}{36}$ 
\end{enumerate}
\solution
		%\input{ncert/12/13/2/17/defs.tex}
	\item A bag contains 4 red and 4 black balls, another bag contains 2 red and 6 black balls. One of the two bags is selected at random and a ball is drawn from the bag which is found to be red. Find the probability that the ball is drawn from the first bag.
\\
\solution
		%\input{ncert/12/13/3/2/main.tex}
  \item
  Cards with numbers 2 to 101 are placed in a box. A card is selected at random.Find the probability that the card has
\begin{enumerate}[label=(\roman*)]
	\item an even number 
	\item a square number
\end{enumerate}
\solution
%\input{exemplar/10/13/3/32/main.tex}
\item
The king, queen and jack of clubs are removed from a deck of 52 playing cards and then well shuffled. Now one card is drawn at random from the remaining cards.  Determine the probability that the card is
\begin{enumerate}[label=(\roman*)]
\item a club
\item 10 of hearts
\end{enumerate}
\solution
%\input{exemplar/10/13/3/29/main.tex}
\item A team of medical students doing their internship have to assist during surgeries
at a city hospital. The probabilities of surgeries rated as very complex, complex,
routine, simple or very simple are respectively, 0.15, 0.20, 0.31, 0.26, .08. Find
the probabilities that a particular surgery will be rated
\begin{enumerate}
	\item complex or very complex;
	\item neither very complex nor very simple;
	\item routine or complex
	\item routine or simple
\end{enumerate}
\solution
%\input{exemplar/11/16/3/8(1)/main.tex}
\item A card is selected from a pack of 52 cards.
\begin{enumerate}[label=(\alph*)]
    \item How many points are there in the sample space?
    \item Calculate the probability that the card is an ace of spades.
    \item Calculate the probability that the card is (i) an ace and (ii) black card.
\end{enumerate}
\solution
%\input{exemplar/11/16/3/4/main2.tex}
\item The probability that a non leap year selected at random will contain 53 sundays.
\\
\solution
%\input{exemplar/10/13/1/19/main.tex}
\item One of the four persons John, Rita, Aslam or Gurpreet will be promoted next
month. Consequently the sample space consists of four elementary outcomes
S = {John promoted, Rita promoted, Aslam promoted, Gurpreet promoted}
You are told that the chances of John’s promotion is same as that of Gurpreet,
Rita’s chances of promotion are twice as likely as Johns. Aslam’s chances are
four times that of John.
\begin{enumerate}
	\item Determine
	\begin{enumerate}
		\item P (John promoted)
		\item P (Rita promoted)
		\item P (Aslam promoted)
		\item P (Gurpreet promoted)
	\end{enumerate}
	\item If A = {John promoted or Gurpreet promoted}, find P (A).
\end{enumerate}
\solution
%\input{exemplar/11/16/3/10/main.tex}
\item A card is drawn from a deck of 52 cards. Find the probability of getting a king or a heart or a red card.\\
\solution
%\input{exemplar/11/16/3/15/main.tex}
\item The probability that a student will pass his examination is 0.73, the probability of
the student getting a compartment is 0.13, and the probability that the student will
either pass or get compartment is 0.96. State True or False.\\
\solution
%\input{exemplar/11/16/3/31/main.tex}
\item A card is selected from a pack of 52 cards\\
\begin{enumerate}[label=(\alph*)]
\item How many points are there in the sample space?
\item Calculate the probability that the cards is an ace of spades.
\item Calculate the probability that the card is (i) an ace (ii)black card.\\
\end{enumerate}
%\input{ncert/11/16/3/4_1/Prob_4.tex}
\item In a non-leap year, the probability of having 53 tuesdays or 53 wednesdays is\\
\solution
%\input{exemplar/11/16/3/18/main.tex}
\item There are 1000 sealed envelopes in a box, 10 of them contain a cash prize of
Rs 100 each, 100 of them contain a cash prize of Rs 50 each and 200 of them
contain a cash prize of Rs 10 each and rest do not contain any cash prize. If they
are well shuffled and an envelope is picked up out, what is the probability that it
contains no cash prize?\\
\solution
%\input{exemplar/10/13/3/34/main.tex}
\item 
A die is thrown and a card is selected at random from a deck of 52 playing cards. The probability of getting an even number on the die and a spade card.\\
\solution
%\input{exemplar/12/13/3/78/main.tex}
\item
If 4-digit numbers greater than 5,000 are randomly formed from the digits 0, 1, 3, 5, and 7, what is the probability of forming a number divisible by 5 when:
\begin{enumerate}
    \item The digits are repeated?
    \item The repetition of digits is not allowed?
\end{enumerate}
\solution
%\input{ncert/11/16/4/9/main.tex}
\item Consider the probability space $\brak{\Omega, \mathcal{G}, P}$ where $\Omega = [0,2]$ and $\mathcal{G} = \cbrak{\phi, \Omega, [0,1], (1,2]}$. Let $X$ and $Y$ be two functions on $\Omega$ defined as
\begin{align*}
    X(\omega) = 
    \begin{cases}
        1 & \text{if }\omega \in [0, 1]\\
        2 & \text{if }\omega \in (1, 2]
    \end{cases}
\end{align*}
and
\begin{align*}
    Y(\omega) = 
    \begin{cases}
        2 & \text{if }\omega \in [0, 1.5]\\
        3 & \text{if }\omega \in (1.5, 2].
    \end{cases}
\end{align*}
Then which one of the following statements is true?
\begin{enumerate}
    \item [(A)] $X$ is a random variable with respect to $\mathcal{G}$, but $Y$ is not a random variable with respect to $\mathcal{G}$.
    \item [(B)] $Y$ is a random variable with respect to $\mathcal{G}$, but $X$ is not a random variable with respect to $\mathcal{G}$.
    \item [(C)] Neither $X$ nor $Y$ is a random variable with respect to $\mathcal{G}$.
    \item [(D)] Both $X$ and $Y$ are random variables with respect to $\mathcal{G}$.
\end{enumerate} \hfill (GATE ST 2023)\\
\solution
%\input{gate/ST/2023/14/main.tex}
	\item  A die is loaded in such a way that each odd number is twice as likely to occur as
each even number. Find $P(G)$, where $G$ is the event that a number greater than
3 occurs on a single roll of the die.
\\
\solution
		%\input{exemplar/11/16/3/5/main.tex}
	\item All the jacks, queens and kings are removed from a deck of 52 playing cards. The remaining cards are well shuffled and then one card is drawn at random. Giving ace a value 1 similar value for other cards, find the probability that the card has a value 
		\begin{enumerate}
			\item 7
			\item greater than 7
			\item less than 7
		\end{enumerate}
		%\input{exemplar/10/13/3/30/main.tex}
  \item A Lot consists of 48 mobile phones of which 42 are good, 3 have only minor defects and 3 have major defects.Varnika will buy a phone if it is good but the trader will only buy a mobile if it has no major defects. One phone is selected at random from the lot. What is the probability that it is
\begin{enumerate}
	\item acceptable to Varnika?
            \item acceptable to the trader?
\end{enumerate}
\solution
	%\input{exemplar/10/13/3/40/main.tex}
 \item A student says that if you throw a die, it will show up 1 or not 1. Therefore, the probability of getting 1 and the probability of getting 'not 1' each is equal to $\frac{1}{2}$. Is this correct? Give reasons.\\
 \solution
        %\input{exemplar/10/13/2/9/main.tex}
   \item Four candidates A, B, C, D have ap-
plied for the assignment to coach a school cricket
team. If A is twice as likely to be selected as B, and
B and C are given about the same chance of being
selected, while C is twice as likely to be selected
as D, what are the probabilities that
\begin{enumerate}
\item C will be selected?
\item A will not be selected?
\end{enumerate}
	%\input{exemplar/11/16/3/9/main.tex}
 \item A bag contain 24 balls of which $x$ balls are red, $2x$ are white and $3x$ are blue. A ball is selected at random, What is the probability that it is
\begin{enumerate}[label=\alph*)]
\item not red ?
\item white ?
\end{enumerate}
%\input{exemplar/10/13/3/41/main.tex}
If the letters of the word ASSASSINATION are arranged at random. Find the Probability that
\begin{enumerate}[label=(\alph*)]
\item Four $S's$ come consecutively in the word
\item Two  $I's$ and two $N's$ come together
\item All $A's$ are not coming together
\item No two $A's$ are coming together
\end{enumerate}
%\input{exemplar/11/16/3/14/main.tex}
	\item One urn contains two black balls (labelled B1 and B2) and one white ball. A
	second urn contains one black ball and two white balls (labelled W1 and W2).
	Suppose the following experiment is performed. One of the two urns is chosen
	at random. Next a ball is randomly chosen from the urn. Then a second ball is
	chosen at random from the same urn without replacing the first ball.
	
	\begin{enumerate}
	\item What is the probability that two black balls are chosen?
	
	\item What is the probability that two balls of opposite colour are chosen?
	\end{enumerate}
	\solution
	%\input{exemplar/11/16/3/12/main1.tex}
\end{enumerate}

\item In a certain lottery 10,000 tickets are sold and ten equal prizes are awarded. What is the probability of not getting a prize if you buy (a) one ticket (b) two tickets (c) 10 tickets ?	
\\
\solution
		%\begin{enumerate}[label=\thesection.\arabic*,ref=\thesection.\theenumi]
	\item One card is drawn from a well-shuffled deck of 52 cards. Find the probability of getting
\begin{enumerate}
\item A king of red colour 
\item A face card 
\item A red face card
\item The jack of hearts
\item A spade
\item The queen of diamonds

\end{enumerate}
\solution
		%\input{ncert/10/15/1/14/main.tex}
	\item Five cards—the ten, jack, queen, king and ace of diamonds, are well-shuffled with their face downwards. One card is then picked up at random.
\begin{enumerate}
\item
What is the probability that the card is the queen? 
\item
If the queen is drawn and put aside, what is the probability that the second card picked up is (a) an ace? (b) a queen?\\
\end{enumerate}
\solution
		%\input{ncert/10/15/1/15/defs.tex}
	\item A bag contains $5$ red balls and some blue balls. If the probability of drawing a blue ball is double that if a red ball, determine the number of blue balls in the bag. 
		\\
\solution
		%\input{ncert/10/15/2/3/defs.tex}
	\item A card is selected from a pack of 52 cards.
 \begin{enumerate}[label=(\alph*)] 
                 \item How many points are there in the sample space?
                 \item Calculate the probability that the card is an ace of spades.
                 \item Calculate the probability that the card is (i) an ace and (ii) black card.
 \end{enumerate}
\solution
		%\input{ncert/11/16/3/4/main.tex}
\item Four cards are drawn from a well-shuffled deck of 52 cards. What is the probability of obtaining 3 diamonds and one spade.
\\
\solution
		%\input{ncert/11/16/4/2/defs.tex}
\item In a certain lottery 10,000 tickets are sold and ten equal prizes are awarded. What is the probability of not getting a prize if you buy (a) one ticket (b) two tickets (c) 10 tickets ?	
\\
\solution
		%\input{ncert/11/16/4/4/defs.tex}
		%
\item 
Out of 100 students, two sections of 40 and 60 are formed. If you and your friend are among the 100 students, what is the probability that
\begin{enumerate}
\item you both enter the same section?
\item you both enter the different sections?
\end{enumerate}
\solution
		%\input{ncert/11/16/4/5/defs.tex}
	\item 
The number lock of a suitcase has 4 wheels each labelled with ten digits i.e. from 0 to 9.The lock opens with a sequence of four digits with no repeats.What is the probability of a person getting the right sequence to open the suitcase.
\\
\solution
		%\input{ncert/11/16/4/10/defs.tex}
		%
\item 
Two cards are drawn at random and without replacement from a pack of 52 playing cards. Find the probability that both the cards are black.
\\
\solution
		%\input{ncert/12/13/2/2/defs.tex}
		\item A box of oranges is inspected by examining three randomly selected oranges drawn without replacement. If all the three oranges are good, the box is approved for sale, otherwise, it is rejected. Find the probability that a box containing 15 oranges out of which 12 are good and 3 are bad ones will be approved for sale.
		\label{ncert/12/13/2/3/defs.tex}
		\item Two balls are drawn at random with replacement from a box containing 10 black and 8 red balls. Find the probability that
		\label{ncert/12/13/2/12}
\begin{enumerate}
\item both balls are red.
\item first ball is black and second is red.
\item one of them is black and other is red.
\end{enumerate}

\item In a hostel, 60\% of the students read Hindi newspaper, 40\% read English newspaper and 20\% read both Hindi and English newspapers. A student is selected at random.
		\label{ncert/12/13/2/15}
\begin{enumerate}
\item Find the probability that she reads neither Hindi nor English newspapers.
\item If she reads Hindi newspaper, find the probability that she reads English newspaper.
\item If she reads English newspaper, find the probability that she reads Hindi newspaper.\\
\end{enumerate}
\item The probability of obtaining an even prime number on each die, when a pair of dice is rolled is 
\begin{enumerate}
    \item $0$ 
    
    \item $\frac{1}{3}$ 
    
    \item $\frac{1}{12}$ 
    
    \item $\frac{1}{36}$ 
\end{enumerate}
\solution
		%\input{ncert/12/13/2/17/defs.tex}
	\item A bag contains 4 red and 4 black balls, another bag contains 2 red and 6 black balls. One of the two bags is selected at random and a ball is drawn from the bag which is found to be red. Find the probability that the ball is drawn from the first bag.
\\
\solution
		%\input{ncert/12/13/3/2/main.tex}
  \item
  Cards with numbers 2 to 101 are placed in a box. A card is selected at random.Find the probability that the card has
\begin{enumerate}[label=(\roman*)]
	\item an even number 
	\item a square number
\end{enumerate}
\solution
%\input{exemplar/10/13/3/32/main.tex}
\item
The king, queen and jack of clubs are removed from a deck of 52 playing cards and then well shuffled. Now one card is drawn at random from the remaining cards.  Determine the probability that the card is
\begin{enumerate}[label=(\roman*)]
\item a club
\item 10 of hearts
\end{enumerate}
\solution
%\input{exemplar/10/13/3/29/main.tex}
\item A team of medical students doing their internship have to assist during surgeries
at a city hospital. The probabilities of surgeries rated as very complex, complex,
routine, simple or very simple are respectively, 0.15, 0.20, 0.31, 0.26, .08. Find
the probabilities that a particular surgery will be rated
\begin{enumerate}
	\item complex or very complex;
	\item neither very complex nor very simple;
	\item routine or complex
	\item routine or simple
\end{enumerate}
\solution
%\input{exemplar/11/16/3/8(1)/main.tex}
\item A card is selected from a pack of 52 cards.
\begin{enumerate}[label=(\alph*)]
    \item How many points are there in the sample space?
    \item Calculate the probability that the card is an ace of spades.
    \item Calculate the probability that the card is (i) an ace and (ii) black card.
\end{enumerate}
\solution
%\input{exemplar/11/16/3/4/main2.tex}
\item The probability that a non leap year selected at random will contain 53 sundays.
\\
\solution
%\input{exemplar/10/13/1/19/main.tex}
\item One of the four persons John, Rita, Aslam or Gurpreet will be promoted next
month. Consequently the sample space consists of four elementary outcomes
S = {John promoted, Rita promoted, Aslam promoted, Gurpreet promoted}
You are told that the chances of John’s promotion is same as that of Gurpreet,
Rita’s chances of promotion are twice as likely as Johns. Aslam’s chances are
four times that of John.
\begin{enumerate}
	\item Determine
	\begin{enumerate}
		\item P (John promoted)
		\item P (Rita promoted)
		\item P (Aslam promoted)
		\item P (Gurpreet promoted)
	\end{enumerate}
	\item If A = {John promoted or Gurpreet promoted}, find P (A).
\end{enumerate}
\solution
%\input{exemplar/11/16/3/10/main.tex}
\item A card is drawn from a deck of 52 cards. Find the probability of getting a king or a heart or a red card.\\
\solution
%\input{exemplar/11/16/3/15/main.tex}
\item The probability that a student will pass his examination is 0.73, the probability of
the student getting a compartment is 0.13, and the probability that the student will
either pass or get compartment is 0.96. State True or False.\\
\solution
%\input{exemplar/11/16/3/31/main.tex}
\item A card is selected from a pack of 52 cards\\
\begin{enumerate}[label=(\alph*)]
\item How many points are there in the sample space?
\item Calculate the probability that the cards is an ace of spades.
\item Calculate the probability that the card is (i) an ace (ii)black card.\\
\end{enumerate}
%\input{ncert/11/16/3/4_1/Prob_4.tex}
\item In a non-leap year, the probability of having 53 tuesdays or 53 wednesdays is\\
\solution
%\input{exemplar/11/16/3/18/main.tex}
\item There are 1000 sealed envelopes in a box, 10 of them contain a cash prize of
Rs 100 each, 100 of them contain a cash prize of Rs 50 each and 200 of them
contain a cash prize of Rs 10 each and rest do not contain any cash prize. If they
are well shuffled and an envelope is picked up out, what is the probability that it
contains no cash prize?\\
\solution
%\input{exemplar/10/13/3/34/main.tex}
\item 
A die is thrown and a card is selected at random from a deck of 52 playing cards. The probability of getting an even number on the die and a spade card.\\
\solution
%\input{exemplar/12/13/3/78/main.tex}
\item
If 4-digit numbers greater than 5,000 are randomly formed from the digits 0, 1, 3, 5, and 7, what is the probability of forming a number divisible by 5 when:
\begin{enumerate}
    \item The digits are repeated?
    \item The repetition of digits is not allowed?
\end{enumerate}
\solution
%\input{ncert/11/16/4/9/main.tex}
\item Consider the probability space $\brak{\Omega, \mathcal{G}, P}$ where $\Omega = [0,2]$ and $\mathcal{G} = \cbrak{\phi, \Omega, [0,1], (1,2]}$. Let $X$ and $Y$ be two functions on $\Omega$ defined as
\begin{align*}
    X(\omega) = 
    \begin{cases}
        1 & \text{if }\omega \in [0, 1]\\
        2 & \text{if }\omega \in (1, 2]
    \end{cases}
\end{align*}
and
\begin{align*}
    Y(\omega) = 
    \begin{cases}
        2 & \text{if }\omega \in [0, 1.5]\\
        3 & \text{if }\omega \in (1.5, 2].
    \end{cases}
\end{align*}
Then which one of the following statements is true?
\begin{enumerate}
    \item [(A)] $X$ is a random variable with respect to $\mathcal{G}$, but $Y$ is not a random variable with respect to $\mathcal{G}$.
    \item [(B)] $Y$ is a random variable with respect to $\mathcal{G}$, but $X$ is not a random variable with respect to $\mathcal{G}$.
    \item [(C)] Neither $X$ nor $Y$ is a random variable with respect to $\mathcal{G}$.
    \item [(D)] Both $X$ and $Y$ are random variables with respect to $\mathcal{G}$.
\end{enumerate} \hfill (GATE ST 2023)\\
\solution
%\input{gate/ST/2023/14/main.tex}
	\item  A die is loaded in such a way that each odd number is twice as likely to occur as
each even number. Find $P(G)$, where $G$ is the event that a number greater than
3 occurs on a single roll of the die.
\\
\solution
		%\input{exemplar/11/16/3/5/main.tex}
	\item All the jacks, queens and kings are removed from a deck of 52 playing cards. The remaining cards are well shuffled and then one card is drawn at random. Giving ace a value 1 similar value for other cards, find the probability that the card has a value 
		\begin{enumerate}
			\item 7
			\item greater than 7
			\item less than 7
		\end{enumerate}
		%\input{exemplar/10/13/3/30/main.tex}
  \item A Lot consists of 48 mobile phones of which 42 are good, 3 have only minor defects and 3 have major defects.Varnika will buy a phone if it is good but the trader will only buy a mobile if it has no major defects. One phone is selected at random from the lot. What is the probability that it is
\begin{enumerate}
	\item acceptable to Varnika?
            \item acceptable to the trader?
\end{enumerate}
\solution
	%\input{exemplar/10/13/3/40/main.tex}
 \item A student says that if you throw a die, it will show up 1 or not 1. Therefore, the probability of getting 1 and the probability of getting 'not 1' each is equal to $\frac{1}{2}$. Is this correct? Give reasons.\\
 \solution
        %\input{exemplar/10/13/2/9/main.tex}
   \item Four candidates A, B, C, D have ap-
plied for the assignment to coach a school cricket
team. If A is twice as likely to be selected as B, and
B and C are given about the same chance of being
selected, while C is twice as likely to be selected
as D, what are the probabilities that
\begin{enumerate}
\item C will be selected?
\item A will not be selected?
\end{enumerate}
	%\input{exemplar/11/16/3/9/main.tex}
 \item A bag contain 24 balls of which $x$ balls are red, $2x$ are white and $3x$ are blue. A ball is selected at random, What is the probability that it is
\begin{enumerate}[label=\alph*)]
\item not red ?
\item white ?
\end{enumerate}
%\input{exemplar/10/13/3/41/main.tex}
If the letters of the word ASSASSINATION are arranged at random. Find the Probability that
\begin{enumerate}[label=(\alph*)]
\item Four $S's$ come consecutively in the word
\item Two  $I's$ and two $N's$ come together
\item All $A's$ are not coming together
\item No two $A's$ are coming together
\end{enumerate}
%\input{exemplar/11/16/3/14/main.tex}
	\item One urn contains two black balls (labelled B1 and B2) and one white ball. A
	second urn contains one black ball and two white balls (labelled W1 and W2).
	Suppose the following experiment is performed. One of the two urns is chosen
	at random. Next a ball is randomly chosen from the urn. Then a second ball is
	chosen at random from the same urn without replacing the first ball.
	
	\begin{enumerate}
	\item What is the probability that two black balls are chosen?
	
	\item What is the probability that two balls of opposite colour are chosen?
	\end{enumerate}
	\solution
	%\input{exemplar/11/16/3/12/main1.tex}
\end{enumerate}

		%
\item 
Out of 100 students, two sections of 40 and 60 are formed. If you and your friend are among the 100 students, what is the probability that
\begin{enumerate}
\item you both enter the same section?
\item you both enter the different sections?
\end{enumerate}
\solution
		%\begin{enumerate}[label=\thesection.\arabic*,ref=\thesection.\theenumi]
	\item One card is drawn from a well-shuffled deck of 52 cards. Find the probability of getting
\begin{enumerate}
\item A king of red colour 
\item A face card 
\item A red face card
\item The jack of hearts
\item A spade
\item The queen of diamonds

\end{enumerate}
\solution
		%\input{ncert/10/15/1/14/main.tex}
	\item Five cards—the ten, jack, queen, king and ace of diamonds, are well-shuffled with their face downwards. One card is then picked up at random.
\begin{enumerate}
\item
What is the probability that the card is the queen? 
\item
If the queen is drawn and put aside, what is the probability that the second card picked up is (a) an ace? (b) a queen?\\
\end{enumerate}
\solution
		%\input{ncert/10/15/1/15/defs.tex}
	\item A bag contains $5$ red balls and some blue balls. If the probability of drawing a blue ball is double that if a red ball, determine the number of blue balls in the bag. 
		\\
\solution
		%\input{ncert/10/15/2/3/defs.tex}
	\item A card is selected from a pack of 52 cards.
 \begin{enumerate}[label=(\alph*)] 
                 \item How many points are there in the sample space?
                 \item Calculate the probability that the card is an ace of spades.
                 \item Calculate the probability that the card is (i) an ace and (ii) black card.
 \end{enumerate}
\solution
		%\input{ncert/11/16/3/4/main.tex}
\item Four cards are drawn from a well-shuffled deck of 52 cards. What is the probability of obtaining 3 diamonds and one spade.
\\
\solution
		%\input{ncert/11/16/4/2/defs.tex}
\item In a certain lottery 10,000 tickets are sold and ten equal prizes are awarded. What is the probability of not getting a prize if you buy (a) one ticket (b) two tickets (c) 10 tickets ?	
\\
\solution
		%\input{ncert/11/16/4/4/defs.tex}
		%
\item 
Out of 100 students, two sections of 40 and 60 are formed. If you and your friend are among the 100 students, what is the probability that
\begin{enumerate}
\item you both enter the same section?
\item you both enter the different sections?
\end{enumerate}
\solution
		%\input{ncert/11/16/4/5/defs.tex}
	\item 
The number lock of a suitcase has 4 wheels each labelled with ten digits i.e. from 0 to 9.The lock opens with a sequence of four digits with no repeats.What is the probability of a person getting the right sequence to open the suitcase.
\\
\solution
		%\input{ncert/11/16/4/10/defs.tex}
		%
\item 
Two cards are drawn at random and without replacement from a pack of 52 playing cards. Find the probability that both the cards are black.
\\
\solution
		%\input{ncert/12/13/2/2/defs.tex}
		\item A box of oranges is inspected by examining three randomly selected oranges drawn without replacement. If all the three oranges are good, the box is approved for sale, otherwise, it is rejected. Find the probability that a box containing 15 oranges out of which 12 are good and 3 are bad ones will be approved for sale.
		\label{ncert/12/13/2/3/defs.tex}
		\item Two balls are drawn at random with replacement from a box containing 10 black and 8 red balls. Find the probability that
		\label{ncert/12/13/2/12}
\begin{enumerate}
\item both balls are red.
\item first ball is black and second is red.
\item one of them is black and other is red.
\end{enumerate}

\item In a hostel, 60\% of the students read Hindi newspaper, 40\% read English newspaper and 20\% read both Hindi and English newspapers. A student is selected at random.
		\label{ncert/12/13/2/15}
\begin{enumerate}
\item Find the probability that she reads neither Hindi nor English newspapers.
\item If she reads Hindi newspaper, find the probability that she reads English newspaper.
\item If she reads English newspaper, find the probability that she reads Hindi newspaper.\\
\end{enumerate}
\item The probability of obtaining an even prime number on each die, when a pair of dice is rolled is 
\begin{enumerate}
    \item $0$ 
    
    \item $\frac{1}{3}$ 
    
    \item $\frac{1}{12}$ 
    
    \item $\frac{1}{36}$ 
\end{enumerate}
\solution
		%\input{ncert/12/13/2/17/defs.tex}
	\item A bag contains 4 red and 4 black balls, another bag contains 2 red and 6 black balls. One of the two bags is selected at random and a ball is drawn from the bag which is found to be red. Find the probability that the ball is drawn from the first bag.
\\
\solution
		%\input{ncert/12/13/3/2/main.tex}
  \item
  Cards with numbers 2 to 101 are placed in a box. A card is selected at random.Find the probability that the card has
\begin{enumerate}[label=(\roman*)]
	\item an even number 
	\item a square number
\end{enumerate}
\solution
%\input{exemplar/10/13/3/32/main.tex}
\item
The king, queen and jack of clubs are removed from a deck of 52 playing cards and then well shuffled. Now one card is drawn at random from the remaining cards.  Determine the probability that the card is
\begin{enumerate}[label=(\roman*)]
\item a club
\item 10 of hearts
\end{enumerate}
\solution
%\input{exemplar/10/13/3/29/main.tex}
\item A team of medical students doing their internship have to assist during surgeries
at a city hospital. The probabilities of surgeries rated as very complex, complex,
routine, simple or very simple are respectively, 0.15, 0.20, 0.31, 0.26, .08. Find
the probabilities that a particular surgery will be rated
\begin{enumerate}
	\item complex or very complex;
	\item neither very complex nor very simple;
	\item routine or complex
	\item routine or simple
\end{enumerate}
\solution
%\input{exemplar/11/16/3/8(1)/main.tex}
\item A card is selected from a pack of 52 cards.
\begin{enumerate}[label=(\alph*)]
    \item How many points are there in the sample space?
    \item Calculate the probability that the card is an ace of spades.
    \item Calculate the probability that the card is (i) an ace and (ii) black card.
\end{enumerate}
\solution
%\input{exemplar/11/16/3/4/main2.tex}
\item The probability that a non leap year selected at random will contain 53 sundays.
\\
\solution
%\input{exemplar/10/13/1/19/main.tex}
\item One of the four persons John, Rita, Aslam or Gurpreet will be promoted next
month. Consequently the sample space consists of four elementary outcomes
S = {John promoted, Rita promoted, Aslam promoted, Gurpreet promoted}
You are told that the chances of John’s promotion is same as that of Gurpreet,
Rita’s chances of promotion are twice as likely as Johns. Aslam’s chances are
four times that of John.
\begin{enumerate}
	\item Determine
	\begin{enumerate}
		\item P (John promoted)
		\item P (Rita promoted)
		\item P (Aslam promoted)
		\item P (Gurpreet promoted)
	\end{enumerate}
	\item If A = {John promoted or Gurpreet promoted}, find P (A).
\end{enumerate}
\solution
%\input{exemplar/11/16/3/10/main.tex}
\item A card is drawn from a deck of 52 cards. Find the probability of getting a king or a heart or a red card.\\
\solution
%\input{exemplar/11/16/3/15/main.tex}
\item The probability that a student will pass his examination is 0.73, the probability of
the student getting a compartment is 0.13, and the probability that the student will
either pass or get compartment is 0.96. State True or False.\\
\solution
%\input{exemplar/11/16/3/31/main.tex}
\item A card is selected from a pack of 52 cards\\
\begin{enumerate}[label=(\alph*)]
\item How many points are there in the sample space?
\item Calculate the probability that the cards is an ace of spades.
\item Calculate the probability that the card is (i) an ace (ii)black card.\\
\end{enumerate}
%\input{ncert/11/16/3/4_1/Prob_4.tex}
\item In a non-leap year, the probability of having 53 tuesdays or 53 wednesdays is\\
\solution
%\input{exemplar/11/16/3/18/main.tex}
\item There are 1000 sealed envelopes in a box, 10 of them contain a cash prize of
Rs 100 each, 100 of them contain a cash prize of Rs 50 each and 200 of them
contain a cash prize of Rs 10 each and rest do not contain any cash prize. If they
are well shuffled and an envelope is picked up out, what is the probability that it
contains no cash prize?\\
\solution
%\input{exemplar/10/13/3/34/main.tex}
\item 
A die is thrown and a card is selected at random from a deck of 52 playing cards. The probability of getting an even number on the die and a spade card.\\
\solution
%\input{exemplar/12/13/3/78/main.tex}
\item
If 4-digit numbers greater than 5,000 are randomly formed from the digits 0, 1, 3, 5, and 7, what is the probability of forming a number divisible by 5 when:
\begin{enumerate}
    \item The digits are repeated?
    \item The repetition of digits is not allowed?
\end{enumerate}
\solution
%\input{ncert/11/16/4/9/main.tex}
\item Consider the probability space $\brak{\Omega, \mathcal{G}, P}$ where $\Omega = [0,2]$ and $\mathcal{G} = \cbrak{\phi, \Omega, [0,1], (1,2]}$. Let $X$ and $Y$ be two functions on $\Omega$ defined as
\begin{align*}
    X(\omega) = 
    \begin{cases}
        1 & \text{if }\omega \in [0, 1]\\
        2 & \text{if }\omega \in (1, 2]
    \end{cases}
\end{align*}
and
\begin{align*}
    Y(\omega) = 
    \begin{cases}
        2 & \text{if }\omega \in [0, 1.5]\\
        3 & \text{if }\omega \in (1.5, 2].
    \end{cases}
\end{align*}
Then which one of the following statements is true?
\begin{enumerate}
    \item [(A)] $X$ is a random variable with respect to $\mathcal{G}$, but $Y$ is not a random variable with respect to $\mathcal{G}$.
    \item [(B)] $Y$ is a random variable with respect to $\mathcal{G}$, but $X$ is not a random variable with respect to $\mathcal{G}$.
    \item [(C)] Neither $X$ nor $Y$ is a random variable with respect to $\mathcal{G}$.
    \item [(D)] Both $X$ and $Y$ are random variables with respect to $\mathcal{G}$.
\end{enumerate} \hfill (GATE ST 2023)\\
\solution
%\input{gate/ST/2023/14/main.tex}
	\item  A die is loaded in such a way that each odd number is twice as likely to occur as
each even number. Find $P(G)$, where $G$ is the event that a number greater than
3 occurs on a single roll of the die.
\\
\solution
		%\input{exemplar/11/16/3/5/main.tex}
	\item All the jacks, queens and kings are removed from a deck of 52 playing cards. The remaining cards are well shuffled and then one card is drawn at random. Giving ace a value 1 similar value for other cards, find the probability that the card has a value 
		\begin{enumerate}
			\item 7
			\item greater than 7
			\item less than 7
		\end{enumerate}
		%\input{exemplar/10/13/3/30/main.tex}
  \item A Lot consists of 48 mobile phones of which 42 are good, 3 have only minor defects and 3 have major defects.Varnika will buy a phone if it is good but the trader will only buy a mobile if it has no major defects. One phone is selected at random from the lot. What is the probability that it is
\begin{enumerate}
	\item acceptable to Varnika?
            \item acceptable to the trader?
\end{enumerate}
\solution
	%\input{exemplar/10/13/3/40/main.tex}
 \item A student says that if you throw a die, it will show up 1 or not 1. Therefore, the probability of getting 1 and the probability of getting 'not 1' each is equal to $\frac{1}{2}$. Is this correct? Give reasons.\\
 \solution
        %\input{exemplar/10/13/2/9/main.tex}
   \item Four candidates A, B, C, D have ap-
plied for the assignment to coach a school cricket
team. If A is twice as likely to be selected as B, and
B and C are given about the same chance of being
selected, while C is twice as likely to be selected
as D, what are the probabilities that
\begin{enumerate}
\item C will be selected?
\item A will not be selected?
\end{enumerate}
	%\input{exemplar/11/16/3/9/main.tex}
 \item A bag contain 24 balls of which $x$ balls are red, $2x$ are white and $3x$ are blue. A ball is selected at random, What is the probability that it is
\begin{enumerate}[label=\alph*)]
\item not red ?
\item white ?
\end{enumerate}
%\input{exemplar/10/13/3/41/main.tex}
If the letters of the word ASSASSINATION are arranged at random. Find the Probability that
\begin{enumerate}[label=(\alph*)]
\item Four $S's$ come consecutively in the word
\item Two  $I's$ and two $N's$ come together
\item All $A's$ are not coming together
\item No two $A's$ are coming together
\end{enumerate}
%\input{exemplar/11/16/3/14/main.tex}
	\item One urn contains two black balls (labelled B1 and B2) and one white ball. A
	second urn contains one black ball and two white balls (labelled W1 and W2).
	Suppose the following experiment is performed. One of the two urns is chosen
	at random. Next a ball is randomly chosen from the urn. Then a second ball is
	chosen at random from the same urn without replacing the first ball.
	
	\begin{enumerate}
	\item What is the probability that two black balls are chosen?
	
	\item What is the probability that two balls of opposite colour are chosen?
	\end{enumerate}
	\solution
	%\input{exemplar/11/16/3/12/main1.tex}
\end{enumerate}

	\item 
The number lock of a suitcase has 4 wheels each labelled with ten digits i.e. from 0 to 9.The lock opens with a sequence of four digits with no repeats.What is the probability of a person getting the right sequence to open the suitcase.
\\
\solution
		%\begin{enumerate}[label=\thesection.\arabic*,ref=\thesection.\theenumi]
	\item One card is drawn from a well-shuffled deck of 52 cards. Find the probability of getting
\begin{enumerate}
\item A king of red colour 
\item A face card 
\item A red face card
\item The jack of hearts
\item A spade
\item The queen of diamonds

\end{enumerate}
\solution
		%\input{ncert/10/15/1/14/main.tex}
	\item Five cards—the ten, jack, queen, king and ace of diamonds, are well-shuffled with their face downwards. One card is then picked up at random.
\begin{enumerate}
\item
What is the probability that the card is the queen? 
\item
If the queen is drawn and put aside, what is the probability that the second card picked up is (a) an ace? (b) a queen?\\
\end{enumerate}
\solution
		%\input{ncert/10/15/1/15/defs.tex}
	\item A bag contains $5$ red balls and some blue balls. If the probability of drawing a blue ball is double that if a red ball, determine the number of blue balls in the bag. 
		\\
\solution
		%\input{ncert/10/15/2/3/defs.tex}
	\item A card is selected from a pack of 52 cards.
 \begin{enumerate}[label=(\alph*)] 
                 \item How many points are there in the sample space?
                 \item Calculate the probability that the card is an ace of spades.
                 \item Calculate the probability that the card is (i) an ace and (ii) black card.
 \end{enumerate}
\solution
		%\input{ncert/11/16/3/4/main.tex}
\item Four cards are drawn from a well-shuffled deck of 52 cards. What is the probability of obtaining 3 diamonds and one spade.
\\
\solution
		%\input{ncert/11/16/4/2/defs.tex}
\item In a certain lottery 10,000 tickets are sold and ten equal prizes are awarded. What is the probability of not getting a prize if you buy (a) one ticket (b) two tickets (c) 10 tickets ?	
\\
\solution
		%\input{ncert/11/16/4/4/defs.tex}
		%
\item 
Out of 100 students, two sections of 40 and 60 are formed. If you and your friend are among the 100 students, what is the probability that
\begin{enumerate}
\item you both enter the same section?
\item you both enter the different sections?
\end{enumerate}
\solution
		%\input{ncert/11/16/4/5/defs.tex}
	\item 
The number lock of a suitcase has 4 wheels each labelled with ten digits i.e. from 0 to 9.The lock opens with a sequence of four digits with no repeats.What is the probability of a person getting the right sequence to open the suitcase.
\\
\solution
		%\input{ncert/11/16/4/10/defs.tex}
		%
\item 
Two cards are drawn at random and without replacement from a pack of 52 playing cards. Find the probability that both the cards are black.
\\
\solution
		%\input{ncert/12/13/2/2/defs.tex}
		\item A box of oranges is inspected by examining three randomly selected oranges drawn without replacement. If all the three oranges are good, the box is approved for sale, otherwise, it is rejected. Find the probability that a box containing 15 oranges out of which 12 are good and 3 are bad ones will be approved for sale.
		\label{ncert/12/13/2/3/defs.tex}
		\item Two balls are drawn at random with replacement from a box containing 10 black and 8 red balls. Find the probability that
		\label{ncert/12/13/2/12}
\begin{enumerate}
\item both balls are red.
\item first ball is black and second is red.
\item one of them is black and other is red.
\end{enumerate}

\item In a hostel, 60\% of the students read Hindi newspaper, 40\% read English newspaper and 20\% read both Hindi and English newspapers. A student is selected at random.
		\label{ncert/12/13/2/15}
\begin{enumerate}
\item Find the probability that she reads neither Hindi nor English newspapers.
\item If she reads Hindi newspaper, find the probability that she reads English newspaper.
\item If she reads English newspaper, find the probability that she reads Hindi newspaper.\\
\end{enumerate}
\item The probability of obtaining an even prime number on each die, when a pair of dice is rolled is 
\begin{enumerate}
    \item $0$ 
    
    \item $\frac{1}{3}$ 
    
    \item $\frac{1}{12}$ 
    
    \item $\frac{1}{36}$ 
\end{enumerate}
\solution
		%\input{ncert/12/13/2/17/defs.tex}
	\item A bag contains 4 red and 4 black balls, another bag contains 2 red and 6 black balls. One of the two bags is selected at random and a ball is drawn from the bag which is found to be red. Find the probability that the ball is drawn from the first bag.
\\
\solution
		%\input{ncert/12/13/3/2/main.tex}
  \item
  Cards with numbers 2 to 101 are placed in a box. A card is selected at random.Find the probability that the card has
\begin{enumerate}[label=(\roman*)]
	\item an even number 
	\item a square number
\end{enumerate}
\solution
%\input{exemplar/10/13/3/32/main.tex}
\item
The king, queen and jack of clubs are removed from a deck of 52 playing cards and then well shuffled. Now one card is drawn at random from the remaining cards.  Determine the probability that the card is
\begin{enumerate}[label=(\roman*)]
\item a club
\item 10 of hearts
\end{enumerate}
\solution
%\input{exemplar/10/13/3/29/main.tex}
\item A team of medical students doing their internship have to assist during surgeries
at a city hospital. The probabilities of surgeries rated as very complex, complex,
routine, simple or very simple are respectively, 0.15, 0.20, 0.31, 0.26, .08. Find
the probabilities that a particular surgery will be rated
\begin{enumerate}
	\item complex or very complex;
	\item neither very complex nor very simple;
	\item routine or complex
	\item routine or simple
\end{enumerate}
\solution
%\input{exemplar/11/16/3/8(1)/main.tex}
\item A card is selected from a pack of 52 cards.
\begin{enumerate}[label=(\alph*)]
    \item How many points are there in the sample space?
    \item Calculate the probability that the card is an ace of spades.
    \item Calculate the probability that the card is (i) an ace and (ii) black card.
\end{enumerate}
\solution
%\input{exemplar/11/16/3/4/main2.tex}
\item The probability that a non leap year selected at random will contain 53 sundays.
\\
\solution
%\input{exemplar/10/13/1/19/main.tex}
\item One of the four persons John, Rita, Aslam or Gurpreet will be promoted next
month. Consequently the sample space consists of four elementary outcomes
S = {John promoted, Rita promoted, Aslam promoted, Gurpreet promoted}
You are told that the chances of John’s promotion is same as that of Gurpreet,
Rita’s chances of promotion are twice as likely as Johns. Aslam’s chances are
four times that of John.
\begin{enumerate}
	\item Determine
	\begin{enumerate}
		\item P (John promoted)
		\item P (Rita promoted)
		\item P (Aslam promoted)
		\item P (Gurpreet promoted)
	\end{enumerate}
	\item If A = {John promoted or Gurpreet promoted}, find P (A).
\end{enumerate}
\solution
%\input{exemplar/11/16/3/10/main.tex}
\item A card is drawn from a deck of 52 cards. Find the probability of getting a king or a heart or a red card.\\
\solution
%\input{exemplar/11/16/3/15/main.tex}
\item The probability that a student will pass his examination is 0.73, the probability of
the student getting a compartment is 0.13, and the probability that the student will
either pass or get compartment is 0.96. State True or False.\\
\solution
%\input{exemplar/11/16/3/31/main.tex}
\item A card is selected from a pack of 52 cards\\
\begin{enumerate}[label=(\alph*)]
\item How many points are there in the sample space?
\item Calculate the probability that the cards is an ace of spades.
\item Calculate the probability that the card is (i) an ace (ii)black card.\\
\end{enumerate}
%\input{ncert/11/16/3/4_1/Prob_4.tex}
\item In a non-leap year, the probability of having 53 tuesdays or 53 wednesdays is\\
\solution
%\input{exemplar/11/16/3/18/main.tex}
\item There are 1000 sealed envelopes in a box, 10 of them contain a cash prize of
Rs 100 each, 100 of them contain a cash prize of Rs 50 each and 200 of them
contain a cash prize of Rs 10 each and rest do not contain any cash prize. If they
are well shuffled and an envelope is picked up out, what is the probability that it
contains no cash prize?\\
\solution
%\input{exemplar/10/13/3/34/main.tex}
\item 
A die is thrown and a card is selected at random from a deck of 52 playing cards. The probability of getting an even number on the die and a spade card.\\
\solution
%\input{exemplar/12/13/3/78/main.tex}
\item
If 4-digit numbers greater than 5,000 are randomly formed from the digits 0, 1, 3, 5, and 7, what is the probability of forming a number divisible by 5 when:
\begin{enumerate}
    \item The digits are repeated?
    \item The repetition of digits is not allowed?
\end{enumerate}
\solution
%\input{ncert/11/16/4/9/main.tex}
\item Consider the probability space $\brak{\Omega, \mathcal{G}, P}$ where $\Omega = [0,2]$ and $\mathcal{G} = \cbrak{\phi, \Omega, [0,1], (1,2]}$. Let $X$ and $Y$ be two functions on $\Omega$ defined as
\begin{align*}
    X(\omega) = 
    \begin{cases}
        1 & \text{if }\omega \in [0, 1]\\
        2 & \text{if }\omega \in (1, 2]
    \end{cases}
\end{align*}
and
\begin{align*}
    Y(\omega) = 
    \begin{cases}
        2 & \text{if }\omega \in [0, 1.5]\\
        3 & \text{if }\omega \in (1.5, 2].
    \end{cases}
\end{align*}
Then which one of the following statements is true?
\begin{enumerate}
    \item [(A)] $X$ is a random variable with respect to $\mathcal{G}$, but $Y$ is not a random variable with respect to $\mathcal{G}$.
    \item [(B)] $Y$ is a random variable with respect to $\mathcal{G}$, but $X$ is not a random variable with respect to $\mathcal{G}$.
    \item [(C)] Neither $X$ nor $Y$ is a random variable with respect to $\mathcal{G}$.
    \item [(D)] Both $X$ and $Y$ are random variables with respect to $\mathcal{G}$.
\end{enumerate} \hfill (GATE ST 2023)\\
\solution
%\input{gate/ST/2023/14/main.tex}
	\item  A die is loaded in such a way that each odd number is twice as likely to occur as
each even number. Find $P(G)$, where $G$ is the event that a number greater than
3 occurs on a single roll of the die.
\\
\solution
		%\input{exemplar/11/16/3/5/main.tex}
	\item All the jacks, queens and kings are removed from a deck of 52 playing cards. The remaining cards are well shuffled and then one card is drawn at random. Giving ace a value 1 similar value for other cards, find the probability that the card has a value 
		\begin{enumerate}
			\item 7
			\item greater than 7
			\item less than 7
		\end{enumerate}
		%\input{exemplar/10/13/3/30/main.tex}
  \item A Lot consists of 48 mobile phones of which 42 are good, 3 have only minor defects and 3 have major defects.Varnika will buy a phone if it is good but the trader will only buy a mobile if it has no major defects. One phone is selected at random from the lot. What is the probability that it is
\begin{enumerate}
	\item acceptable to Varnika?
            \item acceptable to the trader?
\end{enumerate}
\solution
	%\input{exemplar/10/13/3/40/main.tex}
 \item A student says that if you throw a die, it will show up 1 or not 1. Therefore, the probability of getting 1 and the probability of getting 'not 1' each is equal to $\frac{1}{2}$. Is this correct? Give reasons.\\
 \solution
        %\input{exemplar/10/13/2/9/main.tex}
   \item Four candidates A, B, C, D have ap-
plied for the assignment to coach a school cricket
team. If A is twice as likely to be selected as B, and
B and C are given about the same chance of being
selected, while C is twice as likely to be selected
as D, what are the probabilities that
\begin{enumerate}
\item C will be selected?
\item A will not be selected?
\end{enumerate}
	%\input{exemplar/11/16/3/9/main.tex}
 \item A bag contain 24 balls of which $x$ balls are red, $2x$ are white and $3x$ are blue. A ball is selected at random, What is the probability that it is
\begin{enumerate}[label=\alph*)]
\item not red ?
\item white ?
\end{enumerate}
%\input{exemplar/10/13/3/41/main.tex}
If the letters of the word ASSASSINATION are arranged at random. Find the Probability that
\begin{enumerate}[label=(\alph*)]
\item Four $S's$ come consecutively in the word
\item Two  $I's$ and two $N's$ come together
\item All $A's$ are not coming together
\item No two $A's$ are coming together
\end{enumerate}
%\input{exemplar/11/16/3/14/main.tex}
	\item One urn contains two black balls (labelled B1 and B2) and one white ball. A
	second urn contains one black ball and two white balls (labelled W1 and W2).
	Suppose the following experiment is performed. One of the two urns is chosen
	at random. Next a ball is randomly chosen from the urn. Then a second ball is
	chosen at random from the same urn without replacing the first ball.
	
	\begin{enumerate}
	\item What is the probability that two black balls are chosen?
	
	\item What is the probability that two balls of opposite colour are chosen?
	\end{enumerate}
	\solution
	%\input{exemplar/11/16/3/12/main1.tex}
\end{enumerate}

		%
\item 
Two cards are drawn at random and without replacement from a pack of 52 playing cards. Find the probability that both the cards are black.
\\
\solution
		%\begin{enumerate}[label=\thesection.\arabic*,ref=\thesection.\theenumi]
	\item One card is drawn from a well-shuffled deck of 52 cards. Find the probability of getting
\begin{enumerate}
\item A king of red colour 
\item A face card 
\item A red face card
\item The jack of hearts
\item A spade
\item The queen of diamonds

\end{enumerate}
\solution
		%\input{ncert/10/15/1/14/main.tex}
	\item Five cards—the ten, jack, queen, king and ace of diamonds, are well-shuffled with their face downwards. One card is then picked up at random.
\begin{enumerate}
\item
What is the probability that the card is the queen? 
\item
If the queen is drawn and put aside, what is the probability that the second card picked up is (a) an ace? (b) a queen?\\
\end{enumerate}
\solution
		%\input{ncert/10/15/1/15/defs.tex}
	\item A bag contains $5$ red balls and some blue balls. If the probability of drawing a blue ball is double that if a red ball, determine the number of blue balls in the bag. 
		\\
\solution
		%\input{ncert/10/15/2/3/defs.tex}
	\item A card is selected from a pack of 52 cards.
 \begin{enumerate}[label=(\alph*)] 
                 \item How many points are there in the sample space?
                 \item Calculate the probability that the card is an ace of spades.
                 \item Calculate the probability that the card is (i) an ace and (ii) black card.
 \end{enumerate}
\solution
		%\input{ncert/11/16/3/4/main.tex}
\item Four cards are drawn from a well-shuffled deck of 52 cards. What is the probability of obtaining 3 diamonds and one spade.
\\
\solution
		%\input{ncert/11/16/4/2/defs.tex}
\item In a certain lottery 10,000 tickets are sold and ten equal prizes are awarded. What is the probability of not getting a prize if you buy (a) one ticket (b) two tickets (c) 10 tickets ?	
\\
\solution
		%\input{ncert/11/16/4/4/defs.tex}
		%
\item 
Out of 100 students, two sections of 40 and 60 are formed. If you and your friend are among the 100 students, what is the probability that
\begin{enumerate}
\item you both enter the same section?
\item you both enter the different sections?
\end{enumerate}
\solution
		%\input{ncert/11/16/4/5/defs.tex}
	\item 
The number lock of a suitcase has 4 wheels each labelled with ten digits i.e. from 0 to 9.The lock opens with a sequence of four digits with no repeats.What is the probability of a person getting the right sequence to open the suitcase.
\\
\solution
		%\input{ncert/11/16/4/10/defs.tex}
		%
\item 
Two cards are drawn at random and without replacement from a pack of 52 playing cards. Find the probability that both the cards are black.
\\
\solution
		%\input{ncert/12/13/2/2/defs.tex}
		\item A box of oranges is inspected by examining three randomly selected oranges drawn without replacement. If all the three oranges are good, the box is approved for sale, otherwise, it is rejected. Find the probability that a box containing 15 oranges out of which 12 are good and 3 are bad ones will be approved for sale.
		\label{ncert/12/13/2/3/defs.tex}
		\item Two balls are drawn at random with replacement from a box containing 10 black and 8 red balls. Find the probability that
		\label{ncert/12/13/2/12}
\begin{enumerate}
\item both balls are red.
\item first ball is black and second is red.
\item one of them is black and other is red.
\end{enumerate}

\item In a hostel, 60\% of the students read Hindi newspaper, 40\% read English newspaper and 20\% read both Hindi and English newspapers. A student is selected at random.
		\label{ncert/12/13/2/15}
\begin{enumerate}
\item Find the probability that she reads neither Hindi nor English newspapers.
\item If she reads Hindi newspaper, find the probability that she reads English newspaper.
\item If she reads English newspaper, find the probability that she reads Hindi newspaper.\\
\end{enumerate}
\item The probability of obtaining an even prime number on each die, when a pair of dice is rolled is 
\begin{enumerate}
    \item $0$ 
    
    \item $\frac{1}{3}$ 
    
    \item $\frac{1}{12}$ 
    
    \item $\frac{1}{36}$ 
\end{enumerate}
\solution
		%\input{ncert/12/13/2/17/defs.tex}
	\item A bag contains 4 red and 4 black balls, another bag contains 2 red and 6 black balls. One of the two bags is selected at random and a ball is drawn from the bag which is found to be red. Find the probability that the ball is drawn from the first bag.
\\
\solution
		%\input{ncert/12/13/3/2/main.tex}
  \item
  Cards with numbers 2 to 101 are placed in a box. A card is selected at random.Find the probability that the card has
\begin{enumerate}[label=(\roman*)]
	\item an even number 
	\item a square number
\end{enumerate}
\solution
%\input{exemplar/10/13/3/32/main.tex}
\item
The king, queen and jack of clubs are removed from a deck of 52 playing cards and then well shuffled. Now one card is drawn at random from the remaining cards.  Determine the probability that the card is
\begin{enumerate}[label=(\roman*)]
\item a club
\item 10 of hearts
\end{enumerate}
\solution
%\input{exemplar/10/13/3/29/main.tex}
\item A team of medical students doing their internship have to assist during surgeries
at a city hospital. The probabilities of surgeries rated as very complex, complex,
routine, simple or very simple are respectively, 0.15, 0.20, 0.31, 0.26, .08. Find
the probabilities that a particular surgery will be rated
\begin{enumerate}
	\item complex or very complex;
	\item neither very complex nor very simple;
	\item routine or complex
	\item routine or simple
\end{enumerate}
\solution
%\input{exemplar/11/16/3/8(1)/main.tex}
\item A card is selected from a pack of 52 cards.
\begin{enumerate}[label=(\alph*)]
    \item How many points are there in the sample space?
    \item Calculate the probability that the card is an ace of spades.
    \item Calculate the probability that the card is (i) an ace and (ii) black card.
\end{enumerate}
\solution
%\input{exemplar/11/16/3/4/main2.tex}
\item The probability that a non leap year selected at random will contain 53 sundays.
\\
\solution
%\input{exemplar/10/13/1/19/main.tex}
\item One of the four persons John, Rita, Aslam or Gurpreet will be promoted next
month. Consequently the sample space consists of four elementary outcomes
S = {John promoted, Rita promoted, Aslam promoted, Gurpreet promoted}
You are told that the chances of John’s promotion is same as that of Gurpreet,
Rita’s chances of promotion are twice as likely as Johns. Aslam’s chances are
four times that of John.
\begin{enumerate}
	\item Determine
	\begin{enumerate}
		\item P (John promoted)
		\item P (Rita promoted)
		\item P (Aslam promoted)
		\item P (Gurpreet promoted)
	\end{enumerate}
	\item If A = {John promoted or Gurpreet promoted}, find P (A).
\end{enumerate}
\solution
%\input{exemplar/11/16/3/10/main.tex}
\item A card is drawn from a deck of 52 cards. Find the probability of getting a king or a heart or a red card.\\
\solution
%\input{exemplar/11/16/3/15/main.tex}
\item The probability that a student will pass his examination is 0.73, the probability of
the student getting a compartment is 0.13, and the probability that the student will
either pass or get compartment is 0.96. State True or False.\\
\solution
%\input{exemplar/11/16/3/31/main.tex}
\item A card is selected from a pack of 52 cards\\
\begin{enumerate}[label=(\alph*)]
\item How many points are there in the sample space?
\item Calculate the probability that the cards is an ace of spades.
\item Calculate the probability that the card is (i) an ace (ii)black card.\\
\end{enumerate}
%\input{ncert/11/16/3/4_1/Prob_4.tex}
\item In a non-leap year, the probability of having 53 tuesdays or 53 wednesdays is\\
\solution
%\input{exemplar/11/16/3/18/main.tex}
\item There are 1000 sealed envelopes in a box, 10 of them contain a cash prize of
Rs 100 each, 100 of them contain a cash prize of Rs 50 each and 200 of them
contain a cash prize of Rs 10 each and rest do not contain any cash prize. If they
are well shuffled and an envelope is picked up out, what is the probability that it
contains no cash prize?\\
\solution
%\input{exemplar/10/13/3/34/main.tex}
\item 
A die is thrown and a card is selected at random from a deck of 52 playing cards. The probability of getting an even number on the die and a spade card.\\
\solution
%\input{exemplar/12/13/3/78/main.tex}
\item
If 4-digit numbers greater than 5,000 are randomly formed from the digits 0, 1, 3, 5, and 7, what is the probability of forming a number divisible by 5 when:
\begin{enumerate}
    \item The digits are repeated?
    \item The repetition of digits is not allowed?
\end{enumerate}
\solution
%\input{ncert/11/16/4/9/main.tex}
\item Consider the probability space $\brak{\Omega, \mathcal{G}, P}$ where $\Omega = [0,2]$ and $\mathcal{G} = \cbrak{\phi, \Omega, [0,1], (1,2]}$. Let $X$ and $Y$ be two functions on $\Omega$ defined as
\begin{align*}
    X(\omega) = 
    \begin{cases}
        1 & \text{if }\omega \in [0, 1]\\
        2 & \text{if }\omega \in (1, 2]
    \end{cases}
\end{align*}
and
\begin{align*}
    Y(\omega) = 
    \begin{cases}
        2 & \text{if }\omega \in [0, 1.5]\\
        3 & \text{if }\omega \in (1.5, 2].
    \end{cases}
\end{align*}
Then which one of the following statements is true?
\begin{enumerate}
    \item [(A)] $X$ is a random variable with respect to $\mathcal{G}$, but $Y$ is not a random variable with respect to $\mathcal{G}$.
    \item [(B)] $Y$ is a random variable with respect to $\mathcal{G}$, but $X$ is not a random variable with respect to $\mathcal{G}$.
    \item [(C)] Neither $X$ nor $Y$ is a random variable with respect to $\mathcal{G}$.
    \item [(D)] Both $X$ and $Y$ are random variables with respect to $\mathcal{G}$.
\end{enumerate} \hfill (GATE ST 2023)\\
\solution
%\input{gate/ST/2023/14/main.tex}
	\item  A die is loaded in such a way that each odd number is twice as likely to occur as
each even number. Find $P(G)$, where $G$ is the event that a number greater than
3 occurs on a single roll of the die.
\\
\solution
		%\input{exemplar/11/16/3/5/main.tex}
	\item All the jacks, queens and kings are removed from a deck of 52 playing cards. The remaining cards are well shuffled and then one card is drawn at random. Giving ace a value 1 similar value for other cards, find the probability that the card has a value 
		\begin{enumerate}
			\item 7
			\item greater than 7
			\item less than 7
		\end{enumerate}
		%\input{exemplar/10/13/3/30/main.tex}
  \item A Lot consists of 48 mobile phones of which 42 are good, 3 have only minor defects and 3 have major defects.Varnika will buy a phone if it is good but the trader will only buy a mobile if it has no major defects. One phone is selected at random from the lot. What is the probability that it is
\begin{enumerate}
	\item acceptable to Varnika?
            \item acceptable to the trader?
\end{enumerate}
\solution
	%\input{exemplar/10/13/3/40/main.tex}
 \item A student says that if you throw a die, it will show up 1 or not 1. Therefore, the probability of getting 1 and the probability of getting 'not 1' each is equal to $\frac{1}{2}$. Is this correct? Give reasons.\\
 \solution
        %\input{exemplar/10/13/2/9/main.tex}
   \item Four candidates A, B, C, D have ap-
plied for the assignment to coach a school cricket
team. If A is twice as likely to be selected as B, and
B and C are given about the same chance of being
selected, while C is twice as likely to be selected
as D, what are the probabilities that
\begin{enumerate}
\item C will be selected?
\item A will not be selected?
\end{enumerate}
	%\input{exemplar/11/16/3/9/main.tex}
 \item A bag contain 24 balls of which $x$ balls are red, $2x$ are white and $3x$ are blue. A ball is selected at random, What is the probability that it is
\begin{enumerate}[label=\alph*)]
\item not red ?
\item white ?
\end{enumerate}
%\input{exemplar/10/13/3/41/main.tex}
If the letters of the word ASSASSINATION are arranged at random. Find the Probability that
\begin{enumerate}[label=(\alph*)]
\item Four $S's$ come consecutively in the word
\item Two  $I's$ and two $N's$ come together
\item All $A's$ are not coming together
\item No two $A's$ are coming together
\end{enumerate}
%\input{exemplar/11/16/3/14/main.tex}
	\item One urn contains two black balls (labelled B1 and B2) and one white ball. A
	second urn contains one black ball and two white balls (labelled W1 and W2).
	Suppose the following experiment is performed. One of the two urns is chosen
	at random. Next a ball is randomly chosen from the urn. Then a second ball is
	chosen at random from the same urn without replacing the first ball.
	
	\begin{enumerate}
	\item What is the probability that two black balls are chosen?
	
	\item What is the probability that two balls of opposite colour are chosen?
	\end{enumerate}
	\solution
	%\input{exemplar/11/16/3/12/main1.tex}
\end{enumerate}

		\item A box of oranges is inspected by examining three randomly selected oranges drawn without replacement. If all the three oranges are good, the box is approved for sale, otherwise, it is rejected. Find the probability that a box containing 15 oranges out of which 12 are good and 3 are bad ones will be approved for sale.
		\label{ncert/12/13/2/3/defs.tex}
		\item Two balls are drawn at random with replacement from a box containing 10 black and 8 red balls. Find the probability that
		\label{ncert/12/13/2/12}
\begin{enumerate}
\item both balls are red.
\item first ball is black and second is red.
\item one of them is black and other is red.
\end{enumerate}

\item In a hostel, 60\% of the students read Hindi newspaper, 40\% read English newspaper and 20\% read both Hindi and English newspapers. A student is selected at random.
		\label{ncert/12/13/2/15}
\begin{enumerate}
\item Find the probability that she reads neither Hindi nor English newspapers.
\item If she reads Hindi newspaper, find the probability that she reads English newspaper.
\item If she reads English newspaper, find the probability that she reads Hindi newspaper.\\
\end{enumerate}
\item The probability of obtaining an even prime number on each die, when a pair of dice is rolled is 
\begin{enumerate}
    \item $0$ 
    
    \item $\frac{1}{3}$ 
    
    \item $\frac{1}{12}$ 
    
    \item $\frac{1}{36}$ 
\end{enumerate}
\solution
		%\begin{enumerate}[label=\thesection.\arabic*,ref=\thesection.\theenumi]
	\item One card is drawn from a well-shuffled deck of 52 cards. Find the probability of getting
\begin{enumerate}
\item A king of red colour 
\item A face card 
\item A red face card
\item The jack of hearts
\item A spade
\item The queen of diamonds

\end{enumerate}
\solution
		%\input{ncert/10/15/1/14/main.tex}
	\item Five cards—the ten, jack, queen, king and ace of diamonds, are well-shuffled with their face downwards. One card is then picked up at random.
\begin{enumerate}
\item
What is the probability that the card is the queen? 
\item
If the queen is drawn and put aside, what is the probability that the second card picked up is (a) an ace? (b) a queen?\\
\end{enumerate}
\solution
		%\input{ncert/10/15/1/15/defs.tex}
	\item A bag contains $5$ red balls and some blue balls. If the probability of drawing a blue ball is double that if a red ball, determine the number of blue balls in the bag. 
		\\
\solution
		%\input{ncert/10/15/2/3/defs.tex}
	\item A card is selected from a pack of 52 cards.
 \begin{enumerate}[label=(\alph*)] 
                 \item How many points are there in the sample space?
                 \item Calculate the probability that the card is an ace of spades.
                 \item Calculate the probability that the card is (i) an ace and (ii) black card.
 \end{enumerate}
\solution
		%\input{ncert/11/16/3/4/main.tex}
\item Four cards are drawn from a well-shuffled deck of 52 cards. What is the probability of obtaining 3 diamonds and one spade.
\\
\solution
		%\input{ncert/11/16/4/2/defs.tex}
\item In a certain lottery 10,000 tickets are sold and ten equal prizes are awarded. What is the probability of not getting a prize if you buy (a) one ticket (b) two tickets (c) 10 tickets ?	
\\
\solution
		%\input{ncert/11/16/4/4/defs.tex}
		%
\item 
Out of 100 students, two sections of 40 and 60 are formed. If you and your friend are among the 100 students, what is the probability that
\begin{enumerate}
\item you both enter the same section?
\item you both enter the different sections?
\end{enumerate}
\solution
		%\input{ncert/11/16/4/5/defs.tex}
	\item 
The number lock of a suitcase has 4 wheels each labelled with ten digits i.e. from 0 to 9.The lock opens with a sequence of four digits with no repeats.What is the probability of a person getting the right sequence to open the suitcase.
\\
\solution
		%\input{ncert/11/16/4/10/defs.tex}
		%
\item 
Two cards are drawn at random and without replacement from a pack of 52 playing cards. Find the probability that both the cards are black.
\\
\solution
		%\input{ncert/12/13/2/2/defs.tex}
		\item A box of oranges is inspected by examining three randomly selected oranges drawn without replacement. If all the three oranges are good, the box is approved for sale, otherwise, it is rejected. Find the probability that a box containing 15 oranges out of which 12 are good and 3 are bad ones will be approved for sale.
		\label{ncert/12/13/2/3/defs.tex}
		\item Two balls are drawn at random with replacement from a box containing 10 black and 8 red balls. Find the probability that
		\label{ncert/12/13/2/12}
\begin{enumerate}
\item both balls are red.
\item first ball is black and second is red.
\item one of them is black and other is red.
\end{enumerate}

\item In a hostel, 60\% of the students read Hindi newspaper, 40\% read English newspaper and 20\% read both Hindi and English newspapers. A student is selected at random.
		\label{ncert/12/13/2/15}
\begin{enumerate}
\item Find the probability that she reads neither Hindi nor English newspapers.
\item If she reads Hindi newspaper, find the probability that she reads English newspaper.
\item If she reads English newspaper, find the probability that she reads Hindi newspaper.\\
\end{enumerate}
\item The probability of obtaining an even prime number on each die, when a pair of dice is rolled is 
\begin{enumerate}
    \item $0$ 
    
    \item $\frac{1}{3}$ 
    
    \item $\frac{1}{12}$ 
    
    \item $\frac{1}{36}$ 
\end{enumerate}
\solution
		%\input{ncert/12/13/2/17/defs.tex}
	\item A bag contains 4 red and 4 black balls, another bag contains 2 red and 6 black balls. One of the two bags is selected at random and a ball is drawn from the bag which is found to be red. Find the probability that the ball is drawn from the first bag.
\\
\solution
		%\input{ncert/12/13/3/2/main.tex}
  \item
  Cards with numbers 2 to 101 are placed in a box. A card is selected at random.Find the probability that the card has
\begin{enumerate}[label=(\roman*)]
	\item an even number 
	\item a square number
\end{enumerate}
\solution
%\input{exemplar/10/13/3/32/main.tex}
\item
The king, queen and jack of clubs are removed from a deck of 52 playing cards and then well shuffled. Now one card is drawn at random from the remaining cards.  Determine the probability that the card is
\begin{enumerate}[label=(\roman*)]
\item a club
\item 10 of hearts
\end{enumerate}
\solution
%\input{exemplar/10/13/3/29/main.tex}
\item A team of medical students doing their internship have to assist during surgeries
at a city hospital. The probabilities of surgeries rated as very complex, complex,
routine, simple or very simple are respectively, 0.15, 0.20, 0.31, 0.26, .08. Find
the probabilities that a particular surgery will be rated
\begin{enumerate}
	\item complex or very complex;
	\item neither very complex nor very simple;
	\item routine or complex
	\item routine or simple
\end{enumerate}
\solution
%\input{exemplar/11/16/3/8(1)/main.tex}
\item A card is selected from a pack of 52 cards.
\begin{enumerate}[label=(\alph*)]
    \item How many points are there in the sample space?
    \item Calculate the probability that the card is an ace of spades.
    \item Calculate the probability that the card is (i) an ace and (ii) black card.
\end{enumerate}
\solution
%\input{exemplar/11/16/3/4/main2.tex}
\item The probability that a non leap year selected at random will contain 53 sundays.
\\
\solution
%\input{exemplar/10/13/1/19/main.tex}
\item One of the four persons John, Rita, Aslam or Gurpreet will be promoted next
month. Consequently the sample space consists of four elementary outcomes
S = {John promoted, Rita promoted, Aslam promoted, Gurpreet promoted}
You are told that the chances of John’s promotion is same as that of Gurpreet,
Rita’s chances of promotion are twice as likely as Johns. Aslam’s chances are
four times that of John.
\begin{enumerate}
	\item Determine
	\begin{enumerate}
		\item P (John promoted)
		\item P (Rita promoted)
		\item P (Aslam promoted)
		\item P (Gurpreet promoted)
	\end{enumerate}
	\item If A = {John promoted or Gurpreet promoted}, find P (A).
\end{enumerate}
\solution
%\input{exemplar/11/16/3/10/main.tex}
\item A card is drawn from a deck of 52 cards. Find the probability of getting a king or a heart or a red card.\\
\solution
%\input{exemplar/11/16/3/15/main.tex}
\item The probability that a student will pass his examination is 0.73, the probability of
the student getting a compartment is 0.13, and the probability that the student will
either pass or get compartment is 0.96. State True or False.\\
\solution
%\input{exemplar/11/16/3/31/main.tex}
\item A card is selected from a pack of 52 cards\\
\begin{enumerate}[label=(\alph*)]
\item How many points are there in the sample space?
\item Calculate the probability that the cards is an ace of spades.
\item Calculate the probability that the card is (i) an ace (ii)black card.\\
\end{enumerate}
%\input{ncert/11/16/3/4_1/Prob_4.tex}
\item In a non-leap year, the probability of having 53 tuesdays or 53 wednesdays is\\
\solution
%\input{exemplar/11/16/3/18/main.tex}
\item There are 1000 sealed envelopes in a box, 10 of them contain a cash prize of
Rs 100 each, 100 of them contain a cash prize of Rs 50 each and 200 of them
contain a cash prize of Rs 10 each and rest do not contain any cash prize. If they
are well shuffled and an envelope is picked up out, what is the probability that it
contains no cash prize?\\
\solution
%\input{exemplar/10/13/3/34/main.tex}
\item 
A die is thrown and a card is selected at random from a deck of 52 playing cards. The probability of getting an even number on the die and a spade card.\\
\solution
%\input{exemplar/12/13/3/78/main.tex}
\item
If 4-digit numbers greater than 5,000 are randomly formed from the digits 0, 1, 3, 5, and 7, what is the probability of forming a number divisible by 5 when:
\begin{enumerate}
    \item The digits are repeated?
    \item The repetition of digits is not allowed?
\end{enumerate}
\solution
%\input{ncert/11/16/4/9/main.tex}
\item Consider the probability space $\brak{\Omega, \mathcal{G}, P}$ where $\Omega = [0,2]$ and $\mathcal{G} = \cbrak{\phi, \Omega, [0,1], (1,2]}$. Let $X$ and $Y$ be two functions on $\Omega$ defined as
\begin{align*}
    X(\omega) = 
    \begin{cases}
        1 & \text{if }\omega \in [0, 1]\\
        2 & \text{if }\omega \in (1, 2]
    \end{cases}
\end{align*}
and
\begin{align*}
    Y(\omega) = 
    \begin{cases}
        2 & \text{if }\omega \in [0, 1.5]\\
        3 & \text{if }\omega \in (1.5, 2].
    \end{cases}
\end{align*}
Then which one of the following statements is true?
\begin{enumerate}
    \item [(A)] $X$ is a random variable with respect to $\mathcal{G}$, but $Y$ is not a random variable with respect to $\mathcal{G}$.
    \item [(B)] $Y$ is a random variable with respect to $\mathcal{G}$, but $X$ is not a random variable with respect to $\mathcal{G}$.
    \item [(C)] Neither $X$ nor $Y$ is a random variable with respect to $\mathcal{G}$.
    \item [(D)] Both $X$ and $Y$ are random variables with respect to $\mathcal{G}$.
\end{enumerate} \hfill (GATE ST 2023)\\
\solution
%\input{gate/ST/2023/14/main.tex}
	\item  A die is loaded in such a way that each odd number is twice as likely to occur as
each even number. Find $P(G)$, where $G$ is the event that a number greater than
3 occurs on a single roll of the die.
\\
\solution
		%\input{exemplar/11/16/3/5/main.tex}
	\item All the jacks, queens and kings are removed from a deck of 52 playing cards. The remaining cards are well shuffled and then one card is drawn at random. Giving ace a value 1 similar value for other cards, find the probability that the card has a value 
		\begin{enumerate}
			\item 7
			\item greater than 7
			\item less than 7
		\end{enumerate}
		%\input{exemplar/10/13/3/30/main.tex}
  \item A Lot consists of 48 mobile phones of which 42 are good, 3 have only minor defects and 3 have major defects.Varnika will buy a phone if it is good but the trader will only buy a mobile if it has no major defects. One phone is selected at random from the lot. What is the probability that it is
\begin{enumerate}
	\item acceptable to Varnika?
            \item acceptable to the trader?
\end{enumerate}
\solution
	%\input{exemplar/10/13/3/40/main.tex}
 \item A student says that if you throw a die, it will show up 1 or not 1. Therefore, the probability of getting 1 and the probability of getting 'not 1' each is equal to $\frac{1}{2}$. Is this correct? Give reasons.\\
 \solution
        %\input{exemplar/10/13/2/9/main.tex}
   \item Four candidates A, B, C, D have ap-
plied for the assignment to coach a school cricket
team. If A is twice as likely to be selected as B, and
B and C are given about the same chance of being
selected, while C is twice as likely to be selected
as D, what are the probabilities that
\begin{enumerate}
\item C will be selected?
\item A will not be selected?
\end{enumerate}
	%\input{exemplar/11/16/3/9/main.tex}
 \item A bag contain 24 balls of which $x$ balls are red, $2x$ are white and $3x$ are blue. A ball is selected at random, What is the probability that it is
\begin{enumerate}[label=\alph*)]
\item not red ?
\item white ?
\end{enumerate}
%\input{exemplar/10/13/3/41/main.tex}
If the letters of the word ASSASSINATION are arranged at random. Find the Probability that
\begin{enumerate}[label=(\alph*)]
\item Four $S's$ come consecutively in the word
\item Two  $I's$ and two $N's$ come together
\item All $A's$ are not coming together
\item No two $A's$ are coming together
\end{enumerate}
%\input{exemplar/11/16/3/14/main.tex}
	\item One urn contains two black balls (labelled B1 and B2) and one white ball. A
	second urn contains one black ball and two white balls (labelled W1 and W2).
	Suppose the following experiment is performed. One of the two urns is chosen
	at random. Next a ball is randomly chosen from the urn. Then a second ball is
	chosen at random from the same urn without replacing the first ball.
	
	\begin{enumerate}
	\item What is the probability that two black balls are chosen?
	
	\item What is the probability that two balls of opposite colour are chosen?
	\end{enumerate}
	\solution
	%\input{exemplar/11/16/3/12/main1.tex}
\end{enumerate}

	\item A bag contains 4 red and 4 black balls, another bag contains 2 red and 6 black balls. One of the two bags is selected at random and a ball is drawn from the bag which is found to be red. Find the probability that the ball is drawn from the first bag.
\\
\solution
		%\begin{table}[H]
	\centering
\begin{tabular}{|c|c|c|}
\hline
Random variable &Value &Definition\\ \hline
\multirow{3}{*}{X} &0 &Slips of Rs 1\\
&1 &Slips of Rs 5\\
&2 &Slips of Rs 13\\ \hline
\multirow{2}{*}{Y} &0 &Box A\\
&1 &Box B\\\hline
\end{tabular}
\caption{}
\label{tab:Distribution}
\end{table}
See \tabref{tab:Distribution}.
\begin{align}
p_{Y}\brak{k}= \begin{cases} 
      \frac{1}{3} & {k=0} \\
      \frac{2}{3 }& {k=1} 
   \end{cases}
   \\
p_{Y|X}\brak{0|0} = \frac{19}{25}\, 
p_{Y|X}\brak{0|1} = \frac{6}{25}\,
p_{Y|X}\brak{1|0} = \frac{45}{50}\,
p_{Y|X}\brak{1|2} = \frac{5}{50}
\end{align}
The desired probability is the probability that a slip drawn at random is marked other than Rs 1,
\begin{align}
&=1-p_X\brak{0}\\
&= p_X(1) + p_X(2)
\end{align}
Using Bayes theorem,
\begin{align}
&= p_Y\brak{0} \times \pr{Y=0 | X=1} + p_Y\brak{1} \times \pr{Y=1|X=2}\\
&=\frac{1}{3} \times \frac{6}{25} + \frac{2}{3} \times \frac{5}{50}\\
&=\frac{11}{75}
\end{align}

\newpage

%\tableofcontents

\bigskip

\renewcommand{\thefigure}{\theenumi}
\renewcommand{\thetable}{\theenumi}
%\renewcommand{\theequation}{\theenumi}

%\begin{abstract}
%%\boldmath
%In this letter, an algorithm for evaluating the exact analytical bit error rate  (BER)  for the piecewise linear (PL) combiner for  multiple relays is presented. Previous results were available only for upto three relays. The algorithm is unique in the sense that  the actual mathematical expressions, that are prohibitively large, need not be explicitly obtained. The diversity gain due to multiple relays is shown through plots of the analytical BER, well supported by simulations. 
%
%\end{abstract}
% IEEEtran.cls defaults to using nonbold math in the Abstract.
% This preserves the distinction between vectors and scalars. However,
% if the journal you are submitting to favors bold math in the abstract,
% then you can use LaTeX's standard command \boldmath at the very start
% of the abstract to achieve this. Many IEEE journals frown on math
% in the abstract anyway.

% Note that keywords are not normally used for peerreview papers.
%\begin{IEEEkeywords}
%Cooperative diversity, decode and forward, piecewise linear
%\end{IEEEkeywords}



% For peer review papers, you can put extra information on the cover
% page as needed:
% \ifCLASSOPTIONpeerreview
% \begin{center} \bfseries EDICS Category: 3-BBND \end{center}
% \fi
%
% For peerreview papers, this IEEEtran command inserts a page break and
% creates the second title. It will be ignored for other modes.
%\IEEEpeerreviewmaketitle




  \item
  Cards with numbers 2 to 101 are placed in a box. A card is selected at random.Find the probability that the card has
\begin{enumerate}[label=(\roman*)]
	\item an even number 
	\item a square number
\end{enumerate}
\solution
%\begin{table}[H]
	\centering
\begin{tabular}{|c|c|c|}
\hline
Random variable &Value &Definition\\ \hline
\multirow{3}{*}{X} &0 &Slips of Rs 1\\
&1 &Slips of Rs 5\\
&2 &Slips of Rs 13\\ \hline
\multirow{2}{*}{Y} &0 &Box A\\
&1 &Box B\\\hline
\end{tabular}
\caption{}
\label{tab:Distribution}
\end{table}
See \tabref{tab:Distribution}.
\begin{align}
p_{Y}\brak{k}= \begin{cases} 
      \frac{1}{3} & {k=0} \\
      \frac{2}{3 }& {k=1} 
   \end{cases}
   \\
p_{Y|X}\brak{0|0} = \frac{19}{25}\, 
p_{Y|X}\brak{0|1} = \frac{6}{25}\,
p_{Y|X}\brak{1|0} = \frac{45}{50}\,
p_{Y|X}\brak{1|2} = \frac{5}{50}
\end{align}
The desired probability is the probability that a slip drawn at random is marked other than Rs 1,
\begin{align}
&=1-p_X\brak{0}\\
&= p_X(1) + p_X(2)
\end{align}
Using Bayes theorem,
\begin{align}
&= p_Y\brak{0} \times \pr{Y=0 | X=1} + p_Y\brak{1} \times \pr{Y=1|X=2}\\
&=\frac{1}{3} \times \frac{6}{25} + \frac{2}{3} \times \frac{5}{50}\\
&=\frac{11}{75}
\end{align}

\newpage

%\tableofcontents

\bigskip

\renewcommand{\thefigure}{\theenumi}
\renewcommand{\thetable}{\theenumi}
%\renewcommand{\theequation}{\theenumi}

%\begin{abstract}
%%\boldmath
%In this letter, an algorithm for evaluating the exact analytical bit error rate  (BER)  for the piecewise linear (PL) combiner for  multiple relays is presented. Previous results were available only for upto three relays. The algorithm is unique in the sense that  the actual mathematical expressions, that are prohibitively large, need not be explicitly obtained. The diversity gain due to multiple relays is shown through plots of the analytical BER, well supported by simulations. 
%
%\end{abstract}
% IEEEtran.cls defaults to using nonbold math in the Abstract.
% This preserves the distinction between vectors and scalars. However,
% if the journal you are submitting to favors bold math in the abstract,
% then you can use LaTeX's standard command \boldmath at the very start
% of the abstract to achieve this. Many IEEE journals frown on math
% in the abstract anyway.

% Note that keywords are not normally used for peerreview papers.
%\begin{IEEEkeywords}
%Cooperative diversity, decode and forward, piecewise linear
%\end{IEEEkeywords}



% For peer review papers, you can put extra information on the cover
% page as needed:
% \ifCLASSOPTIONpeerreview
% \begin{center} \bfseries EDICS Category: 3-BBND \end{center}
% \fi
%
% For peerreview papers, this IEEEtran command inserts a page break and
% creates the second title. It will be ignored for other modes.
%\IEEEpeerreviewmaketitle




\item
The king, queen and jack of clubs are removed from a deck of 52 playing cards and then well shuffled. Now one card is drawn at random from the remaining cards.  Determine the probability that the card is
\begin{enumerate}[label=(\roman*)]
\item a club
\item 10 of hearts
\end{enumerate}
\solution
%\begin{table}[H]
	\centering
\begin{tabular}{|c|c|c|}
\hline
Random variable &Value &Definition\\ \hline
\multirow{3}{*}{X} &0 &Slips of Rs 1\\
&1 &Slips of Rs 5\\
&2 &Slips of Rs 13\\ \hline
\multirow{2}{*}{Y} &0 &Box A\\
&1 &Box B\\\hline
\end{tabular}
\caption{}
\label{tab:Distribution}
\end{table}
See \tabref{tab:Distribution}.
\begin{align}
p_{Y}\brak{k}= \begin{cases} 
      \frac{1}{3} & {k=0} \\
      \frac{2}{3 }& {k=1} 
   \end{cases}
   \\
p_{Y|X}\brak{0|0} = \frac{19}{25}\, 
p_{Y|X}\brak{0|1} = \frac{6}{25}\,
p_{Y|X}\brak{1|0} = \frac{45}{50}\,
p_{Y|X}\brak{1|2} = \frac{5}{50}
\end{align}
The desired probability is the probability that a slip drawn at random is marked other than Rs 1,
\begin{align}
&=1-p_X\brak{0}\\
&= p_X(1) + p_X(2)
\end{align}
Using Bayes theorem,
\begin{align}
&= p_Y\brak{0} \times \pr{Y=0 | X=1} + p_Y\brak{1} \times \pr{Y=1|X=2}\\
&=\frac{1}{3} \times \frac{6}{25} + \frac{2}{3} \times \frac{5}{50}\\
&=\frac{11}{75}
\end{align}

\newpage

%\tableofcontents

\bigskip

\renewcommand{\thefigure}{\theenumi}
\renewcommand{\thetable}{\theenumi}
%\renewcommand{\theequation}{\theenumi}

%\begin{abstract}
%%\boldmath
%In this letter, an algorithm for evaluating the exact analytical bit error rate  (BER)  for the piecewise linear (PL) combiner for  multiple relays is presented. Previous results were available only for upto three relays. The algorithm is unique in the sense that  the actual mathematical expressions, that are prohibitively large, need not be explicitly obtained. The diversity gain due to multiple relays is shown through plots of the analytical BER, well supported by simulations. 
%
%\end{abstract}
% IEEEtran.cls defaults to using nonbold math in the Abstract.
% This preserves the distinction between vectors and scalars. However,
% if the journal you are submitting to favors bold math in the abstract,
% then you can use LaTeX's standard command \boldmath at the very start
% of the abstract to achieve this. Many IEEE journals frown on math
% in the abstract anyway.

% Note that keywords are not normally used for peerreview papers.
%\begin{IEEEkeywords}
%Cooperative diversity, decode and forward, piecewise linear
%\end{IEEEkeywords}



% For peer review papers, you can put extra information on the cover
% page as needed:
% \ifCLASSOPTIONpeerreview
% \begin{center} \bfseries EDICS Category: 3-BBND \end{center}
% \fi
%
% For peerreview papers, this IEEEtran command inserts a page break and
% creates the second title. It will be ignored for other modes.
%\IEEEpeerreviewmaketitle




\item A team of medical students doing their internship have to assist during surgeries
at a city hospital. The probabilities of surgeries rated as very complex, complex,
routine, simple or very simple are respectively, 0.15, 0.20, 0.31, 0.26, .08. Find
the probabilities that a particular surgery will be rated
\begin{enumerate}
	\item complex or very complex;
	\item neither very complex nor very simple;
	\item routine or complex
	\item routine or simple
\end{enumerate}
\solution
%\begin{table}[H]
	\centering
\begin{tabular}{|c|c|c|}
\hline
Random variable &Value &Definition\\ \hline
\multirow{3}{*}{X} &0 &Slips of Rs 1\\
&1 &Slips of Rs 5\\
&2 &Slips of Rs 13\\ \hline
\multirow{2}{*}{Y} &0 &Box A\\
&1 &Box B\\\hline
\end{tabular}
\caption{}
\label{tab:Distribution}
\end{table}
See \tabref{tab:Distribution}.
\begin{align}
p_{Y}\brak{k}= \begin{cases} 
      \frac{1}{3} & {k=0} \\
      \frac{2}{3 }& {k=1} 
   \end{cases}
   \\
p_{Y|X}\brak{0|0} = \frac{19}{25}\, 
p_{Y|X}\brak{0|1} = \frac{6}{25}\,
p_{Y|X}\brak{1|0} = \frac{45}{50}\,
p_{Y|X}\brak{1|2} = \frac{5}{50}
\end{align}
The desired probability is the probability that a slip drawn at random is marked other than Rs 1,
\begin{align}
&=1-p_X\brak{0}\\
&= p_X(1) + p_X(2)
\end{align}
Using Bayes theorem,
\begin{align}
&= p_Y\brak{0} \times \pr{Y=0 | X=1} + p_Y\brak{1} \times \pr{Y=1|X=2}\\
&=\frac{1}{3} \times \frac{6}{25} + \frac{2}{3} \times \frac{5}{50}\\
&=\frac{11}{75}
\end{align}

\newpage

%\tableofcontents

\bigskip

\renewcommand{\thefigure}{\theenumi}
\renewcommand{\thetable}{\theenumi}
%\renewcommand{\theequation}{\theenumi}

%\begin{abstract}
%%\boldmath
%In this letter, an algorithm for evaluating the exact analytical bit error rate  (BER)  for the piecewise linear (PL) combiner for  multiple relays is presented. Previous results were available only for upto three relays. The algorithm is unique in the sense that  the actual mathematical expressions, that are prohibitively large, need not be explicitly obtained. The diversity gain due to multiple relays is shown through plots of the analytical BER, well supported by simulations. 
%
%\end{abstract}
% IEEEtran.cls defaults to using nonbold math in the Abstract.
% This preserves the distinction between vectors and scalars. However,
% if the journal you are submitting to favors bold math in the abstract,
% then you can use LaTeX's standard command \boldmath at the very start
% of the abstract to achieve this. Many IEEE journals frown on math
% in the abstract anyway.

% Note that keywords are not normally used for peerreview papers.
%\begin{IEEEkeywords}
%Cooperative diversity, decode and forward, piecewise linear
%\end{IEEEkeywords}



% For peer review papers, you can put extra information on the cover
% page as needed:
% \ifCLASSOPTIONpeerreview
% \begin{center} \bfseries EDICS Category: 3-BBND \end{center}
% \fi
%
% For peerreview papers, this IEEEtran command inserts a page break and
% creates the second title. It will be ignored for other modes.
%\IEEEpeerreviewmaketitle




\item A card is selected from a pack of 52 cards.
\begin{enumerate}[label=(\alph*)]
    \item How many points are there in the sample space?
    \item Calculate the probability that the card is an ace of spades.
    \item Calculate the probability that the card is (i) an ace and (ii) black card.
\end{enumerate}
\solution
%Let $X$ be an bernoulli rv defined as in \tabref{tab:exemplar/11/16/3/26}.  Then, 
\begin{equation}
    p =
        \frac{4}{11} 
\end{equation}
\begin{table}[H]
	\centering
	\input{exemplar/11/16/3/26/tables/Table2.tex}
	\caption{}
        \label{tab:exemplar/11/16/3/26}
\end{table}

\item The probability that a non leap year selected at random will contain 53 sundays.
\\
\solution
%\begin{table}[H]
	\centering
\begin{tabular}{|c|c|c|}
\hline
Random variable &Value &Definition\\ \hline
\multirow{3}{*}{X} &0 &Slips of Rs 1\\
&1 &Slips of Rs 5\\
&2 &Slips of Rs 13\\ \hline
\multirow{2}{*}{Y} &0 &Box A\\
&1 &Box B\\\hline
\end{tabular}
\caption{}
\label{tab:Distribution}
\end{table}
See \tabref{tab:Distribution}.
\begin{align}
p_{Y}\brak{k}= \begin{cases} 
      \frac{1}{3} & {k=0} \\
      \frac{2}{3 }& {k=1} 
   \end{cases}
   \\
p_{Y|X}\brak{0|0} = \frac{19}{25}\, 
p_{Y|X}\brak{0|1} = \frac{6}{25}\,
p_{Y|X}\brak{1|0} = \frac{45}{50}\,
p_{Y|X}\brak{1|2} = \frac{5}{50}
\end{align}
The desired probability is the probability that a slip drawn at random is marked other than Rs 1,
\begin{align}
&=1-p_X\brak{0}\\
&= p_X(1) + p_X(2)
\end{align}
Using Bayes theorem,
\begin{align}
&= p_Y\brak{0} \times \pr{Y=0 | X=1} + p_Y\brak{1} \times \pr{Y=1|X=2}\\
&=\frac{1}{3} \times \frac{6}{25} + \frac{2}{3} \times \frac{5}{50}\\
&=\frac{11}{75}
\end{align}

\newpage

%\tableofcontents

\bigskip

\renewcommand{\thefigure}{\theenumi}
\renewcommand{\thetable}{\theenumi}
%\renewcommand{\theequation}{\theenumi}

%\begin{abstract}
%%\boldmath
%In this letter, an algorithm for evaluating the exact analytical bit error rate  (BER)  for the piecewise linear (PL) combiner for  multiple relays is presented. Previous results were available only for upto three relays. The algorithm is unique in the sense that  the actual mathematical expressions, that are prohibitively large, need not be explicitly obtained. The diversity gain due to multiple relays is shown through plots of the analytical BER, well supported by simulations. 
%
%\end{abstract}
% IEEEtran.cls defaults to using nonbold math in the Abstract.
% This preserves the distinction between vectors and scalars. However,
% if the journal you are submitting to favors bold math in the abstract,
% then you can use LaTeX's standard command \boldmath at the very start
% of the abstract to achieve this. Many IEEE journals frown on math
% in the abstract anyway.

% Note that keywords are not normally used for peerreview papers.
%\begin{IEEEkeywords}
%Cooperative diversity, decode and forward, piecewise linear
%\end{IEEEkeywords}



% For peer review papers, you can put extra information on the cover
% page as needed:
% \ifCLASSOPTIONpeerreview
% \begin{center} \bfseries EDICS Category: 3-BBND \end{center}
% \fi
%
% For peerreview papers, this IEEEtran command inserts a page break and
% creates the second title. It will be ignored for other modes.
%\IEEEpeerreviewmaketitle




\item One of the four persons John, Rita, Aslam or Gurpreet will be promoted next
month. Consequently the sample space consists of four elementary outcomes
S = {John promoted, Rita promoted, Aslam promoted, Gurpreet promoted}
You are told that the chances of John’s promotion is same as that of Gurpreet,
Rita’s chances of promotion are twice as likely as Johns. Aslam’s chances are
four times that of John.
\begin{enumerate}
	\item Determine
	\begin{enumerate}
		\item P (John promoted)
		\item P (Rita promoted)
		\item P (Aslam promoted)
		\item P (Gurpreet promoted)
	\end{enumerate}
	\item If A = {John promoted or Gurpreet promoted}, find P (A).
\end{enumerate}
\solution
%\begin{table}[H]
	\centering
\begin{tabular}{|c|c|c|}
\hline
Random variable &Value &Definition\\ \hline
\multirow{3}{*}{X} &0 &Slips of Rs 1\\
&1 &Slips of Rs 5\\
&2 &Slips of Rs 13\\ \hline
\multirow{2}{*}{Y} &0 &Box A\\
&1 &Box B\\\hline
\end{tabular}
\caption{}
\label{tab:Distribution}
\end{table}
See \tabref{tab:Distribution}.
\begin{align}
p_{Y}\brak{k}= \begin{cases} 
      \frac{1}{3} & {k=0} \\
      \frac{2}{3 }& {k=1} 
   \end{cases}
   \\
p_{Y|X}\brak{0|0} = \frac{19}{25}\, 
p_{Y|X}\brak{0|1} = \frac{6}{25}\,
p_{Y|X}\brak{1|0} = \frac{45}{50}\,
p_{Y|X}\brak{1|2} = \frac{5}{50}
\end{align}
The desired probability is the probability that a slip drawn at random is marked other than Rs 1,
\begin{align}
&=1-p_X\brak{0}\\
&= p_X(1) + p_X(2)
\end{align}
Using Bayes theorem,
\begin{align}
&= p_Y\brak{0} \times \pr{Y=0 | X=1} + p_Y\brak{1} \times \pr{Y=1|X=2}\\
&=\frac{1}{3} \times \frac{6}{25} + \frac{2}{3} \times \frac{5}{50}\\
&=\frac{11}{75}
\end{align}

\newpage

%\tableofcontents

\bigskip

\renewcommand{\thefigure}{\theenumi}
\renewcommand{\thetable}{\theenumi}
%\renewcommand{\theequation}{\theenumi}

%\begin{abstract}
%%\boldmath
%In this letter, an algorithm for evaluating the exact analytical bit error rate  (BER)  for the piecewise linear (PL) combiner for  multiple relays is presented. Previous results were available only for upto three relays. The algorithm is unique in the sense that  the actual mathematical expressions, that are prohibitively large, need not be explicitly obtained. The diversity gain due to multiple relays is shown through plots of the analytical BER, well supported by simulations. 
%
%\end{abstract}
% IEEEtran.cls defaults to using nonbold math in the Abstract.
% This preserves the distinction between vectors and scalars. However,
% if the journal you are submitting to favors bold math in the abstract,
% then you can use LaTeX's standard command \boldmath at the very start
% of the abstract to achieve this. Many IEEE journals frown on math
% in the abstract anyway.

% Note that keywords are not normally used for peerreview papers.
%\begin{IEEEkeywords}
%Cooperative diversity, decode and forward, piecewise linear
%\end{IEEEkeywords}



% For peer review papers, you can put extra information on the cover
% page as needed:
% \ifCLASSOPTIONpeerreview
% \begin{center} \bfseries EDICS Category: 3-BBND \end{center}
% \fi
%
% For peerreview papers, this IEEEtran command inserts a page break and
% creates the second title. It will be ignored for other modes.
%\IEEEpeerreviewmaketitle




\item A card is drawn from a deck of 52 cards. Find the probability of getting a king or a heart or a red card.\\
\solution
%\begin{table}[H]
	\centering
\begin{tabular}{|c|c|c|}
\hline
Random variable &Value &Definition\\ \hline
\multirow{3}{*}{X} &0 &Slips of Rs 1\\
&1 &Slips of Rs 5\\
&2 &Slips of Rs 13\\ \hline
\multirow{2}{*}{Y} &0 &Box A\\
&1 &Box B\\\hline
\end{tabular}
\caption{}
\label{tab:Distribution}
\end{table}
See \tabref{tab:Distribution}.
\begin{align}
p_{Y}\brak{k}= \begin{cases} 
      \frac{1}{3} & {k=0} \\
      \frac{2}{3 }& {k=1} 
   \end{cases}
   \\
p_{Y|X}\brak{0|0} = \frac{19}{25}\, 
p_{Y|X}\brak{0|1} = \frac{6}{25}\,
p_{Y|X}\brak{1|0} = \frac{45}{50}\,
p_{Y|X}\brak{1|2} = \frac{5}{50}
\end{align}
The desired probability is the probability that a slip drawn at random is marked other than Rs 1,
\begin{align}
&=1-p_X\brak{0}\\
&= p_X(1) + p_X(2)
\end{align}
Using Bayes theorem,
\begin{align}
&= p_Y\brak{0} \times \pr{Y=0 | X=1} + p_Y\brak{1} \times \pr{Y=1|X=2}\\
&=\frac{1}{3} \times \frac{6}{25} + \frac{2}{3} \times \frac{5}{50}\\
&=\frac{11}{75}
\end{align}

\newpage

%\tableofcontents

\bigskip

\renewcommand{\thefigure}{\theenumi}
\renewcommand{\thetable}{\theenumi}
%\renewcommand{\theequation}{\theenumi}

%\begin{abstract}
%%\boldmath
%In this letter, an algorithm for evaluating the exact analytical bit error rate  (BER)  for the piecewise linear (PL) combiner for  multiple relays is presented. Previous results were available only for upto three relays. The algorithm is unique in the sense that  the actual mathematical expressions, that are prohibitively large, need not be explicitly obtained. The diversity gain due to multiple relays is shown through plots of the analytical BER, well supported by simulations. 
%
%\end{abstract}
% IEEEtran.cls defaults to using nonbold math in the Abstract.
% This preserves the distinction between vectors and scalars. However,
% if the journal you are submitting to favors bold math in the abstract,
% then you can use LaTeX's standard command \boldmath at the very start
% of the abstract to achieve this. Many IEEE journals frown on math
% in the abstract anyway.

% Note that keywords are not normally used for peerreview papers.
%\begin{IEEEkeywords}
%Cooperative diversity, decode and forward, piecewise linear
%\end{IEEEkeywords}



% For peer review papers, you can put extra information on the cover
% page as needed:
% \ifCLASSOPTIONpeerreview
% \begin{center} \bfseries EDICS Category: 3-BBND \end{center}
% \fi
%
% For peerreview papers, this IEEEtran command inserts a page break and
% creates the second title. It will be ignored for other modes.
%\IEEEpeerreviewmaketitle




\item The probability that a student will pass his examination is 0.73, the probability of
the student getting a compartment is 0.13, and the probability that the student will
either pass or get compartment is 0.96. State True or False.\\
\solution
%\begin{table}[H]
	\centering
\begin{tabular}{|c|c|c|}
\hline
Random variable &Value &Definition\\ \hline
\multirow{3}{*}{X} &0 &Slips of Rs 1\\
&1 &Slips of Rs 5\\
&2 &Slips of Rs 13\\ \hline
\multirow{2}{*}{Y} &0 &Box A\\
&1 &Box B\\\hline
\end{tabular}
\caption{}
\label{tab:Distribution}
\end{table}
See \tabref{tab:Distribution}.
\begin{align}
p_{Y}\brak{k}= \begin{cases} 
      \frac{1}{3} & {k=0} \\
      \frac{2}{3 }& {k=1} 
   \end{cases}
   \\
p_{Y|X}\brak{0|0} = \frac{19}{25}\, 
p_{Y|X}\brak{0|1} = \frac{6}{25}\,
p_{Y|X}\brak{1|0} = \frac{45}{50}\,
p_{Y|X}\brak{1|2} = \frac{5}{50}
\end{align}
The desired probability is the probability that a slip drawn at random is marked other than Rs 1,
\begin{align}
&=1-p_X\brak{0}\\
&= p_X(1) + p_X(2)
\end{align}
Using Bayes theorem,
\begin{align}
&= p_Y\brak{0} \times \pr{Y=0 | X=1} + p_Y\brak{1} \times \pr{Y=1|X=2}\\
&=\frac{1}{3} \times \frac{6}{25} + \frac{2}{3} \times \frac{5}{50}\\
&=\frac{11}{75}
\end{align}

\newpage

%\tableofcontents

\bigskip

\renewcommand{\thefigure}{\theenumi}
\renewcommand{\thetable}{\theenumi}
%\renewcommand{\theequation}{\theenumi}

%\begin{abstract}
%%\boldmath
%In this letter, an algorithm for evaluating the exact analytical bit error rate  (BER)  for the piecewise linear (PL) combiner for  multiple relays is presented. Previous results were available only for upto three relays. The algorithm is unique in the sense that  the actual mathematical expressions, that are prohibitively large, need not be explicitly obtained. The diversity gain due to multiple relays is shown through plots of the analytical BER, well supported by simulations. 
%
%\end{abstract}
% IEEEtran.cls defaults to using nonbold math in the Abstract.
% This preserves the distinction between vectors and scalars. However,
% if the journal you are submitting to favors bold math in the abstract,
% then you can use LaTeX's standard command \boldmath at the very start
% of the abstract to achieve this. Many IEEE journals frown on math
% in the abstract anyway.

% Note that keywords are not normally used for peerreview papers.
%\begin{IEEEkeywords}
%Cooperative diversity, decode and forward, piecewise linear
%\end{IEEEkeywords}



% For peer review papers, you can put extra information on the cover
% page as needed:
% \ifCLASSOPTIONpeerreview
% \begin{center} \bfseries EDICS Category: 3-BBND \end{center}
% \fi
%
% For peerreview papers, this IEEEtran command inserts a page break and
% creates the second title. It will be ignored for other modes.
%\IEEEpeerreviewmaketitle




\item A card is selected from a pack of 52 cards\\
\begin{enumerate}[label=(\alph*)]
\item How many points are there in the sample space?
\item Calculate the probability that the cards is an ace of spades.
\item Calculate the probability that the card is (i) an ace (ii)black card.\\
\end{enumerate}
%\input{ncert/11/16/3/4_1/Prob_4.tex}
\item In a non-leap year, the probability of having 53 tuesdays or 53 wednesdays is\\
\solution
%A non-leap year has a total of 365 days, and a week has 7 days.\\
So it can be expressed as 
\begin{align}
365\text{days} &=52\times 7+1 \text{day}
\end{align}
$\implies$ 52 tuesdays or wednesdays\\
Random variable X denotes the days of a week
\begin{align}
p_X\brak{k}&=\frac{1}{7}; \quad \brak{1<k<7}
\end{align}
So the probability of extra day being tuesday or wednesday is
\begin{align}
p_X\brak{3}+p_X\brak{4}&=\frac{1}{7}+\frac{1}{7}=\frac{2}{7}
\end{align}



\item There are 1000 sealed envelopes in a box, 10 of them contain a cash prize of
Rs 100 each, 100 of them contain a cash prize of Rs 50 each and 200 of them
contain a cash prize of Rs 10 each and rest do not contain any cash prize. If they
are well shuffled and an envelope is picked up out, what is the probability that it
contains no cash prize?\\
\solution
%\begin{table}[H]
	\centering
\begin{tabular}{|c|c|c|}
\hline
Random variable &Value &Definition\\ \hline
\multirow{3}{*}{X} &0 &Slips of Rs 1\\
&1 &Slips of Rs 5\\
&2 &Slips of Rs 13\\ \hline
\multirow{2}{*}{Y} &0 &Box A\\
&1 &Box B\\\hline
\end{tabular}
\caption{}
\label{tab:Distribution}
\end{table}
See \tabref{tab:Distribution}.
\begin{align}
p_{Y}\brak{k}= \begin{cases} 
      \frac{1}{3} & {k=0} \\
      \frac{2}{3 }& {k=1} 
   \end{cases}
   \\
p_{Y|X}\brak{0|0} = \frac{19}{25}\, 
p_{Y|X}\brak{0|1} = \frac{6}{25}\,
p_{Y|X}\brak{1|0} = \frac{45}{50}\,
p_{Y|X}\brak{1|2} = \frac{5}{50}
\end{align}
The desired probability is the probability that a slip drawn at random is marked other than Rs 1,
\begin{align}
&=1-p_X\brak{0}\\
&= p_X(1) + p_X(2)
\end{align}
Using Bayes theorem,
\begin{align}
&= p_Y\brak{0} \times \pr{Y=0 | X=1} + p_Y\brak{1} \times \pr{Y=1|X=2}\\
&=\frac{1}{3} \times \frac{6}{25} + \frac{2}{3} \times \frac{5}{50}\\
&=\frac{11}{75}
\end{align}

\newpage

%\tableofcontents

\bigskip

\renewcommand{\thefigure}{\theenumi}
\renewcommand{\thetable}{\theenumi}
%\renewcommand{\theequation}{\theenumi}

%\begin{abstract}
%%\boldmath
%In this letter, an algorithm for evaluating the exact analytical bit error rate  (BER)  for the piecewise linear (PL) combiner for  multiple relays is presented. Previous results were available only for upto three relays. The algorithm is unique in the sense that  the actual mathematical expressions, that are prohibitively large, need not be explicitly obtained. The diversity gain due to multiple relays is shown through plots of the analytical BER, well supported by simulations. 
%
%\end{abstract}
% IEEEtran.cls defaults to using nonbold math in the Abstract.
% This preserves the distinction between vectors and scalars. However,
% if the journal you are submitting to favors bold math in the abstract,
% then you can use LaTeX's standard command \boldmath at the very start
% of the abstract to achieve this. Many IEEE journals frown on math
% in the abstract anyway.

% Note that keywords are not normally used for peerreview papers.
%\begin{IEEEkeywords}
%Cooperative diversity, decode and forward, piecewise linear
%\end{IEEEkeywords}



% For peer review papers, you can put extra information on the cover
% page as needed:
% \ifCLASSOPTIONpeerreview
% \begin{center} \bfseries EDICS Category: 3-BBND \end{center}
% \fi
%
% For peerreview papers, this IEEEtran command inserts a page break and
% creates the second title. It will be ignored for other modes.
%\IEEEpeerreviewmaketitle




\item 
A die is thrown and a card is selected at random from a deck of 52 playing cards. The probability of getting an even number on the die and a spade card.\\
\solution
%\begin{table}[H]
	\centering
\begin{tabular}{|c|c|c|}
\hline
Random variable &Value &Definition\\ \hline
\multirow{3}{*}{X} &0 &Slips of Rs 1\\
&1 &Slips of Rs 5\\
&2 &Slips of Rs 13\\ \hline
\multirow{2}{*}{Y} &0 &Box A\\
&1 &Box B\\\hline
\end{tabular}
\caption{}
\label{tab:Distribution}
\end{table}
See \tabref{tab:Distribution}.
\begin{align}
p_{Y}\brak{k}= \begin{cases} 
      \frac{1}{3} & {k=0} \\
      \frac{2}{3 }& {k=1} 
   \end{cases}
   \\
p_{Y|X}\brak{0|0} = \frac{19}{25}\, 
p_{Y|X}\brak{0|1} = \frac{6}{25}\,
p_{Y|X}\brak{1|0} = \frac{45}{50}\,
p_{Y|X}\brak{1|2} = \frac{5}{50}
\end{align}
The desired probability is the probability that a slip drawn at random is marked other than Rs 1,
\begin{align}
&=1-p_X\brak{0}\\
&= p_X(1) + p_X(2)
\end{align}
Using Bayes theorem,
\begin{align}
&= p_Y\brak{0} \times \pr{Y=0 | X=1} + p_Y\brak{1} \times \pr{Y=1|X=2}\\
&=\frac{1}{3} \times \frac{6}{25} + \frac{2}{3} \times \frac{5}{50}\\
&=\frac{11}{75}
\end{align}

\newpage

%\tableofcontents

\bigskip

\renewcommand{\thefigure}{\theenumi}
\renewcommand{\thetable}{\theenumi}
%\renewcommand{\theequation}{\theenumi}

%\begin{abstract}
%%\boldmath
%In this letter, an algorithm for evaluating the exact analytical bit error rate  (BER)  for the piecewise linear (PL) combiner for  multiple relays is presented. Previous results were available only for upto three relays. The algorithm is unique in the sense that  the actual mathematical expressions, that are prohibitively large, need not be explicitly obtained. The diversity gain due to multiple relays is shown through plots of the analytical BER, well supported by simulations. 
%
%\end{abstract}
% IEEEtran.cls defaults to using nonbold math in the Abstract.
% This preserves the distinction between vectors and scalars. However,
% if the journal you are submitting to favors bold math in the abstract,
% then you can use LaTeX's standard command \boldmath at the very start
% of the abstract to achieve this. Many IEEE journals frown on math
% in the abstract anyway.

% Note that keywords are not normally used for peerreview papers.
%\begin{IEEEkeywords}
%Cooperative diversity, decode and forward, piecewise linear
%\end{IEEEkeywords}



% For peer review papers, you can put extra information on the cover
% page as needed:
% \ifCLASSOPTIONpeerreview
% \begin{center} \bfseries EDICS Category: 3-BBND \end{center}
% \fi
%
% For peerreview papers, this IEEEtran command inserts a page break and
% creates the second title. It will be ignored for other modes.
%\IEEEpeerreviewmaketitle




\item
If 4-digit numbers greater than 5,000 are randomly formed from the digits 0, 1, 3, 5, and 7, what is the probability of forming a number divisible by 5 when:
\begin{enumerate}
    \item The digits are repeated?
    \item The repetition of digits is not allowed?
\end{enumerate}
\solution
%\begin{table}[H]
	\centering
\begin{tabular}{|c|c|c|}
\hline
Random variable &Value &Definition\\ \hline
\multirow{3}{*}{X} &0 &Slips of Rs 1\\
&1 &Slips of Rs 5\\
&2 &Slips of Rs 13\\ \hline
\multirow{2}{*}{Y} &0 &Box A\\
&1 &Box B\\\hline
\end{tabular}
\caption{}
\label{tab:Distribution}
\end{table}
See \tabref{tab:Distribution}.
\begin{align}
p_{Y}\brak{k}= \begin{cases} 
      \frac{1}{3} & {k=0} \\
      \frac{2}{3 }& {k=1} 
   \end{cases}
   \\
p_{Y|X}\brak{0|0} = \frac{19}{25}\, 
p_{Y|X}\brak{0|1} = \frac{6}{25}\,
p_{Y|X}\brak{1|0} = \frac{45}{50}\,
p_{Y|X}\brak{1|2} = \frac{5}{50}
\end{align}
The desired probability is the probability that a slip drawn at random is marked other than Rs 1,
\begin{align}
&=1-p_X\brak{0}\\
&= p_X(1) + p_X(2)
\end{align}
Using Bayes theorem,
\begin{align}
&= p_Y\brak{0} \times \pr{Y=0 | X=1} + p_Y\brak{1} \times \pr{Y=1|X=2}\\
&=\frac{1}{3} \times \frac{6}{25} + \frac{2}{3} \times \frac{5}{50}\\
&=\frac{11}{75}
\end{align}

\newpage

%\tableofcontents

\bigskip

\renewcommand{\thefigure}{\theenumi}
\renewcommand{\thetable}{\theenumi}
%\renewcommand{\theequation}{\theenumi}

%\begin{abstract}
%%\boldmath
%In this letter, an algorithm for evaluating the exact analytical bit error rate  (BER)  for the piecewise linear (PL) combiner for  multiple relays is presented. Previous results were available only for upto three relays. The algorithm is unique in the sense that  the actual mathematical expressions, that are prohibitively large, need not be explicitly obtained. The diversity gain due to multiple relays is shown through plots of the analytical BER, well supported by simulations. 
%
%\end{abstract}
% IEEEtran.cls defaults to using nonbold math in the Abstract.
% This preserves the distinction between vectors and scalars. However,
% if the journal you are submitting to favors bold math in the abstract,
% then you can use LaTeX's standard command \boldmath at the very start
% of the abstract to achieve this. Many IEEE journals frown on math
% in the abstract anyway.

% Note that keywords are not normally used for peerreview papers.
%\begin{IEEEkeywords}
%Cooperative diversity, decode and forward, piecewise linear
%\end{IEEEkeywords}



% For peer review papers, you can put extra information on the cover
% page as needed:
% \ifCLASSOPTIONpeerreview
% \begin{center} \bfseries EDICS Category: 3-BBND \end{center}
% \fi
%
% For peerreview papers, this IEEEtran command inserts a page break and
% creates the second title. It will be ignored for other modes.
%\IEEEpeerreviewmaketitle




\item Consider the probability space $\brak{\Omega, \mathcal{G}, P}$ where $\Omega = [0,2]$ and $\mathcal{G} = \cbrak{\phi, \Omega, [0,1], (1,2]}$. Let $X$ and $Y$ be two functions on $\Omega$ defined as
\begin{align*}
    X(\omega) = 
    \begin{cases}
        1 & \text{if }\omega \in [0, 1]\\
        2 & \text{if }\omega \in (1, 2]
    \end{cases}
\end{align*}
and
\begin{align*}
    Y(\omega) = 
    \begin{cases}
        2 & \text{if }\omega \in [0, 1.5]\\
        3 & \text{if }\omega \in (1.5, 2].
    \end{cases}
\end{align*}
Then which one of the following statements is true?
\begin{enumerate}
    \item [(A)] $X$ is a random variable with respect to $\mathcal{G}$, but $Y$ is not a random variable with respect to $\mathcal{G}$.
    \item [(B)] $Y$ is a random variable with respect to $\mathcal{G}$, but $X$ is not a random variable with respect to $\mathcal{G}$.
    \item [(C)] Neither $X$ nor $Y$ is a random variable with respect to $\mathcal{G}$.
    \item [(D)] Both $X$ and $Y$ are random variables with respect to $\mathcal{G}$.
\end{enumerate} \hfill (GATE ST 2023)\\
\solution
%\begin{table}[H]
	\centering
\begin{tabular}{|c|c|c|}
\hline
Random variable &Value &Definition\\ \hline
\multirow{3}{*}{X} &0 &Slips of Rs 1\\
&1 &Slips of Rs 5\\
&2 &Slips of Rs 13\\ \hline
\multirow{2}{*}{Y} &0 &Box A\\
&1 &Box B\\\hline
\end{tabular}
\caption{}
\label{tab:Distribution}
\end{table}
See \tabref{tab:Distribution}.
\begin{align}
p_{Y}\brak{k}= \begin{cases} 
      \frac{1}{3} & {k=0} \\
      \frac{2}{3 }& {k=1} 
   \end{cases}
   \\
p_{Y|X}\brak{0|0} = \frac{19}{25}\, 
p_{Y|X}\brak{0|1} = \frac{6}{25}\,
p_{Y|X}\brak{1|0} = \frac{45}{50}\,
p_{Y|X}\brak{1|2} = \frac{5}{50}
\end{align}
The desired probability is the probability that a slip drawn at random is marked other than Rs 1,
\begin{align}
&=1-p_X\brak{0}\\
&= p_X(1) + p_X(2)
\end{align}
Using Bayes theorem,
\begin{align}
&= p_Y\brak{0} \times \pr{Y=0 | X=1} + p_Y\brak{1} \times \pr{Y=1|X=2}\\
&=\frac{1}{3} \times \frac{6}{25} + \frac{2}{3} \times \frac{5}{50}\\
&=\frac{11}{75}
\end{align}

\newpage

%\tableofcontents

\bigskip

\renewcommand{\thefigure}{\theenumi}
\renewcommand{\thetable}{\theenumi}
%\renewcommand{\theequation}{\theenumi}

%\begin{abstract}
%%\boldmath
%In this letter, an algorithm for evaluating the exact analytical bit error rate  (BER)  for the piecewise linear (PL) combiner for  multiple relays is presented. Previous results were available only for upto three relays. The algorithm is unique in the sense that  the actual mathematical expressions, that are prohibitively large, need not be explicitly obtained. The diversity gain due to multiple relays is shown through plots of the analytical BER, well supported by simulations. 
%
%\end{abstract}
% IEEEtran.cls defaults to using nonbold math in the Abstract.
% This preserves the distinction between vectors and scalars. However,
% if the journal you are submitting to favors bold math in the abstract,
% then you can use LaTeX's standard command \boldmath at the very start
% of the abstract to achieve this. Many IEEE journals frown on math
% in the abstract anyway.

% Note that keywords are not normally used for peerreview papers.
%\begin{IEEEkeywords}
%Cooperative diversity, decode and forward, piecewise linear
%\end{IEEEkeywords}



% For peer review papers, you can put extra information on the cover
% page as needed:
% \ifCLASSOPTIONpeerreview
% \begin{center} \bfseries EDICS Category: 3-BBND \end{center}
% \fi
%
% For peerreview papers, this IEEEtran command inserts a page break and
% creates the second title. It will be ignored for other modes.
%\IEEEpeerreviewmaketitle




	\item  A die is loaded in such a way that each odd number is twice as likely to occur as
each even number. Find $P(G)$, where $G$ is the event that a number greater than
3 occurs on a single roll of the die.
\\
\solution
		%\begin{table}[H]
	\centering
\begin{tabular}{|c|c|c|}
\hline
Random variable &Value &Definition\\ \hline
\multirow{3}{*}{X} &0 &Slips of Rs 1\\
&1 &Slips of Rs 5\\
&2 &Slips of Rs 13\\ \hline
\multirow{2}{*}{Y} &0 &Box A\\
&1 &Box B\\\hline
\end{tabular}
\caption{}
\label{tab:Distribution}
\end{table}
See \tabref{tab:Distribution}.
\begin{align}
p_{Y}\brak{k}= \begin{cases} 
      \frac{1}{3} & {k=0} \\
      \frac{2}{3 }& {k=1} 
   \end{cases}
   \\
p_{Y|X}\brak{0|0} = \frac{19}{25}\, 
p_{Y|X}\brak{0|1} = \frac{6}{25}\,
p_{Y|X}\brak{1|0} = \frac{45}{50}\,
p_{Y|X}\brak{1|2} = \frac{5}{50}
\end{align}
The desired probability is the probability that a slip drawn at random is marked other than Rs 1,
\begin{align}
&=1-p_X\brak{0}\\
&= p_X(1) + p_X(2)
\end{align}
Using Bayes theorem,
\begin{align}
&= p_Y\brak{0} \times \pr{Y=0 | X=1} + p_Y\brak{1} \times \pr{Y=1|X=2}\\
&=\frac{1}{3} \times \frac{6}{25} + \frac{2}{3} \times \frac{5}{50}\\
&=\frac{11}{75}
\end{align}

\newpage

%\tableofcontents

\bigskip

\renewcommand{\thefigure}{\theenumi}
\renewcommand{\thetable}{\theenumi}
%\renewcommand{\theequation}{\theenumi}

%\begin{abstract}
%%\boldmath
%In this letter, an algorithm for evaluating the exact analytical bit error rate  (BER)  for the piecewise linear (PL) combiner for  multiple relays is presented. Previous results were available only for upto three relays. The algorithm is unique in the sense that  the actual mathematical expressions, that are prohibitively large, need not be explicitly obtained. The diversity gain due to multiple relays is shown through plots of the analytical BER, well supported by simulations. 
%
%\end{abstract}
% IEEEtran.cls defaults to using nonbold math in the Abstract.
% This preserves the distinction between vectors and scalars. However,
% if the journal you are submitting to favors bold math in the abstract,
% then you can use LaTeX's standard command \boldmath at the very start
% of the abstract to achieve this. Many IEEE journals frown on math
% in the abstract anyway.

% Note that keywords are not normally used for peerreview papers.
%\begin{IEEEkeywords}
%Cooperative diversity, decode and forward, piecewise linear
%\end{IEEEkeywords}



% For peer review papers, you can put extra information on the cover
% page as needed:
% \ifCLASSOPTIONpeerreview
% \begin{center} \bfseries EDICS Category: 3-BBND \end{center}
% \fi
%
% For peerreview papers, this IEEEtran command inserts a page break and
% creates the second title. It will be ignored for other modes.
%\IEEEpeerreviewmaketitle




	\item All the jacks, queens and kings are removed from a deck of 52 playing cards. The remaining cards are well shuffled and then one card is drawn at random. Giving ace a value 1 similar value for other cards, find the probability that the card has a value 
		\begin{enumerate}
			\item 7
			\item greater than 7
			\item less than 7
		\end{enumerate}
		%Number of cards left after removing all jacks, queens and kings 
\begin{align}
N	= 52 - 4\times 3
	= 40
\end{align}
%\begin{table}[H]
%\def\arraystretch{1.2}
%\begin{tabular}{|c|c|c|}
%\hline
%	\textbf{Parameter} &\textbf{Value} &\textbf{Description}\\ \hline
%	$X$ &1-10 &Represents the value of the card picked \\ \hline
%\end{tabular}
%\end{table}
Let $1 \le X \le 10$ be the value of the card picked.  Then,
\begin{align}
	p_X(k) &= \Pr(X=k)\ \forall\ 1 \leq k \leq 10\\
	&= \frac{4\times 1}{40}\\
	&= \frac{1}{10}\\
	\therefore p_X(k) &= 
	\begin{cases}
		\frac{1}{10} & 1 \leq k \leq 10\\
		0 & \text{otherwise}
	\end{cases}
\end{align}
and
\begin{align}
	F_{X}(k) &= \sum_{m=0}^{k}p_{X}(m) \quad 1 \leq k \leq 10\\
	&= \frac{k}{10}\\
	\therefore F_{X}(k) &= 
	\begin{cases}
		0 & k \leq 0\\
		\frac{k}{10} & 1\leq k \leq 10\\
		1 & k > 10 
	\end{cases}
\end{align}
\begin{enumerate}
	\item Probability that card has value equal to 7 is
		\begin{align}
			 p_{X}(7)
			= \frac{1}{10}
		\end{align}
	\item Probability that card has value greater than 7 is
		\begin{align}
			1 - F_X(7)
			&= 1 - \frac{7}{10}
			\\
			&= \frac{3}{10}
		\end{align}
	\item Probability that card has value less than 7 is
		\begin{align}
			 F_{X}(6)
			=\frac{6}{10}
		\end{align}
\end{enumerate}

  \item A Lot consists of 48 mobile phones of which 42 are good, 3 have only minor defects and 3 have major defects.Varnika will buy a phone if it is good but the trader will only buy a mobile if it has no major defects. One phone is selected at random from the lot. What is the probability that it is
\begin{enumerate}
	\item acceptable to Varnika?
            \item acceptable to the trader?
\end{enumerate}
\solution
	%\begin{table}[H]
	\centering
\begin{tabular}{|c|c|c|}
\hline
Random variable &Value &Definition\\ \hline
\multirow{3}{*}{X} &0 &Slips of Rs 1\\
&1 &Slips of Rs 5\\
&2 &Slips of Rs 13\\ \hline
\multirow{2}{*}{Y} &0 &Box A\\
&1 &Box B\\\hline
\end{tabular}
\caption{}
\label{tab:Distribution}
\end{table}
See \tabref{tab:Distribution}.
\begin{align}
p_{Y}\brak{k}= \begin{cases} 
      \frac{1}{3} & {k=0} \\
      \frac{2}{3 }& {k=1} 
   \end{cases}
   \\
p_{Y|X}\brak{0|0} = \frac{19}{25}\, 
p_{Y|X}\brak{0|1} = \frac{6}{25}\,
p_{Y|X}\brak{1|0} = \frac{45}{50}\,
p_{Y|X}\brak{1|2} = \frac{5}{50}
\end{align}
The desired probability is the probability that a slip drawn at random is marked other than Rs 1,
\begin{align}
&=1-p_X\brak{0}\\
&= p_X(1) + p_X(2)
\end{align}
Using Bayes theorem,
\begin{align}
&= p_Y\brak{0} \times \pr{Y=0 | X=1} + p_Y\brak{1} \times \pr{Y=1|X=2}\\
&=\frac{1}{3} \times \frac{6}{25} + \frac{2}{3} \times \frac{5}{50}\\
&=\frac{11}{75}
\end{align}

\newpage

%\tableofcontents

\bigskip

\renewcommand{\thefigure}{\theenumi}
\renewcommand{\thetable}{\theenumi}
%\renewcommand{\theequation}{\theenumi}

%\begin{abstract}
%%\boldmath
%In this letter, an algorithm for evaluating the exact analytical bit error rate  (BER)  for the piecewise linear (PL) combiner for  multiple relays is presented. Previous results were available only for upto three relays. The algorithm is unique in the sense that  the actual mathematical expressions, that are prohibitively large, need not be explicitly obtained. The diversity gain due to multiple relays is shown through plots of the analytical BER, well supported by simulations. 
%
%\end{abstract}
% IEEEtran.cls defaults to using nonbold math in the Abstract.
% This preserves the distinction between vectors and scalars. However,
% if the journal you are submitting to favors bold math in the abstract,
% then you can use LaTeX's standard command \boldmath at the very start
% of the abstract to achieve this. Many IEEE journals frown on math
% in the abstract anyway.

% Note that keywords are not normally used for peerreview papers.
%\begin{IEEEkeywords}
%Cooperative diversity, decode and forward, piecewise linear
%\end{IEEEkeywords}



% For peer review papers, you can put extra information on the cover
% page as needed:
% \ifCLASSOPTIONpeerreview
% \begin{center} \bfseries EDICS Category: 3-BBND \end{center}
% \fi
%
% For peerreview papers, this IEEEtran command inserts a page break and
% creates the second title. It will be ignored for other modes.
%\IEEEpeerreviewmaketitle




 \item A student says that if you throw a die, it will show up 1 or not 1. Therefore, the probability of getting 1 and the probability of getting 'not 1' each is equal to $\frac{1}{2}$. Is this correct? Give reasons.\\
 \solution
        %\begin{table}[H]
	\centering
\begin{tabular}{|c|c|c|}
\hline
Random variable &Value &Definition\\ \hline
\multirow{3}{*}{X} &0 &Slips of Rs 1\\
&1 &Slips of Rs 5\\
&2 &Slips of Rs 13\\ \hline
\multirow{2}{*}{Y} &0 &Box A\\
&1 &Box B\\\hline
\end{tabular}
\caption{}
\label{tab:Distribution}
\end{table}
See \tabref{tab:Distribution}.
\begin{align}
p_{Y}\brak{k}= \begin{cases} 
      \frac{1}{3} & {k=0} \\
      \frac{2}{3 }& {k=1} 
   \end{cases}
   \\
p_{Y|X}\brak{0|0} = \frac{19}{25}\, 
p_{Y|X}\brak{0|1} = \frac{6}{25}\,
p_{Y|X}\brak{1|0} = \frac{45}{50}\,
p_{Y|X}\brak{1|2} = \frac{5}{50}
\end{align}
The desired probability is the probability that a slip drawn at random is marked other than Rs 1,
\begin{align}
&=1-p_X\brak{0}\\
&= p_X(1) + p_X(2)
\end{align}
Using Bayes theorem,
\begin{align}
&= p_Y\brak{0} \times \pr{Y=0 | X=1} + p_Y\brak{1} \times \pr{Y=1|X=2}\\
&=\frac{1}{3} \times \frac{6}{25} + \frac{2}{3} \times \frac{5}{50}\\
&=\frac{11}{75}
\end{align}

\newpage

%\tableofcontents

\bigskip

\renewcommand{\thefigure}{\theenumi}
\renewcommand{\thetable}{\theenumi}
%\renewcommand{\theequation}{\theenumi}

%\begin{abstract}
%%\boldmath
%In this letter, an algorithm for evaluating the exact analytical bit error rate  (BER)  for the piecewise linear (PL) combiner for  multiple relays is presented. Previous results were available only for upto three relays. The algorithm is unique in the sense that  the actual mathematical expressions, that are prohibitively large, need not be explicitly obtained. The diversity gain due to multiple relays is shown through plots of the analytical BER, well supported by simulations. 
%
%\end{abstract}
% IEEEtran.cls defaults to using nonbold math in the Abstract.
% This preserves the distinction between vectors and scalars. However,
% if the journal you are submitting to favors bold math in the abstract,
% then you can use LaTeX's standard command \boldmath at the very start
% of the abstract to achieve this. Many IEEE journals frown on math
% in the abstract anyway.

% Note that keywords are not normally used for peerreview papers.
%\begin{IEEEkeywords}
%Cooperative diversity, decode and forward, piecewise linear
%\end{IEEEkeywords}



% For peer review papers, you can put extra information on the cover
% page as needed:
% \ifCLASSOPTIONpeerreview
% \begin{center} \bfseries EDICS Category: 3-BBND \end{center}
% \fi
%
% For peerreview papers, this IEEEtran command inserts a page break and
% creates the second title. It will be ignored for other modes.
%\IEEEpeerreviewmaketitle




   \item Four candidates A, B, C, D have ap-
plied for the assignment to coach a school cricket
team. If A is twice as likely to be selected as B, and
B and C are given about the same chance of being
selected, while C is twice as likely to be selected
as D, what are the probabilities that
\begin{enumerate}
\item C will be selected?
\item A will not be selected?
\end{enumerate}
	%\begin{table}[H]
	\centering
\begin{tabular}{|c|c|c|}
\hline
Random variable &Value &Definition\\ \hline
\multirow{3}{*}{X} &0 &Slips of Rs 1\\
&1 &Slips of Rs 5\\
&2 &Slips of Rs 13\\ \hline
\multirow{2}{*}{Y} &0 &Box A\\
&1 &Box B\\\hline
\end{tabular}
\caption{}
\label{tab:Distribution}
\end{table}
See \tabref{tab:Distribution}.
\begin{align}
p_{Y}\brak{k}= \begin{cases} 
      \frac{1}{3} & {k=0} \\
      \frac{2}{3 }& {k=1} 
   \end{cases}
   \\
p_{Y|X}\brak{0|0} = \frac{19}{25}\, 
p_{Y|X}\brak{0|1} = \frac{6}{25}\,
p_{Y|X}\brak{1|0} = \frac{45}{50}\,
p_{Y|X}\brak{1|2} = \frac{5}{50}
\end{align}
The desired probability is the probability that a slip drawn at random is marked other than Rs 1,
\begin{align}
&=1-p_X\brak{0}\\
&= p_X(1) + p_X(2)
\end{align}
Using Bayes theorem,
\begin{align}
&= p_Y\brak{0} \times \pr{Y=0 | X=1} + p_Y\brak{1} \times \pr{Y=1|X=2}\\
&=\frac{1}{3} \times \frac{6}{25} + \frac{2}{3} \times \frac{5}{50}\\
&=\frac{11}{75}
\end{align}

\newpage

%\tableofcontents

\bigskip

\renewcommand{\thefigure}{\theenumi}
\renewcommand{\thetable}{\theenumi}
%\renewcommand{\theequation}{\theenumi}

%\begin{abstract}
%%\boldmath
%In this letter, an algorithm for evaluating the exact analytical bit error rate  (BER)  for the piecewise linear (PL) combiner for  multiple relays is presented. Previous results were available only for upto three relays. The algorithm is unique in the sense that  the actual mathematical expressions, that are prohibitively large, need not be explicitly obtained. The diversity gain due to multiple relays is shown through plots of the analytical BER, well supported by simulations. 
%
%\end{abstract}
% IEEEtran.cls defaults to using nonbold math in the Abstract.
% This preserves the distinction between vectors and scalars. However,
% if the journal you are submitting to favors bold math in the abstract,
% then you can use LaTeX's standard command \boldmath at the very start
% of the abstract to achieve this. Many IEEE journals frown on math
% in the abstract anyway.

% Note that keywords are not normally used for peerreview papers.
%\begin{IEEEkeywords}
%Cooperative diversity, decode and forward, piecewise linear
%\end{IEEEkeywords}



% For peer review papers, you can put extra information on the cover
% page as needed:
% \ifCLASSOPTIONpeerreview
% \begin{center} \bfseries EDICS Category: 3-BBND \end{center}
% \fi
%
% For peerreview papers, this IEEEtran command inserts a page break and
% creates the second title. It will be ignored for other modes.
%\IEEEpeerreviewmaketitle




 \item A bag contain 24 balls of which $x$ balls are red, $2x$ are white and $3x$ are blue. A ball is selected at random, What is the probability that it is
\begin{enumerate}[label=\alph*)]
\item not red ?
\item white ?
\end{enumerate}
%\begin{table}[H]
	\centering
\begin{tabular}{|c|c|c|}
\hline
Random variable &Value &Definition\\ \hline
\multirow{3}{*}{X} &0 &Slips of Rs 1\\
&1 &Slips of Rs 5\\
&2 &Slips of Rs 13\\ \hline
\multirow{2}{*}{Y} &0 &Box A\\
&1 &Box B\\\hline
\end{tabular}
\caption{}
\label{tab:Distribution}
\end{table}
See \tabref{tab:Distribution}.
\begin{align}
p_{Y}\brak{k}= \begin{cases} 
      \frac{1}{3} & {k=0} \\
      \frac{2}{3 }& {k=1} 
   \end{cases}
   \\
p_{Y|X}\brak{0|0} = \frac{19}{25}\, 
p_{Y|X}\brak{0|1} = \frac{6}{25}\,
p_{Y|X}\brak{1|0} = \frac{45}{50}\,
p_{Y|X}\brak{1|2} = \frac{5}{50}
\end{align}
The desired probability is the probability that a slip drawn at random is marked other than Rs 1,
\begin{align}
&=1-p_X\brak{0}\\
&= p_X(1) + p_X(2)
\end{align}
Using Bayes theorem,
\begin{align}
&= p_Y\brak{0} \times \pr{Y=0 | X=1} + p_Y\brak{1} \times \pr{Y=1|X=2}\\
&=\frac{1}{3} \times \frac{6}{25} + \frac{2}{3} \times \frac{5}{50}\\
&=\frac{11}{75}
\end{align}

\newpage

%\tableofcontents

\bigskip

\renewcommand{\thefigure}{\theenumi}
\renewcommand{\thetable}{\theenumi}
%\renewcommand{\theequation}{\theenumi}

%\begin{abstract}
%%\boldmath
%In this letter, an algorithm for evaluating the exact analytical bit error rate  (BER)  for the piecewise linear (PL) combiner for  multiple relays is presented. Previous results were available only for upto three relays. The algorithm is unique in the sense that  the actual mathematical expressions, that are prohibitively large, need not be explicitly obtained. The diversity gain due to multiple relays is shown through plots of the analytical BER, well supported by simulations. 
%
%\end{abstract}
% IEEEtran.cls defaults to using nonbold math in the Abstract.
% This preserves the distinction between vectors and scalars. However,
% if the journal you are submitting to favors bold math in the abstract,
% then you can use LaTeX's standard command \boldmath at the very start
% of the abstract to achieve this. Many IEEE journals frown on math
% in the abstract anyway.

% Note that keywords are not normally used for peerreview papers.
%\begin{IEEEkeywords}
%Cooperative diversity, decode and forward, piecewise linear
%\end{IEEEkeywords}



% For peer review papers, you can put extra information on the cover
% page as needed:
% \ifCLASSOPTIONpeerreview
% \begin{center} \bfseries EDICS Category: 3-BBND \end{center}
% \fi
%
% For peerreview papers, this IEEEtran command inserts a page break and
% creates the second title. It will be ignored for other modes.
%\IEEEpeerreviewmaketitle




If the letters of the word ASSASSINATION are arranged at random. Find the Probability that
\begin{enumerate}[label=(\alph*)]
\item Four $S's$ come consecutively in the word
\item Two  $I's$ and two $N's$ come together
\item All $A's$ are not coming together
\item No two $A's$ are coming together
\end{enumerate}
%\begin{table}[H]
	\centering
\begin{tabular}{|c|c|c|}
\hline
Random variable &Value &Definition\\ \hline
\multirow{3}{*}{X} &0 &Slips of Rs 1\\
&1 &Slips of Rs 5\\
&2 &Slips of Rs 13\\ \hline
\multirow{2}{*}{Y} &0 &Box A\\
&1 &Box B\\\hline
\end{tabular}
\caption{}
\label{tab:Distribution}
\end{table}
See \tabref{tab:Distribution}.
\begin{align}
p_{Y}\brak{k}= \begin{cases} 
      \frac{1}{3} & {k=0} \\
      \frac{2}{3 }& {k=1} 
   \end{cases}
   \\
p_{Y|X}\brak{0|0} = \frac{19}{25}\, 
p_{Y|X}\brak{0|1} = \frac{6}{25}\,
p_{Y|X}\brak{1|0} = \frac{45}{50}\,
p_{Y|X}\brak{1|2} = \frac{5}{50}
\end{align}
The desired probability is the probability that a slip drawn at random is marked other than Rs 1,
\begin{align}
&=1-p_X\brak{0}\\
&= p_X(1) + p_X(2)
\end{align}
Using Bayes theorem,
\begin{align}
&= p_Y\brak{0} \times \pr{Y=0 | X=1} + p_Y\brak{1} \times \pr{Y=1|X=2}\\
&=\frac{1}{3} \times \frac{6}{25} + \frac{2}{3} \times \frac{5}{50}\\
&=\frac{11}{75}
\end{align}

\newpage

%\tableofcontents

\bigskip

\renewcommand{\thefigure}{\theenumi}
\renewcommand{\thetable}{\theenumi}
%\renewcommand{\theequation}{\theenumi}

%\begin{abstract}
%%\boldmath
%In this letter, an algorithm for evaluating the exact analytical bit error rate  (BER)  for the piecewise linear (PL) combiner for  multiple relays is presented. Previous results were available only for upto three relays. The algorithm is unique in the sense that  the actual mathematical expressions, that are prohibitively large, need not be explicitly obtained. The diversity gain due to multiple relays is shown through plots of the analytical BER, well supported by simulations. 
%
%\end{abstract}
% IEEEtran.cls defaults to using nonbold math in the Abstract.
% This preserves the distinction between vectors and scalars. However,
% if the journal you are submitting to favors bold math in the abstract,
% then you can use LaTeX's standard command \boldmath at the very start
% of the abstract to achieve this. Many IEEE journals frown on math
% in the abstract anyway.

% Note that keywords are not normally used for peerreview papers.
%\begin{IEEEkeywords}
%Cooperative diversity, decode and forward, piecewise linear
%\end{IEEEkeywords}



% For peer review papers, you can put extra information on the cover
% page as needed:
% \ifCLASSOPTIONpeerreview
% \begin{center} \bfseries EDICS Category: 3-BBND \end{center}
% \fi
%
% For peerreview papers, this IEEEtran command inserts a page break and
% creates the second title. It will be ignored for other modes.
%\IEEEpeerreviewmaketitle




	\item One urn contains two black balls (labelled B1 and B2) and one white ball. A
	second urn contains one black ball and two white balls (labelled W1 and W2).
	Suppose the following experiment is performed. One of the two urns is chosen
	at random. Next a ball is randomly chosen from the urn. Then a second ball is
	chosen at random from the same urn without replacing the first ball.
	
	\begin{enumerate}
	\item What is the probability that two black balls are chosen?
	
	\item What is the probability that two balls of opposite colour are chosen?
	\end{enumerate}
	\solution
	%\begin{align}
    \label{eq:12.13.6.18.1}
	\because	\pr{A|B} &> \pr{A},\
\frac{\pr{AB}}{\pr{B}} > \pr{A}
\\
    \label{eq:12.13.6.18.2}
	\implies \pr{AB} &> \pr{A}\pr{B}
	\\
	\text{or, } \frac{\pr{AB}}{\pr{A}} &=\pr{B|A} > \pr{A}
\end{align}

\end{enumerate}

\item In a certain lottery 10,000 tickets are sold and ten equal prizes are awarded. What is the probability of not getting a prize if you buy (a) one ticket (b) two tickets (c) 10 tickets ?	
\\
\solution
		%\begin{enumerate}[label=\thesection.\arabic*,ref=\thesection.\theenumi]
	\item One card is drawn from a well-shuffled deck of 52 cards. Find the probability of getting
\begin{enumerate}
\item A king of red colour 
\item A face card 
\item A red face card
\item The jack of hearts
\item A spade
\item The queen of diamonds

\end{enumerate}
\solution
		%\begin{table}[H]
	\centering
\begin{tabular}{|c|c|c|}
\hline
Random variable &Value &Definition\\ \hline
\multirow{3}{*}{X} &0 &Slips of Rs 1\\
&1 &Slips of Rs 5\\
&2 &Slips of Rs 13\\ \hline
\multirow{2}{*}{Y} &0 &Box A\\
&1 &Box B\\\hline
\end{tabular}
\caption{}
\label{tab:Distribution}
\end{table}
See \tabref{tab:Distribution}.
\begin{align}
p_{Y}\brak{k}= \begin{cases} 
      \frac{1}{3} & {k=0} \\
      \frac{2}{3 }& {k=1} 
   \end{cases}
   \\
p_{Y|X}\brak{0|0} = \frac{19}{25}\, 
p_{Y|X}\brak{0|1} = \frac{6}{25}\,
p_{Y|X}\brak{1|0} = \frac{45}{50}\,
p_{Y|X}\brak{1|2} = \frac{5}{50}
\end{align}
The desired probability is the probability that a slip drawn at random is marked other than Rs 1,
\begin{align}
&=1-p_X\brak{0}\\
&= p_X(1) + p_X(2)
\end{align}
Using Bayes theorem,
\begin{align}
&= p_Y\brak{0} \times \pr{Y=0 | X=1} + p_Y\brak{1} \times \pr{Y=1|X=2}\\
&=\frac{1}{3} \times \frac{6}{25} + \frac{2}{3} \times \frac{5}{50}\\
&=\frac{11}{75}
\end{align}

\newpage

%\tableofcontents

\bigskip

\renewcommand{\thefigure}{\theenumi}
\renewcommand{\thetable}{\theenumi}
%\renewcommand{\theequation}{\theenumi}

%\begin{abstract}
%%\boldmath
%In this letter, an algorithm for evaluating the exact analytical bit error rate  (BER)  for the piecewise linear (PL) combiner for  multiple relays is presented. Previous results were available only for upto three relays. The algorithm is unique in the sense that  the actual mathematical expressions, that are prohibitively large, need not be explicitly obtained. The diversity gain due to multiple relays is shown through plots of the analytical BER, well supported by simulations. 
%
%\end{abstract}
% IEEEtran.cls defaults to using nonbold math in the Abstract.
% This preserves the distinction between vectors and scalars. However,
% if the journal you are submitting to favors bold math in the abstract,
% then you can use LaTeX's standard command \boldmath at the very start
% of the abstract to achieve this. Many IEEE journals frown on math
% in the abstract anyway.

% Note that keywords are not normally used for peerreview papers.
%\begin{IEEEkeywords}
%Cooperative diversity, decode and forward, piecewise linear
%\end{IEEEkeywords}



% For peer review papers, you can put extra information on the cover
% page as needed:
% \ifCLASSOPTIONpeerreview
% \begin{center} \bfseries EDICS Category: 3-BBND \end{center}
% \fi
%
% For peerreview papers, this IEEEtran command inserts a page break and
% creates the second title. It will be ignored for other modes.
%\IEEEpeerreviewmaketitle




	\item Five cards—the ten, jack, queen, king and ace of diamonds, are well-shuffled with their face downwards. One card is then picked up at random.
\begin{enumerate}
\item
What is the probability that the card is the queen? 
\item
If the queen is drawn and put aside, what is the probability that the second card picked up is (a) an ace? (b) a queen?\\
\end{enumerate}
\solution
		%\begin{enumerate}[label=\thesection.\arabic*,ref=\thesection.\theenumi]
	\item One card is drawn from a well-shuffled deck of 52 cards. Find the probability of getting
\begin{enumerate}
\item A king of red colour 
\item A face card 
\item A red face card
\item The jack of hearts
\item A spade
\item The queen of diamonds

\end{enumerate}
\solution
		%\input{ncert/10/15/1/14/main.tex}
	\item Five cards—the ten, jack, queen, king and ace of diamonds, are well-shuffled with their face downwards. One card is then picked up at random.
\begin{enumerate}
\item
What is the probability that the card is the queen? 
\item
If the queen is drawn and put aside, what is the probability that the second card picked up is (a) an ace? (b) a queen?\\
\end{enumerate}
\solution
		%\input{ncert/10/15/1/15/defs.tex}
	\item A bag contains $5$ red balls and some blue balls. If the probability of drawing a blue ball is double that if a red ball, determine the number of blue balls in the bag. 
		\\
\solution
		%\input{ncert/10/15/2/3/defs.tex}
	\item A card is selected from a pack of 52 cards.
 \begin{enumerate}[label=(\alph*)] 
                 \item How many points are there in the sample space?
                 \item Calculate the probability that the card is an ace of spades.
                 \item Calculate the probability that the card is (i) an ace and (ii) black card.
 \end{enumerate}
\solution
		%\input{ncert/11/16/3/4/main.tex}
\item Four cards are drawn from a well-shuffled deck of 52 cards. What is the probability of obtaining 3 diamonds and one spade.
\\
\solution
		%\input{ncert/11/16/4/2/defs.tex}
\item In a certain lottery 10,000 tickets are sold and ten equal prizes are awarded. What is the probability of not getting a prize if you buy (a) one ticket (b) two tickets (c) 10 tickets ?	
\\
\solution
		%\input{ncert/11/16/4/4/defs.tex}
		%
\item 
Out of 100 students, two sections of 40 and 60 are formed. If you and your friend are among the 100 students, what is the probability that
\begin{enumerate}
\item you both enter the same section?
\item you both enter the different sections?
\end{enumerate}
\solution
		%\input{ncert/11/16/4/5/defs.tex}
	\item 
The number lock of a suitcase has 4 wheels each labelled with ten digits i.e. from 0 to 9.The lock opens with a sequence of four digits with no repeats.What is the probability of a person getting the right sequence to open the suitcase.
\\
\solution
		%\input{ncert/11/16/4/10/defs.tex}
		%
\item 
Two cards are drawn at random and without replacement from a pack of 52 playing cards. Find the probability that both the cards are black.
\\
\solution
		%\input{ncert/12/13/2/2/defs.tex}
		\item A box of oranges is inspected by examining three randomly selected oranges drawn without replacement. If all the three oranges are good, the box is approved for sale, otherwise, it is rejected. Find the probability that a box containing 15 oranges out of which 12 are good and 3 are bad ones will be approved for sale.
		\label{ncert/12/13/2/3/defs.tex}
		\item Two balls are drawn at random with replacement from a box containing 10 black and 8 red balls. Find the probability that
		\label{ncert/12/13/2/12}
\begin{enumerate}
\item both balls are red.
\item first ball is black and second is red.
\item one of them is black and other is red.
\end{enumerate}

\item In a hostel, 60\% of the students read Hindi newspaper, 40\% read English newspaper and 20\% read both Hindi and English newspapers. A student is selected at random.
		\label{ncert/12/13/2/15}
\begin{enumerate}
\item Find the probability that she reads neither Hindi nor English newspapers.
\item If she reads Hindi newspaper, find the probability that she reads English newspaper.
\item If she reads English newspaper, find the probability that she reads Hindi newspaper.\\
\end{enumerate}
\item The probability of obtaining an even prime number on each die, when a pair of dice is rolled is 
\begin{enumerate}
    \item $0$ 
    
    \item $\frac{1}{3}$ 
    
    \item $\frac{1}{12}$ 
    
    \item $\frac{1}{36}$ 
\end{enumerate}
\solution
		%\input{ncert/12/13/2/17/defs.tex}
	\item A bag contains 4 red and 4 black balls, another bag contains 2 red and 6 black balls. One of the two bags is selected at random and a ball is drawn from the bag which is found to be red. Find the probability that the ball is drawn from the first bag.
\\
\solution
		%\input{ncert/12/13/3/2/main.tex}
  \item
  Cards with numbers 2 to 101 are placed in a box. A card is selected at random.Find the probability that the card has
\begin{enumerate}[label=(\roman*)]
	\item an even number 
	\item a square number
\end{enumerate}
\solution
%\input{exemplar/10/13/3/32/main.tex}
\item
The king, queen and jack of clubs are removed from a deck of 52 playing cards and then well shuffled. Now one card is drawn at random from the remaining cards.  Determine the probability that the card is
\begin{enumerate}[label=(\roman*)]
\item a club
\item 10 of hearts
\end{enumerate}
\solution
%\input{exemplar/10/13/3/29/main.tex}
\item A team of medical students doing their internship have to assist during surgeries
at a city hospital. The probabilities of surgeries rated as very complex, complex,
routine, simple or very simple are respectively, 0.15, 0.20, 0.31, 0.26, .08. Find
the probabilities that a particular surgery will be rated
\begin{enumerate}
	\item complex or very complex;
	\item neither very complex nor very simple;
	\item routine or complex
	\item routine or simple
\end{enumerate}
\solution
%\input{exemplar/11/16/3/8(1)/main.tex}
\item A card is selected from a pack of 52 cards.
\begin{enumerate}[label=(\alph*)]
    \item How many points are there in the sample space?
    \item Calculate the probability that the card is an ace of spades.
    \item Calculate the probability that the card is (i) an ace and (ii) black card.
\end{enumerate}
\solution
%\input{exemplar/11/16/3/4/main2.tex}
\item The probability that a non leap year selected at random will contain 53 sundays.
\\
\solution
%\input{exemplar/10/13/1/19/main.tex}
\item One of the four persons John, Rita, Aslam or Gurpreet will be promoted next
month. Consequently the sample space consists of four elementary outcomes
S = {John promoted, Rita promoted, Aslam promoted, Gurpreet promoted}
You are told that the chances of John’s promotion is same as that of Gurpreet,
Rita’s chances of promotion are twice as likely as Johns. Aslam’s chances are
four times that of John.
\begin{enumerate}
	\item Determine
	\begin{enumerate}
		\item P (John promoted)
		\item P (Rita promoted)
		\item P (Aslam promoted)
		\item P (Gurpreet promoted)
	\end{enumerate}
	\item If A = {John promoted or Gurpreet promoted}, find P (A).
\end{enumerate}
\solution
%\input{exemplar/11/16/3/10/main.tex}
\item A card is drawn from a deck of 52 cards. Find the probability of getting a king or a heart or a red card.\\
\solution
%\input{exemplar/11/16/3/15/main.tex}
\item The probability that a student will pass his examination is 0.73, the probability of
the student getting a compartment is 0.13, and the probability that the student will
either pass or get compartment is 0.96. State True or False.\\
\solution
%\input{exemplar/11/16/3/31/main.tex}
\item A card is selected from a pack of 52 cards\\
\begin{enumerate}[label=(\alph*)]
\item How many points are there in the sample space?
\item Calculate the probability that the cards is an ace of spades.
\item Calculate the probability that the card is (i) an ace (ii)black card.\\
\end{enumerate}
%\input{ncert/11/16/3/4_1/Prob_4.tex}
\item In a non-leap year, the probability of having 53 tuesdays or 53 wednesdays is\\
\solution
%\input{exemplar/11/16/3/18/main.tex}
\item There are 1000 sealed envelopes in a box, 10 of them contain a cash prize of
Rs 100 each, 100 of them contain a cash prize of Rs 50 each and 200 of them
contain a cash prize of Rs 10 each and rest do not contain any cash prize. If they
are well shuffled and an envelope is picked up out, what is the probability that it
contains no cash prize?\\
\solution
%\input{exemplar/10/13/3/34/main.tex}
\item 
A die is thrown and a card is selected at random from a deck of 52 playing cards. The probability of getting an even number on the die and a spade card.\\
\solution
%\input{exemplar/12/13/3/78/main.tex}
\item
If 4-digit numbers greater than 5,000 are randomly formed from the digits 0, 1, 3, 5, and 7, what is the probability of forming a number divisible by 5 when:
\begin{enumerate}
    \item The digits are repeated?
    \item The repetition of digits is not allowed?
\end{enumerate}
\solution
%\input{ncert/11/16/4/9/main.tex}
\item Consider the probability space $\brak{\Omega, \mathcal{G}, P}$ where $\Omega = [0,2]$ and $\mathcal{G} = \cbrak{\phi, \Omega, [0,1], (1,2]}$. Let $X$ and $Y$ be two functions on $\Omega$ defined as
\begin{align*}
    X(\omega) = 
    \begin{cases}
        1 & \text{if }\omega \in [0, 1]\\
        2 & \text{if }\omega \in (1, 2]
    \end{cases}
\end{align*}
and
\begin{align*}
    Y(\omega) = 
    \begin{cases}
        2 & \text{if }\omega \in [0, 1.5]\\
        3 & \text{if }\omega \in (1.5, 2].
    \end{cases}
\end{align*}
Then which one of the following statements is true?
\begin{enumerate}
    \item [(A)] $X$ is a random variable with respect to $\mathcal{G}$, but $Y$ is not a random variable with respect to $\mathcal{G}$.
    \item [(B)] $Y$ is a random variable with respect to $\mathcal{G}$, but $X$ is not a random variable with respect to $\mathcal{G}$.
    \item [(C)] Neither $X$ nor $Y$ is a random variable with respect to $\mathcal{G}$.
    \item [(D)] Both $X$ and $Y$ are random variables with respect to $\mathcal{G}$.
\end{enumerate} \hfill (GATE ST 2023)\\
\solution
%\input{gate/ST/2023/14/main.tex}
	\item  A die is loaded in such a way that each odd number is twice as likely to occur as
each even number. Find $P(G)$, where $G$ is the event that a number greater than
3 occurs on a single roll of the die.
\\
\solution
		%\input{exemplar/11/16/3/5/main.tex}
	\item All the jacks, queens and kings are removed from a deck of 52 playing cards. The remaining cards are well shuffled and then one card is drawn at random. Giving ace a value 1 similar value for other cards, find the probability that the card has a value 
		\begin{enumerate}
			\item 7
			\item greater than 7
			\item less than 7
		\end{enumerate}
		%\input{exemplar/10/13/3/30/main.tex}
  \item A Lot consists of 48 mobile phones of which 42 are good, 3 have only minor defects and 3 have major defects.Varnika will buy a phone if it is good but the trader will only buy a mobile if it has no major defects. One phone is selected at random from the lot. What is the probability that it is
\begin{enumerate}
	\item acceptable to Varnika?
            \item acceptable to the trader?
\end{enumerate}
\solution
	%\input{exemplar/10/13/3/40/main.tex}
 \item A student says that if you throw a die, it will show up 1 or not 1. Therefore, the probability of getting 1 and the probability of getting 'not 1' each is equal to $\frac{1}{2}$. Is this correct? Give reasons.\\
 \solution
        %\input{exemplar/10/13/2/9/main.tex}
   \item Four candidates A, B, C, D have ap-
plied for the assignment to coach a school cricket
team. If A is twice as likely to be selected as B, and
B and C are given about the same chance of being
selected, while C is twice as likely to be selected
as D, what are the probabilities that
\begin{enumerate}
\item C will be selected?
\item A will not be selected?
\end{enumerate}
	%\input{exemplar/11/16/3/9/main.tex}
 \item A bag contain 24 balls of which $x$ balls are red, $2x$ are white and $3x$ are blue. A ball is selected at random, What is the probability that it is
\begin{enumerate}[label=\alph*)]
\item not red ?
\item white ?
\end{enumerate}
%\input{exemplar/10/13/3/41/main.tex}
If the letters of the word ASSASSINATION are arranged at random. Find the Probability that
\begin{enumerate}[label=(\alph*)]
\item Four $S's$ come consecutively in the word
\item Two  $I's$ and two $N's$ come together
\item All $A's$ are not coming together
\item No two $A's$ are coming together
\end{enumerate}
%\input{exemplar/11/16/3/14/main.tex}
	\item One urn contains two black balls (labelled B1 and B2) and one white ball. A
	second urn contains one black ball and two white balls (labelled W1 and W2).
	Suppose the following experiment is performed. One of the two urns is chosen
	at random. Next a ball is randomly chosen from the urn. Then a second ball is
	chosen at random from the same urn without replacing the first ball.
	
	\begin{enumerate}
	\item What is the probability that two black balls are chosen?
	
	\item What is the probability that two balls of opposite colour are chosen?
	\end{enumerate}
	\solution
	%\input{exemplar/11/16/3/12/main1.tex}
\end{enumerate}

	\item A bag contains $5$ red balls and some blue balls. If the probability of drawing a blue ball is double that if a red ball, determine the number of blue balls in the bag. 
		\\
\solution
		%\begin{enumerate}[label=\thesection.\arabic*,ref=\thesection.\theenumi]
	\item One card is drawn from a well-shuffled deck of 52 cards. Find the probability of getting
\begin{enumerate}
\item A king of red colour 
\item A face card 
\item A red face card
\item The jack of hearts
\item A spade
\item The queen of diamonds

\end{enumerate}
\solution
		%\input{ncert/10/15/1/14/main.tex}
	\item Five cards—the ten, jack, queen, king and ace of diamonds, are well-shuffled with their face downwards. One card is then picked up at random.
\begin{enumerate}
\item
What is the probability that the card is the queen? 
\item
If the queen is drawn and put aside, what is the probability that the second card picked up is (a) an ace? (b) a queen?\\
\end{enumerate}
\solution
		%\input{ncert/10/15/1/15/defs.tex}
	\item A bag contains $5$ red balls and some blue balls. If the probability of drawing a blue ball is double that if a red ball, determine the number of blue balls in the bag. 
		\\
\solution
		%\input{ncert/10/15/2/3/defs.tex}
	\item A card is selected from a pack of 52 cards.
 \begin{enumerate}[label=(\alph*)] 
                 \item How many points are there in the sample space?
                 \item Calculate the probability that the card is an ace of spades.
                 \item Calculate the probability that the card is (i) an ace and (ii) black card.
 \end{enumerate}
\solution
		%\input{ncert/11/16/3/4/main.tex}
\item Four cards are drawn from a well-shuffled deck of 52 cards. What is the probability of obtaining 3 diamonds and one spade.
\\
\solution
		%\input{ncert/11/16/4/2/defs.tex}
\item In a certain lottery 10,000 tickets are sold and ten equal prizes are awarded. What is the probability of not getting a prize if you buy (a) one ticket (b) two tickets (c) 10 tickets ?	
\\
\solution
		%\input{ncert/11/16/4/4/defs.tex}
		%
\item 
Out of 100 students, two sections of 40 and 60 are formed. If you and your friend are among the 100 students, what is the probability that
\begin{enumerate}
\item you both enter the same section?
\item you both enter the different sections?
\end{enumerate}
\solution
		%\input{ncert/11/16/4/5/defs.tex}
	\item 
The number lock of a suitcase has 4 wheels each labelled with ten digits i.e. from 0 to 9.The lock opens with a sequence of four digits with no repeats.What is the probability of a person getting the right sequence to open the suitcase.
\\
\solution
		%\input{ncert/11/16/4/10/defs.tex}
		%
\item 
Two cards are drawn at random and without replacement from a pack of 52 playing cards. Find the probability that both the cards are black.
\\
\solution
		%\input{ncert/12/13/2/2/defs.tex}
		\item A box of oranges is inspected by examining three randomly selected oranges drawn without replacement. If all the three oranges are good, the box is approved for sale, otherwise, it is rejected. Find the probability that a box containing 15 oranges out of which 12 are good and 3 are bad ones will be approved for sale.
		\label{ncert/12/13/2/3/defs.tex}
		\item Two balls are drawn at random with replacement from a box containing 10 black and 8 red balls. Find the probability that
		\label{ncert/12/13/2/12}
\begin{enumerate}
\item both balls are red.
\item first ball is black and second is red.
\item one of them is black and other is red.
\end{enumerate}

\item In a hostel, 60\% of the students read Hindi newspaper, 40\% read English newspaper and 20\% read both Hindi and English newspapers. A student is selected at random.
		\label{ncert/12/13/2/15}
\begin{enumerate}
\item Find the probability that she reads neither Hindi nor English newspapers.
\item If she reads Hindi newspaper, find the probability that she reads English newspaper.
\item If she reads English newspaper, find the probability that she reads Hindi newspaper.\\
\end{enumerate}
\item The probability of obtaining an even prime number on each die, when a pair of dice is rolled is 
\begin{enumerate}
    \item $0$ 
    
    \item $\frac{1}{3}$ 
    
    \item $\frac{1}{12}$ 
    
    \item $\frac{1}{36}$ 
\end{enumerate}
\solution
		%\input{ncert/12/13/2/17/defs.tex}
	\item A bag contains 4 red and 4 black balls, another bag contains 2 red and 6 black balls. One of the two bags is selected at random and a ball is drawn from the bag which is found to be red. Find the probability that the ball is drawn from the first bag.
\\
\solution
		%\input{ncert/12/13/3/2/main.tex}
  \item
  Cards with numbers 2 to 101 are placed in a box. A card is selected at random.Find the probability that the card has
\begin{enumerate}[label=(\roman*)]
	\item an even number 
	\item a square number
\end{enumerate}
\solution
%\input{exemplar/10/13/3/32/main.tex}
\item
The king, queen and jack of clubs are removed from a deck of 52 playing cards and then well shuffled. Now one card is drawn at random from the remaining cards.  Determine the probability that the card is
\begin{enumerate}[label=(\roman*)]
\item a club
\item 10 of hearts
\end{enumerate}
\solution
%\input{exemplar/10/13/3/29/main.tex}
\item A team of medical students doing their internship have to assist during surgeries
at a city hospital. The probabilities of surgeries rated as very complex, complex,
routine, simple or very simple are respectively, 0.15, 0.20, 0.31, 0.26, .08. Find
the probabilities that a particular surgery will be rated
\begin{enumerate}
	\item complex or very complex;
	\item neither very complex nor very simple;
	\item routine or complex
	\item routine or simple
\end{enumerate}
\solution
%\input{exemplar/11/16/3/8(1)/main.tex}
\item A card is selected from a pack of 52 cards.
\begin{enumerate}[label=(\alph*)]
    \item How many points are there in the sample space?
    \item Calculate the probability that the card is an ace of spades.
    \item Calculate the probability that the card is (i) an ace and (ii) black card.
\end{enumerate}
\solution
%\input{exemplar/11/16/3/4/main2.tex}
\item The probability that a non leap year selected at random will contain 53 sundays.
\\
\solution
%\input{exemplar/10/13/1/19/main.tex}
\item One of the four persons John, Rita, Aslam or Gurpreet will be promoted next
month. Consequently the sample space consists of four elementary outcomes
S = {John promoted, Rita promoted, Aslam promoted, Gurpreet promoted}
You are told that the chances of John’s promotion is same as that of Gurpreet,
Rita’s chances of promotion are twice as likely as Johns. Aslam’s chances are
four times that of John.
\begin{enumerate}
	\item Determine
	\begin{enumerate}
		\item P (John promoted)
		\item P (Rita promoted)
		\item P (Aslam promoted)
		\item P (Gurpreet promoted)
	\end{enumerate}
	\item If A = {John promoted or Gurpreet promoted}, find P (A).
\end{enumerate}
\solution
%\input{exemplar/11/16/3/10/main.tex}
\item A card is drawn from a deck of 52 cards. Find the probability of getting a king or a heart or a red card.\\
\solution
%\input{exemplar/11/16/3/15/main.tex}
\item The probability that a student will pass his examination is 0.73, the probability of
the student getting a compartment is 0.13, and the probability that the student will
either pass or get compartment is 0.96. State True or False.\\
\solution
%\input{exemplar/11/16/3/31/main.tex}
\item A card is selected from a pack of 52 cards\\
\begin{enumerate}[label=(\alph*)]
\item How many points are there in the sample space?
\item Calculate the probability that the cards is an ace of spades.
\item Calculate the probability that the card is (i) an ace (ii)black card.\\
\end{enumerate}
%\input{ncert/11/16/3/4_1/Prob_4.tex}
\item In a non-leap year, the probability of having 53 tuesdays or 53 wednesdays is\\
\solution
%\input{exemplar/11/16/3/18/main.tex}
\item There are 1000 sealed envelopes in a box, 10 of them contain a cash prize of
Rs 100 each, 100 of them contain a cash prize of Rs 50 each and 200 of them
contain a cash prize of Rs 10 each and rest do not contain any cash prize. If they
are well shuffled and an envelope is picked up out, what is the probability that it
contains no cash prize?\\
\solution
%\input{exemplar/10/13/3/34/main.tex}
\item 
A die is thrown and a card is selected at random from a deck of 52 playing cards. The probability of getting an even number on the die and a spade card.\\
\solution
%\input{exemplar/12/13/3/78/main.tex}
\item
If 4-digit numbers greater than 5,000 are randomly formed from the digits 0, 1, 3, 5, and 7, what is the probability of forming a number divisible by 5 when:
\begin{enumerate}
    \item The digits are repeated?
    \item The repetition of digits is not allowed?
\end{enumerate}
\solution
%\input{ncert/11/16/4/9/main.tex}
\item Consider the probability space $\brak{\Omega, \mathcal{G}, P}$ where $\Omega = [0,2]$ and $\mathcal{G} = \cbrak{\phi, \Omega, [0,1], (1,2]}$. Let $X$ and $Y$ be two functions on $\Omega$ defined as
\begin{align*}
    X(\omega) = 
    \begin{cases}
        1 & \text{if }\omega \in [0, 1]\\
        2 & \text{if }\omega \in (1, 2]
    \end{cases}
\end{align*}
and
\begin{align*}
    Y(\omega) = 
    \begin{cases}
        2 & \text{if }\omega \in [0, 1.5]\\
        3 & \text{if }\omega \in (1.5, 2].
    \end{cases}
\end{align*}
Then which one of the following statements is true?
\begin{enumerate}
    \item [(A)] $X$ is a random variable with respect to $\mathcal{G}$, but $Y$ is not a random variable with respect to $\mathcal{G}$.
    \item [(B)] $Y$ is a random variable with respect to $\mathcal{G}$, but $X$ is not a random variable with respect to $\mathcal{G}$.
    \item [(C)] Neither $X$ nor $Y$ is a random variable with respect to $\mathcal{G}$.
    \item [(D)] Both $X$ and $Y$ are random variables with respect to $\mathcal{G}$.
\end{enumerate} \hfill (GATE ST 2023)\\
\solution
%\input{gate/ST/2023/14/main.tex}
	\item  A die is loaded in such a way that each odd number is twice as likely to occur as
each even number. Find $P(G)$, where $G$ is the event that a number greater than
3 occurs on a single roll of the die.
\\
\solution
		%\input{exemplar/11/16/3/5/main.tex}
	\item All the jacks, queens and kings are removed from a deck of 52 playing cards. The remaining cards are well shuffled and then one card is drawn at random. Giving ace a value 1 similar value for other cards, find the probability that the card has a value 
		\begin{enumerate}
			\item 7
			\item greater than 7
			\item less than 7
		\end{enumerate}
		%\input{exemplar/10/13/3/30/main.tex}
  \item A Lot consists of 48 mobile phones of which 42 are good, 3 have only minor defects and 3 have major defects.Varnika will buy a phone if it is good but the trader will only buy a mobile if it has no major defects. One phone is selected at random from the lot. What is the probability that it is
\begin{enumerate}
	\item acceptable to Varnika?
            \item acceptable to the trader?
\end{enumerate}
\solution
	%\input{exemplar/10/13/3/40/main.tex}
 \item A student says that if you throw a die, it will show up 1 or not 1. Therefore, the probability of getting 1 and the probability of getting 'not 1' each is equal to $\frac{1}{2}$. Is this correct? Give reasons.\\
 \solution
        %\input{exemplar/10/13/2/9/main.tex}
   \item Four candidates A, B, C, D have ap-
plied for the assignment to coach a school cricket
team. If A is twice as likely to be selected as B, and
B and C are given about the same chance of being
selected, while C is twice as likely to be selected
as D, what are the probabilities that
\begin{enumerate}
\item C will be selected?
\item A will not be selected?
\end{enumerate}
	%\input{exemplar/11/16/3/9/main.tex}
 \item A bag contain 24 balls of which $x$ balls are red, $2x$ are white and $3x$ are blue. A ball is selected at random, What is the probability that it is
\begin{enumerate}[label=\alph*)]
\item not red ?
\item white ?
\end{enumerate}
%\input{exemplar/10/13/3/41/main.tex}
If the letters of the word ASSASSINATION are arranged at random. Find the Probability that
\begin{enumerate}[label=(\alph*)]
\item Four $S's$ come consecutively in the word
\item Two  $I's$ and two $N's$ come together
\item All $A's$ are not coming together
\item No two $A's$ are coming together
\end{enumerate}
%\input{exemplar/11/16/3/14/main.tex}
	\item One urn contains two black balls (labelled B1 and B2) and one white ball. A
	second urn contains one black ball and two white balls (labelled W1 and W2).
	Suppose the following experiment is performed. One of the two urns is chosen
	at random. Next a ball is randomly chosen from the urn. Then a second ball is
	chosen at random from the same urn without replacing the first ball.
	
	\begin{enumerate}
	\item What is the probability that two black balls are chosen?
	
	\item What is the probability that two balls of opposite colour are chosen?
	\end{enumerate}
	\solution
	%\input{exemplar/11/16/3/12/main1.tex}
\end{enumerate}

	\item A card is selected from a pack of 52 cards.
 \begin{enumerate}[label=(\alph*)] 
                 \item How many points are there in the sample space?
                 \item Calculate the probability that the card is an ace of spades.
                 \item Calculate the probability that the card is (i) an ace and (ii) black card.
 \end{enumerate}
\solution
		%\begin{table}[H]
	\centering
\begin{tabular}{|c|c|c|}
\hline
Random variable &Value &Definition\\ \hline
\multirow{3}{*}{X} &0 &Slips of Rs 1\\
&1 &Slips of Rs 5\\
&2 &Slips of Rs 13\\ \hline
\multirow{2}{*}{Y} &0 &Box A\\
&1 &Box B\\\hline
\end{tabular}
\caption{}
\label{tab:Distribution}
\end{table}
See \tabref{tab:Distribution}.
\begin{align}
p_{Y}\brak{k}= \begin{cases} 
      \frac{1}{3} & {k=0} \\
      \frac{2}{3 }& {k=1} 
   \end{cases}
   \\
p_{Y|X}\brak{0|0} = \frac{19}{25}\, 
p_{Y|X}\brak{0|1} = \frac{6}{25}\,
p_{Y|X}\brak{1|0} = \frac{45}{50}\,
p_{Y|X}\brak{1|2} = \frac{5}{50}
\end{align}
The desired probability is the probability that a slip drawn at random is marked other than Rs 1,
\begin{align}
&=1-p_X\brak{0}\\
&= p_X(1) + p_X(2)
\end{align}
Using Bayes theorem,
\begin{align}
&= p_Y\brak{0} \times \pr{Y=0 | X=1} + p_Y\brak{1} \times \pr{Y=1|X=2}\\
&=\frac{1}{3} \times \frac{6}{25} + \frac{2}{3} \times \frac{5}{50}\\
&=\frac{11}{75}
\end{align}

\newpage

%\tableofcontents

\bigskip

\renewcommand{\thefigure}{\theenumi}
\renewcommand{\thetable}{\theenumi}
%\renewcommand{\theequation}{\theenumi}

%\begin{abstract}
%%\boldmath
%In this letter, an algorithm for evaluating the exact analytical bit error rate  (BER)  for the piecewise linear (PL) combiner for  multiple relays is presented. Previous results were available only for upto three relays. The algorithm is unique in the sense that  the actual mathematical expressions, that are prohibitively large, need not be explicitly obtained. The diversity gain due to multiple relays is shown through plots of the analytical BER, well supported by simulations. 
%
%\end{abstract}
% IEEEtran.cls defaults to using nonbold math in the Abstract.
% This preserves the distinction between vectors and scalars. However,
% if the journal you are submitting to favors bold math in the abstract,
% then you can use LaTeX's standard command \boldmath at the very start
% of the abstract to achieve this. Many IEEE journals frown on math
% in the abstract anyway.

% Note that keywords are not normally used for peerreview papers.
%\begin{IEEEkeywords}
%Cooperative diversity, decode and forward, piecewise linear
%\end{IEEEkeywords}



% For peer review papers, you can put extra information on the cover
% page as needed:
% \ifCLASSOPTIONpeerreview
% \begin{center} \bfseries EDICS Category: 3-BBND \end{center}
% \fi
%
% For peerreview papers, this IEEEtran command inserts a page break and
% creates the second title. It will be ignored for other modes.
%\IEEEpeerreviewmaketitle




\item Four cards are drawn from a well-shuffled deck of 52 cards. What is the probability of obtaining 3 diamonds and one spade.
\\
\solution
		%\begin{enumerate}[label=\thesection.\arabic*,ref=\thesection.\theenumi]
	\item One card is drawn from a well-shuffled deck of 52 cards. Find the probability of getting
\begin{enumerate}
\item A king of red colour 
\item A face card 
\item A red face card
\item The jack of hearts
\item A spade
\item The queen of diamonds

\end{enumerate}
\solution
		%\input{ncert/10/15/1/14/main.tex}
	\item Five cards—the ten, jack, queen, king and ace of diamonds, are well-shuffled with their face downwards. One card is then picked up at random.
\begin{enumerate}
\item
What is the probability that the card is the queen? 
\item
If the queen is drawn and put aside, what is the probability that the second card picked up is (a) an ace? (b) a queen?\\
\end{enumerate}
\solution
		%\input{ncert/10/15/1/15/defs.tex}
	\item A bag contains $5$ red balls and some blue balls. If the probability of drawing a blue ball is double that if a red ball, determine the number of blue balls in the bag. 
		\\
\solution
		%\input{ncert/10/15/2/3/defs.tex}
	\item A card is selected from a pack of 52 cards.
 \begin{enumerate}[label=(\alph*)] 
                 \item How many points are there in the sample space?
                 \item Calculate the probability that the card is an ace of spades.
                 \item Calculate the probability that the card is (i) an ace and (ii) black card.
 \end{enumerate}
\solution
		%\input{ncert/11/16/3/4/main.tex}
\item Four cards are drawn from a well-shuffled deck of 52 cards. What is the probability of obtaining 3 diamonds and one spade.
\\
\solution
		%\input{ncert/11/16/4/2/defs.tex}
\item In a certain lottery 10,000 tickets are sold and ten equal prizes are awarded. What is the probability of not getting a prize if you buy (a) one ticket (b) two tickets (c) 10 tickets ?	
\\
\solution
		%\input{ncert/11/16/4/4/defs.tex}
		%
\item 
Out of 100 students, two sections of 40 and 60 are formed. If you and your friend are among the 100 students, what is the probability that
\begin{enumerate}
\item you both enter the same section?
\item you both enter the different sections?
\end{enumerate}
\solution
		%\input{ncert/11/16/4/5/defs.tex}
	\item 
The number lock of a suitcase has 4 wheels each labelled with ten digits i.e. from 0 to 9.The lock opens with a sequence of four digits with no repeats.What is the probability of a person getting the right sequence to open the suitcase.
\\
\solution
		%\input{ncert/11/16/4/10/defs.tex}
		%
\item 
Two cards are drawn at random and without replacement from a pack of 52 playing cards. Find the probability that both the cards are black.
\\
\solution
		%\input{ncert/12/13/2/2/defs.tex}
		\item A box of oranges is inspected by examining three randomly selected oranges drawn without replacement. If all the three oranges are good, the box is approved for sale, otherwise, it is rejected. Find the probability that a box containing 15 oranges out of which 12 are good and 3 are bad ones will be approved for sale.
		\label{ncert/12/13/2/3/defs.tex}
		\item Two balls are drawn at random with replacement from a box containing 10 black and 8 red balls. Find the probability that
		\label{ncert/12/13/2/12}
\begin{enumerate}
\item both balls are red.
\item first ball is black and second is red.
\item one of them is black and other is red.
\end{enumerate}

\item In a hostel, 60\% of the students read Hindi newspaper, 40\% read English newspaper and 20\% read both Hindi and English newspapers. A student is selected at random.
		\label{ncert/12/13/2/15}
\begin{enumerate}
\item Find the probability that she reads neither Hindi nor English newspapers.
\item If she reads Hindi newspaper, find the probability that she reads English newspaper.
\item If she reads English newspaper, find the probability that she reads Hindi newspaper.\\
\end{enumerate}
\item The probability of obtaining an even prime number on each die, when a pair of dice is rolled is 
\begin{enumerate}
    \item $0$ 
    
    \item $\frac{1}{3}$ 
    
    \item $\frac{1}{12}$ 
    
    \item $\frac{1}{36}$ 
\end{enumerate}
\solution
		%\input{ncert/12/13/2/17/defs.tex}
	\item A bag contains 4 red and 4 black balls, another bag contains 2 red and 6 black balls. One of the two bags is selected at random and a ball is drawn from the bag which is found to be red. Find the probability that the ball is drawn from the first bag.
\\
\solution
		%\input{ncert/12/13/3/2/main.tex}
  \item
  Cards with numbers 2 to 101 are placed in a box. A card is selected at random.Find the probability that the card has
\begin{enumerate}[label=(\roman*)]
	\item an even number 
	\item a square number
\end{enumerate}
\solution
%\input{exemplar/10/13/3/32/main.tex}
\item
The king, queen and jack of clubs are removed from a deck of 52 playing cards and then well shuffled. Now one card is drawn at random from the remaining cards.  Determine the probability that the card is
\begin{enumerate}[label=(\roman*)]
\item a club
\item 10 of hearts
\end{enumerate}
\solution
%\input{exemplar/10/13/3/29/main.tex}
\item A team of medical students doing their internship have to assist during surgeries
at a city hospital. The probabilities of surgeries rated as very complex, complex,
routine, simple or very simple are respectively, 0.15, 0.20, 0.31, 0.26, .08. Find
the probabilities that a particular surgery will be rated
\begin{enumerate}
	\item complex or very complex;
	\item neither very complex nor very simple;
	\item routine or complex
	\item routine or simple
\end{enumerate}
\solution
%\input{exemplar/11/16/3/8(1)/main.tex}
\item A card is selected from a pack of 52 cards.
\begin{enumerate}[label=(\alph*)]
    \item How many points are there in the sample space?
    \item Calculate the probability that the card is an ace of spades.
    \item Calculate the probability that the card is (i) an ace and (ii) black card.
\end{enumerate}
\solution
%\input{exemplar/11/16/3/4/main2.tex}
\item The probability that a non leap year selected at random will contain 53 sundays.
\\
\solution
%\input{exemplar/10/13/1/19/main.tex}
\item One of the four persons John, Rita, Aslam or Gurpreet will be promoted next
month. Consequently the sample space consists of four elementary outcomes
S = {John promoted, Rita promoted, Aslam promoted, Gurpreet promoted}
You are told that the chances of John’s promotion is same as that of Gurpreet,
Rita’s chances of promotion are twice as likely as Johns. Aslam’s chances are
four times that of John.
\begin{enumerate}
	\item Determine
	\begin{enumerate}
		\item P (John promoted)
		\item P (Rita promoted)
		\item P (Aslam promoted)
		\item P (Gurpreet promoted)
	\end{enumerate}
	\item If A = {John promoted or Gurpreet promoted}, find P (A).
\end{enumerate}
\solution
%\input{exemplar/11/16/3/10/main.tex}
\item A card is drawn from a deck of 52 cards. Find the probability of getting a king or a heart or a red card.\\
\solution
%\input{exemplar/11/16/3/15/main.tex}
\item The probability that a student will pass his examination is 0.73, the probability of
the student getting a compartment is 0.13, and the probability that the student will
either pass or get compartment is 0.96. State True or False.\\
\solution
%\input{exemplar/11/16/3/31/main.tex}
\item A card is selected from a pack of 52 cards\\
\begin{enumerate}[label=(\alph*)]
\item How many points are there in the sample space?
\item Calculate the probability that the cards is an ace of spades.
\item Calculate the probability that the card is (i) an ace (ii)black card.\\
\end{enumerate}
%\input{ncert/11/16/3/4_1/Prob_4.tex}
\item In a non-leap year, the probability of having 53 tuesdays or 53 wednesdays is\\
\solution
%\input{exemplar/11/16/3/18/main.tex}
\item There are 1000 sealed envelopes in a box, 10 of them contain a cash prize of
Rs 100 each, 100 of them contain a cash prize of Rs 50 each and 200 of them
contain a cash prize of Rs 10 each and rest do not contain any cash prize. If they
are well shuffled and an envelope is picked up out, what is the probability that it
contains no cash prize?\\
\solution
%\input{exemplar/10/13/3/34/main.tex}
\item 
A die is thrown and a card is selected at random from a deck of 52 playing cards. The probability of getting an even number on the die and a spade card.\\
\solution
%\input{exemplar/12/13/3/78/main.tex}
\item
If 4-digit numbers greater than 5,000 are randomly formed from the digits 0, 1, 3, 5, and 7, what is the probability of forming a number divisible by 5 when:
\begin{enumerate}
    \item The digits are repeated?
    \item The repetition of digits is not allowed?
\end{enumerate}
\solution
%\input{ncert/11/16/4/9/main.tex}
\item Consider the probability space $\brak{\Omega, \mathcal{G}, P}$ where $\Omega = [0,2]$ and $\mathcal{G} = \cbrak{\phi, \Omega, [0,1], (1,2]}$. Let $X$ and $Y$ be two functions on $\Omega$ defined as
\begin{align*}
    X(\omega) = 
    \begin{cases}
        1 & \text{if }\omega \in [0, 1]\\
        2 & \text{if }\omega \in (1, 2]
    \end{cases}
\end{align*}
and
\begin{align*}
    Y(\omega) = 
    \begin{cases}
        2 & \text{if }\omega \in [0, 1.5]\\
        3 & \text{if }\omega \in (1.5, 2].
    \end{cases}
\end{align*}
Then which one of the following statements is true?
\begin{enumerate}
    \item [(A)] $X$ is a random variable with respect to $\mathcal{G}$, but $Y$ is not a random variable with respect to $\mathcal{G}$.
    \item [(B)] $Y$ is a random variable with respect to $\mathcal{G}$, but $X$ is not a random variable with respect to $\mathcal{G}$.
    \item [(C)] Neither $X$ nor $Y$ is a random variable with respect to $\mathcal{G}$.
    \item [(D)] Both $X$ and $Y$ are random variables with respect to $\mathcal{G}$.
\end{enumerate} \hfill (GATE ST 2023)\\
\solution
%\input{gate/ST/2023/14/main.tex}
	\item  A die is loaded in such a way that each odd number is twice as likely to occur as
each even number. Find $P(G)$, where $G$ is the event that a number greater than
3 occurs on a single roll of the die.
\\
\solution
		%\input{exemplar/11/16/3/5/main.tex}
	\item All the jacks, queens and kings are removed from a deck of 52 playing cards. The remaining cards are well shuffled and then one card is drawn at random. Giving ace a value 1 similar value for other cards, find the probability that the card has a value 
		\begin{enumerate}
			\item 7
			\item greater than 7
			\item less than 7
		\end{enumerate}
		%\input{exemplar/10/13/3/30/main.tex}
  \item A Lot consists of 48 mobile phones of which 42 are good, 3 have only minor defects and 3 have major defects.Varnika will buy a phone if it is good but the trader will only buy a mobile if it has no major defects. One phone is selected at random from the lot. What is the probability that it is
\begin{enumerate}
	\item acceptable to Varnika?
            \item acceptable to the trader?
\end{enumerate}
\solution
	%\input{exemplar/10/13/3/40/main.tex}
 \item A student says that if you throw a die, it will show up 1 or not 1. Therefore, the probability of getting 1 and the probability of getting 'not 1' each is equal to $\frac{1}{2}$. Is this correct? Give reasons.\\
 \solution
        %\input{exemplar/10/13/2/9/main.tex}
   \item Four candidates A, B, C, D have ap-
plied for the assignment to coach a school cricket
team. If A is twice as likely to be selected as B, and
B and C are given about the same chance of being
selected, while C is twice as likely to be selected
as D, what are the probabilities that
\begin{enumerate}
\item C will be selected?
\item A will not be selected?
\end{enumerate}
	%\input{exemplar/11/16/3/9/main.tex}
 \item A bag contain 24 balls of which $x$ balls are red, $2x$ are white and $3x$ are blue. A ball is selected at random, What is the probability that it is
\begin{enumerate}[label=\alph*)]
\item not red ?
\item white ?
\end{enumerate}
%\input{exemplar/10/13/3/41/main.tex}
If the letters of the word ASSASSINATION are arranged at random. Find the Probability that
\begin{enumerate}[label=(\alph*)]
\item Four $S's$ come consecutively in the word
\item Two  $I's$ and two $N's$ come together
\item All $A's$ are not coming together
\item No two $A's$ are coming together
\end{enumerate}
%\input{exemplar/11/16/3/14/main.tex}
	\item One urn contains two black balls (labelled B1 and B2) and one white ball. A
	second urn contains one black ball and two white balls (labelled W1 and W2).
	Suppose the following experiment is performed. One of the two urns is chosen
	at random. Next a ball is randomly chosen from the urn. Then a second ball is
	chosen at random from the same urn without replacing the first ball.
	
	\begin{enumerate}
	\item What is the probability that two black balls are chosen?
	
	\item What is the probability that two balls of opposite colour are chosen?
	\end{enumerate}
	\solution
	%\input{exemplar/11/16/3/12/main1.tex}
\end{enumerate}

\item In a certain lottery 10,000 tickets are sold and ten equal prizes are awarded. What is the probability of not getting a prize if you buy (a) one ticket (b) two tickets (c) 10 tickets ?	
\\
\solution
		%\begin{enumerate}[label=\thesection.\arabic*,ref=\thesection.\theenumi]
	\item One card is drawn from a well-shuffled deck of 52 cards. Find the probability of getting
\begin{enumerate}
\item A king of red colour 
\item A face card 
\item A red face card
\item The jack of hearts
\item A spade
\item The queen of diamonds

\end{enumerate}
\solution
		%\input{ncert/10/15/1/14/main.tex}
	\item Five cards—the ten, jack, queen, king and ace of diamonds, are well-shuffled with their face downwards. One card is then picked up at random.
\begin{enumerate}
\item
What is the probability that the card is the queen? 
\item
If the queen is drawn and put aside, what is the probability that the second card picked up is (a) an ace? (b) a queen?\\
\end{enumerate}
\solution
		%\input{ncert/10/15/1/15/defs.tex}
	\item A bag contains $5$ red balls and some blue balls. If the probability of drawing a blue ball is double that if a red ball, determine the number of blue balls in the bag. 
		\\
\solution
		%\input{ncert/10/15/2/3/defs.tex}
	\item A card is selected from a pack of 52 cards.
 \begin{enumerate}[label=(\alph*)] 
                 \item How many points are there in the sample space?
                 \item Calculate the probability that the card is an ace of spades.
                 \item Calculate the probability that the card is (i) an ace and (ii) black card.
 \end{enumerate}
\solution
		%\input{ncert/11/16/3/4/main.tex}
\item Four cards are drawn from a well-shuffled deck of 52 cards. What is the probability of obtaining 3 diamonds and one spade.
\\
\solution
		%\input{ncert/11/16/4/2/defs.tex}
\item In a certain lottery 10,000 tickets are sold and ten equal prizes are awarded. What is the probability of not getting a prize if you buy (a) one ticket (b) two tickets (c) 10 tickets ?	
\\
\solution
		%\input{ncert/11/16/4/4/defs.tex}
		%
\item 
Out of 100 students, two sections of 40 and 60 are formed. If you and your friend are among the 100 students, what is the probability that
\begin{enumerate}
\item you both enter the same section?
\item you both enter the different sections?
\end{enumerate}
\solution
		%\input{ncert/11/16/4/5/defs.tex}
	\item 
The number lock of a suitcase has 4 wheels each labelled with ten digits i.e. from 0 to 9.The lock opens with a sequence of four digits with no repeats.What is the probability of a person getting the right sequence to open the suitcase.
\\
\solution
		%\input{ncert/11/16/4/10/defs.tex}
		%
\item 
Two cards are drawn at random and without replacement from a pack of 52 playing cards. Find the probability that both the cards are black.
\\
\solution
		%\input{ncert/12/13/2/2/defs.tex}
		\item A box of oranges is inspected by examining three randomly selected oranges drawn without replacement. If all the three oranges are good, the box is approved for sale, otherwise, it is rejected. Find the probability that a box containing 15 oranges out of which 12 are good and 3 are bad ones will be approved for sale.
		\label{ncert/12/13/2/3/defs.tex}
		\item Two balls are drawn at random with replacement from a box containing 10 black and 8 red balls. Find the probability that
		\label{ncert/12/13/2/12}
\begin{enumerate}
\item both balls are red.
\item first ball is black and second is red.
\item one of them is black and other is red.
\end{enumerate}

\item In a hostel, 60\% of the students read Hindi newspaper, 40\% read English newspaper and 20\% read both Hindi and English newspapers. A student is selected at random.
		\label{ncert/12/13/2/15}
\begin{enumerate}
\item Find the probability that she reads neither Hindi nor English newspapers.
\item If she reads Hindi newspaper, find the probability that she reads English newspaper.
\item If she reads English newspaper, find the probability that she reads Hindi newspaper.\\
\end{enumerate}
\item The probability of obtaining an even prime number on each die, when a pair of dice is rolled is 
\begin{enumerate}
    \item $0$ 
    
    \item $\frac{1}{3}$ 
    
    \item $\frac{1}{12}$ 
    
    \item $\frac{1}{36}$ 
\end{enumerate}
\solution
		%\input{ncert/12/13/2/17/defs.tex}
	\item A bag contains 4 red and 4 black balls, another bag contains 2 red and 6 black balls. One of the two bags is selected at random and a ball is drawn from the bag which is found to be red. Find the probability that the ball is drawn from the first bag.
\\
\solution
		%\input{ncert/12/13/3/2/main.tex}
  \item
  Cards with numbers 2 to 101 are placed in a box. A card is selected at random.Find the probability that the card has
\begin{enumerate}[label=(\roman*)]
	\item an even number 
	\item a square number
\end{enumerate}
\solution
%\input{exemplar/10/13/3/32/main.tex}
\item
The king, queen and jack of clubs are removed from a deck of 52 playing cards and then well shuffled. Now one card is drawn at random from the remaining cards.  Determine the probability that the card is
\begin{enumerate}[label=(\roman*)]
\item a club
\item 10 of hearts
\end{enumerate}
\solution
%\input{exemplar/10/13/3/29/main.tex}
\item A team of medical students doing their internship have to assist during surgeries
at a city hospital. The probabilities of surgeries rated as very complex, complex,
routine, simple or very simple are respectively, 0.15, 0.20, 0.31, 0.26, .08. Find
the probabilities that a particular surgery will be rated
\begin{enumerate}
	\item complex or very complex;
	\item neither very complex nor very simple;
	\item routine or complex
	\item routine or simple
\end{enumerate}
\solution
%\input{exemplar/11/16/3/8(1)/main.tex}
\item A card is selected from a pack of 52 cards.
\begin{enumerate}[label=(\alph*)]
    \item How many points are there in the sample space?
    \item Calculate the probability that the card is an ace of spades.
    \item Calculate the probability that the card is (i) an ace and (ii) black card.
\end{enumerate}
\solution
%\input{exemplar/11/16/3/4/main2.tex}
\item The probability that a non leap year selected at random will contain 53 sundays.
\\
\solution
%\input{exemplar/10/13/1/19/main.tex}
\item One of the four persons John, Rita, Aslam or Gurpreet will be promoted next
month. Consequently the sample space consists of four elementary outcomes
S = {John promoted, Rita promoted, Aslam promoted, Gurpreet promoted}
You are told that the chances of John’s promotion is same as that of Gurpreet,
Rita’s chances of promotion are twice as likely as Johns. Aslam’s chances are
four times that of John.
\begin{enumerate}
	\item Determine
	\begin{enumerate}
		\item P (John promoted)
		\item P (Rita promoted)
		\item P (Aslam promoted)
		\item P (Gurpreet promoted)
	\end{enumerate}
	\item If A = {John promoted or Gurpreet promoted}, find P (A).
\end{enumerate}
\solution
%\input{exemplar/11/16/3/10/main.tex}
\item A card is drawn from a deck of 52 cards. Find the probability of getting a king or a heart or a red card.\\
\solution
%\input{exemplar/11/16/3/15/main.tex}
\item The probability that a student will pass his examination is 0.73, the probability of
the student getting a compartment is 0.13, and the probability that the student will
either pass or get compartment is 0.96. State True or False.\\
\solution
%\input{exemplar/11/16/3/31/main.tex}
\item A card is selected from a pack of 52 cards\\
\begin{enumerate}[label=(\alph*)]
\item How many points are there in the sample space?
\item Calculate the probability that the cards is an ace of spades.
\item Calculate the probability that the card is (i) an ace (ii)black card.\\
\end{enumerate}
%\input{ncert/11/16/3/4_1/Prob_4.tex}
\item In a non-leap year, the probability of having 53 tuesdays or 53 wednesdays is\\
\solution
%\input{exemplar/11/16/3/18/main.tex}
\item There are 1000 sealed envelopes in a box, 10 of them contain a cash prize of
Rs 100 each, 100 of them contain a cash prize of Rs 50 each and 200 of them
contain a cash prize of Rs 10 each and rest do not contain any cash prize. If they
are well shuffled and an envelope is picked up out, what is the probability that it
contains no cash prize?\\
\solution
%\input{exemplar/10/13/3/34/main.tex}
\item 
A die is thrown and a card is selected at random from a deck of 52 playing cards. The probability of getting an even number on the die and a spade card.\\
\solution
%\input{exemplar/12/13/3/78/main.tex}
\item
If 4-digit numbers greater than 5,000 are randomly formed from the digits 0, 1, 3, 5, and 7, what is the probability of forming a number divisible by 5 when:
\begin{enumerate}
    \item The digits are repeated?
    \item The repetition of digits is not allowed?
\end{enumerate}
\solution
%\input{ncert/11/16/4/9/main.tex}
\item Consider the probability space $\brak{\Omega, \mathcal{G}, P}$ where $\Omega = [0,2]$ and $\mathcal{G} = \cbrak{\phi, \Omega, [0,1], (1,2]}$. Let $X$ and $Y$ be two functions on $\Omega$ defined as
\begin{align*}
    X(\omega) = 
    \begin{cases}
        1 & \text{if }\omega \in [0, 1]\\
        2 & \text{if }\omega \in (1, 2]
    \end{cases}
\end{align*}
and
\begin{align*}
    Y(\omega) = 
    \begin{cases}
        2 & \text{if }\omega \in [0, 1.5]\\
        3 & \text{if }\omega \in (1.5, 2].
    \end{cases}
\end{align*}
Then which one of the following statements is true?
\begin{enumerate}
    \item [(A)] $X$ is a random variable with respect to $\mathcal{G}$, but $Y$ is not a random variable with respect to $\mathcal{G}$.
    \item [(B)] $Y$ is a random variable with respect to $\mathcal{G}$, but $X$ is not a random variable with respect to $\mathcal{G}$.
    \item [(C)] Neither $X$ nor $Y$ is a random variable with respect to $\mathcal{G}$.
    \item [(D)] Both $X$ and $Y$ are random variables with respect to $\mathcal{G}$.
\end{enumerate} \hfill (GATE ST 2023)\\
\solution
%\input{gate/ST/2023/14/main.tex}
	\item  A die is loaded in such a way that each odd number is twice as likely to occur as
each even number. Find $P(G)$, where $G$ is the event that a number greater than
3 occurs on a single roll of the die.
\\
\solution
		%\input{exemplar/11/16/3/5/main.tex}
	\item All the jacks, queens and kings are removed from a deck of 52 playing cards. The remaining cards are well shuffled and then one card is drawn at random. Giving ace a value 1 similar value for other cards, find the probability that the card has a value 
		\begin{enumerate}
			\item 7
			\item greater than 7
			\item less than 7
		\end{enumerate}
		%\input{exemplar/10/13/3/30/main.tex}
  \item A Lot consists of 48 mobile phones of which 42 are good, 3 have only minor defects and 3 have major defects.Varnika will buy a phone if it is good but the trader will only buy a mobile if it has no major defects. One phone is selected at random from the lot. What is the probability that it is
\begin{enumerate}
	\item acceptable to Varnika?
            \item acceptable to the trader?
\end{enumerate}
\solution
	%\input{exemplar/10/13/3/40/main.tex}
 \item A student says that if you throw a die, it will show up 1 or not 1. Therefore, the probability of getting 1 and the probability of getting 'not 1' each is equal to $\frac{1}{2}$. Is this correct? Give reasons.\\
 \solution
        %\input{exemplar/10/13/2/9/main.tex}
   \item Four candidates A, B, C, D have ap-
plied for the assignment to coach a school cricket
team. If A is twice as likely to be selected as B, and
B and C are given about the same chance of being
selected, while C is twice as likely to be selected
as D, what are the probabilities that
\begin{enumerate}
\item C will be selected?
\item A will not be selected?
\end{enumerate}
	%\input{exemplar/11/16/3/9/main.tex}
 \item A bag contain 24 balls of which $x$ balls are red, $2x$ are white and $3x$ are blue. A ball is selected at random, What is the probability that it is
\begin{enumerate}[label=\alph*)]
\item not red ?
\item white ?
\end{enumerate}
%\input{exemplar/10/13/3/41/main.tex}
If the letters of the word ASSASSINATION are arranged at random. Find the Probability that
\begin{enumerate}[label=(\alph*)]
\item Four $S's$ come consecutively in the word
\item Two  $I's$ and two $N's$ come together
\item All $A's$ are not coming together
\item No two $A's$ are coming together
\end{enumerate}
%\input{exemplar/11/16/3/14/main.tex}
	\item One urn contains two black balls (labelled B1 and B2) and one white ball. A
	second urn contains one black ball and two white balls (labelled W1 and W2).
	Suppose the following experiment is performed. One of the two urns is chosen
	at random. Next a ball is randomly chosen from the urn. Then a second ball is
	chosen at random from the same urn without replacing the first ball.
	
	\begin{enumerate}
	\item What is the probability that two black balls are chosen?
	
	\item What is the probability that two balls of opposite colour are chosen?
	\end{enumerate}
	\solution
	%\input{exemplar/11/16/3/12/main1.tex}
\end{enumerate}

		%
\item 
Out of 100 students, two sections of 40 and 60 are formed. If you and your friend are among the 100 students, what is the probability that
\begin{enumerate}
\item you both enter the same section?
\item you both enter the different sections?
\end{enumerate}
\solution
		%\begin{enumerate}[label=\thesection.\arabic*,ref=\thesection.\theenumi]
	\item One card is drawn from a well-shuffled deck of 52 cards. Find the probability of getting
\begin{enumerate}
\item A king of red colour 
\item A face card 
\item A red face card
\item The jack of hearts
\item A spade
\item The queen of diamonds

\end{enumerate}
\solution
		%\input{ncert/10/15/1/14/main.tex}
	\item Five cards—the ten, jack, queen, king and ace of diamonds, are well-shuffled with their face downwards. One card is then picked up at random.
\begin{enumerate}
\item
What is the probability that the card is the queen? 
\item
If the queen is drawn and put aside, what is the probability that the second card picked up is (a) an ace? (b) a queen?\\
\end{enumerate}
\solution
		%\input{ncert/10/15/1/15/defs.tex}
	\item A bag contains $5$ red balls and some blue balls. If the probability of drawing a blue ball is double that if a red ball, determine the number of blue balls in the bag. 
		\\
\solution
		%\input{ncert/10/15/2/3/defs.tex}
	\item A card is selected from a pack of 52 cards.
 \begin{enumerate}[label=(\alph*)] 
                 \item How many points are there in the sample space?
                 \item Calculate the probability that the card is an ace of spades.
                 \item Calculate the probability that the card is (i) an ace and (ii) black card.
 \end{enumerate}
\solution
		%\input{ncert/11/16/3/4/main.tex}
\item Four cards are drawn from a well-shuffled deck of 52 cards. What is the probability of obtaining 3 diamonds and one spade.
\\
\solution
		%\input{ncert/11/16/4/2/defs.tex}
\item In a certain lottery 10,000 tickets are sold and ten equal prizes are awarded. What is the probability of not getting a prize if you buy (a) one ticket (b) two tickets (c) 10 tickets ?	
\\
\solution
		%\input{ncert/11/16/4/4/defs.tex}
		%
\item 
Out of 100 students, two sections of 40 and 60 are formed. If you and your friend are among the 100 students, what is the probability that
\begin{enumerate}
\item you both enter the same section?
\item you both enter the different sections?
\end{enumerate}
\solution
		%\input{ncert/11/16/4/5/defs.tex}
	\item 
The number lock of a suitcase has 4 wheels each labelled with ten digits i.e. from 0 to 9.The lock opens with a sequence of four digits with no repeats.What is the probability of a person getting the right sequence to open the suitcase.
\\
\solution
		%\input{ncert/11/16/4/10/defs.tex}
		%
\item 
Two cards are drawn at random and without replacement from a pack of 52 playing cards. Find the probability that both the cards are black.
\\
\solution
		%\input{ncert/12/13/2/2/defs.tex}
		\item A box of oranges is inspected by examining three randomly selected oranges drawn without replacement. If all the three oranges are good, the box is approved for sale, otherwise, it is rejected. Find the probability that a box containing 15 oranges out of which 12 are good and 3 are bad ones will be approved for sale.
		\label{ncert/12/13/2/3/defs.tex}
		\item Two balls are drawn at random with replacement from a box containing 10 black and 8 red balls. Find the probability that
		\label{ncert/12/13/2/12}
\begin{enumerate}
\item both balls are red.
\item first ball is black and second is red.
\item one of them is black and other is red.
\end{enumerate}

\item In a hostel, 60\% of the students read Hindi newspaper, 40\% read English newspaper and 20\% read both Hindi and English newspapers. A student is selected at random.
		\label{ncert/12/13/2/15}
\begin{enumerate}
\item Find the probability that she reads neither Hindi nor English newspapers.
\item If she reads Hindi newspaper, find the probability that she reads English newspaper.
\item If she reads English newspaper, find the probability that she reads Hindi newspaper.\\
\end{enumerate}
\item The probability of obtaining an even prime number on each die, when a pair of dice is rolled is 
\begin{enumerate}
    \item $0$ 
    
    \item $\frac{1}{3}$ 
    
    \item $\frac{1}{12}$ 
    
    \item $\frac{1}{36}$ 
\end{enumerate}
\solution
		%\input{ncert/12/13/2/17/defs.tex}
	\item A bag contains 4 red and 4 black balls, another bag contains 2 red and 6 black balls. One of the two bags is selected at random and a ball is drawn from the bag which is found to be red. Find the probability that the ball is drawn from the first bag.
\\
\solution
		%\input{ncert/12/13/3/2/main.tex}
  \item
  Cards with numbers 2 to 101 are placed in a box. A card is selected at random.Find the probability that the card has
\begin{enumerate}[label=(\roman*)]
	\item an even number 
	\item a square number
\end{enumerate}
\solution
%\input{exemplar/10/13/3/32/main.tex}
\item
The king, queen and jack of clubs are removed from a deck of 52 playing cards and then well shuffled. Now one card is drawn at random from the remaining cards.  Determine the probability that the card is
\begin{enumerate}[label=(\roman*)]
\item a club
\item 10 of hearts
\end{enumerate}
\solution
%\input{exemplar/10/13/3/29/main.tex}
\item A team of medical students doing their internship have to assist during surgeries
at a city hospital. The probabilities of surgeries rated as very complex, complex,
routine, simple or very simple are respectively, 0.15, 0.20, 0.31, 0.26, .08. Find
the probabilities that a particular surgery will be rated
\begin{enumerate}
	\item complex or very complex;
	\item neither very complex nor very simple;
	\item routine or complex
	\item routine or simple
\end{enumerate}
\solution
%\input{exemplar/11/16/3/8(1)/main.tex}
\item A card is selected from a pack of 52 cards.
\begin{enumerate}[label=(\alph*)]
    \item How many points are there in the sample space?
    \item Calculate the probability that the card is an ace of spades.
    \item Calculate the probability that the card is (i) an ace and (ii) black card.
\end{enumerate}
\solution
%\input{exemplar/11/16/3/4/main2.tex}
\item The probability that a non leap year selected at random will contain 53 sundays.
\\
\solution
%\input{exemplar/10/13/1/19/main.tex}
\item One of the four persons John, Rita, Aslam or Gurpreet will be promoted next
month. Consequently the sample space consists of four elementary outcomes
S = {John promoted, Rita promoted, Aslam promoted, Gurpreet promoted}
You are told that the chances of John’s promotion is same as that of Gurpreet,
Rita’s chances of promotion are twice as likely as Johns. Aslam’s chances are
four times that of John.
\begin{enumerate}
	\item Determine
	\begin{enumerate}
		\item P (John promoted)
		\item P (Rita promoted)
		\item P (Aslam promoted)
		\item P (Gurpreet promoted)
	\end{enumerate}
	\item If A = {John promoted or Gurpreet promoted}, find P (A).
\end{enumerate}
\solution
%\input{exemplar/11/16/3/10/main.tex}
\item A card is drawn from a deck of 52 cards. Find the probability of getting a king or a heart or a red card.\\
\solution
%\input{exemplar/11/16/3/15/main.tex}
\item The probability that a student will pass his examination is 0.73, the probability of
the student getting a compartment is 0.13, and the probability that the student will
either pass or get compartment is 0.96. State True or False.\\
\solution
%\input{exemplar/11/16/3/31/main.tex}
\item A card is selected from a pack of 52 cards\\
\begin{enumerate}[label=(\alph*)]
\item How many points are there in the sample space?
\item Calculate the probability that the cards is an ace of spades.
\item Calculate the probability that the card is (i) an ace (ii)black card.\\
\end{enumerate}
%\input{ncert/11/16/3/4_1/Prob_4.tex}
\item In a non-leap year, the probability of having 53 tuesdays or 53 wednesdays is\\
\solution
%\input{exemplar/11/16/3/18/main.tex}
\item There are 1000 sealed envelopes in a box, 10 of them contain a cash prize of
Rs 100 each, 100 of them contain a cash prize of Rs 50 each and 200 of them
contain a cash prize of Rs 10 each and rest do not contain any cash prize. If they
are well shuffled and an envelope is picked up out, what is the probability that it
contains no cash prize?\\
\solution
%\input{exemplar/10/13/3/34/main.tex}
\item 
A die is thrown and a card is selected at random from a deck of 52 playing cards. The probability of getting an even number on the die and a spade card.\\
\solution
%\input{exemplar/12/13/3/78/main.tex}
\item
If 4-digit numbers greater than 5,000 are randomly formed from the digits 0, 1, 3, 5, and 7, what is the probability of forming a number divisible by 5 when:
\begin{enumerate}
    \item The digits are repeated?
    \item The repetition of digits is not allowed?
\end{enumerate}
\solution
%\input{ncert/11/16/4/9/main.tex}
\item Consider the probability space $\brak{\Omega, \mathcal{G}, P}$ where $\Omega = [0,2]$ and $\mathcal{G} = \cbrak{\phi, \Omega, [0,1], (1,2]}$. Let $X$ and $Y$ be two functions on $\Omega$ defined as
\begin{align*}
    X(\omega) = 
    \begin{cases}
        1 & \text{if }\omega \in [0, 1]\\
        2 & \text{if }\omega \in (1, 2]
    \end{cases}
\end{align*}
and
\begin{align*}
    Y(\omega) = 
    \begin{cases}
        2 & \text{if }\omega \in [0, 1.5]\\
        3 & \text{if }\omega \in (1.5, 2].
    \end{cases}
\end{align*}
Then which one of the following statements is true?
\begin{enumerate}
    \item [(A)] $X$ is a random variable with respect to $\mathcal{G}$, but $Y$ is not a random variable with respect to $\mathcal{G}$.
    \item [(B)] $Y$ is a random variable with respect to $\mathcal{G}$, but $X$ is not a random variable with respect to $\mathcal{G}$.
    \item [(C)] Neither $X$ nor $Y$ is a random variable with respect to $\mathcal{G}$.
    \item [(D)] Both $X$ and $Y$ are random variables with respect to $\mathcal{G}$.
\end{enumerate} \hfill (GATE ST 2023)\\
\solution
%\input{gate/ST/2023/14/main.tex}
	\item  A die is loaded in such a way that each odd number is twice as likely to occur as
each even number. Find $P(G)$, where $G$ is the event that a number greater than
3 occurs on a single roll of the die.
\\
\solution
		%\input{exemplar/11/16/3/5/main.tex}
	\item All the jacks, queens and kings are removed from a deck of 52 playing cards. The remaining cards are well shuffled and then one card is drawn at random. Giving ace a value 1 similar value for other cards, find the probability that the card has a value 
		\begin{enumerate}
			\item 7
			\item greater than 7
			\item less than 7
		\end{enumerate}
		%\input{exemplar/10/13/3/30/main.tex}
  \item A Lot consists of 48 mobile phones of which 42 are good, 3 have only minor defects and 3 have major defects.Varnika will buy a phone if it is good but the trader will only buy a mobile if it has no major defects. One phone is selected at random from the lot. What is the probability that it is
\begin{enumerate}
	\item acceptable to Varnika?
            \item acceptable to the trader?
\end{enumerate}
\solution
	%\input{exemplar/10/13/3/40/main.tex}
 \item A student says that if you throw a die, it will show up 1 or not 1. Therefore, the probability of getting 1 and the probability of getting 'not 1' each is equal to $\frac{1}{2}$. Is this correct? Give reasons.\\
 \solution
        %\input{exemplar/10/13/2/9/main.tex}
   \item Four candidates A, B, C, D have ap-
plied for the assignment to coach a school cricket
team. If A is twice as likely to be selected as B, and
B and C are given about the same chance of being
selected, while C is twice as likely to be selected
as D, what are the probabilities that
\begin{enumerate}
\item C will be selected?
\item A will not be selected?
\end{enumerate}
	%\input{exemplar/11/16/3/9/main.tex}
 \item A bag contain 24 balls of which $x$ balls are red, $2x$ are white and $3x$ are blue. A ball is selected at random, What is the probability that it is
\begin{enumerate}[label=\alph*)]
\item not red ?
\item white ?
\end{enumerate}
%\input{exemplar/10/13/3/41/main.tex}
If the letters of the word ASSASSINATION are arranged at random. Find the Probability that
\begin{enumerate}[label=(\alph*)]
\item Four $S's$ come consecutively in the word
\item Two  $I's$ and two $N's$ come together
\item All $A's$ are not coming together
\item No two $A's$ are coming together
\end{enumerate}
%\input{exemplar/11/16/3/14/main.tex}
	\item One urn contains two black balls (labelled B1 and B2) and one white ball. A
	second urn contains one black ball and two white balls (labelled W1 and W2).
	Suppose the following experiment is performed. One of the two urns is chosen
	at random. Next a ball is randomly chosen from the urn. Then a second ball is
	chosen at random from the same urn without replacing the first ball.
	
	\begin{enumerate}
	\item What is the probability that two black balls are chosen?
	
	\item What is the probability that two balls of opposite colour are chosen?
	\end{enumerate}
	\solution
	%\input{exemplar/11/16/3/12/main1.tex}
\end{enumerate}

	\item 
The number lock of a suitcase has 4 wheels each labelled with ten digits i.e. from 0 to 9.The lock opens with a sequence of four digits with no repeats.What is the probability of a person getting the right sequence to open the suitcase.
\\
\solution
		%\begin{enumerate}[label=\thesection.\arabic*,ref=\thesection.\theenumi]
	\item One card is drawn from a well-shuffled deck of 52 cards. Find the probability of getting
\begin{enumerate}
\item A king of red colour 
\item A face card 
\item A red face card
\item The jack of hearts
\item A spade
\item The queen of diamonds

\end{enumerate}
\solution
		%\input{ncert/10/15/1/14/main.tex}
	\item Five cards—the ten, jack, queen, king and ace of diamonds, are well-shuffled with their face downwards. One card is then picked up at random.
\begin{enumerate}
\item
What is the probability that the card is the queen? 
\item
If the queen is drawn and put aside, what is the probability that the second card picked up is (a) an ace? (b) a queen?\\
\end{enumerate}
\solution
		%\input{ncert/10/15/1/15/defs.tex}
	\item A bag contains $5$ red balls and some blue balls. If the probability of drawing a blue ball is double that if a red ball, determine the number of blue balls in the bag. 
		\\
\solution
		%\input{ncert/10/15/2/3/defs.tex}
	\item A card is selected from a pack of 52 cards.
 \begin{enumerate}[label=(\alph*)] 
                 \item How many points are there in the sample space?
                 \item Calculate the probability that the card is an ace of spades.
                 \item Calculate the probability that the card is (i) an ace and (ii) black card.
 \end{enumerate}
\solution
		%\input{ncert/11/16/3/4/main.tex}
\item Four cards are drawn from a well-shuffled deck of 52 cards. What is the probability of obtaining 3 diamonds and one spade.
\\
\solution
		%\input{ncert/11/16/4/2/defs.tex}
\item In a certain lottery 10,000 tickets are sold and ten equal prizes are awarded. What is the probability of not getting a prize if you buy (a) one ticket (b) two tickets (c) 10 tickets ?	
\\
\solution
		%\input{ncert/11/16/4/4/defs.tex}
		%
\item 
Out of 100 students, two sections of 40 and 60 are formed. If you and your friend are among the 100 students, what is the probability that
\begin{enumerate}
\item you both enter the same section?
\item you both enter the different sections?
\end{enumerate}
\solution
		%\input{ncert/11/16/4/5/defs.tex}
	\item 
The number lock of a suitcase has 4 wheels each labelled with ten digits i.e. from 0 to 9.The lock opens with a sequence of four digits with no repeats.What is the probability of a person getting the right sequence to open the suitcase.
\\
\solution
		%\input{ncert/11/16/4/10/defs.tex}
		%
\item 
Two cards are drawn at random and without replacement from a pack of 52 playing cards. Find the probability that both the cards are black.
\\
\solution
		%\input{ncert/12/13/2/2/defs.tex}
		\item A box of oranges is inspected by examining three randomly selected oranges drawn without replacement. If all the three oranges are good, the box is approved for sale, otherwise, it is rejected. Find the probability that a box containing 15 oranges out of which 12 are good and 3 are bad ones will be approved for sale.
		\label{ncert/12/13/2/3/defs.tex}
		\item Two balls are drawn at random with replacement from a box containing 10 black and 8 red balls. Find the probability that
		\label{ncert/12/13/2/12}
\begin{enumerate}
\item both balls are red.
\item first ball is black and second is red.
\item one of them is black and other is red.
\end{enumerate}

\item In a hostel, 60\% of the students read Hindi newspaper, 40\% read English newspaper and 20\% read both Hindi and English newspapers. A student is selected at random.
		\label{ncert/12/13/2/15}
\begin{enumerate}
\item Find the probability that she reads neither Hindi nor English newspapers.
\item If she reads Hindi newspaper, find the probability that she reads English newspaper.
\item If she reads English newspaper, find the probability that she reads Hindi newspaper.\\
\end{enumerate}
\item The probability of obtaining an even prime number on each die, when a pair of dice is rolled is 
\begin{enumerate}
    \item $0$ 
    
    \item $\frac{1}{3}$ 
    
    \item $\frac{1}{12}$ 
    
    \item $\frac{1}{36}$ 
\end{enumerate}
\solution
		%\input{ncert/12/13/2/17/defs.tex}
	\item A bag contains 4 red and 4 black balls, another bag contains 2 red and 6 black balls. One of the two bags is selected at random and a ball is drawn from the bag which is found to be red. Find the probability that the ball is drawn from the first bag.
\\
\solution
		%\input{ncert/12/13/3/2/main.tex}
  \item
  Cards with numbers 2 to 101 are placed in a box. A card is selected at random.Find the probability that the card has
\begin{enumerate}[label=(\roman*)]
	\item an even number 
	\item a square number
\end{enumerate}
\solution
%\input{exemplar/10/13/3/32/main.tex}
\item
The king, queen and jack of clubs are removed from a deck of 52 playing cards and then well shuffled. Now one card is drawn at random from the remaining cards.  Determine the probability that the card is
\begin{enumerate}[label=(\roman*)]
\item a club
\item 10 of hearts
\end{enumerate}
\solution
%\input{exemplar/10/13/3/29/main.tex}
\item A team of medical students doing their internship have to assist during surgeries
at a city hospital. The probabilities of surgeries rated as very complex, complex,
routine, simple or very simple are respectively, 0.15, 0.20, 0.31, 0.26, .08. Find
the probabilities that a particular surgery will be rated
\begin{enumerate}
	\item complex or very complex;
	\item neither very complex nor very simple;
	\item routine or complex
	\item routine or simple
\end{enumerate}
\solution
%\input{exemplar/11/16/3/8(1)/main.tex}
\item A card is selected from a pack of 52 cards.
\begin{enumerate}[label=(\alph*)]
    \item How many points are there in the sample space?
    \item Calculate the probability that the card is an ace of spades.
    \item Calculate the probability that the card is (i) an ace and (ii) black card.
\end{enumerate}
\solution
%\input{exemplar/11/16/3/4/main2.tex}
\item The probability that a non leap year selected at random will contain 53 sundays.
\\
\solution
%\input{exemplar/10/13/1/19/main.tex}
\item One of the four persons John, Rita, Aslam or Gurpreet will be promoted next
month. Consequently the sample space consists of four elementary outcomes
S = {John promoted, Rita promoted, Aslam promoted, Gurpreet promoted}
You are told that the chances of John’s promotion is same as that of Gurpreet,
Rita’s chances of promotion are twice as likely as Johns. Aslam’s chances are
four times that of John.
\begin{enumerate}
	\item Determine
	\begin{enumerate}
		\item P (John promoted)
		\item P (Rita promoted)
		\item P (Aslam promoted)
		\item P (Gurpreet promoted)
	\end{enumerate}
	\item If A = {John promoted or Gurpreet promoted}, find P (A).
\end{enumerate}
\solution
%\input{exemplar/11/16/3/10/main.tex}
\item A card is drawn from a deck of 52 cards. Find the probability of getting a king or a heart or a red card.\\
\solution
%\input{exemplar/11/16/3/15/main.tex}
\item The probability that a student will pass his examination is 0.73, the probability of
the student getting a compartment is 0.13, and the probability that the student will
either pass or get compartment is 0.96. State True or False.\\
\solution
%\input{exemplar/11/16/3/31/main.tex}
\item A card is selected from a pack of 52 cards\\
\begin{enumerate}[label=(\alph*)]
\item How many points are there in the sample space?
\item Calculate the probability that the cards is an ace of spades.
\item Calculate the probability that the card is (i) an ace (ii)black card.\\
\end{enumerate}
%\input{ncert/11/16/3/4_1/Prob_4.tex}
\item In a non-leap year, the probability of having 53 tuesdays or 53 wednesdays is\\
\solution
%\input{exemplar/11/16/3/18/main.tex}
\item There are 1000 sealed envelopes in a box, 10 of them contain a cash prize of
Rs 100 each, 100 of them contain a cash prize of Rs 50 each and 200 of them
contain a cash prize of Rs 10 each and rest do not contain any cash prize. If they
are well shuffled and an envelope is picked up out, what is the probability that it
contains no cash prize?\\
\solution
%\input{exemplar/10/13/3/34/main.tex}
\item 
A die is thrown and a card is selected at random from a deck of 52 playing cards. The probability of getting an even number on the die and a spade card.\\
\solution
%\input{exemplar/12/13/3/78/main.tex}
\item
If 4-digit numbers greater than 5,000 are randomly formed from the digits 0, 1, 3, 5, and 7, what is the probability of forming a number divisible by 5 when:
\begin{enumerate}
    \item The digits are repeated?
    \item The repetition of digits is not allowed?
\end{enumerate}
\solution
%\input{ncert/11/16/4/9/main.tex}
\item Consider the probability space $\brak{\Omega, \mathcal{G}, P}$ where $\Omega = [0,2]$ and $\mathcal{G} = \cbrak{\phi, \Omega, [0,1], (1,2]}$. Let $X$ and $Y$ be two functions on $\Omega$ defined as
\begin{align*}
    X(\omega) = 
    \begin{cases}
        1 & \text{if }\omega \in [0, 1]\\
        2 & \text{if }\omega \in (1, 2]
    \end{cases}
\end{align*}
and
\begin{align*}
    Y(\omega) = 
    \begin{cases}
        2 & \text{if }\omega \in [0, 1.5]\\
        3 & \text{if }\omega \in (1.5, 2].
    \end{cases}
\end{align*}
Then which one of the following statements is true?
\begin{enumerate}
    \item [(A)] $X$ is a random variable with respect to $\mathcal{G}$, but $Y$ is not a random variable with respect to $\mathcal{G}$.
    \item [(B)] $Y$ is a random variable with respect to $\mathcal{G}$, but $X$ is not a random variable with respect to $\mathcal{G}$.
    \item [(C)] Neither $X$ nor $Y$ is a random variable with respect to $\mathcal{G}$.
    \item [(D)] Both $X$ and $Y$ are random variables with respect to $\mathcal{G}$.
\end{enumerate} \hfill (GATE ST 2023)\\
\solution
%\input{gate/ST/2023/14/main.tex}
	\item  A die is loaded in such a way that each odd number is twice as likely to occur as
each even number. Find $P(G)$, where $G$ is the event that a number greater than
3 occurs on a single roll of the die.
\\
\solution
		%\input{exemplar/11/16/3/5/main.tex}
	\item All the jacks, queens and kings are removed from a deck of 52 playing cards. The remaining cards are well shuffled and then one card is drawn at random. Giving ace a value 1 similar value for other cards, find the probability that the card has a value 
		\begin{enumerate}
			\item 7
			\item greater than 7
			\item less than 7
		\end{enumerate}
		%\input{exemplar/10/13/3/30/main.tex}
  \item A Lot consists of 48 mobile phones of which 42 are good, 3 have only minor defects and 3 have major defects.Varnika will buy a phone if it is good but the trader will only buy a mobile if it has no major defects. One phone is selected at random from the lot. What is the probability that it is
\begin{enumerate}
	\item acceptable to Varnika?
            \item acceptable to the trader?
\end{enumerate}
\solution
	%\input{exemplar/10/13/3/40/main.tex}
 \item A student says that if you throw a die, it will show up 1 or not 1. Therefore, the probability of getting 1 and the probability of getting 'not 1' each is equal to $\frac{1}{2}$. Is this correct? Give reasons.\\
 \solution
        %\input{exemplar/10/13/2/9/main.tex}
   \item Four candidates A, B, C, D have ap-
plied for the assignment to coach a school cricket
team. If A is twice as likely to be selected as B, and
B and C are given about the same chance of being
selected, while C is twice as likely to be selected
as D, what are the probabilities that
\begin{enumerate}
\item C will be selected?
\item A will not be selected?
\end{enumerate}
	%\input{exemplar/11/16/3/9/main.tex}
 \item A bag contain 24 balls of which $x$ balls are red, $2x$ are white and $3x$ are blue. A ball is selected at random, What is the probability that it is
\begin{enumerate}[label=\alph*)]
\item not red ?
\item white ?
\end{enumerate}
%\input{exemplar/10/13/3/41/main.tex}
If the letters of the word ASSASSINATION are arranged at random. Find the Probability that
\begin{enumerate}[label=(\alph*)]
\item Four $S's$ come consecutively in the word
\item Two  $I's$ and two $N's$ come together
\item All $A's$ are not coming together
\item No two $A's$ are coming together
\end{enumerate}
%\input{exemplar/11/16/3/14/main.tex}
	\item One urn contains two black balls (labelled B1 and B2) and one white ball. A
	second urn contains one black ball and two white balls (labelled W1 and W2).
	Suppose the following experiment is performed. One of the two urns is chosen
	at random. Next a ball is randomly chosen from the urn. Then a second ball is
	chosen at random from the same urn without replacing the first ball.
	
	\begin{enumerate}
	\item What is the probability that two black balls are chosen?
	
	\item What is the probability that two balls of opposite colour are chosen?
	\end{enumerate}
	\solution
	%\input{exemplar/11/16/3/12/main1.tex}
\end{enumerate}

		%
\item 
Two cards are drawn at random and without replacement from a pack of 52 playing cards. Find the probability that both the cards are black.
\\
\solution
		%\begin{enumerate}[label=\thesection.\arabic*,ref=\thesection.\theenumi]
	\item One card is drawn from a well-shuffled deck of 52 cards. Find the probability of getting
\begin{enumerate}
\item A king of red colour 
\item A face card 
\item A red face card
\item The jack of hearts
\item A spade
\item The queen of diamonds

\end{enumerate}
\solution
		%\input{ncert/10/15/1/14/main.tex}
	\item Five cards—the ten, jack, queen, king and ace of diamonds, are well-shuffled with their face downwards. One card is then picked up at random.
\begin{enumerate}
\item
What is the probability that the card is the queen? 
\item
If the queen is drawn and put aside, what is the probability that the second card picked up is (a) an ace? (b) a queen?\\
\end{enumerate}
\solution
		%\input{ncert/10/15/1/15/defs.tex}
	\item A bag contains $5$ red balls and some blue balls. If the probability of drawing a blue ball is double that if a red ball, determine the number of blue balls in the bag. 
		\\
\solution
		%\input{ncert/10/15/2/3/defs.tex}
	\item A card is selected from a pack of 52 cards.
 \begin{enumerate}[label=(\alph*)] 
                 \item How many points are there in the sample space?
                 \item Calculate the probability that the card is an ace of spades.
                 \item Calculate the probability that the card is (i) an ace and (ii) black card.
 \end{enumerate}
\solution
		%\input{ncert/11/16/3/4/main.tex}
\item Four cards are drawn from a well-shuffled deck of 52 cards. What is the probability of obtaining 3 diamonds and one spade.
\\
\solution
		%\input{ncert/11/16/4/2/defs.tex}
\item In a certain lottery 10,000 tickets are sold and ten equal prizes are awarded. What is the probability of not getting a prize if you buy (a) one ticket (b) two tickets (c) 10 tickets ?	
\\
\solution
		%\input{ncert/11/16/4/4/defs.tex}
		%
\item 
Out of 100 students, two sections of 40 and 60 are formed. If you and your friend are among the 100 students, what is the probability that
\begin{enumerate}
\item you both enter the same section?
\item you both enter the different sections?
\end{enumerate}
\solution
		%\input{ncert/11/16/4/5/defs.tex}
	\item 
The number lock of a suitcase has 4 wheels each labelled with ten digits i.e. from 0 to 9.The lock opens with a sequence of four digits with no repeats.What is the probability of a person getting the right sequence to open the suitcase.
\\
\solution
		%\input{ncert/11/16/4/10/defs.tex}
		%
\item 
Two cards are drawn at random and without replacement from a pack of 52 playing cards. Find the probability that both the cards are black.
\\
\solution
		%\input{ncert/12/13/2/2/defs.tex}
		\item A box of oranges is inspected by examining three randomly selected oranges drawn without replacement. If all the three oranges are good, the box is approved for sale, otherwise, it is rejected. Find the probability that a box containing 15 oranges out of which 12 are good and 3 are bad ones will be approved for sale.
		\label{ncert/12/13/2/3/defs.tex}
		\item Two balls are drawn at random with replacement from a box containing 10 black and 8 red balls. Find the probability that
		\label{ncert/12/13/2/12}
\begin{enumerate}
\item both balls are red.
\item first ball is black and second is red.
\item one of them is black and other is red.
\end{enumerate}

\item In a hostel, 60\% of the students read Hindi newspaper, 40\% read English newspaper and 20\% read both Hindi and English newspapers. A student is selected at random.
		\label{ncert/12/13/2/15}
\begin{enumerate}
\item Find the probability that she reads neither Hindi nor English newspapers.
\item If she reads Hindi newspaper, find the probability that she reads English newspaper.
\item If she reads English newspaper, find the probability that she reads Hindi newspaper.\\
\end{enumerate}
\item The probability of obtaining an even prime number on each die, when a pair of dice is rolled is 
\begin{enumerate}
    \item $0$ 
    
    \item $\frac{1}{3}$ 
    
    \item $\frac{1}{12}$ 
    
    \item $\frac{1}{36}$ 
\end{enumerate}
\solution
		%\input{ncert/12/13/2/17/defs.tex}
	\item A bag contains 4 red and 4 black balls, another bag contains 2 red and 6 black balls. One of the two bags is selected at random and a ball is drawn from the bag which is found to be red. Find the probability that the ball is drawn from the first bag.
\\
\solution
		%\input{ncert/12/13/3/2/main.tex}
  \item
  Cards with numbers 2 to 101 are placed in a box. A card is selected at random.Find the probability that the card has
\begin{enumerate}[label=(\roman*)]
	\item an even number 
	\item a square number
\end{enumerate}
\solution
%\input{exemplar/10/13/3/32/main.tex}
\item
The king, queen and jack of clubs are removed from a deck of 52 playing cards and then well shuffled. Now one card is drawn at random from the remaining cards.  Determine the probability that the card is
\begin{enumerate}[label=(\roman*)]
\item a club
\item 10 of hearts
\end{enumerate}
\solution
%\input{exemplar/10/13/3/29/main.tex}
\item A team of medical students doing their internship have to assist during surgeries
at a city hospital. The probabilities of surgeries rated as very complex, complex,
routine, simple or very simple are respectively, 0.15, 0.20, 0.31, 0.26, .08. Find
the probabilities that a particular surgery will be rated
\begin{enumerate}
	\item complex or very complex;
	\item neither very complex nor very simple;
	\item routine or complex
	\item routine or simple
\end{enumerate}
\solution
%\input{exemplar/11/16/3/8(1)/main.tex}
\item A card is selected from a pack of 52 cards.
\begin{enumerate}[label=(\alph*)]
    \item How many points are there in the sample space?
    \item Calculate the probability that the card is an ace of spades.
    \item Calculate the probability that the card is (i) an ace and (ii) black card.
\end{enumerate}
\solution
%\input{exemplar/11/16/3/4/main2.tex}
\item The probability that a non leap year selected at random will contain 53 sundays.
\\
\solution
%\input{exemplar/10/13/1/19/main.tex}
\item One of the four persons John, Rita, Aslam or Gurpreet will be promoted next
month. Consequently the sample space consists of four elementary outcomes
S = {John promoted, Rita promoted, Aslam promoted, Gurpreet promoted}
You are told that the chances of John’s promotion is same as that of Gurpreet,
Rita’s chances of promotion are twice as likely as Johns. Aslam’s chances are
four times that of John.
\begin{enumerate}
	\item Determine
	\begin{enumerate}
		\item P (John promoted)
		\item P (Rita promoted)
		\item P (Aslam promoted)
		\item P (Gurpreet promoted)
	\end{enumerate}
	\item If A = {John promoted or Gurpreet promoted}, find P (A).
\end{enumerate}
\solution
%\input{exemplar/11/16/3/10/main.tex}
\item A card is drawn from a deck of 52 cards. Find the probability of getting a king or a heart or a red card.\\
\solution
%\input{exemplar/11/16/3/15/main.tex}
\item The probability that a student will pass his examination is 0.73, the probability of
the student getting a compartment is 0.13, and the probability that the student will
either pass or get compartment is 0.96. State True or False.\\
\solution
%\input{exemplar/11/16/3/31/main.tex}
\item A card is selected from a pack of 52 cards\\
\begin{enumerate}[label=(\alph*)]
\item How many points are there in the sample space?
\item Calculate the probability that the cards is an ace of spades.
\item Calculate the probability that the card is (i) an ace (ii)black card.\\
\end{enumerate}
%\input{ncert/11/16/3/4_1/Prob_4.tex}
\item In a non-leap year, the probability of having 53 tuesdays or 53 wednesdays is\\
\solution
%\input{exemplar/11/16/3/18/main.tex}
\item There are 1000 sealed envelopes in a box, 10 of them contain a cash prize of
Rs 100 each, 100 of them contain a cash prize of Rs 50 each and 200 of them
contain a cash prize of Rs 10 each and rest do not contain any cash prize. If they
are well shuffled and an envelope is picked up out, what is the probability that it
contains no cash prize?\\
\solution
%\input{exemplar/10/13/3/34/main.tex}
\item 
A die is thrown and a card is selected at random from a deck of 52 playing cards. The probability of getting an even number on the die and a spade card.\\
\solution
%\input{exemplar/12/13/3/78/main.tex}
\item
If 4-digit numbers greater than 5,000 are randomly formed from the digits 0, 1, 3, 5, and 7, what is the probability of forming a number divisible by 5 when:
\begin{enumerate}
    \item The digits are repeated?
    \item The repetition of digits is not allowed?
\end{enumerate}
\solution
%\input{ncert/11/16/4/9/main.tex}
\item Consider the probability space $\brak{\Omega, \mathcal{G}, P}$ where $\Omega = [0,2]$ and $\mathcal{G} = \cbrak{\phi, \Omega, [0,1], (1,2]}$. Let $X$ and $Y$ be two functions on $\Omega$ defined as
\begin{align*}
    X(\omega) = 
    \begin{cases}
        1 & \text{if }\omega \in [0, 1]\\
        2 & \text{if }\omega \in (1, 2]
    \end{cases}
\end{align*}
and
\begin{align*}
    Y(\omega) = 
    \begin{cases}
        2 & \text{if }\omega \in [0, 1.5]\\
        3 & \text{if }\omega \in (1.5, 2].
    \end{cases}
\end{align*}
Then which one of the following statements is true?
\begin{enumerate}
    \item [(A)] $X$ is a random variable with respect to $\mathcal{G}$, but $Y$ is not a random variable with respect to $\mathcal{G}$.
    \item [(B)] $Y$ is a random variable with respect to $\mathcal{G}$, but $X$ is not a random variable with respect to $\mathcal{G}$.
    \item [(C)] Neither $X$ nor $Y$ is a random variable with respect to $\mathcal{G}$.
    \item [(D)] Both $X$ and $Y$ are random variables with respect to $\mathcal{G}$.
\end{enumerate} \hfill (GATE ST 2023)\\
\solution
%\input{gate/ST/2023/14/main.tex}
	\item  A die is loaded in such a way that each odd number is twice as likely to occur as
each even number. Find $P(G)$, where $G$ is the event that a number greater than
3 occurs on a single roll of the die.
\\
\solution
		%\input{exemplar/11/16/3/5/main.tex}
	\item All the jacks, queens and kings are removed from a deck of 52 playing cards. The remaining cards are well shuffled and then one card is drawn at random. Giving ace a value 1 similar value for other cards, find the probability that the card has a value 
		\begin{enumerate}
			\item 7
			\item greater than 7
			\item less than 7
		\end{enumerate}
		%\input{exemplar/10/13/3/30/main.tex}
  \item A Lot consists of 48 mobile phones of which 42 are good, 3 have only minor defects and 3 have major defects.Varnika will buy a phone if it is good but the trader will only buy a mobile if it has no major defects. One phone is selected at random from the lot. What is the probability that it is
\begin{enumerate}
	\item acceptable to Varnika?
            \item acceptable to the trader?
\end{enumerate}
\solution
	%\input{exemplar/10/13/3/40/main.tex}
 \item A student says that if you throw a die, it will show up 1 or not 1. Therefore, the probability of getting 1 and the probability of getting 'not 1' each is equal to $\frac{1}{2}$. Is this correct? Give reasons.\\
 \solution
        %\input{exemplar/10/13/2/9/main.tex}
   \item Four candidates A, B, C, D have ap-
plied for the assignment to coach a school cricket
team. If A is twice as likely to be selected as B, and
B and C are given about the same chance of being
selected, while C is twice as likely to be selected
as D, what are the probabilities that
\begin{enumerate}
\item C will be selected?
\item A will not be selected?
\end{enumerate}
	%\input{exemplar/11/16/3/9/main.tex}
 \item A bag contain 24 balls of which $x$ balls are red, $2x$ are white and $3x$ are blue. A ball is selected at random, What is the probability that it is
\begin{enumerate}[label=\alph*)]
\item not red ?
\item white ?
\end{enumerate}
%\input{exemplar/10/13/3/41/main.tex}
If the letters of the word ASSASSINATION are arranged at random. Find the Probability that
\begin{enumerate}[label=(\alph*)]
\item Four $S's$ come consecutively in the word
\item Two  $I's$ and two $N's$ come together
\item All $A's$ are not coming together
\item No two $A's$ are coming together
\end{enumerate}
%\input{exemplar/11/16/3/14/main.tex}
	\item One urn contains two black balls (labelled B1 and B2) and one white ball. A
	second urn contains one black ball and two white balls (labelled W1 and W2).
	Suppose the following experiment is performed. One of the two urns is chosen
	at random. Next a ball is randomly chosen from the urn. Then a second ball is
	chosen at random from the same urn without replacing the first ball.
	
	\begin{enumerate}
	\item What is the probability that two black balls are chosen?
	
	\item What is the probability that two balls of opposite colour are chosen?
	\end{enumerate}
	\solution
	%\input{exemplar/11/16/3/12/main1.tex}
\end{enumerate}

		\item A box of oranges is inspected by examining three randomly selected oranges drawn without replacement. If all the three oranges are good, the box is approved for sale, otherwise, it is rejected. Find the probability that a box containing 15 oranges out of which 12 are good and 3 are bad ones will be approved for sale.
		\label{ncert/12/13/2/3/defs.tex}
		\item Two balls are drawn at random with replacement from a box containing 10 black and 8 red balls. Find the probability that
		\label{ncert/12/13/2/12}
\begin{enumerate}
\item both balls are red.
\item first ball is black and second is red.
\item one of them is black and other is red.
\end{enumerate}

\item In a hostel, 60\% of the students read Hindi newspaper, 40\% read English newspaper and 20\% read both Hindi and English newspapers. A student is selected at random.
		\label{ncert/12/13/2/15}
\begin{enumerate}
\item Find the probability that she reads neither Hindi nor English newspapers.
\item If she reads Hindi newspaper, find the probability that she reads English newspaper.
\item If she reads English newspaper, find the probability that she reads Hindi newspaper.\\
\end{enumerate}
\item The probability of obtaining an even prime number on each die, when a pair of dice is rolled is 
\begin{enumerate}
    \item $0$ 
    
    \item $\frac{1}{3}$ 
    
    \item $\frac{1}{12}$ 
    
    \item $\frac{1}{36}$ 
\end{enumerate}
\solution
		%\begin{enumerate}[label=\thesection.\arabic*,ref=\thesection.\theenumi]
	\item One card is drawn from a well-shuffled deck of 52 cards. Find the probability of getting
\begin{enumerate}
\item A king of red colour 
\item A face card 
\item A red face card
\item The jack of hearts
\item A spade
\item The queen of diamonds

\end{enumerate}
\solution
		%\input{ncert/10/15/1/14/main.tex}
	\item Five cards—the ten, jack, queen, king and ace of diamonds, are well-shuffled with their face downwards. One card is then picked up at random.
\begin{enumerate}
\item
What is the probability that the card is the queen? 
\item
If the queen is drawn and put aside, what is the probability that the second card picked up is (a) an ace? (b) a queen?\\
\end{enumerate}
\solution
		%\input{ncert/10/15/1/15/defs.tex}
	\item A bag contains $5$ red balls and some blue balls. If the probability of drawing a blue ball is double that if a red ball, determine the number of blue balls in the bag. 
		\\
\solution
		%\input{ncert/10/15/2/3/defs.tex}
	\item A card is selected from a pack of 52 cards.
 \begin{enumerate}[label=(\alph*)] 
                 \item How many points are there in the sample space?
                 \item Calculate the probability that the card is an ace of spades.
                 \item Calculate the probability that the card is (i) an ace and (ii) black card.
 \end{enumerate}
\solution
		%\input{ncert/11/16/3/4/main.tex}
\item Four cards are drawn from a well-shuffled deck of 52 cards. What is the probability of obtaining 3 diamonds and one spade.
\\
\solution
		%\input{ncert/11/16/4/2/defs.tex}
\item In a certain lottery 10,000 tickets are sold and ten equal prizes are awarded. What is the probability of not getting a prize if you buy (a) one ticket (b) two tickets (c) 10 tickets ?	
\\
\solution
		%\input{ncert/11/16/4/4/defs.tex}
		%
\item 
Out of 100 students, two sections of 40 and 60 are formed. If you and your friend are among the 100 students, what is the probability that
\begin{enumerate}
\item you both enter the same section?
\item you both enter the different sections?
\end{enumerate}
\solution
		%\input{ncert/11/16/4/5/defs.tex}
	\item 
The number lock of a suitcase has 4 wheels each labelled with ten digits i.e. from 0 to 9.The lock opens with a sequence of four digits with no repeats.What is the probability of a person getting the right sequence to open the suitcase.
\\
\solution
		%\input{ncert/11/16/4/10/defs.tex}
		%
\item 
Two cards are drawn at random and without replacement from a pack of 52 playing cards. Find the probability that both the cards are black.
\\
\solution
		%\input{ncert/12/13/2/2/defs.tex}
		\item A box of oranges is inspected by examining three randomly selected oranges drawn without replacement. If all the three oranges are good, the box is approved for sale, otherwise, it is rejected. Find the probability that a box containing 15 oranges out of which 12 are good and 3 are bad ones will be approved for sale.
		\label{ncert/12/13/2/3/defs.tex}
		\item Two balls are drawn at random with replacement from a box containing 10 black and 8 red balls. Find the probability that
		\label{ncert/12/13/2/12}
\begin{enumerate}
\item both balls are red.
\item first ball is black and second is red.
\item one of them is black and other is red.
\end{enumerate}

\item In a hostel, 60\% of the students read Hindi newspaper, 40\% read English newspaper and 20\% read both Hindi and English newspapers. A student is selected at random.
		\label{ncert/12/13/2/15}
\begin{enumerate}
\item Find the probability that she reads neither Hindi nor English newspapers.
\item If she reads Hindi newspaper, find the probability that she reads English newspaper.
\item If she reads English newspaper, find the probability that she reads Hindi newspaper.\\
\end{enumerate}
\item The probability of obtaining an even prime number on each die, when a pair of dice is rolled is 
\begin{enumerate}
    \item $0$ 
    
    \item $\frac{1}{3}$ 
    
    \item $\frac{1}{12}$ 
    
    \item $\frac{1}{36}$ 
\end{enumerate}
\solution
		%\input{ncert/12/13/2/17/defs.tex}
	\item A bag contains 4 red and 4 black balls, another bag contains 2 red and 6 black balls. One of the two bags is selected at random and a ball is drawn from the bag which is found to be red. Find the probability that the ball is drawn from the first bag.
\\
\solution
		%\input{ncert/12/13/3/2/main.tex}
  \item
  Cards with numbers 2 to 101 are placed in a box. A card is selected at random.Find the probability that the card has
\begin{enumerate}[label=(\roman*)]
	\item an even number 
	\item a square number
\end{enumerate}
\solution
%\input{exemplar/10/13/3/32/main.tex}
\item
The king, queen and jack of clubs are removed from a deck of 52 playing cards and then well shuffled. Now one card is drawn at random from the remaining cards.  Determine the probability that the card is
\begin{enumerate}[label=(\roman*)]
\item a club
\item 10 of hearts
\end{enumerate}
\solution
%\input{exemplar/10/13/3/29/main.tex}
\item A team of medical students doing their internship have to assist during surgeries
at a city hospital. The probabilities of surgeries rated as very complex, complex,
routine, simple or very simple are respectively, 0.15, 0.20, 0.31, 0.26, .08. Find
the probabilities that a particular surgery will be rated
\begin{enumerate}
	\item complex or very complex;
	\item neither very complex nor very simple;
	\item routine or complex
	\item routine or simple
\end{enumerate}
\solution
%\input{exemplar/11/16/3/8(1)/main.tex}
\item A card is selected from a pack of 52 cards.
\begin{enumerate}[label=(\alph*)]
    \item How many points are there in the sample space?
    \item Calculate the probability that the card is an ace of spades.
    \item Calculate the probability that the card is (i) an ace and (ii) black card.
\end{enumerate}
\solution
%\input{exemplar/11/16/3/4/main2.tex}
\item The probability that a non leap year selected at random will contain 53 sundays.
\\
\solution
%\input{exemplar/10/13/1/19/main.tex}
\item One of the four persons John, Rita, Aslam or Gurpreet will be promoted next
month. Consequently the sample space consists of four elementary outcomes
S = {John promoted, Rita promoted, Aslam promoted, Gurpreet promoted}
You are told that the chances of John’s promotion is same as that of Gurpreet,
Rita’s chances of promotion are twice as likely as Johns. Aslam’s chances are
four times that of John.
\begin{enumerate}
	\item Determine
	\begin{enumerate}
		\item P (John promoted)
		\item P (Rita promoted)
		\item P (Aslam promoted)
		\item P (Gurpreet promoted)
	\end{enumerate}
	\item If A = {John promoted or Gurpreet promoted}, find P (A).
\end{enumerate}
\solution
%\input{exemplar/11/16/3/10/main.tex}
\item A card is drawn from a deck of 52 cards. Find the probability of getting a king or a heart or a red card.\\
\solution
%\input{exemplar/11/16/3/15/main.tex}
\item The probability that a student will pass his examination is 0.73, the probability of
the student getting a compartment is 0.13, and the probability that the student will
either pass or get compartment is 0.96. State True or False.\\
\solution
%\input{exemplar/11/16/3/31/main.tex}
\item A card is selected from a pack of 52 cards\\
\begin{enumerate}[label=(\alph*)]
\item How many points are there in the sample space?
\item Calculate the probability that the cards is an ace of spades.
\item Calculate the probability that the card is (i) an ace (ii)black card.\\
\end{enumerate}
%\input{ncert/11/16/3/4_1/Prob_4.tex}
\item In a non-leap year, the probability of having 53 tuesdays or 53 wednesdays is\\
\solution
%\input{exemplar/11/16/3/18/main.tex}
\item There are 1000 sealed envelopes in a box, 10 of them contain a cash prize of
Rs 100 each, 100 of them contain a cash prize of Rs 50 each and 200 of them
contain a cash prize of Rs 10 each and rest do not contain any cash prize. If they
are well shuffled and an envelope is picked up out, what is the probability that it
contains no cash prize?\\
\solution
%\input{exemplar/10/13/3/34/main.tex}
\item 
A die is thrown and a card is selected at random from a deck of 52 playing cards. The probability of getting an even number on the die and a spade card.\\
\solution
%\input{exemplar/12/13/3/78/main.tex}
\item
If 4-digit numbers greater than 5,000 are randomly formed from the digits 0, 1, 3, 5, and 7, what is the probability of forming a number divisible by 5 when:
\begin{enumerate}
    \item The digits are repeated?
    \item The repetition of digits is not allowed?
\end{enumerate}
\solution
%\input{ncert/11/16/4/9/main.tex}
\item Consider the probability space $\brak{\Omega, \mathcal{G}, P}$ where $\Omega = [0,2]$ and $\mathcal{G} = \cbrak{\phi, \Omega, [0,1], (1,2]}$. Let $X$ and $Y$ be two functions on $\Omega$ defined as
\begin{align*}
    X(\omega) = 
    \begin{cases}
        1 & \text{if }\omega \in [0, 1]\\
        2 & \text{if }\omega \in (1, 2]
    \end{cases}
\end{align*}
and
\begin{align*}
    Y(\omega) = 
    \begin{cases}
        2 & \text{if }\omega \in [0, 1.5]\\
        3 & \text{if }\omega \in (1.5, 2].
    \end{cases}
\end{align*}
Then which one of the following statements is true?
\begin{enumerate}
    \item [(A)] $X$ is a random variable with respect to $\mathcal{G}$, but $Y$ is not a random variable with respect to $\mathcal{G}$.
    \item [(B)] $Y$ is a random variable with respect to $\mathcal{G}$, but $X$ is not a random variable with respect to $\mathcal{G}$.
    \item [(C)] Neither $X$ nor $Y$ is a random variable with respect to $\mathcal{G}$.
    \item [(D)] Both $X$ and $Y$ are random variables with respect to $\mathcal{G}$.
\end{enumerate} \hfill (GATE ST 2023)\\
\solution
%\input{gate/ST/2023/14/main.tex}
	\item  A die is loaded in such a way that each odd number is twice as likely to occur as
each even number. Find $P(G)$, where $G$ is the event that a number greater than
3 occurs on a single roll of the die.
\\
\solution
		%\input{exemplar/11/16/3/5/main.tex}
	\item All the jacks, queens and kings are removed from a deck of 52 playing cards. The remaining cards are well shuffled and then one card is drawn at random. Giving ace a value 1 similar value for other cards, find the probability that the card has a value 
		\begin{enumerate}
			\item 7
			\item greater than 7
			\item less than 7
		\end{enumerate}
		%\input{exemplar/10/13/3/30/main.tex}
  \item A Lot consists of 48 mobile phones of which 42 are good, 3 have only minor defects and 3 have major defects.Varnika will buy a phone if it is good but the trader will only buy a mobile if it has no major defects. One phone is selected at random from the lot. What is the probability that it is
\begin{enumerate}
	\item acceptable to Varnika?
            \item acceptable to the trader?
\end{enumerate}
\solution
	%\input{exemplar/10/13/3/40/main.tex}
 \item A student says that if you throw a die, it will show up 1 or not 1. Therefore, the probability of getting 1 and the probability of getting 'not 1' each is equal to $\frac{1}{2}$. Is this correct? Give reasons.\\
 \solution
        %\input{exemplar/10/13/2/9/main.tex}
   \item Four candidates A, B, C, D have ap-
plied for the assignment to coach a school cricket
team. If A is twice as likely to be selected as B, and
B and C are given about the same chance of being
selected, while C is twice as likely to be selected
as D, what are the probabilities that
\begin{enumerate}
\item C will be selected?
\item A will not be selected?
\end{enumerate}
	%\input{exemplar/11/16/3/9/main.tex}
 \item A bag contain 24 balls of which $x$ balls are red, $2x$ are white and $3x$ are blue. A ball is selected at random, What is the probability that it is
\begin{enumerate}[label=\alph*)]
\item not red ?
\item white ?
\end{enumerate}
%\input{exemplar/10/13/3/41/main.tex}
If the letters of the word ASSASSINATION are arranged at random. Find the Probability that
\begin{enumerate}[label=(\alph*)]
\item Four $S's$ come consecutively in the word
\item Two  $I's$ and two $N's$ come together
\item All $A's$ are not coming together
\item No two $A's$ are coming together
\end{enumerate}
%\input{exemplar/11/16/3/14/main.tex}
	\item One urn contains two black balls (labelled B1 and B2) and one white ball. A
	second urn contains one black ball and two white balls (labelled W1 and W2).
	Suppose the following experiment is performed. One of the two urns is chosen
	at random. Next a ball is randomly chosen from the urn. Then a second ball is
	chosen at random from the same urn without replacing the first ball.
	
	\begin{enumerate}
	\item What is the probability that two black balls are chosen?
	
	\item What is the probability that two balls of opposite colour are chosen?
	\end{enumerate}
	\solution
	%\input{exemplar/11/16/3/12/main1.tex}
\end{enumerate}

	\item A bag contains 4 red and 4 black balls, another bag contains 2 red and 6 black balls. One of the two bags is selected at random and a ball is drawn from the bag which is found to be red. Find the probability that the ball is drawn from the first bag.
\\
\solution
		%\begin{table}[H]
	\centering
\begin{tabular}{|c|c|c|}
\hline
Random variable &Value &Definition\\ \hline
\multirow{3}{*}{X} &0 &Slips of Rs 1\\
&1 &Slips of Rs 5\\
&2 &Slips of Rs 13\\ \hline
\multirow{2}{*}{Y} &0 &Box A\\
&1 &Box B\\\hline
\end{tabular}
\caption{}
\label{tab:Distribution}
\end{table}
See \tabref{tab:Distribution}.
\begin{align}
p_{Y}\brak{k}= \begin{cases} 
      \frac{1}{3} & {k=0} \\
      \frac{2}{3 }& {k=1} 
   \end{cases}
   \\
p_{Y|X}\brak{0|0} = \frac{19}{25}\, 
p_{Y|X}\brak{0|1} = \frac{6}{25}\,
p_{Y|X}\brak{1|0} = \frac{45}{50}\,
p_{Y|X}\brak{1|2} = \frac{5}{50}
\end{align}
The desired probability is the probability that a slip drawn at random is marked other than Rs 1,
\begin{align}
&=1-p_X\brak{0}\\
&= p_X(1) + p_X(2)
\end{align}
Using Bayes theorem,
\begin{align}
&= p_Y\brak{0} \times \pr{Y=0 | X=1} + p_Y\brak{1} \times \pr{Y=1|X=2}\\
&=\frac{1}{3} \times \frac{6}{25} + \frac{2}{3} \times \frac{5}{50}\\
&=\frac{11}{75}
\end{align}

\newpage

%\tableofcontents

\bigskip

\renewcommand{\thefigure}{\theenumi}
\renewcommand{\thetable}{\theenumi}
%\renewcommand{\theequation}{\theenumi}

%\begin{abstract}
%%\boldmath
%In this letter, an algorithm for evaluating the exact analytical bit error rate  (BER)  for the piecewise linear (PL) combiner for  multiple relays is presented. Previous results were available only for upto three relays. The algorithm is unique in the sense that  the actual mathematical expressions, that are prohibitively large, need not be explicitly obtained. The diversity gain due to multiple relays is shown through plots of the analytical BER, well supported by simulations. 
%
%\end{abstract}
% IEEEtran.cls defaults to using nonbold math in the Abstract.
% This preserves the distinction between vectors and scalars. However,
% if the journal you are submitting to favors bold math in the abstract,
% then you can use LaTeX's standard command \boldmath at the very start
% of the abstract to achieve this. Many IEEE journals frown on math
% in the abstract anyway.

% Note that keywords are not normally used for peerreview papers.
%\begin{IEEEkeywords}
%Cooperative diversity, decode and forward, piecewise linear
%\end{IEEEkeywords}



% For peer review papers, you can put extra information on the cover
% page as needed:
% \ifCLASSOPTIONpeerreview
% \begin{center} \bfseries EDICS Category: 3-BBND \end{center}
% \fi
%
% For peerreview papers, this IEEEtran command inserts a page break and
% creates the second title. It will be ignored for other modes.
%\IEEEpeerreviewmaketitle




  \item
  Cards with numbers 2 to 101 are placed in a box. A card is selected at random.Find the probability that the card has
\begin{enumerate}[label=(\roman*)]
	\item an even number 
	\item a square number
\end{enumerate}
\solution
%\begin{table}[H]
	\centering
\begin{tabular}{|c|c|c|}
\hline
Random variable &Value &Definition\\ \hline
\multirow{3}{*}{X} &0 &Slips of Rs 1\\
&1 &Slips of Rs 5\\
&2 &Slips of Rs 13\\ \hline
\multirow{2}{*}{Y} &0 &Box A\\
&1 &Box B\\\hline
\end{tabular}
\caption{}
\label{tab:Distribution}
\end{table}
See \tabref{tab:Distribution}.
\begin{align}
p_{Y}\brak{k}= \begin{cases} 
      \frac{1}{3} & {k=0} \\
      \frac{2}{3 }& {k=1} 
   \end{cases}
   \\
p_{Y|X}\brak{0|0} = \frac{19}{25}\, 
p_{Y|X}\brak{0|1} = \frac{6}{25}\,
p_{Y|X}\brak{1|0} = \frac{45}{50}\,
p_{Y|X}\brak{1|2} = \frac{5}{50}
\end{align}
The desired probability is the probability that a slip drawn at random is marked other than Rs 1,
\begin{align}
&=1-p_X\brak{0}\\
&= p_X(1) + p_X(2)
\end{align}
Using Bayes theorem,
\begin{align}
&= p_Y\brak{0} \times \pr{Y=0 | X=1} + p_Y\brak{1} \times \pr{Y=1|X=2}\\
&=\frac{1}{3} \times \frac{6}{25} + \frac{2}{3} \times \frac{5}{50}\\
&=\frac{11}{75}
\end{align}

\newpage

%\tableofcontents

\bigskip

\renewcommand{\thefigure}{\theenumi}
\renewcommand{\thetable}{\theenumi}
%\renewcommand{\theequation}{\theenumi}

%\begin{abstract}
%%\boldmath
%In this letter, an algorithm for evaluating the exact analytical bit error rate  (BER)  for the piecewise linear (PL) combiner for  multiple relays is presented. Previous results were available only for upto three relays. The algorithm is unique in the sense that  the actual mathematical expressions, that are prohibitively large, need not be explicitly obtained. The diversity gain due to multiple relays is shown through plots of the analytical BER, well supported by simulations. 
%
%\end{abstract}
% IEEEtran.cls defaults to using nonbold math in the Abstract.
% This preserves the distinction between vectors and scalars. However,
% if the journal you are submitting to favors bold math in the abstract,
% then you can use LaTeX's standard command \boldmath at the very start
% of the abstract to achieve this. Many IEEE journals frown on math
% in the abstract anyway.

% Note that keywords are not normally used for peerreview papers.
%\begin{IEEEkeywords}
%Cooperative diversity, decode and forward, piecewise linear
%\end{IEEEkeywords}



% For peer review papers, you can put extra information on the cover
% page as needed:
% \ifCLASSOPTIONpeerreview
% \begin{center} \bfseries EDICS Category: 3-BBND \end{center}
% \fi
%
% For peerreview papers, this IEEEtran command inserts a page break and
% creates the second title. It will be ignored for other modes.
%\IEEEpeerreviewmaketitle




\item
The king, queen and jack of clubs are removed from a deck of 52 playing cards and then well shuffled. Now one card is drawn at random from the remaining cards.  Determine the probability that the card is
\begin{enumerate}[label=(\roman*)]
\item a club
\item 10 of hearts
\end{enumerate}
\solution
%\begin{table}[H]
	\centering
\begin{tabular}{|c|c|c|}
\hline
Random variable &Value &Definition\\ \hline
\multirow{3}{*}{X} &0 &Slips of Rs 1\\
&1 &Slips of Rs 5\\
&2 &Slips of Rs 13\\ \hline
\multirow{2}{*}{Y} &0 &Box A\\
&1 &Box B\\\hline
\end{tabular}
\caption{}
\label{tab:Distribution}
\end{table}
See \tabref{tab:Distribution}.
\begin{align}
p_{Y}\brak{k}= \begin{cases} 
      \frac{1}{3} & {k=0} \\
      \frac{2}{3 }& {k=1} 
   \end{cases}
   \\
p_{Y|X}\brak{0|0} = \frac{19}{25}\, 
p_{Y|X}\brak{0|1} = \frac{6}{25}\,
p_{Y|X}\brak{1|0} = \frac{45}{50}\,
p_{Y|X}\brak{1|2} = \frac{5}{50}
\end{align}
The desired probability is the probability that a slip drawn at random is marked other than Rs 1,
\begin{align}
&=1-p_X\brak{0}\\
&= p_X(1) + p_X(2)
\end{align}
Using Bayes theorem,
\begin{align}
&= p_Y\brak{0} \times \pr{Y=0 | X=1} + p_Y\brak{1} \times \pr{Y=1|X=2}\\
&=\frac{1}{3} \times \frac{6}{25} + \frac{2}{3} \times \frac{5}{50}\\
&=\frac{11}{75}
\end{align}

\newpage

%\tableofcontents

\bigskip

\renewcommand{\thefigure}{\theenumi}
\renewcommand{\thetable}{\theenumi}
%\renewcommand{\theequation}{\theenumi}

%\begin{abstract}
%%\boldmath
%In this letter, an algorithm for evaluating the exact analytical bit error rate  (BER)  for the piecewise linear (PL) combiner for  multiple relays is presented. Previous results were available only for upto three relays. The algorithm is unique in the sense that  the actual mathematical expressions, that are prohibitively large, need not be explicitly obtained. The diversity gain due to multiple relays is shown through plots of the analytical BER, well supported by simulations. 
%
%\end{abstract}
% IEEEtran.cls defaults to using nonbold math in the Abstract.
% This preserves the distinction between vectors and scalars. However,
% if the journal you are submitting to favors bold math in the abstract,
% then you can use LaTeX's standard command \boldmath at the very start
% of the abstract to achieve this. Many IEEE journals frown on math
% in the abstract anyway.

% Note that keywords are not normally used for peerreview papers.
%\begin{IEEEkeywords}
%Cooperative diversity, decode and forward, piecewise linear
%\end{IEEEkeywords}



% For peer review papers, you can put extra information on the cover
% page as needed:
% \ifCLASSOPTIONpeerreview
% \begin{center} \bfseries EDICS Category: 3-BBND \end{center}
% \fi
%
% For peerreview papers, this IEEEtran command inserts a page break and
% creates the second title. It will be ignored for other modes.
%\IEEEpeerreviewmaketitle




\item A team of medical students doing their internship have to assist during surgeries
at a city hospital. The probabilities of surgeries rated as very complex, complex,
routine, simple or very simple are respectively, 0.15, 0.20, 0.31, 0.26, .08. Find
the probabilities that a particular surgery will be rated
\begin{enumerate}
	\item complex or very complex;
	\item neither very complex nor very simple;
	\item routine or complex
	\item routine or simple
\end{enumerate}
\solution
%\begin{table}[H]
	\centering
\begin{tabular}{|c|c|c|}
\hline
Random variable &Value &Definition\\ \hline
\multirow{3}{*}{X} &0 &Slips of Rs 1\\
&1 &Slips of Rs 5\\
&2 &Slips of Rs 13\\ \hline
\multirow{2}{*}{Y} &0 &Box A\\
&1 &Box B\\\hline
\end{tabular}
\caption{}
\label{tab:Distribution}
\end{table}
See \tabref{tab:Distribution}.
\begin{align}
p_{Y}\brak{k}= \begin{cases} 
      \frac{1}{3} & {k=0} \\
      \frac{2}{3 }& {k=1} 
   \end{cases}
   \\
p_{Y|X}\brak{0|0} = \frac{19}{25}\, 
p_{Y|X}\brak{0|1} = \frac{6}{25}\,
p_{Y|X}\brak{1|0} = \frac{45}{50}\,
p_{Y|X}\brak{1|2} = \frac{5}{50}
\end{align}
The desired probability is the probability that a slip drawn at random is marked other than Rs 1,
\begin{align}
&=1-p_X\brak{0}\\
&= p_X(1) + p_X(2)
\end{align}
Using Bayes theorem,
\begin{align}
&= p_Y\brak{0} \times \pr{Y=0 | X=1} + p_Y\brak{1} \times \pr{Y=1|X=2}\\
&=\frac{1}{3} \times \frac{6}{25} + \frac{2}{3} \times \frac{5}{50}\\
&=\frac{11}{75}
\end{align}

\newpage

%\tableofcontents

\bigskip

\renewcommand{\thefigure}{\theenumi}
\renewcommand{\thetable}{\theenumi}
%\renewcommand{\theequation}{\theenumi}

%\begin{abstract}
%%\boldmath
%In this letter, an algorithm for evaluating the exact analytical bit error rate  (BER)  for the piecewise linear (PL) combiner for  multiple relays is presented. Previous results were available only for upto three relays. The algorithm is unique in the sense that  the actual mathematical expressions, that are prohibitively large, need not be explicitly obtained. The diversity gain due to multiple relays is shown through plots of the analytical BER, well supported by simulations. 
%
%\end{abstract}
% IEEEtran.cls defaults to using nonbold math in the Abstract.
% This preserves the distinction between vectors and scalars. However,
% if the journal you are submitting to favors bold math in the abstract,
% then you can use LaTeX's standard command \boldmath at the very start
% of the abstract to achieve this. Many IEEE journals frown on math
% in the abstract anyway.

% Note that keywords are not normally used for peerreview papers.
%\begin{IEEEkeywords}
%Cooperative diversity, decode and forward, piecewise linear
%\end{IEEEkeywords}



% For peer review papers, you can put extra information on the cover
% page as needed:
% \ifCLASSOPTIONpeerreview
% \begin{center} \bfseries EDICS Category: 3-BBND \end{center}
% \fi
%
% For peerreview papers, this IEEEtran command inserts a page break and
% creates the second title. It will be ignored for other modes.
%\IEEEpeerreviewmaketitle




\item A card is selected from a pack of 52 cards.
\begin{enumerate}[label=(\alph*)]
    \item How many points are there in the sample space?
    \item Calculate the probability that the card is an ace of spades.
    \item Calculate the probability that the card is (i) an ace and (ii) black card.
\end{enumerate}
\solution
%Let $X$ be an bernoulli rv defined as in \tabref{tab:exemplar/11/16/3/26}.  Then, 
\begin{equation}
    p =
        \frac{4}{11} 
\end{equation}
\begin{table}[H]
	\centering
	\input{exemplar/11/16/3/26/tables/Table2.tex}
	\caption{}
        \label{tab:exemplar/11/16/3/26}
\end{table}

\item The probability that a non leap year selected at random will contain 53 sundays.
\\
\solution
%\begin{table}[H]
	\centering
\begin{tabular}{|c|c|c|}
\hline
Random variable &Value &Definition\\ \hline
\multirow{3}{*}{X} &0 &Slips of Rs 1\\
&1 &Slips of Rs 5\\
&2 &Slips of Rs 13\\ \hline
\multirow{2}{*}{Y} &0 &Box A\\
&1 &Box B\\\hline
\end{tabular}
\caption{}
\label{tab:Distribution}
\end{table}
See \tabref{tab:Distribution}.
\begin{align}
p_{Y}\brak{k}= \begin{cases} 
      \frac{1}{3} & {k=0} \\
      \frac{2}{3 }& {k=1} 
   \end{cases}
   \\
p_{Y|X}\brak{0|0} = \frac{19}{25}\, 
p_{Y|X}\brak{0|1} = \frac{6}{25}\,
p_{Y|X}\brak{1|0} = \frac{45}{50}\,
p_{Y|X}\brak{1|2} = \frac{5}{50}
\end{align}
The desired probability is the probability that a slip drawn at random is marked other than Rs 1,
\begin{align}
&=1-p_X\brak{0}\\
&= p_X(1) + p_X(2)
\end{align}
Using Bayes theorem,
\begin{align}
&= p_Y\brak{0} \times \pr{Y=0 | X=1} + p_Y\brak{1} \times \pr{Y=1|X=2}\\
&=\frac{1}{3} \times \frac{6}{25} + \frac{2}{3} \times \frac{5}{50}\\
&=\frac{11}{75}
\end{align}

\newpage

%\tableofcontents

\bigskip

\renewcommand{\thefigure}{\theenumi}
\renewcommand{\thetable}{\theenumi}
%\renewcommand{\theequation}{\theenumi}

%\begin{abstract}
%%\boldmath
%In this letter, an algorithm for evaluating the exact analytical bit error rate  (BER)  for the piecewise linear (PL) combiner for  multiple relays is presented. Previous results were available only for upto three relays. The algorithm is unique in the sense that  the actual mathematical expressions, that are prohibitively large, need not be explicitly obtained. The diversity gain due to multiple relays is shown through plots of the analytical BER, well supported by simulations. 
%
%\end{abstract}
% IEEEtran.cls defaults to using nonbold math in the Abstract.
% This preserves the distinction between vectors and scalars. However,
% if the journal you are submitting to favors bold math in the abstract,
% then you can use LaTeX's standard command \boldmath at the very start
% of the abstract to achieve this. Many IEEE journals frown on math
% in the abstract anyway.

% Note that keywords are not normally used for peerreview papers.
%\begin{IEEEkeywords}
%Cooperative diversity, decode and forward, piecewise linear
%\end{IEEEkeywords}



% For peer review papers, you can put extra information on the cover
% page as needed:
% \ifCLASSOPTIONpeerreview
% \begin{center} \bfseries EDICS Category: 3-BBND \end{center}
% \fi
%
% For peerreview papers, this IEEEtran command inserts a page break and
% creates the second title. It will be ignored for other modes.
%\IEEEpeerreviewmaketitle




\item One of the four persons John, Rita, Aslam or Gurpreet will be promoted next
month. Consequently the sample space consists of four elementary outcomes
S = {John promoted, Rita promoted, Aslam promoted, Gurpreet promoted}
You are told that the chances of John’s promotion is same as that of Gurpreet,
Rita’s chances of promotion are twice as likely as Johns. Aslam’s chances are
four times that of John.
\begin{enumerate}
	\item Determine
	\begin{enumerate}
		\item P (John promoted)
		\item P (Rita promoted)
		\item P (Aslam promoted)
		\item P (Gurpreet promoted)
	\end{enumerate}
	\item If A = {John promoted or Gurpreet promoted}, find P (A).
\end{enumerate}
\solution
%\begin{table}[H]
	\centering
\begin{tabular}{|c|c|c|}
\hline
Random variable &Value &Definition\\ \hline
\multirow{3}{*}{X} &0 &Slips of Rs 1\\
&1 &Slips of Rs 5\\
&2 &Slips of Rs 13\\ \hline
\multirow{2}{*}{Y} &0 &Box A\\
&1 &Box B\\\hline
\end{tabular}
\caption{}
\label{tab:Distribution}
\end{table}
See \tabref{tab:Distribution}.
\begin{align}
p_{Y}\brak{k}= \begin{cases} 
      \frac{1}{3} & {k=0} \\
      \frac{2}{3 }& {k=1} 
   \end{cases}
   \\
p_{Y|X}\brak{0|0} = \frac{19}{25}\, 
p_{Y|X}\brak{0|1} = \frac{6}{25}\,
p_{Y|X}\brak{1|0} = \frac{45}{50}\,
p_{Y|X}\brak{1|2} = \frac{5}{50}
\end{align}
The desired probability is the probability that a slip drawn at random is marked other than Rs 1,
\begin{align}
&=1-p_X\brak{0}\\
&= p_X(1) + p_X(2)
\end{align}
Using Bayes theorem,
\begin{align}
&= p_Y\brak{0} \times \pr{Y=0 | X=1} + p_Y\brak{1} \times \pr{Y=1|X=2}\\
&=\frac{1}{3} \times \frac{6}{25} + \frac{2}{3} \times \frac{5}{50}\\
&=\frac{11}{75}
\end{align}

\newpage

%\tableofcontents

\bigskip

\renewcommand{\thefigure}{\theenumi}
\renewcommand{\thetable}{\theenumi}
%\renewcommand{\theequation}{\theenumi}

%\begin{abstract}
%%\boldmath
%In this letter, an algorithm for evaluating the exact analytical bit error rate  (BER)  for the piecewise linear (PL) combiner for  multiple relays is presented. Previous results were available only for upto three relays. The algorithm is unique in the sense that  the actual mathematical expressions, that are prohibitively large, need not be explicitly obtained. The diversity gain due to multiple relays is shown through plots of the analytical BER, well supported by simulations. 
%
%\end{abstract}
% IEEEtran.cls defaults to using nonbold math in the Abstract.
% This preserves the distinction between vectors and scalars. However,
% if the journal you are submitting to favors bold math in the abstract,
% then you can use LaTeX's standard command \boldmath at the very start
% of the abstract to achieve this. Many IEEE journals frown on math
% in the abstract anyway.

% Note that keywords are not normally used for peerreview papers.
%\begin{IEEEkeywords}
%Cooperative diversity, decode and forward, piecewise linear
%\end{IEEEkeywords}



% For peer review papers, you can put extra information on the cover
% page as needed:
% \ifCLASSOPTIONpeerreview
% \begin{center} \bfseries EDICS Category: 3-BBND \end{center}
% \fi
%
% For peerreview papers, this IEEEtran command inserts a page break and
% creates the second title. It will be ignored for other modes.
%\IEEEpeerreviewmaketitle




\item A card is drawn from a deck of 52 cards. Find the probability of getting a king or a heart or a red card.\\
\solution
%\begin{table}[H]
	\centering
\begin{tabular}{|c|c|c|}
\hline
Random variable &Value &Definition\\ \hline
\multirow{3}{*}{X} &0 &Slips of Rs 1\\
&1 &Slips of Rs 5\\
&2 &Slips of Rs 13\\ \hline
\multirow{2}{*}{Y} &0 &Box A\\
&1 &Box B\\\hline
\end{tabular}
\caption{}
\label{tab:Distribution}
\end{table}
See \tabref{tab:Distribution}.
\begin{align}
p_{Y}\brak{k}= \begin{cases} 
      \frac{1}{3} & {k=0} \\
      \frac{2}{3 }& {k=1} 
   \end{cases}
   \\
p_{Y|X}\brak{0|0} = \frac{19}{25}\, 
p_{Y|X}\brak{0|1} = \frac{6}{25}\,
p_{Y|X}\brak{1|0} = \frac{45}{50}\,
p_{Y|X}\brak{1|2} = \frac{5}{50}
\end{align}
The desired probability is the probability that a slip drawn at random is marked other than Rs 1,
\begin{align}
&=1-p_X\brak{0}\\
&= p_X(1) + p_X(2)
\end{align}
Using Bayes theorem,
\begin{align}
&= p_Y\brak{0} \times \pr{Y=0 | X=1} + p_Y\brak{1} \times \pr{Y=1|X=2}\\
&=\frac{1}{3} \times \frac{6}{25} + \frac{2}{3} \times \frac{5}{50}\\
&=\frac{11}{75}
\end{align}

\newpage

%\tableofcontents

\bigskip

\renewcommand{\thefigure}{\theenumi}
\renewcommand{\thetable}{\theenumi}
%\renewcommand{\theequation}{\theenumi}

%\begin{abstract}
%%\boldmath
%In this letter, an algorithm for evaluating the exact analytical bit error rate  (BER)  for the piecewise linear (PL) combiner for  multiple relays is presented. Previous results were available only for upto three relays. The algorithm is unique in the sense that  the actual mathematical expressions, that are prohibitively large, need not be explicitly obtained. The diversity gain due to multiple relays is shown through plots of the analytical BER, well supported by simulations. 
%
%\end{abstract}
% IEEEtran.cls defaults to using nonbold math in the Abstract.
% This preserves the distinction between vectors and scalars. However,
% if the journal you are submitting to favors bold math in the abstract,
% then you can use LaTeX's standard command \boldmath at the very start
% of the abstract to achieve this. Many IEEE journals frown on math
% in the abstract anyway.

% Note that keywords are not normally used for peerreview papers.
%\begin{IEEEkeywords}
%Cooperative diversity, decode and forward, piecewise linear
%\end{IEEEkeywords}



% For peer review papers, you can put extra information on the cover
% page as needed:
% \ifCLASSOPTIONpeerreview
% \begin{center} \bfseries EDICS Category: 3-BBND \end{center}
% \fi
%
% For peerreview papers, this IEEEtran command inserts a page break and
% creates the second title. It will be ignored for other modes.
%\IEEEpeerreviewmaketitle




\item The probability that a student will pass his examination is 0.73, the probability of
the student getting a compartment is 0.13, and the probability that the student will
either pass or get compartment is 0.96. State True or False.\\
\solution
%\begin{table}[H]
	\centering
\begin{tabular}{|c|c|c|}
\hline
Random variable &Value &Definition\\ \hline
\multirow{3}{*}{X} &0 &Slips of Rs 1\\
&1 &Slips of Rs 5\\
&2 &Slips of Rs 13\\ \hline
\multirow{2}{*}{Y} &0 &Box A\\
&1 &Box B\\\hline
\end{tabular}
\caption{}
\label{tab:Distribution}
\end{table}
See \tabref{tab:Distribution}.
\begin{align}
p_{Y}\brak{k}= \begin{cases} 
      \frac{1}{3} & {k=0} \\
      \frac{2}{3 }& {k=1} 
   \end{cases}
   \\
p_{Y|X}\brak{0|0} = \frac{19}{25}\, 
p_{Y|X}\brak{0|1} = \frac{6}{25}\,
p_{Y|X}\brak{1|0} = \frac{45}{50}\,
p_{Y|X}\brak{1|2} = \frac{5}{50}
\end{align}
The desired probability is the probability that a slip drawn at random is marked other than Rs 1,
\begin{align}
&=1-p_X\brak{0}\\
&= p_X(1) + p_X(2)
\end{align}
Using Bayes theorem,
\begin{align}
&= p_Y\brak{0} \times \pr{Y=0 | X=1} + p_Y\brak{1} \times \pr{Y=1|X=2}\\
&=\frac{1}{3} \times \frac{6}{25} + \frac{2}{3} \times \frac{5}{50}\\
&=\frac{11}{75}
\end{align}

\newpage

%\tableofcontents

\bigskip

\renewcommand{\thefigure}{\theenumi}
\renewcommand{\thetable}{\theenumi}
%\renewcommand{\theequation}{\theenumi}

%\begin{abstract}
%%\boldmath
%In this letter, an algorithm for evaluating the exact analytical bit error rate  (BER)  for the piecewise linear (PL) combiner for  multiple relays is presented. Previous results were available only for upto three relays. The algorithm is unique in the sense that  the actual mathematical expressions, that are prohibitively large, need not be explicitly obtained. The diversity gain due to multiple relays is shown through plots of the analytical BER, well supported by simulations. 
%
%\end{abstract}
% IEEEtran.cls defaults to using nonbold math in the Abstract.
% This preserves the distinction between vectors and scalars. However,
% if the journal you are submitting to favors bold math in the abstract,
% then you can use LaTeX's standard command \boldmath at the very start
% of the abstract to achieve this. Many IEEE journals frown on math
% in the abstract anyway.

% Note that keywords are not normally used for peerreview papers.
%\begin{IEEEkeywords}
%Cooperative diversity, decode and forward, piecewise linear
%\end{IEEEkeywords}



% For peer review papers, you can put extra information on the cover
% page as needed:
% \ifCLASSOPTIONpeerreview
% \begin{center} \bfseries EDICS Category: 3-BBND \end{center}
% \fi
%
% For peerreview papers, this IEEEtran command inserts a page break and
% creates the second title. It will be ignored for other modes.
%\IEEEpeerreviewmaketitle




\item A card is selected from a pack of 52 cards\\
\begin{enumerate}[label=(\alph*)]
\item How many points are there in the sample space?
\item Calculate the probability that the cards is an ace of spades.
\item Calculate the probability that the card is (i) an ace (ii)black card.\\
\end{enumerate}
%\input{ncert/11/16/3/4_1/Prob_4.tex}
\item In a non-leap year, the probability of having 53 tuesdays or 53 wednesdays is\\
\solution
%A non-leap year has a total of 365 days, and a week has 7 days.\\
So it can be expressed as 
\begin{align}
365\text{days} &=52\times 7+1 \text{day}
\end{align}
$\implies$ 52 tuesdays or wednesdays\\
Random variable X denotes the days of a week
\begin{align}
p_X\brak{k}&=\frac{1}{7}; \quad \brak{1<k<7}
\end{align}
So the probability of extra day being tuesday or wednesday is
\begin{align}
p_X\brak{3}+p_X\brak{4}&=\frac{1}{7}+\frac{1}{7}=\frac{2}{7}
\end{align}



\item There are 1000 sealed envelopes in a box, 10 of them contain a cash prize of
Rs 100 each, 100 of them contain a cash prize of Rs 50 each and 200 of them
contain a cash prize of Rs 10 each and rest do not contain any cash prize. If they
are well shuffled and an envelope is picked up out, what is the probability that it
contains no cash prize?\\
\solution
%\begin{table}[H]
	\centering
\begin{tabular}{|c|c|c|}
\hline
Random variable &Value &Definition\\ \hline
\multirow{3}{*}{X} &0 &Slips of Rs 1\\
&1 &Slips of Rs 5\\
&2 &Slips of Rs 13\\ \hline
\multirow{2}{*}{Y} &0 &Box A\\
&1 &Box B\\\hline
\end{tabular}
\caption{}
\label{tab:Distribution}
\end{table}
See \tabref{tab:Distribution}.
\begin{align}
p_{Y}\brak{k}= \begin{cases} 
      \frac{1}{3} & {k=0} \\
      \frac{2}{3 }& {k=1} 
   \end{cases}
   \\
p_{Y|X}\brak{0|0} = \frac{19}{25}\, 
p_{Y|X}\brak{0|1} = \frac{6}{25}\,
p_{Y|X}\brak{1|0} = \frac{45}{50}\,
p_{Y|X}\brak{1|2} = \frac{5}{50}
\end{align}
The desired probability is the probability that a slip drawn at random is marked other than Rs 1,
\begin{align}
&=1-p_X\brak{0}\\
&= p_X(1) + p_X(2)
\end{align}
Using Bayes theorem,
\begin{align}
&= p_Y\brak{0} \times \pr{Y=0 | X=1} + p_Y\brak{1} \times \pr{Y=1|X=2}\\
&=\frac{1}{3} \times \frac{6}{25} + \frac{2}{3} \times \frac{5}{50}\\
&=\frac{11}{75}
\end{align}

\newpage

%\tableofcontents

\bigskip

\renewcommand{\thefigure}{\theenumi}
\renewcommand{\thetable}{\theenumi}
%\renewcommand{\theequation}{\theenumi}

%\begin{abstract}
%%\boldmath
%In this letter, an algorithm for evaluating the exact analytical bit error rate  (BER)  for the piecewise linear (PL) combiner for  multiple relays is presented. Previous results were available only for upto three relays. The algorithm is unique in the sense that  the actual mathematical expressions, that are prohibitively large, need not be explicitly obtained. The diversity gain due to multiple relays is shown through plots of the analytical BER, well supported by simulations. 
%
%\end{abstract}
% IEEEtran.cls defaults to using nonbold math in the Abstract.
% This preserves the distinction between vectors and scalars. However,
% if the journal you are submitting to favors bold math in the abstract,
% then you can use LaTeX's standard command \boldmath at the very start
% of the abstract to achieve this. Many IEEE journals frown on math
% in the abstract anyway.

% Note that keywords are not normally used for peerreview papers.
%\begin{IEEEkeywords}
%Cooperative diversity, decode and forward, piecewise linear
%\end{IEEEkeywords}



% For peer review papers, you can put extra information on the cover
% page as needed:
% \ifCLASSOPTIONpeerreview
% \begin{center} \bfseries EDICS Category: 3-BBND \end{center}
% \fi
%
% For peerreview papers, this IEEEtran command inserts a page break and
% creates the second title. It will be ignored for other modes.
%\IEEEpeerreviewmaketitle




\item 
A die is thrown and a card is selected at random from a deck of 52 playing cards. The probability of getting an even number on the die and a spade card.\\
\solution
%\begin{table}[H]
	\centering
\begin{tabular}{|c|c|c|}
\hline
Random variable &Value &Definition\\ \hline
\multirow{3}{*}{X} &0 &Slips of Rs 1\\
&1 &Slips of Rs 5\\
&2 &Slips of Rs 13\\ \hline
\multirow{2}{*}{Y} &0 &Box A\\
&1 &Box B\\\hline
\end{tabular}
\caption{}
\label{tab:Distribution}
\end{table}
See \tabref{tab:Distribution}.
\begin{align}
p_{Y}\brak{k}= \begin{cases} 
      \frac{1}{3} & {k=0} \\
      \frac{2}{3 }& {k=1} 
   \end{cases}
   \\
p_{Y|X}\brak{0|0} = \frac{19}{25}\, 
p_{Y|X}\brak{0|1} = \frac{6}{25}\,
p_{Y|X}\brak{1|0} = \frac{45}{50}\,
p_{Y|X}\brak{1|2} = \frac{5}{50}
\end{align}
The desired probability is the probability that a slip drawn at random is marked other than Rs 1,
\begin{align}
&=1-p_X\brak{0}\\
&= p_X(1) + p_X(2)
\end{align}
Using Bayes theorem,
\begin{align}
&= p_Y\brak{0} \times \pr{Y=0 | X=1} + p_Y\brak{1} \times \pr{Y=1|X=2}\\
&=\frac{1}{3} \times \frac{6}{25} + \frac{2}{3} \times \frac{5}{50}\\
&=\frac{11}{75}
\end{align}

\newpage

%\tableofcontents

\bigskip

\renewcommand{\thefigure}{\theenumi}
\renewcommand{\thetable}{\theenumi}
%\renewcommand{\theequation}{\theenumi}

%\begin{abstract}
%%\boldmath
%In this letter, an algorithm for evaluating the exact analytical bit error rate  (BER)  for the piecewise linear (PL) combiner for  multiple relays is presented. Previous results were available only for upto three relays. The algorithm is unique in the sense that  the actual mathematical expressions, that are prohibitively large, need not be explicitly obtained. The diversity gain due to multiple relays is shown through plots of the analytical BER, well supported by simulations. 
%
%\end{abstract}
% IEEEtran.cls defaults to using nonbold math in the Abstract.
% This preserves the distinction between vectors and scalars. However,
% if the journal you are submitting to favors bold math in the abstract,
% then you can use LaTeX's standard command \boldmath at the very start
% of the abstract to achieve this. Many IEEE journals frown on math
% in the abstract anyway.

% Note that keywords are not normally used for peerreview papers.
%\begin{IEEEkeywords}
%Cooperative diversity, decode and forward, piecewise linear
%\end{IEEEkeywords}



% For peer review papers, you can put extra information on the cover
% page as needed:
% \ifCLASSOPTIONpeerreview
% \begin{center} \bfseries EDICS Category: 3-BBND \end{center}
% \fi
%
% For peerreview papers, this IEEEtran command inserts a page break and
% creates the second title. It will be ignored for other modes.
%\IEEEpeerreviewmaketitle




\item
If 4-digit numbers greater than 5,000 are randomly formed from the digits 0, 1, 3, 5, and 7, what is the probability of forming a number divisible by 5 when:
\begin{enumerate}
    \item The digits are repeated?
    \item The repetition of digits is not allowed?
\end{enumerate}
\solution
%\begin{table}[H]
	\centering
\begin{tabular}{|c|c|c|}
\hline
Random variable &Value &Definition\\ \hline
\multirow{3}{*}{X} &0 &Slips of Rs 1\\
&1 &Slips of Rs 5\\
&2 &Slips of Rs 13\\ \hline
\multirow{2}{*}{Y} &0 &Box A\\
&1 &Box B\\\hline
\end{tabular}
\caption{}
\label{tab:Distribution}
\end{table}
See \tabref{tab:Distribution}.
\begin{align}
p_{Y}\brak{k}= \begin{cases} 
      \frac{1}{3} & {k=0} \\
      \frac{2}{3 }& {k=1} 
   \end{cases}
   \\
p_{Y|X}\brak{0|0} = \frac{19}{25}\, 
p_{Y|X}\brak{0|1} = \frac{6}{25}\,
p_{Y|X}\brak{1|0} = \frac{45}{50}\,
p_{Y|X}\brak{1|2} = \frac{5}{50}
\end{align}
The desired probability is the probability that a slip drawn at random is marked other than Rs 1,
\begin{align}
&=1-p_X\brak{0}\\
&= p_X(1) + p_X(2)
\end{align}
Using Bayes theorem,
\begin{align}
&= p_Y\brak{0} \times \pr{Y=0 | X=1} + p_Y\brak{1} \times \pr{Y=1|X=2}\\
&=\frac{1}{3} \times \frac{6}{25} + \frac{2}{3} \times \frac{5}{50}\\
&=\frac{11}{75}
\end{align}

\newpage

%\tableofcontents

\bigskip

\renewcommand{\thefigure}{\theenumi}
\renewcommand{\thetable}{\theenumi}
%\renewcommand{\theequation}{\theenumi}

%\begin{abstract}
%%\boldmath
%In this letter, an algorithm for evaluating the exact analytical bit error rate  (BER)  for the piecewise linear (PL) combiner for  multiple relays is presented. Previous results were available only for upto three relays. The algorithm is unique in the sense that  the actual mathematical expressions, that are prohibitively large, need not be explicitly obtained. The diversity gain due to multiple relays is shown through plots of the analytical BER, well supported by simulations. 
%
%\end{abstract}
% IEEEtran.cls defaults to using nonbold math in the Abstract.
% This preserves the distinction between vectors and scalars. However,
% if the journal you are submitting to favors bold math in the abstract,
% then you can use LaTeX's standard command \boldmath at the very start
% of the abstract to achieve this. Many IEEE journals frown on math
% in the abstract anyway.

% Note that keywords are not normally used for peerreview papers.
%\begin{IEEEkeywords}
%Cooperative diversity, decode and forward, piecewise linear
%\end{IEEEkeywords}



% For peer review papers, you can put extra information on the cover
% page as needed:
% \ifCLASSOPTIONpeerreview
% \begin{center} \bfseries EDICS Category: 3-BBND \end{center}
% \fi
%
% For peerreview papers, this IEEEtran command inserts a page break and
% creates the second title. It will be ignored for other modes.
%\IEEEpeerreviewmaketitle




\item Consider the probability space $\brak{\Omega, \mathcal{G}, P}$ where $\Omega = [0,2]$ and $\mathcal{G} = \cbrak{\phi, \Omega, [0,1], (1,2]}$. Let $X$ and $Y$ be two functions on $\Omega$ defined as
\begin{align*}
    X(\omega) = 
    \begin{cases}
        1 & \text{if }\omega \in [0, 1]\\
        2 & \text{if }\omega \in (1, 2]
    \end{cases}
\end{align*}
and
\begin{align*}
    Y(\omega) = 
    \begin{cases}
        2 & \text{if }\omega \in [0, 1.5]\\
        3 & \text{if }\omega \in (1.5, 2].
    \end{cases}
\end{align*}
Then which one of the following statements is true?
\begin{enumerate}
    \item [(A)] $X$ is a random variable with respect to $\mathcal{G}$, but $Y$ is not a random variable with respect to $\mathcal{G}$.
    \item [(B)] $Y$ is a random variable with respect to $\mathcal{G}$, but $X$ is not a random variable with respect to $\mathcal{G}$.
    \item [(C)] Neither $X$ nor $Y$ is a random variable with respect to $\mathcal{G}$.
    \item [(D)] Both $X$ and $Y$ are random variables with respect to $\mathcal{G}$.
\end{enumerate} \hfill (GATE ST 2023)\\
\solution
%\begin{table}[H]
	\centering
\begin{tabular}{|c|c|c|}
\hline
Random variable &Value &Definition\\ \hline
\multirow{3}{*}{X} &0 &Slips of Rs 1\\
&1 &Slips of Rs 5\\
&2 &Slips of Rs 13\\ \hline
\multirow{2}{*}{Y} &0 &Box A\\
&1 &Box B\\\hline
\end{tabular}
\caption{}
\label{tab:Distribution}
\end{table}
See \tabref{tab:Distribution}.
\begin{align}
p_{Y}\brak{k}= \begin{cases} 
      \frac{1}{3} & {k=0} \\
      \frac{2}{3 }& {k=1} 
   \end{cases}
   \\
p_{Y|X}\brak{0|0} = \frac{19}{25}\, 
p_{Y|X}\brak{0|1} = \frac{6}{25}\,
p_{Y|X}\brak{1|0} = \frac{45}{50}\,
p_{Y|X}\brak{1|2} = \frac{5}{50}
\end{align}
The desired probability is the probability that a slip drawn at random is marked other than Rs 1,
\begin{align}
&=1-p_X\brak{0}\\
&= p_X(1) + p_X(2)
\end{align}
Using Bayes theorem,
\begin{align}
&= p_Y\brak{0} \times \pr{Y=0 | X=1} + p_Y\brak{1} \times \pr{Y=1|X=2}\\
&=\frac{1}{3} \times \frac{6}{25} + \frac{2}{3} \times \frac{5}{50}\\
&=\frac{11}{75}
\end{align}

\newpage

%\tableofcontents

\bigskip

\renewcommand{\thefigure}{\theenumi}
\renewcommand{\thetable}{\theenumi}
%\renewcommand{\theequation}{\theenumi}

%\begin{abstract}
%%\boldmath
%In this letter, an algorithm for evaluating the exact analytical bit error rate  (BER)  for the piecewise linear (PL) combiner for  multiple relays is presented. Previous results were available only for upto three relays. The algorithm is unique in the sense that  the actual mathematical expressions, that are prohibitively large, need not be explicitly obtained. The diversity gain due to multiple relays is shown through plots of the analytical BER, well supported by simulations. 
%
%\end{abstract}
% IEEEtran.cls defaults to using nonbold math in the Abstract.
% This preserves the distinction between vectors and scalars. However,
% if the journal you are submitting to favors bold math in the abstract,
% then you can use LaTeX's standard command \boldmath at the very start
% of the abstract to achieve this. Many IEEE journals frown on math
% in the abstract anyway.

% Note that keywords are not normally used for peerreview papers.
%\begin{IEEEkeywords}
%Cooperative diversity, decode and forward, piecewise linear
%\end{IEEEkeywords}



% For peer review papers, you can put extra information on the cover
% page as needed:
% \ifCLASSOPTIONpeerreview
% \begin{center} \bfseries EDICS Category: 3-BBND \end{center}
% \fi
%
% For peerreview papers, this IEEEtran command inserts a page break and
% creates the second title. It will be ignored for other modes.
%\IEEEpeerreviewmaketitle




	\item  A die is loaded in such a way that each odd number is twice as likely to occur as
each even number. Find $P(G)$, where $G$ is the event that a number greater than
3 occurs on a single roll of the die.
\\
\solution
		%\begin{table}[H]
	\centering
\begin{tabular}{|c|c|c|}
\hline
Random variable &Value &Definition\\ \hline
\multirow{3}{*}{X} &0 &Slips of Rs 1\\
&1 &Slips of Rs 5\\
&2 &Slips of Rs 13\\ \hline
\multirow{2}{*}{Y} &0 &Box A\\
&1 &Box B\\\hline
\end{tabular}
\caption{}
\label{tab:Distribution}
\end{table}
See \tabref{tab:Distribution}.
\begin{align}
p_{Y}\brak{k}= \begin{cases} 
      \frac{1}{3} & {k=0} \\
      \frac{2}{3 }& {k=1} 
   \end{cases}
   \\
p_{Y|X}\brak{0|0} = \frac{19}{25}\, 
p_{Y|X}\brak{0|1} = \frac{6}{25}\,
p_{Y|X}\brak{1|0} = \frac{45}{50}\,
p_{Y|X}\brak{1|2} = \frac{5}{50}
\end{align}
The desired probability is the probability that a slip drawn at random is marked other than Rs 1,
\begin{align}
&=1-p_X\brak{0}\\
&= p_X(1) + p_X(2)
\end{align}
Using Bayes theorem,
\begin{align}
&= p_Y\brak{0} \times \pr{Y=0 | X=1} + p_Y\brak{1} \times \pr{Y=1|X=2}\\
&=\frac{1}{3} \times \frac{6}{25} + \frac{2}{3} \times \frac{5}{50}\\
&=\frac{11}{75}
\end{align}

\newpage

%\tableofcontents

\bigskip

\renewcommand{\thefigure}{\theenumi}
\renewcommand{\thetable}{\theenumi}
%\renewcommand{\theequation}{\theenumi}

%\begin{abstract}
%%\boldmath
%In this letter, an algorithm for evaluating the exact analytical bit error rate  (BER)  for the piecewise linear (PL) combiner for  multiple relays is presented. Previous results were available only for upto three relays. The algorithm is unique in the sense that  the actual mathematical expressions, that are prohibitively large, need not be explicitly obtained. The diversity gain due to multiple relays is shown through plots of the analytical BER, well supported by simulations. 
%
%\end{abstract}
% IEEEtran.cls defaults to using nonbold math in the Abstract.
% This preserves the distinction between vectors and scalars. However,
% if the journal you are submitting to favors bold math in the abstract,
% then you can use LaTeX's standard command \boldmath at the very start
% of the abstract to achieve this. Many IEEE journals frown on math
% in the abstract anyway.

% Note that keywords are not normally used for peerreview papers.
%\begin{IEEEkeywords}
%Cooperative diversity, decode and forward, piecewise linear
%\end{IEEEkeywords}



% For peer review papers, you can put extra information on the cover
% page as needed:
% \ifCLASSOPTIONpeerreview
% \begin{center} \bfseries EDICS Category: 3-BBND \end{center}
% \fi
%
% For peerreview papers, this IEEEtran command inserts a page break and
% creates the second title. It will be ignored for other modes.
%\IEEEpeerreviewmaketitle




	\item All the jacks, queens and kings are removed from a deck of 52 playing cards. The remaining cards are well shuffled and then one card is drawn at random. Giving ace a value 1 similar value for other cards, find the probability that the card has a value 
		\begin{enumerate}
			\item 7
			\item greater than 7
			\item less than 7
		\end{enumerate}
		%Number of cards left after removing all jacks, queens and kings 
\begin{align}
N	= 52 - 4\times 3
	= 40
\end{align}
%\begin{table}[H]
%\def\arraystretch{1.2}
%\begin{tabular}{|c|c|c|}
%\hline
%	\textbf{Parameter} &\textbf{Value} &\textbf{Description}\\ \hline
%	$X$ &1-10 &Represents the value of the card picked \\ \hline
%\end{tabular}
%\end{table}
Let $1 \le X \le 10$ be the value of the card picked.  Then,
\begin{align}
	p_X(k) &= \Pr(X=k)\ \forall\ 1 \leq k \leq 10\\
	&= \frac{4\times 1}{40}\\
	&= \frac{1}{10}\\
	\therefore p_X(k) &= 
	\begin{cases}
		\frac{1}{10} & 1 \leq k \leq 10\\
		0 & \text{otherwise}
	\end{cases}
\end{align}
and
\begin{align}
	F_{X}(k) &= \sum_{m=0}^{k}p_{X}(m) \quad 1 \leq k \leq 10\\
	&= \frac{k}{10}\\
	\therefore F_{X}(k) &= 
	\begin{cases}
		0 & k \leq 0\\
		\frac{k}{10} & 1\leq k \leq 10\\
		1 & k > 10 
	\end{cases}
\end{align}
\begin{enumerate}
	\item Probability that card has value equal to 7 is
		\begin{align}
			 p_{X}(7)
			= \frac{1}{10}
		\end{align}
	\item Probability that card has value greater than 7 is
		\begin{align}
			1 - F_X(7)
			&= 1 - \frac{7}{10}
			\\
			&= \frac{3}{10}
		\end{align}
	\item Probability that card has value less than 7 is
		\begin{align}
			 F_{X}(6)
			=\frac{6}{10}
		\end{align}
\end{enumerate}

  \item A Lot consists of 48 mobile phones of which 42 are good, 3 have only minor defects and 3 have major defects.Varnika will buy a phone if it is good but the trader will only buy a mobile if it has no major defects. One phone is selected at random from the lot. What is the probability that it is
\begin{enumerate}
	\item acceptable to Varnika?
            \item acceptable to the trader?
\end{enumerate}
\solution
	%\begin{table}[H]
	\centering
\begin{tabular}{|c|c|c|}
\hline
Random variable &Value &Definition\\ \hline
\multirow{3}{*}{X} &0 &Slips of Rs 1\\
&1 &Slips of Rs 5\\
&2 &Slips of Rs 13\\ \hline
\multirow{2}{*}{Y} &0 &Box A\\
&1 &Box B\\\hline
\end{tabular}
\caption{}
\label{tab:Distribution}
\end{table}
See \tabref{tab:Distribution}.
\begin{align}
p_{Y}\brak{k}= \begin{cases} 
      \frac{1}{3} & {k=0} \\
      \frac{2}{3 }& {k=1} 
   \end{cases}
   \\
p_{Y|X}\brak{0|0} = \frac{19}{25}\, 
p_{Y|X}\brak{0|1} = \frac{6}{25}\,
p_{Y|X}\brak{1|0} = \frac{45}{50}\,
p_{Y|X}\brak{1|2} = \frac{5}{50}
\end{align}
The desired probability is the probability that a slip drawn at random is marked other than Rs 1,
\begin{align}
&=1-p_X\brak{0}\\
&= p_X(1) + p_X(2)
\end{align}
Using Bayes theorem,
\begin{align}
&= p_Y\brak{0} \times \pr{Y=0 | X=1} + p_Y\brak{1} \times \pr{Y=1|X=2}\\
&=\frac{1}{3} \times \frac{6}{25} + \frac{2}{3} \times \frac{5}{50}\\
&=\frac{11}{75}
\end{align}

\newpage

%\tableofcontents

\bigskip

\renewcommand{\thefigure}{\theenumi}
\renewcommand{\thetable}{\theenumi}
%\renewcommand{\theequation}{\theenumi}

%\begin{abstract}
%%\boldmath
%In this letter, an algorithm for evaluating the exact analytical bit error rate  (BER)  for the piecewise linear (PL) combiner for  multiple relays is presented. Previous results were available only for upto three relays. The algorithm is unique in the sense that  the actual mathematical expressions, that are prohibitively large, need not be explicitly obtained. The diversity gain due to multiple relays is shown through plots of the analytical BER, well supported by simulations. 
%
%\end{abstract}
% IEEEtran.cls defaults to using nonbold math in the Abstract.
% This preserves the distinction between vectors and scalars. However,
% if the journal you are submitting to favors bold math in the abstract,
% then you can use LaTeX's standard command \boldmath at the very start
% of the abstract to achieve this. Many IEEE journals frown on math
% in the abstract anyway.

% Note that keywords are not normally used for peerreview papers.
%\begin{IEEEkeywords}
%Cooperative diversity, decode and forward, piecewise linear
%\end{IEEEkeywords}



% For peer review papers, you can put extra information on the cover
% page as needed:
% \ifCLASSOPTIONpeerreview
% \begin{center} \bfseries EDICS Category: 3-BBND \end{center}
% \fi
%
% For peerreview papers, this IEEEtran command inserts a page break and
% creates the second title. It will be ignored for other modes.
%\IEEEpeerreviewmaketitle




 \item A student says that if you throw a die, it will show up 1 or not 1. Therefore, the probability of getting 1 and the probability of getting 'not 1' each is equal to $\frac{1}{2}$. Is this correct? Give reasons.\\
 \solution
        %\begin{table}[H]
	\centering
\begin{tabular}{|c|c|c|}
\hline
Random variable &Value &Definition\\ \hline
\multirow{3}{*}{X} &0 &Slips of Rs 1\\
&1 &Slips of Rs 5\\
&2 &Slips of Rs 13\\ \hline
\multirow{2}{*}{Y} &0 &Box A\\
&1 &Box B\\\hline
\end{tabular}
\caption{}
\label{tab:Distribution}
\end{table}
See \tabref{tab:Distribution}.
\begin{align}
p_{Y}\brak{k}= \begin{cases} 
      \frac{1}{3} & {k=0} \\
      \frac{2}{3 }& {k=1} 
   \end{cases}
   \\
p_{Y|X}\brak{0|0} = \frac{19}{25}\, 
p_{Y|X}\brak{0|1} = \frac{6}{25}\,
p_{Y|X}\brak{1|0} = \frac{45}{50}\,
p_{Y|X}\brak{1|2} = \frac{5}{50}
\end{align}
The desired probability is the probability that a slip drawn at random is marked other than Rs 1,
\begin{align}
&=1-p_X\brak{0}\\
&= p_X(1) + p_X(2)
\end{align}
Using Bayes theorem,
\begin{align}
&= p_Y\brak{0} \times \pr{Y=0 | X=1} + p_Y\brak{1} \times \pr{Y=1|X=2}\\
&=\frac{1}{3} \times \frac{6}{25} + \frac{2}{3} \times \frac{5}{50}\\
&=\frac{11}{75}
\end{align}

\newpage

%\tableofcontents

\bigskip

\renewcommand{\thefigure}{\theenumi}
\renewcommand{\thetable}{\theenumi}
%\renewcommand{\theequation}{\theenumi}

%\begin{abstract}
%%\boldmath
%In this letter, an algorithm for evaluating the exact analytical bit error rate  (BER)  for the piecewise linear (PL) combiner for  multiple relays is presented. Previous results were available only for upto three relays. The algorithm is unique in the sense that  the actual mathematical expressions, that are prohibitively large, need not be explicitly obtained. The diversity gain due to multiple relays is shown through plots of the analytical BER, well supported by simulations. 
%
%\end{abstract}
% IEEEtran.cls defaults to using nonbold math in the Abstract.
% This preserves the distinction between vectors and scalars. However,
% if the journal you are submitting to favors bold math in the abstract,
% then you can use LaTeX's standard command \boldmath at the very start
% of the abstract to achieve this. Many IEEE journals frown on math
% in the abstract anyway.

% Note that keywords are not normally used for peerreview papers.
%\begin{IEEEkeywords}
%Cooperative diversity, decode and forward, piecewise linear
%\end{IEEEkeywords}



% For peer review papers, you can put extra information on the cover
% page as needed:
% \ifCLASSOPTIONpeerreview
% \begin{center} \bfseries EDICS Category: 3-BBND \end{center}
% \fi
%
% For peerreview papers, this IEEEtran command inserts a page break and
% creates the second title. It will be ignored for other modes.
%\IEEEpeerreviewmaketitle




   \item Four candidates A, B, C, D have ap-
plied for the assignment to coach a school cricket
team. If A is twice as likely to be selected as B, and
B and C are given about the same chance of being
selected, while C is twice as likely to be selected
as D, what are the probabilities that
\begin{enumerate}
\item C will be selected?
\item A will not be selected?
\end{enumerate}
	%\begin{table}[H]
	\centering
\begin{tabular}{|c|c|c|}
\hline
Random variable &Value &Definition\\ \hline
\multirow{3}{*}{X} &0 &Slips of Rs 1\\
&1 &Slips of Rs 5\\
&2 &Slips of Rs 13\\ \hline
\multirow{2}{*}{Y} &0 &Box A\\
&1 &Box B\\\hline
\end{tabular}
\caption{}
\label{tab:Distribution}
\end{table}
See \tabref{tab:Distribution}.
\begin{align}
p_{Y}\brak{k}= \begin{cases} 
      \frac{1}{3} & {k=0} \\
      \frac{2}{3 }& {k=1} 
   \end{cases}
   \\
p_{Y|X}\brak{0|0} = \frac{19}{25}\, 
p_{Y|X}\brak{0|1} = \frac{6}{25}\,
p_{Y|X}\brak{1|0} = \frac{45}{50}\,
p_{Y|X}\brak{1|2} = \frac{5}{50}
\end{align}
The desired probability is the probability that a slip drawn at random is marked other than Rs 1,
\begin{align}
&=1-p_X\brak{0}\\
&= p_X(1) + p_X(2)
\end{align}
Using Bayes theorem,
\begin{align}
&= p_Y\brak{0} \times \pr{Y=0 | X=1} + p_Y\brak{1} \times \pr{Y=1|X=2}\\
&=\frac{1}{3} \times \frac{6}{25} + \frac{2}{3} \times \frac{5}{50}\\
&=\frac{11}{75}
\end{align}

\newpage

%\tableofcontents

\bigskip

\renewcommand{\thefigure}{\theenumi}
\renewcommand{\thetable}{\theenumi}
%\renewcommand{\theequation}{\theenumi}

%\begin{abstract}
%%\boldmath
%In this letter, an algorithm for evaluating the exact analytical bit error rate  (BER)  for the piecewise linear (PL) combiner for  multiple relays is presented. Previous results were available only for upto three relays. The algorithm is unique in the sense that  the actual mathematical expressions, that are prohibitively large, need not be explicitly obtained. The diversity gain due to multiple relays is shown through plots of the analytical BER, well supported by simulations. 
%
%\end{abstract}
% IEEEtran.cls defaults to using nonbold math in the Abstract.
% This preserves the distinction between vectors and scalars. However,
% if the journal you are submitting to favors bold math in the abstract,
% then you can use LaTeX's standard command \boldmath at the very start
% of the abstract to achieve this. Many IEEE journals frown on math
% in the abstract anyway.

% Note that keywords are not normally used for peerreview papers.
%\begin{IEEEkeywords}
%Cooperative diversity, decode and forward, piecewise linear
%\end{IEEEkeywords}



% For peer review papers, you can put extra information on the cover
% page as needed:
% \ifCLASSOPTIONpeerreview
% \begin{center} \bfseries EDICS Category: 3-BBND \end{center}
% \fi
%
% For peerreview papers, this IEEEtran command inserts a page break and
% creates the second title. It will be ignored for other modes.
%\IEEEpeerreviewmaketitle




 \item A bag contain 24 balls of which $x$ balls are red, $2x$ are white and $3x$ are blue. A ball is selected at random, What is the probability that it is
\begin{enumerate}[label=\alph*)]
\item not red ?
\item white ?
\end{enumerate}
%\begin{table}[H]
	\centering
\begin{tabular}{|c|c|c|}
\hline
Random variable &Value &Definition\\ \hline
\multirow{3}{*}{X} &0 &Slips of Rs 1\\
&1 &Slips of Rs 5\\
&2 &Slips of Rs 13\\ \hline
\multirow{2}{*}{Y} &0 &Box A\\
&1 &Box B\\\hline
\end{tabular}
\caption{}
\label{tab:Distribution}
\end{table}
See \tabref{tab:Distribution}.
\begin{align}
p_{Y}\brak{k}= \begin{cases} 
      \frac{1}{3} & {k=0} \\
      \frac{2}{3 }& {k=1} 
   \end{cases}
   \\
p_{Y|X}\brak{0|0} = \frac{19}{25}\, 
p_{Y|X}\brak{0|1} = \frac{6}{25}\,
p_{Y|X}\brak{1|0} = \frac{45}{50}\,
p_{Y|X}\brak{1|2} = \frac{5}{50}
\end{align}
The desired probability is the probability that a slip drawn at random is marked other than Rs 1,
\begin{align}
&=1-p_X\brak{0}\\
&= p_X(1) + p_X(2)
\end{align}
Using Bayes theorem,
\begin{align}
&= p_Y\brak{0} \times \pr{Y=0 | X=1} + p_Y\brak{1} \times \pr{Y=1|X=2}\\
&=\frac{1}{3} \times \frac{6}{25} + \frac{2}{3} \times \frac{5}{50}\\
&=\frac{11}{75}
\end{align}

\newpage

%\tableofcontents

\bigskip

\renewcommand{\thefigure}{\theenumi}
\renewcommand{\thetable}{\theenumi}
%\renewcommand{\theequation}{\theenumi}

%\begin{abstract}
%%\boldmath
%In this letter, an algorithm for evaluating the exact analytical bit error rate  (BER)  for the piecewise linear (PL) combiner for  multiple relays is presented. Previous results were available only for upto three relays. The algorithm is unique in the sense that  the actual mathematical expressions, that are prohibitively large, need not be explicitly obtained. The diversity gain due to multiple relays is shown through plots of the analytical BER, well supported by simulations. 
%
%\end{abstract}
% IEEEtran.cls defaults to using nonbold math in the Abstract.
% This preserves the distinction between vectors and scalars. However,
% if the journal you are submitting to favors bold math in the abstract,
% then you can use LaTeX's standard command \boldmath at the very start
% of the abstract to achieve this. Many IEEE journals frown on math
% in the abstract anyway.

% Note that keywords are not normally used for peerreview papers.
%\begin{IEEEkeywords}
%Cooperative diversity, decode and forward, piecewise linear
%\end{IEEEkeywords}



% For peer review papers, you can put extra information on the cover
% page as needed:
% \ifCLASSOPTIONpeerreview
% \begin{center} \bfseries EDICS Category: 3-BBND \end{center}
% \fi
%
% For peerreview papers, this IEEEtran command inserts a page break and
% creates the second title. It will be ignored for other modes.
%\IEEEpeerreviewmaketitle




If the letters of the word ASSASSINATION are arranged at random. Find the Probability that
\begin{enumerate}[label=(\alph*)]
\item Four $S's$ come consecutively in the word
\item Two  $I's$ and two $N's$ come together
\item All $A's$ are not coming together
\item No two $A's$ are coming together
\end{enumerate}
%\begin{table}[H]
	\centering
\begin{tabular}{|c|c|c|}
\hline
Random variable &Value &Definition\\ \hline
\multirow{3}{*}{X} &0 &Slips of Rs 1\\
&1 &Slips of Rs 5\\
&2 &Slips of Rs 13\\ \hline
\multirow{2}{*}{Y} &0 &Box A\\
&1 &Box B\\\hline
\end{tabular}
\caption{}
\label{tab:Distribution}
\end{table}
See \tabref{tab:Distribution}.
\begin{align}
p_{Y}\brak{k}= \begin{cases} 
      \frac{1}{3} & {k=0} \\
      \frac{2}{3 }& {k=1} 
   \end{cases}
   \\
p_{Y|X}\brak{0|0} = \frac{19}{25}\, 
p_{Y|X}\brak{0|1} = \frac{6}{25}\,
p_{Y|X}\brak{1|0} = \frac{45}{50}\,
p_{Y|X}\brak{1|2} = \frac{5}{50}
\end{align}
The desired probability is the probability that a slip drawn at random is marked other than Rs 1,
\begin{align}
&=1-p_X\brak{0}\\
&= p_X(1) + p_X(2)
\end{align}
Using Bayes theorem,
\begin{align}
&= p_Y\brak{0} \times \pr{Y=0 | X=1} + p_Y\brak{1} \times \pr{Y=1|X=2}\\
&=\frac{1}{3} \times \frac{6}{25} + \frac{2}{3} \times \frac{5}{50}\\
&=\frac{11}{75}
\end{align}

\newpage

%\tableofcontents

\bigskip

\renewcommand{\thefigure}{\theenumi}
\renewcommand{\thetable}{\theenumi}
%\renewcommand{\theequation}{\theenumi}

%\begin{abstract}
%%\boldmath
%In this letter, an algorithm for evaluating the exact analytical bit error rate  (BER)  for the piecewise linear (PL) combiner for  multiple relays is presented. Previous results were available only for upto three relays. The algorithm is unique in the sense that  the actual mathematical expressions, that are prohibitively large, need not be explicitly obtained. The diversity gain due to multiple relays is shown through plots of the analytical BER, well supported by simulations. 
%
%\end{abstract}
% IEEEtran.cls defaults to using nonbold math in the Abstract.
% This preserves the distinction between vectors and scalars. However,
% if the journal you are submitting to favors bold math in the abstract,
% then you can use LaTeX's standard command \boldmath at the very start
% of the abstract to achieve this. Many IEEE journals frown on math
% in the abstract anyway.

% Note that keywords are not normally used for peerreview papers.
%\begin{IEEEkeywords}
%Cooperative diversity, decode and forward, piecewise linear
%\end{IEEEkeywords}



% For peer review papers, you can put extra information on the cover
% page as needed:
% \ifCLASSOPTIONpeerreview
% \begin{center} \bfseries EDICS Category: 3-BBND \end{center}
% \fi
%
% For peerreview papers, this IEEEtran command inserts a page break and
% creates the second title. It will be ignored for other modes.
%\IEEEpeerreviewmaketitle




	\item One urn contains two black balls (labelled B1 and B2) and one white ball. A
	second urn contains one black ball and two white balls (labelled W1 and W2).
	Suppose the following experiment is performed. One of the two urns is chosen
	at random. Next a ball is randomly chosen from the urn. Then a second ball is
	chosen at random from the same urn without replacing the first ball.
	
	\begin{enumerate}
	\item What is the probability that two black balls are chosen?
	
	\item What is the probability that two balls of opposite colour are chosen?
	\end{enumerate}
	\solution
	%\begin{align}
    \label{eq:12.13.6.18.1}
	\because	\pr{A|B} &> \pr{A},\
\frac{\pr{AB}}{\pr{B}} > \pr{A}
\\
    \label{eq:12.13.6.18.2}
	\implies \pr{AB} &> \pr{A}\pr{B}
	\\
	\text{or, } \frac{\pr{AB}}{\pr{A}} &=\pr{B|A} > \pr{A}
\end{align}

\end{enumerate}

		%
\item 
Out of 100 students, two sections of 40 and 60 are formed. If you and your friend are among the 100 students, what is the probability that
\begin{enumerate}
\item you both enter the same section?
\item you both enter the different sections?
\end{enumerate}
\solution
		%\begin{enumerate}[label=\thesection.\arabic*,ref=\thesection.\theenumi]
	\item One card is drawn from a well-shuffled deck of 52 cards. Find the probability of getting
\begin{enumerate}
\item A king of red colour 
\item A face card 
\item A red face card
\item The jack of hearts
\item A spade
\item The queen of diamonds

\end{enumerate}
\solution
		%\begin{table}[H]
	\centering
\begin{tabular}{|c|c|c|}
\hline
Random variable &Value &Definition\\ \hline
\multirow{3}{*}{X} &0 &Slips of Rs 1\\
&1 &Slips of Rs 5\\
&2 &Slips of Rs 13\\ \hline
\multirow{2}{*}{Y} &0 &Box A\\
&1 &Box B\\\hline
\end{tabular}
\caption{}
\label{tab:Distribution}
\end{table}
See \tabref{tab:Distribution}.
\begin{align}
p_{Y}\brak{k}= \begin{cases} 
      \frac{1}{3} & {k=0} \\
      \frac{2}{3 }& {k=1} 
   \end{cases}
   \\
p_{Y|X}\brak{0|0} = \frac{19}{25}\, 
p_{Y|X}\brak{0|1} = \frac{6}{25}\,
p_{Y|X}\brak{1|0} = \frac{45}{50}\,
p_{Y|X}\brak{1|2} = \frac{5}{50}
\end{align}
The desired probability is the probability that a slip drawn at random is marked other than Rs 1,
\begin{align}
&=1-p_X\brak{0}\\
&= p_X(1) + p_X(2)
\end{align}
Using Bayes theorem,
\begin{align}
&= p_Y\brak{0} \times \pr{Y=0 | X=1} + p_Y\brak{1} \times \pr{Y=1|X=2}\\
&=\frac{1}{3} \times \frac{6}{25} + \frac{2}{3} \times \frac{5}{50}\\
&=\frac{11}{75}
\end{align}

\newpage

%\tableofcontents

\bigskip

\renewcommand{\thefigure}{\theenumi}
\renewcommand{\thetable}{\theenumi}
%\renewcommand{\theequation}{\theenumi}

%\begin{abstract}
%%\boldmath
%In this letter, an algorithm for evaluating the exact analytical bit error rate  (BER)  for the piecewise linear (PL) combiner for  multiple relays is presented. Previous results were available only for upto three relays. The algorithm is unique in the sense that  the actual mathematical expressions, that are prohibitively large, need not be explicitly obtained. The diversity gain due to multiple relays is shown through plots of the analytical BER, well supported by simulations. 
%
%\end{abstract}
% IEEEtran.cls defaults to using nonbold math in the Abstract.
% This preserves the distinction between vectors and scalars. However,
% if the journal you are submitting to favors bold math in the abstract,
% then you can use LaTeX's standard command \boldmath at the very start
% of the abstract to achieve this. Many IEEE journals frown on math
% in the abstract anyway.

% Note that keywords are not normally used for peerreview papers.
%\begin{IEEEkeywords}
%Cooperative diversity, decode and forward, piecewise linear
%\end{IEEEkeywords}



% For peer review papers, you can put extra information on the cover
% page as needed:
% \ifCLASSOPTIONpeerreview
% \begin{center} \bfseries EDICS Category: 3-BBND \end{center}
% \fi
%
% For peerreview papers, this IEEEtran command inserts a page break and
% creates the second title. It will be ignored for other modes.
%\IEEEpeerreviewmaketitle




	\item Five cards—the ten, jack, queen, king and ace of diamonds, are well-shuffled with their face downwards. One card is then picked up at random.
\begin{enumerate}
\item
What is the probability that the card is the queen? 
\item
If the queen is drawn and put aside, what is the probability that the second card picked up is (a) an ace? (b) a queen?\\
\end{enumerate}
\solution
		%\begin{enumerate}[label=\thesection.\arabic*,ref=\thesection.\theenumi]
	\item One card is drawn from a well-shuffled deck of 52 cards. Find the probability of getting
\begin{enumerate}
\item A king of red colour 
\item A face card 
\item A red face card
\item The jack of hearts
\item A spade
\item The queen of diamonds

\end{enumerate}
\solution
		%\input{ncert/10/15/1/14/main.tex}
	\item Five cards—the ten, jack, queen, king and ace of diamonds, are well-shuffled with their face downwards. One card is then picked up at random.
\begin{enumerate}
\item
What is the probability that the card is the queen? 
\item
If the queen is drawn and put aside, what is the probability that the second card picked up is (a) an ace? (b) a queen?\\
\end{enumerate}
\solution
		%\input{ncert/10/15/1/15/defs.tex}
	\item A bag contains $5$ red balls and some blue balls. If the probability of drawing a blue ball is double that if a red ball, determine the number of blue balls in the bag. 
		\\
\solution
		%\input{ncert/10/15/2/3/defs.tex}
	\item A card is selected from a pack of 52 cards.
 \begin{enumerate}[label=(\alph*)] 
                 \item How many points are there in the sample space?
                 \item Calculate the probability that the card is an ace of spades.
                 \item Calculate the probability that the card is (i) an ace and (ii) black card.
 \end{enumerate}
\solution
		%\input{ncert/11/16/3/4/main.tex}
\item Four cards are drawn from a well-shuffled deck of 52 cards. What is the probability of obtaining 3 diamonds and one spade.
\\
\solution
		%\input{ncert/11/16/4/2/defs.tex}
\item In a certain lottery 10,000 tickets are sold and ten equal prizes are awarded. What is the probability of not getting a prize if you buy (a) one ticket (b) two tickets (c) 10 tickets ?	
\\
\solution
		%\input{ncert/11/16/4/4/defs.tex}
		%
\item 
Out of 100 students, two sections of 40 and 60 are formed. If you and your friend are among the 100 students, what is the probability that
\begin{enumerate}
\item you both enter the same section?
\item you both enter the different sections?
\end{enumerate}
\solution
		%\input{ncert/11/16/4/5/defs.tex}
	\item 
The number lock of a suitcase has 4 wheels each labelled with ten digits i.e. from 0 to 9.The lock opens with a sequence of four digits with no repeats.What is the probability of a person getting the right sequence to open the suitcase.
\\
\solution
		%\input{ncert/11/16/4/10/defs.tex}
		%
\item 
Two cards are drawn at random and without replacement from a pack of 52 playing cards. Find the probability that both the cards are black.
\\
\solution
		%\input{ncert/12/13/2/2/defs.tex}
		\item A box of oranges is inspected by examining three randomly selected oranges drawn without replacement. If all the three oranges are good, the box is approved for sale, otherwise, it is rejected. Find the probability that a box containing 15 oranges out of which 12 are good and 3 are bad ones will be approved for sale.
		\label{ncert/12/13/2/3/defs.tex}
		\item Two balls are drawn at random with replacement from a box containing 10 black and 8 red balls. Find the probability that
		\label{ncert/12/13/2/12}
\begin{enumerate}
\item both balls are red.
\item first ball is black and second is red.
\item one of them is black and other is red.
\end{enumerate}

\item In a hostel, 60\% of the students read Hindi newspaper, 40\% read English newspaper and 20\% read both Hindi and English newspapers. A student is selected at random.
		\label{ncert/12/13/2/15}
\begin{enumerate}
\item Find the probability that she reads neither Hindi nor English newspapers.
\item If she reads Hindi newspaper, find the probability that she reads English newspaper.
\item If she reads English newspaper, find the probability that she reads Hindi newspaper.\\
\end{enumerate}
\item The probability of obtaining an even prime number on each die, when a pair of dice is rolled is 
\begin{enumerate}
    \item $0$ 
    
    \item $\frac{1}{3}$ 
    
    \item $\frac{1}{12}$ 
    
    \item $\frac{1}{36}$ 
\end{enumerate}
\solution
		%\input{ncert/12/13/2/17/defs.tex}
	\item A bag contains 4 red and 4 black balls, another bag contains 2 red and 6 black balls. One of the two bags is selected at random and a ball is drawn from the bag which is found to be red. Find the probability that the ball is drawn from the first bag.
\\
\solution
		%\input{ncert/12/13/3/2/main.tex}
  \item
  Cards with numbers 2 to 101 are placed in a box. A card is selected at random.Find the probability that the card has
\begin{enumerate}[label=(\roman*)]
	\item an even number 
	\item a square number
\end{enumerate}
\solution
%\input{exemplar/10/13/3/32/main.tex}
\item
The king, queen and jack of clubs are removed from a deck of 52 playing cards and then well shuffled. Now one card is drawn at random from the remaining cards.  Determine the probability that the card is
\begin{enumerate}[label=(\roman*)]
\item a club
\item 10 of hearts
\end{enumerate}
\solution
%\input{exemplar/10/13/3/29/main.tex}
\item A team of medical students doing their internship have to assist during surgeries
at a city hospital. The probabilities of surgeries rated as very complex, complex,
routine, simple or very simple are respectively, 0.15, 0.20, 0.31, 0.26, .08. Find
the probabilities that a particular surgery will be rated
\begin{enumerate}
	\item complex or very complex;
	\item neither very complex nor very simple;
	\item routine or complex
	\item routine or simple
\end{enumerate}
\solution
%\input{exemplar/11/16/3/8(1)/main.tex}
\item A card is selected from a pack of 52 cards.
\begin{enumerate}[label=(\alph*)]
    \item How many points are there in the sample space?
    \item Calculate the probability that the card is an ace of spades.
    \item Calculate the probability that the card is (i) an ace and (ii) black card.
\end{enumerate}
\solution
%\input{exemplar/11/16/3/4/main2.tex}
\item The probability that a non leap year selected at random will contain 53 sundays.
\\
\solution
%\input{exemplar/10/13/1/19/main.tex}
\item One of the four persons John, Rita, Aslam or Gurpreet will be promoted next
month. Consequently the sample space consists of four elementary outcomes
S = {John promoted, Rita promoted, Aslam promoted, Gurpreet promoted}
You are told that the chances of John’s promotion is same as that of Gurpreet,
Rita’s chances of promotion are twice as likely as Johns. Aslam’s chances are
four times that of John.
\begin{enumerate}
	\item Determine
	\begin{enumerate}
		\item P (John promoted)
		\item P (Rita promoted)
		\item P (Aslam promoted)
		\item P (Gurpreet promoted)
	\end{enumerate}
	\item If A = {John promoted or Gurpreet promoted}, find P (A).
\end{enumerate}
\solution
%\input{exemplar/11/16/3/10/main.tex}
\item A card is drawn from a deck of 52 cards. Find the probability of getting a king or a heart or a red card.\\
\solution
%\input{exemplar/11/16/3/15/main.tex}
\item The probability that a student will pass his examination is 0.73, the probability of
the student getting a compartment is 0.13, and the probability that the student will
either pass or get compartment is 0.96. State True or False.\\
\solution
%\input{exemplar/11/16/3/31/main.tex}
\item A card is selected from a pack of 52 cards\\
\begin{enumerate}[label=(\alph*)]
\item How many points are there in the sample space?
\item Calculate the probability that the cards is an ace of spades.
\item Calculate the probability that the card is (i) an ace (ii)black card.\\
\end{enumerate}
%\input{ncert/11/16/3/4_1/Prob_4.tex}
\item In a non-leap year, the probability of having 53 tuesdays or 53 wednesdays is\\
\solution
%\input{exemplar/11/16/3/18/main.tex}
\item There are 1000 sealed envelopes in a box, 10 of them contain a cash prize of
Rs 100 each, 100 of them contain a cash prize of Rs 50 each and 200 of them
contain a cash prize of Rs 10 each and rest do not contain any cash prize. If they
are well shuffled and an envelope is picked up out, what is the probability that it
contains no cash prize?\\
\solution
%\input{exemplar/10/13/3/34/main.tex}
\item 
A die is thrown and a card is selected at random from a deck of 52 playing cards. The probability of getting an even number on the die and a spade card.\\
\solution
%\input{exemplar/12/13/3/78/main.tex}
\item
If 4-digit numbers greater than 5,000 are randomly formed from the digits 0, 1, 3, 5, and 7, what is the probability of forming a number divisible by 5 when:
\begin{enumerate}
    \item The digits are repeated?
    \item The repetition of digits is not allowed?
\end{enumerate}
\solution
%\input{ncert/11/16/4/9/main.tex}
\item Consider the probability space $\brak{\Omega, \mathcal{G}, P}$ where $\Omega = [0,2]$ and $\mathcal{G} = \cbrak{\phi, \Omega, [0,1], (1,2]}$. Let $X$ and $Y$ be two functions on $\Omega$ defined as
\begin{align*}
    X(\omega) = 
    \begin{cases}
        1 & \text{if }\omega \in [0, 1]\\
        2 & \text{if }\omega \in (1, 2]
    \end{cases}
\end{align*}
and
\begin{align*}
    Y(\omega) = 
    \begin{cases}
        2 & \text{if }\omega \in [0, 1.5]\\
        3 & \text{if }\omega \in (1.5, 2].
    \end{cases}
\end{align*}
Then which one of the following statements is true?
\begin{enumerate}
    \item [(A)] $X$ is a random variable with respect to $\mathcal{G}$, but $Y$ is not a random variable with respect to $\mathcal{G}$.
    \item [(B)] $Y$ is a random variable with respect to $\mathcal{G}$, but $X$ is not a random variable with respect to $\mathcal{G}$.
    \item [(C)] Neither $X$ nor $Y$ is a random variable with respect to $\mathcal{G}$.
    \item [(D)] Both $X$ and $Y$ are random variables with respect to $\mathcal{G}$.
\end{enumerate} \hfill (GATE ST 2023)\\
\solution
%\input{gate/ST/2023/14/main.tex}
	\item  A die is loaded in such a way that each odd number is twice as likely to occur as
each even number. Find $P(G)$, where $G$ is the event that a number greater than
3 occurs on a single roll of the die.
\\
\solution
		%\input{exemplar/11/16/3/5/main.tex}
	\item All the jacks, queens and kings are removed from a deck of 52 playing cards. The remaining cards are well shuffled and then one card is drawn at random. Giving ace a value 1 similar value for other cards, find the probability that the card has a value 
		\begin{enumerate}
			\item 7
			\item greater than 7
			\item less than 7
		\end{enumerate}
		%\input{exemplar/10/13/3/30/main.tex}
  \item A Lot consists of 48 mobile phones of which 42 are good, 3 have only minor defects and 3 have major defects.Varnika will buy a phone if it is good but the trader will only buy a mobile if it has no major defects. One phone is selected at random from the lot. What is the probability that it is
\begin{enumerate}
	\item acceptable to Varnika?
            \item acceptable to the trader?
\end{enumerate}
\solution
	%\input{exemplar/10/13/3/40/main.tex}
 \item A student says that if you throw a die, it will show up 1 or not 1. Therefore, the probability of getting 1 and the probability of getting 'not 1' each is equal to $\frac{1}{2}$. Is this correct? Give reasons.\\
 \solution
        %\input{exemplar/10/13/2/9/main.tex}
   \item Four candidates A, B, C, D have ap-
plied for the assignment to coach a school cricket
team. If A is twice as likely to be selected as B, and
B and C are given about the same chance of being
selected, while C is twice as likely to be selected
as D, what are the probabilities that
\begin{enumerate}
\item C will be selected?
\item A will not be selected?
\end{enumerate}
	%\input{exemplar/11/16/3/9/main.tex}
 \item A bag contain 24 balls of which $x$ balls are red, $2x$ are white and $3x$ are blue. A ball is selected at random, What is the probability that it is
\begin{enumerate}[label=\alph*)]
\item not red ?
\item white ?
\end{enumerate}
%\input{exemplar/10/13/3/41/main.tex}
If the letters of the word ASSASSINATION are arranged at random. Find the Probability that
\begin{enumerate}[label=(\alph*)]
\item Four $S's$ come consecutively in the word
\item Two  $I's$ and two $N's$ come together
\item All $A's$ are not coming together
\item No two $A's$ are coming together
\end{enumerate}
%\input{exemplar/11/16/3/14/main.tex}
	\item One urn contains two black balls (labelled B1 and B2) and one white ball. A
	second urn contains one black ball and two white balls (labelled W1 and W2).
	Suppose the following experiment is performed. One of the two urns is chosen
	at random. Next a ball is randomly chosen from the urn. Then a second ball is
	chosen at random from the same urn without replacing the first ball.
	
	\begin{enumerate}
	\item What is the probability that two black balls are chosen?
	
	\item What is the probability that two balls of opposite colour are chosen?
	\end{enumerate}
	\solution
	%\input{exemplar/11/16/3/12/main1.tex}
\end{enumerate}

	\item A bag contains $5$ red balls and some blue balls. If the probability of drawing a blue ball is double that if a red ball, determine the number of blue balls in the bag. 
		\\
\solution
		%\begin{enumerate}[label=\thesection.\arabic*,ref=\thesection.\theenumi]
	\item One card is drawn from a well-shuffled deck of 52 cards. Find the probability of getting
\begin{enumerate}
\item A king of red colour 
\item A face card 
\item A red face card
\item The jack of hearts
\item A spade
\item The queen of diamonds

\end{enumerate}
\solution
		%\input{ncert/10/15/1/14/main.tex}
	\item Five cards—the ten, jack, queen, king and ace of diamonds, are well-shuffled with their face downwards. One card is then picked up at random.
\begin{enumerate}
\item
What is the probability that the card is the queen? 
\item
If the queen is drawn and put aside, what is the probability that the second card picked up is (a) an ace? (b) a queen?\\
\end{enumerate}
\solution
		%\input{ncert/10/15/1/15/defs.tex}
	\item A bag contains $5$ red balls and some blue balls. If the probability of drawing a blue ball is double that if a red ball, determine the number of blue balls in the bag. 
		\\
\solution
		%\input{ncert/10/15/2/3/defs.tex}
	\item A card is selected from a pack of 52 cards.
 \begin{enumerate}[label=(\alph*)] 
                 \item How many points are there in the sample space?
                 \item Calculate the probability that the card is an ace of spades.
                 \item Calculate the probability that the card is (i) an ace and (ii) black card.
 \end{enumerate}
\solution
		%\input{ncert/11/16/3/4/main.tex}
\item Four cards are drawn from a well-shuffled deck of 52 cards. What is the probability of obtaining 3 diamonds and one spade.
\\
\solution
		%\input{ncert/11/16/4/2/defs.tex}
\item In a certain lottery 10,000 tickets are sold and ten equal prizes are awarded. What is the probability of not getting a prize if you buy (a) one ticket (b) two tickets (c) 10 tickets ?	
\\
\solution
		%\input{ncert/11/16/4/4/defs.tex}
		%
\item 
Out of 100 students, two sections of 40 and 60 are formed. If you and your friend are among the 100 students, what is the probability that
\begin{enumerate}
\item you both enter the same section?
\item you both enter the different sections?
\end{enumerate}
\solution
		%\input{ncert/11/16/4/5/defs.tex}
	\item 
The number lock of a suitcase has 4 wheels each labelled with ten digits i.e. from 0 to 9.The lock opens with a sequence of four digits with no repeats.What is the probability of a person getting the right sequence to open the suitcase.
\\
\solution
		%\input{ncert/11/16/4/10/defs.tex}
		%
\item 
Two cards are drawn at random and without replacement from a pack of 52 playing cards. Find the probability that both the cards are black.
\\
\solution
		%\input{ncert/12/13/2/2/defs.tex}
		\item A box of oranges is inspected by examining three randomly selected oranges drawn without replacement. If all the three oranges are good, the box is approved for sale, otherwise, it is rejected. Find the probability that a box containing 15 oranges out of which 12 are good and 3 are bad ones will be approved for sale.
		\label{ncert/12/13/2/3/defs.tex}
		\item Two balls are drawn at random with replacement from a box containing 10 black and 8 red balls. Find the probability that
		\label{ncert/12/13/2/12}
\begin{enumerate}
\item both balls are red.
\item first ball is black and second is red.
\item one of them is black and other is red.
\end{enumerate}

\item In a hostel, 60\% of the students read Hindi newspaper, 40\% read English newspaper and 20\% read both Hindi and English newspapers. A student is selected at random.
		\label{ncert/12/13/2/15}
\begin{enumerate}
\item Find the probability that she reads neither Hindi nor English newspapers.
\item If she reads Hindi newspaper, find the probability that she reads English newspaper.
\item If she reads English newspaper, find the probability that she reads Hindi newspaper.\\
\end{enumerate}
\item The probability of obtaining an even prime number on each die, when a pair of dice is rolled is 
\begin{enumerate}
    \item $0$ 
    
    \item $\frac{1}{3}$ 
    
    \item $\frac{1}{12}$ 
    
    \item $\frac{1}{36}$ 
\end{enumerate}
\solution
		%\input{ncert/12/13/2/17/defs.tex}
	\item A bag contains 4 red and 4 black balls, another bag contains 2 red and 6 black balls. One of the two bags is selected at random and a ball is drawn from the bag which is found to be red. Find the probability that the ball is drawn from the first bag.
\\
\solution
		%\input{ncert/12/13/3/2/main.tex}
  \item
  Cards with numbers 2 to 101 are placed in a box. A card is selected at random.Find the probability that the card has
\begin{enumerate}[label=(\roman*)]
	\item an even number 
	\item a square number
\end{enumerate}
\solution
%\input{exemplar/10/13/3/32/main.tex}
\item
The king, queen and jack of clubs are removed from a deck of 52 playing cards and then well shuffled. Now one card is drawn at random from the remaining cards.  Determine the probability that the card is
\begin{enumerate}[label=(\roman*)]
\item a club
\item 10 of hearts
\end{enumerate}
\solution
%\input{exemplar/10/13/3/29/main.tex}
\item A team of medical students doing their internship have to assist during surgeries
at a city hospital. The probabilities of surgeries rated as very complex, complex,
routine, simple or very simple are respectively, 0.15, 0.20, 0.31, 0.26, .08. Find
the probabilities that a particular surgery will be rated
\begin{enumerate}
	\item complex or very complex;
	\item neither very complex nor very simple;
	\item routine or complex
	\item routine or simple
\end{enumerate}
\solution
%\input{exemplar/11/16/3/8(1)/main.tex}
\item A card is selected from a pack of 52 cards.
\begin{enumerate}[label=(\alph*)]
    \item How many points are there in the sample space?
    \item Calculate the probability that the card is an ace of spades.
    \item Calculate the probability that the card is (i) an ace and (ii) black card.
\end{enumerate}
\solution
%\input{exemplar/11/16/3/4/main2.tex}
\item The probability that a non leap year selected at random will contain 53 sundays.
\\
\solution
%\input{exemplar/10/13/1/19/main.tex}
\item One of the four persons John, Rita, Aslam or Gurpreet will be promoted next
month. Consequently the sample space consists of four elementary outcomes
S = {John promoted, Rita promoted, Aslam promoted, Gurpreet promoted}
You are told that the chances of John’s promotion is same as that of Gurpreet,
Rita’s chances of promotion are twice as likely as Johns. Aslam’s chances are
four times that of John.
\begin{enumerate}
	\item Determine
	\begin{enumerate}
		\item P (John promoted)
		\item P (Rita promoted)
		\item P (Aslam promoted)
		\item P (Gurpreet promoted)
	\end{enumerate}
	\item If A = {John promoted or Gurpreet promoted}, find P (A).
\end{enumerate}
\solution
%\input{exemplar/11/16/3/10/main.tex}
\item A card is drawn from a deck of 52 cards. Find the probability of getting a king or a heart or a red card.\\
\solution
%\input{exemplar/11/16/3/15/main.tex}
\item The probability that a student will pass his examination is 0.73, the probability of
the student getting a compartment is 0.13, and the probability that the student will
either pass or get compartment is 0.96. State True or False.\\
\solution
%\input{exemplar/11/16/3/31/main.tex}
\item A card is selected from a pack of 52 cards\\
\begin{enumerate}[label=(\alph*)]
\item How many points are there in the sample space?
\item Calculate the probability that the cards is an ace of spades.
\item Calculate the probability that the card is (i) an ace (ii)black card.\\
\end{enumerate}
%\input{ncert/11/16/3/4_1/Prob_4.tex}
\item In a non-leap year, the probability of having 53 tuesdays or 53 wednesdays is\\
\solution
%\input{exemplar/11/16/3/18/main.tex}
\item There are 1000 sealed envelopes in a box, 10 of them contain a cash prize of
Rs 100 each, 100 of them contain a cash prize of Rs 50 each and 200 of them
contain a cash prize of Rs 10 each and rest do not contain any cash prize. If they
are well shuffled and an envelope is picked up out, what is the probability that it
contains no cash prize?\\
\solution
%\input{exemplar/10/13/3/34/main.tex}
\item 
A die is thrown and a card is selected at random from a deck of 52 playing cards. The probability of getting an even number on the die and a spade card.\\
\solution
%\input{exemplar/12/13/3/78/main.tex}
\item
If 4-digit numbers greater than 5,000 are randomly formed from the digits 0, 1, 3, 5, and 7, what is the probability of forming a number divisible by 5 when:
\begin{enumerate}
    \item The digits are repeated?
    \item The repetition of digits is not allowed?
\end{enumerate}
\solution
%\input{ncert/11/16/4/9/main.tex}
\item Consider the probability space $\brak{\Omega, \mathcal{G}, P}$ where $\Omega = [0,2]$ and $\mathcal{G} = \cbrak{\phi, \Omega, [0,1], (1,2]}$. Let $X$ and $Y$ be two functions on $\Omega$ defined as
\begin{align*}
    X(\omega) = 
    \begin{cases}
        1 & \text{if }\omega \in [0, 1]\\
        2 & \text{if }\omega \in (1, 2]
    \end{cases}
\end{align*}
and
\begin{align*}
    Y(\omega) = 
    \begin{cases}
        2 & \text{if }\omega \in [0, 1.5]\\
        3 & \text{if }\omega \in (1.5, 2].
    \end{cases}
\end{align*}
Then which one of the following statements is true?
\begin{enumerate}
    \item [(A)] $X$ is a random variable with respect to $\mathcal{G}$, but $Y$ is not a random variable with respect to $\mathcal{G}$.
    \item [(B)] $Y$ is a random variable with respect to $\mathcal{G}$, but $X$ is not a random variable with respect to $\mathcal{G}$.
    \item [(C)] Neither $X$ nor $Y$ is a random variable with respect to $\mathcal{G}$.
    \item [(D)] Both $X$ and $Y$ are random variables with respect to $\mathcal{G}$.
\end{enumerate} \hfill (GATE ST 2023)\\
\solution
%\input{gate/ST/2023/14/main.tex}
	\item  A die is loaded in such a way that each odd number is twice as likely to occur as
each even number. Find $P(G)$, where $G$ is the event that a number greater than
3 occurs on a single roll of the die.
\\
\solution
		%\input{exemplar/11/16/3/5/main.tex}
	\item All the jacks, queens and kings are removed from a deck of 52 playing cards. The remaining cards are well shuffled and then one card is drawn at random. Giving ace a value 1 similar value for other cards, find the probability that the card has a value 
		\begin{enumerate}
			\item 7
			\item greater than 7
			\item less than 7
		\end{enumerate}
		%\input{exemplar/10/13/3/30/main.tex}
  \item A Lot consists of 48 mobile phones of which 42 are good, 3 have only minor defects and 3 have major defects.Varnika will buy a phone if it is good but the trader will only buy a mobile if it has no major defects. One phone is selected at random from the lot. What is the probability that it is
\begin{enumerate}
	\item acceptable to Varnika?
            \item acceptable to the trader?
\end{enumerate}
\solution
	%\input{exemplar/10/13/3/40/main.tex}
 \item A student says that if you throw a die, it will show up 1 or not 1. Therefore, the probability of getting 1 and the probability of getting 'not 1' each is equal to $\frac{1}{2}$. Is this correct? Give reasons.\\
 \solution
        %\input{exemplar/10/13/2/9/main.tex}
   \item Four candidates A, B, C, D have ap-
plied for the assignment to coach a school cricket
team. If A is twice as likely to be selected as B, and
B and C are given about the same chance of being
selected, while C is twice as likely to be selected
as D, what are the probabilities that
\begin{enumerate}
\item C will be selected?
\item A will not be selected?
\end{enumerate}
	%\input{exemplar/11/16/3/9/main.tex}
 \item A bag contain 24 balls of which $x$ balls are red, $2x$ are white and $3x$ are blue. A ball is selected at random, What is the probability that it is
\begin{enumerate}[label=\alph*)]
\item not red ?
\item white ?
\end{enumerate}
%\input{exemplar/10/13/3/41/main.tex}
If the letters of the word ASSASSINATION are arranged at random. Find the Probability that
\begin{enumerate}[label=(\alph*)]
\item Four $S's$ come consecutively in the word
\item Two  $I's$ and two $N's$ come together
\item All $A's$ are not coming together
\item No two $A's$ are coming together
\end{enumerate}
%\input{exemplar/11/16/3/14/main.tex}
	\item One urn contains two black balls (labelled B1 and B2) and one white ball. A
	second urn contains one black ball and two white balls (labelled W1 and W2).
	Suppose the following experiment is performed. One of the two urns is chosen
	at random. Next a ball is randomly chosen from the urn. Then a second ball is
	chosen at random from the same urn without replacing the first ball.
	
	\begin{enumerate}
	\item What is the probability that two black balls are chosen?
	
	\item What is the probability that two balls of opposite colour are chosen?
	\end{enumerate}
	\solution
	%\input{exemplar/11/16/3/12/main1.tex}
\end{enumerate}

	\item A card is selected from a pack of 52 cards.
 \begin{enumerate}[label=(\alph*)] 
                 \item How many points are there in the sample space?
                 \item Calculate the probability that the card is an ace of spades.
                 \item Calculate the probability that the card is (i) an ace and (ii) black card.
 \end{enumerate}
\solution
		%\begin{table}[H]
	\centering
\begin{tabular}{|c|c|c|}
\hline
Random variable &Value &Definition\\ \hline
\multirow{3}{*}{X} &0 &Slips of Rs 1\\
&1 &Slips of Rs 5\\
&2 &Slips of Rs 13\\ \hline
\multirow{2}{*}{Y} &0 &Box A\\
&1 &Box B\\\hline
\end{tabular}
\caption{}
\label{tab:Distribution}
\end{table}
See \tabref{tab:Distribution}.
\begin{align}
p_{Y}\brak{k}= \begin{cases} 
      \frac{1}{3} & {k=0} \\
      \frac{2}{3 }& {k=1} 
   \end{cases}
   \\
p_{Y|X}\brak{0|0} = \frac{19}{25}\, 
p_{Y|X}\brak{0|1} = \frac{6}{25}\,
p_{Y|X}\brak{1|0} = \frac{45}{50}\,
p_{Y|X}\brak{1|2} = \frac{5}{50}
\end{align}
The desired probability is the probability that a slip drawn at random is marked other than Rs 1,
\begin{align}
&=1-p_X\brak{0}\\
&= p_X(1) + p_X(2)
\end{align}
Using Bayes theorem,
\begin{align}
&= p_Y\brak{0} \times \pr{Y=0 | X=1} + p_Y\brak{1} \times \pr{Y=1|X=2}\\
&=\frac{1}{3} \times \frac{6}{25} + \frac{2}{3} \times \frac{5}{50}\\
&=\frac{11}{75}
\end{align}

\newpage

%\tableofcontents

\bigskip

\renewcommand{\thefigure}{\theenumi}
\renewcommand{\thetable}{\theenumi}
%\renewcommand{\theequation}{\theenumi}

%\begin{abstract}
%%\boldmath
%In this letter, an algorithm for evaluating the exact analytical bit error rate  (BER)  for the piecewise linear (PL) combiner for  multiple relays is presented. Previous results were available only for upto three relays. The algorithm is unique in the sense that  the actual mathematical expressions, that are prohibitively large, need not be explicitly obtained. The diversity gain due to multiple relays is shown through plots of the analytical BER, well supported by simulations. 
%
%\end{abstract}
% IEEEtran.cls defaults to using nonbold math in the Abstract.
% This preserves the distinction between vectors and scalars. However,
% if the journal you are submitting to favors bold math in the abstract,
% then you can use LaTeX's standard command \boldmath at the very start
% of the abstract to achieve this. Many IEEE journals frown on math
% in the abstract anyway.

% Note that keywords are not normally used for peerreview papers.
%\begin{IEEEkeywords}
%Cooperative diversity, decode and forward, piecewise linear
%\end{IEEEkeywords}



% For peer review papers, you can put extra information on the cover
% page as needed:
% \ifCLASSOPTIONpeerreview
% \begin{center} \bfseries EDICS Category: 3-BBND \end{center}
% \fi
%
% For peerreview papers, this IEEEtran command inserts a page break and
% creates the second title. It will be ignored for other modes.
%\IEEEpeerreviewmaketitle




\item Four cards are drawn from a well-shuffled deck of 52 cards. What is the probability of obtaining 3 diamonds and one spade.
\\
\solution
		%\begin{enumerate}[label=\thesection.\arabic*,ref=\thesection.\theenumi]
	\item One card is drawn from a well-shuffled deck of 52 cards. Find the probability of getting
\begin{enumerate}
\item A king of red colour 
\item A face card 
\item A red face card
\item The jack of hearts
\item A spade
\item The queen of diamonds

\end{enumerate}
\solution
		%\input{ncert/10/15/1/14/main.tex}
	\item Five cards—the ten, jack, queen, king and ace of diamonds, are well-shuffled with their face downwards. One card is then picked up at random.
\begin{enumerate}
\item
What is the probability that the card is the queen? 
\item
If the queen is drawn and put aside, what is the probability that the second card picked up is (a) an ace? (b) a queen?\\
\end{enumerate}
\solution
		%\input{ncert/10/15/1/15/defs.tex}
	\item A bag contains $5$ red balls and some blue balls. If the probability of drawing a blue ball is double that if a red ball, determine the number of blue balls in the bag. 
		\\
\solution
		%\input{ncert/10/15/2/3/defs.tex}
	\item A card is selected from a pack of 52 cards.
 \begin{enumerate}[label=(\alph*)] 
                 \item How many points are there in the sample space?
                 \item Calculate the probability that the card is an ace of spades.
                 \item Calculate the probability that the card is (i) an ace and (ii) black card.
 \end{enumerate}
\solution
		%\input{ncert/11/16/3/4/main.tex}
\item Four cards are drawn from a well-shuffled deck of 52 cards. What is the probability of obtaining 3 diamonds and one spade.
\\
\solution
		%\input{ncert/11/16/4/2/defs.tex}
\item In a certain lottery 10,000 tickets are sold and ten equal prizes are awarded. What is the probability of not getting a prize if you buy (a) one ticket (b) two tickets (c) 10 tickets ?	
\\
\solution
		%\input{ncert/11/16/4/4/defs.tex}
		%
\item 
Out of 100 students, two sections of 40 and 60 are formed. If you and your friend are among the 100 students, what is the probability that
\begin{enumerate}
\item you both enter the same section?
\item you both enter the different sections?
\end{enumerate}
\solution
		%\input{ncert/11/16/4/5/defs.tex}
	\item 
The number lock of a suitcase has 4 wheels each labelled with ten digits i.e. from 0 to 9.The lock opens with a sequence of four digits with no repeats.What is the probability of a person getting the right sequence to open the suitcase.
\\
\solution
		%\input{ncert/11/16/4/10/defs.tex}
		%
\item 
Two cards are drawn at random and without replacement from a pack of 52 playing cards. Find the probability that both the cards are black.
\\
\solution
		%\input{ncert/12/13/2/2/defs.tex}
		\item A box of oranges is inspected by examining three randomly selected oranges drawn without replacement. If all the three oranges are good, the box is approved for sale, otherwise, it is rejected. Find the probability that a box containing 15 oranges out of which 12 are good and 3 are bad ones will be approved for sale.
		\label{ncert/12/13/2/3/defs.tex}
		\item Two balls are drawn at random with replacement from a box containing 10 black and 8 red balls. Find the probability that
		\label{ncert/12/13/2/12}
\begin{enumerate}
\item both balls are red.
\item first ball is black and second is red.
\item one of them is black and other is red.
\end{enumerate}

\item In a hostel, 60\% of the students read Hindi newspaper, 40\% read English newspaper and 20\% read both Hindi and English newspapers. A student is selected at random.
		\label{ncert/12/13/2/15}
\begin{enumerate}
\item Find the probability that she reads neither Hindi nor English newspapers.
\item If she reads Hindi newspaper, find the probability that she reads English newspaper.
\item If she reads English newspaper, find the probability that she reads Hindi newspaper.\\
\end{enumerate}
\item The probability of obtaining an even prime number on each die, when a pair of dice is rolled is 
\begin{enumerate}
    \item $0$ 
    
    \item $\frac{1}{3}$ 
    
    \item $\frac{1}{12}$ 
    
    \item $\frac{1}{36}$ 
\end{enumerate}
\solution
		%\input{ncert/12/13/2/17/defs.tex}
	\item A bag contains 4 red and 4 black balls, another bag contains 2 red and 6 black balls. One of the two bags is selected at random and a ball is drawn from the bag which is found to be red. Find the probability that the ball is drawn from the first bag.
\\
\solution
		%\input{ncert/12/13/3/2/main.tex}
  \item
  Cards with numbers 2 to 101 are placed in a box. A card is selected at random.Find the probability that the card has
\begin{enumerate}[label=(\roman*)]
	\item an even number 
	\item a square number
\end{enumerate}
\solution
%\input{exemplar/10/13/3/32/main.tex}
\item
The king, queen and jack of clubs are removed from a deck of 52 playing cards and then well shuffled. Now one card is drawn at random from the remaining cards.  Determine the probability that the card is
\begin{enumerate}[label=(\roman*)]
\item a club
\item 10 of hearts
\end{enumerate}
\solution
%\input{exemplar/10/13/3/29/main.tex}
\item A team of medical students doing their internship have to assist during surgeries
at a city hospital. The probabilities of surgeries rated as very complex, complex,
routine, simple or very simple are respectively, 0.15, 0.20, 0.31, 0.26, .08. Find
the probabilities that a particular surgery will be rated
\begin{enumerate}
	\item complex or very complex;
	\item neither very complex nor very simple;
	\item routine or complex
	\item routine or simple
\end{enumerate}
\solution
%\input{exemplar/11/16/3/8(1)/main.tex}
\item A card is selected from a pack of 52 cards.
\begin{enumerate}[label=(\alph*)]
    \item How many points are there in the sample space?
    \item Calculate the probability that the card is an ace of spades.
    \item Calculate the probability that the card is (i) an ace and (ii) black card.
\end{enumerate}
\solution
%\input{exemplar/11/16/3/4/main2.tex}
\item The probability that a non leap year selected at random will contain 53 sundays.
\\
\solution
%\input{exemplar/10/13/1/19/main.tex}
\item One of the four persons John, Rita, Aslam or Gurpreet will be promoted next
month. Consequently the sample space consists of four elementary outcomes
S = {John promoted, Rita promoted, Aslam promoted, Gurpreet promoted}
You are told that the chances of John’s promotion is same as that of Gurpreet,
Rita’s chances of promotion are twice as likely as Johns. Aslam’s chances are
four times that of John.
\begin{enumerate}
	\item Determine
	\begin{enumerate}
		\item P (John promoted)
		\item P (Rita promoted)
		\item P (Aslam promoted)
		\item P (Gurpreet promoted)
	\end{enumerate}
	\item If A = {John promoted or Gurpreet promoted}, find P (A).
\end{enumerate}
\solution
%\input{exemplar/11/16/3/10/main.tex}
\item A card is drawn from a deck of 52 cards. Find the probability of getting a king or a heart or a red card.\\
\solution
%\input{exemplar/11/16/3/15/main.tex}
\item The probability that a student will pass his examination is 0.73, the probability of
the student getting a compartment is 0.13, and the probability that the student will
either pass or get compartment is 0.96. State True or False.\\
\solution
%\input{exemplar/11/16/3/31/main.tex}
\item A card is selected from a pack of 52 cards\\
\begin{enumerate}[label=(\alph*)]
\item How many points are there in the sample space?
\item Calculate the probability that the cards is an ace of spades.
\item Calculate the probability that the card is (i) an ace (ii)black card.\\
\end{enumerate}
%\input{ncert/11/16/3/4_1/Prob_4.tex}
\item In a non-leap year, the probability of having 53 tuesdays or 53 wednesdays is\\
\solution
%\input{exemplar/11/16/3/18/main.tex}
\item There are 1000 sealed envelopes in a box, 10 of them contain a cash prize of
Rs 100 each, 100 of them contain a cash prize of Rs 50 each and 200 of them
contain a cash prize of Rs 10 each and rest do not contain any cash prize. If they
are well shuffled and an envelope is picked up out, what is the probability that it
contains no cash prize?\\
\solution
%\input{exemplar/10/13/3/34/main.tex}
\item 
A die is thrown and a card is selected at random from a deck of 52 playing cards. The probability of getting an even number on the die and a spade card.\\
\solution
%\input{exemplar/12/13/3/78/main.tex}
\item
If 4-digit numbers greater than 5,000 are randomly formed from the digits 0, 1, 3, 5, and 7, what is the probability of forming a number divisible by 5 when:
\begin{enumerate}
    \item The digits are repeated?
    \item The repetition of digits is not allowed?
\end{enumerate}
\solution
%\input{ncert/11/16/4/9/main.tex}
\item Consider the probability space $\brak{\Omega, \mathcal{G}, P}$ where $\Omega = [0,2]$ and $\mathcal{G} = \cbrak{\phi, \Omega, [0,1], (1,2]}$. Let $X$ and $Y$ be two functions on $\Omega$ defined as
\begin{align*}
    X(\omega) = 
    \begin{cases}
        1 & \text{if }\omega \in [0, 1]\\
        2 & \text{if }\omega \in (1, 2]
    \end{cases}
\end{align*}
and
\begin{align*}
    Y(\omega) = 
    \begin{cases}
        2 & \text{if }\omega \in [0, 1.5]\\
        3 & \text{if }\omega \in (1.5, 2].
    \end{cases}
\end{align*}
Then which one of the following statements is true?
\begin{enumerate}
    \item [(A)] $X$ is a random variable with respect to $\mathcal{G}$, but $Y$ is not a random variable with respect to $\mathcal{G}$.
    \item [(B)] $Y$ is a random variable with respect to $\mathcal{G}$, but $X$ is not a random variable with respect to $\mathcal{G}$.
    \item [(C)] Neither $X$ nor $Y$ is a random variable with respect to $\mathcal{G}$.
    \item [(D)] Both $X$ and $Y$ are random variables with respect to $\mathcal{G}$.
\end{enumerate} \hfill (GATE ST 2023)\\
\solution
%\input{gate/ST/2023/14/main.tex}
	\item  A die is loaded in such a way that each odd number is twice as likely to occur as
each even number. Find $P(G)$, where $G$ is the event that a number greater than
3 occurs on a single roll of the die.
\\
\solution
		%\input{exemplar/11/16/3/5/main.tex}
	\item All the jacks, queens and kings are removed from a deck of 52 playing cards. The remaining cards are well shuffled and then one card is drawn at random. Giving ace a value 1 similar value for other cards, find the probability that the card has a value 
		\begin{enumerate}
			\item 7
			\item greater than 7
			\item less than 7
		\end{enumerate}
		%\input{exemplar/10/13/3/30/main.tex}
  \item A Lot consists of 48 mobile phones of which 42 are good, 3 have only minor defects and 3 have major defects.Varnika will buy a phone if it is good but the trader will only buy a mobile if it has no major defects. One phone is selected at random from the lot. What is the probability that it is
\begin{enumerate}
	\item acceptable to Varnika?
            \item acceptable to the trader?
\end{enumerate}
\solution
	%\input{exemplar/10/13/3/40/main.tex}
 \item A student says that if you throw a die, it will show up 1 or not 1. Therefore, the probability of getting 1 and the probability of getting 'not 1' each is equal to $\frac{1}{2}$. Is this correct? Give reasons.\\
 \solution
        %\input{exemplar/10/13/2/9/main.tex}
   \item Four candidates A, B, C, D have ap-
plied for the assignment to coach a school cricket
team. If A is twice as likely to be selected as B, and
B and C are given about the same chance of being
selected, while C is twice as likely to be selected
as D, what are the probabilities that
\begin{enumerate}
\item C will be selected?
\item A will not be selected?
\end{enumerate}
	%\input{exemplar/11/16/3/9/main.tex}
 \item A bag contain 24 balls of which $x$ balls are red, $2x$ are white and $3x$ are blue. A ball is selected at random, What is the probability that it is
\begin{enumerate}[label=\alph*)]
\item not red ?
\item white ?
\end{enumerate}
%\input{exemplar/10/13/3/41/main.tex}
If the letters of the word ASSASSINATION are arranged at random. Find the Probability that
\begin{enumerate}[label=(\alph*)]
\item Four $S's$ come consecutively in the word
\item Two  $I's$ and two $N's$ come together
\item All $A's$ are not coming together
\item No two $A's$ are coming together
\end{enumerate}
%\input{exemplar/11/16/3/14/main.tex}
	\item One urn contains two black balls (labelled B1 and B2) and one white ball. A
	second urn contains one black ball and two white balls (labelled W1 and W2).
	Suppose the following experiment is performed. One of the two urns is chosen
	at random. Next a ball is randomly chosen from the urn. Then a second ball is
	chosen at random from the same urn without replacing the first ball.
	
	\begin{enumerate}
	\item What is the probability that two black balls are chosen?
	
	\item What is the probability that two balls of opposite colour are chosen?
	\end{enumerate}
	\solution
	%\input{exemplar/11/16/3/12/main1.tex}
\end{enumerate}

\item In a certain lottery 10,000 tickets are sold and ten equal prizes are awarded. What is the probability of not getting a prize if you buy (a) one ticket (b) two tickets (c) 10 tickets ?	
\\
\solution
		%\begin{enumerate}[label=\thesection.\arabic*,ref=\thesection.\theenumi]
	\item One card is drawn from a well-shuffled deck of 52 cards. Find the probability of getting
\begin{enumerate}
\item A king of red colour 
\item A face card 
\item A red face card
\item The jack of hearts
\item A spade
\item The queen of diamonds

\end{enumerate}
\solution
		%\input{ncert/10/15/1/14/main.tex}
	\item Five cards—the ten, jack, queen, king and ace of diamonds, are well-shuffled with their face downwards. One card is then picked up at random.
\begin{enumerate}
\item
What is the probability that the card is the queen? 
\item
If the queen is drawn and put aside, what is the probability that the second card picked up is (a) an ace? (b) a queen?\\
\end{enumerate}
\solution
		%\input{ncert/10/15/1/15/defs.tex}
	\item A bag contains $5$ red balls and some blue balls. If the probability of drawing a blue ball is double that if a red ball, determine the number of blue balls in the bag. 
		\\
\solution
		%\input{ncert/10/15/2/3/defs.tex}
	\item A card is selected from a pack of 52 cards.
 \begin{enumerate}[label=(\alph*)] 
                 \item How many points are there in the sample space?
                 \item Calculate the probability that the card is an ace of spades.
                 \item Calculate the probability that the card is (i) an ace and (ii) black card.
 \end{enumerate}
\solution
		%\input{ncert/11/16/3/4/main.tex}
\item Four cards are drawn from a well-shuffled deck of 52 cards. What is the probability of obtaining 3 diamonds and one spade.
\\
\solution
		%\input{ncert/11/16/4/2/defs.tex}
\item In a certain lottery 10,000 tickets are sold and ten equal prizes are awarded. What is the probability of not getting a prize if you buy (a) one ticket (b) two tickets (c) 10 tickets ?	
\\
\solution
		%\input{ncert/11/16/4/4/defs.tex}
		%
\item 
Out of 100 students, two sections of 40 and 60 are formed. If you and your friend are among the 100 students, what is the probability that
\begin{enumerate}
\item you both enter the same section?
\item you both enter the different sections?
\end{enumerate}
\solution
		%\input{ncert/11/16/4/5/defs.tex}
	\item 
The number lock of a suitcase has 4 wheels each labelled with ten digits i.e. from 0 to 9.The lock opens with a sequence of four digits with no repeats.What is the probability of a person getting the right sequence to open the suitcase.
\\
\solution
		%\input{ncert/11/16/4/10/defs.tex}
		%
\item 
Two cards are drawn at random and without replacement from a pack of 52 playing cards. Find the probability that both the cards are black.
\\
\solution
		%\input{ncert/12/13/2/2/defs.tex}
		\item A box of oranges is inspected by examining three randomly selected oranges drawn without replacement. If all the three oranges are good, the box is approved for sale, otherwise, it is rejected. Find the probability that a box containing 15 oranges out of which 12 are good and 3 are bad ones will be approved for sale.
		\label{ncert/12/13/2/3/defs.tex}
		\item Two balls are drawn at random with replacement from a box containing 10 black and 8 red balls. Find the probability that
		\label{ncert/12/13/2/12}
\begin{enumerate}
\item both balls are red.
\item first ball is black and second is red.
\item one of them is black and other is red.
\end{enumerate}

\item In a hostel, 60\% of the students read Hindi newspaper, 40\% read English newspaper and 20\% read both Hindi and English newspapers. A student is selected at random.
		\label{ncert/12/13/2/15}
\begin{enumerate}
\item Find the probability that she reads neither Hindi nor English newspapers.
\item If she reads Hindi newspaper, find the probability that she reads English newspaper.
\item If she reads English newspaper, find the probability that she reads Hindi newspaper.\\
\end{enumerate}
\item The probability of obtaining an even prime number on each die, when a pair of dice is rolled is 
\begin{enumerate}
    \item $0$ 
    
    \item $\frac{1}{3}$ 
    
    \item $\frac{1}{12}$ 
    
    \item $\frac{1}{36}$ 
\end{enumerate}
\solution
		%\input{ncert/12/13/2/17/defs.tex}
	\item A bag contains 4 red and 4 black balls, another bag contains 2 red and 6 black balls. One of the two bags is selected at random and a ball is drawn from the bag which is found to be red. Find the probability that the ball is drawn from the first bag.
\\
\solution
		%\input{ncert/12/13/3/2/main.tex}
  \item
  Cards with numbers 2 to 101 are placed in a box. A card is selected at random.Find the probability that the card has
\begin{enumerate}[label=(\roman*)]
	\item an even number 
	\item a square number
\end{enumerate}
\solution
%\input{exemplar/10/13/3/32/main.tex}
\item
The king, queen and jack of clubs are removed from a deck of 52 playing cards and then well shuffled. Now one card is drawn at random from the remaining cards.  Determine the probability that the card is
\begin{enumerate}[label=(\roman*)]
\item a club
\item 10 of hearts
\end{enumerate}
\solution
%\input{exemplar/10/13/3/29/main.tex}
\item A team of medical students doing their internship have to assist during surgeries
at a city hospital. The probabilities of surgeries rated as very complex, complex,
routine, simple or very simple are respectively, 0.15, 0.20, 0.31, 0.26, .08. Find
the probabilities that a particular surgery will be rated
\begin{enumerate}
	\item complex or very complex;
	\item neither very complex nor very simple;
	\item routine or complex
	\item routine or simple
\end{enumerate}
\solution
%\input{exemplar/11/16/3/8(1)/main.tex}
\item A card is selected from a pack of 52 cards.
\begin{enumerate}[label=(\alph*)]
    \item How many points are there in the sample space?
    \item Calculate the probability that the card is an ace of spades.
    \item Calculate the probability that the card is (i) an ace and (ii) black card.
\end{enumerate}
\solution
%\input{exemplar/11/16/3/4/main2.tex}
\item The probability that a non leap year selected at random will contain 53 sundays.
\\
\solution
%\input{exemplar/10/13/1/19/main.tex}
\item One of the four persons John, Rita, Aslam or Gurpreet will be promoted next
month. Consequently the sample space consists of four elementary outcomes
S = {John promoted, Rita promoted, Aslam promoted, Gurpreet promoted}
You are told that the chances of John’s promotion is same as that of Gurpreet,
Rita’s chances of promotion are twice as likely as Johns. Aslam’s chances are
four times that of John.
\begin{enumerate}
	\item Determine
	\begin{enumerate}
		\item P (John promoted)
		\item P (Rita promoted)
		\item P (Aslam promoted)
		\item P (Gurpreet promoted)
	\end{enumerate}
	\item If A = {John promoted or Gurpreet promoted}, find P (A).
\end{enumerate}
\solution
%\input{exemplar/11/16/3/10/main.tex}
\item A card is drawn from a deck of 52 cards. Find the probability of getting a king or a heart or a red card.\\
\solution
%\input{exemplar/11/16/3/15/main.tex}
\item The probability that a student will pass his examination is 0.73, the probability of
the student getting a compartment is 0.13, and the probability that the student will
either pass or get compartment is 0.96. State True or False.\\
\solution
%\input{exemplar/11/16/3/31/main.tex}
\item A card is selected from a pack of 52 cards\\
\begin{enumerate}[label=(\alph*)]
\item How many points are there in the sample space?
\item Calculate the probability that the cards is an ace of spades.
\item Calculate the probability that the card is (i) an ace (ii)black card.\\
\end{enumerate}
%\input{ncert/11/16/3/4_1/Prob_4.tex}
\item In a non-leap year, the probability of having 53 tuesdays or 53 wednesdays is\\
\solution
%\input{exemplar/11/16/3/18/main.tex}
\item There are 1000 sealed envelopes in a box, 10 of them contain a cash prize of
Rs 100 each, 100 of them contain a cash prize of Rs 50 each and 200 of them
contain a cash prize of Rs 10 each and rest do not contain any cash prize. If they
are well shuffled and an envelope is picked up out, what is the probability that it
contains no cash prize?\\
\solution
%\input{exemplar/10/13/3/34/main.tex}
\item 
A die is thrown and a card is selected at random from a deck of 52 playing cards. The probability of getting an even number on the die and a spade card.\\
\solution
%\input{exemplar/12/13/3/78/main.tex}
\item
If 4-digit numbers greater than 5,000 are randomly formed from the digits 0, 1, 3, 5, and 7, what is the probability of forming a number divisible by 5 when:
\begin{enumerate}
    \item The digits are repeated?
    \item The repetition of digits is not allowed?
\end{enumerate}
\solution
%\input{ncert/11/16/4/9/main.tex}
\item Consider the probability space $\brak{\Omega, \mathcal{G}, P}$ where $\Omega = [0,2]$ and $\mathcal{G} = \cbrak{\phi, \Omega, [0,1], (1,2]}$. Let $X$ and $Y$ be two functions on $\Omega$ defined as
\begin{align*}
    X(\omega) = 
    \begin{cases}
        1 & \text{if }\omega \in [0, 1]\\
        2 & \text{if }\omega \in (1, 2]
    \end{cases}
\end{align*}
and
\begin{align*}
    Y(\omega) = 
    \begin{cases}
        2 & \text{if }\omega \in [0, 1.5]\\
        3 & \text{if }\omega \in (1.5, 2].
    \end{cases}
\end{align*}
Then which one of the following statements is true?
\begin{enumerate}
    \item [(A)] $X$ is a random variable with respect to $\mathcal{G}$, but $Y$ is not a random variable with respect to $\mathcal{G}$.
    \item [(B)] $Y$ is a random variable with respect to $\mathcal{G}$, but $X$ is not a random variable with respect to $\mathcal{G}$.
    \item [(C)] Neither $X$ nor $Y$ is a random variable with respect to $\mathcal{G}$.
    \item [(D)] Both $X$ and $Y$ are random variables with respect to $\mathcal{G}$.
\end{enumerate} \hfill (GATE ST 2023)\\
\solution
%\input{gate/ST/2023/14/main.tex}
	\item  A die is loaded in such a way that each odd number is twice as likely to occur as
each even number. Find $P(G)$, where $G$ is the event that a number greater than
3 occurs on a single roll of the die.
\\
\solution
		%\input{exemplar/11/16/3/5/main.tex}
	\item All the jacks, queens and kings are removed from a deck of 52 playing cards. The remaining cards are well shuffled and then one card is drawn at random. Giving ace a value 1 similar value for other cards, find the probability that the card has a value 
		\begin{enumerate}
			\item 7
			\item greater than 7
			\item less than 7
		\end{enumerate}
		%\input{exemplar/10/13/3/30/main.tex}
  \item A Lot consists of 48 mobile phones of which 42 are good, 3 have only minor defects and 3 have major defects.Varnika will buy a phone if it is good but the trader will only buy a mobile if it has no major defects. One phone is selected at random from the lot. What is the probability that it is
\begin{enumerate}
	\item acceptable to Varnika?
            \item acceptable to the trader?
\end{enumerate}
\solution
	%\input{exemplar/10/13/3/40/main.tex}
 \item A student says that if you throw a die, it will show up 1 or not 1. Therefore, the probability of getting 1 and the probability of getting 'not 1' each is equal to $\frac{1}{2}$. Is this correct? Give reasons.\\
 \solution
        %\input{exemplar/10/13/2/9/main.tex}
   \item Four candidates A, B, C, D have ap-
plied for the assignment to coach a school cricket
team. If A is twice as likely to be selected as B, and
B and C are given about the same chance of being
selected, while C is twice as likely to be selected
as D, what are the probabilities that
\begin{enumerate}
\item C will be selected?
\item A will not be selected?
\end{enumerate}
	%\input{exemplar/11/16/3/9/main.tex}
 \item A bag contain 24 balls of which $x$ balls are red, $2x$ are white and $3x$ are blue. A ball is selected at random, What is the probability that it is
\begin{enumerate}[label=\alph*)]
\item not red ?
\item white ?
\end{enumerate}
%\input{exemplar/10/13/3/41/main.tex}
If the letters of the word ASSASSINATION are arranged at random. Find the Probability that
\begin{enumerate}[label=(\alph*)]
\item Four $S's$ come consecutively in the word
\item Two  $I's$ and two $N's$ come together
\item All $A's$ are not coming together
\item No two $A's$ are coming together
\end{enumerate}
%\input{exemplar/11/16/3/14/main.tex}
	\item One urn contains two black balls (labelled B1 and B2) and one white ball. A
	second urn contains one black ball and two white balls (labelled W1 and W2).
	Suppose the following experiment is performed. One of the two urns is chosen
	at random. Next a ball is randomly chosen from the urn. Then a second ball is
	chosen at random from the same urn without replacing the first ball.
	
	\begin{enumerate}
	\item What is the probability that two black balls are chosen?
	
	\item What is the probability that two balls of opposite colour are chosen?
	\end{enumerate}
	\solution
	%\input{exemplar/11/16/3/12/main1.tex}
\end{enumerate}

		%
\item 
Out of 100 students, two sections of 40 and 60 are formed. If you and your friend are among the 100 students, what is the probability that
\begin{enumerate}
\item you both enter the same section?
\item you both enter the different sections?
\end{enumerate}
\solution
		%\begin{enumerate}[label=\thesection.\arabic*,ref=\thesection.\theenumi]
	\item One card is drawn from a well-shuffled deck of 52 cards. Find the probability of getting
\begin{enumerate}
\item A king of red colour 
\item A face card 
\item A red face card
\item The jack of hearts
\item A spade
\item The queen of diamonds

\end{enumerate}
\solution
		%\input{ncert/10/15/1/14/main.tex}
	\item Five cards—the ten, jack, queen, king and ace of diamonds, are well-shuffled with their face downwards. One card is then picked up at random.
\begin{enumerate}
\item
What is the probability that the card is the queen? 
\item
If the queen is drawn and put aside, what is the probability that the second card picked up is (a) an ace? (b) a queen?\\
\end{enumerate}
\solution
		%\input{ncert/10/15/1/15/defs.tex}
	\item A bag contains $5$ red balls and some blue balls. If the probability of drawing a blue ball is double that if a red ball, determine the number of blue balls in the bag. 
		\\
\solution
		%\input{ncert/10/15/2/3/defs.tex}
	\item A card is selected from a pack of 52 cards.
 \begin{enumerate}[label=(\alph*)] 
                 \item How many points are there in the sample space?
                 \item Calculate the probability that the card is an ace of spades.
                 \item Calculate the probability that the card is (i) an ace and (ii) black card.
 \end{enumerate}
\solution
		%\input{ncert/11/16/3/4/main.tex}
\item Four cards are drawn from a well-shuffled deck of 52 cards. What is the probability of obtaining 3 diamonds and one spade.
\\
\solution
		%\input{ncert/11/16/4/2/defs.tex}
\item In a certain lottery 10,000 tickets are sold and ten equal prizes are awarded. What is the probability of not getting a prize if you buy (a) one ticket (b) two tickets (c) 10 tickets ?	
\\
\solution
		%\input{ncert/11/16/4/4/defs.tex}
		%
\item 
Out of 100 students, two sections of 40 and 60 are formed. If you and your friend are among the 100 students, what is the probability that
\begin{enumerate}
\item you both enter the same section?
\item you both enter the different sections?
\end{enumerate}
\solution
		%\input{ncert/11/16/4/5/defs.tex}
	\item 
The number lock of a suitcase has 4 wheels each labelled with ten digits i.e. from 0 to 9.The lock opens with a sequence of four digits with no repeats.What is the probability of a person getting the right sequence to open the suitcase.
\\
\solution
		%\input{ncert/11/16/4/10/defs.tex}
		%
\item 
Two cards are drawn at random and without replacement from a pack of 52 playing cards. Find the probability that both the cards are black.
\\
\solution
		%\input{ncert/12/13/2/2/defs.tex}
		\item A box of oranges is inspected by examining three randomly selected oranges drawn without replacement. If all the three oranges are good, the box is approved for sale, otherwise, it is rejected. Find the probability that a box containing 15 oranges out of which 12 are good and 3 are bad ones will be approved for sale.
		\label{ncert/12/13/2/3/defs.tex}
		\item Two balls are drawn at random with replacement from a box containing 10 black and 8 red balls. Find the probability that
		\label{ncert/12/13/2/12}
\begin{enumerate}
\item both balls are red.
\item first ball is black and second is red.
\item one of them is black and other is red.
\end{enumerate}

\item In a hostel, 60\% of the students read Hindi newspaper, 40\% read English newspaper and 20\% read both Hindi and English newspapers. A student is selected at random.
		\label{ncert/12/13/2/15}
\begin{enumerate}
\item Find the probability that she reads neither Hindi nor English newspapers.
\item If she reads Hindi newspaper, find the probability that she reads English newspaper.
\item If she reads English newspaper, find the probability that she reads Hindi newspaper.\\
\end{enumerate}
\item The probability of obtaining an even prime number on each die, when a pair of dice is rolled is 
\begin{enumerate}
    \item $0$ 
    
    \item $\frac{1}{3}$ 
    
    \item $\frac{1}{12}$ 
    
    \item $\frac{1}{36}$ 
\end{enumerate}
\solution
		%\input{ncert/12/13/2/17/defs.tex}
	\item A bag contains 4 red and 4 black balls, another bag contains 2 red and 6 black balls. One of the two bags is selected at random and a ball is drawn from the bag which is found to be red. Find the probability that the ball is drawn from the first bag.
\\
\solution
		%\input{ncert/12/13/3/2/main.tex}
  \item
  Cards with numbers 2 to 101 are placed in a box. A card is selected at random.Find the probability that the card has
\begin{enumerate}[label=(\roman*)]
	\item an even number 
	\item a square number
\end{enumerate}
\solution
%\input{exemplar/10/13/3/32/main.tex}
\item
The king, queen and jack of clubs are removed from a deck of 52 playing cards and then well shuffled. Now one card is drawn at random from the remaining cards.  Determine the probability that the card is
\begin{enumerate}[label=(\roman*)]
\item a club
\item 10 of hearts
\end{enumerate}
\solution
%\input{exemplar/10/13/3/29/main.tex}
\item A team of medical students doing their internship have to assist during surgeries
at a city hospital. The probabilities of surgeries rated as very complex, complex,
routine, simple or very simple are respectively, 0.15, 0.20, 0.31, 0.26, .08. Find
the probabilities that a particular surgery will be rated
\begin{enumerate}
	\item complex or very complex;
	\item neither very complex nor very simple;
	\item routine or complex
	\item routine or simple
\end{enumerate}
\solution
%\input{exemplar/11/16/3/8(1)/main.tex}
\item A card is selected from a pack of 52 cards.
\begin{enumerate}[label=(\alph*)]
    \item How many points are there in the sample space?
    \item Calculate the probability that the card is an ace of spades.
    \item Calculate the probability that the card is (i) an ace and (ii) black card.
\end{enumerate}
\solution
%\input{exemplar/11/16/3/4/main2.tex}
\item The probability that a non leap year selected at random will contain 53 sundays.
\\
\solution
%\input{exemplar/10/13/1/19/main.tex}
\item One of the four persons John, Rita, Aslam or Gurpreet will be promoted next
month. Consequently the sample space consists of four elementary outcomes
S = {John promoted, Rita promoted, Aslam promoted, Gurpreet promoted}
You are told that the chances of John’s promotion is same as that of Gurpreet,
Rita’s chances of promotion are twice as likely as Johns. Aslam’s chances are
four times that of John.
\begin{enumerate}
	\item Determine
	\begin{enumerate}
		\item P (John promoted)
		\item P (Rita promoted)
		\item P (Aslam promoted)
		\item P (Gurpreet promoted)
	\end{enumerate}
	\item If A = {John promoted or Gurpreet promoted}, find P (A).
\end{enumerate}
\solution
%\input{exemplar/11/16/3/10/main.tex}
\item A card is drawn from a deck of 52 cards. Find the probability of getting a king or a heart or a red card.\\
\solution
%\input{exemplar/11/16/3/15/main.tex}
\item The probability that a student will pass his examination is 0.73, the probability of
the student getting a compartment is 0.13, and the probability that the student will
either pass or get compartment is 0.96. State True or False.\\
\solution
%\input{exemplar/11/16/3/31/main.tex}
\item A card is selected from a pack of 52 cards\\
\begin{enumerate}[label=(\alph*)]
\item How many points are there in the sample space?
\item Calculate the probability that the cards is an ace of spades.
\item Calculate the probability that the card is (i) an ace (ii)black card.\\
\end{enumerate}
%\input{ncert/11/16/3/4_1/Prob_4.tex}
\item In a non-leap year, the probability of having 53 tuesdays or 53 wednesdays is\\
\solution
%\input{exemplar/11/16/3/18/main.tex}
\item There are 1000 sealed envelopes in a box, 10 of them contain a cash prize of
Rs 100 each, 100 of them contain a cash prize of Rs 50 each and 200 of them
contain a cash prize of Rs 10 each and rest do not contain any cash prize. If they
are well shuffled and an envelope is picked up out, what is the probability that it
contains no cash prize?\\
\solution
%\input{exemplar/10/13/3/34/main.tex}
\item 
A die is thrown and a card is selected at random from a deck of 52 playing cards. The probability of getting an even number on the die and a spade card.\\
\solution
%\input{exemplar/12/13/3/78/main.tex}
\item
If 4-digit numbers greater than 5,000 are randomly formed from the digits 0, 1, 3, 5, and 7, what is the probability of forming a number divisible by 5 when:
\begin{enumerate}
    \item The digits are repeated?
    \item The repetition of digits is not allowed?
\end{enumerate}
\solution
%\input{ncert/11/16/4/9/main.tex}
\item Consider the probability space $\brak{\Omega, \mathcal{G}, P}$ where $\Omega = [0,2]$ and $\mathcal{G} = \cbrak{\phi, \Omega, [0,1], (1,2]}$. Let $X$ and $Y$ be two functions on $\Omega$ defined as
\begin{align*}
    X(\omega) = 
    \begin{cases}
        1 & \text{if }\omega \in [0, 1]\\
        2 & \text{if }\omega \in (1, 2]
    \end{cases}
\end{align*}
and
\begin{align*}
    Y(\omega) = 
    \begin{cases}
        2 & \text{if }\omega \in [0, 1.5]\\
        3 & \text{if }\omega \in (1.5, 2].
    \end{cases}
\end{align*}
Then which one of the following statements is true?
\begin{enumerate}
    \item [(A)] $X$ is a random variable with respect to $\mathcal{G}$, but $Y$ is not a random variable with respect to $\mathcal{G}$.
    \item [(B)] $Y$ is a random variable with respect to $\mathcal{G}$, but $X$ is not a random variable with respect to $\mathcal{G}$.
    \item [(C)] Neither $X$ nor $Y$ is a random variable with respect to $\mathcal{G}$.
    \item [(D)] Both $X$ and $Y$ are random variables with respect to $\mathcal{G}$.
\end{enumerate} \hfill (GATE ST 2023)\\
\solution
%\input{gate/ST/2023/14/main.tex}
	\item  A die is loaded in such a way that each odd number is twice as likely to occur as
each even number. Find $P(G)$, where $G$ is the event that a number greater than
3 occurs on a single roll of the die.
\\
\solution
		%\input{exemplar/11/16/3/5/main.tex}
	\item All the jacks, queens and kings are removed from a deck of 52 playing cards. The remaining cards are well shuffled and then one card is drawn at random. Giving ace a value 1 similar value for other cards, find the probability that the card has a value 
		\begin{enumerate}
			\item 7
			\item greater than 7
			\item less than 7
		\end{enumerate}
		%\input{exemplar/10/13/3/30/main.tex}
  \item A Lot consists of 48 mobile phones of which 42 are good, 3 have only minor defects and 3 have major defects.Varnika will buy a phone if it is good but the trader will only buy a mobile if it has no major defects. One phone is selected at random from the lot. What is the probability that it is
\begin{enumerate}
	\item acceptable to Varnika?
            \item acceptable to the trader?
\end{enumerate}
\solution
	%\input{exemplar/10/13/3/40/main.tex}
 \item A student says that if you throw a die, it will show up 1 or not 1. Therefore, the probability of getting 1 and the probability of getting 'not 1' each is equal to $\frac{1}{2}$. Is this correct? Give reasons.\\
 \solution
        %\input{exemplar/10/13/2/9/main.tex}
   \item Four candidates A, B, C, D have ap-
plied for the assignment to coach a school cricket
team. If A is twice as likely to be selected as B, and
B and C are given about the same chance of being
selected, while C is twice as likely to be selected
as D, what are the probabilities that
\begin{enumerate}
\item C will be selected?
\item A will not be selected?
\end{enumerate}
	%\input{exemplar/11/16/3/9/main.tex}
 \item A bag contain 24 balls of which $x$ balls are red, $2x$ are white and $3x$ are blue. A ball is selected at random, What is the probability that it is
\begin{enumerate}[label=\alph*)]
\item not red ?
\item white ?
\end{enumerate}
%\input{exemplar/10/13/3/41/main.tex}
If the letters of the word ASSASSINATION are arranged at random. Find the Probability that
\begin{enumerate}[label=(\alph*)]
\item Four $S's$ come consecutively in the word
\item Two  $I's$ and two $N's$ come together
\item All $A's$ are not coming together
\item No two $A's$ are coming together
\end{enumerate}
%\input{exemplar/11/16/3/14/main.tex}
	\item One urn contains two black balls (labelled B1 and B2) and one white ball. A
	second urn contains one black ball and two white balls (labelled W1 and W2).
	Suppose the following experiment is performed. One of the two urns is chosen
	at random. Next a ball is randomly chosen from the urn. Then a second ball is
	chosen at random from the same urn without replacing the first ball.
	
	\begin{enumerate}
	\item What is the probability that two black balls are chosen?
	
	\item What is the probability that two balls of opposite colour are chosen?
	\end{enumerate}
	\solution
	%\input{exemplar/11/16/3/12/main1.tex}
\end{enumerate}

	\item 
The number lock of a suitcase has 4 wheels each labelled with ten digits i.e. from 0 to 9.The lock opens with a sequence of four digits with no repeats.What is the probability of a person getting the right sequence to open the suitcase.
\\
\solution
		%\begin{enumerate}[label=\thesection.\arabic*,ref=\thesection.\theenumi]
	\item One card is drawn from a well-shuffled deck of 52 cards. Find the probability of getting
\begin{enumerate}
\item A king of red colour 
\item A face card 
\item A red face card
\item The jack of hearts
\item A spade
\item The queen of diamonds

\end{enumerate}
\solution
		%\input{ncert/10/15/1/14/main.tex}
	\item Five cards—the ten, jack, queen, king and ace of diamonds, are well-shuffled with their face downwards. One card is then picked up at random.
\begin{enumerate}
\item
What is the probability that the card is the queen? 
\item
If the queen is drawn and put aside, what is the probability that the second card picked up is (a) an ace? (b) a queen?\\
\end{enumerate}
\solution
		%\input{ncert/10/15/1/15/defs.tex}
	\item A bag contains $5$ red balls and some blue balls. If the probability of drawing a blue ball is double that if a red ball, determine the number of blue balls in the bag. 
		\\
\solution
		%\input{ncert/10/15/2/3/defs.tex}
	\item A card is selected from a pack of 52 cards.
 \begin{enumerate}[label=(\alph*)] 
                 \item How many points are there in the sample space?
                 \item Calculate the probability that the card is an ace of spades.
                 \item Calculate the probability that the card is (i) an ace and (ii) black card.
 \end{enumerate}
\solution
		%\input{ncert/11/16/3/4/main.tex}
\item Four cards are drawn from a well-shuffled deck of 52 cards. What is the probability of obtaining 3 diamonds and one spade.
\\
\solution
		%\input{ncert/11/16/4/2/defs.tex}
\item In a certain lottery 10,000 tickets are sold and ten equal prizes are awarded. What is the probability of not getting a prize if you buy (a) one ticket (b) two tickets (c) 10 tickets ?	
\\
\solution
		%\input{ncert/11/16/4/4/defs.tex}
		%
\item 
Out of 100 students, two sections of 40 and 60 are formed. If you and your friend are among the 100 students, what is the probability that
\begin{enumerate}
\item you both enter the same section?
\item you both enter the different sections?
\end{enumerate}
\solution
		%\input{ncert/11/16/4/5/defs.tex}
	\item 
The number lock of a suitcase has 4 wheels each labelled with ten digits i.e. from 0 to 9.The lock opens with a sequence of four digits with no repeats.What is the probability of a person getting the right sequence to open the suitcase.
\\
\solution
		%\input{ncert/11/16/4/10/defs.tex}
		%
\item 
Two cards are drawn at random and without replacement from a pack of 52 playing cards. Find the probability that both the cards are black.
\\
\solution
		%\input{ncert/12/13/2/2/defs.tex}
		\item A box of oranges is inspected by examining three randomly selected oranges drawn without replacement. If all the three oranges are good, the box is approved for sale, otherwise, it is rejected. Find the probability that a box containing 15 oranges out of which 12 are good and 3 are bad ones will be approved for sale.
		\label{ncert/12/13/2/3/defs.tex}
		\item Two balls are drawn at random with replacement from a box containing 10 black and 8 red balls. Find the probability that
		\label{ncert/12/13/2/12}
\begin{enumerate}
\item both balls are red.
\item first ball is black and second is red.
\item one of them is black and other is red.
\end{enumerate}

\item In a hostel, 60\% of the students read Hindi newspaper, 40\% read English newspaper and 20\% read both Hindi and English newspapers. A student is selected at random.
		\label{ncert/12/13/2/15}
\begin{enumerate}
\item Find the probability that she reads neither Hindi nor English newspapers.
\item If she reads Hindi newspaper, find the probability that she reads English newspaper.
\item If she reads English newspaper, find the probability that she reads Hindi newspaper.\\
\end{enumerate}
\item The probability of obtaining an even prime number on each die, when a pair of dice is rolled is 
\begin{enumerate}
    \item $0$ 
    
    \item $\frac{1}{3}$ 
    
    \item $\frac{1}{12}$ 
    
    \item $\frac{1}{36}$ 
\end{enumerate}
\solution
		%\input{ncert/12/13/2/17/defs.tex}
	\item A bag contains 4 red and 4 black balls, another bag contains 2 red and 6 black balls. One of the two bags is selected at random and a ball is drawn from the bag which is found to be red. Find the probability that the ball is drawn from the first bag.
\\
\solution
		%\input{ncert/12/13/3/2/main.tex}
  \item
  Cards with numbers 2 to 101 are placed in a box. A card is selected at random.Find the probability that the card has
\begin{enumerate}[label=(\roman*)]
	\item an even number 
	\item a square number
\end{enumerate}
\solution
%\input{exemplar/10/13/3/32/main.tex}
\item
The king, queen and jack of clubs are removed from a deck of 52 playing cards and then well shuffled. Now one card is drawn at random from the remaining cards.  Determine the probability that the card is
\begin{enumerate}[label=(\roman*)]
\item a club
\item 10 of hearts
\end{enumerate}
\solution
%\input{exemplar/10/13/3/29/main.tex}
\item A team of medical students doing their internship have to assist during surgeries
at a city hospital. The probabilities of surgeries rated as very complex, complex,
routine, simple or very simple are respectively, 0.15, 0.20, 0.31, 0.26, .08. Find
the probabilities that a particular surgery will be rated
\begin{enumerate}
	\item complex or very complex;
	\item neither very complex nor very simple;
	\item routine or complex
	\item routine or simple
\end{enumerate}
\solution
%\input{exemplar/11/16/3/8(1)/main.tex}
\item A card is selected from a pack of 52 cards.
\begin{enumerate}[label=(\alph*)]
    \item How many points are there in the sample space?
    \item Calculate the probability that the card is an ace of spades.
    \item Calculate the probability that the card is (i) an ace and (ii) black card.
\end{enumerate}
\solution
%\input{exemplar/11/16/3/4/main2.tex}
\item The probability that a non leap year selected at random will contain 53 sundays.
\\
\solution
%\input{exemplar/10/13/1/19/main.tex}
\item One of the four persons John, Rita, Aslam or Gurpreet will be promoted next
month. Consequently the sample space consists of four elementary outcomes
S = {John promoted, Rita promoted, Aslam promoted, Gurpreet promoted}
You are told that the chances of John’s promotion is same as that of Gurpreet,
Rita’s chances of promotion are twice as likely as Johns. Aslam’s chances are
four times that of John.
\begin{enumerate}
	\item Determine
	\begin{enumerate}
		\item P (John promoted)
		\item P (Rita promoted)
		\item P (Aslam promoted)
		\item P (Gurpreet promoted)
	\end{enumerate}
	\item If A = {John promoted or Gurpreet promoted}, find P (A).
\end{enumerate}
\solution
%\input{exemplar/11/16/3/10/main.tex}
\item A card is drawn from a deck of 52 cards. Find the probability of getting a king or a heart or a red card.\\
\solution
%\input{exemplar/11/16/3/15/main.tex}
\item The probability that a student will pass his examination is 0.73, the probability of
the student getting a compartment is 0.13, and the probability that the student will
either pass or get compartment is 0.96. State True or False.\\
\solution
%\input{exemplar/11/16/3/31/main.tex}
\item A card is selected from a pack of 52 cards\\
\begin{enumerate}[label=(\alph*)]
\item How many points are there in the sample space?
\item Calculate the probability that the cards is an ace of spades.
\item Calculate the probability that the card is (i) an ace (ii)black card.\\
\end{enumerate}
%\input{ncert/11/16/3/4_1/Prob_4.tex}
\item In a non-leap year, the probability of having 53 tuesdays or 53 wednesdays is\\
\solution
%\input{exemplar/11/16/3/18/main.tex}
\item There are 1000 sealed envelopes in a box, 10 of them contain a cash prize of
Rs 100 each, 100 of them contain a cash prize of Rs 50 each and 200 of them
contain a cash prize of Rs 10 each and rest do not contain any cash prize. If they
are well shuffled and an envelope is picked up out, what is the probability that it
contains no cash prize?\\
\solution
%\input{exemplar/10/13/3/34/main.tex}
\item 
A die is thrown and a card is selected at random from a deck of 52 playing cards. The probability of getting an even number on the die and a spade card.\\
\solution
%\input{exemplar/12/13/3/78/main.tex}
\item
If 4-digit numbers greater than 5,000 are randomly formed from the digits 0, 1, 3, 5, and 7, what is the probability of forming a number divisible by 5 when:
\begin{enumerate}
    \item The digits are repeated?
    \item The repetition of digits is not allowed?
\end{enumerate}
\solution
%\input{ncert/11/16/4/9/main.tex}
\item Consider the probability space $\brak{\Omega, \mathcal{G}, P}$ where $\Omega = [0,2]$ and $\mathcal{G} = \cbrak{\phi, \Omega, [0,1], (1,2]}$. Let $X$ and $Y$ be two functions on $\Omega$ defined as
\begin{align*}
    X(\omega) = 
    \begin{cases}
        1 & \text{if }\omega \in [0, 1]\\
        2 & \text{if }\omega \in (1, 2]
    \end{cases}
\end{align*}
and
\begin{align*}
    Y(\omega) = 
    \begin{cases}
        2 & \text{if }\omega \in [0, 1.5]\\
        3 & \text{if }\omega \in (1.5, 2].
    \end{cases}
\end{align*}
Then which one of the following statements is true?
\begin{enumerate}
    \item [(A)] $X$ is a random variable with respect to $\mathcal{G}$, but $Y$ is not a random variable with respect to $\mathcal{G}$.
    \item [(B)] $Y$ is a random variable with respect to $\mathcal{G}$, but $X$ is not a random variable with respect to $\mathcal{G}$.
    \item [(C)] Neither $X$ nor $Y$ is a random variable with respect to $\mathcal{G}$.
    \item [(D)] Both $X$ and $Y$ are random variables with respect to $\mathcal{G}$.
\end{enumerate} \hfill (GATE ST 2023)\\
\solution
%\input{gate/ST/2023/14/main.tex}
	\item  A die is loaded in such a way that each odd number is twice as likely to occur as
each even number. Find $P(G)$, where $G$ is the event that a number greater than
3 occurs on a single roll of the die.
\\
\solution
		%\input{exemplar/11/16/3/5/main.tex}
	\item All the jacks, queens and kings are removed from a deck of 52 playing cards. The remaining cards are well shuffled and then one card is drawn at random. Giving ace a value 1 similar value for other cards, find the probability that the card has a value 
		\begin{enumerate}
			\item 7
			\item greater than 7
			\item less than 7
		\end{enumerate}
		%\input{exemplar/10/13/3/30/main.tex}
  \item A Lot consists of 48 mobile phones of which 42 are good, 3 have only minor defects and 3 have major defects.Varnika will buy a phone if it is good but the trader will only buy a mobile if it has no major defects. One phone is selected at random from the lot. What is the probability that it is
\begin{enumerate}
	\item acceptable to Varnika?
            \item acceptable to the trader?
\end{enumerate}
\solution
	%\input{exemplar/10/13/3/40/main.tex}
 \item A student says that if you throw a die, it will show up 1 or not 1. Therefore, the probability of getting 1 and the probability of getting 'not 1' each is equal to $\frac{1}{2}$. Is this correct? Give reasons.\\
 \solution
        %\input{exemplar/10/13/2/9/main.tex}
   \item Four candidates A, B, C, D have ap-
plied for the assignment to coach a school cricket
team. If A is twice as likely to be selected as B, and
B and C are given about the same chance of being
selected, while C is twice as likely to be selected
as D, what are the probabilities that
\begin{enumerate}
\item C will be selected?
\item A will not be selected?
\end{enumerate}
	%\input{exemplar/11/16/3/9/main.tex}
 \item A bag contain 24 balls of which $x$ balls are red, $2x$ are white and $3x$ are blue. A ball is selected at random, What is the probability that it is
\begin{enumerate}[label=\alph*)]
\item not red ?
\item white ?
\end{enumerate}
%\input{exemplar/10/13/3/41/main.tex}
If the letters of the word ASSASSINATION are arranged at random. Find the Probability that
\begin{enumerate}[label=(\alph*)]
\item Four $S's$ come consecutively in the word
\item Two  $I's$ and two $N's$ come together
\item All $A's$ are not coming together
\item No two $A's$ are coming together
\end{enumerate}
%\input{exemplar/11/16/3/14/main.tex}
	\item One urn contains two black balls (labelled B1 and B2) and one white ball. A
	second urn contains one black ball and two white balls (labelled W1 and W2).
	Suppose the following experiment is performed. One of the two urns is chosen
	at random. Next a ball is randomly chosen from the urn. Then a second ball is
	chosen at random from the same urn without replacing the first ball.
	
	\begin{enumerate}
	\item What is the probability that two black balls are chosen?
	
	\item What is the probability that two balls of opposite colour are chosen?
	\end{enumerate}
	\solution
	%\input{exemplar/11/16/3/12/main1.tex}
\end{enumerate}

		%
\item 
Two cards are drawn at random and without replacement from a pack of 52 playing cards. Find the probability that both the cards are black.
\\
\solution
		%\begin{enumerate}[label=\thesection.\arabic*,ref=\thesection.\theenumi]
	\item One card is drawn from a well-shuffled deck of 52 cards. Find the probability of getting
\begin{enumerate}
\item A king of red colour 
\item A face card 
\item A red face card
\item The jack of hearts
\item A spade
\item The queen of diamonds

\end{enumerate}
\solution
		%\input{ncert/10/15/1/14/main.tex}
	\item Five cards—the ten, jack, queen, king and ace of diamonds, are well-shuffled with their face downwards. One card is then picked up at random.
\begin{enumerate}
\item
What is the probability that the card is the queen? 
\item
If the queen is drawn and put aside, what is the probability that the second card picked up is (a) an ace? (b) a queen?\\
\end{enumerate}
\solution
		%\input{ncert/10/15/1/15/defs.tex}
	\item A bag contains $5$ red balls and some blue balls. If the probability of drawing a blue ball is double that if a red ball, determine the number of blue balls in the bag. 
		\\
\solution
		%\input{ncert/10/15/2/3/defs.tex}
	\item A card is selected from a pack of 52 cards.
 \begin{enumerate}[label=(\alph*)] 
                 \item How many points are there in the sample space?
                 \item Calculate the probability that the card is an ace of spades.
                 \item Calculate the probability that the card is (i) an ace and (ii) black card.
 \end{enumerate}
\solution
		%\input{ncert/11/16/3/4/main.tex}
\item Four cards are drawn from a well-shuffled deck of 52 cards. What is the probability of obtaining 3 diamonds and one spade.
\\
\solution
		%\input{ncert/11/16/4/2/defs.tex}
\item In a certain lottery 10,000 tickets are sold and ten equal prizes are awarded. What is the probability of not getting a prize if you buy (a) one ticket (b) two tickets (c) 10 tickets ?	
\\
\solution
		%\input{ncert/11/16/4/4/defs.tex}
		%
\item 
Out of 100 students, two sections of 40 and 60 are formed. If you and your friend are among the 100 students, what is the probability that
\begin{enumerate}
\item you both enter the same section?
\item you both enter the different sections?
\end{enumerate}
\solution
		%\input{ncert/11/16/4/5/defs.tex}
	\item 
The number lock of a suitcase has 4 wheels each labelled with ten digits i.e. from 0 to 9.The lock opens with a sequence of four digits with no repeats.What is the probability of a person getting the right sequence to open the suitcase.
\\
\solution
		%\input{ncert/11/16/4/10/defs.tex}
		%
\item 
Two cards are drawn at random and without replacement from a pack of 52 playing cards. Find the probability that both the cards are black.
\\
\solution
		%\input{ncert/12/13/2/2/defs.tex}
		\item A box of oranges is inspected by examining three randomly selected oranges drawn without replacement. If all the three oranges are good, the box is approved for sale, otherwise, it is rejected. Find the probability that a box containing 15 oranges out of which 12 are good and 3 are bad ones will be approved for sale.
		\label{ncert/12/13/2/3/defs.tex}
		\item Two balls are drawn at random with replacement from a box containing 10 black and 8 red balls. Find the probability that
		\label{ncert/12/13/2/12}
\begin{enumerate}
\item both balls are red.
\item first ball is black and second is red.
\item one of them is black and other is red.
\end{enumerate}

\item In a hostel, 60\% of the students read Hindi newspaper, 40\% read English newspaper and 20\% read both Hindi and English newspapers. A student is selected at random.
		\label{ncert/12/13/2/15}
\begin{enumerate}
\item Find the probability that she reads neither Hindi nor English newspapers.
\item If she reads Hindi newspaper, find the probability that she reads English newspaper.
\item If she reads English newspaper, find the probability that she reads Hindi newspaper.\\
\end{enumerate}
\item The probability of obtaining an even prime number on each die, when a pair of dice is rolled is 
\begin{enumerate}
    \item $0$ 
    
    \item $\frac{1}{3}$ 
    
    \item $\frac{1}{12}$ 
    
    \item $\frac{1}{36}$ 
\end{enumerate}
\solution
		%\input{ncert/12/13/2/17/defs.tex}
	\item A bag contains 4 red and 4 black balls, another bag contains 2 red and 6 black balls. One of the two bags is selected at random and a ball is drawn from the bag which is found to be red. Find the probability that the ball is drawn from the first bag.
\\
\solution
		%\input{ncert/12/13/3/2/main.tex}
  \item
  Cards with numbers 2 to 101 are placed in a box. A card is selected at random.Find the probability that the card has
\begin{enumerate}[label=(\roman*)]
	\item an even number 
	\item a square number
\end{enumerate}
\solution
%\input{exemplar/10/13/3/32/main.tex}
\item
The king, queen and jack of clubs are removed from a deck of 52 playing cards and then well shuffled. Now one card is drawn at random from the remaining cards.  Determine the probability that the card is
\begin{enumerate}[label=(\roman*)]
\item a club
\item 10 of hearts
\end{enumerate}
\solution
%\input{exemplar/10/13/3/29/main.tex}
\item A team of medical students doing their internship have to assist during surgeries
at a city hospital. The probabilities of surgeries rated as very complex, complex,
routine, simple or very simple are respectively, 0.15, 0.20, 0.31, 0.26, .08. Find
the probabilities that a particular surgery will be rated
\begin{enumerate}
	\item complex or very complex;
	\item neither very complex nor very simple;
	\item routine or complex
	\item routine or simple
\end{enumerate}
\solution
%\input{exemplar/11/16/3/8(1)/main.tex}
\item A card is selected from a pack of 52 cards.
\begin{enumerate}[label=(\alph*)]
    \item How many points are there in the sample space?
    \item Calculate the probability that the card is an ace of spades.
    \item Calculate the probability that the card is (i) an ace and (ii) black card.
\end{enumerate}
\solution
%\input{exemplar/11/16/3/4/main2.tex}
\item The probability that a non leap year selected at random will contain 53 sundays.
\\
\solution
%\input{exemplar/10/13/1/19/main.tex}
\item One of the four persons John, Rita, Aslam or Gurpreet will be promoted next
month. Consequently the sample space consists of four elementary outcomes
S = {John promoted, Rita promoted, Aslam promoted, Gurpreet promoted}
You are told that the chances of John’s promotion is same as that of Gurpreet,
Rita’s chances of promotion are twice as likely as Johns. Aslam’s chances are
four times that of John.
\begin{enumerate}
	\item Determine
	\begin{enumerate}
		\item P (John promoted)
		\item P (Rita promoted)
		\item P (Aslam promoted)
		\item P (Gurpreet promoted)
	\end{enumerate}
	\item If A = {John promoted or Gurpreet promoted}, find P (A).
\end{enumerate}
\solution
%\input{exemplar/11/16/3/10/main.tex}
\item A card is drawn from a deck of 52 cards. Find the probability of getting a king or a heart or a red card.\\
\solution
%\input{exemplar/11/16/3/15/main.tex}
\item The probability that a student will pass his examination is 0.73, the probability of
the student getting a compartment is 0.13, and the probability that the student will
either pass or get compartment is 0.96. State True or False.\\
\solution
%\input{exemplar/11/16/3/31/main.tex}
\item A card is selected from a pack of 52 cards\\
\begin{enumerate}[label=(\alph*)]
\item How many points are there in the sample space?
\item Calculate the probability that the cards is an ace of spades.
\item Calculate the probability that the card is (i) an ace (ii)black card.\\
\end{enumerate}
%\input{ncert/11/16/3/4_1/Prob_4.tex}
\item In a non-leap year, the probability of having 53 tuesdays or 53 wednesdays is\\
\solution
%\input{exemplar/11/16/3/18/main.tex}
\item There are 1000 sealed envelopes in a box, 10 of them contain a cash prize of
Rs 100 each, 100 of them contain a cash prize of Rs 50 each and 200 of them
contain a cash prize of Rs 10 each and rest do not contain any cash prize. If they
are well shuffled and an envelope is picked up out, what is the probability that it
contains no cash prize?\\
\solution
%\input{exemplar/10/13/3/34/main.tex}
\item 
A die is thrown and a card is selected at random from a deck of 52 playing cards. The probability of getting an even number on the die and a spade card.\\
\solution
%\input{exemplar/12/13/3/78/main.tex}
\item
If 4-digit numbers greater than 5,000 are randomly formed from the digits 0, 1, 3, 5, and 7, what is the probability of forming a number divisible by 5 when:
\begin{enumerate}
    \item The digits are repeated?
    \item The repetition of digits is not allowed?
\end{enumerate}
\solution
%\input{ncert/11/16/4/9/main.tex}
\item Consider the probability space $\brak{\Omega, \mathcal{G}, P}$ where $\Omega = [0,2]$ and $\mathcal{G} = \cbrak{\phi, \Omega, [0,1], (1,2]}$. Let $X$ and $Y$ be two functions on $\Omega$ defined as
\begin{align*}
    X(\omega) = 
    \begin{cases}
        1 & \text{if }\omega \in [0, 1]\\
        2 & \text{if }\omega \in (1, 2]
    \end{cases}
\end{align*}
and
\begin{align*}
    Y(\omega) = 
    \begin{cases}
        2 & \text{if }\omega \in [0, 1.5]\\
        3 & \text{if }\omega \in (1.5, 2].
    \end{cases}
\end{align*}
Then which one of the following statements is true?
\begin{enumerate}
    \item [(A)] $X$ is a random variable with respect to $\mathcal{G}$, but $Y$ is not a random variable with respect to $\mathcal{G}$.
    \item [(B)] $Y$ is a random variable with respect to $\mathcal{G}$, but $X$ is not a random variable with respect to $\mathcal{G}$.
    \item [(C)] Neither $X$ nor $Y$ is a random variable with respect to $\mathcal{G}$.
    \item [(D)] Both $X$ and $Y$ are random variables with respect to $\mathcal{G}$.
\end{enumerate} \hfill (GATE ST 2023)\\
\solution
%\input{gate/ST/2023/14/main.tex}
	\item  A die is loaded in such a way that each odd number is twice as likely to occur as
each even number. Find $P(G)$, where $G$ is the event that a number greater than
3 occurs on a single roll of the die.
\\
\solution
		%\input{exemplar/11/16/3/5/main.tex}
	\item All the jacks, queens and kings are removed from a deck of 52 playing cards. The remaining cards are well shuffled and then one card is drawn at random. Giving ace a value 1 similar value for other cards, find the probability that the card has a value 
		\begin{enumerate}
			\item 7
			\item greater than 7
			\item less than 7
		\end{enumerate}
		%\input{exemplar/10/13/3/30/main.tex}
  \item A Lot consists of 48 mobile phones of which 42 are good, 3 have only minor defects and 3 have major defects.Varnika will buy a phone if it is good but the trader will only buy a mobile if it has no major defects. One phone is selected at random from the lot. What is the probability that it is
\begin{enumerate}
	\item acceptable to Varnika?
            \item acceptable to the trader?
\end{enumerate}
\solution
	%\input{exemplar/10/13/3/40/main.tex}
 \item A student says that if you throw a die, it will show up 1 or not 1. Therefore, the probability of getting 1 and the probability of getting 'not 1' each is equal to $\frac{1}{2}$. Is this correct? Give reasons.\\
 \solution
        %\input{exemplar/10/13/2/9/main.tex}
   \item Four candidates A, B, C, D have ap-
plied for the assignment to coach a school cricket
team. If A is twice as likely to be selected as B, and
B and C are given about the same chance of being
selected, while C is twice as likely to be selected
as D, what are the probabilities that
\begin{enumerate}
\item C will be selected?
\item A will not be selected?
\end{enumerate}
	%\input{exemplar/11/16/3/9/main.tex}
 \item A bag contain 24 balls of which $x$ balls are red, $2x$ are white and $3x$ are blue. A ball is selected at random, What is the probability that it is
\begin{enumerate}[label=\alph*)]
\item not red ?
\item white ?
\end{enumerate}
%\input{exemplar/10/13/3/41/main.tex}
If the letters of the word ASSASSINATION are arranged at random. Find the Probability that
\begin{enumerate}[label=(\alph*)]
\item Four $S's$ come consecutively in the word
\item Two  $I's$ and two $N's$ come together
\item All $A's$ are not coming together
\item No two $A's$ are coming together
\end{enumerate}
%\input{exemplar/11/16/3/14/main.tex}
	\item One urn contains two black balls (labelled B1 and B2) and one white ball. A
	second urn contains one black ball and two white balls (labelled W1 and W2).
	Suppose the following experiment is performed. One of the two urns is chosen
	at random. Next a ball is randomly chosen from the urn. Then a second ball is
	chosen at random from the same urn without replacing the first ball.
	
	\begin{enumerate}
	\item What is the probability that two black balls are chosen?
	
	\item What is the probability that two balls of opposite colour are chosen?
	\end{enumerate}
	\solution
	%\input{exemplar/11/16/3/12/main1.tex}
\end{enumerate}

		\item A box of oranges is inspected by examining three randomly selected oranges drawn without replacement. If all the three oranges are good, the box is approved for sale, otherwise, it is rejected. Find the probability that a box containing 15 oranges out of which 12 are good and 3 are bad ones will be approved for sale.
		\label{ncert/12/13/2/3/defs.tex}
		\item Two balls are drawn at random with replacement from a box containing 10 black and 8 red balls. Find the probability that
		\label{ncert/12/13/2/12}
\begin{enumerate}
\item both balls are red.
\item first ball is black and second is red.
\item one of them is black and other is red.
\end{enumerate}

\item In a hostel, 60\% of the students read Hindi newspaper, 40\% read English newspaper and 20\% read both Hindi and English newspapers. A student is selected at random.
		\label{ncert/12/13/2/15}
\begin{enumerate}
\item Find the probability that she reads neither Hindi nor English newspapers.
\item If she reads Hindi newspaper, find the probability that she reads English newspaper.
\item If she reads English newspaper, find the probability that she reads Hindi newspaper.\\
\end{enumerate}
\item The probability of obtaining an even prime number on each die, when a pair of dice is rolled is 
\begin{enumerate}
    \item $0$ 
    
    \item $\frac{1}{3}$ 
    
    \item $\frac{1}{12}$ 
    
    \item $\frac{1}{36}$ 
\end{enumerate}
\solution
		%\begin{enumerate}[label=\thesection.\arabic*,ref=\thesection.\theenumi]
	\item One card is drawn from a well-shuffled deck of 52 cards. Find the probability of getting
\begin{enumerate}
\item A king of red colour 
\item A face card 
\item A red face card
\item The jack of hearts
\item A spade
\item The queen of diamonds

\end{enumerate}
\solution
		%\input{ncert/10/15/1/14/main.tex}
	\item Five cards—the ten, jack, queen, king and ace of diamonds, are well-shuffled with their face downwards. One card is then picked up at random.
\begin{enumerate}
\item
What is the probability that the card is the queen? 
\item
If the queen is drawn and put aside, what is the probability that the second card picked up is (a) an ace? (b) a queen?\\
\end{enumerate}
\solution
		%\input{ncert/10/15/1/15/defs.tex}
	\item A bag contains $5$ red balls and some blue balls. If the probability of drawing a blue ball is double that if a red ball, determine the number of blue balls in the bag. 
		\\
\solution
		%\input{ncert/10/15/2/3/defs.tex}
	\item A card is selected from a pack of 52 cards.
 \begin{enumerate}[label=(\alph*)] 
                 \item How many points are there in the sample space?
                 \item Calculate the probability that the card is an ace of spades.
                 \item Calculate the probability that the card is (i) an ace and (ii) black card.
 \end{enumerate}
\solution
		%\input{ncert/11/16/3/4/main.tex}
\item Four cards are drawn from a well-shuffled deck of 52 cards. What is the probability of obtaining 3 diamonds and one spade.
\\
\solution
		%\input{ncert/11/16/4/2/defs.tex}
\item In a certain lottery 10,000 tickets are sold and ten equal prizes are awarded. What is the probability of not getting a prize if you buy (a) one ticket (b) two tickets (c) 10 tickets ?	
\\
\solution
		%\input{ncert/11/16/4/4/defs.tex}
		%
\item 
Out of 100 students, two sections of 40 and 60 are formed. If you and your friend are among the 100 students, what is the probability that
\begin{enumerate}
\item you both enter the same section?
\item you both enter the different sections?
\end{enumerate}
\solution
		%\input{ncert/11/16/4/5/defs.tex}
	\item 
The number lock of a suitcase has 4 wheels each labelled with ten digits i.e. from 0 to 9.The lock opens with a sequence of four digits with no repeats.What is the probability of a person getting the right sequence to open the suitcase.
\\
\solution
		%\input{ncert/11/16/4/10/defs.tex}
		%
\item 
Two cards are drawn at random and without replacement from a pack of 52 playing cards. Find the probability that both the cards are black.
\\
\solution
		%\input{ncert/12/13/2/2/defs.tex}
		\item A box of oranges is inspected by examining three randomly selected oranges drawn without replacement. If all the three oranges are good, the box is approved for sale, otherwise, it is rejected. Find the probability that a box containing 15 oranges out of which 12 are good and 3 are bad ones will be approved for sale.
		\label{ncert/12/13/2/3/defs.tex}
		\item Two balls are drawn at random with replacement from a box containing 10 black and 8 red balls. Find the probability that
		\label{ncert/12/13/2/12}
\begin{enumerate}
\item both balls are red.
\item first ball is black and second is red.
\item one of them is black and other is red.
\end{enumerate}

\item In a hostel, 60\% of the students read Hindi newspaper, 40\% read English newspaper and 20\% read both Hindi and English newspapers. A student is selected at random.
		\label{ncert/12/13/2/15}
\begin{enumerate}
\item Find the probability that she reads neither Hindi nor English newspapers.
\item If she reads Hindi newspaper, find the probability that she reads English newspaper.
\item If she reads English newspaper, find the probability that she reads Hindi newspaper.\\
\end{enumerate}
\item The probability of obtaining an even prime number on each die, when a pair of dice is rolled is 
\begin{enumerate}
    \item $0$ 
    
    \item $\frac{1}{3}$ 
    
    \item $\frac{1}{12}$ 
    
    \item $\frac{1}{36}$ 
\end{enumerate}
\solution
		%\input{ncert/12/13/2/17/defs.tex}
	\item A bag contains 4 red and 4 black balls, another bag contains 2 red and 6 black balls. One of the two bags is selected at random and a ball is drawn from the bag which is found to be red. Find the probability that the ball is drawn from the first bag.
\\
\solution
		%\input{ncert/12/13/3/2/main.tex}
  \item
  Cards with numbers 2 to 101 are placed in a box. A card is selected at random.Find the probability that the card has
\begin{enumerate}[label=(\roman*)]
	\item an even number 
	\item a square number
\end{enumerate}
\solution
%\input{exemplar/10/13/3/32/main.tex}
\item
The king, queen and jack of clubs are removed from a deck of 52 playing cards and then well shuffled. Now one card is drawn at random from the remaining cards.  Determine the probability that the card is
\begin{enumerate}[label=(\roman*)]
\item a club
\item 10 of hearts
\end{enumerate}
\solution
%\input{exemplar/10/13/3/29/main.tex}
\item A team of medical students doing their internship have to assist during surgeries
at a city hospital. The probabilities of surgeries rated as very complex, complex,
routine, simple or very simple are respectively, 0.15, 0.20, 0.31, 0.26, .08. Find
the probabilities that a particular surgery will be rated
\begin{enumerate}
	\item complex or very complex;
	\item neither very complex nor very simple;
	\item routine or complex
	\item routine or simple
\end{enumerate}
\solution
%\input{exemplar/11/16/3/8(1)/main.tex}
\item A card is selected from a pack of 52 cards.
\begin{enumerate}[label=(\alph*)]
    \item How many points are there in the sample space?
    \item Calculate the probability that the card is an ace of spades.
    \item Calculate the probability that the card is (i) an ace and (ii) black card.
\end{enumerate}
\solution
%\input{exemplar/11/16/3/4/main2.tex}
\item The probability that a non leap year selected at random will contain 53 sundays.
\\
\solution
%\input{exemplar/10/13/1/19/main.tex}
\item One of the four persons John, Rita, Aslam or Gurpreet will be promoted next
month. Consequently the sample space consists of four elementary outcomes
S = {John promoted, Rita promoted, Aslam promoted, Gurpreet promoted}
You are told that the chances of John’s promotion is same as that of Gurpreet,
Rita’s chances of promotion are twice as likely as Johns. Aslam’s chances are
four times that of John.
\begin{enumerate}
	\item Determine
	\begin{enumerate}
		\item P (John promoted)
		\item P (Rita promoted)
		\item P (Aslam promoted)
		\item P (Gurpreet promoted)
	\end{enumerate}
	\item If A = {John promoted or Gurpreet promoted}, find P (A).
\end{enumerate}
\solution
%\input{exemplar/11/16/3/10/main.tex}
\item A card is drawn from a deck of 52 cards. Find the probability of getting a king or a heart or a red card.\\
\solution
%\input{exemplar/11/16/3/15/main.tex}
\item The probability that a student will pass his examination is 0.73, the probability of
the student getting a compartment is 0.13, and the probability that the student will
either pass or get compartment is 0.96. State True or False.\\
\solution
%\input{exemplar/11/16/3/31/main.tex}
\item A card is selected from a pack of 52 cards\\
\begin{enumerate}[label=(\alph*)]
\item How many points are there in the sample space?
\item Calculate the probability that the cards is an ace of spades.
\item Calculate the probability that the card is (i) an ace (ii)black card.\\
\end{enumerate}
%\input{ncert/11/16/3/4_1/Prob_4.tex}
\item In a non-leap year, the probability of having 53 tuesdays or 53 wednesdays is\\
\solution
%\input{exemplar/11/16/3/18/main.tex}
\item There are 1000 sealed envelopes in a box, 10 of them contain a cash prize of
Rs 100 each, 100 of them contain a cash prize of Rs 50 each and 200 of them
contain a cash prize of Rs 10 each and rest do not contain any cash prize. If they
are well shuffled and an envelope is picked up out, what is the probability that it
contains no cash prize?\\
\solution
%\input{exemplar/10/13/3/34/main.tex}
\item 
A die is thrown and a card is selected at random from a deck of 52 playing cards. The probability of getting an even number on the die and a spade card.\\
\solution
%\input{exemplar/12/13/3/78/main.tex}
\item
If 4-digit numbers greater than 5,000 are randomly formed from the digits 0, 1, 3, 5, and 7, what is the probability of forming a number divisible by 5 when:
\begin{enumerate}
    \item The digits are repeated?
    \item The repetition of digits is not allowed?
\end{enumerate}
\solution
%\input{ncert/11/16/4/9/main.tex}
\item Consider the probability space $\brak{\Omega, \mathcal{G}, P}$ where $\Omega = [0,2]$ and $\mathcal{G} = \cbrak{\phi, \Omega, [0,1], (1,2]}$. Let $X$ and $Y$ be two functions on $\Omega$ defined as
\begin{align*}
    X(\omega) = 
    \begin{cases}
        1 & \text{if }\omega \in [0, 1]\\
        2 & \text{if }\omega \in (1, 2]
    \end{cases}
\end{align*}
and
\begin{align*}
    Y(\omega) = 
    \begin{cases}
        2 & \text{if }\omega \in [0, 1.5]\\
        3 & \text{if }\omega \in (1.5, 2].
    \end{cases}
\end{align*}
Then which one of the following statements is true?
\begin{enumerate}
    \item [(A)] $X$ is a random variable with respect to $\mathcal{G}$, but $Y$ is not a random variable with respect to $\mathcal{G}$.
    \item [(B)] $Y$ is a random variable with respect to $\mathcal{G}$, but $X$ is not a random variable with respect to $\mathcal{G}$.
    \item [(C)] Neither $X$ nor $Y$ is a random variable with respect to $\mathcal{G}$.
    \item [(D)] Both $X$ and $Y$ are random variables with respect to $\mathcal{G}$.
\end{enumerate} \hfill (GATE ST 2023)\\
\solution
%\input{gate/ST/2023/14/main.tex}
	\item  A die is loaded in such a way that each odd number is twice as likely to occur as
each even number. Find $P(G)$, where $G$ is the event that a number greater than
3 occurs on a single roll of the die.
\\
\solution
		%\input{exemplar/11/16/3/5/main.tex}
	\item All the jacks, queens and kings are removed from a deck of 52 playing cards. The remaining cards are well shuffled and then one card is drawn at random. Giving ace a value 1 similar value for other cards, find the probability that the card has a value 
		\begin{enumerate}
			\item 7
			\item greater than 7
			\item less than 7
		\end{enumerate}
		%\input{exemplar/10/13/3/30/main.tex}
  \item A Lot consists of 48 mobile phones of which 42 are good, 3 have only minor defects and 3 have major defects.Varnika will buy a phone if it is good but the trader will only buy a mobile if it has no major defects. One phone is selected at random from the lot. What is the probability that it is
\begin{enumerate}
	\item acceptable to Varnika?
            \item acceptable to the trader?
\end{enumerate}
\solution
	%\input{exemplar/10/13/3/40/main.tex}
 \item A student says that if you throw a die, it will show up 1 or not 1. Therefore, the probability of getting 1 and the probability of getting 'not 1' each is equal to $\frac{1}{2}$. Is this correct? Give reasons.\\
 \solution
        %\input{exemplar/10/13/2/9/main.tex}
   \item Four candidates A, B, C, D have ap-
plied for the assignment to coach a school cricket
team. If A is twice as likely to be selected as B, and
B and C are given about the same chance of being
selected, while C is twice as likely to be selected
as D, what are the probabilities that
\begin{enumerate}
\item C will be selected?
\item A will not be selected?
\end{enumerate}
	%\input{exemplar/11/16/3/9/main.tex}
 \item A bag contain 24 balls of which $x$ balls are red, $2x$ are white and $3x$ are blue. A ball is selected at random, What is the probability that it is
\begin{enumerate}[label=\alph*)]
\item not red ?
\item white ?
\end{enumerate}
%\input{exemplar/10/13/3/41/main.tex}
If the letters of the word ASSASSINATION are arranged at random. Find the Probability that
\begin{enumerate}[label=(\alph*)]
\item Four $S's$ come consecutively in the word
\item Two  $I's$ and two $N's$ come together
\item All $A's$ are not coming together
\item No two $A's$ are coming together
\end{enumerate}
%\input{exemplar/11/16/3/14/main.tex}
	\item One urn contains two black balls (labelled B1 and B2) and one white ball. A
	second urn contains one black ball and two white balls (labelled W1 and W2).
	Suppose the following experiment is performed. One of the two urns is chosen
	at random. Next a ball is randomly chosen from the urn. Then a second ball is
	chosen at random from the same urn without replacing the first ball.
	
	\begin{enumerate}
	\item What is the probability that two black balls are chosen?
	
	\item What is the probability that two balls of opposite colour are chosen?
	\end{enumerate}
	\solution
	%\input{exemplar/11/16/3/12/main1.tex}
\end{enumerate}

	\item A bag contains 4 red and 4 black balls, another bag contains 2 red and 6 black balls. One of the two bags is selected at random and a ball is drawn from the bag which is found to be red. Find the probability that the ball is drawn from the first bag.
\\
\solution
		%\begin{table}[H]
	\centering
\begin{tabular}{|c|c|c|}
\hline
Random variable &Value &Definition\\ \hline
\multirow{3}{*}{X} &0 &Slips of Rs 1\\
&1 &Slips of Rs 5\\
&2 &Slips of Rs 13\\ \hline
\multirow{2}{*}{Y} &0 &Box A\\
&1 &Box B\\\hline
\end{tabular}
\caption{}
\label{tab:Distribution}
\end{table}
See \tabref{tab:Distribution}.
\begin{align}
p_{Y}\brak{k}= \begin{cases} 
      \frac{1}{3} & {k=0} \\
      \frac{2}{3 }& {k=1} 
   \end{cases}
   \\
p_{Y|X}\brak{0|0} = \frac{19}{25}\, 
p_{Y|X}\brak{0|1} = \frac{6}{25}\,
p_{Y|X}\brak{1|0} = \frac{45}{50}\,
p_{Y|X}\brak{1|2} = \frac{5}{50}
\end{align}
The desired probability is the probability that a slip drawn at random is marked other than Rs 1,
\begin{align}
&=1-p_X\brak{0}\\
&= p_X(1) + p_X(2)
\end{align}
Using Bayes theorem,
\begin{align}
&= p_Y\brak{0} \times \pr{Y=0 | X=1} + p_Y\brak{1} \times \pr{Y=1|X=2}\\
&=\frac{1}{3} \times \frac{6}{25} + \frac{2}{3} \times \frac{5}{50}\\
&=\frac{11}{75}
\end{align}

\newpage

%\tableofcontents

\bigskip

\renewcommand{\thefigure}{\theenumi}
\renewcommand{\thetable}{\theenumi}
%\renewcommand{\theequation}{\theenumi}

%\begin{abstract}
%%\boldmath
%In this letter, an algorithm for evaluating the exact analytical bit error rate  (BER)  for the piecewise linear (PL) combiner for  multiple relays is presented. Previous results were available only for upto three relays. The algorithm is unique in the sense that  the actual mathematical expressions, that are prohibitively large, need not be explicitly obtained. The diversity gain due to multiple relays is shown through plots of the analytical BER, well supported by simulations. 
%
%\end{abstract}
% IEEEtran.cls defaults to using nonbold math in the Abstract.
% This preserves the distinction between vectors and scalars. However,
% if the journal you are submitting to favors bold math in the abstract,
% then you can use LaTeX's standard command \boldmath at the very start
% of the abstract to achieve this. Many IEEE journals frown on math
% in the abstract anyway.

% Note that keywords are not normally used for peerreview papers.
%\begin{IEEEkeywords}
%Cooperative diversity, decode and forward, piecewise linear
%\end{IEEEkeywords}



% For peer review papers, you can put extra information on the cover
% page as needed:
% \ifCLASSOPTIONpeerreview
% \begin{center} \bfseries EDICS Category: 3-BBND \end{center}
% \fi
%
% For peerreview papers, this IEEEtran command inserts a page break and
% creates the second title. It will be ignored for other modes.
%\IEEEpeerreviewmaketitle




  \item
  Cards with numbers 2 to 101 are placed in a box. A card is selected at random.Find the probability that the card has
\begin{enumerate}[label=(\roman*)]
	\item an even number 
	\item a square number
\end{enumerate}
\solution
%\begin{table}[H]
	\centering
\begin{tabular}{|c|c|c|}
\hline
Random variable &Value &Definition\\ \hline
\multirow{3}{*}{X} &0 &Slips of Rs 1\\
&1 &Slips of Rs 5\\
&2 &Slips of Rs 13\\ \hline
\multirow{2}{*}{Y} &0 &Box A\\
&1 &Box B\\\hline
\end{tabular}
\caption{}
\label{tab:Distribution}
\end{table}
See \tabref{tab:Distribution}.
\begin{align}
p_{Y}\brak{k}= \begin{cases} 
      \frac{1}{3} & {k=0} \\
      \frac{2}{3 }& {k=1} 
   \end{cases}
   \\
p_{Y|X}\brak{0|0} = \frac{19}{25}\, 
p_{Y|X}\brak{0|1} = \frac{6}{25}\,
p_{Y|X}\brak{1|0} = \frac{45}{50}\,
p_{Y|X}\brak{1|2} = \frac{5}{50}
\end{align}
The desired probability is the probability that a slip drawn at random is marked other than Rs 1,
\begin{align}
&=1-p_X\brak{0}\\
&= p_X(1) + p_X(2)
\end{align}
Using Bayes theorem,
\begin{align}
&= p_Y\brak{0} \times \pr{Y=0 | X=1} + p_Y\brak{1} \times \pr{Y=1|X=2}\\
&=\frac{1}{3} \times \frac{6}{25} + \frac{2}{3} \times \frac{5}{50}\\
&=\frac{11}{75}
\end{align}

\newpage

%\tableofcontents

\bigskip

\renewcommand{\thefigure}{\theenumi}
\renewcommand{\thetable}{\theenumi}
%\renewcommand{\theequation}{\theenumi}

%\begin{abstract}
%%\boldmath
%In this letter, an algorithm for evaluating the exact analytical bit error rate  (BER)  for the piecewise linear (PL) combiner for  multiple relays is presented. Previous results were available only for upto three relays. The algorithm is unique in the sense that  the actual mathematical expressions, that are prohibitively large, need not be explicitly obtained. The diversity gain due to multiple relays is shown through plots of the analytical BER, well supported by simulations. 
%
%\end{abstract}
% IEEEtran.cls defaults to using nonbold math in the Abstract.
% This preserves the distinction between vectors and scalars. However,
% if the journal you are submitting to favors bold math in the abstract,
% then you can use LaTeX's standard command \boldmath at the very start
% of the abstract to achieve this. Many IEEE journals frown on math
% in the abstract anyway.

% Note that keywords are not normally used for peerreview papers.
%\begin{IEEEkeywords}
%Cooperative diversity, decode and forward, piecewise linear
%\end{IEEEkeywords}



% For peer review papers, you can put extra information on the cover
% page as needed:
% \ifCLASSOPTIONpeerreview
% \begin{center} \bfseries EDICS Category: 3-BBND \end{center}
% \fi
%
% For peerreview papers, this IEEEtran command inserts a page break and
% creates the second title. It will be ignored for other modes.
%\IEEEpeerreviewmaketitle




\item
The king, queen and jack of clubs are removed from a deck of 52 playing cards and then well shuffled. Now one card is drawn at random from the remaining cards.  Determine the probability that the card is
\begin{enumerate}[label=(\roman*)]
\item a club
\item 10 of hearts
\end{enumerate}
\solution
%\begin{table}[H]
	\centering
\begin{tabular}{|c|c|c|}
\hline
Random variable &Value &Definition\\ \hline
\multirow{3}{*}{X} &0 &Slips of Rs 1\\
&1 &Slips of Rs 5\\
&2 &Slips of Rs 13\\ \hline
\multirow{2}{*}{Y} &0 &Box A\\
&1 &Box B\\\hline
\end{tabular}
\caption{}
\label{tab:Distribution}
\end{table}
See \tabref{tab:Distribution}.
\begin{align}
p_{Y}\brak{k}= \begin{cases} 
      \frac{1}{3} & {k=0} \\
      \frac{2}{3 }& {k=1} 
   \end{cases}
   \\
p_{Y|X}\brak{0|0} = \frac{19}{25}\, 
p_{Y|X}\brak{0|1} = \frac{6}{25}\,
p_{Y|X}\brak{1|0} = \frac{45}{50}\,
p_{Y|X}\brak{1|2} = \frac{5}{50}
\end{align}
The desired probability is the probability that a slip drawn at random is marked other than Rs 1,
\begin{align}
&=1-p_X\brak{0}\\
&= p_X(1) + p_X(2)
\end{align}
Using Bayes theorem,
\begin{align}
&= p_Y\brak{0} \times \pr{Y=0 | X=1} + p_Y\brak{1} \times \pr{Y=1|X=2}\\
&=\frac{1}{3} \times \frac{6}{25} + \frac{2}{3} \times \frac{5}{50}\\
&=\frac{11}{75}
\end{align}

\newpage

%\tableofcontents

\bigskip

\renewcommand{\thefigure}{\theenumi}
\renewcommand{\thetable}{\theenumi}
%\renewcommand{\theequation}{\theenumi}

%\begin{abstract}
%%\boldmath
%In this letter, an algorithm for evaluating the exact analytical bit error rate  (BER)  for the piecewise linear (PL) combiner for  multiple relays is presented. Previous results were available only for upto three relays. The algorithm is unique in the sense that  the actual mathematical expressions, that are prohibitively large, need not be explicitly obtained. The diversity gain due to multiple relays is shown through plots of the analytical BER, well supported by simulations. 
%
%\end{abstract}
% IEEEtran.cls defaults to using nonbold math in the Abstract.
% This preserves the distinction between vectors and scalars. However,
% if the journal you are submitting to favors bold math in the abstract,
% then you can use LaTeX's standard command \boldmath at the very start
% of the abstract to achieve this. Many IEEE journals frown on math
% in the abstract anyway.

% Note that keywords are not normally used for peerreview papers.
%\begin{IEEEkeywords}
%Cooperative diversity, decode and forward, piecewise linear
%\end{IEEEkeywords}



% For peer review papers, you can put extra information on the cover
% page as needed:
% \ifCLASSOPTIONpeerreview
% \begin{center} \bfseries EDICS Category: 3-BBND \end{center}
% \fi
%
% For peerreview papers, this IEEEtran command inserts a page break and
% creates the second title. It will be ignored for other modes.
%\IEEEpeerreviewmaketitle




\item A team of medical students doing their internship have to assist during surgeries
at a city hospital. The probabilities of surgeries rated as very complex, complex,
routine, simple or very simple are respectively, 0.15, 0.20, 0.31, 0.26, .08. Find
the probabilities that a particular surgery will be rated
\begin{enumerate}
	\item complex or very complex;
	\item neither very complex nor very simple;
	\item routine or complex
	\item routine or simple
\end{enumerate}
\solution
%\begin{table}[H]
	\centering
\begin{tabular}{|c|c|c|}
\hline
Random variable &Value &Definition\\ \hline
\multirow{3}{*}{X} &0 &Slips of Rs 1\\
&1 &Slips of Rs 5\\
&2 &Slips of Rs 13\\ \hline
\multirow{2}{*}{Y} &0 &Box A\\
&1 &Box B\\\hline
\end{tabular}
\caption{}
\label{tab:Distribution}
\end{table}
See \tabref{tab:Distribution}.
\begin{align}
p_{Y}\brak{k}= \begin{cases} 
      \frac{1}{3} & {k=0} \\
      \frac{2}{3 }& {k=1} 
   \end{cases}
   \\
p_{Y|X}\brak{0|0} = \frac{19}{25}\, 
p_{Y|X}\brak{0|1} = \frac{6}{25}\,
p_{Y|X}\brak{1|0} = \frac{45}{50}\,
p_{Y|X}\brak{1|2} = \frac{5}{50}
\end{align}
The desired probability is the probability that a slip drawn at random is marked other than Rs 1,
\begin{align}
&=1-p_X\brak{0}\\
&= p_X(1) + p_X(2)
\end{align}
Using Bayes theorem,
\begin{align}
&= p_Y\brak{0} \times \pr{Y=0 | X=1} + p_Y\brak{1} \times \pr{Y=1|X=2}\\
&=\frac{1}{3} \times \frac{6}{25} + \frac{2}{3} \times \frac{5}{50}\\
&=\frac{11}{75}
\end{align}

\newpage

%\tableofcontents

\bigskip

\renewcommand{\thefigure}{\theenumi}
\renewcommand{\thetable}{\theenumi}
%\renewcommand{\theequation}{\theenumi}

%\begin{abstract}
%%\boldmath
%In this letter, an algorithm for evaluating the exact analytical bit error rate  (BER)  for the piecewise linear (PL) combiner for  multiple relays is presented. Previous results were available only for upto three relays. The algorithm is unique in the sense that  the actual mathematical expressions, that are prohibitively large, need not be explicitly obtained. The diversity gain due to multiple relays is shown through plots of the analytical BER, well supported by simulations. 
%
%\end{abstract}
% IEEEtran.cls defaults to using nonbold math in the Abstract.
% This preserves the distinction between vectors and scalars. However,
% if the journal you are submitting to favors bold math in the abstract,
% then you can use LaTeX's standard command \boldmath at the very start
% of the abstract to achieve this. Many IEEE journals frown on math
% in the abstract anyway.

% Note that keywords are not normally used for peerreview papers.
%\begin{IEEEkeywords}
%Cooperative diversity, decode and forward, piecewise linear
%\end{IEEEkeywords}



% For peer review papers, you can put extra information on the cover
% page as needed:
% \ifCLASSOPTIONpeerreview
% \begin{center} \bfseries EDICS Category: 3-BBND \end{center}
% \fi
%
% For peerreview papers, this IEEEtran command inserts a page break and
% creates the second title. It will be ignored for other modes.
%\IEEEpeerreviewmaketitle




\item A card is selected from a pack of 52 cards.
\begin{enumerate}[label=(\alph*)]
    \item How many points are there in the sample space?
    \item Calculate the probability that the card is an ace of spades.
    \item Calculate the probability that the card is (i) an ace and (ii) black card.
\end{enumerate}
\solution
%Let $X$ be an bernoulli rv defined as in \tabref{tab:exemplar/11/16/3/26}.  Then, 
\begin{equation}
    p =
        \frac{4}{11} 
\end{equation}
\begin{table}[H]
	\centering
	\input{exemplar/11/16/3/26/tables/Table2.tex}
	\caption{}
        \label{tab:exemplar/11/16/3/26}
\end{table}

\item The probability that a non leap year selected at random will contain 53 sundays.
\\
\solution
%\begin{table}[H]
	\centering
\begin{tabular}{|c|c|c|}
\hline
Random variable &Value &Definition\\ \hline
\multirow{3}{*}{X} &0 &Slips of Rs 1\\
&1 &Slips of Rs 5\\
&2 &Slips of Rs 13\\ \hline
\multirow{2}{*}{Y} &0 &Box A\\
&1 &Box B\\\hline
\end{tabular}
\caption{}
\label{tab:Distribution}
\end{table}
See \tabref{tab:Distribution}.
\begin{align}
p_{Y}\brak{k}= \begin{cases} 
      \frac{1}{3} & {k=0} \\
      \frac{2}{3 }& {k=1} 
   \end{cases}
   \\
p_{Y|X}\brak{0|0} = \frac{19}{25}\, 
p_{Y|X}\brak{0|1} = \frac{6}{25}\,
p_{Y|X}\brak{1|0} = \frac{45}{50}\,
p_{Y|X}\brak{1|2} = \frac{5}{50}
\end{align}
The desired probability is the probability that a slip drawn at random is marked other than Rs 1,
\begin{align}
&=1-p_X\brak{0}\\
&= p_X(1) + p_X(2)
\end{align}
Using Bayes theorem,
\begin{align}
&= p_Y\brak{0} \times \pr{Y=0 | X=1} + p_Y\brak{1} \times \pr{Y=1|X=2}\\
&=\frac{1}{3} \times \frac{6}{25} + \frac{2}{3} \times \frac{5}{50}\\
&=\frac{11}{75}
\end{align}

\newpage

%\tableofcontents

\bigskip

\renewcommand{\thefigure}{\theenumi}
\renewcommand{\thetable}{\theenumi}
%\renewcommand{\theequation}{\theenumi}

%\begin{abstract}
%%\boldmath
%In this letter, an algorithm for evaluating the exact analytical bit error rate  (BER)  for the piecewise linear (PL) combiner for  multiple relays is presented. Previous results were available only for upto three relays. The algorithm is unique in the sense that  the actual mathematical expressions, that are prohibitively large, need not be explicitly obtained. The diversity gain due to multiple relays is shown through plots of the analytical BER, well supported by simulations. 
%
%\end{abstract}
% IEEEtran.cls defaults to using nonbold math in the Abstract.
% This preserves the distinction between vectors and scalars. However,
% if the journal you are submitting to favors bold math in the abstract,
% then you can use LaTeX's standard command \boldmath at the very start
% of the abstract to achieve this. Many IEEE journals frown on math
% in the abstract anyway.

% Note that keywords are not normally used for peerreview papers.
%\begin{IEEEkeywords}
%Cooperative diversity, decode and forward, piecewise linear
%\end{IEEEkeywords}



% For peer review papers, you can put extra information on the cover
% page as needed:
% \ifCLASSOPTIONpeerreview
% \begin{center} \bfseries EDICS Category: 3-BBND \end{center}
% \fi
%
% For peerreview papers, this IEEEtran command inserts a page break and
% creates the second title. It will be ignored for other modes.
%\IEEEpeerreviewmaketitle




\item One of the four persons John, Rita, Aslam or Gurpreet will be promoted next
month. Consequently the sample space consists of four elementary outcomes
S = {John promoted, Rita promoted, Aslam promoted, Gurpreet promoted}
You are told that the chances of John’s promotion is same as that of Gurpreet,
Rita’s chances of promotion are twice as likely as Johns. Aslam’s chances are
four times that of John.
\begin{enumerate}
	\item Determine
	\begin{enumerate}
		\item P (John promoted)
		\item P (Rita promoted)
		\item P (Aslam promoted)
		\item P (Gurpreet promoted)
	\end{enumerate}
	\item If A = {John promoted or Gurpreet promoted}, find P (A).
\end{enumerate}
\solution
%\begin{table}[H]
	\centering
\begin{tabular}{|c|c|c|}
\hline
Random variable &Value &Definition\\ \hline
\multirow{3}{*}{X} &0 &Slips of Rs 1\\
&1 &Slips of Rs 5\\
&2 &Slips of Rs 13\\ \hline
\multirow{2}{*}{Y} &0 &Box A\\
&1 &Box B\\\hline
\end{tabular}
\caption{}
\label{tab:Distribution}
\end{table}
See \tabref{tab:Distribution}.
\begin{align}
p_{Y}\brak{k}= \begin{cases} 
      \frac{1}{3} & {k=0} \\
      \frac{2}{3 }& {k=1} 
   \end{cases}
   \\
p_{Y|X}\brak{0|0} = \frac{19}{25}\, 
p_{Y|X}\brak{0|1} = \frac{6}{25}\,
p_{Y|X}\brak{1|0} = \frac{45}{50}\,
p_{Y|X}\brak{1|2} = \frac{5}{50}
\end{align}
The desired probability is the probability that a slip drawn at random is marked other than Rs 1,
\begin{align}
&=1-p_X\brak{0}\\
&= p_X(1) + p_X(2)
\end{align}
Using Bayes theorem,
\begin{align}
&= p_Y\brak{0} \times \pr{Y=0 | X=1} + p_Y\brak{1} \times \pr{Y=1|X=2}\\
&=\frac{1}{3} \times \frac{6}{25} + \frac{2}{3} \times \frac{5}{50}\\
&=\frac{11}{75}
\end{align}

\newpage

%\tableofcontents

\bigskip

\renewcommand{\thefigure}{\theenumi}
\renewcommand{\thetable}{\theenumi}
%\renewcommand{\theequation}{\theenumi}

%\begin{abstract}
%%\boldmath
%In this letter, an algorithm for evaluating the exact analytical bit error rate  (BER)  for the piecewise linear (PL) combiner for  multiple relays is presented. Previous results were available only for upto three relays. The algorithm is unique in the sense that  the actual mathematical expressions, that are prohibitively large, need not be explicitly obtained. The diversity gain due to multiple relays is shown through plots of the analytical BER, well supported by simulations. 
%
%\end{abstract}
% IEEEtran.cls defaults to using nonbold math in the Abstract.
% This preserves the distinction between vectors and scalars. However,
% if the journal you are submitting to favors bold math in the abstract,
% then you can use LaTeX's standard command \boldmath at the very start
% of the abstract to achieve this. Many IEEE journals frown on math
% in the abstract anyway.

% Note that keywords are not normally used for peerreview papers.
%\begin{IEEEkeywords}
%Cooperative diversity, decode and forward, piecewise linear
%\end{IEEEkeywords}



% For peer review papers, you can put extra information on the cover
% page as needed:
% \ifCLASSOPTIONpeerreview
% \begin{center} \bfseries EDICS Category: 3-BBND \end{center}
% \fi
%
% For peerreview papers, this IEEEtran command inserts a page break and
% creates the second title. It will be ignored for other modes.
%\IEEEpeerreviewmaketitle




\item A card is drawn from a deck of 52 cards. Find the probability of getting a king or a heart or a red card.\\
\solution
%\begin{table}[H]
	\centering
\begin{tabular}{|c|c|c|}
\hline
Random variable &Value &Definition\\ \hline
\multirow{3}{*}{X} &0 &Slips of Rs 1\\
&1 &Slips of Rs 5\\
&2 &Slips of Rs 13\\ \hline
\multirow{2}{*}{Y} &0 &Box A\\
&1 &Box B\\\hline
\end{tabular}
\caption{}
\label{tab:Distribution}
\end{table}
See \tabref{tab:Distribution}.
\begin{align}
p_{Y}\brak{k}= \begin{cases} 
      \frac{1}{3} & {k=0} \\
      \frac{2}{3 }& {k=1} 
   \end{cases}
   \\
p_{Y|X}\brak{0|0} = \frac{19}{25}\, 
p_{Y|X}\brak{0|1} = \frac{6}{25}\,
p_{Y|X}\brak{1|0} = \frac{45}{50}\,
p_{Y|X}\brak{1|2} = \frac{5}{50}
\end{align}
The desired probability is the probability that a slip drawn at random is marked other than Rs 1,
\begin{align}
&=1-p_X\brak{0}\\
&= p_X(1) + p_X(2)
\end{align}
Using Bayes theorem,
\begin{align}
&= p_Y\brak{0} \times \pr{Y=0 | X=1} + p_Y\brak{1} \times \pr{Y=1|X=2}\\
&=\frac{1}{3} \times \frac{6}{25} + \frac{2}{3} \times \frac{5}{50}\\
&=\frac{11}{75}
\end{align}

\newpage

%\tableofcontents

\bigskip

\renewcommand{\thefigure}{\theenumi}
\renewcommand{\thetable}{\theenumi}
%\renewcommand{\theequation}{\theenumi}

%\begin{abstract}
%%\boldmath
%In this letter, an algorithm for evaluating the exact analytical bit error rate  (BER)  for the piecewise linear (PL) combiner for  multiple relays is presented. Previous results were available only for upto three relays. The algorithm is unique in the sense that  the actual mathematical expressions, that are prohibitively large, need not be explicitly obtained. The diversity gain due to multiple relays is shown through plots of the analytical BER, well supported by simulations. 
%
%\end{abstract}
% IEEEtran.cls defaults to using nonbold math in the Abstract.
% This preserves the distinction between vectors and scalars. However,
% if the journal you are submitting to favors bold math in the abstract,
% then you can use LaTeX's standard command \boldmath at the very start
% of the abstract to achieve this. Many IEEE journals frown on math
% in the abstract anyway.

% Note that keywords are not normally used for peerreview papers.
%\begin{IEEEkeywords}
%Cooperative diversity, decode and forward, piecewise linear
%\end{IEEEkeywords}



% For peer review papers, you can put extra information on the cover
% page as needed:
% \ifCLASSOPTIONpeerreview
% \begin{center} \bfseries EDICS Category: 3-BBND \end{center}
% \fi
%
% For peerreview papers, this IEEEtran command inserts a page break and
% creates the second title. It will be ignored for other modes.
%\IEEEpeerreviewmaketitle




\item The probability that a student will pass his examination is 0.73, the probability of
the student getting a compartment is 0.13, and the probability that the student will
either pass or get compartment is 0.96. State True or False.\\
\solution
%\begin{table}[H]
	\centering
\begin{tabular}{|c|c|c|}
\hline
Random variable &Value &Definition\\ \hline
\multirow{3}{*}{X} &0 &Slips of Rs 1\\
&1 &Slips of Rs 5\\
&2 &Slips of Rs 13\\ \hline
\multirow{2}{*}{Y} &0 &Box A\\
&1 &Box B\\\hline
\end{tabular}
\caption{}
\label{tab:Distribution}
\end{table}
See \tabref{tab:Distribution}.
\begin{align}
p_{Y}\brak{k}= \begin{cases} 
      \frac{1}{3} & {k=0} \\
      \frac{2}{3 }& {k=1} 
   \end{cases}
   \\
p_{Y|X}\brak{0|0} = \frac{19}{25}\, 
p_{Y|X}\brak{0|1} = \frac{6}{25}\,
p_{Y|X}\brak{1|0} = \frac{45}{50}\,
p_{Y|X}\brak{1|2} = \frac{5}{50}
\end{align}
The desired probability is the probability that a slip drawn at random is marked other than Rs 1,
\begin{align}
&=1-p_X\brak{0}\\
&= p_X(1) + p_X(2)
\end{align}
Using Bayes theorem,
\begin{align}
&= p_Y\brak{0} \times \pr{Y=0 | X=1} + p_Y\brak{1} \times \pr{Y=1|X=2}\\
&=\frac{1}{3} \times \frac{6}{25} + \frac{2}{3} \times \frac{5}{50}\\
&=\frac{11}{75}
\end{align}

\newpage

%\tableofcontents

\bigskip

\renewcommand{\thefigure}{\theenumi}
\renewcommand{\thetable}{\theenumi}
%\renewcommand{\theequation}{\theenumi}

%\begin{abstract}
%%\boldmath
%In this letter, an algorithm for evaluating the exact analytical bit error rate  (BER)  for the piecewise linear (PL) combiner for  multiple relays is presented. Previous results were available only for upto three relays. The algorithm is unique in the sense that  the actual mathematical expressions, that are prohibitively large, need not be explicitly obtained. The diversity gain due to multiple relays is shown through plots of the analytical BER, well supported by simulations. 
%
%\end{abstract}
% IEEEtran.cls defaults to using nonbold math in the Abstract.
% This preserves the distinction between vectors and scalars. However,
% if the journal you are submitting to favors bold math in the abstract,
% then you can use LaTeX's standard command \boldmath at the very start
% of the abstract to achieve this. Many IEEE journals frown on math
% in the abstract anyway.

% Note that keywords are not normally used for peerreview papers.
%\begin{IEEEkeywords}
%Cooperative diversity, decode and forward, piecewise linear
%\end{IEEEkeywords}



% For peer review papers, you can put extra information on the cover
% page as needed:
% \ifCLASSOPTIONpeerreview
% \begin{center} \bfseries EDICS Category: 3-BBND \end{center}
% \fi
%
% For peerreview papers, this IEEEtran command inserts a page break and
% creates the second title. It will be ignored for other modes.
%\IEEEpeerreviewmaketitle




\item A card is selected from a pack of 52 cards\\
\begin{enumerate}[label=(\alph*)]
\item How many points are there in the sample space?
\item Calculate the probability that the cards is an ace of spades.
\item Calculate the probability that the card is (i) an ace (ii)black card.\\
\end{enumerate}
%\input{ncert/11/16/3/4_1/Prob_4.tex}
\item In a non-leap year, the probability of having 53 tuesdays or 53 wednesdays is\\
\solution
%A non-leap year has a total of 365 days, and a week has 7 days.\\
So it can be expressed as 
\begin{align}
365\text{days} &=52\times 7+1 \text{day}
\end{align}
$\implies$ 52 tuesdays or wednesdays\\
Random variable X denotes the days of a week
\begin{align}
p_X\brak{k}&=\frac{1}{7}; \quad \brak{1<k<7}
\end{align}
So the probability of extra day being tuesday or wednesday is
\begin{align}
p_X\brak{3}+p_X\brak{4}&=\frac{1}{7}+\frac{1}{7}=\frac{2}{7}
\end{align}



\item There are 1000 sealed envelopes in a box, 10 of them contain a cash prize of
Rs 100 each, 100 of them contain a cash prize of Rs 50 each and 200 of them
contain a cash prize of Rs 10 each and rest do not contain any cash prize. If they
are well shuffled and an envelope is picked up out, what is the probability that it
contains no cash prize?\\
\solution
%\begin{table}[H]
	\centering
\begin{tabular}{|c|c|c|}
\hline
Random variable &Value &Definition\\ \hline
\multirow{3}{*}{X} &0 &Slips of Rs 1\\
&1 &Slips of Rs 5\\
&2 &Slips of Rs 13\\ \hline
\multirow{2}{*}{Y} &0 &Box A\\
&1 &Box B\\\hline
\end{tabular}
\caption{}
\label{tab:Distribution}
\end{table}
See \tabref{tab:Distribution}.
\begin{align}
p_{Y}\brak{k}= \begin{cases} 
      \frac{1}{3} & {k=0} \\
      \frac{2}{3 }& {k=1} 
   \end{cases}
   \\
p_{Y|X}\brak{0|0} = \frac{19}{25}\, 
p_{Y|X}\brak{0|1} = \frac{6}{25}\,
p_{Y|X}\brak{1|0} = \frac{45}{50}\,
p_{Y|X}\brak{1|2} = \frac{5}{50}
\end{align}
The desired probability is the probability that a slip drawn at random is marked other than Rs 1,
\begin{align}
&=1-p_X\brak{0}\\
&= p_X(1) + p_X(2)
\end{align}
Using Bayes theorem,
\begin{align}
&= p_Y\brak{0} \times \pr{Y=0 | X=1} + p_Y\brak{1} \times \pr{Y=1|X=2}\\
&=\frac{1}{3} \times \frac{6}{25} + \frac{2}{3} \times \frac{5}{50}\\
&=\frac{11}{75}
\end{align}

\newpage

%\tableofcontents

\bigskip

\renewcommand{\thefigure}{\theenumi}
\renewcommand{\thetable}{\theenumi}
%\renewcommand{\theequation}{\theenumi}

%\begin{abstract}
%%\boldmath
%In this letter, an algorithm for evaluating the exact analytical bit error rate  (BER)  for the piecewise linear (PL) combiner for  multiple relays is presented. Previous results were available only for upto three relays. The algorithm is unique in the sense that  the actual mathematical expressions, that are prohibitively large, need not be explicitly obtained. The diversity gain due to multiple relays is shown through plots of the analytical BER, well supported by simulations. 
%
%\end{abstract}
% IEEEtran.cls defaults to using nonbold math in the Abstract.
% This preserves the distinction between vectors and scalars. However,
% if the journal you are submitting to favors bold math in the abstract,
% then you can use LaTeX's standard command \boldmath at the very start
% of the abstract to achieve this. Many IEEE journals frown on math
% in the abstract anyway.

% Note that keywords are not normally used for peerreview papers.
%\begin{IEEEkeywords}
%Cooperative diversity, decode and forward, piecewise linear
%\end{IEEEkeywords}



% For peer review papers, you can put extra information on the cover
% page as needed:
% \ifCLASSOPTIONpeerreview
% \begin{center} \bfseries EDICS Category: 3-BBND \end{center}
% \fi
%
% For peerreview papers, this IEEEtran command inserts a page break and
% creates the second title. It will be ignored for other modes.
%\IEEEpeerreviewmaketitle




\item 
A die is thrown and a card is selected at random from a deck of 52 playing cards. The probability of getting an even number on the die and a spade card.\\
\solution
%\begin{table}[H]
	\centering
\begin{tabular}{|c|c|c|}
\hline
Random variable &Value &Definition\\ \hline
\multirow{3}{*}{X} &0 &Slips of Rs 1\\
&1 &Slips of Rs 5\\
&2 &Slips of Rs 13\\ \hline
\multirow{2}{*}{Y} &0 &Box A\\
&1 &Box B\\\hline
\end{tabular}
\caption{}
\label{tab:Distribution}
\end{table}
See \tabref{tab:Distribution}.
\begin{align}
p_{Y}\brak{k}= \begin{cases} 
      \frac{1}{3} & {k=0} \\
      \frac{2}{3 }& {k=1} 
   \end{cases}
   \\
p_{Y|X}\brak{0|0} = \frac{19}{25}\, 
p_{Y|X}\brak{0|1} = \frac{6}{25}\,
p_{Y|X}\brak{1|0} = \frac{45}{50}\,
p_{Y|X}\brak{1|2} = \frac{5}{50}
\end{align}
The desired probability is the probability that a slip drawn at random is marked other than Rs 1,
\begin{align}
&=1-p_X\brak{0}\\
&= p_X(1) + p_X(2)
\end{align}
Using Bayes theorem,
\begin{align}
&= p_Y\brak{0} \times \pr{Y=0 | X=1} + p_Y\brak{1} \times \pr{Y=1|X=2}\\
&=\frac{1}{3} \times \frac{6}{25} + \frac{2}{3} \times \frac{5}{50}\\
&=\frac{11}{75}
\end{align}

\newpage

%\tableofcontents

\bigskip

\renewcommand{\thefigure}{\theenumi}
\renewcommand{\thetable}{\theenumi}
%\renewcommand{\theequation}{\theenumi}

%\begin{abstract}
%%\boldmath
%In this letter, an algorithm for evaluating the exact analytical bit error rate  (BER)  for the piecewise linear (PL) combiner for  multiple relays is presented. Previous results were available only for upto three relays. The algorithm is unique in the sense that  the actual mathematical expressions, that are prohibitively large, need not be explicitly obtained. The diversity gain due to multiple relays is shown through plots of the analytical BER, well supported by simulations. 
%
%\end{abstract}
% IEEEtran.cls defaults to using nonbold math in the Abstract.
% This preserves the distinction between vectors and scalars. However,
% if the journal you are submitting to favors bold math in the abstract,
% then you can use LaTeX's standard command \boldmath at the very start
% of the abstract to achieve this. Many IEEE journals frown on math
% in the abstract anyway.

% Note that keywords are not normally used for peerreview papers.
%\begin{IEEEkeywords}
%Cooperative diversity, decode and forward, piecewise linear
%\end{IEEEkeywords}



% For peer review papers, you can put extra information on the cover
% page as needed:
% \ifCLASSOPTIONpeerreview
% \begin{center} \bfseries EDICS Category: 3-BBND \end{center}
% \fi
%
% For peerreview papers, this IEEEtran command inserts a page break and
% creates the second title. It will be ignored for other modes.
%\IEEEpeerreviewmaketitle




\item
If 4-digit numbers greater than 5,000 are randomly formed from the digits 0, 1, 3, 5, and 7, what is the probability of forming a number divisible by 5 when:
\begin{enumerate}
    \item The digits are repeated?
    \item The repetition of digits is not allowed?
\end{enumerate}
\solution
%\begin{table}[H]
	\centering
\begin{tabular}{|c|c|c|}
\hline
Random variable &Value &Definition\\ \hline
\multirow{3}{*}{X} &0 &Slips of Rs 1\\
&1 &Slips of Rs 5\\
&2 &Slips of Rs 13\\ \hline
\multirow{2}{*}{Y} &0 &Box A\\
&1 &Box B\\\hline
\end{tabular}
\caption{}
\label{tab:Distribution}
\end{table}
See \tabref{tab:Distribution}.
\begin{align}
p_{Y}\brak{k}= \begin{cases} 
      \frac{1}{3} & {k=0} \\
      \frac{2}{3 }& {k=1} 
   \end{cases}
   \\
p_{Y|X}\brak{0|0} = \frac{19}{25}\, 
p_{Y|X}\brak{0|1} = \frac{6}{25}\,
p_{Y|X}\brak{1|0} = \frac{45}{50}\,
p_{Y|X}\brak{1|2} = \frac{5}{50}
\end{align}
The desired probability is the probability that a slip drawn at random is marked other than Rs 1,
\begin{align}
&=1-p_X\brak{0}\\
&= p_X(1) + p_X(2)
\end{align}
Using Bayes theorem,
\begin{align}
&= p_Y\brak{0} \times \pr{Y=0 | X=1} + p_Y\brak{1} \times \pr{Y=1|X=2}\\
&=\frac{1}{3} \times \frac{6}{25} + \frac{2}{3} \times \frac{5}{50}\\
&=\frac{11}{75}
\end{align}

\newpage

%\tableofcontents

\bigskip

\renewcommand{\thefigure}{\theenumi}
\renewcommand{\thetable}{\theenumi}
%\renewcommand{\theequation}{\theenumi}

%\begin{abstract}
%%\boldmath
%In this letter, an algorithm for evaluating the exact analytical bit error rate  (BER)  for the piecewise linear (PL) combiner for  multiple relays is presented. Previous results were available only for upto three relays. The algorithm is unique in the sense that  the actual mathematical expressions, that are prohibitively large, need not be explicitly obtained. The diversity gain due to multiple relays is shown through plots of the analytical BER, well supported by simulations. 
%
%\end{abstract}
% IEEEtran.cls defaults to using nonbold math in the Abstract.
% This preserves the distinction between vectors and scalars. However,
% if the journal you are submitting to favors bold math in the abstract,
% then you can use LaTeX's standard command \boldmath at the very start
% of the abstract to achieve this. Many IEEE journals frown on math
% in the abstract anyway.

% Note that keywords are not normally used for peerreview papers.
%\begin{IEEEkeywords}
%Cooperative diversity, decode and forward, piecewise linear
%\end{IEEEkeywords}



% For peer review papers, you can put extra information on the cover
% page as needed:
% \ifCLASSOPTIONpeerreview
% \begin{center} \bfseries EDICS Category: 3-BBND \end{center}
% \fi
%
% For peerreview papers, this IEEEtran command inserts a page break and
% creates the second title. It will be ignored for other modes.
%\IEEEpeerreviewmaketitle




\item Consider the probability space $\brak{\Omega, \mathcal{G}, P}$ where $\Omega = [0,2]$ and $\mathcal{G} = \cbrak{\phi, \Omega, [0,1], (1,2]}$. Let $X$ and $Y$ be two functions on $\Omega$ defined as
\begin{align*}
    X(\omega) = 
    \begin{cases}
        1 & \text{if }\omega \in [0, 1]\\
        2 & \text{if }\omega \in (1, 2]
    \end{cases}
\end{align*}
and
\begin{align*}
    Y(\omega) = 
    \begin{cases}
        2 & \text{if }\omega \in [0, 1.5]\\
        3 & \text{if }\omega \in (1.5, 2].
    \end{cases}
\end{align*}
Then which one of the following statements is true?
\begin{enumerate}
    \item [(A)] $X$ is a random variable with respect to $\mathcal{G}$, but $Y$ is not a random variable with respect to $\mathcal{G}$.
    \item [(B)] $Y$ is a random variable with respect to $\mathcal{G}$, but $X$ is not a random variable with respect to $\mathcal{G}$.
    \item [(C)] Neither $X$ nor $Y$ is a random variable with respect to $\mathcal{G}$.
    \item [(D)] Both $X$ and $Y$ are random variables with respect to $\mathcal{G}$.
\end{enumerate} \hfill (GATE ST 2023)\\
\solution
%\begin{table}[H]
	\centering
\begin{tabular}{|c|c|c|}
\hline
Random variable &Value &Definition\\ \hline
\multirow{3}{*}{X} &0 &Slips of Rs 1\\
&1 &Slips of Rs 5\\
&2 &Slips of Rs 13\\ \hline
\multirow{2}{*}{Y} &0 &Box A\\
&1 &Box B\\\hline
\end{tabular}
\caption{}
\label{tab:Distribution}
\end{table}
See \tabref{tab:Distribution}.
\begin{align}
p_{Y}\brak{k}= \begin{cases} 
      \frac{1}{3} & {k=0} \\
      \frac{2}{3 }& {k=1} 
   \end{cases}
   \\
p_{Y|X}\brak{0|0} = \frac{19}{25}\, 
p_{Y|X}\brak{0|1} = \frac{6}{25}\,
p_{Y|X}\brak{1|0} = \frac{45}{50}\,
p_{Y|X}\brak{1|2} = \frac{5}{50}
\end{align}
The desired probability is the probability that a slip drawn at random is marked other than Rs 1,
\begin{align}
&=1-p_X\brak{0}\\
&= p_X(1) + p_X(2)
\end{align}
Using Bayes theorem,
\begin{align}
&= p_Y\brak{0} \times \pr{Y=0 | X=1} + p_Y\brak{1} \times \pr{Y=1|X=2}\\
&=\frac{1}{3} \times \frac{6}{25} + \frac{2}{3} \times \frac{5}{50}\\
&=\frac{11}{75}
\end{align}

\newpage

%\tableofcontents

\bigskip

\renewcommand{\thefigure}{\theenumi}
\renewcommand{\thetable}{\theenumi}
%\renewcommand{\theequation}{\theenumi}

%\begin{abstract}
%%\boldmath
%In this letter, an algorithm for evaluating the exact analytical bit error rate  (BER)  for the piecewise linear (PL) combiner for  multiple relays is presented. Previous results were available only for upto three relays. The algorithm is unique in the sense that  the actual mathematical expressions, that are prohibitively large, need not be explicitly obtained. The diversity gain due to multiple relays is shown through plots of the analytical BER, well supported by simulations. 
%
%\end{abstract}
% IEEEtran.cls defaults to using nonbold math in the Abstract.
% This preserves the distinction between vectors and scalars. However,
% if the journal you are submitting to favors bold math in the abstract,
% then you can use LaTeX's standard command \boldmath at the very start
% of the abstract to achieve this. Many IEEE journals frown on math
% in the abstract anyway.

% Note that keywords are not normally used for peerreview papers.
%\begin{IEEEkeywords}
%Cooperative diversity, decode and forward, piecewise linear
%\end{IEEEkeywords}



% For peer review papers, you can put extra information on the cover
% page as needed:
% \ifCLASSOPTIONpeerreview
% \begin{center} \bfseries EDICS Category: 3-BBND \end{center}
% \fi
%
% For peerreview papers, this IEEEtran command inserts a page break and
% creates the second title. It will be ignored for other modes.
%\IEEEpeerreviewmaketitle




	\item  A die is loaded in such a way that each odd number is twice as likely to occur as
each even number. Find $P(G)$, where $G$ is the event that a number greater than
3 occurs on a single roll of the die.
\\
\solution
		%\begin{table}[H]
	\centering
\begin{tabular}{|c|c|c|}
\hline
Random variable &Value &Definition\\ \hline
\multirow{3}{*}{X} &0 &Slips of Rs 1\\
&1 &Slips of Rs 5\\
&2 &Slips of Rs 13\\ \hline
\multirow{2}{*}{Y} &0 &Box A\\
&1 &Box B\\\hline
\end{tabular}
\caption{}
\label{tab:Distribution}
\end{table}
See \tabref{tab:Distribution}.
\begin{align}
p_{Y}\brak{k}= \begin{cases} 
      \frac{1}{3} & {k=0} \\
      \frac{2}{3 }& {k=1} 
   \end{cases}
   \\
p_{Y|X}\brak{0|0} = \frac{19}{25}\, 
p_{Y|X}\brak{0|1} = \frac{6}{25}\,
p_{Y|X}\brak{1|0} = \frac{45}{50}\,
p_{Y|X}\brak{1|2} = \frac{5}{50}
\end{align}
The desired probability is the probability that a slip drawn at random is marked other than Rs 1,
\begin{align}
&=1-p_X\brak{0}\\
&= p_X(1) + p_X(2)
\end{align}
Using Bayes theorem,
\begin{align}
&= p_Y\brak{0} \times \pr{Y=0 | X=1} + p_Y\brak{1} \times \pr{Y=1|X=2}\\
&=\frac{1}{3} \times \frac{6}{25} + \frac{2}{3} \times \frac{5}{50}\\
&=\frac{11}{75}
\end{align}

\newpage

%\tableofcontents

\bigskip

\renewcommand{\thefigure}{\theenumi}
\renewcommand{\thetable}{\theenumi}
%\renewcommand{\theequation}{\theenumi}

%\begin{abstract}
%%\boldmath
%In this letter, an algorithm for evaluating the exact analytical bit error rate  (BER)  for the piecewise linear (PL) combiner for  multiple relays is presented. Previous results were available only for upto three relays. The algorithm is unique in the sense that  the actual mathematical expressions, that are prohibitively large, need not be explicitly obtained. The diversity gain due to multiple relays is shown through plots of the analytical BER, well supported by simulations. 
%
%\end{abstract}
% IEEEtran.cls defaults to using nonbold math in the Abstract.
% This preserves the distinction between vectors and scalars. However,
% if the journal you are submitting to favors bold math in the abstract,
% then you can use LaTeX's standard command \boldmath at the very start
% of the abstract to achieve this. Many IEEE journals frown on math
% in the abstract anyway.

% Note that keywords are not normally used for peerreview papers.
%\begin{IEEEkeywords}
%Cooperative diversity, decode and forward, piecewise linear
%\end{IEEEkeywords}



% For peer review papers, you can put extra information on the cover
% page as needed:
% \ifCLASSOPTIONpeerreview
% \begin{center} \bfseries EDICS Category: 3-BBND \end{center}
% \fi
%
% For peerreview papers, this IEEEtran command inserts a page break and
% creates the second title. It will be ignored for other modes.
%\IEEEpeerreviewmaketitle




	\item All the jacks, queens and kings are removed from a deck of 52 playing cards. The remaining cards are well shuffled and then one card is drawn at random. Giving ace a value 1 similar value for other cards, find the probability that the card has a value 
		\begin{enumerate}
			\item 7
			\item greater than 7
			\item less than 7
		\end{enumerate}
		%Number of cards left after removing all jacks, queens and kings 
\begin{align}
N	= 52 - 4\times 3
	= 40
\end{align}
%\begin{table}[H]
%\def\arraystretch{1.2}
%\begin{tabular}{|c|c|c|}
%\hline
%	\textbf{Parameter} &\textbf{Value} &\textbf{Description}\\ \hline
%	$X$ &1-10 &Represents the value of the card picked \\ \hline
%\end{tabular}
%\end{table}
Let $1 \le X \le 10$ be the value of the card picked.  Then,
\begin{align}
	p_X(k) &= \Pr(X=k)\ \forall\ 1 \leq k \leq 10\\
	&= \frac{4\times 1}{40}\\
	&= \frac{1}{10}\\
	\therefore p_X(k) &= 
	\begin{cases}
		\frac{1}{10} & 1 \leq k \leq 10\\
		0 & \text{otherwise}
	\end{cases}
\end{align}
and
\begin{align}
	F_{X}(k) &= \sum_{m=0}^{k}p_{X}(m) \quad 1 \leq k \leq 10\\
	&= \frac{k}{10}\\
	\therefore F_{X}(k) &= 
	\begin{cases}
		0 & k \leq 0\\
		\frac{k}{10} & 1\leq k \leq 10\\
		1 & k > 10 
	\end{cases}
\end{align}
\begin{enumerate}
	\item Probability that card has value equal to 7 is
		\begin{align}
			 p_{X}(7)
			= \frac{1}{10}
		\end{align}
	\item Probability that card has value greater than 7 is
		\begin{align}
			1 - F_X(7)
			&= 1 - \frac{7}{10}
			\\
			&= \frac{3}{10}
		\end{align}
	\item Probability that card has value less than 7 is
		\begin{align}
			 F_{X}(6)
			=\frac{6}{10}
		\end{align}
\end{enumerate}

  \item A Lot consists of 48 mobile phones of which 42 are good, 3 have only minor defects and 3 have major defects.Varnika will buy a phone if it is good but the trader will only buy a mobile if it has no major defects. One phone is selected at random from the lot. What is the probability that it is
\begin{enumerate}
	\item acceptable to Varnika?
            \item acceptable to the trader?
\end{enumerate}
\solution
	%\begin{table}[H]
	\centering
\begin{tabular}{|c|c|c|}
\hline
Random variable &Value &Definition\\ \hline
\multirow{3}{*}{X} &0 &Slips of Rs 1\\
&1 &Slips of Rs 5\\
&2 &Slips of Rs 13\\ \hline
\multirow{2}{*}{Y} &0 &Box A\\
&1 &Box B\\\hline
\end{tabular}
\caption{}
\label{tab:Distribution}
\end{table}
See \tabref{tab:Distribution}.
\begin{align}
p_{Y}\brak{k}= \begin{cases} 
      \frac{1}{3} & {k=0} \\
      \frac{2}{3 }& {k=1} 
   \end{cases}
   \\
p_{Y|X}\brak{0|0} = \frac{19}{25}\, 
p_{Y|X}\brak{0|1} = \frac{6}{25}\,
p_{Y|X}\brak{1|0} = \frac{45}{50}\,
p_{Y|X}\brak{1|2} = \frac{5}{50}
\end{align}
The desired probability is the probability that a slip drawn at random is marked other than Rs 1,
\begin{align}
&=1-p_X\brak{0}\\
&= p_X(1) + p_X(2)
\end{align}
Using Bayes theorem,
\begin{align}
&= p_Y\brak{0} \times \pr{Y=0 | X=1} + p_Y\brak{1} \times \pr{Y=1|X=2}\\
&=\frac{1}{3} \times \frac{6}{25} + \frac{2}{3} \times \frac{5}{50}\\
&=\frac{11}{75}
\end{align}

\newpage

%\tableofcontents

\bigskip

\renewcommand{\thefigure}{\theenumi}
\renewcommand{\thetable}{\theenumi}
%\renewcommand{\theequation}{\theenumi}

%\begin{abstract}
%%\boldmath
%In this letter, an algorithm for evaluating the exact analytical bit error rate  (BER)  for the piecewise linear (PL) combiner for  multiple relays is presented. Previous results were available only for upto three relays. The algorithm is unique in the sense that  the actual mathematical expressions, that are prohibitively large, need not be explicitly obtained. The diversity gain due to multiple relays is shown through plots of the analytical BER, well supported by simulations. 
%
%\end{abstract}
% IEEEtran.cls defaults to using nonbold math in the Abstract.
% This preserves the distinction between vectors and scalars. However,
% if the journal you are submitting to favors bold math in the abstract,
% then you can use LaTeX's standard command \boldmath at the very start
% of the abstract to achieve this. Many IEEE journals frown on math
% in the abstract anyway.

% Note that keywords are not normally used for peerreview papers.
%\begin{IEEEkeywords}
%Cooperative diversity, decode and forward, piecewise linear
%\end{IEEEkeywords}



% For peer review papers, you can put extra information on the cover
% page as needed:
% \ifCLASSOPTIONpeerreview
% \begin{center} \bfseries EDICS Category: 3-BBND \end{center}
% \fi
%
% For peerreview papers, this IEEEtran command inserts a page break and
% creates the second title. It will be ignored for other modes.
%\IEEEpeerreviewmaketitle




 \item A student says that if you throw a die, it will show up 1 or not 1. Therefore, the probability of getting 1 and the probability of getting 'not 1' each is equal to $\frac{1}{2}$. Is this correct? Give reasons.\\
 \solution
        %\begin{table}[H]
	\centering
\begin{tabular}{|c|c|c|}
\hline
Random variable &Value &Definition\\ \hline
\multirow{3}{*}{X} &0 &Slips of Rs 1\\
&1 &Slips of Rs 5\\
&2 &Slips of Rs 13\\ \hline
\multirow{2}{*}{Y} &0 &Box A\\
&1 &Box B\\\hline
\end{tabular}
\caption{}
\label{tab:Distribution}
\end{table}
See \tabref{tab:Distribution}.
\begin{align}
p_{Y}\brak{k}= \begin{cases} 
      \frac{1}{3} & {k=0} \\
      \frac{2}{3 }& {k=1} 
   \end{cases}
   \\
p_{Y|X}\brak{0|0} = \frac{19}{25}\, 
p_{Y|X}\brak{0|1} = \frac{6}{25}\,
p_{Y|X}\brak{1|0} = \frac{45}{50}\,
p_{Y|X}\brak{1|2} = \frac{5}{50}
\end{align}
The desired probability is the probability that a slip drawn at random is marked other than Rs 1,
\begin{align}
&=1-p_X\brak{0}\\
&= p_X(1) + p_X(2)
\end{align}
Using Bayes theorem,
\begin{align}
&= p_Y\brak{0} \times \pr{Y=0 | X=1} + p_Y\brak{1} \times \pr{Y=1|X=2}\\
&=\frac{1}{3} \times \frac{6}{25} + \frac{2}{3} \times \frac{5}{50}\\
&=\frac{11}{75}
\end{align}

\newpage

%\tableofcontents

\bigskip

\renewcommand{\thefigure}{\theenumi}
\renewcommand{\thetable}{\theenumi}
%\renewcommand{\theequation}{\theenumi}

%\begin{abstract}
%%\boldmath
%In this letter, an algorithm for evaluating the exact analytical bit error rate  (BER)  for the piecewise linear (PL) combiner for  multiple relays is presented. Previous results were available only for upto three relays. The algorithm is unique in the sense that  the actual mathematical expressions, that are prohibitively large, need not be explicitly obtained. The diversity gain due to multiple relays is shown through plots of the analytical BER, well supported by simulations. 
%
%\end{abstract}
% IEEEtran.cls defaults to using nonbold math in the Abstract.
% This preserves the distinction between vectors and scalars. However,
% if the journal you are submitting to favors bold math in the abstract,
% then you can use LaTeX's standard command \boldmath at the very start
% of the abstract to achieve this. Many IEEE journals frown on math
% in the abstract anyway.

% Note that keywords are not normally used for peerreview papers.
%\begin{IEEEkeywords}
%Cooperative diversity, decode and forward, piecewise linear
%\end{IEEEkeywords}



% For peer review papers, you can put extra information on the cover
% page as needed:
% \ifCLASSOPTIONpeerreview
% \begin{center} \bfseries EDICS Category: 3-BBND \end{center}
% \fi
%
% For peerreview papers, this IEEEtran command inserts a page break and
% creates the second title. It will be ignored for other modes.
%\IEEEpeerreviewmaketitle




   \item Four candidates A, B, C, D have ap-
plied for the assignment to coach a school cricket
team. If A is twice as likely to be selected as B, and
B and C are given about the same chance of being
selected, while C is twice as likely to be selected
as D, what are the probabilities that
\begin{enumerate}
\item C will be selected?
\item A will not be selected?
\end{enumerate}
	%\begin{table}[H]
	\centering
\begin{tabular}{|c|c|c|}
\hline
Random variable &Value &Definition\\ \hline
\multirow{3}{*}{X} &0 &Slips of Rs 1\\
&1 &Slips of Rs 5\\
&2 &Slips of Rs 13\\ \hline
\multirow{2}{*}{Y} &0 &Box A\\
&1 &Box B\\\hline
\end{tabular}
\caption{}
\label{tab:Distribution}
\end{table}
See \tabref{tab:Distribution}.
\begin{align}
p_{Y}\brak{k}= \begin{cases} 
      \frac{1}{3} & {k=0} \\
      \frac{2}{3 }& {k=1} 
   \end{cases}
   \\
p_{Y|X}\brak{0|0} = \frac{19}{25}\, 
p_{Y|X}\brak{0|1} = \frac{6}{25}\,
p_{Y|X}\brak{1|0} = \frac{45}{50}\,
p_{Y|X}\brak{1|2} = \frac{5}{50}
\end{align}
The desired probability is the probability that a slip drawn at random is marked other than Rs 1,
\begin{align}
&=1-p_X\brak{0}\\
&= p_X(1) + p_X(2)
\end{align}
Using Bayes theorem,
\begin{align}
&= p_Y\brak{0} \times \pr{Y=0 | X=1} + p_Y\brak{1} \times \pr{Y=1|X=2}\\
&=\frac{1}{3} \times \frac{6}{25} + \frac{2}{3} \times \frac{5}{50}\\
&=\frac{11}{75}
\end{align}

\newpage

%\tableofcontents

\bigskip

\renewcommand{\thefigure}{\theenumi}
\renewcommand{\thetable}{\theenumi}
%\renewcommand{\theequation}{\theenumi}

%\begin{abstract}
%%\boldmath
%In this letter, an algorithm for evaluating the exact analytical bit error rate  (BER)  for the piecewise linear (PL) combiner for  multiple relays is presented. Previous results were available only for upto three relays. The algorithm is unique in the sense that  the actual mathematical expressions, that are prohibitively large, need not be explicitly obtained. The diversity gain due to multiple relays is shown through plots of the analytical BER, well supported by simulations. 
%
%\end{abstract}
% IEEEtran.cls defaults to using nonbold math in the Abstract.
% This preserves the distinction between vectors and scalars. However,
% if the journal you are submitting to favors bold math in the abstract,
% then you can use LaTeX's standard command \boldmath at the very start
% of the abstract to achieve this. Many IEEE journals frown on math
% in the abstract anyway.

% Note that keywords are not normally used for peerreview papers.
%\begin{IEEEkeywords}
%Cooperative diversity, decode and forward, piecewise linear
%\end{IEEEkeywords}



% For peer review papers, you can put extra information on the cover
% page as needed:
% \ifCLASSOPTIONpeerreview
% \begin{center} \bfseries EDICS Category: 3-BBND \end{center}
% \fi
%
% For peerreview papers, this IEEEtran command inserts a page break and
% creates the second title. It will be ignored for other modes.
%\IEEEpeerreviewmaketitle




 \item A bag contain 24 balls of which $x$ balls are red, $2x$ are white and $3x$ are blue. A ball is selected at random, What is the probability that it is
\begin{enumerate}[label=\alph*)]
\item not red ?
\item white ?
\end{enumerate}
%\begin{table}[H]
	\centering
\begin{tabular}{|c|c|c|}
\hline
Random variable &Value &Definition\\ \hline
\multirow{3}{*}{X} &0 &Slips of Rs 1\\
&1 &Slips of Rs 5\\
&2 &Slips of Rs 13\\ \hline
\multirow{2}{*}{Y} &0 &Box A\\
&1 &Box B\\\hline
\end{tabular}
\caption{}
\label{tab:Distribution}
\end{table}
See \tabref{tab:Distribution}.
\begin{align}
p_{Y}\brak{k}= \begin{cases} 
      \frac{1}{3} & {k=0} \\
      \frac{2}{3 }& {k=1} 
   \end{cases}
   \\
p_{Y|X}\brak{0|0} = \frac{19}{25}\, 
p_{Y|X}\brak{0|1} = \frac{6}{25}\,
p_{Y|X}\brak{1|0} = \frac{45}{50}\,
p_{Y|X}\brak{1|2} = \frac{5}{50}
\end{align}
The desired probability is the probability that a slip drawn at random is marked other than Rs 1,
\begin{align}
&=1-p_X\brak{0}\\
&= p_X(1) + p_X(2)
\end{align}
Using Bayes theorem,
\begin{align}
&= p_Y\brak{0} \times \pr{Y=0 | X=1} + p_Y\brak{1} \times \pr{Y=1|X=2}\\
&=\frac{1}{3} \times \frac{6}{25} + \frac{2}{3} \times \frac{5}{50}\\
&=\frac{11}{75}
\end{align}

\newpage

%\tableofcontents

\bigskip

\renewcommand{\thefigure}{\theenumi}
\renewcommand{\thetable}{\theenumi}
%\renewcommand{\theequation}{\theenumi}

%\begin{abstract}
%%\boldmath
%In this letter, an algorithm for evaluating the exact analytical bit error rate  (BER)  for the piecewise linear (PL) combiner for  multiple relays is presented. Previous results were available only for upto three relays. The algorithm is unique in the sense that  the actual mathematical expressions, that are prohibitively large, need not be explicitly obtained. The diversity gain due to multiple relays is shown through plots of the analytical BER, well supported by simulations. 
%
%\end{abstract}
% IEEEtran.cls defaults to using nonbold math in the Abstract.
% This preserves the distinction between vectors and scalars. However,
% if the journal you are submitting to favors bold math in the abstract,
% then you can use LaTeX's standard command \boldmath at the very start
% of the abstract to achieve this. Many IEEE journals frown on math
% in the abstract anyway.

% Note that keywords are not normally used for peerreview papers.
%\begin{IEEEkeywords}
%Cooperative diversity, decode and forward, piecewise linear
%\end{IEEEkeywords}



% For peer review papers, you can put extra information on the cover
% page as needed:
% \ifCLASSOPTIONpeerreview
% \begin{center} \bfseries EDICS Category: 3-BBND \end{center}
% \fi
%
% For peerreview papers, this IEEEtran command inserts a page break and
% creates the second title. It will be ignored for other modes.
%\IEEEpeerreviewmaketitle




If the letters of the word ASSASSINATION are arranged at random. Find the Probability that
\begin{enumerate}[label=(\alph*)]
\item Four $S's$ come consecutively in the word
\item Two  $I's$ and two $N's$ come together
\item All $A's$ are not coming together
\item No two $A's$ are coming together
\end{enumerate}
%\begin{table}[H]
	\centering
\begin{tabular}{|c|c|c|}
\hline
Random variable &Value &Definition\\ \hline
\multirow{3}{*}{X} &0 &Slips of Rs 1\\
&1 &Slips of Rs 5\\
&2 &Slips of Rs 13\\ \hline
\multirow{2}{*}{Y} &0 &Box A\\
&1 &Box B\\\hline
\end{tabular}
\caption{}
\label{tab:Distribution}
\end{table}
See \tabref{tab:Distribution}.
\begin{align}
p_{Y}\brak{k}= \begin{cases} 
      \frac{1}{3} & {k=0} \\
      \frac{2}{3 }& {k=1} 
   \end{cases}
   \\
p_{Y|X}\brak{0|0} = \frac{19}{25}\, 
p_{Y|X}\brak{0|1} = \frac{6}{25}\,
p_{Y|X}\brak{1|0} = \frac{45}{50}\,
p_{Y|X}\brak{1|2} = \frac{5}{50}
\end{align}
The desired probability is the probability that a slip drawn at random is marked other than Rs 1,
\begin{align}
&=1-p_X\brak{0}\\
&= p_X(1) + p_X(2)
\end{align}
Using Bayes theorem,
\begin{align}
&= p_Y\brak{0} \times \pr{Y=0 | X=1} + p_Y\brak{1} \times \pr{Y=1|X=2}\\
&=\frac{1}{3} \times \frac{6}{25} + \frac{2}{3} \times \frac{5}{50}\\
&=\frac{11}{75}
\end{align}

\newpage

%\tableofcontents

\bigskip

\renewcommand{\thefigure}{\theenumi}
\renewcommand{\thetable}{\theenumi}
%\renewcommand{\theequation}{\theenumi}

%\begin{abstract}
%%\boldmath
%In this letter, an algorithm for evaluating the exact analytical bit error rate  (BER)  for the piecewise linear (PL) combiner for  multiple relays is presented. Previous results were available only for upto three relays. The algorithm is unique in the sense that  the actual mathematical expressions, that are prohibitively large, need not be explicitly obtained. The diversity gain due to multiple relays is shown through plots of the analytical BER, well supported by simulations. 
%
%\end{abstract}
% IEEEtran.cls defaults to using nonbold math in the Abstract.
% This preserves the distinction between vectors and scalars. However,
% if the journal you are submitting to favors bold math in the abstract,
% then you can use LaTeX's standard command \boldmath at the very start
% of the abstract to achieve this. Many IEEE journals frown on math
% in the abstract anyway.

% Note that keywords are not normally used for peerreview papers.
%\begin{IEEEkeywords}
%Cooperative diversity, decode and forward, piecewise linear
%\end{IEEEkeywords}



% For peer review papers, you can put extra information on the cover
% page as needed:
% \ifCLASSOPTIONpeerreview
% \begin{center} \bfseries EDICS Category: 3-BBND \end{center}
% \fi
%
% For peerreview papers, this IEEEtran command inserts a page break and
% creates the second title. It will be ignored for other modes.
%\IEEEpeerreviewmaketitle




	\item One urn contains two black balls (labelled B1 and B2) and one white ball. A
	second urn contains one black ball and two white balls (labelled W1 and W2).
	Suppose the following experiment is performed. One of the two urns is chosen
	at random. Next a ball is randomly chosen from the urn. Then a second ball is
	chosen at random from the same urn without replacing the first ball.
	
	\begin{enumerate}
	\item What is the probability that two black balls are chosen?
	
	\item What is the probability that two balls of opposite colour are chosen?
	\end{enumerate}
	\solution
	%\begin{align}
    \label{eq:12.13.6.18.1}
	\because	\pr{A|B} &> \pr{A},\
\frac{\pr{AB}}{\pr{B}} > \pr{A}
\\
    \label{eq:12.13.6.18.2}
	\implies \pr{AB} &> \pr{A}\pr{B}
	\\
	\text{or, } \frac{\pr{AB}}{\pr{A}} &=\pr{B|A} > \pr{A}
\end{align}

\end{enumerate}

	\item 
The number lock of a suitcase has 4 wheels each labelled with ten digits i.e. from 0 to 9.The lock opens with a sequence of four digits with no repeats.What is the probability of a person getting the right sequence to open the suitcase.
\\
\solution
		%\begin{enumerate}[label=\thesection.\arabic*,ref=\thesection.\theenumi]
	\item One card is drawn from a well-shuffled deck of 52 cards. Find the probability of getting
\begin{enumerate}
\item A king of red colour 
\item A face card 
\item A red face card
\item The jack of hearts
\item A spade
\item The queen of diamonds

\end{enumerate}
\solution
		%\begin{table}[H]
	\centering
\begin{tabular}{|c|c|c|}
\hline
Random variable &Value &Definition\\ \hline
\multirow{3}{*}{X} &0 &Slips of Rs 1\\
&1 &Slips of Rs 5\\
&2 &Slips of Rs 13\\ \hline
\multirow{2}{*}{Y} &0 &Box A\\
&1 &Box B\\\hline
\end{tabular}
\caption{}
\label{tab:Distribution}
\end{table}
See \tabref{tab:Distribution}.
\begin{align}
p_{Y}\brak{k}= \begin{cases} 
      \frac{1}{3} & {k=0} \\
      \frac{2}{3 }& {k=1} 
   \end{cases}
   \\
p_{Y|X}\brak{0|0} = \frac{19}{25}\, 
p_{Y|X}\brak{0|1} = \frac{6}{25}\,
p_{Y|X}\brak{1|0} = \frac{45}{50}\,
p_{Y|X}\brak{1|2} = \frac{5}{50}
\end{align}
The desired probability is the probability that a slip drawn at random is marked other than Rs 1,
\begin{align}
&=1-p_X\brak{0}\\
&= p_X(1) + p_X(2)
\end{align}
Using Bayes theorem,
\begin{align}
&= p_Y\brak{0} \times \pr{Y=0 | X=1} + p_Y\brak{1} \times \pr{Y=1|X=2}\\
&=\frac{1}{3} \times \frac{6}{25} + \frac{2}{3} \times \frac{5}{50}\\
&=\frac{11}{75}
\end{align}

\newpage

%\tableofcontents

\bigskip

\renewcommand{\thefigure}{\theenumi}
\renewcommand{\thetable}{\theenumi}
%\renewcommand{\theequation}{\theenumi}

%\begin{abstract}
%%\boldmath
%In this letter, an algorithm for evaluating the exact analytical bit error rate  (BER)  for the piecewise linear (PL) combiner for  multiple relays is presented. Previous results were available only for upto three relays. The algorithm is unique in the sense that  the actual mathematical expressions, that are prohibitively large, need not be explicitly obtained. The diversity gain due to multiple relays is shown through plots of the analytical BER, well supported by simulations. 
%
%\end{abstract}
% IEEEtran.cls defaults to using nonbold math in the Abstract.
% This preserves the distinction between vectors and scalars. However,
% if the journal you are submitting to favors bold math in the abstract,
% then you can use LaTeX's standard command \boldmath at the very start
% of the abstract to achieve this. Many IEEE journals frown on math
% in the abstract anyway.

% Note that keywords are not normally used for peerreview papers.
%\begin{IEEEkeywords}
%Cooperative diversity, decode and forward, piecewise linear
%\end{IEEEkeywords}



% For peer review papers, you can put extra information on the cover
% page as needed:
% \ifCLASSOPTIONpeerreview
% \begin{center} \bfseries EDICS Category: 3-BBND \end{center}
% \fi
%
% For peerreview papers, this IEEEtran command inserts a page break and
% creates the second title. It will be ignored for other modes.
%\IEEEpeerreviewmaketitle




	\item Five cards—the ten, jack, queen, king and ace of diamonds, are well-shuffled with their face downwards. One card is then picked up at random.
\begin{enumerate}
\item
What is the probability that the card is the queen? 
\item
If the queen is drawn and put aside, what is the probability that the second card picked up is (a) an ace? (b) a queen?\\
\end{enumerate}
\solution
		%\begin{enumerate}[label=\thesection.\arabic*,ref=\thesection.\theenumi]
	\item One card is drawn from a well-shuffled deck of 52 cards. Find the probability of getting
\begin{enumerate}
\item A king of red colour 
\item A face card 
\item A red face card
\item The jack of hearts
\item A spade
\item The queen of diamonds

\end{enumerate}
\solution
		%\input{ncert/10/15/1/14/main.tex}
	\item Five cards—the ten, jack, queen, king and ace of diamonds, are well-shuffled with their face downwards. One card is then picked up at random.
\begin{enumerate}
\item
What is the probability that the card is the queen? 
\item
If the queen is drawn and put aside, what is the probability that the second card picked up is (a) an ace? (b) a queen?\\
\end{enumerate}
\solution
		%\input{ncert/10/15/1/15/defs.tex}
	\item A bag contains $5$ red balls and some blue balls. If the probability of drawing a blue ball is double that if a red ball, determine the number of blue balls in the bag. 
		\\
\solution
		%\input{ncert/10/15/2/3/defs.tex}
	\item A card is selected from a pack of 52 cards.
 \begin{enumerate}[label=(\alph*)] 
                 \item How many points are there in the sample space?
                 \item Calculate the probability that the card is an ace of spades.
                 \item Calculate the probability that the card is (i) an ace and (ii) black card.
 \end{enumerate}
\solution
		%\input{ncert/11/16/3/4/main.tex}
\item Four cards are drawn from a well-shuffled deck of 52 cards. What is the probability of obtaining 3 diamonds and one spade.
\\
\solution
		%\input{ncert/11/16/4/2/defs.tex}
\item In a certain lottery 10,000 tickets are sold and ten equal prizes are awarded. What is the probability of not getting a prize if you buy (a) one ticket (b) two tickets (c) 10 tickets ?	
\\
\solution
		%\input{ncert/11/16/4/4/defs.tex}
		%
\item 
Out of 100 students, two sections of 40 and 60 are formed. If you and your friend are among the 100 students, what is the probability that
\begin{enumerate}
\item you both enter the same section?
\item you both enter the different sections?
\end{enumerate}
\solution
		%\input{ncert/11/16/4/5/defs.tex}
	\item 
The number lock of a suitcase has 4 wheels each labelled with ten digits i.e. from 0 to 9.The lock opens with a sequence of four digits with no repeats.What is the probability of a person getting the right sequence to open the suitcase.
\\
\solution
		%\input{ncert/11/16/4/10/defs.tex}
		%
\item 
Two cards are drawn at random and without replacement from a pack of 52 playing cards. Find the probability that both the cards are black.
\\
\solution
		%\input{ncert/12/13/2/2/defs.tex}
		\item A box of oranges is inspected by examining three randomly selected oranges drawn without replacement. If all the three oranges are good, the box is approved for sale, otherwise, it is rejected. Find the probability that a box containing 15 oranges out of which 12 are good and 3 are bad ones will be approved for sale.
		\label{ncert/12/13/2/3/defs.tex}
		\item Two balls are drawn at random with replacement from a box containing 10 black and 8 red balls. Find the probability that
		\label{ncert/12/13/2/12}
\begin{enumerate}
\item both balls are red.
\item first ball is black and second is red.
\item one of them is black and other is red.
\end{enumerate}

\item In a hostel, 60\% of the students read Hindi newspaper, 40\% read English newspaper and 20\% read both Hindi and English newspapers. A student is selected at random.
		\label{ncert/12/13/2/15}
\begin{enumerate}
\item Find the probability that she reads neither Hindi nor English newspapers.
\item If she reads Hindi newspaper, find the probability that she reads English newspaper.
\item If she reads English newspaper, find the probability that she reads Hindi newspaper.\\
\end{enumerate}
\item The probability of obtaining an even prime number on each die, when a pair of dice is rolled is 
\begin{enumerate}
    \item $0$ 
    
    \item $\frac{1}{3}$ 
    
    \item $\frac{1}{12}$ 
    
    \item $\frac{1}{36}$ 
\end{enumerate}
\solution
		%\input{ncert/12/13/2/17/defs.tex}
	\item A bag contains 4 red and 4 black balls, another bag contains 2 red and 6 black balls. One of the two bags is selected at random and a ball is drawn from the bag which is found to be red. Find the probability that the ball is drawn from the first bag.
\\
\solution
		%\input{ncert/12/13/3/2/main.tex}
  \item
  Cards with numbers 2 to 101 are placed in a box. A card is selected at random.Find the probability that the card has
\begin{enumerate}[label=(\roman*)]
	\item an even number 
	\item a square number
\end{enumerate}
\solution
%\input{exemplar/10/13/3/32/main.tex}
\item
The king, queen and jack of clubs are removed from a deck of 52 playing cards and then well shuffled. Now one card is drawn at random from the remaining cards.  Determine the probability that the card is
\begin{enumerate}[label=(\roman*)]
\item a club
\item 10 of hearts
\end{enumerate}
\solution
%\input{exemplar/10/13/3/29/main.tex}
\item A team of medical students doing their internship have to assist during surgeries
at a city hospital. The probabilities of surgeries rated as very complex, complex,
routine, simple or very simple are respectively, 0.15, 0.20, 0.31, 0.26, .08. Find
the probabilities that a particular surgery will be rated
\begin{enumerate}
	\item complex or very complex;
	\item neither very complex nor very simple;
	\item routine or complex
	\item routine or simple
\end{enumerate}
\solution
%\input{exemplar/11/16/3/8(1)/main.tex}
\item A card is selected from a pack of 52 cards.
\begin{enumerate}[label=(\alph*)]
    \item How many points are there in the sample space?
    \item Calculate the probability that the card is an ace of spades.
    \item Calculate the probability that the card is (i) an ace and (ii) black card.
\end{enumerate}
\solution
%\input{exemplar/11/16/3/4/main2.tex}
\item The probability that a non leap year selected at random will contain 53 sundays.
\\
\solution
%\input{exemplar/10/13/1/19/main.tex}
\item One of the four persons John, Rita, Aslam or Gurpreet will be promoted next
month. Consequently the sample space consists of four elementary outcomes
S = {John promoted, Rita promoted, Aslam promoted, Gurpreet promoted}
You are told that the chances of John’s promotion is same as that of Gurpreet,
Rita’s chances of promotion are twice as likely as Johns. Aslam’s chances are
four times that of John.
\begin{enumerate}
	\item Determine
	\begin{enumerate}
		\item P (John promoted)
		\item P (Rita promoted)
		\item P (Aslam promoted)
		\item P (Gurpreet promoted)
	\end{enumerate}
	\item If A = {John promoted or Gurpreet promoted}, find P (A).
\end{enumerate}
\solution
%\input{exemplar/11/16/3/10/main.tex}
\item A card is drawn from a deck of 52 cards. Find the probability of getting a king or a heart or a red card.\\
\solution
%\input{exemplar/11/16/3/15/main.tex}
\item The probability that a student will pass his examination is 0.73, the probability of
the student getting a compartment is 0.13, and the probability that the student will
either pass or get compartment is 0.96. State True or False.\\
\solution
%\input{exemplar/11/16/3/31/main.tex}
\item A card is selected from a pack of 52 cards\\
\begin{enumerate}[label=(\alph*)]
\item How many points are there in the sample space?
\item Calculate the probability that the cards is an ace of spades.
\item Calculate the probability that the card is (i) an ace (ii)black card.\\
\end{enumerate}
%\input{ncert/11/16/3/4_1/Prob_4.tex}
\item In a non-leap year, the probability of having 53 tuesdays or 53 wednesdays is\\
\solution
%\input{exemplar/11/16/3/18/main.tex}
\item There are 1000 sealed envelopes in a box, 10 of them contain a cash prize of
Rs 100 each, 100 of them contain a cash prize of Rs 50 each and 200 of them
contain a cash prize of Rs 10 each and rest do not contain any cash prize. If they
are well shuffled and an envelope is picked up out, what is the probability that it
contains no cash prize?\\
\solution
%\input{exemplar/10/13/3/34/main.tex}
\item 
A die is thrown and a card is selected at random from a deck of 52 playing cards. The probability of getting an even number on the die and a spade card.\\
\solution
%\input{exemplar/12/13/3/78/main.tex}
\item
If 4-digit numbers greater than 5,000 are randomly formed from the digits 0, 1, 3, 5, and 7, what is the probability of forming a number divisible by 5 when:
\begin{enumerate}
    \item The digits are repeated?
    \item The repetition of digits is not allowed?
\end{enumerate}
\solution
%\input{ncert/11/16/4/9/main.tex}
\item Consider the probability space $\brak{\Omega, \mathcal{G}, P}$ where $\Omega = [0,2]$ and $\mathcal{G} = \cbrak{\phi, \Omega, [0,1], (1,2]}$. Let $X$ and $Y$ be two functions on $\Omega$ defined as
\begin{align*}
    X(\omega) = 
    \begin{cases}
        1 & \text{if }\omega \in [0, 1]\\
        2 & \text{if }\omega \in (1, 2]
    \end{cases}
\end{align*}
and
\begin{align*}
    Y(\omega) = 
    \begin{cases}
        2 & \text{if }\omega \in [0, 1.5]\\
        3 & \text{if }\omega \in (1.5, 2].
    \end{cases}
\end{align*}
Then which one of the following statements is true?
\begin{enumerate}
    \item [(A)] $X$ is a random variable with respect to $\mathcal{G}$, but $Y$ is not a random variable with respect to $\mathcal{G}$.
    \item [(B)] $Y$ is a random variable with respect to $\mathcal{G}$, but $X$ is not a random variable with respect to $\mathcal{G}$.
    \item [(C)] Neither $X$ nor $Y$ is a random variable with respect to $\mathcal{G}$.
    \item [(D)] Both $X$ and $Y$ are random variables with respect to $\mathcal{G}$.
\end{enumerate} \hfill (GATE ST 2023)\\
\solution
%\input{gate/ST/2023/14/main.tex}
	\item  A die is loaded in such a way that each odd number is twice as likely to occur as
each even number. Find $P(G)$, where $G$ is the event that a number greater than
3 occurs on a single roll of the die.
\\
\solution
		%\input{exemplar/11/16/3/5/main.tex}
	\item All the jacks, queens and kings are removed from a deck of 52 playing cards. The remaining cards are well shuffled and then one card is drawn at random. Giving ace a value 1 similar value for other cards, find the probability that the card has a value 
		\begin{enumerate}
			\item 7
			\item greater than 7
			\item less than 7
		\end{enumerate}
		%\input{exemplar/10/13/3/30/main.tex}
  \item A Lot consists of 48 mobile phones of which 42 are good, 3 have only minor defects and 3 have major defects.Varnika will buy a phone if it is good but the trader will only buy a mobile if it has no major defects. One phone is selected at random from the lot. What is the probability that it is
\begin{enumerate}
	\item acceptable to Varnika?
            \item acceptable to the trader?
\end{enumerate}
\solution
	%\input{exemplar/10/13/3/40/main.tex}
 \item A student says that if you throw a die, it will show up 1 or not 1. Therefore, the probability of getting 1 and the probability of getting 'not 1' each is equal to $\frac{1}{2}$. Is this correct? Give reasons.\\
 \solution
        %\input{exemplar/10/13/2/9/main.tex}
   \item Four candidates A, B, C, D have ap-
plied for the assignment to coach a school cricket
team. If A is twice as likely to be selected as B, and
B and C are given about the same chance of being
selected, while C is twice as likely to be selected
as D, what are the probabilities that
\begin{enumerate}
\item C will be selected?
\item A will not be selected?
\end{enumerate}
	%\input{exemplar/11/16/3/9/main.tex}
 \item A bag contain 24 balls of which $x$ balls are red, $2x$ are white and $3x$ are blue. A ball is selected at random, What is the probability that it is
\begin{enumerate}[label=\alph*)]
\item not red ?
\item white ?
\end{enumerate}
%\input{exemplar/10/13/3/41/main.tex}
If the letters of the word ASSASSINATION are arranged at random. Find the Probability that
\begin{enumerate}[label=(\alph*)]
\item Four $S's$ come consecutively in the word
\item Two  $I's$ and two $N's$ come together
\item All $A's$ are not coming together
\item No two $A's$ are coming together
\end{enumerate}
%\input{exemplar/11/16/3/14/main.tex}
	\item One urn contains two black balls (labelled B1 and B2) and one white ball. A
	second urn contains one black ball and two white balls (labelled W1 and W2).
	Suppose the following experiment is performed. One of the two urns is chosen
	at random. Next a ball is randomly chosen from the urn. Then a second ball is
	chosen at random from the same urn without replacing the first ball.
	
	\begin{enumerate}
	\item What is the probability that two black balls are chosen?
	
	\item What is the probability that two balls of opposite colour are chosen?
	\end{enumerate}
	\solution
	%\input{exemplar/11/16/3/12/main1.tex}
\end{enumerate}

	\item A bag contains $5$ red balls and some blue balls. If the probability of drawing a blue ball is double that if a red ball, determine the number of blue balls in the bag. 
		\\
\solution
		%\begin{enumerate}[label=\thesection.\arabic*,ref=\thesection.\theenumi]
	\item One card is drawn from a well-shuffled deck of 52 cards. Find the probability of getting
\begin{enumerate}
\item A king of red colour 
\item A face card 
\item A red face card
\item The jack of hearts
\item A spade
\item The queen of diamonds

\end{enumerate}
\solution
		%\input{ncert/10/15/1/14/main.tex}
	\item Five cards—the ten, jack, queen, king and ace of diamonds, are well-shuffled with their face downwards. One card is then picked up at random.
\begin{enumerate}
\item
What is the probability that the card is the queen? 
\item
If the queen is drawn and put aside, what is the probability that the second card picked up is (a) an ace? (b) a queen?\\
\end{enumerate}
\solution
		%\input{ncert/10/15/1/15/defs.tex}
	\item A bag contains $5$ red balls and some blue balls. If the probability of drawing a blue ball is double that if a red ball, determine the number of blue balls in the bag. 
		\\
\solution
		%\input{ncert/10/15/2/3/defs.tex}
	\item A card is selected from a pack of 52 cards.
 \begin{enumerate}[label=(\alph*)] 
                 \item How many points are there in the sample space?
                 \item Calculate the probability that the card is an ace of spades.
                 \item Calculate the probability that the card is (i) an ace and (ii) black card.
 \end{enumerate}
\solution
		%\input{ncert/11/16/3/4/main.tex}
\item Four cards are drawn from a well-shuffled deck of 52 cards. What is the probability of obtaining 3 diamonds and one spade.
\\
\solution
		%\input{ncert/11/16/4/2/defs.tex}
\item In a certain lottery 10,000 tickets are sold and ten equal prizes are awarded. What is the probability of not getting a prize if you buy (a) one ticket (b) two tickets (c) 10 tickets ?	
\\
\solution
		%\input{ncert/11/16/4/4/defs.tex}
		%
\item 
Out of 100 students, two sections of 40 and 60 are formed. If you and your friend are among the 100 students, what is the probability that
\begin{enumerate}
\item you both enter the same section?
\item you both enter the different sections?
\end{enumerate}
\solution
		%\input{ncert/11/16/4/5/defs.tex}
	\item 
The number lock of a suitcase has 4 wheels each labelled with ten digits i.e. from 0 to 9.The lock opens with a sequence of four digits with no repeats.What is the probability of a person getting the right sequence to open the suitcase.
\\
\solution
		%\input{ncert/11/16/4/10/defs.tex}
		%
\item 
Two cards are drawn at random and without replacement from a pack of 52 playing cards. Find the probability that both the cards are black.
\\
\solution
		%\input{ncert/12/13/2/2/defs.tex}
		\item A box of oranges is inspected by examining three randomly selected oranges drawn without replacement. If all the three oranges are good, the box is approved for sale, otherwise, it is rejected. Find the probability that a box containing 15 oranges out of which 12 are good and 3 are bad ones will be approved for sale.
		\label{ncert/12/13/2/3/defs.tex}
		\item Two balls are drawn at random with replacement from a box containing 10 black and 8 red balls. Find the probability that
		\label{ncert/12/13/2/12}
\begin{enumerate}
\item both balls are red.
\item first ball is black and second is red.
\item one of them is black and other is red.
\end{enumerate}

\item In a hostel, 60\% of the students read Hindi newspaper, 40\% read English newspaper and 20\% read both Hindi and English newspapers. A student is selected at random.
		\label{ncert/12/13/2/15}
\begin{enumerate}
\item Find the probability that she reads neither Hindi nor English newspapers.
\item If she reads Hindi newspaper, find the probability that she reads English newspaper.
\item If she reads English newspaper, find the probability that she reads Hindi newspaper.\\
\end{enumerate}
\item The probability of obtaining an even prime number on each die, when a pair of dice is rolled is 
\begin{enumerate}
    \item $0$ 
    
    \item $\frac{1}{3}$ 
    
    \item $\frac{1}{12}$ 
    
    \item $\frac{1}{36}$ 
\end{enumerate}
\solution
		%\input{ncert/12/13/2/17/defs.tex}
	\item A bag contains 4 red and 4 black balls, another bag contains 2 red and 6 black balls. One of the two bags is selected at random and a ball is drawn from the bag which is found to be red. Find the probability that the ball is drawn from the first bag.
\\
\solution
		%\input{ncert/12/13/3/2/main.tex}
  \item
  Cards with numbers 2 to 101 are placed in a box. A card is selected at random.Find the probability that the card has
\begin{enumerate}[label=(\roman*)]
	\item an even number 
	\item a square number
\end{enumerate}
\solution
%\input{exemplar/10/13/3/32/main.tex}
\item
The king, queen and jack of clubs are removed from a deck of 52 playing cards and then well shuffled. Now one card is drawn at random from the remaining cards.  Determine the probability that the card is
\begin{enumerate}[label=(\roman*)]
\item a club
\item 10 of hearts
\end{enumerate}
\solution
%\input{exemplar/10/13/3/29/main.tex}
\item A team of medical students doing their internship have to assist during surgeries
at a city hospital. The probabilities of surgeries rated as very complex, complex,
routine, simple or very simple are respectively, 0.15, 0.20, 0.31, 0.26, .08. Find
the probabilities that a particular surgery will be rated
\begin{enumerate}
	\item complex or very complex;
	\item neither very complex nor very simple;
	\item routine or complex
	\item routine or simple
\end{enumerate}
\solution
%\input{exemplar/11/16/3/8(1)/main.tex}
\item A card is selected from a pack of 52 cards.
\begin{enumerate}[label=(\alph*)]
    \item How many points are there in the sample space?
    \item Calculate the probability that the card is an ace of spades.
    \item Calculate the probability that the card is (i) an ace and (ii) black card.
\end{enumerate}
\solution
%\input{exemplar/11/16/3/4/main2.tex}
\item The probability that a non leap year selected at random will contain 53 sundays.
\\
\solution
%\input{exemplar/10/13/1/19/main.tex}
\item One of the four persons John, Rita, Aslam or Gurpreet will be promoted next
month. Consequently the sample space consists of four elementary outcomes
S = {John promoted, Rita promoted, Aslam promoted, Gurpreet promoted}
You are told that the chances of John’s promotion is same as that of Gurpreet,
Rita’s chances of promotion are twice as likely as Johns. Aslam’s chances are
four times that of John.
\begin{enumerate}
	\item Determine
	\begin{enumerate}
		\item P (John promoted)
		\item P (Rita promoted)
		\item P (Aslam promoted)
		\item P (Gurpreet promoted)
	\end{enumerate}
	\item If A = {John promoted or Gurpreet promoted}, find P (A).
\end{enumerate}
\solution
%\input{exemplar/11/16/3/10/main.tex}
\item A card is drawn from a deck of 52 cards. Find the probability of getting a king or a heart or a red card.\\
\solution
%\input{exemplar/11/16/3/15/main.tex}
\item The probability that a student will pass his examination is 0.73, the probability of
the student getting a compartment is 0.13, and the probability that the student will
either pass or get compartment is 0.96. State True or False.\\
\solution
%\input{exemplar/11/16/3/31/main.tex}
\item A card is selected from a pack of 52 cards\\
\begin{enumerate}[label=(\alph*)]
\item How many points are there in the sample space?
\item Calculate the probability that the cards is an ace of spades.
\item Calculate the probability that the card is (i) an ace (ii)black card.\\
\end{enumerate}
%\input{ncert/11/16/3/4_1/Prob_4.tex}
\item In a non-leap year, the probability of having 53 tuesdays or 53 wednesdays is\\
\solution
%\input{exemplar/11/16/3/18/main.tex}
\item There are 1000 sealed envelopes in a box, 10 of them contain a cash prize of
Rs 100 each, 100 of them contain a cash prize of Rs 50 each and 200 of them
contain a cash prize of Rs 10 each and rest do not contain any cash prize. If they
are well shuffled and an envelope is picked up out, what is the probability that it
contains no cash prize?\\
\solution
%\input{exemplar/10/13/3/34/main.tex}
\item 
A die is thrown and a card is selected at random from a deck of 52 playing cards. The probability of getting an even number on the die and a spade card.\\
\solution
%\input{exemplar/12/13/3/78/main.tex}
\item
If 4-digit numbers greater than 5,000 are randomly formed from the digits 0, 1, 3, 5, and 7, what is the probability of forming a number divisible by 5 when:
\begin{enumerate}
    \item The digits are repeated?
    \item The repetition of digits is not allowed?
\end{enumerate}
\solution
%\input{ncert/11/16/4/9/main.tex}
\item Consider the probability space $\brak{\Omega, \mathcal{G}, P}$ where $\Omega = [0,2]$ and $\mathcal{G} = \cbrak{\phi, \Omega, [0,1], (1,2]}$. Let $X$ and $Y$ be two functions on $\Omega$ defined as
\begin{align*}
    X(\omega) = 
    \begin{cases}
        1 & \text{if }\omega \in [0, 1]\\
        2 & \text{if }\omega \in (1, 2]
    \end{cases}
\end{align*}
and
\begin{align*}
    Y(\omega) = 
    \begin{cases}
        2 & \text{if }\omega \in [0, 1.5]\\
        3 & \text{if }\omega \in (1.5, 2].
    \end{cases}
\end{align*}
Then which one of the following statements is true?
\begin{enumerate}
    \item [(A)] $X$ is a random variable with respect to $\mathcal{G}$, but $Y$ is not a random variable with respect to $\mathcal{G}$.
    \item [(B)] $Y$ is a random variable with respect to $\mathcal{G}$, but $X$ is not a random variable with respect to $\mathcal{G}$.
    \item [(C)] Neither $X$ nor $Y$ is a random variable with respect to $\mathcal{G}$.
    \item [(D)] Both $X$ and $Y$ are random variables with respect to $\mathcal{G}$.
\end{enumerate} \hfill (GATE ST 2023)\\
\solution
%\input{gate/ST/2023/14/main.tex}
	\item  A die is loaded in such a way that each odd number is twice as likely to occur as
each even number. Find $P(G)$, where $G$ is the event that a number greater than
3 occurs on a single roll of the die.
\\
\solution
		%\input{exemplar/11/16/3/5/main.tex}
	\item All the jacks, queens and kings are removed from a deck of 52 playing cards. The remaining cards are well shuffled and then one card is drawn at random. Giving ace a value 1 similar value for other cards, find the probability that the card has a value 
		\begin{enumerate}
			\item 7
			\item greater than 7
			\item less than 7
		\end{enumerate}
		%\input{exemplar/10/13/3/30/main.tex}
  \item A Lot consists of 48 mobile phones of which 42 are good, 3 have only minor defects and 3 have major defects.Varnika will buy a phone if it is good but the trader will only buy a mobile if it has no major defects. One phone is selected at random from the lot. What is the probability that it is
\begin{enumerate}
	\item acceptable to Varnika?
            \item acceptable to the trader?
\end{enumerate}
\solution
	%\input{exemplar/10/13/3/40/main.tex}
 \item A student says that if you throw a die, it will show up 1 or not 1. Therefore, the probability of getting 1 and the probability of getting 'not 1' each is equal to $\frac{1}{2}$. Is this correct? Give reasons.\\
 \solution
        %\input{exemplar/10/13/2/9/main.tex}
   \item Four candidates A, B, C, D have ap-
plied for the assignment to coach a school cricket
team. If A is twice as likely to be selected as B, and
B and C are given about the same chance of being
selected, while C is twice as likely to be selected
as D, what are the probabilities that
\begin{enumerate}
\item C will be selected?
\item A will not be selected?
\end{enumerate}
	%\input{exemplar/11/16/3/9/main.tex}
 \item A bag contain 24 balls of which $x$ balls are red, $2x$ are white and $3x$ are blue. A ball is selected at random, What is the probability that it is
\begin{enumerate}[label=\alph*)]
\item not red ?
\item white ?
\end{enumerate}
%\input{exemplar/10/13/3/41/main.tex}
If the letters of the word ASSASSINATION are arranged at random. Find the Probability that
\begin{enumerate}[label=(\alph*)]
\item Four $S's$ come consecutively in the word
\item Two  $I's$ and two $N's$ come together
\item All $A's$ are not coming together
\item No two $A's$ are coming together
\end{enumerate}
%\input{exemplar/11/16/3/14/main.tex}
	\item One urn contains two black balls (labelled B1 and B2) and one white ball. A
	second urn contains one black ball and two white balls (labelled W1 and W2).
	Suppose the following experiment is performed. One of the two urns is chosen
	at random. Next a ball is randomly chosen from the urn. Then a second ball is
	chosen at random from the same urn without replacing the first ball.
	
	\begin{enumerate}
	\item What is the probability that two black balls are chosen?
	
	\item What is the probability that two balls of opposite colour are chosen?
	\end{enumerate}
	\solution
	%\input{exemplar/11/16/3/12/main1.tex}
\end{enumerate}

	\item A card is selected from a pack of 52 cards.
 \begin{enumerate}[label=(\alph*)] 
                 \item How many points are there in the sample space?
                 \item Calculate the probability that the card is an ace of spades.
                 \item Calculate the probability that the card is (i) an ace and (ii) black card.
 \end{enumerate}
\solution
		%\begin{table}[H]
	\centering
\begin{tabular}{|c|c|c|}
\hline
Random variable &Value &Definition\\ \hline
\multirow{3}{*}{X} &0 &Slips of Rs 1\\
&1 &Slips of Rs 5\\
&2 &Slips of Rs 13\\ \hline
\multirow{2}{*}{Y} &0 &Box A\\
&1 &Box B\\\hline
\end{tabular}
\caption{}
\label{tab:Distribution}
\end{table}
See \tabref{tab:Distribution}.
\begin{align}
p_{Y}\brak{k}= \begin{cases} 
      \frac{1}{3} & {k=0} \\
      \frac{2}{3 }& {k=1} 
   \end{cases}
   \\
p_{Y|X}\brak{0|0} = \frac{19}{25}\, 
p_{Y|X}\brak{0|1} = \frac{6}{25}\,
p_{Y|X}\brak{1|0} = \frac{45}{50}\,
p_{Y|X}\brak{1|2} = \frac{5}{50}
\end{align}
The desired probability is the probability that a slip drawn at random is marked other than Rs 1,
\begin{align}
&=1-p_X\brak{0}\\
&= p_X(1) + p_X(2)
\end{align}
Using Bayes theorem,
\begin{align}
&= p_Y\brak{0} \times \pr{Y=0 | X=1} + p_Y\brak{1} \times \pr{Y=1|X=2}\\
&=\frac{1}{3} \times \frac{6}{25} + \frac{2}{3} \times \frac{5}{50}\\
&=\frac{11}{75}
\end{align}

\newpage

%\tableofcontents

\bigskip

\renewcommand{\thefigure}{\theenumi}
\renewcommand{\thetable}{\theenumi}
%\renewcommand{\theequation}{\theenumi}

%\begin{abstract}
%%\boldmath
%In this letter, an algorithm for evaluating the exact analytical bit error rate  (BER)  for the piecewise linear (PL) combiner for  multiple relays is presented. Previous results were available only for upto three relays. The algorithm is unique in the sense that  the actual mathematical expressions, that are prohibitively large, need not be explicitly obtained. The diversity gain due to multiple relays is shown through plots of the analytical BER, well supported by simulations. 
%
%\end{abstract}
% IEEEtran.cls defaults to using nonbold math in the Abstract.
% This preserves the distinction between vectors and scalars. However,
% if the journal you are submitting to favors bold math in the abstract,
% then you can use LaTeX's standard command \boldmath at the very start
% of the abstract to achieve this. Many IEEE journals frown on math
% in the abstract anyway.

% Note that keywords are not normally used for peerreview papers.
%\begin{IEEEkeywords}
%Cooperative diversity, decode and forward, piecewise linear
%\end{IEEEkeywords}



% For peer review papers, you can put extra information on the cover
% page as needed:
% \ifCLASSOPTIONpeerreview
% \begin{center} \bfseries EDICS Category: 3-BBND \end{center}
% \fi
%
% For peerreview papers, this IEEEtran command inserts a page break and
% creates the second title. It will be ignored for other modes.
%\IEEEpeerreviewmaketitle




\item Four cards are drawn from a well-shuffled deck of 52 cards. What is the probability of obtaining 3 diamonds and one spade.
\\
\solution
		%\begin{enumerate}[label=\thesection.\arabic*,ref=\thesection.\theenumi]
	\item One card is drawn from a well-shuffled deck of 52 cards. Find the probability of getting
\begin{enumerate}
\item A king of red colour 
\item A face card 
\item A red face card
\item The jack of hearts
\item A spade
\item The queen of diamonds

\end{enumerate}
\solution
		%\input{ncert/10/15/1/14/main.tex}
	\item Five cards—the ten, jack, queen, king and ace of diamonds, are well-shuffled with their face downwards. One card is then picked up at random.
\begin{enumerate}
\item
What is the probability that the card is the queen? 
\item
If the queen is drawn and put aside, what is the probability that the second card picked up is (a) an ace? (b) a queen?\\
\end{enumerate}
\solution
		%\input{ncert/10/15/1/15/defs.tex}
	\item A bag contains $5$ red balls and some blue balls. If the probability of drawing a blue ball is double that if a red ball, determine the number of blue balls in the bag. 
		\\
\solution
		%\input{ncert/10/15/2/3/defs.tex}
	\item A card is selected from a pack of 52 cards.
 \begin{enumerate}[label=(\alph*)] 
                 \item How many points are there in the sample space?
                 \item Calculate the probability that the card is an ace of spades.
                 \item Calculate the probability that the card is (i) an ace and (ii) black card.
 \end{enumerate}
\solution
		%\input{ncert/11/16/3/4/main.tex}
\item Four cards are drawn from a well-shuffled deck of 52 cards. What is the probability of obtaining 3 diamonds and one spade.
\\
\solution
		%\input{ncert/11/16/4/2/defs.tex}
\item In a certain lottery 10,000 tickets are sold and ten equal prizes are awarded. What is the probability of not getting a prize if you buy (a) one ticket (b) two tickets (c) 10 tickets ?	
\\
\solution
		%\input{ncert/11/16/4/4/defs.tex}
		%
\item 
Out of 100 students, two sections of 40 and 60 are formed. If you and your friend are among the 100 students, what is the probability that
\begin{enumerate}
\item you both enter the same section?
\item you both enter the different sections?
\end{enumerate}
\solution
		%\input{ncert/11/16/4/5/defs.tex}
	\item 
The number lock of a suitcase has 4 wheels each labelled with ten digits i.e. from 0 to 9.The lock opens with a sequence of four digits with no repeats.What is the probability of a person getting the right sequence to open the suitcase.
\\
\solution
		%\input{ncert/11/16/4/10/defs.tex}
		%
\item 
Two cards are drawn at random and without replacement from a pack of 52 playing cards. Find the probability that both the cards are black.
\\
\solution
		%\input{ncert/12/13/2/2/defs.tex}
		\item A box of oranges is inspected by examining three randomly selected oranges drawn without replacement. If all the three oranges are good, the box is approved for sale, otherwise, it is rejected. Find the probability that a box containing 15 oranges out of which 12 are good and 3 are bad ones will be approved for sale.
		\label{ncert/12/13/2/3/defs.tex}
		\item Two balls are drawn at random with replacement from a box containing 10 black and 8 red balls. Find the probability that
		\label{ncert/12/13/2/12}
\begin{enumerate}
\item both balls are red.
\item first ball is black and second is red.
\item one of them is black and other is red.
\end{enumerate}

\item In a hostel, 60\% of the students read Hindi newspaper, 40\% read English newspaper and 20\% read both Hindi and English newspapers. A student is selected at random.
		\label{ncert/12/13/2/15}
\begin{enumerate}
\item Find the probability that she reads neither Hindi nor English newspapers.
\item If she reads Hindi newspaper, find the probability that she reads English newspaper.
\item If she reads English newspaper, find the probability that she reads Hindi newspaper.\\
\end{enumerate}
\item The probability of obtaining an even prime number on each die, when a pair of dice is rolled is 
\begin{enumerate}
    \item $0$ 
    
    \item $\frac{1}{3}$ 
    
    \item $\frac{1}{12}$ 
    
    \item $\frac{1}{36}$ 
\end{enumerate}
\solution
		%\input{ncert/12/13/2/17/defs.tex}
	\item A bag contains 4 red and 4 black balls, another bag contains 2 red and 6 black balls. One of the two bags is selected at random and a ball is drawn from the bag which is found to be red. Find the probability that the ball is drawn from the first bag.
\\
\solution
		%\input{ncert/12/13/3/2/main.tex}
  \item
  Cards with numbers 2 to 101 are placed in a box. A card is selected at random.Find the probability that the card has
\begin{enumerate}[label=(\roman*)]
	\item an even number 
	\item a square number
\end{enumerate}
\solution
%\input{exemplar/10/13/3/32/main.tex}
\item
The king, queen and jack of clubs are removed from a deck of 52 playing cards and then well shuffled. Now one card is drawn at random from the remaining cards.  Determine the probability that the card is
\begin{enumerate}[label=(\roman*)]
\item a club
\item 10 of hearts
\end{enumerate}
\solution
%\input{exemplar/10/13/3/29/main.tex}
\item A team of medical students doing their internship have to assist during surgeries
at a city hospital. The probabilities of surgeries rated as very complex, complex,
routine, simple or very simple are respectively, 0.15, 0.20, 0.31, 0.26, .08. Find
the probabilities that a particular surgery will be rated
\begin{enumerate}
	\item complex or very complex;
	\item neither very complex nor very simple;
	\item routine or complex
	\item routine or simple
\end{enumerate}
\solution
%\input{exemplar/11/16/3/8(1)/main.tex}
\item A card is selected from a pack of 52 cards.
\begin{enumerate}[label=(\alph*)]
    \item How many points are there in the sample space?
    \item Calculate the probability that the card is an ace of spades.
    \item Calculate the probability that the card is (i) an ace and (ii) black card.
\end{enumerate}
\solution
%\input{exemplar/11/16/3/4/main2.tex}
\item The probability that a non leap year selected at random will contain 53 sundays.
\\
\solution
%\input{exemplar/10/13/1/19/main.tex}
\item One of the four persons John, Rita, Aslam or Gurpreet will be promoted next
month. Consequently the sample space consists of four elementary outcomes
S = {John promoted, Rita promoted, Aslam promoted, Gurpreet promoted}
You are told that the chances of John’s promotion is same as that of Gurpreet,
Rita’s chances of promotion are twice as likely as Johns. Aslam’s chances are
four times that of John.
\begin{enumerate}
	\item Determine
	\begin{enumerate}
		\item P (John promoted)
		\item P (Rita promoted)
		\item P (Aslam promoted)
		\item P (Gurpreet promoted)
	\end{enumerate}
	\item If A = {John promoted or Gurpreet promoted}, find P (A).
\end{enumerate}
\solution
%\input{exemplar/11/16/3/10/main.tex}
\item A card is drawn from a deck of 52 cards. Find the probability of getting a king or a heart or a red card.\\
\solution
%\input{exemplar/11/16/3/15/main.tex}
\item The probability that a student will pass his examination is 0.73, the probability of
the student getting a compartment is 0.13, and the probability that the student will
either pass or get compartment is 0.96. State True or False.\\
\solution
%\input{exemplar/11/16/3/31/main.tex}
\item A card is selected from a pack of 52 cards\\
\begin{enumerate}[label=(\alph*)]
\item How many points are there in the sample space?
\item Calculate the probability that the cards is an ace of spades.
\item Calculate the probability that the card is (i) an ace (ii)black card.\\
\end{enumerate}
%\input{ncert/11/16/3/4_1/Prob_4.tex}
\item In a non-leap year, the probability of having 53 tuesdays or 53 wednesdays is\\
\solution
%\input{exemplar/11/16/3/18/main.tex}
\item There are 1000 sealed envelopes in a box, 10 of them contain a cash prize of
Rs 100 each, 100 of them contain a cash prize of Rs 50 each and 200 of them
contain a cash prize of Rs 10 each and rest do not contain any cash prize. If they
are well shuffled and an envelope is picked up out, what is the probability that it
contains no cash prize?\\
\solution
%\input{exemplar/10/13/3/34/main.tex}
\item 
A die is thrown and a card is selected at random from a deck of 52 playing cards. The probability of getting an even number on the die and a spade card.\\
\solution
%\input{exemplar/12/13/3/78/main.tex}
\item
If 4-digit numbers greater than 5,000 are randomly formed from the digits 0, 1, 3, 5, and 7, what is the probability of forming a number divisible by 5 when:
\begin{enumerate}
    \item The digits are repeated?
    \item The repetition of digits is not allowed?
\end{enumerate}
\solution
%\input{ncert/11/16/4/9/main.tex}
\item Consider the probability space $\brak{\Omega, \mathcal{G}, P}$ where $\Omega = [0,2]$ and $\mathcal{G} = \cbrak{\phi, \Omega, [0,1], (1,2]}$. Let $X$ and $Y$ be two functions on $\Omega$ defined as
\begin{align*}
    X(\omega) = 
    \begin{cases}
        1 & \text{if }\omega \in [0, 1]\\
        2 & \text{if }\omega \in (1, 2]
    \end{cases}
\end{align*}
and
\begin{align*}
    Y(\omega) = 
    \begin{cases}
        2 & \text{if }\omega \in [0, 1.5]\\
        3 & \text{if }\omega \in (1.5, 2].
    \end{cases}
\end{align*}
Then which one of the following statements is true?
\begin{enumerate}
    \item [(A)] $X$ is a random variable with respect to $\mathcal{G}$, but $Y$ is not a random variable with respect to $\mathcal{G}$.
    \item [(B)] $Y$ is a random variable with respect to $\mathcal{G}$, but $X$ is not a random variable with respect to $\mathcal{G}$.
    \item [(C)] Neither $X$ nor $Y$ is a random variable with respect to $\mathcal{G}$.
    \item [(D)] Both $X$ and $Y$ are random variables with respect to $\mathcal{G}$.
\end{enumerate} \hfill (GATE ST 2023)\\
\solution
%\input{gate/ST/2023/14/main.tex}
	\item  A die is loaded in such a way that each odd number is twice as likely to occur as
each even number. Find $P(G)$, where $G$ is the event that a number greater than
3 occurs on a single roll of the die.
\\
\solution
		%\input{exemplar/11/16/3/5/main.tex}
	\item All the jacks, queens and kings are removed from a deck of 52 playing cards. The remaining cards are well shuffled and then one card is drawn at random. Giving ace a value 1 similar value for other cards, find the probability that the card has a value 
		\begin{enumerate}
			\item 7
			\item greater than 7
			\item less than 7
		\end{enumerate}
		%\input{exemplar/10/13/3/30/main.tex}
  \item A Lot consists of 48 mobile phones of which 42 are good, 3 have only minor defects and 3 have major defects.Varnika will buy a phone if it is good but the trader will only buy a mobile if it has no major defects. One phone is selected at random from the lot. What is the probability that it is
\begin{enumerate}
	\item acceptable to Varnika?
            \item acceptable to the trader?
\end{enumerate}
\solution
	%\input{exemplar/10/13/3/40/main.tex}
 \item A student says that if you throw a die, it will show up 1 or not 1. Therefore, the probability of getting 1 and the probability of getting 'not 1' each is equal to $\frac{1}{2}$. Is this correct? Give reasons.\\
 \solution
        %\input{exemplar/10/13/2/9/main.tex}
   \item Four candidates A, B, C, D have ap-
plied for the assignment to coach a school cricket
team. If A is twice as likely to be selected as B, and
B and C are given about the same chance of being
selected, while C is twice as likely to be selected
as D, what are the probabilities that
\begin{enumerate}
\item C will be selected?
\item A will not be selected?
\end{enumerate}
	%\input{exemplar/11/16/3/9/main.tex}
 \item A bag contain 24 balls of which $x$ balls are red, $2x$ are white and $3x$ are blue. A ball is selected at random, What is the probability that it is
\begin{enumerate}[label=\alph*)]
\item not red ?
\item white ?
\end{enumerate}
%\input{exemplar/10/13/3/41/main.tex}
If the letters of the word ASSASSINATION are arranged at random. Find the Probability that
\begin{enumerate}[label=(\alph*)]
\item Four $S's$ come consecutively in the word
\item Two  $I's$ and two $N's$ come together
\item All $A's$ are not coming together
\item No two $A's$ are coming together
\end{enumerate}
%\input{exemplar/11/16/3/14/main.tex}
	\item One urn contains two black balls (labelled B1 and B2) and one white ball. A
	second urn contains one black ball and two white balls (labelled W1 and W2).
	Suppose the following experiment is performed. One of the two urns is chosen
	at random. Next a ball is randomly chosen from the urn. Then a second ball is
	chosen at random from the same urn without replacing the first ball.
	
	\begin{enumerate}
	\item What is the probability that two black balls are chosen?
	
	\item What is the probability that two balls of opposite colour are chosen?
	\end{enumerate}
	\solution
	%\input{exemplar/11/16/3/12/main1.tex}
\end{enumerate}

\item In a certain lottery 10,000 tickets are sold and ten equal prizes are awarded. What is the probability of not getting a prize if you buy (a) one ticket (b) two tickets (c) 10 tickets ?	
\\
\solution
		%\begin{enumerate}[label=\thesection.\arabic*,ref=\thesection.\theenumi]
	\item One card is drawn from a well-shuffled deck of 52 cards. Find the probability of getting
\begin{enumerate}
\item A king of red colour 
\item A face card 
\item A red face card
\item The jack of hearts
\item A spade
\item The queen of diamonds

\end{enumerate}
\solution
		%\input{ncert/10/15/1/14/main.tex}
	\item Five cards—the ten, jack, queen, king and ace of diamonds, are well-shuffled with their face downwards. One card is then picked up at random.
\begin{enumerate}
\item
What is the probability that the card is the queen? 
\item
If the queen is drawn and put aside, what is the probability that the second card picked up is (a) an ace? (b) a queen?\\
\end{enumerate}
\solution
		%\input{ncert/10/15/1/15/defs.tex}
	\item A bag contains $5$ red balls and some blue balls. If the probability of drawing a blue ball is double that if a red ball, determine the number of blue balls in the bag. 
		\\
\solution
		%\input{ncert/10/15/2/3/defs.tex}
	\item A card is selected from a pack of 52 cards.
 \begin{enumerate}[label=(\alph*)] 
                 \item How many points are there in the sample space?
                 \item Calculate the probability that the card is an ace of spades.
                 \item Calculate the probability that the card is (i) an ace and (ii) black card.
 \end{enumerate}
\solution
		%\input{ncert/11/16/3/4/main.tex}
\item Four cards are drawn from a well-shuffled deck of 52 cards. What is the probability of obtaining 3 diamonds and one spade.
\\
\solution
		%\input{ncert/11/16/4/2/defs.tex}
\item In a certain lottery 10,000 tickets are sold and ten equal prizes are awarded. What is the probability of not getting a prize if you buy (a) one ticket (b) two tickets (c) 10 tickets ?	
\\
\solution
		%\input{ncert/11/16/4/4/defs.tex}
		%
\item 
Out of 100 students, two sections of 40 and 60 are formed. If you and your friend are among the 100 students, what is the probability that
\begin{enumerate}
\item you both enter the same section?
\item you both enter the different sections?
\end{enumerate}
\solution
		%\input{ncert/11/16/4/5/defs.tex}
	\item 
The number lock of a suitcase has 4 wheels each labelled with ten digits i.e. from 0 to 9.The lock opens with a sequence of four digits with no repeats.What is the probability of a person getting the right sequence to open the suitcase.
\\
\solution
		%\input{ncert/11/16/4/10/defs.tex}
		%
\item 
Two cards are drawn at random and without replacement from a pack of 52 playing cards. Find the probability that both the cards are black.
\\
\solution
		%\input{ncert/12/13/2/2/defs.tex}
		\item A box of oranges is inspected by examining three randomly selected oranges drawn without replacement. If all the three oranges are good, the box is approved for sale, otherwise, it is rejected. Find the probability that a box containing 15 oranges out of which 12 are good and 3 are bad ones will be approved for sale.
		\label{ncert/12/13/2/3/defs.tex}
		\item Two balls are drawn at random with replacement from a box containing 10 black and 8 red balls. Find the probability that
		\label{ncert/12/13/2/12}
\begin{enumerate}
\item both balls are red.
\item first ball is black and second is red.
\item one of them is black and other is red.
\end{enumerate}

\item In a hostel, 60\% of the students read Hindi newspaper, 40\% read English newspaper and 20\% read both Hindi and English newspapers. A student is selected at random.
		\label{ncert/12/13/2/15}
\begin{enumerate}
\item Find the probability that she reads neither Hindi nor English newspapers.
\item If she reads Hindi newspaper, find the probability that she reads English newspaper.
\item If she reads English newspaper, find the probability that she reads Hindi newspaper.\\
\end{enumerate}
\item The probability of obtaining an even prime number on each die, when a pair of dice is rolled is 
\begin{enumerate}
    \item $0$ 
    
    \item $\frac{1}{3}$ 
    
    \item $\frac{1}{12}$ 
    
    \item $\frac{1}{36}$ 
\end{enumerate}
\solution
		%\input{ncert/12/13/2/17/defs.tex}
	\item A bag contains 4 red and 4 black balls, another bag contains 2 red and 6 black balls. One of the two bags is selected at random and a ball is drawn from the bag which is found to be red. Find the probability that the ball is drawn from the first bag.
\\
\solution
		%\input{ncert/12/13/3/2/main.tex}
  \item
  Cards with numbers 2 to 101 are placed in a box. A card is selected at random.Find the probability that the card has
\begin{enumerate}[label=(\roman*)]
	\item an even number 
	\item a square number
\end{enumerate}
\solution
%\input{exemplar/10/13/3/32/main.tex}
\item
The king, queen and jack of clubs are removed from a deck of 52 playing cards and then well shuffled. Now one card is drawn at random from the remaining cards.  Determine the probability that the card is
\begin{enumerate}[label=(\roman*)]
\item a club
\item 10 of hearts
\end{enumerate}
\solution
%\input{exemplar/10/13/3/29/main.tex}
\item A team of medical students doing their internship have to assist during surgeries
at a city hospital. The probabilities of surgeries rated as very complex, complex,
routine, simple or very simple are respectively, 0.15, 0.20, 0.31, 0.26, .08. Find
the probabilities that a particular surgery will be rated
\begin{enumerate}
	\item complex or very complex;
	\item neither very complex nor very simple;
	\item routine or complex
	\item routine or simple
\end{enumerate}
\solution
%\input{exemplar/11/16/3/8(1)/main.tex}
\item A card is selected from a pack of 52 cards.
\begin{enumerate}[label=(\alph*)]
    \item How many points are there in the sample space?
    \item Calculate the probability that the card is an ace of spades.
    \item Calculate the probability that the card is (i) an ace and (ii) black card.
\end{enumerate}
\solution
%\input{exemplar/11/16/3/4/main2.tex}
\item The probability that a non leap year selected at random will contain 53 sundays.
\\
\solution
%\input{exemplar/10/13/1/19/main.tex}
\item One of the four persons John, Rita, Aslam or Gurpreet will be promoted next
month. Consequently the sample space consists of four elementary outcomes
S = {John promoted, Rita promoted, Aslam promoted, Gurpreet promoted}
You are told that the chances of John’s promotion is same as that of Gurpreet,
Rita’s chances of promotion are twice as likely as Johns. Aslam’s chances are
four times that of John.
\begin{enumerate}
	\item Determine
	\begin{enumerate}
		\item P (John promoted)
		\item P (Rita promoted)
		\item P (Aslam promoted)
		\item P (Gurpreet promoted)
	\end{enumerate}
	\item If A = {John promoted or Gurpreet promoted}, find P (A).
\end{enumerate}
\solution
%\input{exemplar/11/16/3/10/main.tex}
\item A card is drawn from a deck of 52 cards. Find the probability of getting a king or a heart or a red card.\\
\solution
%\input{exemplar/11/16/3/15/main.tex}
\item The probability that a student will pass his examination is 0.73, the probability of
the student getting a compartment is 0.13, and the probability that the student will
either pass or get compartment is 0.96. State True or False.\\
\solution
%\input{exemplar/11/16/3/31/main.tex}
\item A card is selected from a pack of 52 cards\\
\begin{enumerate}[label=(\alph*)]
\item How many points are there in the sample space?
\item Calculate the probability that the cards is an ace of spades.
\item Calculate the probability that the card is (i) an ace (ii)black card.\\
\end{enumerate}
%\input{ncert/11/16/3/4_1/Prob_4.tex}
\item In a non-leap year, the probability of having 53 tuesdays or 53 wednesdays is\\
\solution
%\input{exemplar/11/16/3/18/main.tex}
\item There are 1000 sealed envelopes in a box, 10 of them contain a cash prize of
Rs 100 each, 100 of them contain a cash prize of Rs 50 each and 200 of them
contain a cash prize of Rs 10 each and rest do not contain any cash prize. If they
are well shuffled and an envelope is picked up out, what is the probability that it
contains no cash prize?\\
\solution
%\input{exemplar/10/13/3/34/main.tex}
\item 
A die is thrown and a card is selected at random from a deck of 52 playing cards. The probability of getting an even number on the die and a spade card.\\
\solution
%\input{exemplar/12/13/3/78/main.tex}
\item
If 4-digit numbers greater than 5,000 are randomly formed from the digits 0, 1, 3, 5, and 7, what is the probability of forming a number divisible by 5 when:
\begin{enumerate}
    \item The digits are repeated?
    \item The repetition of digits is not allowed?
\end{enumerate}
\solution
%\input{ncert/11/16/4/9/main.tex}
\item Consider the probability space $\brak{\Omega, \mathcal{G}, P}$ where $\Omega = [0,2]$ and $\mathcal{G} = \cbrak{\phi, \Omega, [0,1], (1,2]}$. Let $X$ and $Y$ be two functions on $\Omega$ defined as
\begin{align*}
    X(\omega) = 
    \begin{cases}
        1 & \text{if }\omega \in [0, 1]\\
        2 & \text{if }\omega \in (1, 2]
    \end{cases}
\end{align*}
and
\begin{align*}
    Y(\omega) = 
    \begin{cases}
        2 & \text{if }\omega \in [0, 1.5]\\
        3 & \text{if }\omega \in (1.5, 2].
    \end{cases}
\end{align*}
Then which one of the following statements is true?
\begin{enumerate}
    \item [(A)] $X$ is a random variable with respect to $\mathcal{G}$, but $Y$ is not a random variable with respect to $\mathcal{G}$.
    \item [(B)] $Y$ is a random variable with respect to $\mathcal{G}$, but $X$ is not a random variable with respect to $\mathcal{G}$.
    \item [(C)] Neither $X$ nor $Y$ is a random variable with respect to $\mathcal{G}$.
    \item [(D)] Both $X$ and $Y$ are random variables with respect to $\mathcal{G}$.
\end{enumerate} \hfill (GATE ST 2023)\\
\solution
%\input{gate/ST/2023/14/main.tex}
	\item  A die is loaded in such a way that each odd number is twice as likely to occur as
each even number. Find $P(G)$, where $G$ is the event that a number greater than
3 occurs on a single roll of the die.
\\
\solution
		%\input{exemplar/11/16/3/5/main.tex}
	\item All the jacks, queens and kings are removed from a deck of 52 playing cards. The remaining cards are well shuffled and then one card is drawn at random. Giving ace a value 1 similar value for other cards, find the probability that the card has a value 
		\begin{enumerate}
			\item 7
			\item greater than 7
			\item less than 7
		\end{enumerate}
		%\input{exemplar/10/13/3/30/main.tex}
  \item A Lot consists of 48 mobile phones of which 42 are good, 3 have only minor defects and 3 have major defects.Varnika will buy a phone if it is good but the trader will only buy a mobile if it has no major defects. One phone is selected at random from the lot. What is the probability that it is
\begin{enumerate}
	\item acceptable to Varnika?
            \item acceptable to the trader?
\end{enumerate}
\solution
	%\input{exemplar/10/13/3/40/main.tex}
 \item A student says that if you throw a die, it will show up 1 or not 1. Therefore, the probability of getting 1 and the probability of getting 'not 1' each is equal to $\frac{1}{2}$. Is this correct? Give reasons.\\
 \solution
        %\input{exemplar/10/13/2/9/main.tex}
   \item Four candidates A, B, C, D have ap-
plied for the assignment to coach a school cricket
team. If A is twice as likely to be selected as B, and
B and C are given about the same chance of being
selected, while C is twice as likely to be selected
as D, what are the probabilities that
\begin{enumerate}
\item C will be selected?
\item A will not be selected?
\end{enumerate}
	%\input{exemplar/11/16/3/9/main.tex}
 \item A bag contain 24 balls of which $x$ balls are red, $2x$ are white and $3x$ are blue. A ball is selected at random, What is the probability that it is
\begin{enumerate}[label=\alph*)]
\item not red ?
\item white ?
\end{enumerate}
%\input{exemplar/10/13/3/41/main.tex}
If the letters of the word ASSASSINATION are arranged at random. Find the Probability that
\begin{enumerate}[label=(\alph*)]
\item Four $S's$ come consecutively in the word
\item Two  $I's$ and two $N's$ come together
\item All $A's$ are not coming together
\item No two $A's$ are coming together
\end{enumerate}
%\input{exemplar/11/16/3/14/main.tex}
	\item One urn contains two black balls (labelled B1 and B2) and one white ball. A
	second urn contains one black ball and two white balls (labelled W1 and W2).
	Suppose the following experiment is performed. One of the two urns is chosen
	at random. Next a ball is randomly chosen from the urn. Then a second ball is
	chosen at random from the same urn without replacing the first ball.
	
	\begin{enumerate}
	\item What is the probability that two black balls are chosen?
	
	\item What is the probability that two balls of opposite colour are chosen?
	\end{enumerate}
	\solution
	%\input{exemplar/11/16/3/12/main1.tex}
\end{enumerate}

		%
\item 
Out of 100 students, two sections of 40 and 60 are formed. If you and your friend are among the 100 students, what is the probability that
\begin{enumerate}
\item you both enter the same section?
\item you both enter the different sections?
\end{enumerate}
\solution
		%\begin{enumerate}[label=\thesection.\arabic*,ref=\thesection.\theenumi]
	\item One card is drawn from a well-shuffled deck of 52 cards. Find the probability of getting
\begin{enumerate}
\item A king of red colour 
\item A face card 
\item A red face card
\item The jack of hearts
\item A spade
\item The queen of diamonds

\end{enumerate}
\solution
		%\input{ncert/10/15/1/14/main.tex}
	\item Five cards—the ten, jack, queen, king and ace of diamonds, are well-shuffled with their face downwards. One card is then picked up at random.
\begin{enumerate}
\item
What is the probability that the card is the queen? 
\item
If the queen is drawn and put aside, what is the probability that the second card picked up is (a) an ace? (b) a queen?\\
\end{enumerate}
\solution
		%\input{ncert/10/15/1/15/defs.tex}
	\item A bag contains $5$ red balls and some blue balls. If the probability of drawing a blue ball is double that if a red ball, determine the number of blue balls in the bag. 
		\\
\solution
		%\input{ncert/10/15/2/3/defs.tex}
	\item A card is selected from a pack of 52 cards.
 \begin{enumerate}[label=(\alph*)] 
                 \item How many points are there in the sample space?
                 \item Calculate the probability that the card is an ace of spades.
                 \item Calculate the probability that the card is (i) an ace and (ii) black card.
 \end{enumerate}
\solution
		%\input{ncert/11/16/3/4/main.tex}
\item Four cards are drawn from a well-shuffled deck of 52 cards. What is the probability of obtaining 3 diamonds and one spade.
\\
\solution
		%\input{ncert/11/16/4/2/defs.tex}
\item In a certain lottery 10,000 tickets are sold and ten equal prizes are awarded. What is the probability of not getting a prize if you buy (a) one ticket (b) two tickets (c) 10 tickets ?	
\\
\solution
		%\input{ncert/11/16/4/4/defs.tex}
		%
\item 
Out of 100 students, two sections of 40 and 60 are formed. If you and your friend are among the 100 students, what is the probability that
\begin{enumerate}
\item you both enter the same section?
\item you both enter the different sections?
\end{enumerate}
\solution
		%\input{ncert/11/16/4/5/defs.tex}
	\item 
The number lock of a suitcase has 4 wheels each labelled with ten digits i.e. from 0 to 9.The lock opens with a sequence of four digits with no repeats.What is the probability of a person getting the right sequence to open the suitcase.
\\
\solution
		%\input{ncert/11/16/4/10/defs.tex}
		%
\item 
Two cards are drawn at random and without replacement from a pack of 52 playing cards. Find the probability that both the cards are black.
\\
\solution
		%\input{ncert/12/13/2/2/defs.tex}
		\item A box of oranges is inspected by examining three randomly selected oranges drawn without replacement. If all the three oranges are good, the box is approved for sale, otherwise, it is rejected. Find the probability that a box containing 15 oranges out of which 12 are good and 3 are bad ones will be approved for sale.
		\label{ncert/12/13/2/3/defs.tex}
		\item Two balls are drawn at random with replacement from a box containing 10 black and 8 red balls. Find the probability that
		\label{ncert/12/13/2/12}
\begin{enumerate}
\item both balls are red.
\item first ball is black and second is red.
\item one of them is black and other is red.
\end{enumerate}

\item In a hostel, 60\% of the students read Hindi newspaper, 40\% read English newspaper and 20\% read both Hindi and English newspapers. A student is selected at random.
		\label{ncert/12/13/2/15}
\begin{enumerate}
\item Find the probability that she reads neither Hindi nor English newspapers.
\item If she reads Hindi newspaper, find the probability that she reads English newspaper.
\item If she reads English newspaper, find the probability that she reads Hindi newspaper.\\
\end{enumerate}
\item The probability of obtaining an even prime number on each die, when a pair of dice is rolled is 
\begin{enumerate}
    \item $0$ 
    
    \item $\frac{1}{3}$ 
    
    \item $\frac{1}{12}$ 
    
    \item $\frac{1}{36}$ 
\end{enumerate}
\solution
		%\input{ncert/12/13/2/17/defs.tex}
	\item A bag contains 4 red and 4 black balls, another bag contains 2 red and 6 black balls. One of the two bags is selected at random and a ball is drawn from the bag which is found to be red. Find the probability that the ball is drawn from the first bag.
\\
\solution
		%\input{ncert/12/13/3/2/main.tex}
  \item
  Cards with numbers 2 to 101 are placed in a box. A card is selected at random.Find the probability that the card has
\begin{enumerate}[label=(\roman*)]
	\item an even number 
	\item a square number
\end{enumerate}
\solution
%\input{exemplar/10/13/3/32/main.tex}
\item
The king, queen and jack of clubs are removed from a deck of 52 playing cards and then well shuffled. Now one card is drawn at random from the remaining cards.  Determine the probability that the card is
\begin{enumerate}[label=(\roman*)]
\item a club
\item 10 of hearts
\end{enumerate}
\solution
%\input{exemplar/10/13/3/29/main.tex}
\item A team of medical students doing their internship have to assist during surgeries
at a city hospital. The probabilities of surgeries rated as very complex, complex,
routine, simple or very simple are respectively, 0.15, 0.20, 0.31, 0.26, .08. Find
the probabilities that a particular surgery will be rated
\begin{enumerate}
	\item complex or very complex;
	\item neither very complex nor very simple;
	\item routine or complex
	\item routine or simple
\end{enumerate}
\solution
%\input{exemplar/11/16/3/8(1)/main.tex}
\item A card is selected from a pack of 52 cards.
\begin{enumerate}[label=(\alph*)]
    \item How many points are there in the sample space?
    \item Calculate the probability that the card is an ace of spades.
    \item Calculate the probability that the card is (i) an ace and (ii) black card.
\end{enumerate}
\solution
%\input{exemplar/11/16/3/4/main2.tex}
\item The probability that a non leap year selected at random will contain 53 sundays.
\\
\solution
%\input{exemplar/10/13/1/19/main.tex}
\item One of the four persons John, Rita, Aslam or Gurpreet will be promoted next
month. Consequently the sample space consists of four elementary outcomes
S = {John promoted, Rita promoted, Aslam promoted, Gurpreet promoted}
You are told that the chances of John’s promotion is same as that of Gurpreet,
Rita’s chances of promotion are twice as likely as Johns. Aslam’s chances are
four times that of John.
\begin{enumerate}
	\item Determine
	\begin{enumerate}
		\item P (John promoted)
		\item P (Rita promoted)
		\item P (Aslam promoted)
		\item P (Gurpreet promoted)
	\end{enumerate}
	\item If A = {John promoted or Gurpreet promoted}, find P (A).
\end{enumerate}
\solution
%\input{exemplar/11/16/3/10/main.tex}
\item A card is drawn from a deck of 52 cards. Find the probability of getting a king or a heart or a red card.\\
\solution
%\input{exemplar/11/16/3/15/main.tex}
\item The probability that a student will pass his examination is 0.73, the probability of
the student getting a compartment is 0.13, and the probability that the student will
either pass or get compartment is 0.96. State True or False.\\
\solution
%\input{exemplar/11/16/3/31/main.tex}
\item A card is selected from a pack of 52 cards\\
\begin{enumerate}[label=(\alph*)]
\item How many points are there in the sample space?
\item Calculate the probability that the cards is an ace of spades.
\item Calculate the probability that the card is (i) an ace (ii)black card.\\
\end{enumerate}
%\input{ncert/11/16/3/4_1/Prob_4.tex}
\item In a non-leap year, the probability of having 53 tuesdays or 53 wednesdays is\\
\solution
%\input{exemplar/11/16/3/18/main.tex}
\item There are 1000 sealed envelopes in a box, 10 of them contain a cash prize of
Rs 100 each, 100 of them contain a cash prize of Rs 50 each and 200 of them
contain a cash prize of Rs 10 each and rest do not contain any cash prize. If they
are well shuffled and an envelope is picked up out, what is the probability that it
contains no cash prize?\\
\solution
%\input{exemplar/10/13/3/34/main.tex}
\item 
A die is thrown and a card is selected at random from a deck of 52 playing cards. The probability of getting an even number on the die and a spade card.\\
\solution
%\input{exemplar/12/13/3/78/main.tex}
\item
If 4-digit numbers greater than 5,000 are randomly formed from the digits 0, 1, 3, 5, and 7, what is the probability of forming a number divisible by 5 when:
\begin{enumerate}
    \item The digits are repeated?
    \item The repetition of digits is not allowed?
\end{enumerate}
\solution
%\input{ncert/11/16/4/9/main.tex}
\item Consider the probability space $\brak{\Omega, \mathcal{G}, P}$ where $\Omega = [0,2]$ and $\mathcal{G} = \cbrak{\phi, \Omega, [0,1], (1,2]}$. Let $X$ and $Y$ be two functions on $\Omega$ defined as
\begin{align*}
    X(\omega) = 
    \begin{cases}
        1 & \text{if }\omega \in [0, 1]\\
        2 & \text{if }\omega \in (1, 2]
    \end{cases}
\end{align*}
and
\begin{align*}
    Y(\omega) = 
    \begin{cases}
        2 & \text{if }\omega \in [0, 1.5]\\
        3 & \text{if }\omega \in (1.5, 2].
    \end{cases}
\end{align*}
Then which one of the following statements is true?
\begin{enumerate}
    \item [(A)] $X$ is a random variable with respect to $\mathcal{G}$, but $Y$ is not a random variable with respect to $\mathcal{G}$.
    \item [(B)] $Y$ is a random variable with respect to $\mathcal{G}$, but $X$ is not a random variable with respect to $\mathcal{G}$.
    \item [(C)] Neither $X$ nor $Y$ is a random variable with respect to $\mathcal{G}$.
    \item [(D)] Both $X$ and $Y$ are random variables with respect to $\mathcal{G}$.
\end{enumerate} \hfill (GATE ST 2023)\\
\solution
%\input{gate/ST/2023/14/main.tex}
	\item  A die is loaded in such a way that each odd number is twice as likely to occur as
each even number. Find $P(G)$, where $G$ is the event that a number greater than
3 occurs on a single roll of the die.
\\
\solution
		%\input{exemplar/11/16/3/5/main.tex}
	\item All the jacks, queens and kings are removed from a deck of 52 playing cards. The remaining cards are well shuffled and then one card is drawn at random. Giving ace a value 1 similar value for other cards, find the probability that the card has a value 
		\begin{enumerate}
			\item 7
			\item greater than 7
			\item less than 7
		\end{enumerate}
		%\input{exemplar/10/13/3/30/main.tex}
  \item A Lot consists of 48 mobile phones of which 42 are good, 3 have only minor defects and 3 have major defects.Varnika will buy a phone if it is good but the trader will only buy a mobile if it has no major defects. One phone is selected at random from the lot. What is the probability that it is
\begin{enumerate}
	\item acceptable to Varnika?
            \item acceptable to the trader?
\end{enumerate}
\solution
	%\input{exemplar/10/13/3/40/main.tex}
 \item A student says that if you throw a die, it will show up 1 or not 1. Therefore, the probability of getting 1 and the probability of getting 'not 1' each is equal to $\frac{1}{2}$. Is this correct? Give reasons.\\
 \solution
        %\input{exemplar/10/13/2/9/main.tex}
   \item Four candidates A, B, C, D have ap-
plied for the assignment to coach a school cricket
team. If A is twice as likely to be selected as B, and
B and C are given about the same chance of being
selected, while C is twice as likely to be selected
as D, what are the probabilities that
\begin{enumerate}
\item C will be selected?
\item A will not be selected?
\end{enumerate}
	%\input{exemplar/11/16/3/9/main.tex}
 \item A bag contain 24 balls of which $x$ balls are red, $2x$ are white and $3x$ are blue. A ball is selected at random, What is the probability that it is
\begin{enumerate}[label=\alph*)]
\item not red ?
\item white ?
\end{enumerate}
%\input{exemplar/10/13/3/41/main.tex}
If the letters of the word ASSASSINATION are arranged at random. Find the Probability that
\begin{enumerate}[label=(\alph*)]
\item Four $S's$ come consecutively in the word
\item Two  $I's$ and two $N's$ come together
\item All $A's$ are not coming together
\item No two $A's$ are coming together
\end{enumerate}
%\input{exemplar/11/16/3/14/main.tex}
	\item One urn contains two black balls (labelled B1 and B2) and one white ball. A
	second urn contains one black ball and two white balls (labelled W1 and W2).
	Suppose the following experiment is performed. One of the two urns is chosen
	at random. Next a ball is randomly chosen from the urn. Then a second ball is
	chosen at random from the same urn without replacing the first ball.
	
	\begin{enumerate}
	\item What is the probability that two black balls are chosen?
	
	\item What is the probability that two balls of opposite colour are chosen?
	\end{enumerate}
	\solution
	%\input{exemplar/11/16/3/12/main1.tex}
\end{enumerate}

	\item 
The number lock of a suitcase has 4 wheels each labelled with ten digits i.e. from 0 to 9.The lock opens with a sequence of four digits with no repeats.What is the probability of a person getting the right sequence to open the suitcase.
\\
\solution
		%\begin{enumerate}[label=\thesection.\arabic*,ref=\thesection.\theenumi]
	\item One card is drawn from a well-shuffled deck of 52 cards. Find the probability of getting
\begin{enumerate}
\item A king of red colour 
\item A face card 
\item A red face card
\item The jack of hearts
\item A spade
\item The queen of diamonds

\end{enumerate}
\solution
		%\input{ncert/10/15/1/14/main.tex}
	\item Five cards—the ten, jack, queen, king and ace of diamonds, are well-shuffled with their face downwards. One card is then picked up at random.
\begin{enumerate}
\item
What is the probability that the card is the queen? 
\item
If the queen is drawn and put aside, what is the probability that the second card picked up is (a) an ace? (b) a queen?\\
\end{enumerate}
\solution
		%\input{ncert/10/15/1/15/defs.tex}
	\item A bag contains $5$ red balls and some blue balls. If the probability of drawing a blue ball is double that if a red ball, determine the number of blue balls in the bag. 
		\\
\solution
		%\input{ncert/10/15/2/3/defs.tex}
	\item A card is selected from a pack of 52 cards.
 \begin{enumerate}[label=(\alph*)] 
                 \item How many points are there in the sample space?
                 \item Calculate the probability that the card is an ace of spades.
                 \item Calculate the probability that the card is (i) an ace and (ii) black card.
 \end{enumerate}
\solution
		%\input{ncert/11/16/3/4/main.tex}
\item Four cards are drawn from a well-shuffled deck of 52 cards. What is the probability of obtaining 3 diamonds and one spade.
\\
\solution
		%\input{ncert/11/16/4/2/defs.tex}
\item In a certain lottery 10,000 tickets are sold and ten equal prizes are awarded. What is the probability of not getting a prize if you buy (a) one ticket (b) two tickets (c) 10 tickets ?	
\\
\solution
		%\input{ncert/11/16/4/4/defs.tex}
		%
\item 
Out of 100 students, two sections of 40 and 60 are formed. If you and your friend are among the 100 students, what is the probability that
\begin{enumerate}
\item you both enter the same section?
\item you both enter the different sections?
\end{enumerate}
\solution
		%\input{ncert/11/16/4/5/defs.tex}
	\item 
The number lock of a suitcase has 4 wheels each labelled with ten digits i.e. from 0 to 9.The lock opens with a sequence of four digits with no repeats.What is the probability of a person getting the right sequence to open the suitcase.
\\
\solution
		%\input{ncert/11/16/4/10/defs.tex}
		%
\item 
Two cards are drawn at random and without replacement from a pack of 52 playing cards. Find the probability that both the cards are black.
\\
\solution
		%\input{ncert/12/13/2/2/defs.tex}
		\item A box of oranges is inspected by examining three randomly selected oranges drawn without replacement. If all the three oranges are good, the box is approved for sale, otherwise, it is rejected. Find the probability that a box containing 15 oranges out of which 12 are good and 3 are bad ones will be approved for sale.
		\label{ncert/12/13/2/3/defs.tex}
		\item Two balls are drawn at random with replacement from a box containing 10 black and 8 red balls. Find the probability that
		\label{ncert/12/13/2/12}
\begin{enumerate}
\item both balls are red.
\item first ball is black and second is red.
\item one of them is black and other is red.
\end{enumerate}

\item In a hostel, 60\% of the students read Hindi newspaper, 40\% read English newspaper and 20\% read both Hindi and English newspapers. A student is selected at random.
		\label{ncert/12/13/2/15}
\begin{enumerate}
\item Find the probability that she reads neither Hindi nor English newspapers.
\item If she reads Hindi newspaper, find the probability that she reads English newspaper.
\item If she reads English newspaper, find the probability that she reads Hindi newspaper.\\
\end{enumerate}
\item The probability of obtaining an even prime number on each die, when a pair of dice is rolled is 
\begin{enumerate}
    \item $0$ 
    
    \item $\frac{1}{3}$ 
    
    \item $\frac{1}{12}$ 
    
    \item $\frac{1}{36}$ 
\end{enumerate}
\solution
		%\input{ncert/12/13/2/17/defs.tex}
	\item A bag contains 4 red and 4 black balls, another bag contains 2 red and 6 black balls. One of the two bags is selected at random and a ball is drawn from the bag which is found to be red. Find the probability that the ball is drawn from the first bag.
\\
\solution
		%\input{ncert/12/13/3/2/main.tex}
  \item
  Cards with numbers 2 to 101 are placed in a box. A card is selected at random.Find the probability that the card has
\begin{enumerate}[label=(\roman*)]
	\item an even number 
	\item a square number
\end{enumerate}
\solution
%\input{exemplar/10/13/3/32/main.tex}
\item
The king, queen and jack of clubs are removed from a deck of 52 playing cards and then well shuffled. Now one card is drawn at random from the remaining cards.  Determine the probability that the card is
\begin{enumerate}[label=(\roman*)]
\item a club
\item 10 of hearts
\end{enumerate}
\solution
%\input{exemplar/10/13/3/29/main.tex}
\item A team of medical students doing their internship have to assist during surgeries
at a city hospital. The probabilities of surgeries rated as very complex, complex,
routine, simple or very simple are respectively, 0.15, 0.20, 0.31, 0.26, .08. Find
the probabilities that a particular surgery will be rated
\begin{enumerate}
	\item complex or very complex;
	\item neither very complex nor very simple;
	\item routine or complex
	\item routine or simple
\end{enumerate}
\solution
%\input{exemplar/11/16/3/8(1)/main.tex}
\item A card is selected from a pack of 52 cards.
\begin{enumerate}[label=(\alph*)]
    \item How many points are there in the sample space?
    \item Calculate the probability that the card is an ace of spades.
    \item Calculate the probability that the card is (i) an ace and (ii) black card.
\end{enumerate}
\solution
%\input{exemplar/11/16/3/4/main2.tex}
\item The probability that a non leap year selected at random will contain 53 sundays.
\\
\solution
%\input{exemplar/10/13/1/19/main.tex}
\item One of the four persons John, Rita, Aslam or Gurpreet will be promoted next
month. Consequently the sample space consists of four elementary outcomes
S = {John promoted, Rita promoted, Aslam promoted, Gurpreet promoted}
You are told that the chances of John’s promotion is same as that of Gurpreet,
Rita’s chances of promotion are twice as likely as Johns. Aslam’s chances are
four times that of John.
\begin{enumerate}
	\item Determine
	\begin{enumerate}
		\item P (John promoted)
		\item P (Rita promoted)
		\item P (Aslam promoted)
		\item P (Gurpreet promoted)
	\end{enumerate}
	\item If A = {John promoted or Gurpreet promoted}, find P (A).
\end{enumerate}
\solution
%\input{exemplar/11/16/3/10/main.tex}
\item A card is drawn from a deck of 52 cards. Find the probability of getting a king or a heart or a red card.\\
\solution
%\input{exemplar/11/16/3/15/main.tex}
\item The probability that a student will pass his examination is 0.73, the probability of
the student getting a compartment is 0.13, and the probability that the student will
either pass or get compartment is 0.96. State True or False.\\
\solution
%\input{exemplar/11/16/3/31/main.tex}
\item A card is selected from a pack of 52 cards\\
\begin{enumerate}[label=(\alph*)]
\item How many points are there in the sample space?
\item Calculate the probability that the cards is an ace of spades.
\item Calculate the probability that the card is (i) an ace (ii)black card.\\
\end{enumerate}
%\input{ncert/11/16/3/4_1/Prob_4.tex}
\item In a non-leap year, the probability of having 53 tuesdays or 53 wednesdays is\\
\solution
%\input{exemplar/11/16/3/18/main.tex}
\item There are 1000 sealed envelopes in a box, 10 of them contain a cash prize of
Rs 100 each, 100 of them contain a cash prize of Rs 50 each and 200 of them
contain a cash prize of Rs 10 each and rest do not contain any cash prize. If they
are well shuffled and an envelope is picked up out, what is the probability that it
contains no cash prize?\\
\solution
%\input{exemplar/10/13/3/34/main.tex}
\item 
A die is thrown and a card is selected at random from a deck of 52 playing cards. The probability of getting an even number on the die and a spade card.\\
\solution
%\input{exemplar/12/13/3/78/main.tex}
\item
If 4-digit numbers greater than 5,000 are randomly formed from the digits 0, 1, 3, 5, and 7, what is the probability of forming a number divisible by 5 when:
\begin{enumerate}
    \item The digits are repeated?
    \item The repetition of digits is not allowed?
\end{enumerate}
\solution
%\input{ncert/11/16/4/9/main.tex}
\item Consider the probability space $\brak{\Omega, \mathcal{G}, P}$ where $\Omega = [0,2]$ and $\mathcal{G} = \cbrak{\phi, \Omega, [0,1], (1,2]}$. Let $X$ and $Y$ be two functions on $\Omega$ defined as
\begin{align*}
    X(\omega) = 
    \begin{cases}
        1 & \text{if }\omega \in [0, 1]\\
        2 & \text{if }\omega \in (1, 2]
    \end{cases}
\end{align*}
and
\begin{align*}
    Y(\omega) = 
    \begin{cases}
        2 & \text{if }\omega \in [0, 1.5]\\
        3 & \text{if }\omega \in (1.5, 2].
    \end{cases}
\end{align*}
Then which one of the following statements is true?
\begin{enumerate}
    \item [(A)] $X$ is a random variable with respect to $\mathcal{G}$, but $Y$ is not a random variable with respect to $\mathcal{G}$.
    \item [(B)] $Y$ is a random variable with respect to $\mathcal{G}$, but $X$ is not a random variable with respect to $\mathcal{G}$.
    \item [(C)] Neither $X$ nor $Y$ is a random variable with respect to $\mathcal{G}$.
    \item [(D)] Both $X$ and $Y$ are random variables with respect to $\mathcal{G}$.
\end{enumerate} \hfill (GATE ST 2023)\\
\solution
%\input{gate/ST/2023/14/main.tex}
	\item  A die is loaded in such a way that each odd number is twice as likely to occur as
each even number. Find $P(G)$, where $G$ is the event that a number greater than
3 occurs on a single roll of the die.
\\
\solution
		%\input{exemplar/11/16/3/5/main.tex}
	\item All the jacks, queens and kings are removed from a deck of 52 playing cards. The remaining cards are well shuffled and then one card is drawn at random. Giving ace a value 1 similar value for other cards, find the probability that the card has a value 
		\begin{enumerate}
			\item 7
			\item greater than 7
			\item less than 7
		\end{enumerate}
		%\input{exemplar/10/13/3/30/main.tex}
  \item A Lot consists of 48 mobile phones of which 42 are good, 3 have only minor defects and 3 have major defects.Varnika will buy a phone if it is good but the trader will only buy a mobile if it has no major defects. One phone is selected at random from the lot. What is the probability that it is
\begin{enumerate}
	\item acceptable to Varnika?
            \item acceptable to the trader?
\end{enumerate}
\solution
	%\input{exemplar/10/13/3/40/main.tex}
 \item A student says that if you throw a die, it will show up 1 or not 1. Therefore, the probability of getting 1 and the probability of getting 'not 1' each is equal to $\frac{1}{2}$. Is this correct? Give reasons.\\
 \solution
        %\input{exemplar/10/13/2/9/main.tex}
   \item Four candidates A, B, C, D have ap-
plied for the assignment to coach a school cricket
team. If A is twice as likely to be selected as B, and
B and C are given about the same chance of being
selected, while C is twice as likely to be selected
as D, what are the probabilities that
\begin{enumerate}
\item C will be selected?
\item A will not be selected?
\end{enumerate}
	%\input{exemplar/11/16/3/9/main.tex}
 \item A bag contain 24 balls of which $x$ balls are red, $2x$ are white and $3x$ are blue. A ball is selected at random, What is the probability that it is
\begin{enumerate}[label=\alph*)]
\item not red ?
\item white ?
\end{enumerate}
%\input{exemplar/10/13/3/41/main.tex}
If the letters of the word ASSASSINATION are arranged at random. Find the Probability that
\begin{enumerate}[label=(\alph*)]
\item Four $S's$ come consecutively in the word
\item Two  $I's$ and two $N's$ come together
\item All $A's$ are not coming together
\item No two $A's$ are coming together
\end{enumerate}
%\input{exemplar/11/16/3/14/main.tex}
	\item One urn contains two black balls (labelled B1 and B2) and one white ball. A
	second urn contains one black ball and two white balls (labelled W1 and W2).
	Suppose the following experiment is performed. One of the two urns is chosen
	at random. Next a ball is randomly chosen from the urn. Then a second ball is
	chosen at random from the same urn without replacing the first ball.
	
	\begin{enumerate}
	\item What is the probability that two black balls are chosen?
	
	\item What is the probability that two balls of opposite colour are chosen?
	\end{enumerate}
	\solution
	%\input{exemplar/11/16/3/12/main1.tex}
\end{enumerate}

		%
\item 
Two cards are drawn at random and without replacement from a pack of 52 playing cards. Find the probability that both the cards are black.
\\
\solution
		%\begin{enumerate}[label=\thesection.\arabic*,ref=\thesection.\theenumi]
	\item One card is drawn from a well-shuffled deck of 52 cards. Find the probability of getting
\begin{enumerate}
\item A king of red colour 
\item A face card 
\item A red face card
\item The jack of hearts
\item A spade
\item The queen of diamonds

\end{enumerate}
\solution
		%\input{ncert/10/15/1/14/main.tex}
	\item Five cards—the ten, jack, queen, king and ace of diamonds, are well-shuffled with their face downwards. One card is then picked up at random.
\begin{enumerate}
\item
What is the probability that the card is the queen? 
\item
If the queen is drawn and put aside, what is the probability that the second card picked up is (a) an ace? (b) a queen?\\
\end{enumerate}
\solution
		%\input{ncert/10/15/1/15/defs.tex}
	\item A bag contains $5$ red balls and some blue balls. If the probability of drawing a blue ball is double that if a red ball, determine the number of blue balls in the bag. 
		\\
\solution
		%\input{ncert/10/15/2/3/defs.tex}
	\item A card is selected from a pack of 52 cards.
 \begin{enumerate}[label=(\alph*)] 
                 \item How many points are there in the sample space?
                 \item Calculate the probability that the card is an ace of spades.
                 \item Calculate the probability that the card is (i) an ace and (ii) black card.
 \end{enumerate}
\solution
		%\input{ncert/11/16/3/4/main.tex}
\item Four cards are drawn from a well-shuffled deck of 52 cards. What is the probability of obtaining 3 diamonds and one spade.
\\
\solution
		%\input{ncert/11/16/4/2/defs.tex}
\item In a certain lottery 10,000 tickets are sold and ten equal prizes are awarded. What is the probability of not getting a prize if you buy (a) one ticket (b) two tickets (c) 10 tickets ?	
\\
\solution
		%\input{ncert/11/16/4/4/defs.tex}
		%
\item 
Out of 100 students, two sections of 40 and 60 are formed. If you and your friend are among the 100 students, what is the probability that
\begin{enumerate}
\item you both enter the same section?
\item you both enter the different sections?
\end{enumerate}
\solution
		%\input{ncert/11/16/4/5/defs.tex}
	\item 
The number lock of a suitcase has 4 wheels each labelled with ten digits i.e. from 0 to 9.The lock opens with a sequence of four digits with no repeats.What is the probability of a person getting the right sequence to open the suitcase.
\\
\solution
		%\input{ncert/11/16/4/10/defs.tex}
		%
\item 
Two cards are drawn at random and without replacement from a pack of 52 playing cards. Find the probability that both the cards are black.
\\
\solution
		%\input{ncert/12/13/2/2/defs.tex}
		\item A box of oranges is inspected by examining three randomly selected oranges drawn without replacement. If all the three oranges are good, the box is approved for sale, otherwise, it is rejected. Find the probability that a box containing 15 oranges out of which 12 are good and 3 are bad ones will be approved for sale.
		\label{ncert/12/13/2/3/defs.tex}
		\item Two balls are drawn at random with replacement from a box containing 10 black and 8 red balls. Find the probability that
		\label{ncert/12/13/2/12}
\begin{enumerate}
\item both balls are red.
\item first ball is black and second is red.
\item one of them is black and other is red.
\end{enumerate}

\item In a hostel, 60\% of the students read Hindi newspaper, 40\% read English newspaper and 20\% read both Hindi and English newspapers. A student is selected at random.
		\label{ncert/12/13/2/15}
\begin{enumerate}
\item Find the probability that she reads neither Hindi nor English newspapers.
\item If she reads Hindi newspaper, find the probability that she reads English newspaper.
\item If she reads English newspaper, find the probability that she reads Hindi newspaper.\\
\end{enumerate}
\item The probability of obtaining an even prime number on each die, when a pair of dice is rolled is 
\begin{enumerate}
    \item $0$ 
    
    \item $\frac{1}{3}$ 
    
    \item $\frac{1}{12}$ 
    
    \item $\frac{1}{36}$ 
\end{enumerate}
\solution
		%\input{ncert/12/13/2/17/defs.tex}
	\item A bag contains 4 red and 4 black balls, another bag contains 2 red and 6 black balls. One of the two bags is selected at random and a ball is drawn from the bag which is found to be red. Find the probability that the ball is drawn from the first bag.
\\
\solution
		%\input{ncert/12/13/3/2/main.tex}
  \item
  Cards with numbers 2 to 101 are placed in a box. A card is selected at random.Find the probability that the card has
\begin{enumerate}[label=(\roman*)]
	\item an even number 
	\item a square number
\end{enumerate}
\solution
%\input{exemplar/10/13/3/32/main.tex}
\item
The king, queen and jack of clubs are removed from a deck of 52 playing cards and then well shuffled. Now one card is drawn at random from the remaining cards.  Determine the probability that the card is
\begin{enumerate}[label=(\roman*)]
\item a club
\item 10 of hearts
\end{enumerate}
\solution
%\input{exemplar/10/13/3/29/main.tex}
\item A team of medical students doing their internship have to assist during surgeries
at a city hospital. The probabilities of surgeries rated as very complex, complex,
routine, simple or very simple are respectively, 0.15, 0.20, 0.31, 0.26, .08. Find
the probabilities that a particular surgery will be rated
\begin{enumerate}
	\item complex or very complex;
	\item neither very complex nor very simple;
	\item routine or complex
	\item routine or simple
\end{enumerate}
\solution
%\input{exemplar/11/16/3/8(1)/main.tex}
\item A card is selected from a pack of 52 cards.
\begin{enumerate}[label=(\alph*)]
    \item How many points are there in the sample space?
    \item Calculate the probability that the card is an ace of spades.
    \item Calculate the probability that the card is (i) an ace and (ii) black card.
\end{enumerate}
\solution
%\input{exemplar/11/16/3/4/main2.tex}
\item The probability that a non leap year selected at random will contain 53 sundays.
\\
\solution
%\input{exemplar/10/13/1/19/main.tex}
\item One of the four persons John, Rita, Aslam or Gurpreet will be promoted next
month. Consequently the sample space consists of four elementary outcomes
S = {John promoted, Rita promoted, Aslam promoted, Gurpreet promoted}
You are told that the chances of John’s promotion is same as that of Gurpreet,
Rita’s chances of promotion are twice as likely as Johns. Aslam’s chances are
four times that of John.
\begin{enumerate}
	\item Determine
	\begin{enumerate}
		\item P (John promoted)
		\item P (Rita promoted)
		\item P (Aslam promoted)
		\item P (Gurpreet promoted)
	\end{enumerate}
	\item If A = {John promoted or Gurpreet promoted}, find P (A).
\end{enumerate}
\solution
%\input{exemplar/11/16/3/10/main.tex}
\item A card is drawn from a deck of 52 cards. Find the probability of getting a king or a heart or a red card.\\
\solution
%\input{exemplar/11/16/3/15/main.tex}
\item The probability that a student will pass his examination is 0.73, the probability of
the student getting a compartment is 0.13, and the probability that the student will
either pass or get compartment is 0.96. State True or False.\\
\solution
%\input{exemplar/11/16/3/31/main.tex}
\item A card is selected from a pack of 52 cards\\
\begin{enumerate}[label=(\alph*)]
\item How many points are there in the sample space?
\item Calculate the probability that the cards is an ace of spades.
\item Calculate the probability that the card is (i) an ace (ii)black card.\\
\end{enumerate}
%\input{ncert/11/16/3/4_1/Prob_4.tex}
\item In a non-leap year, the probability of having 53 tuesdays or 53 wednesdays is\\
\solution
%\input{exemplar/11/16/3/18/main.tex}
\item There are 1000 sealed envelopes in a box, 10 of them contain a cash prize of
Rs 100 each, 100 of them contain a cash prize of Rs 50 each and 200 of them
contain a cash prize of Rs 10 each and rest do not contain any cash prize. If they
are well shuffled and an envelope is picked up out, what is the probability that it
contains no cash prize?\\
\solution
%\input{exemplar/10/13/3/34/main.tex}
\item 
A die is thrown and a card is selected at random from a deck of 52 playing cards. The probability of getting an even number on the die and a spade card.\\
\solution
%\input{exemplar/12/13/3/78/main.tex}
\item
If 4-digit numbers greater than 5,000 are randomly formed from the digits 0, 1, 3, 5, and 7, what is the probability of forming a number divisible by 5 when:
\begin{enumerate}
    \item The digits are repeated?
    \item The repetition of digits is not allowed?
\end{enumerate}
\solution
%\input{ncert/11/16/4/9/main.tex}
\item Consider the probability space $\brak{\Omega, \mathcal{G}, P}$ where $\Omega = [0,2]$ and $\mathcal{G} = \cbrak{\phi, \Omega, [0,1], (1,2]}$. Let $X$ and $Y$ be two functions on $\Omega$ defined as
\begin{align*}
    X(\omega) = 
    \begin{cases}
        1 & \text{if }\omega \in [0, 1]\\
        2 & \text{if }\omega \in (1, 2]
    \end{cases}
\end{align*}
and
\begin{align*}
    Y(\omega) = 
    \begin{cases}
        2 & \text{if }\omega \in [0, 1.5]\\
        3 & \text{if }\omega \in (1.5, 2].
    \end{cases}
\end{align*}
Then which one of the following statements is true?
\begin{enumerate}
    \item [(A)] $X$ is a random variable with respect to $\mathcal{G}$, but $Y$ is not a random variable with respect to $\mathcal{G}$.
    \item [(B)] $Y$ is a random variable with respect to $\mathcal{G}$, but $X$ is not a random variable with respect to $\mathcal{G}$.
    \item [(C)] Neither $X$ nor $Y$ is a random variable with respect to $\mathcal{G}$.
    \item [(D)] Both $X$ and $Y$ are random variables with respect to $\mathcal{G}$.
\end{enumerate} \hfill (GATE ST 2023)\\
\solution
%\input{gate/ST/2023/14/main.tex}
	\item  A die is loaded in such a way that each odd number is twice as likely to occur as
each even number. Find $P(G)$, where $G$ is the event that a number greater than
3 occurs on a single roll of the die.
\\
\solution
		%\input{exemplar/11/16/3/5/main.tex}
	\item All the jacks, queens and kings are removed from a deck of 52 playing cards. The remaining cards are well shuffled and then one card is drawn at random. Giving ace a value 1 similar value for other cards, find the probability that the card has a value 
		\begin{enumerate}
			\item 7
			\item greater than 7
			\item less than 7
		\end{enumerate}
		%\input{exemplar/10/13/3/30/main.tex}
  \item A Lot consists of 48 mobile phones of which 42 are good, 3 have only minor defects and 3 have major defects.Varnika will buy a phone if it is good but the trader will only buy a mobile if it has no major defects. One phone is selected at random from the lot. What is the probability that it is
\begin{enumerate}
	\item acceptable to Varnika?
            \item acceptable to the trader?
\end{enumerate}
\solution
	%\input{exemplar/10/13/3/40/main.tex}
 \item A student says that if you throw a die, it will show up 1 or not 1. Therefore, the probability of getting 1 and the probability of getting 'not 1' each is equal to $\frac{1}{2}$. Is this correct? Give reasons.\\
 \solution
        %\input{exemplar/10/13/2/9/main.tex}
   \item Four candidates A, B, C, D have ap-
plied for the assignment to coach a school cricket
team. If A is twice as likely to be selected as B, and
B and C are given about the same chance of being
selected, while C is twice as likely to be selected
as D, what are the probabilities that
\begin{enumerate}
\item C will be selected?
\item A will not be selected?
\end{enumerate}
	%\input{exemplar/11/16/3/9/main.tex}
 \item A bag contain 24 balls of which $x$ balls are red, $2x$ are white and $3x$ are blue. A ball is selected at random, What is the probability that it is
\begin{enumerate}[label=\alph*)]
\item not red ?
\item white ?
\end{enumerate}
%\input{exemplar/10/13/3/41/main.tex}
If the letters of the word ASSASSINATION are arranged at random. Find the Probability that
\begin{enumerate}[label=(\alph*)]
\item Four $S's$ come consecutively in the word
\item Two  $I's$ and two $N's$ come together
\item All $A's$ are not coming together
\item No two $A's$ are coming together
\end{enumerate}
%\input{exemplar/11/16/3/14/main.tex}
	\item One urn contains two black balls (labelled B1 and B2) and one white ball. A
	second urn contains one black ball and two white balls (labelled W1 and W2).
	Suppose the following experiment is performed. One of the two urns is chosen
	at random. Next a ball is randomly chosen from the urn. Then a second ball is
	chosen at random from the same urn without replacing the first ball.
	
	\begin{enumerate}
	\item What is the probability that two black balls are chosen?
	
	\item What is the probability that two balls of opposite colour are chosen?
	\end{enumerate}
	\solution
	%\input{exemplar/11/16/3/12/main1.tex}
\end{enumerate}

		\item A box of oranges is inspected by examining three randomly selected oranges drawn without replacement. If all the three oranges are good, the box is approved for sale, otherwise, it is rejected. Find the probability that a box containing 15 oranges out of which 12 are good and 3 are bad ones will be approved for sale.
		\label{ncert/12/13/2/3/defs.tex}
		\item Two balls are drawn at random with replacement from a box containing 10 black and 8 red balls. Find the probability that
		\label{ncert/12/13/2/12}
\begin{enumerate}
\item both balls are red.
\item first ball is black and second is red.
\item one of them is black and other is red.
\end{enumerate}

\item In a hostel, 60\% of the students read Hindi newspaper, 40\% read English newspaper and 20\% read both Hindi and English newspapers. A student is selected at random.
		\label{ncert/12/13/2/15}
\begin{enumerate}
\item Find the probability that she reads neither Hindi nor English newspapers.
\item If she reads Hindi newspaper, find the probability that she reads English newspaper.
\item If she reads English newspaper, find the probability that she reads Hindi newspaper.\\
\end{enumerate}
\item The probability of obtaining an even prime number on each die, when a pair of dice is rolled is 
\begin{enumerate}
    \item $0$ 
    
    \item $\frac{1}{3}$ 
    
    \item $\frac{1}{12}$ 
    
    \item $\frac{1}{36}$ 
\end{enumerate}
\solution
		%\begin{enumerate}[label=\thesection.\arabic*,ref=\thesection.\theenumi]
	\item One card is drawn from a well-shuffled deck of 52 cards. Find the probability of getting
\begin{enumerate}
\item A king of red colour 
\item A face card 
\item A red face card
\item The jack of hearts
\item A spade
\item The queen of diamonds

\end{enumerate}
\solution
		%\input{ncert/10/15/1/14/main.tex}
	\item Five cards—the ten, jack, queen, king and ace of diamonds, are well-shuffled with their face downwards. One card is then picked up at random.
\begin{enumerate}
\item
What is the probability that the card is the queen? 
\item
If the queen is drawn and put aside, what is the probability that the second card picked up is (a) an ace? (b) a queen?\\
\end{enumerate}
\solution
		%\input{ncert/10/15/1/15/defs.tex}
	\item A bag contains $5$ red balls and some blue balls. If the probability of drawing a blue ball is double that if a red ball, determine the number of blue balls in the bag. 
		\\
\solution
		%\input{ncert/10/15/2/3/defs.tex}
	\item A card is selected from a pack of 52 cards.
 \begin{enumerate}[label=(\alph*)] 
                 \item How many points are there in the sample space?
                 \item Calculate the probability that the card is an ace of spades.
                 \item Calculate the probability that the card is (i) an ace and (ii) black card.
 \end{enumerate}
\solution
		%\input{ncert/11/16/3/4/main.tex}
\item Four cards are drawn from a well-shuffled deck of 52 cards. What is the probability of obtaining 3 diamonds and one spade.
\\
\solution
		%\input{ncert/11/16/4/2/defs.tex}
\item In a certain lottery 10,000 tickets are sold and ten equal prizes are awarded. What is the probability of not getting a prize if you buy (a) one ticket (b) two tickets (c) 10 tickets ?	
\\
\solution
		%\input{ncert/11/16/4/4/defs.tex}
		%
\item 
Out of 100 students, two sections of 40 and 60 are formed. If you and your friend are among the 100 students, what is the probability that
\begin{enumerate}
\item you both enter the same section?
\item you both enter the different sections?
\end{enumerate}
\solution
		%\input{ncert/11/16/4/5/defs.tex}
	\item 
The number lock of a suitcase has 4 wheels each labelled with ten digits i.e. from 0 to 9.The lock opens with a sequence of four digits with no repeats.What is the probability of a person getting the right sequence to open the suitcase.
\\
\solution
		%\input{ncert/11/16/4/10/defs.tex}
		%
\item 
Two cards are drawn at random and without replacement from a pack of 52 playing cards. Find the probability that both the cards are black.
\\
\solution
		%\input{ncert/12/13/2/2/defs.tex}
		\item A box of oranges is inspected by examining three randomly selected oranges drawn without replacement. If all the three oranges are good, the box is approved for sale, otherwise, it is rejected. Find the probability that a box containing 15 oranges out of which 12 are good and 3 are bad ones will be approved for sale.
		\label{ncert/12/13/2/3/defs.tex}
		\item Two balls are drawn at random with replacement from a box containing 10 black and 8 red balls. Find the probability that
		\label{ncert/12/13/2/12}
\begin{enumerate}
\item both balls are red.
\item first ball is black and second is red.
\item one of them is black and other is red.
\end{enumerate}

\item In a hostel, 60\% of the students read Hindi newspaper, 40\% read English newspaper and 20\% read both Hindi and English newspapers. A student is selected at random.
		\label{ncert/12/13/2/15}
\begin{enumerate}
\item Find the probability that she reads neither Hindi nor English newspapers.
\item If she reads Hindi newspaper, find the probability that she reads English newspaper.
\item If she reads English newspaper, find the probability that she reads Hindi newspaper.\\
\end{enumerate}
\item The probability of obtaining an even prime number on each die, when a pair of dice is rolled is 
\begin{enumerate}
    \item $0$ 
    
    \item $\frac{1}{3}$ 
    
    \item $\frac{1}{12}$ 
    
    \item $\frac{1}{36}$ 
\end{enumerate}
\solution
		%\input{ncert/12/13/2/17/defs.tex}
	\item A bag contains 4 red and 4 black balls, another bag contains 2 red and 6 black balls. One of the two bags is selected at random and a ball is drawn from the bag which is found to be red. Find the probability that the ball is drawn from the first bag.
\\
\solution
		%\input{ncert/12/13/3/2/main.tex}
  \item
  Cards with numbers 2 to 101 are placed in a box. A card is selected at random.Find the probability that the card has
\begin{enumerate}[label=(\roman*)]
	\item an even number 
	\item a square number
\end{enumerate}
\solution
%\input{exemplar/10/13/3/32/main.tex}
\item
The king, queen and jack of clubs are removed from a deck of 52 playing cards and then well shuffled. Now one card is drawn at random from the remaining cards.  Determine the probability that the card is
\begin{enumerate}[label=(\roman*)]
\item a club
\item 10 of hearts
\end{enumerate}
\solution
%\input{exemplar/10/13/3/29/main.tex}
\item A team of medical students doing their internship have to assist during surgeries
at a city hospital. The probabilities of surgeries rated as very complex, complex,
routine, simple or very simple are respectively, 0.15, 0.20, 0.31, 0.26, .08. Find
the probabilities that a particular surgery will be rated
\begin{enumerate}
	\item complex or very complex;
	\item neither very complex nor very simple;
	\item routine or complex
	\item routine or simple
\end{enumerate}
\solution
%\input{exemplar/11/16/3/8(1)/main.tex}
\item A card is selected from a pack of 52 cards.
\begin{enumerate}[label=(\alph*)]
    \item How many points are there in the sample space?
    \item Calculate the probability that the card is an ace of spades.
    \item Calculate the probability that the card is (i) an ace and (ii) black card.
\end{enumerate}
\solution
%\input{exemplar/11/16/3/4/main2.tex}
\item The probability that a non leap year selected at random will contain 53 sundays.
\\
\solution
%\input{exemplar/10/13/1/19/main.tex}
\item One of the four persons John, Rita, Aslam or Gurpreet will be promoted next
month. Consequently the sample space consists of four elementary outcomes
S = {John promoted, Rita promoted, Aslam promoted, Gurpreet promoted}
You are told that the chances of John’s promotion is same as that of Gurpreet,
Rita’s chances of promotion are twice as likely as Johns. Aslam’s chances are
four times that of John.
\begin{enumerate}
	\item Determine
	\begin{enumerate}
		\item P (John promoted)
		\item P (Rita promoted)
		\item P (Aslam promoted)
		\item P (Gurpreet promoted)
	\end{enumerate}
	\item If A = {John promoted or Gurpreet promoted}, find P (A).
\end{enumerate}
\solution
%\input{exemplar/11/16/3/10/main.tex}
\item A card is drawn from a deck of 52 cards. Find the probability of getting a king or a heart or a red card.\\
\solution
%\input{exemplar/11/16/3/15/main.tex}
\item The probability that a student will pass his examination is 0.73, the probability of
the student getting a compartment is 0.13, and the probability that the student will
either pass or get compartment is 0.96. State True or False.\\
\solution
%\input{exemplar/11/16/3/31/main.tex}
\item A card is selected from a pack of 52 cards\\
\begin{enumerate}[label=(\alph*)]
\item How many points are there in the sample space?
\item Calculate the probability that the cards is an ace of spades.
\item Calculate the probability that the card is (i) an ace (ii)black card.\\
\end{enumerate}
%\input{ncert/11/16/3/4_1/Prob_4.tex}
\item In a non-leap year, the probability of having 53 tuesdays or 53 wednesdays is\\
\solution
%\input{exemplar/11/16/3/18/main.tex}
\item There are 1000 sealed envelopes in a box, 10 of them contain a cash prize of
Rs 100 each, 100 of them contain a cash prize of Rs 50 each and 200 of them
contain a cash prize of Rs 10 each and rest do not contain any cash prize. If they
are well shuffled and an envelope is picked up out, what is the probability that it
contains no cash prize?\\
\solution
%\input{exemplar/10/13/3/34/main.tex}
\item 
A die is thrown and a card is selected at random from a deck of 52 playing cards. The probability of getting an even number on the die and a spade card.\\
\solution
%\input{exemplar/12/13/3/78/main.tex}
\item
If 4-digit numbers greater than 5,000 are randomly formed from the digits 0, 1, 3, 5, and 7, what is the probability of forming a number divisible by 5 when:
\begin{enumerate}
    \item The digits are repeated?
    \item The repetition of digits is not allowed?
\end{enumerate}
\solution
%\input{ncert/11/16/4/9/main.tex}
\item Consider the probability space $\brak{\Omega, \mathcal{G}, P}$ where $\Omega = [0,2]$ and $\mathcal{G} = \cbrak{\phi, \Omega, [0,1], (1,2]}$. Let $X$ and $Y$ be two functions on $\Omega$ defined as
\begin{align*}
    X(\omega) = 
    \begin{cases}
        1 & \text{if }\omega \in [0, 1]\\
        2 & \text{if }\omega \in (1, 2]
    \end{cases}
\end{align*}
and
\begin{align*}
    Y(\omega) = 
    \begin{cases}
        2 & \text{if }\omega \in [0, 1.5]\\
        3 & \text{if }\omega \in (1.5, 2].
    \end{cases}
\end{align*}
Then which one of the following statements is true?
\begin{enumerate}
    \item [(A)] $X$ is a random variable with respect to $\mathcal{G}$, but $Y$ is not a random variable with respect to $\mathcal{G}$.
    \item [(B)] $Y$ is a random variable with respect to $\mathcal{G}$, but $X$ is not a random variable with respect to $\mathcal{G}$.
    \item [(C)] Neither $X$ nor $Y$ is a random variable with respect to $\mathcal{G}$.
    \item [(D)] Both $X$ and $Y$ are random variables with respect to $\mathcal{G}$.
\end{enumerate} \hfill (GATE ST 2023)\\
\solution
%\input{gate/ST/2023/14/main.tex}
	\item  A die is loaded in such a way that each odd number is twice as likely to occur as
each even number. Find $P(G)$, where $G$ is the event that a number greater than
3 occurs on a single roll of the die.
\\
\solution
		%\input{exemplar/11/16/3/5/main.tex}
	\item All the jacks, queens and kings are removed from a deck of 52 playing cards. The remaining cards are well shuffled and then one card is drawn at random. Giving ace a value 1 similar value for other cards, find the probability that the card has a value 
		\begin{enumerate}
			\item 7
			\item greater than 7
			\item less than 7
		\end{enumerate}
		%\input{exemplar/10/13/3/30/main.tex}
  \item A Lot consists of 48 mobile phones of which 42 are good, 3 have only minor defects and 3 have major defects.Varnika will buy a phone if it is good but the trader will only buy a mobile if it has no major defects. One phone is selected at random from the lot. What is the probability that it is
\begin{enumerate}
	\item acceptable to Varnika?
            \item acceptable to the trader?
\end{enumerate}
\solution
	%\input{exemplar/10/13/3/40/main.tex}
 \item A student says that if you throw a die, it will show up 1 or not 1. Therefore, the probability of getting 1 and the probability of getting 'not 1' each is equal to $\frac{1}{2}$. Is this correct? Give reasons.\\
 \solution
        %\input{exemplar/10/13/2/9/main.tex}
   \item Four candidates A, B, C, D have ap-
plied for the assignment to coach a school cricket
team. If A is twice as likely to be selected as B, and
B and C are given about the same chance of being
selected, while C is twice as likely to be selected
as D, what are the probabilities that
\begin{enumerate}
\item C will be selected?
\item A will not be selected?
\end{enumerate}
	%\input{exemplar/11/16/3/9/main.tex}
 \item A bag contain 24 balls of which $x$ balls are red, $2x$ are white and $3x$ are blue. A ball is selected at random, What is the probability that it is
\begin{enumerate}[label=\alph*)]
\item not red ?
\item white ?
\end{enumerate}
%\input{exemplar/10/13/3/41/main.tex}
If the letters of the word ASSASSINATION are arranged at random. Find the Probability that
\begin{enumerate}[label=(\alph*)]
\item Four $S's$ come consecutively in the word
\item Two  $I's$ and two $N's$ come together
\item All $A's$ are not coming together
\item No two $A's$ are coming together
\end{enumerate}
%\input{exemplar/11/16/3/14/main.tex}
	\item One urn contains two black balls (labelled B1 and B2) and one white ball. A
	second urn contains one black ball and two white balls (labelled W1 and W2).
	Suppose the following experiment is performed. One of the two urns is chosen
	at random. Next a ball is randomly chosen from the urn. Then a second ball is
	chosen at random from the same urn without replacing the first ball.
	
	\begin{enumerate}
	\item What is the probability that two black balls are chosen?
	
	\item What is the probability that two balls of opposite colour are chosen?
	\end{enumerate}
	\solution
	%\input{exemplar/11/16/3/12/main1.tex}
\end{enumerate}

	\item A bag contains 4 red and 4 black balls, another bag contains 2 red and 6 black balls. One of the two bags is selected at random and a ball is drawn from the bag which is found to be red. Find the probability that the ball is drawn from the first bag.
\\
\solution
		%\begin{table}[H]
	\centering
\begin{tabular}{|c|c|c|}
\hline
Random variable &Value &Definition\\ \hline
\multirow{3}{*}{X} &0 &Slips of Rs 1\\
&1 &Slips of Rs 5\\
&2 &Slips of Rs 13\\ \hline
\multirow{2}{*}{Y} &0 &Box A\\
&1 &Box B\\\hline
\end{tabular}
\caption{}
\label{tab:Distribution}
\end{table}
See \tabref{tab:Distribution}.
\begin{align}
p_{Y}\brak{k}= \begin{cases} 
      \frac{1}{3} & {k=0} \\
      \frac{2}{3 }& {k=1} 
   \end{cases}
   \\
p_{Y|X}\brak{0|0} = \frac{19}{25}\, 
p_{Y|X}\brak{0|1} = \frac{6}{25}\,
p_{Y|X}\brak{1|0} = \frac{45}{50}\,
p_{Y|X}\brak{1|2} = \frac{5}{50}
\end{align}
The desired probability is the probability that a slip drawn at random is marked other than Rs 1,
\begin{align}
&=1-p_X\brak{0}\\
&= p_X(1) + p_X(2)
\end{align}
Using Bayes theorem,
\begin{align}
&= p_Y\brak{0} \times \pr{Y=0 | X=1} + p_Y\brak{1} \times \pr{Y=1|X=2}\\
&=\frac{1}{3} \times \frac{6}{25} + \frac{2}{3} \times \frac{5}{50}\\
&=\frac{11}{75}
\end{align}

\newpage

%\tableofcontents

\bigskip

\renewcommand{\thefigure}{\theenumi}
\renewcommand{\thetable}{\theenumi}
%\renewcommand{\theequation}{\theenumi}

%\begin{abstract}
%%\boldmath
%In this letter, an algorithm for evaluating the exact analytical bit error rate  (BER)  for the piecewise linear (PL) combiner for  multiple relays is presented. Previous results were available only for upto three relays. The algorithm is unique in the sense that  the actual mathematical expressions, that are prohibitively large, need not be explicitly obtained. The diversity gain due to multiple relays is shown through plots of the analytical BER, well supported by simulations. 
%
%\end{abstract}
% IEEEtran.cls defaults to using nonbold math in the Abstract.
% This preserves the distinction between vectors and scalars. However,
% if the journal you are submitting to favors bold math in the abstract,
% then you can use LaTeX's standard command \boldmath at the very start
% of the abstract to achieve this. Many IEEE journals frown on math
% in the abstract anyway.

% Note that keywords are not normally used for peerreview papers.
%\begin{IEEEkeywords}
%Cooperative diversity, decode and forward, piecewise linear
%\end{IEEEkeywords}



% For peer review papers, you can put extra information on the cover
% page as needed:
% \ifCLASSOPTIONpeerreview
% \begin{center} \bfseries EDICS Category: 3-BBND \end{center}
% \fi
%
% For peerreview papers, this IEEEtran command inserts a page break and
% creates the second title. It will be ignored for other modes.
%\IEEEpeerreviewmaketitle




  \item
  Cards with numbers 2 to 101 are placed in a box. A card is selected at random.Find the probability that the card has
\begin{enumerate}[label=(\roman*)]
	\item an even number 
	\item a square number
\end{enumerate}
\solution
%\begin{table}[H]
	\centering
\begin{tabular}{|c|c|c|}
\hline
Random variable &Value &Definition\\ \hline
\multirow{3}{*}{X} &0 &Slips of Rs 1\\
&1 &Slips of Rs 5\\
&2 &Slips of Rs 13\\ \hline
\multirow{2}{*}{Y} &0 &Box A\\
&1 &Box B\\\hline
\end{tabular}
\caption{}
\label{tab:Distribution}
\end{table}
See \tabref{tab:Distribution}.
\begin{align}
p_{Y}\brak{k}= \begin{cases} 
      \frac{1}{3} & {k=0} \\
      \frac{2}{3 }& {k=1} 
   \end{cases}
   \\
p_{Y|X}\brak{0|0} = \frac{19}{25}\, 
p_{Y|X}\brak{0|1} = \frac{6}{25}\,
p_{Y|X}\brak{1|0} = \frac{45}{50}\,
p_{Y|X}\brak{1|2} = \frac{5}{50}
\end{align}
The desired probability is the probability that a slip drawn at random is marked other than Rs 1,
\begin{align}
&=1-p_X\brak{0}\\
&= p_X(1) + p_X(2)
\end{align}
Using Bayes theorem,
\begin{align}
&= p_Y\brak{0} \times \pr{Y=0 | X=1} + p_Y\brak{1} \times \pr{Y=1|X=2}\\
&=\frac{1}{3} \times \frac{6}{25} + \frac{2}{3} \times \frac{5}{50}\\
&=\frac{11}{75}
\end{align}

\newpage

%\tableofcontents

\bigskip

\renewcommand{\thefigure}{\theenumi}
\renewcommand{\thetable}{\theenumi}
%\renewcommand{\theequation}{\theenumi}

%\begin{abstract}
%%\boldmath
%In this letter, an algorithm for evaluating the exact analytical bit error rate  (BER)  for the piecewise linear (PL) combiner for  multiple relays is presented. Previous results were available only for upto three relays. The algorithm is unique in the sense that  the actual mathematical expressions, that are prohibitively large, need not be explicitly obtained. The diversity gain due to multiple relays is shown through plots of the analytical BER, well supported by simulations. 
%
%\end{abstract}
% IEEEtran.cls defaults to using nonbold math in the Abstract.
% This preserves the distinction between vectors and scalars. However,
% if the journal you are submitting to favors bold math in the abstract,
% then you can use LaTeX's standard command \boldmath at the very start
% of the abstract to achieve this. Many IEEE journals frown on math
% in the abstract anyway.

% Note that keywords are not normally used for peerreview papers.
%\begin{IEEEkeywords}
%Cooperative diversity, decode and forward, piecewise linear
%\end{IEEEkeywords}



% For peer review papers, you can put extra information on the cover
% page as needed:
% \ifCLASSOPTIONpeerreview
% \begin{center} \bfseries EDICS Category: 3-BBND \end{center}
% \fi
%
% For peerreview papers, this IEEEtran command inserts a page break and
% creates the second title. It will be ignored for other modes.
%\IEEEpeerreviewmaketitle




\item
The king, queen and jack of clubs are removed from a deck of 52 playing cards and then well shuffled. Now one card is drawn at random from the remaining cards.  Determine the probability that the card is
\begin{enumerate}[label=(\roman*)]
\item a club
\item 10 of hearts
\end{enumerate}
\solution
%\begin{table}[H]
	\centering
\begin{tabular}{|c|c|c|}
\hline
Random variable &Value &Definition\\ \hline
\multirow{3}{*}{X} &0 &Slips of Rs 1\\
&1 &Slips of Rs 5\\
&2 &Slips of Rs 13\\ \hline
\multirow{2}{*}{Y} &0 &Box A\\
&1 &Box B\\\hline
\end{tabular}
\caption{}
\label{tab:Distribution}
\end{table}
See \tabref{tab:Distribution}.
\begin{align}
p_{Y}\brak{k}= \begin{cases} 
      \frac{1}{3} & {k=0} \\
      \frac{2}{3 }& {k=1} 
   \end{cases}
   \\
p_{Y|X}\brak{0|0} = \frac{19}{25}\, 
p_{Y|X}\brak{0|1} = \frac{6}{25}\,
p_{Y|X}\brak{1|0} = \frac{45}{50}\,
p_{Y|X}\brak{1|2} = \frac{5}{50}
\end{align}
The desired probability is the probability that a slip drawn at random is marked other than Rs 1,
\begin{align}
&=1-p_X\brak{0}\\
&= p_X(1) + p_X(2)
\end{align}
Using Bayes theorem,
\begin{align}
&= p_Y\brak{0} \times \pr{Y=0 | X=1} + p_Y\brak{1} \times \pr{Y=1|X=2}\\
&=\frac{1}{3} \times \frac{6}{25} + \frac{2}{3} \times \frac{5}{50}\\
&=\frac{11}{75}
\end{align}

\newpage

%\tableofcontents

\bigskip

\renewcommand{\thefigure}{\theenumi}
\renewcommand{\thetable}{\theenumi}
%\renewcommand{\theequation}{\theenumi}

%\begin{abstract}
%%\boldmath
%In this letter, an algorithm for evaluating the exact analytical bit error rate  (BER)  for the piecewise linear (PL) combiner for  multiple relays is presented. Previous results were available only for upto three relays. The algorithm is unique in the sense that  the actual mathematical expressions, that are prohibitively large, need not be explicitly obtained. The diversity gain due to multiple relays is shown through plots of the analytical BER, well supported by simulations. 
%
%\end{abstract}
% IEEEtran.cls defaults to using nonbold math in the Abstract.
% This preserves the distinction between vectors and scalars. However,
% if the journal you are submitting to favors bold math in the abstract,
% then you can use LaTeX's standard command \boldmath at the very start
% of the abstract to achieve this. Many IEEE journals frown on math
% in the abstract anyway.

% Note that keywords are not normally used for peerreview papers.
%\begin{IEEEkeywords}
%Cooperative diversity, decode and forward, piecewise linear
%\end{IEEEkeywords}



% For peer review papers, you can put extra information on the cover
% page as needed:
% \ifCLASSOPTIONpeerreview
% \begin{center} \bfseries EDICS Category: 3-BBND \end{center}
% \fi
%
% For peerreview papers, this IEEEtran command inserts a page break and
% creates the second title. It will be ignored for other modes.
%\IEEEpeerreviewmaketitle




\item A team of medical students doing their internship have to assist during surgeries
at a city hospital. The probabilities of surgeries rated as very complex, complex,
routine, simple or very simple are respectively, 0.15, 0.20, 0.31, 0.26, .08. Find
the probabilities that a particular surgery will be rated
\begin{enumerate}
	\item complex or very complex;
	\item neither very complex nor very simple;
	\item routine or complex
	\item routine or simple
\end{enumerate}
\solution
%\begin{table}[H]
	\centering
\begin{tabular}{|c|c|c|}
\hline
Random variable &Value &Definition\\ \hline
\multirow{3}{*}{X} &0 &Slips of Rs 1\\
&1 &Slips of Rs 5\\
&2 &Slips of Rs 13\\ \hline
\multirow{2}{*}{Y} &0 &Box A\\
&1 &Box B\\\hline
\end{tabular}
\caption{}
\label{tab:Distribution}
\end{table}
See \tabref{tab:Distribution}.
\begin{align}
p_{Y}\brak{k}= \begin{cases} 
      \frac{1}{3} & {k=0} \\
      \frac{2}{3 }& {k=1} 
   \end{cases}
   \\
p_{Y|X}\brak{0|0} = \frac{19}{25}\, 
p_{Y|X}\brak{0|1} = \frac{6}{25}\,
p_{Y|X}\brak{1|0} = \frac{45}{50}\,
p_{Y|X}\brak{1|2} = \frac{5}{50}
\end{align}
The desired probability is the probability that a slip drawn at random is marked other than Rs 1,
\begin{align}
&=1-p_X\brak{0}\\
&= p_X(1) + p_X(2)
\end{align}
Using Bayes theorem,
\begin{align}
&= p_Y\brak{0} \times \pr{Y=0 | X=1} + p_Y\brak{1} \times \pr{Y=1|X=2}\\
&=\frac{1}{3} \times \frac{6}{25} + \frac{2}{3} \times \frac{5}{50}\\
&=\frac{11}{75}
\end{align}

\newpage

%\tableofcontents

\bigskip

\renewcommand{\thefigure}{\theenumi}
\renewcommand{\thetable}{\theenumi}
%\renewcommand{\theequation}{\theenumi}

%\begin{abstract}
%%\boldmath
%In this letter, an algorithm for evaluating the exact analytical bit error rate  (BER)  for the piecewise linear (PL) combiner for  multiple relays is presented. Previous results were available only for upto three relays. The algorithm is unique in the sense that  the actual mathematical expressions, that are prohibitively large, need not be explicitly obtained. The diversity gain due to multiple relays is shown through plots of the analytical BER, well supported by simulations. 
%
%\end{abstract}
% IEEEtran.cls defaults to using nonbold math in the Abstract.
% This preserves the distinction between vectors and scalars. However,
% if the journal you are submitting to favors bold math in the abstract,
% then you can use LaTeX's standard command \boldmath at the very start
% of the abstract to achieve this. Many IEEE journals frown on math
% in the abstract anyway.

% Note that keywords are not normally used for peerreview papers.
%\begin{IEEEkeywords}
%Cooperative diversity, decode and forward, piecewise linear
%\end{IEEEkeywords}



% For peer review papers, you can put extra information on the cover
% page as needed:
% \ifCLASSOPTIONpeerreview
% \begin{center} \bfseries EDICS Category: 3-BBND \end{center}
% \fi
%
% For peerreview papers, this IEEEtran command inserts a page break and
% creates the second title. It will be ignored for other modes.
%\IEEEpeerreviewmaketitle




\item A card is selected from a pack of 52 cards.
\begin{enumerate}[label=(\alph*)]
    \item How many points are there in the sample space?
    \item Calculate the probability that the card is an ace of spades.
    \item Calculate the probability that the card is (i) an ace and (ii) black card.
\end{enumerate}
\solution
%Let $X$ be an bernoulli rv defined as in \tabref{tab:exemplar/11/16/3/26}.  Then, 
\begin{equation}
    p =
        \frac{4}{11} 
\end{equation}
\begin{table}[H]
	\centering
	\input{exemplar/11/16/3/26/tables/Table2.tex}
	\caption{}
        \label{tab:exemplar/11/16/3/26}
\end{table}

\item The probability that a non leap year selected at random will contain 53 sundays.
\\
\solution
%\begin{table}[H]
	\centering
\begin{tabular}{|c|c|c|}
\hline
Random variable &Value &Definition\\ \hline
\multirow{3}{*}{X} &0 &Slips of Rs 1\\
&1 &Slips of Rs 5\\
&2 &Slips of Rs 13\\ \hline
\multirow{2}{*}{Y} &0 &Box A\\
&1 &Box B\\\hline
\end{tabular}
\caption{}
\label{tab:Distribution}
\end{table}
See \tabref{tab:Distribution}.
\begin{align}
p_{Y}\brak{k}= \begin{cases} 
      \frac{1}{3} & {k=0} \\
      \frac{2}{3 }& {k=1} 
   \end{cases}
   \\
p_{Y|X}\brak{0|0} = \frac{19}{25}\, 
p_{Y|X}\brak{0|1} = \frac{6}{25}\,
p_{Y|X}\brak{1|0} = \frac{45}{50}\,
p_{Y|X}\brak{1|2} = \frac{5}{50}
\end{align}
The desired probability is the probability that a slip drawn at random is marked other than Rs 1,
\begin{align}
&=1-p_X\brak{0}\\
&= p_X(1) + p_X(2)
\end{align}
Using Bayes theorem,
\begin{align}
&= p_Y\brak{0} \times \pr{Y=0 | X=1} + p_Y\brak{1} \times \pr{Y=1|X=2}\\
&=\frac{1}{3} \times \frac{6}{25} + \frac{2}{3} \times \frac{5}{50}\\
&=\frac{11}{75}
\end{align}

\newpage

%\tableofcontents

\bigskip

\renewcommand{\thefigure}{\theenumi}
\renewcommand{\thetable}{\theenumi}
%\renewcommand{\theequation}{\theenumi}

%\begin{abstract}
%%\boldmath
%In this letter, an algorithm for evaluating the exact analytical bit error rate  (BER)  for the piecewise linear (PL) combiner for  multiple relays is presented. Previous results were available only for upto three relays. The algorithm is unique in the sense that  the actual mathematical expressions, that are prohibitively large, need not be explicitly obtained. The diversity gain due to multiple relays is shown through plots of the analytical BER, well supported by simulations. 
%
%\end{abstract}
% IEEEtran.cls defaults to using nonbold math in the Abstract.
% This preserves the distinction between vectors and scalars. However,
% if the journal you are submitting to favors bold math in the abstract,
% then you can use LaTeX's standard command \boldmath at the very start
% of the abstract to achieve this. Many IEEE journals frown on math
% in the abstract anyway.

% Note that keywords are not normally used for peerreview papers.
%\begin{IEEEkeywords}
%Cooperative diversity, decode and forward, piecewise linear
%\end{IEEEkeywords}



% For peer review papers, you can put extra information on the cover
% page as needed:
% \ifCLASSOPTIONpeerreview
% \begin{center} \bfseries EDICS Category: 3-BBND \end{center}
% \fi
%
% For peerreview papers, this IEEEtran command inserts a page break and
% creates the second title. It will be ignored for other modes.
%\IEEEpeerreviewmaketitle




\item One of the four persons John, Rita, Aslam or Gurpreet will be promoted next
month. Consequently the sample space consists of four elementary outcomes
S = {John promoted, Rita promoted, Aslam promoted, Gurpreet promoted}
You are told that the chances of John’s promotion is same as that of Gurpreet,
Rita’s chances of promotion are twice as likely as Johns. Aslam’s chances are
four times that of John.
\begin{enumerate}
	\item Determine
	\begin{enumerate}
		\item P (John promoted)
		\item P (Rita promoted)
		\item P (Aslam promoted)
		\item P (Gurpreet promoted)
	\end{enumerate}
	\item If A = {John promoted or Gurpreet promoted}, find P (A).
\end{enumerate}
\solution
%\begin{table}[H]
	\centering
\begin{tabular}{|c|c|c|}
\hline
Random variable &Value &Definition\\ \hline
\multirow{3}{*}{X} &0 &Slips of Rs 1\\
&1 &Slips of Rs 5\\
&2 &Slips of Rs 13\\ \hline
\multirow{2}{*}{Y} &0 &Box A\\
&1 &Box B\\\hline
\end{tabular}
\caption{}
\label{tab:Distribution}
\end{table}
See \tabref{tab:Distribution}.
\begin{align}
p_{Y}\brak{k}= \begin{cases} 
      \frac{1}{3} & {k=0} \\
      \frac{2}{3 }& {k=1} 
   \end{cases}
   \\
p_{Y|X}\brak{0|0} = \frac{19}{25}\, 
p_{Y|X}\brak{0|1} = \frac{6}{25}\,
p_{Y|X}\brak{1|0} = \frac{45}{50}\,
p_{Y|X}\brak{1|2} = \frac{5}{50}
\end{align}
The desired probability is the probability that a slip drawn at random is marked other than Rs 1,
\begin{align}
&=1-p_X\brak{0}\\
&= p_X(1) + p_X(2)
\end{align}
Using Bayes theorem,
\begin{align}
&= p_Y\brak{0} \times \pr{Y=0 | X=1} + p_Y\brak{1} \times \pr{Y=1|X=2}\\
&=\frac{1}{3} \times \frac{6}{25} + \frac{2}{3} \times \frac{5}{50}\\
&=\frac{11}{75}
\end{align}

\newpage

%\tableofcontents

\bigskip

\renewcommand{\thefigure}{\theenumi}
\renewcommand{\thetable}{\theenumi}
%\renewcommand{\theequation}{\theenumi}

%\begin{abstract}
%%\boldmath
%In this letter, an algorithm for evaluating the exact analytical bit error rate  (BER)  for the piecewise linear (PL) combiner for  multiple relays is presented. Previous results were available only for upto three relays. The algorithm is unique in the sense that  the actual mathematical expressions, that are prohibitively large, need not be explicitly obtained. The diversity gain due to multiple relays is shown through plots of the analytical BER, well supported by simulations. 
%
%\end{abstract}
% IEEEtran.cls defaults to using nonbold math in the Abstract.
% This preserves the distinction between vectors and scalars. However,
% if the journal you are submitting to favors bold math in the abstract,
% then you can use LaTeX's standard command \boldmath at the very start
% of the abstract to achieve this. Many IEEE journals frown on math
% in the abstract anyway.

% Note that keywords are not normally used for peerreview papers.
%\begin{IEEEkeywords}
%Cooperative diversity, decode and forward, piecewise linear
%\end{IEEEkeywords}



% For peer review papers, you can put extra information on the cover
% page as needed:
% \ifCLASSOPTIONpeerreview
% \begin{center} \bfseries EDICS Category: 3-BBND \end{center}
% \fi
%
% For peerreview papers, this IEEEtran command inserts a page break and
% creates the second title. It will be ignored for other modes.
%\IEEEpeerreviewmaketitle




\item A card is drawn from a deck of 52 cards. Find the probability of getting a king or a heart or a red card.\\
\solution
%\begin{table}[H]
	\centering
\begin{tabular}{|c|c|c|}
\hline
Random variable &Value &Definition\\ \hline
\multirow{3}{*}{X} &0 &Slips of Rs 1\\
&1 &Slips of Rs 5\\
&2 &Slips of Rs 13\\ \hline
\multirow{2}{*}{Y} &0 &Box A\\
&1 &Box B\\\hline
\end{tabular}
\caption{}
\label{tab:Distribution}
\end{table}
See \tabref{tab:Distribution}.
\begin{align}
p_{Y}\brak{k}= \begin{cases} 
      \frac{1}{3} & {k=0} \\
      \frac{2}{3 }& {k=1} 
   \end{cases}
   \\
p_{Y|X}\brak{0|0} = \frac{19}{25}\, 
p_{Y|X}\brak{0|1} = \frac{6}{25}\,
p_{Y|X}\brak{1|0} = \frac{45}{50}\,
p_{Y|X}\brak{1|2} = \frac{5}{50}
\end{align}
The desired probability is the probability that a slip drawn at random is marked other than Rs 1,
\begin{align}
&=1-p_X\brak{0}\\
&= p_X(1) + p_X(2)
\end{align}
Using Bayes theorem,
\begin{align}
&= p_Y\brak{0} \times \pr{Y=0 | X=1} + p_Y\brak{1} \times \pr{Y=1|X=2}\\
&=\frac{1}{3} \times \frac{6}{25} + \frac{2}{3} \times \frac{5}{50}\\
&=\frac{11}{75}
\end{align}

\newpage

%\tableofcontents

\bigskip

\renewcommand{\thefigure}{\theenumi}
\renewcommand{\thetable}{\theenumi}
%\renewcommand{\theequation}{\theenumi}

%\begin{abstract}
%%\boldmath
%In this letter, an algorithm for evaluating the exact analytical bit error rate  (BER)  for the piecewise linear (PL) combiner for  multiple relays is presented. Previous results were available only for upto three relays. The algorithm is unique in the sense that  the actual mathematical expressions, that are prohibitively large, need not be explicitly obtained. The diversity gain due to multiple relays is shown through plots of the analytical BER, well supported by simulations. 
%
%\end{abstract}
% IEEEtran.cls defaults to using nonbold math in the Abstract.
% This preserves the distinction between vectors and scalars. However,
% if the journal you are submitting to favors bold math in the abstract,
% then you can use LaTeX's standard command \boldmath at the very start
% of the abstract to achieve this. Many IEEE journals frown on math
% in the abstract anyway.

% Note that keywords are not normally used for peerreview papers.
%\begin{IEEEkeywords}
%Cooperative diversity, decode and forward, piecewise linear
%\end{IEEEkeywords}



% For peer review papers, you can put extra information on the cover
% page as needed:
% \ifCLASSOPTIONpeerreview
% \begin{center} \bfseries EDICS Category: 3-BBND \end{center}
% \fi
%
% For peerreview papers, this IEEEtran command inserts a page break and
% creates the second title. It will be ignored for other modes.
%\IEEEpeerreviewmaketitle




\item The probability that a student will pass his examination is 0.73, the probability of
the student getting a compartment is 0.13, and the probability that the student will
either pass or get compartment is 0.96. State True or False.\\
\solution
%\begin{table}[H]
	\centering
\begin{tabular}{|c|c|c|}
\hline
Random variable &Value &Definition\\ \hline
\multirow{3}{*}{X} &0 &Slips of Rs 1\\
&1 &Slips of Rs 5\\
&2 &Slips of Rs 13\\ \hline
\multirow{2}{*}{Y} &0 &Box A\\
&1 &Box B\\\hline
\end{tabular}
\caption{}
\label{tab:Distribution}
\end{table}
See \tabref{tab:Distribution}.
\begin{align}
p_{Y}\brak{k}= \begin{cases} 
      \frac{1}{3} & {k=0} \\
      \frac{2}{3 }& {k=1} 
   \end{cases}
   \\
p_{Y|X}\brak{0|0} = \frac{19}{25}\, 
p_{Y|X}\brak{0|1} = \frac{6}{25}\,
p_{Y|X}\brak{1|0} = \frac{45}{50}\,
p_{Y|X}\brak{1|2} = \frac{5}{50}
\end{align}
The desired probability is the probability that a slip drawn at random is marked other than Rs 1,
\begin{align}
&=1-p_X\brak{0}\\
&= p_X(1) + p_X(2)
\end{align}
Using Bayes theorem,
\begin{align}
&= p_Y\brak{0} \times \pr{Y=0 | X=1} + p_Y\brak{1} \times \pr{Y=1|X=2}\\
&=\frac{1}{3} \times \frac{6}{25} + \frac{2}{3} \times \frac{5}{50}\\
&=\frac{11}{75}
\end{align}

\newpage

%\tableofcontents

\bigskip

\renewcommand{\thefigure}{\theenumi}
\renewcommand{\thetable}{\theenumi}
%\renewcommand{\theequation}{\theenumi}

%\begin{abstract}
%%\boldmath
%In this letter, an algorithm for evaluating the exact analytical bit error rate  (BER)  for the piecewise linear (PL) combiner for  multiple relays is presented. Previous results were available only for upto three relays. The algorithm is unique in the sense that  the actual mathematical expressions, that are prohibitively large, need not be explicitly obtained. The diversity gain due to multiple relays is shown through plots of the analytical BER, well supported by simulations. 
%
%\end{abstract}
% IEEEtran.cls defaults to using nonbold math in the Abstract.
% This preserves the distinction between vectors and scalars. However,
% if the journal you are submitting to favors bold math in the abstract,
% then you can use LaTeX's standard command \boldmath at the very start
% of the abstract to achieve this. Many IEEE journals frown on math
% in the abstract anyway.

% Note that keywords are not normally used for peerreview papers.
%\begin{IEEEkeywords}
%Cooperative diversity, decode and forward, piecewise linear
%\end{IEEEkeywords}



% For peer review papers, you can put extra information on the cover
% page as needed:
% \ifCLASSOPTIONpeerreview
% \begin{center} \bfseries EDICS Category: 3-BBND \end{center}
% \fi
%
% For peerreview papers, this IEEEtran command inserts a page break and
% creates the second title. It will be ignored for other modes.
%\IEEEpeerreviewmaketitle




\item A card is selected from a pack of 52 cards\\
\begin{enumerate}[label=(\alph*)]
\item How many points are there in the sample space?
\item Calculate the probability that the cards is an ace of spades.
\item Calculate the probability that the card is (i) an ace (ii)black card.\\
\end{enumerate}
%\input{ncert/11/16/3/4_1/Prob_4.tex}
\item In a non-leap year, the probability of having 53 tuesdays or 53 wednesdays is\\
\solution
%A non-leap year has a total of 365 days, and a week has 7 days.\\
So it can be expressed as 
\begin{align}
365\text{days} &=52\times 7+1 \text{day}
\end{align}
$\implies$ 52 tuesdays or wednesdays\\
Random variable X denotes the days of a week
\begin{align}
p_X\brak{k}&=\frac{1}{7}; \quad \brak{1<k<7}
\end{align}
So the probability of extra day being tuesday or wednesday is
\begin{align}
p_X\brak{3}+p_X\brak{4}&=\frac{1}{7}+\frac{1}{7}=\frac{2}{7}
\end{align}



\item There are 1000 sealed envelopes in a box, 10 of them contain a cash prize of
Rs 100 each, 100 of them contain a cash prize of Rs 50 each and 200 of them
contain a cash prize of Rs 10 each and rest do not contain any cash prize. If they
are well shuffled and an envelope is picked up out, what is the probability that it
contains no cash prize?\\
\solution
%\begin{table}[H]
	\centering
\begin{tabular}{|c|c|c|}
\hline
Random variable &Value &Definition\\ \hline
\multirow{3}{*}{X} &0 &Slips of Rs 1\\
&1 &Slips of Rs 5\\
&2 &Slips of Rs 13\\ \hline
\multirow{2}{*}{Y} &0 &Box A\\
&1 &Box B\\\hline
\end{tabular}
\caption{}
\label{tab:Distribution}
\end{table}
See \tabref{tab:Distribution}.
\begin{align}
p_{Y}\brak{k}= \begin{cases} 
      \frac{1}{3} & {k=0} \\
      \frac{2}{3 }& {k=1} 
   \end{cases}
   \\
p_{Y|X}\brak{0|0} = \frac{19}{25}\, 
p_{Y|X}\brak{0|1} = \frac{6}{25}\,
p_{Y|X}\brak{1|0} = \frac{45}{50}\,
p_{Y|X}\brak{1|2} = \frac{5}{50}
\end{align}
The desired probability is the probability that a slip drawn at random is marked other than Rs 1,
\begin{align}
&=1-p_X\brak{0}\\
&= p_X(1) + p_X(2)
\end{align}
Using Bayes theorem,
\begin{align}
&= p_Y\brak{0} \times \pr{Y=0 | X=1} + p_Y\brak{1} \times \pr{Y=1|X=2}\\
&=\frac{1}{3} \times \frac{6}{25} + \frac{2}{3} \times \frac{5}{50}\\
&=\frac{11}{75}
\end{align}

\newpage

%\tableofcontents

\bigskip

\renewcommand{\thefigure}{\theenumi}
\renewcommand{\thetable}{\theenumi}
%\renewcommand{\theequation}{\theenumi}

%\begin{abstract}
%%\boldmath
%In this letter, an algorithm for evaluating the exact analytical bit error rate  (BER)  for the piecewise linear (PL) combiner for  multiple relays is presented. Previous results were available only for upto three relays. The algorithm is unique in the sense that  the actual mathematical expressions, that are prohibitively large, need not be explicitly obtained. The diversity gain due to multiple relays is shown through plots of the analytical BER, well supported by simulations. 
%
%\end{abstract}
% IEEEtran.cls defaults to using nonbold math in the Abstract.
% This preserves the distinction between vectors and scalars. However,
% if the journal you are submitting to favors bold math in the abstract,
% then you can use LaTeX's standard command \boldmath at the very start
% of the abstract to achieve this. Many IEEE journals frown on math
% in the abstract anyway.

% Note that keywords are not normally used for peerreview papers.
%\begin{IEEEkeywords}
%Cooperative diversity, decode and forward, piecewise linear
%\end{IEEEkeywords}



% For peer review papers, you can put extra information on the cover
% page as needed:
% \ifCLASSOPTIONpeerreview
% \begin{center} \bfseries EDICS Category: 3-BBND \end{center}
% \fi
%
% For peerreview papers, this IEEEtran command inserts a page break and
% creates the second title. It will be ignored for other modes.
%\IEEEpeerreviewmaketitle




\item 
A die is thrown and a card is selected at random from a deck of 52 playing cards. The probability of getting an even number on the die and a spade card.\\
\solution
%\begin{table}[H]
	\centering
\begin{tabular}{|c|c|c|}
\hline
Random variable &Value &Definition\\ \hline
\multirow{3}{*}{X} &0 &Slips of Rs 1\\
&1 &Slips of Rs 5\\
&2 &Slips of Rs 13\\ \hline
\multirow{2}{*}{Y} &0 &Box A\\
&1 &Box B\\\hline
\end{tabular}
\caption{}
\label{tab:Distribution}
\end{table}
See \tabref{tab:Distribution}.
\begin{align}
p_{Y}\brak{k}= \begin{cases} 
      \frac{1}{3} & {k=0} \\
      \frac{2}{3 }& {k=1} 
   \end{cases}
   \\
p_{Y|X}\brak{0|0} = \frac{19}{25}\, 
p_{Y|X}\brak{0|1} = \frac{6}{25}\,
p_{Y|X}\brak{1|0} = \frac{45}{50}\,
p_{Y|X}\brak{1|2} = \frac{5}{50}
\end{align}
The desired probability is the probability that a slip drawn at random is marked other than Rs 1,
\begin{align}
&=1-p_X\brak{0}\\
&= p_X(1) + p_X(2)
\end{align}
Using Bayes theorem,
\begin{align}
&= p_Y\brak{0} \times \pr{Y=0 | X=1} + p_Y\brak{1} \times \pr{Y=1|X=2}\\
&=\frac{1}{3} \times \frac{6}{25} + \frac{2}{3} \times \frac{5}{50}\\
&=\frac{11}{75}
\end{align}

\newpage

%\tableofcontents

\bigskip

\renewcommand{\thefigure}{\theenumi}
\renewcommand{\thetable}{\theenumi}
%\renewcommand{\theequation}{\theenumi}

%\begin{abstract}
%%\boldmath
%In this letter, an algorithm for evaluating the exact analytical bit error rate  (BER)  for the piecewise linear (PL) combiner for  multiple relays is presented. Previous results were available only for upto three relays. The algorithm is unique in the sense that  the actual mathematical expressions, that are prohibitively large, need not be explicitly obtained. The diversity gain due to multiple relays is shown through plots of the analytical BER, well supported by simulations. 
%
%\end{abstract}
% IEEEtran.cls defaults to using nonbold math in the Abstract.
% This preserves the distinction between vectors and scalars. However,
% if the journal you are submitting to favors bold math in the abstract,
% then you can use LaTeX's standard command \boldmath at the very start
% of the abstract to achieve this. Many IEEE journals frown on math
% in the abstract anyway.

% Note that keywords are not normally used for peerreview papers.
%\begin{IEEEkeywords}
%Cooperative diversity, decode and forward, piecewise linear
%\end{IEEEkeywords}



% For peer review papers, you can put extra information on the cover
% page as needed:
% \ifCLASSOPTIONpeerreview
% \begin{center} \bfseries EDICS Category: 3-BBND \end{center}
% \fi
%
% For peerreview papers, this IEEEtran command inserts a page break and
% creates the second title. It will be ignored for other modes.
%\IEEEpeerreviewmaketitle




\item
If 4-digit numbers greater than 5,000 are randomly formed from the digits 0, 1, 3, 5, and 7, what is the probability of forming a number divisible by 5 when:
\begin{enumerate}
    \item The digits are repeated?
    \item The repetition of digits is not allowed?
\end{enumerate}
\solution
%\begin{table}[H]
	\centering
\begin{tabular}{|c|c|c|}
\hline
Random variable &Value &Definition\\ \hline
\multirow{3}{*}{X} &0 &Slips of Rs 1\\
&1 &Slips of Rs 5\\
&2 &Slips of Rs 13\\ \hline
\multirow{2}{*}{Y} &0 &Box A\\
&1 &Box B\\\hline
\end{tabular}
\caption{}
\label{tab:Distribution}
\end{table}
See \tabref{tab:Distribution}.
\begin{align}
p_{Y}\brak{k}= \begin{cases} 
      \frac{1}{3} & {k=0} \\
      \frac{2}{3 }& {k=1} 
   \end{cases}
   \\
p_{Y|X}\brak{0|0} = \frac{19}{25}\, 
p_{Y|X}\brak{0|1} = \frac{6}{25}\,
p_{Y|X}\brak{1|0} = \frac{45}{50}\,
p_{Y|X}\brak{1|2} = \frac{5}{50}
\end{align}
The desired probability is the probability that a slip drawn at random is marked other than Rs 1,
\begin{align}
&=1-p_X\brak{0}\\
&= p_X(1) + p_X(2)
\end{align}
Using Bayes theorem,
\begin{align}
&= p_Y\brak{0} \times \pr{Y=0 | X=1} + p_Y\brak{1} \times \pr{Y=1|X=2}\\
&=\frac{1}{3} \times \frac{6}{25} + \frac{2}{3} \times \frac{5}{50}\\
&=\frac{11}{75}
\end{align}

\newpage

%\tableofcontents

\bigskip

\renewcommand{\thefigure}{\theenumi}
\renewcommand{\thetable}{\theenumi}
%\renewcommand{\theequation}{\theenumi}

%\begin{abstract}
%%\boldmath
%In this letter, an algorithm for evaluating the exact analytical bit error rate  (BER)  for the piecewise linear (PL) combiner for  multiple relays is presented. Previous results were available only for upto three relays. The algorithm is unique in the sense that  the actual mathematical expressions, that are prohibitively large, need not be explicitly obtained. The diversity gain due to multiple relays is shown through plots of the analytical BER, well supported by simulations. 
%
%\end{abstract}
% IEEEtran.cls defaults to using nonbold math in the Abstract.
% This preserves the distinction between vectors and scalars. However,
% if the journal you are submitting to favors bold math in the abstract,
% then you can use LaTeX's standard command \boldmath at the very start
% of the abstract to achieve this. Many IEEE journals frown on math
% in the abstract anyway.

% Note that keywords are not normally used for peerreview papers.
%\begin{IEEEkeywords}
%Cooperative diversity, decode and forward, piecewise linear
%\end{IEEEkeywords}



% For peer review papers, you can put extra information on the cover
% page as needed:
% \ifCLASSOPTIONpeerreview
% \begin{center} \bfseries EDICS Category: 3-BBND \end{center}
% \fi
%
% For peerreview papers, this IEEEtran command inserts a page break and
% creates the second title. It will be ignored for other modes.
%\IEEEpeerreviewmaketitle




\item Consider the probability space $\brak{\Omega, \mathcal{G}, P}$ where $\Omega = [0,2]$ and $\mathcal{G} = \cbrak{\phi, \Omega, [0,1], (1,2]}$. Let $X$ and $Y$ be two functions on $\Omega$ defined as
\begin{align*}
    X(\omega) = 
    \begin{cases}
        1 & \text{if }\omega \in [0, 1]\\
        2 & \text{if }\omega \in (1, 2]
    \end{cases}
\end{align*}
and
\begin{align*}
    Y(\omega) = 
    \begin{cases}
        2 & \text{if }\omega \in [0, 1.5]\\
        3 & \text{if }\omega \in (1.5, 2].
    \end{cases}
\end{align*}
Then which one of the following statements is true?
\begin{enumerate}
    \item [(A)] $X$ is a random variable with respect to $\mathcal{G}$, but $Y$ is not a random variable with respect to $\mathcal{G}$.
    \item [(B)] $Y$ is a random variable with respect to $\mathcal{G}$, but $X$ is not a random variable with respect to $\mathcal{G}$.
    \item [(C)] Neither $X$ nor $Y$ is a random variable with respect to $\mathcal{G}$.
    \item [(D)] Both $X$ and $Y$ are random variables with respect to $\mathcal{G}$.
\end{enumerate} \hfill (GATE ST 2023)\\
\solution
%\begin{table}[H]
	\centering
\begin{tabular}{|c|c|c|}
\hline
Random variable &Value &Definition\\ \hline
\multirow{3}{*}{X} &0 &Slips of Rs 1\\
&1 &Slips of Rs 5\\
&2 &Slips of Rs 13\\ \hline
\multirow{2}{*}{Y} &0 &Box A\\
&1 &Box B\\\hline
\end{tabular}
\caption{}
\label{tab:Distribution}
\end{table}
See \tabref{tab:Distribution}.
\begin{align}
p_{Y}\brak{k}= \begin{cases} 
      \frac{1}{3} & {k=0} \\
      \frac{2}{3 }& {k=1} 
   \end{cases}
   \\
p_{Y|X}\brak{0|0} = \frac{19}{25}\, 
p_{Y|X}\brak{0|1} = \frac{6}{25}\,
p_{Y|X}\brak{1|0} = \frac{45}{50}\,
p_{Y|X}\brak{1|2} = \frac{5}{50}
\end{align}
The desired probability is the probability that a slip drawn at random is marked other than Rs 1,
\begin{align}
&=1-p_X\brak{0}\\
&= p_X(1) + p_X(2)
\end{align}
Using Bayes theorem,
\begin{align}
&= p_Y\brak{0} \times \pr{Y=0 | X=1} + p_Y\brak{1} \times \pr{Y=1|X=2}\\
&=\frac{1}{3} \times \frac{6}{25} + \frac{2}{3} \times \frac{5}{50}\\
&=\frac{11}{75}
\end{align}

\newpage

%\tableofcontents

\bigskip

\renewcommand{\thefigure}{\theenumi}
\renewcommand{\thetable}{\theenumi}
%\renewcommand{\theequation}{\theenumi}

%\begin{abstract}
%%\boldmath
%In this letter, an algorithm for evaluating the exact analytical bit error rate  (BER)  for the piecewise linear (PL) combiner for  multiple relays is presented. Previous results were available only for upto three relays. The algorithm is unique in the sense that  the actual mathematical expressions, that are prohibitively large, need not be explicitly obtained. The diversity gain due to multiple relays is shown through plots of the analytical BER, well supported by simulations. 
%
%\end{abstract}
% IEEEtran.cls defaults to using nonbold math in the Abstract.
% This preserves the distinction between vectors and scalars. However,
% if the journal you are submitting to favors bold math in the abstract,
% then you can use LaTeX's standard command \boldmath at the very start
% of the abstract to achieve this. Many IEEE journals frown on math
% in the abstract anyway.

% Note that keywords are not normally used for peerreview papers.
%\begin{IEEEkeywords}
%Cooperative diversity, decode and forward, piecewise linear
%\end{IEEEkeywords}



% For peer review papers, you can put extra information on the cover
% page as needed:
% \ifCLASSOPTIONpeerreview
% \begin{center} \bfseries EDICS Category: 3-BBND \end{center}
% \fi
%
% For peerreview papers, this IEEEtran command inserts a page break and
% creates the second title. It will be ignored for other modes.
%\IEEEpeerreviewmaketitle




	\item  A die is loaded in such a way that each odd number is twice as likely to occur as
each even number. Find $P(G)$, where $G$ is the event that a number greater than
3 occurs on a single roll of the die.
\\
\solution
		%\begin{table}[H]
	\centering
\begin{tabular}{|c|c|c|}
\hline
Random variable &Value &Definition\\ \hline
\multirow{3}{*}{X} &0 &Slips of Rs 1\\
&1 &Slips of Rs 5\\
&2 &Slips of Rs 13\\ \hline
\multirow{2}{*}{Y} &0 &Box A\\
&1 &Box B\\\hline
\end{tabular}
\caption{}
\label{tab:Distribution}
\end{table}
See \tabref{tab:Distribution}.
\begin{align}
p_{Y}\brak{k}= \begin{cases} 
      \frac{1}{3} & {k=0} \\
      \frac{2}{3 }& {k=1} 
   \end{cases}
   \\
p_{Y|X}\brak{0|0} = \frac{19}{25}\, 
p_{Y|X}\brak{0|1} = \frac{6}{25}\,
p_{Y|X}\brak{1|0} = \frac{45}{50}\,
p_{Y|X}\brak{1|2} = \frac{5}{50}
\end{align}
The desired probability is the probability that a slip drawn at random is marked other than Rs 1,
\begin{align}
&=1-p_X\brak{0}\\
&= p_X(1) + p_X(2)
\end{align}
Using Bayes theorem,
\begin{align}
&= p_Y\brak{0} \times \pr{Y=0 | X=1} + p_Y\brak{1} \times \pr{Y=1|X=2}\\
&=\frac{1}{3} \times \frac{6}{25} + \frac{2}{3} \times \frac{5}{50}\\
&=\frac{11}{75}
\end{align}

\newpage

%\tableofcontents

\bigskip

\renewcommand{\thefigure}{\theenumi}
\renewcommand{\thetable}{\theenumi}
%\renewcommand{\theequation}{\theenumi}

%\begin{abstract}
%%\boldmath
%In this letter, an algorithm for evaluating the exact analytical bit error rate  (BER)  for the piecewise linear (PL) combiner for  multiple relays is presented. Previous results were available only for upto three relays. The algorithm is unique in the sense that  the actual mathematical expressions, that are prohibitively large, need not be explicitly obtained. The diversity gain due to multiple relays is shown through plots of the analytical BER, well supported by simulations. 
%
%\end{abstract}
% IEEEtran.cls defaults to using nonbold math in the Abstract.
% This preserves the distinction between vectors and scalars. However,
% if the journal you are submitting to favors bold math in the abstract,
% then you can use LaTeX's standard command \boldmath at the very start
% of the abstract to achieve this. Many IEEE journals frown on math
% in the abstract anyway.

% Note that keywords are not normally used for peerreview papers.
%\begin{IEEEkeywords}
%Cooperative diversity, decode and forward, piecewise linear
%\end{IEEEkeywords}



% For peer review papers, you can put extra information on the cover
% page as needed:
% \ifCLASSOPTIONpeerreview
% \begin{center} \bfseries EDICS Category: 3-BBND \end{center}
% \fi
%
% For peerreview papers, this IEEEtran command inserts a page break and
% creates the second title. It will be ignored for other modes.
%\IEEEpeerreviewmaketitle




	\item All the jacks, queens and kings are removed from a deck of 52 playing cards. The remaining cards are well shuffled and then one card is drawn at random. Giving ace a value 1 similar value for other cards, find the probability that the card has a value 
		\begin{enumerate}
			\item 7
			\item greater than 7
			\item less than 7
		\end{enumerate}
		%Number of cards left after removing all jacks, queens and kings 
\begin{align}
N	= 52 - 4\times 3
	= 40
\end{align}
%\begin{table}[H]
%\def\arraystretch{1.2}
%\begin{tabular}{|c|c|c|}
%\hline
%	\textbf{Parameter} &\textbf{Value} &\textbf{Description}\\ \hline
%	$X$ &1-10 &Represents the value of the card picked \\ \hline
%\end{tabular}
%\end{table}
Let $1 \le X \le 10$ be the value of the card picked.  Then,
\begin{align}
	p_X(k) &= \Pr(X=k)\ \forall\ 1 \leq k \leq 10\\
	&= \frac{4\times 1}{40}\\
	&= \frac{1}{10}\\
	\therefore p_X(k) &= 
	\begin{cases}
		\frac{1}{10} & 1 \leq k \leq 10\\
		0 & \text{otherwise}
	\end{cases}
\end{align}
and
\begin{align}
	F_{X}(k) &= \sum_{m=0}^{k}p_{X}(m) \quad 1 \leq k \leq 10\\
	&= \frac{k}{10}\\
	\therefore F_{X}(k) &= 
	\begin{cases}
		0 & k \leq 0\\
		\frac{k}{10} & 1\leq k \leq 10\\
		1 & k > 10 
	\end{cases}
\end{align}
\begin{enumerate}
	\item Probability that card has value equal to 7 is
		\begin{align}
			 p_{X}(7)
			= \frac{1}{10}
		\end{align}
	\item Probability that card has value greater than 7 is
		\begin{align}
			1 - F_X(7)
			&= 1 - \frac{7}{10}
			\\
			&= \frac{3}{10}
		\end{align}
	\item Probability that card has value less than 7 is
		\begin{align}
			 F_{X}(6)
			=\frac{6}{10}
		\end{align}
\end{enumerate}

  \item A Lot consists of 48 mobile phones of which 42 are good, 3 have only minor defects and 3 have major defects.Varnika will buy a phone if it is good but the trader will only buy a mobile if it has no major defects. One phone is selected at random from the lot. What is the probability that it is
\begin{enumerate}
	\item acceptable to Varnika?
            \item acceptable to the trader?
\end{enumerate}
\solution
	%\begin{table}[H]
	\centering
\begin{tabular}{|c|c|c|}
\hline
Random variable &Value &Definition\\ \hline
\multirow{3}{*}{X} &0 &Slips of Rs 1\\
&1 &Slips of Rs 5\\
&2 &Slips of Rs 13\\ \hline
\multirow{2}{*}{Y} &0 &Box A\\
&1 &Box B\\\hline
\end{tabular}
\caption{}
\label{tab:Distribution}
\end{table}
See \tabref{tab:Distribution}.
\begin{align}
p_{Y}\brak{k}= \begin{cases} 
      \frac{1}{3} & {k=0} \\
      \frac{2}{3 }& {k=1} 
   \end{cases}
   \\
p_{Y|X}\brak{0|0} = \frac{19}{25}\, 
p_{Y|X}\brak{0|1} = \frac{6}{25}\,
p_{Y|X}\brak{1|0} = \frac{45}{50}\,
p_{Y|X}\brak{1|2} = \frac{5}{50}
\end{align}
The desired probability is the probability that a slip drawn at random is marked other than Rs 1,
\begin{align}
&=1-p_X\brak{0}\\
&= p_X(1) + p_X(2)
\end{align}
Using Bayes theorem,
\begin{align}
&= p_Y\brak{0} \times \pr{Y=0 | X=1} + p_Y\brak{1} \times \pr{Y=1|X=2}\\
&=\frac{1}{3} \times \frac{6}{25} + \frac{2}{3} \times \frac{5}{50}\\
&=\frac{11}{75}
\end{align}

\newpage

%\tableofcontents

\bigskip

\renewcommand{\thefigure}{\theenumi}
\renewcommand{\thetable}{\theenumi}
%\renewcommand{\theequation}{\theenumi}

%\begin{abstract}
%%\boldmath
%In this letter, an algorithm for evaluating the exact analytical bit error rate  (BER)  for the piecewise linear (PL) combiner for  multiple relays is presented. Previous results were available only for upto three relays. The algorithm is unique in the sense that  the actual mathematical expressions, that are prohibitively large, need not be explicitly obtained. The diversity gain due to multiple relays is shown through plots of the analytical BER, well supported by simulations. 
%
%\end{abstract}
% IEEEtran.cls defaults to using nonbold math in the Abstract.
% This preserves the distinction between vectors and scalars. However,
% if the journal you are submitting to favors bold math in the abstract,
% then you can use LaTeX's standard command \boldmath at the very start
% of the abstract to achieve this. Many IEEE journals frown on math
% in the abstract anyway.

% Note that keywords are not normally used for peerreview papers.
%\begin{IEEEkeywords}
%Cooperative diversity, decode and forward, piecewise linear
%\end{IEEEkeywords}



% For peer review papers, you can put extra information on the cover
% page as needed:
% \ifCLASSOPTIONpeerreview
% \begin{center} \bfseries EDICS Category: 3-BBND \end{center}
% \fi
%
% For peerreview papers, this IEEEtran command inserts a page break and
% creates the second title. It will be ignored for other modes.
%\IEEEpeerreviewmaketitle




 \item A student says that if you throw a die, it will show up 1 or not 1. Therefore, the probability of getting 1 and the probability of getting 'not 1' each is equal to $\frac{1}{2}$. Is this correct? Give reasons.\\
 \solution
        %\begin{table}[H]
	\centering
\begin{tabular}{|c|c|c|}
\hline
Random variable &Value &Definition\\ \hline
\multirow{3}{*}{X} &0 &Slips of Rs 1\\
&1 &Slips of Rs 5\\
&2 &Slips of Rs 13\\ \hline
\multirow{2}{*}{Y} &0 &Box A\\
&1 &Box B\\\hline
\end{tabular}
\caption{}
\label{tab:Distribution}
\end{table}
See \tabref{tab:Distribution}.
\begin{align}
p_{Y}\brak{k}= \begin{cases} 
      \frac{1}{3} & {k=0} \\
      \frac{2}{3 }& {k=1} 
   \end{cases}
   \\
p_{Y|X}\brak{0|0} = \frac{19}{25}\, 
p_{Y|X}\brak{0|1} = \frac{6}{25}\,
p_{Y|X}\brak{1|0} = \frac{45}{50}\,
p_{Y|X}\brak{1|2} = \frac{5}{50}
\end{align}
The desired probability is the probability that a slip drawn at random is marked other than Rs 1,
\begin{align}
&=1-p_X\brak{0}\\
&= p_X(1) + p_X(2)
\end{align}
Using Bayes theorem,
\begin{align}
&= p_Y\brak{0} \times \pr{Y=0 | X=1} + p_Y\brak{1} \times \pr{Y=1|X=2}\\
&=\frac{1}{3} \times \frac{6}{25} + \frac{2}{3} \times \frac{5}{50}\\
&=\frac{11}{75}
\end{align}

\newpage

%\tableofcontents

\bigskip

\renewcommand{\thefigure}{\theenumi}
\renewcommand{\thetable}{\theenumi}
%\renewcommand{\theequation}{\theenumi}

%\begin{abstract}
%%\boldmath
%In this letter, an algorithm for evaluating the exact analytical bit error rate  (BER)  for the piecewise linear (PL) combiner for  multiple relays is presented. Previous results were available only for upto three relays. The algorithm is unique in the sense that  the actual mathematical expressions, that are prohibitively large, need not be explicitly obtained. The diversity gain due to multiple relays is shown through plots of the analytical BER, well supported by simulations. 
%
%\end{abstract}
% IEEEtran.cls defaults to using nonbold math in the Abstract.
% This preserves the distinction between vectors and scalars. However,
% if the journal you are submitting to favors bold math in the abstract,
% then you can use LaTeX's standard command \boldmath at the very start
% of the abstract to achieve this. Many IEEE journals frown on math
% in the abstract anyway.

% Note that keywords are not normally used for peerreview papers.
%\begin{IEEEkeywords}
%Cooperative diversity, decode and forward, piecewise linear
%\end{IEEEkeywords}



% For peer review papers, you can put extra information on the cover
% page as needed:
% \ifCLASSOPTIONpeerreview
% \begin{center} \bfseries EDICS Category: 3-BBND \end{center}
% \fi
%
% For peerreview papers, this IEEEtran command inserts a page break and
% creates the second title. It will be ignored for other modes.
%\IEEEpeerreviewmaketitle




   \item Four candidates A, B, C, D have ap-
plied for the assignment to coach a school cricket
team. If A is twice as likely to be selected as B, and
B and C are given about the same chance of being
selected, while C is twice as likely to be selected
as D, what are the probabilities that
\begin{enumerate}
\item C will be selected?
\item A will not be selected?
\end{enumerate}
	%\begin{table}[H]
	\centering
\begin{tabular}{|c|c|c|}
\hline
Random variable &Value &Definition\\ \hline
\multirow{3}{*}{X} &0 &Slips of Rs 1\\
&1 &Slips of Rs 5\\
&2 &Slips of Rs 13\\ \hline
\multirow{2}{*}{Y} &0 &Box A\\
&1 &Box B\\\hline
\end{tabular}
\caption{}
\label{tab:Distribution}
\end{table}
See \tabref{tab:Distribution}.
\begin{align}
p_{Y}\brak{k}= \begin{cases} 
      \frac{1}{3} & {k=0} \\
      \frac{2}{3 }& {k=1} 
   \end{cases}
   \\
p_{Y|X}\brak{0|0} = \frac{19}{25}\, 
p_{Y|X}\brak{0|1} = \frac{6}{25}\,
p_{Y|X}\brak{1|0} = \frac{45}{50}\,
p_{Y|X}\brak{1|2} = \frac{5}{50}
\end{align}
The desired probability is the probability that a slip drawn at random is marked other than Rs 1,
\begin{align}
&=1-p_X\brak{0}\\
&= p_X(1) + p_X(2)
\end{align}
Using Bayes theorem,
\begin{align}
&= p_Y\brak{0} \times \pr{Y=0 | X=1} + p_Y\brak{1} \times \pr{Y=1|X=2}\\
&=\frac{1}{3} \times \frac{6}{25} + \frac{2}{3} \times \frac{5}{50}\\
&=\frac{11}{75}
\end{align}

\newpage

%\tableofcontents

\bigskip

\renewcommand{\thefigure}{\theenumi}
\renewcommand{\thetable}{\theenumi}
%\renewcommand{\theequation}{\theenumi}

%\begin{abstract}
%%\boldmath
%In this letter, an algorithm for evaluating the exact analytical bit error rate  (BER)  for the piecewise linear (PL) combiner for  multiple relays is presented. Previous results were available only for upto three relays. The algorithm is unique in the sense that  the actual mathematical expressions, that are prohibitively large, need not be explicitly obtained. The diversity gain due to multiple relays is shown through plots of the analytical BER, well supported by simulations. 
%
%\end{abstract}
% IEEEtran.cls defaults to using nonbold math in the Abstract.
% This preserves the distinction between vectors and scalars. However,
% if the journal you are submitting to favors bold math in the abstract,
% then you can use LaTeX's standard command \boldmath at the very start
% of the abstract to achieve this. Many IEEE journals frown on math
% in the abstract anyway.

% Note that keywords are not normally used for peerreview papers.
%\begin{IEEEkeywords}
%Cooperative diversity, decode and forward, piecewise linear
%\end{IEEEkeywords}



% For peer review papers, you can put extra information on the cover
% page as needed:
% \ifCLASSOPTIONpeerreview
% \begin{center} \bfseries EDICS Category: 3-BBND \end{center}
% \fi
%
% For peerreview papers, this IEEEtran command inserts a page break and
% creates the second title. It will be ignored for other modes.
%\IEEEpeerreviewmaketitle




 \item A bag contain 24 balls of which $x$ balls are red, $2x$ are white and $3x$ are blue. A ball is selected at random, What is the probability that it is
\begin{enumerate}[label=\alph*)]
\item not red ?
\item white ?
\end{enumerate}
%\begin{table}[H]
	\centering
\begin{tabular}{|c|c|c|}
\hline
Random variable &Value &Definition\\ \hline
\multirow{3}{*}{X} &0 &Slips of Rs 1\\
&1 &Slips of Rs 5\\
&2 &Slips of Rs 13\\ \hline
\multirow{2}{*}{Y} &0 &Box A\\
&1 &Box B\\\hline
\end{tabular}
\caption{}
\label{tab:Distribution}
\end{table}
See \tabref{tab:Distribution}.
\begin{align}
p_{Y}\brak{k}= \begin{cases} 
      \frac{1}{3} & {k=0} \\
      \frac{2}{3 }& {k=1} 
   \end{cases}
   \\
p_{Y|X}\brak{0|0} = \frac{19}{25}\, 
p_{Y|X}\brak{0|1} = \frac{6}{25}\,
p_{Y|X}\brak{1|0} = \frac{45}{50}\,
p_{Y|X}\brak{1|2} = \frac{5}{50}
\end{align}
The desired probability is the probability that a slip drawn at random is marked other than Rs 1,
\begin{align}
&=1-p_X\brak{0}\\
&= p_X(1) + p_X(2)
\end{align}
Using Bayes theorem,
\begin{align}
&= p_Y\brak{0} \times \pr{Y=0 | X=1} + p_Y\brak{1} \times \pr{Y=1|X=2}\\
&=\frac{1}{3} \times \frac{6}{25} + \frac{2}{3} \times \frac{5}{50}\\
&=\frac{11}{75}
\end{align}

\newpage

%\tableofcontents

\bigskip

\renewcommand{\thefigure}{\theenumi}
\renewcommand{\thetable}{\theenumi}
%\renewcommand{\theequation}{\theenumi}

%\begin{abstract}
%%\boldmath
%In this letter, an algorithm for evaluating the exact analytical bit error rate  (BER)  for the piecewise linear (PL) combiner for  multiple relays is presented. Previous results were available only for upto three relays. The algorithm is unique in the sense that  the actual mathematical expressions, that are prohibitively large, need not be explicitly obtained. The diversity gain due to multiple relays is shown through plots of the analytical BER, well supported by simulations. 
%
%\end{abstract}
% IEEEtran.cls defaults to using nonbold math in the Abstract.
% This preserves the distinction between vectors and scalars. However,
% if the journal you are submitting to favors bold math in the abstract,
% then you can use LaTeX's standard command \boldmath at the very start
% of the abstract to achieve this. Many IEEE journals frown on math
% in the abstract anyway.

% Note that keywords are not normally used for peerreview papers.
%\begin{IEEEkeywords}
%Cooperative diversity, decode and forward, piecewise linear
%\end{IEEEkeywords}



% For peer review papers, you can put extra information on the cover
% page as needed:
% \ifCLASSOPTIONpeerreview
% \begin{center} \bfseries EDICS Category: 3-BBND \end{center}
% \fi
%
% For peerreview papers, this IEEEtran command inserts a page break and
% creates the second title. It will be ignored for other modes.
%\IEEEpeerreviewmaketitle




If the letters of the word ASSASSINATION are arranged at random. Find the Probability that
\begin{enumerate}[label=(\alph*)]
\item Four $S's$ come consecutively in the word
\item Two  $I's$ and two $N's$ come together
\item All $A's$ are not coming together
\item No two $A's$ are coming together
\end{enumerate}
%\begin{table}[H]
	\centering
\begin{tabular}{|c|c|c|}
\hline
Random variable &Value &Definition\\ \hline
\multirow{3}{*}{X} &0 &Slips of Rs 1\\
&1 &Slips of Rs 5\\
&2 &Slips of Rs 13\\ \hline
\multirow{2}{*}{Y} &0 &Box A\\
&1 &Box B\\\hline
\end{tabular}
\caption{}
\label{tab:Distribution}
\end{table}
See \tabref{tab:Distribution}.
\begin{align}
p_{Y}\brak{k}= \begin{cases} 
      \frac{1}{3} & {k=0} \\
      \frac{2}{3 }& {k=1} 
   \end{cases}
   \\
p_{Y|X}\brak{0|0} = \frac{19}{25}\, 
p_{Y|X}\brak{0|1} = \frac{6}{25}\,
p_{Y|X}\brak{1|0} = \frac{45}{50}\,
p_{Y|X}\brak{1|2} = \frac{5}{50}
\end{align}
The desired probability is the probability that a slip drawn at random is marked other than Rs 1,
\begin{align}
&=1-p_X\brak{0}\\
&= p_X(1) + p_X(2)
\end{align}
Using Bayes theorem,
\begin{align}
&= p_Y\brak{0} \times \pr{Y=0 | X=1} + p_Y\brak{1} \times \pr{Y=1|X=2}\\
&=\frac{1}{3} \times \frac{6}{25} + \frac{2}{3} \times \frac{5}{50}\\
&=\frac{11}{75}
\end{align}

\newpage

%\tableofcontents

\bigskip

\renewcommand{\thefigure}{\theenumi}
\renewcommand{\thetable}{\theenumi}
%\renewcommand{\theequation}{\theenumi}

%\begin{abstract}
%%\boldmath
%In this letter, an algorithm for evaluating the exact analytical bit error rate  (BER)  for the piecewise linear (PL) combiner for  multiple relays is presented. Previous results were available only for upto three relays. The algorithm is unique in the sense that  the actual mathematical expressions, that are prohibitively large, need not be explicitly obtained. The diversity gain due to multiple relays is shown through plots of the analytical BER, well supported by simulations. 
%
%\end{abstract}
% IEEEtran.cls defaults to using nonbold math in the Abstract.
% This preserves the distinction between vectors and scalars. However,
% if the journal you are submitting to favors bold math in the abstract,
% then you can use LaTeX's standard command \boldmath at the very start
% of the abstract to achieve this. Many IEEE journals frown on math
% in the abstract anyway.

% Note that keywords are not normally used for peerreview papers.
%\begin{IEEEkeywords}
%Cooperative diversity, decode and forward, piecewise linear
%\end{IEEEkeywords}



% For peer review papers, you can put extra information on the cover
% page as needed:
% \ifCLASSOPTIONpeerreview
% \begin{center} \bfseries EDICS Category: 3-BBND \end{center}
% \fi
%
% For peerreview papers, this IEEEtran command inserts a page break and
% creates the second title. It will be ignored for other modes.
%\IEEEpeerreviewmaketitle




	\item One urn contains two black balls (labelled B1 and B2) and one white ball. A
	second urn contains one black ball and two white balls (labelled W1 and W2).
	Suppose the following experiment is performed. One of the two urns is chosen
	at random. Next a ball is randomly chosen from the urn. Then a second ball is
	chosen at random from the same urn without replacing the first ball.
	
	\begin{enumerate}
	\item What is the probability that two black balls are chosen?
	
	\item What is the probability that two balls of opposite colour are chosen?
	\end{enumerate}
	\solution
	%\begin{align}
    \label{eq:12.13.6.18.1}
	\because	\pr{A|B} &> \pr{A},\
\frac{\pr{AB}}{\pr{B}} > \pr{A}
\\
    \label{eq:12.13.6.18.2}
	\implies \pr{AB} &> \pr{A}\pr{B}
	\\
	\text{or, } \frac{\pr{AB}}{\pr{A}} &=\pr{B|A} > \pr{A}
\end{align}

\end{enumerate}

		%
\item 
Two cards are drawn at random and without replacement from a pack of 52 playing cards. Find the probability that both the cards are black.
\\
\solution
		%\begin{enumerate}[label=\thesection.\arabic*,ref=\thesection.\theenumi]
	\item One card is drawn from a well-shuffled deck of 52 cards. Find the probability of getting
\begin{enumerate}
\item A king of red colour 
\item A face card 
\item A red face card
\item The jack of hearts
\item A spade
\item The queen of diamonds

\end{enumerate}
\solution
		%\begin{table}[H]
	\centering
\begin{tabular}{|c|c|c|}
\hline
Random variable &Value &Definition\\ \hline
\multirow{3}{*}{X} &0 &Slips of Rs 1\\
&1 &Slips of Rs 5\\
&2 &Slips of Rs 13\\ \hline
\multirow{2}{*}{Y} &0 &Box A\\
&1 &Box B\\\hline
\end{tabular}
\caption{}
\label{tab:Distribution}
\end{table}
See \tabref{tab:Distribution}.
\begin{align}
p_{Y}\brak{k}= \begin{cases} 
      \frac{1}{3} & {k=0} \\
      \frac{2}{3 }& {k=1} 
   \end{cases}
   \\
p_{Y|X}\brak{0|0} = \frac{19}{25}\, 
p_{Y|X}\brak{0|1} = \frac{6}{25}\,
p_{Y|X}\brak{1|0} = \frac{45}{50}\,
p_{Y|X}\brak{1|2} = \frac{5}{50}
\end{align}
The desired probability is the probability that a slip drawn at random is marked other than Rs 1,
\begin{align}
&=1-p_X\brak{0}\\
&= p_X(1) + p_X(2)
\end{align}
Using Bayes theorem,
\begin{align}
&= p_Y\brak{0} \times \pr{Y=0 | X=1} + p_Y\brak{1} \times \pr{Y=1|X=2}\\
&=\frac{1}{3} \times \frac{6}{25} + \frac{2}{3} \times \frac{5}{50}\\
&=\frac{11}{75}
\end{align}

\newpage

%\tableofcontents

\bigskip

\renewcommand{\thefigure}{\theenumi}
\renewcommand{\thetable}{\theenumi}
%\renewcommand{\theequation}{\theenumi}

%\begin{abstract}
%%\boldmath
%In this letter, an algorithm for evaluating the exact analytical bit error rate  (BER)  for the piecewise linear (PL) combiner for  multiple relays is presented. Previous results were available only for upto three relays. The algorithm is unique in the sense that  the actual mathematical expressions, that are prohibitively large, need not be explicitly obtained. The diversity gain due to multiple relays is shown through plots of the analytical BER, well supported by simulations. 
%
%\end{abstract}
% IEEEtran.cls defaults to using nonbold math in the Abstract.
% This preserves the distinction between vectors and scalars. However,
% if the journal you are submitting to favors bold math in the abstract,
% then you can use LaTeX's standard command \boldmath at the very start
% of the abstract to achieve this. Many IEEE journals frown on math
% in the abstract anyway.

% Note that keywords are not normally used for peerreview papers.
%\begin{IEEEkeywords}
%Cooperative diversity, decode and forward, piecewise linear
%\end{IEEEkeywords}



% For peer review papers, you can put extra information on the cover
% page as needed:
% \ifCLASSOPTIONpeerreview
% \begin{center} \bfseries EDICS Category: 3-BBND \end{center}
% \fi
%
% For peerreview papers, this IEEEtran command inserts a page break and
% creates the second title. It will be ignored for other modes.
%\IEEEpeerreviewmaketitle




	\item Five cards—the ten, jack, queen, king and ace of diamonds, are well-shuffled with their face downwards. One card is then picked up at random.
\begin{enumerate}
\item
What is the probability that the card is the queen? 
\item
If the queen is drawn and put aside, what is the probability that the second card picked up is (a) an ace? (b) a queen?\\
\end{enumerate}
\solution
		%\begin{enumerate}[label=\thesection.\arabic*,ref=\thesection.\theenumi]
	\item One card is drawn from a well-shuffled deck of 52 cards. Find the probability of getting
\begin{enumerate}
\item A king of red colour 
\item A face card 
\item A red face card
\item The jack of hearts
\item A spade
\item The queen of diamonds

\end{enumerate}
\solution
		%\input{ncert/10/15/1/14/main.tex}
	\item Five cards—the ten, jack, queen, king and ace of diamonds, are well-shuffled with their face downwards. One card is then picked up at random.
\begin{enumerate}
\item
What is the probability that the card is the queen? 
\item
If the queen is drawn and put aside, what is the probability that the second card picked up is (a) an ace? (b) a queen?\\
\end{enumerate}
\solution
		%\input{ncert/10/15/1/15/defs.tex}
	\item A bag contains $5$ red balls and some blue balls. If the probability of drawing a blue ball is double that if a red ball, determine the number of blue balls in the bag. 
		\\
\solution
		%\input{ncert/10/15/2/3/defs.tex}
	\item A card is selected from a pack of 52 cards.
 \begin{enumerate}[label=(\alph*)] 
                 \item How many points are there in the sample space?
                 \item Calculate the probability that the card is an ace of spades.
                 \item Calculate the probability that the card is (i) an ace and (ii) black card.
 \end{enumerate}
\solution
		%\input{ncert/11/16/3/4/main.tex}
\item Four cards are drawn from a well-shuffled deck of 52 cards. What is the probability of obtaining 3 diamonds and one spade.
\\
\solution
		%\input{ncert/11/16/4/2/defs.tex}
\item In a certain lottery 10,000 tickets are sold and ten equal prizes are awarded. What is the probability of not getting a prize if you buy (a) one ticket (b) two tickets (c) 10 tickets ?	
\\
\solution
		%\input{ncert/11/16/4/4/defs.tex}
		%
\item 
Out of 100 students, two sections of 40 and 60 are formed. If you and your friend are among the 100 students, what is the probability that
\begin{enumerate}
\item you both enter the same section?
\item you both enter the different sections?
\end{enumerate}
\solution
		%\input{ncert/11/16/4/5/defs.tex}
	\item 
The number lock of a suitcase has 4 wheels each labelled with ten digits i.e. from 0 to 9.The lock opens with a sequence of four digits with no repeats.What is the probability of a person getting the right sequence to open the suitcase.
\\
\solution
		%\input{ncert/11/16/4/10/defs.tex}
		%
\item 
Two cards are drawn at random and without replacement from a pack of 52 playing cards. Find the probability that both the cards are black.
\\
\solution
		%\input{ncert/12/13/2/2/defs.tex}
		\item A box of oranges is inspected by examining three randomly selected oranges drawn without replacement. If all the three oranges are good, the box is approved for sale, otherwise, it is rejected. Find the probability that a box containing 15 oranges out of which 12 are good and 3 are bad ones will be approved for sale.
		\label{ncert/12/13/2/3/defs.tex}
		\item Two balls are drawn at random with replacement from a box containing 10 black and 8 red balls. Find the probability that
		\label{ncert/12/13/2/12}
\begin{enumerate}
\item both balls are red.
\item first ball is black and second is red.
\item one of them is black and other is red.
\end{enumerate}

\item In a hostel, 60\% of the students read Hindi newspaper, 40\% read English newspaper and 20\% read both Hindi and English newspapers. A student is selected at random.
		\label{ncert/12/13/2/15}
\begin{enumerate}
\item Find the probability that she reads neither Hindi nor English newspapers.
\item If she reads Hindi newspaper, find the probability that she reads English newspaper.
\item If she reads English newspaper, find the probability that she reads Hindi newspaper.\\
\end{enumerate}
\item The probability of obtaining an even prime number on each die, when a pair of dice is rolled is 
\begin{enumerate}
    \item $0$ 
    
    \item $\frac{1}{3}$ 
    
    \item $\frac{1}{12}$ 
    
    \item $\frac{1}{36}$ 
\end{enumerate}
\solution
		%\input{ncert/12/13/2/17/defs.tex}
	\item A bag contains 4 red and 4 black balls, another bag contains 2 red and 6 black balls. One of the two bags is selected at random and a ball is drawn from the bag which is found to be red. Find the probability that the ball is drawn from the first bag.
\\
\solution
		%\input{ncert/12/13/3/2/main.tex}
  \item
  Cards with numbers 2 to 101 are placed in a box. A card is selected at random.Find the probability that the card has
\begin{enumerate}[label=(\roman*)]
	\item an even number 
	\item a square number
\end{enumerate}
\solution
%\input{exemplar/10/13/3/32/main.tex}
\item
The king, queen and jack of clubs are removed from a deck of 52 playing cards and then well shuffled. Now one card is drawn at random from the remaining cards.  Determine the probability that the card is
\begin{enumerate}[label=(\roman*)]
\item a club
\item 10 of hearts
\end{enumerate}
\solution
%\input{exemplar/10/13/3/29/main.tex}
\item A team of medical students doing their internship have to assist during surgeries
at a city hospital. The probabilities of surgeries rated as very complex, complex,
routine, simple or very simple are respectively, 0.15, 0.20, 0.31, 0.26, .08. Find
the probabilities that a particular surgery will be rated
\begin{enumerate}
	\item complex or very complex;
	\item neither very complex nor very simple;
	\item routine or complex
	\item routine or simple
\end{enumerate}
\solution
%\input{exemplar/11/16/3/8(1)/main.tex}
\item A card is selected from a pack of 52 cards.
\begin{enumerate}[label=(\alph*)]
    \item How many points are there in the sample space?
    \item Calculate the probability that the card is an ace of spades.
    \item Calculate the probability that the card is (i) an ace and (ii) black card.
\end{enumerate}
\solution
%\input{exemplar/11/16/3/4/main2.tex}
\item The probability that a non leap year selected at random will contain 53 sundays.
\\
\solution
%\input{exemplar/10/13/1/19/main.tex}
\item One of the four persons John, Rita, Aslam or Gurpreet will be promoted next
month. Consequently the sample space consists of four elementary outcomes
S = {John promoted, Rita promoted, Aslam promoted, Gurpreet promoted}
You are told that the chances of John’s promotion is same as that of Gurpreet,
Rita’s chances of promotion are twice as likely as Johns. Aslam’s chances are
four times that of John.
\begin{enumerate}
	\item Determine
	\begin{enumerate}
		\item P (John promoted)
		\item P (Rita promoted)
		\item P (Aslam promoted)
		\item P (Gurpreet promoted)
	\end{enumerate}
	\item If A = {John promoted or Gurpreet promoted}, find P (A).
\end{enumerate}
\solution
%\input{exemplar/11/16/3/10/main.tex}
\item A card is drawn from a deck of 52 cards. Find the probability of getting a king or a heart or a red card.\\
\solution
%\input{exemplar/11/16/3/15/main.tex}
\item The probability that a student will pass his examination is 0.73, the probability of
the student getting a compartment is 0.13, and the probability that the student will
either pass or get compartment is 0.96. State True or False.\\
\solution
%\input{exemplar/11/16/3/31/main.tex}
\item A card is selected from a pack of 52 cards\\
\begin{enumerate}[label=(\alph*)]
\item How many points are there in the sample space?
\item Calculate the probability that the cards is an ace of spades.
\item Calculate the probability that the card is (i) an ace (ii)black card.\\
\end{enumerate}
%\input{ncert/11/16/3/4_1/Prob_4.tex}
\item In a non-leap year, the probability of having 53 tuesdays or 53 wednesdays is\\
\solution
%\input{exemplar/11/16/3/18/main.tex}
\item There are 1000 sealed envelopes in a box, 10 of them contain a cash prize of
Rs 100 each, 100 of them contain a cash prize of Rs 50 each and 200 of them
contain a cash prize of Rs 10 each and rest do not contain any cash prize. If they
are well shuffled and an envelope is picked up out, what is the probability that it
contains no cash prize?\\
\solution
%\input{exemplar/10/13/3/34/main.tex}
\item 
A die is thrown and a card is selected at random from a deck of 52 playing cards. The probability of getting an even number on the die and a spade card.\\
\solution
%\input{exemplar/12/13/3/78/main.tex}
\item
If 4-digit numbers greater than 5,000 are randomly formed from the digits 0, 1, 3, 5, and 7, what is the probability of forming a number divisible by 5 when:
\begin{enumerate}
    \item The digits are repeated?
    \item The repetition of digits is not allowed?
\end{enumerate}
\solution
%\input{ncert/11/16/4/9/main.tex}
\item Consider the probability space $\brak{\Omega, \mathcal{G}, P}$ where $\Omega = [0,2]$ and $\mathcal{G} = \cbrak{\phi, \Omega, [0,1], (1,2]}$. Let $X$ and $Y$ be two functions on $\Omega$ defined as
\begin{align*}
    X(\omega) = 
    \begin{cases}
        1 & \text{if }\omega \in [0, 1]\\
        2 & \text{if }\omega \in (1, 2]
    \end{cases}
\end{align*}
and
\begin{align*}
    Y(\omega) = 
    \begin{cases}
        2 & \text{if }\omega \in [0, 1.5]\\
        3 & \text{if }\omega \in (1.5, 2].
    \end{cases}
\end{align*}
Then which one of the following statements is true?
\begin{enumerate}
    \item [(A)] $X$ is a random variable with respect to $\mathcal{G}$, but $Y$ is not a random variable with respect to $\mathcal{G}$.
    \item [(B)] $Y$ is a random variable with respect to $\mathcal{G}$, but $X$ is not a random variable with respect to $\mathcal{G}$.
    \item [(C)] Neither $X$ nor $Y$ is a random variable with respect to $\mathcal{G}$.
    \item [(D)] Both $X$ and $Y$ are random variables with respect to $\mathcal{G}$.
\end{enumerate} \hfill (GATE ST 2023)\\
\solution
%\input{gate/ST/2023/14/main.tex}
	\item  A die is loaded in such a way that each odd number is twice as likely to occur as
each even number. Find $P(G)$, where $G$ is the event that a number greater than
3 occurs on a single roll of the die.
\\
\solution
		%\input{exemplar/11/16/3/5/main.tex}
	\item All the jacks, queens and kings are removed from a deck of 52 playing cards. The remaining cards are well shuffled and then one card is drawn at random. Giving ace a value 1 similar value for other cards, find the probability that the card has a value 
		\begin{enumerate}
			\item 7
			\item greater than 7
			\item less than 7
		\end{enumerate}
		%\input{exemplar/10/13/3/30/main.tex}
  \item A Lot consists of 48 mobile phones of which 42 are good, 3 have only minor defects and 3 have major defects.Varnika will buy a phone if it is good but the trader will only buy a mobile if it has no major defects. One phone is selected at random from the lot. What is the probability that it is
\begin{enumerate}
	\item acceptable to Varnika?
            \item acceptable to the trader?
\end{enumerate}
\solution
	%\input{exemplar/10/13/3/40/main.tex}
 \item A student says that if you throw a die, it will show up 1 or not 1. Therefore, the probability of getting 1 and the probability of getting 'not 1' each is equal to $\frac{1}{2}$. Is this correct? Give reasons.\\
 \solution
        %\input{exemplar/10/13/2/9/main.tex}
   \item Four candidates A, B, C, D have ap-
plied for the assignment to coach a school cricket
team. If A is twice as likely to be selected as B, and
B and C are given about the same chance of being
selected, while C is twice as likely to be selected
as D, what are the probabilities that
\begin{enumerate}
\item C will be selected?
\item A will not be selected?
\end{enumerate}
	%\input{exemplar/11/16/3/9/main.tex}
 \item A bag contain 24 balls of which $x$ balls are red, $2x$ are white and $3x$ are blue. A ball is selected at random, What is the probability that it is
\begin{enumerate}[label=\alph*)]
\item not red ?
\item white ?
\end{enumerate}
%\input{exemplar/10/13/3/41/main.tex}
If the letters of the word ASSASSINATION are arranged at random. Find the Probability that
\begin{enumerate}[label=(\alph*)]
\item Four $S's$ come consecutively in the word
\item Two  $I's$ and two $N's$ come together
\item All $A's$ are not coming together
\item No two $A's$ are coming together
\end{enumerate}
%\input{exemplar/11/16/3/14/main.tex}
	\item One urn contains two black balls (labelled B1 and B2) and one white ball. A
	second urn contains one black ball and two white balls (labelled W1 and W2).
	Suppose the following experiment is performed. One of the two urns is chosen
	at random. Next a ball is randomly chosen from the urn. Then a second ball is
	chosen at random from the same urn without replacing the first ball.
	
	\begin{enumerate}
	\item What is the probability that two black balls are chosen?
	
	\item What is the probability that two balls of opposite colour are chosen?
	\end{enumerate}
	\solution
	%\input{exemplar/11/16/3/12/main1.tex}
\end{enumerate}

	\item A bag contains $5$ red balls and some blue balls. If the probability of drawing a blue ball is double that if a red ball, determine the number of blue balls in the bag. 
		\\
\solution
		%\begin{enumerate}[label=\thesection.\arabic*,ref=\thesection.\theenumi]
	\item One card is drawn from a well-shuffled deck of 52 cards. Find the probability of getting
\begin{enumerate}
\item A king of red colour 
\item A face card 
\item A red face card
\item The jack of hearts
\item A spade
\item The queen of diamonds

\end{enumerate}
\solution
		%\input{ncert/10/15/1/14/main.tex}
	\item Five cards—the ten, jack, queen, king and ace of diamonds, are well-shuffled with their face downwards. One card is then picked up at random.
\begin{enumerate}
\item
What is the probability that the card is the queen? 
\item
If the queen is drawn and put aside, what is the probability that the second card picked up is (a) an ace? (b) a queen?\\
\end{enumerate}
\solution
		%\input{ncert/10/15/1/15/defs.tex}
	\item A bag contains $5$ red balls and some blue balls. If the probability of drawing a blue ball is double that if a red ball, determine the number of blue balls in the bag. 
		\\
\solution
		%\input{ncert/10/15/2/3/defs.tex}
	\item A card is selected from a pack of 52 cards.
 \begin{enumerate}[label=(\alph*)] 
                 \item How many points are there in the sample space?
                 \item Calculate the probability that the card is an ace of spades.
                 \item Calculate the probability that the card is (i) an ace and (ii) black card.
 \end{enumerate}
\solution
		%\input{ncert/11/16/3/4/main.tex}
\item Four cards are drawn from a well-shuffled deck of 52 cards. What is the probability of obtaining 3 diamonds and one spade.
\\
\solution
		%\input{ncert/11/16/4/2/defs.tex}
\item In a certain lottery 10,000 tickets are sold and ten equal prizes are awarded. What is the probability of not getting a prize if you buy (a) one ticket (b) two tickets (c) 10 tickets ?	
\\
\solution
		%\input{ncert/11/16/4/4/defs.tex}
		%
\item 
Out of 100 students, two sections of 40 and 60 are formed. If you and your friend are among the 100 students, what is the probability that
\begin{enumerate}
\item you both enter the same section?
\item you both enter the different sections?
\end{enumerate}
\solution
		%\input{ncert/11/16/4/5/defs.tex}
	\item 
The number lock of a suitcase has 4 wheels each labelled with ten digits i.e. from 0 to 9.The lock opens with a sequence of four digits with no repeats.What is the probability of a person getting the right sequence to open the suitcase.
\\
\solution
		%\input{ncert/11/16/4/10/defs.tex}
		%
\item 
Two cards are drawn at random and without replacement from a pack of 52 playing cards. Find the probability that both the cards are black.
\\
\solution
		%\input{ncert/12/13/2/2/defs.tex}
		\item A box of oranges is inspected by examining three randomly selected oranges drawn without replacement. If all the three oranges are good, the box is approved for sale, otherwise, it is rejected. Find the probability that a box containing 15 oranges out of which 12 are good and 3 are bad ones will be approved for sale.
		\label{ncert/12/13/2/3/defs.tex}
		\item Two balls are drawn at random with replacement from a box containing 10 black and 8 red balls. Find the probability that
		\label{ncert/12/13/2/12}
\begin{enumerate}
\item both balls are red.
\item first ball is black and second is red.
\item one of them is black and other is red.
\end{enumerate}

\item In a hostel, 60\% of the students read Hindi newspaper, 40\% read English newspaper and 20\% read both Hindi and English newspapers. A student is selected at random.
		\label{ncert/12/13/2/15}
\begin{enumerate}
\item Find the probability that she reads neither Hindi nor English newspapers.
\item If she reads Hindi newspaper, find the probability that she reads English newspaper.
\item If she reads English newspaper, find the probability that she reads Hindi newspaper.\\
\end{enumerate}
\item The probability of obtaining an even prime number on each die, when a pair of dice is rolled is 
\begin{enumerate}
    \item $0$ 
    
    \item $\frac{1}{3}$ 
    
    \item $\frac{1}{12}$ 
    
    \item $\frac{1}{36}$ 
\end{enumerate}
\solution
		%\input{ncert/12/13/2/17/defs.tex}
	\item A bag contains 4 red and 4 black balls, another bag contains 2 red and 6 black balls. One of the two bags is selected at random and a ball is drawn from the bag which is found to be red. Find the probability that the ball is drawn from the first bag.
\\
\solution
		%\input{ncert/12/13/3/2/main.tex}
  \item
  Cards with numbers 2 to 101 are placed in a box. A card is selected at random.Find the probability that the card has
\begin{enumerate}[label=(\roman*)]
	\item an even number 
	\item a square number
\end{enumerate}
\solution
%\input{exemplar/10/13/3/32/main.tex}
\item
The king, queen and jack of clubs are removed from a deck of 52 playing cards and then well shuffled. Now one card is drawn at random from the remaining cards.  Determine the probability that the card is
\begin{enumerate}[label=(\roman*)]
\item a club
\item 10 of hearts
\end{enumerate}
\solution
%\input{exemplar/10/13/3/29/main.tex}
\item A team of medical students doing their internship have to assist during surgeries
at a city hospital. The probabilities of surgeries rated as very complex, complex,
routine, simple or very simple are respectively, 0.15, 0.20, 0.31, 0.26, .08. Find
the probabilities that a particular surgery will be rated
\begin{enumerate}
	\item complex or very complex;
	\item neither very complex nor very simple;
	\item routine or complex
	\item routine or simple
\end{enumerate}
\solution
%\input{exemplar/11/16/3/8(1)/main.tex}
\item A card is selected from a pack of 52 cards.
\begin{enumerate}[label=(\alph*)]
    \item How many points are there in the sample space?
    \item Calculate the probability that the card is an ace of spades.
    \item Calculate the probability that the card is (i) an ace and (ii) black card.
\end{enumerate}
\solution
%\input{exemplar/11/16/3/4/main2.tex}
\item The probability that a non leap year selected at random will contain 53 sundays.
\\
\solution
%\input{exemplar/10/13/1/19/main.tex}
\item One of the four persons John, Rita, Aslam or Gurpreet will be promoted next
month. Consequently the sample space consists of four elementary outcomes
S = {John promoted, Rita promoted, Aslam promoted, Gurpreet promoted}
You are told that the chances of John’s promotion is same as that of Gurpreet,
Rita’s chances of promotion are twice as likely as Johns. Aslam’s chances are
four times that of John.
\begin{enumerate}
	\item Determine
	\begin{enumerate}
		\item P (John promoted)
		\item P (Rita promoted)
		\item P (Aslam promoted)
		\item P (Gurpreet promoted)
	\end{enumerate}
	\item If A = {John promoted or Gurpreet promoted}, find P (A).
\end{enumerate}
\solution
%\input{exemplar/11/16/3/10/main.tex}
\item A card is drawn from a deck of 52 cards. Find the probability of getting a king or a heart or a red card.\\
\solution
%\input{exemplar/11/16/3/15/main.tex}
\item The probability that a student will pass his examination is 0.73, the probability of
the student getting a compartment is 0.13, and the probability that the student will
either pass or get compartment is 0.96. State True or False.\\
\solution
%\input{exemplar/11/16/3/31/main.tex}
\item A card is selected from a pack of 52 cards\\
\begin{enumerate}[label=(\alph*)]
\item How many points are there in the sample space?
\item Calculate the probability that the cards is an ace of spades.
\item Calculate the probability that the card is (i) an ace (ii)black card.\\
\end{enumerate}
%\input{ncert/11/16/3/4_1/Prob_4.tex}
\item In a non-leap year, the probability of having 53 tuesdays or 53 wednesdays is\\
\solution
%\input{exemplar/11/16/3/18/main.tex}
\item There are 1000 sealed envelopes in a box, 10 of them contain a cash prize of
Rs 100 each, 100 of them contain a cash prize of Rs 50 each and 200 of them
contain a cash prize of Rs 10 each and rest do not contain any cash prize. If they
are well shuffled and an envelope is picked up out, what is the probability that it
contains no cash prize?\\
\solution
%\input{exemplar/10/13/3/34/main.tex}
\item 
A die is thrown and a card is selected at random from a deck of 52 playing cards. The probability of getting an even number on the die and a spade card.\\
\solution
%\input{exemplar/12/13/3/78/main.tex}
\item
If 4-digit numbers greater than 5,000 are randomly formed from the digits 0, 1, 3, 5, and 7, what is the probability of forming a number divisible by 5 when:
\begin{enumerate}
    \item The digits are repeated?
    \item The repetition of digits is not allowed?
\end{enumerate}
\solution
%\input{ncert/11/16/4/9/main.tex}
\item Consider the probability space $\brak{\Omega, \mathcal{G}, P}$ where $\Omega = [0,2]$ and $\mathcal{G} = \cbrak{\phi, \Omega, [0,1], (1,2]}$. Let $X$ and $Y$ be two functions on $\Omega$ defined as
\begin{align*}
    X(\omega) = 
    \begin{cases}
        1 & \text{if }\omega \in [0, 1]\\
        2 & \text{if }\omega \in (1, 2]
    \end{cases}
\end{align*}
and
\begin{align*}
    Y(\omega) = 
    \begin{cases}
        2 & \text{if }\omega \in [0, 1.5]\\
        3 & \text{if }\omega \in (1.5, 2].
    \end{cases}
\end{align*}
Then which one of the following statements is true?
\begin{enumerate}
    \item [(A)] $X$ is a random variable with respect to $\mathcal{G}$, but $Y$ is not a random variable with respect to $\mathcal{G}$.
    \item [(B)] $Y$ is a random variable with respect to $\mathcal{G}$, but $X$ is not a random variable with respect to $\mathcal{G}$.
    \item [(C)] Neither $X$ nor $Y$ is a random variable with respect to $\mathcal{G}$.
    \item [(D)] Both $X$ and $Y$ are random variables with respect to $\mathcal{G}$.
\end{enumerate} \hfill (GATE ST 2023)\\
\solution
%\input{gate/ST/2023/14/main.tex}
	\item  A die is loaded in such a way that each odd number is twice as likely to occur as
each even number. Find $P(G)$, where $G$ is the event that a number greater than
3 occurs on a single roll of the die.
\\
\solution
		%\input{exemplar/11/16/3/5/main.tex}
	\item All the jacks, queens and kings are removed from a deck of 52 playing cards. The remaining cards are well shuffled and then one card is drawn at random. Giving ace a value 1 similar value for other cards, find the probability that the card has a value 
		\begin{enumerate}
			\item 7
			\item greater than 7
			\item less than 7
		\end{enumerate}
		%\input{exemplar/10/13/3/30/main.tex}
  \item A Lot consists of 48 mobile phones of which 42 are good, 3 have only minor defects and 3 have major defects.Varnika will buy a phone if it is good but the trader will only buy a mobile if it has no major defects. One phone is selected at random from the lot. What is the probability that it is
\begin{enumerate}
	\item acceptable to Varnika?
            \item acceptable to the trader?
\end{enumerate}
\solution
	%\input{exemplar/10/13/3/40/main.tex}
 \item A student says that if you throw a die, it will show up 1 or not 1. Therefore, the probability of getting 1 and the probability of getting 'not 1' each is equal to $\frac{1}{2}$. Is this correct? Give reasons.\\
 \solution
        %\input{exemplar/10/13/2/9/main.tex}
   \item Four candidates A, B, C, D have ap-
plied for the assignment to coach a school cricket
team. If A is twice as likely to be selected as B, and
B and C are given about the same chance of being
selected, while C is twice as likely to be selected
as D, what are the probabilities that
\begin{enumerate}
\item C will be selected?
\item A will not be selected?
\end{enumerate}
	%\input{exemplar/11/16/3/9/main.tex}
 \item A bag contain 24 balls of which $x$ balls are red, $2x$ are white and $3x$ are blue. A ball is selected at random, What is the probability that it is
\begin{enumerate}[label=\alph*)]
\item not red ?
\item white ?
\end{enumerate}
%\input{exemplar/10/13/3/41/main.tex}
If the letters of the word ASSASSINATION are arranged at random. Find the Probability that
\begin{enumerate}[label=(\alph*)]
\item Four $S's$ come consecutively in the word
\item Two  $I's$ and two $N's$ come together
\item All $A's$ are not coming together
\item No two $A's$ are coming together
\end{enumerate}
%\input{exemplar/11/16/3/14/main.tex}
	\item One urn contains two black balls (labelled B1 and B2) and one white ball. A
	second urn contains one black ball and two white balls (labelled W1 and W2).
	Suppose the following experiment is performed. One of the two urns is chosen
	at random. Next a ball is randomly chosen from the urn. Then a second ball is
	chosen at random from the same urn without replacing the first ball.
	
	\begin{enumerate}
	\item What is the probability that two black balls are chosen?
	
	\item What is the probability that two balls of opposite colour are chosen?
	\end{enumerate}
	\solution
	%\input{exemplar/11/16/3/12/main1.tex}
\end{enumerate}

	\item A card is selected from a pack of 52 cards.
 \begin{enumerate}[label=(\alph*)] 
                 \item How many points are there in the sample space?
                 \item Calculate the probability that the card is an ace of spades.
                 \item Calculate the probability that the card is (i) an ace and (ii) black card.
 \end{enumerate}
\solution
		%\begin{table}[H]
	\centering
\begin{tabular}{|c|c|c|}
\hline
Random variable &Value &Definition\\ \hline
\multirow{3}{*}{X} &0 &Slips of Rs 1\\
&1 &Slips of Rs 5\\
&2 &Slips of Rs 13\\ \hline
\multirow{2}{*}{Y} &0 &Box A\\
&1 &Box B\\\hline
\end{tabular}
\caption{}
\label{tab:Distribution}
\end{table}
See \tabref{tab:Distribution}.
\begin{align}
p_{Y}\brak{k}= \begin{cases} 
      \frac{1}{3} & {k=0} \\
      \frac{2}{3 }& {k=1} 
   \end{cases}
   \\
p_{Y|X}\brak{0|0} = \frac{19}{25}\, 
p_{Y|X}\brak{0|1} = \frac{6}{25}\,
p_{Y|X}\brak{1|0} = \frac{45}{50}\,
p_{Y|X}\brak{1|2} = \frac{5}{50}
\end{align}
The desired probability is the probability that a slip drawn at random is marked other than Rs 1,
\begin{align}
&=1-p_X\brak{0}\\
&= p_X(1) + p_X(2)
\end{align}
Using Bayes theorem,
\begin{align}
&= p_Y\brak{0} \times \pr{Y=0 | X=1} + p_Y\brak{1} \times \pr{Y=1|X=2}\\
&=\frac{1}{3} \times \frac{6}{25} + \frac{2}{3} \times \frac{5}{50}\\
&=\frac{11}{75}
\end{align}

\newpage

%\tableofcontents

\bigskip

\renewcommand{\thefigure}{\theenumi}
\renewcommand{\thetable}{\theenumi}
%\renewcommand{\theequation}{\theenumi}

%\begin{abstract}
%%\boldmath
%In this letter, an algorithm for evaluating the exact analytical bit error rate  (BER)  for the piecewise linear (PL) combiner for  multiple relays is presented. Previous results were available only for upto three relays. The algorithm is unique in the sense that  the actual mathematical expressions, that are prohibitively large, need not be explicitly obtained. The diversity gain due to multiple relays is shown through plots of the analytical BER, well supported by simulations. 
%
%\end{abstract}
% IEEEtran.cls defaults to using nonbold math in the Abstract.
% This preserves the distinction between vectors and scalars. However,
% if the journal you are submitting to favors bold math in the abstract,
% then you can use LaTeX's standard command \boldmath at the very start
% of the abstract to achieve this. Many IEEE journals frown on math
% in the abstract anyway.

% Note that keywords are not normally used for peerreview papers.
%\begin{IEEEkeywords}
%Cooperative diversity, decode and forward, piecewise linear
%\end{IEEEkeywords}



% For peer review papers, you can put extra information on the cover
% page as needed:
% \ifCLASSOPTIONpeerreview
% \begin{center} \bfseries EDICS Category: 3-BBND \end{center}
% \fi
%
% For peerreview papers, this IEEEtran command inserts a page break and
% creates the second title. It will be ignored for other modes.
%\IEEEpeerreviewmaketitle




\item Four cards are drawn from a well-shuffled deck of 52 cards. What is the probability of obtaining 3 diamonds and one spade.
\\
\solution
		%\begin{enumerate}[label=\thesection.\arabic*,ref=\thesection.\theenumi]
	\item One card is drawn from a well-shuffled deck of 52 cards. Find the probability of getting
\begin{enumerate}
\item A king of red colour 
\item A face card 
\item A red face card
\item The jack of hearts
\item A spade
\item The queen of diamonds

\end{enumerate}
\solution
		%\input{ncert/10/15/1/14/main.tex}
	\item Five cards—the ten, jack, queen, king and ace of diamonds, are well-shuffled with their face downwards. One card is then picked up at random.
\begin{enumerate}
\item
What is the probability that the card is the queen? 
\item
If the queen is drawn and put aside, what is the probability that the second card picked up is (a) an ace? (b) a queen?\\
\end{enumerate}
\solution
		%\input{ncert/10/15/1/15/defs.tex}
	\item A bag contains $5$ red balls and some blue balls. If the probability of drawing a blue ball is double that if a red ball, determine the number of blue balls in the bag. 
		\\
\solution
		%\input{ncert/10/15/2/3/defs.tex}
	\item A card is selected from a pack of 52 cards.
 \begin{enumerate}[label=(\alph*)] 
                 \item How many points are there in the sample space?
                 \item Calculate the probability that the card is an ace of spades.
                 \item Calculate the probability that the card is (i) an ace and (ii) black card.
 \end{enumerate}
\solution
		%\input{ncert/11/16/3/4/main.tex}
\item Four cards are drawn from a well-shuffled deck of 52 cards. What is the probability of obtaining 3 diamonds and one spade.
\\
\solution
		%\input{ncert/11/16/4/2/defs.tex}
\item In a certain lottery 10,000 tickets are sold and ten equal prizes are awarded. What is the probability of not getting a prize if you buy (a) one ticket (b) two tickets (c) 10 tickets ?	
\\
\solution
		%\input{ncert/11/16/4/4/defs.tex}
		%
\item 
Out of 100 students, two sections of 40 and 60 are formed. If you and your friend are among the 100 students, what is the probability that
\begin{enumerate}
\item you both enter the same section?
\item you both enter the different sections?
\end{enumerate}
\solution
		%\input{ncert/11/16/4/5/defs.tex}
	\item 
The number lock of a suitcase has 4 wheels each labelled with ten digits i.e. from 0 to 9.The lock opens with a sequence of four digits with no repeats.What is the probability of a person getting the right sequence to open the suitcase.
\\
\solution
		%\input{ncert/11/16/4/10/defs.tex}
		%
\item 
Two cards are drawn at random and without replacement from a pack of 52 playing cards. Find the probability that both the cards are black.
\\
\solution
		%\input{ncert/12/13/2/2/defs.tex}
		\item A box of oranges is inspected by examining three randomly selected oranges drawn without replacement. If all the three oranges are good, the box is approved for sale, otherwise, it is rejected. Find the probability that a box containing 15 oranges out of which 12 are good and 3 are bad ones will be approved for sale.
		\label{ncert/12/13/2/3/defs.tex}
		\item Two balls are drawn at random with replacement from a box containing 10 black and 8 red balls. Find the probability that
		\label{ncert/12/13/2/12}
\begin{enumerate}
\item both balls are red.
\item first ball is black and second is red.
\item one of them is black and other is red.
\end{enumerate}

\item In a hostel, 60\% of the students read Hindi newspaper, 40\% read English newspaper and 20\% read both Hindi and English newspapers. A student is selected at random.
		\label{ncert/12/13/2/15}
\begin{enumerate}
\item Find the probability that she reads neither Hindi nor English newspapers.
\item If she reads Hindi newspaper, find the probability that she reads English newspaper.
\item If she reads English newspaper, find the probability that she reads Hindi newspaper.\\
\end{enumerate}
\item The probability of obtaining an even prime number on each die, when a pair of dice is rolled is 
\begin{enumerate}
    \item $0$ 
    
    \item $\frac{1}{3}$ 
    
    \item $\frac{1}{12}$ 
    
    \item $\frac{1}{36}$ 
\end{enumerate}
\solution
		%\input{ncert/12/13/2/17/defs.tex}
	\item A bag contains 4 red and 4 black balls, another bag contains 2 red and 6 black balls. One of the two bags is selected at random and a ball is drawn from the bag which is found to be red. Find the probability that the ball is drawn from the first bag.
\\
\solution
		%\input{ncert/12/13/3/2/main.tex}
  \item
  Cards with numbers 2 to 101 are placed in a box. A card is selected at random.Find the probability that the card has
\begin{enumerate}[label=(\roman*)]
	\item an even number 
	\item a square number
\end{enumerate}
\solution
%\input{exemplar/10/13/3/32/main.tex}
\item
The king, queen and jack of clubs are removed from a deck of 52 playing cards and then well shuffled. Now one card is drawn at random from the remaining cards.  Determine the probability that the card is
\begin{enumerate}[label=(\roman*)]
\item a club
\item 10 of hearts
\end{enumerate}
\solution
%\input{exemplar/10/13/3/29/main.tex}
\item A team of medical students doing their internship have to assist during surgeries
at a city hospital. The probabilities of surgeries rated as very complex, complex,
routine, simple or very simple are respectively, 0.15, 0.20, 0.31, 0.26, .08. Find
the probabilities that a particular surgery will be rated
\begin{enumerate}
	\item complex or very complex;
	\item neither very complex nor very simple;
	\item routine or complex
	\item routine or simple
\end{enumerate}
\solution
%\input{exemplar/11/16/3/8(1)/main.tex}
\item A card is selected from a pack of 52 cards.
\begin{enumerate}[label=(\alph*)]
    \item How many points are there in the sample space?
    \item Calculate the probability that the card is an ace of spades.
    \item Calculate the probability that the card is (i) an ace and (ii) black card.
\end{enumerate}
\solution
%\input{exemplar/11/16/3/4/main2.tex}
\item The probability that a non leap year selected at random will contain 53 sundays.
\\
\solution
%\input{exemplar/10/13/1/19/main.tex}
\item One of the four persons John, Rita, Aslam or Gurpreet will be promoted next
month. Consequently the sample space consists of four elementary outcomes
S = {John promoted, Rita promoted, Aslam promoted, Gurpreet promoted}
You are told that the chances of John’s promotion is same as that of Gurpreet,
Rita’s chances of promotion are twice as likely as Johns. Aslam’s chances are
four times that of John.
\begin{enumerate}
	\item Determine
	\begin{enumerate}
		\item P (John promoted)
		\item P (Rita promoted)
		\item P (Aslam promoted)
		\item P (Gurpreet promoted)
	\end{enumerate}
	\item If A = {John promoted or Gurpreet promoted}, find P (A).
\end{enumerate}
\solution
%\input{exemplar/11/16/3/10/main.tex}
\item A card is drawn from a deck of 52 cards. Find the probability of getting a king or a heart or a red card.\\
\solution
%\input{exemplar/11/16/3/15/main.tex}
\item The probability that a student will pass his examination is 0.73, the probability of
the student getting a compartment is 0.13, and the probability that the student will
either pass or get compartment is 0.96. State True or False.\\
\solution
%\input{exemplar/11/16/3/31/main.tex}
\item A card is selected from a pack of 52 cards\\
\begin{enumerate}[label=(\alph*)]
\item How many points are there in the sample space?
\item Calculate the probability that the cards is an ace of spades.
\item Calculate the probability that the card is (i) an ace (ii)black card.\\
\end{enumerate}
%\input{ncert/11/16/3/4_1/Prob_4.tex}
\item In a non-leap year, the probability of having 53 tuesdays or 53 wednesdays is\\
\solution
%\input{exemplar/11/16/3/18/main.tex}
\item There are 1000 sealed envelopes in a box, 10 of them contain a cash prize of
Rs 100 each, 100 of them contain a cash prize of Rs 50 each and 200 of them
contain a cash prize of Rs 10 each and rest do not contain any cash prize. If they
are well shuffled and an envelope is picked up out, what is the probability that it
contains no cash prize?\\
\solution
%\input{exemplar/10/13/3/34/main.tex}
\item 
A die is thrown and a card is selected at random from a deck of 52 playing cards. The probability of getting an even number on the die and a spade card.\\
\solution
%\input{exemplar/12/13/3/78/main.tex}
\item
If 4-digit numbers greater than 5,000 are randomly formed from the digits 0, 1, 3, 5, and 7, what is the probability of forming a number divisible by 5 when:
\begin{enumerate}
    \item The digits are repeated?
    \item The repetition of digits is not allowed?
\end{enumerate}
\solution
%\input{ncert/11/16/4/9/main.tex}
\item Consider the probability space $\brak{\Omega, \mathcal{G}, P}$ where $\Omega = [0,2]$ and $\mathcal{G} = \cbrak{\phi, \Omega, [0,1], (1,2]}$. Let $X$ and $Y$ be two functions on $\Omega$ defined as
\begin{align*}
    X(\omega) = 
    \begin{cases}
        1 & \text{if }\omega \in [0, 1]\\
        2 & \text{if }\omega \in (1, 2]
    \end{cases}
\end{align*}
and
\begin{align*}
    Y(\omega) = 
    \begin{cases}
        2 & \text{if }\omega \in [0, 1.5]\\
        3 & \text{if }\omega \in (1.5, 2].
    \end{cases}
\end{align*}
Then which one of the following statements is true?
\begin{enumerate}
    \item [(A)] $X$ is a random variable with respect to $\mathcal{G}$, but $Y$ is not a random variable with respect to $\mathcal{G}$.
    \item [(B)] $Y$ is a random variable with respect to $\mathcal{G}$, but $X$ is not a random variable with respect to $\mathcal{G}$.
    \item [(C)] Neither $X$ nor $Y$ is a random variable with respect to $\mathcal{G}$.
    \item [(D)] Both $X$ and $Y$ are random variables with respect to $\mathcal{G}$.
\end{enumerate} \hfill (GATE ST 2023)\\
\solution
%\input{gate/ST/2023/14/main.tex}
	\item  A die is loaded in such a way that each odd number is twice as likely to occur as
each even number. Find $P(G)$, where $G$ is the event that a number greater than
3 occurs on a single roll of the die.
\\
\solution
		%\input{exemplar/11/16/3/5/main.tex}
	\item All the jacks, queens and kings are removed from a deck of 52 playing cards. The remaining cards are well shuffled and then one card is drawn at random. Giving ace a value 1 similar value for other cards, find the probability that the card has a value 
		\begin{enumerate}
			\item 7
			\item greater than 7
			\item less than 7
		\end{enumerate}
		%\input{exemplar/10/13/3/30/main.tex}
  \item A Lot consists of 48 mobile phones of which 42 are good, 3 have only minor defects and 3 have major defects.Varnika will buy a phone if it is good but the trader will only buy a mobile if it has no major defects. One phone is selected at random from the lot. What is the probability that it is
\begin{enumerate}
	\item acceptable to Varnika?
            \item acceptable to the trader?
\end{enumerate}
\solution
	%\input{exemplar/10/13/3/40/main.tex}
 \item A student says that if you throw a die, it will show up 1 or not 1. Therefore, the probability of getting 1 and the probability of getting 'not 1' each is equal to $\frac{1}{2}$. Is this correct? Give reasons.\\
 \solution
        %\input{exemplar/10/13/2/9/main.tex}
   \item Four candidates A, B, C, D have ap-
plied for the assignment to coach a school cricket
team. If A is twice as likely to be selected as B, and
B and C are given about the same chance of being
selected, while C is twice as likely to be selected
as D, what are the probabilities that
\begin{enumerate}
\item C will be selected?
\item A will not be selected?
\end{enumerate}
	%\input{exemplar/11/16/3/9/main.tex}
 \item A bag contain 24 balls of which $x$ balls are red, $2x$ are white and $3x$ are blue. A ball is selected at random, What is the probability that it is
\begin{enumerate}[label=\alph*)]
\item not red ?
\item white ?
\end{enumerate}
%\input{exemplar/10/13/3/41/main.tex}
If the letters of the word ASSASSINATION are arranged at random. Find the Probability that
\begin{enumerate}[label=(\alph*)]
\item Four $S's$ come consecutively in the word
\item Two  $I's$ and two $N's$ come together
\item All $A's$ are not coming together
\item No two $A's$ are coming together
\end{enumerate}
%\input{exemplar/11/16/3/14/main.tex}
	\item One urn contains two black balls (labelled B1 and B2) and one white ball. A
	second urn contains one black ball and two white balls (labelled W1 and W2).
	Suppose the following experiment is performed. One of the two urns is chosen
	at random. Next a ball is randomly chosen from the urn. Then a second ball is
	chosen at random from the same urn without replacing the first ball.
	
	\begin{enumerate}
	\item What is the probability that two black balls are chosen?
	
	\item What is the probability that two balls of opposite colour are chosen?
	\end{enumerate}
	\solution
	%\input{exemplar/11/16/3/12/main1.tex}
\end{enumerate}

\item In a certain lottery 10,000 tickets are sold and ten equal prizes are awarded. What is the probability of not getting a prize if you buy (a) one ticket (b) two tickets (c) 10 tickets ?	
\\
\solution
		%\begin{enumerate}[label=\thesection.\arabic*,ref=\thesection.\theenumi]
	\item One card is drawn from a well-shuffled deck of 52 cards. Find the probability of getting
\begin{enumerate}
\item A king of red colour 
\item A face card 
\item A red face card
\item The jack of hearts
\item A spade
\item The queen of diamonds

\end{enumerate}
\solution
		%\input{ncert/10/15/1/14/main.tex}
	\item Five cards—the ten, jack, queen, king and ace of diamonds, are well-shuffled with their face downwards. One card is then picked up at random.
\begin{enumerate}
\item
What is the probability that the card is the queen? 
\item
If the queen is drawn and put aside, what is the probability that the second card picked up is (a) an ace? (b) a queen?\\
\end{enumerate}
\solution
		%\input{ncert/10/15/1/15/defs.tex}
	\item A bag contains $5$ red balls and some blue balls. If the probability of drawing a blue ball is double that if a red ball, determine the number of blue balls in the bag. 
		\\
\solution
		%\input{ncert/10/15/2/3/defs.tex}
	\item A card is selected from a pack of 52 cards.
 \begin{enumerate}[label=(\alph*)] 
                 \item How many points are there in the sample space?
                 \item Calculate the probability that the card is an ace of spades.
                 \item Calculate the probability that the card is (i) an ace and (ii) black card.
 \end{enumerate}
\solution
		%\input{ncert/11/16/3/4/main.tex}
\item Four cards are drawn from a well-shuffled deck of 52 cards. What is the probability of obtaining 3 diamonds and one spade.
\\
\solution
		%\input{ncert/11/16/4/2/defs.tex}
\item In a certain lottery 10,000 tickets are sold and ten equal prizes are awarded. What is the probability of not getting a prize if you buy (a) one ticket (b) two tickets (c) 10 tickets ?	
\\
\solution
		%\input{ncert/11/16/4/4/defs.tex}
		%
\item 
Out of 100 students, two sections of 40 and 60 are formed. If you and your friend are among the 100 students, what is the probability that
\begin{enumerate}
\item you both enter the same section?
\item you both enter the different sections?
\end{enumerate}
\solution
		%\input{ncert/11/16/4/5/defs.tex}
	\item 
The number lock of a suitcase has 4 wheels each labelled with ten digits i.e. from 0 to 9.The lock opens with a sequence of four digits with no repeats.What is the probability of a person getting the right sequence to open the suitcase.
\\
\solution
		%\input{ncert/11/16/4/10/defs.tex}
		%
\item 
Two cards are drawn at random and without replacement from a pack of 52 playing cards. Find the probability that both the cards are black.
\\
\solution
		%\input{ncert/12/13/2/2/defs.tex}
		\item A box of oranges is inspected by examining three randomly selected oranges drawn without replacement. If all the three oranges are good, the box is approved for sale, otherwise, it is rejected. Find the probability that a box containing 15 oranges out of which 12 are good and 3 are bad ones will be approved for sale.
		\label{ncert/12/13/2/3/defs.tex}
		\item Two balls are drawn at random with replacement from a box containing 10 black and 8 red balls. Find the probability that
		\label{ncert/12/13/2/12}
\begin{enumerate}
\item both balls are red.
\item first ball is black and second is red.
\item one of them is black and other is red.
\end{enumerate}

\item In a hostel, 60\% of the students read Hindi newspaper, 40\% read English newspaper and 20\% read both Hindi and English newspapers. A student is selected at random.
		\label{ncert/12/13/2/15}
\begin{enumerate}
\item Find the probability that she reads neither Hindi nor English newspapers.
\item If she reads Hindi newspaper, find the probability that she reads English newspaper.
\item If she reads English newspaper, find the probability that she reads Hindi newspaper.\\
\end{enumerate}
\item The probability of obtaining an even prime number on each die, when a pair of dice is rolled is 
\begin{enumerate}
    \item $0$ 
    
    \item $\frac{1}{3}$ 
    
    \item $\frac{1}{12}$ 
    
    \item $\frac{1}{36}$ 
\end{enumerate}
\solution
		%\input{ncert/12/13/2/17/defs.tex}
	\item A bag contains 4 red and 4 black balls, another bag contains 2 red and 6 black balls. One of the two bags is selected at random and a ball is drawn from the bag which is found to be red. Find the probability that the ball is drawn from the first bag.
\\
\solution
		%\input{ncert/12/13/3/2/main.tex}
  \item
  Cards with numbers 2 to 101 are placed in a box. A card is selected at random.Find the probability that the card has
\begin{enumerate}[label=(\roman*)]
	\item an even number 
	\item a square number
\end{enumerate}
\solution
%\input{exemplar/10/13/3/32/main.tex}
\item
The king, queen and jack of clubs are removed from a deck of 52 playing cards and then well shuffled. Now one card is drawn at random from the remaining cards.  Determine the probability that the card is
\begin{enumerate}[label=(\roman*)]
\item a club
\item 10 of hearts
\end{enumerate}
\solution
%\input{exemplar/10/13/3/29/main.tex}
\item A team of medical students doing their internship have to assist during surgeries
at a city hospital. The probabilities of surgeries rated as very complex, complex,
routine, simple or very simple are respectively, 0.15, 0.20, 0.31, 0.26, .08. Find
the probabilities that a particular surgery will be rated
\begin{enumerate}
	\item complex or very complex;
	\item neither very complex nor very simple;
	\item routine or complex
	\item routine or simple
\end{enumerate}
\solution
%\input{exemplar/11/16/3/8(1)/main.tex}
\item A card is selected from a pack of 52 cards.
\begin{enumerate}[label=(\alph*)]
    \item How many points are there in the sample space?
    \item Calculate the probability that the card is an ace of spades.
    \item Calculate the probability that the card is (i) an ace and (ii) black card.
\end{enumerate}
\solution
%\input{exemplar/11/16/3/4/main2.tex}
\item The probability that a non leap year selected at random will contain 53 sundays.
\\
\solution
%\input{exemplar/10/13/1/19/main.tex}
\item One of the four persons John, Rita, Aslam or Gurpreet will be promoted next
month. Consequently the sample space consists of four elementary outcomes
S = {John promoted, Rita promoted, Aslam promoted, Gurpreet promoted}
You are told that the chances of John’s promotion is same as that of Gurpreet,
Rita’s chances of promotion are twice as likely as Johns. Aslam’s chances are
four times that of John.
\begin{enumerate}
	\item Determine
	\begin{enumerate}
		\item P (John promoted)
		\item P (Rita promoted)
		\item P (Aslam promoted)
		\item P (Gurpreet promoted)
	\end{enumerate}
	\item If A = {John promoted or Gurpreet promoted}, find P (A).
\end{enumerate}
\solution
%\input{exemplar/11/16/3/10/main.tex}
\item A card is drawn from a deck of 52 cards. Find the probability of getting a king or a heart or a red card.\\
\solution
%\input{exemplar/11/16/3/15/main.tex}
\item The probability that a student will pass his examination is 0.73, the probability of
the student getting a compartment is 0.13, and the probability that the student will
either pass or get compartment is 0.96. State True or False.\\
\solution
%\input{exemplar/11/16/3/31/main.tex}
\item A card is selected from a pack of 52 cards\\
\begin{enumerate}[label=(\alph*)]
\item How many points are there in the sample space?
\item Calculate the probability that the cards is an ace of spades.
\item Calculate the probability that the card is (i) an ace (ii)black card.\\
\end{enumerate}
%\input{ncert/11/16/3/4_1/Prob_4.tex}
\item In a non-leap year, the probability of having 53 tuesdays or 53 wednesdays is\\
\solution
%\input{exemplar/11/16/3/18/main.tex}
\item There are 1000 sealed envelopes in a box, 10 of them contain a cash prize of
Rs 100 each, 100 of them contain a cash prize of Rs 50 each and 200 of them
contain a cash prize of Rs 10 each and rest do not contain any cash prize. If they
are well shuffled and an envelope is picked up out, what is the probability that it
contains no cash prize?\\
\solution
%\input{exemplar/10/13/3/34/main.tex}
\item 
A die is thrown and a card is selected at random from a deck of 52 playing cards. The probability of getting an even number on the die and a spade card.\\
\solution
%\input{exemplar/12/13/3/78/main.tex}
\item
If 4-digit numbers greater than 5,000 are randomly formed from the digits 0, 1, 3, 5, and 7, what is the probability of forming a number divisible by 5 when:
\begin{enumerate}
    \item The digits are repeated?
    \item The repetition of digits is not allowed?
\end{enumerate}
\solution
%\input{ncert/11/16/4/9/main.tex}
\item Consider the probability space $\brak{\Omega, \mathcal{G}, P}$ where $\Omega = [0,2]$ and $\mathcal{G} = \cbrak{\phi, \Omega, [0,1], (1,2]}$. Let $X$ and $Y$ be two functions on $\Omega$ defined as
\begin{align*}
    X(\omega) = 
    \begin{cases}
        1 & \text{if }\omega \in [0, 1]\\
        2 & \text{if }\omega \in (1, 2]
    \end{cases}
\end{align*}
and
\begin{align*}
    Y(\omega) = 
    \begin{cases}
        2 & \text{if }\omega \in [0, 1.5]\\
        3 & \text{if }\omega \in (1.5, 2].
    \end{cases}
\end{align*}
Then which one of the following statements is true?
\begin{enumerate}
    \item [(A)] $X$ is a random variable with respect to $\mathcal{G}$, but $Y$ is not a random variable with respect to $\mathcal{G}$.
    \item [(B)] $Y$ is a random variable with respect to $\mathcal{G}$, but $X$ is not a random variable with respect to $\mathcal{G}$.
    \item [(C)] Neither $X$ nor $Y$ is a random variable with respect to $\mathcal{G}$.
    \item [(D)] Both $X$ and $Y$ are random variables with respect to $\mathcal{G}$.
\end{enumerate} \hfill (GATE ST 2023)\\
\solution
%\input{gate/ST/2023/14/main.tex}
	\item  A die is loaded in such a way that each odd number is twice as likely to occur as
each even number. Find $P(G)$, where $G$ is the event that a number greater than
3 occurs on a single roll of the die.
\\
\solution
		%\input{exemplar/11/16/3/5/main.tex}
	\item All the jacks, queens and kings are removed from a deck of 52 playing cards. The remaining cards are well shuffled and then one card is drawn at random. Giving ace a value 1 similar value for other cards, find the probability that the card has a value 
		\begin{enumerate}
			\item 7
			\item greater than 7
			\item less than 7
		\end{enumerate}
		%\input{exemplar/10/13/3/30/main.tex}
  \item A Lot consists of 48 mobile phones of which 42 are good, 3 have only minor defects and 3 have major defects.Varnika will buy a phone if it is good but the trader will only buy a mobile if it has no major defects. One phone is selected at random from the lot. What is the probability that it is
\begin{enumerate}
	\item acceptable to Varnika?
            \item acceptable to the trader?
\end{enumerate}
\solution
	%\input{exemplar/10/13/3/40/main.tex}
 \item A student says that if you throw a die, it will show up 1 or not 1. Therefore, the probability of getting 1 and the probability of getting 'not 1' each is equal to $\frac{1}{2}$. Is this correct? Give reasons.\\
 \solution
        %\input{exemplar/10/13/2/9/main.tex}
   \item Four candidates A, B, C, D have ap-
plied for the assignment to coach a school cricket
team. If A is twice as likely to be selected as B, and
B and C are given about the same chance of being
selected, while C is twice as likely to be selected
as D, what are the probabilities that
\begin{enumerate}
\item C will be selected?
\item A will not be selected?
\end{enumerate}
	%\input{exemplar/11/16/3/9/main.tex}
 \item A bag contain 24 balls of which $x$ balls are red, $2x$ are white and $3x$ are blue. A ball is selected at random, What is the probability that it is
\begin{enumerate}[label=\alph*)]
\item not red ?
\item white ?
\end{enumerate}
%\input{exemplar/10/13/3/41/main.tex}
If the letters of the word ASSASSINATION are arranged at random. Find the Probability that
\begin{enumerate}[label=(\alph*)]
\item Four $S's$ come consecutively in the word
\item Two  $I's$ and two $N's$ come together
\item All $A's$ are not coming together
\item No two $A's$ are coming together
\end{enumerate}
%\input{exemplar/11/16/3/14/main.tex}
	\item One urn contains two black balls (labelled B1 and B2) and one white ball. A
	second urn contains one black ball and two white balls (labelled W1 and W2).
	Suppose the following experiment is performed. One of the two urns is chosen
	at random. Next a ball is randomly chosen from the urn. Then a second ball is
	chosen at random from the same urn without replacing the first ball.
	
	\begin{enumerate}
	\item What is the probability that two black balls are chosen?
	
	\item What is the probability that two balls of opposite colour are chosen?
	\end{enumerate}
	\solution
	%\input{exemplar/11/16/3/12/main1.tex}
\end{enumerate}

		%
\item 
Out of 100 students, two sections of 40 and 60 are formed. If you and your friend are among the 100 students, what is the probability that
\begin{enumerate}
\item you both enter the same section?
\item you both enter the different sections?
\end{enumerate}
\solution
		%\begin{enumerate}[label=\thesection.\arabic*,ref=\thesection.\theenumi]
	\item One card is drawn from a well-shuffled deck of 52 cards. Find the probability of getting
\begin{enumerate}
\item A king of red colour 
\item A face card 
\item A red face card
\item The jack of hearts
\item A spade
\item The queen of diamonds

\end{enumerate}
\solution
		%\input{ncert/10/15/1/14/main.tex}
	\item Five cards—the ten, jack, queen, king and ace of diamonds, are well-shuffled with their face downwards. One card is then picked up at random.
\begin{enumerate}
\item
What is the probability that the card is the queen? 
\item
If the queen is drawn and put aside, what is the probability that the second card picked up is (a) an ace? (b) a queen?\\
\end{enumerate}
\solution
		%\input{ncert/10/15/1/15/defs.tex}
	\item A bag contains $5$ red balls and some blue balls. If the probability of drawing a blue ball is double that if a red ball, determine the number of blue balls in the bag. 
		\\
\solution
		%\input{ncert/10/15/2/3/defs.tex}
	\item A card is selected from a pack of 52 cards.
 \begin{enumerate}[label=(\alph*)] 
                 \item How many points are there in the sample space?
                 \item Calculate the probability that the card is an ace of spades.
                 \item Calculate the probability that the card is (i) an ace and (ii) black card.
 \end{enumerate}
\solution
		%\input{ncert/11/16/3/4/main.tex}
\item Four cards are drawn from a well-shuffled deck of 52 cards. What is the probability of obtaining 3 diamonds and one spade.
\\
\solution
		%\input{ncert/11/16/4/2/defs.tex}
\item In a certain lottery 10,000 tickets are sold and ten equal prizes are awarded. What is the probability of not getting a prize if you buy (a) one ticket (b) two tickets (c) 10 tickets ?	
\\
\solution
		%\input{ncert/11/16/4/4/defs.tex}
		%
\item 
Out of 100 students, two sections of 40 and 60 are formed. If you and your friend are among the 100 students, what is the probability that
\begin{enumerate}
\item you both enter the same section?
\item you both enter the different sections?
\end{enumerate}
\solution
		%\input{ncert/11/16/4/5/defs.tex}
	\item 
The number lock of a suitcase has 4 wheels each labelled with ten digits i.e. from 0 to 9.The lock opens with a sequence of four digits with no repeats.What is the probability of a person getting the right sequence to open the suitcase.
\\
\solution
		%\input{ncert/11/16/4/10/defs.tex}
		%
\item 
Two cards are drawn at random and without replacement from a pack of 52 playing cards. Find the probability that both the cards are black.
\\
\solution
		%\input{ncert/12/13/2/2/defs.tex}
		\item A box of oranges is inspected by examining three randomly selected oranges drawn without replacement. If all the three oranges are good, the box is approved for sale, otherwise, it is rejected. Find the probability that a box containing 15 oranges out of which 12 are good and 3 are bad ones will be approved for sale.
		\label{ncert/12/13/2/3/defs.tex}
		\item Two balls are drawn at random with replacement from a box containing 10 black and 8 red balls. Find the probability that
		\label{ncert/12/13/2/12}
\begin{enumerate}
\item both balls are red.
\item first ball is black and second is red.
\item one of them is black and other is red.
\end{enumerate}

\item In a hostel, 60\% of the students read Hindi newspaper, 40\% read English newspaper and 20\% read both Hindi and English newspapers. A student is selected at random.
		\label{ncert/12/13/2/15}
\begin{enumerate}
\item Find the probability that she reads neither Hindi nor English newspapers.
\item If she reads Hindi newspaper, find the probability that she reads English newspaper.
\item If she reads English newspaper, find the probability that she reads Hindi newspaper.\\
\end{enumerate}
\item The probability of obtaining an even prime number on each die, when a pair of dice is rolled is 
\begin{enumerate}
    \item $0$ 
    
    \item $\frac{1}{3}$ 
    
    \item $\frac{1}{12}$ 
    
    \item $\frac{1}{36}$ 
\end{enumerate}
\solution
		%\input{ncert/12/13/2/17/defs.tex}
	\item A bag contains 4 red and 4 black balls, another bag contains 2 red and 6 black balls. One of the two bags is selected at random and a ball is drawn from the bag which is found to be red. Find the probability that the ball is drawn from the first bag.
\\
\solution
		%\input{ncert/12/13/3/2/main.tex}
  \item
  Cards with numbers 2 to 101 are placed in a box. A card is selected at random.Find the probability that the card has
\begin{enumerate}[label=(\roman*)]
	\item an even number 
	\item a square number
\end{enumerate}
\solution
%\input{exemplar/10/13/3/32/main.tex}
\item
The king, queen and jack of clubs are removed from a deck of 52 playing cards and then well shuffled. Now one card is drawn at random from the remaining cards.  Determine the probability that the card is
\begin{enumerate}[label=(\roman*)]
\item a club
\item 10 of hearts
\end{enumerate}
\solution
%\input{exemplar/10/13/3/29/main.tex}
\item A team of medical students doing their internship have to assist during surgeries
at a city hospital. The probabilities of surgeries rated as very complex, complex,
routine, simple or very simple are respectively, 0.15, 0.20, 0.31, 0.26, .08. Find
the probabilities that a particular surgery will be rated
\begin{enumerate}
	\item complex or very complex;
	\item neither very complex nor very simple;
	\item routine or complex
	\item routine or simple
\end{enumerate}
\solution
%\input{exemplar/11/16/3/8(1)/main.tex}
\item A card is selected from a pack of 52 cards.
\begin{enumerate}[label=(\alph*)]
    \item How many points are there in the sample space?
    \item Calculate the probability that the card is an ace of spades.
    \item Calculate the probability that the card is (i) an ace and (ii) black card.
\end{enumerate}
\solution
%\input{exemplar/11/16/3/4/main2.tex}
\item The probability that a non leap year selected at random will contain 53 sundays.
\\
\solution
%\input{exemplar/10/13/1/19/main.tex}
\item One of the four persons John, Rita, Aslam or Gurpreet will be promoted next
month. Consequently the sample space consists of four elementary outcomes
S = {John promoted, Rita promoted, Aslam promoted, Gurpreet promoted}
You are told that the chances of John’s promotion is same as that of Gurpreet,
Rita’s chances of promotion are twice as likely as Johns. Aslam’s chances are
four times that of John.
\begin{enumerate}
	\item Determine
	\begin{enumerate}
		\item P (John promoted)
		\item P (Rita promoted)
		\item P (Aslam promoted)
		\item P (Gurpreet promoted)
	\end{enumerate}
	\item If A = {John promoted or Gurpreet promoted}, find P (A).
\end{enumerate}
\solution
%\input{exemplar/11/16/3/10/main.tex}
\item A card is drawn from a deck of 52 cards. Find the probability of getting a king or a heart or a red card.\\
\solution
%\input{exemplar/11/16/3/15/main.tex}
\item The probability that a student will pass his examination is 0.73, the probability of
the student getting a compartment is 0.13, and the probability that the student will
either pass or get compartment is 0.96. State True or False.\\
\solution
%\input{exemplar/11/16/3/31/main.tex}
\item A card is selected from a pack of 52 cards\\
\begin{enumerate}[label=(\alph*)]
\item How many points are there in the sample space?
\item Calculate the probability that the cards is an ace of spades.
\item Calculate the probability that the card is (i) an ace (ii)black card.\\
\end{enumerate}
%\input{ncert/11/16/3/4_1/Prob_4.tex}
\item In a non-leap year, the probability of having 53 tuesdays or 53 wednesdays is\\
\solution
%\input{exemplar/11/16/3/18/main.tex}
\item There are 1000 sealed envelopes in a box, 10 of them contain a cash prize of
Rs 100 each, 100 of them contain a cash prize of Rs 50 each and 200 of them
contain a cash prize of Rs 10 each and rest do not contain any cash prize. If they
are well shuffled and an envelope is picked up out, what is the probability that it
contains no cash prize?\\
\solution
%\input{exemplar/10/13/3/34/main.tex}
\item 
A die is thrown and a card is selected at random from a deck of 52 playing cards. The probability of getting an even number on the die and a spade card.\\
\solution
%\input{exemplar/12/13/3/78/main.tex}
\item
If 4-digit numbers greater than 5,000 are randomly formed from the digits 0, 1, 3, 5, and 7, what is the probability of forming a number divisible by 5 when:
\begin{enumerate}
    \item The digits are repeated?
    \item The repetition of digits is not allowed?
\end{enumerate}
\solution
%\input{ncert/11/16/4/9/main.tex}
\item Consider the probability space $\brak{\Omega, \mathcal{G}, P}$ where $\Omega = [0,2]$ and $\mathcal{G} = \cbrak{\phi, \Omega, [0,1], (1,2]}$. Let $X$ and $Y$ be two functions on $\Omega$ defined as
\begin{align*}
    X(\omega) = 
    \begin{cases}
        1 & \text{if }\omega \in [0, 1]\\
        2 & \text{if }\omega \in (1, 2]
    \end{cases}
\end{align*}
and
\begin{align*}
    Y(\omega) = 
    \begin{cases}
        2 & \text{if }\omega \in [0, 1.5]\\
        3 & \text{if }\omega \in (1.5, 2].
    \end{cases}
\end{align*}
Then which one of the following statements is true?
\begin{enumerate}
    \item [(A)] $X$ is a random variable with respect to $\mathcal{G}$, but $Y$ is not a random variable with respect to $\mathcal{G}$.
    \item [(B)] $Y$ is a random variable with respect to $\mathcal{G}$, but $X$ is not a random variable with respect to $\mathcal{G}$.
    \item [(C)] Neither $X$ nor $Y$ is a random variable with respect to $\mathcal{G}$.
    \item [(D)] Both $X$ and $Y$ are random variables with respect to $\mathcal{G}$.
\end{enumerate} \hfill (GATE ST 2023)\\
\solution
%\input{gate/ST/2023/14/main.tex}
	\item  A die is loaded in such a way that each odd number is twice as likely to occur as
each even number. Find $P(G)$, where $G$ is the event that a number greater than
3 occurs on a single roll of the die.
\\
\solution
		%\input{exemplar/11/16/3/5/main.tex}
	\item All the jacks, queens and kings are removed from a deck of 52 playing cards. The remaining cards are well shuffled and then one card is drawn at random. Giving ace a value 1 similar value for other cards, find the probability that the card has a value 
		\begin{enumerate}
			\item 7
			\item greater than 7
			\item less than 7
		\end{enumerate}
		%\input{exemplar/10/13/3/30/main.tex}
  \item A Lot consists of 48 mobile phones of which 42 are good, 3 have only minor defects and 3 have major defects.Varnika will buy a phone if it is good but the trader will only buy a mobile if it has no major defects. One phone is selected at random from the lot. What is the probability that it is
\begin{enumerate}
	\item acceptable to Varnika?
            \item acceptable to the trader?
\end{enumerate}
\solution
	%\input{exemplar/10/13/3/40/main.tex}
 \item A student says that if you throw a die, it will show up 1 or not 1. Therefore, the probability of getting 1 and the probability of getting 'not 1' each is equal to $\frac{1}{2}$. Is this correct? Give reasons.\\
 \solution
        %\input{exemplar/10/13/2/9/main.tex}
   \item Four candidates A, B, C, D have ap-
plied for the assignment to coach a school cricket
team. If A is twice as likely to be selected as B, and
B and C are given about the same chance of being
selected, while C is twice as likely to be selected
as D, what are the probabilities that
\begin{enumerate}
\item C will be selected?
\item A will not be selected?
\end{enumerate}
	%\input{exemplar/11/16/3/9/main.tex}
 \item A bag contain 24 balls of which $x$ balls are red, $2x$ are white and $3x$ are blue. A ball is selected at random, What is the probability that it is
\begin{enumerate}[label=\alph*)]
\item not red ?
\item white ?
\end{enumerate}
%\input{exemplar/10/13/3/41/main.tex}
If the letters of the word ASSASSINATION are arranged at random. Find the Probability that
\begin{enumerate}[label=(\alph*)]
\item Four $S's$ come consecutively in the word
\item Two  $I's$ and two $N's$ come together
\item All $A's$ are not coming together
\item No two $A's$ are coming together
\end{enumerate}
%\input{exemplar/11/16/3/14/main.tex}
	\item One urn contains two black balls (labelled B1 and B2) and one white ball. A
	second urn contains one black ball and two white balls (labelled W1 and W2).
	Suppose the following experiment is performed. One of the two urns is chosen
	at random. Next a ball is randomly chosen from the urn. Then a second ball is
	chosen at random from the same urn without replacing the first ball.
	
	\begin{enumerate}
	\item What is the probability that two black balls are chosen?
	
	\item What is the probability that two balls of opposite colour are chosen?
	\end{enumerate}
	\solution
	%\input{exemplar/11/16/3/12/main1.tex}
\end{enumerate}

	\item 
The number lock of a suitcase has 4 wheels each labelled with ten digits i.e. from 0 to 9.The lock opens with a sequence of four digits with no repeats.What is the probability of a person getting the right sequence to open the suitcase.
\\
\solution
		%\begin{enumerate}[label=\thesection.\arabic*,ref=\thesection.\theenumi]
	\item One card is drawn from a well-shuffled deck of 52 cards. Find the probability of getting
\begin{enumerate}
\item A king of red colour 
\item A face card 
\item A red face card
\item The jack of hearts
\item A spade
\item The queen of diamonds

\end{enumerate}
\solution
		%\input{ncert/10/15/1/14/main.tex}
	\item Five cards—the ten, jack, queen, king and ace of diamonds, are well-shuffled with their face downwards. One card is then picked up at random.
\begin{enumerate}
\item
What is the probability that the card is the queen? 
\item
If the queen is drawn and put aside, what is the probability that the second card picked up is (a) an ace? (b) a queen?\\
\end{enumerate}
\solution
		%\input{ncert/10/15/1/15/defs.tex}
	\item A bag contains $5$ red balls and some blue balls. If the probability of drawing a blue ball is double that if a red ball, determine the number of blue balls in the bag. 
		\\
\solution
		%\input{ncert/10/15/2/3/defs.tex}
	\item A card is selected from a pack of 52 cards.
 \begin{enumerate}[label=(\alph*)] 
                 \item How many points are there in the sample space?
                 \item Calculate the probability that the card is an ace of spades.
                 \item Calculate the probability that the card is (i) an ace and (ii) black card.
 \end{enumerate}
\solution
		%\input{ncert/11/16/3/4/main.tex}
\item Four cards are drawn from a well-shuffled deck of 52 cards. What is the probability of obtaining 3 diamonds and one spade.
\\
\solution
		%\input{ncert/11/16/4/2/defs.tex}
\item In a certain lottery 10,000 tickets are sold and ten equal prizes are awarded. What is the probability of not getting a prize if you buy (a) one ticket (b) two tickets (c) 10 tickets ?	
\\
\solution
		%\input{ncert/11/16/4/4/defs.tex}
		%
\item 
Out of 100 students, two sections of 40 and 60 are formed. If you and your friend are among the 100 students, what is the probability that
\begin{enumerate}
\item you both enter the same section?
\item you both enter the different sections?
\end{enumerate}
\solution
		%\input{ncert/11/16/4/5/defs.tex}
	\item 
The number lock of a suitcase has 4 wheels each labelled with ten digits i.e. from 0 to 9.The lock opens with a sequence of four digits with no repeats.What is the probability of a person getting the right sequence to open the suitcase.
\\
\solution
		%\input{ncert/11/16/4/10/defs.tex}
		%
\item 
Two cards are drawn at random and without replacement from a pack of 52 playing cards. Find the probability that both the cards are black.
\\
\solution
		%\input{ncert/12/13/2/2/defs.tex}
		\item A box of oranges is inspected by examining three randomly selected oranges drawn without replacement. If all the three oranges are good, the box is approved for sale, otherwise, it is rejected. Find the probability that a box containing 15 oranges out of which 12 are good and 3 are bad ones will be approved for sale.
		\label{ncert/12/13/2/3/defs.tex}
		\item Two balls are drawn at random with replacement from a box containing 10 black and 8 red balls. Find the probability that
		\label{ncert/12/13/2/12}
\begin{enumerate}
\item both balls are red.
\item first ball is black and second is red.
\item one of them is black and other is red.
\end{enumerate}

\item In a hostel, 60\% of the students read Hindi newspaper, 40\% read English newspaper and 20\% read both Hindi and English newspapers. A student is selected at random.
		\label{ncert/12/13/2/15}
\begin{enumerate}
\item Find the probability that she reads neither Hindi nor English newspapers.
\item If she reads Hindi newspaper, find the probability that she reads English newspaper.
\item If she reads English newspaper, find the probability that she reads Hindi newspaper.\\
\end{enumerate}
\item The probability of obtaining an even prime number on each die, when a pair of dice is rolled is 
\begin{enumerate}
    \item $0$ 
    
    \item $\frac{1}{3}$ 
    
    \item $\frac{1}{12}$ 
    
    \item $\frac{1}{36}$ 
\end{enumerate}
\solution
		%\input{ncert/12/13/2/17/defs.tex}
	\item A bag contains 4 red and 4 black balls, another bag contains 2 red and 6 black balls. One of the two bags is selected at random and a ball is drawn from the bag which is found to be red. Find the probability that the ball is drawn from the first bag.
\\
\solution
		%\input{ncert/12/13/3/2/main.tex}
  \item
  Cards with numbers 2 to 101 are placed in a box. A card is selected at random.Find the probability that the card has
\begin{enumerate}[label=(\roman*)]
	\item an even number 
	\item a square number
\end{enumerate}
\solution
%\input{exemplar/10/13/3/32/main.tex}
\item
The king, queen and jack of clubs are removed from a deck of 52 playing cards and then well shuffled. Now one card is drawn at random from the remaining cards.  Determine the probability that the card is
\begin{enumerate}[label=(\roman*)]
\item a club
\item 10 of hearts
\end{enumerate}
\solution
%\input{exemplar/10/13/3/29/main.tex}
\item A team of medical students doing their internship have to assist during surgeries
at a city hospital. The probabilities of surgeries rated as very complex, complex,
routine, simple or very simple are respectively, 0.15, 0.20, 0.31, 0.26, .08. Find
the probabilities that a particular surgery will be rated
\begin{enumerate}
	\item complex or very complex;
	\item neither very complex nor very simple;
	\item routine or complex
	\item routine or simple
\end{enumerate}
\solution
%\input{exemplar/11/16/3/8(1)/main.tex}
\item A card is selected from a pack of 52 cards.
\begin{enumerate}[label=(\alph*)]
    \item How many points are there in the sample space?
    \item Calculate the probability that the card is an ace of spades.
    \item Calculate the probability that the card is (i) an ace and (ii) black card.
\end{enumerate}
\solution
%\input{exemplar/11/16/3/4/main2.tex}
\item The probability that a non leap year selected at random will contain 53 sundays.
\\
\solution
%\input{exemplar/10/13/1/19/main.tex}
\item One of the four persons John, Rita, Aslam or Gurpreet will be promoted next
month. Consequently the sample space consists of four elementary outcomes
S = {John promoted, Rita promoted, Aslam promoted, Gurpreet promoted}
You are told that the chances of John’s promotion is same as that of Gurpreet,
Rita’s chances of promotion are twice as likely as Johns. Aslam’s chances are
four times that of John.
\begin{enumerate}
	\item Determine
	\begin{enumerate}
		\item P (John promoted)
		\item P (Rita promoted)
		\item P (Aslam promoted)
		\item P (Gurpreet promoted)
	\end{enumerate}
	\item If A = {John promoted or Gurpreet promoted}, find P (A).
\end{enumerate}
\solution
%\input{exemplar/11/16/3/10/main.tex}
\item A card is drawn from a deck of 52 cards. Find the probability of getting a king or a heart or a red card.\\
\solution
%\input{exemplar/11/16/3/15/main.tex}
\item The probability that a student will pass his examination is 0.73, the probability of
the student getting a compartment is 0.13, and the probability that the student will
either pass or get compartment is 0.96. State True or False.\\
\solution
%\input{exemplar/11/16/3/31/main.tex}
\item A card is selected from a pack of 52 cards\\
\begin{enumerate}[label=(\alph*)]
\item How many points are there in the sample space?
\item Calculate the probability that the cards is an ace of spades.
\item Calculate the probability that the card is (i) an ace (ii)black card.\\
\end{enumerate}
%\input{ncert/11/16/3/4_1/Prob_4.tex}
\item In a non-leap year, the probability of having 53 tuesdays or 53 wednesdays is\\
\solution
%\input{exemplar/11/16/3/18/main.tex}
\item There are 1000 sealed envelopes in a box, 10 of them contain a cash prize of
Rs 100 each, 100 of them contain a cash prize of Rs 50 each and 200 of them
contain a cash prize of Rs 10 each and rest do not contain any cash prize. If they
are well shuffled and an envelope is picked up out, what is the probability that it
contains no cash prize?\\
\solution
%\input{exemplar/10/13/3/34/main.tex}
\item 
A die is thrown and a card is selected at random from a deck of 52 playing cards. The probability of getting an even number on the die and a spade card.\\
\solution
%\input{exemplar/12/13/3/78/main.tex}
\item
If 4-digit numbers greater than 5,000 are randomly formed from the digits 0, 1, 3, 5, and 7, what is the probability of forming a number divisible by 5 when:
\begin{enumerate}
    \item The digits are repeated?
    \item The repetition of digits is not allowed?
\end{enumerate}
\solution
%\input{ncert/11/16/4/9/main.tex}
\item Consider the probability space $\brak{\Omega, \mathcal{G}, P}$ where $\Omega = [0,2]$ and $\mathcal{G} = \cbrak{\phi, \Omega, [0,1], (1,2]}$. Let $X$ and $Y$ be two functions on $\Omega$ defined as
\begin{align*}
    X(\omega) = 
    \begin{cases}
        1 & \text{if }\omega \in [0, 1]\\
        2 & \text{if }\omega \in (1, 2]
    \end{cases}
\end{align*}
and
\begin{align*}
    Y(\omega) = 
    \begin{cases}
        2 & \text{if }\omega \in [0, 1.5]\\
        3 & \text{if }\omega \in (1.5, 2].
    \end{cases}
\end{align*}
Then which one of the following statements is true?
\begin{enumerate}
    \item [(A)] $X$ is a random variable with respect to $\mathcal{G}$, but $Y$ is not a random variable with respect to $\mathcal{G}$.
    \item [(B)] $Y$ is a random variable with respect to $\mathcal{G}$, but $X$ is not a random variable with respect to $\mathcal{G}$.
    \item [(C)] Neither $X$ nor $Y$ is a random variable with respect to $\mathcal{G}$.
    \item [(D)] Both $X$ and $Y$ are random variables with respect to $\mathcal{G}$.
\end{enumerate} \hfill (GATE ST 2023)\\
\solution
%\input{gate/ST/2023/14/main.tex}
	\item  A die is loaded in such a way that each odd number is twice as likely to occur as
each even number. Find $P(G)$, where $G$ is the event that a number greater than
3 occurs on a single roll of the die.
\\
\solution
		%\input{exemplar/11/16/3/5/main.tex}
	\item All the jacks, queens and kings are removed from a deck of 52 playing cards. The remaining cards are well shuffled and then one card is drawn at random. Giving ace a value 1 similar value for other cards, find the probability that the card has a value 
		\begin{enumerate}
			\item 7
			\item greater than 7
			\item less than 7
		\end{enumerate}
		%\input{exemplar/10/13/3/30/main.tex}
  \item A Lot consists of 48 mobile phones of which 42 are good, 3 have only minor defects and 3 have major defects.Varnika will buy a phone if it is good but the trader will only buy a mobile if it has no major defects. One phone is selected at random from the lot. What is the probability that it is
\begin{enumerate}
	\item acceptable to Varnika?
            \item acceptable to the trader?
\end{enumerate}
\solution
	%\input{exemplar/10/13/3/40/main.tex}
 \item A student says that if you throw a die, it will show up 1 or not 1. Therefore, the probability of getting 1 and the probability of getting 'not 1' each is equal to $\frac{1}{2}$. Is this correct? Give reasons.\\
 \solution
        %\input{exemplar/10/13/2/9/main.tex}
   \item Four candidates A, B, C, D have ap-
plied for the assignment to coach a school cricket
team. If A is twice as likely to be selected as B, and
B and C are given about the same chance of being
selected, while C is twice as likely to be selected
as D, what are the probabilities that
\begin{enumerate}
\item C will be selected?
\item A will not be selected?
\end{enumerate}
	%\input{exemplar/11/16/3/9/main.tex}
 \item A bag contain 24 balls of which $x$ balls are red, $2x$ are white and $3x$ are blue. A ball is selected at random, What is the probability that it is
\begin{enumerate}[label=\alph*)]
\item not red ?
\item white ?
\end{enumerate}
%\input{exemplar/10/13/3/41/main.tex}
If the letters of the word ASSASSINATION are arranged at random. Find the Probability that
\begin{enumerate}[label=(\alph*)]
\item Four $S's$ come consecutively in the word
\item Two  $I's$ and two $N's$ come together
\item All $A's$ are not coming together
\item No two $A's$ are coming together
\end{enumerate}
%\input{exemplar/11/16/3/14/main.tex}
	\item One urn contains two black balls (labelled B1 and B2) and one white ball. A
	second urn contains one black ball and two white balls (labelled W1 and W2).
	Suppose the following experiment is performed. One of the two urns is chosen
	at random. Next a ball is randomly chosen from the urn. Then a second ball is
	chosen at random from the same urn without replacing the first ball.
	
	\begin{enumerate}
	\item What is the probability that two black balls are chosen?
	
	\item What is the probability that two balls of opposite colour are chosen?
	\end{enumerate}
	\solution
	%\input{exemplar/11/16/3/12/main1.tex}
\end{enumerate}

		%
\item 
Two cards are drawn at random and without replacement from a pack of 52 playing cards. Find the probability that both the cards are black.
\\
\solution
		%\begin{enumerate}[label=\thesection.\arabic*,ref=\thesection.\theenumi]
	\item One card is drawn from a well-shuffled deck of 52 cards. Find the probability of getting
\begin{enumerate}
\item A king of red colour 
\item A face card 
\item A red face card
\item The jack of hearts
\item A spade
\item The queen of diamonds

\end{enumerate}
\solution
		%\input{ncert/10/15/1/14/main.tex}
	\item Five cards—the ten, jack, queen, king and ace of diamonds, are well-shuffled with their face downwards. One card is then picked up at random.
\begin{enumerate}
\item
What is the probability that the card is the queen? 
\item
If the queen is drawn and put aside, what is the probability that the second card picked up is (a) an ace? (b) a queen?\\
\end{enumerate}
\solution
		%\input{ncert/10/15/1/15/defs.tex}
	\item A bag contains $5$ red balls and some blue balls. If the probability of drawing a blue ball is double that if a red ball, determine the number of blue balls in the bag. 
		\\
\solution
		%\input{ncert/10/15/2/3/defs.tex}
	\item A card is selected from a pack of 52 cards.
 \begin{enumerate}[label=(\alph*)] 
                 \item How many points are there in the sample space?
                 \item Calculate the probability that the card is an ace of spades.
                 \item Calculate the probability that the card is (i) an ace and (ii) black card.
 \end{enumerate}
\solution
		%\input{ncert/11/16/3/4/main.tex}
\item Four cards are drawn from a well-shuffled deck of 52 cards. What is the probability of obtaining 3 diamonds and one spade.
\\
\solution
		%\input{ncert/11/16/4/2/defs.tex}
\item In a certain lottery 10,000 tickets are sold and ten equal prizes are awarded. What is the probability of not getting a prize if you buy (a) one ticket (b) two tickets (c) 10 tickets ?	
\\
\solution
		%\input{ncert/11/16/4/4/defs.tex}
		%
\item 
Out of 100 students, two sections of 40 and 60 are formed. If you and your friend are among the 100 students, what is the probability that
\begin{enumerate}
\item you both enter the same section?
\item you both enter the different sections?
\end{enumerate}
\solution
		%\input{ncert/11/16/4/5/defs.tex}
	\item 
The number lock of a suitcase has 4 wheels each labelled with ten digits i.e. from 0 to 9.The lock opens with a sequence of four digits with no repeats.What is the probability of a person getting the right sequence to open the suitcase.
\\
\solution
		%\input{ncert/11/16/4/10/defs.tex}
		%
\item 
Two cards are drawn at random and without replacement from a pack of 52 playing cards. Find the probability that both the cards are black.
\\
\solution
		%\input{ncert/12/13/2/2/defs.tex}
		\item A box of oranges is inspected by examining three randomly selected oranges drawn without replacement. If all the three oranges are good, the box is approved for sale, otherwise, it is rejected. Find the probability that a box containing 15 oranges out of which 12 are good and 3 are bad ones will be approved for sale.
		\label{ncert/12/13/2/3/defs.tex}
		\item Two balls are drawn at random with replacement from a box containing 10 black and 8 red balls. Find the probability that
		\label{ncert/12/13/2/12}
\begin{enumerate}
\item both balls are red.
\item first ball is black and second is red.
\item one of them is black and other is red.
\end{enumerate}

\item In a hostel, 60\% of the students read Hindi newspaper, 40\% read English newspaper and 20\% read both Hindi and English newspapers. A student is selected at random.
		\label{ncert/12/13/2/15}
\begin{enumerate}
\item Find the probability that she reads neither Hindi nor English newspapers.
\item If she reads Hindi newspaper, find the probability that she reads English newspaper.
\item If she reads English newspaper, find the probability that she reads Hindi newspaper.\\
\end{enumerate}
\item The probability of obtaining an even prime number on each die, when a pair of dice is rolled is 
\begin{enumerate}
    \item $0$ 
    
    \item $\frac{1}{3}$ 
    
    \item $\frac{1}{12}$ 
    
    \item $\frac{1}{36}$ 
\end{enumerate}
\solution
		%\input{ncert/12/13/2/17/defs.tex}
	\item A bag contains 4 red and 4 black balls, another bag contains 2 red and 6 black balls. One of the two bags is selected at random and a ball is drawn from the bag which is found to be red. Find the probability that the ball is drawn from the first bag.
\\
\solution
		%\input{ncert/12/13/3/2/main.tex}
  \item
  Cards with numbers 2 to 101 are placed in a box. A card is selected at random.Find the probability that the card has
\begin{enumerate}[label=(\roman*)]
	\item an even number 
	\item a square number
\end{enumerate}
\solution
%\input{exemplar/10/13/3/32/main.tex}
\item
The king, queen and jack of clubs are removed from a deck of 52 playing cards and then well shuffled. Now one card is drawn at random from the remaining cards.  Determine the probability that the card is
\begin{enumerate}[label=(\roman*)]
\item a club
\item 10 of hearts
\end{enumerate}
\solution
%\input{exemplar/10/13/3/29/main.tex}
\item A team of medical students doing their internship have to assist during surgeries
at a city hospital. The probabilities of surgeries rated as very complex, complex,
routine, simple or very simple are respectively, 0.15, 0.20, 0.31, 0.26, .08. Find
the probabilities that a particular surgery will be rated
\begin{enumerate}
	\item complex or very complex;
	\item neither very complex nor very simple;
	\item routine or complex
	\item routine or simple
\end{enumerate}
\solution
%\input{exemplar/11/16/3/8(1)/main.tex}
\item A card is selected from a pack of 52 cards.
\begin{enumerate}[label=(\alph*)]
    \item How many points are there in the sample space?
    \item Calculate the probability that the card is an ace of spades.
    \item Calculate the probability that the card is (i) an ace and (ii) black card.
\end{enumerate}
\solution
%\input{exemplar/11/16/3/4/main2.tex}
\item The probability that a non leap year selected at random will contain 53 sundays.
\\
\solution
%\input{exemplar/10/13/1/19/main.tex}
\item One of the four persons John, Rita, Aslam or Gurpreet will be promoted next
month. Consequently the sample space consists of four elementary outcomes
S = {John promoted, Rita promoted, Aslam promoted, Gurpreet promoted}
You are told that the chances of John’s promotion is same as that of Gurpreet,
Rita’s chances of promotion are twice as likely as Johns. Aslam’s chances are
four times that of John.
\begin{enumerate}
	\item Determine
	\begin{enumerate}
		\item P (John promoted)
		\item P (Rita promoted)
		\item P (Aslam promoted)
		\item P (Gurpreet promoted)
	\end{enumerate}
	\item If A = {John promoted or Gurpreet promoted}, find P (A).
\end{enumerate}
\solution
%\input{exemplar/11/16/3/10/main.tex}
\item A card is drawn from a deck of 52 cards. Find the probability of getting a king or a heart or a red card.\\
\solution
%\input{exemplar/11/16/3/15/main.tex}
\item The probability that a student will pass his examination is 0.73, the probability of
the student getting a compartment is 0.13, and the probability that the student will
either pass or get compartment is 0.96. State True or False.\\
\solution
%\input{exemplar/11/16/3/31/main.tex}
\item A card is selected from a pack of 52 cards\\
\begin{enumerate}[label=(\alph*)]
\item How many points are there in the sample space?
\item Calculate the probability that the cards is an ace of spades.
\item Calculate the probability that the card is (i) an ace (ii)black card.\\
\end{enumerate}
%\input{ncert/11/16/3/4_1/Prob_4.tex}
\item In a non-leap year, the probability of having 53 tuesdays or 53 wednesdays is\\
\solution
%\input{exemplar/11/16/3/18/main.tex}
\item There are 1000 sealed envelopes in a box, 10 of them contain a cash prize of
Rs 100 each, 100 of them contain a cash prize of Rs 50 each and 200 of them
contain a cash prize of Rs 10 each and rest do not contain any cash prize. If they
are well shuffled and an envelope is picked up out, what is the probability that it
contains no cash prize?\\
\solution
%\input{exemplar/10/13/3/34/main.tex}
\item 
A die is thrown and a card is selected at random from a deck of 52 playing cards. The probability of getting an even number on the die and a spade card.\\
\solution
%\input{exemplar/12/13/3/78/main.tex}
\item
If 4-digit numbers greater than 5,000 are randomly formed from the digits 0, 1, 3, 5, and 7, what is the probability of forming a number divisible by 5 when:
\begin{enumerate}
    \item The digits are repeated?
    \item The repetition of digits is not allowed?
\end{enumerate}
\solution
%\input{ncert/11/16/4/9/main.tex}
\item Consider the probability space $\brak{\Omega, \mathcal{G}, P}$ where $\Omega = [0,2]$ and $\mathcal{G} = \cbrak{\phi, \Omega, [0,1], (1,2]}$. Let $X$ and $Y$ be two functions on $\Omega$ defined as
\begin{align*}
    X(\omega) = 
    \begin{cases}
        1 & \text{if }\omega \in [0, 1]\\
        2 & \text{if }\omega \in (1, 2]
    \end{cases}
\end{align*}
and
\begin{align*}
    Y(\omega) = 
    \begin{cases}
        2 & \text{if }\omega \in [0, 1.5]\\
        3 & \text{if }\omega \in (1.5, 2].
    \end{cases}
\end{align*}
Then which one of the following statements is true?
\begin{enumerate}
    \item [(A)] $X$ is a random variable with respect to $\mathcal{G}$, but $Y$ is not a random variable with respect to $\mathcal{G}$.
    \item [(B)] $Y$ is a random variable with respect to $\mathcal{G}$, but $X$ is not a random variable with respect to $\mathcal{G}$.
    \item [(C)] Neither $X$ nor $Y$ is a random variable with respect to $\mathcal{G}$.
    \item [(D)] Both $X$ and $Y$ are random variables with respect to $\mathcal{G}$.
\end{enumerate} \hfill (GATE ST 2023)\\
\solution
%\input{gate/ST/2023/14/main.tex}
	\item  A die is loaded in such a way that each odd number is twice as likely to occur as
each even number. Find $P(G)$, where $G$ is the event that a number greater than
3 occurs on a single roll of the die.
\\
\solution
		%\input{exemplar/11/16/3/5/main.tex}
	\item All the jacks, queens and kings are removed from a deck of 52 playing cards. The remaining cards are well shuffled and then one card is drawn at random. Giving ace a value 1 similar value for other cards, find the probability that the card has a value 
		\begin{enumerate}
			\item 7
			\item greater than 7
			\item less than 7
		\end{enumerate}
		%\input{exemplar/10/13/3/30/main.tex}
  \item A Lot consists of 48 mobile phones of which 42 are good, 3 have only minor defects and 3 have major defects.Varnika will buy a phone if it is good but the trader will only buy a mobile if it has no major defects. One phone is selected at random from the lot. What is the probability that it is
\begin{enumerate}
	\item acceptable to Varnika?
            \item acceptable to the trader?
\end{enumerate}
\solution
	%\input{exemplar/10/13/3/40/main.tex}
 \item A student says that if you throw a die, it will show up 1 or not 1. Therefore, the probability of getting 1 and the probability of getting 'not 1' each is equal to $\frac{1}{2}$. Is this correct? Give reasons.\\
 \solution
        %\input{exemplar/10/13/2/9/main.tex}
   \item Four candidates A, B, C, D have ap-
plied for the assignment to coach a school cricket
team. If A is twice as likely to be selected as B, and
B and C are given about the same chance of being
selected, while C is twice as likely to be selected
as D, what are the probabilities that
\begin{enumerate}
\item C will be selected?
\item A will not be selected?
\end{enumerate}
	%\input{exemplar/11/16/3/9/main.tex}
 \item A bag contain 24 balls of which $x$ balls are red, $2x$ are white and $3x$ are blue. A ball is selected at random, What is the probability that it is
\begin{enumerate}[label=\alph*)]
\item not red ?
\item white ?
\end{enumerate}
%\input{exemplar/10/13/3/41/main.tex}
If the letters of the word ASSASSINATION are arranged at random. Find the Probability that
\begin{enumerate}[label=(\alph*)]
\item Four $S's$ come consecutively in the word
\item Two  $I's$ and two $N's$ come together
\item All $A's$ are not coming together
\item No two $A's$ are coming together
\end{enumerate}
%\input{exemplar/11/16/3/14/main.tex}
	\item One urn contains two black balls (labelled B1 and B2) and one white ball. A
	second urn contains one black ball and two white balls (labelled W1 and W2).
	Suppose the following experiment is performed. One of the two urns is chosen
	at random. Next a ball is randomly chosen from the urn. Then a second ball is
	chosen at random from the same urn without replacing the first ball.
	
	\begin{enumerate}
	\item What is the probability that two black balls are chosen?
	
	\item What is the probability that two balls of opposite colour are chosen?
	\end{enumerate}
	\solution
	%\input{exemplar/11/16/3/12/main1.tex}
\end{enumerate}

		\item A box of oranges is inspected by examining three randomly selected oranges drawn without replacement. If all the three oranges are good, the box is approved for sale, otherwise, it is rejected. Find the probability that a box containing 15 oranges out of which 12 are good and 3 are bad ones will be approved for sale.
		\label{ncert/12/13/2/3/defs.tex}
		\item Two balls are drawn at random with replacement from a box containing 10 black and 8 red balls. Find the probability that
		\label{ncert/12/13/2/12}
\begin{enumerate}
\item both balls are red.
\item first ball is black and second is red.
\item one of them is black and other is red.
\end{enumerate}

\item In a hostel, 60\% of the students read Hindi newspaper, 40\% read English newspaper and 20\% read both Hindi and English newspapers. A student is selected at random.
		\label{ncert/12/13/2/15}
\begin{enumerate}
\item Find the probability that she reads neither Hindi nor English newspapers.
\item If she reads Hindi newspaper, find the probability that she reads English newspaper.
\item If she reads English newspaper, find the probability that she reads Hindi newspaper.\\
\end{enumerate}
\item The probability of obtaining an even prime number on each die, when a pair of dice is rolled is 
\begin{enumerate}
    \item $0$ 
    
    \item $\frac{1}{3}$ 
    
    \item $\frac{1}{12}$ 
    
    \item $\frac{1}{36}$ 
\end{enumerate}
\solution
		%\begin{enumerate}[label=\thesection.\arabic*,ref=\thesection.\theenumi]
	\item One card is drawn from a well-shuffled deck of 52 cards. Find the probability of getting
\begin{enumerate}
\item A king of red colour 
\item A face card 
\item A red face card
\item The jack of hearts
\item A spade
\item The queen of diamonds

\end{enumerate}
\solution
		%\input{ncert/10/15/1/14/main.tex}
	\item Five cards—the ten, jack, queen, king and ace of diamonds, are well-shuffled with their face downwards. One card is then picked up at random.
\begin{enumerate}
\item
What is the probability that the card is the queen? 
\item
If the queen is drawn and put aside, what is the probability that the second card picked up is (a) an ace? (b) a queen?\\
\end{enumerate}
\solution
		%\input{ncert/10/15/1/15/defs.tex}
	\item A bag contains $5$ red balls and some blue balls. If the probability of drawing a blue ball is double that if a red ball, determine the number of blue balls in the bag. 
		\\
\solution
		%\input{ncert/10/15/2/3/defs.tex}
	\item A card is selected from a pack of 52 cards.
 \begin{enumerate}[label=(\alph*)] 
                 \item How many points are there in the sample space?
                 \item Calculate the probability that the card is an ace of spades.
                 \item Calculate the probability that the card is (i) an ace and (ii) black card.
 \end{enumerate}
\solution
		%\input{ncert/11/16/3/4/main.tex}
\item Four cards are drawn from a well-shuffled deck of 52 cards. What is the probability of obtaining 3 diamonds and one spade.
\\
\solution
		%\input{ncert/11/16/4/2/defs.tex}
\item In a certain lottery 10,000 tickets are sold and ten equal prizes are awarded. What is the probability of not getting a prize if you buy (a) one ticket (b) two tickets (c) 10 tickets ?	
\\
\solution
		%\input{ncert/11/16/4/4/defs.tex}
		%
\item 
Out of 100 students, two sections of 40 and 60 are formed. If you and your friend are among the 100 students, what is the probability that
\begin{enumerate}
\item you both enter the same section?
\item you both enter the different sections?
\end{enumerate}
\solution
		%\input{ncert/11/16/4/5/defs.tex}
	\item 
The number lock of a suitcase has 4 wheels each labelled with ten digits i.e. from 0 to 9.The lock opens with a sequence of four digits with no repeats.What is the probability of a person getting the right sequence to open the suitcase.
\\
\solution
		%\input{ncert/11/16/4/10/defs.tex}
		%
\item 
Two cards are drawn at random and without replacement from a pack of 52 playing cards. Find the probability that both the cards are black.
\\
\solution
		%\input{ncert/12/13/2/2/defs.tex}
		\item A box of oranges is inspected by examining three randomly selected oranges drawn without replacement. If all the three oranges are good, the box is approved for sale, otherwise, it is rejected. Find the probability that a box containing 15 oranges out of which 12 are good and 3 are bad ones will be approved for sale.
		\label{ncert/12/13/2/3/defs.tex}
		\item Two balls are drawn at random with replacement from a box containing 10 black and 8 red balls. Find the probability that
		\label{ncert/12/13/2/12}
\begin{enumerate}
\item both balls are red.
\item first ball is black and second is red.
\item one of them is black and other is red.
\end{enumerate}

\item In a hostel, 60\% of the students read Hindi newspaper, 40\% read English newspaper and 20\% read both Hindi and English newspapers. A student is selected at random.
		\label{ncert/12/13/2/15}
\begin{enumerate}
\item Find the probability that she reads neither Hindi nor English newspapers.
\item If she reads Hindi newspaper, find the probability that she reads English newspaper.
\item If she reads English newspaper, find the probability that she reads Hindi newspaper.\\
\end{enumerate}
\item The probability of obtaining an even prime number on each die, when a pair of dice is rolled is 
\begin{enumerate}
    \item $0$ 
    
    \item $\frac{1}{3}$ 
    
    \item $\frac{1}{12}$ 
    
    \item $\frac{1}{36}$ 
\end{enumerate}
\solution
		%\input{ncert/12/13/2/17/defs.tex}
	\item A bag contains 4 red and 4 black balls, another bag contains 2 red and 6 black balls. One of the two bags is selected at random and a ball is drawn from the bag which is found to be red. Find the probability that the ball is drawn from the first bag.
\\
\solution
		%\input{ncert/12/13/3/2/main.tex}
  \item
  Cards with numbers 2 to 101 are placed in a box. A card is selected at random.Find the probability that the card has
\begin{enumerate}[label=(\roman*)]
	\item an even number 
	\item a square number
\end{enumerate}
\solution
%\input{exemplar/10/13/3/32/main.tex}
\item
The king, queen and jack of clubs are removed from a deck of 52 playing cards and then well shuffled. Now one card is drawn at random from the remaining cards.  Determine the probability that the card is
\begin{enumerate}[label=(\roman*)]
\item a club
\item 10 of hearts
\end{enumerate}
\solution
%\input{exemplar/10/13/3/29/main.tex}
\item A team of medical students doing their internship have to assist during surgeries
at a city hospital. The probabilities of surgeries rated as very complex, complex,
routine, simple or very simple are respectively, 0.15, 0.20, 0.31, 0.26, .08. Find
the probabilities that a particular surgery will be rated
\begin{enumerate}
	\item complex or very complex;
	\item neither very complex nor very simple;
	\item routine or complex
	\item routine or simple
\end{enumerate}
\solution
%\input{exemplar/11/16/3/8(1)/main.tex}
\item A card is selected from a pack of 52 cards.
\begin{enumerate}[label=(\alph*)]
    \item How many points are there in the sample space?
    \item Calculate the probability that the card is an ace of spades.
    \item Calculate the probability that the card is (i) an ace and (ii) black card.
\end{enumerate}
\solution
%\input{exemplar/11/16/3/4/main2.tex}
\item The probability that a non leap year selected at random will contain 53 sundays.
\\
\solution
%\input{exemplar/10/13/1/19/main.tex}
\item One of the four persons John, Rita, Aslam or Gurpreet will be promoted next
month. Consequently the sample space consists of four elementary outcomes
S = {John promoted, Rita promoted, Aslam promoted, Gurpreet promoted}
You are told that the chances of John’s promotion is same as that of Gurpreet,
Rita’s chances of promotion are twice as likely as Johns. Aslam’s chances are
four times that of John.
\begin{enumerate}
	\item Determine
	\begin{enumerate}
		\item P (John promoted)
		\item P (Rita promoted)
		\item P (Aslam promoted)
		\item P (Gurpreet promoted)
	\end{enumerate}
	\item If A = {John promoted or Gurpreet promoted}, find P (A).
\end{enumerate}
\solution
%\input{exemplar/11/16/3/10/main.tex}
\item A card is drawn from a deck of 52 cards. Find the probability of getting a king or a heart or a red card.\\
\solution
%\input{exemplar/11/16/3/15/main.tex}
\item The probability that a student will pass his examination is 0.73, the probability of
the student getting a compartment is 0.13, and the probability that the student will
either pass or get compartment is 0.96. State True or False.\\
\solution
%\input{exemplar/11/16/3/31/main.tex}
\item A card is selected from a pack of 52 cards\\
\begin{enumerate}[label=(\alph*)]
\item How many points are there in the sample space?
\item Calculate the probability that the cards is an ace of spades.
\item Calculate the probability that the card is (i) an ace (ii)black card.\\
\end{enumerate}
%\input{ncert/11/16/3/4_1/Prob_4.tex}
\item In a non-leap year, the probability of having 53 tuesdays or 53 wednesdays is\\
\solution
%\input{exemplar/11/16/3/18/main.tex}
\item There are 1000 sealed envelopes in a box, 10 of them contain a cash prize of
Rs 100 each, 100 of them contain a cash prize of Rs 50 each and 200 of them
contain a cash prize of Rs 10 each and rest do not contain any cash prize. If they
are well shuffled and an envelope is picked up out, what is the probability that it
contains no cash prize?\\
\solution
%\input{exemplar/10/13/3/34/main.tex}
\item 
A die is thrown and a card is selected at random from a deck of 52 playing cards. The probability of getting an even number on the die and a spade card.\\
\solution
%\input{exemplar/12/13/3/78/main.tex}
\item
If 4-digit numbers greater than 5,000 are randomly formed from the digits 0, 1, 3, 5, and 7, what is the probability of forming a number divisible by 5 when:
\begin{enumerate}
    \item The digits are repeated?
    \item The repetition of digits is not allowed?
\end{enumerate}
\solution
%\input{ncert/11/16/4/9/main.tex}
\item Consider the probability space $\brak{\Omega, \mathcal{G}, P}$ where $\Omega = [0,2]$ and $\mathcal{G} = \cbrak{\phi, \Omega, [0,1], (1,2]}$. Let $X$ and $Y$ be two functions on $\Omega$ defined as
\begin{align*}
    X(\omega) = 
    \begin{cases}
        1 & \text{if }\omega \in [0, 1]\\
        2 & \text{if }\omega \in (1, 2]
    \end{cases}
\end{align*}
and
\begin{align*}
    Y(\omega) = 
    \begin{cases}
        2 & \text{if }\omega \in [0, 1.5]\\
        3 & \text{if }\omega \in (1.5, 2].
    \end{cases}
\end{align*}
Then which one of the following statements is true?
\begin{enumerate}
    \item [(A)] $X$ is a random variable with respect to $\mathcal{G}$, but $Y$ is not a random variable with respect to $\mathcal{G}$.
    \item [(B)] $Y$ is a random variable with respect to $\mathcal{G}$, but $X$ is not a random variable with respect to $\mathcal{G}$.
    \item [(C)] Neither $X$ nor $Y$ is a random variable with respect to $\mathcal{G}$.
    \item [(D)] Both $X$ and $Y$ are random variables with respect to $\mathcal{G}$.
\end{enumerate} \hfill (GATE ST 2023)\\
\solution
%\input{gate/ST/2023/14/main.tex}
	\item  A die is loaded in such a way that each odd number is twice as likely to occur as
each even number. Find $P(G)$, where $G$ is the event that a number greater than
3 occurs on a single roll of the die.
\\
\solution
		%\input{exemplar/11/16/3/5/main.tex}
	\item All the jacks, queens and kings are removed from a deck of 52 playing cards. The remaining cards are well shuffled and then one card is drawn at random. Giving ace a value 1 similar value for other cards, find the probability that the card has a value 
		\begin{enumerate}
			\item 7
			\item greater than 7
			\item less than 7
		\end{enumerate}
		%\input{exemplar/10/13/3/30/main.tex}
  \item A Lot consists of 48 mobile phones of which 42 are good, 3 have only minor defects and 3 have major defects.Varnika will buy a phone if it is good but the trader will only buy a mobile if it has no major defects. One phone is selected at random from the lot. What is the probability that it is
\begin{enumerate}
	\item acceptable to Varnika?
            \item acceptable to the trader?
\end{enumerate}
\solution
	%\input{exemplar/10/13/3/40/main.tex}
 \item A student says that if you throw a die, it will show up 1 or not 1. Therefore, the probability of getting 1 and the probability of getting 'not 1' each is equal to $\frac{1}{2}$. Is this correct? Give reasons.\\
 \solution
        %\input{exemplar/10/13/2/9/main.tex}
   \item Four candidates A, B, C, D have ap-
plied for the assignment to coach a school cricket
team. If A is twice as likely to be selected as B, and
B and C are given about the same chance of being
selected, while C is twice as likely to be selected
as D, what are the probabilities that
\begin{enumerate}
\item C will be selected?
\item A will not be selected?
\end{enumerate}
	%\input{exemplar/11/16/3/9/main.tex}
 \item A bag contain 24 balls of which $x$ balls are red, $2x$ are white and $3x$ are blue. A ball is selected at random, What is the probability that it is
\begin{enumerate}[label=\alph*)]
\item not red ?
\item white ?
\end{enumerate}
%\input{exemplar/10/13/3/41/main.tex}
If the letters of the word ASSASSINATION are arranged at random. Find the Probability that
\begin{enumerate}[label=(\alph*)]
\item Four $S's$ come consecutively in the word
\item Two  $I's$ and two $N's$ come together
\item All $A's$ are not coming together
\item No two $A's$ are coming together
\end{enumerate}
%\input{exemplar/11/16/3/14/main.tex}
	\item One urn contains two black balls (labelled B1 and B2) and one white ball. A
	second urn contains one black ball and two white balls (labelled W1 and W2).
	Suppose the following experiment is performed. One of the two urns is chosen
	at random. Next a ball is randomly chosen from the urn. Then a second ball is
	chosen at random from the same urn without replacing the first ball.
	
	\begin{enumerate}
	\item What is the probability that two black balls are chosen?
	
	\item What is the probability that two balls of opposite colour are chosen?
	\end{enumerate}
	\solution
	%\input{exemplar/11/16/3/12/main1.tex}
\end{enumerate}

	\item A bag contains 4 red and 4 black balls, another bag contains 2 red and 6 black balls. One of the two bags is selected at random and a ball is drawn from the bag which is found to be red. Find the probability that the ball is drawn from the first bag.
\\
\solution
		%\begin{table}[H]
	\centering
\begin{tabular}{|c|c|c|}
\hline
Random variable &Value &Definition\\ \hline
\multirow{3}{*}{X} &0 &Slips of Rs 1\\
&1 &Slips of Rs 5\\
&2 &Slips of Rs 13\\ \hline
\multirow{2}{*}{Y} &0 &Box A\\
&1 &Box B\\\hline
\end{tabular}
\caption{}
\label{tab:Distribution}
\end{table}
See \tabref{tab:Distribution}.
\begin{align}
p_{Y}\brak{k}= \begin{cases} 
      \frac{1}{3} & {k=0} \\
      \frac{2}{3 }& {k=1} 
   \end{cases}
   \\
p_{Y|X}\brak{0|0} = \frac{19}{25}\, 
p_{Y|X}\brak{0|1} = \frac{6}{25}\,
p_{Y|X}\brak{1|0} = \frac{45}{50}\,
p_{Y|X}\brak{1|2} = \frac{5}{50}
\end{align}
The desired probability is the probability that a slip drawn at random is marked other than Rs 1,
\begin{align}
&=1-p_X\brak{0}\\
&= p_X(1) + p_X(2)
\end{align}
Using Bayes theorem,
\begin{align}
&= p_Y\brak{0} \times \pr{Y=0 | X=1} + p_Y\brak{1} \times \pr{Y=1|X=2}\\
&=\frac{1}{3} \times \frac{6}{25} + \frac{2}{3} \times \frac{5}{50}\\
&=\frac{11}{75}
\end{align}

\newpage

%\tableofcontents

\bigskip

\renewcommand{\thefigure}{\theenumi}
\renewcommand{\thetable}{\theenumi}
%\renewcommand{\theequation}{\theenumi}

%\begin{abstract}
%%\boldmath
%In this letter, an algorithm for evaluating the exact analytical bit error rate  (BER)  for the piecewise linear (PL) combiner for  multiple relays is presented. Previous results were available only for upto three relays. The algorithm is unique in the sense that  the actual mathematical expressions, that are prohibitively large, need not be explicitly obtained. The diversity gain due to multiple relays is shown through plots of the analytical BER, well supported by simulations. 
%
%\end{abstract}
% IEEEtran.cls defaults to using nonbold math in the Abstract.
% This preserves the distinction between vectors and scalars. However,
% if the journal you are submitting to favors bold math in the abstract,
% then you can use LaTeX's standard command \boldmath at the very start
% of the abstract to achieve this. Many IEEE journals frown on math
% in the abstract anyway.

% Note that keywords are not normally used for peerreview papers.
%\begin{IEEEkeywords}
%Cooperative diversity, decode and forward, piecewise linear
%\end{IEEEkeywords}



% For peer review papers, you can put extra information on the cover
% page as needed:
% \ifCLASSOPTIONpeerreview
% \begin{center} \bfseries EDICS Category: 3-BBND \end{center}
% \fi
%
% For peerreview papers, this IEEEtran command inserts a page break and
% creates the second title. It will be ignored for other modes.
%\IEEEpeerreviewmaketitle




  \item
  Cards with numbers 2 to 101 are placed in a box. A card is selected at random.Find the probability that the card has
\begin{enumerate}[label=(\roman*)]
	\item an even number 
	\item a square number
\end{enumerate}
\solution
%\begin{table}[H]
	\centering
\begin{tabular}{|c|c|c|}
\hline
Random variable &Value &Definition\\ \hline
\multirow{3}{*}{X} &0 &Slips of Rs 1\\
&1 &Slips of Rs 5\\
&2 &Slips of Rs 13\\ \hline
\multirow{2}{*}{Y} &0 &Box A\\
&1 &Box B\\\hline
\end{tabular}
\caption{}
\label{tab:Distribution}
\end{table}
See \tabref{tab:Distribution}.
\begin{align}
p_{Y}\brak{k}= \begin{cases} 
      \frac{1}{3} & {k=0} \\
      \frac{2}{3 }& {k=1} 
   \end{cases}
   \\
p_{Y|X}\brak{0|0} = \frac{19}{25}\, 
p_{Y|X}\brak{0|1} = \frac{6}{25}\,
p_{Y|X}\brak{1|0} = \frac{45}{50}\,
p_{Y|X}\brak{1|2} = \frac{5}{50}
\end{align}
The desired probability is the probability that a slip drawn at random is marked other than Rs 1,
\begin{align}
&=1-p_X\brak{0}\\
&= p_X(1) + p_X(2)
\end{align}
Using Bayes theorem,
\begin{align}
&= p_Y\brak{0} \times \pr{Y=0 | X=1} + p_Y\brak{1} \times \pr{Y=1|X=2}\\
&=\frac{1}{3} \times \frac{6}{25} + \frac{2}{3} \times \frac{5}{50}\\
&=\frac{11}{75}
\end{align}

\newpage

%\tableofcontents

\bigskip

\renewcommand{\thefigure}{\theenumi}
\renewcommand{\thetable}{\theenumi}
%\renewcommand{\theequation}{\theenumi}

%\begin{abstract}
%%\boldmath
%In this letter, an algorithm for evaluating the exact analytical bit error rate  (BER)  for the piecewise linear (PL) combiner for  multiple relays is presented. Previous results were available only for upto three relays. The algorithm is unique in the sense that  the actual mathematical expressions, that are prohibitively large, need not be explicitly obtained. The diversity gain due to multiple relays is shown through plots of the analytical BER, well supported by simulations. 
%
%\end{abstract}
% IEEEtran.cls defaults to using nonbold math in the Abstract.
% This preserves the distinction between vectors and scalars. However,
% if the journal you are submitting to favors bold math in the abstract,
% then you can use LaTeX's standard command \boldmath at the very start
% of the abstract to achieve this. Many IEEE journals frown on math
% in the abstract anyway.

% Note that keywords are not normally used for peerreview papers.
%\begin{IEEEkeywords}
%Cooperative diversity, decode and forward, piecewise linear
%\end{IEEEkeywords}



% For peer review papers, you can put extra information on the cover
% page as needed:
% \ifCLASSOPTIONpeerreview
% \begin{center} \bfseries EDICS Category: 3-BBND \end{center}
% \fi
%
% For peerreview papers, this IEEEtran command inserts a page break and
% creates the second title. It will be ignored for other modes.
%\IEEEpeerreviewmaketitle




\item
The king, queen and jack of clubs are removed from a deck of 52 playing cards and then well shuffled. Now one card is drawn at random from the remaining cards.  Determine the probability that the card is
\begin{enumerate}[label=(\roman*)]
\item a club
\item 10 of hearts
\end{enumerate}
\solution
%\begin{table}[H]
	\centering
\begin{tabular}{|c|c|c|}
\hline
Random variable &Value &Definition\\ \hline
\multirow{3}{*}{X} &0 &Slips of Rs 1\\
&1 &Slips of Rs 5\\
&2 &Slips of Rs 13\\ \hline
\multirow{2}{*}{Y} &0 &Box A\\
&1 &Box B\\\hline
\end{tabular}
\caption{}
\label{tab:Distribution}
\end{table}
See \tabref{tab:Distribution}.
\begin{align}
p_{Y}\brak{k}= \begin{cases} 
      \frac{1}{3} & {k=0} \\
      \frac{2}{3 }& {k=1} 
   \end{cases}
   \\
p_{Y|X}\brak{0|0} = \frac{19}{25}\, 
p_{Y|X}\brak{0|1} = \frac{6}{25}\,
p_{Y|X}\brak{1|0} = \frac{45}{50}\,
p_{Y|X}\brak{1|2} = \frac{5}{50}
\end{align}
The desired probability is the probability that a slip drawn at random is marked other than Rs 1,
\begin{align}
&=1-p_X\brak{0}\\
&= p_X(1) + p_X(2)
\end{align}
Using Bayes theorem,
\begin{align}
&= p_Y\brak{0} \times \pr{Y=0 | X=1} + p_Y\brak{1} \times \pr{Y=1|X=2}\\
&=\frac{1}{3} \times \frac{6}{25} + \frac{2}{3} \times \frac{5}{50}\\
&=\frac{11}{75}
\end{align}

\newpage

%\tableofcontents

\bigskip

\renewcommand{\thefigure}{\theenumi}
\renewcommand{\thetable}{\theenumi}
%\renewcommand{\theequation}{\theenumi}

%\begin{abstract}
%%\boldmath
%In this letter, an algorithm for evaluating the exact analytical bit error rate  (BER)  for the piecewise linear (PL) combiner for  multiple relays is presented. Previous results were available only for upto three relays. The algorithm is unique in the sense that  the actual mathematical expressions, that are prohibitively large, need not be explicitly obtained. The diversity gain due to multiple relays is shown through plots of the analytical BER, well supported by simulations. 
%
%\end{abstract}
% IEEEtran.cls defaults to using nonbold math in the Abstract.
% This preserves the distinction between vectors and scalars. However,
% if the journal you are submitting to favors bold math in the abstract,
% then you can use LaTeX's standard command \boldmath at the very start
% of the abstract to achieve this. Many IEEE journals frown on math
% in the abstract anyway.

% Note that keywords are not normally used for peerreview papers.
%\begin{IEEEkeywords}
%Cooperative diversity, decode and forward, piecewise linear
%\end{IEEEkeywords}



% For peer review papers, you can put extra information on the cover
% page as needed:
% \ifCLASSOPTIONpeerreview
% \begin{center} \bfseries EDICS Category: 3-BBND \end{center}
% \fi
%
% For peerreview papers, this IEEEtran command inserts a page break and
% creates the second title. It will be ignored for other modes.
%\IEEEpeerreviewmaketitle




\item A team of medical students doing their internship have to assist during surgeries
at a city hospital. The probabilities of surgeries rated as very complex, complex,
routine, simple or very simple are respectively, 0.15, 0.20, 0.31, 0.26, .08. Find
the probabilities that a particular surgery will be rated
\begin{enumerate}
	\item complex or very complex;
	\item neither very complex nor very simple;
	\item routine or complex
	\item routine or simple
\end{enumerate}
\solution
%\begin{table}[H]
	\centering
\begin{tabular}{|c|c|c|}
\hline
Random variable &Value &Definition\\ \hline
\multirow{3}{*}{X} &0 &Slips of Rs 1\\
&1 &Slips of Rs 5\\
&2 &Slips of Rs 13\\ \hline
\multirow{2}{*}{Y} &0 &Box A\\
&1 &Box B\\\hline
\end{tabular}
\caption{}
\label{tab:Distribution}
\end{table}
See \tabref{tab:Distribution}.
\begin{align}
p_{Y}\brak{k}= \begin{cases} 
      \frac{1}{3} & {k=0} \\
      \frac{2}{3 }& {k=1} 
   \end{cases}
   \\
p_{Y|X}\brak{0|0} = \frac{19}{25}\, 
p_{Y|X}\brak{0|1} = \frac{6}{25}\,
p_{Y|X}\brak{1|0} = \frac{45}{50}\,
p_{Y|X}\brak{1|2} = \frac{5}{50}
\end{align}
The desired probability is the probability that a slip drawn at random is marked other than Rs 1,
\begin{align}
&=1-p_X\brak{0}\\
&= p_X(1) + p_X(2)
\end{align}
Using Bayes theorem,
\begin{align}
&= p_Y\brak{0} \times \pr{Y=0 | X=1} + p_Y\brak{1} \times \pr{Y=1|X=2}\\
&=\frac{1}{3} \times \frac{6}{25} + \frac{2}{3} \times \frac{5}{50}\\
&=\frac{11}{75}
\end{align}

\newpage

%\tableofcontents

\bigskip

\renewcommand{\thefigure}{\theenumi}
\renewcommand{\thetable}{\theenumi}
%\renewcommand{\theequation}{\theenumi}

%\begin{abstract}
%%\boldmath
%In this letter, an algorithm for evaluating the exact analytical bit error rate  (BER)  for the piecewise linear (PL) combiner for  multiple relays is presented. Previous results were available only for upto three relays. The algorithm is unique in the sense that  the actual mathematical expressions, that are prohibitively large, need not be explicitly obtained. The diversity gain due to multiple relays is shown through plots of the analytical BER, well supported by simulations. 
%
%\end{abstract}
% IEEEtran.cls defaults to using nonbold math in the Abstract.
% This preserves the distinction between vectors and scalars. However,
% if the journal you are submitting to favors bold math in the abstract,
% then you can use LaTeX's standard command \boldmath at the very start
% of the abstract to achieve this. Many IEEE journals frown on math
% in the abstract anyway.

% Note that keywords are not normally used for peerreview papers.
%\begin{IEEEkeywords}
%Cooperative diversity, decode and forward, piecewise linear
%\end{IEEEkeywords}



% For peer review papers, you can put extra information on the cover
% page as needed:
% \ifCLASSOPTIONpeerreview
% \begin{center} \bfseries EDICS Category: 3-BBND \end{center}
% \fi
%
% For peerreview papers, this IEEEtran command inserts a page break and
% creates the second title. It will be ignored for other modes.
%\IEEEpeerreviewmaketitle




\item A card is selected from a pack of 52 cards.
\begin{enumerate}[label=(\alph*)]
    \item How many points are there in the sample space?
    \item Calculate the probability that the card is an ace of spades.
    \item Calculate the probability that the card is (i) an ace and (ii) black card.
\end{enumerate}
\solution
%Let $X$ be an bernoulli rv defined as in \tabref{tab:exemplar/11/16/3/26}.  Then, 
\begin{equation}
    p =
        \frac{4}{11} 
\end{equation}
\begin{table}[H]
	\centering
	\input{exemplar/11/16/3/26/tables/Table2.tex}
	\caption{}
        \label{tab:exemplar/11/16/3/26}
\end{table}

\item The probability that a non leap year selected at random will contain 53 sundays.
\\
\solution
%\begin{table}[H]
	\centering
\begin{tabular}{|c|c|c|}
\hline
Random variable &Value &Definition\\ \hline
\multirow{3}{*}{X} &0 &Slips of Rs 1\\
&1 &Slips of Rs 5\\
&2 &Slips of Rs 13\\ \hline
\multirow{2}{*}{Y} &0 &Box A\\
&1 &Box B\\\hline
\end{tabular}
\caption{}
\label{tab:Distribution}
\end{table}
See \tabref{tab:Distribution}.
\begin{align}
p_{Y}\brak{k}= \begin{cases} 
      \frac{1}{3} & {k=0} \\
      \frac{2}{3 }& {k=1} 
   \end{cases}
   \\
p_{Y|X}\brak{0|0} = \frac{19}{25}\, 
p_{Y|X}\brak{0|1} = \frac{6}{25}\,
p_{Y|X}\brak{1|0} = \frac{45}{50}\,
p_{Y|X}\brak{1|2} = \frac{5}{50}
\end{align}
The desired probability is the probability that a slip drawn at random is marked other than Rs 1,
\begin{align}
&=1-p_X\brak{0}\\
&= p_X(1) + p_X(2)
\end{align}
Using Bayes theorem,
\begin{align}
&= p_Y\brak{0} \times \pr{Y=0 | X=1} + p_Y\brak{1} \times \pr{Y=1|X=2}\\
&=\frac{1}{3} \times \frac{6}{25} + \frac{2}{3} \times \frac{5}{50}\\
&=\frac{11}{75}
\end{align}

\newpage

%\tableofcontents

\bigskip

\renewcommand{\thefigure}{\theenumi}
\renewcommand{\thetable}{\theenumi}
%\renewcommand{\theequation}{\theenumi}

%\begin{abstract}
%%\boldmath
%In this letter, an algorithm for evaluating the exact analytical bit error rate  (BER)  for the piecewise linear (PL) combiner for  multiple relays is presented. Previous results were available only for upto three relays. The algorithm is unique in the sense that  the actual mathematical expressions, that are prohibitively large, need not be explicitly obtained. The diversity gain due to multiple relays is shown through plots of the analytical BER, well supported by simulations. 
%
%\end{abstract}
% IEEEtran.cls defaults to using nonbold math in the Abstract.
% This preserves the distinction between vectors and scalars. However,
% if the journal you are submitting to favors bold math in the abstract,
% then you can use LaTeX's standard command \boldmath at the very start
% of the abstract to achieve this. Many IEEE journals frown on math
% in the abstract anyway.

% Note that keywords are not normally used for peerreview papers.
%\begin{IEEEkeywords}
%Cooperative diversity, decode and forward, piecewise linear
%\end{IEEEkeywords}



% For peer review papers, you can put extra information on the cover
% page as needed:
% \ifCLASSOPTIONpeerreview
% \begin{center} \bfseries EDICS Category: 3-BBND \end{center}
% \fi
%
% For peerreview papers, this IEEEtran command inserts a page break and
% creates the second title. It will be ignored for other modes.
%\IEEEpeerreviewmaketitle




\item One of the four persons John, Rita, Aslam or Gurpreet will be promoted next
month. Consequently the sample space consists of four elementary outcomes
S = {John promoted, Rita promoted, Aslam promoted, Gurpreet promoted}
You are told that the chances of John’s promotion is same as that of Gurpreet,
Rita’s chances of promotion are twice as likely as Johns. Aslam’s chances are
four times that of John.
\begin{enumerate}
	\item Determine
	\begin{enumerate}
		\item P (John promoted)
		\item P (Rita promoted)
		\item P (Aslam promoted)
		\item P (Gurpreet promoted)
	\end{enumerate}
	\item If A = {John promoted or Gurpreet promoted}, find P (A).
\end{enumerate}
\solution
%\begin{table}[H]
	\centering
\begin{tabular}{|c|c|c|}
\hline
Random variable &Value &Definition\\ \hline
\multirow{3}{*}{X} &0 &Slips of Rs 1\\
&1 &Slips of Rs 5\\
&2 &Slips of Rs 13\\ \hline
\multirow{2}{*}{Y} &0 &Box A\\
&1 &Box B\\\hline
\end{tabular}
\caption{}
\label{tab:Distribution}
\end{table}
See \tabref{tab:Distribution}.
\begin{align}
p_{Y}\brak{k}= \begin{cases} 
      \frac{1}{3} & {k=0} \\
      \frac{2}{3 }& {k=1} 
   \end{cases}
   \\
p_{Y|X}\brak{0|0} = \frac{19}{25}\, 
p_{Y|X}\brak{0|1} = \frac{6}{25}\,
p_{Y|X}\brak{1|0} = \frac{45}{50}\,
p_{Y|X}\brak{1|2} = \frac{5}{50}
\end{align}
The desired probability is the probability that a slip drawn at random is marked other than Rs 1,
\begin{align}
&=1-p_X\brak{0}\\
&= p_X(1) + p_X(2)
\end{align}
Using Bayes theorem,
\begin{align}
&= p_Y\brak{0} \times \pr{Y=0 | X=1} + p_Y\brak{1} \times \pr{Y=1|X=2}\\
&=\frac{1}{3} \times \frac{6}{25} + \frac{2}{3} \times \frac{5}{50}\\
&=\frac{11}{75}
\end{align}

\newpage

%\tableofcontents

\bigskip

\renewcommand{\thefigure}{\theenumi}
\renewcommand{\thetable}{\theenumi}
%\renewcommand{\theequation}{\theenumi}

%\begin{abstract}
%%\boldmath
%In this letter, an algorithm for evaluating the exact analytical bit error rate  (BER)  for the piecewise linear (PL) combiner for  multiple relays is presented. Previous results were available only for upto three relays. The algorithm is unique in the sense that  the actual mathematical expressions, that are prohibitively large, need not be explicitly obtained. The diversity gain due to multiple relays is shown through plots of the analytical BER, well supported by simulations. 
%
%\end{abstract}
% IEEEtran.cls defaults to using nonbold math in the Abstract.
% This preserves the distinction between vectors and scalars. However,
% if the journal you are submitting to favors bold math in the abstract,
% then you can use LaTeX's standard command \boldmath at the very start
% of the abstract to achieve this. Many IEEE journals frown on math
% in the abstract anyway.

% Note that keywords are not normally used for peerreview papers.
%\begin{IEEEkeywords}
%Cooperative diversity, decode and forward, piecewise linear
%\end{IEEEkeywords}



% For peer review papers, you can put extra information on the cover
% page as needed:
% \ifCLASSOPTIONpeerreview
% \begin{center} \bfseries EDICS Category: 3-BBND \end{center}
% \fi
%
% For peerreview papers, this IEEEtran command inserts a page break and
% creates the second title. It will be ignored for other modes.
%\IEEEpeerreviewmaketitle




\item A card is drawn from a deck of 52 cards. Find the probability of getting a king or a heart or a red card.\\
\solution
%\begin{table}[H]
	\centering
\begin{tabular}{|c|c|c|}
\hline
Random variable &Value &Definition\\ \hline
\multirow{3}{*}{X} &0 &Slips of Rs 1\\
&1 &Slips of Rs 5\\
&2 &Slips of Rs 13\\ \hline
\multirow{2}{*}{Y} &0 &Box A\\
&1 &Box B\\\hline
\end{tabular}
\caption{}
\label{tab:Distribution}
\end{table}
See \tabref{tab:Distribution}.
\begin{align}
p_{Y}\brak{k}= \begin{cases} 
      \frac{1}{3} & {k=0} \\
      \frac{2}{3 }& {k=1} 
   \end{cases}
   \\
p_{Y|X}\brak{0|0} = \frac{19}{25}\, 
p_{Y|X}\brak{0|1} = \frac{6}{25}\,
p_{Y|X}\brak{1|0} = \frac{45}{50}\,
p_{Y|X}\brak{1|2} = \frac{5}{50}
\end{align}
The desired probability is the probability that a slip drawn at random is marked other than Rs 1,
\begin{align}
&=1-p_X\brak{0}\\
&= p_X(1) + p_X(2)
\end{align}
Using Bayes theorem,
\begin{align}
&= p_Y\brak{0} \times \pr{Y=0 | X=1} + p_Y\brak{1} \times \pr{Y=1|X=2}\\
&=\frac{1}{3} \times \frac{6}{25} + \frac{2}{3} \times \frac{5}{50}\\
&=\frac{11}{75}
\end{align}

\newpage

%\tableofcontents

\bigskip

\renewcommand{\thefigure}{\theenumi}
\renewcommand{\thetable}{\theenumi}
%\renewcommand{\theequation}{\theenumi}

%\begin{abstract}
%%\boldmath
%In this letter, an algorithm for evaluating the exact analytical bit error rate  (BER)  for the piecewise linear (PL) combiner for  multiple relays is presented. Previous results were available only for upto three relays. The algorithm is unique in the sense that  the actual mathematical expressions, that are prohibitively large, need not be explicitly obtained. The diversity gain due to multiple relays is shown through plots of the analytical BER, well supported by simulations. 
%
%\end{abstract}
% IEEEtran.cls defaults to using nonbold math in the Abstract.
% This preserves the distinction between vectors and scalars. However,
% if the journal you are submitting to favors bold math in the abstract,
% then you can use LaTeX's standard command \boldmath at the very start
% of the abstract to achieve this. Many IEEE journals frown on math
% in the abstract anyway.

% Note that keywords are not normally used for peerreview papers.
%\begin{IEEEkeywords}
%Cooperative diversity, decode and forward, piecewise linear
%\end{IEEEkeywords}



% For peer review papers, you can put extra information on the cover
% page as needed:
% \ifCLASSOPTIONpeerreview
% \begin{center} \bfseries EDICS Category: 3-BBND \end{center}
% \fi
%
% For peerreview papers, this IEEEtran command inserts a page break and
% creates the second title. It will be ignored for other modes.
%\IEEEpeerreviewmaketitle




\item The probability that a student will pass his examination is 0.73, the probability of
the student getting a compartment is 0.13, and the probability that the student will
either pass or get compartment is 0.96. State True or False.\\
\solution
%\begin{table}[H]
	\centering
\begin{tabular}{|c|c|c|}
\hline
Random variable &Value &Definition\\ \hline
\multirow{3}{*}{X} &0 &Slips of Rs 1\\
&1 &Slips of Rs 5\\
&2 &Slips of Rs 13\\ \hline
\multirow{2}{*}{Y} &0 &Box A\\
&1 &Box B\\\hline
\end{tabular}
\caption{}
\label{tab:Distribution}
\end{table}
See \tabref{tab:Distribution}.
\begin{align}
p_{Y}\brak{k}= \begin{cases} 
      \frac{1}{3} & {k=0} \\
      \frac{2}{3 }& {k=1} 
   \end{cases}
   \\
p_{Y|X}\brak{0|0} = \frac{19}{25}\, 
p_{Y|X}\brak{0|1} = \frac{6}{25}\,
p_{Y|X}\brak{1|0} = \frac{45}{50}\,
p_{Y|X}\brak{1|2} = \frac{5}{50}
\end{align}
The desired probability is the probability that a slip drawn at random is marked other than Rs 1,
\begin{align}
&=1-p_X\brak{0}\\
&= p_X(1) + p_X(2)
\end{align}
Using Bayes theorem,
\begin{align}
&= p_Y\brak{0} \times \pr{Y=0 | X=1} + p_Y\brak{1} \times \pr{Y=1|X=2}\\
&=\frac{1}{3} \times \frac{6}{25} + \frac{2}{3} \times \frac{5}{50}\\
&=\frac{11}{75}
\end{align}

\newpage

%\tableofcontents

\bigskip

\renewcommand{\thefigure}{\theenumi}
\renewcommand{\thetable}{\theenumi}
%\renewcommand{\theequation}{\theenumi}

%\begin{abstract}
%%\boldmath
%In this letter, an algorithm for evaluating the exact analytical bit error rate  (BER)  for the piecewise linear (PL) combiner for  multiple relays is presented. Previous results were available only for upto three relays. The algorithm is unique in the sense that  the actual mathematical expressions, that are prohibitively large, need not be explicitly obtained. The diversity gain due to multiple relays is shown through plots of the analytical BER, well supported by simulations. 
%
%\end{abstract}
% IEEEtran.cls defaults to using nonbold math in the Abstract.
% This preserves the distinction between vectors and scalars. However,
% if the journal you are submitting to favors bold math in the abstract,
% then you can use LaTeX's standard command \boldmath at the very start
% of the abstract to achieve this. Many IEEE journals frown on math
% in the abstract anyway.

% Note that keywords are not normally used for peerreview papers.
%\begin{IEEEkeywords}
%Cooperative diversity, decode and forward, piecewise linear
%\end{IEEEkeywords}



% For peer review papers, you can put extra information on the cover
% page as needed:
% \ifCLASSOPTIONpeerreview
% \begin{center} \bfseries EDICS Category: 3-BBND \end{center}
% \fi
%
% For peerreview papers, this IEEEtran command inserts a page break and
% creates the second title. It will be ignored for other modes.
%\IEEEpeerreviewmaketitle




\item A card is selected from a pack of 52 cards\\
\begin{enumerate}[label=(\alph*)]
\item How many points are there in the sample space?
\item Calculate the probability that the cards is an ace of spades.
\item Calculate the probability that the card is (i) an ace (ii)black card.\\
\end{enumerate}
%\input{ncert/11/16/3/4_1/Prob_4.tex}
\item In a non-leap year, the probability of having 53 tuesdays or 53 wednesdays is\\
\solution
%A non-leap year has a total of 365 days, and a week has 7 days.\\
So it can be expressed as 
\begin{align}
365\text{days} &=52\times 7+1 \text{day}
\end{align}
$\implies$ 52 tuesdays or wednesdays\\
Random variable X denotes the days of a week
\begin{align}
p_X\brak{k}&=\frac{1}{7}; \quad \brak{1<k<7}
\end{align}
So the probability of extra day being tuesday or wednesday is
\begin{align}
p_X\brak{3}+p_X\brak{4}&=\frac{1}{7}+\frac{1}{7}=\frac{2}{7}
\end{align}



\item There are 1000 sealed envelopes in a box, 10 of them contain a cash prize of
Rs 100 each, 100 of them contain a cash prize of Rs 50 each and 200 of them
contain a cash prize of Rs 10 each and rest do not contain any cash prize. If they
are well shuffled and an envelope is picked up out, what is the probability that it
contains no cash prize?\\
\solution
%\begin{table}[H]
	\centering
\begin{tabular}{|c|c|c|}
\hline
Random variable &Value &Definition\\ \hline
\multirow{3}{*}{X} &0 &Slips of Rs 1\\
&1 &Slips of Rs 5\\
&2 &Slips of Rs 13\\ \hline
\multirow{2}{*}{Y} &0 &Box A\\
&1 &Box B\\\hline
\end{tabular}
\caption{}
\label{tab:Distribution}
\end{table}
See \tabref{tab:Distribution}.
\begin{align}
p_{Y}\brak{k}= \begin{cases} 
      \frac{1}{3} & {k=0} \\
      \frac{2}{3 }& {k=1} 
   \end{cases}
   \\
p_{Y|X}\brak{0|0} = \frac{19}{25}\, 
p_{Y|X}\brak{0|1} = \frac{6}{25}\,
p_{Y|X}\brak{1|0} = \frac{45}{50}\,
p_{Y|X}\brak{1|2} = \frac{5}{50}
\end{align}
The desired probability is the probability that a slip drawn at random is marked other than Rs 1,
\begin{align}
&=1-p_X\brak{0}\\
&= p_X(1) + p_X(2)
\end{align}
Using Bayes theorem,
\begin{align}
&= p_Y\brak{0} \times \pr{Y=0 | X=1} + p_Y\brak{1} \times \pr{Y=1|X=2}\\
&=\frac{1}{3} \times \frac{6}{25} + \frac{2}{3} \times \frac{5}{50}\\
&=\frac{11}{75}
\end{align}

\newpage

%\tableofcontents

\bigskip

\renewcommand{\thefigure}{\theenumi}
\renewcommand{\thetable}{\theenumi}
%\renewcommand{\theequation}{\theenumi}

%\begin{abstract}
%%\boldmath
%In this letter, an algorithm for evaluating the exact analytical bit error rate  (BER)  for the piecewise linear (PL) combiner for  multiple relays is presented. Previous results were available only for upto three relays. The algorithm is unique in the sense that  the actual mathematical expressions, that are prohibitively large, need not be explicitly obtained. The diversity gain due to multiple relays is shown through plots of the analytical BER, well supported by simulations. 
%
%\end{abstract}
% IEEEtran.cls defaults to using nonbold math in the Abstract.
% This preserves the distinction between vectors and scalars. However,
% if the journal you are submitting to favors bold math in the abstract,
% then you can use LaTeX's standard command \boldmath at the very start
% of the abstract to achieve this. Many IEEE journals frown on math
% in the abstract anyway.

% Note that keywords are not normally used for peerreview papers.
%\begin{IEEEkeywords}
%Cooperative diversity, decode and forward, piecewise linear
%\end{IEEEkeywords}



% For peer review papers, you can put extra information on the cover
% page as needed:
% \ifCLASSOPTIONpeerreview
% \begin{center} \bfseries EDICS Category: 3-BBND \end{center}
% \fi
%
% For peerreview papers, this IEEEtran command inserts a page break and
% creates the second title. It will be ignored for other modes.
%\IEEEpeerreviewmaketitle




\item 
A die is thrown and a card is selected at random from a deck of 52 playing cards. The probability of getting an even number on the die and a spade card.\\
\solution
%\begin{table}[H]
	\centering
\begin{tabular}{|c|c|c|}
\hline
Random variable &Value &Definition\\ \hline
\multirow{3}{*}{X} &0 &Slips of Rs 1\\
&1 &Slips of Rs 5\\
&2 &Slips of Rs 13\\ \hline
\multirow{2}{*}{Y} &0 &Box A\\
&1 &Box B\\\hline
\end{tabular}
\caption{}
\label{tab:Distribution}
\end{table}
See \tabref{tab:Distribution}.
\begin{align}
p_{Y}\brak{k}= \begin{cases} 
      \frac{1}{3} & {k=0} \\
      \frac{2}{3 }& {k=1} 
   \end{cases}
   \\
p_{Y|X}\brak{0|0} = \frac{19}{25}\, 
p_{Y|X}\brak{0|1} = \frac{6}{25}\,
p_{Y|X}\brak{1|0} = \frac{45}{50}\,
p_{Y|X}\brak{1|2} = \frac{5}{50}
\end{align}
The desired probability is the probability that a slip drawn at random is marked other than Rs 1,
\begin{align}
&=1-p_X\brak{0}\\
&= p_X(1) + p_X(2)
\end{align}
Using Bayes theorem,
\begin{align}
&= p_Y\brak{0} \times \pr{Y=0 | X=1} + p_Y\brak{1} \times \pr{Y=1|X=2}\\
&=\frac{1}{3} \times \frac{6}{25} + \frac{2}{3} \times \frac{5}{50}\\
&=\frac{11}{75}
\end{align}

\newpage

%\tableofcontents

\bigskip

\renewcommand{\thefigure}{\theenumi}
\renewcommand{\thetable}{\theenumi}
%\renewcommand{\theequation}{\theenumi}

%\begin{abstract}
%%\boldmath
%In this letter, an algorithm for evaluating the exact analytical bit error rate  (BER)  for the piecewise linear (PL) combiner for  multiple relays is presented. Previous results were available only for upto three relays. The algorithm is unique in the sense that  the actual mathematical expressions, that are prohibitively large, need not be explicitly obtained. The diversity gain due to multiple relays is shown through plots of the analytical BER, well supported by simulations. 
%
%\end{abstract}
% IEEEtran.cls defaults to using nonbold math in the Abstract.
% This preserves the distinction between vectors and scalars. However,
% if the journal you are submitting to favors bold math in the abstract,
% then you can use LaTeX's standard command \boldmath at the very start
% of the abstract to achieve this. Many IEEE journals frown on math
% in the abstract anyway.

% Note that keywords are not normally used for peerreview papers.
%\begin{IEEEkeywords}
%Cooperative diversity, decode and forward, piecewise linear
%\end{IEEEkeywords}



% For peer review papers, you can put extra information on the cover
% page as needed:
% \ifCLASSOPTIONpeerreview
% \begin{center} \bfseries EDICS Category: 3-BBND \end{center}
% \fi
%
% For peerreview papers, this IEEEtran command inserts a page break and
% creates the second title. It will be ignored for other modes.
%\IEEEpeerreviewmaketitle




\item
If 4-digit numbers greater than 5,000 are randomly formed from the digits 0, 1, 3, 5, and 7, what is the probability of forming a number divisible by 5 when:
\begin{enumerate}
    \item The digits are repeated?
    \item The repetition of digits is not allowed?
\end{enumerate}
\solution
%\begin{table}[H]
	\centering
\begin{tabular}{|c|c|c|}
\hline
Random variable &Value &Definition\\ \hline
\multirow{3}{*}{X} &0 &Slips of Rs 1\\
&1 &Slips of Rs 5\\
&2 &Slips of Rs 13\\ \hline
\multirow{2}{*}{Y} &0 &Box A\\
&1 &Box B\\\hline
\end{tabular}
\caption{}
\label{tab:Distribution}
\end{table}
See \tabref{tab:Distribution}.
\begin{align}
p_{Y}\brak{k}= \begin{cases} 
      \frac{1}{3} & {k=0} \\
      \frac{2}{3 }& {k=1} 
   \end{cases}
   \\
p_{Y|X}\brak{0|0} = \frac{19}{25}\, 
p_{Y|X}\brak{0|1} = \frac{6}{25}\,
p_{Y|X}\brak{1|0} = \frac{45}{50}\,
p_{Y|X}\brak{1|2} = \frac{5}{50}
\end{align}
The desired probability is the probability that a slip drawn at random is marked other than Rs 1,
\begin{align}
&=1-p_X\brak{0}\\
&= p_X(1) + p_X(2)
\end{align}
Using Bayes theorem,
\begin{align}
&= p_Y\brak{0} \times \pr{Y=0 | X=1} + p_Y\brak{1} \times \pr{Y=1|X=2}\\
&=\frac{1}{3} \times \frac{6}{25} + \frac{2}{3} \times \frac{5}{50}\\
&=\frac{11}{75}
\end{align}

\newpage

%\tableofcontents

\bigskip

\renewcommand{\thefigure}{\theenumi}
\renewcommand{\thetable}{\theenumi}
%\renewcommand{\theequation}{\theenumi}

%\begin{abstract}
%%\boldmath
%In this letter, an algorithm for evaluating the exact analytical bit error rate  (BER)  for the piecewise linear (PL) combiner for  multiple relays is presented. Previous results were available only for upto three relays. The algorithm is unique in the sense that  the actual mathematical expressions, that are prohibitively large, need not be explicitly obtained. The diversity gain due to multiple relays is shown through plots of the analytical BER, well supported by simulations. 
%
%\end{abstract}
% IEEEtran.cls defaults to using nonbold math in the Abstract.
% This preserves the distinction between vectors and scalars. However,
% if the journal you are submitting to favors bold math in the abstract,
% then you can use LaTeX's standard command \boldmath at the very start
% of the abstract to achieve this. Many IEEE journals frown on math
% in the abstract anyway.

% Note that keywords are not normally used for peerreview papers.
%\begin{IEEEkeywords}
%Cooperative diversity, decode and forward, piecewise linear
%\end{IEEEkeywords}



% For peer review papers, you can put extra information on the cover
% page as needed:
% \ifCLASSOPTIONpeerreview
% \begin{center} \bfseries EDICS Category: 3-BBND \end{center}
% \fi
%
% For peerreview papers, this IEEEtran command inserts a page break and
% creates the second title. It will be ignored for other modes.
%\IEEEpeerreviewmaketitle




\item Consider the probability space $\brak{\Omega, \mathcal{G}, P}$ where $\Omega = [0,2]$ and $\mathcal{G} = \cbrak{\phi, \Omega, [0,1], (1,2]}$. Let $X$ and $Y$ be two functions on $\Omega$ defined as
\begin{align*}
    X(\omega) = 
    \begin{cases}
        1 & \text{if }\omega \in [0, 1]\\
        2 & \text{if }\omega \in (1, 2]
    \end{cases}
\end{align*}
and
\begin{align*}
    Y(\omega) = 
    \begin{cases}
        2 & \text{if }\omega \in [0, 1.5]\\
        3 & \text{if }\omega \in (1.5, 2].
    \end{cases}
\end{align*}
Then which one of the following statements is true?
\begin{enumerate}
    \item [(A)] $X$ is a random variable with respect to $\mathcal{G}$, but $Y$ is not a random variable with respect to $\mathcal{G}$.
    \item [(B)] $Y$ is a random variable with respect to $\mathcal{G}$, but $X$ is not a random variable with respect to $\mathcal{G}$.
    \item [(C)] Neither $X$ nor $Y$ is a random variable with respect to $\mathcal{G}$.
    \item [(D)] Both $X$ and $Y$ are random variables with respect to $\mathcal{G}$.
\end{enumerate} \hfill (GATE ST 2023)\\
\solution
%\begin{table}[H]
	\centering
\begin{tabular}{|c|c|c|}
\hline
Random variable &Value &Definition\\ \hline
\multirow{3}{*}{X} &0 &Slips of Rs 1\\
&1 &Slips of Rs 5\\
&2 &Slips of Rs 13\\ \hline
\multirow{2}{*}{Y} &0 &Box A\\
&1 &Box B\\\hline
\end{tabular}
\caption{}
\label{tab:Distribution}
\end{table}
See \tabref{tab:Distribution}.
\begin{align}
p_{Y}\brak{k}= \begin{cases} 
      \frac{1}{3} & {k=0} \\
      \frac{2}{3 }& {k=1} 
   \end{cases}
   \\
p_{Y|X}\brak{0|0} = \frac{19}{25}\, 
p_{Y|X}\brak{0|1} = \frac{6}{25}\,
p_{Y|X}\brak{1|0} = \frac{45}{50}\,
p_{Y|X}\brak{1|2} = \frac{5}{50}
\end{align}
The desired probability is the probability that a slip drawn at random is marked other than Rs 1,
\begin{align}
&=1-p_X\brak{0}\\
&= p_X(1) + p_X(2)
\end{align}
Using Bayes theorem,
\begin{align}
&= p_Y\brak{0} \times \pr{Y=0 | X=1} + p_Y\brak{1} \times \pr{Y=1|X=2}\\
&=\frac{1}{3} \times \frac{6}{25} + \frac{2}{3} \times \frac{5}{50}\\
&=\frac{11}{75}
\end{align}

\newpage

%\tableofcontents

\bigskip

\renewcommand{\thefigure}{\theenumi}
\renewcommand{\thetable}{\theenumi}
%\renewcommand{\theequation}{\theenumi}

%\begin{abstract}
%%\boldmath
%In this letter, an algorithm for evaluating the exact analytical bit error rate  (BER)  for the piecewise linear (PL) combiner for  multiple relays is presented. Previous results were available only for upto three relays. The algorithm is unique in the sense that  the actual mathematical expressions, that are prohibitively large, need not be explicitly obtained. The diversity gain due to multiple relays is shown through plots of the analytical BER, well supported by simulations. 
%
%\end{abstract}
% IEEEtran.cls defaults to using nonbold math in the Abstract.
% This preserves the distinction between vectors and scalars. However,
% if the journal you are submitting to favors bold math in the abstract,
% then you can use LaTeX's standard command \boldmath at the very start
% of the abstract to achieve this. Many IEEE journals frown on math
% in the abstract anyway.

% Note that keywords are not normally used for peerreview papers.
%\begin{IEEEkeywords}
%Cooperative diversity, decode and forward, piecewise linear
%\end{IEEEkeywords}



% For peer review papers, you can put extra information on the cover
% page as needed:
% \ifCLASSOPTIONpeerreview
% \begin{center} \bfseries EDICS Category: 3-BBND \end{center}
% \fi
%
% For peerreview papers, this IEEEtran command inserts a page break and
% creates the second title. It will be ignored for other modes.
%\IEEEpeerreviewmaketitle




	\item  A die is loaded in such a way that each odd number is twice as likely to occur as
each even number. Find $P(G)$, where $G$ is the event that a number greater than
3 occurs on a single roll of the die.
\\
\solution
		%\begin{table}[H]
	\centering
\begin{tabular}{|c|c|c|}
\hline
Random variable &Value &Definition\\ \hline
\multirow{3}{*}{X} &0 &Slips of Rs 1\\
&1 &Slips of Rs 5\\
&2 &Slips of Rs 13\\ \hline
\multirow{2}{*}{Y} &0 &Box A\\
&1 &Box B\\\hline
\end{tabular}
\caption{}
\label{tab:Distribution}
\end{table}
See \tabref{tab:Distribution}.
\begin{align}
p_{Y}\brak{k}= \begin{cases} 
      \frac{1}{3} & {k=0} \\
      \frac{2}{3 }& {k=1} 
   \end{cases}
   \\
p_{Y|X}\brak{0|0} = \frac{19}{25}\, 
p_{Y|X}\brak{0|1} = \frac{6}{25}\,
p_{Y|X}\brak{1|0} = \frac{45}{50}\,
p_{Y|X}\brak{1|2} = \frac{5}{50}
\end{align}
The desired probability is the probability that a slip drawn at random is marked other than Rs 1,
\begin{align}
&=1-p_X\brak{0}\\
&= p_X(1) + p_X(2)
\end{align}
Using Bayes theorem,
\begin{align}
&= p_Y\brak{0} \times \pr{Y=0 | X=1} + p_Y\brak{1} \times \pr{Y=1|X=2}\\
&=\frac{1}{3} \times \frac{6}{25} + \frac{2}{3} \times \frac{5}{50}\\
&=\frac{11}{75}
\end{align}

\newpage

%\tableofcontents

\bigskip

\renewcommand{\thefigure}{\theenumi}
\renewcommand{\thetable}{\theenumi}
%\renewcommand{\theequation}{\theenumi}

%\begin{abstract}
%%\boldmath
%In this letter, an algorithm for evaluating the exact analytical bit error rate  (BER)  for the piecewise linear (PL) combiner for  multiple relays is presented. Previous results were available only for upto three relays. The algorithm is unique in the sense that  the actual mathematical expressions, that are prohibitively large, need not be explicitly obtained. The diversity gain due to multiple relays is shown through plots of the analytical BER, well supported by simulations. 
%
%\end{abstract}
% IEEEtran.cls defaults to using nonbold math in the Abstract.
% This preserves the distinction between vectors and scalars. However,
% if the journal you are submitting to favors bold math in the abstract,
% then you can use LaTeX's standard command \boldmath at the very start
% of the abstract to achieve this. Many IEEE journals frown on math
% in the abstract anyway.

% Note that keywords are not normally used for peerreview papers.
%\begin{IEEEkeywords}
%Cooperative diversity, decode and forward, piecewise linear
%\end{IEEEkeywords}



% For peer review papers, you can put extra information on the cover
% page as needed:
% \ifCLASSOPTIONpeerreview
% \begin{center} \bfseries EDICS Category: 3-BBND \end{center}
% \fi
%
% For peerreview papers, this IEEEtran command inserts a page break and
% creates the second title. It will be ignored for other modes.
%\IEEEpeerreviewmaketitle




	\item All the jacks, queens and kings are removed from a deck of 52 playing cards. The remaining cards are well shuffled and then one card is drawn at random. Giving ace a value 1 similar value for other cards, find the probability that the card has a value 
		\begin{enumerate}
			\item 7
			\item greater than 7
			\item less than 7
		\end{enumerate}
		%Number of cards left after removing all jacks, queens and kings 
\begin{align}
N	= 52 - 4\times 3
	= 40
\end{align}
%\begin{table}[H]
%\def\arraystretch{1.2}
%\begin{tabular}{|c|c|c|}
%\hline
%	\textbf{Parameter} &\textbf{Value} &\textbf{Description}\\ \hline
%	$X$ &1-10 &Represents the value of the card picked \\ \hline
%\end{tabular}
%\end{table}
Let $1 \le X \le 10$ be the value of the card picked.  Then,
\begin{align}
	p_X(k) &= \Pr(X=k)\ \forall\ 1 \leq k \leq 10\\
	&= \frac{4\times 1}{40}\\
	&= \frac{1}{10}\\
	\therefore p_X(k) &= 
	\begin{cases}
		\frac{1}{10} & 1 \leq k \leq 10\\
		0 & \text{otherwise}
	\end{cases}
\end{align}
and
\begin{align}
	F_{X}(k) &= \sum_{m=0}^{k}p_{X}(m) \quad 1 \leq k \leq 10\\
	&= \frac{k}{10}\\
	\therefore F_{X}(k) &= 
	\begin{cases}
		0 & k \leq 0\\
		\frac{k}{10} & 1\leq k \leq 10\\
		1 & k > 10 
	\end{cases}
\end{align}
\begin{enumerate}
	\item Probability that card has value equal to 7 is
		\begin{align}
			 p_{X}(7)
			= \frac{1}{10}
		\end{align}
	\item Probability that card has value greater than 7 is
		\begin{align}
			1 - F_X(7)
			&= 1 - \frac{7}{10}
			\\
			&= \frac{3}{10}
		\end{align}
	\item Probability that card has value less than 7 is
		\begin{align}
			 F_{X}(6)
			=\frac{6}{10}
		\end{align}
\end{enumerate}

  \item A Lot consists of 48 mobile phones of which 42 are good, 3 have only minor defects and 3 have major defects.Varnika will buy a phone if it is good but the trader will only buy a mobile if it has no major defects. One phone is selected at random from the lot. What is the probability that it is
\begin{enumerate}
	\item acceptable to Varnika?
            \item acceptable to the trader?
\end{enumerate}
\solution
	%\begin{table}[H]
	\centering
\begin{tabular}{|c|c|c|}
\hline
Random variable &Value &Definition\\ \hline
\multirow{3}{*}{X} &0 &Slips of Rs 1\\
&1 &Slips of Rs 5\\
&2 &Slips of Rs 13\\ \hline
\multirow{2}{*}{Y} &0 &Box A\\
&1 &Box B\\\hline
\end{tabular}
\caption{}
\label{tab:Distribution}
\end{table}
See \tabref{tab:Distribution}.
\begin{align}
p_{Y}\brak{k}= \begin{cases} 
      \frac{1}{3} & {k=0} \\
      \frac{2}{3 }& {k=1} 
   \end{cases}
   \\
p_{Y|X}\brak{0|0} = \frac{19}{25}\, 
p_{Y|X}\brak{0|1} = \frac{6}{25}\,
p_{Y|X}\brak{1|0} = \frac{45}{50}\,
p_{Y|X}\brak{1|2} = \frac{5}{50}
\end{align}
The desired probability is the probability that a slip drawn at random is marked other than Rs 1,
\begin{align}
&=1-p_X\brak{0}\\
&= p_X(1) + p_X(2)
\end{align}
Using Bayes theorem,
\begin{align}
&= p_Y\brak{0} \times \pr{Y=0 | X=1} + p_Y\brak{1} \times \pr{Y=1|X=2}\\
&=\frac{1}{3} \times \frac{6}{25} + \frac{2}{3} \times \frac{5}{50}\\
&=\frac{11}{75}
\end{align}

\newpage

%\tableofcontents

\bigskip

\renewcommand{\thefigure}{\theenumi}
\renewcommand{\thetable}{\theenumi}
%\renewcommand{\theequation}{\theenumi}

%\begin{abstract}
%%\boldmath
%In this letter, an algorithm for evaluating the exact analytical bit error rate  (BER)  for the piecewise linear (PL) combiner for  multiple relays is presented. Previous results were available only for upto three relays. The algorithm is unique in the sense that  the actual mathematical expressions, that are prohibitively large, need not be explicitly obtained. The diversity gain due to multiple relays is shown through plots of the analytical BER, well supported by simulations. 
%
%\end{abstract}
% IEEEtran.cls defaults to using nonbold math in the Abstract.
% This preserves the distinction between vectors and scalars. However,
% if the journal you are submitting to favors bold math in the abstract,
% then you can use LaTeX's standard command \boldmath at the very start
% of the abstract to achieve this. Many IEEE journals frown on math
% in the abstract anyway.

% Note that keywords are not normally used for peerreview papers.
%\begin{IEEEkeywords}
%Cooperative diversity, decode and forward, piecewise linear
%\end{IEEEkeywords}



% For peer review papers, you can put extra information on the cover
% page as needed:
% \ifCLASSOPTIONpeerreview
% \begin{center} \bfseries EDICS Category: 3-BBND \end{center}
% \fi
%
% For peerreview papers, this IEEEtran command inserts a page break and
% creates the second title. It will be ignored for other modes.
%\IEEEpeerreviewmaketitle




 \item A student says that if you throw a die, it will show up 1 or not 1. Therefore, the probability of getting 1 and the probability of getting 'not 1' each is equal to $\frac{1}{2}$. Is this correct? Give reasons.\\
 \solution
        %\begin{table}[H]
	\centering
\begin{tabular}{|c|c|c|}
\hline
Random variable &Value &Definition\\ \hline
\multirow{3}{*}{X} &0 &Slips of Rs 1\\
&1 &Slips of Rs 5\\
&2 &Slips of Rs 13\\ \hline
\multirow{2}{*}{Y} &0 &Box A\\
&1 &Box B\\\hline
\end{tabular}
\caption{}
\label{tab:Distribution}
\end{table}
See \tabref{tab:Distribution}.
\begin{align}
p_{Y}\brak{k}= \begin{cases} 
      \frac{1}{3} & {k=0} \\
      \frac{2}{3 }& {k=1} 
   \end{cases}
   \\
p_{Y|X}\brak{0|0} = \frac{19}{25}\, 
p_{Y|X}\brak{0|1} = \frac{6}{25}\,
p_{Y|X}\brak{1|0} = \frac{45}{50}\,
p_{Y|X}\brak{1|2} = \frac{5}{50}
\end{align}
The desired probability is the probability that a slip drawn at random is marked other than Rs 1,
\begin{align}
&=1-p_X\brak{0}\\
&= p_X(1) + p_X(2)
\end{align}
Using Bayes theorem,
\begin{align}
&= p_Y\brak{0} \times \pr{Y=0 | X=1} + p_Y\brak{1} \times \pr{Y=1|X=2}\\
&=\frac{1}{3} \times \frac{6}{25} + \frac{2}{3} \times \frac{5}{50}\\
&=\frac{11}{75}
\end{align}

\newpage

%\tableofcontents

\bigskip

\renewcommand{\thefigure}{\theenumi}
\renewcommand{\thetable}{\theenumi}
%\renewcommand{\theequation}{\theenumi}

%\begin{abstract}
%%\boldmath
%In this letter, an algorithm for evaluating the exact analytical bit error rate  (BER)  for the piecewise linear (PL) combiner for  multiple relays is presented. Previous results were available only for upto three relays. The algorithm is unique in the sense that  the actual mathematical expressions, that are prohibitively large, need not be explicitly obtained. The diversity gain due to multiple relays is shown through plots of the analytical BER, well supported by simulations. 
%
%\end{abstract}
% IEEEtran.cls defaults to using nonbold math in the Abstract.
% This preserves the distinction between vectors and scalars. However,
% if the journal you are submitting to favors bold math in the abstract,
% then you can use LaTeX's standard command \boldmath at the very start
% of the abstract to achieve this. Many IEEE journals frown on math
% in the abstract anyway.

% Note that keywords are not normally used for peerreview papers.
%\begin{IEEEkeywords}
%Cooperative diversity, decode and forward, piecewise linear
%\end{IEEEkeywords}



% For peer review papers, you can put extra information on the cover
% page as needed:
% \ifCLASSOPTIONpeerreview
% \begin{center} \bfseries EDICS Category: 3-BBND \end{center}
% \fi
%
% For peerreview papers, this IEEEtran command inserts a page break and
% creates the second title. It will be ignored for other modes.
%\IEEEpeerreviewmaketitle




   \item Four candidates A, B, C, D have ap-
plied for the assignment to coach a school cricket
team. If A is twice as likely to be selected as B, and
B and C are given about the same chance of being
selected, while C is twice as likely to be selected
as D, what are the probabilities that
\begin{enumerate}
\item C will be selected?
\item A will not be selected?
\end{enumerate}
	%\begin{table}[H]
	\centering
\begin{tabular}{|c|c|c|}
\hline
Random variable &Value &Definition\\ \hline
\multirow{3}{*}{X} &0 &Slips of Rs 1\\
&1 &Slips of Rs 5\\
&2 &Slips of Rs 13\\ \hline
\multirow{2}{*}{Y} &0 &Box A\\
&1 &Box B\\\hline
\end{tabular}
\caption{}
\label{tab:Distribution}
\end{table}
See \tabref{tab:Distribution}.
\begin{align}
p_{Y}\brak{k}= \begin{cases} 
      \frac{1}{3} & {k=0} \\
      \frac{2}{3 }& {k=1} 
   \end{cases}
   \\
p_{Y|X}\brak{0|0} = \frac{19}{25}\, 
p_{Y|X}\brak{0|1} = \frac{6}{25}\,
p_{Y|X}\brak{1|0} = \frac{45}{50}\,
p_{Y|X}\brak{1|2} = \frac{5}{50}
\end{align}
The desired probability is the probability that a slip drawn at random is marked other than Rs 1,
\begin{align}
&=1-p_X\brak{0}\\
&= p_X(1) + p_X(2)
\end{align}
Using Bayes theorem,
\begin{align}
&= p_Y\brak{0} \times \pr{Y=0 | X=1} + p_Y\brak{1} \times \pr{Y=1|X=2}\\
&=\frac{1}{3} \times \frac{6}{25} + \frac{2}{3} \times \frac{5}{50}\\
&=\frac{11}{75}
\end{align}

\newpage

%\tableofcontents

\bigskip

\renewcommand{\thefigure}{\theenumi}
\renewcommand{\thetable}{\theenumi}
%\renewcommand{\theequation}{\theenumi}

%\begin{abstract}
%%\boldmath
%In this letter, an algorithm for evaluating the exact analytical bit error rate  (BER)  for the piecewise linear (PL) combiner for  multiple relays is presented. Previous results were available only for upto three relays. The algorithm is unique in the sense that  the actual mathematical expressions, that are prohibitively large, need not be explicitly obtained. The diversity gain due to multiple relays is shown through plots of the analytical BER, well supported by simulations. 
%
%\end{abstract}
% IEEEtran.cls defaults to using nonbold math in the Abstract.
% This preserves the distinction between vectors and scalars. However,
% if the journal you are submitting to favors bold math in the abstract,
% then you can use LaTeX's standard command \boldmath at the very start
% of the abstract to achieve this. Many IEEE journals frown on math
% in the abstract anyway.

% Note that keywords are not normally used for peerreview papers.
%\begin{IEEEkeywords}
%Cooperative diversity, decode and forward, piecewise linear
%\end{IEEEkeywords}



% For peer review papers, you can put extra information on the cover
% page as needed:
% \ifCLASSOPTIONpeerreview
% \begin{center} \bfseries EDICS Category: 3-BBND \end{center}
% \fi
%
% For peerreview papers, this IEEEtran command inserts a page break and
% creates the second title. It will be ignored for other modes.
%\IEEEpeerreviewmaketitle




 \item A bag contain 24 balls of which $x$ balls are red, $2x$ are white and $3x$ are blue. A ball is selected at random, What is the probability that it is
\begin{enumerate}[label=\alph*)]
\item not red ?
\item white ?
\end{enumerate}
%\begin{table}[H]
	\centering
\begin{tabular}{|c|c|c|}
\hline
Random variable &Value &Definition\\ \hline
\multirow{3}{*}{X} &0 &Slips of Rs 1\\
&1 &Slips of Rs 5\\
&2 &Slips of Rs 13\\ \hline
\multirow{2}{*}{Y} &0 &Box A\\
&1 &Box B\\\hline
\end{tabular}
\caption{}
\label{tab:Distribution}
\end{table}
See \tabref{tab:Distribution}.
\begin{align}
p_{Y}\brak{k}= \begin{cases} 
      \frac{1}{3} & {k=0} \\
      \frac{2}{3 }& {k=1} 
   \end{cases}
   \\
p_{Y|X}\brak{0|0} = \frac{19}{25}\, 
p_{Y|X}\brak{0|1} = \frac{6}{25}\,
p_{Y|X}\brak{1|0} = \frac{45}{50}\,
p_{Y|X}\brak{1|2} = \frac{5}{50}
\end{align}
The desired probability is the probability that a slip drawn at random is marked other than Rs 1,
\begin{align}
&=1-p_X\brak{0}\\
&= p_X(1) + p_X(2)
\end{align}
Using Bayes theorem,
\begin{align}
&= p_Y\brak{0} \times \pr{Y=0 | X=1} + p_Y\brak{1} \times \pr{Y=1|X=2}\\
&=\frac{1}{3} \times \frac{6}{25} + \frac{2}{3} \times \frac{5}{50}\\
&=\frac{11}{75}
\end{align}

\newpage

%\tableofcontents

\bigskip

\renewcommand{\thefigure}{\theenumi}
\renewcommand{\thetable}{\theenumi}
%\renewcommand{\theequation}{\theenumi}

%\begin{abstract}
%%\boldmath
%In this letter, an algorithm for evaluating the exact analytical bit error rate  (BER)  for the piecewise linear (PL) combiner for  multiple relays is presented. Previous results were available only for upto three relays. The algorithm is unique in the sense that  the actual mathematical expressions, that are prohibitively large, need not be explicitly obtained. The diversity gain due to multiple relays is shown through plots of the analytical BER, well supported by simulations. 
%
%\end{abstract}
% IEEEtran.cls defaults to using nonbold math in the Abstract.
% This preserves the distinction between vectors and scalars. However,
% if the journal you are submitting to favors bold math in the abstract,
% then you can use LaTeX's standard command \boldmath at the very start
% of the abstract to achieve this. Many IEEE journals frown on math
% in the abstract anyway.

% Note that keywords are not normally used for peerreview papers.
%\begin{IEEEkeywords}
%Cooperative diversity, decode and forward, piecewise linear
%\end{IEEEkeywords}



% For peer review papers, you can put extra information on the cover
% page as needed:
% \ifCLASSOPTIONpeerreview
% \begin{center} \bfseries EDICS Category: 3-BBND \end{center}
% \fi
%
% For peerreview papers, this IEEEtran command inserts a page break and
% creates the second title. It will be ignored for other modes.
%\IEEEpeerreviewmaketitle




If the letters of the word ASSASSINATION are arranged at random. Find the Probability that
\begin{enumerate}[label=(\alph*)]
\item Four $S's$ come consecutively in the word
\item Two  $I's$ and two $N's$ come together
\item All $A's$ are not coming together
\item No two $A's$ are coming together
\end{enumerate}
%\begin{table}[H]
	\centering
\begin{tabular}{|c|c|c|}
\hline
Random variable &Value &Definition\\ \hline
\multirow{3}{*}{X} &0 &Slips of Rs 1\\
&1 &Slips of Rs 5\\
&2 &Slips of Rs 13\\ \hline
\multirow{2}{*}{Y} &0 &Box A\\
&1 &Box B\\\hline
\end{tabular}
\caption{}
\label{tab:Distribution}
\end{table}
See \tabref{tab:Distribution}.
\begin{align}
p_{Y}\brak{k}= \begin{cases} 
      \frac{1}{3} & {k=0} \\
      \frac{2}{3 }& {k=1} 
   \end{cases}
   \\
p_{Y|X}\brak{0|0} = \frac{19}{25}\, 
p_{Y|X}\brak{0|1} = \frac{6}{25}\,
p_{Y|X}\brak{1|0} = \frac{45}{50}\,
p_{Y|X}\brak{1|2} = \frac{5}{50}
\end{align}
The desired probability is the probability that a slip drawn at random is marked other than Rs 1,
\begin{align}
&=1-p_X\brak{0}\\
&= p_X(1) + p_X(2)
\end{align}
Using Bayes theorem,
\begin{align}
&= p_Y\brak{0} \times \pr{Y=0 | X=1} + p_Y\brak{1} \times \pr{Y=1|X=2}\\
&=\frac{1}{3} \times \frac{6}{25} + \frac{2}{3} \times \frac{5}{50}\\
&=\frac{11}{75}
\end{align}

\newpage

%\tableofcontents

\bigskip

\renewcommand{\thefigure}{\theenumi}
\renewcommand{\thetable}{\theenumi}
%\renewcommand{\theequation}{\theenumi}

%\begin{abstract}
%%\boldmath
%In this letter, an algorithm for evaluating the exact analytical bit error rate  (BER)  for the piecewise linear (PL) combiner for  multiple relays is presented. Previous results were available only for upto three relays. The algorithm is unique in the sense that  the actual mathematical expressions, that are prohibitively large, need not be explicitly obtained. The diversity gain due to multiple relays is shown through plots of the analytical BER, well supported by simulations. 
%
%\end{abstract}
% IEEEtran.cls defaults to using nonbold math in the Abstract.
% This preserves the distinction between vectors and scalars. However,
% if the journal you are submitting to favors bold math in the abstract,
% then you can use LaTeX's standard command \boldmath at the very start
% of the abstract to achieve this. Many IEEE journals frown on math
% in the abstract anyway.

% Note that keywords are not normally used for peerreview papers.
%\begin{IEEEkeywords}
%Cooperative diversity, decode and forward, piecewise linear
%\end{IEEEkeywords}



% For peer review papers, you can put extra information on the cover
% page as needed:
% \ifCLASSOPTIONpeerreview
% \begin{center} \bfseries EDICS Category: 3-BBND \end{center}
% \fi
%
% For peerreview papers, this IEEEtran command inserts a page break and
% creates the second title. It will be ignored for other modes.
%\IEEEpeerreviewmaketitle




	\item One urn contains two black balls (labelled B1 and B2) and one white ball. A
	second urn contains one black ball and two white balls (labelled W1 and W2).
	Suppose the following experiment is performed. One of the two urns is chosen
	at random. Next a ball is randomly chosen from the urn. Then a second ball is
	chosen at random from the same urn without replacing the first ball.
	
	\begin{enumerate}
	\item What is the probability that two black balls are chosen?
	
	\item What is the probability that two balls of opposite colour are chosen?
	\end{enumerate}
	\solution
	%\begin{align}
    \label{eq:12.13.6.18.1}
	\because	\pr{A|B} &> \pr{A},\
\frac{\pr{AB}}{\pr{B}} > \pr{A}
\\
    \label{eq:12.13.6.18.2}
	\implies \pr{AB} &> \pr{A}\pr{B}
	\\
	\text{or, } \frac{\pr{AB}}{\pr{A}} &=\pr{B|A} > \pr{A}
\end{align}

\end{enumerate}

		\item A box of oranges is inspected by examining three randomly selected oranges drawn without replacement. If all the three oranges are good, the box is approved for sale, otherwise, it is rejected. Find the probability that a box containing 15 oranges out of which 12 are good and 3 are bad ones will be approved for sale.
		\label{ncert/12/13/2/3/defs.tex}
		\item Two balls are drawn at random with replacement from a box containing 10 black and 8 red balls. Find the probability that
		\label{ncert/12/13/2/12}
\begin{enumerate}
\item both balls are red.
\item first ball is black and second is red.
\item one of them is black and other is red.
\end{enumerate}

\item In a hostel, 60\% of the students read Hindi newspaper, 40\% read English newspaper and 20\% read both Hindi and English newspapers. A student is selected at random.
		\label{ncert/12/13/2/15}
\begin{enumerate}
\item Find the probability that she reads neither Hindi nor English newspapers.
\item If she reads Hindi newspaper, find the probability that she reads English newspaper.
\item If she reads English newspaper, find the probability that she reads Hindi newspaper.\\
\end{enumerate}
\item The probability of obtaining an even prime number on each die, when a pair of dice is rolled is 
\begin{enumerate}
    \item $0$ 
    
    \item $\frac{1}{3}$ 
    
    \item $\frac{1}{12}$ 
    
    \item $\frac{1}{36}$ 
\end{enumerate}
\solution
		%\begin{enumerate}[label=\thesection.\arabic*,ref=\thesection.\theenumi]
	\item One card is drawn from a well-shuffled deck of 52 cards. Find the probability of getting
\begin{enumerate}
\item A king of red colour 
\item A face card 
\item A red face card
\item The jack of hearts
\item A spade
\item The queen of diamonds

\end{enumerate}
\solution
		%\begin{table}[H]
	\centering
\begin{tabular}{|c|c|c|}
\hline
Random variable &Value &Definition\\ \hline
\multirow{3}{*}{X} &0 &Slips of Rs 1\\
&1 &Slips of Rs 5\\
&2 &Slips of Rs 13\\ \hline
\multirow{2}{*}{Y} &0 &Box A\\
&1 &Box B\\\hline
\end{tabular}
\caption{}
\label{tab:Distribution}
\end{table}
See \tabref{tab:Distribution}.
\begin{align}
p_{Y}\brak{k}= \begin{cases} 
      \frac{1}{3} & {k=0} \\
      \frac{2}{3 }& {k=1} 
   \end{cases}
   \\
p_{Y|X}\brak{0|0} = \frac{19}{25}\, 
p_{Y|X}\brak{0|1} = \frac{6}{25}\,
p_{Y|X}\brak{1|0} = \frac{45}{50}\,
p_{Y|X}\brak{1|2} = \frac{5}{50}
\end{align}
The desired probability is the probability that a slip drawn at random is marked other than Rs 1,
\begin{align}
&=1-p_X\brak{0}\\
&= p_X(1) + p_X(2)
\end{align}
Using Bayes theorem,
\begin{align}
&= p_Y\brak{0} \times \pr{Y=0 | X=1} + p_Y\brak{1} \times \pr{Y=1|X=2}\\
&=\frac{1}{3} \times \frac{6}{25} + \frac{2}{3} \times \frac{5}{50}\\
&=\frac{11}{75}
\end{align}

\newpage

%\tableofcontents

\bigskip

\renewcommand{\thefigure}{\theenumi}
\renewcommand{\thetable}{\theenumi}
%\renewcommand{\theequation}{\theenumi}

%\begin{abstract}
%%\boldmath
%In this letter, an algorithm for evaluating the exact analytical bit error rate  (BER)  for the piecewise linear (PL) combiner for  multiple relays is presented. Previous results were available only for upto three relays. The algorithm is unique in the sense that  the actual mathematical expressions, that are prohibitively large, need not be explicitly obtained. The diversity gain due to multiple relays is shown through plots of the analytical BER, well supported by simulations. 
%
%\end{abstract}
% IEEEtran.cls defaults to using nonbold math in the Abstract.
% This preserves the distinction between vectors and scalars. However,
% if the journal you are submitting to favors bold math in the abstract,
% then you can use LaTeX's standard command \boldmath at the very start
% of the abstract to achieve this. Many IEEE journals frown on math
% in the abstract anyway.

% Note that keywords are not normally used for peerreview papers.
%\begin{IEEEkeywords}
%Cooperative diversity, decode and forward, piecewise linear
%\end{IEEEkeywords}



% For peer review papers, you can put extra information on the cover
% page as needed:
% \ifCLASSOPTIONpeerreview
% \begin{center} \bfseries EDICS Category: 3-BBND \end{center}
% \fi
%
% For peerreview papers, this IEEEtran command inserts a page break and
% creates the second title. It will be ignored for other modes.
%\IEEEpeerreviewmaketitle




	\item Five cards—the ten, jack, queen, king and ace of diamonds, are well-shuffled with their face downwards. One card is then picked up at random.
\begin{enumerate}
\item
What is the probability that the card is the queen? 
\item
If the queen is drawn and put aside, what is the probability that the second card picked up is (a) an ace? (b) a queen?\\
\end{enumerate}
\solution
		%\begin{enumerate}[label=\thesection.\arabic*,ref=\thesection.\theenumi]
	\item One card is drawn from a well-shuffled deck of 52 cards. Find the probability of getting
\begin{enumerate}
\item A king of red colour 
\item A face card 
\item A red face card
\item The jack of hearts
\item A spade
\item The queen of diamonds

\end{enumerate}
\solution
		%\input{ncert/10/15/1/14/main.tex}
	\item Five cards—the ten, jack, queen, king and ace of diamonds, are well-shuffled with their face downwards. One card is then picked up at random.
\begin{enumerate}
\item
What is the probability that the card is the queen? 
\item
If the queen is drawn and put aside, what is the probability that the second card picked up is (a) an ace? (b) a queen?\\
\end{enumerate}
\solution
		%\input{ncert/10/15/1/15/defs.tex}
	\item A bag contains $5$ red balls and some blue balls. If the probability of drawing a blue ball is double that if a red ball, determine the number of blue balls in the bag. 
		\\
\solution
		%\input{ncert/10/15/2/3/defs.tex}
	\item A card is selected from a pack of 52 cards.
 \begin{enumerate}[label=(\alph*)] 
                 \item How many points are there in the sample space?
                 \item Calculate the probability that the card is an ace of spades.
                 \item Calculate the probability that the card is (i) an ace and (ii) black card.
 \end{enumerate}
\solution
		%\input{ncert/11/16/3/4/main.tex}
\item Four cards are drawn from a well-shuffled deck of 52 cards. What is the probability of obtaining 3 diamonds and one spade.
\\
\solution
		%\input{ncert/11/16/4/2/defs.tex}
\item In a certain lottery 10,000 tickets are sold and ten equal prizes are awarded. What is the probability of not getting a prize if you buy (a) one ticket (b) two tickets (c) 10 tickets ?	
\\
\solution
		%\input{ncert/11/16/4/4/defs.tex}
		%
\item 
Out of 100 students, two sections of 40 and 60 are formed. If you and your friend are among the 100 students, what is the probability that
\begin{enumerate}
\item you both enter the same section?
\item you both enter the different sections?
\end{enumerate}
\solution
		%\input{ncert/11/16/4/5/defs.tex}
	\item 
The number lock of a suitcase has 4 wheels each labelled with ten digits i.e. from 0 to 9.The lock opens with a sequence of four digits with no repeats.What is the probability of a person getting the right sequence to open the suitcase.
\\
\solution
		%\input{ncert/11/16/4/10/defs.tex}
		%
\item 
Two cards are drawn at random and without replacement from a pack of 52 playing cards. Find the probability that both the cards are black.
\\
\solution
		%\input{ncert/12/13/2/2/defs.tex}
		\item A box of oranges is inspected by examining three randomly selected oranges drawn without replacement. If all the three oranges are good, the box is approved for sale, otherwise, it is rejected. Find the probability that a box containing 15 oranges out of which 12 are good and 3 are bad ones will be approved for sale.
		\label{ncert/12/13/2/3/defs.tex}
		\item Two balls are drawn at random with replacement from a box containing 10 black and 8 red balls. Find the probability that
		\label{ncert/12/13/2/12}
\begin{enumerate}
\item both balls are red.
\item first ball is black and second is red.
\item one of them is black and other is red.
\end{enumerate}

\item In a hostel, 60\% of the students read Hindi newspaper, 40\% read English newspaper and 20\% read both Hindi and English newspapers. A student is selected at random.
		\label{ncert/12/13/2/15}
\begin{enumerate}
\item Find the probability that she reads neither Hindi nor English newspapers.
\item If she reads Hindi newspaper, find the probability that she reads English newspaper.
\item If she reads English newspaper, find the probability that she reads Hindi newspaper.\\
\end{enumerate}
\item The probability of obtaining an even prime number on each die, when a pair of dice is rolled is 
\begin{enumerate}
    \item $0$ 
    
    \item $\frac{1}{3}$ 
    
    \item $\frac{1}{12}$ 
    
    \item $\frac{1}{36}$ 
\end{enumerate}
\solution
		%\input{ncert/12/13/2/17/defs.tex}
	\item A bag contains 4 red and 4 black balls, another bag contains 2 red and 6 black balls. One of the two bags is selected at random and a ball is drawn from the bag which is found to be red. Find the probability that the ball is drawn from the first bag.
\\
\solution
		%\input{ncert/12/13/3/2/main.tex}
  \item
  Cards with numbers 2 to 101 are placed in a box. A card is selected at random.Find the probability that the card has
\begin{enumerate}[label=(\roman*)]
	\item an even number 
	\item a square number
\end{enumerate}
\solution
%\input{exemplar/10/13/3/32/main.tex}
\item
The king, queen and jack of clubs are removed from a deck of 52 playing cards and then well shuffled. Now one card is drawn at random from the remaining cards.  Determine the probability that the card is
\begin{enumerate}[label=(\roman*)]
\item a club
\item 10 of hearts
\end{enumerate}
\solution
%\input{exemplar/10/13/3/29/main.tex}
\item A team of medical students doing their internship have to assist during surgeries
at a city hospital. The probabilities of surgeries rated as very complex, complex,
routine, simple or very simple are respectively, 0.15, 0.20, 0.31, 0.26, .08. Find
the probabilities that a particular surgery will be rated
\begin{enumerate}
	\item complex or very complex;
	\item neither very complex nor very simple;
	\item routine or complex
	\item routine or simple
\end{enumerate}
\solution
%\input{exemplar/11/16/3/8(1)/main.tex}
\item A card is selected from a pack of 52 cards.
\begin{enumerate}[label=(\alph*)]
    \item How many points are there in the sample space?
    \item Calculate the probability that the card is an ace of spades.
    \item Calculate the probability that the card is (i) an ace and (ii) black card.
\end{enumerate}
\solution
%\input{exemplar/11/16/3/4/main2.tex}
\item The probability that a non leap year selected at random will contain 53 sundays.
\\
\solution
%\input{exemplar/10/13/1/19/main.tex}
\item One of the four persons John, Rita, Aslam or Gurpreet will be promoted next
month. Consequently the sample space consists of four elementary outcomes
S = {John promoted, Rita promoted, Aslam promoted, Gurpreet promoted}
You are told that the chances of John’s promotion is same as that of Gurpreet,
Rita’s chances of promotion are twice as likely as Johns. Aslam’s chances are
four times that of John.
\begin{enumerate}
	\item Determine
	\begin{enumerate}
		\item P (John promoted)
		\item P (Rita promoted)
		\item P (Aslam promoted)
		\item P (Gurpreet promoted)
	\end{enumerate}
	\item If A = {John promoted or Gurpreet promoted}, find P (A).
\end{enumerate}
\solution
%\input{exemplar/11/16/3/10/main.tex}
\item A card is drawn from a deck of 52 cards. Find the probability of getting a king or a heart or a red card.\\
\solution
%\input{exemplar/11/16/3/15/main.tex}
\item The probability that a student will pass his examination is 0.73, the probability of
the student getting a compartment is 0.13, and the probability that the student will
either pass or get compartment is 0.96. State True or False.\\
\solution
%\input{exemplar/11/16/3/31/main.tex}
\item A card is selected from a pack of 52 cards\\
\begin{enumerate}[label=(\alph*)]
\item How many points are there in the sample space?
\item Calculate the probability that the cards is an ace of spades.
\item Calculate the probability that the card is (i) an ace (ii)black card.\\
\end{enumerate}
%\input{ncert/11/16/3/4_1/Prob_4.tex}
\item In a non-leap year, the probability of having 53 tuesdays or 53 wednesdays is\\
\solution
%\input{exemplar/11/16/3/18/main.tex}
\item There are 1000 sealed envelopes in a box, 10 of them contain a cash prize of
Rs 100 each, 100 of them contain a cash prize of Rs 50 each and 200 of them
contain a cash prize of Rs 10 each and rest do not contain any cash prize. If they
are well shuffled and an envelope is picked up out, what is the probability that it
contains no cash prize?\\
\solution
%\input{exemplar/10/13/3/34/main.tex}
\item 
A die is thrown and a card is selected at random from a deck of 52 playing cards. The probability of getting an even number on the die and a spade card.\\
\solution
%\input{exemplar/12/13/3/78/main.tex}
\item
If 4-digit numbers greater than 5,000 are randomly formed from the digits 0, 1, 3, 5, and 7, what is the probability of forming a number divisible by 5 when:
\begin{enumerate}
    \item The digits are repeated?
    \item The repetition of digits is not allowed?
\end{enumerate}
\solution
%\input{ncert/11/16/4/9/main.tex}
\item Consider the probability space $\brak{\Omega, \mathcal{G}, P}$ where $\Omega = [0,2]$ and $\mathcal{G} = \cbrak{\phi, \Omega, [0,1], (1,2]}$. Let $X$ and $Y$ be two functions on $\Omega$ defined as
\begin{align*}
    X(\omega) = 
    \begin{cases}
        1 & \text{if }\omega \in [0, 1]\\
        2 & \text{if }\omega \in (1, 2]
    \end{cases}
\end{align*}
and
\begin{align*}
    Y(\omega) = 
    \begin{cases}
        2 & \text{if }\omega \in [0, 1.5]\\
        3 & \text{if }\omega \in (1.5, 2].
    \end{cases}
\end{align*}
Then which one of the following statements is true?
\begin{enumerate}
    \item [(A)] $X$ is a random variable with respect to $\mathcal{G}$, but $Y$ is not a random variable with respect to $\mathcal{G}$.
    \item [(B)] $Y$ is a random variable with respect to $\mathcal{G}$, but $X$ is not a random variable with respect to $\mathcal{G}$.
    \item [(C)] Neither $X$ nor $Y$ is a random variable with respect to $\mathcal{G}$.
    \item [(D)] Both $X$ and $Y$ are random variables with respect to $\mathcal{G}$.
\end{enumerate} \hfill (GATE ST 2023)\\
\solution
%\input{gate/ST/2023/14/main.tex}
	\item  A die is loaded in such a way that each odd number is twice as likely to occur as
each even number. Find $P(G)$, where $G$ is the event that a number greater than
3 occurs on a single roll of the die.
\\
\solution
		%\input{exemplar/11/16/3/5/main.tex}
	\item All the jacks, queens and kings are removed from a deck of 52 playing cards. The remaining cards are well shuffled and then one card is drawn at random. Giving ace a value 1 similar value for other cards, find the probability that the card has a value 
		\begin{enumerate}
			\item 7
			\item greater than 7
			\item less than 7
		\end{enumerate}
		%\input{exemplar/10/13/3/30/main.tex}
  \item A Lot consists of 48 mobile phones of which 42 are good, 3 have only minor defects and 3 have major defects.Varnika will buy a phone if it is good but the trader will only buy a mobile if it has no major defects. One phone is selected at random from the lot. What is the probability that it is
\begin{enumerate}
	\item acceptable to Varnika?
            \item acceptable to the trader?
\end{enumerate}
\solution
	%\input{exemplar/10/13/3/40/main.tex}
 \item A student says that if you throw a die, it will show up 1 or not 1. Therefore, the probability of getting 1 and the probability of getting 'not 1' each is equal to $\frac{1}{2}$. Is this correct? Give reasons.\\
 \solution
        %\input{exemplar/10/13/2/9/main.tex}
   \item Four candidates A, B, C, D have ap-
plied for the assignment to coach a school cricket
team. If A is twice as likely to be selected as B, and
B and C are given about the same chance of being
selected, while C is twice as likely to be selected
as D, what are the probabilities that
\begin{enumerate}
\item C will be selected?
\item A will not be selected?
\end{enumerate}
	%\input{exemplar/11/16/3/9/main.tex}
 \item A bag contain 24 balls of which $x$ balls are red, $2x$ are white and $3x$ are blue. A ball is selected at random, What is the probability that it is
\begin{enumerate}[label=\alph*)]
\item not red ?
\item white ?
\end{enumerate}
%\input{exemplar/10/13/3/41/main.tex}
If the letters of the word ASSASSINATION are arranged at random. Find the Probability that
\begin{enumerate}[label=(\alph*)]
\item Four $S's$ come consecutively in the word
\item Two  $I's$ and two $N's$ come together
\item All $A's$ are not coming together
\item No two $A's$ are coming together
\end{enumerate}
%\input{exemplar/11/16/3/14/main.tex}
	\item One urn contains two black balls (labelled B1 and B2) and one white ball. A
	second urn contains one black ball and two white balls (labelled W1 and W2).
	Suppose the following experiment is performed. One of the two urns is chosen
	at random. Next a ball is randomly chosen from the urn. Then a second ball is
	chosen at random from the same urn without replacing the first ball.
	
	\begin{enumerate}
	\item What is the probability that two black balls are chosen?
	
	\item What is the probability that two balls of opposite colour are chosen?
	\end{enumerate}
	\solution
	%\input{exemplar/11/16/3/12/main1.tex}
\end{enumerate}

	\item A bag contains $5$ red balls and some blue balls. If the probability of drawing a blue ball is double that if a red ball, determine the number of blue balls in the bag. 
		\\
\solution
		%\begin{enumerate}[label=\thesection.\arabic*,ref=\thesection.\theenumi]
	\item One card is drawn from a well-shuffled deck of 52 cards. Find the probability of getting
\begin{enumerate}
\item A king of red colour 
\item A face card 
\item A red face card
\item The jack of hearts
\item A spade
\item The queen of diamonds

\end{enumerate}
\solution
		%\input{ncert/10/15/1/14/main.tex}
	\item Five cards—the ten, jack, queen, king and ace of diamonds, are well-shuffled with their face downwards. One card is then picked up at random.
\begin{enumerate}
\item
What is the probability that the card is the queen? 
\item
If the queen is drawn and put aside, what is the probability that the second card picked up is (a) an ace? (b) a queen?\\
\end{enumerate}
\solution
		%\input{ncert/10/15/1/15/defs.tex}
	\item A bag contains $5$ red balls and some blue balls. If the probability of drawing a blue ball is double that if a red ball, determine the number of blue balls in the bag. 
		\\
\solution
		%\input{ncert/10/15/2/3/defs.tex}
	\item A card is selected from a pack of 52 cards.
 \begin{enumerate}[label=(\alph*)] 
                 \item How many points are there in the sample space?
                 \item Calculate the probability that the card is an ace of spades.
                 \item Calculate the probability that the card is (i) an ace and (ii) black card.
 \end{enumerate}
\solution
		%\input{ncert/11/16/3/4/main.tex}
\item Four cards are drawn from a well-shuffled deck of 52 cards. What is the probability of obtaining 3 diamonds and one spade.
\\
\solution
		%\input{ncert/11/16/4/2/defs.tex}
\item In a certain lottery 10,000 tickets are sold and ten equal prizes are awarded. What is the probability of not getting a prize if you buy (a) one ticket (b) two tickets (c) 10 tickets ?	
\\
\solution
		%\input{ncert/11/16/4/4/defs.tex}
		%
\item 
Out of 100 students, two sections of 40 and 60 are formed. If you and your friend are among the 100 students, what is the probability that
\begin{enumerate}
\item you both enter the same section?
\item you both enter the different sections?
\end{enumerate}
\solution
		%\input{ncert/11/16/4/5/defs.tex}
	\item 
The number lock of a suitcase has 4 wheels each labelled with ten digits i.e. from 0 to 9.The lock opens with a sequence of four digits with no repeats.What is the probability of a person getting the right sequence to open the suitcase.
\\
\solution
		%\input{ncert/11/16/4/10/defs.tex}
		%
\item 
Two cards are drawn at random and without replacement from a pack of 52 playing cards. Find the probability that both the cards are black.
\\
\solution
		%\input{ncert/12/13/2/2/defs.tex}
		\item A box of oranges is inspected by examining three randomly selected oranges drawn without replacement. If all the three oranges are good, the box is approved for sale, otherwise, it is rejected. Find the probability that a box containing 15 oranges out of which 12 are good and 3 are bad ones will be approved for sale.
		\label{ncert/12/13/2/3/defs.tex}
		\item Two balls are drawn at random with replacement from a box containing 10 black and 8 red balls. Find the probability that
		\label{ncert/12/13/2/12}
\begin{enumerate}
\item both balls are red.
\item first ball is black and second is red.
\item one of them is black and other is red.
\end{enumerate}

\item In a hostel, 60\% of the students read Hindi newspaper, 40\% read English newspaper and 20\% read both Hindi and English newspapers. A student is selected at random.
		\label{ncert/12/13/2/15}
\begin{enumerate}
\item Find the probability that she reads neither Hindi nor English newspapers.
\item If she reads Hindi newspaper, find the probability that she reads English newspaper.
\item If she reads English newspaper, find the probability that she reads Hindi newspaper.\\
\end{enumerate}
\item The probability of obtaining an even prime number on each die, when a pair of dice is rolled is 
\begin{enumerate}
    \item $0$ 
    
    \item $\frac{1}{3}$ 
    
    \item $\frac{1}{12}$ 
    
    \item $\frac{1}{36}$ 
\end{enumerate}
\solution
		%\input{ncert/12/13/2/17/defs.tex}
	\item A bag contains 4 red and 4 black balls, another bag contains 2 red and 6 black balls. One of the two bags is selected at random and a ball is drawn from the bag which is found to be red. Find the probability that the ball is drawn from the first bag.
\\
\solution
		%\input{ncert/12/13/3/2/main.tex}
  \item
  Cards with numbers 2 to 101 are placed in a box. A card is selected at random.Find the probability that the card has
\begin{enumerate}[label=(\roman*)]
	\item an even number 
	\item a square number
\end{enumerate}
\solution
%\input{exemplar/10/13/3/32/main.tex}
\item
The king, queen and jack of clubs are removed from a deck of 52 playing cards and then well shuffled. Now one card is drawn at random from the remaining cards.  Determine the probability that the card is
\begin{enumerate}[label=(\roman*)]
\item a club
\item 10 of hearts
\end{enumerate}
\solution
%\input{exemplar/10/13/3/29/main.tex}
\item A team of medical students doing their internship have to assist during surgeries
at a city hospital. The probabilities of surgeries rated as very complex, complex,
routine, simple or very simple are respectively, 0.15, 0.20, 0.31, 0.26, .08. Find
the probabilities that a particular surgery will be rated
\begin{enumerate}
	\item complex or very complex;
	\item neither very complex nor very simple;
	\item routine or complex
	\item routine or simple
\end{enumerate}
\solution
%\input{exemplar/11/16/3/8(1)/main.tex}
\item A card is selected from a pack of 52 cards.
\begin{enumerate}[label=(\alph*)]
    \item How many points are there in the sample space?
    \item Calculate the probability that the card is an ace of spades.
    \item Calculate the probability that the card is (i) an ace and (ii) black card.
\end{enumerate}
\solution
%\input{exemplar/11/16/3/4/main2.tex}
\item The probability that a non leap year selected at random will contain 53 sundays.
\\
\solution
%\input{exemplar/10/13/1/19/main.tex}
\item One of the four persons John, Rita, Aslam or Gurpreet will be promoted next
month. Consequently the sample space consists of four elementary outcomes
S = {John promoted, Rita promoted, Aslam promoted, Gurpreet promoted}
You are told that the chances of John’s promotion is same as that of Gurpreet,
Rita’s chances of promotion are twice as likely as Johns. Aslam’s chances are
four times that of John.
\begin{enumerate}
	\item Determine
	\begin{enumerate}
		\item P (John promoted)
		\item P (Rita promoted)
		\item P (Aslam promoted)
		\item P (Gurpreet promoted)
	\end{enumerate}
	\item If A = {John promoted or Gurpreet promoted}, find P (A).
\end{enumerate}
\solution
%\input{exemplar/11/16/3/10/main.tex}
\item A card is drawn from a deck of 52 cards. Find the probability of getting a king or a heart or a red card.\\
\solution
%\input{exemplar/11/16/3/15/main.tex}
\item The probability that a student will pass his examination is 0.73, the probability of
the student getting a compartment is 0.13, and the probability that the student will
either pass or get compartment is 0.96. State True or False.\\
\solution
%\input{exemplar/11/16/3/31/main.tex}
\item A card is selected from a pack of 52 cards\\
\begin{enumerate}[label=(\alph*)]
\item How many points are there in the sample space?
\item Calculate the probability that the cards is an ace of spades.
\item Calculate the probability that the card is (i) an ace (ii)black card.\\
\end{enumerate}
%\input{ncert/11/16/3/4_1/Prob_4.tex}
\item In a non-leap year, the probability of having 53 tuesdays or 53 wednesdays is\\
\solution
%\input{exemplar/11/16/3/18/main.tex}
\item There are 1000 sealed envelopes in a box, 10 of them contain a cash prize of
Rs 100 each, 100 of them contain a cash prize of Rs 50 each and 200 of them
contain a cash prize of Rs 10 each and rest do not contain any cash prize. If they
are well shuffled and an envelope is picked up out, what is the probability that it
contains no cash prize?\\
\solution
%\input{exemplar/10/13/3/34/main.tex}
\item 
A die is thrown and a card is selected at random from a deck of 52 playing cards. The probability of getting an even number on the die and a spade card.\\
\solution
%\input{exemplar/12/13/3/78/main.tex}
\item
If 4-digit numbers greater than 5,000 are randomly formed from the digits 0, 1, 3, 5, and 7, what is the probability of forming a number divisible by 5 when:
\begin{enumerate}
    \item The digits are repeated?
    \item The repetition of digits is not allowed?
\end{enumerate}
\solution
%\input{ncert/11/16/4/9/main.tex}
\item Consider the probability space $\brak{\Omega, \mathcal{G}, P}$ where $\Omega = [0,2]$ and $\mathcal{G} = \cbrak{\phi, \Omega, [0,1], (1,2]}$. Let $X$ and $Y$ be two functions on $\Omega$ defined as
\begin{align*}
    X(\omega) = 
    \begin{cases}
        1 & \text{if }\omega \in [0, 1]\\
        2 & \text{if }\omega \in (1, 2]
    \end{cases}
\end{align*}
and
\begin{align*}
    Y(\omega) = 
    \begin{cases}
        2 & \text{if }\omega \in [0, 1.5]\\
        3 & \text{if }\omega \in (1.5, 2].
    \end{cases}
\end{align*}
Then which one of the following statements is true?
\begin{enumerate}
    \item [(A)] $X$ is a random variable with respect to $\mathcal{G}$, but $Y$ is not a random variable with respect to $\mathcal{G}$.
    \item [(B)] $Y$ is a random variable with respect to $\mathcal{G}$, but $X$ is not a random variable with respect to $\mathcal{G}$.
    \item [(C)] Neither $X$ nor $Y$ is a random variable with respect to $\mathcal{G}$.
    \item [(D)] Both $X$ and $Y$ are random variables with respect to $\mathcal{G}$.
\end{enumerate} \hfill (GATE ST 2023)\\
\solution
%\input{gate/ST/2023/14/main.tex}
	\item  A die is loaded in such a way that each odd number is twice as likely to occur as
each even number. Find $P(G)$, where $G$ is the event that a number greater than
3 occurs on a single roll of the die.
\\
\solution
		%\input{exemplar/11/16/3/5/main.tex}
	\item All the jacks, queens and kings are removed from a deck of 52 playing cards. The remaining cards are well shuffled and then one card is drawn at random. Giving ace a value 1 similar value for other cards, find the probability that the card has a value 
		\begin{enumerate}
			\item 7
			\item greater than 7
			\item less than 7
		\end{enumerate}
		%\input{exemplar/10/13/3/30/main.tex}
  \item A Lot consists of 48 mobile phones of which 42 are good, 3 have only minor defects and 3 have major defects.Varnika will buy a phone if it is good but the trader will only buy a mobile if it has no major defects. One phone is selected at random from the lot. What is the probability that it is
\begin{enumerate}
	\item acceptable to Varnika?
            \item acceptable to the trader?
\end{enumerate}
\solution
	%\input{exemplar/10/13/3/40/main.tex}
 \item A student says that if you throw a die, it will show up 1 or not 1. Therefore, the probability of getting 1 and the probability of getting 'not 1' each is equal to $\frac{1}{2}$. Is this correct? Give reasons.\\
 \solution
        %\input{exemplar/10/13/2/9/main.tex}
   \item Four candidates A, B, C, D have ap-
plied for the assignment to coach a school cricket
team. If A is twice as likely to be selected as B, and
B and C are given about the same chance of being
selected, while C is twice as likely to be selected
as D, what are the probabilities that
\begin{enumerate}
\item C will be selected?
\item A will not be selected?
\end{enumerate}
	%\input{exemplar/11/16/3/9/main.tex}
 \item A bag contain 24 balls of which $x$ balls are red, $2x$ are white and $3x$ are blue. A ball is selected at random, What is the probability that it is
\begin{enumerate}[label=\alph*)]
\item not red ?
\item white ?
\end{enumerate}
%\input{exemplar/10/13/3/41/main.tex}
If the letters of the word ASSASSINATION are arranged at random. Find the Probability that
\begin{enumerate}[label=(\alph*)]
\item Four $S's$ come consecutively in the word
\item Two  $I's$ and two $N's$ come together
\item All $A's$ are not coming together
\item No two $A's$ are coming together
\end{enumerate}
%\input{exemplar/11/16/3/14/main.tex}
	\item One urn contains two black balls (labelled B1 and B2) and one white ball. A
	second urn contains one black ball and two white balls (labelled W1 and W2).
	Suppose the following experiment is performed. One of the two urns is chosen
	at random. Next a ball is randomly chosen from the urn. Then a second ball is
	chosen at random from the same urn without replacing the first ball.
	
	\begin{enumerate}
	\item What is the probability that two black balls are chosen?
	
	\item What is the probability that two balls of opposite colour are chosen?
	\end{enumerate}
	\solution
	%\input{exemplar/11/16/3/12/main1.tex}
\end{enumerate}

	\item A card is selected from a pack of 52 cards.
 \begin{enumerate}[label=(\alph*)] 
                 \item How many points are there in the sample space?
                 \item Calculate the probability that the card is an ace of spades.
                 \item Calculate the probability that the card is (i) an ace and (ii) black card.
 \end{enumerate}
\solution
		%\begin{table}[H]
	\centering
\begin{tabular}{|c|c|c|}
\hline
Random variable &Value &Definition\\ \hline
\multirow{3}{*}{X} &0 &Slips of Rs 1\\
&1 &Slips of Rs 5\\
&2 &Slips of Rs 13\\ \hline
\multirow{2}{*}{Y} &0 &Box A\\
&1 &Box B\\\hline
\end{tabular}
\caption{}
\label{tab:Distribution}
\end{table}
See \tabref{tab:Distribution}.
\begin{align}
p_{Y}\brak{k}= \begin{cases} 
      \frac{1}{3} & {k=0} \\
      \frac{2}{3 }& {k=1} 
   \end{cases}
   \\
p_{Y|X}\brak{0|0} = \frac{19}{25}\, 
p_{Y|X}\brak{0|1} = \frac{6}{25}\,
p_{Y|X}\brak{1|0} = \frac{45}{50}\,
p_{Y|X}\brak{1|2} = \frac{5}{50}
\end{align}
The desired probability is the probability that a slip drawn at random is marked other than Rs 1,
\begin{align}
&=1-p_X\brak{0}\\
&= p_X(1) + p_X(2)
\end{align}
Using Bayes theorem,
\begin{align}
&= p_Y\brak{0} \times \pr{Y=0 | X=1} + p_Y\brak{1} \times \pr{Y=1|X=2}\\
&=\frac{1}{3} \times \frac{6}{25} + \frac{2}{3} \times \frac{5}{50}\\
&=\frac{11}{75}
\end{align}

\newpage

%\tableofcontents

\bigskip

\renewcommand{\thefigure}{\theenumi}
\renewcommand{\thetable}{\theenumi}
%\renewcommand{\theequation}{\theenumi}

%\begin{abstract}
%%\boldmath
%In this letter, an algorithm for evaluating the exact analytical bit error rate  (BER)  for the piecewise linear (PL) combiner for  multiple relays is presented. Previous results were available only for upto three relays. The algorithm is unique in the sense that  the actual mathematical expressions, that are prohibitively large, need not be explicitly obtained. The diversity gain due to multiple relays is shown through plots of the analytical BER, well supported by simulations. 
%
%\end{abstract}
% IEEEtran.cls defaults to using nonbold math in the Abstract.
% This preserves the distinction between vectors and scalars. However,
% if the journal you are submitting to favors bold math in the abstract,
% then you can use LaTeX's standard command \boldmath at the very start
% of the abstract to achieve this. Many IEEE journals frown on math
% in the abstract anyway.

% Note that keywords are not normally used for peerreview papers.
%\begin{IEEEkeywords}
%Cooperative diversity, decode and forward, piecewise linear
%\end{IEEEkeywords}



% For peer review papers, you can put extra information on the cover
% page as needed:
% \ifCLASSOPTIONpeerreview
% \begin{center} \bfseries EDICS Category: 3-BBND \end{center}
% \fi
%
% For peerreview papers, this IEEEtran command inserts a page break and
% creates the second title. It will be ignored for other modes.
%\IEEEpeerreviewmaketitle




\item Four cards are drawn from a well-shuffled deck of 52 cards. What is the probability of obtaining 3 diamonds and one spade.
\\
\solution
		%\begin{enumerate}[label=\thesection.\arabic*,ref=\thesection.\theenumi]
	\item One card is drawn from a well-shuffled deck of 52 cards. Find the probability of getting
\begin{enumerate}
\item A king of red colour 
\item A face card 
\item A red face card
\item The jack of hearts
\item A spade
\item The queen of diamonds

\end{enumerate}
\solution
		%\input{ncert/10/15/1/14/main.tex}
	\item Five cards—the ten, jack, queen, king and ace of diamonds, are well-shuffled with their face downwards. One card is then picked up at random.
\begin{enumerate}
\item
What is the probability that the card is the queen? 
\item
If the queen is drawn and put aside, what is the probability that the second card picked up is (a) an ace? (b) a queen?\\
\end{enumerate}
\solution
		%\input{ncert/10/15/1/15/defs.tex}
	\item A bag contains $5$ red balls and some blue balls. If the probability of drawing a blue ball is double that if a red ball, determine the number of blue balls in the bag. 
		\\
\solution
		%\input{ncert/10/15/2/3/defs.tex}
	\item A card is selected from a pack of 52 cards.
 \begin{enumerate}[label=(\alph*)] 
                 \item How many points are there in the sample space?
                 \item Calculate the probability that the card is an ace of spades.
                 \item Calculate the probability that the card is (i) an ace and (ii) black card.
 \end{enumerate}
\solution
		%\input{ncert/11/16/3/4/main.tex}
\item Four cards are drawn from a well-shuffled deck of 52 cards. What is the probability of obtaining 3 diamonds and one spade.
\\
\solution
		%\input{ncert/11/16/4/2/defs.tex}
\item In a certain lottery 10,000 tickets are sold and ten equal prizes are awarded. What is the probability of not getting a prize if you buy (a) one ticket (b) two tickets (c) 10 tickets ?	
\\
\solution
		%\input{ncert/11/16/4/4/defs.tex}
		%
\item 
Out of 100 students, two sections of 40 and 60 are formed. If you and your friend are among the 100 students, what is the probability that
\begin{enumerate}
\item you both enter the same section?
\item you both enter the different sections?
\end{enumerate}
\solution
		%\input{ncert/11/16/4/5/defs.tex}
	\item 
The number lock of a suitcase has 4 wheels each labelled with ten digits i.e. from 0 to 9.The lock opens with a sequence of four digits with no repeats.What is the probability of a person getting the right sequence to open the suitcase.
\\
\solution
		%\input{ncert/11/16/4/10/defs.tex}
		%
\item 
Two cards are drawn at random and without replacement from a pack of 52 playing cards. Find the probability that both the cards are black.
\\
\solution
		%\input{ncert/12/13/2/2/defs.tex}
		\item A box of oranges is inspected by examining three randomly selected oranges drawn without replacement. If all the three oranges are good, the box is approved for sale, otherwise, it is rejected. Find the probability that a box containing 15 oranges out of which 12 are good and 3 are bad ones will be approved for sale.
		\label{ncert/12/13/2/3/defs.tex}
		\item Two balls are drawn at random with replacement from a box containing 10 black and 8 red balls. Find the probability that
		\label{ncert/12/13/2/12}
\begin{enumerate}
\item both balls are red.
\item first ball is black and second is red.
\item one of them is black and other is red.
\end{enumerate}

\item In a hostel, 60\% of the students read Hindi newspaper, 40\% read English newspaper and 20\% read both Hindi and English newspapers. A student is selected at random.
		\label{ncert/12/13/2/15}
\begin{enumerate}
\item Find the probability that she reads neither Hindi nor English newspapers.
\item If she reads Hindi newspaper, find the probability that she reads English newspaper.
\item If she reads English newspaper, find the probability that she reads Hindi newspaper.\\
\end{enumerate}
\item The probability of obtaining an even prime number on each die, when a pair of dice is rolled is 
\begin{enumerate}
    \item $0$ 
    
    \item $\frac{1}{3}$ 
    
    \item $\frac{1}{12}$ 
    
    \item $\frac{1}{36}$ 
\end{enumerate}
\solution
		%\input{ncert/12/13/2/17/defs.tex}
	\item A bag contains 4 red and 4 black balls, another bag contains 2 red and 6 black balls. One of the two bags is selected at random and a ball is drawn from the bag which is found to be red. Find the probability that the ball is drawn from the first bag.
\\
\solution
		%\input{ncert/12/13/3/2/main.tex}
  \item
  Cards with numbers 2 to 101 are placed in a box. A card is selected at random.Find the probability that the card has
\begin{enumerate}[label=(\roman*)]
	\item an even number 
	\item a square number
\end{enumerate}
\solution
%\input{exemplar/10/13/3/32/main.tex}
\item
The king, queen and jack of clubs are removed from a deck of 52 playing cards and then well shuffled. Now one card is drawn at random from the remaining cards.  Determine the probability that the card is
\begin{enumerate}[label=(\roman*)]
\item a club
\item 10 of hearts
\end{enumerate}
\solution
%\input{exemplar/10/13/3/29/main.tex}
\item A team of medical students doing their internship have to assist during surgeries
at a city hospital. The probabilities of surgeries rated as very complex, complex,
routine, simple or very simple are respectively, 0.15, 0.20, 0.31, 0.26, .08. Find
the probabilities that a particular surgery will be rated
\begin{enumerate}
	\item complex or very complex;
	\item neither very complex nor very simple;
	\item routine or complex
	\item routine or simple
\end{enumerate}
\solution
%\input{exemplar/11/16/3/8(1)/main.tex}
\item A card is selected from a pack of 52 cards.
\begin{enumerate}[label=(\alph*)]
    \item How many points are there in the sample space?
    \item Calculate the probability that the card is an ace of spades.
    \item Calculate the probability that the card is (i) an ace and (ii) black card.
\end{enumerate}
\solution
%\input{exemplar/11/16/3/4/main2.tex}
\item The probability that a non leap year selected at random will contain 53 sundays.
\\
\solution
%\input{exemplar/10/13/1/19/main.tex}
\item One of the four persons John, Rita, Aslam or Gurpreet will be promoted next
month. Consequently the sample space consists of four elementary outcomes
S = {John promoted, Rita promoted, Aslam promoted, Gurpreet promoted}
You are told that the chances of John’s promotion is same as that of Gurpreet,
Rita’s chances of promotion are twice as likely as Johns. Aslam’s chances are
four times that of John.
\begin{enumerate}
	\item Determine
	\begin{enumerate}
		\item P (John promoted)
		\item P (Rita promoted)
		\item P (Aslam promoted)
		\item P (Gurpreet promoted)
	\end{enumerate}
	\item If A = {John promoted or Gurpreet promoted}, find P (A).
\end{enumerate}
\solution
%\input{exemplar/11/16/3/10/main.tex}
\item A card is drawn from a deck of 52 cards. Find the probability of getting a king or a heart or a red card.\\
\solution
%\input{exemplar/11/16/3/15/main.tex}
\item The probability that a student will pass his examination is 0.73, the probability of
the student getting a compartment is 0.13, and the probability that the student will
either pass or get compartment is 0.96. State True or False.\\
\solution
%\input{exemplar/11/16/3/31/main.tex}
\item A card is selected from a pack of 52 cards\\
\begin{enumerate}[label=(\alph*)]
\item How many points are there in the sample space?
\item Calculate the probability that the cards is an ace of spades.
\item Calculate the probability that the card is (i) an ace (ii)black card.\\
\end{enumerate}
%\input{ncert/11/16/3/4_1/Prob_4.tex}
\item In a non-leap year, the probability of having 53 tuesdays or 53 wednesdays is\\
\solution
%\input{exemplar/11/16/3/18/main.tex}
\item There are 1000 sealed envelopes in a box, 10 of them contain a cash prize of
Rs 100 each, 100 of them contain a cash prize of Rs 50 each and 200 of them
contain a cash prize of Rs 10 each and rest do not contain any cash prize. If they
are well shuffled and an envelope is picked up out, what is the probability that it
contains no cash prize?\\
\solution
%\input{exemplar/10/13/3/34/main.tex}
\item 
A die is thrown and a card is selected at random from a deck of 52 playing cards. The probability of getting an even number on the die and a spade card.\\
\solution
%\input{exemplar/12/13/3/78/main.tex}
\item
If 4-digit numbers greater than 5,000 are randomly formed from the digits 0, 1, 3, 5, and 7, what is the probability of forming a number divisible by 5 when:
\begin{enumerate}
    \item The digits are repeated?
    \item The repetition of digits is not allowed?
\end{enumerate}
\solution
%\input{ncert/11/16/4/9/main.tex}
\item Consider the probability space $\brak{\Omega, \mathcal{G}, P}$ where $\Omega = [0,2]$ and $\mathcal{G} = \cbrak{\phi, \Omega, [0,1], (1,2]}$. Let $X$ and $Y$ be two functions on $\Omega$ defined as
\begin{align*}
    X(\omega) = 
    \begin{cases}
        1 & \text{if }\omega \in [0, 1]\\
        2 & \text{if }\omega \in (1, 2]
    \end{cases}
\end{align*}
and
\begin{align*}
    Y(\omega) = 
    \begin{cases}
        2 & \text{if }\omega \in [0, 1.5]\\
        3 & \text{if }\omega \in (1.5, 2].
    \end{cases}
\end{align*}
Then which one of the following statements is true?
\begin{enumerate}
    \item [(A)] $X$ is a random variable with respect to $\mathcal{G}$, but $Y$ is not a random variable with respect to $\mathcal{G}$.
    \item [(B)] $Y$ is a random variable with respect to $\mathcal{G}$, but $X$ is not a random variable with respect to $\mathcal{G}$.
    \item [(C)] Neither $X$ nor $Y$ is a random variable with respect to $\mathcal{G}$.
    \item [(D)] Both $X$ and $Y$ are random variables with respect to $\mathcal{G}$.
\end{enumerate} \hfill (GATE ST 2023)\\
\solution
%\input{gate/ST/2023/14/main.tex}
	\item  A die is loaded in such a way that each odd number is twice as likely to occur as
each even number. Find $P(G)$, where $G$ is the event that a number greater than
3 occurs on a single roll of the die.
\\
\solution
		%\input{exemplar/11/16/3/5/main.tex}
	\item All the jacks, queens and kings are removed from a deck of 52 playing cards. The remaining cards are well shuffled and then one card is drawn at random. Giving ace a value 1 similar value for other cards, find the probability that the card has a value 
		\begin{enumerate}
			\item 7
			\item greater than 7
			\item less than 7
		\end{enumerate}
		%\input{exemplar/10/13/3/30/main.tex}
  \item A Lot consists of 48 mobile phones of which 42 are good, 3 have only minor defects and 3 have major defects.Varnika will buy a phone if it is good but the trader will only buy a mobile if it has no major defects. One phone is selected at random from the lot. What is the probability that it is
\begin{enumerate}
	\item acceptable to Varnika?
            \item acceptable to the trader?
\end{enumerate}
\solution
	%\input{exemplar/10/13/3/40/main.tex}
 \item A student says that if you throw a die, it will show up 1 or not 1. Therefore, the probability of getting 1 and the probability of getting 'not 1' each is equal to $\frac{1}{2}$. Is this correct? Give reasons.\\
 \solution
        %\input{exemplar/10/13/2/9/main.tex}
   \item Four candidates A, B, C, D have ap-
plied for the assignment to coach a school cricket
team. If A is twice as likely to be selected as B, and
B and C are given about the same chance of being
selected, while C is twice as likely to be selected
as D, what are the probabilities that
\begin{enumerate}
\item C will be selected?
\item A will not be selected?
\end{enumerate}
	%\input{exemplar/11/16/3/9/main.tex}
 \item A bag contain 24 balls of which $x$ balls are red, $2x$ are white and $3x$ are blue. A ball is selected at random, What is the probability that it is
\begin{enumerate}[label=\alph*)]
\item not red ?
\item white ?
\end{enumerate}
%\input{exemplar/10/13/3/41/main.tex}
If the letters of the word ASSASSINATION are arranged at random. Find the Probability that
\begin{enumerate}[label=(\alph*)]
\item Four $S's$ come consecutively in the word
\item Two  $I's$ and two $N's$ come together
\item All $A's$ are not coming together
\item No two $A's$ are coming together
\end{enumerate}
%\input{exemplar/11/16/3/14/main.tex}
	\item One urn contains two black balls (labelled B1 and B2) and one white ball. A
	second urn contains one black ball and two white balls (labelled W1 and W2).
	Suppose the following experiment is performed. One of the two urns is chosen
	at random. Next a ball is randomly chosen from the urn. Then a second ball is
	chosen at random from the same urn without replacing the first ball.
	
	\begin{enumerate}
	\item What is the probability that two black balls are chosen?
	
	\item What is the probability that two balls of opposite colour are chosen?
	\end{enumerate}
	\solution
	%\input{exemplar/11/16/3/12/main1.tex}
\end{enumerate}

\item In a certain lottery 10,000 tickets are sold and ten equal prizes are awarded. What is the probability of not getting a prize if you buy (a) one ticket (b) two tickets (c) 10 tickets ?	
\\
\solution
		%\begin{enumerate}[label=\thesection.\arabic*,ref=\thesection.\theenumi]
	\item One card is drawn from a well-shuffled deck of 52 cards. Find the probability of getting
\begin{enumerate}
\item A king of red colour 
\item A face card 
\item A red face card
\item The jack of hearts
\item A spade
\item The queen of diamonds

\end{enumerate}
\solution
		%\input{ncert/10/15/1/14/main.tex}
	\item Five cards—the ten, jack, queen, king and ace of diamonds, are well-shuffled with their face downwards. One card is then picked up at random.
\begin{enumerate}
\item
What is the probability that the card is the queen? 
\item
If the queen is drawn and put aside, what is the probability that the second card picked up is (a) an ace? (b) a queen?\\
\end{enumerate}
\solution
		%\input{ncert/10/15/1/15/defs.tex}
	\item A bag contains $5$ red balls and some blue balls. If the probability of drawing a blue ball is double that if a red ball, determine the number of blue balls in the bag. 
		\\
\solution
		%\input{ncert/10/15/2/3/defs.tex}
	\item A card is selected from a pack of 52 cards.
 \begin{enumerate}[label=(\alph*)] 
                 \item How many points are there in the sample space?
                 \item Calculate the probability that the card is an ace of spades.
                 \item Calculate the probability that the card is (i) an ace and (ii) black card.
 \end{enumerate}
\solution
		%\input{ncert/11/16/3/4/main.tex}
\item Four cards are drawn from a well-shuffled deck of 52 cards. What is the probability of obtaining 3 diamonds and one spade.
\\
\solution
		%\input{ncert/11/16/4/2/defs.tex}
\item In a certain lottery 10,000 tickets are sold and ten equal prizes are awarded. What is the probability of not getting a prize if you buy (a) one ticket (b) two tickets (c) 10 tickets ?	
\\
\solution
		%\input{ncert/11/16/4/4/defs.tex}
		%
\item 
Out of 100 students, two sections of 40 and 60 are formed. If you and your friend are among the 100 students, what is the probability that
\begin{enumerate}
\item you both enter the same section?
\item you both enter the different sections?
\end{enumerate}
\solution
		%\input{ncert/11/16/4/5/defs.tex}
	\item 
The number lock of a suitcase has 4 wheels each labelled with ten digits i.e. from 0 to 9.The lock opens with a sequence of four digits with no repeats.What is the probability of a person getting the right sequence to open the suitcase.
\\
\solution
		%\input{ncert/11/16/4/10/defs.tex}
		%
\item 
Two cards are drawn at random and without replacement from a pack of 52 playing cards. Find the probability that both the cards are black.
\\
\solution
		%\input{ncert/12/13/2/2/defs.tex}
		\item A box of oranges is inspected by examining three randomly selected oranges drawn without replacement. If all the three oranges are good, the box is approved for sale, otherwise, it is rejected. Find the probability that a box containing 15 oranges out of which 12 are good and 3 are bad ones will be approved for sale.
		\label{ncert/12/13/2/3/defs.tex}
		\item Two balls are drawn at random with replacement from a box containing 10 black and 8 red balls. Find the probability that
		\label{ncert/12/13/2/12}
\begin{enumerate}
\item both balls are red.
\item first ball is black and second is red.
\item one of them is black and other is red.
\end{enumerate}

\item In a hostel, 60\% of the students read Hindi newspaper, 40\% read English newspaper and 20\% read both Hindi and English newspapers. A student is selected at random.
		\label{ncert/12/13/2/15}
\begin{enumerate}
\item Find the probability that she reads neither Hindi nor English newspapers.
\item If she reads Hindi newspaper, find the probability that she reads English newspaper.
\item If she reads English newspaper, find the probability that she reads Hindi newspaper.\\
\end{enumerate}
\item The probability of obtaining an even prime number on each die, when a pair of dice is rolled is 
\begin{enumerate}
    \item $0$ 
    
    \item $\frac{1}{3}$ 
    
    \item $\frac{1}{12}$ 
    
    \item $\frac{1}{36}$ 
\end{enumerate}
\solution
		%\input{ncert/12/13/2/17/defs.tex}
	\item A bag contains 4 red and 4 black balls, another bag contains 2 red and 6 black balls. One of the two bags is selected at random and a ball is drawn from the bag which is found to be red. Find the probability that the ball is drawn from the first bag.
\\
\solution
		%\input{ncert/12/13/3/2/main.tex}
  \item
  Cards with numbers 2 to 101 are placed in a box. A card is selected at random.Find the probability that the card has
\begin{enumerate}[label=(\roman*)]
	\item an even number 
	\item a square number
\end{enumerate}
\solution
%\input{exemplar/10/13/3/32/main.tex}
\item
The king, queen and jack of clubs are removed from a deck of 52 playing cards and then well shuffled. Now one card is drawn at random from the remaining cards.  Determine the probability that the card is
\begin{enumerate}[label=(\roman*)]
\item a club
\item 10 of hearts
\end{enumerate}
\solution
%\input{exemplar/10/13/3/29/main.tex}
\item A team of medical students doing their internship have to assist during surgeries
at a city hospital. The probabilities of surgeries rated as very complex, complex,
routine, simple or very simple are respectively, 0.15, 0.20, 0.31, 0.26, .08. Find
the probabilities that a particular surgery will be rated
\begin{enumerate}
	\item complex or very complex;
	\item neither very complex nor very simple;
	\item routine or complex
	\item routine or simple
\end{enumerate}
\solution
%\input{exemplar/11/16/3/8(1)/main.tex}
\item A card is selected from a pack of 52 cards.
\begin{enumerate}[label=(\alph*)]
    \item How many points are there in the sample space?
    \item Calculate the probability that the card is an ace of spades.
    \item Calculate the probability that the card is (i) an ace and (ii) black card.
\end{enumerate}
\solution
%\input{exemplar/11/16/3/4/main2.tex}
\item The probability that a non leap year selected at random will contain 53 sundays.
\\
\solution
%\input{exemplar/10/13/1/19/main.tex}
\item One of the four persons John, Rita, Aslam or Gurpreet will be promoted next
month. Consequently the sample space consists of four elementary outcomes
S = {John promoted, Rita promoted, Aslam promoted, Gurpreet promoted}
You are told that the chances of John’s promotion is same as that of Gurpreet,
Rita’s chances of promotion are twice as likely as Johns. Aslam’s chances are
four times that of John.
\begin{enumerate}
	\item Determine
	\begin{enumerate}
		\item P (John promoted)
		\item P (Rita promoted)
		\item P (Aslam promoted)
		\item P (Gurpreet promoted)
	\end{enumerate}
	\item If A = {John promoted or Gurpreet promoted}, find P (A).
\end{enumerate}
\solution
%\input{exemplar/11/16/3/10/main.tex}
\item A card is drawn from a deck of 52 cards. Find the probability of getting a king or a heart or a red card.\\
\solution
%\input{exemplar/11/16/3/15/main.tex}
\item The probability that a student will pass his examination is 0.73, the probability of
the student getting a compartment is 0.13, and the probability that the student will
either pass or get compartment is 0.96. State True or False.\\
\solution
%\input{exemplar/11/16/3/31/main.tex}
\item A card is selected from a pack of 52 cards\\
\begin{enumerate}[label=(\alph*)]
\item How many points are there in the sample space?
\item Calculate the probability that the cards is an ace of spades.
\item Calculate the probability that the card is (i) an ace (ii)black card.\\
\end{enumerate}
%\input{ncert/11/16/3/4_1/Prob_4.tex}
\item In a non-leap year, the probability of having 53 tuesdays or 53 wednesdays is\\
\solution
%\input{exemplar/11/16/3/18/main.tex}
\item There are 1000 sealed envelopes in a box, 10 of them contain a cash prize of
Rs 100 each, 100 of them contain a cash prize of Rs 50 each and 200 of them
contain a cash prize of Rs 10 each and rest do not contain any cash prize. If they
are well shuffled and an envelope is picked up out, what is the probability that it
contains no cash prize?\\
\solution
%\input{exemplar/10/13/3/34/main.tex}
\item 
A die is thrown and a card is selected at random from a deck of 52 playing cards. The probability of getting an even number on the die and a spade card.\\
\solution
%\input{exemplar/12/13/3/78/main.tex}
\item
If 4-digit numbers greater than 5,000 are randomly formed from the digits 0, 1, 3, 5, and 7, what is the probability of forming a number divisible by 5 when:
\begin{enumerate}
    \item The digits are repeated?
    \item The repetition of digits is not allowed?
\end{enumerate}
\solution
%\input{ncert/11/16/4/9/main.tex}
\item Consider the probability space $\brak{\Omega, \mathcal{G}, P}$ where $\Omega = [0,2]$ and $\mathcal{G} = \cbrak{\phi, \Omega, [0,1], (1,2]}$. Let $X$ and $Y$ be two functions on $\Omega$ defined as
\begin{align*}
    X(\omega) = 
    \begin{cases}
        1 & \text{if }\omega \in [0, 1]\\
        2 & \text{if }\omega \in (1, 2]
    \end{cases}
\end{align*}
and
\begin{align*}
    Y(\omega) = 
    \begin{cases}
        2 & \text{if }\omega \in [0, 1.5]\\
        3 & \text{if }\omega \in (1.5, 2].
    \end{cases}
\end{align*}
Then which one of the following statements is true?
\begin{enumerate}
    \item [(A)] $X$ is a random variable with respect to $\mathcal{G}$, but $Y$ is not a random variable with respect to $\mathcal{G}$.
    \item [(B)] $Y$ is a random variable with respect to $\mathcal{G}$, but $X$ is not a random variable with respect to $\mathcal{G}$.
    \item [(C)] Neither $X$ nor $Y$ is a random variable with respect to $\mathcal{G}$.
    \item [(D)] Both $X$ and $Y$ are random variables with respect to $\mathcal{G}$.
\end{enumerate} \hfill (GATE ST 2023)\\
\solution
%\input{gate/ST/2023/14/main.tex}
	\item  A die is loaded in such a way that each odd number is twice as likely to occur as
each even number. Find $P(G)$, where $G$ is the event that a number greater than
3 occurs on a single roll of the die.
\\
\solution
		%\input{exemplar/11/16/3/5/main.tex}
	\item All the jacks, queens and kings are removed from a deck of 52 playing cards. The remaining cards are well shuffled and then one card is drawn at random. Giving ace a value 1 similar value for other cards, find the probability that the card has a value 
		\begin{enumerate}
			\item 7
			\item greater than 7
			\item less than 7
		\end{enumerate}
		%\input{exemplar/10/13/3/30/main.tex}
  \item A Lot consists of 48 mobile phones of which 42 are good, 3 have only minor defects and 3 have major defects.Varnika will buy a phone if it is good but the trader will only buy a mobile if it has no major defects. One phone is selected at random from the lot. What is the probability that it is
\begin{enumerate}
	\item acceptable to Varnika?
            \item acceptable to the trader?
\end{enumerate}
\solution
	%\input{exemplar/10/13/3/40/main.tex}
 \item A student says that if you throw a die, it will show up 1 or not 1. Therefore, the probability of getting 1 and the probability of getting 'not 1' each is equal to $\frac{1}{2}$. Is this correct? Give reasons.\\
 \solution
        %\input{exemplar/10/13/2/9/main.tex}
   \item Four candidates A, B, C, D have ap-
plied for the assignment to coach a school cricket
team. If A is twice as likely to be selected as B, and
B and C are given about the same chance of being
selected, while C is twice as likely to be selected
as D, what are the probabilities that
\begin{enumerate}
\item C will be selected?
\item A will not be selected?
\end{enumerate}
	%\input{exemplar/11/16/3/9/main.tex}
 \item A bag contain 24 balls of which $x$ balls are red, $2x$ are white and $3x$ are blue. A ball is selected at random, What is the probability that it is
\begin{enumerate}[label=\alph*)]
\item not red ?
\item white ?
\end{enumerate}
%\input{exemplar/10/13/3/41/main.tex}
If the letters of the word ASSASSINATION are arranged at random. Find the Probability that
\begin{enumerate}[label=(\alph*)]
\item Four $S's$ come consecutively in the word
\item Two  $I's$ and two $N's$ come together
\item All $A's$ are not coming together
\item No two $A's$ are coming together
\end{enumerate}
%\input{exemplar/11/16/3/14/main.tex}
	\item One urn contains two black balls (labelled B1 and B2) and one white ball. A
	second urn contains one black ball and two white balls (labelled W1 and W2).
	Suppose the following experiment is performed. One of the two urns is chosen
	at random. Next a ball is randomly chosen from the urn. Then a second ball is
	chosen at random from the same urn without replacing the first ball.
	
	\begin{enumerate}
	\item What is the probability that two black balls are chosen?
	
	\item What is the probability that two balls of opposite colour are chosen?
	\end{enumerate}
	\solution
	%\input{exemplar/11/16/3/12/main1.tex}
\end{enumerate}

		%
\item 
Out of 100 students, two sections of 40 and 60 are formed. If you and your friend are among the 100 students, what is the probability that
\begin{enumerate}
\item you both enter the same section?
\item you both enter the different sections?
\end{enumerate}
\solution
		%\begin{enumerate}[label=\thesection.\arabic*,ref=\thesection.\theenumi]
	\item One card is drawn from a well-shuffled deck of 52 cards. Find the probability of getting
\begin{enumerate}
\item A king of red colour 
\item A face card 
\item A red face card
\item The jack of hearts
\item A spade
\item The queen of diamonds

\end{enumerate}
\solution
		%\input{ncert/10/15/1/14/main.tex}
	\item Five cards—the ten, jack, queen, king and ace of diamonds, are well-shuffled with their face downwards. One card is then picked up at random.
\begin{enumerate}
\item
What is the probability that the card is the queen? 
\item
If the queen is drawn and put aside, what is the probability that the second card picked up is (a) an ace? (b) a queen?\\
\end{enumerate}
\solution
		%\input{ncert/10/15/1/15/defs.tex}
	\item A bag contains $5$ red balls and some blue balls. If the probability of drawing a blue ball is double that if a red ball, determine the number of blue balls in the bag. 
		\\
\solution
		%\input{ncert/10/15/2/3/defs.tex}
	\item A card is selected from a pack of 52 cards.
 \begin{enumerate}[label=(\alph*)] 
                 \item How many points are there in the sample space?
                 \item Calculate the probability that the card is an ace of spades.
                 \item Calculate the probability that the card is (i) an ace and (ii) black card.
 \end{enumerate}
\solution
		%\input{ncert/11/16/3/4/main.tex}
\item Four cards are drawn from a well-shuffled deck of 52 cards. What is the probability of obtaining 3 diamonds and one spade.
\\
\solution
		%\input{ncert/11/16/4/2/defs.tex}
\item In a certain lottery 10,000 tickets are sold and ten equal prizes are awarded. What is the probability of not getting a prize if you buy (a) one ticket (b) two tickets (c) 10 tickets ?	
\\
\solution
		%\input{ncert/11/16/4/4/defs.tex}
		%
\item 
Out of 100 students, two sections of 40 and 60 are formed. If you and your friend are among the 100 students, what is the probability that
\begin{enumerate}
\item you both enter the same section?
\item you both enter the different sections?
\end{enumerate}
\solution
		%\input{ncert/11/16/4/5/defs.tex}
	\item 
The number lock of a suitcase has 4 wheels each labelled with ten digits i.e. from 0 to 9.The lock opens with a sequence of four digits with no repeats.What is the probability of a person getting the right sequence to open the suitcase.
\\
\solution
		%\input{ncert/11/16/4/10/defs.tex}
		%
\item 
Two cards are drawn at random and without replacement from a pack of 52 playing cards. Find the probability that both the cards are black.
\\
\solution
		%\input{ncert/12/13/2/2/defs.tex}
		\item A box of oranges is inspected by examining three randomly selected oranges drawn without replacement. If all the three oranges are good, the box is approved for sale, otherwise, it is rejected. Find the probability that a box containing 15 oranges out of which 12 are good and 3 are bad ones will be approved for sale.
		\label{ncert/12/13/2/3/defs.tex}
		\item Two balls are drawn at random with replacement from a box containing 10 black and 8 red balls. Find the probability that
		\label{ncert/12/13/2/12}
\begin{enumerate}
\item both balls are red.
\item first ball is black and second is red.
\item one of them is black and other is red.
\end{enumerate}

\item In a hostel, 60\% of the students read Hindi newspaper, 40\% read English newspaper and 20\% read both Hindi and English newspapers. A student is selected at random.
		\label{ncert/12/13/2/15}
\begin{enumerate}
\item Find the probability that she reads neither Hindi nor English newspapers.
\item If she reads Hindi newspaper, find the probability that she reads English newspaper.
\item If she reads English newspaper, find the probability that she reads Hindi newspaper.\\
\end{enumerate}
\item The probability of obtaining an even prime number on each die, when a pair of dice is rolled is 
\begin{enumerate}
    \item $0$ 
    
    \item $\frac{1}{3}$ 
    
    \item $\frac{1}{12}$ 
    
    \item $\frac{1}{36}$ 
\end{enumerate}
\solution
		%\input{ncert/12/13/2/17/defs.tex}
	\item A bag contains 4 red and 4 black balls, another bag contains 2 red and 6 black balls. One of the two bags is selected at random and a ball is drawn from the bag which is found to be red. Find the probability that the ball is drawn from the first bag.
\\
\solution
		%\input{ncert/12/13/3/2/main.tex}
  \item
  Cards with numbers 2 to 101 are placed in a box. A card is selected at random.Find the probability that the card has
\begin{enumerate}[label=(\roman*)]
	\item an even number 
	\item a square number
\end{enumerate}
\solution
%\input{exemplar/10/13/3/32/main.tex}
\item
The king, queen and jack of clubs are removed from a deck of 52 playing cards and then well shuffled. Now one card is drawn at random from the remaining cards.  Determine the probability that the card is
\begin{enumerate}[label=(\roman*)]
\item a club
\item 10 of hearts
\end{enumerate}
\solution
%\input{exemplar/10/13/3/29/main.tex}
\item A team of medical students doing their internship have to assist during surgeries
at a city hospital. The probabilities of surgeries rated as very complex, complex,
routine, simple or very simple are respectively, 0.15, 0.20, 0.31, 0.26, .08. Find
the probabilities that a particular surgery will be rated
\begin{enumerate}
	\item complex or very complex;
	\item neither very complex nor very simple;
	\item routine or complex
	\item routine or simple
\end{enumerate}
\solution
%\input{exemplar/11/16/3/8(1)/main.tex}
\item A card is selected from a pack of 52 cards.
\begin{enumerate}[label=(\alph*)]
    \item How many points are there in the sample space?
    \item Calculate the probability that the card is an ace of spades.
    \item Calculate the probability that the card is (i) an ace and (ii) black card.
\end{enumerate}
\solution
%\input{exemplar/11/16/3/4/main2.tex}
\item The probability that a non leap year selected at random will contain 53 sundays.
\\
\solution
%\input{exemplar/10/13/1/19/main.tex}
\item One of the four persons John, Rita, Aslam or Gurpreet will be promoted next
month. Consequently the sample space consists of four elementary outcomes
S = {John promoted, Rita promoted, Aslam promoted, Gurpreet promoted}
You are told that the chances of John’s promotion is same as that of Gurpreet,
Rita’s chances of promotion are twice as likely as Johns. Aslam’s chances are
four times that of John.
\begin{enumerate}
	\item Determine
	\begin{enumerate}
		\item P (John promoted)
		\item P (Rita promoted)
		\item P (Aslam promoted)
		\item P (Gurpreet promoted)
	\end{enumerate}
	\item If A = {John promoted or Gurpreet promoted}, find P (A).
\end{enumerate}
\solution
%\input{exemplar/11/16/3/10/main.tex}
\item A card is drawn from a deck of 52 cards. Find the probability of getting a king or a heart or a red card.\\
\solution
%\input{exemplar/11/16/3/15/main.tex}
\item The probability that a student will pass his examination is 0.73, the probability of
the student getting a compartment is 0.13, and the probability that the student will
either pass or get compartment is 0.96. State True or False.\\
\solution
%\input{exemplar/11/16/3/31/main.tex}
\item A card is selected from a pack of 52 cards\\
\begin{enumerate}[label=(\alph*)]
\item How many points are there in the sample space?
\item Calculate the probability that the cards is an ace of spades.
\item Calculate the probability that the card is (i) an ace (ii)black card.\\
\end{enumerate}
%\input{ncert/11/16/3/4_1/Prob_4.tex}
\item In a non-leap year, the probability of having 53 tuesdays or 53 wednesdays is\\
\solution
%\input{exemplar/11/16/3/18/main.tex}
\item There are 1000 sealed envelopes in a box, 10 of them contain a cash prize of
Rs 100 each, 100 of them contain a cash prize of Rs 50 each and 200 of them
contain a cash prize of Rs 10 each and rest do not contain any cash prize. If they
are well shuffled and an envelope is picked up out, what is the probability that it
contains no cash prize?\\
\solution
%\input{exemplar/10/13/3/34/main.tex}
\item 
A die is thrown and a card is selected at random from a deck of 52 playing cards. The probability of getting an even number on the die and a spade card.\\
\solution
%\input{exemplar/12/13/3/78/main.tex}
\item
If 4-digit numbers greater than 5,000 are randomly formed from the digits 0, 1, 3, 5, and 7, what is the probability of forming a number divisible by 5 when:
\begin{enumerate}
    \item The digits are repeated?
    \item The repetition of digits is not allowed?
\end{enumerate}
\solution
%\input{ncert/11/16/4/9/main.tex}
\item Consider the probability space $\brak{\Omega, \mathcal{G}, P}$ where $\Omega = [0,2]$ and $\mathcal{G} = \cbrak{\phi, \Omega, [0,1], (1,2]}$. Let $X$ and $Y$ be two functions on $\Omega$ defined as
\begin{align*}
    X(\omega) = 
    \begin{cases}
        1 & \text{if }\omega \in [0, 1]\\
        2 & \text{if }\omega \in (1, 2]
    \end{cases}
\end{align*}
and
\begin{align*}
    Y(\omega) = 
    \begin{cases}
        2 & \text{if }\omega \in [0, 1.5]\\
        3 & \text{if }\omega \in (1.5, 2].
    \end{cases}
\end{align*}
Then which one of the following statements is true?
\begin{enumerate}
    \item [(A)] $X$ is a random variable with respect to $\mathcal{G}$, but $Y$ is not a random variable with respect to $\mathcal{G}$.
    \item [(B)] $Y$ is a random variable with respect to $\mathcal{G}$, but $X$ is not a random variable with respect to $\mathcal{G}$.
    \item [(C)] Neither $X$ nor $Y$ is a random variable with respect to $\mathcal{G}$.
    \item [(D)] Both $X$ and $Y$ are random variables with respect to $\mathcal{G}$.
\end{enumerate} \hfill (GATE ST 2023)\\
\solution
%\input{gate/ST/2023/14/main.tex}
	\item  A die is loaded in such a way that each odd number is twice as likely to occur as
each even number. Find $P(G)$, where $G$ is the event that a number greater than
3 occurs on a single roll of the die.
\\
\solution
		%\input{exemplar/11/16/3/5/main.tex}
	\item All the jacks, queens and kings are removed from a deck of 52 playing cards. The remaining cards are well shuffled and then one card is drawn at random. Giving ace a value 1 similar value for other cards, find the probability that the card has a value 
		\begin{enumerate}
			\item 7
			\item greater than 7
			\item less than 7
		\end{enumerate}
		%\input{exemplar/10/13/3/30/main.tex}
  \item A Lot consists of 48 mobile phones of which 42 are good, 3 have only minor defects and 3 have major defects.Varnika will buy a phone if it is good but the trader will only buy a mobile if it has no major defects. One phone is selected at random from the lot. What is the probability that it is
\begin{enumerate}
	\item acceptable to Varnika?
            \item acceptable to the trader?
\end{enumerate}
\solution
	%\input{exemplar/10/13/3/40/main.tex}
 \item A student says that if you throw a die, it will show up 1 or not 1. Therefore, the probability of getting 1 and the probability of getting 'not 1' each is equal to $\frac{1}{2}$. Is this correct? Give reasons.\\
 \solution
        %\input{exemplar/10/13/2/9/main.tex}
   \item Four candidates A, B, C, D have ap-
plied for the assignment to coach a school cricket
team. If A is twice as likely to be selected as B, and
B and C are given about the same chance of being
selected, while C is twice as likely to be selected
as D, what are the probabilities that
\begin{enumerate}
\item C will be selected?
\item A will not be selected?
\end{enumerate}
	%\input{exemplar/11/16/3/9/main.tex}
 \item A bag contain 24 balls of which $x$ balls are red, $2x$ are white and $3x$ are blue. A ball is selected at random, What is the probability that it is
\begin{enumerate}[label=\alph*)]
\item not red ?
\item white ?
\end{enumerate}
%\input{exemplar/10/13/3/41/main.tex}
If the letters of the word ASSASSINATION are arranged at random. Find the Probability that
\begin{enumerate}[label=(\alph*)]
\item Four $S's$ come consecutively in the word
\item Two  $I's$ and two $N's$ come together
\item All $A's$ are not coming together
\item No two $A's$ are coming together
\end{enumerate}
%\input{exemplar/11/16/3/14/main.tex}
	\item One urn contains two black balls (labelled B1 and B2) and one white ball. A
	second urn contains one black ball and two white balls (labelled W1 and W2).
	Suppose the following experiment is performed. One of the two urns is chosen
	at random. Next a ball is randomly chosen from the urn. Then a second ball is
	chosen at random from the same urn without replacing the first ball.
	
	\begin{enumerate}
	\item What is the probability that two black balls are chosen?
	
	\item What is the probability that two balls of opposite colour are chosen?
	\end{enumerate}
	\solution
	%\input{exemplar/11/16/3/12/main1.tex}
\end{enumerate}

	\item 
The number lock of a suitcase has 4 wheels each labelled with ten digits i.e. from 0 to 9.The lock opens with a sequence of four digits with no repeats.What is the probability of a person getting the right sequence to open the suitcase.
\\
\solution
		%\begin{enumerate}[label=\thesection.\arabic*,ref=\thesection.\theenumi]
	\item One card is drawn from a well-shuffled deck of 52 cards. Find the probability of getting
\begin{enumerate}
\item A king of red colour 
\item A face card 
\item A red face card
\item The jack of hearts
\item A spade
\item The queen of diamonds

\end{enumerate}
\solution
		%\input{ncert/10/15/1/14/main.tex}
	\item Five cards—the ten, jack, queen, king and ace of diamonds, are well-shuffled with their face downwards. One card is then picked up at random.
\begin{enumerate}
\item
What is the probability that the card is the queen? 
\item
If the queen is drawn and put aside, what is the probability that the second card picked up is (a) an ace? (b) a queen?\\
\end{enumerate}
\solution
		%\input{ncert/10/15/1/15/defs.tex}
	\item A bag contains $5$ red balls and some blue balls. If the probability of drawing a blue ball is double that if a red ball, determine the number of blue balls in the bag. 
		\\
\solution
		%\input{ncert/10/15/2/3/defs.tex}
	\item A card is selected from a pack of 52 cards.
 \begin{enumerate}[label=(\alph*)] 
                 \item How many points are there in the sample space?
                 \item Calculate the probability that the card is an ace of spades.
                 \item Calculate the probability that the card is (i) an ace and (ii) black card.
 \end{enumerate}
\solution
		%\input{ncert/11/16/3/4/main.tex}
\item Four cards are drawn from a well-shuffled deck of 52 cards. What is the probability of obtaining 3 diamonds and one spade.
\\
\solution
		%\input{ncert/11/16/4/2/defs.tex}
\item In a certain lottery 10,000 tickets are sold and ten equal prizes are awarded. What is the probability of not getting a prize if you buy (a) one ticket (b) two tickets (c) 10 tickets ?	
\\
\solution
		%\input{ncert/11/16/4/4/defs.tex}
		%
\item 
Out of 100 students, two sections of 40 and 60 are formed. If you and your friend are among the 100 students, what is the probability that
\begin{enumerate}
\item you both enter the same section?
\item you both enter the different sections?
\end{enumerate}
\solution
		%\input{ncert/11/16/4/5/defs.tex}
	\item 
The number lock of a suitcase has 4 wheels each labelled with ten digits i.e. from 0 to 9.The lock opens with a sequence of four digits with no repeats.What is the probability of a person getting the right sequence to open the suitcase.
\\
\solution
		%\input{ncert/11/16/4/10/defs.tex}
		%
\item 
Two cards are drawn at random and without replacement from a pack of 52 playing cards. Find the probability that both the cards are black.
\\
\solution
		%\input{ncert/12/13/2/2/defs.tex}
		\item A box of oranges is inspected by examining three randomly selected oranges drawn without replacement. If all the three oranges are good, the box is approved for sale, otherwise, it is rejected. Find the probability that a box containing 15 oranges out of which 12 are good and 3 are bad ones will be approved for sale.
		\label{ncert/12/13/2/3/defs.tex}
		\item Two balls are drawn at random with replacement from a box containing 10 black and 8 red balls. Find the probability that
		\label{ncert/12/13/2/12}
\begin{enumerate}
\item both balls are red.
\item first ball is black and second is red.
\item one of them is black and other is red.
\end{enumerate}

\item In a hostel, 60\% of the students read Hindi newspaper, 40\% read English newspaper and 20\% read both Hindi and English newspapers. A student is selected at random.
		\label{ncert/12/13/2/15}
\begin{enumerate}
\item Find the probability that she reads neither Hindi nor English newspapers.
\item If she reads Hindi newspaper, find the probability that she reads English newspaper.
\item If she reads English newspaper, find the probability that she reads Hindi newspaper.\\
\end{enumerate}
\item The probability of obtaining an even prime number on each die, when a pair of dice is rolled is 
\begin{enumerate}
    \item $0$ 
    
    \item $\frac{1}{3}$ 
    
    \item $\frac{1}{12}$ 
    
    \item $\frac{1}{36}$ 
\end{enumerate}
\solution
		%\input{ncert/12/13/2/17/defs.tex}
	\item A bag contains 4 red and 4 black balls, another bag contains 2 red and 6 black balls. One of the two bags is selected at random and a ball is drawn from the bag which is found to be red. Find the probability that the ball is drawn from the first bag.
\\
\solution
		%\input{ncert/12/13/3/2/main.tex}
  \item
  Cards with numbers 2 to 101 are placed in a box. A card is selected at random.Find the probability that the card has
\begin{enumerate}[label=(\roman*)]
	\item an even number 
	\item a square number
\end{enumerate}
\solution
%\input{exemplar/10/13/3/32/main.tex}
\item
The king, queen and jack of clubs are removed from a deck of 52 playing cards and then well shuffled. Now one card is drawn at random from the remaining cards.  Determine the probability that the card is
\begin{enumerate}[label=(\roman*)]
\item a club
\item 10 of hearts
\end{enumerate}
\solution
%\input{exemplar/10/13/3/29/main.tex}
\item A team of medical students doing their internship have to assist during surgeries
at a city hospital. The probabilities of surgeries rated as very complex, complex,
routine, simple or very simple are respectively, 0.15, 0.20, 0.31, 0.26, .08. Find
the probabilities that a particular surgery will be rated
\begin{enumerate}
	\item complex or very complex;
	\item neither very complex nor very simple;
	\item routine or complex
	\item routine or simple
\end{enumerate}
\solution
%\input{exemplar/11/16/3/8(1)/main.tex}
\item A card is selected from a pack of 52 cards.
\begin{enumerate}[label=(\alph*)]
    \item How many points are there in the sample space?
    \item Calculate the probability that the card is an ace of spades.
    \item Calculate the probability that the card is (i) an ace and (ii) black card.
\end{enumerate}
\solution
%\input{exemplar/11/16/3/4/main2.tex}
\item The probability that a non leap year selected at random will contain 53 sundays.
\\
\solution
%\input{exemplar/10/13/1/19/main.tex}
\item One of the four persons John, Rita, Aslam or Gurpreet will be promoted next
month. Consequently the sample space consists of four elementary outcomes
S = {John promoted, Rita promoted, Aslam promoted, Gurpreet promoted}
You are told that the chances of John’s promotion is same as that of Gurpreet,
Rita’s chances of promotion are twice as likely as Johns. Aslam’s chances are
four times that of John.
\begin{enumerate}
	\item Determine
	\begin{enumerate}
		\item P (John promoted)
		\item P (Rita promoted)
		\item P (Aslam promoted)
		\item P (Gurpreet promoted)
	\end{enumerate}
	\item If A = {John promoted or Gurpreet promoted}, find P (A).
\end{enumerate}
\solution
%\input{exemplar/11/16/3/10/main.tex}
\item A card is drawn from a deck of 52 cards. Find the probability of getting a king or a heart or a red card.\\
\solution
%\input{exemplar/11/16/3/15/main.tex}
\item The probability that a student will pass his examination is 0.73, the probability of
the student getting a compartment is 0.13, and the probability that the student will
either pass or get compartment is 0.96. State True or False.\\
\solution
%\input{exemplar/11/16/3/31/main.tex}
\item A card is selected from a pack of 52 cards\\
\begin{enumerate}[label=(\alph*)]
\item How many points are there in the sample space?
\item Calculate the probability that the cards is an ace of spades.
\item Calculate the probability that the card is (i) an ace (ii)black card.\\
\end{enumerate}
%\input{ncert/11/16/3/4_1/Prob_4.tex}
\item In a non-leap year, the probability of having 53 tuesdays or 53 wednesdays is\\
\solution
%\input{exemplar/11/16/3/18/main.tex}
\item There are 1000 sealed envelopes in a box, 10 of them contain a cash prize of
Rs 100 each, 100 of them contain a cash prize of Rs 50 each and 200 of them
contain a cash prize of Rs 10 each and rest do not contain any cash prize. If they
are well shuffled and an envelope is picked up out, what is the probability that it
contains no cash prize?\\
\solution
%\input{exemplar/10/13/3/34/main.tex}
\item 
A die is thrown and a card is selected at random from a deck of 52 playing cards. The probability of getting an even number on the die and a spade card.\\
\solution
%\input{exemplar/12/13/3/78/main.tex}
\item
If 4-digit numbers greater than 5,000 are randomly formed from the digits 0, 1, 3, 5, and 7, what is the probability of forming a number divisible by 5 when:
\begin{enumerate}
    \item The digits are repeated?
    \item The repetition of digits is not allowed?
\end{enumerate}
\solution
%\input{ncert/11/16/4/9/main.tex}
\item Consider the probability space $\brak{\Omega, \mathcal{G}, P}$ where $\Omega = [0,2]$ and $\mathcal{G} = \cbrak{\phi, \Omega, [0,1], (1,2]}$. Let $X$ and $Y$ be two functions on $\Omega$ defined as
\begin{align*}
    X(\omega) = 
    \begin{cases}
        1 & \text{if }\omega \in [0, 1]\\
        2 & \text{if }\omega \in (1, 2]
    \end{cases}
\end{align*}
and
\begin{align*}
    Y(\omega) = 
    \begin{cases}
        2 & \text{if }\omega \in [0, 1.5]\\
        3 & \text{if }\omega \in (1.5, 2].
    \end{cases}
\end{align*}
Then which one of the following statements is true?
\begin{enumerate}
    \item [(A)] $X$ is a random variable with respect to $\mathcal{G}$, but $Y$ is not a random variable with respect to $\mathcal{G}$.
    \item [(B)] $Y$ is a random variable with respect to $\mathcal{G}$, but $X$ is not a random variable with respect to $\mathcal{G}$.
    \item [(C)] Neither $X$ nor $Y$ is a random variable with respect to $\mathcal{G}$.
    \item [(D)] Both $X$ and $Y$ are random variables with respect to $\mathcal{G}$.
\end{enumerate} \hfill (GATE ST 2023)\\
\solution
%\input{gate/ST/2023/14/main.tex}
	\item  A die is loaded in such a way that each odd number is twice as likely to occur as
each even number. Find $P(G)$, where $G$ is the event that a number greater than
3 occurs on a single roll of the die.
\\
\solution
		%\input{exemplar/11/16/3/5/main.tex}
	\item All the jacks, queens and kings are removed from a deck of 52 playing cards. The remaining cards are well shuffled and then one card is drawn at random. Giving ace a value 1 similar value for other cards, find the probability that the card has a value 
		\begin{enumerate}
			\item 7
			\item greater than 7
			\item less than 7
		\end{enumerate}
		%\input{exemplar/10/13/3/30/main.tex}
  \item A Lot consists of 48 mobile phones of which 42 are good, 3 have only minor defects and 3 have major defects.Varnika will buy a phone if it is good but the trader will only buy a mobile if it has no major defects. One phone is selected at random from the lot. What is the probability that it is
\begin{enumerate}
	\item acceptable to Varnika?
            \item acceptable to the trader?
\end{enumerate}
\solution
	%\input{exemplar/10/13/3/40/main.tex}
 \item A student says that if you throw a die, it will show up 1 or not 1. Therefore, the probability of getting 1 and the probability of getting 'not 1' each is equal to $\frac{1}{2}$. Is this correct? Give reasons.\\
 \solution
        %\input{exemplar/10/13/2/9/main.tex}
   \item Four candidates A, B, C, D have ap-
plied for the assignment to coach a school cricket
team. If A is twice as likely to be selected as B, and
B and C are given about the same chance of being
selected, while C is twice as likely to be selected
as D, what are the probabilities that
\begin{enumerate}
\item C will be selected?
\item A will not be selected?
\end{enumerate}
	%\input{exemplar/11/16/3/9/main.tex}
 \item A bag contain 24 balls of which $x$ balls are red, $2x$ are white and $3x$ are blue. A ball is selected at random, What is the probability that it is
\begin{enumerate}[label=\alph*)]
\item not red ?
\item white ?
\end{enumerate}
%\input{exemplar/10/13/3/41/main.tex}
If the letters of the word ASSASSINATION are arranged at random. Find the Probability that
\begin{enumerate}[label=(\alph*)]
\item Four $S's$ come consecutively in the word
\item Two  $I's$ and two $N's$ come together
\item All $A's$ are not coming together
\item No two $A's$ are coming together
\end{enumerate}
%\input{exemplar/11/16/3/14/main.tex}
	\item One urn contains two black balls (labelled B1 and B2) and one white ball. A
	second urn contains one black ball and two white balls (labelled W1 and W2).
	Suppose the following experiment is performed. One of the two urns is chosen
	at random. Next a ball is randomly chosen from the urn. Then a second ball is
	chosen at random from the same urn without replacing the first ball.
	
	\begin{enumerate}
	\item What is the probability that two black balls are chosen?
	
	\item What is the probability that two balls of opposite colour are chosen?
	\end{enumerate}
	\solution
	%\input{exemplar/11/16/3/12/main1.tex}
\end{enumerate}

		%
\item 
Two cards are drawn at random and without replacement from a pack of 52 playing cards. Find the probability that both the cards are black.
\\
\solution
		%\begin{enumerate}[label=\thesection.\arabic*,ref=\thesection.\theenumi]
	\item One card is drawn from a well-shuffled deck of 52 cards. Find the probability of getting
\begin{enumerate}
\item A king of red colour 
\item A face card 
\item A red face card
\item The jack of hearts
\item A spade
\item The queen of diamonds

\end{enumerate}
\solution
		%\input{ncert/10/15/1/14/main.tex}
	\item Five cards—the ten, jack, queen, king and ace of diamonds, are well-shuffled with their face downwards. One card is then picked up at random.
\begin{enumerate}
\item
What is the probability that the card is the queen? 
\item
If the queen is drawn and put aside, what is the probability that the second card picked up is (a) an ace? (b) a queen?\\
\end{enumerate}
\solution
		%\input{ncert/10/15/1/15/defs.tex}
	\item A bag contains $5$ red balls and some blue balls. If the probability of drawing a blue ball is double that if a red ball, determine the number of blue balls in the bag. 
		\\
\solution
		%\input{ncert/10/15/2/3/defs.tex}
	\item A card is selected from a pack of 52 cards.
 \begin{enumerate}[label=(\alph*)] 
                 \item How many points are there in the sample space?
                 \item Calculate the probability that the card is an ace of spades.
                 \item Calculate the probability that the card is (i) an ace and (ii) black card.
 \end{enumerate}
\solution
		%\input{ncert/11/16/3/4/main.tex}
\item Four cards are drawn from a well-shuffled deck of 52 cards. What is the probability of obtaining 3 diamonds and one spade.
\\
\solution
		%\input{ncert/11/16/4/2/defs.tex}
\item In a certain lottery 10,000 tickets are sold and ten equal prizes are awarded. What is the probability of not getting a prize if you buy (a) one ticket (b) two tickets (c) 10 tickets ?	
\\
\solution
		%\input{ncert/11/16/4/4/defs.tex}
		%
\item 
Out of 100 students, two sections of 40 and 60 are formed. If you and your friend are among the 100 students, what is the probability that
\begin{enumerate}
\item you both enter the same section?
\item you both enter the different sections?
\end{enumerate}
\solution
		%\input{ncert/11/16/4/5/defs.tex}
	\item 
The number lock of a suitcase has 4 wheels each labelled with ten digits i.e. from 0 to 9.The lock opens with a sequence of four digits with no repeats.What is the probability of a person getting the right sequence to open the suitcase.
\\
\solution
		%\input{ncert/11/16/4/10/defs.tex}
		%
\item 
Two cards are drawn at random and without replacement from a pack of 52 playing cards. Find the probability that both the cards are black.
\\
\solution
		%\input{ncert/12/13/2/2/defs.tex}
		\item A box of oranges is inspected by examining three randomly selected oranges drawn without replacement. If all the three oranges are good, the box is approved for sale, otherwise, it is rejected. Find the probability that a box containing 15 oranges out of which 12 are good and 3 are bad ones will be approved for sale.
		\label{ncert/12/13/2/3/defs.tex}
		\item Two balls are drawn at random with replacement from a box containing 10 black and 8 red balls. Find the probability that
		\label{ncert/12/13/2/12}
\begin{enumerate}
\item both balls are red.
\item first ball is black and second is red.
\item one of them is black and other is red.
\end{enumerate}

\item In a hostel, 60\% of the students read Hindi newspaper, 40\% read English newspaper and 20\% read both Hindi and English newspapers. A student is selected at random.
		\label{ncert/12/13/2/15}
\begin{enumerate}
\item Find the probability that she reads neither Hindi nor English newspapers.
\item If she reads Hindi newspaper, find the probability that she reads English newspaper.
\item If she reads English newspaper, find the probability that she reads Hindi newspaper.\\
\end{enumerate}
\item The probability of obtaining an even prime number on each die, when a pair of dice is rolled is 
\begin{enumerate}
    \item $0$ 
    
    \item $\frac{1}{3}$ 
    
    \item $\frac{1}{12}$ 
    
    \item $\frac{1}{36}$ 
\end{enumerate}
\solution
		%\input{ncert/12/13/2/17/defs.tex}
	\item A bag contains 4 red and 4 black balls, another bag contains 2 red and 6 black balls. One of the two bags is selected at random and a ball is drawn from the bag which is found to be red. Find the probability that the ball is drawn from the first bag.
\\
\solution
		%\input{ncert/12/13/3/2/main.tex}
  \item
  Cards with numbers 2 to 101 are placed in a box. A card is selected at random.Find the probability that the card has
\begin{enumerate}[label=(\roman*)]
	\item an even number 
	\item a square number
\end{enumerate}
\solution
%\input{exemplar/10/13/3/32/main.tex}
\item
The king, queen and jack of clubs are removed from a deck of 52 playing cards and then well shuffled. Now one card is drawn at random from the remaining cards.  Determine the probability that the card is
\begin{enumerate}[label=(\roman*)]
\item a club
\item 10 of hearts
\end{enumerate}
\solution
%\input{exemplar/10/13/3/29/main.tex}
\item A team of medical students doing their internship have to assist during surgeries
at a city hospital. The probabilities of surgeries rated as very complex, complex,
routine, simple or very simple are respectively, 0.15, 0.20, 0.31, 0.26, .08. Find
the probabilities that a particular surgery will be rated
\begin{enumerate}
	\item complex or very complex;
	\item neither very complex nor very simple;
	\item routine or complex
	\item routine or simple
\end{enumerate}
\solution
%\input{exemplar/11/16/3/8(1)/main.tex}
\item A card is selected from a pack of 52 cards.
\begin{enumerate}[label=(\alph*)]
    \item How many points are there in the sample space?
    \item Calculate the probability that the card is an ace of spades.
    \item Calculate the probability that the card is (i) an ace and (ii) black card.
\end{enumerate}
\solution
%\input{exemplar/11/16/3/4/main2.tex}
\item The probability that a non leap year selected at random will contain 53 sundays.
\\
\solution
%\input{exemplar/10/13/1/19/main.tex}
\item One of the four persons John, Rita, Aslam or Gurpreet will be promoted next
month. Consequently the sample space consists of four elementary outcomes
S = {John promoted, Rita promoted, Aslam promoted, Gurpreet promoted}
You are told that the chances of John’s promotion is same as that of Gurpreet,
Rita’s chances of promotion are twice as likely as Johns. Aslam’s chances are
four times that of John.
\begin{enumerate}
	\item Determine
	\begin{enumerate}
		\item P (John promoted)
		\item P (Rita promoted)
		\item P (Aslam promoted)
		\item P (Gurpreet promoted)
	\end{enumerate}
	\item If A = {John promoted or Gurpreet promoted}, find P (A).
\end{enumerate}
\solution
%\input{exemplar/11/16/3/10/main.tex}
\item A card is drawn from a deck of 52 cards. Find the probability of getting a king or a heart or a red card.\\
\solution
%\input{exemplar/11/16/3/15/main.tex}
\item The probability that a student will pass his examination is 0.73, the probability of
the student getting a compartment is 0.13, and the probability that the student will
either pass or get compartment is 0.96. State True or False.\\
\solution
%\input{exemplar/11/16/3/31/main.tex}
\item A card is selected from a pack of 52 cards\\
\begin{enumerate}[label=(\alph*)]
\item How many points are there in the sample space?
\item Calculate the probability that the cards is an ace of spades.
\item Calculate the probability that the card is (i) an ace (ii)black card.\\
\end{enumerate}
%\input{ncert/11/16/3/4_1/Prob_4.tex}
\item In a non-leap year, the probability of having 53 tuesdays or 53 wednesdays is\\
\solution
%\input{exemplar/11/16/3/18/main.tex}
\item There are 1000 sealed envelopes in a box, 10 of them contain a cash prize of
Rs 100 each, 100 of them contain a cash prize of Rs 50 each and 200 of them
contain a cash prize of Rs 10 each and rest do not contain any cash prize. If they
are well shuffled and an envelope is picked up out, what is the probability that it
contains no cash prize?\\
\solution
%\input{exemplar/10/13/3/34/main.tex}
\item 
A die is thrown and a card is selected at random from a deck of 52 playing cards. The probability of getting an even number on the die and a spade card.\\
\solution
%\input{exemplar/12/13/3/78/main.tex}
\item
If 4-digit numbers greater than 5,000 are randomly formed from the digits 0, 1, 3, 5, and 7, what is the probability of forming a number divisible by 5 when:
\begin{enumerate}
    \item The digits are repeated?
    \item The repetition of digits is not allowed?
\end{enumerate}
\solution
%\input{ncert/11/16/4/9/main.tex}
\item Consider the probability space $\brak{\Omega, \mathcal{G}, P}$ where $\Omega = [0,2]$ and $\mathcal{G} = \cbrak{\phi, \Omega, [0,1], (1,2]}$. Let $X$ and $Y$ be two functions on $\Omega$ defined as
\begin{align*}
    X(\omega) = 
    \begin{cases}
        1 & \text{if }\omega \in [0, 1]\\
        2 & \text{if }\omega \in (1, 2]
    \end{cases}
\end{align*}
and
\begin{align*}
    Y(\omega) = 
    \begin{cases}
        2 & \text{if }\omega \in [0, 1.5]\\
        3 & \text{if }\omega \in (1.5, 2].
    \end{cases}
\end{align*}
Then which one of the following statements is true?
\begin{enumerate}
    \item [(A)] $X$ is a random variable with respect to $\mathcal{G}$, but $Y$ is not a random variable with respect to $\mathcal{G}$.
    \item [(B)] $Y$ is a random variable with respect to $\mathcal{G}$, but $X$ is not a random variable with respect to $\mathcal{G}$.
    \item [(C)] Neither $X$ nor $Y$ is a random variable with respect to $\mathcal{G}$.
    \item [(D)] Both $X$ and $Y$ are random variables with respect to $\mathcal{G}$.
\end{enumerate} \hfill (GATE ST 2023)\\
\solution
%\input{gate/ST/2023/14/main.tex}
	\item  A die is loaded in such a way that each odd number is twice as likely to occur as
each even number. Find $P(G)$, where $G$ is the event that a number greater than
3 occurs on a single roll of the die.
\\
\solution
		%\input{exemplar/11/16/3/5/main.tex}
	\item All the jacks, queens and kings are removed from a deck of 52 playing cards. The remaining cards are well shuffled and then one card is drawn at random. Giving ace a value 1 similar value for other cards, find the probability that the card has a value 
		\begin{enumerate}
			\item 7
			\item greater than 7
			\item less than 7
		\end{enumerate}
		%\input{exemplar/10/13/3/30/main.tex}
  \item A Lot consists of 48 mobile phones of which 42 are good, 3 have only minor defects and 3 have major defects.Varnika will buy a phone if it is good but the trader will only buy a mobile if it has no major defects. One phone is selected at random from the lot. What is the probability that it is
\begin{enumerate}
	\item acceptable to Varnika?
            \item acceptable to the trader?
\end{enumerate}
\solution
	%\input{exemplar/10/13/3/40/main.tex}
 \item A student says that if you throw a die, it will show up 1 or not 1. Therefore, the probability of getting 1 and the probability of getting 'not 1' each is equal to $\frac{1}{2}$. Is this correct? Give reasons.\\
 \solution
        %\input{exemplar/10/13/2/9/main.tex}
   \item Four candidates A, B, C, D have ap-
plied for the assignment to coach a school cricket
team. If A is twice as likely to be selected as B, and
B and C are given about the same chance of being
selected, while C is twice as likely to be selected
as D, what are the probabilities that
\begin{enumerate}
\item C will be selected?
\item A will not be selected?
\end{enumerate}
	%\input{exemplar/11/16/3/9/main.tex}
 \item A bag contain 24 balls of which $x$ balls are red, $2x$ are white and $3x$ are blue. A ball is selected at random, What is the probability that it is
\begin{enumerate}[label=\alph*)]
\item not red ?
\item white ?
\end{enumerate}
%\input{exemplar/10/13/3/41/main.tex}
If the letters of the word ASSASSINATION are arranged at random. Find the Probability that
\begin{enumerate}[label=(\alph*)]
\item Four $S's$ come consecutively in the word
\item Two  $I's$ and two $N's$ come together
\item All $A's$ are not coming together
\item No two $A's$ are coming together
\end{enumerate}
%\input{exemplar/11/16/3/14/main.tex}
	\item One urn contains two black balls (labelled B1 and B2) and one white ball. A
	second urn contains one black ball and two white balls (labelled W1 and W2).
	Suppose the following experiment is performed. One of the two urns is chosen
	at random. Next a ball is randomly chosen from the urn. Then a second ball is
	chosen at random from the same urn without replacing the first ball.
	
	\begin{enumerate}
	\item What is the probability that two black balls are chosen?
	
	\item What is the probability that two balls of opposite colour are chosen?
	\end{enumerate}
	\solution
	%\input{exemplar/11/16/3/12/main1.tex}
\end{enumerate}

		\item A box of oranges is inspected by examining three randomly selected oranges drawn without replacement. If all the three oranges are good, the box is approved for sale, otherwise, it is rejected. Find the probability that a box containing 15 oranges out of which 12 are good and 3 are bad ones will be approved for sale.
		\label{ncert/12/13/2/3/defs.tex}
		\item Two balls are drawn at random with replacement from a box containing 10 black and 8 red balls. Find the probability that
		\label{ncert/12/13/2/12}
\begin{enumerate}
\item both balls are red.
\item first ball is black and second is red.
\item one of them is black and other is red.
\end{enumerate}

\item In a hostel, 60\% of the students read Hindi newspaper, 40\% read English newspaper and 20\% read both Hindi and English newspapers. A student is selected at random.
		\label{ncert/12/13/2/15}
\begin{enumerate}
\item Find the probability that she reads neither Hindi nor English newspapers.
\item If she reads Hindi newspaper, find the probability that she reads English newspaper.
\item If she reads English newspaper, find the probability that she reads Hindi newspaper.\\
\end{enumerate}
\item The probability of obtaining an even prime number on each die, when a pair of dice is rolled is 
\begin{enumerate}
    \item $0$ 
    
    \item $\frac{1}{3}$ 
    
    \item $\frac{1}{12}$ 
    
    \item $\frac{1}{36}$ 
\end{enumerate}
\solution
		%\begin{enumerate}[label=\thesection.\arabic*,ref=\thesection.\theenumi]
	\item One card is drawn from a well-shuffled deck of 52 cards. Find the probability of getting
\begin{enumerate}
\item A king of red colour 
\item A face card 
\item A red face card
\item The jack of hearts
\item A spade
\item The queen of diamonds

\end{enumerate}
\solution
		%\input{ncert/10/15/1/14/main.tex}
	\item Five cards—the ten, jack, queen, king and ace of diamonds, are well-shuffled with their face downwards. One card is then picked up at random.
\begin{enumerate}
\item
What is the probability that the card is the queen? 
\item
If the queen is drawn and put aside, what is the probability that the second card picked up is (a) an ace? (b) a queen?\\
\end{enumerate}
\solution
		%\input{ncert/10/15/1/15/defs.tex}
	\item A bag contains $5$ red balls and some blue balls. If the probability of drawing a blue ball is double that if a red ball, determine the number of blue balls in the bag. 
		\\
\solution
		%\input{ncert/10/15/2/3/defs.tex}
	\item A card is selected from a pack of 52 cards.
 \begin{enumerate}[label=(\alph*)] 
                 \item How many points are there in the sample space?
                 \item Calculate the probability that the card is an ace of spades.
                 \item Calculate the probability that the card is (i) an ace and (ii) black card.
 \end{enumerate}
\solution
		%\input{ncert/11/16/3/4/main.tex}
\item Four cards are drawn from a well-shuffled deck of 52 cards. What is the probability of obtaining 3 diamonds and one spade.
\\
\solution
		%\input{ncert/11/16/4/2/defs.tex}
\item In a certain lottery 10,000 tickets are sold and ten equal prizes are awarded. What is the probability of not getting a prize if you buy (a) one ticket (b) two tickets (c) 10 tickets ?	
\\
\solution
		%\input{ncert/11/16/4/4/defs.tex}
		%
\item 
Out of 100 students, two sections of 40 and 60 are formed. If you and your friend are among the 100 students, what is the probability that
\begin{enumerate}
\item you both enter the same section?
\item you both enter the different sections?
\end{enumerate}
\solution
		%\input{ncert/11/16/4/5/defs.tex}
	\item 
The number lock of a suitcase has 4 wheels each labelled with ten digits i.e. from 0 to 9.The lock opens with a sequence of four digits with no repeats.What is the probability of a person getting the right sequence to open the suitcase.
\\
\solution
		%\input{ncert/11/16/4/10/defs.tex}
		%
\item 
Two cards are drawn at random and without replacement from a pack of 52 playing cards. Find the probability that both the cards are black.
\\
\solution
		%\input{ncert/12/13/2/2/defs.tex}
		\item A box of oranges is inspected by examining three randomly selected oranges drawn without replacement. If all the three oranges are good, the box is approved for sale, otherwise, it is rejected. Find the probability that a box containing 15 oranges out of which 12 are good and 3 are bad ones will be approved for sale.
		\label{ncert/12/13/2/3/defs.tex}
		\item Two balls are drawn at random with replacement from a box containing 10 black and 8 red balls. Find the probability that
		\label{ncert/12/13/2/12}
\begin{enumerate}
\item both balls are red.
\item first ball is black and second is red.
\item one of them is black and other is red.
\end{enumerate}

\item In a hostel, 60\% of the students read Hindi newspaper, 40\% read English newspaper and 20\% read both Hindi and English newspapers. A student is selected at random.
		\label{ncert/12/13/2/15}
\begin{enumerate}
\item Find the probability that she reads neither Hindi nor English newspapers.
\item If she reads Hindi newspaper, find the probability that she reads English newspaper.
\item If she reads English newspaper, find the probability that she reads Hindi newspaper.\\
\end{enumerate}
\item The probability of obtaining an even prime number on each die, when a pair of dice is rolled is 
\begin{enumerate}
    \item $0$ 
    
    \item $\frac{1}{3}$ 
    
    \item $\frac{1}{12}$ 
    
    \item $\frac{1}{36}$ 
\end{enumerate}
\solution
		%\input{ncert/12/13/2/17/defs.tex}
	\item A bag contains 4 red and 4 black balls, another bag contains 2 red and 6 black balls. One of the two bags is selected at random and a ball is drawn from the bag which is found to be red. Find the probability that the ball is drawn from the first bag.
\\
\solution
		%\input{ncert/12/13/3/2/main.tex}
  \item
  Cards with numbers 2 to 101 are placed in a box. A card is selected at random.Find the probability that the card has
\begin{enumerate}[label=(\roman*)]
	\item an even number 
	\item a square number
\end{enumerate}
\solution
%\input{exemplar/10/13/3/32/main.tex}
\item
The king, queen and jack of clubs are removed from a deck of 52 playing cards and then well shuffled. Now one card is drawn at random from the remaining cards.  Determine the probability that the card is
\begin{enumerate}[label=(\roman*)]
\item a club
\item 10 of hearts
\end{enumerate}
\solution
%\input{exemplar/10/13/3/29/main.tex}
\item A team of medical students doing their internship have to assist during surgeries
at a city hospital. The probabilities of surgeries rated as very complex, complex,
routine, simple or very simple are respectively, 0.15, 0.20, 0.31, 0.26, .08. Find
the probabilities that a particular surgery will be rated
\begin{enumerate}
	\item complex or very complex;
	\item neither very complex nor very simple;
	\item routine or complex
	\item routine or simple
\end{enumerate}
\solution
%\input{exemplar/11/16/3/8(1)/main.tex}
\item A card is selected from a pack of 52 cards.
\begin{enumerate}[label=(\alph*)]
    \item How many points are there in the sample space?
    \item Calculate the probability that the card is an ace of spades.
    \item Calculate the probability that the card is (i) an ace and (ii) black card.
\end{enumerate}
\solution
%\input{exemplar/11/16/3/4/main2.tex}
\item The probability that a non leap year selected at random will contain 53 sundays.
\\
\solution
%\input{exemplar/10/13/1/19/main.tex}
\item One of the four persons John, Rita, Aslam or Gurpreet will be promoted next
month. Consequently the sample space consists of four elementary outcomes
S = {John promoted, Rita promoted, Aslam promoted, Gurpreet promoted}
You are told that the chances of John’s promotion is same as that of Gurpreet,
Rita’s chances of promotion are twice as likely as Johns. Aslam’s chances are
four times that of John.
\begin{enumerate}
	\item Determine
	\begin{enumerate}
		\item P (John promoted)
		\item P (Rita promoted)
		\item P (Aslam promoted)
		\item P (Gurpreet promoted)
	\end{enumerate}
	\item If A = {John promoted or Gurpreet promoted}, find P (A).
\end{enumerate}
\solution
%\input{exemplar/11/16/3/10/main.tex}
\item A card is drawn from a deck of 52 cards. Find the probability of getting a king or a heart or a red card.\\
\solution
%\input{exemplar/11/16/3/15/main.tex}
\item The probability that a student will pass his examination is 0.73, the probability of
the student getting a compartment is 0.13, and the probability that the student will
either pass or get compartment is 0.96. State True or False.\\
\solution
%\input{exemplar/11/16/3/31/main.tex}
\item A card is selected from a pack of 52 cards\\
\begin{enumerate}[label=(\alph*)]
\item How many points are there in the sample space?
\item Calculate the probability that the cards is an ace of spades.
\item Calculate the probability that the card is (i) an ace (ii)black card.\\
\end{enumerate}
%\input{ncert/11/16/3/4_1/Prob_4.tex}
\item In a non-leap year, the probability of having 53 tuesdays or 53 wednesdays is\\
\solution
%\input{exemplar/11/16/3/18/main.tex}
\item There are 1000 sealed envelopes in a box, 10 of them contain a cash prize of
Rs 100 each, 100 of them contain a cash prize of Rs 50 each and 200 of them
contain a cash prize of Rs 10 each and rest do not contain any cash prize. If they
are well shuffled and an envelope is picked up out, what is the probability that it
contains no cash prize?\\
\solution
%\input{exemplar/10/13/3/34/main.tex}
\item 
A die is thrown and a card is selected at random from a deck of 52 playing cards. The probability of getting an even number on the die and a spade card.\\
\solution
%\input{exemplar/12/13/3/78/main.tex}
\item
If 4-digit numbers greater than 5,000 are randomly formed from the digits 0, 1, 3, 5, and 7, what is the probability of forming a number divisible by 5 when:
\begin{enumerate}
    \item The digits are repeated?
    \item The repetition of digits is not allowed?
\end{enumerate}
\solution
%\input{ncert/11/16/4/9/main.tex}
\item Consider the probability space $\brak{\Omega, \mathcal{G}, P}$ where $\Omega = [0,2]$ and $\mathcal{G} = \cbrak{\phi, \Omega, [0,1], (1,2]}$. Let $X$ and $Y$ be two functions on $\Omega$ defined as
\begin{align*}
    X(\omega) = 
    \begin{cases}
        1 & \text{if }\omega \in [0, 1]\\
        2 & \text{if }\omega \in (1, 2]
    \end{cases}
\end{align*}
and
\begin{align*}
    Y(\omega) = 
    \begin{cases}
        2 & \text{if }\omega \in [0, 1.5]\\
        3 & \text{if }\omega \in (1.5, 2].
    \end{cases}
\end{align*}
Then which one of the following statements is true?
\begin{enumerate}
    \item [(A)] $X$ is a random variable with respect to $\mathcal{G}$, but $Y$ is not a random variable with respect to $\mathcal{G}$.
    \item [(B)] $Y$ is a random variable with respect to $\mathcal{G}$, but $X$ is not a random variable with respect to $\mathcal{G}$.
    \item [(C)] Neither $X$ nor $Y$ is a random variable with respect to $\mathcal{G}$.
    \item [(D)] Both $X$ and $Y$ are random variables with respect to $\mathcal{G}$.
\end{enumerate} \hfill (GATE ST 2023)\\
\solution
%\input{gate/ST/2023/14/main.tex}
	\item  A die is loaded in such a way that each odd number is twice as likely to occur as
each even number. Find $P(G)$, where $G$ is the event that a number greater than
3 occurs on a single roll of the die.
\\
\solution
		%\input{exemplar/11/16/3/5/main.tex}
	\item All the jacks, queens and kings are removed from a deck of 52 playing cards. The remaining cards are well shuffled and then one card is drawn at random. Giving ace a value 1 similar value for other cards, find the probability that the card has a value 
		\begin{enumerate}
			\item 7
			\item greater than 7
			\item less than 7
		\end{enumerate}
		%\input{exemplar/10/13/3/30/main.tex}
  \item A Lot consists of 48 mobile phones of which 42 are good, 3 have only minor defects and 3 have major defects.Varnika will buy a phone if it is good but the trader will only buy a mobile if it has no major defects. One phone is selected at random from the lot. What is the probability that it is
\begin{enumerate}
	\item acceptable to Varnika?
            \item acceptable to the trader?
\end{enumerate}
\solution
	%\input{exemplar/10/13/3/40/main.tex}
 \item A student says that if you throw a die, it will show up 1 or not 1. Therefore, the probability of getting 1 and the probability of getting 'not 1' each is equal to $\frac{1}{2}$. Is this correct? Give reasons.\\
 \solution
        %\input{exemplar/10/13/2/9/main.tex}
   \item Four candidates A, B, C, D have ap-
plied for the assignment to coach a school cricket
team. If A is twice as likely to be selected as B, and
B and C are given about the same chance of being
selected, while C is twice as likely to be selected
as D, what are the probabilities that
\begin{enumerate}
\item C will be selected?
\item A will not be selected?
\end{enumerate}
	%\input{exemplar/11/16/3/9/main.tex}
 \item A bag contain 24 balls of which $x$ balls are red, $2x$ are white and $3x$ are blue. A ball is selected at random, What is the probability that it is
\begin{enumerate}[label=\alph*)]
\item not red ?
\item white ?
\end{enumerate}
%\input{exemplar/10/13/3/41/main.tex}
If the letters of the word ASSASSINATION are arranged at random. Find the Probability that
\begin{enumerate}[label=(\alph*)]
\item Four $S's$ come consecutively in the word
\item Two  $I's$ and two $N's$ come together
\item All $A's$ are not coming together
\item No two $A's$ are coming together
\end{enumerate}
%\input{exemplar/11/16/3/14/main.tex}
	\item One urn contains two black balls (labelled B1 and B2) and one white ball. A
	second urn contains one black ball and two white balls (labelled W1 and W2).
	Suppose the following experiment is performed. One of the two urns is chosen
	at random. Next a ball is randomly chosen from the urn. Then a second ball is
	chosen at random from the same urn without replacing the first ball.
	
	\begin{enumerate}
	\item What is the probability that two black balls are chosen?
	
	\item What is the probability that two balls of opposite colour are chosen?
	\end{enumerate}
	\solution
	%\input{exemplar/11/16/3/12/main1.tex}
\end{enumerate}

	\item A bag contains 4 red and 4 black balls, another bag contains 2 red and 6 black balls. One of the two bags is selected at random and a ball is drawn from the bag which is found to be red. Find the probability that the ball is drawn from the first bag.
\\
\solution
		%\begin{table}[H]
	\centering
\begin{tabular}{|c|c|c|}
\hline
Random variable &Value &Definition\\ \hline
\multirow{3}{*}{X} &0 &Slips of Rs 1\\
&1 &Slips of Rs 5\\
&2 &Slips of Rs 13\\ \hline
\multirow{2}{*}{Y} &0 &Box A\\
&1 &Box B\\\hline
\end{tabular}
\caption{}
\label{tab:Distribution}
\end{table}
See \tabref{tab:Distribution}.
\begin{align}
p_{Y}\brak{k}= \begin{cases} 
      \frac{1}{3} & {k=0} \\
      \frac{2}{3 }& {k=1} 
   \end{cases}
   \\
p_{Y|X}\brak{0|0} = \frac{19}{25}\, 
p_{Y|X}\brak{0|1} = \frac{6}{25}\,
p_{Y|X}\brak{1|0} = \frac{45}{50}\,
p_{Y|X}\brak{1|2} = \frac{5}{50}
\end{align}
The desired probability is the probability that a slip drawn at random is marked other than Rs 1,
\begin{align}
&=1-p_X\brak{0}\\
&= p_X(1) + p_X(2)
\end{align}
Using Bayes theorem,
\begin{align}
&= p_Y\brak{0} \times \pr{Y=0 | X=1} + p_Y\brak{1} \times \pr{Y=1|X=2}\\
&=\frac{1}{3} \times \frac{6}{25} + \frac{2}{3} \times \frac{5}{50}\\
&=\frac{11}{75}
\end{align}

\newpage

%\tableofcontents

\bigskip

\renewcommand{\thefigure}{\theenumi}
\renewcommand{\thetable}{\theenumi}
%\renewcommand{\theequation}{\theenumi}

%\begin{abstract}
%%\boldmath
%In this letter, an algorithm for evaluating the exact analytical bit error rate  (BER)  for the piecewise linear (PL) combiner for  multiple relays is presented. Previous results were available only for upto three relays. The algorithm is unique in the sense that  the actual mathematical expressions, that are prohibitively large, need not be explicitly obtained. The diversity gain due to multiple relays is shown through plots of the analytical BER, well supported by simulations. 
%
%\end{abstract}
% IEEEtran.cls defaults to using nonbold math in the Abstract.
% This preserves the distinction between vectors and scalars. However,
% if the journal you are submitting to favors bold math in the abstract,
% then you can use LaTeX's standard command \boldmath at the very start
% of the abstract to achieve this. Many IEEE journals frown on math
% in the abstract anyway.

% Note that keywords are not normally used for peerreview papers.
%\begin{IEEEkeywords}
%Cooperative diversity, decode and forward, piecewise linear
%\end{IEEEkeywords}



% For peer review papers, you can put extra information on the cover
% page as needed:
% \ifCLASSOPTIONpeerreview
% \begin{center} \bfseries EDICS Category: 3-BBND \end{center}
% \fi
%
% For peerreview papers, this IEEEtran command inserts a page break and
% creates the second title. It will be ignored for other modes.
%\IEEEpeerreviewmaketitle




  \item
  Cards with numbers 2 to 101 are placed in a box. A card is selected at random.Find the probability that the card has
\begin{enumerate}[label=(\roman*)]
	\item an even number 
	\item a square number
\end{enumerate}
\solution
%\begin{table}[H]
	\centering
\begin{tabular}{|c|c|c|}
\hline
Random variable &Value &Definition\\ \hline
\multirow{3}{*}{X} &0 &Slips of Rs 1\\
&1 &Slips of Rs 5\\
&2 &Slips of Rs 13\\ \hline
\multirow{2}{*}{Y} &0 &Box A\\
&1 &Box B\\\hline
\end{tabular}
\caption{}
\label{tab:Distribution}
\end{table}
See \tabref{tab:Distribution}.
\begin{align}
p_{Y}\brak{k}= \begin{cases} 
      \frac{1}{3} & {k=0} \\
      \frac{2}{3 }& {k=1} 
   \end{cases}
   \\
p_{Y|X}\brak{0|0} = \frac{19}{25}\, 
p_{Y|X}\brak{0|1} = \frac{6}{25}\,
p_{Y|X}\brak{1|0} = \frac{45}{50}\,
p_{Y|X}\brak{1|2} = \frac{5}{50}
\end{align}
The desired probability is the probability that a slip drawn at random is marked other than Rs 1,
\begin{align}
&=1-p_X\brak{0}\\
&= p_X(1) + p_X(2)
\end{align}
Using Bayes theorem,
\begin{align}
&= p_Y\brak{0} \times \pr{Y=0 | X=1} + p_Y\brak{1} \times \pr{Y=1|X=2}\\
&=\frac{1}{3} \times \frac{6}{25} + \frac{2}{3} \times \frac{5}{50}\\
&=\frac{11}{75}
\end{align}

\newpage

%\tableofcontents

\bigskip

\renewcommand{\thefigure}{\theenumi}
\renewcommand{\thetable}{\theenumi}
%\renewcommand{\theequation}{\theenumi}

%\begin{abstract}
%%\boldmath
%In this letter, an algorithm for evaluating the exact analytical bit error rate  (BER)  for the piecewise linear (PL) combiner for  multiple relays is presented. Previous results were available only for upto three relays. The algorithm is unique in the sense that  the actual mathematical expressions, that are prohibitively large, need not be explicitly obtained. The diversity gain due to multiple relays is shown through plots of the analytical BER, well supported by simulations. 
%
%\end{abstract}
% IEEEtran.cls defaults to using nonbold math in the Abstract.
% This preserves the distinction between vectors and scalars. However,
% if the journal you are submitting to favors bold math in the abstract,
% then you can use LaTeX's standard command \boldmath at the very start
% of the abstract to achieve this. Many IEEE journals frown on math
% in the abstract anyway.

% Note that keywords are not normally used for peerreview papers.
%\begin{IEEEkeywords}
%Cooperative diversity, decode and forward, piecewise linear
%\end{IEEEkeywords}



% For peer review papers, you can put extra information on the cover
% page as needed:
% \ifCLASSOPTIONpeerreview
% \begin{center} \bfseries EDICS Category: 3-BBND \end{center}
% \fi
%
% For peerreview papers, this IEEEtran command inserts a page break and
% creates the second title. It will be ignored for other modes.
%\IEEEpeerreviewmaketitle




\item
The king, queen and jack of clubs are removed from a deck of 52 playing cards and then well shuffled. Now one card is drawn at random from the remaining cards.  Determine the probability that the card is
\begin{enumerate}[label=(\roman*)]
\item a club
\item 10 of hearts
\end{enumerate}
\solution
%\begin{table}[H]
	\centering
\begin{tabular}{|c|c|c|}
\hline
Random variable &Value &Definition\\ \hline
\multirow{3}{*}{X} &0 &Slips of Rs 1\\
&1 &Slips of Rs 5\\
&2 &Slips of Rs 13\\ \hline
\multirow{2}{*}{Y} &0 &Box A\\
&1 &Box B\\\hline
\end{tabular}
\caption{}
\label{tab:Distribution}
\end{table}
See \tabref{tab:Distribution}.
\begin{align}
p_{Y}\brak{k}= \begin{cases} 
      \frac{1}{3} & {k=0} \\
      \frac{2}{3 }& {k=1} 
   \end{cases}
   \\
p_{Y|X}\brak{0|0} = \frac{19}{25}\, 
p_{Y|X}\brak{0|1} = \frac{6}{25}\,
p_{Y|X}\brak{1|0} = \frac{45}{50}\,
p_{Y|X}\brak{1|2} = \frac{5}{50}
\end{align}
The desired probability is the probability that a slip drawn at random is marked other than Rs 1,
\begin{align}
&=1-p_X\brak{0}\\
&= p_X(1) + p_X(2)
\end{align}
Using Bayes theorem,
\begin{align}
&= p_Y\brak{0} \times \pr{Y=0 | X=1} + p_Y\brak{1} \times \pr{Y=1|X=2}\\
&=\frac{1}{3} \times \frac{6}{25} + \frac{2}{3} \times \frac{5}{50}\\
&=\frac{11}{75}
\end{align}

\newpage

%\tableofcontents

\bigskip

\renewcommand{\thefigure}{\theenumi}
\renewcommand{\thetable}{\theenumi}
%\renewcommand{\theequation}{\theenumi}

%\begin{abstract}
%%\boldmath
%In this letter, an algorithm for evaluating the exact analytical bit error rate  (BER)  for the piecewise linear (PL) combiner for  multiple relays is presented. Previous results were available only for upto three relays. The algorithm is unique in the sense that  the actual mathematical expressions, that are prohibitively large, need not be explicitly obtained. The diversity gain due to multiple relays is shown through plots of the analytical BER, well supported by simulations. 
%
%\end{abstract}
% IEEEtran.cls defaults to using nonbold math in the Abstract.
% This preserves the distinction between vectors and scalars. However,
% if the journal you are submitting to favors bold math in the abstract,
% then you can use LaTeX's standard command \boldmath at the very start
% of the abstract to achieve this. Many IEEE journals frown on math
% in the abstract anyway.

% Note that keywords are not normally used for peerreview papers.
%\begin{IEEEkeywords}
%Cooperative diversity, decode and forward, piecewise linear
%\end{IEEEkeywords}



% For peer review papers, you can put extra information on the cover
% page as needed:
% \ifCLASSOPTIONpeerreview
% \begin{center} \bfseries EDICS Category: 3-BBND \end{center}
% \fi
%
% For peerreview papers, this IEEEtran command inserts a page break and
% creates the second title. It will be ignored for other modes.
%\IEEEpeerreviewmaketitle




\item A team of medical students doing their internship have to assist during surgeries
at a city hospital. The probabilities of surgeries rated as very complex, complex,
routine, simple or very simple are respectively, 0.15, 0.20, 0.31, 0.26, .08. Find
the probabilities that a particular surgery will be rated
\begin{enumerate}
	\item complex or very complex;
	\item neither very complex nor very simple;
	\item routine or complex
	\item routine or simple
\end{enumerate}
\solution
%\begin{table}[H]
	\centering
\begin{tabular}{|c|c|c|}
\hline
Random variable &Value &Definition\\ \hline
\multirow{3}{*}{X} &0 &Slips of Rs 1\\
&1 &Slips of Rs 5\\
&2 &Slips of Rs 13\\ \hline
\multirow{2}{*}{Y} &0 &Box A\\
&1 &Box B\\\hline
\end{tabular}
\caption{}
\label{tab:Distribution}
\end{table}
See \tabref{tab:Distribution}.
\begin{align}
p_{Y}\brak{k}= \begin{cases} 
      \frac{1}{3} & {k=0} \\
      \frac{2}{3 }& {k=1} 
   \end{cases}
   \\
p_{Y|X}\brak{0|0} = \frac{19}{25}\, 
p_{Y|X}\brak{0|1} = \frac{6}{25}\,
p_{Y|X}\brak{1|0} = \frac{45}{50}\,
p_{Y|X}\brak{1|2} = \frac{5}{50}
\end{align}
The desired probability is the probability that a slip drawn at random is marked other than Rs 1,
\begin{align}
&=1-p_X\brak{0}\\
&= p_X(1) + p_X(2)
\end{align}
Using Bayes theorem,
\begin{align}
&= p_Y\brak{0} \times \pr{Y=0 | X=1} + p_Y\brak{1} \times \pr{Y=1|X=2}\\
&=\frac{1}{3} \times \frac{6}{25} + \frac{2}{3} \times \frac{5}{50}\\
&=\frac{11}{75}
\end{align}

\newpage

%\tableofcontents

\bigskip

\renewcommand{\thefigure}{\theenumi}
\renewcommand{\thetable}{\theenumi}
%\renewcommand{\theequation}{\theenumi}

%\begin{abstract}
%%\boldmath
%In this letter, an algorithm for evaluating the exact analytical bit error rate  (BER)  for the piecewise linear (PL) combiner for  multiple relays is presented. Previous results were available only for upto three relays. The algorithm is unique in the sense that  the actual mathematical expressions, that are prohibitively large, need not be explicitly obtained. The diversity gain due to multiple relays is shown through plots of the analytical BER, well supported by simulations. 
%
%\end{abstract}
% IEEEtran.cls defaults to using nonbold math in the Abstract.
% This preserves the distinction between vectors and scalars. However,
% if the journal you are submitting to favors bold math in the abstract,
% then you can use LaTeX's standard command \boldmath at the very start
% of the abstract to achieve this. Many IEEE journals frown on math
% in the abstract anyway.

% Note that keywords are not normally used for peerreview papers.
%\begin{IEEEkeywords}
%Cooperative diversity, decode and forward, piecewise linear
%\end{IEEEkeywords}



% For peer review papers, you can put extra information on the cover
% page as needed:
% \ifCLASSOPTIONpeerreview
% \begin{center} \bfseries EDICS Category: 3-BBND \end{center}
% \fi
%
% For peerreview papers, this IEEEtran command inserts a page break and
% creates the second title. It will be ignored for other modes.
%\IEEEpeerreviewmaketitle




\item A card is selected from a pack of 52 cards.
\begin{enumerate}[label=(\alph*)]
    \item How many points are there in the sample space?
    \item Calculate the probability that the card is an ace of spades.
    \item Calculate the probability that the card is (i) an ace and (ii) black card.
\end{enumerate}
\solution
%Let $X$ be an bernoulli rv defined as in \tabref{tab:exemplar/11/16/3/26}.  Then, 
\begin{equation}
    p =
        \frac{4}{11} 
\end{equation}
\begin{table}[H]
	\centering
	\input{exemplar/11/16/3/26/tables/Table2.tex}
	\caption{}
        \label{tab:exemplar/11/16/3/26}
\end{table}

\item The probability that a non leap year selected at random will contain 53 sundays.
\\
\solution
%\begin{table}[H]
	\centering
\begin{tabular}{|c|c|c|}
\hline
Random variable &Value &Definition\\ \hline
\multirow{3}{*}{X} &0 &Slips of Rs 1\\
&1 &Slips of Rs 5\\
&2 &Slips of Rs 13\\ \hline
\multirow{2}{*}{Y} &0 &Box A\\
&1 &Box B\\\hline
\end{tabular}
\caption{}
\label{tab:Distribution}
\end{table}
See \tabref{tab:Distribution}.
\begin{align}
p_{Y}\brak{k}= \begin{cases} 
      \frac{1}{3} & {k=0} \\
      \frac{2}{3 }& {k=1} 
   \end{cases}
   \\
p_{Y|X}\brak{0|0} = \frac{19}{25}\, 
p_{Y|X}\brak{0|1} = \frac{6}{25}\,
p_{Y|X}\brak{1|0} = \frac{45}{50}\,
p_{Y|X}\brak{1|2} = \frac{5}{50}
\end{align}
The desired probability is the probability that a slip drawn at random is marked other than Rs 1,
\begin{align}
&=1-p_X\brak{0}\\
&= p_X(1) + p_X(2)
\end{align}
Using Bayes theorem,
\begin{align}
&= p_Y\brak{0} \times \pr{Y=0 | X=1} + p_Y\brak{1} \times \pr{Y=1|X=2}\\
&=\frac{1}{3} \times \frac{6}{25} + \frac{2}{3} \times \frac{5}{50}\\
&=\frac{11}{75}
\end{align}

\newpage

%\tableofcontents

\bigskip

\renewcommand{\thefigure}{\theenumi}
\renewcommand{\thetable}{\theenumi}
%\renewcommand{\theequation}{\theenumi}

%\begin{abstract}
%%\boldmath
%In this letter, an algorithm for evaluating the exact analytical bit error rate  (BER)  for the piecewise linear (PL) combiner for  multiple relays is presented. Previous results were available only for upto three relays. The algorithm is unique in the sense that  the actual mathematical expressions, that are prohibitively large, need not be explicitly obtained. The diversity gain due to multiple relays is shown through plots of the analytical BER, well supported by simulations. 
%
%\end{abstract}
% IEEEtran.cls defaults to using nonbold math in the Abstract.
% This preserves the distinction between vectors and scalars. However,
% if the journal you are submitting to favors bold math in the abstract,
% then you can use LaTeX's standard command \boldmath at the very start
% of the abstract to achieve this. Many IEEE journals frown on math
% in the abstract anyway.

% Note that keywords are not normally used for peerreview papers.
%\begin{IEEEkeywords}
%Cooperative diversity, decode and forward, piecewise linear
%\end{IEEEkeywords}



% For peer review papers, you can put extra information on the cover
% page as needed:
% \ifCLASSOPTIONpeerreview
% \begin{center} \bfseries EDICS Category: 3-BBND \end{center}
% \fi
%
% For peerreview papers, this IEEEtran command inserts a page break and
% creates the second title. It will be ignored for other modes.
%\IEEEpeerreviewmaketitle




\item One of the four persons John, Rita, Aslam or Gurpreet will be promoted next
month. Consequently the sample space consists of four elementary outcomes
S = {John promoted, Rita promoted, Aslam promoted, Gurpreet promoted}
You are told that the chances of John’s promotion is same as that of Gurpreet,
Rita’s chances of promotion are twice as likely as Johns. Aslam’s chances are
four times that of John.
\begin{enumerate}
	\item Determine
	\begin{enumerate}
		\item P (John promoted)
		\item P (Rita promoted)
		\item P (Aslam promoted)
		\item P (Gurpreet promoted)
	\end{enumerate}
	\item If A = {John promoted or Gurpreet promoted}, find P (A).
\end{enumerate}
\solution
%\begin{table}[H]
	\centering
\begin{tabular}{|c|c|c|}
\hline
Random variable &Value &Definition\\ \hline
\multirow{3}{*}{X} &0 &Slips of Rs 1\\
&1 &Slips of Rs 5\\
&2 &Slips of Rs 13\\ \hline
\multirow{2}{*}{Y} &0 &Box A\\
&1 &Box B\\\hline
\end{tabular}
\caption{}
\label{tab:Distribution}
\end{table}
See \tabref{tab:Distribution}.
\begin{align}
p_{Y}\brak{k}= \begin{cases} 
      \frac{1}{3} & {k=0} \\
      \frac{2}{3 }& {k=1} 
   \end{cases}
   \\
p_{Y|X}\brak{0|0} = \frac{19}{25}\, 
p_{Y|X}\brak{0|1} = \frac{6}{25}\,
p_{Y|X}\brak{1|0} = \frac{45}{50}\,
p_{Y|X}\brak{1|2} = \frac{5}{50}
\end{align}
The desired probability is the probability that a slip drawn at random is marked other than Rs 1,
\begin{align}
&=1-p_X\brak{0}\\
&= p_X(1) + p_X(2)
\end{align}
Using Bayes theorem,
\begin{align}
&= p_Y\brak{0} \times \pr{Y=0 | X=1} + p_Y\brak{1} \times \pr{Y=1|X=2}\\
&=\frac{1}{3} \times \frac{6}{25} + \frac{2}{3} \times \frac{5}{50}\\
&=\frac{11}{75}
\end{align}

\newpage

%\tableofcontents

\bigskip

\renewcommand{\thefigure}{\theenumi}
\renewcommand{\thetable}{\theenumi}
%\renewcommand{\theequation}{\theenumi}

%\begin{abstract}
%%\boldmath
%In this letter, an algorithm for evaluating the exact analytical bit error rate  (BER)  for the piecewise linear (PL) combiner for  multiple relays is presented. Previous results were available only for upto three relays. The algorithm is unique in the sense that  the actual mathematical expressions, that are prohibitively large, need not be explicitly obtained. The diversity gain due to multiple relays is shown through plots of the analytical BER, well supported by simulations. 
%
%\end{abstract}
% IEEEtran.cls defaults to using nonbold math in the Abstract.
% This preserves the distinction between vectors and scalars. However,
% if the journal you are submitting to favors bold math in the abstract,
% then you can use LaTeX's standard command \boldmath at the very start
% of the abstract to achieve this. Many IEEE journals frown on math
% in the abstract anyway.

% Note that keywords are not normally used for peerreview papers.
%\begin{IEEEkeywords}
%Cooperative diversity, decode and forward, piecewise linear
%\end{IEEEkeywords}



% For peer review papers, you can put extra information on the cover
% page as needed:
% \ifCLASSOPTIONpeerreview
% \begin{center} \bfseries EDICS Category: 3-BBND \end{center}
% \fi
%
% For peerreview papers, this IEEEtran command inserts a page break and
% creates the second title. It will be ignored for other modes.
%\IEEEpeerreviewmaketitle




\item A card is drawn from a deck of 52 cards. Find the probability of getting a king or a heart or a red card.\\
\solution
%\begin{table}[H]
	\centering
\begin{tabular}{|c|c|c|}
\hline
Random variable &Value &Definition\\ \hline
\multirow{3}{*}{X} &0 &Slips of Rs 1\\
&1 &Slips of Rs 5\\
&2 &Slips of Rs 13\\ \hline
\multirow{2}{*}{Y} &0 &Box A\\
&1 &Box B\\\hline
\end{tabular}
\caption{}
\label{tab:Distribution}
\end{table}
See \tabref{tab:Distribution}.
\begin{align}
p_{Y}\brak{k}= \begin{cases} 
      \frac{1}{3} & {k=0} \\
      \frac{2}{3 }& {k=1} 
   \end{cases}
   \\
p_{Y|X}\brak{0|0} = \frac{19}{25}\, 
p_{Y|X}\brak{0|1} = \frac{6}{25}\,
p_{Y|X}\brak{1|0} = \frac{45}{50}\,
p_{Y|X}\brak{1|2} = \frac{5}{50}
\end{align}
The desired probability is the probability that a slip drawn at random is marked other than Rs 1,
\begin{align}
&=1-p_X\brak{0}\\
&= p_X(1) + p_X(2)
\end{align}
Using Bayes theorem,
\begin{align}
&= p_Y\brak{0} \times \pr{Y=0 | X=1} + p_Y\brak{1} \times \pr{Y=1|X=2}\\
&=\frac{1}{3} \times \frac{6}{25} + \frac{2}{3} \times \frac{5}{50}\\
&=\frac{11}{75}
\end{align}

\newpage

%\tableofcontents

\bigskip

\renewcommand{\thefigure}{\theenumi}
\renewcommand{\thetable}{\theenumi}
%\renewcommand{\theequation}{\theenumi}

%\begin{abstract}
%%\boldmath
%In this letter, an algorithm for evaluating the exact analytical bit error rate  (BER)  for the piecewise linear (PL) combiner for  multiple relays is presented. Previous results were available only for upto three relays. The algorithm is unique in the sense that  the actual mathematical expressions, that are prohibitively large, need not be explicitly obtained. The diversity gain due to multiple relays is shown through plots of the analytical BER, well supported by simulations. 
%
%\end{abstract}
% IEEEtran.cls defaults to using nonbold math in the Abstract.
% This preserves the distinction between vectors and scalars. However,
% if the journal you are submitting to favors bold math in the abstract,
% then you can use LaTeX's standard command \boldmath at the very start
% of the abstract to achieve this. Many IEEE journals frown on math
% in the abstract anyway.

% Note that keywords are not normally used for peerreview papers.
%\begin{IEEEkeywords}
%Cooperative diversity, decode and forward, piecewise linear
%\end{IEEEkeywords}



% For peer review papers, you can put extra information on the cover
% page as needed:
% \ifCLASSOPTIONpeerreview
% \begin{center} \bfseries EDICS Category: 3-BBND \end{center}
% \fi
%
% For peerreview papers, this IEEEtran command inserts a page break and
% creates the second title. It will be ignored for other modes.
%\IEEEpeerreviewmaketitle




\item The probability that a student will pass his examination is 0.73, the probability of
the student getting a compartment is 0.13, and the probability that the student will
either pass or get compartment is 0.96. State True or False.\\
\solution
%\begin{table}[H]
	\centering
\begin{tabular}{|c|c|c|}
\hline
Random variable &Value &Definition\\ \hline
\multirow{3}{*}{X} &0 &Slips of Rs 1\\
&1 &Slips of Rs 5\\
&2 &Slips of Rs 13\\ \hline
\multirow{2}{*}{Y} &0 &Box A\\
&1 &Box B\\\hline
\end{tabular}
\caption{}
\label{tab:Distribution}
\end{table}
See \tabref{tab:Distribution}.
\begin{align}
p_{Y}\brak{k}= \begin{cases} 
      \frac{1}{3} & {k=0} \\
      \frac{2}{3 }& {k=1} 
   \end{cases}
   \\
p_{Y|X}\brak{0|0} = \frac{19}{25}\, 
p_{Y|X}\brak{0|1} = \frac{6}{25}\,
p_{Y|X}\brak{1|0} = \frac{45}{50}\,
p_{Y|X}\brak{1|2} = \frac{5}{50}
\end{align}
The desired probability is the probability that a slip drawn at random is marked other than Rs 1,
\begin{align}
&=1-p_X\brak{0}\\
&= p_X(1) + p_X(2)
\end{align}
Using Bayes theorem,
\begin{align}
&= p_Y\brak{0} \times \pr{Y=0 | X=1} + p_Y\brak{1} \times \pr{Y=1|X=2}\\
&=\frac{1}{3} \times \frac{6}{25} + \frac{2}{3} \times \frac{5}{50}\\
&=\frac{11}{75}
\end{align}

\newpage

%\tableofcontents

\bigskip

\renewcommand{\thefigure}{\theenumi}
\renewcommand{\thetable}{\theenumi}
%\renewcommand{\theequation}{\theenumi}

%\begin{abstract}
%%\boldmath
%In this letter, an algorithm for evaluating the exact analytical bit error rate  (BER)  for the piecewise linear (PL) combiner for  multiple relays is presented. Previous results were available only for upto three relays. The algorithm is unique in the sense that  the actual mathematical expressions, that are prohibitively large, need not be explicitly obtained. The diversity gain due to multiple relays is shown through plots of the analytical BER, well supported by simulations. 
%
%\end{abstract}
% IEEEtran.cls defaults to using nonbold math in the Abstract.
% This preserves the distinction between vectors and scalars. However,
% if the journal you are submitting to favors bold math in the abstract,
% then you can use LaTeX's standard command \boldmath at the very start
% of the abstract to achieve this. Many IEEE journals frown on math
% in the abstract anyway.

% Note that keywords are not normally used for peerreview papers.
%\begin{IEEEkeywords}
%Cooperative diversity, decode and forward, piecewise linear
%\end{IEEEkeywords}



% For peer review papers, you can put extra information on the cover
% page as needed:
% \ifCLASSOPTIONpeerreview
% \begin{center} \bfseries EDICS Category: 3-BBND \end{center}
% \fi
%
% For peerreview papers, this IEEEtran command inserts a page break and
% creates the second title. It will be ignored for other modes.
%\IEEEpeerreviewmaketitle




\item A card is selected from a pack of 52 cards\\
\begin{enumerate}[label=(\alph*)]
\item How many points are there in the sample space?
\item Calculate the probability that the cards is an ace of spades.
\item Calculate the probability that the card is (i) an ace (ii)black card.\\
\end{enumerate}
%\input{ncert/11/16/3/4_1/Prob_4.tex}
\item In a non-leap year, the probability of having 53 tuesdays or 53 wednesdays is\\
\solution
%A non-leap year has a total of 365 days, and a week has 7 days.\\
So it can be expressed as 
\begin{align}
365\text{days} &=52\times 7+1 \text{day}
\end{align}
$\implies$ 52 tuesdays or wednesdays\\
Random variable X denotes the days of a week
\begin{align}
p_X\brak{k}&=\frac{1}{7}; \quad \brak{1<k<7}
\end{align}
So the probability of extra day being tuesday or wednesday is
\begin{align}
p_X\brak{3}+p_X\brak{4}&=\frac{1}{7}+\frac{1}{7}=\frac{2}{7}
\end{align}



\item There are 1000 sealed envelopes in a box, 10 of them contain a cash prize of
Rs 100 each, 100 of them contain a cash prize of Rs 50 each and 200 of them
contain a cash prize of Rs 10 each and rest do not contain any cash prize. If they
are well shuffled and an envelope is picked up out, what is the probability that it
contains no cash prize?\\
\solution
%\begin{table}[H]
	\centering
\begin{tabular}{|c|c|c|}
\hline
Random variable &Value &Definition\\ \hline
\multirow{3}{*}{X} &0 &Slips of Rs 1\\
&1 &Slips of Rs 5\\
&2 &Slips of Rs 13\\ \hline
\multirow{2}{*}{Y} &0 &Box A\\
&1 &Box B\\\hline
\end{tabular}
\caption{}
\label{tab:Distribution}
\end{table}
See \tabref{tab:Distribution}.
\begin{align}
p_{Y}\brak{k}= \begin{cases} 
      \frac{1}{3} & {k=0} \\
      \frac{2}{3 }& {k=1} 
   \end{cases}
   \\
p_{Y|X}\brak{0|0} = \frac{19}{25}\, 
p_{Y|X}\brak{0|1} = \frac{6}{25}\,
p_{Y|X}\brak{1|0} = \frac{45}{50}\,
p_{Y|X}\brak{1|2} = \frac{5}{50}
\end{align}
The desired probability is the probability that a slip drawn at random is marked other than Rs 1,
\begin{align}
&=1-p_X\brak{0}\\
&= p_X(1) + p_X(2)
\end{align}
Using Bayes theorem,
\begin{align}
&= p_Y\brak{0} \times \pr{Y=0 | X=1} + p_Y\brak{1} \times \pr{Y=1|X=2}\\
&=\frac{1}{3} \times \frac{6}{25} + \frac{2}{3} \times \frac{5}{50}\\
&=\frac{11}{75}
\end{align}

\newpage

%\tableofcontents

\bigskip

\renewcommand{\thefigure}{\theenumi}
\renewcommand{\thetable}{\theenumi}
%\renewcommand{\theequation}{\theenumi}

%\begin{abstract}
%%\boldmath
%In this letter, an algorithm for evaluating the exact analytical bit error rate  (BER)  for the piecewise linear (PL) combiner for  multiple relays is presented. Previous results were available only for upto three relays. The algorithm is unique in the sense that  the actual mathematical expressions, that are prohibitively large, need not be explicitly obtained. The diversity gain due to multiple relays is shown through plots of the analytical BER, well supported by simulations. 
%
%\end{abstract}
% IEEEtran.cls defaults to using nonbold math in the Abstract.
% This preserves the distinction between vectors and scalars. However,
% if the journal you are submitting to favors bold math in the abstract,
% then you can use LaTeX's standard command \boldmath at the very start
% of the abstract to achieve this. Many IEEE journals frown on math
% in the abstract anyway.

% Note that keywords are not normally used for peerreview papers.
%\begin{IEEEkeywords}
%Cooperative diversity, decode and forward, piecewise linear
%\end{IEEEkeywords}



% For peer review papers, you can put extra information on the cover
% page as needed:
% \ifCLASSOPTIONpeerreview
% \begin{center} \bfseries EDICS Category: 3-BBND \end{center}
% \fi
%
% For peerreview papers, this IEEEtran command inserts a page break and
% creates the second title. It will be ignored for other modes.
%\IEEEpeerreviewmaketitle




\item 
A die is thrown and a card is selected at random from a deck of 52 playing cards. The probability of getting an even number on the die and a spade card.\\
\solution
%\begin{table}[H]
	\centering
\begin{tabular}{|c|c|c|}
\hline
Random variable &Value &Definition\\ \hline
\multirow{3}{*}{X} &0 &Slips of Rs 1\\
&1 &Slips of Rs 5\\
&2 &Slips of Rs 13\\ \hline
\multirow{2}{*}{Y} &0 &Box A\\
&1 &Box B\\\hline
\end{tabular}
\caption{}
\label{tab:Distribution}
\end{table}
See \tabref{tab:Distribution}.
\begin{align}
p_{Y}\brak{k}= \begin{cases} 
      \frac{1}{3} & {k=0} \\
      \frac{2}{3 }& {k=1} 
   \end{cases}
   \\
p_{Y|X}\brak{0|0} = \frac{19}{25}\, 
p_{Y|X}\brak{0|1} = \frac{6}{25}\,
p_{Y|X}\brak{1|0} = \frac{45}{50}\,
p_{Y|X}\brak{1|2} = \frac{5}{50}
\end{align}
The desired probability is the probability that a slip drawn at random is marked other than Rs 1,
\begin{align}
&=1-p_X\brak{0}\\
&= p_X(1) + p_X(2)
\end{align}
Using Bayes theorem,
\begin{align}
&= p_Y\brak{0} \times \pr{Y=0 | X=1} + p_Y\brak{1} \times \pr{Y=1|X=2}\\
&=\frac{1}{3} \times \frac{6}{25} + \frac{2}{3} \times \frac{5}{50}\\
&=\frac{11}{75}
\end{align}

\newpage

%\tableofcontents

\bigskip

\renewcommand{\thefigure}{\theenumi}
\renewcommand{\thetable}{\theenumi}
%\renewcommand{\theequation}{\theenumi}

%\begin{abstract}
%%\boldmath
%In this letter, an algorithm for evaluating the exact analytical bit error rate  (BER)  for the piecewise linear (PL) combiner for  multiple relays is presented. Previous results were available only for upto three relays. The algorithm is unique in the sense that  the actual mathematical expressions, that are prohibitively large, need not be explicitly obtained. The diversity gain due to multiple relays is shown through plots of the analytical BER, well supported by simulations. 
%
%\end{abstract}
% IEEEtran.cls defaults to using nonbold math in the Abstract.
% This preserves the distinction between vectors and scalars. However,
% if the journal you are submitting to favors bold math in the abstract,
% then you can use LaTeX's standard command \boldmath at the very start
% of the abstract to achieve this. Many IEEE journals frown on math
% in the abstract anyway.

% Note that keywords are not normally used for peerreview papers.
%\begin{IEEEkeywords}
%Cooperative diversity, decode and forward, piecewise linear
%\end{IEEEkeywords}



% For peer review papers, you can put extra information on the cover
% page as needed:
% \ifCLASSOPTIONpeerreview
% \begin{center} \bfseries EDICS Category: 3-BBND \end{center}
% \fi
%
% For peerreview papers, this IEEEtran command inserts a page break and
% creates the second title. It will be ignored for other modes.
%\IEEEpeerreviewmaketitle




\item
If 4-digit numbers greater than 5,000 are randomly formed from the digits 0, 1, 3, 5, and 7, what is the probability of forming a number divisible by 5 when:
\begin{enumerate}
    \item The digits are repeated?
    \item The repetition of digits is not allowed?
\end{enumerate}
\solution
%\begin{table}[H]
	\centering
\begin{tabular}{|c|c|c|}
\hline
Random variable &Value &Definition\\ \hline
\multirow{3}{*}{X} &0 &Slips of Rs 1\\
&1 &Slips of Rs 5\\
&2 &Slips of Rs 13\\ \hline
\multirow{2}{*}{Y} &0 &Box A\\
&1 &Box B\\\hline
\end{tabular}
\caption{}
\label{tab:Distribution}
\end{table}
See \tabref{tab:Distribution}.
\begin{align}
p_{Y}\brak{k}= \begin{cases} 
      \frac{1}{3} & {k=0} \\
      \frac{2}{3 }& {k=1} 
   \end{cases}
   \\
p_{Y|X}\brak{0|0} = \frac{19}{25}\, 
p_{Y|X}\brak{0|1} = \frac{6}{25}\,
p_{Y|X}\brak{1|0} = \frac{45}{50}\,
p_{Y|X}\brak{1|2} = \frac{5}{50}
\end{align}
The desired probability is the probability that a slip drawn at random is marked other than Rs 1,
\begin{align}
&=1-p_X\brak{0}\\
&= p_X(1) + p_X(2)
\end{align}
Using Bayes theorem,
\begin{align}
&= p_Y\brak{0} \times \pr{Y=0 | X=1} + p_Y\brak{1} \times \pr{Y=1|X=2}\\
&=\frac{1}{3} \times \frac{6}{25} + \frac{2}{3} \times \frac{5}{50}\\
&=\frac{11}{75}
\end{align}

\newpage

%\tableofcontents

\bigskip

\renewcommand{\thefigure}{\theenumi}
\renewcommand{\thetable}{\theenumi}
%\renewcommand{\theequation}{\theenumi}

%\begin{abstract}
%%\boldmath
%In this letter, an algorithm for evaluating the exact analytical bit error rate  (BER)  for the piecewise linear (PL) combiner for  multiple relays is presented. Previous results were available only for upto three relays. The algorithm is unique in the sense that  the actual mathematical expressions, that are prohibitively large, need not be explicitly obtained. The diversity gain due to multiple relays is shown through plots of the analytical BER, well supported by simulations. 
%
%\end{abstract}
% IEEEtran.cls defaults to using nonbold math in the Abstract.
% This preserves the distinction between vectors and scalars. However,
% if the journal you are submitting to favors bold math in the abstract,
% then you can use LaTeX's standard command \boldmath at the very start
% of the abstract to achieve this. Many IEEE journals frown on math
% in the abstract anyway.

% Note that keywords are not normally used for peerreview papers.
%\begin{IEEEkeywords}
%Cooperative diversity, decode and forward, piecewise linear
%\end{IEEEkeywords}



% For peer review papers, you can put extra information on the cover
% page as needed:
% \ifCLASSOPTIONpeerreview
% \begin{center} \bfseries EDICS Category: 3-BBND \end{center}
% \fi
%
% For peerreview papers, this IEEEtran command inserts a page break and
% creates the second title. It will be ignored for other modes.
%\IEEEpeerreviewmaketitle




\item Consider the probability space $\brak{\Omega, \mathcal{G}, P}$ where $\Omega = [0,2]$ and $\mathcal{G} = \cbrak{\phi, \Omega, [0,1], (1,2]}$. Let $X$ and $Y$ be two functions on $\Omega$ defined as
\begin{align*}
    X(\omega) = 
    \begin{cases}
        1 & \text{if }\omega \in [0, 1]\\
        2 & \text{if }\omega \in (1, 2]
    \end{cases}
\end{align*}
and
\begin{align*}
    Y(\omega) = 
    \begin{cases}
        2 & \text{if }\omega \in [0, 1.5]\\
        3 & \text{if }\omega \in (1.5, 2].
    \end{cases}
\end{align*}
Then which one of the following statements is true?
\begin{enumerate}
    \item [(A)] $X$ is a random variable with respect to $\mathcal{G}$, but $Y$ is not a random variable with respect to $\mathcal{G}$.
    \item [(B)] $Y$ is a random variable with respect to $\mathcal{G}$, but $X$ is not a random variable with respect to $\mathcal{G}$.
    \item [(C)] Neither $X$ nor $Y$ is a random variable with respect to $\mathcal{G}$.
    \item [(D)] Both $X$ and $Y$ are random variables with respect to $\mathcal{G}$.
\end{enumerate} \hfill (GATE ST 2023)\\
\solution
%\begin{table}[H]
	\centering
\begin{tabular}{|c|c|c|}
\hline
Random variable &Value &Definition\\ \hline
\multirow{3}{*}{X} &0 &Slips of Rs 1\\
&1 &Slips of Rs 5\\
&2 &Slips of Rs 13\\ \hline
\multirow{2}{*}{Y} &0 &Box A\\
&1 &Box B\\\hline
\end{tabular}
\caption{}
\label{tab:Distribution}
\end{table}
See \tabref{tab:Distribution}.
\begin{align}
p_{Y}\brak{k}= \begin{cases} 
      \frac{1}{3} & {k=0} \\
      \frac{2}{3 }& {k=1} 
   \end{cases}
   \\
p_{Y|X}\brak{0|0} = \frac{19}{25}\, 
p_{Y|X}\brak{0|1} = \frac{6}{25}\,
p_{Y|X}\brak{1|0} = \frac{45}{50}\,
p_{Y|X}\brak{1|2} = \frac{5}{50}
\end{align}
The desired probability is the probability that a slip drawn at random is marked other than Rs 1,
\begin{align}
&=1-p_X\brak{0}\\
&= p_X(1) + p_X(2)
\end{align}
Using Bayes theorem,
\begin{align}
&= p_Y\brak{0} \times \pr{Y=0 | X=1} + p_Y\brak{1} \times \pr{Y=1|X=2}\\
&=\frac{1}{3} \times \frac{6}{25} + \frac{2}{3} \times \frac{5}{50}\\
&=\frac{11}{75}
\end{align}

\newpage

%\tableofcontents

\bigskip

\renewcommand{\thefigure}{\theenumi}
\renewcommand{\thetable}{\theenumi}
%\renewcommand{\theequation}{\theenumi}

%\begin{abstract}
%%\boldmath
%In this letter, an algorithm for evaluating the exact analytical bit error rate  (BER)  for the piecewise linear (PL) combiner for  multiple relays is presented. Previous results were available only for upto three relays. The algorithm is unique in the sense that  the actual mathematical expressions, that are prohibitively large, need not be explicitly obtained. The diversity gain due to multiple relays is shown through plots of the analytical BER, well supported by simulations. 
%
%\end{abstract}
% IEEEtran.cls defaults to using nonbold math in the Abstract.
% This preserves the distinction between vectors and scalars. However,
% if the journal you are submitting to favors bold math in the abstract,
% then you can use LaTeX's standard command \boldmath at the very start
% of the abstract to achieve this. Many IEEE journals frown on math
% in the abstract anyway.

% Note that keywords are not normally used for peerreview papers.
%\begin{IEEEkeywords}
%Cooperative diversity, decode and forward, piecewise linear
%\end{IEEEkeywords}



% For peer review papers, you can put extra information on the cover
% page as needed:
% \ifCLASSOPTIONpeerreview
% \begin{center} \bfseries EDICS Category: 3-BBND \end{center}
% \fi
%
% For peerreview papers, this IEEEtran command inserts a page break and
% creates the second title. It will be ignored for other modes.
%\IEEEpeerreviewmaketitle




	\item  A die is loaded in such a way that each odd number is twice as likely to occur as
each even number. Find $P(G)$, where $G$ is the event that a number greater than
3 occurs on a single roll of the die.
\\
\solution
		%\begin{table}[H]
	\centering
\begin{tabular}{|c|c|c|}
\hline
Random variable &Value &Definition\\ \hline
\multirow{3}{*}{X} &0 &Slips of Rs 1\\
&1 &Slips of Rs 5\\
&2 &Slips of Rs 13\\ \hline
\multirow{2}{*}{Y} &0 &Box A\\
&1 &Box B\\\hline
\end{tabular}
\caption{}
\label{tab:Distribution}
\end{table}
See \tabref{tab:Distribution}.
\begin{align}
p_{Y}\brak{k}= \begin{cases} 
      \frac{1}{3} & {k=0} \\
      \frac{2}{3 }& {k=1} 
   \end{cases}
   \\
p_{Y|X}\brak{0|0} = \frac{19}{25}\, 
p_{Y|X}\brak{0|1} = \frac{6}{25}\,
p_{Y|X}\brak{1|0} = \frac{45}{50}\,
p_{Y|X}\brak{1|2} = \frac{5}{50}
\end{align}
The desired probability is the probability that a slip drawn at random is marked other than Rs 1,
\begin{align}
&=1-p_X\brak{0}\\
&= p_X(1) + p_X(2)
\end{align}
Using Bayes theorem,
\begin{align}
&= p_Y\brak{0} \times \pr{Y=0 | X=1} + p_Y\brak{1} \times \pr{Y=1|X=2}\\
&=\frac{1}{3} \times \frac{6}{25} + \frac{2}{3} \times \frac{5}{50}\\
&=\frac{11}{75}
\end{align}

\newpage

%\tableofcontents

\bigskip

\renewcommand{\thefigure}{\theenumi}
\renewcommand{\thetable}{\theenumi}
%\renewcommand{\theequation}{\theenumi}

%\begin{abstract}
%%\boldmath
%In this letter, an algorithm for evaluating the exact analytical bit error rate  (BER)  for the piecewise linear (PL) combiner for  multiple relays is presented. Previous results were available only for upto three relays. The algorithm is unique in the sense that  the actual mathematical expressions, that are prohibitively large, need not be explicitly obtained. The diversity gain due to multiple relays is shown through plots of the analytical BER, well supported by simulations. 
%
%\end{abstract}
% IEEEtran.cls defaults to using nonbold math in the Abstract.
% This preserves the distinction between vectors and scalars. However,
% if the journal you are submitting to favors bold math in the abstract,
% then you can use LaTeX's standard command \boldmath at the very start
% of the abstract to achieve this. Many IEEE journals frown on math
% in the abstract anyway.

% Note that keywords are not normally used for peerreview papers.
%\begin{IEEEkeywords}
%Cooperative diversity, decode and forward, piecewise linear
%\end{IEEEkeywords}



% For peer review papers, you can put extra information on the cover
% page as needed:
% \ifCLASSOPTIONpeerreview
% \begin{center} \bfseries EDICS Category: 3-BBND \end{center}
% \fi
%
% For peerreview papers, this IEEEtran command inserts a page break and
% creates the second title. It will be ignored for other modes.
%\IEEEpeerreviewmaketitle




	\item All the jacks, queens and kings are removed from a deck of 52 playing cards. The remaining cards are well shuffled and then one card is drawn at random. Giving ace a value 1 similar value for other cards, find the probability that the card has a value 
		\begin{enumerate}
			\item 7
			\item greater than 7
			\item less than 7
		\end{enumerate}
		%Number of cards left after removing all jacks, queens and kings 
\begin{align}
N	= 52 - 4\times 3
	= 40
\end{align}
%\begin{table}[H]
%\def\arraystretch{1.2}
%\begin{tabular}{|c|c|c|}
%\hline
%	\textbf{Parameter} &\textbf{Value} &\textbf{Description}\\ \hline
%	$X$ &1-10 &Represents the value of the card picked \\ \hline
%\end{tabular}
%\end{table}
Let $1 \le X \le 10$ be the value of the card picked.  Then,
\begin{align}
	p_X(k) &= \Pr(X=k)\ \forall\ 1 \leq k \leq 10\\
	&= \frac{4\times 1}{40}\\
	&= \frac{1}{10}\\
	\therefore p_X(k) &= 
	\begin{cases}
		\frac{1}{10} & 1 \leq k \leq 10\\
		0 & \text{otherwise}
	\end{cases}
\end{align}
and
\begin{align}
	F_{X}(k) &= \sum_{m=0}^{k}p_{X}(m) \quad 1 \leq k \leq 10\\
	&= \frac{k}{10}\\
	\therefore F_{X}(k) &= 
	\begin{cases}
		0 & k \leq 0\\
		\frac{k}{10} & 1\leq k \leq 10\\
		1 & k > 10 
	\end{cases}
\end{align}
\begin{enumerate}
	\item Probability that card has value equal to 7 is
		\begin{align}
			 p_{X}(7)
			= \frac{1}{10}
		\end{align}
	\item Probability that card has value greater than 7 is
		\begin{align}
			1 - F_X(7)
			&= 1 - \frac{7}{10}
			\\
			&= \frac{3}{10}
		\end{align}
	\item Probability that card has value less than 7 is
		\begin{align}
			 F_{X}(6)
			=\frac{6}{10}
		\end{align}
\end{enumerate}

  \item A Lot consists of 48 mobile phones of which 42 are good, 3 have only minor defects and 3 have major defects.Varnika will buy a phone if it is good but the trader will only buy a mobile if it has no major defects. One phone is selected at random from the lot. What is the probability that it is
\begin{enumerate}
	\item acceptable to Varnika?
            \item acceptable to the trader?
\end{enumerate}
\solution
	%\begin{table}[H]
	\centering
\begin{tabular}{|c|c|c|}
\hline
Random variable &Value &Definition\\ \hline
\multirow{3}{*}{X} &0 &Slips of Rs 1\\
&1 &Slips of Rs 5\\
&2 &Slips of Rs 13\\ \hline
\multirow{2}{*}{Y} &0 &Box A\\
&1 &Box B\\\hline
\end{tabular}
\caption{}
\label{tab:Distribution}
\end{table}
See \tabref{tab:Distribution}.
\begin{align}
p_{Y}\brak{k}= \begin{cases} 
      \frac{1}{3} & {k=0} \\
      \frac{2}{3 }& {k=1} 
   \end{cases}
   \\
p_{Y|X}\brak{0|0} = \frac{19}{25}\, 
p_{Y|X}\brak{0|1} = \frac{6}{25}\,
p_{Y|X}\brak{1|0} = \frac{45}{50}\,
p_{Y|X}\brak{1|2} = \frac{5}{50}
\end{align}
The desired probability is the probability that a slip drawn at random is marked other than Rs 1,
\begin{align}
&=1-p_X\brak{0}\\
&= p_X(1) + p_X(2)
\end{align}
Using Bayes theorem,
\begin{align}
&= p_Y\brak{0} \times \pr{Y=0 | X=1} + p_Y\brak{1} \times \pr{Y=1|X=2}\\
&=\frac{1}{3} \times \frac{6}{25} + \frac{2}{3} \times \frac{5}{50}\\
&=\frac{11}{75}
\end{align}

\newpage

%\tableofcontents

\bigskip

\renewcommand{\thefigure}{\theenumi}
\renewcommand{\thetable}{\theenumi}
%\renewcommand{\theequation}{\theenumi}

%\begin{abstract}
%%\boldmath
%In this letter, an algorithm for evaluating the exact analytical bit error rate  (BER)  for the piecewise linear (PL) combiner for  multiple relays is presented. Previous results were available only for upto three relays. The algorithm is unique in the sense that  the actual mathematical expressions, that are prohibitively large, need not be explicitly obtained. The diversity gain due to multiple relays is shown through plots of the analytical BER, well supported by simulations. 
%
%\end{abstract}
% IEEEtran.cls defaults to using nonbold math in the Abstract.
% This preserves the distinction between vectors and scalars. However,
% if the journal you are submitting to favors bold math in the abstract,
% then you can use LaTeX's standard command \boldmath at the very start
% of the abstract to achieve this. Many IEEE journals frown on math
% in the abstract anyway.

% Note that keywords are not normally used for peerreview papers.
%\begin{IEEEkeywords}
%Cooperative diversity, decode and forward, piecewise linear
%\end{IEEEkeywords}



% For peer review papers, you can put extra information on the cover
% page as needed:
% \ifCLASSOPTIONpeerreview
% \begin{center} \bfseries EDICS Category: 3-BBND \end{center}
% \fi
%
% For peerreview papers, this IEEEtran command inserts a page break and
% creates the second title. It will be ignored for other modes.
%\IEEEpeerreviewmaketitle




 \item A student says that if you throw a die, it will show up 1 or not 1. Therefore, the probability of getting 1 and the probability of getting 'not 1' each is equal to $\frac{1}{2}$. Is this correct? Give reasons.\\
 \solution
        %\begin{table}[H]
	\centering
\begin{tabular}{|c|c|c|}
\hline
Random variable &Value &Definition\\ \hline
\multirow{3}{*}{X} &0 &Slips of Rs 1\\
&1 &Slips of Rs 5\\
&2 &Slips of Rs 13\\ \hline
\multirow{2}{*}{Y} &0 &Box A\\
&1 &Box B\\\hline
\end{tabular}
\caption{}
\label{tab:Distribution}
\end{table}
See \tabref{tab:Distribution}.
\begin{align}
p_{Y}\brak{k}= \begin{cases} 
      \frac{1}{3} & {k=0} \\
      \frac{2}{3 }& {k=1} 
   \end{cases}
   \\
p_{Y|X}\brak{0|0} = \frac{19}{25}\, 
p_{Y|X}\brak{0|1} = \frac{6}{25}\,
p_{Y|X}\brak{1|0} = \frac{45}{50}\,
p_{Y|X}\brak{1|2} = \frac{5}{50}
\end{align}
The desired probability is the probability that a slip drawn at random is marked other than Rs 1,
\begin{align}
&=1-p_X\brak{0}\\
&= p_X(1) + p_X(2)
\end{align}
Using Bayes theorem,
\begin{align}
&= p_Y\brak{0} \times \pr{Y=0 | X=1} + p_Y\brak{1} \times \pr{Y=1|X=2}\\
&=\frac{1}{3} \times \frac{6}{25} + \frac{2}{3} \times \frac{5}{50}\\
&=\frac{11}{75}
\end{align}

\newpage

%\tableofcontents

\bigskip

\renewcommand{\thefigure}{\theenumi}
\renewcommand{\thetable}{\theenumi}
%\renewcommand{\theequation}{\theenumi}

%\begin{abstract}
%%\boldmath
%In this letter, an algorithm for evaluating the exact analytical bit error rate  (BER)  for the piecewise linear (PL) combiner for  multiple relays is presented. Previous results were available only for upto three relays. The algorithm is unique in the sense that  the actual mathematical expressions, that are prohibitively large, need not be explicitly obtained. The diversity gain due to multiple relays is shown through plots of the analytical BER, well supported by simulations. 
%
%\end{abstract}
% IEEEtran.cls defaults to using nonbold math in the Abstract.
% This preserves the distinction between vectors and scalars. However,
% if the journal you are submitting to favors bold math in the abstract,
% then you can use LaTeX's standard command \boldmath at the very start
% of the abstract to achieve this. Many IEEE journals frown on math
% in the abstract anyway.

% Note that keywords are not normally used for peerreview papers.
%\begin{IEEEkeywords}
%Cooperative diversity, decode and forward, piecewise linear
%\end{IEEEkeywords}



% For peer review papers, you can put extra information on the cover
% page as needed:
% \ifCLASSOPTIONpeerreview
% \begin{center} \bfseries EDICS Category: 3-BBND \end{center}
% \fi
%
% For peerreview papers, this IEEEtran command inserts a page break and
% creates the second title. It will be ignored for other modes.
%\IEEEpeerreviewmaketitle




   \item Four candidates A, B, C, D have ap-
plied for the assignment to coach a school cricket
team. If A is twice as likely to be selected as B, and
B and C are given about the same chance of being
selected, while C is twice as likely to be selected
as D, what are the probabilities that
\begin{enumerate}
\item C will be selected?
\item A will not be selected?
\end{enumerate}
	%\begin{table}[H]
	\centering
\begin{tabular}{|c|c|c|}
\hline
Random variable &Value &Definition\\ \hline
\multirow{3}{*}{X} &0 &Slips of Rs 1\\
&1 &Slips of Rs 5\\
&2 &Slips of Rs 13\\ \hline
\multirow{2}{*}{Y} &0 &Box A\\
&1 &Box B\\\hline
\end{tabular}
\caption{}
\label{tab:Distribution}
\end{table}
See \tabref{tab:Distribution}.
\begin{align}
p_{Y}\brak{k}= \begin{cases} 
      \frac{1}{3} & {k=0} \\
      \frac{2}{3 }& {k=1} 
   \end{cases}
   \\
p_{Y|X}\brak{0|0} = \frac{19}{25}\, 
p_{Y|X}\brak{0|1} = \frac{6}{25}\,
p_{Y|X}\brak{1|0} = \frac{45}{50}\,
p_{Y|X}\brak{1|2} = \frac{5}{50}
\end{align}
The desired probability is the probability that a slip drawn at random is marked other than Rs 1,
\begin{align}
&=1-p_X\brak{0}\\
&= p_X(1) + p_X(2)
\end{align}
Using Bayes theorem,
\begin{align}
&= p_Y\brak{0} \times \pr{Y=0 | X=1} + p_Y\brak{1} \times \pr{Y=1|X=2}\\
&=\frac{1}{3} \times \frac{6}{25} + \frac{2}{3} \times \frac{5}{50}\\
&=\frac{11}{75}
\end{align}

\newpage

%\tableofcontents

\bigskip

\renewcommand{\thefigure}{\theenumi}
\renewcommand{\thetable}{\theenumi}
%\renewcommand{\theequation}{\theenumi}

%\begin{abstract}
%%\boldmath
%In this letter, an algorithm for evaluating the exact analytical bit error rate  (BER)  for the piecewise linear (PL) combiner for  multiple relays is presented. Previous results were available only for upto three relays. The algorithm is unique in the sense that  the actual mathematical expressions, that are prohibitively large, need not be explicitly obtained. The diversity gain due to multiple relays is shown through plots of the analytical BER, well supported by simulations. 
%
%\end{abstract}
% IEEEtran.cls defaults to using nonbold math in the Abstract.
% This preserves the distinction between vectors and scalars. However,
% if the journal you are submitting to favors bold math in the abstract,
% then you can use LaTeX's standard command \boldmath at the very start
% of the abstract to achieve this. Many IEEE journals frown on math
% in the abstract anyway.

% Note that keywords are not normally used for peerreview papers.
%\begin{IEEEkeywords}
%Cooperative diversity, decode and forward, piecewise linear
%\end{IEEEkeywords}



% For peer review papers, you can put extra information on the cover
% page as needed:
% \ifCLASSOPTIONpeerreview
% \begin{center} \bfseries EDICS Category: 3-BBND \end{center}
% \fi
%
% For peerreview papers, this IEEEtran command inserts a page break and
% creates the second title. It will be ignored for other modes.
%\IEEEpeerreviewmaketitle




 \item A bag contain 24 balls of which $x$ balls are red, $2x$ are white and $3x$ are blue. A ball is selected at random, What is the probability that it is
\begin{enumerate}[label=\alph*)]
\item not red ?
\item white ?
\end{enumerate}
%\begin{table}[H]
	\centering
\begin{tabular}{|c|c|c|}
\hline
Random variable &Value &Definition\\ \hline
\multirow{3}{*}{X} &0 &Slips of Rs 1\\
&1 &Slips of Rs 5\\
&2 &Slips of Rs 13\\ \hline
\multirow{2}{*}{Y} &0 &Box A\\
&1 &Box B\\\hline
\end{tabular}
\caption{}
\label{tab:Distribution}
\end{table}
See \tabref{tab:Distribution}.
\begin{align}
p_{Y}\brak{k}= \begin{cases} 
      \frac{1}{3} & {k=0} \\
      \frac{2}{3 }& {k=1} 
   \end{cases}
   \\
p_{Y|X}\brak{0|0} = \frac{19}{25}\, 
p_{Y|X}\brak{0|1} = \frac{6}{25}\,
p_{Y|X}\brak{1|0} = \frac{45}{50}\,
p_{Y|X}\brak{1|2} = \frac{5}{50}
\end{align}
The desired probability is the probability that a slip drawn at random is marked other than Rs 1,
\begin{align}
&=1-p_X\brak{0}\\
&= p_X(1) + p_X(2)
\end{align}
Using Bayes theorem,
\begin{align}
&= p_Y\brak{0} \times \pr{Y=0 | X=1} + p_Y\brak{1} \times \pr{Y=1|X=2}\\
&=\frac{1}{3} \times \frac{6}{25} + \frac{2}{3} \times \frac{5}{50}\\
&=\frac{11}{75}
\end{align}

\newpage

%\tableofcontents

\bigskip

\renewcommand{\thefigure}{\theenumi}
\renewcommand{\thetable}{\theenumi}
%\renewcommand{\theequation}{\theenumi}

%\begin{abstract}
%%\boldmath
%In this letter, an algorithm for evaluating the exact analytical bit error rate  (BER)  for the piecewise linear (PL) combiner for  multiple relays is presented. Previous results were available only for upto three relays. The algorithm is unique in the sense that  the actual mathematical expressions, that are prohibitively large, need not be explicitly obtained. The diversity gain due to multiple relays is shown through plots of the analytical BER, well supported by simulations. 
%
%\end{abstract}
% IEEEtran.cls defaults to using nonbold math in the Abstract.
% This preserves the distinction between vectors and scalars. However,
% if the journal you are submitting to favors bold math in the abstract,
% then you can use LaTeX's standard command \boldmath at the very start
% of the abstract to achieve this. Many IEEE journals frown on math
% in the abstract anyway.

% Note that keywords are not normally used for peerreview papers.
%\begin{IEEEkeywords}
%Cooperative diversity, decode and forward, piecewise linear
%\end{IEEEkeywords}



% For peer review papers, you can put extra information on the cover
% page as needed:
% \ifCLASSOPTIONpeerreview
% \begin{center} \bfseries EDICS Category: 3-BBND \end{center}
% \fi
%
% For peerreview papers, this IEEEtran command inserts a page break and
% creates the second title. It will be ignored for other modes.
%\IEEEpeerreviewmaketitle




If the letters of the word ASSASSINATION are arranged at random. Find the Probability that
\begin{enumerate}[label=(\alph*)]
\item Four $S's$ come consecutively in the word
\item Two  $I's$ and two $N's$ come together
\item All $A's$ are not coming together
\item No two $A's$ are coming together
\end{enumerate}
%\begin{table}[H]
	\centering
\begin{tabular}{|c|c|c|}
\hline
Random variable &Value &Definition\\ \hline
\multirow{3}{*}{X} &0 &Slips of Rs 1\\
&1 &Slips of Rs 5\\
&2 &Slips of Rs 13\\ \hline
\multirow{2}{*}{Y} &0 &Box A\\
&1 &Box B\\\hline
\end{tabular}
\caption{}
\label{tab:Distribution}
\end{table}
See \tabref{tab:Distribution}.
\begin{align}
p_{Y}\brak{k}= \begin{cases} 
      \frac{1}{3} & {k=0} \\
      \frac{2}{3 }& {k=1} 
   \end{cases}
   \\
p_{Y|X}\brak{0|0} = \frac{19}{25}\, 
p_{Y|X}\brak{0|1} = \frac{6}{25}\,
p_{Y|X}\brak{1|0} = \frac{45}{50}\,
p_{Y|X}\brak{1|2} = \frac{5}{50}
\end{align}
The desired probability is the probability that a slip drawn at random is marked other than Rs 1,
\begin{align}
&=1-p_X\brak{0}\\
&= p_X(1) + p_X(2)
\end{align}
Using Bayes theorem,
\begin{align}
&= p_Y\brak{0} \times \pr{Y=0 | X=1} + p_Y\brak{1} \times \pr{Y=1|X=2}\\
&=\frac{1}{3} \times \frac{6}{25} + \frac{2}{3} \times \frac{5}{50}\\
&=\frac{11}{75}
\end{align}

\newpage

%\tableofcontents

\bigskip

\renewcommand{\thefigure}{\theenumi}
\renewcommand{\thetable}{\theenumi}
%\renewcommand{\theequation}{\theenumi}

%\begin{abstract}
%%\boldmath
%In this letter, an algorithm for evaluating the exact analytical bit error rate  (BER)  for the piecewise linear (PL) combiner for  multiple relays is presented. Previous results were available only for upto three relays. The algorithm is unique in the sense that  the actual mathematical expressions, that are prohibitively large, need not be explicitly obtained. The diversity gain due to multiple relays is shown through plots of the analytical BER, well supported by simulations. 
%
%\end{abstract}
% IEEEtran.cls defaults to using nonbold math in the Abstract.
% This preserves the distinction between vectors and scalars. However,
% if the journal you are submitting to favors bold math in the abstract,
% then you can use LaTeX's standard command \boldmath at the very start
% of the abstract to achieve this. Many IEEE journals frown on math
% in the abstract anyway.

% Note that keywords are not normally used for peerreview papers.
%\begin{IEEEkeywords}
%Cooperative diversity, decode and forward, piecewise linear
%\end{IEEEkeywords}



% For peer review papers, you can put extra information on the cover
% page as needed:
% \ifCLASSOPTIONpeerreview
% \begin{center} \bfseries EDICS Category: 3-BBND \end{center}
% \fi
%
% For peerreview papers, this IEEEtran command inserts a page break and
% creates the second title. It will be ignored for other modes.
%\IEEEpeerreviewmaketitle




	\item One urn contains two black balls (labelled B1 and B2) and one white ball. A
	second urn contains one black ball and two white balls (labelled W1 and W2).
	Suppose the following experiment is performed. One of the two urns is chosen
	at random. Next a ball is randomly chosen from the urn. Then a second ball is
	chosen at random from the same urn without replacing the first ball.
	
	\begin{enumerate}
	\item What is the probability that two black balls are chosen?
	
	\item What is the probability that two balls of opposite colour are chosen?
	\end{enumerate}
	\solution
	%\begin{align}
    \label{eq:12.13.6.18.1}
	\because	\pr{A|B} &> \pr{A},\
\frac{\pr{AB}}{\pr{B}} > \pr{A}
\\
    \label{eq:12.13.6.18.2}
	\implies \pr{AB} &> \pr{A}\pr{B}
	\\
	\text{or, } \frac{\pr{AB}}{\pr{A}} &=\pr{B|A} > \pr{A}
\end{align}

\end{enumerate}

	\item A bag contains 4 red and 4 black balls, another bag contains 2 red and 6 black balls. One of the two bags is selected at random and a ball is drawn from the bag which is found to be red. Find the probability that the ball is drawn from the first bag.
\\
\solution
		%\begin{table}[H]
	\centering
\begin{tabular}{|c|c|c|}
\hline
Random variable &Value &Definition\\ \hline
\multirow{3}{*}{X} &0 &Slips of Rs 1\\
&1 &Slips of Rs 5\\
&2 &Slips of Rs 13\\ \hline
\multirow{2}{*}{Y} &0 &Box A\\
&1 &Box B\\\hline
\end{tabular}
\caption{}
\label{tab:Distribution}
\end{table}
See \tabref{tab:Distribution}.
\begin{align}
p_{Y}\brak{k}= \begin{cases} 
      \frac{1}{3} & {k=0} \\
      \frac{2}{3 }& {k=1} 
   \end{cases}
   \\
p_{Y|X}\brak{0|0} = \frac{19}{25}\, 
p_{Y|X}\brak{0|1} = \frac{6}{25}\,
p_{Y|X}\brak{1|0} = \frac{45}{50}\,
p_{Y|X}\brak{1|2} = \frac{5}{50}
\end{align}
The desired probability is the probability that a slip drawn at random is marked other than Rs 1,
\begin{align}
&=1-p_X\brak{0}\\
&= p_X(1) + p_X(2)
\end{align}
Using Bayes theorem,
\begin{align}
&= p_Y\brak{0} \times \pr{Y=0 | X=1} + p_Y\brak{1} \times \pr{Y=1|X=2}\\
&=\frac{1}{3} \times \frac{6}{25} + \frac{2}{3} \times \frac{5}{50}\\
&=\frac{11}{75}
\end{align}

\newpage

%\tableofcontents

\bigskip

\renewcommand{\thefigure}{\theenumi}
\renewcommand{\thetable}{\theenumi}
%\renewcommand{\theequation}{\theenumi}

%\begin{abstract}
%%\boldmath
%In this letter, an algorithm for evaluating the exact analytical bit error rate  (BER)  for the piecewise linear (PL) combiner for  multiple relays is presented. Previous results were available only for upto three relays. The algorithm is unique in the sense that  the actual mathematical expressions, that are prohibitively large, need not be explicitly obtained. The diversity gain due to multiple relays is shown through plots of the analytical BER, well supported by simulations. 
%
%\end{abstract}
% IEEEtran.cls defaults to using nonbold math in the Abstract.
% This preserves the distinction between vectors and scalars. However,
% if the journal you are submitting to favors bold math in the abstract,
% then you can use LaTeX's standard command \boldmath at the very start
% of the abstract to achieve this. Many IEEE journals frown on math
% in the abstract anyway.

% Note that keywords are not normally used for peerreview papers.
%\begin{IEEEkeywords}
%Cooperative diversity, decode and forward, piecewise linear
%\end{IEEEkeywords}



% For peer review papers, you can put extra information on the cover
% page as needed:
% \ifCLASSOPTIONpeerreview
% \begin{center} \bfseries EDICS Category: 3-BBND \end{center}
% \fi
%
% For peerreview papers, this IEEEtran command inserts a page break and
% creates the second title. It will be ignored for other modes.
%\IEEEpeerreviewmaketitle




  \item
  Cards with numbers 2 to 101 are placed in a box. A card is selected at random.Find the probability that the card has
\begin{enumerate}[label=(\roman*)]
	\item an even number 
	\item a square number
\end{enumerate}
\solution
%\begin{table}[H]
	\centering
\begin{tabular}{|c|c|c|}
\hline
Random variable &Value &Definition\\ \hline
\multirow{3}{*}{X} &0 &Slips of Rs 1\\
&1 &Slips of Rs 5\\
&2 &Slips of Rs 13\\ \hline
\multirow{2}{*}{Y} &0 &Box A\\
&1 &Box B\\\hline
\end{tabular}
\caption{}
\label{tab:Distribution}
\end{table}
See \tabref{tab:Distribution}.
\begin{align}
p_{Y}\brak{k}= \begin{cases} 
      \frac{1}{3} & {k=0} \\
      \frac{2}{3 }& {k=1} 
   \end{cases}
   \\
p_{Y|X}\brak{0|0} = \frac{19}{25}\, 
p_{Y|X}\brak{0|1} = \frac{6}{25}\,
p_{Y|X}\brak{1|0} = \frac{45}{50}\,
p_{Y|X}\brak{1|2} = \frac{5}{50}
\end{align}
The desired probability is the probability that a slip drawn at random is marked other than Rs 1,
\begin{align}
&=1-p_X\brak{0}\\
&= p_X(1) + p_X(2)
\end{align}
Using Bayes theorem,
\begin{align}
&= p_Y\brak{0} \times \pr{Y=0 | X=1} + p_Y\brak{1} \times \pr{Y=1|X=2}\\
&=\frac{1}{3} \times \frac{6}{25} + \frac{2}{3} \times \frac{5}{50}\\
&=\frac{11}{75}
\end{align}

\newpage

%\tableofcontents

\bigskip

\renewcommand{\thefigure}{\theenumi}
\renewcommand{\thetable}{\theenumi}
%\renewcommand{\theequation}{\theenumi}

%\begin{abstract}
%%\boldmath
%In this letter, an algorithm for evaluating the exact analytical bit error rate  (BER)  for the piecewise linear (PL) combiner for  multiple relays is presented. Previous results were available only for upto three relays. The algorithm is unique in the sense that  the actual mathematical expressions, that are prohibitively large, need not be explicitly obtained. The diversity gain due to multiple relays is shown through plots of the analytical BER, well supported by simulations. 
%
%\end{abstract}
% IEEEtran.cls defaults to using nonbold math in the Abstract.
% This preserves the distinction between vectors and scalars. However,
% if the journal you are submitting to favors bold math in the abstract,
% then you can use LaTeX's standard command \boldmath at the very start
% of the abstract to achieve this. Many IEEE journals frown on math
% in the abstract anyway.

% Note that keywords are not normally used for peerreview papers.
%\begin{IEEEkeywords}
%Cooperative diversity, decode and forward, piecewise linear
%\end{IEEEkeywords}



% For peer review papers, you can put extra information on the cover
% page as needed:
% \ifCLASSOPTIONpeerreview
% \begin{center} \bfseries EDICS Category: 3-BBND \end{center}
% \fi
%
% For peerreview papers, this IEEEtran command inserts a page break and
% creates the second title. It will be ignored for other modes.
%\IEEEpeerreviewmaketitle




\item
The king, queen and jack of clubs are removed from a deck of 52 playing cards and then well shuffled. Now one card is drawn at random from the remaining cards.  Determine the probability that the card is
\begin{enumerate}[label=(\roman*)]
\item a club
\item 10 of hearts
\end{enumerate}
\solution
%\begin{table}[H]
	\centering
\begin{tabular}{|c|c|c|}
\hline
Random variable &Value &Definition\\ \hline
\multirow{3}{*}{X} &0 &Slips of Rs 1\\
&1 &Slips of Rs 5\\
&2 &Slips of Rs 13\\ \hline
\multirow{2}{*}{Y} &0 &Box A\\
&1 &Box B\\\hline
\end{tabular}
\caption{}
\label{tab:Distribution}
\end{table}
See \tabref{tab:Distribution}.
\begin{align}
p_{Y}\brak{k}= \begin{cases} 
      \frac{1}{3} & {k=0} \\
      \frac{2}{3 }& {k=1} 
   \end{cases}
   \\
p_{Y|X}\brak{0|0} = \frac{19}{25}\, 
p_{Y|X}\brak{0|1} = \frac{6}{25}\,
p_{Y|X}\brak{1|0} = \frac{45}{50}\,
p_{Y|X}\brak{1|2} = \frac{5}{50}
\end{align}
The desired probability is the probability that a slip drawn at random is marked other than Rs 1,
\begin{align}
&=1-p_X\brak{0}\\
&= p_X(1) + p_X(2)
\end{align}
Using Bayes theorem,
\begin{align}
&= p_Y\brak{0} \times \pr{Y=0 | X=1} + p_Y\brak{1} \times \pr{Y=1|X=2}\\
&=\frac{1}{3} \times \frac{6}{25} + \frac{2}{3} \times \frac{5}{50}\\
&=\frac{11}{75}
\end{align}

\newpage

%\tableofcontents

\bigskip

\renewcommand{\thefigure}{\theenumi}
\renewcommand{\thetable}{\theenumi}
%\renewcommand{\theequation}{\theenumi}

%\begin{abstract}
%%\boldmath
%In this letter, an algorithm for evaluating the exact analytical bit error rate  (BER)  for the piecewise linear (PL) combiner for  multiple relays is presented. Previous results were available only for upto three relays. The algorithm is unique in the sense that  the actual mathematical expressions, that are prohibitively large, need not be explicitly obtained. The diversity gain due to multiple relays is shown through plots of the analytical BER, well supported by simulations. 
%
%\end{abstract}
% IEEEtran.cls defaults to using nonbold math in the Abstract.
% This preserves the distinction between vectors and scalars. However,
% if the journal you are submitting to favors bold math in the abstract,
% then you can use LaTeX's standard command \boldmath at the very start
% of the abstract to achieve this. Many IEEE journals frown on math
% in the abstract anyway.

% Note that keywords are not normally used for peerreview papers.
%\begin{IEEEkeywords}
%Cooperative diversity, decode and forward, piecewise linear
%\end{IEEEkeywords}



% For peer review papers, you can put extra information on the cover
% page as needed:
% \ifCLASSOPTIONpeerreview
% \begin{center} \bfseries EDICS Category: 3-BBND \end{center}
% \fi
%
% For peerreview papers, this IEEEtran command inserts a page break and
% creates the second title. It will be ignored for other modes.
%\IEEEpeerreviewmaketitle




\item A team of medical students doing their internship have to assist during surgeries
at a city hospital. The probabilities of surgeries rated as very complex, complex,
routine, simple or very simple are respectively, 0.15, 0.20, 0.31, 0.26, .08. Find
the probabilities that a particular surgery will be rated
\begin{enumerate}
	\item complex or very complex;
	\item neither very complex nor very simple;
	\item routine or complex
	\item routine or simple
\end{enumerate}
\solution
%\begin{table}[H]
	\centering
\begin{tabular}{|c|c|c|}
\hline
Random variable &Value &Definition\\ \hline
\multirow{3}{*}{X} &0 &Slips of Rs 1\\
&1 &Slips of Rs 5\\
&2 &Slips of Rs 13\\ \hline
\multirow{2}{*}{Y} &0 &Box A\\
&1 &Box B\\\hline
\end{tabular}
\caption{}
\label{tab:Distribution}
\end{table}
See \tabref{tab:Distribution}.
\begin{align}
p_{Y}\brak{k}= \begin{cases} 
      \frac{1}{3} & {k=0} \\
      \frac{2}{3 }& {k=1} 
   \end{cases}
   \\
p_{Y|X}\brak{0|0} = \frac{19}{25}\, 
p_{Y|X}\brak{0|1} = \frac{6}{25}\,
p_{Y|X}\brak{1|0} = \frac{45}{50}\,
p_{Y|X}\brak{1|2} = \frac{5}{50}
\end{align}
The desired probability is the probability that a slip drawn at random is marked other than Rs 1,
\begin{align}
&=1-p_X\brak{0}\\
&= p_X(1) + p_X(2)
\end{align}
Using Bayes theorem,
\begin{align}
&= p_Y\brak{0} \times \pr{Y=0 | X=1} + p_Y\brak{1} \times \pr{Y=1|X=2}\\
&=\frac{1}{3} \times \frac{6}{25} + \frac{2}{3} \times \frac{5}{50}\\
&=\frac{11}{75}
\end{align}

\newpage

%\tableofcontents

\bigskip

\renewcommand{\thefigure}{\theenumi}
\renewcommand{\thetable}{\theenumi}
%\renewcommand{\theequation}{\theenumi}

%\begin{abstract}
%%\boldmath
%In this letter, an algorithm for evaluating the exact analytical bit error rate  (BER)  for the piecewise linear (PL) combiner for  multiple relays is presented. Previous results were available only for upto three relays. The algorithm is unique in the sense that  the actual mathematical expressions, that are prohibitively large, need not be explicitly obtained. The diversity gain due to multiple relays is shown through plots of the analytical BER, well supported by simulations. 
%
%\end{abstract}
% IEEEtran.cls defaults to using nonbold math in the Abstract.
% This preserves the distinction between vectors and scalars. However,
% if the journal you are submitting to favors bold math in the abstract,
% then you can use LaTeX's standard command \boldmath at the very start
% of the abstract to achieve this. Many IEEE journals frown on math
% in the abstract anyway.

% Note that keywords are not normally used for peerreview papers.
%\begin{IEEEkeywords}
%Cooperative diversity, decode and forward, piecewise linear
%\end{IEEEkeywords}



% For peer review papers, you can put extra information on the cover
% page as needed:
% \ifCLASSOPTIONpeerreview
% \begin{center} \bfseries EDICS Category: 3-BBND \end{center}
% \fi
%
% For peerreview papers, this IEEEtran command inserts a page break and
% creates the second title. It will be ignored for other modes.
%\IEEEpeerreviewmaketitle




\item A card is selected from a pack of 52 cards.
\begin{enumerate}[label=(\alph*)]
    \item How many points are there in the sample space?
    \item Calculate the probability that the card is an ace of spades.
    \item Calculate the probability that the card is (i) an ace and (ii) black card.
\end{enumerate}
\solution
%Let $X$ be an bernoulli rv defined as in \tabref{tab:exemplar/11/16/3/26}.  Then, 
\begin{equation}
    p =
        \frac{4}{11} 
\end{equation}
\begin{table}[H]
	\centering
	\input{exemplar/11/16/3/26/tables/Table2.tex}
	\caption{}
        \label{tab:exemplar/11/16/3/26}
\end{table}

\item The probability that a non leap year selected at random will contain 53 sundays.
\\
\solution
%\begin{table}[H]
	\centering
\begin{tabular}{|c|c|c|}
\hline
Random variable &Value &Definition\\ \hline
\multirow{3}{*}{X} &0 &Slips of Rs 1\\
&1 &Slips of Rs 5\\
&2 &Slips of Rs 13\\ \hline
\multirow{2}{*}{Y} &0 &Box A\\
&1 &Box B\\\hline
\end{tabular}
\caption{}
\label{tab:Distribution}
\end{table}
See \tabref{tab:Distribution}.
\begin{align}
p_{Y}\brak{k}= \begin{cases} 
      \frac{1}{3} & {k=0} \\
      \frac{2}{3 }& {k=1} 
   \end{cases}
   \\
p_{Y|X}\brak{0|0} = \frac{19}{25}\, 
p_{Y|X}\brak{0|1} = \frac{6}{25}\,
p_{Y|X}\brak{1|0} = \frac{45}{50}\,
p_{Y|X}\brak{1|2} = \frac{5}{50}
\end{align}
The desired probability is the probability that a slip drawn at random is marked other than Rs 1,
\begin{align}
&=1-p_X\brak{0}\\
&= p_X(1) + p_X(2)
\end{align}
Using Bayes theorem,
\begin{align}
&= p_Y\brak{0} \times \pr{Y=0 | X=1} + p_Y\brak{1} \times \pr{Y=1|X=2}\\
&=\frac{1}{3} \times \frac{6}{25} + \frac{2}{3} \times \frac{5}{50}\\
&=\frac{11}{75}
\end{align}

\newpage

%\tableofcontents

\bigskip

\renewcommand{\thefigure}{\theenumi}
\renewcommand{\thetable}{\theenumi}
%\renewcommand{\theequation}{\theenumi}

%\begin{abstract}
%%\boldmath
%In this letter, an algorithm for evaluating the exact analytical bit error rate  (BER)  for the piecewise linear (PL) combiner for  multiple relays is presented. Previous results were available only for upto three relays. The algorithm is unique in the sense that  the actual mathematical expressions, that are prohibitively large, need not be explicitly obtained. The diversity gain due to multiple relays is shown through plots of the analytical BER, well supported by simulations. 
%
%\end{abstract}
% IEEEtran.cls defaults to using nonbold math in the Abstract.
% This preserves the distinction between vectors and scalars. However,
% if the journal you are submitting to favors bold math in the abstract,
% then you can use LaTeX's standard command \boldmath at the very start
% of the abstract to achieve this. Many IEEE journals frown on math
% in the abstract anyway.

% Note that keywords are not normally used for peerreview papers.
%\begin{IEEEkeywords}
%Cooperative diversity, decode and forward, piecewise linear
%\end{IEEEkeywords}



% For peer review papers, you can put extra information on the cover
% page as needed:
% \ifCLASSOPTIONpeerreview
% \begin{center} \bfseries EDICS Category: 3-BBND \end{center}
% \fi
%
% For peerreview papers, this IEEEtran command inserts a page break and
% creates the second title. It will be ignored for other modes.
%\IEEEpeerreviewmaketitle




\item One of the four persons John, Rita, Aslam or Gurpreet will be promoted next
month. Consequently the sample space consists of four elementary outcomes
S = {John promoted, Rita promoted, Aslam promoted, Gurpreet promoted}
You are told that the chances of John’s promotion is same as that of Gurpreet,
Rita’s chances of promotion are twice as likely as Johns. Aslam’s chances are
four times that of John.
\begin{enumerate}
	\item Determine
	\begin{enumerate}
		\item P (John promoted)
		\item P (Rita promoted)
		\item P (Aslam promoted)
		\item P (Gurpreet promoted)
	\end{enumerate}
	\item If A = {John promoted or Gurpreet promoted}, find P (A).
\end{enumerate}
\solution
%\begin{table}[H]
	\centering
\begin{tabular}{|c|c|c|}
\hline
Random variable &Value &Definition\\ \hline
\multirow{3}{*}{X} &0 &Slips of Rs 1\\
&1 &Slips of Rs 5\\
&2 &Slips of Rs 13\\ \hline
\multirow{2}{*}{Y} &0 &Box A\\
&1 &Box B\\\hline
\end{tabular}
\caption{}
\label{tab:Distribution}
\end{table}
See \tabref{tab:Distribution}.
\begin{align}
p_{Y}\brak{k}= \begin{cases} 
      \frac{1}{3} & {k=0} \\
      \frac{2}{3 }& {k=1} 
   \end{cases}
   \\
p_{Y|X}\brak{0|0} = \frac{19}{25}\, 
p_{Y|X}\brak{0|1} = \frac{6}{25}\,
p_{Y|X}\brak{1|0} = \frac{45}{50}\,
p_{Y|X}\brak{1|2} = \frac{5}{50}
\end{align}
The desired probability is the probability that a slip drawn at random is marked other than Rs 1,
\begin{align}
&=1-p_X\brak{0}\\
&= p_X(1) + p_X(2)
\end{align}
Using Bayes theorem,
\begin{align}
&= p_Y\brak{0} \times \pr{Y=0 | X=1} + p_Y\brak{1} \times \pr{Y=1|X=2}\\
&=\frac{1}{3} \times \frac{6}{25} + \frac{2}{3} \times \frac{5}{50}\\
&=\frac{11}{75}
\end{align}

\newpage

%\tableofcontents

\bigskip

\renewcommand{\thefigure}{\theenumi}
\renewcommand{\thetable}{\theenumi}
%\renewcommand{\theequation}{\theenumi}

%\begin{abstract}
%%\boldmath
%In this letter, an algorithm for evaluating the exact analytical bit error rate  (BER)  for the piecewise linear (PL) combiner for  multiple relays is presented. Previous results were available only for upto three relays. The algorithm is unique in the sense that  the actual mathematical expressions, that are prohibitively large, need not be explicitly obtained. The diversity gain due to multiple relays is shown through plots of the analytical BER, well supported by simulations. 
%
%\end{abstract}
% IEEEtran.cls defaults to using nonbold math in the Abstract.
% This preserves the distinction between vectors and scalars. However,
% if the journal you are submitting to favors bold math in the abstract,
% then you can use LaTeX's standard command \boldmath at the very start
% of the abstract to achieve this. Many IEEE journals frown on math
% in the abstract anyway.

% Note that keywords are not normally used for peerreview papers.
%\begin{IEEEkeywords}
%Cooperative diversity, decode and forward, piecewise linear
%\end{IEEEkeywords}



% For peer review papers, you can put extra information on the cover
% page as needed:
% \ifCLASSOPTIONpeerreview
% \begin{center} \bfseries EDICS Category: 3-BBND \end{center}
% \fi
%
% For peerreview papers, this IEEEtran command inserts a page break and
% creates the second title. It will be ignored for other modes.
%\IEEEpeerreviewmaketitle




\item A card is drawn from a deck of 52 cards. Find the probability of getting a king or a heart or a red card.\\
\solution
%\begin{table}[H]
	\centering
\begin{tabular}{|c|c|c|}
\hline
Random variable &Value &Definition\\ \hline
\multirow{3}{*}{X} &0 &Slips of Rs 1\\
&1 &Slips of Rs 5\\
&2 &Slips of Rs 13\\ \hline
\multirow{2}{*}{Y} &0 &Box A\\
&1 &Box B\\\hline
\end{tabular}
\caption{}
\label{tab:Distribution}
\end{table}
See \tabref{tab:Distribution}.
\begin{align}
p_{Y}\brak{k}= \begin{cases} 
      \frac{1}{3} & {k=0} \\
      \frac{2}{3 }& {k=1} 
   \end{cases}
   \\
p_{Y|X}\brak{0|0} = \frac{19}{25}\, 
p_{Y|X}\brak{0|1} = \frac{6}{25}\,
p_{Y|X}\brak{1|0} = \frac{45}{50}\,
p_{Y|X}\brak{1|2} = \frac{5}{50}
\end{align}
The desired probability is the probability that a slip drawn at random is marked other than Rs 1,
\begin{align}
&=1-p_X\brak{0}\\
&= p_X(1) + p_X(2)
\end{align}
Using Bayes theorem,
\begin{align}
&= p_Y\brak{0} \times \pr{Y=0 | X=1} + p_Y\brak{1} \times \pr{Y=1|X=2}\\
&=\frac{1}{3} \times \frac{6}{25} + \frac{2}{3} \times \frac{5}{50}\\
&=\frac{11}{75}
\end{align}

\newpage

%\tableofcontents

\bigskip

\renewcommand{\thefigure}{\theenumi}
\renewcommand{\thetable}{\theenumi}
%\renewcommand{\theequation}{\theenumi}

%\begin{abstract}
%%\boldmath
%In this letter, an algorithm for evaluating the exact analytical bit error rate  (BER)  for the piecewise linear (PL) combiner for  multiple relays is presented. Previous results were available only for upto three relays. The algorithm is unique in the sense that  the actual mathematical expressions, that are prohibitively large, need not be explicitly obtained. The diversity gain due to multiple relays is shown through plots of the analytical BER, well supported by simulations. 
%
%\end{abstract}
% IEEEtran.cls defaults to using nonbold math in the Abstract.
% This preserves the distinction between vectors and scalars. However,
% if the journal you are submitting to favors bold math in the abstract,
% then you can use LaTeX's standard command \boldmath at the very start
% of the abstract to achieve this. Many IEEE journals frown on math
% in the abstract anyway.

% Note that keywords are not normally used for peerreview papers.
%\begin{IEEEkeywords}
%Cooperative diversity, decode and forward, piecewise linear
%\end{IEEEkeywords}



% For peer review papers, you can put extra information on the cover
% page as needed:
% \ifCLASSOPTIONpeerreview
% \begin{center} \bfseries EDICS Category: 3-BBND \end{center}
% \fi
%
% For peerreview papers, this IEEEtran command inserts a page break and
% creates the second title. It will be ignored for other modes.
%\IEEEpeerreviewmaketitle




\item The probability that a student will pass his examination is 0.73, the probability of
the student getting a compartment is 0.13, and the probability that the student will
either pass or get compartment is 0.96. State True or False.\\
\solution
%\begin{table}[H]
	\centering
\begin{tabular}{|c|c|c|}
\hline
Random variable &Value &Definition\\ \hline
\multirow{3}{*}{X} &0 &Slips of Rs 1\\
&1 &Slips of Rs 5\\
&2 &Slips of Rs 13\\ \hline
\multirow{2}{*}{Y} &0 &Box A\\
&1 &Box B\\\hline
\end{tabular}
\caption{}
\label{tab:Distribution}
\end{table}
See \tabref{tab:Distribution}.
\begin{align}
p_{Y}\brak{k}= \begin{cases} 
      \frac{1}{3} & {k=0} \\
      \frac{2}{3 }& {k=1} 
   \end{cases}
   \\
p_{Y|X}\brak{0|0} = \frac{19}{25}\, 
p_{Y|X}\brak{0|1} = \frac{6}{25}\,
p_{Y|X}\brak{1|0} = \frac{45}{50}\,
p_{Y|X}\brak{1|2} = \frac{5}{50}
\end{align}
The desired probability is the probability that a slip drawn at random is marked other than Rs 1,
\begin{align}
&=1-p_X\brak{0}\\
&= p_X(1) + p_X(2)
\end{align}
Using Bayes theorem,
\begin{align}
&= p_Y\brak{0} \times \pr{Y=0 | X=1} + p_Y\brak{1} \times \pr{Y=1|X=2}\\
&=\frac{1}{3} \times \frac{6}{25} + \frac{2}{3} \times \frac{5}{50}\\
&=\frac{11}{75}
\end{align}

\newpage

%\tableofcontents

\bigskip

\renewcommand{\thefigure}{\theenumi}
\renewcommand{\thetable}{\theenumi}
%\renewcommand{\theequation}{\theenumi}

%\begin{abstract}
%%\boldmath
%In this letter, an algorithm for evaluating the exact analytical bit error rate  (BER)  for the piecewise linear (PL) combiner for  multiple relays is presented. Previous results were available only for upto three relays. The algorithm is unique in the sense that  the actual mathematical expressions, that are prohibitively large, need not be explicitly obtained. The diversity gain due to multiple relays is shown through plots of the analytical BER, well supported by simulations. 
%
%\end{abstract}
% IEEEtran.cls defaults to using nonbold math in the Abstract.
% This preserves the distinction between vectors and scalars. However,
% if the journal you are submitting to favors bold math in the abstract,
% then you can use LaTeX's standard command \boldmath at the very start
% of the abstract to achieve this. Many IEEE journals frown on math
% in the abstract anyway.

% Note that keywords are not normally used for peerreview papers.
%\begin{IEEEkeywords}
%Cooperative diversity, decode and forward, piecewise linear
%\end{IEEEkeywords}



% For peer review papers, you can put extra information on the cover
% page as needed:
% \ifCLASSOPTIONpeerreview
% \begin{center} \bfseries EDICS Category: 3-BBND \end{center}
% \fi
%
% For peerreview papers, this IEEEtran command inserts a page break and
% creates the second title. It will be ignored for other modes.
%\IEEEpeerreviewmaketitle




\item A card is selected from a pack of 52 cards\\
\begin{enumerate}[label=(\alph*)]
\item How many points are there in the sample space?
\item Calculate the probability that the cards is an ace of spades.
\item Calculate the probability that the card is (i) an ace (ii)black card.\\
\end{enumerate}
%\input{ncert/11/16/3/4_1/Prob_4.tex}
\item In a non-leap year, the probability of having 53 tuesdays or 53 wednesdays is\\
\solution
%A non-leap year has a total of 365 days, and a week has 7 days.\\
So it can be expressed as 
\begin{align}
365\text{days} &=52\times 7+1 \text{day}
\end{align}
$\implies$ 52 tuesdays or wednesdays\\
Random variable X denotes the days of a week
\begin{align}
p_X\brak{k}&=\frac{1}{7}; \quad \brak{1<k<7}
\end{align}
So the probability of extra day being tuesday or wednesday is
\begin{align}
p_X\brak{3}+p_X\brak{4}&=\frac{1}{7}+\frac{1}{7}=\frac{2}{7}
\end{align}



\item There are 1000 sealed envelopes in a box, 10 of them contain a cash prize of
Rs 100 each, 100 of them contain a cash prize of Rs 50 each and 200 of them
contain a cash prize of Rs 10 each and rest do not contain any cash prize. If they
are well shuffled and an envelope is picked up out, what is the probability that it
contains no cash prize?\\
\solution
%\begin{table}[H]
	\centering
\begin{tabular}{|c|c|c|}
\hline
Random variable &Value &Definition\\ \hline
\multirow{3}{*}{X} &0 &Slips of Rs 1\\
&1 &Slips of Rs 5\\
&2 &Slips of Rs 13\\ \hline
\multirow{2}{*}{Y} &0 &Box A\\
&1 &Box B\\\hline
\end{tabular}
\caption{}
\label{tab:Distribution}
\end{table}
See \tabref{tab:Distribution}.
\begin{align}
p_{Y}\brak{k}= \begin{cases} 
      \frac{1}{3} & {k=0} \\
      \frac{2}{3 }& {k=1} 
   \end{cases}
   \\
p_{Y|X}\brak{0|0} = \frac{19}{25}\, 
p_{Y|X}\brak{0|1} = \frac{6}{25}\,
p_{Y|X}\brak{1|0} = \frac{45}{50}\,
p_{Y|X}\brak{1|2} = \frac{5}{50}
\end{align}
The desired probability is the probability that a slip drawn at random is marked other than Rs 1,
\begin{align}
&=1-p_X\brak{0}\\
&= p_X(1) + p_X(2)
\end{align}
Using Bayes theorem,
\begin{align}
&= p_Y\brak{0} \times \pr{Y=0 | X=1} + p_Y\brak{1} \times \pr{Y=1|X=2}\\
&=\frac{1}{3} \times \frac{6}{25} + \frac{2}{3} \times \frac{5}{50}\\
&=\frac{11}{75}
\end{align}

\newpage

%\tableofcontents

\bigskip

\renewcommand{\thefigure}{\theenumi}
\renewcommand{\thetable}{\theenumi}
%\renewcommand{\theequation}{\theenumi}

%\begin{abstract}
%%\boldmath
%In this letter, an algorithm for evaluating the exact analytical bit error rate  (BER)  for the piecewise linear (PL) combiner for  multiple relays is presented. Previous results were available only for upto three relays. The algorithm is unique in the sense that  the actual mathematical expressions, that are prohibitively large, need not be explicitly obtained. The diversity gain due to multiple relays is shown through plots of the analytical BER, well supported by simulations. 
%
%\end{abstract}
% IEEEtran.cls defaults to using nonbold math in the Abstract.
% This preserves the distinction between vectors and scalars. However,
% if the journal you are submitting to favors bold math in the abstract,
% then you can use LaTeX's standard command \boldmath at the very start
% of the abstract to achieve this. Many IEEE journals frown on math
% in the abstract anyway.

% Note that keywords are not normally used for peerreview papers.
%\begin{IEEEkeywords}
%Cooperative diversity, decode and forward, piecewise linear
%\end{IEEEkeywords}



% For peer review papers, you can put extra information on the cover
% page as needed:
% \ifCLASSOPTIONpeerreview
% \begin{center} \bfseries EDICS Category: 3-BBND \end{center}
% \fi
%
% For peerreview papers, this IEEEtran command inserts a page break and
% creates the second title. It will be ignored for other modes.
%\IEEEpeerreviewmaketitle




\item 
A die is thrown and a card is selected at random from a deck of 52 playing cards. The probability of getting an even number on the die and a spade card.\\
\solution
%\begin{table}[H]
	\centering
\begin{tabular}{|c|c|c|}
\hline
Random variable &Value &Definition\\ \hline
\multirow{3}{*}{X} &0 &Slips of Rs 1\\
&1 &Slips of Rs 5\\
&2 &Slips of Rs 13\\ \hline
\multirow{2}{*}{Y} &0 &Box A\\
&1 &Box B\\\hline
\end{tabular}
\caption{}
\label{tab:Distribution}
\end{table}
See \tabref{tab:Distribution}.
\begin{align}
p_{Y}\brak{k}= \begin{cases} 
      \frac{1}{3} & {k=0} \\
      \frac{2}{3 }& {k=1} 
   \end{cases}
   \\
p_{Y|X}\brak{0|0} = \frac{19}{25}\, 
p_{Y|X}\brak{0|1} = \frac{6}{25}\,
p_{Y|X}\brak{1|0} = \frac{45}{50}\,
p_{Y|X}\brak{1|2} = \frac{5}{50}
\end{align}
The desired probability is the probability that a slip drawn at random is marked other than Rs 1,
\begin{align}
&=1-p_X\brak{0}\\
&= p_X(1) + p_X(2)
\end{align}
Using Bayes theorem,
\begin{align}
&= p_Y\brak{0} \times \pr{Y=0 | X=1} + p_Y\brak{1} \times \pr{Y=1|X=2}\\
&=\frac{1}{3} \times \frac{6}{25} + \frac{2}{3} \times \frac{5}{50}\\
&=\frac{11}{75}
\end{align}

\newpage

%\tableofcontents

\bigskip

\renewcommand{\thefigure}{\theenumi}
\renewcommand{\thetable}{\theenumi}
%\renewcommand{\theequation}{\theenumi}

%\begin{abstract}
%%\boldmath
%In this letter, an algorithm for evaluating the exact analytical bit error rate  (BER)  for the piecewise linear (PL) combiner for  multiple relays is presented. Previous results were available only for upto three relays. The algorithm is unique in the sense that  the actual mathematical expressions, that are prohibitively large, need not be explicitly obtained. The diversity gain due to multiple relays is shown through plots of the analytical BER, well supported by simulations. 
%
%\end{abstract}
% IEEEtran.cls defaults to using nonbold math in the Abstract.
% This preserves the distinction between vectors and scalars. However,
% if the journal you are submitting to favors bold math in the abstract,
% then you can use LaTeX's standard command \boldmath at the very start
% of the abstract to achieve this. Many IEEE journals frown on math
% in the abstract anyway.

% Note that keywords are not normally used for peerreview papers.
%\begin{IEEEkeywords}
%Cooperative diversity, decode and forward, piecewise linear
%\end{IEEEkeywords}



% For peer review papers, you can put extra information on the cover
% page as needed:
% \ifCLASSOPTIONpeerreview
% \begin{center} \bfseries EDICS Category: 3-BBND \end{center}
% \fi
%
% For peerreview papers, this IEEEtran command inserts a page break and
% creates the second title. It will be ignored for other modes.
%\IEEEpeerreviewmaketitle




\item
If 4-digit numbers greater than 5,000 are randomly formed from the digits 0, 1, 3, 5, and 7, what is the probability of forming a number divisible by 5 when:
\begin{enumerate}
    \item The digits are repeated?
    \item The repetition of digits is not allowed?
\end{enumerate}
\solution
%\begin{table}[H]
	\centering
\begin{tabular}{|c|c|c|}
\hline
Random variable &Value &Definition\\ \hline
\multirow{3}{*}{X} &0 &Slips of Rs 1\\
&1 &Slips of Rs 5\\
&2 &Slips of Rs 13\\ \hline
\multirow{2}{*}{Y} &0 &Box A\\
&1 &Box B\\\hline
\end{tabular}
\caption{}
\label{tab:Distribution}
\end{table}
See \tabref{tab:Distribution}.
\begin{align}
p_{Y}\brak{k}= \begin{cases} 
      \frac{1}{3} & {k=0} \\
      \frac{2}{3 }& {k=1} 
   \end{cases}
   \\
p_{Y|X}\brak{0|0} = \frac{19}{25}\, 
p_{Y|X}\brak{0|1} = \frac{6}{25}\,
p_{Y|X}\brak{1|0} = \frac{45}{50}\,
p_{Y|X}\brak{1|2} = \frac{5}{50}
\end{align}
The desired probability is the probability that a slip drawn at random is marked other than Rs 1,
\begin{align}
&=1-p_X\brak{0}\\
&= p_X(1) + p_X(2)
\end{align}
Using Bayes theorem,
\begin{align}
&= p_Y\brak{0} \times \pr{Y=0 | X=1} + p_Y\brak{1} \times \pr{Y=1|X=2}\\
&=\frac{1}{3} \times \frac{6}{25} + \frac{2}{3} \times \frac{5}{50}\\
&=\frac{11}{75}
\end{align}

\newpage

%\tableofcontents

\bigskip

\renewcommand{\thefigure}{\theenumi}
\renewcommand{\thetable}{\theenumi}
%\renewcommand{\theequation}{\theenumi}

%\begin{abstract}
%%\boldmath
%In this letter, an algorithm for evaluating the exact analytical bit error rate  (BER)  for the piecewise linear (PL) combiner for  multiple relays is presented. Previous results were available only for upto three relays. The algorithm is unique in the sense that  the actual mathematical expressions, that are prohibitively large, need not be explicitly obtained. The diversity gain due to multiple relays is shown through plots of the analytical BER, well supported by simulations. 
%
%\end{abstract}
% IEEEtran.cls defaults to using nonbold math in the Abstract.
% This preserves the distinction between vectors and scalars. However,
% if the journal you are submitting to favors bold math in the abstract,
% then you can use LaTeX's standard command \boldmath at the very start
% of the abstract to achieve this. Many IEEE journals frown on math
% in the abstract anyway.

% Note that keywords are not normally used for peerreview papers.
%\begin{IEEEkeywords}
%Cooperative diversity, decode and forward, piecewise linear
%\end{IEEEkeywords}



% For peer review papers, you can put extra information on the cover
% page as needed:
% \ifCLASSOPTIONpeerreview
% \begin{center} \bfseries EDICS Category: 3-BBND \end{center}
% \fi
%
% For peerreview papers, this IEEEtran command inserts a page break and
% creates the second title. It will be ignored for other modes.
%\IEEEpeerreviewmaketitle




\item Consider the probability space $\brak{\Omega, \mathcal{G}, P}$ where $\Omega = [0,2]$ and $\mathcal{G} = \cbrak{\phi, \Omega, [0,1], (1,2]}$. Let $X$ and $Y$ be two functions on $\Omega$ defined as
\begin{align*}
    X(\omega) = 
    \begin{cases}
        1 & \text{if }\omega \in [0, 1]\\
        2 & \text{if }\omega \in (1, 2]
    \end{cases}
\end{align*}
and
\begin{align*}
    Y(\omega) = 
    \begin{cases}
        2 & \text{if }\omega \in [0, 1.5]\\
        3 & \text{if }\omega \in (1.5, 2].
    \end{cases}
\end{align*}
Then which one of the following statements is true?
\begin{enumerate}
    \item [(A)] $X$ is a random variable with respect to $\mathcal{G}$, but $Y$ is not a random variable with respect to $\mathcal{G}$.
    \item [(B)] $Y$ is a random variable with respect to $\mathcal{G}$, but $X$ is not a random variable with respect to $\mathcal{G}$.
    \item [(C)] Neither $X$ nor $Y$ is a random variable with respect to $\mathcal{G}$.
    \item [(D)] Both $X$ and $Y$ are random variables with respect to $\mathcal{G}$.
\end{enumerate} \hfill (GATE ST 2023)\\
\solution
%\begin{table}[H]
	\centering
\begin{tabular}{|c|c|c|}
\hline
Random variable &Value &Definition\\ \hline
\multirow{3}{*}{X} &0 &Slips of Rs 1\\
&1 &Slips of Rs 5\\
&2 &Slips of Rs 13\\ \hline
\multirow{2}{*}{Y} &0 &Box A\\
&1 &Box B\\\hline
\end{tabular}
\caption{}
\label{tab:Distribution}
\end{table}
See \tabref{tab:Distribution}.
\begin{align}
p_{Y}\brak{k}= \begin{cases} 
      \frac{1}{3} & {k=0} \\
      \frac{2}{3 }& {k=1} 
   \end{cases}
   \\
p_{Y|X}\brak{0|0} = \frac{19}{25}\, 
p_{Y|X}\brak{0|1} = \frac{6}{25}\,
p_{Y|X}\brak{1|0} = \frac{45}{50}\,
p_{Y|X}\brak{1|2} = \frac{5}{50}
\end{align}
The desired probability is the probability that a slip drawn at random is marked other than Rs 1,
\begin{align}
&=1-p_X\brak{0}\\
&= p_X(1) + p_X(2)
\end{align}
Using Bayes theorem,
\begin{align}
&= p_Y\brak{0} \times \pr{Y=0 | X=1} + p_Y\brak{1} \times \pr{Y=1|X=2}\\
&=\frac{1}{3} \times \frac{6}{25} + \frac{2}{3} \times \frac{5}{50}\\
&=\frac{11}{75}
\end{align}

\newpage

%\tableofcontents

\bigskip

\renewcommand{\thefigure}{\theenumi}
\renewcommand{\thetable}{\theenumi}
%\renewcommand{\theequation}{\theenumi}

%\begin{abstract}
%%\boldmath
%In this letter, an algorithm for evaluating the exact analytical bit error rate  (BER)  for the piecewise linear (PL) combiner for  multiple relays is presented. Previous results were available only for upto three relays. The algorithm is unique in the sense that  the actual mathematical expressions, that are prohibitively large, need not be explicitly obtained. The diversity gain due to multiple relays is shown through plots of the analytical BER, well supported by simulations. 
%
%\end{abstract}
% IEEEtran.cls defaults to using nonbold math in the Abstract.
% This preserves the distinction between vectors and scalars. However,
% if the journal you are submitting to favors bold math in the abstract,
% then you can use LaTeX's standard command \boldmath at the very start
% of the abstract to achieve this. Many IEEE journals frown on math
% in the abstract anyway.

% Note that keywords are not normally used for peerreview papers.
%\begin{IEEEkeywords}
%Cooperative diversity, decode and forward, piecewise linear
%\end{IEEEkeywords}



% For peer review papers, you can put extra information on the cover
% page as needed:
% \ifCLASSOPTIONpeerreview
% \begin{center} \bfseries EDICS Category: 3-BBND \end{center}
% \fi
%
% For peerreview papers, this IEEEtran command inserts a page break and
% creates the second title. It will be ignored for other modes.
%\IEEEpeerreviewmaketitle




	\item  A die is loaded in such a way that each odd number is twice as likely to occur as
each even number. Find $P(G)$, where $G$ is the event that a number greater than
3 occurs on a single roll of the die.
\\
\solution
		%\begin{table}[H]
	\centering
\begin{tabular}{|c|c|c|}
\hline
Random variable &Value &Definition\\ \hline
\multirow{3}{*}{X} &0 &Slips of Rs 1\\
&1 &Slips of Rs 5\\
&2 &Slips of Rs 13\\ \hline
\multirow{2}{*}{Y} &0 &Box A\\
&1 &Box B\\\hline
\end{tabular}
\caption{}
\label{tab:Distribution}
\end{table}
See \tabref{tab:Distribution}.
\begin{align}
p_{Y}\brak{k}= \begin{cases} 
      \frac{1}{3} & {k=0} \\
      \frac{2}{3 }& {k=1} 
   \end{cases}
   \\
p_{Y|X}\brak{0|0} = \frac{19}{25}\, 
p_{Y|X}\brak{0|1} = \frac{6}{25}\,
p_{Y|X}\brak{1|0} = \frac{45}{50}\,
p_{Y|X}\brak{1|2} = \frac{5}{50}
\end{align}
The desired probability is the probability that a slip drawn at random is marked other than Rs 1,
\begin{align}
&=1-p_X\brak{0}\\
&= p_X(1) + p_X(2)
\end{align}
Using Bayes theorem,
\begin{align}
&= p_Y\brak{0} \times \pr{Y=0 | X=1} + p_Y\brak{1} \times \pr{Y=1|X=2}\\
&=\frac{1}{3} \times \frac{6}{25} + \frac{2}{3} \times \frac{5}{50}\\
&=\frac{11}{75}
\end{align}

\newpage

%\tableofcontents

\bigskip

\renewcommand{\thefigure}{\theenumi}
\renewcommand{\thetable}{\theenumi}
%\renewcommand{\theequation}{\theenumi}

%\begin{abstract}
%%\boldmath
%In this letter, an algorithm for evaluating the exact analytical bit error rate  (BER)  for the piecewise linear (PL) combiner for  multiple relays is presented. Previous results were available only for upto three relays. The algorithm is unique in the sense that  the actual mathematical expressions, that are prohibitively large, need not be explicitly obtained. The diversity gain due to multiple relays is shown through plots of the analytical BER, well supported by simulations. 
%
%\end{abstract}
% IEEEtran.cls defaults to using nonbold math in the Abstract.
% This preserves the distinction between vectors and scalars. However,
% if the journal you are submitting to favors bold math in the abstract,
% then you can use LaTeX's standard command \boldmath at the very start
% of the abstract to achieve this. Many IEEE journals frown on math
% in the abstract anyway.

% Note that keywords are not normally used for peerreview papers.
%\begin{IEEEkeywords}
%Cooperative diversity, decode and forward, piecewise linear
%\end{IEEEkeywords}



% For peer review papers, you can put extra information on the cover
% page as needed:
% \ifCLASSOPTIONpeerreview
% \begin{center} \bfseries EDICS Category: 3-BBND \end{center}
% \fi
%
% For peerreview papers, this IEEEtran command inserts a page break and
% creates the second title. It will be ignored for other modes.
%\IEEEpeerreviewmaketitle




	\item All the jacks, queens and kings are removed from a deck of 52 playing cards. The remaining cards are well shuffled and then one card is drawn at random. Giving ace a value 1 similar value for other cards, find the probability that the card has a value 
		\begin{enumerate}
			\item 7
			\item greater than 7
			\item less than 7
		\end{enumerate}
		%Number of cards left after removing all jacks, queens and kings 
\begin{align}
N	= 52 - 4\times 3
	= 40
\end{align}
%\begin{table}[H]
%\def\arraystretch{1.2}
%\begin{tabular}{|c|c|c|}
%\hline
%	\textbf{Parameter} &\textbf{Value} &\textbf{Description}\\ \hline
%	$X$ &1-10 &Represents the value of the card picked \\ \hline
%\end{tabular}
%\end{table}
Let $1 \le X \le 10$ be the value of the card picked.  Then,
\begin{align}
	p_X(k) &= \Pr(X=k)\ \forall\ 1 \leq k \leq 10\\
	&= \frac{4\times 1}{40}\\
	&= \frac{1}{10}\\
	\therefore p_X(k) &= 
	\begin{cases}
		\frac{1}{10} & 1 \leq k \leq 10\\
		0 & \text{otherwise}
	\end{cases}
\end{align}
and
\begin{align}
	F_{X}(k) &= \sum_{m=0}^{k}p_{X}(m) \quad 1 \leq k \leq 10\\
	&= \frac{k}{10}\\
	\therefore F_{X}(k) &= 
	\begin{cases}
		0 & k \leq 0\\
		\frac{k}{10} & 1\leq k \leq 10\\
		1 & k > 10 
	\end{cases}
\end{align}
\begin{enumerate}
	\item Probability that card has value equal to 7 is
		\begin{align}
			 p_{X}(7)
			= \frac{1}{10}
		\end{align}
	\item Probability that card has value greater than 7 is
		\begin{align}
			1 - F_X(7)
			&= 1 - \frac{7}{10}
			\\
			&= \frac{3}{10}
		\end{align}
	\item Probability that card has value less than 7 is
		\begin{align}
			 F_{X}(6)
			=\frac{6}{10}
		\end{align}
\end{enumerate}

  \item A Lot consists of 48 mobile phones of which 42 are good, 3 have only minor defects and 3 have major defects.Varnika will buy a phone if it is good but the trader will only buy a mobile if it has no major defects. One phone is selected at random from the lot. What is the probability that it is
\begin{enumerate}
	\item acceptable to Varnika?
            \item acceptable to the trader?
\end{enumerate}
\solution
	%\begin{table}[H]
	\centering
\begin{tabular}{|c|c|c|}
\hline
Random variable &Value &Definition\\ \hline
\multirow{3}{*}{X} &0 &Slips of Rs 1\\
&1 &Slips of Rs 5\\
&2 &Slips of Rs 13\\ \hline
\multirow{2}{*}{Y} &0 &Box A\\
&1 &Box B\\\hline
\end{tabular}
\caption{}
\label{tab:Distribution}
\end{table}
See \tabref{tab:Distribution}.
\begin{align}
p_{Y}\brak{k}= \begin{cases} 
      \frac{1}{3} & {k=0} \\
      \frac{2}{3 }& {k=1} 
   \end{cases}
   \\
p_{Y|X}\brak{0|0} = \frac{19}{25}\, 
p_{Y|X}\brak{0|1} = \frac{6}{25}\,
p_{Y|X}\brak{1|0} = \frac{45}{50}\,
p_{Y|X}\brak{1|2} = \frac{5}{50}
\end{align}
The desired probability is the probability that a slip drawn at random is marked other than Rs 1,
\begin{align}
&=1-p_X\brak{0}\\
&= p_X(1) + p_X(2)
\end{align}
Using Bayes theorem,
\begin{align}
&= p_Y\brak{0} \times \pr{Y=0 | X=1} + p_Y\brak{1} \times \pr{Y=1|X=2}\\
&=\frac{1}{3} \times \frac{6}{25} + \frac{2}{3} \times \frac{5}{50}\\
&=\frac{11}{75}
\end{align}

\newpage

%\tableofcontents

\bigskip

\renewcommand{\thefigure}{\theenumi}
\renewcommand{\thetable}{\theenumi}
%\renewcommand{\theequation}{\theenumi}

%\begin{abstract}
%%\boldmath
%In this letter, an algorithm for evaluating the exact analytical bit error rate  (BER)  for the piecewise linear (PL) combiner for  multiple relays is presented. Previous results were available only for upto three relays. The algorithm is unique in the sense that  the actual mathematical expressions, that are prohibitively large, need not be explicitly obtained. The diversity gain due to multiple relays is shown through plots of the analytical BER, well supported by simulations. 
%
%\end{abstract}
% IEEEtran.cls defaults to using nonbold math in the Abstract.
% This preserves the distinction between vectors and scalars. However,
% if the journal you are submitting to favors bold math in the abstract,
% then you can use LaTeX's standard command \boldmath at the very start
% of the abstract to achieve this. Many IEEE journals frown on math
% in the abstract anyway.

% Note that keywords are not normally used for peerreview papers.
%\begin{IEEEkeywords}
%Cooperative diversity, decode and forward, piecewise linear
%\end{IEEEkeywords}



% For peer review papers, you can put extra information on the cover
% page as needed:
% \ifCLASSOPTIONpeerreview
% \begin{center} \bfseries EDICS Category: 3-BBND \end{center}
% \fi
%
% For peerreview papers, this IEEEtran command inserts a page break and
% creates the second title. It will be ignored for other modes.
%\IEEEpeerreviewmaketitle




 \item A student says that if you throw a die, it will show up 1 or not 1. Therefore, the probability of getting 1 and the probability of getting 'not 1' each is equal to $\frac{1}{2}$. Is this correct? Give reasons.\\
 \solution
        %\begin{table}[H]
	\centering
\begin{tabular}{|c|c|c|}
\hline
Random variable &Value &Definition\\ \hline
\multirow{3}{*}{X} &0 &Slips of Rs 1\\
&1 &Slips of Rs 5\\
&2 &Slips of Rs 13\\ \hline
\multirow{2}{*}{Y} &0 &Box A\\
&1 &Box B\\\hline
\end{tabular}
\caption{}
\label{tab:Distribution}
\end{table}
See \tabref{tab:Distribution}.
\begin{align}
p_{Y}\brak{k}= \begin{cases} 
      \frac{1}{3} & {k=0} \\
      \frac{2}{3 }& {k=1} 
   \end{cases}
   \\
p_{Y|X}\brak{0|0} = \frac{19}{25}\, 
p_{Y|X}\brak{0|1} = \frac{6}{25}\,
p_{Y|X}\brak{1|0} = \frac{45}{50}\,
p_{Y|X}\brak{1|2} = \frac{5}{50}
\end{align}
The desired probability is the probability that a slip drawn at random is marked other than Rs 1,
\begin{align}
&=1-p_X\brak{0}\\
&= p_X(1) + p_X(2)
\end{align}
Using Bayes theorem,
\begin{align}
&= p_Y\brak{0} \times \pr{Y=0 | X=1} + p_Y\brak{1} \times \pr{Y=1|X=2}\\
&=\frac{1}{3} \times \frac{6}{25} + \frac{2}{3} \times \frac{5}{50}\\
&=\frac{11}{75}
\end{align}

\newpage

%\tableofcontents

\bigskip

\renewcommand{\thefigure}{\theenumi}
\renewcommand{\thetable}{\theenumi}
%\renewcommand{\theequation}{\theenumi}

%\begin{abstract}
%%\boldmath
%In this letter, an algorithm for evaluating the exact analytical bit error rate  (BER)  for the piecewise linear (PL) combiner for  multiple relays is presented. Previous results were available only for upto three relays. The algorithm is unique in the sense that  the actual mathematical expressions, that are prohibitively large, need not be explicitly obtained. The diversity gain due to multiple relays is shown through plots of the analytical BER, well supported by simulations. 
%
%\end{abstract}
% IEEEtran.cls defaults to using nonbold math in the Abstract.
% This preserves the distinction between vectors and scalars. However,
% if the journal you are submitting to favors bold math in the abstract,
% then you can use LaTeX's standard command \boldmath at the very start
% of the abstract to achieve this. Many IEEE journals frown on math
% in the abstract anyway.

% Note that keywords are not normally used for peerreview papers.
%\begin{IEEEkeywords}
%Cooperative diversity, decode and forward, piecewise linear
%\end{IEEEkeywords}



% For peer review papers, you can put extra information on the cover
% page as needed:
% \ifCLASSOPTIONpeerreview
% \begin{center} \bfseries EDICS Category: 3-BBND \end{center}
% \fi
%
% For peerreview papers, this IEEEtran command inserts a page break and
% creates the second title. It will be ignored for other modes.
%\IEEEpeerreviewmaketitle




   \item Four candidates A, B, C, D have ap-
plied for the assignment to coach a school cricket
team. If A is twice as likely to be selected as B, and
B and C are given about the same chance of being
selected, while C is twice as likely to be selected
as D, what are the probabilities that
\begin{enumerate}
\item C will be selected?
\item A will not be selected?
\end{enumerate}
	%\begin{table}[H]
	\centering
\begin{tabular}{|c|c|c|}
\hline
Random variable &Value &Definition\\ \hline
\multirow{3}{*}{X} &0 &Slips of Rs 1\\
&1 &Slips of Rs 5\\
&2 &Slips of Rs 13\\ \hline
\multirow{2}{*}{Y} &0 &Box A\\
&1 &Box B\\\hline
\end{tabular}
\caption{}
\label{tab:Distribution}
\end{table}
See \tabref{tab:Distribution}.
\begin{align}
p_{Y}\brak{k}= \begin{cases} 
      \frac{1}{3} & {k=0} \\
      \frac{2}{3 }& {k=1} 
   \end{cases}
   \\
p_{Y|X}\brak{0|0} = \frac{19}{25}\, 
p_{Y|X}\brak{0|1} = \frac{6}{25}\,
p_{Y|X}\brak{1|0} = \frac{45}{50}\,
p_{Y|X}\brak{1|2} = \frac{5}{50}
\end{align}
The desired probability is the probability that a slip drawn at random is marked other than Rs 1,
\begin{align}
&=1-p_X\brak{0}\\
&= p_X(1) + p_X(2)
\end{align}
Using Bayes theorem,
\begin{align}
&= p_Y\brak{0} \times \pr{Y=0 | X=1} + p_Y\brak{1} \times \pr{Y=1|X=2}\\
&=\frac{1}{3} \times \frac{6}{25} + \frac{2}{3} \times \frac{5}{50}\\
&=\frac{11}{75}
\end{align}

\newpage

%\tableofcontents

\bigskip

\renewcommand{\thefigure}{\theenumi}
\renewcommand{\thetable}{\theenumi}
%\renewcommand{\theequation}{\theenumi}

%\begin{abstract}
%%\boldmath
%In this letter, an algorithm for evaluating the exact analytical bit error rate  (BER)  for the piecewise linear (PL) combiner for  multiple relays is presented. Previous results were available only for upto three relays. The algorithm is unique in the sense that  the actual mathematical expressions, that are prohibitively large, need not be explicitly obtained. The diversity gain due to multiple relays is shown through plots of the analytical BER, well supported by simulations. 
%
%\end{abstract}
% IEEEtran.cls defaults to using nonbold math in the Abstract.
% This preserves the distinction between vectors and scalars. However,
% if the journal you are submitting to favors bold math in the abstract,
% then you can use LaTeX's standard command \boldmath at the very start
% of the abstract to achieve this. Many IEEE journals frown on math
% in the abstract anyway.

% Note that keywords are not normally used for peerreview papers.
%\begin{IEEEkeywords}
%Cooperative diversity, decode and forward, piecewise linear
%\end{IEEEkeywords}



% For peer review papers, you can put extra information on the cover
% page as needed:
% \ifCLASSOPTIONpeerreview
% \begin{center} \bfseries EDICS Category: 3-BBND \end{center}
% \fi
%
% For peerreview papers, this IEEEtran command inserts a page break and
% creates the second title. It will be ignored for other modes.
%\IEEEpeerreviewmaketitle




 \item A bag contain 24 balls of which $x$ balls are red, $2x$ are white and $3x$ are blue. A ball is selected at random, What is the probability that it is
\begin{enumerate}[label=\alph*)]
\item not red ?
\item white ?
\end{enumerate}
%\begin{table}[H]
	\centering
\begin{tabular}{|c|c|c|}
\hline
Random variable &Value &Definition\\ \hline
\multirow{3}{*}{X} &0 &Slips of Rs 1\\
&1 &Slips of Rs 5\\
&2 &Slips of Rs 13\\ \hline
\multirow{2}{*}{Y} &0 &Box A\\
&1 &Box B\\\hline
\end{tabular}
\caption{}
\label{tab:Distribution}
\end{table}
See \tabref{tab:Distribution}.
\begin{align}
p_{Y}\brak{k}= \begin{cases} 
      \frac{1}{3} & {k=0} \\
      \frac{2}{3 }& {k=1} 
   \end{cases}
   \\
p_{Y|X}\brak{0|0} = \frac{19}{25}\, 
p_{Y|X}\brak{0|1} = \frac{6}{25}\,
p_{Y|X}\brak{1|0} = \frac{45}{50}\,
p_{Y|X}\brak{1|2} = \frac{5}{50}
\end{align}
The desired probability is the probability that a slip drawn at random is marked other than Rs 1,
\begin{align}
&=1-p_X\brak{0}\\
&= p_X(1) + p_X(2)
\end{align}
Using Bayes theorem,
\begin{align}
&= p_Y\brak{0} \times \pr{Y=0 | X=1} + p_Y\brak{1} \times \pr{Y=1|X=2}\\
&=\frac{1}{3} \times \frac{6}{25} + \frac{2}{3} \times \frac{5}{50}\\
&=\frac{11}{75}
\end{align}

\newpage

%\tableofcontents

\bigskip

\renewcommand{\thefigure}{\theenumi}
\renewcommand{\thetable}{\theenumi}
%\renewcommand{\theequation}{\theenumi}

%\begin{abstract}
%%\boldmath
%In this letter, an algorithm for evaluating the exact analytical bit error rate  (BER)  for the piecewise linear (PL) combiner for  multiple relays is presented. Previous results were available only for upto three relays. The algorithm is unique in the sense that  the actual mathematical expressions, that are prohibitively large, need not be explicitly obtained. The diversity gain due to multiple relays is shown through plots of the analytical BER, well supported by simulations. 
%
%\end{abstract}
% IEEEtran.cls defaults to using nonbold math in the Abstract.
% This preserves the distinction between vectors and scalars. However,
% if the journal you are submitting to favors bold math in the abstract,
% then you can use LaTeX's standard command \boldmath at the very start
% of the abstract to achieve this. Many IEEE journals frown on math
% in the abstract anyway.

% Note that keywords are not normally used for peerreview papers.
%\begin{IEEEkeywords}
%Cooperative diversity, decode and forward, piecewise linear
%\end{IEEEkeywords}



% For peer review papers, you can put extra information on the cover
% page as needed:
% \ifCLASSOPTIONpeerreview
% \begin{center} \bfseries EDICS Category: 3-BBND \end{center}
% \fi
%
% For peerreview papers, this IEEEtran command inserts a page break and
% creates the second title. It will be ignored for other modes.
%\IEEEpeerreviewmaketitle




If the letters of the word ASSASSINATION are arranged at random. Find the Probability that
\begin{enumerate}[label=(\alph*)]
\item Four $S's$ come consecutively in the word
\item Two  $I's$ and two $N's$ come together
\item All $A's$ are not coming together
\item No two $A's$ are coming together
\end{enumerate}
%\begin{table}[H]
	\centering
\begin{tabular}{|c|c|c|}
\hline
Random variable &Value &Definition\\ \hline
\multirow{3}{*}{X} &0 &Slips of Rs 1\\
&1 &Slips of Rs 5\\
&2 &Slips of Rs 13\\ \hline
\multirow{2}{*}{Y} &0 &Box A\\
&1 &Box B\\\hline
\end{tabular}
\caption{}
\label{tab:Distribution}
\end{table}
See \tabref{tab:Distribution}.
\begin{align}
p_{Y}\brak{k}= \begin{cases} 
      \frac{1}{3} & {k=0} \\
      \frac{2}{3 }& {k=1} 
   \end{cases}
   \\
p_{Y|X}\brak{0|0} = \frac{19}{25}\, 
p_{Y|X}\brak{0|1} = \frac{6}{25}\,
p_{Y|X}\brak{1|0} = \frac{45}{50}\,
p_{Y|X}\brak{1|2} = \frac{5}{50}
\end{align}
The desired probability is the probability that a slip drawn at random is marked other than Rs 1,
\begin{align}
&=1-p_X\brak{0}\\
&= p_X(1) + p_X(2)
\end{align}
Using Bayes theorem,
\begin{align}
&= p_Y\brak{0} \times \pr{Y=0 | X=1} + p_Y\brak{1} \times \pr{Y=1|X=2}\\
&=\frac{1}{3} \times \frac{6}{25} + \frac{2}{3} \times \frac{5}{50}\\
&=\frac{11}{75}
\end{align}

\newpage

%\tableofcontents

\bigskip

\renewcommand{\thefigure}{\theenumi}
\renewcommand{\thetable}{\theenumi}
%\renewcommand{\theequation}{\theenumi}

%\begin{abstract}
%%\boldmath
%In this letter, an algorithm for evaluating the exact analytical bit error rate  (BER)  for the piecewise linear (PL) combiner for  multiple relays is presented. Previous results were available only for upto three relays. The algorithm is unique in the sense that  the actual mathematical expressions, that are prohibitively large, need not be explicitly obtained. The diversity gain due to multiple relays is shown through plots of the analytical BER, well supported by simulations. 
%
%\end{abstract}
% IEEEtran.cls defaults to using nonbold math in the Abstract.
% This preserves the distinction between vectors and scalars. However,
% if the journal you are submitting to favors bold math in the abstract,
% then you can use LaTeX's standard command \boldmath at the very start
% of the abstract to achieve this. Many IEEE journals frown on math
% in the abstract anyway.

% Note that keywords are not normally used for peerreview papers.
%\begin{IEEEkeywords}
%Cooperative diversity, decode and forward, piecewise linear
%\end{IEEEkeywords}



% For peer review papers, you can put extra information on the cover
% page as needed:
% \ifCLASSOPTIONpeerreview
% \begin{center} \bfseries EDICS Category: 3-BBND \end{center}
% \fi
%
% For peerreview papers, this IEEEtran command inserts a page break and
% creates the second title. It will be ignored for other modes.
%\IEEEpeerreviewmaketitle




	\item One urn contains two black balls (labelled B1 and B2) and one white ball. A
	second urn contains one black ball and two white balls (labelled W1 and W2).
	Suppose the following experiment is performed. One of the two urns is chosen
	at random. Next a ball is randomly chosen from the urn. Then a second ball is
	chosen at random from the same urn without replacing the first ball.
	
	\begin{enumerate}
	\item What is the probability that two black balls are chosen?
	
	\item What is the probability that two balls of opposite colour are chosen?
	\end{enumerate}
	\solution
	%\begin{align}
    \label{eq:12.13.6.18.1}
	\because	\pr{A|B} &> \pr{A},\
\frac{\pr{AB}}{\pr{B}} > \pr{A}
\\
    \label{eq:12.13.6.18.2}
	\implies \pr{AB} &> \pr{A}\pr{B}
	\\
	\text{or, } \frac{\pr{AB}}{\pr{A}} &=\pr{B|A} > \pr{A}
\end{align}

\end{enumerate}

		%
\item 
Out of 100 students, two sections of 40 and 60 are formed. If you and your friend are among the 100 students, what is the probability that
\begin{enumerate}
\item you both enter the same section?
\item you both enter the different sections?
\end{enumerate}
\solution
		%\begin{enumerate}[label=\thesection.\arabic*,ref=\thesection.\theenumi]
	\item One card is drawn from a well-shuffled deck of 52 cards. Find the probability of getting
\begin{enumerate}
\item A king of red colour 
\item A face card 
\item A red face card
\item The jack of hearts
\item A spade
\item The queen of diamonds

\end{enumerate}
\solution
		%\begin{table}[H]
	\centering
\begin{tabular}{|c|c|c|}
\hline
Random variable &Value &Definition\\ \hline
\multirow{3}{*}{X} &0 &Slips of Rs 1\\
&1 &Slips of Rs 5\\
&2 &Slips of Rs 13\\ \hline
\multirow{2}{*}{Y} &0 &Box A\\
&1 &Box B\\\hline
\end{tabular}
\caption{}
\label{tab:Distribution}
\end{table}
See \tabref{tab:Distribution}.
\begin{align}
p_{Y}\brak{k}= \begin{cases} 
      \frac{1}{3} & {k=0} \\
      \frac{2}{3 }& {k=1} 
   \end{cases}
   \\
p_{Y|X}\brak{0|0} = \frac{19}{25}\, 
p_{Y|X}\brak{0|1} = \frac{6}{25}\,
p_{Y|X}\brak{1|0} = \frac{45}{50}\,
p_{Y|X}\brak{1|2} = \frac{5}{50}
\end{align}
The desired probability is the probability that a slip drawn at random is marked other than Rs 1,
\begin{align}
&=1-p_X\brak{0}\\
&= p_X(1) + p_X(2)
\end{align}
Using Bayes theorem,
\begin{align}
&= p_Y\brak{0} \times \pr{Y=0 | X=1} + p_Y\brak{1} \times \pr{Y=1|X=2}\\
&=\frac{1}{3} \times \frac{6}{25} + \frac{2}{3} \times \frac{5}{50}\\
&=\frac{11}{75}
\end{align}

\newpage

%\tableofcontents

\bigskip

\renewcommand{\thefigure}{\theenumi}
\renewcommand{\thetable}{\theenumi}
%\renewcommand{\theequation}{\theenumi}

%\begin{abstract}
%%\boldmath
%In this letter, an algorithm for evaluating the exact analytical bit error rate  (BER)  for the piecewise linear (PL) combiner for  multiple relays is presented. Previous results were available only for upto three relays. The algorithm is unique in the sense that  the actual mathematical expressions, that are prohibitively large, need not be explicitly obtained. The diversity gain due to multiple relays is shown through plots of the analytical BER, well supported by simulations. 
%
%\end{abstract}
% IEEEtran.cls defaults to using nonbold math in the Abstract.
% This preserves the distinction between vectors and scalars. However,
% if the journal you are submitting to favors bold math in the abstract,
% then you can use LaTeX's standard command \boldmath at the very start
% of the abstract to achieve this. Many IEEE journals frown on math
% in the abstract anyway.

% Note that keywords are not normally used for peerreview papers.
%\begin{IEEEkeywords}
%Cooperative diversity, decode and forward, piecewise linear
%\end{IEEEkeywords}



% For peer review papers, you can put extra information on the cover
% page as needed:
% \ifCLASSOPTIONpeerreview
% \begin{center} \bfseries EDICS Category: 3-BBND \end{center}
% \fi
%
% For peerreview papers, this IEEEtran command inserts a page break and
% creates the second title. It will be ignored for other modes.
%\IEEEpeerreviewmaketitle




	\item Five cards—the ten, jack, queen, king and ace of diamonds, are well-shuffled with their face downwards. One card is then picked up at random.
\begin{enumerate}
\item
What is the probability that the card is the queen? 
\item
If the queen is drawn and put aside, what is the probability that the second card picked up is (a) an ace? (b) a queen?\\
\end{enumerate}
\solution
		%\begin{enumerate}[label=\thesection.\arabic*,ref=\thesection.\theenumi]
	\item One card is drawn from a well-shuffled deck of 52 cards. Find the probability of getting
\begin{enumerate}
\item A king of red colour 
\item A face card 
\item A red face card
\item The jack of hearts
\item A spade
\item The queen of diamonds

\end{enumerate}
\solution
		%\begin{table}[H]
	\centering
\begin{tabular}{|c|c|c|}
\hline
Random variable &Value &Definition\\ \hline
\multirow{3}{*}{X} &0 &Slips of Rs 1\\
&1 &Slips of Rs 5\\
&2 &Slips of Rs 13\\ \hline
\multirow{2}{*}{Y} &0 &Box A\\
&1 &Box B\\\hline
\end{tabular}
\caption{}
\label{tab:Distribution}
\end{table}
See \tabref{tab:Distribution}.
\begin{align}
p_{Y}\brak{k}= \begin{cases} 
      \frac{1}{3} & {k=0} \\
      \frac{2}{3 }& {k=1} 
   \end{cases}
   \\
p_{Y|X}\brak{0|0} = \frac{19}{25}\, 
p_{Y|X}\brak{0|1} = \frac{6}{25}\,
p_{Y|X}\brak{1|0} = \frac{45}{50}\,
p_{Y|X}\brak{1|2} = \frac{5}{50}
\end{align}
The desired probability is the probability that a slip drawn at random is marked other than Rs 1,
\begin{align}
&=1-p_X\brak{0}\\
&= p_X(1) + p_X(2)
\end{align}
Using Bayes theorem,
\begin{align}
&= p_Y\brak{0} \times \pr{Y=0 | X=1} + p_Y\brak{1} \times \pr{Y=1|X=2}\\
&=\frac{1}{3} \times \frac{6}{25} + \frac{2}{3} \times \frac{5}{50}\\
&=\frac{11}{75}
\end{align}

\newpage

%\tableofcontents

\bigskip

\renewcommand{\thefigure}{\theenumi}
\renewcommand{\thetable}{\theenumi}
%\renewcommand{\theequation}{\theenumi}

%\begin{abstract}
%%\boldmath
%In this letter, an algorithm for evaluating the exact analytical bit error rate  (BER)  for the piecewise linear (PL) combiner for  multiple relays is presented. Previous results were available only for upto three relays. The algorithm is unique in the sense that  the actual mathematical expressions, that are prohibitively large, need not be explicitly obtained. The diversity gain due to multiple relays is shown through plots of the analytical BER, well supported by simulations. 
%
%\end{abstract}
% IEEEtran.cls defaults to using nonbold math in the Abstract.
% This preserves the distinction between vectors and scalars. However,
% if the journal you are submitting to favors bold math in the abstract,
% then you can use LaTeX's standard command \boldmath at the very start
% of the abstract to achieve this. Many IEEE journals frown on math
% in the abstract anyway.

% Note that keywords are not normally used for peerreview papers.
%\begin{IEEEkeywords}
%Cooperative diversity, decode and forward, piecewise linear
%\end{IEEEkeywords}



% For peer review papers, you can put extra information on the cover
% page as needed:
% \ifCLASSOPTIONpeerreview
% \begin{center} \bfseries EDICS Category: 3-BBND \end{center}
% \fi
%
% For peerreview papers, this IEEEtran command inserts a page break and
% creates the second title. It will be ignored for other modes.
%\IEEEpeerreviewmaketitle




	\item Five cards—the ten, jack, queen, king and ace of diamonds, are well-shuffled with their face downwards. One card is then picked up at random.
\begin{enumerate}
\item
What is the probability that the card is the queen? 
\item
If the queen is drawn and put aside, what is the probability that the second card picked up is (a) an ace? (b) a queen?\\
\end{enumerate}
\solution
		%\begin{enumerate}[label=\thesection.\arabic*,ref=\thesection.\theenumi]
	\item One card is drawn from a well-shuffled deck of 52 cards. Find the probability of getting
\begin{enumerate}
\item A king of red colour 
\item A face card 
\item A red face card
\item The jack of hearts
\item A spade
\item The queen of diamonds

\end{enumerate}
\solution
		%\input{ncert/10/15/1/14/main.tex}
	\item Five cards—the ten, jack, queen, king and ace of diamonds, are well-shuffled with their face downwards. One card is then picked up at random.
\begin{enumerate}
\item
What is the probability that the card is the queen? 
\item
If the queen is drawn and put aside, what is the probability that the second card picked up is (a) an ace? (b) a queen?\\
\end{enumerate}
\solution
		%\input{ncert/10/15/1/15/defs.tex}
	\item A bag contains $5$ red balls and some blue balls. If the probability of drawing a blue ball is double that if a red ball, determine the number of blue balls in the bag. 
		\\
\solution
		%\input{ncert/10/15/2/3/defs.tex}
	\item A card is selected from a pack of 52 cards.
 \begin{enumerate}[label=(\alph*)] 
                 \item How many points are there in the sample space?
                 \item Calculate the probability that the card is an ace of spades.
                 \item Calculate the probability that the card is (i) an ace and (ii) black card.
 \end{enumerate}
\solution
		%\input{ncert/11/16/3/4/main.tex}
\item Four cards are drawn from a well-shuffled deck of 52 cards. What is the probability of obtaining 3 diamonds and one spade.
\\
\solution
		%\input{ncert/11/16/4/2/defs.tex}
\item In a certain lottery 10,000 tickets are sold and ten equal prizes are awarded. What is the probability of not getting a prize if you buy (a) one ticket (b) two tickets (c) 10 tickets ?	
\\
\solution
		%\input{ncert/11/16/4/4/defs.tex}
		%
\item 
Out of 100 students, two sections of 40 and 60 are formed. If you and your friend are among the 100 students, what is the probability that
\begin{enumerate}
\item you both enter the same section?
\item you both enter the different sections?
\end{enumerate}
\solution
		%\input{ncert/11/16/4/5/defs.tex}
	\item 
The number lock of a suitcase has 4 wheels each labelled with ten digits i.e. from 0 to 9.The lock opens with a sequence of four digits with no repeats.What is the probability of a person getting the right sequence to open the suitcase.
\\
\solution
		%\input{ncert/11/16/4/10/defs.tex}
		%
\item 
Two cards are drawn at random and without replacement from a pack of 52 playing cards. Find the probability that both the cards are black.
\\
\solution
		%\input{ncert/12/13/2/2/defs.tex}
		\item A box of oranges is inspected by examining three randomly selected oranges drawn without replacement. If all the three oranges are good, the box is approved for sale, otherwise, it is rejected. Find the probability that a box containing 15 oranges out of which 12 are good and 3 are bad ones will be approved for sale.
		\label{ncert/12/13/2/3/defs.tex}
		\item Two balls are drawn at random with replacement from a box containing 10 black and 8 red balls. Find the probability that
		\label{ncert/12/13/2/12}
\begin{enumerate}
\item both balls are red.
\item first ball is black and second is red.
\item one of them is black and other is red.
\end{enumerate}

\item In a hostel, 60\% of the students read Hindi newspaper, 40\% read English newspaper and 20\% read both Hindi and English newspapers. A student is selected at random.
		\label{ncert/12/13/2/15}
\begin{enumerate}
\item Find the probability that she reads neither Hindi nor English newspapers.
\item If she reads Hindi newspaper, find the probability that she reads English newspaper.
\item If she reads English newspaper, find the probability that she reads Hindi newspaper.\\
\end{enumerate}
\item The probability of obtaining an even prime number on each die, when a pair of dice is rolled is 
\begin{enumerate}
    \item $0$ 
    
    \item $\frac{1}{3}$ 
    
    \item $\frac{1}{12}$ 
    
    \item $\frac{1}{36}$ 
\end{enumerate}
\solution
		%\input{ncert/12/13/2/17/defs.tex}
	\item A bag contains 4 red and 4 black balls, another bag contains 2 red and 6 black balls. One of the two bags is selected at random and a ball is drawn from the bag which is found to be red. Find the probability that the ball is drawn from the first bag.
\\
\solution
		%\input{ncert/12/13/3/2/main.tex}
  \item
  Cards with numbers 2 to 101 are placed in a box. A card is selected at random.Find the probability that the card has
\begin{enumerate}[label=(\roman*)]
	\item an even number 
	\item a square number
\end{enumerate}
\solution
%\input{exemplar/10/13/3/32/main.tex}
\item
The king, queen and jack of clubs are removed from a deck of 52 playing cards and then well shuffled. Now one card is drawn at random from the remaining cards.  Determine the probability that the card is
\begin{enumerate}[label=(\roman*)]
\item a club
\item 10 of hearts
\end{enumerate}
\solution
%\input{exemplar/10/13/3/29/main.tex}
\item A team of medical students doing their internship have to assist during surgeries
at a city hospital. The probabilities of surgeries rated as very complex, complex,
routine, simple or very simple are respectively, 0.15, 0.20, 0.31, 0.26, .08. Find
the probabilities that a particular surgery will be rated
\begin{enumerate}
	\item complex or very complex;
	\item neither very complex nor very simple;
	\item routine or complex
	\item routine or simple
\end{enumerate}
\solution
%\input{exemplar/11/16/3/8(1)/main.tex}
\item A card is selected from a pack of 52 cards.
\begin{enumerate}[label=(\alph*)]
    \item How many points are there in the sample space?
    \item Calculate the probability that the card is an ace of spades.
    \item Calculate the probability that the card is (i) an ace and (ii) black card.
\end{enumerate}
\solution
%\input{exemplar/11/16/3/4/main2.tex}
\item The probability that a non leap year selected at random will contain 53 sundays.
\\
\solution
%\input{exemplar/10/13/1/19/main.tex}
\item One of the four persons John, Rita, Aslam or Gurpreet will be promoted next
month. Consequently the sample space consists of four elementary outcomes
S = {John promoted, Rita promoted, Aslam promoted, Gurpreet promoted}
You are told that the chances of John’s promotion is same as that of Gurpreet,
Rita’s chances of promotion are twice as likely as Johns. Aslam’s chances are
four times that of John.
\begin{enumerate}
	\item Determine
	\begin{enumerate}
		\item P (John promoted)
		\item P (Rita promoted)
		\item P (Aslam promoted)
		\item P (Gurpreet promoted)
	\end{enumerate}
	\item If A = {John promoted or Gurpreet promoted}, find P (A).
\end{enumerate}
\solution
%\input{exemplar/11/16/3/10/main.tex}
\item A card is drawn from a deck of 52 cards. Find the probability of getting a king or a heart or a red card.\\
\solution
%\input{exemplar/11/16/3/15/main.tex}
\item The probability that a student will pass his examination is 0.73, the probability of
the student getting a compartment is 0.13, and the probability that the student will
either pass or get compartment is 0.96. State True or False.\\
\solution
%\input{exemplar/11/16/3/31/main.tex}
\item A card is selected from a pack of 52 cards\\
\begin{enumerate}[label=(\alph*)]
\item How many points are there in the sample space?
\item Calculate the probability that the cards is an ace of spades.
\item Calculate the probability that the card is (i) an ace (ii)black card.\\
\end{enumerate}
%\input{ncert/11/16/3/4_1/Prob_4.tex}
\item In a non-leap year, the probability of having 53 tuesdays or 53 wednesdays is\\
\solution
%\input{exemplar/11/16/3/18/main.tex}
\item There are 1000 sealed envelopes in a box, 10 of them contain a cash prize of
Rs 100 each, 100 of them contain a cash prize of Rs 50 each and 200 of them
contain a cash prize of Rs 10 each and rest do not contain any cash prize. If they
are well shuffled and an envelope is picked up out, what is the probability that it
contains no cash prize?\\
\solution
%\input{exemplar/10/13/3/34/main.tex}
\item 
A die is thrown and a card is selected at random from a deck of 52 playing cards. The probability of getting an even number on the die and a spade card.\\
\solution
%\input{exemplar/12/13/3/78/main.tex}
\item
If 4-digit numbers greater than 5,000 are randomly formed from the digits 0, 1, 3, 5, and 7, what is the probability of forming a number divisible by 5 when:
\begin{enumerate}
    \item The digits are repeated?
    \item The repetition of digits is not allowed?
\end{enumerate}
\solution
%\input{ncert/11/16/4/9/main.tex}
\item Consider the probability space $\brak{\Omega, \mathcal{G}, P}$ where $\Omega = [0,2]$ and $\mathcal{G} = \cbrak{\phi, \Omega, [0,1], (1,2]}$. Let $X$ and $Y$ be two functions on $\Omega$ defined as
\begin{align*}
    X(\omega) = 
    \begin{cases}
        1 & \text{if }\omega \in [0, 1]\\
        2 & \text{if }\omega \in (1, 2]
    \end{cases}
\end{align*}
and
\begin{align*}
    Y(\omega) = 
    \begin{cases}
        2 & \text{if }\omega \in [0, 1.5]\\
        3 & \text{if }\omega \in (1.5, 2].
    \end{cases}
\end{align*}
Then which one of the following statements is true?
\begin{enumerate}
    \item [(A)] $X$ is a random variable with respect to $\mathcal{G}$, but $Y$ is not a random variable with respect to $\mathcal{G}$.
    \item [(B)] $Y$ is a random variable with respect to $\mathcal{G}$, but $X$ is not a random variable with respect to $\mathcal{G}$.
    \item [(C)] Neither $X$ nor $Y$ is a random variable with respect to $\mathcal{G}$.
    \item [(D)] Both $X$ and $Y$ are random variables with respect to $\mathcal{G}$.
\end{enumerate} \hfill (GATE ST 2023)\\
\solution
%\input{gate/ST/2023/14/main.tex}
	\item  A die is loaded in such a way that each odd number is twice as likely to occur as
each even number. Find $P(G)$, where $G$ is the event that a number greater than
3 occurs on a single roll of the die.
\\
\solution
		%\input{exemplar/11/16/3/5/main.tex}
	\item All the jacks, queens and kings are removed from a deck of 52 playing cards. The remaining cards are well shuffled and then one card is drawn at random. Giving ace a value 1 similar value for other cards, find the probability that the card has a value 
		\begin{enumerate}
			\item 7
			\item greater than 7
			\item less than 7
		\end{enumerate}
		%\input{exemplar/10/13/3/30/main.tex}
  \item A Lot consists of 48 mobile phones of which 42 are good, 3 have only minor defects and 3 have major defects.Varnika will buy a phone if it is good but the trader will only buy a mobile if it has no major defects. One phone is selected at random from the lot. What is the probability that it is
\begin{enumerate}
	\item acceptable to Varnika?
            \item acceptable to the trader?
\end{enumerate}
\solution
	%\input{exemplar/10/13/3/40/main.tex}
 \item A student says that if you throw a die, it will show up 1 or not 1. Therefore, the probability of getting 1 and the probability of getting 'not 1' each is equal to $\frac{1}{2}$. Is this correct? Give reasons.\\
 \solution
        %\input{exemplar/10/13/2/9/main.tex}
   \item Four candidates A, B, C, D have ap-
plied for the assignment to coach a school cricket
team. If A is twice as likely to be selected as B, and
B and C are given about the same chance of being
selected, while C is twice as likely to be selected
as D, what are the probabilities that
\begin{enumerate}
\item C will be selected?
\item A will not be selected?
\end{enumerate}
	%\input{exemplar/11/16/3/9/main.tex}
 \item A bag contain 24 balls of which $x$ balls are red, $2x$ are white and $3x$ are blue. A ball is selected at random, What is the probability that it is
\begin{enumerate}[label=\alph*)]
\item not red ?
\item white ?
\end{enumerate}
%\input{exemplar/10/13/3/41/main.tex}
If the letters of the word ASSASSINATION are arranged at random. Find the Probability that
\begin{enumerate}[label=(\alph*)]
\item Four $S's$ come consecutively in the word
\item Two  $I's$ and two $N's$ come together
\item All $A's$ are not coming together
\item No two $A's$ are coming together
\end{enumerate}
%\input{exemplar/11/16/3/14/main.tex}
	\item One urn contains two black balls (labelled B1 and B2) and one white ball. A
	second urn contains one black ball and two white balls (labelled W1 and W2).
	Suppose the following experiment is performed. One of the two urns is chosen
	at random. Next a ball is randomly chosen from the urn. Then a second ball is
	chosen at random from the same urn without replacing the first ball.
	
	\begin{enumerate}
	\item What is the probability that two black balls are chosen?
	
	\item What is the probability that two balls of opposite colour are chosen?
	\end{enumerate}
	\solution
	%\input{exemplar/11/16/3/12/main1.tex}
\end{enumerate}

	\item A bag contains $5$ red balls and some blue balls. If the probability of drawing a blue ball is double that if a red ball, determine the number of blue balls in the bag. 
		\\
\solution
		%\begin{enumerate}[label=\thesection.\arabic*,ref=\thesection.\theenumi]
	\item One card is drawn from a well-shuffled deck of 52 cards. Find the probability of getting
\begin{enumerate}
\item A king of red colour 
\item A face card 
\item A red face card
\item The jack of hearts
\item A spade
\item The queen of diamonds

\end{enumerate}
\solution
		%\input{ncert/10/15/1/14/main.tex}
	\item Five cards—the ten, jack, queen, king and ace of diamonds, are well-shuffled with their face downwards. One card is then picked up at random.
\begin{enumerate}
\item
What is the probability that the card is the queen? 
\item
If the queen is drawn and put aside, what is the probability that the second card picked up is (a) an ace? (b) a queen?\\
\end{enumerate}
\solution
		%\input{ncert/10/15/1/15/defs.tex}
	\item A bag contains $5$ red balls and some blue balls. If the probability of drawing a blue ball is double that if a red ball, determine the number of blue balls in the bag. 
		\\
\solution
		%\input{ncert/10/15/2/3/defs.tex}
	\item A card is selected from a pack of 52 cards.
 \begin{enumerate}[label=(\alph*)] 
                 \item How many points are there in the sample space?
                 \item Calculate the probability that the card is an ace of spades.
                 \item Calculate the probability that the card is (i) an ace and (ii) black card.
 \end{enumerate}
\solution
		%\input{ncert/11/16/3/4/main.tex}
\item Four cards are drawn from a well-shuffled deck of 52 cards. What is the probability of obtaining 3 diamonds and one spade.
\\
\solution
		%\input{ncert/11/16/4/2/defs.tex}
\item In a certain lottery 10,000 tickets are sold and ten equal prizes are awarded. What is the probability of not getting a prize if you buy (a) one ticket (b) two tickets (c) 10 tickets ?	
\\
\solution
		%\input{ncert/11/16/4/4/defs.tex}
		%
\item 
Out of 100 students, two sections of 40 and 60 are formed. If you and your friend are among the 100 students, what is the probability that
\begin{enumerate}
\item you both enter the same section?
\item you both enter the different sections?
\end{enumerate}
\solution
		%\input{ncert/11/16/4/5/defs.tex}
	\item 
The number lock of a suitcase has 4 wheels each labelled with ten digits i.e. from 0 to 9.The lock opens with a sequence of four digits with no repeats.What is the probability of a person getting the right sequence to open the suitcase.
\\
\solution
		%\input{ncert/11/16/4/10/defs.tex}
		%
\item 
Two cards are drawn at random and without replacement from a pack of 52 playing cards. Find the probability that both the cards are black.
\\
\solution
		%\input{ncert/12/13/2/2/defs.tex}
		\item A box of oranges is inspected by examining three randomly selected oranges drawn without replacement. If all the three oranges are good, the box is approved for sale, otherwise, it is rejected. Find the probability that a box containing 15 oranges out of which 12 are good and 3 are bad ones will be approved for sale.
		\label{ncert/12/13/2/3/defs.tex}
		\item Two balls are drawn at random with replacement from a box containing 10 black and 8 red balls. Find the probability that
		\label{ncert/12/13/2/12}
\begin{enumerate}
\item both balls are red.
\item first ball is black and second is red.
\item one of them is black and other is red.
\end{enumerate}

\item In a hostel, 60\% of the students read Hindi newspaper, 40\% read English newspaper and 20\% read both Hindi and English newspapers. A student is selected at random.
		\label{ncert/12/13/2/15}
\begin{enumerate}
\item Find the probability that she reads neither Hindi nor English newspapers.
\item If she reads Hindi newspaper, find the probability that she reads English newspaper.
\item If she reads English newspaper, find the probability that she reads Hindi newspaper.\\
\end{enumerate}
\item The probability of obtaining an even prime number on each die, when a pair of dice is rolled is 
\begin{enumerate}
    \item $0$ 
    
    \item $\frac{1}{3}$ 
    
    \item $\frac{1}{12}$ 
    
    \item $\frac{1}{36}$ 
\end{enumerate}
\solution
		%\input{ncert/12/13/2/17/defs.tex}
	\item A bag contains 4 red and 4 black balls, another bag contains 2 red and 6 black balls. One of the two bags is selected at random and a ball is drawn from the bag which is found to be red. Find the probability that the ball is drawn from the first bag.
\\
\solution
		%\input{ncert/12/13/3/2/main.tex}
  \item
  Cards with numbers 2 to 101 are placed in a box. A card is selected at random.Find the probability that the card has
\begin{enumerate}[label=(\roman*)]
	\item an even number 
	\item a square number
\end{enumerate}
\solution
%\input{exemplar/10/13/3/32/main.tex}
\item
The king, queen and jack of clubs are removed from a deck of 52 playing cards and then well shuffled. Now one card is drawn at random from the remaining cards.  Determine the probability that the card is
\begin{enumerate}[label=(\roman*)]
\item a club
\item 10 of hearts
\end{enumerate}
\solution
%\input{exemplar/10/13/3/29/main.tex}
\item A team of medical students doing their internship have to assist during surgeries
at a city hospital. The probabilities of surgeries rated as very complex, complex,
routine, simple or very simple are respectively, 0.15, 0.20, 0.31, 0.26, .08. Find
the probabilities that a particular surgery will be rated
\begin{enumerate}
	\item complex or very complex;
	\item neither very complex nor very simple;
	\item routine or complex
	\item routine or simple
\end{enumerate}
\solution
%\input{exemplar/11/16/3/8(1)/main.tex}
\item A card is selected from a pack of 52 cards.
\begin{enumerate}[label=(\alph*)]
    \item How many points are there in the sample space?
    \item Calculate the probability that the card is an ace of spades.
    \item Calculate the probability that the card is (i) an ace and (ii) black card.
\end{enumerate}
\solution
%\input{exemplar/11/16/3/4/main2.tex}
\item The probability that a non leap year selected at random will contain 53 sundays.
\\
\solution
%\input{exemplar/10/13/1/19/main.tex}
\item One of the four persons John, Rita, Aslam or Gurpreet will be promoted next
month. Consequently the sample space consists of four elementary outcomes
S = {John promoted, Rita promoted, Aslam promoted, Gurpreet promoted}
You are told that the chances of John’s promotion is same as that of Gurpreet,
Rita’s chances of promotion are twice as likely as Johns. Aslam’s chances are
four times that of John.
\begin{enumerate}
	\item Determine
	\begin{enumerate}
		\item P (John promoted)
		\item P (Rita promoted)
		\item P (Aslam promoted)
		\item P (Gurpreet promoted)
	\end{enumerate}
	\item If A = {John promoted or Gurpreet promoted}, find P (A).
\end{enumerate}
\solution
%\input{exemplar/11/16/3/10/main.tex}
\item A card is drawn from a deck of 52 cards. Find the probability of getting a king or a heart or a red card.\\
\solution
%\input{exemplar/11/16/3/15/main.tex}
\item The probability that a student will pass his examination is 0.73, the probability of
the student getting a compartment is 0.13, and the probability that the student will
either pass or get compartment is 0.96. State True or False.\\
\solution
%\input{exemplar/11/16/3/31/main.tex}
\item A card is selected from a pack of 52 cards\\
\begin{enumerate}[label=(\alph*)]
\item How many points are there in the sample space?
\item Calculate the probability that the cards is an ace of spades.
\item Calculate the probability that the card is (i) an ace (ii)black card.\\
\end{enumerate}
%\input{ncert/11/16/3/4_1/Prob_4.tex}
\item In a non-leap year, the probability of having 53 tuesdays or 53 wednesdays is\\
\solution
%\input{exemplar/11/16/3/18/main.tex}
\item There are 1000 sealed envelopes in a box, 10 of them contain a cash prize of
Rs 100 each, 100 of them contain a cash prize of Rs 50 each and 200 of them
contain a cash prize of Rs 10 each and rest do not contain any cash prize. If they
are well shuffled and an envelope is picked up out, what is the probability that it
contains no cash prize?\\
\solution
%\input{exemplar/10/13/3/34/main.tex}
\item 
A die is thrown and a card is selected at random from a deck of 52 playing cards. The probability of getting an even number on the die and a spade card.\\
\solution
%\input{exemplar/12/13/3/78/main.tex}
\item
If 4-digit numbers greater than 5,000 are randomly formed from the digits 0, 1, 3, 5, and 7, what is the probability of forming a number divisible by 5 when:
\begin{enumerate}
    \item The digits are repeated?
    \item The repetition of digits is not allowed?
\end{enumerate}
\solution
%\input{ncert/11/16/4/9/main.tex}
\item Consider the probability space $\brak{\Omega, \mathcal{G}, P}$ where $\Omega = [0,2]$ and $\mathcal{G} = \cbrak{\phi, \Omega, [0,1], (1,2]}$. Let $X$ and $Y$ be two functions on $\Omega$ defined as
\begin{align*}
    X(\omega) = 
    \begin{cases}
        1 & \text{if }\omega \in [0, 1]\\
        2 & \text{if }\omega \in (1, 2]
    \end{cases}
\end{align*}
and
\begin{align*}
    Y(\omega) = 
    \begin{cases}
        2 & \text{if }\omega \in [0, 1.5]\\
        3 & \text{if }\omega \in (1.5, 2].
    \end{cases}
\end{align*}
Then which one of the following statements is true?
\begin{enumerate}
    \item [(A)] $X$ is a random variable with respect to $\mathcal{G}$, but $Y$ is not a random variable with respect to $\mathcal{G}$.
    \item [(B)] $Y$ is a random variable with respect to $\mathcal{G}$, but $X$ is not a random variable with respect to $\mathcal{G}$.
    \item [(C)] Neither $X$ nor $Y$ is a random variable with respect to $\mathcal{G}$.
    \item [(D)] Both $X$ and $Y$ are random variables with respect to $\mathcal{G}$.
\end{enumerate} \hfill (GATE ST 2023)\\
\solution
%\input{gate/ST/2023/14/main.tex}
	\item  A die is loaded in such a way that each odd number is twice as likely to occur as
each even number. Find $P(G)$, where $G$ is the event that a number greater than
3 occurs on a single roll of the die.
\\
\solution
		%\input{exemplar/11/16/3/5/main.tex}
	\item All the jacks, queens and kings are removed from a deck of 52 playing cards. The remaining cards are well shuffled and then one card is drawn at random. Giving ace a value 1 similar value for other cards, find the probability that the card has a value 
		\begin{enumerate}
			\item 7
			\item greater than 7
			\item less than 7
		\end{enumerate}
		%\input{exemplar/10/13/3/30/main.tex}
  \item A Lot consists of 48 mobile phones of which 42 are good, 3 have only minor defects and 3 have major defects.Varnika will buy a phone if it is good but the trader will only buy a mobile if it has no major defects. One phone is selected at random from the lot. What is the probability that it is
\begin{enumerate}
	\item acceptable to Varnika?
            \item acceptable to the trader?
\end{enumerate}
\solution
	%\input{exemplar/10/13/3/40/main.tex}
 \item A student says that if you throw a die, it will show up 1 or not 1. Therefore, the probability of getting 1 and the probability of getting 'not 1' each is equal to $\frac{1}{2}$. Is this correct? Give reasons.\\
 \solution
        %\input{exemplar/10/13/2/9/main.tex}
   \item Four candidates A, B, C, D have ap-
plied for the assignment to coach a school cricket
team. If A is twice as likely to be selected as B, and
B and C are given about the same chance of being
selected, while C is twice as likely to be selected
as D, what are the probabilities that
\begin{enumerate}
\item C will be selected?
\item A will not be selected?
\end{enumerate}
	%\input{exemplar/11/16/3/9/main.tex}
 \item A bag contain 24 balls of which $x$ balls are red, $2x$ are white and $3x$ are blue. A ball is selected at random, What is the probability that it is
\begin{enumerate}[label=\alph*)]
\item not red ?
\item white ?
\end{enumerate}
%\input{exemplar/10/13/3/41/main.tex}
If the letters of the word ASSASSINATION are arranged at random. Find the Probability that
\begin{enumerate}[label=(\alph*)]
\item Four $S's$ come consecutively in the word
\item Two  $I's$ and two $N's$ come together
\item All $A's$ are not coming together
\item No two $A's$ are coming together
\end{enumerate}
%\input{exemplar/11/16/3/14/main.tex}
	\item One urn contains two black balls (labelled B1 and B2) and one white ball. A
	second urn contains one black ball and two white balls (labelled W1 and W2).
	Suppose the following experiment is performed. One of the two urns is chosen
	at random. Next a ball is randomly chosen from the urn. Then a second ball is
	chosen at random from the same urn without replacing the first ball.
	
	\begin{enumerate}
	\item What is the probability that two black balls are chosen?
	
	\item What is the probability that two balls of opposite colour are chosen?
	\end{enumerate}
	\solution
	%\input{exemplar/11/16/3/12/main1.tex}
\end{enumerate}

	\item A card is selected from a pack of 52 cards.
 \begin{enumerate}[label=(\alph*)] 
                 \item How many points are there in the sample space?
                 \item Calculate the probability that the card is an ace of spades.
                 \item Calculate the probability that the card is (i) an ace and (ii) black card.
 \end{enumerate}
\solution
		%\begin{table}[H]
	\centering
\begin{tabular}{|c|c|c|}
\hline
Random variable &Value &Definition\\ \hline
\multirow{3}{*}{X} &0 &Slips of Rs 1\\
&1 &Slips of Rs 5\\
&2 &Slips of Rs 13\\ \hline
\multirow{2}{*}{Y} &0 &Box A\\
&1 &Box B\\\hline
\end{tabular}
\caption{}
\label{tab:Distribution}
\end{table}
See \tabref{tab:Distribution}.
\begin{align}
p_{Y}\brak{k}= \begin{cases} 
      \frac{1}{3} & {k=0} \\
      \frac{2}{3 }& {k=1} 
   \end{cases}
   \\
p_{Y|X}\brak{0|0} = \frac{19}{25}\, 
p_{Y|X}\brak{0|1} = \frac{6}{25}\,
p_{Y|X}\brak{1|0} = \frac{45}{50}\,
p_{Y|X}\brak{1|2} = \frac{5}{50}
\end{align}
The desired probability is the probability that a slip drawn at random is marked other than Rs 1,
\begin{align}
&=1-p_X\brak{0}\\
&= p_X(1) + p_X(2)
\end{align}
Using Bayes theorem,
\begin{align}
&= p_Y\brak{0} \times \pr{Y=0 | X=1} + p_Y\brak{1} \times \pr{Y=1|X=2}\\
&=\frac{1}{3} \times \frac{6}{25} + \frac{2}{3} \times \frac{5}{50}\\
&=\frac{11}{75}
\end{align}

\newpage

%\tableofcontents

\bigskip

\renewcommand{\thefigure}{\theenumi}
\renewcommand{\thetable}{\theenumi}
%\renewcommand{\theequation}{\theenumi}

%\begin{abstract}
%%\boldmath
%In this letter, an algorithm for evaluating the exact analytical bit error rate  (BER)  for the piecewise linear (PL) combiner for  multiple relays is presented. Previous results were available only for upto three relays. The algorithm is unique in the sense that  the actual mathematical expressions, that are prohibitively large, need not be explicitly obtained. The diversity gain due to multiple relays is shown through plots of the analytical BER, well supported by simulations. 
%
%\end{abstract}
% IEEEtran.cls defaults to using nonbold math in the Abstract.
% This preserves the distinction between vectors and scalars. However,
% if the journal you are submitting to favors bold math in the abstract,
% then you can use LaTeX's standard command \boldmath at the very start
% of the abstract to achieve this. Many IEEE journals frown on math
% in the abstract anyway.

% Note that keywords are not normally used for peerreview papers.
%\begin{IEEEkeywords}
%Cooperative diversity, decode and forward, piecewise linear
%\end{IEEEkeywords}



% For peer review papers, you can put extra information on the cover
% page as needed:
% \ifCLASSOPTIONpeerreview
% \begin{center} \bfseries EDICS Category: 3-BBND \end{center}
% \fi
%
% For peerreview papers, this IEEEtran command inserts a page break and
% creates the second title. It will be ignored for other modes.
%\IEEEpeerreviewmaketitle




\item Four cards are drawn from a well-shuffled deck of 52 cards. What is the probability of obtaining 3 diamonds and one spade.
\\
\solution
		%\begin{enumerate}[label=\thesection.\arabic*,ref=\thesection.\theenumi]
	\item One card is drawn from a well-shuffled deck of 52 cards. Find the probability of getting
\begin{enumerate}
\item A king of red colour 
\item A face card 
\item A red face card
\item The jack of hearts
\item A spade
\item The queen of diamonds

\end{enumerate}
\solution
		%\input{ncert/10/15/1/14/main.tex}
	\item Five cards—the ten, jack, queen, king and ace of diamonds, are well-shuffled with their face downwards. One card is then picked up at random.
\begin{enumerate}
\item
What is the probability that the card is the queen? 
\item
If the queen is drawn and put aside, what is the probability that the second card picked up is (a) an ace? (b) a queen?\\
\end{enumerate}
\solution
		%\input{ncert/10/15/1/15/defs.tex}
	\item A bag contains $5$ red balls and some blue balls. If the probability of drawing a blue ball is double that if a red ball, determine the number of blue balls in the bag. 
		\\
\solution
		%\input{ncert/10/15/2/3/defs.tex}
	\item A card is selected from a pack of 52 cards.
 \begin{enumerate}[label=(\alph*)] 
                 \item How many points are there in the sample space?
                 \item Calculate the probability that the card is an ace of spades.
                 \item Calculate the probability that the card is (i) an ace and (ii) black card.
 \end{enumerate}
\solution
		%\input{ncert/11/16/3/4/main.tex}
\item Four cards are drawn from a well-shuffled deck of 52 cards. What is the probability of obtaining 3 diamonds and one spade.
\\
\solution
		%\input{ncert/11/16/4/2/defs.tex}
\item In a certain lottery 10,000 tickets are sold and ten equal prizes are awarded. What is the probability of not getting a prize if you buy (a) one ticket (b) two tickets (c) 10 tickets ?	
\\
\solution
		%\input{ncert/11/16/4/4/defs.tex}
		%
\item 
Out of 100 students, two sections of 40 and 60 are formed. If you and your friend are among the 100 students, what is the probability that
\begin{enumerate}
\item you both enter the same section?
\item you both enter the different sections?
\end{enumerate}
\solution
		%\input{ncert/11/16/4/5/defs.tex}
	\item 
The number lock of a suitcase has 4 wheels each labelled with ten digits i.e. from 0 to 9.The lock opens with a sequence of four digits with no repeats.What is the probability of a person getting the right sequence to open the suitcase.
\\
\solution
		%\input{ncert/11/16/4/10/defs.tex}
		%
\item 
Two cards are drawn at random and without replacement from a pack of 52 playing cards. Find the probability that both the cards are black.
\\
\solution
		%\input{ncert/12/13/2/2/defs.tex}
		\item A box of oranges is inspected by examining three randomly selected oranges drawn without replacement. If all the three oranges are good, the box is approved for sale, otherwise, it is rejected. Find the probability that a box containing 15 oranges out of which 12 are good and 3 are bad ones will be approved for sale.
		\label{ncert/12/13/2/3/defs.tex}
		\item Two balls are drawn at random with replacement from a box containing 10 black and 8 red balls. Find the probability that
		\label{ncert/12/13/2/12}
\begin{enumerate}
\item both balls are red.
\item first ball is black and second is red.
\item one of them is black and other is red.
\end{enumerate}

\item In a hostel, 60\% of the students read Hindi newspaper, 40\% read English newspaper and 20\% read both Hindi and English newspapers. A student is selected at random.
		\label{ncert/12/13/2/15}
\begin{enumerate}
\item Find the probability that she reads neither Hindi nor English newspapers.
\item If she reads Hindi newspaper, find the probability that she reads English newspaper.
\item If she reads English newspaper, find the probability that she reads Hindi newspaper.\\
\end{enumerate}
\item The probability of obtaining an even prime number on each die, when a pair of dice is rolled is 
\begin{enumerate}
    \item $0$ 
    
    \item $\frac{1}{3}$ 
    
    \item $\frac{1}{12}$ 
    
    \item $\frac{1}{36}$ 
\end{enumerate}
\solution
		%\input{ncert/12/13/2/17/defs.tex}
	\item A bag contains 4 red and 4 black balls, another bag contains 2 red and 6 black balls. One of the two bags is selected at random and a ball is drawn from the bag which is found to be red. Find the probability that the ball is drawn from the first bag.
\\
\solution
		%\input{ncert/12/13/3/2/main.tex}
  \item
  Cards with numbers 2 to 101 are placed in a box. A card is selected at random.Find the probability that the card has
\begin{enumerate}[label=(\roman*)]
	\item an even number 
	\item a square number
\end{enumerate}
\solution
%\input{exemplar/10/13/3/32/main.tex}
\item
The king, queen and jack of clubs are removed from a deck of 52 playing cards and then well shuffled. Now one card is drawn at random from the remaining cards.  Determine the probability that the card is
\begin{enumerate}[label=(\roman*)]
\item a club
\item 10 of hearts
\end{enumerate}
\solution
%\input{exemplar/10/13/3/29/main.tex}
\item A team of medical students doing their internship have to assist during surgeries
at a city hospital. The probabilities of surgeries rated as very complex, complex,
routine, simple or very simple are respectively, 0.15, 0.20, 0.31, 0.26, .08. Find
the probabilities that a particular surgery will be rated
\begin{enumerate}
	\item complex or very complex;
	\item neither very complex nor very simple;
	\item routine or complex
	\item routine or simple
\end{enumerate}
\solution
%\input{exemplar/11/16/3/8(1)/main.tex}
\item A card is selected from a pack of 52 cards.
\begin{enumerate}[label=(\alph*)]
    \item How many points are there in the sample space?
    \item Calculate the probability that the card is an ace of spades.
    \item Calculate the probability that the card is (i) an ace and (ii) black card.
\end{enumerate}
\solution
%\input{exemplar/11/16/3/4/main2.tex}
\item The probability that a non leap year selected at random will contain 53 sundays.
\\
\solution
%\input{exemplar/10/13/1/19/main.tex}
\item One of the four persons John, Rita, Aslam or Gurpreet will be promoted next
month. Consequently the sample space consists of four elementary outcomes
S = {John promoted, Rita promoted, Aslam promoted, Gurpreet promoted}
You are told that the chances of John’s promotion is same as that of Gurpreet,
Rita’s chances of promotion are twice as likely as Johns. Aslam’s chances are
four times that of John.
\begin{enumerate}
	\item Determine
	\begin{enumerate}
		\item P (John promoted)
		\item P (Rita promoted)
		\item P (Aslam promoted)
		\item P (Gurpreet promoted)
	\end{enumerate}
	\item If A = {John promoted or Gurpreet promoted}, find P (A).
\end{enumerate}
\solution
%\input{exemplar/11/16/3/10/main.tex}
\item A card is drawn from a deck of 52 cards. Find the probability of getting a king or a heart or a red card.\\
\solution
%\input{exemplar/11/16/3/15/main.tex}
\item The probability that a student will pass his examination is 0.73, the probability of
the student getting a compartment is 0.13, and the probability that the student will
either pass or get compartment is 0.96. State True or False.\\
\solution
%\input{exemplar/11/16/3/31/main.tex}
\item A card is selected from a pack of 52 cards\\
\begin{enumerate}[label=(\alph*)]
\item How many points are there in the sample space?
\item Calculate the probability that the cards is an ace of spades.
\item Calculate the probability that the card is (i) an ace (ii)black card.\\
\end{enumerate}
%\input{ncert/11/16/3/4_1/Prob_4.tex}
\item In a non-leap year, the probability of having 53 tuesdays or 53 wednesdays is\\
\solution
%\input{exemplar/11/16/3/18/main.tex}
\item There are 1000 sealed envelopes in a box, 10 of them contain a cash prize of
Rs 100 each, 100 of them contain a cash prize of Rs 50 each and 200 of them
contain a cash prize of Rs 10 each and rest do not contain any cash prize. If they
are well shuffled and an envelope is picked up out, what is the probability that it
contains no cash prize?\\
\solution
%\input{exemplar/10/13/3/34/main.tex}
\item 
A die is thrown and a card is selected at random from a deck of 52 playing cards. The probability of getting an even number on the die and a spade card.\\
\solution
%\input{exemplar/12/13/3/78/main.tex}
\item
If 4-digit numbers greater than 5,000 are randomly formed from the digits 0, 1, 3, 5, and 7, what is the probability of forming a number divisible by 5 when:
\begin{enumerate}
    \item The digits are repeated?
    \item The repetition of digits is not allowed?
\end{enumerate}
\solution
%\input{ncert/11/16/4/9/main.tex}
\item Consider the probability space $\brak{\Omega, \mathcal{G}, P}$ where $\Omega = [0,2]$ and $\mathcal{G} = \cbrak{\phi, \Omega, [0,1], (1,2]}$. Let $X$ and $Y$ be two functions on $\Omega$ defined as
\begin{align*}
    X(\omega) = 
    \begin{cases}
        1 & \text{if }\omega \in [0, 1]\\
        2 & \text{if }\omega \in (1, 2]
    \end{cases}
\end{align*}
and
\begin{align*}
    Y(\omega) = 
    \begin{cases}
        2 & \text{if }\omega \in [0, 1.5]\\
        3 & \text{if }\omega \in (1.5, 2].
    \end{cases}
\end{align*}
Then which one of the following statements is true?
\begin{enumerate}
    \item [(A)] $X$ is a random variable with respect to $\mathcal{G}$, but $Y$ is not a random variable with respect to $\mathcal{G}$.
    \item [(B)] $Y$ is a random variable with respect to $\mathcal{G}$, but $X$ is not a random variable with respect to $\mathcal{G}$.
    \item [(C)] Neither $X$ nor $Y$ is a random variable with respect to $\mathcal{G}$.
    \item [(D)] Both $X$ and $Y$ are random variables with respect to $\mathcal{G}$.
\end{enumerate} \hfill (GATE ST 2023)\\
\solution
%\input{gate/ST/2023/14/main.tex}
	\item  A die is loaded in such a way that each odd number is twice as likely to occur as
each even number. Find $P(G)$, where $G$ is the event that a number greater than
3 occurs on a single roll of the die.
\\
\solution
		%\input{exemplar/11/16/3/5/main.tex}
	\item All the jacks, queens and kings are removed from a deck of 52 playing cards. The remaining cards are well shuffled and then one card is drawn at random. Giving ace a value 1 similar value for other cards, find the probability that the card has a value 
		\begin{enumerate}
			\item 7
			\item greater than 7
			\item less than 7
		\end{enumerate}
		%\input{exemplar/10/13/3/30/main.tex}
  \item A Lot consists of 48 mobile phones of which 42 are good, 3 have only minor defects and 3 have major defects.Varnika will buy a phone if it is good but the trader will only buy a mobile if it has no major defects. One phone is selected at random from the lot. What is the probability that it is
\begin{enumerate}
	\item acceptable to Varnika?
            \item acceptable to the trader?
\end{enumerate}
\solution
	%\input{exemplar/10/13/3/40/main.tex}
 \item A student says that if you throw a die, it will show up 1 or not 1. Therefore, the probability of getting 1 and the probability of getting 'not 1' each is equal to $\frac{1}{2}$. Is this correct? Give reasons.\\
 \solution
        %\input{exemplar/10/13/2/9/main.tex}
   \item Four candidates A, B, C, D have ap-
plied for the assignment to coach a school cricket
team. If A is twice as likely to be selected as B, and
B and C are given about the same chance of being
selected, while C is twice as likely to be selected
as D, what are the probabilities that
\begin{enumerate}
\item C will be selected?
\item A will not be selected?
\end{enumerate}
	%\input{exemplar/11/16/3/9/main.tex}
 \item A bag contain 24 balls of which $x$ balls are red, $2x$ are white and $3x$ are blue. A ball is selected at random, What is the probability that it is
\begin{enumerate}[label=\alph*)]
\item not red ?
\item white ?
\end{enumerate}
%\input{exemplar/10/13/3/41/main.tex}
If the letters of the word ASSASSINATION are arranged at random. Find the Probability that
\begin{enumerate}[label=(\alph*)]
\item Four $S's$ come consecutively in the word
\item Two  $I's$ and two $N's$ come together
\item All $A's$ are not coming together
\item No two $A's$ are coming together
\end{enumerate}
%\input{exemplar/11/16/3/14/main.tex}
	\item One urn contains two black balls (labelled B1 and B2) and one white ball. A
	second urn contains one black ball and two white balls (labelled W1 and W2).
	Suppose the following experiment is performed. One of the two urns is chosen
	at random. Next a ball is randomly chosen from the urn. Then a second ball is
	chosen at random from the same urn without replacing the first ball.
	
	\begin{enumerate}
	\item What is the probability that two black balls are chosen?
	
	\item What is the probability that two balls of opposite colour are chosen?
	\end{enumerate}
	\solution
	%\input{exemplar/11/16/3/12/main1.tex}
\end{enumerate}

\item In a certain lottery 10,000 tickets are sold and ten equal prizes are awarded. What is the probability of not getting a prize if you buy (a) one ticket (b) two tickets (c) 10 tickets ?	
\\
\solution
		%\begin{enumerate}[label=\thesection.\arabic*,ref=\thesection.\theenumi]
	\item One card is drawn from a well-shuffled deck of 52 cards. Find the probability of getting
\begin{enumerate}
\item A king of red colour 
\item A face card 
\item A red face card
\item The jack of hearts
\item A spade
\item The queen of diamonds

\end{enumerate}
\solution
		%\input{ncert/10/15/1/14/main.tex}
	\item Five cards—the ten, jack, queen, king and ace of diamonds, are well-shuffled with their face downwards. One card is then picked up at random.
\begin{enumerate}
\item
What is the probability that the card is the queen? 
\item
If the queen is drawn and put aside, what is the probability that the second card picked up is (a) an ace? (b) a queen?\\
\end{enumerate}
\solution
		%\input{ncert/10/15/1/15/defs.tex}
	\item A bag contains $5$ red balls and some blue balls. If the probability of drawing a blue ball is double that if a red ball, determine the number of blue balls in the bag. 
		\\
\solution
		%\input{ncert/10/15/2/3/defs.tex}
	\item A card is selected from a pack of 52 cards.
 \begin{enumerate}[label=(\alph*)] 
                 \item How many points are there in the sample space?
                 \item Calculate the probability that the card is an ace of spades.
                 \item Calculate the probability that the card is (i) an ace and (ii) black card.
 \end{enumerate}
\solution
		%\input{ncert/11/16/3/4/main.tex}
\item Four cards are drawn from a well-shuffled deck of 52 cards. What is the probability of obtaining 3 diamonds and one spade.
\\
\solution
		%\input{ncert/11/16/4/2/defs.tex}
\item In a certain lottery 10,000 tickets are sold and ten equal prizes are awarded. What is the probability of not getting a prize if you buy (a) one ticket (b) two tickets (c) 10 tickets ?	
\\
\solution
		%\input{ncert/11/16/4/4/defs.tex}
		%
\item 
Out of 100 students, two sections of 40 and 60 are formed. If you and your friend are among the 100 students, what is the probability that
\begin{enumerate}
\item you both enter the same section?
\item you both enter the different sections?
\end{enumerate}
\solution
		%\input{ncert/11/16/4/5/defs.tex}
	\item 
The number lock of a suitcase has 4 wheels each labelled with ten digits i.e. from 0 to 9.The lock opens with a sequence of four digits with no repeats.What is the probability of a person getting the right sequence to open the suitcase.
\\
\solution
		%\input{ncert/11/16/4/10/defs.tex}
		%
\item 
Two cards are drawn at random and without replacement from a pack of 52 playing cards. Find the probability that both the cards are black.
\\
\solution
		%\input{ncert/12/13/2/2/defs.tex}
		\item A box of oranges is inspected by examining three randomly selected oranges drawn without replacement. If all the three oranges are good, the box is approved for sale, otherwise, it is rejected. Find the probability that a box containing 15 oranges out of which 12 are good and 3 are bad ones will be approved for sale.
		\label{ncert/12/13/2/3/defs.tex}
		\item Two balls are drawn at random with replacement from a box containing 10 black and 8 red balls. Find the probability that
		\label{ncert/12/13/2/12}
\begin{enumerate}
\item both balls are red.
\item first ball is black and second is red.
\item one of them is black and other is red.
\end{enumerate}

\item In a hostel, 60\% of the students read Hindi newspaper, 40\% read English newspaper and 20\% read both Hindi and English newspapers. A student is selected at random.
		\label{ncert/12/13/2/15}
\begin{enumerate}
\item Find the probability that she reads neither Hindi nor English newspapers.
\item If she reads Hindi newspaper, find the probability that she reads English newspaper.
\item If she reads English newspaper, find the probability that she reads Hindi newspaper.\\
\end{enumerate}
\item The probability of obtaining an even prime number on each die, when a pair of dice is rolled is 
\begin{enumerate}
    \item $0$ 
    
    \item $\frac{1}{3}$ 
    
    \item $\frac{1}{12}$ 
    
    \item $\frac{1}{36}$ 
\end{enumerate}
\solution
		%\input{ncert/12/13/2/17/defs.tex}
	\item A bag contains 4 red and 4 black balls, another bag contains 2 red and 6 black balls. One of the two bags is selected at random and a ball is drawn from the bag which is found to be red. Find the probability that the ball is drawn from the first bag.
\\
\solution
		%\input{ncert/12/13/3/2/main.tex}
  \item
  Cards with numbers 2 to 101 are placed in a box. A card is selected at random.Find the probability that the card has
\begin{enumerate}[label=(\roman*)]
	\item an even number 
	\item a square number
\end{enumerate}
\solution
%\input{exemplar/10/13/3/32/main.tex}
\item
The king, queen and jack of clubs are removed from a deck of 52 playing cards and then well shuffled. Now one card is drawn at random from the remaining cards.  Determine the probability that the card is
\begin{enumerate}[label=(\roman*)]
\item a club
\item 10 of hearts
\end{enumerate}
\solution
%\input{exemplar/10/13/3/29/main.tex}
\item A team of medical students doing their internship have to assist during surgeries
at a city hospital. The probabilities of surgeries rated as very complex, complex,
routine, simple or very simple are respectively, 0.15, 0.20, 0.31, 0.26, .08. Find
the probabilities that a particular surgery will be rated
\begin{enumerate}
	\item complex or very complex;
	\item neither very complex nor very simple;
	\item routine or complex
	\item routine or simple
\end{enumerate}
\solution
%\input{exemplar/11/16/3/8(1)/main.tex}
\item A card is selected from a pack of 52 cards.
\begin{enumerate}[label=(\alph*)]
    \item How many points are there in the sample space?
    \item Calculate the probability that the card is an ace of spades.
    \item Calculate the probability that the card is (i) an ace and (ii) black card.
\end{enumerate}
\solution
%\input{exemplar/11/16/3/4/main2.tex}
\item The probability that a non leap year selected at random will contain 53 sundays.
\\
\solution
%\input{exemplar/10/13/1/19/main.tex}
\item One of the four persons John, Rita, Aslam or Gurpreet will be promoted next
month. Consequently the sample space consists of four elementary outcomes
S = {John promoted, Rita promoted, Aslam promoted, Gurpreet promoted}
You are told that the chances of John’s promotion is same as that of Gurpreet,
Rita’s chances of promotion are twice as likely as Johns. Aslam’s chances are
four times that of John.
\begin{enumerate}
	\item Determine
	\begin{enumerate}
		\item P (John promoted)
		\item P (Rita promoted)
		\item P (Aslam promoted)
		\item P (Gurpreet promoted)
	\end{enumerate}
	\item If A = {John promoted or Gurpreet promoted}, find P (A).
\end{enumerate}
\solution
%\input{exemplar/11/16/3/10/main.tex}
\item A card is drawn from a deck of 52 cards. Find the probability of getting a king or a heart or a red card.\\
\solution
%\input{exemplar/11/16/3/15/main.tex}
\item The probability that a student will pass his examination is 0.73, the probability of
the student getting a compartment is 0.13, and the probability that the student will
either pass or get compartment is 0.96. State True or False.\\
\solution
%\input{exemplar/11/16/3/31/main.tex}
\item A card is selected from a pack of 52 cards\\
\begin{enumerate}[label=(\alph*)]
\item How many points are there in the sample space?
\item Calculate the probability that the cards is an ace of spades.
\item Calculate the probability that the card is (i) an ace (ii)black card.\\
\end{enumerate}
%\input{ncert/11/16/3/4_1/Prob_4.tex}
\item In a non-leap year, the probability of having 53 tuesdays or 53 wednesdays is\\
\solution
%\input{exemplar/11/16/3/18/main.tex}
\item There are 1000 sealed envelopes in a box, 10 of them contain a cash prize of
Rs 100 each, 100 of them contain a cash prize of Rs 50 each and 200 of them
contain a cash prize of Rs 10 each and rest do not contain any cash prize. If they
are well shuffled and an envelope is picked up out, what is the probability that it
contains no cash prize?\\
\solution
%\input{exemplar/10/13/3/34/main.tex}
\item 
A die is thrown and a card is selected at random from a deck of 52 playing cards. The probability of getting an even number on the die and a spade card.\\
\solution
%\input{exemplar/12/13/3/78/main.tex}
\item
If 4-digit numbers greater than 5,000 are randomly formed from the digits 0, 1, 3, 5, and 7, what is the probability of forming a number divisible by 5 when:
\begin{enumerate}
    \item The digits are repeated?
    \item The repetition of digits is not allowed?
\end{enumerate}
\solution
%\input{ncert/11/16/4/9/main.tex}
\item Consider the probability space $\brak{\Omega, \mathcal{G}, P}$ where $\Omega = [0,2]$ and $\mathcal{G} = \cbrak{\phi, \Omega, [0,1], (1,2]}$. Let $X$ and $Y$ be two functions on $\Omega$ defined as
\begin{align*}
    X(\omega) = 
    \begin{cases}
        1 & \text{if }\omega \in [0, 1]\\
        2 & \text{if }\omega \in (1, 2]
    \end{cases}
\end{align*}
and
\begin{align*}
    Y(\omega) = 
    \begin{cases}
        2 & \text{if }\omega \in [0, 1.5]\\
        3 & \text{if }\omega \in (1.5, 2].
    \end{cases}
\end{align*}
Then which one of the following statements is true?
\begin{enumerate}
    \item [(A)] $X$ is a random variable with respect to $\mathcal{G}$, but $Y$ is not a random variable with respect to $\mathcal{G}$.
    \item [(B)] $Y$ is a random variable with respect to $\mathcal{G}$, but $X$ is not a random variable with respect to $\mathcal{G}$.
    \item [(C)] Neither $X$ nor $Y$ is a random variable with respect to $\mathcal{G}$.
    \item [(D)] Both $X$ and $Y$ are random variables with respect to $\mathcal{G}$.
\end{enumerate} \hfill (GATE ST 2023)\\
\solution
%\input{gate/ST/2023/14/main.tex}
	\item  A die is loaded in such a way that each odd number is twice as likely to occur as
each even number. Find $P(G)$, where $G$ is the event that a number greater than
3 occurs on a single roll of the die.
\\
\solution
		%\input{exemplar/11/16/3/5/main.tex}
	\item All the jacks, queens and kings are removed from a deck of 52 playing cards. The remaining cards are well shuffled and then one card is drawn at random. Giving ace a value 1 similar value for other cards, find the probability that the card has a value 
		\begin{enumerate}
			\item 7
			\item greater than 7
			\item less than 7
		\end{enumerate}
		%\input{exemplar/10/13/3/30/main.tex}
  \item A Lot consists of 48 mobile phones of which 42 are good, 3 have only minor defects and 3 have major defects.Varnika will buy a phone if it is good but the trader will only buy a mobile if it has no major defects. One phone is selected at random from the lot. What is the probability that it is
\begin{enumerate}
	\item acceptable to Varnika?
            \item acceptable to the trader?
\end{enumerate}
\solution
	%\input{exemplar/10/13/3/40/main.tex}
 \item A student says that if you throw a die, it will show up 1 or not 1. Therefore, the probability of getting 1 and the probability of getting 'not 1' each is equal to $\frac{1}{2}$. Is this correct? Give reasons.\\
 \solution
        %\input{exemplar/10/13/2/9/main.tex}
   \item Four candidates A, B, C, D have ap-
plied for the assignment to coach a school cricket
team. If A is twice as likely to be selected as B, and
B and C are given about the same chance of being
selected, while C is twice as likely to be selected
as D, what are the probabilities that
\begin{enumerate}
\item C will be selected?
\item A will not be selected?
\end{enumerate}
	%\input{exemplar/11/16/3/9/main.tex}
 \item A bag contain 24 balls of which $x$ balls are red, $2x$ are white and $3x$ are blue. A ball is selected at random, What is the probability that it is
\begin{enumerate}[label=\alph*)]
\item not red ?
\item white ?
\end{enumerate}
%\input{exemplar/10/13/3/41/main.tex}
If the letters of the word ASSASSINATION are arranged at random. Find the Probability that
\begin{enumerate}[label=(\alph*)]
\item Four $S's$ come consecutively in the word
\item Two  $I's$ and two $N's$ come together
\item All $A's$ are not coming together
\item No two $A's$ are coming together
\end{enumerate}
%\input{exemplar/11/16/3/14/main.tex}
	\item One urn contains two black balls (labelled B1 and B2) and one white ball. A
	second urn contains one black ball and two white balls (labelled W1 and W2).
	Suppose the following experiment is performed. One of the two urns is chosen
	at random. Next a ball is randomly chosen from the urn. Then a second ball is
	chosen at random from the same urn without replacing the first ball.
	
	\begin{enumerate}
	\item What is the probability that two black balls are chosen?
	
	\item What is the probability that two balls of opposite colour are chosen?
	\end{enumerate}
	\solution
	%\input{exemplar/11/16/3/12/main1.tex}
\end{enumerate}

		%
\item 
Out of 100 students, two sections of 40 and 60 are formed. If you and your friend are among the 100 students, what is the probability that
\begin{enumerate}
\item you both enter the same section?
\item you both enter the different sections?
\end{enumerate}
\solution
		%\begin{enumerate}[label=\thesection.\arabic*,ref=\thesection.\theenumi]
	\item One card is drawn from a well-shuffled deck of 52 cards. Find the probability of getting
\begin{enumerate}
\item A king of red colour 
\item A face card 
\item A red face card
\item The jack of hearts
\item A spade
\item The queen of diamonds

\end{enumerate}
\solution
		%\input{ncert/10/15/1/14/main.tex}
	\item Five cards—the ten, jack, queen, king and ace of diamonds, are well-shuffled with their face downwards. One card is then picked up at random.
\begin{enumerate}
\item
What is the probability that the card is the queen? 
\item
If the queen is drawn and put aside, what is the probability that the second card picked up is (a) an ace? (b) a queen?\\
\end{enumerate}
\solution
		%\input{ncert/10/15/1/15/defs.tex}
	\item A bag contains $5$ red balls and some blue balls. If the probability of drawing a blue ball is double that if a red ball, determine the number of blue balls in the bag. 
		\\
\solution
		%\input{ncert/10/15/2/3/defs.tex}
	\item A card is selected from a pack of 52 cards.
 \begin{enumerate}[label=(\alph*)] 
                 \item How many points are there in the sample space?
                 \item Calculate the probability that the card is an ace of spades.
                 \item Calculate the probability that the card is (i) an ace and (ii) black card.
 \end{enumerate}
\solution
		%\input{ncert/11/16/3/4/main.tex}
\item Four cards are drawn from a well-shuffled deck of 52 cards. What is the probability of obtaining 3 diamonds and one spade.
\\
\solution
		%\input{ncert/11/16/4/2/defs.tex}
\item In a certain lottery 10,000 tickets are sold and ten equal prizes are awarded. What is the probability of not getting a prize if you buy (a) one ticket (b) two tickets (c) 10 tickets ?	
\\
\solution
		%\input{ncert/11/16/4/4/defs.tex}
		%
\item 
Out of 100 students, two sections of 40 and 60 are formed. If you and your friend are among the 100 students, what is the probability that
\begin{enumerate}
\item you both enter the same section?
\item you both enter the different sections?
\end{enumerate}
\solution
		%\input{ncert/11/16/4/5/defs.tex}
	\item 
The number lock of a suitcase has 4 wheels each labelled with ten digits i.e. from 0 to 9.The lock opens with a sequence of four digits with no repeats.What is the probability of a person getting the right sequence to open the suitcase.
\\
\solution
		%\input{ncert/11/16/4/10/defs.tex}
		%
\item 
Two cards are drawn at random and without replacement from a pack of 52 playing cards. Find the probability that both the cards are black.
\\
\solution
		%\input{ncert/12/13/2/2/defs.tex}
		\item A box of oranges is inspected by examining three randomly selected oranges drawn without replacement. If all the three oranges are good, the box is approved for sale, otherwise, it is rejected. Find the probability that a box containing 15 oranges out of which 12 are good and 3 are bad ones will be approved for sale.
		\label{ncert/12/13/2/3/defs.tex}
		\item Two balls are drawn at random with replacement from a box containing 10 black and 8 red balls. Find the probability that
		\label{ncert/12/13/2/12}
\begin{enumerate}
\item both balls are red.
\item first ball is black and second is red.
\item one of them is black and other is red.
\end{enumerate}

\item In a hostel, 60\% of the students read Hindi newspaper, 40\% read English newspaper and 20\% read both Hindi and English newspapers. A student is selected at random.
		\label{ncert/12/13/2/15}
\begin{enumerate}
\item Find the probability that she reads neither Hindi nor English newspapers.
\item If she reads Hindi newspaper, find the probability that she reads English newspaper.
\item If she reads English newspaper, find the probability that she reads Hindi newspaper.\\
\end{enumerate}
\item The probability of obtaining an even prime number on each die, when a pair of dice is rolled is 
\begin{enumerate}
    \item $0$ 
    
    \item $\frac{1}{3}$ 
    
    \item $\frac{1}{12}$ 
    
    \item $\frac{1}{36}$ 
\end{enumerate}
\solution
		%\input{ncert/12/13/2/17/defs.tex}
	\item A bag contains 4 red and 4 black balls, another bag contains 2 red and 6 black balls. One of the two bags is selected at random and a ball is drawn from the bag which is found to be red. Find the probability that the ball is drawn from the first bag.
\\
\solution
		%\input{ncert/12/13/3/2/main.tex}
  \item
  Cards with numbers 2 to 101 are placed in a box. A card is selected at random.Find the probability that the card has
\begin{enumerate}[label=(\roman*)]
	\item an even number 
	\item a square number
\end{enumerate}
\solution
%\input{exemplar/10/13/3/32/main.tex}
\item
The king, queen and jack of clubs are removed from a deck of 52 playing cards and then well shuffled. Now one card is drawn at random from the remaining cards.  Determine the probability that the card is
\begin{enumerate}[label=(\roman*)]
\item a club
\item 10 of hearts
\end{enumerate}
\solution
%\input{exemplar/10/13/3/29/main.tex}
\item A team of medical students doing their internship have to assist during surgeries
at a city hospital. The probabilities of surgeries rated as very complex, complex,
routine, simple or very simple are respectively, 0.15, 0.20, 0.31, 0.26, .08. Find
the probabilities that a particular surgery will be rated
\begin{enumerate}
	\item complex or very complex;
	\item neither very complex nor very simple;
	\item routine or complex
	\item routine or simple
\end{enumerate}
\solution
%\input{exemplar/11/16/3/8(1)/main.tex}
\item A card is selected from a pack of 52 cards.
\begin{enumerate}[label=(\alph*)]
    \item How many points are there in the sample space?
    \item Calculate the probability that the card is an ace of spades.
    \item Calculate the probability that the card is (i) an ace and (ii) black card.
\end{enumerate}
\solution
%\input{exemplar/11/16/3/4/main2.tex}
\item The probability that a non leap year selected at random will contain 53 sundays.
\\
\solution
%\input{exemplar/10/13/1/19/main.tex}
\item One of the four persons John, Rita, Aslam or Gurpreet will be promoted next
month. Consequently the sample space consists of four elementary outcomes
S = {John promoted, Rita promoted, Aslam promoted, Gurpreet promoted}
You are told that the chances of John’s promotion is same as that of Gurpreet,
Rita’s chances of promotion are twice as likely as Johns. Aslam’s chances are
four times that of John.
\begin{enumerate}
	\item Determine
	\begin{enumerate}
		\item P (John promoted)
		\item P (Rita promoted)
		\item P (Aslam promoted)
		\item P (Gurpreet promoted)
	\end{enumerate}
	\item If A = {John promoted or Gurpreet promoted}, find P (A).
\end{enumerate}
\solution
%\input{exemplar/11/16/3/10/main.tex}
\item A card is drawn from a deck of 52 cards. Find the probability of getting a king or a heart or a red card.\\
\solution
%\input{exemplar/11/16/3/15/main.tex}
\item The probability that a student will pass his examination is 0.73, the probability of
the student getting a compartment is 0.13, and the probability that the student will
either pass or get compartment is 0.96. State True or False.\\
\solution
%\input{exemplar/11/16/3/31/main.tex}
\item A card is selected from a pack of 52 cards\\
\begin{enumerate}[label=(\alph*)]
\item How many points are there in the sample space?
\item Calculate the probability that the cards is an ace of spades.
\item Calculate the probability that the card is (i) an ace (ii)black card.\\
\end{enumerate}
%\input{ncert/11/16/3/4_1/Prob_4.tex}
\item In a non-leap year, the probability of having 53 tuesdays or 53 wednesdays is\\
\solution
%\input{exemplar/11/16/3/18/main.tex}
\item There are 1000 sealed envelopes in a box, 10 of them contain a cash prize of
Rs 100 each, 100 of them contain a cash prize of Rs 50 each and 200 of them
contain a cash prize of Rs 10 each and rest do not contain any cash prize. If they
are well shuffled and an envelope is picked up out, what is the probability that it
contains no cash prize?\\
\solution
%\input{exemplar/10/13/3/34/main.tex}
\item 
A die is thrown and a card is selected at random from a deck of 52 playing cards. The probability of getting an even number on the die and a spade card.\\
\solution
%\input{exemplar/12/13/3/78/main.tex}
\item
If 4-digit numbers greater than 5,000 are randomly formed from the digits 0, 1, 3, 5, and 7, what is the probability of forming a number divisible by 5 when:
\begin{enumerate}
    \item The digits are repeated?
    \item The repetition of digits is not allowed?
\end{enumerate}
\solution
%\input{ncert/11/16/4/9/main.tex}
\item Consider the probability space $\brak{\Omega, \mathcal{G}, P}$ where $\Omega = [0,2]$ and $\mathcal{G} = \cbrak{\phi, \Omega, [0,1], (1,2]}$. Let $X$ and $Y$ be two functions on $\Omega$ defined as
\begin{align*}
    X(\omega) = 
    \begin{cases}
        1 & \text{if }\omega \in [0, 1]\\
        2 & \text{if }\omega \in (1, 2]
    \end{cases}
\end{align*}
and
\begin{align*}
    Y(\omega) = 
    \begin{cases}
        2 & \text{if }\omega \in [0, 1.5]\\
        3 & \text{if }\omega \in (1.5, 2].
    \end{cases}
\end{align*}
Then which one of the following statements is true?
\begin{enumerate}
    \item [(A)] $X$ is a random variable with respect to $\mathcal{G}$, but $Y$ is not a random variable with respect to $\mathcal{G}$.
    \item [(B)] $Y$ is a random variable with respect to $\mathcal{G}$, but $X$ is not a random variable with respect to $\mathcal{G}$.
    \item [(C)] Neither $X$ nor $Y$ is a random variable with respect to $\mathcal{G}$.
    \item [(D)] Both $X$ and $Y$ are random variables with respect to $\mathcal{G}$.
\end{enumerate} \hfill (GATE ST 2023)\\
\solution
%\input{gate/ST/2023/14/main.tex}
	\item  A die is loaded in such a way that each odd number is twice as likely to occur as
each even number. Find $P(G)$, where $G$ is the event that a number greater than
3 occurs on a single roll of the die.
\\
\solution
		%\input{exemplar/11/16/3/5/main.tex}
	\item All the jacks, queens and kings are removed from a deck of 52 playing cards. The remaining cards are well shuffled and then one card is drawn at random. Giving ace a value 1 similar value for other cards, find the probability that the card has a value 
		\begin{enumerate}
			\item 7
			\item greater than 7
			\item less than 7
		\end{enumerate}
		%\input{exemplar/10/13/3/30/main.tex}
  \item A Lot consists of 48 mobile phones of which 42 are good, 3 have only minor defects and 3 have major defects.Varnika will buy a phone if it is good but the trader will only buy a mobile if it has no major defects. One phone is selected at random from the lot. What is the probability that it is
\begin{enumerate}
	\item acceptable to Varnika?
            \item acceptable to the trader?
\end{enumerate}
\solution
	%\input{exemplar/10/13/3/40/main.tex}
 \item A student says that if you throw a die, it will show up 1 or not 1. Therefore, the probability of getting 1 and the probability of getting 'not 1' each is equal to $\frac{1}{2}$. Is this correct? Give reasons.\\
 \solution
        %\input{exemplar/10/13/2/9/main.tex}
   \item Four candidates A, B, C, D have ap-
plied for the assignment to coach a school cricket
team. If A is twice as likely to be selected as B, and
B and C are given about the same chance of being
selected, while C is twice as likely to be selected
as D, what are the probabilities that
\begin{enumerate}
\item C will be selected?
\item A will not be selected?
\end{enumerate}
	%\input{exemplar/11/16/3/9/main.tex}
 \item A bag contain 24 balls of which $x$ balls are red, $2x$ are white and $3x$ are blue. A ball is selected at random, What is the probability that it is
\begin{enumerate}[label=\alph*)]
\item not red ?
\item white ?
\end{enumerate}
%\input{exemplar/10/13/3/41/main.tex}
If the letters of the word ASSASSINATION are arranged at random. Find the Probability that
\begin{enumerate}[label=(\alph*)]
\item Four $S's$ come consecutively in the word
\item Two  $I's$ and two $N's$ come together
\item All $A's$ are not coming together
\item No two $A's$ are coming together
\end{enumerate}
%\input{exemplar/11/16/3/14/main.tex}
	\item One urn contains two black balls (labelled B1 and B2) and one white ball. A
	second urn contains one black ball and two white balls (labelled W1 and W2).
	Suppose the following experiment is performed. One of the two urns is chosen
	at random. Next a ball is randomly chosen from the urn. Then a second ball is
	chosen at random from the same urn without replacing the first ball.
	
	\begin{enumerate}
	\item What is the probability that two black balls are chosen?
	
	\item What is the probability that two balls of opposite colour are chosen?
	\end{enumerate}
	\solution
	%\input{exemplar/11/16/3/12/main1.tex}
\end{enumerate}

	\item 
The number lock of a suitcase has 4 wheels each labelled with ten digits i.e. from 0 to 9.The lock opens with a sequence of four digits with no repeats.What is the probability of a person getting the right sequence to open the suitcase.
\\
\solution
		%\begin{enumerate}[label=\thesection.\arabic*,ref=\thesection.\theenumi]
	\item One card is drawn from a well-shuffled deck of 52 cards. Find the probability of getting
\begin{enumerate}
\item A king of red colour 
\item A face card 
\item A red face card
\item The jack of hearts
\item A spade
\item The queen of diamonds

\end{enumerate}
\solution
		%\input{ncert/10/15/1/14/main.tex}
	\item Five cards—the ten, jack, queen, king and ace of diamonds, are well-shuffled with their face downwards. One card is then picked up at random.
\begin{enumerate}
\item
What is the probability that the card is the queen? 
\item
If the queen is drawn and put aside, what is the probability that the second card picked up is (a) an ace? (b) a queen?\\
\end{enumerate}
\solution
		%\input{ncert/10/15/1/15/defs.tex}
	\item A bag contains $5$ red balls and some blue balls. If the probability of drawing a blue ball is double that if a red ball, determine the number of blue balls in the bag. 
		\\
\solution
		%\input{ncert/10/15/2/3/defs.tex}
	\item A card is selected from a pack of 52 cards.
 \begin{enumerate}[label=(\alph*)] 
                 \item How many points are there in the sample space?
                 \item Calculate the probability that the card is an ace of spades.
                 \item Calculate the probability that the card is (i) an ace and (ii) black card.
 \end{enumerate}
\solution
		%\input{ncert/11/16/3/4/main.tex}
\item Four cards are drawn from a well-shuffled deck of 52 cards. What is the probability of obtaining 3 diamonds and one spade.
\\
\solution
		%\input{ncert/11/16/4/2/defs.tex}
\item In a certain lottery 10,000 tickets are sold and ten equal prizes are awarded. What is the probability of not getting a prize if you buy (a) one ticket (b) two tickets (c) 10 tickets ?	
\\
\solution
		%\input{ncert/11/16/4/4/defs.tex}
		%
\item 
Out of 100 students, two sections of 40 and 60 are formed. If you and your friend are among the 100 students, what is the probability that
\begin{enumerate}
\item you both enter the same section?
\item you both enter the different sections?
\end{enumerate}
\solution
		%\input{ncert/11/16/4/5/defs.tex}
	\item 
The number lock of a suitcase has 4 wheels each labelled with ten digits i.e. from 0 to 9.The lock opens with a sequence of four digits with no repeats.What is the probability of a person getting the right sequence to open the suitcase.
\\
\solution
		%\input{ncert/11/16/4/10/defs.tex}
		%
\item 
Two cards are drawn at random and without replacement from a pack of 52 playing cards. Find the probability that both the cards are black.
\\
\solution
		%\input{ncert/12/13/2/2/defs.tex}
		\item A box of oranges is inspected by examining three randomly selected oranges drawn without replacement. If all the three oranges are good, the box is approved for sale, otherwise, it is rejected. Find the probability that a box containing 15 oranges out of which 12 are good and 3 are bad ones will be approved for sale.
		\label{ncert/12/13/2/3/defs.tex}
		\item Two balls are drawn at random with replacement from a box containing 10 black and 8 red balls. Find the probability that
		\label{ncert/12/13/2/12}
\begin{enumerate}
\item both balls are red.
\item first ball is black and second is red.
\item one of them is black and other is red.
\end{enumerate}

\item In a hostel, 60\% of the students read Hindi newspaper, 40\% read English newspaper and 20\% read both Hindi and English newspapers. A student is selected at random.
		\label{ncert/12/13/2/15}
\begin{enumerate}
\item Find the probability that she reads neither Hindi nor English newspapers.
\item If she reads Hindi newspaper, find the probability that she reads English newspaper.
\item If she reads English newspaper, find the probability that she reads Hindi newspaper.\\
\end{enumerate}
\item The probability of obtaining an even prime number on each die, when a pair of dice is rolled is 
\begin{enumerate}
    \item $0$ 
    
    \item $\frac{1}{3}$ 
    
    \item $\frac{1}{12}$ 
    
    \item $\frac{1}{36}$ 
\end{enumerate}
\solution
		%\input{ncert/12/13/2/17/defs.tex}
	\item A bag contains 4 red and 4 black balls, another bag contains 2 red and 6 black balls. One of the two bags is selected at random and a ball is drawn from the bag which is found to be red. Find the probability that the ball is drawn from the first bag.
\\
\solution
		%\input{ncert/12/13/3/2/main.tex}
  \item
  Cards with numbers 2 to 101 are placed in a box. A card is selected at random.Find the probability that the card has
\begin{enumerate}[label=(\roman*)]
	\item an even number 
	\item a square number
\end{enumerate}
\solution
%\input{exemplar/10/13/3/32/main.tex}
\item
The king, queen and jack of clubs are removed from a deck of 52 playing cards and then well shuffled. Now one card is drawn at random from the remaining cards.  Determine the probability that the card is
\begin{enumerate}[label=(\roman*)]
\item a club
\item 10 of hearts
\end{enumerate}
\solution
%\input{exemplar/10/13/3/29/main.tex}
\item A team of medical students doing their internship have to assist during surgeries
at a city hospital. The probabilities of surgeries rated as very complex, complex,
routine, simple or very simple are respectively, 0.15, 0.20, 0.31, 0.26, .08. Find
the probabilities that a particular surgery will be rated
\begin{enumerate}
	\item complex or very complex;
	\item neither very complex nor very simple;
	\item routine or complex
	\item routine or simple
\end{enumerate}
\solution
%\input{exemplar/11/16/3/8(1)/main.tex}
\item A card is selected from a pack of 52 cards.
\begin{enumerate}[label=(\alph*)]
    \item How many points are there in the sample space?
    \item Calculate the probability that the card is an ace of spades.
    \item Calculate the probability that the card is (i) an ace and (ii) black card.
\end{enumerate}
\solution
%\input{exemplar/11/16/3/4/main2.tex}
\item The probability that a non leap year selected at random will contain 53 sundays.
\\
\solution
%\input{exemplar/10/13/1/19/main.tex}
\item One of the four persons John, Rita, Aslam or Gurpreet will be promoted next
month. Consequently the sample space consists of four elementary outcomes
S = {John promoted, Rita promoted, Aslam promoted, Gurpreet promoted}
You are told that the chances of John’s promotion is same as that of Gurpreet,
Rita’s chances of promotion are twice as likely as Johns. Aslam’s chances are
four times that of John.
\begin{enumerate}
	\item Determine
	\begin{enumerate}
		\item P (John promoted)
		\item P (Rita promoted)
		\item P (Aslam promoted)
		\item P (Gurpreet promoted)
	\end{enumerate}
	\item If A = {John promoted or Gurpreet promoted}, find P (A).
\end{enumerate}
\solution
%\input{exemplar/11/16/3/10/main.tex}
\item A card is drawn from a deck of 52 cards. Find the probability of getting a king or a heart or a red card.\\
\solution
%\input{exemplar/11/16/3/15/main.tex}
\item The probability that a student will pass his examination is 0.73, the probability of
the student getting a compartment is 0.13, and the probability that the student will
either pass or get compartment is 0.96. State True or False.\\
\solution
%\input{exemplar/11/16/3/31/main.tex}
\item A card is selected from a pack of 52 cards\\
\begin{enumerate}[label=(\alph*)]
\item How many points are there in the sample space?
\item Calculate the probability that the cards is an ace of spades.
\item Calculate the probability that the card is (i) an ace (ii)black card.\\
\end{enumerate}
%\input{ncert/11/16/3/4_1/Prob_4.tex}
\item In a non-leap year, the probability of having 53 tuesdays or 53 wednesdays is\\
\solution
%\input{exemplar/11/16/3/18/main.tex}
\item There are 1000 sealed envelopes in a box, 10 of them contain a cash prize of
Rs 100 each, 100 of them contain a cash prize of Rs 50 each and 200 of them
contain a cash prize of Rs 10 each and rest do not contain any cash prize. If they
are well shuffled and an envelope is picked up out, what is the probability that it
contains no cash prize?\\
\solution
%\input{exemplar/10/13/3/34/main.tex}
\item 
A die is thrown and a card is selected at random from a deck of 52 playing cards. The probability of getting an even number on the die and a spade card.\\
\solution
%\input{exemplar/12/13/3/78/main.tex}
\item
If 4-digit numbers greater than 5,000 are randomly formed from the digits 0, 1, 3, 5, and 7, what is the probability of forming a number divisible by 5 when:
\begin{enumerate}
    \item The digits are repeated?
    \item The repetition of digits is not allowed?
\end{enumerate}
\solution
%\input{ncert/11/16/4/9/main.tex}
\item Consider the probability space $\brak{\Omega, \mathcal{G}, P}$ where $\Omega = [0,2]$ and $\mathcal{G} = \cbrak{\phi, \Omega, [0,1], (1,2]}$. Let $X$ and $Y$ be two functions on $\Omega$ defined as
\begin{align*}
    X(\omega) = 
    \begin{cases}
        1 & \text{if }\omega \in [0, 1]\\
        2 & \text{if }\omega \in (1, 2]
    \end{cases}
\end{align*}
and
\begin{align*}
    Y(\omega) = 
    \begin{cases}
        2 & \text{if }\omega \in [0, 1.5]\\
        3 & \text{if }\omega \in (1.5, 2].
    \end{cases}
\end{align*}
Then which one of the following statements is true?
\begin{enumerate}
    \item [(A)] $X$ is a random variable with respect to $\mathcal{G}$, but $Y$ is not a random variable with respect to $\mathcal{G}$.
    \item [(B)] $Y$ is a random variable with respect to $\mathcal{G}$, but $X$ is not a random variable with respect to $\mathcal{G}$.
    \item [(C)] Neither $X$ nor $Y$ is a random variable with respect to $\mathcal{G}$.
    \item [(D)] Both $X$ and $Y$ are random variables with respect to $\mathcal{G}$.
\end{enumerate} \hfill (GATE ST 2023)\\
\solution
%\input{gate/ST/2023/14/main.tex}
	\item  A die is loaded in such a way that each odd number is twice as likely to occur as
each even number. Find $P(G)$, where $G$ is the event that a number greater than
3 occurs on a single roll of the die.
\\
\solution
		%\input{exemplar/11/16/3/5/main.tex}
	\item All the jacks, queens and kings are removed from a deck of 52 playing cards. The remaining cards are well shuffled and then one card is drawn at random. Giving ace a value 1 similar value for other cards, find the probability that the card has a value 
		\begin{enumerate}
			\item 7
			\item greater than 7
			\item less than 7
		\end{enumerate}
		%\input{exemplar/10/13/3/30/main.tex}
  \item A Lot consists of 48 mobile phones of which 42 are good, 3 have only minor defects and 3 have major defects.Varnika will buy a phone if it is good but the trader will only buy a mobile if it has no major defects. One phone is selected at random from the lot. What is the probability that it is
\begin{enumerate}
	\item acceptable to Varnika?
            \item acceptable to the trader?
\end{enumerate}
\solution
	%\input{exemplar/10/13/3/40/main.tex}
 \item A student says that if you throw a die, it will show up 1 or not 1. Therefore, the probability of getting 1 and the probability of getting 'not 1' each is equal to $\frac{1}{2}$. Is this correct? Give reasons.\\
 \solution
        %\input{exemplar/10/13/2/9/main.tex}
   \item Four candidates A, B, C, D have ap-
plied for the assignment to coach a school cricket
team. If A is twice as likely to be selected as B, and
B and C are given about the same chance of being
selected, while C is twice as likely to be selected
as D, what are the probabilities that
\begin{enumerate}
\item C will be selected?
\item A will not be selected?
\end{enumerate}
	%\input{exemplar/11/16/3/9/main.tex}
 \item A bag contain 24 balls of which $x$ balls are red, $2x$ are white and $3x$ are blue. A ball is selected at random, What is the probability that it is
\begin{enumerate}[label=\alph*)]
\item not red ?
\item white ?
\end{enumerate}
%\input{exemplar/10/13/3/41/main.tex}
If the letters of the word ASSASSINATION are arranged at random. Find the Probability that
\begin{enumerate}[label=(\alph*)]
\item Four $S's$ come consecutively in the word
\item Two  $I's$ and two $N's$ come together
\item All $A's$ are not coming together
\item No two $A's$ are coming together
\end{enumerate}
%\input{exemplar/11/16/3/14/main.tex}
	\item One urn contains two black balls (labelled B1 and B2) and one white ball. A
	second urn contains one black ball and two white balls (labelled W1 and W2).
	Suppose the following experiment is performed. One of the two urns is chosen
	at random. Next a ball is randomly chosen from the urn. Then a second ball is
	chosen at random from the same urn without replacing the first ball.
	
	\begin{enumerate}
	\item What is the probability that two black balls are chosen?
	
	\item What is the probability that two balls of opposite colour are chosen?
	\end{enumerate}
	\solution
	%\input{exemplar/11/16/3/12/main1.tex}
\end{enumerate}

		%
\item 
Two cards are drawn at random and without replacement from a pack of 52 playing cards. Find the probability that both the cards are black.
\\
\solution
		%\begin{enumerate}[label=\thesection.\arabic*,ref=\thesection.\theenumi]
	\item One card is drawn from a well-shuffled deck of 52 cards. Find the probability of getting
\begin{enumerate}
\item A king of red colour 
\item A face card 
\item A red face card
\item The jack of hearts
\item A spade
\item The queen of diamonds

\end{enumerate}
\solution
		%\input{ncert/10/15/1/14/main.tex}
	\item Five cards—the ten, jack, queen, king and ace of diamonds, are well-shuffled with their face downwards. One card is then picked up at random.
\begin{enumerate}
\item
What is the probability that the card is the queen? 
\item
If the queen is drawn and put aside, what is the probability that the second card picked up is (a) an ace? (b) a queen?\\
\end{enumerate}
\solution
		%\input{ncert/10/15/1/15/defs.tex}
	\item A bag contains $5$ red balls and some blue balls. If the probability of drawing a blue ball is double that if a red ball, determine the number of blue balls in the bag. 
		\\
\solution
		%\input{ncert/10/15/2/3/defs.tex}
	\item A card is selected from a pack of 52 cards.
 \begin{enumerate}[label=(\alph*)] 
                 \item How many points are there in the sample space?
                 \item Calculate the probability that the card is an ace of spades.
                 \item Calculate the probability that the card is (i) an ace and (ii) black card.
 \end{enumerate}
\solution
		%\input{ncert/11/16/3/4/main.tex}
\item Four cards are drawn from a well-shuffled deck of 52 cards. What is the probability of obtaining 3 diamonds and one spade.
\\
\solution
		%\input{ncert/11/16/4/2/defs.tex}
\item In a certain lottery 10,000 tickets are sold and ten equal prizes are awarded. What is the probability of not getting a prize if you buy (a) one ticket (b) two tickets (c) 10 tickets ?	
\\
\solution
		%\input{ncert/11/16/4/4/defs.tex}
		%
\item 
Out of 100 students, two sections of 40 and 60 are formed. If you and your friend are among the 100 students, what is the probability that
\begin{enumerate}
\item you both enter the same section?
\item you both enter the different sections?
\end{enumerate}
\solution
		%\input{ncert/11/16/4/5/defs.tex}
	\item 
The number lock of a suitcase has 4 wheels each labelled with ten digits i.e. from 0 to 9.The lock opens with a sequence of four digits with no repeats.What is the probability of a person getting the right sequence to open the suitcase.
\\
\solution
		%\input{ncert/11/16/4/10/defs.tex}
		%
\item 
Two cards are drawn at random and without replacement from a pack of 52 playing cards. Find the probability that both the cards are black.
\\
\solution
		%\input{ncert/12/13/2/2/defs.tex}
		\item A box of oranges is inspected by examining three randomly selected oranges drawn without replacement. If all the three oranges are good, the box is approved for sale, otherwise, it is rejected. Find the probability that a box containing 15 oranges out of which 12 are good and 3 are bad ones will be approved for sale.
		\label{ncert/12/13/2/3/defs.tex}
		\item Two balls are drawn at random with replacement from a box containing 10 black and 8 red balls. Find the probability that
		\label{ncert/12/13/2/12}
\begin{enumerate}
\item both balls are red.
\item first ball is black and second is red.
\item one of them is black and other is red.
\end{enumerate}

\item In a hostel, 60\% of the students read Hindi newspaper, 40\% read English newspaper and 20\% read both Hindi and English newspapers. A student is selected at random.
		\label{ncert/12/13/2/15}
\begin{enumerate}
\item Find the probability that she reads neither Hindi nor English newspapers.
\item If she reads Hindi newspaper, find the probability that she reads English newspaper.
\item If she reads English newspaper, find the probability that she reads Hindi newspaper.\\
\end{enumerate}
\item The probability of obtaining an even prime number on each die, when a pair of dice is rolled is 
\begin{enumerate}
    \item $0$ 
    
    \item $\frac{1}{3}$ 
    
    \item $\frac{1}{12}$ 
    
    \item $\frac{1}{36}$ 
\end{enumerate}
\solution
		%\input{ncert/12/13/2/17/defs.tex}
	\item A bag contains 4 red and 4 black balls, another bag contains 2 red and 6 black balls. One of the two bags is selected at random and a ball is drawn from the bag which is found to be red. Find the probability that the ball is drawn from the first bag.
\\
\solution
		%\input{ncert/12/13/3/2/main.tex}
  \item
  Cards with numbers 2 to 101 are placed in a box. A card is selected at random.Find the probability that the card has
\begin{enumerate}[label=(\roman*)]
	\item an even number 
	\item a square number
\end{enumerate}
\solution
%\input{exemplar/10/13/3/32/main.tex}
\item
The king, queen and jack of clubs are removed from a deck of 52 playing cards and then well shuffled. Now one card is drawn at random from the remaining cards.  Determine the probability that the card is
\begin{enumerate}[label=(\roman*)]
\item a club
\item 10 of hearts
\end{enumerate}
\solution
%\input{exemplar/10/13/3/29/main.tex}
\item A team of medical students doing their internship have to assist during surgeries
at a city hospital. The probabilities of surgeries rated as very complex, complex,
routine, simple or very simple are respectively, 0.15, 0.20, 0.31, 0.26, .08. Find
the probabilities that a particular surgery will be rated
\begin{enumerate}
	\item complex or very complex;
	\item neither very complex nor very simple;
	\item routine or complex
	\item routine or simple
\end{enumerate}
\solution
%\input{exemplar/11/16/3/8(1)/main.tex}
\item A card is selected from a pack of 52 cards.
\begin{enumerate}[label=(\alph*)]
    \item How many points are there in the sample space?
    \item Calculate the probability that the card is an ace of spades.
    \item Calculate the probability that the card is (i) an ace and (ii) black card.
\end{enumerate}
\solution
%\input{exemplar/11/16/3/4/main2.tex}
\item The probability that a non leap year selected at random will contain 53 sundays.
\\
\solution
%\input{exemplar/10/13/1/19/main.tex}
\item One of the four persons John, Rita, Aslam or Gurpreet will be promoted next
month. Consequently the sample space consists of four elementary outcomes
S = {John promoted, Rita promoted, Aslam promoted, Gurpreet promoted}
You are told that the chances of John’s promotion is same as that of Gurpreet,
Rita’s chances of promotion are twice as likely as Johns. Aslam’s chances are
four times that of John.
\begin{enumerate}
	\item Determine
	\begin{enumerate}
		\item P (John promoted)
		\item P (Rita promoted)
		\item P (Aslam promoted)
		\item P (Gurpreet promoted)
	\end{enumerate}
	\item If A = {John promoted or Gurpreet promoted}, find P (A).
\end{enumerate}
\solution
%\input{exemplar/11/16/3/10/main.tex}
\item A card is drawn from a deck of 52 cards. Find the probability of getting a king or a heart or a red card.\\
\solution
%\input{exemplar/11/16/3/15/main.tex}
\item The probability that a student will pass his examination is 0.73, the probability of
the student getting a compartment is 0.13, and the probability that the student will
either pass or get compartment is 0.96. State True or False.\\
\solution
%\input{exemplar/11/16/3/31/main.tex}
\item A card is selected from a pack of 52 cards\\
\begin{enumerate}[label=(\alph*)]
\item How many points are there in the sample space?
\item Calculate the probability that the cards is an ace of spades.
\item Calculate the probability that the card is (i) an ace (ii)black card.\\
\end{enumerate}
%\input{ncert/11/16/3/4_1/Prob_4.tex}
\item In a non-leap year, the probability of having 53 tuesdays or 53 wednesdays is\\
\solution
%\input{exemplar/11/16/3/18/main.tex}
\item There are 1000 sealed envelopes in a box, 10 of them contain a cash prize of
Rs 100 each, 100 of them contain a cash prize of Rs 50 each and 200 of them
contain a cash prize of Rs 10 each and rest do not contain any cash prize. If they
are well shuffled and an envelope is picked up out, what is the probability that it
contains no cash prize?\\
\solution
%\input{exemplar/10/13/3/34/main.tex}
\item 
A die is thrown and a card is selected at random from a deck of 52 playing cards. The probability of getting an even number on the die and a spade card.\\
\solution
%\input{exemplar/12/13/3/78/main.tex}
\item
If 4-digit numbers greater than 5,000 are randomly formed from the digits 0, 1, 3, 5, and 7, what is the probability of forming a number divisible by 5 when:
\begin{enumerate}
    \item The digits are repeated?
    \item The repetition of digits is not allowed?
\end{enumerate}
\solution
%\input{ncert/11/16/4/9/main.tex}
\item Consider the probability space $\brak{\Omega, \mathcal{G}, P}$ where $\Omega = [0,2]$ and $\mathcal{G} = \cbrak{\phi, \Omega, [0,1], (1,2]}$. Let $X$ and $Y$ be two functions on $\Omega$ defined as
\begin{align*}
    X(\omega) = 
    \begin{cases}
        1 & \text{if }\omega \in [0, 1]\\
        2 & \text{if }\omega \in (1, 2]
    \end{cases}
\end{align*}
and
\begin{align*}
    Y(\omega) = 
    \begin{cases}
        2 & \text{if }\omega \in [0, 1.5]\\
        3 & \text{if }\omega \in (1.5, 2].
    \end{cases}
\end{align*}
Then which one of the following statements is true?
\begin{enumerate}
    \item [(A)] $X$ is a random variable with respect to $\mathcal{G}$, but $Y$ is not a random variable with respect to $\mathcal{G}$.
    \item [(B)] $Y$ is a random variable with respect to $\mathcal{G}$, but $X$ is not a random variable with respect to $\mathcal{G}$.
    \item [(C)] Neither $X$ nor $Y$ is a random variable with respect to $\mathcal{G}$.
    \item [(D)] Both $X$ and $Y$ are random variables with respect to $\mathcal{G}$.
\end{enumerate} \hfill (GATE ST 2023)\\
\solution
%\input{gate/ST/2023/14/main.tex}
	\item  A die is loaded in such a way that each odd number is twice as likely to occur as
each even number. Find $P(G)$, where $G$ is the event that a number greater than
3 occurs on a single roll of the die.
\\
\solution
		%\input{exemplar/11/16/3/5/main.tex}
	\item All the jacks, queens and kings are removed from a deck of 52 playing cards. The remaining cards are well shuffled and then one card is drawn at random. Giving ace a value 1 similar value for other cards, find the probability that the card has a value 
		\begin{enumerate}
			\item 7
			\item greater than 7
			\item less than 7
		\end{enumerate}
		%\input{exemplar/10/13/3/30/main.tex}
  \item A Lot consists of 48 mobile phones of which 42 are good, 3 have only minor defects and 3 have major defects.Varnika will buy a phone if it is good but the trader will only buy a mobile if it has no major defects. One phone is selected at random from the lot. What is the probability that it is
\begin{enumerate}
	\item acceptable to Varnika?
            \item acceptable to the trader?
\end{enumerate}
\solution
	%\input{exemplar/10/13/3/40/main.tex}
 \item A student says that if you throw a die, it will show up 1 or not 1. Therefore, the probability of getting 1 and the probability of getting 'not 1' each is equal to $\frac{1}{2}$. Is this correct? Give reasons.\\
 \solution
        %\input{exemplar/10/13/2/9/main.tex}
   \item Four candidates A, B, C, D have ap-
plied for the assignment to coach a school cricket
team. If A is twice as likely to be selected as B, and
B and C are given about the same chance of being
selected, while C is twice as likely to be selected
as D, what are the probabilities that
\begin{enumerate}
\item C will be selected?
\item A will not be selected?
\end{enumerate}
	%\input{exemplar/11/16/3/9/main.tex}
 \item A bag contain 24 balls of which $x$ balls are red, $2x$ are white and $3x$ are blue. A ball is selected at random, What is the probability that it is
\begin{enumerate}[label=\alph*)]
\item not red ?
\item white ?
\end{enumerate}
%\input{exemplar/10/13/3/41/main.tex}
If the letters of the word ASSASSINATION are arranged at random. Find the Probability that
\begin{enumerate}[label=(\alph*)]
\item Four $S's$ come consecutively in the word
\item Two  $I's$ and two $N's$ come together
\item All $A's$ are not coming together
\item No two $A's$ are coming together
\end{enumerate}
%\input{exemplar/11/16/3/14/main.tex}
	\item One urn contains two black balls (labelled B1 and B2) and one white ball. A
	second urn contains one black ball and two white balls (labelled W1 and W2).
	Suppose the following experiment is performed. One of the two urns is chosen
	at random. Next a ball is randomly chosen from the urn. Then a second ball is
	chosen at random from the same urn without replacing the first ball.
	
	\begin{enumerate}
	\item What is the probability that two black balls are chosen?
	
	\item What is the probability that two balls of opposite colour are chosen?
	\end{enumerate}
	\solution
	%\input{exemplar/11/16/3/12/main1.tex}
\end{enumerate}

		\item A box of oranges is inspected by examining three randomly selected oranges drawn without replacement. If all the three oranges are good, the box is approved for sale, otherwise, it is rejected. Find the probability that a box containing 15 oranges out of which 12 are good and 3 are bad ones will be approved for sale.
		\label{ncert/12/13/2/3/defs.tex}
		\item Two balls are drawn at random with replacement from a box containing 10 black and 8 red balls. Find the probability that
		\label{ncert/12/13/2/12}
\begin{enumerate}
\item both balls are red.
\item first ball is black and second is red.
\item one of them is black and other is red.
\end{enumerate}

\item In a hostel, 60\% of the students read Hindi newspaper, 40\% read English newspaper and 20\% read both Hindi and English newspapers. A student is selected at random.
		\label{ncert/12/13/2/15}
\begin{enumerate}
\item Find the probability that she reads neither Hindi nor English newspapers.
\item If she reads Hindi newspaper, find the probability that she reads English newspaper.
\item If she reads English newspaper, find the probability that she reads Hindi newspaper.\\
\end{enumerate}
\item The probability of obtaining an even prime number on each die, when a pair of dice is rolled is 
\begin{enumerate}
    \item $0$ 
    
    \item $\frac{1}{3}$ 
    
    \item $\frac{1}{12}$ 
    
    \item $\frac{1}{36}$ 
\end{enumerate}
\solution
		%\begin{enumerate}[label=\thesection.\arabic*,ref=\thesection.\theenumi]
	\item One card is drawn from a well-shuffled deck of 52 cards. Find the probability of getting
\begin{enumerate}
\item A king of red colour 
\item A face card 
\item A red face card
\item The jack of hearts
\item A spade
\item The queen of diamonds

\end{enumerate}
\solution
		%\input{ncert/10/15/1/14/main.tex}
	\item Five cards—the ten, jack, queen, king and ace of diamonds, are well-shuffled with their face downwards. One card is then picked up at random.
\begin{enumerate}
\item
What is the probability that the card is the queen? 
\item
If the queen is drawn and put aside, what is the probability that the second card picked up is (a) an ace? (b) a queen?\\
\end{enumerate}
\solution
		%\input{ncert/10/15/1/15/defs.tex}
	\item A bag contains $5$ red balls and some blue balls. If the probability of drawing a blue ball is double that if a red ball, determine the number of blue balls in the bag. 
		\\
\solution
		%\input{ncert/10/15/2/3/defs.tex}
	\item A card is selected from a pack of 52 cards.
 \begin{enumerate}[label=(\alph*)] 
                 \item How many points are there in the sample space?
                 \item Calculate the probability that the card is an ace of spades.
                 \item Calculate the probability that the card is (i) an ace and (ii) black card.
 \end{enumerate}
\solution
		%\input{ncert/11/16/3/4/main.tex}
\item Four cards are drawn from a well-shuffled deck of 52 cards. What is the probability of obtaining 3 diamonds and one spade.
\\
\solution
		%\input{ncert/11/16/4/2/defs.tex}
\item In a certain lottery 10,000 tickets are sold and ten equal prizes are awarded. What is the probability of not getting a prize if you buy (a) one ticket (b) two tickets (c) 10 tickets ?	
\\
\solution
		%\input{ncert/11/16/4/4/defs.tex}
		%
\item 
Out of 100 students, two sections of 40 and 60 are formed. If you and your friend are among the 100 students, what is the probability that
\begin{enumerate}
\item you both enter the same section?
\item you both enter the different sections?
\end{enumerate}
\solution
		%\input{ncert/11/16/4/5/defs.tex}
	\item 
The number lock of a suitcase has 4 wheels each labelled with ten digits i.e. from 0 to 9.The lock opens with a sequence of four digits with no repeats.What is the probability of a person getting the right sequence to open the suitcase.
\\
\solution
		%\input{ncert/11/16/4/10/defs.tex}
		%
\item 
Two cards are drawn at random and without replacement from a pack of 52 playing cards. Find the probability that both the cards are black.
\\
\solution
		%\input{ncert/12/13/2/2/defs.tex}
		\item A box of oranges is inspected by examining three randomly selected oranges drawn without replacement. If all the three oranges are good, the box is approved for sale, otherwise, it is rejected. Find the probability that a box containing 15 oranges out of which 12 are good and 3 are bad ones will be approved for sale.
		\label{ncert/12/13/2/3/defs.tex}
		\item Two balls are drawn at random with replacement from a box containing 10 black and 8 red balls. Find the probability that
		\label{ncert/12/13/2/12}
\begin{enumerate}
\item both balls are red.
\item first ball is black and second is red.
\item one of them is black and other is red.
\end{enumerate}

\item In a hostel, 60\% of the students read Hindi newspaper, 40\% read English newspaper and 20\% read both Hindi and English newspapers. A student is selected at random.
		\label{ncert/12/13/2/15}
\begin{enumerate}
\item Find the probability that she reads neither Hindi nor English newspapers.
\item If she reads Hindi newspaper, find the probability that she reads English newspaper.
\item If she reads English newspaper, find the probability that she reads Hindi newspaper.\\
\end{enumerate}
\item The probability of obtaining an even prime number on each die, when a pair of dice is rolled is 
\begin{enumerate}
    \item $0$ 
    
    \item $\frac{1}{3}$ 
    
    \item $\frac{1}{12}$ 
    
    \item $\frac{1}{36}$ 
\end{enumerate}
\solution
		%\input{ncert/12/13/2/17/defs.tex}
	\item A bag contains 4 red and 4 black balls, another bag contains 2 red and 6 black balls. One of the two bags is selected at random and a ball is drawn from the bag which is found to be red. Find the probability that the ball is drawn from the first bag.
\\
\solution
		%\input{ncert/12/13/3/2/main.tex}
  \item
  Cards with numbers 2 to 101 are placed in a box. A card is selected at random.Find the probability that the card has
\begin{enumerate}[label=(\roman*)]
	\item an even number 
	\item a square number
\end{enumerate}
\solution
%\input{exemplar/10/13/3/32/main.tex}
\item
The king, queen and jack of clubs are removed from a deck of 52 playing cards and then well shuffled. Now one card is drawn at random from the remaining cards.  Determine the probability that the card is
\begin{enumerate}[label=(\roman*)]
\item a club
\item 10 of hearts
\end{enumerate}
\solution
%\input{exemplar/10/13/3/29/main.tex}
\item A team of medical students doing their internship have to assist during surgeries
at a city hospital. The probabilities of surgeries rated as very complex, complex,
routine, simple or very simple are respectively, 0.15, 0.20, 0.31, 0.26, .08. Find
the probabilities that a particular surgery will be rated
\begin{enumerate}
	\item complex or very complex;
	\item neither very complex nor very simple;
	\item routine or complex
	\item routine or simple
\end{enumerate}
\solution
%\input{exemplar/11/16/3/8(1)/main.tex}
\item A card is selected from a pack of 52 cards.
\begin{enumerate}[label=(\alph*)]
    \item How many points are there in the sample space?
    \item Calculate the probability that the card is an ace of spades.
    \item Calculate the probability that the card is (i) an ace and (ii) black card.
\end{enumerate}
\solution
%\input{exemplar/11/16/3/4/main2.tex}
\item The probability that a non leap year selected at random will contain 53 sundays.
\\
\solution
%\input{exemplar/10/13/1/19/main.tex}
\item One of the four persons John, Rita, Aslam or Gurpreet will be promoted next
month. Consequently the sample space consists of four elementary outcomes
S = {John promoted, Rita promoted, Aslam promoted, Gurpreet promoted}
You are told that the chances of John’s promotion is same as that of Gurpreet,
Rita’s chances of promotion are twice as likely as Johns. Aslam’s chances are
four times that of John.
\begin{enumerate}
	\item Determine
	\begin{enumerate}
		\item P (John promoted)
		\item P (Rita promoted)
		\item P (Aslam promoted)
		\item P (Gurpreet promoted)
	\end{enumerate}
	\item If A = {John promoted or Gurpreet promoted}, find P (A).
\end{enumerate}
\solution
%\input{exemplar/11/16/3/10/main.tex}
\item A card is drawn from a deck of 52 cards. Find the probability of getting a king or a heart or a red card.\\
\solution
%\input{exemplar/11/16/3/15/main.tex}
\item The probability that a student will pass his examination is 0.73, the probability of
the student getting a compartment is 0.13, and the probability that the student will
either pass or get compartment is 0.96. State True or False.\\
\solution
%\input{exemplar/11/16/3/31/main.tex}
\item A card is selected from a pack of 52 cards\\
\begin{enumerate}[label=(\alph*)]
\item How many points are there in the sample space?
\item Calculate the probability that the cards is an ace of spades.
\item Calculate the probability that the card is (i) an ace (ii)black card.\\
\end{enumerate}
%\input{ncert/11/16/3/4_1/Prob_4.tex}
\item In a non-leap year, the probability of having 53 tuesdays or 53 wednesdays is\\
\solution
%\input{exemplar/11/16/3/18/main.tex}
\item There are 1000 sealed envelopes in a box, 10 of them contain a cash prize of
Rs 100 each, 100 of them contain a cash prize of Rs 50 each and 200 of them
contain a cash prize of Rs 10 each and rest do not contain any cash prize. If they
are well shuffled and an envelope is picked up out, what is the probability that it
contains no cash prize?\\
\solution
%\input{exemplar/10/13/3/34/main.tex}
\item 
A die is thrown and a card is selected at random from a deck of 52 playing cards. The probability of getting an even number on the die and a spade card.\\
\solution
%\input{exemplar/12/13/3/78/main.tex}
\item
If 4-digit numbers greater than 5,000 are randomly formed from the digits 0, 1, 3, 5, and 7, what is the probability of forming a number divisible by 5 when:
\begin{enumerate}
    \item The digits are repeated?
    \item The repetition of digits is not allowed?
\end{enumerate}
\solution
%\input{ncert/11/16/4/9/main.tex}
\item Consider the probability space $\brak{\Omega, \mathcal{G}, P}$ where $\Omega = [0,2]$ and $\mathcal{G} = \cbrak{\phi, \Omega, [0,1], (1,2]}$. Let $X$ and $Y$ be two functions on $\Omega$ defined as
\begin{align*}
    X(\omega) = 
    \begin{cases}
        1 & \text{if }\omega \in [0, 1]\\
        2 & \text{if }\omega \in (1, 2]
    \end{cases}
\end{align*}
and
\begin{align*}
    Y(\omega) = 
    \begin{cases}
        2 & \text{if }\omega \in [0, 1.5]\\
        3 & \text{if }\omega \in (1.5, 2].
    \end{cases}
\end{align*}
Then which one of the following statements is true?
\begin{enumerate}
    \item [(A)] $X$ is a random variable with respect to $\mathcal{G}$, but $Y$ is not a random variable with respect to $\mathcal{G}$.
    \item [(B)] $Y$ is a random variable with respect to $\mathcal{G}$, but $X$ is not a random variable with respect to $\mathcal{G}$.
    \item [(C)] Neither $X$ nor $Y$ is a random variable with respect to $\mathcal{G}$.
    \item [(D)] Both $X$ and $Y$ are random variables with respect to $\mathcal{G}$.
\end{enumerate} \hfill (GATE ST 2023)\\
\solution
%\input{gate/ST/2023/14/main.tex}
	\item  A die is loaded in such a way that each odd number is twice as likely to occur as
each even number. Find $P(G)$, where $G$ is the event that a number greater than
3 occurs on a single roll of the die.
\\
\solution
		%\input{exemplar/11/16/3/5/main.tex}
	\item All the jacks, queens and kings are removed from a deck of 52 playing cards. The remaining cards are well shuffled and then one card is drawn at random. Giving ace a value 1 similar value for other cards, find the probability that the card has a value 
		\begin{enumerate}
			\item 7
			\item greater than 7
			\item less than 7
		\end{enumerate}
		%\input{exemplar/10/13/3/30/main.tex}
  \item A Lot consists of 48 mobile phones of which 42 are good, 3 have only minor defects and 3 have major defects.Varnika will buy a phone if it is good but the trader will only buy a mobile if it has no major defects. One phone is selected at random from the lot. What is the probability that it is
\begin{enumerate}
	\item acceptable to Varnika?
            \item acceptable to the trader?
\end{enumerate}
\solution
	%\input{exemplar/10/13/3/40/main.tex}
 \item A student says that if you throw a die, it will show up 1 or not 1. Therefore, the probability of getting 1 and the probability of getting 'not 1' each is equal to $\frac{1}{2}$. Is this correct? Give reasons.\\
 \solution
        %\input{exemplar/10/13/2/9/main.tex}
   \item Four candidates A, B, C, D have ap-
plied for the assignment to coach a school cricket
team. If A is twice as likely to be selected as B, and
B and C are given about the same chance of being
selected, while C is twice as likely to be selected
as D, what are the probabilities that
\begin{enumerate}
\item C will be selected?
\item A will not be selected?
\end{enumerate}
	%\input{exemplar/11/16/3/9/main.tex}
 \item A bag contain 24 balls of which $x$ balls are red, $2x$ are white and $3x$ are blue. A ball is selected at random, What is the probability that it is
\begin{enumerate}[label=\alph*)]
\item not red ?
\item white ?
\end{enumerate}
%\input{exemplar/10/13/3/41/main.tex}
If the letters of the word ASSASSINATION are arranged at random. Find the Probability that
\begin{enumerate}[label=(\alph*)]
\item Four $S's$ come consecutively in the word
\item Two  $I's$ and two $N's$ come together
\item All $A's$ are not coming together
\item No two $A's$ are coming together
\end{enumerate}
%\input{exemplar/11/16/3/14/main.tex}
	\item One urn contains two black balls (labelled B1 and B2) and one white ball. A
	second urn contains one black ball and two white balls (labelled W1 and W2).
	Suppose the following experiment is performed. One of the two urns is chosen
	at random. Next a ball is randomly chosen from the urn. Then a second ball is
	chosen at random from the same urn without replacing the first ball.
	
	\begin{enumerate}
	\item What is the probability that two black balls are chosen?
	
	\item What is the probability that two balls of opposite colour are chosen?
	\end{enumerate}
	\solution
	%\input{exemplar/11/16/3/12/main1.tex}
\end{enumerate}

	\item A bag contains 4 red and 4 black balls, another bag contains 2 red and 6 black balls. One of the two bags is selected at random and a ball is drawn from the bag which is found to be red. Find the probability that the ball is drawn from the first bag.
\\
\solution
		%\begin{table}[H]
	\centering
\begin{tabular}{|c|c|c|}
\hline
Random variable &Value &Definition\\ \hline
\multirow{3}{*}{X} &0 &Slips of Rs 1\\
&1 &Slips of Rs 5\\
&2 &Slips of Rs 13\\ \hline
\multirow{2}{*}{Y} &0 &Box A\\
&1 &Box B\\\hline
\end{tabular}
\caption{}
\label{tab:Distribution}
\end{table}
See \tabref{tab:Distribution}.
\begin{align}
p_{Y}\brak{k}= \begin{cases} 
      \frac{1}{3} & {k=0} \\
      \frac{2}{3 }& {k=1} 
   \end{cases}
   \\
p_{Y|X}\brak{0|0} = \frac{19}{25}\, 
p_{Y|X}\brak{0|1} = \frac{6}{25}\,
p_{Y|X}\brak{1|0} = \frac{45}{50}\,
p_{Y|X}\brak{1|2} = \frac{5}{50}
\end{align}
The desired probability is the probability that a slip drawn at random is marked other than Rs 1,
\begin{align}
&=1-p_X\brak{0}\\
&= p_X(1) + p_X(2)
\end{align}
Using Bayes theorem,
\begin{align}
&= p_Y\brak{0} \times \pr{Y=0 | X=1} + p_Y\brak{1} \times \pr{Y=1|X=2}\\
&=\frac{1}{3} \times \frac{6}{25} + \frac{2}{3} \times \frac{5}{50}\\
&=\frac{11}{75}
\end{align}

\newpage

%\tableofcontents

\bigskip

\renewcommand{\thefigure}{\theenumi}
\renewcommand{\thetable}{\theenumi}
%\renewcommand{\theequation}{\theenumi}

%\begin{abstract}
%%\boldmath
%In this letter, an algorithm for evaluating the exact analytical bit error rate  (BER)  for the piecewise linear (PL) combiner for  multiple relays is presented. Previous results were available only for upto three relays. The algorithm is unique in the sense that  the actual mathematical expressions, that are prohibitively large, need not be explicitly obtained. The diversity gain due to multiple relays is shown through plots of the analytical BER, well supported by simulations. 
%
%\end{abstract}
% IEEEtran.cls defaults to using nonbold math in the Abstract.
% This preserves the distinction between vectors and scalars. However,
% if the journal you are submitting to favors bold math in the abstract,
% then you can use LaTeX's standard command \boldmath at the very start
% of the abstract to achieve this. Many IEEE journals frown on math
% in the abstract anyway.

% Note that keywords are not normally used for peerreview papers.
%\begin{IEEEkeywords}
%Cooperative diversity, decode and forward, piecewise linear
%\end{IEEEkeywords}



% For peer review papers, you can put extra information on the cover
% page as needed:
% \ifCLASSOPTIONpeerreview
% \begin{center} \bfseries EDICS Category: 3-BBND \end{center}
% \fi
%
% For peerreview papers, this IEEEtran command inserts a page break and
% creates the second title. It will be ignored for other modes.
%\IEEEpeerreviewmaketitle




  \item
  Cards with numbers 2 to 101 are placed in a box. A card is selected at random.Find the probability that the card has
\begin{enumerate}[label=(\roman*)]
	\item an even number 
	\item a square number
\end{enumerate}
\solution
%\begin{table}[H]
	\centering
\begin{tabular}{|c|c|c|}
\hline
Random variable &Value &Definition\\ \hline
\multirow{3}{*}{X} &0 &Slips of Rs 1\\
&1 &Slips of Rs 5\\
&2 &Slips of Rs 13\\ \hline
\multirow{2}{*}{Y} &0 &Box A\\
&1 &Box B\\\hline
\end{tabular}
\caption{}
\label{tab:Distribution}
\end{table}
See \tabref{tab:Distribution}.
\begin{align}
p_{Y}\brak{k}= \begin{cases} 
      \frac{1}{3} & {k=0} \\
      \frac{2}{3 }& {k=1} 
   \end{cases}
   \\
p_{Y|X}\brak{0|0} = \frac{19}{25}\, 
p_{Y|X}\brak{0|1} = \frac{6}{25}\,
p_{Y|X}\brak{1|0} = \frac{45}{50}\,
p_{Y|X}\brak{1|2} = \frac{5}{50}
\end{align}
The desired probability is the probability that a slip drawn at random is marked other than Rs 1,
\begin{align}
&=1-p_X\brak{0}\\
&= p_X(1) + p_X(2)
\end{align}
Using Bayes theorem,
\begin{align}
&= p_Y\brak{0} \times \pr{Y=0 | X=1} + p_Y\brak{1} \times \pr{Y=1|X=2}\\
&=\frac{1}{3} \times \frac{6}{25} + \frac{2}{3} \times \frac{5}{50}\\
&=\frac{11}{75}
\end{align}

\newpage

%\tableofcontents

\bigskip

\renewcommand{\thefigure}{\theenumi}
\renewcommand{\thetable}{\theenumi}
%\renewcommand{\theequation}{\theenumi}

%\begin{abstract}
%%\boldmath
%In this letter, an algorithm for evaluating the exact analytical bit error rate  (BER)  for the piecewise linear (PL) combiner for  multiple relays is presented. Previous results were available only for upto three relays. The algorithm is unique in the sense that  the actual mathematical expressions, that are prohibitively large, need not be explicitly obtained. The diversity gain due to multiple relays is shown through plots of the analytical BER, well supported by simulations. 
%
%\end{abstract}
% IEEEtran.cls defaults to using nonbold math in the Abstract.
% This preserves the distinction between vectors and scalars. However,
% if the journal you are submitting to favors bold math in the abstract,
% then you can use LaTeX's standard command \boldmath at the very start
% of the abstract to achieve this. Many IEEE journals frown on math
% in the abstract anyway.

% Note that keywords are not normally used for peerreview papers.
%\begin{IEEEkeywords}
%Cooperative diversity, decode and forward, piecewise linear
%\end{IEEEkeywords}



% For peer review papers, you can put extra information on the cover
% page as needed:
% \ifCLASSOPTIONpeerreview
% \begin{center} \bfseries EDICS Category: 3-BBND \end{center}
% \fi
%
% For peerreview papers, this IEEEtran command inserts a page break and
% creates the second title. It will be ignored for other modes.
%\IEEEpeerreviewmaketitle




\item
The king, queen and jack of clubs are removed from a deck of 52 playing cards and then well shuffled. Now one card is drawn at random from the remaining cards.  Determine the probability that the card is
\begin{enumerate}[label=(\roman*)]
\item a club
\item 10 of hearts
\end{enumerate}
\solution
%\begin{table}[H]
	\centering
\begin{tabular}{|c|c|c|}
\hline
Random variable &Value &Definition\\ \hline
\multirow{3}{*}{X} &0 &Slips of Rs 1\\
&1 &Slips of Rs 5\\
&2 &Slips of Rs 13\\ \hline
\multirow{2}{*}{Y} &0 &Box A\\
&1 &Box B\\\hline
\end{tabular}
\caption{}
\label{tab:Distribution}
\end{table}
See \tabref{tab:Distribution}.
\begin{align}
p_{Y}\brak{k}= \begin{cases} 
      \frac{1}{3} & {k=0} \\
      \frac{2}{3 }& {k=1} 
   \end{cases}
   \\
p_{Y|X}\brak{0|0} = \frac{19}{25}\, 
p_{Y|X}\brak{0|1} = \frac{6}{25}\,
p_{Y|X}\brak{1|0} = \frac{45}{50}\,
p_{Y|X}\brak{1|2} = \frac{5}{50}
\end{align}
The desired probability is the probability that a slip drawn at random is marked other than Rs 1,
\begin{align}
&=1-p_X\brak{0}\\
&= p_X(1) + p_X(2)
\end{align}
Using Bayes theorem,
\begin{align}
&= p_Y\brak{0} \times \pr{Y=0 | X=1} + p_Y\brak{1} \times \pr{Y=1|X=2}\\
&=\frac{1}{3} \times \frac{6}{25} + \frac{2}{3} \times \frac{5}{50}\\
&=\frac{11}{75}
\end{align}

\newpage

%\tableofcontents

\bigskip

\renewcommand{\thefigure}{\theenumi}
\renewcommand{\thetable}{\theenumi}
%\renewcommand{\theequation}{\theenumi}

%\begin{abstract}
%%\boldmath
%In this letter, an algorithm for evaluating the exact analytical bit error rate  (BER)  for the piecewise linear (PL) combiner for  multiple relays is presented. Previous results were available only for upto three relays. The algorithm is unique in the sense that  the actual mathematical expressions, that are prohibitively large, need not be explicitly obtained. The diversity gain due to multiple relays is shown through plots of the analytical BER, well supported by simulations. 
%
%\end{abstract}
% IEEEtran.cls defaults to using nonbold math in the Abstract.
% This preserves the distinction between vectors and scalars. However,
% if the journal you are submitting to favors bold math in the abstract,
% then you can use LaTeX's standard command \boldmath at the very start
% of the abstract to achieve this. Many IEEE journals frown on math
% in the abstract anyway.

% Note that keywords are not normally used for peerreview papers.
%\begin{IEEEkeywords}
%Cooperative diversity, decode and forward, piecewise linear
%\end{IEEEkeywords}



% For peer review papers, you can put extra information on the cover
% page as needed:
% \ifCLASSOPTIONpeerreview
% \begin{center} \bfseries EDICS Category: 3-BBND \end{center}
% \fi
%
% For peerreview papers, this IEEEtran command inserts a page break and
% creates the second title. It will be ignored for other modes.
%\IEEEpeerreviewmaketitle




\item A team of medical students doing their internship have to assist during surgeries
at a city hospital. The probabilities of surgeries rated as very complex, complex,
routine, simple or very simple are respectively, 0.15, 0.20, 0.31, 0.26, .08. Find
the probabilities that a particular surgery will be rated
\begin{enumerate}
	\item complex or very complex;
	\item neither very complex nor very simple;
	\item routine or complex
	\item routine or simple
\end{enumerate}
\solution
%\begin{table}[H]
	\centering
\begin{tabular}{|c|c|c|}
\hline
Random variable &Value &Definition\\ \hline
\multirow{3}{*}{X} &0 &Slips of Rs 1\\
&1 &Slips of Rs 5\\
&2 &Slips of Rs 13\\ \hline
\multirow{2}{*}{Y} &0 &Box A\\
&1 &Box B\\\hline
\end{tabular}
\caption{}
\label{tab:Distribution}
\end{table}
See \tabref{tab:Distribution}.
\begin{align}
p_{Y}\brak{k}= \begin{cases} 
      \frac{1}{3} & {k=0} \\
      \frac{2}{3 }& {k=1} 
   \end{cases}
   \\
p_{Y|X}\brak{0|0} = \frac{19}{25}\, 
p_{Y|X}\brak{0|1} = \frac{6}{25}\,
p_{Y|X}\brak{1|0} = \frac{45}{50}\,
p_{Y|X}\brak{1|2} = \frac{5}{50}
\end{align}
The desired probability is the probability that a slip drawn at random is marked other than Rs 1,
\begin{align}
&=1-p_X\brak{0}\\
&= p_X(1) + p_X(2)
\end{align}
Using Bayes theorem,
\begin{align}
&= p_Y\brak{0} \times \pr{Y=0 | X=1} + p_Y\brak{1} \times \pr{Y=1|X=2}\\
&=\frac{1}{3} \times \frac{6}{25} + \frac{2}{3} \times \frac{5}{50}\\
&=\frac{11}{75}
\end{align}

\newpage

%\tableofcontents

\bigskip

\renewcommand{\thefigure}{\theenumi}
\renewcommand{\thetable}{\theenumi}
%\renewcommand{\theequation}{\theenumi}

%\begin{abstract}
%%\boldmath
%In this letter, an algorithm for evaluating the exact analytical bit error rate  (BER)  for the piecewise linear (PL) combiner for  multiple relays is presented. Previous results were available only for upto three relays. The algorithm is unique in the sense that  the actual mathematical expressions, that are prohibitively large, need not be explicitly obtained. The diversity gain due to multiple relays is shown through plots of the analytical BER, well supported by simulations. 
%
%\end{abstract}
% IEEEtran.cls defaults to using nonbold math in the Abstract.
% This preserves the distinction between vectors and scalars. However,
% if the journal you are submitting to favors bold math in the abstract,
% then you can use LaTeX's standard command \boldmath at the very start
% of the abstract to achieve this. Many IEEE journals frown on math
% in the abstract anyway.

% Note that keywords are not normally used for peerreview papers.
%\begin{IEEEkeywords}
%Cooperative diversity, decode and forward, piecewise linear
%\end{IEEEkeywords}



% For peer review papers, you can put extra information on the cover
% page as needed:
% \ifCLASSOPTIONpeerreview
% \begin{center} \bfseries EDICS Category: 3-BBND \end{center}
% \fi
%
% For peerreview papers, this IEEEtran command inserts a page break and
% creates the second title. It will be ignored for other modes.
%\IEEEpeerreviewmaketitle




\item A card is selected from a pack of 52 cards.
\begin{enumerate}[label=(\alph*)]
    \item How many points are there in the sample space?
    \item Calculate the probability that the card is an ace of spades.
    \item Calculate the probability that the card is (i) an ace and (ii) black card.
\end{enumerate}
\solution
%Let $X$ be an bernoulli rv defined as in \tabref{tab:exemplar/11/16/3/26}.  Then, 
\begin{equation}
    p =
        \frac{4}{11} 
\end{equation}
\begin{table}[H]
	\centering
	\input{exemplar/11/16/3/26/tables/Table2.tex}
	\caption{}
        \label{tab:exemplar/11/16/3/26}
\end{table}

\item The probability that a non leap year selected at random will contain 53 sundays.
\\
\solution
%\begin{table}[H]
	\centering
\begin{tabular}{|c|c|c|}
\hline
Random variable &Value &Definition\\ \hline
\multirow{3}{*}{X} &0 &Slips of Rs 1\\
&1 &Slips of Rs 5\\
&2 &Slips of Rs 13\\ \hline
\multirow{2}{*}{Y} &0 &Box A\\
&1 &Box B\\\hline
\end{tabular}
\caption{}
\label{tab:Distribution}
\end{table}
See \tabref{tab:Distribution}.
\begin{align}
p_{Y}\brak{k}= \begin{cases} 
      \frac{1}{3} & {k=0} \\
      \frac{2}{3 }& {k=1} 
   \end{cases}
   \\
p_{Y|X}\brak{0|0} = \frac{19}{25}\, 
p_{Y|X}\brak{0|1} = \frac{6}{25}\,
p_{Y|X}\brak{1|0} = \frac{45}{50}\,
p_{Y|X}\brak{1|2} = \frac{5}{50}
\end{align}
The desired probability is the probability that a slip drawn at random is marked other than Rs 1,
\begin{align}
&=1-p_X\brak{0}\\
&= p_X(1) + p_X(2)
\end{align}
Using Bayes theorem,
\begin{align}
&= p_Y\brak{0} \times \pr{Y=0 | X=1} + p_Y\brak{1} \times \pr{Y=1|X=2}\\
&=\frac{1}{3} \times \frac{6}{25} + \frac{2}{3} \times \frac{5}{50}\\
&=\frac{11}{75}
\end{align}

\newpage

%\tableofcontents

\bigskip

\renewcommand{\thefigure}{\theenumi}
\renewcommand{\thetable}{\theenumi}
%\renewcommand{\theequation}{\theenumi}

%\begin{abstract}
%%\boldmath
%In this letter, an algorithm for evaluating the exact analytical bit error rate  (BER)  for the piecewise linear (PL) combiner for  multiple relays is presented. Previous results were available only for upto three relays. The algorithm is unique in the sense that  the actual mathematical expressions, that are prohibitively large, need not be explicitly obtained. The diversity gain due to multiple relays is shown through plots of the analytical BER, well supported by simulations. 
%
%\end{abstract}
% IEEEtran.cls defaults to using nonbold math in the Abstract.
% This preserves the distinction between vectors and scalars. However,
% if the journal you are submitting to favors bold math in the abstract,
% then you can use LaTeX's standard command \boldmath at the very start
% of the abstract to achieve this. Many IEEE journals frown on math
% in the abstract anyway.

% Note that keywords are not normally used for peerreview papers.
%\begin{IEEEkeywords}
%Cooperative diversity, decode and forward, piecewise linear
%\end{IEEEkeywords}



% For peer review papers, you can put extra information on the cover
% page as needed:
% \ifCLASSOPTIONpeerreview
% \begin{center} \bfseries EDICS Category: 3-BBND \end{center}
% \fi
%
% For peerreview papers, this IEEEtran command inserts a page break and
% creates the second title. It will be ignored for other modes.
%\IEEEpeerreviewmaketitle




\item One of the four persons John, Rita, Aslam or Gurpreet will be promoted next
month. Consequently the sample space consists of four elementary outcomes
S = {John promoted, Rita promoted, Aslam promoted, Gurpreet promoted}
You are told that the chances of John’s promotion is same as that of Gurpreet,
Rita’s chances of promotion are twice as likely as Johns. Aslam’s chances are
four times that of John.
\begin{enumerate}
	\item Determine
	\begin{enumerate}
		\item P (John promoted)
		\item P (Rita promoted)
		\item P (Aslam promoted)
		\item P (Gurpreet promoted)
	\end{enumerate}
	\item If A = {John promoted or Gurpreet promoted}, find P (A).
\end{enumerate}
\solution
%\begin{table}[H]
	\centering
\begin{tabular}{|c|c|c|}
\hline
Random variable &Value &Definition\\ \hline
\multirow{3}{*}{X} &0 &Slips of Rs 1\\
&1 &Slips of Rs 5\\
&2 &Slips of Rs 13\\ \hline
\multirow{2}{*}{Y} &0 &Box A\\
&1 &Box B\\\hline
\end{tabular}
\caption{}
\label{tab:Distribution}
\end{table}
See \tabref{tab:Distribution}.
\begin{align}
p_{Y}\brak{k}= \begin{cases} 
      \frac{1}{3} & {k=0} \\
      \frac{2}{3 }& {k=1} 
   \end{cases}
   \\
p_{Y|X}\brak{0|0} = \frac{19}{25}\, 
p_{Y|X}\brak{0|1} = \frac{6}{25}\,
p_{Y|X}\brak{1|0} = \frac{45}{50}\,
p_{Y|X}\brak{1|2} = \frac{5}{50}
\end{align}
The desired probability is the probability that a slip drawn at random is marked other than Rs 1,
\begin{align}
&=1-p_X\brak{0}\\
&= p_X(1) + p_X(2)
\end{align}
Using Bayes theorem,
\begin{align}
&= p_Y\brak{0} \times \pr{Y=0 | X=1} + p_Y\brak{1} \times \pr{Y=1|X=2}\\
&=\frac{1}{3} \times \frac{6}{25} + \frac{2}{3} \times \frac{5}{50}\\
&=\frac{11}{75}
\end{align}

\newpage

%\tableofcontents

\bigskip

\renewcommand{\thefigure}{\theenumi}
\renewcommand{\thetable}{\theenumi}
%\renewcommand{\theequation}{\theenumi}

%\begin{abstract}
%%\boldmath
%In this letter, an algorithm for evaluating the exact analytical bit error rate  (BER)  for the piecewise linear (PL) combiner for  multiple relays is presented. Previous results were available only for upto three relays. The algorithm is unique in the sense that  the actual mathematical expressions, that are prohibitively large, need not be explicitly obtained. The diversity gain due to multiple relays is shown through plots of the analytical BER, well supported by simulations. 
%
%\end{abstract}
% IEEEtran.cls defaults to using nonbold math in the Abstract.
% This preserves the distinction between vectors and scalars. However,
% if the journal you are submitting to favors bold math in the abstract,
% then you can use LaTeX's standard command \boldmath at the very start
% of the abstract to achieve this. Many IEEE journals frown on math
% in the abstract anyway.

% Note that keywords are not normally used for peerreview papers.
%\begin{IEEEkeywords}
%Cooperative diversity, decode and forward, piecewise linear
%\end{IEEEkeywords}



% For peer review papers, you can put extra information on the cover
% page as needed:
% \ifCLASSOPTIONpeerreview
% \begin{center} \bfseries EDICS Category: 3-BBND \end{center}
% \fi
%
% For peerreview papers, this IEEEtran command inserts a page break and
% creates the second title. It will be ignored for other modes.
%\IEEEpeerreviewmaketitle




\item A card is drawn from a deck of 52 cards. Find the probability of getting a king or a heart or a red card.\\
\solution
%\begin{table}[H]
	\centering
\begin{tabular}{|c|c|c|}
\hline
Random variable &Value &Definition\\ \hline
\multirow{3}{*}{X} &0 &Slips of Rs 1\\
&1 &Slips of Rs 5\\
&2 &Slips of Rs 13\\ \hline
\multirow{2}{*}{Y} &0 &Box A\\
&1 &Box B\\\hline
\end{tabular}
\caption{}
\label{tab:Distribution}
\end{table}
See \tabref{tab:Distribution}.
\begin{align}
p_{Y}\brak{k}= \begin{cases} 
      \frac{1}{3} & {k=0} \\
      \frac{2}{3 }& {k=1} 
   \end{cases}
   \\
p_{Y|X}\brak{0|0} = \frac{19}{25}\, 
p_{Y|X}\brak{0|1} = \frac{6}{25}\,
p_{Y|X}\brak{1|0} = \frac{45}{50}\,
p_{Y|X}\brak{1|2} = \frac{5}{50}
\end{align}
The desired probability is the probability that a slip drawn at random is marked other than Rs 1,
\begin{align}
&=1-p_X\brak{0}\\
&= p_X(1) + p_X(2)
\end{align}
Using Bayes theorem,
\begin{align}
&= p_Y\brak{0} \times \pr{Y=0 | X=1} + p_Y\brak{1} \times \pr{Y=1|X=2}\\
&=\frac{1}{3} \times \frac{6}{25} + \frac{2}{3} \times \frac{5}{50}\\
&=\frac{11}{75}
\end{align}

\newpage

%\tableofcontents

\bigskip

\renewcommand{\thefigure}{\theenumi}
\renewcommand{\thetable}{\theenumi}
%\renewcommand{\theequation}{\theenumi}

%\begin{abstract}
%%\boldmath
%In this letter, an algorithm for evaluating the exact analytical bit error rate  (BER)  for the piecewise linear (PL) combiner for  multiple relays is presented. Previous results were available only for upto three relays. The algorithm is unique in the sense that  the actual mathematical expressions, that are prohibitively large, need not be explicitly obtained. The diversity gain due to multiple relays is shown through plots of the analytical BER, well supported by simulations. 
%
%\end{abstract}
% IEEEtran.cls defaults to using nonbold math in the Abstract.
% This preserves the distinction between vectors and scalars. However,
% if the journal you are submitting to favors bold math in the abstract,
% then you can use LaTeX's standard command \boldmath at the very start
% of the abstract to achieve this. Many IEEE journals frown on math
% in the abstract anyway.

% Note that keywords are not normally used for peerreview papers.
%\begin{IEEEkeywords}
%Cooperative diversity, decode and forward, piecewise linear
%\end{IEEEkeywords}



% For peer review papers, you can put extra information on the cover
% page as needed:
% \ifCLASSOPTIONpeerreview
% \begin{center} \bfseries EDICS Category: 3-BBND \end{center}
% \fi
%
% For peerreview papers, this IEEEtran command inserts a page break and
% creates the second title. It will be ignored for other modes.
%\IEEEpeerreviewmaketitle




\item The probability that a student will pass his examination is 0.73, the probability of
the student getting a compartment is 0.13, and the probability that the student will
either pass or get compartment is 0.96. State True or False.\\
\solution
%\begin{table}[H]
	\centering
\begin{tabular}{|c|c|c|}
\hline
Random variable &Value &Definition\\ \hline
\multirow{3}{*}{X} &0 &Slips of Rs 1\\
&1 &Slips of Rs 5\\
&2 &Slips of Rs 13\\ \hline
\multirow{2}{*}{Y} &0 &Box A\\
&1 &Box B\\\hline
\end{tabular}
\caption{}
\label{tab:Distribution}
\end{table}
See \tabref{tab:Distribution}.
\begin{align}
p_{Y}\brak{k}= \begin{cases} 
      \frac{1}{3} & {k=0} \\
      \frac{2}{3 }& {k=1} 
   \end{cases}
   \\
p_{Y|X}\brak{0|0} = \frac{19}{25}\, 
p_{Y|X}\brak{0|1} = \frac{6}{25}\,
p_{Y|X}\brak{1|0} = \frac{45}{50}\,
p_{Y|X}\brak{1|2} = \frac{5}{50}
\end{align}
The desired probability is the probability that a slip drawn at random is marked other than Rs 1,
\begin{align}
&=1-p_X\brak{0}\\
&= p_X(1) + p_X(2)
\end{align}
Using Bayes theorem,
\begin{align}
&= p_Y\brak{0} \times \pr{Y=0 | X=1} + p_Y\brak{1} \times \pr{Y=1|X=2}\\
&=\frac{1}{3} \times \frac{6}{25} + \frac{2}{3} \times \frac{5}{50}\\
&=\frac{11}{75}
\end{align}

\newpage

%\tableofcontents

\bigskip

\renewcommand{\thefigure}{\theenumi}
\renewcommand{\thetable}{\theenumi}
%\renewcommand{\theequation}{\theenumi}

%\begin{abstract}
%%\boldmath
%In this letter, an algorithm for evaluating the exact analytical bit error rate  (BER)  for the piecewise linear (PL) combiner for  multiple relays is presented. Previous results were available only for upto three relays. The algorithm is unique in the sense that  the actual mathematical expressions, that are prohibitively large, need not be explicitly obtained. The diversity gain due to multiple relays is shown through plots of the analytical BER, well supported by simulations. 
%
%\end{abstract}
% IEEEtran.cls defaults to using nonbold math in the Abstract.
% This preserves the distinction between vectors and scalars. However,
% if the journal you are submitting to favors bold math in the abstract,
% then you can use LaTeX's standard command \boldmath at the very start
% of the abstract to achieve this. Many IEEE journals frown on math
% in the abstract anyway.

% Note that keywords are not normally used for peerreview papers.
%\begin{IEEEkeywords}
%Cooperative diversity, decode and forward, piecewise linear
%\end{IEEEkeywords}



% For peer review papers, you can put extra information on the cover
% page as needed:
% \ifCLASSOPTIONpeerreview
% \begin{center} \bfseries EDICS Category: 3-BBND \end{center}
% \fi
%
% For peerreview papers, this IEEEtran command inserts a page break and
% creates the second title. It will be ignored for other modes.
%\IEEEpeerreviewmaketitle




\item A card is selected from a pack of 52 cards\\
\begin{enumerate}[label=(\alph*)]
\item How many points are there in the sample space?
\item Calculate the probability that the cards is an ace of spades.
\item Calculate the probability that the card is (i) an ace (ii)black card.\\
\end{enumerate}
%\input{ncert/11/16/3/4_1/Prob_4.tex}
\item In a non-leap year, the probability of having 53 tuesdays or 53 wednesdays is\\
\solution
%A non-leap year has a total of 365 days, and a week has 7 days.\\
So it can be expressed as 
\begin{align}
365\text{days} &=52\times 7+1 \text{day}
\end{align}
$\implies$ 52 tuesdays or wednesdays\\
Random variable X denotes the days of a week
\begin{align}
p_X\brak{k}&=\frac{1}{7}; \quad \brak{1<k<7}
\end{align}
So the probability of extra day being tuesday or wednesday is
\begin{align}
p_X\brak{3}+p_X\brak{4}&=\frac{1}{7}+\frac{1}{7}=\frac{2}{7}
\end{align}



\item There are 1000 sealed envelopes in a box, 10 of them contain a cash prize of
Rs 100 each, 100 of them contain a cash prize of Rs 50 each and 200 of them
contain a cash prize of Rs 10 each and rest do not contain any cash prize. If they
are well shuffled and an envelope is picked up out, what is the probability that it
contains no cash prize?\\
\solution
%\begin{table}[H]
	\centering
\begin{tabular}{|c|c|c|}
\hline
Random variable &Value &Definition\\ \hline
\multirow{3}{*}{X} &0 &Slips of Rs 1\\
&1 &Slips of Rs 5\\
&2 &Slips of Rs 13\\ \hline
\multirow{2}{*}{Y} &0 &Box A\\
&1 &Box B\\\hline
\end{tabular}
\caption{}
\label{tab:Distribution}
\end{table}
See \tabref{tab:Distribution}.
\begin{align}
p_{Y}\brak{k}= \begin{cases} 
      \frac{1}{3} & {k=0} \\
      \frac{2}{3 }& {k=1} 
   \end{cases}
   \\
p_{Y|X}\brak{0|0} = \frac{19}{25}\, 
p_{Y|X}\brak{0|1} = \frac{6}{25}\,
p_{Y|X}\brak{1|0} = \frac{45}{50}\,
p_{Y|X}\brak{1|2} = \frac{5}{50}
\end{align}
The desired probability is the probability that a slip drawn at random is marked other than Rs 1,
\begin{align}
&=1-p_X\brak{0}\\
&= p_X(1) + p_X(2)
\end{align}
Using Bayes theorem,
\begin{align}
&= p_Y\brak{0} \times \pr{Y=0 | X=1} + p_Y\brak{1} \times \pr{Y=1|X=2}\\
&=\frac{1}{3} \times \frac{6}{25} + \frac{2}{3} \times \frac{5}{50}\\
&=\frac{11}{75}
\end{align}

\newpage

%\tableofcontents

\bigskip

\renewcommand{\thefigure}{\theenumi}
\renewcommand{\thetable}{\theenumi}
%\renewcommand{\theequation}{\theenumi}

%\begin{abstract}
%%\boldmath
%In this letter, an algorithm for evaluating the exact analytical bit error rate  (BER)  for the piecewise linear (PL) combiner for  multiple relays is presented. Previous results were available only for upto three relays. The algorithm is unique in the sense that  the actual mathematical expressions, that are prohibitively large, need not be explicitly obtained. The diversity gain due to multiple relays is shown through plots of the analytical BER, well supported by simulations. 
%
%\end{abstract}
% IEEEtran.cls defaults to using nonbold math in the Abstract.
% This preserves the distinction between vectors and scalars. However,
% if the journal you are submitting to favors bold math in the abstract,
% then you can use LaTeX's standard command \boldmath at the very start
% of the abstract to achieve this. Many IEEE journals frown on math
% in the abstract anyway.

% Note that keywords are not normally used for peerreview papers.
%\begin{IEEEkeywords}
%Cooperative diversity, decode and forward, piecewise linear
%\end{IEEEkeywords}



% For peer review papers, you can put extra information on the cover
% page as needed:
% \ifCLASSOPTIONpeerreview
% \begin{center} \bfseries EDICS Category: 3-BBND \end{center}
% \fi
%
% For peerreview papers, this IEEEtran command inserts a page break and
% creates the second title. It will be ignored for other modes.
%\IEEEpeerreviewmaketitle




\item 
A die is thrown and a card is selected at random from a deck of 52 playing cards. The probability of getting an even number on the die and a spade card.\\
\solution
%\begin{table}[H]
	\centering
\begin{tabular}{|c|c|c|}
\hline
Random variable &Value &Definition\\ \hline
\multirow{3}{*}{X} &0 &Slips of Rs 1\\
&1 &Slips of Rs 5\\
&2 &Slips of Rs 13\\ \hline
\multirow{2}{*}{Y} &0 &Box A\\
&1 &Box B\\\hline
\end{tabular}
\caption{}
\label{tab:Distribution}
\end{table}
See \tabref{tab:Distribution}.
\begin{align}
p_{Y}\brak{k}= \begin{cases} 
      \frac{1}{3} & {k=0} \\
      \frac{2}{3 }& {k=1} 
   \end{cases}
   \\
p_{Y|X}\brak{0|0} = \frac{19}{25}\, 
p_{Y|X}\brak{0|1} = \frac{6}{25}\,
p_{Y|X}\brak{1|0} = \frac{45}{50}\,
p_{Y|X}\brak{1|2} = \frac{5}{50}
\end{align}
The desired probability is the probability that a slip drawn at random is marked other than Rs 1,
\begin{align}
&=1-p_X\brak{0}\\
&= p_X(1) + p_X(2)
\end{align}
Using Bayes theorem,
\begin{align}
&= p_Y\brak{0} \times \pr{Y=0 | X=1} + p_Y\brak{1} \times \pr{Y=1|X=2}\\
&=\frac{1}{3} \times \frac{6}{25} + \frac{2}{3} \times \frac{5}{50}\\
&=\frac{11}{75}
\end{align}

\newpage

%\tableofcontents

\bigskip

\renewcommand{\thefigure}{\theenumi}
\renewcommand{\thetable}{\theenumi}
%\renewcommand{\theequation}{\theenumi}

%\begin{abstract}
%%\boldmath
%In this letter, an algorithm for evaluating the exact analytical bit error rate  (BER)  for the piecewise linear (PL) combiner for  multiple relays is presented. Previous results were available only for upto three relays. The algorithm is unique in the sense that  the actual mathematical expressions, that are prohibitively large, need not be explicitly obtained. The diversity gain due to multiple relays is shown through plots of the analytical BER, well supported by simulations. 
%
%\end{abstract}
% IEEEtran.cls defaults to using nonbold math in the Abstract.
% This preserves the distinction between vectors and scalars. However,
% if the journal you are submitting to favors bold math in the abstract,
% then you can use LaTeX's standard command \boldmath at the very start
% of the abstract to achieve this. Many IEEE journals frown on math
% in the abstract anyway.

% Note that keywords are not normally used for peerreview papers.
%\begin{IEEEkeywords}
%Cooperative diversity, decode and forward, piecewise linear
%\end{IEEEkeywords}



% For peer review papers, you can put extra information on the cover
% page as needed:
% \ifCLASSOPTIONpeerreview
% \begin{center} \bfseries EDICS Category: 3-BBND \end{center}
% \fi
%
% For peerreview papers, this IEEEtran command inserts a page break and
% creates the second title. It will be ignored for other modes.
%\IEEEpeerreviewmaketitle




\item
If 4-digit numbers greater than 5,000 are randomly formed from the digits 0, 1, 3, 5, and 7, what is the probability of forming a number divisible by 5 when:
\begin{enumerate}
    \item The digits are repeated?
    \item The repetition of digits is not allowed?
\end{enumerate}
\solution
%\begin{table}[H]
	\centering
\begin{tabular}{|c|c|c|}
\hline
Random variable &Value &Definition\\ \hline
\multirow{3}{*}{X} &0 &Slips of Rs 1\\
&1 &Slips of Rs 5\\
&2 &Slips of Rs 13\\ \hline
\multirow{2}{*}{Y} &0 &Box A\\
&1 &Box B\\\hline
\end{tabular}
\caption{}
\label{tab:Distribution}
\end{table}
See \tabref{tab:Distribution}.
\begin{align}
p_{Y}\brak{k}= \begin{cases} 
      \frac{1}{3} & {k=0} \\
      \frac{2}{3 }& {k=1} 
   \end{cases}
   \\
p_{Y|X}\brak{0|0} = \frac{19}{25}\, 
p_{Y|X}\brak{0|1} = \frac{6}{25}\,
p_{Y|X}\brak{1|0} = \frac{45}{50}\,
p_{Y|X}\brak{1|2} = \frac{5}{50}
\end{align}
The desired probability is the probability that a slip drawn at random is marked other than Rs 1,
\begin{align}
&=1-p_X\brak{0}\\
&= p_X(1) + p_X(2)
\end{align}
Using Bayes theorem,
\begin{align}
&= p_Y\brak{0} \times \pr{Y=0 | X=1} + p_Y\brak{1} \times \pr{Y=1|X=2}\\
&=\frac{1}{3} \times \frac{6}{25} + \frac{2}{3} \times \frac{5}{50}\\
&=\frac{11}{75}
\end{align}

\newpage

%\tableofcontents

\bigskip

\renewcommand{\thefigure}{\theenumi}
\renewcommand{\thetable}{\theenumi}
%\renewcommand{\theequation}{\theenumi}

%\begin{abstract}
%%\boldmath
%In this letter, an algorithm for evaluating the exact analytical bit error rate  (BER)  for the piecewise linear (PL) combiner for  multiple relays is presented. Previous results were available only for upto three relays. The algorithm is unique in the sense that  the actual mathematical expressions, that are prohibitively large, need not be explicitly obtained. The diversity gain due to multiple relays is shown through plots of the analytical BER, well supported by simulations. 
%
%\end{abstract}
% IEEEtran.cls defaults to using nonbold math in the Abstract.
% This preserves the distinction between vectors and scalars. However,
% if the journal you are submitting to favors bold math in the abstract,
% then you can use LaTeX's standard command \boldmath at the very start
% of the abstract to achieve this. Many IEEE journals frown on math
% in the abstract anyway.

% Note that keywords are not normally used for peerreview papers.
%\begin{IEEEkeywords}
%Cooperative diversity, decode and forward, piecewise linear
%\end{IEEEkeywords}



% For peer review papers, you can put extra information on the cover
% page as needed:
% \ifCLASSOPTIONpeerreview
% \begin{center} \bfseries EDICS Category: 3-BBND \end{center}
% \fi
%
% For peerreview papers, this IEEEtran command inserts a page break and
% creates the second title. It will be ignored for other modes.
%\IEEEpeerreviewmaketitle




\item Consider the probability space $\brak{\Omega, \mathcal{G}, P}$ where $\Omega = [0,2]$ and $\mathcal{G} = \cbrak{\phi, \Omega, [0,1], (1,2]}$. Let $X$ and $Y$ be two functions on $\Omega$ defined as
\begin{align*}
    X(\omega) = 
    \begin{cases}
        1 & \text{if }\omega \in [0, 1]\\
        2 & \text{if }\omega \in (1, 2]
    \end{cases}
\end{align*}
and
\begin{align*}
    Y(\omega) = 
    \begin{cases}
        2 & \text{if }\omega \in [0, 1.5]\\
        3 & \text{if }\omega \in (1.5, 2].
    \end{cases}
\end{align*}
Then which one of the following statements is true?
\begin{enumerate}
    \item [(A)] $X$ is a random variable with respect to $\mathcal{G}$, but $Y$ is not a random variable with respect to $\mathcal{G}$.
    \item [(B)] $Y$ is a random variable with respect to $\mathcal{G}$, but $X$ is not a random variable with respect to $\mathcal{G}$.
    \item [(C)] Neither $X$ nor $Y$ is a random variable with respect to $\mathcal{G}$.
    \item [(D)] Both $X$ and $Y$ are random variables with respect to $\mathcal{G}$.
\end{enumerate} \hfill (GATE ST 2023)\\
\solution
%\begin{table}[H]
	\centering
\begin{tabular}{|c|c|c|}
\hline
Random variable &Value &Definition\\ \hline
\multirow{3}{*}{X} &0 &Slips of Rs 1\\
&1 &Slips of Rs 5\\
&2 &Slips of Rs 13\\ \hline
\multirow{2}{*}{Y} &0 &Box A\\
&1 &Box B\\\hline
\end{tabular}
\caption{}
\label{tab:Distribution}
\end{table}
See \tabref{tab:Distribution}.
\begin{align}
p_{Y}\brak{k}= \begin{cases} 
      \frac{1}{3} & {k=0} \\
      \frac{2}{3 }& {k=1} 
   \end{cases}
   \\
p_{Y|X}\brak{0|0} = \frac{19}{25}\, 
p_{Y|X}\brak{0|1} = \frac{6}{25}\,
p_{Y|X}\brak{1|0} = \frac{45}{50}\,
p_{Y|X}\brak{1|2} = \frac{5}{50}
\end{align}
The desired probability is the probability that a slip drawn at random is marked other than Rs 1,
\begin{align}
&=1-p_X\brak{0}\\
&= p_X(1) + p_X(2)
\end{align}
Using Bayes theorem,
\begin{align}
&= p_Y\brak{0} \times \pr{Y=0 | X=1} + p_Y\brak{1} \times \pr{Y=1|X=2}\\
&=\frac{1}{3} \times \frac{6}{25} + \frac{2}{3} \times \frac{5}{50}\\
&=\frac{11}{75}
\end{align}

\newpage

%\tableofcontents

\bigskip

\renewcommand{\thefigure}{\theenumi}
\renewcommand{\thetable}{\theenumi}
%\renewcommand{\theequation}{\theenumi}

%\begin{abstract}
%%\boldmath
%In this letter, an algorithm for evaluating the exact analytical bit error rate  (BER)  for the piecewise linear (PL) combiner for  multiple relays is presented. Previous results were available only for upto three relays. The algorithm is unique in the sense that  the actual mathematical expressions, that are prohibitively large, need not be explicitly obtained. The diversity gain due to multiple relays is shown through plots of the analytical BER, well supported by simulations. 
%
%\end{abstract}
% IEEEtran.cls defaults to using nonbold math in the Abstract.
% This preserves the distinction between vectors and scalars. However,
% if the journal you are submitting to favors bold math in the abstract,
% then you can use LaTeX's standard command \boldmath at the very start
% of the abstract to achieve this. Many IEEE journals frown on math
% in the abstract anyway.

% Note that keywords are not normally used for peerreview papers.
%\begin{IEEEkeywords}
%Cooperative diversity, decode and forward, piecewise linear
%\end{IEEEkeywords}



% For peer review papers, you can put extra information on the cover
% page as needed:
% \ifCLASSOPTIONpeerreview
% \begin{center} \bfseries EDICS Category: 3-BBND \end{center}
% \fi
%
% For peerreview papers, this IEEEtran command inserts a page break and
% creates the second title. It will be ignored for other modes.
%\IEEEpeerreviewmaketitle




	\item  A die is loaded in such a way that each odd number is twice as likely to occur as
each even number. Find $P(G)$, where $G$ is the event that a number greater than
3 occurs on a single roll of the die.
\\
\solution
		%\begin{table}[H]
	\centering
\begin{tabular}{|c|c|c|}
\hline
Random variable &Value &Definition\\ \hline
\multirow{3}{*}{X} &0 &Slips of Rs 1\\
&1 &Slips of Rs 5\\
&2 &Slips of Rs 13\\ \hline
\multirow{2}{*}{Y} &0 &Box A\\
&1 &Box B\\\hline
\end{tabular}
\caption{}
\label{tab:Distribution}
\end{table}
See \tabref{tab:Distribution}.
\begin{align}
p_{Y}\brak{k}= \begin{cases} 
      \frac{1}{3} & {k=0} \\
      \frac{2}{3 }& {k=1} 
   \end{cases}
   \\
p_{Y|X}\brak{0|0} = \frac{19}{25}\, 
p_{Y|X}\brak{0|1} = \frac{6}{25}\,
p_{Y|X}\brak{1|0} = \frac{45}{50}\,
p_{Y|X}\brak{1|2} = \frac{5}{50}
\end{align}
The desired probability is the probability that a slip drawn at random is marked other than Rs 1,
\begin{align}
&=1-p_X\brak{0}\\
&= p_X(1) + p_X(2)
\end{align}
Using Bayes theorem,
\begin{align}
&= p_Y\brak{0} \times \pr{Y=0 | X=1} + p_Y\brak{1} \times \pr{Y=1|X=2}\\
&=\frac{1}{3} \times \frac{6}{25} + \frac{2}{3} \times \frac{5}{50}\\
&=\frac{11}{75}
\end{align}

\newpage

%\tableofcontents

\bigskip

\renewcommand{\thefigure}{\theenumi}
\renewcommand{\thetable}{\theenumi}
%\renewcommand{\theequation}{\theenumi}

%\begin{abstract}
%%\boldmath
%In this letter, an algorithm for evaluating the exact analytical bit error rate  (BER)  for the piecewise linear (PL) combiner for  multiple relays is presented. Previous results were available only for upto three relays. The algorithm is unique in the sense that  the actual mathematical expressions, that are prohibitively large, need not be explicitly obtained. The diversity gain due to multiple relays is shown through plots of the analytical BER, well supported by simulations. 
%
%\end{abstract}
% IEEEtran.cls defaults to using nonbold math in the Abstract.
% This preserves the distinction between vectors and scalars. However,
% if the journal you are submitting to favors bold math in the abstract,
% then you can use LaTeX's standard command \boldmath at the very start
% of the abstract to achieve this. Many IEEE journals frown on math
% in the abstract anyway.

% Note that keywords are not normally used for peerreview papers.
%\begin{IEEEkeywords}
%Cooperative diversity, decode and forward, piecewise linear
%\end{IEEEkeywords}



% For peer review papers, you can put extra information on the cover
% page as needed:
% \ifCLASSOPTIONpeerreview
% \begin{center} \bfseries EDICS Category: 3-BBND \end{center}
% \fi
%
% For peerreview papers, this IEEEtran command inserts a page break and
% creates the second title. It will be ignored for other modes.
%\IEEEpeerreviewmaketitle




	\item All the jacks, queens and kings are removed from a deck of 52 playing cards. The remaining cards are well shuffled and then one card is drawn at random. Giving ace a value 1 similar value for other cards, find the probability that the card has a value 
		\begin{enumerate}
			\item 7
			\item greater than 7
			\item less than 7
		\end{enumerate}
		%Number of cards left after removing all jacks, queens and kings 
\begin{align}
N	= 52 - 4\times 3
	= 40
\end{align}
%\begin{table}[H]
%\def\arraystretch{1.2}
%\begin{tabular}{|c|c|c|}
%\hline
%	\textbf{Parameter} &\textbf{Value} &\textbf{Description}\\ \hline
%	$X$ &1-10 &Represents the value of the card picked \\ \hline
%\end{tabular}
%\end{table}
Let $1 \le X \le 10$ be the value of the card picked.  Then,
\begin{align}
	p_X(k) &= \Pr(X=k)\ \forall\ 1 \leq k \leq 10\\
	&= \frac{4\times 1}{40}\\
	&= \frac{1}{10}\\
	\therefore p_X(k) &= 
	\begin{cases}
		\frac{1}{10} & 1 \leq k \leq 10\\
		0 & \text{otherwise}
	\end{cases}
\end{align}
and
\begin{align}
	F_{X}(k) &= \sum_{m=0}^{k}p_{X}(m) \quad 1 \leq k \leq 10\\
	&= \frac{k}{10}\\
	\therefore F_{X}(k) &= 
	\begin{cases}
		0 & k \leq 0\\
		\frac{k}{10} & 1\leq k \leq 10\\
		1 & k > 10 
	\end{cases}
\end{align}
\begin{enumerate}
	\item Probability that card has value equal to 7 is
		\begin{align}
			 p_{X}(7)
			= \frac{1}{10}
		\end{align}
	\item Probability that card has value greater than 7 is
		\begin{align}
			1 - F_X(7)
			&= 1 - \frac{7}{10}
			\\
			&= \frac{3}{10}
		\end{align}
	\item Probability that card has value less than 7 is
		\begin{align}
			 F_{X}(6)
			=\frac{6}{10}
		\end{align}
\end{enumerate}

  \item A Lot consists of 48 mobile phones of which 42 are good, 3 have only minor defects and 3 have major defects.Varnika will buy a phone if it is good but the trader will only buy a mobile if it has no major defects. One phone is selected at random from the lot. What is the probability that it is
\begin{enumerate}
	\item acceptable to Varnika?
            \item acceptable to the trader?
\end{enumerate}
\solution
	%\begin{table}[H]
	\centering
\begin{tabular}{|c|c|c|}
\hline
Random variable &Value &Definition\\ \hline
\multirow{3}{*}{X} &0 &Slips of Rs 1\\
&1 &Slips of Rs 5\\
&2 &Slips of Rs 13\\ \hline
\multirow{2}{*}{Y} &0 &Box A\\
&1 &Box B\\\hline
\end{tabular}
\caption{}
\label{tab:Distribution}
\end{table}
See \tabref{tab:Distribution}.
\begin{align}
p_{Y}\brak{k}= \begin{cases} 
      \frac{1}{3} & {k=0} \\
      \frac{2}{3 }& {k=1} 
   \end{cases}
   \\
p_{Y|X}\brak{0|0} = \frac{19}{25}\, 
p_{Y|X}\brak{0|1} = \frac{6}{25}\,
p_{Y|X}\brak{1|0} = \frac{45}{50}\,
p_{Y|X}\brak{1|2} = \frac{5}{50}
\end{align}
The desired probability is the probability that a slip drawn at random is marked other than Rs 1,
\begin{align}
&=1-p_X\brak{0}\\
&= p_X(1) + p_X(2)
\end{align}
Using Bayes theorem,
\begin{align}
&= p_Y\brak{0} \times \pr{Y=0 | X=1} + p_Y\brak{1} \times \pr{Y=1|X=2}\\
&=\frac{1}{3} \times \frac{6}{25} + \frac{2}{3} \times \frac{5}{50}\\
&=\frac{11}{75}
\end{align}

\newpage

%\tableofcontents

\bigskip

\renewcommand{\thefigure}{\theenumi}
\renewcommand{\thetable}{\theenumi}
%\renewcommand{\theequation}{\theenumi}

%\begin{abstract}
%%\boldmath
%In this letter, an algorithm for evaluating the exact analytical bit error rate  (BER)  for the piecewise linear (PL) combiner for  multiple relays is presented. Previous results were available only for upto three relays. The algorithm is unique in the sense that  the actual mathematical expressions, that are prohibitively large, need not be explicitly obtained. The diversity gain due to multiple relays is shown through plots of the analytical BER, well supported by simulations. 
%
%\end{abstract}
% IEEEtran.cls defaults to using nonbold math in the Abstract.
% This preserves the distinction between vectors and scalars. However,
% if the journal you are submitting to favors bold math in the abstract,
% then you can use LaTeX's standard command \boldmath at the very start
% of the abstract to achieve this. Many IEEE journals frown on math
% in the abstract anyway.

% Note that keywords are not normally used for peerreview papers.
%\begin{IEEEkeywords}
%Cooperative diversity, decode and forward, piecewise linear
%\end{IEEEkeywords}



% For peer review papers, you can put extra information on the cover
% page as needed:
% \ifCLASSOPTIONpeerreview
% \begin{center} \bfseries EDICS Category: 3-BBND \end{center}
% \fi
%
% For peerreview papers, this IEEEtran command inserts a page break and
% creates the second title. It will be ignored for other modes.
%\IEEEpeerreviewmaketitle




 \item A student says that if you throw a die, it will show up 1 or not 1. Therefore, the probability of getting 1 and the probability of getting 'not 1' each is equal to $\frac{1}{2}$. Is this correct? Give reasons.\\
 \solution
        %\begin{table}[H]
	\centering
\begin{tabular}{|c|c|c|}
\hline
Random variable &Value &Definition\\ \hline
\multirow{3}{*}{X} &0 &Slips of Rs 1\\
&1 &Slips of Rs 5\\
&2 &Slips of Rs 13\\ \hline
\multirow{2}{*}{Y} &0 &Box A\\
&1 &Box B\\\hline
\end{tabular}
\caption{}
\label{tab:Distribution}
\end{table}
See \tabref{tab:Distribution}.
\begin{align}
p_{Y}\brak{k}= \begin{cases} 
      \frac{1}{3} & {k=0} \\
      \frac{2}{3 }& {k=1} 
   \end{cases}
   \\
p_{Y|X}\brak{0|0} = \frac{19}{25}\, 
p_{Y|X}\brak{0|1} = \frac{6}{25}\,
p_{Y|X}\brak{1|0} = \frac{45}{50}\,
p_{Y|X}\brak{1|2} = \frac{5}{50}
\end{align}
The desired probability is the probability that a slip drawn at random is marked other than Rs 1,
\begin{align}
&=1-p_X\brak{0}\\
&= p_X(1) + p_X(2)
\end{align}
Using Bayes theorem,
\begin{align}
&= p_Y\brak{0} \times \pr{Y=0 | X=1} + p_Y\brak{1} \times \pr{Y=1|X=2}\\
&=\frac{1}{3} \times \frac{6}{25} + \frac{2}{3} \times \frac{5}{50}\\
&=\frac{11}{75}
\end{align}

\newpage

%\tableofcontents

\bigskip

\renewcommand{\thefigure}{\theenumi}
\renewcommand{\thetable}{\theenumi}
%\renewcommand{\theequation}{\theenumi}

%\begin{abstract}
%%\boldmath
%In this letter, an algorithm for evaluating the exact analytical bit error rate  (BER)  for the piecewise linear (PL) combiner for  multiple relays is presented. Previous results were available only for upto three relays. The algorithm is unique in the sense that  the actual mathematical expressions, that are prohibitively large, need not be explicitly obtained. The diversity gain due to multiple relays is shown through plots of the analytical BER, well supported by simulations. 
%
%\end{abstract}
% IEEEtran.cls defaults to using nonbold math in the Abstract.
% This preserves the distinction between vectors and scalars. However,
% if the journal you are submitting to favors bold math in the abstract,
% then you can use LaTeX's standard command \boldmath at the very start
% of the abstract to achieve this. Many IEEE journals frown on math
% in the abstract anyway.

% Note that keywords are not normally used for peerreview papers.
%\begin{IEEEkeywords}
%Cooperative diversity, decode and forward, piecewise linear
%\end{IEEEkeywords}



% For peer review papers, you can put extra information on the cover
% page as needed:
% \ifCLASSOPTIONpeerreview
% \begin{center} \bfseries EDICS Category: 3-BBND \end{center}
% \fi
%
% For peerreview papers, this IEEEtran command inserts a page break and
% creates the second title. It will be ignored for other modes.
%\IEEEpeerreviewmaketitle




   \item Four candidates A, B, C, D have ap-
plied for the assignment to coach a school cricket
team. If A is twice as likely to be selected as B, and
B and C are given about the same chance of being
selected, while C is twice as likely to be selected
as D, what are the probabilities that
\begin{enumerate}
\item C will be selected?
\item A will not be selected?
\end{enumerate}
	%\begin{table}[H]
	\centering
\begin{tabular}{|c|c|c|}
\hline
Random variable &Value &Definition\\ \hline
\multirow{3}{*}{X} &0 &Slips of Rs 1\\
&1 &Slips of Rs 5\\
&2 &Slips of Rs 13\\ \hline
\multirow{2}{*}{Y} &0 &Box A\\
&1 &Box B\\\hline
\end{tabular}
\caption{}
\label{tab:Distribution}
\end{table}
See \tabref{tab:Distribution}.
\begin{align}
p_{Y}\brak{k}= \begin{cases} 
      \frac{1}{3} & {k=0} \\
      \frac{2}{3 }& {k=1} 
   \end{cases}
   \\
p_{Y|X}\brak{0|0} = \frac{19}{25}\, 
p_{Y|X}\brak{0|1} = \frac{6}{25}\,
p_{Y|X}\brak{1|0} = \frac{45}{50}\,
p_{Y|X}\brak{1|2} = \frac{5}{50}
\end{align}
The desired probability is the probability that a slip drawn at random is marked other than Rs 1,
\begin{align}
&=1-p_X\brak{0}\\
&= p_X(1) + p_X(2)
\end{align}
Using Bayes theorem,
\begin{align}
&= p_Y\brak{0} \times \pr{Y=0 | X=1} + p_Y\brak{1} \times \pr{Y=1|X=2}\\
&=\frac{1}{3} \times \frac{6}{25} + \frac{2}{3} \times \frac{5}{50}\\
&=\frac{11}{75}
\end{align}

\newpage

%\tableofcontents

\bigskip

\renewcommand{\thefigure}{\theenumi}
\renewcommand{\thetable}{\theenumi}
%\renewcommand{\theequation}{\theenumi}

%\begin{abstract}
%%\boldmath
%In this letter, an algorithm for evaluating the exact analytical bit error rate  (BER)  for the piecewise linear (PL) combiner for  multiple relays is presented. Previous results were available only for upto three relays. The algorithm is unique in the sense that  the actual mathematical expressions, that are prohibitively large, need not be explicitly obtained. The diversity gain due to multiple relays is shown through plots of the analytical BER, well supported by simulations. 
%
%\end{abstract}
% IEEEtran.cls defaults to using nonbold math in the Abstract.
% This preserves the distinction between vectors and scalars. However,
% if the journal you are submitting to favors bold math in the abstract,
% then you can use LaTeX's standard command \boldmath at the very start
% of the abstract to achieve this. Many IEEE journals frown on math
% in the abstract anyway.

% Note that keywords are not normally used for peerreview papers.
%\begin{IEEEkeywords}
%Cooperative diversity, decode and forward, piecewise linear
%\end{IEEEkeywords}



% For peer review papers, you can put extra information on the cover
% page as needed:
% \ifCLASSOPTIONpeerreview
% \begin{center} \bfseries EDICS Category: 3-BBND \end{center}
% \fi
%
% For peerreview papers, this IEEEtran command inserts a page break and
% creates the second title. It will be ignored for other modes.
%\IEEEpeerreviewmaketitle




 \item A bag contain 24 balls of which $x$ balls are red, $2x$ are white and $3x$ are blue. A ball is selected at random, What is the probability that it is
\begin{enumerate}[label=\alph*)]
\item not red ?
\item white ?
\end{enumerate}
%\begin{table}[H]
	\centering
\begin{tabular}{|c|c|c|}
\hline
Random variable &Value &Definition\\ \hline
\multirow{3}{*}{X} &0 &Slips of Rs 1\\
&1 &Slips of Rs 5\\
&2 &Slips of Rs 13\\ \hline
\multirow{2}{*}{Y} &0 &Box A\\
&1 &Box B\\\hline
\end{tabular}
\caption{}
\label{tab:Distribution}
\end{table}
See \tabref{tab:Distribution}.
\begin{align}
p_{Y}\brak{k}= \begin{cases} 
      \frac{1}{3} & {k=0} \\
      \frac{2}{3 }& {k=1} 
   \end{cases}
   \\
p_{Y|X}\brak{0|0} = \frac{19}{25}\, 
p_{Y|X}\brak{0|1} = \frac{6}{25}\,
p_{Y|X}\brak{1|0} = \frac{45}{50}\,
p_{Y|X}\brak{1|2} = \frac{5}{50}
\end{align}
The desired probability is the probability that a slip drawn at random is marked other than Rs 1,
\begin{align}
&=1-p_X\brak{0}\\
&= p_X(1) + p_X(2)
\end{align}
Using Bayes theorem,
\begin{align}
&= p_Y\brak{0} \times \pr{Y=0 | X=1} + p_Y\brak{1} \times \pr{Y=1|X=2}\\
&=\frac{1}{3} \times \frac{6}{25} + \frac{2}{3} \times \frac{5}{50}\\
&=\frac{11}{75}
\end{align}

\newpage

%\tableofcontents

\bigskip

\renewcommand{\thefigure}{\theenumi}
\renewcommand{\thetable}{\theenumi}
%\renewcommand{\theequation}{\theenumi}

%\begin{abstract}
%%\boldmath
%In this letter, an algorithm for evaluating the exact analytical bit error rate  (BER)  for the piecewise linear (PL) combiner for  multiple relays is presented. Previous results were available only for upto three relays. The algorithm is unique in the sense that  the actual mathematical expressions, that are prohibitively large, need not be explicitly obtained. The diversity gain due to multiple relays is shown through plots of the analytical BER, well supported by simulations. 
%
%\end{abstract}
% IEEEtran.cls defaults to using nonbold math in the Abstract.
% This preserves the distinction between vectors and scalars. However,
% if the journal you are submitting to favors bold math in the abstract,
% then you can use LaTeX's standard command \boldmath at the very start
% of the abstract to achieve this. Many IEEE journals frown on math
% in the abstract anyway.

% Note that keywords are not normally used for peerreview papers.
%\begin{IEEEkeywords}
%Cooperative diversity, decode and forward, piecewise linear
%\end{IEEEkeywords}



% For peer review papers, you can put extra information on the cover
% page as needed:
% \ifCLASSOPTIONpeerreview
% \begin{center} \bfseries EDICS Category: 3-BBND \end{center}
% \fi
%
% For peerreview papers, this IEEEtran command inserts a page break and
% creates the second title. It will be ignored for other modes.
%\IEEEpeerreviewmaketitle




If the letters of the word ASSASSINATION are arranged at random. Find the Probability that
\begin{enumerate}[label=(\alph*)]
\item Four $S's$ come consecutively in the word
\item Two  $I's$ and two $N's$ come together
\item All $A's$ are not coming together
\item No two $A's$ are coming together
\end{enumerate}
%\begin{table}[H]
	\centering
\begin{tabular}{|c|c|c|}
\hline
Random variable &Value &Definition\\ \hline
\multirow{3}{*}{X} &0 &Slips of Rs 1\\
&1 &Slips of Rs 5\\
&2 &Slips of Rs 13\\ \hline
\multirow{2}{*}{Y} &0 &Box A\\
&1 &Box B\\\hline
\end{tabular}
\caption{}
\label{tab:Distribution}
\end{table}
See \tabref{tab:Distribution}.
\begin{align}
p_{Y}\brak{k}= \begin{cases} 
      \frac{1}{3} & {k=0} \\
      \frac{2}{3 }& {k=1} 
   \end{cases}
   \\
p_{Y|X}\brak{0|0} = \frac{19}{25}\, 
p_{Y|X}\brak{0|1} = \frac{6}{25}\,
p_{Y|X}\brak{1|0} = \frac{45}{50}\,
p_{Y|X}\brak{1|2} = \frac{5}{50}
\end{align}
The desired probability is the probability that a slip drawn at random is marked other than Rs 1,
\begin{align}
&=1-p_X\brak{0}\\
&= p_X(1) + p_X(2)
\end{align}
Using Bayes theorem,
\begin{align}
&= p_Y\brak{0} \times \pr{Y=0 | X=1} + p_Y\brak{1} \times \pr{Y=1|X=2}\\
&=\frac{1}{3} \times \frac{6}{25} + \frac{2}{3} \times \frac{5}{50}\\
&=\frac{11}{75}
\end{align}

\newpage

%\tableofcontents

\bigskip

\renewcommand{\thefigure}{\theenumi}
\renewcommand{\thetable}{\theenumi}
%\renewcommand{\theequation}{\theenumi}

%\begin{abstract}
%%\boldmath
%In this letter, an algorithm for evaluating the exact analytical bit error rate  (BER)  for the piecewise linear (PL) combiner for  multiple relays is presented. Previous results were available only for upto three relays. The algorithm is unique in the sense that  the actual mathematical expressions, that are prohibitively large, need not be explicitly obtained. The diversity gain due to multiple relays is shown through plots of the analytical BER, well supported by simulations. 
%
%\end{abstract}
% IEEEtran.cls defaults to using nonbold math in the Abstract.
% This preserves the distinction between vectors and scalars. However,
% if the journal you are submitting to favors bold math in the abstract,
% then you can use LaTeX's standard command \boldmath at the very start
% of the abstract to achieve this. Many IEEE journals frown on math
% in the abstract anyway.

% Note that keywords are not normally used for peerreview papers.
%\begin{IEEEkeywords}
%Cooperative diversity, decode and forward, piecewise linear
%\end{IEEEkeywords}



% For peer review papers, you can put extra information on the cover
% page as needed:
% \ifCLASSOPTIONpeerreview
% \begin{center} \bfseries EDICS Category: 3-BBND \end{center}
% \fi
%
% For peerreview papers, this IEEEtran command inserts a page break and
% creates the second title. It will be ignored for other modes.
%\IEEEpeerreviewmaketitle




	\item One urn contains two black balls (labelled B1 and B2) and one white ball. A
	second urn contains one black ball and two white balls (labelled W1 and W2).
	Suppose the following experiment is performed. One of the two urns is chosen
	at random. Next a ball is randomly chosen from the urn. Then a second ball is
	chosen at random from the same urn without replacing the first ball.
	
	\begin{enumerate}
	\item What is the probability that two black balls are chosen?
	
	\item What is the probability that two balls of opposite colour are chosen?
	\end{enumerate}
	\solution
	%\begin{align}
    \label{eq:12.13.6.18.1}
	\because	\pr{A|B} &> \pr{A},\
\frac{\pr{AB}}{\pr{B}} > \pr{A}
\\
    \label{eq:12.13.6.18.2}
	\implies \pr{AB} &> \pr{A}\pr{B}
	\\
	\text{or, } \frac{\pr{AB}}{\pr{A}} &=\pr{B|A} > \pr{A}
\end{align}

\end{enumerate}

	\item A bag contains $5$ red balls and some blue balls. If the probability of drawing a blue ball is double that if a red ball, determine the number of blue balls in the bag. 
		\\
\solution
		%\begin{enumerate}[label=\thesection.\arabic*,ref=\thesection.\theenumi]
	\item One card is drawn from a well-shuffled deck of 52 cards. Find the probability of getting
\begin{enumerate}
\item A king of red colour 
\item A face card 
\item A red face card
\item The jack of hearts
\item A spade
\item The queen of diamonds

\end{enumerate}
\solution
		%\begin{table}[H]
	\centering
\begin{tabular}{|c|c|c|}
\hline
Random variable &Value &Definition\\ \hline
\multirow{3}{*}{X} &0 &Slips of Rs 1\\
&1 &Slips of Rs 5\\
&2 &Slips of Rs 13\\ \hline
\multirow{2}{*}{Y} &0 &Box A\\
&1 &Box B\\\hline
\end{tabular}
\caption{}
\label{tab:Distribution}
\end{table}
See \tabref{tab:Distribution}.
\begin{align}
p_{Y}\brak{k}= \begin{cases} 
      \frac{1}{3} & {k=0} \\
      \frac{2}{3 }& {k=1} 
   \end{cases}
   \\
p_{Y|X}\brak{0|0} = \frac{19}{25}\, 
p_{Y|X}\brak{0|1} = \frac{6}{25}\,
p_{Y|X}\brak{1|0} = \frac{45}{50}\,
p_{Y|X}\brak{1|2} = \frac{5}{50}
\end{align}
The desired probability is the probability that a slip drawn at random is marked other than Rs 1,
\begin{align}
&=1-p_X\brak{0}\\
&= p_X(1) + p_X(2)
\end{align}
Using Bayes theorem,
\begin{align}
&= p_Y\brak{0} \times \pr{Y=0 | X=1} + p_Y\brak{1} \times \pr{Y=1|X=2}\\
&=\frac{1}{3} \times \frac{6}{25} + \frac{2}{3} \times \frac{5}{50}\\
&=\frac{11}{75}
\end{align}

\newpage

%\tableofcontents

\bigskip

\renewcommand{\thefigure}{\theenumi}
\renewcommand{\thetable}{\theenumi}
%\renewcommand{\theequation}{\theenumi}

%\begin{abstract}
%%\boldmath
%In this letter, an algorithm for evaluating the exact analytical bit error rate  (BER)  for the piecewise linear (PL) combiner for  multiple relays is presented. Previous results were available only for upto three relays. The algorithm is unique in the sense that  the actual mathematical expressions, that are prohibitively large, need not be explicitly obtained. The diversity gain due to multiple relays is shown through plots of the analytical BER, well supported by simulations. 
%
%\end{abstract}
% IEEEtran.cls defaults to using nonbold math in the Abstract.
% This preserves the distinction between vectors and scalars. However,
% if the journal you are submitting to favors bold math in the abstract,
% then you can use LaTeX's standard command \boldmath at the very start
% of the abstract to achieve this. Many IEEE journals frown on math
% in the abstract anyway.

% Note that keywords are not normally used for peerreview papers.
%\begin{IEEEkeywords}
%Cooperative diversity, decode and forward, piecewise linear
%\end{IEEEkeywords}



% For peer review papers, you can put extra information on the cover
% page as needed:
% \ifCLASSOPTIONpeerreview
% \begin{center} \bfseries EDICS Category: 3-BBND \end{center}
% \fi
%
% For peerreview papers, this IEEEtran command inserts a page break and
% creates the second title. It will be ignored for other modes.
%\IEEEpeerreviewmaketitle




	\item Five cards—the ten, jack, queen, king and ace of diamonds, are well-shuffled with their face downwards. One card is then picked up at random.
\begin{enumerate}
\item
What is the probability that the card is the queen? 
\item
If the queen is drawn and put aside, what is the probability that the second card picked up is (a) an ace? (b) a queen?\\
\end{enumerate}
\solution
		%\begin{enumerate}[label=\thesection.\arabic*,ref=\thesection.\theenumi]
	\item One card is drawn from a well-shuffled deck of 52 cards. Find the probability of getting
\begin{enumerate}
\item A king of red colour 
\item A face card 
\item A red face card
\item The jack of hearts
\item A spade
\item The queen of diamonds

\end{enumerate}
\solution
		%\input{ncert/10/15/1/14/main.tex}
	\item Five cards—the ten, jack, queen, king and ace of diamonds, are well-shuffled with their face downwards. One card is then picked up at random.
\begin{enumerate}
\item
What is the probability that the card is the queen? 
\item
If the queen is drawn and put aside, what is the probability that the second card picked up is (a) an ace? (b) a queen?\\
\end{enumerate}
\solution
		%\input{ncert/10/15/1/15/defs.tex}
	\item A bag contains $5$ red balls and some blue balls. If the probability of drawing a blue ball is double that if a red ball, determine the number of blue balls in the bag. 
		\\
\solution
		%\input{ncert/10/15/2/3/defs.tex}
	\item A card is selected from a pack of 52 cards.
 \begin{enumerate}[label=(\alph*)] 
                 \item How many points are there in the sample space?
                 \item Calculate the probability that the card is an ace of spades.
                 \item Calculate the probability that the card is (i) an ace and (ii) black card.
 \end{enumerate}
\solution
		%\input{ncert/11/16/3/4/main.tex}
\item Four cards are drawn from a well-shuffled deck of 52 cards. What is the probability of obtaining 3 diamonds and one spade.
\\
\solution
		%\input{ncert/11/16/4/2/defs.tex}
\item In a certain lottery 10,000 tickets are sold and ten equal prizes are awarded. What is the probability of not getting a prize if you buy (a) one ticket (b) two tickets (c) 10 tickets ?	
\\
\solution
		%\input{ncert/11/16/4/4/defs.tex}
		%
\item 
Out of 100 students, two sections of 40 and 60 are formed. If you and your friend are among the 100 students, what is the probability that
\begin{enumerate}
\item you both enter the same section?
\item you both enter the different sections?
\end{enumerate}
\solution
		%\input{ncert/11/16/4/5/defs.tex}
	\item 
The number lock of a suitcase has 4 wheels each labelled with ten digits i.e. from 0 to 9.The lock opens with a sequence of four digits with no repeats.What is the probability of a person getting the right sequence to open the suitcase.
\\
\solution
		%\input{ncert/11/16/4/10/defs.tex}
		%
\item 
Two cards are drawn at random and without replacement from a pack of 52 playing cards. Find the probability that both the cards are black.
\\
\solution
		%\input{ncert/12/13/2/2/defs.tex}
		\item A box of oranges is inspected by examining three randomly selected oranges drawn without replacement. If all the three oranges are good, the box is approved for sale, otherwise, it is rejected. Find the probability that a box containing 15 oranges out of which 12 are good and 3 are bad ones will be approved for sale.
		\label{ncert/12/13/2/3/defs.tex}
		\item Two balls are drawn at random with replacement from a box containing 10 black and 8 red balls. Find the probability that
		\label{ncert/12/13/2/12}
\begin{enumerate}
\item both balls are red.
\item first ball is black and second is red.
\item one of them is black and other is red.
\end{enumerate}

\item In a hostel, 60\% of the students read Hindi newspaper, 40\% read English newspaper and 20\% read both Hindi and English newspapers. A student is selected at random.
		\label{ncert/12/13/2/15}
\begin{enumerate}
\item Find the probability that she reads neither Hindi nor English newspapers.
\item If she reads Hindi newspaper, find the probability that she reads English newspaper.
\item If she reads English newspaper, find the probability that she reads Hindi newspaper.\\
\end{enumerate}
\item The probability of obtaining an even prime number on each die, when a pair of dice is rolled is 
\begin{enumerate}
    \item $0$ 
    
    \item $\frac{1}{3}$ 
    
    \item $\frac{1}{12}$ 
    
    \item $\frac{1}{36}$ 
\end{enumerate}
\solution
		%\input{ncert/12/13/2/17/defs.tex}
	\item A bag contains 4 red and 4 black balls, another bag contains 2 red and 6 black balls. One of the two bags is selected at random and a ball is drawn from the bag which is found to be red. Find the probability that the ball is drawn from the first bag.
\\
\solution
		%\input{ncert/12/13/3/2/main.tex}
  \item
  Cards with numbers 2 to 101 are placed in a box. A card is selected at random.Find the probability that the card has
\begin{enumerate}[label=(\roman*)]
	\item an even number 
	\item a square number
\end{enumerate}
\solution
%\input{exemplar/10/13/3/32/main.tex}
\item
The king, queen and jack of clubs are removed from a deck of 52 playing cards and then well shuffled. Now one card is drawn at random from the remaining cards.  Determine the probability that the card is
\begin{enumerate}[label=(\roman*)]
\item a club
\item 10 of hearts
\end{enumerate}
\solution
%\input{exemplar/10/13/3/29/main.tex}
\item A team of medical students doing their internship have to assist during surgeries
at a city hospital. The probabilities of surgeries rated as very complex, complex,
routine, simple or very simple are respectively, 0.15, 0.20, 0.31, 0.26, .08. Find
the probabilities that a particular surgery will be rated
\begin{enumerate}
	\item complex or very complex;
	\item neither very complex nor very simple;
	\item routine or complex
	\item routine or simple
\end{enumerate}
\solution
%\input{exemplar/11/16/3/8(1)/main.tex}
\item A card is selected from a pack of 52 cards.
\begin{enumerate}[label=(\alph*)]
    \item How many points are there in the sample space?
    \item Calculate the probability that the card is an ace of spades.
    \item Calculate the probability that the card is (i) an ace and (ii) black card.
\end{enumerate}
\solution
%\input{exemplar/11/16/3/4/main2.tex}
\item The probability that a non leap year selected at random will contain 53 sundays.
\\
\solution
%\input{exemplar/10/13/1/19/main.tex}
\item One of the four persons John, Rita, Aslam or Gurpreet will be promoted next
month. Consequently the sample space consists of four elementary outcomes
S = {John promoted, Rita promoted, Aslam promoted, Gurpreet promoted}
You are told that the chances of John’s promotion is same as that of Gurpreet,
Rita’s chances of promotion are twice as likely as Johns. Aslam’s chances are
four times that of John.
\begin{enumerate}
	\item Determine
	\begin{enumerate}
		\item P (John promoted)
		\item P (Rita promoted)
		\item P (Aslam promoted)
		\item P (Gurpreet promoted)
	\end{enumerate}
	\item If A = {John promoted or Gurpreet promoted}, find P (A).
\end{enumerate}
\solution
%\input{exemplar/11/16/3/10/main.tex}
\item A card is drawn from a deck of 52 cards. Find the probability of getting a king or a heart or a red card.\\
\solution
%\input{exemplar/11/16/3/15/main.tex}
\item The probability that a student will pass his examination is 0.73, the probability of
the student getting a compartment is 0.13, and the probability that the student will
either pass or get compartment is 0.96. State True or False.\\
\solution
%\input{exemplar/11/16/3/31/main.tex}
\item A card is selected from a pack of 52 cards\\
\begin{enumerate}[label=(\alph*)]
\item How many points are there in the sample space?
\item Calculate the probability that the cards is an ace of spades.
\item Calculate the probability that the card is (i) an ace (ii)black card.\\
\end{enumerate}
%\input{ncert/11/16/3/4_1/Prob_4.tex}
\item In a non-leap year, the probability of having 53 tuesdays or 53 wednesdays is\\
\solution
%\input{exemplar/11/16/3/18/main.tex}
\item There are 1000 sealed envelopes in a box, 10 of them contain a cash prize of
Rs 100 each, 100 of them contain a cash prize of Rs 50 each and 200 of them
contain a cash prize of Rs 10 each and rest do not contain any cash prize. If they
are well shuffled and an envelope is picked up out, what is the probability that it
contains no cash prize?\\
\solution
%\input{exemplar/10/13/3/34/main.tex}
\item 
A die is thrown and a card is selected at random from a deck of 52 playing cards. The probability of getting an even number on the die and a spade card.\\
\solution
%\input{exemplar/12/13/3/78/main.tex}
\item
If 4-digit numbers greater than 5,000 are randomly formed from the digits 0, 1, 3, 5, and 7, what is the probability of forming a number divisible by 5 when:
\begin{enumerate}
    \item The digits are repeated?
    \item The repetition of digits is not allowed?
\end{enumerate}
\solution
%\input{ncert/11/16/4/9/main.tex}
\item Consider the probability space $\brak{\Omega, \mathcal{G}, P}$ where $\Omega = [0,2]$ and $\mathcal{G} = \cbrak{\phi, \Omega, [0,1], (1,2]}$. Let $X$ and $Y$ be two functions on $\Omega$ defined as
\begin{align*}
    X(\omega) = 
    \begin{cases}
        1 & \text{if }\omega \in [0, 1]\\
        2 & \text{if }\omega \in (1, 2]
    \end{cases}
\end{align*}
and
\begin{align*}
    Y(\omega) = 
    \begin{cases}
        2 & \text{if }\omega \in [0, 1.5]\\
        3 & \text{if }\omega \in (1.5, 2].
    \end{cases}
\end{align*}
Then which one of the following statements is true?
\begin{enumerate}
    \item [(A)] $X$ is a random variable with respect to $\mathcal{G}$, but $Y$ is not a random variable with respect to $\mathcal{G}$.
    \item [(B)] $Y$ is a random variable with respect to $\mathcal{G}$, but $X$ is not a random variable with respect to $\mathcal{G}$.
    \item [(C)] Neither $X$ nor $Y$ is a random variable with respect to $\mathcal{G}$.
    \item [(D)] Both $X$ and $Y$ are random variables with respect to $\mathcal{G}$.
\end{enumerate} \hfill (GATE ST 2023)\\
\solution
%\input{gate/ST/2023/14/main.tex}
	\item  A die is loaded in such a way that each odd number is twice as likely to occur as
each even number. Find $P(G)$, where $G$ is the event that a number greater than
3 occurs on a single roll of the die.
\\
\solution
		%\input{exemplar/11/16/3/5/main.tex}
	\item All the jacks, queens and kings are removed from a deck of 52 playing cards. The remaining cards are well shuffled and then one card is drawn at random. Giving ace a value 1 similar value for other cards, find the probability that the card has a value 
		\begin{enumerate}
			\item 7
			\item greater than 7
			\item less than 7
		\end{enumerate}
		%\input{exemplar/10/13/3/30/main.tex}
  \item A Lot consists of 48 mobile phones of which 42 are good, 3 have only minor defects and 3 have major defects.Varnika will buy a phone if it is good but the trader will only buy a mobile if it has no major defects. One phone is selected at random from the lot. What is the probability that it is
\begin{enumerate}
	\item acceptable to Varnika?
            \item acceptable to the trader?
\end{enumerate}
\solution
	%\input{exemplar/10/13/3/40/main.tex}
 \item A student says that if you throw a die, it will show up 1 or not 1. Therefore, the probability of getting 1 and the probability of getting 'not 1' each is equal to $\frac{1}{2}$. Is this correct? Give reasons.\\
 \solution
        %\input{exemplar/10/13/2/9/main.tex}
   \item Four candidates A, B, C, D have ap-
plied for the assignment to coach a school cricket
team. If A is twice as likely to be selected as B, and
B and C are given about the same chance of being
selected, while C is twice as likely to be selected
as D, what are the probabilities that
\begin{enumerate}
\item C will be selected?
\item A will not be selected?
\end{enumerate}
	%\input{exemplar/11/16/3/9/main.tex}
 \item A bag contain 24 balls of which $x$ balls are red, $2x$ are white and $3x$ are blue. A ball is selected at random, What is the probability that it is
\begin{enumerate}[label=\alph*)]
\item not red ?
\item white ?
\end{enumerate}
%\input{exemplar/10/13/3/41/main.tex}
If the letters of the word ASSASSINATION are arranged at random. Find the Probability that
\begin{enumerate}[label=(\alph*)]
\item Four $S's$ come consecutively in the word
\item Two  $I's$ and two $N's$ come together
\item All $A's$ are not coming together
\item No two $A's$ are coming together
\end{enumerate}
%\input{exemplar/11/16/3/14/main.tex}
	\item One urn contains two black balls (labelled B1 and B2) and one white ball. A
	second urn contains one black ball and two white balls (labelled W1 and W2).
	Suppose the following experiment is performed. One of the two urns is chosen
	at random. Next a ball is randomly chosen from the urn. Then a second ball is
	chosen at random from the same urn without replacing the first ball.
	
	\begin{enumerate}
	\item What is the probability that two black balls are chosen?
	
	\item What is the probability that two balls of opposite colour are chosen?
	\end{enumerate}
	\solution
	%\input{exemplar/11/16/3/12/main1.tex}
\end{enumerate}

	\item A bag contains $5$ red balls and some blue balls. If the probability of drawing a blue ball is double that if a red ball, determine the number of blue balls in the bag. 
		\\
\solution
		%\begin{enumerate}[label=\thesection.\arabic*,ref=\thesection.\theenumi]
	\item One card is drawn from a well-shuffled deck of 52 cards. Find the probability of getting
\begin{enumerate}
\item A king of red colour 
\item A face card 
\item A red face card
\item The jack of hearts
\item A spade
\item The queen of diamonds

\end{enumerate}
\solution
		%\input{ncert/10/15/1/14/main.tex}
	\item Five cards—the ten, jack, queen, king and ace of diamonds, are well-shuffled with their face downwards. One card is then picked up at random.
\begin{enumerate}
\item
What is the probability that the card is the queen? 
\item
If the queen is drawn and put aside, what is the probability that the second card picked up is (a) an ace? (b) a queen?\\
\end{enumerate}
\solution
		%\input{ncert/10/15/1/15/defs.tex}
	\item A bag contains $5$ red balls and some blue balls. If the probability of drawing a blue ball is double that if a red ball, determine the number of blue balls in the bag. 
		\\
\solution
		%\input{ncert/10/15/2/3/defs.tex}
	\item A card is selected from a pack of 52 cards.
 \begin{enumerate}[label=(\alph*)] 
                 \item How many points are there in the sample space?
                 \item Calculate the probability that the card is an ace of spades.
                 \item Calculate the probability that the card is (i) an ace and (ii) black card.
 \end{enumerate}
\solution
		%\input{ncert/11/16/3/4/main.tex}
\item Four cards are drawn from a well-shuffled deck of 52 cards. What is the probability of obtaining 3 diamonds and one spade.
\\
\solution
		%\input{ncert/11/16/4/2/defs.tex}
\item In a certain lottery 10,000 tickets are sold and ten equal prizes are awarded. What is the probability of not getting a prize if you buy (a) one ticket (b) two tickets (c) 10 tickets ?	
\\
\solution
		%\input{ncert/11/16/4/4/defs.tex}
		%
\item 
Out of 100 students, two sections of 40 and 60 are formed. If you and your friend are among the 100 students, what is the probability that
\begin{enumerate}
\item you both enter the same section?
\item you both enter the different sections?
\end{enumerate}
\solution
		%\input{ncert/11/16/4/5/defs.tex}
	\item 
The number lock of a suitcase has 4 wheels each labelled with ten digits i.e. from 0 to 9.The lock opens with a sequence of four digits with no repeats.What is the probability of a person getting the right sequence to open the suitcase.
\\
\solution
		%\input{ncert/11/16/4/10/defs.tex}
		%
\item 
Two cards are drawn at random and without replacement from a pack of 52 playing cards. Find the probability that both the cards are black.
\\
\solution
		%\input{ncert/12/13/2/2/defs.tex}
		\item A box of oranges is inspected by examining three randomly selected oranges drawn without replacement. If all the three oranges are good, the box is approved for sale, otherwise, it is rejected. Find the probability that a box containing 15 oranges out of which 12 are good and 3 are bad ones will be approved for sale.
		\label{ncert/12/13/2/3/defs.tex}
		\item Two balls are drawn at random with replacement from a box containing 10 black and 8 red balls. Find the probability that
		\label{ncert/12/13/2/12}
\begin{enumerate}
\item both balls are red.
\item first ball is black and second is red.
\item one of them is black and other is red.
\end{enumerate}

\item In a hostel, 60\% of the students read Hindi newspaper, 40\% read English newspaper and 20\% read both Hindi and English newspapers. A student is selected at random.
		\label{ncert/12/13/2/15}
\begin{enumerate}
\item Find the probability that she reads neither Hindi nor English newspapers.
\item If she reads Hindi newspaper, find the probability that she reads English newspaper.
\item If she reads English newspaper, find the probability that she reads Hindi newspaper.\\
\end{enumerate}
\item The probability of obtaining an even prime number on each die, when a pair of dice is rolled is 
\begin{enumerate}
    \item $0$ 
    
    \item $\frac{1}{3}$ 
    
    \item $\frac{1}{12}$ 
    
    \item $\frac{1}{36}$ 
\end{enumerate}
\solution
		%\input{ncert/12/13/2/17/defs.tex}
	\item A bag contains 4 red and 4 black balls, another bag contains 2 red and 6 black balls. One of the two bags is selected at random and a ball is drawn from the bag which is found to be red. Find the probability that the ball is drawn from the first bag.
\\
\solution
		%\input{ncert/12/13/3/2/main.tex}
  \item
  Cards with numbers 2 to 101 are placed in a box. A card is selected at random.Find the probability that the card has
\begin{enumerate}[label=(\roman*)]
	\item an even number 
	\item a square number
\end{enumerate}
\solution
%\input{exemplar/10/13/3/32/main.tex}
\item
The king, queen and jack of clubs are removed from a deck of 52 playing cards and then well shuffled. Now one card is drawn at random from the remaining cards.  Determine the probability that the card is
\begin{enumerate}[label=(\roman*)]
\item a club
\item 10 of hearts
\end{enumerate}
\solution
%\input{exemplar/10/13/3/29/main.tex}
\item A team of medical students doing their internship have to assist during surgeries
at a city hospital. The probabilities of surgeries rated as very complex, complex,
routine, simple or very simple are respectively, 0.15, 0.20, 0.31, 0.26, .08. Find
the probabilities that a particular surgery will be rated
\begin{enumerate}
	\item complex or very complex;
	\item neither very complex nor very simple;
	\item routine or complex
	\item routine or simple
\end{enumerate}
\solution
%\input{exemplar/11/16/3/8(1)/main.tex}
\item A card is selected from a pack of 52 cards.
\begin{enumerate}[label=(\alph*)]
    \item How many points are there in the sample space?
    \item Calculate the probability that the card is an ace of spades.
    \item Calculate the probability that the card is (i) an ace and (ii) black card.
\end{enumerate}
\solution
%\input{exemplar/11/16/3/4/main2.tex}
\item The probability that a non leap year selected at random will contain 53 sundays.
\\
\solution
%\input{exemplar/10/13/1/19/main.tex}
\item One of the four persons John, Rita, Aslam or Gurpreet will be promoted next
month. Consequently the sample space consists of four elementary outcomes
S = {John promoted, Rita promoted, Aslam promoted, Gurpreet promoted}
You are told that the chances of John’s promotion is same as that of Gurpreet,
Rita’s chances of promotion are twice as likely as Johns. Aslam’s chances are
four times that of John.
\begin{enumerate}
	\item Determine
	\begin{enumerate}
		\item P (John promoted)
		\item P (Rita promoted)
		\item P (Aslam promoted)
		\item P (Gurpreet promoted)
	\end{enumerate}
	\item If A = {John promoted or Gurpreet promoted}, find P (A).
\end{enumerate}
\solution
%\input{exemplar/11/16/3/10/main.tex}
\item A card is drawn from a deck of 52 cards. Find the probability of getting a king or a heart or a red card.\\
\solution
%\input{exemplar/11/16/3/15/main.tex}
\item The probability that a student will pass his examination is 0.73, the probability of
the student getting a compartment is 0.13, and the probability that the student will
either pass or get compartment is 0.96. State True or False.\\
\solution
%\input{exemplar/11/16/3/31/main.tex}
\item A card is selected from a pack of 52 cards\\
\begin{enumerate}[label=(\alph*)]
\item How many points are there in the sample space?
\item Calculate the probability that the cards is an ace of spades.
\item Calculate the probability that the card is (i) an ace (ii)black card.\\
\end{enumerate}
%\input{ncert/11/16/3/4_1/Prob_4.tex}
\item In a non-leap year, the probability of having 53 tuesdays or 53 wednesdays is\\
\solution
%\input{exemplar/11/16/3/18/main.tex}
\item There are 1000 sealed envelopes in a box, 10 of them contain a cash prize of
Rs 100 each, 100 of them contain a cash prize of Rs 50 each and 200 of them
contain a cash prize of Rs 10 each and rest do not contain any cash prize. If they
are well shuffled and an envelope is picked up out, what is the probability that it
contains no cash prize?\\
\solution
%\input{exemplar/10/13/3/34/main.tex}
\item 
A die is thrown and a card is selected at random from a deck of 52 playing cards. The probability of getting an even number on the die and a spade card.\\
\solution
%\input{exemplar/12/13/3/78/main.tex}
\item
If 4-digit numbers greater than 5,000 are randomly formed from the digits 0, 1, 3, 5, and 7, what is the probability of forming a number divisible by 5 when:
\begin{enumerate}
    \item The digits are repeated?
    \item The repetition of digits is not allowed?
\end{enumerate}
\solution
%\input{ncert/11/16/4/9/main.tex}
\item Consider the probability space $\brak{\Omega, \mathcal{G}, P}$ where $\Omega = [0,2]$ and $\mathcal{G} = \cbrak{\phi, \Omega, [0,1], (1,2]}$. Let $X$ and $Y$ be two functions on $\Omega$ defined as
\begin{align*}
    X(\omega) = 
    \begin{cases}
        1 & \text{if }\omega \in [0, 1]\\
        2 & \text{if }\omega \in (1, 2]
    \end{cases}
\end{align*}
and
\begin{align*}
    Y(\omega) = 
    \begin{cases}
        2 & \text{if }\omega \in [0, 1.5]\\
        3 & \text{if }\omega \in (1.5, 2].
    \end{cases}
\end{align*}
Then which one of the following statements is true?
\begin{enumerate}
    \item [(A)] $X$ is a random variable with respect to $\mathcal{G}$, but $Y$ is not a random variable with respect to $\mathcal{G}$.
    \item [(B)] $Y$ is a random variable with respect to $\mathcal{G}$, but $X$ is not a random variable with respect to $\mathcal{G}$.
    \item [(C)] Neither $X$ nor $Y$ is a random variable with respect to $\mathcal{G}$.
    \item [(D)] Both $X$ and $Y$ are random variables with respect to $\mathcal{G}$.
\end{enumerate} \hfill (GATE ST 2023)\\
\solution
%\input{gate/ST/2023/14/main.tex}
	\item  A die is loaded in such a way that each odd number is twice as likely to occur as
each even number. Find $P(G)$, where $G$ is the event that a number greater than
3 occurs on a single roll of the die.
\\
\solution
		%\input{exemplar/11/16/3/5/main.tex}
	\item All the jacks, queens and kings are removed from a deck of 52 playing cards. The remaining cards are well shuffled and then one card is drawn at random. Giving ace a value 1 similar value for other cards, find the probability that the card has a value 
		\begin{enumerate}
			\item 7
			\item greater than 7
			\item less than 7
		\end{enumerate}
		%\input{exemplar/10/13/3/30/main.tex}
  \item A Lot consists of 48 mobile phones of which 42 are good, 3 have only minor defects and 3 have major defects.Varnika will buy a phone if it is good but the trader will only buy a mobile if it has no major defects. One phone is selected at random from the lot. What is the probability that it is
\begin{enumerate}
	\item acceptable to Varnika?
            \item acceptable to the trader?
\end{enumerate}
\solution
	%\input{exemplar/10/13/3/40/main.tex}
 \item A student says that if you throw a die, it will show up 1 or not 1. Therefore, the probability of getting 1 and the probability of getting 'not 1' each is equal to $\frac{1}{2}$. Is this correct? Give reasons.\\
 \solution
        %\input{exemplar/10/13/2/9/main.tex}
   \item Four candidates A, B, C, D have ap-
plied for the assignment to coach a school cricket
team. If A is twice as likely to be selected as B, and
B and C are given about the same chance of being
selected, while C is twice as likely to be selected
as D, what are the probabilities that
\begin{enumerate}
\item C will be selected?
\item A will not be selected?
\end{enumerate}
	%\input{exemplar/11/16/3/9/main.tex}
 \item A bag contain 24 balls of which $x$ balls are red, $2x$ are white and $3x$ are blue. A ball is selected at random, What is the probability that it is
\begin{enumerate}[label=\alph*)]
\item not red ?
\item white ?
\end{enumerate}
%\input{exemplar/10/13/3/41/main.tex}
If the letters of the word ASSASSINATION are arranged at random. Find the Probability that
\begin{enumerate}[label=(\alph*)]
\item Four $S's$ come consecutively in the word
\item Two  $I's$ and two $N's$ come together
\item All $A's$ are not coming together
\item No two $A's$ are coming together
\end{enumerate}
%\input{exemplar/11/16/3/14/main.tex}
	\item One urn contains two black balls (labelled B1 and B2) and one white ball. A
	second urn contains one black ball and two white balls (labelled W1 and W2).
	Suppose the following experiment is performed. One of the two urns is chosen
	at random. Next a ball is randomly chosen from the urn. Then a second ball is
	chosen at random from the same urn without replacing the first ball.
	
	\begin{enumerate}
	\item What is the probability that two black balls are chosen?
	
	\item What is the probability that two balls of opposite colour are chosen?
	\end{enumerate}
	\solution
	%\input{exemplar/11/16/3/12/main1.tex}
\end{enumerate}

	\item A card is selected from a pack of 52 cards.
 \begin{enumerate}[label=(\alph*)] 
                 \item How many points are there in the sample space?
                 \item Calculate the probability that the card is an ace of spades.
                 \item Calculate the probability that the card is (i) an ace and (ii) black card.
 \end{enumerate}
\solution
		%\begin{table}[H]
	\centering
\begin{tabular}{|c|c|c|}
\hline
Random variable &Value &Definition\\ \hline
\multirow{3}{*}{X} &0 &Slips of Rs 1\\
&1 &Slips of Rs 5\\
&2 &Slips of Rs 13\\ \hline
\multirow{2}{*}{Y} &0 &Box A\\
&1 &Box B\\\hline
\end{tabular}
\caption{}
\label{tab:Distribution}
\end{table}
See \tabref{tab:Distribution}.
\begin{align}
p_{Y}\brak{k}= \begin{cases} 
      \frac{1}{3} & {k=0} \\
      \frac{2}{3 }& {k=1} 
   \end{cases}
   \\
p_{Y|X}\brak{0|0} = \frac{19}{25}\, 
p_{Y|X}\brak{0|1} = \frac{6}{25}\,
p_{Y|X}\brak{1|0} = \frac{45}{50}\,
p_{Y|X}\brak{1|2} = \frac{5}{50}
\end{align}
The desired probability is the probability that a slip drawn at random is marked other than Rs 1,
\begin{align}
&=1-p_X\brak{0}\\
&= p_X(1) + p_X(2)
\end{align}
Using Bayes theorem,
\begin{align}
&= p_Y\brak{0} \times \pr{Y=0 | X=1} + p_Y\brak{1} \times \pr{Y=1|X=2}\\
&=\frac{1}{3} \times \frac{6}{25} + \frac{2}{3} \times \frac{5}{50}\\
&=\frac{11}{75}
\end{align}

\newpage

%\tableofcontents

\bigskip

\renewcommand{\thefigure}{\theenumi}
\renewcommand{\thetable}{\theenumi}
%\renewcommand{\theequation}{\theenumi}

%\begin{abstract}
%%\boldmath
%In this letter, an algorithm for evaluating the exact analytical bit error rate  (BER)  for the piecewise linear (PL) combiner for  multiple relays is presented. Previous results were available only for upto three relays. The algorithm is unique in the sense that  the actual mathematical expressions, that are prohibitively large, need not be explicitly obtained. The diversity gain due to multiple relays is shown through plots of the analytical BER, well supported by simulations. 
%
%\end{abstract}
% IEEEtran.cls defaults to using nonbold math in the Abstract.
% This preserves the distinction between vectors and scalars. However,
% if the journal you are submitting to favors bold math in the abstract,
% then you can use LaTeX's standard command \boldmath at the very start
% of the abstract to achieve this. Many IEEE journals frown on math
% in the abstract anyway.

% Note that keywords are not normally used for peerreview papers.
%\begin{IEEEkeywords}
%Cooperative diversity, decode and forward, piecewise linear
%\end{IEEEkeywords}



% For peer review papers, you can put extra information on the cover
% page as needed:
% \ifCLASSOPTIONpeerreview
% \begin{center} \bfseries EDICS Category: 3-BBND \end{center}
% \fi
%
% For peerreview papers, this IEEEtran command inserts a page break and
% creates the second title. It will be ignored for other modes.
%\IEEEpeerreviewmaketitle




\item Four cards are drawn from a well-shuffled deck of 52 cards. What is the probability of obtaining 3 diamonds and one spade.
\\
\solution
		%\begin{enumerate}[label=\thesection.\arabic*,ref=\thesection.\theenumi]
	\item One card is drawn from a well-shuffled deck of 52 cards. Find the probability of getting
\begin{enumerate}
\item A king of red colour 
\item A face card 
\item A red face card
\item The jack of hearts
\item A spade
\item The queen of diamonds

\end{enumerate}
\solution
		%\input{ncert/10/15/1/14/main.tex}
	\item Five cards—the ten, jack, queen, king and ace of diamonds, are well-shuffled with their face downwards. One card is then picked up at random.
\begin{enumerate}
\item
What is the probability that the card is the queen? 
\item
If the queen is drawn and put aside, what is the probability that the second card picked up is (a) an ace? (b) a queen?\\
\end{enumerate}
\solution
		%\input{ncert/10/15/1/15/defs.tex}
	\item A bag contains $5$ red balls and some blue balls. If the probability of drawing a blue ball is double that if a red ball, determine the number of blue balls in the bag. 
		\\
\solution
		%\input{ncert/10/15/2/3/defs.tex}
	\item A card is selected from a pack of 52 cards.
 \begin{enumerate}[label=(\alph*)] 
                 \item How many points are there in the sample space?
                 \item Calculate the probability that the card is an ace of spades.
                 \item Calculate the probability that the card is (i) an ace and (ii) black card.
 \end{enumerate}
\solution
		%\input{ncert/11/16/3/4/main.tex}
\item Four cards are drawn from a well-shuffled deck of 52 cards. What is the probability of obtaining 3 diamonds and one spade.
\\
\solution
		%\input{ncert/11/16/4/2/defs.tex}
\item In a certain lottery 10,000 tickets are sold and ten equal prizes are awarded. What is the probability of not getting a prize if you buy (a) one ticket (b) two tickets (c) 10 tickets ?	
\\
\solution
		%\input{ncert/11/16/4/4/defs.tex}
		%
\item 
Out of 100 students, two sections of 40 and 60 are formed. If you and your friend are among the 100 students, what is the probability that
\begin{enumerate}
\item you both enter the same section?
\item you both enter the different sections?
\end{enumerate}
\solution
		%\input{ncert/11/16/4/5/defs.tex}
	\item 
The number lock of a suitcase has 4 wheels each labelled with ten digits i.e. from 0 to 9.The lock opens with a sequence of four digits with no repeats.What is the probability of a person getting the right sequence to open the suitcase.
\\
\solution
		%\input{ncert/11/16/4/10/defs.tex}
		%
\item 
Two cards are drawn at random and without replacement from a pack of 52 playing cards. Find the probability that both the cards are black.
\\
\solution
		%\input{ncert/12/13/2/2/defs.tex}
		\item A box of oranges is inspected by examining three randomly selected oranges drawn without replacement. If all the three oranges are good, the box is approved for sale, otherwise, it is rejected. Find the probability that a box containing 15 oranges out of which 12 are good and 3 are bad ones will be approved for sale.
		\label{ncert/12/13/2/3/defs.tex}
		\item Two balls are drawn at random with replacement from a box containing 10 black and 8 red balls. Find the probability that
		\label{ncert/12/13/2/12}
\begin{enumerate}
\item both balls are red.
\item first ball is black and second is red.
\item one of them is black and other is red.
\end{enumerate}

\item In a hostel, 60\% of the students read Hindi newspaper, 40\% read English newspaper and 20\% read both Hindi and English newspapers. A student is selected at random.
		\label{ncert/12/13/2/15}
\begin{enumerate}
\item Find the probability that she reads neither Hindi nor English newspapers.
\item If she reads Hindi newspaper, find the probability that she reads English newspaper.
\item If she reads English newspaper, find the probability that she reads Hindi newspaper.\\
\end{enumerate}
\item The probability of obtaining an even prime number on each die, when a pair of dice is rolled is 
\begin{enumerate}
    \item $0$ 
    
    \item $\frac{1}{3}$ 
    
    \item $\frac{1}{12}$ 
    
    \item $\frac{1}{36}$ 
\end{enumerate}
\solution
		%\input{ncert/12/13/2/17/defs.tex}
	\item A bag contains 4 red and 4 black balls, another bag contains 2 red and 6 black balls. One of the two bags is selected at random and a ball is drawn from the bag which is found to be red. Find the probability that the ball is drawn from the first bag.
\\
\solution
		%\input{ncert/12/13/3/2/main.tex}
  \item
  Cards with numbers 2 to 101 are placed in a box. A card is selected at random.Find the probability that the card has
\begin{enumerate}[label=(\roman*)]
	\item an even number 
	\item a square number
\end{enumerate}
\solution
%\input{exemplar/10/13/3/32/main.tex}
\item
The king, queen and jack of clubs are removed from a deck of 52 playing cards and then well shuffled. Now one card is drawn at random from the remaining cards.  Determine the probability that the card is
\begin{enumerate}[label=(\roman*)]
\item a club
\item 10 of hearts
\end{enumerate}
\solution
%\input{exemplar/10/13/3/29/main.tex}
\item A team of medical students doing their internship have to assist during surgeries
at a city hospital. The probabilities of surgeries rated as very complex, complex,
routine, simple or very simple are respectively, 0.15, 0.20, 0.31, 0.26, .08. Find
the probabilities that a particular surgery will be rated
\begin{enumerate}
	\item complex or very complex;
	\item neither very complex nor very simple;
	\item routine or complex
	\item routine or simple
\end{enumerate}
\solution
%\input{exemplar/11/16/3/8(1)/main.tex}
\item A card is selected from a pack of 52 cards.
\begin{enumerate}[label=(\alph*)]
    \item How many points are there in the sample space?
    \item Calculate the probability that the card is an ace of spades.
    \item Calculate the probability that the card is (i) an ace and (ii) black card.
\end{enumerate}
\solution
%\input{exemplar/11/16/3/4/main2.tex}
\item The probability that a non leap year selected at random will contain 53 sundays.
\\
\solution
%\input{exemplar/10/13/1/19/main.tex}
\item One of the four persons John, Rita, Aslam or Gurpreet will be promoted next
month. Consequently the sample space consists of four elementary outcomes
S = {John promoted, Rita promoted, Aslam promoted, Gurpreet promoted}
You are told that the chances of John’s promotion is same as that of Gurpreet,
Rita’s chances of promotion are twice as likely as Johns. Aslam’s chances are
four times that of John.
\begin{enumerate}
	\item Determine
	\begin{enumerate}
		\item P (John promoted)
		\item P (Rita promoted)
		\item P (Aslam promoted)
		\item P (Gurpreet promoted)
	\end{enumerate}
	\item If A = {John promoted or Gurpreet promoted}, find P (A).
\end{enumerate}
\solution
%\input{exemplar/11/16/3/10/main.tex}
\item A card is drawn from a deck of 52 cards. Find the probability of getting a king or a heart or a red card.\\
\solution
%\input{exemplar/11/16/3/15/main.tex}
\item The probability that a student will pass his examination is 0.73, the probability of
the student getting a compartment is 0.13, and the probability that the student will
either pass or get compartment is 0.96. State True or False.\\
\solution
%\input{exemplar/11/16/3/31/main.tex}
\item A card is selected from a pack of 52 cards\\
\begin{enumerate}[label=(\alph*)]
\item How many points are there in the sample space?
\item Calculate the probability that the cards is an ace of spades.
\item Calculate the probability that the card is (i) an ace (ii)black card.\\
\end{enumerate}
%\input{ncert/11/16/3/4_1/Prob_4.tex}
\item In a non-leap year, the probability of having 53 tuesdays or 53 wednesdays is\\
\solution
%\input{exemplar/11/16/3/18/main.tex}
\item There are 1000 sealed envelopes in a box, 10 of them contain a cash prize of
Rs 100 each, 100 of them contain a cash prize of Rs 50 each and 200 of them
contain a cash prize of Rs 10 each and rest do not contain any cash prize. If they
are well shuffled and an envelope is picked up out, what is the probability that it
contains no cash prize?\\
\solution
%\input{exemplar/10/13/3/34/main.tex}
\item 
A die is thrown and a card is selected at random from a deck of 52 playing cards. The probability of getting an even number on the die and a spade card.\\
\solution
%\input{exemplar/12/13/3/78/main.tex}
\item
If 4-digit numbers greater than 5,000 are randomly formed from the digits 0, 1, 3, 5, and 7, what is the probability of forming a number divisible by 5 when:
\begin{enumerate}
    \item The digits are repeated?
    \item The repetition of digits is not allowed?
\end{enumerate}
\solution
%\input{ncert/11/16/4/9/main.tex}
\item Consider the probability space $\brak{\Omega, \mathcal{G}, P}$ where $\Omega = [0,2]$ and $\mathcal{G} = \cbrak{\phi, \Omega, [0,1], (1,2]}$. Let $X$ and $Y$ be two functions on $\Omega$ defined as
\begin{align*}
    X(\omega) = 
    \begin{cases}
        1 & \text{if }\omega \in [0, 1]\\
        2 & \text{if }\omega \in (1, 2]
    \end{cases}
\end{align*}
and
\begin{align*}
    Y(\omega) = 
    \begin{cases}
        2 & \text{if }\omega \in [0, 1.5]\\
        3 & \text{if }\omega \in (1.5, 2].
    \end{cases}
\end{align*}
Then which one of the following statements is true?
\begin{enumerate}
    \item [(A)] $X$ is a random variable with respect to $\mathcal{G}$, but $Y$ is not a random variable with respect to $\mathcal{G}$.
    \item [(B)] $Y$ is a random variable with respect to $\mathcal{G}$, but $X$ is not a random variable with respect to $\mathcal{G}$.
    \item [(C)] Neither $X$ nor $Y$ is a random variable with respect to $\mathcal{G}$.
    \item [(D)] Both $X$ and $Y$ are random variables with respect to $\mathcal{G}$.
\end{enumerate} \hfill (GATE ST 2023)\\
\solution
%\input{gate/ST/2023/14/main.tex}
	\item  A die is loaded in such a way that each odd number is twice as likely to occur as
each even number. Find $P(G)$, where $G$ is the event that a number greater than
3 occurs on a single roll of the die.
\\
\solution
		%\input{exemplar/11/16/3/5/main.tex}
	\item All the jacks, queens and kings are removed from a deck of 52 playing cards. The remaining cards are well shuffled and then one card is drawn at random. Giving ace a value 1 similar value for other cards, find the probability that the card has a value 
		\begin{enumerate}
			\item 7
			\item greater than 7
			\item less than 7
		\end{enumerate}
		%\input{exemplar/10/13/3/30/main.tex}
  \item A Lot consists of 48 mobile phones of which 42 are good, 3 have only minor defects and 3 have major defects.Varnika will buy a phone if it is good but the trader will only buy a mobile if it has no major defects. One phone is selected at random from the lot. What is the probability that it is
\begin{enumerate}
	\item acceptable to Varnika?
            \item acceptable to the trader?
\end{enumerate}
\solution
	%\input{exemplar/10/13/3/40/main.tex}
 \item A student says that if you throw a die, it will show up 1 or not 1. Therefore, the probability of getting 1 and the probability of getting 'not 1' each is equal to $\frac{1}{2}$. Is this correct? Give reasons.\\
 \solution
        %\input{exemplar/10/13/2/9/main.tex}
   \item Four candidates A, B, C, D have ap-
plied for the assignment to coach a school cricket
team. If A is twice as likely to be selected as B, and
B and C are given about the same chance of being
selected, while C is twice as likely to be selected
as D, what are the probabilities that
\begin{enumerate}
\item C will be selected?
\item A will not be selected?
\end{enumerate}
	%\input{exemplar/11/16/3/9/main.tex}
 \item A bag contain 24 balls of which $x$ balls are red, $2x$ are white and $3x$ are blue. A ball is selected at random, What is the probability that it is
\begin{enumerate}[label=\alph*)]
\item not red ?
\item white ?
\end{enumerate}
%\input{exemplar/10/13/3/41/main.tex}
If the letters of the word ASSASSINATION are arranged at random. Find the Probability that
\begin{enumerate}[label=(\alph*)]
\item Four $S's$ come consecutively in the word
\item Two  $I's$ and two $N's$ come together
\item All $A's$ are not coming together
\item No two $A's$ are coming together
\end{enumerate}
%\input{exemplar/11/16/3/14/main.tex}
	\item One urn contains two black balls (labelled B1 and B2) and one white ball. A
	second urn contains one black ball and two white balls (labelled W1 and W2).
	Suppose the following experiment is performed. One of the two urns is chosen
	at random. Next a ball is randomly chosen from the urn. Then a second ball is
	chosen at random from the same urn without replacing the first ball.
	
	\begin{enumerate}
	\item What is the probability that two black balls are chosen?
	
	\item What is the probability that two balls of opposite colour are chosen?
	\end{enumerate}
	\solution
	%\input{exemplar/11/16/3/12/main1.tex}
\end{enumerate}

\item In a certain lottery 10,000 tickets are sold and ten equal prizes are awarded. What is the probability of not getting a prize if you buy (a) one ticket (b) two tickets (c) 10 tickets ?	
\\
\solution
		%\begin{enumerate}[label=\thesection.\arabic*,ref=\thesection.\theenumi]
	\item One card is drawn from a well-shuffled deck of 52 cards. Find the probability of getting
\begin{enumerate}
\item A king of red colour 
\item A face card 
\item A red face card
\item The jack of hearts
\item A spade
\item The queen of diamonds

\end{enumerate}
\solution
		%\input{ncert/10/15/1/14/main.tex}
	\item Five cards—the ten, jack, queen, king and ace of diamonds, are well-shuffled with their face downwards. One card is then picked up at random.
\begin{enumerate}
\item
What is the probability that the card is the queen? 
\item
If the queen is drawn and put aside, what is the probability that the second card picked up is (a) an ace? (b) a queen?\\
\end{enumerate}
\solution
		%\input{ncert/10/15/1/15/defs.tex}
	\item A bag contains $5$ red balls and some blue balls. If the probability of drawing a blue ball is double that if a red ball, determine the number of blue balls in the bag. 
		\\
\solution
		%\input{ncert/10/15/2/3/defs.tex}
	\item A card is selected from a pack of 52 cards.
 \begin{enumerate}[label=(\alph*)] 
                 \item How many points are there in the sample space?
                 \item Calculate the probability that the card is an ace of spades.
                 \item Calculate the probability that the card is (i) an ace and (ii) black card.
 \end{enumerate}
\solution
		%\input{ncert/11/16/3/4/main.tex}
\item Four cards are drawn from a well-shuffled deck of 52 cards. What is the probability of obtaining 3 diamonds and one spade.
\\
\solution
		%\input{ncert/11/16/4/2/defs.tex}
\item In a certain lottery 10,000 tickets are sold and ten equal prizes are awarded. What is the probability of not getting a prize if you buy (a) one ticket (b) two tickets (c) 10 tickets ?	
\\
\solution
		%\input{ncert/11/16/4/4/defs.tex}
		%
\item 
Out of 100 students, two sections of 40 and 60 are formed. If you and your friend are among the 100 students, what is the probability that
\begin{enumerate}
\item you both enter the same section?
\item you both enter the different sections?
\end{enumerate}
\solution
		%\input{ncert/11/16/4/5/defs.tex}
	\item 
The number lock of a suitcase has 4 wheels each labelled with ten digits i.e. from 0 to 9.The lock opens with a sequence of four digits with no repeats.What is the probability of a person getting the right sequence to open the suitcase.
\\
\solution
		%\input{ncert/11/16/4/10/defs.tex}
		%
\item 
Two cards are drawn at random and without replacement from a pack of 52 playing cards. Find the probability that both the cards are black.
\\
\solution
		%\input{ncert/12/13/2/2/defs.tex}
		\item A box of oranges is inspected by examining three randomly selected oranges drawn without replacement. If all the three oranges are good, the box is approved for sale, otherwise, it is rejected. Find the probability that a box containing 15 oranges out of which 12 are good and 3 are bad ones will be approved for sale.
		\label{ncert/12/13/2/3/defs.tex}
		\item Two balls are drawn at random with replacement from a box containing 10 black and 8 red balls. Find the probability that
		\label{ncert/12/13/2/12}
\begin{enumerate}
\item both balls are red.
\item first ball is black and second is red.
\item one of them is black and other is red.
\end{enumerate}

\item In a hostel, 60\% of the students read Hindi newspaper, 40\% read English newspaper and 20\% read both Hindi and English newspapers. A student is selected at random.
		\label{ncert/12/13/2/15}
\begin{enumerate}
\item Find the probability that she reads neither Hindi nor English newspapers.
\item If she reads Hindi newspaper, find the probability that she reads English newspaper.
\item If she reads English newspaper, find the probability that she reads Hindi newspaper.\\
\end{enumerate}
\item The probability of obtaining an even prime number on each die, when a pair of dice is rolled is 
\begin{enumerate}
    \item $0$ 
    
    \item $\frac{1}{3}$ 
    
    \item $\frac{1}{12}$ 
    
    \item $\frac{1}{36}$ 
\end{enumerate}
\solution
		%\input{ncert/12/13/2/17/defs.tex}
	\item A bag contains 4 red and 4 black balls, another bag contains 2 red and 6 black balls. One of the two bags is selected at random and a ball is drawn from the bag which is found to be red. Find the probability that the ball is drawn from the first bag.
\\
\solution
		%\input{ncert/12/13/3/2/main.tex}
  \item
  Cards with numbers 2 to 101 are placed in a box. A card is selected at random.Find the probability that the card has
\begin{enumerate}[label=(\roman*)]
	\item an even number 
	\item a square number
\end{enumerate}
\solution
%\input{exemplar/10/13/3/32/main.tex}
\item
The king, queen and jack of clubs are removed from a deck of 52 playing cards and then well shuffled. Now one card is drawn at random from the remaining cards.  Determine the probability that the card is
\begin{enumerate}[label=(\roman*)]
\item a club
\item 10 of hearts
\end{enumerate}
\solution
%\input{exemplar/10/13/3/29/main.tex}
\item A team of medical students doing their internship have to assist during surgeries
at a city hospital. The probabilities of surgeries rated as very complex, complex,
routine, simple or very simple are respectively, 0.15, 0.20, 0.31, 0.26, .08. Find
the probabilities that a particular surgery will be rated
\begin{enumerate}
	\item complex or very complex;
	\item neither very complex nor very simple;
	\item routine or complex
	\item routine or simple
\end{enumerate}
\solution
%\input{exemplar/11/16/3/8(1)/main.tex}
\item A card is selected from a pack of 52 cards.
\begin{enumerate}[label=(\alph*)]
    \item How many points are there in the sample space?
    \item Calculate the probability that the card is an ace of spades.
    \item Calculate the probability that the card is (i) an ace and (ii) black card.
\end{enumerate}
\solution
%\input{exemplar/11/16/3/4/main2.tex}
\item The probability that a non leap year selected at random will contain 53 sundays.
\\
\solution
%\input{exemplar/10/13/1/19/main.tex}
\item One of the four persons John, Rita, Aslam or Gurpreet will be promoted next
month. Consequently the sample space consists of four elementary outcomes
S = {John promoted, Rita promoted, Aslam promoted, Gurpreet promoted}
You are told that the chances of John’s promotion is same as that of Gurpreet,
Rita’s chances of promotion are twice as likely as Johns. Aslam’s chances are
four times that of John.
\begin{enumerate}
	\item Determine
	\begin{enumerate}
		\item P (John promoted)
		\item P (Rita promoted)
		\item P (Aslam promoted)
		\item P (Gurpreet promoted)
	\end{enumerate}
	\item If A = {John promoted or Gurpreet promoted}, find P (A).
\end{enumerate}
\solution
%\input{exemplar/11/16/3/10/main.tex}
\item A card is drawn from a deck of 52 cards. Find the probability of getting a king or a heart or a red card.\\
\solution
%\input{exemplar/11/16/3/15/main.tex}
\item The probability that a student will pass his examination is 0.73, the probability of
the student getting a compartment is 0.13, and the probability that the student will
either pass or get compartment is 0.96. State True or False.\\
\solution
%\input{exemplar/11/16/3/31/main.tex}
\item A card is selected from a pack of 52 cards\\
\begin{enumerate}[label=(\alph*)]
\item How many points are there in the sample space?
\item Calculate the probability that the cards is an ace of spades.
\item Calculate the probability that the card is (i) an ace (ii)black card.\\
\end{enumerate}
%\input{ncert/11/16/3/4_1/Prob_4.tex}
\item In a non-leap year, the probability of having 53 tuesdays or 53 wednesdays is\\
\solution
%\input{exemplar/11/16/3/18/main.tex}
\item There are 1000 sealed envelopes in a box, 10 of them contain a cash prize of
Rs 100 each, 100 of them contain a cash prize of Rs 50 each and 200 of them
contain a cash prize of Rs 10 each and rest do not contain any cash prize. If they
are well shuffled and an envelope is picked up out, what is the probability that it
contains no cash prize?\\
\solution
%\input{exemplar/10/13/3/34/main.tex}
\item 
A die is thrown and a card is selected at random from a deck of 52 playing cards. The probability of getting an even number on the die and a spade card.\\
\solution
%\input{exemplar/12/13/3/78/main.tex}
\item
If 4-digit numbers greater than 5,000 are randomly formed from the digits 0, 1, 3, 5, and 7, what is the probability of forming a number divisible by 5 when:
\begin{enumerate}
    \item The digits are repeated?
    \item The repetition of digits is not allowed?
\end{enumerate}
\solution
%\input{ncert/11/16/4/9/main.tex}
\item Consider the probability space $\brak{\Omega, \mathcal{G}, P}$ where $\Omega = [0,2]$ and $\mathcal{G} = \cbrak{\phi, \Omega, [0,1], (1,2]}$. Let $X$ and $Y$ be two functions on $\Omega$ defined as
\begin{align*}
    X(\omega) = 
    \begin{cases}
        1 & \text{if }\omega \in [0, 1]\\
        2 & \text{if }\omega \in (1, 2]
    \end{cases}
\end{align*}
and
\begin{align*}
    Y(\omega) = 
    \begin{cases}
        2 & \text{if }\omega \in [0, 1.5]\\
        3 & \text{if }\omega \in (1.5, 2].
    \end{cases}
\end{align*}
Then which one of the following statements is true?
\begin{enumerate}
    \item [(A)] $X$ is a random variable with respect to $\mathcal{G}$, but $Y$ is not a random variable with respect to $\mathcal{G}$.
    \item [(B)] $Y$ is a random variable with respect to $\mathcal{G}$, but $X$ is not a random variable with respect to $\mathcal{G}$.
    \item [(C)] Neither $X$ nor $Y$ is a random variable with respect to $\mathcal{G}$.
    \item [(D)] Both $X$ and $Y$ are random variables with respect to $\mathcal{G}$.
\end{enumerate} \hfill (GATE ST 2023)\\
\solution
%\input{gate/ST/2023/14/main.tex}
	\item  A die is loaded in such a way that each odd number is twice as likely to occur as
each even number. Find $P(G)$, where $G$ is the event that a number greater than
3 occurs on a single roll of the die.
\\
\solution
		%\input{exemplar/11/16/3/5/main.tex}
	\item All the jacks, queens and kings are removed from a deck of 52 playing cards. The remaining cards are well shuffled and then one card is drawn at random. Giving ace a value 1 similar value for other cards, find the probability that the card has a value 
		\begin{enumerate}
			\item 7
			\item greater than 7
			\item less than 7
		\end{enumerate}
		%\input{exemplar/10/13/3/30/main.tex}
  \item A Lot consists of 48 mobile phones of which 42 are good, 3 have only minor defects and 3 have major defects.Varnika will buy a phone if it is good but the trader will only buy a mobile if it has no major defects. One phone is selected at random from the lot. What is the probability that it is
\begin{enumerate}
	\item acceptable to Varnika?
            \item acceptable to the trader?
\end{enumerate}
\solution
	%\input{exemplar/10/13/3/40/main.tex}
 \item A student says that if you throw a die, it will show up 1 or not 1. Therefore, the probability of getting 1 and the probability of getting 'not 1' each is equal to $\frac{1}{2}$. Is this correct? Give reasons.\\
 \solution
        %\input{exemplar/10/13/2/9/main.tex}
   \item Four candidates A, B, C, D have ap-
plied for the assignment to coach a school cricket
team. If A is twice as likely to be selected as B, and
B and C are given about the same chance of being
selected, while C is twice as likely to be selected
as D, what are the probabilities that
\begin{enumerate}
\item C will be selected?
\item A will not be selected?
\end{enumerate}
	%\input{exemplar/11/16/3/9/main.tex}
 \item A bag contain 24 balls of which $x$ balls are red, $2x$ are white and $3x$ are blue. A ball is selected at random, What is the probability that it is
\begin{enumerate}[label=\alph*)]
\item not red ?
\item white ?
\end{enumerate}
%\input{exemplar/10/13/3/41/main.tex}
If the letters of the word ASSASSINATION are arranged at random. Find the Probability that
\begin{enumerate}[label=(\alph*)]
\item Four $S's$ come consecutively in the word
\item Two  $I's$ and two $N's$ come together
\item All $A's$ are not coming together
\item No two $A's$ are coming together
\end{enumerate}
%\input{exemplar/11/16/3/14/main.tex}
	\item One urn contains two black balls (labelled B1 and B2) and one white ball. A
	second urn contains one black ball and two white balls (labelled W1 and W2).
	Suppose the following experiment is performed. One of the two urns is chosen
	at random. Next a ball is randomly chosen from the urn. Then a second ball is
	chosen at random from the same urn without replacing the first ball.
	
	\begin{enumerate}
	\item What is the probability that two black balls are chosen?
	
	\item What is the probability that two balls of opposite colour are chosen?
	\end{enumerate}
	\solution
	%\input{exemplar/11/16/3/12/main1.tex}
\end{enumerate}

		%
\item 
Out of 100 students, two sections of 40 and 60 are formed. If you and your friend are among the 100 students, what is the probability that
\begin{enumerate}
\item you both enter the same section?
\item you both enter the different sections?
\end{enumerate}
\solution
		%\begin{enumerate}[label=\thesection.\arabic*,ref=\thesection.\theenumi]
	\item One card is drawn from a well-shuffled deck of 52 cards. Find the probability of getting
\begin{enumerate}
\item A king of red colour 
\item A face card 
\item A red face card
\item The jack of hearts
\item A spade
\item The queen of diamonds

\end{enumerate}
\solution
		%\input{ncert/10/15/1/14/main.tex}
	\item Five cards—the ten, jack, queen, king and ace of diamonds, are well-shuffled with their face downwards. One card is then picked up at random.
\begin{enumerate}
\item
What is the probability that the card is the queen? 
\item
If the queen is drawn and put aside, what is the probability that the second card picked up is (a) an ace? (b) a queen?\\
\end{enumerate}
\solution
		%\input{ncert/10/15/1/15/defs.tex}
	\item A bag contains $5$ red balls and some blue balls. If the probability of drawing a blue ball is double that if a red ball, determine the number of blue balls in the bag. 
		\\
\solution
		%\input{ncert/10/15/2/3/defs.tex}
	\item A card is selected from a pack of 52 cards.
 \begin{enumerate}[label=(\alph*)] 
                 \item How many points are there in the sample space?
                 \item Calculate the probability that the card is an ace of spades.
                 \item Calculate the probability that the card is (i) an ace and (ii) black card.
 \end{enumerate}
\solution
		%\input{ncert/11/16/3/4/main.tex}
\item Four cards are drawn from a well-shuffled deck of 52 cards. What is the probability of obtaining 3 diamonds and one spade.
\\
\solution
		%\input{ncert/11/16/4/2/defs.tex}
\item In a certain lottery 10,000 tickets are sold and ten equal prizes are awarded. What is the probability of not getting a prize if you buy (a) one ticket (b) two tickets (c) 10 tickets ?	
\\
\solution
		%\input{ncert/11/16/4/4/defs.tex}
		%
\item 
Out of 100 students, two sections of 40 and 60 are formed. If you and your friend are among the 100 students, what is the probability that
\begin{enumerate}
\item you both enter the same section?
\item you both enter the different sections?
\end{enumerate}
\solution
		%\input{ncert/11/16/4/5/defs.tex}
	\item 
The number lock of a suitcase has 4 wheels each labelled with ten digits i.e. from 0 to 9.The lock opens with a sequence of four digits with no repeats.What is the probability of a person getting the right sequence to open the suitcase.
\\
\solution
		%\input{ncert/11/16/4/10/defs.tex}
		%
\item 
Two cards are drawn at random and without replacement from a pack of 52 playing cards. Find the probability that both the cards are black.
\\
\solution
		%\input{ncert/12/13/2/2/defs.tex}
		\item A box of oranges is inspected by examining three randomly selected oranges drawn without replacement. If all the three oranges are good, the box is approved for sale, otherwise, it is rejected. Find the probability that a box containing 15 oranges out of which 12 are good and 3 are bad ones will be approved for sale.
		\label{ncert/12/13/2/3/defs.tex}
		\item Two balls are drawn at random with replacement from a box containing 10 black and 8 red balls. Find the probability that
		\label{ncert/12/13/2/12}
\begin{enumerate}
\item both balls are red.
\item first ball is black and second is red.
\item one of them is black and other is red.
\end{enumerate}

\item In a hostel, 60\% of the students read Hindi newspaper, 40\% read English newspaper and 20\% read both Hindi and English newspapers. A student is selected at random.
		\label{ncert/12/13/2/15}
\begin{enumerate}
\item Find the probability that she reads neither Hindi nor English newspapers.
\item If she reads Hindi newspaper, find the probability that she reads English newspaper.
\item If she reads English newspaper, find the probability that she reads Hindi newspaper.\\
\end{enumerate}
\item The probability of obtaining an even prime number on each die, when a pair of dice is rolled is 
\begin{enumerate}
    \item $0$ 
    
    \item $\frac{1}{3}$ 
    
    \item $\frac{1}{12}$ 
    
    \item $\frac{1}{36}$ 
\end{enumerate}
\solution
		%\input{ncert/12/13/2/17/defs.tex}
	\item A bag contains 4 red and 4 black balls, another bag contains 2 red and 6 black balls. One of the two bags is selected at random and a ball is drawn from the bag which is found to be red. Find the probability that the ball is drawn from the first bag.
\\
\solution
		%\input{ncert/12/13/3/2/main.tex}
  \item
  Cards with numbers 2 to 101 are placed in a box. A card is selected at random.Find the probability that the card has
\begin{enumerate}[label=(\roman*)]
	\item an even number 
	\item a square number
\end{enumerate}
\solution
%\input{exemplar/10/13/3/32/main.tex}
\item
The king, queen and jack of clubs are removed from a deck of 52 playing cards and then well shuffled. Now one card is drawn at random from the remaining cards.  Determine the probability that the card is
\begin{enumerate}[label=(\roman*)]
\item a club
\item 10 of hearts
\end{enumerate}
\solution
%\input{exemplar/10/13/3/29/main.tex}
\item A team of medical students doing their internship have to assist during surgeries
at a city hospital. The probabilities of surgeries rated as very complex, complex,
routine, simple or very simple are respectively, 0.15, 0.20, 0.31, 0.26, .08. Find
the probabilities that a particular surgery will be rated
\begin{enumerate}
	\item complex or very complex;
	\item neither very complex nor very simple;
	\item routine or complex
	\item routine or simple
\end{enumerate}
\solution
%\input{exemplar/11/16/3/8(1)/main.tex}
\item A card is selected from a pack of 52 cards.
\begin{enumerate}[label=(\alph*)]
    \item How many points are there in the sample space?
    \item Calculate the probability that the card is an ace of spades.
    \item Calculate the probability that the card is (i) an ace and (ii) black card.
\end{enumerate}
\solution
%\input{exemplar/11/16/3/4/main2.tex}
\item The probability that a non leap year selected at random will contain 53 sundays.
\\
\solution
%\input{exemplar/10/13/1/19/main.tex}
\item One of the four persons John, Rita, Aslam or Gurpreet will be promoted next
month. Consequently the sample space consists of four elementary outcomes
S = {John promoted, Rita promoted, Aslam promoted, Gurpreet promoted}
You are told that the chances of John’s promotion is same as that of Gurpreet,
Rita’s chances of promotion are twice as likely as Johns. Aslam’s chances are
four times that of John.
\begin{enumerate}
	\item Determine
	\begin{enumerate}
		\item P (John promoted)
		\item P (Rita promoted)
		\item P (Aslam promoted)
		\item P (Gurpreet promoted)
	\end{enumerate}
	\item If A = {John promoted or Gurpreet promoted}, find P (A).
\end{enumerate}
\solution
%\input{exemplar/11/16/3/10/main.tex}
\item A card is drawn from a deck of 52 cards. Find the probability of getting a king or a heart or a red card.\\
\solution
%\input{exemplar/11/16/3/15/main.tex}
\item The probability that a student will pass his examination is 0.73, the probability of
the student getting a compartment is 0.13, and the probability that the student will
either pass or get compartment is 0.96. State True or False.\\
\solution
%\input{exemplar/11/16/3/31/main.tex}
\item A card is selected from a pack of 52 cards\\
\begin{enumerate}[label=(\alph*)]
\item How many points are there in the sample space?
\item Calculate the probability that the cards is an ace of spades.
\item Calculate the probability that the card is (i) an ace (ii)black card.\\
\end{enumerate}
%\input{ncert/11/16/3/4_1/Prob_4.tex}
\item In a non-leap year, the probability of having 53 tuesdays or 53 wednesdays is\\
\solution
%\input{exemplar/11/16/3/18/main.tex}
\item There are 1000 sealed envelopes in a box, 10 of them contain a cash prize of
Rs 100 each, 100 of them contain a cash prize of Rs 50 each and 200 of them
contain a cash prize of Rs 10 each and rest do not contain any cash prize. If they
are well shuffled and an envelope is picked up out, what is the probability that it
contains no cash prize?\\
\solution
%\input{exemplar/10/13/3/34/main.tex}
\item 
A die is thrown and a card is selected at random from a deck of 52 playing cards. The probability of getting an even number on the die and a spade card.\\
\solution
%\input{exemplar/12/13/3/78/main.tex}
\item
If 4-digit numbers greater than 5,000 are randomly formed from the digits 0, 1, 3, 5, and 7, what is the probability of forming a number divisible by 5 when:
\begin{enumerate}
    \item The digits are repeated?
    \item The repetition of digits is not allowed?
\end{enumerate}
\solution
%\input{ncert/11/16/4/9/main.tex}
\item Consider the probability space $\brak{\Omega, \mathcal{G}, P}$ where $\Omega = [0,2]$ and $\mathcal{G} = \cbrak{\phi, \Omega, [0,1], (1,2]}$. Let $X$ and $Y$ be two functions on $\Omega$ defined as
\begin{align*}
    X(\omega) = 
    \begin{cases}
        1 & \text{if }\omega \in [0, 1]\\
        2 & \text{if }\omega \in (1, 2]
    \end{cases}
\end{align*}
and
\begin{align*}
    Y(\omega) = 
    \begin{cases}
        2 & \text{if }\omega \in [0, 1.5]\\
        3 & \text{if }\omega \in (1.5, 2].
    \end{cases}
\end{align*}
Then which one of the following statements is true?
\begin{enumerate}
    \item [(A)] $X$ is a random variable with respect to $\mathcal{G}$, but $Y$ is not a random variable with respect to $\mathcal{G}$.
    \item [(B)] $Y$ is a random variable with respect to $\mathcal{G}$, but $X$ is not a random variable with respect to $\mathcal{G}$.
    \item [(C)] Neither $X$ nor $Y$ is a random variable with respect to $\mathcal{G}$.
    \item [(D)] Both $X$ and $Y$ are random variables with respect to $\mathcal{G}$.
\end{enumerate} \hfill (GATE ST 2023)\\
\solution
%\input{gate/ST/2023/14/main.tex}
	\item  A die is loaded in such a way that each odd number is twice as likely to occur as
each even number. Find $P(G)$, where $G$ is the event that a number greater than
3 occurs on a single roll of the die.
\\
\solution
		%\input{exemplar/11/16/3/5/main.tex}
	\item All the jacks, queens and kings are removed from a deck of 52 playing cards. The remaining cards are well shuffled and then one card is drawn at random. Giving ace a value 1 similar value for other cards, find the probability that the card has a value 
		\begin{enumerate}
			\item 7
			\item greater than 7
			\item less than 7
		\end{enumerate}
		%\input{exemplar/10/13/3/30/main.tex}
  \item A Lot consists of 48 mobile phones of which 42 are good, 3 have only minor defects and 3 have major defects.Varnika will buy a phone if it is good but the trader will only buy a mobile if it has no major defects. One phone is selected at random from the lot. What is the probability that it is
\begin{enumerate}
	\item acceptable to Varnika?
            \item acceptable to the trader?
\end{enumerate}
\solution
	%\input{exemplar/10/13/3/40/main.tex}
 \item A student says that if you throw a die, it will show up 1 or not 1. Therefore, the probability of getting 1 and the probability of getting 'not 1' each is equal to $\frac{1}{2}$. Is this correct? Give reasons.\\
 \solution
        %\input{exemplar/10/13/2/9/main.tex}
   \item Four candidates A, B, C, D have ap-
plied for the assignment to coach a school cricket
team. If A is twice as likely to be selected as B, and
B and C are given about the same chance of being
selected, while C is twice as likely to be selected
as D, what are the probabilities that
\begin{enumerate}
\item C will be selected?
\item A will not be selected?
\end{enumerate}
	%\input{exemplar/11/16/3/9/main.tex}
 \item A bag contain 24 balls of which $x$ balls are red, $2x$ are white and $3x$ are blue. A ball is selected at random, What is the probability that it is
\begin{enumerate}[label=\alph*)]
\item not red ?
\item white ?
\end{enumerate}
%\input{exemplar/10/13/3/41/main.tex}
If the letters of the word ASSASSINATION are arranged at random. Find the Probability that
\begin{enumerate}[label=(\alph*)]
\item Four $S's$ come consecutively in the word
\item Two  $I's$ and two $N's$ come together
\item All $A's$ are not coming together
\item No two $A's$ are coming together
\end{enumerate}
%\input{exemplar/11/16/3/14/main.tex}
	\item One urn contains two black balls (labelled B1 and B2) and one white ball. A
	second urn contains one black ball and two white balls (labelled W1 and W2).
	Suppose the following experiment is performed. One of the two urns is chosen
	at random. Next a ball is randomly chosen from the urn. Then a second ball is
	chosen at random from the same urn without replacing the first ball.
	
	\begin{enumerate}
	\item What is the probability that two black balls are chosen?
	
	\item What is the probability that two balls of opposite colour are chosen?
	\end{enumerate}
	\solution
	%\input{exemplar/11/16/3/12/main1.tex}
\end{enumerate}

	\item 
The number lock of a suitcase has 4 wheels each labelled with ten digits i.e. from 0 to 9.The lock opens with a sequence of four digits with no repeats.What is the probability of a person getting the right sequence to open the suitcase.
\\
\solution
		%\begin{enumerate}[label=\thesection.\arabic*,ref=\thesection.\theenumi]
	\item One card is drawn from a well-shuffled deck of 52 cards. Find the probability of getting
\begin{enumerate}
\item A king of red colour 
\item A face card 
\item A red face card
\item The jack of hearts
\item A spade
\item The queen of diamonds

\end{enumerate}
\solution
		%\input{ncert/10/15/1/14/main.tex}
	\item Five cards—the ten, jack, queen, king and ace of diamonds, are well-shuffled with their face downwards. One card is then picked up at random.
\begin{enumerate}
\item
What is the probability that the card is the queen? 
\item
If the queen is drawn and put aside, what is the probability that the second card picked up is (a) an ace? (b) a queen?\\
\end{enumerate}
\solution
		%\input{ncert/10/15/1/15/defs.tex}
	\item A bag contains $5$ red balls and some blue balls. If the probability of drawing a blue ball is double that if a red ball, determine the number of blue balls in the bag. 
		\\
\solution
		%\input{ncert/10/15/2/3/defs.tex}
	\item A card is selected from a pack of 52 cards.
 \begin{enumerate}[label=(\alph*)] 
                 \item How many points are there in the sample space?
                 \item Calculate the probability that the card is an ace of spades.
                 \item Calculate the probability that the card is (i) an ace and (ii) black card.
 \end{enumerate}
\solution
		%\input{ncert/11/16/3/4/main.tex}
\item Four cards are drawn from a well-shuffled deck of 52 cards. What is the probability of obtaining 3 diamonds and one spade.
\\
\solution
		%\input{ncert/11/16/4/2/defs.tex}
\item In a certain lottery 10,000 tickets are sold and ten equal prizes are awarded. What is the probability of not getting a prize if you buy (a) one ticket (b) two tickets (c) 10 tickets ?	
\\
\solution
		%\input{ncert/11/16/4/4/defs.tex}
		%
\item 
Out of 100 students, two sections of 40 and 60 are formed. If you and your friend are among the 100 students, what is the probability that
\begin{enumerate}
\item you both enter the same section?
\item you both enter the different sections?
\end{enumerate}
\solution
		%\input{ncert/11/16/4/5/defs.tex}
	\item 
The number lock of a suitcase has 4 wheels each labelled with ten digits i.e. from 0 to 9.The lock opens with a sequence of four digits with no repeats.What is the probability of a person getting the right sequence to open the suitcase.
\\
\solution
		%\input{ncert/11/16/4/10/defs.tex}
		%
\item 
Two cards are drawn at random and without replacement from a pack of 52 playing cards. Find the probability that both the cards are black.
\\
\solution
		%\input{ncert/12/13/2/2/defs.tex}
		\item A box of oranges is inspected by examining three randomly selected oranges drawn without replacement. If all the three oranges are good, the box is approved for sale, otherwise, it is rejected. Find the probability that a box containing 15 oranges out of which 12 are good and 3 are bad ones will be approved for sale.
		\label{ncert/12/13/2/3/defs.tex}
		\item Two balls are drawn at random with replacement from a box containing 10 black and 8 red balls. Find the probability that
		\label{ncert/12/13/2/12}
\begin{enumerate}
\item both balls are red.
\item first ball is black and second is red.
\item one of them is black and other is red.
\end{enumerate}

\item In a hostel, 60\% of the students read Hindi newspaper, 40\% read English newspaper and 20\% read both Hindi and English newspapers. A student is selected at random.
		\label{ncert/12/13/2/15}
\begin{enumerate}
\item Find the probability that she reads neither Hindi nor English newspapers.
\item If she reads Hindi newspaper, find the probability that she reads English newspaper.
\item If she reads English newspaper, find the probability that she reads Hindi newspaper.\\
\end{enumerate}
\item The probability of obtaining an even prime number on each die, when a pair of dice is rolled is 
\begin{enumerate}
    \item $0$ 
    
    \item $\frac{1}{3}$ 
    
    \item $\frac{1}{12}$ 
    
    \item $\frac{1}{36}$ 
\end{enumerate}
\solution
		%\input{ncert/12/13/2/17/defs.tex}
	\item A bag contains 4 red and 4 black balls, another bag contains 2 red and 6 black balls. One of the two bags is selected at random and a ball is drawn from the bag which is found to be red. Find the probability that the ball is drawn from the first bag.
\\
\solution
		%\input{ncert/12/13/3/2/main.tex}
  \item
  Cards with numbers 2 to 101 are placed in a box. A card is selected at random.Find the probability that the card has
\begin{enumerate}[label=(\roman*)]
	\item an even number 
	\item a square number
\end{enumerate}
\solution
%\input{exemplar/10/13/3/32/main.tex}
\item
The king, queen and jack of clubs are removed from a deck of 52 playing cards and then well shuffled. Now one card is drawn at random from the remaining cards.  Determine the probability that the card is
\begin{enumerate}[label=(\roman*)]
\item a club
\item 10 of hearts
\end{enumerate}
\solution
%\input{exemplar/10/13/3/29/main.tex}
\item A team of medical students doing their internship have to assist during surgeries
at a city hospital. The probabilities of surgeries rated as very complex, complex,
routine, simple or very simple are respectively, 0.15, 0.20, 0.31, 0.26, .08. Find
the probabilities that a particular surgery will be rated
\begin{enumerate}
	\item complex or very complex;
	\item neither very complex nor very simple;
	\item routine or complex
	\item routine or simple
\end{enumerate}
\solution
%\input{exemplar/11/16/3/8(1)/main.tex}
\item A card is selected from a pack of 52 cards.
\begin{enumerate}[label=(\alph*)]
    \item How many points are there in the sample space?
    \item Calculate the probability that the card is an ace of spades.
    \item Calculate the probability that the card is (i) an ace and (ii) black card.
\end{enumerate}
\solution
%\input{exemplar/11/16/3/4/main2.tex}
\item The probability that a non leap year selected at random will contain 53 sundays.
\\
\solution
%\input{exemplar/10/13/1/19/main.tex}
\item One of the four persons John, Rita, Aslam or Gurpreet will be promoted next
month. Consequently the sample space consists of four elementary outcomes
S = {John promoted, Rita promoted, Aslam promoted, Gurpreet promoted}
You are told that the chances of John’s promotion is same as that of Gurpreet,
Rita’s chances of promotion are twice as likely as Johns. Aslam’s chances are
four times that of John.
\begin{enumerate}
	\item Determine
	\begin{enumerate}
		\item P (John promoted)
		\item P (Rita promoted)
		\item P (Aslam promoted)
		\item P (Gurpreet promoted)
	\end{enumerate}
	\item If A = {John promoted or Gurpreet promoted}, find P (A).
\end{enumerate}
\solution
%\input{exemplar/11/16/3/10/main.tex}
\item A card is drawn from a deck of 52 cards. Find the probability of getting a king or a heart or a red card.\\
\solution
%\input{exemplar/11/16/3/15/main.tex}
\item The probability that a student will pass his examination is 0.73, the probability of
the student getting a compartment is 0.13, and the probability that the student will
either pass or get compartment is 0.96. State True or False.\\
\solution
%\input{exemplar/11/16/3/31/main.tex}
\item A card is selected from a pack of 52 cards\\
\begin{enumerate}[label=(\alph*)]
\item How many points are there in the sample space?
\item Calculate the probability that the cards is an ace of spades.
\item Calculate the probability that the card is (i) an ace (ii)black card.\\
\end{enumerate}
%\input{ncert/11/16/3/4_1/Prob_4.tex}
\item In a non-leap year, the probability of having 53 tuesdays or 53 wednesdays is\\
\solution
%\input{exemplar/11/16/3/18/main.tex}
\item There are 1000 sealed envelopes in a box, 10 of them contain a cash prize of
Rs 100 each, 100 of them contain a cash prize of Rs 50 each and 200 of them
contain a cash prize of Rs 10 each and rest do not contain any cash prize. If they
are well shuffled and an envelope is picked up out, what is the probability that it
contains no cash prize?\\
\solution
%\input{exemplar/10/13/3/34/main.tex}
\item 
A die is thrown and a card is selected at random from a deck of 52 playing cards. The probability of getting an even number on the die and a spade card.\\
\solution
%\input{exemplar/12/13/3/78/main.tex}
\item
If 4-digit numbers greater than 5,000 are randomly formed from the digits 0, 1, 3, 5, and 7, what is the probability of forming a number divisible by 5 when:
\begin{enumerate}
    \item The digits are repeated?
    \item The repetition of digits is not allowed?
\end{enumerate}
\solution
%\input{ncert/11/16/4/9/main.tex}
\item Consider the probability space $\brak{\Omega, \mathcal{G}, P}$ where $\Omega = [0,2]$ and $\mathcal{G} = \cbrak{\phi, \Omega, [0,1], (1,2]}$. Let $X$ and $Y$ be two functions on $\Omega$ defined as
\begin{align*}
    X(\omega) = 
    \begin{cases}
        1 & \text{if }\omega \in [0, 1]\\
        2 & \text{if }\omega \in (1, 2]
    \end{cases}
\end{align*}
and
\begin{align*}
    Y(\omega) = 
    \begin{cases}
        2 & \text{if }\omega \in [0, 1.5]\\
        3 & \text{if }\omega \in (1.5, 2].
    \end{cases}
\end{align*}
Then which one of the following statements is true?
\begin{enumerate}
    \item [(A)] $X$ is a random variable with respect to $\mathcal{G}$, but $Y$ is not a random variable with respect to $\mathcal{G}$.
    \item [(B)] $Y$ is a random variable with respect to $\mathcal{G}$, but $X$ is not a random variable with respect to $\mathcal{G}$.
    \item [(C)] Neither $X$ nor $Y$ is a random variable with respect to $\mathcal{G}$.
    \item [(D)] Both $X$ and $Y$ are random variables with respect to $\mathcal{G}$.
\end{enumerate} \hfill (GATE ST 2023)\\
\solution
%\input{gate/ST/2023/14/main.tex}
	\item  A die is loaded in such a way that each odd number is twice as likely to occur as
each even number. Find $P(G)$, where $G$ is the event that a number greater than
3 occurs on a single roll of the die.
\\
\solution
		%\input{exemplar/11/16/3/5/main.tex}
	\item All the jacks, queens and kings are removed from a deck of 52 playing cards. The remaining cards are well shuffled and then one card is drawn at random. Giving ace a value 1 similar value for other cards, find the probability that the card has a value 
		\begin{enumerate}
			\item 7
			\item greater than 7
			\item less than 7
		\end{enumerate}
		%\input{exemplar/10/13/3/30/main.tex}
  \item A Lot consists of 48 mobile phones of which 42 are good, 3 have only minor defects and 3 have major defects.Varnika will buy a phone if it is good but the trader will only buy a mobile if it has no major defects. One phone is selected at random from the lot. What is the probability that it is
\begin{enumerate}
	\item acceptable to Varnika?
            \item acceptable to the trader?
\end{enumerate}
\solution
	%\input{exemplar/10/13/3/40/main.tex}
 \item A student says that if you throw a die, it will show up 1 or not 1. Therefore, the probability of getting 1 and the probability of getting 'not 1' each is equal to $\frac{1}{2}$. Is this correct? Give reasons.\\
 \solution
        %\input{exemplar/10/13/2/9/main.tex}
   \item Four candidates A, B, C, D have ap-
plied for the assignment to coach a school cricket
team. If A is twice as likely to be selected as B, and
B and C are given about the same chance of being
selected, while C is twice as likely to be selected
as D, what are the probabilities that
\begin{enumerate}
\item C will be selected?
\item A will not be selected?
\end{enumerate}
	%\input{exemplar/11/16/3/9/main.tex}
 \item A bag contain 24 balls of which $x$ balls are red, $2x$ are white and $3x$ are blue. A ball is selected at random, What is the probability that it is
\begin{enumerate}[label=\alph*)]
\item not red ?
\item white ?
\end{enumerate}
%\input{exemplar/10/13/3/41/main.tex}
If the letters of the word ASSASSINATION are arranged at random. Find the Probability that
\begin{enumerate}[label=(\alph*)]
\item Four $S's$ come consecutively in the word
\item Two  $I's$ and two $N's$ come together
\item All $A's$ are not coming together
\item No two $A's$ are coming together
\end{enumerate}
%\input{exemplar/11/16/3/14/main.tex}
	\item One urn contains two black balls (labelled B1 and B2) and one white ball. A
	second urn contains one black ball and two white balls (labelled W1 and W2).
	Suppose the following experiment is performed. One of the two urns is chosen
	at random. Next a ball is randomly chosen from the urn. Then a second ball is
	chosen at random from the same urn without replacing the first ball.
	
	\begin{enumerate}
	\item What is the probability that two black balls are chosen?
	
	\item What is the probability that two balls of opposite colour are chosen?
	\end{enumerate}
	\solution
	%\input{exemplar/11/16/3/12/main1.tex}
\end{enumerate}

		%
\item 
Two cards are drawn at random and without replacement from a pack of 52 playing cards. Find the probability that both the cards are black.
\\
\solution
		%\begin{enumerate}[label=\thesection.\arabic*,ref=\thesection.\theenumi]
	\item One card is drawn from a well-shuffled deck of 52 cards. Find the probability of getting
\begin{enumerate}
\item A king of red colour 
\item A face card 
\item A red face card
\item The jack of hearts
\item A spade
\item The queen of diamonds

\end{enumerate}
\solution
		%\input{ncert/10/15/1/14/main.tex}
	\item Five cards—the ten, jack, queen, king and ace of diamonds, are well-shuffled with their face downwards. One card is then picked up at random.
\begin{enumerate}
\item
What is the probability that the card is the queen? 
\item
If the queen is drawn and put aside, what is the probability that the second card picked up is (a) an ace? (b) a queen?\\
\end{enumerate}
\solution
		%\input{ncert/10/15/1/15/defs.tex}
	\item A bag contains $5$ red balls and some blue balls. If the probability of drawing a blue ball is double that if a red ball, determine the number of blue balls in the bag. 
		\\
\solution
		%\input{ncert/10/15/2/3/defs.tex}
	\item A card is selected from a pack of 52 cards.
 \begin{enumerate}[label=(\alph*)] 
                 \item How many points are there in the sample space?
                 \item Calculate the probability that the card is an ace of spades.
                 \item Calculate the probability that the card is (i) an ace and (ii) black card.
 \end{enumerate}
\solution
		%\input{ncert/11/16/3/4/main.tex}
\item Four cards are drawn from a well-shuffled deck of 52 cards. What is the probability of obtaining 3 diamonds and one spade.
\\
\solution
		%\input{ncert/11/16/4/2/defs.tex}
\item In a certain lottery 10,000 tickets are sold and ten equal prizes are awarded. What is the probability of not getting a prize if you buy (a) one ticket (b) two tickets (c) 10 tickets ?	
\\
\solution
		%\input{ncert/11/16/4/4/defs.tex}
		%
\item 
Out of 100 students, two sections of 40 and 60 are formed. If you and your friend are among the 100 students, what is the probability that
\begin{enumerate}
\item you both enter the same section?
\item you both enter the different sections?
\end{enumerate}
\solution
		%\input{ncert/11/16/4/5/defs.tex}
	\item 
The number lock of a suitcase has 4 wheels each labelled with ten digits i.e. from 0 to 9.The lock opens with a sequence of four digits with no repeats.What is the probability of a person getting the right sequence to open the suitcase.
\\
\solution
		%\input{ncert/11/16/4/10/defs.tex}
		%
\item 
Two cards are drawn at random and without replacement from a pack of 52 playing cards. Find the probability that both the cards are black.
\\
\solution
		%\input{ncert/12/13/2/2/defs.tex}
		\item A box of oranges is inspected by examining three randomly selected oranges drawn without replacement. If all the three oranges are good, the box is approved for sale, otherwise, it is rejected. Find the probability that a box containing 15 oranges out of which 12 are good and 3 are bad ones will be approved for sale.
		\label{ncert/12/13/2/3/defs.tex}
		\item Two balls are drawn at random with replacement from a box containing 10 black and 8 red balls. Find the probability that
		\label{ncert/12/13/2/12}
\begin{enumerate}
\item both balls are red.
\item first ball is black and second is red.
\item one of them is black and other is red.
\end{enumerate}

\item In a hostel, 60\% of the students read Hindi newspaper, 40\% read English newspaper and 20\% read both Hindi and English newspapers. A student is selected at random.
		\label{ncert/12/13/2/15}
\begin{enumerate}
\item Find the probability that she reads neither Hindi nor English newspapers.
\item If she reads Hindi newspaper, find the probability that she reads English newspaper.
\item If she reads English newspaper, find the probability that she reads Hindi newspaper.\\
\end{enumerate}
\item The probability of obtaining an even prime number on each die, when a pair of dice is rolled is 
\begin{enumerate}
    \item $0$ 
    
    \item $\frac{1}{3}$ 
    
    \item $\frac{1}{12}$ 
    
    \item $\frac{1}{36}$ 
\end{enumerate}
\solution
		%\input{ncert/12/13/2/17/defs.tex}
	\item A bag contains 4 red and 4 black balls, another bag contains 2 red and 6 black balls. One of the two bags is selected at random and a ball is drawn from the bag which is found to be red. Find the probability that the ball is drawn from the first bag.
\\
\solution
		%\input{ncert/12/13/3/2/main.tex}
  \item
  Cards with numbers 2 to 101 are placed in a box. A card is selected at random.Find the probability that the card has
\begin{enumerate}[label=(\roman*)]
	\item an even number 
	\item a square number
\end{enumerate}
\solution
%\input{exemplar/10/13/3/32/main.tex}
\item
The king, queen and jack of clubs are removed from a deck of 52 playing cards and then well shuffled. Now one card is drawn at random from the remaining cards.  Determine the probability that the card is
\begin{enumerate}[label=(\roman*)]
\item a club
\item 10 of hearts
\end{enumerate}
\solution
%\input{exemplar/10/13/3/29/main.tex}
\item A team of medical students doing their internship have to assist during surgeries
at a city hospital. The probabilities of surgeries rated as very complex, complex,
routine, simple or very simple are respectively, 0.15, 0.20, 0.31, 0.26, .08. Find
the probabilities that a particular surgery will be rated
\begin{enumerate}
	\item complex or very complex;
	\item neither very complex nor very simple;
	\item routine or complex
	\item routine or simple
\end{enumerate}
\solution
%\input{exemplar/11/16/3/8(1)/main.tex}
\item A card is selected from a pack of 52 cards.
\begin{enumerate}[label=(\alph*)]
    \item How many points are there in the sample space?
    \item Calculate the probability that the card is an ace of spades.
    \item Calculate the probability that the card is (i) an ace and (ii) black card.
\end{enumerate}
\solution
%\input{exemplar/11/16/3/4/main2.tex}
\item The probability that a non leap year selected at random will contain 53 sundays.
\\
\solution
%\input{exemplar/10/13/1/19/main.tex}
\item One of the four persons John, Rita, Aslam or Gurpreet will be promoted next
month. Consequently the sample space consists of four elementary outcomes
S = {John promoted, Rita promoted, Aslam promoted, Gurpreet promoted}
You are told that the chances of John’s promotion is same as that of Gurpreet,
Rita’s chances of promotion are twice as likely as Johns. Aslam’s chances are
four times that of John.
\begin{enumerate}
	\item Determine
	\begin{enumerate}
		\item P (John promoted)
		\item P (Rita promoted)
		\item P (Aslam promoted)
		\item P (Gurpreet promoted)
	\end{enumerate}
	\item If A = {John promoted or Gurpreet promoted}, find P (A).
\end{enumerate}
\solution
%\input{exemplar/11/16/3/10/main.tex}
\item A card is drawn from a deck of 52 cards. Find the probability of getting a king or a heart or a red card.\\
\solution
%\input{exemplar/11/16/3/15/main.tex}
\item The probability that a student will pass his examination is 0.73, the probability of
the student getting a compartment is 0.13, and the probability that the student will
either pass or get compartment is 0.96. State True or False.\\
\solution
%\input{exemplar/11/16/3/31/main.tex}
\item A card is selected from a pack of 52 cards\\
\begin{enumerate}[label=(\alph*)]
\item How many points are there in the sample space?
\item Calculate the probability that the cards is an ace of spades.
\item Calculate the probability that the card is (i) an ace (ii)black card.\\
\end{enumerate}
%\input{ncert/11/16/3/4_1/Prob_4.tex}
\item In a non-leap year, the probability of having 53 tuesdays or 53 wednesdays is\\
\solution
%\input{exemplar/11/16/3/18/main.tex}
\item There are 1000 sealed envelopes in a box, 10 of them contain a cash prize of
Rs 100 each, 100 of them contain a cash prize of Rs 50 each and 200 of them
contain a cash prize of Rs 10 each and rest do not contain any cash prize. If they
are well shuffled and an envelope is picked up out, what is the probability that it
contains no cash prize?\\
\solution
%\input{exemplar/10/13/3/34/main.tex}
\item 
A die is thrown and a card is selected at random from a deck of 52 playing cards. The probability of getting an even number on the die and a spade card.\\
\solution
%\input{exemplar/12/13/3/78/main.tex}
\item
If 4-digit numbers greater than 5,000 are randomly formed from the digits 0, 1, 3, 5, and 7, what is the probability of forming a number divisible by 5 when:
\begin{enumerate}
    \item The digits are repeated?
    \item The repetition of digits is not allowed?
\end{enumerate}
\solution
%\input{ncert/11/16/4/9/main.tex}
\item Consider the probability space $\brak{\Omega, \mathcal{G}, P}$ where $\Omega = [0,2]$ and $\mathcal{G} = \cbrak{\phi, \Omega, [0,1], (1,2]}$. Let $X$ and $Y$ be two functions on $\Omega$ defined as
\begin{align*}
    X(\omega) = 
    \begin{cases}
        1 & \text{if }\omega \in [0, 1]\\
        2 & \text{if }\omega \in (1, 2]
    \end{cases}
\end{align*}
and
\begin{align*}
    Y(\omega) = 
    \begin{cases}
        2 & \text{if }\omega \in [0, 1.5]\\
        3 & \text{if }\omega \in (1.5, 2].
    \end{cases}
\end{align*}
Then which one of the following statements is true?
\begin{enumerate}
    \item [(A)] $X$ is a random variable with respect to $\mathcal{G}$, but $Y$ is not a random variable with respect to $\mathcal{G}$.
    \item [(B)] $Y$ is a random variable with respect to $\mathcal{G}$, but $X$ is not a random variable with respect to $\mathcal{G}$.
    \item [(C)] Neither $X$ nor $Y$ is a random variable with respect to $\mathcal{G}$.
    \item [(D)] Both $X$ and $Y$ are random variables with respect to $\mathcal{G}$.
\end{enumerate} \hfill (GATE ST 2023)\\
\solution
%\input{gate/ST/2023/14/main.tex}
	\item  A die is loaded in such a way that each odd number is twice as likely to occur as
each even number. Find $P(G)$, where $G$ is the event that a number greater than
3 occurs on a single roll of the die.
\\
\solution
		%\input{exemplar/11/16/3/5/main.tex}
	\item All the jacks, queens and kings are removed from a deck of 52 playing cards. The remaining cards are well shuffled and then one card is drawn at random. Giving ace a value 1 similar value for other cards, find the probability that the card has a value 
		\begin{enumerate}
			\item 7
			\item greater than 7
			\item less than 7
		\end{enumerate}
		%\input{exemplar/10/13/3/30/main.tex}
  \item A Lot consists of 48 mobile phones of which 42 are good, 3 have only minor defects and 3 have major defects.Varnika will buy a phone if it is good but the trader will only buy a mobile if it has no major defects. One phone is selected at random from the lot. What is the probability that it is
\begin{enumerate}
	\item acceptable to Varnika?
            \item acceptable to the trader?
\end{enumerate}
\solution
	%\input{exemplar/10/13/3/40/main.tex}
 \item A student says that if you throw a die, it will show up 1 or not 1. Therefore, the probability of getting 1 and the probability of getting 'not 1' each is equal to $\frac{1}{2}$. Is this correct? Give reasons.\\
 \solution
        %\input{exemplar/10/13/2/9/main.tex}
   \item Four candidates A, B, C, D have ap-
plied for the assignment to coach a school cricket
team. If A is twice as likely to be selected as B, and
B and C are given about the same chance of being
selected, while C is twice as likely to be selected
as D, what are the probabilities that
\begin{enumerate}
\item C will be selected?
\item A will not be selected?
\end{enumerate}
	%\input{exemplar/11/16/3/9/main.tex}
 \item A bag contain 24 balls of which $x$ balls are red, $2x$ are white and $3x$ are blue. A ball is selected at random, What is the probability that it is
\begin{enumerate}[label=\alph*)]
\item not red ?
\item white ?
\end{enumerate}
%\input{exemplar/10/13/3/41/main.tex}
If the letters of the word ASSASSINATION are arranged at random. Find the Probability that
\begin{enumerate}[label=(\alph*)]
\item Four $S's$ come consecutively in the word
\item Two  $I's$ and two $N's$ come together
\item All $A's$ are not coming together
\item No two $A's$ are coming together
\end{enumerate}
%\input{exemplar/11/16/3/14/main.tex}
	\item One urn contains two black balls (labelled B1 and B2) and one white ball. A
	second urn contains one black ball and two white balls (labelled W1 and W2).
	Suppose the following experiment is performed. One of the two urns is chosen
	at random. Next a ball is randomly chosen from the urn. Then a second ball is
	chosen at random from the same urn without replacing the first ball.
	
	\begin{enumerate}
	\item What is the probability that two black balls are chosen?
	
	\item What is the probability that two balls of opposite colour are chosen?
	\end{enumerate}
	\solution
	%\input{exemplar/11/16/3/12/main1.tex}
\end{enumerate}

		\item A box of oranges is inspected by examining three randomly selected oranges drawn without replacement. If all the three oranges are good, the box is approved for sale, otherwise, it is rejected. Find the probability that a box containing 15 oranges out of which 12 are good and 3 are bad ones will be approved for sale.
		\label{ncert/12/13/2/3/defs.tex}
		\item Two balls are drawn at random with replacement from a box containing 10 black and 8 red balls. Find the probability that
		\label{ncert/12/13/2/12}
\begin{enumerate}
\item both balls are red.
\item first ball is black and second is red.
\item one of them is black and other is red.
\end{enumerate}

\item In a hostel, 60\% of the students read Hindi newspaper, 40\% read English newspaper and 20\% read both Hindi and English newspapers. A student is selected at random.
		\label{ncert/12/13/2/15}
\begin{enumerate}
\item Find the probability that she reads neither Hindi nor English newspapers.
\item If she reads Hindi newspaper, find the probability that she reads English newspaper.
\item If she reads English newspaper, find the probability that she reads Hindi newspaper.\\
\end{enumerate}
\item The probability of obtaining an even prime number on each die, when a pair of dice is rolled is 
\begin{enumerate}
    \item $0$ 
    
    \item $\frac{1}{3}$ 
    
    \item $\frac{1}{12}$ 
    
    \item $\frac{1}{36}$ 
\end{enumerate}
\solution
		%\begin{enumerate}[label=\thesection.\arabic*,ref=\thesection.\theenumi]
	\item One card is drawn from a well-shuffled deck of 52 cards. Find the probability of getting
\begin{enumerate}
\item A king of red colour 
\item A face card 
\item A red face card
\item The jack of hearts
\item A spade
\item The queen of diamonds

\end{enumerate}
\solution
		%\input{ncert/10/15/1/14/main.tex}
	\item Five cards—the ten, jack, queen, king and ace of diamonds, are well-shuffled with their face downwards. One card is then picked up at random.
\begin{enumerate}
\item
What is the probability that the card is the queen? 
\item
If the queen is drawn and put aside, what is the probability that the second card picked up is (a) an ace? (b) a queen?\\
\end{enumerate}
\solution
		%\input{ncert/10/15/1/15/defs.tex}
	\item A bag contains $5$ red balls and some blue balls. If the probability of drawing a blue ball is double that if a red ball, determine the number of blue balls in the bag. 
		\\
\solution
		%\input{ncert/10/15/2/3/defs.tex}
	\item A card is selected from a pack of 52 cards.
 \begin{enumerate}[label=(\alph*)] 
                 \item How many points are there in the sample space?
                 \item Calculate the probability that the card is an ace of spades.
                 \item Calculate the probability that the card is (i) an ace and (ii) black card.
 \end{enumerate}
\solution
		%\input{ncert/11/16/3/4/main.tex}
\item Four cards are drawn from a well-shuffled deck of 52 cards. What is the probability of obtaining 3 diamonds and one spade.
\\
\solution
		%\input{ncert/11/16/4/2/defs.tex}
\item In a certain lottery 10,000 tickets are sold and ten equal prizes are awarded. What is the probability of not getting a prize if you buy (a) one ticket (b) two tickets (c) 10 tickets ?	
\\
\solution
		%\input{ncert/11/16/4/4/defs.tex}
		%
\item 
Out of 100 students, two sections of 40 and 60 are formed. If you and your friend are among the 100 students, what is the probability that
\begin{enumerate}
\item you both enter the same section?
\item you both enter the different sections?
\end{enumerate}
\solution
		%\input{ncert/11/16/4/5/defs.tex}
	\item 
The number lock of a suitcase has 4 wheels each labelled with ten digits i.e. from 0 to 9.The lock opens with a sequence of four digits with no repeats.What is the probability of a person getting the right sequence to open the suitcase.
\\
\solution
		%\input{ncert/11/16/4/10/defs.tex}
		%
\item 
Two cards are drawn at random and without replacement from a pack of 52 playing cards. Find the probability that both the cards are black.
\\
\solution
		%\input{ncert/12/13/2/2/defs.tex}
		\item A box of oranges is inspected by examining three randomly selected oranges drawn without replacement. If all the three oranges are good, the box is approved for sale, otherwise, it is rejected. Find the probability that a box containing 15 oranges out of which 12 are good and 3 are bad ones will be approved for sale.
		\label{ncert/12/13/2/3/defs.tex}
		\item Two balls are drawn at random with replacement from a box containing 10 black and 8 red balls. Find the probability that
		\label{ncert/12/13/2/12}
\begin{enumerate}
\item both balls are red.
\item first ball is black and second is red.
\item one of them is black and other is red.
\end{enumerate}

\item In a hostel, 60\% of the students read Hindi newspaper, 40\% read English newspaper and 20\% read both Hindi and English newspapers. A student is selected at random.
		\label{ncert/12/13/2/15}
\begin{enumerate}
\item Find the probability that she reads neither Hindi nor English newspapers.
\item If she reads Hindi newspaper, find the probability that she reads English newspaper.
\item If she reads English newspaper, find the probability that she reads Hindi newspaper.\\
\end{enumerate}
\item The probability of obtaining an even prime number on each die, when a pair of dice is rolled is 
\begin{enumerate}
    \item $0$ 
    
    \item $\frac{1}{3}$ 
    
    \item $\frac{1}{12}$ 
    
    \item $\frac{1}{36}$ 
\end{enumerate}
\solution
		%\input{ncert/12/13/2/17/defs.tex}
	\item A bag contains 4 red and 4 black balls, another bag contains 2 red and 6 black balls. One of the two bags is selected at random and a ball is drawn from the bag which is found to be red. Find the probability that the ball is drawn from the first bag.
\\
\solution
		%\input{ncert/12/13/3/2/main.tex}
  \item
  Cards with numbers 2 to 101 are placed in a box. A card is selected at random.Find the probability that the card has
\begin{enumerate}[label=(\roman*)]
	\item an even number 
	\item a square number
\end{enumerate}
\solution
%\input{exemplar/10/13/3/32/main.tex}
\item
The king, queen and jack of clubs are removed from a deck of 52 playing cards and then well shuffled. Now one card is drawn at random from the remaining cards.  Determine the probability that the card is
\begin{enumerate}[label=(\roman*)]
\item a club
\item 10 of hearts
\end{enumerate}
\solution
%\input{exemplar/10/13/3/29/main.tex}
\item A team of medical students doing their internship have to assist during surgeries
at a city hospital. The probabilities of surgeries rated as very complex, complex,
routine, simple or very simple are respectively, 0.15, 0.20, 0.31, 0.26, .08. Find
the probabilities that a particular surgery will be rated
\begin{enumerate}
	\item complex or very complex;
	\item neither very complex nor very simple;
	\item routine or complex
	\item routine or simple
\end{enumerate}
\solution
%\input{exemplar/11/16/3/8(1)/main.tex}
\item A card is selected from a pack of 52 cards.
\begin{enumerate}[label=(\alph*)]
    \item How many points are there in the sample space?
    \item Calculate the probability that the card is an ace of spades.
    \item Calculate the probability that the card is (i) an ace and (ii) black card.
\end{enumerate}
\solution
%\input{exemplar/11/16/3/4/main2.tex}
\item The probability that a non leap year selected at random will contain 53 sundays.
\\
\solution
%\input{exemplar/10/13/1/19/main.tex}
\item One of the four persons John, Rita, Aslam or Gurpreet will be promoted next
month. Consequently the sample space consists of four elementary outcomes
S = {John promoted, Rita promoted, Aslam promoted, Gurpreet promoted}
You are told that the chances of John’s promotion is same as that of Gurpreet,
Rita’s chances of promotion are twice as likely as Johns. Aslam’s chances are
four times that of John.
\begin{enumerate}
	\item Determine
	\begin{enumerate}
		\item P (John promoted)
		\item P (Rita promoted)
		\item P (Aslam promoted)
		\item P (Gurpreet promoted)
	\end{enumerate}
	\item If A = {John promoted or Gurpreet promoted}, find P (A).
\end{enumerate}
\solution
%\input{exemplar/11/16/3/10/main.tex}
\item A card is drawn from a deck of 52 cards. Find the probability of getting a king or a heart or a red card.\\
\solution
%\input{exemplar/11/16/3/15/main.tex}
\item The probability that a student will pass his examination is 0.73, the probability of
the student getting a compartment is 0.13, and the probability that the student will
either pass or get compartment is 0.96. State True or False.\\
\solution
%\input{exemplar/11/16/3/31/main.tex}
\item A card is selected from a pack of 52 cards\\
\begin{enumerate}[label=(\alph*)]
\item How many points are there in the sample space?
\item Calculate the probability that the cards is an ace of spades.
\item Calculate the probability that the card is (i) an ace (ii)black card.\\
\end{enumerate}
%\input{ncert/11/16/3/4_1/Prob_4.tex}
\item In a non-leap year, the probability of having 53 tuesdays or 53 wednesdays is\\
\solution
%\input{exemplar/11/16/3/18/main.tex}
\item There are 1000 sealed envelopes in a box, 10 of them contain a cash prize of
Rs 100 each, 100 of them contain a cash prize of Rs 50 each and 200 of them
contain a cash prize of Rs 10 each and rest do not contain any cash prize. If they
are well shuffled and an envelope is picked up out, what is the probability that it
contains no cash prize?\\
\solution
%\input{exemplar/10/13/3/34/main.tex}
\item 
A die is thrown and a card is selected at random from a deck of 52 playing cards. The probability of getting an even number on the die and a spade card.\\
\solution
%\input{exemplar/12/13/3/78/main.tex}
\item
If 4-digit numbers greater than 5,000 are randomly formed from the digits 0, 1, 3, 5, and 7, what is the probability of forming a number divisible by 5 when:
\begin{enumerate}
    \item The digits are repeated?
    \item The repetition of digits is not allowed?
\end{enumerate}
\solution
%\input{ncert/11/16/4/9/main.tex}
\item Consider the probability space $\brak{\Omega, \mathcal{G}, P}$ where $\Omega = [0,2]$ and $\mathcal{G} = \cbrak{\phi, \Omega, [0,1], (1,2]}$. Let $X$ and $Y$ be two functions on $\Omega$ defined as
\begin{align*}
    X(\omega) = 
    \begin{cases}
        1 & \text{if }\omega \in [0, 1]\\
        2 & \text{if }\omega \in (1, 2]
    \end{cases}
\end{align*}
and
\begin{align*}
    Y(\omega) = 
    \begin{cases}
        2 & \text{if }\omega \in [0, 1.5]\\
        3 & \text{if }\omega \in (1.5, 2].
    \end{cases}
\end{align*}
Then which one of the following statements is true?
\begin{enumerate}
    \item [(A)] $X$ is a random variable with respect to $\mathcal{G}$, but $Y$ is not a random variable with respect to $\mathcal{G}$.
    \item [(B)] $Y$ is a random variable with respect to $\mathcal{G}$, but $X$ is not a random variable with respect to $\mathcal{G}$.
    \item [(C)] Neither $X$ nor $Y$ is a random variable with respect to $\mathcal{G}$.
    \item [(D)] Both $X$ and $Y$ are random variables with respect to $\mathcal{G}$.
\end{enumerate} \hfill (GATE ST 2023)\\
\solution
%\input{gate/ST/2023/14/main.tex}
	\item  A die is loaded in such a way that each odd number is twice as likely to occur as
each even number. Find $P(G)$, where $G$ is the event that a number greater than
3 occurs on a single roll of the die.
\\
\solution
		%\input{exemplar/11/16/3/5/main.tex}
	\item All the jacks, queens and kings are removed from a deck of 52 playing cards. The remaining cards are well shuffled and then one card is drawn at random. Giving ace a value 1 similar value for other cards, find the probability that the card has a value 
		\begin{enumerate}
			\item 7
			\item greater than 7
			\item less than 7
		\end{enumerate}
		%\input{exemplar/10/13/3/30/main.tex}
  \item A Lot consists of 48 mobile phones of which 42 are good, 3 have only minor defects and 3 have major defects.Varnika will buy a phone if it is good but the trader will only buy a mobile if it has no major defects. One phone is selected at random from the lot. What is the probability that it is
\begin{enumerate}
	\item acceptable to Varnika?
            \item acceptable to the trader?
\end{enumerate}
\solution
	%\input{exemplar/10/13/3/40/main.tex}
 \item A student says that if you throw a die, it will show up 1 or not 1. Therefore, the probability of getting 1 and the probability of getting 'not 1' each is equal to $\frac{1}{2}$. Is this correct? Give reasons.\\
 \solution
        %\input{exemplar/10/13/2/9/main.tex}
   \item Four candidates A, B, C, D have ap-
plied for the assignment to coach a school cricket
team. If A is twice as likely to be selected as B, and
B and C are given about the same chance of being
selected, while C is twice as likely to be selected
as D, what are the probabilities that
\begin{enumerate}
\item C will be selected?
\item A will not be selected?
\end{enumerate}
	%\input{exemplar/11/16/3/9/main.tex}
 \item A bag contain 24 balls of which $x$ balls are red, $2x$ are white and $3x$ are blue. A ball is selected at random, What is the probability that it is
\begin{enumerate}[label=\alph*)]
\item not red ?
\item white ?
\end{enumerate}
%\input{exemplar/10/13/3/41/main.tex}
If the letters of the word ASSASSINATION are arranged at random. Find the Probability that
\begin{enumerate}[label=(\alph*)]
\item Four $S's$ come consecutively in the word
\item Two  $I's$ and two $N's$ come together
\item All $A's$ are not coming together
\item No two $A's$ are coming together
\end{enumerate}
%\input{exemplar/11/16/3/14/main.tex}
	\item One urn contains two black balls (labelled B1 and B2) and one white ball. A
	second urn contains one black ball and two white balls (labelled W1 and W2).
	Suppose the following experiment is performed. One of the two urns is chosen
	at random. Next a ball is randomly chosen from the urn. Then a second ball is
	chosen at random from the same urn without replacing the first ball.
	
	\begin{enumerate}
	\item What is the probability that two black balls are chosen?
	
	\item What is the probability that two balls of opposite colour are chosen?
	\end{enumerate}
	\solution
	%\input{exemplar/11/16/3/12/main1.tex}
\end{enumerate}

	\item A bag contains 4 red and 4 black balls, another bag contains 2 red and 6 black balls. One of the two bags is selected at random and a ball is drawn from the bag which is found to be red. Find the probability that the ball is drawn from the first bag.
\\
\solution
		%\begin{table}[H]
	\centering
\begin{tabular}{|c|c|c|}
\hline
Random variable &Value &Definition\\ \hline
\multirow{3}{*}{X} &0 &Slips of Rs 1\\
&1 &Slips of Rs 5\\
&2 &Slips of Rs 13\\ \hline
\multirow{2}{*}{Y} &0 &Box A\\
&1 &Box B\\\hline
\end{tabular}
\caption{}
\label{tab:Distribution}
\end{table}
See \tabref{tab:Distribution}.
\begin{align}
p_{Y}\brak{k}= \begin{cases} 
      \frac{1}{3} & {k=0} \\
      \frac{2}{3 }& {k=1} 
   \end{cases}
   \\
p_{Y|X}\brak{0|0} = \frac{19}{25}\, 
p_{Y|X}\brak{0|1} = \frac{6}{25}\,
p_{Y|X}\brak{1|0} = \frac{45}{50}\,
p_{Y|X}\brak{1|2} = \frac{5}{50}
\end{align}
The desired probability is the probability that a slip drawn at random is marked other than Rs 1,
\begin{align}
&=1-p_X\brak{0}\\
&= p_X(1) + p_X(2)
\end{align}
Using Bayes theorem,
\begin{align}
&= p_Y\brak{0} \times \pr{Y=0 | X=1} + p_Y\brak{1} \times \pr{Y=1|X=2}\\
&=\frac{1}{3} \times \frac{6}{25} + \frac{2}{3} \times \frac{5}{50}\\
&=\frac{11}{75}
\end{align}

\newpage

%\tableofcontents

\bigskip

\renewcommand{\thefigure}{\theenumi}
\renewcommand{\thetable}{\theenumi}
%\renewcommand{\theequation}{\theenumi}

%\begin{abstract}
%%\boldmath
%In this letter, an algorithm for evaluating the exact analytical bit error rate  (BER)  for the piecewise linear (PL) combiner for  multiple relays is presented. Previous results were available only for upto three relays. The algorithm is unique in the sense that  the actual mathematical expressions, that are prohibitively large, need not be explicitly obtained. The diversity gain due to multiple relays is shown through plots of the analytical BER, well supported by simulations. 
%
%\end{abstract}
% IEEEtran.cls defaults to using nonbold math in the Abstract.
% This preserves the distinction between vectors and scalars. However,
% if the journal you are submitting to favors bold math in the abstract,
% then you can use LaTeX's standard command \boldmath at the very start
% of the abstract to achieve this. Many IEEE journals frown on math
% in the abstract anyway.

% Note that keywords are not normally used for peerreview papers.
%\begin{IEEEkeywords}
%Cooperative diversity, decode and forward, piecewise linear
%\end{IEEEkeywords}



% For peer review papers, you can put extra information on the cover
% page as needed:
% \ifCLASSOPTIONpeerreview
% \begin{center} \bfseries EDICS Category: 3-BBND \end{center}
% \fi
%
% For peerreview papers, this IEEEtran command inserts a page break and
% creates the second title. It will be ignored for other modes.
%\IEEEpeerreviewmaketitle




  \item
  Cards with numbers 2 to 101 are placed in a box. A card is selected at random.Find the probability that the card has
\begin{enumerate}[label=(\roman*)]
	\item an even number 
	\item a square number
\end{enumerate}
\solution
%\begin{table}[H]
	\centering
\begin{tabular}{|c|c|c|}
\hline
Random variable &Value &Definition\\ \hline
\multirow{3}{*}{X} &0 &Slips of Rs 1\\
&1 &Slips of Rs 5\\
&2 &Slips of Rs 13\\ \hline
\multirow{2}{*}{Y} &0 &Box A\\
&1 &Box B\\\hline
\end{tabular}
\caption{}
\label{tab:Distribution}
\end{table}
See \tabref{tab:Distribution}.
\begin{align}
p_{Y}\brak{k}= \begin{cases} 
      \frac{1}{3} & {k=0} \\
      \frac{2}{3 }& {k=1} 
   \end{cases}
   \\
p_{Y|X}\brak{0|0} = \frac{19}{25}\, 
p_{Y|X}\brak{0|1} = \frac{6}{25}\,
p_{Y|X}\brak{1|0} = \frac{45}{50}\,
p_{Y|X}\brak{1|2} = \frac{5}{50}
\end{align}
The desired probability is the probability that a slip drawn at random is marked other than Rs 1,
\begin{align}
&=1-p_X\brak{0}\\
&= p_X(1) + p_X(2)
\end{align}
Using Bayes theorem,
\begin{align}
&= p_Y\brak{0} \times \pr{Y=0 | X=1} + p_Y\brak{1} \times \pr{Y=1|X=2}\\
&=\frac{1}{3} \times \frac{6}{25} + \frac{2}{3} \times \frac{5}{50}\\
&=\frac{11}{75}
\end{align}

\newpage

%\tableofcontents

\bigskip

\renewcommand{\thefigure}{\theenumi}
\renewcommand{\thetable}{\theenumi}
%\renewcommand{\theequation}{\theenumi}

%\begin{abstract}
%%\boldmath
%In this letter, an algorithm for evaluating the exact analytical bit error rate  (BER)  for the piecewise linear (PL) combiner for  multiple relays is presented. Previous results were available only for upto three relays. The algorithm is unique in the sense that  the actual mathematical expressions, that are prohibitively large, need not be explicitly obtained. The diversity gain due to multiple relays is shown through plots of the analytical BER, well supported by simulations. 
%
%\end{abstract}
% IEEEtran.cls defaults to using nonbold math in the Abstract.
% This preserves the distinction between vectors and scalars. However,
% if the journal you are submitting to favors bold math in the abstract,
% then you can use LaTeX's standard command \boldmath at the very start
% of the abstract to achieve this. Many IEEE journals frown on math
% in the abstract anyway.

% Note that keywords are not normally used for peerreview papers.
%\begin{IEEEkeywords}
%Cooperative diversity, decode and forward, piecewise linear
%\end{IEEEkeywords}



% For peer review papers, you can put extra information on the cover
% page as needed:
% \ifCLASSOPTIONpeerreview
% \begin{center} \bfseries EDICS Category: 3-BBND \end{center}
% \fi
%
% For peerreview papers, this IEEEtran command inserts a page break and
% creates the second title. It will be ignored for other modes.
%\IEEEpeerreviewmaketitle




\item
The king, queen and jack of clubs are removed from a deck of 52 playing cards and then well shuffled. Now one card is drawn at random from the remaining cards.  Determine the probability that the card is
\begin{enumerate}[label=(\roman*)]
\item a club
\item 10 of hearts
\end{enumerate}
\solution
%\begin{table}[H]
	\centering
\begin{tabular}{|c|c|c|}
\hline
Random variable &Value &Definition\\ \hline
\multirow{3}{*}{X} &0 &Slips of Rs 1\\
&1 &Slips of Rs 5\\
&2 &Slips of Rs 13\\ \hline
\multirow{2}{*}{Y} &0 &Box A\\
&1 &Box B\\\hline
\end{tabular}
\caption{}
\label{tab:Distribution}
\end{table}
See \tabref{tab:Distribution}.
\begin{align}
p_{Y}\brak{k}= \begin{cases} 
      \frac{1}{3} & {k=0} \\
      \frac{2}{3 }& {k=1} 
   \end{cases}
   \\
p_{Y|X}\brak{0|0} = \frac{19}{25}\, 
p_{Y|X}\brak{0|1} = \frac{6}{25}\,
p_{Y|X}\brak{1|0} = \frac{45}{50}\,
p_{Y|X}\brak{1|2} = \frac{5}{50}
\end{align}
The desired probability is the probability that a slip drawn at random is marked other than Rs 1,
\begin{align}
&=1-p_X\brak{0}\\
&= p_X(1) + p_X(2)
\end{align}
Using Bayes theorem,
\begin{align}
&= p_Y\brak{0} \times \pr{Y=0 | X=1} + p_Y\brak{1} \times \pr{Y=1|X=2}\\
&=\frac{1}{3} \times \frac{6}{25} + \frac{2}{3} \times \frac{5}{50}\\
&=\frac{11}{75}
\end{align}

\newpage

%\tableofcontents

\bigskip

\renewcommand{\thefigure}{\theenumi}
\renewcommand{\thetable}{\theenumi}
%\renewcommand{\theequation}{\theenumi}

%\begin{abstract}
%%\boldmath
%In this letter, an algorithm for evaluating the exact analytical bit error rate  (BER)  for the piecewise linear (PL) combiner for  multiple relays is presented. Previous results were available only for upto three relays. The algorithm is unique in the sense that  the actual mathematical expressions, that are prohibitively large, need not be explicitly obtained. The diversity gain due to multiple relays is shown through plots of the analytical BER, well supported by simulations. 
%
%\end{abstract}
% IEEEtran.cls defaults to using nonbold math in the Abstract.
% This preserves the distinction between vectors and scalars. However,
% if the journal you are submitting to favors bold math in the abstract,
% then you can use LaTeX's standard command \boldmath at the very start
% of the abstract to achieve this. Many IEEE journals frown on math
% in the abstract anyway.

% Note that keywords are not normally used for peerreview papers.
%\begin{IEEEkeywords}
%Cooperative diversity, decode and forward, piecewise linear
%\end{IEEEkeywords}



% For peer review papers, you can put extra information on the cover
% page as needed:
% \ifCLASSOPTIONpeerreview
% \begin{center} \bfseries EDICS Category: 3-BBND \end{center}
% \fi
%
% For peerreview papers, this IEEEtran command inserts a page break and
% creates the second title. It will be ignored for other modes.
%\IEEEpeerreviewmaketitle




\item A team of medical students doing their internship have to assist during surgeries
at a city hospital. The probabilities of surgeries rated as very complex, complex,
routine, simple or very simple are respectively, 0.15, 0.20, 0.31, 0.26, .08. Find
the probabilities that a particular surgery will be rated
\begin{enumerate}
	\item complex or very complex;
	\item neither very complex nor very simple;
	\item routine or complex
	\item routine or simple
\end{enumerate}
\solution
%\begin{table}[H]
	\centering
\begin{tabular}{|c|c|c|}
\hline
Random variable &Value &Definition\\ \hline
\multirow{3}{*}{X} &0 &Slips of Rs 1\\
&1 &Slips of Rs 5\\
&2 &Slips of Rs 13\\ \hline
\multirow{2}{*}{Y} &0 &Box A\\
&1 &Box B\\\hline
\end{tabular}
\caption{}
\label{tab:Distribution}
\end{table}
See \tabref{tab:Distribution}.
\begin{align}
p_{Y}\brak{k}= \begin{cases} 
      \frac{1}{3} & {k=0} \\
      \frac{2}{3 }& {k=1} 
   \end{cases}
   \\
p_{Y|X}\brak{0|0} = \frac{19}{25}\, 
p_{Y|X}\brak{0|1} = \frac{6}{25}\,
p_{Y|X}\brak{1|0} = \frac{45}{50}\,
p_{Y|X}\brak{1|2} = \frac{5}{50}
\end{align}
The desired probability is the probability that a slip drawn at random is marked other than Rs 1,
\begin{align}
&=1-p_X\brak{0}\\
&= p_X(1) + p_X(2)
\end{align}
Using Bayes theorem,
\begin{align}
&= p_Y\brak{0} \times \pr{Y=0 | X=1} + p_Y\brak{1} \times \pr{Y=1|X=2}\\
&=\frac{1}{3} \times \frac{6}{25} + \frac{2}{3} \times \frac{5}{50}\\
&=\frac{11}{75}
\end{align}

\newpage

%\tableofcontents

\bigskip

\renewcommand{\thefigure}{\theenumi}
\renewcommand{\thetable}{\theenumi}
%\renewcommand{\theequation}{\theenumi}

%\begin{abstract}
%%\boldmath
%In this letter, an algorithm for evaluating the exact analytical bit error rate  (BER)  for the piecewise linear (PL) combiner for  multiple relays is presented. Previous results were available only for upto three relays. The algorithm is unique in the sense that  the actual mathematical expressions, that are prohibitively large, need not be explicitly obtained. The diversity gain due to multiple relays is shown through plots of the analytical BER, well supported by simulations. 
%
%\end{abstract}
% IEEEtran.cls defaults to using nonbold math in the Abstract.
% This preserves the distinction between vectors and scalars. However,
% if the journal you are submitting to favors bold math in the abstract,
% then you can use LaTeX's standard command \boldmath at the very start
% of the abstract to achieve this. Many IEEE journals frown on math
% in the abstract anyway.

% Note that keywords are not normally used for peerreview papers.
%\begin{IEEEkeywords}
%Cooperative diversity, decode and forward, piecewise linear
%\end{IEEEkeywords}



% For peer review papers, you can put extra information on the cover
% page as needed:
% \ifCLASSOPTIONpeerreview
% \begin{center} \bfseries EDICS Category: 3-BBND \end{center}
% \fi
%
% For peerreview papers, this IEEEtran command inserts a page break and
% creates the second title. It will be ignored for other modes.
%\IEEEpeerreviewmaketitle




\item A card is selected from a pack of 52 cards.
\begin{enumerate}[label=(\alph*)]
    \item How many points are there in the sample space?
    \item Calculate the probability that the card is an ace of spades.
    \item Calculate the probability that the card is (i) an ace and (ii) black card.
\end{enumerate}
\solution
%Let $X$ be an bernoulli rv defined as in \tabref{tab:exemplar/11/16/3/26}.  Then, 
\begin{equation}
    p =
        \frac{4}{11} 
\end{equation}
\begin{table}[H]
	\centering
	\input{exemplar/11/16/3/26/tables/Table2.tex}
	\caption{}
        \label{tab:exemplar/11/16/3/26}
\end{table}

\item The probability that a non leap year selected at random will contain 53 sundays.
\\
\solution
%\begin{table}[H]
	\centering
\begin{tabular}{|c|c|c|}
\hline
Random variable &Value &Definition\\ \hline
\multirow{3}{*}{X} &0 &Slips of Rs 1\\
&1 &Slips of Rs 5\\
&2 &Slips of Rs 13\\ \hline
\multirow{2}{*}{Y} &0 &Box A\\
&1 &Box B\\\hline
\end{tabular}
\caption{}
\label{tab:Distribution}
\end{table}
See \tabref{tab:Distribution}.
\begin{align}
p_{Y}\brak{k}= \begin{cases} 
      \frac{1}{3} & {k=0} \\
      \frac{2}{3 }& {k=1} 
   \end{cases}
   \\
p_{Y|X}\brak{0|0} = \frac{19}{25}\, 
p_{Y|X}\brak{0|1} = \frac{6}{25}\,
p_{Y|X}\brak{1|0} = \frac{45}{50}\,
p_{Y|X}\brak{1|2} = \frac{5}{50}
\end{align}
The desired probability is the probability that a slip drawn at random is marked other than Rs 1,
\begin{align}
&=1-p_X\brak{0}\\
&= p_X(1) + p_X(2)
\end{align}
Using Bayes theorem,
\begin{align}
&= p_Y\brak{0} \times \pr{Y=0 | X=1} + p_Y\brak{1} \times \pr{Y=1|X=2}\\
&=\frac{1}{3} \times \frac{6}{25} + \frac{2}{3} \times \frac{5}{50}\\
&=\frac{11}{75}
\end{align}

\newpage

%\tableofcontents

\bigskip

\renewcommand{\thefigure}{\theenumi}
\renewcommand{\thetable}{\theenumi}
%\renewcommand{\theequation}{\theenumi}

%\begin{abstract}
%%\boldmath
%In this letter, an algorithm for evaluating the exact analytical bit error rate  (BER)  for the piecewise linear (PL) combiner for  multiple relays is presented. Previous results were available only for upto three relays. The algorithm is unique in the sense that  the actual mathematical expressions, that are prohibitively large, need not be explicitly obtained. The diversity gain due to multiple relays is shown through plots of the analytical BER, well supported by simulations. 
%
%\end{abstract}
% IEEEtran.cls defaults to using nonbold math in the Abstract.
% This preserves the distinction between vectors and scalars. However,
% if the journal you are submitting to favors bold math in the abstract,
% then you can use LaTeX's standard command \boldmath at the very start
% of the abstract to achieve this. Many IEEE journals frown on math
% in the abstract anyway.

% Note that keywords are not normally used for peerreview papers.
%\begin{IEEEkeywords}
%Cooperative diversity, decode and forward, piecewise linear
%\end{IEEEkeywords}



% For peer review papers, you can put extra information on the cover
% page as needed:
% \ifCLASSOPTIONpeerreview
% \begin{center} \bfseries EDICS Category: 3-BBND \end{center}
% \fi
%
% For peerreview papers, this IEEEtran command inserts a page break and
% creates the second title. It will be ignored for other modes.
%\IEEEpeerreviewmaketitle




\item One of the four persons John, Rita, Aslam or Gurpreet will be promoted next
month. Consequently the sample space consists of four elementary outcomes
S = {John promoted, Rita promoted, Aslam promoted, Gurpreet promoted}
You are told that the chances of John’s promotion is same as that of Gurpreet,
Rita’s chances of promotion are twice as likely as Johns. Aslam’s chances are
four times that of John.
\begin{enumerate}
	\item Determine
	\begin{enumerate}
		\item P (John promoted)
		\item P (Rita promoted)
		\item P (Aslam promoted)
		\item P (Gurpreet promoted)
	\end{enumerate}
	\item If A = {John promoted or Gurpreet promoted}, find P (A).
\end{enumerate}
\solution
%\begin{table}[H]
	\centering
\begin{tabular}{|c|c|c|}
\hline
Random variable &Value &Definition\\ \hline
\multirow{3}{*}{X} &0 &Slips of Rs 1\\
&1 &Slips of Rs 5\\
&2 &Slips of Rs 13\\ \hline
\multirow{2}{*}{Y} &0 &Box A\\
&1 &Box B\\\hline
\end{tabular}
\caption{}
\label{tab:Distribution}
\end{table}
See \tabref{tab:Distribution}.
\begin{align}
p_{Y}\brak{k}= \begin{cases} 
      \frac{1}{3} & {k=0} \\
      \frac{2}{3 }& {k=1} 
   \end{cases}
   \\
p_{Y|X}\brak{0|0} = \frac{19}{25}\, 
p_{Y|X}\brak{0|1} = \frac{6}{25}\,
p_{Y|X}\brak{1|0} = \frac{45}{50}\,
p_{Y|X}\brak{1|2} = \frac{5}{50}
\end{align}
The desired probability is the probability that a slip drawn at random is marked other than Rs 1,
\begin{align}
&=1-p_X\brak{0}\\
&= p_X(1) + p_X(2)
\end{align}
Using Bayes theorem,
\begin{align}
&= p_Y\brak{0} \times \pr{Y=0 | X=1} + p_Y\brak{1} \times \pr{Y=1|X=2}\\
&=\frac{1}{3} \times \frac{6}{25} + \frac{2}{3} \times \frac{5}{50}\\
&=\frac{11}{75}
\end{align}

\newpage

%\tableofcontents

\bigskip

\renewcommand{\thefigure}{\theenumi}
\renewcommand{\thetable}{\theenumi}
%\renewcommand{\theequation}{\theenumi}

%\begin{abstract}
%%\boldmath
%In this letter, an algorithm for evaluating the exact analytical bit error rate  (BER)  for the piecewise linear (PL) combiner for  multiple relays is presented. Previous results were available only for upto three relays. The algorithm is unique in the sense that  the actual mathematical expressions, that are prohibitively large, need not be explicitly obtained. The diversity gain due to multiple relays is shown through plots of the analytical BER, well supported by simulations. 
%
%\end{abstract}
% IEEEtran.cls defaults to using nonbold math in the Abstract.
% This preserves the distinction between vectors and scalars. However,
% if the journal you are submitting to favors bold math in the abstract,
% then you can use LaTeX's standard command \boldmath at the very start
% of the abstract to achieve this. Many IEEE journals frown on math
% in the abstract anyway.

% Note that keywords are not normally used for peerreview papers.
%\begin{IEEEkeywords}
%Cooperative diversity, decode and forward, piecewise linear
%\end{IEEEkeywords}



% For peer review papers, you can put extra information on the cover
% page as needed:
% \ifCLASSOPTIONpeerreview
% \begin{center} \bfseries EDICS Category: 3-BBND \end{center}
% \fi
%
% For peerreview papers, this IEEEtran command inserts a page break and
% creates the second title. It will be ignored for other modes.
%\IEEEpeerreviewmaketitle




\item A card is drawn from a deck of 52 cards. Find the probability of getting a king or a heart or a red card.\\
\solution
%\begin{table}[H]
	\centering
\begin{tabular}{|c|c|c|}
\hline
Random variable &Value &Definition\\ \hline
\multirow{3}{*}{X} &0 &Slips of Rs 1\\
&1 &Slips of Rs 5\\
&2 &Slips of Rs 13\\ \hline
\multirow{2}{*}{Y} &0 &Box A\\
&1 &Box B\\\hline
\end{tabular}
\caption{}
\label{tab:Distribution}
\end{table}
See \tabref{tab:Distribution}.
\begin{align}
p_{Y}\brak{k}= \begin{cases} 
      \frac{1}{3} & {k=0} \\
      \frac{2}{3 }& {k=1} 
   \end{cases}
   \\
p_{Y|X}\brak{0|0} = \frac{19}{25}\, 
p_{Y|X}\brak{0|1} = \frac{6}{25}\,
p_{Y|X}\brak{1|0} = \frac{45}{50}\,
p_{Y|X}\brak{1|2} = \frac{5}{50}
\end{align}
The desired probability is the probability that a slip drawn at random is marked other than Rs 1,
\begin{align}
&=1-p_X\brak{0}\\
&= p_X(1) + p_X(2)
\end{align}
Using Bayes theorem,
\begin{align}
&= p_Y\brak{0} \times \pr{Y=0 | X=1} + p_Y\brak{1} \times \pr{Y=1|X=2}\\
&=\frac{1}{3} \times \frac{6}{25} + \frac{2}{3} \times \frac{5}{50}\\
&=\frac{11}{75}
\end{align}

\newpage

%\tableofcontents

\bigskip

\renewcommand{\thefigure}{\theenumi}
\renewcommand{\thetable}{\theenumi}
%\renewcommand{\theequation}{\theenumi}

%\begin{abstract}
%%\boldmath
%In this letter, an algorithm for evaluating the exact analytical bit error rate  (BER)  for the piecewise linear (PL) combiner for  multiple relays is presented. Previous results were available only for upto three relays. The algorithm is unique in the sense that  the actual mathematical expressions, that are prohibitively large, need not be explicitly obtained. The diversity gain due to multiple relays is shown through plots of the analytical BER, well supported by simulations. 
%
%\end{abstract}
% IEEEtran.cls defaults to using nonbold math in the Abstract.
% This preserves the distinction between vectors and scalars. However,
% if the journal you are submitting to favors bold math in the abstract,
% then you can use LaTeX's standard command \boldmath at the very start
% of the abstract to achieve this. Many IEEE journals frown on math
% in the abstract anyway.

% Note that keywords are not normally used for peerreview papers.
%\begin{IEEEkeywords}
%Cooperative diversity, decode and forward, piecewise linear
%\end{IEEEkeywords}



% For peer review papers, you can put extra information on the cover
% page as needed:
% \ifCLASSOPTIONpeerreview
% \begin{center} \bfseries EDICS Category: 3-BBND \end{center}
% \fi
%
% For peerreview papers, this IEEEtran command inserts a page break and
% creates the second title. It will be ignored for other modes.
%\IEEEpeerreviewmaketitle




\item The probability that a student will pass his examination is 0.73, the probability of
the student getting a compartment is 0.13, and the probability that the student will
either pass or get compartment is 0.96. State True or False.\\
\solution
%\begin{table}[H]
	\centering
\begin{tabular}{|c|c|c|}
\hline
Random variable &Value &Definition\\ \hline
\multirow{3}{*}{X} &0 &Slips of Rs 1\\
&1 &Slips of Rs 5\\
&2 &Slips of Rs 13\\ \hline
\multirow{2}{*}{Y} &0 &Box A\\
&1 &Box B\\\hline
\end{tabular}
\caption{}
\label{tab:Distribution}
\end{table}
See \tabref{tab:Distribution}.
\begin{align}
p_{Y}\brak{k}= \begin{cases} 
      \frac{1}{3} & {k=0} \\
      \frac{2}{3 }& {k=1} 
   \end{cases}
   \\
p_{Y|X}\brak{0|0} = \frac{19}{25}\, 
p_{Y|X}\brak{0|1} = \frac{6}{25}\,
p_{Y|X}\brak{1|0} = \frac{45}{50}\,
p_{Y|X}\brak{1|2} = \frac{5}{50}
\end{align}
The desired probability is the probability that a slip drawn at random is marked other than Rs 1,
\begin{align}
&=1-p_X\brak{0}\\
&= p_X(1) + p_X(2)
\end{align}
Using Bayes theorem,
\begin{align}
&= p_Y\brak{0} \times \pr{Y=0 | X=1} + p_Y\brak{1} \times \pr{Y=1|X=2}\\
&=\frac{1}{3} \times \frac{6}{25} + \frac{2}{3} \times \frac{5}{50}\\
&=\frac{11}{75}
\end{align}

\newpage

%\tableofcontents

\bigskip

\renewcommand{\thefigure}{\theenumi}
\renewcommand{\thetable}{\theenumi}
%\renewcommand{\theequation}{\theenumi}

%\begin{abstract}
%%\boldmath
%In this letter, an algorithm for evaluating the exact analytical bit error rate  (BER)  for the piecewise linear (PL) combiner for  multiple relays is presented. Previous results were available only for upto three relays. The algorithm is unique in the sense that  the actual mathematical expressions, that are prohibitively large, need not be explicitly obtained. The diversity gain due to multiple relays is shown through plots of the analytical BER, well supported by simulations. 
%
%\end{abstract}
% IEEEtran.cls defaults to using nonbold math in the Abstract.
% This preserves the distinction between vectors and scalars. However,
% if the journal you are submitting to favors bold math in the abstract,
% then you can use LaTeX's standard command \boldmath at the very start
% of the abstract to achieve this. Many IEEE journals frown on math
% in the abstract anyway.

% Note that keywords are not normally used for peerreview papers.
%\begin{IEEEkeywords}
%Cooperative diversity, decode and forward, piecewise linear
%\end{IEEEkeywords}



% For peer review papers, you can put extra information on the cover
% page as needed:
% \ifCLASSOPTIONpeerreview
% \begin{center} \bfseries EDICS Category: 3-BBND \end{center}
% \fi
%
% For peerreview papers, this IEEEtran command inserts a page break and
% creates the second title. It will be ignored for other modes.
%\IEEEpeerreviewmaketitle




\item A card is selected from a pack of 52 cards\\
\begin{enumerate}[label=(\alph*)]
\item How many points are there in the sample space?
\item Calculate the probability that the cards is an ace of spades.
\item Calculate the probability that the card is (i) an ace (ii)black card.\\
\end{enumerate}
%\input{ncert/11/16/3/4_1/Prob_4.tex}
\item In a non-leap year, the probability of having 53 tuesdays or 53 wednesdays is\\
\solution
%A non-leap year has a total of 365 days, and a week has 7 days.\\
So it can be expressed as 
\begin{align}
365\text{days} &=52\times 7+1 \text{day}
\end{align}
$\implies$ 52 tuesdays or wednesdays\\
Random variable X denotes the days of a week
\begin{align}
p_X\brak{k}&=\frac{1}{7}; \quad \brak{1<k<7}
\end{align}
So the probability of extra day being tuesday or wednesday is
\begin{align}
p_X\brak{3}+p_X\brak{4}&=\frac{1}{7}+\frac{1}{7}=\frac{2}{7}
\end{align}



\item There are 1000 sealed envelopes in a box, 10 of them contain a cash prize of
Rs 100 each, 100 of them contain a cash prize of Rs 50 each and 200 of them
contain a cash prize of Rs 10 each and rest do not contain any cash prize. If they
are well shuffled and an envelope is picked up out, what is the probability that it
contains no cash prize?\\
\solution
%\begin{table}[H]
	\centering
\begin{tabular}{|c|c|c|}
\hline
Random variable &Value &Definition\\ \hline
\multirow{3}{*}{X} &0 &Slips of Rs 1\\
&1 &Slips of Rs 5\\
&2 &Slips of Rs 13\\ \hline
\multirow{2}{*}{Y} &0 &Box A\\
&1 &Box B\\\hline
\end{tabular}
\caption{}
\label{tab:Distribution}
\end{table}
See \tabref{tab:Distribution}.
\begin{align}
p_{Y}\brak{k}= \begin{cases} 
      \frac{1}{3} & {k=0} \\
      \frac{2}{3 }& {k=1} 
   \end{cases}
   \\
p_{Y|X}\brak{0|0} = \frac{19}{25}\, 
p_{Y|X}\brak{0|1} = \frac{6}{25}\,
p_{Y|X}\brak{1|0} = \frac{45}{50}\,
p_{Y|X}\brak{1|2} = \frac{5}{50}
\end{align}
The desired probability is the probability that a slip drawn at random is marked other than Rs 1,
\begin{align}
&=1-p_X\brak{0}\\
&= p_X(1) + p_X(2)
\end{align}
Using Bayes theorem,
\begin{align}
&= p_Y\brak{0} \times \pr{Y=0 | X=1} + p_Y\brak{1} \times \pr{Y=1|X=2}\\
&=\frac{1}{3} \times \frac{6}{25} + \frac{2}{3} \times \frac{5}{50}\\
&=\frac{11}{75}
\end{align}

\newpage

%\tableofcontents

\bigskip

\renewcommand{\thefigure}{\theenumi}
\renewcommand{\thetable}{\theenumi}
%\renewcommand{\theequation}{\theenumi}

%\begin{abstract}
%%\boldmath
%In this letter, an algorithm for evaluating the exact analytical bit error rate  (BER)  for the piecewise linear (PL) combiner for  multiple relays is presented. Previous results were available only for upto three relays. The algorithm is unique in the sense that  the actual mathematical expressions, that are prohibitively large, need not be explicitly obtained. The diversity gain due to multiple relays is shown through plots of the analytical BER, well supported by simulations. 
%
%\end{abstract}
% IEEEtran.cls defaults to using nonbold math in the Abstract.
% This preserves the distinction between vectors and scalars. However,
% if the journal you are submitting to favors bold math in the abstract,
% then you can use LaTeX's standard command \boldmath at the very start
% of the abstract to achieve this. Many IEEE journals frown on math
% in the abstract anyway.

% Note that keywords are not normally used for peerreview papers.
%\begin{IEEEkeywords}
%Cooperative diversity, decode and forward, piecewise linear
%\end{IEEEkeywords}



% For peer review papers, you can put extra information on the cover
% page as needed:
% \ifCLASSOPTIONpeerreview
% \begin{center} \bfseries EDICS Category: 3-BBND \end{center}
% \fi
%
% For peerreview papers, this IEEEtran command inserts a page break and
% creates the second title. It will be ignored for other modes.
%\IEEEpeerreviewmaketitle




\item 
A die is thrown and a card is selected at random from a deck of 52 playing cards. The probability of getting an even number on the die and a spade card.\\
\solution
%\begin{table}[H]
	\centering
\begin{tabular}{|c|c|c|}
\hline
Random variable &Value &Definition\\ \hline
\multirow{3}{*}{X} &0 &Slips of Rs 1\\
&1 &Slips of Rs 5\\
&2 &Slips of Rs 13\\ \hline
\multirow{2}{*}{Y} &0 &Box A\\
&1 &Box B\\\hline
\end{tabular}
\caption{}
\label{tab:Distribution}
\end{table}
See \tabref{tab:Distribution}.
\begin{align}
p_{Y}\brak{k}= \begin{cases} 
      \frac{1}{3} & {k=0} \\
      \frac{2}{3 }& {k=1} 
   \end{cases}
   \\
p_{Y|X}\brak{0|0} = \frac{19}{25}\, 
p_{Y|X}\brak{0|1} = \frac{6}{25}\,
p_{Y|X}\brak{1|0} = \frac{45}{50}\,
p_{Y|X}\brak{1|2} = \frac{5}{50}
\end{align}
The desired probability is the probability that a slip drawn at random is marked other than Rs 1,
\begin{align}
&=1-p_X\brak{0}\\
&= p_X(1) + p_X(2)
\end{align}
Using Bayes theorem,
\begin{align}
&= p_Y\brak{0} \times \pr{Y=0 | X=1} + p_Y\brak{1} \times \pr{Y=1|X=2}\\
&=\frac{1}{3} \times \frac{6}{25} + \frac{2}{3} \times \frac{5}{50}\\
&=\frac{11}{75}
\end{align}

\newpage

%\tableofcontents

\bigskip

\renewcommand{\thefigure}{\theenumi}
\renewcommand{\thetable}{\theenumi}
%\renewcommand{\theequation}{\theenumi}

%\begin{abstract}
%%\boldmath
%In this letter, an algorithm for evaluating the exact analytical bit error rate  (BER)  for the piecewise linear (PL) combiner for  multiple relays is presented. Previous results were available only for upto three relays. The algorithm is unique in the sense that  the actual mathematical expressions, that are prohibitively large, need not be explicitly obtained. The diversity gain due to multiple relays is shown through plots of the analytical BER, well supported by simulations. 
%
%\end{abstract}
% IEEEtran.cls defaults to using nonbold math in the Abstract.
% This preserves the distinction between vectors and scalars. However,
% if the journal you are submitting to favors bold math in the abstract,
% then you can use LaTeX's standard command \boldmath at the very start
% of the abstract to achieve this. Many IEEE journals frown on math
% in the abstract anyway.

% Note that keywords are not normally used for peerreview papers.
%\begin{IEEEkeywords}
%Cooperative diversity, decode and forward, piecewise linear
%\end{IEEEkeywords}



% For peer review papers, you can put extra information on the cover
% page as needed:
% \ifCLASSOPTIONpeerreview
% \begin{center} \bfseries EDICS Category: 3-BBND \end{center}
% \fi
%
% For peerreview papers, this IEEEtran command inserts a page break and
% creates the second title. It will be ignored for other modes.
%\IEEEpeerreviewmaketitle




\item
If 4-digit numbers greater than 5,000 are randomly formed from the digits 0, 1, 3, 5, and 7, what is the probability of forming a number divisible by 5 when:
\begin{enumerate}
    \item The digits are repeated?
    \item The repetition of digits is not allowed?
\end{enumerate}
\solution
%\begin{table}[H]
	\centering
\begin{tabular}{|c|c|c|}
\hline
Random variable &Value &Definition\\ \hline
\multirow{3}{*}{X} &0 &Slips of Rs 1\\
&1 &Slips of Rs 5\\
&2 &Slips of Rs 13\\ \hline
\multirow{2}{*}{Y} &0 &Box A\\
&1 &Box B\\\hline
\end{tabular}
\caption{}
\label{tab:Distribution}
\end{table}
See \tabref{tab:Distribution}.
\begin{align}
p_{Y}\brak{k}= \begin{cases} 
      \frac{1}{3} & {k=0} \\
      \frac{2}{3 }& {k=1} 
   \end{cases}
   \\
p_{Y|X}\brak{0|0} = \frac{19}{25}\, 
p_{Y|X}\brak{0|1} = \frac{6}{25}\,
p_{Y|X}\brak{1|0} = \frac{45}{50}\,
p_{Y|X}\brak{1|2} = \frac{5}{50}
\end{align}
The desired probability is the probability that a slip drawn at random is marked other than Rs 1,
\begin{align}
&=1-p_X\brak{0}\\
&= p_X(1) + p_X(2)
\end{align}
Using Bayes theorem,
\begin{align}
&= p_Y\brak{0} \times \pr{Y=0 | X=1} + p_Y\brak{1} \times \pr{Y=1|X=2}\\
&=\frac{1}{3} \times \frac{6}{25} + \frac{2}{3} \times \frac{5}{50}\\
&=\frac{11}{75}
\end{align}

\newpage

%\tableofcontents

\bigskip

\renewcommand{\thefigure}{\theenumi}
\renewcommand{\thetable}{\theenumi}
%\renewcommand{\theequation}{\theenumi}

%\begin{abstract}
%%\boldmath
%In this letter, an algorithm for evaluating the exact analytical bit error rate  (BER)  for the piecewise linear (PL) combiner for  multiple relays is presented. Previous results were available only for upto three relays. The algorithm is unique in the sense that  the actual mathematical expressions, that are prohibitively large, need not be explicitly obtained. The diversity gain due to multiple relays is shown through plots of the analytical BER, well supported by simulations. 
%
%\end{abstract}
% IEEEtran.cls defaults to using nonbold math in the Abstract.
% This preserves the distinction between vectors and scalars. However,
% if the journal you are submitting to favors bold math in the abstract,
% then you can use LaTeX's standard command \boldmath at the very start
% of the abstract to achieve this. Many IEEE journals frown on math
% in the abstract anyway.

% Note that keywords are not normally used for peerreview papers.
%\begin{IEEEkeywords}
%Cooperative diversity, decode and forward, piecewise linear
%\end{IEEEkeywords}



% For peer review papers, you can put extra information on the cover
% page as needed:
% \ifCLASSOPTIONpeerreview
% \begin{center} \bfseries EDICS Category: 3-BBND \end{center}
% \fi
%
% For peerreview papers, this IEEEtran command inserts a page break and
% creates the second title. It will be ignored for other modes.
%\IEEEpeerreviewmaketitle




\item Consider the probability space $\brak{\Omega, \mathcal{G}, P}$ where $\Omega = [0,2]$ and $\mathcal{G} = \cbrak{\phi, \Omega, [0,1], (1,2]}$. Let $X$ and $Y$ be two functions on $\Omega$ defined as
\begin{align*}
    X(\omega) = 
    \begin{cases}
        1 & \text{if }\omega \in [0, 1]\\
        2 & \text{if }\omega \in (1, 2]
    \end{cases}
\end{align*}
and
\begin{align*}
    Y(\omega) = 
    \begin{cases}
        2 & \text{if }\omega \in [0, 1.5]\\
        3 & \text{if }\omega \in (1.5, 2].
    \end{cases}
\end{align*}
Then which one of the following statements is true?
\begin{enumerate}
    \item [(A)] $X$ is a random variable with respect to $\mathcal{G}$, but $Y$ is not a random variable with respect to $\mathcal{G}$.
    \item [(B)] $Y$ is a random variable with respect to $\mathcal{G}$, but $X$ is not a random variable with respect to $\mathcal{G}$.
    \item [(C)] Neither $X$ nor $Y$ is a random variable with respect to $\mathcal{G}$.
    \item [(D)] Both $X$ and $Y$ are random variables with respect to $\mathcal{G}$.
\end{enumerate} \hfill (GATE ST 2023)\\
\solution
%\begin{table}[H]
	\centering
\begin{tabular}{|c|c|c|}
\hline
Random variable &Value &Definition\\ \hline
\multirow{3}{*}{X} &0 &Slips of Rs 1\\
&1 &Slips of Rs 5\\
&2 &Slips of Rs 13\\ \hline
\multirow{2}{*}{Y} &0 &Box A\\
&1 &Box B\\\hline
\end{tabular}
\caption{}
\label{tab:Distribution}
\end{table}
See \tabref{tab:Distribution}.
\begin{align}
p_{Y}\brak{k}= \begin{cases} 
      \frac{1}{3} & {k=0} \\
      \frac{2}{3 }& {k=1} 
   \end{cases}
   \\
p_{Y|X}\brak{0|0} = \frac{19}{25}\, 
p_{Y|X}\brak{0|1} = \frac{6}{25}\,
p_{Y|X}\brak{1|0} = \frac{45}{50}\,
p_{Y|X}\brak{1|2} = \frac{5}{50}
\end{align}
The desired probability is the probability that a slip drawn at random is marked other than Rs 1,
\begin{align}
&=1-p_X\brak{0}\\
&= p_X(1) + p_X(2)
\end{align}
Using Bayes theorem,
\begin{align}
&= p_Y\brak{0} \times \pr{Y=0 | X=1} + p_Y\brak{1} \times \pr{Y=1|X=2}\\
&=\frac{1}{3} \times \frac{6}{25} + \frac{2}{3} \times \frac{5}{50}\\
&=\frac{11}{75}
\end{align}

\newpage

%\tableofcontents

\bigskip

\renewcommand{\thefigure}{\theenumi}
\renewcommand{\thetable}{\theenumi}
%\renewcommand{\theequation}{\theenumi}

%\begin{abstract}
%%\boldmath
%In this letter, an algorithm for evaluating the exact analytical bit error rate  (BER)  for the piecewise linear (PL) combiner for  multiple relays is presented. Previous results were available only for upto three relays. The algorithm is unique in the sense that  the actual mathematical expressions, that are prohibitively large, need not be explicitly obtained. The diversity gain due to multiple relays is shown through plots of the analytical BER, well supported by simulations. 
%
%\end{abstract}
% IEEEtran.cls defaults to using nonbold math in the Abstract.
% This preserves the distinction between vectors and scalars. However,
% if the journal you are submitting to favors bold math in the abstract,
% then you can use LaTeX's standard command \boldmath at the very start
% of the abstract to achieve this. Many IEEE journals frown on math
% in the abstract anyway.

% Note that keywords are not normally used for peerreview papers.
%\begin{IEEEkeywords}
%Cooperative diversity, decode and forward, piecewise linear
%\end{IEEEkeywords}



% For peer review papers, you can put extra information on the cover
% page as needed:
% \ifCLASSOPTIONpeerreview
% \begin{center} \bfseries EDICS Category: 3-BBND \end{center}
% \fi
%
% For peerreview papers, this IEEEtran command inserts a page break and
% creates the second title. It will be ignored for other modes.
%\IEEEpeerreviewmaketitle




	\item  A die is loaded in such a way that each odd number is twice as likely to occur as
each even number. Find $P(G)$, where $G$ is the event that a number greater than
3 occurs on a single roll of the die.
\\
\solution
		%\begin{table}[H]
	\centering
\begin{tabular}{|c|c|c|}
\hline
Random variable &Value &Definition\\ \hline
\multirow{3}{*}{X} &0 &Slips of Rs 1\\
&1 &Slips of Rs 5\\
&2 &Slips of Rs 13\\ \hline
\multirow{2}{*}{Y} &0 &Box A\\
&1 &Box B\\\hline
\end{tabular}
\caption{}
\label{tab:Distribution}
\end{table}
See \tabref{tab:Distribution}.
\begin{align}
p_{Y}\brak{k}= \begin{cases} 
      \frac{1}{3} & {k=0} \\
      \frac{2}{3 }& {k=1} 
   \end{cases}
   \\
p_{Y|X}\brak{0|0} = \frac{19}{25}\, 
p_{Y|X}\brak{0|1} = \frac{6}{25}\,
p_{Y|X}\brak{1|0} = \frac{45}{50}\,
p_{Y|X}\brak{1|2} = \frac{5}{50}
\end{align}
The desired probability is the probability that a slip drawn at random is marked other than Rs 1,
\begin{align}
&=1-p_X\brak{0}\\
&= p_X(1) + p_X(2)
\end{align}
Using Bayes theorem,
\begin{align}
&= p_Y\brak{0} \times \pr{Y=0 | X=1} + p_Y\brak{1} \times \pr{Y=1|X=2}\\
&=\frac{1}{3} \times \frac{6}{25} + \frac{2}{3} \times \frac{5}{50}\\
&=\frac{11}{75}
\end{align}

\newpage

%\tableofcontents

\bigskip

\renewcommand{\thefigure}{\theenumi}
\renewcommand{\thetable}{\theenumi}
%\renewcommand{\theequation}{\theenumi}

%\begin{abstract}
%%\boldmath
%In this letter, an algorithm for evaluating the exact analytical bit error rate  (BER)  for the piecewise linear (PL) combiner for  multiple relays is presented. Previous results were available only for upto three relays. The algorithm is unique in the sense that  the actual mathematical expressions, that are prohibitively large, need not be explicitly obtained. The diversity gain due to multiple relays is shown through plots of the analytical BER, well supported by simulations. 
%
%\end{abstract}
% IEEEtran.cls defaults to using nonbold math in the Abstract.
% This preserves the distinction between vectors and scalars. However,
% if the journal you are submitting to favors bold math in the abstract,
% then you can use LaTeX's standard command \boldmath at the very start
% of the abstract to achieve this. Many IEEE journals frown on math
% in the abstract anyway.

% Note that keywords are not normally used for peerreview papers.
%\begin{IEEEkeywords}
%Cooperative diversity, decode and forward, piecewise linear
%\end{IEEEkeywords}



% For peer review papers, you can put extra information on the cover
% page as needed:
% \ifCLASSOPTIONpeerreview
% \begin{center} \bfseries EDICS Category: 3-BBND \end{center}
% \fi
%
% For peerreview papers, this IEEEtran command inserts a page break and
% creates the second title. It will be ignored for other modes.
%\IEEEpeerreviewmaketitle




	\item All the jacks, queens and kings are removed from a deck of 52 playing cards. The remaining cards are well shuffled and then one card is drawn at random. Giving ace a value 1 similar value for other cards, find the probability that the card has a value 
		\begin{enumerate}
			\item 7
			\item greater than 7
			\item less than 7
		\end{enumerate}
		%Number of cards left after removing all jacks, queens and kings 
\begin{align}
N	= 52 - 4\times 3
	= 40
\end{align}
%\begin{table}[H]
%\def\arraystretch{1.2}
%\begin{tabular}{|c|c|c|}
%\hline
%	\textbf{Parameter} &\textbf{Value} &\textbf{Description}\\ \hline
%	$X$ &1-10 &Represents the value of the card picked \\ \hline
%\end{tabular}
%\end{table}
Let $1 \le X \le 10$ be the value of the card picked.  Then,
\begin{align}
	p_X(k) &= \Pr(X=k)\ \forall\ 1 \leq k \leq 10\\
	&= \frac{4\times 1}{40}\\
	&= \frac{1}{10}\\
	\therefore p_X(k) &= 
	\begin{cases}
		\frac{1}{10} & 1 \leq k \leq 10\\
		0 & \text{otherwise}
	\end{cases}
\end{align}
and
\begin{align}
	F_{X}(k) &= \sum_{m=0}^{k}p_{X}(m) \quad 1 \leq k \leq 10\\
	&= \frac{k}{10}\\
	\therefore F_{X}(k) &= 
	\begin{cases}
		0 & k \leq 0\\
		\frac{k}{10} & 1\leq k \leq 10\\
		1 & k > 10 
	\end{cases}
\end{align}
\begin{enumerate}
	\item Probability that card has value equal to 7 is
		\begin{align}
			 p_{X}(7)
			= \frac{1}{10}
		\end{align}
	\item Probability that card has value greater than 7 is
		\begin{align}
			1 - F_X(7)
			&= 1 - \frac{7}{10}
			\\
			&= \frac{3}{10}
		\end{align}
	\item Probability that card has value less than 7 is
		\begin{align}
			 F_{X}(6)
			=\frac{6}{10}
		\end{align}
\end{enumerate}

  \item A Lot consists of 48 mobile phones of which 42 are good, 3 have only minor defects and 3 have major defects.Varnika will buy a phone if it is good but the trader will only buy a mobile if it has no major defects. One phone is selected at random from the lot. What is the probability that it is
\begin{enumerate}
	\item acceptable to Varnika?
            \item acceptable to the trader?
\end{enumerate}
\solution
	%\begin{table}[H]
	\centering
\begin{tabular}{|c|c|c|}
\hline
Random variable &Value &Definition\\ \hline
\multirow{3}{*}{X} &0 &Slips of Rs 1\\
&1 &Slips of Rs 5\\
&2 &Slips of Rs 13\\ \hline
\multirow{2}{*}{Y} &0 &Box A\\
&1 &Box B\\\hline
\end{tabular}
\caption{}
\label{tab:Distribution}
\end{table}
See \tabref{tab:Distribution}.
\begin{align}
p_{Y}\brak{k}= \begin{cases} 
      \frac{1}{3} & {k=0} \\
      \frac{2}{3 }& {k=1} 
   \end{cases}
   \\
p_{Y|X}\brak{0|0} = \frac{19}{25}\, 
p_{Y|X}\brak{0|1} = \frac{6}{25}\,
p_{Y|X}\brak{1|0} = \frac{45}{50}\,
p_{Y|X}\brak{1|2} = \frac{5}{50}
\end{align}
The desired probability is the probability that a slip drawn at random is marked other than Rs 1,
\begin{align}
&=1-p_X\brak{0}\\
&= p_X(1) + p_X(2)
\end{align}
Using Bayes theorem,
\begin{align}
&= p_Y\brak{0} \times \pr{Y=0 | X=1} + p_Y\brak{1} \times \pr{Y=1|X=2}\\
&=\frac{1}{3} \times \frac{6}{25} + \frac{2}{3} \times \frac{5}{50}\\
&=\frac{11}{75}
\end{align}

\newpage

%\tableofcontents

\bigskip

\renewcommand{\thefigure}{\theenumi}
\renewcommand{\thetable}{\theenumi}
%\renewcommand{\theequation}{\theenumi}

%\begin{abstract}
%%\boldmath
%In this letter, an algorithm for evaluating the exact analytical bit error rate  (BER)  for the piecewise linear (PL) combiner for  multiple relays is presented. Previous results were available only for upto three relays. The algorithm is unique in the sense that  the actual mathematical expressions, that are prohibitively large, need not be explicitly obtained. The diversity gain due to multiple relays is shown through plots of the analytical BER, well supported by simulations. 
%
%\end{abstract}
% IEEEtran.cls defaults to using nonbold math in the Abstract.
% This preserves the distinction between vectors and scalars. However,
% if the journal you are submitting to favors bold math in the abstract,
% then you can use LaTeX's standard command \boldmath at the very start
% of the abstract to achieve this. Many IEEE journals frown on math
% in the abstract anyway.

% Note that keywords are not normally used for peerreview papers.
%\begin{IEEEkeywords}
%Cooperative diversity, decode and forward, piecewise linear
%\end{IEEEkeywords}



% For peer review papers, you can put extra information on the cover
% page as needed:
% \ifCLASSOPTIONpeerreview
% \begin{center} \bfseries EDICS Category: 3-BBND \end{center}
% \fi
%
% For peerreview papers, this IEEEtran command inserts a page break and
% creates the second title. It will be ignored for other modes.
%\IEEEpeerreviewmaketitle




 \item A student says that if you throw a die, it will show up 1 or not 1. Therefore, the probability of getting 1 and the probability of getting 'not 1' each is equal to $\frac{1}{2}$. Is this correct? Give reasons.\\
 \solution
        %\begin{table}[H]
	\centering
\begin{tabular}{|c|c|c|}
\hline
Random variable &Value &Definition\\ \hline
\multirow{3}{*}{X} &0 &Slips of Rs 1\\
&1 &Slips of Rs 5\\
&2 &Slips of Rs 13\\ \hline
\multirow{2}{*}{Y} &0 &Box A\\
&1 &Box B\\\hline
\end{tabular}
\caption{}
\label{tab:Distribution}
\end{table}
See \tabref{tab:Distribution}.
\begin{align}
p_{Y}\brak{k}= \begin{cases} 
      \frac{1}{3} & {k=0} \\
      \frac{2}{3 }& {k=1} 
   \end{cases}
   \\
p_{Y|X}\brak{0|0} = \frac{19}{25}\, 
p_{Y|X}\brak{0|1} = \frac{6}{25}\,
p_{Y|X}\brak{1|0} = \frac{45}{50}\,
p_{Y|X}\brak{1|2} = \frac{5}{50}
\end{align}
The desired probability is the probability that a slip drawn at random is marked other than Rs 1,
\begin{align}
&=1-p_X\brak{0}\\
&= p_X(1) + p_X(2)
\end{align}
Using Bayes theorem,
\begin{align}
&= p_Y\brak{0} \times \pr{Y=0 | X=1} + p_Y\brak{1} \times \pr{Y=1|X=2}\\
&=\frac{1}{3} \times \frac{6}{25} + \frac{2}{3} \times \frac{5}{50}\\
&=\frac{11}{75}
\end{align}

\newpage

%\tableofcontents

\bigskip

\renewcommand{\thefigure}{\theenumi}
\renewcommand{\thetable}{\theenumi}
%\renewcommand{\theequation}{\theenumi}

%\begin{abstract}
%%\boldmath
%In this letter, an algorithm for evaluating the exact analytical bit error rate  (BER)  for the piecewise linear (PL) combiner for  multiple relays is presented. Previous results were available only for upto three relays. The algorithm is unique in the sense that  the actual mathematical expressions, that are prohibitively large, need not be explicitly obtained. The diversity gain due to multiple relays is shown through plots of the analytical BER, well supported by simulations. 
%
%\end{abstract}
% IEEEtran.cls defaults to using nonbold math in the Abstract.
% This preserves the distinction between vectors and scalars. However,
% if the journal you are submitting to favors bold math in the abstract,
% then you can use LaTeX's standard command \boldmath at the very start
% of the abstract to achieve this. Many IEEE journals frown on math
% in the abstract anyway.

% Note that keywords are not normally used for peerreview papers.
%\begin{IEEEkeywords}
%Cooperative diversity, decode and forward, piecewise linear
%\end{IEEEkeywords}



% For peer review papers, you can put extra information on the cover
% page as needed:
% \ifCLASSOPTIONpeerreview
% \begin{center} \bfseries EDICS Category: 3-BBND \end{center}
% \fi
%
% For peerreview papers, this IEEEtran command inserts a page break and
% creates the second title. It will be ignored for other modes.
%\IEEEpeerreviewmaketitle




   \item Four candidates A, B, C, D have ap-
plied for the assignment to coach a school cricket
team. If A is twice as likely to be selected as B, and
B and C are given about the same chance of being
selected, while C is twice as likely to be selected
as D, what are the probabilities that
\begin{enumerate}
\item C will be selected?
\item A will not be selected?
\end{enumerate}
	%\begin{table}[H]
	\centering
\begin{tabular}{|c|c|c|}
\hline
Random variable &Value &Definition\\ \hline
\multirow{3}{*}{X} &0 &Slips of Rs 1\\
&1 &Slips of Rs 5\\
&2 &Slips of Rs 13\\ \hline
\multirow{2}{*}{Y} &0 &Box A\\
&1 &Box B\\\hline
\end{tabular}
\caption{}
\label{tab:Distribution}
\end{table}
See \tabref{tab:Distribution}.
\begin{align}
p_{Y}\brak{k}= \begin{cases} 
      \frac{1}{3} & {k=0} \\
      \frac{2}{3 }& {k=1} 
   \end{cases}
   \\
p_{Y|X}\brak{0|0} = \frac{19}{25}\, 
p_{Y|X}\brak{0|1} = \frac{6}{25}\,
p_{Y|X}\brak{1|0} = \frac{45}{50}\,
p_{Y|X}\brak{1|2} = \frac{5}{50}
\end{align}
The desired probability is the probability that a slip drawn at random is marked other than Rs 1,
\begin{align}
&=1-p_X\brak{0}\\
&= p_X(1) + p_X(2)
\end{align}
Using Bayes theorem,
\begin{align}
&= p_Y\brak{0} \times \pr{Y=0 | X=1} + p_Y\brak{1} \times \pr{Y=1|X=2}\\
&=\frac{1}{3} \times \frac{6}{25} + \frac{2}{3} \times \frac{5}{50}\\
&=\frac{11}{75}
\end{align}

\newpage

%\tableofcontents

\bigskip

\renewcommand{\thefigure}{\theenumi}
\renewcommand{\thetable}{\theenumi}
%\renewcommand{\theequation}{\theenumi}

%\begin{abstract}
%%\boldmath
%In this letter, an algorithm for evaluating the exact analytical bit error rate  (BER)  for the piecewise linear (PL) combiner for  multiple relays is presented. Previous results were available only for upto three relays. The algorithm is unique in the sense that  the actual mathematical expressions, that are prohibitively large, need not be explicitly obtained. The diversity gain due to multiple relays is shown through plots of the analytical BER, well supported by simulations. 
%
%\end{abstract}
% IEEEtran.cls defaults to using nonbold math in the Abstract.
% This preserves the distinction between vectors and scalars. However,
% if the journal you are submitting to favors bold math in the abstract,
% then you can use LaTeX's standard command \boldmath at the very start
% of the abstract to achieve this. Many IEEE journals frown on math
% in the abstract anyway.

% Note that keywords are not normally used for peerreview papers.
%\begin{IEEEkeywords}
%Cooperative diversity, decode and forward, piecewise linear
%\end{IEEEkeywords}



% For peer review papers, you can put extra information on the cover
% page as needed:
% \ifCLASSOPTIONpeerreview
% \begin{center} \bfseries EDICS Category: 3-BBND \end{center}
% \fi
%
% For peerreview papers, this IEEEtran command inserts a page break and
% creates the second title. It will be ignored for other modes.
%\IEEEpeerreviewmaketitle




 \item A bag contain 24 balls of which $x$ balls are red, $2x$ are white and $3x$ are blue. A ball is selected at random, What is the probability that it is
\begin{enumerate}[label=\alph*)]
\item not red ?
\item white ?
\end{enumerate}
%\begin{table}[H]
	\centering
\begin{tabular}{|c|c|c|}
\hline
Random variable &Value &Definition\\ \hline
\multirow{3}{*}{X} &0 &Slips of Rs 1\\
&1 &Slips of Rs 5\\
&2 &Slips of Rs 13\\ \hline
\multirow{2}{*}{Y} &0 &Box A\\
&1 &Box B\\\hline
\end{tabular}
\caption{}
\label{tab:Distribution}
\end{table}
See \tabref{tab:Distribution}.
\begin{align}
p_{Y}\brak{k}= \begin{cases} 
      \frac{1}{3} & {k=0} \\
      \frac{2}{3 }& {k=1} 
   \end{cases}
   \\
p_{Y|X}\brak{0|0} = \frac{19}{25}\, 
p_{Y|X}\brak{0|1} = \frac{6}{25}\,
p_{Y|X}\brak{1|0} = \frac{45}{50}\,
p_{Y|X}\brak{1|2} = \frac{5}{50}
\end{align}
The desired probability is the probability that a slip drawn at random is marked other than Rs 1,
\begin{align}
&=1-p_X\brak{0}\\
&= p_X(1) + p_X(2)
\end{align}
Using Bayes theorem,
\begin{align}
&= p_Y\brak{0} \times \pr{Y=0 | X=1} + p_Y\brak{1} \times \pr{Y=1|X=2}\\
&=\frac{1}{3} \times \frac{6}{25} + \frac{2}{3} \times \frac{5}{50}\\
&=\frac{11}{75}
\end{align}

\newpage

%\tableofcontents

\bigskip

\renewcommand{\thefigure}{\theenumi}
\renewcommand{\thetable}{\theenumi}
%\renewcommand{\theequation}{\theenumi}

%\begin{abstract}
%%\boldmath
%In this letter, an algorithm for evaluating the exact analytical bit error rate  (BER)  for the piecewise linear (PL) combiner for  multiple relays is presented. Previous results were available only for upto three relays. The algorithm is unique in the sense that  the actual mathematical expressions, that are prohibitively large, need not be explicitly obtained. The diversity gain due to multiple relays is shown through plots of the analytical BER, well supported by simulations. 
%
%\end{abstract}
% IEEEtran.cls defaults to using nonbold math in the Abstract.
% This preserves the distinction between vectors and scalars. However,
% if the journal you are submitting to favors bold math in the abstract,
% then you can use LaTeX's standard command \boldmath at the very start
% of the abstract to achieve this. Many IEEE journals frown on math
% in the abstract anyway.

% Note that keywords are not normally used for peerreview papers.
%\begin{IEEEkeywords}
%Cooperative diversity, decode and forward, piecewise linear
%\end{IEEEkeywords}



% For peer review papers, you can put extra information on the cover
% page as needed:
% \ifCLASSOPTIONpeerreview
% \begin{center} \bfseries EDICS Category: 3-BBND \end{center}
% \fi
%
% For peerreview papers, this IEEEtran command inserts a page break and
% creates the second title. It will be ignored for other modes.
%\IEEEpeerreviewmaketitle




If the letters of the word ASSASSINATION are arranged at random. Find the Probability that
\begin{enumerate}[label=(\alph*)]
\item Four $S's$ come consecutively in the word
\item Two  $I's$ and two $N's$ come together
\item All $A's$ are not coming together
\item No two $A's$ are coming together
\end{enumerate}
%\begin{table}[H]
	\centering
\begin{tabular}{|c|c|c|}
\hline
Random variable &Value &Definition\\ \hline
\multirow{3}{*}{X} &0 &Slips of Rs 1\\
&1 &Slips of Rs 5\\
&2 &Slips of Rs 13\\ \hline
\multirow{2}{*}{Y} &0 &Box A\\
&1 &Box B\\\hline
\end{tabular}
\caption{}
\label{tab:Distribution}
\end{table}
See \tabref{tab:Distribution}.
\begin{align}
p_{Y}\brak{k}= \begin{cases} 
      \frac{1}{3} & {k=0} \\
      \frac{2}{3 }& {k=1} 
   \end{cases}
   \\
p_{Y|X}\brak{0|0} = \frac{19}{25}\, 
p_{Y|X}\brak{0|1} = \frac{6}{25}\,
p_{Y|X}\brak{1|0} = \frac{45}{50}\,
p_{Y|X}\brak{1|2} = \frac{5}{50}
\end{align}
The desired probability is the probability that a slip drawn at random is marked other than Rs 1,
\begin{align}
&=1-p_X\brak{0}\\
&= p_X(1) + p_X(2)
\end{align}
Using Bayes theorem,
\begin{align}
&= p_Y\brak{0} \times \pr{Y=0 | X=1} + p_Y\brak{1} \times \pr{Y=1|X=2}\\
&=\frac{1}{3} \times \frac{6}{25} + \frac{2}{3} \times \frac{5}{50}\\
&=\frac{11}{75}
\end{align}

\newpage

%\tableofcontents

\bigskip

\renewcommand{\thefigure}{\theenumi}
\renewcommand{\thetable}{\theenumi}
%\renewcommand{\theequation}{\theenumi}

%\begin{abstract}
%%\boldmath
%In this letter, an algorithm for evaluating the exact analytical bit error rate  (BER)  for the piecewise linear (PL) combiner for  multiple relays is presented. Previous results were available only for upto three relays. The algorithm is unique in the sense that  the actual mathematical expressions, that are prohibitively large, need not be explicitly obtained. The diversity gain due to multiple relays is shown through plots of the analytical BER, well supported by simulations. 
%
%\end{abstract}
% IEEEtran.cls defaults to using nonbold math in the Abstract.
% This preserves the distinction between vectors and scalars. However,
% if the journal you are submitting to favors bold math in the abstract,
% then you can use LaTeX's standard command \boldmath at the very start
% of the abstract to achieve this. Many IEEE journals frown on math
% in the abstract anyway.

% Note that keywords are not normally used for peerreview papers.
%\begin{IEEEkeywords}
%Cooperative diversity, decode and forward, piecewise linear
%\end{IEEEkeywords}



% For peer review papers, you can put extra information on the cover
% page as needed:
% \ifCLASSOPTIONpeerreview
% \begin{center} \bfseries EDICS Category: 3-BBND \end{center}
% \fi
%
% For peerreview papers, this IEEEtran command inserts a page break and
% creates the second title. It will be ignored for other modes.
%\IEEEpeerreviewmaketitle




	\item One urn contains two black balls (labelled B1 and B2) and one white ball. A
	second urn contains one black ball and two white balls (labelled W1 and W2).
	Suppose the following experiment is performed. One of the two urns is chosen
	at random. Next a ball is randomly chosen from the urn. Then a second ball is
	chosen at random from the same urn without replacing the first ball.
	
	\begin{enumerate}
	\item What is the probability that two black balls are chosen?
	
	\item What is the probability that two balls of opposite colour are chosen?
	\end{enumerate}
	\solution
	%\begin{align}
    \label{eq:12.13.6.18.1}
	\because	\pr{A|B} &> \pr{A},\
\frac{\pr{AB}}{\pr{B}} > \pr{A}
\\
    \label{eq:12.13.6.18.2}
	\implies \pr{AB} &> \pr{A}\pr{B}
	\\
	\text{or, } \frac{\pr{AB}}{\pr{A}} &=\pr{B|A} > \pr{A}
\end{align}

\end{enumerate}

	\item A card is selected from a pack of 52 cards.
 \begin{enumerate}[label=(\alph*)] 
                 \item How many points are there in the sample space?
                 \item Calculate the probability that the card is an ace of spades.
                 \item Calculate the probability that the card is (i) an ace and (ii) black card.
 \end{enumerate}
\solution
		%\begin{table}[H]
	\centering
\begin{tabular}{|c|c|c|}
\hline
Random variable &Value &Definition\\ \hline
\multirow{3}{*}{X} &0 &Slips of Rs 1\\
&1 &Slips of Rs 5\\
&2 &Slips of Rs 13\\ \hline
\multirow{2}{*}{Y} &0 &Box A\\
&1 &Box B\\\hline
\end{tabular}
\caption{}
\label{tab:Distribution}
\end{table}
See \tabref{tab:Distribution}.
\begin{align}
p_{Y}\brak{k}= \begin{cases} 
      \frac{1}{3} & {k=0} \\
      \frac{2}{3 }& {k=1} 
   \end{cases}
   \\
p_{Y|X}\brak{0|0} = \frac{19}{25}\, 
p_{Y|X}\brak{0|1} = \frac{6}{25}\,
p_{Y|X}\brak{1|0} = \frac{45}{50}\,
p_{Y|X}\brak{1|2} = \frac{5}{50}
\end{align}
The desired probability is the probability that a slip drawn at random is marked other than Rs 1,
\begin{align}
&=1-p_X\brak{0}\\
&= p_X(1) + p_X(2)
\end{align}
Using Bayes theorem,
\begin{align}
&= p_Y\brak{0} \times \pr{Y=0 | X=1} + p_Y\brak{1} \times \pr{Y=1|X=2}\\
&=\frac{1}{3} \times \frac{6}{25} + \frac{2}{3} \times \frac{5}{50}\\
&=\frac{11}{75}
\end{align}

\newpage

%\tableofcontents

\bigskip

\renewcommand{\thefigure}{\theenumi}
\renewcommand{\thetable}{\theenumi}
%\renewcommand{\theequation}{\theenumi}

%\begin{abstract}
%%\boldmath
%In this letter, an algorithm for evaluating the exact analytical bit error rate  (BER)  for the piecewise linear (PL) combiner for  multiple relays is presented. Previous results were available only for upto three relays. The algorithm is unique in the sense that  the actual mathematical expressions, that are prohibitively large, need not be explicitly obtained. The diversity gain due to multiple relays is shown through plots of the analytical BER, well supported by simulations. 
%
%\end{abstract}
% IEEEtran.cls defaults to using nonbold math in the Abstract.
% This preserves the distinction between vectors and scalars. However,
% if the journal you are submitting to favors bold math in the abstract,
% then you can use LaTeX's standard command \boldmath at the very start
% of the abstract to achieve this. Many IEEE journals frown on math
% in the abstract anyway.

% Note that keywords are not normally used for peerreview papers.
%\begin{IEEEkeywords}
%Cooperative diversity, decode and forward, piecewise linear
%\end{IEEEkeywords}



% For peer review papers, you can put extra information on the cover
% page as needed:
% \ifCLASSOPTIONpeerreview
% \begin{center} \bfseries EDICS Category: 3-BBND \end{center}
% \fi
%
% For peerreview papers, this IEEEtran command inserts a page break and
% creates the second title. It will be ignored for other modes.
%\IEEEpeerreviewmaketitle




\item Four cards are drawn from a well-shuffled deck of 52 cards. What is the probability of obtaining 3 diamonds and one spade.
\\
\solution
		%\begin{enumerate}[label=\thesection.\arabic*,ref=\thesection.\theenumi]
	\item One card is drawn from a well-shuffled deck of 52 cards. Find the probability of getting
\begin{enumerate}
\item A king of red colour 
\item A face card 
\item A red face card
\item The jack of hearts
\item A spade
\item The queen of diamonds

\end{enumerate}
\solution
		%\begin{table}[H]
	\centering
\begin{tabular}{|c|c|c|}
\hline
Random variable &Value &Definition\\ \hline
\multirow{3}{*}{X} &0 &Slips of Rs 1\\
&1 &Slips of Rs 5\\
&2 &Slips of Rs 13\\ \hline
\multirow{2}{*}{Y} &0 &Box A\\
&1 &Box B\\\hline
\end{tabular}
\caption{}
\label{tab:Distribution}
\end{table}
See \tabref{tab:Distribution}.
\begin{align}
p_{Y}\brak{k}= \begin{cases} 
      \frac{1}{3} & {k=0} \\
      \frac{2}{3 }& {k=1} 
   \end{cases}
   \\
p_{Y|X}\brak{0|0} = \frac{19}{25}\, 
p_{Y|X}\brak{0|1} = \frac{6}{25}\,
p_{Y|X}\brak{1|0} = \frac{45}{50}\,
p_{Y|X}\brak{1|2} = \frac{5}{50}
\end{align}
The desired probability is the probability that a slip drawn at random is marked other than Rs 1,
\begin{align}
&=1-p_X\brak{0}\\
&= p_X(1) + p_X(2)
\end{align}
Using Bayes theorem,
\begin{align}
&= p_Y\brak{0} \times \pr{Y=0 | X=1} + p_Y\brak{1} \times \pr{Y=1|X=2}\\
&=\frac{1}{3} \times \frac{6}{25} + \frac{2}{3} \times \frac{5}{50}\\
&=\frac{11}{75}
\end{align}

\newpage

%\tableofcontents

\bigskip

\renewcommand{\thefigure}{\theenumi}
\renewcommand{\thetable}{\theenumi}
%\renewcommand{\theequation}{\theenumi}

%\begin{abstract}
%%\boldmath
%In this letter, an algorithm for evaluating the exact analytical bit error rate  (BER)  for the piecewise linear (PL) combiner for  multiple relays is presented. Previous results were available only for upto three relays. The algorithm is unique in the sense that  the actual mathematical expressions, that are prohibitively large, need not be explicitly obtained. The diversity gain due to multiple relays is shown through plots of the analytical BER, well supported by simulations. 
%
%\end{abstract}
% IEEEtran.cls defaults to using nonbold math in the Abstract.
% This preserves the distinction between vectors and scalars. However,
% if the journal you are submitting to favors bold math in the abstract,
% then you can use LaTeX's standard command \boldmath at the very start
% of the abstract to achieve this. Many IEEE journals frown on math
% in the abstract anyway.

% Note that keywords are not normally used for peerreview papers.
%\begin{IEEEkeywords}
%Cooperative diversity, decode and forward, piecewise linear
%\end{IEEEkeywords}



% For peer review papers, you can put extra information on the cover
% page as needed:
% \ifCLASSOPTIONpeerreview
% \begin{center} \bfseries EDICS Category: 3-BBND \end{center}
% \fi
%
% For peerreview papers, this IEEEtran command inserts a page break and
% creates the second title. It will be ignored for other modes.
%\IEEEpeerreviewmaketitle




	\item Five cards—the ten, jack, queen, king and ace of diamonds, are well-shuffled with their face downwards. One card is then picked up at random.
\begin{enumerate}
\item
What is the probability that the card is the queen? 
\item
If the queen is drawn and put aside, what is the probability that the second card picked up is (a) an ace? (b) a queen?\\
\end{enumerate}
\solution
		%\begin{enumerate}[label=\thesection.\arabic*,ref=\thesection.\theenumi]
	\item One card is drawn from a well-shuffled deck of 52 cards. Find the probability of getting
\begin{enumerate}
\item A king of red colour 
\item A face card 
\item A red face card
\item The jack of hearts
\item A spade
\item The queen of diamonds

\end{enumerate}
\solution
		%\input{ncert/10/15/1/14/main.tex}
	\item Five cards—the ten, jack, queen, king and ace of diamonds, are well-shuffled with their face downwards. One card is then picked up at random.
\begin{enumerate}
\item
What is the probability that the card is the queen? 
\item
If the queen is drawn and put aside, what is the probability that the second card picked up is (a) an ace? (b) a queen?\\
\end{enumerate}
\solution
		%\input{ncert/10/15/1/15/defs.tex}
	\item A bag contains $5$ red balls and some blue balls. If the probability of drawing a blue ball is double that if a red ball, determine the number of blue balls in the bag. 
		\\
\solution
		%\input{ncert/10/15/2/3/defs.tex}
	\item A card is selected from a pack of 52 cards.
 \begin{enumerate}[label=(\alph*)] 
                 \item How many points are there in the sample space?
                 \item Calculate the probability that the card is an ace of spades.
                 \item Calculate the probability that the card is (i) an ace and (ii) black card.
 \end{enumerate}
\solution
		%\input{ncert/11/16/3/4/main.tex}
\item Four cards are drawn from a well-shuffled deck of 52 cards. What is the probability of obtaining 3 diamonds and one spade.
\\
\solution
		%\input{ncert/11/16/4/2/defs.tex}
\item In a certain lottery 10,000 tickets are sold and ten equal prizes are awarded. What is the probability of not getting a prize if you buy (a) one ticket (b) two tickets (c) 10 tickets ?	
\\
\solution
		%\input{ncert/11/16/4/4/defs.tex}
		%
\item 
Out of 100 students, two sections of 40 and 60 are formed. If you and your friend are among the 100 students, what is the probability that
\begin{enumerate}
\item you both enter the same section?
\item you both enter the different sections?
\end{enumerate}
\solution
		%\input{ncert/11/16/4/5/defs.tex}
	\item 
The number lock of a suitcase has 4 wheels each labelled with ten digits i.e. from 0 to 9.The lock opens with a sequence of four digits with no repeats.What is the probability of a person getting the right sequence to open the suitcase.
\\
\solution
		%\input{ncert/11/16/4/10/defs.tex}
		%
\item 
Two cards are drawn at random and without replacement from a pack of 52 playing cards. Find the probability that both the cards are black.
\\
\solution
		%\input{ncert/12/13/2/2/defs.tex}
		\item A box of oranges is inspected by examining three randomly selected oranges drawn without replacement. If all the three oranges are good, the box is approved for sale, otherwise, it is rejected. Find the probability that a box containing 15 oranges out of which 12 are good and 3 are bad ones will be approved for sale.
		\label{ncert/12/13/2/3/defs.tex}
		\item Two balls are drawn at random with replacement from a box containing 10 black and 8 red balls. Find the probability that
		\label{ncert/12/13/2/12}
\begin{enumerate}
\item both balls are red.
\item first ball is black and second is red.
\item one of them is black and other is red.
\end{enumerate}

\item In a hostel, 60\% of the students read Hindi newspaper, 40\% read English newspaper and 20\% read both Hindi and English newspapers. A student is selected at random.
		\label{ncert/12/13/2/15}
\begin{enumerate}
\item Find the probability that she reads neither Hindi nor English newspapers.
\item If she reads Hindi newspaper, find the probability that she reads English newspaper.
\item If she reads English newspaper, find the probability that she reads Hindi newspaper.\\
\end{enumerate}
\item The probability of obtaining an even prime number on each die, when a pair of dice is rolled is 
\begin{enumerate}
    \item $0$ 
    
    \item $\frac{1}{3}$ 
    
    \item $\frac{1}{12}$ 
    
    \item $\frac{1}{36}$ 
\end{enumerate}
\solution
		%\input{ncert/12/13/2/17/defs.tex}
	\item A bag contains 4 red and 4 black balls, another bag contains 2 red and 6 black balls. One of the two bags is selected at random and a ball is drawn from the bag which is found to be red. Find the probability that the ball is drawn from the first bag.
\\
\solution
		%\input{ncert/12/13/3/2/main.tex}
  \item
  Cards with numbers 2 to 101 are placed in a box. A card is selected at random.Find the probability that the card has
\begin{enumerate}[label=(\roman*)]
	\item an even number 
	\item a square number
\end{enumerate}
\solution
%\input{exemplar/10/13/3/32/main.tex}
\item
The king, queen and jack of clubs are removed from a deck of 52 playing cards and then well shuffled. Now one card is drawn at random from the remaining cards.  Determine the probability that the card is
\begin{enumerate}[label=(\roman*)]
\item a club
\item 10 of hearts
\end{enumerate}
\solution
%\input{exemplar/10/13/3/29/main.tex}
\item A team of medical students doing their internship have to assist during surgeries
at a city hospital. The probabilities of surgeries rated as very complex, complex,
routine, simple or very simple are respectively, 0.15, 0.20, 0.31, 0.26, .08. Find
the probabilities that a particular surgery will be rated
\begin{enumerate}
	\item complex or very complex;
	\item neither very complex nor very simple;
	\item routine or complex
	\item routine or simple
\end{enumerate}
\solution
%\input{exemplar/11/16/3/8(1)/main.tex}
\item A card is selected from a pack of 52 cards.
\begin{enumerate}[label=(\alph*)]
    \item How many points are there in the sample space?
    \item Calculate the probability that the card is an ace of spades.
    \item Calculate the probability that the card is (i) an ace and (ii) black card.
\end{enumerate}
\solution
%\input{exemplar/11/16/3/4/main2.tex}
\item The probability that a non leap year selected at random will contain 53 sundays.
\\
\solution
%\input{exemplar/10/13/1/19/main.tex}
\item One of the four persons John, Rita, Aslam or Gurpreet will be promoted next
month. Consequently the sample space consists of four elementary outcomes
S = {John promoted, Rita promoted, Aslam promoted, Gurpreet promoted}
You are told that the chances of John’s promotion is same as that of Gurpreet,
Rita’s chances of promotion are twice as likely as Johns. Aslam’s chances are
four times that of John.
\begin{enumerate}
	\item Determine
	\begin{enumerate}
		\item P (John promoted)
		\item P (Rita promoted)
		\item P (Aslam promoted)
		\item P (Gurpreet promoted)
	\end{enumerate}
	\item If A = {John promoted or Gurpreet promoted}, find P (A).
\end{enumerate}
\solution
%\input{exemplar/11/16/3/10/main.tex}
\item A card is drawn from a deck of 52 cards. Find the probability of getting a king or a heart or a red card.\\
\solution
%\input{exemplar/11/16/3/15/main.tex}
\item The probability that a student will pass his examination is 0.73, the probability of
the student getting a compartment is 0.13, and the probability that the student will
either pass or get compartment is 0.96. State True or False.\\
\solution
%\input{exemplar/11/16/3/31/main.tex}
\item A card is selected from a pack of 52 cards\\
\begin{enumerate}[label=(\alph*)]
\item How many points are there in the sample space?
\item Calculate the probability that the cards is an ace of spades.
\item Calculate the probability that the card is (i) an ace (ii)black card.\\
\end{enumerate}
%\input{ncert/11/16/3/4_1/Prob_4.tex}
\item In a non-leap year, the probability of having 53 tuesdays or 53 wednesdays is\\
\solution
%\input{exemplar/11/16/3/18/main.tex}
\item There are 1000 sealed envelopes in a box, 10 of them contain a cash prize of
Rs 100 each, 100 of them contain a cash prize of Rs 50 each and 200 of them
contain a cash prize of Rs 10 each and rest do not contain any cash prize. If they
are well shuffled and an envelope is picked up out, what is the probability that it
contains no cash prize?\\
\solution
%\input{exemplar/10/13/3/34/main.tex}
\item 
A die is thrown and a card is selected at random from a deck of 52 playing cards. The probability of getting an even number on the die and a spade card.\\
\solution
%\input{exemplar/12/13/3/78/main.tex}
\item
If 4-digit numbers greater than 5,000 are randomly formed from the digits 0, 1, 3, 5, and 7, what is the probability of forming a number divisible by 5 when:
\begin{enumerate}
    \item The digits are repeated?
    \item The repetition of digits is not allowed?
\end{enumerate}
\solution
%\input{ncert/11/16/4/9/main.tex}
\item Consider the probability space $\brak{\Omega, \mathcal{G}, P}$ where $\Omega = [0,2]$ and $\mathcal{G} = \cbrak{\phi, \Omega, [0,1], (1,2]}$. Let $X$ and $Y$ be two functions on $\Omega$ defined as
\begin{align*}
    X(\omega) = 
    \begin{cases}
        1 & \text{if }\omega \in [0, 1]\\
        2 & \text{if }\omega \in (1, 2]
    \end{cases}
\end{align*}
and
\begin{align*}
    Y(\omega) = 
    \begin{cases}
        2 & \text{if }\omega \in [0, 1.5]\\
        3 & \text{if }\omega \in (1.5, 2].
    \end{cases}
\end{align*}
Then which one of the following statements is true?
\begin{enumerate}
    \item [(A)] $X$ is a random variable with respect to $\mathcal{G}$, but $Y$ is not a random variable with respect to $\mathcal{G}$.
    \item [(B)] $Y$ is a random variable with respect to $\mathcal{G}$, but $X$ is not a random variable with respect to $\mathcal{G}$.
    \item [(C)] Neither $X$ nor $Y$ is a random variable with respect to $\mathcal{G}$.
    \item [(D)] Both $X$ and $Y$ are random variables with respect to $\mathcal{G}$.
\end{enumerate} \hfill (GATE ST 2023)\\
\solution
%\input{gate/ST/2023/14/main.tex}
	\item  A die is loaded in such a way that each odd number is twice as likely to occur as
each even number. Find $P(G)$, where $G$ is the event that a number greater than
3 occurs on a single roll of the die.
\\
\solution
		%\input{exemplar/11/16/3/5/main.tex}
	\item All the jacks, queens and kings are removed from a deck of 52 playing cards. The remaining cards are well shuffled and then one card is drawn at random. Giving ace a value 1 similar value for other cards, find the probability that the card has a value 
		\begin{enumerate}
			\item 7
			\item greater than 7
			\item less than 7
		\end{enumerate}
		%\input{exemplar/10/13/3/30/main.tex}
  \item A Lot consists of 48 mobile phones of which 42 are good, 3 have only minor defects and 3 have major defects.Varnika will buy a phone if it is good but the trader will only buy a mobile if it has no major defects. One phone is selected at random from the lot. What is the probability that it is
\begin{enumerate}
	\item acceptable to Varnika?
            \item acceptable to the trader?
\end{enumerate}
\solution
	%\input{exemplar/10/13/3/40/main.tex}
 \item A student says that if you throw a die, it will show up 1 or not 1. Therefore, the probability of getting 1 and the probability of getting 'not 1' each is equal to $\frac{1}{2}$. Is this correct? Give reasons.\\
 \solution
        %\input{exemplar/10/13/2/9/main.tex}
   \item Four candidates A, B, C, D have ap-
plied for the assignment to coach a school cricket
team. If A is twice as likely to be selected as B, and
B and C are given about the same chance of being
selected, while C is twice as likely to be selected
as D, what are the probabilities that
\begin{enumerate}
\item C will be selected?
\item A will not be selected?
\end{enumerate}
	%\input{exemplar/11/16/3/9/main.tex}
 \item A bag contain 24 balls of which $x$ balls are red, $2x$ are white and $3x$ are blue. A ball is selected at random, What is the probability that it is
\begin{enumerate}[label=\alph*)]
\item not red ?
\item white ?
\end{enumerate}
%\input{exemplar/10/13/3/41/main.tex}
If the letters of the word ASSASSINATION are arranged at random. Find the Probability that
\begin{enumerate}[label=(\alph*)]
\item Four $S's$ come consecutively in the word
\item Two  $I's$ and two $N's$ come together
\item All $A's$ are not coming together
\item No two $A's$ are coming together
\end{enumerate}
%\input{exemplar/11/16/3/14/main.tex}
	\item One urn contains two black balls (labelled B1 and B2) and one white ball. A
	second urn contains one black ball and two white balls (labelled W1 and W2).
	Suppose the following experiment is performed. One of the two urns is chosen
	at random. Next a ball is randomly chosen from the urn. Then a second ball is
	chosen at random from the same urn without replacing the first ball.
	
	\begin{enumerate}
	\item What is the probability that two black balls are chosen?
	
	\item What is the probability that two balls of opposite colour are chosen?
	\end{enumerate}
	\solution
	%\input{exemplar/11/16/3/12/main1.tex}
\end{enumerate}

	\item A bag contains $5$ red balls and some blue balls. If the probability of drawing a blue ball is double that if a red ball, determine the number of blue balls in the bag. 
		\\
\solution
		%\begin{enumerate}[label=\thesection.\arabic*,ref=\thesection.\theenumi]
	\item One card is drawn from a well-shuffled deck of 52 cards. Find the probability of getting
\begin{enumerate}
\item A king of red colour 
\item A face card 
\item A red face card
\item The jack of hearts
\item A spade
\item The queen of diamonds

\end{enumerate}
\solution
		%\input{ncert/10/15/1/14/main.tex}
	\item Five cards—the ten, jack, queen, king and ace of diamonds, are well-shuffled with their face downwards. One card is then picked up at random.
\begin{enumerate}
\item
What is the probability that the card is the queen? 
\item
If the queen is drawn and put aside, what is the probability that the second card picked up is (a) an ace? (b) a queen?\\
\end{enumerate}
\solution
		%\input{ncert/10/15/1/15/defs.tex}
	\item A bag contains $5$ red balls and some blue balls. If the probability of drawing a blue ball is double that if a red ball, determine the number of blue balls in the bag. 
		\\
\solution
		%\input{ncert/10/15/2/3/defs.tex}
	\item A card is selected from a pack of 52 cards.
 \begin{enumerate}[label=(\alph*)] 
                 \item How many points are there in the sample space?
                 \item Calculate the probability that the card is an ace of spades.
                 \item Calculate the probability that the card is (i) an ace and (ii) black card.
 \end{enumerate}
\solution
		%\input{ncert/11/16/3/4/main.tex}
\item Four cards are drawn from a well-shuffled deck of 52 cards. What is the probability of obtaining 3 diamonds and one spade.
\\
\solution
		%\input{ncert/11/16/4/2/defs.tex}
\item In a certain lottery 10,000 tickets are sold and ten equal prizes are awarded. What is the probability of not getting a prize if you buy (a) one ticket (b) two tickets (c) 10 tickets ?	
\\
\solution
		%\input{ncert/11/16/4/4/defs.tex}
		%
\item 
Out of 100 students, two sections of 40 and 60 are formed. If you and your friend are among the 100 students, what is the probability that
\begin{enumerate}
\item you both enter the same section?
\item you both enter the different sections?
\end{enumerate}
\solution
		%\input{ncert/11/16/4/5/defs.tex}
	\item 
The number lock of a suitcase has 4 wheels each labelled with ten digits i.e. from 0 to 9.The lock opens with a sequence of four digits with no repeats.What is the probability of a person getting the right sequence to open the suitcase.
\\
\solution
		%\input{ncert/11/16/4/10/defs.tex}
		%
\item 
Two cards are drawn at random and without replacement from a pack of 52 playing cards. Find the probability that both the cards are black.
\\
\solution
		%\input{ncert/12/13/2/2/defs.tex}
		\item A box of oranges is inspected by examining three randomly selected oranges drawn without replacement. If all the three oranges are good, the box is approved for sale, otherwise, it is rejected. Find the probability that a box containing 15 oranges out of which 12 are good and 3 are bad ones will be approved for sale.
		\label{ncert/12/13/2/3/defs.tex}
		\item Two balls are drawn at random with replacement from a box containing 10 black and 8 red balls. Find the probability that
		\label{ncert/12/13/2/12}
\begin{enumerate}
\item both balls are red.
\item first ball is black and second is red.
\item one of them is black and other is red.
\end{enumerate}

\item In a hostel, 60\% of the students read Hindi newspaper, 40\% read English newspaper and 20\% read both Hindi and English newspapers. A student is selected at random.
		\label{ncert/12/13/2/15}
\begin{enumerate}
\item Find the probability that she reads neither Hindi nor English newspapers.
\item If she reads Hindi newspaper, find the probability that she reads English newspaper.
\item If she reads English newspaper, find the probability that she reads Hindi newspaper.\\
\end{enumerate}
\item The probability of obtaining an even prime number on each die, when a pair of dice is rolled is 
\begin{enumerate}
    \item $0$ 
    
    \item $\frac{1}{3}$ 
    
    \item $\frac{1}{12}$ 
    
    \item $\frac{1}{36}$ 
\end{enumerate}
\solution
		%\input{ncert/12/13/2/17/defs.tex}
	\item A bag contains 4 red and 4 black balls, another bag contains 2 red and 6 black balls. One of the two bags is selected at random and a ball is drawn from the bag which is found to be red. Find the probability that the ball is drawn from the first bag.
\\
\solution
		%\input{ncert/12/13/3/2/main.tex}
  \item
  Cards with numbers 2 to 101 are placed in a box. A card is selected at random.Find the probability that the card has
\begin{enumerate}[label=(\roman*)]
	\item an even number 
	\item a square number
\end{enumerate}
\solution
%\input{exemplar/10/13/3/32/main.tex}
\item
The king, queen and jack of clubs are removed from a deck of 52 playing cards and then well shuffled. Now one card is drawn at random from the remaining cards.  Determine the probability that the card is
\begin{enumerate}[label=(\roman*)]
\item a club
\item 10 of hearts
\end{enumerate}
\solution
%\input{exemplar/10/13/3/29/main.tex}
\item A team of medical students doing their internship have to assist during surgeries
at a city hospital. The probabilities of surgeries rated as very complex, complex,
routine, simple or very simple are respectively, 0.15, 0.20, 0.31, 0.26, .08. Find
the probabilities that a particular surgery will be rated
\begin{enumerate}
	\item complex or very complex;
	\item neither very complex nor very simple;
	\item routine or complex
	\item routine or simple
\end{enumerate}
\solution
%\input{exemplar/11/16/3/8(1)/main.tex}
\item A card is selected from a pack of 52 cards.
\begin{enumerate}[label=(\alph*)]
    \item How many points are there in the sample space?
    \item Calculate the probability that the card is an ace of spades.
    \item Calculate the probability that the card is (i) an ace and (ii) black card.
\end{enumerate}
\solution
%\input{exemplar/11/16/3/4/main2.tex}
\item The probability that a non leap year selected at random will contain 53 sundays.
\\
\solution
%\input{exemplar/10/13/1/19/main.tex}
\item One of the four persons John, Rita, Aslam or Gurpreet will be promoted next
month. Consequently the sample space consists of four elementary outcomes
S = {John promoted, Rita promoted, Aslam promoted, Gurpreet promoted}
You are told that the chances of John’s promotion is same as that of Gurpreet,
Rita’s chances of promotion are twice as likely as Johns. Aslam’s chances are
four times that of John.
\begin{enumerate}
	\item Determine
	\begin{enumerate}
		\item P (John promoted)
		\item P (Rita promoted)
		\item P (Aslam promoted)
		\item P (Gurpreet promoted)
	\end{enumerate}
	\item If A = {John promoted or Gurpreet promoted}, find P (A).
\end{enumerate}
\solution
%\input{exemplar/11/16/3/10/main.tex}
\item A card is drawn from a deck of 52 cards. Find the probability of getting a king or a heart or a red card.\\
\solution
%\input{exemplar/11/16/3/15/main.tex}
\item The probability that a student will pass his examination is 0.73, the probability of
the student getting a compartment is 0.13, and the probability that the student will
either pass or get compartment is 0.96. State True or False.\\
\solution
%\input{exemplar/11/16/3/31/main.tex}
\item A card is selected from a pack of 52 cards\\
\begin{enumerate}[label=(\alph*)]
\item How many points are there in the sample space?
\item Calculate the probability that the cards is an ace of spades.
\item Calculate the probability that the card is (i) an ace (ii)black card.\\
\end{enumerate}
%\input{ncert/11/16/3/4_1/Prob_4.tex}
\item In a non-leap year, the probability of having 53 tuesdays or 53 wednesdays is\\
\solution
%\input{exemplar/11/16/3/18/main.tex}
\item There are 1000 sealed envelopes in a box, 10 of them contain a cash prize of
Rs 100 each, 100 of them contain a cash prize of Rs 50 each and 200 of them
contain a cash prize of Rs 10 each and rest do not contain any cash prize. If they
are well shuffled and an envelope is picked up out, what is the probability that it
contains no cash prize?\\
\solution
%\input{exemplar/10/13/3/34/main.tex}
\item 
A die is thrown and a card is selected at random from a deck of 52 playing cards. The probability of getting an even number on the die and a spade card.\\
\solution
%\input{exemplar/12/13/3/78/main.tex}
\item
If 4-digit numbers greater than 5,000 are randomly formed from the digits 0, 1, 3, 5, and 7, what is the probability of forming a number divisible by 5 when:
\begin{enumerate}
    \item The digits are repeated?
    \item The repetition of digits is not allowed?
\end{enumerate}
\solution
%\input{ncert/11/16/4/9/main.tex}
\item Consider the probability space $\brak{\Omega, \mathcal{G}, P}$ where $\Omega = [0,2]$ and $\mathcal{G} = \cbrak{\phi, \Omega, [0,1], (1,2]}$. Let $X$ and $Y$ be two functions on $\Omega$ defined as
\begin{align*}
    X(\omega) = 
    \begin{cases}
        1 & \text{if }\omega \in [0, 1]\\
        2 & \text{if }\omega \in (1, 2]
    \end{cases}
\end{align*}
and
\begin{align*}
    Y(\omega) = 
    \begin{cases}
        2 & \text{if }\omega \in [0, 1.5]\\
        3 & \text{if }\omega \in (1.5, 2].
    \end{cases}
\end{align*}
Then which one of the following statements is true?
\begin{enumerate}
    \item [(A)] $X$ is a random variable with respect to $\mathcal{G}$, but $Y$ is not a random variable with respect to $\mathcal{G}$.
    \item [(B)] $Y$ is a random variable with respect to $\mathcal{G}$, but $X$ is not a random variable with respect to $\mathcal{G}$.
    \item [(C)] Neither $X$ nor $Y$ is a random variable with respect to $\mathcal{G}$.
    \item [(D)] Both $X$ and $Y$ are random variables with respect to $\mathcal{G}$.
\end{enumerate} \hfill (GATE ST 2023)\\
\solution
%\input{gate/ST/2023/14/main.tex}
	\item  A die is loaded in such a way that each odd number is twice as likely to occur as
each even number. Find $P(G)$, where $G$ is the event that a number greater than
3 occurs on a single roll of the die.
\\
\solution
		%\input{exemplar/11/16/3/5/main.tex}
	\item All the jacks, queens and kings are removed from a deck of 52 playing cards. The remaining cards are well shuffled and then one card is drawn at random. Giving ace a value 1 similar value for other cards, find the probability that the card has a value 
		\begin{enumerate}
			\item 7
			\item greater than 7
			\item less than 7
		\end{enumerate}
		%\input{exemplar/10/13/3/30/main.tex}
  \item A Lot consists of 48 mobile phones of which 42 are good, 3 have only minor defects and 3 have major defects.Varnika will buy a phone if it is good but the trader will only buy a mobile if it has no major defects. One phone is selected at random from the lot. What is the probability that it is
\begin{enumerate}
	\item acceptable to Varnika?
            \item acceptable to the trader?
\end{enumerate}
\solution
	%\input{exemplar/10/13/3/40/main.tex}
 \item A student says that if you throw a die, it will show up 1 or not 1. Therefore, the probability of getting 1 and the probability of getting 'not 1' each is equal to $\frac{1}{2}$. Is this correct? Give reasons.\\
 \solution
        %\input{exemplar/10/13/2/9/main.tex}
   \item Four candidates A, B, C, D have ap-
plied for the assignment to coach a school cricket
team. If A is twice as likely to be selected as B, and
B and C are given about the same chance of being
selected, while C is twice as likely to be selected
as D, what are the probabilities that
\begin{enumerate}
\item C will be selected?
\item A will not be selected?
\end{enumerate}
	%\input{exemplar/11/16/3/9/main.tex}
 \item A bag contain 24 balls of which $x$ balls are red, $2x$ are white and $3x$ are blue. A ball is selected at random, What is the probability that it is
\begin{enumerate}[label=\alph*)]
\item not red ?
\item white ?
\end{enumerate}
%\input{exemplar/10/13/3/41/main.tex}
If the letters of the word ASSASSINATION are arranged at random. Find the Probability that
\begin{enumerate}[label=(\alph*)]
\item Four $S's$ come consecutively in the word
\item Two  $I's$ and two $N's$ come together
\item All $A's$ are not coming together
\item No two $A's$ are coming together
\end{enumerate}
%\input{exemplar/11/16/3/14/main.tex}
	\item One urn contains two black balls (labelled B1 and B2) and one white ball. A
	second urn contains one black ball and two white balls (labelled W1 and W2).
	Suppose the following experiment is performed. One of the two urns is chosen
	at random. Next a ball is randomly chosen from the urn. Then a second ball is
	chosen at random from the same urn without replacing the first ball.
	
	\begin{enumerate}
	\item What is the probability that two black balls are chosen?
	
	\item What is the probability that two balls of opposite colour are chosen?
	\end{enumerate}
	\solution
	%\input{exemplar/11/16/3/12/main1.tex}
\end{enumerate}

	\item A card is selected from a pack of 52 cards.
 \begin{enumerate}[label=(\alph*)] 
                 \item How many points are there in the sample space?
                 \item Calculate the probability that the card is an ace of spades.
                 \item Calculate the probability that the card is (i) an ace and (ii) black card.
 \end{enumerate}
\solution
		%\begin{table}[H]
	\centering
\begin{tabular}{|c|c|c|}
\hline
Random variable &Value &Definition\\ \hline
\multirow{3}{*}{X} &0 &Slips of Rs 1\\
&1 &Slips of Rs 5\\
&2 &Slips of Rs 13\\ \hline
\multirow{2}{*}{Y} &0 &Box A\\
&1 &Box B\\\hline
\end{tabular}
\caption{}
\label{tab:Distribution}
\end{table}
See \tabref{tab:Distribution}.
\begin{align}
p_{Y}\brak{k}= \begin{cases} 
      \frac{1}{3} & {k=0} \\
      \frac{2}{3 }& {k=1} 
   \end{cases}
   \\
p_{Y|X}\brak{0|0} = \frac{19}{25}\, 
p_{Y|X}\brak{0|1} = \frac{6}{25}\,
p_{Y|X}\brak{1|0} = \frac{45}{50}\,
p_{Y|X}\brak{1|2} = \frac{5}{50}
\end{align}
The desired probability is the probability that a slip drawn at random is marked other than Rs 1,
\begin{align}
&=1-p_X\brak{0}\\
&= p_X(1) + p_X(2)
\end{align}
Using Bayes theorem,
\begin{align}
&= p_Y\brak{0} \times \pr{Y=0 | X=1} + p_Y\brak{1} \times \pr{Y=1|X=2}\\
&=\frac{1}{3} \times \frac{6}{25} + \frac{2}{3} \times \frac{5}{50}\\
&=\frac{11}{75}
\end{align}

\newpage

%\tableofcontents

\bigskip

\renewcommand{\thefigure}{\theenumi}
\renewcommand{\thetable}{\theenumi}
%\renewcommand{\theequation}{\theenumi}

%\begin{abstract}
%%\boldmath
%In this letter, an algorithm for evaluating the exact analytical bit error rate  (BER)  for the piecewise linear (PL) combiner for  multiple relays is presented. Previous results were available only for upto three relays. The algorithm is unique in the sense that  the actual mathematical expressions, that are prohibitively large, need not be explicitly obtained. The diversity gain due to multiple relays is shown through plots of the analytical BER, well supported by simulations. 
%
%\end{abstract}
% IEEEtran.cls defaults to using nonbold math in the Abstract.
% This preserves the distinction between vectors and scalars. However,
% if the journal you are submitting to favors bold math in the abstract,
% then you can use LaTeX's standard command \boldmath at the very start
% of the abstract to achieve this. Many IEEE journals frown on math
% in the abstract anyway.

% Note that keywords are not normally used for peerreview papers.
%\begin{IEEEkeywords}
%Cooperative diversity, decode and forward, piecewise linear
%\end{IEEEkeywords}



% For peer review papers, you can put extra information on the cover
% page as needed:
% \ifCLASSOPTIONpeerreview
% \begin{center} \bfseries EDICS Category: 3-BBND \end{center}
% \fi
%
% For peerreview papers, this IEEEtran command inserts a page break and
% creates the second title. It will be ignored for other modes.
%\IEEEpeerreviewmaketitle




\item Four cards are drawn from a well-shuffled deck of 52 cards. What is the probability of obtaining 3 diamonds and one spade.
\\
\solution
		%\begin{enumerate}[label=\thesection.\arabic*,ref=\thesection.\theenumi]
	\item One card is drawn from a well-shuffled deck of 52 cards. Find the probability of getting
\begin{enumerate}
\item A king of red colour 
\item A face card 
\item A red face card
\item The jack of hearts
\item A spade
\item The queen of diamonds

\end{enumerate}
\solution
		%\input{ncert/10/15/1/14/main.tex}
	\item Five cards—the ten, jack, queen, king and ace of diamonds, are well-shuffled with their face downwards. One card is then picked up at random.
\begin{enumerate}
\item
What is the probability that the card is the queen? 
\item
If the queen is drawn and put aside, what is the probability that the second card picked up is (a) an ace? (b) a queen?\\
\end{enumerate}
\solution
		%\input{ncert/10/15/1/15/defs.tex}
	\item A bag contains $5$ red balls and some blue balls. If the probability of drawing a blue ball is double that if a red ball, determine the number of blue balls in the bag. 
		\\
\solution
		%\input{ncert/10/15/2/3/defs.tex}
	\item A card is selected from a pack of 52 cards.
 \begin{enumerate}[label=(\alph*)] 
                 \item How many points are there in the sample space?
                 \item Calculate the probability that the card is an ace of spades.
                 \item Calculate the probability that the card is (i) an ace and (ii) black card.
 \end{enumerate}
\solution
		%\input{ncert/11/16/3/4/main.tex}
\item Four cards are drawn from a well-shuffled deck of 52 cards. What is the probability of obtaining 3 diamonds and one spade.
\\
\solution
		%\input{ncert/11/16/4/2/defs.tex}
\item In a certain lottery 10,000 tickets are sold and ten equal prizes are awarded. What is the probability of not getting a prize if you buy (a) one ticket (b) two tickets (c) 10 tickets ?	
\\
\solution
		%\input{ncert/11/16/4/4/defs.tex}
		%
\item 
Out of 100 students, two sections of 40 and 60 are formed. If you and your friend are among the 100 students, what is the probability that
\begin{enumerate}
\item you both enter the same section?
\item you both enter the different sections?
\end{enumerate}
\solution
		%\input{ncert/11/16/4/5/defs.tex}
	\item 
The number lock of a suitcase has 4 wheels each labelled with ten digits i.e. from 0 to 9.The lock opens with a sequence of four digits with no repeats.What is the probability of a person getting the right sequence to open the suitcase.
\\
\solution
		%\input{ncert/11/16/4/10/defs.tex}
		%
\item 
Two cards are drawn at random and without replacement from a pack of 52 playing cards. Find the probability that both the cards are black.
\\
\solution
		%\input{ncert/12/13/2/2/defs.tex}
		\item A box of oranges is inspected by examining three randomly selected oranges drawn without replacement. If all the three oranges are good, the box is approved for sale, otherwise, it is rejected. Find the probability that a box containing 15 oranges out of which 12 are good and 3 are bad ones will be approved for sale.
		\label{ncert/12/13/2/3/defs.tex}
		\item Two balls are drawn at random with replacement from a box containing 10 black and 8 red balls. Find the probability that
		\label{ncert/12/13/2/12}
\begin{enumerate}
\item both balls are red.
\item first ball is black and second is red.
\item one of them is black and other is red.
\end{enumerate}

\item In a hostel, 60\% of the students read Hindi newspaper, 40\% read English newspaper and 20\% read both Hindi and English newspapers. A student is selected at random.
		\label{ncert/12/13/2/15}
\begin{enumerate}
\item Find the probability that she reads neither Hindi nor English newspapers.
\item If she reads Hindi newspaper, find the probability that she reads English newspaper.
\item If she reads English newspaper, find the probability that she reads Hindi newspaper.\\
\end{enumerate}
\item The probability of obtaining an even prime number on each die, when a pair of dice is rolled is 
\begin{enumerate}
    \item $0$ 
    
    \item $\frac{1}{3}$ 
    
    \item $\frac{1}{12}$ 
    
    \item $\frac{1}{36}$ 
\end{enumerate}
\solution
		%\input{ncert/12/13/2/17/defs.tex}
	\item A bag contains 4 red and 4 black balls, another bag contains 2 red and 6 black balls. One of the two bags is selected at random and a ball is drawn from the bag which is found to be red. Find the probability that the ball is drawn from the first bag.
\\
\solution
		%\input{ncert/12/13/3/2/main.tex}
  \item
  Cards with numbers 2 to 101 are placed in a box. A card is selected at random.Find the probability that the card has
\begin{enumerate}[label=(\roman*)]
	\item an even number 
	\item a square number
\end{enumerate}
\solution
%\input{exemplar/10/13/3/32/main.tex}
\item
The king, queen and jack of clubs are removed from a deck of 52 playing cards and then well shuffled. Now one card is drawn at random from the remaining cards.  Determine the probability that the card is
\begin{enumerate}[label=(\roman*)]
\item a club
\item 10 of hearts
\end{enumerate}
\solution
%\input{exemplar/10/13/3/29/main.tex}
\item A team of medical students doing their internship have to assist during surgeries
at a city hospital. The probabilities of surgeries rated as very complex, complex,
routine, simple or very simple are respectively, 0.15, 0.20, 0.31, 0.26, .08. Find
the probabilities that a particular surgery will be rated
\begin{enumerate}
	\item complex or very complex;
	\item neither very complex nor very simple;
	\item routine or complex
	\item routine or simple
\end{enumerate}
\solution
%\input{exemplar/11/16/3/8(1)/main.tex}
\item A card is selected from a pack of 52 cards.
\begin{enumerate}[label=(\alph*)]
    \item How many points are there in the sample space?
    \item Calculate the probability that the card is an ace of spades.
    \item Calculate the probability that the card is (i) an ace and (ii) black card.
\end{enumerate}
\solution
%\input{exemplar/11/16/3/4/main2.tex}
\item The probability that a non leap year selected at random will contain 53 sundays.
\\
\solution
%\input{exemplar/10/13/1/19/main.tex}
\item One of the four persons John, Rita, Aslam or Gurpreet will be promoted next
month. Consequently the sample space consists of four elementary outcomes
S = {John promoted, Rita promoted, Aslam promoted, Gurpreet promoted}
You are told that the chances of John’s promotion is same as that of Gurpreet,
Rita’s chances of promotion are twice as likely as Johns. Aslam’s chances are
four times that of John.
\begin{enumerate}
	\item Determine
	\begin{enumerate}
		\item P (John promoted)
		\item P (Rita promoted)
		\item P (Aslam promoted)
		\item P (Gurpreet promoted)
	\end{enumerate}
	\item If A = {John promoted or Gurpreet promoted}, find P (A).
\end{enumerate}
\solution
%\input{exemplar/11/16/3/10/main.tex}
\item A card is drawn from a deck of 52 cards. Find the probability of getting a king or a heart or a red card.\\
\solution
%\input{exemplar/11/16/3/15/main.tex}
\item The probability that a student will pass his examination is 0.73, the probability of
the student getting a compartment is 0.13, and the probability that the student will
either pass or get compartment is 0.96. State True or False.\\
\solution
%\input{exemplar/11/16/3/31/main.tex}
\item A card is selected from a pack of 52 cards\\
\begin{enumerate}[label=(\alph*)]
\item How many points are there in the sample space?
\item Calculate the probability that the cards is an ace of spades.
\item Calculate the probability that the card is (i) an ace (ii)black card.\\
\end{enumerate}
%\input{ncert/11/16/3/4_1/Prob_4.tex}
\item In a non-leap year, the probability of having 53 tuesdays or 53 wednesdays is\\
\solution
%\input{exemplar/11/16/3/18/main.tex}
\item There are 1000 sealed envelopes in a box, 10 of them contain a cash prize of
Rs 100 each, 100 of them contain a cash prize of Rs 50 each and 200 of them
contain a cash prize of Rs 10 each and rest do not contain any cash prize. If they
are well shuffled and an envelope is picked up out, what is the probability that it
contains no cash prize?\\
\solution
%\input{exemplar/10/13/3/34/main.tex}
\item 
A die is thrown and a card is selected at random from a deck of 52 playing cards. The probability of getting an even number on the die and a spade card.\\
\solution
%\input{exemplar/12/13/3/78/main.tex}
\item
If 4-digit numbers greater than 5,000 are randomly formed from the digits 0, 1, 3, 5, and 7, what is the probability of forming a number divisible by 5 when:
\begin{enumerate}
    \item The digits are repeated?
    \item The repetition of digits is not allowed?
\end{enumerate}
\solution
%\input{ncert/11/16/4/9/main.tex}
\item Consider the probability space $\brak{\Omega, \mathcal{G}, P}$ where $\Omega = [0,2]$ and $\mathcal{G} = \cbrak{\phi, \Omega, [0,1], (1,2]}$. Let $X$ and $Y$ be two functions on $\Omega$ defined as
\begin{align*}
    X(\omega) = 
    \begin{cases}
        1 & \text{if }\omega \in [0, 1]\\
        2 & \text{if }\omega \in (1, 2]
    \end{cases}
\end{align*}
and
\begin{align*}
    Y(\omega) = 
    \begin{cases}
        2 & \text{if }\omega \in [0, 1.5]\\
        3 & \text{if }\omega \in (1.5, 2].
    \end{cases}
\end{align*}
Then which one of the following statements is true?
\begin{enumerate}
    \item [(A)] $X$ is a random variable with respect to $\mathcal{G}$, but $Y$ is not a random variable with respect to $\mathcal{G}$.
    \item [(B)] $Y$ is a random variable with respect to $\mathcal{G}$, but $X$ is not a random variable with respect to $\mathcal{G}$.
    \item [(C)] Neither $X$ nor $Y$ is a random variable with respect to $\mathcal{G}$.
    \item [(D)] Both $X$ and $Y$ are random variables with respect to $\mathcal{G}$.
\end{enumerate} \hfill (GATE ST 2023)\\
\solution
%\input{gate/ST/2023/14/main.tex}
	\item  A die is loaded in such a way that each odd number is twice as likely to occur as
each even number. Find $P(G)$, where $G$ is the event that a number greater than
3 occurs on a single roll of the die.
\\
\solution
		%\input{exemplar/11/16/3/5/main.tex}
	\item All the jacks, queens and kings are removed from a deck of 52 playing cards. The remaining cards are well shuffled and then one card is drawn at random. Giving ace a value 1 similar value for other cards, find the probability that the card has a value 
		\begin{enumerate}
			\item 7
			\item greater than 7
			\item less than 7
		\end{enumerate}
		%\input{exemplar/10/13/3/30/main.tex}
  \item A Lot consists of 48 mobile phones of which 42 are good, 3 have only minor defects and 3 have major defects.Varnika will buy a phone if it is good but the trader will only buy a mobile if it has no major defects. One phone is selected at random from the lot. What is the probability that it is
\begin{enumerate}
	\item acceptable to Varnika?
            \item acceptable to the trader?
\end{enumerate}
\solution
	%\input{exemplar/10/13/3/40/main.tex}
 \item A student says that if you throw a die, it will show up 1 or not 1. Therefore, the probability of getting 1 and the probability of getting 'not 1' each is equal to $\frac{1}{2}$. Is this correct? Give reasons.\\
 \solution
        %\input{exemplar/10/13/2/9/main.tex}
   \item Four candidates A, B, C, D have ap-
plied for the assignment to coach a school cricket
team. If A is twice as likely to be selected as B, and
B and C are given about the same chance of being
selected, while C is twice as likely to be selected
as D, what are the probabilities that
\begin{enumerate}
\item C will be selected?
\item A will not be selected?
\end{enumerate}
	%\input{exemplar/11/16/3/9/main.tex}
 \item A bag contain 24 balls of which $x$ balls are red, $2x$ are white and $3x$ are blue. A ball is selected at random, What is the probability that it is
\begin{enumerate}[label=\alph*)]
\item not red ?
\item white ?
\end{enumerate}
%\input{exemplar/10/13/3/41/main.tex}
If the letters of the word ASSASSINATION are arranged at random. Find the Probability that
\begin{enumerate}[label=(\alph*)]
\item Four $S's$ come consecutively in the word
\item Two  $I's$ and two $N's$ come together
\item All $A's$ are not coming together
\item No two $A's$ are coming together
\end{enumerate}
%\input{exemplar/11/16/3/14/main.tex}
	\item One urn contains two black balls (labelled B1 and B2) and one white ball. A
	second urn contains one black ball and two white balls (labelled W1 and W2).
	Suppose the following experiment is performed. One of the two urns is chosen
	at random. Next a ball is randomly chosen from the urn. Then a second ball is
	chosen at random from the same urn without replacing the first ball.
	
	\begin{enumerate}
	\item What is the probability that two black balls are chosen?
	
	\item What is the probability that two balls of opposite colour are chosen?
	\end{enumerate}
	\solution
	%\input{exemplar/11/16/3/12/main1.tex}
\end{enumerate}

\item In a certain lottery 10,000 tickets are sold and ten equal prizes are awarded. What is the probability of not getting a prize if you buy (a) one ticket (b) two tickets (c) 10 tickets ?	
\\
\solution
		%\begin{enumerate}[label=\thesection.\arabic*,ref=\thesection.\theenumi]
	\item One card is drawn from a well-shuffled deck of 52 cards. Find the probability of getting
\begin{enumerate}
\item A king of red colour 
\item A face card 
\item A red face card
\item The jack of hearts
\item A spade
\item The queen of diamonds

\end{enumerate}
\solution
		%\input{ncert/10/15/1/14/main.tex}
	\item Five cards—the ten, jack, queen, king and ace of diamonds, are well-shuffled with their face downwards. One card is then picked up at random.
\begin{enumerate}
\item
What is the probability that the card is the queen? 
\item
If the queen is drawn and put aside, what is the probability that the second card picked up is (a) an ace? (b) a queen?\\
\end{enumerate}
\solution
		%\input{ncert/10/15/1/15/defs.tex}
	\item A bag contains $5$ red balls and some blue balls. If the probability of drawing a blue ball is double that if a red ball, determine the number of blue balls in the bag. 
		\\
\solution
		%\input{ncert/10/15/2/3/defs.tex}
	\item A card is selected from a pack of 52 cards.
 \begin{enumerate}[label=(\alph*)] 
                 \item How many points are there in the sample space?
                 \item Calculate the probability that the card is an ace of spades.
                 \item Calculate the probability that the card is (i) an ace and (ii) black card.
 \end{enumerate}
\solution
		%\input{ncert/11/16/3/4/main.tex}
\item Four cards are drawn from a well-shuffled deck of 52 cards. What is the probability of obtaining 3 diamonds and one spade.
\\
\solution
		%\input{ncert/11/16/4/2/defs.tex}
\item In a certain lottery 10,000 tickets are sold and ten equal prizes are awarded. What is the probability of not getting a prize if you buy (a) one ticket (b) two tickets (c) 10 tickets ?	
\\
\solution
		%\input{ncert/11/16/4/4/defs.tex}
		%
\item 
Out of 100 students, two sections of 40 and 60 are formed. If you and your friend are among the 100 students, what is the probability that
\begin{enumerate}
\item you both enter the same section?
\item you both enter the different sections?
\end{enumerate}
\solution
		%\input{ncert/11/16/4/5/defs.tex}
	\item 
The number lock of a suitcase has 4 wheels each labelled with ten digits i.e. from 0 to 9.The lock opens with a sequence of four digits with no repeats.What is the probability of a person getting the right sequence to open the suitcase.
\\
\solution
		%\input{ncert/11/16/4/10/defs.tex}
		%
\item 
Two cards are drawn at random and without replacement from a pack of 52 playing cards. Find the probability that both the cards are black.
\\
\solution
		%\input{ncert/12/13/2/2/defs.tex}
		\item A box of oranges is inspected by examining three randomly selected oranges drawn without replacement. If all the three oranges are good, the box is approved for sale, otherwise, it is rejected. Find the probability that a box containing 15 oranges out of which 12 are good and 3 are bad ones will be approved for sale.
		\label{ncert/12/13/2/3/defs.tex}
		\item Two balls are drawn at random with replacement from a box containing 10 black and 8 red balls. Find the probability that
		\label{ncert/12/13/2/12}
\begin{enumerate}
\item both balls are red.
\item first ball is black and second is red.
\item one of them is black and other is red.
\end{enumerate}

\item In a hostel, 60\% of the students read Hindi newspaper, 40\% read English newspaper and 20\% read both Hindi and English newspapers. A student is selected at random.
		\label{ncert/12/13/2/15}
\begin{enumerate}
\item Find the probability that she reads neither Hindi nor English newspapers.
\item If she reads Hindi newspaper, find the probability that she reads English newspaper.
\item If she reads English newspaper, find the probability that she reads Hindi newspaper.\\
\end{enumerate}
\item The probability of obtaining an even prime number on each die, when a pair of dice is rolled is 
\begin{enumerate}
    \item $0$ 
    
    \item $\frac{1}{3}$ 
    
    \item $\frac{1}{12}$ 
    
    \item $\frac{1}{36}$ 
\end{enumerate}
\solution
		%\input{ncert/12/13/2/17/defs.tex}
	\item A bag contains 4 red and 4 black balls, another bag contains 2 red and 6 black balls. One of the two bags is selected at random and a ball is drawn from the bag which is found to be red. Find the probability that the ball is drawn from the first bag.
\\
\solution
		%\input{ncert/12/13/3/2/main.tex}
  \item
  Cards with numbers 2 to 101 are placed in a box. A card is selected at random.Find the probability that the card has
\begin{enumerate}[label=(\roman*)]
	\item an even number 
	\item a square number
\end{enumerate}
\solution
%\input{exemplar/10/13/3/32/main.tex}
\item
The king, queen and jack of clubs are removed from a deck of 52 playing cards and then well shuffled. Now one card is drawn at random from the remaining cards.  Determine the probability that the card is
\begin{enumerate}[label=(\roman*)]
\item a club
\item 10 of hearts
\end{enumerate}
\solution
%\input{exemplar/10/13/3/29/main.tex}
\item A team of medical students doing their internship have to assist during surgeries
at a city hospital. The probabilities of surgeries rated as very complex, complex,
routine, simple or very simple are respectively, 0.15, 0.20, 0.31, 0.26, .08. Find
the probabilities that a particular surgery will be rated
\begin{enumerate}
	\item complex or very complex;
	\item neither very complex nor very simple;
	\item routine or complex
	\item routine or simple
\end{enumerate}
\solution
%\input{exemplar/11/16/3/8(1)/main.tex}
\item A card is selected from a pack of 52 cards.
\begin{enumerate}[label=(\alph*)]
    \item How many points are there in the sample space?
    \item Calculate the probability that the card is an ace of spades.
    \item Calculate the probability that the card is (i) an ace and (ii) black card.
\end{enumerate}
\solution
%\input{exemplar/11/16/3/4/main2.tex}
\item The probability that a non leap year selected at random will contain 53 sundays.
\\
\solution
%\input{exemplar/10/13/1/19/main.tex}
\item One of the four persons John, Rita, Aslam or Gurpreet will be promoted next
month. Consequently the sample space consists of four elementary outcomes
S = {John promoted, Rita promoted, Aslam promoted, Gurpreet promoted}
You are told that the chances of John’s promotion is same as that of Gurpreet,
Rita’s chances of promotion are twice as likely as Johns. Aslam’s chances are
four times that of John.
\begin{enumerate}
	\item Determine
	\begin{enumerate}
		\item P (John promoted)
		\item P (Rita promoted)
		\item P (Aslam promoted)
		\item P (Gurpreet promoted)
	\end{enumerate}
	\item If A = {John promoted or Gurpreet promoted}, find P (A).
\end{enumerate}
\solution
%\input{exemplar/11/16/3/10/main.tex}
\item A card is drawn from a deck of 52 cards. Find the probability of getting a king or a heart or a red card.\\
\solution
%\input{exemplar/11/16/3/15/main.tex}
\item The probability that a student will pass his examination is 0.73, the probability of
the student getting a compartment is 0.13, and the probability that the student will
either pass or get compartment is 0.96. State True or False.\\
\solution
%\input{exemplar/11/16/3/31/main.tex}
\item A card is selected from a pack of 52 cards\\
\begin{enumerate}[label=(\alph*)]
\item How many points are there in the sample space?
\item Calculate the probability that the cards is an ace of spades.
\item Calculate the probability that the card is (i) an ace (ii)black card.\\
\end{enumerate}
%\input{ncert/11/16/3/4_1/Prob_4.tex}
\item In a non-leap year, the probability of having 53 tuesdays or 53 wednesdays is\\
\solution
%\input{exemplar/11/16/3/18/main.tex}
\item There are 1000 sealed envelopes in a box, 10 of them contain a cash prize of
Rs 100 each, 100 of them contain a cash prize of Rs 50 each and 200 of them
contain a cash prize of Rs 10 each and rest do not contain any cash prize. If they
are well shuffled and an envelope is picked up out, what is the probability that it
contains no cash prize?\\
\solution
%\input{exemplar/10/13/3/34/main.tex}
\item 
A die is thrown and a card is selected at random from a deck of 52 playing cards. The probability of getting an even number on the die and a spade card.\\
\solution
%\input{exemplar/12/13/3/78/main.tex}
\item
If 4-digit numbers greater than 5,000 are randomly formed from the digits 0, 1, 3, 5, and 7, what is the probability of forming a number divisible by 5 when:
\begin{enumerate}
    \item The digits are repeated?
    \item The repetition of digits is not allowed?
\end{enumerate}
\solution
%\input{ncert/11/16/4/9/main.tex}
\item Consider the probability space $\brak{\Omega, \mathcal{G}, P}$ where $\Omega = [0,2]$ and $\mathcal{G} = \cbrak{\phi, \Omega, [0,1], (1,2]}$. Let $X$ and $Y$ be two functions on $\Omega$ defined as
\begin{align*}
    X(\omega) = 
    \begin{cases}
        1 & \text{if }\omega \in [0, 1]\\
        2 & \text{if }\omega \in (1, 2]
    \end{cases}
\end{align*}
and
\begin{align*}
    Y(\omega) = 
    \begin{cases}
        2 & \text{if }\omega \in [0, 1.5]\\
        3 & \text{if }\omega \in (1.5, 2].
    \end{cases}
\end{align*}
Then which one of the following statements is true?
\begin{enumerate}
    \item [(A)] $X$ is a random variable with respect to $\mathcal{G}$, but $Y$ is not a random variable with respect to $\mathcal{G}$.
    \item [(B)] $Y$ is a random variable with respect to $\mathcal{G}$, but $X$ is not a random variable with respect to $\mathcal{G}$.
    \item [(C)] Neither $X$ nor $Y$ is a random variable with respect to $\mathcal{G}$.
    \item [(D)] Both $X$ and $Y$ are random variables with respect to $\mathcal{G}$.
\end{enumerate} \hfill (GATE ST 2023)\\
\solution
%\input{gate/ST/2023/14/main.tex}
	\item  A die is loaded in such a way that each odd number is twice as likely to occur as
each even number. Find $P(G)$, where $G$ is the event that a number greater than
3 occurs on a single roll of the die.
\\
\solution
		%\input{exemplar/11/16/3/5/main.tex}
	\item All the jacks, queens and kings are removed from a deck of 52 playing cards. The remaining cards are well shuffled and then one card is drawn at random. Giving ace a value 1 similar value for other cards, find the probability that the card has a value 
		\begin{enumerate}
			\item 7
			\item greater than 7
			\item less than 7
		\end{enumerate}
		%\input{exemplar/10/13/3/30/main.tex}
  \item A Lot consists of 48 mobile phones of which 42 are good, 3 have only minor defects and 3 have major defects.Varnika will buy a phone if it is good but the trader will only buy a mobile if it has no major defects. One phone is selected at random from the lot. What is the probability that it is
\begin{enumerate}
	\item acceptable to Varnika?
            \item acceptable to the trader?
\end{enumerate}
\solution
	%\input{exemplar/10/13/3/40/main.tex}
 \item A student says that if you throw a die, it will show up 1 or not 1. Therefore, the probability of getting 1 and the probability of getting 'not 1' each is equal to $\frac{1}{2}$. Is this correct? Give reasons.\\
 \solution
        %\input{exemplar/10/13/2/9/main.tex}
   \item Four candidates A, B, C, D have ap-
plied for the assignment to coach a school cricket
team. If A is twice as likely to be selected as B, and
B and C are given about the same chance of being
selected, while C is twice as likely to be selected
as D, what are the probabilities that
\begin{enumerate}
\item C will be selected?
\item A will not be selected?
\end{enumerate}
	%\input{exemplar/11/16/3/9/main.tex}
 \item A bag contain 24 balls of which $x$ balls are red, $2x$ are white and $3x$ are blue. A ball is selected at random, What is the probability that it is
\begin{enumerate}[label=\alph*)]
\item not red ?
\item white ?
\end{enumerate}
%\input{exemplar/10/13/3/41/main.tex}
If the letters of the word ASSASSINATION are arranged at random. Find the Probability that
\begin{enumerate}[label=(\alph*)]
\item Four $S's$ come consecutively in the word
\item Two  $I's$ and two $N's$ come together
\item All $A's$ are not coming together
\item No two $A's$ are coming together
\end{enumerate}
%\input{exemplar/11/16/3/14/main.tex}
	\item One urn contains two black balls (labelled B1 and B2) and one white ball. A
	second urn contains one black ball and two white balls (labelled W1 and W2).
	Suppose the following experiment is performed. One of the two urns is chosen
	at random. Next a ball is randomly chosen from the urn. Then a second ball is
	chosen at random from the same urn without replacing the first ball.
	
	\begin{enumerate}
	\item What is the probability that two black balls are chosen?
	
	\item What is the probability that two balls of opposite colour are chosen?
	\end{enumerate}
	\solution
	%\input{exemplar/11/16/3/12/main1.tex}
\end{enumerate}

		%
\item 
Out of 100 students, two sections of 40 and 60 are formed. If you and your friend are among the 100 students, what is the probability that
\begin{enumerate}
\item you both enter the same section?
\item you both enter the different sections?
\end{enumerate}
\solution
		%\begin{enumerate}[label=\thesection.\arabic*,ref=\thesection.\theenumi]
	\item One card is drawn from a well-shuffled deck of 52 cards. Find the probability of getting
\begin{enumerate}
\item A king of red colour 
\item A face card 
\item A red face card
\item The jack of hearts
\item A spade
\item The queen of diamonds

\end{enumerate}
\solution
		%\input{ncert/10/15/1/14/main.tex}
	\item Five cards—the ten, jack, queen, king and ace of diamonds, are well-shuffled with their face downwards. One card is then picked up at random.
\begin{enumerate}
\item
What is the probability that the card is the queen? 
\item
If the queen is drawn and put aside, what is the probability that the second card picked up is (a) an ace? (b) a queen?\\
\end{enumerate}
\solution
		%\input{ncert/10/15/1/15/defs.tex}
	\item A bag contains $5$ red balls and some blue balls. If the probability of drawing a blue ball is double that if a red ball, determine the number of blue balls in the bag. 
		\\
\solution
		%\input{ncert/10/15/2/3/defs.tex}
	\item A card is selected from a pack of 52 cards.
 \begin{enumerate}[label=(\alph*)] 
                 \item How many points are there in the sample space?
                 \item Calculate the probability that the card is an ace of spades.
                 \item Calculate the probability that the card is (i) an ace and (ii) black card.
 \end{enumerate}
\solution
		%\input{ncert/11/16/3/4/main.tex}
\item Four cards are drawn from a well-shuffled deck of 52 cards. What is the probability of obtaining 3 diamonds and one spade.
\\
\solution
		%\input{ncert/11/16/4/2/defs.tex}
\item In a certain lottery 10,000 tickets are sold and ten equal prizes are awarded. What is the probability of not getting a prize if you buy (a) one ticket (b) two tickets (c) 10 tickets ?	
\\
\solution
		%\input{ncert/11/16/4/4/defs.tex}
		%
\item 
Out of 100 students, two sections of 40 and 60 are formed. If you and your friend are among the 100 students, what is the probability that
\begin{enumerate}
\item you both enter the same section?
\item you both enter the different sections?
\end{enumerate}
\solution
		%\input{ncert/11/16/4/5/defs.tex}
	\item 
The number lock of a suitcase has 4 wheels each labelled with ten digits i.e. from 0 to 9.The lock opens with a sequence of four digits with no repeats.What is the probability of a person getting the right sequence to open the suitcase.
\\
\solution
		%\input{ncert/11/16/4/10/defs.tex}
		%
\item 
Two cards are drawn at random and without replacement from a pack of 52 playing cards. Find the probability that both the cards are black.
\\
\solution
		%\input{ncert/12/13/2/2/defs.tex}
		\item A box of oranges is inspected by examining three randomly selected oranges drawn without replacement. If all the three oranges are good, the box is approved for sale, otherwise, it is rejected. Find the probability that a box containing 15 oranges out of which 12 are good and 3 are bad ones will be approved for sale.
		\label{ncert/12/13/2/3/defs.tex}
		\item Two balls are drawn at random with replacement from a box containing 10 black and 8 red balls. Find the probability that
		\label{ncert/12/13/2/12}
\begin{enumerate}
\item both balls are red.
\item first ball is black and second is red.
\item one of them is black and other is red.
\end{enumerate}

\item In a hostel, 60\% of the students read Hindi newspaper, 40\% read English newspaper and 20\% read both Hindi and English newspapers. A student is selected at random.
		\label{ncert/12/13/2/15}
\begin{enumerate}
\item Find the probability that she reads neither Hindi nor English newspapers.
\item If she reads Hindi newspaper, find the probability that she reads English newspaper.
\item If she reads English newspaper, find the probability that she reads Hindi newspaper.\\
\end{enumerate}
\item The probability of obtaining an even prime number on each die, when a pair of dice is rolled is 
\begin{enumerate}
    \item $0$ 
    
    \item $\frac{1}{3}$ 
    
    \item $\frac{1}{12}$ 
    
    \item $\frac{1}{36}$ 
\end{enumerate}
\solution
		%\input{ncert/12/13/2/17/defs.tex}
	\item A bag contains 4 red and 4 black balls, another bag contains 2 red and 6 black balls. One of the two bags is selected at random and a ball is drawn from the bag which is found to be red. Find the probability that the ball is drawn from the first bag.
\\
\solution
		%\input{ncert/12/13/3/2/main.tex}
  \item
  Cards with numbers 2 to 101 are placed in a box. A card is selected at random.Find the probability that the card has
\begin{enumerate}[label=(\roman*)]
	\item an even number 
	\item a square number
\end{enumerate}
\solution
%\input{exemplar/10/13/3/32/main.tex}
\item
The king, queen and jack of clubs are removed from a deck of 52 playing cards and then well shuffled. Now one card is drawn at random from the remaining cards.  Determine the probability that the card is
\begin{enumerate}[label=(\roman*)]
\item a club
\item 10 of hearts
\end{enumerate}
\solution
%\input{exemplar/10/13/3/29/main.tex}
\item A team of medical students doing their internship have to assist during surgeries
at a city hospital. The probabilities of surgeries rated as very complex, complex,
routine, simple or very simple are respectively, 0.15, 0.20, 0.31, 0.26, .08. Find
the probabilities that a particular surgery will be rated
\begin{enumerate}
	\item complex or very complex;
	\item neither very complex nor very simple;
	\item routine or complex
	\item routine or simple
\end{enumerate}
\solution
%\input{exemplar/11/16/3/8(1)/main.tex}
\item A card is selected from a pack of 52 cards.
\begin{enumerate}[label=(\alph*)]
    \item How many points are there in the sample space?
    \item Calculate the probability that the card is an ace of spades.
    \item Calculate the probability that the card is (i) an ace and (ii) black card.
\end{enumerate}
\solution
%\input{exemplar/11/16/3/4/main2.tex}
\item The probability that a non leap year selected at random will contain 53 sundays.
\\
\solution
%\input{exemplar/10/13/1/19/main.tex}
\item One of the four persons John, Rita, Aslam or Gurpreet will be promoted next
month. Consequently the sample space consists of four elementary outcomes
S = {John promoted, Rita promoted, Aslam promoted, Gurpreet promoted}
You are told that the chances of John’s promotion is same as that of Gurpreet,
Rita’s chances of promotion are twice as likely as Johns. Aslam’s chances are
four times that of John.
\begin{enumerate}
	\item Determine
	\begin{enumerate}
		\item P (John promoted)
		\item P (Rita promoted)
		\item P (Aslam promoted)
		\item P (Gurpreet promoted)
	\end{enumerate}
	\item If A = {John promoted or Gurpreet promoted}, find P (A).
\end{enumerate}
\solution
%\input{exemplar/11/16/3/10/main.tex}
\item A card is drawn from a deck of 52 cards. Find the probability of getting a king or a heart or a red card.\\
\solution
%\input{exemplar/11/16/3/15/main.tex}
\item The probability that a student will pass his examination is 0.73, the probability of
the student getting a compartment is 0.13, and the probability that the student will
either pass or get compartment is 0.96. State True or False.\\
\solution
%\input{exemplar/11/16/3/31/main.tex}
\item A card is selected from a pack of 52 cards\\
\begin{enumerate}[label=(\alph*)]
\item How many points are there in the sample space?
\item Calculate the probability that the cards is an ace of spades.
\item Calculate the probability that the card is (i) an ace (ii)black card.\\
\end{enumerate}
%\input{ncert/11/16/3/4_1/Prob_4.tex}
\item In a non-leap year, the probability of having 53 tuesdays or 53 wednesdays is\\
\solution
%\input{exemplar/11/16/3/18/main.tex}
\item There are 1000 sealed envelopes in a box, 10 of them contain a cash prize of
Rs 100 each, 100 of them contain a cash prize of Rs 50 each and 200 of them
contain a cash prize of Rs 10 each and rest do not contain any cash prize. If they
are well shuffled and an envelope is picked up out, what is the probability that it
contains no cash prize?\\
\solution
%\input{exemplar/10/13/3/34/main.tex}
\item 
A die is thrown and a card is selected at random from a deck of 52 playing cards. The probability of getting an even number on the die and a spade card.\\
\solution
%\input{exemplar/12/13/3/78/main.tex}
\item
If 4-digit numbers greater than 5,000 are randomly formed from the digits 0, 1, 3, 5, and 7, what is the probability of forming a number divisible by 5 when:
\begin{enumerate}
    \item The digits are repeated?
    \item The repetition of digits is not allowed?
\end{enumerate}
\solution
%\input{ncert/11/16/4/9/main.tex}
\item Consider the probability space $\brak{\Omega, \mathcal{G}, P}$ where $\Omega = [0,2]$ and $\mathcal{G} = \cbrak{\phi, \Omega, [0,1], (1,2]}$. Let $X$ and $Y$ be two functions on $\Omega$ defined as
\begin{align*}
    X(\omega) = 
    \begin{cases}
        1 & \text{if }\omega \in [0, 1]\\
        2 & \text{if }\omega \in (1, 2]
    \end{cases}
\end{align*}
and
\begin{align*}
    Y(\omega) = 
    \begin{cases}
        2 & \text{if }\omega \in [0, 1.5]\\
        3 & \text{if }\omega \in (1.5, 2].
    \end{cases}
\end{align*}
Then which one of the following statements is true?
\begin{enumerate}
    \item [(A)] $X$ is a random variable with respect to $\mathcal{G}$, but $Y$ is not a random variable with respect to $\mathcal{G}$.
    \item [(B)] $Y$ is a random variable with respect to $\mathcal{G}$, but $X$ is not a random variable with respect to $\mathcal{G}$.
    \item [(C)] Neither $X$ nor $Y$ is a random variable with respect to $\mathcal{G}$.
    \item [(D)] Both $X$ and $Y$ are random variables with respect to $\mathcal{G}$.
\end{enumerate} \hfill (GATE ST 2023)\\
\solution
%\input{gate/ST/2023/14/main.tex}
	\item  A die is loaded in such a way that each odd number is twice as likely to occur as
each even number. Find $P(G)$, where $G$ is the event that a number greater than
3 occurs on a single roll of the die.
\\
\solution
		%\input{exemplar/11/16/3/5/main.tex}
	\item All the jacks, queens and kings are removed from a deck of 52 playing cards. The remaining cards are well shuffled and then one card is drawn at random. Giving ace a value 1 similar value for other cards, find the probability that the card has a value 
		\begin{enumerate}
			\item 7
			\item greater than 7
			\item less than 7
		\end{enumerate}
		%\input{exemplar/10/13/3/30/main.tex}
  \item A Lot consists of 48 mobile phones of which 42 are good, 3 have only minor defects and 3 have major defects.Varnika will buy a phone if it is good but the trader will only buy a mobile if it has no major defects. One phone is selected at random from the lot. What is the probability that it is
\begin{enumerate}
	\item acceptable to Varnika?
            \item acceptable to the trader?
\end{enumerate}
\solution
	%\input{exemplar/10/13/3/40/main.tex}
 \item A student says that if you throw a die, it will show up 1 or not 1. Therefore, the probability of getting 1 and the probability of getting 'not 1' each is equal to $\frac{1}{2}$. Is this correct? Give reasons.\\
 \solution
        %\input{exemplar/10/13/2/9/main.tex}
   \item Four candidates A, B, C, D have ap-
plied for the assignment to coach a school cricket
team. If A is twice as likely to be selected as B, and
B and C are given about the same chance of being
selected, while C is twice as likely to be selected
as D, what are the probabilities that
\begin{enumerate}
\item C will be selected?
\item A will not be selected?
\end{enumerate}
	%\input{exemplar/11/16/3/9/main.tex}
 \item A bag contain 24 balls of which $x$ balls are red, $2x$ are white and $3x$ are blue. A ball is selected at random, What is the probability that it is
\begin{enumerate}[label=\alph*)]
\item not red ?
\item white ?
\end{enumerate}
%\input{exemplar/10/13/3/41/main.tex}
If the letters of the word ASSASSINATION are arranged at random. Find the Probability that
\begin{enumerate}[label=(\alph*)]
\item Four $S's$ come consecutively in the word
\item Two  $I's$ and two $N's$ come together
\item All $A's$ are not coming together
\item No two $A's$ are coming together
\end{enumerate}
%\input{exemplar/11/16/3/14/main.tex}
	\item One urn contains two black balls (labelled B1 and B2) and one white ball. A
	second urn contains one black ball and two white balls (labelled W1 and W2).
	Suppose the following experiment is performed. One of the two urns is chosen
	at random. Next a ball is randomly chosen from the urn. Then a second ball is
	chosen at random from the same urn without replacing the first ball.
	
	\begin{enumerate}
	\item What is the probability that two black balls are chosen?
	
	\item What is the probability that two balls of opposite colour are chosen?
	\end{enumerate}
	\solution
	%\input{exemplar/11/16/3/12/main1.tex}
\end{enumerate}

	\item 
The number lock of a suitcase has 4 wheels each labelled with ten digits i.e. from 0 to 9.The lock opens with a sequence of four digits with no repeats.What is the probability of a person getting the right sequence to open the suitcase.
\\
\solution
		%\begin{enumerate}[label=\thesection.\arabic*,ref=\thesection.\theenumi]
	\item One card is drawn from a well-shuffled deck of 52 cards. Find the probability of getting
\begin{enumerate}
\item A king of red colour 
\item A face card 
\item A red face card
\item The jack of hearts
\item A spade
\item The queen of diamonds

\end{enumerate}
\solution
		%\input{ncert/10/15/1/14/main.tex}
	\item Five cards—the ten, jack, queen, king and ace of diamonds, are well-shuffled with their face downwards. One card is then picked up at random.
\begin{enumerate}
\item
What is the probability that the card is the queen? 
\item
If the queen is drawn and put aside, what is the probability that the second card picked up is (a) an ace? (b) a queen?\\
\end{enumerate}
\solution
		%\input{ncert/10/15/1/15/defs.tex}
	\item A bag contains $5$ red balls and some blue balls. If the probability of drawing a blue ball is double that if a red ball, determine the number of blue balls in the bag. 
		\\
\solution
		%\input{ncert/10/15/2/3/defs.tex}
	\item A card is selected from a pack of 52 cards.
 \begin{enumerate}[label=(\alph*)] 
                 \item How many points are there in the sample space?
                 \item Calculate the probability that the card is an ace of spades.
                 \item Calculate the probability that the card is (i) an ace and (ii) black card.
 \end{enumerate}
\solution
		%\input{ncert/11/16/3/4/main.tex}
\item Four cards are drawn from a well-shuffled deck of 52 cards. What is the probability of obtaining 3 diamonds and one spade.
\\
\solution
		%\input{ncert/11/16/4/2/defs.tex}
\item In a certain lottery 10,000 tickets are sold and ten equal prizes are awarded. What is the probability of not getting a prize if you buy (a) one ticket (b) two tickets (c) 10 tickets ?	
\\
\solution
		%\input{ncert/11/16/4/4/defs.tex}
		%
\item 
Out of 100 students, two sections of 40 and 60 are formed. If you and your friend are among the 100 students, what is the probability that
\begin{enumerate}
\item you both enter the same section?
\item you both enter the different sections?
\end{enumerate}
\solution
		%\input{ncert/11/16/4/5/defs.tex}
	\item 
The number lock of a suitcase has 4 wheels each labelled with ten digits i.e. from 0 to 9.The lock opens with a sequence of four digits with no repeats.What is the probability of a person getting the right sequence to open the suitcase.
\\
\solution
		%\input{ncert/11/16/4/10/defs.tex}
		%
\item 
Two cards are drawn at random and without replacement from a pack of 52 playing cards. Find the probability that both the cards are black.
\\
\solution
		%\input{ncert/12/13/2/2/defs.tex}
		\item A box of oranges is inspected by examining three randomly selected oranges drawn without replacement. If all the three oranges are good, the box is approved for sale, otherwise, it is rejected. Find the probability that a box containing 15 oranges out of which 12 are good and 3 are bad ones will be approved for sale.
		\label{ncert/12/13/2/3/defs.tex}
		\item Two balls are drawn at random with replacement from a box containing 10 black and 8 red balls. Find the probability that
		\label{ncert/12/13/2/12}
\begin{enumerate}
\item both balls are red.
\item first ball is black and second is red.
\item one of them is black and other is red.
\end{enumerate}

\item In a hostel, 60\% of the students read Hindi newspaper, 40\% read English newspaper and 20\% read both Hindi and English newspapers. A student is selected at random.
		\label{ncert/12/13/2/15}
\begin{enumerate}
\item Find the probability that she reads neither Hindi nor English newspapers.
\item If she reads Hindi newspaper, find the probability that she reads English newspaper.
\item If she reads English newspaper, find the probability that she reads Hindi newspaper.\\
\end{enumerate}
\item The probability of obtaining an even prime number on each die, when a pair of dice is rolled is 
\begin{enumerate}
    \item $0$ 
    
    \item $\frac{1}{3}$ 
    
    \item $\frac{1}{12}$ 
    
    \item $\frac{1}{36}$ 
\end{enumerate}
\solution
		%\input{ncert/12/13/2/17/defs.tex}
	\item A bag contains 4 red and 4 black balls, another bag contains 2 red and 6 black balls. One of the two bags is selected at random and a ball is drawn from the bag which is found to be red. Find the probability that the ball is drawn from the first bag.
\\
\solution
		%\input{ncert/12/13/3/2/main.tex}
  \item
  Cards with numbers 2 to 101 are placed in a box. A card is selected at random.Find the probability that the card has
\begin{enumerate}[label=(\roman*)]
	\item an even number 
	\item a square number
\end{enumerate}
\solution
%\input{exemplar/10/13/3/32/main.tex}
\item
The king, queen and jack of clubs are removed from a deck of 52 playing cards and then well shuffled. Now one card is drawn at random from the remaining cards.  Determine the probability that the card is
\begin{enumerate}[label=(\roman*)]
\item a club
\item 10 of hearts
\end{enumerate}
\solution
%\input{exemplar/10/13/3/29/main.tex}
\item A team of medical students doing their internship have to assist during surgeries
at a city hospital. The probabilities of surgeries rated as very complex, complex,
routine, simple or very simple are respectively, 0.15, 0.20, 0.31, 0.26, .08. Find
the probabilities that a particular surgery will be rated
\begin{enumerate}
	\item complex or very complex;
	\item neither very complex nor very simple;
	\item routine or complex
	\item routine or simple
\end{enumerate}
\solution
%\input{exemplar/11/16/3/8(1)/main.tex}
\item A card is selected from a pack of 52 cards.
\begin{enumerate}[label=(\alph*)]
    \item How many points are there in the sample space?
    \item Calculate the probability that the card is an ace of spades.
    \item Calculate the probability that the card is (i) an ace and (ii) black card.
\end{enumerate}
\solution
%\input{exemplar/11/16/3/4/main2.tex}
\item The probability that a non leap year selected at random will contain 53 sundays.
\\
\solution
%\input{exemplar/10/13/1/19/main.tex}
\item One of the four persons John, Rita, Aslam or Gurpreet will be promoted next
month. Consequently the sample space consists of four elementary outcomes
S = {John promoted, Rita promoted, Aslam promoted, Gurpreet promoted}
You are told that the chances of John’s promotion is same as that of Gurpreet,
Rita’s chances of promotion are twice as likely as Johns. Aslam’s chances are
four times that of John.
\begin{enumerate}
	\item Determine
	\begin{enumerate}
		\item P (John promoted)
		\item P (Rita promoted)
		\item P (Aslam promoted)
		\item P (Gurpreet promoted)
	\end{enumerate}
	\item If A = {John promoted or Gurpreet promoted}, find P (A).
\end{enumerate}
\solution
%\input{exemplar/11/16/3/10/main.tex}
\item A card is drawn from a deck of 52 cards. Find the probability of getting a king or a heart or a red card.\\
\solution
%\input{exemplar/11/16/3/15/main.tex}
\item The probability that a student will pass his examination is 0.73, the probability of
the student getting a compartment is 0.13, and the probability that the student will
either pass or get compartment is 0.96. State True or False.\\
\solution
%\input{exemplar/11/16/3/31/main.tex}
\item A card is selected from a pack of 52 cards\\
\begin{enumerate}[label=(\alph*)]
\item How many points are there in the sample space?
\item Calculate the probability that the cards is an ace of spades.
\item Calculate the probability that the card is (i) an ace (ii)black card.\\
\end{enumerate}
%\input{ncert/11/16/3/4_1/Prob_4.tex}
\item In a non-leap year, the probability of having 53 tuesdays or 53 wednesdays is\\
\solution
%\input{exemplar/11/16/3/18/main.tex}
\item There are 1000 sealed envelopes in a box, 10 of them contain a cash prize of
Rs 100 each, 100 of them contain a cash prize of Rs 50 each and 200 of them
contain a cash prize of Rs 10 each and rest do not contain any cash prize. If they
are well shuffled and an envelope is picked up out, what is the probability that it
contains no cash prize?\\
\solution
%\input{exemplar/10/13/3/34/main.tex}
\item 
A die is thrown and a card is selected at random from a deck of 52 playing cards. The probability of getting an even number on the die and a spade card.\\
\solution
%\input{exemplar/12/13/3/78/main.tex}
\item
If 4-digit numbers greater than 5,000 are randomly formed from the digits 0, 1, 3, 5, and 7, what is the probability of forming a number divisible by 5 when:
\begin{enumerate}
    \item The digits are repeated?
    \item The repetition of digits is not allowed?
\end{enumerate}
\solution
%\input{ncert/11/16/4/9/main.tex}
\item Consider the probability space $\brak{\Omega, \mathcal{G}, P}$ where $\Omega = [0,2]$ and $\mathcal{G} = \cbrak{\phi, \Omega, [0,1], (1,2]}$. Let $X$ and $Y$ be two functions on $\Omega$ defined as
\begin{align*}
    X(\omega) = 
    \begin{cases}
        1 & \text{if }\omega \in [0, 1]\\
        2 & \text{if }\omega \in (1, 2]
    \end{cases}
\end{align*}
and
\begin{align*}
    Y(\omega) = 
    \begin{cases}
        2 & \text{if }\omega \in [0, 1.5]\\
        3 & \text{if }\omega \in (1.5, 2].
    \end{cases}
\end{align*}
Then which one of the following statements is true?
\begin{enumerate}
    \item [(A)] $X$ is a random variable with respect to $\mathcal{G}$, but $Y$ is not a random variable with respect to $\mathcal{G}$.
    \item [(B)] $Y$ is a random variable with respect to $\mathcal{G}$, but $X$ is not a random variable with respect to $\mathcal{G}$.
    \item [(C)] Neither $X$ nor $Y$ is a random variable with respect to $\mathcal{G}$.
    \item [(D)] Both $X$ and $Y$ are random variables with respect to $\mathcal{G}$.
\end{enumerate} \hfill (GATE ST 2023)\\
\solution
%\input{gate/ST/2023/14/main.tex}
	\item  A die is loaded in such a way that each odd number is twice as likely to occur as
each even number. Find $P(G)$, where $G$ is the event that a number greater than
3 occurs on a single roll of the die.
\\
\solution
		%\input{exemplar/11/16/3/5/main.tex}
	\item All the jacks, queens and kings are removed from a deck of 52 playing cards. The remaining cards are well shuffled and then one card is drawn at random. Giving ace a value 1 similar value for other cards, find the probability that the card has a value 
		\begin{enumerate}
			\item 7
			\item greater than 7
			\item less than 7
		\end{enumerate}
		%\input{exemplar/10/13/3/30/main.tex}
  \item A Lot consists of 48 mobile phones of which 42 are good, 3 have only minor defects and 3 have major defects.Varnika will buy a phone if it is good but the trader will only buy a mobile if it has no major defects. One phone is selected at random from the lot. What is the probability that it is
\begin{enumerate}
	\item acceptable to Varnika?
            \item acceptable to the trader?
\end{enumerate}
\solution
	%\input{exemplar/10/13/3/40/main.tex}
 \item A student says that if you throw a die, it will show up 1 or not 1. Therefore, the probability of getting 1 and the probability of getting 'not 1' each is equal to $\frac{1}{2}$. Is this correct? Give reasons.\\
 \solution
        %\input{exemplar/10/13/2/9/main.tex}
   \item Four candidates A, B, C, D have ap-
plied for the assignment to coach a school cricket
team. If A is twice as likely to be selected as B, and
B and C are given about the same chance of being
selected, while C is twice as likely to be selected
as D, what are the probabilities that
\begin{enumerate}
\item C will be selected?
\item A will not be selected?
\end{enumerate}
	%\input{exemplar/11/16/3/9/main.tex}
 \item A bag contain 24 balls of which $x$ balls are red, $2x$ are white and $3x$ are blue. A ball is selected at random, What is the probability that it is
\begin{enumerate}[label=\alph*)]
\item not red ?
\item white ?
\end{enumerate}
%\input{exemplar/10/13/3/41/main.tex}
If the letters of the word ASSASSINATION are arranged at random. Find the Probability that
\begin{enumerate}[label=(\alph*)]
\item Four $S's$ come consecutively in the word
\item Two  $I's$ and two $N's$ come together
\item All $A's$ are not coming together
\item No two $A's$ are coming together
\end{enumerate}
%\input{exemplar/11/16/3/14/main.tex}
	\item One urn contains two black balls (labelled B1 and B2) and one white ball. A
	second urn contains one black ball and two white balls (labelled W1 and W2).
	Suppose the following experiment is performed. One of the two urns is chosen
	at random. Next a ball is randomly chosen from the urn. Then a second ball is
	chosen at random from the same urn without replacing the first ball.
	
	\begin{enumerate}
	\item What is the probability that two black balls are chosen?
	
	\item What is the probability that two balls of opposite colour are chosen?
	\end{enumerate}
	\solution
	%\input{exemplar/11/16/3/12/main1.tex}
\end{enumerate}

		%
\item 
Two cards are drawn at random and without replacement from a pack of 52 playing cards. Find the probability that both the cards are black.
\\
\solution
		%\begin{enumerate}[label=\thesection.\arabic*,ref=\thesection.\theenumi]
	\item One card is drawn from a well-shuffled deck of 52 cards. Find the probability of getting
\begin{enumerate}
\item A king of red colour 
\item A face card 
\item A red face card
\item The jack of hearts
\item A spade
\item The queen of diamonds

\end{enumerate}
\solution
		%\input{ncert/10/15/1/14/main.tex}
	\item Five cards—the ten, jack, queen, king and ace of diamonds, are well-shuffled with their face downwards. One card is then picked up at random.
\begin{enumerate}
\item
What is the probability that the card is the queen? 
\item
If the queen is drawn and put aside, what is the probability that the second card picked up is (a) an ace? (b) a queen?\\
\end{enumerate}
\solution
		%\input{ncert/10/15/1/15/defs.tex}
	\item A bag contains $5$ red balls and some blue balls. If the probability of drawing a blue ball is double that if a red ball, determine the number of blue balls in the bag. 
		\\
\solution
		%\input{ncert/10/15/2/3/defs.tex}
	\item A card is selected from a pack of 52 cards.
 \begin{enumerate}[label=(\alph*)] 
                 \item How many points are there in the sample space?
                 \item Calculate the probability that the card is an ace of spades.
                 \item Calculate the probability that the card is (i) an ace and (ii) black card.
 \end{enumerate}
\solution
		%\input{ncert/11/16/3/4/main.tex}
\item Four cards are drawn from a well-shuffled deck of 52 cards. What is the probability of obtaining 3 diamonds and one spade.
\\
\solution
		%\input{ncert/11/16/4/2/defs.tex}
\item In a certain lottery 10,000 tickets are sold and ten equal prizes are awarded. What is the probability of not getting a prize if you buy (a) one ticket (b) two tickets (c) 10 tickets ?	
\\
\solution
		%\input{ncert/11/16/4/4/defs.tex}
		%
\item 
Out of 100 students, two sections of 40 and 60 are formed. If you and your friend are among the 100 students, what is the probability that
\begin{enumerate}
\item you both enter the same section?
\item you both enter the different sections?
\end{enumerate}
\solution
		%\input{ncert/11/16/4/5/defs.tex}
	\item 
The number lock of a suitcase has 4 wheels each labelled with ten digits i.e. from 0 to 9.The lock opens with a sequence of four digits with no repeats.What is the probability of a person getting the right sequence to open the suitcase.
\\
\solution
		%\input{ncert/11/16/4/10/defs.tex}
		%
\item 
Two cards are drawn at random and without replacement from a pack of 52 playing cards. Find the probability that both the cards are black.
\\
\solution
		%\input{ncert/12/13/2/2/defs.tex}
		\item A box of oranges is inspected by examining three randomly selected oranges drawn without replacement. If all the three oranges are good, the box is approved for sale, otherwise, it is rejected. Find the probability that a box containing 15 oranges out of which 12 are good and 3 are bad ones will be approved for sale.
		\label{ncert/12/13/2/3/defs.tex}
		\item Two balls are drawn at random with replacement from a box containing 10 black and 8 red balls. Find the probability that
		\label{ncert/12/13/2/12}
\begin{enumerate}
\item both balls are red.
\item first ball is black and second is red.
\item one of them is black and other is red.
\end{enumerate}

\item In a hostel, 60\% of the students read Hindi newspaper, 40\% read English newspaper and 20\% read both Hindi and English newspapers. A student is selected at random.
		\label{ncert/12/13/2/15}
\begin{enumerate}
\item Find the probability that she reads neither Hindi nor English newspapers.
\item If she reads Hindi newspaper, find the probability that she reads English newspaper.
\item If she reads English newspaper, find the probability that she reads Hindi newspaper.\\
\end{enumerate}
\item The probability of obtaining an even prime number on each die, when a pair of dice is rolled is 
\begin{enumerate}
    \item $0$ 
    
    \item $\frac{1}{3}$ 
    
    \item $\frac{1}{12}$ 
    
    \item $\frac{1}{36}$ 
\end{enumerate}
\solution
		%\input{ncert/12/13/2/17/defs.tex}
	\item A bag contains 4 red and 4 black balls, another bag contains 2 red and 6 black balls. One of the two bags is selected at random and a ball is drawn from the bag which is found to be red. Find the probability that the ball is drawn from the first bag.
\\
\solution
		%\input{ncert/12/13/3/2/main.tex}
  \item
  Cards with numbers 2 to 101 are placed in a box. A card is selected at random.Find the probability that the card has
\begin{enumerate}[label=(\roman*)]
	\item an even number 
	\item a square number
\end{enumerate}
\solution
%\input{exemplar/10/13/3/32/main.tex}
\item
The king, queen and jack of clubs are removed from a deck of 52 playing cards and then well shuffled. Now one card is drawn at random from the remaining cards.  Determine the probability that the card is
\begin{enumerate}[label=(\roman*)]
\item a club
\item 10 of hearts
\end{enumerate}
\solution
%\input{exemplar/10/13/3/29/main.tex}
\item A team of medical students doing their internship have to assist during surgeries
at a city hospital. The probabilities of surgeries rated as very complex, complex,
routine, simple or very simple are respectively, 0.15, 0.20, 0.31, 0.26, .08. Find
the probabilities that a particular surgery will be rated
\begin{enumerate}
	\item complex or very complex;
	\item neither very complex nor very simple;
	\item routine or complex
	\item routine or simple
\end{enumerate}
\solution
%\input{exemplar/11/16/3/8(1)/main.tex}
\item A card is selected from a pack of 52 cards.
\begin{enumerate}[label=(\alph*)]
    \item How many points are there in the sample space?
    \item Calculate the probability that the card is an ace of spades.
    \item Calculate the probability that the card is (i) an ace and (ii) black card.
\end{enumerate}
\solution
%\input{exemplar/11/16/3/4/main2.tex}
\item The probability that a non leap year selected at random will contain 53 sundays.
\\
\solution
%\input{exemplar/10/13/1/19/main.tex}
\item One of the four persons John, Rita, Aslam or Gurpreet will be promoted next
month. Consequently the sample space consists of four elementary outcomes
S = {John promoted, Rita promoted, Aslam promoted, Gurpreet promoted}
You are told that the chances of John’s promotion is same as that of Gurpreet,
Rita’s chances of promotion are twice as likely as Johns. Aslam’s chances are
four times that of John.
\begin{enumerate}
	\item Determine
	\begin{enumerate}
		\item P (John promoted)
		\item P (Rita promoted)
		\item P (Aslam promoted)
		\item P (Gurpreet promoted)
	\end{enumerate}
	\item If A = {John promoted or Gurpreet promoted}, find P (A).
\end{enumerate}
\solution
%\input{exemplar/11/16/3/10/main.tex}
\item A card is drawn from a deck of 52 cards. Find the probability of getting a king or a heart or a red card.\\
\solution
%\input{exemplar/11/16/3/15/main.tex}
\item The probability that a student will pass his examination is 0.73, the probability of
the student getting a compartment is 0.13, and the probability that the student will
either pass or get compartment is 0.96. State True or False.\\
\solution
%\input{exemplar/11/16/3/31/main.tex}
\item A card is selected from a pack of 52 cards\\
\begin{enumerate}[label=(\alph*)]
\item How many points are there in the sample space?
\item Calculate the probability that the cards is an ace of spades.
\item Calculate the probability that the card is (i) an ace (ii)black card.\\
\end{enumerate}
%\input{ncert/11/16/3/4_1/Prob_4.tex}
\item In a non-leap year, the probability of having 53 tuesdays or 53 wednesdays is\\
\solution
%\input{exemplar/11/16/3/18/main.tex}
\item There are 1000 sealed envelopes in a box, 10 of them contain a cash prize of
Rs 100 each, 100 of them contain a cash prize of Rs 50 each and 200 of them
contain a cash prize of Rs 10 each and rest do not contain any cash prize. If they
are well shuffled and an envelope is picked up out, what is the probability that it
contains no cash prize?\\
\solution
%\input{exemplar/10/13/3/34/main.tex}
\item 
A die is thrown and a card is selected at random from a deck of 52 playing cards. The probability of getting an even number on the die and a spade card.\\
\solution
%\input{exemplar/12/13/3/78/main.tex}
\item
If 4-digit numbers greater than 5,000 are randomly formed from the digits 0, 1, 3, 5, and 7, what is the probability of forming a number divisible by 5 when:
\begin{enumerate}
    \item The digits are repeated?
    \item The repetition of digits is not allowed?
\end{enumerate}
\solution
%\input{ncert/11/16/4/9/main.tex}
\item Consider the probability space $\brak{\Omega, \mathcal{G}, P}$ where $\Omega = [0,2]$ and $\mathcal{G} = \cbrak{\phi, \Omega, [0,1], (1,2]}$. Let $X$ and $Y$ be two functions on $\Omega$ defined as
\begin{align*}
    X(\omega) = 
    \begin{cases}
        1 & \text{if }\omega \in [0, 1]\\
        2 & \text{if }\omega \in (1, 2]
    \end{cases}
\end{align*}
and
\begin{align*}
    Y(\omega) = 
    \begin{cases}
        2 & \text{if }\omega \in [0, 1.5]\\
        3 & \text{if }\omega \in (1.5, 2].
    \end{cases}
\end{align*}
Then which one of the following statements is true?
\begin{enumerate}
    \item [(A)] $X$ is a random variable with respect to $\mathcal{G}$, but $Y$ is not a random variable with respect to $\mathcal{G}$.
    \item [(B)] $Y$ is a random variable with respect to $\mathcal{G}$, but $X$ is not a random variable with respect to $\mathcal{G}$.
    \item [(C)] Neither $X$ nor $Y$ is a random variable with respect to $\mathcal{G}$.
    \item [(D)] Both $X$ and $Y$ are random variables with respect to $\mathcal{G}$.
\end{enumerate} \hfill (GATE ST 2023)\\
\solution
%\input{gate/ST/2023/14/main.tex}
	\item  A die is loaded in such a way that each odd number is twice as likely to occur as
each even number. Find $P(G)$, where $G$ is the event that a number greater than
3 occurs on a single roll of the die.
\\
\solution
		%\input{exemplar/11/16/3/5/main.tex}
	\item All the jacks, queens and kings are removed from a deck of 52 playing cards. The remaining cards are well shuffled and then one card is drawn at random. Giving ace a value 1 similar value for other cards, find the probability that the card has a value 
		\begin{enumerate}
			\item 7
			\item greater than 7
			\item less than 7
		\end{enumerate}
		%\input{exemplar/10/13/3/30/main.tex}
  \item A Lot consists of 48 mobile phones of which 42 are good, 3 have only minor defects and 3 have major defects.Varnika will buy a phone if it is good but the trader will only buy a mobile if it has no major defects. One phone is selected at random from the lot. What is the probability that it is
\begin{enumerate}
	\item acceptable to Varnika?
            \item acceptable to the trader?
\end{enumerate}
\solution
	%\input{exemplar/10/13/3/40/main.tex}
 \item A student says that if you throw a die, it will show up 1 or not 1. Therefore, the probability of getting 1 and the probability of getting 'not 1' each is equal to $\frac{1}{2}$. Is this correct? Give reasons.\\
 \solution
        %\input{exemplar/10/13/2/9/main.tex}
   \item Four candidates A, B, C, D have ap-
plied for the assignment to coach a school cricket
team. If A is twice as likely to be selected as B, and
B and C are given about the same chance of being
selected, while C is twice as likely to be selected
as D, what are the probabilities that
\begin{enumerate}
\item C will be selected?
\item A will not be selected?
\end{enumerate}
	%\input{exemplar/11/16/3/9/main.tex}
 \item A bag contain 24 balls of which $x$ balls are red, $2x$ are white and $3x$ are blue. A ball is selected at random, What is the probability that it is
\begin{enumerate}[label=\alph*)]
\item not red ?
\item white ?
\end{enumerate}
%\input{exemplar/10/13/3/41/main.tex}
If the letters of the word ASSASSINATION are arranged at random. Find the Probability that
\begin{enumerate}[label=(\alph*)]
\item Four $S's$ come consecutively in the word
\item Two  $I's$ and two $N's$ come together
\item All $A's$ are not coming together
\item No two $A's$ are coming together
\end{enumerate}
%\input{exemplar/11/16/3/14/main.tex}
	\item One urn contains two black balls (labelled B1 and B2) and one white ball. A
	second urn contains one black ball and two white balls (labelled W1 and W2).
	Suppose the following experiment is performed. One of the two urns is chosen
	at random. Next a ball is randomly chosen from the urn. Then a second ball is
	chosen at random from the same urn without replacing the first ball.
	
	\begin{enumerate}
	\item What is the probability that two black balls are chosen?
	
	\item What is the probability that two balls of opposite colour are chosen?
	\end{enumerate}
	\solution
	%\input{exemplar/11/16/3/12/main1.tex}
\end{enumerate}

		\item A box of oranges is inspected by examining three randomly selected oranges drawn without replacement. If all the three oranges are good, the box is approved for sale, otherwise, it is rejected. Find the probability that a box containing 15 oranges out of which 12 are good and 3 are bad ones will be approved for sale.
		\label{ncert/12/13/2/3/defs.tex}
		\item Two balls are drawn at random with replacement from a box containing 10 black and 8 red balls. Find the probability that
		\label{ncert/12/13/2/12}
\begin{enumerate}
\item both balls are red.
\item first ball is black and second is red.
\item one of them is black and other is red.
\end{enumerate}

\item In a hostel, 60\% of the students read Hindi newspaper, 40\% read English newspaper and 20\% read both Hindi and English newspapers. A student is selected at random.
		\label{ncert/12/13/2/15}
\begin{enumerate}
\item Find the probability that she reads neither Hindi nor English newspapers.
\item If she reads Hindi newspaper, find the probability that she reads English newspaper.
\item If she reads English newspaper, find the probability that she reads Hindi newspaper.\\
\end{enumerate}
\item The probability of obtaining an even prime number on each die, when a pair of dice is rolled is 
\begin{enumerate}
    \item $0$ 
    
    \item $\frac{1}{3}$ 
    
    \item $\frac{1}{12}$ 
    
    \item $\frac{1}{36}$ 
\end{enumerate}
\solution
		%\begin{enumerate}[label=\thesection.\arabic*,ref=\thesection.\theenumi]
	\item One card is drawn from a well-shuffled deck of 52 cards. Find the probability of getting
\begin{enumerate}
\item A king of red colour 
\item A face card 
\item A red face card
\item The jack of hearts
\item A spade
\item The queen of diamonds

\end{enumerate}
\solution
		%\input{ncert/10/15/1/14/main.tex}
	\item Five cards—the ten, jack, queen, king and ace of diamonds, are well-shuffled with their face downwards. One card is then picked up at random.
\begin{enumerate}
\item
What is the probability that the card is the queen? 
\item
If the queen is drawn and put aside, what is the probability that the second card picked up is (a) an ace? (b) a queen?\\
\end{enumerate}
\solution
		%\input{ncert/10/15/1/15/defs.tex}
	\item A bag contains $5$ red balls and some blue balls. If the probability of drawing a blue ball is double that if a red ball, determine the number of blue balls in the bag. 
		\\
\solution
		%\input{ncert/10/15/2/3/defs.tex}
	\item A card is selected from a pack of 52 cards.
 \begin{enumerate}[label=(\alph*)] 
                 \item How many points are there in the sample space?
                 \item Calculate the probability that the card is an ace of spades.
                 \item Calculate the probability that the card is (i) an ace and (ii) black card.
 \end{enumerate}
\solution
		%\input{ncert/11/16/3/4/main.tex}
\item Four cards are drawn from a well-shuffled deck of 52 cards. What is the probability of obtaining 3 diamonds and one spade.
\\
\solution
		%\input{ncert/11/16/4/2/defs.tex}
\item In a certain lottery 10,000 tickets are sold and ten equal prizes are awarded. What is the probability of not getting a prize if you buy (a) one ticket (b) two tickets (c) 10 tickets ?	
\\
\solution
		%\input{ncert/11/16/4/4/defs.tex}
		%
\item 
Out of 100 students, two sections of 40 and 60 are formed. If you and your friend are among the 100 students, what is the probability that
\begin{enumerate}
\item you both enter the same section?
\item you both enter the different sections?
\end{enumerate}
\solution
		%\input{ncert/11/16/4/5/defs.tex}
	\item 
The number lock of a suitcase has 4 wheels each labelled with ten digits i.e. from 0 to 9.The lock opens with a sequence of four digits with no repeats.What is the probability of a person getting the right sequence to open the suitcase.
\\
\solution
		%\input{ncert/11/16/4/10/defs.tex}
		%
\item 
Two cards are drawn at random and without replacement from a pack of 52 playing cards. Find the probability that both the cards are black.
\\
\solution
		%\input{ncert/12/13/2/2/defs.tex}
		\item A box of oranges is inspected by examining three randomly selected oranges drawn without replacement. If all the three oranges are good, the box is approved for sale, otherwise, it is rejected. Find the probability that a box containing 15 oranges out of which 12 are good and 3 are bad ones will be approved for sale.
		\label{ncert/12/13/2/3/defs.tex}
		\item Two balls are drawn at random with replacement from a box containing 10 black and 8 red balls. Find the probability that
		\label{ncert/12/13/2/12}
\begin{enumerate}
\item both balls are red.
\item first ball is black and second is red.
\item one of them is black and other is red.
\end{enumerate}

\item In a hostel, 60\% of the students read Hindi newspaper, 40\% read English newspaper and 20\% read both Hindi and English newspapers. A student is selected at random.
		\label{ncert/12/13/2/15}
\begin{enumerate}
\item Find the probability that she reads neither Hindi nor English newspapers.
\item If she reads Hindi newspaper, find the probability that she reads English newspaper.
\item If she reads English newspaper, find the probability that she reads Hindi newspaper.\\
\end{enumerate}
\item The probability of obtaining an even prime number on each die, when a pair of dice is rolled is 
\begin{enumerate}
    \item $0$ 
    
    \item $\frac{1}{3}$ 
    
    \item $\frac{1}{12}$ 
    
    \item $\frac{1}{36}$ 
\end{enumerate}
\solution
		%\input{ncert/12/13/2/17/defs.tex}
	\item A bag contains 4 red and 4 black balls, another bag contains 2 red and 6 black balls. One of the two bags is selected at random and a ball is drawn from the bag which is found to be red. Find the probability that the ball is drawn from the first bag.
\\
\solution
		%\input{ncert/12/13/3/2/main.tex}
  \item
  Cards with numbers 2 to 101 are placed in a box. A card is selected at random.Find the probability that the card has
\begin{enumerate}[label=(\roman*)]
	\item an even number 
	\item a square number
\end{enumerate}
\solution
%\input{exemplar/10/13/3/32/main.tex}
\item
The king, queen and jack of clubs are removed from a deck of 52 playing cards and then well shuffled. Now one card is drawn at random from the remaining cards.  Determine the probability that the card is
\begin{enumerate}[label=(\roman*)]
\item a club
\item 10 of hearts
\end{enumerate}
\solution
%\input{exemplar/10/13/3/29/main.tex}
\item A team of medical students doing their internship have to assist during surgeries
at a city hospital. The probabilities of surgeries rated as very complex, complex,
routine, simple or very simple are respectively, 0.15, 0.20, 0.31, 0.26, .08. Find
the probabilities that a particular surgery will be rated
\begin{enumerate}
	\item complex or very complex;
	\item neither very complex nor very simple;
	\item routine or complex
	\item routine or simple
\end{enumerate}
\solution
%\input{exemplar/11/16/3/8(1)/main.tex}
\item A card is selected from a pack of 52 cards.
\begin{enumerate}[label=(\alph*)]
    \item How many points are there in the sample space?
    \item Calculate the probability that the card is an ace of spades.
    \item Calculate the probability that the card is (i) an ace and (ii) black card.
\end{enumerate}
\solution
%\input{exemplar/11/16/3/4/main2.tex}
\item The probability that a non leap year selected at random will contain 53 sundays.
\\
\solution
%\input{exemplar/10/13/1/19/main.tex}
\item One of the four persons John, Rita, Aslam or Gurpreet will be promoted next
month. Consequently the sample space consists of four elementary outcomes
S = {John promoted, Rita promoted, Aslam promoted, Gurpreet promoted}
You are told that the chances of John’s promotion is same as that of Gurpreet,
Rita’s chances of promotion are twice as likely as Johns. Aslam’s chances are
four times that of John.
\begin{enumerate}
	\item Determine
	\begin{enumerate}
		\item P (John promoted)
		\item P (Rita promoted)
		\item P (Aslam promoted)
		\item P (Gurpreet promoted)
	\end{enumerate}
	\item If A = {John promoted or Gurpreet promoted}, find P (A).
\end{enumerate}
\solution
%\input{exemplar/11/16/3/10/main.tex}
\item A card is drawn from a deck of 52 cards. Find the probability of getting a king or a heart or a red card.\\
\solution
%\input{exemplar/11/16/3/15/main.tex}
\item The probability that a student will pass his examination is 0.73, the probability of
the student getting a compartment is 0.13, and the probability that the student will
either pass or get compartment is 0.96. State True or False.\\
\solution
%\input{exemplar/11/16/3/31/main.tex}
\item A card is selected from a pack of 52 cards\\
\begin{enumerate}[label=(\alph*)]
\item How many points are there in the sample space?
\item Calculate the probability that the cards is an ace of spades.
\item Calculate the probability that the card is (i) an ace (ii)black card.\\
\end{enumerate}
%\input{ncert/11/16/3/4_1/Prob_4.tex}
\item In a non-leap year, the probability of having 53 tuesdays or 53 wednesdays is\\
\solution
%\input{exemplar/11/16/3/18/main.tex}
\item There are 1000 sealed envelopes in a box, 10 of them contain a cash prize of
Rs 100 each, 100 of them contain a cash prize of Rs 50 each and 200 of them
contain a cash prize of Rs 10 each and rest do not contain any cash prize. If they
are well shuffled and an envelope is picked up out, what is the probability that it
contains no cash prize?\\
\solution
%\input{exemplar/10/13/3/34/main.tex}
\item 
A die is thrown and a card is selected at random from a deck of 52 playing cards. The probability of getting an even number on the die and a spade card.\\
\solution
%\input{exemplar/12/13/3/78/main.tex}
\item
If 4-digit numbers greater than 5,000 are randomly formed from the digits 0, 1, 3, 5, and 7, what is the probability of forming a number divisible by 5 when:
\begin{enumerate}
    \item The digits are repeated?
    \item The repetition of digits is not allowed?
\end{enumerate}
\solution
%\input{ncert/11/16/4/9/main.tex}
\item Consider the probability space $\brak{\Omega, \mathcal{G}, P}$ where $\Omega = [0,2]$ and $\mathcal{G} = \cbrak{\phi, \Omega, [0,1], (1,2]}$. Let $X$ and $Y$ be two functions on $\Omega$ defined as
\begin{align*}
    X(\omega) = 
    \begin{cases}
        1 & \text{if }\omega \in [0, 1]\\
        2 & \text{if }\omega \in (1, 2]
    \end{cases}
\end{align*}
and
\begin{align*}
    Y(\omega) = 
    \begin{cases}
        2 & \text{if }\omega \in [0, 1.5]\\
        3 & \text{if }\omega \in (1.5, 2].
    \end{cases}
\end{align*}
Then which one of the following statements is true?
\begin{enumerate}
    \item [(A)] $X$ is a random variable with respect to $\mathcal{G}$, but $Y$ is not a random variable with respect to $\mathcal{G}$.
    \item [(B)] $Y$ is a random variable with respect to $\mathcal{G}$, but $X$ is not a random variable with respect to $\mathcal{G}$.
    \item [(C)] Neither $X$ nor $Y$ is a random variable with respect to $\mathcal{G}$.
    \item [(D)] Both $X$ and $Y$ are random variables with respect to $\mathcal{G}$.
\end{enumerate} \hfill (GATE ST 2023)\\
\solution
%\input{gate/ST/2023/14/main.tex}
	\item  A die is loaded in such a way that each odd number is twice as likely to occur as
each even number. Find $P(G)$, where $G$ is the event that a number greater than
3 occurs on a single roll of the die.
\\
\solution
		%\input{exemplar/11/16/3/5/main.tex}
	\item All the jacks, queens and kings are removed from a deck of 52 playing cards. The remaining cards are well shuffled and then one card is drawn at random. Giving ace a value 1 similar value for other cards, find the probability that the card has a value 
		\begin{enumerate}
			\item 7
			\item greater than 7
			\item less than 7
		\end{enumerate}
		%\input{exemplar/10/13/3/30/main.tex}
  \item A Lot consists of 48 mobile phones of which 42 are good, 3 have only minor defects and 3 have major defects.Varnika will buy a phone if it is good but the trader will only buy a mobile if it has no major defects. One phone is selected at random from the lot. What is the probability that it is
\begin{enumerate}
	\item acceptable to Varnika?
            \item acceptable to the trader?
\end{enumerate}
\solution
	%\input{exemplar/10/13/3/40/main.tex}
 \item A student says that if you throw a die, it will show up 1 or not 1. Therefore, the probability of getting 1 and the probability of getting 'not 1' each is equal to $\frac{1}{2}$. Is this correct? Give reasons.\\
 \solution
        %\input{exemplar/10/13/2/9/main.tex}
   \item Four candidates A, B, C, D have ap-
plied for the assignment to coach a school cricket
team. If A is twice as likely to be selected as B, and
B and C are given about the same chance of being
selected, while C is twice as likely to be selected
as D, what are the probabilities that
\begin{enumerate}
\item C will be selected?
\item A will not be selected?
\end{enumerate}
	%\input{exemplar/11/16/3/9/main.tex}
 \item A bag contain 24 balls of which $x$ balls are red, $2x$ are white and $3x$ are blue. A ball is selected at random, What is the probability that it is
\begin{enumerate}[label=\alph*)]
\item not red ?
\item white ?
\end{enumerate}
%\input{exemplar/10/13/3/41/main.tex}
If the letters of the word ASSASSINATION are arranged at random. Find the Probability that
\begin{enumerate}[label=(\alph*)]
\item Four $S's$ come consecutively in the word
\item Two  $I's$ and two $N's$ come together
\item All $A's$ are not coming together
\item No two $A's$ are coming together
\end{enumerate}
%\input{exemplar/11/16/3/14/main.tex}
	\item One urn contains two black balls (labelled B1 and B2) and one white ball. A
	second urn contains one black ball and two white balls (labelled W1 and W2).
	Suppose the following experiment is performed. One of the two urns is chosen
	at random. Next a ball is randomly chosen from the urn. Then a second ball is
	chosen at random from the same urn without replacing the first ball.
	
	\begin{enumerate}
	\item What is the probability that two black balls are chosen?
	
	\item What is the probability that two balls of opposite colour are chosen?
	\end{enumerate}
	\solution
	%\input{exemplar/11/16/3/12/main1.tex}
\end{enumerate}

	\item A bag contains 4 red and 4 black balls, another bag contains 2 red and 6 black balls. One of the two bags is selected at random and a ball is drawn from the bag which is found to be red. Find the probability that the ball is drawn from the first bag.
\\
\solution
		%\begin{table}[H]
	\centering
\begin{tabular}{|c|c|c|}
\hline
Random variable &Value &Definition\\ \hline
\multirow{3}{*}{X} &0 &Slips of Rs 1\\
&1 &Slips of Rs 5\\
&2 &Slips of Rs 13\\ \hline
\multirow{2}{*}{Y} &0 &Box A\\
&1 &Box B\\\hline
\end{tabular}
\caption{}
\label{tab:Distribution}
\end{table}
See \tabref{tab:Distribution}.
\begin{align}
p_{Y}\brak{k}= \begin{cases} 
      \frac{1}{3} & {k=0} \\
      \frac{2}{3 }& {k=1} 
   \end{cases}
   \\
p_{Y|X}\brak{0|0} = \frac{19}{25}\, 
p_{Y|X}\brak{0|1} = \frac{6}{25}\,
p_{Y|X}\brak{1|0} = \frac{45}{50}\,
p_{Y|X}\brak{1|2} = \frac{5}{50}
\end{align}
The desired probability is the probability that a slip drawn at random is marked other than Rs 1,
\begin{align}
&=1-p_X\brak{0}\\
&= p_X(1) + p_X(2)
\end{align}
Using Bayes theorem,
\begin{align}
&= p_Y\brak{0} \times \pr{Y=0 | X=1} + p_Y\brak{1} \times \pr{Y=1|X=2}\\
&=\frac{1}{3} \times \frac{6}{25} + \frac{2}{3} \times \frac{5}{50}\\
&=\frac{11}{75}
\end{align}

\newpage

%\tableofcontents

\bigskip

\renewcommand{\thefigure}{\theenumi}
\renewcommand{\thetable}{\theenumi}
%\renewcommand{\theequation}{\theenumi}

%\begin{abstract}
%%\boldmath
%In this letter, an algorithm for evaluating the exact analytical bit error rate  (BER)  for the piecewise linear (PL) combiner for  multiple relays is presented. Previous results were available only for upto three relays. The algorithm is unique in the sense that  the actual mathematical expressions, that are prohibitively large, need not be explicitly obtained. The diversity gain due to multiple relays is shown through plots of the analytical BER, well supported by simulations. 
%
%\end{abstract}
% IEEEtran.cls defaults to using nonbold math in the Abstract.
% This preserves the distinction between vectors and scalars. However,
% if the journal you are submitting to favors bold math in the abstract,
% then you can use LaTeX's standard command \boldmath at the very start
% of the abstract to achieve this. Many IEEE journals frown on math
% in the abstract anyway.

% Note that keywords are not normally used for peerreview papers.
%\begin{IEEEkeywords}
%Cooperative diversity, decode and forward, piecewise linear
%\end{IEEEkeywords}



% For peer review papers, you can put extra information on the cover
% page as needed:
% \ifCLASSOPTIONpeerreview
% \begin{center} \bfseries EDICS Category: 3-BBND \end{center}
% \fi
%
% For peerreview papers, this IEEEtran command inserts a page break and
% creates the second title. It will be ignored for other modes.
%\IEEEpeerreviewmaketitle




  \item
  Cards with numbers 2 to 101 are placed in a box. A card is selected at random.Find the probability that the card has
\begin{enumerate}[label=(\roman*)]
	\item an even number 
	\item a square number
\end{enumerate}
\solution
%\begin{table}[H]
	\centering
\begin{tabular}{|c|c|c|}
\hline
Random variable &Value &Definition\\ \hline
\multirow{3}{*}{X} &0 &Slips of Rs 1\\
&1 &Slips of Rs 5\\
&2 &Slips of Rs 13\\ \hline
\multirow{2}{*}{Y} &0 &Box A\\
&1 &Box B\\\hline
\end{tabular}
\caption{}
\label{tab:Distribution}
\end{table}
See \tabref{tab:Distribution}.
\begin{align}
p_{Y}\brak{k}= \begin{cases} 
      \frac{1}{3} & {k=0} \\
      \frac{2}{3 }& {k=1} 
   \end{cases}
   \\
p_{Y|X}\brak{0|0} = \frac{19}{25}\, 
p_{Y|X}\brak{0|1} = \frac{6}{25}\,
p_{Y|X}\brak{1|0} = \frac{45}{50}\,
p_{Y|X}\brak{1|2} = \frac{5}{50}
\end{align}
The desired probability is the probability that a slip drawn at random is marked other than Rs 1,
\begin{align}
&=1-p_X\brak{0}\\
&= p_X(1) + p_X(2)
\end{align}
Using Bayes theorem,
\begin{align}
&= p_Y\brak{0} \times \pr{Y=0 | X=1} + p_Y\brak{1} \times \pr{Y=1|X=2}\\
&=\frac{1}{3} \times \frac{6}{25} + \frac{2}{3} \times \frac{5}{50}\\
&=\frac{11}{75}
\end{align}

\newpage

%\tableofcontents

\bigskip

\renewcommand{\thefigure}{\theenumi}
\renewcommand{\thetable}{\theenumi}
%\renewcommand{\theequation}{\theenumi}

%\begin{abstract}
%%\boldmath
%In this letter, an algorithm for evaluating the exact analytical bit error rate  (BER)  for the piecewise linear (PL) combiner for  multiple relays is presented. Previous results were available only for upto three relays. The algorithm is unique in the sense that  the actual mathematical expressions, that are prohibitively large, need not be explicitly obtained. The diversity gain due to multiple relays is shown through plots of the analytical BER, well supported by simulations. 
%
%\end{abstract}
% IEEEtran.cls defaults to using nonbold math in the Abstract.
% This preserves the distinction between vectors and scalars. However,
% if the journal you are submitting to favors bold math in the abstract,
% then you can use LaTeX's standard command \boldmath at the very start
% of the abstract to achieve this. Many IEEE journals frown on math
% in the abstract anyway.

% Note that keywords are not normally used for peerreview papers.
%\begin{IEEEkeywords}
%Cooperative diversity, decode and forward, piecewise linear
%\end{IEEEkeywords}



% For peer review papers, you can put extra information on the cover
% page as needed:
% \ifCLASSOPTIONpeerreview
% \begin{center} \bfseries EDICS Category: 3-BBND \end{center}
% \fi
%
% For peerreview papers, this IEEEtran command inserts a page break and
% creates the second title. It will be ignored for other modes.
%\IEEEpeerreviewmaketitle




\item
The king, queen and jack of clubs are removed from a deck of 52 playing cards and then well shuffled. Now one card is drawn at random from the remaining cards.  Determine the probability that the card is
\begin{enumerate}[label=(\roman*)]
\item a club
\item 10 of hearts
\end{enumerate}
\solution
%\begin{table}[H]
	\centering
\begin{tabular}{|c|c|c|}
\hline
Random variable &Value &Definition\\ \hline
\multirow{3}{*}{X} &0 &Slips of Rs 1\\
&1 &Slips of Rs 5\\
&2 &Slips of Rs 13\\ \hline
\multirow{2}{*}{Y} &0 &Box A\\
&1 &Box B\\\hline
\end{tabular}
\caption{}
\label{tab:Distribution}
\end{table}
See \tabref{tab:Distribution}.
\begin{align}
p_{Y}\brak{k}= \begin{cases} 
      \frac{1}{3} & {k=0} \\
      \frac{2}{3 }& {k=1} 
   \end{cases}
   \\
p_{Y|X}\brak{0|0} = \frac{19}{25}\, 
p_{Y|X}\brak{0|1} = \frac{6}{25}\,
p_{Y|X}\brak{1|0} = \frac{45}{50}\,
p_{Y|X}\brak{1|2} = \frac{5}{50}
\end{align}
The desired probability is the probability that a slip drawn at random is marked other than Rs 1,
\begin{align}
&=1-p_X\brak{0}\\
&= p_X(1) + p_X(2)
\end{align}
Using Bayes theorem,
\begin{align}
&= p_Y\brak{0} \times \pr{Y=0 | X=1} + p_Y\brak{1} \times \pr{Y=1|X=2}\\
&=\frac{1}{3} \times \frac{6}{25} + \frac{2}{3} \times \frac{5}{50}\\
&=\frac{11}{75}
\end{align}

\newpage

%\tableofcontents

\bigskip

\renewcommand{\thefigure}{\theenumi}
\renewcommand{\thetable}{\theenumi}
%\renewcommand{\theequation}{\theenumi}

%\begin{abstract}
%%\boldmath
%In this letter, an algorithm for evaluating the exact analytical bit error rate  (BER)  for the piecewise linear (PL) combiner for  multiple relays is presented. Previous results were available only for upto three relays. The algorithm is unique in the sense that  the actual mathematical expressions, that are prohibitively large, need not be explicitly obtained. The diversity gain due to multiple relays is shown through plots of the analytical BER, well supported by simulations. 
%
%\end{abstract}
% IEEEtran.cls defaults to using nonbold math in the Abstract.
% This preserves the distinction between vectors and scalars. However,
% if the journal you are submitting to favors bold math in the abstract,
% then you can use LaTeX's standard command \boldmath at the very start
% of the abstract to achieve this. Many IEEE journals frown on math
% in the abstract anyway.

% Note that keywords are not normally used for peerreview papers.
%\begin{IEEEkeywords}
%Cooperative diversity, decode and forward, piecewise linear
%\end{IEEEkeywords}



% For peer review papers, you can put extra information on the cover
% page as needed:
% \ifCLASSOPTIONpeerreview
% \begin{center} \bfseries EDICS Category: 3-BBND \end{center}
% \fi
%
% For peerreview papers, this IEEEtran command inserts a page break and
% creates the second title. It will be ignored for other modes.
%\IEEEpeerreviewmaketitle




\item A team of medical students doing their internship have to assist during surgeries
at a city hospital. The probabilities of surgeries rated as very complex, complex,
routine, simple or very simple are respectively, 0.15, 0.20, 0.31, 0.26, .08. Find
the probabilities that a particular surgery will be rated
\begin{enumerate}
	\item complex or very complex;
	\item neither very complex nor very simple;
	\item routine or complex
	\item routine or simple
\end{enumerate}
\solution
%\begin{table}[H]
	\centering
\begin{tabular}{|c|c|c|}
\hline
Random variable &Value &Definition\\ \hline
\multirow{3}{*}{X} &0 &Slips of Rs 1\\
&1 &Slips of Rs 5\\
&2 &Slips of Rs 13\\ \hline
\multirow{2}{*}{Y} &0 &Box A\\
&1 &Box B\\\hline
\end{tabular}
\caption{}
\label{tab:Distribution}
\end{table}
See \tabref{tab:Distribution}.
\begin{align}
p_{Y}\brak{k}= \begin{cases} 
      \frac{1}{3} & {k=0} \\
      \frac{2}{3 }& {k=1} 
   \end{cases}
   \\
p_{Y|X}\brak{0|0} = \frac{19}{25}\, 
p_{Y|X}\brak{0|1} = \frac{6}{25}\,
p_{Y|X}\brak{1|0} = \frac{45}{50}\,
p_{Y|X}\brak{1|2} = \frac{5}{50}
\end{align}
The desired probability is the probability that a slip drawn at random is marked other than Rs 1,
\begin{align}
&=1-p_X\brak{0}\\
&= p_X(1) + p_X(2)
\end{align}
Using Bayes theorem,
\begin{align}
&= p_Y\brak{0} \times \pr{Y=0 | X=1} + p_Y\brak{1} \times \pr{Y=1|X=2}\\
&=\frac{1}{3} \times \frac{6}{25} + \frac{2}{3} \times \frac{5}{50}\\
&=\frac{11}{75}
\end{align}

\newpage

%\tableofcontents

\bigskip

\renewcommand{\thefigure}{\theenumi}
\renewcommand{\thetable}{\theenumi}
%\renewcommand{\theequation}{\theenumi}

%\begin{abstract}
%%\boldmath
%In this letter, an algorithm for evaluating the exact analytical bit error rate  (BER)  for the piecewise linear (PL) combiner for  multiple relays is presented. Previous results were available only for upto three relays. The algorithm is unique in the sense that  the actual mathematical expressions, that are prohibitively large, need not be explicitly obtained. The diversity gain due to multiple relays is shown through plots of the analytical BER, well supported by simulations. 
%
%\end{abstract}
% IEEEtran.cls defaults to using nonbold math in the Abstract.
% This preserves the distinction between vectors and scalars. However,
% if the journal you are submitting to favors bold math in the abstract,
% then you can use LaTeX's standard command \boldmath at the very start
% of the abstract to achieve this. Many IEEE journals frown on math
% in the abstract anyway.

% Note that keywords are not normally used for peerreview papers.
%\begin{IEEEkeywords}
%Cooperative diversity, decode and forward, piecewise linear
%\end{IEEEkeywords}



% For peer review papers, you can put extra information on the cover
% page as needed:
% \ifCLASSOPTIONpeerreview
% \begin{center} \bfseries EDICS Category: 3-BBND \end{center}
% \fi
%
% For peerreview papers, this IEEEtran command inserts a page break and
% creates the second title. It will be ignored for other modes.
%\IEEEpeerreviewmaketitle




\item A card is selected from a pack of 52 cards.
\begin{enumerate}[label=(\alph*)]
    \item How many points are there in the sample space?
    \item Calculate the probability that the card is an ace of spades.
    \item Calculate the probability that the card is (i) an ace and (ii) black card.
\end{enumerate}
\solution
%Let $X$ be an bernoulli rv defined as in \tabref{tab:exemplar/11/16/3/26}.  Then, 
\begin{equation}
    p =
        \frac{4}{11} 
\end{equation}
\begin{table}[H]
	\centering
	\input{exemplar/11/16/3/26/tables/Table2.tex}
	\caption{}
        \label{tab:exemplar/11/16/3/26}
\end{table}

\item The probability that a non leap year selected at random will contain 53 sundays.
\\
\solution
%\begin{table}[H]
	\centering
\begin{tabular}{|c|c|c|}
\hline
Random variable &Value &Definition\\ \hline
\multirow{3}{*}{X} &0 &Slips of Rs 1\\
&1 &Slips of Rs 5\\
&2 &Slips of Rs 13\\ \hline
\multirow{2}{*}{Y} &0 &Box A\\
&1 &Box B\\\hline
\end{tabular}
\caption{}
\label{tab:Distribution}
\end{table}
See \tabref{tab:Distribution}.
\begin{align}
p_{Y}\brak{k}= \begin{cases} 
      \frac{1}{3} & {k=0} \\
      \frac{2}{3 }& {k=1} 
   \end{cases}
   \\
p_{Y|X}\brak{0|0} = \frac{19}{25}\, 
p_{Y|X}\brak{0|1} = \frac{6}{25}\,
p_{Y|X}\brak{1|0} = \frac{45}{50}\,
p_{Y|X}\brak{1|2} = \frac{5}{50}
\end{align}
The desired probability is the probability that a slip drawn at random is marked other than Rs 1,
\begin{align}
&=1-p_X\brak{0}\\
&= p_X(1) + p_X(2)
\end{align}
Using Bayes theorem,
\begin{align}
&= p_Y\brak{0} \times \pr{Y=0 | X=1} + p_Y\brak{1} \times \pr{Y=1|X=2}\\
&=\frac{1}{3} \times \frac{6}{25} + \frac{2}{3} \times \frac{5}{50}\\
&=\frac{11}{75}
\end{align}

\newpage

%\tableofcontents

\bigskip

\renewcommand{\thefigure}{\theenumi}
\renewcommand{\thetable}{\theenumi}
%\renewcommand{\theequation}{\theenumi}

%\begin{abstract}
%%\boldmath
%In this letter, an algorithm for evaluating the exact analytical bit error rate  (BER)  for the piecewise linear (PL) combiner for  multiple relays is presented. Previous results were available only for upto three relays. The algorithm is unique in the sense that  the actual mathematical expressions, that are prohibitively large, need not be explicitly obtained. The diversity gain due to multiple relays is shown through plots of the analytical BER, well supported by simulations. 
%
%\end{abstract}
% IEEEtran.cls defaults to using nonbold math in the Abstract.
% This preserves the distinction between vectors and scalars. However,
% if the journal you are submitting to favors bold math in the abstract,
% then you can use LaTeX's standard command \boldmath at the very start
% of the abstract to achieve this. Many IEEE journals frown on math
% in the abstract anyway.

% Note that keywords are not normally used for peerreview papers.
%\begin{IEEEkeywords}
%Cooperative diversity, decode and forward, piecewise linear
%\end{IEEEkeywords}



% For peer review papers, you can put extra information on the cover
% page as needed:
% \ifCLASSOPTIONpeerreview
% \begin{center} \bfseries EDICS Category: 3-BBND \end{center}
% \fi
%
% For peerreview papers, this IEEEtran command inserts a page break and
% creates the second title. It will be ignored for other modes.
%\IEEEpeerreviewmaketitle




\item One of the four persons John, Rita, Aslam or Gurpreet will be promoted next
month. Consequently the sample space consists of four elementary outcomes
S = {John promoted, Rita promoted, Aslam promoted, Gurpreet promoted}
You are told that the chances of John’s promotion is same as that of Gurpreet,
Rita’s chances of promotion are twice as likely as Johns. Aslam’s chances are
four times that of John.
\begin{enumerate}
	\item Determine
	\begin{enumerate}
		\item P (John promoted)
		\item P (Rita promoted)
		\item P (Aslam promoted)
		\item P (Gurpreet promoted)
	\end{enumerate}
	\item If A = {John promoted or Gurpreet promoted}, find P (A).
\end{enumerate}
\solution
%\begin{table}[H]
	\centering
\begin{tabular}{|c|c|c|}
\hline
Random variable &Value &Definition\\ \hline
\multirow{3}{*}{X} &0 &Slips of Rs 1\\
&1 &Slips of Rs 5\\
&2 &Slips of Rs 13\\ \hline
\multirow{2}{*}{Y} &0 &Box A\\
&1 &Box B\\\hline
\end{tabular}
\caption{}
\label{tab:Distribution}
\end{table}
See \tabref{tab:Distribution}.
\begin{align}
p_{Y}\brak{k}= \begin{cases} 
      \frac{1}{3} & {k=0} \\
      \frac{2}{3 }& {k=1} 
   \end{cases}
   \\
p_{Y|X}\brak{0|0} = \frac{19}{25}\, 
p_{Y|X}\brak{0|1} = \frac{6}{25}\,
p_{Y|X}\brak{1|0} = \frac{45}{50}\,
p_{Y|X}\brak{1|2} = \frac{5}{50}
\end{align}
The desired probability is the probability that a slip drawn at random is marked other than Rs 1,
\begin{align}
&=1-p_X\brak{0}\\
&= p_X(1) + p_X(2)
\end{align}
Using Bayes theorem,
\begin{align}
&= p_Y\brak{0} \times \pr{Y=0 | X=1} + p_Y\brak{1} \times \pr{Y=1|X=2}\\
&=\frac{1}{3} \times \frac{6}{25} + \frac{2}{3} \times \frac{5}{50}\\
&=\frac{11}{75}
\end{align}

\newpage

%\tableofcontents

\bigskip

\renewcommand{\thefigure}{\theenumi}
\renewcommand{\thetable}{\theenumi}
%\renewcommand{\theequation}{\theenumi}

%\begin{abstract}
%%\boldmath
%In this letter, an algorithm for evaluating the exact analytical bit error rate  (BER)  for the piecewise linear (PL) combiner for  multiple relays is presented. Previous results were available only for upto three relays. The algorithm is unique in the sense that  the actual mathematical expressions, that are prohibitively large, need not be explicitly obtained. The diversity gain due to multiple relays is shown through plots of the analytical BER, well supported by simulations. 
%
%\end{abstract}
% IEEEtran.cls defaults to using nonbold math in the Abstract.
% This preserves the distinction between vectors and scalars. However,
% if the journal you are submitting to favors bold math in the abstract,
% then you can use LaTeX's standard command \boldmath at the very start
% of the abstract to achieve this. Many IEEE journals frown on math
% in the abstract anyway.

% Note that keywords are not normally used for peerreview papers.
%\begin{IEEEkeywords}
%Cooperative diversity, decode and forward, piecewise linear
%\end{IEEEkeywords}



% For peer review papers, you can put extra information on the cover
% page as needed:
% \ifCLASSOPTIONpeerreview
% \begin{center} \bfseries EDICS Category: 3-BBND \end{center}
% \fi
%
% For peerreview papers, this IEEEtran command inserts a page break and
% creates the second title. It will be ignored for other modes.
%\IEEEpeerreviewmaketitle




\item A card is drawn from a deck of 52 cards. Find the probability of getting a king or a heart or a red card.\\
\solution
%\begin{table}[H]
	\centering
\begin{tabular}{|c|c|c|}
\hline
Random variable &Value &Definition\\ \hline
\multirow{3}{*}{X} &0 &Slips of Rs 1\\
&1 &Slips of Rs 5\\
&2 &Slips of Rs 13\\ \hline
\multirow{2}{*}{Y} &0 &Box A\\
&1 &Box B\\\hline
\end{tabular}
\caption{}
\label{tab:Distribution}
\end{table}
See \tabref{tab:Distribution}.
\begin{align}
p_{Y}\brak{k}= \begin{cases} 
      \frac{1}{3} & {k=0} \\
      \frac{2}{3 }& {k=1} 
   \end{cases}
   \\
p_{Y|X}\brak{0|0} = \frac{19}{25}\, 
p_{Y|X}\brak{0|1} = \frac{6}{25}\,
p_{Y|X}\brak{1|0} = \frac{45}{50}\,
p_{Y|X}\brak{1|2} = \frac{5}{50}
\end{align}
The desired probability is the probability that a slip drawn at random is marked other than Rs 1,
\begin{align}
&=1-p_X\brak{0}\\
&= p_X(1) + p_X(2)
\end{align}
Using Bayes theorem,
\begin{align}
&= p_Y\brak{0} \times \pr{Y=0 | X=1} + p_Y\brak{1} \times \pr{Y=1|X=2}\\
&=\frac{1}{3} \times \frac{6}{25} + \frac{2}{3} \times \frac{5}{50}\\
&=\frac{11}{75}
\end{align}

\newpage

%\tableofcontents

\bigskip

\renewcommand{\thefigure}{\theenumi}
\renewcommand{\thetable}{\theenumi}
%\renewcommand{\theequation}{\theenumi}

%\begin{abstract}
%%\boldmath
%In this letter, an algorithm for evaluating the exact analytical bit error rate  (BER)  for the piecewise linear (PL) combiner for  multiple relays is presented. Previous results were available only for upto three relays. The algorithm is unique in the sense that  the actual mathematical expressions, that are prohibitively large, need not be explicitly obtained. The diversity gain due to multiple relays is shown through plots of the analytical BER, well supported by simulations. 
%
%\end{abstract}
% IEEEtran.cls defaults to using nonbold math in the Abstract.
% This preserves the distinction between vectors and scalars. However,
% if the journal you are submitting to favors bold math in the abstract,
% then you can use LaTeX's standard command \boldmath at the very start
% of the abstract to achieve this. Many IEEE journals frown on math
% in the abstract anyway.

% Note that keywords are not normally used for peerreview papers.
%\begin{IEEEkeywords}
%Cooperative diversity, decode and forward, piecewise linear
%\end{IEEEkeywords}



% For peer review papers, you can put extra information on the cover
% page as needed:
% \ifCLASSOPTIONpeerreview
% \begin{center} \bfseries EDICS Category: 3-BBND \end{center}
% \fi
%
% For peerreview papers, this IEEEtran command inserts a page break and
% creates the second title. It will be ignored for other modes.
%\IEEEpeerreviewmaketitle




\item The probability that a student will pass his examination is 0.73, the probability of
the student getting a compartment is 0.13, and the probability that the student will
either pass or get compartment is 0.96. State True or False.\\
\solution
%\begin{table}[H]
	\centering
\begin{tabular}{|c|c|c|}
\hline
Random variable &Value &Definition\\ \hline
\multirow{3}{*}{X} &0 &Slips of Rs 1\\
&1 &Slips of Rs 5\\
&2 &Slips of Rs 13\\ \hline
\multirow{2}{*}{Y} &0 &Box A\\
&1 &Box B\\\hline
\end{tabular}
\caption{}
\label{tab:Distribution}
\end{table}
See \tabref{tab:Distribution}.
\begin{align}
p_{Y}\brak{k}= \begin{cases} 
      \frac{1}{3} & {k=0} \\
      \frac{2}{3 }& {k=1} 
   \end{cases}
   \\
p_{Y|X}\brak{0|0} = \frac{19}{25}\, 
p_{Y|X}\brak{0|1} = \frac{6}{25}\,
p_{Y|X}\brak{1|0} = \frac{45}{50}\,
p_{Y|X}\brak{1|2} = \frac{5}{50}
\end{align}
The desired probability is the probability that a slip drawn at random is marked other than Rs 1,
\begin{align}
&=1-p_X\brak{0}\\
&= p_X(1) + p_X(2)
\end{align}
Using Bayes theorem,
\begin{align}
&= p_Y\brak{0} \times \pr{Y=0 | X=1} + p_Y\brak{1} \times \pr{Y=1|X=2}\\
&=\frac{1}{3} \times \frac{6}{25} + \frac{2}{3} \times \frac{5}{50}\\
&=\frac{11}{75}
\end{align}

\newpage

%\tableofcontents

\bigskip

\renewcommand{\thefigure}{\theenumi}
\renewcommand{\thetable}{\theenumi}
%\renewcommand{\theequation}{\theenumi}

%\begin{abstract}
%%\boldmath
%In this letter, an algorithm for evaluating the exact analytical bit error rate  (BER)  for the piecewise linear (PL) combiner for  multiple relays is presented. Previous results were available only for upto three relays. The algorithm is unique in the sense that  the actual mathematical expressions, that are prohibitively large, need not be explicitly obtained. The diversity gain due to multiple relays is shown through plots of the analytical BER, well supported by simulations. 
%
%\end{abstract}
% IEEEtran.cls defaults to using nonbold math in the Abstract.
% This preserves the distinction between vectors and scalars. However,
% if the journal you are submitting to favors bold math in the abstract,
% then you can use LaTeX's standard command \boldmath at the very start
% of the abstract to achieve this. Many IEEE journals frown on math
% in the abstract anyway.

% Note that keywords are not normally used for peerreview papers.
%\begin{IEEEkeywords}
%Cooperative diversity, decode and forward, piecewise linear
%\end{IEEEkeywords}



% For peer review papers, you can put extra information on the cover
% page as needed:
% \ifCLASSOPTIONpeerreview
% \begin{center} \bfseries EDICS Category: 3-BBND \end{center}
% \fi
%
% For peerreview papers, this IEEEtran command inserts a page break and
% creates the second title. It will be ignored for other modes.
%\IEEEpeerreviewmaketitle




\item A card is selected from a pack of 52 cards\\
\begin{enumerate}[label=(\alph*)]
\item How many points are there in the sample space?
\item Calculate the probability that the cards is an ace of spades.
\item Calculate the probability that the card is (i) an ace (ii)black card.\\
\end{enumerate}
%\input{ncert/11/16/3/4_1/Prob_4.tex}
\item In a non-leap year, the probability of having 53 tuesdays or 53 wednesdays is\\
\solution
%A non-leap year has a total of 365 days, and a week has 7 days.\\
So it can be expressed as 
\begin{align}
365\text{days} &=52\times 7+1 \text{day}
\end{align}
$\implies$ 52 tuesdays or wednesdays\\
Random variable X denotes the days of a week
\begin{align}
p_X\brak{k}&=\frac{1}{7}; \quad \brak{1<k<7}
\end{align}
So the probability of extra day being tuesday or wednesday is
\begin{align}
p_X\brak{3}+p_X\brak{4}&=\frac{1}{7}+\frac{1}{7}=\frac{2}{7}
\end{align}



\item There are 1000 sealed envelopes in a box, 10 of them contain a cash prize of
Rs 100 each, 100 of them contain a cash prize of Rs 50 each and 200 of them
contain a cash prize of Rs 10 each and rest do not contain any cash prize. If they
are well shuffled and an envelope is picked up out, what is the probability that it
contains no cash prize?\\
\solution
%\begin{table}[H]
	\centering
\begin{tabular}{|c|c|c|}
\hline
Random variable &Value &Definition\\ \hline
\multirow{3}{*}{X} &0 &Slips of Rs 1\\
&1 &Slips of Rs 5\\
&2 &Slips of Rs 13\\ \hline
\multirow{2}{*}{Y} &0 &Box A\\
&1 &Box B\\\hline
\end{tabular}
\caption{}
\label{tab:Distribution}
\end{table}
See \tabref{tab:Distribution}.
\begin{align}
p_{Y}\brak{k}= \begin{cases} 
      \frac{1}{3} & {k=0} \\
      \frac{2}{3 }& {k=1} 
   \end{cases}
   \\
p_{Y|X}\brak{0|0} = \frac{19}{25}\, 
p_{Y|X}\brak{0|1} = \frac{6}{25}\,
p_{Y|X}\brak{1|0} = \frac{45}{50}\,
p_{Y|X}\brak{1|2} = \frac{5}{50}
\end{align}
The desired probability is the probability that a slip drawn at random is marked other than Rs 1,
\begin{align}
&=1-p_X\brak{0}\\
&= p_X(1) + p_X(2)
\end{align}
Using Bayes theorem,
\begin{align}
&= p_Y\brak{0} \times \pr{Y=0 | X=1} + p_Y\brak{1} \times \pr{Y=1|X=2}\\
&=\frac{1}{3} \times \frac{6}{25} + \frac{2}{3} \times \frac{5}{50}\\
&=\frac{11}{75}
\end{align}

\newpage

%\tableofcontents

\bigskip

\renewcommand{\thefigure}{\theenumi}
\renewcommand{\thetable}{\theenumi}
%\renewcommand{\theequation}{\theenumi}

%\begin{abstract}
%%\boldmath
%In this letter, an algorithm for evaluating the exact analytical bit error rate  (BER)  for the piecewise linear (PL) combiner for  multiple relays is presented. Previous results were available only for upto three relays. The algorithm is unique in the sense that  the actual mathematical expressions, that are prohibitively large, need not be explicitly obtained. The diversity gain due to multiple relays is shown through plots of the analytical BER, well supported by simulations. 
%
%\end{abstract}
% IEEEtran.cls defaults to using nonbold math in the Abstract.
% This preserves the distinction between vectors and scalars. However,
% if the journal you are submitting to favors bold math in the abstract,
% then you can use LaTeX's standard command \boldmath at the very start
% of the abstract to achieve this. Many IEEE journals frown on math
% in the abstract anyway.

% Note that keywords are not normally used for peerreview papers.
%\begin{IEEEkeywords}
%Cooperative diversity, decode and forward, piecewise linear
%\end{IEEEkeywords}



% For peer review papers, you can put extra information on the cover
% page as needed:
% \ifCLASSOPTIONpeerreview
% \begin{center} \bfseries EDICS Category: 3-BBND \end{center}
% \fi
%
% For peerreview papers, this IEEEtran command inserts a page break and
% creates the second title. It will be ignored for other modes.
%\IEEEpeerreviewmaketitle




\item 
A die is thrown and a card is selected at random from a deck of 52 playing cards. The probability of getting an even number on the die and a spade card.\\
\solution
%\begin{table}[H]
	\centering
\begin{tabular}{|c|c|c|}
\hline
Random variable &Value &Definition\\ \hline
\multirow{3}{*}{X} &0 &Slips of Rs 1\\
&1 &Slips of Rs 5\\
&2 &Slips of Rs 13\\ \hline
\multirow{2}{*}{Y} &0 &Box A\\
&1 &Box B\\\hline
\end{tabular}
\caption{}
\label{tab:Distribution}
\end{table}
See \tabref{tab:Distribution}.
\begin{align}
p_{Y}\brak{k}= \begin{cases} 
      \frac{1}{3} & {k=0} \\
      \frac{2}{3 }& {k=1} 
   \end{cases}
   \\
p_{Y|X}\brak{0|0} = \frac{19}{25}\, 
p_{Y|X}\brak{0|1} = \frac{6}{25}\,
p_{Y|X}\brak{1|0} = \frac{45}{50}\,
p_{Y|X}\brak{1|2} = \frac{5}{50}
\end{align}
The desired probability is the probability that a slip drawn at random is marked other than Rs 1,
\begin{align}
&=1-p_X\brak{0}\\
&= p_X(1) + p_X(2)
\end{align}
Using Bayes theorem,
\begin{align}
&= p_Y\brak{0} \times \pr{Y=0 | X=1} + p_Y\brak{1} \times \pr{Y=1|X=2}\\
&=\frac{1}{3} \times \frac{6}{25} + \frac{2}{3} \times \frac{5}{50}\\
&=\frac{11}{75}
\end{align}

\newpage

%\tableofcontents

\bigskip

\renewcommand{\thefigure}{\theenumi}
\renewcommand{\thetable}{\theenumi}
%\renewcommand{\theequation}{\theenumi}

%\begin{abstract}
%%\boldmath
%In this letter, an algorithm for evaluating the exact analytical bit error rate  (BER)  for the piecewise linear (PL) combiner for  multiple relays is presented. Previous results were available only for upto three relays. The algorithm is unique in the sense that  the actual mathematical expressions, that are prohibitively large, need not be explicitly obtained. The diversity gain due to multiple relays is shown through plots of the analytical BER, well supported by simulations. 
%
%\end{abstract}
% IEEEtran.cls defaults to using nonbold math in the Abstract.
% This preserves the distinction between vectors and scalars. However,
% if the journal you are submitting to favors bold math in the abstract,
% then you can use LaTeX's standard command \boldmath at the very start
% of the abstract to achieve this. Many IEEE journals frown on math
% in the abstract anyway.

% Note that keywords are not normally used for peerreview papers.
%\begin{IEEEkeywords}
%Cooperative diversity, decode and forward, piecewise linear
%\end{IEEEkeywords}



% For peer review papers, you can put extra information on the cover
% page as needed:
% \ifCLASSOPTIONpeerreview
% \begin{center} \bfseries EDICS Category: 3-BBND \end{center}
% \fi
%
% For peerreview papers, this IEEEtran command inserts a page break and
% creates the second title. It will be ignored for other modes.
%\IEEEpeerreviewmaketitle




\item
If 4-digit numbers greater than 5,000 are randomly formed from the digits 0, 1, 3, 5, and 7, what is the probability of forming a number divisible by 5 when:
\begin{enumerate}
    \item The digits are repeated?
    \item The repetition of digits is not allowed?
\end{enumerate}
\solution
%\begin{table}[H]
	\centering
\begin{tabular}{|c|c|c|}
\hline
Random variable &Value &Definition\\ \hline
\multirow{3}{*}{X} &0 &Slips of Rs 1\\
&1 &Slips of Rs 5\\
&2 &Slips of Rs 13\\ \hline
\multirow{2}{*}{Y} &0 &Box A\\
&1 &Box B\\\hline
\end{tabular}
\caption{}
\label{tab:Distribution}
\end{table}
See \tabref{tab:Distribution}.
\begin{align}
p_{Y}\brak{k}= \begin{cases} 
      \frac{1}{3} & {k=0} \\
      \frac{2}{3 }& {k=1} 
   \end{cases}
   \\
p_{Y|X}\brak{0|0} = \frac{19}{25}\, 
p_{Y|X}\brak{0|1} = \frac{6}{25}\,
p_{Y|X}\brak{1|0} = \frac{45}{50}\,
p_{Y|X}\brak{1|2} = \frac{5}{50}
\end{align}
The desired probability is the probability that a slip drawn at random is marked other than Rs 1,
\begin{align}
&=1-p_X\brak{0}\\
&= p_X(1) + p_X(2)
\end{align}
Using Bayes theorem,
\begin{align}
&= p_Y\brak{0} \times \pr{Y=0 | X=1} + p_Y\brak{1} \times \pr{Y=1|X=2}\\
&=\frac{1}{3} \times \frac{6}{25} + \frac{2}{3} \times \frac{5}{50}\\
&=\frac{11}{75}
\end{align}

\newpage

%\tableofcontents

\bigskip

\renewcommand{\thefigure}{\theenumi}
\renewcommand{\thetable}{\theenumi}
%\renewcommand{\theequation}{\theenumi}

%\begin{abstract}
%%\boldmath
%In this letter, an algorithm for evaluating the exact analytical bit error rate  (BER)  for the piecewise linear (PL) combiner for  multiple relays is presented. Previous results were available only for upto three relays. The algorithm is unique in the sense that  the actual mathematical expressions, that are prohibitively large, need not be explicitly obtained. The diversity gain due to multiple relays is shown through plots of the analytical BER, well supported by simulations. 
%
%\end{abstract}
% IEEEtran.cls defaults to using nonbold math in the Abstract.
% This preserves the distinction between vectors and scalars. However,
% if the journal you are submitting to favors bold math in the abstract,
% then you can use LaTeX's standard command \boldmath at the very start
% of the abstract to achieve this. Many IEEE journals frown on math
% in the abstract anyway.

% Note that keywords are not normally used for peerreview papers.
%\begin{IEEEkeywords}
%Cooperative diversity, decode and forward, piecewise linear
%\end{IEEEkeywords}



% For peer review papers, you can put extra information on the cover
% page as needed:
% \ifCLASSOPTIONpeerreview
% \begin{center} \bfseries EDICS Category: 3-BBND \end{center}
% \fi
%
% For peerreview papers, this IEEEtran command inserts a page break and
% creates the second title. It will be ignored for other modes.
%\IEEEpeerreviewmaketitle




\item Consider the probability space $\brak{\Omega, \mathcal{G}, P}$ where $\Omega = [0,2]$ and $\mathcal{G} = \cbrak{\phi, \Omega, [0,1], (1,2]}$. Let $X$ and $Y$ be two functions on $\Omega$ defined as
\begin{align*}
    X(\omega) = 
    \begin{cases}
        1 & \text{if }\omega \in [0, 1]\\
        2 & \text{if }\omega \in (1, 2]
    \end{cases}
\end{align*}
and
\begin{align*}
    Y(\omega) = 
    \begin{cases}
        2 & \text{if }\omega \in [0, 1.5]\\
        3 & \text{if }\omega \in (1.5, 2].
    \end{cases}
\end{align*}
Then which one of the following statements is true?
\begin{enumerate}
    \item [(A)] $X$ is a random variable with respect to $\mathcal{G}$, but $Y$ is not a random variable with respect to $\mathcal{G}$.
    \item [(B)] $Y$ is a random variable with respect to $\mathcal{G}$, but $X$ is not a random variable with respect to $\mathcal{G}$.
    \item [(C)] Neither $X$ nor $Y$ is a random variable with respect to $\mathcal{G}$.
    \item [(D)] Both $X$ and $Y$ are random variables with respect to $\mathcal{G}$.
\end{enumerate} \hfill (GATE ST 2023)\\
\solution
%\begin{table}[H]
	\centering
\begin{tabular}{|c|c|c|}
\hline
Random variable &Value &Definition\\ \hline
\multirow{3}{*}{X} &0 &Slips of Rs 1\\
&1 &Slips of Rs 5\\
&2 &Slips of Rs 13\\ \hline
\multirow{2}{*}{Y} &0 &Box A\\
&1 &Box B\\\hline
\end{tabular}
\caption{}
\label{tab:Distribution}
\end{table}
See \tabref{tab:Distribution}.
\begin{align}
p_{Y}\brak{k}= \begin{cases} 
      \frac{1}{3} & {k=0} \\
      \frac{2}{3 }& {k=1} 
   \end{cases}
   \\
p_{Y|X}\brak{0|0} = \frac{19}{25}\, 
p_{Y|X}\brak{0|1} = \frac{6}{25}\,
p_{Y|X}\brak{1|0} = \frac{45}{50}\,
p_{Y|X}\brak{1|2} = \frac{5}{50}
\end{align}
The desired probability is the probability that a slip drawn at random is marked other than Rs 1,
\begin{align}
&=1-p_X\brak{0}\\
&= p_X(1) + p_X(2)
\end{align}
Using Bayes theorem,
\begin{align}
&= p_Y\brak{0} \times \pr{Y=0 | X=1} + p_Y\brak{1} \times \pr{Y=1|X=2}\\
&=\frac{1}{3} \times \frac{6}{25} + \frac{2}{3} \times \frac{5}{50}\\
&=\frac{11}{75}
\end{align}

\newpage

%\tableofcontents

\bigskip

\renewcommand{\thefigure}{\theenumi}
\renewcommand{\thetable}{\theenumi}
%\renewcommand{\theequation}{\theenumi}

%\begin{abstract}
%%\boldmath
%In this letter, an algorithm for evaluating the exact analytical bit error rate  (BER)  for the piecewise linear (PL) combiner for  multiple relays is presented. Previous results were available only for upto three relays. The algorithm is unique in the sense that  the actual mathematical expressions, that are prohibitively large, need not be explicitly obtained. The diversity gain due to multiple relays is shown through plots of the analytical BER, well supported by simulations. 
%
%\end{abstract}
% IEEEtran.cls defaults to using nonbold math in the Abstract.
% This preserves the distinction between vectors and scalars. However,
% if the journal you are submitting to favors bold math in the abstract,
% then you can use LaTeX's standard command \boldmath at the very start
% of the abstract to achieve this. Many IEEE journals frown on math
% in the abstract anyway.

% Note that keywords are not normally used for peerreview papers.
%\begin{IEEEkeywords}
%Cooperative diversity, decode and forward, piecewise linear
%\end{IEEEkeywords}



% For peer review papers, you can put extra information on the cover
% page as needed:
% \ifCLASSOPTIONpeerreview
% \begin{center} \bfseries EDICS Category: 3-BBND \end{center}
% \fi
%
% For peerreview papers, this IEEEtran command inserts a page break and
% creates the second title. It will be ignored for other modes.
%\IEEEpeerreviewmaketitle




	\item  A die is loaded in such a way that each odd number is twice as likely to occur as
each even number. Find $P(G)$, where $G$ is the event that a number greater than
3 occurs on a single roll of the die.
\\
\solution
		%\begin{table}[H]
	\centering
\begin{tabular}{|c|c|c|}
\hline
Random variable &Value &Definition\\ \hline
\multirow{3}{*}{X} &0 &Slips of Rs 1\\
&1 &Slips of Rs 5\\
&2 &Slips of Rs 13\\ \hline
\multirow{2}{*}{Y} &0 &Box A\\
&1 &Box B\\\hline
\end{tabular}
\caption{}
\label{tab:Distribution}
\end{table}
See \tabref{tab:Distribution}.
\begin{align}
p_{Y}\brak{k}= \begin{cases} 
      \frac{1}{3} & {k=0} \\
      \frac{2}{3 }& {k=1} 
   \end{cases}
   \\
p_{Y|X}\brak{0|0} = \frac{19}{25}\, 
p_{Y|X}\brak{0|1} = \frac{6}{25}\,
p_{Y|X}\brak{1|0} = \frac{45}{50}\,
p_{Y|X}\brak{1|2} = \frac{5}{50}
\end{align}
The desired probability is the probability that a slip drawn at random is marked other than Rs 1,
\begin{align}
&=1-p_X\brak{0}\\
&= p_X(1) + p_X(2)
\end{align}
Using Bayes theorem,
\begin{align}
&= p_Y\brak{0} \times \pr{Y=0 | X=1} + p_Y\brak{1} \times \pr{Y=1|X=2}\\
&=\frac{1}{3} \times \frac{6}{25} + \frac{2}{3} \times \frac{5}{50}\\
&=\frac{11}{75}
\end{align}

\newpage

%\tableofcontents

\bigskip

\renewcommand{\thefigure}{\theenumi}
\renewcommand{\thetable}{\theenumi}
%\renewcommand{\theequation}{\theenumi}

%\begin{abstract}
%%\boldmath
%In this letter, an algorithm for evaluating the exact analytical bit error rate  (BER)  for the piecewise linear (PL) combiner for  multiple relays is presented. Previous results were available only for upto three relays. The algorithm is unique in the sense that  the actual mathematical expressions, that are prohibitively large, need not be explicitly obtained. The diversity gain due to multiple relays is shown through plots of the analytical BER, well supported by simulations. 
%
%\end{abstract}
% IEEEtran.cls defaults to using nonbold math in the Abstract.
% This preserves the distinction between vectors and scalars. However,
% if the journal you are submitting to favors bold math in the abstract,
% then you can use LaTeX's standard command \boldmath at the very start
% of the abstract to achieve this. Many IEEE journals frown on math
% in the abstract anyway.

% Note that keywords are not normally used for peerreview papers.
%\begin{IEEEkeywords}
%Cooperative diversity, decode and forward, piecewise linear
%\end{IEEEkeywords}



% For peer review papers, you can put extra information on the cover
% page as needed:
% \ifCLASSOPTIONpeerreview
% \begin{center} \bfseries EDICS Category: 3-BBND \end{center}
% \fi
%
% For peerreview papers, this IEEEtran command inserts a page break and
% creates the second title. It will be ignored for other modes.
%\IEEEpeerreviewmaketitle




	\item All the jacks, queens and kings are removed from a deck of 52 playing cards. The remaining cards are well shuffled and then one card is drawn at random. Giving ace a value 1 similar value for other cards, find the probability that the card has a value 
		\begin{enumerate}
			\item 7
			\item greater than 7
			\item less than 7
		\end{enumerate}
		%Number of cards left after removing all jacks, queens and kings 
\begin{align}
N	= 52 - 4\times 3
	= 40
\end{align}
%\begin{table}[H]
%\def\arraystretch{1.2}
%\begin{tabular}{|c|c|c|}
%\hline
%	\textbf{Parameter} &\textbf{Value} &\textbf{Description}\\ \hline
%	$X$ &1-10 &Represents the value of the card picked \\ \hline
%\end{tabular}
%\end{table}
Let $1 \le X \le 10$ be the value of the card picked.  Then,
\begin{align}
	p_X(k) &= \Pr(X=k)\ \forall\ 1 \leq k \leq 10\\
	&= \frac{4\times 1}{40}\\
	&= \frac{1}{10}\\
	\therefore p_X(k) &= 
	\begin{cases}
		\frac{1}{10} & 1 \leq k \leq 10\\
		0 & \text{otherwise}
	\end{cases}
\end{align}
and
\begin{align}
	F_{X}(k) &= \sum_{m=0}^{k}p_{X}(m) \quad 1 \leq k \leq 10\\
	&= \frac{k}{10}\\
	\therefore F_{X}(k) &= 
	\begin{cases}
		0 & k \leq 0\\
		\frac{k}{10} & 1\leq k \leq 10\\
		1 & k > 10 
	\end{cases}
\end{align}
\begin{enumerate}
	\item Probability that card has value equal to 7 is
		\begin{align}
			 p_{X}(7)
			= \frac{1}{10}
		\end{align}
	\item Probability that card has value greater than 7 is
		\begin{align}
			1 - F_X(7)
			&= 1 - \frac{7}{10}
			\\
			&= \frac{3}{10}
		\end{align}
	\item Probability that card has value less than 7 is
		\begin{align}
			 F_{X}(6)
			=\frac{6}{10}
		\end{align}
\end{enumerate}

  \item A Lot consists of 48 mobile phones of which 42 are good, 3 have only minor defects and 3 have major defects.Varnika will buy a phone if it is good but the trader will only buy a mobile if it has no major defects. One phone is selected at random from the lot. What is the probability that it is
\begin{enumerate}
	\item acceptable to Varnika?
            \item acceptable to the trader?
\end{enumerate}
\solution
	%\begin{table}[H]
	\centering
\begin{tabular}{|c|c|c|}
\hline
Random variable &Value &Definition\\ \hline
\multirow{3}{*}{X} &0 &Slips of Rs 1\\
&1 &Slips of Rs 5\\
&2 &Slips of Rs 13\\ \hline
\multirow{2}{*}{Y} &0 &Box A\\
&1 &Box B\\\hline
\end{tabular}
\caption{}
\label{tab:Distribution}
\end{table}
See \tabref{tab:Distribution}.
\begin{align}
p_{Y}\brak{k}= \begin{cases} 
      \frac{1}{3} & {k=0} \\
      \frac{2}{3 }& {k=1} 
   \end{cases}
   \\
p_{Y|X}\brak{0|0} = \frac{19}{25}\, 
p_{Y|X}\brak{0|1} = \frac{6}{25}\,
p_{Y|X}\brak{1|0} = \frac{45}{50}\,
p_{Y|X}\brak{1|2} = \frac{5}{50}
\end{align}
The desired probability is the probability that a slip drawn at random is marked other than Rs 1,
\begin{align}
&=1-p_X\brak{0}\\
&= p_X(1) + p_X(2)
\end{align}
Using Bayes theorem,
\begin{align}
&= p_Y\brak{0} \times \pr{Y=0 | X=1} + p_Y\brak{1} \times \pr{Y=1|X=2}\\
&=\frac{1}{3} \times \frac{6}{25} + \frac{2}{3} \times \frac{5}{50}\\
&=\frac{11}{75}
\end{align}

\newpage

%\tableofcontents

\bigskip

\renewcommand{\thefigure}{\theenumi}
\renewcommand{\thetable}{\theenumi}
%\renewcommand{\theequation}{\theenumi}

%\begin{abstract}
%%\boldmath
%In this letter, an algorithm for evaluating the exact analytical bit error rate  (BER)  for the piecewise linear (PL) combiner for  multiple relays is presented. Previous results were available only for upto three relays. The algorithm is unique in the sense that  the actual mathematical expressions, that are prohibitively large, need not be explicitly obtained. The diversity gain due to multiple relays is shown through plots of the analytical BER, well supported by simulations. 
%
%\end{abstract}
% IEEEtran.cls defaults to using nonbold math in the Abstract.
% This preserves the distinction between vectors and scalars. However,
% if the journal you are submitting to favors bold math in the abstract,
% then you can use LaTeX's standard command \boldmath at the very start
% of the abstract to achieve this. Many IEEE journals frown on math
% in the abstract anyway.

% Note that keywords are not normally used for peerreview papers.
%\begin{IEEEkeywords}
%Cooperative diversity, decode and forward, piecewise linear
%\end{IEEEkeywords}



% For peer review papers, you can put extra information on the cover
% page as needed:
% \ifCLASSOPTIONpeerreview
% \begin{center} \bfseries EDICS Category: 3-BBND \end{center}
% \fi
%
% For peerreview papers, this IEEEtran command inserts a page break and
% creates the second title. It will be ignored for other modes.
%\IEEEpeerreviewmaketitle




 \item A student says that if you throw a die, it will show up 1 or not 1. Therefore, the probability of getting 1 and the probability of getting 'not 1' each is equal to $\frac{1}{2}$. Is this correct? Give reasons.\\
 \solution
        %\begin{table}[H]
	\centering
\begin{tabular}{|c|c|c|}
\hline
Random variable &Value &Definition\\ \hline
\multirow{3}{*}{X} &0 &Slips of Rs 1\\
&1 &Slips of Rs 5\\
&2 &Slips of Rs 13\\ \hline
\multirow{2}{*}{Y} &0 &Box A\\
&1 &Box B\\\hline
\end{tabular}
\caption{}
\label{tab:Distribution}
\end{table}
See \tabref{tab:Distribution}.
\begin{align}
p_{Y}\brak{k}= \begin{cases} 
      \frac{1}{3} & {k=0} \\
      \frac{2}{3 }& {k=1} 
   \end{cases}
   \\
p_{Y|X}\brak{0|0} = \frac{19}{25}\, 
p_{Y|X}\brak{0|1} = \frac{6}{25}\,
p_{Y|X}\brak{1|0} = \frac{45}{50}\,
p_{Y|X}\brak{1|2} = \frac{5}{50}
\end{align}
The desired probability is the probability that a slip drawn at random is marked other than Rs 1,
\begin{align}
&=1-p_X\brak{0}\\
&= p_X(1) + p_X(2)
\end{align}
Using Bayes theorem,
\begin{align}
&= p_Y\brak{0} \times \pr{Y=0 | X=1} + p_Y\brak{1} \times \pr{Y=1|X=2}\\
&=\frac{1}{3} \times \frac{6}{25} + \frac{2}{3} \times \frac{5}{50}\\
&=\frac{11}{75}
\end{align}

\newpage

%\tableofcontents

\bigskip

\renewcommand{\thefigure}{\theenumi}
\renewcommand{\thetable}{\theenumi}
%\renewcommand{\theequation}{\theenumi}

%\begin{abstract}
%%\boldmath
%In this letter, an algorithm for evaluating the exact analytical bit error rate  (BER)  for the piecewise linear (PL) combiner for  multiple relays is presented. Previous results were available only for upto three relays. The algorithm is unique in the sense that  the actual mathematical expressions, that are prohibitively large, need not be explicitly obtained. The diversity gain due to multiple relays is shown through plots of the analytical BER, well supported by simulations. 
%
%\end{abstract}
% IEEEtran.cls defaults to using nonbold math in the Abstract.
% This preserves the distinction between vectors and scalars. However,
% if the journal you are submitting to favors bold math in the abstract,
% then you can use LaTeX's standard command \boldmath at the very start
% of the abstract to achieve this. Many IEEE journals frown on math
% in the abstract anyway.

% Note that keywords are not normally used for peerreview papers.
%\begin{IEEEkeywords}
%Cooperative diversity, decode and forward, piecewise linear
%\end{IEEEkeywords}



% For peer review papers, you can put extra information on the cover
% page as needed:
% \ifCLASSOPTIONpeerreview
% \begin{center} \bfseries EDICS Category: 3-BBND \end{center}
% \fi
%
% For peerreview papers, this IEEEtran command inserts a page break and
% creates the second title. It will be ignored for other modes.
%\IEEEpeerreviewmaketitle




   \item Four candidates A, B, C, D have ap-
plied for the assignment to coach a school cricket
team. If A is twice as likely to be selected as B, and
B and C are given about the same chance of being
selected, while C is twice as likely to be selected
as D, what are the probabilities that
\begin{enumerate}
\item C will be selected?
\item A will not be selected?
\end{enumerate}
	%\begin{table}[H]
	\centering
\begin{tabular}{|c|c|c|}
\hline
Random variable &Value &Definition\\ \hline
\multirow{3}{*}{X} &0 &Slips of Rs 1\\
&1 &Slips of Rs 5\\
&2 &Slips of Rs 13\\ \hline
\multirow{2}{*}{Y} &0 &Box A\\
&1 &Box B\\\hline
\end{tabular}
\caption{}
\label{tab:Distribution}
\end{table}
See \tabref{tab:Distribution}.
\begin{align}
p_{Y}\brak{k}= \begin{cases} 
      \frac{1}{3} & {k=0} \\
      \frac{2}{3 }& {k=1} 
   \end{cases}
   \\
p_{Y|X}\brak{0|0} = \frac{19}{25}\, 
p_{Y|X}\brak{0|1} = \frac{6}{25}\,
p_{Y|X}\brak{1|0} = \frac{45}{50}\,
p_{Y|X}\brak{1|2} = \frac{5}{50}
\end{align}
The desired probability is the probability that a slip drawn at random is marked other than Rs 1,
\begin{align}
&=1-p_X\brak{0}\\
&= p_X(1) + p_X(2)
\end{align}
Using Bayes theorem,
\begin{align}
&= p_Y\brak{0} \times \pr{Y=0 | X=1} + p_Y\brak{1} \times \pr{Y=1|X=2}\\
&=\frac{1}{3} \times \frac{6}{25} + \frac{2}{3} \times \frac{5}{50}\\
&=\frac{11}{75}
\end{align}

\newpage

%\tableofcontents

\bigskip

\renewcommand{\thefigure}{\theenumi}
\renewcommand{\thetable}{\theenumi}
%\renewcommand{\theequation}{\theenumi}

%\begin{abstract}
%%\boldmath
%In this letter, an algorithm for evaluating the exact analytical bit error rate  (BER)  for the piecewise linear (PL) combiner for  multiple relays is presented. Previous results were available only for upto three relays. The algorithm is unique in the sense that  the actual mathematical expressions, that are prohibitively large, need not be explicitly obtained. The diversity gain due to multiple relays is shown through plots of the analytical BER, well supported by simulations. 
%
%\end{abstract}
% IEEEtran.cls defaults to using nonbold math in the Abstract.
% This preserves the distinction between vectors and scalars. However,
% if the journal you are submitting to favors bold math in the abstract,
% then you can use LaTeX's standard command \boldmath at the very start
% of the abstract to achieve this. Many IEEE journals frown on math
% in the abstract anyway.

% Note that keywords are not normally used for peerreview papers.
%\begin{IEEEkeywords}
%Cooperative diversity, decode and forward, piecewise linear
%\end{IEEEkeywords}



% For peer review papers, you can put extra information on the cover
% page as needed:
% \ifCLASSOPTIONpeerreview
% \begin{center} \bfseries EDICS Category: 3-BBND \end{center}
% \fi
%
% For peerreview papers, this IEEEtran command inserts a page break and
% creates the second title. It will be ignored for other modes.
%\IEEEpeerreviewmaketitle




 \item A bag contain 24 balls of which $x$ balls are red, $2x$ are white and $3x$ are blue. A ball is selected at random, What is the probability that it is
\begin{enumerate}[label=\alph*)]
\item not red ?
\item white ?
\end{enumerate}
%\begin{table}[H]
	\centering
\begin{tabular}{|c|c|c|}
\hline
Random variable &Value &Definition\\ \hline
\multirow{3}{*}{X} &0 &Slips of Rs 1\\
&1 &Slips of Rs 5\\
&2 &Slips of Rs 13\\ \hline
\multirow{2}{*}{Y} &0 &Box A\\
&1 &Box B\\\hline
\end{tabular}
\caption{}
\label{tab:Distribution}
\end{table}
See \tabref{tab:Distribution}.
\begin{align}
p_{Y}\brak{k}= \begin{cases} 
      \frac{1}{3} & {k=0} \\
      \frac{2}{3 }& {k=1} 
   \end{cases}
   \\
p_{Y|X}\brak{0|0} = \frac{19}{25}\, 
p_{Y|X}\brak{0|1} = \frac{6}{25}\,
p_{Y|X}\brak{1|0} = \frac{45}{50}\,
p_{Y|X}\brak{1|2} = \frac{5}{50}
\end{align}
The desired probability is the probability that a slip drawn at random is marked other than Rs 1,
\begin{align}
&=1-p_X\brak{0}\\
&= p_X(1) + p_X(2)
\end{align}
Using Bayes theorem,
\begin{align}
&= p_Y\brak{0} \times \pr{Y=0 | X=1} + p_Y\brak{1} \times \pr{Y=1|X=2}\\
&=\frac{1}{3} \times \frac{6}{25} + \frac{2}{3} \times \frac{5}{50}\\
&=\frac{11}{75}
\end{align}

\newpage

%\tableofcontents

\bigskip

\renewcommand{\thefigure}{\theenumi}
\renewcommand{\thetable}{\theenumi}
%\renewcommand{\theequation}{\theenumi}

%\begin{abstract}
%%\boldmath
%In this letter, an algorithm for evaluating the exact analytical bit error rate  (BER)  for the piecewise linear (PL) combiner for  multiple relays is presented. Previous results were available only for upto three relays. The algorithm is unique in the sense that  the actual mathematical expressions, that are prohibitively large, need not be explicitly obtained. The diversity gain due to multiple relays is shown through plots of the analytical BER, well supported by simulations. 
%
%\end{abstract}
% IEEEtran.cls defaults to using nonbold math in the Abstract.
% This preserves the distinction between vectors and scalars. However,
% if the journal you are submitting to favors bold math in the abstract,
% then you can use LaTeX's standard command \boldmath at the very start
% of the abstract to achieve this. Many IEEE journals frown on math
% in the abstract anyway.

% Note that keywords are not normally used for peerreview papers.
%\begin{IEEEkeywords}
%Cooperative diversity, decode and forward, piecewise linear
%\end{IEEEkeywords}



% For peer review papers, you can put extra information on the cover
% page as needed:
% \ifCLASSOPTIONpeerreview
% \begin{center} \bfseries EDICS Category: 3-BBND \end{center}
% \fi
%
% For peerreview papers, this IEEEtran command inserts a page break and
% creates the second title. It will be ignored for other modes.
%\IEEEpeerreviewmaketitle




If the letters of the word ASSASSINATION are arranged at random. Find the Probability that
\begin{enumerate}[label=(\alph*)]
\item Four $S's$ come consecutively in the word
\item Two  $I's$ and two $N's$ come together
\item All $A's$ are not coming together
\item No two $A's$ are coming together
\end{enumerate}
%\begin{table}[H]
	\centering
\begin{tabular}{|c|c|c|}
\hline
Random variable &Value &Definition\\ \hline
\multirow{3}{*}{X} &0 &Slips of Rs 1\\
&1 &Slips of Rs 5\\
&2 &Slips of Rs 13\\ \hline
\multirow{2}{*}{Y} &0 &Box A\\
&1 &Box B\\\hline
\end{tabular}
\caption{}
\label{tab:Distribution}
\end{table}
See \tabref{tab:Distribution}.
\begin{align}
p_{Y}\brak{k}= \begin{cases} 
      \frac{1}{3} & {k=0} \\
      \frac{2}{3 }& {k=1} 
   \end{cases}
   \\
p_{Y|X}\brak{0|0} = \frac{19}{25}\, 
p_{Y|X}\brak{0|1} = \frac{6}{25}\,
p_{Y|X}\brak{1|0} = \frac{45}{50}\,
p_{Y|X}\brak{1|2} = \frac{5}{50}
\end{align}
The desired probability is the probability that a slip drawn at random is marked other than Rs 1,
\begin{align}
&=1-p_X\brak{0}\\
&= p_X(1) + p_X(2)
\end{align}
Using Bayes theorem,
\begin{align}
&= p_Y\brak{0} \times \pr{Y=0 | X=1} + p_Y\brak{1} \times \pr{Y=1|X=2}\\
&=\frac{1}{3} \times \frac{6}{25} + \frac{2}{3} \times \frac{5}{50}\\
&=\frac{11}{75}
\end{align}

\newpage

%\tableofcontents

\bigskip

\renewcommand{\thefigure}{\theenumi}
\renewcommand{\thetable}{\theenumi}
%\renewcommand{\theequation}{\theenumi}

%\begin{abstract}
%%\boldmath
%In this letter, an algorithm for evaluating the exact analytical bit error rate  (BER)  for the piecewise linear (PL) combiner for  multiple relays is presented. Previous results were available only for upto three relays. The algorithm is unique in the sense that  the actual mathematical expressions, that are prohibitively large, need not be explicitly obtained. The diversity gain due to multiple relays is shown through plots of the analytical BER, well supported by simulations. 
%
%\end{abstract}
% IEEEtran.cls defaults to using nonbold math in the Abstract.
% This preserves the distinction between vectors and scalars. However,
% if the journal you are submitting to favors bold math in the abstract,
% then you can use LaTeX's standard command \boldmath at the very start
% of the abstract to achieve this. Many IEEE journals frown on math
% in the abstract anyway.

% Note that keywords are not normally used for peerreview papers.
%\begin{IEEEkeywords}
%Cooperative diversity, decode and forward, piecewise linear
%\end{IEEEkeywords}



% For peer review papers, you can put extra information on the cover
% page as needed:
% \ifCLASSOPTIONpeerreview
% \begin{center} \bfseries EDICS Category: 3-BBND \end{center}
% \fi
%
% For peerreview papers, this IEEEtran command inserts a page break and
% creates the second title. It will be ignored for other modes.
%\IEEEpeerreviewmaketitle




	\item One urn contains two black balls (labelled B1 and B2) and one white ball. A
	second urn contains one black ball and two white balls (labelled W1 and W2).
	Suppose the following experiment is performed. One of the two urns is chosen
	at random. Next a ball is randomly chosen from the urn. Then a second ball is
	chosen at random from the same urn without replacing the first ball.
	
	\begin{enumerate}
	\item What is the probability that two black balls are chosen?
	
	\item What is the probability that two balls of opposite colour are chosen?
	\end{enumerate}
	\solution
	%\begin{align}
    \label{eq:12.13.6.18.1}
	\because	\pr{A|B} &> \pr{A},\
\frac{\pr{AB}}{\pr{B}} > \pr{A}
\\
    \label{eq:12.13.6.18.2}
	\implies \pr{AB} &> \pr{A}\pr{B}
	\\
	\text{or, } \frac{\pr{AB}}{\pr{A}} &=\pr{B|A} > \pr{A}
\end{align}

\end{enumerate}

\item In a certain lottery 10,000 tickets are sold and ten equal prizes are awarded. What is the probability of not getting a prize if you buy (a) one ticket (b) two tickets (c) 10 tickets ?	
\\
\solution
		%\begin{enumerate}[label=\thesection.\arabic*,ref=\thesection.\theenumi]
	\item One card is drawn from a well-shuffled deck of 52 cards. Find the probability of getting
\begin{enumerate}
\item A king of red colour 
\item A face card 
\item A red face card
\item The jack of hearts
\item A spade
\item The queen of diamonds

\end{enumerate}
\solution
		%\begin{table}[H]
	\centering
\begin{tabular}{|c|c|c|}
\hline
Random variable &Value &Definition\\ \hline
\multirow{3}{*}{X} &0 &Slips of Rs 1\\
&1 &Slips of Rs 5\\
&2 &Slips of Rs 13\\ \hline
\multirow{2}{*}{Y} &0 &Box A\\
&1 &Box B\\\hline
\end{tabular}
\caption{}
\label{tab:Distribution}
\end{table}
See \tabref{tab:Distribution}.
\begin{align}
p_{Y}\brak{k}= \begin{cases} 
      \frac{1}{3} & {k=0} \\
      \frac{2}{3 }& {k=1} 
   \end{cases}
   \\
p_{Y|X}\brak{0|0} = \frac{19}{25}\, 
p_{Y|X}\brak{0|1} = \frac{6}{25}\,
p_{Y|X}\brak{1|0} = \frac{45}{50}\,
p_{Y|X}\brak{1|2} = \frac{5}{50}
\end{align}
The desired probability is the probability that a slip drawn at random is marked other than Rs 1,
\begin{align}
&=1-p_X\brak{0}\\
&= p_X(1) + p_X(2)
\end{align}
Using Bayes theorem,
\begin{align}
&= p_Y\brak{0} \times \pr{Y=0 | X=1} + p_Y\brak{1} \times \pr{Y=1|X=2}\\
&=\frac{1}{3} \times \frac{6}{25} + \frac{2}{3} \times \frac{5}{50}\\
&=\frac{11}{75}
\end{align}

\newpage

%\tableofcontents

\bigskip

\renewcommand{\thefigure}{\theenumi}
\renewcommand{\thetable}{\theenumi}
%\renewcommand{\theequation}{\theenumi}

%\begin{abstract}
%%\boldmath
%In this letter, an algorithm for evaluating the exact analytical bit error rate  (BER)  for the piecewise linear (PL) combiner for  multiple relays is presented. Previous results were available only for upto three relays. The algorithm is unique in the sense that  the actual mathematical expressions, that are prohibitively large, need not be explicitly obtained. The diversity gain due to multiple relays is shown through plots of the analytical BER, well supported by simulations. 
%
%\end{abstract}
% IEEEtran.cls defaults to using nonbold math in the Abstract.
% This preserves the distinction between vectors and scalars. However,
% if the journal you are submitting to favors bold math in the abstract,
% then you can use LaTeX's standard command \boldmath at the very start
% of the abstract to achieve this. Many IEEE journals frown on math
% in the abstract anyway.

% Note that keywords are not normally used for peerreview papers.
%\begin{IEEEkeywords}
%Cooperative diversity, decode and forward, piecewise linear
%\end{IEEEkeywords}



% For peer review papers, you can put extra information on the cover
% page as needed:
% \ifCLASSOPTIONpeerreview
% \begin{center} \bfseries EDICS Category: 3-BBND \end{center}
% \fi
%
% For peerreview papers, this IEEEtran command inserts a page break and
% creates the second title. It will be ignored for other modes.
%\IEEEpeerreviewmaketitle




	\item Five cards—the ten, jack, queen, king and ace of diamonds, are well-shuffled with their face downwards. One card is then picked up at random.
\begin{enumerate}
\item
What is the probability that the card is the queen? 
\item
If the queen is drawn and put aside, what is the probability that the second card picked up is (a) an ace? (b) a queen?\\
\end{enumerate}
\solution
		%\begin{enumerate}[label=\thesection.\arabic*,ref=\thesection.\theenumi]
	\item One card is drawn from a well-shuffled deck of 52 cards. Find the probability of getting
\begin{enumerate}
\item A king of red colour 
\item A face card 
\item A red face card
\item The jack of hearts
\item A spade
\item The queen of diamonds

\end{enumerate}
\solution
		%\input{ncert/10/15/1/14/main.tex}
	\item Five cards—the ten, jack, queen, king and ace of diamonds, are well-shuffled with their face downwards. One card is then picked up at random.
\begin{enumerate}
\item
What is the probability that the card is the queen? 
\item
If the queen is drawn and put aside, what is the probability that the second card picked up is (a) an ace? (b) a queen?\\
\end{enumerate}
\solution
		%\input{ncert/10/15/1/15/defs.tex}
	\item A bag contains $5$ red balls and some blue balls. If the probability of drawing a blue ball is double that if a red ball, determine the number of blue balls in the bag. 
		\\
\solution
		%\input{ncert/10/15/2/3/defs.tex}
	\item A card is selected from a pack of 52 cards.
 \begin{enumerate}[label=(\alph*)] 
                 \item How many points are there in the sample space?
                 \item Calculate the probability that the card is an ace of spades.
                 \item Calculate the probability that the card is (i) an ace and (ii) black card.
 \end{enumerate}
\solution
		%\input{ncert/11/16/3/4/main.tex}
\item Four cards are drawn from a well-shuffled deck of 52 cards. What is the probability of obtaining 3 diamonds and one spade.
\\
\solution
		%\input{ncert/11/16/4/2/defs.tex}
\item In a certain lottery 10,000 tickets are sold and ten equal prizes are awarded. What is the probability of not getting a prize if you buy (a) one ticket (b) two tickets (c) 10 tickets ?	
\\
\solution
		%\input{ncert/11/16/4/4/defs.tex}
		%
\item 
Out of 100 students, two sections of 40 and 60 are formed. If you and your friend are among the 100 students, what is the probability that
\begin{enumerate}
\item you both enter the same section?
\item you both enter the different sections?
\end{enumerate}
\solution
		%\input{ncert/11/16/4/5/defs.tex}
	\item 
The number lock of a suitcase has 4 wheels each labelled with ten digits i.e. from 0 to 9.The lock opens with a sequence of four digits with no repeats.What is the probability of a person getting the right sequence to open the suitcase.
\\
\solution
		%\input{ncert/11/16/4/10/defs.tex}
		%
\item 
Two cards are drawn at random and without replacement from a pack of 52 playing cards. Find the probability that both the cards are black.
\\
\solution
		%\input{ncert/12/13/2/2/defs.tex}
		\item A box of oranges is inspected by examining three randomly selected oranges drawn without replacement. If all the three oranges are good, the box is approved for sale, otherwise, it is rejected. Find the probability that a box containing 15 oranges out of which 12 are good and 3 are bad ones will be approved for sale.
		\label{ncert/12/13/2/3/defs.tex}
		\item Two balls are drawn at random with replacement from a box containing 10 black and 8 red balls. Find the probability that
		\label{ncert/12/13/2/12}
\begin{enumerate}
\item both balls are red.
\item first ball is black and second is red.
\item one of them is black and other is red.
\end{enumerate}

\item In a hostel, 60\% of the students read Hindi newspaper, 40\% read English newspaper and 20\% read both Hindi and English newspapers. A student is selected at random.
		\label{ncert/12/13/2/15}
\begin{enumerate}
\item Find the probability that she reads neither Hindi nor English newspapers.
\item If she reads Hindi newspaper, find the probability that she reads English newspaper.
\item If she reads English newspaper, find the probability that she reads Hindi newspaper.\\
\end{enumerate}
\item The probability of obtaining an even prime number on each die, when a pair of dice is rolled is 
\begin{enumerate}
    \item $0$ 
    
    \item $\frac{1}{3}$ 
    
    \item $\frac{1}{12}$ 
    
    \item $\frac{1}{36}$ 
\end{enumerate}
\solution
		%\input{ncert/12/13/2/17/defs.tex}
	\item A bag contains 4 red and 4 black balls, another bag contains 2 red and 6 black balls. One of the two bags is selected at random and a ball is drawn from the bag which is found to be red. Find the probability that the ball is drawn from the first bag.
\\
\solution
		%\input{ncert/12/13/3/2/main.tex}
  \item
  Cards with numbers 2 to 101 are placed in a box. A card is selected at random.Find the probability that the card has
\begin{enumerate}[label=(\roman*)]
	\item an even number 
	\item a square number
\end{enumerate}
\solution
%\input{exemplar/10/13/3/32/main.tex}
\item
The king, queen and jack of clubs are removed from a deck of 52 playing cards and then well shuffled. Now one card is drawn at random from the remaining cards.  Determine the probability that the card is
\begin{enumerate}[label=(\roman*)]
\item a club
\item 10 of hearts
\end{enumerate}
\solution
%\input{exemplar/10/13/3/29/main.tex}
\item A team of medical students doing their internship have to assist during surgeries
at a city hospital. The probabilities of surgeries rated as very complex, complex,
routine, simple or very simple are respectively, 0.15, 0.20, 0.31, 0.26, .08. Find
the probabilities that a particular surgery will be rated
\begin{enumerate}
	\item complex or very complex;
	\item neither very complex nor very simple;
	\item routine or complex
	\item routine or simple
\end{enumerate}
\solution
%\input{exemplar/11/16/3/8(1)/main.tex}
\item A card is selected from a pack of 52 cards.
\begin{enumerate}[label=(\alph*)]
    \item How many points are there in the sample space?
    \item Calculate the probability that the card is an ace of spades.
    \item Calculate the probability that the card is (i) an ace and (ii) black card.
\end{enumerate}
\solution
%\input{exemplar/11/16/3/4/main2.tex}
\item The probability that a non leap year selected at random will contain 53 sundays.
\\
\solution
%\input{exemplar/10/13/1/19/main.tex}
\item One of the four persons John, Rita, Aslam or Gurpreet will be promoted next
month. Consequently the sample space consists of four elementary outcomes
S = {John promoted, Rita promoted, Aslam promoted, Gurpreet promoted}
You are told that the chances of John’s promotion is same as that of Gurpreet,
Rita’s chances of promotion are twice as likely as Johns. Aslam’s chances are
four times that of John.
\begin{enumerate}
	\item Determine
	\begin{enumerate}
		\item P (John promoted)
		\item P (Rita promoted)
		\item P (Aslam promoted)
		\item P (Gurpreet promoted)
	\end{enumerate}
	\item If A = {John promoted or Gurpreet promoted}, find P (A).
\end{enumerate}
\solution
%\input{exemplar/11/16/3/10/main.tex}
\item A card is drawn from a deck of 52 cards. Find the probability of getting a king or a heart or a red card.\\
\solution
%\input{exemplar/11/16/3/15/main.tex}
\item The probability that a student will pass his examination is 0.73, the probability of
the student getting a compartment is 0.13, and the probability that the student will
either pass or get compartment is 0.96. State True or False.\\
\solution
%\input{exemplar/11/16/3/31/main.tex}
\item A card is selected from a pack of 52 cards\\
\begin{enumerate}[label=(\alph*)]
\item How many points are there in the sample space?
\item Calculate the probability that the cards is an ace of spades.
\item Calculate the probability that the card is (i) an ace (ii)black card.\\
\end{enumerate}
%\input{ncert/11/16/3/4_1/Prob_4.tex}
\item In a non-leap year, the probability of having 53 tuesdays or 53 wednesdays is\\
\solution
%\input{exemplar/11/16/3/18/main.tex}
\item There are 1000 sealed envelopes in a box, 10 of them contain a cash prize of
Rs 100 each, 100 of them contain a cash prize of Rs 50 each and 200 of them
contain a cash prize of Rs 10 each and rest do not contain any cash prize. If they
are well shuffled and an envelope is picked up out, what is the probability that it
contains no cash prize?\\
\solution
%\input{exemplar/10/13/3/34/main.tex}
\item 
A die is thrown and a card is selected at random from a deck of 52 playing cards. The probability of getting an even number on the die and a spade card.\\
\solution
%\input{exemplar/12/13/3/78/main.tex}
\item
If 4-digit numbers greater than 5,000 are randomly formed from the digits 0, 1, 3, 5, and 7, what is the probability of forming a number divisible by 5 when:
\begin{enumerate}
    \item The digits are repeated?
    \item The repetition of digits is not allowed?
\end{enumerate}
\solution
%\input{ncert/11/16/4/9/main.tex}
\item Consider the probability space $\brak{\Omega, \mathcal{G}, P}$ where $\Omega = [0,2]$ and $\mathcal{G} = \cbrak{\phi, \Omega, [0,1], (1,2]}$. Let $X$ and $Y$ be two functions on $\Omega$ defined as
\begin{align*}
    X(\omega) = 
    \begin{cases}
        1 & \text{if }\omega \in [0, 1]\\
        2 & \text{if }\omega \in (1, 2]
    \end{cases}
\end{align*}
and
\begin{align*}
    Y(\omega) = 
    \begin{cases}
        2 & \text{if }\omega \in [0, 1.5]\\
        3 & \text{if }\omega \in (1.5, 2].
    \end{cases}
\end{align*}
Then which one of the following statements is true?
\begin{enumerate}
    \item [(A)] $X$ is a random variable with respect to $\mathcal{G}$, but $Y$ is not a random variable with respect to $\mathcal{G}$.
    \item [(B)] $Y$ is a random variable with respect to $\mathcal{G}$, but $X$ is not a random variable with respect to $\mathcal{G}$.
    \item [(C)] Neither $X$ nor $Y$ is a random variable with respect to $\mathcal{G}$.
    \item [(D)] Both $X$ and $Y$ are random variables with respect to $\mathcal{G}$.
\end{enumerate} \hfill (GATE ST 2023)\\
\solution
%\input{gate/ST/2023/14/main.tex}
	\item  A die is loaded in such a way that each odd number is twice as likely to occur as
each even number. Find $P(G)$, where $G$ is the event that a number greater than
3 occurs on a single roll of the die.
\\
\solution
		%\input{exemplar/11/16/3/5/main.tex}
	\item All the jacks, queens and kings are removed from a deck of 52 playing cards. The remaining cards are well shuffled and then one card is drawn at random. Giving ace a value 1 similar value for other cards, find the probability that the card has a value 
		\begin{enumerate}
			\item 7
			\item greater than 7
			\item less than 7
		\end{enumerate}
		%\input{exemplar/10/13/3/30/main.tex}
  \item A Lot consists of 48 mobile phones of which 42 are good, 3 have only minor defects and 3 have major defects.Varnika will buy a phone if it is good but the trader will only buy a mobile if it has no major defects. One phone is selected at random from the lot. What is the probability that it is
\begin{enumerate}
	\item acceptable to Varnika?
            \item acceptable to the trader?
\end{enumerate}
\solution
	%\input{exemplar/10/13/3/40/main.tex}
 \item A student says that if you throw a die, it will show up 1 or not 1. Therefore, the probability of getting 1 and the probability of getting 'not 1' each is equal to $\frac{1}{2}$. Is this correct? Give reasons.\\
 \solution
        %\input{exemplar/10/13/2/9/main.tex}
   \item Four candidates A, B, C, D have ap-
plied for the assignment to coach a school cricket
team. If A is twice as likely to be selected as B, and
B and C are given about the same chance of being
selected, while C is twice as likely to be selected
as D, what are the probabilities that
\begin{enumerate}
\item C will be selected?
\item A will not be selected?
\end{enumerate}
	%\input{exemplar/11/16/3/9/main.tex}
 \item A bag contain 24 balls of which $x$ balls are red, $2x$ are white and $3x$ are blue. A ball is selected at random, What is the probability that it is
\begin{enumerate}[label=\alph*)]
\item not red ?
\item white ?
\end{enumerate}
%\input{exemplar/10/13/3/41/main.tex}
If the letters of the word ASSASSINATION are arranged at random. Find the Probability that
\begin{enumerate}[label=(\alph*)]
\item Four $S's$ come consecutively in the word
\item Two  $I's$ and two $N's$ come together
\item All $A's$ are not coming together
\item No two $A's$ are coming together
\end{enumerate}
%\input{exemplar/11/16/3/14/main.tex}
	\item One urn contains two black balls (labelled B1 and B2) and one white ball. A
	second urn contains one black ball and two white balls (labelled W1 and W2).
	Suppose the following experiment is performed. One of the two urns is chosen
	at random. Next a ball is randomly chosen from the urn. Then a second ball is
	chosen at random from the same urn without replacing the first ball.
	
	\begin{enumerate}
	\item What is the probability that two black balls are chosen?
	
	\item What is the probability that two balls of opposite colour are chosen?
	\end{enumerate}
	\solution
	%\input{exemplar/11/16/3/12/main1.tex}
\end{enumerate}

	\item A bag contains $5$ red balls and some blue balls. If the probability of drawing a blue ball is double that if a red ball, determine the number of blue balls in the bag. 
		\\
\solution
		%\begin{enumerate}[label=\thesection.\arabic*,ref=\thesection.\theenumi]
	\item One card is drawn from a well-shuffled deck of 52 cards. Find the probability of getting
\begin{enumerate}
\item A king of red colour 
\item A face card 
\item A red face card
\item The jack of hearts
\item A spade
\item The queen of diamonds

\end{enumerate}
\solution
		%\input{ncert/10/15/1/14/main.tex}
	\item Five cards—the ten, jack, queen, king and ace of diamonds, are well-shuffled with their face downwards. One card is then picked up at random.
\begin{enumerate}
\item
What is the probability that the card is the queen? 
\item
If the queen is drawn and put aside, what is the probability that the second card picked up is (a) an ace? (b) a queen?\\
\end{enumerate}
\solution
		%\input{ncert/10/15/1/15/defs.tex}
	\item A bag contains $5$ red balls and some blue balls. If the probability of drawing a blue ball is double that if a red ball, determine the number of blue balls in the bag. 
		\\
\solution
		%\input{ncert/10/15/2/3/defs.tex}
	\item A card is selected from a pack of 52 cards.
 \begin{enumerate}[label=(\alph*)] 
                 \item How many points are there in the sample space?
                 \item Calculate the probability that the card is an ace of spades.
                 \item Calculate the probability that the card is (i) an ace and (ii) black card.
 \end{enumerate}
\solution
		%\input{ncert/11/16/3/4/main.tex}
\item Four cards are drawn from a well-shuffled deck of 52 cards. What is the probability of obtaining 3 diamonds and one spade.
\\
\solution
		%\input{ncert/11/16/4/2/defs.tex}
\item In a certain lottery 10,000 tickets are sold and ten equal prizes are awarded. What is the probability of not getting a prize if you buy (a) one ticket (b) two tickets (c) 10 tickets ?	
\\
\solution
		%\input{ncert/11/16/4/4/defs.tex}
		%
\item 
Out of 100 students, two sections of 40 and 60 are formed. If you and your friend are among the 100 students, what is the probability that
\begin{enumerate}
\item you both enter the same section?
\item you both enter the different sections?
\end{enumerate}
\solution
		%\input{ncert/11/16/4/5/defs.tex}
	\item 
The number lock of a suitcase has 4 wheels each labelled with ten digits i.e. from 0 to 9.The lock opens with a sequence of four digits with no repeats.What is the probability of a person getting the right sequence to open the suitcase.
\\
\solution
		%\input{ncert/11/16/4/10/defs.tex}
		%
\item 
Two cards are drawn at random and without replacement from a pack of 52 playing cards. Find the probability that both the cards are black.
\\
\solution
		%\input{ncert/12/13/2/2/defs.tex}
		\item A box of oranges is inspected by examining three randomly selected oranges drawn without replacement. If all the three oranges are good, the box is approved for sale, otherwise, it is rejected. Find the probability that a box containing 15 oranges out of which 12 are good and 3 are bad ones will be approved for sale.
		\label{ncert/12/13/2/3/defs.tex}
		\item Two balls are drawn at random with replacement from a box containing 10 black and 8 red balls. Find the probability that
		\label{ncert/12/13/2/12}
\begin{enumerate}
\item both balls are red.
\item first ball is black and second is red.
\item one of them is black and other is red.
\end{enumerate}

\item In a hostel, 60\% of the students read Hindi newspaper, 40\% read English newspaper and 20\% read both Hindi and English newspapers. A student is selected at random.
		\label{ncert/12/13/2/15}
\begin{enumerate}
\item Find the probability that she reads neither Hindi nor English newspapers.
\item If she reads Hindi newspaper, find the probability that she reads English newspaper.
\item If she reads English newspaper, find the probability that she reads Hindi newspaper.\\
\end{enumerate}
\item The probability of obtaining an even prime number on each die, when a pair of dice is rolled is 
\begin{enumerate}
    \item $0$ 
    
    \item $\frac{1}{3}$ 
    
    \item $\frac{1}{12}$ 
    
    \item $\frac{1}{36}$ 
\end{enumerate}
\solution
		%\input{ncert/12/13/2/17/defs.tex}
	\item A bag contains 4 red and 4 black balls, another bag contains 2 red and 6 black balls. One of the two bags is selected at random and a ball is drawn from the bag which is found to be red. Find the probability that the ball is drawn from the first bag.
\\
\solution
		%\input{ncert/12/13/3/2/main.tex}
  \item
  Cards with numbers 2 to 101 are placed in a box. A card is selected at random.Find the probability that the card has
\begin{enumerate}[label=(\roman*)]
	\item an even number 
	\item a square number
\end{enumerate}
\solution
%\input{exemplar/10/13/3/32/main.tex}
\item
The king, queen and jack of clubs are removed from a deck of 52 playing cards and then well shuffled. Now one card is drawn at random from the remaining cards.  Determine the probability that the card is
\begin{enumerate}[label=(\roman*)]
\item a club
\item 10 of hearts
\end{enumerate}
\solution
%\input{exemplar/10/13/3/29/main.tex}
\item A team of medical students doing their internship have to assist during surgeries
at a city hospital. The probabilities of surgeries rated as very complex, complex,
routine, simple or very simple are respectively, 0.15, 0.20, 0.31, 0.26, .08. Find
the probabilities that a particular surgery will be rated
\begin{enumerate}
	\item complex or very complex;
	\item neither very complex nor very simple;
	\item routine or complex
	\item routine or simple
\end{enumerate}
\solution
%\input{exemplar/11/16/3/8(1)/main.tex}
\item A card is selected from a pack of 52 cards.
\begin{enumerate}[label=(\alph*)]
    \item How many points are there in the sample space?
    \item Calculate the probability that the card is an ace of spades.
    \item Calculate the probability that the card is (i) an ace and (ii) black card.
\end{enumerate}
\solution
%\input{exemplar/11/16/3/4/main2.tex}
\item The probability that a non leap year selected at random will contain 53 sundays.
\\
\solution
%\input{exemplar/10/13/1/19/main.tex}
\item One of the four persons John, Rita, Aslam or Gurpreet will be promoted next
month. Consequently the sample space consists of four elementary outcomes
S = {John promoted, Rita promoted, Aslam promoted, Gurpreet promoted}
You are told that the chances of John’s promotion is same as that of Gurpreet,
Rita’s chances of promotion are twice as likely as Johns. Aslam’s chances are
four times that of John.
\begin{enumerate}
	\item Determine
	\begin{enumerate}
		\item P (John promoted)
		\item P (Rita promoted)
		\item P (Aslam promoted)
		\item P (Gurpreet promoted)
	\end{enumerate}
	\item If A = {John promoted or Gurpreet promoted}, find P (A).
\end{enumerate}
\solution
%\input{exemplar/11/16/3/10/main.tex}
\item A card is drawn from a deck of 52 cards. Find the probability of getting a king or a heart or a red card.\\
\solution
%\input{exemplar/11/16/3/15/main.tex}
\item The probability that a student will pass his examination is 0.73, the probability of
the student getting a compartment is 0.13, and the probability that the student will
either pass or get compartment is 0.96. State True or False.\\
\solution
%\input{exemplar/11/16/3/31/main.tex}
\item A card is selected from a pack of 52 cards\\
\begin{enumerate}[label=(\alph*)]
\item How many points are there in the sample space?
\item Calculate the probability that the cards is an ace of spades.
\item Calculate the probability that the card is (i) an ace (ii)black card.\\
\end{enumerate}
%\input{ncert/11/16/3/4_1/Prob_4.tex}
\item In a non-leap year, the probability of having 53 tuesdays or 53 wednesdays is\\
\solution
%\input{exemplar/11/16/3/18/main.tex}
\item There are 1000 sealed envelopes in a box, 10 of them contain a cash prize of
Rs 100 each, 100 of them contain a cash prize of Rs 50 each and 200 of them
contain a cash prize of Rs 10 each and rest do not contain any cash prize. If they
are well shuffled and an envelope is picked up out, what is the probability that it
contains no cash prize?\\
\solution
%\input{exemplar/10/13/3/34/main.tex}
\item 
A die is thrown and a card is selected at random from a deck of 52 playing cards. The probability of getting an even number on the die and a spade card.\\
\solution
%\input{exemplar/12/13/3/78/main.tex}
\item
If 4-digit numbers greater than 5,000 are randomly formed from the digits 0, 1, 3, 5, and 7, what is the probability of forming a number divisible by 5 when:
\begin{enumerate}
    \item The digits are repeated?
    \item The repetition of digits is not allowed?
\end{enumerate}
\solution
%\input{ncert/11/16/4/9/main.tex}
\item Consider the probability space $\brak{\Omega, \mathcal{G}, P}$ where $\Omega = [0,2]$ and $\mathcal{G} = \cbrak{\phi, \Omega, [0,1], (1,2]}$. Let $X$ and $Y$ be two functions on $\Omega$ defined as
\begin{align*}
    X(\omega) = 
    \begin{cases}
        1 & \text{if }\omega \in [0, 1]\\
        2 & \text{if }\omega \in (1, 2]
    \end{cases}
\end{align*}
and
\begin{align*}
    Y(\omega) = 
    \begin{cases}
        2 & \text{if }\omega \in [0, 1.5]\\
        3 & \text{if }\omega \in (1.5, 2].
    \end{cases}
\end{align*}
Then which one of the following statements is true?
\begin{enumerate}
    \item [(A)] $X$ is a random variable with respect to $\mathcal{G}$, but $Y$ is not a random variable with respect to $\mathcal{G}$.
    \item [(B)] $Y$ is a random variable with respect to $\mathcal{G}$, but $X$ is not a random variable with respect to $\mathcal{G}$.
    \item [(C)] Neither $X$ nor $Y$ is a random variable with respect to $\mathcal{G}$.
    \item [(D)] Both $X$ and $Y$ are random variables with respect to $\mathcal{G}$.
\end{enumerate} \hfill (GATE ST 2023)\\
\solution
%\input{gate/ST/2023/14/main.tex}
	\item  A die is loaded in such a way that each odd number is twice as likely to occur as
each even number. Find $P(G)$, where $G$ is the event that a number greater than
3 occurs on a single roll of the die.
\\
\solution
		%\input{exemplar/11/16/3/5/main.tex}
	\item All the jacks, queens and kings are removed from a deck of 52 playing cards. The remaining cards are well shuffled and then one card is drawn at random. Giving ace a value 1 similar value for other cards, find the probability that the card has a value 
		\begin{enumerate}
			\item 7
			\item greater than 7
			\item less than 7
		\end{enumerate}
		%\input{exemplar/10/13/3/30/main.tex}
  \item A Lot consists of 48 mobile phones of which 42 are good, 3 have only minor defects and 3 have major defects.Varnika will buy a phone if it is good but the trader will only buy a mobile if it has no major defects. One phone is selected at random from the lot. What is the probability that it is
\begin{enumerate}
	\item acceptable to Varnika?
            \item acceptable to the trader?
\end{enumerate}
\solution
	%\input{exemplar/10/13/3/40/main.tex}
 \item A student says that if you throw a die, it will show up 1 or not 1. Therefore, the probability of getting 1 and the probability of getting 'not 1' each is equal to $\frac{1}{2}$. Is this correct? Give reasons.\\
 \solution
        %\input{exemplar/10/13/2/9/main.tex}
   \item Four candidates A, B, C, D have ap-
plied for the assignment to coach a school cricket
team. If A is twice as likely to be selected as B, and
B and C are given about the same chance of being
selected, while C is twice as likely to be selected
as D, what are the probabilities that
\begin{enumerate}
\item C will be selected?
\item A will not be selected?
\end{enumerate}
	%\input{exemplar/11/16/3/9/main.tex}
 \item A bag contain 24 balls of which $x$ balls are red, $2x$ are white and $3x$ are blue. A ball is selected at random, What is the probability that it is
\begin{enumerate}[label=\alph*)]
\item not red ?
\item white ?
\end{enumerate}
%\input{exemplar/10/13/3/41/main.tex}
If the letters of the word ASSASSINATION are arranged at random. Find the Probability that
\begin{enumerate}[label=(\alph*)]
\item Four $S's$ come consecutively in the word
\item Two  $I's$ and two $N's$ come together
\item All $A's$ are not coming together
\item No two $A's$ are coming together
\end{enumerate}
%\input{exemplar/11/16/3/14/main.tex}
	\item One urn contains two black balls (labelled B1 and B2) and one white ball. A
	second urn contains one black ball and two white balls (labelled W1 and W2).
	Suppose the following experiment is performed. One of the two urns is chosen
	at random. Next a ball is randomly chosen from the urn. Then a second ball is
	chosen at random from the same urn without replacing the first ball.
	
	\begin{enumerate}
	\item What is the probability that two black balls are chosen?
	
	\item What is the probability that two balls of opposite colour are chosen?
	\end{enumerate}
	\solution
	%\input{exemplar/11/16/3/12/main1.tex}
\end{enumerate}

	\item A card is selected from a pack of 52 cards.
 \begin{enumerate}[label=(\alph*)] 
                 \item How many points are there in the sample space?
                 \item Calculate the probability that the card is an ace of spades.
                 \item Calculate the probability that the card is (i) an ace and (ii) black card.
 \end{enumerate}
\solution
		%\begin{table}[H]
	\centering
\begin{tabular}{|c|c|c|}
\hline
Random variable &Value &Definition\\ \hline
\multirow{3}{*}{X} &0 &Slips of Rs 1\\
&1 &Slips of Rs 5\\
&2 &Slips of Rs 13\\ \hline
\multirow{2}{*}{Y} &0 &Box A\\
&1 &Box B\\\hline
\end{tabular}
\caption{}
\label{tab:Distribution}
\end{table}
See \tabref{tab:Distribution}.
\begin{align}
p_{Y}\brak{k}= \begin{cases} 
      \frac{1}{3} & {k=0} \\
      \frac{2}{3 }& {k=1} 
   \end{cases}
   \\
p_{Y|X}\brak{0|0} = \frac{19}{25}\, 
p_{Y|X}\brak{0|1} = \frac{6}{25}\,
p_{Y|X}\brak{1|0} = \frac{45}{50}\,
p_{Y|X}\brak{1|2} = \frac{5}{50}
\end{align}
The desired probability is the probability that a slip drawn at random is marked other than Rs 1,
\begin{align}
&=1-p_X\brak{0}\\
&= p_X(1) + p_X(2)
\end{align}
Using Bayes theorem,
\begin{align}
&= p_Y\brak{0} \times \pr{Y=0 | X=1} + p_Y\brak{1} \times \pr{Y=1|X=2}\\
&=\frac{1}{3} \times \frac{6}{25} + \frac{2}{3} \times \frac{5}{50}\\
&=\frac{11}{75}
\end{align}

\newpage

%\tableofcontents

\bigskip

\renewcommand{\thefigure}{\theenumi}
\renewcommand{\thetable}{\theenumi}
%\renewcommand{\theequation}{\theenumi}

%\begin{abstract}
%%\boldmath
%In this letter, an algorithm for evaluating the exact analytical bit error rate  (BER)  for the piecewise linear (PL) combiner for  multiple relays is presented. Previous results were available only for upto three relays. The algorithm is unique in the sense that  the actual mathematical expressions, that are prohibitively large, need not be explicitly obtained. The diversity gain due to multiple relays is shown through plots of the analytical BER, well supported by simulations. 
%
%\end{abstract}
% IEEEtran.cls defaults to using nonbold math in the Abstract.
% This preserves the distinction between vectors and scalars. However,
% if the journal you are submitting to favors bold math in the abstract,
% then you can use LaTeX's standard command \boldmath at the very start
% of the abstract to achieve this. Many IEEE journals frown on math
% in the abstract anyway.

% Note that keywords are not normally used for peerreview papers.
%\begin{IEEEkeywords}
%Cooperative diversity, decode and forward, piecewise linear
%\end{IEEEkeywords}



% For peer review papers, you can put extra information on the cover
% page as needed:
% \ifCLASSOPTIONpeerreview
% \begin{center} \bfseries EDICS Category: 3-BBND \end{center}
% \fi
%
% For peerreview papers, this IEEEtran command inserts a page break and
% creates the second title. It will be ignored for other modes.
%\IEEEpeerreviewmaketitle




\item Four cards are drawn from a well-shuffled deck of 52 cards. What is the probability of obtaining 3 diamonds and one spade.
\\
\solution
		%\begin{enumerate}[label=\thesection.\arabic*,ref=\thesection.\theenumi]
	\item One card is drawn from a well-shuffled deck of 52 cards. Find the probability of getting
\begin{enumerate}
\item A king of red colour 
\item A face card 
\item A red face card
\item The jack of hearts
\item A spade
\item The queen of diamonds

\end{enumerate}
\solution
		%\input{ncert/10/15/1/14/main.tex}
	\item Five cards—the ten, jack, queen, king and ace of diamonds, are well-shuffled with their face downwards. One card is then picked up at random.
\begin{enumerate}
\item
What is the probability that the card is the queen? 
\item
If the queen is drawn and put aside, what is the probability that the second card picked up is (a) an ace? (b) a queen?\\
\end{enumerate}
\solution
		%\input{ncert/10/15/1/15/defs.tex}
	\item A bag contains $5$ red balls and some blue balls. If the probability of drawing a blue ball is double that if a red ball, determine the number of blue balls in the bag. 
		\\
\solution
		%\input{ncert/10/15/2/3/defs.tex}
	\item A card is selected from a pack of 52 cards.
 \begin{enumerate}[label=(\alph*)] 
                 \item How many points are there in the sample space?
                 \item Calculate the probability that the card is an ace of spades.
                 \item Calculate the probability that the card is (i) an ace and (ii) black card.
 \end{enumerate}
\solution
		%\input{ncert/11/16/3/4/main.tex}
\item Four cards are drawn from a well-shuffled deck of 52 cards. What is the probability of obtaining 3 diamonds and one spade.
\\
\solution
		%\input{ncert/11/16/4/2/defs.tex}
\item In a certain lottery 10,000 tickets are sold and ten equal prizes are awarded. What is the probability of not getting a prize if you buy (a) one ticket (b) two tickets (c) 10 tickets ?	
\\
\solution
		%\input{ncert/11/16/4/4/defs.tex}
		%
\item 
Out of 100 students, two sections of 40 and 60 are formed. If you and your friend are among the 100 students, what is the probability that
\begin{enumerate}
\item you both enter the same section?
\item you both enter the different sections?
\end{enumerate}
\solution
		%\input{ncert/11/16/4/5/defs.tex}
	\item 
The number lock of a suitcase has 4 wheels each labelled with ten digits i.e. from 0 to 9.The lock opens with a sequence of four digits with no repeats.What is the probability of a person getting the right sequence to open the suitcase.
\\
\solution
		%\input{ncert/11/16/4/10/defs.tex}
		%
\item 
Two cards are drawn at random and without replacement from a pack of 52 playing cards. Find the probability that both the cards are black.
\\
\solution
		%\input{ncert/12/13/2/2/defs.tex}
		\item A box of oranges is inspected by examining three randomly selected oranges drawn without replacement. If all the three oranges are good, the box is approved for sale, otherwise, it is rejected. Find the probability that a box containing 15 oranges out of which 12 are good and 3 are bad ones will be approved for sale.
		\label{ncert/12/13/2/3/defs.tex}
		\item Two balls are drawn at random with replacement from a box containing 10 black and 8 red balls. Find the probability that
		\label{ncert/12/13/2/12}
\begin{enumerate}
\item both balls are red.
\item first ball is black and second is red.
\item one of them is black and other is red.
\end{enumerate}

\item In a hostel, 60\% of the students read Hindi newspaper, 40\% read English newspaper and 20\% read both Hindi and English newspapers. A student is selected at random.
		\label{ncert/12/13/2/15}
\begin{enumerate}
\item Find the probability that she reads neither Hindi nor English newspapers.
\item If she reads Hindi newspaper, find the probability that she reads English newspaper.
\item If she reads English newspaper, find the probability that she reads Hindi newspaper.\\
\end{enumerate}
\item The probability of obtaining an even prime number on each die, when a pair of dice is rolled is 
\begin{enumerate}
    \item $0$ 
    
    \item $\frac{1}{3}$ 
    
    \item $\frac{1}{12}$ 
    
    \item $\frac{1}{36}$ 
\end{enumerate}
\solution
		%\input{ncert/12/13/2/17/defs.tex}
	\item A bag contains 4 red and 4 black balls, another bag contains 2 red and 6 black balls. One of the two bags is selected at random and a ball is drawn from the bag which is found to be red. Find the probability that the ball is drawn from the first bag.
\\
\solution
		%\input{ncert/12/13/3/2/main.tex}
  \item
  Cards with numbers 2 to 101 are placed in a box. A card is selected at random.Find the probability that the card has
\begin{enumerate}[label=(\roman*)]
	\item an even number 
	\item a square number
\end{enumerate}
\solution
%\input{exemplar/10/13/3/32/main.tex}
\item
The king, queen and jack of clubs are removed from a deck of 52 playing cards and then well shuffled. Now one card is drawn at random from the remaining cards.  Determine the probability that the card is
\begin{enumerate}[label=(\roman*)]
\item a club
\item 10 of hearts
\end{enumerate}
\solution
%\input{exemplar/10/13/3/29/main.tex}
\item A team of medical students doing their internship have to assist during surgeries
at a city hospital. The probabilities of surgeries rated as very complex, complex,
routine, simple or very simple are respectively, 0.15, 0.20, 0.31, 0.26, .08. Find
the probabilities that a particular surgery will be rated
\begin{enumerate}
	\item complex or very complex;
	\item neither very complex nor very simple;
	\item routine or complex
	\item routine or simple
\end{enumerate}
\solution
%\input{exemplar/11/16/3/8(1)/main.tex}
\item A card is selected from a pack of 52 cards.
\begin{enumerate}[label=(\alph*)]
    \item How many points are there in the sample space?
    \item Calculate the probability that the card is an ace of spades.
    \item Calculate the probability that the card is (i) an ace and (ii) black card.
\end{enumerate}
\solution
%\input{exemplar/11/16/3/4/main2.tex}
\item The probability that a non leap year selected at random will contain 53 sundays.
\\
\solution
%\input{exemplar/10/13/1/19/main.tex}
\item One of the four persons John, Rita, Aslam or Gurpreet will be promoted next
month. Consequently the sample space consists of four elementary outcomes
S = {John promoted, Rita promoted, Aslam promoted, Gurpreet promoted}
You are told that the chances of John’s promotion is same as that of Gurpreet,
Rita’s chances of promotion are twice as likely as Johns. Aslam’s chances are
four times that of John.
\begin{enumerate}
	\item Determine
	\begin{enumerate}
		\item P (John promoted)
		\item P (Rita promoted)
		\item P (Aslam promoted)
		\item P (Gurpreet promoted)
	\end{enumerate}
	\item If A = {John promoted or Gurpreet promoted}, find P (A).
\end{enumerate}
\solution
%\input{exemplar/11/16/3/10/main.tex}
\item A card is drawn from a deck of 52 cards. Find the probability of getting a king or a heart or a red card.\\
\solution
%\input{exemplar/11/16/3/15/main.tex}
\item The probability that a student will pass his examination is 0.73, the probability of
the student getting a compartment is 0.13, and the probability that the student will
either pass or get compartment is 0.96. State True or False.\\
\solution
%\input{exemplar/11/16/3/31/main.tex}
\item A card is selected from a pack of 52 cards\\
\begin{enumerate}[label=(\alph*)]
\item How many points are there in the sample space?
\item Calculate the probability that the cards is an ace of spades.
\item Calculate the probability that the card is (i) an ace (ii)black card.\\
\end{enumerate}
%\input{ncert/11/16/3/4_1/Prob_4.tex}
\item In a non-leap year, the probability of having 53 tuesdays or 53 wednesdays is\\
\solution
%\input{exemplar/11/16/3/18/main.tex}
\item There are 1000 sealed envelopes in a box, 10 of them contain a cash prize of
Rs 100 each, 100 of them contain a cash prize of Rs 50 each and 200 of them
contain a cash prize of Rs 10 each and rest do not contain any cash prize. If they
are well shuffled and an envelope is picked up out, what is the probability that it
contains no cash prize?\\
\solution
%\input{exemplar/10/13/3/34/main.tex}
\item 
A die is thrown and a card is selected at random from a deck of 52 playing cards. The probability of getting an even number on the die and a spade card.\\
\solution
%\input{exemplar/12/13/3/78/main.tex}
\item
If 4-digit numbers greater than 5,000 are randomly formed from the digits 0, 1, 3, 5, and 7, what is the probability of forming a number divisible by 5 when:
\begin{enumerate}
    \item The digits are repeated?
    \item The repetition of digits is not allowed?
\end{enumerate}
\solution
%\input{ncert/11/16/4/9/main.tex}
\item Consider the probability space $\brak{\Omega, \mathcal{G}, P}$ where $\Omega = [0,2]$ and $\mathcal{G} = \cbrak{\phi, \Omega, [0,1], (1,2]}$. Let $X$ and $Y$ be two functions on $\Omega$ defined as
\begin{align*}
    X(\omega) = 
    \begin{cases}
        1 & \text{if }\omega \in [0, 1]\\
        2 & \text{if }\omega \in (1, 2]
    \end{cases}
\end{align*}
and
\begin{align*}
    Y(\omega) = 
    \begin{cases}
        2 & \text{if }\omega \in [0, 1.5]\\
        3 & \text{if }\omega \in (1.5, 2].
    \end{cases}
\end{align*}
Then which one of the following statements is true?
\begin{enumerate}
    \item [(A)] $X$ is a random variable with respect to $\mathcal{G}$, but $Y$ is not a random variable with respect to $\mathcal{G}$.
    \item [(B)] $Y$ is a random variable with respect to $\mathcal{G}$, but $X$ is not a random variable with respect to $\mathcal{G}$.
    \item [(C)] Neither $X$ nor $Y$ is a random variable with respect to $\mathcal{G}$.
    \item [(D)] Both $X$ and $Y$ are random variables with respect to $\mathcal{G}$.
\end{enumerate} \hfill (GATE ST 2023)\\
\solution
%\input{gate/ST/2023/14/main.tex}
	\item  A die is loaded in such a way that each odd number is twice as likely to occur as
each even number. Find $P(G)$, where $G$ is the event that a number greater than
3 occurs on a single roll of the die.
\\
\solution
		%\input{exemplar/11/16/3/5/main.tex}
	\item All the jacks, queens and kings are removed from a deck of 52 playing cards. The remaining cards are well shuffled and then one card is drawn at random. Giving ace a value 1 similar value for other cards, find the probability that the card has a value 
		\begin{enumerate}
			\item 7
			\item greater than 7
			\item less than 7
		\end{enumerate}
		%\input{exemplar/10/13/3/30/main.tex}
  \item A Lot consists of 48 mobile phones of which 42 are good, 3 have only minor defects and 3 have major defects.Varnika will buy a phone if it is good but the trader will only buy a mobile if it has no major defects. One phone is selected at random from the lot. What is the probability that it is
\begin{enumerate}
	\item acceptable to Varnika?
            \item acceptable to the trader?
\end{enumerate}
\solution
	%\input{exemplar/10/13/3/40/main.tex}
 \item A student says that if you throw a die, it will show up 1 or not 1. Therefore, the probability of getting 1 and the probability of getting 'not 1' each is equal to $\frac{1}{2}$. Is this correct? Give reasons.\\
 \solution
        %\input{exemplar/10/13/2/9/main.tex}
   \item Four candidates A, B, C, D have ap-
plied for the assignment to coach a school cricket
team. If A is twice as likely to be selected as B, and
B and C are given about the same chance of being
selected, while C is twice as likely to be selected
as D, what are the probabilities that
\begin{enumerate}
\item C will be selected?
\item A will not be selected?
\end{enumerate}
	%\input{exemplar/11/16/3/9/main.tex}
 \item A bag contain 24 balls of which $x$ balls are red, $2x$ are white and $3x$ are blue. A ball is selected at random, What is the probability that it is
\begin{enumerate}[label=\alph*)]
\item not red ?
\item white ?
\end{enumerate}
%\input{exemplar/10/13/3/41/main.tex}
If the letters of the word ASSASSINATION are arranged at random. Find the Probability that
\begin{enumerate}[label=(\alph*)]
\item Four $S's$ come consecutively in the word
\item Two  $I's$ and two $N's$ come together
\item All $A's$ are not coming together
\item No two $A's$ are coming together
\end{enumerate}
%\input{exemplar/11/16/3/14/main.tex}
	\item One urn contains two black balls (labelled B1 and B2) and one white ball. A
	second urn contains one black ball and two white balls (labelled W1 and W2).
	Suppose the following experiment is performed. One of the two urns is chosen
	at random. Next a ball is randomly chosen from the urn. Then a second ball is
	chosen at random from the same urn without replacing the first ball.
	
	\begin{enumerate}
	\item What is the probability that two black balls are chosen?
	
	\item What is the probability that two balls of opposite colour are chosen?
	\end{enumerate}
	\solution
	%\input{exemplar/11/16/3/12/main1.tex}
\end{enumerate}

\item In a certain lottery 10,000 tickets are sold and ten equal prizes are awarded. What is the probability of not getting a prize if you buy (a) one ticket (b) two tickets (c) 10 tickets ?	
\\
\solution
		%\begin{enumerate}[label=\thesection.\arabic*,ref=\thesection.\theenumi]
	\item One card is drawn from a well-shuffled deck of 52 cards. Find the probability of getting
\begin{enumerate}
\item A king of red colour 
\item A face card 
\item A red face card
\item The jack of hearts
\item A spade
\item The queen of diamonds

\end{enumerate}
\solution
		%\input{ncert/10/15/1/14/main.tex}
	\item Five cards—the ten, jack, queen, king and ace of diamonds, are well-shuffled with their face downwards. One card is then picked up at random.
\begin{enumerate}
\item
What is the probability that the card is the queen? 
\item
If the queen is drawn and put aside, what is the probability that the second card picked up is (a) an ace? (b) a queen?\\
\end{enumerate}
\solution
		%\input{ncert/10/15/1/15/defs.tex}
	\item A bag contains $5$ red balls and some blue balls. If the probability of drawing a blue ball is double that if a red ball, determine the number of blue balls in the bag. 
		\\
\solution
		%\input{ncert/10/15/2/3/defs.tex}
	\item A card is selected from a pack of 52 cards.
 \begin{enumerate}[label=(\alph*)] 
                 \item How many points are there in the sample space?
                 \item Calculate the probability that the card is an ace of spades.
                 \item Calculate the probability that the card is (i) an ace and (ii) black card.
 \end{enumerate}
\solution
		%\input{ncert/11/16/3/4/main.tex}
\item Four cards are drawn from a well-shuffled deck of 52 cards. What is the probability of obtaining 3 diamonds and one spade.
\\
\solution
		%\input{ncert/11/16/4/2/defs.tex}
\item In a certain lottery 10,000 tickets are sold and ten equal prizes are awarded. What is the probability of not getting a prize if you buy (a) one ticket (b) two tickets (c) 10 tickets ?	
\\
\solution
		%\input{ncert/11/16/4/4/defs.tex}
		%
\item 
Out of 100 students, two sections of 40 and 60 are formed. If you and your friend are among the 100 students, what is the probability that
\begin{enumerate}
\item you both enter the same section?
\item you both enter the different sections?
\end{enumerate}
\solution
		%\input{ncert/11/16/4/5/defs.tex}
	\item 
The number lock of a suitcase has 4 wheels each labelled with ten digits i.e. from 0 to 9.The lock opens with a sequence of four digits with no repeats.What is the probability of a person getting the right sequence to open the suitcase.
\\
\solution
		%\input{ncert/11/16/4/10/defs.tex}
		%
\item 
Two cards are drawn at random and without replacement from a pack of 52 playing cards. Find the probability that both the cards are black.
\\
\solution
		%\input{ncert/12/13/2/2/defs.tex}
		\item A box of oranges is inspected by examining three randomly selected oranges drawn without replacement. If all the three oranges are good, the box is approved for sale, otherwise, it is rejected. Find the probability that a box containing 15 oranges out of which 12 are good and 3 are bad ones will be approved for sale.
		\label{ncert/12/13/2/3/defs.tex}
		\item Two balls are drawn at random with replacement from a box containing 10 black and 8 red balls. Find the probability that
		\label{ncert/12/13/2/12}
\begin{enumerate}
\item both balls are red.
\item first ball is black and second is red.
\item one of them is black and other is red.
\end{enumerate}

\item In a hostel, 60\% of the students read Hindi newspaper, 40\% read English newspaper and 20\% read both Hindi and English newspapers. A student is selected at random.
		\label{ncert/12/13/2/15}
\begin{enumerate}
\item Find the probability that she reads neither Hindi nor English newspapers.
\item If she reads Hindi newspaper, find the probability that she reads English newspaper.
\item If she reads English newspaper, find the probability that she reads Hindi newspaper.\\
\end{enumerate}
\item The probability of obtaining an even prime number on each die, when a pair of dice is rolled is 
\begin{enumerate}
    \item $0$ 
    
    \item $\frac{1}{3}$ 
    
    \item $\frac{1}{12}$ 
    
    \item $\frac{1}{36}$ 
\end{enumerate}
\solution
		%\input{ncert/12/13/2/17/defs.tex}
	\item A bag contains 4 red and 4 black balls, another bag contains 2 red and 6 black balls. One of the two bags is selected at random and a ball is drawn from the bag which is found to be red. Find the probability that the ball is drawn from the first bag.
\\
\solution
		%\input{ncert/12/13/3/2/main.tex}
  \item
  Cards with numbers 2 to 101 are placed in a box. A card is selected at random.Find the probability that the card has
\begin{enumerate}[label=(\roman*)]
	\item an even number 
	\item a square number
\end{enumerate}
\solution
%\input{exemplar/10/13/3/32/main.tex}
\item
The king, queen and jack of clubs are removed from a deck of 52 playing cards and then well shuffled. Now one card is drawn at random from the remaining cards.  Determine the probability that the card is
\begin{enumerate}[label=(\roman*)]
\item a club
\item 10 of hearts
\end{enumerate}
\solution
%\input{exemplar/10/13/3/29/main.tex}
\item A team of medical students doing their internship have to assist during surgeries
at a city hospital. The probabilities of surgeries rated as very complex, complex,
routine, simple or very simple are respectively, 0.15, 0.20, 0.31, 0.26, .08. Find
the probabilities that a particular surgery will be rated
\begin{enumerate}
	\item complex or very complex;
	\item neither very complex nor very simple;
	\item routine or complex
	\item routine or simple
\end{enumerate}
\solution
%\input{exemplar/11/16/3/8(1)/main.tex}
\item A card is selected from a pack of 52 cards.
\begin{enumerate}[label=(\alph*)]
    \item How many points are there in the sample space?
    \item Calculate the probability that the card is an ace of spades.
    \item Calculate the probability that the card is (i) an ace and (ii) black card.
\end{enumerate}
\solution
%\input{exemplar/11/16/3/4/main2.tex}
\item The probability that a non leap year selected at random will contain 53 sundays.
\\
\solution
%\input{exemplar/10/13/1/19/main.tex}
\item One of the four persons John, Rita, Aslam or Gurpreet will be promoted next
month. Consequently the sample space consists of four elementary outcomes
S = {John promoted, Rita promoted, Aslam promoted, Gurpreet promoted}
You are told that the chances of John’s promotion is same as that of Gurpreet,
Rita’s chances of promotion are twice as likely as Johns. Aslam’s chances are
four times that of John.
\begin{enumerate}
	\item Determine
	\begin{enumerate}
		\item P (John promoted)
		\item P (Rita promoted)
		\item P (Aslam promoted)
		\item P (Gurpreet promoted)
	\end{enumerate}
	\item If A = {John promoted or Gurpreet promoted}, find P (A).
\end{enumerate}
\solution
%\input{exemplar/11/16/3/10/main.tex}
\item A card is drawn from a deck of 52 cards. Find the probability of getting a king or a heart or a red card.\\
\solution
%\input{exemplar/11/16/3/15/main.tex}
\item The probability that a student will pass his examination is 0.73, the probability of
the student getting a compartment is 0.13, and the probability that the student will
either pass or get compartment is 0.96. State True or False.\\
\solution
%\input{exemplar/11/16/3/31/main.tex}
\item A card is selected from a pack of 52 cards\\
\begin{enumerate}[label=(\alph*)]
\item How many points are there in the sample space?
\item Calculate the probability that the cards is an ace of spades.
\item Calculate the probability that the card is (i) an ace (ii)black card.\\
\end{enumerate}
%\input{ncert/11/16/3/4_1/Prob_4.tex}
\item In a non-leap year, the probability of having 53 tuesdays or 53 wednesdays is\\
\solution
%\input{exemplar/11/16/3/18/main.tex}
\item There are 1000 sealed envelopes in a box, 10 of them contain a cash prize of
Rs 100 each, 100 of them contain a cash prize of Rs 50 each and 200 of them
contain a cash prize of Rs 10 each and rest do not contain any cash prize. If they
are well shuffled and an envelope is picked up out, what is the probability that it
contains no cash prize?\\
\solution
%\input{exemplar/10/13/3/34/main.tex}
\item 
A die is thrown and a card is selected at random from a deck of 52 playing cards. The probability of getting an even number on the die and a spade card.\\
\solution
%\input{exemplar/12/13/3/78/main.tex}
\item
If 4-digit numbers greater than 5,000 are randomly formed from the digits 0, 1, 3, 5, and 7, what is the probability of forming a number divisible by 5 when:
\begin{enumerate}
    \item The digits are repeated?
    \item The repetition of digits is not allowed?
\end{enumerate}
\solution
%\input{ncert/11/16/4/9/main.tex}
\item Consider the probability space $\brak{\Omega, \mathcal{G}, P}$ where $\Omega = [0,2]$ and $\mathcal{G} = \cbrak{\phi, \Omega, [0,1], (1,2]}$. Let $X$ and $Y$ be two functions on $\Omega$ defined as
\begin{align*}
    X(\omega) = 
    \begin{cases}
        1 & \text{if }\omega \in [0, 1]\\
        2 & \text{if }\omega \in (1, 2]
    \end{cases}
\end{align*}
and
\begin{align*}
    Y(\omega) = 
    \begin{cases}
        2 & \text{if }\omega \in [0, 1.5]\\
        3 & \text{if }\omega \in (1.5, 2].
    \end{cases}
\end{align*}
Then which one of the following statements is true?
\begin{enumerate}
    \item [(A)] $X$ is a random variable with respect to $\mathcal{G}$, but $Y$ is not a random variable with respect to $\mathcal{G}$.
    \item [(B)] $Y$ is a random variable with respect to $\mathcal{G}$, but $X$ is not a random variable with respect to $\mathcal{G}$.
    \item [(C)] Neither $X$ nor $Y$ is a random variable with respect to $\mathcal{G}$.
    \item [(D)] Both $X$ and $Y$ are random variables with respect to $\mathcal{G}$.
\end{enumerate} \hfill (GATE ST 2023)\\
\solution
%\input{gate/ST/2023/14/main.tex}
	\item  A die is loaded in such a way that each odd number is twice as likely to occur as
each even number. Find $P(G)$, where $G$ is the event that a number greater than
3 occurs on a single roll of the die.
\\
\solution
		%\input{exemplar/11/16/3/5/main.tex}
	\item All the jacks, queens and kings are removed from a deck of 52 playing cards. The remaining cards are well shuffled and then one card is drawn at random. Giving ace a value 1 similar value for other cards, find the probability that the card has a value 
		\begin{enumerate}
			\item 7
			\item greater than 7
			\item less than 7
		\end{enumerate}
		%\input{exemplar/10/13/3/30/main.tex}
  \item A Lot consists of 48 mobile phones of which 42 are good, 3 have only minor defects and 3 have major defects.Varnika will buy a phone if it is good but the trader will only buy a mobile if it has no major defects. One phone is selected at random from the lot. What is the probability that it is
\begin{enumerate}
	\item acceptable to Varnika?
            \item acceptable to the trader?
\end{enumerate}
\solution
	%\input{exemplar/10/13/3/40/main.tex}
 \item A student says that if you throw a die, it will show up 1 or not 1. Therefore, the probability of getting 1 and the probability of getting 'not 1' each is equal to $\frac{1}{2}$. Is this correct? Give reasons.\\
 \solution
        %\input{exemplar/10/13/2/9/main.tex}
   \item Four candidates A, B, C, D have ap-
plied for the assignment to coach a school cricket
team. If A is twice as likely to be selected as B, and
B and C are given about the same chance of being
selected, while C is twice as likely to be selected
as D, what are the probabilities that
\begin{enumerate}
\item C will be selected?
\item A will not be selected?
\end{enumerate}
	%\input{exemplar/11/16/3/9/main.tex}
 \item A bag contain 24 balls of which $x$ balls are red, $2x$ are white and $3x$ are blue. A ball is selected at random, What is the probability that it is
\begin{enumerate}[label=\alph*)]
\item not red ?
\item white ?
\end{enumerate}
%\input{exemplar/10/13/3/41/main.tex}
If the letters of the word ASSASSINATION are arranged at random. Find the Probability that
\begin{enumerate}[label=(\alph*)]
\item Four $S's$ come consecutively in the word
\item Two  $I's$ and two $N's$ come together
\item All $A's$ are not coming together
\item No two $A's$ are coming together
\end{enumerate}
%\input{exemplar/11/16/3/14/main.tex}
	\item One urn contains two black balls (labelled B1 and B2) and one white ball. A
	second urn contains one black ball and two white balls (labelled W1 and W2).
	Suppose the following experiment is performed. One of the two urns is chosen
	at random. Next a ball is randomly chosen from the urn. Then a second ball is
	chosen at random from the same urn without replacing the first ball.
	
	\begin{enumerate}
	\item What is the probability that two black balls are chosen?
	
	\item What is the probability that two balls of opposite colour are chosen?
	\end{enumerate}
	\solution
	%\input{exemplar/11/16/3/12/main1.tex}
\end{enumerate}

		%
\item 
Out of 100 students, two sections of 40 and 60 are formed. If you and your friend are among the 100 students, what is the probability that
\begin{enumerate}
\item you both enter the same section?
\item you both enter the different sections?
\end{enumerate}
\solution
		%\begin{enumerate}[label=\thesection.\arabic*,ref=\thesection.\theenumi]
	\item One card is drawn from a well-shuffled deck of 52 cards. Find the probability of getting
\begin{enumerate}
\item A king of red colour 
\item A face card 
\item A red face card
\item The jack of hearts
\item A spade
\item The queen of diamonds

\end{enumerate}
\solution
		%\input{ncert/10/15/1/14/main.tex}
	\item Five cards—the ten, jack, queen, king and ace of diamonds, are well-shuffled with their face downwards. One card is then picked up at random.
\begin{enumerate}
\item
What is the probability that the card is the queen? 
\item
If the queen is drawn and put aside, what is the probability that the second card picked up is (a) an ace? (b) a queen?\\
\end{enumerate}
\solution
		%\input{ncert/10/15/1/15/defs.tex}
	\item A bag contains $5$ red balls and some blue balls. If the probability of drawing a blue ball is double that if a red ball, determine the number of blue balls in the bag. 
		\\
\solution
		%\input{ncert/10/15/2/3/defs.tex}
	\item A card is selected from a pack of 52 cards.
 \begin{enumerate}[label=(\alph*)] 
                 \item How many points are there in the sample space?
                 \item Calculate the probability that the card is an ace of spades.
                 \item Calculate the probability that the card is (i) an ace and (ii) black card.
 \end{enumerate}
\solution
		%\input{ncert/11/16/3/4/main.tex}
\item Four cards are drawn from a well-shuffled deck of 52 cards. What is the probability of obtaining 3 diamonds and one spade.
\\
\solution
		%\input{ncert/11/16/4/2/defs.tex}
\item In a certain lottery 10,000 tickets are sold and ten equal prizes are awarded. What is the probability of not getting a prize if you buy (a) one ticket (b) two tickets (c) 10 tickets ?	
\\
\solution
		%\input{ncert/11/16/4/4/defs.tex}
		%
\item 
Out of 100 students, two sections of 40 and 60 are formed. If you and your friend are among the 100 students, what is the probability that
\begin{enumerate}
\item you both enter the same section?
\item you both enter the different sections?
\end{enumerate}
\solution
		%\input{ncert/11/16/4/5/defs.tex}
	\item 
The number lock of a suitcase has 4 wheels each labelled with ten digits i.e. from 0 to 9.The lock opens with a sequence of four digits with no repeats.What is the probability of a person getting the right sequence to open the suitcase.
\\
\solution
		%\input{ncert/11/16/4/10/defs.tex}
		%
\item 
Two cards are drawn at random and without replacement from a pack of 52 playing cards. Find the probability that both the cards are black.
\\
\solution
		%\input{ncert/12/13/2/2/defs.tex}
		\item A box of oranges is inspected by examining three randomly selected oranges drawn without replacement. If all the three oranges are good, the box is approved for sale, otherwise, it is rejected. Find the probability that a box containing 15 oranges out of which 12 are good and 3 are bad ones will be approved for sale.
		\label{ncert/12/13/2/3/defs.tex}
		\item Two balls are drawn at random with replacement from a box containing 10 black and 8 red balls. Find the probability that
		\label{ncert/12/13/2/12}
\begin{enumerate}
\item both balls are red.
\item first ball is black and second is red.
\item one of them is black and other is red.
\end{enumerate}

\item In a hostel, 60\% of the students read Hindi newspaper, 40\% read English newspaper and 20\% read both Hindi and English newspapers. A student is selected at random.
		\label{ncert/12/13/2/15}
\begin{enumerate}
\item Find the probability that she reads neither Hindi nor English newspapers.
\item If she reads Hindi newspaper, find the probability that she reads English newspaper.
\item If she reads English newspaper, find the probability that she reads Hindi newspaper.\\
\end{enumerate}
\item The probability of obtaining an even prime number on each die, when a pair of dice is rolled is 
\begin{enumerate}
    \item $0$ 
    
    \item $\frac{1}{3}$ 
    
    \item $\frac{1}{12}$ 
    
    \item $\frac{1}{36}$ 
\end{enumerate}
\solution
		%\input{ncert/12/13/2/17/defs.tex}
	\item A bag contains 4 red and 4 black balls, another bag contains 2 red and 6 black balls. One of the two bags is selected at random and a ball is drawn from the bag which is found to be red. Find the probability that the ball is drawn from the first bag.
\\
\solution
		%\input{ncert/12/13/3/2/main.tex}
  \item
  Cards with numbers 2 to 101 are placed in a box. A card is selected at random.Find the probability that the card has
\begin{enumerate}[label=(\roman*)]
	\item an even number 
	\item a square number
\end{enumerate}
\solution
%\input{exemplar/10/13/3/32/main.tex}
\item
The king, queen and jack of clubs are removed from a deck of 52 playing cards and then well shuffled. Now one card is drawn at random from the remaining cards.  Determine the probability that the card is
\begin{enumerate}[label=(\roman*)]
\item a club
\item 10 of hearts
\end{enumerate}
\solution
%\input{exemplar/10/13/3/29/main.tex}
\item A team of medical students doing their internship have to assist during surgeries
at a city hospital. The probabilities of surgeries rated as very complex, complex,
routine, simple or very simple are respectively, 0.15, 0.20, 0.31, 0.26, .08. Find
the probabilities that a particular surgery will be rated
\begin{enumerate}
	\item complex or very complex;
	\item neither very complex nor very simple;
	\item routine or complex
	\item routine or simple
\end{enumerate}
\solution
%\input{exemplar/11/16/3/8(1)/main.tex}
\item A card is selected from a pack of 52 cards.
\begin{enumerate}[label=(\alph*)]
    \item How many points are there in the sample space?
    \item Calculate the probability that the card is an ace of spades.
    \item Calculate the probability that the card is (i) an ace and (ii) black card.
\end{enumerate}
\solution
%\input{exemplar/11/16/3/4/main2.tex}
\item The probability that a non leap year selected at random will contain 53 sundays.
\\
\solution
%\input{exemplar/10/13/1/19/main.tex}
\item One of the four persons John, Rita, Aslam or Gurpreet will be promoted next
month. Consequently the sample space consists of four elementary outcomes
S = {John promoted, Rita promoted, Aslam promoted, Gurpreet promoted}
You are told that the chances of John’s promotion is same as that of Gurpreet,
Rita’s chances of promotion are twice as likely as Johns. Aslam’s chances are
four times that of John.
\begin{enumerate}
	\item Determine
	\begin{enumerate}
		\item P (John promoted)
		\item P (Rita promoted)
		\item P (Aslam promoted)
		\item P (Gurpreet promoted)
	\end{enumerate}
	\item If A = {John promoted or Gurpreet promoted}, find P (A).
\end{enumerate}
\solution
%\input{exemplar/11/16/3/10/main.tex}
\item A card is drawn from a deck of 52 cards. Find the probability of getting a king or a heart or a red card.\\
\solution
%\input{exemplar/11/16/3/15/main.tex}
\item The probability that a student will pass his examination is 0.73, the probability of
the student getting a compartment is 0.13, and the probability that the student will
either pass or get compartment is 0.96. State True or False.\\
\solution
%\input{exemplar/11/16/3/31/main.tex}
\item A card is selected from a pack of 52 cards\\
\begin{enumerate}[label=(\alph*)]
\item How many points are there in the sample space?
\item Calculate the probability that the cards is an ace of spades.
\item Calculate the probability that the card is (i) an ace (ii)black card.\\
\end{enumerate}
%\input{ncert/11/16/3/4_1/Prob_4.tex}
\item In a non-leap year, the probability of having 53 tuesdays or 53 wednesdays is\\
\solution
%\input{exemplar/11/16/3/18/main.tex}
\item There are 1000 sealed envelopes in a box, 10 of them contain a cash prize of
Rs 100 each, 100 of them contain a cash prize of Rs 50 each and 200 of them
contain a cash prize of Rs 10 each and rest do not contain any cash prize. If they
are well shuffled and an envelope is picked up out, what is the probability that it
contains no cash prize?\\
\solution
%\input{exemplar/10/13/3/34/main.tex}
\item 
A die is thrown and a card is selected at random from a deck of 52 playing cards. The probability of getting an even number on the die and a spade card.\\
\solution
%\input{exemplar/12/13/3/78/main.tex}
\item
If 4-digit numbers greater than 5,000 are randomly formed from the digits 0, 1, 3, 5, and 7, what is the probability of forming a number divisible by 5 when:
\begin{enumerate}
    \item The digits are repeated?
    \item The repetition of digits is not allowed?
\end{enumerate}
\solution
%\input{ncert/11/16/4/9/main.tex}
\item Consider the probability space $\brak{\Omega, \mathcal{G}, P}$ where $\Omega = [0,2]$ and $\mathcal{G} = \cbrak{\phi, \Omega, [0,1], (1,2]}$. Let $X$ and $Y$ be two functions on $\Omega$ defined as
\begin{align*}
    X(\omega) = 
    \begin{cases}
        1 & \text{if }\omega \in [0, 1]\\
        2 & \text{if }\omega \in (1, 2]
    \end{cases}
\end{align*}
and
\begin{align*}
    Y(\omega) = 
    \begin{cases}
        2 & \text{if }\omega \in [0, 1.5]\\
        3 & \text{if }\omega \in (1.5, 2].
    \end{cases}
\end{align*}
Then which one of the following statements is true?
\begin{enumerate}
    \item [(A)] $X$ is a random variable with respect to $\mathcal{G}$, but $Y$ is not a random variable with respect to $\mathcal{G}$.
    \item [(B)] $Y$ is a random variable with respect to $\mathcal{G}$, but $X$ is not a random variable with respect to $\mathcal{G}$.
    \item [(C)] Neither $X$ nor $Y$ is a random variable with respect to $\mathcal{G}$.
    \item [(D)] Both $X$ and $Y$ are random variables with respect to $\mathcal{G}$.
\end{enumerate} \hfill (GATE ST 2023)\\
\solution
%\input{gate/ST/2023/14/main.tex}
	\item  A die is loaded in such a way that each odd number is twice as likely to occur as
each even number. Find $P(G)$, where $G$ is the event that a number greater than
3 occurs on a single roll of the die.
\\
\solution
		%\input{exemplar/11/16/3/5/main.tex}
	\item All the jacks, queens and kings are removed from a deck of 52 playing cards. The remaining cards are well shuffled and then one card is drawn at random. Giving ace a value 1 similar value for other cards, find the probability that the card has a value 
		\begin{enumerate}
			\item 7
			\item greater than 7
			\item less than 7
		\end{enumerate}
		%\input{exemplar/10/13/3/30/main.tex}
  \item A Lot consists of 48 mobile phones of which 42 are good, 3 have only minor defects and 3 have major defects.Varnika will buy a phone if it is good but the trader will only buy a mobile if it has no major defects. One phone is selected at random from the lot. What is the probability that it is
\begin{enumerate}
	\item acceptable to Varnika?
            \item acceptable to the trader?
\end{enumerate}
\solution
	%\input{exemplar/10/13/3/40/main.tex}
 \item A student says that if you throw a die, it will show up 1 or not 1. Therefore, the probability of getting 1 and the probability of getting 'not 1' each is equal to $\frac{1}{2}$. Is this correct? Give reasons.\\
 \solution
        %\input{exemplar/10/13/2/9/main.tex}
   \item Four candidates A, B, C, D have ap-
plied for the assignment to coach a school cricket
team. If A is twice as likely to be selected as B, and
B and C are given about the same chance of being
selected, while C is twice as likely to be selected
as D, what are the probabilities that
\begin{enumerate}
\item C will be selected?
\item A will not be selected?
\end{enumerate}
	%\input{exemplar/11/16/3/9/main.tex}
 \item A bag contain 24 balls of which $x$ balls are red, $2x$ are white and $3x$ are blue. A ball is selected at random, What is the probability that it is
\begin{enumerate}[label=\alph*)]
\item not red ?
\item white ?
\end{enumerate}
%\input{exemplar/10/13/3/41/main.tex}
If the letters of the word ASSASSINATION are arranged at random. Find the Probability that
\begin{enumerate}[label=(\alph*)]
\item Four $S's$ come consecutively in the word
\item Two  $I's$ and two $N's$ come together
\item All $A's$ are not coming together
\item No two $A's$ are coming together
\end{enumerate}
%\input{exemplar/11/16/3/14/main.tex}
	\item One urn contains two black balls (labelled B1 and B2) and one white ball. A
	second urn contains one black ball and two white balls (labelled W1 and W2).
	Suppose the following experiment is performed. One of the two urns is chosen
	at random. Next a ball is randomly chosen from the urn. Then a second ball is
	chosen at random from the same urn without replacing the first ball.
	
	\begin{enumerate}
	\item What is the probability that two black balls are chosen?
	
	\item What is the probability that two balls of opposite colour are chosen?
	\end{enumerate}
	\solution
	%\input{exemplar/11/16/3/12/main1.tex}
\end{enumerate}

	\item 
The number lock of a suitcase has 4 wheels each labelled with ten digits i.e. from 0 to 9.The lock opens with a sequence of four digits with no repeats.What is the probability of a person getting the right sequence to open the suitcase.
\\
\solution
		%\begin{enumerate}[label=\thesection.\arabic*,ref=\thesection.\theenumi]
	\item One card is drawn from a well-shuffled deck of 52 cards. Find the probability of getting
\begin{enumerate}
\item A king of red colour 
\item A face card 
\item A red face card
\item The jack of hearts
\item A spade
\item The queen of diamonds

\end{enumerate}
\solution
		%\input{ncert/10/15/1/14/main.tex}
	\item Five cards—the ten, jack, queen, king and ace of diamonds, are well-shuffled with their face downwards. One card is then picked up at random.
\begin{enumerate}
\item
What is the probability that the card is the queen? 
\item
If the queen is drawn and put aside, what is the probability that the second card picked up is (a) an ace? (b) a queen?\\
\end{enumerate}
\solution
		%\input{ncert/10/15/1/15/defs.tex}
	\item A bag contains $5$ red balls and some blue balls. If the probability of drawing a blue ball is double that if a red ball, determine the number of blue balls in the bag. 
		\\
\solution
		%\input{ncert/10/15/2/3/defs.tex}
	\item A card is selected from a pack of 52 cards.
 \begin{enumerate}[label=(\alph*)] 
                 \item How many points are there in the sample space?
                 \item Calculate the probability that the card is an ace of spades.
                 \item Calculate the probability that the card is (i) an ace and (ii) black card.
 \end{enumerate}
\solution
		%\input{ncert/11/16/3/4/main.tex}
\item Four cards are drawn from a well-shuffled deck of 52 cards. What is the probability of obtaining 3 diamonds and one spade.
\\
\solution
		%\input{ncert/11/16/4/2/defs.tex}
\item In a certain lottery 10,000 tickets are sold and ten equal prizes are awarded. What is the probability of not getting a prize if you buy (a) one ticket (b) two tickets (c) 10 tickets ?	
\\
\solution
		%\input{ncert/11/16/4/4/defs.tex}
		%
\item 
Out of 100 students, two sections of 40 and 60 are formed. If you and your friend are among the 100 students, what is the probability that
\begin{enumerate}
\item you both enter the same section?
\item you both enter the different sections?
\end{enumerate}
\solution
		%\input{ncert/11/16/4/5/defs.tex}
	\item 
The number lock of a suitcase has 4 wheels each labelled with ten digits i.e. from 0 to 9.The lock opens with a sequence of four digits with no repeats.What is the probability of a person getting the right sequence to open the suitcase.
\\
\solution
		%\input{ncert/11/16/4/10/defs.tex}
		%
\item 
Two cards are drawn at random and without replacement from a pack of 52 playing cards. Find the probability that both the cards are black.
\\
\solution
		%\input{ncert/12/13/2/2/defs.tex}
		\item A box of oranges is inspected by examining three randomly selected oranges drawn without replacement. If all the three oranges are good, the box is approved for sale, otherwise, it is rejected. Find the probability that a box containing 15 oranges out of which 12 are good and 3 are bad ones will be approved for sale.
		\label{ncert/12/13/2/3/defs.tex}
		\item Two balls are drawn at random with replacement from a box containing 10 black and 8 red balls. Find the probability that
		\label{ncert/12/13/2/12}
\begin{enumerate}
\item both balls are red.
\item first ball is black and second is red.
\item one of them is black and other is red.
\end{enumerate}

\item In a hostel, 60\% of the students read Hindi newspaper, 40\% read English newspaper and 20\% read both Hindi and English newspapers. A student is selected at random.
		\label{ncert/12/13/2/15}
\begin{enumerate}
\item Find the probability that she reads neither Hindi nor English newspapers.
\item If she reads Hindi newspaper, find the probability that she reads English newspaper.
\item If she reads English newspaper, find the probability that she reads Hindi newspaper.\\
\end{enumerate}
\item The probability of obtaining an even prime number on each die, when a pair of dice is rolled is 
\begin{enumerate}
    \item $0$ 
    
    \item $\frac{1}{3}$ 
    
    \item $\frac{1}{12}$ 
    
    \item $\frac{1}{36}$ 
\end{enumerate}
\solution
		%\input{ncert/12/13/2/17/defs.tex}
	\item A bag contains 4 red and 4 black balls, another bag contains 2 red and 6 black balls. One of the two bags is selected at random and a ball is drawn from the bag which is found to be red. Find the probability that the ball is drawn from the first bag.
\\
\solution
		%\input{ncert/12/13/3/2/main.tex}
  \item
  Cards with numbers 2 to 101 are placed in a box. A card is selected at random.Find the probability that the card has
\begin{enumerate}[label=(\roman*)]
	\item an even number 
	\item a square number
\end{enumerate}
\solution
%\input{exemplar/10/13/3/32/main.tex}
\item
The king, queen and jack of clubs are removed from a deck of 52 playing cards and then well shuffled. Now one card is drawn at random from the remaining cards.  Determine the probability that the card is
\begin{enumerate}[label=(\roman*)]
\item a club
\item 10 of hearts
\end{enumerate}
\solution
%\input{exemplar/10/13/3/29/main.tex}
\item A team of medical students doing their internship have to assist during surgeries
at a city hospital. The probabilities of surgeries rated as very complex, complex,
routine, simple or very simple are respectively, 0.15, 0.20, 0.31, 0.26, .08. Find
the probabilities that a particular surgery will be rated
\begin{enumerate}
	\item complex or very complex;
	\item neither very complex nor very simple;
	\item routine or complex
	\item routine or simple
\end{enumerate}
\solution
%\input{exemplar/11/16/3/8(1)/main.tex}
\item A card is selected from a pack of 52 cards.
\begin{enumerate}[label=(\alph*)]
    \item How many points are there in the sample space?
    \item Calculate the probability that the card is an ace of spades.
    \item Calculate the probability that the card is (i) an ace and (ii) black card.
\end{enumerate}
\solution
%\input{exemplar/11/16/3/4/main2.tex}
\item The probability that a non leap year selected at random will contain 53 sundays.
\\
\solution
%\input{exemplar/10/13/1/19/main.tex}
\item One of the four persons John, Rita, Aslam or Gurpreet will be promoted next
month. Consequently the sample space consists of four elementary outcomes
S = {John promoted, Rita promoted, Aslam promoted, Gurpreet promoted}
You are told that the chances of John’s promotion is same as that of Gurpreet,
Rita’s chances of promotion are twice as likely as Johns. Aslam’s chances are
four times that of John.
\begin{enumerate}
	\item Determine
	\begin{enumerate}
		\item P (John promoted)
		\item P (Rita promoted)
		\item P (Aslam promoted)
		\item P (Gurpreet promoted)
	\end{enumerate}
	\item If A = {John promoted or Gurpreet promoted}, find P (A).
\end{enumerate}
\solution
%\input{exemplar/11/16/3/10/main.tex}
\item A card is drawn from a deck of 52 cards. Find the probability of getting a king or a heart or a red card.\\
\solution
%\input{exemplar/11/16/3/15/main.tex}
\item The probability that a student will pass his examination is 0.73, the probability of
the student getting a compartment is 0.13, and the probability that the student will
either pass or get compartment is 0.96. State True or False.\\
\solution
%\input{exemplar/11/16/3/31/main.tex}
\item A card is selected from a pack of 52 cards\\
\begin{enumerate}[label=(\alph*)]
\item How many points are there in the sample space?
\item Calculate the probability that the cards is an ace of spades.
\item Calculate the probability that the card is (i) an ace (ii)black card.\\
\end{enumerate}
%\input{ncert/11/16/3/4_1/Prob_4.tex}
\item In a non-leap year, the probability of having 53 tuesdays or 53 wednesdays is\\
\solution
%\input{exemplar/11/16/3/18/main.tex}
\item There are 1000 sealed envelopes in a box, 10 of them contain a cash prize of
Rs 100 each, 100 of them contain a cash prize of Rs 50 each and 200 of them
contain a cash prize of Rs 10 each and rest do not contain any cash prize. If they
are well shuffled and an envelope is picked up out, what is the probability that it
contains no cash prize?\\
\solution
%\input{exemplar/10/13/3/34/main.tex}
\item 
A die is thrown and a card is selected at random from a deck of 52 playing cards. The probability of getting an even number on the die and a spade card.\\
\solution
%\input{exemplar/12/13/3/78/main.tex}
\item
If 4-digit numbers greater than 5,000 are randomly formed from the digits 0, 1, 3, 5, and 7, what is the probability of forming a number divisible by 5 when:
\begin{enumerate}
    \item The digits are repeated?
    \item The repetition of digits is not allowed?
\end{enumerate}
\solution
%\input{ncert/11/16/4/9/main.tex}
\item Consider the probability space $\brak{\Omega, \mathcal{G}, P}$ where $\Omega = [0,2]$ and $\mathcal{G} = \cbrak{\phi, \Omega, [0,1], (1,2]}$. Let $X$ and $Y$ be two functions on $\Omega$ defined as
\begin{align*}
    X(\omega) = 
    \begin{cases}
        1 & \text{if }\omega \in [0, 1]\\
        2 & \text{if }\omega \in (1, 2]
    \end{cases}
\end{align*}
and
\begin{align*}
    Y(\omega) = 
    \begin{cases}
        2 & \text{if }\omega \in [0, 1.5]\\
        3 & \text{if }\omega \in (1.5, 2].
    \end{cases}
\end{align*}
Then which one of the following statements is true?
\begin{enumerate}
    \item [(A)] $X$ is a random variable with respect to $\mathcal{G}$, but $Y$ is not a random variable with respect to $\mathcal{G}$.
    \item [(B)] $Y$ is a random variable with respect to $\mathcal{G}$, but $X$ is not a random variable with respect to $\mathcal{G}$.
    \item [(C)] Neither $X$ nor $Y$ is a random variable with respect to $\mathcal{G}$.
    \item [(D)] Both $X$ and $Y$ are random variables with respect to $\mathcal{G}$.
\end{enumerate} \hfill (GATE ST 2023)\\
\solution
%\input{gate/ST/2023/14/main.tex}
	\item  A die is loaded in such a way that each odd number is twice as likely to occur as
each even number. Find $P(G)$, where $G$ is the event that a number greater than
3 occurs on a single roll of the die.
\\
\solution
		%\input{exemplar/11/16/3/5/main.tex}
	\item All the jacks, queens and kings are removed from a deck of 52 playing cards. The remaining cards are well shuffled and then one card is drawn at random. Giving ace a value 1 similar value for other cards, find the probability that the card has a value 
		\begin{enumerate}
			\item 7
			\item greater than 7
			\item less than 7
		\end{enumerate}
		%\input{exemplar/10/13/3/30/main.tex}
  \item A Lot consists of 48 mobile phones of which 42 are good, 3 have only minor defects and 3 have major defects.Varnika will buy a phone if it is good but the trader will only buy a mobile if it has no major defects. One phone is selected at random from the lot. What is the probability that it is
\begin{enumerate}
	\item acceptable to Varnika?
            \item acceptable to the trader?
\end{enumerate}
\solution
	%\input{exemplar/10/13/3/40/main.tex}
 \item A student says that if you throw a die, it will show up 1 or not 1. Therefore, the probability of getting 1 and the probability of getting 'not 1' each is equal to $\frac{1}{2}$. Is this correct? Give reasons.\\
 \solution
        %\input{exemplar/10/13/2/9/main.tex}
   \item Four candidates A, B, C, D have ap-
plied for the assignment to coach a school cricket
team. If A is twice as likely to be selected as B, and
B and C are given about the same chance of being
selected, while C is twice as likely to be selected
as D, what are the probabilities that
\begin{enumerate}
\item C will be selected?
\item A will not be selected?
\end{enumerate}
	%\input{exemplar/11/16/3/9/main.tex}
 \item A bag contain 24 balls of which $x$ balls are red, $2x$ are white and $3x$ are blue. A ball is selected at random, What is the probability that it is
\begin{enumerate}[label=\alph*)]
\item not red ?
\item white ?
\end{enumerate}
%\input{exemplar/10/13/3/41/main.tex}
If the letters of the word ASSASSINATION are arranged at random. Find the Probability that
\begin{enumerate}[label=(\alph*)]
\item Four $S's$ come consecutively in the word
\item Two  $I's$ and two $N's$ come together
\item All $A's$ are not coming together
\item No two $A's$ are coming together
\end{enumerate}
%\input{exemplar/11/16/3/14/main.tex}
	\item One urn contains two black balls (labelled B1 and B2) and one white ball. A
	second urn contains one black ball and two white balls (labelled W1 and W2).
	Suppose the following experiment is performed. One of the two urns is chosen
	at random. Next a ball is randomly chosen from the urn. Then a second ball is
	chosen at random from the same urn without replacing the first ball.
	
	\begin{enumerate}
	\item What is the probability that two black balls are chosen?
	
	\item What is the probability that two balls of opposite colour are chosen?
	\end{enumerate}
	\solution
	%\input{exemplar/11/16/3/12/main1.tex}
\end{enumerate}

		%
\item 
Two cards are drawn at random and without replacement from a pack of 52 playing cards. Find the probability that both the cards are black.
\\
\solution
		%\begin{enumerate}[label=\thesection.\arabic*,ref=\thesection.\theenumi]
	\item One card is drawn from a well-shuffled deck of 52 cards. Find the probability of getting
\begin{enumerate}
\item A king of red colour 
\item A face card 
\item A red face card
\item The jack of hearts
\item A spade
\item The queen of diamonds

\end{enumerate}
\solution
		%\input{ncert/10/15/1/14/main.tex}
	\item Five cards—the ten, jack, queen, king and ace of diamonds, are well-shuffled with their face downwards. One card is then picked up at random.
\begin{enumerate}
\item
What is the probability that the card is the queen? 
\item
If the queen is drawn and put aside, what is the probability that the second card picked up is (a) an ace? (b) a queen?\\
\end{enumerate}
\solution
		%\input{ncert/10/15/1/15/defs.tex}
	\item A bag contains $5$ red balls and some blue balls. If the probability of drawing a blue ball is double that if a red ball, determine the number of blue balls in the bag. 
		\\
\solution
		%\input{ncert/10/15/2/3/defs.tex}
	\item A card is selected from a pack of 52 cards.
 \begin{enumerate}[label=(\alph*)] 
                 \item How many points are there in the sample space?
                 \item Calculate the probability that the card is an ace of spades.
                 \item Calculate the probability that the card is (i) an ace and (ii) black card.
 \end{enumerate}
\solution
		%\input{ncert/11/16/3/4/main.tex}
\item Four cards are drawn from a well-shuffled deck of 52 cards. What is the probability of obtaining 3 diamonds and one spade.
\\
\solution
		%\input{ncert/11/16/4/2/defs.tex}
\item In a certain lottery 10,000 tickets are sold and ten equal prizes are awarded. What is the probability of not getting a prize if you buy (a) one ticket (b) two tickets (c) 10 tickets ?	
\\
\solution
		%\input{ncert/11/16/4/4/defs.tex}
		%
\item 
Out of 100 students, two sections of 40 and 60 are formed. If you and your friend are among the 100 students, what is the probability that
\begin{enumerate}
\item you both enter the same section?
\item you both enter the different sections?
\end{enumerate}
\solution
		%\input{ncert/11/16/4/5/defs.tex}
	\item 
The number lock of a suitcase has 4 wheels each labelled with ten digits i.e. from 0 to 9.The lock opens with a sequence of four digits with no repeats.What is the probability of a person getting the right sequence to open the suitcase.
\\
\solution
		%\input{ncert/11/16/4/10/defs.tex}
		%
\item 
Two cards are drawn at random and without replacement from a pack of 52 playing cards. Find the probability that both the cards are black.
\\
\solution
		%\input{ncert/12/13/2/2/defs.tex}
		\item A box of oranges is inspected by examining three randomly selected oranges drawn without replacement. If all the three oranges are good, the box is approved for sale, otherwise, it is rejected. Find the probability that a box containing 15 oranges out of which 12 are good and 3 are bad ones will be approved for sale.
		\label{ncert/12/13/2/3/defs.tex}
		\item Two balls are drawn at random with replacement from a box containing 10 black and 8 red balls. Find the probability that
		\label{ncert/12/13/2/12}
\begin{enumerate}
\item both balls are red.
\item first ball is black and second is red.
\item one of them is black and other is red.
\end{enumerate}

\item In a hostel, 60\% of the students read Hindi newspaper, 40\% read English newspaper and 20\% read both Hindi and English newspapers. A student is selected at random.
		\label{ncert/12/13/2/15}
\begin{enumerate}
\item Find the probability that she reads neither Hindi nor English newspapers.
\item If she reads Hindi newspaper, find the probability that she reads English newspaper.
\item If she reads English newspaper, find the probability that she reads Hindi newspaper.\\
\end{enumerate}
\item The probability of obtaining an even prime number on each die, when a pair of dice is rolled is 
\begin{enumerate}
    \item $0$ 
    
    \item $\frac{1}{3}$ 
    
    \item $\frac{1}{12}$ 
    
    \item $\frac{1}{36}$ 
\end{enumerate}
\solution
		%\input{ncert/12/13/2/17/defs.tex}
	\item A bag contains 4 red and 4 black balls, another bag contains 2 red and 6 black balls. One of the two bags is selected at random and a ball is drawn from the bag which is found to be red. Find the probability that the ball is drawn from the first bag.
\\
\solution
		%\input{ncert/12/13/3/2/main.tex}
  \item
  Cards with numbers 2 to 101 are placed in a box. A card is selected at random.Find the probability that the card has
\begin{enumerate}[label=(\roman*)]
	\item an even number 
	\item a square number
\end{enumerate}
\solution
%\input{exemplar/10/13/3/32/main.tex}
\item
The king, queen and jack of clubs are removed from a deck of 52 playing cards and then well shuffled. Now one card is drawn at random from the remaining cards.  Determine the probability that the card is
\begin{enumerate}[label=(\roman*)]
\item a club
\item 10 of hearts
\end{enumerate}
\solution
%\input{exemplar/10/13/3/29/main.tex}
\item A team of medical students doing their internship have to assist during surgeries
at a city hospital. The probabilities of surgeries rated as very complex, complex,
routine, simple or very simple are respectively, 0.15, 0.20, 0.31, 0.26, .08. Find
the probabilities that a particular surgery will be rated
\begin{enumerate}
	\item complex or very complex;
	\item neither very complex nor very simple;
	\item routine or complex
	\item routine or simple
\end{enumerate}
\solution
%\input{exemplar/11/16/3/8(1)/main.tex}
\item A card is selected from a pack of 52 cards.
\begin{enumerate}[label=(\alph*)]
    \item How many points are there in the sample space?
    \item Calculate the probability that the card is an ace of spades.
    \item Calculate the probability that the card is (i) an ace and (ii) black card.
\end{enumerate}
\solution
%\input{exemplar/11/16/3/4/main2.tex}
\item The probability that a non leap year selected at random will contain 53 sundays.
\\
\solution
%\input{exemplar/10/13/1/19/main.tex}
\item One of the four persons John, Rita, Aslam or Gurpreet will be promoted next
month. Consequently the sample space consists of four elementary outcomes
S = {John promoted, Rita promoted, Aslam promoted, Gurpreet promoted}
You are told that the chances of John’s promotion is same as that of Gurpreet,
Rita’s chances of promotion are twice as likely as Johns. Aslam’s chances are
four times that of John.
\begin{enumerate}
	\item Determine
	\begin{enumerate}
		\item P (John promoted)
		\item P (Rita promoted)
		\item P (Aslam promoted)
		\item P (Gurpreet promoted)
	\end{enumerate}
	\item If A = {John promoted or Gurpreet promoted}, find P (A).
\end{enumerate}
\solution
%\input{exemplar/11/16/3/10/main.tex}
\item A card is drawn from a deck of 52 cards. Find the probability of getting a king or a heart or a red card.\\
\solution
%\input{exemplar/11/16/3/15/main.tex}
\item The probability that a student will pass his examination is 0.73, the probability of
the student getting a compartment is 0.13, and the probability that the student will
either pass or get compartment is 0.96. State True or False.\\
\solution
%\input{exemplar/11/16/3/31/main.tex}
\item A card is selected from a pack of 52 cards\\
\begin{enumerate}[label=(\alph*)]
\item How many points are there in the sample space?
\item Calculate the probability that the cards is an ace of spades.
\item Calculate the probability that the card is (i) an ace (ii)black card.\\
\end{enumerate}
%\input{ncert/11/16/3/4_1/Prob_4.tex}
\item In a non-leap year, the probability of having 53 tuesdays or 53 wednesdays is\\
\solution
%\input{exemplar/11/16/3/18/main.tex}
\item There are 1000 sealed envelopes in a box, 10 of them contain a cash prize of
Rs 100 each, 100 of them contain a cash prize of Rs 50 each and 200 of them
contain a cash prize of Rs 10 each and rest do not contain any cash prize. If they
are well shuffled and an envelope is picked up out, what is the probability that it
contains no cash prize?\\
\solution
%\input{exemplar/10/13/3/34/main.tex}
\item 
A die is thrown and a card is selected at random from a deck of 52 playing cards. The probability of getting an even number on the die and a spade card.\\
\solution
%\input{exemplar/12/13/3/78/main.tex}
\item
If 4-digit numbers greater than 5,000 are randomly formed from the digits 0, 1, 3, 5, and 7, what is the probability of forming a number divisible by 5 when:
\begin{enumerate}
    \item The digits are repeated?
    \item The repetition of digits is not allowed?
\end{enumerate}
\solution
%\input{ncert/11/16/4/9/main.tex}
\item Consider the probability space $\brak{\Omega, \mathcal{G}, P}$ where $\Omega = [0,2]$ and $\mathcal{G} = \cbrak{\phi, \Omega, [0,1], (1,2]}$. Let $X$ and $Y$ be two functions on $\Omega$ defined as
\begin{align*}
    X(\omega) = 
    \begin{cases}
        1 & \text{if }\omega \in [0, 1]\\
        2 & \text{if }\omega \in (1, 2]
    \end{cases}
\end{align*}
and
\begin{align*}
    Y(\omega) = 
    \begin{cases}
        2 & \text{if }\omega \in [0, 1.5]\\
        3 & \text{if }\omega \in (1.5, 2].
    \end{cases}
\end{align*}
Then which one of the following statements is true?
\begin{enumerate}
    \item [(A)] $X$ is a random variable with respect to $\mathcal{G}$, but $Y$ is not a random variable with respect to $\mathcal{G}$.
    \item [(B)] $Y$ is a random variable with respect to $\mathcal{G}$, but $X$ is not a random variable with respect to $\mathcal{G}$.
    \item [(C)] Neither $X$ nor $Y$ is a random variable with respect to $\mathcal{G}$.
    \item [(D)] Both $X$ and $Y$ are random variables with respect to $\mathcal{G}$.
\end{enumerate} \hfill (GATE ST 2023)\\
\solution
%\input{gate/ST/2023/14/main.tex}
	\item  A die is loaded in such a way that each odd number is twice as likely to occur as
each even number. Find $P(G)$, where $G$ is the event that a number greater than
3 occurs on a single roll of the die.
\\
\solution
		%\input{exemplar/11/16/3/5/main.tex}
	\item All the jacks, queens and kings are removed from a deck of 52 playing cards. The remaining cards are well shuffled and then one card is drawn at random. Giving ace a value 1 similar value for other cards, find the probability that the card has a value 
		\begin{enumerate}
			\item 7
			\item greater than 7
			\item less than 7
		\end{enumerate}
		%\input{exemplar/10/13/3/30/main.tex}
  \item A Lot consists of 48 mobile phones of which 42 are good, 3 have only minor defects and 3 have major defects.Varnika will buy a phone if it is good but the trader will only buy a mobile if it has no major defects. One phone is selected at random from the lot. What is the probability that it is
\begin{enumerate}
	\item acceptable to Varnika?
            \item acceptable to the trader?
\end{enumerate}
\solution
	%\input{exemplar/10/13/3/40/main.tex}
 \item A student says that if you throw a die, it will show up 1 or not 1. Therefore, the probability of getting 1 and the probability of getting 'not 1' each is equal to $\frac{1}{2}$. Is this correct? Give reasons.\\
 \solution
        %\input{exemplar/10/13/2/9/main.tex}
   \item Four candidates A, B, C, D have ap-
plied for the assignment to coach a school cricket
team. If A is twice as likely to be selected as B, and
B and C are given about the same chance of being
selected, while C is twice as likely to be selected
as D, what are the probabilities that
\begin{enumerate}
\item C will be selected?
\item A will not be selected?
\end{enumerate}
	%\input{exemplar/11/16/3/9/main.tex}
 \item A bag contain 24 balls of which $x$ balls are red, $2x$ are white and $3x$ are blue. A ball is selected at random, What is the probability that it is
\begin{enumerate}[label=\alph*)]
\item not red ?
\item white ?
\end{enumerate}
%\input{exemplar/10/13/3/41/main.tex}
If the letters of the word ASSASSINATION are arranged at random. Find the Probability that
\begin{enumerate}[label=(\alph*)]
\item Four $S's$ come consecutively in the word
\item Two  $I's$ and two $N's$ come together
\item All $A's$ are not coming together
\item No two $A's$ are coming together
\end{enumerate}
%\input{exemplar/11/16/3/14/main.tex}
	\item One urn contains two black balls (labelled B1 and B2) and one white ball. A
	second urn contains one black ball and two white balls (labelled W1 and W2).
	Suppose the following experiment is performed. One of the two urns is chosen
	at random. Next a ball is randomly chosen from the urn. Then a second ball is
	chosen at random from the same urn without replacing the first ball.
	
	\begin{enumerate}
	\item What is the probability that two black balls are chosen?
	
	\item What is the probability that two balls of opposite colour are chosen?
	\end{enumerate}
	\solution
	%\input{exemplar/11/16/3/12/main1.tex}
\end{enumerate}

		\item A box of oranges is inspected by examining three randomly selected oranges drawn without replacement. If all the three oranges are good, the box is approved for sale, otherwise, it is rejected. Find the probability that a box containing 15 oranges out of which 12 are good and 3 are bad ones will be approved for sale.
		\label{ncert/12/13/2/3/defs.tex}
		\item Two balls are drawn at random with replacement from a box containing 10 black and 8 red balls. Find the probability that
		\label{ncert/12/13/2/12}
\begin{enumerate}
\item both balls are red.
\item first ball is black and second is red.
\item one of them is black and other is red.
\end{enumerate}

\item In a hostel, 60\% of the students read Hindi newspaper, 40\% read English newspaper and 20\% read both Hindi and English newspapers. A student is selected at random.
		\label{ncert/12/13/2/15}
\begin{enumerate}
\item Find the probability that she reads neither Hindi nor English newspapers.
\item If she reads Hindi newspaper, find the probability that she reads English newspaper.
\item If she reads English newspaper, find the probability that she reads Hindi newspaper.\\
\end{enumerate}
\item The probability of obtaining an even prime number on each die, when a pair of dice is rolled is 
\begin{enumerate}
    \item $0$ 
    
    \item $\frac{1}{3}$ 
    
    \item $\frac{1}{12}$ 
    
    \item $\frac{1}{36}$ 
\end{enumerate}
\solution
		%\begin{enumerate}[label=\thesection.\arabic*,ref=\thesection.\theenumi]
	\item One card is drawn from a well-shuffled deck of 52 cards. Find the probability of getting
\begin{enumerate}
\item A king of red colour 
\item A face card 
\item A red face card
\item The jack of hearts
\item A spade
\item The queen of diamonds

\end{enumerate}
\solution
		%\input{ncert/10/15/1/14/main.tex}
	\item Five cards—the ten, jack, queen, king and ace of diamonds, are well-shuffled with their face downwards. One card is then picked up at random.
\begin{enumerate}
\item
What is the probability that the card is the queen? 
\item
If the queen is drawn and put aside, what is the probability that the second card picked up is (a) an ace? (b) a queen?\\
\end{enumerate}
\solution
		%\input{ncert/10/15/1/15/defs.tex}
	\item A bag contains $5$ red balls and some blue balls. If the probability of drawing a blue ball is double that if a red ball, determine the number of blue balls in the bag. 
		\\
\solution
		%\input{ncert/10/15/2/3/defs.tex}
	\item A card is selected from a pack of 52 cards.
 \begin{enumerate}[label=(\alph*)] 
                 \item How many points are there in the sample space?
                 \item Calculate the probability that the card is an ace of spades.
                 \item Calculate the probability that the card is (i) an ace and (ii) black card.
 \end{enumerate}
\solution
		%\input{ncert/11/16/3/4/main.tex}
\item Four cards are drawn from a well-shuffled deck of 52 cards. What is the probability of obtaining 3 diamonds and one spade.
\\
\solution
		%\input{ncert/11/16/4/2/defs.tex}
\item In a certain lottery 10,000 tickets are sold and ten equal prizes are awarded. What is the probability of not getting a prize if you buy (a) one ticket (b) two tickets (c) 10 tickets ?	
\\
\solution
		%\input{ncert/11/16/4/4/defs.tex}
		%
\item 
Out of 100 students, two sections of 40 and 60 are formed. If you and your friend are among the 100 students, what is the probability that
\begin{enumerate}
\item you both enter the same section?
\item you both enter the different sections?
\end{enumerate}
\solution
		%\input{ncert/11/16/4/5/defs.tex}
	\item 
The number lock of a suitcase has 4 wheels each labelled with ten digits i.e. from 0 to 9.The lock opens with a sequence of four digits with no repeats.What is the probability of a person getting the right sequence to open the suitcase.
\\
\solution
		%\input{ncert/11/16/4/10/defs.tex}
		%
\item 
Two cards are drawn at random and without replacement from a pack of 52 playing cards. Find the probability that both the cards are black.
\\
\solution
		%\input{ncert/12/13/2/2/defs.tex}
		\item A box of oranges is inspected by examining three randomly selected oranges drawn without replacement. If all the three oranges are good, the box is approved for sale, otherwise, it is rejected. Find the probability that a box containing 15 oranges out of which 12 are good and 3 are bad ones will be approved for sale.
		\label{ncert/12/13/2/3/defs.tex}
		\item Two balls are drawn at random with replacement from a box containing 10 black and 8 red balls. Find the probability that
		\label{ncert/12/13/2/12}
\begin{enumerate}
\item both balls are red.
\item first ball is black and second is red.
\item one of them is black and other is red.
\end{enumerate}

\item In a hostel, 60\% of the students read Hindi newspaper, 40\% read English newspaper and 20\% read both Hindi and English newspapers. A student is selected at random.
		\label{ncert/12/13/2/15}
\begin{enumerate}
\item Find the probability that she reads neither Hindi nor English newspapers.
\item If she reads Hindi newspaper, find the probability that she reads English newspaper.
\item If she reads English newspaper, find the probability that she reads Hindi newspaper.\\
\end{enumerate}
\item The probability of obtaining an even prime number on each die, when a pair of dice is rolled is 
\begin{enumerate}
    \item $0$ 
    
    \item $\frac{1}{3}$ 
    
    \item $\frac{1}{12}$ 
    
    \item $\frac{1}{36}$ 
\end{enumerate}
\solution
		%\input{ncert/12/13/2/17/defs.tex}
	\item A bag contains 4 red and 4 black balls, another bag contains 2 red and 6 black balls. One of the two bags is selected at random and a ball is drawn from the bag which is found to be red. Find the probability that the ball is drawn from the first bag.
\\
\solution
		%\input{ncert/12/13/3/2/main.tex}
  \item
  Cards with numbers 2 to 101 are placed in a box. A card is selected at random.Find the probability that the card has
\begin{enumerate}[label=(\roman*)]
	\item an even number 
	\item a square number
\end{enumerate}
\solution
%\input{exemplar/10/13/3/32/main.tex}
\item
The king, queen and jack of clubs are removed from a deck of 52 playing cards and then well shuffled. Now one card is drawn at random from the remaining cards.  Determine the probability that the card is
\begin{enumerate}[label=(\roman*)]
\item a club
\item 10 of hearts
\end{enumerate}
\solution
%\input{exemplar/10/13/3/29/main.tex}
\item A team of medical students doing their internship have to assist during surgeries
at a city hospital. The probabilities of surgeries rated as very complex, complex,
routine, simple or very simple are respectively, 0.15, 0.20, 0.31, 0.26, .08. Find
the probabilities that a particular surgery will be rated
\begin{enumerate}
	\item complex or very complex;
	\item neither very complex nor very simple;
	\item routine or complex
	\item routine or simple
\end{enumerate}
\solution
%\input{exemplar/11/16/3/8(1)/main.tex}
\item A card is selected from a pack of 52 cards.
\begin{enumerate}[label=(\alph*)]
    \item How many points are there in the sample space?
    \item Calculate the probability that the card is an ace of spades.
    \item Calculate the probability that the card is (i) an ace and (ii) black card.
\end{enumerate}
\solution
%\input{exemplar/11/16/3/4/main2.tex}
\item The probability that a non leap year selected at random will contain 53 sundays.
\\
\solution
%\input{exemplar/10/13/1/19/main.tex}
\item One of the four persons John, Rita, Aslam or Gurpreet will be promoted next
month. Consequently the sample space consists of four elementary outcomes
S = {John promoted, Rita promoted, Aslam promoted, Gurpreet promoted}
You are told that the chances of John’s promotion is same as that of Gurpreet,
Rita’s chances of promotion are twice as likely as Johns. Aslam’s chances are
four times that of John.
\begin{enumerate}
	\item Determine
	\begin{enumerate}
		\item P (John promoted)
		\item P (Rita promoted)
		\item P (Aslam promoted)
		\item P (Gurpreet promoted)
	\end{enumerate}
	\item If A = {John promoted or Gurpreet promoted}, find P (A).
\end{enumerate}
\solution
%\input{exemplar/11/16/3/10/main.tex}
\item A card is drawn from a deck of 52 cards. Find the probability of getting a king or a heart or a red card.\\
\solution
%\input{exemplar/11/16/3/15/main.tex}
\item The probability that a student will pass his examination is 0.73, the probability of
the student getting a compartment is 0.13, and the probability that the student will
either pass or get compartment is 0.96. State True or False.\\
\solution
%\input{exemplar/11/16/3/31/main.tex}
\item A card is selected from a pack of 52 cards\\
\begin{enumerate}[label=(\alph*)]
\item How many points are there in the sample space?
\item Calculate the probability that the cards is an ace of spades.
\item Calculate the probability that the card is (i) an ace (ii)black card.\\
\end{enumerate}
%\input{ncert/11/16/3/4_1/Prob_4.tex}
\item In a non-leap year, the probability of having 53 tuesdays or 53 wednesdays is\\
\solution
%\input{exemplar/11/16/3/18/main.tex}
\item There are 1000 sealed envelopes in a box, 10 of them contain a cash prize of
Rs 100 each, 100 of them contain a cash prize of Rs 50 each and 200 of them
contain a cash prize of Rs 10 each and rest do not contain any cash prize. If they
are well shuffled and an envelope is picked up out, what is the probability that it
contains no cash prize?\\
\solution
%\input{exemplar/10/13/3/34/main.tex}
\item 
A die is thrown and a card is selected at random from a deck of 52 playing cards. The probability of getting an even number on the die and a spade card.\\
\solution
%\input{exemplar/12/13/3/78/main.tex}
\item
If 4-digit numbers greater than 5,000 are randomly formed from the digits 0, 1, 3, 5, and 7, what is the probability of forming a number divisible by 5 when:
\begin{enumerate}
    \item The digits are repeated?
    \item The repetition of digits is not allowed?
\end{enumerate}
\solution
%\input{ncert/11/16/4/9/main.tex}
\item Consider the probability space $\brak{\Omega, \mathcal{G}, P}$ where $\Omega = [0,2]$ and $\mathcal{G} = \cbrak{\phi, \Omega, [0,1], (1,2]}$. Let $X$ and $Y$ be two functions on $\Omega$ defined as
\begin{align*}
    X(\omega) = 
    \begin{cases}
        1 & \text{if }\omega \in [0, 1]\\
        2 & \text{if }\omega \in (1, 2]
    \end{cases}
\end{align*}
and
\begin{align*}
    Y(\omega) = 
    \begin{cases}
        2 & \text{if }\omega \in [0, 1.5]\\
        3 & \text{if }\omega \in (1.5, 2].
    \end{cases}
\end{align*}
Then which one of the following statements is true?
\begin{enumerate}
    \item [(A)] $X$ is a random variable with respect to $\mathcal{G}$, but $Y$ is not a random variable with respect to $\mathcal{G}$.
    \item [(B)] $Y$ is a random variable with respect to $\mathcal{G}$, but $X$ is not a random variable with respect to $\mathcal{G}$.
    \item [(C)] Neither $X$ nor $Y$ is a random variable with respect to $\mathcal{G}$.
    \item [(D)] Both $X$ and $Y$ are random variables with respect to $\mathcal{G}$.
\end{enumerate} \hfill (GATE ST 2023)\\
\solution
%\input{gate/ST/2023/14/main.tex}
	\item  A die is loaded in such a way that each odd number is twice as likely to occur as
each even number. Find $P(G)$, where $G$ is the event that a number greater than
3 occurs on a single roll of the die.
\\
\solution
		%\input{exemplar/11/16/3/5/main.tex}
	\item All the jacks, queens and kings are removed from a deck of 52 playing cards. The remaining cards are well shuffled and then one card is drawn at random. Giving ace a value 1 similar value for other cards, find the probability that the card has a value 
		\begin{enumerate}
			\item 7
			\item greater than 7
			\item less than 7
		\end{enumerate}
		%\input{exemplar/10/13/3/30/main.tex}
  \item A Lot consists of 48 mobile phones of which 42 are good, 3 have only minor defects and 3 have major defects.Varnika will buy a phone if it is good but the trader will only buy a mobile if it has no major defects. One phone is selected at random from the lot. What is the probability that it is
\begin{enumerate}
	\item acceptable to Varnika?
            \item acceptable to the trader?
\end{enumerate}
\solution
	%\input{exemplar/10/13/3/40/main.tex}
 \item A student says that if you throw a die, it will show up 1 or not 1. Therefore, the probability of getting 1 and the probability of getting 'not 1' each is equal to $\frac{1}{2}$. Is this correct? Give reasons.\\
 \solution
        %\input{exemplar/10/13/2/9/main.tex}
   \item Four candidates A, B, C, D have ap-
plied for the assignment to coach a school cricket
team. If A is twice as likely to be selected as B, and
B and C are given about the same chance of being
selected, while C is twice as likely to be selected
as D, what are the probabilities that
\begin{enumerate}
\item C will be selected?
\item A will not be selected?
\end{enumerate}
	%\input{exemplar/11/16/3/9/main.tex}
 \item A bag contain 24 balls of which $x$ balls are red, $2x$ are white and $3x$ are blue. A ball is selected at random, What is the probability that it is
\begin{enumerate}[label=\alph*)]
\item not red ?
\item white ?
\end{enumerate}
%\input{exemplar/10/13/3/41/main.tex}
If the letters of the word ASSASSINATION are arranged at random. Find the Probability that
\begin{enumerate}[label=(\alph*)]
\item Four $S's$ come consecutively in the word
\item Two  $I's$ and two $N's$ come together
\item All $A's$ are not coming together
\item No two $A's$ are coming together
\end{enumerate}
%\input{exemplar/11/16/3/14/main.tex}
	\item One urn contains two black balls (labelled B1 and B2) and one white ball. A
	second urn contains one black ball and two white balls (labelled W1 and W2).
	Suppose the following experiment is performed. One of the two urns is chosen
	at random. Next a ball is randomly chosen from the urn. Then a second ball is
	chosen at random from the same urn without replacing the first ball.
	
	\begin{enumerate}
	\item What is the probability that two black balls are chosen?
	
	\item What is the probability that two balls of opposite colour are chosen?
	\end{enumerate}
	\solution
	%\input{exemplar/11/16/3/12/main1.tex}
\end{enumerate}

	\item A bag contains 4 red and 4 black balls, another bag contains 2 red and 6 black balls. One of the two bags is selected at random and a ball is drawn from the bag which is found to be red. Find the probability that the ball is drawn from the first bag.
\\
\solution
		%\begin{table}[H]
	\centering
\begin{tabular}{|c|c|c|}
\hline
Random variable &Value &Definition\\ \hline
\multirow{3}{*}{X} &0 &Slips of Rs 1\\
&1 &Slips of Rs 5\\
&2 &Slips of Rs 13\\ \hline
\multirow{2}{*}{Y} &0 &Box A\\
&1 &Box B\\\hline
\end{tabular}
\caption{}
\label{tab:Distribution}
\end{table}
See \tabref{tab:Distribution}.
\begin{align}
p_{Y}\brak{k}= \begin{cases} 
      \frac{1}{3} & {k=0} \\
      \frac{2}{3 }& {k=1} 
   \end{cases}
   \\
p_{Y|X}\brak{0|0} = \frac{19}{25}\, 
p_{Y|X}\brak{0|1} = \frac{6}{25}\,
p_{Y|X}\brak{1|0} = \frac{45}{50}\,
p_{Y|X}\brak{1|2} = \frac{5}{50}
\end{align}
The desired probability is the probability that a slip drawn at random is marked other than Rs 1,
\begin{align}
&=1-p_X\brak{0}\\
&= p_X(1) + p_X(2)
\end{align}
Using Bayes theorem,
\begin{align}
&= p_Y\brak{0} \times \pr{Y=0 | X=1} + p_Y\brak{1} \times \pr{Y=1|X=2}\\
&=\frac{1}{3} \times \frac{6}{25} + \frac{2}{3} \times \frac{5}{50}\\
&=\frac{11}{75}
\end{align}

\newpage

%\tableofcontents

\bigskip

\renewcommand{\thefigure}{\theenumi}
\renewcommand{\thetable}{\theenumi}
%\renewcommand{\theequation}{\theenumi}

%\begin{abstract}
%%\boldmath
%In this letter, an algorithm for evaluating the exact analytical bit error rate  (BER)  for the piecewise linear (PL) combiner for  multiple relays is presented. Previous results were available only for upto three relays. The algorithm is unique in the sense that  the actual mathematical expressions, that are prohibitively large, need not be explicitly obtained. The diversity gain due to multiple relays is shown through plots of the analytical BER, well supported by simulations. 
%
%\end{abstract}
% IEEEtran.cls defaults to using nonbold math in the Abstract.
% This preserves the distinction between vectors and scalars. However,
% if the journal you are submitting to favors bold math in the abstract,
% then you can use LaTeX's standard command \boldmath at the very start
% of the abstract to achieve this. Many IEEE journals frown on math
% in the abstract anyway.

% Note that keywords are not normally used for peerreview papers.
%\begin{IEEEkeywords}
%Cooperative diversity, decode and forward, piecewise linear
%\end{IEEEkeywords}



% For peer review papers, you can put extra information on the cover
% page as needed:
% \ifCLASSOPTIONpeerreview
% \begin{center} \bfseries EDICS Category: 3-BBND \end{center}
% \fi
%
% For peerreview papers, this IEEEtran command inserts a page break and
% creates the second title. It will be ignored for other modes.
%\IEEEpeerreviewmaketitle




  \item
  Cards with numbers 2 to 101 are placed in a box. A card is selected at random.Find the probability that the card has
\begin{enumerate}[label=(\roman*)]
	\item an even number 
	\item a square number
\end{enumerate}
\solution
%\begin{table}[H]
	\centering
\begin{tabular}{|c|c|c|}
\hline
Random variable &Value &Definition\\ \hline
\multirow{3}{*}{X} &0 &Slips of Rs 1\\
&1 &Slips of Rs 5\\
&2 &Slips of Rs 13\\ \hline
\multirow{2}{*}{Y} &0 &Box A\\
&1 &Box B\\\hline
\end{tabular}
\caption{}
\label{tab:Distribution}
\end{table}
See \tabref{tab:Distribution}.
\begin{align}
p_{Y}\brak{k}= \begin{cases} 
      \frac{1}{3} & {k=0} \\
      \frac{2}{3 }& {k=1} 
   \end{cases}
   \\
p_{Y|X}\brak{0|0} = \frac{19}{25}\, 
p_{Y|X}\brak{0|1} = \frac{6}{25}\,
p_{Y|X}\brak{1|0} = \frac{45}{50}\,
p_{Y|X}\brak{1|2} = \frac{5}{50}
\end{align}
The desired probability is the probability that a slip drawn at random is marked other than Rs 1,
\begin{align}
&=1-p_X\brak{0}\\
&= p_X(1) + p_X(2)
\end{align}
Using Bayes theorem,
\begin{align}
&= p_Y\brak{0} \times \pr{Y=0 | X=1} + p_Y\brak{1} \times \pr{Y=1|X=2}\\
&=\frac{1}{3} \times \frac{6}{25} + \frac{2}{3} \times \frac{5}{50}\\
&=\frac{11}{75}
\end{align}

\newpage

%\tableofcontents

\bigskip

\renewcommand{\thefigure}{\theenumi}
\renewcommand{\thetable}{\theenumi}
%\renewcommand{\theequation}{\theenumi}

%\begin{abstract}
%%\boldmath
%In this letter, an algorithm for evaluating the exact analytical bit error rate  (BER)  for the piecewise linear (PL) combiner for  multiple relays is presented. Previous results were available only for upto three relays. The algorithm is unique in the sense that  the actual mathematical expressions, that are prohibitively large, need not be explicitly obtained. The diversity gain due to multiple relays is shown through plots of the analytical BER, well supported by simulations. 
%
%\end{abstract}
% IEEEtran.cls defaults to using nonbold math in the Abstract.
% This preserves the distinction between vectors and scalars. However,
% if the journal you are submitting to favors bold math in the abstract,
% then you can use LaTeX's standard command \boldmath at the very start
% of the abstract to achieve this. Many IEEE journals frown on math
% in the abstract anyway.

% Note that keywords are not normally used for peerreview papers.
%\begin{IEEEkeywords}
%Cooperative diversity, decode and forward, piecewise linear
%\end{IEEEkeywords}



% For peer review papers, you can put extra information on the cover
% page as needed:
% \ifCLASSOPTIONpeerreview
% \begin{center} \bfseries EDICS Category: 3-BBND \end{center}
% \fi
%
% For peerreview papers, this IEEEtran command inserts a page break and
% creates the second title. It will be ignored for other modes.
%\IEEEpeerreviewmaketitle




\item
The king, queen and jack of clubs are removed from a deck of 52 playing cards and then well shuffled. Now one card is drawn at random from the remaining cards.  Determine the probability that the card is
\begin{enumerate}[label=(\roman*)]
\item a club
\item 10 of hearts
\end{enumerate}
\solution
%\begin{table}[H]
	\centering
\begin{tabular}{|c|c|c|}
\hline
Random variable &Value &Definition\\ \hline
\multirow{3}{*}{X} &0 &Slips of Rs 1\\
&1 &Slips of Rs 5\\
&2 &Slips of Rs 13\\ \hline
\multirow{2}{*}{Y} &0 &Box A\\
&1 &Box B\\\hline
\end{tabular}
\caption{}
\label{tab:Distribution}
\end{table}
See \tabref{tab:Distribution}.
\begin{align}
p_{Y}\brak{k}= \begin{cases} 
      \frac{1}{3} & {k=0} \\
      \frac{2}{3 }& {k=1} 
   \end{cases}
   \\
p_{Y|X}\brak{0|0} = \frac{19}{25}\, 
p_{Y|X}\brak{0|1} = \frac{6}{25}\,
p_{Y|X}\brak{1|0} = \frac{45}{50}\,
p_{Y|X}\brak{1|2} = \frac{5}{50}
\end{align}
The desired probability is the probability that a slip drawn at random is marked other than Rs 1,
\begin{align}
&=1-p_X\brak{0}\\
&= p_X(1) + p_X(2)
\end{align}
Using Bayes theorem,
\begin{align}
&= p_Y\brak{0} \times \pr{Y=0 | X=1} + p_Y\brak{1} \times \pr{Y=1|X=2}\\
&=\frac{1}{3} \times \frac{6}{25} + \frac{2}{3} \times \frac{5}{50}\\
&=\frac{11}{75}
\end{align}

\newpage

%\tableofcontents

\bigskip

\renewcommand{\thefigure}{\theenumi}
\renewcommand{\thetable}{\theenumi}
%\renewcommand{\theequation}{\theenumi}

%\begin{abstract}
%%\boldmath
%In this letter, an algorithm for evaluating the exact analytical bit error rate  (BER)  for the piecewise linear (PL) combiner for  multiple relays is presented. Previous results were available only for upto three relays. The algorithm is unique in the sense that  the actual mathematical expressions, that are prohibitively large, need not be explicitly obtained. The diversity gain due to multiple relays is shown through plots of the analytical BER, well supported by simulations. 
%
%\end{abstract}
% IEEEtran.cls defaults to using nonbold math in the Abstract.
% This preserves the distinction between vectors and scalars. However,
% if the journal you are submitting to favors bold math in the abstract,
% then you can use LaTeX's standard command \boldmath at the very start
% of the abstract to achieve this. Many IEEE journals frown on math
% in the abstract anyway.

% Note that keywords are not normally used for peerreview papers.
%\begin{IEEEkeywords}
%Cooperative diversity, decode and forward, piecewise linear
%\end{IEEEkeywords}



% For peer review papers, you can put extra information on the cover
% page as needed:
% \ifCLASSOPTIONpeerreview
% \begin{center} \bfseries EDICS Category: 3-BBND \end{center}
% \fi
%
% For peerreview papers, this IEEEtran command inserts a page break and
% creates the second title. It will be ignored for other modes.
%\IEEEpeerreviewmaketitle




\item A team of medical students doing their internship have to assist during surgeries
at a city hospital. The probabilities of surgeries rated as very complex, complex,
routine, simple or very simple are respectively, 0.15, 0.20, 0.31, 0.26, .08. Find
the probabilities that a particular surgery will be rated
\begin{enumerate}
	\item complex or very complex;
	\item neither very complex nor very simple;
	\item routine or complex
	\item routine or simple
\end{enumerate}
\solution
%\begin{table}[H]
	\centering
\begin{tabular}{|c|c|c|}
\hline
Random variable &Value &Definition\\ \hline
\multirow{3}{*}{X} &0 &Slips of Rs 1\\
&1 &Slips of Rs 5\\
&2 &Slips of Rs 13\\ \hline
\multirow{2}{*}{Y} &0 &Box A\\
&1 &Box B\\\hline
\end{tabular}
\caption{}
\label{tab:Distribution}
\end{table}
See \tabref{tab:Distribution}.
\begin{align}
p_{Y}\brak{k}= \begin{cases} 
      \frac{1}{3} & {k=0} \\
      \frac{2}{3 }& {k=1} 
   \end{cases}
   \\
p_{Y|X}\brak{0|0} = \frac{19}{25}\, 
p_{Y|X}\brak{0|1} = \frac{6}{25}\,
p_{Y|X}\brak{1|0} = \frac{45}{50}\,
p_{Y|X}\brak{1|2} = \frac{5}{50}
\end{align}
The desired probability is the probability that a slip drawn at random is marked other than Rs 1,
\begin{align}
&=1-p_X\brak{0}\\
&= p_X(1) + p_X(2)
\end{align}
Using Bayes theorem,
\begin{align}
&= p_Y\brak{0} \times \pr{Y=0 | X=1} + p_Y\brak{1} \times \pr{Y=1|X=2}\\
&=\frac{1}{3} \times \frac{6}{25} + \frac{2}{3} \times \frac{5}{50}\\
&=\frac{11}{75}
\end{align}

\newpage

%\tableofcontents

\bigskip

\renewcommand{\thefigure}{\theenumi}
\renewcommand{\thetable}{\theenumi}
%\renewcommand{\theequation}{\theenumi}

%\begin{abstract}
%%\boldmath
%In this letter, an algorithm for evaluating the exact analytical bit error rate  (BER)  for the piecewise linear (PL) combiner for  multiple relays is presented. Previous results were available only for upto three relays. The algorithm is unique in the sense that  the actual mathematical expressions, that are prohibitively large, need not be explicitly obtained. The diversity gain due to multiple relays is shown through plots of the analytical BER, well supported by simulations. 
%
%\end{abstract}
% IEEEtran.cls defaults to using nonbold math in the Abstract.
% This preserves the distinction between vectors and scalars. However,
% if the journal you are submitting to favors bold math in the abstract,
% then you can use LaTeX's standard command \boldmath at the very start
% of the abstract to achieve this. Many IEEE journals frown on math
% in the abstract anyway.

% Note that keywords are not normally used for peerreview papers.
%\begin{IEEEkeywords}
%Cooperative diversity, decode and forward, piecewise linear
%\end{IEEEkeywords}



% For peer review papers, you can put extra information on the cover
% page as needed:
% \ifCLASSOPTIONpeerreview
% \begin{center} \bfseries EDICS Category: 3-BBND \end{center}
% \fi
%
% For peerreview papers, this IEEEtran command inserts a page break and
% creates the second title. It will be ignored for other modes.
%\IEEEpeerreviewmaketitle




\item A card is selected from a pack of 52 cards.
\begin{enumerate}[label=(\alph*)]
    \item How many points are there in the sample space?
    \item Calculate the probability that the card is an ace of spades.
    \item Calculate the probability that the card is (i) an ace and (ii) black card.
\end{enumerate}
\solution
%Let $X$ be an bernoulli rv defined as in \tabref{tab:exemplar/11/16/3/26}.  Then, 
\begin{equation}
    p =
        \frac{4}{11} 
\end{equation}
\begin{table}[H]
	\centering
	\input{exemplar/11/16/3/26/tables/Table2.tex}
	\caption{}
        \label{tab:exemplar/11/16/3/26}
\end{table}

\item The probability that a non leap year selected at random will contain 53 sundays.
\\
\solution
%\begin{table}[H]
	\centering
\begin{tabular}{|c|c|c|}
\hline
Random variable &Value &Definition\\ \hline
\multirow{3}{*}{X} &0 &Slips of Rs 1\\
&1 &Slips of Rs 5\\
&2 &Slips of Rs 13\\ \hline
\multirow{2}{*}{Y} &0 &Box A\\
&1 &Box B\\\hline
\end{tabular}
\caption{}
\label{tab:Distribution}
\end{table}
See \tabref{tab:Distribution}.
\begin{align}
p_{Y}\brak{k}= \begin{cases} 
      \frac{1}{3} & {k=0} \\
      \frac{2}{3 }& {k=1} 
   \end{cases}
   \\
p_{Y|X}\brak{0|0} = \frac{19}{25}\, 
p_{Y|X}\brak{0|1} = \frac{6}{25}\,
p_{Y|X}\brak{1|0} = \frac{45}{50}\,
p_{Y|X}\brak{1|2} = \frac{5}{50}
\end{align}
The desired probability is the probability that a slip drawn at random is marked other than Rs 1,
\begin{align}
&=1-p_X\brak{0}\\
&= p_X(1) + p_X(2)
\end{align}
Using Bayes theorem,
\begin{align}
&= p_Y\brak{0} \times \pr{Y=0 | X=1} + p_Y\brak{1} \times \pr{Y=1|X=2}\\
&=\frac{1}{3} \times \frac{6}{25} + \frac{2}{3} \times \frac{5}{50}\\
&=\frac{11}{75}
\end{align}

\newpage

%\tableofcontents

\bigskip

\renewcommand{\thefigure}{\theenumi}
\renewcommand{\thetable}{\theenumi}
%\renewcommand{\theequation}{\theenumi}

%\begin{abstract}
%%\boldmath
%In this letter, an algorithm for evaluating the exact analytical bit error rate  (BER)  for the piecewise linear (PL) combiner for  multiple relays is presented. Previous results were available only for upto three relays. The algorithm is unique in the sense that  the actual mathematical expressions, that are prohibitively large, need not be explicitly obtained. The diversity gain due to multiple relays is shown through plots of the analytical BER, well supported by simulations. 
%
%\end{abstract}
% IEEEtran.cls defaults to using nonbold math in the Abstract.
% This preserves the distinction between vectors and scalars. However,
% if the journal you are submitting to favors bold math in the abstract,
% then you can use LaTeX's standard command \boldmath at the very start
% of the abstract to achieve this. Many IEEE journals frown on math
% in the abstract anyway.

% Note that keywords are not normally used for peerreview papers.
%\begin{IEEEkeywords}
%Cooperative diversity, decode and forward, piecewise linear
%\end{IEEEkeywords}



% For peer review papers, you can put extra information on the cover
% page as needed:
% \ifCLASSOPTIONpeerreview
% \begin{center} \bfseries EDICS Category: 3-BBND \end{center}
% \fi
%
% For peerreview papers, this IEEEtran command inserts a page break and
% creates the second title. It will be ignored for other modes.
%\IEEEpeerreviewmaketitle




\item One of the four persons John, Rita, Aslam or Gurpreet will be promoted next
month. Consequently the sample space consists of four elementary outcomes
S = {John promoted, Rita promoted, Aslam promoted, Gurpreet promoted}
You are told that the chances of John’s promotion is same as that of Gurpreet,
Rita’s chances of promotion are twice as likely as Johns. Aslam’s chances are
four times that of John.
\begin{enumerate}
	\item Determine
	\begin{enumerate}
		\item P (John promoted)
		\item P (Rita promoted)
		\item P (Aslam promoted)
		\item P (Gurpreet promoted)
	\end{enumerate}
	\item If A = {John promoted or Gurpreet promoted}, find P (A).
\end{enumerate}
\solution
%\begin{table}[H]
	\centering
\begin{tabular}{|c|c|c|}
\hline
Random variable &Value &Definition\\ \hline
\multirow{3}{*}{X} &0 &Slips of Rs 1\\
&1 &Slips of Rs 5\\
&2 &Slips of Rs 13\\ \hline
\multirow{2}{*}{Y} &0 &Box A\\
&1 &Box B\\\hline
\end{tabular}
\caption{}
\label{tab:Distribution}
\end{table}
See \tabref{tab:Distribution}.
\begin{align}
p_{Y}\brak{k}= \begin{cases} 
      \frac{1}{3} & {k=0} \\
      \frac{2}{3 }& {k=1} 
   \end{cases}
   \\
p_{Y|X}\brak{0|0} = \frac{19}{25}\, 
p_{Y|X}\brak{0|1} = \frac{6}{25}\,
p_{Y|X}\brak{1|0} = \frac{45}{50}\,
p_{Y|X}\brak{1|2} = \frac{5}{50}
\end{align}
The desired probability is the probability that a slip drawn at random is marked other than Rs 1,
\begin{align}
&=1-p_X\brak{0}\\
&= p_X(1) + p_X(2)
\end{align}
Using Bayes theorem,
\begin{align}
&= p_Y\brak{0} \times \pr{Y=0 | X=1} + p_Y\brak{1} \times \pr{Y=1|X=2}\\
&=\frac{1}{3} \times \frac{6}{25} + \frac{2}{3} \times \frac{5}{50}\\
&=\frac{11}{75}
\end{align}

\newpage

%\tableofcontents

\bigskip

\renewcommand{\thefigure}{\theenumi}
\renewcommand{\thetable}{\theenumi}
%\renewcommand{\theequation}{\theenumi}

%\begin{abstract}
%%\boldmath
%In this letter, an algorithm for evaluating the exact analytical bit error rate  (BER)  for the piecewise linear (PL) combiner for  multiple relays is presented. Previous results were available only for upto three relays. The algorithm is unique in the sense that  the actual mathematical expressions, that are prohibitively large, need not be explicitly obtained. The diversity gain due to multiple relays is shown through plots of the analytical BER, well supported by simulations. 
%
%\end{abstract}
% IEEEtran.cls defaults to using nonbold math in the Abstract.
% This preserves the distinction between vectors and scalars. However,
% if the journal you are submitting to favors bold math in the abstract,
% then you can use LaTeX's standard command \boldmath at the very start
% of the abstract to achieve this. Many IEEE journals frown on math
% in the abstract anyway.

% Note that keywords are not normally used for peerreview papers.
%\begin{IEEEkeywords}
%Cooperative diversity, decode and forward, piecewise linear
%\end{IEEEkeywords}



% For peer review papers, you can put extra information on the cover
% page as needed:
% \ifCLASSOPTIONpeerreview
% \begin{center} \bfseries EDICS Category: 3-BBND \end{center}
% \fi
%
% For peerreview papers, this IEEEtran command inserts a page break and
% creates the second title. It will be ignored for other modes.
%\IEEEpeerreviewmaketitle




\item A card is drawn from a deck of 52 cards. Find the probability of getting a king or a heart or a red card.\\
\solution
%\begin{table}[H]
	\centering
\begin{tabular}{|c|c|c|}
\hline
Random variable &Value &Definition\\ \hline
\multirow{3}{*}{X} &0 &Slips of Rs 1\\
&1 &Slips of Rs 5\\
&2 &Slips of Rs 13\\ \hline
\multirow{2}{*}{Y} &0 &Box A\\
&1 &Box B\\\hline
\end{tabular}
\caption{}
\label{tab:Distribution}
\end{table}
See \tabref{tab:Distribution}.
\begin{align}
p_{Y}\brak{k}= \begin{cases} 
      \frac{1}{3} & {k=0} \\
      \frac{2}{3 }& {k=1} 
   \end{cases}
   \\
p_{Y|X}\brak{0|0} = \frac{19}{25}\, 
p_{Y|X}\brak{0|1} = \frac{6}{25}\,
p_{Y|X}\brak{1|0} = \frac{45}{50}\,
p_{Y|X}\brak{1|2} = \frac{5}{50}
\end{align}
The desired probability is the probability that a slip drawn at random is marked other than Rs 1,
\begin{align}
&=1-p_X\brak{0}\\
&= p_X(1) + p_X(2)
\end{align}
Using Bayes theorem,
\begin{align}
&= p_Y\brak{0} \times \pr{Y=0 | X=1} + p_Y\brak{1} \times \pr{Y=1|X=2}\\
&=\frac{1}{3} \times \frac{6}{25} + \frac{2}{3} \times \frac{5}{50}\\
&=\frac{11}{75}
\end{align}

\newpage

%\tableofcontents

\bigskip

\renewcommand{\thefigure}{\theenumi}
\renewcommand{\thetable}{\theenumi}
%\renewcommand{\theequation}{\theenumi}

%\begin{abstract}
%%\boldmath
%In this letter, an algorithm for evaluating the exact analytical bit error rate  (BER)  for the piecewise linear (PL) combiner for  multiple relays is presented. Previous results were available only for upto three relays. The algorithm is unique in the sense that  the actual mathematical expressions, that are prohibitively large, need not be explicitly obtained. The diversity gain due to multiple relays is shown through plots of the analytical BER, well supported by simulations. 
%
%\end{abstract}
% IEEEtran.cls defaults to using nonbold math in the Abstract.
% This preserves the distinction between vectors and scalars. However,
% if the journal you are submitting to favors bold math in the abstract,
% then you can use LaTeX's standard command \boldmath at the very start
% of the abstract to achieve this. Many IEEE journals frown on math
% in the abstract anyway.

% Note that keywords are not normally used for peerreview papers.
%\begin{IEEEkeywords}
%Cooperative diversity, decode and forward, piecewise linear
%\end{IEEEkeywords}



% For peer review papers, you can put extra information on the cover
% page as needed:
% \ifCLASSOPTIONpeerreview
% \begin{center} \bfseries EDICS Category: 3-BBND \end{center}
% \fi
%
% For peerreview papers, this IEEEtran command inserts a page break and
% creates the second title. It will be ignored for other modes.
%\IEEEpeerreviewmaketitle




\item The probability that a student will pass his examination is 0.73, the probability of
the student getting a compartment is 0.13, and the probability that the student will
either pass or get compartment is 0.96. State True or False.\\
\solution
%\begin{table}[H]
	\centering
\begin{tabular}{|c|c|c|}
\hline
Random variable &Value &Definition\\ \hline
\multirow{3}{*}{X} &0 &Slips of Rs 1\\
&1 &Slips of Rs 5\\
&2 &Slips of Rs 13\\ \hline
\multirow{2}{*}{Y} &0 &Box A\\
&1 &Box B\\\hline
\end{tabular}
\caption{}
\label{tab:Distribution}
\end{table}
See \tabref{tab:Distribution}.
\begin{align}
p_{Y}\brak{k}= \begin{cases} 
      \frac{1}{3} & {k=0} \\
      \frac{2}{3 }& {k=1} 
   \end{cases}
   \\
p_{Y|X}\brak{0|0} = \frac{19}{25}\, 
p_{Y|X}\brak{0|1} = \frac{6}{25}\,
p_{Y|X}\brak{1|0} = \frac{45}{50}\,
p_{Y|X}\brak{1|2} = \frac{5}{50}
\end{align}
The desired probability is the probability that a slip drawn at random is marked other than Rs 1,
\begin{align}
&=1-p_X\brak{0}\\
&= p_X(1) + p_X(2)
\end{align}
Using Bayes theorem,
\begin{align}
&= p_Y\brak{0} \times \pr{Y=0 | X=1} + p_Y\brak{1} \times \pr{Y=1|X=2}\\
&=\frac{1}{3} \times \frac{6}{25} + \frac{2}{3} \times \frac{5}{50}\\
&=\frac{11}{75}
\end{align}

\newpage

%\tableofcontents

\bigskip

\renewcommand{\thefigure}{\theenumi}
\renewcommand{\thetable}{\theenumi}
%\renewcommand{\theequation}{\theenumi}

%\begin{abstract}
%%\boldmath
%In this letter, an algorithm for evaluating the exact analytical bit error rate  (BER)  for the piecewise linear (PL) combiner for  multiple relays is presented. Previous results were available only for upto three relays. The algorithm is unique in the sense that  the actual mathematical expressions, that are prohibitively large, need not be explicitly obtained. The diversity gain due to multiple relays is shown through plots of the analytical BER, well supported by simulations. 
%
%\end{abstract}
% IEEEtran.cls defaults to using nonbold math in the Abstract.
% This preserves the distinction between vectors and scalars. However,
% if the journal you are submitting to favors bold math in the abstract,
% then you can use LaTeX's standard command \boldmath at the very start
% of the abstract to achieve this. Many IEEE journals frown on math
% in the abstract anyway.

% Note that keywords are not normally used for peerreview papers.
%\begin{IEEEkeywords}
%Cooperative diversity, decode and forward, piecewise linear
%\end{IEEEkeywords}



% For peer review papers, you can put extra information on the cover
% page as needed:
% \ifCLASSOPTIONpeerreview
% \begin{center} \bfseries EDICS Category: 3-BBND \end{center}
% \fi
%
% For peerreview papers, this IEEEtran command inserts a page break and
% creates the second title. It will be ignored for other modes.
%\IEEEpeerreviewmaketitle




\item A card is selected from a pack of 52 cards\\
\begin{enumerate}[label=(\alph*)]
\item How many points are there in the sample space?
\item Calculate the probability that the cards is an ace of spades.
\item Calculate the probability that the card is (i) an ace (ii)black card.\\
\end{enumerate}
%\input{ncert/11/16/3/4_1/Prob_4.tex}
\item In a non-leap year, the probability of having 53 tuesdays or 53 wednesdays is\\
\solution
%A non-leap year has a total of 365 days, and a week has 7 days.\\
So it can be expressed as 
\begin{align}
365\text{days} &=52\times 7+1 \text{day}
\end{align}
$\implies$ 52 tuesdays or wednesdays\\
Random variable X denotes the days of a week
\begin{align}
p_X\brak{k}&=\frac{1}{7}; \quad \brak{1<k<7}
\end{align}
So the probability of extra day being tuesday or wednesday is
\begin{align}
p_X\brak{3}+p_X\brak{4}&=\frac{1}{7}+\frac{1}{7}=\frac{2}{7}
\end{align}



\item There are 1000 sealed envelopes in a box, 10 of them contain a cash prize of
Rs 100 each, 100 of them contain a cash prize of Rs 50 each and 200 of them
contain a cash prize of Rs 10 each and rest do not contain any cash prize. If they
are well shuffled and an envelope is picked up out, what is the probability that it
contains no cash prize?\\
\solution
%\begin{table}[H]
	\centering
\begin{tabular}{|c|c|c|}
\hline
Random variable &Value &Definition\\ \hline
\multirow{3}{*}{X} &0 &Slips of Rs 1\\
&1 &Slips of Rs 5\\
&2 &Slips of Rs 13\\ \hline
\multirow{2}{*}{Y} &0 &Box A\\
&1 &Box B\\\hline
\end{tabular}
\caption{}
\label{tab:Distribution}
\end{table}
See \tabref{tab:Distribution}.
\begin{align}
p_{Y}\brak{k}= \begin{cases} 
      \frac{1}{3} & {k=0} \\
      \frac{2}{3 }& {k=1} 
   \end{cases}
   \\
p_{Y|X}\brak{0|0} = \frac{19}{25}\, 
p_{Y|X}\brak{0|1} = \frac{6}{25}\,
p_{Y|X}\brak{1|0} = \frac{45}{50}\,
p_{Y|X}\brak{1|2} = \frac{5}{50}
\end{align}
The desired probability is the probability that a slip drawn at random is marked other than Rs 1,
\begin{align}
&=1-p_X\brak{0}\\
&= p_X(1) + p_X(2)
\end{align}
Using Bayes theorem,
\begin{align}
&= p_Y\brak{0} \times \pr{Y=0 | X=1} + p_Y\brak{1} \times \pr{Y=1|X=2}\\
&=\frac{1}{3} \times \frac{6}{25} + \frac{2}{3} \times \frac{5}{50}\\
&=\frac{11}{75}
\end{align}

\newpage

%\tableofcontents

\bigskip

\renewcommand{\thefigure}{\theenumi}
\renewcommand{\thetable}{\theenumi}
%\renewcommand{\theequation}{\theenumi}

%\begin{abstract}
%%\boldmath
%In this letter, an algorithm for evaluating the exact analytical bit error rate  (BER)  for the piecewise linear (PL) combiner for  multiple relays is presented. Previous results were available only for upto three relays. The algorithm is unique in the sense that  the actual mathematical expressions, that are prohibitively large, need not be explicitly obtained. The diversity gain due to multiple relays is shown through plots of the analytical BER, well supported by simulations. 
%
%\end{abstract}
% IEEEtran.cls defaults to using nonbold math in the Abstract.
% This preserves the distinction between vectors and scalars. However,
% if the journal you are submitting to favors bold math in the abstract,
% then you can use LaTeX's standard command \boldmath at the very start
% of the abstract to achieve this. Many IEEE journals frown on math
% in the abstract anyway.

% Note that keywords are not normally used for peerreview papers.
%\begin{IEEEkeywords}
%Cooperative diversity, decode and forward, piecewise linear
%\end{IEEEkeywords}



% For peer review papers, you can put extra information on the cover
% page as needed:
% \ifCLASSOPTIONpeerreview
% \begin{center} \bfseries EDICS Category: 3-BBND \end{center}
% \fi
%
% For peerreview papers, this IEEEtran command inserts a page break and
% creates the second title. It will be ignored for other modes.
%\IEEEpeerreviewmaketitle




\item 
A die is thrown and a card is selected at random from a deck of 52 playing cards. The probability of getting an even number on the die and a spade card.\\
\solution
%\begin{table}[H]
	\centering
\begin{tabular}{|c|c|c|}
\hline
Random variable &Value &Definition\\ \hline
\multirow{3}{*}{X} &0 &Slips of Rs 1\\
&1 &Slips of Rs 5\\
&2 &Slips of Rs 13\\ \hline
\multirow{2}{*}{Y} &0 &Box A\\
&1 &Box B\\\hline
\end{tabular}
\caption{}
\label{tab:Distribution}
\end{table}
See \tabref{tab:Distribution}.
\begin{align}
p_{Y}\brak{k}= \begin{cases} 
      \frac{1}{3} & {k=0} \\
      \frac{2}{3 }& {k=1} 
   \end{cases}
   \\
p_{Y|X}\brak{0|0} = \frac{19}{25}\, 
p_{Y|X}\brak{0|1} = \frac{6}{25}\,
p_{Y|X}\brak{1|0} = \frac{45}{50}\,
p_{Y|X}\brak{1|2} = \frac{5}{50}
\end{align}
The desired probability is the probability that a slip drawn at random is marked other than Rs 1,
\begin{align}
&=1-p_X\brak{0}\\
&= p_X(1) + p_X(2)
\end{align}
Using Bayes theorem,
\begin{align}
&= p_Y\brak{0} \times \pr{Y=0 | X=1} + p_Y\brak{1} \times \pr{Y=1|X=2}\\
&=\frac{1}{3} \times \frac{6}{25} + \frac{2}{3} \times \frac{5}{50}\\
&=\frac{11}{75}
\end{align}

\newpage

%\tableofcontents

\bigskip

\renewcommand{\thefigure}{\theenumi}
\renewcommand{\thetable}{\theenumi}
%\renewcommand{\theequation}{\theenumi}

%\begin{abstract}
%%\boldmath
%In this letter, an algorithm for evaluating the exact analytical bit error rate  (BER)  for the piecewise linear (PL) combiner for  multiple relays is presented. Previous results were available only for upto three relays. The algorithm is unique in the sense that  the actual mathematical expressions, that are prohibitively large, need not be explicitly obtained. The diversity gain due to multiple relays is shown through plots of the analytical BER, well supported by simulations. 
%
%\end{abstract}
% IEEEtran.cls defaults to using nonbold math in the Abstract.
% This preserves the distinction between vectors and scalars. However,
% if the journal you are submitting to favors bold math in the abstract,
% then you can use LaTeX's standard command \boldmath at the very start
% of the abstract to achieve this. Many IEEE journals frown on math
% in the abstract anyway.

% Note that keywords are not normally used for peerreview papers.
%\begin{IEEEkeywords}
%Cooperative diversity, decode and forward, piecewise linear
%\end{IEEEkeywords}



% For peer review papers, you can put extra information on the cover
% page as needed:
% \ifCLASSOPTIONpeerreview
% \begin{center} \bfseries EDICS Category: 3-BBND \end{center}
% \fi
%
% For peerreview papers, this IEEEtran command inserts a page break and
% creates the second title. It will be ignored for other modes.
%\IEEEpeerreviewmaketitle




\item
If 4-digit numbers greater than 5,000 are randomly formed from the digits 0, 1, 3, 5, and 7, what is the probability of forming a number divisible by 5 when:
\begin{enumerate}
    \item The digits are repeated?
    \item The repetition of digits is not allowed?
\end{enumerate}
\solution
%\begin{table}[H]
	\centering
\begin{tabular}{|c|c|c|}
\hline
Random variable &Value &Definition\\ \hline
\multirow{3}{*}{X} &0 &Slips of Rs 1\\
&1 &Slips of Rs 5\\
&2 &Slips of Rs 13\\ \hline
\multirow{2}{*}{Y} &0 &Box A\\
&1 &Box B\\\hline
\end{tabular}
\caption{}
\label{tab:Distribution}
\end{table}
See \tabref{tab:Distribution}.
\begin{align}
p_{Y}\brak{k}= \begin{cases} 
      \frac{1}{3} & {k=0} \\
      \frac{2}{3 }& {k=1} 
   \end{cases}
   \\
p_{Y|X}\brak{0|0} = \frac{19}{25}\, 
p_{Y|X}\brak{0|1} = \frac{6}{25}\,
p_{Y|X}\brak{1|0} = \frac{45}{50}\,
p_{Y|X}\brak{1|2} = \frac{5}{50}
\end{align}
The desired probability is the probability that a slip drawn at random is marked other than Rs 1,
\begin{align}
&=1-p_X\brak{0}\\
&= p_X(1) + p_X(2)
\end{align}
Using Bayes theorem,
\begin{align}
&= p_Y\brak{0} \times \pr{Y=0 | X=1} + p_Y\brak{1} \times \pr{Y=1|X=2}\\
&=\frac{1}{3} \times \frac{6}{25} + \frac{2}{3} \times \frac{5}{50}\\
&=\frac{11}{75}
\end{align}

\newpage

%\tableofcontents

\bigskip

\renewcommand{\thefigure}{\theenumi}
\renewcommand{\thetable}{\theenumi}
%\renewcommand{\theequation}{\theenumi}

%\begin{abstract}
%%\boldmath
%In this letter, an algorithm for evaluating the exact analytical bit error rate  (BER)  for the piecewise linear (PL) combiner for  multiple relays is presented. Previous results were available only for upto three relays. The algorithm is unique in the sense that  the actual mathematical expressions, that are prohibitively large, need not be explicitly obtained. The diversity gain due to multiple relays is shown through plots of the analytical BER, well supported by simulations. 
%
%\end{abstract}
% IEEEtran.cls defaults to using nonbold math in the Abstract.
% This preserves the distinction between vectors and scalars. However,
% if the journal you are submitting to favors bold math in the abstract,
% then you can use LaTeX's standard command \boldmath at the very start
% of the abstract to achieve this. Many IEEE journals frown on math
% in the abstract anyway.

% Note that keywords are not normally used for peerreview papers.
%\begin{IEEEkeywords}
%Cooperative diversity, decode and forward, piecewise linear
%\end{IEEEkeywords}



% For peer review papers, you can put extra information on the cover
% page as needed:
% \ifCLASSOPTIONpeerreview
% \begin{center} \bfseries EDICS Category: 3-BBND \end{center}
% \fi
%
% For peerreview papers, this IEEEtran command inserts a page break and
% creates the second title. It will be ignored for other modes.
%\IEEEpeerreviewmaketitle




\item Consider the probability space $\brak{\Omega, \mathcal{G}, P}$ where $\Omega = [0,2]$ and $\mathcal{G} = \cbrak{\phi, \Omega, [0,1], (1,2]}$. Let $X$ and $Y$ be two functions on $\Omega$ defined as
\begin{align*}
    X(\omega) = 
    \begin{cases}
        1 & \text{if }\omega \in [0, 1]\\
        2 & \text{if }\omega \in (1, 2]
    \end{cases}
\end{align*}
and
\begin{align*}
    Y(\omega) = 
    \begin{cases}
        2 & \text{if }\omega \in [0, 1.5]\\
        3 & \text{if }\omega \in (1.5, 2].
    \end{cases}
\end{align*}
Then which one of the following statements is true?
\begin{enumerate}
    \item [(A)] $X$ is a random variable with respect to $\mathcal{G}$, but $Y$ is not a random variable with respect to $\mathcal{G}$.
    \item [(B)] $Y$ is a random variable with respect to $\mathcal{G}$, but $X$ is not a random variable with respect to $\mathcal{G}$.
    \item [(C)] Neither $X$ nor $Y$ is a random variable with respect to $\mathcal{G}$.
    \item [(D)] Both $X$ and $Y$ are random variables with respect to $\mathcal{G}$.
\end{enumerate} \hfill (GATE ST 2023)\\
\solution
%\begin{table}[H]
	\centering
\begin{tabular}{|c|c|c|}
\hline
Random variable &Value &Definition\\ \hline
\multirow{3}{*}{X} &0 &Slips of Rs 1\\
&1 &Slips of Rs 5\\
&2 &Slips of Rs 13\\ \hline
\multirow{2}{*}{Y} &0 &Box A\\
&1 &Box B\\\hline
\end{tabular}
\caption{}
\label{tab:Distribution}
\end{table}
See \tabref{tab:Distribution}.
\begin{align}
p_{Y}\brak{k}= \begin{cases} 
      \frac{1}{3} & {k=0} \\
      \frac{2}{3 }& {k=1} 
   \end{cases}
   \\
p_{Y|X}\brak{0|0} = \frac{19}{25}\, 
p_{Y|X}\brak{0|1} = \frac{6}{25}\,
p_{Y|X}\brak{1|0} = \frac{45}{50}\,
p_{Y|X}\brak{1|2} = \frac{5}{50}
\end{align}
The desired probability is the probability that a slip drawn at random is marked other than Rs 1,
\begin{align}
&=1-p_X\brak{0}\\
&= p_X(1) + p_X(2)
\end{align}
Using Bayes theorem,
\begin{align}
&= p_Y\brak{0} \times \pr{Y=0 | X=1} + p_Y\brak{1} \times \pr{Y=1|X=2}\\
&=\frac{1}{3} \times \frac{6}{25} + \frac{2}{3} \times \frac{5}{50}\\
&=\frac{11}{75}
\end{align}

\newpage

%\tableofcontents

\bigskip

\renewcommand{\thefigure}{\theenumi}
\renewcommand{\thetable}{\theenumi}
%\renewcommand{\theequation}{\theenumi}

%\begin{abstract}
%%\boldmath
%In this letter, an algorithm for evaluating the exact analytical bit error rate  (BER)  for the piecewise linear (PL) combiner for  multiple relays is presented. Previous results were available only for upto three relays. The algorithm is unique in the sense that  the actual mathematical expressions, that are prohibitively large, need not be explicitly obtained. The diversity gain due to multiple relays is shown through plots of the analytical BER, well supported by simulations. 
%
%\end{abstract}
% IEEEtran.cls defaults to using nonbold math in the Abstract.
% This preserves the distinction between vectors and scalars. However,
% if the journal you are submitting to favors bold math in the abstract,
% then you can use LaTeX's standard command \boldmath at the very start
% of the abstract to achieve this. Many IEEE journals frown on math
% in the abstract anyway.

% Note that keywords are not normally used for peerreview papers.
%\begin{IEEEkeywords}
%Cooperative diversity, decode and forward, piecewise linear
%\end{IEEEkeywords}



% For peer review papers, you can put extra information on the cover
% page as needed:
% \ifCLASSOPTIONpeerreview
% \begin{center} \bfseries EDICS Category: 3-BBND \end{center}
% \fi
%
% For peerreview papers, this IEEEtran command inserts a page break and
% creates the second title. It will be ignored for other modes.
%\IEEEpeerreviewmaketitle




	\item  A die is loaded in such a way that each odd number is twice as likely to occur as
each even number. Find $P(G)$, where $G$ is the event that a number greater than
3 occurs on a single roll of the die.
\\
\solution
		%\begin{table}[H]
	\centering
\begin{tabular}{|c|c|c|}
\hline
Random variable &Value &Definition\\ \hline
\multirow{3}{*}{X} &0 &Slips of Rs 1\\
&1 &Slips of Rs 5\\
&2 &Slips of Rs 13\\ \hline
\multirow{2}{*}{Y} &0 &Box A\\
&1 &Box B\\\hline
\end{tabular}
\caption{}
\label{tab:Distribution}
\end{table}
See \tabref{tab:Distribution}.
\begin{align}
p_{Y}\brak{k}= \begin{cases} 
      \frac{1}{3} & {k=0} \\
      \frac{2}{3 }& {k=1} 
   \end{cases}
   \\
p_{Y|X}\brak{0|0} = \frac{19}{25}\, 
p_{Y|X}\brak{0|1} = \frac{6}{25}\,
p_{Y|X}\brak{1|0} = \frac{45}{50}\,
p_{Y|X}\brak{1|2} = \frac{5}{50}
\end{align}
The desired probability is the probability that a slip drawn at random is marked other than Rs 1,
\begin{align}
&=1-p_X\brak{0}\\
&= p_X(1) + p_X(2)
\end{align}
Using Bayes theorem,
\begin{align}
&= p_Y\brak{0} \times \pr{Y=0 | X=1} + p_Y\brak{1} \times \pr{Y=1|X=2}\\
&=\frac{1}{3} \times \frac{6}{25} + \frac{2}{3} \times \frac{5}{50}\\
&=\frac{11}{75}
\end{align}

\newpage

%\tableofcontents

\bigskip

\renewcommand{\thefigure}{\theenumi}
\renewcommand{\thetable}{\theenumi}
%\renewcommand{\theequation}{\theenumi}

%\begin{abstract}
%%\boldmath
%In this letter, an algorithm for evaluating the exact analytical bit error rate  (BER)  for the piecewise linear (PL) combiner for  multiple relays is presented. Previous results were available only for upto three relays. The algorithm is unique in the sense that  the actual mathematical expressions, that are prohibitively large, need not be explicitly obtained. The diversity gain due to multiple relays is shown through plots of the analytical BER, well supported by simulations. 
%
%\end{abstract}
% IEEEtran.cls defaults to using nonbold math in the Abstract.
% This preserves the distinction between vectors and scalars. However,
% if the journal you are submitting to favors bold math in the abstract,
% then you can use LaTeX's standard command \boldmath at the very start
% of the abstract to achieve this. Many IEEE journals frown on math
% in the abstract anyway.

% Note that keywords are not normally used for peerreview papers.
%\begin{IEEEkeywords}
%Cooperative diversity, decode and forward, piecewise linear
%\end{IEEEkeywords}



% For peer review papers, you can put extra information on the cover
% page as needed:
% \ifCLASSOPTIONpeerreview
% \begin{center} \bfseries EDICS Category: 3-BBND \end{center}
% \fi
%
% For peerreview papers, this IEEEtran command inserts a page break and
% creates the second title. It will be ignored for other modes.
%\IEEEpeerreviewmaketitle




	\item All the jacks, queens and kings are removed from a deck of 52 playing cards. The remaining cards are well shuffled and then one card is drawn at random. Giving ace a value 1 similar value for other cards, find the probability that the card has a value 
		\begin{enumerate}
			\item 7
			\item greater than 7
			\item less than 7
		\end{enumerate}
		%Number of cards left after removing all jacks, queens and kings 
\begin{align}
N	= 52 - 4\times 3
	= 40
\end{align}
%\begin{table}[H]
%\def\arraystretch{1.2}
%\begin{tabular}{|c|c|c|}
%\hline
%	\textbf{Parameter} &\textbf{Value} &\textbf{Description}\\ \hline
%	$X$ &1-10 &Represents the value of the card picked \\ \hline
%\end{tabular}
%\end{table}
Let $1 \le X \le 10$ be the value of the card picked.  Then,
\begin{align}
	p_X(k) &= \Pr(X=k)\ \forall\ 1 \leq k \leq 10\\
	&= \frac{4\times 1}{40}\\
	&= \frac{1}{10}\\
	\therefore p_X(k) &= 
	\begin{cases}
		\frac{1}{10} & 1 \leq k \leq 10\\
		0 & \text{otherwise}
	\end{cases}
\end{align}
and
\begin{align}
	F_{X}(k) &= \sum_{m=0}^{k}p_{X}(m) \quad 1 \leq k \leq 10\\
	&= \frac{k}{10}\\
	\therefore F_{X}(k) &= 
	\begin{cases}
		0 & k \leq 0\\
		\frac{k}{10} & 1\leq k \leq 10\\
		1 & k > 10 
	\end{cases}
\end{align}
\begin{enumerate}
	\item Probability that card has value equal to 7 is
		\begin{align}
			 p_{X}(7)
			= \frac{1}{10}
		\end{align}
	\item Probability that card has value greater than 7 is
		\begin{align}
			1 - F_X(7)
			&= 1 - \frac{7}{10}
			\\
			&= \frac{3}{10}
		\end{align}
	\item Probability that card has value less than 7 is
		\begin{align}
			 F_{X}(6)
			=\frac{6}{10}
		\end{align}
\end{enumerate}

  \item A Lot consists of 48 mobile phones of which 42 are good, 3 have only minor defects and 3 have major defects.Varnika will buy a phone if it is good but the trader will only buy a mobile if it has no major defects. One phone is selected at random from the lot. What is the probability that it is
\begin{enumerate}
	\item acceptable to Varnika?
            \item acceptable to the trader?
\end{enumerate}
\solution
	%\begin{table}[H]
	\centering
\begin{tabular}{|c|c|c|}
\hline
Random variable &Value &Definition\\ \hline
\multirow{3}{*}{X} &0 &Slips of Rs 1\\
&1 &Slips of Rs 5\\
&2 &Slips of Rs 13\\ \hline
\multirow{2}{*}{Y} &0 &Box A\\
&1 &Box B\\\hline
\end{tabular}
\caption{}
\label{tab:Distribution}
\end{table}
See \tabref{tab:Distribution}.
\begin{align}
p_{Y}\brak{k}= \begin{cases} 
      \frac{1}{3} & {k=0} \\
      \frac{2}{3 }& {k=1} 
   \end{cases}
   \\
p_{Y|X}\brak{0|0} = \frac{19}{25}\, 
p_{Y|X}\brak{0|1} = \frac{6}{25}\,
p_{Y|X}\brak{1|0} = \frac{45}{50}\,
p_{Y|X}\brak{1|2} = \frac{5}{50}
\end{align}
The desired probability is the probability that a slip drawn at random is marked other than Rs 1,
\begin{align}
&=1-p_X\brak{0}\\
&= p_X(1) + p_X(2)
\end{align}
Using Bayes theorem,
\begin{align}
&= p_Y\brak{0} \times \pr{Y=0 | X=1} + p_Y\brak{1} \times \pr{Y=1|X=2}\\
&=\frac{1}{3} \times \frac{6}{25} + \frac{2}{3} \times \frac{5}{50}\\
&=\frac{11}{75}
\end{align}

\newpage

%\tableofcontents

\bigskip

\renewcommand{\thefigure}{\theenumi}
\renewcommand{\thetable}{\theenumi}
%\renewcommand{\theequation}{\theenumi}

%\begin{abstract}
%%\boldmath
%In this letter, an algorithm for evaluating the exact analytical bit error rate  (BER)  for the piecewise linear (PL) combiner for  multiple relays is presented. Previous results were available only for upto three relays. The algorithm is unique in the sense that  the actual mathematical expressions, that are prohibitively large, need not be explicitly obtained. The diversity gain due to multiple relays is shown through plots of the analytical BER, well supported by simulations. 
%
%\end{abstract}
% IEEEtran.cls defaults to using nonbold math in the Abstract.
% This preserves the distinction between vectors and scalars. However,
% if the journal you are submitting to favors bold math in the abstract,
% then you can use LaTeX's standard command \boldmath at the very start
% of the abstract to achieve this. Many IEEE journals frown on math
% in the abstract anyway.

% Note that keywords are not normally used for peerreview papers.
%\begin{IEEEkeywords}
%Cooperative diversity, decode and forward, piecewise linear
%\end{IEEEkeywords}



% For peer review papers, you can put extra information on the cover
% page as needed:
% \ifCLASSOPTIONpeerreview
% \begin{center} \bfseries EDICS Category: 3-BBND \end{center}
% \fi
%
% For peerreview papers, this IEEEtran command inserts a page break and
% creates the second title. It will be ignored for other modes.
%\IEEEpeerreviewmaketitle




 \item A student says that if you throw a die, it will show up 1 or not 1. Therefore, the probability of getting 1 and the probability of getting 'not 1' each is equal to $\frac{1}{2}$. Is this correct? Give reasons.\\
 \solution
        %\begin{table}[H]
	\centering
\begin{tabular}{|c|c|c|}
\hline
Random variable &Value &Definition\\ \hline
\multirow{3}{*}{X} &0 &Slips of Rs 1\\
&1 &Slips of Rs 5\\
&2 &Slips of Rs 13\\ \hline
\multirow{2}{*}{Y} &0 &Box A\\
&1 &Box B\\\hline
\end{tabular}
\caption{}
\label{tab:Distribution}
\end{table}
See \tabref{tab:Distribution}.
\begin{align}
p_{Y}\brak{k}= \begin{cases} 
      \frac{1}{3} & {k=0} \\
      \frac{2}{3 }& {k=1} 
   \end{cases}
   \\
p_{Y|X}\brak{0|0} = \frac{19}{25}\, 
p_{Y|X}\brak{0|1} = \frac{6}{25}\,
p_{Y|X}\brak{1|0} = \frac{45}{50}\,
p_{Y|X}\brak{1|2} = \frac{5}{50}
\end{align}
The desired probability is the probability that a slip drawn at random is marked other than Rs 1,
\begin{align}
&=1-p_X\brak{0}\\
&= p_X(1) + p_X(2)
\end{align}
Using Bayes theorem,
\begin{align}
&= p_Y\brak{0} \times \pr{Y=0 | X=1} + p_Y\brak{1} \times \pr{Y=1|X=2}\\
&=\frac{1}{3} \times \frac{6}{25} + \frac{2}{3} \times \frac{5}{50}\\
&=\frac{11}{75}
\end{align}

\newpage

%\tableofcontents

\bigskip

\renewcommand{\thefigure}{\theenumi}
\renewcommand{\thetable}{\theenumi}
%\renewcommand{\theequation}{\theenumi}

%\begin{abstract}
%%\boldmath
%In this letter, an algorithm for evaluating the exact analytical bit error rate  (BER)  for the piecewise linear (PL) combiner for  multiple relays is presented. Previous results were available only for upto three relays. The algorithm is unique in the sense that  the actual mathematical expressions, that are prohibitively large, need not be explicitly obtained. The diversity gain due to multiple relays is shown through plots of the analytical BER, well supported by simulations. 
%
%\end{abstract}
% IEEEtran.cls defaults to using nonbold math in the Abstract.
% This preserves the distinction between vectors and scalars. However,
% if the journal you are submitting to favors bold math in the abstract,
% then you can use LaTeX's standard command \boldmath at the very start
% of the abstract to achieve this. Many IEEE journals frown on math
% in the abstract anyway.

% Note that keywords are not normally used for peerreview papers.
%\begin{IEEEkeywords}
%Cooperative diversity, decode and forward, piecewise linear
%\end{IEEEkeywords}



% For peer review papers, you can put extra information on the cover
% page as needed:
% \ifCLASSOPTIONpeerreview
% \begin{center} \bfseries EDICS Category: 3-BBND \end{center}
% \fi
%
% For peerreview papers, this IEEEtran command inserts a page break and
% creates the second title. It will be ignored for other modes.
%\IEEEpeerreviewmaketitle




   \item Four candidates A, B, C, D have ap-
plied for the assignment to coach a school cricket
team. If A is twice as likely to be selected as B, and
B and C are given about the same chance of being
selected, while C is twice as likely to be selected
as D, what are the probabilities that
\begin{enumerate}
\item C will be selected?
\item A will not be selected?
\end{enumerate}
	%\begin{table}[H]
	\centering
\begin{tabular}{|c|c|c|}
\hline
Random variable &Value &Definition\\ \hline
\multirow{3}{*}{X} &0 &Slips of Rs 1\\
&1 &Slips of Rs 5\\
&2 &Slips of Rs 13\\ \hline
\multirow{2}{*}{Y} &0 &Box A\\
&1 &Box B\\\hline
\end{tabular}
\caption{}
\label{tab:Distribution}
\end{table}
See \tabref{tab:Distribution}.
\begin{align}
p_{Y}\brak{k}= \begin{cases} 
      \frac{1}{3} & {k=0} \\
      \frac{2}{3 }& {k=1} 
   \end{cases}
   \\
p_{Y|X}\brak{0|0} = \frac{19}{25}\, 
p_{Y|X}\brak{0|1} = \frac{6}{25}\,
p_{Y|X}\brak{1|0} = \frac{45}{50}\,
p_{Y|X}\brak{1|2} = \frac{5}{50}
\end{align}
The desired probability is the probability that a slip drawn at random is marked other than Rs 1,
\begin{align}
&=1-p_X\brak{0}\\
&= p_X(1) + p_X(2)
\end{align}
Using Bayes theorem,
\begin{align}
&= p_Y\brak{0} \times \pr{Y=0 | X=1} + p_Y\brak{1} \times \pr{Y=1|X=2}\\
&=\frac{1}{3} \times \frac{6}{25} + \frac{2}{3} \times \frac{5}{50}\\
&=\frac{11}{75}
\end{align}

\newpage

%\tableofcontents

\bigskip

\renewcommand{\thefigure}{\theenumi}
\renewcommand{\thetable}{\theenumi}
%\renewcommand{\theequation}{\theenumi}

%\begin{abstract}
%%\boldmath
%In this letter, an algorithm for evaluating the exact analytical bit error rate  (BER)  for the piecewise linear (PL) combiner for  multiple relays is presented. Previous results were available only for upto three relays. The algorithm is unique in the sense that  the actual mathematical expressions, that are prohibitively large, need not be explicitly obtained. The diversity gain due to multiple relays is shown through plots of the analytical BER, well supported by simulations. 
%
%\end{abstract}
% IEEEtran.cls defaults to using nonbold math in the Abstract.
% This preserves the distinction between vectors and scalars. However,
% if the journal you are submitting to favors bold math in the abstract,
% then you can use LaTeX's standard command \boldmath at the very start
% of the abstract to achieve this. Many IEEE journals frown on math
% in the abstract anyway.

% Note that keywords are not normally used for peerreview papers.
%\begin{IEEEkeywords}
%Cooperative diversity, decode and forward, piecewise linear
%\end{IEEEkeywords}



% For peer review papers, you can put extra information on the cover
% page as needed:
% \ifCLASSOPTIONpeerreview
% \begin{center} \bfseries EDICS Category: 3-BBND \end{center}
% \fi
%
% For peerreview papers, this IEEEtran command inserts a page break and
% creates the second title. It will be ignored for other modes.
%\IEEEpeerreviewmaketitle




 \item A bag contain 24 balls of which $x$ balls are red, $2x$ are white and $3x$ are blue. A ball is selected at random, What is the probability that it is
\begin{enumerate}[label=\alph*)]
\item not red ?
\item white ?
\end{enumerate}
%\begin{table}[H]
	\centering
\begin{tabular}{|c|c|c|}
\hline
Random variable &Value &Definition\\ \hline
\multirow{3}{*}{X} &0 &Slips of Rs 1\\
&1 &Slips of Rs 5\\
&2 &Slips of Rs 13\\ \hline
\multirow{2}{*}{Y} &0 &Box A\\
&1 &Box B\\\hline
\end{tabular}
\caption{}
\label{tab:Distribution}
\end{table}
See \tabref{tab:Distribution}.
\begin{align}
p_{Y}\brak{k}= \begin{cases} 
      \frac{1}{3} & {k=0} \\
      \frac{2}{3 }& {k=1} 
   \end{cases}
   \\
p_{Y|X}\brak{0|0} = \frac{19}{25}\, 
p_{Y|X}\brak{0|1} = \frac{6}{25}\,
p_{Y|X}\brak{1|0} = \frac{45}{50}\,
p_{Y|X}\brak{1|2} = \frac{5}{50}
\end{align}
The desired probability is the probability that a slip drawn at random is marked other than Rs 1,
\begin{align}
&=1-p_X\brak{0}\\
&= p_X(1) + p_X(2)
\end{align}
Using Bayes theorem,
\begin{align}
&= p_Y\brak{0} \times \pr{Y=0 | X=1} + p_Y\brak{1} \times \pr{Y=1|X=2}\\
&=\frac{1}{3} \times \frac{6}{25} + \frac{2}{3} \times \frac{5}{50}\\
&=\frac{11}{75}
\end{align}

\newpage

%\tableofcontents

\bigskip

\renewcommand{\thefigure}{\theenumi}
\renewcommand{\thetable}{\theenumi}
%\renewcommand{\theequation}{\theenumi}

%\begin{abstract}
%%\boldmath
%In this letter, an algorithm for evaluating the exact analytical bit error rate  (BER)  for the piecewise linear (PL) combiner for  multiple relays is presented. Previous results were available only for upto three relays. The algorithm is unique in the sense that  the actual mathematical expressions, that are prohibitively large, need not be explicitly obtained. The diversity gain due to multiple relays is shown through plots of the analytical BER, well supported by simulations. 
%
%\end{abstract}
% IEEEtran.cls defaults to using nonbold math in the Abstract.
% This preserves the distinction between vectors and scalars. However,
% if the journal you are submitting to favors bold math in the abstract,
% then you can use LaTeX's standard command \boldmath at the very start
% of the abstract to achieve this. Many IEEE journals frown on math
% in the abstract anyway.

% Note that keywords are not normally used for peerreview papers.
%\begin{IEEEkeywords}
%Cooperative diversity, decode and forward, piecewise linear
%\end{IEEEkeywords}



% For peer review papers, you can put extra information on the cover
% page as needed:
% \ifCLASSOPTIONpeerreview
% \begin{center} \bfseries EDICS Category: 3-BBND \end{center}
% \fi
%
% For peerreview papers, this IEEEtran command inserts a page break and
% creates the second title. It will be ignored for other modes.
%\IEEEpeerreviewmaketitle




If the letters of the word ASSASSINATION are arranged at random. Find the Probability that
\begin{enumerate}[label=(\alph*)]
\item Four $S's$ come consecutively in the word
\item Two  $I's$ and two $N's$ come together
\item All $A's$ are not coming together
\item No two $A's$ are coming together
\end{enumerate}
%\begin{table}[H]
	\centering
\begin{tabular}{|c|c|c|}
\hline
Random variable &Value &Definition\\ \hline
\multirow{3}{*}{X} &0 &Slips of Rs 1\\
&1 &Slips of Rs 5\\
&2 &Slips of Rs 13\\ \hline
\multirow{2}{*}{Y} &0 &Box A\\
&1 &Box B\\\hline
\end{tabular}
\caption{}
\label{tab:Distribution}
\end{table}
See \tabref{tab:Distribution}.
\begin{align}
p_{Y}\brak{k}= \begin{cases} 
      \frac{1}{3} & {k=0} \\
      \frac{2}{3 }& {k=1} 
   \end{cases}
   \\
p_{Y|X}\brak{0|0} = \frac{19}{25}\, 
p_{Y|X}\brak{0|1} = \frac{6}{25}\,
p_{Y|X}\brak{1|0} = \frac{45}{50}\,
p_{Y|X}\brak{1|2} = \frac{5}{50}
\end{align}
The desired probability is the probability that a slip drawn at random is marked other than Rs 1,
\begin{align}
&=1-p_X\brak{0}\\
&= p_X(1) + p_X(2)
\end{align}
Using Bayes theorem,
\begin{align}
&= p_Y\brak{0} \times \pr{Y=0 | X=1} + p_Y\brak{1} \times \pr{Y=1|X=2}\\
&=\frac{1}{3} \times \frac{6}{25} + \frac{2}{3} \times \frac{5}{50}\\
&=\frac{11}{75}
\end{align}

\newpage

%\tableofcontents

\bigskip

\renewcommand{\thefigure}{\theenumi}
\renewcommand{\thetable}{\theenumi}
%\renewcommand{\theequation}{\theenumi}

%\begin{abstract}
%%\boldmath
%In this letter, an algorithm for evaluating the exact analytical bit error rate  (BER)  for the piecewise linear (PL) combiner for  multiple relays is presented. Previous results were available only for upto three relays. The algorithm is unique in the sense that  the actual mathematical expressions, that are prohibitively large, need not be explicitly obtained. The diversity gain due to multiple relays is shown through plots of the analytical BER, well supported by simulations. 
%
%\end{abstract}
% IEEEtran.cls defaults to using nonbold math in the Abstract.
% This preserves the distinction between vectors and scalars. However,
% if the journal you are submitting to favors bold math in the abstract,
% then you can use LaTeX's standard command \boldmath at the very start
% of the abstract to achieve this. Many IEEE journals frown on math
% in the abstract anyway.

% Note that keywords are not normally used for peerreview papers.
%\begin{IEEEkeywords}
%Cooperative diversity, decode and forward, piecewise linear
%\end{IEEEkeywords}



% For peer review papers, you can put extra information on the cover
% page as needed:
% \ifCLASSOPTIONpeerreview
% \begin{center} \bfseries EDICS Category: 3-BBND \end{center}
% \fi
%
% For peerreview papers, this IEEEtran command inserts a page break and
% creates the second title. It will be ignored for other modes.
%\IEEEpeerreviewmaketitle




	\item One urn contains two black balls (labelled B1 and B2) and one white ball. A
	second urn contains one black ball and two white balls (labelled W1 and W2).
	Suppose the following experiment is performed. One of the two urns is chosen
	at random. Next a ball is randomly chosen from the urn. Then a second ball is
	chosen at random from the same urn without replacing the first ball.
	
	\begin{enumerate}
	\item What is the probability that two black balls are chosen?
	
	\item What is the probability that two balls of opposite colour are chosen?
	\end{enumerate}
	\solution
	%\begin{align}
    \label{eq:12.13.6.18.1}
	\because	\pr{A|B} &> \pr{A},\
\frac{\pr{AB}}{\pr{B}} > \pr{A}
\\
    \label{eq:12.13.6.18.2}
	\implies \pr{AB} &> \pr{A}\pr{B}
	\\
	\text{or, } \frac{\pr{AB}}{\pr{A}} &=\pr{B|A} > \pr{A}
\end{align}

\end{enumerate}

		%
\item 
Out of 100 students, two sections of 40 and 60 are formed. If you and your friend are among the 100 students, what is the probability that
\begin{enumerate}
\item you both enter the same section?
\item you both enter the different sections?
\end{enumerate}
\solution
		%\begin{enumerate}[label=\thesection.\arabic*,ref=\thesection.\theenumi]
	\item One card is drawn from a well-shuffled deck of 52 cards. Find the probability of getting
\begin{enumerate}
\item A king of red colour 
\item A face card 
\item A red face card
\item The jack of hearts
\item A spade
\item The queen of diamonds

\end{enumerate}
\solution
		%\begin{table}[H]
	\centering
\begin{tabular}{|c|c|c|}
\hline
Random variable &Value &Definition\\ \hline
\multirow{3}{*}{X} &0 &Slips of Rs 1\\
&1 &Slips of Rs 5\\
&2 &Slips of Rs 13\\ \hline
\multirow{2}{*}{Y} &0 &Box A\\
&1 &Box B\\\hline
\end{tabular}
\caption{}
\label{tab:Distribution}
\end{table}
See \tabref{tab:Distribution}.
\begin{align}
p_{Y}\brak{k}= \begin{cases} 
      \frac{1}{3} & {k=0} \\
      \frac{2}{3 }& {k=1} 
   \end{cases}
   \\
p_{Y|X}\brak{0|0} = \frac{19}{25}\, 
p_{Y|X}\brak{0|1} = \frac{6}{25}\,
p_{Y|X}\brak{1|0} = \frac{45}{50}\,
p_{Y|X}\brak{1|2} = \frac{5}{50}
\end{align}
The desired probability is the probability that a slip drawn at random is marked other than Rs 1,
\begin{align}
&=1-p_X\brak{0}\\
&= p_X(1) + p_X(2)
\end{align}
Using Bayes theorem,
\begin{align}
&= p_Y\brak{0} \times \pr{Y=0 | X=1} + p_Y\brak{1} \times \pr{Y=1|X=2}\\
&=\frac{1}{3} \times \frac{6}{25} + \frac{2}{3} \times \frac{5}{50}\\
&=\frac{11}{75}
\end{align}

\newpage

%\tableofcontents

\bigskip

\renewcommand{\thefigure}{\theenumi}
\renewcommand{\thetable}{\theenumi}
%\renewcommand{\theequation}{\theenumi}

%\begin{abstract}
%%\boldmath
%In this letter, an algorithm for evaluating the exact analytical bit error rate  (BER)  for the piecewise linear (PL) combiner for  multiple relays is presented. Previous results were available only for upto three relays. The algorithm is unique in the sense that  the actual mathematical expressions, that are prohibitively large, need not be explicitly obtained. The diversity gain due to multiple relays is shown through plots of the analytical BER, well supported by simulations. 
%
%\end{abstract}
% IEEEtran.cls defaults to using nonbold math in the Abstract.
% This preserves the distinction between vectors and scalars. However,
% if the journal you are submitting to favors bold math in the abstract,
% then you can use LaTeX's standard command \boldmath at the very start
% of the abstract to achieve this. Many IEEE journals frown on math
% in the abstract anyway.

% Note that keywords are not normally used for peerreview papers.
%\begin{IEEEkeywords}
%Cooperative diversity, decode and forward, piecewise linear
%\end{IEEEkeywords}



% For peer review papers, you can put extra information on the cover
% page as needed:
% \ifCLASSOPTIONpeerreview
% \begin{center} \bfseries EDICS Category: 3-BBND \end{center}
% \fi
%
% For peerreview papers, this IEEEtran command inserts a page break and
% creates the second title. It will be ignored for other modes.
%\IEEEpeerreviewmaketitle




	\item Five cards—the ten, jack, queen, king and ace of diamonds, are well-shuffled with their face downwards. One card is then picked up at random.
\begin{enumerate}
\item
What is the probability that the card is the queen? 
\item
If the queen is drawn and put aside, what is the probability that the second card picked up is (a) an ace? (b) a queen?\\
\end{enumerate}
\solution
		%\begin{enumerate}[label=\thesection.\arabic*,ref=\thesection.\theenumi]
	\item One card is drawn from a well-shuffled deck of 52 cards. Find the probability of getting
\begin{enumerate}
\item A king of red colour 
\item A face card 
\item A red face card
\item The jack of hearts
\item A spade
\item The queen of diamonds

\end{enumerate}
\solution
		%\input{ncert/10/15/1/14/main.tex}
	\item Five cards—the ten, jack, queen, king and ace of diamonds, are well-shuffled with their face downwards. One card is then picked up at random.
\begin{enumerate}
\item
What is the probability that the card is the queen? 
\item
If the queen is drawn and put aside, what is the probability that the second card picked up is (a) an ace? (b) a queen?\\
\end{enumerate}
\solution
		%\input{ncert/10/15/1/15/defs.tex}
	\item A bag contains $5$ red balls and some blue balls. If the probability of drawing a blue ball is double that if a red ball, determine the number of blue balls in the bag. 
		\\
\solution
		%\input{ncert/10/15/2/3/defs.tex}
	\item A card is selected from a pack of 52 cards.
 \begin{enumerate}[label=(\alph*)] 
                 \item How many points are there in the sample space?
                 \item Calculate the probability that the card is an ace of spades.
                 \item Calculate the probability that the card is (i) an ace and (ii) black card.
 \end{enumerate}
\solution
		%\input{ncert/11/16/3/4/main.tex}
\item Four cards are drawn from a well-shuffled deck of 52 cards. What is the probability of obtaining 3 diamonds and one spade.
\\
\solution
		%\input{ncert/11/16/4/2/defs.tex}
\item In a certain lottery 10,000 tickets are sold and ten equal prizes are awarded. What is the probability of not getting a prize if you buy (a) one ticket (b) two tickets (c) 10 tickets ?	
\\
\solution
		%\input{ncert/11/16/4/4/defs.tex}
		%
\item 
Out of 100 students, two sections of 40 and 60 are formed. If you and your friend are among the 100 students, what is the probability that
\begin{enumerate}
\item you both enter the same section?
\item you both enter the different sections?
\end{enumerate}
\solution
		%\input{ncert/11/16/4/5/defs.tex}
	\item 
The number lock of a suitcase has 4 wheels each labelled with ten digits i.e. from 0 to 9.The lock opens with a sequence of four digits with no repeats.What is the probability of a person getting the right sequence to open the suitcase.
\\
\solution
		%\input{ncert/11/16/4/10/defs.tex}
		%
\item 
Two cards are drawn at random and without replacement from a pack of 52 playing cards. Find the probability that both the cards are black.
\\
\solution
		%\input{ncert/12/13/2/2/defs.tex}
		\item A box of oranges is inspected by examining three randomly selected oranges drawn without replacement. If all the three oranges are good, the box is approved for sale, otherwise, it is rejected. Find the probability that a box containing 15 oranges out of which 12 are good and 3 are bad ones will be approved for sale.
		\label{ncert/12/13/2/3/defs.tex}
		\item Two balls are drawn at random with replacement from a box containing 10 black and 8 red balls. Find the probability that
		\label{ncert/12/13/2/12}
\begin{enumerate}
\item both balls are red.
\item first ball is black and second is red.
\item one of them is black and other is red.
\end{enumerate}

\item In a hostel, 60\% of the students read Hindi newspaper, 40\% read English newspaper and 20\% read both Hindi and English newspapers. A student is selected at random.
		\label{ncert/12/13/2/15}
\begin{enumerate}
\item Find the probability that she reads neither Hindi nor English newspapers.
\item If she reads Hindi newspaper, find the probability that she reads English newspaper.
\item If she reads English newspaper, find the probability that she reads Hindi newspaper.\\
\end{enumerate}
\item The probability of obtaining an even prime number on each die, when a pair of dice is rolled is 
\begin{enumerate}
    \item $0$ 
    
    \item $\frac{1}{3}$ 
    
    \item $\frac{1}{12}$ 
    
    \item $\frac{1}{36}$ 
\end{enumerate}
\solution
		%\input{ncert/12/13/2/17/defs.tex}
	\item A bag contains 4 red and 4 black balls, another bag contains 2 red and 6 black balls. One of the two bags is selected at random and a ball is drawn from the bag which is found to be red. Find the probability that the ball is drawn from the first bag.
\\
\solution
		%\input{ncert/12/13/3/2/main.tex}
  \item
  Cards with numbers 2 to 101 are placed in a box. A card is selected at random.Find the probability that the card has
\begin{enumerate}[label=(\roman*)]
	\item an even number 
	\item a square number
\end{enumerate}
\solution
%\input{exemplar/10/13/3/32/main.tex}
\item
The king, queen and jack of clubs are removed from a deck of 52 playing cards and then well shuffled. Now one card is drawn at random from the remaining cards.  Determine the probability that the card is
\begin{enumerate}[label=(\roman*)]
\item a club
\item 10 of hearts
\end{enumerate}
\solution
%\input{exemplar/10/13/3/29/main.tex}
\item A team of medical students doing their internship have to assist during surgeries
at a city hospital. The probabilities of surgeries rated as very complex, complex,
routine, simple or very simple are respectively, 0.15, 0.20, 0.31, 0.26, .08. Find
the probabilities that a particular surgery will be rated
\begin{enumerate}
	\item complex or very complex;
	\item neither very complex nor very simple;
	\item routine or complex
	\item routine or simple
\end{enumerate}
\solution
%\input{exemplar/11/16/3/8(1)/main.tex}
\item A card is selected from a pack of 52 cards.
\begin{enumerate}[label=(\alph*)]
    \item How many points are there in the sample space?
    \item Calculate the probability that the card is an ace of spades.
    \item Calculate the probability that the card is (i) an ace and (ii) black card.
\end{enumerate}
\solution
%\input{exemplar/11/16/3/4/main2.tex}
\item The probability that a non leap year selected at random will contain 53 sundays.
\\
\solution
%\input{exemplar/10/13/1/19/main.tex}
\item One of the four persons John, Rita, Aslam or Gurpreet will be promoted next
month. Consequently the sample space consists of four elementary outcomes
S = {John promoted, Rita promoted, Aslam promoted, Gurpreet promoted}
You are told that the chances of John’s promotion is same as that of Gurpreet,
Rita’s chances of promotion are twice as likely as Johns. Aslam’s chances are
four times that of John.
\begin{enumerate}
	\item Determine
	\begin{enumerate}
		\item P (John promoted)
		\item P (Rita promoted)
		\item P (Aslam promoted)
		\item P (Gurpreet promoted)
	\end{enumerate}
	\item If A = {John promoted or Gurpreet promoted}, find P (A).
\end{enumerate}
\solution
%\input{exemplar/11/16/3/10/main.tex}
\item A card is drawn from a deck of 52 cards. Find the probability of getting a king or a heart or a red card.\\
\solution
%\input{exemplar/11/16/3/15/main.tex}
\item The probability that a student will pass his examination is 0.73, the probability of
the student getting a compartment is 0.13, and the probability that the student will
either pass or get compartment is 0.96. State True or False.\\
\solution
%\input{exemplar/11/16/3/31/main.tex}
\item A card is selected from a pack of 52 cards\\
\begin{enumerate}[label=(\alph*)]
\item How many points are there in the sample space?
\item Calculate the probability that the cards is an ace of spades.
\item Calculate the probability that the card is (i) an ace (ii)black card.\\
\end{enumerate}
%\input{ncert/11/16/3/4_1/Prob_4.tex}
\item In a non-leap year, the probability of having 53 tuesdays or 53 wednesdays is\\
\solution
%\input{exemplar/11/16/3/18/main.tex}
\item There are 1000 sealed envelopes in a box, 10 of them contain a cash prize of
Rs 100 each, 100 of them contain a cash prize of Rs 50 each and 200 of them
contain a cash prize of Rs 10 each and rest do not contain any cash prize. If they
are well shuffled and an envelope is picked up out, what is the probability that it
contains no cash prize?\\
\solution
%\input{exemplar/10/13/3/34/main.tex}
\item 
A die is thrown and a card is selected at random from a deck of 52 playing cards. The probability of getting an even number on the die and a spade card.\\
\solution
%\input{exemplar/12/13/3/78/main.tex}
\item
If 4-digit numbers greater than 5,000 are randomly formed from the digits 0, 1, 3, 5, and 7, what is the probability of forming a number divisible by 5 when:
\begin{enumerate}
    \item The digits are repeated?
    \item The repetition of digits is not allowed?
\end{enumerate}
\solution
%\input{ncert/11/16/4/9/main.tex}
\item Consider the probability space $\brak{\Omega, \mathcal{G}, P}$ where $\Omega = [0,2]$ and $\mathcal{G} = \cbrak{\phi, \Omega, [0,1], (1,2]}$. Let $X$ and $Y$ be two functions on $\Omega$ defined as
\begin{align*}
    X(\omega) = 
    \begin{cases}
        1 & \text{if }\omega \in [0, 1]\\
        2 & \text{if }\omega \in (1, 2]
    \end{cases}
\end{align*}
and
\begin{align*}
    Y(\omega) = 
    \begin{cases}
        2 & \text{if }\omega \in [0, 1.5]\\
        3 & \text{if }\omega \in (1.5, 2].
    \end{cases}
\end{align*}
Then which one of the following statements is true?
\begin{enumerate}
    \item [(A)] $X$ is a random variable with respect to $\mathcal{G}$, but $Y$ is not a random variable with respect to $\mathcal{G}$.
    \item [(B)] $Y$ is a random variable with respect to $\mathcal{G}$, but $X$ is not a random variable with respect to $\mathcal{G}$.
    \item [(C)] Neither $X$ nor $Y$ is a random variable with respect to $\mathcal{G}$.
    \item [(D)] Both $X$ and $Y$ are random variables with respect to $\mathcal{G}$.
\end{enumerate} \hfill (GATE ST 2023)\\
\solution
%\input{gate/ST/2023/14/main.tex}
	\item  A die is loaded in such a way that each odd number is twice as likely to occur as
each even number. Find $P(G)$, where $G$ is the event that a number greater than
3 occurs on a single roll of the die.
\\
\solution
		%\input{exemplar/11/16/3/5/main.tex}
	\item All the jacks, queens and kings are removed from a deck of 52 playing cards. The remaining cards are well shuffled and then one card is drawn at random. Giving ace a value 1 similar value for other cards, find the probability that the card has a value 
		\begin{enumerate}
			\item 7
			\item greater than 7
			\item less than 7
		\end{enumerate}
		%\input{exemplar/10/13/3/30/main.tex}
  \item A Lot consists of 48 mobile phones of which 42 are good, 3 have only minor defects and 3 have major defects.Varnika will buy a phone if it is good but the trader will only buy a mobile if it has no major defects. One phone is selected at random from the lot. What is the probability that it is
\begin{enumerate}
	\item acceptable to Varnika?
            \item acceptable to the trader?
\end{enumerate}
\solution
	%\input{exemplar/10/13/3/40/main.tex}
 \item A student says that if you throw a die, it will show up 1 or not 1. Therefore, the probability of getting 1 and the probability of getting 'not 1' each is equal to $\frac{1}{2}$. Is this correct? Give reasons.\\
 \solution
        %\input{exemplar/10/13/2/9/main.tex}
   \item Four candidates A, B, C, D have ap-
plied for the assignment to coach a school cricket
team. If A is twice as likely to be selected as B, and
B and C are given about the same chance of being
selected, while C is twice as likely to be selected
as D, what are the probabilities that
\begin{enumerate}
\item C will be selected?
\item A will not be selected?
\end{enumerate}
	%\input{exemplar/11/16/3/9/main.tex}
 \item A bag contain 24 balls of which $x$ balls are red, $2x$ are white and $3x$ are blue. A ball is selected at random, What is the probability that it is
\begin{enumerate}[label=\alph*)]
\item not red ?
\item white ?
\end{enumerate}
%\input{exemplar/10/13/3/41/main.tex}
If the letters of the word ASSASSINATION are arranged at random. Find the Probability that
\begin{enumerate}[label=(\alph*)]
\item Four $S's$ come consecutively in the word
\item Two  $I's$ and two $N's$ come together
\item All $A's$ are not coming together
\item No two $A's$ are coming together
\end{enumerate}
%\input{exemplar/11/16/3/14/main.tex}
	\item One urn contains two black balls (labelled B1 and B2) and one white ball. A
	second urn contains one black ball and two white balls (labelled W1 and W2).
	Suppose the following experiment is performed. One of the two urns is chosen
	at random. Next a ball is randomly chosen from the urn. Then a second ball is
	chosen at random from the same urn without replacing the first ball.
	
	\begin{enumerate}
	\item What is the probability that two black balls are chosen?
	
	\item What is the probability that two balls of opposite colour are chosen?
	\end{enumerate}
	\solution
	%\input{exemplar/11/16/3/12/main1.tex}
\end{enumerate}

	\item A bag contains $5$ red balls and some blue balls. If the probability of drawing a blue ball is double that if a red ball, determine the number of blue balls in the bag. 
		\\
\solution
		%\begin{enumerate}[label=\thesection.\arabic*,ref=\thesection.\theenumi]
	\item One card is drawn from a well-shuffled deck of 52 cards. Find the probability of getting
\begin{enumerate}
\item A king of red colour 
\item A face card 
\item A red face card
\item The jack of hearts
\item A spade
\item The queen of diamonds

\end{enumerate}
\solution
		%\input{ncert/10/15/1/14/main.tex}
	\item Five cards—the ten, jack, queen, king and ace of diamonds, are well-shuffled with their face downwards. One card is then picked up at random.
\begin{enumerate}
\item
What is the probability that the card is the queen? 
\item
If the queen is drawn and put aside, what is the probability that the second card picked up is (a) an ace? (b) a queen?\\
\end{enumerate}
\solution
		%\input{ncert/10/15/1/15/defs.tex}
	\item A bag contains $5$ red balls and some blue balls. If the probability of drawing a blue ball is double that if a red ball, determine the number of blue balls in the bag. 
		\\
\solution
		%\input{ncert/10/15/2/3/defs.tex}
	\item A card is selected from a pack of 52 cards.
 \begin{enumerate}[label=(\alph*)] 
                 \item How many points are there in the sample space?
                 \item Calculate the probability that the card is an ace of spades.
                 \item Calculate the probability that the card is (i) an ace and (ii) black card.
 \end{enumerate}
\solution
		%\input{ncert/11/16/3/4/main.tex}
\item Four cards are drawn from a well-shuffled deck of 52 cards. What is the probability of obtaining 3 diamonds and one spade.
\\
\solution
		%\input{ncert/11/16/4/2/defs.tex}
\item In a certain lottery 10,000 tickets are sold and ten equal prizes are awarded. What is the probability of not getting a prize if you buy (a) one ticket (b) two tickets (c) 10 tickets ?	
\\
\solution
		%\input{ncert/11/16/4/4/defs.tex}
		%
\item 
Out of 100 students, two sections of 40 and 60 are formed. If you and your friend are among the 100 students, what is the probability that
\begin{enumerate}
\item you both enter the same section?
\item you both enter the different sections?
\end{enumerate}
\solution
		%\input{ncert/11/16/4/5/defs.tex}
	\item 
The number lock of a suitcase has 4 wheels each labelled with ten digits i.e. from 0 to 9.The lock opens with a sequence of four digits with no repeats.What is the probability of a person getting the right sequence to open the suitcase.
\\
\solution
		%\input{ncert/11/16/4/10/defs.tex}
		%
\item 
Two cards are drawn at random and without replacement from a pack of 52 playing cards. Find the probability that both the cards are black.
\\
\solution
		%\input{ncert/12/13/2/2/defs.tex}
		\item A box of oranges is inspected by examining three randomly selected oranges drawn without replacement. If all the three oranges are good, the box is approved for sale, otherwise, it is rejected. Find the probability that a box containing 15 oranges out of which 12 are good and 3 are bad ones will be approved for sale.
		\label{ncert/12/13/2/3/defs.tex}
		\item Two balls are drawn at random with replacement from a box containing 10 black and 8 red balls. Find the probability that
		\label{ncert/12/13/2/12}
\begin{enumerate}
\item both balls are red.
\item first ball is black and second is red.
\item one of them is black and other is red.
\end{enumerate}

\item In a hostel, 60\% of the students read Hindi newspaper, 40\% read English newspaper and 20\% read both Hindi and English newspapers. A student is selected at random.
		\label{ncert/12/13/2/15}
\begin{enumerate}
\item Find the probability that she reads neither Hindi nor English newspapers.
\item If she reads Hindi newspaper, find the probability that she reads English newspaper.
\item If she reads English newspaper, find the probability that she reads Hindi newspaper.\\
\end{enumerate}
\item The probability of obtaining an even prime number on each die, when a pair of dice is rolled is 
\begin{enumerate}
    \item $0$ 
    
    \item $\frac{1}{3}$ 
    
    \item $\frac{1}{12}$ 
    
    \item $\frac{1}{36}$ 
\end{enumerate}
\solution
		%\input{ncert/12/13/2/17/defs.tex}
	\item A bag contains 4 red and 4 black balls, another bag contains 2 red and 6 black balls. One of the two bags is selected at random and a ball is drawn from the bag which is found to be red. Find the probability that the ball is drawn from the first bag.
\\
\solution
		%\input{ncert/12/13/3/2/main.tex}
  \item
  Cards with numbers 2 to 101 are placed in a box. A card is selected at random.Find the probability that the card has
\begin{enumerate}[label=(\roman*)]
	\item an even number 
	\item a square number
\end{enumerate}
\solution
%\input{exemplar/10/13/3/32/main.tex}
\item
The king, queen and jack of clubs are removed from a deck of 52 playing cards and then well shuffled. Now one card is drawn at random from the remaining cards.  Determine the probability that the card is
\begin{enumerate}[label=(\roman*)]
\item a club
\item 10 of hearts
\end{enumerate}
\solution
%\input{exemplar/10/13/3/29/main.tex}
\item A team of medical students doing their internship have to assist during surgeries
at a city hospital. The probabilities of surgeries rated as very complex, complex,
routine, simple or very simple are respectively, 0.15, 0.20, 0.31, 0.26, .08. Find
the probabilities that a particular surgery will be rated
\begin{enumerate}
	\item complex or very complex;
	\item neither very complex nor very simple;
	\item routine or complex
	\item routine or simple
\end{enumerate}
\solution
%\input{exemplar/11/16/3/8(1)/main.tex}
\item A card is selected from a pack of 52 cards.
\begin{enumerate}[label=(\alph*)]
    \item How many points are there in the sample space?
    \item Calculate the probability that the card is an ace of spades.
    \item Calculate the probability that the card is (i) an ace and (ii) black card.
\end{enumerate}
\solution
%\input{exemplar/11/16/3/4/main2.tex}
\item The probability that a non leap year selected at random will contain 53 sundays.
\\
\solution
%\input{exemplar/10/13/1/19/main.tex}
\item One of the four persons John, Rita, Aslam or Gurpreet will be promoted next
month. Consequently the sample space consists of four elementary outcomes
S = {John promoted, Rita promoted, Aslam promoted, Gurpreet promoted}
You are told that the chances of John’s promotion is same as that of Gurpreet,
Rita’s chances of promotion are twice as likely as Johns. Aslam’s chances are
four times that of John.
\begin{enumerate}
	\item Determine
	\begin{enumerate}
		\item P (John promoted)
		\item P (Rita promoted)
		\item P (Aslam promoted)
		\item P (Gurpreet promoted)
	\end{enumerate}
	\item If A = {John promoted or Gurpreet promoted}, find P (A).
\end{enumerate}
\solution
%\input{exemplar/11/16/3/10/main.tex}
\item A card is drawn from a deck of 52 cards. Find the probability of getting a king or a heart or a red card.\\
\solution
%\input{exemplar/11/16/3/15/main.tex}
\item The probability that a student will pass his examination is 0.73, the probability of
the student getting a compartment is 0.13, and the probability that the student will
either pass or get compartment is 0.96. State True or False.\\
\solution
%\input{exemplar/11/16/3/31/main.tex}
\item A card is selected from a pack of 52 cards\\
\begin{enumerate}[label=(\alph*)]
\item How many points are there in the sample space?
\item Calculate the probability that the cards is an ace of spades.
\item Calculate the probability that the card is (i) an ace (ii)black card.\\
\end{enumerate}
%\input{ncert/11/16/3/4_1/Prob_4.tex}
\item In a non-leap year, the probability of having 53 tuesdays or 53 wednesdays is\\
\solution
%\input{exemplar/11/16/3/18/main.tex}
\item There are 1000 sealed envelopes in a box, 10 of them contain a cash prize of
Rs 100 each, 100 of them contain a cash prize of Rs 50 each and 200 of them
contain a cash prize of Rs 10 each and rest do not contain any cash prize. If they
are well shuffled and an envelope is picked up out, what is the probability that it
contains no cash prize?\\
\solution
%\input{exemplar/10/13/3/34/main.tex}
\item 
A die is thrown and a card is selected at random from a deck of 52 playing cards. The probability of getting an even number on the die and a spade card.\\
\solution
%\input{exemplar/12/13/3/78/main.tex}
\item
If 4-digit numbers greater than 5,000 are randomly formed from the digits 0, 1, 3, 5, and 7, what is the probability of forming a number divisible by 5 when:
\begin{enumerate}
    \item The digits are repeated?
    \item The repetition of digits is not allowed?
\end{enumerate}
\solution
%\input{ncert/11/16/4/9/main.tex}
\item Consider the probability space $\brak{\Omega, \mathcal{G}, P}$ where $\Omega = [0,2]$ and $\mathcal{G} = \cbrak{\phi, \Omega, [0,1], (1,2]}$. Let $X$ and $Y$ be two functions on $\Omega$ defined as
\begin{align*}
    X(\omega) = 
    \begin{cases}
        1 & \text{if }\omega \in [0, 1]\\
        2 & \text{if }\omega \in (1, 2]
    \end{cases}
\end{align*}
and
\begin{align*}
    Y(\omega) = 
    \begin{cases}
        2 & \text{if }\omega \in [0, 1.5]\\
        3 & \text{if }\omega \in (1.5, 2].
    \end{cases}
\end{align*}
Then which one of the following statements is true?
\begin{enumerate}
    \item [(A)] $X$ is a random variable with respect to $\mathcal{G}$, but $Y$ is not a random variable with respect to $\mathcal{G}$.
    \item [(B)] $Y$ is a random variable with respect to $\mathcal{G}$, but $X$ is not a random variable with respect to $\mathcal{G}$.
    \item [(C)] Neither $X$ nor $Y$ is a random variable with respect to $\mathcal{G}$.
    \item [(D)] Both $X$ and $Y$ are random variables with respect to $\mathcal{G}$.
\end{enumerate} \hfill (GATE ST 2023)\\
\solution
%\input{gate/ST/2023/14/main.tex}
	\item  A die is loaded in such a way that each odd number is twice as likely to occur as
each even number. Find $P(G)$, where $G$ is the event that a number greater than
3 occurs on a single roll of the die.
\\
\solution
		%\input{exemplar/11/16/3/5/main.tex}
	\item All the jacks, queens and kings are removed from a deck of 52 playing cards. The remaining cards are well shuffled and then one card is drawn at random. Giving ace a value 1 similar value for other cards, find the probability that the card has a value 
		\begin{enumerate}
			\item 7
			\item greater than 7
			\item less than 7
		\end{enumerate}
		%\input{exemplar/10/13/3/30/main.tex}
  \item A Lot consists of 48 mobile phones of which 42 are good, 3 have only minor defects and 3 have major defects.Varnika will buy a phone if it is good but the trader will only buy a mobile if it has no major defects. One phone is selected at random from the lot. What is the probability that it is
\begin{enumerate}
	\item acceptable to Varnika?
            \item acceptable to the trader?
\end{enumerate}
\solution
	%\input{exemplar/10/13/3/40/main.tex}
 \item A student says that if you throw a die, it will show up 1 or not 1. Therefore, the probability of getting 1 and the probability of getting 'not 1' each is equal to $\frac{1}{2}$. Is this correct? Give reasons.\\
 \solution
        %\input{exemplar/10/13/2/9/main.tex}
   \item Four candidates A, B, C, D have ap-
plied for the assignment to coach a school cricket
team. If A is twice as likely to be selected as B, and
B and C are given about the same chance of being
selected, while C is twice as likely to be selected
as D, what are the probabilities that
\begin{enumerate}
\item C will be selected?
\item A will not be selected?
\end{enumerate}
	%\input{exemplar/11/16/3/9/main.tex}
 \item A bag contain 24 balls of which $x$ balls are red, $2x$ are white and $3x$ are blue. A ball is selected at random, What is the probability that it is
\begin{enumerate}[label=\alph*)]
\item not red ?
\item white ?
\end{enumerate}
%\input{exemplar/10/13/3/41/main.tex}
If the letters of the word ASSASSINATION are arranged at random. Find the Probability that
\begin{enumerate}[label=(\alph*)]
\item Four $S's$ come consecutively in the word
\item Two  $I's$ and two $N's$ come together
\item All $A's$ are not coming together
\item No two $A's$ are coming together
\end{enumerate}
%\input{exemplar/11/16/3/14/main.tex}
	\item One urn contains two black balls (labelled B1 and B2) and one white ball. A
	second urn contains one black ball and two white balls (labelled W1 and W2).
	Suppose the following experiment is performed. One of the two urns is chosen
	at random. Next a ball is randomly chosen from the urn. Then a second ball is
	chosen at random from the same urn without replacing the first ball.
	
	\begin{enumerate}
	\item What is the probability that two black balls are chosen?
	
	\item What is the probability that two balls of opposite colour are chosen?
	\end{enumerate}
	\solution
	%\input{exemplar/11/16/3/12/main1.tex}
\end{enumerate}

	\item A card is selected from a pack of 52 cards.
 \begin{enumerate}[label=(\alph*)] 
                 \item How many points are there in the sample space?
                 \item Calculate the probability that the card is an ace of spades.
                 \item Calculate the probability that the card is (i) an ace and (ii) black card.
 \end{enumerate}
\solution
		%\begin{table}[H]
	\centering
\begin{tabular}{|c|c|c|}
\hline
Random variable &Value &Definition\\ \hline
\multirow{3}{*}{X} &0 &Slips of Rs 1\\
&1 &Slips of Rs 5\\
&2 &Slips of Rs 13\\ \hline
\multirow{2}{*}{Y} &0 &Box A\\
&1 &Box B\\\hline
\end{tabular}
\caption{}
\label{tab:Distribution}
\end{table}
See \tabref{tab:Distribution}.
\begin{align}
p_{Y}\brak{k}= \begin{cases} 
      \frac{1}{3} & {k=0} \\
      \frac{2}{3 }& {k=1} 
   \end{cases}
   \\
p_{Y|X}\brak{0|0} = \frac{19}{25}\, 
p_{Y|X}\brak{0|1} = \frac{6}{25}\,
p_{Y|X}\brak{1|0} = \frac{45}{50}\,
p_{Y|X}\brak{1|2} = \frac{5}{50}
\end{align}
The desired probability is the probability that a slip drawn at random is marked other than Rs 1,
\begin{align}
&=1-p_X\brak{0}\\
&= p_X(1) + p_X(2)
\end{align}
Using Bayes theorem,
\begin{align}
&= p_Y\brak{0} \times \pr{Y=0 | X=1} + p_Y\brak{1} \times \pr{Y=1|X=2}\\
&=\frac{1}{3} \times \frac{6}{25} + \frac{2}{3} \times \frac{5}{50}\\
&=\frac{11}{75}
\end{align}

\newpage

%\tableofcontents

\bigskip

\renewcommand{\thefigure}{\theenumi}
\renewcommand{\thetable}{\theenumi}
%\renewcommand{\theequation}{\theenumi}

%\begin{abstract}
%%\boldmath
%In this letter, an algorithm for evaluating the exact analytical bit error rate  (BER)  for the piecewise linear (PL) combiner for  multiple relays is presented. Previous results were available only for upto three relays. The algorithm is unique in the sense that  the actual mathematical expressions, that are prohibitively large, need not be explicitly obtained. The diversity gain due to multiple relays is shown through plots of the analytical BER, well supported by simulations. 
%
%\end{abstract}
% IEEEtran.cls defaults to using nonbold math in the Abstract.
% This preserves the distinction between vectors and scalars. However,
% if the journal you are submitting to favors bold math in the abstract,
% then you can use LaTeX's standard command \boldmath at the very start
% of the abstract to achieve this. Many IEEE journals frown on math
% in the abstract anyway.

% Note that keywords are not normally used for peerreview papers.
%\begin{IEEEkeywords}
%Cooperative diversity, decode and forward, piecewise linear
%\end{IEEEkeywords}



% For peer review papers, you can put extra information on the cover
% page as needed:
% \ifCLASSOPTIONpeerreview
% \begin{center} \bfseries EDICS Category: 3-BBND \end{center}
% \fi
%
% For peerreview papers, this IEEEtran command inserts a page break and
% creates the second title. It will be ignored for other modes.
%\IEEEpeerreviewmaketitle




\item Four cards are drawn from a well-shuffled deck of 52 cards. What is the probability of obtaining 3 diamonds and one spade.
\\
\solution
		%\begin{enumerate}[label=\thesection.\arabic*,ref=\thesection.\theenumi]
	\item One card is drawn from a well-shuffled deck of 52 cards. Find the probability of getting
\begin{enumerate}
\item A king of red colour 
\item A face card 
\item A red face card
\item The jack of hearts
\item A spade
\item The queen of diamonds

\end{enumerate}
\solution
		%\input{ncert/10/15/1/14/main.tex}
	\item Five cards—the ten, jack, queen, king and ace of diamonds, are well-shuffled with their face downwards. One card is then picked up at random.
\begin{enumerate}
\item
What is the probability that the card is the queen? 
\item
If the queen is drawn and put aside, what is the probability that the second card picked up is (a) an ace? (b) a queen?\\
\end{enumerate}
\solution
		%\input{ncert/10/15/1/15/defs.tex}
	\item A bag contains $5$ red balls and some blue balls. If the probability of drawing a blue ball is double that if a red ball, determine the number of blue balls in the bag. 
		\\
\solution
		%\input{ncert/10/15/2/3/defs.tex}
	\item A card is selected from a pack of 52 cards.
 \begin{enumerate}[label=(\alph*)] 
                 \item How many points are there in the sample space?
                 \item Calculate the probability that the card is an ace of spades.
                 \item Calculate the probability that the card is (i) an ace and (ii) black card.
 \end{enumerate}
\solution
		%\input{ncert/11/16/3/4/main.tex}
\item Four cards are drawn from a well-shuffled deck of 52 cards. What is the probability of obtaining 3 diamonds and one spade.
\\
\solution
		%\input{ncert/11/16/4/2/defs.tex}
\item In a certain lottery 10,000 tickets are sold and ten equal prizes are awarded. What is the probability of not getting a prize if you buy (a) one ticket (b) two tickets (c) 10 tickets ?	
\\
\solution
		%\input{ncert/11/16/4/4/defs.tex}
		%
\item 
Out of 100 students, two sections of 40 and 60 are formed. If you and your friend are among the 100 students, what is the probability that
\begin{enumerate}
\item you both enter the same section?
\item you both enter the different sections?
\end{enumerate}
\solution
		%\input{ncert/11/16/4/5/defs.tex}
	\item 
The number lock of a suitcase has 4 wheels each labelled with ten digits i.e. from 0 to 9.The lock opens with a sequence of four digits with no repeats.What is the probability of a person getting the right sequence to open the suitcase.
\\
\solution
		%\input{ncert/11/16/4/10/defs.tex}
		%
\item 
Two cards are drawn at random and without replacement from a pack of 52 playing cards. Find the probability that both the cards are black.
\\
\solution
		%\input{ncert/12/13/2/2/defs.tex}
		\item A box of oranges is inspected by examining three randomly selected oranges drawn without replacement. If all the three oranges are good, the box is approved for sale, otherwise, it is rejected. Find the probability that a box containing 15 oranges out of which 12 are good and 3 are bad ones will be approved for sale.
		\label{ncert/12/13/2/3/defs.tex}
		\item Two balls are drawn at random with replacement from a box containing 10 black and 8 red balls. Find the probability that
		\label{ncert/12/13/2/12}
\begin{enumerate}
\item both balls are red.
\item first ball is black and second is red.
\item one of them is black and other is red.
\end{enumerate}

\item In a hostel, 60\% of the students read Hindi newspaper, 40\% read English newspaper and 20\% read both Hindi and English newspapers. A student is selected at random.
		\label{ncert/12/13/2/15}
\begin{enumerate}
\item Find the probability that she reads neither Hindi nor English newspapers.
\item If she reads Hindi newspaper, find the probability that she reads English newspaper.
\item If she reads English newspaper, find the probability that she reads Hindi newspaper.\\
\end{enumerate}
\item The probability of obtaining an even prime number on each die, when a pair of dice is rolled is 
\begin{enumerate}
    \item $0$ 
    
    \item $\frac{1}{3}$ 
    
    \item $\frac{1}{12}$ 
    
    \item $\frac{1}{36}$ 
\end{enumerate}
\solution
		%\input{ncert/12/13/2/17/defs.tex}
	\item A bag contains 4 red and 4 black balls, another bag contains 2 red and 6 black balls. One of the two bags is selected at random and a ball is drawn from the bag which is found to be red. Find the probability that the ball is drawn from the first bag.
\\
\solution
		%\input{ncert/12/13/3/2/main.tex}
  \item
  Cards with numbers 2 to 101 are placed in a box. A card is selected at random.Find the probability that the card has
\begin{enumerate}[label=(\roman*)]
	\item an even number 
	\item a square number
\end{enumerate}
\solution
%\input{exemplar/10/13/3/32/main.tex}
\item
The king, queen and jack of clubs are removed from a deck of 52 playing cards and then well shuffled. Now one card is drawn at random from the remaining cards.  Determine the probability that the card is
\begin{enumerate}[label=(\roman*)]
\item a club
\item 10 of hearts
\end{enumerate}
\solution
%\input{exemplar/10/13/3/29/main.tex}
\item A team of medical students doing their internship have to assist during surgeries
at a city hospital. The probabilities of surgeries rated as very complex, complex,
routine, simple or very simple are respectively, 0.15, 0.20, 0.31, 0.26, .08. Find
the probabilities that a particular surgery will be rated
\begin{enumerate}
	\item complex or very complex;
	\item neither very complex nor very simple;
	\item routine or complex
	\item routine or simple
\end{enumerate}
\solution
%\input{exemplar/11/16/3/8(1)/main.tex}
\item A card is selected from a pack of 52 cards.
\begin{enumerate}[label=(\alph*)]
    \item How many points are there in the sample space?
    \item Calculate the probability that the card is an ace of spades.
    \item Calculate the probability that the card is (i) an ace and (ii) black card.
\end{enumerate}
\solution
%\input{exemplar/11/16/3/4/main2.tex}
\item The probability that a non leap year selected at random will contain 53 sundays.
\\
\solution
%\input{exemplar/10/13/1/19/main.tex}
\item One of the four persons John, Rita, Aslam or Gurpreet will be promoted next
month. Consequently the sample space consists of four elementary outcomes
S = {John promoted, Rita promoted, Aslam promoted, Gurpreet promoted}
You are told that the chances of John’s promotion is same as that of Gurpreet,
Rita’s chances of promotion are twice as likely as Johns. Aslam’s chances are
four times that of John.
\begin{enumerate}
	\item Determine
	\begin{enumerate}
		\item P (John promoted)
		\item P (Rita promoted)
		\item P (Aslam promoted)
		\item P (Gurpreet promoted)
	\end{enumerate}
	\item If A = {John promoted or Gurpreet promoted}, find P (A).
\end{enumerate}
\solution
%\input{exemplar/11/16/3/10/main.tex}
\item A card is drawn from a deck of 52 cards. Find the probability of getting a king or a heart or a red card.\\
\solution
%\input{exemplar/11/16/3/15/main.tex}
\item The probability that a student will pass his examination is 0.73, the probability of
the student getting a compartment is 0.13, and the probability that the student will
either pass or get compartment is 0.96. State True or False.\\
\solution
%\input{exemplar/11/16/3/31/main.tex}
\item A card is selected from a pack of 52 cards\\
\begin{enumerate}[label=(\alph*)]
\item How many points are there in the sample space?
\item Calculate the probability that the cards is an ace of spades.
\item Calculate the probability that the card is (i) an ace (ii)black card.\\
\end{enumerate}
%\input{ncert/11/16/3/4_1/Prob_4.tex}
\item In a non-leap year, the probability of having 53 tuesdays or 53 wednesdays is\\
\solution
%\input{exemplar/11/16/3/18/main.tex}
\item There are 1000 sealed envelopes in a box, 10 of them contain a cash prize of
Rs 100 each, 100 of them contain a cash prize of Rs 50 each and 200 of them
contain a cash prize of Rs 10 each and rest do not contain any cash prize. If they
are well shuffled and an envelope is picked up out, what is the probability that it
contains no cash prize?\\
\solution
%\input{exemplar/10/13/3/34/main.tex}
\item 
A die is thrown and a card is selected at random from a deck of 52 playing cards. The probability of getting an even number on the die and a spade card.\\
\solution
%\input{exemplar/12/13/3/78/main.tex}
\item
If 4-digit numbers greater than 5,000 are randomly formed from the digits 0, 1, 3, 5, and 7, what is the probability of forming a number divisible by 5 when:
\begin{enumerate}
    \item The digits are repeated?
    \item The repetition of digits is not allowed?
\end{enumerate}
\solution
%\input{ncert/11/16/4/9/main.tex}
\item Consider the probability space $\brak{\Omega, \mathcal{G}, P}$ where $\Omega = [0,2]$ and $\mathcal{G} = \cbrak{\phi, \Omega, [0,1], (1,2]}$. Let $X$ and $Y$ be two functions on $\Omega$ defined as
\begin{align*}
    X(\omega) = 
    \begin{cases}
        1 & \text{if }\omega \in [0, 1]\\
        2 & \text{if }\omega \in (1, 2]
    \end{cases}
\end{align*}
and
\begin{align*}
    Y(\omega) = 
    \begin{cases}
        2 & \text{if }\omega \in [0, 1.5]\\
        3 & \text{if }\omega \in (1.5, 2].
    \end{cases}
\end{align*}
Then which one of the following statements is true?
\begin{enumerate}
    \item [(A)] $X$ is a random variable with respect to $\mathcal{G}$, but $Y$ is not a random variable with respect to $\mathcal{G}$.
    \item [(B)] $Y$ is a random variable with respect to $\mathcal{G}$, but $X$ is not a random variable with respect to $\mathcal{G}$.
    \item [(C)] Neither $X$ nor $Y$ is a random variable with respect to $\mathcal{G}$.
    \item [(D)] Both $X$ and $Y$ are random variables with respect to $\mathcal{G}$.
\end{enumerate} \hfill (GATE ST 2023)\\
\solution
%\input{gate/ST/2023/14/main.tex}
	\item  A die is loaded in such a way that each odd number is twice as likely to occur as
each even number. Find $P(G)$, where $G$ is the event that a number greater than
3 occurs on a single roll of the die.
\\
\solution
		%\input{exemplar/11/16/3/5/main.tex}
	\item All the jacks, queens and kings are removed from a deck of 52 playing cards. The remaining cards are well shuffled and then one card is drawn at random. Giving ace a value 1 similar value for other cards, find the probability that the card has a value 
		\begin{enumerate}
			\item 7
			\item greater than 7
			\item less than 7
		\end{enumerate}
		%\input{exemplar/10/13/3/30/main.tex}
  \item A Lot consists of 48 mobile phones of which 42 are good, 3 have only minor defects and 3 have major defects.Varnika will buy a phone if it is good but the trader will only buy a mobile if it has no major defects. One phone is selected at random from the lot. What is the probability that it is
\begin{enumerate}
	\item acceptable to Varnika?
            \item acceptable to the trader?
\end{enumerate}
\solution
	%\input{exemplar/10/13/3/40/main.tex}
 \item A student says that if you throw a die, it will show up 1 or not 1. Therefore, the probability of getting 1 and the probability of getting 'not 1' each is equal to $\frac{1}{2}$. Is this correct? Give reasons.\\
 \solution
        %\input{exemplar/10/13/2/9/main.tex}
   \item Four candidates A, B, C, D have ap-
plied for the assignment to coach a school cricket
team. If A is twice as likely to be selected as B, and
B and C are given about the same chance of being
selected, while C is twice as likely to be selected
as D, what are the probabilities that
\begin{enumerate}
\item C will be selected?
\item A will not be selected?
\end{enumerate}
	%\input{exemplar/11/16/3/9/main.tex}
 \item A bag contain 24 balls of which $x$ balls are red, $2x$ are white and $3x$ are blue. A ball is selected at random, What is the probability that it is
\begin{enumerate}[label=\alph*)]
\item not red ?
\item white ?
\end{enumerate}
%\input{exemplar/10/13/3/41/main.tex}
If the letters of the word ASSASSINATION are arranged at random. Find the Probability that
\begin{enumerate}[label=(\alph*)]
\item Four $S's$ come consecutively in the word
\item Two  $I's$ and two $N's$ come together
\item All $A's$ are not coming together
\item No two $A's$ are coming together
\end{enumerate}
%\input{exemplar/11/16/3/14/main.tex}
	\item One urn contains two black balls (labelled B1 and B2) and one white ball. A
	second urn contains one black ball and two white balls (labelled W1 and W2).
	Suppose the following experiment is performed. One of the two urns is chosen
	at random. Next a ball is randomly chosen from the urn. Then a second ball is
	chosen at random from the same urn without replacing the first ball.
	
	\begin{enumerate}
	\item What is the probability that two black balls are chosen?
	
	\item What is the probability that two balls of opposite colour are chosen?
	\end{enumerate}
	\solution
	%\input{exemplar/11/16/3/12/main1.tex}
\end{enumerate}

\item In a certain lottery 10,000 tickets are sold and ten equal prizes are awarded. What is the probability of not getting a prize if you buy (a) one ticket (b) two tickets (c) 10 tickets ?	
\\
\solution
		%\begin{enumerate}[label=\thesection.\arabic*,ref=\thesection.\theenumi]
	\item One card is drawn from a well-shuffled deck of 52 cards. Find the probability of getting
\begin{enumerate}
\item A king of red colour 
\item A face card 
\item A red face card
\item The jack of hearts
\item A spade
\item The queen of diamonds

\end{enumerate}
\solution
		%\input{ncert/10/15/1/14/main.tex}
	\item Five cards—the ten, jack, queen, king and ace of diamonds, are well-shuffled with their face downwards. One card is then picked up at random.
\begin{enumerate}
\item
What is the probability that the card is the queen? 
\item
If the queen is drawn and put aside, what is the probability that the second card picked up is (a) an ace? (b) a queen?\\
\end{enumerate}
\solution
		%\input{ncert/10/15/1/15/defs.tex}
	\item A bag contains $5$ red balls and some blue balls. If the probability of drawing a blue ball is double that if a red ball, determine the number of blue balls in the bag. 
		\\
\solution
		%\input{ncert/10/15/2/3/defs.tex}
	\item A card is selected from a pack of 52 cards.
 \begin{enumerate}[label=(\alph*)] 
                 \item How many points are there in the sample space?
                 \item Calculate the probability that the card is an ace of spades.
                 \item Calculate the probability that the card is (i) an ace and (ii) black card.
 \end{enumerate}
\solution
		%\input{ncert/11/16/3/4/main.tex}
\item Four cards are drawn from a well-shuffled deck of 52 cards. What is the probability of obtaining 3 diamonds and one spade.
\\
\solution
		%\input{ncert/11/16/4/2/defs.tex}
\item In a certain lottery 10,000 tickets are sold and ten equal prizes are awarded. What is the probability of not getting a prize if you buy (a) one ticket (b) two tickets (c) 10 tickets ?	
\\
\solution
		%\input{ncert/11/16/4/4/defs.tex}
		%
\item 
Out of 100 students, two sections of 40 and 60 are formed. If you and your friend are among the 100 students, what is the probability that
\begin{enumerate}
\item you both enter the same section?
\item you both enter the different sections?
\end{enumerate}
\solution
		%\input{ncert/11/16/4/5/defs.tex}
	\item 
The number lock of a suitcase has 4 wheels each labelled with ten digits i.e. from 0 to 9.The lock opens with a sequence of four digits with no repeats.What is the probability of a person getting the right sequence to open the suitcase.
\\
\solution
		%\input{ncert/11/16/4/10/defs.tex}
		%
\item 
Two cards are drawn at random and without replacement from a pack of 52 playing cards. Find the probability that both the cards are black.
\\
\solution
		%\input{ncert/12/13/2/2/defs.tex}
		\item A box of oranges is inspected by examining three randomly selected oranges drawn without replacement. If all the three oranges are good, the box is approved for sale, otherwise, it is rejected. Find the probability that a box containing 15 oranges out of which 12 are good and 3 are bad ones will be approved for sale.
		\label{ncert/12/13/2/3/defs.tex}
		\item Two balls are drawn at random with replacement from a box containing 10 black and 8 red balls. Find the probability that
		\label{ncert/12/13/2/12}
\begin{enumerate}
\item both balls are red.
\item first ball is black and second is red.
\item one of them is black and other is red.
\end{enumerate}

\item In a hostel, 60\% of the students read Hindi newspaper, 40\% read English newspaper and 20\% read both Hindi and English newspapers. A student is selected at random.
		\label{ncert/12/13/2/15}
\begin{enumerate}
\item Find the probability that she reads neither Hindi nor English newspapers.
\item If she reads Hindi newspaper, find the probability that she reads English newspaper.
\item If she reads English newspaper, find the probability that she reads Hindi newspaper.\\
\end{enumerate}
\item The probability of obtaining an even prime number on each die, when a pair of dice is rolled is 
\begin{enumerate}
    \item $0$ 
    
    \item $\frac{1}{3}$ 
    
    \item $\frac{1}{12}$ 
    
    \item $\frac{1}{36}$ 
\end{enumerate}
\solution
		%\input{ncert/12/13/2/17/defs.tex}
	\item A bag contains 4 red and 4 black balls, another bag contains 2 red and 6 black balls. One of the two bags is selected at random and a ball is drawn from the bag which is found to be red. Find the probability that the ball is drawn from the first bag.
\\
\solution
		%\input{ncert/12/13/3/2/main.tex}
  \item
  Cards with numbers 2 to 101 are placed in a box. A card is selected at random.Find the probability that the card has
\begin{enumerate}[label=(\roman*)]
	\item an even number 
	\item a square number
\end{enumerate}
\solution
%\input{exemplar/10/13/3/32/main.tex}
\item
The king, queen and jack of clubs are removed from a deck of 52 playing cards and then well shuffled. Now one card is drawn at random from the remaining cards.  Determine the probability that the card is
\begin{enumerate}[label=(\roman*)]
\item a club
\item 10 of hearts
\end{enumerate}
\solution
%\input{exemplar/10/13/3/29/main.tex}
\item A team of medical students doing their internship have to assist during surgeries
at a city hospital. The probabilities of surgeries rated as very complex, complex,
routine, simple or very simple are respectively, 0.15, 0.20, 0.31, 0.26, .08. Find
the probabilities that a particular surgery will be rated
\begin{enumerate}
	\item complex or very complex;
	\item neither very complex nor very simple;
	\item routine or complex
	\item routine or simple
\end{enumerate}
\solution
%\input{exemplar/11/16/3/8(1)/main.tex}
\item A card is selected from a pack of 52 cards.
\begin{enumerate}[label=(\alph*)]
    \item How many points are there in the sample space?
    \item Calculate the probability that the card is an ace of spades.
    \item Calculate the probability that the card is (i) an ace and (ii) black card.
\end{enumerate}
\solution
%\input{exemplar/11/16/3/4/main2.tex}
\item The probability that a non leap year selected at random will contain 53 sundays.
\\
\solution
%\input{exemplar/10/13/1/19/main.tex}
\item One of the four persons John, Rita, Aslam or Gurpreet will be promoted next
month. Consequently the sample space consists of four elementary outcomes
S = {John promoted, Rita promoted, Aslam promoted, Gurpreet promoted}
You are told that the chances of John’s promotion is same as that of Gurpreet,
Rita’s chances of promotion are twice as likely as Johns. Aslam’s chances are
four times that of John.
\begin{enumerate}
	\item Determine
	\begin{enumerate}
		\item P (John promoted)
		\item P (Rita promoted)
		\item P (Aslam promoted)
		\item P (Gurpreet promoted)
	\end{enumerate}
	\item If A = {John promoted or Gurpreet promoted}, find P (A).
\end{enumerate}
\solution
%\input{exemplar/11/16/3/10/main.tex}
\item A card is drawn from a deck of 52 cards. Find the probability of getting a king or a heart or a red card.\\
\solution
%\input{exemplar/11/16/3/15/main.tex}
\item The probability that a student will pass his examination is 0.73, the probability of
the student getting a compartment is 0.13, and the probability that the student will
either pass or get compartment is 0.96. State True or False.\\
\solution
%\input{exemplar/11/16/3/31/main.tex}
\item A card is selected from a pack of 52 cards\\
\begin{enumerate}[label=(\alph*)]
\item How many points are there in the sample space?
\item Calculate the probability that the cards is an ace of spades.
\item Calculate the probability that the card is (i) an ace (ii)black card.\\
\end{enumerate}
%\input{ncert/11/16/3/4_1/Prob_4.tex}
\item In a non-leap year, the probability of having 53 tuesdays or 53 wednesdays is\\
\solution
%\input{exemplar/11/16/3/18/main.tex}
\item There are 1000 sealed envelopes in a box, 10 of them contain a cash prize of
Rs 100 each, 100 of them contain a cash prize of Rs 50 each and 200 of them
contain a cash prize of Rs 10 each and rest do not contain any cash prize. If they
are well shuffled and an envelope is picked up out, what is the probability that it
contains no cash prize?\\
\solution
%\input{exemplar/10/13/3/34/main.tex}
\item 
A die is thrown and a card is selected at random from a deck of 52 playing cards. The probability of getting an even number on the die and a spade card.\\
\solution
%\input{exemplar/12/13/3/78/main.tex}
\item
If 4-digit numbers greater than 5,000 are randomly formed from the digits 0, 1, 3, 5, and 7, what is the probability of forming a number divisible by 5 when:
\begin{enumerate}
    \item The digits are repeated?
    \item The repetition of digits is not allowed?
\end{enumerate}
\solution
%\input{ncert/11/16/4/9/main.tex}
\item Consider the probability space $\brak{\Omega, \mathcal{G}, P}$ where $\Omega = [0,2]$ and $\mathcal{G} = \cbrak{\phi, \Omega, [0,1], (1,2]}$. Let $X$ and $Y$ be two functions on $\Omega$ defined as
\begin{align*}
    X(\omega) = 
    \begin{cases}
        1 & \text{if }\omega \in [0, 1]\\
        2 & \text{if }\omega \in (1, 2]
    \end{cases}
\end{align*}
and
\begin{align*}
    Y(\omega) = 
    \begin{cases}
        2 & \text{if }\omega \in [0, 1.5]\\
        3 & \text{if }\omega \in (1.5, 2].
    \end{cases}
\end{align*}
Then which one of the following statements is true?
\begin{enumerate}
    \item [(A)] $X$ is a random variable with respect to $\mathcal{G}$, but $Y$ is not a random variable with respect to $\mathcal{G}$.
    \item [(B)] $Y$ is a random variable with respect to $\mathcal{G}$, but $X$ is not a random variable with respect to $\mathcal{G}$.
    \item [(C)] Neither $X$ nor $Y$ is a random variable with respect to $\mathcal{G}$.
    \item [(D)] Both $X$ and $Y$ are random variables with respect to $\mathcal{G}$.
\end{enumerate} \hfill (GATE ST 2023)\\
\solution
%\input{gate/ST/2023/14/main.tex}
	\item  A die is loaded in such a way that each odd number is twice as likely to occur as
each even number. Find $P(G)$, where $G$ is the event that a number greater than
3 occurs on a single roll of the die.
\\
\solution
		%\input{exemplar/11/16/3/5/main.tex}
	\item All the jacks, queens and kings are removed from a deck of 52 playing cards. The remaining cards are well shuffled and then one card is drawn at random. Giving ace a value 1 similar value for other cards, find the probability that the card has a value 
		\begin{enumerate}
			\item 7
			\item greater than 7
			\item less than 7
		\end{enumerate}
		%\input{exemplar/10/13/3/30/main.tex}
  \item A Lot consists of 48 mobile phones of which 42 are good, 3 have only minor defects and 3 have major defects.Varnika will buy a phone if it is good but the trader will only buy a mobile if it has no major defects. One phone is selected at random from the lot. What is the probability that it is
\begin{enumerate}
	\item acceptable to Varnika?
            \item acceptable to the trader?
\end{enumerate}
\solution
	%\input{exemplar/10/13/3/40/main.tex}
 \item A student says that if you throw a die, it will show up 1 or not 1. Therefore, the probability of getting 1 and the probability of getting 'not 1' each is equal to $\frac{1}{2}$. Is this correct? Give reasons.\\
 \solution
        %\input{exemplar/10/13/2/9/main.tex}
   \item Four candidates A, B, C, D have ap-
plied for the assignment to coach a school cricket
team. If A is twice as likely to be selected as B, and
B and C are given about the same chance of being
selected, while C is twice as likely to be selected
as D, what are the probabilities that
\begin{enumerate}
\item C will be selected?
\item A will not be selected?
\end{enumerate}
	%\input{exemplar/11/16/3/9/main.tex}
 \item A bag contain 24 balls of which $x$ balls are red, $2x$ are white and $3x$ are blue. A ball is selected at random, What is the probability that it is
\begin{enumerate}[label=\alph*)]
\item not red ?
\item white ?
\end{enumerate}
%\input{exemplar/10/13/3/41/main.tex}
If the letters of the word ASSASSINATION are arranged at random. Find the Probability that
\begin{enumerate}[label=(\alph*)]
\item Four $S's$ come consecutively in the word
\item Two  $I's$ and two $N's$ come together
\item All $A's$ are not coming together
\item No two $A's$ are coming together
\end{enumerate}
%\input{exemplar/11/16/3/14/main.tex}
	\item One urn contains two black balls (labelled B1 and B2) and one white ball. A
	second urn contains one black ball and two white balls (labelled W1 and W2).
	Suppose the following experiment is performed. One of the two urns is chosen
	at random. Next a ball is randomly chosen from the urn. Then a second ball is
	chosen at random from the same urn without replacing the first ball.
	
	\begin{enumerate}
	\item What is the probability that two black balls are chosen?
	
	\item What is the probability that two balls of opposite colour are chosen?
	\end{enumerate}
	\solution
	%\input{exemplar/11/16/3/12/main1.tex}
\end{enumerate}

		%
\item 
Out of 100 students, two sections of 40 and 60 are formed. If you and your friend are among the 100 students, what is the probability that
\begin{enumerate}
\item you both enter the same section?
\item you both enter the different sections?
\end{enumerate}
\solution
		%\begin{enumerate}[label=\thesection.\arabic*,ref=\thesection.\theenumi]
	\item One card is drawn from a well-shuffled deck of 52 cards. Find the probability of getting
\begin{enumerate}
\item A king of red colour 
\item A face card 
\item A red face card
\item The jack of hearts
\item A spade
\item The queen of diamonds

\end{enumerate}
\solution
		%\input{ncert/10/15/1/14/main.tex}
	\item Five cards—the ten, jack, queen, king and ace of diamonds, are well-shuffled with their face downwards. One card is then picked up at random.
\begin{enumerate}
\item
What is the probability that the card is the queen? 
\item
If the queen is drawn and put aside, what is the probability that the second card picked up is (a) an ace? (b) a queen?\\
\end{enumerate}
\solution
		%\input{ncert/10/15/1/15/defs.tex}
	\item A bag contains $5$ red balls and some blue balls. If the probability of drawing a blue ball is double that if a red ball, determine the number of blue balls in the bag. 
		\\
\solution
		%\input{ncert/10/15/2/3/defs.tex}
	\item A card is selected from a pack of 52 cards.
 \begin{enumerate}[label=(\alph*)] 
                 \item How many points are there in the sample space?
                 \item Calculate the probability that the card is an ace of spades.
                 \item Calculate the probability that the card is (i) an ace and (ii) black card.
 \end{enumerate}
\solution
		%\input{ncert/11/16/3/4/main.tex}
\item Four cards are drawn from a well-shuffled deck of 52 cards. What is the probability of obtaining 3 diamonds and one spade.
\\
\solution
		%\input{ncert/11/16/4/2/defs.tex}
\item In a certain lottery 10,000 tickets are sold and ten equal prizes are awarded. What is the probability of not getting a prize if you buy (a) one ticket (b) two tickets (c) 10 tickets ?	
\\
\solution
		%\input{ncert/11/16/4/4/defs.tex}
		%
\item 
Out of 100 students, two sections of 40 and 60 are formed. If you and your friend are among the 100 students, what is the probability that
\begin{enumerate}
\item you both enter the same section?
\item you both enter the different sections?
\end{enumerate}
\solution
		%\input{ncert/11/16/4/5/defs.tex}
	\item 
The number lock of a suitcase has 4 wheels each labelled with ten digits i.e. from 0 to 9.The lock opens with a sequence of four digits with no repeats.What is the probability of a person getting the right sequence to open the suitcase.
\\
\solution
		%\input{ncert/11/16/4/10/defs.tex}
		%
\item 
Two cards are drawn at random and without replacement from a pack of 52 playing cards. Find the probability that both the cards are black.
\\
\solution
		%\input{ncert/12/13/2/2/defs.tex}
		\item A box of oranges is inspected by examining three randomly selected oranges drawn without replacement. If all the three oranges are good, the box is approved for sale, otherwise, it is rejected. Find the probability that a box containing 15 oranges out of which 12 are good and 3 are bad ones will be approved for sale.
		\label{ncert/12/13/2/3/defs.tex}
		\item Two balls are drawn at random with replacement from a box containing 10 black and 8 red balls. Find the probability that
		\label{ncert/12/13/2/12}
\begin{enumerate}
\item both balls are red.
\item first ball is black and second is red.
\item one of them is black and other is red.
\end{enumerate}

\item In a hostel, 60\% of the students read Hindi newspaper, 40\% read English newspaper and 20\% read both Hindi and English newspapers. A student is selected at random.
		\label{ncert/12/13/2/15}
\begin{enumerate}
\item Find the probability that she reads neither Hindi nor English newspapers.
\item If she reads Hindi newspaper, find the probability that she reads English newspaper.
\item If she reads English newspaper, find the probability that she reads Hindi newspaper.\\
\end{enumerate}
\item The probability of obtaining an even prime number on each die, when a pair of dice is rolled is 
\begin{enumerate}
    \item $0$ 
    
    \item $\frac{1}{3}$ 
    
    \item $\frac{1}{12}$ 
    
    \item $\frac{1}{36}$ 
\end{enumerate}
\solution
		%\input{ncert/12/13/2/17/defs.tex}
	\item A bag contains 4 red and 4 black balls, another bag contains 2 red and 6 black balls. One of the two bags is selected at random and a ball is drawn from the bag which is found to be red. Find the probability that the ball is drawn from the first bag.
\\
\solution
		%\input{ncert/12/13/3/2/main.tex}
  \item
  Cards with numbers 2 to 101 are placed in a box. A card is selected at random.Find the probability that the card has
\begin{enumerate}[label=(\roman*)]
	\item an even number 
	\item a square number
\end{enumerate}
\solution
%\input{exemplar/10/13/3/32/main.tex}
\item
The king, queen and jack of clubs are removed from a deck of 52 playing cards and then well shuffled. Now one card is drawn at random from the remaining cards.  Determine the probability that the card is
\begin{enumerate}[label=(\roman*)]
\item a club
\item 10 of hearts
\end{enumerate}
\solution
%\input{exemplar/10/13/3/29/main.tex}
\item A team of medical students doing their internship have to assist during surgeries
at a city hospital. The probabilities of surgeries rated as very complex, complex,
routine, simple or very simple are respectively, 0.15, 0.20, 0.31, 0.26, .08. Find
the probabilities that a particular surgery will be rated
\begin{enumerate}
	\item complex or very complex;
	\item neither very complex nor very simple;
	\item routine or complex
	\item routine or simple
\end{enumerate}
\solution
%\input{exemplar/11/16/3/8(1)/main.tex}
\item A card is selected from a pack of 52 cards.
\begin{enumerate}[label=(\alph*)]
    \item How many points are there in the sample space?
    \item Calculate the probability that the card is an ace of spades.
    \item Calculate the probability that the card is (i) an ace and (ii) black card.
\end{enumerate}
\solution
%\input{exemplar/11/16/3/4/main2.tex}
\item The probability that a non leap year selected at random will contain 53 sundays.
\\
\solution
%\input{exemplar/10/13/1/19/main.tex}
\item One of the four persons John, Rita, Aslam or Gurpreet will be promoted next
month. Consequently the sample space consists of four elementary outcomes
S = {John promoted, Rita promoted, Aslam promoted, Gurpreet promoted}
You are told that the chances of John’s promotion is same as that of Gurpreet,
Rita’s chances of promotion are twice as likely as Johns. Aslam’s chances are
four times that of John.
\begin{enumerate}
	\item Determine
	\begin{enumerate}
		\item P (John promoted)
		\item P (Rita promoted)
		\item P (Aslam promoted)
		\item P (Gurpreet promoted)
	\end{enumerate}
	\item If A = {John promoted or Gurpreet promoted}, find P (A).
\end{enumerate}
\solution
%\input{exemplar/11/16/3/10/main.tex}
\item A card is drawn from a deck of 52 cards. Find the probability of getting a king or a heart or a red card.\\
\solution
%\input{exemplar/11/16/3/15/main.tex}
\item The probability that a student will pass his examination is 0.73, the probability of
the student getting a compartment is 0.13, and the probability that the student will
either pass or get compartment is 0.96. State True or False.\\
\solution
%\input{exemplar/11/16/3/31/main.tex}
\item A card is selected from a pack of 52 cards\\
\begin{enumerate}[label=(\alph*)]
\item How many points are there in the sample space?
\item Calculate the probability that the cards is an ace of spades.
\item Calculate the probability that the card is (i) an ace (ii)black card.\\
\end{enumerate}
%\input{ncert/11/16/3/4_1/Prob_4.tex}
\item In a non-leap year, the probability of having 53 tuesdays or 53 wednesdays is\\
\solution
%\input{exemplar/11/16/3/18/main.tex}
\item There are 1000 sealed envelopes in a box, 10 of them contain a cash prize of
Rs 100 each, 100 of them contain a cash prize of Rs 50 each and 200 of them
contain a cash prize of Rs 10 each and rest do not contain any cash prize. If they
are well shuffled and an envelope is picked up out, what is the probability that it
contains no cash prize?\\
\solution
%\input{exemplar/10/13/3/34/main.tex}
\item 
A die is thrown and a card is selected at random from a deck of 52 playing cards. The probability of getting an even number on the die and a spade card.\\
\solution
%\input{exemplar/12/13/3/78/main.tex}
\item
If 4-digit numbers greater than 5,000 are randomly formed from the digits 0, 1, 3, 5, and 7, what is the probability of forming a number divisible by 5 when:
\begin{enumerate}
    \item The digits are repeated?
    \item The repetition of digits is not allowed?
\end{enumerate}
\solution
%\input{ncert/11/16/4/9/main.tex}
\item Consider the probability space $\brak{\Omega, \mathcal{G}, P}$ where $\Omega = [0,2]$ and $\mathcal{G} = \cbrak{\phi, \Omega, [0,1], (1,2]}$. Let $X$ and $Y$ be two functions on $\Omega$ defined as
\begin{align*}
    X(\omega) = 
    \begin{cases}
        1 & \text{if }\omega \in [0, 1]\\
        2 & \text{if }\omega \in (1, 2]
    \end{cases}
\end{align*}
and
\begin{align*}
    Y(\omega) = 
    \begin{cases}
        2 & \text{if }\omega \in [0, 1.5]\\
        3 & \text{if }\omega \in (1.5, 2].
    \end{cases}
\end{align*}
Then which one of the following statements is true?
\begin{enumerate}
    \item [(A)] $X$ is a random variable with respect to $\mathcal{G}$, but $Y$ is not a random variable with respect to $\mathcal{G}$.
    \item [(B)] $Y$ is a random variable with respect to $\mathcal{G}$, but $X$ is not a random variable with respect to $\mathcal{G}$.
    \item [(C)] Neither $X$ nor $Y$ is a random variable with respect to $\mathcal{G}$.
    \item [(D)] Both $X$ and $Y$ are random variables with respect to $\mathcal{G}$.
\end{enumerate} \hfill (GATE ST 2023)\\
\solution
%\input{gate/ST/2023/14/main.tex}
	\item  A die is loaded in such a way that each odd number is twice as likely to occur as
each even number. Find $P(G)$, where $G$ is the event that a number greater than
3 occurs on a single roll of the die.
\\
\solution
		%\input{exemplar/11/16/3/5/main.tex}
	\item All the jacks, queens and kings are removed from a deck of 52 playing cards. The remaining cards are well shuffled and then one card is drawn at random. Giving ace a value 1 similar value for other cards, find the probability that the card has a value 
		\begin{enumerate}
			\item 7
			\item greater than 7
			\item less than 7
		\end{enumerate}
		%\input{exemplar/10/13/3/30/main.tex}
  \item A Lot consists of 48 mobile phones of which 42 are good, 3 have only minor defects and 3 have major defects.Varnika will buy a phone if it is good but the trader will only buy a mobile if it has no major defects. One phone is selected at random from the lot. What is the probability that it is
\begin{enumerate}
	\item acceptable to Varnika?
            \item acceptable to the trader?
\end{enumerate}
\solution
	%\input{exemplar/10/13/3/40/main.tex}
 \item A student says that if you throw a die, it will show up 1 or not 1. Therefore, the probability of getting 1 and the probability of getting 'not 1' each is equal to $\frac{1}{2}$. Is this correct? Give reasons.\\
 \solution
        %\input{exemplar/10/13/2/9/main.tex}
   \item Four candidates A, B, C, D have ap-
plied for the assignment to coach a school cricket
team. If A is twice as likely to be selected as B, and
B and C are given about the same chance of being
selected, while C is twice as likely to be selected
as D, what are the probabilities that
\begin{enumerate}
\item C will be selected?
\item A will not be selected?
\end{enumerate}
	%\input{exemplar/11/16/3/9/main.tex}
 \item A bag contain 24 balls of which $x$ balls are red, $2x$ are white and $3x$ are blue. A ball is selected at random, What is the probability that it is
\begin{enumerate}[label=\alph*)]
\item not red ?
\item white ?
\end{enumerate}
%\input{exemplar/10/13/3/41/main.tex}
If the letters of the word ASSASSINATION are arranged at random. Find the Probability that
\begin{enumerate}[label=(\alph*)]
\item Four $S's$ come consecutively in the word
\item Two  $I's$ and two $N's$ come together
\item All $A's$ are not coming together
\item No two $A's$ are coming together
\end{enumerate}
%\input{exemplar/11/16/3/14/main.tex}
	\item One urn contains two black balls (labelled B1 and B2) and one white ball. A
	second urn contains one black ball and two white balls (labelled W1 and W2).
	Suppose the following experiment is performed. One of the two urns is chosen
	at random. Next a ball is randomly chosen from the urn. Then a second ball is
	chosen at random from the same urn without replacing the first ball.
	
	\begin{enumerate}
	\item What is the probability that two black balls are chosen?
	
	\item What is the probability that two balls of opposite colour are chosen?
	\end{enumerate}
	\solution
	%\input{exemplar/11/16/3/12/main1.tex}
\end{enumerate}

	\item 
The number lock of a suitcase has 4 wheels each labelled with ten digits i.e. from 0 to 9.The lock opens with a sequence of four digits with no repeats.What is the probability of a person getting the right sequence to open the suitcase.
\\
\solution
		%\begin{enumerate}[label=\thesection.\arabic*,ref=\thesection.\theenumi]
	\item One card is drawn from a well-shuffled deck of 52 cards. Find the probability of getting
\begin{enumerate}
\item A king of red colour 
\item A face card 
\item A red face card
\item The jack of hearts
\item A spade
\item The queen of diamonds

\end{enumerate}
\solution
		%\input{ncert/10/15/1/14/main.tex}
	\item Five cards—the ten, jack, queen, king and ace of diamonds, are well-shuffled with their face downwards. One card is then picked up at random.
\begin{enumerate}
\item
What is the probability that the card is the queen? 
\item
If the queen is drawn and put aside, what is the probability that the second card picked up is (a) an ace? (b) a queen?\\
\end{enumerate}
\solution
		%\input{ncert/10/15/1/15/defs.tex}
	\item A bag contains $5$ red balls and some blue balls. If the probability of drawing a blue ball is double that if a red ball, determine the number of blue balls in the bag. 
		\\
\solution
		%\input{ncert/10/15/2/3/defs.tex}
	\item A card is selected from a pack of 52 cards.
 \begin{enumerate}[label=(\alph*)] 
                 \item How many points are there in the sample space?
                 \item Calculate the probability that the card is an ace of spades.
                 \item Calculate the probability that the card is (i) an ace and (ii) black card.
 \end{enumerate}
\solution
		%\input{ncert/11/16/3/4/main.tex}
\item Four cards are drawn from a well-shuffled deck of 52 cards. What is the probability of obtaining 3 diamonds and one spade.
\\
\solution
		%\input{ncert/11/16/4/2/defs.tex}
\item In a certain lottery 10,000 tickets are sold and ten equal prizes are awarded. What is the probability of not getting a prize if you buy (a) one ticket (b) two tickets (c) 10 tickets ?	
\\
\solution
		%\input{ncert/11/16/4/4/defs.tex}
		%
\item 
Out of 100 students, two sections of 40 and 60 are formed. If you and your friend are among the 100 students, what is the probability that
\begin{enumerate}
\item you both enter the same section?
\item you both enter the different sections?
\end{enumerate}
\solution
		%\input{ncert/11/16/4/5/defs.tex}
	\item 
The number lock of a suitcase has 4 wheels each labelled with ten digits i.e. from 0 to 9.The lock opens with a sequence of four digits with no repeats.What is the probability of a person getting the right sequence to open the suitcase.
\\
\solution
		%\input{ncert/11/16/4/10/defs.tex}
		%
\item 
Two cards are drawn at random and without replacement from a pack of 52 playing cards. Find the probability that both the cards are black.
\\
\solution
		%\input{ncert/12/13/2/2/defs.tex}
		\item A box of oranges is inspected by examining three randomly selected oranges drawn without replacement. If all the three oranges are good, the box is approved for sale, otherwise, it is rejected. Find the probability that a box containing 15 oranges out of which 12 are good and 3 are bad ones will be approved for sale.
		\label{ncert/12/13/2/3/defs.tex}
		\item Two balls are drawn at random with replacement from a box containing 10 black and 8 red balls. Find the probability that
		\label{ncert/12/13/2/12}
\begin{enumerate}
\item both balls are red.
\item first ball is black and second is red.
\item one of them is black and other is red.
\end{enumerate}

\item In a hostel, 60\% of the students read Hindi newspaper, 40\% read English newspaper and 20\% read both Hindi and English newspapers. A student is selected at random.
		\label{ncert/12/13/2/15}
\begin{enumerate}
\item Find the probability that she reads neither Hindi nor English newspapers.
\item If she reads Hindi newspaper, find the probability that she reads English newspaper.
\item If she reads English newspaper, find the probability that she reads Hindi newspaper.\\
\end{enumerate}
\item The probability of obtaining an even prime number on each die, when a pair of dice is rolled is 
\begin{enumerate}
    \item $0$ 
    
    \item $\frac{1}{3}$ 
    
    \item $\frac{1}{12}$ 
    
    \item $\frac{1}{36}$ 
\end{enumerate}
\solution
		%\input{ncert/12/13/2/17/defs.tex}
	\item A bag contains 4 red and 4 black balls, another bag contains 2 red and 6 black balls. One of the two bags is selected at random and a ball is drawn from the bag which is found to be red. Find the probability that the ball is drawn from the first bag.
\\
\solution
		%\input{ncert/12/13/3/2/main.tex}
  \item
  Cards with numbers 2 to 101 are placed in a box. A card is selected at random.Find the probability that the card has
\begin{enumerate}[label=(\roman*)]
	\item an even number 
	\item a square number
\end{enumerate}
\solution
%\input{exemplar/10/13/3/32/main.tex}
\item
The king, queen and jack of clubs are removed from a deck of 52 playing cards and then well shuffled. Now one card is drawn at random from the remaining cards.  Determine the probability that the card is
\begin{enumerate}[label=(\roman*)]
\item a club
\item 10 of hearts
\end{enumerate}
\solution
%\input{exemplar/10/13/3/29/main.tex}
\item A team of medical students doing their internship have to assist during surgeries
at a city hospital. The probabilities of surgeries rated as very complex, complex,
routine, simple or very simple are respectively, 0.15, 0.20, 0.31, 0.26, .08. Find
the probabilities that a particular surgery will be rated
\begin{enumerate}
	\item complex or very complex;
	\item neither very complex nor very simple;
	\item routine or complex
	\item routine or simple
\end{enumerate}
\solution
%\input{exemplar/11/16/3/8(1)/main.tex}
\item A card is selected from a pack of 52 cards.
\begin{enumerate}[label=(\alph*)]
    \item How many points are there in the sample space?
    \item Calculate the probability that the card is an ace of spades.
    \item Calculate the probability that the card is (i) an ace and (ii) black card.
\end{enumerate}
\solution
%\input{exemplar/11/16/3/4/main2.tex}
\item The probability that a non leap year selected at random will contain 53 sundays.
\\
\solution
%\input{exemplar/10/13/1/19/main.tex}
\item One of the four persons John, Rita, Aslam or Gurpreet will be promoted next
month. Consequently the sample space consists of four elementary outcomes
S = {John promoted, Rita promoted, Aslam promoted, Gurpreet promoted}
You are told that the chances of John’s promotion is same as that of Gurpreet,
Rita’s chances of promotion are twice as likely as Johns. Aslam’s chances are
four times that of John.
\begin{enumerate}
	\item Determine
	\begin{enumerate}
		\item P (John promoted)
		\item P (Rita promoted)
		\item P (Aslam promoted)
		\item P (Gurpreet promoted)
	\end{enumerate}
	\item If A = {John promoted or Gurpreet promoted}, find P (A).
\end{enumerate}
\solution
%\input{exemplar/11/16/3/10/main.tex}
\item A card is drawn from a deck of 52 cards. Find the probability of getting a king or a heart or a red card.\\
\solution
%\input{exemplar/11/16/3/15/main.tex}
\item The probability that a student will pass his examination is 0.73, the probability of
the student getting a compartment is 0.13, and the probability that the student will
either pass or get compartment is 0.96. State True or False.\\
\solution
%\input{exemplar/11/16/3/31/main.tex}
\item A card is selected from a pack of 52 cards\\
\begin{enumerate}[label=(\alph*)]
\item How many points are there in the sample space?
\item Calculate the probability that the cards is an ace of spades.
\item Calculate the probability that the card is (i) an ace (ii)black card.\\
\end{enumerate}
%\input{ncert/11/16/3/4_1/Prob_4.tex}
\item In a non-leap year, the probability of having 53 tuesdays or 53 wednesdays is\\
\solution
%\input{exemplar/11/16/3/18/main.tex}
\item There are 1000 sealed envelopes in a box, 10 of them contain a cash prize of
Rs 100 each, 100 of them contain a cash prize of Rs 50 each and 200 of them
contain a cash prize of Rs 10 each and rest do not contain any cash prize. If they
are well shuffled and an envelope is picked up out, what is the probability that it
contains no cash prize?\\
\solution
%\input{exemplar/10/13/3/34/main.tex}
\item 
A die is thrown and a card is selected at random from a deck of 52 playing cards. The probability of getting an even number on the die and a spade card.\\
\solution
%\input{exemplar/12/13/3/78/main.tex}
\item
If 4-digit numbers greater than 5,000 are randomly formed from the digits 0, 1, 3, 5, and 7, what is the probability of forming a number divisible by 5 when:
\begin{enumerate}
    \item The digits are repeated?
    \item The repetition of digits is not allowed?
\end{enumerate}
\solution
%\input{ncert/11/16/4/9/main.tex}
\item Consider the probability space $\brak{\Omega, \mathcal{G}, P}$ where $\Omega = [0,2]$ and $\mathcal{G} = \cbrak{\phi, \Omega, [0,1], (1,2]}$. Let $X$ and $Y$ be two functions on $\Omega$ defined as
\begin{align*}
    X(\omega) = 
    \begin{cases}
        1 & \text{if }\omega \in [0, 1]\\
        2 & \text{if }\omega \in (1, 2]
    \end{cases}
\end{align*}
and
\begin{align*}
    Y(\omega) = 
    \begin{cases}
        2 & \text{if }\omega \in [0, 1.5]\\
        3 & \text{if }\omega \in (1.5, 2].
    \end{cases}
\end{align*}
Then which one of the following statements is true?
\begin{enumerate}
    \item [(A)] $X$ is a random variable with respect to $\mathcal{G}$, but $Y$ is not a random variable with respect to $\mathcal{G}$.
    \item [(B)] $Y$ is a random variable with respect to $\mathcal{G}$, but $X$ is not a random variable with respect to $\mathcal{G}$.
    \item [(C)] Neither $X$ nor $Y$ is a random variable with respect to $\mathcal{G}$.
    \item [(D)] Both $X$ and $Y$ are random variables with respect to $\mathcal{G}$.
\end{enumerate} \hfill (GATE ST 2023)\\
\solution
%\input{gate/ST/2023/14/main.tex}
	\item  A die is loaded in such a way that each odd number is twice as likely to occur as
each even number. Find $P(G)$, where $G$ is the event that a number greater than
3 occurs on a single roll of the die.
\\
\solution
		%\input{exemplar/11/16/3/5/main.tex}
	\item All the jacks, queens and kings are removed from a deck of 52 playing cards. The remaining cards are well shuffled and then one card is drawn at random. Giving ace a value 1 similar value for other cards, find the probability that the card has a value 
		\begin{enumerate}
			\item 7
			\item greater than 7
			\item less than 7
		\end{enumerate}
		%\input{exemplar/10/13/3/30/main.tex}
  \item A Lot consists of 48 mobile phones of which 42 are good, 3 have only minor defects and 3 have major defects.Varnika will buy a phone if it is good but the trader will only buy a mobile if it has no major defects. One phone is selected at random from the lot. What is the probability that it is
\begin{enumerate}
	\item acceptable to Varnika?
            \item acceptable to the trader?
\end{enumerate}
\solution
	%\input{exemplar/10/13/3/40/main.tex}
 \item A student says that if you throw a die, it will show up 1 or not 1. Therefore, the probability of getting 1 and the probability of getting 'not 1' each is equal to $\frac{1}{2}$. Is this correct? Give reasons.\\
 \solution
        %\input{exemplar/10/13/2/9/main.tex}
   \item Four candidates A, B, C, D have ap-
plied for the assignment to coach a school cricket
team. If A is twice as likely to be selected as B, and
B and C are given about the same chance of being
selected, while C is twice as likely to be selected
as D, what are the probabilities that
\begin{enumerate}
\item C will be selected?
\item A will not be selected?
\end{enumerate}
	%\input{exemplar/11/16/3/9/main.tex}
 \item A bag contain 24 balls of which $x$ balls are red, $2x$ are white and $3x$ are blue. A ball is selected at random, What is the probability that it is
\begin{enumerate}[label=\alph*)]
\item not red ?
\item white ?
\end{enumerate}
%\input{exemplar/10/13/3/41/main.tex}
If the letters of the word ASSASSINATION are arranged at random. Find the Probability that
\begin{enumerate}[label=(\alph*)]
\item Four $S's$ come consecutively in the word
\item Two  $I's$ and two $N's$ come together
\item All $A's$ are not coming together
\item No two $A's$ are coming together
\end{enumerate}
%\input{exemplar/11/16/3/14/main.tex}
	\item One urn contains two black balls (labelled B1 and B2) and one white ball. A
	second urn contains one black ball and two white balls (labelled W1 and W2).
	Suppose the following experiment is performed. One of the two urns is chosen
	at random. Next a ball is randomly chosen from the urn. Then a second ball is
	chosen at random from the same urn without replacing the first ball.
	
	\begin{enumerate}
	\item What is the probability that two black balls are chosen?
	
	\item What is the probability that two balls of opposite colour are chosen?
	\end{enumerate}
	\solution
	%\input{exemplar/11/16/3/12/main1.tex}
\end{enumerate}

		%
\item 
Two cards are drawn at random and without replacement from a pack of 52 playing cards. Find the probability that both the cards are black.
\\
\solution
		%\begin{enumerate}[label=\thesection.\arabic*,ref=\thesection.\theenumi]
	\item One card is drawn from a well-shuffled deck of 52 cards. Find the probability of getting
\begin{enumerate}
\item A king of red colour 
\item A face card 
\item A red face card
\item The jack of hearts
\item A spade
\item The queen of diamonds

\end{enumerate}
\solution
		%\input{ncert/10/15/1/14/main.tex}
	\item Five cards—the ten, jack, queen, king and ace of diamonds, are well-shuffled with their face downwards. One card is then picked up at random.
\begin{enumerate}
\item
What is the probability that the card is the queen? 
\item
If the queen is drawn and put aside, what is the probability that the second card picked up is (a) an ace? (b) a queen?\\
\end{enumerate}
\solution
		%\input{ncert/10/15/1/15/defs.tex}
	\item A bag contains $5$ red balls and some blue balls. If the probability of drawing a blue ball is double that if a red ball, determine the number of blue balls in the bag. 
		\\
\solution
		%\input{ncert/10/15/2/3/defs.tex}
	\item A card is selected from a pack of 52 cards.
 \begin{enumerate}[label=(\alph*)] 
                 \item How many points are there in the sample space?
                 \item Calculate the probability that the card is an ace of spades.
                 \item Calculate the probability that the card is (i) an ace and (ii) black card.
 \end{enumerate}
\solution
		%\input{ncert/11/16/3/4/main.tex}
\item Four cards are drawn from a well-shuffled deck of 52 cards. What is the probability of obtaining 3 diamonds and one spade.
\\
\solution
		%\input{ncert/11/16/4/2/defs.tex}
\item In a certain lottery 10,000 tickets are sold and ten equal prizes are awarded. What is the probability of not getting a prize if you buy (a) one ticket (b) two tickets (c) 10 tickets ?	
\\
\solution
		%\input{ncert/11/16/4/4/defs.tex}
		%
\item 
Out of 100 students, two sections of 40 and 60 are formed. If you and your friend are among the 100 students, what is the probability that
\begin{enumerate}
\item you both enter the same section?
\item you both enter the different sections?
\end{enumerate}
\solution
		%\input{ncert/11/16/4/5/defs.tex}
	\item 
The number lock of a suitcase has 4 wheels each labelled with ten digits i.e. from 0 to 9.The lock opens with a sequence of four digits with no repeats.What is the probability of a person getting the right sequence to open the suitcase.
\\
\solution
		%\input{ncert/11/16/4/10/defs.tex}
		%
\item 
Two cards are drawn at random and without replacement from a pack of 52 playing cards. Find the probability that both the cards are black.
\\
\solution
		%\input{ncert/12/13/2/2/defs.tex}
		\item A box of oranges is inspected by examining three randomly selected oranges drawn without replacement. If all the three oranges are good, the box is approved for sale, otherwise, it is rejected. Find the probability that a box containing 15 oranges out of which 12 are good and 3 are bad ones will be approved for sale.
		\label{ncert/12/13/2/3/defs.tex}
		\item Two balls are drawn at random with replacement from a box containing 10 black and 8 red balls. Find the probability that
		\label{ncert/12/13/2/12}
\begin{enumerate}
\item both balls are red.
\item first ball is black and second is red.
\item one of them is black and other is red.
\end{enumerate}

\item In a hostel, 60\% of the students read Hindi newspaper, 40\% read English newspaper and 20\% read both Hindi and English newspapers. A student is selected at random.
		\label{ncert/12/13/2/15}
\begin{enumerate}
\item Find the probability that she reads neither Hindi nor English newspapers.
\item If she reads Hindi newspaper, find the probability that she reads English newspaper.
\item If she reads English newspaper, find the probability that she reads Hindi newspaper.\\
\end{enumerate}
\item The probability of obtaining an even prime number on each die, when a pair of dice is rolled is 
\begin{enumerate}
    \item $0$ 
    
    \item $\frac{1}{3}$ 
    
    \item $\frac{1}{12}$ 
    
    \item $\frac{1}{36}$ 
\end{enumerate}
\solution
		%\input{ncert/12/13/2/17/defs.tex}
	\item A bag contains 4 red and 4 black balls, another bag contains 2 red and 6 black balls. One of the two bags is selected at random and a ball is drawn from the bag which is found to be red. Find the probability that the ball is drawn from the first bag.
\\
\solution
		%\input{ncert/12/13/3/2/main.tex}
  \item
  Cards with numbers 2 to 101 are placed in a box. A card is selected at random.Find the probability that the card has
\begin{enumerate}[label=(\roman*)]
	\item an even number 
	\item a square number
\end{enumerate}
\solution
%\input{exemplar/10/13/3/32/main.tex}
\item
The king, queen and jack of clubs are removed from a deck of 52 playing cards and then well shuffled. Now one card is drawn at random from the remaining cards.  Determine the probability that the card is
\begin{enumerate}[label=(\roman*)]
\item a club
\item 10 of hearts
\end{enumerate}
\solution
%\input{exemplar/10/13/3/29/main.tex}
\item A team of medical students doing their internship have to assist during surgeries
at a city hospital. The probabilities of surgeries rated as very complex, complex,
routine, simple or very simple are respectively, 0.15, 0.20, 0.31, 0.26, .08. Find
the probabilities that a particular surgery will be rated
\begin{enumerate}
	\item complex or very complex;
	\item neither very complex nor very simple;
	\item routine or complex
	\item routine or simple
\end{enumerate}
\solution
%\input{exemplar/11/16/3/8(1)/main.tex}
\item A card is selected from a pack of 52 cards.
\begin{enumerate}[label=(\alph*)]
    \item How many points are there in the sample space?
    \item Calculate the probability that the card is an ace of spades.
    \item Calculate the probability that the card is (i) an ace and (ii) black card.
\end{enumerate}
\solution
%\input{exemplar/11/16/3/4/main2.tex}
\item The probability that a non leap year selected at random will contain 53 sundays.
\\
\solution
%\input{exemplar/10/13/1/19/main.tex}
\item One of the four persons John, Rita, Aslam or Gurpreet will be promoted next
month. Consequently the sample space consists of four elementary outcomes
S = {John promoted, Rita promoted, Aslam promoted, Gurpreet promoted}
You are told that the chances of John’s promotion is same as that of Gurpreet,
Rita’s chances of promotion are twice as likely as Johns. Aslam’s chances are
four times that of John.
\begin{enumerate}
	\item Determine
	\begin{enumerate}
		\item P (John promoted)
		\item P (Rita promoted)
		\item P (Aslam promoted)
		\item P (Gurpreet promoted)
	\end{enumerate}
	\item If A = {John promoted or Gurpreet promoted}, find P (A).
\end{enumerate}
\solution
%\input{exemplar/11/16/3/10/main.tex}
\item A card is drawn from a deck of 52 cards. Find the probability of getting a king or a heart or a red card.\\
\solution
%\input{exemplar/11/16/3/15/main.tex}
\item The probability that a student will pass his examination is 0.73, the probability of
the student getting a compartment is 0.13, and the probability that the student will
either pass or get compartment is 0.96. State True or False.\\
\solution
%\input{exemplar/11/16/3/31/main.tex}
\item A card is selected from a pack of 52 cards\\
\begin{enumerate}[label=(\alph*)]
\item How many points are there in the sample space?
\item Calculate the probability that the cards is an ace of spades.
\item Calculate the probability that the card is (i) an ace (ii)black card.\\
\end{enumerate}
%\input{ncert/11/16/3/4_1/Prob_4.tex}
\item In a non-leap year, the probability of having 53 tuesdays or 53 wednesdays is\\
\solution
%\input{exemplar/11/16/3/18/main.tex}
\item There are 1000 sealed envelopes in a box, 10 of them contain a cash prize of
Rs 100 each, 100 of them contain a cash prize of Rs 50 each and 200 of them
contain a cash prize of Rs 10 each and rest do not contain any cash prize. If they
are well shuffled and an envelope is picked up out, what is the probability that it
contains no cash prize?\\
\solution
%\input{exemplar/10/13/3/34/main.tex}
\item 
A die is thrown and a card is selected at random from a deck of 52 playing cards. The probability of getting an even number on the die and a spade card.\\
\solution
%\input{exemplar/12/13/3/78/main.tex}
\item
If 4-digit numbers greater than 5,000 are randomly formed from the digits 0, 1, 3, 5, and 7, what is the probability of forming a number divisible by 5 when:
\begin{enumerate}
    \item The digits are repeated?
    \item The repetition of digits is not allowed?
\end{enumerate}
\solution
%\input{ncert/11/16/4/9/main.tex}
\item Consider the probability space $\brak{\Omega, \mathcal{G}, P}$ where $\Omega = [0,2]$ and $\mathcal{G} = \cbrak{\phi, \Omega, [0,1], (1,2]}$. Let $X$ and $Y$ be two functions on $\Omega$ defined as
\begin{align*}
    X(\omega) = 
    \begin{cases}
        1 & \text{if }\omega \in [0, 1]\\
        2 & \text{if }\omega \in (1, 2]
    \end{cases}
\end{align*}
and
\begin{align*}
    Y(\omega) = 
    \begin{cases}
        2 & \text{if }\omega \in [0, 1.5]\\
        3 & \text{if }\omega \in (1.5, 2].
    \end{cases}
\end{align*}
Then which one of the following statements is true?
\begin{enumerate}
    \item [(A)] $X$ is a random variable with respect to $\mathcal{G}$, but $Y$ is not a random variable with respect to $\mathcal{G}$.
    \item [(B)] $Y$ is a random variable with respect to $\mathcal{G}$, but $X$ is not a random variable with respect to $\mathcal{G}$.
    \item [(C)] Neither $X$ nor $Y$ is a random variable with respect to $\mathcal{G}$.
    \item [(D)] Both $X$ and $Y$ are random variables with respect to $\mathcal{G}$.
\end{enumerate} \hfill (GATE ST 2023)\\
\solution
%\input{gate/ST/2023/14/main.tex}
	\item  A die is loaded in such a way that each odd number is twice as likely to occur as
each even number. Find $P(G)$, where $G$ is the event that a number greater than
3 occurs on a single roll of the die.
\\
\solution
		%\input{exemplar/11/16/3/5/main.tex}
	\item All the jacks, queens and kings are removed from a deck of 52 playing cards. The remaining cards are well shuffled and then one card is drawn at random. Giving ace a value 1 similar value for other cards, find the probability that the card has a value 
		\begin{enumerate}
			\item 7
			\item greater than 7
			\item less than 7
		\end{enumerate}
		%\input{exemplar/10/13/3/30/main.tex}
  \item A Lot consists of 48 mobile phones of which 42 are good, 3 have only minor defects and 3 have major defects.Varnika will buy a phone if it is good but the trader will only buy a mobile if it has no major defects. One phone is selected at random from the lot. What is the probability that it is
\begin{enumerate}
	\item acceptable to Varnika?
            \item acceptable to the trader?
\end{enumerate}
\solution
	%\input{exemplar/10/13/3/40/main.tex}
 \item A student says that if you throw a die, it will show up 1 or not 1. Therefore, the probability of getting 1 and the probability of getting 'not 1' each is equal to $\frac{1}{2}$. Is this correct? Give reasons.\\
 \solution
        %\input{exemplar/10/13/2/9/main.tex}
   \item Four candidates A, B, C, D have ap-
plied for the assignment to coach a school cricket
team. If A is twice as likely to be selected as B, and
B and C are given about the same chance of being
selected, while C is twice as likely to be selected
as D, what are the probabilities that
\begin{enumerate}
\item C will be selected?
\item A will not be selected?
\end{enumerate}
	%\input{exemplar/11/16/3/9/main.tex}
 \item A bag contain 24 balls of which $x$ balls are red, $2x$ are white and $3x$ are blue. A ball is selected at random, What is the probability that it is
\begin{enumerate}[label=\alph*)]
\item not red ?
\item white ?
\end{enumerate}
%\input{exemplar/10/13/3/41/main.tex}
If the letters of the word ASSASSINATION are arranged at random. Find the Probability that
\begin{enumerate}[label=(\alph*)]
\item Four $S's$ come consecutively in the word
\item Two  $I's$ and two $N's$ come together
\item All $A's$ are not coming together
\item No two $A's$ are coming together
\end{enumerate}
%\input{exemplar/11/16/3/14/main.tex}
	\item One urn contains two black balls (labelled B1 and B2) and one white ball. A
	second urn contains one black ball and two white balls (labelled W1 and W2).
	Suppose the following experiment is performed. One of the two urns is chosen
	at random. Next a ball is randomly chosen from the urn. Then a second ball is
	chosen at random from the same urn without replacing the first ball.
	
	\begin{enumerate}
	\item What is the probability that two black balls are chosen?
	
	\item What is the probability that two balls of opposite colour are chosen?
	\end{enumerate}
	\solution
	%\input{exemplar/11/16/3/12/main1.tex}
\end{enumerate}

		\item A box of oranges is inspected by examining three randomly selected oranges drawn without replacement. If all the three oranges are good, the box is approved for sale, otherwise, it is rejected. Find the probability that a box containing 15 oranges out of which 12 are good and 3 are bad ones will be approved for sale.
		\label{ncert/12/13/2/3/defs.tex}
		\item Two balls are drawn at random with replacement from a box containing 10 black and 8 red balls. Find the probability that
		\label{ncert/12/13/2/12}
\begin{enumerate}
\item both balls are red.
\item first ball is black and second is red.
\item one of them is black and other is red.
\end{enumerate}

\item In a hostel, 60\% of the students read Hindi newspaper, 40\% read English newspaper and 20\% read both Hindi and English newspapers. A student is selected at random.
		\label{ncert/12/13/2/15}
\begin{enumerate}
\item Find the probability that she reads neither Hindi nor English newspapers.
\item If she reads Hindi newspaper, find the probability that she reads English newspaper.
\item If she reads English newspaper, find the probability that she reads Hindi newspaper.\\
\end{enumerate}
\item The probability of obtaining an even prime number on each die, when a pair of dice is rolled is 
\begin{enumerate}
    \item $0$ 
    
    \item $\frac{1}{3}$ 
    
    \item $\frac{1}{12}$ 
    
    \item $\frac{1}{36}$ 
\end{enumerate}
\solution
		%\begin{enumerate}[label=\thesection.\arabic*,ref=\thesection.\theenumi]
	\item One card is drawn from a well-shuffled deck of 52 cards. Find the probability of getting
\begin{enumerate}
\item A king of red colour 
\item A face card 
\item A red face card
\item The jack of hearts
\item A spade
\item The queen of diamonds

\end{enumerate}
\solution
		%\input{ncert/10/15/1/14/main.tex}
	\item Five cards—the ten, jack, queen, king and ace of diamonds, are well-shuffled with their face downwards. One card is then picked up at random.
\begin{enumerate}
\item
What is the probability that the card is the queen? 
\item
If the queen is drawn and put aside, what is the probability that the second card picked up is (a) an ace? (b) a queen?\\
\end{enumerate}
\solution
		%\input{ncert/10/15/1/15/defs.tex}
	\item A bag contains $5$ red balls and some blue balls. If the probability of drawing a blue ball is double that if a red ball, determine the number of blue balls in the bag. 
		\\
\solution
		%\input{ncert/10/15/2/3/defs.tex}
	\item A card is selected from a pack of 52 cards.
 \begin{enumerate}[label=(\alph*)] 
                 \item How many points are there in the sample space?
                 \item Calculate the probability that the card is an ace of spades.
                 \item Calculate the probability that the card is (i) an ace and (ii) black card.
 \end{enumerate}
\solution
		%\input{ncert/11/16/3/4/main.tex}
\item Four cards are drawn from a well-shuffled deck of 52 cards. What is the probability of obtaining 3 diamonds and one spade.
\\
\solution
		%\input{ncert/11/16/4/2/defs.tex}
\item In a certain lottery 10,000 tickets are sold and ten equal prizes are awarded. What is the probability of not getting a prize if you buy (a) one ticket (b) two tickets (c) 10 tickets ?	
\\
\solution
		%\input{ncert/11/16/4/4/defs.tex}
		%
\item 
Out of 100 students, two sections of 40 and 60 are formed. If you and your friend are among the 100 students, what is the probability that
\begin{enumerate}
\item you both enter the same section?
\item you both enter the different sections?
\end{enumerate}
\solution
		%\input{ncert/11/16/4/5/defs.tex}
	\item 
The number lock of a suitcase has 4 wheels each labelled with ten digits i.e. from 0 to 9.The lock opens with a sequence of four digits with no repeats.What is the probability of a person getting the right sequence to open the suitcase.
\\
\solution
		%\input{ncert/11/16/4/10/defs.tex}
		%
\item 
Two cards are drawn at random and without replacement from a pack of 52 playing cards. Find the probability that both the cards are black.
\\
\solution
		%\input{ncert/12/13/2/2/defs.tex}
		\item A box of oranges is inspected by examining three randomly selected oranges drawn without replacement. If all the three oranges are good, the box is approved for sale, otherwise, it is rejected. Find the probability that a box containing 15 oranges out of which 12 are good and 3 are bad ones will be approved for sale.
		\label{ncert/12/13/2/3/defs.tex}
		\item Two balls are drawn at random with replacement from a box containing 10 black and 8 red balls. Find the probability that
		\label{ncert/12/13/2/12}
\begin{enumerate}
\item both balls are red.
\item first ball is black and second is red.
\item one of them is black and other is red.
\end{enumerate}

\item In a hostel, 60\% of the students read Hindi newspaper, 40\% read English newspaper and 20\% read both Hindi and English newspapers. A student is selected at random.
		\label{ncert/12/13/2/15}
\begin{enumerate}
\item Find the probability that she reads neither Hindi nor English newspapers.
\item If she reads Hindi newspaper, find the probability that she reads English newspaper.
\item If she reads English newspaper, find the probability that she reads Hindi newspaper.\\
\end{enumerate}
\item The probability of obtaining an even prime number on each die, when a pair of dice is rolled is 
\begin{enumerate}
    \item $0$ 
    
    \item $\frac{1}{3}$ 
    
    \item $\frac{1}{12}$ 
    
    \item $\frac{1}{36}$ 
\end{enumerate}
\solution
		%\input{ncert/12/13/2/17/defs.tex}
	\item A bag contains 4 red and 4 black balls, another bag contains 2 red and 6 black balls. One of the two bags is selected at random and a ball is drawn from the bag which is found to be red. Find the probability that the ball is drawn from the first bag.
\\
\solution
		%\input{ncert/12/13/3/2/main.tex}
  \item
  Cards with numbers 2 to 101 are placed in a box. A card is selected at random.Find the probability that the card has
\begin{enumerate}[label=(\roman*)]
	\item an even number 
	\item a square number
\end{enumerate}
\solution
%\input{exemplar/10/13/3/32/main.tex}
\item
The king, queen and jack of clubs are removed from a deck of 52 playing cards and then well shuffled. Now one card is drawn at random from the remaining cards.  Determine the probability that the card is
\begin{enumerate}[label=(\roman*)]
\item a club
\item 10 of hearts
\end{enumerate}
\solution
%\input{exemplar/10/13/3/29/main.tex}
\item A team of medical students doing their internship have to assist during surgeries
at a city hospital. The probabilities of surgeries rated as very complex, complex,
routine, simple or very simple are respectively, 0.15, 0.20, 0.31, 0.26, .08. Find
the probabilities that a particular surgery will be rated
\begin{enumerate}
	\item complex or very complex;
	\item neither very complex nor very simple;
	\item routine or complex
	\item routine or simple
\end{enumerate}
\solution
%\input{exemplar/11/16/3/8(1)/main.tex}
\item A card is selected from a pack of 52 cards.
\begin{enumerate}[label=(\alph*)]
    \item How many points are there in the sample space?
    \item Calculate the probability that the card is an ace of spades.
    \item Calculate the probability that the card is (i) an ace and (ii) black card.
\end{enumerate}
\solution
%\input{exemplar/11/16/3/4/main2.tex}
\item The probability that a non leap year selected at random will contain 53 sundays.
\\
\solution
%\input{exemplar/10/13/1/19/main.tex}
\item One of the four persons John, Rita, Aslam or Gurpreet will be promoted next
month. Consequently the sample space consists of four elementary outcomes
S = {John promoted, Rita promoted, Aslam promoted, Gurpreet promoted}
You are told that the chances of John’s promotion is same as that of Gurpreet,
Rita’s chances of promotion are twice as likely as Johns. Aslam’s chances are
four times that of John.
\begin{enumerate}
	\item Determine
	\begin{enumerate}
		\item P (John promoted)
		\item P (Rita promoted)
		\item P (Aslam promoted)
		\item P (Gurpreet promoted)
	\end{enumerate}
	\item If A = {John promoted or Gurpreet promoted}, find P (A).
\end{enumerate}
\solution
%\input{exemplar/11/16/3/10/main.tex}
\item A card is drawn from a deck of 52 cards. Find the probability of getting a king or a heart or a red card.\\
\solution
%\input{exemplar/11/16/3/15/main.tex}
\item The probability that a student will pass his examination is 0.73, the probability of
the student getting a compartment is 0.13, and the probability that the student will
either pass or get compartment is 0.96. State True or False.\\
\solution
%\input{exemplar/11/16/3/31/main.tex}
\item A card is selected from a pack of 52 cards\\
\begin{enumerate}[label=(\alph*)]
\item How many points are there in the sample space?
\item Calculate the probability that the cards is an ace of spades.
\item Calculate the probability that the card is (i) an ace (ii)black card.\\
\end{enumerate}
%\input{ncert/11/16/3/4_1/Prob_4.tex}
\item In a non-leap year, the probability of having 53 tuesdays or 53 wednesdays is\\
\solution
%\input{exemplar/11/16/3/18/main.tex}
\item There are 1000 sealed envelopes in a box, 10 of them contain a cash prize of
Rs 100 each, 100 of them contain a cash prize of Rs 50 each and 200 of them
contain a cash prize of Rs 10 each and rest do not contain any cash prize. If they
are well shuffled and an envelope is picked up out, what is the probability that it
contains no cash prize?\\
\solution
%\input{exemplar/10/13/3/34/main.tex}
\item 
A die is thrown and a card is selected at random from a deck of 52 playing cards. The probability of getting an even number on the die and a spade card.\\
\solution
%\input{exemplar/12/13/3/78/main.tex}
\item
If 4-digit numbers greater than 5,000 are randomly formed from the digits 0, 1, 3, 5, and 7, what is the probability of forming a number divisible by 5 when:
\begin{enumerate}
    \item The digits are repeated?
    \item The repetition of digits is not allowed?
\end{enumerate}
\solution
%\input{ncert/11/16/4/9/main.tex}
\item Consider the probability space $\brak{\Omega, \mathcal{G}, P}$ where $\Omega = [0,2]$ and $\mathcal{G} = \cbrak{\phi, \Omega, [0,1], (1,2]}$. Let $X$ and $Y$ be two functions on $\Omega$ defined as
\begin{align*}
    X(\omega) = 
    \begin{cases}
        1 & \text{if }\omega \in [0, 1]\\
        2 & \text{if }\omega \in (1, 2]
    \end{cases}
\end{align*}
and
\begin{align*}
    Y(\omega) = 
    \begin{cases}
        2 & \text{if }\omega \in [0, 1.5]\\
        3 & \text{if }\omega \in (1.5, 2].
    \end{cases}
\end{align*}
Then which one of the following statements is true?
\begin{enumerate}
    \item [(A)] $X$ is a random variable with respect to $\mathcal{G}$, but $Y$ is not a random variable with respect to $\mathcal{G}$.
    \item [(B)] $Y$ is a random variable with respect to $\mathcal{G}$, but $X$ is not a random variable with respect to $\mathcal{G}$.
    \item [(C)] Neither $X$ nor $Y$ is a random variable with respect to $\mathcal{G}$.
    \item [(D)] Both $X$ and $Y$ are random variables with respect to $\mathcal{G}$.
\end{enumerate} \hfill (GATE ST 2023)\\
\solution
%\input{gate/ST/2023/14/main.tex}
	\item  A die is loaded in such a way that each odd number is twice as likely to occur as
each even number. Find $P(G)$, where $G$ is the event that a number greater than
3 occurs on a single roll of the die.
\\
\solution
		%\input{exemplar/11/16/3/5/main.tex}
	\item All the jacks, queens and kings are removed from a deck of 52 playing cards. The remaining cards are well shuffled and then one card is drawn at random. Giving ace a value 1 similar value for other cards, find the probability that the card has a value 
		\begin{enumerate}
			\item 7
			\item greater than 7
			\item less than 7
		\end{enumerate}
		%\input{exemplar/10/13/3/30/main.tex}
  \item A Lot consists of 48 mobile phones of which 42 are good, 3 have only minor defects and 3 have major defects.Varnika will buy a phone if it is good but the trader will only buy a mobile if it has no major defects. One phone is selected at random from the lot. What is the probability that it is
\begin{enumerate}
	\item acceptable to Varnika?
            \item acceptable to the trader?
\end{enumerate}
\solution
	%\input{exemplar/10/13/3/40/main.tex}
 \item A student says that if you throw a die, it will show up 1 or not 1. Therefore, the probability of getting 1 and the probability of getting 'not 1' each is equal to $\frac{1}{2}$. Is this correct? Give reasons.\\
 \solution
        %\input{exemplar/10/13/2/9/main.tex}
   \item Four candidates A, B, C, D have ap-
plied for the assignment to coach a school cricket
team. If A is twice as likely to be selected as B, and
B and C are given about the same chance of being
selected, while C is twice as likely to be selected
as D, what are the probabilities that
\begin{enumerate}
\item C will be selected?
\item A will not be selected?
\end{enumerate}
	%\input{exemplar/11/16/3/9/main.tex}
 \item A bag contain 24 balls of which $x$ balls are red, $2x$ are white and $3x$ are blue. A ball is selected at random, What is the probability that it is
\begin{enumerate}[label=\alph*)]
\item not red ?
\item white ?
\end{enumerate}
%\input{exemplar/10/13/3/41/main.tex}
If the letters of the word ASSASSINATION are arranged at random. Find the Probability that
\begin{enumerate}[label=(\alph*)]
\item Four $S's$ come consecutively in the word
\item Two  $I's$ and two $N's$ come together
\item All $A's$ are not coming together
\item No two $A's$ are coming together
\end{enumerate}
%\input{exemplar/11/16/3/14/main.tex}
	\item One urn contains two black balls (labelled B1 and B2) and one white ball. A
	second urn contains one black ball and two white balls (labelled W1 and W2).
	Suppose the following experiment is performed. One of the two urns is chosen
	at random. Next a ball is randomly chosen from the urn. Then a second ball is
	chosen at random from the same urn without replacing the first ball.
	
	\begin{enumerate}
	\item What is the probability that two black balls are chosen?
	
	\item What is the probability that two balls of opposite colour are chosen?
	\end{enumerate}
	\solution
	%\input{exemplar/11/16/3/12/main1.tex}
\end{enumerate}

	\item A bag contains 4 red and 4 black balls, another bag contains 2 red and 6 black balls. One of the two bags is selected at random and a ball is drawn from the bag which is found to be red. Find the probability that the ball is drawn from the first bag.
\\
\solution
		%\begin{table}[H]
	\centering
\begin{tabular}{|c|c|c|}
\hline
Random variable &Value &Definition\\ \hline
\multirow{3}{*}{X} &0 &Slips of Rs 1\\
&1 &Slips of Rs 5\\
&2 &Slips of Rs 13\\ \hline
\multirow{2}{*}{Y} &0 &Box A\\
&1 &Box B\\\hline
\end{tabular}
\caption{}
\label{tab:Distribution}
\end{table}
See \tabref{tab:Distribution}.
\begin{align}
p_{Y}\brak{k}= \begin{cases} 
      \frac{1}{3} & {k=0} \\
      \frac{2}{3 }& {k=1} 
   \end{cases}
   \\
p_{Y|X}\brak{0|0} = \frac{19}{25}\, 
p_{Y|X}\brak{0|1} = \frac{6}{25}\,
p_{Y|X}\brak{1|0} = \frac{45}{50}\,
p_{Y|X}\brak{1|2} = \frac{5}{50}
\end{align}
The desired probability is the probability that a slip drawn at random is marked other than Rs 1,
\begin{align}
&=1-p_X\brak{0}\\
&= p_X(1) + p_X(2)
\end{align}
Using Bayes theorem,
\begin{align}
&= p_Y\brak{0} \times \pr{Y=0 | X=1} + p_Y\brak{1} \times \pr{Y=1|X=2}\\
&=\frac{1}{3} \times \frac{6}{25} + \frac{2}{3} \times \frac{5}{50}\\
&=\frac{11}{75}
\end{align}

\newpage

%\tableofcontents

\bigskip

\renewcommand{\thefigure}{\theenumi}
\renewcommand{\thetable}{\theenumi}
%\renewcommand{\theequation}{\theenumi}

%\begin{abstract}
%%\boldmath
%In this letter, an algorithm for evaluating the exact analytical bit error rate  (BER)  for the piecewise linear (PL) combiner for  multiple relays is presented. Previous results were available only for upto three relays. The algorithm is unique in the sense that  the actual mathematical expressions, that are prohibitively large, need not be explicitly obtained. The diversity gain due to multiple relays is shown through plots of the analytical BER, well supported by simulations. 
%
%\end{abstract}
% IEEEtran.cls defaults to using nonbold math in the Abstract.
% This preserves the distinction between vectors and scalars. However,
% if the journal you are submitting to favors bold math in the abstract,
% then you can use LaTeX's standard command \boldmath at the very start
% of the abstract to achieve this. Many IEEE journals frown on math
% in the abstract anyway.

% Note that keywords are not normally used for peerreview papers.
%\begin{IEEEkeywords}
%Cooperative diversity, decode and forward, piecewise linear
%\end{IEEEkeywords}



% For peer review papers, you can put extra information on the cover
% page as needed:
% \ifCLASSOPTIONpeerreview
% \begin{center} \bfseries EDICS Category: 3-BBND \end{center}
% \fi
%
% For peerreview papers, this IEEEtran command inserts a page break and
% creates the second title. It will be ignored for other modes.
%\IEEEpeerreviewmaketitle




  \item
  Cards with numbers 2 to 101 are placed in a box. A card is selected at random.Find the probability that the card has
\begin{enumerate}[label=(\roman*)]
	\item an even number 
	\item a square number
\end{enumerate}
\solution
%\begin{table}[H]
	\centering
\begin{tabular}{|c|c|c|}
\hline
Random variable &Value &Definition\\ \hline
\multirow{3}{*}{X} &0 &Slips of Rs 1\\
&1 &Slips of Rs 5\\
&2 &Slips of Rs 13\\ \hline
\multirow{2}{*}{Y} &0 &Box A\\
&1 &Box B\\\hline
\end{tabular}
\caption{}
\label{tab:Distribution}
\end{table}
See \tabref{tab:Distribution}.
\begin{align}
p_{Y}\brak{k}= \begin{cases} 
      \frac{1}{3} & {k=0} \\
      \frac{2}{3 }& {k=1} 
   \end{cases}
   \\
p_{Y|X}\brak{0|0} = \frac{19}{25}\, 
p_{Y|X}\brak{0|1} = \frac{6}{25}\,
p_{Y|X}\brak{1|0} = \frac{45}{50}\,
p_{Y|X}\brak{1|2} = \frac{5}{50}
\end{align}
The desired probability is the probability that a slip drawn at random is marked other than Rs 1,
\begin{align}
&=1-p_X\brak{0}\\
&= p_X(1) + p_X(2)
\end{align}
Using Bayes theorem,
\begin{align}
&= p_Y\brak{0} \times \pr{Y=0 | X=1} + p_Y\brak{1} \times \pr{Y=1|X=2}\\
&=\frac{1}{3} \times \frac{6}{25} + \frac{2}{3} \times \frac{5}{50}\\
&=\frac{11}{75}
\end{align}

\newpage

%\tableofcontents

\bigskip

\renewcommand{\thefigure}{\theenumi}
\renewcommand{\thetable}{\theenumi}
%\renewcommand{\theequation}{\theenumi}

%\begin{abstract}
%%\boldmath
%In this letter, an algorithm for evaluating the exact analytical bit error rate  (BER)  for the piecewise linear (PL) combiner for  multiple relays is presented. Previous results were available only for upto three relays. The algorithm is unique in the sense that  the actual mathematical expressions, that are prohibitively large, need not be explicitly obtained. The diversity gain due to multiple relays is shown through plots of the analytical BER, well supported by simulations. 
%
%\end{abstract}
% IEEEtran.cls defaults to using nonbold math in the Abstract.
% This preserves the distinction between vectors and scalars. However,
% if the journal you are submitting to favors bold math in the abstract,
% then you can use LaTeX's standard command \boldmath at the very start
% of the abstract to achieve this. Many IEEE journals frown on math
% in the abstract anyway.

% Note that keywords are not normally used for peerreview papers.
%\begin{IEEEkeywords}
%Cooperative diversity, decode and forward, piecewise linear
%\end{IEEEkeywords}



% For peer review papers, you can put extra information on the cover
% page as needed:
% \ifCLASSOPTIONpeerreview
% \begin{center} \bfseries EDICS Category: 3-BBND \end{center}
% \fi
%
% For peerreview papers, this IEEEtran command inserts a page break and
% creates the second title. It will be ignored for other modes.
%\IEEEpeerreviewmaketitle




\item
The king, queen and jack of clubs are removed from a deck of 52 playing cards and then well shuffled. Now one card is drawn at random from the remaining cards.  Determine the probability that the card is
\begin{enumerate}[label=(\roman*)]
\item a club
\item 10 of hearts
\end{enumerate}
\solution
%\begin{table}[H]
	\centering
\begin{tabular}{|c|c|c|}
\hline
Random variable &Value &Definition\\ \hline
\multirow{3}{*}{X} &0 &Slips of Rs 1\\
&1 &Slips of Rs 5\\
&2 &Slips of Rs 13\\ \hline
\multirow{2}{*}{Y} &0 &Box A\\
&1 &Box B\\\hline
\end{tabular}
\caption{}
\label{tab:Distribution}
\end{table}
See \tabref{tab:Distribution}.
\begin{align}
p_{Y}\brak{k}= \begin{cases} 
      \frac{1}{3} & {k=0} \\
      \frac{2}{3 }& {k=1} 
   \end{cases}
   \\
p_{Y|X}\brak{0|0} = \frac{19}{25}\, 
p_{Y|X}\brak{0|1} = \frac{6}{25}\,
p_{Y|X}\brak{1|0} = \frac{45}{50}\,
p_{Y|X}\brak{1|2} = \frac{5}{50}
\end{align}
The desired probability is the probability that a slip drawn at random is marked other than Rs 1,
\begin{align}
&=1-p_X\brak{0}\\
&= p_X(1) + p_X(2)
\end{align}
Using Bayes theorem,
\begin{align}
&= p_Y\brak{0} \times \pr{Y=0 | X=1} + p_Y\brak{1} \times \pr{Y=1|X=2}\\
&=\frac{1}{3} \times \frac{6}{25} + \frac{2}{3} \times \frac{5}{50}\\
&=\frac{11}{75}
\end{align}

\newpage

%\tableofcontents

\bigskip

\renewcommand{\thefigure}{\theenumi}
\renewcommand{\thetable}{\theenumi}
%\renewcommand{\theequation}{\theenumi}

%\begin{abstract}
%%\boldmath
%In this letter, an algorithm for evaluating the exact analytical bit error rate  (BER)  for the piecewise linear (PL) combiner for  multiple relays is presented. Previous results were available only for upto three relays. The algorithm is unique in the sense that  the actual mathematical expressions, that are prohibitively large, need not be explicitly obtained. The diversity gain due to multiple relays is shown through plots of the analytical BER, well supported by simulations. 
%
%\end{abstract}
% IEEEtran.cls defaults to using nonbold math in the Abstract.
% This preserves the distinction between vectors and scalars. However,
% if the journal you are submitting to favors bold math in the abstract,
% then you can use LaTeX's standard command \boldmath at the very start
% of the abstract to achieve this. Many IEEE journals frown on math
% in the abstract anyway.

% Note that keywords are not normally used for peerreview papers.
%\begin{IEEEkeywords}
%Cooperative diversity, decode and forward, piecewise linear
%\end{IEEEkeywords}



% For peer review papers, you can put extra information on the cover
% page as needed:
% \ifCLASSOPTIONpeerreview
% \begin{center} \bfseries EDICS Category: 3-BBND \end{center}
% \fi
%
% For peerreview papers, this IEEEtran command inserts a page break and
% creates the second title. It will be ignored for other modes.
%\IEEEpeerreviewmaketitle




\item A team of medical students doing their internship have to assist during surgeries
at a city hospital. The probabilities of surgeries rated as very complex, complex,
routine, simple or very simple are respectively, 0.15, 0.20, 0.31, 0.26, .08. Find
the probabilities that a particular surgery will be rated
\begin{enumerate}
	\item complex or very complex;
	\item neither very complex nor very simple;
	\item routine or complex
	\item routine or simple
\end{enumerate}
\solution
%\begin{table}[H]
	\centering
\begin{tabular}{|c|c|c|}
\hline
Random variable &Value &Definition\\ \hline
\multirow{3}{*}{X} &0 &Slips of Rs 1\\
&1 &Slips of Rs 5\\
&2 &Slips of Rs 13\\ \hline
\multirow{2}{*}{Y} &0 &Box A\\
&1 &Box B\\\hline
\end{tabular}
\caption{}
\label{tab:Distribution}
\end{table}
See \tabref{tab:Distribution}.
\begin{align}
p_{Y}\brak{k}= \begin{cases} 
      \frac{1}{3} & {k=0} \\
      \frac{2}{3 }& {k=1} 
   \end{cases}
   \\
p_{Y|X}\brak{0|0} = \frac{19}{25}\, 
p_{Y|X}\brak{0|1} = \frac{6}{25}\,
p_{Y|X}\brak{1|0} = \frac{45}{50}\,
p_{Y|X}\brak{1|2} = \frac{5}{50}
\end{align}
The desired probability is the probability that a slip drawn at random is marked other than Rs 1,
\begin{align}
&=1-p_X\brak{0}\\
&= p_X(1) + p_X(2)
\end{align}
Using Bayes theorem,
\begin{align}
&= p_Y\brak{0} \times \pr{Y=0 | X=1} + p_Y\brak{1} \times \pr{Y=1|X=2}\\
&=\frac{1}{3} \times \frac{6}{25} + \frac{2}{3} \times \frac{5}{50}\\
&=\frac{11}{75}
\end{align}

\newpage

%\tableofcontents

\bigskip

\renewcommand{\thefigure}{\theenumi}
\renewcommand{\thetable}{\theenumi}
%\renewcommand{\theequation}{\theenumi}

%\begin{abstract}
%%\boldmath
%In this letter, an algorithm for evaluating the exact analytical bit error rate  (BER)  for the piecewise linear (PL) combiner for  multiple relays is presented. Previous results were available only for upto three relays. The algorithm is unique in the sense that  the actual mathematical expressions, that are prohibitively large, need not be explicitly obtained. The diversity gain due to multiple relays is shown through plots of the analytical BER, well supported by simulations. 
%
%\end{abstract}
% IEEEtran.cls defaults to using nonbold math in the Abstract.
% This preserves the distinction between vectors and scalars. However,
% if the journal you are submitting to favors bold math in the abstract,
% then you can use LaTeX's standard command \boldmath at the very start
% of the abstract to achieve this. Many IEEE journals frown on math
% in the abstract anyway.

% Note that keywords are not normally used for peerreview papers.
%\begin{IEEEkeywords}
%Cooperative diversity, decode and forward, piecewise linear
%\end{IEEEkeywords}



% For peer review papers, you can put extra information on the cover
% page as needed:
% \ifCLASSOPTIONpeerreview
% \begin{center} \bfseries EDICS Category: 3-BBND \end{center}
% \fi
%
% For peerreview papers, this IEEEtran command inserts a page break and
% creates the second title. It will be ignored for other modes.
%\IEEEpeerreviewmaketitle




\item A card is selected from a pack of 52 cards.
\begin{enumerate}[label=(\alph*)]
    \item How many points are there in the sample space?
    \item Calculate the probability that the card is an ace of spades.
    \item Calculate the probability that the card is (i) an ace and (ii) black card.
\end{enumerate}
\solution
%Let $X$ be an bernoulli rv defined as in \tabref{tab:exemplar/11/16/3/26}.  Then, 
\begin{equation}
    p =
        \frac{4}{11} 
\end{equation}
\begin{table}[H]
	\centering
	\input{exemplar/11/16/3/26/tables/Table2.tex}
	\caption{}
        \label{tab:exemplar/11/16/3/26}
\end{table}

\item The probability that a non leap year selected at random will contain 53 sundays.
\\
\solution
%\begin{table}[H]
	\centering
\begin{tabular}{|c|c|c|}
\hline
Random variable &Value &Definition\\ \hline
\multirow{3}{*}{X} &0 &Slips of Rs 1\\
&1 &Slips of Rs 5\\
&2 &Slips of Rs 13\\ \hline
\multirow{2}{*}{Y} &0 &Box A\\
&1 &Box B\\\hline
\end{tabular}
\caption{}
\label{tab:Distribution}
\end{table}
See \tabref{tab:Distribution}.
\begin{align}
p_{Y}\brak{k}= \begin{cases} 
      \frac{1}{3} & {k=0} \\
      \frac{2}{3 }& {k=1} 
   \end{cases}
   \\
p_{Y|X}\brak{0|0} = \frac{19}{25}\, 
p_{Y|X}\brak{0|1} = \frac{6}{25}\,
p_{Y|X}\brak{1|0} = \frac{45}{50}\,
p_{Y|X}\brak{1|2} = \frac{5}{50}
\end{align}
The desired probability is the probability that a slip drawn at random is marked other than Rs 1,
\begin{align}
&=1-p_X\brak{0}\\
&= p_X(1) + p_X(2)
\end{align}
Using Bayes theorem,
\begin{align}
&= p_Y\brak{0} \times \pr{Y=0 | X=1} + p_Y\brak{1} \times \pr{Y=1|X=2}\\
&=\frac{1}{3} \times \frac{6}{25} + \frac{2}{3} \times \frac{5}{50}\\
&=\frac{11}{75}
\end{align}

\newpage

%\tableofcontents

\bigskip

\renewcommand{\thefigure}{\theenumi}
\renewcommand{\thetable}{\theenumi}
%\renewcommand{\theequation}{\theenumi}

%\begin{abstract}
%%\boldmath
%In this letter, an algorithm for evaluating the exact analytical bit error rate  (BER)  for the piecewise linear (PL) combiner for  multiple relays is presented. Previous results were available only for upto three relays. The algorithm is unique in the sense that  the actual mathematical expressions, that are prohibitively large, need not be explicitly obtained. The diversity gain due to multiple relays is shown through plots of the analytical BER, well supported by simulations. 
%
%\end{abstract}
% IEEEtran.cls defaults to using nonbold math in the Abstract.
% This preserves the distinction between vectors and scalars. However,
% if the journal you are submitting to favors bold math in the abstract,
% then you can use LaTeX's standard command \boldmath at the very start
% of the abstract to achieve this. Many IEEE journals frown on math
% in the abstract anyway.

% Note that keywords are not normally used for peerreview papers.
%\begin{IEEEkeywords}
%Cooperative diversity, decode and forward, piecewise linear
%\end{IEEEkeywords}



% For peer review papers, you can put extra information on the cover
% page as needed:
% \ifCLASSOPTIONpeerreview
% \begin{center} \bfseries EDICS Category: 3-BBND \end{center}
% \fi
%
% For peerreview papers, this IEEEtran command inserts a page break and
% creates the second title. It will be ignored for other modes.
%\IEEEpeerreviewmaketitle




\item One of the four persons John, Rita, Aslam or Gurpreet will be promoted next
month. Consequently the sample space consists of four elementary outcomes
S = {John promoted, Rita promoted, Aslam promoted, Gurpreet promoted}
You are told that the chances of John’s promotion is same as that of Gurpreet,
Rita’s chances of promotion are twice as likely as Johns. Aslam’s chances are
four times that of John.
\begin{enumerate}
	\item Determine
	\begin{enumerate}
		\item P (John promoted)
		\item P (Rita promoted)
		\item P (Aslam promoted)
		\item P (Gurpreet promoted)
	\end{enumerate}
	\item If A = {John promoted or Gurpreet promoted}, find P (A).
\end{enumerate}
\solution
%\begin{table}[H]
	\centering
\begin{tabular}{|c|c|c|}
\hline
Random variable &Value &Definition\\ \hline
\multirow{3}{*}{X} &0 &Slips of Rs 1\\
&1 &Slips of Rs 5\\
&2 &Slips of Rs 13\\ \hline
\multirow{2}{*}{Y} &0 &Box A\\
&1 &Box B\\\hline
\end{tabular}
\caption{}
\label{tab:Distribution}
\end{table}
See \tabref{tab:Distribution}.
\begin{align}
p_{Y}\brak{k}= \begin{cases} 
      \frac{1}{3} & {k=0} \\
      \frac{2}{3 }& {k=1} 
   \end{cases}
   \\
p_{Y|X}\brak{0|0} = \frac{19}{25}\, 
p_{Y|X}\brak{0|1} = \frac{6}{25}\,
p_{Y|X}\brak{1|0} = \frac{45}{50}\,
p_{Y|X}\brak{1|2} = \frac{5}{50}
\end{align}
The desired probability is the probability that a slip drawn at random is marked other than Rs 1,
\begin{align}
&=1-p_X\brak{0}\\
&= p_X(1) + p_X(2)
\end{align}
Using Bayes theorem,
\begin{align}
&= p_Y\brak{0} \times \pr{Y=0 | X=1} + p_Y\brak{1} \times \pr{Y=1|X=2}\\
&=\frac{1}{3} \times \frac{6}{25} + \frac{2}{3} \times \frac{5}{50}\\
&=\frac{11}{75}
\end{align}

\newpage

%\tableofcontents

\bigskip

\renewcommand{\thefigure}{\theenumi}
\renewcommand{\thetable}{\theenumi}
%\renewcommand{\theequation}{\theenumi}

%\begin{abstract}
%%\boldmath
%In this letter, an algorithm for evaluating the exact analytical bit error rate  (BER)  for the piecewise linear (PL) combiner for  multiple relays is presented. Previous results were available only for upto three relays. The algorithm is unique in the sense that  the actual mathematical expressions, that are prohibitively large, need not be explicitly obtained. The diversity gain due to multiple relays is shown through plots of the analytical BER, well supported by simulations. 
%
%\end{abstract}
% IEEEtran.cls defaults to using nonbold math in the Abstract.
% This preserves the distinction between vectors and scalars. However,
% if the journal you are submitting to favors bold math in the abstract,
% then you can use LaTeX's standard command \boldmath at the very start
% of the abstract to achieve this. Many IEEE journals frown on math
% in the abstract anyway.

% Note that keywords are not normally used for peerreview papers.
%\begin{IEEEkeywords}
%Cooperative diversity, decode and forward, piecewise linear
%\end{IEEEkeywords}



% For peer review papers, you can put extra information on the cover
% page as needed:
% \ifCLASSOPTIONpeerreview
% \begin{center} \bfseries EDICS Category: 3-BBND \end{center}
% \fi
%
% For peerreview papers, this IEEEtran command inserts a page break and
% creates the second title. It will be ignored for other modes.
%\IEEEpeerreviewmaketitle




\item A card is drawn from a deck of 52 cards. Find the probability of getting a king or a heart or a red card.\\
\solution
%\begin{table}[H]
	\centering
\begin{tabular}{|c|c|c|}
\hline
Random variable &Value &Definition\\ \hline
\multirow{3}{*}{X} &0 &Slips of Rs 1\\
&1 &Slips of Rs 5\\
&2 &Slips of Rs 13\\ \hline
\multirow{2}{*}{Y} &0 &Box A\\
&1 &Box B\\\hline
\end{tabular}
\caption{}
\label{tab:Distribution}
\end{table}
See \tabref{tab:Distribution}.
\begin{align}
p_{Y}\brak{k}= \begin{cases} 
      \frac{1}{3} & {k=0} \\
      \frac{2}{3 }& {k=1} 
   \end{cases}
   \\
p_{Y|X}\brak{0|0} = \frac{19}{25}\, 
p_{Y|X}\brak{0|1} = \frac{6}{25}\,
p_{Y|X}\brak{1|0} = \frac{45}{50}\,
p_{Y|X}\brak{1|2} = \frac{5}{50}
\end{align}
The desired probability is the probability that a slip drawn at random is marked other than Rs 1,
\begin{align}
&=1-p_X\brak{0}\\
&= p_X(1) + p_X(2)
\end{align}
Using Bayes theorem,
\begin{align}
&= p_Y\brak{0} \times \pr{Y=0 | X=1} + p_Y\brak{1} \times \pr{Y=1|X=2}\\
&=\frac{1}{3} \times \frac{6}{25} + \frac{2}{3} \times \frac{5}{50}\\
&=\frac{11}{75}
\end{align}

\newpage

%\tableofcontents

\bigskip

\renewcommand{\thefigure}{\theenumi}
\renewcommand{\thetable}{\theenumi}
%\renewcommand{\theequation}{\theenumi}

%\begin{abstract}
%%\boldmath
%In this letter, an algorithm for evaluating the exact analytical bit error rate  (BER)  for the piecewise linear (PL) combiner for  multiple relays is presented. Previous results were available only for upto three relays. The algorithm is unique in the sense that  the actual mathematical expressions, that are prohibitively large, need not be explicitly obtained. The diversity gain due to multiple relays is shown through plots of the analytical BER, well supported by simulations. 
%
%\end{abstract}
% IEEEtran.cls defaults to using nonbold math in the Abstract.
% This preserves the distinction between vectors and scalars. However,
% if the journal you are submitting to favors bold math in the abstract,
% then you can use LaTeX's standard command \boldmath at the very start
% of the abstract to achieve this. Many IEEE journals frown on math
% in the abstract anyway.

% Note that keywords are not normally used for peerreview papers.
%\begin{IEEEkeywords}
%Cooperative diversity, decode and forward, piecewise linear
%\end{IEEEkeywords}



% For peer review papers, you can put extra information on the cover
% page as needed:
% \ifCLASSOPTIONpeerreview
% \begin{center} \bfseries EDICS Category: 3-BBND \end{center}
% \fi
%
% For peerreview papers, this IEEEtran command inserts a page break and
% creates the second title. It will be ignored for other modes.
%\IEEEpeerreviewmaketitle




\item The probability that a student will pass his examination is 0.73, the probability of
the student getting a compartment is 0.13, and the probability that the student will
either pass or get compartment is 0.96. State True or False.\\
\solution
%\begin{table}[H]
	\centering
\begin{tabular}{|c|c|c|}
\hline
Random variable &Value &Definition\\ \hline
\multirow{3}{*}{X} &0 &Slips of Rs 1\\
&1 &Slips of Rs 5\\
&2 &Slips of Rs 13\\ \hline
\multirow{2}{*}{Y} &0 &Box A\\
&1 &Box B\\\hline
\end{tabular}
\caption{}
\label{tab:Distribution}
\end{table}
See \tabref{tab:Distribution}.
\begin{align}
p_{Y}\brak{k}= \begin{cases} 
      \frac{1}{3} & {k=0} \\
      \frac{2}{3 }& {k=1} 
   \end{cases}
   \\
p_{Y|X}\brak{0|0} = \frac{19}{25}\, 
p_{Y|X}\brak{0|1} = \frac{6}{25}\,
p_{Y|X}\brak{1|0} = \frac{45}{50}\,
p_{Y|X}\brak{1|2} = \frac{5}{50}
\end{align}
The desired probability is the probability that a slip drawn at random is marked other than Rs 1,
\begin{align}
&=1-p_X\brak{0}\\
&= p_X(1) + p_X(2)
\end{align}
Using Bayes theorem,
\begin{align}
&= p_Y\brak{0} \times \pr{Y=0 | X=1} + p_Y\brak{1} \times \pr{Y=1|X=2}\\
&=\frac{1}{3} \times \frac{6}{25} + \frac{2}{3} \times \frac{5}{50}\\
&=\frac{11}{75}
\end{align}

\newpage

%\tableofcontents

\bigskip

\renewcommand{\thefigure}{\theenumi}
\renewcommand{\thetable}{\theenumi}
%\renewcommand{\theequation}{\theenumi}

%\begin{abstract}
%%\boldmath
%In this letter, an algorithm for evaluating the exact analytical bit error rate  (BER)  for the piecewise linear (PL) combiner for  multiple relays is presented. Previous results were available only for upto three relays. The algorithm is unique in the sense that  the actual mathematical expressions, that are prohibitively large, need not be explicitly obtained. The diversity gain due to multiple relays is shown through plots of the analytical BER, well supported by simulations. 
%
%\end{abstract}
% IEEEtran.cls defaults to using nonbold math in the Abstract.
% This preserves the distinction between vectors and scalars. However,
% if the journal you are submitting to favors bold math in the abstract,
% then you can use LaTeX's standard command \boldmath at the very start
% of the abstract to achieve this. Many IEEE journals frown on math
% in the abstract anyway.

% Note that keywords are not normally used for peerreview papers.
%\begin{IEEEkeywords}
%Cooperative diversity, decode and forward, piecewise linear
%\end{IEEEkeywords}



% For peer review papers, you can put extra information on the cover
% page as needed:
% \ifCLASSOPTIONpeerreview
% \begin{center} \bfseries EDICS Category: 3-BBND \end{center}
% \fi
%
% For peerreview papers, this IEEEtran command inserts a page break and
% creates the second title. It will be ignored for other modes.
%\IEEEpeerreviewmaketitle




\item A card is selected from a pack of 52 cards\\
\begin{enumerate}[label=(\alph*)]
\item How many points are there in the sample space?
\item Calculate the probability that the cards is an ace of spades.
\item Calculate the probability that the card is (i) an ace (ii)black card.\\
\end{enumerate}
%\input{ncert/11/16/3/4_1/Prob_4.tex}
\item In a non-leap year, the probability of having 53 tuesdays or 53 wednesdays is\\
\solution
%A non-leap year has a total of 365 days, and a week has 7 days.\\
So it can be expressed as 
\begin{align}
365\text{days} &=52\times 7+1 \text{day}
\end{align}
$\implies$ 52 tuesdays or wednesdays\\
Random variable X denotes the days of a week
\begin{align}
p_X\brak{k}&=\frac{1}{7}; \quad \brak{1<k<7}
\end{align}
So the probability of extra day being tuesday or wednesday is
\begin{align}
p_X\brak{3}+p_X\brak{4}&=\frac{1}{7}+\frac{1}{7}=\frac{2}{7}
\end{align}



\item There are 1000 sealed envelopes in a box, 10 of them contain a cash prize of
Rs 100 each, 100 of them contain a cash prize of Rs 50 each and 200 of them
contain a cash prize of Rs 10 each and rest do not contain any cash prize. If they
are well shuffled and an envelope is picked up out, what is the probability that it
contains no cash prize?\\
\solution
%\begin{table}[H]
	\centering
\begin{tabular}{|c|c|c|}
\hline
Random variable &Value &Definition\\ \hline
\multirow{3}{*}{X} &0 &Slips of Rs 1\\
&1 &Slips of Rs 5\\
&2 &Slips of Rs 13\\ \hline
\multirow{2}{*}{Y} &0 &Box A\\
&1 &Box B\\\hline
\end{tabular}
\caption{}
\label{tab:Distribution}
\end{table}
See \tabref{tab:Distribution}.
\begin{align}
p_{Y}\brak{k}= \begin{cases} 
      \frac{1}{3} & {k=0} \\
      \frac{2}{3 }& {k=1} 
   \end{cases}
   \\
p_{Y|X}\brak{0|0} = \frac{19}{25}\, 
p_{Y|X}\brak{0|1} = \frac{6}{25}\,
p_{Y|X}\brak{1|0} = \frac{45}{50}\,
p_{Y|X}\brak{1|2} = \frac{5}{50}
\end{align}
The desired probability is the probability that a slip drawn at random is marked other than Rs 1,
\begin{align}
&=1-p_X\brak{0}\\
&= p_X(1) + p_X(2)
\end{align}
Using Bayes theorem,
\begin{align}
&= p_Y\brak{0} \times \pr{Y=0 | X=1} + p_Y\brak{1} \times \pr{Y=1|X=2}\\
&=\frac{1}{3} \times \frac{6}{25} + \frac{2}{3} \times \frac{5}{50}\\
&=\frac{11}{75}
\end{align}

\newpage

%\tableofcontents

\bigskip

\renewcommand{\thefigure}{\theenumi}
\renewcommand{\thetable}{\theenumi}
%\renewcommand{\theequation}{\theenumi}

%\begin{abstract}
%%\boldmath
%In this letter, an algorithm for evaluating the exact analytical bit error rate  (BER)  for the piecewise linear (PL) combiner for  multiple relays is presented. Previous results were available only for upto three relays. The algorithm is unique in the sense that  the actual mathematical expressions, that are prohibitively large, need not be explicitly obtained. The diversity gain due to multiple relays is shown through plots of the analytical BER, well supported by simulations. 
%
%\end{abstract}
% IEEEtran.cls defaults to using nonbold math in the Abstract.
% This preserves the distinction between vectors and scalars. However,
% if the journal you are submitting to favors bold math in the abstract,
% then you can use LaTeX's standard command \boldmath at the very start
% of the abstract to achieve this. Many IEEE journals frown on math
% in the abstract anyway.

% Note that keywords are not normally used for peerreview papers.
%\begin{IEEEkeywords}
%Cooperative diversity, decode and forward, piecewise linear
%\end{IEEEkeywords}



% For peer review papers, you can put extra information on the cover
% page as needed:
% \ifCLASSOPTIONpeerreview
% \begin{center} \bfseries EDICS Category: 3-BBND \end{center}
% \fi
%
% For peerreview papers, this IEEEtran command inserts a page break and
% creates the second title. It will be ignored for other modes.
%\IEEEpeerreviewmaketitle




\item 
A die is thrown and a card is selected at random from a deck of 52 playing cards. The probability of getting an even number on the die and a spade card.\\
\solution
%\begin{table}[H]
	\centering
\begin{tabular}{|c|c|c|}
\hline
Random variable &Value &Definition\\ \hline
\multirow{3}{*}{X} &0 &Slips of Rs 1\\
&1 &Slips of Rs 5\\
&2 &Slips of Rs 13\\ \hline
\multirow{2}{*}{Y} &0 &Box A\\
&1 &Box B\\\hline
\end{tabular}
\caption{}
\label{tab:Distribution}
\end{table}
See \tabref{tab:Distribution}.
\begin{align}
p_{Y}\brak{k}= \begin{cases} 
      \frac{1}{3} & {k=0} \\
      \frac{2}{3 }& {k=1} 
   \end{cases}
   \\
p_{Y|X}\brak{0|0} = \frac{19}{25}\, 
p_{Y|X}\brak{0|1} = \frac{6}{25}\,
p_{Y|X}\brak{1|0} = \frac{45}{50}\,
p_{Y|X}\brak{1|2} = \frac{5}{50}
\end{align}
The desired probability is the probability that a slip drawn at random is marked other than Rs 1,
\begin{align}
&=1-p_X\brak{0}\\
&= p_X(1) + p_X(2)
\end{align}
Using Bayes theorem,
\begin{align}
&= p_Y\brak{0} \times \pr{Y=0 | X=1} + p_Y\brak{1} \times \pr{Y=1|X=2}\\
&=\frac{1}{3} \times \frac{6}{25} + \frac{2}{3} \times \frac{5}{50}\\
&=\frac{11}{75}
\end{align}

\newpage

%\tableofcontents

\bigskip

\renewcommand{\thefigure}{\theenumi}
\renewcommand{\thetable}{\theenumi}
%\renewcommand{\theequation}{\theenumi}

%\begin{abstract}
%%\boldmath
%In this letter, an algorithm for evaluating the exact analytical bit error rate  (BER)  for the piecewise linear (PL) combiner for  multiple relays is presented. Previous results were available only for upto three relays. The algorithm is unique in the sense that  the actual mathematical expressions, that are prohibitively large, need not be explicitly obtained. The diversity gain due to multiple relays is shown through plots of the analytical BER, well supported by simulations. 
%
%\end{abstract}
% IEEEtran.cls defaults to using nonbold math in the Abstract.
% This preserves the distinction between vectors and scalars. However,
% if the journal you are submitting to favors bold math in the abstract,
% then you can use LaTeX's standard command \boldmath at the very start
% of the abstract to achieve this. Many IEEE journals frown on math
% in the abstract anyway.

% Note that keywords are not normally used for peerreview papers.
%\begin{IEEEkeywords}
%Cooperative diversity, decode and forward, piecewise linear
%\end{IEEEkeywords}



% For peer review papers, you can put extra information on the cover
% page as needed:
% \ifCLASSOPTIONpeerreview
% \begin{center} \bfseries EDICS Category: 3-BBND \end{center}
% \fi
%
% For peerreview papers, this IEEEtran command inserts a page break and
% creates the second title. It will be ignored for other modes.
%\IEEEpeerreviewmaketitle




\item
If 4-digit numbers greater than 5,000 are randomly formed from the digits 0, 1, 3, 5, and 7, what is the probability of forming a number divisible by 5 when:
\begin{enumerate}
    \item The digits are repeated?
    \item The repetition of digits is not allowed?
\end{enumerate}
\solution
%\begin{table}[H]
	\centering
\begin{tabular}{|c|c|c|}
\hline
Random variable &Value &Definition\\ \hline
\multirow{3}{*}{X} &0 &Slips of Rs 1\\
&1 &Slips of Rs 5\\
&2 &Slips of Rs 13\\ \hline
\multirow{2}{*}{Y} &0 &Box A\\
&1 &Box B\\\hline
\end{tabular}
\caption{}
\label{tab:Distribution}
\end{table}
See \tabref{tab:Distribution}.
\begin{align}
p_{Y}\brak{k}= \begin{cases} 
      \frac{1}{3} & {k=0} \\
      \frac{2}{3 }& {k=1} 
   \end{cases}
   \\
p_{Y|X}\brak{0|0} = \frac{19}{25}\, 
p_{Y|X}\brak{0|1} = \frac{6}{25}\,
p_{Y|X}\brak{1|0} = \frac{45}{50}\,
p_{Y|X}\brak{1|2} = \frac{5}{50}
\end{align}
The desired probability is the probability that a slip drawn at random is marked other than Rs 1,
\begin{align}
&=1-p_X\brak{0}\\
&= p_X(1) + p_X(2)
\end{align}
Using Bayes theorem,
\begin{align}
&= p_Y\brak{0} \times \pr{Y=0 | X=1} + p_Y\brak{1} \times \pr{Y=1|X=2}\\
&=\frac{1}{3} \times \frac{6}{25} + \frac{2}{3} \times \frac{5}{50}\\
&=\frac{11}{75}
\end{align}

\newpage

%\tableofcontents

\bigskip

\renewcommand{\thefigure}{\theenumi}
\renewcommand{\thetable}{\theenumi}
%\renewcommand{\theequation}{\theenumi}

%\begin{abstract}
%%\boldmath
%In this letter, an algorithm for evaluating the exact analytical bit error rate  (BER)  for the piecewise linear (PL) combiner for  multiple relays is presented. Previous results were available only for upto three relays. The algorithm is unique in the sense that  the actual mathematical expressions, that are prohibitively large, need not be explicitly obtained. The diversity gain due to multiple relays is shown through plots of the analytical BER, well supported by simulations. 
%
%\end{abstract}
% IEEEtran.cls defaults to using nonbold math in the Abstract.
% This preserves the distinction between vectors and scalars. However,
% if the journal you are submitting to favors bold math in the abstract,
% then you can use LaTeX's standard command \boldmath at the very start
% of the abstract to achieve this. Many IEEE journals frown on math
% in the abstract anyway.

% Note that keywords are not normally used for peerreview papers.
%\begin{IEEEkeywords}
%Cooperative diversity, decode and forward, piecewise linear
%\end{IEEEkeywords}



% For peer review papers, you can put extra information on the cover
% page as needed:
% \ifCLASSOPTIONpeerreview
% \begin{center} \bfseries EDICS Category: 3-BBND \end{center}
% \fi
%
% For peerreview papers, this IEEEtran command inserts a page break and
% creates the second title. It will be ignored for other modes.
%\IEEEpeerreviewmaketitle




\item Consider the probability space $\brak{\Omega, \mathcal{G}, P}$ where $\Omega = [0,2]$ and $\mathcal{G} = \cbrak{\phi, \Omega, [0,1], (1,2]}$. Let $X$ and $Y$ be two functions on $\Omega$ defined as
\begin{align*}
    X(\omega) = 
    \begin{cases}
        1 & \text{if }\omega \in [0, 1]\\
        2 & \text{if }\omega \in (1, 2]
    \end{cases}
\end{align*}
and
\begin{align*}
    Y(\omega) = 
    \begin{cases}
        2 & \text{if }\omega \in [0, 1.5]\\
        3 & \text{if }\omega \in (1.5, 2].
    \end{cases}
\end{align*}
Then which one of the following statements is true?
\begin{enumerate}
    \item [(A)] $X$ is a random variable with respect to $\mathcal{G}$, but $Y$ is not a random variable with respect to $\mathcal{G}$.
    \item [(B)] $Y$ is a random variable with respect to $\mathcal{G}$, but $X$ is not a random variable with respect to $\mathcal{G}$.
    \item [(C)] Neither $X$ nor $Y$ is a random variable with respect to $\mathcal{G}$.
    \item [(D)] Both $X$ and $Y$ are random variables with respect to $\mathcal{G}$.
\end{enumerate} \hfill (GATE ST 2023)\\
\solution
%\begin{table}[H]
	\centering
\begin{tabular}{|c|c|c|}
\hline
Random variable &Value &Definition\\ \hline
\multirow{3}{*}{X} &0 &Slips of Rs 1\\
&1 &Slips of Rs 5\\
&2 &Slips of Rs 13\\ \hline
\multirow{2}{*}{Y} &0 &Box A\\
&1 &Box B\\\hline
\end{tabular}
\caption{}
\label{tab:Distribution}
\end{table}
See \tabref{tab:Distribution}.
\begin{align}
p_{Y}\brak{k}= \begin{cases} 
      \frac{1}{3} & {k=0} \\
      \frac{2}{3 }& {k=1} 
   \end{cases}
   \\
p_{Y|X}\brak{0|0} = \frac{19}{25}\, 
p_{Y|X}\brak{0|1} = \frac{6}{25}\,
p_{Y|X}\brak{1|0} = \frac{45}{50}\,
p_{Y|X}\brak{1|2} = \frac{5}{50}
\end{align}
The desired probability is the probability that a slip drawn at random is marked other than Rs 1,
\begin{align}
&=1-p_X\brak{0}\\
&= p_X(1) + p_X(2)
\end{align}
Using Bayes theorem,
\begin{align}
&= p_Y\brak{0} \times \pr{Y=0 | X=1} + p_Y\brak{1} \times \pr{Y=1|X=2}\\
&=\frac{1}{3} \times \frac{6}{25} + \frac{2}{3} \times \frac{5}{50}\\
&=\frac{11}{75}
\end{align}

\newpage

%\tableofcontents

\bigskip

\renewcommand{\thefigure}{\theenumi}
\renewcommand{\thetable}{\theenumi}
%\renewcommand{\theequation}{\theenumi}

%\begin{abstract}
%%\boldmath
%In this letter, an algorithm for evaluating the exact analytical bit error rate  (BER)  for the piecewise linear (PL) combiner for  multiple relays is presented. Previous results were available only for upto three relays. The algorithm is unique in the sense that  the actual mathematical expressions, that are prohibitively large, need not be explicitly obtained. The diversity gain due to multiple relays is shown through plots of the analytical BER, well supported by simulations. 
%
%\end{abstract}
% IEEEtran.cls defaults to using nonbold math in the Abstract.
% This preserves the distinction between vectors and scalars. However,
% if the journal you are submitting to favors bold math in the abstract,
% then you can use LaTeX's standard command \boldmath at the very start
% of the abstract to achieve this. Many IEEE journals frown on math
% in the abstract anyway.

% Note that keywords are not normally used for peerreview papers.
%\begin{IEEEkeywords}
%Cooperative diversity, decode and forward, piecewise linear
%\end{IEEEkeywords}



% For peer review papers, you can put extra information on the cover
% page as needed:
% \ifCLASSOPTIONpeerreview
% \begin{center} \bfseries EDICS Category: 3-BBND \end{center}
% \fi
%
% For peerreview papers, this IEEEtran command inserts a page break and
% creates the second title. It will be ignored for other modes.
%\IEEEpeerreviewmaketitle




	\item  A die is loaded in such a way that each odd number is twice as likely to occur as
each even number. Find $P(G)$, where $G$ is the event that a number greater than
3 occurs on a single roll of the die.
\\
\solution
		%\begin{table}[H]
	\centering
\begin{tabular}{|c|c|c|}
\hline
Random variable &Value &Definition\\ \hline
\multirow{3}{*}{X} &0 &Slips of Rs 1\\
&1 &Slips of Rs 5\\
&2 &Slips of Rs 13\\ \hline
\multirow{2}{*}{Y} &0 &Box A\\
&1 &Box B\\\hline
\end{tabular}
\caption{}
\label{tab:Distribution}
\end{table}
See \tabref{tab:Distribution}.
\begin{align}
p_{Y}\brak{k}= \begin{cases} 
      \frac{1}{3} & {k=0} \\
      \frac{2}{3 }& {k=1} 
   \end{cases}
   \\
p_{Y|X}\brak{0|0} = \frac{19}{25}\, 
p_{Y|X}\brak{0|1} = \frac{6}{25}\,
p_{Y|X}\brak{1|0} = \frac{45}{50}\,
p_{Y|X}\brak{1|2} = \frac{5}{50}
\end{align}
The desired probability is the probability that a slip drawn at random is marked other than Rs 1,
\begin{align}
&=1-p_X\brak{0}\\
&= p_X(1) + p_X(2)
\end{align}
Using Bayes theorem,
\begin{align}
&= p_Y\brak{0} \times \pr{Y=0 | X=1} + p_Y\brak{1} \times \pr{Y=1|X=2}\\
&=\frac{1}{3} \times \frac{6}{25} + \frac{2}{3} \times \frac{5}{50}\\
&=\frac{11}{75}
\end{align}

\newpage

%\tableofcontents

\bigskip

\renewcommand{\thefigure}{\theenumi}
\renewcommand{\thetable}{\theenumi}
%\renewcommand{\theequation}{\theenumi}

%\begin{abstract}
%%\boldmath
%In this letter, an algorithm for evaluating the exact analytical bit error rate  (BER)  for the piecewise linear (PL) combiner for  multiple relays is presented. Previous results were available only for upto three relays. The algorithm is unique in the sense that  the actual mathematical expressions, that are prohibitively large, need not be explicitly obtained. The diversity gain due to multiple relays is shown through plots of the analytical BER, well supported by simulations. 
%
%\end{abstract}
% IEEEtran.cls defaults to using nonbold math in the Abstract.
% This preserves the distinction between vectors and scalars. However,
% if the journal you are submitting to favors bold math in the abstract,
% then you can use LaTeX's standard command \boldmath at the very start
% of the abstract to achieve this. Many IEEE journals frown on math
% in the abstract anyway.

% Note that keywords are not normally used for peerreview papers.
%\begin{IEEEkeywords}
%Cooperative diversity, decode and forward, piecewise linear
%\end{IEEEkeywords}



% For peer review papers, you can put extra information on the cover
% page as needed:
% \ifCLASSOPTIONpeerreview
% \begin{center} \bfseries EDICS Category: 3-BBND \end{center}
% \fi
%
% For peerreview papers, this IEEEtran command inserts a page break and
% creates the second title. It will be ignored for other modes.
%\IEEEpeerreviewmaketitle




	\item All the jacks, queens and kings are removed from a deck of 52 playing cards. The remaining cards are well shuffled and then one card is drawn at random. Giving ace a value 1 similar value for other cards, find the probability that the card has a value 
		\begin{enumerate}
			\item 7
			\item greater than 7
			\item less than 7
		\end{enumerate}
		%Number of cards left after removing all jacks, queens and kings 
\begin{align}
N	= 52 - 4\times 3
	= 40
\end{align}
%\begin{table}[H]
%\def\arraystretch{1.2}
%\begin{tabular}{|c|c|c|}
%\hline
%	\textbf{Parameter} &\textbf{Value} &\textbf{Description}\\ \hline
%	$X$ &1-10 &Represents the value of the card picked \\ \hline
%\end{tabular}
%\end{table}
Let $1 \le X \le 10$ be the value of the card picked.  Then,
\begin{align}
	p_X(k) &= \Pr(X=k)\ \forall\ 1 \leq k \leq 10\\
	&= \frac{4\times 1}{40}\\
	&= \frac{1}{10}\\
	\therefore p_X(k) &= 
	\begin{cases}
		\frac{1}{10} & 1 \leq k \leq 10\\
		0 & \text{otherwise}
	\end{cases}
\end{align}
and
\begin{align}
	F_{X}(k) &= \sum_{m=0}^{k}p_{X}(m) \quad 1 \leq k \leq 10\\
	&= \frac{k}{10}\\
	\therefore F_{X}(k) &= 
	\begin{cases}
		0 & k \leq 0\\
		\frac{k}{10} & 1\leq k \leq 10\\
		1 & k > 10 
	\end{cases}
\end{align}
\begin{enumerate}
	\item Probability that card has value equal to 7 is
		\begin{align}
			 p_{X}(7)
			= \frac{1}{10}
		\end{align}
	\item Probability that card has value greater than 7 is
		\begin{align}
			1 - F_X(7)
			&= 1 - \frac{7}{10}
			\\
			&= \frac{3}{10}
		\end{align}
	\item Probability that card has value less than 7 is
		\begin{align}
			 F_{X}(6)
			=\frac{6}{10}
		\end{align}
\end{enumerate}

  \item A Lot consists of 48 mobile phones of which 42 are good, 3 have only minor defects and 3 have major defects.Varnika will buy a phone if it is good but the trader will only buy a mobile if it has no major defects. One phone is selected at random from the lot. What is the probability that it is
\begin{enumerate}
	\item acceptable to Varnika?
            \item acceptable to the trader?
\end{enumerate}
\solution
	%\begin{table}[H]
	\centering
\begin{tabular}{|c|c|c|}
\hline
Random variable &Value &Definition\\ \hline
\multirow{3}{*}{X} &0 &Slips of Rs 1\\
&1 &Slips of Rs 5\\
&2 &Slips of Rs 13\\ \hline
\multirow{2}{*}{Y} &0 &Box A\\
&1 &Box B\\\hline
\end{tabular}
\caption{}
\label{tab:Distribution}
\end{table}
See \tabref{tab:Distribution}.
\begin{align}
p_{Y}\brak{k}= \begin{cases} 
      \frac{1}{3} & {k=0} \\
      \frac{2}{3 }& {k=1} 
   \end{cases}
   \\
p_{Y|X}\brak{0|0} = \frac{19}{25}\, 
p_{Y|X}\brak{0|1} = \frac{6}{25}\,
p_{Y|X}\brak{1|0} = \frac{45}{50}\,
p_{Y|X}\brak{1|2} = \frac{5}{50}
\end{align}
The desired probability is the probability that a slip drawn at random is marked other than Rs 1,
\begin{align}
&=1-p_X\brak{0}\\
&= p_X(1) + p_X(2)
\end{align}
Using Bayes theorem,
\begin{align}
&= p_Y\brak{0} \times \pr{Y=0 | X=1} + p_Y\brak{1} \times \pr{Y=1|X=2}\\
&=\frac{1}{3} \times \frac{6}{25} + \frac{2}{3} \times \frac{5}{50}\\
&=\frac{11}{75}
\end{align}

\newpage

%\tableofcontents

\bigskip

\renewcommand{\thefigure}{\theenumi}
\renewcommand{\thetable}{\theenumi}
%\renewcommand{\theequation}{\theenumi}

%\begin{abstract}
%%\boldmath
%In this letter, an algorithm for evaluating the exact analytical bit error rate  (BER)  for the piecewise linear (PL) combiner for  multiple relays is presented. Previous results were available only for upto three relays. The algorithm is unique in the sense that  the actual mathematical expressions, that are prohibitively large, need not be explicitly obtained. The diversity gain due to multiple relays is shown through plots of the analytical BER, well supported by simulations. 
%
%\end{abstract}
% IEEEtran.cls defaults to using nonbold math in the Abstract.
% This preserves the distinction between vectors and scalars. However,
% if the journal you are submitting to favors bold math in the abstract,
% then you can use LaTeX's standard command \boldmath at the very start
% of the abstract to achieve this. Many IEEE journals frown on math
% in the abstract anyway.

% Note that keywords are not normally used for peerreview papers.
%\begin{IEEEkeywords}
%Cooperative diversity, decode and forward, piecewise linear
%\end{IEEEkeywords}



% For peer review papers, you can put extra information on the cover
% page as needed:
% \ifCLASSOPTIONpeerreview
% \begin{center} \bfseries EDICS Category: 3-BBND \end{center}
% \fi
%
% For peerreview papers, this IEEEtran command inserts a page break and
% creates the second title. It will be ignored for other modes.
%\IEEEpeerreviewmaketitle




 \item A student says that if you throw a die, it will show up 1 or not 1. Therefore, the probability of getting 1 and the probability of getting 'not 1' each is equal to $\frac{1}{2}$. Is this correct? Give reasons.\\
 \solution
        %\begin{table}[H]
	\centering
\begin{tabular}{|c|c|c|}
\hline
Random variable &Value &Definition\\ \hline
\multirow{3}{*}{X} &0 &Slips of Rs 1\\
&1 &Slips of Rs 5\\
&2 &Slips of Rs 13\\ \hline
\multirow{2}{*}{Y} &0 &Box A\\
&1 &Box B\\\hline
\end{tabular}
\caption{}
\label{tab:Distribution}
\end{table}
See \tabref{tab:Distribution}.
\begin{align}
p_{Y}\brak{k}= \begin{cases} 
      \frac{1}{3} & {k=0} \\
      \frac{2}{3 }& {k=1} 
   \end{cases}
   \\
p_{Y|X}\brak{0|0} = \frac{19}{25}\, 
p_{Y|X}\brak{0|1} = \frac{6}{25}\,
p_{Y|X}\brak{1|0} = \frac{45}{50}\,
p_{Y|X}\brak{1|2} = \frac{5}{50}
\end{align}
The desired probability is the probability that a slip drawn at random is marked other than Rs 1,
\begin{align}
&=1-p_X\brak{0}\\
&= p_X(1) + p_X(2)
\end{align}
Using Bayes theorem,
\begin{align}
&= p_Y\brak{0} \times \pr{Y=0 | X=1} + p_Y\brak{1} \times \pr{Y=1|X=2}\\
&=\frac{1}{3} \times \frac{6}{25} + \frac{2}{3} \times \frac{5}{50}\\
&=\frac{11}{75}
\end{align}

\newpage

%\tableofcontents

\bigskip

\renewcommand{\thefigure}{\theenumi}
\renewcommand{\thetable}{\theenumi}
%\renewcommand{\theequation}{\theenumi}

%\begin{abstract}
%%\boldmath
%In this letter, an algorithm for evaluating the exact analytical bit error rate  (BER)  for the piecewise linear (PL) combiner for  multiple relays is presented. Previous results were available only for upto three relays. The algorithm is unique in the sense that  the actual mathematical expressions, that are prohibitively large, need not be explicitly obtained. The diversity gain due to multiple relays is shown through plots of the analytical BER, well supported by simulations. 
%
%\end{abstract}
% IEEEtran.cls defaults to using nonbold math in the Abstract.
% This preserves the distinction between vectors and scalars. However,
% if the journal you are submitting to favors bold math in the abstract,
% then you can use LaTeX's standard command \boldmath at the very start
% of the abstract to achieve this. Many IEEE journals frown on math
% in the abstract anyway.

% Note that keywords are not normally used for peerreview papers.
%\begin{IEEEkeywords}
%Cooperative diversity, decode and forward, piecewise linear
%\end{IEEEkeywords}



% For peer review papers, you can put extra information on the cover
% page as needed:
% \ifCLASSOPTIONpeerreview
% \begin{center} \bfseries EDICS Category: 3-BBND \end{center}
% \fi
%
% For peerreview papers, this IEEEtran command inserts a page break and
% creates the second title. It will be ignored for other modes.
%\IEEEpeerreviewmaketitle




   \item Four candidates A, B, C, D have ap-
plied for the assignment to coach a school cricket
team. If A is twice as likely to be selected as B, and
B and C are given about the same chance of being
selected, while C is twice as likely to be selected
as D, what are the probabilities that
\begin{enumerate}
\item C will be selected?
\item A will not be selected?
\end{enumerate}
	%\begin{table}[H]
	\centering
\begin{tabular}{|c|c|c|}
\hline
Random variable &Value &Definition\\ \hline
\multirow{3}{*}{X} &0 &Slips of Rs 1\\
&1 &Slips of Rs 5\\
&2 &Slips of Rs 13\\ \hline
\multirow{2}{*}{Y} &0 &Box A\\
&1 &Box B\\\hline
\end{tabular}
\caption{}
\label{tab:Distribution}
\end{table}
See \tabref{tab:Distribution}.
\begin{align}
p_{Y}\brak{k}= \begin{cases} 
      \frac{1}{3} & {k=0} \\
      \frac{2}{3 }& {k=1} 
   \end{cases}
   \\
p_{Y|X}\brak{0|0} = \frac{19}{25}\, 
p_{Y|X}\brak{0|1} = \frac{6}{25}\,
p_{Y|X}\brak{1|0} = \frac{45}{50}\,
p_{Y|X}\brak{1|2} = \frac{5}{50}
\end{align}
The desired probability is the probability that a slip drawn at random is marked other than Rs 1,
\begin{align}
&=1-p_X\brak{0}\\
&= p_X(1) + p_X(2)
\end{align}
Using Bayes theorem,
\begin{align}
&= p_Y\brak{0} \times \pr{Y=0 | X=1} + p_Y\brak{1} \times \pr{Y=1|X=2}\\
&=\frac{1}{3} \times \frac{6}{25} + \frac{2}{3} \times \frac{5}{50}\\
&=\frac{11}{75}
\end{align}

\newpage

%\tableofcontents

\bigskip

\renewcommand{\thefigure}{\theenumi}
\renewcommand{\thetable}{\theenumi}
%\renewcommand{\theequation}{\theenumi}

%\begin{abstract}
%%\boldmath
%In this letter, an algorithm for evaluating the exact analytical bit error rate  (BER)  for the piecewise linear (PL) combiner for  multiple relays is presented. Previous results were available only for upto three relays. The algorithm is unique in the sense that  the actual mathematical expressions, that are prohibitively large, need not be explicitly obtained. The diversity gain due to multiple relays is shown through plots of the analytical BER, well supported by simulations. 
%
%\end{abstract}
% IEEEtran.cls defaults to using nonbold math in the Abstract.
% This preserves the distinction between vectors and scalars. However,
% if the journal you are submitting to favors bold math in the abstract,
% then you can use LaTeX's standard command \boldmath at the very start
% of the abstract to achieve this. Many IEEE journals frown on math
% in the abstract anyway.

% Note that keywords are not normally used for peerreview papers.
%\begin{IEEEkeywords}
%Cooperative diversity, decode and forward, piecewise linear
%\end{IEEEkeywords}



% For peer review papers, you can put extra information on the cover
% page as needed:
% \ifCLASSOPTIONpeerreview
% \begin{center} \bfseries EDICS Category: 3-BBND \end{center}
% \fi
%
% For peerreview papers, this IEEEtran command inserts a page break and
% creates the second title. It will be ignored for other modes.
%\IEEEpeerreviewmaketitle




 \item A bag contain 24 balls of which $x$ balls are red, $2x$ are white and $3x$ are blue. A ball is selected at random, What is the probability that it is
\begin{enumerate}[label=\alph*)]
\item not red ?
\item white ?
\end{enumerate}
%\begin{table}[H]
	\centering
\begin{tabular}{|c|c|c|}
\hline
Random variable &Value &Definition\\ \hline
\multirow{3}{*}{X} &0 &Slips of Rs 1\\
&1 &Slips of Rs 5\\
&2 &Slips of Rs 13\\ \hline
\multirow{2}{*}{Y} &0 &Box A\\
&1 &Box B\\\hline
\end{tabular}
\caption{}
\label{tab:Distribution}
\end{table}
See \tabref{tab:Distribution}.
\begin{align}
p_{Y}\brak{k}= \begin{cases} 
      \frac{1}{3} & {k=0} \\
      \frac{2}{3 }& {k=1} 
   \end{cases}
   \\
p_{Y|X}\brak{0|0} = \frac{19}{25}\, 
p_{Y|X}\brak{0|1} = \frac{6}{25}\,
p_{Y|X}\brak{1|0} = \frac{45}{50}\,
p_{Y|X}\brak{1|2} = \frac{5}{50}
\end{align}
The desired probability is the probability that a slip drawn at random is marked other than Rs 1,
\begin{align}
&=1-p_X\brak{0}\\
&= p_X(1) + p_X(2)
\end{align}
Using Bayes theorem,
\begin{align}
&= p_Y\brak{0} \times \pr{Y=0 | X=1} + p_Y\brak{1} \times \pr{Y=1|X=2}\\
&=\frac{1}{3} \times \frac{6}{25} + \frac{2}{3} \times \frac{5}{50}\\
&=\frac{11}{75}
\end{align}

\newpage

%\tableofcontents

\bigskip

\renewcommand{\thefigure}{\theenumi}
\renewcommand{\thetable}{\theenumi}
%\renewcommand{\theequation}{\theenumi}

%\begin{abstract}
%%\boldmath
%In this letter, an algorithm for evaluating the exact analytical bit error rate  (BER)  for the piecewise linear (PL) combiner for  multiple relays is presented. Previous results were available only for upto three relays. The algorithm is unique in the sense that  the actual mathematical expressions, that are prohibitively large, need not be explicitly obtained. The diversity gain due to multiple relays is shown through plots of the analytical BER, well supported by simulations. 
%
%\end{abstract}
% IEEEtran.cls defaults to using nonbold math in the Abstract.
% This preserves the distinction between vectors and scalars. However,
% if the journal you are submitting to favors bold math in the abstract,
% then you can use LaTeX's standard command \boldmath at the very start
% of the abstract to achieve this. Many IEEE journals frown on math
% in the abstract anyway.

% Note that keywords are not normally used for peerreview papers.
%\begin{IEEEkeywords}
%Cooperative diversity, decode and forward, piecewise linear
%\end{IEEEkeywords}



% For peer review papers, you can put extra information on the cover
% page as needed:
% \ifCLASSOPTIONpeerreview
% \begin{center} \bfseries EDICS Category: 3-BBND \end{center}
% \fi
%
% For peerreview papers, this IEEEtran command inserts a page break and
% creates the second title. It will be ignored for other modes.
%\IEEEpeerreviewmaketitle




If the letters of the word ASSASSINATION are arranged at random. Find the Probability that
\begin{enumerate}[label=(\alph*)]
\item Four $S's$ come consecutively in the word
\item Two  $I's$ and two $N's$ come together
\item All $A's$ are not coming together
\item No two $A's$ are coming together
\end{enumerate}
%\begin{table}[H]
	\centering
\begin{tabular}{|c|c|c|}
\hline
Random variable &Value &Definition\\ \hline
\multirow{3}{*}{X} &0 &Slips of Rs 1\\
&1 &Slips of Rs 5\\
&2 &Slips of Rs 13\\ \hline
\multirow{2}{*}{Y} &0 &Box A\\
&1 &Box B\\\hline
\end{tabular}
\caption{}
\label{tab:Distribution}
\end{table}
See \tabref{tab:Distribution}.
\begin{align}
p_{Y}\brak{k}= \begin{cases} 
      \frac{1}{3} & {k=0} \\
      \frac{2}{3 }& {k=1} 
   \end{cases}
   \\
p_{Y|X}\brak{0|0} = \frac{19}{25}\, 
p_{Y|X}\brak{0|1} = \frac{6}{25}\,
p_{Y|X}\brak{1|0} = \frac{45}{50}\,
p_{Y|X}\brak{1|2} = \frac{5}{50}
\end{align}
The desired probability is the probability that a slip drawn at random is marked other than Rs 1,
\begin{align}
&=1-p_X\brak{0}\\
&= p_X(1) + p_X(2)
\end{align}
Using Bayes theorem,
\begin{align}
&= p_Y\brak{0} \times \pr{Y=0 | X=1} + p_Y\brak{1} \times \pr{Y=1|X=2}\\
&=\frac{1}{3} \times \frac{6}{25} + \frac{2}{3} \times \frac{5}{50}\\
&=\frac{11}{75}
\end{align}

\newpage

%\tableofcontents

\bigskip

\renewcommand{\thefigure}{\theenumi}
\renewcommand{\thetable}{\theenumi}
%\renewcommand{\theequation}{\theenumi}

%\begin{abstract}
%%\boldmath
%In this letter, an algorithm for evaluating the exact analytical bit error rate  (BER)  for the piecewise linear (PL) combiner for  multiple relays is presented. Previous results were available only for upto three relays. The algorithm is unique in the sense that  the actual mathematical expressions, that are prohibitively large, need not be explicitly obtained. The diversity gain due to multiple relays is shown through plots of the analytical BER, well supported by simulations. 
%
%\end{abstract}
% IEEEtran.cls defaults to using nonbold math in the Abstract.
% This preserves the distinction between vectors and scalars. However,
% if the journal you are submitting to favors bold math in the abstract,
% then you can use LaTeX's standard command \boldmath at the very start
% of the abstract to achieve this. Many IEEE journals frown on math
% in the abstract anyway.

% Note that keywords are not normally used for peerreview papers.
%\begin{IEEEkeywords}
%Cooperative diversity, decode and forward, piecewise linear
%\end{IEEEkeywords}



% For peer review papers, you can put extra information on the cover
% page as needed:
% \ifCLASSOPTIONpeerreview
% \begin{center} \bfseries EDICS Category: 3-BBND \end{center}
% \fi
%
% For peerreview papers, this IEEEtran command inserts a page break and
% creates the second title. It will be ignored for other modes.
%\IEEEpeerreviewmaketitle




	\item One urn contains two black balls (labelled B1 and B2) and one white ball. A
	second urn contains one black ball and two white balls (labelled W1 and W2).
	Suppose the following experiment is performed. One of the two urns is chosen
	at random. Next a ball is randomly chosen from the urn. Then a second ball is
	chosen at random from the same urn without replacing the first ball.
	
	\begin{enumerate}
	\item What is the probability that two black balls are chosen?
	
	\item What is the probability that two balls of opposite colour are chosen?
	\end{enumerate}
	\solution
	%\begin{align}
    \label{eq:12.13.6.18.1}
	\because	\pr{A|B} &> \pr{A},\
\frac{\pr{AB}}{\pr{B}} > \pr{A}
\\
    \label{eq:12.13.6.18.2}
	\implies \pr{AB} &> \pr{A}\pr{B}
	\\
	\text{or, } \frac{\pr{AB}}{\pr{A}} &=\pr{B|A} > \pr{A}
\end{align}

\end{enumerate}

	\item 
The number lock of a suitcase has 4 wheels each labelled with ten digits i.e. from 0 to 9.The lock opens with a sequence of four digits with no repeats.What is the probability of a person getting the right sequence to open the suitcase.
\\
\solution
		%\begin{enumerate}[label=\thesection.\arabic*,ref=\thesection.\theenumi]
	\item One card is drawn from a well-shuffled deck of 52 cards. Find the probability of getting
\begin{enumerate}
\item A king of red colour 
\item A face card 
\item A red face card
\item The jack of hearts
\item A spade
\item The queen of diamonds

\end{enumerate}
\solution
		%\begin{table}[H]
	\centering
\begin{tabular}{|c|c|c|}
\hline
Random variable &Value &Definition\\ \hline
\multirow{3}{*}{X} &0 &Slips of Rs 1\\
&1 &Slips of Rs 5\\
&2 &Slips of Rs 13\\ \hline
\multirow{2}{*}{Y} &0 &Box A\\
&1 &Box B\\\hline
\end{tabular}
\caption{}
\label{tab:Distribution}
\end{table}
See \tabref{tab:Distribution}.
\begin{align}
p_{Y}\brak{k}= \begin{cases} 
      \frac{1}{3} & {k=0} \\
      \frac{2}{3 }& {k=1} 
   \end{cases}
   \\
p_{Y|X}\brak{0|0} = \frac{19}{25}\, 
p_{Y|X}\brak{0|1} = \frac{6}{25}\,
p_{Y|X}\brak{1|0} = \frac{45}{50}\,
p_{Y|X}\brak{1|2} = \frac{5}{50}
\end{align}
The desired probability is the probability that a slip drawn at random is marked other than Rs 1,
\begin{align}
&=1-p_X\brak{0}\\
&= p_X(1) + p_X(2)
\end{align}
Using Bayes theorem,
\begin{align}
&= p_Y\brak{0} \times \pr{Y=0 | X=1} + p_Y\brak{1} \times \pr{Y=1|X=2}\\
&=\frac{1}{3} \times \frac{6}{25} + \frac{2}{3} \times \frac{5}{50}\\
&=\frac{11}{75}
\end{align}

\newpage

%\tableofcontents

\bigskip

\renewcommand{\thefigure}{\theenumi}
\renewcommand{\thetable}{\theenumi}
%\renewcommand{\theequation}{\theenumi}

%\begin{abstract}
%%\boldmath
%In this letter, an algorithm for evaluating the exact analytical bit error rate  (BER)  for the piecewise linear (PL) combiner for  multiple relays is presented. Previous results were available only for upto three relays. The algorithm is unique in the sense that  the actual mathematical expressions, that are prohibitively large, need not be explicitly obtained. The diversity gain due to multiple relays is shown through plots of the analytical BER, well supported by simulations. 
%
%\end{abstract}
% IEEEtran.cls defaults to using nonbold math in the Abstract.
% This preserves the distinction between vectors and scalars. However,
% if the journal you are submitting to favors bold math in the abstract,
% then you can use LaTeX's standard command \boldmath at the very start
% of the abstract to achieve this. Many IEEE journals frown on math
% in the abstract anyway.

% Note that keywords are not normally used for peerreview papers.
%\begin{IEEEkeywords}
%Cooperative diversity, decode and forward, piecewise linear
%\end{IEEEkeywords}



% For peer review papers, you can put extra information on the cover
% page as needed:
% \ifCLASSOPTIONpeerreview
% \begin{center} \bfseries EDICS Category: 3-BBND \end{center}
% \fi
%
% For peerreview papers, this IEEEtran command inserts a page break and
% creates the second title. It will be ignored for other modes.
%\IEEEpeerreviewmaketitle




	\item Five cards—the ten, jack, queen, king and ace of diamonds, are well-shuffled with their face downwards. One card is then picked up at random.
\begin{enumerate}
\item
What is the probability that the card is the queen? 
\item
If the queen is drawn and put aside, what is the probability that the second card picked up is (a) an ace? (b) a queen?\\
\end{enumerate}
\solution
		%\begin{enumerate}[label=\thesection.\arabic*,ref=\thesection.\theenumi]
	\item One card is drawn from a well-shuffled deck of 52 cards. Find the probability of getting
\begin{enumerate}
\item A king of red colour 
\item A face card 
\item A red face card
\item The jack of hearts
\item A spade
\item The queen of diamonds

\end{enumerate}
\solution
		%\input{ncert/10/15/1/14/main.tex}
	\item Five cards—the ten, jack, queen, king and ace of diamonds, are well-shuffled with their face downwards. One card is then picked up at random.
\begin{enumerate}
\item
What is the probability that the card is the queen? 
\item
If the queen is drawn and put aside, what is the probability that the second card picked up is (a) an ace? (b) a queen?\\
\end{enumerate}
\solution
		%\input{ncert/10/15/1/15/defs.tex}
	\item A bag contains $5$ red balls and some blue balls. If the probability of drawing a blue ball is double that if a red ball, determine the number of blue balls in the bag. 
		\\
\solution
		%\input{ncert/10/15/2/3/defs.tex}
	\item A card is selected from a pack of 52 cards.
 \begin{enumerate}[label=(\alph*)] 
                 \item How many points are there in the sample space?
                 \item Calculate the probability that the card is an ace of spades.
                 \item Calculate the probability that the card is (i) an ace and (ii) black card.
 \end{enumerate}
\solution
		%\input{ncert/11/16/3/4/main.tex}
\item Four cards are drawn from a well-shuffled deck of 52 cards. What is the probability of obtaining 3 diamonds and one spade.
\\
\solution
		%\input{ncert/11/16/4/2/defs.tex}
\item In a certain lottery 10,000 tickets are sold and ten equal prizes are awarded. What is the probability of not getting a prize if you buy (a) one ticket (b) two tickets (c) 10 tickets ?	
\\
\solution
		%\input{ncert/11/16/4/4/defs.tex}
		%
\item 
Out of 100 students, two sections of 40 and 60 are formed. If you and your friend are among the 100 students, what is the probability that
\begin{enumerate}
\item you both enter the same section?
\item you both enter the different sections?
\end{enumerate}
\solution
		%\input{ncert/11/16/4/5/defs.tex}
	\item 
The number lock of a suitcase has 4 wheels each labelled with ten digits i.e. from 0 to 9.The lock opens with a sequence of four digits with no repeats.What is the probability of a person getting the right sequence to open the suitcase.
\\
\solution
		%\input{ncert/11/16/4/10/defs.tex}
		%
\item 
Two cards are drawn at random and without replacement from a pack of 52 playing cards. Find the probability that both the cards are black.
\\
\solution
		%\input{ncert/12/13/2/2/defs.tex}
		\item A box of oranges is inspected by examining three randomly selected oranges drawn without replacement. If all the three oranges are good, the box is approved for sale, otherwise, it is rejected. Find the probability that a box containing 15 oranges out of which 12 are good and 3 are bad ones will be approved for sale.
		\label{ncert/12/13/2/3/defs.tex}
		\item Two balls are drawn at random with replacement from a box containing 10 black and 8 red balls. Find the probability that
		\label{ncert/12/13/2/12}
\begin{enumerate}
\item both balls are red.
\item first ball is black and second is red.
\item one of them is black and other is red.
\end{enumerate}

\item In a hostel, 60\% of the students read Hindi newspaper, 40\% read English newspaper and 20\% read both Hindi and English newspapers. A student is selected at random.
		\label{ncert/12/13/2/15}
\begin{enumerate}
\item Find the probability that she reads neither Hindi nor English newspapers.
\item If she reads Hindi newspaper, find the probability that she reads English newspaper.
\item If she reads English newspaper, find the probability that she reads Hindi newspaper.\\
\end{enumerate}
\item The probability of obtaining an even prime number on each die, when a pair of dice is rolled is 
\begin{enumerate}
    \item $0$ 
    
    \item $\frac{1}{3}$ 
    
    \item $\frac{1}{12}$ 
    
    \item $\frac{1}{36}$ 
\end{enumerate}
\solution
		%\input{ncert/12/13/2/17/defs.tex}
	\item A bag contains 4 red and 4 black balls, another bag contains 2 red and 6 black balls. One of the two bags is selected at random and a ball is drawn from the bag which is found to be red. Find the probability that the ball is drawn from the first bag.
\\
\solution
		%\input{ncert/12/13/3/2/main.tex}
  \item
  Cards with numbers 2 to 101 are placed in a box. A card is selected at random.Find the probability that the card has
\begin{enumerate}[label=(\roman*)]
	\item an even number 
	\item a square number
\end{enumerate}
\solution
%\input{exemplar/10/13/3/32/main.tex}
\item
The king, queen and jack of clubs are removed from a deck of 52 playing cards and then well shuffled. Now one card is drawn at random from the remaining cards.  Determine the probability that the card is
\begin{enumerate}[label=(\roman*)]
\item a club
\item 10 of hearts
\end{enumerate}
\solution
%\input{exemplar/10/13/3/29/main.tex}
\item A team of medical students doing their internship have to assist during surgeries
at a city hospital. The probabilities of surgeries rated as very complex, complex,
routine, simple or very simple are respectively, 0.15, 0.20, 0.31, 0.26, .08. Find
the probabilities that a particular surgery will be rated
\begin{enumerate}
	\item complex or very complex;
	\item neither very complex nor very simple;
	\item routine or complex
	\item routine or simple
\end{enumerate}
\solution
%\input{exemplar/11/16/3/8(1)/main.tex}
\item A card is selected from a pack of 52 cards.
\begin{enumerate}[label=(\alph*)]
    \item How many points are there in the sample space?
    \item Calculate the probability that the card is an ace of spades.
    \item Calculate the probability that the card is (i) an ace and (ii) black card.
\end{enumerate}
\solution
%\input{exemplar/11/16/3/4/main2.tex}
\item The probability that a non leap year selected at random will contain 53 sundays.
\\
\solution
%\input{exemplar/10/13/1/19/main.tex}
\item One of the four persons John, Rita, Aslam or Gurpreet will be promoted next
month. Consequently the sample space consists of four elementary outcomes
S = {John promoted, Rita promoted, Aslam promoted, Gurpreet promoted}
You are told that the chances of John’s promotion is same as that of Gurpreet,
Rita’s chances of promotion are twice as likely as Johns. Aslam’s chances are
four times that of John.
\begin{enumerate}
	\item Determine
	\begin{enumerate}
		\item P (John promoted)
		\item P (Rita promoted)
		\item P (Aslam promoted)
		\item P (Gurpreet promoted)
	\end{enumerate}
	\item If A = {John promoted or Gurpreet promoted}, find P (A).
\end{enumerate}
\solution
%\input{exemplar/11/16/3/10/main.tex}
\item A card is drawn from a deck of 52 cards. Find the probability of getting a king or a heart or a red card.\\
\solution
%\input{exemplar/11/16/3/15/main.tex}
\item The probability that a student will pass his examination is 0.73, the probability of
the student getting a compartment is 0.13, and the probability that the student will
either pass or get compartment is 0.96. State True or False.\\
\solution
%\input{exemplar/11/16/3/31/main.tex}
\item A card is selected from a pack of 52 cards\\
\begin{enumerate}[label=(\alph*)]
\item How many points are there in the sample space?
\item Calculate the probability that the cards is an ace of spades.
\item Calculate the probability that the card is (i) an ace (ii)black card.\\
\end{enumerate}
%\input{ncert/11/16/3/4_1/Prob_4.tex}
\item In a non-leap year, the probability of having 53 tuesdays or 53 wednesdays is\\
\solution
%\input{exemplar/11/16/3/18/main.tex}
\item There are 1000 sealed envelopes in a box, 10 of them contain a cash prize of
Rs 100 each, 100 of them contain a cash prize of Rs 50 each and 200 of them
contain a cash prize of Rs 10 each and rest do not contain any cash prize. If they
are well shuffled and an envelope is picked up out, what is the probability that it
contains no cash prize?\\
\solution
%\input{exemplar/10/13/3/34/main.tex}
\item 
A die is thrown and a card is selected at random from a deck of 52 playing cards. The probability of getting an even number on the die and a spade card.\\
\solution
%\input{exemplar/12/13/3/78/main.tex}
\item
If 4-digit numbers greater than 5,000 are randomly formed from the digits 0, 1, 3, 5, and 7, what is the probability of forming a number divisible by 5 when:
\begin{enumerate}
    \item The digits are repeated?
    \item The repetition of digits is not allowed?
\end{enumerate}
\solution
%\input{ncert/11/16/4/9/main.tex}
\item Consider the probability space $\brak{\Omega, \mathcal{G}, P}$ where $\Omega = [0,2]$ and $\mathcal{G} = \cbrak{\phi, \Omega, [0,1], (1,2]}$. Let $X$ and $Y$ be two functions on $\Omega$ defined as
\begin{align*}
    X(\omega) = 
    \begin{cases}
        1 & \text{if }\omega \in [0, 1]\\
        2 & \text{if }\omega \in (1, 2]
    \end{cases}
\end{align*}
and
\begin{align*}
    Y(\omega) = 
    \begin{cases}
        2 & \text{if }\omega \in [0, 1.5]\\
        3 & \text{if }\omega \in (1.5, 2].
    \end{cases}
\end{align*}
Then which one of the following statements is true?
\begin{enumerate}
    \item [(A)] $X$ is a random variable with respect to $\mathcal{G}$, but $Y$ is not a random variable with respect to $\mathcal{G}$.
    \item [(B)] $Y$ is a random variable with respect to $\mathcal{G}$, but $X$ is not a random variable with respect to $\mathcal{G}$.
    \item [(C)] Neither $X$ nor $Y$ is a random variable with respect to $\mathcal{G}$.
    \item [(D)] Both $X$ and $Y$ are random variables with respect to $\mathcal{G}$.
\end{enumerate} \hfill (GATE ST 2023)\\
\solution
%\input{gate/ST/2023/14/main.tex}
	\item  A die is loaded in such a way that each odd number is twice as likely to occur as
each even number. Find $P(G)$, where $G$ is the event that a number greater than
3 occurs on a single roll of the die.
\\
\solution
		%\input{exemplar/11/16/3/5/main.tex}
	\item All the jacks, queens and kings are removed from a deck of 52 playing cards. The remaining cards are well shuffled and then one card is drawn at random. Giving ace a value 1 similar value for other cards, find the probability that the card has a value 
		\begin{enumerate}
			\item 7
			\item greater than 7
			\item less than 7
		\end{enumerate}
		%\input{exemplar/10/13/3/30/main.tex}
  \item A Lot consists of 48 mobile phones of which 42 are good, 3 have only minor defects and 3 have major defects.Varnika will buy a phone if it is good but the trader will only buy a mobile if it has no major defects. One phone is selected at random from the lot. What is the probability that it is
\begin{enumerate}
	\item acceptable to Varnika?
            \item acceptable to the trader?
\end{enumerate}
\solution
	%\input{exemplar/10/13/3/40/main.tex}
 \item A student says that if you throw a die, it will show up 1 or not 1. Therefore, the probability of getting 1 and the probability of getting 'not 1' each is equal to $\frac{1}{2}$. Is this correct? Give reasons.\\
 \solution
        %\input{exemplar/10/13/2/9/main.tex}
   \item Four candidates A, B, C, D have ap-
plied for the assignment to coach a school cricket
team. If A is twice as likely to be selected as B, and
B and C are given about the same chance of being
selected, while C is twice as likely to be selected
as D, what are the probabilities that
\begin{enumerate}
\item C will be selected?
\item A will not be selected?
\end{enumerate}
	%\input{exemplar/11/16/3/9/main.tex}
 \item A bag contain 24 balls of which $x$ balls are red, $2x$ are white and $3x$ are blue. A ball is selected at random, What is the probability that it is
\begin{enumerate}[label=\alph*)]
\item not red ?
\item white ?
\end{enumerate}
%\input{exemplar/10/13/3/41/main.tex}
If the letters of the word ASSASSINATION are arranged at random. Find the Probability that
\begin{enumerate}[label=(\alph*)]
\item Four $S's$ come consecutively in the word
\item Two  $I's$ and two $N's$ come together
\item All $A's$ are not coming together
\item No two $A's$ are coming together
\end{enumerate}
%\input{exemplar/11/16/3/14/main.tex}
	\item One urn contains two black balls (labelled B1 and B2) and one white ball. A
	second urn contains one black ball and two white balls (labelled W1 and W2).
	Suppose the following experiment is performed. One of the two urns is chosen
	at random. Next a ball is randomly chosen from the urn. Then a second ball is
	chosen at random from the same urn without replacing the first ball.
	
	\begin{enumerate}
	\item What is the probability that two black balls are chosen?
	
	\item What is the probability that two balls of opposite colour are chosen?
	\end{enumerate}
	\solution
	%\input{exemplar/11/16/3/12/main1.tex}
\end{enumerate}

	\item A bag contains $5$ red balls and some blue balls. If the probability of drawing a blue ball is double that if a red ball, determine the number of blue balls in the bag. 
		\\
\solution
		%\begin{enumerate}[label=\thesection.\arabic*,ref=\thesection.\theenumi]
	\item One card is drawn from a well-shuffled deck of 52 cards. Find the probability of getting
\begin{enumerate}
\item A king of red colour 
\item A face card 
\item A red face card
\item The jack of hearts
\item A spade
\item The queen of diamonds

\end{enumerate}
\solution
		%\input{ncert/10/15/1/14/main.tex}
	\item Five cards—the ten, jack, queen, king and ace of diamonds, are well-shuffled with their face downwards. One card is then picked up at random.
\begin{enumerate}
\item
What is the probability that the card is the queen? 
\item
If the queen is drawn and put aside, what is the probability that the second card picked up is (a) an ace? (b) a queen?\\
\end{enumerate}
\solution
		%\input{ncert/10/15/1/15/defs.tex}
	\item A bag contains $5$ red balls and some blue balls. If the probability of drawing a blue ball is double that if a red ball, determine the number of blue balls in the bag. 
		\\
\solution
		%\input{ncert/10/15/2/3/defs.tex}
	\item A card is selected from a pack of 52 cards.
 \begin{enumerate}[label=(\alph*)] 
                 \item How many points are there in the sample space?
                 \item Calculate the probability that the card is an ace of spades.
                 \item Calculate the probability that the card is (i) an ace and (ii) black card.
 \end{enumerate}
\solution
		%\input{ncert/11/16/3/4/main.tex}
\item Four cards are drawn from a well-shuffled deck of 52 cards. What is the probability of obtaining 3 diamonds and one spade.
\\
\solution
		%\input{ncert/11/16/4/2/defs.tex}
\item In a certain lottery 10,000 tickets are sold and ten equal prizes are awarded. What is the probability of not getting a prize if you buy (a) one ticket (b) two tickets (c) 10 tickets ?	
\\
\solution
		%\input{ncert/11/16/4/4/defs.tex}
		%
\item 
Out of 100 students, two sections of 40 and 60 are formed. If you and your friend are among the 100 students, what is the probability that
\begin{enumerate}
\item you both enter the same section?
\item you both enter the different sections?
\end{enumerate}
\solution
		%\input{ncert/11/16/4/5/defs.tex}
	\item 
The number lock of a suitcase has 4 wheels each labelled with ten digits i.e. from 0 to 9.The lock opens with a sequence of four digits with no repeats.What is the probability of a person getting the right sequence to open the suitcase.
\\
\solution
		%\input{ncert/11/16/4/10/defs.tex}
		%
\item 
Two cards are drawn at random and without replacement from a pack of 52 playing cards. Find the probability that both the cards are black.
\\
\solution
		%\input{ncert/12/13/2/2/defs.tex}
		\item A box of oranges is inspected by examining three randomly selected oranges drawn without replacement. If all the three oranges are good, the box is approved for sale, otherwise, it is rejected. Find the probability that a box containing 15 oranges out of which 12 are good and 3 are bad ones will be approved for sale.
		\label{ncert/12/13/2/3/defs.tex}
		\item Two balls are drawn at random with replacement from a box containing 10 black and 8 red balls. Find the probability that
		\label{ncert/12/13/2/12}
\begin{enumerate}
\item both balls are red.
\item first ball is black and second is red.
\item one of them is black and other is red.
\end{enumerate}

\item In a hostel, 60\% of the students read Hindi newspaper, 40\% read English newspaper and 20\% read both Hindi and English newspapers. A student is selected at random.
		\label{ncert/12/13/2/15}
\begin{enumerate}
\item Find the probability that she reads neither Hindi nor English newspapers.
\item If she reads Hindi newspaper, find the probability that she reads English newspaper.
\item If she reads English newspaper, find the probability that she reads Hindi newspaper.\\
\end{enumerate}
\item The probability of obtaining an even prime number on each die, when a pair of dice is rolled is 
\begin{enumerate}
    \item $0$ 
    
    \item $\frac{1}{3}$ 
    
    \item $\frac{1}{12}$ 
    
    \item $\frac{1}{36}$ 
\end{enumerate}
\solution
		%\input{ncert/12/13/2/17/defs.tex}
	\item A bag contains 4 red and 4 black balls, another bag contains 2 red and 6 black balls. One of the two bags is selected at random and a ball is drawn from the bag which is found to be red. Find the probability that the ball is drawn from the first bag.
\\
\solution
		%\input{ncert/12/13/3/2/main.tex}
  \item
  Cards with numbers 2 to 101 are placed in a box. A card is selected at random.Find the probability that the card has
\begin{enumerate}[label=(\roman*)]
	\item an even number 
	\item a square number
\end{enumerate}
\solution
%\input{exemplar/10/13/3/32/main.tex}
\item
The king, queen and jack of clubs are removed from a deck of 52 playing cards and then well shuffled. Now one card is drawn at random from the remaining cards.  Determine the probability that the card is
\begin{enumerate}[label=(\roman*)]
\item a club
\item 10 of hearts
\end{enumerate}
\solution
%\input{exemplar/10/13/3/29/main.tex}
\item A team of medical students doing their internship have to assist during surgeries
at a city hospital. The probabilities of surgeries rated as very complex, complex,
routine, simple or very simple are respectively, 0.15, 0.20, 0.31, 0.26, .08. Find
the probabilities that a particular surgery will be rated
\begin{enumerate}
	\item complex or very complex;
	\item neither very complex nor very simple;
	\item routine or complex
	\item routine or simple
\end{enumerate}
\solution
%\input{exemplar/11/16/3/8(1)/main.tex}
\item A card is selected from a pack of 52 cards.
\begin{enumerate}[label=(\alph*)]
    \item How many points are there in the sample space?
    \item Calculate the probability that the card is an ace of spades.
    \item Calculate the probability that the card is (i) an ace and (ii) black card.
\end{enumerate}
\solution
%\input{exemplar/11/16/3/4/main2.tex}
\item The probability that a non leap year selected at random will contain 53 sundays.
\\
\solution
%\input{exemplar/10/13/1/19/main.tex}
\item One of the four persons John, Rita, Aslam or Gurpreet will be promoted next
month. Consequently the sample space consists of four elementary outcomes
S = {John promoted, Rita promoted, Aslam promoted, Gurpreet promoted}
You are told that the chances of John’s promotion is same as that of Gurpreet,
Rita’s chances of promotion are twice as likely as Johns. Aslam’s chances are
four times that of John.
\begin{enumerate}
	\item Determine
	\begin{enumerate}
		\item P (John promoted)
		\item P (Rita promoted)
		\item P (Aslam promoted)
		\item P (Gurpreet promoted)
	\end{enumerate}
	\item If A = {John promoted or Gurpreet promoted}, find P (A).
\end{enumerate}
\solution
%\input{exemplar/11/16/3/10/main.tex}
\item A card is drawn from a deck of 52 cards. Find the probability of getting a king or a heart or a red card.\\
\solution
%\input{exemplar/11/16/3/15/main.tex}
\item The probability that a student will pass his examination is 0.73, the probability of
the student getting a compartment is 0.13, and the probability that the student will
either pass or get compartment is 0.96. State True or False.\\
\solution
%\input{exemplar/11/16/3/31/main.tex}
\item A card is selected from a pack of 52 cards\\
\begin{enumerate}[label=(\alph*)]
\item How many points are there in the sample space?
\item Calculate the probability that the cards is an ace of spades.
\item Calculate the probability that the card is (i) an ace (ii)black card.\\
\end{enumerate}
%\input{ncert/11/16/3/4_1/Prob_4.tex}
\item In a non-leap year, the probability of having 53 tuesdays or 53 wednesdays is\\
\solution
%\input{exemplar/11/16/3/18/main.tex}
\item There are 1000 sealed envelopes in a box, 10 of them contain a cash prize of
Rs 100 each, 100 of them contain a cash prize of Rs 50 each and 200 of them
contain a cash prize of Rs 10 each and rest do not contain any cash prize. If they
are well shuffled and an envelope is picked up out, what is the probability that it
contains no cash prize?\\
\solution
%\input{exemplar/10/13/3/34/main.tex}
\item 
A die is thrown and a card is selected at random from a deck of 52 playing cards. The probability of getting an even number on the die and a spade card.\\
\solution
%\input{exemplar/12/13/3/78/main.tex}
\item
If 4-digit numbers greater than 5,000 are randomly formed from the digits 0, 1, 3, 5, and 7, what is the probability of forming a number divisible by 5 when:
\begin{enumerate}
    \item The digits are repeated?
    \item The repetition of digits is not allowed?
\end{enumerate}
\solution
%\input{ncert/11/16/4/9/main.tex}
\item Consider the probability space $\brak{\Omega, \mathcal{G}, P}$ where $\Omega = [0,2]$ and $\mathcal{G} = \cbrak{\phi, \Omega, [0,1], (1,2]}$. Let $X$ and $Y$ be two functions on $\Omega$ defined as
\begin{align*}
    X(\omega) = 
    \begin{cases}
        1 & \text{if }\omega \in [0, 1]\\
        2 & \text{if }\omega \in (1, 2]
    \end{cases}
\end{align*}
and
\begin{align*}
    Y(\omega) = 
    \begin{cases}
        2 & \text{if }\omega \in [0, 1.5]\\
        3 & \text{if }\omega \in (1.5, 2].
    \end{cases}
\end{align*}
Then which one of the following statements is true?
\begin{enumerate}
    \item [(A)] $X$ is a random variable with respect to $\mathcal{G}$, but $Y$ is not a random variable with respect to $\mathcal{G}$.
    \item [(B)] $Y$ is a random variable with respect to $\mathcal{G}$, but $X$ is not a random variable with respect to $\mathcal{G}$.
    \item [(C)] Neither $X$ nor $Y$ is a random variable with respect to $\mathcal{G}$.
    \item [(D)] Both $X$ and $Y$ are random variables with respect to $\mathcal{G}$.
\end{enumerate} \hfill (GATE ST 2023)\\
\solution
%\input{gate/ST/2023/14/main.tex}
	\item  A die is loaded in such a way that each odd number is twice as likely to occur as
each even number. Find $P(G)$, where $G$ is the event that a number greater than
3 occurs on a single roll of the die.
\\
\solution
		%\input{exemplar/11/16/3/5/main.tex}
	\item All the jacks, queens and kings are removed from a deck of 52 playing cards. The remaining cards are well shuffled and then one card is drawn at random. Giving ace a value 1 similar value for other cards, find the probability that the card has a value 
		\begin{enumerate}
			\item 7
			\item greater than 7
			\item less than 7
		\end{enumerate}
		%\input{exemplar/10/13/3/30/main.tex}
  \item A Lot consists of 48 mobile phones of which 42 are good, 3 have only minor defects and 3 have major defects.Varnika will buy a phone if it is good but the trader will only buy a mobile if it has no major defects. One phone is selected at random from the lot. What is the probability that it is
\begin{enumerate}
	\item acceptable to Varnika?
            \item acceptable to the trader?
\end{enumerate}
\solution
	%\input{exemplar/10/13/3/40/main.tex}
 \item A student says that if you throw a die, it will show up 1 or not 1. Therefore, the probability of getting 1 and the probability of getting 'not 1' each is equal to $\frac{1}{2}$. Is this correct? Give reasons.\\
 \solution
        %\input{exemplar/10/13/2/9/main.tex}
   \item Four candidates A, B, C, D have ap-
plied for the assignment to coach a school cricket
team. If A is twice as likely to be selected as B, and
B and C are given about the same chance of being
selected, while C is twice as likely to be selected
as D, what are the probabilities that
\begin{enumerate}
\item C will be selected?
\item A will not be selected?
\end{enumerate}
	%\input{exemplar/11/16/3/9/main.tex}
 \item A bag contain 24 balls of which $x$ balls are red, $2x$ are white and $3x$ are blue. A ball is selected at random, What is the probability that it is
\begin{enumerate}[label=\alph*)]
\item not red ?
\item white ?
\end{enumerate}
%\input{exemplar/10/13/3/41/main.tex}
If the letters of the word ASSASSINATION are arranged at random. Find the Probability that
\begin{enumerate}[label=(\alph*)]
\item Four $S's$ come consecutively in the word
\item Two  $I's$ and two $N's$ come together
\item All $A's$ are not coming together
\item No two $A's$ are coming together
\end{enumerate}
%\input{exemplar/11/16/3/14/main.tex}
	\item One urn contains two black balls (labelled B1 and B2) and one white ball. A
	second urn contains one black ball and two white balls (labelled W1 and W2).
	Suppose the following experiment is performed. One of the two urns is chosen
	at random. Next a ball is randomly chosen from the urn. Then a second ball is
	chosen at random from the same urn without replacing the first ball.
	
	\begin{enumerate}
	\item What is the probability that two black balls are chosen?
	
	\item What is the probability that two balls of opposite colour are chosen?
	\end{enumerate}
	\solution
	%\input{exemplar/11/16/3/12/main1.tex}
\end{enumerate}

	\item A card is selected from a pack of 52 cards.
 \begin{enumerate}[label=(\alph*)] 
                 \item How many points are there in the sample space?
                 \item Calculate the probability that the card is an ace of spades.
                 \item Calculate the probability that the card is (i) an ace and (ii) black card.
 \end{enumerate}
\solution
		%\begin{table}[H]
	\centering
\begin{tabular}{|c|c|c|}
\hline
Random variable &Value &Definition\\ \hline
\multirow{3}{*}{X} &0 &Slips of Rs 1\\
&1 &Slips of Rs 5\\
&2 &Slips of Rs 13\\ \hline
\multirow{2}{*}{Y} &0 &Box A\\
&1 &Box B\\\hline
\end{tabular}
\caption{}
\label{tab:Distribution}
\end{table}
See \tabref{tab:Distribution}.
\begin{align}
p_{Y}\brak{k}= \begin{cases} 
      \frac{1}{3} & {k=0} \\
      \frac{2}{3 }& {k=1} 
   \end{cases}
   \\
p_{Y|X}\brak{0|0} = \frac{19}{25}\, 
p_{Y|X}\brak{0|1} = \frac{6}{25}\,
p_{Y|X}\brak{1|0} = \frac{45}{50}\,
p_{Y|X}\brak{1|2} = \frac{5}{50}
\end{align}
The desired probability is the probability that a slip drawn at random is marked other than Rs 1,
\begin{align}
&=1-p_X\brak{0}\\
&= p_X(1) + p_X(2)
\end{align}
Using Bayes theorem,
\begin{align}
&= p_Y\brak{0} \times \pr{Y=0 | X=1} + p_Y\brak{1} \times \pr{Y=1|X=2}\\
&=\frac{1}{3} \times \frac{6}{25} + \frac{2}{3} \times \frac{5}{50}\\
&=\frac{11}{75}
\end{align}

\newpage

%\tableofcontents

\bigskip

\renewcommand{\thefigure}{\theenumi}
\renewcommand{\thetable}{\theenumi}
%\renewcommand{\theequation}{\theenumi}

%\begin{abstract}
%%\boldmath
%In this letter, an algorithm for evaluating the exact analytical bit error rate  (BER)  for the piecewise linear (PL) combiner for  multiple relays is presented. Previous results were available only for upto three relays. The algorithm is unique in the sense that  the actual mathematical expressions, that are prohibitively large, need not be explicitly obtained. The diversity gain due to multiple relays is shown through plots of the analytical BER, well supported by simulations. 
%
%\end{abstract}
% IEEEtran.cls defaults to using nonbold math in the Abstract.
% This preserves the distinction between vectors and scalars. However,
% if the journal you are submitting to favors bold math in the abstract,
% then you can use LaTeX's standard command \boldmath at the very start
% of the abstract to achieve this. Many IEEE journals frown on math
% in the abstract anyway.

% Note that keywords are not normally used for peerreview papers.
%\begin{IEEEkeywords}
%Cooperative diversity, decode and forward, piecewise linear
%\end{IEEEkeywords}



% For peer review papers, you can put extra information on the cover
% page as needed:
% \ifCLASSOPTIONpeerreview
% \begin{center} \bfseries EDICS Category: 3-BBND \end{center}
% \fi
%
% For peerreview papers, this IEEEtran command inserts a page break and
% creates the second title. It will be ignored for other modes.
%\IEEEpeerreviewmaketitle




\item Four cards are drawn from a well-shuffled deck of 52 cards. What is the probability of obtaining 3 diamonds and one spade.
\\
\solution
		%\begin{enumerate}[label=\thesection.\arabic*,ref=\thesection.\theenumi]
	\item One card is drawn from a well-shuffled deck of 52 cards. Find the probability of getting
\begin{enumerate}
\item A king of red colour 
\item A face card 
\item A red face card
\item The jack of hearts
\item A spade
\item The queen of diamonds

\end{enumerate}
\solution
		%\input{ncert/10/15/1/14/main.tex}
	\item Five cards—the ten, jack, queen, king and ace of diamonds, are well-shuffled with their face downwards. One card is then picked up at random.
\begin{enumerate}
\item
What is the probability that the card is the queen? 
\item
If the queen is drawn and put aside, what is the probability that the second card picked up is (a) an ace? (b) a queen?\\
\end{enumerate}
\solution
		%\input{ncert/10/15/1/15/defs.tex}
	\item A bag contains $5$ red balls and some blue balls. If the probability of drawing a blue ball is double that if a red ball, determine the number of blue balls in the bag. 
		\\
\solution
		%\input{ncert/10/15/2/3/defs.tex}
	\item A card is selected from a pack of 52 cards.
 \begin{enumerate}[label=(\alph*)] 
                 \item How many points are there in the sample space?
                 \item Calculate the probability that the card is an ace of spades.
                 \item Calculate the probability that the card is (i) an ace and (ii) black card.
 \end{enumerate}
\solution
		%\input{ncert/11/16/3/4/main.tex}
\item Four cards are drawn from a well-shuffled deck of 52 cards. What is the probability of obtaining 3 diamonds and one spade.
\\
\solution
		%\input{ncert/11/16/4/2/defs.tex}
\item In a certain lottery 10,000 tickets are sold and ten equal prizes are awarded. What is the probability of not getting a prize if you buy (a) one ticket (b) two tickets (c) 10 tickets ?	
\\
\solution
		%\input{ncert/11/16/4/4/defs.tex}
		%
\item 
Out of 100 students, two sections of 40 and 60 are formed. If you and your friend are among the 100 students, what is the probability that
\begin{enumerate}
\item you both enter the same section?
\item you both enter the different sections?
\end{enumerate}
\solution
		%\input{ncert/11/16/4/5/defs.tex}
	\item 
The number lock of a suitcase has 4 wheels each labelled with ten digits i.e. from 0 to 9.The lock opens with a sequence of four digits with no repeats.What is the probability of a person getting the right sequence to open the suitcase.
\\
\solution
		%\input{ncert/11/16/4/10/defs.tex}
		%
\item 
Two cards are drawn at random and without replacement from a pack of 52 playing cards. Find the probability that both the cards are black.
\\
\solution
		%\input{ncert/12/13/2/2/defs.tex}
		\item A box of oranges is inspected by examining three randomly selected oranges drawn without replacement. If all the three oranges are good, the box is approved for sale, otherwise, it is rejected. Find the probability that a box containing 15 oranges out of which 12 are good and 3 are bad ones will be approved for sale.
		\label{ncert/12/13/2/3/defs.tex}
		\item Two balls are drawn at random with replacement from a box containing 10 black and 8 red balls. Find the probability that
		\label{ncert/12/13/2/12}
\begin{enumerate}
\item both balls are red.
\item first ball is black and second is red.
\item one of them is black and other is red.
\end{enumerate}

\item In a hostel, 60\% of the students read Hindi newspaper, 40\% read English newspaper and 20\% read both Hindi and English newspapers. A student is selected at random.
		\label{ncert/12/13/2/15}
\begin{enumerate}
\item Find the probability that she reads neither Hindi nor English newspapers.
\item If she reads Hindi newspaper, find the probability that she reads English newspaper.
\item If she reads English newspaper, find the probability that she reads Hindi newspaper.\\
\end{enumerate}
\item The probability of obtaining an even prime number on each die, when a pair of dice is rolled is 
\begin{enumerate}
    \item $0$ 
    
    \item $\frac{1}{3}$ 
    
    \item $\frac{1}{12}$ 
    
    \item $\frac{1}{36}$ 
\end{enumerate}
\solution
		%\input{ncert/12/13/2/17/defs.tex}
	\item A bag contains 4 red and 4 black balls, another bag contains 2 red and 6 black balls. One of the two bags is selected at random and a ball is drawn from the bag which is found to be red. Find the probability that the ball is drawn from the first bag.
\\
\solution
		%\input{ncert/12/13/3/2/main.tex}
  \item
  Cards with numbers 2 to 101 are placed in a box. A card is selected at random.Find the probability that the card has
\begin{enumerate}[label=(\roman*)]
	\item an even number 
	\item a square number
\end{enumerate}
\solution
%\input{exemplar/10/13/3/32/main.tex}
\item
The king, queen and jack of clubs are removed from a deck of 52 playing cards and then well shuffled. Now one card is drawn at random from the remaining cards.  Determine the probability that the card is
\begin{enumerate}[label=(\roman*)]
\item a club
\item 10 of hearts
\end{enumerate}
\solution
%\input{exemplar/10/13/3/29/main.tex}
\item A team of medical students doing their internship have to assist during surgeries
at a city hospital. The probabilities of surgeries rated as very complex, complex,
routine, simple or very simple are respectively, 0.15, 0.20, 0.31, 0.26, .08. Find
the probabilities that a particular surgery will be rated
\begin{enumerate}
	\item complex or very complex;
	\item neither very complex nor very simple;
	\item routine or complex
	\item routine or simple
\end{enumerate}
\solution
%\input{exemplar/11/16/3/8(1)/main.tex}
\item A card is selected from a pack of 52 cards.
\begin{enumerate}[label=(\alph*)]
    \item How many points are there in the sample space?
    \item Calculate the probability that the card is an ace of spades.
    \item Calculate the probability that the card is (i) an ace and (ii) black card.
\end{enumerate}
\solution
%\input{exemplar/11/16/3/4/main2.tex}
\item The probability that a non leap year selected at random will contain 53 sundays.
\\
\solution
%\input{exemplar/10/13/1/19/main.tex}
\item One of the four persons John, Rita, Aslam or Gurpreet will be promoted next
month. Consequently the sample space consists of four elementary outcomes
S = {John promoted, Rita promoted, Aslam promoted, Gurpreet promoted}
You are told that the chances of John’s promotion is same as that of Gurpreet,
Rita’s chances of promotion are twice as likely as Johns. Aslam’s chances are
four times that of John.
\begin{enumerate}
	\item Determine
	\begin{enumerate}
		\item P (John promoted)
		\item P (Rita promoted)
		\item P (Aslam promoted)
		\item P (Gurpreet promoted)
	\end{enumerate}
	\item If A = {John promoted or Gurpreet promoted}, find P (A).
\end{enumerate}
\solution
%\input{exemplar/11/16/3/10/main.tex}
\item A card is drawn from a deck of 52 cards. Find the probability of getting a king or a heart or a red card.\\
\solution
%\input{exemplar/11/16/3/15/main.tex}
\item The probability that a student will pass his examination is 0.73, the probability of
the student getting a compartment is 0.13, and the probability that the student will
either pass or get compartment is 0.96. State True or False.\\
\solution
%\input{exemplar/11/16/3/31/main.tex}
\item A card is selected from a pack of 52 cards\\
\begin{enumerate}[label=(\alph*)]
\item How many points are there in the sample space?
\item Calculate the probability that the cards is an ace of spades.
\item Calculate the probability that the card is (i) an ace (ii)black card.\\
\end{enumerate}
%\input{ncert/11/16/3/4_1/Prob_4.tex}
\item In a non-leap year, the probability of having 53 tuesdays or 53 wednesdays is\\
\solution
%\input{exemplar/11/16/3/18/main.tex}
\item There are 1000 sealed envelopes in a box, 10 of them contain a cash prize of
Rs 100 each, 100 of them contain a cash prize of Rs 50 each and 200 of them
contain a cash prize of Rs 10 each and rest do not contain any cash prize. If they
are well shuffled and an envelope is picked up out, what is the probability that it
contains no cash prize?\\
\solution
%\input{exemplar/10/13/3/34/main.tex}
\item 
A die is thrown and a card is selected at random from a deck of 52 playing cards. The probability of getting an even number on the die and a spade card.\\
\solution
%\input{exemplar/12/13/3/78/main.tex}
\item
If 4-digit numbers greater than 5,000 are randomly formed from the digits 0, 1, 3, 5, and 7, what is the probability of forming a number divisible by 5 when:
\begin{enumerate}
    \item The digits are repeated?
    \item The repetition of digits is not allowed?
\end{enumerate}
\solution
%\input{ncert/11/16/4/9/main.tex}
\item Consider the probability space $\brak{\Omega, \mathcal{G}, P}$ where $\Omega = [0,2]$ and $\mathcal{G} = \cbrak{\phi, \Omega, [0,1], (1,2]}$. Let $X$ and $Y$ be two functions on $\Omega$ defined as
\begin{align*}
    X(\omega) = 
    \begin{cases}
        1 & \text{if }\omega \in [0, 1]\\
        2 & \text{if }\omega \in (1, 2]
    \end{cases}
\end{align*}
and
\begin{align*}
    Y(\omega) = 
    \begin{cases}
        2 & \text{if }\omega \in [0, 1.5]\\
        3 & \text{if }\omega \in (1.5, 2].
    \end{cases}
\end{align*}
Then which one of the following statements is true?
\begin{enumerate}
    \item [(A)] $X$ is a random variable with respect to $\mathcal{G}$, but $Y$ is not a random variable with respect to $\mathcal{G}$.
    \item [(B)] $Y$ is a random variable with respect to $\mathcal{G}$, but $X$ is not a random variable with respect to $\mathcal{G}$.
    \item [(C)] Neither $X$ nor $Y$ is a random variable with respect to $\mathcal{G}$.
    \item [(D)] Both $X$ and $Y$ are random variables with respect to $\mathcal{G}$.
\end{enumerate} \hfill (GATE ST 2023)\\
\solution
%\input{gate/ST/2023/14/main.tex}
	\item  A die is loaded in such a way that each odd number is twice as likely to occur as
each even number. Find $P(G)$, where $G$ is the event that a number greater than
3 occurs on a single roll of the die.
\\
\solution
		%\input{exemplar/11/16/3/5/main.tex}
	\item All the jacks, queens and kings are removed from a deck of 52 playing cards. The remaining cards are well shuffled and then one card is drawn at random. Giving ace a value 1 similar value for other cards, find the probability that the card has a value 
		\begin{enumerate}
			\item 7
			\item greater than 7
			\item less than 7
		\end{enumerate}
		%\input{exemplar/10/13/3/30/main.tex}
  \item A Lot consists of 48 mobile phones of which 42 are good, 3 have only minor defects and 3 have major defects.Varnika will buy a phone if it is good but the trader will only buy a mobile if it has no major defects. One phone is selected at random from the lot. What is the probability that it is
\begin{enumerate}
	\item acceptable to Varnika?
            \item acceptable to the trader?
\end{enumerate}
\solution
	%\input{exemplar/10/13/3/40/main.tex}
 \item A student says that if you throw a die, it will show up 1 or not 1. Therefore, the probability of getting 1 and the probability of getting 'not 1' each is equal to $\frac{1}{2}$. Is this correct? Give reasons.\\
 \solution
        %\input{exemplar/10/13/2/9/main.tex}
   \item Four candidates A, B, C, D have ap-
plied for the assignment to coach a school cricket
team. If A is twice as likely to be selected as B, and
B and C are given about the same chance of being
selected, while C is twice as likely to be selected
as D, what are the probabilities that
\begin{enumerate}
\item C will be selected?
\item A will not be selected?
\end{enumerate}
	%\input{exemplar/11/16/3/9/main.tex}
 \item A bag contain 24 balls of which $x$ balls are red, $2x$ are white and $3x$ are blue. A ball is selected at random, What is the probability that it is
\begin{enumerate}[label=\alph*)]
\item not red ?
\item white ?
\end{enumerate}
%\input{exemplar/10/13/3/41/main.tex}
If the letters of the word ASSASSINATION are arranged at random. Find the Probability that
\begin{enumerate}[label=(\alph*)]
\item Four $S's$ come consecutively in the word
\item Two  $I's$ and two $N's$ come together
\item All $A's$ are not coming together
\item No two $A's$ are coming together
\end{enumerate}
%\input{exemplar/11/16/3/14/main.tex}
	\item One urn contains two black balls (labelled B1 and B2) and one white ball. A
	second urn contains one black ball and two white balls (labelled W1 and W2).
	Suppose the following experiment is performed. One of the two urns is chosen
	at random. Next a ball is randomly chosen from the urn. Then a second ball is
	chosen at random from the same urn without replacing the first ball.
	
	\begin{enumerate}
	\item What is the probability that two black balls are chosen?
	
	\item What is the probability that two balls of opposite colour are chosen?
	\end{enumerate}
	\solution
	%\input{exemplar/11/16/3/12/main1.tex}
\end{enumerate}

\item In a certain lottery 10,000 tickets are sold and ten equal prizes are awarded. What is the probability of not getting a prize if you buy (a) one ticket (b) two tickets (c) 10 tickets ?	
\\
\solution
		%\begin{enumerate}[label=\thesection.\arabic*,ref=\thesection.\theenumi]
	\item One card is drawn from a well-shuffled deck of 52 cards. Find the probability of getting
\begin{enumerate}
\item A king of red colour 
\item A face card 
\item A red face card
\item The jack of hearts
\item A spade
\item The queen of diamonds

\end{enumerate}
\solution
		%\input{ncert/10/15/1/14/main.tex}
	\item Five cards—the ten, jack, queen, king and ace of diamonds, are well-shuffled with their face downwards. One card is then picked up at random.
\begin{enumerate}
\item
What is the probability that the card is the queen? 
\item
If the queen is drawn and put aside, what is the probability that the second card picked up is (a) an ace? (b) a queen?\\
\end{enumerate}
\solution
		%\input{ncert/10/15/1/15/defs.tex}
	\item A bag contains $5$ red balls and some blue balls. If the probability of drawing a blue ball is double that if a red ball, determine the number of blue balls in the bag. 
		\\
\solution
		%\input{ncert/10/15/2/3/defs.tex}
	\item A card is selected from a pack of 52 cards.
 \begin{enumerate}[label=(\alph*)] 
                 \item How many points are there in the sample space?
                 \item Calculate the probability that the card is an ace of spades.
                 \item Calculate the probability that the card is (i) an ace and (ii) black card.
 \end{enumerate}
\solution
		%\input{ncert/11/16/3/4/main.tex}
\item Four cards are drawn from a well-shuffled deck of 52 cards. What is the probability of obtaining 3 diamonds and one spade.
\\
\solution
		%\input{ncert/11/16/4/2/defs.tex}
\item In a certain lottery 10,000 tickets are sold and ten equal prizes are awarded. What is the probability of not getting a prize if you buy (a) one ticket (b) two tickets (c) 10 tickets ?	
\\
\solution
		%\input{ncert/11/16/4/4/defs.tex}
		%
\item 
Out of 100 students, two sections of 40 and 60 are formed. If you and your friend are among the 100 students, what is the probability that
\begin{enumerate}
\item you both enter the same section?
\item you both enter the different sections?
\end{enumerate}
\solution
		%\input{ncert/11/16/4/5/defs.tex}
	\item 
The number lock of a suitcase has 4 wheels each labelled with ten digits i.e. from 0 to 9.The lock opens with a sequence of four digits with no repeats.What is the probability of a person getting the right sequence to open the suitcase.
\\
\solution
		%\input{ncert/11/16/4/10/defs.tex}
		%
\item 
Two cards are drawn at random and without replacement from a pack of 52 playing cards. Find the probability that both the cards are black.
\\
\solution
		%\input{ncert/12/13/2/2/defs.tex}
		\item A box of oranges is inspected by examining three randomly selected oranges drawn without replacement. If all the three oranges are good, the box is approved for sale, otherwise, it is rejected. Find the probability that a box containing 15 oranges out of which 12 are good and 3 are bad ones will be approved for sale.
		\label{ncert/12/13/2/3/defs.tex}
		\item Two balls are drawn at random with replacement from a box containing 10 black and 8 red balls. Find the probability that
		\label{ncert/12/13/2/12}
\begin{enumerate}
\item both balls are red.
\item first ball is black and second is red.
\item one of them is black and other is red.
\end{enumerate}

\item In a hostel, 60\% of the students read Hindi newspaper, 40\% read English newspaper and 20\% read both Hindi and English newspapers. A student is selected at random.
		\label{ncert/12/13/2/15}
\begin{enumerate}
\item Find the probability that she reads neither Hindi nor English newspapers.
\item If she reads Hindi newspaper, find the probability that she reads English newspaper.
\item If she reads English newspaper, find the probability that she reads Hindi newspaper.\\
\end{enumerate}
\item The probability of obtaining an even prime number on each die, when a pair of dice is rolled is 
\begin{enumerate}
    \item $0$ 
    
    \item $\frac{1}{3}$ 
    
    \item $\frac{1}{12}$ 
    
    \item $\frac{1}{36}$ 
\end{enumerate}
\solution
		%\input{ncert/12/13/2/17/defs.tex}
	\item A bag contains 4 red and 4 black balls, another bag contains 2 red and 6 black balls. One of the two bags is selected at random and a ball is drawn from the bag which is found to be red. Find the probability that the ball is drawn from the first bag.
\\
\solution
		%\input{ncert/12/13/3/2/main.tex}
  \item
  Cards with numbers 2 to 101 are placed in a box. A card is selected at random.Find the probability that the card has
\begin{enumerate}[label=(\roman*)]
	\item an even number 
	\item a square number
\end{enumerate}
\solution
%\input{exemplar/10/13/3/32/main.tex}
\item
The king, queen and jack of clubs are removed from a deck of 52 playing cards and then well shuffled. Now one card is drawn at random from the remaining cards.  Determine the probability that the card is
\begin{enumerate}[label=(\roman*)]
\item a club
\item 10 of hearts
\end{enumerate}
\solution
%\input{exemplar/10/13/3/29/main.tex}
\item A team of medical students doing their internship have to assist during surgeries
at a city hospital. The probabilities of surgeries rated as very complex, complex,
routine, simple or very simple are respectively, 0.15, 0.20, 0.31, 0.26, .08. Find
the probabilities that a particular surgery will be rated
\begin{enumerate}
	\item complex or very complex;
	\item neither very complex nor very simple;
	\item routine or complex
	\item routine or simple
\end{enumerate}
\solution
%\input{exemplar/11/16/3/8(1)/main.tex}
\item A card is selected from a pack of 52 cards.
\begin{enumerate}[label=(\alph*)]
    \item How many points are there in the sample space?
    \item Calculate the probability that the card is an ace of spades.
    \item Calculate the probability that the card is (i) an ace and (ii) black card.
\end{enumerate}
\solution
%\input{exemplar/11/16/3/4/main2.tex}
\item The probability that a non leap year selected at random will contain 53 sundays.
\\
\solution
%\input{exemplar/10/13/1/19/main.tex}
\item One of the four persons John, Rita, Aslam or Gurpreet will be promoted next
month. Consequently the sample space consists of four elementary outcomes
S = {John promoted, Rita promoted, Aslam promoted, Gurpreet promoted}
You are told that the chances of John’s promotion is same as that of Gurpreet,
Rita’s chances of promotion are twice as likely as Johns. Aslam’s chances are
four times that of John.
\begin{enumerate}
	\item Determine
	\begin{enumerate}
		\item P (John promoted)
		\item P (Rita promoted)
		\item P (Aslam promoted)
		\item P (Gurpreet promoted)
	\end{enumerate}
	\item If A = {John promoted or Gurpreet promoted}, find P (A).
\end{enumerate}
\solution
%\input{exemplar/11/16/3/10/main.tex}
\item A card is drawn from a deck of 52 cards. Find the probability of getting a king or a heart or a red card.\\
\solution
%\input{exemplar/11/16/3/15/main.tex}
\item The probability that a student will pass his examination is 0.73, the probability of
the student getting a compartment is 0.13, and the probability that the student will
either pass or get compartment is 0.96. State True or False.\\
\solution
%\input{exemplar/11/16/3/31/main.tex}
\item A card is selected from a pack of 52 cards\\
\begin{enumerate}[label=(\alph*)]
\item How many points are there in the sample space?
\item Calculate the probability that the cards is an ace of spades.
\item Calculate the probability that the card is (i) an ace (ii)black card.\\
\end{enumerate}
%\input{ncert/11/16/3/4_1/Prob_4.tex}
\item In a non-leap year, the probability of having 53 tuesdays or 53 wednesdays is\\
\solution
%\input{exemplar/11/16/3/18/main.tex}
\item There are 1000 sealed envelopes in a box, 10 of them contain a cash prize of
Rs 100 each, 100 of them contain a cash prize of Rs 50 each and 200 of them
contain a cash prize of Rs 10 each and rest do not contain any cash prize. If they
are well shuffled and an envelope is picked up out, what is the probability that it
contains no cash prize?\\
\solution
%\input{exemplar/10/13/3/34/main.tex}
\item 
A die is thrown and a card is selected at random from a deck of 52 playing cards. The probability of getting an even number on the die and a spade card.\\
\solution
%\input{exemplar/12/13/3/78/main.tex}
\item
If 4-digit numbers greater than 5,000 are randomly formed from the digits 0, 1, 3, 5, and 7, what is the probability of forming a number divisible by 5 when:
\begin{enumerate}
    \item The digits are repeated?
    \item The repetition of digits is not allowed?
\end{enumerate}
\solution
%\input{ncert/11/16/4/9/main.tex}
\item Consider the probability space $\brak{\Omega, \mathcal{G}, P}$ where $\Omega = [0,2]$ and $\mathcal{G} = \cbrak{\phi, \Omega, [0,1], (1,2]}$. Let $X$ and $Y$ be two functions on $\Omega$ defined as
\begin{align*}
    X(\omega) = 
    \begin{cases}
        1 & \text{if }\omega \in [0, 1]\\
        2 & \text{if }\omega \in (1, 2]
    \end{cases}
\end{align*}
and
\begin{align*}
    Y(\omega) = 
    \begin{cases}
        2 & \text{if }\omega \in [0, 1.5]\\
        3 & \text{if }\omega \in (1.5, 2].
    \end{cases}
\end{align*}
Then which one of the following statements is true?
\begin{enumerate}
    \item [(A)] $X$ is a random variable with respect to $\mathcal{G}$, but $Y$ is not a random variable with respect to $\mathcal{G}$.
    \item [(B)] $Y$ is a random variable with respect to $\mathcal{G}$, but $X$ is not a random variable with respect to $\mathcal{G}$.
    \item [(C)] Neither $X$ nor $Y$ is a random variable with respect to $\mathcal{G}$.
    \item [(D)] Both $X$ and $Y$ are random variables with respect to $\mathcal{G}$.
\end{enumerate} \hfill (GATE ST 2023)\\
\solution
%\input{gate/ST/2023/14/main.tex}
	\item  A die is loaded in such a way that each odd number is twice as likely to occur as
each even number. Find $P(G)$, where $G$ is the event that a number greater than
3 occurs on a single roll of the die.
\\
\solution
		%\input{exemplar/11/16/3/5/main.tex}
	\item All the jacks, queens and kings are removed from a deck of 52 playing cards. The remaining cards are well shuffled and then one card is drawn at random. Giving ace a value 1 similar value for other cards, find the probability that the card has a value 
		\begin{enumerate}
			\item 7
			\item greater than 7
			\item less than 7
		\end{enumerate}
		%\input{exemplar/10/13/3/30/main.tex}
  \item A Lot consists of 48 mobile phones of which 42 are good, 3 have only minor defects and 3 have major defects.Varnika will buy a phone if it is good but the trader will only buy a mobile if it has no major defects. One phone is selected at random from the lot. What is the probability that it is
\begin{enumerate}
	\item acceptable to Varnika?
            \item acceptable to the trader?
\end{enumerate}
\solution
	%\input{exemplar/10/13/3/40/main.tex}
 \item A student says that if you throw a die, it will show up 1 or not 1. Therefore, the probability of getting 1 and the probability of getting 'not 1' each is equal to $\frac{1}{2}$. Is this correct? Give reasons.\\
 \solution
        %\input{exemplar/10/13/2/9/main.tex}
   \item Four candidates A, B, C, D have ap-
plied for the assignment to coach a school cricket
team. If A is twice as likely to be selected as B, and
B and C are given about the same chance of being
selected, while C is twice as likely to be selected
as D, what are the probabilities that
\begin{enumerate}
\item C will be selected?
\item A will not be selected?
\end{enumerate}
	%\input{exemplar/11/16/3/9/main.tex}
 \item A bag contain 24 balls of which $x$ balls are red, $2x$ are white and $3x$ are blue. A ball is selected at random, What is the probability that it is
\begin{enumerate}[label=\alph*)]
\item not red ?
\item white ?
\end{enumerate}
%\input{exemplar/10/13/3/41/main.tex}
If the letters of the word ASSASSINATION are arranged at random. Find the Probability that
\begin{enumerate}[label=(\alph*)]
\item Four $S's$ come consecutively in the word
\item Two  $I's$ and two $N's$ come together
\item All $A's$ are not coming together
\item No two $A's$ are coming together
\end{enumerate}
%\input{exemplar/11/16/3/14/main.tex}
	\item One urn contains two black balls (labelled B1 and B2) and one white ball. A
	second urn contains one black ball and two white balls (labelled W1 and W2).
	Suppose the following experiment is performed. One of the two urns is chosen
	at random. Next a ball is randomly chosen from the urn. Then a second ball is
	chosen at random from the same urn without replacing the first ball.
	
	\begin{enumerate}
	\item What is the probability that two black balls are chosen?
	
	\item What is the probability that two balls of opposite colour are chosen?
	\end{enumerate}
	\solution
	%\input{exemplar/11/16/3/12/main1.tex}
\end{enumerate}

		%
\item 
Out of 100 students, two sections of 40 and 60 are formed. If you and your friend are among the 100 students, what is the probability that
\begin{enumerate}
\item you both enter the same section?
\item you both enter the different sections?
\end{enumerate}
\solution
		%\begin{enumerate}[label=\thesection.\arabic*,ref=\thesection.\theenumi]
	\item One card is drawn from a well-shuffled deck of 52 cards. Find the probability of getting
\begin{enumerate}
\item A king of red colour 
\item A face card 
\item A red face card
\item The jack of hearts
\item A spade
\item The queen of diamonds

\end{enumerate}
\solution
		%\input{ncert/10/15/1/14/main.tex}
	\item Five cards—the ten, jack, queen, king and ace of diamonds, are well-shuffled with their face downwards. One card is then picked up at random.
\begin{enumerate}
\item
What is the probability that the card is the queen? 
\item
If the queen is drawn and put aside, what is the probability that the second card picked up is (a) an ace? (b) a queen?\\
\end{enumerate}
\solution
		%\input{ncert/10/15/1/15/defs.tex}
	\item A bag contains $5$ red balls and some blue balls. If the probability of drawing a blue ball is double that if a red ball, determine the number of blue balls in the bag. 
		\\
\solution
		%\input{ncert/10/15/2/3/defs.tex}
	\item A card is selected from a pack of 52 cards.
 \begin{enumerate}[label=(\alph*)] 
                 \item How many points are there in the sample space?
                 \item Calculate the probability that the card is an ace of spades.
                 \item Calculate the probability that the card is (i) an ace and (ii) black card.
 \end{enumerate}
\solution
		%\input{ncert/11/16/3/4/main.tex}
\item Four cards are drawn from a well-shuffled deck of 52 cards. What is the probability of obtaining 3 diamonds and one spade.
\\
\solution
		%\input{ncert/11/16/4/2/defs.tex}
\item In a certain lottery 10,000 tickets are sold and ten equal prizes are awarded. What is the probability of not getting a prize if you buy (a) one ticket (b) two tickets (c) 10 tickets ?	
\\
\solution
		%\input{ncert/11/16/4/4/defs.tex}
		%
\item 
Out of 100 students, two sections of 40 and 60 are formed. If you and your friend are among the 100 students, what is the probability that
\begin{enumerate}
\item you both enter the same section?
\item you both enter the different sections?
\end{enumerate}
\solution
		%\input{ncert/11/16/4/5/defs.tex}
	\item 
The number lock of a suitcase has 4 wheels each labelled with ten digits i.e. from 0 to 9.The lock opens with a sequence of four digits with no repeats.What is the probability of a person getting the right sequence to open the suitcase.
\\
\solution
		%\input{ncert/11/16/4/10/defs.tex}
		%
\item 
Two cards are drawn at random and without replacement from a pack of 52 playing cards. Find the probability that both the cards are black.
\\
\solution
		%\input{ncert/12/13/2/2/defs.tex}
		\item A box of oranges is inspected by examining three randomly selected oranges drawn without replacement. If all the three oranges are good, the box is approved for sale, otherwise, it is rejected. Find the probability that a box containing 15 oranges out of which 12 are good and 3 are bad ones will be approved for sale.
		\label{ncert/12/13/2/3/defs.tex}
		\item Two balls are drawn at random with replacement from a box containing 10 black and 8 red balls. Find the probability that
		\label{ncert/12/13/2/12}
\begin{enumerate}
\item both balls are red.
\item first ball is black and second is red.
\item one of them is black and other is red.
\end{enumerate}

\item In a hostel, 60\% of the students read Hindi newspaper, 40\% read English newspaper and 20\% read both Hindi and English newspapers. A student is selected at random.
		\label{ncert/12/13/2/15}
\begin{enumerate}
\item Find the probability that she reads neither Hindi nor English newspapers.
\item If she reads Hindi newspaper, find the probability that she reads English newspaper.
\item If she reads English newspaper, find the probability that she reads Hindi newspaper.\\
\end{enumerate}
\item The probability of obtaining an even prime number on each die, when a pair of dice is rolled is 
\begin{enumerate}
    \item $0$ 
    
    \item $\frac{1}{3}$ 
    
    \item $\frac{1}{12}$ 
    
    \item $\frac{1}{36}$ 
\end{enumerate}
\solution
		%\input{ncert/12/13/2/17/defs.tex}
	\item A bag contains 4 red and 4 black balls, another bag contains 2 red and 6 black balls. One of the two bags is selected at random and a ball is drawn from the bag which is found to be red. Find the probability that the ball is drawn from the first bag.
\\
\solution
		%\input{ncert/12/13/3/2/main.tex}
  \item
  Cards with numbers 2 to 101 are placed in a box. A card is selected at random.Find the probability that the card has
\begin{enumerate}[label=(\roman*)]
	\item an even number 
	\item a square number
\end{enumerate}
\solution
%\input{exemplar/10/13/3/32/main.tex}
\item
The king, queen and jack of clubs are removed from a deck of 52 playing cards and then well shuffled. Now one card is drawn at random from the remaining cards.  Determine the probability that the card is
\begin{enumerate}[label=(\roman*)]
\item a club
\item 10 of hearts
\end{enumerate}
\solution
%\input{exemplar/10/13/3/29/main.tex}
\item A team of medical students doing their internship have to assist during surgeries
at a city hospital. The probabilities of surgeries rated as very complex, complex,
routine, simple or very simple are respectively, 0.15, 0.20, 0.31, 0.26, .08. Find
the probabilities that a particular surgery will be rated
\begin{enumerate}
	\item complex or very complex;
	\item neither very complex nor very simple;
	\item routine or complex
	\item routine or simple
\end{enumerate}
\solution
%\input{exemplar/11/16/3/8(1)/main.tex}
\item A card is selected from a pack of 52 cards.
\begin{enumerate}[label=(\alph*)]
    \item How many points are there in the sample space?
    \item Calculate the probability that the card is an ace of spades.
    \item Calculate the probability that the card is (i) an ace and (ii) black card.
\end{enumerate}
\solution
%\input{exemplar/11/16/3/4/main2.tex}
\item The probability that a non leap year selected at random will contain 53 sundays.
\\
\solution
%\input{exemplar/10/13/1/19/main.tex}
\item One of the four persons John, Rita, Aslam or Gurpreet will be promoted next
month. Consequently the sample space consists of four elementary outcomes
S = {John promoted, Rita promoted, Aslam promoted, Gurpreet promoted}
You are told that the chances of John’s promotion is same as that of Gurpreet,
Rita’s chances of promotion are twice as likely as Johns. Aslam’s chances are
four times that of John.
\begin{enumerate}
	\item Determine
	\begin{enumerate}
		\item P (John promoted)
		\item P (Rita promoted)
		\item P (Aslam promoted)
		\item P (Gurpreet promoted)
	\end{enumerate}
	\item If A = {John promoted or Gurpreet promoted}, find P (A).
\end{enumerate}
\solution
%\input{exemplar/11/16/3/10/main.tex}
\item A card is drawn from a deck of 52 cards. Find the probability of getting a king or a heart or a red card.\\
\solution
%\input{exemplar/11/16/3/15/main.tex}
\item The probability that a student will pass his examination is 0.73, the probability of
the student getting a compartment is 0.13, and the probability that the student will
either pass or get compartment is 0.96. State True or False.\\
\solution
%\input{exemplar/11/16/3/31/main.tex}
\item A card is selected from a pack of 52 cards\\
\begin{enumerate}[label=(\alph*)]
\item How many points are there in the sample space?
\item Calculate the probability that the cards is an ace of spades.
\item Calculate the probability that the card is (i) an ace (ii)black card.\\
\end{enumerate}
%\input{ncert/11/16/3/4_1/Prob_4.tex}
\item In a non-leap year, the probability of having 53 tuesdays or 53 wednesdays is\\
\solution
%\input{exemplar/11/16/3/18/main.tex}
\item There are 1000 sealed envelopes in a box, 10 of them contain a cash prize of
Rs 100 each, 100 of them contain a cash prize of Rs 50 each and 200 of them
contain a cash prize of Rs 10 each and rest do not contain any cash prize. If they
are well shuffled and an envelope is picked up out, what is the probability that it
contains no cash prize?\\
\solution
%\input{exemplar/10/13/3/34/main.tex}
\item 
A die is thrown and a card is selected at random from a deck of 52 playing cards. The probability of getting an even number on the die and a spade card.\\
\solution
%\input{exemplar/12/13/3/78/main.tex}
\item
If 4-digit numbers greater than 5,000 are randomly formed from the digits 0, 1, 3, 5, and 7, what is the probability of forming a number divisible by 5 when:
\begin{enumerate}
    \item The digits are repeated?
    \item The repetition of digits is not allowed?
\end{enumerate}
\solution
%\input{ncert/11/16/4/9/main.tex}
\item Consider the probability space $\brak{\Omega, \mathcal{G}, P}$ where $\Omega = [0,2]$ and $\mathcal{G} = \cbrak{\phi, \Omega, [0,1], (1,2]}$. Let $X$ and $Y$ be two functions on $\Omega$ defined as
\begin{align*}
    X(\omega) = 
    \begin{cases}
        1 & \text{if }\omega \in [0, 1]\\
        2 & \text{if }\omega \in (1, 2]
    \end{cases}
\end{align*}
and
\begin{align*}
    Y(\omega) = 
    \begin{cases}
        2 & \text{if }\omega \in [0, 1.5]\\
        3 & \text{if }\omega \in (1.5, 2].
    \end{cases}
\end{align*}
Then which one of the following statements is true?
\begin{enumerate}
    \item [(A)] $X$ is a random variable with respect to $\mathcal{G}$, but $Y$ is not a random variable with respect to $\mathcal{G}$.
    \item [(B)] $Y$ is a random variable with respect to $\mathcal{G}$, but $X$ is not a random variable with respect to $\mathcal{G}$.
    \item [(C)] Neither $X$ nor $Y$ is a random variable with respect to $\mathcal{G}$.
    \item [(D)] Both $X$ and $Y$ are random variables with respect to $\mathcal{G}$.
\end{enumerate} \hfill (GATE ST 2023)\\
\solution
%\input{gate/ST/2023/14/main.tex}
	\item  A die is loaded in such a way that each odd number is twice as likely to occur as
each even number. Find $P(G)$, where $G$ is the event that a number greater than
3 occurs on a single roll of the die.
\\
\solution
		%\input{exemplar/11/16/3/5/main.tex}
	\item All the jacks, queens and kings are removed from a deck of 52 playing cards. The remaining cards are well shuffled and then one card is drawn at random. Giving ace a value 1 similar value for other cards, find the probability that the card has a value 
		\begin{enumerate}
			\item 7
			\item greater than 7
			\item less than 7
		\end{enumerate}
		%\input{exemplar/10/13/3/30/main.tex}
  \item A Lot consists of 48 mobile phones of which 42 are good, 3 have only minor defects and 3 have major defects.Varnika will buy a phone if it is good but the trader will only buy a mobile if it has no major defects. One phone is selected at random from the lot. What is the probability that it is
\begin{enumerate}
	\item acceptable to Varnika?
            \item acceptable to the trader?
\end{enumerate}
\solution
	%\input{exemplar/10/13/3/40/main.tex}
 \item A student says that if you throw a die, it will show up 1 or not 1. Therefore, the probability of getting 1 and the probability of getting 'not 1' each is equal to $\frac{1}{2}$. Is this correct? Give reasons.\\
 \solution
        %\input{exemplar/10/13/2/9/main.tex}
   \item Four candidates A, B, C, D have ap-
plied for the assignment to coach a school cricket
team. If A is twice as likely to be selected as B, and
B and C are given about the same chance of being
selected, while C is twice as likely to be selected
as D, what are the probabilities that
\begin{enumerate}
\item C will be selected?
\item A will not be selected?
\end{enumerate}
	%\input{exemplar/11/16/3/9/main.tex}
 \item A bag contain 24 balls of which $x$ balls are red, $2x$ are white and $3x$ are blue. A ball is selected at random, What is the probability that it is
\begin{enumerate}[label=\alph*)]
\item not red ?
\item white ?
\end{enumerate}
%\input{exemplar/10/13/3/41/main.tex}
If the letters of the word ASSASSINATION are arranged at random. Find the Probability that
\begin{enumerate}[label=(\alph*)]
\item Four $S's$ come consecutively in the word
\item Two  $I's$ and two $N's$ come together
\item All $A's$ are not coming together
\item No two $A's$ are coming together
\end{enumerate}
%\input{exemplar/11/16/3/14/main.tex}
	\item One urn contains two black balls (labelled B1 and B2) and one white ball. A
	second urn contains one black ball and two white balls (labelled W1 and W2).
	Suppose the following experiment is performed. One of the two urns is chosen
	at random. Next a ball is randomly chosen from the urn. Then a second ball is
	chosen at random from the same urn without replacing the first ball.
	
	\begin{enumerate}
	\item What is the probability that two black balls are chosen?
	
	\item What is the probability that two balls of opposite colour are chosen?
	\end{enumerate}
	\solution
	%\input{exemplar/11/16/3/12/main1.tex}
\end{enumerate}

	\item 
The number lock of a suitcase has 4 wheels each labelled with ten digits i.e. from 0 to 9.The lock opens with a sequence of four digits with no repeats.What is the probability of a person getting the right sequence to open the suitcase.
\\
\solution
		%\begin{enumerate}[label=\thesection.\arabic*,ref=\thesection.\theenumi]
	\item One card is drawn from a well-shuffled deck of 52 cards. Find the probability of getting
\begin{enumerate}
\item A king of red colour 
\item A face card 
\item A red face card
\item The jack of hearts
\item A spade
\item The queen of diamonds

\end{enumerate}
\solution
		%\input{ncert/10/15/1/14/main.tex}
	\item Five cards—the ten, jack, queen, king and ace of diamonds, are well-shuffled with their face downwards. One card is then picked up at random.
\begin{enumerate}
\item
What is the probability that the card is the queen? 
\item
If the queen is drawn and put aside, what is the probability that the second card picked up is (a) an ace? (b) a queen?\\
\end{enumerate}
\solution
		%\input{ncert/10/15/1/15/defs.tex}
	\item A bag contains $5$ red balls and some blue balls. If the probability of drawing a blue ball is double that if a red ball, determine the number of blue balls in the bag. 
		\\
\solution
		%\input{ncert/10/15/2/3/defs.tex}
	\item A card is selected from a pack of 52 cards.
 \begin{enumerate}[label=(\alph*)] 
                 \item How many points are there in the sample space?
                 \item Calculate the probability that the card is an ace of spades.
                 \item Calculate the probability that the card is (i) an ace and (ii) black card.
 \end{enumerate}
\solution
		%\input{ncert/11/16/3/4/main.tex}
\item Four cards are drawn from a well-shuffled deck of 52 cards. What is the probability of obtaining 3 diamonds and one spade.
\\
\solution
		%\input{ncert/11/16/4/2/defs.tex}
\item In a certain lottery 10,000 tickets are sold and ten equal prizes are awarded. What is the probability of not getting a prize if you buy (a) one ticket (b) two tickets (c) 10 tickets ?	
\\
\solution
		%\input{ncert/11/16/4/4/defs.tex}
		%
\item 
Out of 100 students, two sections of 40 and 60 are formed. If you and your friend are among the 100 students, what is the probability that
\begin{enumerate}
\item you both enter the same section?
\item you both enter the different sections?
\end{enumerate}
\solution
		%\input{ncert/11/16/4/5/defs.tex}
	\item 
The number lock of a suitcase has 4 wheels each labelled with ten digits i.e. from 0 to 9.The lock opens with a sequence of four digits with no repeats.What is the probability of a person getting the right sequence to open the suitcase.
\\
\solution
		%\input{ncert/11/16/4/10/defs.tex}
		%
\item 
Two cards are drawn at random and without replacement from a pack of 52 playing cards. Find the probability that both the cards are black.
\\
\solution
		%\input{ncert/12/13/2/2/defs.tex}
		\item A box of oranges is inspected by examining three randomly selected oranges drawn without replacement. If all the three oranges are good, the box is approved for sale, otherwise, it is rejected. Find the probability that a box containing 15 oranges out of which 12 are good and 3 are bad ones will be approved for sale.
		\label{ncert/12/13/2/3/defs.tex}
		\item Two balls are drawn at random with replacement from a box containing 10 black and 8 red balls. Find the probability that
		\label{ncert/12/13/2/12}
\begin{enumerate}
\item both balls are red.
\item first ball is black and second is red.
\item one of them is black and other is red.
\end{enumerate}

\item In a hostel, 60\% of the students read Hindi newspaper, 40\% read English newspaper and 20\% read both Hindi and English newspapers. A student is selected at random.
		\label{ncert/12/13/2/15}
\begin{enumerate}
\item Find the probability that she reads neither Hindi nor English newspapers.
\item If she reads Hindi newspaper, find the probability that she reads English newspaper.
\item If she reads English newspaper, find the probability that she reads Hindi newspaper.\\
\end{enumerate}
\item The probability of obtaining an even prime number on each die, when a pair of dice is rolled is 
\begin{enumerate}
    \item $0$ 
    
    \item $\frac{1}{3}$ 
    
    \item $\frac{1}{12}$ 
    
    \item $\frac{1}{36}$ 
\end{enumerate}
\solution
		%\input{ncert/12/13/2/17/defs.tex}
	\item A bag contains 4 red and 4 black balls, another bag contains 2 red and 6 black balls. One of the two bags is selected at random and a ball is drawn from the bag which is found to be red. Find the probability that the ball is drawn from the first bag.
\\
\solution
		%\input{ncert/12/13/3/2/main.tex}
  \item
  Cards with numbers 2 to 101 are placed in a box. A card is selected at random.Find the probability that the card has
\begin{enumerate}[label=(\roman*)]
	\item an even number 
	\item a square number
\end{enumerate}
\solution
%\input{exemplar/10/13/3/32/main.tex}
\item
The king, queen and jack of clubs are removed from a deck of 52 playing cards and then well shuffled. Now one card is drawn at random from the remaining cards.  Determine the probability that the card is
\begin{enumerate}[label=(\roman*)]
\item a club
\item 10 of hearts
\end{enumerate}
\solution
%\input{exemplar/10/13/3/29/main.tex}
\item A team of medical students doing their internship have to assist during surgeries
at a city hospital. The probabilities of surgeries rated as very complex, complex,
routine, simple or very simple are respectively, 0.15, 0.20, 0.31, 0.26, .08. Find
the probabilities that a particular surgery will be rated
\begin{enumerate}
	\item complex or very complex;
	\item neither very complex nor very simple;
	\item routine or complex
	\item routine or simple
\end{enumerate}
\solution
%\input{exemplar/11/16/3/8(1)/main.tex}
\item A card is selected from a pack of 52 cards.
\begin{enumerate}[label=(\alph*)]
    \item How many points are there in the sample space?
    \item Calculate the probability that the card is an ace of spades.
    \item Calculate the probability that the card is (i) an ace and (ii) black card.
\end{enumerate}
\solution
%\input{exemplar/11/16/3/4/main2.tex}
\item The probability that a non leap year selected at random will contain 53 sundays.
\\
\solution
%\input{exemplar/10/13/1/19/main.tex}
\item One of the four persons John, Rita, Aslam or Gurpreet will be promoted next
month. Consequently the sample space consists of four elementary outcomes
S = {John promoted, Rita promoted, Aslam promoted, Gurpreet promoted}
You are told that the chances of John’s promotion is same as that of Gurpreet,
Rita’s chances of promotion are twice as likely as Johns. Aslam’s chances are
four times that of John.
\begin{enumerate}
	\item Determine
	\begin{enumerate}
		\item P (John promoted)
		\item P (Rita promoted)
		\item P (Aslam promoted)
		\item P (Gurpreet promoted)
	\end{enumerate}
	\item If A = {John promoted or Gurpreet promoted}, find P (A).
\end{enumerate}
\solution
%\input{exemplar/11/16/3/10/main.tex}
\item A card is drawn from a deck of 52 cards. Find the probability of getting a king or a heart or a red card.\\
\solution
%\input{exemplar/11/16/3/15/main.tex}
\item The probability that a student will pass his examination is 0.73, the probability of
the student getting a compartment is 0.13, and the probability that the student will
either pass or get compartment is 0.96. State True or False.\\
\solution
%\input{exemplar/11/16/3/31/main.tex}
\item A card is selected from a pack of 52 cards\\
\begin{enumerate}[label=(\alph*)]
\item How many points are there in the sample space?
\item Calculate the probability that the cards is an ace of spades.
\item Calculate the probability that the card is (i) an ace (ii)black card.\\
\end{enumerate}
%\input{ncert/11/16/3/4_1/Prob_4.tex}
\item In a non-leap year, the probability of having 53 tuesdays or 53 wednesdays is\\
\solution
%\input{exemplar/11/16/3/18/main.tex}
\item There are 1000 sealed envelopes in a box, 10 of them contain a cash prize of
Rs 100 each, 100 of them contain a cash prize of Rs 50 each and 200 of them
contain a cash prize of Rs 10 each and rest do not contain any cash prize. If they
are well shuffled and an envelope is picked up out, what is the probability that it
contains no cash prize?\\
\solution
%\input{exemplar/10/13/3/34/main.tex}
\item 
A die is thrown and a card is selected at random from a deck of 52 playing cards. The probability of getting an even number on the die and a spade card.\\
\solution
%\input{exemplar/12/13/3/78/main.tex}
\item
If 4-digit numbers greater than 5,000 are randomly formed from the digits 0, 1, 3, 5, and 7, what is the probability of forming a number divisible by 5 when:
\begin{enumerate}
    \item The digits are repeated?
    \item The repetition of digits is not allowed?
\end{enumerate}
\solution
%\input{ncert/11/16/4/9/main.tex}
\item Consider the probability space $\brak{\Omega, \mathcal{G}, P}$ where $\Omega = [0,2]$ and $\mathcal{G} = \cbrak{\phi, \Omega, [0,1], (1,2]}$. Let $X$ and $Y$ be two functions on $\Omega$ defined as
\begin{align*}
    X(\omega) = 
    \begin{cases}
        1 & \text{if }\omega \in [0, 1]\\
        2 & \text{if }\omega \in (1, 2]
    \end{cases}
\end{align*}
and
\begin{align*}
    Y(\omega) = 
    \begin{cases}
        2 & \text{if }\omega \in [0, 1.5]\\
        3 & \text{if }\omega \in (1.5, 2].
    \end{cases}
\end{align*}
Then which one of the following statements is true?
\begin{enumerate}
    \item [(A)] $X$ is a random variable with respect to $\mathcal{G}$, but $Y$ is not a random variable with respect to $\mathcal{G}$.
    \item [(B)] $Y$ is a random variable with respect to $\mathcal{G}$, but $X$ is not a random variable with respect to $\mathcal{G}$.
    \item [(C)] Neither $X$ nor $Y$ is a random variable with respect to $\mathcal{G}$.
    \item [(D)] Both $X$ and $Y$ are random variables with respect to $\mathcal{G}$.
\end{enumerate} \hfill (GATE ST 2023)\\
\solution
%\input{gate/ST/2023/14/main.tex}
	\item  A die is loaded in such a way that each odd number is twice as likely to occur as
each even number. Find $P(G)$, where $G$ is the event that a number greater than
3 occurs on a single roll of the die.
\\
\solution
		%\input{exemplar/11/16/3/5/main.tex}
	\item All the jacks, queens and kings are removed from a deck of 52 playing cards. The remaining cards are well shuffled and then one card is drawn at random. Giving ace a value 1 similar value for other cards, find the probability that the card has a value 
		\begin{enumerate}
			\item 7
			\item greater than 7
			\item less than 7
		\end{enumerate}
		%\input{exemplar/10/13/3/30/main.tex}
  \item A Lot consists of 48 mobile phones of which 42 are good, 3 have only minor defects and 3 have major defects.Varnika will buy a phone if it is good but the trader will only buy a mobile if it has no major defects. One phone is selected at random from the lot. What is the probability that it is
\begin{enumerate}
	\item acceptable to Varnika?
            \item acceptable to the trader?
\end{enumerate}
\solution
	%\input{exemplar/10/13/3/40/main.tex}
 \item A student says that if you throw a die, it will show up 1 or not 1. Therefore, the probability of getting 1 and the probability of getting 'not 1' each is equal to $\frac{1}{2}$. Is this correct? Give reasons.\\
 \solution
        %\input{exemplar/10/13/2/9/main.tex}
   \item Four candidates A, B, C, D have ap-
plied for the assignment to coach a school cricket
team. If A is twice as likely to be selected as B, and
B and C are given about the same chance of being
selected, while C is twice as likely to be selected
as D, what are the probabilities that
\begin{enumerate}
\item C will be selected?
\item A will not be selected?
\end{enumerate}
	%\input{exemplar/11/16/3/9/main.tex}
 \item A bag contain 24 balls of which $x$ balls are red, $2x$ are white and $3x$ are blue. A ball is selected at random, What is the probability that it is
\begin{enumerate}[label=\alph*)]
\item not red ?
\item white ?
\end{enumerate}
%\input{exemplar/10/13/3/41/main.tex}
If the letters of the word ASSASSINATION are arranged at random. Find the Probability that
\begin{enumerate}[label=(\alph*)]
\item Four $S's$ come consecutively in the word
\item Two  $I's$ and two $N's$ come together
\item All $A's$ are not coming together
\item No two $A's$ are coming together
\end{enumerate}
%\input{exemplar/11/16/3/14/main.tex}
	\item One urn contains two black balls (labelled B1 and B2) and one white ball. A
	second urn contains one black ball and two white balls (labelled W1 and W2).
	Suppose the following experiment is performed. One of the two urns is chosen
	at random. Next a ball is randomly chosen from the urn. Then a second ball is
	chosen at random from the same urn without replacing the first ball.
	
	\begin{enumerate}
	\item What is the probability that two black balls are chosen?
	
	\item What is the probability that two balls of opposite colour are chosen?
	\end{enumerate}
	\solution
	%\input{exemplar/11/16/3/12/main1.tex}
\end{enumerate}

		%
\item 
Two cards are drawn at random and without replacement from a pack of 52 playing cards. Find the probability that both the cards are black.
\\
\solution
		%\begin{enumerate}[label=\thesection.\arabic*,ref=\thesection.\theenumi]
	\item One card is drawn from a well-shuffled deck of 52 cards. Find the probability of getting
\begin{enumerate}
\item A king of red colour 
\item A face card 
\item A red face card
\item The jack of hearts
\item A spade
\item The queen of diamonds

\end{enumerate}
\solution
		%\input{ncert/10/15/1/14/main.tex}
	\item Five cards—the ten, jack, queen, king and ace of diamonds, are well-shuffled with their face downwards. One card is then picked up at random.
\begin{enumerate}
\item
What is the probability that the card is the queen? 
\item
If the queen is drawn and put aside, what is the probability that the second card picked up is (a) an ace? (b) a queen?\\
\end{enumerate}
\solution
		%\input{ncert/10/15/1/15/defs.tex}
	\item A bag contains $5$ red balls and some blue balls. If the probability of drawing a blue ball is double that if a red ball, determine the number of blue balls in the bag. 
		\\
\solution
		%\input{ncert/10/15/2/3/defs.tex}
	\item A card is selected from a pack of 52 cards.
 \begin{enumerate}[label=(\alph*)] 
                 \item How many points are there in the sample space?
                 \item Calculate the probability that the card is an ace of spades.
                 \item Calculate the probability that the card is (i) an ace and (ii) black card.
 \end{enumerate}
\solution
		%\input{ncert/11/16/3/4/main.tex}
\item Four cards are drawn from a well-shuffled deck of 52 cards. What is the probability of obtaining 3 diamonds and one spade.
\\
\solution
		%\input{ncert/11/16/4/2/defs.tex}
\item In a certain lottery 10,000 tickets are sold and ten equal prizes are awarded. What is the probability of not getting a prize if you buy (a) one ticket (b) two tickets (c) 10 tickets ?	
\\
\solution
		%\input{ncert/11/16/4/4/defs.tex}
		%
\item 
Out of 100 students, two sections of 40 and 60 are formed. If you and your friend are among the 100 students, what is the probability that
\begin{enumerate}
\item you both enter the same section?
\item you both enter the different sections?
\end{enumerate}
\solution
		%\input{ncert/11/16/4/5/defs.tex}
	\item 
The number lock of a suitcase has 4 wheels each labelled with ten digits i.e. from 0 to 9.The lock opens with a sequence of four digits with no repeats.What is the probability of a person getting the right sequence to open the suitcase.
\\
\solution
		%\input{ncert/11/16/4/10/defs.tex}
		%
\item 
Two cards are drawn at random and without replacement from a pack of 52 playing cards. Find the probability that both the cards are black.
\\
\solution
		%\input{ncert/12/13/2/2/defs.tex}
		\item A box of oranges is inspected by examining three randomly selected oranges drawn without replacement. If all the three oranges are good, the box is approved for sale, otherwise, it is rejected. Find the probability that a box containing 15 oranges out of which 12 are good and 3 are bad ones will be approved for sale.
		\label{ncert/12/13/2/3/defs.tex}
		\item Two balls are drawn at random with replacement from a box containing 10 black and 8 red balls. Find the probability that
		\label{ncert/12/13/2/12}
\begin{enumerate}
\item both balls are red.
\item first ball is black and second is red.
\item one of them is black and other is red.
\end{enumerate}

\item In a hostel, 60\% of the students read Hindi newspaper, 40\% read English newspaper and 20\% read both Hindi and English newspapers. A student is selected at random.
		\label{ncert/12/13/2/15}
\begin{enumerate}
\item Find the probability that she reads neither Hindi nor English newspapers.
\item If she reads Hindi newspaper, find the probability that she reads English newspaper.
\item If she reads English newspaper, find the probability that she reads Hindi newspaper.\\
\end{enumerate}
\item The probability of obtaining an even prime number on each die, when a pair of dice is rolled is 
\begin{enumerate}
    \item $0$ 
    
    \item $\frac{1}{3}$ 
    
    \item $\frac{1}{12}$ 
    
    \item $\frac{1}{36}$ 
\end{enumerate}
\solution
		%\input{ncert/12/13/2/17/defs.tex}
	\item A bag contains 4 red and 4 black balls, another bag contains 2 red and 6 black balls. One of the two bags is selected at random and a ball is drawn from the bag which is found to be red. Find the probability that the ball is drawn from the first bag.
\\
\solution
		%\input{ncert/12/13/3/2/main.tex}
  \item
  Cards with numbers 2 to 101 are placed in a box. A card is selected at random.Find the probability that the card has
\begin{enumerate}[label=(\roman*)]
	\item an even number 
	\item a square number
\end{enumerate}
\solution
%\input{exemplar/10/13/3/32/main.tex}
\item
The king, queen and jack of clubs are removed from a deck of 52 playing cards and then well shuffled. Now one card is drawn at random from the remaining cards.  Determine the probability that the card is
\begin{enumerate}[label=(\roman*)]
\item a club
\item 10 of hearts
\end{enumerate}
\solution
%\input{exemplar/10/13/3/29/main.tex}
\item A team of medical students doing their internship have to assist during surgeries
at a city hospital. The probabilities of surgeries rated as very complex, complex,
routine, simple or very simple are respectively, 0.15, 0.20, 0.31, 0.26, .08. Find
the probabilities that a particular surgery will be rated
\begin{enumerate}
	\item complex or very complex;
	\item neither very complex nor very simple;
	\item routine or complex
	\item routine or simple
\end{enumerate}
\solution
%\input{exemplar/11/16/3/8(1)/main.tex}
\item A card is selected from a pack of 52 cards.
\begin{enumerate}[label=(\alph*)]
    \item How many points are there in the sample space?
    \item Calculate the probability that the card is an ace of spades.
    \item Calculate the probability that the card is (i) an ace and (ii) black card.
\end{enumerate}
\solution
%\input{exemplar/11/16/3/4/main2.tex}
\item The probability that a non leap year selected at random will contain 53 sundays.
\\
\solution
%\input{exemplar/10/13/1/19/main.tex}
\item One of the four persons John, Rita, Aslam or Gurpreet will be promoted next
month. Consequently the sample space consists of four elementary outcomes
S = {John promoted, Rita promoted, Aslam promoted, Gurpreet promoted}
You are told that the chances of John’s promotion is same as that of Gurpreet,
Rita’s chances of promotion are twice as likely as Johns. Aslam’s chances are
four times that of John.
\begin{enumerate}
	\item Determine
	\begin{enumerate}
		\item P (John promoted)
		\item P (Rita promoted)
		\item P (Aslam promoted)
		\item P (Gurpreet promoted)
	\end{enumerate}
	\item If A = {John promoted or Gurpreet promoted}, find P (A).
\end{enumerate}
\solution
%\input{exemplar/11/16/3/10/main.tex}
\item A card is drawn from a deck of 52 cards. Find the probability of getting a king or a heart or a red card.\\
\solution
%\input{exemplar/11/16/3/15/main.tex}
\item The probability that a student will pass his examination is 0.73, the probability of
the student getting a compartment is 0.13, and the probability that the student will
either pass or get compartment is 0.96. State True or False.\\
\solution
%\input{exemplar/11/16/3/31/main.tex}
\item A card is selected from a pack of 52 cards\\
\begin{enumerate}[label=(\alph*)]
\item How many points are there in the sample space?
\item Calculate the probability that the cards is an ace of spades.
\item Calculate the probability that the card is (i) an ace (ii)black card.\\
\end{enumerate}
%\input{ncert/11/16/3/4_1/Prob_4.tex}
\item In a non-leap year, the probability of having 53 tuesdays or 53 wednesdays is\\
\solution
%\input{exemplar/11/16/3/18/main.tex}
\item There are 1000 sealed envelopes in a box, 10 of them contain a cash prize of
Rs 100 each, 100 of them contain a cash prize of Rs 50 each and 200 of them
contain a cash prize of Rs 10 each and rest do not contain any cash prize. If they
are well shuffled and an envelope is picked up out, what is the probability that it
contains no cash prize?\\
\solution
%\input{exemplar/10/13/3/34/main.tex}
\item 
A die is thrown and a card is selected at random from a deck of 52 playing cards. The probability of getting an even number on the die and a spade card.\\
\solution
%\input{exemplar/12/13/3/78/main.tex}
\item
If 4-digit numbers greater than 5,000 are randomly formed from the digits 0, 1, 3, 5, and 7, what is the probability of forming a number divisible by 5 when:
\begin{enumerate}
    \item The digits are repeated?
    \item The repetition of digits is not allowed?
\end{enumerate}
\solution
%\input{ncert/11/16/4/9/main.tex}
\item Consider the probability space $\brak{\Omega, \mathcal{G}, P}$ where $\Omega = [0,2]$ and $\mathcal{G} = \cbrak{\phi, \Omega, [0,1], (1,2]}$. Let $X$ and $Y$ be two functions on $\Omega$ defined as
\begin{align*}
    X(\omega) = 
    \begin{cases}
        1 & \text{if }\omega \in [0, 1]\\
        2 & \text{if }\omega \in (1, 2]
    \end{cases}
\end{align*}
and
\begin{align*}
    Y(\omega) = 
    \begin{cases}
        2 & \text{if }\omega \in [0, 1.5]\\
        3 & \text{if }\omega \in (1.5, 2].
    \end{cases}
\end{align*}
Then which one of the following statements is true?
\begin{enumerate}
    \item [(A)] $X$ is a random variable with respect to $\mathcal{G}$, but $Y$ is not a random variable with respect to $\mathcal{G}$.
    \item [(B)] $Y$ is a random variable with respect to $\mathcal{G}$, but $X$ is not a random variable with respect to $\mathcal{G}$.
    \item [(C)] Neither $X$ nor $Y$ is a random variable with respect to $\mathcal{G}$.
    \item [(D)] Both $X$ and $Y$ are random variables with respect to $\mathcal{G}$.
\end{enumerate} \hfill (GATE ST 2023)\\
\solution
%\input{gate/ST/2023/14/main.tex}
	\item  A die is loaded in such a way that each odd number is twice as likely to occur as
each even number. Find $P(G)$, where $G$ is the event that a number greater than
3 occurs on a single roll of the die.
\\
\solution
		%\input{exemplar/11/16/3/5/main.tex}
	\item All the jacks, queens and kings are removed from a deck of 52 playing cards. The remaining cards are well shuffled and then one card is drawn at random. Giving ace a value 1 similar value for other cards, find the probability that the card has a value 
		\begin{enumerate}
			\item 7
			\item greater than 7
			\item less than 7
		\end{enumerate}
		%\input{exemplar/10/13/3/30/main.tex}
  \item A Lot consists of 48 mobile phones of which 42 are good, 3 have only minor defects and 3 have major defects.Varnika will buy a phone if it is good but the trader will only buy a mobile if it has no major defects. One phone is selected at random from the lot. What is the probability that it is
\begin{enumerate}
	\item acceptable to Varnika?
            \item acceptable to the trader?
\end{enumerate}
\solution
	%\input{exemplar/10/13/3/40/main.tex}
 \item A student says that if you throw a die, it will show up 1 or not 1. Therefore, the probability of getting 1 and the probability of getting 'not 1' each is equal to $\frac{1}{2}$. Is this correct? Give reasons.\\
 \solution
        %\input{exemplar/10/13/2/9/main.tex}
   \item Four candidates A, B, C, D have ap-
plied for the assignment to coach a school cricket
team. If A is twice as likely to be selected as B, and
B and C are given about the same chance of being
selected, while C is twice as likely to be selected
as D, what are the probabilities that
\begin{enumerate}
\item C will be selected?
\item A will not be selected?
\end{enumerate}
	%\input{exemplar/11/16/3/9/main.tex}
 \item A bag contain 24 balls of which $x$ balls are red, $2x$ are white and $3x$ are blue. A ball is selected at random, What is the probability that it is
\begin{enumerate}[label=\alph*)]
\item not red ?
\item white ?
\end{enumerate}
%\input{exemplar/10/13/3/41/main.tex}
If the letters of the word ASSASSINATION are arranged at random. Find the Probability that
\begin{enumerate}[label=(\alph*)]
\item Four $S's$ come consecutively in the word
\item Two  $I's$ and two $N's$ come together
\item All $A's$ are not coming together
\item No two $A's$ are coming together
\end{enumerate}
%\input{exemplar/11/16/3/14/main.tex}
	\item One urn contains two black balls (labelled B1 and B2) and one white ball. A
	second urn contains one black ball and two white balls (labelled W1 and W2).
	Suppose the following experiment is performed. One of the two urns is chosen
	at random. Next a ball is randomly chosen from the urn. Then a second ball is
	chosen at random from the same urn without replacing the first ball.
	
	\begin{enumerate}
	\item What is the probability that two black balls are chosen?
	
	\item What is the probability that two balls of opposite colour are chosen?
	\end{enumerate}
	\solution
	%\input{exemplar/11/16/3/12/main1.tex}
\end{enumerate}

		\item A box of oranges is inspected by examining three randomly selected oranges drawn without replacement. If all the three oranges are good, the box is approved for sale, otherwise, it is rejected. Find the probability that a box containing 15 oranges out of which 12 are good and 3 are bad ones will be approved for sale.
		\label{ncert/12/13/2/3/defs.tex}
		\item Two balls are drawn at random with replacement from a box containing 10 black and 8 red balls. Find the probability that
		\label{ncert/12/13/2/12}
\begin{enumerate}
\item both balls are red.
\item first ball is black and second is red.
\item one of them is black and other is red.
\end{enumerate}

\item In a hostel, 60\% of the students read Hindi newspaper, 40\% read English newspaper and 20\% read both Hindi and English newspapers. A student is selected at random.
		\label{ncert/12/13/2/15}
\begin{enumerate}
\item Find the probability that she reads neither Hindi nor English newspapers.
\item If she reads Hindi newspaper, find the probability that she reads English newspaper.
\item If she reads English newspaper, find the probability that she reads Hindi newspaper.\\
\end{enumerate}
\item The probability of obtaining an even prime number on each die, when a pair of dice is rolled is 
\begin{enumerate}
    \item $0$ 
    
    \item $\frac{1}{3}$ 
    
    \item $\frac{1}{12}$ 
    
    \item $\frac{1}{36}$ 
\end{enumerate}
\solution
		%\begin{enumerate}[label=\thesection.\arabic*,ref=\thesection.\theenumi]
	\item One card is drawn from a well-shuffled deck of 52 cards. Find the probability of getting
\begin{enumerate}
\item A king of red colour 
\item A face card 
\item A red face card
\item The jack of hearts
\item A spade
\item The queen of diamonds

\end{enumerate}
\solution
		%\input{ncert/10/15/1/14/main.tex}
	\item Five cards—the ten, jack, queen, king and ace of diamonds, are well-shuffled with their face downwards. One card is then picked up at random.
\begin{enumerate}
\item
What is the probability that the card is the queen? 
\item
If the queen is drawn and put aside, what is the probability that the second card picked up is (a) an ace? (b) a queen?\\
\end{enumerate}
\solution
		%\input{ncert/10/15/1/15/defs.tex}
	\item A bag contains $5$ red balls and some blue balls. If the probability of drawing a blue ball is double that if a red ball, determine the number of blue balls in the bag. 
		\\
\solution
		%\input{ncert/10/15/2/3/defs.tex}
	\item A card is selected from a pack of 52 cards.
 \begin{enumerate}[label=(\alph*)] 
                 \item How many points are there in the sample space?
                 \item Calculate the probability that the card is an ace of spades.
                 \item Calculate the probability that the card is (i) an ace and (ii) black card.
 \end{enumerate}
\solution
		%\input{ncert/11/16/3/4/main.tex}
\item Four cards are drawn from a well-shuffled deck of 52 cards. What is the probability of obtaining 3 diamonds and one spade.
\\
\solution
		%\input{ncert/11/16/4/2/defs.tex}
\item In a certain lottery 10,000 tickets are sold and ten equal prizes are awarded. What is the probability of not getting a prize if you buy (a) one ticket (b) two tickets (c) 10 tickets ?	
\\
\solution
		%\input{ncert/11/16/4/4/defs.tex}
		%
\item 
Out of 100 students, two sections of 40 and 60 are formed. If you and your friend are among the 100 students, what is the probability that
\begin{enumerate}
\item you both enter the same section?
\item you both enter the different sections?
\end{enumerate}
\solution
		%\input{ncert/11/16/4/5/defs.tex}
	\item 
The number lock of a suitcase has 4 wheels each labelled with ten digits i.e. from 0 to 9.The lock opens with a sequence of four digits with no repeats.What is the probability of a person getting the right sequence to open the suitcase.
\\
\solution
		%\input{ncert/11/16/4/10/defs.tex}
		%
\item 
Two cards are drawn at random and without replacement from a pack of 52 playing cards. Find the probability that both the cards are black.
\\
\solution
		%\input{ncert/12/13/2/2/defs.tex}
		\item A box of oranges is inspected by examining three randomly selected oranges drawn without replacement. If all the three oranges are good, the box is approved for sale, otherwise, it is rejected. Find the probability that a box containing 15 oranges out of which 12 are good and 3 are bad ones will be approved for sale.
		\label{ncert/12/13/2/3/defs.tex}
		\item Two balls are drawn at random with replacement from a box containing 10 black and 8 red balls. Find the probability that
		\label{ncert/12/13/2/12}
\begin{enumerate}
\item both balls are red.
\item first ball is black and second is red.
\item one of them is black and other is red.
\end{enumerate}

\item In a hostel, 60\% of the students read Hindi newspaper, 40\% read English newspaper and 20\% read both Hindi and English newspapers. A student is selected at random.
		\label{ncert/12/13/2/15}
\begin{enumerate}
\item Find the probability that she reads neither Hindi nor English newspapers.
\item If she reads Hindi newspaper, find the probability that she reads English newspaper.
\item If she reads English newspaper, find the probability that she reads Hindi newspaper.\\
\end{enumerate}
\item The probability of obtaining an even prime number on each die, when a pair of dice is rolled is 
\begin{enumerate}
    \item $0$ 
    
    \item $\frac{1}{3}$ 
    
    \item $\frac{1}{12}$ 
    
    \item $\frac{1}{36}$ 
\end{enumerate}
\solution
		%\input{ncert/12/13/2/17/defs.tex}
	\item A bag contains 4 red and 4 black balls, another bag contains 2 red and 6 black balls. One of the two bags is selected at random and a ball is drawn from the bag which is found to be red. Find the probability that the ball is drawn from the first bag.
\\
\solution
		%\input{ncert/12/13/3/2/main.tex}
  \item
  Cards with numbers 2 to 101 are placed in a box. A card is selected at random.Find the probability that the card has
\begin{enumerate}[label=(\roman*)]
	\item an even number 
	\item a square number
\end{enumerate}
\solution
%\input{exemplar/10/13/3/32/main.tex}
\item
The king, queen and jack of clubs are removed from a deck of 52 playing cards and then well shuffled. Now one card is drawn at random from the remaining cards.  Determine the probability that the card is
\begin{enumerate}[label=(\roman*)]
\item a club
\item 10 of hearts
\end{enumerate}
\solution
%\input{exemplar/10/13/3/29/main.tex}
\item A team of medical students doing their internship have to assist during surgeries
at a city hospital. The probabilities of surgeries rated as very complex, complex,
routine, simple or very simple are respectively, 0.15, 0.20, 0.31, 0.26, .08. Find
the probabilities that a particular surgery will be rated
\begin{enumerate}
	\item complex or very complex;
	\item neither very complex nor very simple;
	\item routine or complex
	\item routine or simple
\end{enumerate}
\solution
%\input{exemplar/11/16/3/8(1)/main.tex}
\item A card is selected from a pack of 52 cards.
\begin{enumerate}[label=(\alph*)]
    \item How many points are there in the sample space?
    \item Calculate the probability that the card is an ace of spades.
    \item Calculate the probability that the card is (i) an ace and (ii) black card.
\end{enumerate}
\solution
%\input{exemplar/11/16/3/4/main2.tex}
\item The probability that a non leap year selected at random will contain 53 sundays.
\\
\solution
%\input{exemplar/10/13/1/19/main.tex}
\item One of the four persons John, Rita, Aslam or Gurpreet will be promoted next
month. Consequently the sample space consists of four elementary outcomes
S = {John promoted, Rita promoted, Aslam promoted, Gurpreet promoted}
You are told that the chances of John’s promotion is same as that of Gurpreet,
Rita’s chances of promotion are twice as likely as Johns. Aslam’s chances are
four times that of John.
\begin{enumerate}
	\item Determine
	\begin{enumerate}
		\item P (John promoted)
		\item P (Rita promoted)
		\item P (Aslam promoted)
		\item P (Gurpreet promoted)
	\end{enumerate}
	\item If A = {John promoted or Gurpreet promoted}, find P (A).
\end{enumerate}
\solution
%\input{exemplar/11/16/3/10/main.tex}
\item A card is drawn from a deck of 52 cards. Find the probability of getting a king or a heart or a red card.\\
\solution
%\input{exemplar/11/16/3/15/main.tex}
\item The probability that a student will pass his examination is 0.73, the probability of
the student getting a compartment is 0.13, and the probability that the student will
either pass or get compartment is 0.96. State True or False.\\
\solution
%\input{exemplar/11/16/3/31/main.tex}
\item A card is selected from a pack of 52 cards\\
\begin{enumerate}[label=(\alph*)]
\item How many points are there in the sample space?
\item Calculate the probability that the cards is an ace of spades.
\item Calculate the probability that the card is (i) an ace (ii)black card.\\
\end{enumerate}
%\input{ncert/11/16/3/4_1/Prob_4.tex}
\item In a non-leap year, the probability of having 53 tuesdays or 53 wednesdays is\\
\solution
%\input{exemplar/11/16/3/18/main.tex}
\item There are 1000 sealed envelopes in a box, 10 of them contain a cash prize of
Rs 100 each, 100 of them contain a cash prize of Rs 50 each and 200 of them
contain a cash prize of Rs 10 each and rest do not contain any cash prize. If they
are well shuffled and an envelope is picked up out, what is the probability that it
contains no cash prize?\\
\solution
%\input{exemplar/10/13/3/34/main.tex}
\item 
A die is thrown and a card is selected at random from a deck of 52 playing cards. The probability of getting an even number on the die and a spade card.\\
\solution
%\input{exemplar/12/13/3/78/main.tex}
\item
If 4-digit numbers greater than 5,000 are randomly formed from the digits 0, 1, 3, 5, and 7, what is the probability of forming a number divisible by 5 when:
\begin{enumerate}
    \item The digits are repeated?
    \item The repetition of digits is not allowed?
\end{enumerate}
\solution
%\input{ncert/11/16/4/9/main.tex}
\item Consider the probability space $\brak{\Omega, \mathcal{G}, P}$ where $\Omega = [0,2]$ and $\mathcal{G} = \cbrak{\phi, \Omega, [0,1], (1,2]}$. Let $X$ and $Y$ be two functions on $\Omega$ defined as
\begin{align*}
    X(\omega) = 
    \begin{cases}
        1 & \text{if }\omega \in [0, 1]\\
        2 & \text{if }\omega \in (1, 2]
    \end{cases}
\end{align*}
and
\begin{align*}
    Y(\omega) = 
    \begin{cases}
        2 & \text{if }\omega \in [0, 1.5]\\
        3 & \text{if }\omega \in (1.5, 2].
    \end{cases}
\end{align*}
Then which one of the following statements is true?
\begin{enumerate}
    \item [(A)] $X$ is a random variable with respect to $\mathcal{G}$, but $Y$ is not a random variable with respect to $\mathcal{G}$.
    \item [(B)] $Y$ is a random variable with respect to $\mathcal{G}$, but $X$ is not a random variable with respect to $\mathcal{G}$.
    \item [(C)] Neither $X$ nor $Y$ is a random variable with respect to $\mathcal{G}$.
    \item [(D)] Both $X$ and $Y$ are random variables with respect to $\mathcal{G}$.
\end{enumerate} \hfill (GATE ST 2023)\\
\solution
%\input{gate/ST/2023/14/main.tex}
	\item  A die is loaded in such a way that each odd number is twice as likely to occur as
each even number. Find $P(G)$, where $G$ is the event that a number greater than
3 occurs on a single roll of the die.
\\
\solution
		%\input{exemplar/11/16/3/5/main.tex}
	\item All the jacks, queens and kings are removed from a deck of 52 playing cards. The remaining cards are well shuffled and then one card is drawn at random. Giving ace a value 1 similar value for other cards, find the probability that the card has a value 
		\begin{enumerate}
			\item 7
			\item greater than 7
			\item less than 7
		\end{enumerate}
		%\input{exemplar/10/13/3/30/main.tex}
  \item A Lot consists of 48 mobile phones of which 42 are good, 3 have only minor defects and 3 have major defects.Varnika will buy a phone if it is good but the trader will only buy a mobile if it has no major defects. One phone is selected at random from the lot. What is the probability that it is
\begin{enumerate}
	\item acceptable to Varnika?
            \item acceptable to the trader?
\end{enumerate}
\solution
	%\input{exemplar/10/13/3/40/main.tex}
 \item A student says that if you throw a die, it will show up 1 or not 1. Therefore, the probability of getting 1 and the probability of getting 'not 1' each is equal to $\frac{1}{2}$. Is this correct? Give reasons.\\
 \solution
        %\input{exemplar/10/13/2/9/main.tex}
   \item Four candidates A, B, C, D have ap-
plied for the assignment to coach a school cricket
team. If A is twice as likely to be selected as B, and
B and C are given about the same chance of being
selected, while C is twice as likely to be selected
as D, what are the probabilities that
\begin{enumerate}
\item C will be selected?
\item A will not be selected?
\end{enumerate}
	%\input{exemplar/11/16/3/9/main.tex}
 \item A bag contain 24 balls of which $x$ balls are red, $2x$ are white and $3x$ are blue. A ball is selected at random, What is the probability that it is
\begin{enumerate}[label=\alph*)]
\item not red ?
\item white ?
\end{enumerate}
%\input{exemplar/10/13/3/41/main.tex}
If the letters of the word ASSASSINATION are arranged at random. Find the Probability that
\begin{enumerate}[label=(\alph*)]
\item Four $S's$ come consecutively in the word
\item Two  $I's$ and two $N's$ come together
\item All $A's$ are not coming together
\item No two $A's$ are coming together
\end{enumerate}
%\input{exemplar/11/16/3/14/main.tex}
	\item One urn contains two black balls (labelled B1 and B2) and one white ball. A
	second urn contains one black ball and two white balls (labelled W1 and W2).
	Suppose the following experiment is performed. One of the two urns is chosen
	at random. Next a ball is randomly chosen from the urn. Then a second ball is
	chosen at random from the same urn without replacing the first ball.
	
	\begin{enumerate}
	\item What is the probability that two black balls are chosen?
	
	\item What is the probability that two balls of opposite colour are chosen?
	\end{enumerate}
	\solution
	%\input{exemplar/11/16/3/12/main1.tex}
\end{enumerate}

	\item A bag contains 4 red and 4 black balls, another bag contains 2 red and 6 black balls. One of the two bags is selected at random and a ball is drawn from the bag which is found to be red. Find the probability that the ball is drawn from the first bag.
\\
\solution
		%\begin{table}[H]
	\centering
\begin{tabular}{|c|c|c|}
\hline
Random variable &Value &Definition\\ \hline
\multirow{3}{*}{X} &0 &Slips of Rs 1\\
&1 &Slips of Rs 5\\
&2 &Slips of Rs 13\\ \hline
\multirow{2}{*}{Y} &0 &Box A\\
&1 &Box B\\\hline
\end{tabular}
\caption{}
\label{tab:Distribution}
\end{table}
See \tabref{tab:Distribution}.
\begin{align}
p_{Y}\brak{k}= \begin{cases} 
      \frac{1}{3} & {k=0} \\
      \frac{2}{3 }& {k=1} 
   \end{cases}
   \\
p_{Y|X}\brak{0|0} = \frac{19}{25}\, 
p_{Y|X}\brak{0|1} = \frac{6}{25}\,
p_{Y|X}\brak{1|0} = \frac{45}{50}\,
p_{Y|X}\brak{1|2} = \frac{5}{50}
\end{align}
The desired probability is the probability that a slip drawn at random is marked other than Rs 1,
\begin{align}
&=1-p_X\brak{0}\\
&= p_X(1) + p_X(2)
\end{align}
Using Bayes theorem,
\begin{align}
&= p_Y\brak{0} \times \pr{Y=0 | X=1} + p_Y\brak{1} \times \pr{Y=1|X=2}\\
&=\frac{1}{3} \times \frac{6}{25} + \frac{2}{3} \times \frac{5}{50}\\
&=\frac{11}{75}
\end{align}

\newpage

%\tableofcontents

\bigskip

\renewcommand{\thefigure}{\theenumi}
\renewcommand{\thetable}{\theenumi}
%\renewcommand{\theequation}{\theenumi}

%\begin{abstract}
%%\boldmath
%In this letter, an algorithm for evaluating the exact analytical bit error rate  (BER)  for the piecewise linear (PL) combiner for  multiple relays is presented. Previous results were available only for upto three relays. The algorithm is unique in the sense that  the actual mathematical expressions, that are prohibitively large, need not be explicitly obtained. The diversity gain due to multiple relays is shown through plots of the analytical BER, well supported by simulations. 
%
%\end{abstract}
% IEEEtran.cls defaults to using nonbold math in the Abstract.
% This preserves the distinction between vectors and scalars. However,
% if the journal you are submitting to favors bold math in the abstract,
% then you can use LaTeX's standard command \boldmath at the very start
% of the abstract to achieve this. Many IEEE journals frown on math
% in the abstract anyway.

% Note that keywords are not normally used for peerreview papers.
%\begin{IEEEkeywords}
%Cooperative diversity, decode and forward, piecewise linear
%\end{IEEEkeywords}



% For peer review papers, you can put extra information on the cover
% page as needed:
% \ifCLASSOPTIONpeerreview
% \begin{center} \bfseries EDICS Category: 3-BBND \end{center}
% \fi
%
% For peerreview papers, this IEEEtran command inserts a page break and
% creates the second title. It will be ignored for other modes.
%\IEEEpeerreviewmaketitle




  \item
  Cards with numbers 2 to 101 are placed in a box. A card is selected at random.Find the probability that the card has
\begin{enumerate}[label=(\roman*)]
	\item an even number 
	\item a square number
\end{enumerate}
\solution
%\begin{table}[H]
	\centering
\begin{tabular}{|c|c|c|}
\hline
Random variable &Value &Definition\\ \hline
\multirow{3}{*}{X} &0 &Slips of Rs 1\\
&1 &Slips of Rs 5\\
&2 &Slips of Rs 13\\ \hline
\multirow{2}{*}{Y} &0 &Box A\\
&1 &Box B\\\hline
\end{tabular}
\caption{}
\label{tab:Distribution}
\end{table}
See \tabref{tab:Distribution}.
\begin{align}
p_{Y}\brak{k}= \begin{cases} 
      \frac{1}{3} & {k=0} \\
      \frac{2}{3 }& {k=1} 
   \end{cases}
   \\
p_{Y|X}\brak{0|0} = \frac{19}{25}\, 
p_{Y|X}\brak{0|1} = \frac{6}{25}\,
p_{Y|X}\brak{1|0} = \frac{45}{50}\,
p_{Y|X}\brak{1|2} = \frac{5}{50}
\end{align}
The desired probability is the probability that a slip drawn at random is marked other than Rs 1,
\begin{align}
&=1-p_X\brak{0}\\
&= p_X(1) + p_X(2)
\end{align}
Using Bayes theorem,
\begin{align}
&= p_Y\brak{0} \times \pr{Y=0 | X=1} + p_Y\brak{1} \times \pr{Y=1|X=2}\\
&=\frac{1}{3} \times \frac{6}{25} + \frac{2}{3} \times \frac{5}{50}\\
&=\frac{11}{75}
\end{align}

\newpage

%\tableofcontents

\bigskip

\renewcommand{\thefigure}{\theenumi}
\renewcommand{\thetable}{\theenumi}
%\renewcommand{\theequation}{\theenumi}

%\begin{abstract}
%%\boldmath
%In this letter, an algorithm for evaluating the exact analytical bit error rate  (BER)  for the piecewise linear (PL) combiner for  multiple relays is presented. Previous results were available only for upto three relays. The algorithm is unique in the sense that  the actual mathematical expressions, that are prohibitively large, need not be explicitly obtained. The diversity gain due to multiple relays is shown through plots of the analytical BER, well supported by simulations. 
%
%\end{abstract}
% IEEEtran.cls defaults to using nonbold math in the Abstract.
% This preserves the distinction between vectors and scalars. However,
% if the journal you are submitting to favors bold math in the abstract,
% then you can use LaTeX's standard command \boldmath at the very start
% of the abstract to achieve this. Many IEEE journals frown on math
% in the abstract anyway.

% Note that keywords are not normally used for peerreview papers.
%\begin{IEEEkeywords}
%Cooperative diversity, decode and forward, piecewise linear
%\end{IEEEkeywords}



% For peer review papers, you can put extra information on the cover
% page as needed:
% \ifCLASSOPTIONpeerreview
% \begin{center} \bfseries EDICS Category: 3-BBND \end{center}
% \fi
%
% For peerreview papers, this IEEEtran command inserts a page break and
% creates the second title. It will be ignored for other modes.
%\IEEEpeerreviewmaketitle




\item
The king, queen and jack of clubs are removed from a deck of 52 playing cards and then well shuffled. Now one card is drawn at random from the remaining cards.  Determine the probability that the card is
\begin{enumerate}[label=(\roman*)]
\item a club
\item 10 of hearts
\end{enumerate}
\solution
%\begin{table}[H]
	\centering
\begin{tabular}{|c|c|c|}
\hline
Random variable &Value &Definition\\ \hline
\multirow{3}{*}{X} &0 &Slips of Rs 1\\
&1 &Slips of Rs 5\\
&2 &Slips of Rs 13\\ \hline
\multirow{2}{*}{Y} &0 &Box A\\
&1 &Box B\\\hline
\end{tabular}
\caption{}
\label{tab:Distribution}
\end{table}
See \tabref{tab:Distribution}.
\begin{align}
p_{Y}\brak{k}= \begin{cases} 
      \frac{1}{3} & {k=0} \\
      \frac{2}{3 }& {k=1} 
   \end{cases}
   \\
p_{Y|X}\brak{0|0} = \frac{19}{25}\, 
p_{Y|X}\brak{0|1} = \frac{6}{25}\,
p_{Y|X}\brak{1|0} = \frac{45}{50}\,
p_{Y|X}\brak{1|2} = \frac{5}{50}
\end{align}
The desired probability is the probability that a slip drawn at random is marked other than Rs 1,
\begin{align}
&=1-p_X\brak{0}\\
&= p_X(1) + p_X(2)
\end{align}
Using Bayes theorem,
\begin{align}
&= p_Y\brak{0} \times \pr{Y=0 | X=1} + p_Y\brak{1} \times \pr{Y=1|X=2}\\
&=\frac{1}{3} \times \frac{6}{25} + \frac{2}{3} \times \frac{5}{50}\\
&=\frac{11}{75}
\end{align}

\newpage

%\tableofcontents

\bigskip

\renewcommand{\thefigure}{\theenumi}
\renewcommand{\thetable}{\theenumi}
%\renewcommand{\theequation}{\theenumi}

%\begin{abstract}
%%\boldmath
%In this letter, an algorithm for evaluating the exact analytical bit error rate  (BER)  for the piecewise linear (PL) combiner for  multiple relays is presented. Previous results were available only for upto three relays. The algorithm is unique in the sense that  the actual mathematical expressions, that are prohibitively large, need not be explicitly obtained. The diversity gain due to multiple relays is shown through plots of the analytical BER, well supported by simulations. 
%
%\end{abstract}
% IEEEtran.cls defaults to using nonbold math in the Abstract.
% This preserves the distinction between vectors and scalars. However,
% if the journal you are submitting to favors bold math in the abstract,
% then you can use LaTeX's standard command \boldmath at the very start
% of the abstract to achieve this. Many IEEE journals frown on math
% in the abstract anyway.

% Note that keywords are not normally used for peerreview papers.
%\begin{IEEEkeywords}
%Cooperative diversity, decode and forward, piecewise linear
%\end{IEEEkeywords}



% For peer review papers, you can put extra information on the cover
% page as needed:
% \ifCLASSOPTIONpeerreview
% \begin{center} \bfseries EDICS Category: 3-BBND \end{center}
% \fi
%
% For peerreview papers, this IEEEtran command inserts a page break and
% creates the second title. It will be ignored for other modes.
%\IEEEpeerreviewmaketitle




\item A team of medical students doing their internship have to assist during surgeries
at a city hospital. The probabilities of surgeries rated as very complex, complex,
routine, simple or very simple are respectively, 0.15, 0.20, 0.31, 0.26, .08. Find
the probabilities that a particular surgery will be rated
\begin{enumerate}
	\item complex or very complex;
	\item neither very complex nor very simple;
	\item routine or complex
	\item routine or simple
\end{enumerate}
\solution
%\begin{table}[H]
	\centering
\begin{tabular}{|c|c|c|}
\hline
Random variable &Value &Definition\\ \hline
\multirow{3}{*}{X} &0 &Slips of Rs 1\\
&1 &Slips of Rs 5\\
&2 &Slips of Rs 13\\ \hline
\multirow{2}{*}{Y} &0 &Box A\\
&1 &Box B\\\hline
\end{tabular}
\caption{}
\label{tab:Distribution}
\end{table}
See \tabref{tab:Distribution}.
\begin{align}
p_{Y}\brak{k}= \begin{cases} 
      \frac{1}{3} & {k=0} \\
      \frac{2}{3 }& {k=1} 
   \end{cases}
   \\
p_{Y|X}\brak{0|0} = \frac{19}{25}\, 
p_{Y|X}\brak{0|1} = \frac{6}{25}\,
p_{Y|X}\brak{1|0} = \frac{45}{50}\,
p_{Y|X}\brak{1|2} = \frac{5}{50}
\end{align}
The desired probability is the probability that a slip drawn at random is marked other than Rs 1,
\begin{align}
&=1-p_X\brak{0}\\
&= p_X(1) + p_X(2)
\end{align}
Using Bayes theorem,
\begin{align}
&= p_Y\brak{0} \times \pr{Y=0 | X=1} + p_Y\brak{1} \times \pr{Y=1|X=2}\\
&=\frac{1}{3} \times \frac{6}{25} + \frac{2}{3} \times \frac{5}{50}\\
&=\frac{11}{75}
\end{align}

\newpage

%\tableofcontents

\bigskip

\renewcommand{\thefigure}{\theenumi}
\renewcommand{\thetable}{\theenumi}
%\renewcommand{\theequation}{\theenumi}

%\begin{abstract}
%%\boldmath
%In this letter, an algorithm for evaluating the exact analytical bit error rate  (BER)  for the piecewise linear (PL) combiner for  multiple relays is presented. Previous results were available only for upto three relays. The algorithm is unique in the sense that  the actual mathematical expressions, that are prohibitively large, need not be explicitly obtained. The diversity gain due to multiple relays is shown through plots of the analytical BER, well supported by simulations. 
%
%\end{abstract}
% IEEEtran.cls defaults to using nonbold math in the Abstract.
% This preserves the distinction between vectors and scalars. However,
% if the journal you are submitting to favors bold math in the abstract,
% then you can use LaTeX's standard command \boldmath at the very start
% of the abstract to achieve this. Many IEEE journals frown on math
% in the abstract anyway.

% Note that keywords are not normally used for peerreview papers.
%\begin{IEEEkeywords}
%Cooperative diversity, decode and forward, piecewise linear
%\end{IEEEkeywords}



% For peer review papers, you can put extra information on the cover
% page as needed:
% \ifCLASSOPTIONpeerreview
% \begin{center} \bfseries EDICS Category: 3-BBND \end{center}
% \fi
%
% For peerreview papers, this IEEEtran command inserts a page break and
% creates the second title. It will be ignored for other modes.
%\IEEEpeerreviewmaketitle




\item A card is selected from a pack of 52 cards.
\begin{enumerate}[label=(\alph*)]
    \item How many points are there in the sample space?
    \item Calculate the probability that the card is an ace of spades.
    \item Calculate the probability that the card is (i) an ace and (ii) black card.
\end{enumerate}
\solution
%Let $X$ be an bernoulli rv defined as in \tabref{tab:exemplar/11/16/3/26}.  Then, 
\begin{equation}
    p =
        \frac{4}{11} 
\end{equation}
\begin{table}[H]
	\centering
	\input{exemplar/11/16/3/26/tables/Table2.tex}
	\caption{}
        \label{tab:exemplar/11/16/3/26}
\end{table}

\item The probability that a non leap year selected at random will contain 53 sundays.
\\
\solution
%\begin{table}[H]
	\centering
\begin{tabular}{|c|c|c|}
\hline
Random variable &Value &Definition\\ \hline
\multirow{3}{*}{X} &0 &Slips of Rs 1\\
&1 &Slips of Rs 5\\
&2 &Slips of Rs 13\\ \hline
\multirow{2}{*}{Y} &0 &Box A\\
&1 &Box B\\\hline
\end{tabular}
\caption{}
\label{tab:Distribution}
\end{table}
See \tabref{tab:Distribution}.
\begin{align}
p_{Y}\brak{k}= \begin{cases} 
      \frac{1}{3} & {k=0} \\
      \frac{2}{3 }& {k=1} 
   \end{cases}
   \\
p_{Y|X}\brak{0|0} = \frac{19}{25}\, 
p_{Y|X}\brak{0|1} = \frac{6}{25}\,
p_{Y|X}\brak{1|0} = \frac{45}{50}\,
p_{Y|X}\brak{1|2} = \frac{5}{50}
\end{align}
The desired probability is the probability that a slip drawn at random is marked other than Rs 1,
\begin{align}
&=1-p_X\brak{0}\\
&= p_X(1) + p_X(2)
\end{align}
Using Bayes theorem,
\begin{align}
&= p_Y\brak{0} \times \pr{Y=0 | X=1} + p_Y\brak{1} \times \pr{Y=1|X=2}\\
&=\frac{1}{3} \times \frac{6}{25} + \frac{2}{3} \times \frac{5}{50}\\
&=\frac{11}{75}
\end{align}

\newpage

%\tableofcontents

\bigskip

\renewcommand{\thefigure}{\theenumi}
\renewcommand{\thetable}{\theenumi}
%\renewcommand{\theequation}{\theenumi}

%\begin{abstract}
%%\boldmath
%In this letter, an algorithm for evaluating the exact analytical bit error rate  (BER)  for the piecewise linear (PL) combiner for  multiple relays is presented. Previous results were available only for upto three relays. The algorithm is unique in the sense that  the actual mathematical expressions, that are prohibitively large, need not be explicitly obtained. The diversity gain due to multiple relays is shown through plots of the analytical BER, well supported by simulations. 
%
%\end{abstract}
% IEEEtran.cls defaults to using nonbold math in the Abstract.
% This preserves the distinction between vectors and scalars. However,
% if the journal you are submitting to favors bold math in the abstract,
% then you can use LaTeX's standard command \boldmath at the very start
% of the abstract to achieve this. Many IEEE journals frown on math
% in the abstract anyway.

% Note that keywords are not normally used for peerreview papers.
%\begin{IEEEkeywords}
%Cooperative diversity, decode and forward, piecewise linear
%\end{IEEEkeywords}



% For peer review papers, you can put extra information on the cover
% page as needed:
% \ifCLASSOPTIONpeerreview
% \begin{center} \bfseries EDICS Category: 3-BBND \end{center}
% \fi
%
% For peerreview papers, this IEEEtran command inserts a page break and
% creates the second title. It will be ignored for other modes.
%\IEEEpeerreviewmaketitle




\item One of the four persons John, Rita, Aslam or Gurpreet will be promoted next
month. Consequently the sample space consists of four elementary outcomes
S = {John promoted, Rita promoted, Aslam promoted, Gurpreet promoted}
You are told that the chances of John’s promotion is same as that of Gurpreet,
Rita’s chances of promotion are twice as likely as Johns. Aslam’s chances are
four times that of John.
\begin{enumerate}
	\item Determine
	\begin{enumerate}
		\item P (John promoted)
		\item P (Rita promoted)
		\item P (Aslam promoted)
		\item P (Gurpreet promoted)
	\end{enumerate}
	\item If A = {John promoted or Gurpreet promoted}, find P (A).
\end{enumerate}
\solution
%\begin{table}[H]
	\centering
\begin{tabular}{|c|c|c|}
\hline
Random variable &Value &Definition\\ \hline
\multirow{3}{*}{X} &0 &Slips of Rs 1\\
&1 &Slips of Rs 5\\
&2 &Slips of Rs 13\\ \hline
\multirow{2}{*}{Y} &0 &Box A\\
&1 &Box B\\\hline
\end{tabular}
\caption{}
\label{tab:Distribution}
\end{table}
See \tabref{tab:Distribution}.
\begin{align}
p_{Y}\brak{k}= \begin{cases} 
      \frac{1}{3} & {k=0} \\
      \frac{2}{3 }& {k=1} 
   \end{cases}
   \\
p_{Y|X}\brak{0|0} = \frac{19}{25}\, 
p_{Y|X}\brak{0|1} = \frac{6}{25}\,
p_{Y|X}\brak{1|0} = \frac{45}{50}\,
p_{Y|X}\brak{1|2} = \frac{5}{50}
\end{align}
The desired probability is the probability that a slip drawn at random is marked other than Rs 1,
\begin{align}
&=1-p_X\brak{0}\\
&= p_X(1) + p_X(2)
\end{align}
Using Bayes theorem,
\begin{align}
&= p_Y\brak{0} \times \pr{Y=0 | X=1} + p_Y\brak{1} \times \pr{Y=1|X=2}\\
&=\frac{1}{3} \times \frac{6}{25} + \frac{2}{3} \times \frac{5}{50}\\
&=\frac{11}{75}
\end{align}

\newpage

%\tableofcontents

\bigskip

\renewcommand{\thefigure}{\theenumi}
\renewcommand{\thetable}{\theenumi}
%\renewcommand{\theequation}{\theenumi}

%\begin{abstract}
%%\boldmath
%In this letter, an algorithm for evaluating the exact analytical bit error rate  (BER)  for the piecewise linear (PL) combiner for  multiple relays is presented. Previous results were available only for upto three relays. The algorithm is unique in the sense that  the actual mathematical expressions, that are prohibitively large, need not be explicitly obtained. The diversity gain due to multiple relays is shown through plots of the analytical BER, well supported by simulations. 
%
%\end{abstract}
% IEEEtran.cls defaults to using nonbold math in the Abstract.
% This preserves the distinction between vectors and scalars. However,
% if the journal you are submitting to favors bold math in the abstract,
% then you can use LaTeX's standard command \boldmath at the very start
% of the abstract to achieve this. Many IEEE journals frown on math
% in the abstract anyway.

% Note that keywords are not normally used for peerreview papers.
%\begin{IEEEkeywords}
%Cooperative diversity, decode and forward, piecewise linear
%\end{IEEEkeywords}



% For peer review papers, you can put extra information on the cover
% page as needed:
% \ifCLASSOPTIONpeerreview
% \begin{center} \bfseries EDICS Category: 3-BBND \end{center}
% \fi
%
% For peerreview papers, this IEEEtran command inserts a page break and
% creates the second title. It will be ignored for other modes.
%\IEEEpeerreviewmaketitle




\item A card is drawn from a deck of 52 cards. Find the probability of getting a king or a heart or a red card.\\
\solution
%\begin{table}[H]
	\centering
\begin{tabular}{|c|c|c|}
\hline
Random variable &Value &Definition\\ \hline
\multirow{3}{*}{X} &0 &Slips of Rs 1\\
&1 &Slips of Rs 5\\
&2 &Slips of Rs 13\\ \hline
\multirow{2}{*}{Y} &0 &Box A\\
&1 &Box B\\\hline
\end{tabular}
\caption{}
\label{tab:Distribution}
\end{table}
See \tabref{tab:Distribution}.
\begin{align}
p_{Y}\brak{k}= \begin{cases} 
      \frac{1}{3} & {k=0} \\
      \frac{2}{3 }& {k=1} 
   \end{cases}
   \\
p_{Y|X}\brak{0|0} = \frac{19}{25}\, 
p_{Y|X}\brak{0|1} = \frac{6}{25}\,
p_{Y|X}\brak{1|0} = \frac{45}{50}\,
p_{Y|X}\brak{1|2} = \frac{5}{50}
\end{align}
The desired probability is the probability that a slip drawn at random is marked other than Rs 1,
\begin{align}
&=1-p_X\brak{0}\\
&= p_X(1) + p_X(2)
\end{align}
Using Bayes theorem,
\begin{align}
&= p_Y\brak{0} \times \pr{Y=0 | X=1} + p_Y\brak{1} \times \pr{Y=1|X=2}\\
&=\frac{1}{3} \times \frac{6}{25} + \frac{2}{3} \times \frac{5}{50}\\
&=\frac{11}{75}
\end{align}

\newpage

%\tableofcontents

\bigskip

\renewcommand{\thefigure}{\theenumi}
\renewcommand{\thetable}{\theenumi}
%\renewcommand{\theequation}{\theenumi}

%\begin{abstract}
%%\boldmath
%In this letter, an algorithm for evaluating the exact analytical bit error rate  (BER)  for the piecewise linear (PL) combiner for  multiple relays is presented. Previous results were available only for upto three relays. The algorithm is unique in the sense that  the actual mathematical expressions, that are prohibitively large, need not be explicitly obtained. The diversity gain due to multiple relays is shown through plots of the analytical BER, well supported by simulations. 
%
%\end{abstract}
% IEEEtran.cls defaults to using nonbold math in the Abstract.
% This preserves the distinction between vectors and scalars. However,
% if the journal you are submitting to favors bold math in the abstract,
% then you can use LaTeX's standard command \boldmath at the very start
% of the abstract to achieve this. Many IEEE journals frown on math
% in the abstract anyway.

% Note that keywords are not normally used for peerreview papers.
%\begin{IEEEkeywords}
%Cooperative diversity, decode and forward, piecewise linear
%\end{IEEEkeywords}



% For peer review papers, you can put extra information on the cover
% page as needed:
% \ifCLASSOPTIONpeerreview
% \begin{center} \bfseries EDICS Category: 3-BBND \end{center}
% \fi
%
% For peerreview papers, this IEEEtran command inserts a page break and
% creates the second title. It will be ignored for other modes.
%\IEEEpeerreviewmaketitle




\item The probability that a student will pass his examination is 0.73, the probability of
the student getting a compartment is 0.13, and the probability that the student will
either pass or get compartment is 0.96. State True or False.\\
\solution
%\begin{table}[H]
	\centering
\begin{tabular}{|c|c|c|}
\hline
Random variable &Value &Definition\\ \hline
\multirow{3}{*}{X} &0 &Slips of Rs 1\\
&1 &Slips of Rs 5\\
&2 &Slips of Rs 13\\ \hline
\multirow{2}{*}{Y} &0 &Box A\\
&1 &Box B\\\hline
\end{tabular}
\caption{}
\label{tab:Distribution}
\end{table}
See \tabref{tab:Distribution}.
\begin{align}
p_{Y}\brak{k}= \begin{cases} 
      \frac{1}{3} & {k=0} \\
      \frac{2}{3 }& {k=1} 
   \end{cases}
   \\
p_{Y|X}\brak{0|0} = \frac{19}{25}\, 
p_{Y|X}\brak{0|1} = \frac{6}{25}\,
p_{Y|X}\brak{1|0} = \frac{45}{50}\,
p_{Y|X}\brak{1|2} = \frac{5}{50}
\end{align}
The desired probability is the probability that a slip drawn at random is marked other than Rs 1,
\begin{align}
&=1-p_X\brak{0}\\
&= p_X(1) + p_X(2)
\end{align}
Using Bayes theorem,
\begin{align}
&= p_Y\brak{0} \times \pr{Y=0 | X=1} + p_Y\brak{1} \times \pr{Y=1|X=2}\\
&=\frac{1}{3} \times \frac{6}{25} + \frac{2}{3} \times \frac{5}{50}\\
&=\frac{11}{75}
\end{align}

\newpage

%\tableofcontents

\bigskip

\renewcommand{\thefigure}{\theenumi}
\renewcommand{\thetable}{\theenumi}
%\renewcommand{\theequation}{\theenumi}

%\begin{abstract}
%%\boldmath
%In this letter, an algorithm for evaluating the exact analytical bit error rate  (BER)  for the piecewise linear (PL) combiner for  multiple relays is presented. Previous results were available only for upto three relays. The algorithm is unique in the sense that  the actual mathematical expressions, that are prohibitively large, need not be explicitly obtained. The diversity gain due to multiple relays is shown through plots of the analytical BER, well supported by simulations. 
%
%\end{abstract}
% IEEEtran.cls defaults to using nonbold math in the Abstract.
% This preserves the distinction between vectors and scalars. However,
% if the journal you are submitting to favors bold math in the abstract,
% then you can use LaTeX's standard command \boldmath at the very start
% of the abstract to achieve this. Many IEEE journals frown on math
% in the abstract anyway.

% Note that keywords are not normally used for peerreview papers.
%\begin{IEEEkeywords}
%Cooperative diversity, decode and forward, piecewise linear
%\end{IEEEkeywords}



% For peer review papers, you can put extra information on the cover
% page as needed:
% \ifCLASSOPTIONpeerreview
% \begin{center} \bfseries EDICS Category: 3-BBND \end{center}
% \fi
%
% For peerreview papers, this IEEEtran command inserts a page break and
% creates the second title. It will be ignored for other modes.
%\IEEEpeerreviewmaketitle




\item A card is selected from a pack of 52 cards\\
\begin{enumerate}[label=(\alph*)]
\item How many points are there in the sample space?
\item Calculate the probability that the cards is an ace of spades.
\item Calculate the probability that the card is (i) an ace (ii)black card.\\
\end{enumerate}
%\input{ncert/11/16/3/4_1/Prob_4.tex}
\item In a non-leap year, the probability of having 53 tuesdays or 53 wednesdays is\\
\solution
%A non-leap year has a total of 365 days, and a week has 7 days.\\
So it can be expressed as 
\begin{align}
365\text{days} &=52\times 7+1 \text{day}
\end{align}
$\implies$ 52 tuesdays or wednesdays\\
Random variable X denotes the days of a week
\begin{align}
p_X\brak{k}&=\frac{1}{7}; \quad \brak{1<k<7}
\end{align}
So the probability of extra day being tuesday or wednesday is
\begin{align}
p_X\brak{3}+p_X\brak{4}&=\frac{1}{7}+\frac{1}{7}=\frac{2}{7}
\end{align}



\item There are 1000 sealed envelopes in a box, 10 of them contain a cash prize of
Rs 100 each, 100 of them contain a cash prize of Rs 50 each and 200 of them
contain a cash prize of Rs 10 each and rest do not contain any cash prize. If they
are well shuffled and an envelope is picked up out, what is the probability that it
contains no cash prize?\\
\solution
%\begin{table}[H]
	\centering
\begin{tabular}{|c|c|c|}
\hline
Random variable &Value &Definition\\ \hline
\multirow{3}{*}{X} &0 &Slips of Rs 1\\
&1 &Slips of Rs 5\\
&2 &Slips of Rs 13\\ \hline
\multirow{2}{*}{Y} &0 &Box A\\
&1 &Box B\\\hline
\end{tabular}
\caption{}
\label{tab:Distribution}
\end{table}
See \tabref{tab:Distribution}.
\begin{align}
p_{Y}\brak{k}= \begin{cases} 
      \frac{1}{3} & {k=0} \\
      \frac{2}{3 }& {k=1} 
   \end{cases}
   \\
p_{Y|X}\brak{0|0} = \frac{19}{25}\, 
p_{Y|X}\brak{0|1} = \frac{6}{25}\,
p_{Y|X}\brak{1|0} = \frac{45}{50}\,
p_{Y|X}\brak{1|2} = \frac{5}{50}
\end{align}
The desired probability is the probability that a slip drawn at random is marked other than Rs 1,
\begin{align}
&=1-p_X\brak{0}\\
&= p_X(1) + p_X(2)
\end{align}
Using Bayes theorem,
\begin{align}
&= p_Y\brak{0} \times \pr{Y=0 | X=1} + p_Y\brak{1} \times \pr{Y=1|X=2}\\
&=\frac{1}{3} \times \frac{6}{25} + \frac{2}{3} \times \frac{5}{50}\\
&=\frac{11}{75}
\end{align}

\newpage

%\tableofcontents

\bigskip

\renewcommand{\thefigure}{\theenumi}
\renewcommand{\thetable}{\theenumi}
%\renewcommand{\theequation}{\theenumi}

%\begin{abstract}
%%\boldmath
%In this letter, an algorithm for evaluating the exact analytical bit error rate  (BER)  for the piecewise linear (PL) combiner for  multiple relays is presented. Previous results were available only for upto three relays. The algorithm is unique in the sense that  the actual mathematical expressions, that are prohibitively large, need not be explicitly obtained. The diversity gain due to multiple relays is shown through plots of the analytical BER, well supported by simulations. 
%
%\end{abstract}
% IEEEtran.cls defaults to using nonbold math in the Abstract.
% This preserves the distinction between vectors and scalars. However,
% if the journal you are submitting to favors bold math in the abstract,
% then you can use LaTeX's standard command \boldmath at the very start
% of the abstract to achieve this. Many IEEE journals frown on math
% in the abstract anyway.

% Note that keywords are not normally used for peerreview papers.
%\begin{IEEEkeywords}
%Cooperative diversity, decode and forward, piecewise linear
%\end{IEEEkeywords}



% For peer review papers, you can put extra information on the cover
% page as needed:
% \ifCLASSOPTIONpeerreview
% \begin{center} \bfseries EDICS Category: 3-BBND \end{center}
% \fi
%
% For peerreview papers, this IEEEtran command inserts a page break and
% creates the second title. It will be ignored for other modes.
%\IEEEpeerreviewmaketitle




\item 
A die is thrown and a card is selected at random from a deck of 52 playing cards. The probability of getting an even number on the die and a spade card.\\
\solution
%\begin{table}[H]
	\centering
\begin{tabular}{|c|c|c|}
\hline
Random variable &Value &Definition\\ \hline
\multirow{3}{*}{X} &0 &Slips of Rs 1\\
&1 &Slips of Rs 5\\
&2 &Slips of Rs 13\\ \hline
\multirow{2}{*}{Y} &0 &Box A\\
&1 &Box B\\\hline
\end{tabular}
\caption{}
\label{tab:Distribution}
\end{table}
See \tabref{tab:Distribution}.
\begin{align}
p_{Y}\brak{k}= \begin{cases} 
      \frac{1}{3} & {k=0} \\
      \frac{2}{3 }& {k=1} 
   \end{cases}
   \\
p_{Y|X}\brak{0|0} = \frac{19}{25}\, 
p_{Y|X}\brak{0|1} = \frac{6}{25}\,
p_{Y|X}\brak{1|0} = \frac{45}{50}\,
p_{Y|X}\brak{1|2} = \frac{5}{50}
\end{align}
The desired probability is the probability that a slip drawn at random is marked other than Rs 1,
\begin{align}
&=1-p_X\brak{0}\\
&= p_X(1) + p_X(2)
\end{align}
Using Bayes theorem,
\begin{align}
&= p_Y\brak{0} \times \pr{Y=0 | X=1} + p_Y\brak{1} \times \pr{Y=1|X=2}\\
&=\frac{1}{3} \times \frac{6}{25} + \frac{2}{3} \times \frac{5}{50}\\
&=\frac{11}{75}
\end{align}

\newpage

%\tableofcontents

\bigskip

\renewcommand{\thefigure}{\theenumi}
\renewcommand{\thetable}{\theenumi}
%\renewcommand{\theequation}{\theenumi}

%\begin{abstract}
%%\boldmath
%In this letter, an algorithm for evaluating the exact analytical bit error rate  (BER)  for the piecewise linear (PL) combiner for  multiple relays is presented. Previous results were available only for upto three relays. The algorithm is unique in the sense that  the actual mathematical expressions, that are prohibitively large, need not be explicitly obtained. The diversity gain due to multiple relays is shown through plots of the analytical BER, well supported by simulations. 
%
%\end{abstract}
% IEEEtran.cls defaults to using nonbold math in the Abstract.
% This preserves the distinction between vectors and scalars. However,
% if the journal you are submitting to favors bold math in the abstract,
% then you can use LaTeX's standard command \boldmath at the very start
% of the abstract to achieve this. Many IEEE journals frown on math
% in the abstract anyway.

% Note that keywords are not normally used for peerreview papers.
%\begin{IEEEkeywords}
%Cooperative diversity, decode and forward, piecewise linear
%\end{IEEEkeywords}



% For peer review papers, you can put extra information on the cover
% page as needed:
% \ifCLASSOPTIONpeerreview
% \begin{center} \bfseries EDICS Category: 3-BBND \end{center}
% \fi
%
% For peerreview papers, this IEEEtran command inserts a page break and
% creates the second title. It will be ignored for other modes.
%\IEEEpeerreviewmaketitle




\item
If 4-digit numbers greater than 5,000 are randomly formed from the digits 0, 1, 3, 5, and 7, what is the probability of forming a number divisible by 5 when:
\begin{enumerate}
    \item The digits are repeated?
    \item The repetition of digits is not allowed?
\end{enumerate}
\solution
%\begin{table}[H]
	\centering
\begin{tabular}{|c|c|c|}
\hline
Random variable &Value &Definition\\ \hline
\multirow{3}{*}{X} &0 &Slips of Rs 1\\
&1 &Slips of Rs 5\\
&2 &Slips of Rs 13\\ \hline
\multirow{2}{*}{Y} &0 &Box A\\
&1 &Box B\\\hline
\end{tabular}
\caption{}
\label{tab:Distribution}
\end{table}
See \tabref{tab:Distribution}.
\begin{align}
p_{Y}\brak{k}= \begin{cases} 
      \frac{1}{3} & {k=0} \\
      \frac{2}{3 }& {k=1} 
   \end{cases}
   \\
p_{Y|X}\brak{0|0} = \frac{19}{25}\, 
p_{Y|X}\brak{0|1} = \frac{6}{25}\,
p_{Y|X}\brak{1|0} = \frac{45}{50}\,
p_{Y|X}\brak{1|2} = \frac{5}{50}
\end{align}
The desired probability is the probability that a slip drawn at random is marked other than Rs 1,
\begin{align}
&=1-p_X\brak{0}\\
&= p_X(1) + p_X(2)
\end{align}
Using Bayes theorem,
\begin{align}
&= p_Y\brak{0} \times \pr{Y=0 | X=1} + p_Y\brak{1} \times \pr{Y=1|X=2}\\
&=\frac{1}{3} \times \frac{6}{25} + \frac{2}{3} \times \frac{5}{50}\\
&=\frac{11}{75}
\end{align}

\newpage

%\tableofcontents

\bigskip

\renewcommand{\thefigure}{\theenumi}
\renewcommand{\thetable}{\theenumi}
%\renewcommand{\theequation}{\theenumi}

%\begin{abstract}
%%\boldmath
%In this letter, an algorithm for evaluating the exact analytical bit error rate  (BER)  for the piecewise linear (PL) combiner for  multiple relays is presented. Previous results were available only for upto three relays. The algorithm is unique in the sense that  the actual mathematical expressions, that are prohibitively large, need not be explicitly obtained. The diversity gain due to multiple relays is shown through plots of the analytical BER, well supported by simulations. 
%
%\end{abstract}
% IEEEtran.cls defaults to using nonbold math in the Abstract.
% This preserves the distinction between vectors and scalars. However,
% if the journal you are submitting to favors bold math in the abstract,
% then you can use LaTeX's standard command \boldmath at the very start
% of the abstract to achieve this. Many IEEE journals frown on math
% in the abstract anyway.

% Note that keywords are not normally used for peerreview papers.
%\begin{IEEEkeywords}
%Cooperative diversity, decode and forward, piecewise linear
%\end{IEEEkeywords}



% For peer review papers, you can put extra information on the cover
% page as needed:
% \ifCLASSOPTIONpeerreview
% \begin{center} \bfseries EDICS Category: 3-BBND \end{center}
% \fi
%
% For peerreview papers, this IEEEtran command inserts a page break and
% creates the second title. It will be ignored for other modes.
%\IEEEpeerreviewmaketitle




\item Consider the probability space $\brak{\Omega, \mathcal{G}, P}$ where $\Omega = [0,2]$ and $\mathcal{G} = \cbrak{\phi, \Omega, [0,1], (1,2]}$. Let $X$ and $Y$ be two functions on $\Omega$ defined as
\begin{align*}
    X(\omega) = 
    \begin{cases}
        1 & \text{if }\omega \in [0, 1]\\
        2 & \text{if }\omega \in (1, 2]
    \end{cases}
\end{align*}
and
\begin{align*}
    Y(\omega) = 
    \begin{cases}
        2 & \text{if }\omega \in [0, 1.5]\\
        3 & \text{if }\omega \in (1.5, 2].
    \end{cases}
\end{align*}
Then which one of the following statements is true?
\begin{enumerate}
    \item [(A)] $X$ is a random variable with respect to $\mathcal{G}$, but $Y$ is not a random variable with respect to $\mathcal{G}$.
    \item [(B)] $Y$ is a random variable with respect to $\mathcal{G}$, but $X$ is not a random variable with respect to $\mathcal{G}$.
    \item [(C)] Neither $X$ nor $Y$ is a random variable with respect to $\mathcal{G}$.
    \item [(D)] Both $X$ and $Y$ are random variables with respect to $\mathcal{G}$.
\end{enumerate} \hfill (GATE ST 2023)\\
\solution
%\begin{table}[H]
	\centering
\begin{tabular}{|c|c|c|}
\hline
Random variable &Value &Definition\\ \hline
\multirow{3}{*}{X} &0 &Slips of Rs 1\\
&1 &Slips of Rs 5\\
&2 &Slips of Rs 13\\ \hline
\multirow{2}{*}{Y} &0 &Box A\\
&1 &Box B\\\hline
\end{tabular}
\caption{}
\label{tab:Distribution}
\end{table}
See \tabref{tab:Distribution}.
\begin{align}
p_{Y}\brak{k}= \begin{cases} 
      \frac{1}{3} & {k=0} \\
      \frac{2}{3 }& {k=1} 
   \end{cases}
   \\
p_{Y|X}\brak{0|0} = \frac{19}{25}\, 
p_{Y|X}\brak{0|1} = \frac{6}{25}\,
p_{Y|X}\brak{1|0} = \frac{45}{50}\,
p_{Y|X}\brak{1|2} = \frac{5}{50}
\end{align}
The desired probability is the probability that a slip drawn at random is marked other than Rs 1,
\begin{align}
&=1-p_X\brak{0}\\
&= p_X(1) + p_X(2)
\end{align}
Using Bayes theorem,
\begin{align}
&= p_Y\brak{0} \times \pr{Y=0 | X=1} + p_Y\brak{1} \times \pr{Y=1|X=2}\\
&=\frac{1}{3} \times \frac{6}{25} + \frac{2}{3} \times \frac{5}{50}\\
&=\frac{11}{75}
\end{align}

\newpage

%\tableofcontents

\bigskip

\renewcommand{\thefigure}{\theenumi}
\renewcommand{\thetable}{\theenumi}
%\renewcommand{\theequation}{\theenumi}

%\begin{abstract}
%%\boldmath
%In this letter, an algorithm for evaluating the exact analytical bit error rate  (BER)  for the piecewise linear (PL) combiner for  multiple relays is presented. Previous results were available only for upto three relays. The algorithm is unique in the sense that  the actual mathematical expressions, that are prohibitively large, need not be explicitly obtained. The diversity gain due to multiple relays is shown through plots of the analytical BER, well supported by simulations. 
%
%\end{abstract}
% IEEEtran.cls defaults to using nonbold math in the Abstract.
% This preserves the distinction between vectors and scalars. However,
% if the journal you are submitting to favors bold math in the abstract,
% then you can use LaTeX's standard command \boldmath at the very start
% of the abstract to achieve this. Many IEEE journals frown on math
% in the abstract anyway.

% Note that keywords are not normally used for peerreview papers.
%\begin{IEEEkeywords}
%Cooperative diversity, decode and forward, piecewise linear
%\end{IEEEkeywords}



% For peer review papers, you can put extra information on the cover
% page as needed:
% \ifCLASSOPTIONpeerreview
% \begin{center} \bfseries EDICS Category: 3-BBND \end{center}
% \fi
%
% For peerreview papers, this IEEEtran command inserts a page break and
% creates the second title. It will be ignored for other modes.
%\IEEEpeerreviewmaketitle




	\item  A die is loaded in such a way that each odd number is twice as likely to occur as
each even number. Find $P(G)$, where $G$ is the event that a number greater than
3 occurs on a single roll of the die.
\\
\solution
		%\begin{table}[H]
	\centering
\begin{tabular}{|c|c|c|}
\hline
Random variable &Value &Definition\\ \hline
\multirow{3}{*}{X} &0 &Slips of Rs 1\\
&1 &Slips of Rs 5\\
&2 &Slips of Rs 13\\ \hline
\multirow{2}{*}{Y} &0 &Box A\\
&1 &Box B\\\hline
\end{tabular}
\caption{}
\label{tab:Distribution}
\end{table}
See \tabref{tab:Distribution}.
\begin{align}
p_{Y}\brak{k}= \begin{cases} 
      \frac{1}{3} & {k=0} \\
      \frac{2}{3 }& {k=1} 
   \end{cases}
   \\
p_{Y|X}\brak{0|0} = \frac{19}{25}\, 
p_{Y|X}\brak{0|1} = \frac{6}{25}\,
p_{Y|X}\brak{1|0} = \frac{45}{50}\,
p_{Y|X}\brak{1|2} = \frac{5}{50}
\end{align}
The desired probability is the probability that a slip drawn at random is marked other than Rs 1,
\begin{align}
&=1-p_X\brak{0}\\
&= p_X(1) + p_X(2)
\end{align}
Using Bayes theorem,
\begin{align}
&= p_Y\brak{0} \times \pr{Y=0 | X=1} + p_Y\brak{1} \times \pr{Y=1|X=2}\\
&=\frac{1}{3} \times \frac{6}{25} + \frac{2}{3} \times \frac{5}{50}\\
&=\frac{11}{75}
\end{align}

\newpage

%\tableofcontents

\bigskip

\renewcommand{\thefigure}{\theenumi}
\renewcommand{\thetable}{\theenumi}
%\renewcommand{\theequation}{\theenumi}

%\begin{abstract}
%%\boldmath
%In this letter, an algorithm for evaluating the exact analytical bit error rate  (BER)  for the piecewise linear (PL) combiner for  multiple relays is presented. Previous results were available only for upto three relays. The algorithm is unique in the sense that  the actual mathematical expressions, that are prohibitively large, need not be explicitly obtained. The diversity gain due to multiple relays is shown through plots of the analytical BER, well supported by simulations. 
%
%\end{abstract}
% IEEEtran.cls defaults to using nonbold math in the Abstract.
% This preserves the distinction between vectors and scalars. However,
% if the journal you are submitting to favors bold math in the abstract,
% then you can use LaTeX's standard command \boldmath at the very start
% of the abstract to achieve this. Many IEEE journals frown on math
% in the abstract anyway.

% Note that keywords are not normally used for peerreview papers.
%\begin{IEEEkeywords}
%Cooperative diversity, decode and forward, piecewise linear
%\end{IEEEkeywords}



% For peer review papers, you can put extra information on the cover
% page as needed:
% \ifCLASSOPTIONpeerreview
% \begin{center} \bfseries EDICS Category: 3-BBND \end{center}
% \fi
%
% For peerreview papers, this IEEEtran command inserts a page break and
% creates the second title. It will be ignored for other modes.
%\IEEEpeerreviewmaketitle




	\item All the jacks, queens and kings are removed from a deck of 52 playing cards. The remaining cards are well shuffled and then one card is drawn at random. Giving ace a value 1 similar value for other cards, find the probability that the card has a value 
		\begin{enumerate}
			\item 7
			\item greater than 7
			\item less than 7
		\end{enumerate}
		%Number of cards left after removing all jacks, queens and kings 
\begin{align}
N	= 52 - 4\times 3
	= 40
\end{align}
%\begin{table}[H]
%\def\arraystretch{1.2}
%\begin{tabular}{|c|c|c|}
%\hline
%	\textbf{Parameter} &\textbf{Value} &\textbf{Description}\\ \hline
%	$X$ &1-10 &Represents the value of the card picked \\ \hline
%\end{tabular}
%\end{table}
Let $1 \le X \le 10$ be the value of the card picked.  Then,
\begin{align}
	p_X(k) &= \Pr(X=k)\ \forall\ 1 \leq k \leq 10\\
	&= \frac{4\times 1}{40}\\
	&= \frac{1}{10}\\
	\therefore p_X(k) &= 
	\begin{cases}
		\frac{1}{10} & 1 \leq k \leq 10\\
		0 & \text{otherwise}
	\end{cases}
\end{align}
and
\begin{align}
	F_{X}(k) &= \sum_{m=0}^{k}p_{X}(m) \quad 1 \leq k \leq 10\\
	&= \frac{k}{10}\\
	\therefore F_{X}(k) &= 
	\begin{cases}
		0 & k \leq 0\\
		\frac{k}{10} & 1\leq k \leq 10\\
		1 & k > 10 
	\end{cases}
\end{align}
\begin{enumerate}
	\item Probability that card has value equal to 7 is
		\begin{align}
			 p_{X}(7)
			= \frac{1}{10}
		\end{align}
	\item Probability that card has value greater than 7 is
		\begin{align}
			1 - F_X(7)
			&= 1 - \frac{7}{10}
			\\
			&= \frac{3}{10}
		\end{align}
	\item Probability that card has value less than 7 is
		\begin{align}
			 F_{X}(6)
			=\frac{6}{10}
		\end{align}
\end{enumerate}

  \item A Lot consists of 48 mobile phones of which 42 are good, 3 have only minor defects and 3 have major defects.Varnika will buy a phone if it is good but the trader will only buy a mobile if it has no major defects. One phone is selected at random from the lot. What is the probability that it is
\begin{enumerate}
	\item acceptable to Varnika?
            \item acceptable to the trader?
\end{enumerate}
\solution
	%\begin{table}[H]
	\centering
\begin{tabular}{|c|c|c|}
\hline
Random variable &Value &Definition\\ \hline
\multirow{3}{*}{X} &0 &Slips of Rs 1\\
&1 &Slips of Rs 5\\
&2 &Slips of Rs 13\\ \hline
\multirow{2}{*}{Y} &0 &Box A\\
&1 &Box B\\\hline
\end{tabular}
\caption{}
\label{tab:Distribution}
\end{table}
See \tabref{tab:Distribution}.
\begin{align}
p_{Y}\brak{k}= \begin{cases} 
      \frac{1}{3} & {k=0} \\
      \frac{2}{3 }& {k=1} 
   \end{cases}
   \\
p_{Y|X}\brak{0|0} = \frac{19}{25}\, 
p_{Y|X}\brak{0|1} = \frac{6}{25}\,
p_{Y|X}\brak{1|0} = \frac{45}{50}\,
p_{Y|X}\brak{1|2} = \frac{5}{50}
\end{align}
The desired probability is the probability that a slip drawn at random is marked other than Rs 1,
\begin{align}
&=1-p_X\brak{0}\\
&= p_X(1) + p_X(2)
\end{align}
Using Bayes theorem,
\begin{align}
&= p_Y\brak{0} \times \pr{Y=0 | X=1} + p_Y\brak{1} \times \pr{Y=1|X=2}\\
&=\frac{1}{3} \times \frac{6}{25} + \frac{2}{3} \times \frac{5}{50}\\
&=\frac{11}{75}
\end{align}

\newpage

%\tableofcontents

\bigskip

\renewcommand{\thefigure}{\theenumi}
\renewcommand{\thetable}{\theenumi}
%\renewcommand{\theequation}{\theenumi}

%\begin{abstract}
%%\boldmath
%In this letter, an algorithm for evaluating the exact analytical bit error rate  (BER)  for the piecewise linear (PL) combiner for  multiple relays is presented. Previous results were available only for upto three relays. The algorithm is unique in the sense that  the actual mathematical expressions, that are prohibitively large, need not be explicitly obtained. The diversity gain due to multiple relays is shown through plots of the analytical BER, well supported by simulations. 
%
%\end{abstract}
% IEEEtran.cls defaults to using nonbold math in the Abstract.
% This preserves the distinction between vectors and scalars. However,
% if the journal you are submitting to favors bold math in the abstract,
% then you can use LaTeX's standard command \boldmath at the very start
% of the abstract to achieve this. Many IEEE journals frown on math
% in the abstract anyway.

% Note that keywords are not normally used for peerreview papers.
%\begin{IEEEkeywords}
%Cooperative diversity, decode and forward, piecewise linear
%\end{IEEEkeywords}



% For peer review papers, you can put extra information on the cover
% page as needed:
% \ifCLASSOPTIONpeerreview
% \begin{center} \bfseries EDICS Category: 3-BBND \end{center}
% \fi
%
% For peerreview papers, this IEEEtran command inserts a page break and
% creates the second title. It will be ignored for other modes.
%\IEEEpeerreviewmaketitle




 \item A student says that if you throw a die, it will show up 1 or not 1. Therefore, the probability of getting 1 and the probability of getting 'not 1' each is equal to $\frac{1}{2}$. Is this correct? Give reasons.\\
 \solution
        %\begin{table}[H]
	\centering
\begin{tabular}{|c|c|c|}
\hline
Random variable &Value &Definition\\ \hline
\multirow{3}{*}{X} &0 &Slips of Rs 1\\
&1 &Slips of Rs 5\\
&2 &Slips of Rs 13\\ \hline
\multirow{2}{*}{Y} &0 &Box A\\
&1 &Box B\\\hline
\end{tabular}
\caption{}
\label{tab:Distribution}
\end{table}
See \tabref{tab:Distribution}.
\begin{align}
p_{Y}\brak{k}= \begin{cases} 
      \frac{1}{3} & {k=0} \\
      \frac{2}{3 }& {k=1} 
   \end{cases}
   \\
p_{Y|X}\brak{0|0} = \frac{19}{25}\, 
p_{Y|X}\brak{0|1} = \frac{6}{25}\,
p_{Y|X}\brak{1|0} = \frac{45}{50}\,
p_{Y|X}\brak{1|2} = \frac{5}{50}
\end{align}
The desired probability is the probability that a slip drawn at random is marked other than Rs 1,
\begin{align}
&=1-p_X\brak{0}\\
&= p_X(1) + p_X(2)
\end{align}
Using Bayes theorem,
\begin{align}
&= p_Y\brak{0} \times \pr{Y=0 | X=1} + p_Y\brak{1} \times \pr{Y=1|X=2}\\
&=\frac{1}{3} \times \frac{6}{25} + \frac{2}{3} \times \frac{5}{50}\\
&=\frac{11}{75}
\end{align}

\newpage

%\tableofcontents

\bigskip

\renewcommand{\thefigure}{\theenumi}
\renewcommand{\thetable}{\theenumi}
%\renewcommand{\theequation}{\theenumi}

%\begin{abstract}
%%\boldmath
%In this letter, an algorithm for evaluating the exact analytical bit error rate  (BER)  for the piecewise linear (PL) combiner for  multiple relays is presented. Previous results were available only for upto three relays. The algorithm is unique in the sense that  the actual mathematical expressions, that are prohibitively large, need not be explicitly obtained. The diversity gain due to multiple relays is shown through plots of the analytical BER, well supported by simulations. 
%
%\end{abstract}
% IEEEtran.cls defaults to using nonbold math in the Abstract.
% This preserves the distinction between vectors and scalars. However,
% if the journal you are submitting to favors bold math in the abstract,
% then you can use LaTeX's standard command \boldmath at the very start
% of the abstract to achieve this. Many IEEE journals frown on math
% in the abstract anyway.

% Note that keywords are not normally used for peerreview papers.
%\begin{IEEEkeywords}
%Cooperative diversity, decode and forward, piecewise linear
%\end{IEEEkeywords}



% For peer review papers, you can put extra information on the cover
% page as needed:
% \ifCLASSOPTIONpeerreview
% \begin{center} \bfseries EDICS Category: 3-BBND \end{center}
% \fi
%
% For peerreview papers, this IEEEtran command inserts a page break and
% creates the second title. It will be ignored for other modes.
%\IEEEpeerreviewmaketitle




   \item Four candidates A, B, C, D have ap-
plied for the assignment to coach a school cricket
team. If A is twice as likely to be selected as B, and
B and C are given about the same chance of being
selected, while C is twice as likely to be selected
as D, what are the probabilities that
\begin{enumerate}
\item C will be selected?
\item A will not be selected?
\end{enumerate}
	%\begin{table}[H]
	\centering
\begin{tabular}{|c|c|c|}
\hline
Random variable &Value &Definition\\ \hline
\multirow{3}{*}{X} &0 &Slips of Rs 1\\
&1 &Slips of Rs 5\\
&2 &Slips of Rs 13\\ \hline
\multirow{2}{*}{Y} &0 &Box A\\
&1 &Box B\\\hline
\end{tabular}
\caption{}
\label{tab:Distribution}
\end{table}
See \tabref{tab:Distribution}.
\begin{align}
p_{Y}\brak{k}= \begin{cases} 
      \frac{1}{3} & {k=0} \\
      \frac{2}{3 }& {k=1} 
   \end{cases}
   \\
p_{Y|X}\brak{0|0} = \frac{19}{25}\, 
p_{Y|X}\brak{0|1} = \frac{6}{25}\,
p_{Y|X}\brak{1|0} = \frac{45}{50}\,
p_{Y|X}\brak{1|2} = \frac{5}{50}
\end{align}
The desired probability is the probability that a slip drawn at random is marked other than Rs 1,
\begin{align}
&=1-p_X\brak{0}\\
&= p_X(1) + p_X(2)
\end{align}
Using Bayes theorem,
\begin{align}
&= p_Y\brak{0} \times \pr{Y=0 | X=1} + p_Y\brak{1} \times \pr{Y=1|X=2}\\
&=\frac{1}{3} \times \frac{6}{25} + \frac{2}{3} \times \frac{5}{50}\\
&=\frac{11}{75}
\end{align}

\newpage

%\tableofcontents

\bigskip

\renewcommand{\thefigure}{\theenumi}
\renewcommand{\thetable}{\theenumi}
%\renewcommand{\theequation}{\theenumi}

%\begin{abstract}
%%\boldmath
%In this letter, an algorithm for evaluating the exact analytical bit error rate  (BER)  for the piecewise linear (PL) combiner for  multiple relays is presented. Previous results were available only for upto three relays. The algorithm is unique in the sense that  the actual mathematical expressions, that are prohibitively large, need not be explicitly obtained. The diversity gain due to multiple relays is shown through plots of the analytical BER, well supported by simulations. 
%
%\end{abstract}
% IEEEtran.cls defaults to using nonbold math in the Abstract.
% This preserves the distinction between vectors and scalars. However,
% if the journal you are submitting to favors bold math in the abstract,
% then you can use LaTeX's standard command \boldmath at the very start
% of the abstract to achieve this. Many IEEE journals frown on math
% in the abstract anyway.

% Note that keywords are not normally used for peerreview papers.
%\begin{IEEEkeywords}
%Cooperative diversity, decode and forward, piecewise linear
%\end{IEEEkeywords}



% For peer review papers, you can put extra information on the cover
% page as needed:
% \ifCLASSOPTIONpeerreview
% \begin{center} \bfseries EDICS Category: 3-BBND \end{center}
% \fi
%
% For peerreview papers, this IEEEtran command inserts a page break and
% creates the second title. It will be ignored for other modes.
%\IEEEpeerreviewmaketitle




 \item A bag contain 24 balls of which $x$ balls are red, $2x$ are white and $3x$ are blue. A ball is selected at random, What is the probability that it is
\begin{enumerate}[label=\alph*)]
\item not red ?
\item white ?
\end{enumerate}
%\begin{table}[H]
	\centering
\begin{tabular}{|c|c|c|}
\hline
Random variable &Value &Definition\\ \hline
\multirow{3}{*}{X} &0 &Slips of Rs 1\\
&1 &Slips of Rs 5\\
&2 &Slips of Rs 13\\ \hline
\multirow{2}{*}{Y} &0 &Box A\\
&1 &Box B\\\hline
\end{tabular}
\caption{}
\label{tab:Distribution}
\end{table}
See \tabref{tab:Distribution}.
\begin{align}
p_{Y}\brak{k}= \begin{cases} 
      \frac{1}{3} & {k=0} \\
      \frac{2}{3 }& {k=1} 
   \end{cases}
   \\
p_{Y|X}\brak{0|0} = \frac{19}{25}\, 
p_{Y|X}\brak{0|1} = \frac{6}{25}\,
p_{Y|X}\brak{1|0} = \frac{45}{50}\,
p_{Y|X}\brak{1|2} = \frac{5}{50}
\end{align}
The desired probability is the probability that a slip drawn at random is marked other than Rs 1,
\begin{align}
&=1-p_X\brak{0}\\
&= p_X(1) + p_X(2)
\end{align}
Using Bayes theorem,
\begin{align}
&= p_Y\brak{0} \times \pr{Y=0 | X=1} + p_Y\brak{1} \times \pr{Y=1|X=2}\\
&=\frac{1}{3} \times \frac{6}{25} + \frac{2}{3} \times \frac{5}{50}\\
&=\frac{11}{75}
\end{align}

\newpage

%\tableofcontents

\bigskip

\renewcommand{\thefigure}{\theenumi}
\renewcommand{\thetable}{\theenumi}
%\renewcommand{\theequation}{\theenumi}

%\begin{abstract}
%%\boldmath
%In this letter, an algorithm for evaluating the exact analytical bit error rate  (BER)  for the piecewise linear (PL) combiner for  multiple relays is presented. Previous results were available only for upto three relays. The algorithm is unique in the sense that  the actual mathematical expressions, that are prohibitively large, need not be explicitly obtained. The diversity gain due to multiple relays is shown through plots of the analytical BER, well supported by simulations. 
%
%\end{abstract}
% IEEEtran.cls defaults to using nonbold math in the Abstract.
% This preserves the distinction between vectors and scalars. However,
% if the journal you are submitting to favors bold math in the abstract,
% then you can use LaTeX's standard command \boldmath at the very start
% of the abstract to achieve this. Many IEEE journals frown on math
% in the abstract anyway.

% Note that keywords are not normally used for peerreview papers.
%\begin{IEEEkeywords}
%Cooperative diversity, decode and forward, piecewise linear
%\end{IEEEkeywords}



% For peer review papers, you can put extra information on the cover
% page as needed:
% \ifCLASSOPTIONpeerreview
% \begin{center} \bfseries EDICS Category: 3-BBND \end{center}
% \fi
%
% For peerreview papers, this IEEEtran command inserts a page break and
% creates the second title. It will be ignored for other modes.
%\IEEEpeerreviewmaketitle




If the letters of the word ASSASSINATION are arranged at random. Find the Probability that
\begin{enumerate}[label=(\alph*)]
\item Four $S's$ come consecutively in the word
\item Two  $I's$ and two $N's$ come together
\item All $A's$ are not coming together
\item No two $A's$ are coming together
\end{enumerate}
%\begin{table}[H]
	\centering
\begin{tabular}{|c|c|c|}
\hline
Random variable &Value &Definition\\ \hline
\multirow{3}{*}{X} &0 &Slips of Rs 1\\
&1 &Slips of Rs 5\\
&2 &Slips of Rs 13\\ \hline
\multirow{2}{*}{Y} &0 &Box A\\
&1 &Box B\\\hline
\end{tabular}
\caption{}
\label{tab:Distribution}
\end{table}
See \tabref{tab:Distribution}.
\begin{align}
p_{Y}\brak{k}= \begin{cases} 
      \frac{1}{3} & {k=0} \\
      \frac{2}{3 }& {k=1} 
   \end{cases}
   \\
p_{Y|X}\brak{0|0} = \frac{19}{25}\, 
p_{Y|X}\brak{0|1} = \frac{6}{25}\,
p_{Y|X}\brak{1|0} = \frac{45}{50}\,
p_{Y|X}\brak{1|2} = \frac{5}{50}
\end{align}
The desired probability is the probability that a slip drawn at random is marked other than Rs 1,
\begin{align}
&=1-p_X\brak{0}\\
&= p_X(1) + p_X(2)
\end{align}
Using Bayes theorem,
\begin{align}
&= p_Y\brak{0} \times \pr{Y=0 | X=1} + p_Y\brak{1} \times \pr{Y=1|X=2}\\
&=\frac{1}{3} \times \frac{6}{25} + \frac{2}{3} \times \frac{5}{50}\\
&=\frac{11}{75}
\end{align}

\newpage

%\tableofcontents

\bigskip

\renewcommand{\thefigure}{\theenumi}
\renewcommand{\thetable}{\theenumi}
%\renewcommand{\theequation}{\theenumi}

%\begin{abstract}
%%\boldmath
%In this letter, an algorithm for evaluating the exact analytical bit error rate  (BER)  for the piecewise linear (PL) combiner for  multiple relays is presented. Previous results were available only for upto three relays. The algorithm is unique in the sense that  the actual mathematical expressions, that are prohibitively large, need not be explicitly obtained. The diversity gain due to multiple relays is shown through plots of the analytical BER, well supported by simulations. 
%
%\end{abstract}
% IEEEtran.cls defaults to using nonbold math in the Abstract.
% This preserves the distinction between vectors and scalars. However,
% if the journal you are submitting to favors bold math in the abstract,
% then you can use LaTeX's standard command \boldmath at the very start
% of the abstract to achieve this. Many IEEE journals frown on math
% in the abstract anyway.

% Note that keywords are not normally used for peerreview papers.
%\begin{IEEEkeywords}
%Cooperative diversity, decode and forward, piecewise linear
%\end{IEEEkeywords}



% For peer review papers, you can put extra information on the cover
% page as needed:
% \ifCLASSOPTIONpeerreview
% \begin{center} \bfseries EDICS Category: 3-BBND \end{center}
% \fi
%
% For peerreview papers, this IEEEtran command inserts a page break and
% creates the second title. It will be ignored for other modes.
%\IEEEpeerreviewmaketitle




	\item One urn contains two black balls (labelled B1 and B2) and one white ball. A
	second urn contains one black ball and two white balls (labelled W1 and W2).
	Suppose the following experiment is performed. One of the two urns is chosen
	at random. Next a ball is randomly chosen from the urn. Then a second ball is
	chosen at random from the same urn without replacing the first ball.
	
	\begin{enumerate}
	\item What is the probability that two black balls are chosen?
	
	\item What is the probability that two balls of opposite colour are chosen?
	\end{enumerate}
	\solution
	%\begin{align}
    \label{eq:12.13.6.18.1}
	\because	\pr{A|B} &> \pr{A},\
\frac{\pr{AB}}{\pr{B}} > \pr{A}
\\
    \label{eq:12.13.6.18.2}
	\implies \pr{AB} &> \pr{A}\pr{B}
	\\
	\text{or, } \frac{\pr{AB}}{\pr{A}} &=\pr{B|A} > \pr{A}
\end{align}

\end{enumerate}

		%
\item 
Two cards are drawn at random and without replacement from a pack of 52 playing cards. Find the probability that both the cards are black.
\\
\solution
		%\begin{enumerate}[label=\thesection.\arabic*,ref=\thesection.\theenumi]
	\item One card is drawn from a well-shuffled deck of 52 cards. Find the probability of getting
\begin{enumerate}
\item A king of red colour 
\item A face card 
\item A red face card
\item The jack of hearts
\item A spade
\item The queen of diamonds

\end{enumerate}
\solution
		%\begin{table}[H]
	\centering
\begin{tabular}{|c|c|c|}
\hline
Random variable &Value &Definition\\ \hline
\multirow{3}{*}{X} &0 &Slips of Rs 1\\
&1 &Slips of Rs 5\\
&2 &Slips of Rs 13\\ \hline
\multirow{2}{*}{Y} &0 &Box A\\
&1 &Box B\\\hline
\end{tabular}
\caption{}
\label{tab:Distribution}
\end{table}
See \tabref{tab:Distribution}.
\begin{align}
p_{Y}\brak{k}= \begin{cases} 
      \frac{1}{3} & {k=0} \\
      \frac{2}{3 }& {k=1} 
   \end{cases}
   \\
p_{Y|X}\brak{0|0} = \frac{19}{25}\, 
p_{Y|X}\brak{0|1} = \frac{6}{25}\,
p_{Y|X}\brak{1|0} = \frac{45}{50}\,
p_{Y|X}\brak{1|2} = \frac{5}{50}
\end{align}
The desired probability is the probability that a slip drawn at random is marked other than Rs 1,
\begin{align}
&=1-p_X\brak{0}\\
&= p_X(1) + p_X(2)
\end{align}
Using Bayes theorem,
\begin{align}
&= p_Y\brak{0} \times \pr{Y=0 | X=1} + p_Y\brak{1} \times \pr{Y=1|X=2}\\
&=\frac{1}{3} \times \frac{6}{25} + \frac{2}{3} \times \frac{5}{50}\\
&=\frac{11}{75}
\end{align}

\newpage

%\tableofcontents

\bigskip

\renewcommand{\thefigure}{\theenumi}
\renewcommand{\thetable}{\theenumi}
%\renewcommand{\theequation}{\theenumi}

%\begin{abstract}
%%\boldmath
%In this letter, an algorithm for evaluating the exact analytical bit error rate  (BER)  for the piecewise linear (PL) combiner for  multiple relays is presented. Previous results were available only for upto three relays. The algorithm is unique in the sense that  the actual mathematical expressions, that are prohibitively large, need not be explicitly obtained. The diversity gain due to multiple relays is shown through plots of the analytical BER, well supported by simulations. 
%
%\end{abstract}
% IEEEtran.cls defaults to using nonbold math in the Abstract.
% This preserves the distinction between vectors and scalars. However,
% if the journal you are submitting to favors bold math in the abstract,
% then you can use LaTeX's standard command \boldmath at the very start
% of the abstract to achieve this. Many IEEE journals frown on math
% in the abstract anyway.

% Note that keywords are not normally used for peerreview papers.
%\begin{IEEEkeywords}
%Cooperative diversity, decode and forward, piecewise linear
%\end{IEEEkeywords}



% For peer review papers, you can put extra information on the cover
% page as needed:
% \ifCLASSOPTIONpeerreview
% \begin{center} \bfseries EDICS Category: 3-BBND \end{center}
% \fi
%
% For peerreview papers, this IEEEtran command inserts a page break and
% creates the second title. It will be ignored for other modes.
%\IEEEpeerreviewmaketitle




	\item Five cards—the ten, jack, queen, king and ace of diamonds, are well-shuffled with their face downwards. One card is then picked up at random.
\begin{enumerate}
\item
What is the probability that the card is the queen? 
\item
If the queen is drawn and put aside, what is the probability that the second card picked up is (a) an ace? (b) a queen?\\
\end{enumerate}
\solution
		%\begin{enumerate}[label=\thesection.\arabic*,ref=\thesection.\theenumi]
	\item One card is drawn from a well-shuffled deck of 52 cards. Find the probability of getting
\begin{enumerate}
\item A king of red colour 
\item A face card 
\item A red face card
\item The jack of hearts
\item A spade
\item The queen of diamonds

\end{enumerate}
\solution
		%\input{ncert/10/15/1/14/main.tex}
	\item Five cards—the ten, jack, queen, king and ace of diamonds, are well-shuffled with their face downwards. One card is then picked up at random.
\begin{enumerate}
\item
What is the probability that the card is the queen? 
\item
If the queen is drawn and put aside, what is the probability that the second card picked up is (a) an ace? (b) a queen?\\
\end{enumerate}
\solution
		%\input{ncert/10/15/1/15/defs.tex}
	\item A bag contains $5$ red balls and some blue balls. If the probability of drawing a blue ball is double that if a red ball, determine the number of blue balls in the bag. 
		\\
\solution
		%\input{ncert/10/15/2/3/defs.tex}
	\item A card is selected from a pack of 52 cards.
 \begin{enumerate}[label=(\alph*)] 
                 \item How many points are there in the sample space?
                 \item Calculate the probability that the card is an ace of spades.
                 \item Calculate the probability that the card is (i) an ace and (ii) black card.
 \end{enumerate}
\solution
		%\input{ncert/11/16/3/4/main.tex}
\item Four cards are drawn from a well-shuffled deck of 52 cards. What is the probability of obtaining 3 diamonds and one spade.
\\
\solution
		%\input{ncert/11/16/4/2/defs.tex}
\item In a certain lottery 10,000 tickets are sold and ten equal prizes are awarded. What is the probability of not getting a prize if you buy (a) one ticket (b) two tickets (c) 10 tickets ?	
\\
\solution
		%\input{ncert/11/16/4/4/defs.tex}
		%
\item 
Out of 100 students, two sections of 40 and 60 are formed. If you and your friend are among the 100 students, what is the probability that
\begin{enumerate}
\item you both enter the same section?
\item you both enter the different sections?
\end{enumerate}
\solution
		%\input{ncert/11/16/4/5/defs.tex}
	\item 
The number lock of a suitcase has 4 wheels each labelled with ten digits i.e. from 0 to 9.The lock opens with a sequence of four digits with no repeats.What is the probability of a person getting the right sequence to open the suitcase.
\\
\solution
		%\input{ncert/11/16/4/10/defs.tex}
		%
\item 
Two cards are drawn at random and without replacement from a pack of 52 playing cards. Find the probability that both the cards are black.
\\
\solution
		%\input{ncert/12/13/2/2/defs.tex}
		\item A box of oranges is inspected by examining three randomly selected oranges drawn without replacement. If all the three oranges are good, the box is approved for sale, otherwise, it is rejected. Find the probability that a box containing 15 oranges out of which 12 are good and 3 are bad ones will be approved for sale.
		\label{ncert/12/13/2/3/defs.tex}
		\item Two balls are drawn at random with replacement from a box containing 10 black and 8 red balls. Find the probability that
		\label{ncert/12/13/2/12}
\begin{enumerate}
\item both balls are red.
\item first ball is black and second is red.
\item one of them is black and other is red.
\end{enumerate}

\item In a hostel, 60\% of the students read Hindi newspaper, 40\% read English newspaper and 20\% read both Hindi and English newspapers. A student is selected at random.
		\label{ncert/12/13/2/15}
\begin{enumerate}
\item Find the probability that she reads neither Hindi nor English newspapers.
\item If she reads Hindi newspaper, find the probability that she reads English newspaper.
\item If she reads English newspaper, find the probability that she reads Hindi newspaper.\\
\end{enumerate}
\item The probability of obtaining an even prime number on each die, when a pair of dice is rolled is 
\begin{enumerate}
    \item $0$ 
    
    \item $\frac{1}{3}$ 
    
    \item $\frac{1}{12}$ 
    
    \item $\frac{1}{36}$ 
\end{enumerate}
\solution
		%\input{ncert/12/13/2/17/defs.tex}
	\item A bag contains 4 red and 4 black balls, another bag contains 2 red and 6 black balls. One of the two bags is selected at random and a ball is drawn from the bag which is found to be red. Find the probability that the ball is drawn from the first bag.
\\
\solution
		%\input{ncert/12/13/3/2/main.tex}
  \item
  Cards with numbers 2 to 101 are placed in a box. A card is selected at random.Find the probability that the card has
\begin{enumerate}[label=(\roman*)]
	\item an even number 
	\item a square number
\end{enumerate}
\solution
%\input{exemplar/10/13/3/32/main.tex}
\item
The king, queen and jack of clubs are removed from a deck of 52 playing cards and then well shuffled. Now one card is drawn at random from the remaining cards.  Determine the probability that the card is
\begin{enumerate}[label=(\roman*)]
\item a club
\item 10 of hearts
\end{enumerate}
\solution
%\input{exemplar/10/13/3/29/main.tex}
\item A team of medical students doing their internship have to assist during surgeries
at a city hospital. The probabilities of surgeries rated as very complex, complex,
routine, simple or very simple are respectively, 0.15, 0.20, 0.31, 0.26, .08. Find
the probabilities that a particular surgery will be rated
\begin{enumerate}
	\item complex or very complex;
	\item neither very complex nor very simple;
	\item routine or complex
	\item routine or simple
\end{enumerate}
\solution
%\input{exemplar/11/16/3/8(1)/main.tex}
\item A card is selected from a pack of 52 cards.
\begin{enumerate}[label=(\alph*)]
    \item How many points are there in the sample space?
    \item Calculate the probability that the card is an ace of spades.
    \item Calculate the probability that the card is (i) an ace and (ii) black card.
\end{enumerate}
\solution
%\input{exemplar/11/16/3/4/main2.tex}
\item The probability that a non leap year selected at random will contain 53 sundays.
\\
\solution
%\input{exemplar/10/13/1/19/main.tex}
\item One of the four persons John, Rita, Aslam or Gurpreet will be promoted next
month. Consequently the sample space consists of four elementary outcomes
S = {John promoted, Rita promoted, Aslam promoted, Gurpreet promoted}
You are told that the chances of John’s promotion is same as that of Gurpreet,
Rita’s chances of promotion are twice as likely as Johns. Aslam’s chances are
four times that of John.
\begin{enumerate}
	\item Determine
	\begin{enumerate}
		\item P (John promoted)
		\item P (Rita promoted)
		\item P (Aslam promoted)
		\item P (Gurpreet promoted)
	\end{enumerate}
	\item If A = {John promoted or Gurpreet promoted}, find P (A).
\end{enumerate}
\solution
%\input{exemplar/11/16/3/10/main.tex}
\item A card is drawn from a deck of 52 cards. Find the probability of getting a king or a heart or a red card.\\
\solution
%\input{exemplar/11/16/3/15/main.tex}
\item The probability that a student will pass his examination is 0.73, the probability of
the student getting a compartment is 0.13, and the probability that the student will
either pass or get compartment is 0.96. State True or False.\\
\solution
%\input{exemplar/11/16/3/31/main.tex}
\item A card is selected from a pack of 52 cards\\
\begin{enumerate}[label=(\alph*)]
\item How many points are there in the sample space?
\item Calculate the probability that the cards is an ace of spades.
\item Calculate the probability that the card is (i) an ace (ii)black card.\\
\end{enumerate}
%\input{ncert/11/16/3/4_1/Prob_4.tex}
\item In a non-leap year, the probability of having 53 tuesdays or 53 wednesdays is\\
\solution
%\input{exemplar/11/16/3/18/main.tex}
\item There are 1000 sealed envelopes in a box, 10 of them contain a cash prize of
Rs 100 each, 100 of them contain a cash prize of Rs 50 each and 200 of them
contain a cash prize of Rs 10 each and rest do not contain any cash prize. If they
are well shuffled and an envelope is picked up out, what is the probability that it
contains no cash prize?\\
\solution
%\input{exemplar/10/13/3/34/main.tex}
\item 
A die is thrown and a card is selected at random from a deck of 52 playing cards. The probability of getting an even number on the die and a spade card.\\
\solution
%\input{exemplar/12/13/3/78/main.tex}
\item
If 4-digit numbers greater than 5,000 are randomly formed from the digits 0, 1, 3, 5, and 7, what is the probability of forming a number divisible by 5 when:
\begin{enumerate}
    \item The digits are repeated?
    \item The repetition of digits is not allowed?
\end{enumerate}
\solution
%\input{ncert/11/16/4/9/main.tex}
\item Consider the probability space $\brak{\Omega, \mathcal{G}, P}$ where $\Omega = [0,2]$ and $\mathcal{G} = \cbrak{\phi, \Omega, [0,1], (1,2]}$. Let $X$ and $Y$ be two functions on $\Omega$ defined as
\begin{align*}
    X(\omega) = 
    \begin{cases}
        1 & \text{if }\omega \in [0, 1]\\
        2 & \text{if }\omega \in (1, 2]
    \end{cases}
\end{align*}
and
\begin{align*}
    Y(\omega) = 
    \begin{cases}
        2 & \text{if }\omega \in [0, 1.5]\\
        3 & \text{if }\omega \in (1.5, 2].
    \end{cases}
\end{align*}
Then which one of the following statements is true?
\begin{enumerate}
    \item [(A)] $X$ is a random variable with respect to $\mathcal{G}$, but $Y$ is not a random variable with respect to $\mathcal{G}$.
    \item [(B)] $Y$ is a random variable with respect to $\mathcal{G}$, but $X$ is not a random variable with respect to $\mathcal{G}$.
    \item [(C)] Neither $X$ nor $Y$ is a random variable with respect to $\mathcal{G}$.
    \item [(D)] Both $X$ and $Y$ are random variables with respect to $\mathcal{G}$.
\end{enumerate} \hfill (GATE ST 2023)\\
\solution
%\input{gate/ST/2023/14/main.tex}
	\item  A die is loaded in such a way that each odd number is twice as likely to occur as
each even number. Find $P(G)$, where $G$ is the event that a number greater than
3 occurs on a single roll of the die.
\\
\solution
		%\input{exemplar/11/16/3/5/main.tex}
	\item All the jacks, queens and kings are removed from a deck of 52 playing cards. The remaining cards are well shuffled and then one card is drawn at random. Giving ace a value 1 similar value for other cards, find the probability that the card has a value 
		\begin{enumerate}
			\item 7
			\item greater than 7
			\item less than 7
		\end{enumerate}
		%\input{exemplar/10/13/3/30/main.tex}
  \item A Lot consists of 48 mobile phones of which 42 are good, 3 have only minor defects and 3 have major defects.Varnika will buy a phone if it is good but the trader will only buy a mobile if it has no major defects. One phone is selected at random from the lot. What is the probability that it is
\begin{enumerate}
	\item acceptable to Varnika?
            \item acceptable to the trader?
\end{enumerate}
\solution
	%\input{exemplar/10/13/3/40/main.tex}
 \item A student says that if you throw a die, it will show up 1 or not 1. Therefore, the probability of getting 1 and the probability of getting 'not 1' each is equal to $\frac{1}{2}$. Is this correct? Give reasons.\\
 \solution
        %\input{exemplar/10/13/2/9/main.tex}
   \item Four candidates A, B, C, D have ap-
plied for the assignment to coach a school cricket
team. If A is twice as likely to be selected as B, and
B and C are given about the same chance of being
selected, while C is twice as likely to be selected
as D, what are the probabilities that
\begin{enumerate}
\item C will be selected?
\item A will not be selected?
\end{enumerate}
	%\input{exemplar/11/16/3/9/main.tex}
 \item A bag contain 24 balls of which $x$ balls are red, $2x$ are white and $3x$ are blue. A ball is selected at random, What is the probability that it is
\begin{enumerate}[label=\alph*)]
\item not red ?
\item white ?
\end{enumerate}
%\input{exemplar/10/13/3/41/main.tex}
If the letters of the word ASSASSINATION are arranged at random. Find the Probability that
\begin{enumerate}[label=(\alph*)]
\item Four $S's$ come consecutively in the word
\item Two  $I's$ and two $N's$ come together
\item All $A's$ are not coming together
\item No two $A's$ are coming together
\end{enumerate}
%\input{exemplar/11/16/3/14/main.tex}
	\item One urn contains two black balls (labelled B1 and B2) and one white ball. A
	second urn contains one black ball and two white balls (labelled W1 and W2).
	Suppose the following experiment is performed. One of the two urns is chosen
	at random. Next a ball is randomly chosen from the urn. Then a second ball is
	chosen at random from the same urn without replacing the first ball.
	
	\begin{enumerate}
	\item What is the probability that two black balls are chosen?
	
	\item What is the probability that two balls of opposite colour are chosen?
	\end{enumerate}
	\solution
	%\input{exemplar/11/16/3/12/main1.tex}
\end{enumerate}

	\item A bag contains $5$ red balls and some blue balls. If the probability of drawing a blue ball is double that if a red ball, determine the number of blue balls in the bag. 
		\\
\solution
		%\begin{enumerate}[label=\thesection.\arabic*,ref=\thesection.\theenumi]
	\item One card is drawn from a well-shuffled deck of 52 cards. Find the probability of getting
\begin{enumerate}
\item A king of red colour 
\item A face card 
\item A red face card
\item The jack of hearts
\item A spade
\item The queen of diamonds

\end{enumerate}
\solution
		%\input{ncert/10/15/1/14/main.tex}
	\item Five cards—the ten, jack, queen, king and ace of diamonds, are well-shuffled with their face downwards. One card is then picked up at random.
\begin{enumerate}
\item
What is the probability that the card is the queen? 
\item
If the queen is drawn and put aside, what is the probability that the second card picked up is (a) an ace? (b) a queen?\\
\end{enumerate}
\solution
		%\input{ncert/10/15/1/15/defs.tex}
	\item A bag contains $5$ red balls and some blue balls. If the probability of drawing a blue ball is double that if a red ball, determine the number of blue balls in the bag. 
		\\
\solution
		%\input{ncert/10/15/2/3/defs.tex}
	\item A card is selected from a pack of 52 cards.
 \begin{enumerate}[label=(\alph*)] 
                 \item How many points are there in the sample space?
                 \item Calculate the probability that the card is an ace of spades.
                 \item Calculate the probability that the card is (i) an ace and (ii) black card.
 \end{enumerate}
\solution
		%\input{ncert/11/16/3/4/main.tex}
\item Four cards are drawn from a well-shuffled deck of 52 cards. What is the probability of obtaining 3 diamonds and one spade.
\\
\solution
		%\input{ncert/11/16/4/2/defs.tex}
\item In a certain lottery 10,000 tickets are sold and ten equal prizes are awarded. What is the probability of not getting a prize if you buy (a) one ticket (b) two tickets (c) 10 tickets ?	
\\
\solution
		%\input{ncert/11/16/4/4/defs.tex}
		%
\item 
Out of 100 students, two sections of 40 and 60 are formed. If you and your friend are among the 100 students, what is the probability that
\begin{enumerate}
\item you both enter the same section?
\item you both enter the different sections?
\end{enumerate}
\solution
		%\input{ncert/11/16/4/5/defs.tex}
	\item 
The number lock of a suitcase has 4 wheels each labelled with ten digits i.e. from 0 to 9.The lock opens with a sequence of four digits with no repeats.What is the probability of a person getting the right sequence to open the suitcase.
\\
\solution
		%\input{ncert/11/16/4/10/defs.tex}
		%
\item 
Two cards are drawn at random and without replacement from a pack of 52 playing cards. Find the probability that both the cards are black.
\\
\solution
		%\input{ncert/12/13/2/2/defs.tex}
		\item A box of oranges is inspected by examining three randomly selected oranges drawn without replacement. If all the three oranges are good, the box is approved for sale, otherwise, it is rejected. Find the probability that a box containing 15 oranges out of which 12 are good and 3 are bad ones will be approved for sale.
		\label{ncert/12/13/2/3/defs.tex}
		\item Two balls are drawn at random with replacement from a box containing 10 black and 8 red balls. Find the probability that
		\label{ncert/12/13/2/12}
\begin{enumerate}
\item both balls are red.
\item first ball is black and second is red.
\item one of them is black and other is red.
\end{enumerate}

\item In a hostel, 60\% of the students read Hindi newspaper, 40\% read English newspaper and 20\% read both Hindi and English newspapers. A student is selected at random.
		\label{ncert/12/13/2/15}
\begin{enumerate}
\item Find the probability that she reads neither Hindi nor English newspapers.
\item If she reads Hindi newspaper, find the probability that she reads English newspaper.
\item If she reads English newspaper, find the probability that she reads Hindi newspaper.\\
\end{enumerate}
\item The probability of obtaining an even prime number on each die, when a pair of dice is rolled is 
\begin{enumerate}
    \item $0$ 
    
    \item $\frac{1}{3}$ 
    
    \item $\frac{1}{12}$ 
    
    \item $\frac{1}{36}$ 
\end{enumerate}
\solution
		%\input{ncert/12/13/2/17/defs.tex}
	\item A bag contains 4 red and 4 black balls, another bag contains 2 red and 6 black balls. One of the two bags is selected at random and a ball is drawn from the bag which is found to be red. Find the probability that the ball is drawn from the first bag.
\\
\solution
		%\input{ncert/12/13/3/2/main.tex}
  \item
  Cards with numbers 2 to 101 are placed in a box. A card is selected at random.Find the probability that the card has
\begin{enumerate}[label=(\roman*)]
	\item an even number 
	\item a square number
\end{enumerate}
\solution
%\input{exemplar/10/13/3/32/main.tex}
\item
The king, queen and jack of clubs are removed from a deck of 52 playing cards and then well shuffled. Now one card is drawn at random from the remaining cards.  Determine the probability that the card is
\begin{enumerate}[label=(\roman*)]
\item a club
\item 10 of hearts
\end{enumerate}
\solution
%\input{exemplar/10/13/3/29/main.tex}
\item A team of medical students doing their internship have to assist during surgeries
at a city hospital. The probabilities of surgeries rated as very complex, complex,
routine, simple or very simple are respectively, 0.15, 0.20, 0.31, 0.26, .08. Find
the probabilities that a particular surgery will be rated
\begin{enumerate}
	\item complex or very complex;
	\item neither very complex nor very simple;
	\item routine or complex
	\item routine or simple
\end{enumerate}
\solution
%\input{exemplar/11/16/3/8(1)/main.tex}
\item A card is selected from a pack of 52 cards.
\begin{enumerate}[label=(\alph*)]
    \item How many points are there in the sample space?
    \item Calculate the probability that the card is an ace of spades.
    \item Calculate the probability that the card is (i) an ace and (ii) black card.
\end{enumerate}
\solution
%\input{exemplar/11/16/3/4/main2.tex}
\item The probability that a non leap year selected at random will contain 53 sundays.
\\
\solution
%\input{exemplar/10/13/1/19/main.tex}
\item One of the four persons John, Rita, Aslam or Gurpreet will be promoted next
month. Consequently the sample space consists of four elementary outcomes
S = {John promoted, Rita promoted, Aslam promoted, Gurpreet promoted}
You are told that the chances of John’s promotion is same as that of Gurpreet,
Rita’s chances of promotion are twice as likely as Johns. Aslam’s chances are
four times that of John.
\begin{enumerate}
	\item Determine
	\begin{enumerate}
		\item P (John promoted)
		\item P (Rita promoted)
		\item P (Aslam promoted)
		\item P (Gurpreet promoted)
	\end{enumerate}
	\item If A = {John promoted or Gurpreet promoted}, find P (A).
\end{enumerate}
\solution
%\input{exemplar/11/16/3/10/main.tex}
\item A card is drawn from a deck of 52 cards. Find the probability of getting a king or a heart or a red card.\\
\solution
%\input{exemplar/11/16/3/15/main.tex}
\item The probability that a student will pass his examination is 0.73, the probability of
the student getting a compartment is 0.13, and the probability that the student will
either pass or get compartment is 0.96. State True or False.\\
\solution
%\input{exemplar/11/16/3/31/main.tex}
\item A card is selected from a pack of 52 cards\\
\begin{enumerate}[label=(\alph*)]
\item How many points are there in the sample space?
\item Calculate the probability that the cards is an ace of spades.
\item Calculate the probability that the card is (i) an ace (ii)black card.\\
\end{enumerate}
%\input{ncert/11/16/3/4_1/Prob_4.tex}
\item In a non-leap year, the probability of having 53 tuesdays or 53 wednesdays is\\
\solution
%\input{exemplar/11/16/3/18/main.tex}
\item There are 1000 sealed envelopes in a box, 10 of them contain a cash prize of
Rs 100 each, 100 of them contain a cash prize of Rs 50 each and 200 of them
contain a cash prize of Rs 10 each and rest do not contain any cash prize. If they
are well shuffled and an envelope is picked up out, what is the probability that it
contains no cash prize?\\
\solution
%\input{exemplar/10/13/3/34/main.tex}
\item 
A die is thrown and a card is selected at random from a deck of 52 playing cards. The probability of getting an even number on the die and a spade card.\\
\solution
%\input{exemplar/12/13/3/78/main.tex}
\item
If 4-digit numbers greater than 5,000 are randomly formed from the digits 0, 1, 3, 5, and 7, what is the probability of forming a number divisible by 5 when:
\begin{enumerate}
    \item The digits are repeated?
    \item The repetition of digits is not allowed?
\end{enumerate}
\solution
%\input{ncert/11/16/4/9/main.tex}
\item Consider the probability space $\brak{\Omega, \mathcal{G}, P}$ where $\Omega = [0,2]$ and $\mathcal{G} = \cbrak{\phi, \Omega, [0,1], (1,2]}$. Let $X$ and $Y$ be two functions on $\Omega$ defined as
\begin{align*}
    X(\omega) = 
    \begin{cases}
        1 & \text{if }\omega \in [0, 1]\\
        2 & \text{if }\omega \in (1, 2]
    \end{cases}
\end{align*}
and
\begin{align*}
    Y(\omega) = 
    \begin{cases}
        2 & \text{if }\omega \in [0, 1.5]\\
        3 & \text{if }\omega \in (1.5, 2].
    \end{cases}
\end{align*}
Then which one of the following statements is true?
\begin{enumerate}
    \item [(A)] $X$ is a random variable with respect to $\mathcal{G}$, but $Y$ is not a random variable with respect to $\mathcal{G}$.
    \item [(B)] $Y$ is a random variable with respect to $\mathcal{G}$, but $X$ is not a random variable with respect to $\mathcal{G}$.
    \item [(C)] Neither $X$ nor $Y$ is a random variable with respect to $\mathcal{G}$.
    \item [(D)] Both $X$ and $Y$ are random variables with respect to $\mathcal{G}$.
\end{enumerate} \hfill (GATE ST 2023)\\
\solution
%\input{gate/ST/2023/14/main.tex}
	\item  A die is loaded in such a way that each odd number is twice as likely to occur as
each even number. Find $P(G)$, where $G$ is the event that a number greater than
3 occurs on a single roll of the die.
\\
\solution
		%\input{exemplar/11/16/3/5/main.tex}
	\item All the jacks, queens and kings are removed from a deck of 52 playing cards. The remaining cards are well shuffled and then one card is drawn at random. Giving ace a value 1 similar value for other cards, find the probability that the card has a value 
		\begin{enumerate}
			\item 7
			\item greater than 7
			\item less than 7
		\end{enumerate}
		%\input{exemplar/10/13/3/30/main.tex}
  \item A Lot consists of 48 mobile phones of which 42 are good, 3 have only minor defects and 3 have major defects.Varnika will buy a phone if it is good but the trader will only buy a mobile if it has no major defects. One phone is selected at random from the lot. What is the probability that it is
\begin{enumerate}
	\item acceptable to Varnika?
            \item acceptable to the trader?
\end{enumerate}
\solution
	%\input{exemplar/10/13/3/40/main.tex}
 \item A student says that if you throw a die, it will show up 1 or not 1. Therefore, the probability of getting 1 and the probability of getting 'not 1' each is equal to $\frac{1}{2}$. Is this correct? Give reasons.\\
 \solution
        %\input{exemplar/10/13/2/9/main.tex}
   \item Four candidates A, B, C, D have ap-
plied for the assignment to coach a school cricket
team. If A is twice as likely to be selected as B, and
B and C are given about the same chance of being
selected, while C is twice as likely to be selected
as D, what are the probabilities that
\begin{enumerate}
\item C will be selected?
\item A will not be selected?
\end{enumerate}
	%\input{exemplar/11/16/3/9/main.tex}
 \item A bag contain 24 balls of which $x$ balls are red, $2x$ are white and $3x$ are blue. A ball is selected at random, What is the probability that it is
\begin{enumerate}[label=\alph*)]
\item not red ?
\item white ?
\end{enumerate}
%\input{exemplar/10/13/3/41/main.tex}
If the letters of the word ASSASSINATION are arranged at random. Find the Probability that
\begin{enumerate}[label=(\alph*)]
\item Four $S's$ come consecutively in the word
\item Two  $I's$ and two $N's$ come together
\item All $A's$ are not coming together
\item No two $A's$ are coming together
\end{enumerate}
%\input{exemplar/11/16/3/14/main.tex}
	\item One urn contains two black balls (labelled B1 and B2) and one white ball. A
	second urn contains one black ball and two white balls (labelled W1 and W2).
	Suppose the following experiment is performed. One of the two urns is chosen
	at random. Next a ball is randomly chosen from the urn. Then a second ball is
	chosen at random from the same urn without replacing the first ball.
	
	\begin{enumerate}
	\item What is the probability that two black balls are chosen?
	
	\item What is the probability that two balls of opposite colour are chosen?
	\end{enumerate}
	\solution
	%\input{exemplar/11/16/3/12/main1.tex}
\end{enumerate}

	\item A card is selected from a pack of 52 cards.
 \begin{enumerate}[label=(\alph*)] 
                 \item How many points are there in the sample space?
                 \item Calculate the probability that the card is an ace of spades.
                 \item Calculate the probability that the card is (i) an ace and (ii) black card.
 \end{enumerate}
\solution
		%\begin{table}[H]
	\centering
\begin{tabular}{|c|c|c|}
\hline
Random variable &Value &Definition\\ \hline
\multirow{3}{*}{X} &0 &Slips of Rs 1\\
&1 &Slips of Rs 5\\
&2 &Slips of Rs 13\\ \hline
\multirow{2}{*}{Y} &0 &Box A\\
&1 &Box B\\\hline
\end{tabular}
\caption{}
\label{tab:Distribution}
\end{table}
See \tabref{tab:Distribution}.
\begin{align}
p_{Y}\brak{k}= \begin{cases} 
      \frac{1}{3} & {k=0} \\
      \frac{2}{3 }& {k=1} 
   \end{cases}
   \\
p_{Y|X}\brak{0|0} = \frac{19}{25}\, 
p_{Y|X}\brak{0|1} = \frac{6}{25}\,
p_{Y|X}\brak{1|0} = \frac{45}{50}\,
p_{Y|X}\brak{1|2} = \frac{5}{50}
\end{align}
The desired probability is the probability that a slip drawn at random is marked other than Rs 1,
\begin{align}
&=1-p_X\brak{0}\\
&= p_X(1) + p_X(2)
\end{align}
Using Bayes theorem,
\begin{align}
&= p_Y\brak{0} \times \pr{Y=0 | X=1} + p_Y\brak{1} \times \pr{Y=1|X=2}\\
&=\frac{1}{3} \times \frac{6}{25} + \frac{2}{3} \times \frac{5}{50}\\
&=\frac{11}{75}
\end{align}

\newpage

%\tableofcontents

\bigskip

\renewcommand{\thefigure}{\theenumi}
\renewcommand{\thetable}{\theenumi}
%\renewcommand{\theequation}{\theenumi}

%\begin{abstract}
%%\boldmath
%In this letter, an algorithm for evaluating the exact analytical bit error rate  (BER)  for the piecewise linear (PL) combiner for  multiple relays is presented. Previous results were available only for upto three relays. The algorithm is unique in the sense that  the actual mathematical expressions, that are prohibitively large, need not be explicitly obtained. The diversity gain due to multiple relays is shown through plots of the analytical BER, well supported by simulations. 
%
%\end{abstract}
% IEEEtran.cls defaults to using nonbold math in the Abstract.
% This preserves the distinction between vectors and scalars. However,
% if the journal you are submitting to favors bold math in the abstract,
% then you can use LaTeX's standard command \boldmath at the very start
% of the abstract to achieve this. Many IEEE journals frown on math
% in the abstract anyway.

% Note that keywords are not normally used for peerreview papers.
%\begin{IEEEkeywords}
%Cooperative diversity, decode and forward, piecewise linear
%\end{IEEEkeywords}



% For peer review papers, you can put extra information on the cover
% page as needed:
% \ifCLASSOPTIONpeerreview
% \begin{center} \bfseries EDICS Category: 3-BBND \end{center}
% \fi
%
% For peerreview papers, this IEEEtran command inserts a page break and
% creates the second title. It will be ignored for other modes.
%\IEEEpeerreviewmaketitle




\item Four cards are drawn from a well-shuffled deck of 52 cards. What is the probability of obtaining 3 diamonds and one spade.
\\
\solution
		%\begin{enumerate}[label=\thesection.\arabic*,ref=\thesection.\theenumi]
	\item One card is drawn from a well-shuffled deck of 52 cards. Find the probability of getting
\begin{enumerate}
\item A king of red colour 
\item A face card 
\item A red face card
\item The jack of hearts
\item A spade
\item The queen of diamonds

\end{enumerate}
\solution
		%\input{ncert/10/15/1/14/main.tex}
	\item Five cards—the ten, jack, queen, king and ace of diamonds, are well-shuffled with their face downwards. One card is then picked up at random.
\begin{enumerate}
\item
What is the probability that the card is the queen? 
\item
If the queen is drawn and put aside, what is the probability that the second card picked up is (a) an ace? (b) a queen?\\
\end{enumerate}
\solution
		%\input{ncert/10/15/1/15/defs.tex}
	\item A bag contains $5$ red balls and some blue balls. If the probability of drawing a blue ball is double that if a red ball, determine the number of blue balls in the bag. 
		\\
\solution
		%\input{ncert/10/15/2/3/defs.tex}
	\item A card is selected from a pack of 52 cards.
 \begin{enumerate}[label=(\alph*)] 
                 \item How many points are there in the sample space?
                 \item Calculate the probability that the card is an ace of spades.
                 \item Calculate the probability that the card is (i) an ace and (ii) black card.
 \end{enumerate}
\solution
		%\input{ncert/11/16/3/4/main.tex}
\item Four cards are drawn from a well-shuffled deck of 52 cards. What is the probability of obtaining 3 diamonds and one spade.
\\
\solution
		%\input{ncert/11/16/4/2/defs.tex}
\item In a certain lottery 10,000 tickets are sold and ten equal prizes are awarded. What is the probability of not getting a prize if you buy (a) one ticket (b) two tickets (c) 10 tickets ?	
\\
\solution
		%\input{ncert/11/16/4/4/defs.tex}
		%
\item 
Out of 100 students, two sections of 40 and 60 are formed. If you and your friend are among the 100 students, what is the probability that
\begin{enumerate}
\item you both enter the same section?
\item you both enter the different sections?
\end{enumerate}
\solution
		%\input{ncert/11/16/4/5/defs.tex}
	\item 
The number lock of a suitcase has 4 wheels each labelled with ten digits i.e. from 0 to 9.The lock opens with a sequence of four digits with no repeats.What is the probability of a person getting the right sequence to open the suitcase.
\\
\solution
		%\input{ncert/11/16/4/10/defs.tex}
		%
\item 
Two cards are drawn at random and without replacement from a pack of 52 playing cards. Find the probability that both the cards are black.
\\
\solution
		%\input{ncert/12/13/2/2/defs.tex}
		\item A box of oranges is inspected by examining three randomly selected oranges drawn without replacement. If all the three oranges are good, the box is approved for sale, otherwise, it is rejected. Find the probability that a box containing 15 oranges out of which 12 are good and 3 are bad ones will be approved for sale.
		\label{ncert/12/13/2/3/defs.tex}
		\item Two balls are drawn at random with replacement from a box containing 10 black and 8 red balls. Find the probability that
		\label{ncert/12/13/2/12}
\begin{enumerate}
\item both balls are red.
\item first ball is black and second is red.
\item one of them is black and other is red.
\end{enumerate}

\item In a hostel, 60\% of the students read Hindi newspaper, 40\% read English newspaper and 20\% read both Hindi and English newspapers. A student is selected at random.
		\label{ncert/12/13/2/15}
\begin{enumerate}
\item Find the probability that she reads neither Hindi nor English newspapers.
\item If she reads Hindi newspaper, find the probability that she reads English newspaper.
\item If she reads English newspaper, find the probability that she reads Hindi newspaper.\\
\end{enumerate}
\item The probability of obtaining an even prime number on each die, when a pair of dice is rolled is 
\begin{enumerate}
    \item $0$ 
    
    \item $\frac{1}{3}$ 
    
    \item $\frac{1}{12}$ 
    
    \item $\frac{1}{36}$ 
\end{enumerate}
\solution
		%\input{ncert/12/13/2/17/defs.tex}
	\item A bag contains 4 red and 4 black balls, another bag contains 2 red and 6 black balls. One of the two bags is selected at random and a ball is drawn from the bag which is found to be red. Find the probability that the ball is drawn from the first bag.
\\
\solution
		%\input{ncert/12/13/3/2/main.tex}
  \item
  Cards with numbers 2 to 101 are placed in a box. A card is selected at random.Find the probability that the card has
\begin{enumerate}[label=(\roman*)]
	\item an even number 
	\item a square number
\end{enumerate}
\solution
%\input{exemplar/10/13/3/32/main.tex}
\item
The king, queen and jack of clubs are removed from a deck of 52 playing cards and then well shuffled. Now one card is drawn at random from the remaining cards.  Determine the probability that the card is
\begin{enumerate}[label=(\roman*)]
\item a club
\item 10 of hearts
\end{enumerate}
\solution
%\input{exemplar/10/13/3/29/main.tex}
\item A team of medical students doing their internship have to assist during surgeries
at a city hospital. The probabilities of surgeries rated as very complex, complex,
routine, simple or very simple are respectively, 0.15, 0.20, 0.31, 0.26, .08. Find
the probabilities that a particular surgery will be rated
\begin{enumerate}
	\item complex or very complex;
	\item neither very complex nor very simple;
	\item routine or complex
	\item routine or simple
\end{enumerate}
\solution
%\input{exemplar/11/16/3/8(1)/main.tex}
\item A card is selected from a pack of 52 cards.
\begin{enumerate}[label=(\alph*)]
    \item How many points are there in the sample space?
    \item Calculate the probability that the card is an ace of spades.
    \item Calculate the probability that the card is (i) an ace and (ii) black card.
\end{enumerate}
\solution
%\input{exemplar/11/16/3/4/main2.tex}
\item The probability that a non leap year selected at random will contain 53 sundays.
\\
\solution
%\input{exemplar/10/13/1/19/main.tex}
\item One of the four persons John, Rita, Aslam or Gurpreet will be promoted next
month. Consequently the sample space consists of four elementary outcomes
S = {John promoted, Rita promoted, Aslam promoted, Gurpreet promoted}
You are told that the chances of John’s promotion is same as that of Gurpreet,
Rita’s chances of promotion are twice as likely as Johns. Aslam’s chances are
four times that of John.
\begin{enumerate}
	\item Determine
	\begin{enumerate}
		\item P (John promoted)
		\item P (Rita promoted)
		\item P (Aslam promoted)
		\item P (Gurpreet promoted)
	\end{enumerate}
	\item If A = {John promoted or Gurpreet promoted}, find P (A).
\end{enumerate}
\solution
%\input{exemplar/11/16/3/10/main.tex}
\item A card is drawn from a deck of 52 cards. Find the probability of getting a king or a heart or a red card.\\
\solution
%\input{exemplar/11/16/3/15/main.tex}
\item The probability that a student will pass his examination is 0.73, the probability of
the student getting a compartment is 0.13, and the probability that the student will
either pass or get compartment is 0.96. State True or False.\\
\solution
%\input{exemplar/11/16/3/31/main.tex}
\item A card is selected from a pack of 52 cards\\
\begin{enumerate}[label=(\alph*)]
\item How many points are there in the sample space?
\item Calculate the probability that the cards is an ace of spades.
\item Calculate the probability that the card is (i) an ace (ii)black card.\\
\end{enumerate}
%\input{ncert/11/16/3/4_1/Prob_4.tex}
\item In a non-leap year, the probability of having 53 tuesdays or 53 wednesdays is\\
\solution
%\input{exemplar/11/16/3/18/main.tex}
\item There are 1000 sealed envelopes in a box, 10 of them contain a cash prize of
Rs 100 each, 100 of them contain a cash prize of Rs 50 each and 200 of them
contain a cash prize of Rs 10 each and rest do not contain any cash prize. If they
are well shuffled and an envelope is picked up out, what is the probability that it
contains no cash prize?\\
\solution
%\input{exemplar/10/13/3/34/main.tex}
\item 
A die is thrown and a card is selected at random from a deck of 52 playing cards. The probability of getting an even number on the die and a spade card.\\
\solution
%\input{exemplar/12/13/3/78/main.tex}
\item
If 4-digit numbers greater than 5,000 are randomly formed from the digits 0, 1, 3, 5, and 7, what is the probability of forming a number divisible by 5 when:
\begin{enumerate}
    \item The digits are repeated?
    \item The repetition of digits is not allowed?
\end{enumerate}
\solution
%\input{ncert/11/16/4/9/main.tex}
\item Consider the probability space $\brak{\Omega, \mathcal{G}, P}$ where $\Omega = [0,2]$ and $\mathcal{G} = \cbrak{\phi, \Omega, [0,1], (1,2]}$. Let $X$ and $Y$ be two functions on $\Omega$ defined as
\begin{align*}
    X(\omega) = 
    \begin{cases}
        1 & \text{if }\omega \in [0, 1]\\
        2 & \text{if }\omega \in (1, 2]
    \end{cases}
\end{align*}
and
\begin{align*}
    Y(\omega) = 
    \begin{cases}
        2 & \text{if }\omega \in [0, 1.5]\\
        3 & \text{if }\omega \in (1.5, 2].
    \end{cases}
\end{align*}
Then which one of the following statements is true?
\begin{enumerate}
    \item [(A)] $X$ is a random variable with respect to $\mathcal{G}$, but $Y$ is not a random variable with respect to $\mathcal{G}$.
    \item [(B)] $Y$ is a random variable with respect to $\mathcal{G}$, but $X$ is not a random variable with respect to $\mathcal{G}$.
    \item [(C)] Neither $X$ nor $Y$ is a random variable with respect to $\mathcal{G}$.
    \item [(D)] Both $X$ and $Y$ are random variables with respect to $\mathcal{G}$.
\end{enumerate} \hfill (GATE ST 2023)\\
\solution
%\input{gate/ST/2023/14/main.tex}
	\item  A die is loaded in such a way that each odd number is twice as likely to occur as
each even number. Find $P(G)$, where $G$ is the event that a number greater than
3 occurs on a single roll of the die.
\\
\solution
		%\input{exemplar/11/16/3/5/main.tex}
	\item All the jacks, queens and kings are removed from a deck of 52 playing cards. The remaining cards are well shuffled and then one card is drawn at random. Giving ace a value 1 similar value for other cards, find the probability that the card has a value 
		\begin{enumerate}
			\item 7
			\item greater than 7
			\item less than 7
		\end{enumerate}
		%\input{exemplar/10/13/3/30/main.tex}
  \item A Lot consists of 48 mobile phones of which 42 are good, 3 have only minor defects and 3 have major defects.Varnika will buy a phone if it is good but the trader will only buy a mobile if it has no major defects. One phone is selected at random from the lot. What is the probability that it is
\begin{enumerate}
	\item acceptable to Varnika?
            \item acceptable to the trader?
\end{enumerate}
\solution
	%\input{exemplar/10/13/3/40/main.tex}
 \item A student says that if you throw a die, it will show up 1 or not 1. Therefore, the probability of getting 1 and the probability of getting 'not 1' each is equal to $\frac{1}{2}$. Is this correct? Give reasons.\\
 \solution
        %\input{exemplar/10/13/2/9/main.tex}
   \item Four candidates A, B, C, D have ap-
plied for the assignment to coach a school cricket
team. If A is twice as likely to be selected as B, and
B and C are given about the same chance of being
selected, while C is twice as likely to be selected
as D, what are the probabilities that
\begin{enumerate}
\item C will be selected?
\item A will not be selected?
\end{enumerate}
	%\input{exemplar/11/16/3/9/main.tex}
 \item A bag contain 24 balls of which $x$ balls are red, $2x$ are white and $3x$ are blue. A ball is selected at random, What is the probability that it is
\begin{enumerate}[label=\alph*)]
\item not red ?
\item white ?
\end{enumerate}
%\input{exemplar/10/13/3/41/main.tex}
If the letters of the word ASSASSINATION are arranged at random. Find the Probability that
\begin{enumerate}[label=(\alph*)]
\item Four $S's$ come consecutively in the word
\item Two  $I's$ and two $N's$ come together
\item All $A's$ are not coming together
\item No two $A's$ are coming together
\end{enumerate}
%\input{exemplar/11/16/3/14/main.tex}
	\item One urn contains two black balls (labelled B1 and B2) and one white ball. A
	second urn contains one black ball and two white balls (labelled W1 and W2).
	Suppose the following experiment is performed. One of the two urns is chosen
	at random. Next a ball is randomly chosen from the urn. Then a second ball is
	chosen at random from the same urn without replacing the first ball.
	
	\begin{enumerate}
	\item What is the probability that two black balls are chosen?
	
	\item What is the probability that two balls of opposite colour are chosen?
	\end{enumerate}
	\solution
	%\input{exemplar/11/16/3/12/main1.tex}
\end{enumerate}

\item In a certain lottery 10,000 tickets are sold and ten equal prizes are awarded. What is the probability of not getting a prize if you buy (a) one ticket (b) two tickets (c) 10 tickets ?	
\\
\solution
		%\begin{enumerate}[label=\thesection.\arabic*,ref=\thesection.\theenumi]
	\item One card is drawn from a well-shuffled deck of 52 cards. Find the probability of getting
\begin{enumerate}
\item A king of red colour 
\item A face card 
\item A red face card
\item The jack of hearts
\item A spade
\item The queen of diamonds

\end{enumerate}
\solution
		%\input{ncert/10/15/1/14/main.tex}
	\item Five cards—the ten, jack, queen, king and ace of diamonds, are well-shuffled with their face downwards. One card is then picked up at random.
\begin{enumerate}
\item
What is the probability that the card is the queen? 
\item
If the queen is drawn and put aside, what is the probability that the second card picked up is (a) an ace? (b) a queen?\\
\end{enumerate}
\solution
		%\input{ncert/10/15/1/15/defs.tex}
	\item A bag contains $5$ red balls and some blue balls. If the probability of drawing a blue ball is double that if a red ball, determine the number of blue balls in the bag. 
		\\
\solution
		%\input{ncert/10/15/2/3/defs.tex}
	\item A card is selected from a pack of 52 cards.
 \begin{enumerate}[label=(\alph*)] 
                 \item How many points are there in the sample space?
                 \item Calculate the probability that the card is an ace of spades.
                 \item Calculate the probability that the card is (i) an ace and (ii) black card.
 \end{enumerate}
\solution
		%\input{ncert/11/16/3/4/main.tex}
\item Four cards are drawn from a well-shuffled deck of 52 cards. What is the probability of obtaining 3 diamonds and one spade.
\\
\solution
		%\input{ncert/11/16/4/2/defs.tex}
\item In a certain lottery 10,000 tickets are sold and ten equal prizes are awarded. What is the probability of not getting a prize if you buy (a) one ticket (b) two tickets (c) 10 tickets ?	
\\
\solution
		%\input{ncert/11/16/4/4/defs.tex}
		%
\item 
Out of 100 students, two sections of 40 and 60 are formed. If you and your friend are among the 100 students, what is the probability that
\begin{enumerate}
\item you both enter the same section?
\item you both enter the different sections?
\end{enumerate}
\solution
		%\input{ncert/11/16/4/5/defs.tex}
	\item 
The number lock of a suitcase has 4 wheels each labelled with ten digits i.e. from 0 to 9.The lock opens with a sequence of four digits with no repeats.What is the probability of a person getting the right sequence to open the suitcase.
\\
\solution
		%\input{ncert/11/16/4/10/defs.tex}
		%
\item 
Two cards are drawn at random and without replacement from a pack of 52 playing cards. Find the probability that both the cards are black.
\\
\solution
		%\input{ncert/12/13/2/2/defs.tex}
		\item A box of oranges is inspected by examining three randomly selected oranges drawn without replacement. If all the three oranges are good, the box is approved for sale, otherwise, it is rejected. Find the probability that a box containing 15 oranges out of which 12 are good and 3 are bad ones will be approved for sale.
		\label{ncert/12/13/2/3/defs.tex}
		\item Two balls are drawn at random with replacement from a box containing 10 black and 8 red balls. Find the probability that
		\label{ncert/12/13/2/12}
\begin{enumerate}
\item both balls are red.
\item first ball is black and second is red.
\item one of them is black and other is red.
\end{enumerate}

\item In a hostel, 60\% of the students read Hindi newspaper, 40\% read English newspaper and 20\% read both Hindi and English newspapers. A student is selected at random.
		\label{ncert/12/13/2/15}
\begin{enumerate}
\item Find the probability that she reads neither Hindi nor English newspapers.
\item If she reads Hindi newspaper, find the probability that she reads English newspaper.
\item If she reads English newspaper, find the probability that she reads Hindi newspaper.\\
\end{enumerate}
\item The probability of obtaining an even prime number on each die, when a pair of dice is rolled is 
\begin{enumerate}
    \item $0$ 
    
    \item $\frac{1}{3}$ 
    
    \item $\frac{1}{12}$ 
    
    \item $\frac{1}{36}$ 
\end{enumerate}
\solution
		%\input{ncert/12/13/2/17/defs.tex}
	\item A bag contains 4 red and 4 black balls, another bag contains 2 red and 6 black balls. One of the two bags is selected at random and a ball is drawn from the bag which is found to be red. Find the probability that the ball is drawn from the first bag.
\\
\solution
		%\input{ncert/12/13/3/2/main.tex}
  \item
  Cards with numbers 2 to 101 are placed in a box. A card is selected at random.Find the probability that the card has
\begin{enumerate}[label=(\roman*)]
	\item an even number 
	\item a square number
\end{enumerate}
\solution
%\input{exemplar/10/13/3/32/main.tex}
\item
The king, queen and jack of clubs are removed from a deck of 52 playing cards and then well shuffled. Now one card is drawn at random from the remaining cards.  Determine the probability that the card is
\begin{enumerate}[label=(\roman*)]
\item a club
\item 10 of hearts
\end{enumerate}
\solution
%\input{exemplar/10/13/3/29/main.tex}
\item A team of medical students doing their internship have to assist during surgeries
at a city hospital. The probabilities of surgeries rated as very complex, complex,
routine, simple or very simple are respectively, 0.15, 0.20, 0.31, 0.26, .08. Find
the probabilities that a particular surgery will be rated
\begin{enumerate}
	\item complex or very complex;
	\item neither very complex nor very simple;
	\item routine or complex
	\item routine or simple
\end{enumerate}
\solution
%\input{exemplar/11/16/3/8(1)/main.tex}
\item A card is selected from a pack of 52 cards.
\begin{enumerate}[label=(\alph*)]
    \item How many points are there in the sample space?
    \item Calculate the probability that the card is an ace of spades.
    \item Calculate the probability that the card is (i) an ace and (ii) black card.
\end{enumerate}
\solution
%\input{exemplar/11/16/3/4/main2.tex}
\item The probability that a non leap year selected at random will contain 53 sundays.
\\
\solution
%\input{exemplar/10/13/1/19/main.tex}
\item One of the four persons John, Rita, Aslam or Gurpreet will be promoted next
month. Consequently the sample space consists of four elementary outcomes
S = {John promoted, Rita promoted, Aslam promoted, Gurpreet promoted}
You are told that the chances of John’s promotion is same as that of Gurpreet,
Rita’s chances of promotion are twice as likely as Johns. Aslam’s chances are
four times that of John.
\begin{enumerate}
	\item Determine
	\begin{enumerate}
		\item P (John promoted)
		\item P (Rita promoted)
		\item P (Aslam promoted)
		\item P (Gurpreet promoted)
	\end{enumerate}
	\item If A = {John promoted or Gurpreet promoted}, find P (A).
\end{enumerate}
\solution
%\input{exemplar/11/16/3/10/main.tex}
\item A card is drawn from a deck of 52 cards. Find the probability of getting a king or a heart or a red card.\\
\solution
%\input{exemplar/11/16/3/15/main.tex}
\item The probability that a student will pass his examination is 0.73, the probability of
the student getting a compartment is 0.13, and the probability that the student will
either pass or get compartment is 0.96. State True or False.\\
\solution
%\input{exemplar/11/16/3/31/main.tex}
\item A card is selected from a pack of 52 cards\\
\begin{enumerate}[label=(\alph*)]
\item How many points are there in the sample space?
\item Calculate the probability that the cards is an ace of spades.
\item Calculate the probability that the card is (i) an ace (ii)black card.\\
\end{enumerate}
%\input{ncert/11/16/3/4_1/Prob_4.tex}
\item In a non-leap year, the probability of having 53 tuesdays or 53 wednesdays is\\
\solution
%\input{exemplar/11/16/3/18/main.tex}
\item There are 1000 sealed envelopes in a box, 10 of them contain a cash prize of
Rs 100 each, 100 of them contain a cash prize of Rs 50 each and 200 of them
contain a cash prize of Rs 10 each and rest do not contain any cash prize. If they
are well shuffled and an envelope is picked up out, what is the probability that it
contains no cash prize?\\
\solution
%\input{exemplar/10/13/3/34/main.tex}
\item 
A die is thrown and a card is selected at random from a deck of 52 playing cards. The probability of getting an even number on the die and a spade card.\\
\solution
%\input{exemplar/12/13/3/78/main.tex}
\item
If 4-digit numbers greater than 5,000 are randomly formed from the digits 0, 1, 3, 5, and 7, what is the probability of forming a number divisible by 5 when:
\begin{enumerate}
    \item The digits are repeated?
    \item The repetition of digits is not allowed?
\end{enumerate}
\solution
%\input{ncert/11/16/4/9/main.tex}
\item Consider the probability space $\brak{\Omega, \mathcal{G}, P}$ where $\Omega = [0,2]$ and $\mathcal{G} = \cbrak{\phi, \Omega, [0,1], (1,2]}$. Let $X$ and $Y$ be two functions on $\Omega$ defined as
\begin{align*}
    X(\omega) = 
    \begin{cases}
        1 & \text{if }\omega \in [0, 1]\\
        2 & \text{if }\omega \in (1, 2]
    \end{cases}
\end{align*}
and
\begin{align*}
    Y(\omega) = 
    \begin{cases}
        2 & \text{if }\omega \in [0, 1.5]\\
        3 & \text{if }\omega \in (1.5, 2].
    \end{cases}
\end{align*}
Then which one of the following statements is true?
\begin{enumerate}
    \item [(A)] $X$ is a random variable with respect to $\mathcal{G}$, but $Y$ is not a random variable with respect to $\mathcal{G}$.
    \item [(B)] $Y$ is a random variable with respect to $\mathcal{G}$, but $X$ is not a random variable with respect to $\mathcal{G}$.
    \item [(C)] Neither $X$ nor $Y$ is a random variable with respect to $\mathcal{G}$.
    \item [(D)] Both $X$ and $Y$ are random variables with respect to $\mathcal{G}$.
\end{enumerate} \hfill (GATE ST 2023)\\
\solution
%\input{gate/ST/2023/14/main.tex}
	\item  A die is loaded in such a way that each odd number is twice as likely to occur as
each even number. Find $P(G)$, where $G$ is the event that a number greater than
3 occurs on a single roll of the die.
\\
\solution
		%\input{exemplar/11/16/3/5/main.tex}
	\item All the jacks, queens and kings are removed from a deck of 52 playing cards. The remaining cards are well shuffled and then one card is drawn at random. Giving ace a value 1 similar value for other cards, find the probability that the card has a value 
		\begin{enumerate}
			\item 7
			\item greater than 7
			\item less than 7
		\end{enumerate}
		%\input{exemplar/10/13/3/30/main.tex}
  \item A Lot consists of 48 mobile phones of which 42 are good, 3 have only minor defects and 3 have major defects.Varnika will buy a phone if it is good but the trader will only buy a mobile if it has no major defects. One phone is selected at random from the lot. What is the probability that it is
\begin{enumerate}
	\item acceptable to Varnika?
            \item acceptable to the trader?
\end{enumerate}
\solution
	%\input{exemplar/10/13/3/40/main.tex}
 \item A student says that if you throw a die, it will show up 1 or not 1. Therefore, the probability of getting 1 and the probability of getting 'not 1' each is equal to $\frac{1}{2}$. Is this correct? Give reasons.\\
 \solution
        %\input{exemplar/10/13/2/9/main.tex}
   \item Four candidates A, B, C, D have ap-
plied for the assignment to coach a school cricket
team. If A is twice as likely to be selected as B, and
B and C are given about the same chance of being
selected, while C is twice as likely to be selected
as D, what are the probabilities that
\begin{enumerate}
\item C will be selected?
\item A will not be selected?
\end{enumerate}
	%\input{exemplar/11/16/3/9/main.tex}
 \item A bag contain 24 balls of which $x$ balls are red, $2x$ are white and $3x$ are blue. A ball is selected at random, What is the probability that it is
\begin{enumerate}[label=\alph*)]
\item not red ?
\item white ?
\end{enumerate}
%\input{exemplar/10/13/3/41/main.tex}
If the letters of the word ASSASSINATION are arranged at random. Find the Probability that
\begin{enumerate}[label=(\alph*)]
\item Four $S's$ come consecutively in the word
\item Two  $I's$ and two $N's$ come together
\item All $A's$ are not coming together
\item No two $A's$ are coming together
\end{enumerate}
%\input{exemplar/11/16/3/14/main.tex}
	\item One urn contains two black balls (labelled B1 and B2) and one white ball. A
	second urn contains one black ball and two white balls (labelled W1 and W2).
	Suppose the following experiment is performed. One of the two urns is chosen
	at random. Next a ball is randomly chosen from the urn. Then a second ball is
	chosen at random from the same urn without replacing the first ball.
	
	\begin{enumerate}
	\item What is the probability that two black balls are chosen?
	
	\item What is the probability that two balls of opposite colour are chosen?
	\end{enumerate}
	\solution
	%\input{exemplar/11/16/3/12/main1.tex}
\end{enumerate}

		%
\item 
Out of 100 students, two sections of 40 and 60 are formed. If you and your friend are among the 100 students, what is the probability that
\begin{enumerate}
\item you both enter the same section?
\item you both enter the different sections?
\end{enumerate}
\solution
		%\begin{enumerate}[label=\thesection.\arabic*,ref=\thesection.\theenumi]
	\item One card is drawn from a well-shuffled deck of 52 cards. Find the probability of getting
\begin{enumerate}
\item A king of red colour 
\item A face card 
\item A red face card
\item The jack of hearts
\item A spade
\item The queen of diamonds

\end{enumerate}
\solution
		%\input{ncert/10/15/1/14/main.tex}
	\item Five cards—the ten, jack, queen, king and ace of diamonds, are well-shuffled with their face downwards. One card is then picked up at random.
\begin{enumerate}
\item
What is the probability that the card is the queen? 
\item
If the queen is drawn and put aside, what is the probability that the second card picked up is (a) an ace? (b) a queen?\\
\end{enumerate}
\solution
		%\input{ncert/10/15/1/15/defs.tex}
	\item A bag contains $5$ red balls and some blue balls. If the probability of drawing a blue ball is double that if a red ball, determine the number of blue balls in the bag. 
		\\
\solution
		%\input{ncert/10/15/2/3/defs.tex}
	\item A card is selected from a pack of 52 cards.
 \begin{enumerate}[label=(\alph*)] 
                 \item How many points are there in the sample space?
                 \item Calculate the probability that the card is an ace of spades.
                 \item Calculate the probability that the card is (i) an ace and (ii) black card.
 \end{enumerate}
\solution
		%\input{ncert/11/16/3/4/main.tex}
\item Four cards are drawn from a well-shuffled deck of 52 cards. What is the probability of obtaining 3 diamonds and one spade.
\\
\solution
		%\input{ncert/11/16/4/2/defs.tex}
\item In a certain lottery 10,000 tickets are sold and ten equal prizes are awarded. What is the probability of not getting a prize if you buy (a) one ticket (b) two tickets (c) 10 tickets ?	
\\
\solution
		%\input{ncert/11/16/4/4/defs.tex}
		%
\item 
Out of 100 students, two sections of 40 and 60 are formed. If you and your friend are among the 100 students, what is the probability that
\begin{enumerate}
\item you both enter the same section?
\item you both enter the different sections?
\end{enumerate}
\solution
		%\input{ncert/11/16/4/5/defs.tex}
	\item 
The number lock of a suitcase has 4 wheels each labelled with ten digits i.e. from 0 to 9.The lock opens with a sequence of four digits with no repeats.What is the probability of a person getting the right sequence to open the suitcase.
\\
\solution
		%\input{ncert/11/16/4/10/defs.tex}
		%
\item 
Two cards are drawn at random and without replacement from a pack of 52 playing cards. Find the probability that both the cards are black.
\\
\solution
		%\input{ncert/12/13/2/2/defs.tex}
		\item A box of oranges is inspected by examining three randomly selected oranges drawn without replacement. If all the three oranges are good, the box is approved for sale, otherwise, it is rejected. Find the probability that a box containing 15 oranges out of which 12 are good and 3 are bad ones will be approved for sale.
		\label{ncert/12/13/2/3/defs.tex}
		\item Two balls are drawn at random with replacement from a box containing 10 black and 8 red balls. Find the probability that
		\label{ncert/12/13/2/12}
\begin{enumerate}
\item both balls are red.
\item first ball is black and second is red.
\item one of them is black and other is red.
\end{enumerate}

\item In a hostel, 60\% of the students read Hindi newspaper, 40\% read English newspaper and 20\% read both Hindi and English newspapers. A student is selected at random.
		\label{ncert/12/13/2/15}
\begin{enumerate}
\item Find the probability that she reads neither Hindi nor English newspapers.
\item If she reads Hindi newspaper, find the probability that she reads English newspaper.
\item If she reads English newspaper, find the probability that she reads Hindi newspaper.\\
\end{enumerate}
\item The probability of obtaining an even prime number on each die, when a pair of dice is rolled is 
\begin{enumerate}
    \item $0$ 
    
    \item $\frac{1}{3}$ 
    
    \item $\frac{1}{12}$ 
    
    \item $\frac{1}{36}$ 
\end{enumerate}
\solution
		%\input{ncert/12/13/2/17/defs.tex}
	\item A bag contains 4 red and 4 black balls, another bag contains 2 red and 6 black balls. One of the two bags is selected at random and a ball is drawn from the bag which is found to be red. Find the probability that the ball is drawn from the first bag.
\\
\solution
		%\input{ncert/12/13/3/2/main.tex}
  \item
  Cards with numbers 2 to 101 are placed in a box. A card is selected at random.Find the probability that the card has
\begin{enumerate}[label=(\roman*)]
	\item an even number 
	\item a square number
\end{enumerate}
\solution
%\input{exemplar/10/13/3/32/main.tex}
\item
The king, queen and jack of clubs are removed from a deck of 52 playing cards and then well shuffled. Now one card is drawn at random from the remaining cards.  Determine the probability that the card is
\begin{enumerate}[label=(\roman*)]
\item a club
\item 10 of hearts
\end{enumerate}
\solution
%\input{exemplar/10/13/3/29/main.tex}
\item A team of medical students doing their internship have to assist during surgeries
at a city hospital. The probabilities of surgeries rated as very complex, complex,
routine, simple or very simple are respectively, 0.15, 0.20, 0.31, 0.26, .08. Find
the probabilities that a particular surgery will be rated
\begin{enumerate}
	\item complex or very complex;
	\item neither very complex nor very simple;
	\item routine or complex
	\item routine or simple
\end{enumerate}
\solution
%\input{exemplar/11/16/3/8(1)/main.tex}
\item A card is selected from a pack of 52 cards.
\begin{enumerate}[label=(\alph*)]
    \item How many points are there in the sample space?
    \item Calculate the probability that the card is an ace of spades.
    \item Calculate the probability that the card is (i) an ace and (ii) black card.
\end{enumerate}
\solution
%\input{exemplar/11/16/3/4/main2.tex}
\item The probability that a non leap year selected at random will contain 53 sundays.
\\
\solution
%\input{exemplar/10/13/1/19/main.tex}
\item One of the four persons John, Rita, Aslam or Gurpreet will be promoted next
month. Consequently the sample space consists of four elementary outcomes
S = {John promoted, Rita promoted, Aslam promoted, Gurpreet promoted}
You are told that the chances of John’s promotion is same as that of Gurpreet,
Rita’s chances of promotion are twice as likely as Johns. Aslam’s chances are
four times that of John.
\begin{enumerate}
	\item Determine
	\begin{enumerate}
		\item P (John promoted)
		\item P (Rita promoted)
		\item P (Aslam promoted)
		\item P (Gurpreet promoted)
	\end{enumerate}
	\item If A = {John promoted or Gurpreet promoted}, find P (A).
\end{enumerate}
\solution
%\input{exemplar/11/16/3/10/main.tex}
\item A card is drawn from a deck of 52 cards. Find the probability of getting a king or a heart or a red card.\\
\solution
%\input{exemplar/11/16/3/15/main.tex}
\item The probability that a student will pass his examination is 0.73, the probability of
the student getting a compartment is 0.13, and the probability that the student will
either pass or get compartment is 0.96. State True or False.\\
\solution
%\input{exemplar/11/16/3/31/main.tex}
\item A card is selected from a pack of 52 cards\\
\begin{enumerate}[label=(\alph*)]
\item How many points are there in the sample space?
\item Calculate the probability that the cards is an ace of spades.
\item Calculate the probability that the card is (i) an ace (ii)black card.\\
\end{enumerate}
%\input{ncert/11/16/3/4_1/Prob_4.tex}
\item In a non-leap year, the probability of having 53 tuesdays or 53 wednesdays is\\
\solution
%\input{exemplar/11/16/3/18/main.tex}
\item There are 1000 sealed envelopes in a box, 10 of them contain a cash prize of
Rs 100 each, 100 of them contain a cash prize of Rs 50 each and 200 of them
contain a cash prize of Rs 10 each and rest do not contain any cash prize. If they
are well shuffled and an envelope is picked up out, what is the probability that it
contains no cash prize?\\
\solution
%\input{exemplar/10/13/3/34/main.tex}
\item 
A die is thrown and a card is selected at random from a deck of 52 playing cards. The probability of getting an even number on the die and a spade card.\\
\solution
%\input{exemplar/12/13/3/78/main.tex}
\item
If 4-digit numbers greater than 5,000 are randomly formed from the digits 0, 1, 3, 5, and 7, what is the probability of forming a number divisible by 5 when:
\begin{enumerate}
    \item The digits are repeated?
    \item The repetition of digits is not allowed?
\end{enumerate}
\solution
%\input{ncert/11/16/4/9/main.tex}
\item Consider the probability space $\brak{\Omega, \mathcal{G}, P}$ where $\Omega = [0,2]$ and $\mathcal{G} = \cbrak{\phi, \Omega, [0,1], (1,2]}$. Let $X$ and $Y$ be two functions on $\Omega$ defined as
\begin{align*}
    X(\omega) = 
    \begin{cases}
        1 & \text{if }\omega \in [0, 1]\\
        2 & \text{if }\omega \in (1, 2]
    \end{cases}
\end{align*}
and
\begin{align*}
    Y(\omega) = 
    \begin{cases}
        2 & \text{if }\omega \in [0, 1.5]\\
        3 & \text{if }\omega \in (1.5, 2].
    \end{cases}
\end{align*}
Then which one of the following statements is true?
\begin{enumerate}
    \item [(A)] $X$ is a random variable with respect to $\mathcal{G}$, but $Y$ is not a random variable with respect to $\mathcal{G}$.
    \item [(B)] $Y$ is a random variable with respect to $\mathcal{G}$, but $X$ is not a random variable with respect to $\mathcal{G}$.
    \item [(C)] Neither $X$ nor $Y$ is a random variable with respect to $\mathcal{G}$.
    \item [(D)] Both $X$ and $Y$ are random variables with respect to $\mathcal{G}$.
\end{enumerate} \hfill (GATE ST 2023)\\
\solution
%\input{gate/ST/2023/14/main.tex}
	\item  A die is loaded in such a way that each odd number is twice as likely to occur as
each even number. Find $P(G)$, where $G$ is the event that a number greater than
3 occurs on a single roll of the die.
\\
\solution
		%\input{exemplar/11/16/3/5/main.tex}
	\item All the jacks, queens and kings are removed from a deck of 52 playing cards. The remaining cards are well shuffled and then one card is drawn at random. Giving ace a value 1 similar value for other cards, find the probability that the card has a value 
		\begin{enumerate}
			\item 7
			\item greater than 7
			\item less than 7
		\end{enumerate}
		%\input{exemplar/10/13/3/30/main.tex}
  \item A Lot consists of 48 mobile phones of which 42 are good, 3 have only minor defects and 3 have major defects.Varnika will buy a phone if it is good but the trader will only buy a mobile if it has no major defects. One phone is selected at random from the lot. What is the probability that it is
\begin{enumerate}
	\item acceptable to Varnika?
            \item acceptable to the trader?
\end{enumerate}
\solution
	%\input{exemplar/10/13/3/40/main.tex}
 \item A student says that if you throw a die, it will show up 1 or not 1. Therefore, the probability of getting 1 and the probability of getting 'not 1' each is equal to $\frac{1}{2}$. Is this correct? Give reasons.\\
 \solution
        %\input{exemplar/10/13/2/9/main.tex}
   \item Four candidates A, B, C, D have ap-
plied for the assignment to coach a school cricket
team. If A is twice as likely to be selected as B, and
B and C are given about the same chance of being
selected, while C is twice as likely to be selected
as D, what are the probabilities that
\begin{enumerate}
\item C will be selected?
\item A will not be selected?
\end{enumerate}
	%\input{exemplar/11/16/3/9/main.tex}
 \item A bag contain 24 balls of which $x$ balls are red, $2x$ are white and $3x$ are blue. A ball is selected at random, What is the probability that it is
\begin{enumerate}[label=\alph*)]
\item not red ?
\item white ?
\end{enumerate}
%\input{exemplar/10/13/3/41/main.tex}
If the letters of the word ASSASSINATION are arranged at random. Find the Probability that
\begin{enumerate}[label=(\alph*)]
\item Four $S's$ come consecutively in the word
\item Two  $I's$ and two $N's$ come together
\item All $A's$ are not coming together
\item No two $A's$ are coming together
\end{enumerate}
%\input{exemplar/11/16/3/14/main.tex}
	\item One urn contains two black balls (labelled B1 and B2) and one white ball. A
	second urn contains one black ball and two white balls (labelled W1 and W2).
	Suppose the following experiment is performed. One of the two urns is chosen
	at random. Next a ball is randomly chosen from the urn. Then a second ball is
	chosen at random from the same urn without replacing the first ball.
	
	\begin{enumerate}
	\item What is the probability that two black balls are chosen?
	
	\item What is the probability that two balls of opposite colour are chosen?
	\end{enumerate}
	\solution
	%\input{exemplar/11/16/3/12/main1.tex}
\end{enumerate}

	\item 
The number lock of a suitcase has 4 wheels each labelled with ten digits i.e. from 0 to 9.The lock opens with a sequence of four digits with no repeats.What is the probability of a person getting the right sequence to open the suitcase.
\\
\solution
		%\begin{enumerate}[label=\thesection.\arabic*,ref=\thesection.\theenumi]
	\item One card is drawn from a well-shuffled deck of 52 cards. Find the probability of getting
\begin{enumerate}
\item A king of red colour 
\item A face card 
\item A red face card
\item The jack of hearts
\item A spade
\item The queen of diamonds

\end{enumerate}
\solution
		%\input{ncert/10/15/1/14/main.tex}
	\item Five cards—the ten, jack, queen, king and ace of diamonds, are well-shuffled with their face downwards. One card is then picked up at random.
\begin{enumerate}
\item
What is the probability that the card is the queen? 
\item
If the queen is drawn and put aside, what is the probability that the second card picked up is (a) an ace? (b) a queen?\\
\end{enumerate}
\solution
		%\input{ncert/10/15/1/15/defs.tex}
	\item A bag contains $5$ red balls and some blue balls. If the probability of drawing a blue ball is double that if a red ball, determine the number of blue balls in the bag. 
		\\
\solution
		%\input{ncert/10/15/2/3/defs.tex}
	\item A card is selected from a pack of 52 cards.
 \begin{enumerate}[label=(\alph*)] 
                 \item How many points are there in the sample space?
                 \item Calculate the probability that the card is an ace of spades.
                 \item Calculate the probability that the card is (i) an ace and (ii) black card.
 \end{enumerate}
\solution
		%\input{ncert/11/16/3/4/main.tex}
\item Four cards are drawn from a well-shuffled deck of 52 cards. What is the probability of obtaining 3 diamonds and one spade.
\\
\solution
		%\input{ncert/11/16/4/2/defs.tex}
\item In a certain lottery 10,000 tickets are sold and ten equal prizes are awarded. What is the probability of not getting a prize if you buy (a) one ticket (b) two tickets (c) 10 tickets ?	
\\
\solution
		%\input{ncert/11/16/4/4/defs.tex}
		%
\item 
Out of 100 students, two sections of 40 and 60 are formed. If you and your friend are among the 100 students, what is the probability that
\begin{enumerate}
\item you both enter the same section?
\item you both enter the different sections?
\end{enumerate}
\solution
		%\input{ncert/11/16/4/5/defs.tex}
	\item 
The number lock of a suitcase has 4 wheels each labelled with ten digits i.e. from 0 to 9.The lock opens with a sequence of four digits with no repeats.What is the probability of a person getting the right sequence to open the suitcase.
\\
\solution
		%\input{ncert/11/16/4/10/defs.tex}
		%
\item 
Two cards are drawn at random and without replacement from a pack of 52 playing cards. Find the probability that both the cards are black.
\\
\solution
		%\input{ncert/12/13/2/2/defs.tex}
		\item A box of oranges is inspected by examining three randomly selected oranges drawn without replacement. If all the three oranges are good, the box is approved for sale, otherwise, it is rejected. Find the probability that a box containing 15 oranges out of which 12 are good and 3 are bad ones will be approved for sale.
		\label{ncert/12/13/2/3/defs.tex}
		\item Two balls are drawn at random with replacement from a box containing 10 black and 8 red balls. Find the probability that
		\label{ncert/12/13/2/12}
\begin{enumerate}
\item both balls are red.
\item first ball is black and second is red.
\item one of them is black and other is red.
\end{enumerate}

\item In a hostel, 60\% of the students read Hindi newspaper, 40\% read English newspaper and 20\% read both Hindi and English newspapers. A student is selected at random.
		\label{ncert/12/13/2/15}
\begin{enumerate}
\item Find the probability that she reads neither Hindi nor English newspapers.
\item If she reads Hindi newspaper, find the probability that she reads English newspaper.
\item If she reads English newspaper, find the probability that she reads Hindi newspaper.\\
\end{enumerate}
\item The probability of obtaining an even prime number on each die, when a pair of dice is rolled is 
\begin{enumerate}
    \item $0$ 
    
    \item $\frac{1}{3}$ 
    
    \item $\frac{1}{12}$ 
    
    \item $\frac{1}{36}$ 
\end{enumerate}
\solution
		%\input{ncert/12/13/2/17/defs.tex}
	\item A bag contains 4 red and 4 black balls, another bag contains 2 red and 6 black balls. One of the two bags is selected at random and a ball is drawn from the bag which is found to be red. Find the probability that the ball is drawn from the first bag.
\\
\solution
		%\input{ncert/12/13/3/2/main.tex}
  \item
  Cards with numbers 2 to 101 are placed in a box. A card is selected at random.Find the probability that the card has
\begin{enumerate}[label=(\roman*)]
	\item an even number 
	\item a square number
\end{enumerate}
\solution
%\input{exemplar/10/13/3/32/main.tex}
\item
The king, queen and jack of clubs are removed from a deck of 52 playing cards and then well shuffled. Now one card is drawn at random from the remaining cards.  Determine the probability that the card is
\begin{enumerate}[label=(\roman*)]
\item a club
\item 10 of hearts
\end{enumerate}
\solution
%\input{exemplar/10/13/3/29/main.tex}
\item A team of medical students doing their internship have to assist during surgeries
at a city hospital. The probabilities of surgeries rated as very complex, complex,
routine, simple or very simple are respectively, 0.15, 0.20, 0.31, 0.26, .08. Find
the probabilities that a particular surgery will be rated
\begin{enumerate}
	\item complex or very complex;
	\item neither very complex nor very simple;
	\item routine or complex
	\item routine or simple
\end{enumerate}
\solution
%\input{exemplar/11/16/3/8(1)/main.tex}
\item A card is selected from a pack of 52 cards.
\begin{enumerate}[label=(\alph*)]
    \item How many points are there in the sample space?
    \item Calculate the probability that the card is an ace of spades.
    \item Calculate the probability that the card is (i) an ace and (ii) black card.
\end{enumerate}
\solution
%\input{exemplar/11/16/3/4/main2.tex}
\item The probability that a non leap year selected at random will contain 53 sundays.
\\
\solution
%\input{exemplar/10/13/1/19/main.tex}
\item One of the four persons John, Rita, Aslam or Gurpreet will be promoted next
month. Consequently the sample space consists of four elementary outcomes
S = {John promoted, Rita promoted, Aslam promoted, Gurpreet promoted}
You are told that the chances of John’s promotion is same as that of Gurpreet,
Rita’s chances of promotion are twice as likely as Johns. Aslam’s chances are
four times that of John.
\begin{enumerate}
	\item Determine
	\begin{enumerate}
		\item P (John promoted)
		\item P (Rita promoted)
		\item P (Aslam promoted)
		\item P (Gurpreet promoted)
	\end{enumerate}
	\item If A = {John promoted or Gurpreet promoted}, find P (A).
\end{enumerate}
\solution
%\input{exemplar/11/16/3/10/main.tex}
\item A card is drawn from a deck of 52 cards. Find the probability of getting a king or a heart or a red card.\\
\solution
%\input{exemplar/11/16/3/15/main.tex}
\item The probability that a student will pass his examination is 0.73, the probability of
the student getting a compartment is 0.13, and the probability that the student will
either pass or get compartment is 0.96. State True or False.\\
\solution
%\input{exemplar/11/16/3/31/main.tex}
\item A card is selected from a pack of 52 cards\\
\begin{enumerate}[label=(\alph*)]
\item How many points are there in the sample space?
\item Calculate the probability that the cards is an ace of spades.
\item Calculate the probability that the card is (i) an ace (ii)black card.\\
\end{enumerate}
%\input{ncert/11/16/3/4_1/Prob_4.tex}
\item In a non-leap year, the probability of having 53 tuesdays or 53 wednesdays is\\
\solution
%\input{exemplar/11/16/3/18/main.tex}
\item There are 1000 sealed envelopes in a box, 10 of them contain a cash prize of
Rs 100 each, 100 of them contain a cash prize of Rs 50 each and 200 of them
contain a cash prize of Rs 10 each and rest do not contain any cash prize. If they
are well shuffled and an envelope is picked up out, what is the probability that it
contains no cash prize?\\
\solution
%\input{exemplar/10/13/3/34/main.tex}
\item 
A die is thrown and a card is selected at random from a deck of 52 playing cards. The probability of getting an even number on the die and a spade card.\\
\solution
%\input{exemplar/12/13/3/78/main.tex}
\item
If 4-digit numbers greater than 5,000 are randomly formed from the digits 0, 1, 3, 5, and 7, what is the probability of forming a number divisible by 5 when:
\begin{enumerate}
    \item The digits are repeated?
    \item The repetition of digits is not allowed?
\end{enumerate}
\solution
%\input{ncert/11/16/4/9/main.tex}
\item Consider the probability space $\brak{\Omega, \mathcal{G}, P}$ where $\Omega = [0,2]$ and $\mathcal{G} = \cbrak{\phi, \Omega, [0,1], (1,2]}$. Let $X$ and $Y$ be two functions on $\Omega$ defined as
\begin{align*}
    X(\omega) = 
    \begin{cases}
        1 & \text{if }\omega \in [0, 1]\\
        2 & \text{if }\omega \in (1, 2]
    \end{cases}
\end{align*}
and
\begin{align*}
    Y(\omega) = 
    \begin{cases}
        2 & \text{if }\omega \in [0, 1.5]\\
        3 & \text{if }\omega \in (1.5, 2].
    \end{cases}
\end{align*}
Then which one of the following statements is true?
\begin{enumerate}
    \item [(A)] $X$ is a random variable with respect to $\mathcal{G}$, but $Y$ is not a random variable with respect to $\mathcal{G}$.
    \item [(B)] $Y$ is a random variable with respect to $\mathcal{G}$, but $X$ is not a random variable with respect to $\mathcal{G}$.
    \item [(C)] Neither $X$ nor $Y$ is a random variable with respect to $\mathcal{G}$.
    \item [(D)] Both $X$ and $Y$ are random variables with respect to $\mathcal{G}$.
\end{enumerate} \hfill (GATE ST 2023)\\
\solution
%\input{gate/ST/2023/14/main.tex}
	\item  A die is loaded in such a way that each odd number is twice as likely to occur as
each even number. Find $P(G)$, where $G$ is the event that a number greater than
3 occurs on a single roll of the die.
\\
\solution
		%\input{exemplar/11/16/3/5/main.tex}
	\item All the jacks, queens and kings are removed from a deck of 52 playing cards. The remaining cards are well shuffled and then one card is drawn at random. Giving ace a value 1 similar value for other cards, find the probability that the card has a value 
		\begin{enumerate}
			\item 7
			\item greater than 7
			\item less than 7
		\end{enumerate}
		%\input{exemplar/10/13/3/30/main.tex}
  \item A Lot consists of 48 mobile phones of which 42 are good, 3 have only minor defects and 3 have major defects.Varnika will buy a phone if it is good but the trader will only buy a mobile if it has no major defects. One phone is selected at random from the lot. What is the probability that it is
\begin{enumerate}
	\item acceptable to Varnika?
            \item acceptable to the trader?
\end{enumerate}
\solution
	%\input{exemplar/10/13/3/40/main.tex}
 \item A student says that if you throw a die, it will show up 1 or not 1. Therefore, the probability of getting 1 and the probability of getting 'not 1' each is equal to $\frac{1}{2}$. Is this correct? Give reasons.\\
 \solution
        %\input{exemplar/10/13/2/9/main.tex}
   \item Four candidates A, B, C, D have ap-
plied for the assignment to coach a school cricket
team. If A is twice as likely to be selected as B, and
B and C are given about the same chance of being
selected, while C is twice as likely to be selected
as D, what are the probabilities that
\begin{enumerate}
\item C will be selected?
\item A will not be selected?
\end{enumerate}
	%\input{exemplar/11/16/3/9/main.tex}
 \item A bag contain 24 balls of which $x$ balls are red, $2x$ are white and $3x$ are blue. A ball is selected at random, What is the probability that it is
\begin{enumerate}[label=\alph*)]
\item not red ?
\item white ?
\end{enumerate}
%\input{exemplar/10/13/3/41/main.tex}
If the letters of the word ASSASSINATION are arranged at random. Find the Probability that
\begin{enumerate}[label=(\alph*)]
\item Four $S's$ come consecutively in the word
\item Two  $I's$ and two $N's$ come together
\item All $A's$ are not coming together
\item No two $A's$ are coming together
\end{enumerate}
%\input{exemplar/11/16/3/14/main.tex}
	\item One urn contains two black balls (labelled B1 and B2) and one white ball. A
	second urn contains one black ball and two white balls (labelled W1 and W2).
	Suppose the following experiment is performed. One of the two urns is chosen
	at random. Next a ball is randomly chosen from the urn. Then a second ball is
	chosen at random from the same urn without replacing the first ball.
	
	\begin{enumerate}
	\item What is the probability that two black balls are chosen?
	
	\item What is the probability that two balls of opposite colour are chosen?
	\end{enumerate}
	\solution
	%\input{exemplar/11/16/3/12/main1.tex}
\end{enumerate}

		%
\item 
Two cards are drawn at random and without replacement from a pack of 52 playing cards. Find the probability that both the cards are black.
\\
\solution
		%\begin{enumerate}[label=\thesection.\arabic*,ref=\thesection.\theenumi]
	\item One card is drawn from a well-shuffled deck of 52 cards. Find the probability of getting
\begin{enumerate}
\item A king of red colour 
\item A face card 
\item A red face card
\item The jack of hearts
\item A spade
\item The queen of diamonds

\end{enumerate}
\solution
		%\input{ncert/10/15/1/14/main.tex}
	\item Five cards—the ten, jack, queen, king and ace of diamonds, are well-shuffled with their face downwards. One card is then picked up at random.
\begin{enumerate}
\item
What is the probability that the card is the queen? 
\item
If the queen is drawn and put aside, what is the probability that the second card picked up is (a) an ace? (b) a queen?\\
\end{enumerate}
\solution
		%\input{ncert/10/15/1/15/defs.tex}
	\item A bag contains $5$ red balls and some blue balls. If the probability of drawing a blue ball is double that if a red ball, determine the number of blue balls in the bag. 
		\\
\solution
		%\input{ncert/10/15/2/3/defs.tex}
	\item A card is selected from a pack of 52 cards.
 \begin{enumerate}[label=(\alph*)] 
                 \item How many points are there in the sample space?
                 \item Calculate the probability that the card is an ace of spades.
                 \item Calculate the probability that the card is (i) an ace and (ii) black card.
 \end{enumerate}
\solution
		%\input{ncert/11/16/3/4/main.tex}
\item Four cards are drawn from a well-shuffled deck of 52 cards. What is the probability of obtaining 3 diamonds and one spade.
\\
\solution
		%\input{ncert/11/16/4/2/defs.tex}
\item In a certain lottery 10,000 tickets are sold and ten equal prizes are awarded. What is the probability of not getting a prize if you buy (a) one ticket (b) two tickets (c) 10 tickets ?	
\\
\solution
		%\input{ncert/11/16/4/4/defs.tex}
		%
\item 
Out of 100 students, two sections of 40 and 60 are formed. If you and your friend are among the 100 students, what is the probability that
\begin{enumerate}
\item you both enter the same section?
\item you both enter the different sections?
\end{enumerate}
\solution
		%\input{ncert/11/16/4/5/defs.tex}
	\item 
The number lock of a suitcase has 4 wheels each labelled with ten digits i.e. from 0 to 9.The lock opens with a sequence of four digits with no repeats.What is the probability of a person getting the right sequence to open the suitcase.
\\
\solution
		%\input{ncert/11/16/4/10/defs.tex}
		%
\item 
Two cards are drawn at random and without replacement from a pack of 52 playing cards. Find the probability that both the cards are black.
\\
\solution
		%\input{ncert/12/13/2/2/defs.tex}
		\item A box of oranges is inspected by examining three randomly selected oranges drawn without replacement. If all the three oranges are good, the box is approved for sale, otherwise, it is rejected. Find the probability that a box containing 15 oranges out of which 12 are good and 3 are bad ones will be approved for sale.
		\label{ncert/12/13/2/3/defs.tex}
		\item Two balls are drawn at random with replacement from a box containing 10 black and 8 red balls. Find the probability that
		\label{ncert/12/13/2/12}
\begin{enumerate}
\item both balls are red.
\item first ball is black and second is red.
\item one of them is black and other is red.
\end{enumerate}

\item In a hostel, 60\% of the students read Hindi newspaper, 40\% read English newspaper and 20\% read both Hindi and English newspapers. A student is selected at random.
		\label{ncert/12/13/2/15}
\begin{enumerate}
\item Find the probability that she reads neither Hindi nor English newspapers.
\item If she reads Hindi newspaper, find the probability that she reads English newspaper.
\item If she reads English newspaper, find the probability that she reads Hindi newspaper.\\
\end{enumerate}
\item The probability of obtaining an even prime number on each die, when a pair of dice is rolled is 
\begin{enumerate}
    \item $0$ 
    
    \item $\frac{1}{3}$ 
    
    \item $\frac{1}{12}$ 
    
    \item $\frac{1}{36}$ 
\end{enumerate}
\solution
		%\input{ncert/12/13/2/17/defs.tex}
	\item A bag contains 4 red and 4 black balls, another bag contains 2 red and 6 black balls. One of the two bags is selected at random and a ball is drawn from the bag which is found to be red. Find the probability that the ball is drawn from the first bag.
\\
\solution
		%\input{ncert/12/13/3/2/main.tex}
  \item
  Cards with numbers 2 to 101 are placed in a box. A card is selected at random.Find the probability that the card has
\begin{enumerate}[label=(\roman*)]
	\item an even number 
	\item a square number
\end{enumerate}
\solution
%\input{exemplar/10/13/3/32/main.tex}
\item
The king, queen and jack of clubs are removed from a deck of 52 playing cards and then well shuffled. Now one card is drawn at random from the remaining cards.  Determine the probability that the card is
\begin{enumerate}[label=(\roman*)]
\item a club
\item 10 of hearts
\end{enumerate}
\solution
%\input{exemplar/10/13/3/29/main.tex}
\item A team of medical students doing their internship have to assist during surgeries
at a city hospital. The probabilities of surgeries rated as very complex, complex,
routine, simple or very simple are respectively, 0.15, 0.20, 0.31, 0.26, .08. Find
the probabilities that a particular surgery will be rated
\begin{enumerate}
	\item complex or very complex;
	\item neither very complex nor very simple;
	\item routine or complex
	\item routine or simple
\end{enumerate}
\solution
%\input{exemplar/11/16/3/8(1)/main.tex}
\item A card is selected from a pack of 52 cards.
\begin{enumerate}[label=(\alph*)]
    \item How many points are there in the sample space?
    \item Calculate the probability that the card is an ace of spades.
    \item Calculate the probability that the card is (i) an ace and (ii) black card.
\end{enumerate}
\solution
%\input{exemplar/11/16/3/4/main2.tex}
\item The probability that a non leap year selected at random will contain 53 sundays.
\\
\solution
%\input{exemplar/10/13/1/19/main.tex}
\item One of the four persons John, Rita, Aslam or Gurpreet will be promoted next
month. Consequently the sample space consists of four elementary outcomes
S = {John promoted, Rita promoted, Aslam promoted, Gurpreet promoted}
You are told that the chances of John’s promotion is same as that of Gurpreet,
Rita’s chances of promotion are twice as likely as Johns. Aslam’s chances are
four times that of John.
\begin{enumerate}
	\item Determine
	\begin{enumerate}
		\item P (John promoted)
		\item P (Rita promoted)
		\item P (Aslam promoted)
		\item P (Gurpreet promoted)
	\end{enumerate}
	\item If A = {John promoted or Gurpreet promoted}, find P (A).
\end{enumerate}
\solution
%\input{exemplar/11/16/3/10/main.tex}
\item A card is drawn from a deck of 52 cards. Find the probability of getting a king or a heart or a red card.\\
\solution
%\input{exemplar/11/16/3/15/main.tex}
\item The probability that a student will pass his examination is 0.73, the probability of
the student getting a compartment is 0.13, and the probability that the student will
either pass or get compartment is 0.96. State True or False.\\
\solution
%\input{exemplar/11/16/3/31/main.tex}
\item A card is selected from a pack of 52 cards\\
\begin{enumerate}[label=(\alph*)]
\item How many points are there in the sample space?
\item Calculate the probability that the cards is an ace of spades.
\item Calculate the probability that the card is (i) an ace (ii)black card.\\
\end{enumerate}
%\input{ncert/11/16/3/4_1/Prob_4.tex}
\item In a non-leap year, the probability of having 53 tuesdays or 53 wednesdays is\\
\solution
%\input{exemplar/11/16/3/18/main.tex}
\item There are 1000 sealed envelopes in a box, 10 of them contain a cash prize of
Rs 100 each, 100 of them contain a cash prize of Rs 50 each and 200 of them
contain a cash prize of Rs 10 each and rest do not contain any cash prize. If they
are well shuffled and an envelope is picked up out, what is the probability that it
contains no cash prize?\\
\solution
%\input{exemplar/10/13/3/34/main.tex}
\item 
A die is thrown and a card is selected at random from a deck of 52 playing cards. The probability of getting an even number on the die and a spade card.\\
\solution
%\input{exemplar/12/13/3/78/main.tex}
\item
If 4-digit numbers greater than 5,000 are randomly formed from the digits 0, 1, 3, 5, and 7, what is the probability of forming a number divisible by 5 when:
\begin{enumerate}
    \item The digits are repeated?
    \item The repetition of digits is not allowed?
\end{enumerate}
\solution
%\input{ncert/11/16/4/9/main.tex}
\item Consider the probability space $\brak{\Omega, \mathcal{G}, P}$ where $\Omega = [0,2]$ and $\mathcal{G} = \cbrak{\phi, \Omega, [0,1], (1,2]}$. Let $X$ and $Y$ be two functions on $\Omega$ defined as
\begin{align*}
    X(\omega) = 
    \begin{cases}
        1 & \text{if }\omega \in [0, 1]\\
        2 & \text{if }\omega \in (1, 2]
    \end{cases}
\end{align*}
and
\begin{align*}
    Y(\omega) = 
    \begin{cases}
        2 & \text{if }\omega \in [0, 1.5]\\
        3 & \text{if }\omega \in (1.5, 2].
    \end{cases}
\end{align*}
Then which one of the following statements is true?
\begin{enumerate}
    \item [(A)] $X$ is a random variable with respect to $\mathcal{G}$, but $Y$ is not a random variable with respect to $\mathcal{G}$.
    \item [(B)] $Y$ is a random variable with respect to $\mathcal{G}$, but $X$ is not a random variable with respect to $\mathcal{G}$.
    \item [(C)] Neither $X$ nor $Y$ is a random variable with respect to $\mathcal{G}$.
    \item [(D)] Both $X$ and $Y$ are random variables with respect to $\mathcal{G}$.
\end{enumerate} \hfill (GATE ST 2023)\\
\solution
%\input{gate/ST/2023/14/main.tex}
	\item  A die is loaded in such a way that each odd number is twice as likely to occur as
each even number. Find $P(G)$, where $G$ is the event that a number greater than
3 occurs on a single roll of the die.
\\
\solution
		%\input{exemplar/11/16/3/5/main.tex}
	\item All the jacks, queens and kings are removed from a deck of 52 playing cards. The remaining cards are well shuffled and then one card is drawn at random. Giving ace a value 1 similar value for other cards, find the probability that the card has a value 
		\begin{enumerate}
			\item 7
			\item greater than 7
			\item less than 7
		\end{enumerate}
		%\input{exemplar/10/13/3/30/main.tex}
  \item A Lot consists of 48 mobile phones of which 42 are good, 3 have only minor defects and 3 have major defects.Varnika will buy a phone if it is good but the trader will only buy a mobile if it has no major defects. One phone is selected at random from the lot. What is the probability that it is
\begin{enumerate}
	\item acceptable to Varnika?
            \item acceptable to the trader?
\end{enumerate}
\solution
	%\input{exemplar/10/13/3/40/main.tex}
 \item A student says that if you throw a die, it will show up 1 or not 1. Therefore, the probability of getting 1 and the probability of getting 'not 1' each is equal to $\frac{1}{2}$. Is this correct? Give reasons.\\
 \solution
        %\input{exemplar/10/13/2/9/main.tex}
   \item Four candidates A, B, C, D have ap-
plied for the assignment to coach a school cricket
team. If A is twice as likely to be selected as B, and
B and C are given about the same chance of being
selected, while C is twice as likely to be selected
as D, what are the probabilities that
\begin{enumerate}
\item C will be selected?
\item A will not be selected?
\end{enumerate}
	%\input{exemplar/11/16/3/9/main.tex}
 \item A bag contain 24 balls of which $x$ balls are red, $2x$ are white and $3x$ are blue. A ball is selected at random, What is the probability that it is
\begin{enumerate}[label=\alph*)]
\item not red ?
\item white ?
\end{enumerate}
%\input{exemplar/10/13/3/41/main.tex}
If the letters of the word ASSASSINATION are arranged at random. Find the Probability that
\begin{enumerate}[label=(\alph*)]
\item Four $S's$ come consecutively in the word
\item Two  $I's$ and two $N's$ come together
\item All $A's$ are not coming together
\item No two $A's$ are coming together
\end{enumerate}
%\input{exemplar/11/16/3/14/main.tex}
	\item One urn contains two black balls (labelled B1 and B2) and one white ball. A
	second urn contains one black ball and two white balls (labelled W1 and W2).
	Suppose the following experiment is performed. One of the two urns is chosen
	at random. Next a ball is randomly chosen from the urn. Then a second ball is
	chosen at random from the same urn without replacing the first ball.
	
	\begin{enumerate}
	\item What is the probability that two black balls are chosen?
	
	\item What is the probability that two balls of opposite colour are chosen?
	\end{enumerate}
	\solution
	%\input{exemplar/11/16/3/12/main1.tex}
\end{enumerate}

		\item A box of oranges is inspected by examining three randomly selected oranges drawn without replacement. If all the three oranges are good, the box is approved for sale, otherwise, it is rejected. Find the probability that a box containing 15 oranges out of which 12 are good and 3 are bad ones will be approved for sale.
		\label{ncert/12/13/2/3/defs.tex}
		\item Two balls are drawn at random with replacement from a box containing 10 black and 8 red balls. Find the probability that
		\label{ncert/12/13/2/12}
\begin{enumerate}
\item both balls are red.
\item first ball is black and second is red.
\item one of them is black and other is red.
\end{enumerate}

\item In a hostel, 60\% of the students read Hindi newspaper, 40\% read English newspaper and 20\% read both Hindi and English newspapers. A student is selected at random.
		\label{ncert/12/13/2/15}
\begin{enumerate}
\item Find the probability that she reads neither Hindi nor English newspapers.
\item If she reads Hindi newspaper, find the probability that she reads English newspaper.
\item If she reads English newspaper, find the probability that she reads Hindi newspaper.\\
\end{enumerate}
\item The probability of obtaining an even prime number on each die, when a pair of dice is rolled is 
\begin{enumerate}
    \item $0$ 
    
    \item $\frac{1}{3}$ 
    
    \item $\frac{1}{12}$ 
    
    \item $\frac{1}{36}$ 
\end{enumerate}
\solution
		%\begin{enumerate}[label=\thesection.\arabic*,ref=\thesection.\theenumi]
	\item One card is drawn from a well-shuffled deck of 52 cards. Find the probability of getting
\begin{enumerate}
\item A king of red colour 
\item A face card 
\item A red face card
\item The jack of hearts
\item A spade
\item The queen of diamonds

\end{enumerate}
\solution
		%\input{ncert/10/15/1/14/main.tex}
	\item Five cards—the ten, jack, queen, king and ace of diamonds, are well-shuffled with their face downwards. One card is then picked up at random.
\begin{enumerate}
\item
What is the probability that the card is the queen? 
\item
If the queen is drawn and put aside, what is the probability that the second card picked up is (a) an ace? (b) a queen?\\
\end{enumerate}
\solution
		%\input{ncert/10/15/1/15/defs.tex}
	\item A bag contains $5$ red balls and some blue balls. If the probability of drawing a blue ball is double that if a red ball, determine the number of blue balls in the bag. 
		\\
\solution
		%\input{ncert/10/15/2/3/defs.tex}
	\item A card is selected from a pack of 52 cards.
 \begin{enumerate}[label=(\alph*)] 
                 \item How many points are there in the sample space?
                 \item Calculate the probability that the card is an ace of spades.
                 \item Calculate the probability that the card is (i) an ace and (ii) black card.
 \end{enumerate}
\solution
		%\input{ncert/11/16/3/4/main.tex}
\item Four cards are drawn from a well-shuffled deck of 52 cards. What is the probability of obtaining 3 diamonds and one spade.
\\
\solution
		%\input{ncert/11/16/4/2/defs.tex}
\item In a certain lottery 10,000 tickets are sold and ten equal prizes are awarded. What is the probability of not getting a prize if you buy (a) one ticket (b) two tickets (c) 10 tickets ?	
\\
\solution
		%\input{ncert/11/16/4/4/defs.tex}
		%
\item 
Out of 100 students, two sections of 40 and 60 are formed. If you and your friend are among the 100 students, what is the probability that
\begin{enumerate}
\item you both enter the same section?
\item you both enter the different sections?
\end{enumerate}
\solution
		%\input{ncert/11/16/4/5/defs.tex}
	\item 
The number lock of a suitcase has 4 wheels each labelled with ten digits i.e. from 0 to 9.The lock opens with a sequence of four digits with no repeats.What is the probability of a person getting the right sequence to open the suitcase.
\\
\solution
		%\input{ncert/11/16/4/10/defs.tex}
		%
\item 
Two cards are drawn at random and without replacement from a pack of 52 playing cards. Find the probability that both the cards are black.
\\
\solution
		%\input{ncert/12/13/2/2/defs.tex}
		\item A box of oranges is inspected by examining three randomly selected oranges drawn without replacement. If all the three oranges are good, the box is approved for sale, otherwise, it is rejected. Find the probability that a box containing 15 oranges out of which 12 are good and 3 are bad ones will be approved for sale.
		\label{ncert/12/13/2/3/defs.tex}
		\item Two balls are drawn at random with replacement from a box containing 10 black and 8 red balls. Find the probability that
		\label{ncert/12/13/2/12}
\begin{enumerate}
\item both balls are red.
\item first ball is black and second is red.
\item one of them is black and other is red.
\end{enumerate}

\item In a hostel, 60\% of the students read Hindi newspaper, 40\% read English newspaper and 20\% read both Hindi and English newspapers. A student is selected at random.
		\label{ncert/12/13/2/15}
\begin{enumerate}
\item Find the probability that she reads neither Hindi nor English newspapers.
\item If she reads Hindi newspaper, find the probability that she reads English newspaper.
\item If she reads English newspaper, find the probability that she reads Hindi newspaper.\\
\end{enumerate}
\item The probability of obtaining an even prime number on each die, when a pair of dice is rolled is 
\begin{enumerate}
    \item $0$ 
    
    \item $\frac{1}{3}$ 
    
    \item $\frac{1}{12}$ 
    
    \item $\frac{1}{36}$ 
\end{enumerate}
\solution
		%\input{ncert/12/13/2/17/defs.tex}
	\item A bag contains 4 red and 4 black balls, another bag contains 2 red and 6 black balls. One of the two bags is selected at random and a ball is drawn from the bag which is found to be red. Find the probability that the ball is drawn from the first bag.
\\
\solution
		%\input{ncert/12/13/3/2/main.tex}
  \item
  Cards with numbers 2 to 101 are placed in a box. A card is selected at random.Find the probability that the card has
\begin{enumerate}[label=(\roman*)]
	\item an even number 
	\item a square number
\end{enumerate}
\solution
%\input{exemplar/10/13/3/32/main.tex}
\item
The king, queen and jack of clubs are removed from a deck of 52 playing cards and then well shuffled. Now one card is drawn at random from the remaining cards.  Determine the probability that the card is
\begin{enumerate}[label=(\roman*)]
\item a club
\item 10 of hearts
\end{enumerate}
\solution
%\input{exemplar/10/13/3/29/main.tex}
\item A team of medical students doing their internship have to assist during surgeries
at a city hospital. The probabilities of surgeries rated as very complex, complex,
routine, simple or very simple are respectively, 0.15, 0.20, 0.31, 0.26, .08. Find
the probabilities that a particular surgery will be rated
\begin{enumerate}
	\item complex or very complex;
	\item neither very complex nor very simple;
	\item routine or complex
	\item routine or simple
\end{enumerate}
\solution
%\input{exemplar/11/16/3/8(1)/main.tex}
\item A card is selected from a pack of 52 cards.
\begin{enumerate}[label=(\alph*)]
    \item How many points are there in the sample space?
    \item Calculate the probability that the card is an ace of spades.
    \item Calculate the probability that the card is (i) an ace and (ii) black card.
\end{enumerate}
\solution
%\input{exemplar/11/16/3/4/main2.tex}
\item The probability that a non leap year selected at random will contain 53 sundays.
\\
\solution
%\input{exemplar/10/13/1/19/main.tex}
\item One of the four persons John, Rita, Aslam or Gurpreet will be promoted next
month. Consequently the sample space consists of four elementary outcomes
S = {John promoted, Rita promoted, Aslam promoted, Gurpreet promoted}
You are told that the chances of John’s promotion is same as that of Gurpreet,
Rita’s chances of promotion are twice as likely as Johns. Aslam’s chances are
four times that of John.
\begin{enumerate}
	\item Determine
	\begin{enumerate}
		\item P (John promoted)
		\item P (Rita promoted)
		\item P (Aslam promoted)
		\item P (Gurpreet promoted)
	\end{enumerate}
	\item If A = {John promoted or Gurpreet promoted}, find P (A).
\end{enumerate}
\solution
%\input{exemplar/11/16/3/10/main.tex}
\item A card is drawn from a deck of 52 cards. Find the probability of getting a king or a heart or a red card.\\
\solution
%\input{exemplar/11/16/3/15/main.tex}
\item The probability that a student will pass his examination is 0.73, the probability of
the student getting a compartment is 0.13, and the probability that the student will
either pass or get compartment is 0.96. State True or False.\\
\solution
%\input{exemplar/11/16/3/31/main.tex}
\item A card is selected from a pack of 52 cards\\
\begin{enumerate}[label=(\alph*)]
\item How many points are there in the sample space?
\item Calculate the probability that the cards is an ace of spades.
\item Calculate the probability that the card is (i) an ace (ii)black card.\\
\end{enumerate}
%\input{ncert/11/16/3/4_1/Prob_4.tex}
\item In a non-leap year, the probability of having 53 tuesdays or 53 wednesdays is\\
\solution
%\input{exemplar/11/16/3/18/main.tex}
\item There are 1000 sealed envelopes in a box, 10 of them contain a cash prize of
Rs 100 each, 100 of them contain a cash prize of Rs 50 each and 200 of them
contain a cash prize of Rs 10 each and rest do not contain any cash prize. If they
are well shuffled and an envelope is picked up out, what is the probability that it
contains no cash prize?\\
\solution
%\input{exemplar/10/13/3/34/main.tex}
\item 
A die is thrown and a card is selected at random from a deck of 52 playing cards. The probability of getting an even number on the die and a spade card.\\
\solution
%\input{exemplar/12/13/3/78/main.tex}
\item
If 4-digit numbers greater than 5,000 are randomly formed from the digits 0, 1, 3, 5, and 7, what is the probability of forming a number divisible by 5 when:
\begin{enumerate}
    \item The digits are repeated?
    \item The repetition of digits is not allowed?
\end{enumerate}
\solution
%\input{ncert/11/16/4/9/main.tex}
\item Consider the probability space $\brak{\Omega, \mathcal{G}, P}$ where $\Omega = [0,2]$ and $\mathcal{G} = \cbrak{\phi, \Omega, [0,1], (1,2]}$. Let $X$ and $Y$ be two functions on $\Omega$ defined as
\begin{align*}
    X(\omega) = 
    \begin{cases}
        1 & \text{if }\omega \in [0, 1]\\
        2 & \text{if }\omega \in (1, 2]
    \end{cases}
\end{align*}
and
\begin{align*}
    Y(\omega) = 
    \begin{cases}
        2 & \text{if }\omega \in [0, 1.5]\\
        3 & \text{if }\omega \in (1.5, 2].
    \end{cases}
\end{align*}
Then which one of the following statements is true?
\begin{enumerate}
    \item [(A)] $X$ is a random variable with respect to $\mathcal{G}$, but $Y$ is not a random variable with respect to $\mathcal{G}$.
    \item [(B)] $Y$ is a random variable with respect to $\mathcal{G}$, but $X$ is not a random variable with respect to $\mathcal{G}$.
    \item [(C)] Neither $X$ nor $Y$ is a random variable with respect to $\mathcal{G}$.
    \item [(D)] Both $X$ and $Y$ are random variables with respect to $\mathcal{G}$.
\end{enumerate} \hfill (GATE ST 2023)\\
\solution
%\input{gate/ST/2023/14/main.tex}
	\item  A die is loaded in such a way that each odd number is twice as likely to occur as
each even number. Find $P(G)$, where $G$ is the event that a number greater than
3 occurs on a single roll of the die.
\\
\solution
		%\input{exemplar/11/16/3/5/main.tex}
	\item All the jacks, queens and kings are removed from a deck of 52 playing cards. The remaining cards are well shuffled and then one card is drawn at random. Giving ace a value 1 similar value for other cards, find the probability that the card has a value 
		\begin{enumerate}
			\item 7
			\item greater than 7
			\item less than 7
		\end{enumerate}
		%\input{exemplar/10/13/3/30/main.tex}
  \item A Lot consists of 48 mobile phones of which 42 are good, 3 have only minor defects and 3 have major defects.Varnika will buy a phone if it is good but the trader will only buy a mobile if it has no major defects. One phone is selected at random from the lot. What is the probability that it is
\begin{enumerate}
	\item acceptable to Varnika?
            \item acceptable to the trader?
\end{enumerate}
\solution
	%\input{exemplar/10/13/3/40/main.tex}
 \item A student says that if you throw a die, it will show up 1 or not 1. Therefore, the probability of getting 1 and the probability of getting 'not 1' each is equal to $\frac{1}{2}$. Is this correct? Give reasons.\\
 \solution
        %\input{exemplar/10/13/2/9/main.tex}
   \item Four candidates A, B, C, D have ap-
plied for the assignment to coach a school cricket
team. If A is twice as likely to be selected as B, and
B and C are given about the same chance of being
selected, while C is twice as likely to be selected
as D, what are the probabilities that
\begin{enumerate}
\item C will be selected?
\item A will not be selected?
\end{enumerate}
	%\input{exemplar/11/16/3/9/main.tex}
 \item A bag contain 24 balls of which $x$ balls are red, $2x$ are white and $3x$ are blue. A ball is selected at random, What is the probability that it is
\begin{enumerate}[label=\alph*)]
\item not red ?
\item white ?
\end{enumerate}
%\input{exemplar/10/13/3/41/main.tex}
If the letters of the word ASSASSINATION are arranged at random. Find the Probability that
\begin{enumerate}[label=(\alph*)]
\item Four $S's$ come consecutively in the word
\item Two  $I's$ and two $N's$ come together
\item All $A's$ are not coming together
\item No two $A's$ are coming together
\end{enumerate}
%\input{exemplar/11/16/3/14/main.tex}
	\item One urn contains two black balls (labelled B1 and B2) and one white ball. A
	second urn contains one black ball and two white balls (labelled W1 and W2).
	Suppose the following experiment is performed. One of the two urns is chosen
	at random. Next a ball is randomly chosen from the urn. Then a second ball is
	chosen at random from the same urn without replacing the first ball.
	
	\begin{enumerate}
	\item What is the probability that two black balls are chosen?
	
	\item What is the probability that two balls of opposite colour are chosen?
	\end{enumerate}
	\solution
	%\input{exemplar/11/16/3/12/main1.tex}
\end{enumerate}

	\item A bag contains 4 red and 4 black balls, another bag contains 2 red and 6 black balls. One of the two bags is selected at random and a ball is drawn from the bag which is found to be red. Find the probability that the ball is drawn from the first bag.
\\
\solution
		%\begin{table}[H]
	\centering
\begin{tabular}{|c|c|c|}
\hline
Random variable &Value &Definition\\ \hline
\multirow{3}{*}{X} &0 &Slips of Rs 1\\
&1 &Slips of Rs 5\\
&2 &Slips of Rs 13\\ \hline
\multirow{2}{*}{Y} &0 &Box A\\
&1 &Box B\\\hline
\end{tabular}
\caption{}
\label{tab:Distribution}
\end{table}
See \tabref{tab:Distribution}.
\begin{align}
p_{Y}\brak{k}= \begin{cases} 
      \frac{1}{3} & {k=0} \\
      \frac{2}{3 }& {k=1} 
   \end{cases}
   \\
p_{Y|X}\brak{0|0} = \frac{19}{25}\, 
p_{Y|X}\brak{0|1} = \frac{6}{25}\,
p_{Y|X}\brak{1|0} = \frac{45}{50}\,
p_{Y|X}\brak{1|2} = \frac{5}{50}
\end{align}
The desired probability is the probability that a slip drawn at random is marked other than Rs 1,
\begin{align}
&=1-p_X\brak{0}\\
&= p_X(1) + p_X(2)
\end{align}
Using Bayes theorem,
\begin{align}
&= p_Y\brak{0} \times \pr{Y=0 | X=1} + p_Y\brak{1} \times \pr{Y=1|X=2}\\
&=\frac{1}{3} \times \frac{6}{25} + \frac{2}{3} \times \frac{5}{50}\\
&=\frac{11}{75}
\end{align}

\newpage

%\tableofcontents

\bigskip

\renewcommand{\thefigure}{\theenumi}
\renewcommand{\thetable}{\theenumi}
%\renewcommand{\theequation}{\theenumi}

%\begin{abstract}
%%\boldmath
%In this letter, an algorithm for evaluating the exact analytical bit error rate  (BER)  for the piecewise linear (PL) combiner for  multiple relays is presented. Previous results were available only for upto three relays. The algorithm is unique in the sense that  the actual mathematical expressions, that are prohibitively large, need not be explicitly obtained. The diversity gain due to multiple relays is shown through plots of the analytical BER, well supported by simulations. 
%
%\end{abstract}
% IEEEtran.cls defaults to using nonbold math in the Abstract.
% This preserves the distinction between vectors and scalars. However,
% if the journal you are submitting to favors bold math in the abstract,
% then you can use LaTeX's standard command \boldmath at the very start
% of the abstract to achieve this. Many IEEE journals frown on math
% in the abstract anyway.

% Note that keywords are not normally used for peerreview papers.
%\begin{IEEEkeywords}
%Cooperative diversity, decode and forward, piecewise linear
%\end{IEEEkeywords}



% For peer review papers, you can put extra information on the cover
% page as needed:
% \ifCLASSOPTIONpeerreview
% \begin{center} \bfseries EDICS Category: 3-BBND \end{center}
% \fi
%
% For peerreview papers, this IEEEtran command inserts a page break and
% creates the second title. It will be ignored for other modes.
%\IEEEpeerreviewmaketitle




  \item
  Cards with numbers 2 to 101 are placed in a box. A card is selected at random.Find the probability that the card has
\begin{enumerate}[label=(\roman*)]
	\item an even number 
	\item a square number
\end{enumerate}
\solution
%\begin{table}[H]
	\centering
\begin{tabular}{|c|c|c|}
\hline
Random variable &Value &Definition\\ \hline
\multirow{3}{*}{X} &0 &Slips of Rs 1\\
&1 &Slips of Rs 5\\
&2 &Slips of Rs 13\\ \hline
\multirow{2}{*}{Y} &0 &Box A\\
&1 &Box B\\\hline
\end{tabular}
\caption{}
\label{tab:Distribution}
\end{table}
See \tabref{tab:Distribution}.
\begin{align}
p_{Y}\brak{k}= \begin{cases} 
      \frac{1}{3} & {k=0} \\
      \frac{2}{3 }& {k=1} 
   \end{cases}
   \\
p_{Y|X}\brak{0|0} = \frac{19}{25}\, 
p_{Y|X}\brak{0|1} = \frac{6}{25}\,
p_{Y|X}\brak{1|0} = \frac{45}{50}\,
p_{Y|X}\brak{1|2} = \frac{5}{50}
\end{align}
The desired probability is the probability that a slip drawn at random is marked other than Rs 1,
\begin{align}
&=1-p_X\brak{0}\\
&= p_X(1) + p_X(2)
\end{align}
Using Bayes theorem,
\begin{align}
&= p_Y\brak{0} \times \pr{Y=0 | X=1} + p_Y\brak{1} \times \pr{Y=1|X=2}\\
&=\frac{1}{3} \times \frac{6}{25} + \frac{2}{3} \times \frac{5}{50}\\
&=\frac{11}{75}
\end{align}

\newpage

%\tableofcontents

\bigskip

\renewcommand{\thefigure}{\theenumi}
\renewcommand{\thetable}{\theenumi}
%\renewcommand{\theequation}{\theenumi}

%\begin{abstract}
%%\boldmath
%In this letter, an algorithm for evaluating the exact analytical bit error rate  (BER)  for the piecewise linear (PL) combiner for  multiple relays is presented. Previous results were available only for upto three relays. The algorithm is unique in the sense that  the actual mathematical expressions, that are prohibitively large, need not be explicitly obtained. The diversity gain due to multiple relays is shown through plots of the analytical BER, well supported by simulations. 
%
%\end{abstract}
% IEEEtran.cls defaults to using nonbold math in the Abstract.
% This preserves the distinction between vectors and scalars. However,
% if the journal you are submitting to favors bold math in the abstract,
% then you can use LaTeX's standard command \boldmath at the very start
% of the abstract to achieve this. Many IEEE journals frown on math
% in the abstract anyway.

% Note that keywords are not normally used for peerreview papers.
%\begin{IEEEkeywords}
%Cooperative diversity, decode and forward, piecewise linear
%\end{IEEEkeywords}



% For peer review papers, you can put extra information on the cover
% page as needed:
% \ifCLASSOPTIONpeerreview
% \begin{center} \bfseries EDICS Category: 3-BBND \end{center}
% \fi
%
% For peerreview papers, this IEEEtran command inserts a page break and
% creates the second title. It will be ignored for other modes.
%\IEEEpeerreviewmaketitle




\item
The king, queen and jack of clubs are removed from a deck of 52 playing cards and then well shuffled. Now one card is drawn at random from the remaining cards.  Determine the probability that the card is
\begin{enumerate}[label=(\roman*)]
\item a club
\item 10 of hearts
\end{enumerate}
\solution
%\begin{table}[H]
	\centering
\begin{tabular}{|c|c|c|}
\hline
Random variable &Value &Definition\\ \hline
\multirow{3}{*}{X} &0 &Slips of Rs 1\\
&1 &Slips of Rs 5\\
&2 &Slips of Rs 13\\ \hline
\multirow{2}{*}{Y} &0 &Box A\\
&1 &Box B\\\hline
\end{tabular}
\caption{}
\label{tab:Distribution}
\end{table}
See \tabref{tab:Distribution}.
\begin{align}
p_{Y}\brak{k}= \begin{cases} 
      \frac{1}{3} & {k=0} \\
      \frac{2}{3 }& {k=1} 
   \end{cases}
   \\
p_{Y|X}\brak{0|0} = \frac{19}{25}\, 
p_{Y|X}\brak{0|1} = \frac{6}{25}\,
p_{Y|X}\brak{1|0} = \frac{45}{50}\,
p_{Y|X}\brak{1|2} = \frac{5}{50}
\end{align}
The desired probability is the probability that a slip drawn at random is marked other than Rs 1,
\begin{align}
&=1-p_X\brak{0}\\
&= p_X(1) + p_X(2)
\end{align}
Using Bayes theorem,
\begin{align}
&= p_Y\brak{0} \times \pr{Y=0 | X=1} + p_Y\brak{1} \times \pr{Y=1|X=2}\\
&=\frac{1}{3} \times \frac{6}{25} + \frac{2}{3} \times \frac{5}{50}\\
&=\frac{11}{75}
\end{align}

\newpage

%\tableofcontents

\bigskip

\renewcommand{\thefigure}{\theenumi}
\renewcommand{\thetable}{\theenumi}
%\renewcommand{\theequation}{\theenumi}

%\begin{abstract}
%%\boldmath
%In this letter, an algorithm for evaluating the exact analytical bit error rate  (BER)  for the piecewise linear (PL) combiner for  multiple relays is presented. Previous results were available only for upto three relays. The algorithm is unique in the sense that  the actual mathematical expressions, that are prohibitively large, need not be explicitly obtained. The diversity gain due to multiple relays is shown through plots of the analytical BER, well supported by simulations. 
%
%\end{abstract}
% IEEEtran.cls defaults to using nonbold math in the Abstract.
% This preserves the distinction between vectors and scalars. However,
% if the journal you are submitting to favors bold math in the abstract,
% then you can use LaTeX's standard command \boldmath at the very start
% of the abstract to achieve this. Many IEEE journals frown on math
% in the abstract anyway.

% Note that keywords are not normally used for peerreview papers.
%\begin{IEEEkeywords}
%Cooperative diversity, decode and forward, piecewise linear
%\end{IEEEkeywords}



% For peer review papers, you can put extra information on the cover
% page as needed:
% \ifCLASSOPTIONpeerreview
% \begin{center} \bfseries EDICS Category: 3-BBND \end{center}
% \fi
%
% For peerreview papers, this IEEEtran command inserts a page break and
% creates the second title. It will be ignored for other modes.
%\IEEEpeerreviewmaketitle




\item A team of medical students doing their internship have to assist during surgeries
at a city hospital. The probabilities of surgeries rated as very complex, complex,
routine, simple or very simple are respectively, 0.15, 0.20, 0.31, 0.26, .08. Find
the probabilities that a particular surgery will be rated
\begin{enumerate}
	\item complex or very complex;
	\item neither very complex nor very simple;
	\item routine or complex
	\item routine or simple
\end{enumerate}
\solution
%\begin{table}[H]
	\centering
\begin{tabular}{|c|c|c|}
\hline
Random variable &Value &Definition\\ \hline
\multirow{3}{*}{X} &0 &Slips of Rs 1\\
&1 &Slips of Rs 5\\
&2 &Slips of Rs 13\\ \hline
\multirow{2}{*}{Y} &0 &Box A\\
&1 &Box B\\\hline
\end{tabular}
\caption{}
\label{tab:Distribution}
\end{table}
See \tabref{tab:Distribution}.
\begin{align}
p_{Y}\brak{k}= \begin{cases} 
      \frac{1}{3} & {k=0} \\
      \frac{2}{3 }& {k=1} 
   \end{cases}
   \\
p_{Y|X}\brak{0|0} = \frac{19}{25}\, 
p_{Y|X}\brak{0|1} = \frac{6}{25}\,
p_{Y|X}\brak{1|0} = \frac{45}{50}\,
p_{Y|X}\brak{1|2} = \frac{5}{50}
\end{align}
The desired probability is the probability that a slip drawn at random is marked other than Rs 1,
\begin{align}
&=1-p_X\brak{0}\\
&= p_X(1) + p_X(2)
\end{align}
Using Bayes theorem,
\begin{align}
&= p_Y\brak{0} \times \pr{Y=0 | X=1} + p_Y\brak{1} \times \pr{Y=1|X=2}\\
&=\frac{1}{3} \times \frac{6}{25} + \frac{2}{3} \times \frac{5}{50}\\
&=\frac{11}{75}
\end{align}

\newpage

%\tableofcontents

\bigskip

\renewcommand{\thefigure}{\theenumi}
\renewcommand{\thetable}{\theenumi}
%\renewcommand{\theequation}{\theenumi}

%\begin{abstract}
%%\boldmath
%In this letter, an algorithm for evaluating the exact analytical bit error rate  (BER)  for the piecewise linear (PL) combiner for  multiple relays is presented. Previous results were available only for upto three relays. The algorithm is unique in the sense that  the actual mathematical expressions, that are prohibitively large, need not be explicitly obtained. The diversity gain due to multiple relays is shown through plots of the analytical BER, well supported by simulations. 
%
%\end{abstract}
% IEEEtran.cls defaults to using nonbold math in the Abstract.
% This preserves the distinction between vectors and scalars. However,
% if the journal you are submitting to favors bold math in the abstract,
% then you can use LaTeX's standard command \boldmath at the very start
% of the abstract to achieve this. Many IEEE journals frown on math
% in the abstract anyway.

% Note that keywords are not normally used for peerreview papers.
%\begin{IEEEkeywords}
%Cooperative diversity, decode and forward, piecewise linear
%\end{IEEEkeywords}



% For peer review papers, you can put extra information on the cover
% page as needed:
% \ifCLASSOPTIONpeerreview
% \begin{center} \bfseries EDICS Category: 3-BBND \end{center}
% \fi
%
% For peerreview papers, this IEEEtran command inserts a page break and
% creates the second title. It will be ignored for other modes.
%\IEEEpeerreviewmaketitle




\item A card is selected from a pack of 52 cards.
\begin{enumerate}[label=(\alph*)]
    \item How many points are there in the sample space?
    \item Calculate the probability that the card is an ace of spades.
    \item Calculate the probability that the card is (i) an ace and (ii) black card.
\end{enumerate}
\solution
%Let $X$ be an bernoulli rv defined as in \tabref{tab:exemplar/11/16/3/26}.  Then, 
\begin{equation}
    p =
        \frac{4}{11} 
\end{equation}
\begin{table}[H]
	\centering
	\input{exemplar/11/16/3/26/tables/Table2.tex}
	\caption{}
        \label{tab:exemplar/11/16/3/26}
\end{table}

\item The probability that a non leap year selected at random will contain 53 sundays.
\\
\solution
%\begin{table}[H]
	\centering
\begin{tabular}{|c|c|c|}
\hline
Random variable &Value &Definition\\ \hline
\multirow{3}{*}{X} &0 &Slips of Rs 1\\
&1 &Slips of Rs 5\\
&2 &Slips of Rs 13\\ \hline
\multirow{2}{*}{Y} &0 &Box A\\
&1 &Box B\\\hline
\end{tabular}
\caption{}
\label{tab:Distribution}
\end{table}
See \tabref{tab:Distribution}.
\begin{align}
p_{Y}\brak{k}= \begin{cases} 
      \frac{1}{3} & {k=0} \\
      \frac{2}{3 }& {k=1} 
   \end{cases}
   \\
p_{Y|X}\brak{0|0} = \frac{19}{25}\, 
p_{Y|X}\brak{0|1} = \frac{6}{25}\,
p_{Y|X}\brak{1|0} = \frac{45}{50}\,
p_{Y|X}\brak{1|2} = \frac{5}{50}
\end{align}
The desired probability is the probability that a slip drawn at random is marked other than Rs 1,
\begin{align}
&=1-p_X\brak{0}\\
&= p_X(1) + p_X(2)
\end{align}
Using Bayes theorem,
\begin{align}
&= p_Y\brak{0} \times \pr{Y=0 | X=1} + p_Y\brak{1} \times \pr{Y=1|X=2}\\
&=\frac{1}{3} \times \frac{6}{25} + \frac{2}{3} \times \frac{5}{50}\\
&=\frac{11}{75}
\end{align}

\newpage

%\tableofcontents

\bigskip

\renewcommand{\thefigure}{\theenumi}
\renewcommand{\thetable}{\theenumi}
%\renewcommand{\theequation}{\theenumi}

%\begin{abstract}
%%\boldmath
%In this letter, an algorithm for evaluating the exact analytical bit error rate  (BER)  for the piecewise linear (PL) combiner for  multiple relays is presented. Previous results were available only for upto three relays. The algorithm is unique in the sense that  the actual mathematical expressions, that are prohibitively large, need not be explicitly obtained. The diversity gain due to multiple relays is shown through plots of the analytical BER, well supported by simulations. 
%
%\end{abstract}
% IEEEtran.cls defaults to using nonbold math in the Abstract.
% This preserves the distinction between vectors and scalars. However,
% if the journal you are submitting to favors bold math in the abstract,
% then you can use LaTeX's standard command \boldmath at the very start
% of the abstract to achieve this. Many IEEE journals frown on math
% in the abstract anyway.

% Note that keywords are not normally used for peerreview papers.
%\begin{IEEEkeywords}
%Cooperative diversity, decode and forward, piecewise linear
%\end{IEEEkeywords}



% For peer review papers, you can put extra information on the cover
% page as needed:
% \ifCLASSOPTIONpeerreview
% \begin{center} \bfseries EDICS Category: 3-BBND \end{center}
% \fi
%
% For peerreview papers, this IEEEtran command inserts a page break and
% creates the second title. It will be ignored for other modes.
%\IEEEpeerreviewmaketitle




\item One of the four persons John, Rita, Aslam or Gurpreet will be promoted next
month. Consequently the sample space consists of four elementary outcomes
S = {John promoted, Rita promoted, Aslam promoted, Gurpreet promoted}
You are told that the chances of John’s promotion is same as that of Gurpreet,
Rita’s chances of promotion are twice as likely as Johns. Aslam’s chances are
four times that of John.
\begin{enumerate}
	\item Determine
	\begin{enumerate}
		\item P (John promoted)
		\item P (Rita promoted)
		\item P (Aslam promoted)
		\item P (Gurpreet promoted)
	\end{enumerate}
	\item If A = {John promoted or Gurpreet promoted}, find P (A).
\end{enumerate}
\solution
%\begin{table}[H]
	\centering
\begin{tabular}{|c|c|c|}
\hline
Random variable &Value &Definition\\ \hline
\multirow{3}{*}{X} &0 &Slips of Rs 1\\
&1 &Slips of Rs 5\\
&2 &Slips of Rs 13\\ \hline
\multirow{2}{*}{Y} &0 &Box A\\
&1 &Box B\\\hline
\end{tabular}
\caption{}
\label{tab:Distribution}
\end{table}
See \tabref{tab:Distribution}.
\begin{align}
p_{Y}\brak{k}= \begin{cases} 
      \frac{1}{3} & {k=0} \\
      \frac{2}{3 }& {k=1} 
   \end{cases}
   \\
p_{Y|X}\brak{0|0} = \frac{19}{25}\, 
p_{Y|X}\brak{0|1} = \frac{6}{25}\,
p_{Y|X}\brak{1|0} = \frac{45}{50}\,
p_{Y|X}\brak{1|2} = \frac{5}{50}
\end{align}
The desired probability is the probability that a slip drawn at random is marked other than Rs 1,
\begin{align}
&=1-p_X\brak{0}\\
&= p_X(1) + p_X(2)
\end{align}
Using Bayes theorem,
\begin{align}
&= p_Y\brak{0} \times \pr{Y=0 | X=1} + p_Y\brak{1} \times \pr{Y=1|X=2}\\
&=\frac{1}{3} \times \frac{6}{25} + \frac{2}{3} \times \frac{5}{50}\\
&=\frac{11}{75}
\end{align}

\newpage

%\tableofcontents

\bigskip

\renewcommand{\thefigure}{\theenumi}
\renewcommand{\thetable}{\theenumi}
%\renewcommand{\theequation}{\theenumi}

%\begin{abstract}
%%\boldmath
%In this letter, an algorithm for evaluating the exact analytical bit error rate  (BER)  for the piecewise linear (PL) combiner for  multiple relays is presented. Previous results were available only for upto three relays. The algorithm is unique in the sense that  the actual mathematical expressions, that are prohibitively large, need not be explicitly obtained. The diversity gain due to multiple relays is shown through plots of the analytical BER, well supported by simulations. 
%
%\end{abstract}
% IEEEtran.cls defaults to using nonbold math in the Abstract.
% This preserves the distinction between vectors and scalars. However,
% if the journal you are submitting to favors bold math in the abstract,
% then you can use LaTeX's standard command \boldmath at the very start
% of the abstract to achieve this. Many IEEE journals frown on math
% in the abstract anyway.

% Note that keywords are not normally used for peerreview papers.
%\begin{IEEEkeywords}
%Cooperative diversity, decode and forward, piecewise linear
%\end{IEEEkeywords}



% For peer review papers, you can put extra information on the cover
% page as needed:
% \ifCLASSOPTIONpeerreview
% \begin{center} \bfseries EDICS Category: 3-BBND \end{center}
% \fi
%
% For peerreview papers, this IEEEtran command inserts a page break and
% creates the second title. It will be ignored for other modes.
%\IEEEpeerreviewmaketitle




\item A card is drawn from a deck of 52 cards. Find the probability of getting a king or a heart or a red card.\\
\solution
%\begin{table}[H]
	\centering
\begin{tabular}{|c|c|c|}
\hline
Random variable &Value &Definition\\ \hline
\multirow{3}{*}{X} &0 &Slips of Rs 1\\
&1 &Slips of Rs 5\\
&2 &Slips of Rs 13\\ \hline
\multirow{2}{*}{Y} &0 &Box A\\
&1 &Box B\\\hline
\end{tabular}
\caption{}
\label{tab:Distribution}
\end{table}
See \tabref{tab:Distribution}.
\begin{align}
p_{Y}\brak{k}= \begin{cases} 
      \frac{1}{3} & {k=0} \\
      \frac{2}{3 }& {k=1} 
   \end{cases}
   \\
p_{Y|X}\brak{0|0} = \frac{19}{25}\, 
p_{Y|X}\brak{0|1} = \frac{6}{25}\,
p_{Y|X}\brak{1|0} = \frac{45}{50}\,
p_{Y|X}\brak{1|2} = \frac{5}{50}
\end{align}
The desired probability is the probability that a slip drawn at random is marked other than Rs 1,
\begin{align}
&=1-p_X\brak{0}\\
&= p_X(1) + p_X(2)
\end{align}
Using Bayes theorem,
\begin{align}
&= p_Y\brak{0} \times \pr{Y=0 | X=1} + p_Y\brak{1} \times \pr{Y=1|X=2}\\
&=\frac{1}{3} \times \frac{6}{25} + \frac{2}{3} \times \frac{5}{50}\\
&=\frac{11}{75}
\end{align}

\newpage

%\tableofcontents

\bigskip

\renewcommand{\thefigure}{\theenumi}
\renewcommand{\thetable}{\theenumi}
%\renewcommand{\theequation}{\theenumi}

%\begin{abstract}
%%\boldmath
%In this letter, an algorithm for evaluating the exact analytical bit error rate  (BER)  for the piecewise linear (PL) combiner for  multiple relays is presented. Previous results were available only for upto three relays. The algorithm is unique in the sense that  the actual mathematical expressions, that are prohibitively large, need not be explicitly obtained. The diversity gain due to multiple relays is shown through plots of the analytical BER, well supported by simulations. 
%
%\end{abstract}
% IEEEtran.cls defaults to using nonbold math in the Abstract.
% This preserves the distinction between vectors and scalars. However,
% if the journal you are submitting to favors bold math in the abstract,
% then you can use LaTeX's standard command \boldmath at the very start
% of the abstract to achieve this. Many IEEE journals frown on math
% in the abstract anyway.

% Note that keywords are not normally used for peerreview papers.
%\begin{IEEEkeywords}
%Cooperative diversity, decode and forward, piecewise linear
%\end{IEEEkeywords}



% For peer review papers, you can put extra information on the cover
% page as needed:
% \ifCLASSOPTIONpeerreview
% \begin{center} \bfseries EDICS Category: 3-BBND \end{center}
% \fi
%
% For peerreview papers, this IEEEtran command inserts a page break and
% creates the second title. It will be ignored for other modes.
%\IEEEpeerreviewmaketitle




\item The probability that a student will pass his examination is 0.73, the probability of
the student getting a compartment is 0.13, and the probability that the student will
either pass or get compartment is 0.96. State True or False.\\
\solution
%\begin{table}[H]
	\centering
\begin{tabular}{|c|c|c|}
\hline
Random variable &Value &Definition\\ \hline
\multirow{3}{*}{X} &0 &Slips of Rs 1\\
&1 &Slips of Rs 5\\
&2 &Slips of Rs 13\\ \hline
\multirow{2}{*}{Y} &0 &Box A\\
&1 &Box B\\\hline
\end{tabular}
\caption{}
\label{tab:Distribution}
\end{table}
See \tabref{tab:Distribution}.
\begin{align}
p_{Y}\brak{k}= \begin{cases} 
      \frac{1}{3} & {k=0} \\
      \frac{2}{3 }& {k=1} 
   \end{cases}
   \\
p_{Y|X}\brak{0|0} = \frac{19}{25}\, 
p_{Y|X}\brak{0|1} = \frac{6}{25}\,
p_{Y|X}\brak{1|0} = \frac{45}{50}\,
p_{Y|X}\brak{1|2} = \frac{5}{50}
\end{align}
The desired probability is the probability that a slip drawn at random is marked other than Rs 1,
\begin{align}
&=1-p_X\brak{0}\\
&= p_X(1) + p_X(2)
\end{align}
Using Bayes theorem,
\begin{align}
&= p_Y\brak{0} \times \pr{Y=0 | X=1} + p_Y\brak{1} \times \pr{Y=1|X=2}\\
&=\frac{1}{3} \times \frac{6}{25} + \frac{2}{3} \times \frac{5}{50}\\
&=\frac{11}{75}
\end{align}

\newpage

%\tableofcontents

\bigskip

\renewcommand{\thefigure}{\theenumi}
\renewcommand{\thetable}{\theenumi}
%\renewcommand{\theequation}{\theenumi}

%\begin{abstract}
%%\boldmath
%In this letter, an algorithm for evaluating the exact analytical bit error rate  (BER)  for the piecewise linear (PL) combiner for  multiple relays is presented. Previous results were available only for upto three relays. The algorithm is unique in the sense that  the actual mathematical expressions, that are prohibitively large, need not be explicitly obtained. The diversity gain due to multiple relays is shown through plots of the analytical BER, well supported by simulations. 
%
%\end{abstract}
% IEEEtran.cls defaults to using nonbold math in the Abstract.
% This preserves the distinction between vectors and scalars. However,
% if the journal you are submitting to favors bold math in the abstract,
% then you can use LaTeX's standard command \boldmath at the very start
% of the abstract to achieve this. Many IEEE journals frown on math
% in the abstract anyway.

% Note that keywords are not normally used for peerreview papers.
%\begin{IEEEkeywords}
%Cooperative diversity, decode and forward, piecewise linear
%\end{IEEEkeywords}



% For peer review papers, you can put extra information on the cover
% page as needed:
% \ifCLASSOPTIONpeerreview
% \begin{center} \bfseries EDICS Category: 3-BBND \end{center}
% \fi
%
% For peerreview papers, this IEEEtran command inserts a page break and
% creates the second title. It will be ignored for other modes.
%\IEEEpeerreviewmaketitle




\item A card is selected from a pack of 52 cards\\
\begin{enumerate}[label=(\alph*)]
\item How many points are there in the sample space?
\item Calculate the probability that the cards is an ace of spades.
\item Calculate the probability that the card is (i) an ace (ii)black card.\\
\end{enumerate}
%\input{ncert/11/16/3/4_1/Prob_4.tex}
\item In a non-leap year, the probability of having 53 tuesdays or 53 wednesdays is\\
\solution
%A non-leap year has a total of 365 days, and a week has 7 days.\\
So it can be expressed as 
\begin{align}
365\text{days} &=52\times 7+1 \text{day}
\end{align}
$\implies$ 52 tuesdays or wednesdays\\
Random variable X denotes the days of a week
\begin{align}
p_X\brak{k}&=\frac{1}{7}; \quad \brak{1<k<7}
\end{align}
So the probability of extra day being tuesday or wednesday is
\begin{align}
p_X\brak{3}+p_X\brak{4}&=\frac{1}{7}+\frac{1}{7}=\frac{2}{7}
\end{align}



\item There are 1000 sealed envelopes in a box, 10 of them contain a cash prize of
Rs 100 each, 100 of them contain a cash prize of Rs 50 each and 200 of them
contain a cash prize of Rs 10 each and rest do not contain any cash prize. If they
are well shuffled and an envelope is picked up out, what is the probability that it
contains no cash prize?\\
\solution
%\begin{table}[H]
	\centering
\begin{tabular}{|c|c|c|}
\hline
Random variable &Value &Definition\\ \hline
\multirow{3}{*}{X} &0 &Slips of Rs 1\\
&1 &Slips of Rs 5\\
&2 &Slips of Rs 13\\ \hline
\multirow{2}{*}{Y} &0 &Box A\\
&1 &Box B\\\hline
\end{tabular}
\caption{}
\label{tab:Distribution}
\end{table}
See \tabref{tab:Distribution}.
\begin{align}
p_{Y}\brak{k}= \begin{cases} 
      \frac{1}{3} & {k=0} \\
      \frac{2}{3 }& {k=1} 
   \end{cases}
   \\
p_{Y|X}\brak{0|0} = \frac{19}{25}\, 
p_{Y|X}\brak{0|1} = \frac{6}{25}\,
p_{Y|X}\brak{1|0} = \frac{45}{50}\,
p_{Y|X}\brak{1|2} = \frac{5}{50}
\end{align}
The desired probability is the probability that a slip drawn at random is marked other than Rs 1,
\begin{align}
&=1-p_X\brak{0}\\
&= p_X(1) + p_X(2)
\end{align}
Using Bayes theorem,
\begin{align}
&= p_Y\brak{0} \times \pr{Y=0 | X=1} + p_Y\brak{1} \times \pr{Y=1|X=2}\\
&=\frac{1}{3} \times \frac{6}{25} + \frac{2}{3} \times \frac{5}{50}\\
&=\frac{11}{75}
\end{align}

\newpage

%\tableofcontents

\bigskip

\renewcommand{\thefigure}{\theenumi}
\renewcommand{\thetable}{\theenumi}
%\renewcommand{\theequation}{\theenumi}

%\begin{abstract}
%%\boldmath
%In this letter, an algorithm for evaluating the exact analytical bit error rate  (BER)  for the piecewise linear (PL) combiner for  multiple relays is presented. Previous results were available only for upto three relays. The algorithm is unique in the sense that  the actual mathematical expressions, that are prohibitively large, need not be explicitly obtained. The diversity gain due to multiple relays is shown through plots of the analytical BER, well supported by simulations. 
%
%\end{abstract}
% IEEEtran.cls defaults to using nonbold math in the Abstract.
% This preserves the distinction between vectors and scalars. However,
% if the journal you are submitting to favors bold math in the abstract,
% then you can use LaTeX's standard command \boldmath at the very start
% of the abstract to achieve this. Many IEEE journals frown on math
% in the abstract anyway.

% Note that keywords are not normally used for peerreview papers.
%\begin{IEEEkeywords}
%Cooperative diversity, decode and forward, piecewise linear
%\end{IEEEkeywords}



% For peer review papers, you can put extra information on the cover
% page as needed:
% \ifCLASSOPTIONpeerreview
% \begin{center} \bfseries EDICS Category: 3-BBND \end{center}
% \fi
%
% For peerreview papers, this IEEEtran command inserts a page break and
% creates the second title. It will be ignored for other modes.
%\IEEEpeerreviewmaketitle




\item 
A die is thrown and a card is selected at random from a deck of 52 playing cards. The probability of getting an even number on the die and a spade card.\\
\solution
%\begin{table}[H]
	\centering
\begin{tabular}{|c|c|c|}
\hline
Random variable &Value &Definition\\ \hline
\multirow{3}{*}{X} &0 &Slips of Rs 1\\
&1 &Slips of Rs 5\\
&2 &Slips of Rs 13\\ \hline
\multirow{2}{*}{Y} &0 &Box A\\
&1 &Box B\\\hline
\end{tabular}
\caption{}
\label{tab:Distribution}
\end{table}
See \tabref{tab:Distribution}.
\begin{align}
p_{Y}\brak{k}= \begin{cases} 
      \frac{1}{3} & {k=0} \\
      \frac{2}{3 }& {k=1} 
   \end{cases}
   \\
p_{Y|X}\brak{0|0} = \frac{19}{25}\, 
p_{Y|X}\brak{0|1} = \frac{6}{25}\,
p_{Y|X}\brak{1|0} = \frac{45}{50}\,
p_{Y|X}\brak{1|2} = \frac{5}{50}
\end{align}
The desired probability is the probability that a slip drawn at random is marked other than Rs 1,
\begin{align}
&=1-p_X\brak{0}\\
&= p_X(1) + p_X(2)
\end{align}
Using Bayes theorem,
\begin{align}
&= p_Y\brak{0} \times \pr{Y=0 | X=1} + p_Y\brak{1} \times \pr{Y=1|X=2}\\
&=\frac{1}{3} \times \frac{6}{25} + \frac{2}{3} \times \frac{5}{50}\\
&=\frac{11}{75}
\end{align}

\newpage

%\tableofcontents

\bigskip

\renewcommand{\thefigure}{\theenumi}
\renewcommand{\thetable}{\theenumi}
%\renewcommand{\theequation}{\theenumi}

%\begin{abstract}
%%\boldmath
%In this letter, an algorithm for evaluating the exact analytical bit error rate  (BER)  for the piecewise linear (PL) combiner for  multiple relays is presented. Previous results were available only for upto three relays. The algorithm is unique in the sense that  the actual mathematical expressions, that are prohibitively large, need not be explicitly obtained. The diversity gain due to multiple relays is shown through plots of the analytical BER, well supported by simulations. 
%
%\end{abstract}
% IEEEtran.cls defaults to using nonbold math in the Abstract.
% This preserves the distinction between vectors and scalars. However,
% if the journal you are submitting to favors bold math in the abstract,
% then you can use LaTeX's standard command \boldmath at the very start
% of the abstract to achieve this. Many IEEE journals frown on math
% in the abstract anyway.

% Note that keywords are not normally used for peerreview papers.
%\begin{IEEEkeywords}
%Cooperative diversity, decode and forward, piecewise linear
%\end{IEEEkeywords}



% For peer review papers, you can put extra information on the cover
% page as needed:
% \ifCLASSOPTIONpeerreview
% \begin{center} \bfseries EDICS Category: 3-BBND \end{center}
% \fi
%
% For peerreview papers, this IEEEtran command inserts a page break and
% creates the second title. It will be ignored for other modes.
%\IEEEpeerreviewmaketitle




\item
If 4-digit numbers greater than 5,000 are randomly formed from the digits 0, 1, 3, 5, and 7, what is the probability of forming a number divisible by 5 when:
\begin{enumerate}
    \item The digits are repeated?
    \item The repetition of digits is not allowed?
\end{enumerate}
\solution
%\begin{table}[H]
	\centering
\begin{tabular}{|c|c|c|}
\hline
Random variable &Value &Definition\\ \hline
\multirow{3}{*}{X} &0 &Slips of Rs 1\\
&1 &Slips of Rs 5\\
&2 &Slips of Rs 13\\ \hline
\multirow{2}{*}{Y} &0 &Box A\\
&1 &Box B\\\hline
\end{tabular}
\caption{}
\label{tab:Distribution}
\end{table}
See \tabref{tab:Distribution}.
\begin{align}
p_{Y}\brak{k}= \begin{cases} 
      \frac{1}{3} & {k=0} \\
      \frac{2}{3 }& {k=1} 
   \end{cases}
   \\
p_{Y|X}\brak{0|0} = \frac{19}{25}\, 
p_{Y|X}\brak{0|1} = \frac{6}{25}\,
p_{Y|X}\brak{1|0} = \frac{45}{50}\,
p_{Y|X}\brak{1|2} = \frac{5}{50}
\end{align}
The desired probability is the probability that a slip drawn at random is marked other than Rs 1,
\begin{align}
&=1-p_X\brak{0}\\
&= p_X(1) + p_X(2)
\end{align}
Using Bayes theorem,
\begin{align}
&= p_Y\brak{0} \times \pr{Y=0 | X=1} + p_Y\brak{1} \times \pr{Y=1|X=2}\\
&=\frac{1}{3} \times \frac{6}{25} + \frac{2}{3} \times \frac{5}{50}\\
&=\frac{11}{75}
\end{align}

\newpage

%\tableofcontents

\bigskip

\renewcommand{\thefigure}{\theenumi}
\renewcommand{\thetable}{\theenumi}
%\renewcommand{\theequation}{\theenumi}

%\begin{abstract}
%%\boldmath
%In this letter, an algorithm for evaluating the exact analytical bit error rate  (BER)  for the piecewise linear (PL) combiner for  multiple relays is presented. Previous results were available only for upto three relays. The algorithm is unique in the sense that  the actual mathematical expressions, that are prohibitively large, need not be explicitly obtained. The diversity gain due to multiple relays is shown through plots of the analytical BER, well supported by simulations. 
%
%\end{abstract}
% IEEEtran.cls defaults to using nonbold math in the Abstract.
% This preserves the distinction between vectors and scalars. However,
% if the journal you are submitting to favors bold math in the abstract,
% then you can use LaTeX's standard command \boldmath at the very start
% of the abstract to achieve this. Many IEEE journals frown on math
% in the abstract anyway.

% Note that keywords are not normally used for peerreview papers.
%\begin{IEEEkeywords}
%Cooperative diversity, decode and forward, piecewise linear
%\end{IEEEkeywords}



% For peer review papers, you can put extra information on the cover
% page as needed:
% \ifCLASSOPTIONpeerreview
% \begin{center} \bfseries EDICS Category: 3-BBND \end{center}
% \fi
%
% For peerreview papers, this IEEEtran command inserts a page break and
% creates the second title. It will be ignored for other modes.
%\IEEEpeerreviewmaketitle




\item Consider the probability space $\brak{\Omega, \mathcal{G}, P}$ where $\Omega = [0,2]$ and $\mathcal{G} = \cbrak{\phi, \Omega, [0,1], (1,2]}$. Let $X$ and $Y$ be two functions on $\Omega$ defined as
\begin{align*}
    X(\omega) = 
    \begin{cases}
        1 & \text{if }\omega \in [0, 1]\\
        2 & \text{if }\omega \in (1, 2]
    \end{cases}
\end{align*}
and
\begin{align*}
    Y(\omega) = 
    \begin{cases}
        2 & \text{if }\omega \in [0, 1.5]\\
        3 & \text{if }\omega \in (1.5, 2].
    \end{cases}
\end{align*}
Then which one of the following statements is true?
\begin{enumerate}
    \item [(A)] $X$ is a random variable with respect to $\mathcal{G}$, but $Y$ is not a random variable with respect to $\mathcal{G}$.
    \item [(B)] $Y$ is a random variable with respect to $\mathcal{G}$, but $X$ is not a random variable with respect to $\mathcal{G}$.
    \item [(C)] Neither $X$ nor $Y$ is a random variable with respect to $\mathcal{G}$.
    \item [(D)] Both $X$ and $Y$ are random variables with respect to $\mathcal{G}$.
\end{enumerate} \hfill (GATE ST 2023)\\
\solution
%\begin{table}[H]
	\centering
\begin{tabular}{|c|c|c|}
\hline
Random variable &Value &Definition\\ \hline
\multirow{3}{*}{X} &0 &Slips of Rs 1\\
&1 &Slips of Rs 5\\
&2 &Slips of Rs 13\\ \hline
\multirow{2}{*}{Y} &0 &Box A\\
&1 &Box B\\\hline
\end{tabular}
\caption{}
\label{tab:Distribution}
\end{table}
See \tabref{tab:Distribution}.
\begin{align}
p_{Y}\brak{k}= \begin{cases} 
      \frac{1}{3} & {k=0} \\
      \frac{2}{3 }& {k=1} 
   \end{cases}
   \\
p_{Y|X}\brak{0|0} = \frac{19}{25}\, 
p_{Y|X}\brak{0|1} = \frac{6}{25}\,
p_{Y|X}\brak{1|0} = \frac{45}{50}\,
p_{Y|X}\brak{1|2} = \frac{5}{50}
\end{align}
The desired probability is the probability that a slip drawn at random is marked other than Rs 1,
\begin{align}
&=1-p_X\brak{0}\\
&= p_X(1) + p_X(2)
\end{align}
Using Bayes theorem,
\begin{align}
&= p_Y\brak{0} \times \pr{Y=0 | X=1} + p_Y\brak{1} \times \pr{Y=1|X=2}\\
&=\frac{1}{3} \times \frac{6}{25} + \frac{2}{3} \times \frac{5}{50}\\
&=\frac{11}{75}
\end{align}

\newpage

%\tableofcontents

\bigskip

\renewcommand{\thefigure}{\theenumi}
\renewcommand{\thetable}{\theenumi}
%\renewcommand{\theequation}{\theenumi}

%\begin{abstract}
%%\boldmath
%In this letter, an algorithm for evaluating the exact analytical bit error rate  (BER)  for the piecewise linear (PL) combiner for  multiple relays is presented. Previous results were available only for upto three relays. The algorithm is unique in the sense that  the actual mathematical expressions, that are prohibitively large, need not be explicitly obtained. The diversity gain due to multiple relays is shown through plots of the analytical BER, well supported by simulations. 
%
%\end{abstract}
% IEEEtran.cls defaults to using nonbold math in the Abstract.
% This preserves the distinction between vectors and scalars. However,
% if the journal you are submitting to favors bold math in the abstract,
% then you can use LaTeX's standard command \boldmath at the very start
% of the abstract to achieve this. Many IEEE journals frown on math
% in the abstract anyway.

% Note that keywords are not normally used for peerreview papers.
%\begin{IEEEkeywords}
%Cooperative diversity, decode and forward, piecewise linear
%\end{IEEEkeywords}



% For peer review papers, you can put extra information on the cover
% page as needed:
% \ifCLASSOPTIONpeerreview
% \begin{center} \bfseries EDICS Category: 3-BBND \end{center}
% \fi
%
% For peerreview papers, this IEEEtran command inserts a page break and
% creates the second title. It will be ignored for other modes.
%\IEEEpeerreviewmaketitle




	\item  A die is loaded in such a way that each odd number is twice as likely to occur as
each even number. Find $P(G)$, where $G$ is the event that a number greater than
3 occurs on a single roll of the die.
\\
\solution
		%\begin{table}[H]
	\centering
\begin{tabular}{|c|c|c|}
\hline
Random variable &Value &Definition\\ \hline
\multirow{3}{*}{X} &0 &Slips of Rs 1\\
&1 &Slips of Rs 5\\
&2 &Slips of Rs 13\\ \hline
\multirow{2}{*}{Y} &0 &Box A\\
&1 &Box B\\\hline
\end{tabular}
\caption{}
\label{tab:Distribution}
\end{table}
See \tabref{tab:Distribution}.
\begin{align}
p_{Y}\brak{k}= \begin{cases} 
      \frac{1}{3} & {k=0} \\
      \frac{2}{3 }& {k=1} 
   \end{cases}
   \\
p_{Y|X}\brak{0|0} = \frac{19}{25}\, 
p_{Y|X}\brak{0|1} = \frac{6}{25}\,
p_{Y|X}\brak{1|0} = \frac{45}{50}\,
p_{Y|X}\brak{1|2} = \frac{5}{50}
\end{align}
The desired probability is the probability that a slip drawn at random is marked other than Rs 1,
\begin{align}
&=1-p_X\brak{0}\\
&= p_X(1) + p_X(2)
\end{align}
Using Bayes theorem,
\begin{align}
&= p_Y\brak{0} \times \pr{Y=0 | X=1} + p_Y\brak{1} \times \pr{Y=1|X=2}\\
&=\frac{1}{3} \times \frac{6}{25} + \frac{2}{3} \times \frac{5}{50}\\
&=\frac{11}{75}
\end{align}

\newpage

%\tableofcontents

\bigskip

\renewcommand{\thefigure}{\theenumi}
\renewcommand{\thetable}{\theenumi}
%\renewcommand{\theequation}{\theenumi}

%\begin{abstract}
%%\boldmath
%In this letter, an algorithm for evaluating the exact analytical bit error rate  (BER)  for the piecewise linear (PL) combiner for  multiple relays is presented. Previous results were available only for upto three relays. The algorithm is unique in the sense that  the actual mathematical expressions, that are prohibitively large, need not be explicitly obtained. The diversity gain due to multiple relays is shown through plots of the analytical BER, well supported by simulations. 
%
%\end{abstract}
% IEEEtran.cls defaults to using nonbold math in the Abstract.
% This preserves the distinction between vectors and scalars. However,
% if the journal you are submitting to favors bold math in the abstract,
% then you can use LaTeX's standard command \boldmath at the very start
% of the abstract to achieve this. Many IEEE journals frown on math
% in the abstract anyway.

% Note that keywords are not normally used for peerreview papers.
%\begin{IEEEkeywords}
%Cooperative diversity, decode and forward, piecewise linear
%\end{IEEEkeywords}



% For peer review papers, you can put extra information on the cover
% page as needed:
% \ifCLASSOPTIONpeerreview
% \begin{center} \bfseries EDICS Category: 3-BBND \end{center}
% \fi
%
% For peerreview papers, this IEEEtran command inserts a page break and
% creates the second title. It will be ignored for other modes.
%\IEEEpeerreviewmaketitle




	\item All the jacks, queens and kings are removed from a deck of 52 playing cards. The remaining cards are well shuffled and then one card is drawn at random. Giving ace a value 1 similar value for other cards, find the probability that the card has a value 
		\begin{enumerate}
			\item 7
			\item greater than 7
			\item less than 7
		\end{enumerate}
		%Number of cards left after removing all jacks, queens and kings 
\begin{align}
N	= 52 - 4\times 3
	= 40
\end{align}
%\begin{table}[H]
%\def\arraystretch{1.2}
%\begin{tabular}{|c|c|c|}
%\hline
%	\textbf{Parameter} &\textbf{Value} &\textbf{Description}\\ \hline
%	$X$ &1-10 &Represents the value of the card picked \\ \hline
%\end{tabular}
%\end{table}
Let $1 \le X \le 10$ be the value of the card picked.  Then,
\begin{align}
	p_X(k) &= \Pr(X=k)\ \forall\ 1 \leq k \leq 10\\
	&= \frac{4\times 1}{40}\\
	&= \frac{1}{10}\\
	\therefore p_X(k) &= 
	\begin{cases}
		\frac{1}{10} & 1 \leq k \leq 10\\
		0 & \text{otherwise}
	\end{cases}
\end{align}
and
\begin{align}
	F_{X}(k) &= \sum_{m=0}^{k}p_{X}(m) \quad 1 \leq k \leq 10\\
	&= \frac{k}{10}\\
	\therefore F_{X}(k) &= 
	\begin{cases}
		0 & k \leq 0\\
		\frac{k}{10} & 1\leq k \leq 10\\
		1 & k > 10 
	\end{cases}
\end{align}
\begin{enumerate}
	\item Probability that card has value equal to 7 is
		\begin{align}
			 p_{X}(7)
			= \frac{1}{10}
		\end{align}
	\item Probability that card has value greater than 7 is
		\begin{align}
			1 - F_X(7)
			&= 1 - \frac{7}{10}
			\\
			&= \frac{3}{10}
		\end{align}
	\item Probability that card has value less than 7 is
		\begin{align}
			 F_{X}(6)
			=\frac{6}{10}
		\end{align}
\end{enumerate}

  \item A Lot consists of 48 mobile phones of which 42 are good, 3 have only minor defects and 3 have major defects.Varnika will buy a phone if it is good but the trader will only buy a mobile if it has no major defects. One phone is selected at random from the lot. What is the probability that it is
\begin{enumerate}
	\item acceptable to Varnika?
            \item acceptable to the trader?
\end{enumerate}
\solution
	%\begin{table}[H]
	\centering
\begin{tabular}{|c|c|c|}
\hline
Random variable &Value &Definition\\ \hline
\multirow{3}{*}{X} &0 &Slips of Rs 1\\
&1 &Slips of Rs 5\\
&2 &Slips of Rs 13\\ \hline
\multirow{2}{*}{Y} &0 &Box A\\
&1 &Box B\\\hline
\end{tabular}
\caption{}
\label{tab:Distribution}
\end{table}
See \tabref{tab:Distribution}.
\begin{align}
p_{Y}\brak{k}= \begin{cases} 
      \frac{1}{3} & {k=0} \\
      \frac{2}{3 }& {k=1} 
   \end{cases}
   \\
p_{Y|X}\brak{0|0} = \frac{19}{25}\, 
p_{Y|X}\brak{0|1} = \frac{6}{25}\,
p_{Y|X}\brak{1|0} = \frac{45}{50}\,
p_{Y|X}\brak{1|2} = \frac{5}{50}
\end{align}
The desired probability is the probability that a slip drawn at random is marked other than Rs 1,
\begin{align}
&=1-p_X\brak{0}\\
&= p_X(1) + p_X(2)
\end{align}
Using Bayes theorem,
\begin{align}
&= p_Y\brak{0} \times \pr{Y=0 | X=1} + p_Y\brak{1} \times \pr{Y=1|X=2}\\
&=\frac{1}{3} \times \frac{6}{25} + \frac{2}{3} \times \frac{5}{50}\\
&=\frac{11}{75}
\end{align}

\newpage

%\tableofcontents

\bigskip

\renewcommand{\thefigure}{\theenumi}
\renewcommand{\thetable}{\theenumi}
%\renewcommand{\theequation}{\theenumi}

%\begin{abstract}
%%\boldmath
%In this letter, an algorithm for evaluating the exact analytical bit error rate  (BER)  for the piecewise linear (PL) combiner for  multiple relays is presented. Previous results were available only for upto three relays. The algorithm is unique in the sense that  the actual mathematical expressions, that are prohibitively large, need not be explicitly obtained. The diversity gain due to multiple relays is shown through plots of the analytical BER, well supported by simulations. 
%
%\end{abstract}
% IEEEtran.cls defaults to using nonbold math in the Abstract.
% This preserves the distinction between vectors and scalars. However,
% if the journal you are submitting to favors bold math in the abstract,
% then you can use LaTeX's standard command \boldmath at the very start
% of the abstract to achieve this. Many IEEE journals frown on math
% in the abstract anyway.

% Note that keywords are not normally used for peerreview papers.
%\begin{IEEEkeywords}
%Cooperative diversity, decode and forward, piecewise linear
%\end{IEEEkeywords}



% For peer review papers, you can put extra information on the cover
% page as needed:
% \ifCLASSOPTIONpeerreview
% \begin{center} \bfseries EDICS Category: 3-BBND \end{center}
% \fi
%
% For peerreview papers, this IEEEtran command inserts a page break and
% creates the second title. It will be ignored for other modes.
%\IEEEpeerreviewmaketitle




 \item A student says that if you throw a die, it will show up 1 or not 1. Therefore, the probability of getting 1 and the probability of getting 'not 1' each is equal to $\frac{1}{2}$. Is this correct? Give reasons.\\
 \solution
        %\begin{table}[H]
	\centering
\begin{tabular}{|c|c|c|}
\hline
Random variable &Value &Definition\\ \hline
\multirow{3}{*}{X} &0 &Slips of Rs 1\\
&1 &Slips of Rs 5\\
&2 &Slips of Rs 13\\ \hline
\multirow{2}{*}{Y} &0 &Box A\\
&1 &Box B\\\hline
\end{tabular}
\caption{}
\label{tab:Distribution}
\end{table}
See \tabref{tab:Distribution}.
\begin{align}
p_{Y}\brak{k}= \begin{cases} 
      \frac{1}{3} & {k=0} \\
      \frac{2}{3 }& {k=1} 
   \end{cases}
   \\
p_{Y|X}\brak{0|0} = \frac{19}{25}\, 
p_{Y|X}\brak{0|1} = \frac{6}{25}\,
p_{Y|X}\brak{1|0} = \frac{45}{50}\,
p_{Y|X}\brak{1|2} = \frac{5}{50}
\end{align}
The desired probability is the probability that a slip drawn at random is marked other than Rs 1,
\begin{align}
&=1-p_X\brak{0}\\
&= p_X(1) + p_X(2)
\end{align}
Using Bayes theorem,
\begin{align}
&= p_Y\brak{0} \times \pr{Y=0 | X=1} + p_Y\brak{1} \times \pr{Y=1|X=2}\\
&=\frac{1}{3} \times \frac{6}{25} + \frac{2}{3} \times \frac{5}{50}\\
&=\frac{11}{75}
\end{align}

\newpage

%\tableofcontents

\bigskip

\renewcommand{\thefigure}{\theenumi}
\renewcommand{\thetable}{\theenumi}
%\renewcommand{\theequation}{\theenumi}

%\begin{abstract}
%%\boldmath
%In this letter, an algorithm for evaluating the exact analytical bit error rate  (BER)  for the piecewise linear (PL) combiner for  multiple relays is presented. Previous results were available only for upto three relays. The algorithm is unique in the sense that  the actual mathematical expressions, that are prohibitively large, need not be explicitly obtained. The diversity gain due to multiple relays is shown through plots of the analytical BER, well supported by simulations. 
%
%\end{abstract}
% IEEEtran.cls defaults to using nonbold math in the Abstract.
% This preserves the distinction between vectors and scalars. However,
% if the journal you are submitting to favors bold math in the abstract,
% then you can use LaTeX's standard command \boldmath at the very start
% of the abstract to achieve this. Many IEEE journals frown on math
% in the abstract anyway.

% Note that keywords are not normally used for peerreview papers.
%\begin{IEEEkeywords}
%Cooperative diversity, decode and forward, piecewise linear
%\end{IEEEkeywords}



% For peer review papers, you can put extra information on the cover
% page as needed:
% \ifCLASSOPTIONpeerreview
% \begin{center} \bfseries EDICS Category: 3-BBND \end{center}
% \fi
%
% For peerreview papers, this IEEEtran command inserts a page break and
% creates the second title. It will be ignored for other modes.
%\IEEEpeerreviewmaketitle




   \item Four candidates A, B, C, D have ap-
plied for the assignment to coach a school cricket
team. If A is twice as likely to be selected as B, and
B and C are given about the same chance of being
selected, while C is twice as likely to be selected
as D, what are the probabilities that
\begin{enumerate}
\item C will be selected?
\item A will not be selected?
\end{enumerate}
	%\begin{table}[H]
	\centering
\begin{tabular}{|c|c|c|}
\hline
Random variable &Value &Definition\\ \hline
\multirow{3}{*}{X} &0 &Slips of Rs 1\\
&1 &Slips of Rs 5\\
&2 &Slips of Rs 13\\ \hline
\multirow{2}{*}{Y} &0 &Box A\\
&1 &Box B\\\hline
\end{tabular}
\caption{}
\label{tab:Distribution}
\end{table}
See \tabref{tab:Distribution}.
\begin{align}
p_{Y}\brak{k}= \begin{cases} 
      \frac{1}{3} & {k=0} \\
      \frac{2}{3 }& {k=1} 
   \end{cases}
   \\
p_{Y|X}\brak{0|0} = \frac{19}{25}\, 
p_{Y|X}\brak{0|1} = \frac{6}{25}\,
p_{Y|X}\brak{1|0} = \frac{45}{50}\,
p_{Y|X}\brak{1|2} = \frac{5}{50}
\end{align}
The desired probability is the probability that a slip drawn at random is marked other than Rs 1,
\begin{align}
&=1-p_X\brak{0}\\
&= p_X(1) + p_X(2)
\end{align}
Using Bayes theorem,
\begin{align}
&= p_Y\brak{0} \times \pr{Y=0 | X=1} + p_Y\brak{1} \times \pr{Y=1|X=2}\\
&=\frac{1}{3} \times \frac{6}{25} + \frac{2}{3} \times \frac{5}{50}\\
&=\frac{11}{75}
\end{align}

\newpage

%\tableofcontents

\bigskip

\renewcommand{\thefigure}{\theenumi}
\renewcommand{\thetable}{\theenumi}
%\renewcommand{\theequation}{\theenumi}

%\begin{abstract}
%%\boldmath
%In this letter, an algorithm for evaluating the exact analytical bit error rate  (BER)  for the piecewise linear (PL) combiner for  multiple relays is presented. Previous results were available only for upto three relays. The algorithm is unique in the sense that  the actual mathematical expressions, that are prohibitively large, need not be explicitly obtained. The diversity gain due to multiple relays is shown through plots of the analytical BER, well supported by simulations. 
%
%\end{abstract}
% IEEEtran.cls defaults to using nonbold math in the Abstract.
% This preserves the distinction between vectors and scalars. However,
% if the journal you are submitting to favors bold math in the abstract,
% then you can use LaTeX's standard command \boldmath at the very start
% of the abstract to achieve this. Many IEEE journals frown on math
% in the abstract anyway.

% Note that keywords are not normally used for peerreview papers.
%\begin{IEEEkeywords}
%Cooperative diversity, decode and forward, piecewise linear
%\end{IEEEkeywords}



% For peer review papers, you can put extra information on the cover
% page as needed:
% \ifCLASSOPTIONpeerreview
% \begin{center} \bfseries EDICS Category: 3-BBND \end{center}
% \fi
%
% For peerreview papers, this IEEEtran command inserts a page break and
% creates the second title. It will be ignored for other modes.
%\IEEEpeerreviewmaketitle




 \item A bag contain 24 balls of which $x$ balls are red, $2x$ are white and $3x$ are blue. A ball is selected at random, What is the probability that it is
\begin{enumerate}[label=\alph*)]
\item not red ?
\item white ?
\end{enumerate}
%\begin{table}[H]
	\centering
\begin{tabular}{|c|c|c|}
\hline
Random variable &Value &Definition\\ \hline
\multirow{3}{*}{X} &0 &Slips of Rs 1\\
&1 &Slips of Rs 5\\
&2 &Slips of Rs 13\\ \hline
\multirow{2}{*}{Y} &0 &Box A\\
&1 &Box B\\\hline
\end{tabular}
\caption{}
\label{tab:Distribution}
\end{table}
See \tabref{tab:Distribution}.
\begin{align}
p_{Y}\brak{k}= \begin{cases} 
      \frac{1}{3} & {k=0} \\
      \frac{2}{3 }& {k=1} 
   \end{cases}
   \\
p_{Y|X}\brak{0|0} = \frac{19}{25}\, 
p_{Y|X}\brak{0|1} = \frac{6}{25}\,
p_{Y|X}\brak{1|0} = \frac{45}{50}\,
p_{Y|X}\brak{1|2} = \frac{5}{50}
\end{align}
The desired probability is the probability that a slip drawn at random is marked other than Rs 1,
\begin{align}
&=1-p_X\brak{0}\\
&= p_X(1) + p_X(2)
\end{align}
Using Bayes theorem,
\begin{align}
&= p_Y\brak{0} \times \pr{Y=0 | X=1} + p_Y\brak{1} \times \pr{Y=1|X=2}\\
&=\frac{1}{3} \times \frac{6}{25} + \frac{2}{3} \times \frac{5}{50}\\
&=\frac{11}{75}
\end{align}

\newpage

%\tableofcontents

\bigskip

\renewcommand{\thefigure}{\theenumi}
\renewcommand{\thetable}{\theenumi}
%\renewcommand{\theequation}{\theenumi}

%\begin{abstract}
%%\boldmath
%In this letter, an algorithm for evaluating the exact analytical bit error rate  (BER)  for the piecewise linear (PL) combiner for  multiple relays is presented. Previous results were available only for upto three relays. The algorithm is unique in the sense that  the actual mathematical expressions, that are prohibitively large, need not be explicitly obtained. The diversity gain due to multiple relays is shown through plots of the analytical BER, well supported by simulations. 
%
%\end{abstract}
% IEEEtran.cls defaults to using nonbold math in the Abstract.
% This preserves the distinction between vectors and scalars. However,
% if the journal you are submitting to favors bold math in the abstract,
% then you can use LaTeX's standard command \boldmath at the very start
% of the abstract to achieve this. Many IEEE journals frown on math
% in the abstract anyway.

% Note that keywords are not normally used for peerreview papers.
%\begin{IEEEkeywords}
%Cooperative diversity, decode and forward, piecewise linear
%\end{IEEEkeywords}



% For peer review papers, you can put extra information on the cover
% page as needed:
% \ifCLASSOPTIONpeerreview
% \begin{center} \bfseries EDICS Category: 3-BBND \end{center}
% \fi
%
% For peerreview papers, this IEEEtran command inserts a page break and
% creates the second title. It will be ignored for other modes.
%\IEEEpeerreviewmaketitle




If the letters of the word ASSASSINATION are arranged at random. Find the Probability that
\begin{enumerate}[label=(\alph*)]
\item Four $S's$ come consecutively in the word
\item Two  $I's$ and two $N's$ come together
\item All $A's$ are not coming together
\item No two $A's$ are coming together
\end{enumerate}
%\begin{table}[H]
	\centering
\begin{tabular}{|c|c|c|}
\hline
Random variable &Value &Definition\\ \hline
\multirow{3}{*}{X} &0 &Slips of Rs 1\\
&1 &Slips of Rs 5\\
&2 &Slips of Rs 13\\ \hline
\multirow{2}{*}{Y} &0 &Box A\\
&1 &Box B\\\hline
\end{tabular}
\caption{}
\label{tab:Distribution}
\end{table}
See \tabref{tab:Distribution}.
\begin{align}
p_{Y}\brak{k}= \begin{cases} 
      \frac{1}{3} & {k=0} \\
      \frac{2}{3 }& {k=1} 
   \end{cases}
   \\
p_{Y|X}\brak{0|0} = \frac{19}{25}\, 
p_{Y|X}\brak{0|1} = \frac{6}{25}\,
p_{Y|X}\brak{1|0} = \frac{45}{50}\,
p_{Y|X}\brak{1|2} = \frac{5}{50}
\end{align}
The desired probability is the probability that a slip drawn at random is marked other than Rs 1,
\begin{align}
&=1-p_X\brak{0}\\
&= p_X(1) + p_X(2)
\end{align}
Using Bayes theorem,
\begin{align}
&= p_Y\brak{0} \times \pr{Y=0 | X=1} + p_Y\brak{1} \times \pr{Y=1|X=2}\\
&=\frac{1}{3} \times \frac{6}{25} + \frac{2}{3} \times \frac{5}{50}\\
&=\frac{11}{75}
\end{align}

\newpage

%\tableofcontents

\bigskip

\renewcommand{\thefigure}{\theenumi}
\renewcommand{\thetable}{\theenumi}
%\renewcommand{\theequation}{\theenumi}

%\begin{abstract}
%%\boldmath
%In this letter, an algorithm for evaluating the exact analytical bit error rate  (BER)  for the piecewise linear (PL) combiner for  multiple relays is presented. Previous results were available only for upto three relays. The algorithm is unique in the sense that  the actual mathematical expressions, that are prohibitively large, need not be explicitly obtained. The diversity gain due to multiple relays is shown through plots of the analytical BER, well supported by simulations. 
%
%\end{abstract}
% IEEEtran.cls defaults to using nonbold math in the Abstract.
% This preserves the distinction between vectors and scalars. However,
% if the journal you are submitting to favors bold math in the abstract,
% then you can use LaTeX's standard command \boldmath at the very start
% of the abstract to achieve this. Many IEEE journals frown on math
% in the abstract anyway.

% Note that keywords are not normally used for peerreview papers.
%\begin{IEEEkeywords}
%Cooperative diversity, decode and forward, piecewise linear
%\end{IEEEkeywords}



% For peer review papers, you can put extra information on the cover
% page as needed:
% \ifCLASSOPTIONpeerreview
% \begin{center} \bfseries EDICS Category: 3-BBND \end{center}
% \fi
%
% For peerreview papers, this IEEEtran command inserts a page break and
% creates the second title. It will be ignored for other modes.
%\IEEEpeerreviewmaketitle




	\item One urn contains two black balls (labelled B1 and B2) and one white ball. A
	second urn contains one black ball and two white balls (labelled W1 and W2).
	Suppose the following experiment is performed. One of the two urns is chosen
	at random. Next a ball is randomly chosen from the urn. Then a second ball is
	chosen at random from the same urn without replacing the first ball.
	
	\begin{enumerate}
	\item What is the probability that two black balls are chosen?
	
	\item What is the probability that two balls of opposite colour are chosen?
	\end{enumerate}
	\solution
	%\begin{align}
    \label{eq:12.13.6.18.1}
	\because	\pr{A|B} &> \pr{A},\
\frac{\pr{AB}}{\pr{B}} > \pr{A}
\\
    \label{eq:12.13.6.18.2}
	\implies \pr{AB} &> \pr{A}\pr{B}
	\\
	\text{or, } \frac{\pr{AB}}{\pr{A}} &=\pr{B|A} > \pr{A}
\end{align}

\end{enumerate}

		\item A box of oranges is inspected by examining three randomly selected oranges drawn without replacement. If all the three oranges are good, the box is approved for sale, otherwise, it is rejected. Find the probability that a box containing 15 oranges out of which 12 are good and 3 are bad ones will be approved for sale.
		\label{ncert/12/13/2/3/defs.tex}
		\item Two balls are drawn at random with replacement from a box containing 10 black and 8 red balls. Find the probability that
		\label{ncert/12/13/2/12}
\begin{enumerate}
\item both balls are red.
\item first ball is black and second is red.
\item one of them is black and other is red.
\end{enumerate}

\item In a hostel, 60\% of the students read Hindi newspaper, 40\% read English newspaper and 20\% read both Hindi and English newspapers. A student is selected at random.
		\label{ncert/12/13/2/15}
\begin{enumerate}
\item Find the probability that she reads neither Hindi nor English newspapers.
\item If she reads Hindi newspaper, find the probability that she reads English newspaper.
\item If she reads English newspaper, find the probability that she reads Hindi newspaper.\\
\end{enumerate}
\item The probability of obtaining an even prime number on each die, when a pair of dice is rolled is 
\begin{enumerate}
    \item $0$ 
    
    \item $\frac{1}{3}$ 
    
    \item $\frac{1}{12}$ 
    
    \item $\frac{1}{36}$ 
\end{enumerate}
\solution
		%\begin{enumerate}[label=\thesection.\arabic*,ref=\thesection.\theenumi]
	\item One card is drawn from a well-shuffled deck of 52 cards. Find the probability of getting
\begin{enumerate}
\item A king of red colour 
\item A face card 
\item A red face card
\item The jack of hearts
\item A spade
\item The queen of diamonds

\end{enumerate}
\solution
		%\begin{table}[H]
	\centering
\begin{tabular}{|c|c|c|}
\hline
Random variable &Value &Definition\\ \hline
\multirow{3}{*}{X} &0 &Slips of Rs 1\\
&1 &Slips of Rs 5\\
&2 &Slips of Rs 13\\ \hline
\multirow{2}{*}{Y} &0 &Box A\\
&1 &Box B\\\hline
\end{tabular}
\caption{}
\label{tab:Distribution}
\end{table}
See \tabref{tab:Distribution}.
\begin{align}
p_{Y}\brak{k}= \begin{cases} 
      \frac{1}{3} & {k=0} \\
      \frac{2}{3 }& {k=1} 
   \end{cases}
   \\
p_{Y|X}\brak{0|0} = \frac{19}{25}\, 
p_{Y|X}\brak{0|1} = \frac{6}{25}\,
p_{Y|X}\brak{1|0} = \frac{45}{50}\,
p_{Y|X}\brak{1|2} = \frac{5}{50}
\end{align}
The desired probability is the probability that a slip drawn at random is marked other than Rs 1,
\begin{align}
&=1-p_X\brak{0}\\
&= p_X(1) + p_X(2)
\end{align}
Using Bayes theorem,
\begin{align}
&= p_Y\brak{0} \times \pr{Y=0 | X=1} + p_Y\brak{1} \times \pr{Y=1|X=2}\\
&=\frac{1}{3} \times \frac{6}{25} + \frac{2}{3} \times \frac{5}{50}\\
&=\frac{11}{75}
\end{align}

\newpage

%\tableofcontents

\bigskip

\renewcommand{\thefigure}{\theenumi}
\renewcommand{\thetable}{\theenumi}
%\renewcommand{\theequation}{\theenumi}

%\begin{abstract}
%%\boldmath
%In this letter, an algorithm for evaluating the exact analytical bit error rate  (BER)  for the piecewise linear (PL) combiner for  multiple relays is presented. Previous results were available only for upto three relays. The algorithm is unique in the sense that  the actual mathematical expressions, that are prohibitively large, need not be explicitly obtained. The diversity gain due to multiple relays is shown through plots of the analytical BER, well supported by simulations. 
%
%\end{abstract}
% IEEEtran.cls defaults to using nonbold math in the Abstract.
% This preserves the distinction between vectors and scalars. However,
% if the journal you are submitting to favors bold math in the abstract,
% then you can use LaTeX's standard command \boldmath at the very start
% of the abstract to achieve this. Many IEEE journals frown on math
% in the abstract anyway.

% Note that keywords are not normally used for peerreview papers.
%\begin{IEEEkeywords}
%Cooperative diversity, decode and forward, piecewise linear
%\end{IEEEkeywords}



% For peer review papers, you can put extra information on the cover
% page as needed:
% \ifCLASSOPTIONpeerreview
% \begin{center} \bfseries EDICS Category: 3-BBND \end{center}
% \fi
%
% For peerreview papers, this IEEEtran command inserts a page break and
% creates the second title. It will be ignored for other modes.
%\IEEEpeerreviewmaketitle




	\item Five cards—the ten, jack, queen, king and ace of diamonds, are well-shuffled with their face downwards. One card is then picked up at random.
\begin{enumerate}
\item
What is the probability that the card is the queen? 
\item
If the queen is drawn and put aside, what is the probability that the second card picked up is (a) an ace? (b) a queen?\\
\end{enumerate}
\solution
		%\begin{enumerate}[label=\thesection.\arabic*,ref=\thesection.\theenumi]
	\item One card is drawn from a well-shuffled deck of 52 cards. Find the probability of getting
\begin{enumerate}
\item A king of red colour 
\item A face card 
\item A red face card
\item The jack of hearts
\item A spade
\item The queen of diamonds

\end{enumerate}
\solution
		%\input{ncert/10/15/1/14/main.tex}
	\item Five cards—the ten, jack, queen, king and ace of diamonds, are well-shuffled with their face downwards. One card is then picked up at random.
\begin{enumerate}
\item
What is the probability that the card is the queen? 
\item
If the queen is drawn and put aside, what is the probability that the second card picked up is (a) an ace? (b) a queen?\\
\end{enumerate}
\solution
		%\input{ncert/10/15/1/15/defs.tex}
	\item A bag contains $5$ red balls and some blue balls. If the probability of drawing a blue ball is double that if a red ball, determine the number of blue balls in the bag. 
		\\
\solution
		%\input{ncert/10/15/2/3/defs.tex}
	\item A card is selected from a pack of 52 cards.
 \begin{enumerate}[label=(\alph*)] 
                 \item How many points are there in the sample space?
                 \item Calculate the probability that the card is an ace of spades.
                 \item Calculate the probability that the card is (i) an ace and (ii) black card.
 \end{enumerate}
\solution
		%\input{ncert/11/16/3/4/main.tex}
\item Four cards are drawn from a well-shuffled deck of 52 cards. What is the probability of obtaining 3 diamonds and one spade.
\\
\solution
		%\input{ncert/11/16/4/2/defs.tex}
\item In a certain lottery 10,000 tickets are sold and ten equal prizes are awarded. What is the probability of not getting a prize if you buy (a) one ticket (b) two tickets (c) 10 tickets ?	
\\
\solution
		%\input{ncert/11/16/4/4/defs.tex}
		%
\item 
Out of 100 students, two sections of 40 and 60 are formed. If you and your friend are among the 100 students, what is the probability that
\begin{enumerate}
\item you both enter the same section?
\item you both enter the different sections?
\end{enumerate}
\solution
		%\input{ncert/11/16/4/5/defs.tex}
	\item 
The number lock of a suitcase has 4 wheels each labelled with ten digits i.e. from 0 to 9.The lock opens with a sequence of four digits with no repeats.What is the probability of a person getting the right sequence to open the suitcase.
\\
\solution
		%\input{ncert/11/16/4/10/defs.tex}
		%
\item 
Two cards are drawn at random and without replacement from a pack of 52 playing cards. Find the probability that both the cards are black.
\\
\solution
		%\input{ncert/12/13/2/2/defs.tex}
		\item A box of oranges is inspected by examining three randomly selected oranges drawn without replacement. If all the three oranges are good, the box is approved for sale, otherwise, it is rejected. Find the probability that a box containing 15 oranges out of which 12 are good and 3 are bad ones will be approved for sale.
		\label{ncert/12/13/2/3/defs.tex}
		\item Two balls are drawn at random with replacement from a box containing 10 black and 8 red balls. Find the probability that
		\label{ncert/12/13/2/12}
\begin{enumerate}
\item both balls are red.
\item first ball is black and second is red.
\item one of them is black and other is red.
\end{enumerate}

\item In a hostel, 60\% of the students read Hindi newspaper, 40\% read English newspaper and 20\% read both Hindi and English newspapers. A student is selected at random.
		\label{ncert/12/13/2/15}
\begin{enumerate}
\item Find the probability that she reads neither Hindi nor English newspapers.
\item If she reads Hindi newspaper, find the probability that she reads English newspaper.
\item If she reads English newspaper, find the probability that she reads Hindi newspaper.\\
\end{enumerate}
\item The probability of obtaining an even prime number on each die, when a pair of dice is rolled is 
\begin{enumerate}
    \item $0$ 
    
    \item $\frac{1}{3}$ 
    
    \item $\frac{1}{12}$ 
    
    \item $\frac{1}{36}$ 
\end{enumerate}
\solution
		%\input{ncert/12/13/2/17/defs.tex}
	\item A bag contains 4 red and 4 black balls, another bag contains 2 red and 6 black balls. One of the two bags is selected at random and a ball is drawn from the bag which is found to be red. Find the probability that the ball is drawn from the first bag.
\\
\solution
		%\input{ncert/12/13/3/2/main.tex}
  \item
  Cards with numbers 2 to 101 are placed in a box. A card is selected at random.Find the probability that the card has
\begin{enumerate}[label=(\roman*)]
	\item an even number 
	\item a square number
\end{enumerate}
\solution
%\input{exemplar/10/13/3/32/main.tex}
\item
The king, queen and jack of clubs are removed from a deck of 52 playing cards and then well shuffled. Now one card is drawn at random from the remaining cards.  Determine the probability that the card is
\begin{enumerate}[label=(\roman*)]
\item a club
\item 10 of hearts
\end{enumerate}
\solution
%\input{exemplar/10/13/3/29/main.tex}
\item A team of medical students doing their internship have to assist during surgeries
at a city hospital. The probabilities of surgeries rated as very complex, complex,
routine, simple or very simple are respectively, 0.15, 0.20, 0.31, 0.26, .08. Find
the probabilities that a particular surgery will be rated
\begin{enumerate}
	\item complex or very complex;
	\item neither very complex nor very simple;
	\item routine or complex
	\item routine or simple
\end{enumerate}
\solution
%\input{exemplar/11/16/3/8(1)/main.tex}
\item A card is selected from a pack of 52 cards.
\begin{enumerate}[label=(\alph*)]
    \item How many points are there in the sample space?
    \item Calculate the probability that the card is an ace of spades.
    \item Calculate the probability that the card is (i) an ace and (ii) black card.
\end{enumerate}
\solution
%\input{exemplar/11/16/3/4/main2.tex}
\item The probability that a non leap year selected at random will contain 53 sundays.
\\
\solution
%\input{exemplar/10/13/1/19/main.tex}
\item One of the four persons John, Rita, Aslam or Gurpreet will be promoted next
month. Consequently the sample space consists of four elementary outcomes
S = {John promoted, Rita promoted, Aslam promoted, Gurpreet promoted}
You are told that the chances of John’s promotion is same as that of Gurpreet,
Rita’s chances of promotion are twice as likely as Johns. Aslam’s chances are
four times that of John.
\begin{enumerate}
	\item Determine
	\begin{enumerate}
		\item P (John promoted)
		\item P (Rita promoted)
		\item P (Aslam promoted)
		\item P (Gurpreet promoted)
	\end{enumerate}
	\item If A = {John promoted or Gurpreet promoted}, find P (A).
\end{enumerate}
\solution
%\input{exemplar/11/16/3/10/main.tex}
\item A card is drawn from a deck of 52 cards. Find the probability of getting a king or a heart or a red card.\\
\solution
%\input{exemplar/11/16/3/15/main.tex}
\item The probability that a student will pass his examination is 0.73, the probability of
the student getting a compartment is 0.13, and the probability that the student will
either pass or get compartment is 0.96. State True or False.\\
\solution
%\input{exemplar/11/16/3/31/main.tex}
\item A card is selected from a pack of 52 cards\\
\begin{enumerate}[label=(\alph*)]
\item How many points are there in the sample space?
\item Calculate the probability that the cards is an ace of spades.
\item Calculate the probability that the card is (i) an ace (ii)black card.\\
\end{enumerate}
%\input{ncert/11/16/3/4_1/Prob_4.tex}
\item In a non-leap year, the probability of having 53 tuesdays or 53 wednesdays is\\
\solution
%\input{exemplar/11/16/3/18/main.tex}
\item There are 1000 sealed envelopes in a box, 10 of them contain a cash prize of
Rs 100 each, 100 of them contain a cash prize of Rs 50 each and 200 of them
contain a cash prize of Rs 10 each and rest do not contain any cash prize. If they
are well shuffled and an envelope is picked up out, what is the probability that it
contains no cash prize?\\
\solution
%\input{exemplar/10/13/3/34/main.tex}
\item 
A die is thrown and a card is selected at random from a deck of 52 playing cards. The probability of getting an even number on the die and a spade card.\\
\solution
%\input{exemplar/12/13/3/78/main.tex}
\item
If 4-digit numbers greater than 5,000 are randomly formed from the digits 0, 1, 3, 5, and 7, what is the probability of forming a number divisible by 5 when:
\begin{enumerate}
    \item The digits are repeated?
    \item The repetition of digits is not allowed?
\end{enumerate}
\solution
%\input{ncert/11/16/4/9/main.tex}
\item Consider the probability space $\brak{\Omega, \mathcal{G}, P}$ where $\Omega = [0,2]$ and $\mathcal{G} = \cbrak{\phi, \Omega, [0,1], (1,2]}$. Let $X$ and $Y$ be two functions on $\Omega$ defined as
\begin{align*}
    X(\omega) = 
    \begin{cases}
        1 & \text{if }\omega \in [0, 1]\\
        2 & \text{if }\omega \in (1, 2]
    \end{cases}
\end{align*}
and
\begin{align*}
    Y(\omega) = 
    \begin{cases}
        2 & \text{if }\omega \in [0, 1.5]\\
        3 & \text{if }\omega \in (1.5, 2].
    \end{cases}
\end{align*}
Then which one of the following statements is true?
\begin{enumerate}
    \item [(A)] $X$ is a random variable with respect to $\mathcal{G}$, but $Y$ is not a random variable with respect to $\mathcal{G}$.
    \item [(B)] $Y$ is a random variable with respect to $\mathcal{G}$, but $X$ is not a random variable with respect to $\mathcal{G}$.
    \item [(C)] Neither $X$ nor $Y$ is a random variable with respect to $\mathcal{G}$.
    \item [(D)] Both $X$ and $Y$ are random variables with respect to $\mathcal{G}$.
\end{enumerate} \hfill (GATE ST 2023)\\
\solution
%\input{gate/ST/2023/14/main.tex}
	\item  A die is loaded in such a way that each odd number is twice as likely to occur as
each even number. Find $P(G)$, where $G$ is the event that a number greater than
3 occurs on a single roll of the die.
\\
\solution
		%\input{exemplar/11/16/3/5/main.tex}
	\item All the jacks, queens and kings are removed from a deck of 52 playing cards. The remaining cards are well shuffled and then one card is drawn at random. Giving ace a value 1 similar value for other cards, find the probability that the card has a value 
		\begin{enumerate}
			\item 7
			\item greater than 7
			\item less than 7
		\end{enumerate}
		%\input{exemplar/10/13/3/30/main.tex}
  \item A Lot consists of 48 mobile phones of which 42 are good, 3 have only minor defects and 3 have major defects.Varnika will buy a phone if it is good but the trader will only buy a mobile if it has no major defects. One phone is selected at random from the lot. What is the probability that it is
\begin{enumerate}
	\item acceptable to Varnika?
            \item acceptable to the trader?
\end{enumerate}
\solution
	%\input{exemplar/10/13/3/40/main.tex}
 \item A student says that if you throw a die, it will show up 1 or not 1. Therefore, the probability of getting 1 and the probability of getting 'not 1' each is equal to $\frac{1}{2}$. Is this correct? Give reasons.\\
 \solution
        %\input{exemplar/10/13/2/9/main.tex}
   \item Four candidates A, B, C, D have ap-
plied for the assignment to coach a school cricket
team. If A is twice as likely to be selected as B, and
B and C are given about the same chance of being
selected, while C is twice as likely to be selected
as D, what are the probabilities that
\begin{enumerate}
\item C will be selected?
\item A will not be selected?
\end{enumerate}
	%\input{exemplar/11/16/3/9/main.tex}
 \item A bag contain 24 balls of which $x$ balls are red, $2x$ are white and $3x$ are blue. A ball is selected at random, What is the probability that it is
\begin{enumerate}[label=\alph*)]
\item not red ?
\item white ?
\end{enumerate}
%\input{exemplar/10/13/3/41/main.tex}
If the letters of the word ASSASSINATION are arranged at random. Find the Probability that
\begin{enumerate}[label=(\alph*)]
\item Four $S's$ come consecutively in the word
\item Two  $I's$ and two $N's$ come together
\item All $A's$ are not coming together
\item No two $A's$ are coming together
\end{enumerate}
%\input{exemplar/11/16/3/14/main.tex}
	\item One urn contains two black balls (labelled B1 and B2) and one white ball. A
	second urn contains one black ball and two white balls (labelled W1 and W2).
	Suppose the following experiment is performed. One of the two urns is chosen
	at random. Next a ball is randomly chosen from the urn. Then a second ball is
	chosen at random from the same urn without replacing the first ball.
	
	\begin{enumerate}
	\item What is the probability that two black balls are chosen?
	
	\item What is the probability that two balls of opposite colour are chosen?
	\end{enumerate}
	\solution
	%\input{exemplar/11/16/3/12/main1.tex}
\end{enumerate}

	\item A bag contains $5$ red balls and some blue balls. If the probability of drawing a blue ball is double that if a red ball, determine the number of blue balls in the bag. 
		\\
\solution
		%\begin{enumerate}[label=\thesection.\arabic*,ref=\thesection.\theenumi]
	\item One card is drawn from a well-shuffled deck of 52 cards. Find the probability of getting
\begin{enumerate}
\item A king of red colour 
\item A face card 
\item A red face card
\item The jack of hearts
\item A spade
\item The queen of diamonds

\end{enumerate}
\solution
		%\input{ncert/10/15/1/14/main.tex}
	\item Five cards—the ten, jack, queen, king and ace of diamonds, are well-shuffled with their face downwards. One card is then picked up at random.
\begin{enumerate}
\item
What is the probability that the card is the queen? 
\item
If the queen is drawn and put aside, what is the probability that the second card picked up is (a) an ace? (b) a queen?\\
\end{enumerate}
\solution
		%\input{ncert/10/15/1/15/defs.tex}
	\item A bag contains $5$ red balls and some blue balls. If the probability of drawing a blue ball is double that if a red ball, determine the number of blue balls in the bag. 
		\\
\solution
		%\input{ncert/10/15/2/3/defs.tex}
	\item A card is selected from a pack of 52 cards.
 \begin{enumerate}[label=(\alph*)] 
                 \item How many points are there in the sample space?
                 \item Calculate the probability that the card is an ace of spades.
                 \item Calculate the probability that the card is (i) an ace and (ii) black card.
 \end{enumerate}
\solution
		%\input{ncert/11/16/3/4/main.tex}
\item Four cards are drawn from a well-shuffled deck of 52 cards. What is the probability of obtaining 3 diamonds and one spade.
\\
\solution
		%\input{ncert/11/16/4/2/defs.tex}
\item In a certain lottery 10,000 tickets are sold and ten equal prizes are awarded. What is the probability of not getting a prize if you buy (a) one ticket (b) two tickets (c) 10 tickets ?	
\\
\solution
		%\input{ncert/11/16/4/4/defs.tex}
		%
\item 
Out of 100 students, two sections of 40 and 60 are formed. If you and your friend are among the 100 students, what is the probability that
\begin{enumerate}
\item you both enter the same section?
\item you both enter the different sections?
\end{enumerate}
\solution
		%\input{ncert/11/16/4/5/defs.tex}
	\item 
The number lock of a suitcase has 4 wheels each labelled with ten digits i.e. from 0 to 9.The lock opens with a sequence of four digits with no repeats.What is the probability of a person getting the right sequence to open the suitcase.
\\
\solution
		%\input{ncert/11/16/4/10/defs.tex}
		%
\item 
Two cards are drawn at random and without replacement from a pack of 52 playing cards. Find the probability that both the cards are black.
\\
\solution
		%\input{ncert/12/13/2/2/defs.tex}
		\item A box of oranges is inspected by examining three randomly selected oranges drawn without replacement. If all the three oranges are good, the box is approved for sale, otherwise, it is rejected. Find the probability that a box containing 15 oranges out of which 12 are good and 3 are bad ones will be approved for sale.
		\label{ncert/12/13/2/3/defs.tex}
		\item Two balls are drawn at random with replacement from a box containing 10 black and 8 red balls. Find the probability that
		\label{ncert/12/13/2/12}
\begin{enumerate}
\item both balls are red.
\item first ball is black and second is red.
\item one of them is black and other is red.
\end{enumerate}

\item In a hostel, 60\% of the students read Hindi newspaper, 40\% read English newspaper and 20\% read both Hindi and English newspapers. A student is selected at random.
		\label{ncert/12/13/2/15}
\begin{enumerate}
\item Find the probability that she reads neither Hindi nor English newspapers.
\item If she reads Hindi newspaper, find the probability that she reads English newspaper.
\item If she reads English newspaper, find the probability that she reads Hindi newspaper.\\
\end{enumerate}
\item The probability of obtaining an even prime number on each die, when a pair of dice is rolled is 
\begin{enumerate}
    \item $0$ 
    
    \item $\frac{1}{3}$ 
    
    \item $\frac{1}{12}$ 
    
    \item $\frac{1}{36}$ 
\end{enumerate}
\solution
		%\input{ncert/12/13/2/17/defs.tex}
	\item A bag contains 4 red and 4 black balls, another bag contains 2 red and 6 black balls. One of the two bags is selected at random and a ball is drawn from the bag which is found to be red. Find the probability that the ball is drawn from the first bag.
\\
\solution
		%\input{ncert/12/13/3/2/main.tex}
  \item
  Cards with numbers 2 to 101 are placed in a box. A card is selected at random.Find the probability that the card has
\begin{enumerate}[label=(\roman*)]
	\item an even number 
	\item a square number
\end{enumerate}
\solution
%\input{exemplar/10/13/3/32/main.tex}
\item
The king, queen and jack of clubs are removed from a deck of 52 playing cards and then well shuffled. Now one card is drawn at random from the remaining cards.  Determine the probability that the card is
\begin{enumerate}[label=(\roman*)]
\item a club
\item 10 of hearts
\end{enumerate}
\solution
%\input{exemplar/10/13/3/29/main.tex}
\item A team of medical students doing their internship have to assist during surgeries
at a city hospital. The probabilities of surgeries rated as very complex, complex,
routine, simple or very simple are respectively, 0.15, 0.20, 0.31, 0.26, .08. Find
the probabilities that a particular surgery will be rated
\begin{enumerate}
	\item complex or very complex;
	\item neither very complex nor very simple;
	\item routine or complex
	\item routine or simple
\end{enumerate}
\solution
%\input{exemplar/11/16/3/8(1)/main.tex}
\item A card is selected from a pack of 52 cards.
\begin{enumerate}[label=(\alph*)]
    \item How many points are there in the sample space?
    \item Calculate the probability that the card is an ace of spades.
    \item Calculate the probability that the card is (i) an ace and (ii) black card.
\end{enumerate}
\solution
%\input{exemplar/11/16/3/4/main2.tex}
\item The probability that a non leap year selected at random will contain 53 sundays.
\\
\solution
%\input{exemplar/10/13/1/19/main.tex}
\item One of the four persons John, Rita, Aslam or Gurpreet will be promoted next
month. Consequently the sample space consists of four elementary outcomes
S = {John promoted, Rita promoted, Aslam promoted, Gurpreet promoted}
You are told that the chances of John’s promotion is same as that of Gurpreet,
Rita’s chances of promotion are twice as likely as Johns. Aslam’s chances are
four times that of John.
\begin{enumerate}
	\item Determine
	\begin{enumerate}
		\item P (John promoted)
		\item P (Rita promoted)
		\item P (Aslam promoted)
		\item P (Gurpreet promoted)
	\end{enumerate}
	\item If A = {John promoted or Gurpreet promoted}, find P (A).
\end{enumerate}
\solution
%\input{exemplar/11/16/3/10/main.tex}
\item A card is drawn from a deck of 52 cards. Find the probability of getting a king or a heart or a red card.\\
\solution
%\input{exemplar/11/16/3/15/main.tex}
\item The probability that a student will pass his examination is 0.73, the probability of
the student getting a compartment is 0.13, and the probability that the student will
either pass or get compartment is 0.96. State True or False.\\
\solution
%\input{exemplar/11/16/3/31/main.tex}
\item A card is selected from a pack of 52 cards\\
\begin{enumerate}[label=(\alph*)]
\item How many points are there in the sample space?
\item Calculate the probability that the cards is an ace of spades.
\item Calculate the probability that the card is (i) an ace (ii)black card.\\
\end{enumerate}
%\input{ncert/11/16/3/4_1/Prob_4.tex}
\item In a non-leap year, the probability of having 53 tuesdays or 53 wednesdays is\\
\solution
%\input{exemplar/11/16/3/18/main.tex}
\item There are 1000 sealed envelopes in a box, 10 of them contain a cash prize of
Rs 100 each, 100 of them contain a cash prize of Rs 50 each and 200 of them
contain a cash prize of Rs 10 each and rest do not contain any cash prize. If they
are well shuffled and an envelope is picked up out, what is the probability that it
contains no cash prize?\\
\solution
%\input{exemplar/10/13/3/34/main.tex}
\item 
A die is thrown and a card is selected at random from a deck of 52 playing cards. The probability of getting an even number on the die and a spade card.\\
\solution
%\input{exemplar/12/13/3/78/main.tex}
\item
If 4-digit numbers greater than 5,000 are randomly formed from the digits 0, 1, 3, 5, and 7, what is the probability of forming a number divisible by 5 when:
\begin{enumerate}
    \item The digits are repeated?
    \item The repetition of digits is not allowed?
\end{enumerate}
\solution
%\input{ncert/11/16/4/9/main.tex}
\item Consider the probability space $\brak{\Omega, \mathcal{G}, P}$ where $\Omega = [0,2]$ and $\mathcal{G} = \cbrak{\phi, \Omega, [0,1], (1,2]}$. Let $X$ and $Y$ be two functions on $\Omega$ defined as
\begin{align*}
    X(\omega) = 
    \begin{cases}
        1 & \text{if }\omega \in [0, 1]\\
        2 & \text{if }\omega \in (1, 2]
    \end{cases}
\end{align*}
and
\begin{align*}
    Y(\omega) = 
    \begin{cases}
        2 & \text{if }\omega \in [0, 1.5]\\
        3 & \text{if }\omega \in (1.5, 2].
    \end{cases}
\end{align*}
Then which one of the following statements is true?
\begin{enumerate}
    \item [(A)] $X$ is a random variable with respect to $\mathcal{G}$, but $Y$ is not a random variable with respect to $\mathcal{G}$.
    \item [(B)] $Y$ is a random variable with respect to $\mathcal{G}$, but $X$ is not a random variable with respect to $\mathcal{G}$.
    \item [(C)] Neither $X$ nor $Y$ is a random variable with respect to $\mathcal{G}$.
    \item [(D)] Both $X$ and $Y$ are random variables with respect to $\mathcal{G}$.
\end{enumerate} \hfill (GATE ST 2023)\\
\solution
%\input{gate/ST/2023/14/main.tex}
	\item  A die is loaded in such a way that each odd number is twice as likely to occur as
each even number. Find $P(G)$, where $G$ is the event that a number greater than
3 occurs on a single roll of the die.
\\
\solution
		%\input{exemplar/11/16/3/5/main.tex}
	\item All the jacks, queens and kings are removed from a deck of 52 playing cards. The remaining cards are well shuffled and then one card is drawn at random. Giving ace a value 1 similar value for other cards, find the probability that the card has a value 
		\begin{enumerate}
			\item 7
			\item greater than 7
			\item less than 7
		\end{enumerate}
		%\input{exemplar/10/13/3/30/main.tex}
  \item A Lot consists of 48 mobile phones of which 42 are good, 3 have only minor defects and 3 have major defects.Varnika will buy a phone if it is good but the trader will only buy a mobile if it has no major defects. One phone is selected at random from the lot. What is the probability that it is
\begin{enumerate}
	\item acceptable to Varnika?
            \item acceptable to the trader?
\end{enumerate}
\solution
	%\input{exemplar/10/13/3/40/main.tex}
 \item A student says that if you throw a die, it will show up 1 or not 1. Therefore, the probability of getting 1 and the probability of getting 'not 1' each is equal to $\frac{1}{2}$. Is this correct? Give reasons.\\
 \solution
        %\input{exemplar/10/13/2/9/main.tex}
   \item Four candidates A, B, C, D have ap-
plied for the assignment to coach a school cricket
team. If A is twice as likely to be selected as B, and
B and C are given about the same chance of being
selected, while C is twice as likely to be selected
as D, what are the probabilities that
\begin{enumerate}
\item C will be selected?
\item A will not be selected?
\end{enumerate}
	%\input{exemplar/11/16/3/9/main.tex}
 \item A bag contain 24 balls of which $x$ balls are red, $2x$ are white and $3x$ are blue. A ball is selected at random, What is the probability that it is
\begin{enumerate}[label=\alph*)]
\item not red ?
\item white ?
\end{enumerate}
%\input{exemplar/10/13/3/41/main.tex}
If the letters of the word ASSASSINATION are arranged at random. Find the Probability that
\begin{enumerate}[label=(\alph*)]
\item Four $S's$ come consecutively in the word
\item Two  $I's$ and two $N's$ come together
\item All $A's$ are not coming together
\item No two $A's$ are coming together
\end{enumerate}
%\input{exemplar/11/16/3/14/main.tex}
	\item One urn contains two black balls (labelled B1 and B2) and one white ball. A
	second urn contains one black ball and two white balls (labelled W1 and W2).
	Suppose the following experiment is performed. One of the two urns is chosen
	at random. Next a ball is randomly chosen from the urn. Then a second ball is
	chosen at random from the same urn without replacing the first ball.
	
	\begin{enumerate}
	\item What is the probability that two black balls are chosen?
	
	\item What is the probability that two balls of opposite colour are chosen?
	\end{enumerate}
	\solution
	%\input{exemplar/11/16/3/12/main1.tex}
\end{enumerate}

	\item A card is selected from a pack of 52 cards.
 \begin{enumerate}[label=(\alph*)] 
                 \item How many points are there in the sample space?
                 \item Calculate the probability that the card is an ace of spades.
                 \item Calculate the probability that the card is (i) an ace and (ii) black card.
 \end{enumerate}
\solution
		%\begin{table}[H]
	\centering
\begin{tabular}{|c|c|c|}
\hline
Random variable &Value &Definition\\ \hline
\multirow{3}{*}{X} &0 &Slips of Rs 1\\
&1 &Slips of Rs 5\\
&2 &Slips of Rs 13\\ \hline
\multirow{2}{*}{Y} &0 &Box A\\
&1 &Box B\\\hline
\end{tabular}
\caption{}
\label{tab:Distribution}
\end{table}
See \tabref{tab:Distribution}.
\begin{align}
p_{Y}\brak{k}= \begin{cases} 
      \frac{1}{3} & {k=0} \\
      \frac{2}{3 }& {k=1} 
   \end{cases}
   \\
p_{Y|X}\brak{0|0} = \frac{19}{25}\, 
p_{Y|X}\brak{0|1} = \frac{6}{25}\,
p_{Y|X}\brak{1|0} = \frac{45}{50}\,
p_{Y|X}\brak{1|2} = \frac{5}{50}
\end{align}
The desired probability is the probability that a slip drawn at random is marked other than Rs 1,
\begin{align}
&=1-p_X\brak{0}\\
&= p_X(1) + p_X(2)
\end{align}
Using Bayes theorem,
\begin{align}
&= p_Y\brak{0} \times \pr{Y=0 | X=1} + p_Y\brak{1} \times \pr{Y=1|X=2}\\
&=\frac{1}{3} \times \frac{6}{25} + \frac{2}{3} \times \frac{5}{50}\\
&=\frac{11}{75}
\end{align}

\newpage

%\tableofcontents

\bigskip

\renewcommand{\thefigure}{\theenumi}
\renewcommand{\thetable}{\theenumi}
%\renewcommand{\theequation}{\theenumi}

%\begin{abstract}
%%\boldmath
%In this letter, an algorithm for evaluating the exact analytical bit error rate  (BER)  for the piecewise linear (PL) combiner for  multiple relays is presented. Previous results were available only for upto three relays. The algorithm is unique in the sense that  the actual mathematical expressions, that are prohibitively large, need not be explicitly obtained. The diversity gain due to multiple relays is shown through plots of the analytical BER, well supported by simulations. 
%
%\end{abstract}
% IEEEtran.cls defaults to using nonbold math in the Abstract.
% This preserves the distinction between vectors and scalars. However,
% if the journal you are submitting to favors bold math in the abstract,
% then you can use LaTeX's standard command \boldmath at the very start
% of the abstract to achieve this. Many IEEE journals frown on math
% in the abstract anyway.

% Note that keywords are not normally used for peerreview papers.
%\begin{IEEEkeywords}
%Cooperative diversity, decode and forward, piecewise linear
%\end{IEEEkeywords}



% For peer review papers, you can put extra information on the cover
% page as needed:
% \ifCLASSOPTIONpeerreview
% \begin{center} \bfseries EDICS Category: 3-BBND \end{center}
% \fi
%
% For peerreview papers, this IEEEtran command inserts a page break and
% creates the second title. It will be ignored for other modes.
%\IEEEpeerreviewmaketitle




\item Four cards are drawn from a well-shuffled deck of 52 cards. What is the probability of obtaining 3 diamonds and one spade.
\\
\solution
		%\begin{enumerate}[label=\thesection.\arabic*,ref=\thesection.\theenumi]
	\item One card is drawn from a well-shuffled deck of 52 cards. Find the probability of getting
\begin{enumerate}
\item A king of red colour 
\item A face card 
\item A red face card
\item The jack of hearts
\item A spade
\item The queen of diamonds

\end{enumerate}
\solution
		%\input{ncert/10/15/1/14/main.tex}
	\item Five cards—the ten, jack, queen, king and ace of diamonds, are well-shuffled with their face downwards. One card is then picked up at random.
\begin{enumerate}
\item
What is the probability that the card is the queen? 
\item
If the queen is drawn and put aside, what is the probability that the second card picked up is (a) an ace? (b) a queen?\\
\end{enumerate}
\solution
		%\input{ncert/10/15/1/15/defs.tex}
	\item A bag contains $5$ red balls and some blue balls. If the probability of drawing a blue ball is double that if a red ball, determine the number of blue balls in the bag. 
		\\
\solution
		%\input{ncert/10/15/2/3/defs.tex}
	\item A card is selected from a pack of 52 cards.
 \begin{enumerate}[label=(\alph*)] 
                 \item How many points are there in the sample space?
                 \item Calculate the probability that the card is an ace of spades.
                 \item Calculate the probability that the card is (i) an ace and (ii) black card.
 \end{enumerate}
\solution
		%\input{ncert/11/16/3/4/main.tex}
\item Four cards are drawn from a well-shuffled deck of 52 cards. What is the probability of obtaining 3 diamonds and one spade.
\\
\solution
		%\input{ncert/11/16/4/2/defs.tex}
\item In a certain lottery 10,000 tickets are sold and ten equal prizes are awarded. What is the probability of not getting a prize if you buy (a) one ticket (b) two tickets (c) 10 tickets ?	
\\
\solution
		%\input{ncert/11/16/4/4/defs.tex}
		%
\item 
Out of 100 students, two sections of 40 and 60 are formed. If you and your friend are among the 100 students, what is the probability that
\begin{enumerate}
\item you both enter the same section?
\item you both enter the different sections?
\end{enumerate}
\solution
		%\input{ncert/11/16/4/5/defs.tex}
	\item 
The number lock of a suitcase has 4 wheels each labelled with ten digits i.e. from 0 to 9.The lock opens with a sequence of four digits with no repeats.What is the probability of a person getting the right sequence to open the suitcase.
\\
\solution
		%\input{ncert/11/16/4/10/defs.tex}
		%
\item 
Two cards are drawn at random and without replacement from a pack of 52 playing cards. Find the probability that both the cards are black.
\\
\solution
		%\input{ncert/12/13/2/2/defs.tex}
		\item A box of oranges is inspected by examining three randomly selected oranges drawn without replacement. If all the three oranges are good, the box is approved for sale, otherwise, it is rejected. Find the probability that a box containing 15 oranges out of which 12 are good and 3 are bad ones will be approved for sale.
		\label{ncert/12/13/2/3/defs.tex}
		\item Two balls are drawn at random with replacement from a box containing 10 black and 8 red balls. Find the probability that
		\label{ncert/12/13/2/12}
\begin{enumerate}
\item both balls are red.
\item first ball is black and second is red.
\item one of them is black and other is red.
\end{enumerate}

\item In a hostel, 60\% of the students read Hindi newspaper, 40\% read English newspaper and 20\% read both Hindi and English newspapers. A student is selected at random.
		\label{ncert/12/13/2/15}
\begin{enumerate}
\item Find the probability that she reads neither Hindi nor English newspapers.
\item If she reads Hindi newspaper, find the probability that she reads English newspaper.
\item If she reads English newspaper, find the probability that she reads Hindi newspaper.\\
\end{enumerate}
\item The probability of obtaining an even prime number on each die, when a pair of dice is rolled is 
\begin{enumerate}
    \item $0$ 
    
    \item $\frac{1}{3}$ 
    
    \item $\frac{1}{12}$ 
    
    \item $\frac{1}{36}$ 
\end{enumerate}
\solution
		%\input{ncert/12/13/2/17/defs.tex}
	\item A bag contains 4 red and 4 black balls, another bag contains 2 red and 6 black balls. One of the two bags is selected at random and a ball is drawn from the bag which is found to be red. Find the probability that the ball is drawn from the first bag.
\\
\solution
		%\input{ncert/12/13/3/2/main.tex}
  \item
  Cards with numbers 2 to 101 are placed in a box. A card is selected at random.Find the probability that the card has
\begin{enumerate}[label=(\roman*)]
	\item an even number 
	\item a square number
\end{enumerate}
\solution
%\input{exemplar/10/13/3/32/main.tex}
\item
The king, queen and jack of clubs are removed from a deck of 52 playing cards and then well shuffled. Now one card is drawn at random from the remaining cards.  Determine the probability that the card is
\begin{enumerate}[label=(\roman*)]
\item a club
\item 10 of hearts
\end{enumerate}
\solution
%\input{exemplar/10/13/3/29/main.tex}
\item A team of medical students doing their internship have to assist during surgeries
at a city hospital. The probabilities of surgeries rated as very complex, complex,
routine, simple or very simple are respectively, 0.15, 0.20, 0.31, 0.26, .08. Find
the probabilities that a particular surgery will be rated
\begin{enumerate}
	\item complex or very complex;
	\item neither very complex nor very simple;
	\item routine or complex
	\item routine or simple
\end{enumerate}
\solution
%\input{exemplar/11/16/3/8(1)/main.tex}
\item A card is selected from a pack of 52 cards.
\begin{enumerate}[label=(\alph*)]
    \item How many points are there in the sample space?
    \item Calculate the probability that the card is an ace of spades.
    \item Calculate the probability that the card is (i) an ace and (ii) black card.
\end{enumerate}
\solution
%\input{exemplar/11/16/3/4/main2.tex}
\item The probability that a non leap year selected at random will contain 53 sundays.
\\
\solution
%\input{exemplar/10/13/1/19/main.tex}
\item One of the four persons John, Rita, Aslam or Gurpreet will be promoted next
month. Consequently the sample space consists of four elementary outcomes
S = {John promoted, Rita promoted, Aslam promoted, Gurpreet promoted}
You are told that the chances of John’s promotion is same as that of Gurpreet,
Rita’s chances of promotion are twice as likely as Johns. Aslam’s chances are
four times that of John.
\begin{enumerate}
	\item Determine
	\begin{enumerate}
		\item P (John promoted)
		\item P (Rita promoted)
		\item P (Aslam promoted)
		\item P (Gurpreet promoted)
	\end{enumerate}
	\item If A = {John promoted or Gurpreet promoted}, find P (A).
\end{enumerate}
\solution
%\input{exemplar/11/16/3/10/main.tex}
\item A card is drawn from a deck of 52 cards. Find the probability of getting a king or a heart or a red card.\\
\solution
%\input{exemplar/11/16/3/15/main.tex}
\item The probability that a student will pass his examination is 0.73, the probability of
the student getting a compartment is 0.13, and the probability that the student will
either pass or get compartment is 0.96. State True or False.\\
\solution
%\input{exemplar/11/16/3/31/main.tex}
\item A card is selected from a pack of 52 cards\\
\begin{enumerate}[label=(\alph*)]
\item How many points are there in the sample space?
\item Calculate the probability that the cards is an ace of spades.
\item Calculate the probability that the card is (i) an ace (ii)black card.\\
\end{enumerate}
%\input{ncert/11/16/3/4_1/Prob_4.tex}
\item In a non-leap year, the probability of having 53 tuesdays or 53 wednesdays is\\
\solution
%\input{exemplar/11/16/3/18/main.tex}
\item There are 1000 sealed envelopes in a box, 10 of them contain a cash prize of
Rs 100 each, 100 of them contain a cash prize of Rs 50 each and 200 of them
contain a cash prize of Rs 10 each and rest do not contain any cash prize. If they
are well shuffled and an envelope is picked up out, what is the probability that it
contains no cash prize?\\
\solution
%\input{exemplar/10/13/3/34/main.tex}
\item 
A die is thrown and a card is selected at random from a deck of 52 playing cards. The probability of getting an even number on the die and a spade card.\\
\solution
%\input{exemplar/12/13/3/78/main.tex}
\item
If 4-digit numbers greater than 5,000 are randomly formed from the digits 0, 1, 3, 5, and 7, what is the probability of forming a number divisible by 5 when:
\begin{enumerate}
    \item The digits are repeated?
    \item The repetition of digits is not allowed?
\end{enumerate}
\solution
%\input{ncert/11/16/4/9/main.tex}
\item Consider the probability space $\brak{\Omega, \mathcal{G}, P}$ where $\Omega = [0,2]$ and $\mathcal{G} = \cbrak{\phi, \Omega, [0,1], (1,2]}$. Let $X$ and $Y$ be two functions on $\Omega$ defined as
\begin{align*}
    X(\omega) = 
    \begin{cases}
        1 & \text{if }\omega \in [0, 1]\\
        2 & \text{if }\omega \in (1, 2]
    \end{cases}
\end{align*}
and
\begin{align*}
    Y(\omega) = 
    \begin{cases}
        2 & \text{if }\omega \in [0, 1.5]\\
        3 & \text{if }\omega \in (1.5, 2].
    \end{cases}
\end{align*}
Then which one of the following statements is true?
\begin{enumerate}
    \item [(A)] $X$ is a random variable with respect to $\mathcal{G}$, but $Y$ is not a random variable with respect to $\mathcal{G}$.
    \item [(B)] $Y$ is a random variable with respect to $\mathcal{G}$, but $X$ is not a random variable with respect to $\mathcal{G}$.
    \item [(C)] Neither $X$ nor $Y$ is a random variable with respect to $\mathcal{G}$.
    \item [(D)] Both $X$ and $Y$ are random variables with respect to $\mathcal{G}$.
\end{enumerate} \hfill (GATE ST 2023)\\
\solution
%\input{gate/ST/2023/14/main.tex}
	\item  A die is loaded in such a way that each odd number is twice as likely to occur as
each even number. Find $P(G)$, where $G$ is the event that a number greater than
3 occurs on a single roll of the die.
\\
\solution
		%\input{exemplar/11/16/3/5/main.tex}
	\item All the jacks, queens and kings are removed from a deck of 52 playing cards. The remaining cards are well shuffled and then one card is drawn at random. Giving ace a value 1 similar value for other cards, find the probability that the card has a value 
		\begin{enumerate}
			\item 7
			\item greater than 7
			\item less than 7
		\end{enumerate}
		%\input{exemplar/10/13/3/30/main.tex}
  \item A Lot consists of 48 mobile phones of which 42 are good, 3 have only minor defects and 3 have major defects.Varnika will buy a phone if it is good but the trader will only buy a mobile if it has no major defects. One phone is selected at random from the lot. What is the probability that it is
\begin{enumerate}
	\item acceptable to Varnika?
            \item acceptable to the trader?
\end{enumerate}
\solution
	%\input{exemplar/10/13/3/40/main.tex}
 \item A student says that if you throw a die, it will show up 1 or not 1. Therefore, the probability of getting 1 and the probability of getting 'not 1' each is equal to $\frac{1}{2}$. Is this correct? Give reasons.\\
 \solution
        %\input{exemplar/10/13/2/9/main.tex}
   \item Four candidates A, B, C, D have ap-
plied for the assignment to coach a school cricket
team. If A is twice as likely to be selected as B, and
B and C are given about the same chance of being
selected, while C is twice as likely to be selected
as D, what are the probabilities that
\begin{enumerate}
\item C will be selected?
\item A will not be selected?
\end{enumerate}
	%\input{exemplar/11/16/3/9/main.tex}
 \item A bag contain 24 balls of which $x$ balls are red, $2x$ are white and $3x$ are blue. A ball is selected at random, What is the probability that it is
\begin{enumerate}[label=\alph*)]
\item not red ?
\item white ?
\end{enumerate}
%\input{exemplar/10/13/3/41/main.tex}
If the letters of the word ASSASSINATION are arranged at random. Find the Probability that
\begin{enumerate}[label=(\alph*)]
\item Four $S's$ come consecutively in the word
\item Two  $I's$ and two $N's$ come together
\item All $A's$ are not coming together
\item No two $A's$ are coming together
\end{enumerate}
%\input{exemplar/11/16/3/14/main.tex}
	\item One urn contains two black balls (labelled B1 and B2) and one white ball. A
	second urn contains one black ball and two white balls (labelled W1 and W2).
	Suppose the following experiment is performed. One of the two urns is chosen
	at random. Next a ball is randomly chosen from the urn. Then a second ball is
	chosen at random from the same urn without replacing the first ball.
	
	\begin{enumerate}
	\item What is the probability that two black balls are chosen?
	
	\item What is the probability that two balls of opposite colour are chosen?
	\end{enumerate}
	\solution
	%\input{exemplar/11/16/3/12/main1.tex}
\end{enumerate}

\item In a certain lottery 10,000 tickets are sold and ten equal prizes are awarded. What is the probability of not getting a prize if you buy (a) one ticket (b) two tickets (c) 10 tickets ?	
\\
\solution
		%\begin{enumerate}[label=\thesection.\arabic*,ref=\thesection.\theenumi]
	\item One card is drawn from a well-shuffled deck of 52 cards. Find the probability of getting
\begin{enumerate}
\item A king of red colour 
\item A face card 
\item A red face card
\item The jack of hearts
\item A spade
\item The queen of diamonds

\end{enumerate}
\solution
		%\input{ncert/10/15/1/14/main.tex}
	\item Five cards—the ten, jack, queen, king and ace of diamonds, are well-shuffled with their face downwards. One card is then picked up at random.
\begin{enumerate}
\item
What is the probability that the card is the queen? 
\item
If the queen is drawn and put aside, what is the probability that the second card picked up is (a) an ace? (b) a queen?\\
\end{enumerate}
\solution
		%\input{ncert/10/15/1/15/defs.tex}
	\item A bag contains $5$ red balls and some blue balls. If the probability of drawing a blue ball is double that if a red ball, determine the number of blue balls in the bag. 
		\\
\solution
		%\input{ncert/10/15/2/3/defs.tex}
	\item A card is selected from a pack of 52 cards.
 \begin{enumerate}[label=(\alph*)] 
                 \item How many points are there in the sample space?
                 \item Calculate the probability that the card is an ace of spades.
                 \item Calculate the probability that the card is (i) an ace and (ii) black card.
 \end{enumerate}
\solution
		%\input{ncert/11/16/3/4/main.tex}
\item Four cards are drawn from a well-shuffled deck of 52 cards. What is the probability of obtaining 3 diamonds and one spade.
\\
\solution
		%\input{ncert/11/16/4/2/defs.tex}
\item In a certain lottery 10,000 tickets are sold and ten equal prizes are awarded. What is the probability of not getting a prize if you buy (a) one ticket (b) two tickets (c) 10 tickets ?	
\\
\solution
		%\input{ncert/11/16/4/4/defs.tex}
		%
\item 
Out of 100 students, two sections of 40 and 60 are formed. If you and your friend are among the 100 students, what is the probability that
\begin{enumerate}
\item you both enter the same section?
\item you both enter the different sections?
\end{enumerate}
\solution
		%\input{ncert/11/16/4/5/defs.tex}
	\item 
The number lock of a suitcase has 4 wheels each labelled with ten digits i.e. from 0 to 9.The lock opens with a sequence of four digits with no repeats.What is the probability of a person getting the right sequence to open the suitcase.
\\
\solution
		%\input{ncert/11/16/4/10/defs.tex}
		%
\item 
Two cards are drawn at random and without replacement from a pack of 52 playing cards. Find the probability that both the cards are black.
\\
\solution
		%\input{ncert/12/13/2/2/defs.tex}
		\item A box of oranges is inspected by examining three randomly selected oranges drawn without replacement. If all the three oranges are good, the box is approved for sale, otherwise, it is rejected. Find the probability that a box containing 15 oranges out of which 12 are good and 3 are bad ones will be approved for sale.
		\label{ncert/12/13/2/3/defs.tex}
		\item Two balls are drawn at random with replacement from a box containing 10 black and 8 red balls. Find the probability that
		\label{ncert/12/13/2/12}
\begin{enumerate}
\item both balls are red.
\item first ball is black and second is red.
\item one of them is black and other is red.
\end{enumerate}

\item In a hostel, 60\% of the students read Hindi newspaper, 40\% read English newspaper and 20\% read both Hindi and English newspapers. A student is selected at random.
		\label{ncert/12/13/2/15}
\begin{enumerate}
\item Find the probability that she reads neither Hindi nor English newspapers.
\item If she reads Hindi newspaper, find the probability that she reads English newspaper.
\item If she reads English newspaper, find the probability that she reads Hindi newspaper.\\
\end{enumerate}
\item The probability of obtaining an even prime number on each die, when a pair of dice is rolled is 
\begin{enumerate}
    \item $0$ 
    
    \item $\frac{1}{3}$ 
    
    \item $\frac{1}{12}$ 
    
    \item $\frac{1}{36}$ 
\end{enumerate}
\solution
		%\input{ncert/12/13/2/17/defs.tex}
	\item A bag contains 4 red and 4 black balls, another bag contains 2 red and 6 black balls. One of the two bags is selected at random and a ball is drawn from the bag which is found to be red. Find the probability that the ball is drawn from the first bag.
\\
\solution
		%\input{ncert/12/13/3/2/main.tex}
  \item
  Cards with numbers 2 to 101 are placed in a box. A card is selected at random.Find the probability that the card has
\begin{enumerate}[label=(\roman*)]
	\item an even number 
	\item a square number
\end{enumerate}
\solution
%\input{exemplar/10/13/3/32/main.tex}
\item
The king, queen and jack of clubs are removed from a deck of 52 playing cards and then well shuffled. Now one card is drawn at random from the remaining cards.  Determine the probability that the card is
\begin{enumerate}[label=(\roman*)]
\item a club
\item 10 of hearts
\end{enumerate}
\solution
%\input{exemplar/10/13/3/29/main.tex}
\item A team of medical students doing their internship have to assist during surgeries
at a city hospital. The probabilities of surgeries rated as very complex, complex,
routine, simple or very simple are respectively, 0.15, 0.20, 0.31, 0.26, .08. Find
the probabilities that a particular surgery will be rated
\begin{enumerate}
	\item complex or very complex;
	\item neither very complex nor very simple;
	\item routine or complex
	\item routine or simple
\end{enumerate}
\solution
%\input{exemplar/11/16/3/8(1)/main.tex}
\item A card is selected from a pack of 52 cards.
\begin{enumerate}[label=(\alph*)]
    \item How many points are there in the sample space?
    \item Calculate the probability that the card is an ace of spades.
    \item Calculate the probability that the card is (i) an ace and (ii) black card.
\end{enumerate}
\solution
%\input{exemplar/11/16/3/4/main2.tex}
\item The probability that a non leap year selected at random will contain 53 sundays.
\\
\solution
%\input{exemplar/10/13/1/19/main.tex}
\item One of the four persons John, Rita, Aslam or Gurpreet will be promoted next
month. Consequently the sample space consists of four elementary outcomes
S = {John promoted, Rita promoted, Aslam promoted, Gurpreet promoted}
You are told that the chances of John’s promotion is same as that of Gurpreet,
Rita’s chances of promotion are twice as likely as Johns. Aslam’s chances are
four times that of John.
\begin{enumerate}
	\item Determine
	\begin{enumerate}
		\item P (John promoted)
		\item P (Rita promoted)
		\item P (Aslam promoted)
		\item P (Gurpreet promoted)
	\end{enumerate}
	\item If A = {John promoted or Gurpreet promoted}, find P (A).
\end{enumerate}
\solution
%\input{exemplar/11/16/3/10/main.tex}
\item A card is drawn from a deck of 52 cards. Find the probability of getting a king or a heart or a red card.\\
\solution
%\input{exemplar/11/16/3/15/main.tex}
\item The probability that a student will pass his examination is 0.73, the probability of
the student getting a compartment is 0.13, and the probability that the student will
either pass or get compartment is 0.96. State True or False.\\
\solution
%\input{exemplar/11/16/3/31/main.tex}
\item A card is selected from a pack of 52 cards\\
\begin{enumerate}[label=(\alph*)]
\item How many points are there in the sample space?
\item Calculate the probability that the cards is an ace of spades.
\item Calculate the probability that the card is (i) an ace (ii)black card.\\
\end{enumerate}
%\input{ncert/11/16/3/4_1/Prob_4.tex}
\item In a non-leap year, the probability of having 53 tuesdays or 53 wednesdays is\\
\solution
%\input{exemplar/11/16/3/18/main.tex}
\item There are 1000 sealed envelopes in a box, 10 of them contain a cash prize of
Rs 100 each, 100 of them contain a cash prize of Rs 50 each and 200 of them
contain a cash prize of Rs 10 each and rest do not contain any cash prize. If they
are well shuffled and an envelope is picked up out, what is the probability that it
contains no cash prize?\\
\solution
%\input{exemplar/10/13/3/34/main.tex}
\item 
A die is thrown and a card is selected at random from a deck of 52 playing cards. The probability of getting an even number on the die and a spade card.\\
\solution
%\input{exemplar/12/13/3/78/main.tex}
\item
If 4-digit numbers greater than 5,000 are randomly formed from the digits 0, 1, 3, 5, and 7, what is the probability of forming a number divisible by 5 when:
\begin{enumerate}
    \item The digits are repeated?
    \item The repetition of digits is not allowed?
\end{enumerate}
\solution
%\input{ncert/11/16/4/9/main.tex}
\item Consider the probability space $\brak{\Omega, \mathcal{G}, P}$ where $\Omega = [0,2]$ and $\mathcal{G} = \cbrak{\phi, \Omega, [0,1], (1,2]}$. Let $X$ and $Y$ be two functions on $\Omega$ defined as
\begin{align*}
    X(\omega) = 
    \begin{cases}
        1 & \text{if }\omega \in [0, 1]\\
        2 & \text{if }\omega \in (1, 2]
    \end{cases}
\end{align*}
and
\begin{align*}
    Y(\omega) = 
    \begin{cases}
        2 & \text{if }\omega \in [0, 1.5]\\
        3 & \text{if }\omega \in (1.5, 2].
    \end{cases}
\end{align*}
Then which one of the following statements is true?
\begin{enumerate}
    \item [(A)] $X$ is a random variable with respect to $\mathcal{G}$, but $Y$ is not a random variable with respect to $\mathcal{G}$.
    \item [(B)] $Y$ is a random variable with respect to $\mathcal{G}$, but $X$ is not a random variable with respect to $\mathcal{G}$.
    \item [(C)] Neither $X$ nor $Y$ is a random variable with respect to $\mathcal{G}$.
    \item [(D)] Both $X$ and $Y$ are random variables with respect to $\mathcal{G}$.
\end{enumerate} \hfill (GATE ST 2023)\\
\solution
%\input{gate/ST/2023/14/main.tex}
	\item  A die is loaded in such a way that each odd number is twice as likely to occur as
each even number. Find $P(G)$, where $G$ is the event that a number greater than
3 occurs on a single roll of the die.
\\
\solution
		%\input{exemplar/11/16/3/5/main.tex}
	\item All the jacks, queens and kings are removed from a deck of 52 playing cards. The remaining cards are well shuffled and then one card is drawn at random. Giving ace a value 1 similar value for other cards, find the probability that the card has a value 
		\begin{enumerate}
			\item 7
			\item greater than 7
			\item less than 7
		\end{enumerate}
		%\input{exemplar/10/13/3/30/main.tex}
  \item A Lot consists of 48 mobile phones of which 42 are good, 3 have only minor defects and 3 have major defects.Varnika will buy a phone if it is good but the trader will only buy a mobile if it has no major defects. One phone is selected at random from the lot. What is the probability that it is
\begin{enumerate}
	\item acceptable to Varnika?
            \item acceptable to the trader?
\end{enumerate}
\solution
	%\input{exemplar/10/13/3/40/main.tex}
 \item A student says that if you throw a die, it will show up 1 or not 1. Therefore, the probability of getting 1 and the probability of getting 'not 1' each is equal to $\frac{1}{2}$. Is this correct? Give reasons.\\
 \solution
        %\input{exemplar/10/13/2/9/main.tex}
   \item Four candidates A, B, C, D have ap-
plied for the assignment to coach a school cricket
team. If A is twice as likely to be selected as B, and
B and C are given about the same chance of being
selected, while C is twice as likely to be selected
as D, what are the probabilities that
\begin{enumerate}
\item C will be selected?
\item A will not be selected?
\end{enumerate}
	%\input{exemplar/11/16/3/9/main.tex}
 \item A bag contain 24 balls of which $x$ balls are red, $2x$ are white and $3x$ are blue. A ball is selected at random, What is the probability that it is
\begin{enumerate}[label=\alph*)]
\item not red ?
\item white ?
\end{enumerate}
%\input{exemplar/10/13/3/41/main.tex}
If the letters of the word ASSASSINATION are arranged at random. Find the Probability that
\begin{enumerate}[label=(\alph*)]
\item Four $S's$ come consecutively in the word
\item Two  $I's$ and two $N's$ come together
\item All $A's$ are not coming together
\item No two $A's$ are coming together
\end{enumerate}
%\input{exemplar/11/16/3/14/main.tex}
	\item One urn contains two black balls (labelled B1 and B2) and one white ball. A
	second urn contains one black ball and two white balls (labelled W1 and W2).
	Suppose the following experiment is performed. One of the two urns is chosen
	at random. Next a ball is randomly chosen from the urn. Then a second ball is
	chosen at random from the same urn without replacing the first ball.
	
	\begin{enumerate}
	\item What is the probability that two black balls are chosen?
	
	\item What is the probability that two balls of opposite colour are chosen?
	\end{enumerate}
	\solution
	%\input{exemplar/11/16/3/12/main1.tex}
\end{enumerate}

		%
\item 
Out of 100 students, two sections of 40 and 60 are formed. If you and your friend are among the 100 students, what is the probability that
\begin{enumerate}
\item you both enter the same section?
\item you both enter the different sections?
\end{enumerate}
\solution
		%\begin{enumerate}[label=\thesection.\arabic*,ref=\thesection.\theenumi]
	\item One card is drawn from a well-shuffled deck of 52 cards. Find the probability of getting
\begin{enumerate}
\item A king of red colour 
\item A face card 
\item A red face card
\item The jack of hearts
\item A spade
\item The queen of diamonds

\end{enumerate}
\solution
		%\input{ncert/10/15/1/14/main.tex}
	\item Five cards—the ten, jack, queen, king and ace of diamonds, are well-shuffled with their face downwards. One card is then picked up at random.
\begin{enumerate}
\item
What is the probability that the card is the queen? 
\item
If the queen is drawn and put aside, what is the probability that the second card picked up is (a) an ace? (b) a queen?\\
\end{enumerate}
\solution
		%\input{ncert/10/15/1/15/defs.tex}
	\item A bag contains $5$ red balls and some blue balls. If the probability of drawing a blue ball is double that if a red ball, determine the number of blue balls in the bag. 
		\\
\solution
		%\input{ncert/10/15/2/3/defs.tex}
	\item A card is selected from a pack of 52 cards.
 \begin{enumerate}[label=(\alph*)] 
                 \item How many points are there in the sample space?
                 \item Calculate the probability that the card is an ace of spades.
                 \item Calculate the probability that the card is (i) an ace and (ii) black card.
 \end{enumerate}
\solution
		%\input{ncert/11/16/3/4/main.tex}
\item Four cards are drawn from a well-shuffled deck of 52 cards. What is the probability of obtaining 3 diamonds and one spade.
\\
\solution
		%\input{ncert/11/16/4/2/defs.tex}
\item In a certain lottery 10,000 tickets are sold and ten equal prizes are awarded. What is the probability of not getting a prize if you buy (a) one ticket (b) two tickets (c) 10 tickets ?	
\\
\solution
		%\input{ncert/11/16/4/4/defs.tex}
		%
\item 
Out of 100 students, two sections of 40 and 60 are formed. If you and your friend are among the 100 students, what is the probability that
\begin{enumerate}
\item you both enter the same section?
\item you both enter the different sections?
\end{enumerate}
\solution
		%\input{ncert/11/16/4/5/defs.tex}
	\item 
The number lock of a suitcase has 4 wheels each labelled with ten digits i.e. from 0 to 9.The lock opens with a sequence of four digits with no repeats.What is the probability of a person getting the right sequence to open the suitcase.
\\
\solution
		%\input{ncert/11/16/4/10/defs.tex}
		%
\item 
Two cards are drawn at random and without replacement from a pack of 52 playing cards. Find the probability that both the cards are black.
\\
\solution
		%\input{ncert/12/13/2/2/defs.tex}
		\item A box of oranges is inspected by examining three randomly selected oranges drawn without replacement. If all the three oranges are good, the box is approved for sale, otherwise, it is rejected. Find the probability that a box containing 15 oranges out of which 12 are good and 3 are bad ones will be approved for sale.
		\label{ncert/12/13/2/3/defs.tex}
		\item Two balls are drawn at random with replacement from a box containing 10 black and 8 red balls. Find the probability that
		\label{ncert/12/13/2/12}
\begin{enumerate}
\item both balls are red.
\item first ball is black and second is red.
\item one of them is black and other is red.
\end{enumerate}

\item In a hostel, 60\% of the students read Hindi newspaper, 40\% read English newspaper and 20\% read both Hindi and English newspapers. A student is selected at random.
		\label{ncert/12/13/2/15}
\begin{enumerate}
\item Find the probability that she reads neither Hindi nor English newspapers.
\item If she reads Hindi newspaper, find the probability that she reads English newspaper.
\item If she reads English newspaper, find the probability that she reads Hindi newspaper.\\
\end{enumerate}
\item The probability of obtaining an even prime number on each die, when a pair of dice is rolled is 
\begin{enumerate}
    \item $0$ 
    
    \item $\frac{1}{3}$ 
    
    \item $\frac{1}{12}$ 
    
    \item $\frac{1}{36}$ 
\end{enumerate}
\solution
		%\input{ncert/12/13/2/17/defs.tex}
	\item A bag contains 4 red and 4 black balls, another bag contains 2 red and 6 black balls. One of the two bags is selected at random and a ball is drawn from the bag which is found to be red. Find the probability that the ball is drawn from the first bag.
\\
\solution
		%\input{ncert/12/13/3/2/main.tex}
  \item
  Cards with numbers 2 to 101 are placed in a box. A card is selected at random.Find the probability that the card has
\begin{enumerate}[label=(\roman*)]
	\item an even number 
	\item a square number
\end{enumerate}
\solution
%\input{exemplar/10/13/3/32/main.tex}
\item
The king, queen and jack of clubs are removed from a deck of 52 playing cards and then well shuffled. Now one card is drawn at random from the remaining cards.  Determine the probability that the card is
\begin{enumerate}[label=(\roman*)]
\item a club
\item 10 of hearts
\end{enumerate}
\solution
%\input{exemplar/10/13/3/29/main.tex}
\item A team of medical students doing their internship have to assist during surgeries
at a city hospital. The probabilities of surgeries rated as very complex, complex,
routine, simple or very simple are respectively, 0.15, 0.20, 0.31, 0.26, .08. Find
the probabilities that a particular surgery will be rated
\begin{enumerate}
	\item complex or very complex;
	\item neither very complex nor very simple;
	\item routine or complex
	\item routine or simple
\end{enumerate}
\solution
%\input{exemplar/11/16/3/8(1)/main.tex}
\item A card is selected from a pack of 52 cards.
\begin{enumerate}[label=(\alph*)]
    \item How many points are there in the sample space?
    \item Calculate the probability that the card is an ace of spades.
    \item Calculate the probability that the card is (i) an ace and (ii) black card.
\end{enumerate}
\solution
%\input{exemplar/11/16/3/4/main2.tex}
\item The probability that a non leap year selected at random will contain 53 sundays.
\\
\solution
%\input{exemplar/10/13/1/19/main.tex}
\item One of the four persons John, Rita, Aslam or Gurpreet will be promoted next
month. Consequently the sample space consists of four elementary outcomes
S = {John promoted, Rita promoted, Aslam promoted, Gurpreet promoted}
You are told that the chances of John’s promotion is same as that of Gurpreet,
Rita’s chances of promotion are twice as likely as Johns. Aslam’s chances are
four times that of John.
\begin{enumerate}
	\item Determine
	\begin{enumerate}
		\item P (John promoted)
		\item P (Rita promoted)
		\item P (Aslam promoted)
		\item P (Gurpreet promoted)
	\end{enumerate}
	\item If A = {John promoted or Gurpreet promoted}, find P (A).
\end{enumerate}
\solution
%\input{exemplar/11/16/3/10/main.tex}
\item A card is drawn from a deck of 52 cards. Find the probability of getting a king or a heart or a red card.\\
\solution
%\input{exemplar/11/16/3/15/main.tex}
\item The probability that a student will pass his examination is 0.73, the probability of
the student getting a compartment is 0.13, and the probability that the student will
either pass or get compartment is 0.96. State True or False.\\
\solution
%\input{exemplar/11/16/3/31/main.tex}
\item A card is selected from a pack of 52 cards\\
\begin{enumerate}[label=(\alph*)]
\item How many points are there in the sample space?
\item Calculate the probability that the cards is an ace of spades.
\item Calculate the probability that the card is (i) an ace (ii)black card.\\
\end{enumerate}
%\input{ncert/11/16/3/4_1/Prob_4.tex}
\item In a non-leap year, the probability of having 53 tuesdays or 53 wednesdays is\\
\solution
%\input{exemplar/11/16/3/18/main.tex}
\item There are 1000 sealed envelopes in a box, 10 of them contain a cash prize of
Rs 100 each, 100 of them contain a cash prize of Rs 50 each and 200 of them
contain a cash prize of Rs 10 each and rest do not contain any cash prize. If they
are well shuffled and an envelope is picked up out, what is the probability that it
contains no cash prize?\\
\solution
%\input{exemplar/10/13/3/34/main.tex}
\item 
A die is thrown and a card is selected at random from a deck of 52 playing cards. The probability of getting an even number on the die and a spade card.\\
\solution
%\input{exemplar/12/13/3/78/main.tex}
\item
If 4-digit numbers greater than 5,000 are randomly formed from the digits 0, 1, 3, 5, and 7, what is the probability of forming a number divisible by 5 when:
\begin{enumerate}
    \item The digits are repeated?
    \item The repetition of digits is not allowed?
\end{enumerate}
\solution
%\input{ncert/11/16/4/9/main.tex}
\item Consider the probability space $\brak{\Omega, \mathcal{G}, P}$ where $\Omega = [0,2]$ and $\mathcal{G} = \cbrak{\phi, \Omega, [0,1], (1,2]}$. Let $X$ and $Y$ be two functions on $\Omega$ defined as
\begin{align*}
    X(\omega) = 
    \begin{cases}
        1 & \text{if }\omega \in [0, 1]\\
        2 & \text{if }\omega \in (1, 2]
    \end{cases}
\end{align*}
and
\begin{align*}
    Y(\omega) = 
    \begin{cases}
        2 & \text{if }\omega \in [0, 1.5]\\
        3 & \text{if }\omega \in (1.5, 2].
    \end{cases}
\end{align*}
Then which one of the following statements is true?
\begin{enumerate}
    \item [(A)] $X$ is a random variable with respect to $\mathcal{G}$, but $Y$ is not a random variable with respect to $\mathcal{G}$.
    \item [(B)] $Y$ is a random variable with respect to $\mathcal{G}$, but $X$ is not a random variable with respect to $\mathcal{G}$.
    \item [(C)] Neither $X$ nor $Y$ is a random variable with respect to $\mathcal{G}$.
    \item [(D)] Both $X$ and $Y$ are random variables with respect to $\mathcal{G}$.
\end{enumerate} \hfill (GATE ST 2023)\\
\solution
%\input{gate/ST/2023/14/main.tex}
	\item  A die is loaded in such a way that each odd number is twice as likely to occur as
each even number. Find $P(G)$, where $G$ is the event that a number greater than
3 occurs on a single roll of the die.
\\
\solution
		%\input{exemplar/11/16/3/5/main.tex}
	\item All the jacks, queens and kings are removed from a deck of 52 playing cards. The remaining cards are well shuffled and then one card is drawn at random. Giving ace a value 1 similar value for other cards, find the probability that the card has a value 
		\begin{enumerate}
			\item 7
			\item greater than 7
			\item less than 7
		\end{enumerate}
		%\input{exemplar/10/13/3/30/main.tex}
  \item A Lot consists of 48 mobile phones of which 42 are good, 3 have only minor defects and 3 have major defects.Varnika will buy a phone if it is good but the trader will only buy a mobile if it has no major defects. One phone is selected at random from the lot. What is the probability that it is
\begin{enumerate}
	\item acceptable to Varnika?
            \item acceptable to the trader?
\end{enumerate}
\solution
	%\input{exemplar/10/13/3/40/main.tex}
 \item A student says that if you throw a die, it will show up 1 or not 1. Therefore, the probability of getting 1 and the probability of getting 'not 1' each is equal to $\frac{1}{2}$. Is this correct? Give reasons.\\
 \solution
        %\input{exemplar/10/13/2/9/main.tex}
   \item Four candidates A, B, C, D have ap-
plied for the assignment to coach a school cricket
team. If A is twice as likely to be selected as B, and
B and C are given about the same chance of being
selected, while C is twice as likely to be selected
as D, what are the probabilities that
\begin{enumerate}
\item C will be selected?
\item A will not be selected?
\end{enumerate}
	%\input{exemplar/11/16/3/9/main.tex}
 \item A bag contain 24 balls of which $x$ balls are red, $2x$ are white and $3x$ are blue. A ball is selected at random, What is the probability that it is
\begin{enumerate}[label=\alph*)]
\item not red ?
\item white ?
\end{enumerate}
%\input{exemplar/10/13/3/41/main.tex}
If the letters of the word ASSASSINATION are arranged at random. Find the Probability that
\begin{enumerate}[label=(\alph*)]
\item Four $S's$ come consecutively in the word
\item Two  $I's$ and two $N's$ come together
\item All $A's$ are not coming together
\item No two $A's$ are coming together
\end{enumerate}
%\input{exemplar/11/16/3/14/main.tex}
	\item One urn contains two black balls (labelled B1 and B2) and one white ball. A
	second urn contains one black ball and two white balls (labelled W1 and W2).
	Suppose the following experiment is performed. One of the two urns is chosen
	at random. Next a ball is randomly chosen from the urn. Then a second ball is
	chosen at random from the same urn without replacing the first ball.
	
	\begin{enumerate}
	\item What is the probability that two black balls are chosen?
	
	\item What is the probability that two balls of opposite colour are chosen?
	\end{enumerate}
	\solution
	%\input{exemplar/11/16/3/12/main1.tex}
\end{enumerate}

	\item 
The number lock of a suitcase has 4 wheels each labelled with ten digits i.e. from 0 to 9.The lock opens with a sequence of four digits with no repeats.What is the probability of a person getting the right sequence to open the suitcase.
\\
\solution
		%\begin{enumerate}[label=\thesection.\arabic*,ref=\thesection.\theenumi]
	\item One card is drawn from a well-shuffled deck of 52 cards. Find the probability of getting
\begin{enumerate}
\item A king of red colour 
\item A face card 
\item A red face card
\item The jack of hearts
\item A spade
\item The queen of diamonds

\end{enumerate}
\solution
		%\input{ncert/10/15/1/14/main.tex}
	\item Five cards—the ten, jack, queen, king and ace of diamonds, are well-shuffled with their face downwards. One card is then picked up at random.
\begin{enumerate}
\item
What is the probability that the card is the queen? 
\item
If the queen is drawn and put aside, what is the probability that the second card picked up is (a) an ace? (b) a queen?\\
\end{enumerate}
\solution
		%\input{ncert/10/15/1/15/defs.tex}
	\item A bag contains $5$ red balls and some blue balls. If the probability of drawing a blue ball is double that if a red ball, determine the number of blue balls in the bag. 
		\\
\solution
		%\input{ncert/10/15/2/3/defs.tex}
	\item A card is selected from a pack of 52 cards.
 \begin{enumerate}[label=(\alph*)] 
                 \item How many points are there in the sample space?
                 \item Calculate the probability that the card is an ace of spades.
                 \item Calculate the probability that the card is (i) an ace and (ii) black card.
 \end{enumerate}
\solution
		%\input{ncert/11/16/3/4/main.tex}
\item Four cards are drawn from a well-shuffled deck of 52 cards. What is the probability of obtaining 3 diamonds and one spade.
\\
\solution
		%\input{ncert/11/16/4/2/defs.tex}
\item In a certain lottery 10,000 tickets are sold and ten equal prizes are awarded. What is the probability of not getting a prize if you buy (a) one ticket (b) two tickets (c) 10 tickets ?	
\\
\solution
		%\input{ncert/11/16/4/4/defs.tex}
		%
\item 
Out of 100 students, two sections of 40 and 60 are formed. If you and your friend are among the 100 students, what is the probability that
\begin{enumerate}
\item you both enter the same section?
\item you both enter the different sections?
\end{enumerate}
\solution
		%\input{ncert/11/16/4/5/defs.tex}
	\item 
The number lock of a suitcase has 4 wheels each labelled with ten digits i.e. from 0 to 9.The lock opens with a sequence of four digits with no repeats.What is the probability of a person getting the right sequence to open the suitcase.
\\
\solution
		%\input{ncert/11/16/4/10/defs.tex}
		%
\item 
Two cards are drawn at random and without replacement from a pack of 52 playing cards. Find the probability that both the cards are black.
\\
\solution
		%\input{ncert/12/13/2/2/defs.tex}
		\item A box of oranges is inspected by examining three randomly selected oranges drawn without replacement. If all the three oranges are good, the box is approved for sale, otherwise, it is rejected. Find the probability that a box containing 15 oranges out of which 12 are good and 3 are bad ones will be approved for sale.
		\label{ncert/12/13/2/3/defs.tex}
		\item Two balls are drawn at random with replacement from a box containing 10 black and 8 red balls. Find the probability that
		\label{ncert/12/13/2/12}
\begin{enumerate}
\item both balls are red.
\item first ball is black and second is red.
\item one of them is black and other is red.
\end{enumerate}

\item In a hostel, 60\% of the students read Hindi newspaper, 40\% read English newspaper and 20\% read both Hindi and English newspapers. A student is selected at random.
		\label{ncert/12/13/2/15}
\begin{enumerate}
\item Find the probability that she reads neither Hindi nor English newspapers.
\item If she reads Hindi newspaper, find the probability that she reads English newspaper.
\item If she reads English newspaper, find the probability that she reads Hindi newspaper.\\
\end{enumerate}
\item The probability of obtaining an even prime number on each die, when a pair of dice is rolled is 
\begin{enumerate}
    \item $0$ 
    
    \item $\frac{1}{3}$ 
    
    \item $\frac{1}{12}$ 
    
    \item $\frac{1}{36}$ 
\end{enumerate}
\solution
		%\input{ncert/12/13/2/17/defs.tex}
	\item A bag contains 4 red and 4 black balls, another bag contains 2 red and 6 black balls. One of the two bags is selected at random and a ball is drawn from the bag which is found to be red. Find the probability that the ball is drawn from the first bag.
\\
\solution
		%\input{ncert/12/13/3/2/main.tex}
  \item
  Cards with numbers 2 to 101 are placed in a box. A card is selected at random.Find the probability that the card has
\begin{enumerate}[label=(\roman*)]
	\item an even number 
	\item a square number
\end{enumerate}
\solution
%\input{exemplar/10/13/3/32/main.tex}
\item
The king, queen and jack of clubs are removed from a deck of 52 playing cards and then well shuffled. Now one card is drawn at random from the remaining cards.  Determine the probability that the card is
\begin{enumerate}[label=(\roman*)]
\item a club
\item 10 of hearts
\end{enumerate}
\solution
%\input{exemplar/10/13/3/29/main.tex}
\item A team of medical students doing their internship have to assist during surgeries
at a city hospital. The probabilities of surgeries rated as very complex, complex,
routine, simple or very simple are respectively, 0.15, 0.20, 0.31, 0.26, .08. Find
the probabilities that a particular surgery will be rated
\begin{enumerate}
	\item complex or very complex;
	\item neither very complex nor very simple;
	\item routine or complex
	\item routine or simple
\end{enumerate}
\solution
%\input{exemplar/11/16/3/8(1)/main.tex}
\item A card is selected from a pack of 52 cards.
\begin{enumerate}[label=(\alph*)]
    \item How many points are there in the sample space?
    \item Calculate the probability that the card is an ace of spades.
    \item Calculate the probability that the card is (i) an ace and (ii) black card.
\end{enumerate}
\solution
%\input{exemplar/11/16/3/4/main2.tex}
\item The probability that a non leap year selected at random will contain 53 sundays.
\\
\solution
%\input{exemplar/10/13/1/19/main.tex}
\item One of the four persons John, Rita, Aslam or Gurpreet will be promoted next
month. Consequently the sample space consists of four elementary outcomes
S = {John promoted, Rita promoted, Aslam promoted, Gurpreet promoted}
You are told that the chances of John’s promotion is same as that of Gurpreet,
Rita’s chances of promotion are twice as likely as Johns. Aslam’s chances are
four times that of John.
\begin{enumerate}
	\item Determine
	\begin{enumerate}
		\item P (John promoted)
		\item P (Rita promoted)
		\item P (Aslam promoted)
		\item P (Gurpreet promoted)
	\end{enumerate}
	\item If A = {John promoted or Gurpreet promoted}, find P (A).
\end{enumerate}
\solution
%\input{exemplar/11/16/3/10/main.tex}
\item A card is drawn from a deck of 52 cards. Find the probability of getting a king or a heart or a red card.\\
\solution
%\input{exemplar/11/16/3/15/main.tex}
\item The probability that a student will pass his examination is 0.73, the probability of
the student getting a compartment is 0.13, and the probability that the student will
either pass or get compartment is 0.96. State True or False.\\
\solution
%\input{exemplar/11/16/3/31/main.tex}
\item A card is selected from a pack of 52 cards\\
\begin{enumerate}[label=(\alph*)]
\item How many points are there in the sample space?
\item Calculate the probability that the cards is an ace of spades.
\item Calculate the probability that the card is (i) an ace (ii)black card.\\
\end{enumerate}
%\input{ncert/11/16/3/4_1/Prob_4.tex}
\item In a non-leap year, the probability of having 53 tuesdays or 53 wednesdays is\\
\solution
%\input{exemplar/11/16/3/18/main.tex}
\item There are 1000 sealed envelopes in a box, 10 of them contain a cash prize of
Rs 100 each, 100 of them contain a cash prize of Rs 50 each and 200 of them
contain a cash prize of Rs 10 each and rest do not contain any cash prize. If they
are well shuffled and an envelope is picked up out, what is the probability that it
contains no cash prize?\\
\solution
%\input{exemplar/10/13/3/34/main.tex}
\item 
A die is thrown and a card is selected at random from a deck of 52 playing cards. The probability of getting an even number on the die and a spade card.\\
\solution
%\input{exemplar/12/13/3/78/main.tex}
\item
If 4-digit numbers greater than 5,000 are randomly formed from the digits 0, 1, 3, 5, and 7, what is the probability of forming a number divisible by 5 when:
\begin{enumerate}
    \item The digits are repeated?
    \item The repetition of digits is not allowed?
\end{enumerate}
\solution
%\input{ncert/11/16/4/9/main.tex}
\item Consider the probability space $\brak{\Omega, \mathcal{G}, P}$ where $\Omega = [0,2]$ and $\mathcal{G} = \cbrak{\phi, \Omega, [0,1], (1,2]}$. Let $X$ and $Y$ be two functions on $\Omega$ defined as
\begin{align*}
    X(\omega) = 
    \begin{cases}
        1 & \text{if }\omega \in [0, 1]\\
        2 & \text{if }\omega \in (1, 2]
    \end{cases}
\end{align*}
and
\begin{align*}
    Y(\omega) = 
    \begin{cases}
        2 & \text{if }\omega \in [0, 1.5]\\
        3 & \text{if }\omega \in (1.5, 2].
    \end{cases}
\end{align*}
Then which one of the following statements is true?
\begin{enumerate}
    \item [(A)] $X$ is a random variable with respect to $\mathcal{G}$, but $Y$ is not a random variable with respect to $\mathcal{G}$.
    \item [(B)] $Y$ is a random variable with respect to $\mathcal{G}$, but $X$ is not a random variable with respect to $\mathcal{G}$.
    \item [(C)] Neither $X$ nor $Y$ is a random variable with respect to $\mathcal{G}$.
    \item [(D)] Both $X$ and $Y$ are random variables with respect to $\mathcal{G}$.
\end{enumerate} \hfill (GATE ST 2023)\\
\solution
%\input{gate/ST/2023/14/main.tex}
	\item  A die is loaded in such a way that each odd number is twice as likely to occur as
each even number. Find $P(G)$, where $G$ is the event that a number greater than
3 occurs on a single roll of the die.
\\
\solution
		%\input{exemplar/11/16/3/5/main.tex}
	\item All the jacks, queens and kings are removed from a deck of 52 playing cards. The remaining cards are well shuffled and then one card is drawn at random. Giving ace a value 1 similar value for other cards, find the probability that the card has a value 
		\begin{enumerate}
			\item 7
			\item greater than 7
			\item less than 7
		\end{enumerate}
		%\input{exemplar/10/13/3/30/main.tex}
  \item A Lot consists of 48 mobile phones of which 42 are good, 3 have only minor defects and 3 have major defects.Varnika will buy a phone if it is good but the trader will only buy a mobile if it has no major defects. One phone is selected at random from the lot. What is the probability that it is
\begin{enumerate}
	\item acceptable to Varnika?
            \item acceptable to the trader?
\end{enumerate}
\solution
	%\input{exemplar/10/13/3/40/main.tex}
 \item A student says that if you throw a die, it will show up 1 or not 1. Therefore, the probability of getting 1 and the probability of getting 'not 1' each is equal to $\frac{1}{2}$. Is this correct? Give reasons.\\
 \solution
        %\input{exemplar/10/13/2/9/main.tex}
   \item Four candidates A, B, C, D have ap-
plied for the assignment to coach a school cricket
team. If A is twice as likely to be selected as B, and
B and C are given about the same chance of being
selected, while C is twice as likely to be selected
as D, what are the probabilities that
\begin{enumerate}
\item C will be selected?
\item A will not be selected?
\end{enumerate}
	%\input{exemplar/11/16/3/9/main.tex}
 \item A bag contain 24 balls of which $x$ balls are red, $2x$ are white and $3x$ are blue. A ball is selected at random, What is the probability that it is
\begin{enumerate}[label=\alph*)]
\item not red ?
\item white ?
\end{enumerate}
%\input{exemplar/10/13/3/41/main.tex}
If the letters of the word ASSASSINATION are arranged at random. Find the Probability that
\begin{enumerate}[label=(\alph*)]
\item Four $S's$ come consecutively in the word
\item Two  $I's$ and two $N's$ come together
\item All $A's$ are not coming together
\item No two $A's$ are coming together
\end{enumerate}
%\input{exemplar/11/16/3/14/main.tex}
	\item One urn contains two black balls (labelled B1 and B2) and one white ball. A
	second urn contains one black ball and two white balls (labelled W1 and W2).
	Suppose the following experiment is performed. One of the two urns is chosen
	at random. Next a ball is randomly chosen from the urn. Then a second ball is
	chosen at random from the same urn without replacing the first ball.
	
	\begin{enumerate}
	\item What is the probability that two black balls are chosen?
	
	\item What is the probability that two balls of opposite colour are chosen?
	\end{enumerate}
	\solution
	%\input{exemplar/11/16/3/12/main1.tex}
\end{enumerate}

		%
\item 
Two cards are drawn at random and without replacement from a pack of 52 playing cards. Find the probability that both the cards are black.
\\
\solution
		%\begin{enumerate}[label=\thesection.\arabic*,ref=\thesection.\theenumi]
	\item One card is drawn from a well-shuffled deck of 52 cards. Find the probability of getting
\begin{enumerate}
\item A king of red colour 
\item A face card 
\item A red face card
\item The jack of hearts
\item A spade
\item The queen of diamonds

\end{enumerate}
\solution
		%\input{ncert/10/15/1/14/main.tex}
	\item Five cards—the ten, jack, queen, king and ace of diamonds, are well-shuffled with their face downwards. One card is then picked up at random.
\begin{enumerate}
\item
What is the probability that the card is the queen? 
\item
If the queen is drawn and put aside, what is the probability that the second card picked up is (a) an ace? (b) a queen?\\
\end{enumerate}
\solution
		%\input{ncert/10/15/1/15/defs.tex}
	\item A bag contains $5$ red balls and some blue balls. If the probability of drawing a blue ball is double that if a red ball, determine the number of blue balls in the bag. 
		\\
\solution
		%\input{ncert/10/15/2/3/defs.tex}
	\item A card is selected from a pack of 52 cards.
 \begin{enumerate}[label=(\alph*)] 
                 \item How many points are there in the sample space?
                 \item Calculate the probability that the card is an ace of spades.
                 \item Calculate the probability that the card is (i) an ace and (ii) black card.
 \end{enumerate}
\solution
		%\input{ncert/11/16/3/4/main.tex}
\item Four cards are drawn from a well-shuffled deck of 52 cards. What is the probability of obtaining 3 diamonds and one spade.
\\
\solution
		%\input{ncert/11/16/4/2/defs.tex}
\item In a certain lottery 10,000 tickets are sold and ten equal prizes are awarded. What is the probability of not getting a prize if you buy (a) one ticket (b) two tickets (c) 10 tickets ?	
\\
\solution
		%\input{ncert/11/16/4/4/defs.tex}
		%
\item 
Out of 100 students, two sections of 40 and 60 are formed. If you and your friend are among the 100 students, what is the probability that
\begin{enumerate}
\item you both enter the same section?
\item you both enter the different sections?
\end{enumerate}
\solution
		%\input{ncert/11/16/4/5/defs.tex}
	\item 
The number lock of a suitcase has 4 wheels each labelled with ten digits i.e. from 0 to 9.The lock opens with a sequence of four digits with no repeats.What is the probability of a person getting the right sequence to open the suitcase.
\\
\solution
		%\input{ncert/11/16/4/10/defs.tex}
		%
\item 
Two cards are drawn at random and without replacement from a pack of 52 playing cards. Find the probability that both the cards are black.
\\
\solution
		%\input{ncert/12/13/2/2/defs.tex}
		\item A box of oranges is inspected by examining three randomly selected oranges drawn without replacement. If all the three oranges are good, the box is approved for sale, otherwise, it is rejected. Find the probability that a box containing 15 oranges out of which 12 are good and 3 are bad ones will be approved for sale.
		\label{ncert/12/13/2/3/defs.tex}
		\item Two balls are drawn at random with replacement from a box containing 10 black and 8 red balls. Find the probability that
		\label{ncert/12/13/2/12}
\begin{enumerate}
\item both balls are red.
\item first ball is black and second is red.
\item one of them is black and other is red.
\end{enumerate}

\item In a hostel, 60\% of the students read Hindi newspaper, 40\% read English newspaper and 20\% read both Hindi and English newspapers. A student is selected at random.
		\label{ncert/12/13/2/15}
\begin{enumerate}
\item Find the probability that she reads neither Hindi nor English newspapers.
\item If she reads Hindi newspaper, find the probability that she reads English newspaper.
\item If she reads English newspaper, find the probability that she reads Hindi newspaper.\\
\end{enumerate}
\item The probability of obtaining an even prime number on each die, when a pair of dice is rolled is 
\begin{enumerate}
    \item $0$ 
    
    \item $\frac{1}{3}$ 
    
    \item $\frac{1}{12}$ 
    
    \item $\frac{1}{36}$ 
\end{enumerate}
\solution
		%\input{ncert/12/13/2/17/defs.tex}
	\item A bag contains 4 red and 4 black balls, another bag contains 2 red and 6 black balls. One of the two bags is selected at random and a ball is drawn from the bag which is found to be red. Find the probability that the ball is drawn from the first bag.
\\
\solution
		%\input{ncert/12/13/3/2/main.tex}
  \item
  Cards with numbers 2 to 101 are placed in a box. A card is selected at random.Find the probability that the card has
\begin{enumerate}[label=(\roman*)]
	\item an even number 
	\item a square number
\end{enumerate}
\solution
%\input{exemplar/10/13/3/32/main.tex}
\item
The king, queen and jack of clubs are removed from a deck of 52 playing cards and then well shuffled. Now one card is drawn at random from the remaining cards.  Determine the probability that the card is
\begin{enumerate}[label=(\roman*)]
\item a club
\item 10 of hearts
\end{enumerate}
\solution
%\input{exemplar/10/13/3/29/main.tex}
\item A team of medical students doing their internship have to assist during surgeries
at a city hospital. The probabilities of surgeries rated as very complex, complex,
routine, simple or very simple are respectively, 0.15, 0.20, 0.31, 0.26, .08. Find
the probabilities that a particular surgery will be rated
\begin{enumerate}
	\item complex or very complex;
	\item neither very complex nor very simple;
	\item routine or complex
	\item routine or simple
\end{enumerate}
\solution
%\input{exemplar/11/16/3/8(1)/main.tex}
\item A card is selected from a pack of 52 cards.
\begin{enumerate}[label=(\alph*)]
    \item How many points are there in the sample space?
    \item Calculate the probability that the card is an ace of spades.
    \item Calculate the probability that the card is (i) an ace and (ii) black card.
\end{enumerate}
\solution
%\input{exemplar/11/16/3/4/main2.tex}
\item The probability that a non leap year selected at random will contain 53 sundays.
\\
\solution
%\input{exemplar/10/13/1/19/main.tex}
\item One of the four persons John, Rita, Aslam or Gurpreet will be promoted next
month. Consequently the sample space consists of four elementary outcomes
S = {John promoted, Rita promoted, Aslam promoted, Gurpreet promoted}
You are told that the chances of John’s promotion is same as that of Gurpreet,
Rita’s chances of promotion are twice as likely as Johns. Aslam’s chances are
four times that of John.
\begin{enumerate}
	\item Determine
	\begin{enumerate}
		\item P (John promoted)
		\item P (Rita promoted)
		\item P (Aslam promoted)
		\item P (Gurpreet promoted)
	\end{enumerate}
	\item If A = {John promoted or Gurpreet promoted}, find P (A).
\end{enumerate}
\solution
%\input{exemplar/11/16/3/10/main.tex}
\item A card is drawn from a deck of 52 cards. Find the probability of getting a king or a heart or a red card.\\
\solution
%\input{exemplar/11/16/3/15/main.tex}
\item The probability that a student will pass his examination is 0.73, the probability of
the student getting a compartment is 0.13, and the probability that the student will
either pass or get compartment is 0.96. State True or False.\\
\solution
%\input{exemplar/11/16/3/31/main.tex}
\item A card is selected from a pack of 52 cards\\
\begin{enumerate}[label=(\alph*)]
\item How many points are there in the sample space?
\item Calculate the probability that the cards is an ace of spades.
\item Calculate the probability that the card is (i) an ace (ii)black card.\\
\end{enumerate}
%\input{ncert/11/16/3/4_1/Prob_4.tex}
\item In a non-leap year, the probability of having 53 tuesdays or 53 wednesdays is\\
\solution
%\input{exemplar/11/16/3/18/main.tex}
\item There are 1000 sealed envelopes in a box, 10 of them contain a cash prize of
Rs 100 each, 100 of them contain a cash prize of Rs 50 each and 200 of them
contain a cash prize of Rs 10 each and rest do not contain any cash prize. If they
are well shuffled and an envelope is picked up out, what is the probability that it
contains no cash prize?\\
\solution
%\input{exemplar/10/13/3/34/main.tex}
\item 
A die is thrown and a card is selected at random from a deck of 52 playing cards. The probability of getting an even number on the die and a spade card.\\
\solution
%\input{exemplar/12/13/3/78/main.tex}
\item
If 4-digit numbers greater than 5,000 are randomly formed from the digits 0, 1, 3, 5, and 7, what is the probability of forming a number divisible by 5 when:
\begin{enumerate}
    \item The digits are repeated?
    \item The repetition of digits is not allowed?
\end{enumerate}
\solution
%\input{ncert/11/16/4/9/main.tex}
\item Consider the probability space $\brak{\Omega, \mathcal{G}, P}$ where $\Omega = [0,2]$ and $\mathcal{G} = \cbrak{\phi, \Omega, [0,1], (1,2]}$. Let $X$ and $Y$ be two functions on $\Omega$ defined as
\begin{align*}
    X(\omega) = 
    \begin{cases}
        1 & \text{if }\omega \in [0, 1]\\
        2 & \text{if }\omega \in (1, 2]
    \end{cases}
\end{align*}
and
\begin{align*}
    Y(\omega) = 
    \begin{cases}
        2 & \text{if }\omega \in [0, 1.5]\\
        3 & \text{if }\omega \in (1.5, 2].
    \end{cases}
\end{align*}
Then which one of the following statements is true?
\begin{enumerate}
    \item [(A)] $X$ is a random variable with respect to $\mathcal{G}$, but $Y$ is not a random variable with respect to $\mathcal{G}$.
    \item [(B)] $Y$ is a random variable with respect to $\mathcal{G}$, but $X$ is not a random variable with respect to $\mathcal{G}$.
    \item [(C)] Neither $X$ nor $Y$ is a random variable with respect to $\mathcal{G}$.
    \item [(D)] Both $X$ and $Y$ are random variables with respect to $\mathcal{G}$.
\end{enumerate} \hfill (GATE ST 2023)\\
\solution
%\input{gate/ST/2023/14/main.tex}
	\item  A die is loaded in such a way that each odd number is twice as likely to occur as
each even number. Find $P(G)$, where $G$ is the event that a number greater than
3 occurs on a single roll of the die.
\\
\solution
		%\input{exemplar/11/16/3/5/main.tex}
	\item All the jacks, queens and kings are removed from a deck of 52 playing cards. The remaining cards are well shuffled and then one card is drawn at random. Giving ace a value 1 similar value for other cards, find the probability that the card has a value 
		\begin{enumerate}
			\item 7
			\item greater than 7
			\item less than 7
		\end{enumerate}
		%\input{exemplar/10/13/3/30/main.tex}
  \item A Lot consists of 48 mobile phones of which 42 are good, 3 have only minor defects and 3 have major defects.Varnika will buy a phone if it is good but the trader will only buy a mobile if it has no major defects. One phone is selected at random from the lot. What is the probability that it is
\begin{enumerate}
	\item acceptable to Varnika?
            \item acceptable to the trader?
\end{enumerate}
\solution
	%\input{exemplar/10/13/3/40/main.tex}
 \item A student says that if you throw a die, it will show up 1 or not 1. Therefore, the probability of getting 1 and the probability of getting 'not 1' each is equal to $\frac{1}{2}$. Is this correct? Give reasons.\\
 \solution
        %\input{exemplar/10/13/2/9/main.tex}
   \item Four candidates A, B, C, D have ap-
plied for the assignment to coach a school cricket
team. If A is twice as likely to be selected as B, and
B and C are given about the same chance of being
selected, while C is twice as likely to be selected
as D, what are the probabilities that
\begin{enumerate}
\item C will be selected?
\item A will not be selected?
\end{enumerate}
	%\input{exemplar/11/16/3/9/main.tex}
 \item A bag contain 24 balls of which $x$ balls are red, $2x$ are white and $3x$ are blue. A ball is selected at random, What is the probability that it is
\begin{enumerate}[label=\alph*)]
\item not red ?
\item white ?
\end{enumerate}
%\input{exemplar/10/13/3/41/main.tex}
If the letters of the word ASSASSINATION are arranged at random. Find the Probability that
\begin{enumerate}[label=(\alph*)]
\item Four $S's$ come consecutively in the word
\item Two  $I's$ and two $N's$ come together
\item All $A's$ are not coming together
\item No two $A's$ are coming together
\end{enumerate}
%\input{exemplar/11/16/3/14/main.tex}
	\item One urn contains two black balls (labelled B1 and B2) and one white ball. A
	second urn contains one black ball and two white balls (labelled W1 and W2).
	Suppose the following experiment is performed. One of the two urns is chosen
	at random. Next a ball is randomly chosen from the urn. Then a second ball is
	chosen at random from the same urn without replacing the first ball.
	
	\begin{enumerate}
	\item What is the probability that two black balls are chosen?
	
	\item What is the probability that two balls of opposite colour are chosen?
	\end{enumerate}
	\solution
	%\input{exemplar/11/16/3/12/main1.tex}
\end{enumerate}

		\item A box of oranges is inspected by examining three randomly selected oranges drawn without replacement. If all the three oranges are good, the box is approved for sale, otherwise, it is rejected. Find the probability that a box containing 15 oranges out of which 12 are good and 3 are bad ones will be approved for sale.
		\label{ncert/12/13/2/3/defs.tex}
		\item Two balls are drawn at random with replacement from a box containing 10 black and 8 red balls. Find the probability that
		\label{ncert/12/13/2/12}
\begin{enumerate}
\item both balls are red.
\item first ball is black and second is red.
\item one of them is black and other is red.
\end{enumerate}

\item In a hostel, 60\% of the students read Hindi newspaper, 40\% read English newspaper and 20\% read both Hindi and English newspapers. A student is selected at random.
		\label{ncert/12/13/2/15}
\begin{enumerate}
\item Find the probability that she reads neither Hindi nor English newspapers.
\item If she reads Hindi newspaper, find the probability that she reads English newspaper.
\item If she reads English newspaper, find the probability that she reads Hindi newspaper.\\
\end{enumerate}
\item The probability of obtaining an even prime number on each die, when a pair of dice is rolled is 
\begin{enumerate}
    \item $0$ 
    
    \item $\frac{1}{3}$ 
    
    \item $\frac{1}{12}$ 
    
    \item $\frac{1}{36}$ 
\end{enumerate}
\solution
		%\begin{enumerate}[label=\thesection.\arabic*,ref=\thesection.\theenumi]
	\item One card is drawn from a well-shuffled deck of 52 cards. Find the probability of getting
\begin{enumerate}
\item A king of red colour 
\item A face card 
\item A red face card
\item The jack of hearts
\item A spade
\item The queen of diamonds

\end{enumerate}
\solution
		%\input{ncert/10/15/1/14/main.tex}
	\item Five cards—the ten, jack, queen, king and ace of diamonds, are well-shuffled with their face downwards. One card is then picked up at random.
\begin{enumerate}
\item
What is the probability that the card is the queen? 
\item
If the queen is drawn and put aside, what is the probability that the second card picked up is (a) an ace? (b) a queen?\\
\end{enumerate}
\solution
		%\input{ncert/10/15/1/15/defs.tex}
	\item A bag contains $5$ red balls and some blue balls. If the probability of drawing a blue ball is double that if a red ball, determine the number of blue balls in the bag. 
		\\
\solution
		%\input{ncert/10/15/2/3/defs.tex}
	\item A card is selected from a pack of 52 cards.
 \begin{enumerate}[label=(\alph*)] 
                 \item How many points are there in the sample space?
                 \item Calculate the probability that the card is an ace of spades.
                 \item Calculate the probability that the card is (i) an ace and (ii) black card.
 \end{enumerate}
\solution
		%\input{ncert/11/16/3/4/main.tex}
\item Four cards are drawn from a well-shuffled deck of 52 cards. What is the probability of obtaining 3 diamonds and one spade.
\\
\solution
		%\input{ncert/11/16/4/2/defs.tex}
\item In a certain lottery 10,000 tickets are sold and ten equal prizes are awarded. What is the probability of not getting a prize if you buy (a) one ticket (b) two tickets (c) 10 tickets ?	
\\
\solution
		%\input{ncert/11/16/4/4/defs.tex}
		%
\item 
Out of 100 students, two sections of 40 and 60 are formed. If you and your friend are among the 100 students, what is the probability that
\begin{enumerate}
\item you both enter the same section?
\item you both enter the different sections?
\end{enumerate}
\solution
		%\input{ncert/11/16/4/5/defs.tex}
	\item 
The number lock of a suitcase has 4 wheels each labelled with ten digits i.e. from 0 to 9.The lock opens with a sequence of four digits with no repeats.What is the probability of a person getting the right sequence to open the suitcase.
\\
\solution
		%\input{ncert/11/16/4/10/defs.tex}
		%
\item 
Two cards are drawn at random and without replacement from a pack of 52 playing cards. Find the probability that both the cards are black.
\\
\solution
		%\input{ncert/12/13/2/2/defs.tex}
		\item A box of oranges is inspected by examining three randomly selected oranges drawn without replacement. If all the three oranges are good, the box is approved for sale, otherwise, it is rejected. Find the probability that a box containing 15 oranges out of which 12 are good and 3 are bad ones will be approved for sale.
		\label{ncert/12/13/2/3/defs.tex}
		\item Two balls are drawn at random with replacement from a box containing 10 black and 8 red balls. Find the probability that
		\label{ncert/12/13/2/12}
\begin{enumerate}
\item both balls are red.
\item first ball is black and second is red.
\item one of them is black and other is red.
\end{enumerate}

\item In a hostel, 60\% of the students read Hindi newspaper, 40\% read English newspaper and 20\% read both Hindi and English newspapers. A student is selected at random.
		\label{ncert/12/13/2/15}
\begin{enumerate}
\item Find the probability that she reads neither Hindi nor English newspapers.
\item If she reads Hindi newspaper, find the probability that she reads English newspaper.
\item If she reads English newspaper, find the probability that she reads Hindi newspaper.\\
\end{enumerate}
\item The probability of obtaining an even prime number on each die, when a pair of dice is rolled is 
\begin{enumerate}
    \item $0$ 
    
    \item $\frac{1}{3}$ 
    
    \item $\frac{1}{12}$ 
    
    \item $\frac{1}{36}$ 
\end{enumerate}
\solution
		%\input{ncert/12/13/2/17/defs.tex}
	\item A bag contains 4 red and 4 black balls, another bag contains 2 red and 6 black balls. One of the two bags is selected at random and a ball is drawn from the bag which is found to be red. Find the probability that the ball is drawn from the first bag.
\\
\solution
		%\input{ncert/12/13/3/2/main.tex}
  \item
  Cards with numbers 2 to 101 are placed in a box. A card is selected at random.Find the probability that the card has
\begin{enumerate}[label=(\roman*)]
	\item an even number 
	\item a square number
\end{enumerate}
\solution
%\input{exemplar/10/13/3/32/main.tex}
\item
The king, queen and jack of clubs are removed from a deck of 52 playing cards and then well shuffled. Now one card is drawn at random from the remaining cards.  Determine the probability that the card is
\begin{enumerate}[label=(\roman*)]
\item a club
\item 10 of hearts
\end{enumerate}
\solution
%\input{exemplar/10/13/3/29/main.tex}
\item A team of medical students doing their internship have to assist during surgeries
at a city hospital. The probabilities of surgeries rated as very complex, complex,
routine, simple or very simple are respectively, 0.15, 0.20, 0.31, 0.26, .08. Find
the probabilities that a particular surgery will be rated
\begin{enumerate}
	\item complex or very complex;
	\item neither very complex nor very simple;
	\item routine or complex
	\item routine or simple
\end{enumerate}
\solution
%\input{exemplar/11/16/3/8(1)/main.tex}
\item A card is selected from a pack of 52 cards.
\begin{enumerate}[label=(\alph*)]
    \item How many points are there in the sample space?
    \item Calculate the probability that the card is an ace of spades.
    \item Calculate the probability that the card is (i) an ace and (ii) black card.
\end{enumerate}
\solution
%\input{exemplar/11/16/3/4/main2.tex}
\item The probability that a non leap year selected at random will contain 53 sundays.
\\
\solution
%\input{exemplar/10/13/1/19/main.tex}
\item One of the four persons John, Rita, Aslam or Gurpreet will be promoted next
month. Consequently the sample space consists of four elementary outcomes
S = {John promoted, Rita promoted, Aslam promoted, Gurpreet promoted}
You are told that the chances of John’s promotion is same as that of Gurpreet,
Rita’s chances of promotion are twice as likely as Johns. Aslam’s chances are
four times that of John.
\begin{enumerate}
	\item Determine
	\begin{enumerate}
		\item P (John promoted)
		\item P (Rita promoted)
		\item P (Aslam promoted)
		\item P (Gurpreet promoted)
	\end{enumerate}
	\item If A = {John promoted or Gurpreet promoted}, find P (A).
\end{enumerate}
\solution
%\input{exemplar/11/16/3/10/main.tex}
\item A card is drawn from a deck of 52 cards. Find the probability of getting a king or a heart or a red card.\\
\solution
%\input{exemplar/11/16/3/15/main.tex}
\item The probability that a student will pass his examination is 0.73, the probability of
the student getting a compartment is 0.13, and the probability that the student will
either pass or get compartment is 0.96. State True or False.\\
\solution
%\input{exemplar/11/16/3/31/main.tex}
\item A card is selected from a pack of 52 cards\\
\begin{enumerate}[label=(\alph*)]
\item How many points are there in the sample space?
\item Calculate the probability that the cards is an ace of spades.
\item Calculate the probability that the card is (i) an ace (ii)black card.\\
\end{enumerate}
%\input{ncert/11/16/3/4_1/Prob_4.tex}
\item In a non-leap year, the probability of having 53 tuesdays or 53 wednesdays is\\
\solution
%\input{exemplar/11/16/3/18/main.tex}
\item There are 1000 sealed envelopes in a box, 10 of them contain a cash prize of
Rs 100 each, 100 of them contain a cash prize of Rs 50 each and 200 of them
contain a cash prize of Rs 10 each and rest do not contain any cash prize. If they
are well shuffled and an envelope is picked up out, what is the probability that it
contains no cash prize?\\
\solution
%\input{exemplar/10/13/3/34/main.tex}
\item 
A die is thrown and a card is selected at random from a deck of 52 playing cards. The probability of getting an even number on the die and a spade card.\\
\solution
%\input{exemplar/12/13/3/78/main.tex}
\item
If 4-digit numbers greater than 5,000 are randomly formed from the digits 0, 1, 3, 5, and 7, what is the probability of forming a number divisible by 5 when:
\begin{enumerate}
    \item The digits are repeated?
    \item The repetition of digits is not allowed?
\end{enumerate}
\solution
%\input{ncert/11/16/4/9/main.tex}
\item Consider the probability space $\brak{\Omega, \mathcal{G}, P}$ where $\Omega = [0,2]$ and $\mathcal{G} = \cbrak{\phi, \Omega, [0,1], (1,2]}$. Let $X$ and $Y$ be two functions on $\Omega$ defined as
\begin{align*}
    X(\omega) = 
    \begin{cases}
        1 & \text{if }\omega \in [0, 1]\\
        2 & \text{if }\omega \in (1, 2]
    \end{cases}
\end{align*}
and
\begin{align*}
    Y(\omega) = 
    \begin{cases}
        2 & \text{if }\omega \in [0, 1.5]\\
        3 & \text{if }\omega \in (1.5, 2].
    \end{cases}
\end{align*}
Then which one of the following statements is true?
\begin{enumerate}
    \item [(A)] $X$ is a random variable with respect to $\mathcal{G}$, but $Y$ is not a random variable with respect to $\mathcal{G}$.
    \item [(B)] $Y$ is a random variable with respect to $\mathcal{G}$, but $X$ is not a random variable with respect to $\mathcal{G}$.
    \item [(C)] Neither $X$ nor $Y$ is a random variable with respect to $\mathcal{G}$.
    \item [(D)] Both $X$ and $Y$ are random variables with respect to $\mathcal{G}$.
\end{enumerate} \hfill (GATE ST 2023)\\
\solution
%\input{gate/ST/2023/14/main.tex}
	\item  A die is loaded in such a way that each odd number is twice as likely to occur as
each even number. Find $P(G)$, where $G$ is the event that a number greater than
3 occurs on a single roll of the die.
\\
\solution
		%\input{exemplar/11/16/3/5/main.tex}
	\item All the jacks, queens and kings are removed from a deck of 52 playing cards. The remaining cards are well shuffled and then one card is drawn at random. Giving ace a value 1 similar value for other cards, find the probability that the card has a value 
		\begin{enumerate}
			\item 7
			\item greater than 7
			\item less than 7
		\end{enumerate}
		%\input{exemplar/10/13/3/30/main.tex}
  \item A Lot consists of 48 mobile phones of which 42 are good, 3 have only minor defects and 3 have major defects.Varnika will buy a phone if it is good but the trader will only buy a mobile if it has no major defects. One phone is selected at random from the lot. What is the probability that it is
\begin{enumerate}
	\item acceptable to Varnika?
            \item acceptable to the trader?
\end{enumerate}
\solution
	%\input{exemplar/10/13/3/40/main.tex}
 \item A student says that if you throw a die, it will show up 1 or not 1. Therefore, the probability of getting 1 and the probability of getting 'not 1' each is equal to $\frac{1}{2}$. Is this correct? Give reasons.\\
 \solution
        %\input{exemplar/10/13/2/9/main.tex}
   \item Four candidates A, B, C, D have ap-
plied for the assignment to coach a school cricket
team. If A is twice as likely to be selected as B, and
B and C are given about the same chance of being
selected, while C is twice as likely to be selected
as D, what are the probabilities that
\begin{enumerate}
\item C will be selected?
\item A will not be selected?
\end{enumerate}
	%\input{exemplar/11/16/3/9/main.tex}
 \item A bag contain 24 balls of which $x$ balls are red, $2x$ are white and $3x$ are blue. A ball is selected at random, What is the probability that it is
\begin{enumerate}[label=\alph*)]
\item not red ?
\item white ?
\end{enumerate}
%\input{exemplar/10/13/3/41/main.tex}
If the letters of the word ASSASSINATION are arranged at random. Find the Probability that
\begin{enumerate}[label=(\alph*)]
\item Four $S's$ come consecutively in the word
\item Two  $I's$ and two $N's$ come together
\item All $A's$ are not coming together
\item No two $A's$ are coming together
\end{enumerate}
%\input{exemplar/11/16/3/14/main.tex}
	\item One urn contains two black balls (labelled B1 and B2) and one white ball. A
	second urn contains one black ball and two white balls (labelled W1 and W2).
	Suppose the following experiment is performed. One of the two urns is chosen
	at random. Next a ball is randomly chosen from the urn. Then a second ball is
	chosen at random from the same urn without replacing the first ball.
	
	\begin{enumerate}
	\item What is the probability that two black balls are chosen?
	
	\item What is the probability that two balls of opposite colour are chosen?
	\end{enumerate}
	\solution
	%\input{exemplar/11/16/3/12/main1.tex}
\end{enumerate}

	\item A bag contains 4 red and 4 black balls, another bag contains 2 red and 6 black balls. One of the two bags is selected at random and a ball is drawn from the bag which is found to be red. Find the probability that the ball is drawn from the first bag.
\\
\solution
		%\begin{table}[H]
	\centering
\begin{tabular}{|c|c|c|}
\hline
Random variable &Value &Definition\\ \hline
\multirow{3}{*}{X} &0 &Slips of Rs 1\\
&1 &Slips of Rs 5\\
&2 &Slips of Rs 13\\ \hline
\multirow{2}{*}{Y} &0 &Box A\\
&1 &Box B\\\hline
\end{tabular}
\caption{}
\label{tab:Distribution}
\end{table}
See \tabref{tab:Distribution}.
\begin{align}
p_{Y}\brak{k}= \begin{cases} 
      \frac{1}{3} & {k=0} \\
      \frac{2}{3 }& {k=1} 
   \end{cases}
   \\
p_{Y|X}\brak{0|0} = \frac{19}{25}\, 
p_{Y|X}\brak{0|1} = \frac{6}{25}\,
p_{Y|X}\brak{1|0} = \frac{45}{50}\,
p_{Y|X}\brak{1|2} = \frac{5}{50}
\end{align}
The desired probability is the probability that a slip drawn at random is marked other than Rs 1,
\begin{align}
&=1-p_X\brak{0}\\
&= p_X(1) + p_X(2)
\end{align}
Using Bayes theorem,
\begin{align}
&= p_Y\brak{0} \times \pr{Y=0 | X=1} + p_Y\brak{1} \times \pr{Y=1|X=2}\\
&=\frac{1}{3} \times \frac{6}{25} + \frac{2}{3} \times \frac{5}{50}\\
&=\frac{11}{75}
\end{align}

\newpage

%\tableofcontents

\bigskip

\renewcommand{\thefigure}{\theenumi}
\renewcommand{\thetable}{\theenumi}
%\renewcommand{\theequation}{\theenumi}

%\begin{abstract}
%%\boldmath
%In this letter, an algorithm for evaluating the exact analytical bit error rate  (BER)  for the piecewise linear (PL) combiner for  multiple relays is presented. Previous results were available only for upto three relays. The algorithm is unique in the sense that  the actual mathematical expressions, that are prohibitively large, need not be explicitly obtained. The diversity gain due to multiple relays is shown through plots of the analytical BER, well supported by simulations. 
%
%\end{abstract}
% IEEEtran.cls defaults to using nonbold math in the Abstract.
% This preserves the distinction between vectors and scalars. However,
% if the journal you are submitting to favors bold math in the abstract,
% then you can use LaTeX's standard command \boldmath at the very start
% of the abstract to achieve this. Many IEEE journals frown on math
% in the abstract anyway.

% Note that keywords are not normally used for peerreview papers.
%\begin{IEEEkeywords}
%Cooperative diversity, decode and forward, piecewise linear
%\end{IEEEkeywords}



% For peer review papers, you can put extra information on the cover
% page as needed:
% \ifCLASSOPTIONpeerreview
% \begin{center} \bfseries EDICS Category: 3-BBND \end{center}
% \fi
%
% For peerreview papers, this IEEEtran command inserts a page break and
% creates the second title. It will be ignored for other modes.
%\IEEEpeerreviewmaketitle




  \item
  Cards with numbers 2 to 101 are placed in a box. A card is selected at random.Find the probability that the card has
\begin{enumerate}[label=(\roman*)]
	\item an even number 
	\item a square number
\end{enumerate}
\solution
%\begin{table}[H]
	\centering
\begin{tabular}{|c|c|c|}
\hline
Random variable &Value &Definition\\ \hline
\multirow{3}{*}{X} &0 &Slips of Rs 1\\
&1 &Slips of Rs 5\\
&2 &Slips of Rs 13\\ \hline
\multirow{2}{*}{Y} &0 &Box A\\
&1 &Box B\\\hline
\end{tabular}
\caption{}
\label{tab:Distribution}
\end{table}
See \tabref{tab:Distribution}.
\begin{align}
p_{Y}\brak{k}= \begin{cases} 
      \frac{1}{3} & {k=0} \\
      \frac{2}{3 }& {k=1} 
   \end{cases}
   \\
p_{Y|X}\brak{0|0} = \frac{19}{25}\, 
p_{Y|X}\brak{0|1} = \frac{6}{25}\,
p_{Y|X}\brak{1|0} = \frac{45}{50}\,
p_{Y|X}\brak{1|2} = \frac{5}{50}
\end{align}
The desired probability is the probability that a slip drawn at random is marked other than Rs 1,
\begin{align}
&=1-p_X\brak{0}\\
&= p_X(1) + p_X(2)
\end{align}
Using Bayes theorem,
\begin{align}
&= p_Y\brak{0} \times \pr{Y=0 | X=1} + p_Y\brak{1} \times \pr{Y=1|X=2}\\
&=\frac{1}{3} \times \frac{6}{25} + \frac{2}{3} \times \frac{5}{50}\\
&=\frac{11}{75}
\end{align}

\newpage

%\tableofcontents

\bigskip

\renewcommand{\thefigure}{\theenumi}
\renewcommand{\thetable}{\theenumi}
%\renewcommand{\theequation}{\theenumi}

%\begin{abstract}
%%\boldmath
%In this letter, an algorithm for evaluating the exact analytical bit error rate  (BER)  for the piecewise linear (PL) combiner for  multiple relays is presented. Previous results were available only for upto three relays. The algorithm is unique in the sense that  the actual mathematical expressions, that are prohibitively large, need not be explicitly obtained. The diversity gain due to multiple relays is shown through plots of the analytical BER, well supported by simulations. 
%
%\end{abstract}
% IEEEtran.cls defaults to using nonbold math in the Abstract.
% This preserves the distinction between vectors and scalars. However,
% if the journal you are submitting to favors bold math in the abstract,
% then you can use LaTeX's standard command \boldmath at the very start
% of the abstract to achieve this. Many IEEE journals frown on math
% in the abstract anyway.

% Note that keywords are not normally used for peerreview papers.
%\begin{IEEEkeywords}
%Cooperative diversity, decode and forward, piecewise linear
%\end{IEEEkeywords}



% For peer review papers, you can put extra information on the cover
% page as needed:
% \ifCLASSOPTIONpeerreview
% \begin{center} \bfseries EDICS Category: 3-BBND \end{center}
% \fi
%
% For peerreview papers, this IEEEtran command inserts a page break and
% creates the second title. It will be ignored for other modes.
%\IEEEpeerreviewmaketitle




\item
The king, queen and jack of clubs are removed from a deck of 52 playing cards and then well shuffled. Now one card is drawn at random from the remaining cards.  Determine the probability that the card is
\begin{enumerate}[label=(\roman*)]
\item a club
\item 10 of hearts
\end{enumerate}
\solution
%\begin{table}[H]
	\centering
\begin{tabular}{|c|c|c|}
\hline
Random variable &Value &Definition\\ \hline
\multirow{3}{*}{X} &0 &Slips of Rs 1\\
&1 &Slips of Rs 5\\
&2 &Slips of Rs 13\\ \hline
\multirow{2}{*}{Y} &0 &Box A\\
&1 &Box B\\\hline
\end{tabular}
\caption{}
\label{tab:Distribution}
\end{table}
See \tabref{tab:Distribution}.
\begin{align}
p_{Y}\brak{k}= \begin{cases} 
      \frac{1}{3} & {k=0} \\
      \frac{2}{3 }& {k=1} 
   \end{cases}
   \\
p_{Y|X}\brak{0|0} = \frac{19}{25}\, 
p_{Y|X}\brak{0|1} = \frac{6}{25}\,
p_{Y|X}\brak{1|0} = \frac{45}{50}\,
p_{Y|X}\brak{1|2} = \frac{5}{50}
\end{align}
The desired probability is the probability that a slip drawn at random is marked other than Rs 1,
\begin{align}
&=1-p_X\brak{0}\\
&= p_X(1) + p_X(2)
\end{align}
Using Bayes theorem,
\begin{align}
&= p_Y\brak{0} \times \pr{Y=0 | X=1} + p_Y\brak{1} \times \pr{Y=1|X=2}\\
&=\frac{1}{3} \times \frac{6}{25} + \frac{2}{3} \times \frac{5}{50}\\
&=\frac{11}{75}
\end{align}

\newpage

%\tableofcontents

\bigskip

\renewcommand{\thefigure}{\theenumi}
\renewcommand{\thetable}{\theenumi}
%\renewcommand{\theequation}{\theenumi}

%\begin{abstract}
%%\boldmath
%In this letter, an algorithm for evaluating the exact analytical bit error rate  (BER)  for the piecewise linear (PL) combiner for  multiple relays is presented. Previous results were available only for upto three relays. The algorithm is unique in the sense that  the actual mathematical expressions, that are prohibitively large, need not be explicitly obtained. The diversity gain due to multiple relays is shown through plots of the analytical BER, well supported by simulations. 
%
%\end{abstract}
% IEEEtran.cls defaults to using nonbold math in the Abstract.
% This preserves the distinction between vectors and scalars. However,
% if the journal you are submitting to favors bold math in the abstract,
% then you can use LaTeX's standard command \boldmath at the very start
% of the abstract to achieve this. Many IEEE journals frown on math
% in the abstract anyway.

% Note that keywords are not normally used for peerreview papers.
%\begin{IEEEkeywords}
%Cooperative diversity, decode and forward, piecewise linear
%\end{IEEEkeywords}



% For peer review papers, you can put extra information on the cover
% page as needed:
% \ifCLASSOPTIONpeerreview
% \begin{center} \bfseries EDICS Category: 3-BBND \end{center}
% \fi
%
% For peerreview papers, this IEEEtran command inserts a page break and
% creates the second title. It will be ignored for other modes.
%\IEEEpeerreviewmaketitle




\item A team of medical students doing their internship have to assist during surgeries
at a city hospital. The probabilities of surgeries rated as very complex, complex,
routine, simple or very simple are respectively, 0.15, 0.20, 0.31, 0.26, .08. Find
the probabilities that a particular surgery will be rated
\begin{enumerate}
	\item complex or very complex;
	\item neither very complex nor very simple;
	\item routine or complex
	\item routine or simple
\end{enumerate}
\solution
%\begin{table}[H]
	\centering
\begin{tabular}{|c|c|c|}
\hline
Random variable &Value &Definition\\ \hline
\multirow{3}{*}{X} &0 &Slips of Rs 1\\
&1 &Slips of Rs 5\\
&2 &Slips of Rs 13\\ \hline
\multirow{2}{*}{Y} &0 &Box A\\
&1 &Box B\\\hline
\end{tabular}
\caption{}
\label{tab:Distribution}
\end{table}
See \tabref{tab:Distribution}.
\begin{align}
p_{Y}\brak{k}= \begin{cases} 
      \frac{1}{3} & {k=0} \\
      \frac{2}{3 }& {k=1} 
   \end{cases}
   \\
p_{Y|X}\brak{0|0} = \frac{19}{25}\, 
p_{Y|X}\brak{0|1} = \frac{6}{25}\,
p_{Y|X}\brak{1|0} = \frac{45}{50}\,
p_{Y|X}\brak{1|2} = \frac{5}{50}
\end{align}
The desired probability is the probability that a slip drawn at random is marked other than Rs 1,
\begin{align}
&=1-p_X\brak{0}\\
&= p_X(1) + p_X(2)
\end{align}
Using Bayes theorem,
\begin{align}
&= p_Y\brak{0} \times \pr{Y=0 | X=1} + p_Y\brak{1} \times \pr{Y=1|X=2}\\
&=\frac{1}{3} \times \frac{6}{25} + \frac{2}{3} \times \frac{5}{50}\\
&=\frac{11}{75}
\end{align}

\newpage

%\tableofcontents

\bigskip

\renewcommand{\thefigure}{\theenumi}
\renewcommand{\thetable}{\theenumi}
%\renewcommand{\theequation}{\theenumi}

%\begin{abstract}
%%\boldmath
%In this letter, an algorithm for evaluating the exact analytical bit error rate  (BER)  for the piecewise linear (PL) combiner for  multiple relays is presented. Previous results were available only for upto three relays. The algorithm is unique in the sense that  the actual mathematical expressions, that are prohibitively large, need not be explicitly obtained. The diversity gain due to multiple relays is shown through plots of the analytical BER, well supported by simulations. 
%
%\end{abstract}
% IEEEtran.cls defaults to using nonbold math in the Abstract.
% This preserves the distinction between vectors and scalars. However,
% if the journal you are submitting to favors bold math in the abstract,
% then you can use LaTeX's standard command \boldmath at the very start
% of the abstract to achieve this. Many IEEE journals frown on math
% in the abstract anyway.

% Note that keywords are not normally used for peerreview papers.
%\begin{IEEEkeywords}
%Cooperative diversity, decode and forward, piecewise linear
%\end{IEEEkeywords}



% For peer review papers, you can put extra information on the cover
% page as needed:
% \ifCLASSOPTIONpeerreview
% \begin{center} \bfseries EDICS Category: 3-BBND \end{center}
% \fi
%
% For peerreview papers, this IEEEtran command inserts a page break and
% creates the second title. It will be ignored for other modes.
%\IEEEpeerreviewmaketitle




\item A card is selected from a pack of 52 cards.
\begin{enumerate}[label=(\alph*)]
    \item How many points are there in the sample space?
    \item Calculate the probability that the card is an ace of spades.
    \item Calculate the probability that the card is (i) an ace and (ii) black card.
\end{enumerate}
\solution
%Let $X$ be an bernoulli rv defined as in \tabref{tab:exemplar/11/16/3/26}.  Then, 
\begin{equation}
    p =
        \frac{4}{11} 
\end{equation}
\begin{table}[H]
	\centering
	\input{exemplar/11/16/3/26/tables/Table2.tex}
	\caption{}
        \label{tab:exemplar/11/16/3/26}
\end{table}

\item The probability that a non leap year selected at random will contain 53 sundays.
\\
\solution
%\begin{table}[H]
	\centering
\begin{tabular}{|c|c|c|}
\hline
Random variable &Value &Definition\\ \hline
\multirow{3}{*}{X} &0 &Slips of Rs 1\\
&1 &Slips of Rs 5\\
&2 &Slips of Rs 13\\ \hline
\multirow{2}{*}{Y} &0 &Box A\\
&1 &Box B\\\hline
\end{tabular}
\caption{}
\label{tab:Distribution}
\end{table}
See \tabref{tab:Distribution}.
\begin{align}
p_{Y}\brak{k}= \begin{cases} 
      \frac{1}{3} & {k=0} \\
      \frac{2}{3 }& {k=1} 
   \end{cases}
   \\
p_{Y|X}\brak{0|0} = \frac{19}{25}\, 
p_{Y|X}\brak{0|1} = \frac{6}{25}\,
p_{Y|X}\brak{1|0} = \frac{45}{50}\,
p_{Y|X}\brak{1|2} = \frac{5}{50}
\end{align}
The desired probability is the probability that a slip drawn at random is marked other than Rs 1,
\begin{align}
&=1-p_X\brak{0}\\
&= p_X(1) + p_X(2)
\end{align}
Using Bayes theorem,
\begin{align}
&= p_Y\brak{0} \times \pr{Y=0 | X=1} + p_Y\brak{1} \times \pr{Y=1|X=2}\\
&=\frac{1}{3} \times \frac{6}{25} + \frac{2}{3} \times \frac{5}{50}\\
&=\frac{11}{75}
\end{align}

\newpage

%\tableofcontents

\bigskip

\renewcommand{\thefigure}{\theenumi}
\renewcommand{\thetable}{\theenumi}
%\renewcommand{\theequation}{\theenumi}

%\begin{abstract}
%%\boldmath
%In this letter, an algorithm for evaluating the exact analytical bit error rate  (BER)  for the piecewise linear (PL) combiner for  multiple relays is presented. Previous results were available only for upto three relays. The algorithm is unique in the sense that  the actual mathematical expressions, that are prohibitively large, need not be explicitly obtained. The diversity gain due to multiple relays is shown through plots of the analytical BER, well supported by simulations. 
%
%\end{abstract}
% IEEEtran.cls defaults to using nonbold math in the Abstract.
% This preserves the distinction between vectors and scalars. However,
% if the journal you are submitting to favors bold math in the abstract,
% then you can use LaTeX's standard command \boldmath at the very start
% of the abstract to achieve this. Many IEEE journals frown on math
% in the abstract anyway.

% Note that keywords are not normally used for peerreview papers.
%\begin{IEEEkeywords}
%Cooperative diversity, decode and forward, piecewise linear
%\end{IEEEkeywords}



% For peer review papers, you can put extra information on the cover
% page as needed:
% \ifCLASSOPTIONpeerreview
% \begin{center} \bfseries EDICS Category: 3-BBND \end{center}
% \fi
%
% For peerreview papers, this IEEEtran command inserts a page break and
% creates the second title. It will be ignored for other modes.
%\IEEEpeerreviewmaketitle




\item One of the four persons John, Rita, Aslam or Gurpreet will be promoted next
month. Consequently the sample space consists of four elementary outcomes
S = {John promoted, Rita promoted, Aslam promoted, Gurpreet promoted}
You are told that the chances of John’s promotion is same as that of Gurpreet,
Rita’s chances of promotion are twice as likely as Johns. Aslam’s chances are
four times that of John.
\begin{enumerate}
	\item Determine
	\begin{enumerate}
		\item P (John promoted)
		\item P (Rita promoted)
		\item P (Aslam promoted)
		\item P (Gurpreet promoted)
	\end{enumerate}
	\item If A = {John promoted or Gurpreet promoted}, find P (A).
\end{enumerate}
\solution
%\begin{table}[H]
	\centering
\begin{tabular}{|c|c|c|}
\hline
Random variable &Value &Definition\\ \hline
\multirow{3}{*}{X} &0 &Slips of Rs 1\\
&1 &Slips of Rs 5\\
&2 &Slips of Rs 13\\ \hline
\multirow{2}{*}{Y} &0 &Box A\\
&1 &Box B\\\hline
\end{tabular}
\caption{}
\label{tab:Distribution}
\end{table}
See \tabref{tab:Distribution}.
\begin{align}
p_{Y}\brak{k}= \begin{cases} 
      \frac{1}{3} & {k=0} \\
      \frac{2}{3 }& {k=1} 
   \end{cases}
   \\
p_{Y|X}\brak{0|0} = \frac{19}{25}\, 
p_{Y|X}\brak{0|1} = \frac{6}{25}\,
p_{Y|X}\brak{1|0} = \frac{45}{50}\,
p_{Y|X}\brak{1|2} = \frac{5}{50}
\end{align}
The desired probability is the probability that a slip drawn at random is marked other than Rs 1,
\begin{align}
&=1-p_X\brak{0}\\
&= p_X(1) + p_X(2)
\end{align}
Using Bayes theorem,
\begin{align}
&= p_Y\brak{0} \times \pr{Y=0 | X=1} + p_Y\brak{1} \times \pr{Y=1|X=2}\\
&=\frac{1}{3} \times \frac{6}{25} + \frac{2}{3} \times \frac{5}{50}\\
&=\frac{11}{75}
\end{align}

\newpage

%\tableofcontents

\bigskip

\renewcommand{\thefigure}{\theenumi}
\renewcommand{\thetable}{\theenumi}
%\renewcommand{\theequation}{\theenumi}

%\begin{abstract}
%%\boldmath
%In this letter, an algorithm for evaluating the exact analytical bit error rate  (BER)  for the piecewise linear (PL) combiner for  multiple relays is presented. Previous results were available only for upto three relays. The algorithm is unique in the sense that  the actual mathematical expressions, that are prohibitively large, need not be explicitly obtained. The diversity gain due to multiple relays is shown through plots of the analytical BER, well supported by simulations. 
%
%\end{abstract}
% IEEEtran.cls defaults to using nonbold math in the Abstract.
% This preserves the distinction between vectors and scalars. However,
% if the journal you are submitting to favors bold math in the abstract,
% then you can use LaTeX's standard command \boldmath at the very start
% of the abstract to achieve this. Many IEEE journals frown on math
% in the abstract anyway.

% Note that keywords are not normally used for peerreview papers.
%\begin{IEEEkeywords}
%Cooperative diversity, decode and forward, piecewise linear
%\end{IEEEkeywords}



% For peer review papers, you can put extra information on the cover
% page as needed:
% \ifCLASSOPTIONpeerreview
% \begin{center} \bfseries EDICS Category: 3-BBND \end{center}
% \fi
%
% For peerreview papers, this IEEEtran command inserts a page break and
% creates the second title. It will be ignored for other modes.
%\IEEEpeerreviewmaketitle




\item A card is drawn from a deck of 52 cards. Find the probability of getting a king or a heart or a red card.\\
\solution
%\begin{table}[H]
	\centering
\begin{tabular}{|c|c|c|}
\hline
Random variable &Value &Definition\\ \hline
\multirow{3}{*}{X} &0 &Slips of Rs 1\\
&1 &Slips of Rs 5\\
&2 &Slips of Rs 13\\ \hline
\multirow{2}{*}{Y} &0 &Box A\\
&1 &Box B\\\hline
\end{tabular}
\caption{}
\label{tab:Distribution}
\end{table}
See \tabref{tab:Distribution}.
\begin{align}
p_{Y}\brak{k}= \begin{cases} 
      \frac{1}{3} & {k=0} \\
      \frac{2}{3 }& {k=1} 
   \end{cases}
   \\
p_{Y|X}\brak{0|0} = \frac{19}{25}\, 
p_{Y|X}\brak{0|1} = \frac{6}{25}\,
p_{Y|X}\brak{1|0} = \frac{45}{50}\,
p_{Y|X}\brak{1|2} = \frac{5}{50}
\end{align}
The desired probability is the probability that a slip drawn at random is marked other than Rs 1,
\begin{align}
&=1-p_X\brak{0}\\
&= p_X(1) + p_X(2)
\end{align}
Using Bayes theorem,
\begin{align}
&= p_Y\brak{0} \times \pr{Y=0 | X=1} + p_Y\brak{1} \times \pr{Y=1|X=2}\\
&=\frac{1}{3} \times \frac{6}{25} + \frac{2}{3} \times \frac{5}{50}\\
&=\frac{11}{75}
\end{align}

\newpage

%\tableofcontents

\bigskip

\renewcommand{\thefigure}{\theenumi}
\renewcommand{\thetable}{\theenumi}
%\renewcommand{\theequation}{\theenumi}

%\begin{abstract}
%%\boldmath
%In this letter, an algorithm for evaluating the exact analytical bit error rate  (BER)  for the piecewise linear (PL) combiner for  multiple relays is presented. Previous results were available only for upto three relays. The algorithm is unique in the sense that  the actual mathematical expressions, that are prohibitively large, need not be explicitly obtained. The diversity gain due to multiple relays is shown through plots of the analytical BER, well supported by simulations. 
%
%\end{abstract}
% IEEEtran.cls defaults to using nonbold math in the Abstract.
% This preserves the distinction between vectors and scalars. However,
% if the journal you are submitting to favors bold math in the abstract,
% then you can use LaTeX's standard command \boldmath at the very start
% of the abstract to achieve this. Many IEEE journals frown on math
% in the abstract anyway.

% Note that keywords are not normally used for peerreview papers.
%\begin{IEEEkeywords}
%Cooperative diversity, decode and forward, piecewise linear
%\end{IEEEkeywords}



% For peer review papers, you can put extra information on the cover
% page as needed:
% \ifCLASSOPTIONpeerreview
% \begin{center} \bfseries EDICS Category: 3-BBND \end{center}
% \fi
%
% For peerreview papers, this IEEEtran command inserts a page break and
% creates the second title. It will be ignored for other modes.
%\IEEEpeerreviewmaketitle




\item The probability that a student will pass his examination is 0.73, the probability of
the student getting a compartment is 0.13, and the probability that the student will
either pass or get compartment is 0.96. State True or False.\\
\solution
%\begin{table}[H]
	\centering
\begin{tabular}{|c|c|c|}
\hline
Random variable &Value &Definition\\ \hline
\multirow{3}{*}{X} &0 &Slips of Rs 1\\
&1 &Slips of Rs 5\\
&2 &Slips of Rs 13\\ \hline
\multirow{2}{*}{Y} &0 &Box A\\
&1 &Box B\\\hline
\end{tabular}
\caption{}
\label{tab:Distribution}
\end{table}
See \tabref{tab:Distribution}.
\begin{align}
p_{Y}\brak{k}= \begin{cases} 
      \frac{1}{3} & {k=0} \\
      \frac{2}{3 }& {k=1} 
   \end{cases}
   \\
p_{Y|X}\brak{0|0} = \frac{19}{25}\, 
p_{Y|X}\brak{0|1} = \frac{6}{25}\,
p_{Y|X}\brak{1|0} = \frac{45}{50}\,
p_{Y|X}\brak{1|2} = \frac{5}{50}
\end{align}
The desired probability is the probability that a slip drawn at random is marked other than Rs 1,
\begin{align}
&=1-p_X\brak{0}\\
&= p_X(1) + p_X(2)
\end{align}
Using Bayes theorem,
\begin{align}
&= p_Y\brak{0} \times \pr{Y=0 | X=1} + p_Y\brak{1} \times \pr{Y=1|X=2}\\
&=\frac{1}{3} \times \frac{6}{25} + \frac{2}{3} \times \frac{5}{50}\\
&=\frac{11}{75}
\end{align}

\newpage

%\tableofcontents

\bigskip

\renewcommand{\thefigure}{\theenumi}
\renewcommand{\thetable}{\theenumi}
%\renewcommand{\theequation}{\theenumi}

%\begin{abstract}
%%\boldmath
%In this letter, an algorithm for evaluating the exact analytical bit error rate  (BER)  for the piecewise linear (PL) combiner for  multiple relays is presented. Previous results were available only for upto three relays. The algorithm is unique in the sense that  the actual mathematical expressions, that are prohibitively large, need not be explicitly obtained. The diversity gain due to multiple relays is shown through plots of the analytical BER, well supported by simulations. 
%
%\end{abstract}
% IEEEtran.cls defaults to using nonbold math in the Abstract.
% This preserves the distinction between vectors and scalars. However,
% if the journal you are submitting to favors bold math in the abstract,
% then you can use LaTeX's standard command \boldmath at the very start
% of the abstract to achieve this. Many IEEE journals frown on math
% in the abstract anyway.

% Note that keywords are not normally used for peerreview papers.
%\begin{IEEEkeywords}
%Cooperative diversity, decode and forward, piecewise linear
%\end{IEEEkeywords}



% For peer review papers, you can put extra information on the cover
% page as needed:
% \ifCLASSOPTIONpeerreview
% \begin{center} \bfseries EDICS Category: 3-BBND \end{center}
% \fi
%
% For peerreview papers, this IEEEtran command inserts a page break and
% creates the second title. It will be ignored for other modes.
%\IEEEpeerreviewmaketitle




\item A card is selected from a pack of 52 cards\\
\begin{enumerate}[label=(\alph*)]
\item How many points are there in the sample space?
\item Calculate the probability that the cards is an ace of spades.
\item Calculate the probability that the card is (i) an ace (ii)black card.\\
\end{enumerate}
%\input{ncert/11/16/3/4_1/Prob_4.tex}
\item In a non-leap year, the probability of having 53 tuesdays or 53 wednesdays is\\
\solution
%A non-leap year has a total of 365 days, and a week has 7 days.\\
So it can be expressed as 
\begin{align}
365\text{days} &=52\times 7+1 \text{day}
\end{align}
$\implies$ 52 tuesdays or wednesdays\\
Random variable X denotes the days of a week
\begin{align}
p_X\brak{k}&=\frac{1}{7}; \quad \brak{1<k<7}
\end{align}
So the probability of extra day being tuesday or wednesday is
\begin{align}
p_X\brak{3}+p_X\brak{4}&=\frac{1}{7}+\frac{1}{7}=\frac{2}{7}
\end{align}



\item There are 1000 sealed envelopes in a box, 10 of them contain a cash prize of
Rs 100 each, 100 of them contain a cash prize of Rs 50 each and 200 of them
contain a cash prize of Rs 10 each and rest do not contain any cash prize. If they
are well shuffled and an envelope is picked up out, what is the probability that it
contains no cash prize?\\
\solution
%\begin{table}[H]
	\centering
\begin{tabular}{|c|c|c|}
\hline
Random variable &Value &Definition\\ \hline
\multirow{3}{*}{X} &0 &Slips of Rs 1\\
&1 &Slips of Rs 5\\
&2 &Slips of Rs 13\\ \hline
\multirow{2}{*}{Y} &0 &Box A\\
&1 &Box B\\\hline
\end{tabular}
\caption{}
\label{tab:Distribution}
\end{table}
See \tabref{tab:Distribution}.
\begin{align}
p_{Y}\brak{k}= \begin{cases} 
      \frac{1}{3} & {k=0} \\
      \frac{2}{3 }& {k=1} 
   \end{cases}
   \\
p_{Y|X}\brak{0|0} = \frac{19}{25}\, 
p_{Y|X}\brak{0|1} = \frac{6}{25}\,
p_{Y|X}\brak{1|0} = \frac{45}{50}\,
p_{Y|X}\brak{1|2} = \frac{5}{50}
\end{align}
The desired probability is the probability that a slip drawn at random is marked other than Rs 1,
\begin{align}
&=1-p_X\brak{0}\\
&= p_X(1) + p_X(2)
\end{align}
Using Bayes theorem,
\begin{align}
&= p_Y\brak{0} \times \pr{Y=0 | X=1} + p_Y\brak{1} \times \pr{Y=1|X=2}\\
&=\frac{1}{3} \times \frac{6}{25} + \frac{2}{3} \times \frac{5}{50}\\
&=\frac{11}{75}
\end{align}

\newpage

%\tableofcontents

\bigskip

\renewcommand{\thefigure}{\theenumi}
\renewcommand{\thetable}{\theenumi}
%\renewcommand{\theequation}{\theenumi}

%\begin{abstract}
%%\boldmath
%In this letter, an algorithm for evaluating the exact analytical bit error rate  (BER)  for the piecewise linear (PL) combiner for  multiple relays is presented. Previous results were available only for upto three relays. The algorithm is unique in the sense that  the actual mathematical expressions, that are prohibitively large, need not be explicitly obtained. The diversity gain due to multiple relays is shown through plots of the analytical BER, well supported by simulations. 
%
%\end{abstract}
% IEEEtran.cls defaults to using nonbold math in the Abstract.
% This preserves the distinction between vectors and scalars. However,
% if the journal you are submitting to favors bold math in the abstract,
% then you can use LaTeX's standard command \boldmath at the very start
% of the abstract to achieve this. Many IEEE journals frown on math
% in the abstract anyway.

% Note that keywords are not normally used for peerreview papers.
%\begin{IEEEkeywords}
%Cooperative diversity, decode and forward, piecewise linear
%\end{IEEEkeywords}



% For peer review papers, you can put extra information on the cover
% page as needed:
% \ifCLASSOPTIONpeerreview
% \begin{center} \bfseries EDICS Category: 3-BBND \end{center}
% \fi
%
% For peerreview papers, this IEEEtran command inserts a page break and
% creates the second title. It will be ignored for other modes.
%\IEEEpeerreviewmaketitle




\item 
A die is thrown and a card is selected at random from a deck of 52 playing cards. The probability of getting an even number on the die and a spade card.\\
\solution
%\begin{table}[H]
	\centering
\begin{tabular}{|c|c|c|}
\hline
Random variable &Value &Definition\\ \hline
\multirow{3}{*}{X} &0 &Slips of Rs 1\\
&1 &Slips of Rs 5\\
&2 &Slips of Rs 13\\ \hline
\multirow{2}{*}{Y} &0 &Box A\\
&1 &Box B\\\hline
\end{tabular}
\caption{}
\label{tab:Distribution}
\end{table}
See \tabref{tab:Distribution}.
\begin{align}
p_{Y}\brak{k}= \begin{cases} 
      \frac{1}{3} & {k=0} \\
      \frac{2}{3 }& {k=1} 
   \end{cases}
   \\
p_{Y|X}\brak{0|0} = \frac{19}{25}\, 
p_{Y|X}\brak{0|1} = \frac{6}{25}\,
p_{Y|X}\brak{1|0} = \frac{45}{50}\,
p_{Y|X}\brak{1|2} = \frac{5}{50}
\end{align}
The desired probability is the probability that a slip drawn at random is marked other than Rs 1,
\begin{align}
&=1-p_X\brak{0}\\
&= p_X(1) + p_X(2)
\end{align}
Using Bayes theorem,
\begin{align}
&= p_Y\brak{0} \times \pr{Y=0 | X=1} + p_Y\brak{1} \times \pr{Y=1|X=2}\\
&=\frac{1}{3} \times \frac{6}{25} + \frac{2}{3} \times \frac{5}{50}\\
&=\frac{11}{75}
\end{align}

\newpage

%\tableofcontents

\bigskip

\renewcommand{\thefigure}{\theenumi}
\renewcommand{\thetable}{\theenumi}
%\renewcommand{\theequation}{\theenumi}

%\begin{abstract}
%%\boldmath
%In this letter, an algorithm for evaluating the exact analytical bit error rate  (BER)  for the piecewise linear (PL) combiner for  multiple relays is presented. Previous results were available only for upto three relays. The algorithm is unique in the sense that  the actual mathematical expressions, that are prohibitively large, need not be explicitly obtained. The diversity gain due to multiple relays is shown through plots of the analytical BER, well supported by simulations. 
%
%\end{abstract}
% IEEEtran.cls defaults to using nonbold math in the Abstract.
% This preserves the distinction between vectors and scalars. However,
% if the journal you are submitting to favors bold math in the abstract,
% then you can use LaTeX's standard command \boldmath at the very start
% of the abstract to achieve this. Many IEEE journals frown on math
% in the abstract anyway.

% Note that keywords are not normally used for peerreview papers.
%\begin{IEEEkeywords}
%Cooperative diversity, decode and forward, piecewise linear
%\end{IEEEkeywords}



% For peer review papers, you can put extra information on the cover
% page as needed:
% \ifCLASSOPTIONpeerreview
% \begin{center} \bfseries EDICS Category: 3-BBND \end{center}
% \fi
%
% For peerreview papers, this IEEEtran command inserts a page break and
% creates the second title. It will be ignored for other modes.
%\IEEEpeerreviewmaketitle




\item
If 4-digit numbers greater than 5,000 are randomly formed from the digits 0, 1, 3, 5, and 7, what is the probability of forming a number divisible by 5 when:
\begin{enumerate}
    \item The digits are repeated?
    \item The repetition of digits is not allowed?
\end{enumerate}
\solution
%\begin{table}[H]
	\centering
\begin{tabular}{|c|c|c|}
\hline
Random variable &Value &Definition\\ \hline
\multirow{3}{*}{X} &0 &Slips of Rs 1\\
&1 &Slips of Rs 5\\
&2 &Slips of Rs 13\\ \hline
\multirow{2}{*}{Y} &0 &Box A\\
&1 &Box B\\\hline
\end{tabular}
\caption{}
\label{tab:Distribution}
\end{table}
See \tabref{tab:Distribution}.
\begin{align}
p_{Y}\brak{k}= \begin{cases} 
      \frac{1}{3} & {k=0} \\
      \frac{2}{3 }& {k=1} 
   \end{cases}
   \\
p_{Y|X}\brak{0|0} = \frac{19}{25}\, 
p_{Y|X}\brak{0|1} = \frac{6}{25}\,
p_{Y|X}\brak{1|0} = \frac{45}{50}\,
p_{Y|X}\brak{1|2} = \frac{5}{50}
\end{align}
The desired probability is the probability that a slip drawn at random is marked other than Rs 1,
\begin{align}
&=1-p_X\brak{0}\\
&= p_X(1) + p_X(2)
\end{align}
Using Bayes theorem,
\begin{align}
&= p_Y\brak{0} \times \pr{Y=0 | X=1} + p_Y\brak{1} \times \pr{Y=1|X=2}\\
&=\frac{1}{3} \times \frac{6}{25} + \frac{2}{3} \times \frac{5}{50}\\
&=\frac{11}{75}
\end{align}

\newpage

%\tableofcontents

\bigskip

\renewcommand{\thefigure}{\theenumi}
\renewcommand{\thetable}{\theenumi}
%\renewcommand{\theequation}{\theenumi}

%\begin{abstract}
%%\boldmath
%In this letter, an algorithm for evaluating the exact analytical bit error rate  (BER)  for the piecewise linear (PL) combiner for  multiple relays is presented. Previous results were available only for upto three relays. The algorithm is unique in the sense that  the actual mathematical expressions, that are prohibitively large, need not be explicitly obtained. The diversity gain due to multiple relays is shown through plots of the analytical BER, well supported by simulations. 
%
%\end{abstract}
% IEEEtran.cls defaults to using nonbold math in the Abstract.
% This preserves the distinction between vectors and scalars. However,
% if the journal you are submitting to favors bold math in the abstract,
% then you can use LaTeX's standard command \boldmath at the very start
% of the abstract to achieve this. Many IEEE journals frown on math
% in the abstract anyway.

% Note that keywords are not normally used for peerreview papers.
%\begin{IEEEkeywords}
%Cooperative diversity, decode and forward, piecewise linear
%\end{IEEEkeywords}



% For peer review papers, you can put extra information on the cover
% page as needed:
% \ifCLASSOPTIONpeerreview
% \begin{center} \bfseries EDICS Category: 3-BBND \end{center}
% \fi
%
% For peerreview papers, this IEEEtran command inserts a page break and
% creates the second title. It will be ignored for other modes.
%\IEEEpeerreviewmaketitle




\item Consider the probability space $\brak{\Omega, \mathcal{G}, P}$ where $\Omega = [0,2]$ and $\mathcal{G} = \cbrak{\phi, \Omega, [0,1], (1,2]}$. Let $X$ and $Y$ be two functions on $\Omega$ defined as
\begin{align*}
    X(\omega) = 
    \begin{cases}
        1 & \text{if }\omega \in [0, 1]\\
        2 & \text{if }\omega \in (1, 2]
    \end{cases}
\end{align*}
and
\begin{align*}
    Y(\omega) = 
    \begin{cases}
        2 & \text{if }\omega \in [0, 1.5]\\
        3 & \text{if }\omega \in (1.5, 2].
    \end{cases}
\end{align*}
Then which one of the following statements is true?
\begin{enumerate}
    \item [(A)] $X$ is a random variable with respect to $\mathcal{G}$, but $Y$ is not a random variable with respect to $\mathcal{G}$.
    \item [(B)] $Y$ is a random variable with respect to $\mathcal{G}$, but $X$ is not a random variable with respect to $\mathcal{G}$.
    \item [(C)] Neither $X$ nor $Y$ is a random variable with respect to $\mathcal{G}$.
    \item [(D)] Both $X$ and $Y$ are random variables with respect to $\mathcal{G}$.
\end{enumerate} \hfill (GATE ST 2023)\\
\solution
%\begin{table}[H]
	\centering
\begin{tabular}{|c|c|c|}
\hline
Random variable &Value &Definition\\ \hline
\multirow{3}{*}{X} &0 &Slips of Rs 1\\
&1 &Slips of Rs 5\\
&2 &Slips of Rs 13\\ \hline
\multirow{2}{*}{Y} &0 &Box A\\
&1 &Box B\\\hline
\end{tabular}
\caption{}
\label{tab:Distribution}
\end{table}
See \tabref{tab:Distribution}.
\begin{align}
p_{Y}\brak{k}= \begin{cases} 
      \frac{1}{3} & {k=0} \\
      \frac{2}{3 }& {k=1} 
   \end{cases}
   \\
p_{Y|X}\brak{0|0} = \frac{19}{25}\, 
p_{Y|X}\brak{0|1} = \frac{6}{25}\,
p_{Y|X}\brak{1|0} = \frac{45}{50}\,
p_{Y|X}\brak{1|2} = \frac{5}{50}
\end{align}
The desired probability is the probability that a slip drawn at random is marked other than Rs 1,
\begin{align}
&=1-p_X\brak{0}\\
&= p_X(1) + p_X(2)
\end{align}
Using Bayes theorem,
\begin{align}
&= p_Y\brak{0} \times \pr{Y=0 | X=1} + p_Y\brak{1} \times \pr{Y=1|X=2}\\
&=\frac{1}{3} \times \frac{6}{25} + \frac{2}{3} \times \frac{5}{50}\\
&=\frac{11}{75}
\end{align}

\newpage

%\tableofcontents

\bigskip

\renewcommand{\thefigure}{\theenumi}
\renewcommand{\thetable}{\theenumi}
%\renewcommand{\theequation}{\theenumi}

%\begin{abstract}
%%\boldmath
%In this letter, an algorithm for evaluating the exact analytical bit error rate  (BER)  for the piecewise linear (PL) combiner for  multiple relays is presented. Previous results were available only for upto three relays. The algorithm is unique in the sense that  the actual mathematical expressions, that are prohibitively large, need not be explicitly obtained. The diversity gain due to multiple relays is shown through plots of the analytical BER, well supported by simulations. 
%
%\end{abstract}
% IEEEtran.cls defaults to using nonbold math in the Abstract.
% This preserves the distinction between vectors and scalars. However,
% if the journal you are submitting to favors bold math in the abstract,
% then you can use LaTeX's standard command \boldmath at the very start
% of the abstract to achieve this. Many IEEE journals frown on math
% in the abstract anyway.

% Note that keywords are not normally used for peerreview papers.
%\begin{IEEEkeywords}
%Cooperative diversity, decode and forward, piecewise linear
%\end{IEEEkeywords}



% For peer review papers, you can put extra information on the cover
% page as needed:
% \ifCLASSOPTIONpeerreview
% \begin{center} \bfseries EDICS Category: 3-BBND \end{center}
% \fi
%
% For peerreview papers, this IEEEtran command inserts a page break and
% creates the second title. It will be ignored for other modes.
%\IEEEpeerreviewmaketitle




	\item  A die is loaded in such a way that each odd number is twice as likely to occur as
each even number. Find $P(G)$, where $G$ is the event that a number greater than
3 occurs on a single roll of the die.
\\
\solution
		%\begin{table}[H]
	\centering
\begin{tabular}{|c|c|c|}
\hline
Random variable &Value &Definition\\ \hline
\multirow{3}{*}{X} &0 &Slips of Rs 1\\
&1 &Slips of Rs 5\\
&2 &Slips of Rs 13\\ \hline
\multirow{2}{*}{Y} &0 &Box A\\
&1 &Box B\\\hline
\end{tabular}
\caption{}
\label{tab:Distribution}
\end{table}
See \tabref{tab:Distribution}.
\begin{align}
p_{Y}\brak{k}= \begin{cases} 
      \frac{1}{3} & {k=0} \\
      \frac{2}{3 }& {k=1} 
   \end{cases}
   \\
p_{Y|X}\brak{0|0} = \frac{19}{25}\, 
p_{Y|X}\brak{0|1} = \frac{6}{25}\,
p_{Y|X}\brak{1|0} = \frac{45}{50}\,
p_{Y|X}\brak{1|2} = \frac{5}{50}
\end{align}
The desired probability is the probability that a slip drawn at random is marked other than Rs 1,
\begin{align}
&=1-p_X\brak{0}\\
&= p_X(1) + p_X(2)
\end{align}
Using Bayes theorem,
\begin{align}
&= p_Y\brak{0} \times \pr{Y=0 | X=1} + p_Y\brak{1} \times \pr{Y=1|X=2}\\
&=\frac{1}{3} \times \frac{6}{25} + \frac{2}{3} \times \frac{5}{50}\\
&=\frac{11}{75}
\end{align}

\newpage

%\tableofcontents

\bigskip

\renewcommand{\thefigure}{\theenumi}
\renewcommand{\thetable}{\theenumi}
%\renewcommand{\theequation}{\theenumi}

%\begin{abstract}
%%\boldmath
%In this letter, an algorithm for evaluating the exact analytical bit error rate  (BER)  for the piecewise linear (PL) combiner for  multiple relays is presented. Previous results were available only for upto three relays. The algorithm is unique in the sense that  the actual mathematical expressions, that are prohibitively large, need not be explicitly obtained. The diversity gain due to multiple relays is shown through plots of the analytical BER, well supported by simulations. 
%
%\end{abstract}
% IEEEtran.cls defaults to using nonbold math in the Abstract.
% This preserves the distinction between vectors and scalars. However,
% if the journal you are submitting to favors bold math in the abstract,
% then you can use LaTeX's standard command \boldmath at the very start
% of the abstract to achieve this. Many IEEE journals frown on math
% in the abstract anyway.

% Note that keywords are not normally used for peerreview papers.
%\begin{IEEEkeywords}
%Cooperative diversity, decode and forward, piecewise linear
%\end{IEEEkeywords}



% For peer review papers, you can put extra information on the cover
% page as needed:
% \ifCLASSOPTIONpeerreview
% \begin{center} \bfseries EDICS Category: 3-BBND \end{center}
% \fi
%
% For peerreview papers, this IEEEtran command inserts a page break and
% creates the second title. It will be ignored for other modes.
%\IEEEpeerreviewmaketitle




	\item All the jacks, queens and kings are removed from a deck of 52 playing cards. The remaining cards are well shuffled and then one card is drawn at random. Giving ace a value 1 similar value for other cards, find the probability that the card has a value 
		\begin{enumerate}
			\item 7
			\item greater than 7
			\item less than 7
		\end{enumerate}
		%Number of cards left after removing all jacks, queens and kings 
\begin{align}
N	= 52 - 4\times 3
	= 40
\end{align}
%\begin{table}[H]
%\def\arraystretch{1.2}
%\begin{tabular}{|c|c|c|}
%\hline
%	\textbf{Parameter} &\textbf{Value} &\textbf{Description}\\ \hline
%	$X$ &1-10 &Represents the value of the card picked \\ \hline
%\end{tabular}
%\end{table}
Let $1 \le X \le 10$ be the value of the card picked.  Then,
\begin{align}
	p_X(k) &= \Pr(X=k)\ \forall\ 1 \leq k \leq 10\\
	&= \frac{4\times 1}{40}\\
	&= \frac{1}{10}\\
	\therefore p_X(k) &= 
	\begin{cases}
		\frac{1}{10} & 1 \leq k \leq 10\\
		0 & \text{otherwise}
	\end{cases}
\end{align}
and
\begin{align}
	F_{X}(k) &= \sum_{m=0}^{k}p_{X}(m) \quad 1 \leq k \leq 10\\
	&= \frac{k}{10}\\
	\therefore F_{X}(k) &= 
	\begin{cases}
		0 & k \leq 0\\
		\frac{k}{10} & 1\leq k \leq 10\\
		1 & k > 10 
	\end{cases}
\end{align}
\begin{enumerate}
	\item Probability that card has value equal to 7 is
		\begin{align}
			 p_{X}(7)
			= \frac{1}{10}
		\end{align}
	\item Probability that card has value greater than 7 is
		\begin{align}
			1 - F_X(7)
			&= 1 - \frac{7}{10}
			\\
			&= \frac{3}{10}
		\end{align}
	\item Probability that card has value less than 7 is
		\begin{align}
			 F_{X}(6)
			=\frac{6}{10}
		\end{align}
\end{enumerate}

  \item A Lot consists of 48 mobile phones of which 42 are good, 3 have only minor defects and 3 have major defects.Varnika will buy a phone if it is good but the trader will only buy a mobile if it has no major defects. One phone is selected at random from the lot. What is the probability that it is
\begin{enumerate}
	\item acceptable to Varnika?
            \item acceptable to the trader?
\end{enumerate}
\solution
	%\begin{table}[H]
	\centering
\begin{tabular}{|c|c|c|}
\hline
Random variable &Value &Definition\\ \hline
\multirow{3}{*}{X} &0 &Slips of Rs 1\\
&1 &Slips of Rs 5\\
&2 &Slips of Rs 13\\ \hline
\multirow{2}{*}{Y} &0 &Box A\\
&1 &Box B\\\hline
\end{tabular}
\caption{}
\label{tab:Distribution}
\end{table}
See \tabref{tab:Distribution}.
\begin{align}
p_{Y}\brak{k}= \begin{cases} 
      \frac{1}{3} & {k=0} \\
      \frac{2}{3 }& {k=1} 
   \end{cases}
   \\
p_{Y|X}\brak{0|0} = \frac{19}{25}\, 
p_{Y|X}\brak{0|1} = \frac{6}{25}\,
p_{Y|X}\brak{1|0} = \frac{45}{50}\,
p_{Y|X}\brak{1|2} = \frac{5}{50}
\end{align}
The desired probability is the probability that a slip drawn at random is marked other than Rs 1,
\begin{align}
&=1-p_X\brak{0}\\
&= p_X(1) + p_X(2)
\end{align}
Using Bayes theorem,
\begin{align}
&= p_Y\brak{0} \times \pr{Y=0 | X=1} + p_Y\brak{1} \times \pr{Y=1|X=2}\\
&=\frac{1}{3} \times \frac{6}{25} + \frac{2}{3} \times \frac{5}{50}\\
&=\frac{11}{75}
\end{align}

\newpage

%\tableofcontents

\bigskip

\renewcommand{\thefigure}{\theenumi}
\renewcommand{\thetable}{\theenumi}
%\renewcommand{\theequation}{\theenumi}

%\begin{abstract}
%%\boldmath
%In this letter, an algorithm for evaluating the exact analytical bit error rate  (BER)  for the piecewise linear (PL) combiner for  multiple relays is presented. Previous results were available only for upto three relays. The algorithm is unique in the sense that  the actual mathematical expressions, that are prohibitively large, need not be explicitly obtained. The diversity gain due to multiple relays is shown through plots of the analytical BER, well supported by simulations. 
%
%\end{abstract}
% IEEEtran.cls defaults to using nonbold math in the Abstract.
% This preserves the distinction between vectors and scalars. However,
% if the journal you are submitting to favors bold math in the abstract,
% then you can use LaTeX's standard command \boldmath at the very start
% of the abstract to achieve this. Many IEEE journals frown on math
% in the abstract anyway.

% Note that keywords are not normally used for peerreview papers.
%\begin{IEEEkeywords}
%Cooperative diversity, decode and forward, piecewise linear
%\end{IEEEkeywords}



% For peer review papers, you can put extra information on the cover
% page as needed:
% \ifCLASSOPTIONpeerreview
% \begin{center} \bfseries EDICS Category: 3-BBND \end{center}
% \fi
%
% For peerreview papers, this IEEEtran command inserts a page break and
% creates the second title. It will be ignored for other modes.
%\IEEEpeerreviewmaketitle




 \item A student says that if you throw a die, it will show up 1 or not 1. Therefore, the probability of getting 1 and the probability of getting 'not 1' each is equal to $\frac{1}{2}$. Is this correct? Give reasons.\\
 \solution
        %\begin{table}[H]
	\centering
\begin{tabular}{|c|c|c|}
\hline
Random variable &Value &Definition\\ \hline
\multirow{3}{*}{X} &0 &Slips of Rs 1\\
&1 &Slips of Rs 5\\
&2 &Slips of Rs 13\\ \hline
\multirow{2}{*}{Y} &0 &Box A\\
&1 &Box B\\\hline
\end{tabular}
\caption{}
\label{tab:Distribution}
\end{table}
See \tabref{tab:Distribution}.
\begin{align}
p_{Y}\brak{k}= \begin{cases} 
      \frac{1}{3} & {k=0} \\
      \frac{2}{3 }& {k=1} 
   \end{cases}
   \\
p_{Y|X}\brak{0|0} = \frac{19}{25}\, 
p_{Y|X}\brak{0|1} = \frac{6}{25}\,
p_{Y|X}\brak{1|0} = \frac{45}{50}\,
p_{Y|X}\brak{1|2} = \frac{5}{50}
\end{align}
The desired probability is the probability that a slip drawn at random is marked other than Rs 1,
\begin{align}
&=1-p_X\brak{0}\\
&= p_X(1) + p_X(2)
\end{align}
Using Bayes theorem,
\begin{align}
&= p_Y\brak{0} \times \pr{Y=0 | X=1} + p_Y\brak{1} \times \pr{Y=1|X=2}\\
&=\frac{1}{3} \times \frac{6}{25} + \frac{2}{3} \times \frac{5}{50}\\
&=\frac{11}{75}
\end{align}

\newpage

%\tableofcontents

\bigskip

\renewcommand{\thefigure}{\theenumi}
\renewcommand{\thetable}{\theenumi}
%\renewcommand{\theequation}{\theenumi}

%\begin{abstract}
%%\boldmath
%In this letter, an algorithm for evaluating the exact analytical bit error rate  (BER)  for the piecewise linear (PL) combiner for  multiple relays is presented. Previous results were available only for upto three relays. The algorithm is unique in the sense that  the actual mathematical expressions, that are prohibitively large, need not be explicitly obtained. The diversity gain due to multiple relays is shown through plots of the analytical BER, well supported by simulations. 
%
%\end{abstract}
% IEEEtran.cls defaults to using nonbold math in the Abstract.
% This preserves the distinction between vectors and scalars. However,
% if the journal you are submitting to favors bold math in the abstract,
% then you can use LaTeX's standard command \boldmath at the very start
% of the abstract to achieve this. Many IEEE journals frown on math
% in the abstract anyway.

% Note that keywords are not normally used for peerreview papers.
%\begin{IEEEkeywords}
%Cooperative diversity, decode and forward, piecewise linear
%\end{IEEEkeywords}



% For peer review papers, you can put extra information on the cover
% page as needed:
% \ifCLASSOPTIONpeerreview
% \begin{center} \bfseries EDICS Category: 3-BBND \end{center}
% \fi
%
% For peerreview papers, this IEEEtran command inserts a page break and
% creates the second title. It will be ignored for other modes.
%\IEEEpeerreviewmaketitle




   \item Four candidates A, B, C, D have ap-
plied for the assignment to coach a school cricket
team. If A is twice as likely to be selected as B, and
B and C are given about the same chance of being
selected, while C is twice as likely to be selected
as D, what are the probabilities that
\begin{enumerate}
\item C will be selected?
\item A will not be selected?
\end{enumerate}
	%\begin{table}[H]
	\centering
\begin{tabular}{|c|c|c|}
\hline
Random variable &Value &Definition\\ \hline
\multirow{3}{*}{X} &0 &Slips of Rs 1\\
&1 &Slips of Rs 5\\
&2 &Slips of Rs 13\\ \hline
\multirow{2}{*}{Y} &0 &Box A\\
&1 &Box B\\\hline
\end{tabular}
\caption{}
\label{tab:Distribution}
\end{table}
See \tabref{tab:Distribution}.
\begin{align}
p_{Y}\brak{k}= \begin{cases} 
      \frac{1}{3} & {k=0} \\
      \frac{2}{3 }& {k=1} 
   \end{cases}
   \\
p_{Y|X}\brak{0|0} = \frac{19}{25}\, 
p_{Y|X}\brak{0|1} = \frac{6}{25}\,
p_{Y|X}\brak{1|0} = \frac{45}{50}\,
p_{Y|X}\brak{1|2} = \frac{5}{50}
\end{align}
The desired probability is the probability that a slip drawn at random is marked other than Rs 1,
\begin{align}
&=1-p_X\brak{0}\\
&= p_X(1) + p_X(2)
\end{align}
Using Bayes theorem,
\begin{align}
&= p_Y\brak{0} \times \pr{Y=0 | X=1} + p_Y\brak{1} \times \pr{Y=1|X=2}\\
&=\frac{1}{3} \times \frac{6}{25} + \frac{2}{3} \times \frac{5}{50}\\
&=\frac{11}{75}
\end{align}

\newpage

%\tableofcontents

\bigskip

\renewcommand{\thefigure}{\theenumi}
\renewcommand{\thetable}{\theenumi}
%\renewcommand{\theequation}{\theenumi}

%\begin{abstract}
%%\boldmath
%In this letter, an algorithm for evaluating the exact analytical bit error rate  (BER)  for the piecewise linear (PL) combiner for  multiple relays is presented. Previous results were available only for upto three relays. The algorithm is unique in the sense that  the actual mathematical expressions, that are prohibitively large, need not be explicitly obtained. The diversity gain due to multiple relays is shown through plots of the analytical BER, well supported by simulations. 
%
%\end{abstract}
% IEEEtran.cls defaults to using nonbold math in the Abstract.
% This preserves the distinction between vectors and scalars. However,
% if the journal you are submitting to favors bold math in the abstract,
% then you can use LaTeX's standard command \boldmath at the very start
% of the abstract to achieve this. Many IEEE journals frown on math
% in the abstract anyway.

% Note that keywords are not normally used for peerreview papers.
%\begin{IEEEkeywords}
%Cooperative diversity, decode and forward, piecewise linear
%\end{IEEEkeywords}



% For peer review papers, you can put extra information on the cover
% page as needed:
% \ifCLASSOPTIONpeerreview
% \begin{center} \bfseries EDICS Category: 3-BBND \end{center}
% \fi
%
% For peerreview papers, this IEEEtran command inserts a page break and
% creates the second title. It will be ignored for other modes.
%\IEEEpeerreviewmaketitle




 \item A bag contain 24 balls of which $x$ balls are red, $2x$ are white and $3x$ are blue. A ball is selected at random, What is the probability that it is
\begin{enumerate}[label=\alph*)]
\item not red ?
\item white ?
\end{enumerate}
%\begin{table}[H]
	\centering
\begin{tabular}{|c|c|c|}
\hline
Random variable &Value &Definition\\ \hline
\multirow{3}{*}{X} &0 &Slips of Rs 1\\
&1 &Slips of Rs 5\\
&2 &Slips of Rs 13\\ \hline
\multirow{2}{*}{Y} &0 &Box A\\
&1 &Box B\\\hline
\end{tabular}
\caption{}
\label{tab:Distribution}
\end{table}
See \tabref{tab:Distribution}.
\begin{align}
p_{Y}\brak{k}= \begin{cases} 
      \frac{1}{3} & {k=0} \\
      \frac{2}{3 }& {k=1} 
   \end{cases}
   \\
p_{Y|X}\brak{0|0} = \frac{19}{25}\, 
p_{Y|X}\brak{0|1} = \frac{6}{25}\,
p_{Y|X}\brak{1|0} = \frac{45}{50}\,
p_{Y|X}\brak{1|2} = \frac{5}{50}
\end{align}
The desired probability is the probability that a slip drawn at random is marked other than Rs 1,
\begin{align}
&=1-p_X\brak{0}\\
&= p_X(1) + p_X(2)
\end{align}
Using Bayes theorem,
\begin{align}
&= p_Y\brak{0} \times \pr{Y=0 | X=1} + p_Y\brak{1} \times \pr{Y=1|X=2}\\
&=\frac{1}{3} \times \frac{6}{25} + \frac{2}{3} \times \frac{5}{50}\\
&=\frac{11}{75}
\end{align}

\newpage

%\tableofcontents

\bigskip

\renewcommand{\thefigure}{\theenumi}
\renewcommand{\thetable}{\theenumi}
%\renewcommand{\theequation}{\theenumi}

%\begin{abstract}
%%\boldmath
%In this letter, an algorithm for evaluating the exact analytical bit error rate  (BER)  for the piecewise linear (PL) combiner for  multiple relays is presented. Previous results were available only for upto three relays. The algorithm is unique in the sense that  the actual mathematical expressions, that are prohibitively large, need not be explicitly obtained. The diversity gain due to multiple relays is shown through plots of the analytical BER, well supported by simulations. 
%
%\end{abstract}
% IEEEtran.cls defaults to using nonbold math in the Abstract.
% This preserves the distinction between vectors and scalars. However,
% if the journal you are submitting to favors bold math in the abstract,
% then you can use LaTeX's standard command \boldmath at the very start
% of the abstract to achieve this. Many IEEE journals frown on math
% in the abstract anyway.

% Note that keywords are not normally used for peerreview papers.
%\begin{IEEEkeywords}
%Cooperative diversity, decode and forward, piecewise linear
%\end{IEEEkeywords}



% For peer review papers, you can put extra information on the cover
% page as needed:
% \ifCLASSOPTIONpeerreview
% \begin{center} \bfseries EDICS Category: 3-BBND \end{center}
% \fi
%
% For peerreview papers, this IEEEtran command inserts a page break and
% creates the second title. It will be ignored for other modes.
%\IEEEpeerreviewmaketitle




If the letters of the word ASSASSINATION are arranged at random. Find the Probability that
\begin{enumerate}[label=(\alph*)]
\item Four $S's$ come consecutively in the word
\item Two  $I's$ and two $N's$ come together
\item All $A's$ are not coming together
\item No two $A's$ are coming together
\end{enumerate}
%\begin{table}[H]
	\centering
\begin{tabular}{|c|c|c|}
\hline
Random variable &Value &Definition\\ \hline
\multirow{3}{*}{X} &0 &Slips of Rs 1\\
&1 &Slips of Rs 5\\
&2 &Slips of Rs 13\\ \hline
\multirow{2}{*}{Y} &0 &Box A\\
&1 &Box B\\\hline
\end{tabular}
\caption{}
\label{tab:Distribution}
\end{table}
See \tabref{tab:Distribution}.
\begin{align}
p_{Y}\brak{k}= \begin{cases} 
      \frac{1}{3} & {k=0} \\
      \frac{2}{3 }& {k=1} 
   \end{cases}
   \\
p_{Y|X}\brak{0|0} = \frac{19}{25}\, 
p_{Y|X}\brak{0|1} = \frac{6}{25}\,
p_{Y|X}\brak{1|0} = \frac{45}{50}\,
p_{Y|X}\brak{1|2} = \frac{5}{50}
\end{align}
The desired probability is the probability that a slip drawn at random is marked other than Rs 1,
\begin{align}
&=1-p_X\brak{0}\\
&= p_X(1) + p_X(2)
\end{align}
Using Bayes theorem,
\begin{align}
&= p_Y\brak{0} \times \pr{Y=0 | X=1} + p_Y\brak{1} \times \pr{Y=1|X=2}\\
&=\frac{1}{3} \times \frac{6}{25} + \frac{2}{3} \times \frac{5}{50}\\
&=\frac{11}{75}
\end{align}

\newpage

%\tableofcontents

\bigskip

\renewcommand{\thefigure}{\theenumi}
\renewcommand{\thetable}{\theenumi}
%\renewcommand{\theequation}{\theenumi}

%\begin{abstract}
%%\boldmath
%In this letter, an algorithm for evaluating the exact analytical bit error rate  (BER)  for the piecewise linear (PL) combiner for  multiple relays is presented. Previous results were available only for upto three relays. The algorithm is unique in the sense that  the actual mathematical expressions, that are prohibitively large, need not be explicitly obtained. The diversity gain due to multiple relays is shown through plots of the analytical BER, well supported by simulations. 
%
%\end{abstract}
% IEEEtran.cls defaults to using nonbold math in the Abstract.
% This preserves the distinction between vectors and scalars. However,
% if the journal you are submitting to favors bold math in the abstract,
% then you can use LaTeX's standard command \boldmath at the very start
% of the abstract to achieve this. Many IEEE journals frown on math
% in the abstract anyway.

% Note that keywords are not normally used for peerreview papers.
%\begin{IEEEkeywords}
%Cooperative diversity, decode and forward, piecewise linear
%\end{IEEEkeywords}



% For peer review papers, you can put extra information on the cover
% page as needed:
% \ifCLASSOPTIONpeerreview
% \begin{center} \bfseries EDICS Category: 3-BBND \end{center}
% \fi
%
% For peerreview papers, this IEEEtran command inserts a page break and
% creates the second title. It will be ignored for other modes.
%\IEEEpeerreviewmaketitle




	\item One urn contains two black balls (labelled B1 and B2) and one white ball. A
	second urn contains one black ball and two white balls (labelled W1 and W2).
	Suppose the following experiment is performed. One of the two urns is chosen
	at random. Next a ball is randomly chosen from the urn. Then a second ball is
	chosen at random from the same urn without replacing the first ball.
	
	\begin{enumerate}
	\item What is the probability that two black balls are chosen?
	
	\item What is the probability that two balls of opposite colour are chosen?
	\end{enumerate}
	\solution
	%\begin{align}
    \label{eq:12.13.6.18.1}
	\because	\pr{A|B} &> \pr{A},\
\frac{\pr{AB}}{\pr{B}} > \pr{A}
\\
    \label{eq:12.13.6.18.2}
	\implies \pr{AB} &> \pr{A}\pr{B}
	\\
	\text{or, } \frac{\pr{AB}}{\pr{A}} &=\pr{B|A} > \pr{A}
\end{align}

\end{enumerate}

	\item A bag contains 4 red and 4 black balls, another bag contains 2 red and 6 black balls. One of the two bags is selected at random and a ball is drawn from the bag which is found to be red. Find the probability that the ball is drawn from the first bag.
\\
\solution
		%\begin{table}[H]
	\centering
\begin{tabular}{|c|c|c|}
\hline
Random variable &Value &Definition\\ \hline
\multirow{3}{*}{X} &0 &Slips of Rs 1\\
&1 &Slips of Rs 5\\
&2 &Slips of Rs 13\\ \hline
\multirow{2}{*}{Y} &0 &Box A\\
&1 &Box B\\\hline
\end{tabular}
\caption{}
\label{tab:Distribution}
\end{table}
See \tabref{tab:Distribution}.
\begin{align}
p_{Y}\brak{k}= \begin{cases} 
      \frac{1}{3} & {k=0} \\
      \frac{2}{3 }& {k=1} 
   \end{cases}
   \\
p_{Y|X}\brak{0|0} = \frac{19}{25}\, 
p_{Y|X}\brak{0|1} = \frac{6}{25}\,
p_{Y|X}\brak{1|0} = \frac{45}{50}\,
p_{Y|X}\brak{1|2} = \frac{5}{50}
\end{align}
The desired probability is the probability that a slip drawn at random is marked other than Rs 1,
\begin{align}
&=1-p_X\brak{0}\\
&= p_X(1) + p_X(2)
\end{align}
Using Bayes theorem,
\begin{align}
&= p_Y\brak{0} \times \pr{Y=0 | X=1} + p_Y\brak{1} \times \pr{Y=1|X=2}\\
&=\frac{1}{3} \times \frac{6}{25} + \frac{2}{3} \times \frac{5}{50}\\
&=\frac{11}{75}
\end{align}

\newpage

%\tableofcontents

\bigskip

\renewcommand{\thefigure}{\theenumi}
\renewcommand{\thetable}{\theenumi}
%\renewcommand{\theequation}{\theenumi}

%\begin{abstract}
%%\boldmath
%In this letter, an algorithm for evaluating the exact analytical bit error rate  (BER)  for the piecewise linear (PL) combiner for  multiple relays is presented. Previous results were available only for upto three relays. The algorithm is unique in the sense that  the actual mathematical expressions, that are prohibitively large, need not be explicitly obtained. The diversity gain due to multiple relays is shown through plots of the analytical BER, well supported by simulations. 
%
%\end{abstract}
% IEEEtran.cls defaults to using nonbold math in the Abstract.
% This preserves the distinction between vectors and scalars. However,
% if the journal you are submitting to favors bold math in the abstract,
% then you can use LaTeX's standard command \boldmath at the very start
% of the abstract to achieve this. Many IEEE journals frown on math
% in the abstract anyway.

% Note that keywords are not normally used for peerreview papers.
%\begin{IEEEkeywords}
%Cooperative diversity, decode and forward, piecewise linear
%\end{IEEEkeywords}



% For peer review papers, you can put extra information on the cover
% page as needed:
% \ifCLASSOPTIONpeerreview
% \begin{center} \bfseries EDICS Category: 3-BBND \end{center}
% \fi
%
% For peerreview papers, this IEEEtran command inserts a page break and
% creates the second title. It will be ignored for other modes.
%\IEEEpeerreviewmaketitle




  \item
  Cards with numbers 2 to 101 are placed in a box. A card is selected at random.Find the probability that the card has
\begin{enumerate}[label=(\roman*)]
	\item an even number 
	\item a square number
\end{enumerate}
\solution
%\begin{table}[H]
	\centering
\begin{tabular}{|c|c|c|}
\hline
Random variable &Value &Definition\\ \hline
\multirow{3}{*}{X} &0 &Slips of Rs 1\\
&1 &Slips of Rs 5\\
&2 &Slips of Rs 13\\ \hline
\multirow{2}{*}{Y} &0 &Box A\\
&1 &Box B\\\hline
\end{tabular}
\caption{}
\label{tab:Distribution}
\end{table}
See \tabref{tab:Distribution}.
\begin{align}
p_{Y}\brak{k}= \begin{cases} 
      \frac{1}{3} & {k=0} \\
      \frac{2}{3 }& {k=1} 
   \end{cases}
   \\
p_{Y|X}\brak{0|0} = \frac{19}{25}\, 
p_{Y|X}\brak{0|1} = \frac{6}{25}\,
p_{Y|X}\brak{1|0} = \frac{45}{50}\,
p_{Y|X}\brak{1|2} = \frac{5}{50}
\end{align}
The desired probability is the probability that a slip drawn at random is marked other than Rs 1,
\begin{align}
&=1-p_X\brak{0}\\
&= p_X(1) + p_X(2)
\end{align}
Using Bayes theorem,
\begin{align}
&= p_Y\brak{0} \times \pr{Y=0 | X=1} + p_Y\brak{1} \times \pr{Y=1|X=2}\\
&=\frac{1}{3} \times \frac{6}{25} + \frac{2}{3} \times \frac{5}{50}\\
&=\frac{11}{75}
\end{align}

\newpage

%\tableofcontents

\bigskip

\renewcommand{\thefigure}{\theenumi}
\renewcommand{\thetable}{\theenumi}
%\renewcommand{\theequation}{\theenumi}

%\begin{abstract}
%%\boldmath
%In this letter, an algorithm for evaluating the exact analytical bit error rate  (BER)  for the piecewise linear (PL) combiner for  multiple relays is presented. Previous results were available only for upto three relays. The algorithm is unique in the sense that  the actual mathematical expressions, that are prohibitively large, need not be explicitly obtained. The diversity gain due to multiple relays is shown through plots of the analytical BER, well supported by simulations. 
%
%\end{abstract}
% IEEEtran.cls defaults to using nonbold math in the Abstract.
% This preserves the distinction between vectors and scalars. However,
% if the journal you are submitting to favors bold math in the abstract,
% then you can use LaTeX's standard command \boldmath at the very start
% of the abstract to achieve this. Many IEEE journals frown on math
% in the abstract anyway.

% Note that keywords are not normally used for peerreview papers.
%\begin{IEEEkeywords}
%Cooperative diversity, decode and forward, piecewise linear
%\end{IEEEkeywords}



% For peer review papers, you can put extra information on the cover
% page as needed:
% \ifCLASSOPTIONpeerreview
% \begin{center} \bfseries EDICS Category: 3-BBND \end{center}
% \fi
%
% For peerreview papers, this IEEEtran command inserts a page break and
% creates the second title. It will be ignored for other modes.
%\IEEEpeerreviewmaketitle




\item
The king, queen and jack of clubs are removed from a deck of 52 playing cards and then well shuffled. Now one card is drawn at random from the remaining cards.  Determine the probability that the card is
\begin{enumerate}[label=(\roman*)]
\item a club
\item 10 of hearts
\end{enumerate}
\solution
%\begin{table}[H]
	\centering
\begin{tabular}{|c|c|c|}
\hline
Random variable &Value &Definition\\ \hline
\multirow{3}{*}{X} &0 &Slips of Rs 1\\
&1 &Slips of Rs 5\\
&2 &Slips of Rs 13\\ \hline
\multirow{2}{*}{Y} &0 &Box A\\
&1 &Box B\\\hline
\end{tabular}
\caption{}
\label{tab:Distribution}
\end{table}
See \tabref{tab:Distribution}.
\begin{align}
p_{Y}\brak{k}= \begin{cases} 
      \frac{1}{3} & {k=0} \\
      \frac{2}{3 }& {k=1} 
   \end{cases}
   \\
p_{Y|X}\brak{0|0} = \frac{19}{25}\, 
p_{Y|X}\brak{0|1} = \frac{6}{25}\,
p_{Y|X}\brak{1|0} = \frac{45}{50}\,
p_{Y|X}\brak{1|2} = \frac{5}{50}
\end{align}
The desired probability is the probability that a slip drawn at random is marked other than Rs 1,
\begin{align}
&=1-p_X\brak{0}\\
&= p_X(1) + p_X(2)
\end{align}
Using Bayes theorem,
\begin{align}
&= p_Y\brak{0} \times \pr{Y=0 | X=1} + p_Y\brak{1} \times \pr{Y=1|X=2}\\
&=\frac{1}{3} \times \frac{6}{25} + \frac{2}{3} \times \frac{5}{50}\\
&=\frac{11}{75}
\end{align}

\newpage

%\tableofcontents

\bigskip

\renewcommand{\thefigure}{\theenumi}
\renewcommand{\thetable}{\theenumi}
%\renewcommand{\theequation}{\theenumi}

%\begin{abstract}
%%\boldmath
%In this letter, an algorithm for evaluating the exact analytical bit error rate  (BER)  for the piecewise linear (PL) combiner for  multiple relays is presented. Previous results were available only for upto three relays. The algorithm is unique in the sense that  the actual mathematical expressions, that are prohibitively large, need not be explicitly obtained. The diversity gain due to multiple relays is shown through plots of the analytical BER, well supported by simulations. 
%
%\end{abstract}
% IEEEtran.cls defaults to using nonbold math in the Abstract.
% This preserves the distinction between vectors and scalars. However,
% if the journal you are submitting to favors bold math in the abstract,
% then you can use LaTeX's standard command \boldmath at the very start
% of the abstract to achieve this. Many IEEE journals frown on math
% in the abstract anyway.

% Note that keywords are not normally used for peerreview papers.
%\begin{IEEEkeywords}
%Cooperative diversity, decode and forward, piecewise linear
%\end{IEEEkeywords}



% For peer review papers, you can put extra information on the cover
% page as needed:
% \ifCLASSOPTIONpeerreview
% \begin{center} \bfseries EDICS Category: 3-BBND \end{center}
% \fi
%
% For peerreview papers, this IEEEtran command inserts a page break and
% creates the second title. It will be ignored for other modes.
%\IEEEpeerreviewmaketitle




\item A team of medical students doing their internship have to assist during surgeries
at a city hospital. The probabilities of surgeries rated as very complex, complex,
routine, simple or very simple are respectively, 0.15, 0.20, 0.31, 0.26, .08. Find
the probabilities that a particular surgery will be rated
\begin{enumerate}
	\item complex or very complex;
	\item neither very complex nor very simple;
	\item routine or complex
	\item routine or simple
\end{enumerate}
\solution
%\begin{table}[H]
	\centering
\begin{tabular}{|c|c|c|}
\hline
Random variable &Value &Definition\\ \hline
\multirow{3}{*}{X} &0 &Slips of Rs 1\\
&1 &Slips of Rs 5\\
&2 &Slips of Rs 13\\ \hline
\multirow{2}{*}{Y} &0 &Box A\\
&1 &Box B\\\hline
\end{tabular}
\caption{}
\label{tab:Distribution}
\end{table}
See \tabref{tab:Distribution}.
\begin{align}
p_{Y}\brak{k}= \begin{cases} 
      \frac{1}{3} & {k=0} \\
      \frac{2}{3 }& {k=1} 
   \end{cases}
   \\
p_{Y|X}\brak{0|0} = \frac{19}{25}\, 
p_{Y|X}\brak{0|1} = \frac{6}{25}\,
p_{Y|X}\brak{1|0} = \frac{45}{50}\,
p_{Y|X}\brak{1|2} = \frac{5}{50}
\end{align}
The desired probability is the probability that a slip drawn at random is marked other than Rs 1,
\begin{align}
&=1-p_X\brak{0}\\
&= p_X(1) + p_X(2)
\end{align}
Using Bayes theorem,
\begin{align}
&= p_Y\brak{0} \times \pr{Y=0 | X=1} + p_Y\brak{1} \times \pr{Y=1|X=2}\\
&=\frac{1}{3} \times \frac{6}{25} + \frac{2}{3} \times \frac{5}{50}\\
&=\frac{11}{75}
\end{align}

\newpage

%\tableofcontents

\bigskip

\renewcommand{\thefigure}{\theenumi}
\renewcommand{\thetable}{\theenumi}
%\renewcommand{\theequation}{\theenumi}

%\begin{abstract}
%%\boldmath
%In this letter, an algorithm for evaluating the exact analytical bit error rate  (BER)  for the piecewise linear (PL) combiner for  multiple relays is presented. Previous results were available only for upto three relays. The algorithm is unique in the sense that  the actual mathematical expressions, that are prohibitively large, need not be explicitly obtained. The diversity gain due to multiple relays is shown through plots of the analytical BER, well supported by simulations. 
%
%\end{abstract}
% IEEEtran.cls defaults to using nonbold math in the Abstract.
% This preserves the distinction between vectors and scalars. However,
% if the journal you are submitting to favors bold math in the abstract,
% then you can use LaTeX's standard command \boldmath at the very start
% of the abstract to achieve this. Many IEEE journals frown on math
% in the abstract anyway.

% Note that keywords are not normally used for peerreview papers.
%\begin{IEEEkeywords}
%Cooperative diversity, decode and forward, piecewise linear
%\end{IEEEkeywords}



% For peer review papers, you can put extra information on the cover
% page as needed:
% \ifCLASSOPTIONpeerreview
% \begin{center} \bfseries EDICS Category: 3-BBND \end{center}
% \fi
%
% For peerreview papers, this IEEEtran command inserts a page break and
% creates the second title. It will be ignored for other modes.
%\IEEEpeerreviewmaketitle




\item A card is selected from a pack of 52 cards.
\begin{enumerate}[label=(\alph*)]
    \item How many points are there in the sample space?
    \item Calculate the probability that the card is an ace of spades.
    \item Calculate the probability that the card is (i) an ace and (ii) black card.
\end{enumerate}
\solution
%Let $X$ be an bernoulli rv defined as in \tabref{tab:exemplar/11/16/3/26}.  Then, 
\begin{equation}
    p =
        \frac{4}{11} 
\end{equation}
\begin{table}[H]
	\centering
	\input{exemplar/11/16/3/26/tables/Table2.tex}
	\caption{}
        \label{tab:exemplar/11/16/3/26}
\end{table}

\item The probability that a non leap year selected at random will contain 53 sundays.
\\
\solution
%\begin{table}[H]
	\centering
\begin{tabular}{|c|c|c|}
\hline
Random variable &Value &Definition\\ \hline
\multirow{3}{*}{X} &0 &Slips of Rs 1\\
&1 &Slips of Rs 5\\
&2 &Slips of Rs 13\\ \hline
\multirow{2}{*}{Y} &0 &Box A\\
&1 &Box B\\\hline
\end{tabular}
\caption{}
\label{tab:Distribution}
\end{table}
See \tabref{tab:Distribution}.
\begin{align}
p_{Y}\brak{k}= \begin{cases} 
      \frac{1}{3} & {k=0} \\
      \frac{2}{3 }& {k=1} 
   \end{cases}
   \\
p_{Y|X}\brak{0|0} = \frac{19}{25}\, 
p_{Y|X}\brak{0|1} = \frac{6}{25}\,
p_{Y|X}\brak{1|0} = \frac{45}{50}\,
p_{Y|X}\brak{1|2} = \frac{5}{50}
\end{align}
The desired probability is the probability that a slip drawn at random is marked other than Rs 1,
\begin{align}
&=1-p_X\brak{0}\\
&= p_X(1) + p_X(2)
\end{align}
Using Bayes theorem,
\begin{align}
&= p_Y\brak{0} \times \pr{Y=0 | X=1} + p_Y\brak{1} \times \pr{Y=1|X=2}\\
&=\frac{1}{3} \times \frac{6}{25} + \frac{2}{3} \times \frac{5}{50}\\
&=\frac{11}{75}
\end{align}

\newpage

%\tableofcontents

\bigskip

\renewcommand{\thefigure}{\theenumi}
\renewcommand{\thetable}{\theenumi}
%\renewcommand{\theequation}{\theenumi}

%\begin{abstract}
%%\boldmath
%In this letter, an algorithm for evaluating the exact analytical bit error rate  (BER)  for the piecewise linear (PL) combiner for  multiple relays is presented. Previous results were available only for upto three relays. The algorithm is unique in the sense that  the actual mathematical expressions, that are prohibitively large, need not be explicitly obtained. The diversity gain due to multiple relays is shown through plots of the analytical BER, well supported by simulations. 
%
%\end{abstract}
% IEEEtran.cls defaults to using nonbold math in the Abstract.
% This preserves the distinction between vectors and scalars. However,
% if the journal you are submitting to favors bold math in the abstract,
% then you can use LaTeX's standard command \boldmath at the very start
% of the abstract to achieve this. Many IEEE journals frown on math
% in the abstract anyway.

% Note that keywords are not normally used for peerreview papers.
%\begin{IEEEkeywords}
%Cooperative diversity, decode and forward, piecewise linear
%\end{IEEEkeywords}



% For peer review papers, you can put extra information on the cover
% page as needed:
% \ifCLASSOPTIONpeerreview
% \begin{center} \bfseries EDICS Category: 3-BBND \end{center}
% \fi
%
% For peerreview papers, this IEEEtran command inserts a page break and
% creates the second title. It will be ignored for other modes.
%\IEEEpeerreviewmaketitle




\item One of the four persons John, Rita, Aslam or Gurpreet will be promoted next
month. Consequently the sample space consists of four elementary outcomes
S = {John promoted, Rita promoted, Aslam promoted, Gurpreet promoted}
You are told that the chances of John’s promotion is same as that of Gurpreet,
Rita’s chances of promotion are twice as likely as Johns. Aslam’s chances are
four times that of John.
\begin{enumerate}
	\item Determine
	\begin{enumerate}
		\item P (John promoted)
		\item P (Rita promoted)
		\item P (Aslam promoted)
		\item P (Gurpreet promoted)
	\end{enumerate}
	\item If A = {John promoted or Gurpreet promoted}, find P (A).
\end{enumerate}
\solution
%\begin{table}[H]
	\centering
\begin{tabular}{|c|c|c|}
\hline
Random variable &Value &Definition\\ \hline
\multirow{3}{*}{X} &0 &Slips of Rs 1\\
&1 &Slips of Rs 5\\
&2 &Slips of Rs 13\\ \hline
\multirow{2}{*}{Y} &0 &Box A\\
&1 &Box B\\\hline
\end{tabular}
\caption{}
\label{tab:Distribution}
\end{table}
See \tabref{tab:Distribution}.
\begin{align}
p_{Y}\brak{k}= \begin{cases} 
      \frac{1}{3} & {k=0} \\
      \frac{2}{3 }& {k=1} 
   \end{cases}
   \\
p_{Y|X}\brak{0|0} = \frac{19}{25}\, 
p_{Y|X}\brak{0|1} = \frac{6}{25}\,
p_{Y|X}\brak{1|0} = \frac{45}{50}\,
p_{Y|X}\brak{1|2} = \frac{5}{50}
\end{align}
The desired probability is the probability that a slip drawn at random is marked other than Rs 1,
\begin{align}
&=1-p_X\brak{0}\\
&= p_X(1) + p_X(2)
\end{align}
Using Bayes theorem,
\begin{align}
&= p_Y\brak{0} \times \pr{Y=0 | X=1} + p_Y\brak{1} \times \pr{Y=1|X=2}\\
&=\frac{1}{3} \times \frac{6}{25} + \frac{2}{3} \times \frac{5}{50}\\
&=\frac{11}{75}
\end{align}

\newpage

%\tableofcontents

\bigskip

\renewcommand{\thefigure}{\theenumi}
\renewcommand{\thetable}{\theenumi}
%\renewcommand{\theequation}{\theenumi}

%\begin{abstract}
%%\boldmath
%In this letter, an algorithm for evaluating the exact analytical bit error rate  (BER)  for the piecewise linear (PL) combiner for  multiple relays is presented. Previous results were available only for upto three relays. The algorithm is unique in the sense that  the actual mathematical expressions, that are prohibitively large, need not be explicitly obtained. The diversity gain due to multiple relays is shown through plots of the analytical BER, well supported by simulations. 
%
%\end{abstract}
% IEEEtran.cls defaults to using nonbold math in the Abstract.
% This preserves the distinction between vectors and scalars. However,
% if the journal you are submitting to favors bold math in the abstract,
% then you can use LaTeX's standard command \boldmath at the very start
% of the abstract to achieve this. Many IEEE journals frown on math
% in the abstract anyway.

% Note that keywords are not normally used for peerreview papers.
%\begin{IEEEkeywords}
%Cooperative diversity, decode and forward, piecewise linear
%\end{IEEEkeywords}



% For peer review papers, you can put extra information on the cover
% page as needed:
% \ifCLASSOPTIONpeerreview
% \begin{center} \bfseries EDICS Category: 3-BBND \end{center}
% \fi
%
% For peerreview papers, this IEEEtran command inserts a page break and
% creates the second title. It will be ignored for other modes.
%\IEEEpeerreviewmaketitle




\item A card is drawn from a deck of 52 cards. Find the probability of getting a king or a heart or a red card.\\
\solution
%\begin{table}[H]
	\centering
\begin{tabular}{|c|c|c|}
\hline
Random variable &Value &Definition\\ \hline
\multirow{3}{*}{X} &0 &Slips of Rs 1\\
&1 &Slips of Rs 5\\
&2 &Slips of Rs 13\\ \hline
\multirow{2}{*}{Y} &0 &Box A\\
&1 &Box B\\\hline
\end{tabular}
\caption{}
\label{tab:Distribution}
\end{table}
See \tabref{tab:Distribution}.
\begin{align}
p_{Y}\brak{k}= \begin{cases} 
      \frac{1}{3} & {k=0} \\
      \frac{2}{3 }& {k=1} 
   \end{cases}
   \\
p_{Y|X}\brak{0|0} = \frac{19}{25}\, 
p_{Y|X}\brak{0|1} = \frac{6}{25}\,
p_{Y|X}\brak{1|0} = \frac{45}{50}\,
p_{Y|X}\brak{1|2} = \frac{5}{50}
\end{align}
The desired probability is the probability that a slip drawn at random is marked other than Rs 1,
\begin{align}
&=1-p_X\brak{0}\\
&= p_X(1) + p_X(2)
\end{align}
Using Bayes theorem,
\begin{align}
&= p_Y\brak{0} \times \pr{Y=0 | X=1} + p_Y\brak{1} \times \pr{Y=1|X=2}\\
&=\frac{1}{3} \times \frac{6}{25} + \frac{2}{3} \times \frac{5}{50}\\
&=\frac{11}{75}
\end{align}

\newpage

%\tableofcontents

\bigskip

\renewcommand{\thefigure}{\theenumi}
\renewcommand{\thetable}{\theenumi}
%\renewcommand{\theequation}{\theenumi}

%\begin{abstract}
%%\boldmath
%In this letter, an algorithm for evaluating the exact analytical bit error rate  (BER)  for the piecewise linear (PL) combiner for  multiple relays is presented. Previous results were available only for upto three relays. The algorithm is unique in the sense that  the actual mathematical expressions, that are prohibitively large, need not be explicitly obtained. The diversity gain due to multiple relays is shown through plots of the analytical BER, well supported by simulations. 
%
%\end{abstract}
% IEEEtran.cls defaults to using nonbold math in the Abstract.
% This preserves the distinction between vectors and scalars. However,
% if the journal you are submitting to favors bold math in the abstract,
% then you can use LaTeX's standard command \boldmath at the very start
% of the abstract to achieve this. Many IEEE journals frown on math
% in the abstract anyway.

% Note that keywords are not normally used for peerreview papers.
%\begin{IEEEkeywords}
%Cooperative diversity, decode and forward, piecewise linear
%\end{IEEEkeywords}



% For peer review papers, you can put extra information on the cover
% page as needed:
% \ifCLASSOPTIONpeerreview
% \begin{center} \bfseries EDICS Category: 3-BBND \end{center}
% \fi
%
% For peerreview papers, this IEEEtran command inserts a page break and
% creates the second title. It will be ignored for other modes.
%\IEEEpeerreviewmaketitle




\item The probability that a student will pass his examination is 0.73, the probability of
the student getting a compartment is 0.13, and the probability that the student will
either pass or get compartment is 0.96. State True or False.\\
\solution
%\begin{table}[H]
	\centering
\begin{tabular}{|c|c|c|}
\hline
Random variable &Value &Definition\\ \hline
\multirow{3}{*}{X} &0 &Slips of Rs 1\\
&1 &Slips of Rs 5\\
&2 &Slips of Rs 13\\ \hline
\multirow{2}{*}{Y} &0 &Box A\\
&1 &Box B\\\hline
\end{tabular}
\caption{}
\label{tab:Distribution}
\end{table}
See \tabref{tab:Distribution}.
\begin{align}
p_{Y}\brak{k}= \begin{cases} 
      \frac{1}{3} & {k=0} \\
      \frac{2}{3 }& {k=1} 
   \end{cases}
   \\
p_{Y|X}\brak{0|0} = \frac{19}{25}\, 
p_{Y|X}\brak{0|1} = \frac{6}{25}\,
p_{Y|X}\brak{1|0} = \frac{45}{50}\,
p_{Y|X}\brak{1|2} = \frac{5}{50}
\end{align}
The desired probability is the probability that a slip drawn at random is marked other than Rs 1,
\begin{align}
&=1-p_X\brak{0}\\
&= p_X(1) + p_X(2)
\end{align}
Using Bayes theorem,
\begin{align}
&= p_Y\brak{0} \times \pr{Y=0 | X=1} + p_Y\brak{1} \times \pr{Y=1|X=2}\\
&=\frac{1}{3} \times \frac{6}{25} + \frac{2}{3} \times \frac{5}{50}\\
&=\frac{11}{75}
\end{align}

\newpage

%\tableofcontents

\bigskip

\renewcommand{\thefigure}{\theenumi}
\renewcommand{\thetable}{\theenumi}
%\renewcommand{\theequation}{\theenumi}

%\begin{abstract}
%%\boldmath
%In this letter, an algorithm for evaluating the exact analytical bit error rate  (BER)  for the piecewise linear (PL) combiner for  multiple relays is presented. Previous results were available only for upto three relays. The algorithm is unique in the sense that  the actual mathematical expressions, that are prohibitively large, need not be explicitly obtained. The diversity gain due to multiple relays is shown through plots of the analytical BER, well supported by simulations. 
%
%\end{abstract}
% IEEEtran.cls defaults to using nonbold math in the Abstract.
% This preserves the distinction between vectors and scalars. However,
% if the journal you are submitting to favors bold math in the abstract,
% then you can use LaTeX's standard command \boldmath at the very start
% of the abstract to achieve this. Many IEEE journals frown on math
% in the abstract anyway.

% Note that keywords are not normally used for peerreview papers.
%\begin{IEEEkeywords}
%Cooperative diversity, decode and forward, piecewise linear
%\end{IEEEkeywords}



% For peer review papers, you can put extra information on the cover
% page as needed:
% \ifCLASSOPTIONpeerreview
% \begin{center} \bfseries EDICS Category: 3-BBND \end{center}
% \fi
%
% For peerreview papers, this IEEEtran command inserts a page break and
% creates the second title. It will be ignored for other modes.
%\IEEEpeerreviewmaketitle




\item A card is selected from a pack of 52 cards\\
\begin{enumerate}[label=(\alph*)]
\item How many points are there in the sample space?
\item Calculate the probability that the cards is an ace of spades.
\item Calculate the probability that the card is (i) an ace (ii)black card.\\
\end{enumerate}
%\input{ncert/11/16/3/4_1/Prob_4.tex}
\item In a non-leap year, the probability of having 53 tuesdays or 53 wednesdays is\\
\solution
%A non-leap year has a total of 365 days, and a week has 7 days.\\
So it can be expressed as 
\begin{align}
365\text{days} &=52\times 7+1 \text{day}
\end{align}
$\implies$ 52 tuesdays or wednesdays\\
Random variable X denotes the days of a week
\begin{align}
p_X\brak{k}&=\frac{1}{7}; \quad \brak{1<k<7}
\end{align}
So the probability of extra day being tuesday or wednesday is
\begin{align}
p_X\brak{3}+p_X\brak{4}&=\frac{1}{7}+\frac{1}{7}=\frac{2}{7}
\end{align}



\item There are 1000 sealed envelopes in a box, 10 of them contain a cash prize of
Rs 100 each, 100 of them contain a cash prize of Rs 50 each and 200 of them
contain a cash prize of Rs 10 each and rest do not contain any cash prize. If they
are well shuffled and an envelope is picked up out, what is the probability that it
contains no cash prize?\\
\solution
%\begin{table}[H]
	\centering
\begin{tabular}{|c|c|c|}
\hline
Random variable &Value &Definition\\ \hline
\multirow{3}{*}{X} &0 &Slips of Rs 1\\
&1 &Slips of Rs 5\\
&2 &Slips of Rs 13\\ \hline
\multirow{2}{*}{Y} &0 &Box A\\
&1 &Box B\\\hline
\end{tabular}
\caption{}
\label{tab:Distribution}
\end{table}
See \tabref{tab:Distribution}.
\begin{align}
p_{Y}\brak{k}= \begin{cases} 
      \frac{1}{3} & {k=0} \\
      \frac{2}{3 }& {k=1} 
   \end{cases}
   \\
p_{Y|X}\brak{0|0} = \frac{19}{25}\, 
p_{Y|X}\brak{0|1} = \frac{6}{25}\,
p_{Y|X}\brak{1|0} = \frac{45}{50}\,
p_{Y|X}\brak{1|2} = \frac{5}{50}
\end{align}
The desired probability is the probability that a slip drawn at random is marked other than Rs 1,
\begin{align}
&=1-p_X\brak{0}\\
&= p_X(1) + p_X(2)
\end{align}
Using Bayes theorem,
\begin{align}
&= p_Y\brak{0} \times \pr{Y=0 | X=1} + p_Y\brak{1} \times \pr{Y=1|X=2}\\
&=\frac{1}{3} \times \frac{6}{25} + \frac{2}{3} \times \frac{5}{50}\\
&=\frac{11}{75}
\end{align}

\newpage

%\tableofcontents

\bigskip

\renewcommand{\thefigure}{\theenumi}
\renewcommand{\thetable}{\theenumi}
%\renewcommand{\theequation}{\theenumi}

%\begin{abstract}
%%\boldmath
%In this letter, an algorithm for evaluating the exact analytical bit error rate  (BER)  for the piecewise linear (PL) combiner for  multiple relays is presented. Previous results were available only for upto three relays. The algorithm is unique in the sense that  the actual mathematical expressions, that are prohibitively large, need not be explicitly obtained. The diversity gain due to multiple relays is shown through plots of the analytical BER, well supported by simulations. 
%
%\end{abstract}
% IEEEtran.cls defaults to using nonbold math in the Abstract.
% This preserves the distinction between vectors and scalars. However,
% if the journal you are submitting to favors bold math in the abstract,
% then you can use LaTeX's standard command \boldmath at the very start
% of the abstract to achieve this. Many IEEE journals frown on math
% in the abstract anyway.

% Note that keywords are not normally used for peerreview papers.
%\begin{IEEEkeywords}
%Cooperative diversity, decode and forward, piecewise linear
%\end{IEEEkeywords}



% For peer review papers, you can put extra information on the cover
% page as needed:
% \ifCLASSOPTIONpeerreview
% \begin{center} \bfseries EDICS Category: 3-BBND \end{center}
% \fi
%
% For peerreview papers, this IEEEtran command inserts a page break and
% creates the second title. It will be ignored for other modes.
%\IEEEpeerreviewmaketitle




\item 
A die is thrown and a card is selected at random from a deck of 52 playing cards. The probability of getting an even number on the die and a spade card.\\
\solution
%\begin{table}[H]
	\centering
\begin{tabular}{|c|c|c|}
\hline
Random variable &Value &Definition\\ \hline
\multirow{3}{*}{X} &0 &Slips of Rs 1\\
&1 &Slips of Rs 5\\
&2 &Slips of Rs 13\\ \hline
\multirow{2}{*}{Y} &0 &Box A\\
&1 &Box B\\\hline
\end{tabular}
\caption{}
\label{tab:Distribution}
\end{table}
See \tabref{tab:Distribution}.
\begin{align}
p_{Y}\brak{k}= \begin{cases} 
      \frac{1}{3} & {k=0} \\
      \frac{2}{3 }& {k=1} 
   \end{cases}
   \\
p_{Y|X}\brak{0|0} = \frac{19}{25}\, 
p_{Y|X}\brak{0|1} = \frac{6}{25}\,
p_{Y|X}\brak{1|0} = \frac{45}{50}\,
p_{Y|X}\brak{1|2} = \frac{5}{50}
\end{align}
The desired probability is the probability that a slip drawn at random is marked other than Rs 1,
\begin{align}
&=1-p_X\brak{0}\\
&= p_X(1) + p_X(2)
\end{align}
Using Bayes theorem,
\begin{align}
&= p_Y\brak{0} \times \pr{Y=0 | X=1} + p_Y\brak{1} \times \pr{Y=1|X=2}\\
&=\frac{1}{3} \times \frac{6}{25} + \frac{2}{3} \times \frac{5}{50}\\
&=\frac{11}{75}
\end{align}

\newpage

%\tableofcontents

\bigskip

\renewcommand{\thefigure}{\theenumi}
\renewcommand{\thetable}{\theenumi}
%\renewcommand{\theequation}{\theenumi}

%\begin{abstract}
%%\boldmath
%In this letter, an algorithm for evaluating the exact analytical bit error rate  (BER)  for the piecewise linear (PL) combiner for  multiple relays is presented. Previous results were available only for upto three relays. The algorithm is unique in the sense that  the actual mathematical expressions, that are prohibitively large, need not be explicitly obtained. The diversity gain due to multiple relays is shown through plots of the analytical BER, well supported by simulations. 
%
%\end{abstract}
% IEEEtran.cls defaults to using nonbold math in the Abstract.
% This preserves the distinction between vectors and scalars. However,
% if the journal you are submitting to favors bold math in the abstract,
% then you can use LaTeX's standard command \boldmath at the very start
% of the abstract to achieve this. Many IEEE journals frown on math
% in the abstract anyway.

% Note that keywords are not normally used for peerreview papers.
%\begin{IEEEkeywords}
%Cooperative diversity, decode and forward, piecewise linear
%\end{IEEEkeywords}



% For peer review papers, you can put extra information on the cover
% page as needed:
% \ifCLASSOPTIONpeerreview
% \begin{center} \bfseries EDICS Category: 3-BBND \end{center}
% \fi
%
% For peerreview papers, this IEEEtran command inserts a page break and
% creates the second title. It will be ignored for other modes.
%\IEEEpeerreviewmaketitle




\item
If 4-digit numbers greater than 5,000 are randomly formed from the digits 0, 1, 3, 5, and 7, what is the probability of forming a number divisible by 5 when:
\begin{enumerate}
    \item The digits are repeated?
    \item The repetition of digits is not allowed?
\end{enumerate}
\solution
%\begin{table}[H]
	\centering
\begin{tabular}{|c|c|c|}
\hline
Random variable &Value &Definition\\ \hline
\multirow{3}{*}{X} &0 &Slips of Rs 1\\
&1 &Slips of Rs 5\\
&2 &Slips of Rs 13\\ \hline
\multirow{2}{*}{Y} &0 &Box A\\
&1 &Box B\\\hline
\end{tabular}
\caption{}
\label{tab:Distribution}
\end{table}
See \tabref{tab:Distribution}.
\begin{align}
p_{Y}\brak{k}= \begin{cases} 
      \frac{1}{3} & {k=0} \\
      \frac{2}{3 }& {k=1} 
   \end{cases}
   \\
p_{Y|X}\brak{0|0} = \frac{19}{25}\, 
p_{Y|X}\brak{0|1} = \frac{6}{25}\,
p_{Y|X}\brak{1|0} = \frac{45}{50}\,
p_{Y|X}\brak{1|2} = \frac{5}{50}
\end{align}
The desired probability is the probability that a slip drawn at random is marked other than Rs 1,
\begin{align}
&=1-p_X\brak{0}\\
&= p_X(1) + p_X(2)
\end{align}
Using Bayes theorem,
\begin{align}
&= p_Y\brak{0} \times \pr{Y=0 | X=1} + p_Y\brak{1} \times \pr{Y=1|X=2}\\
&=\frac{1}{3} \times \frac{6}{25} + \frac{2}{3} \times \frac{5}{50}\\
&=\frac{11}{75}
\end{align}

\newpage

%\tableofcontents

\bigskip

\renewcommand{\thefigure}{\theenumi}
\renewcommand{\thetable}{\theenumi}
%\renewcommand{\theequation}{\theenumi}

%\begin{abstract}
%%\boldmath
%In this letter, an algorithm for evaluating the exact analytical bit error rate  (BER)  for the piecewise linear (PL) combiner for  multiple relays is presented. Previous results were available only for upto three relays. The algorithm is unique in the sense that  the actual mathematical expressions, that are prohibitively large, need not be explicitly obtained. The diversity gain due to multiple relays is shown through plots of the analytical BER, well supported by simulations. 
%
%\end{abstract}
% IEEEtran.cls defaults to using nonbold math in the Abstract.
% This preserves the distinction between vectors and scalars. However,
% if the journal you are submitting to favors bold math in the abstract,
% then you can use LaTeX's standard command \boldmath at the very start
% of the abstract to achieve this. Many IEEE journals frown on math
% in the abstract anyway.

% Note that keywords are not normally used for peerreview papers.
%\begin{IEEEkeywords}
%Cooperative diversity, decode and forward, piecewise linear
%\end{IEEEkeywords}



% For peer review papers, you can put extra information on the cover
% page as needed:
% \ifCLASSOPTIONpeerreview
% \begin{center} \bfseries EDICS Category: 3-BBND \end{center}
% \fi
%
% For peerreview papers, this IEEEtran command inserts a page break and
% creates the second title. It will be ignored for other modes.
%\IEEEpeerreviewmaketitle




\item Consider the probability space $\brak{\Omega, \mathcal{G}, P}$ where $\Omega = [0,2]$ and $\mathcal{G} = \cbrak{\phi, \Omega, [0,1], (1,2]}$. Let $X$ and $Y$ be two functions on $\Omega$ defined as
\begin{align*}
    X(\omega) = 
    \begin{cases}
        1 & \text{if }\omega \in [0, 1]\\
        2 & \text{if }\omega \in (1, 2]
    \end{cases}
\end{align*}
and
\begin{align*}
    Y(\omega) = 
    \begin{cases}
        2 & \text{if }\omega \in [0, 1.5]\\
        3 & \text{if }\omega \in (1.5, 2].
    \end{cases}
\end{align*}
Then which one of the following statements is true?
\begin{enumerate}
    \item [(A)] $X$ is a random variable with respect to $\mathcal{G}$, but $Y$ is not a random variable with respect to $\mathcal{G}$.
    \item [(B)] $Y$ is a random variable with respect to $\mathcal{G}$, but $X$ is not a random variable with respect to $\mathcal{G}$.
    \item [(C)] Neither $X$ nor $Y$ is a random variable with respect to $\mathcal{G}$.
    \item [(D)] Both $X$ and $Y$ are random variables with respect to $\mathcal{G}$.
\end{enumerate} \hfill (GATE ST 2023)\\
\solution
%\begin{table}[H]
	\centering
\begin{tabular}{|c|c|c|}
\hline
Random variable &Value &Definition\\ \hline
\multirow{3}{*}{X} &0 &Slips of Rs 1\\
&1 &Slips of Rs 5\\
&2 &Slips of Rs 13\\ \hline
\multirow{2}{*}{Y} &0 &Box A\\
&1 &Box B\\\hline
\end{tabular}
\caption{}
\label{tab:Distribution}
\end{table}
See \tabref{tab:Distribution}.
\begin{align}
p_{Y}\brak{k}= \begin{cases} 
      \frac{1}{3} & {k=0} \\
      \frac{2}{3 }& {k=1} 
   \end{cases}
   \\
p_{Y|X}\brak{0|0} = \frac{19}{25}\, 
p_{Y|X}\brak{0|1} = \frac{6}{25}\,
p_{Y|X}\brak{1|0} = \frac{45}{50}\,
p_{Y|X}\brak{1|2} = \frac{5}{50}
\end{align}
The desired probability is the probability that a slip drawn at random is marked other than Rs 1,
\begin{align}
&=1-p_X\brak{0}\\
&= p_X(1) + p_X(2)
\end{align}
Using Bayes theorem,
\begin{align}
&= p_Y\brak{0} \times \pr{Y=0 | X=1} + p_Y\brak{1} \times \pr{Y=1|X=2}\\
&=\frac{1}{3} \times \frac{6}{25} + \frac{2}{3} \times \frac{5}{50}\\
&=\frac{11}{75}
\end{align}

\newpage

%\tableofcontents

\bigskip

\renewcommand{\thefigure}{\theenumi}
\renewcommand{\thetable}{\theenumi}
%\renewcommand{\theequation}{\theenumi}

%\begin{abstract}
%%\boldmath
%In this letter, an algorithm for evaluating the exact analytical bit error rate  (BER)  for the piecewise linear (PL) combiner for  multiple relays is presented. Previous results were available only for upto three relays. The algorithm is unique in the sense that  the actual mathematical expressions, that are prohibitively large, need not be explicitly obtained. The diversity gain due to multiple relays is shown through plots of the analytical BER, well supported by simulations. 
%
%\end{abstract}
% IEEEtran.cls defaults to using nonbold math in the Abstract.
% This preserves the distinction between vectors and scalars. However,
% if the journal you are submitting to favors bold math in the abstract,
% then you can use LaTeX's standard command \boldmath at the very start
% of the abstract to achieve this. Many IEEE journals frown on math
% in the abstract anyway.

% Note that keywords are not normally used for peerreview papers.
%\begin{IEEEkeywords}
%Cooperative diversity, decode and forward, piecewise linear
%\end{IEEEkeywords}



% For peer review papers, you can put extra information on the cover
% page as needed:
% \ifCLASSOPTIONpeerreview
% \begin{center} \bfseries EDICS Category: 3-BBND \end{center}
% \fi
%
% For peerreview papers, this IEEEtran command inserts a page break and
% creates the second title. It will be ignored for other modes.
%\IEEEpeerreviewmaketitle




	\item  A die is loaded in such a way that each odd number is twice as likely to occur as
each even number. Find $P(G)$, where $G$ is the event that a number greater than
3 occurs on a single roll of the die.
\\
\solution
		%\begin{table}[H]
	\centering
\begin{tabular}{|c|c|c|}
\hline
Random variable &Value &Definition\\ \hline
\multirow{3}{*}{X} &0 &Slips of Rs 1\\
&1 &Slips of Rs 5\\
&2 &Slips of Rs 13\\ \hline
\multirow{2}{*}{Y} &0 &Box A\\
&1 &Box B\\\hline
\end{tabular}
\caption{}
\label{tab:Distribution}
\end{table}
See \tabref{tab:Distribution}.
\begin{align}
p_{Y}\brak{k}= \begin{cases} 
      \frac{1}{3} & {k=0} \\
      \frac{2}{3 }& {k=1} 
   \end{cases}
   \\
p_{Y|X}\brak{0|0} = \frac{19}{25}\, 
p_{Y|X}\brak{0|1} = \frac{6}{25}\,
p_{Y|X}\brak{1|0} = \frac{45}{50}\,
p_{Y|X}\brak{1|2} = \frac{5}{50}
\end{align}
The desired probability is the probability that a slip drawn at random is marked other than Rs 1,
\begin{align}
&=1-p_X\brak{0}\\
&= p_X(1) + p_X(2)
\end{align}
Using Bayes theorem,
\begin{align}
&= p_Y\brak{0} \times \pr{Y=0 | X=1} + p_Y\brak{1} \times \pr{Y=1|X=2}\\
&=\frac{1}{3} \times \frac{6}{25} + \frac{2}{3} \times \frac{5}{50}\\
&=\frac{11}{75}
\end{align}

\newpage

%\tableofcontents

\bigskip

\renewcommand{\thefigure}{\theenumi}
\renewcommand{\thetable}{\theenumi}
%\renewcommand{\theequation}{\theenumi}

%\begin{abstract}
%%\boldmath
%In this letter, an algorithm for evaluating the exact analytical bit error rate  (BER)  for the piecewise linear (PL) combiner for  multiple relays is presented. Previous results were available only for upto three relays. The algorithm is unique in the sense that  the actual mathematical expressions, that are prohibitively large, need not be explicitly obtained. The diversity gain due to multiple relays is shown through plots of the analytical BER, well supported by simulations. 
%
%\end{abstract}
% IEEEtran.cls defaults to using nonbold math in the Abstract.
% This preserves the distinction between vectors and scalars. However,
% if the journal you are submitting to favors bold math in the abstract,
% then you can use LaTeX's standard command \boldmath at the very start
% of the abstract to achieve this. Many IEEE journals frown on math
% in the abstract anyway.

% Note that keywords are not normally used for peerreview papers.
%\begin{IEEEkeywords}
%Cooperative diversity, decode and forward, piecewise linear
%\end{IEEEkeywords}



% For peer review papers, you can put extra information on the cover
% page as needed:
% \ifCLASSOPTIONpeerreview
% \begin{center} \bfseries EDICS Category: 3-BBND \end{center}
% \fi
%
% For peerreview papers, this IEEEtran command inserts a page break and
% creates the second title. It will be ignored for other modes.
%\IEEEpeerreviewmaketitle




	\item All the jacks, queens and kings are removed from a deck of 52 playing cards. The remaining cards are well shuffled and then one card is drawn at random. Giving ace a value 1 similar value for other cards, find the probability that the card has a value 
		\begin{enumerate}
			\item 7
			\item greater than 7
			\item less than 7
		\end{enumerate}
		%Number of cards left after removing all jacks, queens and kings 
\begin{align}
N	= 52 - 4\times 3
	= 40
\end{align}
%\begin{table}[H]
%\def\arraystretch{1.2}
%\begin{tabular}{|c|c|c|}
%\hline
%	\textbf{Parameter} &\textbf{Value} &\textbf{Description}\\ \hline
%	$X$ &1-10 &Represents the value of the card picked \\ \hline
%\end{tabular}
%\end{table}
Let $1 \le X \le 10$ be the value of the card picked.  Then,
\begin{align}
	p_X(k) &= \Pr(X=k)\ \forall\ 1 \leq k \leq 10\\
	&= \frac{4\times 1}{40}\\
	&= \frac{1}{10}\\
	\therefore p_X(k) &= 
	\begin{cases}
		\frac{1}{10} & 1 \leq k \leq 10\\
		0 & \text{otherwise}
	\end{cases}
\end{align}
and
\begin{align}
	F_{X}(k) &= \sum_{m=0}^{k}p_{X}(m) \quad 1 \leq k \leq 10\\
	&= \frac{k}{10}\\
	\therefore F_{X}(k) &= 
	\begin{cases}
		0 & k \leq 0\\
		\frac{k}{10} & 1\leq k \leq 10\\
		1 & k > 10 
	\end{cases}
\end{align}
\begin{enumerate}
	\item Probability that card has value equal to 7 is
		\begin{align}
			 p_{X}(7)
			= \frac{1}{10}
		\end{align}
	\item Probability that card has value greater than 7 is
		\begin{align}
			1 - F_X(7)
			&= 1 - \frac{7}{10}
			\\
			&= \frac{3}{10}
		\end{align}
	\item Probability that card has value less than 7 is
		\begin{align}
			 F_{X}(6)
			=\frac{6}{10}
		\end{align}
\end{enumerate}

  \item A Lot consists of 48 mobile phones of which 42 are good, 3 have only minor defects and 3 have major defects.Varnika will buy a phone if it is good but the trader will only buy a mobile if it has no major defects. One phone is selected at random from the lot. What is the probability that it is
\begin{enumerate}
	\item acceptable to Varnika?
            \item acceptable to the trader?
\end{enumerate}
\solution
	%\begin{table}[H]
	\centering
\begin{tabular}{|c|c|c|}
\hline
Random variable &Value &Definition\\ \hline
\multirow{3}{*}{X} &0 &Slips of Rs 1\\
&1 &Slips of Rs 5\\
&2 &Slips of Rs 13\\ \hline
\multirow{2}{*}{Y} &0 &Box A\\
&1 &Box B\\\hline
\end{tabular}
\caption{}
\label{tab:Distribution}
\end{table}
See \tabref{tab:Distribution}.
\begin{align}
p_{Y}\brak{k}= \begin{cases} 
      \frac{1}{3} & {k=0} \\
      \frac{2}{3 }& {k=1} 
   \end{cases}
   \\
p_{Y|X}\brak{0|0} = \frac{19}{25}\, 
p_{Y|X}\brak{0|1} = \frac{6}{25}\,
p_{Y|X}\brak{1|0} = \frac{45}{50}\,
p_{Y|X}\brak{1|2} = \frac{5}{50}
\end{align}
The desired probability is the probability that a slip drawn at random is marked other than Rs 1,
\begin{align}
&=1-p_X\brak{0}\\
&= p_X(1) + p_X(2)
\end{align}
Using Bayes theorem,
\begin{align}
&= p_Y\brak{0} \times \pr{Y=0 | X=1} + p_Y\brak{1} \times \pr{Y=1|X=2}\\
&=\frac{1}{3} \times \frac{6}{25} + \frac{2}{3} \times \frac{5}{50}\\
&=\frac{11}{75}
\end{align}

\newpage

%\tableofcontents

\bigskip

\renewcommand{\thefigure}{\theenumi}
\renewcommand{\thetable}{\theenumi}
%\renewcommand{\theequation}{\theenumi}

%\begin{abstract}
%%\boldmath
%In this letter, an algorithm for evaluating the exact analytical bit error rate  (BER)  for the piecewise linear (PL) combiner for  multiple relays is presented. Previous results were available only for upto three relays. The algorithm is unique in the sense that  the actual mathematical expressions, that are prohibitively large, need not be explicitly obtained. The diversity gain due to multiple relays is shown through plots of the analytical BER, well supported by simulations. 
%
%\end{abstract}
% IEEEtran.cls defaults to using nonbold math in the Abstract.
% This preserves the distinction between vectors and scalars. However,
% if the journal you are submitting to favors bold math in the abstract,
% then you can use LaTeX's standard command \boldmath at the very start
% of the abstract to achieve this. Many IEEE journals frown on math
% in the abstract anyway.

% Note that keywords are not normally used for peerreview papers.
%\begin{IEEEkeywords}
%Cooperative diversity, decode and forward, piecewise linear
%\end{IEEEkeywords}



% For peer review papers, you can put extra information on the cover
% page as needed:
% \ifCLASSOPTIONpeerreview
% \begin{center} \bfseries EDICS Category: 3-BBND \end{center}
% \fi
%
% For peerreview papers, this IEEEtran command inserts a page break and
% creates the second title. It will be ignored for other modes.
%\IEEEpeerreviewmaketitle




 \item A student says that if you throw a die, it will show up 1 or not 1. Therefore, the probability of getting 1 and the probability of getting 'not 1' each is equal to $\frac{1}{2}$. Is this correct? Give reasons.\\
 \solution
        %\begin{table}[H]
	\centering
\begin{tabular}{|c|c|c|}
\hline
Random variable &Value &Definition\\ \hline
\multirow{3}{*}{X} &0 &Slips of Rs 1\\
&1 &Slips of Rs 5\\
&2 &Slips of Rs 13\\ \hline
\multirow{2}{*}{Y} &0 &Box A\\
&1 &Box B\\\hline
\end{tabular}
\caption{}
\label{tab:Distribution}
\end{table}
See \tabref{tab:Distribution}.
\begin{align}
p_{Y}\brak{k}= \begin{cases} 
      \frac{1}{3} & {k=0} \\
      \frac{2}{3 }& {k=1} 
   \end{cases}
   \\
p_{Y|X}\brak{0|0} = \frac{19}{25}\, 
p_{Y|X}\brak{0|1} = \frac{6}{25}\,
p_{Y|X}\brak{1|0} = \frac{45}{50}\,
p_{Y|X}\brak{1|2} = \frac{5}{50}
\end{align}
The desired probability is the probability that a slip drawn at random is marked other than Rs 1,
\begin{align}
&=1-p_X\brak{0}\\
&= p_X(1) + p_X(2)
\end{align}
Using Bayes theorem,
\begin{align}
&= p_Y\brak{0} \times \pr{Y=0 | X=1} + p_Y\brak{1} \times \pr{Y=1|X=2}\\
&=\frac{1}{3} \times \frac{6}{25} + \frac{2}{3} \times \frac{5}{50}\\
&=\frac{11}{75}
\end{align}

\newpage

%\tableofcontents

\bigskip

\renewcommand{\thefigure}{\theenumi}
\renewcommand{\thetable}{\theenumi}
%\renewcommand{\theequation}{\theenumi}

%\begin{abstract}
%%\boldmath
%In this letter, an algorithm for evaluating the exact analytical bit error rate  (BER)  for the piecewise linear (PL) combiner for  multiple relays is presented. Previous results were available only for upto three relays. The algorithm is unique in the sense that  the actual mathematical expressions, that are prohibitively large, need not be explicitly obtained. The diversity gain due to multiple relays is shown through plots of the analytical BER, well supported by simulations. 
%
%\end{abstract}
% IEEEtran.cls defaults to using nonbold math in the Abstract.
% This preserves the distinction between vectors and scalars. However,
% if the journal you are submitting to favors bold math in the abstract,
% then you can use LaTeX's standard command \boldmath at the very start
% of the abstract to achieve this. Many IEEE journals frown on math
% in the abstract anyway.

% Note that keywords are not normally used for peerreview papers.
%\begin{IEEEkeywords}
%Cooperative diversity, decode and forward, piecewise linear
%\end{IEEEkeywords}



% For peer review papers, you can put extra information on the cover
% page as needed:
% \ifCLASSOPTIONpeerreview
% \begin{center} \bfseries EDICS Category: 3-BBND \end{center}
% \fi
%
% For peerreview papers, this IEEEtran command inserts a page break and
% creates the second title. It will be ignored for other modes.
%\IEEEpeerreviewmaketitle




   \item Four candidates A, B, C, D have ap-
plied for the assignment to coach a school cricket
team. If A is twice as likely to be selected as B, and
B and C are given about the same chance of being
selected, while C is twice as likely to be selected
as D, what are the probabilities that
\begin{enumerate}
\item C will be selected?
\item A will not be selected?
\end{enumerate}
	%\begin{table}[H]
	\centering
\begin{tabular}{|c|c|c|}
\hline
Random variable &Value &Definition\\ \hline
\multirow{3}{*}{X} &0 &Slips of Rs 1\\
&1 &Slips of Rs 5\\
&2 &Slips of Rs 13\\ \hline
\multirow{2}{*}{Y} &0 &Box A\\
&1 &Box B\\\hline
\end{tabular}
\caption{}
\label{tab:Distribution}
\end{table}
See \tabref{tab:Distribution}.
\begin{align}
p_{Y}\brak{k}= \begin{cases} 
      \frac{1}{3} & {k=0} \\
      \frac{2}{3 }& {k=1} 
   \end{cases}
   \\
p_{Y|X}\brak{0|0} = \frac{19}{25}\, 
p_{Y|X}\brak{0|1} = \frac{6}{25}\,
p_{Y|X}\brak{1|0} = \frac{45}{50}\,
p_{Y|X}\brak{1|2} = \frac{5}{50}
\end{align}
The desired probability is the probability that a slip drawn at random is marked other than Rs 1,
\begin{align}
&=1-p_X\brak{0}\\
&= p_X(1) + p_X(2)
\end{align}
Using Bayes theorem,
\begin{align}
&= p_Y\brak{0} \times \pr{Y=0 | X=1} + p_Y\brak{1} \times \pr{Y=1|X=2}\\
&=\frac{1}{3} \times \frac{6}{25} + \frac{2}{3} \times \frac{5}{50}\\
&=\frac{11}{75}
\end{align}

\newpage

%\tableofcontents

\bigskip

\renewcommand{\thefigure}{\theenumi}
\renewcommand{\thetable}{\theenumi}
%\renewcommand{\theequation}{\theenumi}

%\begin{abstract}
%%\boldmath
%In this letter, an algorithm for evaluating the exact analytical bit error rate  (BER)  for the piecewise linear (PL) combiner for  multiple relays is presented. Previous results were available only for upto three relays. The algorithm is unique in the sense that  the actual mathematical expressions, that are prohibitively large, need not be explicitly obtained. The diversity gain due to multiple relays is shown through plots of the analytical BER, well supported by simulations. 
%
%\end{abstract}
% IEEEtran.cls defaults to using nonbold math in the Abstract.
% This preserves the distinction between vectors and scalars. However,
% if the journal you are submitting to favors bold math in the abstract,
% then you can use LaTeX's standard command \boldmath at the very start
% of the abstract to achieve this. Many IEEE journals frown on math
% in the abstract anyway.

% Note that keywords are not normally used for peerreview papers.
%\begin{IEEEkeywords}
%Cooperative diversity, decode and forward, piecewise linear
%\end{IEEEkeywords}



% For peer review papers, you can put extra information on the cover
% page as needed:
% \ifCLASSOPTIONpeerreview
% \begin{center} \bfseries EDICS Category: 3-BBND \end{center}
% \fi
%
% For peerreview papers, this IEEEtran command inserts a page break and
% creates the second title. It will be ignored for other modes.
%\IEEEpeerreviewmaketitle




 \item A bag contain 24 balls of which $x$ balls are red, $2x$ are white and $3x$ are blue. A ball is selected at random, What is the probability that it is
\begin{enumerate}[label=\alph*)]
\item not red ?
\item white ?
\end{enumerate}
%\begin{table}[H]
	\centering
\begin{tabular}{|c|c|c|}
\hline
Random variable &Value &Definition\\ \hline
\multirow{3}{*}{X} &0 &Slips of Rs 1\\
&1 &Slips of Rs 5\\
&2 &Slips of Rs 13\\ \hline
\multirow{2}{*}{Y} &0 &Box A\\
&1 &Box B\\\hline
\end{tabular}
\caption{}
\label{tab:Distribution}
\end{table}
See \tabref{tab:Distribution}.
\begin{align}
p_{Y}\brak{k}= \begin{cases} 
      \frac{1}{3} & {k=0} \\
      \frac{2}{3 }& {k=1} 
   \end{cases}
   \\
p_{Y|X}\brak{0|0} = \frac{19}{25}\, 
p_{Y|X}\brak{0|1} = \frac{6}{25}\,
p_{Y|X}\brak{1|0} = \frac{45}{50}\,
p_{Y|X}\brak{1|2} = \frac{5}{50}
\end{align}
The desired probability is the probability that a slip drawn at random is marked other than Rs 1,
\begin{align}
&=1-p_X\brak{0}\\
&= p_X(1) + p_X(2)
\end{align}
Using Bayes theorem,
\begin{align}
&= p_Y\brak{0} \times \pr{Y=0 | X=1} + p_Y\brak{1} \times \pr{Y=1|X=2}\\
&=\frac{1}{3} \times \frac{6}{25} + \frac{2}{3} \times \frac{5}{50}\\
&=\frac{11}{75}
\end{align}

\newpage

%\tableofcontents

\bigskip

\renewcommand{\thefigure}{\theenumi}
\renewcommand{\thetable}{\theenumi}
%\renewcommand{\theequation}{\theenumi}

%\begin{abstract}
%%\boldmath
%In this letter, an algorithm for evaluating the exact analytical bit error rate  (BER)  for the piecewise linear (PL) combiner for  multiple relays is presented. Previous results were available only for upto three relays. The algorithm is unique in the sense that  the actual mathematical expressions, that are prohibitively large, need not be explicitly obtained. The diversity gain due to multiple relays is shown through plots of the analytical BER, well supported by simulations. 
%
%\end{abstract}
% IEEEtran.cls defaults to using nonbold math in the Abstract.
% This preserves the distinction between vectors and scalars. However,
% if the journal you are submitting to favors bold math in the abstract,
% then you can use LaTeX's standard command \boldmath at the very start
% of the abstract to achieve this. Many IEEE journals frown on math
% in the abstract anyway.

% Note that keywords are not normally used for peerreview papers.
%\begin{IEEEkeywords}
%Cooperative diversity, decode and forward, piecewise linear
%\end{IEEEkeywords}



% For peer review papers, you can put extra information on the cover
% page as needed:
% \ifCLASSOPTIONpeerreview
% \begin{center} \bfseries EDICS Category: 3-BBND \end{center}
% \fi
%
% For peerreview papers, this IEEEtran command inserts a page break and
% creates the second title. It will be ignored for other modes.
%\IEEEpeerreviewmaketitle




If the letters of the word ASSASSINATION are arranged at random. Find the Probability that
\begin{enumerate}[label=(\alph*)]
\item Four $S's$ come consecutively in the word
\item Two  $I's$ and two $N's$ come together
\item All $A's$ are not coming together
\item No two $A's$ are coming together
\end{enumerate}
%\begin{table}[H]
	\centering
\begin{tabular}{|c|c|c|}
\hline
Random variable &Value &Definition\\ \hline
\multirow{3}{*}{X} &0 &Slips of Rs 1\\
&1 &Slips of Rs 5\\
&2 &Slips of Rs 13\\ \hline
\multirow{2}{*}{Y} &0 &Box A\\
&1 &Box B\\\hline
\end{tabular}
\caption{}
\label{tab:Distribution}
\end{table}
See \tabref{tab:Distribution}.
\begin{align}
p_{Y}\brak{k}= \begin{cases} 
      \frac{1}{3} & {k=0} \\
      \frac{2}{3 }& {k=1} 
   \end{cases}
   \\
p_{Y|X}\brak{0|0} = \frac{19}{25}\, 
p_{Y|X}\brak{0|1} = \frac{6}{25}\,
p_{Y|X}\brak{1|0} = \frac{45}{50}\,
p_{Y|X}\brak{1|2} = \frac{5}{50}
\end{align}
The desired probability is the probability that a slip drawn at random is marked other than Rs 1,
\begin{align}
&=1-p_X\brak{0}\\
&= p_X(1) + p_X(2)
\end{align}
Using Bayes theorem,
\begin{align}
&= p_Y\brak{0} \times \pr{Y=0 | X=1} + p_Y\brak{1} \times \pr{Y=1|X=2}\\
&=\frac{1}{3} \times \frac{6}{25} + \frac{2}{3} \times \frac{5}{50}\\
&=\frac{11}{75}
\end{align}

\newpage

%\tableofcontents

\bigskip

\renewcommand{\thefigure}{\theenumi}
\renewcommand{\thetable}{\theenumi}
%\renewcommand{\theequation}{\theenumi}

%\begin{abstract}
%%\boldmath
%In this letter, an algorithm for evaluating the exact analytical bit error rate  (BER)  for the piecewise linear (PL) combiner for  multiple relays is presented. Previous results were available only for upto three relays. The algorithm is unique in the sense that  the actual mathematical expressions, that are prohibitively large, need not be explicitly obtained. The diversity gain due to multiple relays is shown through plots of the analytical BER, well supported by simulations. 
%
%\end{abstract}
% IEEEtran.cls defaults to using nonbold math in the Abstract.
% This preserves the distinction between vectors and scalars. However,
% if the journal you are submitting to favors bold math in the abstract,
% then you can use LaTeX's standard command \boldmath at the very start
% of the abstract to achieve this. Many IEEE journals frown on math
% in the abstract anyway.

% Note that keywords are not normally used for peerreview papers.
%\begin{IEEEkeywords}
%Cooperative diversity, decode and forward, piecewise linear
%\end{IEEEkeywords}



% For peer review papers, you can put extra information on the cover
% page as needed:
% \ifCLASSOPTIONpeerreview
% \begin{center} \bfseries EDICS Category: 3-BBND \end{center}
% \fi
%
% For peerreview papers, this IEEEtran command inserts a page break and
% creates the second title. It will be ignored for other modes.
%\IEEEpeerreviewmaketitle




	\item One urn contains two black balls (labelled B1 and B2) and one white ball. A
	second urn contains one black ball and two white balls (labelled W1 and W2).
	Suppose the following experiment is performed. One of the two urns is chosen
	at random. Next a ball is randomly chosen from the urn. Then a second ball is
	chosen at random from the same urn without replacing the first ball.
	
	\begin{enumerate}
	\item What is the probability that two black balls are chosen?
	
	\item What is the probability that two balls of opposite colour are chosen?
	\end{enumerate}
	\solution
	%\begin{align}
    \label{eq:12.13.6.18.1}
	\because	\pr{A|B} &> \pr{A},\
\frac{\pr{AB}}{\pr{B}} > \pr{A}
\\
    \label{eq:12.13.6.18.2}
	\implies \pr{AB} &> \pr{A}\pr{B}
	\\
	\text{or, } \frac{\pr{AB}}{\pr{A}} &=\pr{B|A} > \pr{A}
\end{align}

\end{enumerate}

	\item 
The number lock of a suitcase has 4 wheels each labelled with ten digits i.e. from 0 to 9.The lock opens with a sequence of four digits with no repeats.What is the probability of a person getting the right sequence to open the suitcase.
\\
\solution
		%\begin{enumerate}[label=\thesection.\arabic*,ref=\thesection.\theenumi]
	\item One card is drawn from a well-shuffled deck of 52 cards. Find the probability of getting
\begin{enumerate}
\item A king of red colour 
\item A face card 
\item A red face card
\item The jack of hearts
\item A spade
\item The queen of diamonds

\end{enumerate}
\solution
		%\begin{table}[H]
	\centering
\begin{tabular}{|c|c|c|}
\hline
Random variable &Value &Definition\\ \hline
\multirow{3}{*}{X} &0 &Slips of Rs 1\\
&1 &Slips of Rs 5\\
&2 &Slips of Rs 13\\ \hline
\multirow{2}{*}{Y} &0 &Box A\\
&1 &Box B\\\hline
\end{tabular}
\caption{}
\label{tab:Distribution}
\end{table}
See \tabref{tab:Distribution}.
\begin{align}
p_{Y}\brak{k}= \begin{cases} 
      \frac{1}{3} & {k=0} \\
      \frac{2}{3 }& {k=1} 
   \end{cases}
   \\
p_{Y|X}\brak{0|0} = \frac{19}{25}\, 
p_{Y|X}\brak{0|1} = \frac{6}{25}\,
p_{Y|X}\brak{1|0} = \frac{45}{50}\,
p_{Y|X}\brak{1|2} = \frac{5}{50}
\end{align}
The desired probability is the probability that a slip drawn at random is marked other than Rs 1,
\begin{align}
&=1-p_X\brak{0}\\
&= p_X(1) + p_X(2)
\end{align}
Using Bayes theorem,
\begin{align}
&= p_Y\brak{0} \times \pr{Y=0 | X=1} + p_Y\brak{1} \times \pr{Y=1|X=2}\\
&=\frac{1}{3} \times \frac{6}{25} + \frac{2}{3} \times \frac{5}{50}\\
&=\frac{11}{75}
\end{align}

\newpage

%\tableofcontents

\bigskip

\renewcommand{\thefigure}{\theenumi}
\renewcommand{\thetable}{\theenumi}
%\renewcommand{\theequation}{\theenumi}

%\begin{abstract}
%%\boldmath
%In this letter, an algorithm for evaluating the exact analytical bit error rate  (BER)  for the piecewise linear (PL) combiner for  multiple relays is presented. Previous results were available only for upto three relays. The algorithm is unique in the sense that  the actual mathematical expressions, that are prohibitively large, need not be explicitly obtained. The diversity gain due to multiple relays is shown through plots of the analytical BER, well supported by simulations. 
%
%\end{abstract}
% IEEEtran.cls defaults to using nonbold math in the Abstract.
% This preserves the distinction between vectors and scalars. However,
% if the journal you are submitting to favors bold math in the abstract,
% then you can use LaTeX's standard command \boldmath at the very start
% of the abstract to achieve this. Many IEEE journals frown on math
% in the abstract anyway.

% Note that keywords are not normally used for peerreview papers.
%\begin{IEEEkeywords}
%Cooperative diversity, decode and forward, piecewise linear
%\end{IEEEkeywords}



% For peer review papers, you can put extra information on the cover
% page as needed:
% \ifCLASSOPTIONpeerreview
% \begin{center} \bfseries EDICS Category: 3-BBND \end{center}
% \fi
%
% For peerreview papers, this IEEEtran command inserts a page break and
% creates the second title. It will be ignored for other modes.
%\IEEEpeerreviewmaketitle




	\item Five cards—the ten, jack, queen, king and ace of diamonds, are well-shuffled with their face downwards. One card is then picked up at random.
\begin{enumerate}
\item
What is the probability that the card is the queen? 
\item
If the queen is drawn and put aside, what is the probability that the second card picked up is (a) an ace? (b) a queen?\\
\end{enumerate}
\solution
		%\begin{enumerate}[label=\thesection.\arabic*,ref=\thesection.\theenumi]
	\item One card is drawn from a well-shuffled deck of 52 cards. Find the probability of getting
\begin{enumerate}
\item A king of red colour 
\item A face card 
\item A red face card
\item The jack of hearts
\item A spade
\item The queen of diamonds

\end{enumerate}
\solution
		%\begin{table}[H]
	\centering
\begin{tabular}{|c|c|c|}
\hline
Random variable &Value &Definition\\ \hline
\multirow{3}{*}{X} &0 &Slips of Rs 1\\
&1 &Slips of Rs 5\\
&2 &Slips of Rs 13\\ \hline
\multirow{2}{*}{Y} &0 &Box A\\
&1 &Box B\\\hline
\end{tabular}
\caption{}
\label{tab:Distribution}
\end{table}
See \tabref{tab:Distribution}.
\begin{align}
p_{Y}\brak{k}= \begin{cases} 
      \frac{1}{3} & {k=0} \\
      \frac{2}{3 }& {k=1} 
   \end{cases}
   \\
p_{Y|X}\brak{0|0} = \frac{19}{25}\, 
p_{Y|X}\brak{0|1} = \frac{6}{25}\,
p_{Y|X}\brak{1|0} = \frac{45}{50}\,
p_{Y|X}\brak{1|2} = \frac{5}{50}
\end{align}
The desired probability is the probability that a slip drawn at random is marked other than Rs 1,
\begin{align}
&=1-p_X\brak{0}\\
&= p_X(1) + p_X(2)
\end{align}
Using Bayes theorem,
\begin{align}
&= p_Y\brak{0} \times \pr{Y=0 | X=1} + p_Y\brak{1} \times \pr{Y=1|X=2}\\
&=\frac{1}{3} \times \frac{6}{25} + \frac{2}{3} \times \frac{5}{50}\\
&=\frac{11}{75}
\end{align}

\newpage

%\tableofcontents

\bigskip

\renewcommand{\thefigure}{\theenumi}
\renewcommand{\thetable}{\theenumi}
%\renewcommand{\theequation}{\theenumi}

%\begin{abstract}
%%\boldmath
%In this letter, an algorithm for evaluating the exact analytical bit error rate  (BER)  for the piecewise linear (PL) combiner for  multiple relays is presented. Previous results were available only for upto three relays. The algorithm is unique in the sense that  the actual mathematical expressions, that are prohibitively large, need not be explicitly obtained. The diversity gain due to multiple relays is shown through plots of the analytical BER, well supported by simulations. 
%
%\end{abstract}
% IEEEtran.cls defaults to using nonbold math in the Abstract.
% This preserves the distinction between vectors and scalars. However,
% if the journal you are submitting to favors bold math in the abstract,
% then you can use LaTeX's standard command \boldmath at the very start
% of the abstract to achieve this. Many IEEE journals frown on math
% in the abstract anyway.

% Note that keywords are not normally used for peerreview papers.
%\begin{IEEEkeywords}
%Cooperative diversity, decode and forward, piecewise linear
%\end{IEEEkeywords}



% For peer review papers, you can put extra information on the cover
% page as needed:
% \ifCLASSOPTIONpeerreview
% \begin{center} \bfseries EDICS Category: 3-BBND \end{center}
% \fi
%
% For peerreview papers, this IEEEtran command inserts a page break and
% creates the second title. It will be ignored for other modes.
%\IEEEpeerreviewmaketitle




	\item Five cards—the ten, jack, queen, king and ace of diamonds, are well-shuffled with their face downwards. One card is then picked up at random.
\begin{enumerate}
\item
What is the probability that the card is the queen? 
\item
If the queen is drawn and put aside, what is the probability that the second card picked up is (a) an ace? (b) a queen?\\
\end{enumerate}
\solution
		%\begin{enumerate}[label=\thesection.\arabic*,ref=\thesection.\theenumi]
	\item One card is drawn from a well-shuffled deck of 52 cards. Find the probability of getting
\begin{enumerate}
\item A king of red colour 
\item A face card 
\item A red face card
\item The jack of hearts
\item A spade
\item The queen of diamonds

\end{enumerate}
\solution
		%\input{ncert/10/15/1/14/main.tex}
	\item Five cards—the ten, jack, queen, king and ace of diamonds, are well-shuffled with their face downwards. One card is then picked up at random.
\begin{enumerate}
\item
What is the probability that the card is the queen? 
\item
If the queen is drawn and put aside, what is the probability that the second card picked up is (a) an ace? (b) a queen?\\
\end{enumerate}
\solution
		%\input{ncert/10/15/1/15/defs.tex}
	\item A bag contains $5$ red balls and some blue balls. If the probability of drawing a blue ball is double that if a red ball, determine the number of blue balls in the bag. 
		\\
\solution
		%\input{ncert/10/15/2/3/defs.tex}
	\item A card is selected from a pack of 52 cards.
 \begin{enumerate}[label=(\alph*)] 
                 \item How many points are there in the sample space?
                 \item Calculate the probability that the card is an ace of spades.
                 \item Calculate the probability that the card is (i) an ace and (ii) black card.
 \end{enumerate}
\solution
		%\input{ncert/11/16/3/4/main.tex}
\item Four cards are drawn from a well-shuffled deck of 52 cards. What is the probability of obtaining 3 diamonds and one spade.
\\
\solution
		%\input{ncert/11/16/4/2/defs.tex}
\item In a certain lottery 10,000 tickets are sold and ten equal prizes are awarded. What is the probability of not getting a prize if you buy (a) one ticket (b) two tickets (c) 10 tickets ?	
\\
\solution
		%\input{ncert/11/16/4/4/defs.tex}
		%
\item 
Out of 100 students, two sections of 40 and 60 are formed. If you and your friend are among the 100 students, what is the probability that
\begin{enumerate}
\item you both enter the same section?
\item you both enter the different sections?
\end{enumerate}
\solution
		%\input{ncert/11/16/4/5/defs.tex}
	\item 
The number lock of a suitcase has 4 wheels each labelled with ten digits i.e. from 0 to 9.The lock opens with a sequence of four digits with no repeats.What is the probability of a person getting the right sequence to open the suitcase.
\\
\solution
		%\input{ncert/11/16/4/10/defs.tex}
		%
\item 
Two cards are drawn at random and without replacement from a pack of 52 playing cards. Find the probability that both the cards are black.
\\
\solution
		%\input{ncert/12/13/2/2/defs.tex}
		\item A box of oranges is inspected by examining three randomly selected oranges drawn without replacement. If all the three oranges are good, the box is approved for sale, otherwise, it is rejected. Find the probability that a box containing 15 oranges out of which 12 are good and 3 are bad ones will be approved for sale.
		\label{ncert/12/13/2/3/defs.tex}
		\item Two balls are drawn at random with replacement from a box containing 10 black and 8 red balls. Find the probability that
		\label{ncert/12/13/2/12}
\begin{enumerate}
\item both balls are red.
\item first ball is black and second is red.
\item one of them is black and other is red.
\end{enumerate}

\item In a hostel, 60\% of the students read Hindi newspaper, 40\% read English newspaper and 20\% read both Hindi and English newspapers. A student is selected at random.
		\label{ncert/12/13/2/15}
\begin{enumerate}
\item Find the probability that she reads neither Hindi nor English newspapers.
\item If she reads Hindi newspaper, find the probability that she reads English newspaper.
\item If she reads English newspaper, find the probability that she reads Hindi newspaper.\\
\end{enumerate}
\item The probability of obtaining an even prime number on each die, when a pair of dice is rolled is 
\begin{enumerate}
    \item $0$ 
    
    \item $\frac{1}{3}$ 
    
    \item $\frac{1}{12}$ 
    
    \item $\frac{1}{36}$ 
\end{enumerate}
\solution
		%\input{ncert/12/13/2/17/defs.tex}
	\item A bag contains 4 red and 4 black balls, another bag contains 2 red and 6 black balls. One of the two bags is selected at random and a ball is drawn from the bag which is found to be red. Find the probability that the ball is drawn from the first bag.
\\
\solution
		%\input{ncert/12/13/3/2/main.tex}
  \item
  Cards with numbers 2 to 101 are placed in a box. A card is selected at random.Find the probability that the card has
\begin{enumerate}[label=(\roman*)]
	\item an even number 
	\item a square number
\end{enumerate}
\solution
%\input{exemplar/10/13/3/32/main.tex}
\item
The king, queen and jack of clubs are removed from a deck of 52 playing cards and then well shuffled. Now one card is drawn at random from the remaining cards.  Determine the probability that the card is
\begin{enumerate}[label=(\roman*)]
\item a club
\item 10 of hearts
\end{enumerate}
\solution
%\input{exemplar/10/13/3/29/main.tex}
\item A team of medical students doing their internship have to assist during surgeries
at a city hospital. The probabilities of surgeries rated as very complex, complex,
routine, simple or very simple are respectively, 0.15, 0.20, 0.31, 0.26, .08. Find
the probabilities that a particular surgery will be rated
\begin{enumerate}
	\item complex or very complex;
	\item neither very complex nor very simple;
	\item routine or complex
	\item routine or simple
\end{enumerate}
\solution
%\input{exemplar/11/16/3/8(1)/main.tex}
\item A card is selected from a pack of 52 cards.
\begin{enumerate}[label=(\alph*)]
    \item How many points are there in the sample space?
    \item Calculate the probability that the card is an ace of spades.
    \item Calculate the probability that the card is (i) an ace and (ii) black card.
\end{enumerate}
\solution
%\input{exemplar/11/16/3/4/main2.tex}
\item The probability that a non leap year selected at random will contain 53 sundays.
\\
\solution
%\input{exemplar/10/13/1/19/main.tex}
\item One of the four persons John, Rita, Aslam or Gurpreet will be promoted next
month. Consequently the sample space consists of four elementary outcomes
S = {John promoted, Rita promoted, Aslam promoted, Gurpreet promoted}
You are told that the chances of John’s promotion is same as that of Gurpreet,
Rita’s chances of promotion are twice as likely as Johns. Aslam’s chances are
four times that of John.
\begin{enumerate}
	\item Determine
	\begin{enumerate}
		\item P (John promoted)
		\item P (Rita promoted)
		\item P (Aslam promoted)
		\item P (Gurpreet promoted)
	\end{enumerate}
	\item If A = {John promoted or Gurpreet promoted}, find P (A).
\end{enumerate}
\solution
%\input{exemplar/11/16/3/10/main.tex}
\item A card is drawn from a deck of 52 cards. Find the probability of getting a king or a heart or a red card.\\
\solution
%\input{exemplar/11/16/3/15/main.tex}
\item The probability that a student will pass his examination is 0.73, the probability of
the student getting a compartment is 0.13, and the probability that the student will
either pass or get compartment is 0.96. State True or False.\\
\solution
%\input{exemplar/11/16/3/31/main.tex}
\item A card is selected from a pack of 52 cards\\
\begin{enumerate}[label=(\alph*)]
\item How many points are there in the sample space?
\item Calculate the probability that the cards is an ace of spades.
\item Calculate the probability that the card is (i) an ace (ii)black card.\\
\end{enumerate}
%\input{ncert/11/16/3/4_1/Prob_4.tex}
\item In a non-leap year, the probability of having 53 tuesdays or 53 wednesdays is\\
\solution
%\input{exemplar/11/16/3/18/main.tex}
\item There are 1000 sealed envelopes in a box, 10 of them contain a cash prize of
Rs 100 each, 100 of them contain a cash prize of Rs 50 each and 200 of them
contain a cash prize of Rs 10 each and rest do not contain any cash prize. If they
are well shuffled and an envelope is picked up out, what is the probability that it
contains no cash prize?\\
\solution
%\input{exemplar/10/13/3/34/main.tex}
\item 
A die is thrown and a card is selected at random from a deck of 52 playing cards. The probability of getting an even number on the die and a spade card.\\
\solution
%\input{exemplar/12/13/3/78/main.tex}
\item
If 4-digit numbers greater than 5,000 are randomly formed from the digits 0, 1, 3, 5, and 7, what is the probability of forming a number divisible by 5 when:
\begin{enumerate}
    \item The digits are repeated?
    \item The repetition of digits is not allowed?
\end{enumerate}
\solution
%\input{ncert/11/16/4/9/main.tex}
\item Consider the probability space $\brak{\Omega, \mathcal{G}, P}$ where $\Omega = [0,2]$ and $\mathcal{G} = \cbrak{\phi, \Omega, [0,1], (1,2]}$. Let $X$ and $Y$ be two functions on $\Omega$ defined as
\begin{align*}
    X(\omega) = 
    \begin{cases}
        1 & \text{if }\omega \in [0, 1]\\
        2 & \text{if }\omega \in (1, 2]
    \end{cases}
\end{align*}
and
\begin{align*}
    Y(\omega) = 
    \begin{cases}
        2 & \text{if }\omega \in [0, 1.5]\\
        3 & \text{if }\omega \in (1.5, 2].
    \end{cases}
\end{align*}
Then which one of the following statements is true?
\begin{enumerate}
    \item [(A)] $X$ is a random variable with respect to $\mathcal{G}$, but $Y$ is not a random variable with respect to $\mathcal{G}$.
    \item [(B)] $Y$ is a random variable with respect to $\mathcal{G}$, but $X$ is not a random variable with respect to $\mathcal{G}$.
    \item [(C)] Neither $X$ nor $Y$ is a random variable with respect to $\mathcal{G}$.
    \item [(D)] Both $X$ and $Y$ are random variables with respect to $\mathcal{G}$.
\end{enumerate} \hfill (GATE ST 2023)\\
\solution
%\input{gate/ST/2023/14/main.tex}
	\item  A die is loaded in such a way that each odd number is twice as likely to occur as
each even number. Find $P(G)$, where $G$ is the event that a number greater than
3 occurs on a single roll of the die.
\\
\solution
		%\input{exemplar/11/16/3/5/main.tex}
	\item All the jacks, queens and kings are removed from a deck of 52 playing cards. The remaining cards are well shuffled and then one card is drawn at random. Giving ace a value 1 similar value for other cards, find the probability that the card has a value 
		\begin{enumerate}
			\item 7
			\item greater than 7
			\item less than 7
		\end{enumerate}
		%\input{exemplar/10/13/3/30/main.tex}
  \item A Lot consists of 48 mobile phones of which 42 are good, 3 have only minor defects and 3 have major defects.Varnika will buy a phone if it is good but the trader will only buy a mobile if it has no major defects. One phone is selected at random from the lot. What is the probability that it is
\begin{enumerate}
	\item acceptable to Varnika?
            \item acceptable to the trader?
\end{enumerate}
\solution
	%\input{exemplar/10/13/3/40/main.tex}
 \item A student says that if you throw a die, it will show up 1 or not 1. Therefore, the probability of getting 1 and the probability of getting 'not 1' each is equal to $\frac{1}{2}$. Is this correct? Give reasons.\\
 \solution
        %\input{exemplar/10/13/2/9/main.tex}
   \item Four candidates A, B, C, D have ap-
plied for the assignment to coach a school cricket
team. If A is twice as likely to be selected as B, and
B and C are given about the same chance of being
selected, while C is twice as likely to be selected
as D, what are the probabilities that
\begin{enumerate}
\item C will be selected?
\item A will not be selected?
\end{enumerate}
	%\input{exemplar/11/16/3/9/main.tex}
 \item A bag contain 24 balls of which $x$ balls are red, $2x$ are white and $3x$ are blue. A ball is selected at random, What is the probability that it is
\begin{enumerate}[label=\alph*)]
\item not red ?
\item white ?
\end{enumerate}
%\input{exemplar/10/13/3/41/main.tex}
If the letters of the word ASSASSINATION are arranged at random. Find the Probability that
\begin{enumerate}[label=(\alph*)]
\item Four $S's$ come consecutively in the word
\item Two  $I's$ and two $N's$ come together
\item All $A's$ are not coming together
\item No two $A's$ are coming together
\end{enumerate}
%\input{exemplar/11/16/3/14/main.tex}
	\item One urn contains two black balls (labelled B1 and B2) and one white ball. A
	second urn contains one black ball and two white balls (labelled W1 and W2).
	Suppose the following experiment is performed. One of the two urns is chosen
	at random. Next a ball is randomly chosen from the urn. Then a second ball is
	chosen at random from the same urn without replacing the first ball.
	
	\begin{enumerate}
	\item What is the probability that two black balls are chosen?
	
	\item What is the probability that two balls of opposite colour are chosen?
	\end{enumerate}
	\solution
	%\input{exemplar/11/16/3/12/main1.tex}
\end{enumerate}

	\item A bag contains $5$ red balls and some blue balls. If the probability of drawing a blue ball is double that if a red ball, determine the number of blue balls in the bag. 
		\\
\solution
		%\begin{enumerate}[label=\thesection.\arabic*,ref=\thesection.\theenumi]
	\item One card is drawn from a well-shuffled deck of 52 cards. Find the probability of getting
\begin{enumerate}
\item A king of red colour 
\item A face card 
\item A red face card
\item The jack of hearts
\item A spade
\item The queen of diamonds

\end{enumerate}
\solution
		%\input{ncert/10/15/1/14/main.tex}
	\item Five cards—the ten, jack, queen, king and ace of diamonds, are well-shuffled with their face downwards. One card is then picked up at random.
\begin{enumerate}
\item
What is the probability that the card is the queen? 
\item
If the queen is drawn and put aside, what is the probability that the second card picked up is (a) an ace? (b) a queen?\\
\end{enumerate}
\solution
		%\input{ncert/10/15/1/15/defs.tex}
	\item A bag contains $5$ red balls and some blue balls. If the probability of drawing a blue ball is double that if a red ball, determine the number of blue balls in the bag. 
		\\
\solution
		%\input{ncert/10/15/2/3/defs.tex}
	\item A card is selected from a pack of 52 cards.
 \begin{enumerate}[label=(\alph*)] 
                 \item How many points are there in the sample space?
                 \item Calculate the probability that the card is an ace of spades.
                 \item Calculate the probability that the card is (i) an ace and (ii) black card.
 \end{enumerate}
\solution
		%\input{ncert/11/16/3/4/main.tex}
\item Four cards are drawn from a well-shuffled deck of 52 cards. What is the probability of obtaining 3 diamonds and one spade.
\\
\solution
		%\input{ncert/11/16/4/2/defs.tex}
\item In a certain lottery 10,000 tickets are sold and ten equal prizes are awarded. What is the probability of not getting a prize if you buy (a) one ticket (b) two tickets (c) 10 tickets ?	
\\
\solution
		%\input{ncert/11/16/4/4/defs.tex}
		%
\item 
Out of 100 students, two sections of 40 and 60 are formed. If you and your friend are among the 100 students, what is the probability that
\begin{enumerate}
\item you both enter the same section?
\item you both enter the different sections?
\end{enumerate}
\solution
		%\input{ncert/11/16/4/5/defs.tex}
	\item 
The number lock of a suitcase has 4 wheels each labelled with ten digits i.e. from 0 to 9.The lock opens with a sequence of four digits with no repeats.What is the probability of a person getting the right sequence to open the suitcase.
\\
\solution
		%\input{ncert/11/16/4/10/defs.tex}
		%
\item 
Two cards are drawn at random and without replacement from a pack of 52 playing cards. Find the probability that both the cards are black.
\\
\solution
		%\input{ncert/12/13/2/2/defs.tex}
		\item A box of oranges is inspected by examining three randomly selected oranges drawn without replacement. If all the three oranges are good, the box is approved for sale, otherwise, it is rejected. Find the probability that a box containing 15 oranges out of which 12 are good and 3 are bad ones will be approved for sale.
		\label{ncert/12/13/2/3/defs.tex}
		\item Two balls are drawn at random with replacement from a box containing 10 black and 8 red balls. Find the probability that
		\label{ncert/12/13/2/12}
\begin{enumerate}
\item both balls are red.
\item first ball is black and second is red.
\item one of them is black and other is red.
\end{enumerate}

\item In a hostel, 60\% of the students read Hindi newspaper, 40\% read English newspaper and 20\% read both Hindi and English newspapers. A student is selected at random.
		\label{ncert/12/13/2/15}
\begin{enumerate}
\item Find the probability that she reads neither Hindi nor English newspapers.
\item If she reads Hindi newspaper, find the probability that she reads English newspaper.
\item If she reads English newspaper, find the probability that she reads Hindi newspaper.\\
\end{enumerate}
\item The probability of obtaining an even prime number on each die, when a pair of dice is rolled is 
\begin{enumerate}
    \item $0$ 
    
    \item $\frac{1}{3}$ 
    
    \item $\frac{1}{12}$ 
    
    \item $\frac{1}{36}$ 
\end{enumerate}
\solution
		%\input{ncert/12/13/2/17/defs.tex}
	\item A bag contains 4 red and 4 black balls, another bag contains 2 red and 6 black balls. One of the two bags is selected at random and a ball is drawn from the bag which is found to be red. Find the probability that the ball is drawn from the first bag.
\\
\solution
		%\input{ncert/12/13/3/2/main.tex}
  \item
  Cards with numbers 2 to 101 are placed in a box. A card is selected at random.Find the probability that the card has
\begin{enumerate}[label=(\roman*)]
	\item an even number 
	\item a square number
\end{enumerate}
\solution
%\input{exemplar/10/13/3/32/main.tex}
\item
The king, queen and jack of clubs are removed from a deck of 52 playing cards and then well shuffled. Now one card is drawn at random from the remaining cards.  Determine the probability that the card is
\begin{enumerate}[label=(\roman*)]
\item a club
\item 10 of hearts
\end{enumerate}
\solution
%\input{exemplar/10/13/3/29/main.tex}
\item A team of medical students doing their internship have to assist during surgeries
at a city hospital. The probabilities of surgeries rated as very complex, complex,
routine, simple or very simple are respectively, 0.15, 0.20, 0.31, 0.26, .08. Find
the probabilities that a particular surgery will be rated
\begin{enumerate}
	\item complex or very complex;
	\item neither very complex nor very simple;
	\item routine or complex
	\item routine or simple
\end{enumerate}
\solution
%\input{exemplar/11/16/3/8(1)/main.tex}
\item A card is selected from a pack of 52 cards.
\begin{enumerate}[label=(\alph*)]
    \item How many points are there in the sample space?
    \item Calculate the probability that the card is an ace of spades.
    \item Calculate the probability that the card is (i) an ace and (ii) black card.
\end{enumerate}
\solution
%\input{exemplar/11/16/3/4/main2.tex}
\item The probability that a non leap year selected at random will contain 53 sundays.
\\
\solution
%\input{exemplar/10/13/1/19/main.tex}
\item One of the four persons John, Rita, Aslam or Gurpreet will be promoted next
month. Consequently the sample space consists of four elementary outcomes
S = {John promoted, Rita promoted, Aslam promoted, Gurpreet promoted}
You are told that the chances of John’s promotion is same as that of Gurpreet,
Rita’s chances of promotion are twice as likely as Johns. Aslam’s chances are
four times that of John.
\begin{enumerate}
	\item Determine
	\begin{enumerate}
		\item P (John promoted)
		\item P (Rita promoted)
		\item P (Aslam promoted)
		\item P (Gurpreet promoted)
	\end{enumerate}
	\item If A = {John promoted or Gurpreet promoted}, find P (A).
\end{enumerate}
\solution
%\input{exemplar/11/16/3/10/main.tex}
\item A card is drawn from a deck of 52 cards. Find the probability of getting a king or a heart or a red card.\\
\solution
%\input{exemplar/11/16/3/15/main.tex}
\item The probability that a student will pass his examination is 0.73, the probability of
the student getting a compartment is 0.13, and the probability that the student will
either pass or get compartment is 0.96. State True or False.\\
\solution
%\input{exemplar/11/16/3/31/main.tex}
\item A card is selected from a pack of 52 cards\\
\begin{enumerate}[label=(\alph*)]
\item How many points are there in the sample space?
\item Calculate the probability that the cards is an ace of spades.
\item Calculate the probability that the card is (i) an ace (ii)black card.\\
\end{enumerate}
%\input{ncert/11/16/3/4_1/Prob_4.tex}
\item In a non-leap year, the probability of having 53 tuesdays or 53 wednesdays is\\
\solution
%\input{exemplar/11/16/3/18/main.tex}
\item There are 1000 sealed envelopes in a box, 10 of them contain a cash prize of
Rs 100 each, 100 of them contain a cash prize of Rs 50 each and 200 of them
contain a cash prize of Rs 10 each and rest do not contain any cash prize. If they
are well shuffled and an envelope is picked up out, what is the probability that it
contains no cash prize?\\
\solution
%\input{exemplar/10/13/3/34/main.tex}
\item 
A die is thrown and a card is selected at random from a deck of 52 playing cards. The probability of getting an even number on the die and a spade card.\\
\solution
%\input{exemplar/12/13/3/78/main.tex}
\item
If 4-digit numbers greater than 5,000 are randomly formed from the digits 0, 1, 3, 5, and 7, what is the probability of forming a number divisible by 5 when:
\begin{enumerate}
    \item The digits are repeated?
    \item The repetition of digits is not allowed?
\end{enumerate}
\solution
%\input{ncert/11/16/4/9/main.tex}
\item Consider the probability space $\brak{\Omega, \mathcal{G}, P}$ where $\Omega = [0,2]$ and $\mathcal{G} = \cbrak{\phi, \Omega, [0,1], (1,2]}$. Let $X$ and $Y$ be two functions on $\Omega$ defined as
\begin{align*}
    X(\omega) = 
    \begin{cases}
        1 & \text{if }\omega \in [0, 1]\\
        2 & \text{if }\omega \in (1, 2]
    \end{cases}
\end{align*}
and
\begin{align*}
    Y(\omega) = 
    \begin{cases}
        2 & \text{if }\omega \in [0, 1.5]\\
        3 & \text{if }\omega \in (1.5, 2].
    \end{cases}
\end{align*}
Then which one of the following statements is true?
\begin{enumerate}
    \item [(A)] $X$ is a random variable with respect to $\mathcal{G}$, but $Y$ is not a random variable with respect to $\mathcal{G}$.
    \item [(B)] $Y$ is a random variable with respect to $\mathcal{G}$, but $X$ is not a random variable with respect to $\mathcal{G}$.
    \item [(C)] Neither $X$ nor $Y$ is a random variable with respect to $\mathcal{G}$.
    \item [(D)] Both $X$ and $Y$ are random variables with respect to $\mathcal{G}$.
\end{enumerate} \hfill (GATE ST 2023)\\
\solution
%\input{gate/ST/2023/14/main.tex}
	\item  A die is loaded in such a way that each odd number is twice as likely to occur as
each even number. Find $P(G)$, where $G$ is the event that a number greater than
3 occurs on a single roll of the die.
\\
\solution
		%\input{exemplar/11/16/3/5/main.tex}
	\item All the jacks, queens and kings are removed from a deck of 52 playing cards. The remaining cards are well shuffled and then one card is drawn at random. Giving ace a value 1 similar value for other cards, find the probability that the card has a value 
		\begin{enumerate}
			\item 7
			\item greater than 7
			\item less than 7
		\end{enumerate}
		%\input{exemplar/10/13/3/30/main.tex}
  \item A Lot consists of 48 mobile phones of which 42 are good, 3 have only minor defects and 3 have major defects.Varnika will buy a phone if it is good but the trader will only buy a mobile if it has no major defects. One phone is selected at random from the lot. What is the probability that it is
\begin{enumerate}
	\item acceptable to Varnika?
            \item acceptable to the trader?
\end{enumerate}
\solution
	%\input{exemplar/10/13/3/40/main.tex}
 \item A student says that if you throw a die, it will show up 1 or not 1. Therefore, the probability of getting 1 and the probability of getting 'not 1' each is equal to $\frac{1}{2}$. Is this correct? Give reasons.\\
 \solution
        %\input{exemplar/10/13/2/9/main.tex}
   \item Four candidates A, B, C, D have ap-
plied for the assignment to coach a school cricket
team. If A is twice as likely to be selected as B, and
B and C are given about the same chance of being
selected, while C is twice as likely to be selected
as D, what are the probabilities that
\begin{enumerate}
\item C will be selected?
\item A will not be selected?
\end{enumerate}
	%\input{exemplar/11/16/3/9/main.tex}
 \item A bag contain 24 balls of which $x$ balls are red, $2x$ are white and $3x$ are blue. A ball is selected at random, What is the probability that it is
\begin{enumerate}[label=\alph*)]
\item not red ?
\item white ?
\end{enumerate}
%\input{exemplar/10/13/3/41/main.tex}
If the letters of the word ASSASSINATION are arranged at random. Find the Probability that
\begin{enumerate}[label=(\alph*)]
\item Four $S's$ come consecutively in the word
\item Two  $I's$ and two $N's$ come together
\item All $A's$ are not coming together
\item No two $A's$ are coming together
\end{enumerate}
%\input{exemplar/11/16/3/14/main.tex}
	\item One urn contains two black balls (labelled B1 and B2) and one white ball. A
	second urn contains one black ball and two white balls (labelled W1 and W2).
	Suppose the following experiment is performed. One of the two urns is chosen
	at random. Next a ball is randomly chosen from the urn. Then a second ball is
	chosen at random from the same urn without replacing the first ball.
	
	\begin{enumerate}
	\item What is the probability that two black balls are chosen?
	
	\item What is the probability that two balls of opposite colour are chosen?
	\end{enumerate}
	\solution
	%\input{exemplar/11/16/3/12/main1.tex}
\end{enumerate}

	\item A card is selected from a pack of 52 cards.
 \begin{enumerate}[label=(\alph*)] 
                 \item How many points are there in the sample space?
                 \item Calculate the probability that the card is an ace of spades.
                 \item Calculate the probability that the card is (i) an ace and (ii) black card.
 \end{enumerate}
\solution
		%\begin{table}[H]
	\centering
\begin{tabular}{|c|c|c|}
\hline
Random variable &Value &Definition\\ \hline
\multirow{3}{*}{X} &0 &Slips of Rs 1\\
&1 &Slips of Rs 5\\
&2 &Slips of Rs 13\\ \hline
\multirow{2}{*}{Y} &0 &Box A\\
&1 &Box B\\\hline
\end{tabular}
\caption{}
\label{tab:Distribution}
\end{table}
See \tabref{tab:Distribution}.
\begin{align}
p_{Y}\brak{k}= \begin{cases} 
      \frac{1}{3} & {k=0} \\
      \frac{2}{3 }& {k=1} 
   \end{cases}
   \\
p_{Y|X}\brak{0|0} = \frac{19}{25}\, 
p_{Y|X}\brak{0|1} = \frac{6}{25}\,
p_{Y|X}\brak{1|0} = \frac{45}{50}\,
p_{Y|X}\brak{1|2} = \frac{5}{50}
\end{align}
The desired probability is the probability that a slip drawn at random is marked other than Rs 1,
\begin{align}
&=1-p_X\brak{0}\\
&= p_X(1) + p_X(2)
\end{align}
Using Bayes theorem,
\begin{align}
&= p_Y\brak{0} \times \pr{Y=0 | X=1} + p_Y\brak{1} \times \pr{Y=1|X=2}\\
&=\frac{1}{3} \times \frac{6}{25} + \frac{2}{3} \times \frac{5}{50}\\
&=\frac{11}{75}
\end{align}

\newpage

%\tableofcontents

\bigskip

\renewcommand{\thefigure}{\theenumi}
\renewcommand{\thetable}{\theenumi}
%\renewcommand{\theequation}{\theenumi}

%\begin{abstract}
%%\boldmath
%In this letter, an algorithm for evaluating the exact analytical bit error rate  (BER)  for the piecewise linear (PL) combiner for  multiple relays is presented. Previous results were available only for upto three relays. The algorithm is unique in the sense that  the actual mathematical expressions, that are prohibitively large, need not be explicitly obtained. The diversity gain due to multiple relays is shown through plots of the analytical BER, well supported by simulations. 
%
%\end{abstract}
% IEEEtran.cls defaults to using nonbold math in the Abstract.
% This preserves the distinction between vectors and scalars. However,
% if the journal you are submitting to favors bold math in the abstract,
% then you can use LaTeX's standard command \boldmath at the very start
% of the abstract to achieve this. Many IEEE journals frown on math
% in the abstract anyway.

% Note that keywords are not normally used for peerreview papers.
%\begin{IEEEkeywords}
%Cooperative diversity, decode and forward, piecewise linear
%\end{IEEEkeywords}



% For peer review papers, you can put extra information on the cover
% page as needed:
% \ifCLASSOPTIONpeerreview
% \begin{center} \bfseries EDICS Category: 3-BBND \end{center}
% \fi
%
% For peerreview papers, this IEEEtran command inserts a page break and
% creates the second title. It will be ignored for other modes.
%\IEEEpeerreviewmaketitle




\item Four cards are drawn from a well-shuffled deck of 52 cards. What is the probability of obtaining 3 diamonds and one spade.
\\
\solution
		%\begin{enumerate}[label=\thesection.\arabic*,ref=\thesection.\theenumi]
	\item One card is drawn from a well-shuffled deck of 52 cards. Find the probability of getting
\begin{enumerate}
\item A king of red colour 
\item A face card 
\item A red face card
\item The jack of hearts
\item A spade
\item The queen of diamonds

\end{enumerate}
\solution
		%\input{ncert/10/15/1/14/main.tex}
	\item Five cards—the ten, jack, queen, king and ace of diamonds, are well-shuffled with their face downwards. One card is then picked up at random.
\begin{enumerate}
\item
What is the probability that the card is the queen? 
\item
If the queen is drawn and put aside, what is the probability that the second card picked up is (a) an ace? (b) a queen?\\
\end{enumerate}
\solution
		%\input{ncert/10/15/1/15/defs.tex}
	\item A bag contains $5$ red balls and some blue balls. If the probability of drawing a blue ball is double that if a red ball, determine the number of blue balls in the bag. 
		\\
\solution
		%\input{ncert/10/15/2/3/defs.tex}
	\item A card is selected from a pack of 52 cards.
 \begin{enumerate}[label=(\alph*)] 
                 \item How many points are there in the sample space?
                 \item Calculate the probability that the card is an ace of spades.
                 \item Calculate the probability that the card is (i) an ace and (ii) black card.
 \end{enumerate}
\solution
		%\input{ncert/11/16/3/4/main.tex}
\item Four cards are drawn from a well-shuffled deck of 52 cards. What is the probability of obtaining 3 diamonds and one spade.
\\
\solution
		%\input{ncert/11/16/4/2/defs.tex}
\item In a certain lottery 10,000 tickets are sold and ten equal prizes are awarded. What is the probability of not getting a prize if you buy (a) one ticket (b) two tickets (c) 10 tickets ?	
\\
\solution
		%\input{ncert/11/16/4/4/defs.tex}
		%
\item 
Out of 100 students, two sections of 40 and 60 are formed. If you and your friend are among the 100 students, what is the probability that
\begin{enumerate}
\item you both enter the same section?
\item you both enter the different sections?
\end{enumerate}
\solution
		%\input{ncert/11/16/4/5/defs.tex}
	\item 
The number lock of a suitcase has 4 wheels each labelled with ten digits i.e. from 0 to 9.The lock opens with a sequence of four digits with no repeats.What is the probability of a person getting the right sequence to open the suitcase.
\\
\solution
		%\input{ncert/11/16/4/10/defs.tex}
		%
\item 
Two cards are drawn at random and without replacement from a pack of 52 playing cards. Find the probability that both the cards are black.
\\
\solution
		%\input{ncert/12/13/2/2/defs.tex}
		\item A box of oranges is inspected by examining three randomly selected oranges drawn without replacement. If all the three oranges are good, the box is approved for sale, otherwise, it is rejected. Find the probability that a box containing 15 oranges out of which 12 are good and 3 are bad ones will be approved for sale.
		\label{ncert/12/13/2/3/defs.tex}
		\item Two balls are drawn at random with replacement from a box containing 10 black and 8 red balls. Find the probability that
		\label{ncert/12/13/2/12}
\begin{enumerate}
\item both balls are red.
\item first ball is black and second is red.
\item one of them is black and other is red.
\end{enumerate}

\item In a hostel, 60\% of the students read Hindi newspaper, 40\% read English newspaper and 20\% read both Hindi and English newspapers. A student is selected at random.
		\label{ncert/12/13/2/15}
\begin{enumerate}
\item Find the probability that she reads neither Hindi nor English newspapers.
\item If she reads Hindi newspaper, find the probability that she reads English newspaper.
\item If she reads English newspaper, find the probability that she reads Hindi newspaper.\\
\end{enumerate}
\item The probability of obtaining an even prime number on each die, when a pair of dice is rolled is 
\begin{enumerate}
    \item $0$ 
    
    \item $\frac{1}{3}$ 
    
    \item $\frac{1}{12}$ 
    
    \item $\frac{1}{36}$ 
\end{enumerate}
\solution
		%\input{ncert/12/13/2/17/defs.tex}
	\item A bag contains 4 red and 4 black balls, another bag contains 2 red and 6 black balls. One of the two bags is selected at random and a ball is drawn from the bag which is found to be red. Find the probability that the ball is drawn from the first bag.
\\
\solution
		%\input{ncert/12/13/3/2/main.tex}
  \item
  Cards with numbers 2 to 101 are placed in a box. A card is selected at random.Find the probability that the card has
\begin{enumerate}[label=(\roman*)]
	\item an even number 
	\item a square number
\end{enumerate}
\solution
%\input{exemplar/10/13/3/32/main.tex}
\item
The king, queen and jack of clubs are removed from a deck of 52 playing cards and then well shuffled. Now one card is drawn at random from the remaining cards.  Determine the probability that the card is
\begin{enumerate}[label=(\roman*)]
\item a club
\item 10 of hearts
\end{enumerate}
\solution
%\input{exemplar/10/13/3/29/main.tex}
\item A team of medical students doing their internship have to assist during surgeries
at a city hospital. The probabilities of surgeries rated as very complex, complex,
routine, simple or very simple are respectively, 0.15, 0.20, 0.31, 0.26, .08. Find
the probabilities that a particular surgery will be rated
\begin{enumerate}
	\item complex or very complex;
	\item neither very complex nor very simple;
	\item routine or complex
	\item routine or simple
\end{enumerate}
\solution
%\input{exemplar/11/16/3/8(1)/main.tex}
\item A card is selected from a pack of 52 cards.
\begin{enumerate}[label=(\alph*)]
    \item How many points are there in the sample space?
    \item Calculate the probability that the card is an ace of spades.
    \item Calculate the probability that the card is (i) an ace and (ii) black card.
\end{enumerate}
\solution
%\input{exemplar/11/16/3/4/main2.tex}
\item The probability that a non leap year selected at random will contain 53 sundays.
\\
\solution
%\input{exemplar/10/13/1/19/main.tex}
\item One of the four persons John, Rita, Aslam or Gurpreet will be promoted next
month. Consequently the sample space consists of four elementary outcomes
S = {John promoted, Rita promoted, Aslam promoted, Gurpreet promoted}
You are told that the chances of John’s promotion is same as that of Gurpreet,
Rita’s chances of promotion are twice as likely as Johns. Aslam’s chances are
four times that of John.
\begin{enumerate}
	\item Determine
	\begin{enumerate}
		\item P (John promoted)
		\item P (Rita promoted)
		\item P (Aslam promoted)
		\item P (Gurpreet promoted)
	\end{enumerate}
	\item If A = {John promoted or Gurpreet promoted}, find P (A).
\end{enumerate}
\solution
%\input{exemplar/11/16/3/10/main.tex}
\item A card is drawn from a deck of 52 cards. Find the probability of getting a king or a heart or a red card.\\
\solution
%\input{exemplar/11/16/3/15/main.tex}
\item The probability that a student will pass his examination is 0.73, the probability of
the student getting a compartment is 0.13, and the probability that the student will
either pass or get compartment is 0.96. State True or False.\\
\solution
%\input{exemplar/11/16/3/31/main.tex}
\item A card is selected from a pack of 52 cards\\
\begin{enumerate}[label=(\alph*)]
\item How many points are there in the sample space?
\item Calculate the probability that the cards is an ace of spades.
\item Calculate the probability that the card is (i) an ace (ii)black card.\\
\end{enumerate}
%\input{ncert/11/16/3/4_1/Prob_4.tex}
\item In a non-leap year, the probability of having 53 tuesdays or 53 wednesdays is\\
\solution
%\input{exemplar/11/16/3/18/main.tex}
\item There are 1000 sealed envelopes in a box, 10 of them contain a cash prize of
Rs 100 each, 100 of them contain a cash prize of Rs 50 each and 200 of them
contain a cash prize of Rs 10 each and rest do not contain any cash prize. If they
are well shuffled and an envelope is picked up out, what is the probability that it
contains no cash prize?\\
\solution
%\input{exemplar/10/13/3/34/main.tex}
\item 
A die is thrown and a card is selected at random from a deck of 52 playing cards. The probability of getting an even number on the die and a spade card.\\
\solution
%\input{exemplar/12/13/3/78/main.tex}
\item
If 4-digit numbers greater than 5,000 are randomly formed from the digits 0, 1, 3, 5, and 7, what is the probability of forming a number divisible by 5 when:
\begin{enumerate}
    \item The digits are repeated?
    \item The repetition of digits is not allowed?
\end{enumerate}
\solution
%\input{ncert/11/16/4/9/main.tex}
\item Consider the probability space $\brak{\Omega, \mathcal{G}, P}$ where $\Omega = [0,2]$ and $\mathcal{G} = \cbrak{\phi, \Omega, [0,1], (1,2]}$. Let $X$ and $Y$ be two functions on $\Omega$ defined as
\begin{align*}
    X(\omega) = 
    \begin{cases}
        1 & \text{if }\omega \in [0, 1]\\
        2 & \text{if }\omega \in (1, 2]
    \end{cases}
\end{align*}
and
\begin{align*}
    Y(\omega) = 
    \begin{cases}
        2 & \text{if }\omega \in [0, 1.5]\\
        3 & \text{if }\omega \in (1.5, 2].
    \end{cases}
\end{align*}
Then which one of the following statements is true?
\begin{enumerate}
    \item [(A)] $X$ is a random variable with respect to $\mathcal{G}$, but $Y$ is not a random variable with respect to $\mathcal{G}$.
    \item [(B)] $Y$ is a random variable with respect to $\mathcal{G}$, but $X$ is not a random variable with respect to $\mathcal{G}$.
    \item [(C)] Neither $X$ nor $Y$ is a random variable with respect to $\mathcal{G}$.
    \item [(D)] Both $X$ and $Y$ are random variables with respect to $\mathcal{G}$.
\end{enumerate} \hfill (GATE ST 2023)\\
\solution
%\input{gate/ST/2023/14/main.tex}
	\item  A die is loaded in such a way that each odd number is twice as likely to occur as
each even number. Find $P(G)$, where $G$ is the event that a number greater than
3 occurs on a single roll of the die.
\\
\solution
		%\input{exemplar/11/16/3/5/main.tex}
	\item All the jacks, queens and kings are removed from a deck of 52 playing cards. The remaining cards are well shuffled and then one card is drawn at random. Giving ace a value 1 similar value for other cards, find the probability that the card has a value 
		\begin{enumerate}
			\item 7
			\item greater than 7
			\item less than 7
		\end{enumerate}
		%\input{exemplar/10/13/3/30/main.tex}
  \item A Lot consists of 48 mobile phones of which 42 are good, 3 have only minor defects and 3 have major defects.Varnika will buy a phone if it is good but the trader will only buy a mobile if it has no major defects. One phone is selected at random from the lot. What is the probability that it is
\begin{enumerate}
	\item acceptable to Varnika?
            \item acceptable to the trader?
\end{enumerate}
\solution
	%\input{exemplar/10/13/3/40/main.tex}
 \item A student says that if you throw a die, it will show up 1 or not 1. Therefore, the probability of getting 1 and the probability of getting 'not 1' each is equal to $\frac{1}{2}$. Is this correct? Give reasons.\\
 \solution
        %\input{exemplar/10/13/2/9/main.tex}
   \item Four candidates A, B, C, D have ap-
plied for the assignment to coach a school cricket
team. If A is twice as likely to be selected as B, and
B and C are given about the same chance of being
selected, while C is twice as likely to be selected
as D, what are the probabilities that
\begin{enumerate}
\item C will be selected?
\item A will not be selected?
\end{enumerate}
	%\input{exemplar/11/16/3/9/main.tex}
 \item A bag contain 24 balls of which $x$ balls are red, $2x$ are white and $3x$ are blue. A ball is selected at random, What is the probability that it is
\begin{enumerate}[label=\alph*)]
\item not red ?
\item white ?
\end{enumerate}
%\input{exemplar/10/13/3/41/main.tex}
If the letters of the word ASSASSINATION are arranged at random. Find the Probability that
\begin{enumerate}[label=(\alph*)]
\item Four $S's$ come consecutively in the word
\item Two  $I's$ and two $N's$ come together
\item All $A's$ are not coming together
\item No two $A's$ are coming together
\end{enumerate}
%\input{exemplar/11/16/3/14/main.tex}
	\item One urn contains two black balls (labelled B1 and B2) and one white ball. A
	second urn contains one black ball and two white balls (labelled W1 and W2).
	Suppose the following experiment is performed. One of the two urns is chosen
	at random. Next a ball is randomly chosen from the urn. Then a second ball is
	chosen at random from the same urn without replacing the first ball.
	
	\begin{enumerate}
	\item What is the probability that two black balls are chosen?
	
	\item What is the probability that two balls of opposite colour are chosen?
	\end{enumerate}
	\solution
	%\input{exemplar/11/16/3/12/main1.tex}
\end{enumerate}

\item In a certain lottery 10,000 tickets are sold and ten equal prizes are awarded. What is the probability of not getting a prize if you buy (a) one ticket (b) two tickets (c) 10 tickets ?	
\\
\solution
		%\begin{enumerate}[label=\thesection.\arabic*,ref=\thesection.\theenumi]
	\item One card is drawn from a well-shuffled deck of 52 cards. Find the probability of getting
\begin{enumerate}
\item A king of red colour 
\item A face card 
\item A red face card
\item The jack of hearts
\item A spade
\item The queen of diamonds

\end{enumerate}
\solution
		%\input{ncert/10/15/1/14/main.tex}
	\item Five cards—the ten, jack, queen, king and ace of diamonds, are well-shuffled with their face downwards. One card is then picked up at random.
\begin{enumerate}
\item
What is the probability that the card is the queen? 
\item
If the queen is drawn and put aside, what is the probability that the second card picked up is (a) an ace? (b) a queen?\\
\end{enumerate}
\solution
		%\input{ncert/10/15/1/15/defs.tex}
	\item A bag contains $5$ red balls and some blue balls. If the probability of drawing a blue ball is double that if a red ball, determine the number of blue balls in the bag. 
		\\
\solution
		%\input{ncert/10/15/2/3/defs.tex}
	\item A card is selected from a pack of 52 cards.
 \begin{enumerate}[label=(\alph*)] 
                 \item How many points are there in the sample space?
                 \item Calculate the probability that the card is an ace of spades.
                 \item Calculate the probability that the card is (i) an ace and (ii) black card.
 \end{enumerate}
\solution
		%\input{ncert/11/16/3/4/main.tex}
\item Four cards are drawn from a well-shuffled deck of 52 cards. What is the probability of obtaining 3 diamonds and one spade.
\\
\solution
		%\input{ncert/11/16/4/2/defs.tex}
\item In a certain lottery 10,000 tickets are sold and ten equal prizes are awarded. What is the probability of not getting a prize if you buy (a) one ticket (b) two tickets (c) 10 tickets ?	
\\
\solution
		%\input{ncert/11/16/4/4/defs.tex}
		%
\item 
Out of 100 students, two sections of 40 and 60 are formed. If you and your friend are among the 100 students, what is the probability that
\begin{enumerate}
\item you both enter the same section?
\item you both enter the different sections?
\end{enumerate}
\solution
		%\input{ncert/11/16/4/5/defs.tex}
	\item 
The number lock of a suitcase has 4 wheels each labelled with ten digits i.e. from 0 to 9.The lock opens with a sequence of four digits with no repeats.What is the probability of a person getting the right sequence to open the suitcase.
\\
\solution
		%\input{ncert/11/16/4/10/defs.tex}
		%
\item 
Two cards are drawn at random and without replacement from a pack of 52 playing cards. Find the probability that both the cards are black.
\\
\solution
		%\input{ncert/12/13/2/2/defs.tex}
		\item A box of oranges is inspected by examining three randomly selected oranges drawn without replacement. If all the three oranges are good, the box is approved for sale, otherwise, it is rejected. Find the probability that a box containing 15 oranges out of which 12 are good and 3 are bad ones will be approved for sale.
		\label{ncert/12/13/2/3/defs.tex}
		\item Two balls are drawn at random with replacement from a box containing 10 black and 8 red balls. Find the probability that
		\label{ncert/12/13/2/12}
\begin{enumerate}
\item both balls are red.
\item first ball is black and second is red.
\item one of them is black and other is red.
\end{enumerate}

\item In a hostel, 60\% of the students read Hindi newspaper, 40\% read English newspaper and 20\% read both Hindi and English newspapers. A student is selected at random.
		\label{ncert/12/13/2/15}
\begin{enumerate}
\item Find the probability that she reads neither Hindi nor English newspapers.
\item If she reads Hindi newspaper, find the probability that she reads English newspaper.
\item If she reads English newspaper, find the probability that she reads Hindi newspaper.\\
\end{enumerate}
\item The probability of obtaining an even prime number on each die, when a pair of dice is rolled is 
\begin{enumerate}
    \item $0$ 
    
    \item $\frac{1}{3}$ 
    
    \item $\frac{1}{12}$ 
    
    \item $\frac{1}{36}$ 
\end{enumerate}
\solution
		%\input{ncert/12/13/2/17/defs.tex}
	\item A bag contains 4 red and 4 black balls, another bag contains 2 red and 6 black balls. One of the two bags is selected at random and a ball is drawn from the bag which is found to be red. Find the probability that the ball is drawn from the first bag.
\\
\solution
		%\input{ncert/12/13/3/2/main.tex}
  \item
  Cards with numbers 2 to 101 are placed in a box. A card is selected at random.Find the probability that the card has
\begin{enumerate}[label=(\roman*)]
	\item an even number 
	\item a square number
\end{enumerate}
\solution
%\input{exemplar/10/13/3/32/main.tex}
\item
The king, queen and jack of clubs are removed from a deck of 52 playing cards and then well shuffled. Now one card is drawn at random from the remaining cards.  Determine the probability that the card is
\begin{enumerate}[label=(\roman*)]
\item a club
\item 10 of hearts
\end{enumerate}
\solution
%\input{exemplar/10/13/3/29/main.tex}
\item A team of medical students doing their internship have to assist during surgeries
at a city hospital. The probabilities of surgeries rated as very complex, complex,
routine, simple or very simple are respectively, 0.15, 0.20, 0.31, 0.26, .08. Find
the probabilities that a particular surgery will be rated
\begin{enumerate}
	\item complex or very complex;
	\item neither very complex nor very simple;
	\item routine or complex
	\item routine or simple
\end{enumerate}
\solution
%\input{exemplar/11/16/3/8(1)/main.tex}
\item A card is selected from a pack of 52 cards.
\begin{enumerate}[label=(\alph*)]
    \item How many points are there in the sample space?
    \item Calculate the probability that the card is an ace of spades.
    \item Calculate the probability that the card is (i) an ace and (ii) black card.
\end{enumerate}
\solution
%\input{exemplar/11/16/3/4/main2.tex}
\item The probability that a non leap year selected at random will contain 53 sundays.
\\
\solution
%\input{exemplar/10/13/1/19/main.tex}
\item One of the four persons John, Rita, Aslam or Gurpreet will be promoted next
month. Consequently the sample space consists of four elementary outcomes
S = {John promoted, Rita promoted, Aslam promoted, Gurpreet promoted}
You are told that the chances of John’s promotion is same as that of Gurpreet,
Rita’s chances of promotion are twice as likely as Johns. Aslam’s chances are
four times that of John.
\begin{enumerate}
	\item Determine
	\begin{enumerate}
		\item P (John promoted)
		\item P (Rita promoted)
		\item P (Aslam promoted)
		\item P (Gurpreet promoted)
	\end{enumerate}
	\item If A = {John promoted or Gurpreet promoted}, find P (A).
\end{enumerate}
\solution
%\input{exemplar/11/16/3/10/main.tex}
\item A card is drawn from a deck of 52 cards. Find the probability of getting a king or a heart or a red card.\\
\solution
%\input{exemplar/11/16/3/15/main.tex}
\item The probability that a student will pass his examination is 0.73, the probability of
the student getting a compartment is 0.13, and the probability that the student will
either pass or get compartment is 0.96. State True or False.\\
\solution
%\input{exemplar/11/16/3/31/main.tex}
\item A card is selected from a pack of 52 cards\\
\begin{enumerate}[label=(\alph*)]
\item How many points are there in the sample space?
\item Calculate the probability that the cards is an ace of spades.
\item Calculate the probability that the card is (i) an ace (ii)black card.\\
\end{enumerate}
%\input{ncert/11/16/3/4_1/Prob_4.tex}
\item In a non-leap year, the probability of having 53 tuesdays or 53 wednesdays is\\
\solution
%\input{exemplar/11/16/3/18/main.tex}
\item There are 1000 sealed envelopes in a box, 10 of them contain a cash prize of
Rs 100 each, 100 of them contain a cash prize of Rs 50 each and 200 of them
contain a cash prize of Rs 10 each and rest do not contain any cash prize. If they
are well shuffled and an envelope is picked up out, what is the probability that it
contains no cash prize?\\
\solution
%\input{exemplar/10/13/3/34/main.tex}
\item 
A die is thrown and a card is selected at random from a deck of 52 playing cards. The probability of getting an even number on the die and a spade card.\\
\solution
%\input{exemplar/12/13/3/78/main.tex}
\item
If 4-digit numbers greater than 5,000 are randomly formed from the digits 0, 1, 3, 5, and 7, what is the probability of forming a number divisible by 5 when:
\begin{enumerate}
    \item The digits are repeated?
    \item The repetition of digits is not allowed?
\end{enumerate}
\solution
%\input{ncert/11/16/4/9/main.tex}
\item Consider the probability space $\brak{\Omega, \mathcal{G}, P}$ where $\Omega = [0,2]$ and $\mathcal{G} = \cbrak{\phi, \Omega, [0,1], (1,2]}$. Let $X$ and $Y$ be two functions on $\Omega$ defined as
\begin{align*}
    X(\omega) = 
    \begin{cases}
        1 & \text{if }\omega \in [0, 1]\\
        2 & \text{if }\omega \in (1, 2]
    \end{cases}
\end{align*}
and
\begin{align*}
    Y(\omega) = 
    \begin{cases}
        2 & \text{if }\omega \in [0, 1.5]\\
        3 & \text{if }\omega \in (1.5, 2].
    \end{cases}
\end{align*}
Then which one of the following statements is true?
\begin{enumerate}
    \item [(A)] $X$ is a random variable with respect to $\mathcal{G}$, but $Y$ is not a random variable with respect to $\mathcal{G}$.
    \item [(B)] $Y$ is a random variable with respect to $\mathcal{G}$, but $X$ is not a random variable with respect to $\mathcal{G}$.
    \item [(C)] Neither $X$ nor $Y$ is a random variable with respect to $\mathcal{G}$.
    \item [(D)] Both $X$ and $Y$ are random variables with respect to $\mathcal{G}$.
\end{enumerate} \hfill (GATE ST 2023)\\
\solution
%\input{gate/ST/2023/14/main.tex}
	\item  A die is loaded in such a way that each odd number is twice as likely to occur as
each even number. Find $P(G)$, where $G$ is the event that a number greater than
3 occurs on a single roll of the die.
\\
\solution
		%\input{exemplar/11/16/3/5/main.tex}
	\item All the jacks, queens and kings are removed from a deck of 52 playing cards. The remaining cards are well shuffled and then one card is drawn at random. Giving ace a value 1 similar value for other cards, find the probability that the card has a value 
		\begin{enumerate}
			\item 7
			\item greater than 7
			\item less than 7
		\end{enumerate}
		%\input{exemplar/10/13/3/30/main.tex}
  \item A Lot consists of 48 mobile phones of which 42 are good, 3 have only minor defects and 3 have major defects.Varnika will buy a phone if it is good but the trader will only buy a mobile if it has no major defects. One phone is selected at random from the lot. What is the probability that it is
\begin{enumerate}
	\item acceptable to Varnika?
            \item acceptable to the trader?
\end{enumerate}
\solution
	%\input{exemplar/10/13/3/40/main.tex}
 \item A student says that if you throw a die, it will show up 1 or not 1. Therefore, the probability of getting 1 and the probability of getting 'not 1' each is equal to $\frac{1}{2}$. Is this correct? Give reasons.\\
 \solution
        %\input{exemplar/10/13/2/9/main.tex}
   \item Four candidates A, B, C, D have ap-
plied for the assignment to coach a school cricket
team. If A is twice as likely to be selected as B, and
B and C are given about the same chance of being
selected, while C is twice as likely to be selected
as D, what are the probabilities that
\begin{enumerate}
\item C will be selected?
\item A will not be selected?
\end{enumerate}
	%\input{exemplar/11/16/3/9/main.tex}
 \item A bag contain 24 balls of which $x$ balls are red, $2x$ are white and $3x$ are blue. A ball is selected at random, What is the probability that it is
\begin{enumerate}[label=\alph*)]
\item not red ?
\item white ?
\end{enumerate}
%\input{exemplar/10/13/3/41/main.tex}
If the letters of the word ASSASSINATION are arranged at random. Find the Probability that
\begin{enumerate}[label=(\alph*)]
\item Four $S's$ come consecutively in the word
\item Two  $I's$ and two $N's$ come together
\item All $A's$ are not coming together
\item No two $A's$ are coming together
\end{enumerate}
%\input{exemplar/11/16/3/14/main.tex}
	\item One urn contains two black balls (labelled B1 and B2) and one white ball. A
	second urn contains one black ball and two white balls (labelled W1 and W2).
	Suppose the following experiment is performed. One of the two urns is chosen
	at random. Next a ball is randomly chosen from the urn. Then a second ball is
	chosen at random from the same urn without replacing the first ball.
	
	\begin{enumerate}
	\item What is the probability that two black balls are chosen?
	
	\item What is the probability that two balls of opposite colour are chosen?
	\end{enumerate}
	\solution
	%\input{exemplar/11/16/3/12/main1.tex}
\end{enumerate}

		%
\item 
Out of 100 students, two sections of 40 and 60 are formed. If you and your friend are among the 100 students, what is the probability that
\begin{enumerate}
\item you both enter the same section?
\item you both enter the different sections?
\end{enumerate}
\solution
		%\begin{enumerate}[label=\thesection.\arabic*,ref=\thesection.\theenumi]
	\item One card is drawn from a well-shuffled deck of 52 cards. Find the probability of getting
\begin{enumerate}
\item A king of red colour 
\item A face card 
\item A red face card
\item The jack of hearts
\item A spade
\item The queen of diamonds

\end{enumerate}
\solution
		%\input{ncert/10/15/1/14/main.tex}
	\item Five cards—the ten, jack, queen, king and ace of diamonds, are well-shuffled with their face downwards. One card is then picked up at random.
\begin{enumerate}
\item
What is the probability that the card is the queen? 
\item
If the queen is drawn and put aside, what is the probability that the second card picked up is (a) an ace? (b) a queen?\\
\end{enumerate}
\solution
		%\input{ncert/10/15/1/15/defs.tex}
	\item A bag contains $5$ red balls and some blue balls. If the probability of drawing a blue ball is double that if a red ball, determine the number of blue balls in the bag. 
		\\
\solution
		%\input{ncert/10/15/2/3/defs.tex}
	\item A card is selected from a pack of 52 cards.
 \begin{enumerate}[label=(\alph*)] 
                 \item How many points are there in the sample space?
                 \item Calculate the probability that the card is an ace of spades.
                 \item Calculate the probability that the card is (i) an ace and (ii) black card.
 \end{enumerate}
\solution
		%\input{ncert/11/16/3/4/main.tex}
\item Four cards are drawn from a well-shuffled deck of 52 cards. What is the probability of obtaining 3 diamonds and one spade.
\\
\solution
		%\input{ncert/11/16/4/2/defs.tex}
\item In a certain lottery 10,000 tickets are sold and ten equal prizes are awarded. What is the probability of not getting a prize if you buy (a) one ticket (b) two tickets (c) 10 tickets ?	
\\
\solution
		%\input{ncert/11/16/4/4/defs.tex}
		%
\item 
Out of 100 students, two sections of 40 and 60 are formed. If you and your friend are among the 100 students, what is the probability that
\begin{enumerate}
\item you both enter the same section?
\item you both enter the different sections?
\end{enumerate}
\solution
		%\input{ncert/11/16/4/5/defs.tex}
	\item 
The number lock of a suitcase has 4 wheels each labelled with ten digits i.e. from 0 to 9.The lock opens with a sequence of four digits with no repeats.What is the probability of a person getting the right sequence to open the suitcase.
\\
\solution
		%\input{ncert/11/16/4/10/defs.tex}
		%
\item 
Two cards are drawn at random and without replacement from a pack of 52 playing cards. Find the probability that both the cards are black.
\\
\solution
		%\input{ncert/12/13/2/2/defs.tex}
		\item A box of oranges is inspected by examining three randomly selected oranges drawn without replacement. If all the three oranges are good, the box is approved for sale, otherwise, it is rejected. Find the probability that a box containing 15 oranges out of which 12 are good and 3 are bad ones will be approved for sale.
		\label{ncert/12/13/2/3/defs.tex}
		\item Two balls are drawn at random with replacement from a box containing 10 black and 8 red balls. Find the probability that
		\label{ncert/12/13/2/12}
\begin{enumerate}
\item both balls are red.
\item first ball is black and second is red.
\item one of them is black and other is red.
\end{enumerate}

\item In a hostel, 60\% of the students read Hindi newspaper, 40\% read English newspaper and 20\% read both Hindi and English newspapers. A student is selected at random.
		\label{ncert/12/13/2/15}
\begin{enumerate}
\item Find the probability that she reads neither Hindi nor English newspapers.
\item If she reads Hindi newspaper, find the probability that she reads English newspaper.
\item If she reads English newspaper, find the probability that she reads Hindi newspaper.\\
\end{enumerate}
\item The probability of obtaining an even prime number on each die, when a pair of dice is rolled is 
\begin{enumerate}
    \item $0$ 
    
    \item $\frac{1}{3}$ 
    
    \item $\frac{1}{12}$ 
    
    \item $\frac{1}{36}$ 
\end{enumerate}
\solution
		%\input{ncert/12/13/2/17/defs.tex}
	\item A bag contains 4 red and 4 black balls, another bag contains 2 red and 6 black balls. One of the two bags is selected at random and a ball is drawn from the bag which is found to be red. Find the probability that the ball is drawn from the first bag.
\\
\solution
		%\input{ncert/12/13/3/2/main.tex}
  \item
  Cards with numbers 2 to 101 are placed in a box. A card is selected at random.Find the probability that the card has
\begin{enumerate}[label=(\roman*)]
	\item an even number 
	\item a square number
\end{enumerate}
\solution
%\input{exemplar/10/13/3/32/main.tex}
\item
The king, queen and jack of clubs are removed from a deck of 52 playing cards and then well shuffled. Now one card is drawn at random from the remaining cards.  Determine the probability that the card is
\begin{enumerate}[label=(\roman*)]
\item a club
\item 10 of hearts
\end{enumerate}
\solution
%\input{exemplar/10/13/3/29/main.tex}
\item A team of medical students doing their internship have to assist during surgeries
at a city hospital. The probabilities of surgeries rated as very complex, complex,
routine, simple or very simple are respectively, 0.15, 0.20, 0.31, 0.26, .08. Find
the probabilities that a particular surgery will be rated
\begin{enumerate}
	\item complex or very complex;
	\item neither very complex nor very simple;
	\item routine or complex
	\item routine or simple
\end{enumerate}
\solution
%\input{exemplar/11/16/3/8(1)/main.tex}
\item A card is selected from a pack of 52 cards.
\begin{enumerate}[label=(\alph*)]
    \item How many points are there in the sample space?
    \item Calculate the probability that the card is an ace of spades.
    \item Calculate the probability that the card is (i) an ace and (ii) black card.
\end{enumerate}
\solution
%\input{exemplar/11/16/3/4/main2.tex}
\item The probability that a non leap year selected at random will contain 53 sundays.
\\
\solution
%\input{exemplar/10/13/1/19/main.tex}
\item One of the four persons John, Rita, Aslam or Gurpreet will be promoted next
month. Consequently the sample space consists of four elementary outcomes
S = {John promoted, Rita promoted, Aslam promoted, Gurpreet promoted}
You are told that the chances of John’s promotion is same as that of Gurpreet,
Rita’s chances of promotion are twice as likely as Johns. Aslam’s chances are
four times that of John.
\begin{enumerate}
	\item Determine
	\begin{enumerate}
		\item P (John promoted)
		\item P (Rita promoted)
		\item P (Aslam promoted)
		\item P (Gurpreet promoted)
	\end{enumerate}
	\item If A = {John promoted or Gurpreet promoted}, find P (A).
\end{enumerate}
\solution
%\input{exemplar/11/16/3/10/main.tex}
\item A card is drawn from a deck of 52 cards. Find the probability of getting a king or a heart or a red card.\\
\solution
%\input{exemplar/11/16/3/15/main.tex}
\item The probability that a student will pass his examination is 0.73, the probability of
the student getting a compartment is 0.13, and the probability that the student will
either pass or get compartment is 0.96. State True or False.\\
\solution
%\input{exemplar/11/16/3/31/main.tex}
\item A card is selected from a pack of 52 cards\\
\begin{enumerate}[label=(\alph*)]
\item How many points are there in the sample space?
\item Calculate the probability that the cards is an ace of spades.
\item Calculate the probability that the card is (i) an ace (ii)black card.\\
\end{enumerate}
%\input{ncert/11/16/3/4_1/Prob_4.tex}
\item In a non-leap year, the probability of having 53 tuesdays or 53 wednesdays is\\
\solution
%\input{exemplar/11/16/3/18/main.tex}
\item There are 1000 sealed envelopes in a box, 10 of them contain a cash prize of
Rs 100 each, 100 of them contain a cash prize of Rs 50 each and 200 of them
contain a cash prize of Rs 10 each and rest do not contain any cash prize. If they
are well shuffled and an envelope is picked up out, what is the probability that it
contains no cash prize?\\
\solution
%\input{exemplar/10/13/3/34/main.tex}
\item 
A die is thrown and a card is selected at random from a deck of 52 playing cards. The probability of getting an even number on the die and a spade card.\\
\solution
%\input{exemplar/12/13/3/78/main.tex}
\item
If 4-digit numbers greater than 5,000 are randomly formed from the digits 0, 1, 3, 5, and 7, what is the probability of forming a number divisible by 5 when:
\begin{enumerate}
    \item The digits are repeated?
    \item The repetition of digits is not allowed?
\end{enumerate}
\solution
%\input{ncert/11/16/4/9/main.tex}
\item Consider the probability space $\brak{\Omega, \mathcal{G}, P}$ where $\Omega = [0,2]$ and $\mathcal{G} = \cbrak{\phi, \Omega, [0,1], (1,2]}$. Let $X$ and $Y$ be two functions on $\Omega$ defined as
\begin{align*}
    X(\omega) = 
    \begin{cases}
        1 & \text{if }\omega \in [0, 1]\\
        2 & \text{if }\omega \in (1, 2]
    \end{cases}
\end{align*}
and
\begin{align*}
    Y(\omega) = 
    \begin{cases}
        2 & \text{if }\omega \in [0, 1.5]\\
        3 & \text{if }\omega \in (1.5, 2].
    \end{cases}
\end{align*}
Then which one of the following statements is true?
\begin{enumerate}
    \item [(A)] $X$ is a random variable with respect to $\mathcal{G}$, but $Y$ is not a random variable with respect to $\mathcal{G}$.
    \item [(B)] $Y$ is a random variable with respect to $\mathcal{G}$, but $X$ is not a random variable with respect to $\mathcal{G}$.
    \item [(C)] Neither $X$ nor $Y$ is a random variable with respect to $\mathcal{G}$.
    \item [(D)] Both $X$ and $Y$ are random variables with respect to $\mathcal{G}$.
\end{enumerate} \hfill (GATE ST 2023)\\
\solution
%\input{gate/ST/2023/14/main.tex}
	\item  A die is loaded in such a way that each odd number is twice as likely to occur as
each even number. Find $P(G)$, where $G$ is the event that a number greater than
3 occurs on a single roll of the die.
\\
\solution
		%\input{exemplar/11/16/3/5/main.tex}
	\item All the jacks, queens and kings are removed from a deck of 52 playing cards. The remaining cards are well shuffled and then one card is drawn at random. Giving ace a value 1 similar value for other cards, find the probability that the card has a value 
		\begin{enumerate}
			\item 7
			\item greater than 7
			\item less than 7
		\end{enumerate}
		%\input{exemplar/10/13/3/30/main.tex}
  \item A Lot consists of 48 mobile phones of which 42 are good, 3 have only minor defects and 3 have major defects.Varnika will buy a phone if it is good but the trader will only buy a mobile if it has no major defects. One phone is selected at random from the lot. What is the probability that it is
\begin{enumerate}
	\item acceptable to Varnika?
            \item acceptable to the trader?
\end{enumerate}
\solution
	%\input{exemplar/10/13/3/40/main.tex}
 \item A student says that if you throw a die, it will show up 1 or not 1. Therefore, the probability of getting 1 and the probability of getting 'not 1' each is equal to $\frac{1}{2}$. Is this correct? Give reasons.\\
 \solution
        %\input{exemplar/10/13/2/9/main.tex}
   \item Four candidates A, B, C, D have ap-
plied for the assignment to coach a school cricket
team. If A is twice as likely to be selected as B, and
B and C are given about the same chance of being
selected, while C is twice as likely to be selected
as D, what are the probabilities that
\begin{enumerate}
\item C will be selected?
\item A will not be selected?
\end{enumerate}
	%\input{exemplar/11/16/3/9/main.tex}
 \item A bag contain 24 balls of which $x$ balls are red, $2x$ are white and $3x$ are blue. A ball is selected at random, What is the probability that it is
\begin{enumerate}[label=\alph*)]
\item not red ?
\item white ?
\end{enumerate}
%\input{exemplar/10/13/3/41/main.tex}
If the letters of the word ASSASSINATION are arranged at random. Find the Probability that
\begin{enumerate}[label=(\alph*)]
\item Four $S's$ come consecutively in the word
\item Two  $I's$ and two $N's$ come together
\item All $A's$ are not coming together
\item No two $A's$ are coming together
\end{enumerate}
%\input{exemplar/11/16/3/14/main.tex}
	\item One urn contains two black balls (labelled B1 and B2) and one white ball. A
	second urn contains one black ball and two white balls (labelled W1 and W2).
	Suppose the following experiment is performed. One of the two urns is chosen
	at random. Next a ball is randomly chosen from the urn. Then a second ball is
	chosen at random from the same urn without replacing the first ball.
	
	\begin{enumerate}
	\item What is the probability that two black balls are chosen?
	
	\item What is the probability that two balls of opposite colour are chosen?
	\end{enumerate}
	\solution
	%\input{exemplar/11/16/3/12/main1.tex}
\end{enumerate}

	\item 
The number lock of a suitcase has 4 wheels each labelled with ten digits i.e. from 0 to 9.The lock opens with a sequence of four digits with no repeats.What is the probability of a person getting the right sequence to open the suitcase.
\\
\solution
		%\begin{enumerate}[label=\thesection.\arabic*,ref=\thesection.\theenumi]
	\item One card is drawn from a well-shuffled deck of 52 cards. Find the probability of getting
\begin{enumerate}
\item A king of red colour 
\item A face card 
\item A red face card
\item The jack of hearts
\item A spade
\item The queen of diamonds

\end{enumerate}
\solution
		%\input{ncert/10/15/1/14/main.tex}
	\item Five cards—the ten, jack, queen, king and ace of diamonds, are well-shuffled with their face downwards. One card is then picked up at random.
\begin{enumerate}
\item
What is the probability that the card is the queen? 
\item
If the queen is drawn and put aside, what is the probability that the second card picked up is (a) an ace? (b) a queen?\\
\end{enumerate}
\solution
		%\input{ncert/10/15/1/15/defs.tex}
	\item A bag contains $5$ red balls and some blue balls. If the probability of drawing a blue ball is double that if a red ball, determine the number of blue balls in the bag. 
		\\
\solution
		%\input{ncert/10/15/2/3/defs.tex}
	\item A card is selected from a pack of 52 cards.
 \begin{enumerate}[label=(\alph*)] 
                 \item How many points are there in the sample space?
                 \item Calculate the probability that the card is an ace of spades.
                 \item Calculate the probability that the card is (i) an ace and (ii) black card.
 \end{enumerate}
\solution
		%\input{ncert/11/16/3/4/main.tex}
\item Four cards are drawn from a well-shuffled deck of 52 cards. What is the probability of obtaining 3 diamonds and one spade.
\\
\solution
		%\input{ncert/11/16/4/2/defs.tex}
\item In a certain lottery 10,000 tickets are sold and ten equal prizes are awarded. What is the probability of not getting a prize if you buy (a) one ticket (b) two tickets (c) 10 tickets ?	
\\
\solution
		%\input{ncert/11/16/4/4/defs.tex}
		%
\item 
Out of 100 students, two sections of 40 and 60 are formed. If you and your friend are among the 100 students, what is the probability that
\begin{enumerate}
\item you both enter the same section?
\item you both enter the different sections?
\end{enumerate}
\solution
		%\input{ncert/11/16/4/5/defs.tex}
	\item 
The number lock of a suitcase has 4 wheels each labelled with ten digits i.e. from 0 to 9.The lock opens with a sequence of four digits with no repeats.What is the probability of a person getting the right sequence to open the suitcase.
\\
\solution
		%\input{ncert/11/16/4/10/defs.tex}
		%
\item 
Two cards are drawn at random and without replacement from a pack of 52 playing cards. Find the probability that both the cards are black.
\\
\solution
		%\input{ncert/12/13/2/2/defs.tex}
		\item A box of oranges is inspected by examining three randomly selected oranges drawn without replacement. If all the three oranges are good, the box is approved for sale, otherwise, it is rejected. Find the probability that a box containing 15 oranges out of which 12 are good and 3 are bad ones will be approved for sale.
		\label{ncert/12/13/2/3/defs.tex}
		\item Two balls are drawn at random with replacement from a box containing 10 black and 8 red balls. Find the probability that
		\label{ncert/12/13/2/12}
\begin{enumerate}
\item both balls are red.
\item first ball is black and second is red.
\item one of them is black and other is red.
\end{enumerate}

\item In a hostel, 60\% of the students read Hindi newspaper, 40\% read English newspaper and 20\% read both Hindi and English newspapers. A student is selected at random.
		\label{ncert/12/13/2/15}
\begin{enumerate}
\item Find the probability that she reads neither Hindi nor English newspapers.
\item If she reads Hindi newspaper, find the probability that she reads English newspaper.
\item If she reads English newspaper, find the probability that she reads Hindi newspaper.\\
\end{enumerate}
\item The probability of obtaining an even prime number on each die, when a pair of dice is rolled is 
\begin{enumerate}
    \item $0$ 
    
    \item $\frac{1}{3}$ 
    
    \item $\frac{1}{12}$ 
    
    \item $\frac{1}{36}$ 
\end{enumerate}
\solution
		%\input{ncert/12/13/2/17/defs.tex}
	\item A bag contains 4 red and 4 black balls, another bag contains 2 red and 6 black balls. One of the two bags is selected at random and a ball is drawn from the bag which is found to be red. Find the probability that the ball is drawn from the first bag.
\\
\solution
		%\input{ncert/12/13/3/2/main.tex}
  \item
  Cards with numbers 2 to 101 are placed in a box. A card is selected at random.Find the probability that the card has
\begin{enumerate}[label=(\roman*)]
	\item an even number 
	\item a square number
\end{enumerate}
\solution
%\input{exemplar/10/13/3/32/main.tex}
\item
The king, queen and jack of clubs are removed from a deck of 52 playing cards and then well shuffled. Now one card is drawn at random from the remaining cards.  Determine the probability that the card is
\begin{enumerate}[label=(\roman*)]
\item a club
\item 10 of hearts
\end{enumerate}
\solution
%\input{exemplar/10/13/3/29/main.tex}
\item A team of medical students doing their internship have to assist during surgeries
at a city hospital. The probabilities of surgeries rated as very complex, complex,
routine, simple or very simple are respectively, 0.15, 0.20, 0.31, 0.26, .08. Find
the probabilities that a particular surgery will be rated
\begin{enumerate}
	\item complex or very complex;
	\item neither very complex nor very simple;
	\item routine or complex
	\item routine or simple
\end{enumerate}
\solution
%\input{exemplar/11/16/3/8(1)/main.tex}
\item A card is selected from a pack of 52 cards.
\begin{enumerate}[label=(\alph*)]
    \item How many points are there in the sample space?
    \item Calculate the probability that the card is an ace of spades.
    \item Calculate the probability that the card is (i) an ace and (ii) black card.
\end{enumerate}
\solution
%\input{exemplar/11/16/3/4/main2.tex}
\item The probability that a non leap year selected at random will contain 53 sundays.
\\
\solution
%\input{exemplar/10/13/1/19/main.tex}
\item One of the four persons John, Rita, Aslam or Gurpreet will be promoted next
month. Consequently the sample space consists of four elementary outcomes
S = {John promoted, Rita promoted, Aslam promoted, Gurpreet promoted}
You are told that the chances of John’s promotion is same as that of Gurpreet,
Rita’s chances of promotion are twice as likely as Johns. Aslam’s chances are
four times that of John.
\begin{enumerate}
	\item Determine
	\begin{enumerate}
		\item P (John promoted)
		\item P (Rita promoted)
		\item P (Aslam promoted)
		\item P (Gurpreet promoted)
	\end{enumerate}
	\item If A = {John promoted or Gurpreet promoted}, find P (A).
\end{enumerate}
\solution
%\input{exemplar/11/16/3/10/main.tex}
\item A card is drawn from a deck of 52 cards. Find the probability of getting a king or a heart or a red card.\\
\solution
%\input{exemplar/11/16/3/15/main.tex}
\item The probability that a student will pass his examination is 0.73, the probability of
the student getting a compartment is 0.13, and the probability that the student will
either pass or get compartment is 0.96. State True or False.\\
\solution
%\input{exemplar/11/16/3/31/main.tex}
\item A card is selected from a pack of 52 cards\\
\begin{enumerate}[label=(\alph*)]
\item How many points are there in the sample space?
\item Calculate the probability that the cards is an ace of spades.
\item Calculate the probability that the card is (i) an ace (ii)black card.\\
\end{enumerate}
%\input{ncert/11/16/3/4_1/Prob_4.tex}
\item In a non-leap year, the probability of having 53 tuesdays or 53 wednesdays is\\
\solution
%\input{exemplar/11/16/3/18/main.tex}
\item There are 1000 sealed envelopes in a box, 10 of them contain a cash prize of
Rs 100 each, 100 of them contain a cash prize of Rs 50 each and 200 of them
contain a cash prize of Rs 10 each and rest do not contain any cash prize. If they
are well shuffled and an envelope is picked up out, what is the probability that it
contains no cash prize?\\
\solution
%\input{exemplar/10/13/3/34/main.tex}
\item 
A die is thrown and a card is selected at random from a deck of 52 playing cards. The probability of getting an even number on the die and a spade card.\\
\solution
%\input{exemplar/12/13/3/78/main.tex}
\item
If 4-digit numbers greater than 5,000 are randomly formed from the digits 0, 1, 3, 5, and 7, what is the probability of forming a number divisible by 5 when:
\begin{enumerate}
    \item The digits are repeated?
    \item The repetition of digits is not allowed?
\end{enumerate}
\solution
%\input{ncert/11/16/4/9/main.tex}
\item Consider the probability space $\brak{\Omega, \mathcal{G}, P}$ where $\Omega = [0,2]$ and $\mathcal{G} = \cbrak{\phi, \Omega, [0,1], (1,2]}$. Let $X$ and $Y$ be two functions on $\Omega$ defined as
\begin{align*}
    X(\omega) = 
    \begin{cases}
        1 & \text{if }\omega \in [0, 1]\\
        2 & \text{if }\omega \in (1, 2]
    \end{cases}
\end{align*}
and
\begin{align*}
    Y(\omega) = 
    \begin{cases}
        2 & \text{if }\omega \in [0, 1.5]\\
        3 & \text{if }\omega \in (1.5, 2].
    \end{cases}
\end{align*}
Then which one of the following statements is true?
\begin{enumerate}
    \item [(A)] $X$ is a random variable with respect to $\mathcal{G}$, but $Y$ is not a random variable with respect to $\mathcal{G}$.
    \item [(B)] $Y$ is a random variable with respect to $\mathcal{G}$, but $X$ is not a random variable with respect to $\mathcal{G}$.
    \item [(C)] Neither $X$ nor $Y$ is a random variable with respect to $\mathcal{G}$.
    \item [(D)] Both $X$ and $Y$ are random variables with respect to $\mathcal{G}$.
\end{enumerate} \hfill (GATE ST 2023)\\
\solution
%\input{gate/ST/2023/14/main.tex}
	\item  A die is loaded in such a way that each odd number is twice as likely to occur as
each even number. Find $P(G)$, where $G$ is the event that a number greater than
3 occurs on a single roll of the die.
\\
\solution
		%\input{exemplar/11/16/3/5/main.tex}
	\item All the jacks, queens and kings are removed from a deck of 52 playing cards. The remaining cards are well shuffled and then one card is drawn at random. Giving ace a value 1 similar value for other cards, find the probability that the card has a value 
		\begin{enumerate}
			\item 7
			\item greater than 7
			\item less than 7
		\end{enumerate}
		%\input{exemplar/10/13/3/30/main.tex}
  \item A Lot consists of 48 mobile phones of which 42 are good, 3 have only minor defects and 3 have major defects.Varnika will buy a phone if it is good but the trader will only buy a mobile if it has no major defects. One phone is selected at random from the lot. What is the probability that it is
\begin{enumerate}
	\item acceptable to Varnika?
            \item acceptable to the trader?
\end{enumerate}
\solution
	%\input{exemplar/10/13/3/40/main.tex}
 \item A student says that if you throw a die, it will show up 1 or not 1. Therefore, the probability of getting 1 and the probability of getting 'not 1' each is equal to $\frac{1}{2}$. Is this correct? Give reasons.\\
 \solution
        %\input{exemplar/10/13/2/9/main.tex}
   \item Four candidates A, B, C, D have ap-
plied for the assignment to coach a school cricket
team. If A is twice as likely to be selected as B, and
B and C are given about the same chance of being
selected, while C is twice as likely to be selected
as D, what are the probabilities that
\begin{enumerate}
\item C will be selected?
\item A will not be selected?
\end{enumerate}
	%\input{exemplar/11/16/3/9/main.tex}
 \item A bag contain 24 balls of which $x$ balls are red, $2x$ are white and $3x$ are blue. A ball is selected at random, What is the probability that it is
\begin{enumerate}[label=\alph*)]
\item not red ?
\item white ?
\end{enumerate}
%\input{exemplar/10/13/3/41/main.tex}
If the letters of the word ASSASSINATION are arranged at random. Find the Probability that
\begin{enumerate}[label=(\alph*)]
\item Four $S's$ come consecutively in the word
\item Two  $I's$ and two $N's$ come together
\item All $A's$ are not coming together
\item No two $A's$ are coming together
\end{enumerate}
%\input{exemplar/11/16/3/14/main.tex}
	\item One urn contains two black balls (labelled B1 and B2) and one white ball. A
	second urn contains one black ball and two white balls (labelled W1 and W2).
	Suppose the following experiment is performed. One of the two urns is chosen
	at random. Next a ball is randomly chosen from the urn. Then a second ball is
	chosen at random from the same urn without replacing the first ball.
	
	\begin{enumerate}
	\item What is the probability that two black balls are chosen?
	
	\item What is the probability that two balls of opposite colour are chosen?
	\end{enumerate}
	\solution
	%\input{exemplar/11/16/3/12/main1.tex}
\end{enumerate}

		%
\item 
Two cards are drawn at random and without replacement from a pack of 52 playing cards. Find the probability that both the cards are black.
\\
\solution
		%\begin{enumerate}[label=\thesection.\arabic*,ref=\thesection.\theenumi]
	\item One card is drawn from a well-shuffled deck of 52 cards. Find the probability of getting
\begin{enumerate}
\item A king of red colour 
\item A face card 
\item A red face card
\item The jack of hearts
\item A spade
\item The queen of diamonds

\end{enumerate}
\solution
		%\input{ncert/10/15/1/14/main.tex}
	\item Five cards—the ten, jack, queen, king and ace of diamonds, are well-shuffled with their face downwards. One card is then picked up at random.
\begin{enumerate}
\item
What is the probability that the card is the queen? 
\item
If the queen is drawn and put aside, what is the probability that the second card picked up is (a) an ace? (b) a queen?\\
\end{enumerate}
\solution
		%\input{ncert/10/15/1/15/defs.tex}
	\item A bag contains $5$ red balls and some blue balls. If the probability of drawing a blue ball is double that if a red ball, determine the number of blue balls in the bag. 
		\\
\solution
		%\input{ncert/10/15/2/3/defs.tex}
	\item A card is selected from a pack of 52 cards.
 \begin{enumerate}[label=(\alph*)] 
                 \item How many points are there in the sample space?
                 \item Calculate the probability that the card is an ace of spades.
                 \item Calculate the probability that the card is (i) an ace and (ii) black card.
 \end{enumerate}
\solution
		%\input{ncert/11/16/3/4/main.tex}
\item Four cards are drawn from a well-shuffled deck of 52 cards. What is the probability of obtaining 3 diamonds and one spade.
\\
\solution
		%\input{ncert/11/16/4/2/defs.tex}
\item In a certain lottery 10,000 tickets are sold and ten equal prizes are awarded. What is the probability of not getting a prize if you buy (a) one ticket (b) two tickets (c) 10 tickets ?	
\\
\solution
		%\input{ncert/11/16/4/4/defs.tex}
		%
\item 
Out of 100 students, two sections of 40 and 60 are formed. If you and your friend are among the 100 students, what is the probability that
\begin{enumerate}
\item you both enter the same section?
\item you both enter the different sections?
\end{enumerate}
\solution
		%\input{ncert/11/16/4/5/defs.tex}
	\item 
The number lock of a suitcase has 4 wheels each labelled with ten digits i.e. from 0 to 9.The lock opens with a sequence of four digits with no repeats.What is the probability of a person getting the right sequence to open the suitcase.
\\
\solution
		%\input{ncert/11/16/4/10/defs.tex}
		%
\item 
Two cards are drawn at random and without replacement from a pack of 52 playing cards. Find the probability that both the cards are black.
\\
\solution
		%\input{ncert/12/13/2/2/defs.tex}
		\item A box of oranges is inspected by examining three randomly selected oranges drawn without replacement. If all the three oranges are good, the box is approved for sale, otherwise, it is rejected. Find the probability that a box containing 15 oranges out of which 12 are good and 3 are bad ones will be approved for sale.
		\label{ncert/12/13/2/3/defs.tex}
		\item Two balls are drawn at random with replacement from a box containing 10 black and 8 red balls. Find the probability that
		\label{ncert/12/13/2/12}
\begin{enumerate}
\item both balls are red.
\item first ball is black and second is red.
\item one of them is black and other is red.
\end{enumerate}

\item In a hostel, 60\% of the students read Hindi newspaper, 40\% read English newspaper and 20\% read both Hindi and English newspapers. A student is selected at random.
		\label{ncert/12/13/2/15}
\begin{enumerate}
\item Find the probability that she reads neither Hindi nor English newspapers.
\item If she reads Hindi newspaper, find the probability that she reads English newspaper.
\item If she reads English newspaper, find the probability that she reads Hindi newspaper.\\
\end{enumerate}
\item The probability of obtaining an even prime number on each die, when a pair of dice is rolled is 
\begin{enumerate}
    \item $0$ 
    
    \item $\frac{1}{3}$ 
    
    \item $\frac{1}{12}$ 
    
    \item $\frac{1}{36}$ 
\end{enumerate}
\solution
		%\input{ncert/12/13/2/17/defs.tex}
	\item A bag contains 4 red and 4 black balls, another bag contains 2 red and 6 black balls. One of the two bags is selected at random and a ball is drawn from the bag which is found to be red. Find the probability that the ball is drawn from the first bag.
\\
\solution
		%\input{ncert/12/13/3/2/main.tex}
  \item
  Cards with numbers 2 to 101 are placed in a box. A card is selected at random.Find the probability that the card has
\begin{enumerate}[label=(\roman*)]
	\item an even number 
	\item a square number
\end{enumerate}
\solution
%\input{exemplar/10/13/3/32/main.tex}
\item
The king, queen and jack of clubs are removed from a deck of 52 playing cards and then well shuffled. Now one card is drawn at random from the remaining cards.  Determine the probability that the card is
\begin{enumerate}[label=(\roman*)]
\item a club
\item 10 of hearts
\end{enumerate}
\solution
%\input{exemplar/10/13/3/29/main.tex}
\item A team of medical students doing their internship have to assist during surgeries
at a city hospital. The probabilities of surgeries rated as very complex, complex,
routine, simple or very simple are respectively, 0.15, 0.20, 0.31, 0.26, .08. Find
the probabilities that a particular surgery will be rated
\begin{enumerate}
	\item complex or very complex;
	\item neither very complex nor very simple;
	\item routine or complex
	\item routine or simple
\end{enumerate}
\solution
%\input{exemplar/11/16/3/8(1)/main.tex}
\item A card is selected from a pack of 52 cards.
\begin{enumerate}[label=(\alph*)]
    \item How many points are there in the sample space?
    \item Calculate the probability that the card is an ace of spades.
    \item Calculate the probability that the card is (i) an ace and (ii) black card.
\end{enumerate}
\solution
%\input{exemplar/11/16/3/4/main2.tex}
\item The probability that a non leap year selected at random will contain 53 sundays.
\\
\solution
%\input{exemplar/10/13/1/19/main.tex}
\item One of the four persons John, Rita, Aslam or Gurpreet will be promoted next
month. Consequently the sample space consists of four elementary outcomes
S = {John promoted, Rita promoted, Aslam promoted, Gurpreet promoted}
You are told that the chances of John’s promotion is same as that of Gurpreet,
Rita’s chances of promotion are twice as likely as Johns. Aslam’s chances are
four times that of John.
\begin{enumerate}
	\item Determine
	\begin{enumerate}
		\item P (John promoted)
		\item P (Rita promoted)
		\item P (Aslam promoted)
		\item P (Gurpreet promoted)
	\end{enumerate}
	\item If A = {John promoted or Gurpreet promoted}, find P (A).
\end{enumerate}
\solution
%\input{exemplar/11/16/3/10/main.tex}
\item A card is drawn from a deck of 52 cards. Find the probability of getting a king or a heart or a red card.\\
\solution
%\input{exemplar/11/16/3/15/main.tex}
\item The probability that a student will pass his examination is 0.73, the probability of
the student getting a compartment is 0.13, and the probability that the student will
either pass or get compartment is 0.96. State True or False.\\
\solution
%\input{exemplar/11/16/3/31/main.tex}
\item A card is selected from a pack of 52 cards\\
\begin{enumerate}[label=(\alph*)]
\item How many points are there in the sample space?
\item Calculate the probability that the cards is an ace of spades.
\item Calculate the probability that the card is (i) an ace (ii)black card.\\
\end{enumerate}
%\input{ncert/11/16/3/4_1/Prob_4.tex}
\item In a non-leap year, the probability of having 53 tuesdays or 53 wednesdays is\\
\solution
%\input{exemplar/11/16/3/18/main.tex}
\item There are 1000 sealed envelopes in a box, 10 of them contain a cash prize of
Rs 100 each, 100 of them contain a cash prize of Rs 50 each and 200 of them
contain a cash prize of Rs 10 each and rest do not contain any cash prize. If they
are well shuffled and an envelope is picked up out, what is the probability that it
contains no cash prize?\\
\solution
%\input{exemplar/10/13/3/34/main.tex}
\item 
A die is thrown and a card is selected at random from a deck of 52 playing cards. The probability of getting an even number on the die and a spade card.\\
\solution
%\input{exemplar/12/13/3/78/main.tex}
\item
If 4-digit numbers greater than 5,000 are randomly formed from the digits 0, 1, 3, 5, and 7, what is the probability of forming a number divisible by 5 when:
\begin{enumerate}
    \item The digits are repeated?
    \item The repetition of digits is not allowed?
\end{enumerate}
\solution
%\input{ncert/11/16/4/9/main.tex}
\item Consider the probability space $\brak{\Omega, \mathcal{G}, P}$ where $\Omega = [0,2]$ and $\mathcal{G} = \cbrak{\phi, \Omega, [0,1], (1,2]}$. Let $X$ and $Y$ be two functions on $\Omega$ defined as
\begin{align*}
    X(\omega) = 
    \begin{cases}
        1 & \text{if }\omega \in [0, 1]\\
        2 & \text{if }\omega \in (1, 2]
    \end{cases}
\end{align*}
and
\begin{align*}
    Y(\omega) = 
    \begin{cases}
        2 & \text{if }\omega \in [0, 1.5]\\
        3 & \text{if }\omega \in (1.5, 2].
    \end{cases}
\end{align*}
Then which one of the following statements is true?
\begin{enumerate}
    \item [(A)] $X$ is a random variable with respect to $\mathcal{G}$, but $Y$ is not a random variable with respect to $\mathcal{G}$.
    \item [(B)] $Y$ is a random variable with respect to $\mathcal{G}$, but $X$ is not a random variable with respect to $\mathcal{G}$.
    \item [(C)] Neither $X$ nor $Y$ is a random variable with respect to $\mathcal{G}$.
    \item [(D)] Both $X$ and $Y$ are random variables with respect to $\mathcal{G}$.
\end{enumerate} \hfill (GATE ST 2023)\\
\solution
%\input{gate/ST/2023/14/main.tex}
	\item  A die is loaded in such a way that each odd number is twice as likely to occur as
each even number. Find $P(G)$, where $G$ is the event that a number greater than
3 occurs on a single roll of the die.
\\
\solution
		%\input{exemplar/11/16/3/5/main.tex}
	\item All the jacks, queens and kings are removed from a deck of 52 playing cards. The remaining cards are well shuffled and then one card is drawn at random. Giving ace a value 1 similar value for other cards, find the probability that the card has a value 
		\begin{enumerate}
			\item 7
			\item greater than 7
			\item less than 7
		\end{enumerate}
		%\input{exemplar/10/13/3/30/main.tex}
  \item A Lot consists of 48 mobile phones of which 42 are good, 3 have only minor defects and 3 have major defects.Varnika will buy a phone if it is good but the trader will only buy a mobile if it has no major defects. One phone is selected at random from the lot. What is the probability that it is
\begin{enumerate}
	\item acceptable to Varnika?
            \item acceptable to the trader?
\end{enumerate}
\solution
	%\input{exemplar/10/13/3/40/main.tex}
 \item A student says that if you throw a die, it will show up 1 or not 1. Therefore, the probability of getting 1 and the probability of getting 'not 1' each is equal to $\frac{1}{2}$. Is this correct? Give reasons.\\
 \solution
        %\input{exemplar/10/13/2/9/main.tex}
   \item Four candidates A, B, C, D have ap-
plied for the assignment to coach a school cricket
team. If A is twice as likely to be selected as B, and
B and C are given about the same chance of being
selected, while C is twice as likely to be selected
as D, what are the probabilities that
\begin{enumerate}
\item C will be selected?
\item A will not be selected?
\end{enumerate}
	%\input{exemplar/11/16/3/9/main.tex}
 \item A bag contain 24 balls of which $x$ balls are red, $2x$ are white and $3x$ are blue. A ball is selected at random, What is the probability that it is
\begin{enumerate}[label=\alph*)]
\item not red ?
\item white ?
\end{enumerate}
%\input{exemplar/10/13/3/41/main.tex}
If the letters of the word ASSASSINATION are arranged at random. Find the Probability that
\begin{enumerate}[label=(\alph*)]
\item Four $S's$ come consecutively in the word
\item Two  $I's$ and two $N's$ come together
\item All $A's$ are not coming together
\item No two $A's$ are coming together
\end{enumerate}
%\input{exemplar/11/16/3/14/main.tex}
	\item One urn contains two black balls (labelled B1 and B2) and one white ball. A
	second urn contains one black ball and two white balls (labelled W1 and W2).
	Suppose the following experiment is performed. One of the two urns is chosen
	at random. Next a ball is randomly chosen from the urn. Then a second ball is
	chosen at random from the same urn without replacing the first ball.
	
	\begin{enumerate}
	\item What is the probability that two black balls are chosen?
	
	\item What is the probability that two balls of opposite colour are chosen?
	\end{enumerate}
	\solution
	%\input{exemplar/11/16/3/12/main1.tex}
\end{enumerate}

		\item A box of oranges is inspected by examining three randomly selected oranges drawn without replacement. If all the three oranges are good, the box is approved for sale, otherwise, it is rejected. Find the probability that a box containing 15 oranges out of which 12 are good and 3 are bad ones will be approved for sale.
		\label{ncert/12/13/2/3/defs.tex}
		\item Two balls are drawn at random with replacement from a box containing 10 black and 8 red balls. Find the probability that
		\label{ncert/12/13/2/12}
\begin{enumerate}
\item both balls are red.
\item first ball is black and second is red.
\item one of them is black and other is red.
\end{enumerate}

\item In a hostel, 60\% of the students read Hindi newspaper, 40\% read English newspaper and 20\% read both Hindi and English newspapers. A student is selected at random.
		\label{ncert/12/13/2/15}
\begin{enumerate}
\item Find the probability that she reads neither Hindi nor English newspapers.
\item If she reads Hindi newspaper, find the probability that she reads English newspaper.
\item If she reads English newspaper, find the probability that she reads Hindi newspaper.\\
\end{enumerate}
\item The probability of obtaining an even prime number on each die, when a pair of dice is rolled is 
\begin{enumerate}
    \item $0$ 
    
    \item $\frac{1}{3}$ 
    
    \item $\frac{1}{12}$ 
    
    \item $\frac{1}{36}$ 
\end{enumerate}
\solution
		%\begin{enumerate}[label=\thesection.\arabic*,ref=\thesection.\theenumi]
	\item One card is drawn from a well-shuffled deck of 52 cards. Find the probability of getting
\begin{enumerate}
\item A king of red colour 
\item A face card 
\item A red face card
\item The jack of hearts
\item A spade
\item The queen of diamonds

\end{enumerate}
\solution
		%\input{ncert/10/15/1/14/main.tex}
	\item Five cards—the ten, jack, queen, king and ace of diamonds, are well-shuffled with their face downwards. One card is then picked up at random.
\begin{enumerate}
\item
What is the probability that the card is the queen? 
\item
If the queen is drawn and put aside, what is the probability that the second card picked up is (a) an ace? (b) a queen?\\
\end{enumerate}
\solution
		%\input{ncert/10/15/1/15/defs.tex}
	\item A bag contains $5$ red balls and some blue balls. If the probability of drawing a blue ball is double that if a red ball, determine the number of blue balls in the bag. 
		\\
\solution
		%\input{ncert/10/15/2/3/defs.tex}
	\item A card is selected from a pack of 52 cards.
 \begin{enumerate}[label=(\alph*)] 
                 \item How many points are there in the sample space?
                 \item Calculate the probability that the card is an ace of spades.
                 \item Calculate the probability that the card is (i) an ace and (ii) black card.
 \end{enumerate}
\solution
		%\input{ncert/11/16/3/4/main.tex}
\item Four cards are drawn from a well-shuffled deck of 52 cards. What is the probability of obtaining 3 diamonds and one spade.
\\
\solution
		%\input{ncert/11/16/4/2/defs.tex}
\item In a certain lottery 10,000 tickets are sold and ten equal prizes are awarded. What is the probability of not getting a prize if you buy (a) one ticket (b) two tickets (c) 10 tickets ?	
\\
\solution
		%\input{ncert/11/16/4/4/defs.tex}
		%
\item 
Out of 100 students, two sections of 40 and 60 are formed. If you and your friend are among the 100 students, what is the probability that
\begin{enumerate}
\item you both enter the same section?
\item you both enter the different sections?
\end{enumerate}
\solution
		%\input{ncert/11/16/4/5/defs.tex}
	\item 
The number lock of a suitcase has 4 wheels each labelled with ten digits i.e. from 0 to 9.The lock opens with a sequence of four digits with no repeats.What is the probability of a person getting the right sequence to open the suitcase.
\\
\solution
		%\input{ncert/11/16/4/10/defs.tex}
		%
\item 
Two cards are drawn at random and without replacement from a pack of 52 playing cards. Find the probability that both the cards are black.
\\
\solution
		%\input{ncert/12/13/2/2/defs.tex}
		\item A box of oranges is inspected by examining three randomly selected oranges drawn without replacement. If all the three oranges are good, the box is approved for sale, otherwise, it is rejected. Find the probability that a box containing 15 oranges out of which 12 are good and 3 are bad ones will be approved for sale.
		\label{ncert/12/13/2/3/defs.tex}
		\item Two balls are drawn at random with replacement from a box containing 10 black and 8 red balls. Find the probability that
		\label{ncert/12/13/2/12}
\begin{enumerate}
\item both balls are red.
\item first ball is black and second is red.
\item one of them is black and other is red.
\end{enumerate}

\item In a hostel, 60\% of the students read Hindi newspaper, 40\% read English newspaper and 20\% read both Hindi and English newspapers. A student is selected at random.
		\label{ncert/12/13/2/15}
\begin{enumerate}
\item Find the probability that she reads neither Hindi nor English newspapers.
\item If she reads Hindi newspaper, find the probability that she reads English newspaper.
\item If she reads English newspaper, find the probability that she reads Hindi newspaper.\\
\end{enumerate}
\item The probability of obtaining an even prime number on each die, when a pair of dice is rolled is 
\begin{enumerate}
    \item $0$ 
    
    \item $\frac{1}{3}$ 
    
    \item $\frac{1}{12}$ 
    
    \item $\frac{1}{36}$ 
\end{enumerate}
\solution
		%\input{ncert/12/13/2/17/defs.tex}
	\item A bag contains 4 red and 4 black balls, another bag contains 2 red and 6 black balls. One of the two bags is selected at random and a ball is drawn from the bag which is found to be red. Find the probability that the ball is drawn from the first bag.
\\
\solution
		%\input{ncert/12/13/3/2/main.tex}
  \item
  Cards with numbers 2 to 101 are placed in a box. A card is selected at random.Find the probability that the card has
\begin{enumerate}[label=(\roman*)]
	\item an even number 
	\item a square number
\end{enumerate}
\solution
%\input{exemplar/10/13/3/32/main.tex}
\item
The king, queen and jack of clubs are removed from a deck of 52 playing cards and then well shuffled. Now one card is drawn at random from the remaining cards.  Determine the probability that the card is
\begin{enumerate}[label=(\roman*)]
\item a club
\item 10 of hearts
\end{enumerate}
\solution
%\input{exemplar/10/13/3/29/main.tex}
\item A team of medical students doing their internship have to assist during surgeries
at a city hospital. The probabilities of surgeries rated as very complex, complex,
routine, simple or very simple are respectively, 0.15, 0.20, 0.31, 0.26, .08. Find
the probabilities that a particular surgery will be rated
\begin{enumerate}
	\item complex or very complex;
	\item neither very complex nor very simple;
	\item routine or complex
	\item routine or simple
\end{enumerate}
\solution
%\input{exemplar/11/16/3/8(1)/main.tex}
\item A card is selected from a pack of 52 cards.
\begin{enumerate}[label=(\alph*)]
    \item How many points are there in the sample space?
    \item Calculate the probability that the card is an ace of spades.
    \item Calculate the probability that the card is (i) an ace and (ii) black card.
\end{enumerate}
\solution
%\input{exemplar/11/16/3/4/main2.tex}
\item The probability that a non leap year selected at random will contain 53 sundays.
\\
\solution
%\input{exemplar/10/13/1/19/main.tex}
\item One of the four persons John, Rita, Aslam or Gurpreet will be promoted next
month. Consequently the sample space consists of four elementary outcomes
S = {John promoted, Rita promoted, Aslam promoted, Gurpreet promoted}
You are told that the chances of John’s promotion is same as that of Gurpreet,
Rita’s chances of promotion are twice as likely as Johns. Aslam’s chances are
four times that of John.
\begin{enumerate}
	\item Determine
	\begin{enumerate}
		\item P (John promoted)
		\item P (Rita promoted)
		\item P (Aslam promoted)
		\item P (Gurpreet promoted)
	\end{enumerate}
	\item If A = {John promoted or Gurpreet promoted}, find P (A).
\end{enumerate}
\solution
%\input{exemplar/11/16/3/10/main.tex}
\item A card is drawn from a deck of 52 cards. Find the probability of getting a king or a heart or a red card.\\
\solution
%\input{exemplar/11/16/3/15/main.tex}
\item The probability that a student will pass his examination is 0.73, the probability of
the student getting a compartment is 0.13, and the probability that the student will
either pass or get compartment is 0.96. State True or False.\\
\solution
%\input{exemplar/11/16/3/31/main.tex}
\item A card is selected from a pack of 52 cards\\
\begin{enumerate}[label=(\alph*)]
\item How many points are there in the sample space?
\item Calculate the probability that the cards is an ace of spades.
\item Calculate the probability that the card is (i) an ace (ii)black card.\\
\end{enumerate}
%\input{ncert/11/16/3/4_1/Prob_4.tex}
\item In a non-leap year, the probability of having 53 tuesdays or 53 wednesdays is\\
\solution
%\input{exemplar/11/16/3/18/main.tex}
\item There are 1000 sealed envelopes in a box, 10 of them contain a cash prize of
Rs 100 each, 100 of them contain a cash prize of Rs 50 each and 200 of them
contain a cash prize of Rs 10 each and rest do not contain any cash prize. If they
are well shuffled and an envelope is picked up out, what is the probability that it
contains no cash prize?\\
\solution
%\input{exemplar/10/13/3/34/main.tex}
\item 
A die is thrown and a card is selected at random from a deck of 52 playing cards. The probability of getting an even number on the die and a spade card.\\
\solution
%\input{exemplar/12/13/3/78/main.tex}
\item
If 4-digit numbers greater than 5,000 are randomly formed from the digits 0, 1, 3, 5, and 7, what is the probability of forming a number divisible by 5 when:
\begin{enumerate}
    \item The digits are repeated?
    \item The repetition of digits is not allowed?
\end{enumerate}
\solution
%\input{ncert/11/16/4/9/main.tex}
\item Consider the probability space $\brak{\Omega, \mathcal{G}, P}$ where $\Omega = [0,2]$ and $\mathcal{G} = \cbrak{\phi, \Omega, [0,1], (1,2]}$. Let $X$ and $Y$ be two functions on $\Omega$ defined as
\begin{align*}
    X(\omega) = 
    \begin{cases}
        1 & \text{if }\omega \in [0, 1]\\
        2 & \text{if }\omega \in (1, 2]
    \end{cases}
\end{align*}
and
\begin{align*}
    Y(\omega) = 
    \begin{cases}
        2 & \text{if }\omega \in [0, 1.5]\\
        3 & \text{if }\omega \in (1.5, 2].
    \end{cases}
\end{align*}
Then which one of the following statements is true?
\begin{enumerate}
    \item [(A)] $X$ is a random variable with respect to $\mathcal{G}$, but $Y$ is not a random variable with respect to $\mathcal{G}$.
    \item [(B)] $Y$ is a random variable with respect to $\mathcal{G}$, but $X$ is not a random variable with respect to $\mathcal{G}$.
    \item [(C)] Neither $X$ nor $Y$ is a random variable with respect to $\mathcal{G}$.
    \item [(D)] Both $X$ and $Y$ are random variables with respect to $\mathcal{G}$.
\end{enumerate} \hfill (GATE ST 2023)\\
\solution
%\input{gate/ST/2023/14/main.tex}
	\item  A die is loaded in such a way that each odd number is twice as likely to occur as
each even number. Find $P(G)$, where $G$ is the event that a number greater than
3 occurs on a single roll of the die.
\\
\solution
		%\input{exemplar/11/16/3/5/main.tex}
	\item All the jacks, queens and kings are removed from a deck of 52 playing cards. The remaining cards are well shuffled and then one card is drawn at random. Giving ace a value 1 similar value for other cards, find the probability that the card has a value 
		\begin{enumerate}
			\item 7
			\item greater than 7
			\item less than 7
		\end{enumerate}
		%\input{exemplar/10/13/3/30/main.tex}
  \item A Lot consists of 48 mobile phones of which 42 are good, 3 have only minor defects and 3 have major defects.Varnika will buy a phone if it is good but the trader will only buy a mobile if it has no major defects. One phone is selected at random from the lot. What is the probability that it is
\begin{enumerate}
	\item acceptable to Varnika?
            \item acceptable to the trader?
\end{enumerate}
\solution
	%\input{exemplar/10/13/3/40/main.tex}
 \item A student says that if you throw a die, it will show up 1 or not 1. Therefore, the probability of getting 1 and the probability of getting 'not 1' each is equal to $\frac{1}{2}$. Is this correct? Give reasons.\\
 \solution
        %\input{exemplar/10/13/2/9/main.tex}
   \item Four candidates A, B, C, D have ap-
plied for the assignment to coach a school cricket
team. If A is twice as likely to be selected as B, and
B and C are given about the same chance of being
selected, while C is twice as likely to be selected
as D, what are the probabilities that
\begin{enumerate}
\item C will be selected?
\item A will not be selected?
\end{enumerate}
	%\input{exemplar/11/16/3/9/main.tex}
 \item A bag contain 24 balls of which $x$ balls are red, $2x$ are white and $3x$ are blue. A ball is selected at random, What is the probability that it is
\begin{enumerate}[label=\alph*)]
\item not red ?
\item white ?
\end{enumerate}
%\input{exemplar/10/13/3/41/main.tex}
If the letters of the word ASSASSINATION are arranged at random. Find the Probability that
\begin{enumerate}[label=(\alph*)]
\item Four $S's$ come consecutively in the word
\item Two  $I's$ and two $N's$ come together
\item All $A's$ are not coming together
\item No two $A's$ are coming together
\end{enumerate}
%\input{exemplar/11/16/3/14/main.tex}
	\item One urn contains two black balls (labelled B1 and B2) and one white ball. A
	second urn contains one black ball and two white balls (labelled W1 and W2).
	Suppose the following experiment is performed. One of the two urns is chosen
	at random. Next a ball is randomly chosen from the urn. Then a second ball is
	chosen at random from the same urn without replacing the first ball.
	
	\begin{enumerate}
	\item What is the probability that two black balls are chosen?
	
	\item What is the probability that two balls of opposite colour are chosen?
	\end{enumerate}
	\solution
	%\input{exemplar/11/16/3/12/main1.tex}
\end{enumerate}

	\item A bag contains 4 red and 4 black balls, another bag contains 2 red and 6 black balls. One of the two bags is selected at random and a ball is drawn from the bag which is found to be red. Find the probability that the ball is drawn from the first bag.
\\
\solution
		%\begin{table}[H]
	\centering
\begin{tabular}{|c|c|c|}
\hline
Random variable &Value &Definition\\ \hline
\multirow{3}{*}{X} &0 &Slips of Rs 1\\
&1 &Slips of Rs 5\\
&2 &Slips of Rs 13\\ \hline
\multirow{2}{*}{Y} &0 &Box A\\
&1 &Box B\\\hline
\end{tabular}
\caption{}
\label{tab:Distribution}
\end{table}
See \tabref{tab:Distribution}.
\begin{align}
p_{Y}\brak{k}= \begin{cases} 
      \frac{1}{3} & {k=0} \\
      \frac{2}{3 }& {k=1} 
   \end{cases}
   \\
p_{Y|X}\brak{0|0} = \frac{19}{25}\, 
p_{Y|X}\brak{0|1} = \frac{6}{25}\,
p_{Y|X}\brak{1|0} = \frac{45}{50}\,
p_{Y|X}\brak{1|2} = \frac{5}{50}
\end{align}
The desired probability is the probability that a slip drawn at random is marked other than Rs 1,
\begin{align}
&=1-p_X\brak{0}\\
&= p_X(1) + p_X(2)
\end{align}
Using Bayes theorem,
\begin{align}
&= p_Y\brak{0} \times \pr{Y=0 | X=1} + p_Y\brak{1} \times \pr{Y=1|X=2}\\
&=\frac{1}{3} \times \frac{6}{25} + \frac{2}{3} \times \frac{5}{50}\\
&=\frac{11}{75}
\end{align}

\newpage

%\tableofcontents

\bigskip

\renewcommand{\thefigure}{\theenumi}
\renewcommand{\thetable}{\theenumi}
%\renewcommand{\theequation}{\theenumi}

%\begin{abstract}
%%\boldmath
%In this letter, an algorithm for evaluating the exact analytical bit error rate  (BER)  for the piecewise linear (PL) combiner for  multiple relays is presented. Previous results were available only for upto three relays. The algorithm is unique in the sense that  the actual mathematical expressions, that are prohibitively large, need not be explicitly obtained. The diversity gain due to multiple relays is shown through plots of the analytical BER, well supported by simulations. 
%
%\end{abstract}
% IEEEtran.cls defaults to using nonbold math in the Abstract.
% This preserves the distinction between vectors and scalars. However,
% if the journal you are submitting to favors bold math in the abstract,
% then you can use LaTeX's standard command \boldmath at the very start
% of the abstract to achieve this. Many IEEE journals frown on math
% in the abstract anyway.

% Note that keywords are not normally used for peerreview papers.
%\begin{IEEEkeywords}
%Cooperative diversity, decode and forward, piecewise linear
%\end{IEEEkeywords}



% For peer review papers, you can put extra information on the cover
% page as needed:
% \ifCLASSOPTIONpeerreview
% \begin{center} \bfseries EDICS Category: 3-BBND \end{center}
% \fi
%
% For peerreview papers, this IEEEtran command inserts a page break and
% creates the second title. It will be ignored for other modes.
%\IEEEpeerreviewmaketitle




  \item
  Cards with numbers 2 to 101 are placed in a box. A card is selected at random.Find the probability that the card has
\begin{enumerate}[label=(\roman*)]
	\item an even number 
	\item a square number
\end{enumerate}
\solution
%\begin{table}[H]
	\centering
\begin{tabular}{|c|c|c|}
\hline
Random variable &Value &Definition\\ \hline
\multirow{3}{*}{X} &0 &Slips of Rs 1\\
&1 &Slips of Rs 5\\
&2 &Slips of Rs 13\\ \hline
\multirow{2}{*}{Y} &0 &Box A\\
&1 &Box B\\\hline
\end{tabular}
\caption{}
\label{tab:Distribution}
\end{table}
See \tabref{tab:Distribution}.
\begin{align}
p_{Y}\brak{k}= \begin{cases} 
      \frac{1}{3} & {k=0} \\
      \frac{2}{3 }& {k=1} 
   \end{cases}
   \\
p_{Y|X}\brak{0|0} = \frac{19}{25}\, 
p_{Y|X}\brak{0|1} = \frac{6}{25}\,
p_{Y|X}\brak{1|0} = \frac{45}{50}\,
p_{Y|X}\brak{1|2} = \frac{5}{50}
\end{align}
The desired probability is the probability that a slip drawn at random is marked other than Rs 1,
\begin{align}
&=1-p_X\brak{0}\\
&= p_X(1) + p_X(2)
\end{align}
Using Bayes theorem,
\begin{align}
&= p_Y\brak{0} \times \pr{Y=0 | X=1} + p_Y\brak{1} \times \pr{Y=1|X=2}\\
&=\frac{1}{3} \times \frac{6}{25} + \frac{2}{3} \times \frac{5}{50}\\
&=\frac{11}{75}
\end{align}

\newpage

%\tableofcontents

\bigskip

\renewcommand{\thefigure}{\theenumi}
\renewcommand{\thetable}{\theenumi}
%\renewcommand{\theequation}{\theenumi}

%\begin{abstract}
%%\boldmath
%In this letter, an algorithm for evaluating the exact analytical bit error rate  (BER)  for the piecewise linear (PL) combiner for  multiple relays is presented. Previous results were available only for upto three relays. The algorithm is unique in the sense that  the actual mathematical expressions, that are prohibitively large, need not be explicitly obtained. The diversity gain due to multiple relays is shown through plots of the analytical BER, well supported by simulations. 
%
%\end{abstract}
% IEEEtran.cls defaults to using nonbold math in the Abstract.
% This preserves the distinction between vectors and scalars. However,
% if the journal you are submitting to favors bold math in the abstract,
% then you can use LaTeX's standard command \boldmath at the very start
% of the abstract to achieve this. Many IEEE journals frown on math
% in the abstract anyway.

% Note that keywords are not normally used for peerreview papers.
%\begin{IEEEkeywords}
%Cooperative diversity, decode and forward, piecewise linear
%\end{IEEEkeywords}



% For peer review papers, you can put extra information on the cover
% page as needed:
% \ifCLASSOPTIONpeerreview
% \begin{center} \bfseries EDICS Category: 3-BBND \end{center}
% \fi
%
% For peerreview papers, this IEEEtran command inserts a page break and
% creates the second title. It will be ignored for other modes.
%\IEEEpeerreviewmaketitle




\item
The king, queen and jack of clubs are removed from a deck of 52 playing cards and then well shuffled. Now one card is drawn at random from the remaining cards.  Determine the probability that the card is
\begin{enumerate}[label=(\roman*)]
\item a club
\item 10 of hearts
\end{enumerate}
\solution
%\begin{table}[H]
	\centering
\begin{tabular}{|c|c|c|}
\hline
Random variable &Value &Definition\\ \hline
\multirow{3}{*}{X} &0 &Slips of Rs 1\\
&1 &Slips of Rs 5\\
&2 &Slips of Rs 13\\ \hline
\multirow{2}{*}{Y} &0 &Box A\\
&1 &Box B\\\hline
\end{tabular}
\caption{}
\label{tab:Distribution}
\end{table}
See \tabref{tab:Distribution}.
\begin{align}
p_{Y}\brak{k}= \begin{cases} 
      \frac{1}{3} & {k=0} \\
      \frac{2}{3 }& {k=1} 
   \end{cases}
   \\
p_{Y|X}\brak{0|0} = \frac{19}{25}\, 
p_{Y|X}\brak{0|1} = \frac{6}{25}\,
p_{Y|X}\brak{1|0} = \frac{45}{50}\,
p_{Y|X}\brak{1|2} = \frac{5}{50}
\end{align}
The desired probability is the probability that a slip drawn at random is marked other than Rs 1,
\begin{align}
&=1-p_X\brak{0}\\
&= p_X(1) + p_X(2)
\end{align}
Using Bayes theorem,
\begin{align}
&= p_Y\brak{0} \times \pr{Y=0 | X=1} + p_Y\brak{1} \times \pr{Y=1|X=2}\\
&=\frac{1}{3} \times \frac{6}{25} + \frac{2}{3} \times \frac{5}{50}\\
&=\frac{11}{75}
\end{align}

\newpage

%\tableofcontents

\bigskip

\renewcommand{\thefigure}{\theenumi}
\renewcommand{\thetable}{\theenumi}
%\renewcommand{\theequation}{\theenumi}

%\begin{abstract}
%%\boldmath
%In this letter, an algorithm for evaluating the exact analytical bit error rate  (BER)  for the piecewise linear (PL) combiner for  multiple relays is presented. Previous results were available only for upto three relays. The algorithm is unique in the sense that  the actual mathematical expressions, that are prohibitively large, need not be explicitly obtained. The diversity gain due to multiple relays is shown through plots of the analytical BER, well supported by simulations. 
%
%\end{abstract}
% IEEEtran.cls defaults to using nonbold math in the Abstract.
% This preserves the distinction between vectors and scalars. However,
% if the journal you are submitting to favors bold math in the abstract,
% then you can use LaTeX's standard command \boldmath at the very start
% of the abstract to achieve this. Many IEEE journals frown on math
% in the abstract anyway.

% Note that keywords are not normally used for peerreview papers.
%\begin{IEEEkeywords}
%Cooperative diversity, decode and forward, piecewise linear
%\end{IEEEkeywords}



% For peer review papers, you can put extra information on the cover
% page as needed:
% \ifCLASSOPTIONpeerreview
% \begin{center} \bfseries EDICS Category: 3-BBND \end{center}
% \fi
%
% For peerreview papers, this IEEEtran command inserts a page break and
% creates the second title. It will be ignored for other modes.
%\IEEEpeerreviewmaketitle




\item A team of medical students doing their internship have to assist during surgeries
at a city hospital. The probabilities of surgeries rated as very complex, complex,
routine, simple or very simple are respectively, 0.15, 0.20, 0.31, 0.26, .08. Find
the probabilities that a particular surgery will be rated
\begin{enumerate}
	\item complex or very complex;
	\item neither very complex nor very simple;
	\item routine or complex
	\item routine or simple
\end{enumerate}
\solution
%\begin{table}[H]
	\centering
\begin{tabular}{|c|c|c|}
\hline
Random variable &Value &Definition\\ \hline
\multirow{3}{*}{X} &0 &Slips of Rs 1\\
&1 &Slips of Rs 5\\
&2 &Slips of Rs 13\\ \hline
\multirow{2}{*}{Y} &0 &Box A\\
&1 &Box B\\\hline
\end{tabular}
\caption{}
\label{tab:Distribution}
\end{table}
See \tabref{tab:Distribution}.
\begin{align}
p_{Y}\brak{k}= \begin{cases} 
      \frac{1}{3} & {k=0} \\
      \frac{2}{3 }& {k=1} 
   \end{cases}
   \\
p_{Y|X}\brak{0|0} = \frac{19}{25}\, 
p_{Y|X}\brak{0|1} = \frac{6}{25}\,
p_{Y|X}\brak{1|0} = \frac{45}{50}\,
p_{Y|X}\brak{1|2} = \frac{5}{50}
\end{align}
The desired probability is the probability that a slip drawn at random is marked other than Rs 1,
\begin{align}
&=1-p_X\brak{0}\\
&= p_X(1) + p_X(2)
\end{align}
Using Bayes theorem,
\begin{align}
&= p_Y\brak{0} \times \pr{Y=0 | X=1} + p_Y\brak{1} \times \pr{Y=1|X=2}\\
&=\frac{1}{3} \times \frac{6}{25} + \frac{2}{3} \times \frac{5}{50}\\
&=\frac{11}{75}
\end{align}

\newpage

%\tableofcontents

\bigskip

\renewcommand{\thefigure}{\theenumi}
\renewcommand{\thetable}{\theenumi}
%\renewcommand{\theequation}{\theenumi}

%\begin{abstract}
%%\boldmath
%In this letter, an algorithm for evaluating the exact analytical bit error rate  (BER)  for the piecewise linear (PL) combiner for  multiple relays is presented. Previous results were available only for upto three relays. The algorithm is unique in the sense that  the actual mathematical expressions, that are prohibitively large, need not be explicitly obtained. The diversity gain due to multiple relays is shown through plots of the analytical BER, well supported by simulations. 
%
%\end{abstract}
% IEEEtran.cls defaults to using nonbold math in the Abstract.
% This preserves the distinction between vectors and scalars. However,
% if the journal you are submitting to favors bold math in the abstract,
% then you can use LaTeX's standard command \boldmath at the very start
% of the abstract to achieve this. Many IEEE journals frown on math
% in the abstract anyway.

% Note that keywords are not normally used for peerreview papers.
%\begin{IEEEkeywords}
%Cooperative diversity, decode and forward, piecewise linear
%\end{IEEEkeywords}



% For peer review papers, you can put extra information on the cover
% page as needed:
% \ifCLASSOPTIONpeerreview
% \begin{center} \bfseries EDICS Category: 3-BBND \end{center}
% \fi
%
% For peerreview papers, this IEEEtran command inserts a page break and
% creates the second title. It will be ignored for other modes.
%\IEEEpeerreviewmaketitle




\item A card is selected from a pack of 52 cards.
\begin{enumerate}[label=(\alph*)]
    \item How many points are there in the sample space?
    \item Calculate the probability that the card is an ace of spades.
    \item Calculate the probability that the card is (i) an ace and (ii) black card.
\end{enumerate}
\solution
%Let $X$ be an bernoulli rv defined as in \tabref{tab:exemplar/11/16/3/26}.  Then, 
\begin{equation}
    p =
        \frac{4}{11} 
\end{equation}
\begin{table}[H]
	\centering
	\input{exemplar/11/16/3/26/tables/Table2.tex}
	\caption{}
        \label{tab:exemplar/11/16/3/26}
\end{table}

\item The probability that a non leap year selected at random will contain 53 sundays.
\\
\solution
%\begin{table}[H]
	\centering
\begin{tabular}{|c|c|c|}
\hline
Random variable &Value &Definition\\ \hline
\multirow{3}{*}{X} &0 &Slips of Rs 1\\
&1 &Slips of Rs 5\\
&2 &Slips of Rs 13\\ \hline
\multirow{2}{*}{Y} &0 &Box A\\
&1 &Box B\\\hline
\end{tabular}
\caption{}
\label{tab:Distribution}
\end{table}
See \tabref{tab:Distribution}.
\begin{align}
p_{Y}\brak{k}= \begin{cases} 
      \frac{1}{3} & {k=0} \\
      \frac{2}{3 }& {k=1} 
   \end{cases}
   \\
p_{Y|X}\brak{0|0} = \frac{19}{25}\, 
p_{Y|X}\brak{0|1} = \frac{6}{25}\,
p_{Y|X}\brak{1|0} = \frac{45}{50}\,
p_{Y|X}\brak{1|2} = \frac{5}{50}
\end{align}
The desired probability is the probability that a slip drawn at random is marked other than Rs 1,
\begin{align}
&=1-p_X\brak{0}\\
&= p_X(1) + p_X(2)
\end{align}
Using Bayes theorem,
\begin{align}
&= p_Y\brak{0} \times \pr{Y=0 | X=1} + p_Y\brak{1} \times \pr{Y=1|X=2}\\
&=\frac{1}{3} \times \frac{6}{25} + \frac{2}{3} \times \frac{5}{50}\\
&=\frac{11}{75}
\end{align}

\newpage

%\tableofcontents

\bigskip

\renewcommand{\thefigure}{\theenumi}
\renewcommand{\thetable}{\theenumi}
%\renewcommand{\theequation}{\theenumi}

%\begin{abstract}
%%\boldmath
%In this letter, an algorithm for evaluating the exact analytical bit error rate  (BER)  for the piecewise linear (PL) combiner for  multiple relays is presented. Previous results were available only for upto three relays. The algorithm is unique in the sense that  the actual mathematical expressions, that are prohibitively large, need not be explicitly obtained. The diversity gain due to multiple relays is shown through plots of the analytical BER, well supported by simulations. 
%
%\end{abstract}
% IEEEtran.cls defaults to using nonbold math in the Abstract.
% This preserves the distinction between vectors and scalars. However,
% if the journal you are submitting to favors bold math in the abstract,
% then you can use LaTeX's standard command \boldmath at the very start
% of the abstract to achieve this. Many IEEE journals frown on math
% in the abstract anyway.

% Note that keywords are not normally used for peerreview papers.
%\begin{IEEEkeywords}
%Cooperative diversity, decode and forward, piecewise linear
%\end{IEEEkeywords}



% For peer review papers, you can put extra information on the cover
% page as needed:
% \ifCLASSOPTIONpeerreview
% \begin{center} \bfseries EDICS Category: 3-BBND \end{center}
% \fi
%
% For peerreview papers, this IEEEtran command inserts a page break and
% creates the second title. It will be ignored for other modes.
%\IEEEpeerreviewmaketitle




\item One of the four persons John, Rita, Aslam or Gurpreet will be promoted next
month. Consequently the sample space consists of four elementary outcomes
S = {John promoted, Rita promoted, Aslam promoted, Gurpreet promoted}
You are told that the chances of John’s promotion is same as that of Gurpreet,
Rita’s chances of promotion are twice as likely as Johns. Aslam’s chances are
four times that of John.
\begin{enumerate}
	\item Determine
	\begin{enumerate}
		\item P (John promoted)
		\item P (Rita promoted)
		\item P (Aslam promoted)
		\item P (Gurpreet promoted)
	\end{enumerate}
	\item If A = {John promoted or Gurpreet promoted}, find P (A).
\end{enumerate}
\solution
%\begin{table}[H]
	\centering
\begin{tabular}{|c|c|c|}
\hline
Random variable &Value &Definition\\ \hline
\multirow{3}{*}{X} &0 &Slips of Rs 1\\
&1 &Slips of Rs 5\\
&2 &Slips of Rs 13\\ \hline
\multirow{2}{*}{Y} &0 &Box A\\
&1 &Box B\\\hline
\end{tabular}
\caption{}
\label{tab:Distribution}
\end{table}
See \tabref{tab:Distribution}.
\begin{align}
p_{Y}\brak{k}= \begin{cases} 
      \frac{1}{3} & {k=0} \\
      \frac{2}{3 }& {k=1} 
   \end{cases}
   \\
p_{Y|X}\brak{0|0} = \frac{19}{25}\, 
p_{Y|X}\brak{0|1} = \frac{6}{25}\,
p_{Y|X}\brak{1|0} = \frac{45}{50}\,
p_{Y|X}\brak{1|2} = \frac{5}{50}
\end{align}
The desired probability is the probability that a slip drawn at random is marked other than Rs 1,
\begin{align}
&=1-p_X\brak{0}\\
&= p_X(1) + p_X(2)
\end{align}
Using Bayes theorem,
\begin{align}
&= p_Y\brak{0} \times \pr{Y=0 | X=1} + p_Y\brak{1} \times \pr{Y=1|X=2}\\
&=\frac{1}{3} \times \frac{6}{25} + \frac{2}{3} \times \frac{5}{50}\\
&=\frac{11}{75}
\end{align}

\newpage

%\tableofcontents

\bigskip

\renewcommand{\thefigure}{\theenumi}
\renewcommand{\thetable}{\theenumi}
%\renewcommand{\theequation}{\theenumi}

%\begin{abstract}
%%\boldmath
%In this letter, an algorithm for evaluating the exact analytical bit error rate  (BER)  for the piecewise linear (PL) combiner for  multiple relays is presented. Previous results were available only for upto three relays. The algorithm is unique in the sense that  the actual mathematical expressions, that are prohibitively large, need not be explicitly obtained. The diversity gain due to multiple relays is shown through plots of the analytical BER, well supported by simulations. 
%
%\end{abstract}
% IEEEtran.cls defaults to using nonbold math in the Abstract.
% This preserves the distinction between vectors and scalars. However,
% if the journal you are submitting to favors bold math in the abstract,
% then you can use LaTeX's standard command \boldmath at the very start
% of the abstract to achieve this. Many IEEE journals frown on math
% in the abstract anyway.

% Note that keywords are not normally used for peerreview papers.
%\begin{IEEEkeywords}
%Cooperative diversity, decode and forward, piecewise linear
%\end{IEEEkeywords}



% For peer review papers, you can put extra information on the cover
% page as needed:
% \ifCLASSOPTIONpeerreview
% \begin{center} \bfseries EDICS Category: 3-BBND \end{center}
% \fi
%
% For peerreview papers, this IEEEtran command inserts a page break and
% creates the second title. It will be ignored for other modes.
%\IEEEpeerreviewmaketitle




\item A card is drawn from a deck of 52 cards. Find the probability of getting a king or a heart or a red card.\\
\solution
%\begin{table}[H]
	\centering
\begin{tabular}{|c|c|c|}
\hline
Random variable &Value &Definition\\ \hline
\multirow{3}{*}{X} &0 &Slips of Rs 1\\
&1 &Slips of Rs 5\\
&2 &Slips of Rs 13\\ \hline
\multirow{2}{*}{Y} &0 &Box A\\
&1 &Box B\\\hline
\end{tabular}
\caption{}
\label{tab:Distribution}
\end{table}
See \tabref{tab:Distribution}.
\begin{align}
p_{Y}\brak{k}= \begin{cases} 
      \frac{1}{3} & {k=0} \\
      \frac{2}{3 }& {k=1} 
   \end{cases}
   \\
p_{Y|X}\brak{0|0} = \frac{19}{25}\, 
p_{Y|X}\brak{0|1} = \frac{6}{25}\,
p_{Y|X}\brak{1|0} = \frac{45}{50}\,
p_{Y|X}\brak{1|2} = \frac{5}{50}
\end{align}
The desired probability is the probability that a slip drawn at random is marked other than Rs 1,
\begin{align}
&=1-p_X\brak{0}\\
&= p_X(1) + p_X(2)
\end{align}
Using Bayes theorem,
\begin{align}
&= p_Y\brak{0} \times \pr{Y=0 | X=1} + p_Y\brak{1} \times \pr{Y=1|X=2}\\
&=\frac{1}{3} \times \frac{6}{25} + \frac{2}{3} \times \frac{5}{50}\\
&=\frac{11}{75}
\end{align}

\newpage

%\tableofcontents

\bigskip

\renewcommand{\thefigure}{\theenumi}
\renewcommand{\thetable}{\theenumi}
%\renewcommand{\theequation}{\theenumi}

%\begin{abstract}
%%\boldmath
%In this letter, an algorithm for evaluating the exact analytical bit error rate  (BER)  for the piecewise linear (PL) combiner for  multiple relays is presented. Previous results were available only for upto three relays. The algorithm is unique in the sense that  the actual mathematical expressions, that are prohibitively large, need not be explicitly obtained. The diversity gain due to multiple relays is shown through plots of the analytical BER, well supported by simulations. 
%
%\end{abstract}
% IEEEtran.cls defaults to using nonbold math in the Abstract.
% This preserves the distinction between vectors and scalars. However,
% if the journal you are submitting to favors bold math in the abstract,
% then you can use LaTeX's standard command \boldmath at the very start
% of the abstract to achieve this. Many IEEE journals frown on math
% in the abstract anyway.

% Note that keywords are not normally used for peerreview papers.
%\begin{IEEEkeywords}
%Cooperative diversity, decode and forward, piecewise linear
%\end{IEEEkeywords}



% For peer review papers, you can put extra information on the cover
% page as needed:
% \ifCLASSOPTIONpeerreview
% \begin{center} \bfseries EDICS Category: 3-BBND \end{center}
% \fi
%
% For peerreview papers, this IEEEtran command inserts a page break and
% creates the second title. It will be ignored for other modes.
%\IEEEpeerreviewmaketitle




\item The probability that a student will pass his examination is 0.73, the probability of
the student getting a compartment is 0.13, and the probability that the student will
either pass or get compartment is 0.96. State True or False.\\
\solution
%\begin{table}[H]
	\centering
\begin{tabular}{|c|c|c|}
\hline
Random variable &Value &Definition\\ \hline
\multirow{3}{*}{X} &0 &Slips of Rs 1\\
&1 &Slips of Rs 5\\
&2 &Slips of Rs 13\\ \hline
\multirow{2}{*}{Y} &0 &Box A\\
&1 &Box B\\\hline
\end{tabular}
\caption{}
\label{tab:Distribution}
\end{table}
See \tabref{tab:Distribution}.
\begin{align}
p_{Y}\brak{k}= \begin{cases} 
      \frac{1}{3} & {k=0} \\
      \frac{2}{3 }& {k=1} 
   \end{cases}
   \\
p_{Y|X}\brak{0|0} = \frac{19}{25}\, 
p_{Y|X}\brak{0|1} = \frac{6}{25}\,
p_{Y|X}\brak{1|0} = \frac{45}{50}\,
p_{Y|X}\brak{1|2} = \frac{5}{50}
\end{align}
The desired probability is the probability that a slip drawn at random is marked other than Rs 1,
\begin{align}
&=1-p_X\brak{0}\\
&= p_X(1) + p_X(2)
\end{align}
Using Bayes theorem,
\begin{align}
&= p_Y\brak{0} \times \pr{Y=0 | X=1} + p_Y\brak{1} \times \pr{Y=1|X=2}\\
&=\frac{1}{3} \times \frac{6}{25} + \frac{2}{3} \times \frac{5}{50}\\
&=\frac{11}{75}
\end{align}

\newpage

%\tableofcontents

\bigskip

\renewcommand{\thefigure}{\theenumi}
\renewcommand{\thetable}{\theenumi}
%\renewcommand{\theequation}{\theenumi}

%\begin{abstract}
%%\boldmath
%In this letter, an algorithm for evaluating the exact analytical bit error rate  (BER)  for the piecewise linear (PL) combiner for  multiple relays is presented. Previous results were available only for upto three relays. The algorithm is unique in the sense that  the actual mathematical expressions, that are prohibitively large, need not be explicitly obtained. The diversity gain due to multiple relays is shown through plots of the analytical BER, well supported by simulations. 
%
%\end{abstract}
% IEEEtran.cls defaults to using nonbold math in the Abstract.
% This preserves the distinction between vectors and scalars. However,
% if the journal you are submitting to favors bold math in the abstract,
% then you can use LaTeX's standard command \boldmath at the very start
% of the abstract to achieve this. Many IEEE journals frown on math
% in the abstract anyway.

% Note that keywords are not normally used for peerreview papers.
%\begin{IEEEkeywords}
%Cooperative diversity, decode and forward, piecewise linear
%\end{IEEEkeywords}



% For peer review papers, you can put extra information on the cover
% page as needed:
% \ifCLASSOPTIONpeerreview
% \begin{center} \bfseries EDICS Category: 3-BBND \end{center}
% \fi
%
% For peerreview papers, this IEEEtran command inserts a page break and
% creates the second title. It will be ignored for other modes.
%\IEEEpeerreviewmaketitle




\item A card is selected from a pack of 52 cards\\
\begin{enumerate}[label=(\alph*)]
\item How many points are there in the sample space?
\item Calculate the probability that the cards is an ace of spades.
\item Calculate the probability that the card is (i) an ace (ii)black card.\\
\end{enumerate}
%\input{ncert/11/16/3/4_1/Prob_4.tex}
\item In a non-leap year, the probability of having 53 tuesdays or 53 wednesdays is\\
\solution
%A non-leap year has a total of 365 days, and a week has 7 days.\\
So it can be expressed as 
\begin{align}
365\text{days} &=52\times 7+1 \text{day}
\end{align}
$\implies$ 52 tuesdays or wednesdays\\
Random variable X denotes the days of a week
\begin{align}
p_X\brak{k}&=\frac{1}{7}; \quad \brak{1<k<7}
\end{align}
So the probability of extra day being tuesday or wednesday is
\begin{align}
p_X\brak{3}+p_X\brak{4}&=\frac{1}{7}+\frac{1}{7}=\frac{2}{7}
\end{align}



\item There are 1000 sealed envelopes in a box, 10 of them contain a cash prize of
Rs 100 each, 100 of them contain a cash prize of Rs 50 each and 200 of them
contain a cash prize of Rs 10 each and rest do not contain any cash prize. If they
are well shuffled and an envelope is picked up out, what is the probability that it
contains no cash prize?\\
\solution
%\begin{table}[H]
	\centering
\begin{tabular}{|c|c|c|}
\hline
Random variable &Value &Definition\\ \hline
\multirow{3}{*}{X} &0 &Slips of Rs 1\\
&1 &Slips of Rs 5\\
&2 &Slips of Rs 13\\ \hline
\multirow{2}{*}{Y} &0 &Box A\\
&1 &Box B\\\hline
\end{tabular}
\caption{}
\label{tab:Distribution}
\end{table}
See \tabref{tab:Distribution}.
\begin{align}
p_{Y}\brak{k}= \begin{cases} 
      \frac{1}{3} & {k=0} \\
      \frac{2}{3 }& {k=1} 
   \end{cases}
   \\
p_{Y|X}\brak{0|0} = \frac{19}{25}\, 
p_{Y|X}\brak{0|1} = \frac{6}{25}\,
p_{Y|X}\brak{1|0} = \frac{45}{50}\,
p_{Y|X}\brak{1|2} = \frac{5}{50}
\end{align}
The desired probability is the probability that a slip drawn at random is marked other than Rs 1,
\begin{align}
&=1-p_X\brak{0}\\
&= p_X(1) + p_X(2)
\end{align}
Using Bayes theorem,
\begin{align}
&= p_Y\brak{0} \times \pr{Y=0 | X=1} + p_Y\brak{1} \times \pr{Y=1|X=2}\\
&=\frac{1}{3} \times \frac{6}{25} + \frac{2}{3} \times \frac{5}{50}\\
&=\frac{11}{75}
\end{align}

\newpage

%\tableofcontents

\bigskip

\renewcommand{\thefigure}{\theenumi}
\renewcommand{\thetable}{\theenumi}
%\renewcommand{\theequation}{\theenumi}

%\begin{abstract}
%%\boldmath
%In this letter, an algorithm for evaluating the exact analytical bit error rate  (BER)  for the piecewise linear (PL) combiner for  multiple relays is presented. Previous results were available only for upto three relays. The algorithm is unique in the sense that  the actual mathematical expressions, that are prohibitively large, need not be explicitly obtained. The diversity gain due to multiple relays is shown through plots of the analytical BER, well supported by simulations. 
%
%\end{abstract}
% IEEEtran.cls defaults to using nonbold math in the Abstract.
% This preserves the distinction between vectors and scalars. However,
% if the journal you are submitting to favors bold math in the abstract,
% then you can use LaTeX's standard command \boldmath at the very start
% of the abstract to achieve this. Many IEEE journals frown on math
% in the abstract anyway.

% Note that keywords are not normally used for peerreview papers.
%\begin{IEEEkeywords}
%Cooperative diversity, decode and forward, piecewise linear
%\end{IEEEkeywords}



% For peer review papers, you can put extra information on the cover
% page as needed:
% \ifCLASSOPTIONpeerreview
% \begin{center} \bfseries EDICS Category: 3-BBND \end{center}
% \fi
%
% For peerreview papers, this IEEEtran command inserts a page break and
% creates the second title. It will be ignored for other modes.
%\IEEEpeerreviewmaketitle




\item 
A die is thrown and a card is selected at random from a deck of 52 playing cards. The probability of getting an even number on the die and a spade card.\\
\solution
%\begin{table}[H]
	\centering
\begin{tabular}{|c|c|c|}
\hline
Random variable &Value &Definition\\ \hline
\multirow{3}{*}{X} &0 &Slips of Rs 1\\
&1 &Slips of Rs 5\\
&2 &Slips of Rs 13\\ \hline
\multirow{2}{*}{Y} &0 &Box A\\
&1 &Box B\\\hline
\end{tabular}
\caption{}
\label{tab:Distribution}
\end{table}
See \tabref{tab:Distribution}.
\begin{align}
p_{Y}\brak{k}= \begin{cases} 
      \frac{1}{3} & {k=0} \\
      \frac{2}{3 }& {k=1} 
   \end{cases}
   \\
p_{Y|X}\brak{0|0} = \frac{19}{25}\, 
p_{Y|X}\brak{0|1} = \frac{6}{25}\,
p_{Y|X}\brak{1|0} = \frac{45}{50}\,
p_{Y|X}\brak{1|2} = \frac{5}{50}
\end{align}
The desired probability is the probability that a slip drawn at random is marked other than Rs 1,
\begin{align}
&=1-p_X\brak{0}\\
&= p_X(1) + p_X(2)
\end{align}
Using Bayes theorem,
\begin{align}
&= p_Y\brak{0} \times \pr{Y=0 | X=1} + p_Y\brak{1} \times \pr{Y=1|X=2}\\
&=\frac{1}{3} \times \frac{6}{25} + \frac{2}{3} \times \frac{5}{50}\\
&=\frac{11}{75}
\end{align}

\newpage

%\tableofcontents

\bigskip

\renewcommand{\thefigure}{\theenumi}
\renewcommand{\thetable}{\theenumi}
%\renewcommand{\theequation}{\theenumi}

%\begin{abstract}
%%\boldmath
%In this letter, an algorithm for evaluating the exact analytical bit error rate  (BER)  for the piecewise linear (PL) combiner for  multiple relays is presented. Previous results were available only for upto three relays. The algorithm is unique in the sense that  the actual mathematical expressions, that are prohibitively large, need not be explicitly obtained. The diversity gain due to multiple relays is shown through plots of the analytical BER, well supported by simulations. 
%
%\end{abstract}
% IEEEtran.cls defaults to using nonbold math in the Abstract.
% This preserves the distinction between vectors and scalars. However,
% if the journal you are submitting to favors bold math in the abstract,
% then you can use LaTeX's standard command \boldmath at the very start
% of the abstract to achieve this. Many IEEE journals frown on math
% in the abstract anyway.

% Note that keywords are not normally used for peerreview papers.
%\begin{IEEEkeywords}
%Cooperative diversity, decode and forward, piecewise linear
%\end{IEEEkeywords}



% For peer review papers, you can put extra information on the cover
% page as needed:
% \ifCLASSOPTIONpeerreview
% \begin{center} \bfseries EDICS Category: 3-BBND \end{center}
% \fi
%
% For peerreview papers, this IEEEtran command inserts a page break and
% creates the second title. It will be ignored for other modes.
%\IEEEpeerreviewmaketitle




\item
If 4-digit numbers greater than 5,000 are randomly formed from the digits 0, 1, 3, 5, and 7, what is the probability of forming a number divisible by 5 when:
\begin{enumerate}
    \item The digits are repeated?
    \item The repetition of digits is not allowed?
\end{enumerate}
\solution
%\begin{table}[H]
	\centering
\begin{tabular}{|c|c|c|}
\hline
Random variable &Value &Definition\\ \hline
\multirow{3}{*}{X} &0 &Slips of Rs 1\\
&1 &Slips of Rs 5\\
&2 &Slips of Rs 13\\ \hline
\multirow{2}{*}{Y} &0 &Box A\\
&1 &Box B\\\hline
\end{tabular}
\caption{}
\label{tab:Distribution}
\end{table}
See \tabref{tab:Distribution}.
\begin{align}
p_{Y}\brak{k}= \begin{cases} 
      \frac{1}{3} & {k=0} \\
      \frac{2}{3 }& {k=1} 
   \end{cases}
   \\
p_{Y|X}\brak{0|0} = \frac{19}{25}\, 
p_{Y|X}\brak{0|1} = \frac{6}{25}\,
p_{Y|X}\brak{1|0} = \frac{45}{50}\,
p_{Y|X}\brak{1|2} = \frac{5}{50}
\end{align}
The desired probability is the probability that a slip drawn at random is marked other than Rs 1,
\begin{align}
&=1-p_X\brak{0}\\
&= p_X(1) + p_X(2)
\end{align}
Using Bayes theorem,
\begin{align}
&= p_Y\brak{0} \times \pr{Y=0 | X=1} + p_Y\brak{1} \times \pr{Y=1|X=2}\\
&=\frac{1}{3} \times \frac{6}{25} + \frac{2}{3} \times \frac{5}{50}\\
&=\frac{11}{75}
\end{align}

\newpage

%\tableofcontents

\bigskip

\renewcommand{\thefigure}{\theenumi}
\renewcommand{\thetable}{\theenumi}
%\renewcommand{\theequation}{\theenumi}

%\begin{abstract}
%%\boldmath
%In this letter, an algorithm for evaluating the exact analytical bit error rate  (BER)  for the piecewise linear (PL) combiner for  multiple relays is presented. Previous results were available only for upto three relays. The algorithm is unique in the sense that  the actual mathematical expressions, that are prohibitively large, need not be explicitly obtained. The diversity gain due to multiple relays is shown through plots of the analytical BER, well supported by simulations. 
%
%\end{abstract}
% IEEEtran.cls defaults to using nonbold math in the Abstract.
% This preserves the distinction between vectors and scalars. However,
% if the journal you are submitting to favors bold math in the abstract,
% then you can use LaTeX's standard command \boldmath at the very start
% of the abstract to achieve this. Many IEEE journals frown on math
% in the abstract anyway.

% Note that keywords are not normally used for peerreview papers.
%\begin{IEEEkeywords}
%Cooperative diversity, decode and forward, piecewise linear
%\end{IEEEkeywords}



% For peer review papers, you can put extra information on the cover
% page as needed:
% \ifCLASSOPTIONpeerreview
% \begin{center} \bfseries EDICS Category: 3-BBND \end{center}
% \fi
%
% For peerreview papers, this IEEEtran command inserts a page break and
% creates the second title. It will be ignored for other modes.
%\IEEEpeerreviewmaketitle




\item Consider the probability space $\brak{\Omega, \mathcal{G}, P}$ where $\Omega = [0,2]$ and $\mathcal{G} = \cbrak{\phi, \Omega, [0,1], (1,2]}$. Let $X$ and $Y$ be two functions on $\Omega$ defined as
\begin{align*}
    X(\omega) = 
    \begin{cases}
        1 & \text{if }\omega \in [0, 1]\\
        2 & \text{if }\omega \in (1, 2]
    \end{cases}
\end{align*}
and
\begin{align*}
    Y(\omega) = 
    \begin{cases}
        2 & \text{if }\omega \in [0, 1.5]\\
        3 & \text{if }\omega \in (1.5, 2].
    \end{cases}
\end{align*}
Then which one of the following statements is true?
\begin{enumerate}
    \item [(A)] $X$ is a random variable with respect to $\mathcal{G}$, but $Y$ is not a random variable with respect to $\mathcal{G}$.
    \item [(B)] $Y$ is a random variable with respect to $\mathcal{G}$, but $X$ is not a random variable with respect to $\mathcal{G}$.
    \item [(C)] Neither $X$ nor $Y$ is a random variable with respect to $\mathcal{G}$.
    \item [(D)] Both $X$ and $Y$ are random variables with respect to $\mathcal{G}$.
\end{enumerate} \hfill (GATE ST 2023)\\
\solution
%\begin{table}[H]
	\centering
\begin{tabular}{|c|c|c|}
\hline
Random variable &Value &Definition\\ \hline
\multirow{3}{*}{X} &0 &Slips of Rs 1\\
&1 &Slips of Rs 5\\
&2 &Slips of Rs 13\\ \hline
\multirow{2}{*}{Y} &0 &Box A\\
&1 &Box B\\\hline
\end{tabular}
\caption{}
\label{tab:Distribution}
\end{table}
See \tabref{tab:Distribution}.
\begin{align}
p_{Y}\brak{k}= \begin{cases} 
      \frac{1}{3} & {k=0} \\
      \frac{2}{3 }& {k=1} 
   \end{cases}
   \\
p_{Y|X}\brak{0|0} = \frac{19}{25}\, 
p_{Y|X}\brak{0|1} = \frac{6}{25}\,
p_{Y|X}\brak{1|0} = \frac{45}{50}\,
p_{Y|X}\brak{1|2} = \frac{5}{50}
\end{align}
The desired probability is the probability that a slip drawn at random is marked other than Rs 1,
\begin{align}
&=1-p_X\brak{0}\\
&= p_X(1) + p_X(2)
\end{align}
Using Bayes theorem,
\begin{align}
&= p_Y\brak{0} \times \pr{Y=0 | X=1} + p_Y\brak{1} \times \pr{Y=1|X=2}\\
&=\frac{1}{3} \times \frac{6}{25} + \frac{2}{3} \times \frac{5}{50}\\
&=\frac{11}{75}
\end{align}

\newpage

%\tableofcontents

\bigskip

\renewcommand{\thefigure}{\theenumi}
\renewcommand{\thetable}{\theenumi}
%\renewcommand{\theequation}{\theenumi}

%\begin{abstract}
%%\boldmath
%In this letter, an algorithm for evaluating the exact analytical bit error rate  (BER)  for the piecewise linear (PL) combiner for  multiple relays is presented. Previous results were available only for upto three relays. The algorithm is unique in the sense that  the actual mathematical expressions, that are prohibitively large, need not be explicitly obtained. The diversity gain due to multiple relays is shown through plots of the analytical BER, well supported by simulations. 
%
%\end{abstract}
% IEEEtran.cls defaults to using nonbold math in the Abstract.
% This preserves the distinction between vectors and scalars. However,
% if the journal you are submitting to favors bold math in the abstract,
% then you can use LaTeX's standard command \boldmath at the very start
% of the abstract to achieve this. Many IEEE journals frown on math
% in the abstract anyway.

% Note that keywords are not normally used for peerreview papers.
%\begin{IEEEkeywords}
%Cooperative diversity, decode and forward, piecewise linear
%\end{IEEEkeywords}



% For peer review papers, you can put extra information on the cover
% page as needed:
% \ifCLASSOPTIONpeerreview
% \begin{center} \bfseries EDICS Category: 3-BBND \end{center}
% \fi
%
% For peerreview papers, this IEEEtran command inserts a page break and
% creates the second title. It will be ignored for other modes.
%\IEEEpeerreviewmaketitle




	\item  A die is loaded in such a way that each odd number is twice as likely to occur as
each even number. Find $P(G)$, where $G$ is the event that a number greater than
3 occurs on a single roll of the die.
\\
\solution
		%\begin{table}[H]
	\centering
\begin{tabular}{|c|c|c|}
\hline
Random variable &Value &Definition\\ \hline
\multirow{3}{*}{X} &0 &Slips of Rs 1\\
&1 &Slips of Rs 5\\
&2 &Slips of Rs 13\\ \hline
\multirow{2}{*}{Y} &0 &Box A\\
&1 &Box B\\\hline
\end{tabular}
\caption{}
\label{tab:Distribution}
\end{table}
See \tabref{tab:Distribution}.
\begin{align}
p_{Y}\brak{k}= \begin{cases} 
      \frac{1}{3} & {k=0} \\
      \frac{2}{3 }& {k=1} 
   \end{cases}
   \\
p_{Y|X}\brak{0|0} = \frac{19}{25}\, 
p_{Y|X}\brak{0|1} = \frac{6}{25}\,
p_{Y|X}\brak{1|0} = \frac{45}{50}\,
p_{Y|X}\brak{1|2} = \frac{5}{50}
\end{align}
The desired probability is the probability that a slip drawn at random is marked other than Rs 1,
\begin{align}
&=1-p_X\brak{0}\\
&= p_X(1) + p_X(2)
\end{align}
Using Bayes theorem,
\begin{align}
&= p_Y\brak{0} \times \pr{Y=0 | X=1} + p_Y\brak{1} \times \pr{Y=1|X=2}\\
&=\frac{1}{3} \times \frac{6}{25} + \frac{2}{3} \times \frac{5}{50}\\
&=\frac{11}{75}
\end{align}

\newpage

%\tableofcontents

\bigskip

\renewcommand{\thefigure}{\theenumi}
\renewcommand{\thetable}{\theenumi}
%\renewcommand{\theequation}{\theenumi}

%\begin{abstract}
%%\boldmath
%In this letter, an algorithm for evaluating the exact analytical bit error rate  (BER)  for the piecewise linear (PL) combiner for  multiple relays is presented. Previous results were available only for upto three relays. The algorithm is unique in the sense that  the actual mathematical expressions, that are prohibitively large, need not be explicitly obtained. The diversity gain due to multiple relays is shown through plots of the analytical BER, well supported by simulations. 
%
%\end{abstract}
% IEEEtran.cls defaults to using nonbold math in the Abstract.
% This preserves the distinction between vectors and scalars. However,
% if the journal you are submitting to favors bold math in the abstract,
% then you can use LaTeX's standard command \boldmath at the very start
% of the abstract to achieve this. Many IEEE journals frown on math
% in the abstract anyway.

% Note that keywords are not normally used for peerreview papers.
%\begin{IEEEkeywords}
%Cooperative diversity, decode and forward, piecewise linear
%\end{IEEEkeywords}



% For peer review papers, you can put extra information on the cover
% page as needed:
% \ifCLASSOPTIONpeerreview
% \begin{center} \bfseries EDICS Category: 3-BBND \end{center}
% \fi
%
% For peerreview papers, this IEEEtran command inserts a page break and
% creates the second title. It will be ignored for other modes.
%\IEEEpeerreviewmaketitle




	\item All the jacks, queens and kings are removed from a deck of 52 playing cards. The remaining cards are well shuffled and then one card is drawn at random. Giving ace a value 1 similar value for other cards, find the probability that the card has a value 
		\begin{enumerate}
			\item 7
			\item greater than 7
			\item less than 7
		\end{enumerate}
		%Number of cards left after removing all jacks, queens and kings 
\begin{align}
N	= 52 - 4\times 3
	= 40
\end{align}
%\begin{table}[H]
%\def\arraystretch{1.2}
%\begin{tabular}{|c|c|c|}
%\hline
%	\textbf{Parameter} &\textbf{Value} &\textbf{Description}\\ \hline
%	$X$ &1-10 &Represents the value of the card picked \\ \hline
%\end{tabular}
%\end{table}
Let $1 \le X \le 10$ be the value of the card picked.  Then,
\begin{align}
	p_X(k) &= \Pr(X=k)\ \forall\ 1 \leq k \leq 10\\
	&= \frac{4\times 1}{40}\\
	&= \frac{1}{10}\\
	\therefore p_X(k) &= 
	\begin{cases}
		\frac{1}{10} & 1 \leq k \leq 10\\
		0 & \text{otherwise}
	\end{cases}
\end{align}
and
\begin{align}
	F_{X}(k) &= \sum_{m=0}^{k}p_{X}(m) \quad 1 \leq k \leq 10\\
	&= \frac{k}{10}\\
	\therefore F_{X}(k) &= 
	\begin{cases}
		0 & k \leq 0\\
		\frac{k}{10} & 1\leq k \leq 10\\
		1 & k > 10 
	\end{cases}
\end{align}
\begin{enumerate}
	\item Probability that card has value equal to 7 is
		\begin{align}
			 p_{X}(7)
			= \frac{1}{10}
		\end{align}
	\item Probability that card has value greater than 7 is
		\begin{align}
			1 - F_X(7)
			&= 1 - \frac{7}{10}
			\\
			&= \frac{3}{10}
		\end{align}
	\item Probability that card has value less than 7 is
		\begin{align}
			 F_{X}(6)
			=\frac{6}{10}
		\end{align}
\end{enumerate}

  \item A Lot consists of 48 mobile phones of which 42 are good, 3 have only minor defects and 3 have major defects.Varnika will buy a phone if it is good but the trader will only buy a mobile if it has no major defects. One phone is selected at random from the lot. What is the probability that it is
\begin{enumerate}
	\item acceptable to Varnika?
            \item acceptable to the trader?
\end{enumerate}
\solution
	%\begin{table}[H]
	\centering
\begin{tabular}{|c|c|c|}
\hline
Random variable &Value &Definition\\ \hline
\multirow{3}{*}{X} &0 &Slips of Rs 1\\
&1 &Slips of Rs 5\\
&2 &Slips of Rs 13\\ \hline
\multirow{2}{*}{Y} &0 &Box A\\
&1 &Box B\\\hline
\end{tabular}
\caption{}
\label{tab:Distribution}
\end{table}
See \tabref{tab:Distribution}.
\begin{align}
p_{Y}\brak{k}= \begin{cases} 
      \frac{1}{3} & {k=0} \\
      \frac{2}{3 }& {k=1} 
   \end{cases}
   \\
p_{Y|X}\brak{0|0} = \frac{19}{25}\, 
p_{Y|X}\brak{0|1} = \frac{6}{25}\,
p_{Y|X}\brak{1|0} = \frac{45}{50}\,
p_{Y|X}\brak{1|2} = \frac{5}{50}
\end{align}
The desired probability is the probability that a slip drawn at random is marked other than Rs 1,
\begin{align}
&=1-p_X\brak{0}\\
&= p_X(1) + p_X(2)
\end{align}
Using Bayes theorem,
\begin{align}
&= p_Y\brak{0} \times \pr{Y=0 | X=1} + p_Y\brak{1} \times \pr{Y=1|X=2}\\
&=\frac{1}{3} \times \frac{6}{25} + \frac{2}{3} \times \frac{5}{50}\\
&=\frac{11}{75}
\end{align}

\newpage

%\tableofcontents

\bigskip

\renewcommand{\thefigure}{\theenumi}
\renewcommand{\thetable}{\theenumi}
%\renewcommand{\theequation}{\theenumi}

%\begin{abstract}
%%\boldmath
%In this letter, an algorithm for evaluating the exact analytical bit error rate  (BER)  for the piecewise linear (PL) combiner for  multiple relays is presented. Previous results were available only for upto three relays. The algorithm is unique in the sense that  the actual mathematical expressions, that are prohibitively large, need not be explicitly obtained. The diversity gain due to multiple relays is shown through plots of the analytical BER, well supported by simulations. 
%
%\end{abstract}
% IEEEtran.cls defaults to using nonbold math in the Abstract.
% This preserves the distinction between vectors and scalars. However,
% if the journal you are submitting to favors bold math in the abstract,
% then you can use LaTeX's standard command \boldmath at the very start
% of the abstract to achieve this. Many IEEE journals frown on math
% in the abstract anyway.

% Note that keywords are not normally used for peerreview papers.
%\begin{IEEEkeywords}
%Cooperative diversity, decode and forward, piecewise linear
%\end{IEEEkeywords}



% For peer review papers, you can put extra information on the cover
% page as needed:
% \ifCLASSOPTIONpeerreview
% \begin{center} \bfseries EDICS Category: 3-BBND \end{center}
% \fi
%
% For peerreview papers, this IEEEtran command inserts a page break and
% creates the second title. It will be ignored for other modes.
%\IEEEpeerreviewmaketitle




 \item A student says that if you throw a die, it will show up 1 or not 1. Therefore, the probability of getting 1 and the probability of getting 'not 1' each is equal to $\frac{1}{2}$. Is this correct? Give reasons.\\
 \solution
        %\begin{table}[H]
	\centering
\begin{tabular}{|c|c|c|}
\hline
Random variable &Value &Definition\\ \hline
\multirow{3}{*}{X} &0 &Slips of Rs 1\\
&1 &Slips of Rs 5\\
&2 &Slips of Rs 13\\ \hline
\multirow{2}{*}{Y} &0 &Box A\\
&1 &Box B\\\hline
\end{tabular}
\caption{}
\label{tab:Distribution}
\end{table}
See \tabref{tab:Distribution}.
\begin{align}
p_{Y}\brak{k}= \begin{cases} 
      \frac{1}{3} & {k=0} \\
      \frac{2}{3 }& {k=1} 
   \end{cases}
   \\
p_{Y|X}\brak{0|0} = \frac{19}{25}\, 
p_{Y|X}\brak{0|1} = \frac{6}{25}\,
p_{Y|X}\brak{1|0} = \frac{45}{50}\,
p_{Y|X}\brak{1|2} = \frac{5}{50}
\end{align}
The desired probability is the probability that a slip drawn at random is marked other than Rs 1,
\begin{align}
&=1-p_X\brak{0}\\
&= p_X(1) + p_X(2)
\end{align}
Using Bayes theorem,
\begin{align}
&= p_Y\brak{0} \times \pr{Y=0 | X=1} + p_Y\brak{1} \times \pr{Y=1|X=2}\\
&=\frac{1}{3} \times \frac{6}{25} + \frac{2}{3} \times \frac{5}{50}\\
&=\frac{11}{75}
\end{align}

\newpage

%\tableofcontents

\bigskip

\renewcommand{\thefigure}{\theenumi}
\renewcommand{\thetable}{\theenumi}
%\renewcommand{\theequation}{\theenumi}

%\begin{abstract}
%%\boldmath
%In this letter, an algorithm for evaluating the exact analytical bit error rate  (BER)  for the piecewise linear (PL) combiner for  multiple relays is presented. Previous results were available only for upto three relays. The algorithm is unique in the sense that  the actual mathematical expressions, that are prohibitively large, need not be explicitly obtained. The diversity gain due to multiple relays is shown through plots of the analytical BER, well supported by simulations. 
%
%\end{abstract}
% IEEEtran.cls defaults to using nonbold math in the Abstract.
% This preserves the distinction between vectors and scalars. However,
% if the journal you are submitting to favors bold math in the abstract,
% then you can use LaTeX's standard command \boldmath at the very start
% of the abstract to achieve this. Many IEEE journals frown on math
% in the abstract anyway.

% Note that keywords are not normally used for peerreview papers.
%\begin{IEEEkeywords}
%Cooperative diversity, decode and forward, piecewise linear
%\end{IEEEkeywords}



% For peer review papers, you can put extra information on the cover
% page as needed:
% \ifCLASSOPTIONpeerreview
% \begin{center} \bfseries EDICS Category: 3-BBND \end{center}
% \fi
%
% For peerreview papers, this IEEEtran command inserts a page break and
% creates the second title. It will be ignored for other modes.
%\IEEEpeerreviewmaketitle




   \item Four candidates A, B, C, D have ap-
plied for the assignment to coach a school cricket
team. If A is twice as likely to be selected as B, and
B and C are given about the same chance of being
selected, while C is twice as likely to be selected
as D, what are the probabilities that
\begin{enumerate}
\item C will be selected?
\item A will not be selected?
\end{enumerate}
	%\begin{table}[H]
	\centering
\begin{tabular}{|c|c|c|}
\hline
Random variable &Value &Definition\\ \hline
\multirow{3}{*}{X} &0 &Slips of Rs 1\\
&1 &Slips of Rs 5\\
&2 &Slips of Rs 13\\ \hline
\multirow{2}{*}{Y} &0 &Box A\\
&1 &Box B\\\hline
\end{tabular}
\caption{}
\label{tab:Distribution}
\end{table}
See \tabref{tab:Distribution}.
\begin{align}
p_{Y}\brak{k}= \begin{cases} 
      \frac{1}{3} & {k=0} \\
      \frac{2}{3 }& {k=1} 
   \end{cases}
   \\
p_{Y|X}\brak{0|0} = \frac{19}{25}\, 
p_{Y|X}\brak{0|1} = \frac{6}{25}\,
p_{Y|X}\brak{1|0} = \frac{45}{50}\,
p_{Y|X}\brak{1|2} = \frac{5}{50}
\end{align}
The desired probability is the probability that a slip drawn at random is marked other than Rs 1,
\begin{align}
&=1-p_X\brak{0}\\
&= p_X(1) + p_X(2)
\end{align}
Using Bayes theorem,
\begin{align}
&= p_Y\brak{0} \times \pr{Y=0 | X=1} + p_Y\brak{1} \times \pr{Y=1|X=2}\\
&=\frac{1}{3} \times \frac{6}{25} + \frac{2}{3} \times \frac{5}{50}\\
&=\frac{11}{75}
\end{align}

\newpage

%\tableofcontents

\bigskip

\renewcommand{\thefigure}{\theenumi}
\renewcommand{\thetable}{\theenumi}
%\renewcommand{\theequation}{\theenumi}

%\begin{abstract}
%%\boldmath
%In this letter, an algorithm for evaluating the exact analytical bit error rate  (BER)  for the piecewise linear (PL) combiner for  multiple relays is presented. Previous results were available only for upto three relays. The algorithm is unique in the sense that  the actual mathematical expressions, that are prohibitively large, need not be explicitly obtained. The diversity gain due to multiple relays is shown through plots of the analytical BER, well supported by simulations. 
%
%\end{abstract}
% IEEEtran.cls defaults to using nonbold math in the Abstract.
% This preserves the distinction between vectors and scalars. However,
% if the journal you are submitting to favors bold math in the abstract,
% then you can use LaTeX's standard command \boldmath at the very start
% of the abstract to achieve this. Many IEEE journals frown on math
% in the abstract anyway.

% Note that keywords are not normally used for peerreview papers.
%\begin{IEEEkeywords}
%Cooperative diversity, decode and forward, piecewise linear
%\end{IEEEkeywords}



% For peer review papers, you can put extra information on the cover
% page as needed:
% \ifCLASSOPTIONpeerreview
% \begin{center} \bfseries EDICS Category: 3-BBND \end{center}
% \fi
%
% For peerreview papers, this IEEEtran command inserts a page break and
% creates the second title. It will be ignored for other modes.
%\IEEEpeerreviewmaketitle




 \item A bag contain 24 balls of which $x$ balls are red, $2x$ are white and $3x$ are blue. A ball is selected at random, What is the probability that it is
\begin{enumerate}[label=\alph*)]
\item not red ?
\item white ?
\end{enumerate}
%\begin{table}[H]
	\centering
\begin{tabular}{|c|c|c|}
\hline
Random variable &Value &Definition\\ \hline
\multirow{3}{*}{X} &0 &Slips of Rs 1\\
&1 &Slips of Rs 5\\
&2 &Slips of Rs 13\\ \hline
\multirow{2}{*}{Y} &0 &Box A\\
&1 &Box B\\\hline
\end{tabular}
\caption{}
\label{tab:Distribution}
\end{table}
See \tabref{tab:Distribution}.
\begin{align}
p_{Y}\brak{k}= \begin{cases} 
      \frac{1}{3} & {k=0} \\
      \frac{2}{3 }& {k=1} 
   \end{cases}
   \\
p_{Y|X}\brak{0|0} = \frac{19}{25}\, 
p_{Y|X}\brak{0|1} = \frac{6}{25}\,
p_{Y|X}\brak{1|0} = \frac{45}{50}\,
p_{Y|X}\brak{1|2} = \frac{5}{50}
\end{align}
The desired probability is the probability that a slip drawn at random is marked other than Rs 1,
\begin{align}
&=1-p_X\brak{0}\\
&= p_X(1) + p_X(2)
\end{align}
Using Bayes theorem,
\begin{align}
&= p_Y\brak{0} \times \pr{Y=0 | X=1} + p_Y\brak{1} \times \pr{Y=1|X=2}\\
&=\frac{1}{3} \times \frac{6}{25} + \frac{2}{3} \times \frac{5}{50}\\
&=\frac{11}{75}
\end{align}

\newpage

%\tableofcontents

\bigskip

\renewcommand{\thefigure}{\theenumi}
\renewcommand{\thetable}{\theenumi}
%\renewcommand{\theequation}{\theenumi}

%\begin{abstract}
%%\boldmath
%In this letter, an algorithm for evaluating the exact analytical bit error rate  (BER)  for the piecewise linear (PL) combiner for  multiple relays is presented. Previous results were available only for upto three relays. The algorithm is unique in the sense that  the actual mathematical expressions, that are prohibitively large, need not be explicitly obtained. The diversity gain due to multiple relays is shown through plots of the analytical BER, well supported by simulations. 
%
%\end{abstract}
% IEEEtran.cls defaults to using nonbold math in the Abstract.
% This preserves the distinction between vectors and scalars. However,
% if the journal you are submitting to favors bold math in the abstract,
% then you can use LaTeX's standard command \boldmath at the very start
% of the abstract to achieve this. Many IEEE journals frown on math
% in the abstract anyway.

% Note that keywords are not normally used for peerreview papers.
%\begin{IEEEkeywords}
%Cooperative diversity, decode and forward, piecewise linear
%\end{IEEEkeywords}



% For peer review papers, you can put extra information on the cover
% page as needed:
% \ifCLASSOPTIONpeerreview
% \begin{center} \bfseries EDICS Category: 3-BBND \end{center}
% \fi
%
% For peerreview papers, this IEEEtran command inserts a page break and
% creates the second title. It will be ignored for other modes.
%\IEEEpeerreviewmaketitle




If the letters of the word ASSASSINATION are arranged at random. Find the Probability that
\begin{enumerate}[label=(\alph*)]
\item Four $S's$ come consecutively in the word
\item Two  $I's$ and two $N's$ come together
\item All $A's$ are not coming together
\item No two $A's$ are coming together
\end{enumerate}
%\begin{table}[H]
	\centering
\begin{tabular}{|c|c|c|}
\hline
Random variable &Value &Definition\\ \hline
\multirow{3}{*}{X} &0 &Slips of Rs 1\\
&1 &Slips of Rs 5\\
&2 &Slips of Rs 13\\ \hline
\multirow{2}{*}{Y} &0 &Box A\\
&1 &Box B\\\hline
\end{tabular}
\caption{}
\label{tab:Distribution}
\end{table}
See \tabref{tab:Distribution}.
\begin{align}
p_{Y}\brak{k}= \begin{cases} 
      \frac{1}{3} & {k=0} \\
      \frac{2}{3 }& {k=1} 
   \end{cases}
   \\
p_{Y|X}\brak{0|0} = \frac{19}{25}\, 
p_{Y|X}\brak{0|1} = \frac{6}{25}\,
p_{Y|X}\brak{1|0} = \frac{45}{50}\,
p_{Y|X}\brak{1|2} = \frac{5}{50}
\end{align}
The desired probability is the probability that a slip drawn at random is marked other than Rs 1,
\begin{align}
&=1-p_X\brak{0}\\
&= p_X(1) + p_X(2)
\end{align}
Using Bayes theorem,
\begin{align}
&= p_Y\brak{0} \times \pr{Y=0 | X=1} + p_Y\brak{1} \times \pr{Y=1|X=2}\\
&=\frac{1}{3} \times \frac{6}{25} + \frac{2}{3} \times \frac{5}{50}\\
&=\frac{11}{75}
\end{align}

\newpage

%\tableofcontents

\bigskip

\renewcommand{\thefigure}{\theenumi}
\renewcommand{\thetable}{\theenumi}
%\renewcommand{\theequation}{\theenumi}

%\begin{abstract}
%%\boldmath
%In this letter, an algorithm for evaluating the exact analytical bit error rate  (BER)  for the piecewise linear (PL) combiner for  multiple relays is presented. Previous results were available only for upto three relays. The algorithm is unique in the sense that  the actual mathematical expressions, that are prohibitively large, need not be explicitly obtained. The diversity gain due to multiple relays is shown through plots of the analytical BER, well supported by simulations. 
%
%\end{abstract}
% IEEEtran.cls defaults to using nonbold math in the Abstract.
% This preserves the distinction between vectors and scalars. However,
% if the journal you are submitting to favors bold math in the abstract,
% then you can use LaTeX's standard command \boldmath at the very start
% of the abstract to achieve this. Many IEEE journals frown on math
% in the abstract anyway.

% Note that keywords are not normally used for peerreview papers.
%\begin{IEEEkeywords}
%Cooperative diversity, decode and forward, piecewise linear
%\end{IEEEkeywords}



% For peer review papers, you can put extra information on the cover
% page as needed:
% \ifCLASSOPTIONpeerreview
% \begin{center} \bfseries EDICS Category: 3-BBND \end{center}
% \fi
%
% For peerreview papers, this IEEEtran command inserts a page break and
% creates the second title. It will be ignored for other modes.
%\IEEEpeerreviewmaketitle




	\item One urn contains two black balls (labelled B1 and B2) and one white ball. A
	second urn contains one black ball and two white balls (labelled W1 and W2).
	Suppose the following experiment is performed. One of the two urns is chosen
	at random. Next a ball is randomly chosen from the urn. Then a second ball is
	chosen at random from the same urn without replacing the first ball.
	
	\begin{enumerate}
	\item What is the probability that two black balls are chosen?
	
	\item What is the probability that two balls of opposite colour are chosen?
	\end{enumerate}
	\solution
	%\begin{align}
    \label{eq:12.13.6.18.1}
	\because	\pr{A|B} &> \pr{A},\
\frac{\pr{AB}}{\pr{B}} > \pr{A}
\\
    \label{eq:12.13.6.18.2}
	\implies \pr{AB} &> \pr{A}\pr{B}
	\\
	\text{or, } \frac{\pr{AB}}{\pr{A}} &=\pr{B|A} > \pr{A}
\end{align}

\end{enumerate}

	\item A bag contains $5$ red balls and some blue balls. If the probability of drawing a blue ball is double that if a red ball, determine the number of blue balls in the bag. 
		\\
\solution
		%\begin{enumerate}[label=\thesection.\arabic*,ref=\thesection.\theenumi]
	\item One card is drawn from a well-shuffled deck of 52 cards. Find the probability of getting
\begin{enumerate}
\item A king of red colour 
\item A face card 
\item A red face card
\item The jack of hearts
\item A spade
\item The queen of diamonds

\end{enumerate}
\solution
		%\begin{table}[H]
	\centering
\begin{tabular}{|c|c|c|}
\hline
Random variable &Value &Definition\\ \hline
\multirow{3}{*}{X} &0 &Slips of Rs 1\\
&1 &Slips of Rs 5\\
&2 &Slips of Rs 13\\ \hline
\multirow{2}{*}{Y} &0 &Box A\\
&1 &Box B\\\hline
\end{tabular}
\caption{}
\label{tab:Distribution}
\end{table}
See \tabref{tab:Distribution}.
\begin{align}
p_{Y}\brak{k}= \begin{cases} 
      \frac{1}{3} & {k=0} \\
      \frac{2}{3 }& {k=1} 
   \end{cases}
   \\
p_{Y|X}\brak{0|0} = \frac{19}{25}\, 
p_{Y|X}\brak{0|1} = \frac{6}{25}\,
p_{Y|X}\brak{1|0} = \frac{45}{50}\,
p_{Y|X}\brak{1|2} = \frac{5}{50}
\end{align}
The desired probability is the probability that a slip drawn at random is marked other than Rs 1,
\begin{align}
&=1-p_X\brak{0}\\
&= p_X(1) + p_X(2)
\end{align}
Using Bayes theorem,
\begin{align}
&= p_Y\brak{0} \times \pr{Y=0 | X=1} + p_Y\brak{1} \times \pr{Y=1|X=2}\\
&=\frac{1}{3} \times \frac{6}{25} + \frac{2}{3} \times \frac{5}{50}\\
&=\frac{11}{75}
\end{align}

\newpage

%\tableofcontents

\bigskip

\renewcommand{\thefigure}{\theenumi}
\renewcommand{\thetable}{\theenumi}
%\renewcommand{\theequation}{\theenumi}

%\begin{abstract}
%%\boldmath
%In this letter, an algorithm for evaluating the exact analytical bit error rate  (BER)  for the piecewise linear (PL) combiner for  multiple relays is presented. Previous results were available only for upto three relays. The algorithm is unique in the sense that  the actual mathematical expressions, that are prohibitively large, need not be explicitly obtained. The diversity gain due to multiple relays is shown through plots of the analytical BER, well supported by simulations. 
%
%\end{abstract}
% IEEEtran.cls defaults to using nonbold math in the Abstract.
% This preserves the distinction between vectors and scalars. However,
% if the journal you are submitting to favors bold math in the abstract,
% then you can use LaTeX's standard command \boldmath at the very start
% of the abstract to achieve this. Many IEEE journals frown on math
% in the abstract anyway.

% Note that keywords are not normally used for peerreview papers.
%\begin{IEEEkeywords}
%Cooperative diversity, decode and forward, piecewise linear
%\end{IEEEkeywords}



% For peer review papers, you can put extra information on the cover
% page as needed:
% \ifCLASSOPTIONpeerreview
% \begin{center} \bfseries EDICS Category: 3-BBND \end{center}
% \fi
%
% For peerreview papers, this IEEEtran command inserts a page break and
% creates the second title. It will be ignored for other modes.
%\IEEEpeerreviewmaketitle




	\item Five cards—the ten, jack, queen, king and ace of diamonds, are well-shuffled with their face downwards. One card is then picked up at random.
\begin{enumerate}
\item
What is the probability that the card is the queen? 
\item
If the queen is drawn and put aside, what is the probability that the second card picked up is (a) an ace? (b) a queen?\\
\end{enumerate}
\solution
		%\begin{enumerate}[label=\thesection.\arabic*,ref=\thesection.\theenumi]
	\item One card is drawn from a well-shuffled deck of 52 cards. Find the probability of getting
\begin{enumerate}
\item A king of red colour 
\item A face card 
\item A red face card
\item The jack of hearts
\item A spade
\item The queen of diamonds

\end{enumerate}
\solution
		%\input{ncert/10/15/1/14/main.tex}
	\item Five cards—the ten, jack, queen, king and ace of diamonds, are well-shuffled with their face downwards. One card is then picked up at random.
\begin{enumerate}
\item
What is the probability that the card is the queen? 
\item
If the queen is drawn and put aside, what is the probability that the second card picked up is (a) an ace? (b) a queen?\\
\end{enumerate}
\solution
		%\input{ncert/10/15/1/15/defs.tex}
	\item A bag contains $5$ red balls and some blue balls. If the probability of drawing a blue ball is double that if a red ball, determine the number of blue balls in the bag. 
		\\
\solution
		%\input{ncert/10/15/2/3/defs.tex}
	\item A card is selected from a pack of 52 cards.
 \begin{enumerate}[label=(\alph*)] 
                 \item How many points are there in the sample space?
                 \item Calculate the probability that the card is an ace of spades.
                 \item Calculate the probability that the card is (i) an ace and (ii) black card.
 \end{enumerate}
\solution
		%\input{ncert/11/16/3/4/main.tex}
\item Four cards are drawn from a well-shuffled deck of 52 cards. What is the probability of obtaining 3 diamonds and one spade.
\\
\solution
		%\input{ncert/11/16/4/2/defs.tex}
\item In a certain lottery 10,000 tickets are sold and ten equal prizes are awarded. What is the probability of not getting a prize if you buy (a) one ticket (b) two tickets (c) 10 tickets ?	
\\
\solution
		%\input{ncert/11/16/4/4/defs.tex}
		%
\item 
Out of 100 students, two sections of 40 and 60 are formed. If you and your friend are among the 100 students, what is the probability that
\begin{enumerate}
\item you both enter the same section?
\item you both enter the different sections?
\end{enumerate}
\solution
		%\input{ncert/11/16/4/5/defs.tex}
	\item 
The number lock of a suitcase has 4 wheels each labelled with ten digits i.e. from 0 to 9.The lock opens with a sequence of four digits with no repeats.What is the probability of a person getting the right sequence to open the suitcase.
\\
\solution
		%\input{ncert/11/16/4/10/defs.tex}
		%
\item 
Two cards are drawn at random and without replacement from a pack of 52 playing cards. Find the probability that both the cards are black.
\\
\solution
		%\input{ncert/12/13/2/2/defs.tex}
		\item A box of oranges is inspected by examining three randomly selected oranges drawn without replacement. If all the three oranges are good, the box is approved for sale, otherwise, it is rejected. Find the probability that a box containing 15 oranges out of which 12 are good and 3 are bad ones will be approved for sale.
		\label{ncert/12/13/2/3/defs.tex}
		\item Two balls are drawn at random with replacement from a box containing 10 black and 8 red balls. Find the probability that
		\label{ncert/12/13/2/12}
\begin{enumerate}
\item both balls are red.
\item first ball is black and second is red.
\item one of them is black and other is red.
\end{enumerate}

\item In a hostel, 60\% of the students read Hindi newspaper, 40\% read English newspaper and 20\% read both Hindi and English newspapers. A student is selected at random.
		\label{ncert/12/13/2/15}
\begin{enumerate}
\item Find the probability that she reads neither Hindi nor English newspapers.
\item If she reads Hindi newspaper, find the probability that she reads English newspaper.
\item If she reads English newspaper, find the probability that she reads Hindi newspaper.\\
\end{enumerate}
\item The probability of obtaining an even prime number on each die, when a pair of dice is rolled is 
\begin{enumerate}
    \item $0$ 
    
    \item $\frac{1}{3}$ 
    
    \item $\frac{1}{12}$ 
    
    \item $\frac{1}{36}$ 
\end{enumerate}
\solution
		%\input{ncert/12/13/2/17/defs.tex}
	\item A bag contains 4 red and 4 black balls, another bag contains 2 red and 6 black balls. One of the two bags is selected at random and a ball is drawn from the bag which is found to be red. Find the probability that the ball is drawn from the first bag.
\\
\solution
		%\input{ncert/12/13/3/2/main.tex}
  \item
  Cards with numbers 2 to 101 are placed in a box. A card is selected at random.Find the probability that the card has
\begin{enumerate}[label=(\roman*)]
	\item an even number 
	\item a square number
\end{enumerate}
\solution
%\input{exemplar/10/13/3/32/main.tex}
\item
The king, queen and jack of clubs are removed from a deck of 52 playing cards and then well shuffled. Now one card is drawn at random from the remaining cards.  Determine the probability that the card is
\begin{enumerate}[label=(\roman*)]
\item a club
\item 10 of hearts
\end{enumerate}
\solution
%\input{exemplar/10/13/3/29/main.tex}
\item A team of medical students doing their internship have to assist during surgeries
at a city hospital. The probabilities of surgeries rated as very complex, complex,
routine, simple or very simple are respectively, 0.15, 0.20, 0.31, 0.26, .08. Find
the probabilities that a particular surgery will be rated
\begin{enumerate}
	\item complex or very complex;
	\item neither very complex nor very simple;
	\item routine or complex
	\item routine or simple
\end{enumerate}
\solution
%\input{exemplar/11/16/3/8(1)/main.tex}
\item A card is selected from a pack of 52 cards.
\begin{enumerate}[label=(\alph*)]
    \item How many points are there in the sample space?
    \item Calculate the probability that the card is an ace of spades.
    \item Calculate the probability that the card is (i) an ace and (ii) black card.
\end{enumerate}
\solution
%\input{exemplar/11/16/3/4/main2.tex}
\item The probability that a non leap year selected at random will contain 53 sundays.
\\
\solution
%\input{exemplar/10/13/1/19/main.tex}
\item One of the four persons John, Rita, Aslam or Gurpreet will be promoted next
month. Consequently the sample space consists of four elementary outcomes
S = {John promoted, Rita promoted, Aslam promoted, Gurpreet promoted}
You are told that the chances of John’s promotion is same as that of Gurpreet,
Rita’s chances of promotion are twice as likely as Johns. Aslam’s chances are
four times that of John.
\begin{enumerate}
	\item Determine
	\begin{enumerate}
		\item P (John promoted)
		\item P (Rita promoted)
		\item P (Aslam promoted)
		\item P (Gurpreet promoted)
	\end{enumerate}
	\item If A = {John promoted or Gurpreet promoted}, find P (A).
\end{enumerate}
\solution
%\input{exemplar/11/16/3/10/main.tex}
\item A card is drawn from a deck of 52 cards. Find the probability of getting a king or a heart or a red card.\\
\solution
%\input{exemplar/11/16/3/15/main.tex}
\item The probability that a student will pass his examination is 0.73, the probability of
the student getting a compartment is 0.13, and the probability that the student will
either pass or get compartment is 0.96. State True or False.\\
\solution
%\input{exemplar/11/16/3/31/main.tex}
\item A card is selected from a pack of 52 cards\\
\begin{enumerate}[label=(\alph*)]
\item How many points are there in the sample space?
\item Calculate the probability that the cards is an ace of spades.
\item Calculate the probability that the card is (i) an ace (ii)black card.\\
\end{enumerate}
%\input{ncert/11/16/3/4_1/Prob_4.tex}
\item In a non-leap year, the probability of having 53 tuesdays or 53 wednesdays is\\
\solution
%\input{exemplar/11/16/3/18/main.tex}
\item There are 1000 sealed envelopes in a box, 10 of them contain a cash prize of
Rs 100 each, 100 of them contain a cash prize of Rs 50 each and 200 of them
contain a cash prize of Rs 10 each and rest do not contain any cash prize. If they
are well shuffled and an envelope is picked up out, what is the probability that it
contains no cash prize?\\
\solution
%\input{exemplar/10/13/3/34/main.tex}
\item 
A die is thrown and a card is selected at random from a deck of 52 playing cards. The probability of getting an even number on the die and a spade card.\\
\solution
%\input{exemplar/12/13/3/78/main.tex}
\item
If 4-digit numbers greater than 5,000 are randomly formed from the digits 0, 1, 3, 5, and 7, what is the probability of forming a number divisible by 5 when:
\begin{enumerate}
    \item The digits are repeated?
    \item The repetition of digits is not allowed?
\end{enumerate}
\solution
%\input{ncert/11/16/4/9/main.tex}
\item Consider the probability space $\brak{\Omega, \mathcal{G}, P}$ where $\Omega = [0,2]$ and $\mathcal{G} = \cbrak{\phi, \Omega, [0,1], (1,2]}$. Let $X$ and $Y$ be two functions on $\Omega$ defined as
\begin{align*}
    X(\omega) = 
    \begin{cases}
        1 & \text{if }\omega \in [0, 1]\\
        2 & \text{if }\omega \in (1, 2]
    \end{cases}
\end{align*}
and
\begin{align*}
    Y(\omega) = 
    \begin{cases}
        2 & \text{if }\omega \in [0, 1.5]\\
        3 & \text{if }\omega \in (1.5, 2].
    \end{cases}
\end{align*}
Then which one of the following statements is true?
\begin{enumerate}
    \item [(A)] $X$ is a random variable with respect to $\mathcal{G}$, but $Y$ is not a random variable with respect to $\mathcal{G}$.
    \item [(B)] $Y$ is a random variable with respect to $\mathcal{G}$, but $X$ is not a random variable with respect to $\mathcal{G}$.
    \item [(C)] Neither $X$ nor $Y$ is a random variable with respect to $\mathcal{G}$.
    \item [(D)] Both $X$ and $Y$ are random variables with respect to $\mathcal{G}$.
\end{enumerate} \hfill (GATE ST 2023)\\
\solution
%\input{gate/ST/2023/14/main.tex}
	\item  A die is loaded in such a way that each odd number is twice as likely to occur as
each even number. Find $P(G)$, where $G$ is the event that a number greater than
3 occurs on a single roll of the die.
\\
\solution
		%\input{exemplar/11/16/3/5/main.tex}
	\item All the jacks, queens and kings are removed from a deck of 52 playing cards. The remaining cards are well shuffled and then one card is drawn at random. Giving ace a value 1 similar value for other cards, find the probability that the card has a value 
		\begin{enumerate}
			\item 7
			\item greater than 7
			\item less than 7
		\end{enumerate}
		%\input{exemplar/10/13/3/30/main.tex}
  \item A Lot consists of 48 mobile phones of which 42 are good, 3 have only minor defects and 3 have major defects.Varnika will buy a phone if it is good but the trader will only buy a mobile if it has no major defects. One phone is selected at random from the lot. What is the probability that it is
\begin{enumerate}
	\item acceptable to Varnika?
            \item acceptable to the trader?
\end{enumerate}
\solution
	%\input{exemplar/10/13/3/40/main.tex}
 \item A student says that if you throw a die, it will show up 1 or not 1. Therefore, the probability of getting 1 and the probability of getting 'not 1' each is equal to $\frac{1}{2}$. Is this correct? Give reasons.\\
 \solution
        %\input{exemplar/10/13/2/9/main.tex}
   \item Four candidates A, B, C, D have ap-
plied for the assignment to coach a school cricket
team. If A is twice as likely to be selected as B, and
B and C are given about the same chance of being
selected, while C is twice as likely to be selected
as D, what are the probabilities that
\begin{enumerate}
\item C will be selected?
\item A will not be selected?
\end{enumerate}
	%\input{exemplar/11/16/3/9/main.tex}
 \item A bag contain 24 balls of which $x$ balls are red, $2x$ are white and $3x$ are blue. A ball is selected at random, What is the probability that it is
\begin{enumerate}[label=\alph*)]
\item not red ?
\item white ?
\end{enumerate}
%\input{exemplar/10/13/3/41/main.tex}
If the letters of the word ASSASSINATION are arranged at random. Find the Probability that
\begin{enumerate}[label=(\alph*)]
\item Four $S's$ come consecutively in the word
\item Two  $I's$ and two $N's$ come together
\item All $A's$ are not coming together
\item No two $A's$ are coming together
\end{enumerate}
%\input{exemplar/11/16/3/14/main.tex}
	\item One urn contains two black balls (labelled B1 and B2) and one white ball. A
	second urn contains one black ball and two white balls (labelled W1 and W2).
	Suppose the following experiment is performed. One of the two urns is chosen
	at random. Next a ball is randomly chosen from the urn. Then a second ball is
	chosen at random from the same urn without replacing the first ball.
	
	\begin{enumerate}
	\item What is the probability that two black balls are chosen?
	
	\item What is the probability that two balls of opposite colour are chosen?
	\end{enumerate}
	\solution
	%\input{exemplar/11/16/3/12/main1.tex}
\end{enumerate}

	\item A bag contains $5$ red balls and some blue balls. If the probability of drawing a blue ball is double that if a red ball, determine the number of blue balls in the bag. 
		\\
\solution
		%\begin{enumerate}[label=\thesection.\arabic*,ref=\thesection.\theenumi]
	\item One card is drawn from a well-shuffled deck of 52 cards. Find the probability of getting
\begin{enumerate}
\item A king of red colour 
\item A face card 
\item A red face card
\item The jack of hearts
\item A spade
\item The queen of diamonds

\end{enumerate}
\solution
		%\input{ncert/10/15/1/14/main.tex}
	\item Five cards—the ten, jack, queen, king and ace of diamonds, are well-shuffled with their face downwards. One card is then picked up at random.
\begin{enumerate}
\item
What is the probability that the card is the queen? 
\item
If the queen is drawn and put aside, what is the probability that the second card picked up is (a) an ace? (b) a queen?\\
\end{enumerate}
\solution
		%\input{ncert/10/15/1/15/defs.tex}
	\item A bag contains $5$ red balls and some blue balls. If the probability of drawing a blue ball is double that if a red ball, determine the number of blue balls in the bag. 
		\\
\solution
		%\input{ncert/10/15/2/3/defs.tex}
	\item A card is selected from a pack of 52 cards.
 \begin{enumerate}[label=(\alph*)] 
                 \item How many points are there in the sample space?
                 \item Calculate the probability that the card is an ace of spades.
                 \item Calculate the probability that the card is (i) an ace and (ii) black card.
 \end{enumerate}
\solution
		%\input{ncert/11/16/3/4/main.tex}
\item Four cards are drawn from a well-shuffled deck of 52 cards. What is the probability of obtaining 3 diamonds and one spade.
\\
\solution
		%\input{ncert/11/16/4/2/defs.tex}
\item In a certain lottery 10,000 tickets are sold and ten equal prizes are awarded. What is the probability of not getting a prize if you buy (a) one ticket (b) two tickets (c) 10 tickets ?	
\\
\solution
		%\input{ncert/11/16/4/4/defs.tex}
		%
\item 
Out of 100 students, two sections of 40 and 60 are formed. If you and your friend are among the 100 students, what is the probability that
\begin{enumerate}
\item you both enter the same section?
\item you both enter the different sections?
\end{enumerate}
\solution
		%\input{ncert/11/16/4/5/defs.tex}
	\item 
The number lock of a suitcase has 4 wheels each labelled with ten digits i.e. from 0 to 9.The lock opens with a sequence of four digits with no repeats.What is the probability of a person getting the right sequence to open the suitcase.
\\
\solution
		%\input{ncert/11/16/4/10/defs.tex}
		%
\item 
Two cards are drawn at random and without replacement from a pack of 52 playing cards. Find the probability that both the cards are black.
\\
\solution
		%\input{ncert/12/13/2/2/defs.tex}
		\item A box of oranges is inspected by examining three randomly selected oranges drawn without replacement. If all the three oranges are good, the box is approved for sale, otherwise, it is rejected. Find the probability that a box containing 15 oranges out of which 12 are good and 3 are bad ones will be approved for sale.
		\label{ncert/12/13/2/3/defs.tex}
		\item Two balls are drawn at random with replacement from a box containing 10 black and 8 red balls. Find the probability that
		\label{ncert/12/13/2/12}
\begin{enumerate}
\item both balls are red.
\item first ball is black and second is red.
\item one of them is black and other is red.
\end{enumerate}

\item In a hostel, 60\% of the students read Hindi newspaper, 40\% read English newspaper and 20\% read both Hindi and English newspapers. A student is selected at random.
		\label{ncert/12/13/2/15}
\begin{enumerate}
\item Find the probability that she reads neither Hindi nor English newspapers.
\item If she reads Hindi newspaper, find the probability that she reads English newspaper.
\item If she reads English newspaper, find the probability that she reads Hindi newspaper.\\
\end{enumerate}
\item The probability of obtaining an even prime number on each die, when a pair of dice is rolled is 
\begin{enumerate}
    \item $0$ 
    
    \item $\frac{1}{3}$ 
    
    \item $\frac{1}{12}$ 
    
    \item $\frac{1}{36}$ 
\end{enumerate}
\solution
		%\input{ncert/12/13/2/17/defs.tex}
	\item A bag contains 4 red and 4 black balls, another bag contains 2 red and 6 black balls. One of the two bags is selected at random and a ball is drawn from the bag which is found to be red. Find the probability that the ball is drawn from the first bag.
\\
\solution
		%\input{ncert/12/13/3/2/main.tex}
  \item
  Cards with numbers 2 to 101 are placed in a box. A card is selected at random.Find the probability that the card has
\begin{enumerate}[label=(\roman*)]
	\item an even number 
	\item a square number
\end{enumerate}
\solution
%\input{exemplar/10/13/3/32/main.tex}
\item
The king, queen and jack of clubs are removed from a deck of 52 playing cards and then well shuffled. Now one card is drawn at random from the remaining cards.  Determine the probability that the card is
\begin{enumerate}[label=(\roman*)]
\item a club
\item 10 of hearts
\end{enumerate}
\solution
%\input{exemplar/10/13/3/29/main.tex}
\item A team of medical students doing their internship have to assist during surgeries
at a city hospital. The probabilities of surgeries rated as very complex, complex,
routine, simple or very simple are respectively, 0.15, 0.20, 0.31, 0.26, .08. Find
the probabilities that a particular surgery will be rated
\begin{enumerate}
	\item complex or very complex;
	\item neither very complex nor very simple;
	\item routine or complex
	\item routine or simple
\end{enumerate}
\solution
%\input{exemplar/11/16/3/8(1)/main.tex}
\item A card is selected from a pack of 52 cards.
\begin{enumerate}[label=(\alph*)]
    \item How many points are there in the sample space?
    \item Calculate the probability that the card is an ace of spades.
    \item Calculate the probability that the card is (i) an ace and (ii) black card.
\end{enumerate}
\solution
%\input{exemplar/11/16/3/4/main2.tex}
\item The probability that a non leap year selected at random will contain 53 sundays.
\\
\solution
%\input{exemplar/10/13/1/19/main.tex}
\item One of the four persons John, Rita, Aslam or Gurpreet will be promoted next
month. Consequently the sample space consists of four elementary outcomes
S = {John promoted, Rita promoted, Aslam promoted, Gurpreet promoted}
You are told that the chances of John’s promotion is same as that of Gurpreet,
Rita’s chances of promotion are twice as likely as Johns. Aslam’s chances are
four times that of John.
\begin{enumerate}
	\item Determine
	\begin{enumerate}
		\item P (John promoted)
		\item P (Rita promoted)
		\item P (Aslam promoted)
		\item P (Gurpreet promoted)
	\end{enumerate}
	\item If A = {John promoted or Gurpreet promoted}, find P (A).
\end{enumerate}
\solution
%\input{exemplar/11/16/3/10/main.tex}
\item A card is drawn from a deck of 52 cards. Find the probability of getting a king or a heart or a red card.\\
\solution
%\input{exemplar/11/16/3/15/main.tex}
\item The probability that a student will pass his examination is 0.73, the probability of
the student getting a compartment is 0.13, and the probability that the student will
either pass or get compartment is 0.96. State True or False.\\
\solution
%\input{exemplar/11/16/3/31/main.tex}
\item A card is selected from a pack of 52 cards\\
\begin{enumerate}[label=(\alph*)]
\item How many points are there in the sample space?
\item Calculate the probability that the cards is an ace of spades.
\item Calculate the probability that the card is (i) an ace (ii)black card.\\
\end{enumerate}
%\input{ncert/11/16/3/4_1/Prob_4.tex}
\item In a non-leap year, the probability of having 53 tuesdays or 53 wednesdays is\\
\solution
%\input{exemplar/11/16/3/18/main.tex}
\item There are 1000 sealed envelopes in a box, 10 of them contain a cash prize of
Rs 100 each, 100 of them contain a cash prize of Rs 50 each and 200 of them
contain a cash prize of Rs 10 each and rest do not contain any cash prize. If they
are well shuffled and an envelope is picked up out, what is the probability that it
contains no cash prize?\\
\solution
%\input{exemplar/10/13/3/34/main.tex}
\item 
A die is thrown and a card is selected at random from a deck of 52 playing cards. The probability of getting an even number on the die and a spade card.\\
\solution
%\input{exemplar/12/13/3/78/main.tex}
\item
If 4-digit numbers greater than 5,000 are randomly formed from the digits 0, 1, 3, 5, and 7, what is the probability of forming a number divisible by 5 when:
\begin{enumerate}
    \item The digits are repeated?
    \item The repetition of digits is not allowed?
\end{enumerate}
\solution
%\input{ncert/11/16/4/9/main.tex}
\item Consider the probability space $\brak{\Omega, \mathcal{G}, P}$ where $\Omega = [0,2]$ and $\mathcal{G} = \cbrak{\phi, \Omega, [0,1], (1,2]}$. Let $X$ and $Y$ be two functions on $\Omega$ defined as
\begin{align*}
    X(\omega) = 
    \begin{cases}
        1 & \text{if }\omega \in [0, 1]\\
        2 & \text{if }\omega \in (1, 2]
    \end{cases}
\end{align*}
and
\begin{align*}
    Y(\omega) = 
    \begin{cases}
        2 & \text{if }\omega \in [0, 1.5]\\
        3 & \text{if }\omega \in (1.5, 2].
    \end{cases}
\end{align*}
Then which one of the following statements is true?
\begin{enumerate}
    \item [(A)] $X$ is a random variable with respect to $\mathcal{G}$, but $Y$ is not a random variable with respect to $\mathcal{G}$.
    \item [(B)] $Y$ is a random variable with respect to $\mathcal{G}$, but $X$ is not a random variable with respect to $\mathcal{G}$.
    \item [(C)] Neither $X$ nor $Y$ is a random variable with respect to $\mathcal{G}$.
    \item [(D)] Both $X$ and $Y$ are random variables with respect to $\mathcal{G}$.
\end{enumerate} \hfill (GATE ST 2023)\\
\solution
%\input{gate/ST/2023/14/main.tex}
	\item  A die is loaded in such a way that each odd number is twice as likely to occur as
each even number. Find $P(G)$, where $G$ is the event that a number greater than
3 occurs on a single roll of the die.
\\
\solution
		%\input{exemplar/11/16/3/5/main.tex}
	\item All the jacks, queens and kings are removed from a deck of 52 playing cards. The remaining cards are well shuffled and then one card is drawn at random. Giving ace a value 1 similar value for other cards, find the probability that the card has a value 
		\begin{enumerate}
			\item 7
			\item greater than 7
			\item less than 7
		\end{enumerate}
		%\input{exemplar/10/13/3/30/main.tex}
  \item A Lot consists of 48 mobile phones of which 42 are good, 3 have only minor defects and 3 have major defects.Varnika will buy a phone if it is good but the trader will only buy a mobile if it has no major defects. One phone is selected at random from the lot. What is the probability that it is
\begin{enumerate}
	\item acceptable to Varnika?
            \item acceptable to the trader?
\end{enumerate}
\solution
	%\input{exemplar/10/13/3/40/main.tex}
 \item A student says that if you throw a die, it will show up 1 or not 1. Therefore, the probability of getting 1 and the probability of getting 'not 1' each is equal to $\frac{1}{2}$. Is this correct? Give reasons.\\
 \solution
        %\input{exemplar/10/13/2/9/main.tex}
   \item Four candidates A, B, C, D have ap-
plied for the assignment to coach a school cricket
team. If A is twice as likely to be selected as B, and
B and C are given about the same chance of being
selected, while C is twice as likely to be selected
as D, what are the probabilities that
\begin{enumerate}
\item C will be selected?
\item A will not be selected?
\end{enumerate}
	%\input{exemplar/11/16/3/9/main.tex}
 \item A bag contain 24 balls of which $x$ balls are red, $2x$ are white and $3x$ are blue. A ball is selected at random, What is the probability that it is
\begin{enumerate}[label=\alph*)]
\item not red ?
\item white ?
\end{enumerate}
%\input{exemplar/10/13/3/41/main.tex}
If the letters of the word ASSASSINATION are arranged at random. Find the Probability that
\begin{enumerate}[label=(\alph*)]
\item Four $S's$ come consecutively in the word
\item Two  $I's$ and two $N's$ come together
\item All $A's$ are not coming together
\item No two $A's$ are coming together
\end{enumerate}
%\input{exemplar/11/16/3/14/main.tex}
	\item One urn contains two black balls (labelled B1 and B2) and one white ball. A
	second urn contains one black ball and two white balls (labelled W1 and W2).
	Suppose the following experiment is performed. One of the two urns is chosen
	at random. Next a ball is randomly chosen from the urn. Then a second ball is
	chosen at random from the same urn without replacing the first ball.
	
	\begin{enumerate}
	\item What is the probability that two black balls are chosen?
	
	\item What is the probability that two balls of opposite colour are chosen?
	\end{enumerate}
	\solution
	%\input{exemplar/11/16/3/12/main1.tex}
\end{enumerate}

	\item A card is selected from a pack of 52 cards.
 \begin{enumerate}[label=(\alph*)] 
                 \item How many points are there in the sample space?
                 \item Calculate the probability that the card is an ace of spades.
                 \item Calculate the probability that the card is (i) an ace and (ii) black card.
 \end{enumerate}
\solution
		%\begin{table}[H]
	\centering
\begin{tabular}{|c|c|c|}
\hline
Random variable &Value &Definition\\ \hline
\multirow{3}{*}{X} &0 &Slips of Rs 1\\
&1 &Slips of Rs 5\\
&2 &Slips of Rs 13\\ \hline
\multirow{2}{*}{Y} &0 &Box A\\
&1 &Box B\\\hline
\end{tabular}
\caption{}
\label{tab:Distribution}
\end{table}
See \tabref{tab:Distribution}.
\begin{align}
p_{Y}\brak{k}= \begin{cases} 
      \frac{1}{3} & {k=0} \\
      \frac{2}{3 }& {k=1} 
   \end{cases}
   \\
p_{Y|X}\brak{0|0} = \frac{19}{25}\, 
p_{Y|X}\brak{0|1} = \frac{6}{25}\,
p_{Y|X}\brak{1|0} = \frac{45}{50}\,
p_{Y|X}\brak{1|2} = \frac{5}{50}
\end{align}
The desired probability is the probability that a slip drawn at random is marked other than Rs 1,
\begin{align}
&=1-p_X\brak{0}\\
&= p_X(1) + p_X(2)
\end{align}
Using Bayes theorem,
\begin{align}
&= p_Y\brak{0} \times \pr{Y=0 | X=1} + p_Y\brak{1} \times \pr{Y=1|X=2}\\
&=\frac{1}{3} \times \frac{6}{25} + \frac{2}{3} \times \frac{5}{50}\\
&=\frac{11}{75}
\end{align}

\newpage

%\tableofcontents

\bigskip

\renewcommand{\thefigure}{\theenumi}
\renewcommand{\thetable}{\theenumi}
%\renewcommand{\theequation}{\theenumi}

%\begin{abstract}
%%\boldmath
%In this letter, an algorithm for evaluating the exact analytical bit error rate  (BER)  for the piecewise linear (PL) combiner for  multiple relays is presented. Previous results were available only for upto three relays. The algorithm is unique in the sense that  the actual mathematical expressions, that are prohibitively large, need not be explicitly obtained. The diversity gain due to multiple relays is shown through plots of the analytical BER, well supported by simulations. 
%
%\end{abstract}
% IEEEtran.cls defaults to using nonbold math in the Abstract.
% This preserves the distinction between vectors and scalars. However,
% if the journal you are submitting to favors bold math in the abstract,
% then you can use LaTeX's standard command \boldmath at the very start
% of the abstract to achieve this. Many IEEE journals frown on math
% in the abstract anyway.

% Note that keywords are not normally used for peerreview papers.
%\begin{IEEEkeywords}
%Cooperative diversity, decode and forward, piecewise linear
%\end{IEEEkeywords}



% For peer review papers, you can put extra information on the cover
% page as needed:
% \ifCLASSOPTIONpeerreview
% \begin{center} \bfseries EDICS Category: 3-BBND \end{center}
% \fi
%
% For peerreview papers, this IEEEtran command inserts a page break and
% creates the second title. It will be ignored for other modes.
%\IEEEpeerreviewmaketitle




\item Four cards are drawn from a well-shuffled deck of 52 cards. What is the probability of obtaining 3 diamonds and one spade.
\\
\solution
		%\begin{enumerate}[label=\thesection.\arabic*,ref=\thesection.\theenumi]
	\item One card is drawn from a well-shuffled deck of 52 cards. Find the probability of getting
\begin{enumerate}
\item A king of red colour 
\item A face card 
\item A red face card
\item The jack of hearts
\item A spade
\item The queen of diamonds

\end{enumerate}
\solution
		%\input{ncert/10/15/1/14/main.tex}
	\item Five cards—the ten, jack, queen, king and ace of diamonds, are well-shuffled with their face downwards. One card is then picked up at random.
\begin{enumerate}
\item
What is the probability that the card is the queen? 
\item
If the queen is drawn and put aside, what is the probability that the second card picked up is (a) an ace? (b) a queen?\\
\end{enumerate}
\solution
		%\input{ncert/10/15/1/15/defs.tex}
	\item A bag contains $5$ red balls and some blue balls. If the probability of drawing a blue ball is double that if a red ball, determine the number of blue balls in the bag. 
		\\
\solution
		%\input{ncert/10/15/2/3/defs.tex}
	\item A card is selected from a pack of 52 cards.
 \begin{enumerate}[label=(\alph*)] 
                 \item How many points are there in the sample space?
                 \item Calculate the probability that the card is an ace of spades.
                 \item Calculate the probability that the card is (i) an ace and (ii) black card.
 \end{enumerate}
\solution
		%\input{ncert/11/16/3/4/main.tex}
\item Four cards are drawn from a well-shuffled deck of 52 cards. What is the probability of obtaining 3 diamonds and one spade.
\\
\solution
		%\input{ncert/11/16/4/2/defs.tex}
\item In a certain lottery 10,000 tickets are sold and ten equal prizes are awarded. What is the probability of not getting a prize if you buy (a) one ticket (b) two tickets (c) 10 tickets ?	
\\
\solution
		%\input{ncert/11/16/4/4/defs.tex}
		%
\item 
Out of 100 students, two sections of 40 and 60 are formed. If you and your friend are among the 100 students, what is the probability that
\begin{enumerate}
\item you both enter the same section?
\item you both enter the different sections?
\end{enumerate}
\solution
		%\input{ncert/11/16/4/5/defs.tex}
	\item 
The number lock of a suitcase has 4 wheels each labelled with ten digits i.e. from 0 to 9.The lock opens with a sequence of four digits with no repeats.What is the probability of a person getting the right sequence to open the suitcase.
\\
\solution
		%\input{ncert/11/16/4/10/defs.tex}
		%
\item 
Two cards are drawn at random and without replacement from a pack of 52 playing cards. Find the probability that both the cards are black.
\\
\solution
		%\input{ncert/12/13/2/2/defs.tex}
		\item A box of oranges is inspected by examining three randomly selected oranges drawn without replacement. If all the three oranges are good, the box is approved for sale, otherwise, it is rejected. Find the probability that a box containing 15 oranges out of which 12 are good and 3 are bad ones will be approved for sale.
		\label{ncert/12/13/2/3/defs.tex}
		\item Two balls are drawn at random with replacement from a box containing 10 black and 8 red balls. Find the probability that
		\label{ncert/12/13/2/12}
\begin{enumerate}
\item both balls are red.
\item first ball is black and second is red.
\item one of them is black and other is red.
\end{enumerate}

\item In a hostel, 60\% of the students read Hindi newspaper, 40\% read English newspaper and 20\% read both Hindi and English newspapers. A student is selected at random.
		\label{ncert/12/13/2/15}
\begin{enumerate}
\item Find the probability that she reads neither Hindi nor English newspapers.
\item If she reads Hindi newspaper, find the probability that she reads English newspaper.
\item If she reads English newspaper, find the probability that she reads Hindi newspaper.\\
\end{enumerate}
\item The probability of obtaining an even prime number on each die, when a pair of dice is rolled is 
\begin{enumerate}
    \item $0$ 
    
    \item $\frac{1}{3}$ 
    
    \item $\frac{1}{12}$ 
    
    \item $\frac{1}{36}$ 
\end{enumerate}
\solution
		%\input{ncert/12/13/2/17/defs.tex}
	\item A bag contains 4 red and 4 black balls, another bag contains 2 red and 6 black balls. One of the two bags is selected at random and a ball is drawn from the bag which is found to be red. Find the probability that the ball is drawn from the first bag.
\\
\solution
		%\input{ncert/12/13/3/2/main.tex}
  \item
  Cards with numbers 2 to 101 are placed in a box. A card is selected at random.Find the probability that the card has
\begin{enumerate}[label=(\roman*)]
	\item an even number 
	\item a square number
\end{enumerate}
\solution
%\input{exemplar/10/13/3/32/main.tex}
\item
The king, queen and jack of clubs are removed from a deck of 52 playing cards and then well shuffled. Now one card is drawn at random from the remaining cards.  Determine the probability that the card is
\begin{enumerate}[label=(\roman*)]
\item a club
\item 10 of hearts
\end{enumerate}
\solution
%\input{exemplar/10/13/3/29/main.tex}
\item A team of medical students doing their internship have to assist during surgeries
at a city hospital. The probabilities of surgeries rated as very complex, complex,
routine, simple or very simple are respectively, 0.15, 0.20, 0.31, 0.26, .08. Find
the probabilities that a particular surgery will be rated
\begin{enumerate}
	\item complex or very complex;
	\item neither very complex nor very simple;
	\item routine or complex
	\item routine or simple
\end{enumerate}
\solution
%\input{exemplar/11/16/3/8(1)/main.tex}
\item A card is selected from a pack of 52 cards.
\begin{enumerate}[label=(\alph*)]
    \item How many points are there in the sample space?
    \item Calculate the probability that the card is an ace of spades.
    \item Calculate the probability that the card is (i) an ace and (ii) black card.
\end{enumerate}
\solution
%\input{exemplar/11/16/3/4/main2.tex}
\item The probability that a non leap year selected at random will contain 53 sundays.
\\
\solution
%\input{exemplar/10/13/1/19/main.tex}
\item One of the four persons John, Rita, Aslam or Gurpreet will be promoted next
month. Consequently the sample space consists of four elementary outcomes
S = {John promoted, Rita promoted, Aslam promoted, Gurpreet promoted}
You are told that the chances of John’s promotion is same as that of Gurpreet,
Rita’s chances of promotion are twice as likely as Johns. Aslam’s chances are
four times that of John.
\begin{enumerate}
	\item Determine
	\begin{enumerate}
		\item P (John promoted)
		\item P (Rita promoted)
		\item P (Aslam promoted)
		\item P (Gurpreet promoted)
	\end{enumerate}
	\item If A = {John promoted or Gurpreet promoted}, find P (A).
\end{enumerate}
\solution
%\input{exemplar/11/16/3/10/main.tex}
\item A card is drawn from a deck of 52 cards. Find the probability of getting a king or a heart or a red card.\\
\solution
%\input{exemplar/11/16/3/15/main.tex}
\item The probability that a student will pass his examination is 0.73, the probability of
the student getting a compartment is 0.13, and the probability that the student will
either pass or get compartment is 0.96. State True or False.\\
\solution
%\input{exemplar/11/16/3/31/main.tex}
\item A card is selected from a pack of 52 cards\\
\begin{enumerate}[label=(\alph*)]
\item How many points are there in the sample space?
\item Calculate the probability that the cards is an ace of spades.
\item Calculate the probability that the card is (i) an ace (ii)black card.\\
\end{enumerate}
%\input{ncert/11/16/3/4_1/Prob_4.tex}
\item In a non-leap year, the probability of having 53 tuesdays or 53 wednesdays is\\
\solution
%\input{exemplar/11/16/3/18/main.tex}
\item There are 1000 sealed envelopes in a box, 10 of them contain a cash prize of
Rs 100 each, 100 of them contain a cash prize of Rs 50 each and 200 of them
contain a cash prize of Rs 10 each and rest do not contain any cash prize. If they
are well shuffled and an envelope is picked up out, what is the probability that it
contains no cash prize?\\
\solution
%\input{exemplar/10/13/3/34/main.tex}
\item 
A die is thrown and a card is selected at random from a deck of 52 playing cards. The probability of getting an even number on the die and a spade card.\\
\solution
%\input{exemplar/12/13/3/78/main.tex}
\item
If 4-digit numbers greater than 5,000 are randomly formed from the digits 0, 1, 3, 5, and 7, what is the probability of forming a number divisible by 5 when:
\begin{enumerate}
    \item The digits are repeated?
    \item The repetition of digits is not allowed?
\end{enumerate}
\solution
%\input{ncert/11/16/4/9/main.tex}
\item Consider the probability space $\brak{\Omega, \mathcal{G}, P}$ where $\Omega = [0,2]$ and $\mathcal{G} = \cbrak{\phi, \Omega, [0,1], (1,2]}$. Let $X$ and $Y$ be two functions on $\Omega$ defined as
\begin{align*}
    X(\omega) = 
    \begin{cases}
        1 & \text{if }\omega \in [0, 1]\\
        2 & \text{if }\omega \in (1, 2]
    \end{cases}
\end{align*}
and
\begin{align*}
    Y(\omega) = 
    \begin{cases}
        2 & \text{if }\omega \in [0, 1.5]\\
        3 & \text{if }\omega \in (1.5, 2].
    \end{cases}
\end{align*}
Then which one of the following statements is true?
\begin{enumerate}
    \item [(A)] $X$ is a random variable with respect to $\mathcal{G}$, but $Y$ is not a random variable with respect to $\mathcal{G}$.
    \item [(B)] $Y$ is a random variable with respect to $\mathcal{G}$, but $X$ is not a random variable with respect to $\mathcal{G}$.
    \item [(C)] Neither $X$ nor $Y$ is a random variable with respect to $\mathcal{G}$.
    \item [(D)] Both $X$ and $Y$ are random variables with respect to $\mathcal{G}$.
\end{enumerate} \hfill (GATE ST 2023)\\
\solution
%\input{gate/ST/2023/14/main.tex}
	\item  A die is loaded in such a way that each odd number is twice as likely to occur as
each even number. Find $P(G)$, where $G$ is the event that a number greater than
3 occurs on a single roll of the die.
\\
\solution
		%\input{exemplar/11/16/3/5/main.tex}
	\item All the jacks, queens and kings are removed from a deck of 52 playing cards. The remaining cards are well shuffled and then one card is drawn at random. Giving ace a value 1 similar value for other cards, find the probability that the card has a value 
		\begin{enumerate}
			\item 7
			\item greater than 7
			\item less than 7
		\end{enumerate}
		%\input{exemplar/10/13/3/30/main.tex}
  \item A Lot consists of 48 mobile phones of which 42 are good, 3 have only minor defects and 3 have major defects.Varnika will buy a phone if it is good but the trader will only buy a mobile if it has no major defects. One phone is selected at random from the lot. What is the probability that it is
\begin{enumerate}
	\item acceptable to Varnika?
            \item acceptable to the trader?
\end{enumerate}
\solution
	%\input{exemplar/10/13/3/40/main.tex}
 \item A student says that if you throw a die, it will show up 1 or not 1. Therefore, the probability of getting 1 and the probability of getting 'not 1' each is equal to $\frac{1}{2}$. Is this correct? Give reasons.\\
 \solution
        %\input{exemplar/10/13/2/9/main.tex}
   \item Four candidates A, B, C, D have ap-
plied for the assignment to coach a school cricket
team. If A is twice as likely to be selected as B, and
B and C are given about the same chance of being
selected, while C is twice as likely to be selected
as D, what are the probabilities that
\begin{enumerate}
\item C will be selected?
\item A will not be selected?
\end{enumerate}
	%\input{exemplar/11/16/3/9/main.tex}
 \item A bag contain 24 balls of which $x$ balls are red, $2x$ are white and $3x$ are blue. A ball is selected at random, What is the probability that it is
\begin{enumerate}[label=\alph*)]
\item not red ?
\item white ?
\end{enumerate}
%\input{exemplar/10/13/3/41/main.tex}
If the letters of the word ASSASSINATION are arranged at random. Find the Probability that
\begin{enumerate}[label=(\alph*)]
\item Four $S's$ come consecutively in the word
\item Two  $I's$ and two $N's$ come together
\item All $A's$ are not coming together
\item No two $A's$ are coming together
\end{enumerate}
%\input{exemplar/11/16/3/14/main.tex}
	\item One urn contains two black balls (labelled B1 and B2) and one white ball. A
	second urn contains one black ball and two white balls (labelled W1 and W2).
	Suppose the following experiment is performed. One of the two urns is chosen
	at random. Next a ball is randomly chosen from the urn. Then a second ball is
	chosen at random from the same urn without replacing the first ball.
	
	\begin{enumerate}
	\item What is the probability that two black balls are chosen?
	
	\item What is the probability that two balls of opposite colour are chosen?
	\end{enumerate}
	\solution
	%\input{exemplar/11/16/3/12/main1.tex}
\end{enumerate}

\item In a certain lottery 10,000 tickets are sold and ten equal prizes are awarded. What is the probability of not getting a prize if you buy (a) one ticket (b) two tickets (c) 10 tickets ?	
\\
\solution
		%\begin{enumerate}[label=\thesection.\arabic*,ref=\thesection.\theenumi]
	\item One card is drawn from a well-shuffled deck of 52 cards. Find the probability of getting
\begin{enumerate}
\item A king of red colour 
\item A face card 
\item A red face card
\item The jack of hearts
\item A spade
\item The queen of diamonds

\end{enumerate}
\solution
		%\input{ncert/10/15/1/14/main.tex}
	\item Five cards—the ten, jack, queen, king and ace of diamonds, are well-shuffled with their face downwards. One card is then picked up at random.
\begin{enumerate}
\item
What is the probability that the card is the queen? 
\item
If the queen is drawn and put aside, what is the probability that the second card picked up is (a) an ace? (b) a queen?\\
\end{enumerate}
\solution
		%\input{ncert/10/15/1/15/defs.tex}
	\item A bag contains $5$ red balls and some blue balls. If the probability of drawing a blue ball is double that if a red ball, determine the number of blue balls in the bag. 
		\\
\solution
		%\input{ncert/10/15/2/3/defs.tex}
	\item A card is selected from a pack of 52 cards.
 \begin{enumerate}[label=(\alph*)] 
                 \item How many points are there in the sample space?
                 \item Calculate the probability that the card is an ace of spades.
                 \item Calculate the probability that the card is (i) an ace and (ii) black card.
 \end{enumerate}
\solution
		%\input{ncert/11/16/3/4/main.tex}
\item Four cards are drawn from a well-shuffled deck of 52 cards. What is the probability of obtaining 3 diamonds and one spade.
\\
\solution
		%\input{ncert/11/16/4/2/defs.tex}
\item In a certain lottery 10,000 tickets are sold and ten equal prizes are awarded. What is the probability of not getting a prize if you buy (a) one ticket (b) two tickets (c) 10 tickets ?	
\\
\solution
		%\input{ncert/11/16/4/4/defs.tex}
		%
\item 
Out of 100 students, two sections of 40 and 60 are formed. If you and your friend are among the 100 students, what is the probability that
\begin{enumerate}
\item you both enter the same section?
\item you both enter the different sections?
\end{enumerate}
\solution
		%\input{ncert/11/16/4/5/defs.tex}
	\item 
The number lock of a suitcase has 4 wheels each labelled with ten digits i.e. from 0 to 9.The lock opens with a sequence of four digits with no repeats.What is the probability of a person getting the right sequence to open the suitcase.
\\
\solution
		%\input{ncert/11/16/4/10/defs.tex}
		%
\item 
Two cards are drawn at random and without replacement from a pack of 52 playing cards. Find the probability that both the cards are black.
\\
\solution
		%\input{ncert/12/13/2/2/defs.tex}
		\item A box of oranges is inspected by examining three randomly selected oranges drawn without replacement. If all the three oranges are good, the box is approved for sale, otherwise, it is rejected. Find the probability that a box containing 15 oranges out of which 12 are good and 3 are bad ones will be approved for sale.
		\label{ncert/12/13/2/3/defs.tex}
		\item Two balls are drawn at random with replacement from a box containing 10 black and 8 red balls. Find the probability that
		\label{ncert/12/13/2/12}
\begin{enumerate}
\item both balls are red.
\item first ball is black and second is red.
\item one of them is black and other is red.
\end{enumerate}

\item In a hostel, 60\% of the students read Hindi newspaper, 40\% read English newspaper and 20\% read both Hindi and English newspapers. A student is selected at random.
		\label{ncert/12/13/2/15}
\begin{enumerate}
\item Find the probability that she reads neither Hindi nor English newspapers.
\item If she reads Hindi newspaper, find the probability that she reads English newspaper.
\item If she reads English newspaper, find the probability that she reads Hindi newspaper.\\
\end{enumerate}
\item The probability of obtaining an even prime number on each die, when a pair of dice is rolled is 
\begin{enumerate}
    \item $0$ 
    
    \item $\frac{1}{3}$ 
    
    \item $\frac{1}{12}$ 
    
    \item $\frac{1}{36}$ 
\end{enumerate}
\solution
		%\input{ncert/12/13/2/17/defs.tex}
	\item A bag contains 4 red and 4 black balls, another bag contains 2 red and 6 black balls. One of the two bags is selected at random and a ball is drawn from the bag which is found to be red. Find the probability that the ball is drawn from the first bag.
\\
\solution
		%\input{ncert/12/13/3/2/main.tex}
  \item
  Cards with numbers 2 to 101 are placed in a box. A card is selected at random.Find the probability that the card has
\begin{enumerate}[label=(\roman*)]
	\item an even number 
	\item a square number
\end{enumerate}
\solution
%\input{exemplar/10/13/3/32/main.tex}
\item
The king, queen and jack of clubs are removed from a deck of 52 playing cards and then well shuffled. Now one card is drawn at random from the remaining cards.  Determine the probability that the card is
\begin{enumerate}[label=(\roman*)]
\item a club
\item 10 of hearts
\end{enumerate}
\solution
%\input{exemplar/10/13/3/29/main.tex}
\item A team of medical students doing their internship have to assist during surgeries
at a city hospital. The probabilities of surgeries rated as very complex, complex,
routine, simple or very simple are respectively, 0.15, 0.20, 0.31, 0.26, .08. Find
the probabilities that a particular surgery will be rated
\begin{enumerate}
	\item complex or very complex;
	\item neither very complex nor very simple;
	\item routine or complex
	\item routine or simple
\end{enumerate}
\solution
%\input{exemplar/11/16/3/8(1)/main.tex}
\item A card is selected from a pack of 52 cards.
\begin{enumerate}[label=(\alph*)]
    \item How many points are there in the sample space?
    \item Calculate the probability that the card is an ace of spades.
    \item Calculate the probability that the card is (i) an ace and (ii) black card.
\end{enumerate}
\solution
%\input{exemplar/11/16/3/4/main2.tex}
\item The probability that a non leap year selected at random will contain 53 sundays.
\\
\solution
%\input{exemplar/10/13/1/19/main.tex}
\item One of the four persons John, Rita, Aslam or Gurpreet will be promoted next
month. Consequently the sample space consists of four elementary outcomes
S = {John promoted, Rita promoted, Aslam promoted, Gurpreet promoted}
You are told that the chances of John’s promotion is same as that of Gurpreet,
Rita’s chances of promotion are twice as likely as Johns. Aslam’s chances are
four times that of John.
\begin{enumerate}
	\item Determine
	\begin{enumerate}
		\item P (John promoted)
		\item P (Rita promoted)
		\item P (Aslam promoted)
		\item P (Gurpreet promoted)
	\end{enumerate}
	\item If A = {John promoted or Gurpreet promoted}, find P (A).
\end{enumerate}
\solution
%\input{exemplar/11/16/3/10/main.tex}
\item A card is drawn from a deck of 52 cards. Find the probability of getting a king or a heart or a red card.\\
\solution
%\input{exemplar/11/16/3/15/main.tex}
\item The probability that a student will pass his examination is 0.73, the probability of
the student getting a compartment is 0.13, and the probability that the student will
either pass or get compartment is 0.96. State True or False.\\
\solution
%\input{exemplar/11/16/3/31/main.tex}
\item A card is selected from a pack of 52 cards\\
\begin{enumerate}[label=(\alph*)]
\item How many points are there in the sample space?
\item Calculate the probability that the cards is an ace of spades.
\item Calculate the probability that the card is (i) an ace (ii)black card.\\
\end{enumerate}
%\input{ncert/11/16/3/4_1/Prob_4.tex}
\item In a non-leap year, the probability of having 53 tuesdays or 53 wednesdays is\\
\solution
%\input{exemplar/11/16/3/18/main.tex}
\item There are 1000 sealed envelopes in a box, 10 of them contain a cash prize of
Rs 100 each, 100 of them contain a cash prize of Rs 50 each and 200 of them
contain a cash prize of Rs 10 each and rest do not contain any cash prize. If they
are well shuffled and an envelope is picked up out, what is the probability that it
contains no cash prize?\\
\solution
%\input{exemplar/10/13/3/34/main.tex}
\item 
A die is thrown and a card is selected at random from a deck of 52 playing cards. The probability of getting an even number on the die and a spade card.\\
\solution
%\input{exemplar/12/13/3/78/main.tex}
\item
If 4-digit numbers greater than 5,000 are randomly formed from the digits 0, 1, 3, 5, and 7, what is the probability of forming a number divisible by 5 when:
\begin{enumerate}
    \item The digits are repeated?
    \item The repetition of digits is not allowed?
\end{enumerate}
\solution
%\input{ncert/11/16/4/9/main.tex}
\item Consider the probability space $\brak{\Omega, \mathcal{G}, P}$ where $\Omega = [0,2]$ and $\mathcal{G} = \cbrak{\phi, \Omega, [0,1], (1,2]}$. Let $X$ and $Y$ be two functions on $\Omega$ defined as
\begin{align*}
    X(\omega) = 
    \begin{cases}
        1 & \text{if }\omega \in [0, 1]\\
        2 & \text{if }\omega \in (1, 2]
    \end{cases}
\end{align*}
and
\begin{align*}
    Y(\omega) = 
    \begin{cases}
        2 & \text{if }\omega \in [0, 1.5]\\
        3 & \text{if }\omega \in (1.5, 2].
    \end{cases}
\end{align*}
Then which one of the following statements is true?
\begin{enumerate}
    \item [(A)] $X$ is a random variable with respect to $\mathcal{G}$, but $Y$ is not a random variable with respect to $\mathcal{G}$.
    \item [(B)] $Y$ is a random variable with respect to $\mathcal{G}$, but $X$ is not a random variable with respect to $\mathcal{G}$.
    \item [(C)] Neither $X$ nor $Y$ is a random variable with respect to $\mathcal{G}$.
    \item [(D)] Both $X$ and $Y$ are random variables with respect to $\mathcal{G}$.
\end{enumerate} \hfill (GATE ST 2023)\\
\solution
%\input{gate/ST/2023/14/main.tex}
	\item  A die is loaded in such a way that each odd number is twice as likely to occur as
each even number. Find $P(G)$, where $G$ is the event that a number greater than
3 occurs on a single roll of the die.
\\
\solution
		%\input{exemplar/11/16/3/5/main.tex}
	\item All the jacks, queens and kings are removed from a deck of 52 playing cards. The remaining cards are well shuffled and then one card is drawn at random. Giving ace a value 1 similar value for other cards, find the probability that the card has a value 
		\begin{enumerate}
			\item 7
			\item greater than 7
			\item less than 7
		\end{enumerate}
		%\input{exemplar/10/13/3/30/main.tex}
  \item A Lot consists of 48 mobile phones of which 42 are good, 3 have only minor defects and 3 have major defects.Varnika will buy a phone if it is good but the trader will only buy a mobile if it has no major defects. One phone is selected at random from the lot. What is the probability that it is
\begin{enumerate}
	\item acceptable to Varnika?
            \item acceptable to the trader?
\end{enumerate}
\solution
	%\input{exemplar/10/13/3/40/main.tex}
 \item A student says that if you throw a die, it will show up 1 or not 1. Therefore, the probability of getting 1 and the probability of getting 'not 1' each is equal to $\frac{1}{2}$. Is this correct? Give reasons.\\
 \solution
        %\input{exemplar/10/13/2/9/main.tex}
   \item Four candidates A, B, C, D have ap-
plied for the assignment to coach a school cricket
team. If A is twice as likely to be selected as B, and
B and C are given about the same chance of being
selected, while C is twice as likely to be selected
as D, what are the probabilities that
\begin{enumerate}
\item C will be selected?
\item A will not be selected?
\end{enumerate}
	%\input{exemplar/11/16/3/9/main.tex}
 \item A bag contain 24 balls of which $x$ balls are red, $2x$ are white and $3x$ are blue. A ball is selected at random, What is the probability that it is
\begin{enumerate}[label=\alph*)]
\item not red ?
\item white ?
\end{enumerate}
%\input{exemplar/10/13/3/41/main.tex}
If the letters of the word ASSASSINATION are arranged at random. Find the Probability that
\begin{enumerate}[label=(\alph*)]
\item Four $S's$ come consecutively in the word
\item Two  $I's$ and two $N's$ come together
\item All $A's$ are not coming together
\item No two $A's$ are coming together
\end{enumerate}
%\input{exemplar/11/16/3/14/main.tex}
	\item One urn contains two black balls (labelled B1 and B2) and one white ball. A
	second urn contains one black ball and two white balls (labelled W1 and W2).
	Suppose the following experiment is performed. One of the two urns is chosen
	at random. Next a ball is randomly chosen from the urn. Then a second ball is
	chosen at random from the same urn without replacing the first ball.
	
	\begin{enumerate}
	\item What is the probability that two black balls are chosen?
	
	\item What is the probability that two balls of opposite colour are chosen?
	\end{enumerate}
	\solution
	%\input{exemplar/11/16/3/12/main1.tex}
\end{enumerate}

		%
\item 
Out of 100 students, two sections of 40 and 60 are formed. If you and your friend are among the 100 students, what is the probability that
\begin{enumerate}
\item you both enter the same section?
\item you both enter the different sections?
\end{enumerate}
\solution
		%\begin{enumerate}[label=\thesection.\arabic*,ref=\thesection.\theenumi]
	\item One card is drawn from a well-shuffled deck of 52 cards. Find the probability of getting
\begin{enumerate}
\item A king of red colour 
\item A face card 
\item A red face card
\item The jack of hearts
\item A spade
\item The queen of diamonds

\end{enumerate}
\solution
		%\input{ncert/10/15/1/14/main.tex}
	\item Five cards—the ten, jack, queen, king and ace of diamonds, are well-shuffled with their face downwards. One card is then picked up at random.
\begin{enumerate}
\item
What is the probability that the card is the queen? 
\item
If the queen is drawn and put aside, what is the probability that the second card picked up is (a) an ace? (b) a queen?\\
\end{enumerate}
\solution
		%\input{ncert/10/15/1/15/defs.tex}
	\item A bag contains $5$ red balls and some blue balls. If the probability of drawing a blue ball is double that if a red ball, determine the number of blue balls in the bag. 
		\\
\solution
		%\input{ncert/10/15/2/3/defs.tex}
	\item A card is selected from a pack of 52 cards.
 \begin{enumerate}[label=(\alph*)] 
                 \item How many points are there in the sample space?
                 \item Calculate the probability that the card is an ace of spades.
                 \item Calculate the probability that the card is (i) an ace and (ii) black card.
 \end{enumerate}
\solution
		%\input{ncert/11/16/3/4/main.tex}
\item Four cards are drawn from a well-shuffled deck of 52 cards. What is the probability of obtaining 3 diamonds and one spade.
\\
\solution
		%\input{ncert/11/16/4/2/defs.tex}
\item In a certain lottery 10,000 tickets are sold and ten equal prizes are awarded. What is the probability of not getting a prize if you buy (a) one ticket (b) two tickets (c) 10 tickets ?	
\\
\solution
		%\input{ncert/11/16/4/4/defs.tex}
		%
\item 
Out of 100 students, two sections of 40 and 60 are formed. If you and your friend are among the 100 students, what is the probability that
\begin{enumerate}
\item you both enter the same section?
\item you both enter the different sections?
\end{enumerate}
\solution
		%\input{ncert/11/16/4/5/defs.tex}
	\item 
The number lock of a suitcase has 4 wheels each labelled with ten digits i.e. from 0 to 9.The lock opens with a sequence of four digits with no repeats.What is the probability of a person getting the right sequence to open the suitcase.
\\
\solution
		%\input{ncert/11/16/4/10/defs.tex}
		%
\item 
Two cards are drawn at random and without replacement from a pack of 52 playing cards. Find the probability that both the cards are black.
\\
\solution
		%\input{ncert/12/13/2/2/defs.tex}
		\item A box of oranges is inspected by examining three randomly selected oranges drawn without replacement. If all the three oranges are good, the box is approved for sale, otherwise, it is rejected. Find the probability that a box containing 15 oranges out of which 12 are good and 3 are bad ones will be approved for sale.
		\label{ncert/12/13/2/3/defs.tex}
		\item Two balls are drawn at random with replacement from a box containing 10 black and 8 red balls. Find the probability that
		\label{ncert/12/13/2/12}
\begin{enumerate}
\item both balls are red.
\item first ball is black and second is red.
\item one of them is black and other is red.
\end{enumerate}

\item In a hostel, 60\% of the students read Hindi newspaper, 40\% read English newspaper and 20\% read both Hindi and English newspapers. A student is selected at random.
		\label{ncert/12/13/2/15}
\begin{enumerate}
\item Find the probability that she reads neither Hindi nor English newspapers.
\item If she reads Hindi newspaper, find the probability that she reads English newspaper.
\item If she reads English newspaper, find the probability that she reads Hindi newspaper.\\
\end{enumerate}
\item The probability of obtaining an even prime number on each die, when a pair of dice is rolled is 
\begin{enumerate}
    \item $0$ 
    
    \item $\frac{1}{3}$ 
    
    \item $\frac{1}{12}$ 
    
    \item $\frac{1}{36}$ 
\end{enumerate}
\solution
		%\input{ncert/12/13/2/17/defs.tex}
	\item A bag contains 4 red and 4 black balls, another bag contains 2 red and 6 black balls. One of the two bags is selected at random and a ball is drawn from the bag which is found to be red. Find the probability that the ball is drawn from the first bag.
\\
\solution
		%\input{ncert/12/13/3/2/main.tex}
  \item
  Cards with numbers 2 to 101 are placed in a box. A card is selected at random.Find the probability that the card has
\begin{enumerate}[label=(\roman*)]
	\item an even number 
	\item a square number
\end{enumerate}
\solution
%\input{exemplar/10/13/3/32/main.tex}
\item
The king, queen and jack of clubs are removed from a deck of 52 playing cards and then well shuffled. Now one card is drawn at random from the remaining cards.  Determine the probability that the card is
\begin{enumerate}[label=(\roman*)]
\item a club
\item 10 of hearts
\end{enumerate}
\solution
%\input{exemplar/10/13/3/29/main.tex}
\item A team of medical students doing their internship have to assist during surgeries
at a city hospital. The probabilities of surgeries rated as very complex, complex,
routine, simple or very simple are respectively, 0.15, 0.20, 0.31, 0.26, .08. Find
the probabilities that a particular surgery will be rated
\begin{enumerate}
	\item complex or very complex;
	\item neither very complex nor very simple;
	\item routine or complex
	\item routine or simple
\end{enumerate}
\solution
%\input{exemplar/11/16/3/8(1)/main.tex}
\item A card is selected from a pack of 52 cards.
\begin{enumerate}[label=(\alph*)]
    \item How many points are there in the sample space?
    \item Calculate the probability that the card is an ace of spades.
    \item Calculate the probability that the card is (i) an ace and (ii) black card.
\end{enumerate}
\solution
%\input{exemplar/11/16/3/4/main2.tex}
\item The probability that a non leap year selected at random will contain 53 sundays.
\\
\solution
%\input{exemplar/10/13/1/19/main.tex}
\item One of the four persons John, Rita, Aslam or Gurpreet will be promoted next
month. Consequently the sample space consists of four elementary outcomes
S = {John promoted, Rita promoted, Aslam promoted, Gurpreet promoted}
You are told that the chances of John’s promotion is same as that of Gurpreet,
Rita’s chances of promotion are twice as likely as Johns. Aslam’s chances are
four times that of John.
\begin{enumerate}
	\item Determine
	\begin{enumerate}
		\item P (John promoted)
		\item P (Rita promoted)
		\item P (Aslam promoted)
		\item P (Gurpreet promoted)
	\end{enumerate}
	\item If A = {John promoted or Gurpreet promoted}, find P (A).
\end{enumerate}
\solution
%\input{exemplar/11/16/3/10/main.tex}
\item A card is drawn from a deck of 52 cards. Find the probability of getting a king or a heart or a red card.\\
\solution
%\input{exemplar/11/16/3/15/main.tex}
\item The probability that a student will pass his examination is 0.73, the probability of
the student getting a compartment is 0.13, and the probability that the student will
either pass or get compartment is 0.96. State True or False.\\
\solution
%\input{exemplar/11/16/3/31/main.tex}
\item A card is selected from a pack of 52 cards\\
\begin{enumerate}[label=(\alph*)]
\item How many points are there in the sample space?
\item Calculate the probability that the cards is an ace of spades.
\item Calculate the probability that the card is (i) an ace (ii)black card.\\
\end{enumerate}
%\input{ncert/11/16/3/4_1/Prob_4.tex}
\item In a non-leap year, the probability of having 53 tuesdays or 53 wednesdays is\\
\solution
%\input{exemplar/11/16/3/18/main.tex}
\item There are 1000 sealed envelopes in a box, 10 of them contain a cash prize of
Rs 100 each, 100 of them contain a cash prize of Rs 50 each and 200 of them
contain a cash prize of Rs 10 each and rest do not contain any cash prize. If they
are well shuffled and an envelope is picked up out, what is the probability that it
contains no cash prize?\\
\solution
%\input{exemplar/10/13/3/34/main.tex}
\item 
A die is thrown and a card is selected at random from a deck of 52 playing cards. The probability of getting an even number on the die and a spade card.\\
\solution
%\input{exemplar/12/13/3/78/main.tex}
\item
If 4-digit numbers greater than 5,000 are randomly formed from the digits 0, 1, 3, 5, and 7, what is the probability of forming a number divisible by 5 when:
\begin{enumerate}
    \item The digits are repeated?
    \item The repetition of digits is not allowed?
\end{enumerate}
\solution
%\input{ncert/11/16/4/9/main.tex}
\item Consider the probability space $\brak{\Omega, \mathcal{G}, P}$ where $\Omega = [0,2]$ and $\mathcal{G} = \cbrak{\phi, \Omega, [0,1], (1,2]}$. Let $X$ and $Y$ be two functions on $\Omega$ defined as
\begin{align*}
    X(\omega) = 
    \begin{cases}
        1 & \text{if }\omega \in [0, 1]\\
        2 & \text{if }\omega \in (1, 2]
    \end{cases}
\end{align*}
and
\begin{align*}
    Y(\omega) = 
    \begin{cases}
        2 & \text{if }\omega \in [0, 1.5]\\
        3 & \text{if }\omega \in (1.5, 2].
    \end{cases}
\end{align*}
Then which one of the following statements is true?
\begin{enumerate}
    \item [(A)] $X$ is a random variable with respect to $\mathcal{G}$, but $Y$ is not a random variable with respect to $\mathcal{G}$.
    \item [(B)] $Y$ is a random variable with respect to $\mathcal{G}$, but $X$ is not a random variable with respect to $\mathcal{G}$.
    \item [(C)] Neither $X$ nor $Y$ is a random variable with respect to $\mathcal{G}$.
    \item [(D)] Both $X$ and $Y$ are random variables with respect to $\mathcal{G}$.
\end{enumerate} \hfill (GATE ST 2023)\\
\solution
%\input{gate/ST/2023/14/main.tex}
	\item  A die is loaded in such a way that each odd number is twice as likely to occur as
each even number. Find $P(G)$, where $G$ is the event that a number greater than
3 occurs on a single roll of the die.
\\
\solution
		%\input{exemplar/11/16/3/5/main.tex}
	\item All the jacks, queens and kings are removed from a deck of 52 playing cards. The remaining cards are well shuffled and then one card is drawn at random. Giving ace a value 1 similar value for other cards, find the probability that the card has a value 
		\begin{enumerate}
			\item 7
			\item greater than 7
			\item less than 7
		\end{enumerate}
		%\input{exemplar/10/13/3/30/main.tex}
  \item A Lot consists of 48 mobile phones of which 42 are good, 3 have only minor defects and 3 have major defects.Varnika will buy a phone if it is good but the trader will only buy a mobile if it has no major defects. One phone is selected at random from the lot. What is the probability that it is
\begin{enumerate}
	\item acceptable to Varnika?
            \item acceptable to the trader?
\end{enumerate}
\solution
	%\input{exemplar/10/13/3/40/main.tex}
 \item A student says that if you throw a die, it will show up 1 or not 1. Therefore, the probability of getting 1 and the probability of getting 'not 1' each is equal to $\frac{1}{2}$. Is this correct? Give reasons.\\
 \solution
        %\input{exemplar/10/13/2/9/main.tex}
   \item Four candidates A, B, C, D have ap-
plied for the assignment to coach a school cricket
team. If A is twice as likely to be selected as B, and
B and C are given about the same chance of being
selected, while C is twice as likely to be selected
as D, what are the probabilities that
\begin{enumerate}
\item C will be selected?
\item A will not be selected?
\end{enumerate}
	%\input{exemplar/11/16/3/9/main.tex}
 \item A bag contain 24 balls of which $x$ balls are red, $2x$ are white and $3x$ are blue. A ball is selected at random, What is the probability that it is
\begin{enumerate}[label=\alph*)]
\item not red ?
\item white ?
\end{enumerate}
%\input{exemplar/10/13/3/41/main.tex}
If the letters of the word ASSASSINATION are arranged at random. Find the Probability that
\begin{enumerate}[label=(\alph*)]
\item Four $S's$ come consecutively in the word
\item Two  $I's$ and two $N's$ come together
\item All $A's$ are not coming together
\item No two $A's$ are coming together
\end{enumerate}
%\input{exemplar/11/16/3/14/main.tex}
	\item One urn contains two black balls (labelled B1 and B2) and one white ball. A
	second urn contains one black ball and two white balls (labelled W1 and W2).
	Suppose the following experiment is performed. One of the two urns is chosen
	at random. Next a ball is randomly chosen from the urn. Then a second ball is
	chosen at random from the same urn without replacing the first ball.
	
	\begin{enumerate}
	\item What is the probability that two black balls are chosen?
	
	\item What is the probability that two balls of opposite colour are chosen?
	\end{enumerate}
	\solution
	%\input{exemplar/11/16/3/12/main1.tex}
\end{enumerate}

	\item 
The number lock of a suitcase has 4 wheels each labelled with ten digits i.e. from 0 to 9.The lock opens with a sequence of four digits with no repeats.What is the probability of a person getting the right sequence to open the suitcase.
\\
\solution
		%\begin{enumerate}[label=\thesection.\arabic*,ref=\thesection.\theenumi]
	\item One card is drawn from a well-shuffled deck of 52 cards. Find the probability of getting
\begin{enumerate}
\item A king of red colour 
\item A face card 
\item A red face card
\item The jack of hearts
\item A spade
\item The queen of diamonds

\end{enumerate}
\solution
		%\input{ncert/10/15/1/14/main.tex}
	\item Five cards—the ten, jack, queen, king and ace of diamonds, are well-shuffled with their face downwards. One card is then picked up at random.
\begin{enumerate}
\item
What is the probability that the card is the queen? 
\item
If the queen is drawn and put aside, what is the probability that the second card picked up is (a) an ace? (b) a queen?\\
\end{enumerate}
\solution
		%\input{ncert/10/15/1/15/defs.tex}
	\item A bag contains $5$ red balls and some blue balls. If the probability of drawing a blue ball is double that if a red ball, determine the number of blue balls in the bag. 
		\\
\solution
		%\input{ncert/10/15/2/3/defs.tex}
	\item A card is selected from a pack of 52 cards.
 \begin{enumerate}[label=(\alph*)] 
                 \item How many points are there in the sample space?
                 \item Calculate the probability that the card is an ace of spades.
                 \item Calculate the probability that the card is (i) an ace and (ii) black card.
 \end{enumerate}
\solution
		%\input{ncert/11/16/3/4/main.tex}
\item Four cards are drawn from a well-shuffled deck of 52 cards. What is the probability of obtaining 3 diamonds and one spade.
\\
\solution
		%\input{ncert/11/16/4/2/defs.tex}
\item In a certain lottery 10,000 tickets are sold and ten equal prizes are awarded. What is the probability of not getting a prize if you buy (a) one ticket (b) two tickets (c) 10 tickets ?	
\\
\solution
		%\input{ncert/11/16/4/4/defs.tex}
		%
\item 
Out of 100 students, two sections of 40 and 60 are formed. If you and your friend are among the 100 students, what is the probability that
\begin{enumerate}
\item you both enter the same section?
\item you both enter the different sections?
\end{enumerate}
\solution
		%\input{ncert/11/16/4/5/defs.tex}
	\item 
The number lock of a suitcase has 4 wheels each labelled with ten digits i.e. from 0 to 9.The lock opens with a sequence of four digits with no repeats.What is the probability of a person getting the right sequence to open the suitcase.
\\
\solution
		%\input{ncert/11/16/4/10/defs.tex}
		%
\item 
Two cards are drawn at random and without replacement from a pack of 52 playing cards. Find the probability that both the cards are black.
\\
\solution
		%\input{ncert/12/13/2/2/defs.tex}
		\item A box of oranges is inspected by examining three randomly selected oranges drawn without replacement. If all the three oranges are good, the box is approved for sale, otherwise, it is rejected. Find the probability that a box containing 15 oranges out of which 12 are good and 3 are bad ones will be approved for sale.
		\label{ncert/12/13/2/3/defs.tex}
		\item Two balls are drawn at random with replacement from a box containing 10 black and 8 red balls. Find the probability that
		\label{ncert/12/13/2/12}
\begin{enumerate}
\item both balls are red.
\item first ball is black and second is red.
\item one of them is black and other is red.
\end{enumerate}

\item In a hostel, 60\% of the students read Hindi newspaper, 40\% read English newspaper and 20\% read both Hindi and English newspapers. A student is selected at random.
		\label{ncert/12/13/2/15}
\begin{enumerate}
\item Find the probability that she reads neither Hindi nor English newspapers.
\item If she reads Hindi newspaper, find the probability that she reads English newspaper.
\item If she reads English newspaper, find the probability that she reads Hindi newspaper.\\
\end{enumerate}
\item The probability of obtaining an even prime number on each die, when a pair of dice is rolled is 
\begin{enumerate}
    \item $0$ 
    
    \item $\frac{1}{3}$ 
    
    \item $\frac{1}{12}$ 
    
    \item $\frac{1}{36}$ 
\end{enumerate}
\solution
		%\input{ncert/12/13/2/17/defs.tex}
	\item A bag contains 4 red and 4 black balls, another bag contains 2 red and 6 black balls. One of the two bags is selected at random and a ball is drawn from the bag which is found to be red. Find the probability that the ball is drawn from the first bag.
\\
\solution
		%\input{ncert/12/13/3/2/main.tex}
  \item
  Cards with numbers 2 to 101 are placed in a box. A card is selected at random.Find the probability that the card has
\begin{enumerate}[label=(\roman*)]
	\item an even number 
	\item a square number
\end{enumerate}
\solution
%\input{exemplar/10/13/3/32/main.tex}
\item
The king, queen and jack of clubs are removed from a deck of 52 playing cards and then well shuffled. Now one card is drawn at random from the remaining cards.  Determine the probability that the card is
\begin{enumerate}[label=(\roman*)]
\item a club
\item 10 of hearts
\end{enumerate}
\solution
%\input{exemplar/10/13/3/29/main.tex}
\item A team of medical students doing their internship have to assist during surgeries
at a city hospital. The probabilities of surgeries rated as very complex, complex,
routine, simple or very simple are respectively, 0.15, 0.20, 0.31, 0.26, .08. Find
the probabilities that a particular surgery will be rated
\begin{enumerate}
	\item complex or very complex;
	\item neither very complex nor very simple;
	\item routine or complex
	\item routine or simple
\end{enumerate}
\solution
%\input{exemplar/11/16/3/8(1)/main.tex}
\item A card is selected from a pack of 52 cards.
\begin{enumerate}[label=(\alph*)]
    \item How many points are there in the sample space?
    \item Calculate the probability that the card is an ace of spades.
    \item Calculate the probability that the card is (i) an ace and (ii) black card.
\end{enumerate}
\solution
%\input{exemplar/11/16/3/4/main2.tex}
\item The probability that a non leap year selected at random will contain 53 sundays.
\\
\solution
%\input{exemplar/10/13/1/19/main.tex}
\item One of the four persons John, Rita, Aslam or Gurpreet will be promoted next
month. Consequently the sample space consists of four elementary outcomes
S = {John promoted, Rita promoted, Aslam promoted, Gurpreet promoted}
You are told that the chances of John’s promotion is same as that of Gurpreet,
Rita’s chances of promotion are twice as likely as Johns. Aslam’s chances are
four times that of John.
\begin{enumerate}
	\item Determine
	\begin{enumerate}
		\item P (John promoted)
		\item P (Rita promoted)
		\item P (Aslam promoted)
		\item P (Gurpreet promoted)
	\end{enumerate}
	\item If A = {John promoted or Gurpreet promoted}, find P (A).
\end{enumerate}
\solution
%\input{exemplar/11/16/3/10/main.tex}
\item A card is drawn from a deck of 52 cards. Find the probability of getting a king or a heart or a red card.\\
\solution
%\input{exemplar/11/16/3/15/main.tex}
\item The probability that a student will pass his examination is 0.73, the probability of
the student getting a compartment is 0.13, and the probability that the student will
either pass or get compartment is 0.96. State True or False.\\
\solution
%\input{exemplar/11/16/3/31/main.tex}
\item A card is selected from a pack of 52 cards\\
\begin{enumerate}[label=(\alph*)]
\item How many points are there in the sample space?
\item Calculate the probability that the cards is an ace of spades.
\item Calculate the probability that the card is (i) an ace (ii)black card.\\
\end{enumerate}
%\input{ncert/11/16/3/4_1/Prob_4.tex}
\item In a non-leap year, the probability of having 53 tuesdays or 53 wednesdays is\\
\solution
%\input{exemplar/11/16/3/18/main.tex}
\item There are 1000 sealed envelopes in a box, 10 of them contain a cash prize of
Rs 100 each, 100 of them contain a cash prize of Rs 50 each and 200 of them
contain a cash prize of Rs 10 each and rest do not contain any cash prize. If they
are well shuffled and an envelope is picked up out, what is the probability that it
contains no cash prize?\\
\solution
%\input{exemplar/10/13/3/34/main.tex}
\item 
A die is thrown and a card is selected at random from a deck of 52 playing cards. The probability of getting an even number on the die and a spade card.\\
\solution
%\input{exemplar/12/13/3/78/main.tex}
\item
If 4-digit numbers greater than 5,000 are randomly formed from the digits 0, 1, 3, 5, and 7, what is the probability of forming a number divisible by 5 when:
\begin{enumerate}
    \item The digits are repeated?
    \item The repetition of digits is not allowed?
\end{enumerate}
\solution
%\input{ncert/11/16/4/9/main.tex}
\item Consider the probability space $\brak{\Omega, \mathcal{G}, P}$ where $\Omega = [0,2]$ and $\mathcal{G} = \cbrak{\phi, \Omega, [0,1], (1,2]}$. Let $X$ and $Y$ be two functions on $\Omega$ defined as
\begin{align*}
    X(\omega) = 
    \begin{cases}
        1 & \text{if }\omega \in [0, 1]\\
        2 & \text{if }\omega \in (1, 2]
    \end{cases}
\end{align*}
and
\begin{align*}
    Y(\omega) = 
    \begin{cases}
        2 & \text{if }\omega \in [0, 1.5]\\
        3 & \text{if }\omega \in (1.5, 2].
    \end{cases}
\end{align*}
Then which one of the following statements is true?
\begin{enumerate}
    \item [(A)] $X$ is a random variable with respect to $\mathcal{G}$, but $Y$ is not a random variable with respect to $\mathcal{G}$.
    \item [(B)] $Y$ is a random variable with respect to $\mathcal{G}$, but $X$ is not a random variable with respect to $\mathcal{G}$.
    \item [(C)] Neither $X$ nor $Y$ is a random variable with respect to $\mathcal{G}$.
    \item [(D)] Both $X$ and $Y$ are random variables with respect to $\mathcal{G}$.
\end{enumerate} \hfill (GATE ST 2023)\\
\solution
%\input{gate/ST/2023/14/main.tex}
	\item  A die is loaded in such a way that each odd number is twice as likely to occur as
each even number. Find $P(G)$, where $G$ is the event that a number greater than
3 occurs on a single roll of the die.
\\
\solution
		%\input{exemplar/11/16/3/5/main.tex}
	\item All the jacks, queens and kings are removed from a deck of 52 playing cards. The remaining cards are well shuffled and then one card is drawn at random. Giving ace a value 1 similar value for other cards, find the probability that the card has a value 
		\begin{enumerate}
			\item 7
			\item greater than 7
			\item less than 7
		\end{enumerate}
		%\input{exemplar/10/13/3/30/main.tex}
  \item A Lot consists of 48 mobile phones of which 42 are good, 3 have only minor defects and 3 have major defects.Varnika will buy a phone if it is good but the trader will only buy a mobile if it has no major defects. One phone is selected at random from the lot. What is the probability that it is
\begin{enumerate}
	\item acceptable to Varnika?
            \item acceptable to the trader?
\end{enumerate}
\solution
	%\input{exemplar/10/13/3/40/main.tex}
 \item A student says that if you throw a die, it will show up 1 or not 1. Therefore, the probability of getting 1 and the probability of getting 'not 1' each is equal to $\frac{1}{2}$. Is this correct? Give reasons.\\
 \solution
        %\input{exemplar/10/13/2/9/main.tex}
   \item Four candidates A, B, C, D have ap-
plied for the assignment to coach a school cricket
team. If A is twice as likely to be selected as B, and
B and C are given about the same chance of being
selected, while C is twice as likely to be selected
as D, what are the probabilities that
\begin{enumerate}
\item C will be selected?
\item A will not be selected?
\end{enumerate}
	%\input{exemplar/11/16/3/9/main.tex}
 \item A bag contain 24 balls of which $x$ balls are red, $2x$ are white and $3x$ are blue. A ball is selected at random, What is the probability that it is
\begin{enumerate}[label=\alph*)]
\item not red ?
\item white ?
\end{enumerate}
%\input{exemplar/10/13/3/41/main.tex}
If the letters of the word ASSASSINATION are arranged at random. Find the Probability that
\begin{enumerate}[label=(\alph*)]
\item Four $S's$ come consecutively in the word
\item Two  $I's$ and two $N's$ come together
\item All $A's$ are not coming together
\item No two $A's$ are coming together
\end{enumerate}
%\input{exemplar/11/16/3/14/main.tex}
	\item One urn contains two black balls (labelled B1 and B2) and one white ball. A
	second urn contains one black ball and two white balls (labelled W1 and W2).
	Suppose the following experiment is performed. One of the two urns is chosen
	at random. Next a ball is randomly chosen from the urn. Then a second ball is
	chosen at random from the same urn without replacing the first ball.
	
	\begin{enumerate}
	\item What is the probability that two black balls are chosen?
	
	\item What is the probability that two balls of opposite colour are chosen?
	\end{enumerate}
	\solution
	%\input{exemplar/11/16/3/12/main1.tex}
\end{enumerate}

		%
\item 
Two cards are drawn at random and without replacement from a pack of 52 playing cards. Find the probability that both the cards are black.
\\
\solution
		%\begin{enumerate}[label=\thesection.\arabic*,ref=\thesection.\theenumi]
	\item One card is drawn from a well-shuffled deck of 52 cards. Find the probability of getting
\begin{enumerate}
\item A king of red colour 
\item A face card 
\item A red face card
\item The jack of hearts
\item A spade
\item The queen of diamonds

\end{enumerate}
\solution
		%\input{ncert/10/15/1/14/main.tex}
	\item Five cards—the ten, jack, queen, king and ace of diamonds, are well-shuffled with their face downwards. One card is then picked up at random.
\begin{enumerate}
\item
What is the probability that the card is the queen? 
\item
If the queen is drawn and put aside, what is the probability that the second card picked up is (a) an ace? (b) a queen?\\
\end{enumerate}
\solution
		%\input{ncert/10/15/1/15/defs.tex}
	\item A bag contains $5$ red balls and some blue balls. If the probability of drawing a blue ball is double that if a red ball, determine the number of blue balls in the bag. 
		\\
\solution
		%\input{ncert/10/15/2/3/defs.tex}
	\item A card is selected from a pack of 52 cards.
 \begin{enumerate}[label=(\alph*)] 
                 \item How many points are there in the sample space?
                 \item Calculate the probability that the card is an ace of spades.
                 \item Calculate the probability that the card is (i) an ace and (ii) black card.
 \end{enumerate}
\solution
		%\input{ncert/11/16/3/4/main.tex}
\item Four cards are drawn from a well-shuffled deck of 52 cards. What is the probability of obtaining 3 diamonds and one spade.
\\
\solution
		%\input{ncert/11/16/4/2/defs.tex}
\item In a certain lottery 10,000 tickets are sold and ten equal prizes are awarded. What is the probability of not getting a prize if you buy (a) one ticket (b) two tickets (c) 10 tickets ?	
\\
\solution
		%\input{ncert/11/16/4/4/defs.tex}
		%
\item 
Out of 100 students, two sections of 40 and 60 are formed. If you and your friend are among the 100 students, what is the probability that
\begin{enumerate}
\item you both enter the same section?
\item you both enter the different sections?
\end{enumerate}
\solution
		%\input{ncert/11/16/4/5/defs.tex}
	\item 
The number lock of a suitcase has 4 wheels each labelled with ten digits i.e. from 0 to 9.The lock opens with a sequence of four digits with no repeats.What is the probability of a person getting the right sequence to open the suitcase.
\\
\solution
		%\input{ncert/11/16/4/10/defs.tex}
		%
\item 
Two cards are drawn at random and without replacement from a pack of 52 playing cards. Find the probability that both the cards are black.
\\
\solution
		%\input{ncert/12/13/2/2/defs.tex}
		\item A box of oranges is inspected by examining three randomly selected oranges drawn without replacement. If all the three oranges are good, the box is approved for sale, otherwise, it is rejected. Find the probability that a box containing 15 oranges out of which 12 are good and 3 are bad ones will be approved for sale.
		\label{ncert/12/13/2/3/defs.tex}
		\item Two balls are drawn at random with replacement from a box containing 10 black and 8 red balls. Find the probability that
		\label{ncert/12/13/2/12}
\begin{enumerate}
\item both balls are red.
\item first ball is black and second is red.
\item one of them is black and other is red.
\end{enumerate}

\item In a hostel, 60\% of the students read Hindi newspaper, 40\% read English newspaper and 20\% read both Hindi and English newspapers. A student is selected at random.
		\label{ncert/12/13/2/15}
\begin{enumerate}
\item Find the probability that she reads neither Hindi nor English newspapers.
\item If she reads Hindi newspaper, find the probability that she reads English newspaper.
\item If she reads English newspaper, find the probability that she reads Hindi newspaper.\\
\end{enumerate}
\item The probability of obtaining an even prime number on each die, when a pair of dice is rolled is 
\begin{enumerate}
    \item $0$ 
    
    \item $\frac{1}{3}$ 
    
    \item $\frac{1}{12}$ 
    
    \item $\frac{1}{36}$ 
\end{enumerate}
\solution
		%\input{ncert/12/13/2/17/defs.tex}
	\item A bag contains 4 red and 4 black balls, another bag contains 2 red and 6 black balls. One of the two bags is selected at random and a ball is drawn from the bag which is found to be red. Find the probability that the ball is drawn from the first bag.
\\
\solution
		%\input{ncert/12/13/3/2/main.tex}
  \item
  Cards with numbers 2 to 101 are placed in a box. A card is selected at random.Find the probability that the card has
\begin{enumerate}[label=(\roman*)]
	\item an even number 
	\item a square number
\end{enumerate}
\solution
%\input{exemplar/10/13/3/32/main.tex}
\item
The king, queen and jack of clubs are removed from a deck of 52 playing cards and then well shuffled. Now one card is drawn at random from the remaining cards.  Determine the probability that the card is
\begin{enumerate}[label=(\roman*)]
\item a club
\item 10 of hearts
\end{enumerate}
\solution
%\input{exemplar/10/13/3/29/main.tex}
\item A team of medical students doing their internship have to assist during surgeries
at a city hospital. The probabilities of surgeries rated as very complex, complex,
routine, simple or very simple are respectively, 0.15, 0.20, 0.31, 0.26, .08. Find
the probabilities that a particular surgery will be rated
\begin{enumerate}
	\item complex or very complex;
	\item neither very complex nor very simple;
	\item routine or complex
	\item routine or simple
\end{enumerate}
\solution
%\input{exemplar/11/16/3/8(1)/main.tex}
\item A card is selected from a pack of 52 cards.
\begin{enumerate}[label=(\alph*)]
    \item How many points are there in the sample space?
    \item Calculate the probability that the card is an ace of spades.
    \item Calculate the probability that the card is (i) an ace and (ii) black card.
\end{enumerate}
\solution
%\input{exemplar/11/16/3/4/main2.tex}
\item The probability that a non leap year selected at random will contain 53 sundays.
\\
\solution
%\input{exemplar/10/13/1/19/main.tex}
\item One of the four persons John, Rita, Aslam or Gurpreet will be promoted next
month. Consequently the sample space consists of four elementary outcomes
S = {John promoted, Rita promoted, Aslam promoted, Gurpreet promoted}
You are told that the chances of John’s promotion is same as that of Gurpreet,
Rita’s chances of promotion are twice as likely as Johns. Aslam’s chances are
four times that of John.
\begin{enumerate}
	\item Determine
	\begin{enumerate}
		\item P (John promoted)
		\item P (Rita promoted)
		\item P (Aslam promoted)
		\item P (Gurpreet promoted)
	\end{enumerate}
	\item If A = {John promoted or Gurpreet promoted}, find P (A).
\end{enumerate}
\solution
%\input{exemplar/11/16/3/10/main.tex}
\item A card is drawn from a deck of 52 cards. Find the probability of getting a king or a heart or a red card.\\
\solution
%\input{exemplar/11/16/3/15/main.tex}
\item The probability that a student will pass his examination is 0.73, the probability of
the student getting a compartment is 0.13, and the probability that the student will
either pass or get compartment is 0.96. State True or False.\\
\solution
%\input{exemplar/11/16/3/31/main.tex}
\item A card is selected from a pack of 52 cards\\
\begin{enumerate}[label=(\alph*)]
\item How many points are there in the sample space?
\item Calculate the probability that the cards is an ace of spades.
\item Calculate the probability that the card is (i) an ace (ii)black card.\\
\end{enumerate}
%\input{ncert/11/16/3/4_1/Prob_4.tex}
\item In a non-leap year, the probability of having 53 tuesdays or 53 wednesdays is\\
\solution
%\input{exemplar/11/16/3/18/main.tex}
\item There are 1000 sealed envelopes in a box, 10 of them contain a cash prize of
Rs 100 each, 100 of them contain a cash prize of Rs 50 each and 200 of them
contain a cash prize of Rs 10 each and rest do not contain any cash prize. If they
are well shuffled and an envelope is picked up out, what is the probability that it
contains no cash prize?\\
\solution
%\input{exemplar/10/13/3/34/main.tex}
\item 
A die is thrown and a card is selected at random from a deck of 52 playing cards. The probability of getting an even number on the die and a spade card.\\
\solution
%\input{exemplar/12/13/3/78/main.tex}
\item
If 4-digit numbers greater than 5,000 are randomly formed from the digits 0, 1, 3, 5, and 7, what is the probability of forming a number divisible by 5 when:
\begin{enumerate}
    \item The digits are repeated?
    \item The repetition of digits is not allowed?
\end{enumerate}
\solution
%\input{ncert/11/16/4/9/main.tex}
\item Consider the probability space $\brak{\Omega, \mathcal{G}, P}$ where $\Omega = [0,2]$ and $\mathcal{G} = \cbrak{\phi, \Omega, [0,1], (1,2]}$. Let $X$ and $Y$ be two functions on $\Omega$ defined as
\begin{align*}
    X(\omega) = 
    \begin{cases}
        1 & \text{if }\omega \in [0, 1]\\
        2 & \text{if }\omega \in (1, 2]
    \end{cases}
\end{align*}
and
\begin{align*}
    Y(\omega) = 
    \begin{cases}
        2 & \text{if }\omega \in [0, 1.5]\\
        3 & \text{if }\omega \in (1.5, 2].
    \end{cases}
\end{align*}
Then which one of the following statements is true?
\begin{enumerate}
    \item [(A)] $X$ is a random variable with respect to $\mathcal{G}$, but $Y$ is not a random variable with respect to $\mathcal{G}$.
    \item [(B)] $Y$ is a random variable with respect to $\mathcal{G}$, but $X$ is not a random variable with respect to $\mathcal{G}$.
    \item [(C)] Neither $X$ nor $Y$ is a random variable with respect to $\mathcal{G}$.
    \item [(D)] Both $X$ and $Y$ are random variables with respect to $\mathcal{G}$.
\end{enumerate} \hfill (GATE ST 2023)\\
\solution
%\input{gate/ST/2023/14/main.tex}
	\item  A die is loaded in such a way that each odd number is twice as likely to occur as
each even number. Find $P(G)$, where $G$ is the event that a number greater than
3 occurs on a single roll of the die.
\\
\solution
		%\input{exemplar/11/16/3/5/main.tex}
	\item All the jacks, queens and kings are removed from a deck of 52 playing cards. The remaining cards are well shuffled and then one card is drawn at random. Giving ace a value 1 similar value for other cards, find the probability that the card has a value 
		\begin{enumerate}
			\item 7
			\item greater than 7
			\item less than 7
		\end{enumerate}
		%\input{exemplar/10/13/3/30/main.tex}
  \item A Lot consists of 48 mobile phones of which 42 are good, 3 have only minor defects and 3 have major defects.Varnika will buy a phone if it is good but the trader will only buy a mobile if it has no major defects. One phone is selected at random from the lot. What is the probability that it is
\begin{enumerate}
	\item acceptable to Varnika?
            \item acceptable to the trader?
\end{enumerate}
\solution
	%\input{exemplar/10/13/3/40/main.tex}
 \item A student says that if you throw a die, it will show up 1 or not 1. Therefore, the probability of getting 1 and the probability of getting 'not 1' each is equal to $\frac{1}{2}$. Is this correct? Give reasons.\\
 \solution
        %\input{exemplar/10/13/2/9/main.tex}
   \item Four candidates A, B, C, D have ap-
plied for the assignment to coach a school cricket
team. If A is twice as likely to be selected as B, and
B and C are given about the same chance of being
selected, while C is twice as likely to be selected
as D, what are the probabilities that
\begin{enumerate}
\item C will be selected?
\item A will not be selected?
\end{enumerate}
	%\input{exemplar/11/16/3/9/main.tex}
 \item A bag contain 24 balls of which $x$ balls are red, $2x$ are white and $3x$ are blue. A ball is selected at random, What is the probability that it is
\begin{enumerate}[label=\alph*)]
\item not red ?
\item white ?
\end{enumerate}
%\input{exemplar/10/13/3/41/main.tex}
If the letters of the word ASSASSINATION are arranged at random. Find the Probability that
\begin{enumerate}[label=(\alph*)]
\item Four $S's$ come consecutively in the word
\item Two  $I's$ and two $N's$ come together
\item All $A's$ are not coming together
\item No two $A's$ are coming together
\end{enumerate}
%\input{exemplar/11/16/3/14/main.tex}
	\item One urn contains two black balls (labelled B1 and B2) and one white ball. A
	second urn contains one black ball and two white balls (labelled W1 and W2).
	Suppose the following experiment is performed. One of the two urns is chosen
	at random. Next a ball is randomly chosen from the urn. Then a second ball is
	chosen at random from the same urn without replacing the first ball.
	
	\begin{enumerate}
	\item What is the probability that two black balls are chosen?
	
	\item What is the probability that two balls of opposite colour are chosen?
	\end{enumerate}
	\solution
	%\input{exemplar/11/16/3/12/main1.tex}
\end{enumerate}

		\item A box of oranges is inspected by examining three randomly selected oranges drawn without replacement. If all the three oranges are good, the box is approved for sale, otherwise, it is rejected. Find the probability that a box containing 15 oranges out of which 12 are good and 3 are bad ones will be approved for sale.
		\label{ncert/12/13/2/3/defs.tex}
		\item Two balls are drawn at random with replacement from a box containing 10 black and 8 red balls. Find the probability that
		\label{ncert/12/13/2/12}
\begin{enumerate}
\item both balls are red.
\item first ball is black and second is red.
\item one of them is black and other is red.
\end{enumerate}

\item In a hostel, 60\% of the students read Hindi newspaper, 40\% read English newspaper and 20\% read both Hindi and English newspapers. A student is selected at random.
		\label{ncert/12/13/2/15}
\begin{enumerate}
\item Find the probability that she reads neither Hindi nor English newspapers.
\item If she reads Hindi newspaper, find the probability that she reads English newspaper.
\item If she reads English newspaper, find the probability that she reads Hindi newspaper.\\
\end{enumerate}
\item The probability of obtaining an even prime number on each die, when a pair of dice is rolled is 
\begin{enumerate}
    \item $0$ 
    
    \item $\frac{1}{3}$ 
    
    \item $\frac{1}{12}$ 
    
    \item $\frac{1}{36}$ 
\end{enumerate}
\solution
		%\begin{enumerate}[label=\thesection.\arabic*,ref=\thesection.\theenumi]
	\item One card is drawn from a well-shuffled deck of 52 cards. Find the probability of getting
\begin{enumerate}
\item A king of red colour 
\item A face card 
\item A red face card
\item The jack of hearts
\item A spade
\item The queen of diamonds

\end{enumerate}
\solution
		%\input{ncert/10/15/1/14/main.tex}
	\item Five cards—the ten, jack, queen, king and ace of diamonds, are well-shuffled with their face downwards. One card is then picked up at random.
\begin{enumerate}
\item
What is the probability that the card is the queen? 
\item
If the queen is drawn and put aside, what is the probability that the second card picked up is (a) an ace? (b) a queen?\\
\end{enumerate}
\solution
		%\input{ncert/10/15/1/15/defs.tex}
	\item A bag contains $5$ red balls and some blue balls. If the probability of drawing a blue ball is double that if a red ball, determine the number of blue balls in the bag. 
		\\
\solution
		%\input{ncert/10/15/2/3/defs.tex}
	\item A card is selected from a pack of 52 cards.
 \begin{enumerate}[label=(\alph*)] 
                 \item How many points are there in the sample space?
                 \item Calculate the probability that the card is an ace of spades.
                 \item Calculate the probability that the card is (i) an ace and (ii) black card.
 \end{enumerate}
\solution
		%\input{ncert/11/16/3/4/main.tex}
\item Four cards are drawn from a well-shuffled deck of 52 cards. What is the probability of obtaining 3 diamonds and one spade.
\\
\solution
		%\input{ncert/11/16/4/2/defs.tex}
\item In a certain lottery 10,000 tickets are sold and ten equal prizes are awarded. What is the probability of not getting a prize if you buy (a) one ticket (b) two tickets (c) 10 tickets ?	
\\
\solution
		%\input{ncert/11/16/4/4/defs.tex}
		%
\item 
Out of 100 students, two sections of 40 and 60 are formed. If you and your friend are among the 100 students, what is the probability that
\begin{enumerate}
\item you both enter the same section?
\item you both enter the different sections?
\end{enumerate}
\solution
		%\input{ncert/11/16/4/5/defs.tex}
	\item 
The number lock of a suitcase has 4 wheels each labelled with ten digits i.e. from 0 to 9.The lock opens with a sequence of four digits with no repeats.What is the probability of a person getting the right sequence to open the suitcase.
\\
\solution
		%\input{ncert/11/16/4/10/defs.tex}
		%
\item 
Two cards are drawn at random and without replacement from a pack of 52 playing cards. Find the probability that both the cards are black.
\\
\solution
		%\input{ncert/12/13/2/2/defs.tex}
		\item A box of oranges is inspected by examining three randomly selected oranges drawn without replacement. If all the three oranges are good, the box is approved for sale, otherwise, it is rejected. Find the probability that a box containing 15 oranges out of which 12 are good and 3 are bad ones will be approved for sale.
		\label{ncert/12/13/2/3/defs.tex}
		\item Two balls are drawn at random with replacement from a box containing 10 black and 8 red balls. Find the probability that
		\label{ncert/12/13/2/12}
\begin{enumerate}
\item both balls are red.
\item first ball is black and second is red.
\item one of them is black and other is red.
\end{enumerate}

\item In a hostel, 60\% of the students read Hindi newspaper, 40\% read English newspaper and 20\% read both Hindi and English newspapers. A student is selected at random.
		\label{ncert/12/13/2/15}
\begin{enumerate}
\item Find the probability that she reads neither Hindi nor English newspapers.
\item If she reads Hindi newspaper, find the probability that she reads English newspaper.
\item If she reads English newspaper, find the probability that she reads Hindi newspaper.\\
\end{enumerate}
\item The probability of obtaining an even prime number on each die, when a pair of dice is rolled is 
\begin{enumerate}
    \item $0$ 
    
    \item $\frac{1}{3}$ 
    
    \item $\frac{1}{12}$ 
    
    \item $\frac{1}{36}$ 
\end{enumerate}
\solution
		%\input{ncert/12/13/2/17/defs.tex}
	\item A bag contains 4 red and 4 black balls, another bag contains 2 red and 6 black balls. One of the two bags is selected at random and a ball is drawn from the bag which is found to be red. Find the probability that the ball is drawn from the first bag.
\\
\solution
		%\input{ncert/12/13/3/2/main.tex}
  \item
  Cards with numbers 2 to 101 are placed in a box. A card is selected at random.Find the probability that the card has
\begin{enumerate}[label=(\roman*)]
	\item an even number 
	\item a square number
\end{enumerate}
\solution
%\input{exemplar/10/13/3/32/main.tex}
\item
The king, queen and jack of clubs are removed from a deck of 52 playing cards and then well shuffled. Now one card is drawn at random from the remaining cards.  Determine the probability that the card is
\begin{enumerate}[label=(\roman*)]
\item a club
\item 10 of hearts
\end{enumerate}
\solution
%\input{exemplar/10/13/3/29/main.tex}
\item A team of medical students doing their internship have to assist during surgeries
at a city hospital. The probabilities of surgeries rated as very complex, complex,
routine, simple or very simple are respectively, 0.15, 0.20, 0.31, 0.26, .08. Find
the probabilities that a particular surgery will be rated
\begin{enumerate}
	\item complex or very complex;
	\item neither very complex nor very simple;
	\item routine or complex
	\item routine or simple
\end{enumerate}
\solution
%\input{exemplar/11/16/3/8(1)/main.tex}
\item A card is selected from a pack of 52 cards.
\begin{enumerate}[label=(\alph*)]
    \item How many points are there in the sample space?
    \item Calculate the probability that the card is an ace of spades.
    \item Calculate the probability that the card is (i) an ace and (ii) black card.
\end{enumerate}
\solution
%\input{exemplar/11/16/3/4/main2.tex}
\item The probability that a non leap year selected at random will contain 53 sundays.
\\
\solution
%\input{exemplar/10/13/1/19/main.tex}
\item One of the four persons John, Rita, Aslam or Gurpreet will be promoted next
month. Consequently the sample space consists of four elementary outcomes
S = {John promoted, Rita promoted, Aslam promoted, Gurpreet promoted}
You are told that the chances of John’s promotion is same as that of Gurpreet,
Rita’s chances of promotion are twice as likely as Johns. Aslam’s chances are
four times that of John.
\begin{enumerate}
	\item Determine
	\begin{enumerate}
		\item P (John promoted)
		\item P (Rita promoted)
		\item P (Aslam promoted)
		\item P (Gurpreet promoted)
	\end{enumerate}
	\item If A = {John promoted or Gurpreet promoted}, find P (A).
\end{enumerate}
\solution
%\input{exemplar/11/16/3/10/main.tex}
\item A card is drawn from a deck of 52 cards. Find the probability of getting a king or a heart or a red card.\\
\solution
%\input{exemplar/11/16/3/15/main.tex}
\item The probability that a student will pass his examination is 0.73, the probability of
the student getting a compartment is 0.13, and the probability that the student will
either pass or get compartment is 0.96. State True or False.\\
\solution
%\input{exemplar/11/16/3/31/main.tex}
\item A card is selected from a pack of 52 cards\\
\begin{enumerate}[label=(\alph*)]
\item How many points are there in the sample space?
\item Calculate the probability that the cards is an ace of spades.
\item Calculate the probability that the card is (i) an ace (ii)black card.\\
\end{enumerate}
%\input{ncert/11/16/3/4_1/Prob_4.tex}
\item In a non-leap year, the probability of having 53 tuesdays or 53 wednesdays is\\
\solution
%\input{exemplar/11/16/3/18/main.tex}
\item There are 1000 sealed envelopes in a box, 10 of them contain a cash prize of
Rs 100 each, 100 of them contain a cash prize of Rs 50 each and 200 of them
contain a cash prize of Rs 10 each and rest do not contain any cash prize. If they
are well shuffled and an envelope is picked up out, what is the probability that it
contains no cash prize?\\
\solution
%\input{exemplar/10/13/3/34/main.tex}
\item 
A die is thrown and a card is selected at random from a deck of 52 playing cards. The probability of getting an even number on the die and a spade card.\\
\solution
%\input{exemplar/12/13/3/78/main.tex}
\item
If 4-digit numbers greater than 5,000 are randomly formed from the digits 0, 1, 3, 5, and 7, what is the probability of forming a number divisible by 5 when:
\begin{enumerate}
    \item The digits are repeated?
    \item The repetition of digits is not allowed?
\end{enumerate}
\solution
%\input{ncert/11/16/4/9/main.tex}
\item Consider the probability space $\brak{\Omega, \mathcal{G}, P}$ where $\Omega = [0,2]$ and $\mathcal{G} = \cbrak{\phi, \Omega, [0,1], (1,2]}$. Let $X$ and $Y$ be two functions on $\Omega$ defined as
\begin{align*}
    X(\omega) = 
    \begin{cases}
        1 & \text{if }\omega \in [0, 1]\\
        2 & \text{if }\omega \in (1, 2]
    \end{cases}
\end{align*}
and
\begin{align*}
    Y(\omega) = 
    \begin{cases}
        2 & \text{if }\omega \in [0, 1.5]\\
        3 & \text{if }\omega \in (1.5, 2].
    \end{cases}
\end{align*}
Then which one of the following statements is true?
\begin{enumerate}
    \item [(A)] $X$ is a random variable with respect to $\mathcal{G}$, but $Y$ is not a random variable with respect to $\mathcal{G}$.
    \item [(B)] $Y$ is a random variable with respect to $\mathcal{G}$, but $X$ is not a random variable with respect to $\mathcal{G}$.
    \item [(C)] Neither $X$ nor $Y$ is a random variable with respect to $\mathcal{G}$.
    \item [(D)] Both $X$ and $Y$ are random variables with respect to $\mathcal{G}$.
\end{enumerate} \hfill (GATE ST 2023)\\
\solution
%\input{gate/ST/2023/14/main.tex}
	\item  A die is loaded in such a way that each odd number is twice as likely to occur as
each even number. Find $P(G)$, where $G$ is the event that a number greater than
3 occurs on a single roll of the die.
\\
\solution
		%\input{exemplar/11/16/3/5/main.tex}
	\item All the jacks, queens and kings are removed from a deck of 52 playing cards. The remaining cards are well shuffled and then one card is drawn at random. Giving ace a value 1 similar value for other cards, find the probability that the card has a value 
		\begin{enumerate}
			\item 7
			\item greater than 7
			\item less than 7
		\end{enumerate}
		%\input{exemplar/10/13/3/30/main.tex}
  \item A Lot consists of 48 mobile phones of which 42 are good, 3 have only minor defects and 3 have major defects.Varnika will buy a phone if it is good but the trader will only buy a mobile if it has no major defects. One phone is selected at random from the lot. What is the probability that it is
\begin{enumerate}
	\item acceptable to Varnika?
            \item acceptable to the trader?
\end{enumerate}
\solution
	%\input{exemplar/10/13/3/40/main.tex}
 \item A student says that if you throw a die, it will show up 1 or not 1. Therefore, the probability of getting 1 and the probability of getting 'not 1' each is equal to $\frac{1}{2}$. Is this correct? Give reasons.\\
 \solution
        %\input{exemplar/10/13/2/9/main.tex}
   \item Four candidates A, B, C, D have ap-
plied for the assignment to coach a school cricket
team. If A is twice as likely to be selected as B, and
B and C are given about the same chance of being
selected, while C is twice as likely to be selected
as D, what are the probabilities that
\begin{enumerate}
\item C will be selected?
\item A will not be selected?
\end{enumerate}
	%\input{exemplar/11/16/3/9/main.tex}
 \item A bag contain 24 balls of which $x$ balls are red, $2x$ are white and $3x$ are blue. A ball is selected at random, What is the probability that it is
\begin{enumerate}[label=\alph*)]
\item not red ?
\item white ?
\end{enumerate}
%\input{exemplar/10/13/3/41/main.tex}
If the letters of the word ASSASSINATION are arranged at random. Find the Probability that
\begin{enumerate}[label=(\alph*)]
\item Four $S's$ come consecutively in the word
\item Two  $I's$ and two $N's$ come together
\item All $A's$ are not coming together
\item No two $A's$ are coming together
\end{enumerate}
%\input{exemplar/11/16/3/14/main.tex}
	\item One urn contains two black balls (labelled B1 and B2) and one white ball. A
	second urn contains one black ball and two white balls (labelled W1 and W2).
	Suppose the following experiment is performed. One of the two urns is chosen
	at random. Next a ball is randomly chosen from the urn. Then a second ball is
	chosen at random from the same urn without replacing the first ball.
	
	\begin{enumerate}
	\item What is the probability that two black balls are chosen?
	
	\item What is the probability that two balls of opposite colour are chosen?
	\end{enumerate}
	\solution
	%\input{exemplar/11/16/3/12/main1.tex}
\end{enumerate}

	\item A bag contains 4 red and 4 black balls, another bag contains 2 red and 6 black balls. One of the two bags is selected at random and a ball is drawn from the bag which is found to be red. Find the probability that the ball is drawn from the first bag.
\\
\solution
		%\begin{table}[H]
	\centering
\begin{tabular}{|c|c|c|}
\hline
Random variable &Value &Definition\\ \hline
\multirow{3}{*}{X} &0 &Slips of Rs 1\\
&1 &Slips of Rs 5\\
&2 &Slips of Rs 13\\ \hline
\multirow{2}{*}{Y} &0 &Box A\\
&1 &Box B\\\hline
\end{tabular}
\caption{}
\label{tab:Distribution}
\end{table}
See \tabref{tab:Distribution}.
\begin{align}
p_{Y}\brak{k}= \begin{cases} 
      \frac{1}{3} & {k=0} \\
      \frac{2}{3 }& {k=1} 
   \end{cases}
   \\
p_{Y|X}\brak{0|0} = \frac{19}{25}\, 
p_{Y|X}\brak{0|1} = \frac{6}{25}\,
p_{Y|X}\brak{1|0} = \frac{45}{50}\,
p_{Y|X}\brak{1|2} = \frac{5}{50}
\end{align}
The desired probability is the probability that a slip drawn at random is marked other than Rs 1,
\begin{align}
&=1-p_X\brak{0}\\
&= p_X(1) + p_X(2)
\end{align}
Using Bayes theorem,
\begin{align}
&= p_Y\brak{0} \times \pr{Y=0 | X=1} + p_Y\brak{1} \times \pr{Y=1|X=2}\\
&=\frac{1}{3} \times \frac{6}{25} + \frac{2}{3} \times \frac{5}{50}\\
&=\frac{11}{75}
\end{align}

\newpage

%\tableofcontents

\bigskip

\renewcommand{\thefigure}{\theenumi}
\renewcommand{\thetable}{\theenumi}
%\renewcommand{\theequation}{\theenumi}

%\begin{abstract}
%%\boldmath
%In this letter, an algorithm for evaluating the exact analytical bit error rate  (BER)  for the piecewise linear (PL) combiner for  multiple relays is presented. Previous results were available only for upto three relays. The algorithm is unique in the sense that  the actual mathematical expressions, that are prohibitively large, need not be explicitly obtained. The diversity gain due to multiple relays is shown through plots of the analytical BER, well supported by simulations. 
%
%\end{abstract}
% IEEEtran.cls defaults to using nonbold math in the Abstract.
% This preserves the distinction between vectors and scalars. However,
% if the journal you are submitting to favors bold math in the abstract,
% then you can use LaTeX's standard command \boldmath at the very start
% of the abstract to achieve this. Many IEEE journals frown on math
% in the abstract anyway.

% Note that keywords are not normally used for peerreview papers.
%\begin{IEEEkeywords}
%Cooperative diversity, decode and forward, piecewise linear
%\end{IEEEkeywords}



% For peer review papers, you can put extra information on the cover
% page as needed:
% \ifCLASSOPTIONpeerreview
% \begin{center} \bfseries EDICS Category: 3-BBND \end{center}
% \fi
%
% For peerreview papers, this IEEEtran command inserts a page break and
% creates the second title. It will be ignored for other modes.
%\IEEEpeerreviewmaketitle




  \item
  Cards with numbers 2 to 101 are placed in a box. A card is selected at random.Find the probability that the card has
\begin{enumerate}[label=(\roman*)]
	\item an even number 
	\item a square number
\end{enumerate}
\solution
%\begin{table}[H]
	\centering
\begin{tabular}{|c|c|c|}
\hline
Random variable &Value &Definition\\ \hline
\multirow{3}{*}{X} &0 &Slips of Rs 1\\
&1 &Slips of Rs 5\\
&2 &Slips of Rs 13\\ \hline
\multirow{2}{*}{Y} &0 &Box A\\
&1 &Box B\\\hline
\end{tabular}
\caption{}
\label{tab:Distribution}
\end{table}
See \tabref{tab:Distribution}.
\begin{align}
p_{Y}\brak{k}= \begin{cases} 
      \frac{1}{3} & {k=0} \\
      \frac{2}{3 }& {k=1} 
   \end{cases}
   \\
p_{Y|X}\brak{0|0} = \frac{19}{25}\, 
p_{Y|X}\brak{0|1} = \frac{6}{25}\,
p_{Y|X}\brak{1|0} = \frac{45}{50}\,
p_{Y|X}\brak{1|2} = \frac{5}{50}
\end{align}
The desired probability is the probability that a slip drawn at random is marked other than Rs 1,
\begin{align}
&=1-p_X\brak{0}\\
&= p_X(1) + p_X(2)
\end{align}
Using Bayes theorem,
\begin{align}
&= p_Y\brak{0} \times \pr{Y=0 | X=1} + p_Y\brak{1} \times \pr{Y=1|X=2}\\
&=\frac{1}{3} \times \frac{6}{25} + \frac{2}{3} \times \frac{5}{50}\\
&=\frac{11}{75}
\end{align}

\newpage

%\tableofcontents

\bigskip

\renewcommand{\thefigure}{\theenumi}
\renewcommand{\thetable}{\theenumi}
%\renewcommand{\theequation}{\theenumi}

%\begin{abstract}
%%\boldmath
%In this letter, an algorithm for evaluating the exact analytical bit error rate  (BER)  for the piecewise linear (PL) combiner for  multiple relays is presented. Previous results were available only for upto three relays. The algorithm is unique in the sense that  the actual mathematical expressions, that are prohibitively large, need not be explicitly obtained. The diversity gain due to multiple relays is shown through plots of the analytical BER, well supported by simulations. 
%
%\end{abstract}
% IEEEtran.cls defaults to using nonbold math in the Abstract.
% This preserves the distinction between vectors and scalars. However,
% if the journal you are submitting to favors bold math in the abstract,
% then you can use LaTeX's standard command \boldmath at the very start
% of the abstract to achieve this. Many IEEE journals frown on math
% in the abstract anyway.

% Note that keywords are not normally used for peerreview papers.
%\begin{IEEEkeywords}
%Cooperative diversity, decode and forward, piecewise linear
%\end{IEEEkeywords}



% For peer review papers, you can put extra information on the cover
% page as needed:
% \ifCLASSOPTIONpeerreview
% \begin{center} \bfseries EDICS Category: 3-BBND \end{center}
% \fi
%
% For peerreview papers, this IEEEtran command inserts a page break and
% creates the second title. It will be ignored for other modes.
%\IEEEpeerreviewmaketitle




\item
The king, queen and jack of clubs are removed from a deck of 52 playing cards and then well shuffled. Now one card is drawn at random from the remaining cards.  Determine the probability that the card is
\begin{enumerate}[label=(\roman*)]
\item a club
\item 10 of hearts
\end{enumerate}
\solution
%\begin{table}[H]
	\centering
\begin{tabular}{|c|c|c|}
\hline
Random variable &Value &Definition\\ \hline
\multirow{3}{*}{X} &0 &Slips of Rs 1\\
&1 &Slips of Rs 5\\
&2 &Slips of Rs 13\\ \hline
\multirow{2}{*}{Y} &0 &Box A\\
&1 &Box B\\\hline
\end{tabular}
\caption{}
\label{tab:Distribution}
\end{table}
See \tabref{tab:Distribution}.
\begin{align}
p_{Y}\brak{k}= \begin{cases} 
      \frac{1}{3} & {k=0} \\
      \frac{2}{3 }& {k=1} 
   \end{cases}
   \\
p_{Y|X}\brak{0|0} = \frac{19}{25}\, 
p_{Y|X}\brak{0|1} = \frac{6}{25}\,
p_{Y|X}\brak{1|0} = \frac{45}{50}\,
p_{Y|X}\brak{1|2} = \frac{5}{50}
\end{align}
The desired probability is the probability that a slip drawn at random is marked other than Rs 1,
\begin{align}
&=1-p_X\brak{0}\\
&= p_X(1) + p_X(2)
\end{align}
Using Bayes theorem,
\begin{align}
&= p_Y\brak{0} \times \pr{Y=0 | X=1} + p_Y\brak{1} \times \pr{Y=1|X=2}\\
&=\frac{1}{3} \times \frac{6}{25} + \frac{2}{3} \times \frac{5}{50}\\
&=\frac{11}{75}
\end{align}

\newpage

%\tableofcontents

\bigskip

\renewcommand{\thefigure}{\theenumi}
\renewcommand{\thetable}{\theenumi}
%\renewcommand{\theequation}{\theenumi}

%\begin{abstract}
%%\boldmath
%In this letter, an algorithm for evaluating the exact analytical bit error rate  (BER)  for the piecewise linear (PL) combiner for  multiple relays is presented. Previous results were available only for upto three relays. The algorithm is unique in the sense that  the actual mathematical expressions, that are prohibitively large, need not be explicitly obtained. The diversity gain due to multiple relays is shown through plots of the analytical BER, well supported by simulations. 
%
%\end{abstract}
% IEEEtran.cls defaults to using nonbold math in the Abstract.
% This preserves the distinction between vectors and scalars. However,
% if the journal you are submitting to favors bold math in the abstract,
% then you can use LaTeX's standard command \boldmath at the very start
% of the abstract to achieve this. Many IEEE journals frown on math
% in the abstract anyway.

% Note that keywords are not normally used for peerreview papers.
%\begin{IEEEkeywords}
%Cooperative diversity, decode and forward, piecewise linear
%\end{IEEEkeywords}



% For peer review papers, you can put extra information on the cover
% page as needed:
% \ifCLASSOPTIONpeerreview
% \begin{center} \bfseries EDICS Category: 3-BBND \end{center}
% \fi
%
% For peerreview papers, this IEEEtran command inserts a page break and
% creates the second title. It will be ignored for other modes.
%\IEEEpeerreviewmaketitle




\item A team of medical students doing their internship have to assist during surgeries
at a city hospital. The probabilities of surgeries rated as very complex, complex,
routine, simple or very simple are respectively, 0.15, 0.20, 0.31, 0.26, .08. Find
the probabilities that a particular surgery will be rated
\begin{enumerate}
	\item complex or very complex;
	\item neither very complex nor very simple;
	\item routine or complex
	\item routine or simple
\end{enumerate}
\solution
%\begin{table}[H]
	\centering
\begin{tabular}{|c|c|c|}
\hline
Random variable &Value &Definition\\ \hline
\multirow{3}{*}{X} &0 &Slips of Rs 1\\
&1 &Slips of Rs 5\\
&2 &Slips of Rs 13\\ \hline
\multirow{2}{*}{Y} &0 &Box A\\
&1 &Box B\\\hline
\end{tabular}
\caption{}
\label{tab:Distribution}
\end{table}
See \tabref{tab:Distribution}.
\begin{align}
p_{Y}\brak{k}= \begin{cases} 
      \frac{1}{3} & {k=0} \\
      \frac{2}{3 }& {k=1} 
   \end{cases}
   \\
p_{Y|X}\brak{0|0} = \frac{19}{25}\, 
p_{Y|X}\brak{0|1} = \frac{6}{25}\,
p_{Y|X}\brak{1|0} = \frac{45}{50}\,
p_{Y|X}\brak{1|2} = \frac{5}{50}
\end{align}
The desired probability is the probability that a slip drawn at random is marked other than Rs 1,
\begin{align}
&=1-p_X\brak{0}\\
&= p_X(1) + p_X(2)
\end{align}
Using Bayes theorem,
\begin{align}
&= p_Y\brak{0} \times \pr{Y=0 | X=1} + p_Y\brak{1} \times \pr{Y=1|X=2}\\
&=\frac{1}{3} \times \frac{6}{25} + \frac{2}{3} \times \frac{5}{50}\\
&=\frac{11}{75}
\end{align}

\newpage

%\tableofcontents

\bigskip

\renewcommand{\thefigure}{\theenumi}
\renewcommand{\thetable}{\theenumi}
%\renewcommand{\theequation}{\theenumi}

%\begin{abstract}
%%\boldmath
%In this letter, an algorithm for evaluating the exact analytical bit error rate  (BER)  for the piecewise linear (PL) combiner for  multiple relays is presented. Previous results were available only for upto three relays. The algorithm is unique in the sense that  the actual mathematical expressions, that are prohibitively large, need not be explicitly obtained. The diversity gain due to multiple relays is shown through plots of the analytical BER, well supported by simulations. 
%
%\end{abstract}
% IEEEtran.cls defaults to using nonbold math in the Abstract.
% This preserves the distinction between vectors and scalars. However,
% if the journal you are submitting to favors bold math in the abstract,
% then you can use LaTeX's standard command \boldmath at the very start
% of the abstract to achieve this. Many IEEE journals frown on math
% in the abstract anyway.

% Note that keywords are not normally used for peerreview papers.
%\begin{IEEEkeywords}
%Cooperative diversity, decode and forward, piecewise linear
%\end{IEEEkeywords}



% For peer review papers, you can put extra information on the cover
% page as needed:
% \ifCLASSOPTIONpeerreview
% \begin{center} \bfseries EDICS Category: 3-BBND \end{center}
% \fi
%
% For peerreview papers, this IEEEtran command inserts a page break and
% creates the second title. It will be ignored for other modes.
%\IEEEpeerreviewmaketitle




\item A card is selected from a pack of 52 cards.
\begin{enumerate}[label=(\alph*)]
    \item How many points are there in the sample space?
    \item Calculate the probability that the card is an ace of spades.
    \item Calculate the probability that the card is (i) an ace and (ii) black card.
\end{enumerate}
\solution
%Let $X$ be an bernoulli rv defined as in \tabref{tab:exemplar/11/16/3/26}.  Then, 
\begin{equation}
    p =
        \frac{4}{11} 
\end{equation}
\begin{table}[H]
	\centering
	\input{exemplar/11/16/3/26/tables/Table2.tex}
	\caption{}
        \label{tab:exemplar/11/16/3/26}
\end{table}

\item The probability that a non leap year selected at random will contain 53 sundays.
\\
\solution
%\begin{table}[H]
	\centering
\begin{tabular}{|c|c|c|}
\hline
Random variable &Value &Definition\\ \hline
\multirow{3}{*}{X} &0 &Slips of Rs 1\\
&1 &Slips of Rs 5\\
&2 &Slips of Rs 13\\ \hline
\multirow{2}{*}{Y} &0 &Box A\\
&1 &Box B\\\hline
\end{tabular}
\caption{}
\label{tab:Distribution}
\end{table}
See \tabref{tab:Distribution}.
\begin{align}
p_{Y}\brak{k}= \begin{cases} 
      \frac{1}{3} & {k=0} \\
      \frac{2}{3 }& {k=1} 
   \end{cases}
   \\
p_{Y|X}\brak{0|0} = \frac{19}{25}\, 
p_{Y|X}\brak{0|1} = \frac{6}{25}\,
p_{Y|X}\brak{1|0} = \frac{45}{50}\,
p_{Y|X}\brak{1|2} = \frac{5}{50}
\end{align}
The desired probability is the probability that a slip drawn at random is marked other than Rs 1,
\begin{align}
&=1-p_X\brak{0}\\
&= p_X(1) + p_X(2)
\end{align}
Using Bayes theorem,
\begin{align}
&= p_Y\brak{0} \times \pr{Y=0 | X=1} + p_Y\brak{1} \times \pr{Y=1|X=2}\\
&=\frac{1}{3} \times \frac{6}{25} + \frac{2}{3} \times \frac{5}{50}\\
&=\frac{11}{75}
\end{align}

\newpage

%\tableofcontents

\bigskip

\renewcommand{\thefigure}{\theenumi}
\renewcommand{\thetable}{\theenumi}
%\renewcommand{\theequation}{\theenumi}

%\begin{abstract}
%%\boldmath
%In this letter, an algorithm for evaluating the exact analytical bit error rate  (BER)  for the piecewise linear (PL) combiner for  multiple relays is presented. Previous results were available only for upto three relays. The algorithm is unique in the sense that  the actual mathematical expressions, that are prohibitively large, need not be explicitly obtained. The diversity gain due to multiple relays is shown through plots of the analytical BER, well supported by simulations. 
%
%\end{abstract}
% IEEEtran.cls defaults to using nonbold math in the Abstract.
% This preserves the distinction between vectors and scalars. However,
% if the journal you are submitting to favors bold math in the abstract,
% then you can use LaTeX's standard command \boldmath at the very start
% of the abstract to achieve this. Many IEEE journals frown on math
% in the abstract anyway.

% Note that keywords are not normally used for peerreview papers.
%\begin{IEEEkeywords}
%Cooperative diversity, decode and forward, piecewise linear
%\end{IEEEkeywords}



% For peer review papers, you can put extra information on the cover
% page as needed:
% \ifCLASSOPTIONpeerreview
% \begin{center} \bfseries EDICS Category: 3-BBND \end{center}
% \fi
%
% For peerreview papers, this IEEEtran command inserts a page break and
% creates the second title. It will be ignored for other modes.
%\IEEEpeerreviewmaketitle




\item One of the four persons John, Rita, Aslam or Gurpreet will be promoted next
month. Consequently the sample space consists of four elementary outcomes
S = {John promoted, Rita promoted, Aslam promoted, Gurpreet promoted}
You are told that the chances of John’s promotion is same as that of Gurpreet,
Rita’s chances of promotion are twice as likely as Johns. Aslam’s chances are
four times that of John.
\begin{enumerate}
	\item Determine
	\begin{enumerate}
		\item P (John promoted)
		\item P (Rita promoted)
		\item P (Aslam promoted)
		\item P (Gurpreet promoted)
	\end{enumerate}
	\item If A = {John promoted or Gurpreet promoted}, find P (A).
\end{enumerate}
\solution
%\begin{table}[H]
	\centering
\begin{tabular}{|c|c|c|}
\hline
Random variable &Value &Definition\\ \hline
\multirow{3}{*}{X} &0 &Slips of Rs 1\\
&1 &Slips of Rs 5\\
&2 &Slips of Rs 13\\ \hline
\multirow{2}{*}{Y} &0 &Box A\\
&1 &Box B\\\hline
\end{tabular}
\caption{}
\label{tab:Distribution}
\end{table}
See \tabref{tab:Distribution}.
\begin{align}
p_{Y}\brak{k}= \begin{cases} 
      \frac{1}{3} & {k=0} \\
      \frac{2}{3 }& {k=1} 
   \end{cases}
   \\
p_{Y|X}\brak{0|0} = \frac{19}{25}\, 
p_{Y|X}\brak{0|1} = \frac{6}{25}\,
p_{Y|X}\brak{1|0} = \frac{45}{50}\,
p_{Y|X}\brak{1|2} = \frac{5}{50}
\end{align}
The desired probability is the probability that a slip drawn at random is marked other than Rs 1,
\begin{align}
&=1-p_X\brak{0}\\
&= p_X(1) + p_X(2)
\end{align}
Using Bayes theorem,
\begin{align}
&= p_Y\brak{0} \times \pr{Y=0 | X=1} + p_Y\brak{1} \times \pr{Y=1|X=2}\\
&=\frac{1}{3} \times \frac{6}{25} + \frac{2}{3} \times \frac{5}{50}\\
&=\frac{11}{75}
\end{align}

\newpage

%\tableofcontents

\bigskip

\renewcommand{\thefigure}{\theenumi}
\renewcommand{\thetable}{\theenumi}
%\renewcommand{\theequation}{\theenumi}

%\begin{abstract}
%%\boldmath
%In this letter, an algorithm for evaluating the exact analytical bit error rate  (BER)  for the piecewise linear (PL) combiner for  multiple relays is presented. Previous results were available only for upto three relays. The algorithm is unique in the sense that  the actual mathematical expressions, that are prohibitively large, need not be explicitly obtained. The diversity gain due to multiple relays is shown through plots of the analytical BER, well supported by simulations. 
%
%\end{abstract}
% IEEEtran.cls defaults to using nonbold math in the Abstract.
% This preserves the distinction between vectors and scalars. However,
% if the journal you are submitting to favors bold math in the abstract,
% then you can use LaTeX's standard command \boldmath at the very start
% of the abstract to achieve this. Many IEEE journals frown on math
% in the abstract anyway.

% Note that keywords are not normally used for peerreview papers.
%\begin{IEEEkeywords}
%Cooperative diversity, decode and forward, piecewise linear
%\end{IEEEkeywords}



% For peer review papers, you can put extra information on the cover
% page as needed:
% \ifCLASSOPTIONpeerreview
% \begin{center} \bfseries EDICS Category: 3-BBND \end{center}
% \fi
%
% For peerreview papers, this IEEEtran command inserts a page break and
% creates the second title. It will be ignored for other modes.
%\IEEEpeerreviewmaketitle




\item A card is drawn from a deck of 52 cards. Find the probability of getting a king or a heart or a red card.\\
\solution
%\begin{table}[H]
	\centering
\begin{tabular}{|c|c|c|}
\hline
Random variable &Value &Definition\\ \hline
\multirow{3}{*}{X} &0 &Slips of Rs 1\\
&1 &Slips of Rs 5\\
&2 &Slips of Rs 13\\ \hline
\multirow{2}{*}{Y} &0 &Box A\\
&1 &Box B\\\hline
\end{tabular}
\caption{}
\label{tab:Distribution}
\end{table}
See \tabref{tab:Distribution}.
\begin{align}
p_{Y}\brak{k}= \begin{cases} 
      \frac{1}{3} & {k=0} \\
      \frac{2}{3 }& {k=1} 
   \end{cases}
   \\
p_{Y|X}\brak{0|0} = \frac{19}{25}\, 
p_{Y|X}\brak{0|1} = \frac{6}{25}\,
p_{Y|X}\brak{1|0} = \frac{45}{50}\,
p_{Y|X}\brak{1|2} = \frac{5}{50}
\end{align}
The desired probability is the probability that a slip drawn at random is marked other than Rs 1,
\begin{align}
&=1-p_X\brak{0}\\
&= p_X(1) + p_X(2)
\end{align}
Using Bayes theorem,
\begin{align}
&= p_Y\brak{0} \times \pr{Y=0 | X=1} + p_Y\brak{1} \times \pr{Y=1|X=2}\\
&=\frac{1}{3} \times \frac{6}{25} + \frac{2}{3} \times \frac{5}{50}\\
&=\frac{11}{75}
\end{align}

\newpage

%\tableofcontents

\bigskip

\renewcommand{\thefigure}{\theenumi}
\renewcommand{\thetable}{\theenumi}
%\renewcommand{\theequation}{\theenumi}

%\begin{abstract}
%%\boldmath
%In this letter, an algorithm for evaluating the exact analytical bit error rate  (BER)  for the piecewise linear (PL) combiner for  multiple relays is presented. Previous results were available only for upto three relays. The algorithm is unique in the sense that  the actual mathematical expressions, that are prohibitively large, need not be explicitly obtained. The diversity gain due to multiple relays is shown through plots of the analytical BER, well supported by simulations. 
%
%\end{abstract}
% IEEEtran.cls defaults to using nonbold math in the Abstract.
% This preserves the distinction between vectors and scalars. However,
% if the journal you are submitting to favors bold math in the abstract,
% then you can use LaTeX's standard command \boldmath at the very start
% of the abstract to achieve this. Many IEEE journals frown on math
% in the abstract anyway.

% Note that keywords are not normally used for peerreview papers.
%\begin{IEEEkeywords}
%Cooperative diversity, decode and forward, piecewise linear
%\end{IEEEkeywords}



% For peer review papers, you can put extra information on the cover
% page as needed:
% \ifCLASSOPTIONpeerreview
% \begin{center} \bfseries EDICS Category: 3-BBND \end{center}
% \fi
%
% For peerreview papers, this IEEEtran command inserts a page break and
% creates the second title. It will be ignored for other modes.
%\IEEEpeerreviewmaketitle




\item The probability that a student will pass his examination is 0.73, the probability of
the student getting a compartment is 0.13, and the probability that the student will
either pass or get compartment is 0.96. State True or False.\\
\solution
%\begin{table}[H]
	\centering
\begin{tabular}{|c|c|c|}
\hline
Random variable &Value &Definition\\ \hline
\multirow{3}{*}{X} &0 &Slips of Rs 1\\
&1 &Slips of Rs 5\\
&2 &Slips of Rs 13\\ \hline
\multirow{2}{*}{Y} &0 &Box A\\
&1 &Box B\\\hline
\end{tabular}
\caption{}
\label{tab:Distribution}
\end{table}
See \tabref{tab:Distribution}.
\begin{align}
p_{Y}\brak{k}= \begin{cases} 
      \frac{1}{3} & {k=0} \\
      \frac{2}{3 }& {k=1} 
   \end{cases}
   \\
p_{Y|X}\brak{0|0} = \frac{19}{25}\, 
p_{Y|X}\brak{0|1} = \frac{6}{25}\,
p_{Y|X}\brak{1|0} = \frac{45}{50}\,
p_{Y|X}\brak{1|2} = \frac{5}{50}
\end{align}
The desired probability is the probability that a slip drawn at random is marked other than Rs 1,
\begin{align}
&=1-p_X\brak{0}\\
&= p_X(1) + p_X(2)
\end{align}
Using Bayes theorem,
\begin{align}
&= p_Y\brak{0} \times \pr{Y=0 | X=1} + p_Y\brak{1} \times \pr{Y=1|X=2}\\
&=\frac{1}{3} \times \frac{6}{25} + \frac{2}{3} \times \frac{5}{50}\\
&=\frac{11}{75}
\end{align}

\newpage

%\tableofcontents

\bigskip

\renewcommand{\thefigure}{\theenumi}
\renewcommand{\thetable}{\theenumi}
%\renewcommand{\theequation}{\theenumi}

%\begin{abstract}
%%\boldmath
%In this letter, an algorithm for evaluating the exact analytical bit error rate  (BER)  for the piecewise linear (PL) combiner for  multiple relays is presented. Previous results were available only for upto three relays. The algorithm is unique in the sense that  the actual mathematical expressions, that are prohibitively large, need not be explicitly obtained. The diversity gain due to multiple relays is shown through plots of the analytical BER, well supported by simulations. 
%
%\end{abstract}
% IEEEtran.cls defaults to using nonbold math in the Abstract.
% This preserves the distinction between vectors and scalars. However,
% if the journal you are submitting to favors bold math in the abstract,
% then you can use LaTeX's standard command \boldmath at the very start
% of the abstract to achieve this. Many IEEE journals frown on math
% in the abstract anyway.

% Note that keywords are not normally used for peerreview papers.
%\begin{IEEEkeywords}
%Cooperative diversity, decode and forward, piecewise linear
%\end{IEEEkeywords}



% For peer review papers, you can put extra information on the cover
% page as needed:
% \ifCLASSOPTIONpeerreview
% \begin{center} \bfseries EDICS Category: 3-BBND \end{center}
% \fi
%
% For peerreview papers, this IEEEtran command inserts a page break and
% creates the second title. It will be ignored for other modes.
%\IEEEpeerreviewmaketitle




\item A card is selected from a pack of 52 cards\\
\begin{enumerate}[label=(\alph*)]
\item How many points are there in the sample space?
\item Calculate the probability that the cards is an ace of spades.
\item Calculate the probability that the card is (i) an ace (ii)black card.\\
\end{enumerate}
%\input{ncert/11/16/3/4_1/Prob_4.tex}
\item In a non-leap year, the probability of having 53 tuesdays or 53 wednesdays is\\
\solution
%A non-leap year has a total of 365 days, and a week has 7 days.\\
So it can be expressed as 
\begin{align}
365\text{days} &=52\times 7+1 \text{day}
\end{align}
$\implies$ 52 tuesdays or wednesdays\\
Random variable X denotes the days of a week
\begin{align}
p_X\brak{k}&=\frac{1}{7}; \quad \brak{1<k<7}
\end{align}
So the probability of extra day being tuesday or wednesday is
\begin{align}
p_X\brak{3}+p_X\brak{4}&=\frac{1}{7}+\frac{1}{7}=\frac{2}{7}
\end{align}



\item There are 1000 sealed envelopes in a box, 10 of them contain a cash prize of
Rs 100 each, 100 of them contain a cash prize of Rs 50 each and 200 of them
contain a cash prize of Rs 10 each and rest do not contain any cash prize. If they
are well shuffled and an envelope is picked up out, what is the probability that it
contains no cash prize?\\
\solution
%\begin{table}[H]
	\centering
\begin{tabular}{|c|c|c|}
\hline
Random variable &Value &Definition\\ \hline
\multirow{3}{*}{X} &0 &Slips of Rs 1\\
&1 &Slips of Rs 5\\
&2 &Slips of Rs 13\\ \hline
\multirow{2}{*}{Y} &0 &Box A\\
&1 &Box B\\\hline
\end{tabular}
\caption{}
\label{tab:Distribution}
\end{table}
See \tabref{tab:Distribution}.
\begin{align}
p_{Y}\brak{k}= \begin{cases} 
      \frac{1}{3} & {k=0} \\
      \frac{2}{3 }& {k=1} 
   \end{cases}
   \\
p_{Y|X}\brak{0|0} = \frac{19}{25}\, 
p_{Y|X}\brak{0|1} = \frac{6}{25}\,
p_{Y|X}\brak{1|0} = \frac{45}{50}\,
p_{Y|X}\brak{1|2} = \frac{5}{50}
\end{align}
The desired probability is the probability that a slip drawn at random is marked other than Rs 1,
\begin{align}
&=1-p_X\brak{0}\\
&= p_X(1) + p_X(2)
\end{align}
Using Bayes theorem,
\begin{align}
&= p_Y\brak{0} \times \pr{Y=0 | X=1} + p_Y\brak{1} \times \pr{Y=1|X=2}\\
&=\frac{1}{3} \times \frac{6}{25} + \frac{2}{3} \times \frac{5}{50}\\
&=\frac{11}{75}
\end{align}

\newpage

%\tableofcontents

\bigskip

\renewcommand{\thefigure}{\theenumi}
\renewcommand{\thetable}{\theenumi}
%\renewcommand{\theequation}{\theenumi}

%\begin{abstract}
%%\boldmath
%In this letter, an algorithm for evaluating the exact analytical bit error rate  (BER)  for the piecewise linear (PL) combiner for  multiple relays is presented. Previous results were available only for upto three relays. The algorithm is unique in the sense that  the actual mathematical expressions, that are prohibitively large, need not be explicitly obtained. The diversity gain due to multiple relays is shown through plots of the analytical BER, well supported by simulations. 
%
%\end{abstract}
% IEEEtran.cls defaults to using nonbold math in the Abstract.
% This preserves the distinction between vectors and scalars. However,
% if the journal you are submitting to favors bold math in the abstract,
% then you can use LaTeX's standard command \boldmath at the very start
% of the abstract to achieve this. Many IEEE journals frown on math
% in the abstract anyway.

% Note that keywords are not normally used for peerreview papers.
%\begin{IEEEkeywords}
%Cooperative diversity, decode and forward, piecewise linear
%\end{IEEEkeywords}



% For peer review papers, you can put extra information on the cover
% page as needed:
% \ifCLASSOPTIONpeerreview
% \begin{center} \bfseries EDICS Category: 3-BBND \end{center}
% \fi
%
% For peerreview papers, this IEEEtran command inserts a page break and
% creates the second title. It will be ignored for other modes.
%\IEEEpeerreviewmaketitle




\item 
A die is thrown and a card is selected at random from a deck of 52 playing cards. The probability of getting an even number on the die and a spade card.\\
\solution
%\begin{table}[H]
	\centering
\begin{tabular}{|c|c|c|}
\hline
Random variable &Value &Definition\\ \hline
\multirow{3}{*}{X} &0 &Slips of Rs 1\\
&1 &Slips of Rs 5\\
&2 &Slips of Rs 13\\ \hline
\multirow{2}{*}{Y} &0 &Box A\\
&1 &Box B\\\hline
\end{tabular}
\caption{}
\label{tab:Distribution}
\end{table}
See \tabref{tab:Distribution}.
\begin{align}
p_{Y}\brak{k}= \begin{cases} 
      \frac{1}{3} & {k=0} \\
      \frac{2}{3 }& {k=1} 
   \end{cases}
   \\
p_{Y|X}\brak{0|0} = \frac{19}{25}\, 
p_{Y|X}\brak{0|1} = \frac{6}{25}\,
p_{Y|X}\brak{1|0} = \frac{45}{50}\,
p_{Y|X}\brak{1|2} = \frac{5}{50}
\end{align}
The desired probability is the probability that a slip drawn at random is marked other than Rs 1,
\begin{align}
&=1-p_X\brak{0}\\
&= p_X(1) + p_X(2)
\end{align}
Using Bayes theorem,
\begin{align}
&= p_Y\brak{0} \times \pr{Y=0 | X=1} + p_Y\brak{1} \times \pr{Y=1|X=2}\\
&=\frac{1}{3} \times \frac{6}{25} + \frac{2}{3} \times \frac{5}{50}\\
&=\frac{11}{75}
\end{align}

\newpage

%\tableofcontents

\bigskip

\renewcommand{\thefigure}{\theenumi}
\renewcommand{\thetable}{\theenumi}
%\renewcommand{\theequation}{\theenumi}

%\begin{abstract}
%%\boldmath
%In this letter, an algorithm for evaluating the exact analytical bit error rate  (BER)  for the piecewise linear (PL) combiner for  multiple relays is presented. Previous results were available only for upto three relays. The algorithm is unique in the sense that  the actual mathematical expressions, that are prohibitively large, need not be explicitly obtained. The diversity gain due to multiple relays is shown through plots of the analytical BER, well supported by simulations. 
%
%\end{abstract}
% IEEEtran.cls defaults to using nonbold math in the Abstract.
% This preserves the distinction between vectors and scalars. However,
% if the journal you are submitting to favors bold math in the abstract,
% then you can use LaTeX's standard command \boldmath at the very start
% of the abstract to achieve this. Many IEEE journals frown on math
% in the abstract anyway.

% Note that keywords are not normally used for peerreview papers.
%\begin{IEEEkeywords}
%Cooperative diversity, decode and forward, piecewise linear
%\end{IEEEkeywords}



% For peer review papers, you can put extra information on the cover
% page as needed:
% \ifCLASSOPTIONpeerreview
% \begin{center} \bfseries EDICS Category: 3-BBND \end{center}
% \fi
%
% For peerreview papers, this IEEEtran command inserts a page break and
% creates the second title. It will be ignored for other modes.
%\IEEEpeerreviewmaketitle




\item
If 4-digit numbers greater than 5,000 are randomly formed from the digits 0, 1, 3, 5, and 7, what is the probability of forming a number divisible by 5 when:
\begin{enumerate}
    \item The digits are repeated?
    \item The repetition of digits is not allowed?
\end{enumerate}
\solution
%\begin{table}[H]
	\centering
\begin{tabular}{|c|c|c|}
\hline
Random variable &Value &Definition\\ \hline
\multirow{3}{*}{X} &0 &Slips of Rs 1\\
&1 &Slips of Rs 5\\
&2 &Slips of Rs 13\\ \hline
\multirow{2}{*}{Y} &0 &Box A\\
&1 &Box B\\\hline
\end{tabular}
\caption{}
\label{tab:Distribution}
\end{table}
See \tabref{tab:Distribution}.
\begin{align}
p_{Y}\brak{k}= \begin{cases} 
      \frac{1}{3} & {k=0} \\
      \frac{2}{3 }& {k=1} 
   \end{cases}
   \\
p_{Y|X}\brak{0|0} = \frac{19}{25}\, 
p_{Y|X}\brak{0|1} = \frac{6}{25}\,
p_{Y|X}\brak{1|0} = \frac{45}{50}\,
p_{Y|X}\brak{1|2} = \frac{5}{50}
\end{align}
The desired probability is the probability that a slip drawn at random is marked other than Rs 1,
\begin{align}
&=1-p_X\brak{0}\\
&= p_X(1) + p_X(2)
\end{align}
Using Bayes theorem,
\begin{align}
&= p_Y\brak{0} \times \pr{Y=0 | X=1} + p_Y\brak{1} \times \pr{Y=1|X=2}\\
&=\frac{1}{3} \times \frac{6}{25} + \frac{2}{3} \times \frac{5}{50}\\
&=\frac{11}{75}
\end{align}

\newpage

%\tableofcontents

\bigskip

\renewcommand{\thefigure}{\theenumi}
\renewcommand{\thetable}{\theenumi}
%\renewcommand{\theequation}{\theenumi}

%\begin{abstract}
%%\boldmath
%In this letter, an algorithm for evaluating the exact analytical bit error rate  (BER)  for the piecewise linear (PL) combiner for  multiple relays is presented. Previous results were available only for upto three relays. The algorithm is unique in the sense that  the actual mathematical expressions, that are prohibitively large, need not be explicitly obtained. The diversity gain due to multiple relays is shown through plots of the analytical BER, well supported by simulations. 
%
%\end{abstract}
% IEEEtran.cls defaults to using nonbold math in the Abstract.
% This preserves the distinction between vectors and scalars. However,
% if the journal you are submitting to favors bold math in the abstract,
% then you can use LaTeX's standard command \boldmath at the very start
% of the abstract to achieve this. Many IEEE journals frown on math
% in the abstract anyway.

% Note that keywords are not normally used for peerreview papers.
%\begin{IEEEkeywords}
%Cooperative diversity, decode and forward, piecewise linear
%\end{IEEEkeywords}



% For peer review papers, you can put extra information on the cover
% page as needed:
% \ifCLASSOPTIONpeerreview
% \begin{center} \bfseries EDICS Category: 3-BBND \end{center}
% \fi
%
% For peerreview papers, this IEEEtran command inserts a page break and
% creates the second title. It will be ignored for other modes.
%\IEEEpeerreviewmaketitle




\item Consider the probability space $\brak{\Omega, \mathcal{G}, P}$ where $\Omega = [0,2]$ and $\mathcal{G} = \cbrak{\phi, \Omega, [0,1], (1,2]}$. Let $X$ and $Y$ be two functions on $\Omega$ defined as
\begin{align*}
    X(\omega) = 
    \begin{cases}
        1 & \text{if }\omega \in [0, 1]\\
        2 & \text{if }\omega \in (1, 2]
    \end{cases}
\end{align*}
and
\begin{align*}
    Y(\omega) = 
    \begin{cases}
        2 & \text{if }\omega \in [0, 1.5]\\
        3 & \text{if }\omega \in (1.5, 2].
    \end{cases}
\end{align*}
Then which one of the following statements is true?
\begin{enumerate}
    \item [(A)] $X$ is a random variable with respect to $\mathcal{G}$, but $Y$ is not a random variable with respect to $\mathcal{G}$.
    \item [(B)] $Y$ is a random variable with respect to $\mathcal{G}$, but $X$ is not a random variable with respect to $\mathcal{G}$.
    \item [(C)] Neither $X$ nor $Y$ is a random variable with respect to $\mathcal{G}$.
    \item [(D)] Both $X$ and $Y$ are random variables with respect to $\mathcal{G}$.
\end{enumerate} \hfill (GATE ST 2023)\\
\solution
%\begin{table}[H]
	\centering
\begin{tabular}{|c|c|c|}
\hline
Random variable &Value &Definition\\ \hline
\multirow{3}{*}{X} &0 &Slips of Rs 1\\
&1 &Slips of Rs 5\\
&2 &Slips of Rs 13\\ \hline
\multirow{2}{*}{Y} &0 &Box A\\
&1 &Box B\\\hline
\end{tabular}
\caption{}
\label{tab:Distribution}
\end{table}
See \tabref{tab:Distribution}.
\begin{align}
p_{Y}\brak{k}= \begin{cases} 
      \frac{1}{3} & {k=0} \\
      \frac{2}{3 }& {k=1} 
   \end{cases}
   \\
p_{Y|X}\brak{0|0} = \frac{19}{25}\, 
p_{Y|X}\brak{0|1} = \frac{6}{25}\,
p_{Y|X}\brak{1|0} = \frac{45}{50}\,
p_{Y|X}\brak{1|2} = \frac{5}{50}
\end{align}
The desired probability is the probability that a slip drawn at random is marked other than Rs 1,
\begin{align}
&=1-p_X\brak{0}\\
&= p_X(1) + p_X(2)
\end{align}
Using Bayes theorem,
\begin{align}
&= p_Y\brak{0} \times \pr{Y=0 | X=1} + p_Y\brak{1} \times \pr{Y=1|X=2}\\
&=\frac{1}{3} \times \frac{6}{25} + \frac{2}{3} \times \frac{5}{50}\\
&=\frac{11}{75}
\end{align}

\newpage

%\tableofcontents

\bigskip

\renewcommand{\thefigure}{\theenumi}
\renewcommand{\thetable}{\theenumi}
%\renewcommand{\theequation}{\theenumi}

%\begin{abstract}
%%\boldmath
%In this letter, an algorithm for evaluating the exact analytical bit error rate  (BER)  for the piecewise linear (PL) combiner for  multiple relays is presented. Previous results were available only for upto three relays. The algorithm is unique in the sense that  the actual mathematical expressions, that are prohibitively large, need not be explicitly obtained. The diversity gain due to multiple relays is shown through plots of the analytical BER, well supported by simulations. 
%
%\end{abstract}
% IEEEtran.cls defaults to using nonbold math in the Abstract.
% This preserves the distinction between vectors and scalars. However,
% if the journal you are submitting to favors bold math in the abstract,
% then you can use LaTeX's standard command \boldmath at the very start
% of the abstract to achieve this. Many IEEE journals frown on math
% in the abstract anyway.

% Note that keywords are not normally used for peerreview papers.
%\begin{IEEEkeywords}
%Cooperative diversity, decode and forward, piecewise linear
%\end{IEEEkeywords}



% For peer review papers, you can put extra information on the cover
% page as needed:
% \ifCLASSOPTIONpeerreview
% \begin{center} \bfseries EDICS Category: 3-BBND \end{center}
% \fi
%
% For peerreview papers, this IEEEtran command inserts a page break and
% creates the second title. It will be ignored for other modes.
%\IEEEpeerreviewmaketitle




	\item  A die is loaded in such a way that each odd number is twice as likely to occur as
each even number. Find $P(G)$, where $G$ is the event that a number greater than
3 occurs on a single roll of the die.
\\
\solution
		%\begin{table}[H]
	\centering
\begin{tabular}{|c|c|c|}
\hline
Random variable &Value &Definition\\ \hline
\multirow{3}{*}{X} &0 &Slips of Rs 1\\
&1 &Slips of Rs 5\\
&2 &Slips of Rs 13\\ \hline
\multirow{2}{*}{Y} &0 &Box A\\
&1 &Box B\\\hline
\end{tabular}
\caption{}
\label{tab:Distribution}
\end{table}
See \tabref{tab:Distribution}.
\begin{align}
p_{Y}\brak{k}= \begin{cases} 
      \frac{1}{3} & {k=0} \\
      \frac{2}{3 }& {k=1} 
   \end{cases}
   \\
p_{Y|X}\brak{0|0} = \frac{19}{25}\, 
p_{Y|X}\brak{0|1} = \frac{6}{25}\,
p_{Y|X}\brak{1|0} = \frac{45}{50}\,
p_{Y|X}\brak{1|2} = \frac{5}{50}
\end{align}
The desired probability is the probability that a slip drawn at random is marked other than Rs 1,
\begin{align}
&=1-p_X\brak{0}\\
&= p_X(1) + p_X(2)
\end{align}
Using Bayes theorem,
\begin{align}
&= p_Y\brak{0} \times \pr{Y=0 | X=1} + p_Y\brak{1} \times \pr{Y=1|X=2}\\
&=\frac{1}{3} \times \frac{6}{25} + \frac{2}{3} \times \frac{5}{50}\\
&=\frac{11}{75}
\end{align}

\newpage

%\tableofcontents

\bigskip

\renewcommand{\thefigure}{\theenumi}
\renewcommand{\thetable}{\theenumi}
%\renewcommand{\theequation}{\theenumi}

%\begin{abstract}
%%\boldmath
%In this letter, an algorithm for evaluating the exact analytical bit error rate  (BER)  for the piecewise linear (PL) combiner for  multiple relays is presented. Previous results were available only for upto three relays. The algorithm is unique in the sense that  the actual mathematical expressions, that are prohibitively large, need not be explicitly obtained. The diversity gain due to multiple relays is shown through plots of the analytical BER, well supported by simulations. 
%
%\end{abstract}
% IEEEtran.cls defaults to using nonbold math in the Abstract.
% This preserves the distinction between vectors and scalars. However,
% if the journal you are submitting to favors bold math in the abstract,
% then you can use LaTeX's standard command \boldmath at the very start
% of the abstract to achieve this. Many IEEE journals frown on math
% in the abstract anyway.

% Note that keywords are not normally used for peerreview papers.
%\begin{IEEEkeywords}
%Cooperative diversity, decode and forward, piecewise linear
%\end{IEEEkeywords}



% For peer review papers, you can put extra information on the cover
% page as needed:
% \ifCLASSOPTIONpeerreview
% \begin{center} \bfseries EDICS Category: 3-BBND \end{center}
% \fi
%
% For peerreview papers, this IEEEtran command inserts a page break and
% creates the second title. It will be ignored for other modes.
%\IEEEpeerreviewmaketitle




	\item All the jacks, queens and kings are removed from a deck of 52 playing cards. The remaining cards are well shuffled and then one card is drawn at random. Giving ace a value 1 similar value for other cards, find the probability that the card has a value 
		\begin{enumerate}
			\item 7
			\item greater than 7
			\item less than 7
		\end{enumerate}
		%Number of cards left after removing all jacks, queens and kings 
\begin{align}
N	= 52 - 4\times 3
	= 40
\end{align}
%\begin{table}[H]
%\def\arraystretch{1.2}
%\begin{tabular}{|c|c|c|}
%\hline
%	\textbf{Parameter} &\textbf{Value} &\textbf{Description}\\ \hline
%	$X$ &1-10 &Represents the value of the card picked \\ \hline
%\end{tabular}
%\end{table}
Let $1 \le X \le 10$ be the value of the card picked.  Then,
\begin{align}
	p_X(k) &= \Pr(X=k)\ \forall\ 1 \leq k \leq 10\\
	&= \frac{4\times 1}{40}\\
	&= \frac{1}{10}\\
	\therefore p_X(k) &= 
	\begin{cases}
		\frac{1}{10} & 1 \leq k \leq 10\\
		0 & \text{otherwise}
	\end{cases}
\end{align}
and
\begin{align}
	F_{X}(k) &= \sum_{m=0}^{k}p_{X}(m) \quad 1 \leq k \leq 10\\
	&= \frac{k}{10}\\
	\therefore F_{X}(k) &= 
	\begin{cases}
		0 & k \leq 0\\
		\frac{k}{10} & 1\leq k \leq 10\\
		1 & k > 10 
	\end{cases}
\end{align}
\begin{enumerate}
	\item Probability that card has value equal to 7 is
		\begin{align}
			 p_{X}(7)
			= \frac{1}{10}
		\end{align}
	\item Probability that card has value greater than 7 is
		\begin{align}
			1 - F_X(7)
			&= 1 - \frac{7}{10}
			\\
			&= \frac{3}{10}
		\end{align}
	\item Probability that card has value less than 7 is
		\begin{align}
			 F_{X}(6)
			=\frac{6}{10}
		\end{align}
\end{enumerate}

  \item A Lot consists of 48 mobile phones of which 42 are good, 3 have only minor defects and 3 have major defects.Varnika will buy a phone if it is good but the trader will only buy a mobile if it has no major defects. One phone is selected at random from the lot. What is the probability that it is
\begin{enumerate}
	\item acceptable to Varnika?
            \item acceptable to the trader?
\end{enumerate}
\solution
	%\begin{table}[H]
	\centering
\begin{tabular}{|c|c|c|}
\hline
Random variable &Value &Definition\\ \hline
\multirow{3}{*}{X} &0 &Slips of Rs 1\\
&1 &Slips of Rs 5\\
&2 &Slips of Rs 13\\ \hline
\multirow{2}{*}{Y} &0 &Box A\\
&1 &Box B\\\hline
\end{tabular}
\caption{}
\label{tab:Distribution}
\end{table}
See \tabref{tab:Distribution}.
\begin{align}
p_{Y}\brak{k}= \begin{cases} 
      \frac{1}{3} & {k=0} \\
      \frac{2}{3 }& {k=1} 
   \end{cases}
   \\
p_{Y|X}\brak{0|0} = \frac{19}{25}\, 
p_{Y|X}\brak{0|1} = \frac{6}{25}\,
p_{Y|X}\brak{1|0} = \frac{45}{50}\,
p_{Y|X}\brak{1|2} = \frac{5}{50}
\end{align}
The desired probability is the probability that a slip drawn at random is marked other than Rs 1,
\begin{align}
&=1-p_X\brak{0}\\
&= p_X(1) + p_X(2)
\end{align}
Using Bayes theorem,
\begin{align}
&= p_Y\brak{0} \times \pr{Y=0 | X=1} + p_Y\brak{1} \times \pr{Y=1|X=2}\\
&=\frac{1}{3} \times \frac{6}{25} + \frac{2}{3} \times \frac{5}{50}\\
&=\frac{11}{75}
\end{align}

\newpage

%\tableofcontents

\bigskip

\renewcommand{\thefigure}{\theenumi}
\renewcommand{\thetable}{\theenumi}
%\renewcommand{\theequation}{\theenumi}

%\begin{abstract}
%%\boldmath
%In this letter, an algorithm for evaluating the exact analytical bit error rate  (BER)  for the piecewise linear (PL) combiner for  multiple relays is presented. Previous results were available only for upto three relays. The algorithm is unique in the sense that  the actual mathematical expressions, that are prohibitively large, need not be explicitly obtained. The diversity gain due to multiple relays is shown through plots of the analytical BER, well supported by simulations. 
%
%\end{abstract}
% IEEEtran.cls defaults to using nonbold math in the Abstract.
% This preserves the distinction between vectors and scalars. However,
% if the journal you are submitting to favors bold math in the abstract,
% then you can use LaTeX's standard command \boldmath at the very start
% of the abstract to achieve this. Many IEEE journals frown on math
% in the abstract anyway.

% Note that keywords are not normally used for peerreview papers.
%\begin{IEEEkeywords}
%Cooperative diversity, decode and forward, piecewise linear
%\end{IEEEkeywords}



% For peer review papers, you can put extra information on the cover
% page as needed:
% \ifCLASSOPTIONpeerreview
% \begin{center} \bfseries EDICS Category: 3-BBND \end{center}
% \fi
%
% For peerreview papers, this IEEEtran command inserts a page break and
% creates the second title. It will be ignored for other modes.
%\IEEEpeerreviewmaketitle




 \item A student says that if you throw a die, it will show up 1 or not 1. Therefore, the probability of getting 1 and the probability of getting 'not 1' each is equal to $\frac{1}{2}$. Is this correct? Give reasons.\\
 \solution
        %\begin{table}[H]
	\centering
\begin{tabular}{|c|c|c|}
\hline
Random variable &Value &Definition\\ \hline
\multirow{3}{*}{X} &0 &Slips of Rs 1\\
&1 &Slips of Rs 5\\
&2 &Slips of Rs 13\\ \hline
\multirow{2}{*}{Y} &0 &Box A\\
&1 &Box B\\\hline
\end{tabular}
\caption{}
\label{tab:Distribution}
\end{table}
See \tabref{tab:Distribution}.
\begin{align}
p_{Y}\brak{k}= \begin{cases} 
      \frac{1}{3} & {k=0} \\
      \frac{2}{3 }& {k=1} 
   \end{cases}
   \\
p_{Y|X}\brak{0|0} = \frac{19}{25}\, 
p_{Y|X}\brak{0|1} = \frac{6}{25}\,
p_{Y|X}\brak{1|0} = \frac{45}{50}\,
p_{Y|X}\brak{1|2} = \frac{5}{50}
\end{align}
The desired probability is the probability that a slip drawn at random is marked other than Rs 1,
\begin{align}
&=1-p_X\brak{0}\\
&= p_X(1) + p_X(2)
\end{align}
Using Bayes theorem,
\begin{align}
&= p_Y\brak{0} \times \pr{Y=0 | X=1} + p_Y\brak{1} \times \pr{Y=1|X=2}\\
&=\frac{1}{3} \times \frac{6}{25} + \frac{2}{3} \times \frac{5}{50}\\
&=\frac{11}{75}
\end{align}

\newpage

%\tableofcontents

\bigskip

\renewcommand{\thefigure}{\theenumi}
\renewcommand{\thetable}{\theenumi}
%\renewcommand{\theequation}{\theenumi}

%\begin{abstract}
%%\boldmath
%In this letter, an algorithm for evaluating the exact analytical bit error rate  (BER)  for the piecewise linear (PL) combiner for  multiple relays is presented. Previous results were available only for upto three relays. The algorithm is unique in the sense that  the actual mathematical expressions, that are prohibitively large, need not be explicitly obtained. The diversity gain due to multiple relays is shown through plots of the analytical BER, well supported by simulations. 
%
%\end{abstract}
% IEEEtran.cls defaults to using nonbold math in the Abstract.
% This preserves the distinction between vectors and scalars. However,
% if the journal you are submitting to favors bold math in the abstract,
% then you can use LaTeX's standard command \boldmath at the very start
% of the abstract to achieve this. Many IEEE journals frown on math
% in the abstract anyway.

% Note that keywords are not normally used for peerreview papers.
%\begin{IEEEkeywords}
%Cooperative diversity, decode and forward, piecewise linear
%\end{IEEEkeywords}



% For peer review papers, you can put extra information on the cover
% page as needed:
% \ifCLASSOPTIONpeerreview
% \begin{center} \bfseries EDICS Category: 3-BBND \end{center}
% \fi
%
% For peerreview papers, this IEEEtran command inserts a page break and
% creates the second title. It will be ignored for other modes.
%\IEEEpeerreviewmaketitle




   \item Four candidates A, B, C, D have ap-
plied for the assignment to coach a school cricket
team. If A is twice as likely to be selected as B, and
B and C are given about the same chance of being
selected, while C is twice as likely to be selected
as D, what are the probabilities that
\begin{enumerate}
\item C will be selected?
\item A will not be selected?
\end{enumerate}
	%\begin{table}[H]
	\centering
\begin{tabular}{|c|c|c|}
\hline
Random variable &Value &Definition\\ \hline
\multirow{3}{*}{X} &0 &Slips of Rs 1\\
&1 &Slips of Rs 5\\
&2 &Slips of Rs 13\\ \hline
\multirow{2}{*}{Y} &0 &Box A\\
&1 &Box B\\\hline
\end{tabular}
\caption{}
\label{tab:Distribution}
\end{table}
See \tabref{tab:Distribution}.
\begin{align}
p_{Y}\brak{k}= \begin{cases} 
      \frac{1}{3} & {k=0} \\
      \frac{2}{3 }& {k=1} 
   \end{cases}
   \\
p_{Y|X}\brak{0|0} = \frac{19}{25}\, 
p_{Y|X}\brak{0|1} = \frac{6}{25}\,
p_{Y|X}\brak{1|0} = \frac{45}{50}\,
p_{Y|X}\brak{1|2} = \frac{5}{50}
\end{align}
The desired probability is the probability that a slip drawn at random is marked other than Rs 1,
\begin{align}
&=1-p_X\brak{0}\\
&= p_X(1) + p_X(2)
\end{align}
Using Bayes theorem,
\begin{align}
&= p_Y\brak{0} \times \pr{Y=0 | X=1} + p_Y\brak{1} \times \pr{Y=1|X=2}\\
&=\frac{1}{3} \times \frac{6}{25} + \frac{2}{3} \times \frac{5}{50}\\
&=\frac{11}{75}
\end{align}

\newpage

%\tableofcontents

\bigskip

\renewcommand{\thefigure}{\theenumi}
\renewcommand{\thetable}{\theenumi}
%\renewcommand{\theequation}{\theenumi}

%\begin{abstract}
%%\boldmath
%In this letter, an algorithm for evaluating the exact analytical bit error rate  (BER)  for the piecewise linear (PL) combiner for  multiple relays is presented. Previous results were available only for upto three relays. The algorithm is unique in the sense that  the actual mathematical expressions, that are prohibitively large, need not be explicitly obtained. The diversity gain due to multiple relays is shown through plots of the analytical BER, well supported by simulations. 
%
%\end{abstract}
% IEEEtran.cls defaults to using nonbold math in the Abstract.
% This preserves the distinction between vectors and scalars. However,
% if the journal you are submitting to favors bold math in the abstract,
% then you can use LaTeX's standard command \boldmath at the very start
% of the abstract to achieve this. Many IEEE journals frown on math
% in the abstract anyway.

% Note that keywords are not normally used for peerreview papers.
%\begin{IEEEkeywords}
%Cooperative diversity, decode and forward, piecewise linear
%\end{IEEEkeywords}



% For peer review papers, you can put extra information on the cover
% page as needed:
% \ifCLASSOPTIONpeerreview
% \begin{center} \bfseries EDICS Category: 3-BBND \end{center}
% \fi
%
% For peerreview papers, this IEEEtran command inserts a page break and
% creates the second title. It will be ignored for other modes.
%\IEEEpeerreviewmaketitle




 \item A bag contain 24 balls of which $x$ balls are red, $2x$ are white and $3x$ are blue. A ball is selected at random, What is the probability that it is
\begin{enumerate}[label=\alph*)]
\item not red ?
\item white ?
\end{enumerate}
%\begin{table}[H]
	\centering
\begin{tabular}{|c|c|c|}
\hline
Random variable &Value &Definition\\ \hline
\multirow{3}{*}{X} &0 &Slips of Rs 1\\
&1 &Slips of Rs 5\\
&2 &Slips of Rs 13\\ \hline
\multirow{2}{*}{Y} &0 &Box A\\
&1 &Box B\\\hline
\end{tabular}
\caption{}
\label{tab:Distribution}
\end{table}
See \tabref{tab:Distribution}.
\begin{align}
p_{Y}\brak{k}= \begin{cases} 
      \frac{1}{3} & {k=0} \\
      \frac{2}{3 }& {k=1} 
   \end{cases}
   \\
p_{Y|X}\brak{0|0} = \frac{19}{25}\, 
p_{Y|X}\brak{0|1} = \frac{6}{25}\,
p_{Y|X}\brak{1|0} = \frac{45}{50}\,
p_{Y|X}\brak{1|2} = \frac{5}{50}
\end{align}
The desired probability is the probability that a slip drawn at random is marked other than Rs 1,
\begin{align}
&=1-p_X\brak{0}\\
&= p_X(1) + p_X(2)
\end{align}
Using Bayes theorem,
\begin{align}
&= p_Y\brak{0} \times \pr{Y=0 | X=1} + p_Y\brak{1} \times \pr{Y=1|X=2}\\
&=\frac{1}{3} \times \frac{6}{25} + \frac{2}{3} \times \frac{5}{50}\\
&=\frac{11}{75}
\end{align}

\newpage

%\tableofcontents

\bigskip

\renewcommand{\thefigure}{\theenumi}
\renewcommand{\thetable}{\theenumi}
%\renewcommand{\theequation}{\theenumi}

%\begin{abstract}
%%\boldmath
%In this letter, an algorithm for evaluating the exact analytical bit error rate  (BER)  for the piecewise linear (PL) combiner for  multiple relays is presented. Previous results were available only for upto three relays. The algorithm is unique in the sense that  the actual mathematical expressions, that are prohibitively large, need not be explicitly obtained. The diversity gain due to multiple relays is shown through plots of the analytical BER, well supported by simulations. 
%
%\end{abstract}
% IEEEtran.cls defaults to using nonbold math in the Abstract.
% This preserves the distinction between vectors and scalars. However,
% if the journal you are submitting to favors bold math in the abstract,
% then you can use LaTeX's standard command \boldmath at the very start
% of the abstract to achieve this. Many IEEE journals frown on math
% in the abstract anyway.

% Note that keywords are not normally used for peerreview papers.
%\begin{IEEEkeywords}
%Cooperative diversity, decode and forward, piecewise linear
%\end{IEEEkeywords}



% For peer review papers, you can put extra information on the cover
% page as needed:
% \ifCLASSOPTIONpeerreview
% \begin{center} \bfseries EDICS Category: 3-BBND \end{center}
% \fi
%
% For peerreview papers, this IEEEtran command inserts a page break and
% creates the second title. It will be ignored for other modes.
%\IEEEpeerreviewmaketitle




If the letters of the word ASSASSINATION are arranged at random. Find the Probability that
\begin{enumerate}[label=(\alph*)]
\item Four $S's$ come consecutively in the word
\item Two  $I's$ and two $N's$ come together
\item All $A's$ are not coming together
\item No two $A's$ are coming together
\end{enumerate}
%\begin{table}[H]
	\centering
\begin{tabular}{|c|c|c|}
\hline
Random variable &Value &Definition\\ \hline
\multirow{3}{*}{X} &0 &Slips of Rs 1\\
&1 &Slips of Rs 5\\
&2 &Slips of Rs 13\\ \hline
\multirow{2}{*}{Y} &0 &Box A\\
&1 &Box B\\\hline
\end{tabular}
\caption{}
\label{tab:Distribution}
\end{table}
See \tabref{tab:Distribution}.
\begin{align}
p_{Y}\brak{k}= \begin{cases} 
      \frac{1}{3} & {k=0} \\
      \frac{2}{3 }& {k=1} 
   \end{cases}
   \\
p_{Y|X}\brak{0|0} = \frac{19}{25}\, 
p_{Y|X}\brak{0|1} = \frac{6}{25}\,
p_{Y|X}\brak{1|0} = \frac{45}{50}\,
p_{Y|X}\brak{1|2} = \frac{5}{50}
\end{align}
The desired probability is the probability that a slip drawn at random is marked other than Rs 1,
\begin{align}
&=1-p_X\brak{0}\\
&= p_X(1) + p_X(2)
\end{align}
Using Bayes theorem,
\begin{align}
&= p_Y\brak{0} \times \pr{Y=0 | X=1} + p_Y\brak{1} \times \pr{Y=1|X=2}\\
&=\frac{1}{3} \times \frac{6}{25} + \frac{2}{3} \times \frac{5}{50}\\
&=\frac{11}{75}
\end{align}

\newpage

%\tableofcontents

\bigskip

\renewcommand{\thefigure}{\theenumi}
\renewcommand{\thetable}{\theenumi}
%\renewcommand{\theequation}{\theenumi}

%\begin{abstract}
%%\boldmath
%In this letter, an algorithm for evaluating the exact analytical bit error rate  (BER)  for the piecewise linear (PL) combiner for  multiple relays is presented. Previous results were available only for upto three relays. The algorithm is unique in the sense that  the actual mathematical expressions, that are prohibitively large, need not be explicitly obtained. The diversity gain due to multiple relays is shown through plots of the analytical BER, well supported by simulations. 
%
%\end{abstract}
% IEEEtran.cls defaults to using nonbold math in the Abstract.
% This preserves the distinction between vectors and scalars. However,
% if the journal you are submitting to favors bold math in the abstract,
% then you can use LaTeX's standard command \boldmath at the very start
% of the abstract to achieve this. Many IEEE journals frown on math
% in the abstract anyway.

% Note that keywords are not normally used for peerreview papers.
%\begin{IEEEkeywords}
%Cooperative diversity, decode and forward, piecewise linear
%\end{IEEEkeywords}



% For peer review papers, you can put extra information on the cover
% page as needed:
% \ifCLASSOPTIONpeerreview
% \begin{center} \bfseries EDICS Category: 3-BBND \end{center}
% \fi
%
% For peerreview papers, this IEEEtran command inserts a page break and
% creates the second title. It will be ignored for other modes.
%\IEEEpeerreviewmaketitle




	\item One urn contains two black balls (labelled B1 and B2) and one white ball. A
	second urn contains one black ball and two white balls (labelled W1 and W2).
	Suppose the following experiment is performed. One of the two urns is chosen
	at random. Next a ball is randomly chosen from the urn. Then a second ball is
	chosen at random from the same urn without replacing the first ball.
	
	\begin{enumerate}
	\item What is the probability that two black balls are chosen?
	
	\item What is the probability that two balls of opposite colour are chosen?
	\end{enumerate}
	\solution
	%\begin{align}
    \label{eq:12.13.6.18.1}
	\because	\pr{A|B} &> \pr{A},\
\frac{\pr{AB}}{\pr{B}} > \pr{A}
\\
    \label{eq:12.13.6.18.2}
	\implies \pr{AB} &> \pr{A}\pr{B}
	\\
	\text{or, } \frac{\pr{AB}}{\pr{A}} &=\pr{B|A} > \pr{A}
\end{align}

\end{enumerate}

	\item A card is selected from a pack of 52 cards.
 \begin{enumerate}[label=(\alph*)] 
                 \item How many points are there in the sample space?
                 \item Calculate the probability that the card is an ace of spades.
                 \item Calculate the probability that the card is (i) an ace and (ii) black card.
 \end{enumerate}
\solution
		%\begin{table}[H]
	\centering
\begin{tabular}{|c|c|c|}
\hline
Random variable &Value &Definition\\ \hline
\multirow{3}{*}{X} &0 &Slips of Rs 1\\
&1 &Slips of Rs 5\\
&2 &Slips of Rs 13\\ \hline
\multirow{2}{*}{Y} &0 &Box A\\
&1 &Box B\\\hline
\end{tabular}
\caption{}
\label{tab:Distribution}
\end{table}
See \tabref{tab:Distribution}.
\begin{align}
p_{Y}\brak{k}= \begin{cases} 
      \frac{1}{3} & {k=0} \\
      \frac{2}{3 }& {k=1} 
   \end{cases}
   \\
p_{Y|X}\brak{0|0} = \frac{19}{25}\, 
p_{Y|X}\brak{0|1} = \frac{6}{25}\,
p_{Y|X}\brak{1|0} = \frac{45}{50}\,
p_{Y|X}\brak{1|2} = \frac{5}{50}
\end{align}
The desired probability is the probability that a slip drawn at random is marked other than Rs 1,
\begin{align}
&=1-p_X\brak{0}\\
&= p_X(1) + p_X(2)
\end{align}
Using Bayes theorem,
\begin{align}
&= p_Y\brak{0} \times \pr{Y=0 | X=1} + p_Y\brak{1} \times \pr{Y=1|X=2}\\
&=\frac{1}{3} \times \frac{6}{25} + \frac{2}{3} \times \frac{5}{50}\\
&=\frac{11}{75}
\end{align}

\newpage

%\tableofcontents

\bigskip

\renewcommand{\thefigure}{\theenumi}
\renewcommand{\thetable}{\theenumi}
%\renewcommand{\theequation}{\theenumi}

%\begin{abstract}
%%\boldmath
%In this letter, an algorithm for evaluating the exact analytical bit error rate  (BER)  for the piecewise linear (PL) combiner for  multiple relays is presented. Previous results were available only for upto three relays. The algorithm is unique in the sense that  the actual mathematical expressions, that are prohibitively large, need not be explicitly obtained. The diversity gain due to multiple relays is shown through plots of the analytical BER, well supported by simulations. 
%
%\end{abstract}
% IEEEtran.cls defaults to using nonbold math in the Abstract.
% This preserves the distinction between vectors and scalars. However,
% if the journal you are submitting to favors bold math in the abstract,
% then you can use LaTeX's standard command \boldmath at the very start
% of the abstract to achieve this. Many IEEE journals frown on math
% in the abstract anyway.

% Note that keywords are not normally used for peerreview papers.
%\begin{IEEEkeywords}
%Cooperative diversity, decode and forward, piecewise linear
%\end{IEEEkeywords}



% For peer review papers, you can put extra information on the cover
% page as needed:
% \ifCLASSOPTIONpeerreview
% \begin{center} \bfseries EDICS Category: 3-BBND \end{center}
% \fi
%
% For peerreview papers, this IEEEtran command inserts a page break and
% creates the second title. It will be ignored for other modes.
%\IEEEpeerreviewmaketitle




\item Four cards are drawn from a well-shuffled deck of 52 cards. What is the probability of obtaining 3 diamonds and one spade.
\\
\solution
		%\begin{enumerate}[label=\thesection.\arabic*,ref=\thesection.\theenumi]
	\item One card is drawn from a well-shuffled deck of 52 cards. Find the probability of getting
\begin{enumerate}
\item A king of red colour 
\item A face card 
\item A red face card
\item The jack of hearts
\item A spade
\item The queen of diamonds

\end{enumerate}
\solution
		%\begin{table}[H]
	\centering
\begin{tabular}{|c|c|c|}
\hline
Random variable &Value &Definition\\ \hline
\multirow{3}{*}{X} &0 &Slips of Rs 1\\
&1 &Slips of Rs 5\\
&2 &Slips of Rs 13\\ \hline
\multirow{2}{*}{Y} &0 &Box A\\
&1 &Box B\\\hline
\end{tabular}
\caption{}
\label{tab:Distribution}
\end{table}
See \tabref{tab:Distribution}.
\begin{align}
p_{Y}\brak{k}= \begin{cases} 
      \frac{1}{3} & {k=0} \\
      \frac{2}{3 }& {k=1} 
   \end{cases}
   \\
p_{Y|X}\brak{0|0} = \frac{19}{25}\, 
p_{Y|X}\brak{0|1} = \frac{6}{25}\,
p_{Y|X}\brak{1|0} = \frac{45}{50}\,
p_{Y|X}\brak{1|2} = \frac{5}{50}
\end{align}
The desired probability is the probability that a slip drawn at random is marked other than Rs 1,
\begin{align}
&=1-p_X\brak{0}\\
&= p_X(1) + p_X(2)
\end{align}
Using Bayes theorem,
\begin{align}
&= p_Y\brak{0} \times \pr{Y=0 | X=1} + p_Y\brak{1} \times \pr{Y=1|X=2}\\
&=\frac{1}{3} \times \frac{6}{25} + \frac{2}{3} \times \frac{5}{50}\\
&=\frac{11}{75}
\end{align}

\newpage

%\tableofcontents

\bigskip

\renewcommand{\thefigure}{\theenumi}
\renewcommand{\thetable}{\theenumi}
%\renewcommand{\theequation}{\theenumi}

%\begin{abstract}
%%\boldmath
%In this letter, an algorithm for evaluating the exact analytical bit error rate  (BER)  for the piecewise linear (PL) combiner for  multiple relays is presented. Previous results were available only for upto three relays. The algorithm is unique in the sense that  the actual mathematical expressions, that are prohibitively large, need not be explicitly obtained. The diversity gain due to multiple relays is shown through plots of the analytical BER, well supported by simulations. 
%
%\end{abstract}
% IEEEtran.cls defaults to using nonbold math in the Abstract.
% This preserves the distinction between vectors and scalars. However,
% if the journal you are submitting to favors bold math in the abstract,
% then you can use LaTeX's standard command \boldmath at the very start
% of the abstract to achieve this. Many IEEE journals frown on math
% in the abstract anyway.

% Note that keywords are not normally used for peerreview papers.
%\begin{IEEEkeywords}
%Cooperative diversity, decode and forward, piecewise linear
%\end{IEEEkeywords}



% For peer review papers, you can put extra information on the cover
% page as needed:
% \ifCLASSOPTIONpeerreview
% \begin{center} \bfseries EDICS Category: 3-BBND \end{center}
% \fi
%
% For peerreview papers, this IEEEtran command inserts a page break and
% creates the second title. It will be ignored for other modes.
%\IEEEpeerreviewmaketitle




	\item Five cards—the ten, jack, queen, king and ace of diamonds, are well-shuffled with their face downwards. One card is then picked up at random.
\begin{enumerate}
\item
What is the probability that the card is the queen? 
\item
If the queen is drawn and put aside, what is the probability that the second card picked up is (a) an ace? (b) a queen?\\
\end{enumerate}
\solution
		%\begin{enumerate}[label=\thesection.\arabic*,ref=\thesection.\theenumi]
	\item One card is drawn from a well-shuffled deck of 52 cards. Find the probability of getting
\begin{enumerate}
\item A king of red colour 
\item A face card 
\item A red face card
\item The jack of hearts
\item A spade
\item The queen of diamonds

\end{enumerate}
\solution
		%\input{ncert/10/15/1/14/main.tex}
	\item Five cards—the ten, jack, queen, king and ace of diamonds, are well-shuffled with their face downwards. One card is then picked up at random.
\begin{enumerate}
\item
What is the probability that the card is the queen? 
\item
If the queen is drawn and put aside, what is the probability that the second card picked up is (a) an ace? (b) a queen?\\
\end{enumerate}
\solution
		%\input{ncert/10/15/1/15/defs.tex}
	\item A bag contains $5$ red balls and some blue balls. If the probability of drawing a blue ball is double that if a red ball, determine the number of blue balls in the bag. 
		\\
\solution
		%\input{ncert/10/15/2/3/defs.tex}
	\item A card is selected from a pack of 52 cards.
 \begin{enumerate}[label=(\alph*)] 
                 \item How many points are there in the sample space?
                 \item Calculate the probability that the card is an ace of spades.
                 \item Calculate the probability that the card is (i) an ace and (ii) black card.
 \end{enumerate}
\solution
		%\input{ncert/11/16/3/4/main.tex}
\item Four cards are drawn from a well-shuffled deck of 52 cards. What is the probability of obtaining 3 diamonds and one spade.
\\
\solution
		%\input{ncert/11/16/4/2/defs.tex}
\item In a certain lottery 10,000 tickets are sold and ten equal prizes are awarded. What is the probability of not getting a prize if you buy (a) one ticket (b) two tickets (c) 10 tickets ?	
\\
\solution
		%\input{ncert/11/16/4/4/defs.tex}
		%
\item 
Out of 100 students, two sections of 40 and 60 are formed. If you and your friend are among the 100 students, what is the probability that
\begin{enumerate}
\item you both enter the same section?
\item you both enter the different sections?
\end{enumerate}
\solution
		%\input{ncert/11/16/4/5/defs.tex}
	\item 
The number lock of a suitcase has 4 wheels each labelled with ten digits i.e. from 0 to 9.The lock opens with a sequence of four digits with no repeats.What is the probability of a person getting the right sequence to open the suitcase.
\\
\solution
		%\input{ncert/11/16/4/10/defs.tex}
		%
\item 
Two cards are drawn at random and without replacement from a pack of 52 playing cards. Find the probability that both the cards are black.
\\
\solution
		%\input{ncert/12/13/2/2/defs.tex}
		\item A box of oranges is inspected by examining three randomly selected oranges drawn without replacement. If all the three oranges are good, the box is approved for sale, otherwise, it is rejected. Find the probability that a box containing 15 oranges out of which 12 are good and 3 are bad ones will be approved for sale.
		\label{ncert/12/13/2/3/defs.tex}
		\item Two balls are drawn at random with replacement from a box containing 10 black and 8 red balls. Find the probability that
		\label{ncert/12/13/2/12}
\begin{enumerate}
\item both balls are red.
\item first ball is black and second is red.
\item one of them is black and other is red.
\end{enumerate}

\item In a hostel, 60\% of the students read Hindi newspaper, 40\% read English newspaper and 20\% read both Hindi and English newspapers. A student is selected at random.
		\label{ncert/12/13/2/15}
\begin{enumerate}
\item Find the probability that she reads neither Hindi nor English newspapers.
\item If she reads Hindi newspaper, find the probability that she reads English newspaper.
\item If she reads English newspaper, find the probability that she reads Hindi newspaper.\\
\end{enumerate}
\item The probability of obtaining an even prime number on each die, when a pair of dice is rolled is 
\begin{enumerate}
    \item $0$ 
    
    \item $\frac{1}{3}$ 
    
    \item $\frac{1}{12}$ 
    
    \item $\frac{1}{36}$ 
\end{enumerate}
\solution
		%\input{ncert/12/13/2/17/defs.tex}
	\item A bag contains 4 red and 4 black balls, another bag contains 2 red and 6 black balls. One of the two bags is selected at random and a ball is drawn from the bag which is found to be red. Find the probability that the ball is drawn from the first bag.
\\
\solution
		%\input{ncert/12/13/3/2/main.tex}
  \item
  Cards with numbers 2 to 101 are placed in a box. A card is selected at random.Find the probability that the card has
\begin{enumerate}[label=(\roman*)]
	\item an even number 
	\item a square number
\end{enumerate}
\solution
%\input{exemplar/10/13/3/32/main.tex}
\item
The king, queen and jack of clubs are removed from a deck of 52 playing cards and then well shuffled. Now one card is drawn at random from the remaining cards.  Determine the probability that the card is
\begin{enumerate}[label=(\roman*)]
\item a club
\item 10 of hearts
\end{enumerate}
\solution
%\input{exemplar/10/13/3/29/main.tex}
\item A team of medical students doing their internship have to assist during surgeries
at a city hospital. The probabilities of surgeries rated as very complex, complex,
routine, simple or very simple are respectively, 0.15, 0.20, 0.31, 0.26, .08. Find
the probabilities that a particular surgery will be rated
\begin{enumerate}
	\item complex or very complex;
	\item neither very complex nor very simple;
	\item routine or complex
	\item routine or simple
\end{enumerate}
\solution
%\input{exemplar/11/16/3/8(1)/main.tex}
\item A card is selected from a pack of 52 cards.
\begin{enumerate}[label=(\alph*)]
    \item How many points are there in the sample space?
    \item Calculate the probability that the card is an ace of spades.
    \item Calculate the probability that the card is (i) an ace and (ii) black card.
\end{enumerate}
\solution
%\input{exemplar/11/16/3/4/main2.tex}
\item The probability that a non leap year selected at random will contain 53 sundays.
\\
\solution
%\input{exemplar/10/13/1/19/main.tex}
\item One of the four persons John, Rita, Aslam or Gurpreet will be promoted next
month. Consequently the sample space consists of four elementary outcomes
S = {John promoted, Rita promoted, Aslam promoted, Gurpreet promoted}
You are told that the chances of John’s promotion is same as that of Gurpreet,
Rita’s chances of promotion are twice as likely as Johns. Aslam’s chances are
four times that of John.
\begin{enumerate}
	\item Determine
	\begin{enumerate}
		\item P (John promoted)
		\item P (Rita promoted)
		\item P (Aslam promoted)
		\item P (Gurpreet promoted)
	\end{enumerate}
	\item If A = {John promoted or Gurpreet promoted}, find P (A).
\end{enumerate}
\solution
%\input{exemplar/11/16/3/10/main.tex}
\item A card is drawn from a deck of 52 cards. Find the probability of getting a king or a heart or a red card.\\
\solution
%\input{exemplar/11/16/3/15/main.tex}
\item The probability that a student will pass his examination is 0.73, the probability of
the student getting a compartment is 0.13, and the probability that the student will
either pass or get compartment is 0.96. State True or False.\\
\solution
%\input{exemplar/11/16/3/31/main.tex}
\item A card is selected from a pack of 52 cards\\
\begin{enumerate}[label=(\alph*)]
\item How many points are there in the sample space?
\item Calculate the probability that the cards is an ace of spades.
\item Calculate the probability that the card is (i) an ace (ii)black card.\\
\end{enumerate}
%\input{ncert/11/16/3/4_1/Prob_4.tex}
\item In a non-leap year, the probability of having 53 tuesdays or 53 wednesdays is\\
\solution
%\input{exemplar/11/16/3/18/main.tex}
\item There are 1000 sealed envelopes in a box, 10 of them contain a cash prize of
Rs 100 each, 100 of them contain a cash prize of Rs 50 each and 200 of them
contain a cash prize of Rs 10 each and rest do not contain any cash prize. If they
are well shuffled and an envelope is picked up out, what is the probability that it
contains no cash prize?\\
\solution
%\input{exemplar/10/13/3/34/main.tex}
\item 
A die is thrown and a card is selected at random from a deck of 52 playing cards. The probability of getting an even number on the die and a spade card.\\
\solution
%\input{exemplar/12/13/3/78/main.tex}
\item
If 4-digit numbers greater than 5,000 are randomly formed from the digits 0, 1, 3, 5, and 7, what is the probability of forming a number divisible by 5 when:
\begin{enumerate}
    \item The digits are repeated?
    \item The repetition of digits is not allowed?
\end{enumerate}
\solution
%\input{ncert/11/16/4/9/main.tex}
\item Consider the probability space $\brak{\Omega, \mathcal{G}, P}$ where $\Omega = [0,2]$ and $\mathcal{G} = \cbrak{\phi, \Omega, [0,1], (1,2]}$. Let $X$ and $Y$ be two functions on $\Omega$ defined as
\begin{align*}
    X(\omega) = 
    \begin{cases}
        1 & \text{if }\omega \in [0, 1]\\
        2 & \text{if }\omega \in (1, 2]
    \end{cases}
\end{align*}
and
\begin{align*}
    Y(\omega) = 
    \begin{cases}
        2 & \text{if }\omega \in [0, 1.5]\\
        3 & \text{if }\omega \in (1.5, 2].
    \end{cases}
\end{align*}
Then which one of the following statements is true?
\begin{enumerate}
    \item [(A)] $X$ is a random variable with respect to $\mathcal{G}$, but $Y$ is not a random variable with respect to $\mathcal{G}$.
    \item [(B)] $Y$ is a random variable with respect to $\mathcal{G}$, but $X$ is not a random variable with respect to $\mathcal{G}$.
    \item [(C)] Neither $X$ nor $Y$ is a random variable with respect to $\mathcal{G}$.
    \item [(D)] Both $X$ and $Y$ are random variables with respect to $\mathcal{G}$.
\end{enumerate} \hfill (GATE ST 2023)\\
\solution
%\input{gate/ST/2023/14/main.tex}
	\item  A die is loaded in such a way that each odd number is twice as likely to occur as
each even number. Find $P(G)$, where $G$ is the event that a number greater than
3 occurs on a single roll of the die.
\\
\solution
		%\input{exemplar/11/16/3/5/main.tex}
	\item All the jacks, queens and kings are removed from a deck of 52 playing cards. The remaining cards are well shuffled and then one card is drawn at random. Giving ace a value 1 similar value for other cards, find the probability that the card has a value 
		\begin{enumerate}
			\item 7
			\item greater than 7
			\item less than 7
		\end{enumerate}
		%\input{exemplar/10/13/3/30/main.tex}
  \item A Lot consists of 48 mobile phones of which 42 are good, 3 have only minor defects and 3 have major defects.Varnika will buy a phone if it is good but the trader will only buy a mobile if it has no major defects. One phone is selected at random from the lot. What is the probability that it is
\begin{enumerate}
	\item acceptable to Varnika?
            \item acceptable to the trader?
\end{enumerate}
\solution
	%\input{exemplar/10/13/3/40/main.tex}
 \item A student says that if you throw a die, it will show up 1 or not 1. Therefore, the probability of getting 1 and the probability of getting 'not 1' each is equal to $\frac{1}{2}$. Is this correct? Give reasons.\\
 \solution
        %\input{exemplar/10/13/2/9/main.tex}
   \item Four candidates A, B, C, D have ap-
plied for the assignment to coach a school cricket
team. If A is twice as likely to be selected as B, and
B and C are given about the same chance of being
selected, while C is twice as likely to be selected
as D, what are the probabilities that
\begin{enumerate}
\item C will be selected?
\item A will not be selected?
\end{enumerate}
	%\input{exemplar/11/16/3/9/main.tex}
 \item A bag contain 24 balls of which $x$ balls are red, $2x$ are white and $3x$ are blue. A ball is selected at random, What is the probability that it is
\begin{enumerate}[label=\alph*)]
\item not red ?
\item white ?
\end{enumerate}
%\input{exemplar/10/13/3/41/main.tex}
If the letters of the word ASSASSINATION are arranged at random. Find the Probability that
\begin{enumerate}[label=(\alph*)]
\item Four $S's$ come consecutively in the word
\item Two  $I's$ and two $N's$ come together
\item All $A's$ are not coming together
\item No two $A's$ are coming together
\end{enumerate}
%\input{exemplar/11/16/3/14/main.tex}
	\item One urn contains two black balls (labelled B1 and B2) and one white ball. A
	second urn contains one black ball and two white balls (labelled W1 and W2).
	Suppose the following experiment is performed. One of the two urns is chosen
	at random. Next a ball is randomly chosen from the urn. Then a second ball is
	chosen at random from the same urn without replacing the first ball.
	
	\begin{enumerate}
	\item What is the probability that two black balls are chosen?
	
	\item What is the probability that two balls of opposite colour are chosen?
	\end{enumerate}
	\solution
	%\input{exemplar/11/16/3/12/main1.tex}
\end{enumerate}

	\item A bag contains $5$ red balls and some blue balls. If the probability of drawing a blue ball is double that if a red ball, determine the number of blue balls in the bag. 
		\\
\solution
		%\begin{enumerate}[label=\thesection.\arabic*,ref=\thesection.\theenumi]
	\item One card is drawn from a well-shuffled deck of 52 cards. Find the probability of getting
\begin{enumerate}
\item A king of red colour 
\item A face card 
\item A red face card
\item The jack of hearts
\item A spade
\item The queen of diamonds

\end{enumerate}
\solution
		%\input{ncert/10/15/1/14/main.tex}
	\item Five cards—the ten, jack, queen, king and ace of diamonds, are well-shuffled with their face downwards. One card is then picked up at random.
\begin{enumerate}
\item
What is the probability that the card is the queen? 
\item
If the queen is drawn and put aside, what is the probability that the second card picked up is (a) an ace? (b) a queen?\\
\end{enumerate}
\solution
		%\input{ncert/10/15/1/15/defs.tex}
	\item A bag contains $5$ red balls and some blue balls. If the probability of drawing a blue ball is double that if a red ball, determine the number of blue balls in the bag. 
		\\
\solution
		%\input{ncert/10/15/2/3/defs.tex}
	\item A card is selected from a pack of 52 cards.
 \begin{enumerate}[label=(\alph*)] 
                 \item How many points are there in the sample space?
                 \item Calculate the probability that the card is an ace of spades.
                 \item Calculate the probability that the card is (i) an ace and (ii) black card.
 \end{enumerate}
\solution
		%\input{ncert/11/16/3/4/main.tex}
\item Four cards are drawn from a well-shuffled deck of 52 cards. What is the probability of obtaining 3 diamonds and one spade.
\\
\solution
		%\input{ncert/11/16/4/2/defs.tex}
\item In a certain lottery 10,000 tickets are sold and ten equal prizes are awarded. What is the probability of not getting a prize if you buy (a) one ticket (b) two tickets (c) 10 tickets ?	
\\
\solution
		%\input{ncert/11/16/4/4/defs.tex}
		%
\item 
Out of 100 students, two sections of 40 and 60 are formed. If you and your friend are among the 100 students, what is the probability that
\begin{enumerate}
\item you both enter the same section?
\item you both enter the different sections?
\end{enumerate}
\solution
		%\input{ncert/11/16/4/5/defs.tex}
	\item 
The number lock of a suitcase has 4 wheels each labelled with ten digits i.e. from 0 to 9.The lock opens with a sequence of four digits with no repeats.What is the probability of a person getting the right sequence to open the suitcase.
\\
\solution
		%\input{ncert/11/16/4/10/defs.tex}
		%
\item 
Two cards are drawn at random and without replacement from a pack of 52 playing cards. Find the probability that both the cards are black.
\\
\solution
		%\input{ncert/12/13/2/2/defs.tex}
		\item A box of oranges is inspected by examining three randomly selected oranges drawn without replacement. If all the three oranges are good, the box is approved for sale, otherwise, it is rejected. Find the probability that a box containing 15 oranges out of which 12 are good and 3 are bad ones will be approved for sale.
		\label{ncert/12/13/2/3/defs.tex}
		\item Two balls are drawn at random with replacement from a box containing 10 black and 8 red balls. Find the probability that
		\label{ncert/12/13/2/12}
\begin{enumerate}
\item both balls are red.
\item first ball is black and second is red.
\item one of them is black and other is red.
\end{enumerate}

\item In a hostel, 60\% of the students read Hindi newspaper, 40\% read English newspaper and 20\% read both Hindi and English newspapers. A student is selected at random.
		\label{ncert/12/13/2/15}
\begin{enumerate}
\item Find the probability that she reads neither Hindi nor English newspapers.
\item If she reads Hindi newspaper, find the probability that she reads English newspaper.
\item If she reads English newspaper, find the probability that she reads Hindi newspaper.\\
\end{enumerate}
\item The probability of obtaining an even prime number on each die, when a pair of dice is rolled is 
\begin{enumerate}
    \item $0$ 
    
    \item $\frac{1}{3}$ 
    
    \item $\frac{1}{12}$ 
    
    \item $\frac{1}{36}$ 
\end{enumerate}
\solution
		%\input{ncert/12/13/2/17/defs.tex}
	\item A bag contains 4 red and 4 black balls, another bag contains 2 red and 6 black balls. One of the two bags is selected at random and a ball is drawn from the bag which is found to be red. Find the probability that the ball is drawn from the first bag.
\\
\solution
		%\input{ncert/12/13/3/2/main.tex}
  \item
  Cards with numbers 2 to 101 are placed in a box. A card is selected at random.Find the probability that the card has
\begin{enumerate}[label=(\roman*)]
	\item an even number 
	\item a square number
\end{enumerate}
\solution
%\input{exemplar/10/13/3/32/main.tex}
\item
The king, queen and jack of clubs are removed from a deck of 52 playing cards and then well shuffled. Now one card is drawn at random from the remaining cards.  Determine the probability that the card is
\begin{enumerate}[label=(\roman*)]
\item a club
\item 10 of hearts
\end{enumerate}
\solution
%\input{exemplar/10/13/3/29/main.tex}
\item A team of medical students doing their internship have to assist during surgeries
at a city hospital. The probabilities of surgeries rated as very complex, complex,
routine, simple or very simple are respectively, 0.15, 0.20, 0.31, 0.26, .08. Find
the probabilities that a particular surgery will be rated
\begin{enumerate}
	\item complex or very complex;
	\item neither very complex nor very simple;
	\item routine or complex
	\item routine or simple
\end{enumerate}
\solution
%\input{exemplar/11/16/3/8(1)/main.tex}
\item A card is selected from a pack of 52 cards.
\begin{enumerate}[label=(\alph*)]
    \item How many points are there in the sample space?
    \item Calculate the probability that the card is an ace of spades.
    \item Calculate the probability that the card is (i) an ace and (ii) black card.
\end{enumerate}
\solution
%\input{exemplar/11/16/3/4/main2.tex}
\item The probability that a non leap year selected at random will contain 53 sundays.
\\
\solution
%\input{exemplar/10/13/1/19/main.tex}
\item One of the four persons John, Rita, Aslam or Gurpreet will be promoted next
month. Consequently the sample space consists of four elementary outcomes
S = {John promoted, Rita promoted, Aslam promoted, Gurpreet promoted}
You are told that the chances of John’s promotion is same as that of Gurpreet,
Rita’s chances of promotion are twice as likely as Johns. Aslam’s chances are
four times that of John.
\begin{enumerate}
	\item Determine
	\begin{enumerate}
		\item P (John promoted)
		\item P (Rita promoted)
		\item P (Aslam promoted)
		\item P (Gurpreet promoted)
	\end{enumerate}
	\item If A = {John promoted or Gurpreet promoted}, find P (A).
\end{enumerate}
\solution
%\input{exemplar/11/16/3/10/main.tex}
\item A card is drawn from a deck of 52 cards. Find the probability of getting a king or a heart or a red card.\\
\solution
%\input{exemplar/11/16/3/15/main.tex}
\item The probability that a student will pass his examination is 0.73, the probability of
the student getting a compartment is 0.13, and the probability that the student will
either pass or get compartment is 0.96. State True or False.\\
\solution
%\input{exemplar/11/16/3/31/main.tex}
\item A card is selected from a pack of 52 cards\\
\begin{enumerate}[label=(\alph*)]
\item How many points are there in the sample space?
\item Calculate the probability that the cards is an ace of spades.
\item Calculate the probability that the card is (i) an ace (ii)black card.\\
\end{enumerate}
%\input{ncert/11/16/3/4_1/Prob_4.tex}
\item In a non-leap year, the probability of having 53 tuesdays or 53 wednesdays is\\
\solution
%\input{exemplar/11/16/3/18/main.tex}
\item There are 1000 sealed envelopes in a box, 10 of them contain a cash prize of
Rs 100 each, 100 of them contain a cash prize of Rs 50 each and 200 of them
contain a cash prize of Rs 10 each and rest do not contain any cash prize. If they
are well shuffled and an envelope is picked up out, what is the probability that it
contains no cash prize?\\
\solution
%\input{exemplar/10/13/3/34/main.tex}
\item 
A die is thrown and a card is selected at random from a deck of 52 playing cards. The probability of getting an even number on the die and a spade card.\\
\solution
%\input{exemplar/12/13/3/78/main.tex}
\item
If 4-digit numbers greater than 5,000 are randomly formed from the digits 0, 1, 3, 5, and 7, what is the probability of forming a number divisible by 5 when:
\begin{enumerate}
    \item The digits are repeated?
    \item The repetition of digits is not allowed?
\end{enumerate}
\solution
%\input{ncert/11/16/4/9/main.tex}
\item Consider the probability space $\brak{\Omega, \mathcal{G}, P}$ where $\Omega = [0,2]$ and $\mathcal{G} = \cbrak{\phi, \Omega, [0,1], (1,2]}$. Let $X$ and $Y$ be two functions on $\Omega$ defined as
\begin{align*}
    X(\omega) = 
    \begin{cases}
        1 & \text{if }\omega \in [0, 1]\\
        2 & \text{if }\omega \in (1, 2]
    \end{cases}
\end{align*}
and
\begin{align*}
    Y(\omega) = 
    \begin{cases}
        2 & \text{if }\omega \in [0, 1.5]\\
        3 & \text{if }\omega \in (1.5, 2].
    \end{cases}
\end{align*}
Then which one of the following statements is true?
\begin{enumerate}
    \item [(A)] $X$ is a random variable with respect to $\mathcal{G}$, but $Y$ is not a random variable with respect to $\mathcal{G}$.
    \item [(B)] $Y$ is a random variable with respect to $\mathcal{G}$, but $X$ is not a random variable with respect to $\mathcal{G}$.
    \item [(C)] Neither $X$ nor $Y$ is a random variable with respect to $\mathcal{G}$.
    \item [(D)] Both $X$ and $Y$ are random variables with respect to $\mathcal{G}$.
\end{enumerate} \hfill (GATE ST 2023)\\
\solution
%\input{gate/ST/2023/14/main.tex}
	\item  A die is loaded in such a way that each odd number is twice as likely to occur as
each even number. Find $P(G)$, where $G$ is the event that a number greater than
3 occurs on a single roll of the die.
\\
\solution
		%\input{exemplar/11/16/3/5/main.tex}
	\item All the jacks, queens and kings are removed from a deck of 52 playing cards. The remaining cards are well shuffled and then one card is drawn at random. Giving ace a value 1 similar value for other cards, find the probability that the card has a value 
		\begin{enumerate}
			\item 7
			\item greater than 7
			\item less than 7
		\end{enumerate}
		%\input{exemplar/10/13/3/30/main.tex}
  \item A Lot consists of 48 mobile phones of which 42 are good, 3 have only minor defects and 3 have major defects.Varnika will buy a phone if it is good but the trader will only buy a mobile if it has no major defects. One phone is selected at random from the lot. What is the probability that it is
\begin{enumerate}
	\item acceptable to Varnika?
            \item acceptable to the trader?
\end{enumerate}
\solution
	%\input{exemplar/10/13/3/40/main.tex}
 \item A student says that if you throw a die, it will show up 1 or not 1. Therefore, the probability of getting 1 and the probability of getting 'not 1' each is equal to $\frac{1}{2}$. Is this correct? Give reasons.\\
 \solution
        %\input{exemplar/10/13/2/9/main.tex}
   \item Four candidates A, B, C, D have ap-
plied for the assignment to coach a school cricket
team. If A is twice as likely to be selected as B, and
B and C are given about the same chance of being
selected, while C is twice as likely to be selected
as D, what are the probabilities that
\begin{enumerate}
\item C will be selected?
\item A will not be selected?
\end{enumerate}
	%\input{exemplar/11/16/3/9/main.tex}
 \item A bag contain 24 balls of which $x$ balls are red, $2x$ are white and $3x$ are blue. A ball is selected at random, What is the probability that it is
\begin{enumerate}[label=\alph*)]
\item not red ?
\item white ?
\end{enumerate}
%\input{exemplar/10/13/3/41/main.tex}
If the letters of the word ASSASSINATION are arranged at random. Find the Probability that
\begin{enumerate}[label=(\alph*)]
\item Four $S's$ come consecutively in the word
\item Two  $I's$ and two $N's$ come together
\item All $A's$ are not coming together
\item No two $A's$ are coming together
\end{enumerate}
%\input{exemplar/11/16/3/14/main.tex}
	\item One urn contains two black balls (labelled B1 and B2) and one white ball. A
	second urn contains one black ball and two white balls (labelled W1 and W2).
	Suppose the following experiment is performed. One of the two urns is chosen
	at random. Next a ball is randomly chosen from the urn. Then a second ball is
	chosen at random from the same urn without replacing the first ball.
	
	\begin{enumerate}
	\item What is the probability that two black balls are chosen?
	
	\item What is the probability that two balls of opposite colour are chosen?
	\end{enumerate}
	\solution
	%\input{exemplar/11/16/3/12/main1.tex}
\end{enumerate}

	\item A card is selected from a pack of 52 cards.
 \begin{enumerate}[label=(\alph*)] 
                 \item How many points are there in the sample space?
                 \item Calculate the probability that the card is an ace of spades.
                 \item Calculate the probability that the card is (i) an ace and (ii) black card.
 \end{enumerate}
\solution
		%\begin{table}[H]
	\centering
\begin{tabular}{|c|c|c|}
\hline
Random variable &Value &Definition\\ \hline
\multirow{3}{*}{X} &0 &Slips of Rs 1\\
&1 &Slips of Rs 5\\
&2 &Slips of Rs 13\\ \hline
\multirow{2}{*}{Y} &0 &Box A\\
&1 &Box B\\\hline
\end{tabular}
\caption{}
\label{tab:Distribution}
\end{table}
See \tabref{tab:Distribution}.
\begin{align}
p_{Y}\brak{k}= \begin{cases} 
      \frac{1}{3} & {k=0} \\
      \frac{2}{3 }& {k=1} 
   \end{cases}
   \\
p_{Y|X}\brak{0|0} = \frac{19}{25}\, 
p_{Y|X}\brak{0|1} = \frac{6}{25}\,
p_{Y|X}\brak{1|0} = \frac{45}{50}\,
p_{Y|X}\brak{1|2} = \frac{5}{50}
\end{align}
The desired probability is the probability that a slip drawn at random is marked other than Rs 1,
\begin{align}
&=1-p_X\brak{0}\\
&= p_X(1) + p_X(2)
\end{align}
Using Bayes theorem,
\begin{align}
&= p_Y\brak{0} \times \pr{Y=0 | X=1} + p_Y\brak{1} \times \pr{Y=1|X=2}\\
&=\frac{1}{3} \times \frac{6}{25} + \frac{2}{3} \times \frac{5}{50}\\
&=\frac{11}{75}
\end{align}

\newpage

%\tableofcontents

\bigskip

\renewcommand{\thefigure}{\theenumi}
\renewcommand{\thetable}{\theenumi}
%\renewcommand{\theequation}{\theenumi}

%\begin{abstract}
%%\boldmath
%In this letter, an algorithm for evaluating the exact analytical bit error rate  (BER)  for the piecewise linear (PL) combiner for  multiple relays is presented. Previous results were available only for upto three relays. The algorithm is unique in the sense that  the actual mathematical expressions, that are prohibitively large, need not be explicitly obtained. The diversity gain due to multiple relays is shown through plots of the analytical BER, well supported by simulations. 
%
%\end{abstract}
% IEEEtran.cls defaults to using nonbold math in the Abstract.
% This preserves the distinction between vectors and scalars. However,
% if the journal you are submitting to favors bold math in the abstract,
% then you can use LaTeX's standard command \boldmath at the very start
% of the abstract to achieve this. Many IEEE journals frown on math
% in the abstract anyway.

% Note that keywords are not normally used for peerreview papers.
%\begin{IEEEkeywords}
%Cooperative diversity, decode and forward, piecewise linear
%\end{IEEEkeywords}



% For peer review papers, you can put extra information on the cover
% page as needed:
% \ifCLASSOPTIONpeerreview
% \begin{center} \bfseries EDICS Category: 3-BBND \end{center}
% \fi
%
% For peerreview papers, this IEEEtran command inserts a page break and
% creates the second title. It will be ignored for other modes.
%\IEEEpeerreviewmaketitle




\item Four cards are drawn from a well-shuffled deck of 52 cards. What is the probability of obtaining 3 diamonds and one spade.
\\
\solution
		%\begin{enumerate}[label=\thesection.\arabic*,ref=\thesection.\theenumi]
	\item One card is drawn from a well-shuffled deck of 52 cards. Find the probability of getting
\begin{enumerate}
\item A king of red colour 
\item A face card 
\item A red face card
\item The jack of hearts
\item A spade
\item The queen of diamonds

\end{enumerate}
\solution
		%\input{ncert/10/15/1/14/main.tex}
	\item Five cards—the ten, jack, queen, king and ace of diamonds, are well-shuffled with their face downwards. One card is then picked up at random.
\begin{enumerate}
\item
What is the probability that the card is the queen? 
\item
If the queen is drawn and put aside, what is the probability that the second card picked up is (a) an ace? (b) a queen?\\
\end{enumerate}
\solution
		%\input{ncert/10/15/1/15/defs.tex}
	\item A bag contains $5$ red balls and some blue balls. If the probability of drawing a blue ball is double that if a red ball, determine the number of blue balls in the bag. 
		\\
\solution
		%\input{ncert/10/15/2/3/defs.tex}
	\item A card is selected from a pack of 52 cards.
 \begin{enumerate}[label=(\alph*)] 
                 \item How many points are there in the sample space?
                 \item Calculate the probability that the card is an ace of spades.
                 \item Calculate the probability that the card is (i) an ace and (ii) black card.
 \end{enumerate}
\solution
		%\input{ncert/11/16/3/4/main.tex}
\item Four cards are drawn from a well-shuffled deck of 52 cards. What is the probability of obtaining 3 diamonds and one spade.
\\
\solution
		%\input{ncert/11/16/4/2/defs.tex}
\item In a certain lottery 10,000 tickets are sold and ten equal prizes are awarded. What is the probability of not getting a prize if you buy (a) one ticket (b) two tickets (c) 10 tickets ?	
\\
\solution
		%\input{ncert/11/16/4/4/defs.tex}
		%
\item 
Out of 100 students, two sections of 40 and 60 are formed. If you and your friend are among the 100 students, what is the probability that
\begin{enumerate}
\item you both enter the same section?
\item you both enter the different sections?
\end{enumerate}
\solution
		%\input{ncert/11/16/4/5/defs.tex}
	\item 
The number lock of a suitcase has 4 wheels each labelled with ten digits i.e. from 0 to 9.The lock opens with a sequence of four digits with no repeats.What is the probability of a person getting the right sequence to open the suitcase.
\\
\solution
		%\input{ncert/11/16/4/10/defs.tex}
		%
\item 
Two cards are drawn at random and without replacement from a pack of 52 playing cards. Find the probability that both the cards are black.
\\
\solution
		%\input{ncert/12/13/2/2/defs.tex}
		\item A box of oranges is inspected by examining three randomly selected oranges drawn without replacement. If all the three oranges are good, the box is approved for sale, otherwise, it is rejected. Find the probability that a box containing 15 oranges out of which 12 are good and 3 are bad ones will be approved for sale.
		\label{ncert/12/13/2/3/defs.tex}
		\item Two balls are drawn at random with replacement from a box containing 10 black and 8 red balls. Find the probability that
		\label{ncert/12/13/2/12}
\begin{enumerate}
\item both balls are red.
\item first ball is black and second is red.
\item one of them is black and other is red.
\end{enumerate}

\item In a hostel, 60\% of the students read Hindi newspaper, 40\% read English newspaper and 20\% read both Hindi and English newspapers. A student is selected at random.
		\label{ncert/12/13/2/15}
\begin{enumerate}
\item Find the probability that she reads neither Hindi nor English newspapers.
\item If she reads Hindi newspaper, find the probability that she reads English newspaper.
\item If she reads English newspaper, find the probability that she reads Hindi newspaper.\\
\end{enumerate}
\item The probability of obtaining an even prime number on each die, when a pair of dice is rolled is 
\begin{enumerate}
    \item $0$ 
    
    \item $\frac{1}{3}$ 
    
    \item $\frac{1}{12}$ 
    
    \item $\frac{1}{36}$ 
\end{enumerate}
\solution
		%\input{ncert/12/13/2/17/defs.tex}
	\item A bag contains 4 red and 4 black balls, another bag contains 2 red and 6 black balls. One of the two bags is selected at random and a ball is drawn from the bag which is found to be red. Find the probability that the ball is drawn from the first bag.
\\
\solution
		%\input{ncert/12/13/3/2/main.tex}
  \item
  Cards with numbers 2 to 101 are placed in a box. A card is selected at random.Find the probability that the card has
\begin{enumerate}[label=(\roman*)]
	\item an even number 
	\item a square number
\end{enumerate}
\solution
%\input{exemplar/10/13/3/32/main.tex}
\item
The king, queen and jack of clubs are removed from a deck of 52 playing cards and then well shuffled. Now one card is drawn at random from the remaining cards.  Determine the probability that the card is
\begin{enumerate}[label=(\roman*)]
\item a club
\item 10 of hearts
\end{enumerate}
\solution
%\input{exemplar/10/13/3/29/main.tex}
\item A team of medical students doing their internship have to assist during surgeries
at a city hospital. The probabilities of surgeries rated as very complex, complex,
routine, simple or very simple are respectively, 0.15, 0.20, 0.31, 0.26, .08. Find
the probabilities that a particular surgery will be rated
\begin{enumerate}
	\item complex or very complex;
	\item neither very complex nor very simple;
	\item routine or complex
	\item routine or simple
\end{enumerate}
\solution
%\input{exemplar/11/16/3/8(1)/main.tex}
\item A card is selected from a pack of 52 cards.
\begin{enumerate}[label=(\alph*)]
    \item How many points are there in the sample space?
    \item Calculate the probability that the card is an ace of spades.
    \item Calculate the probability that the card is (i) an ace and (ii) black card.
\end{enumerate}
\solution
%\input{exemplar/11/16/3/4/main2.tex}
\item The probability that a non leap year selected at random will contain 53 sundays.
\\
\solution
%\input{exemplar/10/13/1/19/main.tex}
\item One of the four persons John, Rita, Aslam or Gurpreet will be promoted next
month. Consequently the sample space consists of four elementary outcomes
S = {John promoted, Rita promoted, Aslam promoted, Gurpreet promoted}
You are told that the chances of John’s promotion is same as that of Gurpreet,
Rita’s chances of promotion are twice as likely as Johns. Aslam’s chances are
four times that of John.
\begin{enumerate}
	\item Determine
	\begin{enumerate}
		\item P (John promoted)
		\item P (Rita promoted)
		\item P (Aslam promoted)
		\item P (Gurpreet promoted)
	\end{enumerate}
	\item If A = {John promoted or Gurpreet promoted}, find P (A).
\end{enumerate}
\solution
%\input{exemplar/11/16/3/10/main.tex}
\item A card is drawn from a deck of 52 cards. Find the probability of getting a king or a heart or a red card.\\
\solution
%\input{exemplar/11/16/3/15/main.tex}
\item The probability that a student will pass his examination is 0.73, the probability of
the student getting a compartment is 0.13, and the probability that the student will
either pass or get compartment is 0.96. State True or False.\\
\solution
%\input{exemplar/11/16/3/31/main.tex}
\item A card is selected from a pack of 52 cards\\
\begin{enumerate}[label=(\alph*)]
\item How many points are there in the sample space?
\item Calculate the probability that the cards is an ace of spades.
\item Calculate the probability that the card is (i) an ace (ii)black card.\\
\end{enumerate}
%\input{ncert/11/16/3/4_1/Prob_4.tex}
\item In a non-leap year, the probability of having 53 tuesdays or 53 wednesdays is\\
\solution
%\input{exemplar/11/16/3/18/main.tex}
\item There are 1000 sealed envelopes in a box, 10 of them contain a cash prize of
Rs 100 each, 100 of them contain a cash prize of Rs 50 each and 200 of them
contain a cash prize of Rs 10 each and rest do not contain any cash prize. If they
are well shuffled and an envelope is picked up out, what is the probability that it
contains no cash prize?\\
\solution
%\input{exemplar/10/13/3/34/main.tex}
\item 
A die is thrown and a card is selected at random from a deck of 52 playing cards. The probability of getting an even number on the die and a spade card.\\
\solution
%\input{exemplar/12/13/3/78/main.tex}
\item
If 4-digit numbers greater than 5,000 are randomly formed from the digits 0, 1, 3, 5, and 7, what is the probability of forming a number divisible by 5 when:
\begin{enumerate}
    \item The digits are repeated?
    \item The repetition of digits is not allowed?
\end{enumerate}
\solution
%\input{ncert/11/16/4/9/main.tex}
\item Consider the probability space $\brak{\Omega, \mathcal{G}, P}$ where $\Omega = [0,2]$ and $\mathcal{G} = \cbrak{\phi, \Omega, [0,1], (1,2]}$. Let $X$ and $Y$ be two functions on $\Omega$ defined as
\begin{align*}
    X(\omega) = 
    \begin{cases}
        1 & \text{if }\omega \in [0, 1]\\
        2 & \text{if }\omega \in (1, 2]
    \end{cases}
\end{align*}
and
\begin{align*}
    Y(\omega) = 
    \begin{cases}
        2 & \text{if }\omega \in [0, 1.5]\\
        3 & \text{if }\omega \in (1.5, 2].
    \end{cases}
\end{align*}
Then which one of the following statements is true?
\begin{enumerate}
    \item [(A)] $X$ is a random variable with respect to $\mathcal{G}$, but $Y$ is not a random variable with respect to $\mathcal{G}$.
    \item [(B)] $Y$ is a random variable with respect to $\mathcal{G}$, but $X$ is not a random variable with respect to $\mathcal{G}$.
    \item [(C)] Neither $X$ nor $Y$ is a random variable with respect to $\mathcal{G}$.
    \item [(D)] Both $X$ and $Y$ are random variables with respect to $\mathcal{G}$.
\end{enumerate} \hfill (GATE ST 2023)\\
\solution
%\input{gate/ST/2023/14/main.tex}
	\item  A die is loaded in such a way that each odd number is twice as likely to occur as
each even number. Find $P(G)$, where $G$ is the event that a number greater than
3 occurs on a single roll of the die.
\\
\solution
		%\input{exemplar/11/16/3/5/main.tex}
	\item All the jacks, queens and kings are removed from a deck of 52 playing cards. The remaining cards are well shuffled and then one card is drawn at random. Giving ace a value 1 similar value for other cards, find the probability that the card has a value 
		\begin{enumerate}
			\item 7
			\item greater than 7
			\item less than 7
		\end{enumerate}
		%\input{exemplar/10/13/3/30/main.tex}
  \item A Lot consists of 48 mobile phones of which 42 are good, 3 have only minor defects and 3 have major defects.Varnika will buy a phone if it is good but the trader will only buy a mobile if it has no major defects. One phone is selected at random from the lot. What is the probability that it is
\begin{enumerate}
	\item acceptable to Varnika?
            \item acceptable to the trader?
\end{enumerate}
\solution
	%\input{exemplar/10/13/3/40/main.tex}
 \item A student says that if you throw a die, it will show up 1 or not 1. Therefore, the probability of getting 1 and the probability of getting 'not 1' each is equal to $\frac{1}{2}$. Is this correct? Give reasons.\\
 \solution
        %\input{exemplar/10/13/2/9/main.tex}
   \item Four candidates A, B, C, D have ap-
plied for the assignment to coach a school cricket
team. If A is twice as likely to be selected as B, and
B and C are given about the same chance of being
selected, while C is twice as likely to be selected
as D, what are the probabilities that
\begin{enumerate}
\item C will be selected?
\item A will not be selected?
\end{enumerate}
	%\input{exemplar/11/16/3/9/main.tex}
 \item A bag contain 24 balls of which $x$ balls are red, $2x$ are white and $3x$ are blue. A ball is selected at random, What is the probability that it is
\begin{enumerate}[label=\alph*)]
\item not red ?
\item white ?
\end{enumerate}
%\input{exemplar/10/13/3/41/main.tex}
If the letters of the word ASSASSINATION are arranged at random. Find the Probability that
\begin{enumerate}[label=(\alph*)]
\item Four $S's$ come consecutively in the word
\item Two  $I's$ and two $N's$ come together
\item All $A's$ are not coming together
\item No two $A's$ are coming together
\end{enumerate}
%\input{exemplar/11/16/3/14/main.tex}
	\item One urn contains two black balls (labelled B1 and B2) and one white ball. A
	second urn contains one black ball and two white balls (labelled W1 and W2).
	Suppose the following experiment is performed. One of the two urns is chosen
	at random. Next a ball is randomly chosen from the urn. Then a second ball is
	chosen at random from the same urn without replacing the first ball.
	
	\begin{enumerate}
	\item What is the probability that two black balls are chosen?
	
	\item What is the probability that two balls of opposite colour are chosen?
	\end{enumerate}
	\solution
	%\input{exemplar/11/16/3/12/main1.tex}
\end{enumerate}

\item In a certain lottery 10,000 tickets are sold and ten equal prizes are awarded. What is the probability of not getting a prize if you buy (a) one ticket (b) two tickets (c) 10 tickets ?	
\\
\solution
		%\begin{enumerate}[label=\thesection.\arabic*,ref=\thesection.\theenumi]
	\item One card is drawn from a well-shuffled deck of 52 cards. Find the probability of getting
\begin{enumerate}
\item A king of red colour 
\item A face card 
\item A red face card
\item The jack of hearts
\item A spade
\item The queen of diamonds

\end{enumerate}
\solution
		%\input{ncert/10/15/1/14/main.tex}
	\item Five cards—the ten, jack, queen, king and ace of diamonds, are well-shuffled with their face downwards. One card is then picked up at random.
\begin{enumerate}
\item
What is the probability that the card is the queen? 
\item
If the queen is drawn and put aside, what is the probability that the second card picked up is (a) an ace? (b) a queen?\\
\end{enumerate}
\solution
		%\input{ncert/10/15/1/15/defs.tex}
	\item A bag contains $5$ red balls and some blue balls. If the probability of drawing a blue ball is double that if a red ball, determine the number of blue balls in the bag. 
		\\
\solution
		%\input{ncert/10/15/2/3/defs.tex}
	\item A card is selected from a pack of 52 cards.
 \begin{enumerate}[label=(\alph*)] 
                 \item How many points are there in the sample space?
                 \item Calculate the probability that the card is an ace of spades.
                 \item Calculate the probability that the card is (i) an ace and (ii) black card.
 \end{enumerate}
\solution
		%\input{ncert/11/16/3/4/main.tex}
\item Four cards are drawn from a well-shuffled deck of 52 cards. What is the probability of obtaining 3 diamonds and one spade.
\\
\solution
		%\input{ncert/11/16/4/2/defs.tex}
\item In a certain lottery 10,000 tickets are sold and ten equal prizes are awarded. What is the probability of not getting a prize if you buy (a) one ticket (b) two tickets (c) 10 tickets ?	
\\
\solution
		%\input{ncert/11/16/4/4/defs.tex}
		%
\item 
Out of 100 students, two sections of 40 and 60 are formed. If you and your friend are among the 100 students, what is the probability that
\begin{enumerate}
\item you both enter the same section?
\item you both enter the different sections?
\end{enumerate}
\solution
		%\input{ncert/11/16/4/5/defs.tex}
	\item 
The number lock of a suitcase has 4 wheels each labelled with ten digits i.e. from 0 to 9.The lock opens with a sequence of four digits with no repeats.What is the probability of a person getting the right sequence to open the suitcase.
\\
\solution
		%\input{ncert/11/16/4/10/defs.tex}
		%
\item 
Two cards are drawn at random and without replacement from a pack of 52 playing cards. Find the probability that both the cards are black.
\\
\solution
		%\input{ncert/12/13/2/2/defs.tex}
		\item A box of oranges is inspected by examining three randomly selected oranges drawn without replacement. If all the three oranges are good, the box is approved for sale, otherwise, it is rejected. Find the probability that a box containing 15 oranges out of which 12 are good and 3 are bad ones will be approved for sale.
		\label{ncert/12/13/2/3/defs.tex}
		\item Two balls are drawn at random with replacement from a box containing 10 black and 8 red balls. Find the probability that
		\label{ncert/12/13/2/12}
\begin{enumerate}
\item both balls are red.
\item first ball is black and second is red.
\item one of them is black and other is red.
\end{enumerate}

\item In a hostel, 60\% of the students read Hindi newspaper, 40\% read English newspaper and 20\% read both Hindi and English newspapers. A student is selected at random.
		\label{ncert/12/13/2/15}
\begin{enumerate}
\item Find the probability that she reads neither Hindi nor English newspapers.
\item If she reads Hindi newspaper, find the probability that she reads English newspaper.
\item If she reads English newspaper, find the probability that she reads Hindi newspaper.\\
\end{enumerate}
\item The probability of obtaining an even prime number on each die, when a pair of dice is rolled is 
\begin{enumerate}
    \item $0$ 
    
    \item $\frac{1}{3}$ 
    
    \item $\frac{1}{12}$ 
    
    \item $\frac{1}{36}$ 
\end{enumerate}
\solution
		%\input{ncert/12/13/2/17/defs.tex}
	\item A bag contains 4 red and 4 black balls, another bag contains 2 red and 6 black balls. One of the two bags is selected at random and a ball is drawn from the bag which is found to be red. Find the probability that the ball is drawn from the first bag.
\\
\solution
		%\input{ncert/12/13/3/2/main.tex}
  \item
  Cards with numbers 2 to 101 are placed in a box. A card is selected at random.Find the probability that the card has
\begin{enumerate}[label=(\roman*)]
	\item an even number 
	\item a square number
\end{enumerate}
\solution
%\input{exemplar/10/13/3/32/main.tex}
\item
The king, queen and jack of clubs are removed from a deck of 52 playing cards and then well shuffled. Now one card is drawn at random from the remaining cards.  Determine the probability that the card is
\begin{enumerate}[label=(\roman*)]
\item a club
\item 10 of hearts
\end{enumerate}
\solution
%\input{exemplar/10/13/3/29/main.tex}
\item A team of medical students doing their internship have to assist during surgeries
at a city hospital. The probabilities of surgeries rated as very complex, complex,
routine, simple or very simple are respectively, 0.15, 0.20, 0.31, 0.26, .08. Find
the probabilities that a particular surgery will be rated
\begin{enumerate}
	\item complex or very complex;
	\item neither very complex nor very simple;
	\item routine or complex
	\item routine or simple
\end{enumerate}
\solution
%\input{exemplar/11/16/3/8(1)/main.tex}
\item A card is selected from a pack of 52 cards.
\begin{enumerate}[label=(\alph*)]
    \item How many points are there in the sample space?
    \item Calculate the probability that the card is an ace of spades.
    \item Calculate the probability that the card is (i) an ace and (ii) black card.
\end{enumerate}
\solution
%\input{exemplar/11/16/3/4/main2.tex}
\item The probability that a non leap year selected at random will contain 53 sundays.
\\
\solution
%\input{exemplar/10/13/1/19/main.tex}
\item One of the four persons John, Rita, Aslam or Gurpreet will be promoted next
month. Consequently the sample space consists of four elementary outcomes
S = {John promoted, Rita promoted, Aslam promoted, Gurpreet promoted}
You are told that the chances of John’s promotion is same as that of Gurpreet,
Rita’s chances of promotion are twice as likely as Johns. Aslam’s chances are
four times that of John.
\begin{enumerate}
	\item Determine
	\begin{enumerate}
		\item P (John promoted)
		\item P (Rita promoted)
		\item P (Aslam promoted)
		\item P (Gurpreet promoted)
	\end{enumerate}
	\item If A = {John promoted or Gurpreet promoted}, find P (A).
\end{enumerate}
\solution
%\input{exemplar/11/16/3/10/main.tex}
\item A card is drawn from a deck of 52 cards. Find the probability of getting a king or a heart or a red card.\\
\solution
%\input{exemplar/11/16/3/15/main.tex}
\item The probability that a student will pass his examination is 0.73, the probability of
the student getting a compartment is 0.13, and the probability that the student will
either pass or get compartment is 0.96. State True or False.\\
\solution
%\input{exemplar/11/16/3/31/main.tex}
\item A card is selected from a pack of 52 cards\\
\begin{enumerate}[label=(\alph*)]
\item How many points are there in the sample space?
\item Calculate the probability that the cards is an ace of spades.
\item Calculate the probability that the card is (i) an ace (ii)black card.\\
\end{enumerate}
%\input{ncert/11/16/3/4_1/Prob_4.tex}
\item In a non-leap year, the probability of having 53 tuesdays or 53 wednesdays is\\
\solution
%\input{exemplar/11/16/3/18/main.tex}
\item There are 1000 sealed envelopes in a box, 10 of them contain a cash prize of
Rs 100 each, 100 of them contain a cash prize of Rs 50 each and 200 of them
contain a cash prize of Rs 10 each and rest do not contain any cash prize. If they
are well shuffled and an envelope is picked up out, what is the probability that it
contains no cash prize?\\
\solution
%\input{exemplar/10/13/3/34/main.tex}
\item 
A die is thrown and a card is selected at random from a deck of 52 playing cards. The probability of getting an even number on the die and a spade card.\\
\solution
%\input{exemplar/12/13/3/78/main.tex}
\item
If 4-digit numbers greater than 5,000 are randomly formed from the digits 0, 1, 3, 5, and 7, what is the probability of forming a number divisible by 5 when:
\begin{enumerate}
    \item The digits are repeated?
    \item The repetition of digits is not allowed?
\end{enumerate}
\solution
%\input{ncert/11/16/4/9/main.tex}
\item Consider the probability space $\brak{\Omega, \mathcal{G}, P}$ where $\Omega = [0,2]$ and $\mathcal{G} = \cbrak{\phi, \Omega, [0,1], (1,2]}$. Let $X$ and $Y$ be two functions on $\Omega$ defined as
\begin{align*}
    X(\omega) = 
    \begin{cases}
        1 & \text{if }\omega \in [0, 1]\\
        2 & \text{if }\omega \in (1, 2]
    \end{cases}
\end{align*}
and
\begin{align*}
    Y(\omega) = 
    \begin{cases}
        2 & \text{if }\omega \in [0, 1.5]\\
        3 & \text{if }\omega \in (1.5, 2].
    \end{cases}
\end{align*}
Then which one of the following statements is true?
\begin{enumerate}
    \item [(A)] $X$ is a random variable with respect to $\mathcal{G}$, but $Y$ is not a random variable with respect to $\mathcal{G}$.
    \item [(B)] $Y$ is a random variable with respect to $\mathcal{G}$, but $X$ is not a random variable with respect to $\mathcal{G}$.
    \item [(C)] Neither $X$ nor $Y$ is a random variable with respect to $\mathcal{G}$.
    \item [(D)] Both $X$ and $Y$ are random variables with respect to $\mathcal{G}$.
\end{enumerate} \hfill (GATE ST 2023)\\
\solution
%\input{gate/ST/2023/14/main.tex}
	\item  A die is loaded in such a way that each odd number is twice as likely to occur as
each even number. Find $P(G)$, where $G$ is the event that a number greater than
3 occurs on a single roll of the die.
\\
\solution
		%\input{exemplar/11/16/3/5/main.tex}
	\item All the jacks, queens and kings are removed from a deck of 52 playing cards. The remaining cards are well shuffled and then one card is drawn at random. Giving ace a value 1 similar value for other cards, find the probability that the card has a value 
		\begin{enumerate}
			\item 7
			\item greater than 7
			\item less than 7
		\end{enumerate}
		%\input{exemplar/10/13/3/30/main.tex}
  \item A Lot consists of 48 mobile phones of which 42 are good, 3 have only minor defects and 3 have major defects.Varnika will buy a phone if it is good but the trader will only buy a mobile if it has no major defects. One phone is selected at random from the lot. What is the probability that it is
\begin{enumerate}
	\item acceptable to Varnika?
            \item acceptable to the trader?
\end{enumerate}
\solution
	%\input{exemplar/10/13/3/40/main.tex}
 \item A student says that if you throw a die, it will show up 1 or not 1. Therefore, the probability of getting 1 and the probability of getting 'not 1' each is equal to $\frac{1}{2}$. Is this correct? Give reasons.\\
 \solution
        %\input{exemplar/10/13/2/9/main.tex}
   \item Four candidates A, B, C, D have ap-
plied for the assignment to coach a school cricket
team. If A is twice as likely to be selected as B, and
B and C are given about the same chance of being
selected, while C is twice as likely to be selected
as D, what are the probabilities that
\begin{enumerate}
\item C will be selected?
\item A will not be selected?
\end{enumerate}
	%\input{exemplar/11/16/3/9/main.tex}
 \item A bag contain 24 balls of which $x$ balls are red, $2x$ are white and $3x$ are blue. A ball is selected at random, What is the probability that it is
\begin{enumerate}[label=\alph*)]
\item not red ?
\item white ?
\end{enumerate}
%\input{exemplar/10/13/3/41/main.tex}
If the letters of the word ASSASSINATION are arranged at random. Find the Probability that
\begin{enumerate}[label=(\alph*)]
\item Four $S's$ come consecutively in the word
\item Two  $I's$ and two $N's$ come together
\item All $A's$ are not coming together
\item No two $A's$ are coming together
\end{enumerate}
%\input{exemplar/11/16/3/14/main.tex}
	\item One urn contains two black balls (labelled B1 and B2) and one white ball. A
	second urn contains one black ball and two white balls (labelled W1 and W2).
	Suppose the following experiment is performed. One of the two urns is chosen
	at random. Next a ball is randomly chosen from the urn. Then a second ball is
	chosen at random from the same urn without replacing the first ball.
	
	\begin{enumerate}
	\item What is the probability that two black balls are chosen?
	
	\item What is the probability that two balls of opposite colour are chosen?
	\end{enumerate}
	\solution
	%\input{exemplar/11/16/3/12/main1.tex}
\end{enumerate}

		%
\item 
Out of 100 students, two sections of 40 and 60 are formed. If you and your friend are among the 100 students, what is the probability that
\begin{enumerate}
\item you both enter the same section?
\item you both enter the different sections?
\end{enumerate}
\solution
		%\begin{enumerate}[label=\thesection.\arabic*,ref=\thesection.\theenumi]
	\item One card is drawn from a well-shuffled deck of 52 cards. Find the probability of getting
\begin{enumerate}
\item A king of red colour 
\item A face card 
\item A red face card
\item The jack of hearts
\item A spade
\item The queen of diamonds

\end{enumerate}
\solution
		%\input{ncert/10/15/1/14/main.tex}
	\item Five cards—the ten, jack, queen, king and ace of diamonds, are well-shuffled with their face downwards. One card is then picked up at random.
\begin{enumerate}
\item
What is the probability that the card is the queen? 
\item
If the queen is drawn and put aside, what is the probability that the second card picked up is (a) an ace? (b) a queen?\\
\end{enumerate}
\solution
		%\input{ncert/10/15/1/15/defs.tex}
	\item A bag contains $5$ red balls and some blue balls. If the probability of drawing a blue ball is double that if a red ball, determine the number of blue balls in the bag. 
		\\
\solution
		%\input{ncert/10/15/2/3/defs.tex}
	\item A card is selected from a pack of 52 cards.
 \begin{enumerate}[label=(\alph*)] 
                 \item How many points are there in the sample space?
                 \item Calculate the probability that the card is an ace of spades.
                 \item Calculate the probability that the card is (i) an ace and (ii) black card.
 \end{enumerate}
\solution
		%\input{ncert/11/16/3/4/main.tex}
\item Four cards are drawn from a well-shuffled deck of 52 cards. What is the probability of obtaining 3 diamonds and one spade.
\\
\solution
		%\input{ncert/11/16/4/2/defs.tex}
\item In a certain lottery 10,000 tickets are sold and ten equal prizes are awarded. What is the probability of not getting a prize if you buy (a) one ticket (b) two tickets (c) 10 tickets ?	
\\
\solution
		%\input{ncert/11/16/4/4/defs.tex}
		%
\item 
Out of 100 students, two sections of 40 and 60 are formed. If you and your friend are among the 100 students, what is the probability that
\begin{enumerate}
\item you both enter the same section?
\item you both enter the different sections?
\end{enumerate}
\solution
		%\input{ncert/11/16/4/5/defs.tex}
	\item 
The number lock of a suitcase has 4 wheels each labelled with ten digits i.e. from 0 to 9.The lock opens with a sequence of four digits with no repeats.What is the probability of a person getting the right sequence to open the suitcase.
\\
\solution
		%\input{ncert/11/16/4/10/defs.tex}
		%
\item 
Two cards are drawn at random and without replacement from a pack of 52 playing cards. Find the probability that both the cards are black.
\\
\solution
		%\input{ncert/12/13/2/2/defs.tex}
		\item A box of oranges is inspected by examining three randomly selected oranges drawn without replacement. If all the three oranges are good, the box is approved for sale, otherwise, it is rejected. Find the probability that a box containing 15 oranges out of which 12 are good and 3 are bad ones will be approved for sale.
		\label{ncert/12/13/2/3/defs.tex}
		\item Two balls are drawn at random with replacement from a box containing 10 black and 8 red balls. Find the probability that
		\label{ncert/12/13/2/12}
\begin{enumerate}
\item both balls are red.
\item first ball is black and second is red.
\item one of them is black and other is red.
\end{enumerate}

\item In a hostel, 60\% of the students read Hindi newspaper, 40\% read English newspaper and 20\% read both Hindi and English newspapers. A student is selected at random.
		\label{ncert/12/13/2/15}
\begin{enumerate}
\item Find the probability that she reads neither Hindi nor English newspapers.
\item If she reads Hindi newspaper, find the probability that she reads English newspaper.
\item If she reads English newspaper, find the probability that she reads Hindi newspaper.\\
\end{enumerate}
\item The probability of obtaining an even prime number on each die, when a pair of dice is rolled is 
\begin{enumerate}
    \item $0$ 
    
    \item $\frac{1}{3}$ 
    
    \item $\frac{1}{12}$ 
    
    \item $\frac{1}{36}$ 
\end{enumerate}
\solution
		%\input{ncert/12/13/2/17/defs.tex}
	\item A bag contains 4 red and 4 black balls, another bag contains 2 red and 6 black balls. One of the two bags is selected at random and a ball is drawn from the bag which is found to be red. Find the probability that the ball is drawn from the first bag.
\\
\solution
		%\input{ncert/12/13/3/2/main.tex}
  \item
  Cards with numbers 2 to 101 are placed in a box. A card is selected at random.Find the probability that the card has
\begin{enumerate}[label=(\roman*)]
	\item an even number 
	\item a square number
\end{enumerate}
\solution
%\input{exemplar/10/13/3/32/main.tex}
\item
The king, queen and jack of clubs are removed from a deck of 52 playing cards and then well shuffled. Now one card is drawn at random from the remaining cards.  Determine the probability that the card is
\begin{enumerate}[label=(\roman*)]
\item a club
\item 10 of hearts
\end{enumerate}
\solution
%\input{exemplar/10/13/3/29/main.tex}
\item A team of medical students doing their internship have to assist during surgeries
at a city hospital. The probabilities of surgeries rated as very complex, complex,
routine, simple or very simple are respectively, 0.15, 0.20, 0.31, 0.26, .08. Find
the probabilities that a particular surgery will be rated
\begin{enumerate}
	\item complex or very complex;
	\item neither very complex nor very simple;
	\item routine or complex
	\item routine or simple
\end{enumerate}
\solution
%\input{exemplar/11/16/3/8(1)/main.tex}
\item A card is selected from a pack of 52 cards.
\begin{enumerate}[label=(\alph*)]
    \item How many points are there in the sample space?
    \item Calculate the probability that the card is an ace of spades.
    \item Calculate the probability that the card is (i) an ace and (ii) black card.
\end{enumerate}
\solution
%\input{exemplar/11/16/3/4/main2.tex}
\item The probability that a non leap year selected at random will contain 53 sundays.
\\
\solution
%\input{exemplar/10/13/1/19/main.tex}
\item One of the four persons John, Rita, Aslam or Gurpreet will be promoted next
month. Consequently the sample space consists of four elementary outcomes
S = {John promoted, Rita promoted, Aslam promoted, Gurpreet promoted}
You are told that the chances of John’s promotion is same as that of Gurpreet,
Rita’s chances of promotion are twice as likely as Johns. Aslam’s chances are
four times that of John.
\begin{enumerate}
	\item Determine
	\begin{enumerate}
		\item P (John promoted)
		\item P (Rita promoted)
		\item P (Aslam promoted)
		\item P (Gurpreet promoted)
	\end{enumerate}
	\item If A = {John promoted or Gurpreet promoted}, find P (A).
\end{enumerate}
\solution
%\input{exemplar/11/16/3/10/main.tex}
\item A card is drawn from a deck of 52 cards. Find the probability of getting a king or a heart or a red card.\\
\solution
%\input{exemplar/11/16/3/15/main.tex}
\item The probability that a student will pass his examination is 0.73, the probability of
the student getting a compartment is 0.13, and the probability that the student will
either pass or get compartment is 0.96. State True or False.\\
\solution
%\input{exemplar/11/16/3/31/main.tex}
\item A card is selected from a pack of 52 cards\\
\begin{enumerate}[label=(\alph*)]
\item How many points are there in the sample space?
\item Calculate the probability that the cards is an ace of spades.
\item Calculate the probability that the card is (i) an ace (ii)black card.\\
\end{enumerate}
%\input{ncert/11/16/3/4_1/Prob_4.tex}
\item In a non-leap year, the probability of having 53 tuesdays or 53 wednesdays is\\
\solution
%\input{exemplar/11/16/3/18/main.tex}
\item There are 1000 sealed envelopes in a box, 10 of them contain a cash prize of
Rs 100 each, 100 of them contain a cash prize of Rs 50 each and 200 of them
contain a cash prize of Rs 10 each and rest do not contain any cash prize. If they
are well shuffled and an envelope is picked up out, what is the probability that it
contains no cash prize?\\
\solution
%\input{exemplar/10/13/3/34/main.tex}
\item 
A die is thrown and a card is selected at random from a deck of 52 playing cards. The probability of getting an even number on the die and a spade card.\\
\solution
%\input{exemplar/12/13/3/78/main.tex}
\item
If 4-digit numbers greater than 5,000 are randomly formed from the digits 0, 1, 3, 5, and 7, what is the probability of forming a number divisible by 5 when:
\begin{enumerate}
    \item The digits are repeated?
    \item The repetition of digits is not allowed?
\end{enumerate}
\solution
%\input{ncert/11/16/4/9/main.tex}
\item Consider the probability space $\brak{\Omega, \mathcal{G}, P}$ where $\Omega = [0,2]$ and $\mathcal{G} = \cbrak{\phi, \Omega, [0,1], (1,2]}$. Let $X$ and $Y$ be two functions on $\Omega$ defined as
\begin{align*}
    X(\omega) = 
    \begin{cases}
        1 & \text{if }\omega \in [0, 1]\\
        2 & \text{if }\omega \in (1, 2]
    \end{cases}
\end{align*}
and
\begin{align*}
    Y(\omega) = 
    \begin{cases}
        2 & \text{if }\omega \in [0, 1.5]\\
        3 & \text{if }\omega \in (1.5, 2].
    \end{cases}
\end{align*}
Then which one of the following statements is true?
\begin{enumerate}
    \item [(A)] $X$ is a random variable with respect to $\mathcal{G}$, but $Y$ is not a random variable with respect to $\mathcal{G}$.
    \item [(B)] $Y$ is a random variable with respect to $\mathcal{G}$, but $X$ is not a random variable with respect to $\mathcal{G}$.
    \item [(C)] Neither $X$ nor $Y$ is a random variable with respect to $\mathcal{G}$.
    \item [(D)] Both $X$ and $Y$ are random variables with respect to $\mathcal{G}$.
\end{enumerate} \hfill (GATE ST 2023)\\
\solution
%\input{gate/ST/2023/14/main.tex}
	\item  A die is loaded in such a way that each odd number is twice as likely to occur as
each even number. Find $P(G)$, where $G$ is the event that a number greater than
3 occurs on a single roll of the die.
\\
\solution
		%\input{exemplar/11/16/3/5/main.tex}
	\item All the jacks, queens and kings are removed from a deck of 52 playing cards. The remaining cards are well shuffled and then one card is drawn at random. Giving ace a value 1 similar value for other cards, find the probability that the card has a value 
		\begin{enumerate}
			\item 7
			\item greater than 7
			\item less than 7
		\end{enumerate}
		%\input{exemplar/10/13/3/30/main.tex}
  \item A Lot consists of 48 mobile phones of which 42 are good, 3 have only minor defects and 3 have major defects.Varnika will buy a phone if it is good but the trader will only buy a mobile if it has no major defects. One phone is selected at random from the lot. What is the probability that it is
\begin{enumerate}
	\item acceptable to Varnika?
            \item acceptable to the trader?
\end{enumerate}
\solution
	%\input{exemplar/10/13/3/40/main.tex}
 \item A student says that if you throw a die, it will show up 1 or not 1. Therefore, the probability of getting 1 and the probability of getting 'not 1' each is equal to $\frac{1}{2}$. Is this correct? Give reasons.\\
 \solution
        %\input{exemplar/10/13/2/9/main.tex}
   \item Four candidates A, B, C, D have ap-
plied for the assignment to coach a school cricket
team. If A is twice as likely to be selected as B, and
B and C are given about the same chance of being
selected, while C is twice as likely to be selected
as D, what are the probabilities that
\begin{enumerate}
\item C will be selected?
\item A will not be selected?
\end{enumerate}
	%\input{exemplar/11/16/3/9/main.tex}
 \item A bag contain 24 balls of which $x$ balls are red, $2x$ are white and $3x$ are blue. A ball is selected at random, What is the probability that it is
\begin{enumerate}[label=\alph*)]
\item not red ?
\item white ?
\end{enumerate}
%\input{exemplar/10/13/3/41/main.tex}
If the letters of the word ASSASSINATION are arranged at random. Find the Probability that
\begin{enumerate}[label=(\alph*)]
\item Four $S's$ come consecutively in the word
\item Two  $I's$ and two $N's$ come together
\item All $A's$ are not coming together
\item No two $A's$ are coming together
\end{enumerate}
%\input{exemplar/11/16/3/14/main.tex}
	\item One urn contains two black balls (labelled B1 and B2) and one white ball. A
	second urn contains one black ball and two white balls (labelled W1 and W2).
	Suppose the following experiment is performed. One of the two urns is chosen
	at random. Next a ball is randomly chosen from the urn. Then a second ball is
	chosen at random from the same urn without replacing the first ball.
	
	\begin{enumerate}
	\item What is the probability that two black balls are chosen?
	
	\item What is the probability that two balls of opposite colour are chosen?
	\end{enumerate}
	\solution
	%\input{exemplar/11/16/3/12/main1.tex}
\end{enumerate}

	\item 
The number lock of a suitcase has 4 wheels each labelled with ten digits i.e. from 0 to 9.The lock opens with a sequence of four digits with no repeats.What is the probability of a person getting the right sequence to open the suitcase.
\\
\solution
		%\begin{enumerate}[label=\thesection.\arabic*,ref=\thesection.\theenumi]
	\item One card is drawn from a well-shuffled deck of 52 cards. Find the probability of getting
\begin{enumerate}
\item A king of red colour 
\item A face card 
\item A red face card
\item The jack of hearts
\item A spade
\item The queen of diamonds

\end{enumerate}
\solution
		%\input{ncert/10/15/1/14/main.tex}
	\item Five cards—the ten, jack, queen, king and ace of diamonds, are well-shuffled with their face downwards. One card is then picked up at random.
\begin{enumerate}
\item
What is the probability that the card is the queen? 
\item
If the queen is drawn and put aside, what is the probability that the second card picked up is (a) an ace? (b) a queen?\\
\end{enumerate}
\solution
		%\input{ncert/10/15/1/15/defs.tex}
	\item A bag contains $5$ red balls and some blue balls. If the probability of drawing a blue ball is double that if a red ball, determine the number of blue balls in the bag. 
		\\
\solution
		%\input{ncert/10/15/2/3/defs.tex}
	\item A card is selected from a pack of 52 cards.
 \begin{enumerate}[label=(\alph*)] 
                 \item How many points are there in the sample space?
                 \item Calculate the probability that the card is an ace of spades.
                 \item Calculate the probability that the card is (i) an ace and (ii) black card.
 \end{enumerate}
\solution
		%\input{ncert/11/16/3/4/main.tex}
\item Four cards are drawn from a well-shuffled deck of 52 cards. What is the probability of obtaining 3 diamonds and one spade.
\\
\solution
		%\input{ncert/11/16/4/2/defs.tex}
\item In a certain lottery 10,000 tickets are sold and ten equal prizes are awarded. What is the probability of not getting a prize if you buy (a) one ticket (b) two tickets (c) 10 tickets ?	
\\
\solution
		%\input{ncert/11/16/4/4/defs.tex}
		%
\item 
Out of 100 students, two sections of 40 and 60 are formed. If you and your friend are among the 100 students, what is the probability that
\begin{enumerate}
\item you both enter the same section?
\item you both enter the different sections?
\end{enumerate}
\solution
		%\input{ncert/11/16/4/5/defs.tex}
	\item 
The number lock of a suitcase has 4 wheels each labelled with ten digits i.e. from 0 to 9.The lock opens with a sequence of four digits with no repeats.What is the probability of a person getting the right sequence to open the suitcase.
\\
\solution
		%\input{ncert/11/16/4/10/defs.tex}
		%
\item 
Two cards are drawn at random and without replacement from a pack of 52 playing cards. Find the probability that both the cards are black.
\\
\solution
		%\input{ncert/12/13/2/2/defs.tex}
		\item A box of oranges is inspected by examining three randomly selected oranges drawn without replacement. If all the three oranges are good, the box is approved for sale, otherwise, it is rejected. Find the probability that a box containing 15 oranges out of which 12 are good and 3 are bad ones will be approved for sale.
		\label{ncert/12/13/2/3/defs.tex}
		\item Two balls are drawn at random with replacement from a box containing 10 black and 8 red balls. Find the probability that
		\label{ncert/12/13/2/12}
\begin{enumerate}
\item both balls are red.
\item first ball is black and second is red.
\item one of them is black and other is red.
\end{enumerate}

\item In a hostel, 60\% of the students read Hindi newspaper, 40\% read English newspaper and 20\% read both Hindi and English newspapers. A student is selected at random.
		\label{ncert/12/13/2/15}
\begin{enumerate}
\item Find the probability that she reads neither Hindi nor English newspapers.
\item If she reads Hindi newspaper, find the probability that she reads English newspaper.
\item If she reads English newspaper, find the probability that she reads Hindi newspaper.\\
\end{enumerate}
\item The probability of obtaining an even prime number on each die, when a pair of dice is rolled is 
\begin{enumerate}
    \item $0$ 
    
    \item $\frac{1}{3}$ 
    
    \item $\frac{1}{12}$ 
    
    \item $\frac{1}{36}$ 
\end{enumerate}
\solution
		%\input{ncert/12/13/2/17/defs.tex}
	\item A bag contains 4 red and 4 black balls, another bag contains 2 red and 6 black balls. One of the two bags is selected at random and a ball is drawn from the bag which is found to be red. Find the probability that the ball is drawn from the first bag.
\\
\solution
		%\input{ncert/12/13/3/2/main.tex}
  \item
  Cards with numbers 2 to 101 are placed in a box. A card is selected at random.Find the probability that the card has
\begin{enumerate}[label=(\roman*)]
	\item an even number 
	\item a square number
\end{enumerate}
\solution
%\input{exemplar/10/13/3/32/main.tex}
\item
The king, queen and jack of clubs are removed from a deck of 52 playing cards and then well shuffled. Now one card is drawn at random from the remaining cards.  Determine the probability that the card is
\begin{enumerate}[label=(\roman*)]
\item a club
\item 10 of hearts
\end{enumerate}
\solution
%\input{exemplar/10/13/3/29/main.tex}
\item A team of medical students doing their internship have to assist during surgeries
at a city hospital. The probabilities of surgeries rated as very complex, complex,
routine, simple or very simple are respectively, 0.15, 0.20, 0.31, 0.26, .08. Find
the probabilities that a particular surgery will be rated
\begin{enumerate}
	\item complex or very complex;
	\item neither very complex nor very simple;
	\item routine or complex
	\item routine or simple
\end{enumerate}
\solution
%\input{exemplar/11/16/3/8(1)/main.tex}
\item A card is selected from a pack of 52 cards.
\begin{enumerate}[label=(\alph*)]
    \item How many points are there in the sample space?
    \item Calculate the probability that the card is an ace of spades.
    \item Calculate the probability that the card is (i) an ace and (ii) black card.
\end{enumerate}
\solution
%\input{exemplar/11/16/3/4/main2.tex}
\item The probability that a non leap year selected at random will contain 53 sundays.
\\
\solution
%\input{exemplar/10/13/1/19/main.tex}
\item One of the four persons John, Rita, Aslam or Gurpreet will be promoted next
month. Consequently the sample space consists of four elementary outcomes
S = {John promoted, Rita promoted, Aslam promoted, Gurpreet promoted}
You are told that the chances of John’s promotion is same as that of Gurpreet,
Rita’s chances of promotion are twice as likely as Johns. Aslam’s chances are
four times that of John.
\begin{enumerate}
	\item Determine
	\begin{enumerate}
		\item P (John promoted)
		\item P (Rita promoted)
		\item P (Aslam promoted)
		\item P (Gurpreet promoted)
	\end{enumerate}
	\item If A = {John promoted or Gurpreet promoted}, find P (A).
\end{enumerate}
\solution
%\input{exemplar/11/16/3/10/main.tex}
\item A card is drawn from a deck of 52 cards. Find the probability of getting a king or a heart or a red card.\\
\solution
%\input{exemplar/11/16/3/15/main.tex}
\item The probability that a student will pass his examination is 0.73, the probability of
the student getting a compartment is 0.13, and the probability that the student will
either pass or get compartment is 0.96. State True or False.\\
\solution
%\input{exemplar/11/16/3/31/main.tex}
\item A card is selected from a pack of 52 cards\\
\begin{enumerate}[label=(\alph*)]
\item How many points are there in the sample space?
\item Calculate the probability that the cards is an ace of spades.
\item Calculate the probability that the card is (i) an ace (ii)black card.\\
\end{enumerate}
%\input{ncert/11/16/3/4_1/Prob_4.tex}
\item In a non-leap year, the probability of having 53 tuesdays or 53 wednesdays is\\
\solution
%\input{exemplar/11/16/3/18/main.tex}
\item There are 1000 sealed envelopes in a box, 10 of them contain a cash prize of
Rs 100 each, 100 of them contain a cash prize of Rs 50 each and 200 of them
contain a cash prize of Rs 10 each and rest do not contain any cash prize. If they
are well shuffled and an envelope is picked up out, what is the probability that it
contains no cash prize?\\
\solution
%\input{exemplar/10/13/3/34/main.tex}
\item 
A die is thrown and a card is selected at random from a deck of 52 playing cards. The probability of getting an even number on the die and a spade card.\\
\solution
%\input{exemplar/12/13/3/78/main.tex}
\item
If 4-digit numbers greater than 5,000 are randomly formed from the digits 0, 1, 3, 5, and 7, what is the probability of forming a number divisible by 5 when:
\begin{enumerate}
    \item The digits are repeated?
    \item The repetition of digits is not allowed?
\end{enumerate}
\solution
%\input{ncert/11/16/4/9/main.tex}
\item Consider the probability space $\brak{\Omega, \mathcal{G}, P}$ where $\Omega = [0,2]$ and $\mathcal{G} = \cbrak{\phi, \Omega, [0,1], (1,2]}$. Let $X$ and $Y$ be two functions on $\Omega$ defined as
\begin{align*}
    X(\omega) = 
    \begin{cases}
        1 & \text{if }\omega \in [0, 1]\\
        2 & \text{if }\omega \in (1, 2]
    \end{cases}
\end{align*}
and
\begin{align*}
    Y(\omega) = 
    \begin{cases}
        2 & \text{if }\omega \in [0, 1.5]\\
        3 & \text{if }\omega \in (1.5, 2].
    \end{cases}
\end{align*}
Then which one of the following statements is true?
\begin{enumerate}
    \item [(A)] $X$ is a random variable with respect to $\mathcal{G}$, but $Y$ is not a random variable with respect to $\mathcal{G}$.
    \item [(B)] $Y$ is a random variable with respect to $\mathcal{G}$, but $X$ is not a random variable with respect to $\mathcal{G}$.
    \item [(C)] Neither $X$ nor $Y$ is a random variable with respect to $\mathcal{G}$.
    \item [(D)] Both $X$ and $Y$ are random variables with respect to $\mathcal{G}$.
\end{enumerate} \hfill (GATE ST 2023)\\
\solution
%\input{gate/ST/2023/14/main.tex}
	\item  A die is loaded in such a way that each odd number is twice as likely to occur as
each even number. Find $P(G)$, where $G$ is the event that a number greater than
3 occurs on a single roll of the die.
\\
\solution
		%\input{exemplar/11/16/3/5/main.tex}
	\item All the jacks, queens and kings are removed from a deck of 52 playing cards. The remaining cards are well shuffled and then one card is drawn at random. Giving ace a value 1 similar value for other cards, find the probability that the card has a value 
		\begin{enumerate}
			\item 7
			\item greater than 7
			\item less than 7
		\end{enumerate}
		%\input{exemplar/10/13/3/30/main.tex}
  \item A Lot consists of 48 mobile phones of which 42 are good, 3 have only minor defects and 3 have major defects.Varnika will buy a phone if it is good but the trader will only buy a mobile if it has no major defects. One phone is selected at random from the lot. What is the probability that it is
\begin{enumerate}
	\item acceptable to Varnika?
            \item acceptable to the trader?
\end{enumerate}
\solution
	%\input{exemplar/10/13/3/40/main.tex}
 \item A student says that if you throw a die, it will show up 1 or not 1. Therefore, the probability of getting 1 and the probability of getting 'not 1' each is equal to $\frac{1}{2}$. Is this correct? Give reasons.\\
 \solution
        %\input{exemplar/10/13/2/9/main.tex}
   \item Four candidates A, B, C, D have ap-
plied for the assignment to coach a school cricket
team. If A is twice as likely to be selected as B, and
B and C are given about the same chance of being
selected, while C is twice as likely to be selected
as D, what are the probabilities that
\begin{enumerate}
\item C will be selected?
\item A will not be selected?
\end{enumerate}
	%\input{exemplar/11/16/3/9/main.tex}
 \item A bag contain 24 balls of which $x$ balls are red, $2x$ are white and $3x$ are blue. A ball is selected at random, What is the probability that it is
\begin{enumerate}[label=\alph*)]
\item not red ?
\item white ?
\end{enumerate}
%\input{exemplar/10/13/3/41/main.tex}
If the letters of the word ASSASSINATION are arranged at random. Find the Probability that
\begin{enumerate}[label=(\alph*)]
\item Four $S's$ come consecutively in the word
\item Two  $I's$ and two $N's$ come together
\item All $A's$ are not coming together
\item No two $A's$ are coming together
\end{enumerate}
%\input{exemplar/11/16/3/14/main.tex}
	\item One urn contains two black balls (labelled B1 and B2) and one white ball. A
	second urn contains one black ball and two white balls (labelled W1 and W2).
	Suppose the following experiment is performed. One of the two urns is chosen
	at random. Next a ball is randomly chosen from the urn. Then a second ball is
	chosen at random from the same urn without replacing the first ball.
	
	\begin{enumerate}
	\item What is the probability that two black balls are chosen?
	
	\item What is the probability that two balls of opposite colour are chosen?
	\end{enumerate}
	\solution
	%\input{exemplar/11/16/3/12/main1.tex}
\end{enumerate}

		%
\item 
Two cards are drawn at random and without replacement from a pack of 52 playing cards. Find the probability that both the cards are black.
\\
\solution
		%\begin{enumerate}[label=\thesection.\arabic*,ref=\thesection.\theenumi]
	\item One card is drawn from a well-shuffled deck of 52 cards. Find the probability of getting
\begin{enumerate}
\item A king of red colour 
\item A face card 
\item A red face card
\item The jack of hearts
\item A spade
\item The queen of diamonds

\end{enumerate}
\solution
		%\input{ncert/10/15/1/14/main.tex}
	\item Five cards—the ten, jack, queen, king and ace of diamonds, are well-shuffled with their face downwards. One card is then picked up at random.
\begin{enumerate}
\item
What is the probability that the card is the queen? 
\item
If the queen is drawn and put aside, what is the probability that the second card picked up is (a) an ace? (b) a queen?\\
\end{enumerate}
\solution
		%\input{ncert/10/15/1/15/defs.tex}
	\item A bag contains $5$ red balls and some blue balls. If the probability of drawing a blue ball is double that if a red ball, determine the number of blue balls in the bag. 
		\\
\solution
		%\input{ncert/10/15/2/3/defs.tex}
	\item A card is selected from a pack of 52 cards.
 \begin{enumerate}[label=(\alph*)] 
                 \item How many points are there in the sample space?
                 \item Calculate the probability that the card is an ace of spades.
                 \item Calculate the probability that the card is (i) an ace and (ii) black card.
 \end{enumerate}
\solution
		%\input{ncert/11/16/3/4/main.tex}
\item Four cards are drawn from a well-shuffled deck of 52 cards. What is the probability of obtaining 3 diamonds and one spade.
\\
\solution
		%\input{ncert/11/16/4/2/defs.tex}
\item In a certain lottery 10,000 tickets are sold and ten equal prizes are awarded. What is the probability of not getting a prize if you buy (a) one ticket (b) two tickets (c) 10 tickets ?	
\\
\solution
		%\input{ncert/11/16/4/4/defs.tex}
		%
\item 
Out of 100 students, two sections of 40 and 60 are formed. If you and your friend are among the 100 students, what is the probability that
\begin{enumerate}
\item you both enter the same section?
\item you both enter the different sections?
\end{enumerate}
\solution
		%\input{ncert/11/16/4/5/defs.tex}
	\item 
The number lock of a suitcase has 4 wheels each labelled with ten digits i.e. from 0 to 9.The lock opens with a sequence of four digits with no repeats.What is the probability of a person getting the right sequence to open the suitcase.
\\
\solution
		%\input{ncert/11/16/4/10/defs.tex}
		%
\item 
Two cards are drawn at random and without replacement from a pack of 52 playing cards. Find the probability that both the cards are black.
\\
\solution
		%\input{ncert/12/13/2/2/defs.tex}
		\item A box of oranges is inspected by examining three randomly selected oranges drawn without replacement. If all the three oranges are good, the box is approved for sale, otherwise, it is rejected. Find the probability that a box containing 15 oranges out of which 12 are good and 3 are bad ones will be approved for sale.
		\label{ncert/12/13/2/3/defs.tex}
		\item Two balls are drawn at random with replacement from a box containing 10 black and 8 red balls. Find the probability that
		\label{ncert/12/13/2/12}
\begin{enumerate}
\item both balls are red.
\item first ball is black and second is red.
\item one of them is black and other is red.
\end{enumerate}

\item In a hostel, 60\% of the students read Hindi newspaper, 40\% read English newspaper and 20\% read both Hindi and English newspapers. A student is selected at random.
		\label{ncert/12/13/2/15}
\begin{enumerate}
\item Find the probability that she reads neither Hindi nor English newspapers.
\item If she reads Hindi newspaper, find the probability that she reads English newspaper.
\item If she reads English newspaper, find the probability that she reads Hindi newspaper.\\
\end{enumerate}
\item The probability of obtaining an even prime number on each die, when a pair of dice is rolled is 
\begin{enumerate}
    \item $0$ 
    
    \item $\frac{1}{3}$ 
    
    \item $\frac{1}{12}$ 
    
    \item $\frac{1}{36}$ 
\end{enumerate}
\solution
		%\input{ncert/12/13/2/17/defs.tex}
	\item A bag contains 4 red and 4 black balls, another bag contains 2 red and 6 black balls. One of the two bags is selected at random and a ball is drawn from the bag which is found to be red. Find the probability that the ball is drawn from the first bag.
\\
\solution
		%\input{ncert/12/13/3/2/main.tex}
  \item
  Cards with numbers 2 to 101 are placed in a box. A card is selected at random.Find the probability that the card has
\begin{enumerate}[label=(\roman*)]
	\item an even number 
	\item a square number
\end{enumerate}
\solution
%\input{exemplar/10/13/3/32/main.tex}
\item
The king, queen and jack of clubs are removed from a deck of 52 playing cards and then well shuffled. Now one card is drawn at random from the remaining cards.  Determine the probability that the card is
\begin{enumerate}[label=(\roman*)]
\item a club
\item 10 of hearts
\end{enumerate}
\solution
%\input{exemplar/10/13/3/29/main.tex}
\item A team of medical students doing their internship have to assist during surgeries
at a city hospital. The probabilities of surgeries rated as very complex, complex,
routine, simple or very simple are respectively, 0.15, 0.20, 0.31, 0.26, .08. Find
the probabilities that a particular surgery will be rated
\begin{enumerate}
	\item complex or very complex;
	\item neither very complex nor very simple;
	\item routine or complex
	\item routine or simple
\end{enumerate}
\solution
%\input{exemplar/11/16/3/8(1)/main.tex}
\item A card is selected from a pack of 52 cards.
\begin{enumerate}[label=(\alph*)]
    \item How many points are there in the sample space?
    \item Calculate the probability that the card is an ace of spades.
    \item Calculate the probability that the card is (i) an ace and (ii) black card.
\end{enumerate}
\solution
%\input{exemplar/11/16/3/4/main2.tex}
\item The probability that a non leap year selected at random will contain 53 sundays.
\\
\solution
%\input{exemplar/10/13/1/19/main.tex}
\item One of the four persons John, Rita, Aslam or Gurpreet will be promoted next
month. Consequently the sample space consists of four elementary outcomes
S = {John promoted, Rita promoted, Aslam promoted, Gurpreet promoted}
You are told that the chances of John’s promotion is same as that of Gurpreet,
Rita’s chances of promotion are twice as likely as Johns. Aslam’s chances are
four times that of John.
\begin{enumerate}
	\item Determine
	\begin{enumerate}
		\item P (John promoted)
		\item P (Rita promoted)
		\item P (Aslam promoted)
		\item P (Gurpreet promoted)
	\end{enumerate}
	\item If A = {John promoted or Gurpreet promoted}, find P (A).
\end{enumerate}
\solution
%\input{exemplar/11/16/3/10/main.tex}
\item A card is drawn from a deck of 52 cards. Find the probability of getting a king or a heart or a red card.\\
\solution
%\input{exemplar/11/16/3/15/main.tex}
\item The probability that a student will pass his examination is 0.73, the probability of
the student getting a compartment is 0.13, and the probability that the student will
either pass or get compartment is 0.96. State True or False.\\
\solution
%\input{exemplar/11/16/3/31/main.tex}
\item A card is selected from a pack of 52 cards\\
\begin{enumerate}[label=(\alph*)]
\item How many points are there in the sample space?
\item Calculate the probability that the cards is an ace of spades.
\item Calculate the probability that the card is (i) an ace (ii)black card.\\
\end{enumerate}
%\input{ncert/11/16/3/4_1/Prob_4.tex}
\item In a non-leap year, the probability of having 53 tuesdays or 53 wednesdays is\\
\solution
%\input{exemplar/11/16/3/18/main.tex}
\item There are 1000 sealed envelopes in a box, 10 of them contain a cash prize of
Rs 100 each, 100 of them contain a cash prize of Rs 50 each and 200 of them
contain a cash prize of Rs 10 each and rest do not contain any cash prize. If they
are well shuffled and an envelope is picked up out, what is the probability that it
contains no cash prize?\\
\solution
%\input{exemplar/10/13/3/34/main.tex}
\item 
A die is thrown and a card is selected at random from a deck of 52 playing cards. The probability of getting an even number on the die and a spade card.\\
\solution
%\input{exemplar/12/13/3/78/main.tex}
\item
If 4-digit numbers greater than 5,000 are randomly formed from the digits 0, 1, 3, 5, and 7, what is the probability of forming a number divisible by 5 when:
\begin{enumerate}
    \item The digits are repeated?
    \item The repetition of digits is not allowed?
\end{enumerate}
\solution
%\input{ncert/11/16/4/9/main.tex}
\item Consider the probability space $\brak{\Omega, \mathcal{G}, P}$ where $\Omega = [0,2]$ and $\mathcal{G} = \cbrak{\phi, \Omega, [0,1], (1,2]}$. Let $X$ and $Y$ be two functions on $\Omega$ defined as
\begin{align*}
    X(\omega) = 
    \begin{cases}
        1 & \text{if }\omega \in [0, 1]\\
        2 & \text{if }\omega \in (1, 2]
    \end{cases}
\end{align*}
and
\begin{align*}
    Y(\omega) = 
    \begin{cases}
        2 & \text{if }\omega \in [0, 1.5]\\
        3 & \text{if }\omega \in (1.5, 2].
    \end{cases}
\end{align*}
Then which one of the following statements is true?
\begin{enumerate}
    \item [(A)] $X$ is a random variable with respect to $\mathcal{G}$, but $Y$ is not a random variable with respect to $\mathcal{G}$.
    \item [(B)] $Y$ is a random variable with respect to $\mathcal{G}$, but $X$ is not a random variable with respect to $\mathcal{G}$.
    \item [(C)] Neither $X$ nor $Y$ is a random variable with respect to $\mathcal{G}$.
    \item [(D)] Both $X$ and $Y$ are random variables with respect to $\mathcal{G}$.
\end{enumerate} \hfill (GATE ST 2023)\\
\solution
%\input{gate/ST/2023/14/main.tex}
	\item  A die is loaded in such a way that each odd number is twice as likely to occur as
each even number. Find $P(G)$, where $G$ is the event that a number greater than
3 occurs on a single roll of the die.
\\
\solution
		%\input{exemplar/11/16/3/5/main.tex}
	\item All the jacks, queens and kings are removed from a deck of 52 playing cards. The remaining cards are well shuffled and then one card is drawn at random. Giving ace a value 1 similar value for other cards, find the probability that the card has a value 
		\begin{enumerate}
			\item 7
			\item greater than 7
			\item less than 7
		\end{enumerate}
		%\input{exemplar/10/13/3/30/main.tex}
  \item A Lot consists of 48 mobile phones of which 42 are good, 3 have only minor defects and 3 have major defects.Varnika will buy a phone if it is good but the trader will only buy a mobile if it has no major defects. One phone is selected at random from the lot. What is the probability that it is
\begin{enumerate}
	\item acceptable to Varnika?
            \item acceptable to the trader?
\end{enumerate}
\solution
	%\input{exemplar/10/13/3/40/main.tex}
 \item A student says that if you throw a die, it will show up 1 or not 1. Therefore, the probability of getting 1 and the probability of getting 'not 1' each is equal to $\frac{1}{2}$. Is this correct? Give reasons.\\
 \solution
        %\input{exemplar/10/13/2/9/main.tex}
   \item Four candidates A, B, C, D have ap-
plied for the assignment to coach a school cricket
team. If A is twice as likely to be selected as B, and
B and C are given about the same chance of being
selected, while C is twice as likely to be selected
as D, what are the probabilities that
\begin{enumerate}
\item C will be selected?
\item A will not be selected?
\end{enumerate}
	%\input{exemplar/11/16/3/9/main.tex}
 \item A bag contain 24 balls of which $x$ balls are red, $2x$ are white and $3x$ are blue. A ball is selected at random, What is the probability that it is
\begin{enumerate}[label=\alph*)]
\item not red ?
\item white ?
\end{enumerate}
%\input{exemplar/10/13/3/41/main.tex}
If the letters of the word ASSASSINATION are arranged at random. Find the Probability that
\begin{enumerate}[label=(\alph*)]
\item Four $S's$ come consecutively in the word
\item Two  $I's$ and two $N's$ come together
\item All $A's$ are not coming together
\item No two $A's$ are coming together
\end{enumerate}
%\input{exemplar/11/16/3/14/main.tex}
	\item One urn contains two black balls (labelled B1 and B2) and one white ball. A
	second urn contains one black ball and two white balls (labelled W1 and W2).
	Suppose the following experiment is performed. One of the two urns is chosen
	at random. Next a ball is randomly chosen from the urn. Then a second ball is
	chosen at random from the same urn without replacing the first ball.
	
	\begin{enumerate}
	\item What is the probability that two black balls are chosen?
	
	\item What is the probability that two balls of opposite colour are chosen?
	\end{enumerate}
	\solution
	%\input{exemplar/11/16/3/12/main1.tex}
\end{enumerate}

		\item A box of oranges is inspected by examining three randomly selected oranges drawn without replacement. If all the three oranges are good, the box is approved for sale, otherwise, it is rejected. Find the probability that a box containing 15 oranges out of which 12 are good and 3 are bad ones will be approved for sale.
		\label{ncert/12/13/2/3/defs.tex}
		\item Two balls are drawn at random with replacement from a box containing 10 black and 8 red balls. Find the probability that
		\label{ncert/12/13/2/12}
\begin{enumerate}
\item both balls are red.
\item first ball is black and second is red.
\item one of them is black and other is red.
\end{enumerate}

\item In a hostel, 60\% of the students read Hindi newspaper, 40\% read English newspaper and 20\% read both Hindi and English newspapers. A student is selected at random.
		\label{ncert/12/13/2/15}
\begin{enumerate}
\item Find the probability that she reads neither Hindi nor English newspapers.
\item If she reads Hindi newspaper, find the probability that she reads English newspaper.
\item If she reads English newspaper, find the probability that she reads Hindi newspaper.\\
\end{enumerate}
\item The probability of obtaining an even prime number on each die, when a pair of dice is rolled is 
\begin{enumerate}
    \item $0$ 
    
    \item $\frac{1}{3}$ 
    
    \item $\frac{1}{12}$ 
    
    \item $\frac{1}{36}$ 
\end{enumerate}
\solution
		%\begin{enumerate}[label=\thesection.\arabic*,ref=\thesection.\theenumi]
	\item One card is drawn from a well-shuffled deck of 52 cards. Find the probability of getting
\begin{enumerate}
\item A king of red colour 
\item A face card 
\item A red face card
\item The jack of hearts
\item A spade
\item The queen of diamonds

\end{enumerate}
\solution
		%\input{ncert/10/15/1/14/main.tex}
	\item Five cards—the ten, jack, queen, king and ace of diamonds, are well-shuffled with their face downwards. One card is then picked up at random.
\begin{enumerate}
\item
What is the probability that the card is the queen? 
\item
If the queen is drawn and put aside, what is the probability that the second card picked up is (a) an ace? (b) a queen?\\
\end{enumerate}
\solution
		%\input{ncert/10/15/1/15/defs.tex}
	\item A bag contains $5$ red balls and some blue balls. If the probability of drawing a blue ball is double that if a red ball, determine the number of blue balls in the bag. 
		\\
\solution
		%\input{ncert/10/15/2/3/defs.tex}
	\item A card is selected from a pack of 52 cards.
 \begin{enumerate}[label=(\alph*)] 
                 \item How many points are there in the sample space?
                 \item Calculate the probability that the card is an ace of spades.
                 \item Calculate the probability that the card is (i) an ace and (ii) black card.
 \end{enumerate}
\solution
		%\input{ncert/11/16/3/4/main.tex}
\item Four cards are drawn from a well-shuffled deck of 52 cards. What is the probability of obtaining 3 diamonds and one spade.
\\
\solution
		%\input{ncert/11/16/4/2/defs.tex}
\item In a certain lottery 10,000 tickets are sold and ten equal prizes are awarded. What is the probability of not getting a prize if you buy (a) one ticket (b) two tickets (c) 10 tickets ?	
\\
\solution
		%\input{ncert/11/16/4/4/defs.tex}
		%
\item 
Out of 100 students, two sections of 40 and 60 are formed. If you and your friend are among the 100 students, what is the probability that
\begin{enumerate}
\item you both enter the same section?
\item you both enter the different sections?
\end{enumerate}
\solution
		%\input{ncert/11/16/4/5/defs.tex}
	\item 
The number lock of a suitcase has 4 wheels each labelled with ten digits i.e. from 0 to 9.The lock opens with a sequence of four digits with no repeats.What is the probability of a person getting the right sequence to open the suitcase.
\\
\solution
		%\input{ncert/11/16/4/10/defs.tex}
		%
\item 
Two cards are drawn at random and without replacement from a pack of 52 playing cards. Find the probability that both the cards are black.
\\
\solution
		%\input{ncert/12/13/2/2/defs.tex}
		\item A box of oranges is inspected by examining three randomly selected oranges drawn without replacement. If all the three oranges are good, the box is approved for sale, otherwise, it is rejected. Find the probability that a box containing 15 oranges out of which 12 are good and 3 are bad ones will be approved for sale.
		\label{ncert/12/13/2/3/defs.tex}
		\item Two balls are drawn at random with replacement from a box containing 10 black and 8 red balls. Find the probability that
		\label{ncert/12/13/2/12}
\begin{enumerate}
\item both balls are red.
\item first ball is black and second is red.
\item one of them is black and other is red.
\end{enumerate}

\item In a hostel, 60\% of the students read Hindi newspaper, 40\% read English newspaper and 20\% read both Hindi and English newspapers. A student is selected at random.
		\label{ncert/12/13/2/15}
\begin{enumerate}
\item Find the probability that she reads neither Hindi nor English newspapers.
\item If she reads Hindi newspaper, find the probability that she reads English newspaper.
\item If she reads English newspaper, find the probability that she reads Hindi newspaper.\\
\end{enumerate}
\item The probability of obtaining an even prime number on each die, when a pair of dice is rolled is 
\begin{enumerate}
    \item $0$ 
    
    \item $\frac{1}{3}$ 
    
    \item $\frac{1}{12}$ 
    
    \item $\frac{1}{36}$ 
\end{enumerate}
\solution
		%\input{ncert/12/13/2/17/defs.tex}
	\item A bag contains 4 red and 4 black balls, another bag contains 2 red and 6 black balls. One of the two bags is selected at random and a ball is drawn from the bag which is found to be red. Find the probability that the ball is drawn from the first bag.
\\
\solution
		%\input{ncert/12/13/3/2/main.tex}
  \item
  Cards with numbers 2 to 101 are placed in a box. A card is selected at random.Find the probability that the card has
\begin{enumerate}[label=(\roman*)]
	\item an even number 
	\item a square number
\end{enumerate}
\solution
%\input{exemplar/10/13/3/32/main.tex}
\item
The king, queen and jack of clubs are removed from a deck of 52 playing cards and then well shuffled. Now one card is drawn at random from the remaining cards.  Determine the probability that the card is
\begin{enumerate}[label=(\roman*)]
\item a club
\item 10 of hearts
\end{enumerate}
\solution
%\input{exemplar/10/13/3/29/main.tex}
\item A team of medical students doing their internship have to assist during surgeries
at a city hospital. The probabilities of surgeries rated as very complex, complex,
routine, simple or very simple are respectively, 0.15, 0.20, 0.31, 0.26, .08. Find
the probabilities that a particular surgery will be rated
\begin{enumerate}
	\item complex or very complex;
	\item neither very complex nor very simple;
	\item routine or complex
	\item routine or simple
\end{enumerate}
\solution
%\input{exemplar/11/16/3/8(1)/main.tex}
\item A card is selected from a pack of 52 cards.
\begin{enumerate}[label=(\alph*)]
    \item How many points are there in the sample space?
    \item Calculate the probability that the card is an ace of spades.
    \item Calculate the probability that the card is (i) an ace and (ii) black card.
\end{enumerate}
\solution
%\input{exemplar/11/16/3/4/main2.tex}
\item The probability that a non leap year selected at random will contain 53 sundays.
\\
\solution
%\input{exemplar/10/13/1/19/main.tex}
\item One of the four persons John, Rita, Aslam or Gurpreet will be promoted next
month. Consequently the sample space consists of four elementary outcomes
S = {John promoted, Rita promoted, Aslam promoted, Gurpreet promoted}
You are told that the chances of John’s promotion is same as that of Gurpreet,
Rita’s chances of promotion are twice as likely as Johns. Aslam’s chances are
four times that of John.
\begin{enumerate}
	\item Determine
	\begin{enumerate}
		\item P (John promoted)
		\item P (Rita promoted)
		\item P (Aslam promoted)
		\item P (Gurpreet promoted)
	\end{enumerate}
	\item If A = {John promoted or Gurpreet promoted}, find P (A).
\end{enumerate}
\solution
%\input{exemplar/11/16/3/10/main.tex}
\item A card is drawn from a deck of 52 cards. Find the probability of getting a king or a heart or a red card.\\
\solution
%\input{exemplar/11/16/3/15/main.tex}
\item The probability that a student will pass his examination is 0.73, the probability of
the student getting a compartment is 0.13, and the probability that the student will
either pass or get compartment is 0.96. State True or False.\\
\solution
%\input{exemplar/11/16/3/31/main.tex}
\item A card is selected from a pack of 52 cards\\
\begin{enumerate}[label=(\alph*)]
\item How many points are there in the sample space?
\item Calculate the probability that the cards is an ace of spades.
\item Calculate the probability that the card is (i) an ace (ii)black card.\\
\end{enumerate}
%\input{ncert/11/16/3/4_1/Prob_4.tex}
\item In a non-leap year, the probability of having 53 tuesdays or 53 wednesdays is\\
\solution
%\input{exemplar/11/16/3/18/main.tex}
\item There are 1000 sealed envelopes in a box, 10 of them contain a cash prize of
Rs 100 each, 100 of them contain a cash prize of Rs 50 each and 200 of them
contain a cash prize of Rs 10 each and rest do not contain any cash prize. If they
are well shuffled and an envelope is picked up out, what is the probability that it
contains no cash prize?\\
\solution
%\input{exemplar/10/13/3/34/main.tex}
\item 
A die is thrown and a card is selected at random from a deck of 52 playing cards. The probability of getting an even number on the die and a spade card.\\
\solution
%\input{exemplar/12/13/3/78/main.tex}
\item
If 4-digit numbers greater than 5,000 are randomly formed from the digits 0, 1, 3, 5, and 7, what is the probability of forming a number divisible by 5 when:
\begin{enumerate}
    \item The digits are repeated?
    \item The repetition of digits is not allowed?
\end{enumerate}
\solution
%\input{ncert/11/16/4/9/main.tex}
\item Consider the probability space $\brak{\Omega, \mathcal{G}, P}$ where $\Omega = [0,2]$ and $\mathcal{G} = \cbrak{\phi, \Omega, [0,1], (1,2]}$. Let $X$ and $Y$ be two functions on $\Omega$ defined as
\begin{align*}
    X(\omega) = 
    \begin{cases}
        1 & \text{if }\omega \in [0, 1]\\
        2 & \text{if }\omega \in (1, 2]
    \end{cases}
\end{align*}
and
\begin{align*}
    Y(\omega) = 
    \begin{cases}
        2 & \text{if }\omega \in [0, 1.5]\\
        3 & \text{if }\omega \in (1.5, 2].
    \end{cases}
\end{align*}
Then which one of the following statements is true?
\begin{enumerate}
    \item [(A)] $X$ is a random variable with respect to $\mathcal{G}$, but $Y$ is not a random variable with respect to $\mathcal{G}$.
    \item [(B)] $Y$ is a random variable with respect to $\mathcal{G}$, but $X$ is not a random variable with respect to $\mathcal{G}$.
    \item [(C)] Neither $X$ nor $Y$ is a random variable with respect to $\mathcal{G}$.
    \item [(D)] Both $X$ and $Y$ are random variables with respect to $\mathcal{G}$.
\end{enumerate} \hfill (GATE ST 2023)\\
\solution
%\input{gate/ST/2023/14/main.tex}
	\item  A die is loaded in such a way that each odd number is twice as likely to occur as
each even number. Find $P(G)$, where $G$ is the event that a number greater than
3 occurs on a single roll of the die.
\\
\solution
		%\input{exemplar/11/16/3/5/main.tex}
	\item All the jacks, queens and kings are removed from a deck of 52 playing cards. The remaining cards are well shuffled and then one card is drawn at random. Giving ace a value 1 similar value for other cards, find the probability that the card has a value 
		\begin{enumerate}
			\item 7
			\item greater than 7
			\item less than 7
		\end{enumerate}
		%\input{exemplar/10/13/3/30/main.tex}
  \item A Lot consists of 48 mobile phones of which 42 are good, 3 have only minor defects and 3 have major defects.Varnika will buy a phone if it is good but the trader will only buy a mobile if it has no major defects. One phone is selected at random from the lot. What is the probability that it is
\begin{enumerate}
	\item acceptable to Varnika?
            \item acceptable to the trader?
\end{enumerate}
\solution
	%\input{exemplar/10/13/3/40/main.tex}
 \item A student says that if you throw a die, it will show up 1 or not 1. Therefore, the probability of getting 1 and the probability of getting 'not 1' each is equal to $\frac{1}{2}$. Is this correct? Give reasons.\\
 \solution
        %\input{exemplar/10/13/2/9/main.tex}
   \item Four candidates A, B, C, D have ap-
plied for the assignment to coach a school cricket
team. If A is twice as likely to be selected as B, and
B and C are given about the same chance of being
selected, while C is twice as likely to be selected
as D, what are the probabilities that
\begin{enumerate}
\item C will be selected?
\item A will not be selected?
\end{enumerate}
	%\input{exemplar/11/16/3/9/main.tex}
 \item A bag contain 24 balls of which $x$ balls are red, $2x$ are white and $3x$ are blue. A ball is selected at random, What is the probability that it is
\begin{enumerate}[label=\alph*)]
\item not red ?
\item white ?
\end{enumerate}
%\input{exemplar/10/13/3/41/main.tex}
If the letters of the word ASSASSINATION are arranged at random. Find the Probability that
\begin{enumerate}[label=(\alph*)]
\item Four $S's$ come consecutively in the word
\item Two  $I's$ and two $N's$ come together
\item All $A's$ are not coming together
\item No two $A's$ are coming together
\end{enumerate}
%\input{exemplar/11/16/3/14/main.tex}
	\item One urn contains two black balls (labelled B1 and B2) and one white ball. A
	second urn contains one black ball and two white balls (labelled W1 and W2).
	Suppose the following experiment is performed. One of the two urns is chosen
	at random. Next a ball is randomly chosen from the urn. Then a second ball is
	chosen at random from the same urn without replacing the first ball.
	
	\begin{enumerate}
	\item What is the probability that two black balls are chosen?
	
	\item What is the probability that two balls of opposite colour are chosen?
	\end{enumerate}
	\solution
	%\input{exemplar/11/16/3/12/main1.tex}
\end{enumerate}

	\item A bag contains 4 red and 4 black balls, another bag contains 2 red and 6 black balls. One of the two bags is selected at random and a ball is drawn from the bag which is found to be red. Find the probability that the ball is drawn from the first bag.
\\
\solution
		%\begin{table}[H]
	\centering
\begin{tabular}{|c|c|c|}
\hline
Random variable &Value &Definition\\ \hline
\multirow{3}{*}{X} &0 &Slips of Rs 1\\
&1 &Slips of Rs 5\\
&2 &Slips of Rs 13\\ \hline
\multirow{2}{*}{Y} &0 &Box A\\
&1 &Box B\\\hline
\end{tabular}
\caption{}
\label{tab:Distribution}
\end{table}
See \tabref{tab:Distribution}.
\begin{align}
p_{Y}\brak{k}= \begin{cases} 
      \frac{1}{3} & {k=0} \\
      \frac{2}{3 }& {k=1} 
   \end{cases}
   \\
p_{Y|X}\brak{0|0} = \frac{19}{25}\, 
p_{Y|X}\brak{0|1} = \frac{6}{25}\,
p_{Y|X}\brak{1|0} = \frac{45}{50}\,
p_{Y|X}\brak{1|2} = \frac{5}{50}
\end{align}
The desired probability is the probability that a slip drawn at random is marked other than Rs 1,
\begin{align}
&=1-p_X\brak{0}\\
&= p_X(1) + p_X(2)
\end{align}
Using Bayes theorem,
\begin{align}
&= p_Y\brak{0} \times \pr{Y=0 | X=1} + p_Y\brak{1} \times \pr{Y=1|X=2}\\
&=\frac{1}{3} \times \frac{6}{25} + \frac{2}{3} \times \frac{5}{50}\\
&=\frac{11}{75}
\end{align}

\newpage

%\tableofcontents

\bigskip

\renewcommand{\thefigure}{\theenumi}
\renewcommand{\thetable}{\theenumi}
%\renewcommand{\theequation}{\theenumi}

%\begin{abstract}
%%\boldmath
%In this letter, an algorithm for evaluating the exact analytical bit error rate  (BER)  for the piecewise linear (PL) combiner for  multiple relays is presented. Previous results were available only for upto three relays. The algorithm is unique in the sense that  the actual mathematical expressions, that are prohibitively large, need not be explicitly obtained. The diversity gain due to multiple relays is shown through plots of the analytical BER, well supported by simulations. 
%
%\end{abstract}
% IEEEtran.cls defaults to using nonbold math in the Abstract.
% This preserves the distinction between vectors and scalars. However,
% if the journal you are submitting to favors bold math in the abstract,
% then you can use LaTeX's standard command \boldmath at the very start
% of the abstract to achieve this. Many IEEE journals frown on math
% in the abstract anyway.

% Note that keywords are not normally used for peerreview papers.
%\begin{IEEEkeywords}
%Cooperative diversity, decode and forward, piecewise linear
%\end{IEEEkeywords}



% For peer review papers, you can put extra information on the cover
% page as needed:
% \ifCLASSOPTIONpeerreview
% \begin{center} \bfseries EDICS Category: 3-BBND \end{center}
% \fi
%
% For peerreview papers, this IEEEtran command inserts a page break and
% creates the second title. It will be ignored for other modes.
%\IEEEpeerreviewmaketitle




  \item
  Cards with numbers 2 to 101 are placed in a box. A card is selected at random.Find the probability that the card has
\begin{enumerate}[label=(\roman*)]
	\item an even number 
	\item a square number
\end{enumerate}
\solution
%\begin{table}[H]
	\centering
\begin{tabular}{|c|c|c|}
\hline
Random variable &Value &Definition\\ \hline
\multirow{3}{*}{X} &0 &Slips of Rs 1\\
&1 &Slips of Rs 5\\
&2 &Slips of Rs 13\\ \hline
\multirow{2}{*}{Y} &0 &Box A\\
&1 &Box B\\\hline
\end{tabular}
\caption{}
\label{tab:Distribution}
\end{table}
See \tabref{tab:Distribution}.
\begin{align}
p_{Y}\brak{k}= \begin{cases} 
      \frac{1}{3} & {k=0} \\
      \frac{2}{3 }& {k=1} 
   \end{cases}
   \\
p_{Y|X}\brak{0|0} = \frac{19}{25}\, 
p_{Y|X}\brak{0|1} = \frac{6}{25}\,
p_{Y|X}\brak{1|0} = \frac{45}{50}\,
p_{Y|X}\brak{1|2} = \frac{5}{50}
\end{align}
The desired probability is the probability that a slip drawn at random is marked other than Rs 1,
\begin{align}
&=1-p_X\brak{0}\\
&= p_X(1) + p_X(2)
\end{align}
Using Bayes theorem,
\begin{align}
&= p_Y\brak{0} \times \pr{Y=0 | X=1} + p_Y\brak{1} \times \pr{Y=1|X=2}\\
&=\frac{1}{3} \times \frac{6}{25} + \frac{2}{3} \times \frac{5}{50}\\
&=\frac{11}{75}
\end{align}

\newpage

%\tableofcontents

\bigskip

\renewcommand{\thefigure}{\theenumi}
\renewcommand{\thetable}{\theenumi}
%\renewcommand{\theequation}{\theenumi}

%\begin{abstract}
%%\boldmath
%In this letter, an algorithm for evaluating the exact analytical bit error rate  (BER)  for the piecewise linear (PL) combiner for  multiple relays is presented. Previous results were available only for upto three relays. The algorithm is unique in the sense that  the actual mathematical expressions, that are prohibitively large, need not be explicitly obtained. The diversity gain due to multiple relays is shown through plots of the analytical BER, well supported by simulations. 
%
%\end{abstract}
% IEEEtran.cls defaults to using nonbold math in the Abstract.
% This preserves the distinction between vectors and scalars. However,
% if the journal you are submitting to favors bold math in the abstract,
% then you can use LaTeX's standard command \boldmath at the very start
% of the abstract to achieve this. Many IEEE journals frown on math
% in the abstract anyway.

% Note that keywords are not normally used for peerreview papers.
%\begin{IEEEkeywords}
%Cooperative diversity, decode and forward, piecewise linear
%\end{IEEEkeywords}



% For peer review papers, you can put extra information on the cover
% page as needed:
% \ifCLASSOPTIONpeerreview
% \begin{center} \bfseries EDICS Category: 3-BBND \end{center}
% \fi
%
% For peerreview papers, this IEEEtran command inserts a page break and
% creates the second title. It will be ignored for other modes.
%\IEEEpeerreviewmaketitle




\item
The king, queen and jack of clubs are removed from a deck of 52 playing cards and then well shuffled. Now one card is drawn at random from the remaining cards.  Determine the probability that the card is
\begin{enumerate}[label=(\roman*)]
\item a club
\item 10 of hearts
\end{enumerate}
\solution
%\begin{table}[H]
	\centering
\begin{tabular}{|c|c|c|}
\hline
Random variable &Value &Definition\\ \hline
\multirow{3}{*}{X} &0 &Slips of Rs 1\\
&1 &Slips of Rs 5\\
&2 &Slips of Rs 13\\ \hline
\multirow{2}{*}{Y} &0 &Box A\\
&1 &Box B\\\hline
\end{tabular}
\caption{}
\label{tab:Distribution}
\end{table}
See \tabref{tab:Distribution}.
\begin{align}
p_{Y}\brak{k}= \begin{cases} 
      \frac{1}{3} & {k=0} \\
      \frac{2}{3 }& {k=1} 
   \end{cases}
   \\
p_{Y|X}\brak{0|0} = \frac{19}{25}\, 
p_{Y|X}\brak{0|1} = \frac{6}{25}\,
p_{Y|X}\brak{1|0} = \frac{45}{50}\,
p_{Y|X}\brak{1|2} = \frac{5}{50}
\end{align}
The desired probability is the probability that a slip drawn at random is marked other than Rs 1,
\begin{align}
&=1-p_X\brak{0}\\
&= p_X(1) + p_X(2)
\end{align}
Using Bayes theorem,
\begin{align}
&= p_Y\brak{0} \times \pr{Y=0 | X=1} + p_Y\brak{1} \times \pr{Y=1|X=2}\\
&=\frac{1}{3} \times \frac{6}{25} + \frac{2}{3} \times \frac{5}{50}\\
&=\frac{11}{75}
\end{align}

\newpage

%\tableofcontents

\bigskip

\renewcommand{\thefigure}{\theenumi}
\renewcommand{\thetable}{\theenumi}
%\renewcommand{\theequation}{\theenumi}

%\begin{abstract}
%%\boldmath
%In this letter, an algorithm for evaluating the exact analytical bit error rate  (BER)  for the piecewise linear (PL) combiner for  multiple relays is presented. Previous results were available only for upto three relays. The algorithm is unique in the sense that  the actual mathematical expressions, that are prohibitively large, need not be explicitly obtained. The diversity gain due to multiple relays is shown through plots of the analytical BER, well supported by simulations. 
%
%\end{abstract}
% IEEEtran.cls defaults to using nonbold math in the Abstract.
% This preserves the distinction between vectors and scalars. However,
% if the journal you are submitting to favors bold math in the abstract,
% then you can use LaTeX's standard command \boldmath at the very start
% of the abstract to achieve this. Many IEEE journals frown on math
% in the abstract anyway.

% Note that keywords are not normally used for peerreview papers.
%\begin{IEEEkeywords}
%Cooperative diversity, decode and forward, piecewise linear
%\end{IEEEkeywords}



% For peer review papers, you can put extra information on the cover
% page as needed:
% \ifCLASSOPTIONpeerreview
% \begin{center} \bfseries EDICS Category: 3-BBND \end{center}
% \fi
%
% For peerreview papers, this IEEEtran command inserts a page break and
% creates the second title. It will be ignored for other modes.
%\IEEEpeerreviewmaketitle




\item A team of medical students doing their internship have to assist during surgeries
at a city hospital. The probabilities of surgeries rated as very complex, complex,
routine, simple or very simple are respectively, 0.15, 0.20, 0.31, 0.26, .08. Find
the probabilities that a particular surgery will be rated
\begin{enumerate}
	\item complex or very complex;
	\item neither very complex nor very simple;
	\item routine or complex
	\item routine or simple
\end{enumerate}
\solution
%\begin{table}[H]
	\centering
\begin{tabular}{|c|c|c|}
\hline
Random variable &Value &Definition\\ \hline
\multirow{3}{*}{X} &0 &Slips of Rs 1\\
&1 &Slips of Rs 5\\
&2 &Slips of Rs 13\\ \hline
\multirow{2}{*}{Y} &0 &Box A\\
&1 &Box B\\\hline
\end{tabular}
\caption{}
\label{tab:Distribution}
\end{table}
See \tabref{tab:Distribution}.
\begin{align}
p_{Y}\brak{k}= \begin{cases} 
      \frac{1}{3} & {k=0} \\
      \frac{2}{3 }& {k=1} 
   \end{cases}
   \\
p_{Y|X}\brak{0|0} = \frac{19}{25}\, 
p_{Y|X}\brak{0|1} = \frac{6}{25}\,
p_{Y|X}\brak{1|0} = \frac{45}{50}\,
p_{Y|X}\brak{1|2} = \frac{5}{50}
\end{align}
The desired probability is the probability that a slip drawn at random is marked other than Rs 1,
\begin{align}
&=1-p_X\brak{0}\\
&= p_X(1) + p_X(2)
\end{align}
Using Bayes theorem,
\begin{align}
&= p_Y\brak{0} \times \pr{Y=0 | X=1} + p_Y\brak{1} \times \pr{Y=1|X=2}\\
&=\frac{1}{3} \times \frac{6}{25} + \frac{2}{3} \times \frac{5}{50}\\
&=\frac{11}{75}
\end{align}

\newpage

%\tableofcontents

\bigskip

\renewcommand{\thefigure}{\theenumi}
\renewcommand{\thetable}{\theenumi}
%\renewcommand{\theequation}{\theenumi}

%\begin{abstract}
%%\boldmath
%In this letter, an algorithm for evaluating the exact analytical bit error rate  (BER)  for the piecewise linear (PL) combiner for  multiple relays is presented. Previous results were available only for upto three relays. The algorithm is unique in the sense that  the actual mathematical expressions, that are prohibitively large, need not be explicitly obtained. The diversity gain due to multiple relays is shown through plots of the analytical BER, well supported by simulations. 
%
%\end{abstract}
% IEEEtran.cls defaults to using nonbold math in the Abstract.
% This preserves the distinction between vectors and scalars. However,
% if the journal you are submitting to favors bold math in the abstract,
% then you can use LaTeX's standard command \boldmath at the very start
% of the abstract to achieve this. Many IEEE journals frown on math
% in the abstract anyway.

% Note that keywords are not normally used for peerreview papers.
%\begin{IEEEkeywords}
%Cooperative diversity, decode and forward, piecewise linear
%\end{IEEEkeywords}



% For peer review papers, you can put extra information on the cover
% page as needed:
% \ifCLASSOPTIONpeerreview
% \begin{center} \bfseries EDICS Category: 3-BBND \end{center}
% \fi
%
% For peerreview papers, this IEEEtran command inserts a page break and
% creates the second title. It will be ignored for other modes.
%\IEEEpeerreviewmaketitle




\item A card is selected from a pack of 52 cards.
\begin{enumerate}[label=(\alph*)]
    \item How many points are there in the sample space?
    \item Calculate the probability that the card is an ace of spades.
    \item Calculate the probability that the card is (i) an ace and (ii) black card.
\end{enumerate}
\solution
%Let $X$ be an bernoulli rv defined as in \tabref{tab:exemplar/11/16/3/26}.  Then, 
\begin{equation}
    p =
        \frac{4}{11} 
\end{equation}
\begin{table}[H]
	\centering
	\input{exemplar/11/16/3/26/tables/Table2.tex}
	\caption{}
        \label{tab:exemplar/11/16/3/26}
\end{table}

\item The probability that a non leap year selected at random will contain 53 sundays.
\\
\solution
%\begin{table}[H]
	\centering
\begin{tabular}{|c|c|c|}
\hline
Random variable &Value &Definition\\ \hline
\multirow{3}{*}{X} &0 &Slips of Rs 1\\
&1 &Slips of Rs 5\\
&2 &Slips of Rs 13\\ \hline
\multirow{2}{*}{Y} &0 &Box A\\
&1 &Box B\\\hline
\end{tabular}
\caption{}
\label{tab:Distribution}
\end{table}
See \tabref{tab:Distribution}.
\begin{align}
p_{Y}\brak{k}= \begin{cases} 
      \frac{1}{3} & {k=0} \\
      \frac{2}{3 }& {k=1} 
   \end{cases}
   \\
p_{Y|X}\brak{0|0} = \frac{19}{25}\, 
p_{Y|X}\brak{0|1} = \frac{6}{25}\,
p_{Y|X}\brak{1|0} = \frac{45}{50}\,
p_{Y|X}\brak{1|2} = \frac{5}{50}
\end{align}
The desired probability is the probability that a slip drawn at random is marked other than Rs 1,
\begin{align}
&=1-p_X\brak{0}\\
&= p_X(1) + p_X(2)
\end{align}
Using Bayes theorem,
\begin{align}
&= p_Y\brak{0} \times \pr{Y=0 | X=1} + p_Y\brak{1} \times \pr{Y=1|X=2}\\
&=\frac{1}{3} \times \frac{6}{25} + \frac{2}{3} \times \frac{5}{50}\\
&=\frac{11}{75}
\end{align}

\newpage

%\tableofcontents

\bigskip

\renewcommand{\thefigure}{\theenumi}
\renewcommand{\thetable}{\theenumi}
%\renewcommand{\theequation}{\theenumi}

%\begin{abstract}
%%\boldmath
%In this letter, an algorithm for evaluating the exact analytical bit error rate  (BER)  for the piecewise linear (PL) combiner for  multiple relays is presented. Previous results were available only for upto three relays. The algorithm is unique in the sense that  the actual mathematical expressions, that are prohibitively large, need not be explicitly obtained. The diversity gain due to multiple relays is shown through plots of the analytical BER, well supported by simulations. 
%
%\end{abstract}
% IEEEtran.cls defaults to using nonbold math in the Abstract.
% This preserves the distinction between vectors and scalars. However,
% if the journal you are submitting to favors bold math in the abstract,
% then you can use LaTeX's standard command \boldmath at the very start
% of the abstract to achieve this. Many IEEE journals frown on math
% in the abstract anyway.

% Note that keywords are not normally used for peerreview papers.
%\begin{IEEEkeywords}
%Cooperative diversity, decode and forward, piecewise linear
%\end{IEEEkeywords}



% For peer review papers, you can put extra information on the cover
% page as needed:
% \ifCLASSOPTIONpeerreview
% \begin{center} \bfseries EDICS Category: 3-BBND \end{center}
% \fi
%
% For peerreview papers, this IEEEtran command inserts a page break and
% creates the second title. It will be ignored for other modes.
%\IEEEpeerreviewmaketitle




\item One of the four persons John, Rita, Aslam or Gurpreet will be promoted next
month. Consequently the sample space consists of four elementary outcomes
S = {John promoted, Rita promoted, Aslam promoted, Gurpreet promoted}
You are told that the chances of John’s promotion is same as that of Gurpreet,
Rita’s chances of promotion are twice as likely as Johns. Aslam’s chances are
four times that of John.
\begin{enumerate}
	\item Determine
	\begin{enumerate}
		\item P (John promoted)
		\item P (Rita promoted)
		\item P (Aslam promoted)
		\item P (Gurpreet promoted)
	\end{enumerate}
	\item If A = {John promoted or Gurpreet promoted}, find P (A).
\end{enumerate}
\solution
%\begin{table}[H]
	\centering
\begin{tabular}{|c|c|c|}
\hline
Random variable &Value &Definition\\ \hline
\multirow{3}{*}{X} &0 &Slips of Rs 1\\
&1 &Slips of Rs 5\\
&2 &Slips of Rs 13\\ \hline
\multirow{2}{*}{Y} &0 &Box A\\
&1 &Box B\\\hline
\end{tabular}
\caption{}
\label{tab:Distribution}
\end{table}
See \tabref{tab:Distribution}.
\begin{align}
p_{Y}\brak{k}= \begin{cases} 
      \frac{1}{3} & {k=0} \\
      \frac{2}{3 }& {k=1} 
   \end{cases}
   \\
p_{Y|X}\brak{0|0} = \frac{19}{25}\, 
p_{Y|X}\brak{0|1} = \frac{6}{25}\,
p_{Y|X}\brak{1|0} = \frac{45}{50}\,
p_{Y|X}\brak{1|2} = \frac{5}{50}
\end{align}
The desired probability is the probability that a slip drawn at random is marked other than Rs 1,
\begin{align}
&=1-p_X\brak{0}\\
&= p_X(1) + p_X(2)
\end{align}
Using Bayes theorem,
\begin{align}
&= p_Y\brak{0} \times \pr{Y=0 | X=1} + p_Y\brak{1} \times \pr{Y=1|X=2}\\
&=\frac{1}{3} \times \frac{6}{25} + \frac{2}{3} \times \frac{5}{50}\\
&=\frac{11}{75}
\end{align}

\newpage

%\tableofcontents

\bigskip

\renewcommand{\thefigure}{\theenumi}
\renewcommand{\thetable}{\theenumi}
%\renewcommand{\theequation}{\theenumi}

%\begin{abstract}
%%\boldmath
%In this letter, an algorithm for evaluating the exact analytical bit error rate  (BER)  for the piecewise linear (PL) combiner for  multiple relays is presented. Previous results were available only for upto three relays. The algorithm is unique in the sense that  the actual mathematical expressions, that are prohibitively large, need not be explicitly obtained. The diversity gain due to multiple relays is shown through plots of the analytical BER, well supported by simulations. 
%
%\end{abstract}
% IEEEtran.cls defaults to using nonbold math in the Abstract.
% This preserves the distinction between vectors and scalars. However,
% if the journal you are submitting to favors bold math in the abstract,
% then you can use LaTeX's standard command \boldmath at the very start
% of the abstract to achieve this. Many IEEE journals frown on math
% in the abstract anyway.

% Note that keywords are not normally used for peerreview papers.
%\begin{IEEEkeywords}
%Cooperative diversity, decode and forward, piecewise linear
%\end{IEEEkeywords}



% For peer review papers, you can put extra information on the cover
% page as needed:
% \ifCLASSOPTIONpeerreview
% \begin{center} \bfseries EDICS Category: 3-BBND \end{center}
% \fi
%
% For peerreview papers, this IEEEtran command inserts a page break and
% creates the second title. It will be ignored for other modes.
%\IEEEpeerreviewmaketitle




\item A card is drawn from a deck of 52 cards. Find the probability of getting a king or a heart or a red card.\\
\solution
%\begin{table}[H]
	\centering
\begin{tabular}{|c|c|c|}
\hline
Random variable &Value &Definition\\ \hline
\multirow{3}{*}{X} &0 &Slips of Rs 1\\
&1 &Slips of Rs 5\\
&2 &Slips of Rs 13\\ \hline
\multirow{2}{*}{Y} &0 &Box A\\
&1 &Box B\\\hline
\end{tabular}
\caption{}
\label{tab:Distribution}
\end{table}
See \tabref{tab:Distribution}.
\begin{align}
p_{Y}\brak{k}= \begin{cases} 
      \frac{1}{3} & {k=0} \\
      \frac{2}{3 }& {k=1} 
   \end{cases}
   \\
p_{Y|X}\brak{0|0} = \frac{19}{25}\, 
p_{Y|X}\brak{0|1} = \frac{6}{25}\,
p_{Y|X}\brak{1|0} = \frac{45}{50}\,
p_{Y|X}\brak{1|2} = \frac{5}{50}
\end{align}
The desired probability is the probability that a slip drawn at random is marked other than Rs 1,
\begin{align}
&=1-p_X\brak{0}\\
&= p_X(1) + p_X(2)
\end{align}
Using Bayes theorem,
\begin{align}
&= p_Y\brak{0} \times \pr{Y=0 | X=1} + p_Y\brak{1} \times \pr{Y=1|X=2}\\
&=\frac{1}{3} \times \frac{6}{25} + \frac{2}{3} \times \frac{5}{50}\\
&=\frac{11}{75}
\end{align}

\newpage

%\tableofcontents

\bigskip

\renewcommand{\thefigure}{\theenumi}
\renewcommand{\thetable}{\theenumi}
%\renewcommand{\theequation}{\theenumi}

%\begin{abstract}
%%\boldmath
%In this letter, an algorithm for evaluating the exact analytical bit error rate  (BER)  for the piecewise linear (PL) combiner for  multiple relays is presented. Previous results were available only for upto three relays. The algorithm is unique in the sense that  the actual mathematical expressions, that are prohibitively large, need not be explicitly obtained. The diversity gain due to multiple relays is shown through plots of the analytical BER, well supported by simulations. 
%
%\end{abstract}
% IEEEtran.cls defaults to using nonbold math in the Abstract.
% This preserves the distinction between vectors and scalars. However,
% if the journal you are submitting to favors bold math in the abstract,
% then you can use LaTeX's standard command \boldmath at the very start
% of the abstract to achieve this. Many IEEE journals frown on math
% in the abstract anyway.

% Note that keywords are not normally used for peerreview papers.
%\begin{IEEEkeywords}
%Cooperative diversity, decode and forward, piecewise linear
%\end{IEEEkeywords}



% For peer review papers, you can put extra information on the cover
% page as needed:
% \ifCLASSOPTIONpeerreview
% \begin{center} \bfseries EDICS Category: 3-BBND \end{center}
% \fi
%
% For peerreview papers, this IEEEtran command inserts a page break and
% creates the second title. It will be ignored for other modes.
%\IEEEpeerreviewmaketitle




\item The probability that a student will pass his examination is 0.73, the probability of
the student getting a compartment is 0.13, and the probability that the student will
either pass or get compartment is 0.96. State True or False.\\
\solution
%\begin{table}[H]
	\centering
\begin{tabular}{|c|c|c|}
\hline
Random variable &Value &Definition\\ \hline
\multirow{3}{*}{X} &0 &Slips of Rs 1\\
&1 &Slips of Rs 5\\
&2 &Slips of Rs 13\\ \hline
\multirow{2}{*}{Y} &0 &Box A\\
&1 &Box B\\\hline
\end{tabular}
\caption{}
\label{tab:Distribution}
\end{table}
See \tabref{tab:Distribution}.
\begin{align}
p_{Y}\brak{k}= \begin{cases} 
      \frac{1}{3} & {k=0} \\
      \frac{2}{3 }& {k=1} 
   \end{cases}
   \\
p_{Y|X}\brak{0|0} = \frac{19}{25}\, 
p_{Y|X}\brak{0|1} = \frac{6}{25}\,
p_{Y|X}\brak{1|0} = \frac{45}{50}\,
p_{Y|X}\brak{1|2} = \frac{5}{50}
\end{align}
The desired probability is the probability that a slip drawn at random is marked other than Rs 1,
\begin{align}
&=1-p_X\brak{0}\\
&= p_X(1) + p_X(2)
\end{align}
Using Bayes theorem,
\begin{align}
&= p_Y\brak{0} \times \pr{Y=0 | X=1} + p_Y\brak{1} \times \pr{Y=1|X=2}\\
&=\frac{1}{3} \times \frac{6}{25} + \frac{2}{3} \times \frac{5}{50}\\
&=\frac{11}{75}
\end{align}

\newpage

%\tableofcontents

\bigskip

\renewcommand{\thefigure}{\theenumi}
\renewcommand{\thetable}{\theenumi}
%\renewcommand{\theequation}{\theenumi}

%\begin{abstract}
%%\boldmath
%In this letter, an algorithm for evaluating the exact analytical bit error rate  (BER)  for the piecewise linear (PL) combiner for  multiple relays is presented. Previous results were available only for upto three relays. The algorithm is unique in the sense that  the actual mathematical expressions, that are prohibitively large, need not be explicitly obtained. The diversity gain due to multiple relays is shown through plots of the analytical BER, well supported by simulations. 
%
%\end{abstract}
% IEEEtran.cls defaults to using nonbold math in the Abstract.
% This preserves the distinction between vectors and scalars. However,
% if the journal you are submitting to favors bold math in the abstract,
% then you can use LaTeX's standard command \boldmath at the very start
% of the abstract to achieve this. Many IEEE journals frown on math
% in the abstract anyway.

% Note that keywords are not normally used for peerreview papers.
%\begin{IEEEkeywords}
%Cooperative diversity, decode and forward, piecewise linear
%\end{IEEEkeywords}



% For peer review papers, you can put extra information on the cover
% page as needed:
% \ifCLASSOPTIONpeerreview
% \begin{center} \bfseries EDICS Category: 3-BBND \end{center}
% \fi
%
% For peerreview papers, this IEEEtran command inserts a page break and
% creates the second title. It will be ignored for other modes.
%\IEEEpeerreviewmaketitle




\item A card is selected from a pack of 52 cards\\
\begin{enumerate}[label=(\alph*)]
\item How many points are there in the sample space?
\item Calculate the probability that the cards is an ace of spades.
\item Calculate the probability that the card is (i) an ace (ii)black card.\\
\end{enumerate}
%\input{ncert/11/16/3/4_1/Prob_4.tex}
\item In a non-leap year, the probability of having 53 tuesdays or 53 wednesdays is\\
\solution
%A non-leap year has a total of 365 days, and a week has 7 days.\\
So it can be expressed as 
\begin{align}
365\text{days} &=52\times 7+1 \text{day}
\end{align}
$\implies$ 52 tuesdays or wednesdays\\
Random variable X denotes the days of a week
\begin{align}
p_X\brak{k}&=\frac{1}{7}; \quad \brak{1<k<7}
\end{align}
So the probability of extra day being tuesday or wednesday is
\begin{align}
p_X\brak{3}+p_X\brak{4}&=\frac{1}{7}+\frac{1}{7}=\frac{2}{7}
\end{align}



\item There are 1000 sealed envelopes in a box, 10 of them contain a cash prize of
Rs 100 each, 100 of them contain a cash prize of Rs 50 each and 200 of them
contain a cash prize of Rs 10 each and rest do not contain any cash prize. If they
are well shuffled and an envelope is picked up out, what is the probability that it
contains no cash prize?\\
\solution
%\begin{table}[H]
	\centering
\begin{tabular}{|c|c|c|}
\hline
Random variable &Value &Definition\\ \hline
\multirow{3}{*}{X} &0 &Slips of Rs 1\\
&1 &Slips of Rs 5\\
&2 &Slips of Rs 13\\ \hline
\multirow{2}{*}{Y} &0 &Box A\\
&1 &Box B\\\hline
\end{tabular}
\caption{}
\label{tab:Distribution}
\end{table}
See \tabref{tab:Distribution}.
\begin{align}
p_{Y}\brak{k}= \begin{cases} 
      \frac{1}{3} & {k=0} \\
      \frac{2}{3 }& {k=1} 
   \end{cases}
   \\
p_{Y|X}\brak{0|0} = \frac{19}{25}\, 
p_{Y|X}\brak{0|1} = \frac{6}{25}\,
p_{Y|X}\brak{1|0} = \frac{45}{50}\,
p_{Y|X}\brak{1|2} = \frac{5}{50}
\end{align}
The desired probability is the probability that a slip drawn at random is marked other than Rs 1,
\begin{align}
&=1-p_X\brak{0}\\
&= p_X(1) + p_X(2)
\end{align}
Using Bayes theorem,
\begin{align}
&= p_Y\brak{0} \times \pr{Y=0 | X=1} + p_Y\brak{1} \times \pr{Y=1|X=2}\\
&=\frac{1}{3} \times \frac{6}{25} + \frac{2}{3} \times \frac{5}{50}\\
&=\frac{11}{75}
\end{align}

\newpage

%\tableofcontents

\bigskip

\renewcommand{\thefigure}{\theenumi}
\renewcommand{\thetable}{\theenumi}
%\renewcommand{\theequation}{\theenumi}

%\begin{abstract}
%%\boldmath
%In this letter, an algorithm for evaluating the exact analytical bit error rate  (BER)  for the piecewise linear (PL) combiner for  multiple relays is presented. Previous results were available only for upto three relays. The algorithm is unique in the sense that  the actual mathematical expressions, that are prohibitively large, need not be explicitly obtained. The diversity gain due to multiple relays is shown through plots of the analytical BER, well supported by simulations. 
%
%\end{abstract}
% IEEEtran.cls defaults to using nonbold math in the Abstract.
% This preserves the distinction between vectors and scalars. However,
% if the journal you are submitting to favors bold math in the abstract,
% then you can use LaTeX's standard command \boldmath at the very start
% of the abstract to achieve this. Many IEEE journals frown on math
% in the abstract anyway.

% Note that keywords are not normally used for peerreview papers.
%\begin{IEEEkeywords}
%Cooperative diversity, decode and forward, piecewise linear
%\end{IEEEkeywords}



% For peer review papers, you can put extra information on the cover
% page as needed:
% \ifCLASSOPTIONpeerreview
% \begin{center} \bfseries EDICS Category: 3-BBND \end{center}
% \fi
%
% For peerreview papers, this IEEEtran command inserts a page break and
% creates the second title. It will be ignored for other modes.
%\IEEEpeerreviewmaketitle




\item 
A die is thrown and a card is selected at random from a deck of 52 playing cards. The probability of getting an even number on the die and a spade card.\\
\solution
%\begin{table}[H]
	\centering
\begin{tabular}{|c|c|c|}
\hline
Random variable &Value &Definition\\ \hline
\multirow{3}{*}{X} &0 &Slips of Rs 1\\
&1 &Slips of Rs 5\\
&2 &Slips of Rs 13\\ \hline
\multirow{2}{*}{Y} &0 &Box A\\
&1 &Box B\\\hline
\end{tabular}
\caption{}
\label{tab:Distribution}
\end{table}
See \tabref{tab:Distribution}.
\begin{align}
p_{Y}\brak{k}= \begin{cases} 
      \frac{1}{3} & {k=0} \\
      \frac{2}{3 }& {k=1} 
   \end{cases}
   \\
p_{Y|X}\brak{0|0} = \frac{19}{25}\, 
p_{Y|X}\brak{0|1} = \frac{6}{25}\,
p_{Y|X}\brak{1|0} = \frac{45}{50}\,
p_{Y|X}\brak{1|2} = \frac{5}{50}
\end{align}
The desired probability is the probability that a slip drawn at random is marked other than Rs 1,
\begin{align}
&=1-p_X\brak{0}\\
&= p_X(1) + p_X(2)
\end{align}
Using Bayes theorem,
\begin{align}
&= p_Y\brak{0} \times \pr{Y=0 | X=1} + p_Y\brak{1} \times \pr{Y=1|X=2}\\
&=\frac{1}{3} \times \frac{6}{25} + \frac{2}{3} \times \frac{5}{50}\\
&=\frac{11}{75}
\end{align}

\newpage

%\tableofcontents

\bigskip

\renewcommand{\thefigure}{\theenumi}
\renewcommand{\thetable}{\theenumi}
%\renewcommand{\theequation}{\theenumi}

%\begin{abstract}
%%\boldmath
%In this letter, an algorithm for evaluating the exact analytical bit error rate  (BER)  for the piecewise linear (PL) combiner for  multiple relays is presented. Previous results were available only for upto three relays. The algorithm is unique in the sense that  the actual mathematical expressions, that are prohibitively large, need not be explicitly obtained. The diversity gain due to multiple relays is shown through plots of the analytical BER, well supported by simulations. 
%
%\end{abstract}
% IEEEtran.cls defaults to using nonbold math in the Abstract.
% This preserves the distinction between vectors and scalars. However,
% if the journal you are submitting to favors bold math in the abstract,
% then you can use LaTeX's standard command \boldmath at the very start
% of the abstract to achieve this. Many IEEE journals frown on math
% in the abstract anyway.

% Note that keywords are not normally used for peerreview papers.
%\begin{IEEEkeywords}
%Cooperative diversity, decode and forward, piecewise linear
%\end{IEEEkeywords}



% For peer review papers, you can put extra information on the cover
% page as needed:
% \ifCLASSOPTIONpeerreview
% \begin{center} \bfseries EDICS Category: 3-BBND \end{center}
% \fi
%
% For peerreview papers, this IEEEtran command inserts a page break and
% creates the second title. It will be ignored for other modes.
%\IEEEpeerreviewmaketitle




\item
If 4-digit numbers greater than 5,000 are randomly formed from the digits 0, 1, 3, 5, and 7, what is the probability of forming a number divisible by 5 when:
\begin{enumerate}
    \item The digits are repeated?
    \item The repetition of digits is not allowed?
\end{enumerate}
\solution
%\begin{table}[H]
	\centering
\begin{tabular}{|c|c|c|}
\hline
Random variable &Value &Definition\\ \hline
\multirow{3}{*}{X} &0 &Slips of Rs 1\\
&1 &Slips of Rs 5\\
&2 &Slips of Rs 13\\ \hline
\multirow{2}{*}{Y} &0 &Box A\\
&1 &Box B\\\hline
\end{tabular}
\caption{}
\label{tab:Distribution}
\end{table}
See \tabref{tab:Distribution}.
\begin{align}
p_{Y}\brak{k}= \begin{cases} 
      \frac{1}{3} & {k=0} \\
      \frac{2}{3 }& {k=1} 
   \end{cases}
   \\
p_{Y|X}\brak{0|0} = \frac{19}{25}\, 
p_{Y|X}\brak{0|1} = \frac{6}{25}\,
p_{Y|X}\brak{1|0} = \frac{45}{50}\,
p_{Y|X}\brak{1|2} = \frac{5}{50}
\end{align}
The desired probability is the probability that a slip drawn at random is marked other than Rs 1,
\begin{align}
&=1-p_X\brak{0}\\
&= p_X(1) + p_X(2)
\end{align}
Using Bayes theorem,
\begin{align}
&= p_Y\brak{0} \times \pr{Y=0 | X=1} + p_Y\brak{1} \times \pr{Y=1|X=2}\\
&=\frac{1}{3} \times \frac{6}{25} + \frac{2}{3} \times \frac{5}{50}\\
&=\frac{11}{75}
\end{align}

\newpage

%\tableofcontents

\bigskip

\renewcommand{\thefigure}{\theenumi}
\renewcommand{\thetable}{\theenumi}
%\renewcommand{\theequation}{\theenumi}

%\begin{abstract}
%%\boldmath
%In this letter, an algorithm for evaluating the exact analytical bit error rate  (BER)  for the piecewise linear (PL) combiner for  multiple relays is presented. Previous results were available only for upto three relays. The algorithm is unique in the sense that  the actual mathematical expressions, that are prohibitively large, need not be explicitly obtained. The diversity gain due to multiple relays is shown through plots of the analytical BER, well supported by simulations. 
%
%\end{abstract}
% IEEEtran.cls defaults to using nonbold math in the Abstract.
% This preserves the distinction between vectors and scalars. However,
% if the journal you are submitting to favors bold math in the abstract,
% then you can use LaTeX's standard command \boldmath at the very start
% of the abstract to achieve this. Many IEEE journals frown on math
% in the abstract anyway.

% Note that keywords are not normally used for peerreview papers.
%\begin{IEEEkeywords}
%Cooperative diversity, decode and forward, piecewise linear
%\end{IEEEkeywords}



% For peer review papers, you can put extra information on the cover
% page as needed:
% \ifCLASSOPTIONpeerreview
% \begin{center} \bfseries EDICS Category: 3-BBND \end{center}
% \fi
%
% For peerreview papers, this IEEEtran command inserts a page break and
% creates the second title. It will be ignored for other modes.
%\IEEEpeerreviewmaketitle




\item Consider the probability space $\brak{\Omega, \mathcal{G}, P}$ where $\Omega = [0,2]$ and $\mathcal{G} = \cbrak{\phi, \Omega, [0,1], (1,2]}$. Let $X$ and $Y$ be two functions on $\Omega$ defined as
\begin{align*}
    X(\omega) = 
    \begin{cases}
        1 & \text{if }\omega \in [0, 1]\\
        2 & \text{if }\omega \in (1, 2]
    \end{cases}
\end{align*}
and
\begin{align*}
    Y(\omega) = 
    \begin{cases}
        2 & \text{if }\omega \in [0, 1.5]\\
        3 & \text{if }\omega \in (1.5, 2].
    \end{cases}
\end{align*}
Then which one of the following statements is true?
\begin{enumerate}
    \item [(A)] $X$ is a random variable with respect to $\mathcal{G}$, but $Y$ is not a random variable with respect to $\mathcal{G}$.
    \item [(B)] $Y$ is a random variable with respect to $\mathcal{G}$, but $X$ is not a random variable with respect to $\mathcal{G}$.
    \item [(C)] Neither $X$ nor $Y$ is a random variable with respect to $\mathcal{G}$.
    \item [(D)] Both $X$ and $Y$ are random variables with respect to $\mathcal{G}$.
\end{enumerate} \hfill (GATE ST 2023)\\
\solution
%\begin{table}[H]
	\centering
\begin{tabular}{|c|c|c|}
\hline
Random variable &Value &Definition\\ \hline
\multirow{3}{*}{X} &0 &Slips of Rs 1\\
&1 &Slips of Rs 5\\
&2 &Slips of Rs 13\\ \hline
\multirow{2}{*}{Y} &0 &Box A\\
&1 &Box B\\\hline
\end{tabular}
\caption{}
\label{tab:Distribution}
\end{table}
See \tabref{tab:Distribution}.
\begin{align}
p_{Y}\brak{k}= \begin{cases} 
      \frac{1}{3} & {k=0} \\
      \frac{2}{3 }& {k=1} 
   \end{cases}
   \\
p_{Y|X}\brak{0|0} = \frac{19}{25}\, 
p_{Y|X}\brak{0|1} = \frac{6}{25}\,
p_{Y|X}\brak{1|0} = \frac{45}{50}\,
p_{Y|X}\brak{1|2} = \frac{5}{50}
\end{align}
The desired probability is the probability that a slip drawn at random is marked other than Rs 1,
\begin{align}
&=1-p_X\brak{0}\\
&= p_X(1) + p_X(2)
\end{align}
Using Bayes theorem,
\begin{align}
&= p_Y\brak{0} \times \pr{Y=0 | X=1} + p_Y\brak{1} \times \pr{Y=1|X=2}\\
&=\frac{1}{3} \times \frac{6}{25} + \frac{2}{3} \times \frac{5}{50}\\
&=\frac{11}{75}
\end{align}

\newpage

%\tableofcontents

\bigskip

\renewcommand{\thefigure}{\theenumi}
\renewcommand{\thetable}{\theenumi}
%\renewcommand{\theequation}{\theenumi}

%\begin{abstract}
%%\boldmath
%In this letter, an algorithm for evaluating the exact analytical bit error rate  (BER)  for the piecewise linear (PL) combiner for  multiple relays is presented. Previous results were available only for upto three relays. The algorithm is unique in the sense that  the actual mathematical expressions, that are prohibitively large, need not be explicitly obtained. The diversity gain due to multiple relays is shown through plots of the analytical BER, well supported by simulations. 
%
%\end{abstract}
% IEEEtran.cls defaults to using nonbold math in the Abstract.
% This preserves the distinction between vectors and scalars. However,
% if the journal you are submitting to favors bold math in the abstract,
% then you can use LaTeX's standard command \boldmath at the very start
% of the abstract to achieve this. Many IEEE journals frown on math
% in the abstract anyway.

% Note that keywords are not normally used for peerreview papers.
%\begin{IEEEkeywords}
%Cooperative diversity, decode and forward, piecewise linear
%\end{IEEEkeywords}



% For peer review papers, you can put extra information on the cover
% page as needed:
% \ifCLASSOPTIONpeerreview
% \begin{center} \bfseries EDICS Category: 3-BBND \end{center}
% \fi
%
% For peerreview papers, this IEEEtran command inserts a page break and
% creates the second title. It will be ignored for other modes.
%\IEEEpeerreviewmaketitle




	\item  A die is loaded in such a way that each odd number is twice as likely to occur as
each even number. Find $P(G)$, where $G$ is the event that a number greater than
3 occurs on a single roll of the die.
\\
\solution
		%\begin{table}[H]
	\centering
\begin{tabular}{|c|c|c|}
\hline
Random variable &Value &Definition\\ \hline
\multirow{3}{*}{X} &0 &Slips of Rs 1\\
&1 &Slips of Rs 5\\
&2 &Slips of Rs 13\\ \hline
\multirow{2}{*}{Y} &0 &Box A\\
&1 &Box B\\\hline
\end{tabular}
\caption{}
\label{tab:Distribution}
\end{table}
See \tabref{tab:Distribution}.
\begin{align}
p_{Y}\brak{k}= \begin{cases} 
      \frac{1}{3} & {k=0} \\
      \frac{2}{3 }& {k=1} 
   \end{cases}
   \\
p_{Y|X}\brak{0|0} = \frac{19}{25}\, 
p_{Y|X}\brak{0|1} = \frac{6}{25}\,
p_{Y|X}\brak{1|0} = \frac{45}{50}\,
p_{Y|X}\brak{1|2} = \frac{5}{50}
\end{align}
The desired probability is the probability that a slip drawn at random is marked other than Rs 1,
\begin{align}
&=1-p_X\brak{0}\\
&= p_X(1) + p_X(2)
\end{align}
Using Bayes theorem,
\begin{align}
&= p_Y\brak{0} \times \pr{Y=0 | X=1} + p_Y\brak{1} \times \pr{Y=1|X=2}\\
&=\frac{1}{3} \times \frac{6}{25} + \frac{2}{3} \times \frac{5}{50}\\
&=\frac{11}{75}
\end{align}

\newpage

%\tableofcontents

\bigskip

\renewcommand{\thefigure}{\theenumi}
\renewcommand{\thetable}{\theenumi}
%\renewcommand{\theequation}{\theenumi}

%\begin{abstract}
%%\boldmath
%In this letter, an algorithm for evaluating the exact analytical bit error rate  (BER)  for the piecewise linear (PL) combiner for  multiple relays is presented. Previous results were available only for upto three relays. The algorithm is unique in the sense that  the actual mathematical expressions, that are prohibitively large, need not be explicitly obtained. The diversity gain due to multiple relays is shown through plots of the analytical BER, well supported by simulations. 
%
%\end{abstract}
% IEEEtran.cls defaults to using nonbold math in the Abstract.
% This preserves the distinction between vectors and scalars. However,
% if the journal you are submitting to favors bold math in the abstract,
% then you can use LaTeX's standard command \boldmath at the very start
% of the abstract to achieve this. Many IEEE journals frown on math
% in the abstract anyway.

% Note that keywords are not normally used for peerreview papers.
%\begin{IEEEkeywords}
%Cooperative diversity, decode and forward, piecewise linear
%\end{IEEEkeywords}



% For peer review papers, you can put extra information on the cover
% page as needed:
% \ifCLASSOPTIONpeerreview
% \begin{center} \bfseries EDICS Category: 3-BBND \end{center}
% \fi
%
% For peerreview papers, this IEEEtran command inserts a page break and
% creates the second title. It will be ignored for other modes.
%\IEEEpeerreviewmaketitle




	\item All the jacks, queens and kings are removed from a deck of 52 playing cards. The remaining cards are well shuffled and then one card is drawn at random. Giving ace a value 1 similar value for other cards, find the probability that the card has a value 
		\begin{enumerate}
			\item 7
			\item greater than 7
			\item less than 7
		\end{enumerate}
		%Number of cards left after removing all jacks, queens and kings 
\begin{align}
N	= 52 - 4\times 3
	= 40
\end{align}
%\begin{table}[H]
%\def\arraystretch{1.2}
%\begin{tabular}{|c|c|c|}
%\hline
%	\textbf{Parameter} &\textbf{Value} &\textbf{Description}\\ \hline
%	$X$ &1-10 &Represents the value of the card picked \\ \hline
%\end{tabular}
%\end{table}
Let $1 \le X \le 10$ be the value of the card picked.  Then,
\begin{align}
	p_X(k) &= \Pr(X=k)\ \forall\ 1 \leq k \leq 10\\
	&= \frac{4\times 1}{40}\\
	&= \frac{1}{10}\\
	\therefore p_X(k) &= 
	\begin{cases}
		\frac{1}{10} & 1 \leq k \leq 10\\
		0 & \text{otherwise}
	\end{cases}
\end{align}
and
\begin{align}
	F_{X}(k) &= \sum_{m=0}^{k}p_{X}(m) \quad 1 \leq k \leq 10\\
	&= \frac{k}{10}\\
	\therefore F_{X}(k) &= 
	\begin{cases}
		0 & k \leq 0\\
		\frac{k}{10} & 1\leq k \leq 10\\
		1 & k > 10 
	\end{cases}
\end{align}
\begin{enumerate}
	\item Probability that card has value equal to 7 is
		\begin{align}
			 p_{X}(7)
			= \frac{1}{10}
		\end{align}
	\item Probability that card has value greater than 7 is
		\begin{align}
			1 - F_X(7)
			&= 1 - \frac{7}{10}
			\\
			&= \frac{3}{10}
		\end{align}
	\item Probability that card has value less than 7 is
		\begin{align}
			 F_{X}(6)
			=\frac{6}{10}
		\end{align}
\end{enumerate}

  \item A Lot consists of 48 mobile phones of which 42 are good, 3 have only minor defects and 3 have major defects.Varnika will buy a phone if it is good but the trader will only buy a mobile if it has no major defects. One phone is selected at random from the lot. What is the probability that it is
\begin{enumerate}
	\item acceptable to Varnika?
            \item acceptable to the trader?
\end{enumerate}
\solution
	%\begin{table}[H]
	\centering
\begin{tabular}{|c|c|c|}
\hline
Random variable &Value &Definition\\ \hline
\multirow{3}{*}{X} &0 &Slips of Rs 1\\
&1 &Slips of Rs 5\\
&2 &Slips of Rs 13\\ \hline
\multirow{2}{*}{Y} &0 &Box A\\
&1 &Box B\\\hline
\end{tabular}
\caption{}
\label{tab:Distribution}
\end{table}
See \tabref{tab:Distribution}.
\begin{align}
p_{Y}\brak{k}= \begin{cases} 
      \frac{1}{3} & {k=0} \\
      \frac{2}{3 }& {k=1} 
   \end{cases}
   \\
p_{Y|X}\brak{0|0} = \frac{19}{25}\, 
p_{Y|X}\brak{0|1} = \frac{6}{25}\,
p_{Y|X}\brak{1|0} = \frac{45}{50}\,
p_{Y|X}\brak{1|2} = \frac{5}{50}
\end{align}
The desired probability is the probability that a slip drawn at random is marked other than Rs 1,
\begin{align}
&=1-p_X\brak{0}\\
&= p_X(1) + p_X(2)
\end{align}
Using Bayes theorem,
\begin{align}
&= p_Y\brak{0} \times \pr{Y=0 | X=1} + p_Y\brak{1} \times \pr{Y=1|X=2}\\
&=\frac{1}{3} \times \frac{6}{25} + \frac{2}{3} \times \frac{5}{50}\\
&=\frac{11}{75}
\end{align}

\newpage

%\tableofcontents

\bigskip

\renewcommand{\thefigure}{\theenumi}
\renewcommand{\thetable}{\theenumi}
%\renewcommand{\theequation}{\theenumi}

%\begin{abstract}
%%\boldmath
%In this letter, an algorithm for evaluating the exact analytical bit error rate  (BER)  for the piecewise linear (PL) combiner for  multiple relays is presented. Previous results were available only for upto three relays. The algorithm is unique in the sense that  the actual mathematical expressions, that are prohibitively large, need not be explicitly obtained. The diversity gain due to multiple relays is shown through plots of the analytical BER, well supported by simulations. 
%
%\end{abstract}
% IEEEtran.cls defaults to using nonbold math in the Abstract.
% This preserves the distinction between vectors and scalars. However,
% if the journal you are submitting to favors bold math in the abstract,
% then you can use LaTeX's standard command \boldmath at the very start
% of the abstract to achieve this. Many IEEE journals frown on math
% in the abstract anyway.

% Note that keywords are not normally used for peerreview papers.
%\begin{IEEEkeywords}
%Cooperative diversity, decode and forward, piecewise linear
%\end{IEEEkeywords}



% For peer review papers, you can put extra information on the cover
% page as needed:
% \ifCLASSOPTIONpeerreview
% \begin{center} \bfseries EDICS Category: 3-BBND \end{center}
% \fi
%
% For peerreview papers, this IEEEtran command inserts a page break and
% creates the second title. It will be ignored for other modes.
%\IEEEpeerreviewmaketitle




 \item A student says that if you throw a die, it will show up 1 or not 1. Therefore, the probability of getting 1 and the probability of getting 'not 1' each is equal to $\frac{1}{2}$. Is this correct? Give reasons.\\
 \solution
        %\begin{table}[H]
	\centering
\begin{tabular}{|c|c|c|}
\hline
Random variable &Value &Definition\\ \hline
\multirow{3}{*}{X} &0 &Slips of Rs 1\\
&1 &Slips of Rs 5\\
&2 &Slips of Rs 13\\ \hline
\multirow{2}{*}{Y} &0 &Box A\\
&1 &Box B\\\hline
\end{tabular}
\caption{}
\label{tab:Distribution}
\end{table}
See \tabref{tab:Distribution}.
\begin{align}
p_{Y}\brak{k}= \begin{cases} 
      \frac{1}{3} & {k=0} \\
      \frac{2}{3 }& {k=1} 
   \end{cases}
   \\
p_{Y|X}\brak{0|0} = \frac{19}{25}\, 
p_{Y|X}\brak{0|1} = \frac{6}{25}\,
p_{Y|X}\brak{1|0} = \frac{45}{50}\,
p_{Y|X}\brak{1|2} = \frac{5}{50}
\end{align}
The desired probability is the probability that a slip drawn at random is marked other than Rs 1,
\begin{align}
&=1-p_X\brak{0}\\
&= p_X(1) + p_X(2)
\end{align}
Using Bayes theorem,
\begin{align}
&= p_Y\brak{0} \times \pr{Y=0 | X=1} + p_Y\brak{1} \times \pr{Y=1|X=2}\\
&=\frac{1}{3} \times \frac{6}{25} + \frac{2}{3} \times \frac{5}{50}\\
&=\frac{11}{75}
\end{align}

\newpage

%\tableofcontents

\bigskip

\renewcommand{\thefigure}{\theenumi}
\renewcommand{\thetable}{\theenumi}
%\renewcommand{\theequation}{\theenumi}

%\begin{abstract}
%%\boldmath
%In this letter, an algorithm for evaluating the exact analytical bit error rate  (BER)  for the piecewise linear (PL) combiner for  multiple relays is presented. Previous results were available only for upto three relays. The algorithm is unique in the sense that  the actual mathematical expressions, that are prohibitively large, need not be explicitly obtained. The diversity gain due to multiple relays is shown through plots of the analytical BER, well supported by simulations. 
%
%\end{abstract}
% IEEEtran.cls defaults to using nonbold math in the Abstract.
% This preserves the distinction between vectors and scalars. However,
% if the journal you are submitting to favors bold math in the abstract,
% then you can use LaTeX's standard command \boldmath at the very start
% of the abstract to achieve this. Many IEEE journals frown on math
% in the abstract anyway.

% Note that keywords are not normally used for peerreview papers.
%\begin{IEEEkeywords}
%Cooperative diversity, decode and forward, piecewise linear
%\end{IEEEkeywords}



% For peer review papers, you can put extra information on the cover
% page as needed:
% \ifCLASSOPTIONpeerreview
% \begin{center} \bfseries EDICS Category: 3-BBND \end{center}
% \fi
%
% For peerreview papers, this IEEEtran command inserts a page break and
% creates the second title. It will be ignored for other modes.
%\IEEEpeerreviewmaketitle




   \item Four candidates A, B, C, D have ap-
plied for the assignment to coach a school cricket
team. If A is twice as likely to be selected as B, and
B and C are given about the same chance of being
selected, while C is twice as likely to be selected
as D, what are the probabilities that
\begin{enumerate}
\item C will be selected?
\item A will not be selected?
\end{enumerate}
	%\begin{table}[H]
	\centering
\begin{tabular}{|c|c|c|}
\hline
Random variable &Value &Definition\\ \hline
\multirow{3}{*}{X} &0 &Slips of Rs 1\\
&1 &Slips of Rs 5\\
&2 &Slips of Rs 13\\ \hline
\multirow{2}{*}{Y} &0 &Box A\\
&1 &Box B\\\hline
\end{tabular}
\caption{}
\label{tab:Distribution}
\end{table}
See \tabref{tab:Distribution}.
\begin{align}
p_{Y}\brak{k}= \begin{cases} 
      \frac{1}{3} & {k=0} \\
      \frac{2}{3 }& {k=1} 
   \end{cases}
   \\
p_{Y|X}\brak{0|0} = \frac{19}{25}\, 
p_{Y|X}\brak{0|1} = \frac{6}{25}\,
p_{Y|X}\brak{1|0} = \frac{45}{50}\,
p_{Y|X}\brak{1|2} = \frac{5}{50}
\end{align}
The desired probability is the probability that a slip drawn at random is marked other than Rs 1,
\begin{align}
&=1-p_X\brak{0}\\
&= p_X(1) + p_X(2)
\end{align}
Using Bayes theorem,
\begin{align}
&= p_Y\brak{0} \times \pr{Y=0 | X=1} + p_Y\brak{1} \times \pr{Y=1|X=2}\\
&=\frac{1}{3} \times \frac{6}{25} + \frac{2}{3} \times \frac{5}{50}\\
&=\frac{11}{75}
\end{align}

\newpage

%\tableofcontents

\bigskip

\renewcommand{\thefigure}{\theenumi}
\renewcommand{\thetable}{\theenumi}
%\renewcommand{\theequation}{\theenumi}

%\begin{abstract}
%%\boldmath
%In this letter, an algorithm for evaluating the exact analytical bit error rate  (BER)  for the piecewise linear (PL) combiner for  multiple relays is presented. Previous results were available only for upto three relays. The algorithm is unique in the sense that  the actual mathematical expressions, that are prohibitively large, need not be explicitly obtained. The diversity gain due to multiple relays is shown through plots of the analytical BER, well supported by simulations. 
%
%\end{abstract}
% IEEEtran.cls defaults to using nonbold math in the Abstract.
% This preserves the distinction between vectors and scalars. However,
% if the journal you are submitting to favors bold math in the abstract,
% then you can use LaTeX's standard command \boldmath at the very start
% of the abstract to achieve this. Many IEEE journals frown on math
% in the abstract anyway.

% Note that keywords are not normally used for peerreview papers.
%\begin{IEEEkeywords}
%Cooperative diversity, decode and forward, piecewise linear
%\end{IEEEkeywords}



% For peer review papers, you can put extra information on the cover
% page as needed:
% \ifCLASSOPTIONpeerreview
% \begin{center} \bfseries EDICS Category: 3-BBND \end{center}
% \fi
%
% For peerreview papers, this IEEEtran command inserts a page break and
% creates the second title. It will be ignored for other modes.
%\IEEEpeerreviewmaketitle




 \item A bag contain 24 balls of which $x$ balls are red, $2x$ are white and $3x$ are blue. A ball is selected at random, What is the probability that it is
\begin{enumerate}[label=\alph*)]
\item not red ?
\item white ?
\end{enumerate}
%\begin{table}[H]
	\centering
\begin{tabular}{|c|c|c|}
\hline
Random variable &Value &Definition\\ \hline
\multirow{3}{*}{X} &0 &Slips of Rs 1\\
&1 &Slips of Rs 5\\
&2 &Slips of Rs 13\\ \hline
\multirow{2}{*}{Y} &0 &Box A\\
&1 &Box B\\\hline
\end{tabular}
\caption{}
\label{tab:Distribution}
\end{table}
See \tabref{tab:Distribution}.
\begin{align}
p_{Y}\brak{k}= \begin{cases} 
      \frac{1}{3} & {k=0} \\
      \frac{2}{3 }& {k=1} 
   \end{cases}
   \\
p_{Y|X}\brak{0|0} = \frac{19}{25}\, 
p_{Y|X}\brak{0|1} = \frac{6}{25}\,
p_{Y|X}\brak{1|0} = \frac{45}{50}\,
p_{Y|X}\brak{1|2} = \frac{5}{50}
\end{align}
The desired probability is the probability that a slip drawn at random is marked other than Rs 1,
\begin{align}
&=1-p_X\brak{0}\\
&= p_X(1) + p_X(2)
\end{align}
Using Bayes theorem,
\begin{align}
&= p_Y\brak{0} \times \pr{Y=0 | X=1} + p_Y\brak{1} \times \pr{Y=1|X=2}\\
&=\frac{1}{3} \times \frac{6}{25} + \frac{2}{3} \times \frac{5}{50}\\
&=\frac{11}{75}
\end{align}

\newpage

%\tableofcontents

\bigskip

\renewcommand{\thefigure}{\theenumi}
\renewcommand{\thetable}{\theenumi}
%\renewcommand{\theequation}{\theenumi}

%\begin{abstract}
%%\boldmath
%In this letter, an algorithm for evaluating the exact analytical bit error rate  (BER)  for the piecewise linear (PL) combiner for  multiple relays is presented. Previous results were available only for upto three relays. The algorithm is unique in the sense that  the actual mathematical expressions, that are prohibitively large, need not be explicitly obtained. The diversity gain due to multiple relays is shown through plots of the analytical BER, well supported by simulations. 
%
%\end{abstract}
% IEEEtran.cls defaults to using nonbold math in the Abstract.
% This preserves the distinction between vectors and scalars. However,
% if the journal you are submitting to favors bold math in the abstract,
% then you can use LaTeX's standard command \boldmath at the very start
% of the abstract to achieve this. Many IEEE journals frown on math
% in the abstract anyway.

% Note that keywords are not normally used for peerreview papers.
%\begin{IEEEkeywords}
%Cooperative diversity, decode and forward, piecewise linear
%\end{IEEEkeywords}



% For peer review papers, you can put extra information on the cover
% page as needed:
% \ifCLASSOPTIONpeerreview
% \begin{center} \bfseries EDICS Category: 3-BBND \end{center}
% \fi
%
% For peerreview papers, this IEEEtran command inserts a page break and
% creates the second title. It will be ignored for other modes.
%\IEEEpeerreviewmaketitle




If the letters of the word ASSASSINATION are arranged at random. Find the Probability that
\begin{enumerate}[label=(\alph*)]
\item Four $S's$ come consecutively in the word
\item Two  $I's$ and two $N's$ come together
\item All $A's$ are not coming together
\item No two $A's$ are coming together
\end{enumerate}
%\begin{table}[H]
	\centering
\begin{tabular}{|c|c|c|}
\hline
Random variable &Value &Definition\\ \hline
\multirow{3}{*}{X} &0 &Slips of Rs 1\\
&1 &Slips of Rs 5\\
&2 &Slips of Rs 13\\ \hline
\multirow{2}{*}{Y} &0 &Box A\\
&1 &Box B\\\hline
\end{tabular}
\caption{}
\label{tab:Distribution}
\end{table}
See \tabref{tab:Distribution}.
\begin{align}
p_{Y}\brak{k}= \begin{cases} 
      \frac{1}{3} & {k=0} \\
      \frac{2}{3 }& {k=1} 
   \end{cases}
   \\
p_{Y|X}\brak{0|0} = \frac{19}{25}\, 
p_{Y|X}\brak{0|1} = \frac{6}{25}\,
p_{Y|X}\brak{1|0} = \frac{45}{50}\,
p_{Y|X}\brak{1|2} = \frac{5}{50}
\end{align}
The desired probability is the probability that a slip drawn at random is marked other than Rs 1,
\begin{align}
&=1-p_X\brak{0}\\
&= p_X(1) + p_X(2)
\end{align}
Using Bayes theorem,
\begin{align}
&= p_Y\brak{0} \times \pr{Y=0 | X=1} + p_Y\brak{1} \times \pr{Y=1|X=2}\\
&=\frac{1}{3} \times \frac{6}{25} + \frac{2}{3} \times \frac{5}{50}\\
&=\frac{11}{75}
\end{align}

\newpage

%\tableofcontents

\bigskip

\renewcommand{\thefigure}{\theenumi}
\renewcommand{\thetable}{\theenumi}
%\renewcommand{\theequation}{\theenumi}

%\begin{abstract}
%%\boldmath
%In this letter, an algorithm for evaluating the exact analytical bit error rate  (BER)  for the piecewise linear (PL) combiner for  multiple relays is presented. Previous results were available only for upto three relays. The algorithm is unique in the sense that  the actual mathematical expressions, that are prohibitively large, need not be explicitly obtained. The diversity gain due to multiple relays is shown through plots of the analytical BER, well supported by simulations. 
%
%\end{abstract}
% IEEEtran.cls defaults to using nonbold math in the Abstract.
% This preserves the distinction between vectors and scalars. However,
% if the journal you are submitting to favors bold math in the abstract,
% then you can use LaTeX's standard command \boldmath at the very start
% of the abstract to achieve this. Many IEEE journals frown on math
% in the abstract anyway.

% Note that keywords are not normally used for peerreview papers.
%\begin{IEEEkeywords}
%Cooperative diversity, decode and forward, piecewise linear
%\end{IEEEkeywords}



% For peer review papers, you can put extra information on the cover
% page as needed:
% \ifCLASSOPTIONpeerreview
% \begin{center} \bfseries EDICS Category: 3-BBND \end{center}
% \fi
%
% For peerreview papers, this IEEEtran command inserts a page break and
% creates the second title. It will be ignored for other modes.
%\IEEEpeerreviewmaketitle




	\item One urn contains two black balls (labelled B1 and B2) and one white ball. A
	second urn contains one black ball and two white balls (labelled W1 and W2).
	Suppose the following experiment is performed. One of the two urns is chosen
	at random. Next a ball is randomly chosen from the urn. Then a second ball is
	chosen at random from the same urn without replacing the first ball.
	
	\begin{enumerate}
	\item What is the probability that two black balls are chosen?
	
	\item What is the probability that two balls of opposite colour are chosen?
	\end{enumerate}
	\solution
	%\begin{align}
    \label{eq:12.13.6.18.1}
	\because	\pr{A|B} &> \pr{A},\
\frac{\pr{AB}}{\pr{B}} > \pr{A}
\\
    \label{eq:12.13.6.18.2}
	\implies \pr{AB} &> \pr{A}\pr{B}
	\\
	\text{or, } \frac{\pr{AB}}{\pr{A}} &=\pr{B|A} > \pr{A}
\end{align}

\end{enumerate}

\item In a certain lottery 10,000 tickets are sold and ten equal prizes are awarded. What is the probability of not getting a prize if you buy (a) one ticket (b) two tickets (c) 10 tickets ?	
\\
\solution
		%\begin{enumerate}[label=\thesection.\arabic*,ref=\thesection.\theenumi]
	\item One card is drawn from a well-shuffled deck of 52 cards. Find the probability of getting
\begin{enumerate}
\item A king of red colour 
\item A face card 
\item A red face card
\item The jack of hearts
\item A spade
\item The queen of diamonds

\end{enumerate}
\solution
		%\begin{table}[H]
	\centering
\begin{tabular}{|c|c|c|}
\hline
Random variable &Value &Definition\\ \hline
\multirow{3}{*}{X} &0 &Slips of Rs 1\\
&1 &Slips of Rs 5\\
&2 &Slips of Rs 13\\ \hline
\multirow{2}{*}{Y} &0 &Box A\\
&1 &Box B\\\hline
\end{tabular}
\caption{}
\label{tab:Distribution}
\end{table}
See \tabref{tab:Distribution}.
\begin{align}
p_{Y}\brak{k}= \begin{cases} 
      \frac{1}{3} & {k=0} \\
      \frac{2}{3 }& {k=1} 
   \end{cases}
   \\
p_{Y|X}\brak{0|0} = \frac{19}{25}\, 
p_{Y|X}\brak{0|1} = \frac{6}{25}\,
p_{Y|X}\brak{1|0} = \frac{45}{50}\,
p_{Y|X}\brak{1|2} = \frac{5}{50}
\end{align}
The desired probability is the probability that a slip drawn at random is marked other than Rs 1,
\begin{align}
&=1-p_X\brak{0}\\
&= p_X(1) + p_X(2)
\end{align}
Using Bayes theorem,
\begin{align}
&= p_Y\brak{0} \times \pr{Y=0 | X=1} + p_Y\brak{1} \times \pr{Y=1|X=2}\\
&=\frac{1}{3} \times \frac{6}{25} + \frac{2}{3} \times \frac{5}{50}\\
&=\frac{11}{75}
\end{align}

\newpage

%\tableofcontents

\bigskip

\renewcommand{\thefigure}{\theenumi}
\renewcommand{\thetable}{\theenumi}
%\renewcommand{\theequation}{\theenumi}

%\begin{abstract}
%%\boldmath
%In this letter, an algorithm for evaluating the exact analytical bit error rate  (BER)  for the piecewise linear (PL) combiner for  multiple relays is presented. Previous results were available only for upto three relays. The algorithm is unique in the sense that  the actual mathematical expressions, that are prohibitively large, need not be explicitly obtained. The diversity gain due to multiple relays is shown through plots of the analytical BER, well supported by simulations. 
%
%\end{abstract}
% IEEEtran.cls defaults to using nonbold math in the Abstract.
% This preserves the distinction between vectors and scalars. However,
% if the journal you are submitting to favors bold math in the abstract,
% then you can use LaTeX's standard command \boldmath at the very start
% of the abstract to achieve this. Many IEEE journals frown on math
% in the abstract anyway.

% Note that keywords are not normally used for peerreview papers.
%\begin{IEEEkeywords}
%Cooperative diversity, decode and forward, piecewise linear
%\end{IEEEkeywords}



% For peer review papers, you can put extra information on the cover
% page as needed:
% \ifCLASSOPTIONpeerreview
% \begin{center} \bfseries EDICS Category: 3-BBND \end{center}
% \fi
%
% For peerreview papers, this IEEEtran command inserts a page break and
% creates the second title. It will be ignored for other modes.
%\IEEEpeerreviewmaketitle




	\item Five cards—the ten, jack, queen, king and ace of diamonds, are well-shuffled with their face downwards. One card is then picked up at random.
\begin{enumerate}
\item
What is the probability that the card is the queen? 
\item
If the queen is drawn and put aside, what is the probability that the second card picked up is (a) an ace? (b) a queen?\\
\end{enumerate}
\solution
		%\begin{enumerate}[label=\thesection.\arabic*,ref=\thesection.\theenumi]
	\item One card is drawn from a well-shuffled deck of 52 cards. Find the probability of getting
\begin{enumerate}
\item A king of red colour 
\item A face card 
\item A red face card
\item The jack of hearts
\item A spade
\item The queen of diamonds

\end{enumerate}
\solution
		%\input{ncert/10/15/1/14/main.tex}
	\item Five cards—the ten, jack, queen, king and ace of diamonds, are well-shuffled with their face downwards. One card is then picked up at random.
\begin{enumerate}
\item
What is the probability that the card is the queen? 
\item
If the queen is drawn and put aside, what is the probability that the second card picked up is (a) an ace? (b) a queen?\\
\end{enumerate}
\solution
		%\input{ncert/10/15/1/15/defs.tex}
	\item A bag contains $5$ red balls and some blue balls. If the probability of drawing a blue ball is double that if a red ball, determine the number of blue balls in the bag. 
		\\
\solution
		%\input{ncert/10/15/2/3/defs.tex}
	\item A card is selected from a pack of 52 cards.
 \begin{enumerate}[label=(\alph*)] 
                 \item How many points are there in the sample space?
                 \item Calculate the probability that the card is an ace of spades.
                 \item Calculate the probability that the card is (i) an ace and (ii) black card.
 \end{enumerate}
\solution
		%\input{ncert/11/16/3/4/main.tex}
\item Four cards are drawn from a well-shuffled deck of 52 cards. What is the probability of obtaining 3 diamonds and one spade.
\\
\solution
		%\input{ncert/11/16/4/2/defs.tex}
\item In a certain lottery 10,000 tickets are sold and ten equal prizes are awarded. What is the probability of not getting a prize if you buy (a) one ticket (b) two tickets (c) 10 tickets ?	
\\
\solution
		%\input{ncert/11/16/4/4/defs.tex}
		%
\item 
Out of 100 students, two sections of 40 and 60 are formed. If you and your friend are among the 100 students, what is the probability that
\begin{enumerate}
\item you both enter the same section?
\item you both enter the different sections?
\end{enumerate}
\solution
		%\input{ncert/11/16/4/5/defs.tex}
	\item 
The number lock of a suitcase has 4 wheels each labelled with ten digits i.e. from 0 to 9.The lock opens with a sequence of four digits with no repeats.What is the probability of a person getting the right sequence to open the suitcase.
\\
\solution
		%\input{ncert/11/16/4/10/defs.tex}
		%
\item 
Two cards are drawn at random and without replacement from a pack of 52 playing cards. Find the probability that both the cards are black.
\\
\solution
		%\input{ncert/12/13/2/2/defs.tex}
		\item A box of oranges is inspected by examining three randomly selected oranges drawn without replacement. If all the three oranges are good, the box is approved for sale, otherwise, it is rejected. Find the probability that a box containing 15 oranges out of which 12 are good and 3 are bad ones will be approved for sale.
		\label{ncert/12/13/2/3/defs.tex}
		\item Two balls are drawn at random with replacement from a box containing 10 black and 8 red balls. Find the probability that
		\label{ncert/12/13/2/12}
\begin{enumerate}
\item both balls are red.
\item first ball is black and second is red.
\item one of them is black and other is red.
\end{enumerate}

\item In a hostel, 60\% of the students read Hindi newspaper, 40\% read English newspaper and 20\% read both Hindi and English newspapers. A student is selected at random.
		\label{ncert/12/13/2/15}
\begin{enumerate}
\item Find the probability that she reads neither Hindi nor English newspapers.
\item If she reads Hindi newspaper, find the probability that she reads English newspaper.
\item If she reads English newspaper, find the probability that she reads Hindi newspaper.\\
\end{enumerate}
\item The probability of obtaining an even prime number on each die, when a pair of dice is rolled is 
\begin{enumerate}
    \item $0$ 
    
    \item $\frac{1}{3}$ 
    
    \item $\frac{1}{12}$ 
    
    \item $\frac{1}{36}$ 
\end{enumerate}
\solution
		%\input{ncert/12/13/2/17/defs.tex}
	\item A bag contains 4 red and 4 black balls, another bag contains 2 red and 6 black balls. One of the two bags is selected at random and a ball is drawn from the bag which is found to be red. Find the probability that the ball is drawn from the first bag.
\\
\solution
		%\input{ncert/12/13/3/2/main.tex}
  \item
  Cards with numbers 2 to 101 are placed in a box. A card is selected at random.Find the probability that the card has
\begin{enumerate}[label=(\roman*)]
	\item an even number 
	\item a square number
\end{enumerate}
\solution
%\input{exemplar/10/13/3/32/main.tex}
\item
The king, queen and jack of clubs are removed from a deck of 52 playing cards and then well shuffled. Now one card is drawn at random from the remaining cards.  Determine the probability that the card is
\begin{enumerate}[label=(\roman*)]
\item a club
\item 10 of hearts
\end{enumerate}
\solution
%\input{exemplar/10/13/3/29/main.tex}
\item A team of medical students doing their internship have to assist during surgeries
at a city hospital. The probabilities of surgeries rated as very complex, complex,
routine, simple or very simple are respectively, 0.15, 0.20, 0.31, 0.26, .08. Find
the probabilities that a particular surgery will be rated
\begin{enumerate}
	\item complex or very complex;
	\item neither very complex nor very simple;
	\item routine or complex
	\item routine or simple
\end{enumerate}
\solution
%\input{exemplar/11/16/3/8(1)/main.tex}
\item A card is selected from a pack of 52 cards.
\begin{enumerate}[label=(\alph*)]
    \item How many points are there in the sample space?
    \item Calculate the probability that the card is an ace of spades.
    \item Calculate the probability that the card is (i) an ace and (ii) black card.
\end{enumerate}
\solution
%\input{exemplar/11/16/3/4/main2.tex}
\item The probability that a non leap year selected at random will contain 53 sundays.
\\
\solution
%\input{exemplar/10/13/1/19/main.tex}
\item One of the four persons John, Rita, Aslam or Gurpreet will be promoted next
month. Consequently the sample space consists of four elementary outcomes
S = {John promoted, Rita promoted, Aslam promoted, Gurpreet promoted}
You are told that the chances of John’s promotion is same as that of Gurpreet,
Rita’s chances of promotion are twice as likely as Johns. Aslam’s chances are
four times that of John.
\begin{enumerate}
	\item Determine
	\begin{enumerate}
		\item P (John promoted)
		\item P (Rita promoted)
		\item P (Aslam promoted)
		\item P (Gurpreet promoted)
	\end{enumerate}
	\item If A = {John promoted or Gurpreet promoted}, find P (A).
\end{enumerate}
\solution
%\input{exemplar/11/16/3/10/main.tex}
\item A card is drawn from a deck of 52 cards. Find the probability of getting a king or a heart or a red card.\\
\solution
%\input{exemplar/11/16/3/15/main.tex}
\item The probability that a student will pass his examination is 0.73, the probability of
the student getting a compartment is 0.13, and the probability that the student will
either pass or get compartment is 0.96. State True or False.\\
\solution
%\input{exemplar/11/16/3/31/main.tex}
\item A card is selected from a pack of 52 cards\\
\begin{enumerate}[label=(\alph*)]
\item How many points are there in the sample space?
\item Calculate the probability that the cards is an ace of spades.
\item Calculate the probability that the card is (i) an ace (ii)black card.\\
\end{enumerate}
%\input{ncert/11/16/3/4_1/Prob_4.tex}
\item In a non-leap year, the probability of having 53 tuesdays or 53 wednesdays is\\
\solution
%\input{exemplar/11/16/3/18/main.tex}
\item There are 1000 sealed envelopes in a box, 10 of them contain a cash prize of
Rs 100 each, 100 of them contain a cash prize of Rs 50 each and 200 of them
contain a cash prize of Rs 10 each and rest do not contain any cash prize. If they
are well shuffled and an envelope is picked up out, what is the probability that it
contains no cash prize?\\
\solution
%\input{exemplar/10/13/3/34/main.tex}
\item 
A die is thrown and a card is selected at random from a deck of 52 playing cards. The probability of getting an even number on the die and a spade card.\\
\solution
%\input{exemplar/12/13/3/78/main.tex}
\item
If 4-digit numbers greater than 5,000 are randomly formed from the digits 0, 1, 3, 5, and 7, what is the probability of forming a number divisible by 5 when:
\begin{enumerate}
    \item The digits are repeated?
    \item The repetition of digits is not allowed?
\end{enumerate}
\solution
%\input{ncert/11/16/4/9/main.tex}
\item Consider the probability space $\brak{\Omega, \mathcal{G}, P}$ where $\Omega = [0,2]$ and $\mathcal{G} = \cbrak{\phi, \Omega, [0,1], (1,2]}$. Let $X$ and $Y$ be two functions on $\Omega$ defined as
\begin{align*}
    X(\omega) = 
    \begin{cases}
        1 & \text{if }\omega \in [0, 1]\\
        2 & \text{if }\omega \in (1, 2]
    \end{cases}
\end{align*}
and
\begin{align*}
    Y(\omega) = 
    \begin{cases}
        2 & \text{if }\omega \in [0, 1.5]\\
        3 & \text{if }\omega \in (1.5, 2].
    \end{cases}
\end{align*}
Then which one of the following statements is true?
\begin{enumerate}
    \item [(A)] $X$ is a random variable with respect to $\mathcal{G}$, but $Y$ is not a random variable with respect to $\mathcal{G}$.
    \item [(B)] $Y$ is a random variable with respect to $\mathcal{G}$, but $X$ is not a random variable with respect to $\mathcal{G}$.
    \item [(C)] Neither $X$ nor $Y$ is a random variable with respect to $\mathcal{G}$.
    \item [(D)] Both $X$ and $Y$ are random variables with respect to $\mathcal{G}$.
\end{enumerate} \hfill (GATE ST 2023)\\
\solution
%\input{gate/ST/2023/14/main.tex}
	\item  A die is loaded in such a way that each odd number is twice as likely to occur as
each even number. Find $P(G)$, where $G$ is the event that a number greater than
3 occurs on a single roll of the die.
\\
\solution
		%\input{exemplar/11/16/3/5/main.tex}
	\item All the jacks, queens and kings are removed from a deck of 52 playing cards. The remaining cards are well shuffled and then one card is drawn at random. Giving ace a value 1 similar value for other cards, find the probability that the card has a value 
		\begin{enumerate}
			\item 7
			\item greater than 7
			\item less than 7
		\end{enumerate}
		%\input{exemplar/10/13/3/30/main.tex}
  \item A Lot consists of 48 mobile phones of which 42 are good, 3 have only minor defects and 3 have major defects.Varnika will buy a phone if it is good but the trader will only buy a mobile if it has no major defects. One phone is selected at random from the lot. What is the probability that it is
\begin{enumerate}
	\item acceptable to Varnika?
            \item acceptable to the trader?
\end{enumerate}
\solution
	%\input{exemplar/10/13/3/40/main.tex}
 \item A student says that if you throw a die, it will show up 1 or not 1. Therefore, the probability of getting 1 and the probability of getting 'not 1' each is equal to $\frac{1}{2}$. Is this correct? Give reasons.\\
 \solution
        %\input{exemplar/10/13/2/9/main.tex}
   \item Four candidates A, B, C, D have ap-
plied for the assignment to coach a school cricket
team. If A is twice as likely to be selected as B, and
B and C are given about the same chance of being
selected, while C is twice as likely to be selected
as D, what are the probabilities that
\begin{enumerate}
\item C will be selected?
\item A will not be selected?
\end{enumerate}
	%\input{exemplar/11/16/3/9/main.tex}
 \item A bag contain 24 balls of which $x$ balls are red, $2x$ are white and $3x$ are blue. A ball is selected at random, What is the probability that it is
\begin{enumerate}[label=\alph*)]
\item not red ?
\item white ?
\end{enumerate}
%\input{exemplar/10/13/3/41/main.tex}
If the letters of the word ASSASSINATION are arranged at random. Find the Probability that
\begin{enumerate}[label=(\alph*)]
\item Four $S's$ come consecutively in the word
\item Two  $I's$ and two $N's$ come together
\item All $A's$ are not coming together
\item No two $A's$ are coming together
\end{enumerate}
%\input{exemplar/11/16/3/14/main.tex}
	\item One urn contains two black balls (labelled B1 and B2) and one white ball. A
	second urn contains one black ball and two white balls (labelled W1 and W2).
	Suppose the following experiment is performed. One of the two urns is chosen
	at random. Next a ball is randomly chosen from the urn. Then a second ball is
	chosen at random from the same urn without replacing the first ball.
	
	\begin{enumerate}
	\item What is the probability that two black balls are chosen?
	
	\item What is the probability that two balls of opposite colour are chosen?
	\end{enumerate}
	\solution
	%\input{exemplar/11/16/3/12/main1.tex}
\end{enumerate}

	\item A bag contains $5$ red balls and some blue balls. If the probability of drawing a blue ball is double that if a red ball, determine the number of blue balls in the bag. 
		\\
\solution
		%\begin{enumerate}[label=\thesection.\arabic*,ref=\thesection.\theenumi]
	\item One card is drawn from a well-shuffled deck of 52 cards. Find the probability of getting
\begin{enumerate}
\item A king of red colour 
\item A face card 
\item A red face card
\item The jack of hearts
\item A spade
\item The queen of diamonds

\end{enumerate}
\solution
		%\input{ncert/10/15/1/14/main.tex}
	\item Five cards—the ten, jack, queen, king and ace of diamonds, are well-shuffled with their face downwards. One card is then picked up at random.
\begin{enumerate}
\item
What is the probability that the card is the queen? 
\item
If the queen is drawn and put aside, what is the probability that the second card picked up is (a) an ace? (b) a queen?\\
\end{enumerate}
\solution
		%\input{ncert/10/15/1/15/defs.tex}
	\item A bag contains $5$ red balls and some blue balls. If the probability of drawing a blue ball is double that if a red ball, determine the number of blue balls in the bag. 
		\\
\solution
		%\input{ncert/10/15/2/3/defs.tex}
	\item A card is selected from a pack of 52 cards.
 \begin{enumerate}[label=(\alph*)] 
                 \item How many points are there in the sample space?
                 \item Calculate the probability that the card is an ace of spades.
                 \item Calculate the probability that the card is (i) an ace and (ii) black card.
 \end{enumerate}
\solution
		%\input{ncert/11/16/3/4/main.tex}
\item Four cards are drawn from a well-shuffled deck of 52 cards. What is the probability of obtaining 3 diamonds and one spade.
\\
\solution
		%\input{ncert/11/16/4/2/defs.tex}
\item In a certain lottery 10,000 tickets are sold and ten equal prizes are awarded. What is the probability of not getting a prize if you buy (a) one ticket (b) two tickets (c) 10 tickets ?	
\\
\solution
		%\input{ncert/11/16/4/4/defs.tex}
		%
\item 
Out of 100 students, two sections of 40 and 60 are formed. If you and your friend are among the 100 students, what is the probability that
\begin{enumerate}
\item you both enter the same section?
\item you both enter the different sections?
\end{enumerate}
\solution
		%\input{ncert/11/16/4/5/defs.tex}
	\item 
The number lock of a suitcase has 4 wheels each labelled with ten digits i.e. from 0 to 9.The lock opens with a sequence of four digits with no repeats.What is the probability of a person getting the right sequence to open the suitcase.
\\
\solution
		%\input{ncert/11/16/4/10/defs.tex}
		%
\item 
Two cards are drawn at random and without replacement from a pack of 52 playing cards. Find the probability that both the cards are black.
\\
\solution
		%\input{ncert/12/13/2/2/defs.tex}
		\item A box of oranges is inspected by examining three randomly selected oranges drawn without replacement. If all the three oranges are good, the box is approved for sale, otherwise, it is rejected. Find the probability that a box containing 15 oranges out of which 12 are good and 3 are bad ones will be approved for sale.
		\label{ncert/12/13/2/3/defs.tex}
		\item Two balls are drawn at random with replacement from a box containing 10 black and 8 red balls. Find the probability that
		\label{ncert/12/13/2/12}
\begin{enumerate}
\item both balls are red.
\item first ball is black and second is red.
\item one of them is black and other is red.
\end{enumerate}

\item In a hostel, 60\% of the students read Hindi newspaper, 40\% read English newspaper and 20\% read both Hindi and English newspapers. A student is selected at random.
		\label{ncert/12/13/2/15}
\begin{enumerate}
\item Find the probability that she reads neither Hindi nor English newspapers.
\item If she reads Hindi newspaper, find the probability that she reads English newspaper.
\item If she reads English newspaper, find the probability that she reads Hindi newspaper.\\
\end{enumerate}
\item The probability of obtaining an even prime number on each die, when a pair of dice is rolled is 
\begin{enumerate}
    \item $0$ 
    
    \item $\frac{1}{3}$ 
    
    \item $\frac{1}{12}$ 
    
    \item $\frac{1}{36}$ 
\end{enumerate}
\solution
		%\input{ncert/12/13/2/17/defs.tex}
	\item A bag contains 4 red and 4 black balls, another bag contains 2 red and 6 black balls. One of the two bags is selected at random and a ball is drawn from the bag which is found to be red. Find the probability that the ball is drawn from the first bag.
\\
\solution
		%\input{ncert/12/13/3/2/main.tex}
  \item
  Cards with numbers 2 to 101 are placed in a box. A card is selected at random.Find the probability that the card has
\begin{enumerate}[label=(\roman*)]
	\item an even number 
	\item a square number
\end{enumerate}
\solution
%\input{exemplar/10/13/3/32/main.tex}
\item
The king, queen and jack of clubs are removed from a deck of 52 playing cards and then well shuffled. Now one card is drawn at random from the remaining cards.  Determine the probability that the card is
\begin{enumerate}[label=(\roman*)]
\item a club
\item 10 of hearts
\end{enumerate}
\solution
%\input{exemplar/10/13/3/29/main.tex}
\item A team of medical students doing their internship have to assist during surgeries
at a city hospital. The probabilities of surgeries rated as very complex, complex,
routine, simple or very simple are respectively, 0.15, 0.20, 0.31, 0.26, .08. Find
the probabilities that a particular surgery will be rated
\begin{enumerate}
	\item complex or very complex;
	\item neither very complex nor very simple;
	\item routine or complex
	\item routine or simple
\end{enumerate}
\solution
%\input{exemplar/11/16/3/8(1)/main.tex}
\item A card is selected from a pack of 52 cards.
\begin{enumerate}[label=(\alph*)]
    \item How many points are there in the sample space?
    \item Calculate the probability that the card is an ace of spades.
    \item Calculate the probability that the card is (i) an ace and (ii) black card.
\end{enumerate}
\solution
%\input{exemplar/11/16/3/4/main2.tex}
\item The probability that a non leap year selected at random will contain 53 sundays.
\\
\solution
%\input{exemplar/10/13/1/19/main.tex}
\item One of the four persons John, Rita, Aslam or Gurpreet will be promoted next
month. Consequently the sample space consists of four elementary outcomes
S = {John promoted, Rita promoted, Aslam promoted, Gurpreet promoted}
You are told that the chances of John’s promotion is same as that of Gurpreet,
Rita’s chances of promotion are twice as likely as Johns. Aslam’s chances are
four times that of John.
\begin{enumerate}
	\item Determine
	\begin{enumerate}
		\item P (John promoted)
		\item P (Rita promoted)
		\item P (Aslam promoted)
		\item P (Gurpreet promoted)
	\end{enumerate}
	\item If A = {John promoted or Gurpreet promoted}, find P (A).
\end{enumerate}
\solution
%\input{exemplar/11/16/3/10/main.tex}
\item A card is drawn from a deck of 52 cards. Find the probability of getting a king or a heart or a red card.\\
\solution
%\input{exemplar/11/16/3/15/main.tex}
\item The probability that a student will pass his examination is 0.73, the probability of
the student getting a compartment is 0.13, and the probability that the student will
either pass or get compartment is 0.96. State True or False.\\
\solution
%\input{exemplar/11/16/3/31/main.tex}
\item A card is selected from a pack of 52 cards\\
\begin{enumerate}[label=(\alph*)]
\item How many points are there in the sample space?
\item Calculate the probability that the cards is an ace of spades.
\item Calculate the probability that the card is (i) an ace (ii)black card.\\
\end{enumerate}
%\input{ncert/11/16/3/4_1/Prob_4.tex}
\item In a non-leap year, the probability of having 53 tuesdays or 53 wednesdays is\\
\solution
%\input{exemplar/11/16/3/18/main.tex}
\item There are 1000 sealed envelopes in a box, 10 of them contain a cash prize of
Rs 100 each, 100 of them contain a cash prize of Rs 50 each and 200 of them
contain a cash prize of Rs 10 each and rest do not contain any cash prize. If they
are well shuffled and an envelope is picked up out, what is the probability that it
contains no cash prize?\\
\solution
%\input{exemplar/10/13/3/34/main.tex}
\item 
A die is thrown and a card is selected at random from a deck of 52 playing cards. The probability of getting an even number on the die and a spade card.\\
\solution
%\input{exemplar/12/13/3/78/main.tex}
\item
If 4-digit numbers greater than 5,000 are randomly formed from the digits 0, 1, 3, 5, and 7, what is the probability of forming a number divisible by 5 when:
\begin{enumerate}
    \item The digits are repeated?
    \item The repetition of digits is not allowed?
\end{enumerate}
\solution
%\input{ncert/11/16/4/9/main.tex}
\item Consider the probability space $\brak{\Omega, \mathcal{G}, P}$ where $\Omega = [0,2]$ and $\mathcal{G} = \cbrak{\phi, \Omega, [0,1], (1,2]}$. Let $X$ and $Y$ be two functions on $\Omega$ defined as
\begin{align*}
    X(\omega) = 
    \begin{cases}
        1 & \text{if }\omega \in [0, 1]\\
        2 & \text{if }\omega \in (1, 2]
    \end{cases}
\end{align*}
and
\begin{align*}
    Y(\omega) = 
    \begin{cases}
        2 & \text{if }\omega \in [0, 1.5]\\
        3 & \text{if }\omega \in (1.5, 2].
    \end{cases}
\end{align*}
Then which one of the following statements is true?
\begin{enumerate}
    \item [(A)] $X$ is a random variable with respect to $\mathcal{G}$, but $Y$ is not a random variable with respect to $\mathcal{G}$.
    \item [(B)] $Y$ is a random variable with respect to $\mathcal{G}$, but $X$ is not a random variable with respect to $\mathcal{G}$.
    \item [(C)] Neither $X$ nor $Y$ is a random variable with respect to $\mathcal{G}$.
    \item [(D)] Both $X$ and $Y$ are random variables with respect to $\mathcal{G}$.
\end{enumerate} \hfill (GATE ST 2023)\\
\solution
%\input{gate/ST/2023/14/main.tex}
	\item  A die is loaded in such a way that each odd number is twice as likely to occur as
each even number. Find $P(G)$, where $G$ is the event that a number greater than
3 occurs on a single roll of the die.
\\
\solution
		%\input{exemplar/11/16/3/5/main.tex}
	\item All the jacks, queens and kings are removed from a deck of 52 playing cards. The remaining cards are well shuffled and then one card is drawn at random. Giving ace a value 1 similar value for other cards, find the probability that the card has a value 
		\begin{enumerate}
			\item 7
			\item greater than 7
			\item less than 7
		\end{enumerate}
		%\input{exemplar/10/13/3/30/main.tex}
  \item A Lot consists of 48 mobile phones of which 42 are good, 3 have only minor defects and 3 have major defects.Varnika will buy a phone if it is good but the trader will only buy a mobile if it has no major defects. One phone is selected at random from the lot. What is the probability that it is
\begin{enumerate}
	\item acceptable to Varnika?
            \item acceptable to the trader?
\end{enumerate}
\solution
	%\input{exemplar/10/13/3/40/main.tex}
 \item A student says that if you throw a die, it will show up 1 or not 1. Therefore, the probability of getting 1 and the probability of getting 'not 1' each is equal to $\frac{1}{2}$. Is this correct? Give reasons.\\
 \solution
        %\input{exemplar/10/13/2/9/main.tex}
   \item Four candidates A, B, C, D have ap-
plied for the assignment to coach a school cricket
team. If A is twice as likely to be selected as B, and
B and C are given about the same chance of being
selected, while C is twice as likely to be selected
as D, what are the probabilities that
\begin{enumerate}
\item C will be selected?
\item A will not be selected?
\end{enumerate}
	%\input{exemplar/11/16/3/9/main.tex}
 \item A bag contain 24 balls of which $x$ balls are red, $2x$ are white and $3x$ are blue. A ball is selected at random, What is the probability that it is
\begin{enumerate}[label=\alph*)]
\item not red ?
\item white ?
\end{enumerate}
%\input{exemplar/10/13/3/41/main.tex}
If the letters of the word ASSASSINATION are arranged at random. Find the Probability that
\begin{enumerate}[label=(\alph*)]
\item Four $S's$ come consecutively in the word
\item Two  $I's$ and two $N's$ come together
\item All $A's$ are not coming together
\item No two $A's$ are coming together
\end{enumerate}
%\input{exemplar/11/16/3/14/main.tex}
	\item One urn contains two black balls (labelled B1 and B2) and one white ball. A
	second urn contains one black ball and two white balls (labelled W1 and W2).
	Suppose the following experiment is performed. One of the two urns is chosen
	at random. Next a ball is randomly chosen from the urn. Then a second ball is
	chosen at random from the same urn without replacing the first ball.
	
	\begin{enumerate}
	\item What is the probability that two black balls are chosen?
	
	\item What is the probability that two balls of opposite colour are chosen?
	\end{enumerate}
	\solution
	%\input{exemplar/11/16/3/12/main1.tex}
\end{enumerate}

	\item A card is selected from a pack of 52 cards.
 \begin{enumerate}[label=(\alph*)] 
                 \item How many points are there in the sample space?
                 \item Calculate the probability that the card is an ace of spades.
                 \item Calculate the probability that the card is (i) an ace and (ii) black card.
 \end{enumerate}
\solution
		%\begin{table}[H]
	\centering
\begin{tabular}{|c|c|c|}
\hline
Random variable &Value &Definition\\ \hline
\multirow{3}{*}{X} &0 &Slips of Rs 1\\
&1 &Slips of Rs 5\\
&2 &Slips of Rs 13\\ \hline
\multirow{2}{*}{Y} &0 &Box A\\
&1 &Box B\\\hline
\end{tabular}
\caption{}
\label{tab:Distribution}
\end{table}
See \tabref{tab:Distribution}.
\begin{align}
p_{Y}\brak{k}= \begin{cases} 
      \frac{1}{3} & {k=0} \\
      \frac{2}{3 }& {k=1} 
   \end{cases}
   \\
p_{Y|X}\brak{0|0} = \frac{19}{25}\, 
p_{Y|X}\brak{0|1} = \frac{6}{25}\,
p_{Y|X}\brak{1|0} = \frac{45}{50}\,
p_{Y|X}\brak{1|2} = \frac{5}{50}
\end{align}
The desired probability is the probability that a slip drawn at random is marked other than Rs 1,
\begin{align}
&=1-p_X\brak{0}\\
&= p_X(1) + p_X(2)
\end{align}
Using Bayes theorem,
\begin{align}
&= p_Y\brak{0} \times \pr{Y=0 | X=1} + p_Y\brak{1} \times \pr{Y=1|X=2}\\
&=\frac{1}{3} \times \frac{6}{25} + \frac{2}{3} \times \frac{5}{50}\\
&=\frac{11}{75}
\end{align}

\newpage

%\tableofcontents

\bigskip

\renewcommand{\thefigure}{\theenumi}
\renewcommand{\thetable}{\theenumi}
%\renewcommand{\theequation}{\theenumi}

%\begin{abstract}
%%\boldmath
%In this letter, an algorithm for evaluating the exact analytical bit error rate  (BER)  for the piecewise linear (PL) combiner for  multiple relays is presented. Previous results were available only for upto three relays. The algorithm is unique in the sense that  the actual mathematical expressions, that are prohibitively large, need not be explicitly obtained. The diversity gain due to multiple relays is shown through plots of the analytical BER, well supported by simulations. 
%
%\end{abstract}
% IEEEtran.cls defaults to using nonbold math in the Abstract.
% This preserves the distinction between vectors and scalars. However,
% if the journal you are submitting to favors bold math in the abstract,
% then you can use LaTeX's standard command \boldmath at the very start
% of the abstract to achieve this. Many IEEE journals frown on math
% in the abstract anyway.

% Note that keywords are not normally used for peerreview papers.
%\begin{IEEEkeywords}
%Cooperative diversity, decode and forward, piecewise linear
%\end{IEEEkeywords}



% For peer review papers, you can put extra information on the cover
% page as needed:
% \ifCLASSOPTIONpeerreview
% \begin{center} \bfseries EDICS Category: 3-BBND \end{center}
% \fi
%
% For peerreview papers, this IEEEtran command inserts a page break and
% creates the second title. It will be ignored for other modes.
%\IEEEpeerreviewmaketitle




\item Four cards are drawn from a well-shuffled deck of 52 cards. What is the probability of obtaining 3 diamonds and one spade.
\\
\solution
		%\begin{enumerate}[label=\thesection.\arabic*,ref=\thesection.\theenumi]
	\item One card is drawn from a well-shuffled deck of 52 cards. Find the probability of getting
\begin{enumerate}
\item A king of red colour 
\item A face card 
\item A red face card
\item The jack of hearts
\item A spade
\item The queen of diamonds

\end{enumerate}
\solution
		%\input{ncert/10/15/1/14/main.tex}
	\item Five cards—the ten, jack, queen, king and ace of diamonds, are well-shuffled with their face downwards. One card is then picked up at random.
\begin{enumerate}
\item
What is the probability that the card is the queen? 
\item
If the queen is drawn and put aside, what is the probability that the second card picked up is (a) an ace? (b) a queen?\\
\end{enumerate}
\solution
		%\input{ncert/10/15/1/15/defs.tex}
	\item A bag contains $5$ red balls and some blue balls. If the probability of drawing a blue ball is double that if a red ball, determine the number of blue balls in the bag. 
		\\
\solution
		%\input{ncert/10/15/2/3/defs.tex}
	\item A card is selected from a pack of 52 cards.
 \begin{enumerate}[label=(\alph*)] 
                 \item How many points are there in the sample space?
                 \item Calculate the probability that the card is an ace of spades.
                 \item Calculate the probability that the card is (i) an ace and (ii) black card.
 \end{enumerate}
\solution
		%\input{ncert/11/16/3/4/main.tex}
\item Four cards are drawn from a well-shuffled deck of 52 cards. What is the probability of obtaining 3 diamonds and one spade.
\\
\solution
		%\input{ncert/11/16/4/2/defs.tex}
\item In a certain lottery 10,000 tickets are sold and ten equal prizes are awarded. What is the probability of not getting a prize if you buy (a) one ticket (b) two tickets (c) 10 tickets ?	
\\
\solution
		%\input{ncert/11/16/4/4/defs.tex}
		%
\item 
Out of 100 students, two sections of 40 and 60 are formed. If you and your friend are among the 100 students, what is the probability that
\begin{enumerate}
\item you both enter the same section?
\item you both enter the different sections?
\end{enumerate}
\solution
		%\input{ncert/11/16/4/5/defs.tex}
	\item 
The number lock of a suitcase has 4 wheels each labelled with ten digits i.e. from 0 to 9.The lock opens with a sequence of four digits with no repeats.What is the probability of a person getting the right sequence to open the suitcase.
\\
\solution
		%\input{ncert/11/16/4/10/defs.tex}
		%
\item 
Two cards are drawn at random and without replacement from a pack of 52 playing cards. Find the probability that both the cards are black.
\\
\solution
		%\input{ncert/12/13/2/2/defs.tex}
		\item A box of oranges is inspected by examining three randomly selected oranges drawn without replacement. If all the three oranges are good, the box is approved for sale, otherwise, it is rejected. Find the probability that a box containing 15 oranges out of which 12 are good and 3 are bad ones will be approved for sale.
		\label{ncert/12/13/2/3/defs.tex}
		\item Two balls are drawn at random with replacement from a box containing 10 black and 8 red balls. Find the probability that
		\label{ncert/12/13/2/12}
\begin{enumerate}
\item both balls are red.
\item first ball is black and second is red.
\item one of them is black and other is red.
\end{enumerate}

\item In a hostel, 60\% of the students read Hindi newspaper, 40\% read English newspaper and 20\% read both Hindi and English newspapers. A student is selected at random.
		\label{ncert/12/13/2/15}
\begin{enumerate}
\item Find the probability that she reads neither Hindi nor English newspapers.
\item If she reads Hindi newspaper, find the probability that she reads English newspaper.
\item If she reads English newspaper, find the probability that she reads Hindi newspaper.\\
\end{enumerate}
\item The probability of obtaining an even prime number on each die, when a pair of dice is rolled is 
\begin{enumerate}
    \item $0$ 
    
    \item $\frac{1}{3}$ 
    
    \item $\frac{1}{12}$ 
    
    \item $\frac{1}{36}$ 
\end{enumerate}
\solution
		%\input{ncert/12/13/2/17/defs.tex}
	\item A bag contains 4 red and 4 black balls, another bag contains 2 red and 6 black balls. One of the two bags is selected at random and a ball is drawn from the bag which is found to be red. Find the probability that the ball is drawn from the first bag.
\\
\solution
		%\input{ncert/12/13/3/2/main.tex}
  \item
  Cards with numbers 2 to 101 are placed in a box. A card is selected at random.Find the probability that the card has
\begin{enumerate}[label=(\roman*)]
	\item an even number 
	\item a square number
\end{enumerate}
\solution
%\input{exemplar/10/13/3/32/main.tex}
\item
The king, queen and jack of clubs are removed from a deck of 52 playing cards and then well shuffled. Now one card is drawn at random from the remaining cards.  Determine the probability that the card is
\begin{enumerate}[label=(\roman*)]
\item a club
\item 10 of hearts
\end{enumerate}
\solution
%\input{exemplar/10/13/3/29/main.tex}
\item A team of medical students doing their internship have to assist during surgeries
at a city hospital. The probabilities of surgeries rated as very complex, complex,
routine, simple or very simple are respectively, 0.15, 0.20, 0.31, 0.26, .08. Find
the probabilities that a particular surgery will be rated
\begin{enumerate}
	\item complex or very complex;
	\item neither very complex nor very simple;
	\item routine or complex
	\item routine or simple
\end{enumerate}
\solution
%\input{exemplar/11/16/3/8(1)/main.tex}
\item A card is selected from a pack of 52 cards.
\begin{enumerate}[label=(\alph*)]
    \item How many points are there in the sample space?
    \item Calculate the probability that the card is an ace of spades.
    \item Calculate the probability that the card is (i) an ace and (ii) black card.
\end{enumerate}
\solution
%\input{exemplar/11/16/3/4/main2.tex}
\item The probability that a non leap year selected at random will contain 53 sundays.
\\
\solution
%\input{exemplar/10/13/1/19/main.tex}
\item One of the four persons John, Rita, Aslam or Gurpreet will be promoted next
month. Consequently the sample space consists of four elementary outcomes
S = {John promoted, Rita promoted, Aslam promoted, Gurpreet promoted}
You are told that the chances of John’s promotion is same as that of Gurpreet,
Rita’s chances of promotion are twice as likely as Johns. Aslam’s chances are
four times that of John.
\begin{enumerate}
	\item Determine
	\begin{enumerate}
		\item P (John promoted)
		\item P (Rita promoted)
		\item P (Aslam promoted)
		\item P (Gurpreet promoted)
	\end{enumerate}
	\item If A = {John promoted or Gurpreet promoted}, find P (A).
\end{enumerate}
\solution
%\input{exemplar/11/16/3/10/main.tex}
\item A card is drawn from a deck of 52 cards. Find the probability of getting a king or a heart or a red card.\\
\solution
%\input{exemplar/11/16/3/15/main.tex}
\item The probability that a student will pass his examination is 0.73, the probability of
the student getting a compartment is 0.13, and the probability that the student will
either pass or get compartment is 0.96. State True or False.\\
\solution
%\input{exemplar/11/16/3/31/main.tex}
\item A card is selected from a pack of 52 cards\\
\begin{enumerate}[label=(\alph*)]
\item How many points are there in the sample space?
\item Calculate the probability that the cards is an ace of spades.
\item Calculate the probability that the card is (i) an ace (ii)black card.\\
\end{enumerate}
%\input{ncert/11/16/3/4_1/Prob_4.tex}
\item In a non-leap year, the probability of having 53 tuesdays or 53 wednesdays is\\
\solution
%\input{exemplar/11/16/3/18/main.tex}
\item There are 1000 sealed envelopes in a box, 10 of them contain a cash prize of
Rs 100 each, 100 of them contain a cash prize of Rs 50 each and 200 of them
contain a cash prize of Rs 10 each and rest do not contain any cash prize. If they
are well shuffled and an envelope is picked up out, what is the probability that it
contains no cash prize?\\
\solution
%\input{exemplar/10/13/3/34/main.tex}
\item 
A die is thrown and a card is selected at random from a deck of 52 playing cards. The probability of getting an even number on the die and a spade card.\\
\solution
%\input{exemplar/12/13/3/78/main.tex}
\item
If 4-digit numbers greater than 5,000 are randomly formed from the digits 0, 1, 3, 5, and 7, what is the probability of forming a number divisible by 5 when:
\begin{enumerate}
    \item The digits are repeated?
    \item The repetition of digits is not allowed?
\end{enumerate}
\solution
%\input{ncert/11/16/4/9/main.tex}
\item Consider the probability space $\brak{\Omega, \mathcal{G}, P}$ where $\Omega = [0,2]$ and $\mathcal{G} = \cbrak{\phi, \Omega, [0,1], (1,2]}$. Let $X$ and $Y$ be two functions on $\Omega$ defined as
\begin{align*}
    X(\omega) = 
    \begin{cases}
        1 & \text{if }\omega \in [0, 1]\\
        2 & \text{if }\omega \in (1, 2]
    \end{cases}
\end{align*}
and
\begin{align*}
    Y(\omega) = 
    \begin{cases}
        2 & \text{if }\omega \in [0, 1.5]\\
        3 & \text{if }\omega \in (1.5, 2].
    \end{cases}
\end{align*}
Then which one of the following statements is true?
\begin{enumerate}
    \item [(A)] $X$ is a random variable with respect to $\mathcal{G}$, but $Y$ is not a random variable with respect to $\mathcal{G}$.
    \item [(B)] $Y$ is a random variable with respect to $\mathcal{G}$, but $X$ is not a random variable with respect to $\mathcal{G}$.
    \item [(C)] Neither $X$ nor $Y$ is a random variable with respect to $\mathcal{G}$.
    \item [(D)] Both $X$ and $Y$ are random variables with respect to $\mathcal{G}$.
\end{enumerate} \hfill (GATE ST 2023)\\
\solution
%\input{gate/ST/2023/14/main.tex}
	\item  A die is loaded in such a way that each odd number is twice as likely to occur as
each even number. Find $P(G)$, where $G$ is the event that a number greater than
3 occurs on a single roll of the die.
\\
\solution
		%\input{exemplar/11/16/3/5/main.tex}
	\item All the jacks, queens and kings are removed from a deck of 52 playing cards. The remaining cards are well shuffled and then one card is drawn at random. Giving ace a value 1 similar value for other cards, find the probability that the card has a value 
		\begin{enumerate}
			\item 7
			\item greater than 7
			\item less than 7
		\end{enumerate}
		%\input{exemplar/10/13/3/30/main.tex}
  \item A Lot consists of 48 mobile phones of which 42 are good, 3 have only minor defects and 3 have major defects.Varnika will buy a phone if it is good but the trader will only buy a mobile if it has no major defects. One phone is selected at random from the lot. What is the probability that it is
\begin{enumerate}
	\item acceptable to Varnika?
            \item acceptable to the trader?
\end{enumerate}
\solution
	%\input{exemplar/10/13/3/40/main.tex}
 \item A student says that if you throw a die, it will show up 1 or not 1. Therefore, the probability of getting 1 and the probability of getting 'not 1' each is equal to $\frac{1}{2}$. Is this correct? Give reasons.\\
 \solution
        %\input{exemplar/10/13/2/9/main.tex}
   \item Four candidates A, B, C, D have ap-
plied for the assignment to coach a school cricket
team. If A is twice as likely to be selected as B, and
B and C are given about the same chance of being
selected, while C is twice as likely to be selected
as D, what are the probabilities that
\begin{enumerate}
\item C will be selected?
\item A will not be selected?
\end{enumerate}
	%\input{exemplar/11/16/3/9/main.tex}
 \item A bag contain 24 balls of which $x$ balls are red, $2x$ are white and $3x$ are blue. A ball is selected at random, What is the probability that it is
\begin{enumerate}[label=\alph*)]
\item not red ?
\item white ?
\end{enumerate}
%\input{exemplar/10/13/3/41/main.tex}
If the letters of the word ASSASSINATION are arranged at random. Find the Probability that
\begin{enumerate}[label=(\alph*)]
\item Four $S's$ come consecutively in the word
\item Two  $I's$ and two $N's$ come together
\item All $A's$ are not coming together
\item No two $A's$ are coming together
\end{enumerate}
%\input{exemplar/11/16/3/14/main.tex}
	\item One urn contains two black balls (labelled B1 and B2) and one white ball. A
	second urn contains one black ball and two white balls (labelled W1 and W2).
	Suppose the following experiment is performed. One of the two urns is chosen
	at random. Next a ball is randomly chosen from the urn. Then a second ball is
	chosen at random from the same urn without replacing the first ball.
	
	\begin{enumerate}
	\item What is the probability that two black balls are chosen?
	
	\item What is the probability that two balls of opposite colour are chosen?
	\end{enumerate}
	\solution
	%\input{exemplar/11/16/3/12/main1.tex}
\end{enumerate}

\item In a certain lottery 10,000 tickets are sold and ten equal prizes are awarded. What is the probability of not getting a prize if you buy (a) one ticket (b) two tickets (c) 10 tickets ?	
\\
\solution
		%\begin{enumerate}[label=\thesection.\arabic*,ref=\thesection.\theenumi]
	\item One card is drawn from a well-shuffled deck of 52 cards. Find the probability of getting
\begin{enumerate}
\item A king of red colour 
\item A face card 
\item A red face card
\item The jack of hearts
\item A spade
\item The queen of diamonds

\end{enumerate}
\solution
		%\input{ncert/10/15/1/14/main.tex}
	\item Five cards—the ten, jack, queen, king and ace of diamonds, are well-shuffled with their face downwards. One card is then picked up at random.
\begin{enumerate}
\item
What is the probability that the card is the queen? 
\item
If the queen is drawn and put aside, what is the probability that the second card picked up is (a) an ace? (b) a queen?\\
\end{enumerate}
\solution
		%\input{ncert/10/15/1/15/defs.tex}
	\item A bag contains $5$ red balls and some blue balls. If the probability of drawing a blue ball is double that if a red ball, determine the number of blue balls in the bag. 
		\\
\solution
		%\input{ncert/10/15/2/3/defs.tex}
	\item A card is selected from a pack of 52 cards.
 \begin{enumerate}[label=(\alph*)] 
                 \item How many points are there in the sample space?
                 \item Calculate the probability that the card is an ace of spades.
                 \item Calculate the probability that the card is (i) an ace and (ii) black card.
 \end{enumerate}
\solution
		%\input{ncert/11/16/3/4/main.tex}
\item Four cards are drawn from a well-shuffled deck of 52 cards. What is the probability of obtaining 3 diamonds and one spade.
\\
\solution
		%\input{ncert/11/16/4/2/defs.tex}
\item In a certain lottery 10,000 tickets are sold and ten equal prizes are awarded. What is the probability of not getting a prize if you buy (a) one ticket (b) two tickets (c) 10 tickets ?	
\\
\solution
		%\input{ncert/11/16/4/4/defs.tex}
		%
\item 
Out of 100 students, two sections of 40 and 60 are formed. If you and your friend are among the 100 students, what is the probability that
\begin{enumerate}
\item you both enter the same section?
\item you both enter the different sections?
\end{enumerate}
\solution
		%\input{ncert/11/16/4/5/defs.tex}
	\item 
The number lock of a suitcase has 4 wheels each labelled with ten digits i.e. from 0 to 9.The lock opens with a sequence of four digits with no repeats.What is the probability of a person getting the right sequence to open the suitcase.
\\
\solution
		%\input{ncert/11/16/4/10/defs.tex}
		%
\item 
Two cards are drawn at random and without replacement from a pack of 52 playing cards. Find the probability that both the cards are black.
\\
\solution
		%\input{ncert/12/13/2/2/defs.tex}
		\item A box of oranges is inspected by examining three randomly selected oranges drawn without replacement. If all the three oranges are good, the box is approved for sale, otherwise, it is rejected. Find the probability that a box containing 15 oranges out of which 12 are good and 3 are bad ones will be approved for sale.
		\label{ncert/12/13/2/3/defs.tex}
		\item Two balls are drawn at random with replacement from a box containing 10 black and 8 red balls. Find the probability that
		\label{ncert/12/13/2/12}
\begin{enumerate}
\item both balls are red.
\item first ball is black and second is red.
\item one of them is black and other is red.
\end{enumerate}

\item In a hostel, 60\% of the students read Hindi newspaper, 40\% read English newspaper and 20\% read both Hindi and English newspapers. A student is selected at random.
		\label{ncert/12/13/2/15}
\begin{enumerate}
\item Find the probability that she reads neither Hindi nor English newspapers.
\item If she reads Hindi newspaper, find the probability that she reads English newspaper.
\item If she reads English newspaper, find the probability that she reads Hindi newspaper.\\
\end{enumerate}
\item The probability of obtaining an even prime number on each die, when a pair of dice is rolled is 
\begin{enumerate}
    \item $0$ 
    
    \item $\frac{1}{3}$ 
    
    \item $\frac{1}{12}$ 
    
    \item $\frac{1}{36}$ 
\end{enumerate}
\solution
		%\input{ncert/12/13/2/17/defs.tex}
	\item A bag contains 4 red and 4 black balls, another bag contains 2 red and 6 black balls. One of the two bags is selected at random and a ball is drawn from the bag which is found to be red. Find the probability that the ball is drawn from the first bag.
\\
\solution
		%\input{ncert/12/13/3/2/main.tex}
  \item
  Cards with numbers 2 to 101 are placed in a box. A card is selected at random.Find the probability that the card has
\begin{enumerate}[label=(\roman*)]
	\item an even number 
	\item a square number
\end{enumerate}
\solution
%\input{exemplar/10/13/3/32/main.tex}
\item
The king, queen and jack of clubs are removed from a deck of 52 playing cards and then well shuffled. Now one card is drawn at random from the remaining cards.  Determine the probability that the card is
\begin{enumerate}[label=(\roman*)]
\item a club
\item 10 of hearts
\end{enumerate}
\solution
%\input{exemplar/10/13/3/29/main.tex}
\item A team of medical students doing their internship have to assist during surgeries
at a city hospital. The probabilities of surgeries rated as very complex, complex,
routine, simple or very simple are respectively, 0.15, 0.20, 0.31, 0.26, .08. Find
the probabilities that a particular surgery will be rated
\begin{enumerate}
	\item complex or very complex;
	\item neither very complex nor very simple;
	\item routine or complex
	\item routine or simple
\end{enumerate}
\solution
%\input{exemplar/11/16/3/8(1)/main.tex}
\item A card is selected from a pack of 52 cards.
\begin{enumerate}[label=(\alph*)]
    \item How many points are there in the sample space?
    \item Calculate the probability that the card is an ace of spades.
    \item Calculate the probability that the card is (i) an ace and (ii) black card.
\end{enumerate}
\solution
%\input{exemplar/11/16/3/4/main2.tex}
\item The probability that a non leap year selected at random will contain 53 sundays.
\\
\solution
%\input{exemplar/10/13/1/19/main.tex}
\item One of the four persons John, Rita, Aslam or Gurpreet will be promoted next
month. Consequently the sample space consists of four elementary outcomes
S = {John promoted, Rita promoted, Aslam promoted, Gurpreet promoted}
You are told that the chances of John’s promotion is same as that of Gurpreet,
Rita’s chances of promotion are twice as likely as Johns. Aslam’s chances are
four times that of John.
\begin{enumerate}
	\item Determine
	\begin{enumerate}
		\item P (John promoted)
		\item P (Rita promoted)
		\item P (Aslam promoted)
		\item P (Gurpreet promoted)
	\end{enumerate}
	\item If A = {John promoted or Gurpreet promoted}, find P (A).
\end{enumerate}
\solution
%\input{exemplar/11/16/3/10/main.tex}
\item A card is drawn from a deck of 52 cards. Find the probability of getting a king or a heart or a red card.\\
\solution
%\input{exemplar/11/16/3/15/main.tex}
\item The probability that a student will pass his examination is 0.73, the probability of
the student getting a compartment is 0.13, and the probability that the student will
either pass or get compartment is 0.96. State True or False.\\
\solution
%\input{exemplar/11/16/3/31/main.tex}
\item A card is selected from a pack of 52 cards\\
\begin{enumerate}[label=(\alph*)]
\item How many points are there in the sample space?
\item Calculate the probability that the cards is an ace of spades.
\item Calculate the probability that the card is (i) an ace (ii)black card.\\
\end{enumerate}
%\input{ncert/11/16/3/4_1/Prob_4.tex}
\item In a non-leap year, the probability of having 53 tuesdays or 53 wednesdays is\\
\solution
%\input{exemplar/11/16/3/18/main.tex}
\item There are 1000 sealed envelopes in a box, 10 of them contain a cash prize of
Rs 100 each, 100 of them contain a cash prize of Rs 50 each and 200 of them
contain a cash prize of Rs 10 each and rest do not contain any cash prize. If they
are well shuffled and an envelope is picked up out, what is the probability that it
contains no cash prize?\\
\solution
%\input{exemplar/10/13/3/34/main.tex}
\item 
A die is thrown and a card is selected at random from a deck of 52 playing cards. The probability of getting an even number on the die and a spade card.\\
\solution
%\input{exemplar/12/13/3/78/main.tex}
\item
If 4-digit numbers greater than 5,000 are randomly formed from the digits 0, 1, 3, 5, and 7, what is the probability of forming a number divisible by 5 when:
\begin{enumerate}
    \item The digits are repeated?
    \item The repetition of digits is not allowed?
\end{enumerate}
\solution
%\input{ncert/11/16/4/9/main.tex}
\item Consider the probability space $\brak{\Omega, \mathcal{G}, P}$ where $\Omega = [0,2]$ and $\mathcal{G} = \cbrak{\phi, \Omega, [0,1], (1,2]}$. Let $X$ and $Y$ be two functions on $\Omega$ defined as
\begin{align*}
    X(\omega) = 
    \begin{cases}
        1 & \text{if }\omega \in [0, 1]\\
        2 & \text{if }\omega \in (1, 2]
    \end{cases}
\end{align*}
and
\begin{align*}
    Y(\omega) = 
    \begin{cases}
        2 & \text{if }\omega \in [0, 1.5]\\
        3 & \text{if }\omega \in (1.5, 2].
    \end{cases}
\end{align*}
Then which one of the following statements is true?
\begin{enumerate}
    \item [(A)] $X$ is a random variable with respect to $\mathcal{G}$, but $Y$ is not a random variable with respect to $\mathcal{G}$.
    \item [(B)] $Y$ is a random variable with respect to $\mathcal{G}$, but $X$ is not a random variable with respect to $\mathcal{G}$.
    \item [(C)] Neither $X$ nor $Y$ is a random variable with respect to $\mathcal{G}$.
    \item [(D)] Both $X$ and $Y$ are random variables with respect to $\mathcal{G}$.
\end{enumerate} \hfill (GATE ST 2023)\\
\solution
%\input{gate/ST/2023/14/main.tex}
	\item  A die is loaded in such a way that each odd number is twice as likely to occur as
each even number. Find $P(G)$, where $G$ is the event that a number greater than
3 occurs on a single roll of the die.
\\
\solution
		%\input{exemplar/11/16/3/5/main.tex}
	\item All the jacks, queens and kings are removed from a deck of 52 playing cards. The remaining cards are well shuffled and then one card is drawn at random. Giving ace a value 1 similar value for other cards, find the probability that the card has a value 
		\begin{enumerate}
			\item 7
			\item greater than 7
			\item less than 7
		\end{enumerate}
		%\input{exemplar/10/13/3/30/main.tex}
  \item A Lot consists of 48 mobile phones of which 42 are good, 3 have only minor defects and 3 have major defects.Varnika will buy a phone if it is good but the trader will only buy a mobile if it has no major defects. One phone is selected at random from the lot. What is the probability that it is
\begin{enumerate}
	\item acceptable to Varnika?
            \item acceptable to the trader?
\end{enumerate}
\solution
	%\input{exemplar/10/13/3/40/main.tex}
 \item A student says that if you throw a die, it will show up 1 or not 1. Therefore, the probability of getting 1 and the probability of getting 'not 1' each is equal to $\frac{1}{2}$. Is this correct? Give reasons.\\
 \solution
        %\input{exemplar/10/13/2/9/main.tex}
   \item Four candidates A, B, C, D have ap-
plied for the assignment to coach a school cricket
team. If A is twice as likely to be selected as B, and
B and C are given about the same chance of being
selected, while C is twice as likely to be selected
as D, what are the probabilities that
\begin{enumerate}
\item C will be selected?
\item A will not be selected?
\end{enumerate}
	%\input{exemplar/11/16/3/9/main.tex}
 \item A bag contain 24 balls of which $x$ balls are red, $2x$ are white and $3x$ are blue. A ball is selected at random, What is the probability that it is
\begin{enumerate}[label=\alph*)]
\item not red ?
\item white ?
\end{enumerate}
%\input{exemplar/10/13/3/41/main.tex}
If the letters of the word ASSASSINATION are arranged at random. Find the Probability that
\begin{enumerate}[label=(\alph*)]
\item Four $S's$ come consecutively in the word
\item Two  $I's$ and two $N's$ come together
\item All $A's$ are not coming together
\item No two $A's$ are coming together
\end{enumerate}
%\input{exemplar/11/16/3/14/main.tex}
	\item One urn contains two black balls (labelled B1 and B2) and one white ball. A
	second urn contains one black ball and two white balls (labelled W1 and W2).
	Suppose the following experiment is performed. One of the two urns is chosen
	at random. Next a ball is randomly chosen from the urn. Then a second ball is
	chosen at random from the same urn without replacing the first ball.
	
	\begin{enumerate}
	\item What is the probability that two black balls are chosen?
	
	\item What is the probability that two balls of opposite colour are chosen?
	\end{enumerate}
	\solution
	%\input{exemplar/11/16/3/12/main1.tex}
\end{enumerate}

		%
\item 
Out of 100 students, two sections of 40 and 60 are formed. If you and your friend are among the 100 students, what is the probability that
\begin{enumerate}
\item you both enter the same section?
\item you both enter the different sections?
\end{enumerate}
\solution
		%\begin{enumerate}[label=\thesection.\arabic*,ref=\thesection.\theenumi]
	\item One card is drawn from a well-shuffled deck of 52 cards. Find the probability of getting
\begin{enumerate}
\item A king of red colour 
\item A face card 
\item A red face card
\item The jack of hearts
\item A spade
\item The queen of diamonds

\end{enumerate}
\solution
		%\input{ncert/10/15/1/14/main.tex}
	\item Five cards—the ten, jack, queen, king and ace of diamonds, are well-shuffled with their face downwards. One card is then picked up at random.
\begin{enumerate}
\item
What is the probability that the card is the queen? 
\item
If the queen is drawn and put aside, what is the probability that the second card picked up is (a) an ace? (b) a queen?\\
\end{enumerate}
\solution
		%\input{ncert/10/15/1/15/defs.tex}
	\item A bag contains $5$ red balls and some blue balls. If the probability of drawing a blue ball is double that if a red ball, determine the number of blue balls in the bag. 
		\\
\solution
		%\input{ncert/10/15/2/3/defs.tex}
	\item A card is selected from a pack of 52 cards.
 \begin{enumerate}[label=(\alph*)] 
                 \item How many points are there in the sample space?
                 \item Calculate the probability that the card is an ace of spades.
                 \item Calculate the probability that the card is (i) an ace and (ii) black card.
 \end{enumerate}
\solution
		%\input{ncert/11/16/3/4/main.tex}
\item Four cards are drawn from a well-shuffled deck of 52 cards. What is the probability of obtaining 3 diamonds and one spade.
\\
\solution
		%\input{ncert/11/16/4/2/defs.tex}
\item In a certain lottery 10,000 tickets are sold and ten equal prizes are awarded. What is the probability of not getting a prize if you buy (a) one ticket (b) two tickets (c) 10 tickets ?	
\\
\solution
		%\input{ncert/11/16/4/4/defs.tex}
		%
\item 
Out of 100 students, two sections of 40 and 60 are formed. If you and your friend are among the 100 students, what is the probability that
\begin{enumerate}
\item you both enter the same section?
\item you both enter the different sections?
\end{enumerate}
\solution
		%\input{ncert/11/16/4/5/defs.tex}
	\item 
The number lock of a suitcase has 4 wheels each labelled with ten digits i.e. from 0 to 9.The lock opens with a sequence of four digits with no repeats.What is the probability of a person getting the right sequence to open the suitcase.
\\
\solution
		%\input{ncert/11/16/4/10/defs.tex}
		%
\item 
Two cards are drawn at random and without replacement from a pack of 52 playing cards. Find the probability that both the cards are black.
\\
\solution
		%\input{ncert/12/13/2/2/defs.tex}
		\item A box of oranges is inspected by examining three randomly selected oranges drawn without replacement. If all the three oranges are good, the box is approved for sale, otherwise, it is rejected. Find the probability that a box containing 15 oranges out of which 12 are good and 3 are bad ones will be approved for sale.
		\label{ncert/12/13/2/3/defs.tex}
		\item Two balls are drawn at random with replacement from a box containing 10 black and 8 red balls. Find the probability that
		\label{ncert/12/13/2/12}
\begin{enumerate}
\item both balls are red.
\item first ball is black and second is red.
\item one of them is black and other is red.
\end{enumerate}

\item In a hostel, 60\% of the students read Hindi newspaper, 40\% read English newspaper and 20\% read both Hindi and English newspapers. A student is selected at random.
		\label{ncert/12/13/2/15}
\begin{enumerate}
\item Find the probability that she reads neither Hindi nor English newspapers.
\item If she reads Hindi newspaper, find the probability that she reads English newspaper.
\item If she reads English newspaper, find the probability that she reads Hindi newspaper.\\
\end{enumerate}
\item The probability of obtaining an even prime number on each die, when a pair of dice is rolled is 
\begin{enumerate}
    \item $0$ 
    
    \item $\frac{1}{3}$ 
    
    \item $\frac{1}{12}$ 
    
    \item $\frac{1}{36}$ 
\end{enumerate}
\solution
		%\input{ncert/12/13/2/17/defs.tex}
	\item A bag contains 4 red and 4 black balls, another bag contains 2 red and 6 black balls. One of the two bags is selected at random and a ball is drawn from the bag which is found to be red. Find the probability that the ball is drawn from the first bag.
\\
\solution
		%\input{ncert/12/13/3/2/main.tex}
  \item
  Cards with numbers 2 to 101 are placed in a box. A card is selected at random.Find the probability that the card has
\begin{enumerate}[label=(\roman*)]
	\item an even number 
	\item a square number
\end{enumerate}
\solution
%\input{exemplar/10/13/3/32/main.tex}
\item
The king, queen and jack of clubs are removed from a deck of 52 playing cards and then well shuffled. Now one card is drawn at random from the remaining cards.  Determine the probability that the card is
\begin{enumerate}[label=(\roman*)]
\item a club
\item 10 of hearts
\end{enumerate}
\solution
%\input{exemplar/10/13/3/29/main.tex}
\item A team of medical students doing their internship have to assist during surgeries
at a city hospital. The probabilities of surgeries rated as very complex, complex,
routine, simple or very simple are respectively, 0.15, 0.20, 0.31, 0.26, .08. Find
the probabilities that a particular surgery will be rated
\begin{enumerate}
	\item complex or very complex;
	\item neither very complex nor very simple;
	\item routine or complex
	\item routine or simple
\end{enumerate}
\solution
%\input{exemplar/11/16/3/8(1)/main.tex}
\item A card is selected from a pack of 52 cards.
\begin{enumerate}[label=(\alph*)]
    \item How many points are there in the sample space?
    \item Calculate the probability that the card is an ace of spades.
    \item Calculate the probability that the card is (i) an ace and (ii) black card.
\end{enumerate}
\solution
%\input{exemplar/11/16/3/4/main2.tex}
\item The probability that a non leap year selected at random will contain 53 sundays.
\\
\solution
%\input{exemplar/10/13/1/19/main.tex}
\item One of the four persons John, Rita, Aslam or Gurpreet will be promoted next
month. Consequently the sample space consists of four elementary outcomes
S = {John promoted, Rita promoted, Aslam promoted, Gurpreet promoted}
You are told that the chances of John’s promotion is same as that of Gurpreet,
Rita’s chances of promotion are twice as likely as Johns. Aslam’s chances are
four times that of John.
\begin{enumerate}
	\item Determine
	\begin{enumerate}
		\item P (John promoted)
		\item P (Rita promoted)
		\item P (Aslam promoted)
		\item P (Gurpreet promoted)
	\end{enumerate}
	\item If A = {John promoted or Gurpreet promoted}, find P (A).
\end{enumerate}
\solution
%\input{exemplar/11/16/3/10/main.tex}
\item A card is drawn from a deck of 52 cards. Find the probability of getting a king or a heart or a red card.\\
\solution
%\input{exemplar/11/16/3/15/main.tex}
\item The probability that a student will pass his examination is 0.73, the probability of
the student getting a compartment is 0.13, and the probability that the student will
either pass or get compartment is 0.96. State True or False.\\
\solution
%\input{exemplar/11/16/3/31/main.tex}
\item A card is selected from a pack of 52 cards\\
\begin{enumerate}[label=(\alph*)]
\item How many points are there in the sample space?
\item Calculate the probability that the cards is an ace of spades.
\item Calculate the probability that the card is (i) an ace (ii)black card.\\
\end{enumerate}
%\input{ncert/11/16/3/4_1/Prob_4.tex}
\item In a non-leap year, the probability of having 53 tuesdays or 53 wednesdays is\\
\solution
%\input{exemplar/11/16/3/18/main.tex}
\item There are 1000 sealed envelopes in a box, 10 of them contain a cash prize of
Rs 100 each, 100 of them contain a cash prize of Rs 50 each and 200 of them
contain a cash prize of Rs 10 each and rest do not contain any cash prize. If they
are well shuffled and an envelope is picked up out, what is the probability that it
contains no cash prize?\\
\solution
%\input{exemplar/10/13/3/34/main.tex}
\item 
A die is thrown and a card is selected at random from a deck of 52 playing cards. The probability of getting an even number on the die and a spade card.\\
\solution
%\input{exemplar/12/13/3/78/main.tex}
\item
If 4-digit numbers greater than 5,000 are randomly formed from the digits 0, 1, 3, 5, and 7, what is the probability of forming a number divisible by 5 when:
\begin{enumerate}
    \item The digits are repeated?
    \item The repetition of digits is not allowed?
\end{enumerate}
\solution
%\input{ncert/11/16/4/9/main.tex}
\item Consider the probability space $\brak{\Omega, \mathcal{G}, P}$ where $\Omega = [0,2]$ and $\mathcal{G} = \cbrak{\phi, \Omega, [0,1], (1,2]}$. Let $X$ and $Y$ be two functions on $\Omega$ defined as
\begin{align*}
    X(\omega) = 
    \begin{cases}
        1 & \text{if }\omega \in [0, 1]\\
        2 & \text{if }\omega \in (1, 2]
    \end{cases}
\end{align*}
and
\begin{align*}
    Y(\omega) = 
    \begin{cases}
        2 & \text{if }\omega \in [0, 1.5]\\
        3 & \text{if }\omega \in (1.5, 2].
    \end{cases}
\end{align*}
Then which one of the following statements is true?
\begin{enumerate}
    \item [(A)] $X$ is a random variable with respect to $\mathcal{G}$, but $Y$ is not a random variable with respect to $\mathcal{G}$.
    \item [(B)] $Y$ is a random variable with respect to $\mathcal{G}$, but $X$ is not a random variable with respect to $\mathcal{G}$.
    \item [(C)] Neither $X$ nor $Y$ is a random variable with respect to $\mathcal{G}$.
    \item [(D)] Both $X$ and $Y$ are random variables with respect to $\mathcal{G}$.
\end{enumerate} \hfill (GATE ST 2023)\\
\solution
%\input{gate/ST/2023/14/main.tex}
	\item  A die is loaded in such a way that each odd number is twice as likely to occur as
each even number. Find $P(G)$, where $G$ is the event that a number greater than
3 occurs on a single roll of the die.
\\
\solution
		%\input{exemplar/11/16/3/5/main.tex}
	\item All the jacks, queens and kings are removed from a deck of 52 playing cards. The remaining cards are well shuffled and then one card is drawn at random. Giving ace a value 1 similar value for other cards, find the probability that the card has a value 
		\begin{enumerate}
			\item 7
			\item greater than 7
			\item less than 7
		\end{enumerate}
		%\input{exemplar/10/13/3/30/main.tex}
  \item A Lot consists of 48 mobile phones of which 42 are good, 3 have only minor defects and 3 have major defects.Varnika will buy a phone if it is good but the trader will only buy a mobile if it has no major defects. One phone is selected at random from the lot. What is the probability that it is
\begin{enumerate}
	\item acceptable to Varnika?
            \item acceptable to the trader?
\end{enumerate}
\solution
	%\input{exemplar/10/13/3/40/main.tex}
 \item A student says that if you throw a die, it will show up 1 or not 1. Therefore, the probability of getting 1 and the probability of getting 'not 1' each is equal to $\frac{1}{2}$. Is this correct? Give reasons.\\
 \solution
        %\input{exemplar/10/13/2/9/main.tex}
   \item Four candidates A, B, C, D have ap-
plied for the assignment to coach a school cricket
team. If A is twice as likely to be selected as B, and
B and C are given about the same chance of being
selected, while C is twice as likely to be selected
as D, what are the probabilities that
\begin{enumerate}
\item C will be selected?
\item A will not be selected?
\end{enumerate}
	%\input{exemplar/11/16/3/9/main.tex}
 \item A bag contain 24 balls of which $x$ balls are red, $2x$ are white and $3x$ are blue. A ball is selected at random, What is the probability that it is
\begin{enumerate}[label=\alph*)]
\item not red ?
\item white ?
\end{enumerate}
%\input{exemplar/10/13/3/41/main.tex}
If the letters of the word ASSASSINATION are arranged at random. Find the Probability that
\begin{enumerate}[label=(\alph*)]
\item Four $S's$ come consecutively in the word
\item Two  $I's$ and two $N's$ come together
\item All $A's$ are not coming together
\item No two $A's$ are coming together
\end{enumerate}
%\input{exemplar/11/16/3/14/main.tex}
	\item One urn contains two black balls (labelled B1 and B2) and one white ball. A
	second urn contains one black ball and two white balls (labelled W1 and W2).
	Suppose the following experiment is performed. One of the two urns is chosen
	at random. Next a ball is randomly chosen from the urn. Then a second ball is
	chosen at random from the same urn without replacing the first ball.
	
	\begin{enumerate}
	\item What is the probability that two black balls are chosen?
	
	\item What is the probability that two balls of opposite colour are chosen?
	\end{enumerate}
	\solution
	%\input{exemplar/11/16/3/12/main1.tex}
\end{enumerate}

	\item 
The number lock of a suitcase has 4 wheels each labelled with ten digits i.e. from 0 to 9.The lock opens with a sequence of four digits with no repeats.What is the probability of a person getting the right sequence to open the suitcase.
\\
\solution
		%\begin{enumerate}[label=\thesection.\arabic*,ref=\thesection.\theenumi]
	\item One card is drawn from a well-shuffled deck of 52 cards. Find the probability of getting
\begin{enumerate}
\item A king of red colour 
\item A face card 
\item A red face card
\item The jack of hearts
\item A spade
\item The queen of diamonds

\end{enumerate}
\solution
		%\input{ncert/10/15/1/14/main.tex}
	\item Five cards—the ten, jack, queen, king and ace of diamonds, are well-shuffled with their face downwards. One card is then picked up at random.
\begin{enumerate}
\item
What is the probability that the card is the queen? 
\item
If the queen is drawn and put aside, what is the probability that the second card picked up is (a) an ace? (b) a queen?\\
\end{enumerate}
\solution
		%\input{ncert/10/15/1/15/defs.tex}
	\item A bag contains $5$ red balls and some blue balls. If the probability of drawing a blue ball is double that if a red ball, determine the number of blue balls in the bag. 
		\\
\solution
		%\input{ncert/10/15/2/3/defs.tex}
	\item A card is selected from a pack of 52 cards.
 \begin{enumerate}[label=(\alph*)] 
                 \item How many points are there in the sample space?
                 \item Calculate the probability that the card is an ace of spades.
                 \item Calculate the probability that the card is (i) an ace and (ii) black card.
 \end{enumerate}
\solution
		%\input{ncert/11/16/3/4/main.tex}
\item Four cards are drawn from a well-shuffled deck of 52 cards. What is the probability of obtaining 3 diamonds and one spade.
\\
\solution
		%\input{ncert/11/16/4/2/defs.tex}
\item In a certain lottery 10,000 tickets are sold and ten equal prizes are awarded. What is the probability of not getting a prize if you buy (a) one ticket (b) two tickets (c) 10 tickets ?	
\\
\solution
		%\input{ncert/11/16/4/4/defs.tex}
		%
\item 
Out of 100 students, two sections of 40 and 60 are formed. If you and your friend are among the 100 students, what is the probability that
\begin{enumerate}
\item you both enter the same section?
\item you both enter the different sections?
\end{enumerate}
\solution
		%\input{ncert/11/16/4/5/defs.tex}
	\item 
The number lock of a suitcase has 4 wheels each labelled with ten digits i.e. from 0 to 9.The lock opens with a sequence of four digits with no repeats.What is the probability of a person getting the right sequence to open the suitcase.
\\
\solution
		%\input{ncert/11/16/4/10/defs.tex}
		%
\item 
Two cards are drawn at random and without replacement from a pack of 52 playing cards. Find the probability that both the cards are black.
\\
\solution
		%\input{ncert/12/13/2/2/defs.tex}
		\item A box of oranges is inspected by examining three randomly selected oranges drawn without replacement. If all the three oranges are good, the box is approved for sale, otherwise, it is rejected. Find the probability that a box containing 15 oranges out of which 12 are good and 3 are bad ones will be approved for sale.
		\label{ncert/12/13/2/3/defs.tex}
		\item Two balls are drawn at random with replacement from a box containing 10 black and 8 red balls. Find the probability that
		\label{ncert/12/13/2/12}
\begin{enumerate}
\item both balls are red.
\item first ball is black and second is red.
\item one of them is black and other is red.
\end{enumerate}

\item In a hostel, 60\% of the students read Hindi newspaper, 40\% read English newspaper and 20\% read both Hindi and English newspapers. A student is selected at random.
		\label{ncert/12/13/2/15}
\begin{enumerate}
\item Find the probability that she reads neither Hindi nor English newspapers.
\item If she reads Hindi newspaper, find the probability that she reads English newspaper.
\item If she reads English newspaper, find the probability that she reads Hindi newspaper.\\
\end{enumerate}
\item The probability of obtaining an even prime number on each die, when a pair of dice is rolled is 
\begin{enumerate}
    \item $0$ 
    
    \item $\frac{1}{3}$ 
    
    \item $\frac{1}{12}$ 
    
    \item $\frac{1}{36}$ 
\end{enumerate}
\solution
		%\input{ncert/12/13/2/17/defs.tex}
	\item A bag contains 4 red and 4 black balls, another bag contains 2 red and 6 black balls. One of the two bags is selected at random and a ball is drawn from the bag which is found to be red. Find the probability that the ball is drawn from the first bag.
\\
\solution
		%\input{ncert/12/13/3/2/main.tex}
  \item
  Cards with numbers 2 to 101 are placed in a box. A card is selected at random.Find the probability that the card has
\begin{enumerate}[label=(\roman*)]
	\item an even number 
	\item a square number
\end{enumerate}
\solution
%\input{exemplar/10/13/3/32/main.tex}
\item
The king, queen and jack of clubs are removed from a deck of 52 playing cards and then well shuffled. Now one card is drawn at random from the remaining cards.  Determine the probability that the card is
\begin{enumerate}[label=(\roman*)]
\item a club
\item 10 of hearts
\end{enumerate}
\solution
%\input{exemplar/10/13/3/29/main.tex}
\item A team of medical students doing their internship have to assist during surgeries
at a city hospital. The probabilities of surgeries rated as very complex, complex,
routine, simple or very simple are respectively, 0.15, 0.20, 0.31, 0.26, .08. Find
the probabilities that a particular surgery will be rated
\begin{enumerate}
	\item complex or very complex;
	\item neither very complex nor very simple;
	\item routine or complex
	\item routine or simple
\end{enumerate}
\solution
%\input{exemplar/11/16/3/8(1)/main.tex}
\item A card is selected from a pack of 52 cards.
\begin{enumerate}[label=(\alph*)]
    \item How many points are there in the sample space?
    \item Calculate the probability that the card is an ace of spades.
    \item Calculate the probability that the card is (i) an ace and (ii) black card.
\end{enumerate}
\solution
%\input{exemplar/11/16/3/4/main2.tex}
\item The probability that a non leap year selected at random will contain 53 sundays.
\\
\solution
%\input{exemplar/10/13/1/19/main.tex}
\item One of the four persons John, Rita, Aslam or Gurpreet will be promoted next
month. Consequently the sample space consists of four elementary outcomes
S = {John promoted, Rita promoted, Aslam promoted, Gurpreet promoted}
You are told that the chances of John’s promotion is same as that of Gurpreet,
Rita’s chances of promotion are twice as likely as Johns. Aslam’s chances are
four times that of John.
\begin{enumerate}
	\item Determine
	\begin{enumerate}
		\item P (John promoted)
		\item P (Rita promoted)
		\item P (Aslam promoted)
		\item P (Gurpreet promoted)
	\end{enumerate}
	\item If A = {John promoted or Gurpreet promoted}, find P (A).
\end{enumerate}
\solution
%\input{exemplar/11/16/3/10/main.tex}
\item A card is drawn from a deck of 52 cards. Find the probability of getting a king or a heart or a red card.\\
\solution
%\input{exemplar/11/16/3/15/main.tex}
\item The probability that a student will pass his examination is 0.73, the probability of
the student getting a compartment is 0.13, and the probability that the student will
either pass or get compartment is 0.96. State True or False.\\
\solution
%\input{exemplar/11/16/3/31/main.tex}
\item A card is selected from a pack of 52 cards\\
\begin{enumerate}[label=(\alph*)]
\item How many points are there in the sample space?
\item Calculate the probability that the cards is an ace of spades.
\item Calculate the probability that the card is (i) an ace (ii)black card.\\
\end{enumerate}
%\input{ncert/11/16/3/4_1/Prob_4.tex}
\item In a non-leap year, the probability of having 53 tuesdays or 53 wednesdays is\\
\solution
%\input{exemplar/11/16/3/18/main.tex}
\item There are 1000 sealed envelopes in a box, 10 of them contain a cash prize of
Rs 100 each, 100 of them contain a cash prize of Rs 50 each and 200 of them
contain a cash prize of Rs 10 each and rest do not contain any cash prize. If they
are well shuffled and an envelope is picked up out, what is the probability that it
contains no cash prize?\\
\solution
%\input{exemplar/10/13/3/34/main.tex}
\item 
A die is thrown and a card is selected at random from a deck of 52 playing cards. The probability of getting an even number on the die and a spade card.\\
\solution
%\input{exemplar/12/13/3/78/main.tex}
\item
If 4-digit numbers greater than 5,000 are randomly formed from the digits 0, 1, 3, 5, and 7, what is the probability of forming a number divisible by 5 when:
\begin{enumerate}
    \item The digits are repeated?
    \item The repetition of digits is not allowed?
\end{enumerate}
\solution
%\input{ncert/11/16/4/9/main.tex}
\item Consider the probability space $\brak{\Omega, \mathcal{G}, P}$ where $\Omega = [0,2]$ and $\mathcal{G} = \cbrak{\phi, \Omega, [0,1], (1,2]}$. Let $X$ and $Y$ be two functions on $\Omega$ defined as
\begin{align*}
    X(\omega) = 
    \begin{cases}
        1 & \text{if }\omega \in [0, 1]\\
        2 & \text{if }\omega \in (1, 2]
    \end{cases}
\end{align*}
and
\begin{align*}
    Y(\omega) = 
    \begin{cases}
        2 & \text{if }\omega \in [0, 1.5]\\
        3 & \text{if }\omega \in (1.5, 2].
    \end{cases}
\end{align*}
Then which one of the following statements is true?
\begin{enumerate}
    \item [(A)] $X$ is a random variable with respect to $\mathcal{G}$, but $Y$ is not a random variable with respect to $\mathcal{G}$.
    \item [(B)] $Y$ is a random variable with respect to $\mathcal{G}$, but $X$ is not a random variable with respect to $\mathcal{G}$.
    \item [(C)] Neither $X$ nor $Y$ is a random variable with respect to $\mathcal{G}$.
    \item [(D)] Both $X$ and $Y$ are random variables with respect to $\mathcal{G}$.
\end{enumerate} \hfill (GATE ST 2023)\\
\solution
%\input{gate/ST/2023/14/main.tex}
	\item  A die is loaded in such a way that each odd number is twice as likely to occur as
each even number. Find $P(G)$, where $G$ is the event that a number greater than
3 occurs on a single roll of the die.
\\
\solution
		%\input{exemplar/11/16/3/5/main.tex}
	\item All the jacks, queens and kings are removed from a deck of 52 playing cards. The remaining cards are well shuffled and then one card is drawn at random. Giving ace a value 1 similar value for other cards, find the probability that the card has a value 
		\begin{enumerate}
			\item 7
			\item greater than 7
			\item less than 7
		\end{enumerate}
		%\input{exemplar/10/13/3/30/main.tex}
  \item A Lot consists of 48 mobile phones of which 42 are good, 3 have only minor defects and 3 have major defects.Varnika will buy a phone if it is good but the trader will only buy a mobile if it has no major defects. One phone is selected at random from the lot. What is the probability that it is
\begin{enumerate}
	\item acceptable to Varnika?
            \item acceptable to the trader?
\end{enumerate}
\solution
	%\input{exemplar/10/13/3/40/main.tex}
 \item A student says that if you throw a die, it will show up 1 or not 1. Therefore, the probability of getting 1 and the probability of getting 'not 1' each is equal to $\frac{1}{2}$. Is this correct? Give reasons.\\
 \solution
        %\input{exemplar/10/13/2/9/main.tex}
   \item Four candidates A, B, C, D have ap-
plied for the assignment to coach a school cricket
team. If A is twice as likely to be selected as B, and
B and C are given about the same chance of being
selected, while C is twice as likely to be selected
as D, what are the probabilities that
\begin{enumerate}
\item C will be selected?
\item A will not be selected?
\end{enumerate}
	%\input{exemplar/11/16/3/9/main.tex}
 \item A bag contain 24 balls of which $x$ balls are red, $2x$ are white and $3x$ are blue. A ball is selected at random, What is the probability that it is
\begin{enumerate}[label=\alph*)]
\item not red ?
\item white ?
\end{enumerate}
%\input{exemplar/10/13/3/41/main.tex}
If the letters of the word ASSASSINATION are arranged at random. Find the Probability that
\begin{enumerate}[label=(\alph*)]
\item Four $S's$ come consecutively in the word
\item Two  $I's$ and two $N's$ come together
\item All $A's$ are not coming together
\item No two $A's$ are coming together
\end{enumerate}
%\input{exemplar/11/16/3/14/main.tex}
	\item One urn contains two black balls (labelled B1 and B2) and one white ball. A
	second urn contains one black ball and two white balls (labelled W1 and W2).
	Suppose the following experiment is performed. One of the two urns is chosen
	at random. Next a ball is randomly chosen from the urn. Then a second ball is
	chosen at random from the same urn without replacing the first ball.
	
	\begin{enumerate}
	\item What is the probability that two black balls are chosen?
	
	\item What is the probability that two balls of opposite colour are chosen?
	\end{enumerate}
	\solution
	%\input{exemplar/11/16/3/12/main1.tex}
\end{enumerate}

		%
\item 
Two cards are drawn at random and without replacement from a pack of 52 playing cards. Find the probability that both the cards are black.
\\
\solution
		%\begin{enumerate}[label=\thesection.\arabic*,ref=\thesection.\theenumi]
	\item One card is drawn from a well-shuffled deck of 52 cards. Find the probability of getting
\begin{enumerate}
\item A king of red colour 
\item A face card 
\item A red face card
\item The jack of hearts
\item A spade
\item The queen of diamonds

\end{enumerate}
\solution
		%\input{ncert/10/15/1/14/main.tex}
	\item Five cards—the ten, jack, queen, king and ace of diamonds, are well-shuffled with their face downwards. One card is then picked up at random.
\begin{enumerate}
\item
What is the probability that the card is the queen? 
\item
If the queen is drawn and put aside, what is the probability that the second card picked up is (a) an ace? (b) a queen?\\
\end{enumerate}
\solution
		%\input{ncert/10/15/1/15/defs.tex}
	\item A bag contains $5$ red balls and some blue balls. If the probability of drawing a blue ball is double that if a red ball, determine the number of blue balls in the bag. 
		\\
\solution
		%\input{ncert/10/15/2/3/defs.tex}
	\item A card is selected from a pack of 52 cards.
 \begin{enumerate}[label=(\alph*)] 
                 \item How many points are there in the sample space?
                 \item Calculate the probability that the card is an ace of spades.
                 \item Calculate the probability that the card is (i) an ace and (ii) black card.
 \end{enumerate}
\solution
		%\input{ncert/11/16/3/4/main.tex}
\item Four cards are drawn from a well-shuffled deck of 52 cards. What is the probability of obtaining 3 diamonds and one spade.
\\
\solution
		%\input{ncert/11/16/4/2/defs.tex}
\item In a certain lottery 10,000 tickets are sold and ten equal prizes are awarded. What is the probability of not getting a prize if you buy (a) one ticket (b) two tickets (c) 10 tickets ?	
\\
\solution
		%\input{ncert/11/16/4/4/defs.tex}
		%
\item 
Out of 100 students, two sections of 40 and 60 are formed. If you and your friend are among the 100 students, what is the probability that
\begin{enumerate}
\item you both enter the same section?
\item you both enter the different sections?
\end{enumerate}
\solution
		%\input{ncert/11/16/4/5/defs.tex}
	\item 
The number lock of a suitcase has 4 wheels each labelled with ten digits i.e. from 0 to 9.The lock opens with a sequence of four digits with no repeats.What is the probability of a person getting the right sequence to open the suitcase.
\\
\solution
		%\input{ncert/11/16/4/10/defs.tex}
		%
\item 
Two cards are drawn at random and without replacement from a pack of 52 playing cards. Find the probability that both the cards are black.
\\
\solution
		%\input{ncert/12/13/2/2/defs.tex}
		\item A box of oranges is inspected by examining three randomly selected oranges drawn without replacement. If all the three oranges are good, the box is approved for sale, otherwise, it is rejected. Find the probability that a box containing 15 oranges out of which 12 are good and 3 are bad ones will be approved for sale.
		\label{ncert/12/13/2/3/defs.tex}
		\item Two balls are drawn at random with replacement from a box containing 10 black and 8 red balls. Find the probability that
		\label{ncert/12/13/2/12}
\begin{enumerate}
\item both balls are red.
\item first ball is black and second is red.
\item one of them is black and other is red.
\end{enumerate}

\item In a hostel, 60\% of the students read Hindi newspaper, 40\% read English newspaper and 20\% read both Hindi and English newspapers. A student is selected at random.
		\label{ncert/12/13/2/15}
\begin{enumerate}
\item Find the probability that she reads neither Hindi nor English newspapers.
\item If she reads Hindi newspaper, find the probability that she reads English newspaper.
\item If she reads English newspaper, find the probability that she reads Hindi newspaper.\\
\end{enumerate}
\item The probability of obtaining an even prime number on each die, when a pair of dice is rolled is 
\begin{enumerate}
    \item $0$ 
    
    \item $\frac{1}{3}$ 
    
    \item $\frac{1}{12}$ 
    
    \item $\frac{1}{36}$ 
\end{enumerate}
\solution
		%\input{ncert/12/13/2/17/defs.tex}
	\item A bag contains 4 red and 4 black balls, another bag contains 2 red and 6 black balls. One of the two bags is selected at random and a ball is drawn from the bag which is found to be red. Find the probability that the ball is drawn from the first bag.
\\
\solution
		%\input{ncert/12/13/3/2/main.tex}
  \item
  Cards with numbers 2 to 101 are placed in a box. A card is selected at random.Find the probability that the card has
\begin{enumerate}[label=(\roman*)]
	\item an even number 
	\item a square number
\end{enumerate}
\solution
%\input{exemplar/10/13/3/32/main.tex}
\item
The king, queen and jack of clubs are removed from a deck of 52 playing cards and then well shuffled. Now one card is drawn at random from the remaining cards.  Determine the probability that the card is
\begin{enumerate}[label=(\roman*)]
\item a club
\item 10 of hearts
\end{enumerate}
\solution
%\input{exemplar/10/13/3/29/main.tex}
\item A team of medical students doing their internship have to assist during surgeries
at a city hospital. The probabilities of surgeries rated as very complex, complex,
routine, simple or very simple are respectively, 0.15, 0.20, 0.31, 0.26, .08. Find
the probabilities that a particular surgery will be rated
\begin{enumerate}
	\item complex or very complex;
	\item neither very complex nor very simple;
	\item routine or complex
	\item routine or simple
\end{enumerate}
\solution
%\input{exemplar/11/16/3/8(1)/main.tex}
\item A card is selected from a pack of 52 cards.
\begin{enumerate}[label=(\alph*)]
    \item How many points are there in the sample space?
    \item Calculate the probability that the card is an ace of spades.
    \item Calculate the probability that the card is (i) an ace and (ii) black card.
\end{enumerate}
\solution
%\input{exemplar/11/16/3/4/main2.tex}
\item The probability that a non leap year selected at random will contain 53 sundays.
\\
\solution
%\input{exemplar/10/13/1/19/main.tex}
\item One of the four persons John, Rita, Aslam or Gurpreet will be promoted next
month. Consequently the sample space consists of four elementary outcomes
S = {John promoted, Rita promoted, Aslam promoted, Gurpreet promoted}
You are told that the chances of John’s promotion is same as that of Gurpreet,
Rita’s chances of promotion are twice as likely as Johns. Aslam’s chances are
four times that of John.
\begin{enumerate}
	\item Determine
	\begin{enumerate}
		\item P (John promoted)
		\item P (Rita promoted)
		\item P (Aslam promoted)
		\item P (Gurpreet promoted)
	\end{enumerate}
	\item If A = {John promoted or Gurpreet promoted}, find P (A).
\end{enumerate}
\solution
%\input{exemplar/11/16/3/10/main.tex}
\item A card is drawn from a deck of 52 cards. Find the probability of getting a king or a heart or a red card.\\
\solution
%\input{exemplar/11/16/3/15/main.tex}
\item The probability that a student will pass his examination is 0.73, the probability of
the student getting a compartment is 0.13, and the probability that the student will
either pass or get compartment is 0.96. State True or False.\\
\solution
%\input{exemplar/11/16/3/31/main.tex}
\item A card is selected from a pack of 52 cards\\
\begin{enumerate}[label=(\alph*)]
\item How many points are there in the sample space?
\item Calculate the probability that the cards is an ace of spades.
\item Calculate the probability that the card is (i) an ace (ii)black card.\\
\end{enumerate}
%\input{ncert/11/16/3/4_1/Prob_4.tex}
\item In a non-leap year, the probability of having 53 tuesdays or 53 wednesdays is\\
\solution
%\input{exemplar/11/16/3/18/main.tex}
\item There are 1000 sealed envelopes in a box, 10 of them contain a cash prize of
Rs 100 each, 100 of them contain a cash prize of Rs 50 each and 200 of them
contain a cash prize of Rs 10 each and rest do not contain any cash prize. If they
are well shuffled and an envelope is picked up out, what is the probability that it
contains no cash prize?\\
\solution
%\input{exemplar/10/13/3/34/main.tex}
\item 
A die is thrown and a card is selected at random from a deck of 52 playing cards. The probability of getting an even number on the die and a spade card.\\
\solution
%\input{exemplar/12/13/3/78/main.tex}
\item
If 4-digit numbers greater than 5,000 are randomly formed from the digits 0, 1, 3, 5, and 7, what is the probability of forming a number divisible by 5 when:
\begin{enumerate}
    \item The digits are repeated?
    \item The repetition of digits is not allowed?
\end{enumerate}
\solution
%\input{ncert/11/16/4/9/main.tex}
\item Consider the probability space $\brak{\Omega, \mathcal{G}, P}$ where $\Omega = [0,2]$ and $\mathcal{G} = \cbrak{\phi, \Omega, [0,1], (1,2]}$. Let $X$ and $Y$ be two functions on $\Omega$ defined as
\begin{align*}
    X(\omega) = 
    \begin{cases}
        1 & \text{if }\omega \in [0, 1]\\
        2 & \text{if }\omega \in (1, 2]
    \end{cases}
\end{align*}
and
\begin{align*}
    Y(\omega) = 
    \begin{cases}
        2 & \text{if }\omega \in [0, 1.5]\\
        3 & \text{if }\omega \in (1.5, 2].
    \end{cases}
\end{align*}
Then which one of the following statements is true?
\begin{enumerate}
    \item [(A)] $X$ is a random variable with respect to $\mathcal{G}$, but $Y$ is not a random variable with respect to $\mathcal{G}$.
    \item [(B)] $Y$ is a random variable with respect to $\mathcal{G}$, but $X$ is not a random variable with respect to $\mathcal{G}$.
    \item [(C)] Neither $X$ nor $Y$ is a random variable with respect to $\mathcal{G}$.
    \item [(D)] Both $X$ and $Y$ are random variables with respect to $\mathcal{G}$.
\end{enumerate} \hfill (GATE ST 2023)\\
\solution
%\input{gate/ST/2023/14/main.tex}
	\item  A die is loaded in such a way that each odd number is twice as likely to occur as
each even number. Find $P(G)$, where $G$ is the event that a number greater than
3 occurs on a single roll of the die.
\\
\solution
		%\input{exemplar/11/16/3/5/main.tex}
	\item All the jacks, queens and kings are removed from a deck of 52 playing cards. The remaining cards are well shuffled and then one card is drawn at random. Giving ace a value 1 similar value for other cards, find the probability that the card has a value 
		\begin{enumerate}
			\item 7
			\item greater than 7
			\item less than 7
		\end{enumerate}
		%\input{exemplar/10/13/3/30/main.tex}
  \item A Lot consists of 48 mobile phones of which 42 are good, 3 have only minor defects and 3 have major defects.Varnika will buy a phone if it is good but the trader will only buy a mobile if it has no major defects. One phone is selected at random from the lot. What is the probability that it is
\begin{enumerate}
	\item acceptable to Varnika?
            \item acceptable to the trader?
\end{enumerate}
\solution
	%\input{exemplar/10/13/3/40/main.tex}
 \item A student says that if you throw a die, it will show up 1 or not 1. Therefore, the probability of getting 1 and the probability of getting 'not 1' each is equal to $\frac{1}{2}$. Is this correct? Give reasons.\\
 \solution
        %\input{exemplar/10/13/2/9/main.tex}
   \item Four candidates A, B, C, D have ap-
plied for the assignment to coach a school cricket
team. If A is twice as likely to be selected as B, and
B and C are given about the same chance of being
selected, while C is twice as likely to be selected
as D, what are the probabilities that
\begin{enumerate}
\item C will be selected?
\item A will not be selected?
\end{enumerate}
	%\input{exemplar/11/16/3/9/main.tex}
 \item A bag contain 24 balls of which $x$ balls are red, $2x$ are white and $3x$ are blue. A ball is selected at random, What is the probability that it is
\begin{enumerate}[label=\alph*)]
\item not red ?
\item white ?
\end{enumerate}
%\input{exemplar/10/13/3/41/main.tex}
If the letters of the word ASSASSINATION are arranged at random. Find the Probability that
\begin{enumerate}[label=(\alph*)]
\item Four $S's$ come consecutively in the word
\item Two  $I's$ and two $N's$ come together
\item All $A's$ are not coming together
\item No two $A's$ are coming together
\end{enumerate}
%\input{exemplar/11/16/3/14/main.tex}
	\item One urn contains two black balls (labelled B1 and B2) and one white ball. A
	second urn contains one black ball and two white balls (labelled W1 and W2).
	Suppose the following experiment is performed. One of the two urns is chosen
	at random. Next a ball is randomly chosen from the urn. Then a second ball is
	chosen at random from the same urn without replacing the first ball.
	
	\begin{enumerate}
	\item What is the probability that two black balls are chosen?
	
	\item What is the probability that two balls of opposite colour are chosen?
	\end{enumerate}
	\solution
	%\input{exemplar/11/16/3/12/main1.tex}
\end{enumerate}

		\item A box of oranges is inspected by examining three randomly selected oranges drawn without replacement. If all the three oranges are good, the box is approved for sale, otherwise, it is rejected. Find the probability that a box containing 15 oranges out of which 12 are good and 3 are bad ones will be approved for sale.
		\label{ncert/12/13/2/3/defs.tex}
		\item Two balls are drawn at random with replacement from a box containing 10 black and 8 red balls. Find the probability that
		\label{ncert/12/13/2/12}
\begin{enumerate}
\item both balls are red.
\item first ball is black and second is red.
\item one of them is black and other is red.
\end{enumerate}

\item In a hostel, 60\% of the students read Hindi newspaper, 40\% read English newspaper and 20\% read both Hindi and English newspapers. A student is selected at random.
		\label{ncert/12/13/2/15}
\begin{enumerate}
\item Find the probability that she reads neither Hindi nor English newspapers.
\item If she reads Hindi newspaper, find the probability that she reads English newspaper.
\item If she reads English newspaper, find the probability that she reads Hindi newspaper.\\
\end{enumerate}
\item The probability of obtaining an even prime number on each die, when a pair of dice is rolled is 
\begin{enumerate}
    \item $0$ 
    
    \item $\frac{1}{3}$ 
    
    \item $\frac{1}{12}$ 
    
    \item $\frac{1}{36}$ 
\end{enumerate}
\solution
		%\begin{enumerate}[label=\thesection.\arabic*,ref=\thesection.\theenumi]
	\item One card is drawn from a well-shuffled deck of 52 cards. Find the probability of getting
\begin{enumerate}
\item A king of red colour 
\item A face card 
\item A red face card
\item The jack of hearts
\item A spade
\item The queen of diamonds

\end{enumerate}
\solution
		%\input{ncert/10/15/1/14/main.tex}
	\item Five cards—the ten, jack, queen, king and ace of diamonds, are well-shuffled with their face downwards. One card is then picked up at random.
\begin{enumerate}
\item
What is the probability that the card is the queen? 
\item
If the queen is drawn and put aside, what is the probability that the second card picked up is (a) an ace? (b) a queen?\\
\end{enumerate}
\solution
		%\input{ncert/10/15/1/15/defs.tex}
	\item A bag contains $5$ red balls and some blue balls. If the probability of drawing a blue ball is double that if a red ball, determine the number of blue balls in the bag. 
		\\
\solution
		%\input{ncert/10/15/2/3/defs.tex}
	\item A card is selected from a pack of 52 cards.
 \begin{enumerate}[label=(\alph*)] 
                 \item How many points are there in the sample space?
                 \item Calculate the probability that the card is an ace of spades.
                 \item Calculate the probability that the card is (i) an ace and (ii) black card.
 \end{enumerate}
\solution
		%\input{ncert/11/16/3/4/main.tex}
\item Four cards are drawn from a well-shuffled deck of 52 cards. What is the probability of obtaining 3 diamonds and one spade.
\\
\solution
		%\input{ncert/11/16/4/2/defs.tex}
\item In a certain lottery 10,000 tickets are sold and ten equal prizes are awarded. What is the probability of not getting a prize if you buy (a) one ticket (b) two tickets (c) 10 tickets ?	
\\
\solution
		%\input{ncert/11/16/4/4/defs.tex}
		%
\item 
Out of 100 students, two sections of 40 and 60 are formed. If you and your friend are among the 100 students, what is the probability that
\begin{enumerate}
\item you both enter the same section?
\item you both enter the different sections?
\end{enumerate}
\solution
		%\input{ncert/11/16/4/5/defs.tex}
	\item 
The number lock of a suitcase has 4 wheels each labelled with ten digits i.e. from 0 to 9.The lock opens with a sequence of four digits with no repeats.What is the probability of a person getting the right sequence to open the suitcase.
\\
\solution
		%\input{ncert/11/16/4/10/defs.tex}
		%
\item 
Two cards are drawn at random and without replacement from a pack of 52 playing cards. Find the probability that both the cards are black.
\\
\solution
		%\input{ncert/12/13/2/2/defs.tex}
		\item A box of oranges is inspected by examining three randomly selected oranges drawn without replacement. If all the three oranges are good, the box is approved for sale, otherwise, it is rejected. Find the probability that a box containing 15 oranges out of which 12 are good and 3 are bad ones will be approved for sale.
		\label{ncert/12/13/2/3/defs.tex}
		\item Two balls are drawn at random with replacement from a box containing 10 black and 8 red balls. Find the probability that
		\label{ncert/12/13/2/12}
\begin{enumerate}
\item both balls are red.
\item first ball is black and second is red.
\item one of them is black and other is red.
\end{enumerate}

\item In a hostel, 60\% of the students read Hindi newspaper, 40\% read English newspaper and 20\% read both Hindi and English newspapers. A student is selected at random.
		\label{ncert/12/13/2/15}
\begin{enumerate}
\item Find the probability that she reads neither Hindi nor English newspapers.
\item If she reads Hindi newspaper, find the probability that she reads English newspaper.
\item If she reads English newspaper, find the probability that she reads Hindi newspaper.\\
\end{enumerate}
\item The probability of obtaining an even prime number on each die, when a pair of dice is rolled is 
\begin{enumerate}
    \item $0$ 
    
    \item $\frac{1}{3}$ 
    
    \item $\frac{1}{12}$ 
    
    \item $\frac{1}{36}$ 
\end{enumerate}
\solution
		%\input{ncert/12/13/2/17/defs.tex}
	\item A bag contains 4 red and 4 black balls, another bag contains 2 red and 6 black balls. One of the two bags is selected at random and a ball is drawn from the bag which is found to be red. Find the probability that the ball is drawn from the first bag.
\\
\solution
		%\input{ncert/12/13/3/2/main.tex}
  \item
  Cards with numbers 2 to 101 are placed in a box. A card is selected at random.Find the probability that the card has
\begin{enumerate}[label=(\roman*)]
	\item an even number 
	\item a square number
\end{enumerate}
\solution
%\input{exemplar/10/13/3/32/main.tex}
\item
The king, queen and jack of clubs are removed from a deck of 52 playing cards and then well shuffled. Now one card is drawn at random from the remaining cards.  Determine the probability that the card is
\begin{enumerate}[label=(\roman*)]
\item a club
\item 10 of hearts
\end{enumerate}
\solution
%\input{exemplar/10/13/3/29/main.tex}
\item A team of medical students doing their internship have to assist during surgeries
at a city hospital. The probabilities of surgeries rated as very complex, complex,
routine, simple or very simple are respectively, 0.15, 0.20, 0.31, 0.26, .08. Find
the probabilities that a particular surgery will be rated
\begin{enumerate}
	\item complex or very complex;
	\item neither very complex nor very simple;
	\item routine or complex
	\item routine or simple
\end{enumerate}
\solution
%\input{exemplar/11/16/3/8(1)/main.tex}
\item A card is selected from a pack of 52 cards.
\begin{enumerate}[label=(\alph*)]
    \item How many points are there in the sample space?
    \item Calculate the probability that the card is an ace of spades.
    \item Calculate the probability that the card is (i) an ace and (ii) black card.
\end{enumerate}
\solution
%\input{exemplar/11/16/3/4/main2.tex}
\item The probability that a non leap year selected at random will contain 53 sundays.
\\
\solution
%\input{exemplar/10/13/1/19/main.tex}
\item One of the four persons John, Rita, Aslam or Gurpreet will be promoted next
month. Consequently the sample space consists of four elementary outcomes
S = {John promoted, Rita promoted, Aslam promoted, Gurpreet promoted}
You are told that the chances of John’s promotion is same as that of Gurpreet,
Rita’s chances of promotion are twice as likely as Johns. Aslam’s chances are
four times that of John.
\begin{enumerate}
	\item Determine
	\begin{enumerate}
		\item P (John promoted)
		\item P (Rita promoted)
		\item P (Aslam promoted)
		\item P (Gurpreet promoted)
	\end{enumerate}
	\item If A = {John promoted or Gurpreet promoted}, find P (A).
\end{enumerate}
\solution
%\input{exemplar/11/16/3/10/main.tex}
\item A card is drawn from a deck of 52 cards. Find the probability of getting a king or a heart or a red card.\\
\solution
%\input{exemplar/11/16/3/15/main.tex}
\item The probability that a student will pass his examination is 0.73, the probability of
the student getting a compartment is 0.13, and the probability that the student will
either pass or get compartment is 0.96. State True or False.\\
\solution
%\input{exemplar/11/16/3/31/main.tex}
\item A card is selected from a pack of 52 cards\\
\begin{enumerate}[label=(\alph*)]
\item How many points are there in the sample space?
\item Calculate the probability that the cards is an ace of spades.
\item Calculate the probability that the card is (i) an ace (ii)black card.\\
\end{enumerate}
%\input{ncert/11/16/3/4_1/Prob_4.tex}
\item In a non-leap year, the probability of having 53 tuesdays or 53 wednesdays is\\
\solution
%\input{exemplar/11/16/3/18/main.tex}
\item There are 1000 sealed envelopes in a box, 10 of them contain a cash prize of
Rs 100 each, 100 of them contain a cash prize of Rs 50 each and 200 of them
contain a cash prize of Rs 10 each and rest do not contain any cash prize. If they
are well shuffled and an envelope is picked up out, what is the probability that it
contains no cash prize?\\
\solution
%\input{exemplar/10/13/3/34/main.tex}
\item 
A die is thrown and a card is selected at random from a deck of 52 playing cards. The probability of getting an even number on the die and a spade card.\\
\solution
%\input{exemplar/12/13/3/78/main.tex}
\item
If 4-digit numbers greater than 5,000 are randomly formed from the digits 0, 1, 3, 5, and 7, what is the probability of forming a number divisible by 5 when:
\begin{enumerate}
    \item The digits are repeated?
    \item The repetition of digits is not allowed?
\end{enumerate}
\solution
%\input{ncert/11/16/4/9/main.tex}
\item Consider the probability space $\brak{\Omega, \mathcal{G}, P}$ where $\Omega = [0,2]$ and $\mathcal{G} = \cbrak{\phi, \Omega, [0,1], (1,2]}$. Let $X$ and $Y$ be two functions on $\Omega$ defined as
\begin{align*}
    X(\omega) = 
    \begin{cases}
        1 & \text{if }\omega \in [0, 1]\\
        2 & \text{if }\omega \in (1, 2]
    \end{cases}
\end{align*}
and
\begin{align*}
    Y(\omega) = 
    \begin{cases}
        2 & \text{if }\omega \in [0, 1.5]\\
        3 & \text{if }\omega \in (1.5, 2].
    \end{cases}
\end{align*}
Then which one of the following statements is true?
\begin{enumerate}
    \item [(A)] $X$ is a random variable with respect to $\mathcal{G}$, but $Y$ is not a random variable with respect to $\mathcal{G}$.
    \item [(B)] $Y$ is a random variable with respect to $\mathcal{G}$, but $X$ is not a random variable with respect to $\mathcal{G}$.
    \item [(C)] Neither $X$ nor $Y$ is a random variable with respect to $\mathcal{G}$.
    \item [(D)] Both $X$ and $Y$ are random variables with respect to $\mathcal{G}$.
\end{enumerate} \hfill (GATE ST 2023)\\
\solution
%\input{gate/ST/2023/14/main.tex}
	\item  A die is loaded in such a way that each odd number is twice as likely to occur as
each even number. Find $P(G)$, where $G$ is the event that a number greater than
3 occurs on a single roll of the die.
\\
\solution
		%\input{exemplar/11/16/3/5/main.tex}
	\item All the jacks, queens and kings are removed from a deck of 52 playing cards. The remaining cards are well shuffled and then one card is drawn at random. Giving ace a value 1 similar value for other cards, find the probability that the card has a value 
		\begin{enumerate}
			\item 7
			\item greater than 7
			\item less than 7
		\end{enumerate}
		%\input{exemplar/10/13/3/30/main.tex}
  \item A Lot consists of 48 mobile phones of which 42 are good, 3 have only minor defects and 3 have major defects.Varnika will buy a phone if it is good but the trader will only buy a mobile if it has no major defects. One phone is selected at random from the lot. What is the probability that it is
\begin{enumerate}
	\item acceptable to Varnika?
            \item acceptable to the trader?
\end{enumerate}
\solution
	%\input{exemplar/10/13/3/40/main.tex}
 \item A student says that if you throw a die, it will show up 1 or not 1. Therefore, the probability of getting 1 and the probability of getting 'not 1' each is equal to $\frac{1}{2}$. Is this correct? Give reasons.\\
 \solution
        %\input{exemplar/10/13/2/9/main.tex}
   \item Four candidates A, B, C, D have ap-
plied for the assignment to coach a school cricket
team. If A is twice as likely to be selected as B, and
B and C are given about the same chance of being
selected, while C is twice as likely to be selected
as D, what are the probabilities that
\begin{enumerate}
\item C will be selected?
\item A will not be selected?
\end{enumerate}
	%\input{exemplar/11/16/3/9/main.tex}
 \item A bag contain 24 balls of which $x$ balls are red, $2x$ are white and $3x$ are blue. A ball is selected at random, What is the probability that it is
\begin{enumerate}[label=\alph*)]
\item not red ?
\item white ?
\end{enumerate}
%\input{exemplar/10/13/3/41/main.tex}
If the letters of the word ASSASSINATION are arranged at random. Find the Probability that
\begin{enumerate}[label=(\alph*)]
\item Four $S's$ come consecutively in the word
\item Two  $I's$ and two $N's$ come together
\item All $A's$ are not coming together
\item No two $A's$ are coming together
\end{enumerate}
%\input{exemplar/11/16/3/14/main.tex}
	\item One urn contains two black balls (labelled B1 and B2) and one white ball. A
	second urn contains one black ball and two white balls (labelled W1 and W2).
	Suppose the following experiment is performed. One of the two urns is chosen
	at random. Next a ball is randomly chosen from the urn. Then a second ball is
	chosen at random from the same urn without replacing the first ball.
	
	\begin{enumerate}
	\item What is the probability that two black balls are chosen?
	
	\item What is the probability that two balls of opposite colour are chosen?
	\end{enumerate}
	\solution
	%\input{exemplar/11/16/3/12/main1.tex}
\end{enumerate}

	\item A bag contains 4 red and 4 black balls, another bag contains 2 red and 6 black balls. One of the two bags is selected at random and a ball is drawn from the bag which is found to be red. Find the probability that the ball is drawn from the first bag.
\\
\solution
		%\begin{table}[H]
	\centering
\begin{tabular}{|c|c|c|}
\hline
Random variable &Value &Definition\\ \hline
\multirow{3}{*}{X} &0 &Slips of Rs 1\\
&1 &Slips of Rs 5\\
&2 &Slips of Rs 13\\ \hline
\multirow{2}{*}{Y} &0 &Box A\\
&1 &Box B\\\hline
\end{tabular}
\caption{}
\label{tab:Distribution}
\end{table}
See \tabref{tab:Distribution}.
\begin{align}
p_{Y}\brak{k}= \begin{cases} 
      \frac{1}{3} & {k=0} \\
      \frac{2}{3 }& {k=1} 
   \end{cases}
   \\
p_{Y|X}\brak{0|0} = \frac{19}{25}\, 
p_{Y|X}\brak{0|1} = \frac{6}{25}\,
p_{Y|X}\brak{1|0} = \frac{45}{50}\,
p_{Y|X}\brak{1|2} = \frac{5}{50}
\end{align}
The desired probability is the probability that a slip drawn at random is marked other than Rs 1,
\begin{align}
&=1-p_X\brak{0}\\
&= p_X(1) + p_X(2)
\end{align}
Using Bayes theorem,
\begin{align}
&= p_Y\brak{0} \times \pr{Y=0 | X=1} + p_Y\brak{1} \times \pr{Y=1|X=2}\\
&=\frac{1}{3} \times \frac{6}{25} + \frac{2}{3} \times \frac{5}{50}\\
&=\frac{11}{75}
\end{align}

\newpage

%\tableofcontents

\bigskip

\renewcommand{\thefigure}{\theenumi}
\renewcommand{\thetable}{\theenumi}
%\renewcommand{\theequation}{\theenumi}

%\begin{abstract}
%%\boldmath
%In this letter, an algorithm for evaluating the exact analytical bit error rate  (BER)  for the piecewise linear (PL) combiner for  multiple relays is presented. Previous results were available only for upto three relays. The algorithm is unique in the sense that  the actual mathematical expressions, that are prohibitively large, need not be explicitly obtained. The diversity gain due to multiple relays is shown through plots of the analytical BER, well supported by simulations. 
%
%\end{abstract}
% IEEEtran.cls defaults to using nonbold math in the Abstract.
% This preserves the distinction between vectors and scalars. However,
% if the journal you are submitting to favors bold math in the abstract,
% then you can use LaTeX's standard command \boldmath at the very start
% of the abstract to achieve this. Many IEEE journals frown on math
% in the abstract anyway.

% Note that keywords are not normally used for peerreview papers.
%\begin{IEEEkeywords}
%Cooperative diversity, decode and forward, piecewise linear
%\end{IEEEkeywords}



% For peer review papers, you can put extra information on the cover
% page as needed:
% \ifCLASSOPTIONpeerreview
% \begin{center} \bfseries EDICS Category: 3-BBND \end{center}
% \fi
%
% For peerreview papers, this IEEEtran command inserts a page break and
% creates the second title. It will be ignored for other modes.
%\IEEEpeerreviewmaketitle




  \item
  Cards with numbers 2 to 101 are placed in a box. A card is selected at random.Find the probability that the card has
\begin{enumerate}[label=(\roman*)]
	\item an even number 
	\item a square number
\end{enumerate}
\solution
%\begin{table}[H]
	\centering
\begin{tabular}{|c|c|c|}
\hline
Random variable &Value &Definition\\ \hline
\multirow{3}{*}{X} &0 &Slips of Rs 1\\
&1 &Slips of Rs 5\\
&2 &Slips of Rs 13\\ \hline
\multirow{2}{*}{Y} &0 &Box A\\
&1 &Box B\\\hline
\end{tabular}
\caption{}
\label{tab:Distribution}
\end{table}
See \tabref{tab:Distribution}.
\begin{align}
p_{Y}\brak{k}= \begin{cases} 
      \frac{1}{3} & {k=0} \\
      \frac{2}{3 }& {k=1} 
   \end{cases}
   \\
p_{Y|X}\brak{0|0} = \frac{19}{25}\, 
p_{Y|X}\brak{0|1} = \frac{6}{25}\,
p_{Y|X}\brak{1|0} = \frac{45}{50}\,
p_{Y|X}\brak{1|2} = \frac{5}{50}
\end{align}
The desired probability is the probability that a slip drawn at random is marked other than Rs 1,
\begin{align}
&=1-p_X\brak{0}\\
&= p_X(1) + p_X(2)
\end{align}
Using Bayes theorem,
\begin{align}
&= p_Y\brak{0} \times \pr{Y=0 | X=1} + p_Y\brak{1} \times \pr{Y=1|X=2}\\
&=\frac{1}{3} \times \frac{6}{25} + \frac{2}{3} \times \frac{5}{50}\\
&=\frac{11}{75}
\end{align}

\newpage

%\tableofcontents

\bigskip

\renewcommand{\thefigure}{\theenumi}
\renewcommand{\thetable}{\theenumi}
%\renewcommand{\theequation}{\theenumi}

%\begin{abstract}
%%\boldmath
%In this letter, an algorithm for evaluating the exact analytical bit error rate  (BER)  for the piecewise linear (PL) combiner for  multiple relays is presented. Previous results were available only for upto three relays. The algorithm is unique in the sense that  the actual mathematical expressions, that are prohibitively large, need not be explicitly obtained. The diversity gain due to multiple relays is shown through plots of the analytical BER, well supported by simulations. 
%
%\end{abstract}
% IEEEtran.cls defaults to using nonbold math in the Abstract.
% This preserves the distinction between vectors and scalars. However,
% if the journal you are submitting to favors bold math in the abstract,
% then you can use LaTeX's standard command \boldmath at the very start
% of the abstract to achieve this. Many IEEE journals frown on math
% in the abstract anyway.

% Note that keywords are not normally used for peerreview papers.
%\begin{IEEEkeywords}
%Cooperative diversity, decode and forward, piecewise linear
%\end{IEEEkeywords}



% For peer review papers, you can put extra information on the cover
% page as needed:
% \ifCLASSOPTIONpeerreview
% \begin{center} \bfseries EDICS Category: 3-BBND \end{center}
% \fi
%
% For peerreview papers, this IEEEtran command inserts a page break and
% creates the second title. It will be ignored for other modes.
%\IEEEpeerreviewmaketitle




\item
The king, queen and jack of clubs are removed from a deck of 52 playing cards and then well shuffled. Now one card is drawn at random from the remaining cards.  Determine the probability that the card is
\begin{enumerate}[label=(\roman*)]
\item a club
\item 10 of hearts
\end{enumerate}
\solution
%\begin{table}[H]
	\centering
\begin{tabular}{|c|c|c|}
\hline
Random variable &Value &Definition\\ \hline
\multirow{3}{*}{X} &0 &Slips of Rs 1\\
&1 &Slips of Rs 5\\
&2 &Slips of Rs 13\\ \hline
\multirow{2}{*}{Y} &0 &Box A\\
&1 &Box B\\\hline
\end{tabular}
\caption{}
\label{tab:Distribution}
\end{table}
See \tabref{tab:Distribution}.
\begin{align}
p_{Y}\brak{k}= \begin{cases} 
      \frac{1}{3} & {k=0} \\
      \frac{2}{3 }& {k=1} 
   \end{cases}
   \\
p_{Y|X}\brak{0|0} = \frac{19}{25}\, 
p_{Y|X}\brak{0|1} = \frac{6}{25}\,
p_{Y|X}\brak{1|0} = \frac{45}{50}\,
p_{Y|X}\brak{1|2} = \frac{5}{50}
\end{align}
The desired probability is the probability that a slip drawn at random is marked other than Rs 1,
\begin{align}
&=1-p_X\brak{0}\\
&= p_X(1) + p_X(2)
\end{align}
Using Bayes theorem,
\begin{align}
&= p_Y\brak{0} \times \pr{Y=0 | X=1} + p_Y\brak{1} \times \pr{Y=1|X=2}\\
&=\frac{1}{3} \times \frac{6}{25} + \frac{2}{3} \times \frac{5}{50}\\
&=\frac{11}{75}
\end{align}

\newpage

%\tableofcontents

\bigskip

\renewcommand{\thefigure}{\theenumi}
\renewcommand{\thetable}{\theenumi}
%\renewcommand{\theequation}{\theenumi}

%\begin{abstract}
%%\boldmath
%In this letter, an algorithm for evaluating the exact analytical bit error rate  (BER)  for the piecewise linear (PL) combiner for  multiple relays is presented. Previous results were available only for upto three relays. The algorithm is unique in the sense that  the actual mathematical expressions, that are prohibitively large, need not be explicitly obtained. The diversity gain due to multiple relays is shown through plots of the analytical BER, well supported by simulations. 
%
%\end{abstract}
% IEEEtran.cls defaults to using nonbold math in the Abstract.
% This preserves the distinction between vectors and scalars. However,
% if the journal you are submitting to favors bold math in the abstract,
% then you can use LaTeX's standard command \boldmath at the very start
% of the abstract to achieve this. Many IEEE journals frown on math
% in the abstract anyway.

% Note that keywords are not normally used for peerreview papers.
%\begin{IEEEkeywords}
%Cooperative diversity, decode and forward, piecewise linear
%\end{IEEEkeywords}



% For peer review papers, you can put extra information on the cover
% page as needed:
% \ifCLASSOPTIONpeerreview
% \begin{center} \bfseries EDICS Category: 3-BBND \end{center}
% \fi
%
% For peerreview papers, this IEEEtran command inserts a page break and
% creates the second title. It will be ignored for other modes.
%\IEEEpeerreviewmaketitle




\item A team of medical students doing their internship have to assist during surgeries
at a city hospital. The probabilities of surgeries rated as very complex, complex,
routine, simple or very simple are respectively, 0.15, 0.20, 0.31, 0.26, .08. Find
the probabilities that a particular surgery will be rated
\begin{enumerate}
	\item complex or very complex;
	\item neither very complex nor very simple;
	\item routine or complex
	\item routine or simple
\end{enumerate}
\solution
%\begin{table}[H]
	\centering
\begin{tabular}{|c|c|c|}
\hline
Random variable &Value &Definition\\ \hline
\multirow{3}{*}{X} &0 &Slips of Rs 1\\
&1 &Slips of Rs 5\\
&2 &Slips of Rs 13\\ \hline
\multirow{2}{*}{Y} &0 &Box A\\
&1 &Box B\\\hline
\end{tabular}
\caption{}
\label{tab:Distribution}
\end{table}
See \tabref{tab:Distribution}.
\begin{align}
p_{Y}\brak{k}= \begin{cases} 
      \frac{1}{3} & {k=0} \\
      \frac{2}{3 }& {k=1} 
   \end{cases}
   \\
p_{Y|X}\brak{0|0} = \frac{19}{25}\, 
p_{Y|X}\brak{0|1} = \frac{6}{25}\,
p_{Y|X}\brak{1|0} = \frac{45}{50}\,
p_{Y|X}\brak{1|2} = \frac{5}{50}
\end{align}
The desired probability is the probability that a slip drawn at random is marked other than Rs 1,
\begin{align}
&=1-p_X\brak{0}\\
&= p_X(1) + p_X(2)
\end{align}
Using Bayes theorem,
\begin{align}
&= p_Y\brak{0} \times \pr{Y=0 | X=1} + p_Y\brak{1} \times \pr{Y=1|X=2}\\
&=\frac{1}{3} \times \frac{6}{25} + \frac{2}{3} \times \frac{5}{50}\\
&=\frac{11}{75}
\end{align}

\newpage

%\tableofcontents

\bigskip

\renewcommand{\thefigure}{\theenumi}
\renewcommand{\thetable}{\theenumi}
%\renewcommand{\theequation}{\theenumi}

%\begin{abstract}
%%\boldmath
%In this letter, an algorithm for evaluating the exact analytical bit error rate  (BER)  for the piecewise linear (PL) combiner for  multiple relays is presented. Previous results were available only for upto three relays. The algorithm is unique in the sense that  the actual mathematical expressions, that are prohibitively large, need not be explicitly obtained. The diversity gain due to multiple relays is shown through plots of the analytical BER, well supported by simulations. 
%
%\end{abstract}
% IEEEtran.cls defaults to using nonbold math in the Abstract.
% This preserves the distinction between vectors and scalars. However,
% if the journal you are submitting to favors bold math in the abstract,
% then you can use LaTeX's standard command \boldmath at the very start
% of the abstract to achieve this. Many IEEE journals frown on math
% in the abstract anyway.

% Note that keywords are not normally used for peerreview papers.
%\begin{IEEEkeywords}
%Cooperative diversity, decode and forward, piecewise linear
%\end{IEEEkeywords}



% For peer review papers, you can put extra information on the cover
% page as needed:
% \ifCLASSOPTIONpeerreview
% \begin{center} \bfseries EDICS Category: 3-BBND \end{center}
% \fi
%
% For peerreview papers, this IEEEtran command inserts a page break and
% creates the second title. It will be ignored for other modes.
%\IEEEpeerreviewmaketitle




\item A card is selected from a pack of 52 cards.
\begin{enumerate}[label=(\alph*)]
    \item How many points are there in the sample space?
    \item Calculate the probability that the card is an ace of spades.
    \item Calculate the probability that the card is (i) an ace and (ii) black card.
\end{enumerate}
\solution
%Let $X$ be an bernoulli rv defined as in \tabref{tab:exemplar/11/16/3/26}.  Then, 
\begin{equation}
    p =
        \frac{4}{11} 
\end{equation}
\begin{table}[H]
	\centering
	\input{exemplar/11/16/3/26/tables/Table2.tex}
	\caption{}
        \label{tab:exemplar/11/16/3/26}
\end{table}

\item The probability that a non leap year selected at random will contain 53 sundays.
\\
\solution
%\begin{table}[H]
	\centering
\begin{tabular}{|c|c|c|}
\hline
Random variable &Value &Definition\\ \hline
\multirow{3}{*}{X} &0 &Slips of Rs 1\\
&1 &Slips of Rs 5\\
&2 &Slips of Rs 13\\ \hline
\multirow{2}{*}{Y} &0 &Box A\\
&1 &Box B\\\hline
\end{tabular}
\caption{}
\label{tab:Distribution}
\end{table}
See \tabref{tab:Distribution}.
\begin{align}
p_{Y}\brak{k}= \begin{cases} 
      \frac{1}{3} & {k=0} \\
      \frac{2}{3 }& {k=1} 
   \end{cases}
   \\
p_{Y|X}\brak{0|0} = \frac{19}{25}\, 
p_{Y|X}\brak{0|1} = \frac{6}{25}\,
p_{Y|X}\brak{1|0} = \frac{45}{50}\,
p_{Y|X}\brak{1|2} = \frac{5}{50}
\end{align}
The desired probability is the probability that a slip drawn at random is marked other than Rs 1,
\begin{align}
&=1-p_X\brak{0}\\
&= p_X(1) + p_X(2)
\end{align}
Using Bayes theorem,
\begin{align}
&= p_Y\brak{0} \times \pr{Y=0 | X=1} + p_Y\brak{1} \times \pr{Y=1|X=2}\\
&=\frac{1}{3} \times \frac{6}{25} + \frac{2}{3} \times \frac{5}{50}\\
&=\frac{11}{75}
\end{align}

\newpage

%\tableofcontents

\bigskip

\renewcommand{\thefigure}{\theenumi}
\renewcommand{\thetable}{\theenumi}
%\renewcommand{\theequation}{\theenumi}

%\begin{abstract}
%%\boldmath
%In this letter, an algorithm for evaluating the exact analytical bit error rate  (BER)  for the piecewise linear (PL) combiner for  multiple relays is presented. Previous results were available only for upto three relays. The algorithm is unique in the sense that  the actual mathematical expressions, that are prohibitively large, need not be explicitly obtained. The diversity gain due to multiple relays is shown through plots of the analytical BER, well supported by simulations. 
%
%\end{abstract}
% IEEEtran.cls defaults to using nonbold math in the Abstract.
% This preserves the distinction between vectors and scalars. However,
% if the journal you are submitting to favors bold math in the abstract,
% then you can use LaTeX's standard command \boldmath at the very start
% of the abstract to achieve this. Many IEEE journals frown on math
% in the abstract anyway.

% Note that keywords are not normally used for peerreview papers.
%\begin{IEEEkeywords}
%Cooperative diversity, decode and forward, piecewise linear
%\end{IEEEkeywords}



% For peer review papers, you can put extra information on the cover
% page as needed:
% \ifCLASSOPTIONpeerreview
% \begin{center} \bfseries EDICS Category: 3-BBND \end{center}
% \fi
%
% For peerreview papers, this IEEEtran command inserts a page break and
% creates the second title. It will be ignored for other modes.
%\IEEEpeerreviewmaketitle




\item One of the four persons John, Rita, Aslam or Gurpreet will be promoted next
month. Consequently the sample space consists of four elementary outcomes
S = {John promoted, Rita promoted, Aslam promoted, Gurpreet promoted}
You are told that the chances of John’s promotion is same as that of Gurpreet,
Rita’s chances of promotion are twice as likely as Johns. Aslam’s chances are
four times that of John.
\begin{enumerate}
	\item Determine
	\begin{enumerate}
		\item P (John promoted)
		\item P (Rita promoted)
		\item P (Aslam promoted)
		\item P (Gurpreet promoted)
	\end{enumerate}
	\item If A = {John promoted or Gurpreet promoted}, find P (A).
\end{enumerate}
\solution
%\begin{table}[H]
	\centering
\begin{tabular}{|c|c|c|}
\hline
Random variable &Value &Definition\\ \hline
\multirow{3}{*}{X} &0 &Slips of Rs 1\\
&1 &Slips of Rs 5\\
&2 &Slips of Rs 13\\ \hline
\multirow{2}{*}{Y} &0 &Box A\\
&1 &Box B\\\hline
\end{tabular}
\caption{}
\label{tab:Distribution}
\end{table}
See \tabref{tab:Distribution}.
\begin{align}
p_{Y}\brak{k}= \begin{cases} 
      \frac{1}{3} & {k=0} \\
      \frac{2}{3 }& {k=1} 
   \end{cases}
   \\
p_{Y|X}\brak{0|0} = \frac{19}{25}\, 
p_{Y|X}\brak{0|1} = \frac{6}{25}\,
p_{Y|X}\brak{1|0} = \frac{45}{50}\,
p_{Y|X}\brak{1|2} = \frac{5}{50}
\end{align}
The desired probability is the probability that a slip drawn at random is marked other than Rs 1,
\begin{align}
&=1-p_X\brak{0}\\
&= p_X(1) + p_X(2)
\end{align}
Using Bayes theorem,
\begin{align}
&= p_Y\brak{0} \times \pr{Y=0 | X=1} + p_Y\brak{1} \times \pr{Y=1|X=2}\\
&=\frac{1}{3} \times \frac{6}{25} + \frac{2}{3} \times \frac{5}{50}\\
&=\frac{11}{75}
\end{align}

\newpage

%\tableofcontents

\bigskip

\renewcommand{\thefigure}{\theenumi}
\renewcommand{\thetable}{\theenumi}
%\renewcommand{\theequation}{\theenumi}

%\begin{abstract}
%%\boldmath
%In this letter, an algorithm for evaluating the exact analytical bit error rate  (BER)  for the piecewise linear (PL) combiner for  multiple relays is presented. Previous results were available only for upto three relays. The algorithm is unique in the sense that  the actual mathematical expressions, that are prohibitively large, need not be explicitly obtained. The diversity gain due to multiple relays is shown through plots of the analytical BER, well supported by simulations. 
%
%\end{abstract}
% IEEEtran.cls defaults to using nonbold math in the Abstract.
% This preserves the distinction between vectors and scalars. However,
% if the journal you are submitting to favors bold math in the abstract,
% then you can use LaTeX's standard command \boldmath at the very start
% of the abstract to achieve this. Many IEEE journals frown on math
% in the abstract anyway.

% Note that keywords are not normally used for peerreview papers.
%\begin{IEEEkeywords}
%Cooperative diversity, decode and forward, piecewise linear
%\end{IEEEkeywords}



% For peer review papers, you can put extra information on the cover
% page as needed:
% \ifCLASSOPTIONpeerreview
% \begin{center} \bfseries EDICS Category: 3-BBND \end{center}
% \fi
%
% For peerreview papers, this IEEEtran command inserts a page break and
% creates the second title. It will be ignored for other modes.
%\IEEEpeerreviewmaketitle




\item A card is drawn from a deck of 52 cards. Find the probability of getting a king or a heart or a red card.\\
\solution
%\begin{table}[H]
	\centering
\begin{tabular}{|c|c|c|}
\hline
Random variable &Value &Definition\\ \hline
\multirow{3}{*}{X} &0 &Slips of Rs 1\\
&1 &Slips of Rs 5\\
&2 &Slips of Rs 13\\ \hline
\multirow{2}{*}{Y} &0 &Box A\\
&1 &Box B\\\hline
\end{tabular}
\caption{}
\label{tab:Distribution}
\end{table}
See \tabref{tab:Distribution}.
\begin{align}
p_{Y}\brak{k}= \begin{cases} 
      \frac{1}{3} & {k=0} \\
      \frac{2}{3 }& {k=1} 
   \end{cases}
   \\
p_{Y|X}\brak{0|0} = \frac{19}{25}\, 
p_{Y|X}\brak{0|1} = \frac{6}{25}\,
p_{Y|X}\brak{1|0} = \frac{45}{50}\,
p_{Y|X}\brak{1|2} = \frac{5}{50}
\end{align}
The desired probability is the probability that a slip drawn at random is marked other than Rs 1,
\begin{align}
&=1-p_X\brak{0}\\
&= p_X(1) + p_X(2)
\end{align}
Using Bayes theorem,
\begin{align}
&= p_Y\brak{0} \times \pr{Y=0 | X=1} + p_Y\brak{1} \times \pr{Y=1|X=2}\\
&=\frac{1}{3} \times \frac{6}{25} + \frac{2}{3} \times \frac{5}{50}\\
&=\frac{11}{75}
\end{align}

\newpage

%\tableofcontents

\bigskip

\renewcommand{\thefigure}{\theenumi}
\renewcommand{\thetable}{\theenumi}
%\renewcommand{\theequation}{\theenumi}

%\begin{abstract}
%%\boldmath
%In this letter, an algorithm for evaluating the exact analytical bit error rate  (BER)  for the piecewise linear (PL) combiner for  multiple relays is presented. Previous results were available only for upto three relays. The algorithm is unique in the sense that  the actual mathematical expressions, that are prohibitively large, need not be explicitly obtained. The diversity gain due to multiple relays is shown through plots of the analytical BER, well supported by simulations. 
%
%\end{abstract}
% IEEEtran.cls defaults to using nonbold math in the Abstract.
% This preserves the distinction between vectors and scalars. However,
% if the journal you are submitting to favors bold math in the abstract,
% then you can use LaTeX's standard command \boldmath at the very start
% of the abstract to achieve this. Many IEEE journals frown on math
% in the abstract anyway.

% Note that keywords are not normally used for peerreview papers.
%\begin{IEEEkeywords}
%Cooperative diversity, decode and forward, piecewise linear
%\end{IEEEkeywords}



% For peer review papers, you can put extra information on the cover
% page as needed:
% \ifCLASSOPTIONpeerreview
% \begin{center} \bfseries EDICS Category: 3-BBND \end{center}
% \fi
%
% For peerreview papers, this IEEEtran command inserts a page break and
% creates the second title. It will be ignored for other modes.
%\IEEEpeerreviewmaketitle




\item The probability that a student will pass his examination is 0.73, the probability of
the student getting a compartment is 0.13, and the probability that the student will
either pass or get compartment is 0.96. State True or False.\\
\solution
%\begin{table}[H]
	\centering
\begin{tabular}{|c|c|c|}
\hline
Random variable &Value &Definition\\ \hline
\multirow{3}{*}{X} &0 &Slips of Rs 1\\
&1 &Slips of Rs 5\\
&2 &Slips of Rs 13\\ \hline
\multirow{2}{*}{Y} &0 &Box A\\
&1 &Box B\\\hline
\end{tabular}
\caption{}
\label{tab:Distribution}
\end{table}
See \tabref{tab:Distribution}.
\begin{align}
p_{Y}\brak{k}= \begin{cases} 
      \frac{1}{3} & {k=0} \\
      \frac{2}{3 }& {k=1} 
   \end{cases}
   \\
p_{Y|X}\brak{0|0} = \frac{19}{25}\, 
p_{Y|X}\brak{0|1} = \frac{6}{25}\,
p_{Y|X}\brak{1|0} = \frac{45}{50}\,
p_{Y|X}\brak{1|2} = \frac{5}{50}
\end{align}
The desired probability is the probability that a slip drawn at random is marked other than Rs 1,
\begin{align}
&=1-p_X\brak{0}\\
&= p_X(1) + p_X(2)
\end{align}
Using Bayes theorem,
\begin{align}
&= p_Y\brak{0} \times \pr{Y=0 | X=1} + p_Y\brak{1} \times \pr{Y=1|X=2}\\
&=\frac{1}{3} \times \frac{6}{25} + \frac{2}{3} \times \frac{5}{50}\\
&=\frac{11}{75}
\end{align}

\newpage

%\tableofcontents

\bigskip

\renewcommand{\thefigure}{\theenumi}
\renewcommand{\thetable}{\theenumi}
%\renewcommand{\theequation}{\theenumi}

%\begin{abstract}
%%\boldmath
%In this letter, an algorithm for evaluating the exact analytical bit error rate  (BER)  for the piecewise linear (PL) combiner for  multiple relays is presented. Previous results were available only for upto three relays. The algorithm is unique in the sense that  the actual mathematical expressions, that are prohibitively large, need not be explicitly obtained. The diversity gain due to multiple relays is shown through plots of the analytical BER, well supported by simulations. 
%
%\end{abstract}
% IEEEtran.cls defaults to using nonbold math in the Abstract.
% This preserves the distinction between vectors and scalars. However,
% if the journal you are submitting to favors bold math in the abstract,
% then you can use LaTeX's standard command \boldmath at the very start
% of the abstract to achieve this. Many IEEE journals frown on math
% in the abstract anyway.

% Note that keywords are not normally used for peerreview papers.
%\begin{IEEEkeywords}
%Cooperative diversity, decode and forward, piecewise linear
%\end{IEEEkeywords}



% For peer review papers, you can put extra information on the cover
% page as needed:
% \ifCLASSOPTIONpeerreview
% \begin{center} \bfseries EDICS Category: 3-BBND \end{center}
% \fi
%
% For peerreview papers, this IEEEtran command inserts a page break and
% creates the second title. It will be ignored for other modes.
%\IEEEpeerreviewmaketitle




\item A card is selected from a pack of 52 cards\\
\begin{enumerate}[label=(\alph*)]
\item How many points are there in the sample space?
\item Calculate the probability that the cards is an ace of spades.
\item Calculate the probability that the card is (i) an ace (ii)black card.\\
\end{enumerate}
%\input{ncert/11/16/3/4_1/Prob_4.tex}
\item In a non-leap year, the probability of having 53 tuesdays or 53 wednesdays is\\
\solution
%A non-leap year has a total of 365 days, and a week has 7 days.\\
So it can be expressed as 
\begin{align}
365\text{days} &=52\times 7+1 \text{day}
\end{align}
$\implies$ 52 tuesdays or wednesdays\\
Random variable X denotes the days of a week
\begin{align}
p_X\brak{k}&=\frac{1}{7}; \quad \brak{1<k<7}
\end{align}
So the probability of extra day being tuesday or wednesday is
\begin{align}
p_X\brak{3}+p_X\brak{4}&=\frac{1}{7}+\frac{1}{7}=\frac{2}{7}
\end{align}



\item There are 1000 sealed envelopes in a box, 10 of them contain a cash prize of
Rs 100 each, 100 of them contain a cash prize of Rs 50 each and 200 of them
contain a cash prize of Rs 10 each and rest do not contain any cash prize. If they
are well shuffled and an envelope is picked up out, what is the probability that it
contains no cash prize?\\
\solution
%\begin{table}[H]
	\centering
\begin{tabular}{|c|c|c|}
\hline
Random variable &Value &Definition\\ \hline
\multirow{3}{*}{X} &0 &Slips of Rs 1\\
&1 &Slips of Rs 5\\
&2 &Slips of Rs 13\\ \hline
\multirow{2}{*}{Y} &0 &Box A\\
&1 &Box B\\\hline
\end{tabular}
\caption{}
\label{tab:Distribution}
\end{table}
See \tabref{tab:Distribution}.
\begin{align}
p_{Y}\brak{k}= \begin{cases} 
      \frac{1}{3} & {k=0} \\
      \frac{2}{3 }& {k=1} 
   \end{cases}
   \\
p_{Y|X}\brak{0|0} = \frac{19}{25}\, 
p_{Y|X}\brak{0|1} = \frac{6}{25}\,
p_{Y|X}\brak{1|0} = \frac{45}{50}\,
p_{Y|X}\brak{1|2} = \frac{5}{50}
\end{align}
The desired probability is the probability that a slip drawn at random is marked other than Rs 1,
\begin{align}
&=1-p_X\brak{0}\\
&= p_X(1) + p_X(2)
\end{align}
Using Bayes theorem,
\begin{align}
&= p_Y\brak{0} \times \pr{Y=0 | X=1} + p_Y\brak{1} \times \pr{Y=1|X=2}\\
&=\frac{1}{3} \times \frac{6}{25} + \frac{2}{3} \times \frac{5}{50}\\
&=\frac{11}{75}
\end{align}

\newpage

%\tableofcontents

\bigskip

\renewcommand{\thefigure}{\theenumi}
\renewcommand{\thetable}{\theenumi}
%\renewcommand{\theequation}{\theenumi}

%\begin{abstract}
%%\boldmath
%In this letter, an algorithm for evaluating the exact analytical bit error rate  (BER)  for the piecewise linear (PL) combiner for  multiple relays is presented. Previous results were available only for upto three relays. The algorithm is unique in the sense that  the actual mathematical expressions, that are prohibitively large, need not be explicitly obtained. The diversity gain due to multiple relays is shown through plots of the analytical BER, well supported by simulations. 
%
%\end{abstract}
% IEEEtran.cls defaults to using nonbold math in the Abstract.
% This preserves the distinction between vectors and scalars. However,
% if the journal you are submitting to favors bold math in the abstract,
% then you can use LaTeX's standard command \boldmath at the very start
% of the abstract to achieve this. Many IEEE journals frown on math
% in the abstract anyway.

% Note that keywords are not normally used for peerreview papers.
%\begin{IEEEkeywords}
%Cooperative diversity, decode and forward, piecewise linear
%\end{IEEEkeywords}



% For peer review papers, you can put extra information on the cover
% page as needed:
% \ifCLASSOPTIONpeerreview
% \begin{center} \bfseries EDICS Category: 3-BBND \end{center}
% \fi
%
% For peerreview papers, this IEEEtran command inserts a page break and
% creates the second title. It will be ignored for other modes.
%\IEEEpeerreviewmaketitle




\item 
A die is thrown and a card is selected at random from a deck of 52 playing cards. The probability of getting an even number on the die and a spade card.\\
\solution
%\begin{table}[H]
	\centering
\begin{tabular}{|c|c|c|}
\hline
Random variable &Value &Definition\\ \hline
\multirow{3}{*}{X} &0 &Slips of Rs 1\\
&1 &Slips of Rs 5\\
&2 &Slips of Rs 13\\ \hline
\multirow{2}{*}{Y} &0 &Box A\\
&1 &Box B\\\hline
\end{tabular}
\caption{}
\label{tab:Distribution}
\end{table}
See \tabref{tab:Distribution}.
\begin{align}
p_{Y}\brak{k}= \begin{cases} 
      \frac{1}{3} & {k=0} \\
      \frac{2}{3 }& {k=1} 
   \end{cases}
   \\
p_{Y|X}\brak{0|0} = \frac{19}{25}\, 
p_{Y|X}\brak{0|1} = \frac{6}{25}\,
p_{Y|X}\brak{1|0} = \frac{45}{50}\,
p_{Y|X}\brak{1|2} = \frac{5}{50}
\end{align}
The desired probability is the probability that a slip drawn at random is marked other than Rs 1,
\begin{align}
&=1-p_X\brak{0}\\
&= p_X(1) + p_X(2)
\end{align}
Using Bayes theorem,
\begin{align}
&= p_Y\brak{0} \times \pr{Y=0 | X=1} + p_Y\brak{1} \times \pr{Y=1|X=2}\\
&=\frac{1}{3} \times \frac{6}{25} + \frac{2}{3} \times \frac{5}{50}\\
&=\frac{11}{75}
\end{align}

\newpage

%\tableofcontents

\bigskip

\renewcommand{\thefigure}{\theenumi}
\renewcommand{\thetable}{\theenumi}
%\renewcommand{\theequation}{\theenumi}

%\begin{abstract}
%%\boldmath
%In this letter, an algorithm for evaluating the exact analytical bit error rate  (BER)  for the piecewise linear (PL) combiner for  multiple relays is presented. Previous results were available only for upto three relays. The algorithm is unique in the sense that  the actual mathematical expressions, that are prohibitively large, need not be explicitly obtained. The diversity gain due to multiple relays is shown through plots of the analytical BER, well supported by simulations. 
%
%\end{abstract}
% IEEEtran.cls defaults to using nonbold math in the Abstract.
% This preserves the distinction between vectors and scalars. However,
% if the journal you are submitting to favors bold math in the abstract,
% then you can use LaTeX's standard command \boldmath at the very start
% of the abstract to achieve this. Many IEEE journals frown on math
% in the abstract anyway.

% Note that keywords are not normally used for peerreview papers.
%\begin{IEEEkeywords}
%Cooperative diversity, decode and forward, piecewise linear
%\end{IEEEkeywords}



% For peer review papers, you can put extra information on the cover
% page as needed:
% \ifCLASSOPTIONpeerreview
% \begin{center} \bfseries EDICS Category: 3-BBND \end{center}
% \fi
%
% For peerreview papers, this IEEEtran command inserts a page break and
% creates the second title. It will be ignored for other modes.
%\IEEEpeerreviewmaketitle




\item
If 4-digit numbers greater than 5,000 are randomly formed from the digits 0, 1, 3, 5, and 7, what is the probability of forming a number divisible by 5 when:
\begin{enumerate}
    \item The digits are repeated?
    \item The repetition of digits is not allowed?
\end{enumerate}
\solution
%\begin{table}[H]
	\centering
\begin{tabular}{|c|c|c|}
\hline
Random variable &Value &Definition\\ \hline
\multirow{3}{*}{X} &0 &Slips of Rs 1\\
&1 &Slips of Rs 5\\
&2 &Slips of Rs 13\\ \hline
\multirow{2}{*}{Y} &0 &Box A\\
&1 &Box B\\\hline
\end{tabular}
\caption{}
\label{tab:Distribution}
\end{table}
See \tabref{tab:Distribution}.
\begin{align}
p_{Y}\brak{k}= \begin{cases} 
      \frac{1}{3} & {k=0} \\
      \frac{2}{3 }& {k=1} 
   \end{cases}
   \\
p_{Y|X}\brak{0|0} = \frac{19}{25}\, 
p_{Y|X}\brak{0|1} = \frac{6}{25}\,
p_{Y|X}\brak{1|0} = \frac{45}{50}\,
p_{Y|X}\brak{1|2} = \frac{5}{50}
\end{align}
The desired probability is the probability that a slip drawn at random is marked other than Rs 1,
\begin{align}
&=1-p_X\brak{0}\\
&= p_X(1) + p_X(2)
\end{align}
Using Bayes theorem,
\begin{align}
&= p_Y\brak{0} \times \pr{Y=0 | X=1} + p_Y\brak{1} \times \pr{Y=1|X=2}\\
&=\frac{1}{3} \times \frac{6}{25} + \frac{2}{3} \times \frac{5}{50}\\
&=\frac{11}{75}
\end{align}

\newpage

%\tableofcontents

\bigskip

\renewcommand{\thefigure}{\theenumi}
\renewcommand{\thetable}{\theenumi}
%\renewcommand{\theequation}{\theenumi}

%\begin{abstract}
%%\boldmath
%In this letter, an algorithm for evaluating the exact analytical bit error rate  (BER)  for the piecewise linear (PL) combiner for  multiple relays is presented. Previous results were available only for upto three relays. The algorithm is unique in the sense that  the actual mathematical expressions, that are prohibitively large, need not be explicitly obtained. The diversity gain due to multiple relays is shown through plots of the analytical BER, well supported by simulations. 
%
%\end{abstract}
% IEEEtran.cls defaults to using nonbold math in the Abstract.
% This preserves the distinction between vectors and scalars. However,
% if the journal you are submitting to favors bold math in the abstract,
% then you can use LaTeX's standard command \boldmath at the very start
% of the abstract to achieve this. Many IEEE journals frown on math
% in the abstract anyway.

% Note that keywords are not normally used for peerreview papers.
%\begin{IEEEkeywords}
%Cooperative diversity, decode and forward, piecewise linear
%\end{IEEEkeywords}



% For peer review papers, you can put extra information on the cover
% page as needed:
% \ifCLASSOPTIONpeerreview
% \begin{center} \bfseries EDICS Category: 3-BBND \end{center}
% \fi
%
% For peerreview papers, this IEEEtran command inserts a page break and
% creates the second title. It will be ignored for other modes.
%\IEEEpeerreviewmaketitle




\item Consider the probability space $\brak{\Omega, \mathcal{G}, P}$ where $\Omega = [0,2]$ and $\mathcal{G} = \cbrak{\phi, \Omega, [0,1], (1,2]}$. Let $X$ and $Y$ be two functions on $\Omega$ defined as
\begin{align*}
    X(\omega) = 
    \begin{cases}
        1 & \text{if }\omega \in [0, 1]\\
        2 & \text{if }\omega \in (1, 2]
    \end{cases}
\end{align*}
and
\begin{align*}
    Y(\omega) = 
    \begin{cases}
        2 & \text{if }\omega \in [0, 1.5]\\
        3 & \text{if }\omega \in (1.5, 2].
    \end{cases}
\end{align*}
Then which one of the following statements is true?
\begin{enumerate}
    \item [(A)] $X$ is a random variable with respect to $\mathcal{G}$, but $Y$ is not a random variable with respect to $\mathcal{G}$.
    \item [(B)] $Y$ is a random variable with respect to $\mathcal{G}$, but $X$ is not a random variable with respect to $\mathcal{G}$.
    \item [(C)] Neither $X$ nor $Y$ is a random variable with respect to $\mathcal{G}$.
    \item [(D)] Both $X$ and $Y$ are random variables with respect to $\mathcal{G}$.
\end{enumerate} \hfill (GATE ST 2023)\\
\solution
%\begin{table}[H]
	\centering
\begin{tabular}{|c|c|c|}
\hline
Random variable &Value &Definition\\ \hline
\multirow{3}{*}{X} &0 &Slips of Rs 1\\
&1 &Slips of Rs 5\\
&2 &Slips of Rs 13\\ \hline
\multirow{2}{*}{Y} &0 &Box A\\
&1 &Box B\\\hline
\end{tabular}
\caption{}
\label{tab:Distribution}
\end{table}
See \tabref{tab:Distribution}.
\begin{align}
p_{Y}\brak{k}= \begin{cases} 
      \frac{1}{3} & {k=0} \\
      \frac{2}{3 }& {k=1} 
   \end{cases}
   \\
p_{Y|X}\brak{0|0} = \frac{19}{25}\, 
p_{Y|X}\brak{0|1} = \frac{6}{25}\,
p_{Y|X}\brak{1|0} = \frac{45}{50}\,
p_{Y|X}\brak{1|2} = \frac{5}{50}
\end{align}
The desired probability is the probability that a slip drawn at random is marked other than Rs 1,
\begin{align}
&=1-p_X\brak{0}\\
&= p_X(1) + p_X(2)
\end{align}
Using Bayes theorem,
\begin{align}
&= p_Y\brak{0} \times \pr{Y=0 | X=1} + p_Y\brak{1} \times \pr{Y=1|X=2}\\
&=\frac{1}{3} \times \frac{6}{25} + \frac{2}{3} \times \frac{5}{50}\\
&=\frac{11}{75}
\end{align}

\newpage

%\tableofcontents

\bigskip

\renewcommand{\thefigure}{\theenumi}
\renewcommand{\thetable}{\theenumi}
%\renewcommand{\theequation}{\theenumi}

%\begin{abstract}
%%\boldmath
%In this letter, an algorithm for evaluating the exact analytical bit error rate  (BER)  for the piecewise linear (PL) combiner for  multiple relays is presented. Previous results were available only for upto three relays. The algorithm is unique in the sense that  the actual mathematical expressions, that are prohibitively large, need not be explicitly obtained. The diversity gain due to multiple relays is shown through plots of the analytical BER, well supported by simulations. 
%
%\end{abstract}
% IEEEtran.cls defaults to using nonbold math in the Abstract.
% This preserves the distinction between vectors and scalars. However,
% if the journal you are submitting to favors bold math in the abstract,
% then you can use LaTeX's standard command \boldmath at the very start
% of the abstract to achieve this. Many IEEE journals frown on math
% in the abstract anyway.

% Note that keywords are not normally used for peerreview papers.
%\begin{IEEEkeywords}
%Cooperative diversity, decode and forward, piecewise linear
%\end{IEEEkeywords}



% For peer review papers, you can put extra information on the cover
% page as needed:
% \ifCLASSOPTIONpeerreview
% \begin{center} \bfseries EDICS Category: 3-BBND \end{center}
% \fi
%
% For peerreview papers, this IEEEtran command inserts a page break and
% creates the second title. It will be ignored for other modes.
%\IEEEpeerreviewmaketitle




	\item  A die is loaded in such a way that each odd number is twice as likely to occur as
each even number. Find $P(G)$, where $G$ is the event that a number greater than
3 occurs on a single roll of the die.
\\
\solution
		%\begin{table}[H]
	\centering
\begin{tabular}{|c|c|c|}
\hline
Random variable &Value &Definition\\ \hline
\multirow{3}{*}{X} &0 &Slips of Rs 1\\
&1 &Slips of Rs 5\\
&2 &Slips of Rs 13\\ \hline
\multirow{2}{*}{Y} &0 &Box A\\
&1 &Box B\\\hline
\end{tabular}
\caption{}
\label{tab:Distribution}
\end{table}
See \tabref{tab:Distribution}.
\begin{align}
p_{Y}\brak{k}= \begin{cases} 
      \frac{1}{3} & {k=0} \\
      \frac{2}{3 }& {k=1} 
   \end{cases}
   \\
p_{Y|X}\brak{0|0} = \frac{19}{25}\, 
p_{Y|X}\brak{0|1} = \frac{6}{25}\,
p_{Y|X}\brak{1|0} = \frac{45}{50}\,
p_{Y|X}\brak{1|2} = \frac{5}{50}
\end{align}
The desired probability is the probability that a slip drawn at random is marked other than Rs 1,
\begin{align}
&=1-p_X\brak{0}\\
&= p_X(1) + p_X(2)
\end{align}
Using Bayes theorem,
\begin{align}
&= p_Y\brak{0} \times \pr{Y=0 | X=1} + p_Y\brak{1} \times \pr{Y=1|X=2}\\
&=\frac{1}{3} \times \frac{6}{25} + \frac{2}{3} \times \frac{5}{50}\\
&=\frac{11}{75}
\end{align}

\newpage

%\tableofcontents

\bigskip

\renewcommand{\thefigure}{\theenumi}
\renewcommand{\thetable}{\theenumi}
%\renewcommand{\theequation}{\theenumi}

%\begin{abstract}
%%\boldmath
%In this letter, an algorithm for evaluating the exact analytical bit error rate  (BER)  for the piecewise linear (PL) combiner for  multiple relays is presented. Previous results were available only for upto three relays. The algorithm is unique in the sense that  the actual mathematical expressions, that are prohibitively large, need not be explicitly obtained. The diversity gain due to multiple relays is shown through plots of the analytical BER, well supported by simulations. 
%
%\end{abstract}
% IEEEtran.cls defaults to using nonbold math in the Abstract.
% This preserves the distinction between vectors and scalars. However,
% if the journal you are submitting to favors bold math in the abstract,
% then you can use LaTeX's standard command \boldmath at the very start
% of the abstract to achieve this. Many IEEE journals frown on math
% in the abstract anyway.

% Note that keywords are not normally used for peerreview papers.
%\begin{IEEEkeywords}
%Cooperative diversity, decode and forward, piecewise linear
%\end{IEEEkeywords}



% For peer review papers, you can put extra information on the cover
% page as needed:
% \ifCLASSOPTIONpeerreview
% \begin{center} \bfseries EDICS Category: 3-BBND \end{center}
% \fi
%
% For peerreview papers, this IEEEtran command inserts a page break and
% creates the second title. It will be ignored for other modes.
%\IEEEpeerreviewmaketitle




	\item All the jacks, queens and kings are removed from a deck of 52 playing cards. The remaining cards are well shuffled and then one card is drawn at random. Giving ace a value 1 similar value for other cards, find the probability that the card has a value 
		\begin{enumerate}
			\item 7
			\item greater than 7
			\item less than 7
		\end{enumerate}
		%Number of cards left after removing all jacks, queens and kings 
\begin{align}
N	= 52 - 4\times 3
	= 40
\end{align}
%\begin{table}[H]
%\def\arraystretch{1.2}
%\begin{tabular}{|c|c|c|}
%\hline
%	\textbf{Parameter} &\textbf{Value} &\textbf{Description}\\ \hline
%	$X$ &1-10 &Represents the value of the card picked \\ \hline
%\end{tabular}
%\end{table}
Let $1 \le X \le 10$ be the value of the card picked.  Then,
\begin{align}
	p_X(k) &= \Pr(X=k)\ \forall\ 1 \leq k \leq 10\\
	&= \frac{4\times 1}{40}\\
	&= \frac{1}{10}\\
	\therefore p_X(k) &= 
	\begin{cases}
		\frac{1}{10} & 1 \leq k \leq 10\\
		0 & \text{otherwise}
	\end{cases}
\end{align}
and
\begin{align}
	F_{X}(k) &= \sum_{m=0}^{k}p_{X}(m) \quad 1 \leq k \leq 10\\
	&= \frac{k}{10}\\
	\therefore F_{X}(k) &= 
	\begin{cases}
		0 & k \leq 0\\
		\frac{k}{10} & 1\leq k \leq 10\\
		1 & k > 10 
	\end{cases}
\end{align}
\begin{enumerate}
	\item Probability that card has value equal to 7 is
		\begin{align}
			 p_{X}(7)
			= \frac{1}{10}
		\end{align}
	\item Probability that card has value greater than 7 is
		\begin{align}
			1 - F_X(7)
			&= 1 - \frac{7}{10}
			\\
			&= \frac{3}{10}
		\end{align}
	\item Probability that card has value less than 7 is
		\begin{align}
			 F_{X}(6)
			=\frac{6}{10}
		\end{align}
\end{enumerate}

  \item A Lot consists of 48 mobile phones of which 42 are good, 3 have only minor defects and 3 have major defects.Varnika will buy a phone if it is good but the trader will only buy a mobile if it has no major defects. One phone is selected at random from the lot. What is the probability that it is
\begin{enumerate}
	\item acceptable to Varnika?
            \item acceptable to the trader?
\end{enumerate}
\solution
	%\begin{table}[H]
	\centering
\begin{tabular}{|c|c|c|}
\hline
Random variable &Value &Definition\\ \hline
\multirow{3}{*}{X} &0 &Slips of Rs 1\\
&1 &Slips of Rs 5\\
&2 &Slips of Rs 13\\ \hline
\multirow{2}{*}{Y} &0 &Box A\\
&1 &Box B\\\hline
\end{tabular}
\caption{}
\label{tab:Distribution}
\end{table}
See \tabref{tab:Distribution}.
\begin{align}
p_{Y}\brak{k}= \begin{cases} 
      \frac{1}{3} & {k=0} \\
      \frac{2}{3 }& {k=1} 
   \end{cases}
   \\
p_{Y|X}\brak{0|0} = \frac{19}{25}\, 
p_{Y|X}\brak{0|1} = \frac{6}{25}\,
p_{Y|X}\brak{1|0} = \frac{45}{50}\,
p_{Y|X}\brak{1|2} = \frac{5}{50}
\end{align}
The desired probability is the probability that a slip drawn at random is marked other than Rs 1,
\begin{align}
&=1-p_X\brak{0}\\
&= p_X(1) + p_X(2)
\end{align}
Using Bayes theorem,
\begin{align}
&= p_Y\brak{0} \times \pr{Y=0 | X=1} + p_Y\brak{1} \times \pr{Y=1|X=2}\\
&=\frac{1}{3} \times \frac{6}{25} + \frac{2}{3} \times \frac{5}{50}\\
&=\frac{11}{75}
\end{align}

\newpage

%\tableofcontents

\bigskip

\renewcommand{\thefigure}{\theenumi}
\renewcommand{\thetable}{\theenumi}
%\renewcommand{\theequation}{\theenumi}

%\begin{abstract}
%%\boldmath
%In this letter, an algorithm for evaluating the exact analytical bit error rate  (BER)  for the piecewise linear (PL) combiner for  multiple relays is presented. Previous results were available only for upto three relays. The algorithm is unique in the sense that  the actual mathematical expressions, that are prohibitively large, need not be explicitly obtained. The diversity gain due to multiple relays is shown through plots of the analytical BER, well supported by simulations. 
%
%\end{abstract}
% IEEEtran.cls defaults to using nonbold math in the Abstract.
% This preserves the distinction between vectors and scalars. However,
% if the journal you are submitting to favors bold math in the abstract,
% then you can use LaTeX's standard command \boldmath at the very start
% of the abstract to achieve this. Many IEEE journals frown on math
% in the abstract anyway.

% Note that keywords are not normally used for peerreview papers.
%\begin{IEEEkeywords}
%Cooperative diversity, decode and forward, piecewise linear
%\end{IEEEkeywords}



% For peer review papers, you can put extra information on the cover
% page as needed:
% \ifCLASSOPTIONpeerreview
% \begin{center} \bfseries EDICS Category: 3-BBND \end{center}
% \fi
%
% For peerreview papers, this IEEEtran command inserts a page break and
% creates the second title. It will be ignored for other modes.
%\IEEEpeerreviewmaketitle




 \item A student says that if you throw a die, it will show up 1 or not 1. Therefore, the probability of getting 1 and the probability of getting 'not 1' each is equal to $\frac{1}{2}$. Is this correct? Give reasons.\\
 \solution
        %\begin{table}[H]
	\centering
\begin{tabular}{|c|c|c|}
\hline
Random variable &Value &Definition\\ \hline
\multirow{3}{*}{X} &0 &Slips of Rs 1\\
&1 &Slips of Rs 5\\
&2 &Slips of Rs 13\\ \hline
\multirow{2}{*}{Y} &0 &Box A\\
&1 &Box B\\\hline
\end{tabular}
\caption{}
\label{tab:Distribution}
\end{table}
See \tabref{tab:Distribution}.
\begin{align}
p_{Y}\brak{k}= \begin{cases} 
      \frac{1}{3} & {k=0} \\
      \frac{2}{3 }& {k=1} 
   \end{cases}
   \\
p_{Y|X}\brak{0|0} = \frac{19}{25}\, 
p_{Y|X}\brak{0|1} = \frac{6}{25}\,
p_{Y|X}\brak{1|0} = \frac{45}{50}\,
p_{Y|X}\brak{1|2} = \frac{5}{50}
\end{align}
The desired probability is the probability that a slip drawn at random is marked other than Rs 1,
\begin{align}
&=1-p_X\brak{0}\\
&= p_X(1) + p_X(2)
\end{align}
Using Bayes theorem,
\begin{align}
&= p_Y\brak{0} \times \pr{Y=0 | X=1} + p_Y\brak{1} \times \pr{Y=1|X=2}\\
&=\frac{1}{3} \times \frac{6}{25} + \frac{2}{3} \times \frac{5}{50}\\
&=\frac{11}{75}
\end{align}

\newpage

%\tableofcontents

\bigskip

\renewcommand{\thefigure}{\theenumi}
\renewcommand{\thetable}{\theenumi}
%\renewcommand{\theequation}{\theenumi}

%\begin{abstract}
%%\boldmath
%In this letter, an algorithm for evaluating the exact analytical bit error rate  (BER)  for the piecewise linear (PL) combiner for  multiple relays is presented. Previous results were available only for upto three relays. The algorithm is unique in the sense that  the actual mathematical expressions, that are prohibitively large, need not be explicitly obtained. The diversity gain due to multiple relays is shown through plots of the analytical BER, well supported by simulations. 
%
%\end{abstract}
% IEEEtran.cls defaults to using nonbold math in the Abstract.
% This preserves the distinction between vectors and scalars. However,
% if the journal you are submitting to favors bold math in the abstract,
% then you can use LaTeX's standard command \boldmath at the very start
% of the abstract to achieve this. Many IEEE journals frown on math
% in the abstract anyway.

% Note that keywords are not normally used for peerreview papers.
%\begin{IEEEkeywords}
%Cooperative diversity, decode and forward, piecewise linear
%\end{IEEEkeywords}



% For peer review papers, you can put extra information on the cover
% page as needed:
% \ifCLASSOPTIONpeerreview
% \begin{center} \bfseries EDICS Category: 3-BBND \end{center}
% \fi
%
% For peerreview papers, this IEEEtran command inserts a page break and
% creates the second title. It will be ignored for other modes.
%\IEEEpeerreviewmaketitle




   \item Four candidates A, B, C, D have ap-
plied for the assignment to coach a school cricket
team. If A is twice as likely to be selected as B, and
B and C are given about the same chance of being
selected, while C is twice as likely to be selected
as D, what are the probabilities that
\begin{enumerate}
\item C will be selected?
\item A will not be selected?
\end{enumerate}
	%\begin{table}[H]
	\centering
\begin{tabular}{|c|c|c|}
\hline
Random variable &Value &Definition\\ \hline
\multirow{3}{*}{X} &0 &Slips of Rs 1\\
&1 &Slips of Rs 5\\
&2 &Slips of Rs 13\\ \hline
\multirow{2}{*}{Y} &0 &Box A\\
&1 &Box B\\\hline
\end{tabular}
\caption{}
\label{tab:Distribution}
\end{table}
See \tabref{tab:Distribution}.
\begin{align}
p_{Y}\brak{k}= \begin{cases} 
      \frac{1}{3} & {k=0} \\
      \frac{2}{3 }& {k=1} 
   \end{cases}
   \\
p_{Y|X}\brak{0|0} = \frac{19}{25}\, 
p_{Y|X}\brak{0|1} = \frac{6}{25}\,
p_{Y|X}\brak{1|0} = \frac{45}{50}\,
p_{Y|X}\brak{1|2} = \frac{5}{50}
\end{align}
The desired probability is the probability that a slip drawn at random is marked other than Rs 1,
\begin{align}
&=1-p_X\brak{0}\\
&= p_X(1) + p_X(2)
\end{align}
Using Bayes theorem,
\begin{align}
&= p_Y\brak{0} \times \pr{Y=0 | X=1} + p_Y\brak{1} \times \pr{Y=1|X=2}\\
&=\frac{1}{3} \times \frac{6}{25} + \frac{2}{3} \times \frac{5}{50}\\
&=\frac{11}{75}
\end{align}

\newpage

%\tableofcontents

\bigskip

\renewcommand{\thefigure}{\theenumi}
\renewcommand{\thetable}{\theenumi}
%\renewcommand{\theequation}{\theenumi}

%\begin{abstract}
%%\boldmath
%In this letter, an algorithm for evaluating the exact analytical bit error rate  (BER)  for the piecewise linear (PL) combiner for  multiple relays is presented. Previous results were available only for upto three relays. The algorithm is unique in the sense that  the actual mathematical expressions, that are prohibitively large, need not be explicitly obtained. The diversity gain due to multiple relays is shown through plots of the analytical BER, well supported by simulations. 
%
%\end{abstract}
% IEEEtran.cls defaults to using nonbold math in the Abstract.
% This preserves the distinction between vectors and scalars. However,
% if the journal you are submitting to favors bold math in the abstract,
% then you can use LaTeX's standard command \boldmath at the very start
% of the abstract to achieve this. Many IEEE journals frown on math
% in the abstract anyway.

% Note that keywords are not normally used for peerreview papers.
%\begin{IEEEkeywords}
%Cooperative diversity, decode and forward, piecewise linear
%\end{IEEEkeywords}



% For peer review papers, you can put extra information on the cover
% page as needed:
% \ifCLASSOPTIONpeerreview
% \begin{center} \bfseries EDICS Category: 3-BBND \end{center}
% \fi
%
% For peerreview papers, this IEEEtran command inserts a page break and
% creates the second title. It will be ignored for other modes.
%\IEEEpeerreviewmaketitle




 \item A bag contain 24 balls of which $x$ balls are red, $2x$ are white and $3x$ are blue. A ball is selected at random, What is the probability that it is
\begin{enumerate}[label=\alph*)]
\item not red ?
\item white ?
\end{enumerate}
%\begin{table}[H]
	\centering
\begin{tabular}{|c|c|c|}
\hline
Random variable &Value &Definition\\ \hline
\multirow{3}{*}{X} &0 &Slips of Rs 1\\
&1 &Slips of Rs 5\\
&2 &Slips of Rs 13\\ \hline
\multirow{2}{*}{Y} &0 &Box A\\
&1 &Box B\\\hline
\end{tabular}
\caption{}
\label{tab:Distribution}
\end{table}
See \tabref{tab:Distribution}.
\begin{align}
p_{Y}\brak{k}= \begin{cases} 
      \frac{1}{3} & {k=0} \\
      \frac{2}{3 }& {k=1} 
   \end{cases}
   \\
p_{Y|X}\brak{0|0} = \frac{19}{25}\, 
p_{Y|X}\brak{0|1} = \frac{6}{25}\,
p_{Y|X}\brak{1|0} = \frac{45}{50}\,
p_{Y|X}\brak{1|2} = \frac{5}{50}
\end{align}
The desired probability is the probability that a slip drawn at random is marked other than Rs 1,
\begin{align}
&=1-p_X\brak{0}\\
&= p_X(1) + p_X(2)
\end{align}
Using Bayes theorem,
\begin{align}
&= p_Y\brak{0} \times \pr{Y=0 | X=1} + p_Y\brak{1} \times \pr{Y=1|X=2}\\
&=\frac{1}{3} \times \frac{6}{25} + \frac{2}{3} \times \frac{5}{50}\\
&=\frac{11}{75}
\end{align}

\newpage

%\tableofcontents

\bigskip

\renewcommand{\thefigure}{\theenumi}
\renewcommand{\thetable}{\theenumi}
%\renewcommand{\theequation}{\theenumi}

%\begin{abstract}
%%\boldmath
%In this letter, an algorithm for evaluating the exact analytical bit error rate  (BER)  for the piecewise linear (PL) combiner for  multiple relays is presented. Previous results were available only for upto three relays. The algorithm is unique in the sense that  the actual mathematical expressions, that are prohibitively large, need not be explicitly obtained. The diversity gain due to multiple relays is shown through plots of the analytical BER, well supported by simulations. 
%
%\end{abstract}
% IEEEtran.cls defaults to using nonbold math in the Abstract.
% This preserves the distinction between vectors and scalars. However,
% if the journal you are submitting to favors bold math in the abstract,
% then you can use LaTeX's standard command \boldmath at the very start
% of the abstract to achieve this. Many IEEE journals frown on math
% in the abstract anyway.

% Note that keywords are not normally used for peerreview papers.
%\begin{IEEEkeywords}
%Cooperative diversity, decode and forward, piecewise linear
%\end{IEEEkeywords}



% For peer review papers, you can put extra information on the cover
% page as needed:
% \ifCLASSOPTIONpeerreview
% \begin{center} \bfseries EDICS Category: 3-BBND \end{center}
% \fi
%
% For peerreview papers, this IEEEtran command inserts a page break and
% creates the second title. It will be ignored for other modes.
%\IEEEpeerreviewmaketitle




If the letters of the word ASSASSINATION are arranged at random. Find the Probability that
\begin{enumerate}[label=(\alph*)]
\item Four $S's$ come consecutively in the word
\item Two  $I's$ and two $N's$ come together
\item All $A's$ are not coming together
\item No two $A's$ are coming together
\end{enumerate}
%\begin{table}[H]
	\centering
\begin{tabular}{|c|c|c|}
\hline
Random variable &Value &Definition\\ \hline
\multirow{3}{*}{X} &0 &Slips of Rs 1\\
&1 &Slips of Rs 5\\
&2 &Slips of Rs 13\\ \hline
\multirow{2}{*}{Y} &0 &Box A\\
&1 &Box B\\\hline
\end{tabular}
\caption{}
\label{tab:Distribution}
\end{table}
See \tabref{tab:Distribution}.
\begin{align}
p_{Y}\brak{k}= \begin{cases} 
      \frac{1}{3} & {k=0} \\
      \frac{2}{3 }& {k=1} 
   \end{cases}
   \\
p_{Y|X}\brak{0|0} = \frac{19}{25}\, 
p_{Y|X}\brak{0|1} = \frac{6}{25}\,
p_{Y|X}\brak{1|0} = \frac{45}{50}\,
p_{Y|X}\brak{1|2} = \frac{5}{50}
\end{align}
The desired probability is the probability that a slip drawn at random is marked other than Rs 1,
\begin{align}
&=1-p_X\brak{0}\\
&= p_X(1) + p_X(2)
\end{align}
Using Bayes theorem,
\begin{align}
&= p_Y\brak{0} \times \pr{Y=0 | X=1} + p_Y\brak{1} \times \pr{Y=1|X=2}\\
&=\frac{1}{3} \times \frac{6}{25} + \frac{2}{3} \times \frac{5}{50}\\
&=\frac{11}{75}
\end{align}

\newpage

%\tableofcontents

\bigskip

\renewcommand{\thefigure}{\theenumi}
\renewcommand{\thetable}{\theenumi}
%\renewcommand{\theequation}{\theenumi}

%\begin{abstract}
%%\boldmath
%In this letter, an algorithm for evaluating the exact analytical bit error rate  (BER)  for the piecewise linear (PL) combiner for  multiple relays is presented. Previous results were available only for upto three relays. The algorithm is unique in the sense that  the actual mathematical expressions, that are prohibitively large, need not be explicitly obtained. The diversity gain due to multiple relays is shown through plots of the analytical BER, well supported by simulations. 
%
%\end{abstract}
% IEEEtran.cls defaults to using nonbold math in the Abstract.
% This preserves the distinction between vectors and scalars. However,
% if the journal you are submitting to favors bold math in the abstract,
% then you can use LaTeX's standard command \boldmath at the very start
% of the abstract to achieve this. Many IEEE journals frown on math
% in the abstract anyway.

% Note that keywords are not normally used for peerreview papers.
%\begin{IEEEkeywords}
%Cooperative diversity, decode and forward, piecewise linear
%\end{IEEEkeywords}



% For peer review papers, you can put extra information on the cover
% page as needed:
% \ifCLASSOPTIONpeerreview
% \begin{center} \bfseries EDICS Category: 3-BBND \end{center}
% \fi
%
% For peerreview papers, this IEEEtran command inserts a page break and
% creates the second title. It will be ignored for other modes.
%\IEEEpeerreviewmaketitle




	\item One urn contains two black balls (labelled B1 and B2) and one white ball. A
	second urn contains one black ball and two white balls (labelled W1 and W2).
	Suppose the following experiment is performed. One of the two urns is chosen
	at random. Next a ball is randomly chosen from the urn. Then a second ball is
	chosen at random from the same urn without replacing the first ball.
	
	\begin{enumerate}
	\item What is the probability that two black balls are chosen?
	
	\item What is the probability that two balls of opposite colour are chosen?
	\end{enumerate}
	\solution
	%\begin{align}
    \label{eq:12.13.6.18.1}
	\because	\pr{A|B} &> \pr{A},\
\frac{\pr{AB}}{\pr{B}} > \pr{A}
\\
    \label{eq:12.13.6.18.2}
	\implies \pr{AB} &> \pr{A}\pr{B}
	\\
	\text{or, } \frac{\pr{AB}}{\pr{A}} &=\pr{B|A} > \pr{A}
\end{align}

\end{enumerate}

		%
\item 
Out of 100 students, two sections of 40 and 60 are formed. If you and your friend are among the 100 students, what is the probability that
\begin{enumerate}
\item you both enter the same section?
\item you both enter the different sections?
\end{enumerate}
\solution
		%\begin{enumerate}[label=\thesection.\arabic*,ref=\thesection.\theenumi]
	\item One card is drawn from a well-shuffled deck of 52 cards. Find the probability of getting
\begin{enumerate}
\item A king of red colour 
\item A face card 
\item A red face card
\item The jack of hearts
\item A spade
\item The queen of diamonds

\end{enumerate}
\solution
		%\begin{table}[H]
	\centering
\begin{tabular}{|c|c|c|}
\hline
Random variable &Value &Definition\\ \hline
\multirow{3}{*}{X} &0 &Slips of Rs 1\\
&1 &Slips of Rs 5\\
&2 &Slips of Rs 13\\ \hline
\multirow{2}{*}{Y} &0 &Box A\\
&1 &Box B\\\hline
\end{tabular}
\caption{}
\label{tab:Distribution}
\end{table}
See \tabref{tab:Distribution}.
\begin{align}
p_{Y}\brak{k}= \begin{cases} 
      \frac{1}{3} & {k=0} \\
      \frac{2}{3 }& {k=1} 
   \end{cases}
   \\
p_{Y|X}\brak{0|0} = \frac{19}{25}\, 
p_{Y|X}\brak{0|1} = \frac{6}{25}\,
p_{Y|X}\brak{1|0} = \frac{45}{50}\,
p_{Y|X}\brak{1|2} = \frac{5}{50}
\end{align}
The desired probability is the probability that a slip drawn at random is marked other than Rs 1,
\begin{align}
&=1-p_X\brak{0}\\
&= p_X(1) + p_X(2)
\end{align}
Using Bayes theorem,
\begin{align}
&= p_Y\brak{0} \times \pr{Y=0 | X=1} + p_Y\brak{1} \times \pr{Y=1|X=2}\\
&=\frac{1}{3} \times \frac{6}{25} + \frac{2}{3} \times \frac{5}{50}\\
&=\frac{11}{75}
\end{align}

\newpage

%\tableofcontents

\bigskip

\renewcommand{\thefigure}{\theenumi}
\renewcommand{\thetable}{\theenumi}
%\renewcommand{\theequation}{\theenumi}

%\begin{abstract}
%%\boldmath
%In this letter, an algorithm for evaluating the exact analytical bit error rate  (BER)  for the piecewise linear (PL) combiner for  multiple relays is presented. Previous results were available only for upto three relays. The algorithm is unique in the sense that  the actual mathematical expressions, that are prohibitively large, need not be explicitly obtained. The diversity gain due to multiple relays is shown through plots of the analytical BER, well supported by simulations. 
%
%\end{abstract}
% IEEEtran.cls defaults to using nonbold math in the Abstract.
% This preserves the distinction between vectors and scalars. However,
% if the journal you are submitting to favors bold math in the abstract,
% then you can use LaTeX's standard command \boldmath at the very start
% of the abstract to achieve this. Many IEEE journals frown on math
% in the abstract anyway.

% Note that keywords are not normally used for peerreview papers.
%\begin{IEEEkeywords}
%Cooperative diversity, decode and forward, piecewise linear
%\end{IEEEkeywords}



% For peer review papers, you can put extra information on the cover
% page as needed:
% \ifCLASSOPTIONpeerreview
% \begin{center} \bfseries EDICS Category: 3-BBND \end{center}
% \fi
%
% For peerreview papers, this IEEEtran command inserts a page break and
% creates the second title. It will be ignored for other modes.
%\IEEEpeerreviewmaketitle




	\item Five cards—the ten, jack, queen, king and ace of diamonds, are well-shuffled with their face downwards. One card is then picked up at random.
\begin{enumerate}
\item
What is the probability that the card is the queen? 
\item
If the queen is drawn and put aside, what is the probability that the second card picked up is (a) an ace? (b) a queen?\\
\end{enumerate}
\solution
		%\begin{enumerate}[label=\thesection.\arabic*,ref=\thesection.\theenumi]
	\item One card is drawn from a well-shuffled deck of 52 cards. Find the probability of getting
\begin{enumerate}
\item A king of red colour 
\item A face card 
\item A red face card
\item The jack of hearts
\item A spade
\item The queen of diamonds

\end{enumerate}
\solution
		%\input{ncert/10/15/1/14/main.tex}
	\item Five cards—the ten, jack, queen, king and ace of diamonds, are well-shuffled with their face downwards. One card is then picked up at random.
\begin{enumerate}
\item
What is the probability that the card is the queen? 
\item
If the queen is drawn and put aside, what is the probability that the second card picked up is (a) an ace? (b) a queen?\\
\end{enumerate}
\solution
		%\input{ncert/10/15/1/15/defs.tex}
	\item A bag contains $5$ red balls and some blue balls. If the probability of drawing a blue ball is double that if a red ball, determine the number of blue balls in the bag. 
		\\
\solution
		%\input{ncert/10/15/2/3/defs.tex}
	\item A card is selected from a pack of 52 cards.
 \begin{enumerate}[label=(\alph*)] 
                 \item How many points are there in the sample space?
                 \item Calculate the probability that the card is an ace of spades.
                 \item Calculate the probability that the card is (i) an ace and (ii) black card.
 \end{enumerate}
\solution
		%\input{ncert/11/16/3/4/main.tex}
\item Four cards are drawn from a well-shuffled deck of 52 cards. What is the probability of obtaining 3 diamonds and one spade.
\\
\solution
		%\input{ncert/11/16/4/2/defs.tex}
\item In a certain lottery 10,000 tickets are sold and ten equal prizes are awarded. What is the probability of not getting a prize if you buy (a) one ticket (b) two tickets (c) 10 tickets ?	
\\
\solution
		%\input{ncert/11/16/4/4/defs.tex}
		%
\item 
Out of 100 students, two sections of 40 and 60 are formed. If you and your friend are among the 100 students, what is the probability that
\begin{enumerate}
\item you both enter the same section?
\item you both enter the different sections?
\end{enumerate}
\solution
		%\input{ncert/11/16/4/5/defs.tex}
	\item 
The number lock of a suitcase has 4 wheels each labelled with ten digits i.e. from 0 to 9.The lock opens with a sequence of four digits with no repeats.What is the probability of a person getting the right sequence to open the suitcase.
\\
\solution
		%\input{ncert/11/16/4/10/defs.tex}
		%
\item 
Two cards are drawn at random and without replacement from a pack of 52 playing cards. Find the probability that both the cards are black.
\\
\solution
		%\input{ncert/12/13/2/2/defs.tex}
		\item A box of oranges is inspected by examining three randomly selected oranges drawn without replacement. If all the three oranges are good, the box is approved for sale, otherwise, it is rejected. Find the probability that a box containing 15 oranges out of which 12 are good and 3 are bad ones will be approved for sale.
		\label{ncert/12/13/2/3/defs.tex}
		\item Two balls are drawn at random with replacement from a box containing 10 black and 8 red balls. Find the probability that
		\label{ncert/12/13/2/12}
\begin{enumerate}
\item both balls are red.
\item first ball is black and second is red.
\item one of them is black and other is red.
\end{enumerate}

\item In a hostel, 60\% of the students read Hindi newspaper, 40\% read English newspaper and 20\% read both Hindi and English newspapers. A student is selected at random.
		\label{ncert/12/13/2/15}
\begin{enumerate}
\item Find the probability that she reads neither Hindi nor English newspapers.
\item If she reads Hindi newspaper, find the probability that she reads English newspaper.
\item If she reads English newspaper, find the probability that she reads Hindi newspaper.\\
\end{enumerate}
\item The probability of obtaining an even prime number on each die, when a pair of dice is rolled is 
\begin{enumerate}
    \item $0$ 
    
    \item $\frac{1}{3}$ 
    
    \item $\frac{1}{12}$ 
    
    \item $\frac{1}{36}$ 
\end{enumerate}
\solution
		%\input{ncert/12/13/2/17/defs.tex}
	\item A bag contains 4 red and 4 black balls, another bag contains 2 red and 6 black balls. One of the two bags is selected at random and a ball is drawn from the bag which is found to be red. Find the probability that the ball is drawn from the first bag.
\\
\solution
		%\input{ncert/12/13/3/2/main.tex}
  \item
  Cards with numbers 2 to 101 are placed in a box. A card is selected at random.Find the probability that the card has
\begin{enumerate}[label=(\roman*)]
	\item an even number 
	\item a square number
\end{enumerate}
\solution
%\input{exemplar/10/13/3/32/main.tex}
\item
The king, queen and jack of clubs are removed from a deck of 52 playing cards and then well shuffled. Now one card is drawn at random from the remaining cards.  Determine the probability that the card is
\begin{enumerate}[label=(\roman*)]
\item a club
\item 10 of hearts
\end{enumerate}
\solution
%\input{exemplar/10/13/3/29/main.tex}
\item A team of medical students doing their internship have to assist during surgeries
at a city hospital. The probabilities of surgeries rated as very complex, complex,
routine, simple or very simple are respectively, 0.15, 0.20, 0.31, 0.26, .08. Find
the probabilities that a particular surgery will be rated
\begin{enumerate}
	\item complex or very complex;
	\item neither very complex nor very simple;
	\item routine or complex
	\item routine or simple
\end{enumerate}
\solution
%\input{exemplar/11/16/3/8(1)/main.tex}
\item A card is selected from a pack of 52 cards.
\begin{enumerate}[label=(\alph*)]
    \item How many points are there in the sample space?
    \item Calculate the probability that the card is an ace of spades.
    \item Calculate the probability that the card is (i) an ace and (ii) black card.
\end{enumerate}
\solution
%\input{exemplar/11/16/3/4/main2.tex}
\item The probability that a non leap year selected at random will contain 53 sundays.
\\
\solution
%\input{exemplar/10/13/1/19/main.tex}
\item One of the four persons John, Rita, Aslam or Gurpreet will be promoted next
month. Consequently the sample space consists of four elementary outcomes
S = {John promoted, Rita promoted, Aslam promoted, Gurpreet promoted}
You are told that the chances of John’s promotion is same as that of Gurpreet,
Rita’s chances of promotion are twice as likely as Johns. Aslam’s chances are
four times that of John.
\begin{enumerate}
	\item Determine
	\begin{enumerate}
		\item P (John promoted)
		\item P (Rita promoted)
		\item P (Aslam promoted)
		\item P (Gurpreet promoted)
	\end{enumerate}
	\item If A = {John promoted or Gurpreet promoted}, find P (A).
\end{enumerate}
\solution
%\input{exemplar/11/16/3/10/main.tex}
\item A card is drawn from a deck of 52 cards. Find the probability of getting a king or a heart or a red card.\\
\solution
%\input{exemplar/11/16/3/15/main.tex}
\item The probability that a student will pass his examination is 0.73, the probability of
the student getting a compartment is 0.13, and the probability that the student will
either pass or get compartment is 0.96. State True or False.\\
\solution
%\input{exemplar/11/16/3/31/main.tex}
\item A card is selected from a pack of 52 cards\\
\begin{enumerate}[label=(\alph*)]
\item How many points are there in the sample space?
\item Calculate the probability that the cards is an ace of spades.
\item Calculate the probability that the card is (i) an ace (ii)black card.\\
\end{enumerate}
%\input{ncert/11/16/3/4_1/Prob_4.tex}
\item In a non-leap year, the probability of having 53 tuesdays or 53 wednesdays is\\
\solution
%\input{exemplar/11/16/3/18/main.tex}
\item There are 1000 sealed envelopes in a box, 10 of them contain a cash prize of
Rs 100 each, 100 of them contain a cash prize of Rs 50 each and 200 of them
contain a cash prize of Rs 10 each and rest do not contain any cash prize. If they
are well shuffled and an envelope is picked up out, what is the probability that it
contains no cash prize?\\
\solution
%\input{exemplar/10/13/3/34/main.tex}
\item 
A die is thrown and a card is selected at random from a deck of 52 playing cards. The probability of getting an even number on the die and a spade card.\\
\solution
%\input{exemplar/12/13/3/78/main.tex}
\item
If 4-digit numbers greater than 5,000 are randomly formed from the digits 0, 1, 3, 5, and 7, what is the probability of forming a number divisible by 5 when:
\begin{enumerate}
    \item The digits are repeated?
    \item The repetition of digits is not allowed?
\end{enumerate}
\solution
%\input{ncert/11/16/4/9/main.tex}
\item Consider the probability space $\brak{\Omega, \mathcal{G}, P}$ where $\Omega = [0,2]$ and $\mathcal{G} = \cbrak{\phi, \Omega, [0,1], (1,2]}$. Let $X$ and $Y$ be two functions on $\Omega$ defined as
\begin{align*}
    X(\omega) = 
    \begin{cases}
        1 & \text{if }\omega \in [0, 1]\\
        2 & \text{if }\omega \in (1, 2]
    \end{cases}
\end{align*}
and
\begin{align*}
    Y(\omega) = 
    \begin{cases}
        2 & \text{if }\omega \in [0, 1.5]\\
        3 & \text{if }\omega \in (1.5, 2].
    \end{cases}
\end{align*}
Then which one of the following statements is true?
\begin{enumerate}
    \item [(A)] $X$ is a random variable with respect to $\mathcal{G}$, but $Y$ is not a random variable with respect to $\mathcal{G}$.
    \item [(B)] $Y$ is a random variable with respect to $\mathcal{G}$, but $X$ is not a random variable with respect to $\mathcal{G}$.
    \item [(C)] Neither $X$ nor $Y$ is a random variable with respect to $\mathcal{G}$.
    \item [(D)] Both $X$ and $Y$ are random variables with respect to $\mathcal{G}$.
\end{enumerate} \hfill (GATE ST 2023)\\
\solution
%\input{gate/ST/2023/14/main.tex}
	\item  A die is loaded in such a way that each odd number is twice as likely to occur as
each even number. Find $P(G)$, where $G$ is the event that a number greater than
3 occurs on a single roll of the die.
\\
\solution
		%\input{exemplar/11/16/3/5/main.tex}
	\item All the jacks, queens and kings are removed from a deck of 52 playing cards. The remaining cards are well shuffled and then one card is drawn at random. Giving ace a value 1 similar value for other cards, find the probability that the card has a value 
		\begin{enumerate}
			\item 7
			\item greater than 7
			\item less than 7
		\end{enumerate}
		%\input{exemplar/10/13/3/30/main.tex}
  \item A Lot consists of 48 mobile phones of which 42 are good, 3 have only minor defects and 3 have major defects.Varnika will buy a phone if it is good but the trader will only buy a mobile if it has no major defects. One phone is selected at random from the lot. What is the probability that it is
\begin{enumerate}
	\item acceptable to Varnika?
            \item acceptable to the trader?
\end{enumerate}
\solution
	%\input{exemplar/10/13/3/40/main.tex}
 \item A student says that if you throw a die, it will show up 1 or not 1. Therefore, the probability of getting 1 and the probability of getting 'not 1' each is equal to $\frac{1}{2}$. Is this correct? Give reasons.\\
 \solution
        %\input{exemplar/10/13/2/9/main.tex}
   \item Four candidates A, B, C, D have ap-
plied for the assignment to coach a school cricket
team. If A is twice as likely to be selected as B, and
B and C are given about the same chance of being
selected, while C is twice as likely to be selected
as D, what are the probabilities that
\begin{enumerate}
\item C will be selected?
\item A will not be selected?
\end{enumerate}
	%\input{exemplar/11/16/3/9/main.tex}
 \item A bag contain 24 balls of which $x$ balls are red, $2x$ are white and $3x$ are blue. A ball is selected at random, What is the probability that it is
\begin{enumerate}[label=\alph*)]
\item not red ?
\item white ?
\end{enumerate}
%\input{exemplar/10/13/3/41/main.tex}
If the letters of the word ASSASSINATION are arranged at random. Find the Probability that
\begin{enumerate}[label=(\alph*)]
\item Four $S's$ come consecutively in the word
\item Two  $I's$ and two $N's$ come together
\item All $A's$ are not coming together
\item No two $A's$ are coming together
\end{enumerate}
%\input{exemplar/11/16/3/14/main.tex}
	\item One urn contains two black balls (labelled B1 and B2) and one white ball. A
	second urn contains one black ball and two white balls (labelled W1 and W2).
	Suppose the following experiment is performed. One of the two urns is chosen
	at random. Next a ball is randomly chosen from the urn. Then a second ball is
	chosen at random from the same urn without replacing the first ball.
	
	\begin{enumerate}
	\item What is the probability that two black balls are chosen?
	
	\item What is the probability that two balls of opposite colour are chosen?
	\end{enumerate}
	\solution
	%\input{exemplar/11/16/3/12/main1.tex}
\end{enumerate}

	\item A bag contains $5$ red balls and some blue balls. If the probability of drawing a blue ball is double that if a red ball, determine the number of blue balls in the bag. 
		\\
\solution
		%\begin{enumerate}[label=\thesection.\arabic*,ref=\thesection.\theenumi]
	\item One card is drawn from a well-shuffled deck of 52 cards. Find the probability of getting
\begin{enumerate}
\item A king of red colour 
\item A face card 
\item A red face card
\item The jack of hearts
\item A spade
\item The queen of diamonds

\end{enumerate}
\solution
		%\input{ncert/10/15/1/14/main.tex}
	\item Five cards—the ten, jack, queen, king and ace of diamonds, are well-shuffled with their face downwards. One card is then picked up at random.
\begin{enumerate}
\item
What is the probability that the card is the queen? 
\item
If the queen is drawn and put aside, what is the probability that the second card picked up is (a) an ace? (b) a queen?\\
\end{enumerate}
\solution
		%\input{ncert/10/15/1/15/defs.tex}
	\item A bag contains $5$ red balls and some blue balls. If the probability of drawing a blue ball is double that if a red ball, determine the number of blue balls in the bag. 
		\\
\solution
		%\input{ncert/10/15/2/3/defs.tex}
	\item A card is selected from a pack of 52 cards.
 \begin{enumerate}[label=(\alph*)] 
                 \item How many points are there in the sample space?
                 \item Calculate the probability that the card is an ace of spades.
                 \item Calculate the probability that the card is (i) an ace and (ii) black card.
 \end{enumerate}
\solution
		%\input{ncert/11/16/3/4/main.tex}
\item Four cards are drawn from a well-shuffled deck of 52 cards. What is the probability of obtaining 3 diamonds and one spade.
\\
\solution
		%\input{ncert/11/16/4/2/defs.tex}
\item In a certain lottery 10,000 tickets are sold and ten equal prizes are awarded. What is the probability of not getting a prize if you buy (a) one ticket (b) two tickets (c) 10 tickets ?	
\\
\solution
		%\input{ncert/11/16/4/4/defs.tex}
		%
\item 
Out of 100 students, two sections of 40 and 60 are formed. If you and your friend are among the 100 students, what is the probability that
\begin{enumerate}
\item you both enter the same section?
\item you both enter the different sections?
\end{enumerate}
\solution
		%\input{ncert/11/16/4/5/defs.tex}
	\item 
The number lock of a suitcase has 4 wheels each labelled with ten digits i.e. from 0 to 9.The lock opens with a sequence of four digits with no repeats.What is the probability of a person getting the right sequence to open the suitcase.
\\
\solution
		%\input{ncert/11/16/4/10/defs.tex}
		%
\item 
Two cards are drawn at random and without replacement from a pack of 52 playing cards. Find the probability that both the cards are black.
\\
\solution
		%\input{ncert/12/13/2/2/defs.tex}
		\item A box of oranges is inspected by examining three randomly selected oranges drawn without replacement. If all the three oranges are good, the box is approved for sale, otherwise, it is rejected. Find the probability that a box containing 15 oranges out of which 12 are good and 3 are bad ones will be approved for sale.
		\label{ncert/12/13/2/3/defs.tex}
		\item Two balls are drawn at random with replacement from a box containing 10 black and 8 red balls. Find the probability that
		\label{ncert/12/13/2/12}
\begin{enumerate}
\item both balls are red.
\item first ball is black and second is red.
\item one of them is black and other is red.
\end{enumerate}

\item In a hostel, 60\% of the students read Hindi newspaper, 40\% read English newspaper and 20\% read both Hindi and English newspapers. A student is selected at random.
		\label{ncert/12/13/2/15}
\begin{enumerate}
\item Find the probability that she reads neither Hindi nor English newspapers.
\item If she reads Hindi newspaper, find the probability that she reads English newspaper.
\item If she reads English newspaper, find the probability that she reads Hindi newspaper.\\
\end{enumerate}
\item The probability of obtaining an even prime number on each die, when a pair of dice is rolled is 
\begin{enumerate}
    \item $0$ 
    
    \item $\frac{1}{3}$ 
    
    \item $\frac{1}{12}$ 
    
    \item $\frac{1}{36}$ 
\end{enumerate}
\solution
		%\input{ncert/12/13/2/17/defs.tex}
	\item A bag contains 4 red and 4 black balls, another bag contains 2 red and 6 black balls. One of the two bags is selected at random and a ball is drawn from the bag which is found to be red. Find the probability that the ball is drawn from the first bag.
\\
\solution
		%\input{ncert/12/13/3/2/main.tex}
  \item
  Cards with numbers 2 to 101 are placed in a box. A card is selected at random.Find the probability that the card has
\begin{enumerate}[label=(\roman*)]
	\item an even number 
	\item a square number
\end{enumerate}
\solution
%\input{exemplar/10/13/3/32/main.tex}
\item
The king, queen and jack of clubs are removed from a deck of 52 playing cards and then well shuffled. Now one card is drawn at random from the remaining cards.  Determine the probability that the card is
\begin{enumerate}[label=(\roman*)]
\item a club
\item 10 of hearts
\end{enumerate}
\solution
%\input{exemplar/10/13/3/29/main.tex}
\item A team of medical students doing their internship have to assist during surgeries
at a city hospital. The probabilities of surgeries rated as very complex, complex,
routine, simple or very simple are respectively, 0.15, 0.20, 0.31, 0.26, .08. Find
the probabilities that a particular surgery will be rated
\begin{enumerate}
	\item complex or very complex;
	\item neither very complex nor very simple;
	\item routine or complex
	\item routine or simple
\end{enumerate}
\solution
%\input{exemplar/11/16/3/8(1)/main.tex}
\item A card is selected from a pack of 52 cards.
\begin{enumerate}[label=(\alph*)]
    \item How many points are there in the sample space?
    \item Calculate the probability that the card is an ace of spades.
    \item Calculate the probability that the card is (i) an ace and (ii) black card.
\end{enumerate}
\solution
%\input{exemplar/11/16/3/4/main2.tex}
\item The probability that a non leap year selected at random will contain 53 sundays.
\\
\solution
%\input{exemplar/10/13/1/19/main.tex}
\item One of the four persons John, Rita, Aslam or Gurpreet will be promoted next
month. Consequently the sample space consists of four elementary outcomes
S = {John promoted, Rita promoted, Aslam promoted, Gurpreet promoted}
You are told that the chances of John’s promotion is same as that of Gurpreet,
Rita’s chances of promotion are twice as likely as Johns. Aslam’s chances are
four times that of John.
\begin{enumerate}
	\item Determine
	\begin{enumerate}
		\item P (John promoted)
		\item P (Rita promoted)
		\item P (Aslam promoted)
		\item P (Gurpreet promoted)
	\end{enumerate}
	\item If A = {John promoted or Gurpreet promoted}, find P (A).
\end{enumerate}
\solution
%\input{exemplar/11/16/3/10/main.tex}
\item A card is drawn from a deck of 52 cards. Find the probability of getting a king or a heart or a red card.\\
\solution
%\input{exemplar/11/16/3/15/main.tex}
\item The probability that a student will pass his examination is 0.73, the probability of
the student getting a compartment is 0.13, and the probability that the student will
either pass or get compartment is 0.96. State True or False.\\
\solution
%\input{exemplar/11/16/3/31/main.tex}
\item A card is selected from a pack of 52 cards\\
\begin{enumerate}[label=(\alph*)]
\item How many points are there in the sample space?
\item Calculate the probability that the cards is an ace of spades.
\item Calculate the probability that the card is (i) an ace (ii)black card.\\
\end{enumerate}
%\input{ncert/11/16/3/4_1/Prob_4.tex}
\item In a non-leap year, the probability of having 53 tuesdays or 53 wednesdays is\\
\solution
%\input{exemplar/11/16/3/18/main.tex}
\item There are 1000 sealed envelopes in a box, 10 of them contain a cash prize of
Rs 100 each, 100 of them contain a cash prize of Rs 50 each and 200 of them
contain a cash prize of Rs 10 each and rest do not contain any cash prize. If they
are well shuffled and an envelope is picked up out, what is the probability that it
contains no cash prize?\\
\solution
%\input{exemplar/10/13/3/34/main.tex}
\item 
A die is thrown and a card is selected at random from a deck of 52 playing cards. The probability of getting an even number on the die and a spade card.\\
\solution
%\input{exemplar/12/13/3/78/main.tex}
\item
If 4-digit numbers greater than 5,000 are randomly formed from the digits 0, 1, 3, 5, and 7, what is the probability of forming a number divisible by 5 when:
\begin{enumerate}
    \item The digits are repeated?
    \item The repetition of digits is not allowed?
\end{enumerate}
\solution
%\input{ncert/11/16/4/9/main.tex}
\item Consider the probability space $\brak{\Omega, \mathcal{G}, P}$ where $\Omega = [0,2]$ and $\mathcal{G} = \cbrak{\phi, \Omega, [0,1], (1,2]}$. Let $X$ and $Y$ be two functions on $\Omega$ defined as
\begin{align*}
    X(\omega) = 
    \begin{cases}
        1 & \text{if }\omega \in [0, 1]\\
        2 & \text{if }\omega \in (1, 2]
    \end{cases}
\end{align*}
and
\begin{align*}
    Y(\omega) = 
    \begin{cases}
        2 & \text{if }\omega \in [0, 1.5]\\
        3 & \text{if }\omega \in (1.5, 2].
    \end{cases}
\end{align*}
Then which one of the following statements is true?
\begin{enumerate}
    \item [(A)] $X$ is a random variable with respect to $\mathcal{G}$, but $Y$ is not a random variable with respect to $\mathcal{G}$.
    \item [(B)] $Y$ is a random variable with respect to $\mathcal{G}$, but $X$ is not a random variable with respect to $\mathcal{G}$.
    \item [(C)] Neither $X$ nor $Y$ is a random variable with respect to $\mathcal{G}$.
    \item [(D)] Both $X$ and $Y$ are random variables with respect to $\mathcal{G}$.
\end{enumerate} \hfill (GATE ST 2023)\\
\solution
%\input{gate/ST/2023/14/main.tex}
	\item  A die is loaded in such a way that each odd number is twice as likely to occur as
each even number. Find $P(G)$, where $G$ is the event that a number greater than
3 occurs on a single roll of the die.
\\
\solution
		%\input{exemplar/11/16/3/5/main.tex}
	\item All the jacks, queens and kings are removed from a deck of 52 playing cards. The remaining cards are well shuffled and then one card is drawn at random. Giving ace a value 1 similar value for other cards, find the probability that the card has a value 
		\begin{enumerate}
			\item 7
			\item greater than 7
			\item less than 7
		\end{enumerate}
		%\input{exemplar/10/13/3/30/main.tex}
  \item A Lot consists of 48 mobile phones of which 42 are good, 3 have only minor defects and 3 have major defects.Varnika will buy a phone if it is good but the trader will only buy a mobile if it has no major defects. One phone is selected at random from the lot. What is the probability that it is
\begin{enumerate}
	\item acceptable to Varnika?
            \item acceptable to the trader?
\end{enumerate}
\solution
	%\input{exemplar/10/13/3/40/main.tex}
 \item A student says that if you throw a die, it will show up 1 or not 1. Therefore, the probability of getting 1 and the probability of getting 'not 1' each is equal to $\frac{1}{2}$. Is this correct? Give reasons.\\
 \solution
        %\input{exemplar/10/13/2/9/main.tex}
   \item Four candidates A, B, C, D have ap-
plied for the assignment to coach a school cricket
team. If A is twice as likely to be selected as B, and
B and C are given about the same chance of being
selected, while C is twice as likely to be selected
as D, what are the probabilities that
\begin{enumerate}
\item C will be selected?
\item A will not be selected?
\end{enumerate}
	%\input{exemplar/11/16/3/9/main.tex}
 \item A bag contain 24 balls of which $x$ balls are red, $2x$ are white and $3x$ are blue. A ball is selected at random, What is the probability that it is
\begin{enumerate}[label=\alph*)]
\item not red ?
\item white ?
\end{enumerate}
%\input{exemplar/10/13/3/41/main.tex}
If the letters of the word ASSASSINATION are arranged at random. Find the Probability that
\begin{enumerate}[label=(\alph*)]
\item Four $S's$ come consecutively in the word
\item Two  $I's$ and two $N's$ come together
\item All $A's$ are not coming together
\item No two $A's$ are coming together
\end{enumerate}
%\input{exemplar/11/16/3/14/main.tex}
	\item One urn contains two black balls (labelled B1 and B2) and one white ball. A
	second urn contains one black ball and two white balls (labelled W1 and W2).
	Suppose the following experiment is performed. One of the two urns is chosen
	at random. Next a ball is randomly chosen from the urn. Then a second ball is
	chosen at random from the same urn without replacing the first ball.
	
	\begin{enumerate}
	\item What is the probability that two black balls are chosen?
	
	\item What is the probability that two balls of opposite colour are chosen?
	\end{enumerate}
	\solution
	%\input{exemplar/11/16/3/12/main1.tex}
\end{enumerate}

	\item A card is selected from a pack of 52 cards.
 \begin{enumerate}[label=(\alph*)] 
                 \item How many points are there in the sample space?
                 \item Calculate the probability that the card is an ace of spades.
                 \item Calculate the probability that the card is (i) an ace and (ii) black card.
 \end{enumerate}
\solution
		%\begin{table}[H]
	\centering
\begin{tabular}{|c|c|c|}
\hline
Random variable &Value &Definition\\ \hline
\multirow{3}{*}{X} &0 &Slips of Rs 1\\
&1 &Slips of Rs 5\\
&2 &Slips of Rs 13\\ \hline
\multirow{2}{*}{Y} &0 &Box A\\
&1 &Box B\\\hline
\end{tabular}
\caption{}
\label{tab:Distribution}
\end{table}
See \tabref{tab:Distribution}.
\begin{align}
p_{Y}\brak{k}= \begin{cases} 
      \frac{1}{3} & {k=0} \\
      \frac{2}{3 }& {k=1} 
   \end{cases}
   \\
p_{Y|X}\brak{0|0} = \frac{19}{25}\, 
p_{Y|X}\brak{0|1} = \frac{6}{25}\,
p_{Y|X}\brak{1|0} = \frac{45}{50}\,
p_{Y|X}\brak{1|2} = \frac{5}{50}
\end{align}
The desired probability is the probability that a slip drawn at random is marked other than Rs 1,
\begin{align}
&=1-p_X\brak{0}\\
&= p_X(1) + p_X(2)
\end{align}
Using Bayes theorem,
\begin{align}
&= p_Y\brak{0} \times \pr{Y=0 | X=1} + p_Y\brak{1} \times \pr{Y=1|X=2}\\
&=\frac{1}{3} \times \frac{6}{25} + \frac{2}{3} \times \frac{5}{50}\\
&=\frac{11}{75}
\end{align}

\newpage

%\tableofcontents

\bigskip

\renewcommand{\thefigure}{\theenumi}
\renewcommand{\thetable}{\theenumi}
%\renewcommand{\theequation}{\theenumi}

%\begin{abstract}
%%\boldmath
%In this letter, an algorithm for evaluating the exact analytical bit error rate  (BER)  for the piecewise linear (PL) combiner for  multiple relays is presented. Previous results were available only for upto three relays. The algorithm is unique in the sense that  the actual mathematical expressions, that are prohibitively large, need not be explicitly obtained. The diversity gain due to multiple relays is shown through plots of the analytical BER, well supported by simulations. 
%
%\end{abstract}
% IEEEtran.cls defaults to using nonbold math in the Abstract.
% This preserves the distinction between vectors and scalars. However,
% if the journal you are submitting to favors bold math in the abstract,
% then you can use LaTeX's standard command \boldmath at the very start
% of the abstract to achieve this. Many IEEE journals frown on math
% in the abstract anyway.

% Note that keywords are not normally used for peerreview papers.
%\begin{IEEEkeywords}
%Cooperative diversity, decode and forward, piecewise linear
%\end{IEEEkeywords}



% For peer review papers, you can put extra information on the cover
% page as needed:
% \ifCLASSOPTIONpeerreview
% \begin{center} \bfseries EDICS Category: 3-BBND \end{center}
% \fi
%
% For peerreview papers, this IEEEtran command inserts a page break and
% creates the second title. It will be ignored for other modes.
%\IEEEpeerreviewmaketitle




\item Four cards are drawn from a well-shuffled deck of 52 cards. What is the probability of obtaining 3 diamonds and one spade.
\\
\solution
		%\begin{enumerate}[label=\thesection.\arabic*,ref=\thesection.\theenumi]
	\item One card is drawn from a well-shuffled deck of 52 cards. Find the probability of getting
\begin{enumerate}
\item A king of red colour 
\item A face card 
\item A red face card
\item The jack of hearts
\item A spade
\item The queen of diamonds

\end{enumerate}
\solution
		%\input{ncert/10/15/1/14/main.tex}
	\item Five cards—the ten, jack, queen, king and ace of diamonds, are well-shuffled with their face downwards. One card is then picked up at random.
\begin{enumerate}
\item
What is the probability that the card is the queen? 
\item
If the queen is drawn and put aside, what is the probability that the second card picked up is (a) an ace? (b) a queen?\\
\end{enumerate}
\solution
		%\input{ncert/10/15/1/15/defs.tex}
	\item A bag contains $5$ red balls and some blue balls. If the probability of drawing a blue ball is double that if a red ball, determine the number of blue balls in the bag. 
		\\
\solution
		%\input{ncert/10/15/2/3/defs.tex}
	\item A card is selected from a pack of 52 cards.
 \begin{enumerate}[label=(\alph*)] 
                 \item How many points are there in the sample space?
                 \item Calculate the probability that the card is an ace of spades.
                 \item Calculate the probability that the card is (i) an ace and (ii) black card.
 \end{enumerate}
\solution
		%\input{ncert/11/16/3/4/main.tex}
\item Four cards are drawn from a well-shuffled deck of 52 cards. What is the probability of obtaining 3 diamonds and one spade.
\\
\solution
		%\input{ncert/11/16/4/2/defs.tex}
\item In a certain lottery 10,000 tickets are sold and ten equal prizes are awarded. What is the probability of not getting a prize if you buy (a) one ticket (b) two tickets (c) 10 tickets ?	
\\
\solution
		%\input{ncert/11/16/4/4/defs.tex}
		%
\item 
Out of 100 students, two sections of 40 and 60 are formed. If you and your friend are among the 100 students, what is the probability that
\begin{enumerate}
\item you both enter the same section?
\item you both enter the different sections?
\end{enumerate}
\solution
		%\input{ncert/11/16/4/5/defs.tex}
	\item 
The number lock of a suitcase has 4 wheels each labelled with ten digits i.e. from 0 to 9.The lock opens with a sequence of four digits with no repeats.What is the probability of a person getting the right sequence to open the suitcase.
\\
\solution
		%\input{ncert/11/16/4/10/defs.tex}
		%
\item 
Two cards are drawn at random and without replacement from a pack of 52 playing cards. Find the probability that both the cards are black.
\\
\solution
		%\input{ncert/12/13/2/2/defs.tex}
		\item A box of oranges is inspected by examining three randomly selected oranges drawn without replacement. If all the three oranges are good, the box is approved for sale, otherwise, it is rejected. Find the probability that a box containing 15 oranges out of which 12 are good and 3 are bad ones will be approved for sale.
		\label{ncert/12/13/2/3/defs.tex}
		\item Two balls are drawn at random with replacement from a box containing 10 black and 8 red balls. Find the probability that
		\label{ncert/12/13/2/12}
\begin{enumerate}
\item both balls are red.
\item first ball is black and second is red.
\item one of them is black and other is red.
\end{enumerate}

\item In a hostel, 60\% of the students read Hindi newspaper, 40\% read English newspaper and 20\% read both Hindi and English newspapers. A student is selected at random.
		\label{ncert/12/13/2/15}
\begin{enumerate}
\item Find the probability that she reads neither Hindi nor English newspapers.
\item If she reads Hindi newspaper, find the probability that she reads English newspaper.
\item If she reads English newspaper, find the probability that she reads Hindi newspaper.\\
\end{enumerate}
\item The probability of obtaining an even prime number on each die, when a pair of dice is rolled is 
\begin{enumerate}
    \item $0$ 
    
    \item $\frac{1}{3}$ 
    
    \item $\frac{1}{12}$ 
    
    \item $\frac{1}{36}$ 
\end{enumerate}
\solution
		%\input{ncert/12/13/2/17/defs.tex}
	\item A bag contains 4 red and 4 black balls, another bag contains 2 red and 6 black balls. One of the two bags is selected at random and a ball is drawn from the bag which is found to be red. Find the probability that the ball is drawn from the first bag.
\\
\solution
		%\input{ncert/12/13/3/2/main.tex}
  \item
  Cards with numbers 2 to 101 are placed in a box. A card is selected at random.Find the probability that the card has
\begin{enumerate}[label=(\roman*)]
	\item an even number 
	\item a square number
\end{enumerate}
\solution
%\input{exemplar/10/13/3/32/main.tex}
\item
The king, queen and jack of clubs are removed from a deck of 52 playing cards and then well shuffled. Now one card is drawn at random from the remaining cards.  Determine the probability that the card is
\begin{enumerate}[label=(\roman*)]
\item a club
\item 10 of hearts
\end{enumerate}
\solution
%\input{exemplar/10/13/3/29/main.tex}
\item A team of medical students doing their internship have to assist during surgeries
at a city hospital. The probabilities of surgeries rated as very complex, complex,
routine, simple or very simple are respectively, 0.15, 0.20, 0.31, 0.26, .08. Find
the probabilities that a particular surgery will be rated
\begin{enumerate}
	\item complex or very complex;
	\item neither very complex nor very simple;
	\item routine or complex
	\item routine or simple
\end{enumerate}
\solution
%\input{exemplar/11/16/3/8(1)/main.tex}
\item A card is selected from a pack of 52 cards.
\begin{enumerate}[label=(\alph*)]
    \item How many points are there in the sample space?
    \item Calculate the probability that the card is an ace of spades.
    \item Calculate the probability that the card is (i) an ace and (ii) black card.
\end{enumerate}
\solution
%\input{exemplar/11/16/3/4/main2.tex}
\item The probability that a non leap year selected at random will contain 53 sundays.
\\
\solution
%\input{exemplar/10/13/1/19/main.tex}
\item One of the four persons John, Rita, Aslam or Gurpreet will be promoted next
month. Consequently the sample space consists of four elementary outcomes
S = {John promoted, Rita promoted, Aslam promoted, Gurpreet promoted}
You are told that the chances of John’s promotion is same as that of Gurpreet,
Rita’s chances of promotion are twice as likely as Johns. Aslam’s chances are
four times that of John.
\begin{enumerate}
	\item Determine
	\begin{enumerate}
		\item P (John promoted)
		\item P (Rita promoted)
		\item P (Aslam promoted)
		\item P (Gurpreet promoted)
	\end{enumerate}
	\item If A = {John promoted or Gurpreet promoted}, find P (A).
\end{enumerate}
\solution
%\input{exemplar/11/16/3/10/main.tex}
\item A card is drawn from a deck of 52 cards. Find the probability of getting a king or a heart or a red card.\\
\solution
%\input{exemplar/11/16/3/15/main.tex}
\item The probability that a student will pass his examination is 0.73, the probability of
the student getting a compartment is 0.13, and the probability that the student will
either pass or get compartment is 0.96. State True or False.\\
\solution
%\input{exemplar/11/16/3/31/main.tex}
\item A card is selected from a pack of 52 cards\\
\begin{enumerate}[label=(\alph*)]
\item How many points are there in the sample space?
\item Calculate the probability that the cards is an ace of spades.
\item Calculate the probability that the card is (i) an ace (ii)black card.\\
\end{enumerate}
%\input{ncert/11/16/3/4_1/Prob_4.tex}
\item In a non-leap year, the probability of having 53 tuesdays or 53 wednesdays is\\
\solution
%\input{exemplar/11/16/3/18/main.tex}
\item There are 1000 sealed envelopes in a box, 10 of them contain a cash prize of
Rs 100 each, 100 of them contain a cash prize of Rs 50 each and 200 of them
contain a cash prize of Rs 10 each and rest do not contain any cash prize. If they
are well shuffled and an envelope is picked up out, what is the probability that it
contains no cash prize?\\
\solution
%\input{exemplar/10/13/3/34/main.tex}
\item 
A die is thrown and a card is selected at random from a deck of 52 playing cards. The probability of getting an even number on the die and a spade card.\\
\solution
%\input{exemplar/12/13/3/78/main.tex}
\item
If 4-digit numbers greater than 5,000 are randomly formed from the digits 0, 1, 3, 5, and 7, what is the probability of forming a number divisible by 5 when:
\begin{enumerate}
    \item The digits are repeated?
    \item The repetition of digits is not allowed?
\end{enumerate}
\solution
%\input{ncert/11/16/4/9/main.tex}
\item Consider the probability space $\brak{\Omega, \mathcal{G}, P}$ where $\Omega = [0,2]$ and $\mathcal{G} = \cbrak{\phi, \Omega, [0,1], (1,2]}$. Let $X$ and $Y$ be two functions on $\Omega$ defined as
\begin{align*}
    X(\omega) = 
    \begin{cases}
        1 & \text{if }\omega \in [0, 1]\\
        2 & \text{if }\omega \in (1, 2]
    \end{cases}
\end{align*}
and
\begin{align*}
    Y(\omega) = 
    \begin{cases}
        2 & \text{if }\omega \in [0, 1.5]\\
        3 & \text{if }\omega \in (1.5, 2].
    \end{cases}
\end{align*}
Then which one of the following statements is true?
\begin{enumerate}
    \item [(A)] $X$ is a random variable with respect to $\mathcal{G}$, but $Y$ is not a random variable with respect to $\mathcal{G}$.
    \item [(B)] $Y$ is a random variable with respect to $\mathcal{G}$, but $X$ is not a random variable with respect to $\mathcal{G}$.
    \item [(C)] Neither $X$ nor $Y$ is a random variable with respect to $\mathcal{G}$.
    \item [(D)] Both $X$ and $Y$ are random variables with respect to $\mathcal{G}$.
\end{enumerate} \hfill (GATE ST 2023)\\
\solution
%\input{gate/ST/2023/14/main.tex}
	\item  A die is loaded in such a way that each odd number is twice as likely to occur as
each even number. Find $P(G)$, where $G$ is the event that a number greater than
3 occurs on a single roll of the die.
\\
\solution
		%\input{exemplar/11/16/3/5/main.tex}
	\item All the jacks, queens and kings are removed from a deck of 52 playing cards. The remaining cards are well shuffled and then one card is drawn at random. Giving ace a value 1 similar value for other cards, find the probability that the card has a value 
		\begin{enumerate}
			\item 7
			\item greater than 7
			\item less than 7
		\end{enumerate}
		%\input{exemplar/10/13/3/30/main.tex}
  \item A Lot consists of 48 mobile phones of which 42 are good, 3 have only minor defects and 3 have major defects.Varnika will buy a phone if it is good but the trader will only buy a mobile if it has no major defects. One phone is selected at random from the lot. What is the probability that it is
\begin{enumerate}
	\item acceptable to Varnika?
            \item acceptable to the trader?
\end{enumerate}
\solution
	%\input{exemplar/10/13/3/40/main.tex}
 \item A student says that if you throw a die, it will show up 1 or not 1. Therefore, the probability of getting 1 and the probability of getting 'not 1' each is equal to $\frac{1}{2}$. Is this correct? Give reasons.\\
 \solution
        %\input{exemplar/10/13/2/9/main.tex}
   \item Four candidates A, B, C, D have ap-
plied for the assignment to coach a school cricket
team. If A is twice as likely to be selected as B, and
B and C are given about the same chance of being
selected, while C is twice as likely to be selected
as D, what are the probabilities that
\begin{enumerate}
\item C will be selected?
\item A will not be selected?
\end{enumerate}
	%\input{exemplar/11/16/3/9/main.tex}
 \item A bag contain 24 balls of which $x$ balls are red, $2x$ are white and $3x$ are blue. A ball is selected at random, What is the probability that it is
\begin{enumerate}[label=\alph*)]
\item not red ?
\item white ?
\end{enumerate}
%\input{exemplar/10/13/3/41/main.tex}
If the letters of the word ASSASSINATION are arranged at random. Find the Probability that
\begin{enumerate}[label=(\alph*)]
\item Four $S's$ come consecutively in the word
\item Two  $I's$ and two $N's$ come together
\item All $A's$ are not coming together
\item No two $A's$ are coming together
\end{enumerate}
%\input{exemplar/11/16/3/14/main.tex}
	\item One urn contains two black balls (labelled B1 and B2) and one white ball. A
	second urn contains one black ball and two white balls (labelled W1 and W2).
	Suppose the following experiment is performed. One of the two urns is chosen
	at random. Next a ball is randomly chosen from the urn. Then a second ball is
	chosen at random from the same urn without replacing the first ball.
	
	\begin{enumerate}
	\item What is the probability that two black balls are chosen?
	
	\item What is the probability that two balls of opposite colour are chosen?
	\end{enumerate}
	\solution
	%\input{exemplar/11/16/3/12/main1.tex}
\end{enumerate}

\item In a certain lottery 10,000 tickets are sold and ten equal prizes are awarded. What is the probability of not getting a prize if you buy (a) one ticket (b) two tickets (c) 10 tickets ?	
\\
\solution
		%\begin{enumerate}[label=\thesection.\arabic*,ref=\thesection.\theenumi]
	\item One card is drawn from a well-shuffled deck of 52 cards. Find the probability of getting
\begin{enumerate}
\item A king of red colour 
\item A face card 
\item A red face card
\item The jack of hearts
\item A spade
\item The queen of diamonds

\end{enumerate}
\solution
		%\input{ncert/10/15/1/14/main.tex}
	\item Five cards—the ten, jack, queen, king and ace of diamonds, are well-shuffled with their face downwards. One card is then picked up at random.
\begin{enumerate}
\item
What is the probability that the card is the queen? 
\item
If the queen is drawn and put aside, what is the probability that the second card picked up is (a) an ace? (b) a queen?\\
\end{enumerate}
\solution
		%\input{ncert/10/15/1/15/defs.tex}
	\item A bag contains $5$ red balls and some blue balls. If the probability of drawing a blue ball is double that if a red ball, determine the number of blue balls in the bag. 
		\\
\solution
		%\input{ncert/10/15/2/3/defs.tex}
	\item A card is selected from a pack of 52 cards.
 \begin{enumerate}[label=(\alph*)] 
                 \item How many points are there in the sample space?
                 \item Calculate the probability that the card is an ace of spades.
                 \item Calculate the probability that the card is (i) an ace and (ii) black card.
 \end{enumerate}
\solution
		%\input{ncert/11/16/3/4/main.tex}
\item Four cards are drawn from a well-shuffled deck of 52 cards. What is the probability of obtaining 3 diamonds and one spade.
\\
\solution
		%\input{ncert/11/16/4/2/defs.tex}
\item In a certain lottery 10,000 tickets are sold and ten equal prizes are awarded. What is the probability of not getting a prize if you buy (a) one ticket (b) two tickets (c) 10 tickets ?	
\\
\solution
		%\input{ncert/11/16/4/4/defs.tex}
		%
\item 
Out of 100 students, two sections of 40 and 60 are formed. If you and your friend are among the 100 students, what is the probability that
\begin{enumerate}
\item you both enter the same section?
\item you both enter the different sections?
\end{enumerate}
\solution
		%\input{ncert/11/16/4/5/defs.tex}
	\item 
The number lock of a suitcase has 4 wheels each labelled with ten digits i.e. from 0 to 9.The lock opens with a sequence of four digits with no repeats.What is the probability of a person getting the right sequence to open the suitcase.
\\
\solution
		%\input{ncert/11/16/4/10/defs.tex}
		%
\item 
Two cards are drawn at random and without replacement from a pack of 52 playing cards. Find the probability that both the cards are black.
\\
\solution
		%\input{ncert/12/13/2/2/defs.tex}
		\item A box of oranges is inspected by examining three randomly selected oranges drawn without replacement. If all the three oranges are good, the box is approved for sale, otherwise, it is rejected. Find the probability that a box containing 15 oranges out of which 12 are good and 3 are bad ones will be approved for sale.
		\label{ncert/12/13/2/3/defs.tex}
		\item Two balls are drawn at random with replacement from a box containing 10 black and 8 red balls. Find the probability that
		\label{ncert/12/13/2/12}
\begin{enumerate}
\item both balls are red.
\item first ball is black and second is red.
\item one of them is black and other is red.
\end{enumerate}

\item In a hostel, 60\% of the students read Hindi newspaper, 40\% read English newspaper and 20\% read both Hindi and English newspapers. A student is selected at random.
		\label{ncert/12/13/2/15}
\begin{enumerate}
\item Find the probability that she reads neither Hindi nor English newspapers.
\item If she reads Hindi newspaper, find the probability that she reads English newspaper.
\item If she reads English newspaper, find the probability that she reads Hindi newspaper.\\
\end{enumerate}
\item The probability of obtaining an even prime number on each die, when a pair of dice is rolled is 
\begin{enumerate}
    \item $0$ 
    
    \item $\frac{1}{3}$ 
    
    \item $\frac{1}{12}$ 
    
    \item $\frac{1}{36}$ 
\end{enumerate}
\solution
		%\input{ncert/12/13/2/17/defs.tex}
	\item A bag contains 4 red and 4 black balls, another bag contains 2 red and 6 black balls. One of the two bags is selected at random and a ball is drawn from the bag which is found to be red. Find the probability that the ball is drawn from the first bag.
\\
\solution
		%\input{ncert/12/13/3/2/main.tex}
  \item
  Cards with numbers 2 to 101 are placed in a box. A card is selected at random.Find the probability that the card has
\begin{enumerate}[label=(\roman*)]
	\item an even number 
	\item a square number
\end{enumerate}
\solution
%\input{exemplar/10/13/3/32/main.tex}
\item
The king, queen and jack of clubs are removed from a deck of 52 playing cards and then well shuffled. Now one card is drawn at random from the remaining cards.  Determine the probability that the card is
\begin{enumerate}[label=(\roman*)]
\item a club
\item 10 of hearts
\end{enumerate}
\solution
%\input{exemplar/10/13/3/29/main.tex}
\item A team of medical students doing their internship have to assist during surgeries
at a city hospital. The probabilities of surgeries rated as very complex, complex,
routine, simple or very simple are respectively, 0.15, 0.20, 0.31, 0.26, .08. Find
the probabilities that a particular surgery will be rated
\begin{enumerate}
	\item complex or very complex;
	\item neither very complex nor very simple;
	\item routine or complex
	\item routine or simple
\end{enumerate}
\solution
%\input{exemplar/11/16/3/8(1)/main.tex}
\item A card is selected from a pack of 52 cards.
\begin{enumerate}[label=(\alph*)]
    \item How many points are there in the sample space?
    \item Calculate the probability that the card is an ace of spades.
    \item Calculate the probability that the card is (i) an ace and (ii) black card.
\end{enumerate}
\solution
%\input{exemplar/11/16/3/4/main2.tex}
\item The probability that a non leap year selected at random will contain 53 sundays.
\\
\solution
%\input{exemplar/10/13/1/19/main.tex}
\item One of the four persons John, Rita, Aslam or Gurpreet will be promoted next
month. Consequently the sample space consists of four elementary outcomes
S = {John promoted, Rita promoted, Aslam promoted, Gurpreet promoted}
You are told that the chances of John’s promotion is same as that of Gurpreet,
Rita’s chances of promotion are twice as likely as Johns. Aslam’s chances are
four times that of John.
\begin{enumerate}
	\item Determine
	\begin{enumerate}
		\item P (John promoted)
		\item P (Rita promoted)
		\item P (Aslam promoted)
		\item P (Gurpreet promoted)
	\end{enumerate}
	\item If A = {John promoted or Gurpreet promoted}, find P (A).
\end{enumerate}
\solution
%\input{exemplar/11/16/3/10/main.tex}
\item A card is drawn from a deck of 52 cards. Find the probability of getting a king or a heart or a red card.\\
\solution
%\input{exemplar/11/16/3/15/main.tex}
\item The probability that a student will pass his examination is 0.73, the probability of
the student getting a compartment is 0.13, and the probability that the student will
either pass or get compartment is 0.96. State True or False.\\
\solution
%\input{exemplar/11/16/3/31/main.tex}
\item A card is selected from a pack of 52 cards\\
\begin{enumerate}[label=(\alph*)]
\item How many points are there in the sample space?
\item Calculate the probability that the cards is an ace of spades.
\item Calculate the probability that the card is (i) an ace (ii)black card.\\
\end{enumerate}
%\input{ncert/11/16/3/4_1/Prob_4.tex}
\item In a non-leap year, the probability of having 53 tuesdays or 53 wednesdays is\\
\solution
%\input{exemplar/11/16/3/18/main.tex}
\item There are 1000 sealed envelopes in a box, 10 of them contain a cash prize of
Rs 100 each, 100 of them contain a cash prize of Rs 50 each and 200 of them
contain a cash prize of Rs 10 each and rest do not contain any cash prize. If they
are well shuffled and an envelope is picked up out, what is the probability that it
contains no cash prize?\\
\solution
%\input{exemplar/10/13/3/34/main.tex}
\item 
A die is thrown and a card is selected at random from a deck of 52 playing cards. The probability of getting an even number on the die and a spade card.\\
\solution
%\input{exemplar/12/13/3/78/main.tex}
\item
If 4-digit numbers greater than 5,000 are randomly formed from the digits 0, 1, 3, 5, and 7, what is the probability of forming a number divisible by 5 when:
\begin{enumerate}
    \item The digits are repeated?
    \item The repetition of digits is not allowed?
\end{enumerate}
\solution
%\input{ncert/11/16/4/9/main.tex}
\item Consider the probability space $\brak{\Omega, \mathcal{G}, P}$ where $\Omega = [0,2]$ and $\mathcal{G} = \cbrak{\phi, \Omega, [0,1], (1,2]}$. Let $X$ and $Y$ be two functions on $\Omega$ defined as
\begin{align*}
    X(\omega) = 
    \begin{cases}
        1 & \text{if }\omega \in [0, 1]\\
        2 & \text{if }\omega \in (1, 2]
    \end{cases}
\end{align*}
and
\begin{align*}
    Y(\omega) = 
    \begin{cases}
        2 & \text{if }\omega \in [0, 1.5]\\
        3 & \text{if }\omega \in (1.5, 2].
    \end{cases}
\end{align*}
Then which one of the following statements is true?
\begin{enumerate}
    \item [(A)] $X$ is a random variable with respect to $\mathcal{G}$, but $Y$ is not a random variable with respect to $\mathcal{G}$.
    \item [(B)] $Y$ is a random variable with respect to $\mathcal{G}$, but $X$ is not a random variable with respect to $\mathcal{G}$.
    \item [(C)] Neither $X$ nor $Y$ is a random variable with respect to $\mathcal{G}$.
    \item [(D)] Both $X$ and $Y$ are random variables with respect to $\mathcal{G}$.
\end{enumerate} \hfill (GATE ST 2023)\\
\solution
%\input{gate/ST/2023/14/main.tex}
	\item  A die is loaded in such a way that each odd number is twice as likely to occur as
each even number. Find $P(G)$, where $G$ is the event that a number greater than
3 occurs on a single roll of the die.
\\
\solution
		%\input{exemplar/11/16/3/5/main.tex}
	\item All the jacks, queens and kings are removed from a deck of 52 playing cards. The remaining cards are well shuffled and then one card is drawn at random. Giving ace a value 1 similar value for other cards, find the probability that the card has a value 
		\begin{enumerate}
			\item 7
			\item greater than 7
			\item less than 7
		\end{enumerate}
		%\input{exemplar/10/13/3/30/main.tex}
  \item A Lot consists of 48 mobile phones of which 42 are good, 3 have only minor defects and 3 have major defects.Varnika will buy a phone if it is good but the trader will only buy a mobile if it has no major defects. One phone is selected at random from the lot. What is the probability that it is
\begin{enumerate}
	\item acceptable to Varnika?
            \item acceptable to the trader?
\end{enumerate}
\solution
	%\input{exemplar/10/13/3/40/main.tex}
 \item A student says that if you throw a die, it will show up 1 or not 1. Therefore, the probability of getting 1 and the probability of getting 'not 1' each is equal to $\frac{1}{2}$. Is this correct? Give reasons.\\
 \solution
        %\input{exemplar/10/13/2/9/main.tex}
   \item Four candidates A, B, C, D have ap-
plied for the assignment to coach a school cricket
team. If A is twice as likely to be selected as B, and
B and C are given about the same chance of being
selected, while C is twice as likely to be selected
as D, what are the probabilities that
\begin{enumerate}
\item C will be selected?
\item A will not be selected?
\end{enumerate}
	%\input{exemplar/11/16/3/9/main.tex}
 \item A bag contain 24 balls of which $x$ balls are red, $2x$ are white and $3x$ are blue. A ball is selected at random, What is the probability that it is
\begin{enumerate}[label=\alph*)]
\item not red ?
\item white ?
\end{enumerate}
%\input{exemplar/10/13/3/41/main.tex}
If the letters of the word ASSASSINATION are arranged at random. Find the Probability that
\begin{enumerate}[label=(\alph*)]
\item Four $S's$ come consecutively in the word
\item Two  $I's$ and two $N's$ come together
\item All $A's$ are not coming together
\item No two $A's$ are coming together
\end{enumerate}
%\input{exemplar/11/16/3/14/main.tex}
	\item One urn contains two black balls (labelled B1 and B2) and one white ball. A
	second urn contains one black ball and two white balls (labelled W1 and W2).
	Suppose the following experiment is performed. One of the two urns is chosen
	at random. Next a ball is randomly chosen from the urn. Then a second ball is
	chosen at random from the same urn without replacing the first ball.
	
	\begin{enumerate}
	\item What is the probability that two black balls are chosen?
	
	\item What is the probability that two balls of opposite colour are chosen?
	\end{enumerate}
	\solution
	%\input{exemplar/11/16/3/12/main1.tex}
\end{enumerate}

		%
\item 
Out of 100 students, two sections of 40 and 60 are formed. If you and your friend are among the 100 students, what is the probability that
\begin{enumerate}
\item you both enter the same section?
\item you both enter the different sections?
\end{enumerate}
\solution
		%\begin{enumerate}[label=\thesection.\arabic*,ref=\thesection.\theenumi]
	\item One card is drawn from a well-shuffled deck of 52 cards. Find the probability of getting
\begin{enumerate}
\item A king of red colour 
\item A face card 
\item A red face card
\item The jack of hearts
\item A spade
\item The queen of diamonds

\end{enumerate}
\solution
		%\input{ncert/10/15/1/14/main.tex}
	\item Five cards—the ten, jack, queen, king and ace of diamonds, are well-shuffled with their face downwards. One card is then picked up at random.
\begin{enumerate}
\item
What is the probability that the card is the queen? 
\item
If the queen is drawn and put aside, what is the probability that the second card picked up is (a) an ace? (b) a queen?\\
\end{enumerate}
\solution
		%\input{ncert/10/15/1/15/defs.tex}
	\item A bag contains $5$ red balls and some blue balls. If the probability of drawing a blue ball is double that if a red ball, determine the number of blue balls in the bag. 
		\\
\solution
		%\input{ncert/10/15/2/3/defs.tex}
	\item A card is selected from a pack of 52 cards.
 \begin{enumerate}[label=(\alph*)] 
                 \item How many points are there in the sample space?
                 \item Calculate the probability that the card is an ace of spades.
                 \item Calculate the probability that the card is (i) an ace and (ii) black card.
 \end{enumerate}
\solution
		%\input{ncert/11/16/3/4/main.tex}
\item Four cards are drawn from a well-shuffled deck of 52 cards. What is the probability of obtaining 3 diamonds and one spade.
\\
\solution
		%\input{ncert/11/16/4/2/defs.tex}
\item In a certain lottery 10,000 tickets are sold and ten equal prizes are awarded. What is the probability of not getting a prize if you buy (a) one ticket (b) two tickets (c) 10 tickets ?	
\\
\solution
		%\input{ncert/11/16/4/4/defs.tex}
		%
\item 
Out of 100 students, two sections of 40 and 60 are formed. If you and your friend are among the 100 students, what is the probability that
\begin{enumerate}
\item you both enter the same section?
\item you both enter the different sections?
\end{enumerate}
\solution
		%\input{ncert/11/16/4/5/defs.tex}
	\item 
The number lock of a suitcase has 4 wheels each labelled with ten digits i.e. from 0 to 9.The lock opens with a sequence of four digits with no repeats.What is the probability of a person getting the right sequence to open the suitcase.
\\
\solution
		%\input{ncert/11/16/4/10/defs.tex}
		%
\item 
Two cards are drawn at random and without replacement from a pack of 52 playing cards. Find the probability that both the cards are black.
\\
\solution
		%\input{ncert/12/13/2/2/defs.tex}
		\item A box of oranges is inspected by examining three randomly selected oranges drawn without replacement. If all the three oranges are good, the box is approved for sale, otherwise, it is rejected. Find the probability that a box containing 15 oranges out of which 12 are good and 3 are bad ones will be approved for sale.
		\label{ncert/12/13/2/3/defs.tex}
		\item Two balls are drawn at random with replacement from a box containing 10 black and 8 red balls. Find the probability that
		\label{ncert/12/13/2/12}
\begin{enumerate}
\item both balls are red.
\item first ball is black and second is red.
\item one of them is black and other is red.
\end{enumerate}

\item In a hostel, 60\% of the students read Hindi newspaper, 40\% read English newspaper and 20\% read both Hindi and English newspapers. A student is selected at random.
		\label{ncert/12/13/2/15}
\begin{enumerate}
\item Find the probability that she reads neither Hindi nor English newspapers.
\item If she reads Hindi newspaper, find the probability that she reads English newspaper.
\item If she reads English newspaper, find the probability that she reads Hindi newspaper.\\
\end{enumerate}
\item The probability of obtaining an even prime number on each die, when a pair of dice is rolled is 
\begin{enumerate}
    \item $0$ 
    
    \item $\frac{1}{3}$ 
    
    \item $\frac{1}{12}$ 
    
    \item $\frac{1}{36}$ 
\end{enumerate}
\solution
		%\input{ncert/12/13/2/17/defs.tex}
	\item A bag contains 4 red and 4 black balls, another bag contains 2 red and 6 black balls. One of the two bags is selected at random and a ball is drawn from the bag which is found to be red. Find the probability that the ball is drawn from the first bag.
\\
\solution
		%\input{ncert/12/13/3/2/main.tex}
  \item
  Cards with numbers 2 to 101 are placed in a box. A card is selected at random.Find the probability that the card has
\begin{enumerate}[label=(\roman*)]
	\item an even number 
	\item a square number
\end{enumerate}
\solution
%\input{exemplar/10/13/3/32/main.tex}
\item
The king, queen and jack of clubs are removed from a deck of 52 playing cards and then well shuffled. Now one card is drawn at random from the remaining cards.  Determine the probability that the card is
\begin{enumerate}[label=(\roman*)]
\item a club
\item 10 of hearts
\end{enumerate}
\solution
%\input{exemplar/10/13/3/29/main.tex}
\item A team of medical students doing their internship have to assist during surgeries
at a city hospital. The probabilities of surgeries rated as very complex, complex,
routine, simple or very simple are respectively, 0.15, 0.20, 0.31, 0.26, .08. Find
the probabilities that a particular surgery will be rated
\begin{enumerate}
	\item complex or very complex;
	\item neither very complex nor very simple;
	\item routine or complex
	\item routine or simple
\end{enumerate}
\solution
%\input{exemplar/11/16/3/8(1)/main.tex}
\item A card is selected from a pack of 52 cards.
\begin{enumerate}[label=(\alph*)]
    \item How many points are there in the sample space?
    \item Calculate the probability that the card is an ace of spades.
    \item Calculate the probability that the card is (i) an ace and (ii) black card.
\end{enumerate}
\solution
%\input{exemplar/11/16/3/4/main2.tex}
\item The probability that a non leap year selected at random will contain 53 sundays.
\\
\solution
%\input{exemplar/10/13/1/19/main.tex}
\item One of the four persons John, Rita, Aslam or Gurpreet will be promoted next
month. Consequently the sample space consists of four elementary outcomes
S = {John promoted, Rita promoted, Aslam promoted, Gurpreet promoted}
You are told that the chances of John’s promotion is same as that of Gurpreet,
Rita’s chances of promotion are twice as likely as Johns. Aslam’s chances are
four times that of John.
\begin{enumerate}
	\item Determine
	\begin{enumerate}
		\item P (John promoted)
		\item P (Rita promoted)
		\item P (Aslam promoted)
		\item P (Gurpreet promoted)
	\end{enumerate}
	\item If A = {John promoted or Gurpreet promoted}, find P (A).
\end{enumerate}
\solution
%\input{exemplar/11/16/3/10/main.tex}
\item A card is drawn from a deck of 52 cards. Find the probability of getting a king or a heart or a red card.\\
\solution
%\input{exemplar/11/16/3/15/main.tex}
\item The probability that a student will pass his examination is 0.73, the probability of
the student getting a compartment is 0.13, and the probability that the student will
either pass or get compartment is 0.96. State True or False.\\
\solution
%\input{exemplar/11/16/3/31/main.tex}
\item A card is selected from a pack of 52 cards\\
\begin{enumerate}[label=(\alph*)]
\item How many points are there in the sample space?
\item Calculate the probability that the cards is an ace of spades.
\item Calculate the probability that the card is (i) an ace (ii)black card.\\
\end{enumerate}
%\input{ncert/11/16/3/4_1/Prob_4.tex}
\item In a non-leap year, the probability of having 53 tuesdays or 53 wednesdays is\\
\solution
%\input{exemplar/11/16/3/18/main.tex}
\item There are 1000 sealed envelopes in a box, 10 of them contain a cash prize of
Rs 100 each, 100 of them contain a cash prize of Rs 50 each and 200 of them
contain a cash prize of Rs 10 each and rest do not contain any cash prize. If they
are well shuffled and an envelope is picked up out, what is the probability that it
contains no cash prize?\\
\solution
%\input{exemplar/10/13/3/34/main.tex}
\item 
A die is thrown and a card is selected at random from a deck of 52 playing cards. The probability of getting an even number on the die and a spade card.\\
\solution
%\input{exemplar/12/13/3/78/main.tex}
\item
If 4-digit numbers greater than 5,000 are randomly formed from the digits 0, 1, 3, 5, and 7, what is the probability of forming a number divisible by 5 when:
\begin{enumerate}
    \item The digits are repeated?
    \item The repetition of digits is not allowed?
\end{enumerate}
\solution
%\input{ncert/11/16/4/9/main.tex}
\item Consider the probability space $\brak{\Omega, \mathcal{G}, P}$ where $\Omega = [0,2]$ and $\mathcal{G} = \cbrak{\phi, \Omega, [0,1], (1,2]}$. Let $X$ and $Y$ be two functions on $\Omega$ defined as
\begin{align*}
    X(\omega) = 
    \begin{cases}
        1 & \text{if }\omega \in [0, 1]\\
        2 & \text{if }\omega \in (1, 2]
    \end{cases}
\end{align*}
and
\begin{align*}
    Y(\omega) = 
    \begin{cases}
        2 & \text{if }\omega \in [0, 1.5]\\
        3 & \text{if }\omega \in (1.5, 2].
    \end{cases}
\end{align*}
Then which one of the following statements is true?
\begin{enumerate}
    \item [(A)] $X$ is a random variable with respect to $\mathcal{G}$, but $Y$ is not a random variable with respect to $\mathcal{G}$.
    \item [(B)] $Y$ is a random variable with respect to $\mathcal{G}$, but $X$ is not a random variable with respect to $\mathcal{G}$.
    \item [(C)] Neither $X$ nor $Y$ is a random variable with respect to $\mathcal{G}$.
    \item [(D)] Both $X$ and $Y$ are random variables with respect to $\mathcal{G}$.
\end{enumerate} \hfill (GATE ST 2023)\\
\solution
%\input{gate/ST/2023/14/main.tex}
	\item  A die is loaded in such a way that each odd number is twice as likely to occur as
each even number. Find $P(G)$, where $G$ is the event that a number greater than
3 occurs on a single roll of the die.
\\
\solution
		%\input{exemplar/11/16/3/5/main.tex}
	\item All the jacks, queens and kings are removed from a deck of 52 playing cards. The remaining cards are well shuffled and then one card is drawn at random. Giving ace a value 1 similar value for other cards, find the probability that the card has a value 
		\begin{enumerate}
			\item 7
			\item greater than 7
			\item less than 7
		\end{enumerate}
		%\input{exemplar/10/13/3/30/main.tex}
  \item A Lot consists of 48 mobile phones of which 42 are good, 3 have only minor defects and 3 have major defects.Varnika will buy a phone if it is good but the trader will only buy a mobile if it has no major defects. One phone is selected at random from the lot. What is the probability that it is
\begin{enumerate}
	\item acceptable to Varnika?
            \item acceptable to the trader?
\end{enumerate}
\solution
	%\input{exemplar/10/13/3/40/main.tex}
 \item A student says that if you throw a die, it will show up 1 or not 1. Therefore, the probability of getting 1 and the probability of getting 'not 1' each is equal to $\frac{1}{2}$. Is this correct? Give reasons.\\
 \solution
        %\input{exemplar/10/13/2/9/main.tex}
   \item Four candidates A, B, C, D have ap-
plied for the assignment to coach a school cricket
team. If A is twice as likely to be selected as B, and
B and C are given about the same chance of being
selected, while C is twice as likely to be selected
as D, what are the probabilities that
\begin{enumerate}
\item C will be selected?
\item A will not be selected?
\end{enumerate}
	%\input{exemplar/11/16/3/9/main.tex}
 \item A bag contain 24 balls of which $x$ balls are red, $2x$ are white and $3x$ are blue. A ball is selected at random, What is the probability that it is
\begin{enumerate}[label=\alph*)]
\item not red ?
\item white ?
\end{enumerate}
%\input{exemplar/10/13/3/41/main.tex}
If the letters of the word ASSASSINATION are arranged at random. Find the Probability that
\begin{enumerate}[label=(\alph*)]
\item Four $S's$ come consecutively in the word
\item Two  $I's$ and two $N's$ come together
\item All $A's$ are not coming together
\item No two $A's$ are coming together
\end{enumerate}
%\input{exemplar/11/16/3/14/main.tex}
	\item One urn contains two black balls (labelled B1 and B2) and one white ball. A
	second urn contains one black ball and two white balls (labelled W1 and W2).
	Suppose the following experiment is performed. One of the two urns is chosen
	at random. Next a ball is randomly chosen from the urn. Then a second ball is
	chosen at random from the same urn without replacing the first ball.
	
	\begin{enumerate}
	\item What is the probability that two black balls are chosen?
	
	\item What is the probability that two balls of opposite colour are chosen?
	\end{enumerate}
	\solution
	%\input{exemplar/11/16/3/12/main1.tex}
\end{enumerate}

	\item 
The number lock of a suitcase has 4 wheels each labelled with ten digits i.e. from 0 to 9.The lock opens with a sequence of four digits with no repeats.What is the probability of a person getting the right sequence to open the suitcase.
\\
\solution
		%\begin{enumerate}[label=\thesection.\arabic*,ref=\thesection.\theenumi]
	\item One card is drawn from a well-shuffled deck of 52 cards. Find the probability of getting
\begin{enumerate}
\item A king of red colour 
\item A face card 
\item A red face card
\item The jack of hearts
\item A spade
\item The queen of diamonds

\end{enumerate}
\solution
		%\input{ncert/10/15/1/14/main.tex}
	\item Five cards—the ten, jack, queen, king and ace of diamonds, are well-shuffled with their face downwards. One card is then picked up at random.
\begin{enumerate}
\item
What is the probability that the card is the queen? 
\item
If the queen is drawn and put aside, what is the probability that the second card picked up is (a) an ace? (b) a queen?\\
\end{enumerate}
\solution
		%\input{ncert/10/15/1/15/defs.tex}
	\item A bag contains $5$ red balls and some blue balls. If the probability of drawing a blue ball is double that if a red ball, determine the number of blue balls in the bag. 
		\\
\solution
		%\input{ncert/10/15/2/3/defs.tex}
	\item A card is selected from a pack of 52 cards.
 \begin{enumerate}[label=(\alph*)] 
                 \item How many points are there in the sample space?
                 \item Calculate the probability that the card is an ace of spades.
                 \item Calculate the probability that the card is (i) an ace and (ii) black card.
 \end{enumerate}
\solution
		%\input{ncert/11/16/3/4/main.tex}
\item Four cards are drawn from a well-shuffled deck of 52 cards. What is the probability of obtaining 3 diamonds and one spade.
\\
\solution
		%\input{ncert/11/16/4/2/defs.tex}
\item In a certain lottery 10,000 tickets are sold and ten equal prizes are awarded. What is the probability of not getting a prize if you buy (a) one ticket (b) two tickets (c) 10 tickets ?	
\\
\solution
		%\input{ncert/11/16/4/4/defs.tex}
		%
\item 
Out of 100 students, two sections of 40 and 60 are formed. If you and your friend are among the 100 students, what is the probability that
\begin{enumerate}
\item you both enter the same section?
\item you both enter the different sections?
\end{enumerate}
\solution
		%\input{ncert/11/16/4/5/defs.tex}
	\item 
The number lock of a suitcase has 4 wheels each labelled with ten digits i.e. from 0 to 9.The lock opens with a sequence of four digits with no repeats.What is the probability of a person getting the right sequence to open the suitcase.
\\
\solution
		%\input{ncert/11/16/4/10/defs.tex}
		%
\item 
Two cards are drawn at random and without replacement from a pack of 52 playing cards. Find the probability that both the cards are black.
\\
\solution
		%\input{ncert/12/13/2/2/defs.tex}
		\item A box of oranges is inspected by examining three randomly selected oranges drawn without replacement. If all the three oranges are good, the box is approved for sale, otherwise, it is rejected. Find the probability that a box containing 15 oranges out of which 12 are good and 3 are bad ones will be approved for sale.
		\label{ncert/12/13/2/3/defs.tex}
		\item Two balls are drawn at random with replacement from a box containing 10 black and 8 red balls. Find the probability that
		\label{ncert/12/13/2/12}
\begin{enumerate}
\item both balls are red.
\item first ball is black and second is red.
\item one of them is black and other is red.
\end{enumerate}

\item In a hostel, 60\% of the students read Hindi newspaper, 40\% read English newspaper and 20\% read both Hindi and English newspapers. A student is selected at random.
		\label{ncert/12/13/2/15}
\begin{enumerate}
\item Find the probability that she reads neither Hindi nor English newspapers.
\item If she reads Hindi newspaper, find the probability that she reads English newspaper.
\item If she reads English newspaper, find the probability that she reads Hindi newspaper.\\
\end{enumerate}
\item The probability of obtaining an even prime number on each die, when a pair of dice is rolled is 
\begin{enumerate}
    \item $0$ 
    
    \item $\frac{1}{3}$ 
    
    \item $\frac{1}{12}$ 
    
    \item $\frac{1}{36}$ 
\end{enumerate}
\solution
		%\input{ncert/12/13/2/17/defs.tex}
	\item A bag contains 4 red and 4 black balls, another bag contains 2 red and 6 black balls. One of the two bags is selected at random and a ball is drawn from the bag which is found to be red. Find the probability that the ball is drawn from the first bag.
\\
\solution
		%\input{ncert/12/13/3/2/main.tex}
  \item
  Cards with numbers 2 to 101 are placed in a box. A card is selected at random.Find the probability that the card has
\begin{enumerate}[label=(\roman*)]
	\item an even number 
	\item a square number
\end{enumerate}
\solution
%\input{exemplar/10/13/3/32/main.tex}
\item
The king, queen and jack of clubs are removed from a deck of 52 playing cards and then well shuffled. Now one card is drawn at random from the remaining cards.  Determine the probability that the card is
\begin{enumerate}[label=(\roman*)]
\item a club
\item 10 of hearts
\end{enumerate}
\solution
%\input{exemplar/10/13/3/29/main.tex}
\item A team of medical students doing their internship have to assist during surgeries
at a city hospital. The probabilities of surgeries rated as very complex, complex,
routine, simple or very simple are respectively, 0.15, 0.20, 0.31, 0.26, .08. Find
the probabilities that a particular surgery will be rated
\begin{enumerate}
	\item complex or very complex;
	\item neither very complex nor very simple;
	\item routine or complex
	\item routine or simple
\end{enumerate}
\solution
%\input{exemplar/11/16/3/8(1)/main.tex}
\item A card is selected from a pack of 52 cards.
\begin{enumerate}[label=(\alph*)]
    \item How many points are there in the sample space?
    \item Calculate the probability that the card is an ace of spades.
    \item Calculate the probability that the card is (i) an ace and (ii) black card.
\end{enumerate}
\solution
%\input{exemplar/11/16/3/4/main2.tex}
\item The probability that a non leap year selected at random will contain 53 sundays.
\\
\solution
%\input{exemplar/10/13/1/19/main.tex}
\item One of the four persons John, Rita, Aslam or Gurpreet will be promoted next
month. Consequently the sample space consists of four elementary outcomes
S = {John promoted, Rita promoted, Aslam promoted, Gurpreet promoted}
You are told that the chances of John’s promotion is same as that of Gurpreet,
Rita’s chances of promotion are twice as likely as Johns. Aslam’s chances are
four times that of John.
\begin{enumerate}
	\item Determine
	\begin{enumerate}
		\item P (John promoted)
		\item P (Rita promoted)
		\item P (Aslam promoted)
		\item P (Gurpreet promoted)
	\end{enumerate}
	\item If A = {John promoted or Gurpreet promoted}, find P (A).
\end{enumerate}
\solution
%\input{exemplar/11/16/3/10/main.tex}
\item A card is drawn from a deck of 52 cards. Find the probability of getting a king or a heart or a red card.\\
\solution
%\input{exemplar/11/16/3/15/main.tex}
\item The probability that a student will pass his examination is 0.73, the probability of
the student getting a compartment is 0.13, and the probability that the student will
either pass or get compartment is 0.96. State True or False.\\
\solution
%\input{exemplar/11/16/3/31/main.tex}
\item A card is selected from a pack of 52 cards\\
\begin{enumerate}[label=(\alph*)]
\item How many points are there in the sample space?
\item Calculate the probability that the cards is an ace of spades.
\item Calculate the probability that the card is (i) an ace (ii)black card.\\
\end{enumerate}
%\input{ncert/11/16/3/4_1/Prob_4.tex}
\item In a non-leap year, the probability of having 53 tuesdays or 53 wednesdays is\\
\solution
%\input{exemplar/11/16/3/18/main.tex}
\item There are 1000 sealed envelopes in a box, 10 of them contain a cash prize of
Rs 100 each, 100 of them contain a cash prize of Rs 50 each and 200 of them
contain a cash prize of Rs 10 each and rest do not contain any cash prize. If they
are well shuffled and an envelope is picked up out, what is the probability that it
contains no cash prize?\\
\solution
%\input{exemplar/10/13/3/34/main.tex}
\item 
A die is thrown and a card is selected at random from a deck of 52 playing cards. The probability of getting an even number on the die and a spade card.\\
\solution
%\input{exemplar/12/13/3/78/main.tex}
\item
If 4-digit numbers greater than 5,000 are randomly formed from the digits 0, 1, 3, 5, and 7, what is the probability of forming a number divisible by 5 when:
\begin{enumerate}
    \item The digits are repeated?
    \item The repetition of digits is not allowed?
\end{enumerate}
\solution
%\input{ncert/11/16/4/9/main.tex}
\item Consider the probability space $\brak{\Omega, \mathcal{G}, P}$ where $\Omega = [0,2]$ and $\mathcal{G} = \cbrak{\phi, \Omega, [0,1], (1,2]}$. Let $X$ and $Y$ be two functions on $\Omega$ defined as
\begin{align*}
    X(\omega) = 
    \begin{cases}
        1 & \text{if }\omega \in [0, 1]\\
        2 & \text{if }\omega \in (1, 2]
    \end{cases}
\end{align*}
and
\begin{align*}
    Y(\omega) = 
    \begin{cases}
        2 & \text{if }\omega \in [0, 1.5]\\
        3 & \text{if }\omega \in (1.5, 2].
    \end{cases}
\end{align*}
Then which one of the following statements is true?
\begin{enumerate}
    \item [(A)] $X$ is a random variable with respect to $\mathcal{G}$, but $Y$ is not a random variable with respect to $\mathcal{G}$.
    \item [(B)] $Y$ is a random variable with respect to $\mathcal{G}$, but $X$ is not a random variable with respect to $\mathcal{G}$.
    \item [(C)] Neither $X$ nor $Y$ is a random variable with respect to $\mathcal{G}$.
    \item [(D)] Both $X$ and $Y$ are random variables with respect to $\mathcal{G}$.
\end{enumerate} \hfill (GATE ST 2023)\\
\solution
%\input{gate/ST/2023/14/main.tex}
	\item  A die is loaded in such a way that each odd number is twice as likely to occur as
each even number. Find $P(G)$, where $G$ is the event that a number greater than
3 occurs on a single roll of the die.
\\
\solution
		%\input{exemplar/11/16/3/5/main.tex}
	\item All the jacks, queens and kings are removed from a deck of 52 playing cards. The remaining cards are well shuffled and then one card is drawn at random. Giving ace a value 1 similar value for other cards, find the probability that the card has a value 
		\begin{enumerate}
			\item 7
			\item greater than 7
			\item less than 7
		\end{enumerate}
		%\input{exemplar/10/13/3/30/main.tex}
  \item A Lot consists of 48 mobile phones of which 42 are good, 3 have only minor defects and 3 have major defects.Varnika will buy a phone if it is good but the trader will only buy a mobile if it has no major defects. One phone is selected at random from the lot. What is the probability that it is
\begin{enumerate}
	\item acceptable to Varnika?
            \item acceptable to the trader?
\end{enumerate}
\solution
	%\input{exemplar/10/13/3/40/main.tex}
 \item A student says that if you throw a die, it will show up 1 or not 1. Therefore, the probability of getting 1 and the probability of getting 'not 1' each is equal to $\frac{1}{2}$. Is this correct? Give reasons.\\
 \solution
        %\input{exemplar/10/13/2/9/main.tex}
   \item Four candidates A, B, C, D have ap-
plied for the assignment to coach a school cricket
team. If A is twice as likely to be selected as B, and
B and C are given about the same chance of being
selected, while C is twice as likely to be selected
as D, what are the probabilities that
\begin{enumerate}
\item C will be selected?
\item A will not be selected?
\end{enumerate}
	%\input{exemplar/11/16/3/9/main.tex}
 \item A bag contain 24 balls of which $x$ balls are red, $2x$ are white and $3x$ are blue. A ball is selected at random, What is the probability that it is
\begin{enumerate}[label=\alph*)]
\item not red ?
\item white ?
\end{enumerate}
%\input{exemplar/10/13/3/41/main.tex}
If the letters of the word ASSASSINATION are arranged at random. Find the Probability that
\begin{enumerate}[label=(\alph*)]
\item Four $S's$ come consecutively in the word
\item Two  $I's$ and two $N's$ come together
\item All $A's$ are not coming together
\item No two $A's$ are coming together
\end{enumerate}
%\input{exemplar/11/16/3/14/main.tex}
	\item One urn contains two black balls (labelled B1 and B2) and one white ball. A
	second urn contains one black ball and two white balls (labelled W1 and W2).
	Suppose the following experiment is performed. One of the two urns is chosen
	at random. Next a ball is randomly chosen from the urn. Then a second ball is
	chosen at random from the same urn without replacing the first ball.
	
	\begin{enumerate}
	\item What is the probability that two black balls are chosen?
	
	\item What is the probability that two balls of opposite colour are chosen?
	\end{enumerate}
	\solution
	%\input{exemplar/11/16/3/12/main1.tex}
\end{enumerate}

		%
\item 
Two cards are drawn at random and without replacement from a pack of 52 playing cards. Find the probability that both the cards are black.
\\
\solution
		%\begin{enumerate}[label=\thesection.\arabic*,ref=\thesection.\theenumi]
	\item One card is drawn from a well-shuffled deck of 52 cards. Find the probability of getting
\begin{enumerate}
\item A king of red colour 
\item A face card 
\item A red face card
\item The jack of hearts
\item A spade
\item The queen of diamonds

\end{enumerate}
\solution
		%\input{ncert/10/15/1/14/main.tex}
	\item Five cards—the ten, jack, queen, king and ace of diamonds, are well-shuffled with their face downwards. One card is then picked up at random.
\begin{enumerate}
\item
What is the probability that the card is the queen? 
\item
If the queen is drawn and put aside, what is the probability that the second card picked up is (a) an ace? (b) a queen?\\
\end{enumerate}
\solution
		%\input{ncert/10/15/1/15/defs.tex}
	\item A bag contains $5$ red balls and some blue balls. If the probability of drawing a blue ball is double that if a red ball, determine the number of blue balls in the bag. 
		\\
\solution
		%\input{ncert/10/15/2/3/defs.tex}
	\item A card is selected from a pack of 52 cards.
 \begin{enumerate}[label=(\alph*)] 
                 \item How many points are there in the sample space?
                 \item Calculate the probability that the card is an ace of spades.
                 \item Calculate the probability that the card is (i) an ace and (ii) black card.
 \end{enumerate}
\solution
		%\input{ncert/11/16/3/4/main.tex}
\item Four cards are drawn from a well-shuffled deck of 52 cards. What is the probability of obtaining 3 diamonds and one spade.
\\
\solution
		%\input{ncert/11/16/4/2/defs.tex}
\item In a certain lottery 10,000 tickets are sold and ten equal prizes are awarded. What is the probability of not getting a prize if you buy (a) one ticket (b) two tickets (c) 10 tickets ?	
\\
\solution
		%\input{ncert/11/16/4/4/defs.tex}
		%
\item 
Out of 100 students, two sections of 40 and 60 are formed. If you and your friend are among the 100 students, what is the probability that
\begin{enumerate}
\item you both enter the same section?
\item you both enter the different sections?
\end{enumerate}
\solution
		%\input{ncert/11/16/4/5/defs.tex}
	\item 
The number lock of a suitcase has 4 wheels each labelled with ten digits i.e. from 0 to 9.The lock opens with a sequence of four digits with no repeats.What is the probability of a person getting the right sequence to open the suitcase.
\\
\solution
		%\input{ncert/11/16/4/10/defs.tex}
		%
\item 
Two cards are drawn at random and without replacement from a pack of 52 playing cards. Find the probability that both the cards are black.
\\
\solution
		%\input{ncert/12/13/2/2/defs.tex}
		\item A box of oranges is inspected by examining three randomly selected oranges drawn without replacement. If all the three oranges are good, the box is approved for sale, otherwise, it is rejected. Find the probability that a box containing 15 oranges out of which 12 are good and 3 are bad ones will be approved for sale.
		\label{ncert/12/13/2/3/defs.tex}
		\item Two balls are drawn at random with replacement from a box containing 10 black and 8 red balls. Find the probability that
		\label{ncert/12/13/2/12}
\begin{enumerate}
\item both balls are red.
\item first ball is black and second is red.
\item one of them is black and other is red.
\end{enumerate}

\item In a hostel, 60\% of the students read Hindi newspaper, 40\% read English newspaper and 20\% read both Hindi and English newspapers. A student is selected at random.
		\label{ncert/12/13/2/15}
\begin{enumerate}
\item Find the probability that she reads neither Hindi nor English newspapers.
\item If she reads Hindi newspaper, find the probability that she reads English newspaper.
\item If she reads English newspaper, find the probability that she reads Hindi newspaper.\\
\end{enumerate}
\item The probability of obtaining an even prime number on each die, when a pair of dice is rolled is 
\begin{enumerate}
    \item $0$ 
    
    \item $\frac{1}{3}$ 
    
    \item $\frac{1}{12}$ 
    
    \item $\frac{1}{36}$ 
\end{enumerate}
\solution
		%\input{ncert/12/13/2/17/defs.tex}
	\item A bag contains 4 red and 4 black balls, another bag contains 2 red and 6 black balls. One of the two bags is selected at random and a ball is drawn from the bag which is found to be red. Find the probability that the ball is drawn from the first bag.
\\
\solution
		%\input{ncert/12/13/3/2/main.tex}
  \item
  Cards with numbers 2 to 101 are placed in a box. A card is selected at random.Find the probability that the card has
\begin{enumerate}[label=(\roman*)]
	\item an even number 
	\item a square number
\end{enumerate}
\solution
%\input{exemplar/10/13/3/32/main.tex}
\item
The king, queen and jack of clubs are removed from a deck of 52 playing cards and then well shuffled. Now one card is drawn at random from the remaining cards.  Determine the probability that the card is
\begin{enumerate}[label=(\roman*)]
\item a club
\item 10 of hearts
\end{enumerate}
\solution
%\input{exemplar/10/13/3/29/main.tex}
\item A team of medical students doing their internship have to assist during surgeries
at a city hospital. The probabilities of surgeries rated as very complex, complex,
routine, simple or very simple are respectively, 0.15, 0.20, 0.31, 0.26, .08. Find
the probabilities that a particular surgery will be rated
\begin{enumerate}
	\item complex or very complex;
	\item neither very complex nor very simple;
	\item routine or complex
	\item routine or simple
\end{enumerate}
\solution
%\input{exemplar/11/16/3/8(1)/main.tex}
\item A card is selected from a pack of 52 cards.
\begin{enumerate}[label=(\alph*)]
    \item How many points are there in the sample space?
    \item Calculate the probability that the card is an ace of spades.
    \item Calculate the probability that the card is (i) an ace and (ii) black card.
\end{enumerate}
\solution
%\input{exemplar/11/16/3/4/main2.tex}
\item The probability that a non leap year selected at random will contain 53 sundays.
\\
\solution
%\input{exemplar/10/13/1/19/main.tex}
\item One of the four persons John, Rita, Aslam or Gurpreet will be promoted next
month. Consequently the sample space consists of four elementary outcomes
S = {John promoted, Rita promoted, Aslam promoted, Gurpreet promoted}
You are told that the chances of John’s promotion is same as that of Gurpreet,
Rita’s chances of promotion are twice as likely as Johns. Aslam’s chances are
four times that of John.
\begin{enumerate}
	\item Determine
	\begin{enumerate}
		\item P (John promoted)
		\item P (Rita promoted)
		\item P (Aslam promoted)
		\item P (Gurpreet promoted)
	\end{enumerate}
	\item If A = {John promoted or Gurpreet promoted}, find P (A).
\end{enumerate}
\solution
%\input{exemplar/11/16/3/10/main.tex}
\item A card is drawn from a deck of 52 cards. Find the probability of getting a king or a heart or a red card.\\
\solution
%\input{exemplar/11/16/3/15/main.tex}
\item The probability that a student will pass his examination is 0.73, the probability of
the student getting a compartment is 0.13, and the probability that the student will
either pass or get compartment is 0.96. State True or False.\\
\solution
%\input{exemplar/11/16/3/31/main.tex}
\item A card is selected from a pack of 52 cards\\
\begin{enumerate}[label=(\alph*)]
\item How many points are there in the sample space?
\item Calculate the probability that the cards is an ace of spades.
\item Calculate the probability that the card is (i) an ace (ii)black card.\\
\end{enumerate}
%\input{ncert/11/16/3/4_1/Prob_4.tex}
\item In a non-leap year, the probability of having 53 tuesdays or 53 wednesdays is\\
\solution
%\input{exemplar/11/16/3/18/main.tex}
\item There are 1000 sealed envelopes in a box, 10 of them contain a cash prize of
Rs 100 each, 100 of them contain a cash prize of Rs 50 each and 200 of them
contain a cash prize of Rs 10 each and rest do not contain any cash prize. If they
are well shuffled and an envelope is picked up out, what is the probability that it
contains no cash prize?\\
\solution
%\input{exemplar/10/13/3/34/main.tex}
\item 
A die is thrown and a card is selected at random from a deck of 52 playing cards. The probability of getting an even number on the die and a spade card.\\
\solution
%\input{exemplar/12/13/3/78/main.tex}
\item
If 4-digit numbers greater than 5,000 are randomly formed from the digits 0, 1, 3, 5, and 7, what is the probability of forming a number divisible by 5 when:
\begin{enumerate}
    \item The digits are repeated?
    \item The repetition of digits is not allowed?
\end{enumerate}
\solution
%\input{ncert/11/16/4/9/main.tex}
\item Consider the probability space $\brak{\Omega, \mathcal{G}, P}$ where $\Omega = [0,2]$ and $\mathcal{G} = \cbrak{\phi, \Omega, [0,1], (1,2]}$. Let $X$ and $Y$ be two functions on $\Omega$ defined as
\begin{align*}
    X(\omega) = 
    \begin{cases}
        1 & \text{if }\omega \in [0, 1]\\
        2 & \text{if }\omega \in (1, 2]
    \end{cases}
\end{align*}
and
\begin{align*}
    Y(\omega) = 
    \begin{cases}
        2 & \text{if }\omega \in [0, 1.5]\\
        3 & \text{if }\omega \in (1.5, 2].
    \end{cases}
\end{align*}
Then which one of the following statements is true?
\begin{enumerate}
    \item [(A)] $X$ is a random variable with respect to $\mathcal{G}$, but $Y$ is not a random variable with respect to $\mathcal{G}$.
    \item [(B)] $Y$ is a random variable with respect to $\mathcal{G}$, but $X$ is not a random variable with respect to $\mathcal{G}$.
    \item [(C)] Neither $X$ nor $Y$ is a random variable with respect to $\mathcal{G}$.
    \item [(D)] Both $X$ and $Y$ are random variables with respect to $\mathcal{G}$.
\end{enumerate} \hfill (GATE ST 2023)\\
\solution
%\input{gate/ST/2023/14/main.tex}
	\item  A die is loaded in such a way that each odd number is twice as likely to occur as
each even number. Find $P(G)$, where $G$ is the event that a number greater than
3 occurs on a single roll of the die.
\\
\solution
		%\input{exemplar/11/16/3/5/main.tex}
	\item All the jacks, queens and kings are removed from a deck of 52 playing cards. The remaining cards are well shuffled and then one card is drawn at random. Giving ace a value 1 similar value for other cards, find the probability that the card has a value 
		\begin{enumerate}
			\item 7
			\item greater than 7
			\item less than 7
		\end{enumerate}
		%\input{exemplar/10/13/3/30/main.tex}
  \item A Lot consists of 48 mobile phones of which 42 are good, 3 have only minor defects and 3 have major defects.Varnika will buy a phone if it is good but the trader will only buy a mobile if it has no major defects. One phone is selected at random from the lot. What is the probability that it is
\begin{enumerate}
	\item acceptable to Varnika?
            \item acceptable to the trader?
\end{enumerate}
\solution
	%\input{exemplar/10/13/3/40/main.tex}
 \item A student says that if you throw a die, it will show up 1 or not 1. Therefore, the probability of getting 1 and the probability of getting 'not 1' each is equal to $\frac{1}{2}$. Is this correct? Give reasons.\\
 \solution
        %\input{exemplar/10/13/2/9/main.tex}
   \item Four candidates A, B, C, D have ap-
plied for the assignment to coach a school cricket
team. If A is twice as likely to be selected as B, and
B and C are given about the same chance of being
selected, while C is twice as likely to be selected
as D, what are the probabilities that
\begin{enumerate}
\item C will be selected?
\item A will not be selected?
\end{enumerate}
	%\input{exemplar/11/16/3/9/main.tex}
 \item A bag contain 24 balls of which $x$ balls are red, $2x$ are white and $3x$ are blue. A ball is selected at random, What is the probability that it is
\begin{enumerate}[label=\alph*)]
\item not red ?
\item white ?
\end{enumerate}
%\input{exemplar/10/13/3/41/main.tex}
If the letters of the word ASSASSINATION are arranged at random. Find the Probability that
\begin{enumerate}[label=(\alph*)]
\item Four $S's$ come consecutively in the word
\item Two  $I's$ and two $N's$ come together
\item All $A's$ are not coming together
\item No two $A's$ are coming together
\end{enumerate}
%\input{exemplar/11/16/3/14/main.tex}
	\item One urn contains two black balls (labelled B1 and B2) and one white ball. A
	second urn contains one black ball and two white balls (labelled W1 and W2).
	Suppose the following experiment is performed. One of the two urns is chosen
	at random. Next a ball is randomly chosen from the urn. Then a second ball is
	chosen at random from the same urn without replacing the first ball.
	
	\begin{enumerate}
	\item What is the probability that two black balls are chosen?
	
	\item What is the probability that two balls of opposite colour are chosen?
	\end{enumerate}
	\solution
	%\input{exemplar/11/16/3/12/main1.tex}
\end{enumerate}

		\item A box of oranges is inspected by examining three randomly selected oranges drawn without replacement. If all the three oranges are good, the box is approved for sale, otherwise, it is rejected. Find the probability that a box containing 15 oranges out of which 12 are good and 3 are bad ones will be approved for sale.
		\label{ncert/12/13/2/3/defs.tex}
		\item Two balls are drawn at random with replacement from a box containing 10 black and 8 red balls. Find the probability that
		\label{ncert/12/13/2/12}
\begin{enumerate}
\item both balls are red.
\item first ball is black and second is red.
\item one of them is black and other is red.
\end{enumerate}

\item In a hostel, 60\% of the students read Hindi newspaper, 40\% read English newspaper and 20\% read both Hindi and English newspapers. A student is selected at random.
		\label{ncert/12/13/2/15}
\begin{enumerate}
\item Find the probability that she reads neither Hindi nor English newspapers.
\item If she reads Hindi newspaper, find the probability that she reads English newspaper.
\item If she reads English newspaper, find the probability that she reads Hindi newspaper.\\
\end{enumerate}
\item The probability of obtaining an even prime number on each die, when a pair of dice is rolled is 
\begin{enumerate}
    \item $0$ 
    
    \item $\frac{1}{3}$ 
    
    \item $\frac{1}{12}$ 
    
    \item $\frac{1}{36}$ 
\end{enumerate}
\solution
		%\begin{enumerate}[label=\thesection.\arabic*,ref=\thesection.\theenumi]
	\item One card is drawn from a well-shuffled deck of 52 cards. Find the probability of getting
\begin{enumerate}
\item A king of red colour 
\item A face card 
\item A red face card
\item The jack of hearts
\item A spade
\item The queen of diamonds

\end{enumerate}
\solution
		%\input{ncert/10/15/1/14/main.tex}
	\item Five cards—the ten, jack, queen, king and ace of diamonds, are well-shuffled with their face downwards. One card is then picked up at random.
\begin{enumerate}
\item
What is the probability that the card is the queen? 
\item
If the queen is drawn and put aside, what is the probability that the second card picked up is (a) an ace? (b) a queen?\\
\end{enumerate}
\solution
		%\input{ncert/10/15/1/15/defs.tex}
	\item A bag contains $5$ red balls and some blue balls. If the probability of drawing a blue ball is double that if a red ball, determine the number of blue balls in the bag. 
		\\
\solution
		%\input{ncert/10/15/2/3/defs.tex}
	\item A card is selected from a pack of 52 cards.
 \begin{enumerate}[label=(\alph*)] 
                 \item How many points are there in the sample space?
                 \item Calculate the probability that the card is an ace of spades.
                 \item Calculate the probability that the card is (i) an ace and (ii) black card.
 \end{enumerate}
\solution
		%\input{ncert/11/16/3/4/main.tex}
\item Four cards are drawn from a well-shuffled deck of 52 cards. What is the probability of obtaining 3 diamonds and one spade.
\\
\solution
		%\input{ncert/11/16/4/2/defs.tex}
\item In a certain lottery 10,000 tickets are sold and ten equal prizes are awarded. What is the probability of not getting a prize if you buy (a) one ticket (b) two tickets (c) 10 tickets ?	
\\
\solution
		%\input{ncert/11/16/4/4/defs.tex}
		%
\item 
Out of 100 students, two sections of 40 and 60 are formed. If you and your friend are among the 100 students, what is the probability that
\begin{enumerate}
\item you both enter the same section?
\item you both enter the different sections?
\end{enumerate}
\solution
		%\input{ncert/11/16/4/5/defs.tex}
	\item 
The number lock of a suitcase has 4 wheels each labelled with ten digits i.e. from 0 to 9.The lock opens with a sequence of four digits with no repeats.What is the probability of a person getting the right sequence to open the suitcase.
\\
\solution
		%\input{ncert/11/16/4/10/defs.tex}
		%
\item 
Two cards are drawn at random and without replacement from a pack of 52 playing cards. Find the probability that both the cards are black.
\\
\solution
		%\input{ncert/12/13/2/2/defs.tex}
		\item A box of oranges is inspected by examining three randomly selected oranges drawn without replacement. If all the three oranges are good, the box is approved for sale, otherwise, it is rejected. Find the probability that a box containing 15 oranges out of which 12 are good and 3 are bad ones will be approved for sale.
		\label{ncert/12/13/2/3/defs.tex}
		\item Two balls are drawn at random with replacement from a box containing 10 black and 8 red balls. Find the probability that
		\label{ncert/12/13/2/12}
\begin{enumerate}
\item both balls are red.
\item first ball is black and second is red.
\item one of them is black and other is red.
\end{enumerate}

\item In a hostel, 60\% of the students read Hindi newspaper, 40\% read English newspaper and 20\% read both Hindi and English newspapers. A student is selected at random.
		\label{ncert/12/13/2/15}
\begin{enumerate}
\item Find the probability that she reads neither Hindi nor English newspapers.
\item If she reads Hindi newspaper, find the probability that she reads English newspaper.
\item If she reads English newspaper, find the probability that she reads Hindi newspaper.\\
\end{enumerate}
\item The probability of obtaining an even prime number on each die, when a pair of dice is rolled is 
\begin{enumerate}
    \item $0$ 
    
    \item $\frac{1}{3}$ 
    
    \item $\frac{1}{12}$ 
    
    \item $\frac{1}{36}$ 
\end{enumerate}
\solution
		%\input{ncert/12/13/2/17/defs.tex}
	\item A bag contains 4 red and 4 black balls, another bag contains 2 red and 6 black balls. One of the two bags is selected at random and a ball is drawn from the bag which is found to be red. Find the probability that the ball is drawn from the first bag.
\\
\solution
		%\input{ncert/12/13/3/2/main.tex}
  \item
  Cards with numbers 2 to 101 are placed in a box. A card is selected at random.Find the probability that the card has
\begin{enumerate}[label=(\roman*)]
	\item an even number 
	\item a square number
\end{enumerate}
\solution
%\input{exemplar/10/13/3/32/main.tex}
\item
The king, queen and jack of clubs are removed from a deck of 52 playing cards and then well shuffled. Now one card is drawn at random from the remaining cards.  Determine the probability that the card is
\begin{enumerate}[label=(\roman*)]
\item a club
\item 10 of hearts
\end{enumerate}
\solution
%\input{exemplar/10/13/3/29/main.tex}
\item A team of medical students doing their internship have to assist during surgeries
at a city hospital. The probabilities of surgeries rated as very complex, complex,
routine, simple or very simple are respectively, 0.15, 0.20, 0.31, 0.26, .08. Find
the probabilities that a particular surgery will be rated
\begin{enumerate}
	\item complex or very complex;
	\item neither very complex nor very simple;
	\item routine or complex
	\item routine or simple
\end{enumerate}
\solution
%\input{exemplar/11/16/3/8(1)/main.tex}
\item A card is selected from a pack of 52 cards.
\begin{enumerate}[label=(\alph*)]
    \item How many points are there in the sample space?
    \item Calculate the probability that the card is an ace of spades.
    \item Calculate the probability that the card is (i) an ace and (ii) black card.
\end{enumerate}
\solution
%\input{exemplar/11/16/3/4/main2.tex}
\item The probability that a non leap year selected at random will contain 53 sundays.
\\
\solution
%\input{exemplar/10/13/1/19/main.tex}
\item One of the four persons John, Rita, Aslam or Gurpreet will be promoted next
month. Consequently the sample space consists of four elementary outcomes
S = {John promoted, Rita promoted, Aslam promoted, Gurpreet promoted}
You are told that the chances of John’s promotion is same as that of Gurpreet,
Rita’s chances of promotion are twice as likely as Johns. Aslam’s chances are
four times that of John.
\begin{enumerate}
	\item Determine
	\begin{enumerate}
		\item P (John promoted)
		\item P (Rita promoted)
		\item P (Aslam promoted)
		\item P (Gurpreet promoted)
	\end{enumerate}
	\item If A = {John promoted or Gurpreet promoted}, find P (A).
\end{enumerate}
\solution
%\input{exemplar/11/16/3/10/main.tex}
\item A card is drawn from a deck of 52 cards. Find the probability of getting a king or a heart or a red card.\\
\solution
%\input{exemplar/11/16/3/15/main.tex}
\item The probability that a student will pass his examination is 0.73, the probability of
the student getting a compartment is 0.13, and the probability that the student will
either pass or get compartment is 0.96. State True or False.\\
\solution
%\input{exemplar/11/16/3/31/main.tex}
\item A card is selected from a pack of 52 cards\\
\begin{enumerate}[label=(\alph*)]
\item How many points are there in the sample space?
\item Calculate the probability that the cards is an ace of spades.
\item Calculate the probability that the card is (i) an ace (ii)black card.\\
\end{enumerate}
%\input{ncert/11/16/3/4_1/Prob_4.tex}
\item In a non-leap year, the probability of having 53 tuesdays or 53 wednesdays is\\
\solution
%\input{exemplar/11/16/3/18/main.tex}
\item There are 1000 sealed envelopes in a box, 10 of them contain a cash prize of
Rs 100 each, 100 of them contain a cash prize of Rs 50 each and 200 of them
contain a cash prize of Rs 10 each and rest do not contain any cash prize. If they
are well shuffled and an envelope is picked up out, what is the probability that it
contains no cash prize?\\
\solution
%\input{exemplar/10/13/3/34/main.tex}
\item 
A die is thrown and a card is selected at random from a deck of 52 playing cards. The probability of getting an even number on the die and a spade card.\\
\solution
%\input{exemplar/12/13/3/78/main.tex}
\item
If 4-digit numbers greater than 5,000 are randomly formed from the digits 0, 1, 3, 5, and 7, what is the probability of forming a number divisible by 5 when:
\begin{enumerate}
    \item The digits are repeated?
    \item The repetition of digits is not allowed?
\end{enumerate}
\solution
%\input{ncert/11/16/4/9/main.tex}
\item Consider the probability space $\brak{\Omega, \mathcal{G}, P}$ where $\Omega = [0,2]$ and $\mathcal{G} = \cbrak{\phi, \Omega, [0,1], (1,2]}$. Let $X$ and $Y$ be two functions on $\Omega$ defined as
\begin{align*}
    X(\omega) = 
    \begin{cases}
        1 & \text{if }\omega \in [0, 1]\\
        2 & \text{if }\omega \in (1, 2]
    \end{cases}
\end{align*}
and
\begin{align*}
    Y(\omega) = 
    \begin{cases}
        2 & \text{if }\omega \in [0, 1.5]\\
        3 & \text{if }\omega \in (1.5, 2].
    \end{cases}
\end{align*}
Then which one of the following statements is true?
\begin{enumerate}
    \item [(A)] $X$ is a random variable with respect to $\mathcal{G}$, but $Y$ is not a random variable with respect to $\mathcal{G}$.
    \item [(B)] $Y$ is a random variable with respect to $\mathcal{G}$, but $X$ is not a random variable with respect to $\mathcal{G}$.
    \item [(C)] Neither $X$ nor $Y$ is a random variable with respect to $\mathcal{G}$.
    \item [(D)] Both $X$ and $Y$ are random variables with respect to $\mathcal{G}$.
\end{enumerate} \hfill (GATE ST 2023)\\
\solution
%\input{gate/ST/2023/14/main.tex}
	\item  A die is loaded in such a way that each odd number is twice as likely to occur as
each even number. Find $P(G)$, where $G$ is the event that a number greater than
3 occurs on a single roll of the die.
\\
\solution
		%\input{exemplar/11/16/3/5/main.tex}
	\item All the jacks, queens and kings are removed from a deck of 52 playing cards. The remaining cards are well shuffled and then one card is drawn at random. Giving ace a value 1 similar value for other cards, find the probability that the card has a value 
		\begin{enumerate}
			\item 7
			\item greater than 7
			\item less than 7
		\end{enumerate}
		%\input{exemplar/10/13/3/30/main.tex}
  \item A Lot consists of 48 mobile phones of which 42 are good, 3 have only minor defects and 3 have major defects.Varnika will buy a phone if it is good but the trader will only buy a mobile if it has no major defects. One phone is selected at random from the lot. What is the probability that it is
\begin{enumerate}
	\item acceptable to Varnika?
            \item acceptable to the trader?
\end{enumerate}
\solution
	%\input{exemplar/10/13/3/40/main.tex}
 \item A student says that if you throw a die, it will show up 1 or not 1. Therefore, the probability of getting 1 and the probability of getting 'not 1' each is equal to $\frac{1}{2}$. Is this correct? Give reasons.\\
 \solution
        %\input{exemplar/10/13/2/9/main.tex}
   \item Four candidates A, B, C, D have ap-
plied for the assignment to coach a school cricket
team. If A is twice as likely to be selected as B, and
B and C are given about the same chance of being
selected, while C is twice as likely to be selected
as D, what are the probabilities that
\begin{enumerate}
\item C will be selected?
\item A will not be selected?
\end{enumerate}
	%\input{exemplar/11/16/3/9/main.tex}
 \item A bag contain 24 balls of which $x$ balls are red, $2x$ are white and $3x$ are blue. A ball is selected at random, What is the probability that it is
\begin{enumerate}[label=\alph*)]
\item not red ?
\item white ?
\end{enumerate}
%\input{exemplar/10/13/3/41/main.tex}
If the letters of the word ASSASSINATION are arranged at random. Find the Probability that
\begin{enumerate}[label=(\alph*)]
\item Four $S's$ come consecutively in the word
\item Two  $I's$ and two $N's$ come together
\item All $A's$ are not coming together
\item No two $A's$ are coming together
\end{enumerate}
%\input{exemplar/11/16/3/14/main.tex}
	\item One urn contains two black balls (labelled B1 and B2) and one white ball. A
	second urn contains one black ball and two white balls (labelled W1 and W2).
	Suppose the following experiment is performed. One of the two urns is chosen
	at random. Next a ball is randomly chosen from the urn. Then a second ball is
	chosen at random from the same urn without replacing the first ball.
	
	\begin{enumerate}
	\item What is the probability that two black balls are chosen?
	
	\item What is the probability that two balls of opposite colour are chosen?
	\end{enumerate}
	\solution
	%\input{exemplar/11/16/3/12/main1.tex}
\end{enumerate}

	\item A bag contains 4 red and 4 black balls, another bag contains 2 red and 6 black balls. One of the two bags is selected at random and a ball is drawn from the bag which is found to be red. Find the probability that the ball is drawn from the first bag.
\\
\solution
		%\begin{table}[H]
	\centering
\begin{tabular}{|c|c|c|}
\hline
Random variable &Value &Definition\\ \hline
\multirow{3}{*}{X} &0 &Slips of Rs 1\\
&1 &Slips of Rs 5\\
&2 &Slips of Rs 13\\ \hline
\multirow{2}{*}{Y} &0 &Box A\\
&1 &Box B\\\hline
\end{tabular}
\caption{}
\label{tab:Distribution}
\end{table}
See \tabref{tab:Distribution}.
\begin{align}
p_{Y}\brak{k}= \begin{cases} 
      \frac{1}{3} & {k=0} \\
      \frac{2}{3 }& {k=1} 
   \end{cases}
   \\
p_{Y|X}\brak{0|0} = \frac{19}{25}\, 
p_{Y|X}\brak{0|1} = \frac{6}{25}\,
p_{Y|X}\brak{1|0} = \frac{45}{50}\,
p_{Y|X}\brak{1|2} = \frac{5}{50}
\end{align}
The desired probability is the probability that a slip drawn at random is marked other than Rs 1,
\begin{align}
&=1-p_X\brak{0}\\
&= p_X(1) + p_X(2)
\end{align}
Using Bayes theorem,
\begin{align}
&= p_Y\brak{0} \times \pr{Y=0 | X=1} + p_Y\brak{1} \times \pr{Y=1|X=2}\\
&=\frac{1}{3} \times \frac{6}{25} + \frac{2}{3} \times \frac{5}{50}\\
&=\frac{11}{75}
\end{align}

\newpage

%\tableofcontents

\bigskip

\renewcommand{\thefigure}{\theenumi}
\renewcommand{\thetable}{\theenumi}
%\renewcommand{\theequation}{\theenumi}

%\begin{abstract}
%%\boldmath
%In this letter, an algorithm for evaluating the exact analytical bit error rate  (BER)  for the piecewise linear (PL) combiner for  multiple relays is presented. Previous results were available only for upto three relays. The algorithm is unique in the sense that  the actual mathematical expressions, that are prohibitively large, need not be explicitly obtained. The diversity gain due to multiple relays is shown through plots of the analytical BER, well supported by simulations. 
%
%\end{abstract}
% IEEEtran.cls defaults to using nonbold math in the Abstract.
% This preserves the distinction between vectors and scalars. However,
% if the journal you are submitting to favors bold math in the abstract,
% then you can use LaTeX's standard command \boldmath at the very start
% of the abstract to achieve this. Many IEEE journals frown on math
% in the abstract anyway.

% Note that keywords are not normally used for peerreview papers.
%\begin{IEEEkeywords}
%Cooperative diversity, decode and forward, piecewise linear
%\end{IEEEkeywords}



% For peer review papers, you can put extra information on the cover
% page as needed:
% \ifCLASSOPTIONpeerreview
% \begin{center} \bfseries EDICS Category: 3-BBND \end{center}
% \fi
%
% For peerreview papers, this IEEEtran command inserts a page break and
% creates the second title. It will be ignored for other modes.
%\IEEEpeerreviewmaketitle




  \item
  Cards with numbers 2 to 101 are placed in a box. A card is selected at random.Find the probability that the card has
\begin{enumerate}[label=(\roman*)]
	\item an even number 
	\item a square number
\end{enumerate}
\solution
%\begin{table}[H]
	\centering
\begin{tabular}{|c|c|c|}
\hline
Random variable &Value &Definition\\ \hline
\multirow{3}{*}{X} &0 &Slips of Rs 1\\
&1 &Slips of Rs 5\\
&2 &Slips of Rs 13\\ \hline
\multirow{2}{*}{Y} &0 &Box A\\
&1 &Box B\\\hline
\end{tabular}
\caption{}
\label{tab:Distribution}
\end{table}
See \tabref{tab:Distribution}.
\begin{align}
p_{Y}\brak{k}= \begin{cases} 
      \frac{1}{3} & {k=0} \\
      \frac{2}{3 }& {k=1} 
   \end{cases}
   \\
p_{Y|X}\brak{0|0} = \frac{19}{25}\, 
p_{Y|X}\brak{0|1} = \frac{6}{25}\,
p_{Y|X}\brak{1|0} = \frac{45}{50}\,
p_{Y|X}\brak{1|2} = \frac{5}{50}
\end{align}
The desired probability is the probability that a slip drawn at random is marked other than Rs 1,
\begin{align}
&=1-p_X\brak{0}\\
&= p_X(1) + p_X(2)
\end{align}
Using Bayes theorem,
\begin{align}
&= p_Y\brak{0} \times \pr{Y=0 | X=1} + p_Y\brak{1} \times \pr{Y=1|X=2}\\
&=\frac{1}{3} \times \frac{6}{25} + \frac{2}{3} \times \frac{5}{50}\\
&=\frac{11}{75}
\end{align}

\newpage

%\tableofcontents

\bigskip

\renewcommand{\thefigure}{\theenumi}
\renewcommand{\thetable}{\theenumi}
%\renewcommand{\theequation}{\theenumi}

%\begin{abstract}
%%\boldmath
%In this letter, an algorithm for evaluating the exact analytical bit error rate  (BER)  for the piecewise linear (PL) combiner for  multiple relays is presented. Previous results were available only for upto three relays. The algorithm is unique in the sense that  the actual mathematical expressions, that are prohibitively large, need not be explicitly obtained. The diversity gain due to multiple relays is shown through plots of the analytical BER, well supported by simulations. 
%
%\end{abstract}
% IEEEtran.cls defaults to using nonbold math in the Abstract.
% This preserves the distinction between vectors and scalars. However,
% if the journal you are submitting to favors bold math in the abstract,
% then you can use LaTeX's standard command \boldmath at the very start
% of the abstract to achieve this. Many IEEE journals frown on math
% in the abstract anyway.

% Note that keywords are not normally used for peerreview papers.
%\begin{IEEEkeywords}
%Cooperative diversity, decode and forward, piecewise linear
%\end{IEEEkeywords}



% For peer review papers, you can put extra information on the cover
% page as needed:
% \ifCLASSOPTIONpeerreview
% \begin{center} \bfseries EDICS Category: 3-BBND \end{center}
% \fi
%
% For peerreview papers, this IEEEtran command inserts a page break and
% creates the second title. It will be ignored for other modes.
%\IEEEpeerreviewmaketitle




\item
The king, queen and jack of clubs are removed from a deck of 52 playing cards and then well shuffled. Now one card is drawn at random from the remaining cards.  Determine the probability that the card is
\begin{enumerate}[label=(\roman*)]
\item a club
\item 10 of hearts
\end{enumerate}
\solution
%\begin{table}[H]
	\centering
\begin{tabular}{|c|c|c|}
\hline
Random variable &Value &Definition\\ \hline
\multirow{3}{*}{X} &0 &Slips of Rs 1\\
&1 &Slips of Rs 5\\
&2 &Slips of Rs 13\\ \hline
\multirow{2}{*}{Y} &0 &Box A\\
&1 &Box B\\\hline
\end{tabular}
\caption{}
\label{tab:Distribution}
\end{table}
See \tabref{tab:Distribution}.
\begin{align}
p_{Y}\brak{k}= \begin{cases} 
      \frac{1}{3} & {k=0} \\
      \frac{2}{3 }& {k=1} 
   \end{cases}
   \\
p_{Y|X}\brak{0|0} = \frac{19}{25}\, 
p_{Y|X}\brak{0|1} = \frac{6}{25}\,
p_{Y|X}\brak{1|0} = \frac{45}{50}\,
p_{Y|X}\brak{1|2} = \frac{5}{50}
\end{align}
The desired probability is the probability that a slip drawn at random is marked other than Rs 1,
\begin{align}
&=1-p_X\brak{0}\\
&= p_X(1) + p_X(2)
\end{align}
Using Bayes theorem,
\begin{align}
&= p_Y\brak{0} \times \pr{Y=0 | X=1} + p_Y\brak{1} \times \pr{Y=1|X=2}\\
&=\frac{1}{3} \times \frac{6}{25} + \frac{2}{3} \times \frac{5}{50}\\
&=\frac{11}{75}
\end{align}

\newpage

%\tableofcontents

\bigskip

\renewcommand{\thefigure}{\theenumi}
\renewcommand{\thetable}{\theenumi}
%\renewcommand{\theequation}{\theenumi}

%\begin{abstract}
%%\boldmath
%In this letter, an algorithm for evaluating the exact analytical bit error rate  (BER)  for the piecewise linear (PL) combiner for  multiple relays is presented. Previous results were available only for upto three relays. The algorithm is unique in the sense that  the actual mathematical expressions, that are prohibitively large, need not be explicitly obtained. The diversity gain due to multiple relays is shown through plots of the analytical BER, well supported by simulations. 
%
%\end{abstract}
% IEEEtran.cls defaults to using nonbold math in the Abstract.
% This preserves the distinction between vectors and scalars. However,
% if the journal you are submitting to favors bold math in the abstract,
% then you can use LaTeX's standard command \boldmath at the very start
% of the abstract to achieve this. Many IEEE journals frown on math
% in the abstract anyway.

% Note that keywords are not normally used for peerreview papers.
%\begin{IEEEkeywords}
%Cooperative diversity, decode and forward, piecewise linear
%\end{IEEEkeywords}



% For peer review papers, you can put extra information on the cover
% page as needed:
% \ifCLASSOPTIONpeerreview
% \begin{center} \bfseries EDICS Category: 3-BBND \end{center}
% \fi
%
% For peerreview papers, this IEEEtran command inserts a page break and
% creates the second title. It will be ignored for other modes.
%\IEEEpeerreviewmaketitle




\item A team of medical students doing their internship have to assist during surgeries
at a city hospital. The probabilities of surgeries rated as very complex, complex,
routine, simple or very simple are respectively, 0.15, 0.20, 0.31, 0.26, .08. Find
the probabilities that a particular surgery will be rated
\begin{enumerate}
	\item complex or very complex;
	\item neither very complex nor very simple;
	\item routine or complex
	\item routine or simple
\end{enumerate}
\solution
%\begin{table}[H]
	\centering
\begin{tabular}{|c|c|c|}
\hline
Random variable &Value &Definition\\ \hline
\multirow{3}{*}{X} &0 &Slips of Rs 1\\
&1 &Slips of Rs 5\\
&2 &Slips of Rs 13\\ \hline
\multirow{2}{*}{Y} &0 &Box A\\
&1 &Box B\\\hline
\end{tabular}
\caption{}
\label{tab:Distribution}
\end{table}
See \tabref{tab:Distribution}.
\begin{align}
p_{Y}\brak{k}= \begin{cases} 
      \frac{1}{3} & {k=0} \\
      \frac{2}{3 }& {k=1} 
   \end{cases}
   \\
p_{Y|X}\brak{0|0} = \frac{19}{25}\, 
p_{Y|X}\brak{0|1} = \frac{6}{25}\,
p_{Y|X}\brak{1|0} = \frac{45}{50}\,
p_{Y|X}\brak{1|2} = \frac{5}{50}
\end{align}
The desired probability is the probability that a slip drawn at random is marked other than Rs 1,
\begin{align}
&=1-p_X\brak{0}\\
&= p_X(1) + p_X(2)
\end{align}
Using Bayes theorem,
\begin{align}
&= p_Y\brak{0} \times \pr{Y=0 | X=1} + p_Y\brak{1} \times \pr{Y=1|X=2}\\
&=\frac{1}{3} \times \frac{6}{25} + \frac{2}{3} \times \frac{5}{50}\\
&=\frac{11}{75}
\end{align}

\newpage

%\tableofcontents

\bigskip

\renewcommand{\thefigure}{\theenumi}
\renewcommand{\thetable}{\theenumi}
%\renewcommand{\theequation}{\theenumi}

%\begin{abstract}
%%\boldmath
%In this letter, an algorithm for evaluating the exact analytical bit error rate  (BER)  for the piecewise linear (PL) combiner for  multiple relays is presented. Previous results were available only for upto three relays. The algorithm is unique in the sense that  the actual mathematical expressions, that are prohibitively large, need not be explicitly obtained. The diversity gain due to multiple relays is shown through plots of the analytical BER, well supported by simulations. 
%
%\end{abstract}
% IEEEtran.cls defaults to using nonbold math in the Abstract.
% This preserves the distinction between vectors and scalars. However,
% if the journal you are submitting to favors bold math in the abstract,
% then you can use LaTeX's standard command \boldmath at the very start
% of the abstract to achieve this. Many IEEE journals frown on math
% in the abstract anyway.

% Note that keywords are not normally used for peerreview papers.
%\begin{IEEEkeywords}
%Cooperative diversity, decode and forward, piecewise linear
%\end{IEEEkeywords}



% For peer review papers, you can put extra information on the cover
% page as needed:
% \ifCLASSOPTIONpeerreview
% \begin{center} \bfseries EDICS Category: 3-BBND \end{center}
% \fi
%
% For peerreview papers, this IEEEtran command inserts a page break and
% creates the second title. It will be ignored for other modes.
%\IEEEpeerreviewmaketitle




\item A card is selected from a pack of 52 cards.
\begin{enumerate}[label=(\alph*)]
    \item How many points are there in the sample space?
    \item Calculate the probability that the card is an ace of spades.
    \item Calculate the probability that the card is (i) an ace and (ii) black card.
\end{enumerate}
\solution
%Let $X$ be an bernoulli rv defined as in \tabref{tab:exemplar/11/16/3/26}.  Then, 
\begin{equation}
    p =
        \frac{4}{11} 
\end{equation}
\begin{table}[H]
	\centering
	\input{exemplar/11/16/3/26/tables/Table2.tex}
	\caption{}
        \label{tab:exemplar/11/16/3/26}
\end{table}

\item The probability that a non leap year selected at random will contain 53 sundays.
\\
\solution
%\begin{table}[H]
	\centering
\begin{tabular}{|c|c|c|}
\hline
Random variable &Value &Definition\\ \hline
\multirow{3}{*}{X} &0 &Slips of Rs 1\\
&1 &Slips of Rs 5\\
&2 &Slips of Rs 13\\ \hline
\multirow{2}{*}{Y} &0 &Box A\\
&1 &Box B\\\hline
\end{tabular}
\caption{}
\label{tab:Distribution}
\end{table}
See \tabref{tab:Distribution}.
\begin{align}
p_{Y}\brak{k}= \begin{cases} 
      \frac{1}{3} & {k=0} \\
      \frac{2}{3 }& {k=1} 
   \end{cases}
   \\
p_{Y|X}\brak{0|0} = \frac{19}{25}\, 
p_{Y|X}\brak{0|1} = \frac{6}{25}\,
p_{Y|X}\brak{1|0} = \frac{45}{50}\,
p_{Y|X}\brak{1|2} = \frac{5}{50}
\end{align}
The desired probability is the probability that a slip drawn at random is marked other than Rs 1,
\begin{align}
&=1-p_X\brak{0}\\
&= p_X(1) + p_X(2)
\end{align}
Using Bayes theorem,
\begin{align}
&= p_Y\brak{0} \times \pr{Y=0 | X=1} + p_Y\brak{1} \times \pr{Y=1|X=2}\\
&=\frac{1}{3} \times \frac{6}{25} + \frac{2}{3} \times \frac{5}{50}\\
&=\frac{11}{75}
\end{align}

\newpage

%\tableofcontents

\bigskip

\renewcommand{\thefigure}{\theenumi}
\renewcommand{\thetable}{\theenumi}
%\renewcommand{\theequation}{\theenumi}

%\begin{abstract}
%%\boldmath
%In this letter, an algorithm for evaluating the exact analytical bit error rate  (BER)  for the piecewise linear (PL) combiner for  multiple relays is presented. Previous results were available only for upto three relays. The algorithm is unique in the sense that  the actual mathematical expressions, that are prohibitively large, need not be explicitly obtained. The diversity gain due to multiple relays is shown through plots of the analytical BER, well supported by simulations. 
%
%\end{abstract}
% IEEEtran.cls defaults to using nonbold math in the Abstract.
% This preserves the distinction between vectors and scalars. However,
% if the journal you are submitting to favors bold math in the abstract,
% then you can use LaTeX's standard command \boldmath at the very start
% of the abstract to achieve this. Many IEEE journals frown on math
% in the abstract anyway.

% Note that keywords are not normally used for peerreview papers.
%\begin{IEEEkeywords}
%Cooperative diversity, decode and forward, piecewise linear
%\end{IEEEkeywords}



% For peer review papers, you can put extra information on the cover
% page as needed:
% \ifCLASSOPTIONpeerreview
% \begin{center} \bfseries EDICS Category: 3-BBND \end{center}
% \fi
%
% For peerreview papers, this IEEEtran command inserts a page break and
% creates the second title. It will be ignored for other modes.
%\IEEEpeerreviewmaketitle




\item One of the four persons John, Rita, Aslam or Gurpreet will be promoted next
month. Consequently the sample space consists of four elementary outcomes
S = {John promoted, Rita promoted, Aslam promoted, Gurpreet promoted}
You are told that the chances of John’s promotion is same as that of Gurpreet,
Rita’s chances of promotion are twice as likely as Johns. Aslam’s chances are
four times that of John.
\begin{enumerate}
	\item Determine
	\begin{enumerate}
		\item P (John promoted)
		\item P (Rita promoted)
		\item P (Aslam promoted)
		\item P (Gurpreet promoted)
	\end{enumerate}
	\item If A = {John promoted or Gurpreet promoted}, find P (A).
\end{enumerate}
\solution
%\begin{table}[H]
	\centering
\begin{tabular}{|c|c|c|}
\hline
Random variable &Value &Definition\\ \hline
\multirow{3}{*}{X} &0 &Slips of Rs 1\\
&1 &Slips of Rs 5\\
&2 &Slips of Rs 13\\ \hline
\multirow{2}{*}{Y} &0 &Box A\\
&1 &Box B\\\hline
\end{tabular}
\caption{}
\label{tab:Distribution}
\end{table}
See \tabref{tab:Distribution}.
\begin{align}
p_{Y}\brak{k}= \begin{cases} 
      \frac{1}{3} & {k=0} \\
      \frac{2}{3 }& {k=1} 
   \end{cases}
   \\
p_{Y|X}\brak{0|0} = \frac{19}{25}\, 
p_{Y|X}\brak{0|1} = \frac{6}{25}\,
p_{Y|X}\brak{1|0} = \frac{45}{50}\,
p_{Y|X}\brak{1|2} = \frac{5}{50}
\end{align}
The desired probability is the probability that a slip drawn at random is marked other than Rs 1,
\begin{align}
&=1-p_X\brak{0}\\
&= p_X(1) + p_X(2)
\end{align}
Using Bayes theorem,
\begin{align}
&= p_Y\brak{0} \times \pr{Y=0 | X=1} + p_Y\brak{1} \times \pr{Y=1|X=2}\\
&=\frac{1}{3} \times \frac{6}{25} + \frac{2}{3} \times \frac{5}{50}\\
&=\frac{11}{75}
\end{align}

\newpage

%\tableofcontents

\bigskip

\renewcommand{\thefigure}{\theenumi}
\renewcommand{\thetable}{\theenumi}
%\renewcommand{\theequation}{\theenumi}

%\begin{abstract}
%%\boldmath
%In this letter, an algorithm for evaluating the exact analytical bit error rate  (BER)  for the piecewise linear (PL) combiner for  multiple relays is presented. Previous results were available only for upto three relays. The algorithm is unique in the sense that  the actual mathematical expressions, that are prohibitively large, need not be explicitly obtained. The diversity gain due to multiple relays is shown through plots of the analytical BER, well supported by simulations. 
%
%\end{abstract}
% IEEEtran.cls defaults to using nonbold math in the Abstract.
% This preserves the distinction between vectors and scalars. However,
% if the journal you are submitting to favors bold math in the abstract,
% then you can use LaTeX's standard command \boldmath at the very start
% of the abstract to achieve this. Many IEEE journals frown on math
% in the abstract anyway.

% Note that keywords are not normally used for peerreview papers.
%\begin{IEEEkeywords}
%Cooperative diversity, decode and forward, piecewise linear
%\end{IEEEkeywords}



% For peer review papers, you can put extra information on the cover
% page as needed:
% \ifCLASSOPTIONpeerreview
% \begin{center} \bfseries EDICS Category: 3-BBND \end{center}
% \fi
%
% For peerreview papers, this IEEEtran command inserts a page break and
% creates the second title. It will be ignored for other modes.
%\IEEEpeerreviewmaketitle




\item A card is drawn from a deck of 52 cards. Find the probability of getting a king or a heart or a red card.\\
\solution
%\begin{table}[H]
	\centering
\begin{tabular}{|c|c|c|}
\hline
Random variable &Value &Definition\\ \hline
\multirow{3}{*}{X} &0 &Slips of Rs 1\\
&1 &Slips of Rs 5\\
&2 &Slips of Rs 13\\ \hline
\multirow{2}{*}{Y} &0 &Box A\\
&1 &Box B\\\hline
\end{tabular}
\caption{}
\label{tab:Distribution}
\end{table}
See \tabref{tab:Distribution}.
\begin{align}
p_{Y}\brak{k}= \begin{cases} 
      \frac{1}{3} & {k=0} \\
      \frac{2}{3 }& {k=1} 
   \end{cases}
   \\
p_{Y|X}\brak{0|0} = \frac{19}{25}\, 
p_{Y|X}\brak{0|1} = \frac{6}{25}\,
p_{Y|X}\brak{1|0} = \frac{45}{50}\,
p_{Y|X}\brak{1|2} = \frac{5}{50}
\end{align}
The desired probability is the probability that a slip drawn at random is marked other than Rs 1,
\begin{align}
&=1-p_X\brak{0}\\
&= p_X(1) + p_X(2)
\end{align}
Using Bayes theorem,
\begin{align}
&= p_Y\brak{0} \times \pr{Y=0 | X=1} + p_Y\brak{1} \times \pr{Y=1|X=2}\\
&=\frac{1}{3} \times \frac{6}{25} + \frac{2}{3} \times \frac{5}{50}\\
&=\frac{11}{75}
\end{align}

\newpage

%\tableofcontents

\bigskip

\renewcommand{\thefigure}{\theenumi}
\renewcommand{\thetable}{\theenumi}
%\renewcommand{\theequation}{\theenumi}

%\begin{abstract}
%%\boldmath
%In this letter, an algorithm for evaluating the exact analytical bit error rate  (BER)  for the piecewise linear (PL) combiner for  multiple relays is presented. Previous results were available only for upto three relays. The algorithm is unique in the sense that  the actual mathematical expressions, that are prohibitively large, need not be explicitly obtained. The diversity gain due to multiple relays is shown through plots of the analytical BER, well supported by simulations. 
%
%\end{abstract}
% IEEEtran.cls defaults to using nonbold math in the Abstract.
% This preserves the distinction between vectors and scalars. However,
% if the journal you are submitting to favors bold math in the abstract,
% then you can use LaTeX's standard command \boldmath at the very start
% of the abstract to achieve this. Many IEEE journals frown on math
% in the abstract anyway.

% Note that keywords are not normally used for peerreview papers.
%\begin{IEEEkeywords}
%Cooperative diversity, decode and forward, piecewise linear
%\end{IEEEkeywords}



% For peer review papers, you can put extra information on the cover
% page as needed:
% \ifCLASSOPTIONpeerreview
% \begin{center} \bfseries EDICS Category: 3-BBND \end{center}
% \fi
%
% For peerreview papers, this IEEEtran command inserts a page break and
% creates the second title. It will be ignored for other modes.
%\IEEEpeerreviewmaketitle




\item The probability that a student will pass his examination is 0.73, the probability of
the student getting a compartment is 0.13, and the probability that the student will
either pass or get compartment is 0.96. State True or False.\\
\solution
%\begin{table}[H]
	\centering
\begin{tabular}{|c|c|c|}
\hline
Random variable &Value &Definition\\ \hline
\multirow{3}{*}{X} &0 &Slips of Rs 1\\
&1 &Slips of Rs 5\\
&2 &Slips of Rs 13\\ \hline
\multirow{2}{*}{Y} &0 &Box A\\
&1 &Box B\\\hline
\end{tabular}
\caption{}
\label{tab:Distribution}
\end{table}
See \tabref{tab:Distribution}.
\begin{align}
p_{Y}\brak{k}= \begin{cases} 
      \frac{1}{3} & {k=0} \\
      \frac{2}{3 }& {k=1} 
   \end{cases}
   \\
p_{Y|X}\brak{0|0} = \frac{19}{25}\, 
p_{Y|X}\brak{0|1} = \frac{6}{25}\,
p_{Y|X}\brak{1|0} = \frac{45}{50}\,
p_{Y|X}\brak{1|2} = \frac{5}{50}
\end{align}
The desired probability is the probability that a slip drawn at random is marked other than Rs 1,
\begin{align}
&=1-p_X\brak{0}\\
&= p_X(1) + p_X(2)
\end{align}
Using Bayes theorem,
\begin{align}
&= p_Y\brak{0} \times \pr{Y=0 | X=1} + p_Y\brak{1} \times \pr{Y=1|X=2}\\
&=\frac{1}{3} \times \frac{6}{25} + \frac{2}{3} \times \frac{5}{50}\\
&=\frac{11}{75}
\end{align}

\newpage

%\tableofcontents

\bigskip

\renewcommand{\thefigure}{\theenumi}
\renewcommand{\thetable}{\theenumi}
%\renewcommand{\theequation}{\theenumi}

%\begin{abstract}
%%\boldmath
%In this letter, an algorithm for evaluating the exact analytical bit error rate  (BER)  for the piecewise linear (PL) combiner for  multiple relays is presented. Previous results were available only for upto three relays. The algorithm is unique in the sense that  the actual mathematical expressions, that are prohibitively large, need not be explicitly obtained. The diversity gain due to multiple relays is shown through plots of the analytical BER, well supported by simulations. 
%
%\end{abstract}
% IEEEtran.cls defaults to using nonbold math in the Abstract.
% This preserves the distinction between vectors and scalars. However,
% if the journal you are submitting to favors bold math in the abstract,
% then you can use LaTeX's standard command \boldmath at the very start
% of the abstract to achieve this. Many IEEE journals frown on math
% in the abstract anyway.

% Note that keywords are not normally used for peerreview papers.
%\begin{IEEEkeywords}
%Cooperative diversity, decode and forward, piecewise linear
%\end{IEEEkeywords}



% For peer review papers, you can put extra information on the cover
% page as needed:
% \ifCLASSOPTIONpeerreview
% \begin{center} \bfseries EDICS Category: 3-BBND \end{center}
% \fi
%
% For peerreview papers, this IEEEtran command inserts a page break and
% creates the second title. It will be ignored for other modes.
%\IEEEpeerreviewmaketitle




\item A card is selected from a pack of 52 cards\\
\begin{enumerate}[label=(\alph*)]
\item How many points are there in the sample space?
\item Calculate the probability that the cards is an ace of spades.
\item Calculate the probability that the card is (i) an ace (ii)black card.\\
\end{enumerate}
%\input{ncert/11/16/3/4_1/Prob_4.tex}
\item In a non-leap year, the probability of having 53 tuesdays or 53 wednesdays is\\
\solution
%A non-leap year has a total of 365 days, and a week has 7 days.\\
So it can be expressed as 
\begin{align}
365\text{days} &=52\times 7+1 \text{day}
\end{align}
$\implies$ 52 tuesdays or wednesdays\\
Random variable X denotes the days of a week
\begin{align}
p_X\brak{k}&=\frac{1}{7}; \quad \brak{1<k<7}
\end{align}
So the probability of extra day being tuesday or wednesday is
\begin{align}
p_X\brak{3}+p_X\brak{4}&=\frac{1}{7}+\frac{1}{7}=\frac{2}{7}
\end{align}



\item There are 1000 sealed envelopes in a box, 10 of them contain a cash prize of
Rs 100 each, 100 of them contain a cash prize of Rs 50 each and 200 of them
contain a cash prize of Rs 10 each and rest do not contain any cash prize. If they
are well shuffled and an envelope is picked up out, what is the probability that it
contains no cash prize?\\
\solution
%\begin{table}[H]
	\centering
\begin{tabular}{|c|c|c|}
\hline
Random variable &Value &Definition\\ \hline
\multirow{3}{*}{X} &0 &Slips of Rs 1\\
&1 &Slips of Rs 5\\
&2 &Slips of Rs 13\\ \hline
\multirow{2}{*}{Y} &0 &Box A\\
&1 &Box B\\\hline
\end{tabular}
\caption{}
\label{tab:Distribution}
\end{table}
See \tabref{tab:Distribution}.
\begin{align}
p_{Y}\brak{k}= \begin{cases} 
      \frac{1}{3} & {k=0} \\
      \frac{2}{3 }& {k=1} 
   \end{cases}
   \\
p_{Y|X}\brak{0|0} = \frac{19}{25}\, 
p_{Y|X}\brak{0|1} = \frac{6}{25}\,
p_{Y|X}\brak{1|0} = \frac{45}{50}\,
p_{Y|X}\brak{1|2} = \frac{5}{50}
\end{align}
The desired probability is the probability that a slip drawn at random is marked other than Rs 1,
\begin{align}
&=1-p_X\brak{0}\\
&= p_X(1) + p_X(2)
\end{align}
Using Bayes theorem,
\begin{align}
&= p_Y\brak{0} \times \pr{Y=0 | X=1} + p_Y\brak{1} \times \pr{Y=1|X=2}\\
&=\frac{1}{3} \times \frac{6}{25} + \frac{2}{3} \times \frac{5}{50}\\
&=\frac{11}{75}
\end{align}

\newpage

%\tableofcontents

\bigskip

\renewcommand{\thefigure}{\theenumi}
\renewcommand{\thetable}{\theenumi}
%\renewcommand{\theequation}{\theenumi}

%\begin{abstract}
%%\boldmath
%In this letter, an algorithm for evaluating the exact analytical bit error rate  (BER)  for the piecewise linear (PL) combiner for  multiple relays is presented. Previous results were available only for upto three relays. The algorithm is unique in the sense that  the actual mathematical expressions, that are prohibitively large, need not be explicitly obtained. The diversity gain due to multiple relays is shown through plots of the analytical BER, well supported by simulations. 
%
%\end{abstract}
% IEEEtran.cls defaults to using nonbold math in the Abstract.
% This preserves the distinction between vectors and scalars. However,
% if the journal you are submitting to favors bold math in the abstract,
% then you can use LaTeX's standard command \boldmath at the very start
% of the abstract to achieve this. Many IEEE journals frown on math
% in the abstract anyway.

% Note that keywords are not normally used for peerreview papers.
%\begin{IEEEkeywords}
%Cooperative diversity, decode and forward, piecewise linear
%\end{IEEEkeywords}



% For peer review papers, you can put extra information on the cover
% page as needed:
% \ifCLASSOPTIONpeerreview
% \begin{center} \bfseries EDICS Category: 3-BBND \end{center}
% \fi
%
% For peerreview papers, this IEEEtran command inserts a page break and
% creates the second title. It will be ignored for other modes.
%\IEEEpeerreviewmaketitle




\item 
A die is thrown and a card is selected at random from a deck of 52 playing cards. The probability of getting an even number on the die and a spade card.\\
\solution
%\begin{table}[H]
	\centering
\begin{tabular}{|c|c|c|}
\hline
Random variable &Value &Definition\\ \hline
\multirow{3}{*}{X} &0 &Slips of Rs 1\\
&1 &Slips of Rs 5\\
&2 &Slips of Rs 13\\ \hline
\multirow{2}{*}{Y} &0 &Box A\\
&1 &Box B\\\hline
\end{tabular}
\caption{}
\label{tab:Distribution}
\end{table}
See \tabref{tab:Distribution}.
\begin{align}
p_{Y}\brak{k}= \begin{cases} 
      \frac{1}{3} & {k=0} \\
      \frac{2}{3 }& {k=1} 
   \end{cases}
   \\
p_{Y|X}\brak{0|0} = \frac{19}{25}\, 
p_{Y|X}\brak{0|1} = \frac{6}{25}\,
p_{Y|X}\brak{1|0} = \frac{45}{50}\,
p_{Y|X}\brak{1|2} = \frac{5}{50}
\end{align}
The desired probability is the probability that a slip drawn at random is marked other than Rs 1,
\begin{align}
&=1-p_X\brak{0}\\
&= p_X(1) + p_X(2)
\end{align}
Using Bayes theorem,
\begin{align}
&= p_Y\brak{0} \times \pr{Y=0 | X=1} + p_Y\brak{1} \times \pr{Y=1|X=2}\\
&=\frac{1}{3} \times \frac{6}{25} + \frac{2}{3} \times \frac{5}{50}\\
&=\frac{11}{75}
\end{align}

\newpage

%\tableofcontents

\bigskip

\renewcommand{\thefigure}{\theenumi}
\renewcommand{\thetable}{\theenumi}
%\renewcommand{\theequation}{\theenumi}

%\begin{abstract}
%%\boldmath
%In this letter, an algorithm for evaluating the exact analytical bit error rate  (BER)  for the piecewise linear (PL) combiner for  multiple relays is presented. Previous results were available only for upto three relays. The algorithm is unique in the sense that  the actual mathematical expressions, that are prohibitively large, need not be explicitly obtained. The diversity gain due to multiple relays is shown through plots of the analytical BER, well supported by simulations. 
%
%\end{abstract}
% IEEEtran.cls defaults to using nonbold math in the Abstract.
% This preserves the distinction between vectors and scalars. However,
% if the journal you are submitting to favors bold math in the abstract,
% then you can use LaTeX's standard command \boldmath at the very start
% of the abstract to achieve this. Many IEEE journals frown on math
% in the abstract anyway.

% Note that keywords are not normally used for peerreview papers.
%\begin{IEEEkeywords}
%Cooperative diversity, decode and forward, piecewise linear
%\end{IEEEkeywords}



% For peer review papers, you can put extra information on the cover
% page as needed:
% \ifCLASSOPTIONpeerreview
% \begin{center} \bfseries EDICS Category: 3-BBND \end{center}
% \fi
%
% For peerreview papers, this IEEEtran command inserts a page break and
% creates the second title. It will be ignored for other modes.
%\IEEEpeerreviewmaketitle




\item
If 4-digit numbers greater than 5,000 are randomly formed from the digits 0, 1, 3, 5, and 7, what is the probability of forming a number divisible by 5 when:
\begin{enumerate}
    \item The digits are repeated?
    \item The repetition of digits is not allowed?
\end{enumerate}
\solution
%\begin{table}[H]
	\centering
\begin{tabular}{|c|c|c|}
\hline
Random variable &Value &Definition\\ \hline
\multirow{3}{*}{X} &0 &Slips of Rs 1\\
&1 &Slips of Rs 5\\
&2 &Slips of Rs 13\\ \hline
\multirow{2}{*}{Y} &0 &Box A\\
&1 &Box B\\\hline
\end{tabular}
\caption{}
\label{tab:Distribution}
\end{table}
See \tabref{tab:Distribution}.
\begin{align}
p_{Y}\brak{k}= \begin{cases} 
      \frac{1}{3} & {k=0} \\
      \frac{2}{3 }& {k=1} 
   \end{cases}
   \\
p_{Y|X}\brak{0|0} = \frac{19}{25}\, 
p_{Y|X}\brak{0|1} = \frac{6}{25}\,
p_{Y|X}\brak{1|0} = \frac{45}{50}\,
p_{Y|X}\brak{1|2} = \frac{5}{50}
\end{align}
The desired probability is the probability that a slip drawn at random is marked other than Rs 1,
\begin{align}
&=1-p_X\brak{0}\\
&= p_X(1) + p_X(2)
\end{align}
Using Bayes theorem,
\begin{align}
&= p_Y\brak{0} \times \pr{Y=0 | X=1} + p_Y\brak{1} \times \pr{Y=1|X=2}\\
&=\frac{1}{3} \times \frac{6}{25} + \frac{2}{3} \times \frac{5}{50}\\
&=\frac{11}{75}
\end{align}

\newpage

%\tableofcontents

\bigskip

\renewcommand{\thefigure}{\theenumi}
\renewcommand{\thetable}{\theenumi}
%\renewcommand{\theequation}{\theenumi}

%\begin{abstract}
%%\boldmath
%In this letter, an algorithm for evaluating the exact analytical bit error rate  (BER)  for the piecewise linear (PL) combiner for  multiple relays is presented. Previous results were available only for upto three relays. The algorithm is unique in the sense that  the actual mathematical expressions, that are prohibitively large, need not be explicitly obtained. The diversity gain due to multiple relays is shown through plots of the analytical BER, well supported by simulations. 
%
%\end{abstract}
% IEEEtran.cls defaults to using nonbold math in the Abstract.
% This preserves the distinction between vectors and scalars. However,
% if the journal you are submitting to favors bold math in the abstract,
% then you can use LaTeX's standard command \boldmath at the very start
% of the abstract to achieve this. Many IEEE journals frown on math
% in the abstract anyway.

% Note that keywords are not normally used for peerreview papers.
%\begin{IEEEkeywords}
%Cooperative diversity, decode and forward, piecewise linear
%\end{IEEEkeywords}



% For peer review papers, you can put extra information on the cover
% page as needed:
% \ifCLASSOPTIONpeerreview
% \begin{center} \bfseries EDICS Category: 3-BBND \end{center}
% \fi
%
% For peerreview papers, this IEEEtran command inserts a page break and
% creates the second title. It will be ignored for other modes.
%\IEEEpeerreviewmaketitle




\item Consider the probability space $\brak{\Omega, \mathcal{G}, P}$ where $\Omega = [0,2]$ and $\mathcal{G} = \cbrak{\phi, \Omega, [0,1], (1,2]}$. Let $X$ and $Y$ be two functions on $\Omega$ defined as
\begin{align*}
    X(\omega) = 
    \begin{cases}
        1 & \text{if }\omega \in [0, 1]\\
        2 & \text{if }\omega \in (1, 2]
    \end{cases}
\end{align*}
and
\begin{align*}
    Y(\omega) = 
    \begin{cases}
        2 & \text{if }\omega \in [0, 1.5]\\
        3 & \text{if }\omega \in (1.5, 2].
    \end{cases}
\end{align*}
Then which one of the following statements is true?
\begin{enumerate}
    \item [(A)] $X$ is a random variable with respect to $\mathcal{G}$, but $Y$ is not a random variable with respect to $\mathcal{G}$.
    \item [(B)] $Y$ is a random variable with respect to $\mathcal{G}$, but $X$ is not a random variable with respect to $\mathcal{G}$.
    \item [(C)] Neither $X$ nor $Y$ is a random variable with respect to $\mathcal{G}$.
    \item [(D)] Both $X$ and $Y$ are random variables with respect to $\mathcal{G}$.
\end{enumerate} \hfill (GATE ST 2023)\\
\solution
%\begin{table}[H]
	\centering
\begin{tabular}{|c|c|c|}
\hline
Random variable &Value &Definition\\ \hline
\multirow{3}{*}{X} &0 &Slips of Rs 1\\
&1 &Slips of Rs 5\\
&2 &Slips of Rs 13\\ \hline
\multirow{2}{*}{Y} &0 &Box A\\
&1 &Box B\\\hline
\end{tabular}
\caption{}
\label{tab:Distribution}
\end{table}
See \tabref{tab:Distribution}.
\begin{align}
p_{Y}\brak{k}= \begin{cases} 
      \frac{1}{3} & {k=0} \\
      \frac{2}{3 }& {k=1} 
   \end{cases}
   \\
p_{Y|X}\brak{0|0} = \frac{19}{25}\, 
p_{Y|X}\brak{0|1} = \frac{6}{25}\,
p_{Y|X}\brak{1|0} = \frac{45}{50}\,
p_{Y|X}\brak{1|2} = \frac{5}{50}
\end{align}
The desired probability is the probability that a slip drawn at random is marked other than Rs 1,
\begin{align}
&=1-p_X\brak{0}\\
&= p_X(1) + p_X(2)
\end{align}
Using Bayes theorem,
\begin{align}
&= p_Y\brak{0} \times \pr{Y=0 | X=1} + p_Y\brak{1} \times \pr{Y=1|X=2}\\
&=\frac{1}{3} \times \frac{6}{25} + \frac{2}{3} \times \frac{5}{50}\\
&=\frac{11}{75}
\end{align}

\newpage

%\tableofcontents

\bigskip

\renewcommand{\thefigure}{\theenumi}
\renewcommand{\thetable}{\theenumi}
%\renewcommand{\theequation}{\theenumi}

%\begin{abstract}
%%\boldmath
%In this letter, an algorithm for evaluating the exact analytical bit error rate  (BER)  for the piecewise linear (PL) combiner for  multiple relays is presented. Previous results were available only for upto three relays. The algorithm is unique in the sense that  the actual mathematical expressions, that are prohibitively large, need not be explicitly obtained. The diversity gain due to multiple relays is shown through plots of the analytical BER, well supported by simulations. 
%
%\end{abstract}
% IEEEtran.cls defaults to using nonbold math in the Abstract.
% This preserves the distinction between vectors and scalars. However,
% if the journal you are submitting to favors bold math in the abstract,
% then you can use LaTeX's standard command \boldmath at the very start
% of the abstract to achieve this. Many IEEE journals frown on math
% in the abstract anyway.

% Note that keywords are not normally used for peerreview papers.
%\begin{IEEEkeywords}
%Cooperative diversity, decode and forward, piecewise linear
%\end{IEEEkeywords}



% For peer review papers, you can put extra information on the cover
% page as needed:
% \ifCLASSOPTIONpeerreview
% \begin{center} \bfseries EDICS Category: 3-BBND \end{center}
% \fi
%
% For peerreview papers, this IEEEtran command inserts a page break and
% creates the second title. It will be ignored for other modes.
%\IEEEpeerreviewmaketitle




	\item  A die is loaded in such a way that each odd number is twice as likely to occur as
each even number. Find $P(G)$, where $G$ is the event that a number greater than
3 occurs on a single roll of the die.
\\
\solution
		%\begin{table}[H]
	\centering
\begin{tabular}{|c|c|c|}
\hline
Random variable &Value &Definition\\ \hline
\multirow{3}{*}{X} &0 &Slips of Rs 1\\
&1 &Slips of Rs 5\\
&2 &Slips of Rs 13\\ \hline
\multirow{2}{*}{Y} &0 &Box A\\
&1 &Box B\\\hline
\end{tabular}
\caption{}
\label{tab:Distribution}
\end{table}
See \tabref{tab:Distribution}.
\begin{align}
p_{Y}\brak{k}= \begin{cases} 
      \frac{1}{3} & {k=0} \\
      \frac{2}{3 }& {k=1} 
   \end{cases}
   \\
p_{Y|X}\brak{0|0} = \frac{19}{25}\, 
p_{Y|X}\brak{0|1} = \frac{6}{25}\,
p_{Y|X}\brak{1|0} = \frac{45}{50}\,
p_{Y|X}\brak{1|2} = \frac{5}{50}
\end{align}
The desired probability is the probability that a slip drawn at random is marked other than Rs 1,
\begin{align}
&=1-p_X\brak{0}\\
&= p_X(1) + p_X(2)
\end{align}
Using Bayes theorem,
\begin{align}
&= p_Y\brak{0} \times \pr{Y=0 | X=1} + p_Y\brak{1} \times \pr{Y=1|X=2}\\
&=\frac{1}{3} \times \frac{6}{25} + \frac{2}{3} \times \frac{5}{50}\\
&=\frac{11}{75}
\end{align}

\newpage

%\tableofcontents

\bigskip

\renewcommand{\thefigure}{\theenumi}
\renewcommand{\thetable}{\theenumi}
%\renewcommand{\theequation}{\theenumi}

%\begin{abstract}
%%\boldmath
%In this letter, an algorithm for evaluating the exact analytical bit error rate  (BER)  for the piecewise linear (PL) combiner for  multiple relays is presented. Previous results were available only for upto three relays. The algorithm is unique in the sense that  the actual mathematical expressions, that are prohibitively large, need not be explicitly obtained. The diversity gain due to multiple relays is shown through plots of the analytical BER, well supported by simulations. 
%
%\end{abstract}
% IEEEtran.cls defaults to using nonbold math in the Abstract.
% This preserves the distinction between vectors and scalars. However,
% if the journal you are submitting to favors bold math in the abstract,
% then you can use LaTeX's standard command \boldmath at the very start
% of the abstract to achieve this. Many IEEE journals frown on math
% in the abstract anyway.

% Note that keywords are not normally used for peerreview papers.
%\begin{IEEEkeywords}
%Cooperative diversity, decode and forward, piecewise linear
%\end{IEEEkeywords}



% For peer review papers, you can put extra information on the cover
% page as needed:
% \ifCLASSOPTIONpeerreview
% \begin{center} \bfseries EDICS Category: 3-BBND \end{center}
% \fi
%
% For peerreview papers, this IEEEtran command inserts a page break and
% creates the second title. It will be ignored for other modes.
%\IEEEpeerreviewmaketitle




	\item All the jacks, queens and kings are removed from a deck of 52 playing cards. The remaining cards are well shuffled and then one card is drawn at random. Giving ace a value 1 similar value for other cards, find the probability that the card has a value 
		\begin{enumerate}
			\item 7
			\item greater than 7
			\item less than 7
		\end{enumerate}
		%Number of cards left after removing all jacks, queens and kings 
\begin{align}
N	= 52 - 4\times 3
	= 40
\end{align}
%\begin{table}[H]
%\def\arraystretch{1.2}
%\begin{tabular}{|c|c|c|}
%\hline
%	\textbf{Parameter} &\textbf{Value} &\textbf{Description}\\ \hline
%	$X$ &1-10 &Represents the value of the card picked \\ \hline
%\end{tabular}
%\end{table}
Let $1 \le X \le 10$ be the value of the card picked.  Then,
\begin{align}
	p_X(k) &= \Pr(X=k)\ \forall\ 1 \leq k \leq 10\\
	&= \frac{4\times 1}{40}\\
	&= \frac{1}{10}\\
	\therefore p_X(k) &= 
	\begin{cases}
		\frac{1}{10} & 1 \leq k \leq 10\\
		0 & \text{otherwise}
	\end{cases}
\end{align}
and
\begin{align}
	F_{X}(k) &= \sum_{m=0}^{k}p_{X}(m) \quad 1 \leq k \leq 10\\
	&= \frac{k}{10}\\
	\therefore F_{X}(k) &= 
	\begin{cases}
		0 & k \leq 0\\
		\frac{k}{10} & 1\leq k \leq 10\\
		1 & k > 10 
	\end{cases}
\end{align}
\begin{enumerate}
	\item Probability that card has value equal to 7 is
		\begin{align}
			 p_{X}(7)
			= \frac{1}{10}
		\end{align}
	\item Probability that card has value greater than 7 is
		\begin{align}
			1 - F_X(7)
			&= 1 - \frac{7}{10}
			\\
			&= \frac{3}{10}
		\end{align}
	\item Probability that card has value less than 7 is
		\begin{align}
			 F_{X}(6)
			=\frac{6}{10}
		\end{align}
\end{enumerate}

  \item A Lot consists of 48 mobile phones of which 42 are good, 3 have only minor defects and 3 have major defects.Varnika will buy a phone if it is good but the trader will only buy a mobile if it has no major defects. One phone is selected at random from the lot. What is the probability that it is
\begin{enumerate}
	\item acceptable to Varnika?
            \item acceptable to the trader?
\end{enumerate}
\solution
	%\begin{table}[H]
	\centering
\begin{tabular}{|c|c|c|}
\hline
Random variable &Value &Definition\\ \hline
\multirow{3}{*}{X} &0 &Slips of Rs 1\\
&1 &Slips of Rs 5\\
&2 &Slips of Rs 13\\ \hline
\multirow{2}{*}{Y} &0 &Box A\\
&1 &Box B\\\hline
\end{tabular}
\caption{}
\label{tab:Distribution}
\end{table}
See \tabref{tab:Distribution}.
\begin{align}
p_{Y}\brak{k}= \begin{cases} 
      \frac{1}{3} & {k=0} \\
      \frac{2}{3 }& {k=1} 
   \end{cases}
   \\
p_{Y|X}\brak{0|0} = \frac{19}{25}\, 
p_{Y|X}\brak{0|1} = \frac{6}{25}\,
p_{Y|X}\brak{1|0} = \frac{45}{50}\,
p_{Y|X}\brak{1|2} = \frac{5}{50}
\end{align}
The desired probability is the probability that a slip drawn at random is marked other than Rs 1,
\begin{align}
&=1-p_X\brak{0}\\
&= p_X(1) + p_X(2)
\end{align}
Using Bayes theorem,
\begin{align}
&= p_Y\brak{0} \times \pr{Y=0 | X=1} + p_Y\brak{1} \times \pr{Y=1|X=2}\\
&=\frac{1}{3} \times \frac{6}{25} + \frac{2}{3} \times \frac{5}{50}\\
&=\frac{11}{75}
\end{align}

\newpage

%\tableofcontents

\bigskip

\renewcommand{\thefigure}{\theenumi}
\renewcommand{\thetable}{\theenumi}
%\renewcommand{\theequation}{\theenumi}

%\begin{abstract}
%%\boldmath
%In this letter, an algorithm for evaluating the exact analytical bit error rate  (BER)  for the piecewise linear (PL) combiner for  multiple relays is presented. Previous results were available only for upto three relays. The algorithm is unique in the sense that  the actual mathematical expressions, that are prohibitively large, need not be explicitly obtained. The diversity gain due to multiple relays is shown through plots of the analytical BER, well supported by simulations. 
%
%\end{abstract}
% IEEEtran.cls defaults to using nonbold math in the Abstract.
% This preserves the distinction between vectors and scalars. However,
% if the journal you are submitting to favors bold math in the abstract,
% then you can use LaTeX's standard command \boldmath at the very start
% of the abstract to achieve this. Many IEEE journals frown on math
% in the abstract anyway.

% Note that keywords are not normally used for peerreview papers.
%\begin{IEEEkeywords}
%Cooperative diversity, decode and forward, piecewise linear
%\end{IEEEkeywords}



% For peer review papers, you can put extra information on the cover
% page as needed:
% \ifCLASSOPTIONpeerreview
% \begin{center} \bfseries EDICS Category: 3-BBND \end{center}
% \fi
%
% For peerreview papers, this IEEEtran command inserts a page break and
% creates the second title. It will be ignored for other modes.
%\IEEEpeerreviewmaketitle




 \item A student says that if you throw a die, it will show up 1 or not 1. Therefore, the probability of getting 1 and the probability of getting 'not 1' each is equal to $\frac{1}{2}$. Is this correct? Give reasons.\\
 \solution
        %\begin{table}[H]
	\centering
\begin{tabular}{|c|c|c|}
\hline
Random variable &Value &Definition\\ \hline
\multirow{3}{*}{X} &0 &Slips of Rs 1\\
&1 &Slips of Rs 5\\
&2 &Slips of Rs 13\\ \hline
\multirow{2}{*}{Y} &0 &Box A\\
&1 &Box B\\\hline
\end{tabular}
\caption{}
\label{tab:Distribution}
\end{table}
See \tabref{tab:Distribution}.
\begin{align}
p_{Y}\brak{k}= \begin{cases} 
      \frac{1}{3} & {k=0} \\
      \frac{2}{3 }& {k=1} 
   \end{cases}
   \\
p_{Y|X}\brak{0|0} = \frac{19}{25}\, 
p_{Y|X}\brak{0|1} = \frac{6}{25}\,
p_{Y|X}\brak{1|0} = \frac{45}{50}\,
p_{Y|X}\brak{1|2} = \frac{5}{50}
\end{align}
The desired probability is the probability that a slip drawn at random is marked other than Rs 1,
\begin{align}
&=1-p_X\brak{0}\\
&= p_X(1) + p_X(2)
\end{align}
Using Bayes theorem,
\begin{align}
&= p_Y\brak{0} \times \pr{Y=0 | X=1} + p_Y\brak{1} \times \pr{Y=1|X=2}\\
&=\frac{1}{3} \times \frac{6}{25} + \frac{2}{3} \times \frac{5}{50}\\
&=\frac{11}{75}
\end{align}

\newpage

%\tableofcontents

\bigskip

\renewcommand{\thefigure}{\theenumi}
\renewcommand{\thetable}{\theenumi}
%\renewcommand{\theequation}{\theenumi}

%\begin{abstract}
%%\boldmath
%In this letter, an algorithm for evaluating the exact analytical bit error rate  (BER)  for the piecewise linear (PL) combiner for  multiple relays is presented. Previous results were available only for upto three relays. The algorithm is unique in the sense that  the actual mathematical expressions, that are prohibitively large, need not be explicitly obtained. The diversity gain due to multiple relays is shown through plots of the analytical BER, well supported by simulations. 
%
%\end{abstract}
% IEEEtran.cls defaults to using nonbold math in the Abstract.
% This preserves the distinction between vectors and scalars. However,
% if the journal you are submitting to favors bold math in the abstract,
% then you can use LaTeX's standard command \boldmath at the very start
% of the abstract to achieve this. Many IEEE journals frown on math
% in the abstract anyway.

% Note that keywords are not normally used for peerreview papers.
%\begin{IEEEkeywords}
%Cooperative diversity, decode and forward, piecewise linear
%\end{IEEEkeywords}



% For peer review papers, you can put extra information on the cover
% page as needed:
% \ifCLASSOPTIONpeerreview
% \begin{center} \bfseries EDICS Category: 3-BBND \end{center}
% \fi
%
% For peerreview papers, this IEEEtran command inserts a page break and
% creates the second title. It will be ignored for other modes.
%\IEEEpeerreviewmaketitle




   \item Four candidates A, B, C, D have ap-
plied for the assignment to coach a school cricket
team. If A is twice as likely to be selected as B, and
B and C are given about the same chance of being
selected, while C is twice as likely to be selected
as D, what are the probabilities that
\begin{enumerate}
\item C will be selected?
\item A will not be selected?
\end{enumerate}
	%\begin{table}[H]
	\centering
\begin{tabular}{|c|c|c|}
\hline
Random variable &Value &Definition\\ \hline
\multirow{3}{*}{X} &0 &Slips of Rs 1\\
&1 &Slips of Rs 5\\
&2 &Slips of Rs 13\\ \hline
\multirow{2}{*}{Y} &0 &Box A\\
&1 &Box B\\\hline
\end{tabular}
\caption{}
\label{tab:Distribution}
\end{table}
See \tabref{tab:Distribution}.
\begin{align}
p_{Y}\brak{k}= \begin{cases} 
      \frac{1}{3} & {k=0} \\
      \frac{2}{3 }& {k=1} 
   \end{cases}
   \\
p_{Y|X}\brak{0|0} = \frac{19}{25}\, 
p_{Y|X}\brak{0|1} = \frac{6}{25}\,
p_{Y|X}\brak{1|0} = \frac{45}{50}\,
p_{Y|X}\brak{1|2} = \frac{5}{50}
\end{align}
The desired probability is the probability that a slip drawn at random is marked other than Rs 1,
\begin{align}
&=1-p_X\brak{0}\\
&= p_X(1) + p_X(2)
\end{align}
Using Bayes theorem,
\begin{align}
&= p_Y\brak{0} \times \pr{Y=0 | X=1} + p_Y\brak{1} \times \pr{Y=1|X=2}\\
&=\frac{1}{3} \times \frac{6}{25} + \frac{2}{3} \times \frac{5}{50}\\
&=\frac{11}{75}
\end{align}

\newpage

%\tableofcontents

\bigskip

\renewcommand{\thefigure}{\theenumi}
\renewcommand{\thetable}{\theenumi}
%\renewcommand{\theequation}{\theenumi}

%\begin{abstract}
%%\boldmath
%In this letter, an algorithm for evaluating the exact analytical bit error rate  (BER)  for the piecewise linear (PL) combiner for  multiple relays is presented. Previous results were available only for upto three relays. The algorithm is unique in the sense that  the actual mathematical expressions, that are prohibitively large, need not be explicitly obtained. The diversity gain due to multiple relays is shown through plots of the analytical BER, well supported by simulations. 
%
%\end{abstract}
% IEEEtran.cls defaults to using nonbold math in the Abstract.
% This preserves the distinction between vectors and scalars. However,
% if the journal you are submitting to favors bold math in the abstract,
% then you can use LaTeX's standard command \boldmath at the very start
% of the abstract to achieve this. Many IEEE journals frown on math
% in the abstract anyway.

% Note that keywords are not normally used for peerreview papers.
%\begin{IEEEkeywords}
%Cooperative diversity, decode and forward, piecewise linear
%\end{IEEEkeywords}



% For peer review papers, you can put extra information on the cover
% page as needed:
% \ifCLASSOPTIONpeerreview
% \begin{center} \bfseries EDICS Category: 3-BBND \end{center}
% \fi
%
% For peerreview papers, this IEEEtran command inserts a page break and
% creates the second title. It will be ignored for other modes.
%\IEEEpeerreviewmaketitle




 \item A bag contain 24 balls of which $x$ balls are red, $2x$ are white and $3x$ are blue. A ball is selected at random, What is the probability that it is
\begin{enumerate}[label=\alph*)]
\item not red ?
\item white ?
\end{enumerate}
%\begin{table}[H]
	\centering
\begin{tabular}{|c|c|c|}
\hline
Random variable &Value &Definition\\ \hline
\multirow{3}{*}{X} &0 &Slips of Rs 1\\
&1 &Slips of Rs 5\\
&2 &Slips of Rs 13\\ \hline
\multirow{2}{*}{Y} &0 &Box A\\
&1 &Box B\\\hline
\end{tabular}
\caption{}
\label{tab:Distribution}
\end{table}
See \tabref{tab:Distribution}.
\begin{align}
p_{Y}\brak{k}= \begin{cases} 
      \frac{1}{3} & {k=0} \\
      \frac{2}{3 }& {k=1} 
   \end{cases}
   \\
p_{Y|X}\brak{0|0} = \frac{19}{25}\, 
p_{Y|X}\brak{0|1} = \frac{6}{25}\,
p_{Y|X}\brak{1|0} = \frac{45}{50}\,
p_{Y|X}\brak{1|2} = \frac{5}{50}
\end{align}
The desired probability is the probability that a slip drawn at random is marked other than Rs 1,
\begin{align}
&=1-p_X\brak{0}\\
&= p_X(1) + p_X(2)
\end{align}
Using Bayes theorem,
\begin{align}
&= p_Y\brak{0} \times \pr{Y=0 | X=1} + p_Y\brak{1} \times \pr{Y=1|X=2}\\
&=\frac{1}{3} \times \frac{6}{25} + \frac{2}{3} \times \frac{5}{50}\\
&=\frac{11}{75}
\end{align}

\newpage

%\tableofcontents

\bigskip

\renewcommand{\thefigure}{\theenumi}
\renewcommand{\thetable}{\theenumi}
%\renewcommand{\theequation}{\theenumi}

%\begin{abstract}
%%\boldmath
%In this letter, an algorithm for evaluating the exact analytical bit error rate  (BER)  for the piecewise linear (PL) combiner for  multiple relays is presented. Previous results were available only for upto three relays. The algorithm is unique in the sense that  the actual mathematical expressions, that are prohibitively large, need not be explicitly obtained. The diversity gain due to multiple relays is shown through plots of the analytical BER, well supported by simulations. 
%
%\end{abstract}
% IEEEtran.cls defaults to using nonbold math in the Abstract.
% This preserves the distinction between vectors and scalars. However,
% if the journal you are submitting to favors bold math in the abstract,
% then you can use LaTeX's standard command \boldmath at the very start
% of the abstract to achieve this. Many IEEE journals frown on math
% in the abstract anyway.

% Note that keywords are not normally used for peerreview papers.
%\begin{IEEEkeywords}
%Cooperative diversity, decode and forward, piecewise linear
%\end{IEEEkeywords}



% For peer review papers, you can put extra information on the cover
% page as needed:
% \ifCLASSOPTIONpeerreview
% \begin{center} \bfseries EDICS Category: 3-BBND \end{center}
% \fi
%
% For peerreview papers, this IEEEtran command inserts a page break and
% creates the second title. It will be ignored for other modes.
%\IEEEpeerreviewmaketitle




If the letters of the word ASSASSINATION are arranged at random. Find the Probability that
\begin{enumerate}[label=(\alph*)]
\item Four $S's$ come consecutively in the word
\item Two  $I's$ and two $N's$ come together
\item All $A's$ are not coming together
\item No two $A's$ are coming together
\end{enumerate}
%\begin{table}[H]
	\centering
\begin{tabular}{|c|c|c|}
\hline
Random variable &Value &Definition\\ \hline
\multirow{3}{*}{X} &0 &Slips of Rs 1\\
&1 &Slips of Rs 5\\
&2 &Slips of Rs 13\\ \hline
\multirow{2}{*}{Y} &0 &Box A\\
&1 &Box B\\\hline
\end{tabular}
\caption{}
\label{tab:Distribution}
\end{table}
See \tabref{tab:Distribution}.
\begin{align}
p_{Y}\brak{k}= \begin{cases} 
      \frac{1}{3} & {k=0} \\
      \frac{2}{3 }& {k=1} 
   \end{cases}
   \\
p_{Y|X}\brak{0|0} = \frac{19}{25}\, 
p_{Y|X}\brak{0|1} = \frac{6}{25}\,
p_{Y|X}\brak{1|0} = \frac{45}{50}\,
p_{Y|X}\brak{1|2} = \frac{5}{50}
\end{align}
The desired probability is the probability that a slip drawn at random is marked other than Rs 1,
\begin{align}
&=1-p_X\brak{0}\\
&= p_X(1) + p_X(2)
\end{align}
Using Bayes theorem,
\begin{align}
&= p_Y\brak{0} \times \pr{Y=0 | X=1} + p_Y\brak{1} \times \pr{Y=1|X=2}\\
&=\frac{1}{3} \times \frac{6}{25} + \frac{2}{3} \times \frac{5}{50}\\
&=\frac{11}{75}
\end{align}

\newpage

%\tableofcontents

\bigskip

\renewcommand{\thefigure}{\theenumi}
\renewcommand{\thetable}{\theenumi}
%\renewcommand{\theequation}{\theenumi}

%\begin{abstract}
%%\boldmath
%In this letter, an algorithm for evaluating the exact analytical bit error rate  (BER)  for the piecewise linear (PL) combiner for  multiple relays is presented. Previous results were available only for upto three relays. The algorithm is unique in the sense that  the actual mathematical expressions, that are prohibitively large, need not be explicitly obtained. The diversity gain due to multiple relays is shown through plots of the analytical BER, well supported by simulations. 
%
%\end{abstract}
% IEEEtran.cls defaults to using nonbold math in the Abstract.
% This preserves the distinction between vectors and scalars. However,
% if the journal you are submitting to favors bold math in the abstract,
% then you can use LaTeX's standard command \boldmath at the very start
% of the abstract to achieve this. Many IEEE journals frown on math
% in the abstract anyway.

% Note that keywords are not normally used for peerreview papers.
%\begin{IEEEkeywords}
%Cooperative diversity, decode and forward, piecewise linear
%\end{IEEEkeywords}



% For peer review papers, you can put extra information on the cover
% page as needed:
% \ifCLASSOPTIONpeerreview
% \begin{center} \bfseries EDICS Category: 3-BBND \end{center}
% \fi
%
% For peerreview papers, this IEEEtran command inserts a page break and
% creates the second title. It will be ignored for other modes.
%\IEEEpeerreviewmaketitle




	\item One urn contains two black balls (labelled B1 and B2) and one white ball. A
	second urn contains one black ball and two white balls (labelled W1 and W2).
	Suppose the following experiment is performed. One of the two urns is chosen
	at random. Next a ball is randomly chosen from the urn. Then a second ball is
	chosen at random from the same urn without replacing the first ball.
	
	\begin{enumerate}
	\item What is the probability that two black balls are chosen?
	
	\item What is the probability that two balls of opposite colour are chosen?
	\end{enumerate}
	\solution
	%\begin{align}
    \label{eq:12.13.6.18.1}
	\because	\pr{A|B} &> \pr{A},\
\frac{\pr{AB}}{\pr{B}} > \pr{A}
\\
    \label{eq:12.13.6.18.2}
	\implies \pr{AB} &> \pr{A}\pr{B}
	\\
	\text{or, } \frac{\pr{AB}}{\pr{A}} &=\pr{B|A} > \pr{A}
\end{align}

\end{enumerate}

	\item 
The number lock of a suitcase has 4 wheels each labelled with ten digits i.e. from 0 to 9.The lock opens with a sequence of four digits with no repeats.What is the probability of a person getting the right sequence to open the suitcase.
\\
\solution
		%\begin{enumerate}[label=\thesection.\arabic*,ref=\thesection.\theenumi]
	\item One card is drawn from a well-shuffled deck of 52 cards. Find the probability of getting
\begin{enumerate}
\item A king of red colour 
\item A face card 
\item A red face card
\item The jack of hearts
\item A spade
\item The queen of diamonds

\end{enumerate}
\solution
		%\begin{table}[H]
	\centering
\begin{tabular}{|c|c|c|}
\hline
Random variable &Value &Definition\\ \hline
\multirow{3}{*}{X} &0 &Slips of Rs 1\\
&1 &Slips of Rs 5\\
&2 &Slips of Rs 13\\ \hline
\multirow{2}{*}{Y} &0 &Box A\\
&1 &Box B\\\hline
\end{tabular}
\caption{}
\label{tab:Distribution}
\end{table}
See \tabref{tab:Distribution}.
\begin{align}
p_{Y}\brak{k}= \begin{cases} 
      \frac{1}{3} & {k=0} \\
      \frac{2}{3 }& {k=1} 
   \end{cases}
   \\
p_{Y|X}\brak{0|0} = \frac{19}{25}\, 
p_{Y|X}\brak{0|1} = \frac{6}{25}\,
p_{Y|X}\brak{1|0} = \frac{45}{50}\,
p_{Y|X}\brak{1|2} = \frac{5}{50}
\end{align}
The desired probability is the probability that a slip drawn at random is marked other than Rs 1,
\begin{align}
&=1-p_X\brak{0}\\
&= p_X(1) + p_X(2)
\end{align}
Using Bayes theorem,
\begin{align}
&= p_Y\brak{0} \times \pr{Y=0 | X=1} + p_Y\brak{1} \times \pr{Y=1|X=2}\\
&=\frac{1}{3} \times \frac{6}{25} + \frac{2}{3} \times \frac{5}{50}\\
&=\frac{11}{75}
\end{align}

\newpage

%\tableofcontents

\bigskip

\renewcommand{\thefigure}{\theenumi}
\renewcommand{\thetable}{\theenumi}
%\renewcommand{\theequation}{\theenumi}

%\begin{abstract}
%%\boldmath
%In this letter, an algorithm for evaluating the exact analytical bit error rate  (BER)  for the piecewise linear (PL) combiner for  multiple relays is presented. Previous results were available only for upto three relays. The algorithm is unique in the sense that  the actual mathematical expressions, that are prohibitively large, need not be explicitly obtained. The diversity gain due to multiple relays is shown through plots of the analytical BER, well supported by simulations. 
%
%\end{abstract}
% IEEEtran.cls defaults to using nonbold math in the Abstract.
% This preserves the distinction between vectors and scalars. However,
% if the journal you are submitting to favors bold math in the abstract,
% then you can use LaTeX's standard command \boldmath at the very start
% of the abstract to achieve this. Many IEEE journals frown on math
% in the abstract anyway.

% Note that keywords are not normally used for peerreview papers.
%\begin{IEEEkeywords}
%Cooperative diversity, decode and forward, piecewise linear
%\end{IEEEkeywords}



% For peer review papers, you can put extra information on the cover
% page as needed:
% \ifCLASSOPTIONpeerreview
% \begin{center} \bfseries EDICS Category: 3-BBND \end{center}
% \fi
%
% For peerreview papers, this IEEEtran command inserts a page break and
% creates the second title. It will be ignored for other modes.
%\IEEEpeerreviewmaketitle




	\item Five cards—the ten, jack, queen, king and ace of diamonds, are well-shuffled with their face downwards. One card is then picked up at random.
\begin{enumerate}
\item
What is the probability that the card is the queen? 
\item
If the queen is drawn and put aside, what is the probability that the second card picked up is (a) an ace? (b) a queen?\\
\end{enumerate}
\solution
		%\begin{enumerate}[label=\thesection.\arabic*,ref=\thesection.\theenumi]
	\item One card is drawn from a well-shuffled deck of 52 cards. Find the probability of getting
\begin{enumerate}
\item A king of red colour 
\item A face card 
\item A red face card
\item The jack of hearts
\item A spade
\item The queen of diamonds

\end{enumerate}
\solution
		%\input{ncert/10/15/1/14/main.tex}
	\item Five cards—the ten, jack, queen, king and ace of diamonds, are well-shuffled with their face downwards. One card is then picked up at random.
\begin{enumerate}
\item
What is the probability that the card is the queen? 
\item
If the queen is drawn and put aside, what is the probability that the second card picked up is (a) an ace? (b) a queen?\\
\end{enumerate}
\solution
		%\input{ncert/10/15/1/15/defs.tex}
	\item A bag contains $5$ red balls and some blue balls. If the probability of drawing a blue ball is double that if a red ball, determine the number of blue balls in the bag. 
		\\
\solution
		%\input{ncert/10/15/2/3/defs.tex}
	\item A card is selected from a pack of 52 cards.
 \begin{enumerate}[label=(\alph*)] 
                 \item How many points are there in the sample space?
                 \item Calculate the probability that the card is an ace of spades.
                 \item Calculate the probability that the card is (i) an ace and (ii) black card.
 \end{enumerate}
\solution
		%\input{ncert/11/16/3/4/main.tex}
\item Four cards are drawn from a well-shuffled deck of 52 cards. What is the probability of obtaining 3 diamonds and one spade.
\\
\solution
		%\input{ncert/11/16/4/2/defs.tex}
\item In a certain lottery 10,000 tickets are sold and ten equal prizes are awarded. What is the probability of not getting a prize if you buy (a) one ticket (b) two tickets (c) 10 tickets ?	
\\
\solution
		%\input{ncert/11/16/4/4/defs.tex}
		%
\item 
Out of 100 students, two sections of 40 and 60 are formed. If you and your friend are among the 100 students, what is the probability that
\begin{enumerate}
\item you both enter the same section?
\item you both enter the different sections?
\end{enumerate}
\solution
		%\input{ncert/11/16/4/5/defs.tex}
	\item 
The number lock of a suitcase has 4 wheels each labelled with ten digits i.e. from 0 to 9.The lock opens with a sequence of four digits with no repeats.What is the probability of a person getting the right sequence to open the suitcase.
\\
\solution
		%\input{ncert/11/16/4/10/defs.tex}
		%
\item 
Two cards are drawn at random and without replacement from a pack of 52 playing cards. Find the probability that both the cards are black.
\\
\solution
		%\input{ncert/12/13/2/2/defs.tex}
		\item A box of oranges is inspected by examining three randomly selected oranges drawn without replacement. If all the three oranges are good, the box is approved for sale, otherwise, it is rejected. Find the probability that a box containing 15 oranges out of which 12 are good and 3 are bad ones will be approved for sale.
		\label{ncert/12/13/2/3/defs.tex}
		\item Two balls are drawn at random with replacement from a box containing 10 black and 8 red balls. Find the probability that
		\label{ncert/12/13/2/12}
\begin{enumerate}
\item both balls are red.
\item first ball is black and second is red.
\item one of them is black and other is red.
\end{enumerate}

\item In a hostel, 60\% of the students read Hindi newspaper, 40\% read English newspaper and 20\% read both Hindi and English newspapers. A student is selected at random.
		\label{ncert/12/13/2/15}
\begin{enumerate}
\item Find the probability that she reads neither Hindi nor English newspapers.
\item If she reads Hindi newspaper, find the probability that she reads English newspaper.
\item If she reads English newspaper, find the probability that she reads Hindi newspaper.\\
\end{enumerate}
\item The probability of obtaining an even prime number on each die, when a pair of dice is rolled is 
\begin{enumerate}
    \item $0$ 
    
    \item $\frac{1}{3}$ 
    
    \item $\frac{1}{12}$ 
    
    \item $\frac{1}{36}$ 
\end{enumerate}
\solution
		%\input{ncert/12/13/2/17/defs.tex}
	\item A bag contains 4 red and 4 black balls, another bag contains 2 red and 6 black balls. One of the two bags is selected at random and a ball is drawn from the bag which is found to be red. Find the probability that the ball is drawn from the first bag.
\\
\solution
		%\input{ncert/12/13/3/2/main.tex}
  \item
  Cards with numbers 2 to 101 are placed in a box. A card is selected at random.Find the probability that the card has
\begin{enumerate}[label=(\roman*)]
	\item an even number 
	\item a square number
\end{enumerate}
\solution
%\input{exemplar/10/13/3/32/main.tex}
\item
The king, queen and jack of clubs are removed from a deck of 52 playing cards and then well shuffled. Now one card is drawn at random from the remaining cards.  Determine the probability that the card is
\begin{enumerate}[label=(\roman*)]
\item a club
\item 10 of hearts
\end{enumerate}
\solution
%\input{exemplar/10/13/3/29/main.tex}
\item A team of medical students doing their internship have to assist during surgeries
at a city hospital. The probabilities of surgeries rated as very complex, complex,
routine, simple or very simple are respectively, 0.15, 0.20, 0.31, 0.26, .08. Find
the probabilities that a particular surgery will be rated
\begin{enumerate}
	\item complex or very complex;
	\item neither very complex nor very simple;
	\item routine or complex
	\item routine or simple
\end{enumerate}
\solution
%\input{exemplar/11/16/3/8(1)/main.tex}
\item A card is selected from a pack of 52 cards.
\begin{enumerate}[label=(\alph*)]
    \item How many points are there in the sample space?
    \item Calculate the probability that the card is an ace of spades.
    \item Calculate the probability that the card is (i) an ace and (ii) black card.
\end{enumerate}
\solution
%\input{exemplar/11/16/3/4/main2.tex}
\item The probability that a non leap year selected at random will contain 53 sundays.
\\
\solution
%\input{exemplar/10/13/1/19/main.tex}
\item One of the four persons John, Rita, Aslam or Gurpreet will be promoted next
month. Consequently the sample space consists of four elementary outcomes
S = {John promoted, Rita promoted, Aslam promoted, Gurpreet promoted}
You are told that the chances of John’s promotion is same as that of Gurpreet,
Rita’s chances of promotion are twice as likely as Johns. Aslam’s chances are
four times that of John.
\begin{enumerate}
	\item Determine
	\begin{enumerate}
		\item P (John promoted)
		\item P (Rita promoted)
		\item P (Aslam promoted)
		\item P (Gurpreet promoted)
	\end{enumerate}
	\item If A = {John promoted or Gurpreet promoted}, find P (A).
\end{enumerate}
\solution
%\input{exemplar/11/16/3/10/main.tex}
\item A card is drawn from a deck of 52 cards. Find the probability of getting a king or a heart or a red card.\\
\solution
%\input{exemplar/11/16/3/15/main.tex}
\item The probability that a student will pass his examination is 0.73, the probability of
the student getting a compartment is 0.13, and the probability that the student will
either pass or get compartment is 0.96. State True or False.\\
\solution
%\input{exemplar/11/16/3/31/main.tex}
\item A card is selected from a pack of 52 cards\\
\begin{enumerate}[label=(\alph*)]
\item How many points are there in the sample space?
\item Calculate the probability that the cards is an ace of spades.
\item Calculate the probability that the card is (i) an ace (ii)black card.\\
\end{enumerate}
%\input{ncert/11/16/3/4_1/Prob_4.tex}
\item In a non-leap year, the probability of having 53 tuesdays or 53 wednesdays is\\
\solution
%\input{exemplar/11/16/3/18/main.tex}
\item There are 1000 sealed envelopes in a box, 10 of them contain a cash prize of
Rs 100 each, 100 of them contain a cash prize of Rs 50 each and 200 of them
contain a cash prize of Rs 10 each and rest do not contain any cash prize. If they
are well shuffled and an envelope is picked up out, what is the probability that it
contains no cash prize?\\
\solution
%\input{exemplar/10/13/3/34/main.tex}
\item 
A die is thrown and a card is selected at random from a deck of 52 playing cards. The probability of getting an even number on the die and a spade card.\\
\solution
%\input{exemplar/12/13/3/78/main.tex}
\item
If 4-digit numbers greater than 5,000 are randomly formed from the digits 0, 1, 3, 5, and 7, what is the probability of forming a number divisible by 5 when:
\begin{enumerate}
    \item The digits are repeated?
    \item The repetition of digits is not allowed?
\end{enumerate}
\solution
%\input{ncert/11/16/4/9/main.tex}
\item Consider the probability space $\brak{\Omega, \mathcal{G}, P}$ where $\Omega = [0,2]$ and $\mathcal{G} = \cbrak{\phi, \Omega, [0,1], (1,2]}$. Let $X$ and $Y$ be two functions on $\Omega$ defined as
\begin{align*}
    X(\omega) = 
    \begin{cases}
        1 & \text{if }\omega \in [0, 1]\\
        2 & \text{if }\omega \in (1, 2]
    \end{cases}
\end{align*}
and
\begin{align*}
    Y(\omega) = 
    \begin{cases}
        2 & \text{if }\omega \in [0, 1.5]\\
        3 & \text{if }\omega \in (1.5, 2].
    \end{cases}
\end{align*}
Then which one of the following statements is true?
\begin{enumerate}
    \item [(A)] $X$ is a random variable with respect to $\mathcal{G}$, but $Y$ is not a random variable with respect to $\mathcal{G}$.
    \item [(B)] $Y$ is a random variable with respect to $\mathcal{G}$, but $X$ is not a random variable with respect to $\mathcal{G}$.
    \item [(C)] Neither $X$ nor $Y$ is a random variable with respect to $\mathcal{G}$.
    \item [(D)] Both $X$ and $Y$ are random variables with respect to $\mathcal{G}$.
\end{enumerate} \hfill (GATE ST 2023)\\
\solution
%\input{gate/ST/2023/14/main.tex}
	\item  A die is loaded in such a way that each odd number is twice as likely to occur as
each even number. Find $P(G)$, where $G$ is the event that a number greater than
3 occurs on a single roll of the die.
\\
\solution
		%\input{exemplar/11/16/3/5/main.tex}
	\item All the jacks, queens and kings are removed from a deck of 52 playing cards. The remaining cards are well shuffled and then one card is drawn at random. Giving ace a value 1 similar value for other cards, find the probability that the card has a value 
		\begin{enumerate}
			\item 7
			\item greater than 7
			\item less than 7
		\end{enumerate}
		%\input{exemplar/10/13/3/30/main.tex}
  \item A Lot consists of 48 mobile phones of which 42 are good, 3 have only minor defects and 3 have major defects.Varnika will buy a phone if it is good but the trader will only buy a mobile if it has no major defects. One phone is selected at random from the lot. What is the probability that it is
\begin{enumerate}
	\item acceptable to Varnika?
            \item acceptable to the trader?
\end{enumerate}
\solution
	%\input{exemplar/10/13/3/40/main.tex}
 \item A student says that if you throw a die, it will show up 1 or not 1. Therefore, the probability of getting 1 and the probability of getting 'not 1' each is equal to $\frac{1}{2}$. Is this correct? Give reasons.\\
 \solution
        %\input{exemplar/10/13/2/9/main.tex}
   \item Four candidates A, B, C, D have ap-
plied for the assignment to coach a school cricket
team. If A is twice as likely to be selected as B, and
B and C are given about the same chance of being
selected, while C is twice as likely to be selected
as D, what are the probabilities that
\begin{enumerate}
\item C will be selected?
\item A will not be selected?
\end{enumerate}
	%\input{exemplar/11/16/3/9/main.tex}
 \item A bag contain 24 balls of which $x$ balls are red, $2x$ are white and $3x$ are blue. A ball is selected at random, What is the probability that it is
\begin{enumerate}[label=\alph*)]
\item not red ?
\item white ?
\end{enumerate}
%\input{exemplar/10/13/3/41/main.tex}
If the letters of the word ASSASSINATION are arranged at random. Find the Probability that
\begin{enumerate}[label=(\alph*)]
\item Four $S's$ come consecutively in the word
\item Two  $I's$ and two $N's$ come together
\item All $A's$ are not coming together
\item No two $A's$ are coming together
\end{enumerate}
%\input{exemplar/11/16/3/14/main.tex}
	\item One urn contains two black balls (labelled B1 and B2) and one white ball. A
	second urn contains one black ball and two white balls (labelled W1 and W2).
	Suppose the following experiment is performed. One of the two urns is chosen
	at random. Next a ball is randomly chosen from the urn. Then a second ball is
	chosen at random from the same urn without replacing the first ball.
	
	\begin{enumerate}
	\item What is the probability that two black balls are chosen?
	
	\item What is the probability that two balls of opposite colour are chosen?
	\end{enumerate}
	\solution
	%\input{exemplar/11/16/3/12/main1.tex}
\end{enumerate}

	\item A bag contains $5$ red balls and some blue balls. If the probability of drawing a blue ball is double that if a red ball, determine the number of blue balls in the bag. 
		\\
\solution
		%\begin{enumerate}[label=\thesection.\arabic*,ref=\thesection.\theenumi]
	\item One card is drawn from a well-shuffled deck of 52 cards. Find the probability of getting
\begin{enumerate}
\item A king of red colour 
\item A face card 
\item A red face card
\item The jack of hearts
\item A spade
\item The queen of diamonds

\end{enumerate}
\solution
		%\input{ncert/10/15/1/14/main.tex}
	\item Five cards—the ten, jack, queen, king and ace of diamonds, are well-shuffled with their face downwards. One card is then picked up at random.
\begin{enumerate}
\item
What is the probability that the card is the queen? 
\item
If the queen is drawn and put aside, what is the probability that the second card picked up is (a) an ace? (b) a queen?\\
\end{enumerate}
\solution
		%\input{ncert/10/15/1/15/defs.tex}
	\item A bag contains $5$ red balls and some blue balls. If the probability of drawing a blue ball is double that if a red ball, determine the number of blue balls in the bag. 
		\\
\solution
		%\input{ncert/10/15/2/3/defs.tex}
	\item A card is selected from a pack of 52 cards.
 \begin{enumerate}[label=(\alph*)] 
                 \item How many points are there in the sample space?
                 \item Calculate the probability that the card is an ace of spades.
                 \item Calculate the probability that the card is (i) an ace and (ii) black card.
 \end{enumerate}
\solution
		%\input{ncert/11/16/3/4/main.tex}
\item Four cards are drawn from a well-shuffled deck of 52 cards. What is the probability of obtaining 3 diamonds and one spade.
\\
\solution
		%\input{ncert/11/16/4/2/defs.tex}
\item In a certain lottery 10,000 tickets are sold and ten equal prizes are awarded. What is the probability of not getting a prize if you buy (a) one ticket (b) two tickets (c) 10 tickets ?	
\\
\solution
		%\input{ncert/11/16/4/4/defs.tex}
		%
\item 
Out of 100 students, two sections of 40 and 60 are formed. If you and your friend are among the 100 students, what is the probability that
\begin{enumerate}
\item you both enter the same section?
\item you both enter the different sections?
\end{enumerate}
\solution
		%\input{ncert/11/16/4/5/defs.tex}
	\item 
The number lock of a suitcase has 4 wheels each labelled with ten digits i.e. from 0 to 9.The lock opens with a sequence of four digits with no repeats.What is the probability of a person getting the right sequence to open the suitcase.
\\
\solution
		%\input{ncert/11/16/4/10/defs.tex}
		%
\item 
Two cards are drawn at random and without replacement from a pack of 52 playing cards. Find the probability that both the cards are black.
\\
\solution
		%\input{ncert/12/13/2/2/defs.tex}
		\item A box of oranges is inspected by examining three randomly selected oranges drawn without replacement. If all the three oranges are good, the box is approved for sale, otherwise, it is rejected. Find the probability that a box containing 15 oranges out of which 12 are good and 3 are bad ones will be approved for sale.
		\label{ncert/12/13/2/3/defs.tex}
		\item Two balls are drawn at random with replacement from a box containing 10 black and 8 red balls. Find the probability that
		\label{ncert/12/13/2/12}
\begin{enumerate}
\item both balls are red.
\item first ball is black and second is red.
\item one of them is black and other is red.
\end{enumerate}

\item In a hostel, 60\% of the students read Hindi newspaper, 40\% read English newspaper and 20\% read both Hindi and English newspapers. A student is selected at random.
		\label{ncert/12/13/2/15}
\begin{enumerate}
\item Find the probability that she reads neither Hindi nor English newspapers.
\item If she reads Hindi newspaper, find the probability that she reads English newspaper.
\item If she reads English newspaper, find the probability that she reads Hindi newspaper.\\
\end{enumerate}
\item The probability of obtaining an even prime number on each die, when a pair of dice is rolled is 
\begin{enumerate}
    \item $0$ 
    
    \item $\frac{1}{3}$ 
    
    \item $\frac{1}{12}$ 
    
    \item $\frac{1}{36}$ 
\end{enumerate}
\solution
		%\input{ncert/12/13/2/17/defs.tex}
	\item A bag contains 4 red and 4 black balls, another bag contains 2 red and 6 black balls. One of the two bags is selected at random and a ball is drawn from the bag which is found to be red. Find the probability that the ball is drawn from the first bag.
\\
\solution
		%\input{ncert/12/13/3/2/main.tex}
  \item
  Cards with numbers 2 to 101 are placed in a box. A card is selected at random.Find the probability that the card has
\begin{enumerate}[label=(\roman*)]
	\item an even number 
	\item a square number
\end{enumerate}
\solution
%\input{exemplar/10/13/3/32/main.tex}
\item
The king, queen and jack of clubs are removed from a deck of 52 playing cards and then well shuffled. Now one card is drawn at random from the remaining cards.  Determine the probability that the card is
\begin{enumerate}[label=(\roman*)]
\item a club
\item 10 of hearts
\end{enumerate}
\solution
%\input{exemplar/10/13/3/29/main.tex}
\item A team of medical students doing their internship have to assist during surgeries
at a city hospital. The probabilities of surgeries rated as very complex, complex,
routine, simple or very simple are respectively, 0.15, 0.20, 0.31, 0.26, .08. Find
the probabilities that a particular surgery will be rated
\begin{enumerate}
	\item complex or very complex;
	\item neither very complex nor very simple;
	\item routine or complex
	\item routine or simple
\end{enumerate}
\solution
%\input{exemplar/11/16/3/8(1)/main.tex}
\item A card is selected from a pack of 52 cards.
\begin{enumerate}[label=(\alph*)]
    \item How many points are there in the sample space?
    \item Calculate the probability that the card is an ace of spades.
    \item Calculate the probability that the card is (i) an ace and (ii) black card.
\end{enumerate}
\solution
%\input{exemplar/11/16/3/4/main2.tex}
\item The probability that a non leap year selected at random will contain 53 sundays.
\\
\solution
%\input{exemplar/10/13/1/19/main.tex}
\item One of the four persons John, Rita, Aslam or Gurpreet will be promoted next
month. Consequently the sample space consists of four elementary outcomes
S = {John promoted, Rita promoted, Aslam promoted, Gurpreet promoted}
You are told that the chances of John’s promotion is same as that of Gurpreet,
Rita’s chances of promotion are twice as likely as Johns. Aslam’s chances are
four times that of John.
\begin{enumerate}
	\item Determine
	\begin{enumerate}
		\item P (John promoted)
		\item P (Rita promoted)
		\item P (Aslam promoted)
		\item P (Gurpreet promoted)
	\end{enumerate}
	\item If A = {John promoted or Gurpreet promoted}, find P (A).
\end{enumerate}
\solution
%\input{exemplar/11/16/3/10/main.tex}
\item A card is drawn from a deck of 52 cards. Find the probability of getting a king or a heart or a red card.\\
\solution
%\input{exemplar/11/16/3/15/main.tex}
\item The probability that a student will pass his examination is 0.73, the probability of
the student getting a compartment is 0.13, and the probability that the student will
either pass or get compartment is 0.96. State True or False.\\
\solution
%\input{exemplar/11/16/3/31/main.tex}
\item A card is selected from a pack of 52 cards\\
\begin{enumerate}[label=(\alph*)]
\item How many points are there in the sample space?
\item Calculate the probability that the cards is an ace of spades.
\item Calculate the probability that the card is (i) an ace (ii)black card.\\
\end{enumerate}
%\input{ncert/11/16/3/4_1/Prob_4.tex}
\item In a non-leap year, the probability of having 53 tuesdays or 53 wednesdays is\\
\solution
%\input{exemplar/11/16/3/18/main.tex}
\item There are 1000 sealed envelopes in a box, 10 of them contain a cash prize of
Rs 100 each, 100 of them contain a cash prize of Rs 50 each and 200 of them
contain a cash prize of Rs 10 each and rest do not contain any cash prize. If they
are well shuffled and an envelope is picked up out, what is the probability that it
contains no cash prize?\\
\solution
%\input{exemplar/10/13/3/34/main.tex}
\item 
A die is thrown and a card is selected at random from a deck of 52 playing cards. The probability of getting an even number on the die and a spade card.\\
\solution
%\input{exemplar/12/13/3/78/main.tex}
\item
If 4-digit numbers greater than 5,000 are randomly formed from the digits 0, 1, 3, 5, and 7, what is the probability of forming a number divisible by 5 when:
\begin{enumerate}
    \item The digits are repeated?
    \item The repetition of digits is not allowed?
\end{enumerate}
\solution
%\input{ncert/11/16/4/9/main.tex}
\item Consider the probability space $\brak{\Omega, \mathcal{G}, P}$ where $\Omega = [0,2]$ and $\mathcal{G} = \cbrak{\phi, \Omega, [0,1], (1,2]}$. Let $X$ and $Y$ be two functions on $\Omega$ defined as
\begin{align*}
    X(\omega) = 
    \begin{cases}
        1 & \text{if }\omega \in [0, 1]\\
        2 & \text{if }\omega \in (1, 2]
    \end{cases}
\end{align*}
and
\begin{align*}
    Y(\omega) = 
    \begin{cases}
        2 & \text{if }\omega \in [0, 1.5]\\
        3 & \text{if }\omega \in (1.5, 2].
    \end{cases}
\end{align*}
Then which one of the following statements is true?
\begin{enumerate}
    \item [(A)] $X$ is a random variable with respect to $\mathcal{G}$, but $Y$ is not a random variable with respect to $\mathcal{G}$.
    \item [(B)] $Y$ is a random variable with respect to $\mathcal{G}$, but $X$ is not a random variable with respect to $\mathcal{G}$.
    \item [(C)] Neither $X$ nor $Y$ is a random variable with respect to $\mathcal{G}$.
    \item [(D)] Both $X$ and $Y$ are random variables with respect to $\mathcal{G}$.
\end{enumerate} \hfill (GATE ST 2023)\\
\solution
%\input{gate/ST/2023/14/main.tex}
	\item  A die is loaded in such a way that each odd number is twice as likely to occur as
each even number. Find $P(G)$, where $G$ is the event that a number greater than
3 occurs on a single roll of the die.
\\
\solution
		%\input{exemplar/11/16/3/5/main.tex}
	\item All the jacks, queens and kings are removed from a deck of 52 playing cards. The remaining cards are well shuffled and then one card is drawn at random. Giving ace a value 1 similar value for other cards, find the probability that the card has a value 
		\begin{enumerate}
			\item 7
			\item greater than 7
			\item less than 7
		\end{enumerate}
		%\input{exemplar/10/13/3/30/main.tex}
  \item A Lot consists of 48 mobile phones of which 42 are good, 3 have only minor defects and 3 have major defects.Varnika will buy a phone if it is good but the trader will only buy a mobile if it has no major defects. One phone is selected at random from the lot. What is the probability that it is
\begin{enumerate}
	\item acceptable to Varnika?
            \item acceptable to the trader?
\end{enumerate}
\solution
	%\input{exemplar/10/13/3/40/main.tex}
 \item A student says that if you throw a die, it will show up 1 or not 1. Therefore, the probability of getting 1 and the probability of getting 'not 1' each is equal to $\frac{1}{2}$. Is this correct? Give reasons.\\
 \solution
        %\input{exemplar/10/13/2/9/main.tex}
   \item Four candidates A, B, C, D have ap-
plied for the assignment to coach a school cricket
team. If A is twice as likely to be selected as B, and
B and C are given about the same chance of being
selected, while C is twice as likely to be selected
as D, what are the probabilities that
\begin{enumerate}
\item C will be selected?
\item A will not be selected?
\end{enumerate}
	%\input{exemplar/11/16/3/9/main.tex}
 \item A bag contain 24 balls of which $x$ balls are red, $2x$ are white and $3x$ are blue. A ball is selected at random, What is the probability that it is
\begin{enumerate}[label=\alph*)]
\item not red ?
\item white ?
\end{enumerate}
%\input{exemplar/10/13/3/41/main.tex}
If the letters of the word ASSASSINATION are arranged at random. Find the Probability that
\begin{enumerate}[label=(\alph*)]
\item Four $S's$ come consecutively in the word
\item Two  $I's$ and two $N's$ come together
\item All $A's$ are not coming together
\item No two $A's$ are coming together
\end{enumerate}
%\input{exemplar/11/16/3/14/main.tex}
	\item One urn contains two black balls (labelled B1 and B2) and one white ball. A
	second urn contains one black ball and two white balls (labelled W1 and W2).
	Suppose the following experiment is performed. One of the two urns is chosen
	at random. Next a ball is randomly chosen from the urn. Then a second ball is
	chosen at random from the same urn without replacing the first ball.
	
	\begin{enumerate}
	\item What is the probability that two black balls are chosen?
	
	\item What is the probability that two balls of opposite colour are chosen?
	\end{enumerate}
	\solution
	%\input{exemplar/11/16/3/12/main1.tex}
\end{enumerate}

	\item A card is selected from a pack of 52 cards.
 \begin{enumerate}[label=(\alph*)] 
                 \item How many points are there in the sample space?
                 \item Calculate the probability that the card is an ace of spades.
                 \item Calculate the probability that the card is (i) an ace and (ii) black card.
 \end{enumerate}
\solution
		%\begin{table}[H]
	\centering
\begin{tabular}{|c|c|c|}
\hline
Random variable &Value &Definition\\ \hline
\multirow{3}{*}{X} &0 &Slips of Rs 1\\
&1 &Slips of Rs 5\\
&2 &Slips of Rs 13\\ \hline
\multirow{2}{*}{Y} &0 &Box A\\
&1 &Box B\\\hline
\end{tabular}
\caption{}
\label{tab:Distribution}
\end{table}
See \tabref{tab:Distribution}.
\begin{align}
p_{Y}\brak{k}= \begin{cases} 
      \frac{1}{3} & {k=0} \\
      \frac{2}{3 }& {k=1} 
   \end{cases}
   \\
p_{Y|X}\brak{0|0} = \frac{19}{25}\, 
p_{Y|X}\brak{0|1} = \frac{6}{25}\,
p_{Y|X}\brak{1|0} = \frac{45}{50}\,
p_{Y|X}\brak{1|2} = \frac{5}{50}
\end{align}
The desired probability is the probability that a slip drawn at random is marked other than Rs 1,
\begin{align}
&=1-p_X\brak{0}\\
&= p_X(1) + p_X(2)
\end{align}
Using Bayes theorem,
\begin{align}
&= p_Y\brak{0} \times \pr{Y=0 | X=1} + p_Y\brak{1} \times \pr{Y=1|X=2}\\
&=\frac{1}{3} \times \frac{6}{25} + \frac{2}{3} \times \frac{5}{50}\\
&=\frac{11}{75}
\end{align}

\newpage

%\tableofcontents

\bigskip

\renewcommand{\thefigure}{\theenumi}
\renewcommand{\thetable}{\theenumi}
%\renewcommand{\theequation}{\theenumi}

%\begin{abstract}
%%\boldmath
%In this letter, an algorithm for evaluating the exact analytical bit error rate  (BER)  for the piecewise linear (PL) combiner for  multiple relays is presented. Previous results were available only for upto three relays. The algorithm is unique in the sense that  the actual mathematical expressions, that are prohibitively large, need not be explicitly obtained. The diversity gain due to multiple relays is shown through plots of the analytical BER, well supported by simulations. 
%
%\end{abstract}
% IEEEtran.cls defaults to using nonbold math in the Abstract.
% This preserves the distinction between vectors and scalars. However,
% if the journal you are submitting to favors bold math in the abstract,
% then you can use LaTeX's standard command \boldmath at the very start
% of the abstract to achieve this. Many IEEE journals frown on math
% in the abstract anyway.

% Note that keywords are not normally used for peerreview papers.
%\begin{IEEEkeywords}
%Cooperative diversity, decode and forward, piecewise linear
%\end{IEEEkeywords}



% For peer review papers, you can put extra information on the cover
% page as needed:
% \ifCLASSOPTIONpeerreview
% \begin{center} \bfseries EDICS Category: 3-BBND \end{center}
% \fi
%
% For peerreview papers, this IEEEtran command inserts a page break and
% creates the second title. It will be ignored for other modes.
%\IEEEpeerreviewmaketitle




\item Four cards are drawn from a well-shuffled deck of 52 cards. What is the probability of obtaining 3 diamonds and one spade.
\\
\solution
		%\begin{enumerate}[label=\thesection.\arabic*,ref=\thesection.\theenumi]
	\item One card is drawn from a well-shuffled deck of 52 cards. Find the probability of getting
\begin{enumerate}
\item A king of red colour 
\item A face card 
\item A red face card
\item The jack of hearts
\item A spade
\item The queen of diamonds

\end{enumerate}
\solution
		%\input{ncert/10/15/1/14/main.tex}
	\item Five cards—the ten, jack, queen, king and ace of diamonds, are well-shuffled with their face downwards. One card is then picked up at random.
\begin{enumerate}
\item
What is the probability that the card is the queen? 
\item
If the queen is drawn and put aside, what is the probability that the second card picked up is (a) an ace? (b) a queen?\\
\end{enumerate}
\solution
		%\input{ncert/10/15/1/15/defs.tex}
	\item A bag contains $5$ red balls and some blue balls. If the probability of drawing a blue ball is double that if a red ball, determine the number of blue balls in the bag. 
		\\
\solution
		%\input{ncert/10/15/2/3/defs.tex}
	\item A card is selected from a pack of 52 cards.
 \begin{enumerate}[label=(\alph*)] 
                 \item How many points are there in the sample space?
                 \item Calculate the probability that the card is an ace of spades.
                 \item Calculate the probability that the card is (i) an ace and (ii) black card.
 \end{enumerate}
\solution
		%\input{ncert/11/16/3/4/main.tex}
\item Four cards are drawn from a well-shuffled deck of 52 cards. What is the probability of obtaining 3 diamonds and one spade.
\\
\solution
		%\input{ncert/11/16/4/2/defs.tex}
\item In a certain lottery 10,000 tickets are sold and ten equal prizes are awarded. What is the probability of not getting a prize if you buy (a) one ticket (b) two tickets (c) 10 tickets ?	
\\
\solution
		%\input{ncert/11/16/4/4/defs.tex}
		%
\item 
Out of 100 students, two sections of 40 and 60 are formed. If you and your friend are among the 100 students, what is the probability that
\begin{enumerate}
\item you both enter the same section?
\item you both enter the different sections?
\end{enumerate}
\solution
		%\input{ncert/11/16/4/5/defs.tex}
	\item 
The number lock of a suitcase has 4 wheels each labelled with ten digits i.e. from 0 to 9.The lock opens with a sequence of four digits with no repeats.What is the probability of a person getting the right sequence to open the suitcase.
\\
\solution
		%\input{ncert/11/16/4/10/defs.tex}
		%
\item 
Two cards are drawn at random and without replacement from a pack of 52 playing cards. Find the probability that both the cards are black.
\\
\solution
		%\input{ncert/12/13/2/2/defs.tex}
		\item A box of oranges is inspected by examining three randomly selected oranges drawn without replacement. If all the three oranges are good, the box is approved for sale, otherwise, it is rejected. Find the probability that a box containing 15 oranges out of which 12 are good and 3 are bad ones will be approved for sale.
		\label{ncert/12/13/2/3/defs.tex}
		\item Two balls are drawn at random with replacement from a box containing 10 black and 8 red balls. Find the probability that
		\label{ncert/12/13/2/12}
\begin{enumerate}
\item both balls are red.
\item first ball is black and second is red.
\item one of them is black and other is red.
\end{enumerate}

\item In a hostel, 60\% of the students read Hindi newspaper, 40\% read English newspaper and 20\% read both Hindi and English newspapers. A student is selected at random.
		\label{ncert/12/13/2/15}
\begin{enumerate}
\item Find the probability that she reads neither Hindi nor English newspapers.
\item If she reads Hindi newspaper, find the probability that she reads English newspaper.
\item If she reads English newspaper, find the probability that she reads Hindi newspaper.\\
\end{enumerate}
\item The probability of obtaining an even prime number on each die, when a pair of dice is rolled is 
\begin{enumerate}
    \item $0$ 
    
    \item $\frac{1}{3}$ 
    
    \item $\frac{1}{12}$ 
    
    \item $\frac{1}{36}$ 
\end{enumerate}
\solution
		%\input{ncert/12/13/2/17/defs.tex}
	\item A bag contains 4 red and 4 black balls, another bag contains 2 red and 6 black balls. One of the two bags is selected at random and a ball is drawn from the bag which is found to be red. Find the probability that the ball is drawn from the first bag.
\\
\solution
		%\input{ncert/12/13/3/2/main.tex}
  \item
  Cards with numbers 2 to 101 are placed in a box. A card is selected at random.Find the probability that the card has
\begin{enumerate}[label=(\roman*)]
	\item an even number 
	\item a square number
\end{enumerate}
\solution
%\input{exemplar/10/13/3/32/main.tex}
\item
The king, queen and jack of clubs are removed from a deck of 52 playing cards and then well shuffled. Now one card is drawn at random from the remaining cards.  Determine the probability that the card is
\begin{enumerate}[label=(\roman*)]
\item a club
\item 10 of hearts
\end{enumerate}
\solution
%\input{exemplar/10/13/3/29/main.tex}
\item A team of medical students doing their internship have to assist during surgeries
at a city hospital. The probabilities of surgeries rated as very complex, complex,
routine, simple or very simple are respectively, 0.15, 0.20, 0.31, 0.26, .08. Find
the probabilities that a particular surgery will be rated
\begin{enumerate}
	\item complex or very complex;
	\item neither very complex nor very simple;
	\item routine or complex
	\item routine or simple
\end{enumerate}
\solution
%\input{exemplar/11/16/3/8(1)/main.tex}
\item A card is selected from a pack of 52 cards.
\begin{enumerate}[label=(\alph*)]
    \item How many points are there in the sample space?
    \item Calculate the probability that the card is an ace of spades.
    \item Calculate the probability that the card is (i) an ace and (ii) black card.
\end{enumerate}
\solution
%\input{exemplar/11/16/3/4/main2.tex}
\item The probability that a non leap year selected at random will contain 53 sundays.
\\
\solution
%\input{exemplar/10/13/1/19/main.tex}
\item One of the four persons John, Rita, Aslam or Gurpreet will be promoted next
month. Consequently the sample space consists of four elementary outcomes
S = {John promoted, Rita promoted, Aslam promoted, Gurpreet promoted}
You are told that the chances of John’s promotion is same as that of Gurpreet,
Rita’s chances of promotion are twice as likely as Johns. Aslam’s chances are
four times that of John.
\begin{enumerate}
	\item Determine
	\begin{enumerate}
		\item P (John promoted)
		\item P (Rita promoted)
		\item P (Aslam promoted)
		\item P (Gurpreet promoted)
	\end{enumerate}
	\item If A = {John promoted or Gurpreet promoted}, find P (A).
\end{enumerate}
\solution
%\input{exemplar/11/16/3/10/main.tex}
\item A card is drawn from a deck of 52 cards. Find the probability of getting a king or a heart or a red card.\\
\solution
%\input{exemplar/11/16/3/15/main.tex}
\item The probability that a student will pass his examination is 0.73, the probability of
the student getting a compartment is 0.13, and the probability that the student will
either pass or get compartment is 0.96. State True or False.\\
\solution
%\input{exemplar/11/16/3/31/main.tex}
\item A card is selected from a pack of 52 cards\\
\begin{enumerate}[label=(\alph*)]
\item How many points are there in the sample space?
\item Calculate the probability that the cards is an ace of spades.
\item Calculate the probability that the card is (i) an ace (ii)black card.\\
\end{enumerate}
%\input{ncert/11/16/3/4_1/Prob_4.tex}
\item In a non-leap year, the probability of having 53 tuesdays or 53 wednesdays is\\
\solution
%\input{exemplar/11/16/3/18/main.tex}
\item There are 1000 sealed envelopes in a box, 10 of them contain a cash prize of
Rs 100 each, 100 of them contain a cash prize of Rs 50 each and 200 of them
contain a cash prize of Rs 10 each and rest do not contain any cash prize. If they
are well shuffled and an envelope is picked up out, what is the probability that it
contains no cash prize?\\
\solution
%\input{exemplar/10/13/3/34/main.tex}
\item 
A die is thrown and a card is selected at random from a deck of 52 playing cards. The probability of getting an even number on the die and a spade card.\\
\solution
%\input{exemplar/12/13/3/78/main.tex}
\item
If 4-digit numbers greater than 5,000 are randomly formed from the digits 0, 1, 3, 5, and 7, what is the probability of forming a number divisible by 5 when:
\begin{enumerate}
    \item The digits are repeated?
    \item The repetition of digits is not allowed?
\end{enumerate}
\solution
%\input{ncert/11/16/4/9/main.tex}
\item Consider the probability space $\brak{\Omega, \mathcal{G}, P}$ where $\Omega = [0,2]$ and $\mathcal{G} = \cbrak{\phi, \Omega, [0,1], (1,2]}$. Let $X$ and $Y$ be two functions on $\Omega$ defined as
\begin{align*}
    X(\omega) = 
    \begin{cases}
        1 & \text{if }\omega \in [0, 1]\\
        2 & \text{if }\omega \in (1, 2]
    \end{cases}
\end{align*}
and
\begin{align*}
    Y(\omega) = 
    \begin{cases}
        2 & \text{if }\omega \in [0, 1.5]\\
        3 & \text{if }\omega \in (1.5, 2].
    \end{cases}
\end{align*}
Then which one of the following statements is true?
\begin{enumerate}
    \item [(A)] $X$ is a random variable with respect to $\mathcal{G}$, but $Y$ is not a random variable with respect to $\mathcal{G}$.
    \item [(B)] $Y$ is a random variable with respect to $\mathcal{G}$, but $X$ is not a random variable with respect to $\mathcal{G}$.
    \item [(C)] Neither $X$ nor $Y$ is a random variable with respect to $\mathcal{G}$.
    \item [(D)] Both $X$ and $Y$ are random variables with respect to $\mathcal{G}$.
\end{enumerate} \hfill (GATE ST 2023)\\
\solution
%\input{gate/ST/2023/14/main.tex}
	\item  A die is loaded in such a way that each odd number is twice as likely to occur as
each even number. Find $P(G)$, where $G$ is the event that a number greater than
3 occurs on a single roll of the die.
\\
\solution
		%\input{exemplar/11/16/3/5/main.tex}
	\item All the jacks, queens and kings are removed from a deck of 52 playing cards. The remaining cards are well shuffled and then one card is drawn at random. Giving ace a value 1 similar value for other cards, find the probability that the card has a value 
		\begin{enumerate}
			\item 7
			\item greater than 7
			\item less than 7
		\end{enumerate}
		%\input{exemplar/10/13/3/30/main.tex}
  \item A Lot consists of 48 mobile phones of which 42 are good, 3 have only minor defects and 3 have major defects.Varnika will buy a phone if it is good but the trader will only buy a mobile if it has no major defects. One phone is selected at random from the lot. What is the probability that it is
\begin{enumerate}
	\item acceptable to Varnika?
            \item acceptable to the trader?
\end{enumerate}
\solution
	%\input{exemplar/10/13/3/40/main.tex}
 \item A student says that if you throw a die, it will show up 1 or not 1. Therefore, the probability of getting 1 and the probability of getting 'not 1' each is equal to $\frac{1}{2}$. Is this correct? Give reasons.\\
 \solution
        %\input{exemplar/10/13/2/9/main.tex}
   \item Four candidates A, B, C, D have ap-
plied for the assignment to coach a school cricket
team. If A is twice as likely to be selected as B, and
B and C are given about the same chance of being
selected, while C is twice as likely to be selected
as D, what are the probabilities that
\begin{enumerate}
\item C will be selected?
\item A will not be selected?
\end{enumerate}
	%\input{exemplar/11/16/3/9/main.tex}
 \item A bag contain 24 balls of which $x$ balls are red, $2x$ are white and $3x$ are blue. A ball is selected at random, What is the probability that it is
\begin{enumerate}[label=\alph*)]
\item not red ?
\item white ?
\end{enumerate}
%\input{exemplar/10/13/3/41/main.tex}
If the letters of the word ASSASSINATION are arranged at random. Find the Probability that
\begin{enumerate}[label=(\alph*)]
\item Four $S's$ come consecutively in the word
\item Two  $I's$ and two $N's$ come together
\item All $A's$ are not coming together
\item No two $A's$ are coming together
\end{enumerate}
%\input{exemplar/11/16/3/14/main.tex}
	\item One urn contains two black balls (labelled B1 and B2) and one white ball. A
	second urn contains one black ball and two white balls (labelled W1 and W2).
	Suppose the following experiment is performed. One of the two urns is chosen
	at random. Next a ball is randomly chosen from the urn. Then a second ball is
	chosen at random from the same urn without replacing the first ball.
	
	\begin{enumerate}
	\item What is the probability that two black balls are chosen?
	
	\item What is the probability that two balls of opposite colour are chosen?
	\end{enumerate}
	\solution
	%\input{exemplar/11/16/3/12/main1.tex}
\end{enumerate}

\item In a certain lottery 10,000 tickets are sold and ten equal prizes are awarded. What is the probability of not getting a prize if you buy (a) one ticket (b) two tickets (c) 10 tickets ?	
\\
\solution
		%\begin{enumerate}[label=\thesection.\arabic*,ref=\thesection.\theenumi]
	\item One card is drawn from a well-shuffled deck of 52 cards. Find the probability of getting
\begin{enumerate}
\item A king of red colour 
\item A face card 
\item A red face card
\item The jack of hearts
\item A spade
\item The queen of diamonds

\end{enumerate}
\solution
		%\input{ncert/10/15/1/14/main.tex}
	\item Five cards—the ten, jack, queen, king and ace of diamonds, are well-shuffled with their face downwards. One card is then picked up at random.
\begin{enumerate}
\item
What is the probability that the card is the queen? 
\item
If the queen is drawn and put aside, what is the probability that the second card picked up is (a) an ace? (b) a queen?\\
\end{enumerate}
\solution
		%\input{ncert/10/15/1/15/defs.tex}
	\item A bag contains $5$ red balls and some blue balls. If the probability of drawing a blue ball is double that if a red ball, determine the number of blue balls in the bag. 
		\\
\solution
		%\input{ncert/10/15/2/3/defs.tex}
	\item A card is selected from a pack of 52 cards.
 \begin{enumerate}[label=(\alph*)] 
                 \item How many points are there in the sample space?
                 \item Calculate the probability that the card is an ace of spades.
                 \item Calculate the probability that the card is (i) an ace and (ii) black card.
 \end{enumerate}
\solution
		%\input{ncert/11/16/3/4/main.tex}
\item Four cards are drawn from a well-shuffled deck of 52 cards. What is the probability of obtaining 3 diamonds and one spade.
\\
\solution
		%\input{ncert/11/16/4/2/defs.tex}
\item In a certain lottery 10,000 tickets are sold and ten equal prizes are awarded. What is the probability of not getting a prize if you buy (a) one ticket (b) two tickets (c) 10 tickets ?	
\\
\solution
		%\input{ncert/11/16/4/4/defs.tex}
		%
\item 
Out of 100 students, two sections of 40 and 60 are formed. If you and your friend are among the 100 students, what is the probability that
\begin{enumerate}
\item you both enter the same section?
\item you both enter the different sections?
\end{enumerate}
\solution
		%\input{ncert/11/16/4/5/defs.tex}
	\item 
The number lock of a suitcase has 4 wheels each labelled with ten digits i.e. from 0 to 9.The lock opens with a sequence of four digits with no repeats.What is the probability of a person getting the right sequence to open the suitcase.
\\
\solution
		%\input{ncert/11/16/4/10/defs.tex}
		%
\item 
Two cards are drawn at random and without replacement from a pack of 52 playing cards. Find the probability that both the cards are black.
\\
\solution
		%\input{ncert/12/13/2/2/defs.tex}
		\item A box of oranges is inspected by examining three randomly selected oranges drawn without replacement. If all the three oranges are good, the box is approved for sale, otherwise, it is rejected. Find the probability that a box containing 15 oranges out of which 12 are good and 3 are bad ones will be approved for sale.
		\label{ncert/12/13/2/3/defs.tex}
		\item Two balls are drawn at random with replacement from a box containing 10 black and 8 red balls. Find the probability that
		\label{ncert/12/13/2/12}
\begin{enumerate}
\item both balls are red.
\item first ball is black and second is red.
\item one of them is black and other is red.
\end{enumerate}

\item In a hostel, 60\% of the students read Hindi newspaper, 40\% read English newspaper and 20\% read both Hindi and English newspapers. A student is selected at random.
		\label{ncert/12/13/2/15}
\begin{enumerate}
\item Find the probability that she reads neither Hindi nor English newspapers.
\item If she reads Hindi newspaper, find the probability that she reads English newspaper.
\item If she reads English newspaper, find the probability that she reads Hindi newspaper.\\
\end{enumerate}
\item The probability of obtaining an even prime number on each die, when a pair of dice is rolled is 
\begin{enumerate}
    \item $0$ 
    
    \item $\frac{1}{3}$ 
    
    \item $\frac{1}{12}$ 
    
    \item $\frac{1}{36}$ 
\end{enumerate}
\solution
		%\input{ncert/12/13/2/17/defs.tex}
	\item A bag contains 4 red and 4 black balls, another bag contains 2 red and 6 black balls. One of the two bags is selected at random and a ball is drawn from the bag which is found to be red. Find the probability that the ball is drawn from the first bag.
\\
\solution
		%\input{ncert/12/13/3/2/main.tex}
  \item
  Cards with numbers 2 to 101 are placed in a box. A card is selected at random.Find the probability that the card has
\begin{enumerate}[label=(\roman*)]
	\item an even number 
	\item a square number
\end{enumerate}
\solution
%\input{exemplar/10/13/3/32/main.tex}
\item
The king, queen and jack of clubs are removed from a deck of 52 playing cards and then well shuffled. Now one card is drawn at random from the remaining cards.  Determine the probability that the card is
\begin{enumerate}[label=(\roman*)]
\item a club
\item 10 of hearts
\end{enumerate}
\solution
%\input{exemplar/10/13/3/29/main.tex}
\item A team of medical students doing their internship have to assist during surgeries
at a city hospital. The probabilities of surgeries rated as very complex, complex,
routine, simple or very simple are respectively, 0.15, 0.20, 0.31, 0.26, .08. Find
the probabilities that a particular surgery will be rated
\begin{enumerate}
	\item complex or very complex;
	\item neither very complex nor very simple;
	\item routine or complex
	\item routine or simple
\end{enumerate}
\solution
%\input{exemplar/11/16/3/8(1)/main.tex}
\item A card is selected from a pack of 52 cards.
\begin{enumerate}[label=(\alph*)]
    \item How many points are there in the sample space?
    \item Calculate the probability that the card is an ace of spades.
    \item Calculate the probability that the card is (i) an ace and (ii) black card.
\end{enumerate}
\solution
%\input{exemplar/11/16/3/4/main2.tex}
\item The probability that a non leap year selected at random will contain 53 sundays.
\\
\solution
%\input{exemplar/10/13/1/19/main.tex}
\item One of the four persons John, Rita, Aslam or Gurpreet will be promoted next
month. Consequently the sample space consists of four elementary outcomes
S = {John promoted, Rita promoted, Aslam promoted, Gurpreet promoted}
You are told that the chances of John’s promotion is same as that of Gurpreet,
Rita’s chances of promotion are twice as likely as Johns. Aslam’s chances are
four times that of John.
\begin{enumerate}
	\item Determine
	\begin{enumerate}
		\item P (John promoted)
		\item P (Rita promoted)
		\item P (Aslam promoted)
		\item P (Gurpreet promoted)
	\end{enumerate}
	\item If A = {John promoted or Gurpreet promoted}, find P (A).
\end{enumerate}
\solution
%\input{exemplar/11/16/3/10/main.tex}
\item A card is drawn from a deck of 52 cards. Find the probability of getting a king or a heart or a red card.\\
\solution
%\input{exemplar/11/16/3/15/main.tex}
\item The probability that a student will pass his examination is 0.73, the probability of
the student getting a compartment is 0.13, and the probability that the student will
either pass or get compartment is 0.96. State True or False.\\
\solution
%\input{exemplar/11/16/3/31/main.tex}
\item A card is selected from a pack of 52 cards\\
\begin{enumerate}[label=(\alph*)]
\item How many points are there in the sample space?
\item Calculate the probability that the cards is an ace of spades.
\item Calculate the probability that the card is (i) an ace (ii)black card.\\
\end{enumerate}
%\input{ncert/11/16/3/4_1/Prob_4.tex}
\item In a non-leap year, the probability of having 53 tuesdays or 53 wednesdays is\\
\solution
%\input{exemplar/11/16/3/18/main.tex}
\item There are 1000 sealed envelopes in a box, 10 of them contain a cash prize of
Rs 100 each, 100 of them contain a cash prize of Rs 50 each and 200 of them
contain a cash prize of Rs 10 each and rest do not contain any cash prize. If they
are well shuffled and an envelope is picked up out, what is the probability that it
contains no cash prize?\\
\solution
%\input{exemplar/10/13/3/34/main.tex}
\item 
A die is thrown and a card is selected at random from a deck of 52 playing cards. The probability of getting an even number on the die and a spade card.\\
\solution
%\input{exemplar/12/13/3/78/main.tex}
\item
If 4-digit numbers greater than 5,000 are randomly formed from the digits 0, 1, 3, 5, and 7, what is the probability of forming a number divisible by 5 when:
\begin{enumerate}
    \item The digits are repeated?
    \item The repetition of digits is not allowed?
\end{enumerate}
\solution
%\input{ncert/11/16/4/9/main.tex}
\item Consider the probability space $\brak{\Omega, \mathcal{G}, P}$ where $\Omega = [0,2]$ and $\mathcal{G} = \cbrak{\phi, \Omega, [0,1], (1,2]}$. Let $X$ and $Y$ be two functions on $\Omega$ defined as
\begin{align*}
    X(\omega) = 
    \begin{cases}
        1 & \text{if }\omega \in [0, 1]\\
        2 & \text{if }\omega \in (1, 2]
    \end{cases}
\end{align*}
and
\begin{align*}
    Y(\omega) = 
    \begin{cases}
        2 & \text{if }\omega \in [0, 1.5]\\
        3 & \text{if }\omega \in (1.5, 2].
    \end{cases}
\end{align*}
Then which one of the following statements is true?
\begin{enumerate}
    \item [(A)] $X$ is a random variable with respect to $\mathcal{G}$, but $Y$ is not a random variable with respect to $\mathcal{G}$.
    \item [(B)] $Y$ is a random variable with respect to $\mathcal{G}$, but $X$ is not a random variable with respect to $\mathcal{G}$.
    \item [(C)] Neither $X$ nor $Y$ is a random variable with respect to $\mathcal{G}$.
    \item [(D)] Both $X$ and $Y$ are random variables with respect to $\mathcal{G}$.
\end{enumerate} \hfill (GATE ST 2023)\\
\solution
%\input{gate/ST/2023/14/main.tex}
	\item  A die is loaded in such a way that each odd number is twice as likely to occur as
each even number. Find $P(G)$, where $G$ is the event that a number greater than
3 occurs on a single roll of the die.
\\
\solution
		%\input{exemplar/11/16/3/5/main.tex}
	\item All the jacks, queens and kings are removed from a deck of 52 playing cards. The remaining cards are well shuffled and then one card is drawn at random. Giving ace a value 1 similar value for other cards, find the probability that the card has a value 
		\begin{enumerate}
			\item 7
			\item greater than 7
			\item less than 7
		\end{enumerate}
		%\input{exemplar/10/13/3/30/main.tex}
  \item A Lot consists of 48 mobile phones of which 42 are good, 3 have only minor defects and 3 have major defects.Varnika will buy a phone if it is good but the trader will only buy a mobile if it has no major defects. One phone is selected at random from the lot. What is the probability that it is
\begin{enumerate}
	\item acceptable to Varnika?
            \item acceptable to the trader?
\end{enumerate}
\solution
	%\input{exemplar/10/13/3/40/main.tex}
 \item A student says that if you throw a die, it will show up 1 or not 1. Therefore, the probability of getting 1 and the probability of getting 'not 1' each is equal to $\frac{1}{2}$. Is this correct? Give reasons.\\
 \solution
        %\input{exemplar/10/13/2/9/main.tex}
   \item Four candidates A, B, C, D have ap-
plied for the assignment to coach a school cricket
team. If A is twice as likely to be selected as B, and
B and C are given about the same chance of being
selected, while C is twice as likely to be selected
as D, what are the probabilities that
\begin{enumerate}
\item C will be selected?
\item A will not be selected?
\end{enumerate}
	%\input{exemplar/11/16/3/9/main.tex}
 \item A bag contain 24 balls of which $x$ balls are red, $2x$ are white and $3x$ are blue. A ball is selected at random, What is the probability that it is
\begin{enumerate}[label=\alph*)]
\item not red ?
\item white ?
\end{enumerate}
%\input{exemplar/10/13/3/41/main.tex}
If the letters of the word ASSASSINATION are arranged at random. Find the Probability that
\begin{enumerate}[label=(\alph*)]
\item Four $S's$ come consecutively in the word
\item Two  $I's$ and two $N's$ come together
\item All $A's$ are not coming together
\item No two $A's$ are coming together
\end{enumerate}
%\input{exemplar/11/16/3/14/main.tex}
	\item One urn contains two black balls (labelled B1 and B2) and one white ball. A
	second urn contains one black ball and two white balls (labelled W1 and W2).
	Suppose the following experiment is performed. One of the two urns is chosen
	at random. Next a ball is randomly chosen from the urn. Then a second ball is
	chosen at random from the same urn without replacing the first ball.
	
	\begin{enumerate}
	\item What is the probability that two black balls are chosen?
	
	\item What is the probability that two balls of opposite colour are chosen?
	\end{enumerate}
	\solution
	%\input{exemplar/11/16/3/12/main1.tex}
\end{enumerate}

		%
\item 
Out of 100 students, two sections of 40 and 60 are formed. If you and your friend are among the 100 students, what is the probability that
\begin{enumerate}
\item you both enter the same section?
\item you both enter the different sections?
\end{enumerate}
\solution
		%\begin{enumerate}[label=\thesection.\arabic*,ref=\thesection.\theenumi]
	\item One card is drawn from a well-shuffled deck of 52 cards. Find the probability of getting
\begin{enumerate}
\item A king of red colour 
\item A face card 
\item A red face card
\item The jack of hearts
\item A spade
\item The queen of diamonds

\end{enumerate}
\solution
		%\input{ncert/10/15/1/14/main.tex}
	\item Five cards—the ten, jack, queen, king and ace of diamonds, are well-shuffled with their face downwards. One card is then picked up at random.
\begin{enumerate}
\item
What is the probability that the card is the queen? 
\item
If the queen is drawn and put aside, what is the probability that the second card picked up is (a) an ace? (b) a queen?\\
\end{enumerate}
\solution
		%\input{ncert/10/15/1/15/defs.tex}
	\item A bag contains $5$ red balls and some blue balls. If the probability of drawing a blue ball is double that if a red ball, determine the number of blue balls in the bag. 
		\\
\solution
		%\input{ncert/10/15/2/3/defs.tex}
	\item A card is selected from a pack of 52 cards.
 \begin{enumerate}[label=(\alph*)] 
                 \item How many points are there in the sample space?
                 \item Calculate the probability that the card is an ace of spades.
                 \item Calculate the probability that the card is (i) an ace and (ii) black card.
 \end{enumerate}
\solution
		%\input{ncert/11/16/3/4/main.tex}
\item Four cards are drawn from a well-shuffled deck of 52 cards. What is the probability of obtaining 3 diamonds and one spade.
\\
\solution
		%\input{ncert/11/16/4/2/defs.tex}
\item In a certain lottery 10,000 tickets are sold and ten equal prizes are awarded. What is the probability of not getting a prize if you buy (a) one ticket (b) two tickets (c) 10 tickets ?	
\\
\solution
		%\input{ncert/11/16/4/4/defs.tex}
		%
\item 
Out of 100 students, two sections of 40 and 60 are formed. If you and your friend are among the 100 students, what is the probability that
\begin{enumerate}
\item you both enter the same section?
\item you both enter the different sections?
\end{enumerate}
\solution
		%\input{ncert/11/16/4/5/defs.tex}
	\item 
The number lock of a suitcase has 4 wheels each labelled with ten digits i.e. from 0 to 9.The lock opens with a sequence of four digits with no repeats.What is the probability of a person getting the right sequence to open the suitcase.
\\
\solution
		%\input{ncert/11/16/4/10/defs.tex}
		%
\item 
Two cards are drawn at random and without replacement from a pack of 52 playing cards. Find the probability that both the cards are black.
\\
\solution
		%\input{ncert/12/13/2/2/defs.tex}
		\item A box of oranges is inspected by examining three randomly selected oranges drawn without replacement. If all the three oranges are good, the box is approved for sale, otherwise, it is rejected. Find the probability that a box containing 15 oranges out of which 12 are good and 3 are bad ones will be approved for sale.
		\label{ncert/12/13/2/3/defs.tex}
		\item Two balls are drawn at random with replacement from a box containing 10 black and 8 red balls. Find the probability that
		\label{ncert/12/13/2/12}
\begin{enumerate}
\item both balls are red.
\item first ball is black and second is red.
\item one of them is black and other is red.
\end{enumerate}

\item In a hostel, 60\% of the students read Hindi newspaper, 40\% read English newspaper and 20\% read both Hindi and English newspapers. A student is selected at random.
		\label{ncert/12/13/2/15}
\begin{enumerate}
\item Find the probability that she reads neither Hindi nor English newspapers.
\item If she reads Hindi newspaper, find the probability that she reads English newspaper.
\item If she reads English newspaper, find the probability that she reads Hindi newspaper.\\
\end{enumerate}
\item The probability of obtaining an even prime number on each die, when a pair of dice is rolled is 
\begin{enumerate}
    \item $0$ 
    
    \item $\frac{1}{3}$ 
    
    \item $\frac{1}{12}$ 
    
    \item $\frac{1}{36}$ 
\end{enumerate}
\solution
		%\input{ncert/12/13/2/17/defs.tex}
	\item A bag contains 4 red and 4 black balls, another bag contains 2 red and 6 black balls. One of the two bags is selected at random and a ball is drawn from the bag which is found to be red. Find the probability that the ball is drawn from the first bag.
\\
\solution
		%\input{ncert/12/13/3/2/main.tex}
  \item
  Cards with numbers 2 to 101 are placed in a box. A card is selected at random.Find the probability that the card has
\begin{enumerate}[label=(\roman*)]
	\item an even number 
	\item a square number
\end{enumerate}
\solution
%\input{exemplar/10/13/3/32/main.tex}
\item
The king, queen and jack of clubs are removed from a deck of 52 playing cards and then well shuffled. Now one card is drawn at random from the remaining cards.  Determine the probability that the card is
\begin{enumerate}[label=(\roman*)]
\item a club
\item 10 of hearts
\end{enumerate}
\solution
%\input{exemplar/10/13/3/29/main.tex}
\item A team of medical students doing their internship have to assist during surgeries
at a city hospital. The probabilities of surgeries rated as very complex, complex,
routine, simple or very simple are respectively, 0.15, 0.20, 0.31, 0.26, .08. Find
the probabilities that a particular surgery will be rated
\begin{enumerate}
	\item complex or very complex;
	\item neither very complex nor very simple;
	\item routine or complex
	\item routine or simple
\end{enumerate}
\solution
%\input{exemplar/11/16/3/8(1)/main.tex}
\item A card is selected from a pack of 52 cards.
\begin{enumerate}[label=(\alph*)]
    \item How many points are there in the sample space?
    \item Calculate the probability that the card is an ace of spades.
    \item Calculate the probability that the card is (i) an ace and (ii) black card.
\end{enumerate}
\solution
%\input{exemplar/11/16/3/4/main2.tex}
\item The probability that a non leap year selected at random will contain 53 sundays.
\\
\solution
%\input{exemplar/10/13/1/19/main.tex}
\item One of the four persons John, Rita, Aslam or Gurpreet will be promoted next
month. Consequently the sample space consists of four elementary outcomes
S = {John promoted, Rita promoted, Aslam promoted, Gurpreet promoted}
You are told that the chances of John’s promotion is same as that of Gurpreet,
Rita’s chances of promotion are twice as likely as Johns. Aslam’s chances are
four times that of John.
\begin{enumerate}
	\item Determine
	\begin{enumerate}
		\item P (John promoted)
		\item P (Rita promoted)
		\item P (Aslam promoted)
		\item P (Gurpreet promoted)
	\end{enumerate}
	\item If A = {John promoted or Gurpreet promoted}, find P (A).
\end{enumerate}
\solution
%\input{exemplar/11/16/3/10/main.tex}
\item A card is drawn from a deck of 52 cards. Find the probability of getting a king or a heart or a red card.\\
\solution
%\input{exemplar/11/16/3/15/main.tex}
\item The probability that a student will pass his examination is 0.73, the probability of
the student getting a compartment is 0.13, and the probability that the student will
either pass or get compartment is 0.96. State True or False.\\
\solution
%\input{exemplar/11/16/3/31/main.tex}
\item A card is selected from a pack of 52 cards\\
\begin{enumerate}[label=(\alph*)]
\item How many points are there in the sample space?
\item Calculate the probability that the cards is an ace of spades.
\item Calculate the probability that the card is (i) an ace (ii)black card.\\
\end{enumerate}
%\input{ncert/11/16/3/4_1/Prob_4.tex}
\item In a non-leap year, the probability of having 53 tuesdays or 53 wednesdays is\\
\solution
%\input{exemplar/11/16/3/18/main.tex}
\item There are 1000 sealed envelopes in a box, 10 of them contain a cash prize of
Rs 100 each, 100 of them contain a cash prize of Rs 50 each and 200 of them
contain a cash prize of Rs 10 each and rest do not contain any cash prize. If they
are well shuffled and an envelope is picked up out, what is the probability that it
contains no cash prize?\\
\solution
%\input{exemplar/10/13/3/34/main.tex}
\item 
A die is thrown and a card is selected at random from a deck of 52 playing cards. The probability of getting an even number on the die and a spade card.\\
\solution
%\input{exemplar/12/13/3/78/main.tex}
\item
If 4-digit numbers greater than 5,000 are randomly formed from the digits 0, 1, 3, 5, and 7, what is the probability of forming a number divisible by 5 when:
\begin{enumerate}
    \item The digits are repeated?
    \item The repetition of digits is not allowed?
\end{enumerate}
\solution
%\input{ncert/11/16/4/9/main.tex}
\item Consider the probability space $\brak{\Omega, \mathcal{G}, P}$ where $\Omega = [0,2]$ and $\mathcal{G} = \cbrak{\phi, \Omega, [0,1], (1,2]}$. Let $X$ and $Y$ be two functions on $\Omega$ defined as
\begin{align*}
    X(\omega) = 
    \begin{cases}
        1 & \text{if }\omega \in [0, 1]\\
        2 & \text{if }\omega \in (1, 2]
    \end{cases}
\end{align*}
and
\begin{align*}
    Y(\omega) = 
    \begin{cases}
        2 & \text{if }\omega \in [0, 1.5]\\
        3 & \text{if }\omega \in (1.5, 2].
    \end{cases}
\end{align*}
Then which one of the following statements is true?
\begin{enumerate}
    \item [(A)] $X$ is a random variable with respect to $\mathcal{G}$, but $Y$ is not a random variable with respect to $\mathcal{G}$.
    \item [(B)] $Y$ is a random variable with respect to $\mathcal{G}$, but $X$ is not a random variable with respect to $\mathcal{G}$.
    \item [(C)] Neither $X$ nor $Y$ is a random variable with respect to $\mathcal{G}$.
    \item [(D)] Both $X$ and $Y$ are random variables with respect to $\mathcal{G}$.
\end{enumerate} \hfill (GATE ST 2023)\\
\solution
%\input{gate/ST/2023/14/main.tex}
	\item  A die is loaded in such a way that each odd number is twice as likely to occur as
each even number. Find $P(G)$, where $G$ is the event that a number greater than
3 occurs on a single roll of the die.
\\
\solution
		%\input{exemplar/11/16/3/5/main.tex}
	\item All the jacks, queens and kings are removed from a deck of 52 playing cards. The remaining cards are well shuffled and then one card is drawn at random. Giving ace a value 1 similar value for other cards, find the probability that the card has a value 
		\begin{enumerate}
			\item 7
			\item greater than 7
			\item less than 7
		\end{enumerate}
		%\input{exemplar/10/13/3/30/main.tex}
  \item A Lot consists of 48 mobile phones of which 42 are good, 3 have only minor defects and 3 have major defects.Varnika will buy a phone if it is good but the trader will only buy a mobile if it has no major defects. One phone is selected at random from the lot. What is the probability that it is
\begin{enumerate}
	\item acceptable to Varnika?
            \item acceptable to the trader?
\end{enumerate}
\solution
	%\input{exemplar/10/13/3/40/main.tex}
 \item A student says that if you throw a die, it will show up 1 or not 1. Therefore, the probability of getting 1 and the probability of getting 'not 1' each is equal to $\frac{1}{2}$. Is this correct? Give reasons.\\
 \solution
        %\input{exemplar/10/13/2/9/main.tex}
   \item Four candidates A, B, C, D have ap-
plied for the assignment to coach a school cricket
team. If A is twice as likely to be selected as B, and
B and C are given about the same chance of being
selected, while C is twice as likely to be selected
as D, what are the probabilities that
\begin{enumerate}
\item C will be selected?
\item A will not be selected?
\end{enumerate}
	%\input{exemplar/11/16/3/9/main.tex}
 \item A bag contain 24 balls of which $x$ balls are red, $2x$ are white and $3x$ are blue. A ball is selected at random, What is the probability that it is
\begin{enumerate}[label=\alph*)]
\item not red ?
\item white ?
\end{enumerate}
%\input{exemplar/10/13/3/41/main.tex}
If the letters of the word ASSASSINATION are arranged at random. Find the Probability that
\begin{enumerate}[label=(\alph*)]
\item Four $S's$ come consecutively in the word
\item Two  $I's$ and two $N's$ come together
\item All $A's$ are not coming together
\item No two $A's$ are coming together
\end{enumerate}
%\input{exemplar/11/16/3/14/main.tex}
	\item One urn contains two black balls (labelled B1 and B2) and one white ball. A
	second urn contains one black ball and two white balls (labelled W1 and W2).
	Suppose the following experiment is performed. One of the two urns is chosen
	at random. Next a ball is randomly chosen from the urn. Then a second ball is
	chosen at random from the same urn without replacing the first ball.
	
	\begin{enumerate}
	\item What is the probability that two black balls are chosen?
	
	\item What is the probability that two balls of opposite colour are chosen?
	\end{enumerate}
	\solution
	%\input{exemplar/11/16/3/12/main1.tex}
\end{enumerate}

	\item 
The number lock of a suitcase has 4 wheels each labelled with ten digits i.e. from 0 to 9.The lock opens with a sequence of four digits with no repeats.What is the probability of a person getting the right sequence to open the suitcase.
\\
\solution
		%\begin{enumerate}[label=\thesection.\arabic*,ref=\thesection.\theenumi]
	\item One card is drawn from a well-shuffled deck of 52 cards. Find the probability of getting
\begin{enumerate}
\item A king of red colour 
\item A face card 
\item A red face card
\item The jack of hearts
\item A spade
\item The queen of diamonds

\end{enumerate}
\solution
		%\input{ncert/10/15/1/14/main.tex}
	\item Five cards—the ten, jack, queen, king and ace of diamonds, are well-shuffled with their face downwards. One card is then picked up at random.
\begin{enumerate}
\item
What is the probability that the card is the queen? 
\item
If the queen is drawn and put aside, what is the probability that the second card picked up is (a) an ace? (b) a queen?\\
\end{enumerate}
\solution
		%\input{ncert/10/15/1/15/defs.tex}
	\item A bag contains $5$ red balls and some blue balls. If the probability of drawing a blue ball is double that if a red ball, determine the number of blue balls in the bag. 
		\\
\solution
		%\input{ncert/10/15/2/3/defs.tex}
	\item A card is selected from a pack of 52 cards.
 \begin{enumerate}[label=(\alph*)] 
                 \item How many points are there in the sample space?
                 \item Calculate the probability that the card is an ace of spades.
                 \item Calculate the probability that the card is (i) an ace and (ii) black card.
 \end{enumerate}
\solution
		%\input{ncert/11/16/3/4/main.tex}
\item Four cards are drawn from a well-shuffled deck of 52 cards. What is the probability of obtaining 3 diamonds and one spade.
\\
\solution
		%\input{ncert/11/16/4/2/defs.tex}
\item In a certain lottery 10,000 tickets are sold and ten equal prizes are awarded. What is the probability of not getting a prize if you buy (a) one ticket (b) two tickets (c) 10 tickets ?	
\\
\solution
		%\input{ncert/11/16/4/4/defs.tex}
		%
\item 
Out of 100 students, two sections of 40 and 60 are formed. If you and your friend are among the 100 students, what is the probability that
\begin{enumerate}
\item you both enter the same section?
\item you both enter the different sections?
\end{enumerate}
\solution
		%\input{ncert/11/16/4/5/defs.tex}
	\item 
The number lock of a suitcase has 4 wheels each labelled with ten digits i.e. from 0 to 9.The lock opens with a sequence of four digits with no repeats.What is the probability of a person getting the right sequence to open the suitcase.
\\
\solution
		%\input{ncert/11/16/4/10/defs.tex}
		%
\item 
Two cards are drawn at random and without replacement from a pack of 52 playing cards. Find the probability that both the cards are black.
\\
\solution
		%\input{ncert/12/13/2/2/defs.tex}
		\item A box of oranges is inspected by examining three randomly selected oranges drawn without replacement. If all the three oranges are good, the box is approved for sale, otherwise, it is rejected. Find the probability that a box containing 15 oranges out of which 12 are good and 3 are bad ones will be approved for sale.
		\label{ncert/12/13/2/3/defs.tex}
		\item Two balls are drawn at random with replacement from a box containing 10 black and 8 red balls. Find the probability that
		\label{ncert/12/13/2/12}
\begin{enumerate}
\item both balls are red.
\item first ball is black and second is red.
\item one of them is black and other is red.
\end{enumerate}

\item In a hostel, 60\% of the students read Hindi newspaper, 40\% read English newspaper and 20\% read both Hindi and English newspapers. A student is selected at random.
		\label{ncert/12/13/2/15}
\begin{enumerate}
\item Find the probability that she reads neither Hindi nor English newspapers.
\item If she reads Hindi newspaper, find the probability that she reads English newspaper.
\item If she reads English newspaper, find the probability that she reads Hindi newspaper.\\
\end{enumerate}
\item The probability of obtaining an even prime number on each die, when a pair of dice is rolled is 
\begin{enumerate}
    \item $0$ 
    
    \item $\frac{1}{3}$ 
    
    \item $\frac{1}{12}$ 
    
    \item $\frac{1}{36}$ 
\end{enumerate}
\solution
		%\input{ncert/12/13/2/17/defs.tex}
	\item A bag contains 4 red and 4 black balls, another bag contains 2 red and 6 black balls. One of the two bags is selected at random and a ball is drawn from the bag which is found to be red. Find the probability that the ball is drawn from the first bag.
\\
\solution
		%\input{ncert/12/13/3/2/main.tex}
  \item
  Cards with numbers 2 to 101 are placed in a box. A card is selected at random.Find the probability that the card has
\begin{enumerate}[label=(\roman*)]
	\item an even number 
	\item a square number
\end{enumerate}
\solution
%\input{exemplar/10/13/3/32/main.tex}
\item
The king, queen and jack of clubs are removed from a deck of 52 playing cards and then well shuffled. Now one card is drawn at random from the remaining cards.  Determine the probability that the card is
\begin{enumerate}[label=(\roman*)]
\item a club
\item 10 of hearts
\end{enumerate}
\solution
%\input{exemplar/10/13/3/29/main.tex}
\item A team of medical students doing their internship have to assist during surgeries
at a city hospital. The probabilities of surgeries rated as very complex, complex,
routine, simple or very simple are respectively, 0.15, 0.20, 0.31, 0.26, .08. Find
the probabilities that a particular surgery will be rated
\begin{enumerate}
	\item complex or very complex;
	\item neither very complex nor very simple;
	\item routine or complex
	\item routine or simple
\end{enumerate}
\solution
%\input{exemplar/11/16/3/8(1)/main.tex}
\item A card is selected from a pack of 52 cards.
\begin{enumerate}[label=(\alph*)]
    \item How many points are there in the sample space?
    \item Calculate the probability that the card is an ace of spades.
    \item Calculate the probability that the card is (i) an ace and (ii) black card.
\end{enumerate}
\solution
%\input{exemplar/11/16/3/4/main2.tex}
\item The probability that a non leap year selected at random will contain 53 sundays.
\\
\solution
%\input{exemplar/10/13/1/19/main.tex}
\item One of the four persons John, Rita, Aslam or Gurpreet will be promoted next
month. Consequently the sample space consists of four elementary outcomes
S = {John promoted, Rita promoted, Aslam promoted, Gurpreet promoted}
You are told that the chances of John’s promotion is same as that of Gurpreet,
Rita’s chances of promotion are twice as likely as Johns. Aslam’s chances are
four times that of John.
\begin{enumerate}
	\item Determine
	\begin{enumerate}
		\item P (John promoted)
		\item P (Rita promoted)
		\item P (Aslam promoted)
		\item P (Gurpreet promoted)
	\end{enumerate}
	\item If A = {John promoted or Gurpreet promoted}, find P (A).
\end{enumerate}
\solution
%\input{exemplar/11/16/3/10/main.tex}
\item A card is drawn from a deck of 52 cards. Find the probability of getting a king or a heart or a red card.\\
\solution
%\input{exemplar/11/16/3/15/main.tex}
\item The probability that a student will pass his examination is 0.73, the probability of
the student getting a compartment is 0.13, and the probability that the student will
either pass or get compartment is 0.96. State True or False.\\
\solution
%\input{exemplar/11/16/3/31/main.tex}
\item A card is selected from a pack of 52 cards\\
\begin{enumerate}[label=(\alph*)]
\item How many points are there in the sample space?
\item Calculate the probability that the cards is an ace of spades.
\item Calculate the probability that the card is (i) an ace (ii)black card.\\
\end{enumerate}
%\input{ncert/11/16/3/4_1/Prob_4.tex}
\item In a non-leap year, the probability of having 53 tuesdays or 53 wednesdays is\\
\solution
%\input{exemplar/11/16/3/18/main.tex}
\item There are 1000 sealed envelopes in a box, 10 of them contain a cash prize of
Rs 100 each, 100 of them contain a cash prize of Rs 50 each and 200 of them
contain a cash prize of Rs 10 each and rest do not contain any cash prize. If they
are well shuffled and an envelope is picked up out, what is the probability that it
contains no cash prize?\\
\solution
%\input{exemplar/10/13/3/34/main.tex}
\item 
A die is thrown and a card is selected at random from a deck of 52 playing cards. The probability of getting an even number on the die and a spade card.\\
\solution
%\input{exemplar/12/13/3/78/main.tex}
\item
If 4-digit numbers greater than 5,000 are randomly formed from the digits 0, 1, 3, 5, and 7, what is the probability of forming a number divisible by 5 when:
\begin{enumerate}
    \item The digits are repeated?
    \item The repetition of digits is not allowed?
\end{enumerate}
\solution
%\input{ncert/11/16/4/9/main.tex}
\item Consider the probability space $\brak{\Omega, \mathcal{G}, P}$ where $\Omega = [0,2]$ and $\mathcal{G} = \cbrak{\phi, \Omega, [0,1], (1,2]}$. Let $X$ and $Y$ be two functions on $\Omega$ defined as
\begin{align*}
    X(\omega) = 
    \begin{cases}
        1 & \text{if }\omega \in [0, 1]\\
        2 & \text{if }\omega \in (1, 2]
    \end{cases}
\end{align*}
and
\begin{align*}
    Y(\omega) = 
    \begin{cases}
        2 & \text{if }\omega \in [0, 1.5]\\
        3 & \text{if }\omega \in (1.5, 2].
    \end{cases}
\end{align*}
Then which one of the following statements is true?
\begin{enumerate}
    \item [(A)] $X$ is a random variable with respect to $\mathcal{G}$, but $Y$ is not a random variable with respect to $\mathcal{G}$.
    \item [(B)] $Y$ is a random variable with respect to $\mathcal{G}$, but $X$ is not a random variable with respect to $\mathcal{G}$.
    \item [(C)] Neither $X$ nor $Y$ is a random variable with respect to $\mathcal{G}$.
    \item [(D)] Both $X$ and $Y$ are random variables with respect to $\mathcal{G}$.
\end{enumerate} \hfill (GATE ST 2023)\\
\solution
%\input{gate/ST/2023/14/main.tex}
	\item  A die is loaded in such a way that each odd number is twice as likely to occur as
each even number. Find $P(G)$, where $G$ is the event that a number greater than
3 occurs on a single roll of the die.
\\
\solution
		%\input{exemplar/11/16/3/5/main.tex}
	\item All the jacks, queens and kings are removed from a deck of 52 playing cards. The remaining cards are well shuffled and then one card is drawn at random. Giving ace a value 1 similar value for other cards, find the probability that the card has a value 
		\begin{enumerate}
			\item 7
			\item greater than 7
			\item less than 7
		\end{enumerate}
		%\input{exemplar/10/13/3/30/main.tex}
  \item A Lot consists of 48 mobile phones of which 42 are good, 3 have only minor defects and 3 have major defects.Varnika will buy a phone if it is good but the trader will only buy a mobile if it has no major defects. One phone is selected at random from the lot. What is the probability that it is
\begin{enumerate}
	\item acceptable to Varnika?
            \item acceptable to the trader?
\end{enumerate}
\solution
	%\input{exemplar/10/13/3/40/main.tex}
 \item A student says that if you throw a die, it will show up 1 or not 1. Therefore, the probability of getting 1 and the probability of getting 'not 1' each is equal to $\frac{1}{2}$. Is this correct? Give reasons.\\
 \solution
        %\input{exemplar/10/13/2/9/main.tex}
   \item Four candidates A, B, C, D have ap-
plied for the assignment to coach a school cricket
team. If A is twice as likely to be selected as B, and
B and C are given about the same chance of being
selected, while C is twice as likely to be selected
as D, what are the probabilities that
\begin{enumerate}
\item C will be selected?
\item A will not be selected?
\end{enumerate}
	%\input{exemplar/11/16/3/9/main.tex}
 \item A bag contain 24 balls of which $x$ balls are red, $2x$ are white and $3x$ are blue. A ball is selected at random, What is the probability that it is
\begin{enumerate}[label=\alph*)]
\item not red ?
\item white ?
\end{enumerate}
%\input{exemplar/10/13/3/41/main.tex}
If the letters of the word ASSASSINATION are arranged at random. Find the Probability that
\begin{enumerate}[label=(\alph*)]
\item Four $S's$ come consecutively in the word
\item Two  $I's$ and two $N's$ come together
\item All $A's$ are not coming together
\item No two $A's$ are coming together
\end{enumerate}
%\input{exemplar/11/16/3/14/main.tex}
	\item One urn contains two black balls (labelled B1 and B2) and one white ball. A
	second urn contains one black ball and two white balls (labelled W1 and W2).
	Suppose the following experiment is performed. One of the two urns is chosen
	at random. Next a ball is randomly chosen from the urn. Then a second ball is
	chosen at random from the same urn without replacing the first ball.
	
	\begin{enumerate}
	\item What is the probability that two black balls are chosen?
	
	\item What is the probability that two balls of opposite colour are chosen?
	\end{enumerate}
	\solution
	%\input{exemplar/11/16/3/12/main1.tex}
\end{enumerate}

		%
\item 
Two cards are drawn at random and without replacement from a pack of 52 playing cards. Find the probability that both the cards are black.
\\
\solution
		%\begin{enumerate}[label=\thesection.\arabic*,ref=\thesection.\theenumi]
	\item One card is drawn from a well-shuffled deck of 52 cards. Find the probability of getting
\begin{enumerate}
\item A king of red colour 
\item A face card 
\item A red face card
\item The jack of hearts
\item A spade
\item The queen of diamonds

\end{enumerate}
\solution
		%\input{ncert/10/15/1/14/main.tex}
	\item Five cards—the ten, jack, queen, king and ace of diamonds, are well-shuffled with their face downwards. One card is then picked up at random.
\begin{enumerate}
\item
What is the probability that the card is the queen? 
\item
If the queen is drawn and put aside, what is the probability that the second card picked up is (a) an ace? (b) a queen?\\
\end{enumerate}
\solution
		%\input{ncert/10/15/1/15/defs.tex}
	\item A bag contains $5$ red balls and some blue balls. If the probability of drawing a blue ball is double that if a red ball, determine the number of blue balls in the bag. 
		\\
\solution
		%\input{ncert/10/15/2/3/defs.tex}
	\item A card is selected from a pack of 52 cards.
 \begin{enumerate}[label=(\alph*)] 
                 \item How many points are there in the sample space?
                 \item Calculate the probability that the card is an ace of spades.
                 \item Calculate the probability that the card is (i) an ace and (ii) black card.
 \end{enumerate}
\solution
		%\input{ncert/11/16/3/4/main.tex}
\item Four cards are drawn from a well-shuffled deck of 52 cards. What is the probability of obtaining 3 diamonds and one spade.
\\
\solution
		%\input{ncert/11/16/4/2/defs.tex}
\item In a certain lottery 10,000 tickets are sold and ten equal prizes are awarded. What is the probability of not getting a prize if you buy (a) one ticket (b) two tickets (c) 10 tickets ?	
\\
\solution
		%\input{ncert/11/16/4/4/defs.tex}
		%
\item 
Out of 100 students, two sections of 40 and 60 are formed. If you and your friend are among the 100 students, what is the probability that
\begin{enumerate}
\item you both enter the same section?
\item you both enter the different sections?
\end{enumerate}
\solution
		%\input{ncert/11/16/4/5/defs.tex}
	\item 
The number lock of a suitcase has 4 wheels each labelled with ten digits i.e. from 0 to 9.The lock opens with a sequence of four digits with no repeats.What is the probability of a person getting the right sequence to open the suitcase.
\\
\solution
		%\input{ncert/11/16/4/10/defs.tex}
		%
\item 
Two cards are drawn at random and without replacement from a pack of 52 playing cards. Find the probability that both the cards are black.
\\
\solution
		%\input{ncert/12/13/2/2/defs.tex}
		\item A box of oranges is inspected by examining three randomly selected oranges drawn without replacement. If all the three oranges are good, the box is approved for sale, otherwise, it is rejected. Find the probability that a box containing 15 oranges out of which 12 are good and 3 are bad ones will be approved for sale.
		\label{ncert/12/13/2/3/defs.tex}
		\item Two balls are drawn at random with replacement from a box containing 10 black and 8 red balls. Find the probability that
		\label{ncert/12/13/2/12}
\begin{enumerate}
\item both balls are red.
\item first ball is black and second is red.
\item one of them is black and other is red.
\end{enumerate}

\item In a hostel, 60\% of the students read Hindi newspaper, 40\% read English newspaper and 20\% read both Hindi and English newspapers. A student is selected at random.
		\label{ncert/12/13/2/15}
\begin{enumerate}
\item Find the probability that she reads neither Hindi nor English newspapers.
\item If she reads Hindi newspaper, find the probability that she reads English newspaper.
\item If she reads English newspaper, find the probability that she reads Hindi newspaper.\\
\end{enumerate}
\item The probability of obtaining an even prime number on each die, when a pair of dice is rolled is 
\begin{enumerate}
    \item $0$ 
    
    \item $\frac{1}{3}$ 
    
    \item $\frac{1}{12}$ 
    
    \item $\frac{1}{36}$ 
\end{enumerate}
\solution
		%\input{ncert/12/13/2/17/defs.tex}
	\item A bag contains 4 red and 4 black balls, another bag contains 2 red and 6 black balls. One of the two bags is selected at random and a ball is drawn from the bag which is found to be red. Find the probability that the ball is drawn from the first bag.
\\
\solution
		%\input{ncert/12/13/3/2/main.tex}
  \item
  Cards with numbers 2 to 101 are placed in a box. A card is selected at random.Find the probability that the card has
\begin{enumerate}[label=(\roman*)]
	\item an even number 
	\item a square number
\end{enumerate}
\solution
%\input{exemplar/10/13/3/32/main.tex}
\item
The king, queen and jack of clubs are removed from a deck of 52 playing cards and then well shuffled. Now one card is drawn at random from the remaining cards.  Determine the probability that the card is
\begin{enumerate}[label=(\roman*)]
\item a club
\item 10 of hearts
\end{enumerate}
\solution
%\input{exemplar/10/13/3/29/main.tex}
\item A team of medical students doing their internship have to assist during surgeries
at a city hospital. The probabilities of surgeries rated as very complex, complex,
routine, simple or very simple are respectively, 0.15, 0.20, 0.31, 0.26, .08. Find
the probabilities that a particular surgery will be rated
\begin{enumerate}
	\item complex or very complex;
	\item neither very complex nor very simple;
	\item routine or complex
	\item routine or simple
\end{enumerate}
\solution
%\input{exemplar/11/16/3/8(1)/main.tex}
\item A card is selected from a pack of 52 cards.
\begin{enumerate}[label=(\alph*)]
    \item How many points are there in the sample space?
    \item Calculate the probability that the card is an ace of spades.
    \item Calculate the probability that the card is (i) an ace and (ii) black card.
\end{enumerate}
\solution
%\input{exemplar/11/16/3/4/main2.tex}
\item The probability that a non leap year selected at random will contain 53 sundays.
\\
\solution
%\input{exemplar/10/13/1/19/main.tex}
\item One of the four persons John, Rita, Aslam or Gurpreet will be promoted next
month. Consequently the sample space consists of four elementary outcomes
S = {John promoted, Rita promoted, Aslam promoted, Gurpreet promoted}
You are told that the chances of John’s promotion is same as that of Gurpreet,
Rita’s chances of promotion are twice as likely as Johns. Aslam’s chances are
four times that of John.
\begin{enumerate}
	\item Determine
	\begin{enumerate}
		\item P (John promoted)
		\item P (Rita promoted)
		\item P (Aslam promoted)
		\item P (Gurpreet promoted)
	\end{enumerate}
	\item If A = {John promoted or Gurpreet promoted}, find P (A).
\end{enumerate}
\solution
%\input{exemplar/11/16/3/10/main.tex}
\item A card is drawn from a deck of 52 cards. Find the probability of getting a king or a heart or a red card.\\
\solution
%\input{exemplar/11/16/3/15/main.tex}
\item The probability that a student will pass his examination is 0.73, the probability of
the student getting a compartment is 0.13, and the probability that the student will
either pass or get compartment is 0.96. State True or False.\\
\solution
%\input{exemplar/11/16/3/31/main.tex}
\item A card is selected from a pack of 52 cards\\
\begin{enumerate}[label=(\alph*)]
\item How many points are there in the sample space?
\item Calculate the probability that the cards is an ace of spades.
\item Calculate the probability that the card is (i) an ace (ii)black card.\\
\end{enumerate}
%\input{ncert/11/16/3/4_1/Prob_4.tex}
\item In a non-leap year, the probability of having 53 tuesdays or 53 wednesdays is\\
\solution
%\input{exemplar/11/16/3/18/main.tex}
\item There are 1000 sealed envelopes in a box, 10 of them contain a cash prize of
Rs 100 each, 100 of them contain a cash prize of Rs 50 each and 200 of them
contain a cash prize of Rs 10 each and rest do not contain any cash prize. If they
are well shuffled and an envelope is picked up out, what is the probability that it
contains no cash prize?\\
\solution
%\input{exemplar/10/13/3/34/main.tex}
\item 
A die is thrown and a card is selected at random from a deck of 52 playing cards. The probability of getting an even number on the die and a spade card.\\
\solution
%\input{exemplar/12/13/3/78/main.tex}
\item
If 4-digit numbers greater than 5,000 are randomly formed from the digits 0, 1, 3, 5, and 7, what is the probability of forming a number divisible by 5 when:
\begin{enumerate}
    \item The digits are repeated?
    \item The repetition of digits is not allowed?
\end{enumerate}
\solution
%\input{ncert/11/16/4/9/main.tex}
\item Consider the probability space $\brak{\Omega, \mathcal{G}, P}$ where $\Omega = [0,2]$ and $\mathcal{G} = \cbrak{\phi, \Omega, [0,1], (1,2]}$. Let $X$ and $Y$ be two functions on $\Omega$ defined as
\begin{align*}
    X(\omega) = 
    \begin{cases}
        1 & \text{if }\omega \in [0, 1]\\
        2 & \text{if }\omega \in (1, 2]
    \end{cases}
\end{align*}
and
\begin{align*}
    Y(\omega) = 
    \begin{cases}
        2 & \text{if }\omega \in [0, 1.5]\\
        3 & \text{if }\omega \in (1.5, 2].
    \end{cases}
\end{align*}
Then which one of the following statements is true?
\begin{enumerate}
    \item [(A)] $X$ is a random variable with respect to $\mathcal{G}$, but $Y$ is not a random variable with respect to $\mathcal{G}$.
    \item [(B)] $Y$ is a random variable with respect to $\mathcal{G}$, but $X$ is not a random variable with respect to $\mathcal{G}$.
    \item [(C)] Neither $X$ nor $Y$ is a random variable with respect to $\mathcal{G}$.
    \item [(D)] Both $X$ and $Y$ are random variables with respect to $\mathcal{G}$.
\end{enumerate} \hfill (GATE ST 2023)\\
\solution
%\input{gate/ST/2023/14/main.tex}
	\item  A die is loaded in such a way that each odd number is twice as likely to occur as
each even number. Find $P(G)$, where $G$ is the event that a number greater than
3 occurs on a single roll of the die.
\\
\solution
		%\input{exemplar/11/16/3/5/main.tex}
	\item All the jacks, queens and kings are removed from a deck of 52 playing cards. The remaining cards are well shuffled and then one card is drawn at random. Giving ace a value 1 similar value for other cards, find the probability that the card has a value 
		\begin{enumerate}
			\item 7
			\item greater than 7
			\item less than 7
		\end{enumerate}
		%\input{exemplar/10/13/3/30/main.tex}
  \item A Lot consists of 48 mobile phones of which 42 are good, 3 have only minor defects and 3 have major defects.Varnika will buy a phone if it is good but the trader will only buy a mobile if it has no major defects. One phone is selected at random from the lot. What is the probability that it is
\begin{enumerate}
	\item acceptable to Varnika?
            \item acceptable to the trader?
\end{enumerate}
\solution
	%\input{exemplar/10/13/3/40/main.tex}
 \item A student says that if you throw a die, it will show up 1 or not 1. Therefore, the probability of getting 1 and the probability of getting 'not 1' each is equal to $\frac{1}{2}$. Is this correct? Give reasons.\\
 \solution
        %\input{exemplar/10/13/2/9/main.tex}
   \item Four candidates A, B, C, D have ap-
plied for the assignment to coach a school cricket
team. If A is twice as likely to be selected as B, and
B and C are given about the same chance of being
selected, while C is twice as likely to be selected
as D, what are the probabilities that
\begin{enumerate}
\item C will be selected?
\item A will not be selected?
\end{enumerate}
	%\input{exemplar/11/16/3/9/main.tex}
 \item A bag contain 24 balls of which $x$ balls are red, $2x$ are white and $3x$ are blue. A ball is selected at random, What is the probability that it is
\begin{enumerate}[label=\alph*)]
\item not red ?
\item white ?
\end{enumerate}
%\input{exemplar/10/13/3/41/main.tex}
If the letters of the word ASSASSINATION are arranged at random. Find the Probability that
\begin{enumerate}[label=(\alph*)]
\item Four $S's$ come consecutively in the word
\item Two  $I's$ and two $N's$ come together
\item All $A's$ are not coming together
\item No two $A's$ are coming together
\end{enumerate}
%\input{exemplar/11/16/3/14/main.tex}
	\item One urn contains two black balls (labelled B1 and B2) and one white ball. A
	second urn contains one black ball and two white balls (labelled W1 and W2).
	Suppose the following experiment is performed. One of the two urns is chosen
	at random. Next a ball is randomly chosen from the urn. Then a second ball is
	chosen at random from the same urn without replacing the first ball.
	
	\begin{enumerate}
	\item What is the probability that two black balls are chosen?
	
	\item What is the probability that two balls of opposite colour are chosen?
	\end{enumerate}
	\solution
	%\input{exemplar/11/16/3/12/main1.tex}
\end{enumerate}

		\item A box of oranges is inspected by examining three randomly selected oranges drawn without replacement. If all the three oranges are good, the box is approved for sale, otherwise, it is rejected. Find the probability that a box containing 15 oranges out of which 12 are good and 3 are bad ones will be approved for sale.
		\label{ncert/12/13/2/3/defs.tex}
		\item Two balls are drawn at random with replacement from a box containing 10 black and 8 red balls. Find the probability that
		\label{ncert/12/13/2/12}
\begin{enumerate}
\item both balls are red.
\item first ball is black and second is red.
\item one of them is black and other is red.
\end{enumerate}

\item In a hostel, 60\% of the students read Hindi newspaper, 40\% read English newspaper and 20\% read both Hindi and English newspapers. A student is selected at random.
		\label{ncert/12/13/2/15}
\begin{enumerate}
\item Find the probability that she reads neither Hindi nor English newspapers.
\item If she reads Hindi newspaper, find the probability that she reads English newspaper.
\item If she reads English newspaper, find the probability that she reads Hindi newspaper.\\
\end{enumerate}
\item The probability of obtaining an even prime number on each die, when a pair of dice is rolled is 
\begin{enumerate}
    \item $0$ 
    
    \item $\frac{1}{3}$ 
    
    \item $\frac{1}{12}$ 
    
    \item $\frac{1}{36}$ 
\end{enumerate}
\solution
		%\begin{enumerate}[label=\thesection.\arabic*,ref=\thesection.\theenumi]
	\item One card is drawn from a well-shuffled deck of 52 cards. Find the probability of getting
\begin{enumerate}
\item A king of red colour 
\item A face card 
\item A red face card
\item The jack of hearts
\item A spade
\item The queen of diamonds

\end{enumerate}
\solution
		%\input{ncert/10/15/1/14/main.tex}
	\item Five cards—the ten, jack, queen, king and ace of diamonds, are well-shuffled with their face downwards. One card is then picked up at random.
\begin{enumerate}
\item
What is the probability that the card is the queen? 
\item
If the queen is drawn and put aside, what is the probability that the second card picked up is (a) an ace? (b) a queen?\\
\end{enumerate}
\solution
		%\input{ncert/10/15/1/15/defs.tex}
	\item A bag contains $5$ red balls and some blue balls. If the probability of drawing a blue ball is double that if a red ball, determine the number of blue balls in the bag. 
		\\
\solution
		%\input{ncert/10/15/2/3/defs.tex}
	\item A card is selected from a pack of 52 cards.
 \begin{enumerate}[label=(\alph*)] 
                 \item How many points are there in the sample space?
                 \item Calculate the probability that the card is an ace of spades.
                 \item Calculate the probability that the card is (i) an ace and (ii) black card.
 \end{enumerate}
\solution
		%\input{ncert/11/16/3/4/main.tex}
\item Four cards are drawn from a well-shuffled deck of 52 cards. What is the probability of obtaining 3 diamonds and one spade.
\\
\solution
		%\input{ncert/11/16/4/2/defs.tex}
\item In a certain lottery 10,000 tickets are sold and ten equal prizes are awarded. What is the probability of not getting a prize if you buy (a) one ticket (b) two tickets (c) 10 tickets ?	
\\
\solution
		%\input{ncert/11/16/4/4/defs.tex}
		%
\item 
Out of 100 students, two sections of 40 and 60 are formed. If you and your friend are among the 100 students, what is the probability that
\begin{enumerate}
\item you both enter the same section?
\item you both enter the different sections?
\end{enumerate}
\solution
		%\input{ncert/11/16/4/5/defs.tex}
	\item 
The number lock of a suitcase has 4 wheels each labelled with ten digits i.e. from 0 to 9.The lock opens with a sequence of four digits with no repeats.What is the probability of a person getting the right sequence to open the suitcase.
\\
\solution
		%\input{ncert/11/16/4/10/defs.tex}
		%
\item 
Two cards are drawn at random and without replacement from a pack of 52 playing cards. Find the probability that both the cards are black.
\\
\solution
		%\input{ncert/12/13/2/2/defs.tex}
		\item A box of oranges is inspected by examining three randomly selected oranges drawn without replacement. If all the three oranges are good, the box is approved for sale, otherwise, it is rejected. Find the probability that a box containing 15 oranges out of which 12 are good and 3 are bad ones will be approved for sale.
		\label{ncert/12/13/2/3/defs.tex}
		\item Two balls are drawn at random with replacement from a box containing 10 black and 8 red balls. Find the probability that
		\label{ncert/12/13/2/12}
\begin{enumerate}
\item both balls are red.
\item first ball is black and second is red.
\item one of them is black and other is red.
\end{enumerate}

\item In a hostel, 60\% of the students read Hindi newspaper, 40\% read English newspaper and 20\% read both Hindi and English newspapers. A student is selected at random.
		\label{ncert/12/13/2/15}
\begin{enumerate}
\item Find the probability that she reads neither Hindi nor English newspapers.
\item If she reads Hindi newspaper, find the probability that she reads English newspaper.
\item If she reads English newspaper, find the probability that she reads Hindi newspaper.\\
\end{enumerate}
\item The probability of obtaining an even prime number on each die, when a pair of dice is rolled is 
\begin{enumerate}
    \item $0$ 
    
    \item $\frac{1}{3}$ 
    
    \item $\frac{1}{12}$ 
    
    \item $\frac{1}{36}$ 
\end{enumerate}
\solution
		%\input{ncert/12/13/2/17/defs.tex}
	\item A bag contains 4 red and 4 black balls, another bag contains 2 red and 6 black balls. One of the two bags is selected at random and a ball is drawn from the bag which is found to be red. Find the probability that the ball is drawn from the first bag.
\\
\solution
		%\input{ncert/12/13/3/2/main.tex}
  \item
  Cards with numbers 2 to 101 are placed in a box. A card is selected at random.Find the probability that the card has
\begin{enumerate}[label=(\roman*)]
	\item an even number 
	\item a square number
\end{enumerate}
\solution
%\input{exemplar/10/13/3/32/main.tex}
\item
The king, queen and jack of clubs are removed from a deck of 52 playing cards and then well shuffled. Now one card is drawn at random from the remaining cards.  Determine the probability that the card is
\begin{enumerate}[label=(\roman*)]
\item a club
\item 10 of hearts
\end{enumerate}
\solution
%\input{exemplar/10/13/3/29/main.tex}
\item A team of medical students doing their internship have to assist during surgeries
at a city hospital. The probabilities of surgeries rated as very complex, complex,
routine, simple or very simple are respectively, 0.15, 0.20, 0.31, 0.26, .08. Find
the probabilities that a particular surgery will be rated
\begin{enumerate}
	\item complex or very complex;
	\item neither very complex nor very simple;
	\item routine or complex
	\item routine or simple
\end{enumerate}
\solution
%\input{exemplar/11/16/3/8(1)/main.tex}
\item A card is selected from a pack of 52 cards.
\begin{enumerate}[label=(\alph*)]
    \item How many points are there in the sample space?
    \item Calculate the probability that the card is an ace of spades.
    \item Calculate the probability that the card is (i) an ace and (ii) black card.
\end{enumerate}
\solution
%\input{exemplar/11/16/3/4/main2.tex}
\item The probability that a non leap year selected at random will contain 53 sundays.
\\
\solution
%\input{exemplar/10/13/1/19/main.tex}
\item One of the four persons John, Rita, Aslam or Gurpreet will be promoted next
month. Consequently the sample space consists of four elementary outcomes
S = {John promoted, Rita promoted, Aslam promoted, Gurpreet promoted}
You are told that the chances of John’s promotion is same as that of Gurpreet,
Rita’s chances of promotion are twice as likely as Johns. Aslam’s chances are
four times that of John.
\begin{enumerate}
	\item Determine
	\begin{enumerate}
		\item P (John promoted)
		\item P (Rita promoted)
		\item P (Aslam promoted)
		\item P (Gurpreet promoted)
	\end{enumerate}
	\item If A = {John promoted or Gurpreet promoted}, find P (A).
\end{enumerate}
\solution
%\input{exemplar/11/16/3/10/main.tex}
\item A card is drawn from a deck of 52 cards. Find the probability of getting a king or a heart or a red card.\\
\solution
%\input{exemplar/11/16/3/15/main.tex}
\item The probability that a student will pass his examination is 0.73, the probability of
the student getting a compartment is 0.13, and the probability that the student will
either pass or get compartment is 0.96. State True or False.\\
\solution
%\input{exemplar/11/16/3/31/main.tex}
\item A card is selected from a pack of 52 cards\\
\begin{enumerate}[label=(\alph*)]
\item How many points are there in the sample space?
\item Calculate the probability that the cards is an ace of spades.
\item Calculate the probability that the card is (i) an ace (ii)black card.\\
\end{enumerate}
%\input{ncert/11/16/3/4_1/Prob_4.tex}
\item In a non-leap year, the probability of having 53 tuesdays or 53 wednesdays is\\
\solution
%\input{exemplar/11/16/3/18/main.tex}
\item There are 1000 sealed envelopes in a box, 10 of them contain a cash prize of
Rs 100 each, 100 of them contain a cash prize of Rs 50 each and 200 of them
contain a cash prize of Rs 10 each and rest do not contain any cash prize. If they
are well shuffled and an envelope is picked up out, what is the probability that it
contains no cash prize?\\
\solution
%\input{exemplar/10/13/3/34/main.tex}
\item 
A die is thrown and a card is selected at random from a deck of 52 playing cards. The probability of getting an even number on the die and a spade card.\\
\solution
%\input{exemplar/12/13/3/78/main.tex}
\item
If 4-digit numbers greater than 5,000 are randomly formed from the digits 0, 1, 3, 5, and 7, what is the probability of forming a number divisible by 5 when:
\begin{enumerate}
    \item The digits are repeated?
    \item The repetition of digits is not allowed?
\end{enumerate}
\solution
%\input{ncert/11/16/4/9/main.tex}
\item Consider the probability space $\brak{\Omega, \mathcal{G}, P}$ where $\Omega = [0,2]$ and $\mathcal{G} = \cbrak{\phi, \Omega, [0,1], (1,2]}$. Let $X$ and $Y$ be two functions on $\Omega$ defined as
\begin{align*}
    X(\omega) = 
    \begin{cases}
        1 & \text{if }\omega \in [0, 1]\\
        2 & \text{if }\omega \in (1, 2]
    \end{cases}
\end{align*}
and
\begin{align*}
    Y(\omega) = 
    \begin{cases}
        2 & \text{if }\omega \in [0, 1.5]\\
        3 & \text{if }\omega \in (1.5, 2].
    \end{cases}
\end{align*}
Then which one of the following statements is true?
\begin{enumerate}
    \item [(A)] $X$ is a random variable with respect to $\mathcal{G}$, but $Y$ is not a random variable with respect to $\mathcal{G}$.
    \item [(B)] $Y$ is a random variable with respect to $\mathcal{G}$, but $X$ is not a random variable with respect to $\mathcal{G}$.
    \item [(C)] Neither $X$ nor $Y$ is a random variable with respect to $\mathcal{G}$.
    \item [(D)] Both $X$ and $Y$ are random variables with respect to $\mathcal{G}$.
\end{enumerate} \hfill (GATE ST 2023)\\
\solution
%\input{gate/ST/2023/14/main.tex}
	\item  A die is loaded in such a way that each odd number is twice as likely to occur as
each even number. Find $P(G)$, where $G$ is the event that a number greater than
3 occurs on a single roll of the die.
\\
\solution
		%\input{exemplar/11/16/3/5/main.tex}
	\item All the jacks, queens and kings are removed from a deck of 52 playing cards. The remaining cards are well shuffled and then one card is drawn at random. Giving ace a value 1 similar value for other cards, find the probability that the card has a value 
		\begin{enumerate}
			\item 7
			\item greater than 7
			\item less than 7
		\end{enumerate}
		%\input{exemplar/10/13/3/30/main.tex}
  \item A Lot consists of 48 mobile phones of which 42 are good, 3 have only minor defects and 3 have major defects.Varnika will buy a phone if it is good but the trader will only buy a mobile if it has no major defects. One phone is selected at random from the lot. What is the probability that it is
\begin{enumerate}
	\item acceptable to Varnika?
            \item acceptable to the trader?
\end{enumerate}
\solution
	%\input{exemplar/10/13/3/40/main.tex}
 \item A student says that if you throw a die, it will show up 1 or not 1. Therefore, the probability of getting 1 and the probability of getting 'not 1' each is equal to $\frac{1}{2}$. Is this correct? Give reasons.\\
 \solution
        %\input{exemplar/10/13/2/9/main.tex}
   \item Four candidates A, B, C, D have ap-
plied for the assignment to coach a school cricket
team. If A is twice as likely to be selected as B, and
B and C are given about the same chance of being
selected, while C is twice as likely to be selected
as D, what are the probabilities that
\begin{enumerate}
\item C will be selected?
\item A will not be selected?
\end{enumerate}
	%\input{exemplar/11/16/3/9/main.tex}
 \item A bag contain 24 balls of which $x$ balls are red, $2x$ are white and $3x$ are blue. A ball is selected at random, What is the probability that it is
\begin{enumerate}[label=\alph*)]
\item not red ?
\item white ?
\end{enumerate}
%\input{exemplar/10/13/3/41/main.tex}
If the letters of the word ASSASSINATION are arranged at random. Find the Probability that
\begin{enumerate}[label=(\alph*)]
\item Four $S's$ come consecutively in the word
\item Two  $I's$ and two $N's$ come together
\item All $A's$ are not coming together
\item No two $A's$ are coming together
\end{enumerate}
%\input{exemplar/11/16/3/14/main.tex}
	\item One urn contains two black balls (labelled B1 and B2) and one white ball. A
	second urn contains one black ball and two white balls (labelled W1 and W2).
	Suppose the following experiment is performed. One of the two urns is chosen
	at random. Next a ball is randomly chosen from the urn. Then a second ball is
	chosen at random from the same urn without replacing the first ball.
	
	\begin{enumerate}
	\item What is the probability that two black balls are chosen?
	
	\item What is the probability that two balls of opposite colour are chosen?
	\end{enumerate}
	\solution
	%\input{exemplar/11/16/3/12/main1.tex}
\end{enumerate}

	\item A bag contains 4 red and 4 black balls, another bag contains 2 red and 6 black balls. One of the two bags is selected at random and a ball is drawn from the bag which is found to be red. Find the probability that the ball is drawn from the first bag.
\\
\solution
		%\begin{table}[H]
	\centering
\begin{tabular}{|c|c|c|}
\hline
Random variable &Value &Definition\\ \hline
\multirow{3}{*}{X} &0 &Slips of Rs 1\\
&1 &Slips of Rs 5\\
&2 &Slips of Rs 13\\ \hline
\multirow{2}{*}{Y} &0 &Box A\\
&1 &Box B\\\hline
\end{tabular}
\caption{}
\label{tab:Distribution}
\end{table}
See \tabref{tab:Distribution}.
\begin{align}
p_{Y}\brak{k}= \begin{cases} 
      \frac{1}{3} & {k=0} \\
      \frac{2}{3 }& {k=1} 
   \end{cases}
   \\
p_{Y|X}\brak{0|0} = \frac{19}{25}\, 
p_{Y|X}\brak{0|1} = \frac{6}{25}\,
p_{Y|X}\brak{1|0} = \frac{45}{50}\,
p_{Y|X}\brak{1|2} = \frac{5}{50}
\end{align}
The desired probability is the probability that a slip drawn at random is marked other than Rs 1,
\begin{align}
&=1-p_X\brak{0}\\
&= p_X(1) + p_X(2)
\end{align}
Using Bayes theorem,
\begin{align}
&= p_Y\brak{0} \times \pr{Y=0 | X=1} + p_Y\brak{1} \times \pr{Y=1|X=2}\\
&=\frac{1}{3} \times \frac{6}{25} + \frac{2}{3} \times \frac{5}{50}\\
&=\frac{11}{75}
\end{align}

\newpage

%\tableofcontents

\bigskip

\renewcommand{\thefigure}{\theenumi}
\renewcommand{\thetable}{\theenumi}
%\renewcommand{\theequation}{\theenumi}

%\begin{abstract}
%%\boldmath
%In this letter, an algorithm for evaluating the exact analytical bit error rate  (BER)  for the piecewise linear (PL) combiner for  multiple relays is presented. Previous results were available only for upto three relays. The algorithm is unique in the sense that  the actual mathematical expressions, that are prohibitively large, need not be explicitly obtained. The diversity gain due to multiple relays is shown through plots of the analytical BER, well supported by simulations. 
%
%\end{abstract}
% IEEEtran.cls defaults to using nonbold math in the Abstract.
% This preserves the distinction between vectors and scalars. However,
% if the journal you are submitting to favors bold math in the abstract,
% then you can use LaTeX's standard command \boldmath at the very start
% of the abstract to achieve this. Many IEEE journals frown on math
% in the abstract anyway.

% Note that keywords are not normally used for peerreview papers.
%\begin{IEEEkeywords}
%Cooperative diversity, decode and forward, piecewise linear
%\end{IEEEkeywords}



% For peer review papers, you can put extra information on the cover
% page as needed:
% \ifCLASSOPTIONpeerreview
% \begin{center} \bfseries EDICS Category: 3-BBND \end{center}
% \fi
%
% For peerreview papers, this IEEEtran command inserts a page break and
% creates the second title. It will be ignored for other modes.
%\IEEEpeerreviewmaketitle




  \item
  Cards with numbers 2 to 101 are placed in a box. A card is selected at random.Find the probability that the card has
\begin{enumerate}[label=(\roman*)]
	\item an even number 
	\item a square number
\end{enumerate}
\solution
%\begin{table}[H]
	\centering
\begin{tabular}{|c|c|c|}
\hline
Random variable &Value &Definition\\ \hline
\multirow{3}{*}{X} &0 &Slips of Rs 1\\
&1 &Slips of Rs 5\\
&2 &Slips of Rs 13\\ \hline
\multirow{2}{*}{Y} &0 &Box A\\
&1 &Box B\\\hline
\end{tabular}
\caption{}
\label{tab:Distribution}
\end{table}
See \tabref{tab:Distribution}.
\begin{align}
p_{Y}\brak{k}= \begin{cases} 
      \frac{1}{3} & {k=0} \\
      \frac{2}{3 }& {k=1} 
   \end{cases}
   \\
p_{Y|X}\brak{0|0} = \frac{19}{25}\, 
p_{Y|X}\brak{0|1} = \frac{6}{25}\,
p_{Y|X}\brak{1|0} = \frac{45}{50}\,
p_{Y|X}\brak{1|2} = \frac{5}{50}
\end{align}
The desired probability is the probability that a slip drawn at random is marked other than Rs 1,
\begin{align}
&=1-p_X\brak{0}\\
&= p_X(1) + p_X(2)
\end{align}
Using Bayes theorem,
\begin{align}
&= p_Y\brak{0} \times \pr{Y=0 | X=1} + p_Y\brak{1} \times \pr{Y=1|X=2}\\
&=\frac{1}{3} \times \frac{6}{25} + \frac{2}{3} \times \frac{5}{50}\\
&=\frac{11}{75}
\end{align}

\newpage

%\tableofcontents

\bigskip

\renewcommand{\thefigure}{\theenumi}
\renewcommand{\thetable}{\theenumi}
%\renewcommand{\theequation}{\theenumi}

%\begin{abstract}
%%\boldmath
%In this letter, an algorithm for evaluating the exact analytical bit error rate  (BER)  for the piecewise linear (PL) combiner for  multiple relays is presented. Previous results were available only for upto three relays. The algorithm is unique in the sense that  the actual mathematical expressions, that are prohibitively large, need not be explicitly obtained. The diversity gain due to multiple relays is shown through plots of the analytical BER, well supported by simulations. 
%
%\end{abstract}
% IEEEtran.cls defaults to using nonbold math in the Abstract.
% This preserves the distinction between vectors and scalars. However,
% if the journal you are submitting to favors bold math in the abstract,
% then you can use LaTeX's standard command \boldmath at the very start
% of the abstract to achieve this. Many IEEE journals frown on math
% in the abstract anyway.

% Note that keywords are not normally used for peerreview papers.
%\begin{IEEEkeywords}
%Cooperative diversity, decode and forward, piecewise linear
%\end{IEEEkeywords}



% For peer review papers, you can put extra information on the cover
% page as needed:
% \ifCLASSOPTIONpeerreview
% \begin{center} \bfseries EDICS Category: 3-BBND \end{center}
% \fi
%
% For peerreview papers, this IEEEtran command inserts a page break and
% creates the second title. It will be ignored for other modes.
%\IEEEpeerreviewmaketitle




\item
The king, queen and jack of clubs are removed from a deck of 52 playing cards and then well shuffled. Now one card is drawn at random from the remaining cards.  Determine the probability that the card is
\begin{enumerate}[label=(\roman*)]
\item a club
\item 10 of hearts
\end{enumerate}
\solution
%\begin{table}[H]
	\centering
\begin{tabular}{|c|c|c|}
\hline
Random variable &Value &Definition\\ \hline
\multirow{3}{*}{X} &0 &Slips of Rs 1\\
&1 &Slips of Rs 5\\
&2 &Slips of Rs 13\\ \hline
\multirow{2}{*}{Y} &0 &Box A\\
&1 &Box B\\\hline
\end{tabular}
\caption{}
\label{tab:Distribution}
\end{table}
See \tabref{tab:Distribution}.
\begin{align}
p_{Y}\brak{k}= \begin{cases} 
      \frac{1}{3} & {k=0} \\
      \frac{2}{3 }& {k=1} 
   \end{cases}
   \\
p_{Y|X}\brak{0|0} = \frac{19}{25}\, 
p_{Y|X}\brak{0|1} = \frac{6}{25}\,
p_{Y|X}\brak{1|0} = \frac{45}{50}\,
p_{Y|X}\brak{1|2} = \frac{5}{50}
\end{align}
The desired probability is the probability that a slip drawn at random is marked other than Rs 1,
\begin{align}
&=1-p_X\brak{0}\\
&= p_X(1) + p_X(2)
\end{align}
Using Bayes theorem,
\begin{align}
&= p_Y\brak{0} \times \pr{Y=0 | X=1} + p_Y\brak{1} \times \pr{Y=1|X=2}\\
&=\frac{1}{3} \times \frac{6}{25} + \frac{2}{3} \times \frac{5}{50}\\
&=\frac{11}{75}
\end{align}

\newpage

%\tableofcontents

\bigskip

\renewcommand{\thefigure}{\theenumi}
\renewcommand{\thetable}{\theenumi}
%\renewcommand{\theequation}{\theenumi}

%\begin{abstract}
%%\boldmath
%In this letter, an algorithm for evaluating the exact analytical bit error rate  (BER)  for the piecewise linear (PL) combiner for  multiple relays is presented. Previous results were available only for upto three relays. The algorithm is unique in the sense that  the actual mathematical expressions, that are prohibitively large, need not be explicitly obtained. The diversity gain due to multiple relays is shown through plots of the analytical BER, well supported by simulations. 
%
%\end{abstract}
% IEEEtran.cls defaults to using nonbold math in the Abstract.
% This preserves the distinction between vectors and scalars. However,
% if the journal you are submitting to favors bold math in the abstract,
% then you can use LaTeX's standard command \boldmath at the very start
% of the abstract to achieve this. Many IEEE journals frown on math
% in the abstract anyway.

% Note that keywords are not normally used for peerreview papers.
%\begin{IEEEkeywords}
%Cooperative diversity, decode and forward, piecewise linear
%\end{IEEEkeywords}



% For peer review papers, you can put extra information on the cover
% page as needed:
% \ifCLASSOPTIONpeerreview
% \begin{center} \bfseries EDICS Category: 3-BBND \end{center}
% \fi
%
% For peerreview papers, this IEEEtran command inserts a page break and
% creates the second title. It will be ignored for other modes.
%\IEEEpeerreviewmaketitle




\item A team of medical students doing their internship have to assist during surgeries
at a city hospital. The probabilities of surgeries rated as very complex, complex,
routine, simple or very simple are respectively, 0.15, 0.20, 0.31, 0.26, .08. Find
the probabilities that a particular surgery will be rated
\begin{enumerate}
	\item complex or very complex;
	\item neither very complex nor very simple;
	\item routine or complex
	\item routine or simple
\end{enumerate}
\solution
%\begin{table}[H]
	\centering
\begin{tabular}{|c|c|c|}
\hline
Random variable &Value &Definition\\ \hline
\multirow{3}{*}{X} &0 &Slips of Rs 1\\
&1 &Slips of Rs 5\\
&2 &Slips of Rs 13\\ \hline
\multirow{2}{*}{Y} &0 &Box A\\
&1 &Box B\\\hline
\end{tabular}
\caption{}
\label{tab:Distribution}
\end{table}
See \tabref{tab:Distribution}.
\begin{align}
p_{Y}\brak{k}= \begin{cases} 
      \frac{1}{3} & {k=0} \\
      \frac{2}{3 }& {k=1} 
   \end{cases}
   \\
p_{Y|X}\brak{0|0} = \frac{19}{25}\, 
p_{Y|X}\brak{0|1} = \frac{6}{25}\,
p_{Y|X}\brak{1|0} = \frac{45}{50}\,
p_{Y|X}\brak{1|2} = \frac{5}{50}
\end{align}
The desired probability is the probability that a slip drawn at random is marked other than Rs 1,
\begin{align}
&=1-p_X\brak{0}\\
&= p_X(1) + p_X(2)
\end{align}
Using Bayes theorem,
\begin{align}
&= p_Y\brak{0} \times \pr{Y=0 | X=1} + p_Y\brak{1} \times \pr{Y=1|X=2}\\
&=\frac{1}{3} \times \frac{6}{25} + \frac{2}{3} \times \frac{5}{50}\\
&=\frac{11}{75}
\end{align}

\newpage

%\tableofcontents

\bigskip

\renewcommand{\thefigure}{\theenumi}
\renewcommand{\thetable}{\theenumi}
%\renewcommand{\theequation}{\theenumi}

%\begin{abstract}
%%\boldmath
%In this letter, an algorithm for evaluating the exact analytical bit error rate  (BER)  for the piecewise linear (PL) combiner for  multiple relays is presented. Previous results were available only for upto three relays. The algorithm is unique in the sense that  the actual mathematical expressions, that are prohibitively large, need not be explicitly obtained. The diversity gain due to multiple relays is shown through plots of the analytical BER, well supported by simulations. 
%
%\end{abstract}
% IEEEtran.cls defaults to using nonbold math in the Abstract.
% This preserves the distinction between vectors and scalars. However,
% if the journal you are submitting to favors bold math in the abstract,
% then you can use LaTeX's standard command \boldmath at the very start
% of the abstract to achieve this. Many IEEE journals frown on math
% in the abstract anyway.

% Note that keywords are not normally used for peerreview papers.
%\begin{IEEEkeywords}
%Cooperative diversity, decode and forward, piecewise linear
%\end{IEEEkeywords}



% For peer review papers, you can put extra information on the cover
% page as needed:
% \ifCLASSOPTIONpeerreview
% \begin{center} \bfseries EDICS Category: 3-BBND \end{center}
% \fi
%
% For peerreview papers, this IEEEtran command inserts a page break and
% creates the second title. It will be ignored for other modes.
%\IEEEpeerreviewmaketitle




\item A card is selected from a pack of 52 cards.
\begin{enumerate}[label=(\alph*)]
    \item How many points are there in the sample space?
    \item Calculate the probability that the card is an ace of spades.
    \item Calculate the probability that the card is (i) an ace and (ii) black card.
\end{enumerate}
\solution
%Let $X$ be an bernoulli rv defined as in \tabref{tab:exemplar/11/16/3/26}.  Then, 
\begin{equation}
    p =
        \frac{4}{11} 
\end{equation}
\begin{table}[H]
	\centering
	\input{exemplar/11/16/3/26/tables/Table2.tex}
	\caption{}
        \label{tab:exemplar/11/16/3/26}
\end{table}

\item The probability that a non leap year selected at random will contain 53 sundays.
\\
\solution
%\begin{table}[H]
	\centering
\begin{tabular}{|c|c|c|}
\hline
Random variable &Value &Definition\\ \hline
\multirow{3}{*}{X} &0 &Slips of Rs 1\\
&1 &Slips of Rs 5\\
&2 &Slips of Rs 13\\ \hline
\multirow{2}{*}{Y} &0 &Box A\\
&1 &Box B\\\hline
\end{tabular}
\caption{}
\label{tab:Distribution}
\end{table}
See \tabref{tab:Distribution}.
\begin{align}
p_{Y}\brak{k}= \begin{cases} 
      \frac{1}{3} & {k=0} \\
      \frac{2}{3 }& {k=1} 
   \end{cases}
   \\
p_{Y|X}\brak{0|0} = \frac{19}{25}\, 
p_{Y|X}\brak{0|1} = \frac{6}{25}\,
p_{Y|X}\brak{1|0} = \frac{45}{50}\,
p_{Y|X}\brak{1|2} = \frac{5}{50}
\end{align}
The desired probability is the probability that a slip drawn at random is marked other than Rs 1,
\begin{align}
&=1-p_X\brak{0}\\
&= p_X(1) + p_X(2)
\end{align}
Using Bayes theorem,
\begin{align}
&= p_Y\brak{0} \times \pr{Y=0 | X=1} + p_Y\brak{1} \times \pr{Y=1|X=2}\\
&=\frac{1}{3} \times \frac{6}{25} + \frac{2}{3} \times \frac{5}{50}\\
&=\frac{11}{75}
\end{align}

\newpage

%\tableofcontents

\bigskip

\renewcommand{\thefigure}{\theenumi}
\renewcommand{\thetable}{\theenumi}
%\renewcommand{\theequation}{\theenumi}

%\begin{abstract}
%%\boldmath
%In this letter, an algorithm for evaluating the exact analytical bit error rate  (BER)  for the piecewise linear (PL) combiner for  multiple relays is presented. Previous results were available only for upto three relays. The algorithm is unique in the sense that  the actual mathematical expressions, that are prohibitively large, need not be explicitly obtained. The diversity gain due to multiple relays is shown through plots of the analytical BER, well supported by simulations. 
%
%\end{abstract}
% IEEEtran.cls defaults to using nonbold math in the Abstract.
% This preserves the distinction between vectors and scalars. However,
% if the journal you are submitting to favors bold math in the abstract,
% then you can use LaTeX's standard command \boldmath at the very start
% of the abstract to achieve this. Many IEEE journals frown on math
% in the abstract anyway.

% Note that keywords are not normally used for peerreview papers.
%\begin{IEEEkeywords}
%Cooperative diversity, decode and forward, piecewise linear
%\end{IEEEkeywords}



% For peer review papers, you can put extra information on the cover
% page as needed:
% \ifCLASSOPTIONpeerreview
% \begin{center} \bfseries EDICS Category: 3-BBND \end{center}
% \fi
%
% For peerreview papers, this IEEEtran command inserts a page break and
% creates the second title. It will be ignored for other modes.
%\IEEEpeerreviewmaketitle




\item One of the four persons John, Rita, Aslam or Gurpreet will be promoted next
month. Consequently the sample space consists of four elementary outcomes
S = {John promoted, Rita promoted, Aslam promoted, Gurpreet promoted}
You are told that the chances of John’s promotion is same as that of Gurpreet,
Rita’s chances of promotion are twice as likely as Johns. Aslam’s chances are
four times that of John.
\begin{enumerate}
	\item Determine
	\begin{enumerate}
		\item P (John promoted)
		\item P (Rita promoted)
		\item P (Aslam promoted)
		\item P (Gurpreet promoted)
	\end{enumerate}
	\item If A = {John promoted or Gurpreet promoted}, find P (A).
\end{enumerate}
\solution
%\begin{table}[H]
	\centering
\begin{tabular}{|c|c|c|}
\hline
Random variable &Value &Definition\\ \hline
\multirow{3}{*}{X} &0 &Slips of Rs 1\\
&1 &Slips of Rs 5\\
&2 &Slips of Rs 13\\ \hline
\multirow{2}{*}{Y} &0 &Box A\\
&1 &Box B\\\hline
\end{tabular}
\caption{}
\label{tab:Distribution}
\end{table}
See \tabref{tab:Distribution}.
\begin{align}
p_{Y}\brak{k}= \begin{cases} 
      \frac{1}{3} & {k=0} \\
      \frac{2}{3 }& {k=1} 
   \end{cases}
   \\
p_{Y|X}\brak{0|0} = \frac{19}{25}\, 
p_{Y|X}\brak{0|1} = \frac{6}{25}\,
p_{Y|X}\brak{1|0} = \frac{45}{50}\,
p_{Y|X}\brak{1|2} = \frac{5}{50}
\end{align}
The desired probability is the probability that a slip drawn at random is marked other than Rs 1,
\begin{align}
&=1-p_X\brak{0}\\
&= p_X(1) + p_X(2)
\end{align}
Using Bayes theorem,
\begin{align}
&= p_Y\brak{0} \times \pr{Y=0 | X=1} + p_Y\brak{1} \times \pr{Y=1|X=2}\\
&=\frac{1}{3} \times \frac{6}{25} + \frac{2}{3} \times \frac{5}{50}\\
&=\frac{11}{75}
\end{align}

\newpage

%\tableofcontents

\bigskip

\renewcommand{\thefigure}{\theenumi}
\renewcommand{\thetable}{\theenumi}
%\renewcommand{\theequation}{\theenumi}

%\begin{abstract}
%%\boldmath
%In this letter, an algorithm for evaluating the exact analytical bit error rate  (BER)  for the piecewise linear (PL) combiner for  multiple relays is presented. Previous results were available only for upto three relays. The algorithm is unique in the sense that  the actual mathematical expressions, that are prohibitively large, need not be explicitly obtained. The diversity gain due to multiple relays is shown through plots of the analytical BER, well supported by simulations. 
%
%\end{abstract}
% IEEEtran.cls defaults to using nonbold math in the Abstract.
% This preserves the distinction between vectors and scalars. However,
% if the journal you are submitting to favors bold math in the abstract,
% then you can use LaTeX's standard command \boldmath at the very start
% of the abstract to achieve this. Many IEEE journals frown on math
% in the abstract anyway.

% Note that keywords are not normally used for peerreview papers.
%\begin{IEEEkeywords}
%Cooperative diversity, decode and forward, piecewise linear
%\end{IEEEkeywords}



% For peer review papers, you can put extra information on the cover
% page as needed:
% \ifCLASSOPTIONpeerreview
% \begin{center} \bfseries EDICS Category: 3-BBND \end{center}
% \fi
%
% For peerreview papers, this IEEEtran command inserts a page break and
% creates the second title. It will be ignored for other modes.
%\IEEEpeerreviewmaketitle




\item A card is drawn from a deck of 52 cards. Find the probability of getting a king or a heart or a red card.\\
\solution
%\begin{table}[H]
	\centering
\begin{tabular}{|c|c|c|}
\hline
Random variable &Value &Definition\\ \hline
\multirow{3}{*}{X} &0 &Slips of Rs 1\\
&1 &Slips of Rs 5\\
&2 &Slips of Rs 13\\ \hline
\multirow{2}{*}{Y} &0 &Box A\\
&1 &Box B\\\hline
\end{tabular}
\caption{}
\label{tab:Distribution}
\end{table}
See \tabref{tab:Distribution}.
\begin{align}
p_{Y}\brak{k}= \begin{cases} 
      \frac{1}{3} & {k=0} \\
      \frac{2}{3 }& {k=1} 
   \end{cases}
   \\
p_{Y|X}\brak{0|0} = \frac{19}{25}\, 
p_{Y|X}\brak{0|1} = \frac{6}{25}\,
p_{Y|X}\brak{1|0} = \frac{45}{50}\,
p_{Y|X}\brak{1|2} = \frac{5}{50}
\end{align}
The desired probability is the probability that a slip drawn at random is marked other than Rs 1,
\begin{align}
&=1-p_X\brak{0}\\
&= p_X(1) + p_X(2)
\end{align}
Using Bayes theorem,
\begin{align}
&= p_Y\brak{0} \times \pr{Y=0 | X=1} + p_Y\brak{1} \times \pr{Y=1|X=2}\\
&=\frac{1}{3} \times \frac{6}{25} + \frac{2}{3} \times \frac{5}{50}\\
&=\frac{11}{75}
\end{align}

\newpage

%\tableofcontents

\bigskip

\renewcommand{\thefigure}{\theenumi}
\renewcommand{\thetable}{\theenumi}
%\renewcommand{\theequation}{\theenumi}

%\begin{abstract}
%%\boldmath
%In this letter, an algorithm for evaluating the exact analytical bit error rate  (BER)  for the piecewise linear (PL) combiner for  multiple relays is presented. Previous results were available only for upto three relays. The algorithm is unique in the sense that  the actual mathematical expressions, that are prohibitively large, need not be explicitly obtained. The diversity gain due to multiple relays is shown through plots of the analytical BER, well supported by simulations. 
%
%\end{abstract}
% IEEEtran.cls defaults to using nonbold math in the Abstract.
% This preserves the distinction between vectors and scalars. However,
% if the journal you are submitting to favors bold math in the abstract,
% then you can use LaTeX's standard command \boldmath at the very start
% of the abstract to achieve this. Many IEEE journals frown on math
% in the abstract anyway.

% Note that keywords are not normally used for peerreview papers.
%\begin{IEEEkeywords}
%Cooperative diversity, decode and forward, piecewise linear
%\end{IEEEkeywords}



% For peer review papers, you can put extra information on the cover
% page as needed:
% \ifCLASSOPTIONpeerreview
% \begin{center} \bfseries EDICS Category: 3-BBND \end{center}
% \fi
%
% For peerreview papers, this IEEEtran command inserts a page break and
% creates the second title. It will be ignored for other modes.
%\IEEEpeerreviewmaketitle




\item The probability that a student will pass his examination is 0.73, the probability of
the student getting a compartment is 0.13, and the probability that the student will
either pass or get compartment is 0.96. State True or False.\\
\solution
%\begin{table}[H]
	\centering
\begin{tabular}{|c|c|c|}
\hline
Random variable &Value &Definition\\ \hline
\multirow{3}{*}{X} &0 &Slips of Rs 1\\
&1 &Slips of Rs 5\\
&2 &Slips of Rs 13\\ \hline
\multirow{2}{*}{Y} &0 &Box A\\
&1 &Box B\\\hline
\end{tabular}
\caption{}
\label{tab:Distribution}
\end{table}
See \tabref{tab:Distribution}.
\begin{align}
p_{Y}\brak{k}= \begin{cases} 
      \frac{1}{3} & {k=0} \\
      \frac{2}{3 }& {k=1} 
   \end{cases}
   \\
p_{Y|X}\brak{0|0} = \frac{19}{25}\, 
p_{Y|X}\brak{0|1} = \frac{6}{25}\,
p_{Y|X}\brak{1|0} = \frac{45}{50}\,
p_{Y|X}\brak{1|2} = \frac{5}{50}
\end{align}
The desired probability is the probability that a slip drawn at random is marked other than Rs 1,
\begin{align}
&=1-p_X\brak{0}\\
&= p_X(1) + p_X(2)
\end{align}
Using Bayes theorem,
\begin{align}
&= p_Y\brak{0} \times \pr{Y=0 | X=1} + p_Y\brak{1} \times \pr{Y=1|X=2}\\
&=\frac{1}{3} \times \frac{6}{25} + \frac{2}{3} \times \frac{5}{50}\\
&=\frac{11}{75}
\end{align}

\newpage

%\tableofcontents

\bigskip

\renewcommand{\thefigure}{\theenumi}
\renewcommand{\thetable}{\theenumi}
%\renewcommand{\theequation}{\theenumi}

%\begin{abstract}
%%\boldmath
%In this letter, an algorithm for evaluating the exact analytical bit error rate  (BER)  for the piecewise linear (PL) combiner for  multiple relays is presented. Previous results were available only for upto three relays. The algorithm is unique in the sense that  the actual mathematical expressions, that are prohibitively large, need not be explicitly obtained. The diversity gain due to multiple relays is shown through plots of the analytical BER, well supported by simulations. 
%
%\end{abstract}
% IEEEtran.cls defaults to using nonbold math in the Abstract.
% This preserves the distinction between vectors and scalars. However,
% if the journal you are submitting to favors bold math in the abstract,
% then you can use LaTeX's standard command \boldmath at the very start
% of the abstract to achieve this. Many IEEE journals frown on math
% in the abstract anyway.

% Note that keywords are not normally used for peerreview papers.
%\begin{IEEEkeywords}
%Cooperative diversity, decode and forward, piecewise linear
%\end{IEEEkeywords}



% For peer review papers, you can put extra information on the cover
% page as needed:
% \ifCLASSOPTIONpeerreview
% \begin{center} \bfseries EDICS Category: 3-BBND \end{center}
% \fi
%
% For peerreview papers, this IEEEtran command inserts a page break and
% creates the second title. It will be ignored for other modes.
%\IEEEpeerreviewmaketitle




\item A card is selected from a pack of 52 cards\\
\begin{enumerate}[label=(\alph*)]
\item How many points are there in the sample space?
\item Calculate the probability that the cards is an ace of spades.
\item Calculate the probability that the card is (i) an ace (ii)black card.\\
\end{enumerate}
%\input{ncert/11/16/3/4_1/Prob_4.tex}
\item In a non-leap year, the probability of having 53 tuesdays or 53 wednesdays is\\
\solution
%A non-leap year has a total of 365 days, and a week has 7 days.\\
So it can be expressed as 
\begin{align}
365\text{days} &=52\times 7+1 \text{day}
\end{align}
$\implies$ 52 tuesdays or wednesdays\\
Random variable X denotes the days of a week
\begin{align}
p_X\brak{k}&=\frac{1}{7}; \quad \brak{1<k<7}
\end{align}
So the probability of extra day being tuesday or wednesday is
\begin{align}
p_X\brak{3}+p_X\brak{4}&=\frac{1}{7}+\frac{1}{7}=\frac{2}{7}
\end{align}



\item There are 1000 sealed envelopes in a box, 10 of them contain a cash prize of
Rs 100 each, 100 of them contain a cash prize of Rs 50 each and 200 of them
contain a cash prize of Rs 10 each and rest do not contain any cash prize. If they
are well shuffled and an envelope is picked up out, what is the probability that it
contains no cash prize?\\
\solution
%\begin{table}[H]
	\centering
\begin{tabular}{|c|c|c|}
\hline
Random variable &Value &Definition\\ \hline
\multirow{3}{*}{X} &0 &Slips of Rs 1\\
&1 &Slips of Rs 5\\
&2 &Slips of Rs 13\\ \hline
\multirow{2}{*}{Y} &0 &Box A\\
&1 &Box B\\\hline
\end{tabular}
\caption{}
\label{tab:Distribution}
\end{table}
See \tabref{tab:Distribution}.
\begin{align}
p_{Y}\brak{k}= \begin{cases} 
      \frac{1}{3} & {k=0} \\
      \frac{2}{3 }& {k=1} 
   \end{cases}
   \\
p_{Y|X}\brak{0|0} = \frac{19}{25}\, 
p_{Y|X}\brak{0|1} = \frac{6}{25}\,
p_{Y|X}\brak{1|0} = \frac{45}{50}\,
p_{Y|X}\brak{1|2} = \frac{5}{50}
\end{align}
The desired probability is the probability that a slip drawn at random is marked other than Rs 1,
\begin{align}
&=1-p_X\brak{0}\\
&= p_X(1) + p_X(2)
\end{align}
Using Bayes theorem,
\begin{align}
&= p_Y\brak{0} \times \pr{Y=0 | X=1} + p_Y\brak{1} \times \pr{Y=1|X=2}\\
&=\frac{1}{3} \times \frac{6}{25} + \frac{2}{3} \times \frac{5}{50}\\
&=\frac{11}{75}
\end{align}

\newpage

%\tableofcontents

\bigskip

\renewcommand{\thefigure}{\theenumi}
\renewcommand{\thetable}{\theenumi}
%\renewcommand{\theequation}{\theenumi}

%\begin{abstract}
%%\boldmath
%In this letter, an algorithm for evaluating the exact analytical bit error rate  (BER)  for the piecewise linear (PL) combiner for  multiple relays is presented. Previous results were available only for upto three relays. The algorithm is unique in the sense that  the actual mathematical expressions, that are prohibitively large, need not be explicitly obtained. The diversity gain due to multiple relays is shown through plots of the analytical BER, well supported by simulations. 
%
%\end{abstract}
% IEEEtran.cls defaults to using nonbold math in the Abstract.
% This preserves the distinction between vectors and scalars. However,
% if the journal you are submitting to favors bold math in the abstract,
% then you can use LaTeX's standard command \boldmath at the very start
% of the abstract to achieve this. Many IEEE journals frown on math
% in the abstract anyway.

% Note that keywords are not normally used for peerreview papers.
%\begin{IEEEkeywords}
%Cooperative diversity, decode and forward, piecewise linear
%\end{IEEEkeywords}



% For peer review papers, you can put extra information on the cover
% page as needed:
% \ifCLASSOPTIONpeerreview
% \begin{center} \bfseries EDICS Category: 3-BBND \end{center}
% \fi
%
% For peerreview papers, this IEEEtran command inserts a page break and
% creates the second title. It will be ignored for other modes.
%\IEEEpeerreviewmaketitle




\item 
A die is thrown and a card is selected at random from a deck of 52 playing cards. The probability of getting an even number on the die and a spade card.\\
\solution
%\begin{table}[H]
	\centering
\begin{tabular}{|c|c|c|}
\hline
Random variable &Value &Definition\\ \hline
\multirow{3}{*}{X} &0 &Slips of Rs 1\\
&1 &Slips of Rs 5\\
&2 &Slips of Rs 13\\ \hline
\multirow{2}{*}{Y} &0 &Box A\\
&1 &Box B\\\hline
\end{tabular}
\caption{}
\label{tab:Distribution}
\end{table}
See \tabref{tab:Distribution}.
\begin{align}
p_{Y}\brak{k}= \begin{cases} 
      \frac{1}{3} & {k=0} \\
      \frac{2}{3 }& {k=1} 
   \end{cases}
   \\
p_{Y|X}\brak{0|0} = \frac{19}{25}\, 
p_{Y|X}\brak{0|1} = \frac{6}{25}\,
p_{Y|X}\brak{1|0} = \frac{45}{50}\,
p_{Y|X}\brak{1|2} = \frac{5}{50}
\end{align}
The desired probability is the probability that a slip drawn at random is marked other than Rs 1,
\begin{align}
&=1-p_X\brak{0}\\
&= p_X(1) + p_X(2)
\end{align}
Using Bayes theorem,
\begin{align}
&= p_Y\brak{0} \times \pr{Y=0 | X=1} + p_Y\brak{1} \times \pr{Y=1|X=2}\\
&=\frac{1}{3} \times \frac{6}{25} + \frac{2}{3} \times \frac{5}{50}\\
&=\frac{11}{75}
\end{align}

\newpage

%\tableofcontents

\bigskip

\renewcommand{\thefigure}{\theenumi}
\renewcommand{\thetable}{\theenumi}
%\renewcommand{\theequation}{\theenumi}

%\begin{abstract}
%%\boldmath
%In this letter, an algorithm for evaluating the exact analytical bit error rate  (BER)  for the piecewise linear (PL) combiner for  multiple relays is presented. Previous results were available only for upto three relays. The algorithm is unique in the sense that  the actual mathematical expressions, that are prohibitively large, need not be explicitly obtained. The diversity gain due to multiple relays is shown through plots of the analytical BER, well supported by simulations. 
%
%\end{abstract}
% IEEEtran.cls defaults to using nonbold math in the Abstract.
% This preserves the distinction between vectors and scalars. However,
% if the journal you are submitting to favors bold math in the abstract,
% then you can use LaTeX's standard command \boldmath at the very start
% of the abstract to achieve this. Many IEEE journals frown on math
% in the abstract anyway.

% Note that keywords are not normally used for peerreview papers.
%\begin{IEEEkeywords}
%Cooperative diversity, decode and forward, piecewise linear
%\end{IEEEkeywords}



% For peer review papers, you can put extra information on the cover
% page as needed:
% \ifCLASSOPTIONpeerreview
% \begin{center} \bfseries EDICS Category: 3-BBND \end{center}
% \fi
%
% For peerreview papers, this IEEEtran command inserts a page break and
% creates the second title. It will be ignored for other modes.
%\IEEEpeerreviewmaketitle




\item
If 4-digit numbers greater than 5,000 are randomly formed from the digits 0, 1, 3, 5, and 7, what is the probability of forming a number divisible by 5 when:
\begin{enumerate}
    \item The digits are repeated?
    \item The repetition of digits is not allowed?
\end{enumerate}
\solution
%\begin{table}[H]
	\centering
\begin{tabular}{|c|c|c|}
\hline
Random variable &Value &Definition\\ \hline
\multirow{3}{*}{X} &0 &Slips of Rs 1\\
&1 &Slips of Rs 5\\
&2 &Slips of Rs 13\\ \hline
\multirow{2}{*}{Y} &0 &Box A\\
&1 &Box B\\\hline
\end{tabular}
\caption{}
\label{tab:Distribution}
\end{table}
See \tabref{tab:Distribution}.
\begin{align}
p_{Y}\brak{k}= \begin{cases} 
      \frac{1}{3} & {k=0} \\
      \frac{2}{3 }& {k=1} 
   \end{cases}
   \\
p_{Y|X}\brak{0|0} = \frac{19}{25}\, 
p_{Y|X}\brak{0|1} = \frac{6}{25}\,
p_{Y|X}\brak{1|0} = \frac{45}{50}\,
p_{Y|X}\brak{1|2} = \frac{5}{50}
\end{align}
The desired probability is the probability that a slip drawn at random is marked other than Rs 1,
\begin{align}
&=1-p_X\brak{0}\\
&= p_X(1) + p_X(2)
\end{align}
Using Bayes theorem,
\begin{align}
&= p_Y\brak{0} \times \pr{Y=0 | X=1} + p_Y\brak{1} \times \pr{Y=1|X=2}\\
&=\frac{1}{3} \times \frac{6}{25} + \frac{2}{3} \times \frac{5}{50}\\
&=\frac{11}{75}
\end{align}

\newpage

%\tableofcontents

\bigskip

\renewcommand{\thefigure}{\theenumi}
\renewcommand{\thetable}{\theenumi}
%\renewcommand{\theequation}{\theenumi}

%\begin{abstract}
%%\boldmath
%In this letter, an algorithm for evaluating the exact analytical bit error rate  (BER)  for the piecewise linear (PL) combiner for  multiple relays is presented. Previous results were available only for upto three relays. The algorithm is unique in the sense that  the actual mathematical expressions, that are prohibitively large, need not be explicitly obtained. The diversity gain due to multiple relays is shown through plots of the analytical BER, well supported by simulations. 
%
%\end{abstract}
% IEEEtran.cls defaults to using nonbold math in the Abstract.
% This preserves the distinction between vectors and scalars. However,
% if the journal you are submitting to favors bold math in the abstract,
% then you can use LaTeX's standard command \boldmath at the very start
% of the abstract to achieve this. Many IEEE journals frown on math
% in the abstract anyway.

% Note that keywords are not normally used for peerreview papers.
%\begin{IEEEkeywords}
%Cooperative diversity, decode and forward, piecewise linear
%\end{IEEEkeywords}



% For peer review papers, you can put extra information on the cover
% page as needed:
% \ifCLASSOPTIONpeerreview
% \begin{center} \bfseries EDICS Category: 3-BBND \end{center}
% \fi
%
% For peerreview papers, this IEEEtran command inserts a page break and
% creates the second title. It will be ignored for other modes.
%\IEEEpeerreviewmaketitle




\item Consider the probability space $\brak{\Omega, \mathcal{G}, P}$ where $\Omega = [0,2]$ and $\mathcal{G} = \cbrak{\phi, \Omega, [0,1], (1,2]}$. Let $X$ and $Y$ be two functions on $\Omega$ defined as
\begin{align*}
    X(\omega) = 
    \begin{cases}
        1 & \text{if }\omega \in [0, 1]\\
        2 & \text{if }\omega \in (1, 2]
    \end{cases}
\end{align*}
and
\begin{align*}
    Y(\omega) = 
    \begin{cases}
        2 & \text{if }\omega \in [0, 1.5]\\
        3 & \text{if }\omega \in (1.5, 2].
    \end{cases}
\end{align*}
Then which one of the following statements is true?
\begin{enumerate}
    \item [(A)] $X$ is a random variable with respect to $\mathcal{G}$, but $Y$ is not a random variable with respect to $\mathcal{G}$.
    \item [(B)] $Y$ is a random variable with respect to $\mathcal{G}$, but $X$ is not a random variable with respect to $\mathcal{G}$.
    \item [(C)] Neither $X$ nor $Y$ is a random variable with respect to $\mathcal{G}$.
    \item [(D)] Both $X$ and $Y$ are random variables with respect to $\mathcal{G}$.
\end{enumerate} \hfill (GATE ST 2023)\\
\solution
%\begin{table}[H]
	\centering
\begin{tabular}{|c|c|c|}
\hline
Random variable &Value &Definition\\ \hline
\multirow{3}{*}{X} &0 &Slips of Rs 1\\
&1 &Slips of Rs 5\\
&2 &Slips of Rs 13\\ \hline
\multirow{2}{*}{Y} &0 &Box A\\
&1 &Box B\\\hline
\end{tabular}
\caption{}
\label{tab:Distribution}
\end{table}
See \tabref{tab:Distribution}.
\begin{align}
p_{Y}\brak{k}= \begin{cases} 
      \frac{1}{3} & {k=0} \\
      \frac{2}{3 }& {k=1} 
   \end{cases}
   \\
p_{Y|X}\brak{0|0} = \frac{19}{25}\, 
p_{Y|X}\brak{0|1} = \frac{6}{25}\,
p_{Y|X}\brak{1|0} = \frac{45}{50}\,
p_{Y|X}\brak{1|2} = \frac{5}{50}
\end{align}
The desired probability is the probability that a slip drawn at random is marked other than Rs 1,
\begin{align}
&=1-p_X\brak{0}\\
&= p_X(1) + p_X(2)
\end{align}
Using Bayes theorem,
\begin{align}
&= p_Y\brak{0} \times \pr{Y=0 | X=1} + p_Y\brak{1} \times \pr{Y=1|X=2}\\
&=\frac{1}{3} \times \frac{6}{25} + \frac{2}{3} \times \frac{5}{50}\\
&=\frac{11}{75}
\end{align}

\newpage

%\tableofcontents

\bigskip

\renewcommand{\thefigure}{\theenumi}
\renewcommand{\thetable}{\theenumi}
%\renewcommand{\theequation}{\theenumi}

%\begin{abstract}
%%\boldmath
%In this letter, an algorithm for evaluating the exact analytical bit error rate  (BER)  for the piecewise linear (PL) combiner for  multiple relays is presented. Previous results were available only for upto three relays. The algorithm is unique in the sense that  the actual mathematical expressions, that are prohibitively large, need not be explicitly obtained. The diversity gain due to multiple relays is shown through plots of the analytical BER, well supported by simulations. 
%
%\end{abstract}
% IEEEtran.cls defaults to using nonbold math in the Abstract.
% This preserves the distinction between vectors and scalars. However,
% if the journal you are submitting to favors bold math in the abstract,
% then you can use LaTeX's standard command \boldmath at the very start
% of the abstract to achieve this. Many IEEE journals frown on math
% in the abstract anyway.

% Note that keywords are not normally used for peerreview papers.
%\begin{IEEEkeywords}
%Cooperative diversity, decode and forward, piecewise linear
%\end{IEEEkeywords}



% For peer review papers, you can put extra information on the cover
% page as needed:
% \ifCLASSOPTIONpeerreview
% \begin{center} \bfseries EDICS Category: 3-BBND \end{center}
% \fi
%
% For peerreview papers, this IEEEtran command inserts a page break and
% creates the second title. It will be ignored for other modes.
%\IEEEpeerreviewmaketitle




	\item  A die is loaded in such a way that each odd number is twice as likely to occur as
each even number. Find $P(G)$, where $G$ is the event that a number greater than
3 occurs on a single roll of the die.
\\
\solution
		%\begin{table}[H]
	\centering
\begin{tabular}{|c|c|c|}
\hline
Random variable &Value &Definition\\ \hline
\multirow{3}{*}{X} &0 &Slips of Rs 1\\
&1 &Slips of Rs 5\\
&2 &Slips of Rs 13\\ \hline
\multirow{2}{*}{Y} &0 &Box A\\
&1 &Box B\\\hline
\end{tabular}
\caption{}
\label{tab:Distribution}
\end{table}
See \tabref{tab:Distribution}.
\begin{align}
p_{Y}\brak{k}= \begin{cases} 
      \frac{1}{3} & {k=0} \\
      \frac{2}{3 }& {k=1} 
   \end{cases}
   \\
p_{Y|X}\brak{0|0} = \frac{19}{25}\, 
p_{Y|X}\brak{0|1} = \frac{6}{25}\,
p_{Y|X}\brak{1|0} = \frac{45}{50}\,
p_{Y|X}\brak{1|2} = \frac{5}{50}
\end{align}
The desired probability is the probability that a slip drawn at random is marked other than Rs 1,
\begin{align}
&=1-p_X\brak{0}\\
&= p_X(1) + p_X(2)
\end{align}
Using Bayes theorem,
\begin{align}
&= p_Y\brak{0} \times \pr{Y=0 | X=1} + p_Y\brak{1} \times \pr{Y=1|X=2}\\
&=\frac{1}{3} \times \frac{6}{25} + \frac{2}{3} \times \frac{5}{50}\\
&=\frac{11}{75}
\end{align}

\newpage

%\tableofcontents

\bigskip

\renewcommand{\thefigure}{\theenumi}
\renewcommand{\thetable}{\theenumi}
%\renewcommand{\theequation}{\theenumi}

%\begin{abstract}
%%\boldmath
%In this letter, an algorithm for evaluating the exact analytical bit error rate  (BER)  for the piecewise linear (PL) combiner for  multiple relays is presented. Previous results were available only for upto three relays. The algorithm is unique in the sense that  the actual mathematical expressions, that are prohibitively large, need not be explicitly obtained. The diversity gain due to multiple relays is shown through plots of the analytical BER, well supported by simulations. 
%
%\end{abstract}
% IEEEtran.cls defaults to using nonbold math in the Abstract.
% This preserves the distinction between vectors and scalars. However,
% if the journal you are submitting to favors bold math in the abstract,
% then you can use LaTeX's standard command \boldmath at the very start
% of the abstract to achieve this. Many IEEE journals frown on math
% in the abstract anyway.

% Note that keywords are not normally used for peerreview papers.
%\begin{IEEEkeywords}
%Cooperative diversity, decode and forward, piecewise linear
%\end{IEEEkeywords}



% For peer review papers, you can put extra information on the cover
% page as needed:
% \ifCLASSOPTIONpeerreview
% \begin{center} \bfseries EDICS Category: 3-BBND \end{center}
% \fi
%
% For peerreview papers, this IEEEtran command inserts a page break and
% creates the second title. It will be ignored for other modes.
%\IEEEpeerreviewmaketitle




	\item All the jacks, queens and kings are removed from a deck of 52 playing cards. The remaining cards are well shuffled and then one card is drawn at random. Giving ace a value 1 similar value for other cards, find the probability that the card has a value 
		\begin{enumerate}
			\item 7
			\item greater than 7
			\item less than 7
		\end{enumerate}
		%Number of cards left after removing all jacks, queens and kings 
\begin{align}
N	= 52 - 4\times 3
	= 40
\end{align}
%\begin{table}[H]
%\def\arraystretch{1.2}
%\begin{tabular}{|c|c|c|}
%\hline
%	\textbf{Parameter} &\textbf{Value} &\textbf{Description}\\ \hline
%	$X$ &1-10 &Represents the value of the card picked \\ \hline
%\end{tabular}
%\end{table}
Let $1 \le X \le 10$ be the value of the card picked.  Then,
\begin{align}
	p_X(k) &= \Pr(X=k)\ \forall\ 1 \leq k \leq 10\\
	&= \frac{4\times 1}{40}\\
	&= \frac{1}{10}\\
	\therefore p_X(k) &= 
	\begin{cases}
		\frac{1}{10} & 1 \leq k \leq 10\\
		0 & \text{otherwise}
	\end{cases}
\end{align}
and
\begin{align}
	F_{X}(k) &= \sum_{m=0}^{k}p_{X}(m) \quad 1 \leq k \leq 10\\
	&= \frac{k}{10}\\
	\therefore F_{X}(k) &= 
	\begin{cases}
		0 & k \leq 0\\
		\frac{k}{10} & 1\leq k \leq 10\\
		1 & k > 10 
	\end{cases}
\end{align}
\begin{enumerate}
	\item Probability that card has value equal to 7 is
		\begin{align}
			 p_{X}(7)
			= \frac{1}{10}
		\end{align}
	\item Probability that card has value greater than 7 is
		\begin{align}
			1 - F_X(7)
			&= 1 - \frac{7}{10}
			\\
			&= \frac{3}{10}
		\end{align}
	\item Probability that card has value less than 7 is
		\begin{align}
			 F_{X}(6)
			=\frac{6}{10}
		\end{align}
\end{enumerate}

  \item A Lot consists of 48 mobile phones of which 42 are good, 3 have only minor defects and 3 have major defects.Varnika will buy a phone if it is good but the trader will only buy a mobile if it has no major defects. One phone is selected at random from the lot. What is the probability that it is
\begin{enumerate}
	\item acceptable to Varnika?
            \item acceptable to the trader?
\end{enumerate}
\solution
	%\begin{table}[H]
	\centering
\begin{tabular}{|c|c|c|}
\hline
Random variable &Value &Definition\\ \hline
\multirow{3}{*}{X} &0 &Slips of Rs 1\\
&1 &Slips of Rs 5\\
&2 &Slips of Rs 13\\ \hline
\multirow{2}{*}{Y} &0 &Box A\\
&1 &Box B\\\hline
\end{tabular}
\caption{}
\label{tab:Distribution}
\end{table}
See \tabref{tab:Distribution}.
\begin{align}
p_{Y}\brak{k}= \begin{cases} 
      \frac{1}{3} & {k=0} \\
      \frac{2}{3 }& {k=1} 
   \end{cases}
   \\
p_{Y|X}\brak{0|0} = \frac{19}{25}\, 
p_{Y|X}\brak{0|1} = \frac{6}{25}\,
p_{Y|X}\brak{1|0} = \frac{45}{50}\,
p_{Y|X}\brak{1|2} = \frac{5}{50}
\end{align}
The desired probability is the probability that a slip drawn at random is marked other than Rs 1,
\begin{align}
&=1-p_X\brak{0}\\
&= p_X(1) + p_X(2)
\end{align}
Using Bayes theorem,
\begin{align}
&= p_Y\brak{0} \times \pr{Y=0 | X=1} + p_Y\brak{1} \times \pr{Y=1|X=2}\\
&=\frac{1}{3} \times \frac{6}{25} + \frac{2}{3} \times \frac{5}{50}\\
&=\frac{11}{75}
\end{align}

\newpage

%\tableofcontents

\bigskip

\renewcommand{\thefigure}{\theenumi}
\renewcommand{\thetable}{\theenumi}
%\renewcommand{\theequation}{\theenumi}

%\begin{abstract}
%%\boldmath
%In this letter, an algorithm for evaluating the exact analytical bit error rate  (BER)  for the piecewise linear (PL) combiner for  multiple relays is presented. Previous results were available only for upto three relays. The algorithm is unique in the sense that  the actual mathematical expressions, that are prohibitively large, need not be explicitly obtained. The diversity gain due to multiple relays is shown through plots of the analytical BER, well supported by simulations. 
%
%\end{abstract}
% IEEEtran.cls defaults to using nonbold math in the Abstract.
% This preserves the distinction between vectors and scalars. However,
% if the journal you are submitting to favors bold math in the abstract,
% then you can use LaTeX's standard command \boldmath at the very start
% of the abstract to achieve this. Many IEEE journals frown on math
% in the abstract anyway.

% Note that keywords are not normally used for peerreview papers.
%\begin{IEEEkeywords}
%Cooperative diversity, decode and forward, piecewise linear
%\end{IEEEkeywords}



% For peer review papers, you can put extra information on the cover
% page as needed:
% \ifCLASSOPTIONpeerreview
% \begin{center} \bfseries EDICS Category: 3-BBND \end{center}
% \fi
%
% For peerreview papers, this IEEEtran command inserts a page break and
% creates the second title. It will be ignored for other modes.
%\IEEEpeerreviewmaketitle




 \item A student says that if you throw a die, it will show up 1 or not 1. Therefore, the probability of getting 1 and the probability of getting 'not 1' each is equal to $\frac{1}{2}$. Is this correct? Give reasons.\\
 \solution
        %\begin{table}[H]
	\centering
\begin{tabular}{|c|c|c|}
\hline
Random variable &Value &Definition\\ \hline
\multirow{3}{*}{X} &0 &Slips of Rs 1\\
&1 &Slips of Rs 5\\
&2 &Slips of Rs 13\\ \hline
\multirow{2}{*}{Y} &0 &Box A\\
&1 &Box B\\\hline
\end{tabular}
\caption{}
\label{tab:Distribution}
\end{table}
See \tabref{tab:Distribution}.
\begin{align}
p_{Y}\brak{k}= \begin{cases} 
      \frac{1}{3} & {k=0} \\
      \frac{2}{3 }& {k=1} 
   \end{cases}
   \\
p_{Y|X}\brak{0|0} = \frac{19}{25}\, 
p_{Y|X}\brak{0|1} = \frac{6}{25}\,
p_{Y|X}\brak{1|0} = \frac{45}{50}\,
p_{Y|X}\brak{1|2} = \frac{5}{50}
\end{align}
The desired probability is the probability that a slip drawn at random is marked other than Rs 1,
\begin{align}
&=1-p_X\brak{0}\\
&= p_X(1) + p_X(2)
\end{align}
Using Bayes theorem,
\begin{align}
&= p_Y\brak{0} \times \pr{Y=0 | X=1} + p_Y\brak{1} \times \pr{Y=1|X=2}\\
&=\frac{1}{3} \times \frac{6}{25} + \frac{2}{3} \times \frac{5}{50}\\
&=\frac{11}{75}
\end{align}

\newpage

%\tableofcontents

\bigskip

\renewcommand{\thefigure}{\theenumi}
\renewcommand{\thetable}{\theenumi}
%\renewcommand{\theequation}{\theenumi}

%\begin{abstract}
%%\boldmath
%In this letter, an algorithm for evaluating the exact analytical bit error rate  (BER)  for the piecewise linear (PL) combiner for  multiple relays is presented. Previous results were available only for upto three relays. The algorithm is unique in the sense that  the actual mathematical expressions, that are prohibitively large, need not be explicitly obtained. The diversity gain due to multiple relays is shown through plots of the analytical BER, well supported by simulations. 
%
%\end{abstract}
% IEEEtran.cls defaults to using nonbold math in the Abstract.
% This preserves the distinction between vectors and scalars. However,
% if the journal you are submitting to favors bold math in the abstract,
% then you can use LaTeX's standard command \boldmath at the very start
% of the abstract to achieve this. Many IEEE journals frown on math
% in the abstract anyway.

% Note that keywords are not normally used for peerreview papers.
%\begin{IEEEkeywords}
%Cooperative diversity, decode and forward, piecewise linear
%\end{IEEEkeywords}



% For peer review papers, you can put extra information on the cover
% page as needed:
% \ifCLASSOPTIONpeerreview
% \begin{center} \bfseries EDICS Category: 3-BBND \end{center}
% \fi
%
% For peerreview papers, this IEEEtran command inserts a page break and
% creates the second title. It will be ignored for other modes.
%\IEEEpeerreviewmaketitle




   \item Four candidates A, B, C, D have ap-
plied for the assignment to coach a school cricket
team. If A is twice as likely to be selected as B, and
B and C are given about the same chance of being
selected, while C is twice as likely to be selected
as D, what are the probabilities that
\begin{enumerate}
\item C will be selected?
\item A will not be selected?
\end{enumerate}
	%\begin{table}[H]
	\centering
\begin{tabular}{|c|c|c|}
\hline
Random variable &Value &Definition\\ \hline
\multirow{3}{*}{X} &0 &Slips of Rs 1\\
&1 &Slips of Rs 5\\
&2 &Slips of Rs 13\\ \hline
\multirow{2}{*}{Y} &0 &Box A\\
&1 &Box B\\\hline
\end{tabular}
\caption{}
\label{tab:Distribution}
\end{table}
See \tabref{tab:Distribution}.
\begin{align}
p_{Y}\brak{k}= \begin{cases} 
      \frac{1}{3} & {k=0} \\
      \frac{2}{3 }& {k=1} 
   \end{cases}
   \\
p_{Y|X}\brak{0|0} = \frac{19}{25}\, 
p_{Y|X}\brak{0|1} = \frac{6}{25}\,
p_{Y|X}\brak{1|0} = \frac{45}{50}\,
p_{Y|X}\brak{1|2} = \frac{5}{50}
\end{align}
The desired probability is the probability that a slip drawn at random is marked other than Rs 1,
\begin{align}
&=1-p_X\brak{0}\\
&= p_X(1) + p_X(2)
\end{align}
Using Bayes theorem,
\begin{align}
&= p_Y\brak{0} \times \pr{Y=0 | X=1} + p_Y\brak{1} \times \pr{Y=1|X=2}\\
&=\frac{1}{3} \times \frac{6}{25} + \frac{2}{3} \times \frac{5}{50}\\
&=\frac{11}{75}
\end{align}

\newpage

%\tableofcontents

\bigskip

\renewcommand{\thefigure}{\theenumi}
\renewcommand{\thetable}{\theenumi}
%\renewcommand{\theequation}{\theenumi}

%\begin{abstract}
%%\boldmath
%In this letter, an algorithm for evaluating the exact analytical bit error rate  (BER)  for the piecewise linear (PL) combiner for  multiple relays is presented. Previous results were available only for upto three relays. The algorithm is unique in the sense that  the actual mathematical expressions, that are prohibitively large, need not be explicitly obtained. The diversity gain due to multiple relays is shown through plots of the analytical BER, well supported by simulations. 
%
%\end{abstract}
% IEEEtran.cls defaults to using nonbold math in the Abstract.
% This preserves the distinction between vectors and scalars. However,
% if the journal you are submitting to favors bold math in the abstract,
% then you can use LaTeX's standard command \boldmath at the very start
% of the abstract to achieve this. Many IEEE journals frown on math
% in the abstract anyway.

% Note that keywords are not normally used for peerreview papers.
%\begin{IEEEkeywords}
%Cooperative diversity, decode and forward, piecewise linear
%\end{IEEEkeywords}



% For peer review papers, you can put extra information on the cover
% page as needed:
% \ifCLASSOPTIONpeerreview
% \begin{center} \bfseries EDICS Category: 3-BBND \end{center}
% \fi
%
% For peerreview papers, this IEEEtran command inserts a page break and
% creates the second title. It will be ignored for other modes.
%\IEEEpeerreviewmaketitle




 \item A bag contain 24 balls of which $x$ balls are red, $2x$ are white and $3x$ are blue. A ball is selected at random, What is the probability that it is
\begin{enumerate}[label=\alph*)]
\item not red ?
\item white ?
\end{enumerate}
%\begin{table}[H]
	\centering
\begin{tabular}{|c|c|c|}
\hline
Random variable &Value &Definition\\ \hline
\multirow{3}{*}{X} &0 &Slips of Rs 1\\
&1 &Slips of Rs 5\\
&2 &Slips of Rs 13\\ \hline
\multirow{2}{*}{Y} &0 &Box A\\
&1 &Box B\\\hline
\end{tabular}
\caption{}
\label{tab:Distribution}
\end{table}
See \tabref{tab:Distribution}.
\begin{align}
p_{Y}\brak{k}= \begin{cases} 
      \frac{1}{3} & {k=0} \\
      \frac{2}{3 }& {k=1} 
   \end{cases}
   \\
p_{Y|X}\brak{0|0} = \frac{19}{25}\, 
p_{Y|X}\brak{0|1} = \frac{6}{25}\,
p_{Y|X}\brak{1|0} = \frac{45}{50}\,
p_{Y|X}\brak{1|2} = \frac{5}{50}
\end{align}
The desired probability is the probability that a slip drawn at random is marked other than Rs 1,
\begin{align}
&=1-p_X\brak{0}\\
&= p_X(1) + p_X(2)
\end{align}
Using Bayes theorem,
\begin{align}
&= p_Y\brak{0} \times \pr{Y=0 | X=1} + p_Y\brak{1} \times \pr{Y=1|X=2}\\
&=\frac{1}{3} \times \frac{6}{25} + \frac{2}{3} \times \frac{5}{50}\\
&=\frac{11}{75}
\end{align}

\newpage

%\tableofcontents

\bigskip

\renewcommand{\thefigure}{\theenumi}
\renewcommand{\thetable}{\theenumi}
%\renewcommand{\theequation}{\theenumi}

%\begin{abstract}
%%\boldmath
%In this letter, an algorithm for evaluating the exact analytical bit error rate  (BER)  for the piecewise linear (PL) combiner for  multiple relays is presented. Previous results were available only for upto three relays. The algorithm is unique in the sense that  the actual mathematical expressions, that are prohibitively large, need not be explicitly obtained. The diversity gain due to multiple relays is shown through plots of the analytical BER, well supported by simulations. 
%
%\end{abstract}
% IEEEtran.cls defaults to using nonbold math in the Abstract.
% This preserves the distinction between vectors and scalars. However,
% if the journal you are submitting to favors bold math in the abstract,
% then you can use LaTeX's standard command \boldmath at the very start
% of the abstract to achieve this. Many IEEE journals frown on math
% in the abstract anyway.

% Note that keywords are not normally used for peerreview papers.
%\begin{IEEEkeywords}
%Cooperative diversity, decode and forward, piecewise linear
%\end{IEEEkeywords}



% For peer review papers, you can put extra information on the cover
% page as needed:
% \ifCLASSOPTIONpeerreview
% \begin{center} \bfseries EDICS Category: 3-BBND \end{center}
% \fi
%
% For peerreview papers, this IEEEtran command inserts a page break and
% creates the second title. It will be ignored for other modes.
%\IEEEpeerreviewmaketitle




If the letters of the word ASSASSINATION are arranged at random. Find the Probability that
\begin{enumerate}[label=(\alph*)]
\item Four $S's$ come consecutively in the word
\item Two  $I's$ and two $N's$ come together
\item All $A's$ are not coming together
\item No two $A's$ are coming together
\end{enumerate}
%\begin{table}[H]
	\centering
\begin{tabular}{|c|c|c|}
\hline
Random variable &Value &Definition\\ \hline
\multirow{3}{*}{X} &0 &Slips of Rs 1\\
&1 &Slips of Rs 5\\
&2 &Slips of Rs 13\\ \hline
\multirow{2}{*}{Y} &0 &Box A\\
&1 &Box B\\\hline
\end{tabular}
\caption{}
\label{tab:Distribution}
\end{table}
See \tabref{tab:Distribution}.
\begin{align}
p_{Y}\brak{k}= \begin{cases} 
      \frac{1}{3} & {k=0} \\
      \frac{2}{3 }& {k=1} 
   \end{cases}
   \\
p_{Y|X}\brak{0|0} = \frac{19}{25}\, 
p_{Y|X}\brak{0|1} = \frac{6}{25}\,
p_{Y|X}\brak{1|0} = \frac{45}{50}\,
p_{Y|X}\brak{1|2} = \frac{5}{50}
\end{align}
The desired probability is the probability that a slip drawn at random is marked other than Rs 1,
\begin{align}
&=1-p_X\brak{0}\\
&= p_X(1) + p_X(2)
\end{align}
Using Bayes theorem,
\begin{align}
&= p_Y\brak{0} \times \pr{Y=0 | X=1} + p_Y\brak{1} \times \pr{Y=1|X=2}\\
&=\frac{1}{3} \times \frac{6}{25} + \frac{2}{3} \times \frac{5}{50}\\
&=\frac{11}{75}
\end{align}

\newpage

%\tableofcontents

\bigskip

\renewcommand{\thefigure}{\theenumi}
\renewcommand{\thetable}{\theenumi}
%\renewcommand{\theequation}{\theenumi}

%\begin{abstract}
%%\boldmath
%In this letter, an algorithm for evaluating the exact analytical bit error rate  (BER)  for the piecewise linear (PL) combiner for  multiple relays is presented. Previous results were available only for upto three relays. The algorithm is unique in the sense that  the actual mathematical expressions, that are prohibitively large, need not be explicitly obtained. The diversity gain due to multiple relays is shown through plots of the analytical BER, well supported by simulations. 
%
%\end{abstract}
% IEEEtran.cls defaults to using nonbold math in the Abstract.
% This preserves the distinction between vectors and scalars. However,
% if the journal you are submitting to favors bold math in the abstract,
% then you can use LaTeX's standard command \boldmath at the very start
% of the abstract to achieve this. Many IEEE journals frown on math
% in the abstract anyway.

% Note that keywords are not normally used for peerreview papers.
%\begin{IEEEkeywords}
%Cooperative diversity, decode and forward, piecewise linear
%\end{IEEEkeywords}



% For peer review papers, you can put extra information on the cover
% page as needed:
% \ifCLASSOPTIONpeerreview
% \begin{center} \bfseries EDICS Category: 3-BBND \end{center}
% \fi
%
% For peerreview papers, this IEEEtran command inserts a page break and
% creates the second title. It will be ignored for other modes.
%\IEEEpeerreviewmaketitle




	\item One urn contains two black balls (labelled B1 and B2) and one white ball. A
	second urn contains one black ball and two white balls (labelled W1 and W2).
	Suppose the following experiment is performed. One of the two urns is chosen
	at random. Next a ball is randomly chosen from the urn. Then a second ball is
	chosen at random from the same urn without replacing the first ball.
	
	\begin{enumerate}
	\item What is the probability that two black balls are chosen?
	
	\item What is the probability that two balls of opposite colour are chosen?
	\end{enumerate}
	\solution
	%\begin{align}
    \label{eq:12.13.6.18.1}
	\because	\pr{A|B} &> \pr{A},\
\frac{\pr{AB}}{\pr{B}} > \pr{A}
\\
    \label{eq:12.13.6.18.2}
	\implies \pr{AB} &> \pr{A}\pr{B}
	\\
	\text{or, } \frac{\pr{AB}}{\pr{A}} &=\pr{B|A} > \pr{A}
\end{align}

\end{enumerate}

		%
\item 
Two cards are drawn at random and without replacement from a pack of 52 playing cards. Find the probability that both the cards are black.
\\
\solution
		%\begin{enumerate}[label=\thesection.\arabic*,ref=\thesection.\theenumi]
	\item One card is drawn from a well-shuffled deck of 52 cards. Find the probability of getting
\begin{enumerate}
\item A king of red colour 
\item A face card 
\item A red face card
\item The jack of hearts
\item A spade
\item The queen of diamonds

\end{enumerate}
\solution
		%\begin{table}[H]
	\centering
\begin{tabular}{|c|c|c|}
\hline
Random variable &Value &Definition\\ \hline
\multirow{3}{*}{X} &0 &Slips of Rs 1\\
&1 &Slips of Rs 5\\
&2 &Slips of Rs 13\\ \hline
\multirow{2}{*}{Y} &0 &Box A\\
&1 &Box B\\\hline
\end{tabular}
\caption{}
\label{tab:Distribution}
\end{table}
See \tabref{tab:Distribution}.
\begin{align}
p_{Y}\brak{k}= \begin{cases} 
      \frac{1}{3} & {k=0} \\
      \frac{2}{3 }& {k=1} 
   \end{cases}
   \\
p_{Y|X}\brak{0|0} = \frac{19}{25}\, 
p_{Y|X}\brak{0|1} = \frac{6}{25}\,
p_{Y|X}\brak{1|0} = \frac{45}{50}\,
p_{Y|X}\brak{1|2} = \frac{5}{50}
\end{align}
The desired probability is the probability that a slip drawn at random is marked other than Rs 1,
\begin{align}
&=1-p_X\brak{0}\\
&= p_X(1) + p_X(2)
\end{align}
Using Bayes theorem,
\begin{align}
&= p_Y\brak{0} \times \pr{Y=0 | X=1} + p_Y\brak{1} \times \pr{Y=1|X=2}\\
&=\frac{1}{3} \times \frac{6}{25} + \frac{2}{3} \times \frac{5}{50}\\
&=\frac{11}{75}
\end{align}

\newpage

%\tableofcontents

\bigskip

\renewcommand{\thefigure}{\theenumi}
\renewcommand{\thetable}{\theenumi}
%\renewcommand{\theequation}{\theenumi}

%\begin{abstract}
%%\boldmath
%In this letter, an algorithm for evaluating the exact analytical bit error rate  (BER)  for the piecewise linear (PL) combiner for  multiple relays is presented. Previous results were available only for upto three relays. The algorithm is unique in the sense that  the actual mathematical expressions, that are prohibitively large, need not be explicitly obtained. The diversity gain due to multiple relays is shown through plots of the analytical BER, well supported by simulations. 
%
%\end{abstract}
% IEEEtran.cls defaults to using nonbold math in the Abstract.
% This preserves the distinction between vectors and scalars. However,
% if the journal you are submitting to favors bold math in the abstract,
% then you can use LaTeX's standard command \boldmath at the very start
% of the abstract to achieve this. Many IEEE journals frown on math
% in the abstract anyway.

% Note that keywords are not normally used for peerreview papers.
%\begin{IEEEkeywords}
%Cooperative diversity, decode and forward, piecewise linear
%\end{IEEEkeywords}



% For peer review papers, you can put extra information on the cover
% page as needed:
% \ifCLASSOPTIONpeerreview
% \begin{center} \bfseries EDICS Category: 3-BBND \end{center}
% \fi
%
% For peerreview papers, this IEEEtran command inserts a page break and
% creates the second title. It will be ignored for other modes.
%\IEEEpeerreviewmaketitle




	\item Five cards—the ten, jack, queen, king and ace of diamonds, are well-shuffled with their face downwards. One card is then picked up at random.
\begin{enumerate}
\item
What is the probability that the card is the queen? 
\item
If the queen is drawn and put aside, what is the probability that the second card picked up is (a) an ace? (b) a queen?\\
\end{enumerate}
\solution
		%\begin{enumerate}[label=\thesection.\arabic*,ref=\thesection.\theenumi]
	\item One card is drawn from a well-shuffled deck of 52 cards. Find the probability of getting
\begin{enumerate}
\item A king of red colour 
\item A face card 
\item A red face card
\item The jack of hearts
\item A spade
\item The queen of diamonds

\end{enumerate}
\solution
		%\input{ncert/10/15/1/14/main.tex}
	\item Five cards—the ten, jack, queen, king and ace of diamonds, are well-shuffled with their face downwards. One card is then picked up at random.
\begin{enumerate}
\item
What is the probability that the card is the queen? 
\item
If the queen is drawn and put aside, what is the probability that the second card picked up is (a) an ace? (b) a queen?\\
\end{enumerate}
\solution
		%\input{ncert/10/15/1/15/defs.tex}
	\item A bag contains $5$ red balls and some blue balls. If the probability of drawing a blue ball is double that if a red ball, determine the number of blue balls in the bag. 
		\\
\solution
		%\input{ncert/10/15/2/3/defs.tex}
	\item A card is selected from a pack of 52 cards.
 \begin{enumerate}[label=(\alph*)] 
                 \item How many points are there in the sample space?
                 \item Calculate the probability that the card is an ace of spades.
                 \item Calculate the probability that the card is (i) an ace and (ii) black card.
 \end{enumerate}
\solution
		%\input{ncert/11/16/3/4/main.tex}
\item Four cards are drawn from a well-shuffled deck of 52 cards. What is the probability of obtaining 3 diamonds and one spade.
\\
\solution
		%\input{ncert/11/16/4/2/defs.tex}
\item In a certain lottery 10,000 tickets are sold and ten equal prizes are awarded. What is the probability of not getting a prize if you buy (a) one ticket (b) two tickets (c) 10 tickets ?	
\\
\solution
		%\input{ncert/11/16/4/4/defs.tex}
		%
\item 
Out of 100 students, two sections of 40 and 60 are formed. If you and your friend are among the 100 students, what is the probability that
\begin{enumerate}
\item you both enter the same section?
\item you both enter the different sections?
\end{enumerate}
\solution
		%\input{ncert/11/16/4/5/defs.tex}
	\item 
The number lock of a suitcase has 4 wheels each labelled with ten digits i.e. from 0 to 9.The lock opens with a sequence of four digits with no repeats.What is the probability of a person getting the right sequence to open the suitcase.
\\
\solution
		%\input{ncert/11/16/4/10/defs.tex}
		%
\item 
Two cards are drawn at random and without replacement from a pack of 52 playing cards. Find the probability that both the cards are black.
\\
\solution
		%\input{ncert/12/13/2/2/defs.tex}
		\item A box of oranges is inspected by examining three randomly selected oranges drawn without replacement. If all the three oranges are good, the box is approved for sale, otherwise, it is rejected. Find the probability that a box containing 15 oranges out of which 12 are good and 3 are bad ones will be approved for sale.
		\label{ncert/12/13/2/3/defs.tex}
		\item Two balls are drawn at random with replacement from a box containing 10 black and 8 red balls. Find the probability that
		\label{ncert/12/13/2/12}
\begin{enumerate}
\item both balls are red.
\item first ball is black and second is red.
\item one of them is black and other is red.
\end{enumerate}

\item In a hostel, 60\% of the students read Hindi newspaper, 40\% read English newspaper and 20\% read both Hindi and English newspapers. A student is selected at random.
		\label{ncert/12/13/2/15}
\begin{enumerate}
\item Find the probability that she reads neither Hindi nor English newspapers.
\item If she reads Hindi newspaper, find the probability that she reads English newspaper.
\item If she reads English newspaper, find the probability that she reads Hindi newspaper.\\
\end{enumerate}
\item The probability of obtaining an even prime number on each die, when a pair of dice is rolled is 
\begin{enumerate}
    \item $0$ 
    
    \item $\frac{1}{3}$ 
    
    \item $\frac{1}{12}$ 
    
    \item $\frac{1}{36}$ 
\end{enumerate}
\solution
		%\input{ncert/12/13/2/17/defs.tex}
	\item A bag contains 4 red and 4 black balls, another bag contains 2 red and 6 black balls. One of the two bags is selected at random and a ball is drawn from the bag which is found to be red. Find the probability that the ball is drawn from the first bag.
\\
\solution
		%\input{ncert/12/13/3/2/main.tex}
  \item
  Cards with numbers 2 to 101 are placed in a box. A card is selected at random.Find the probability that the card has
\begin{enumerate}[label=(\roman*)]
	\item an even number 
	\item a square number
\end{enumerate}
\solution
%\input{exemplar/10/13/3/32/main.tex}
\item
The king, queen and jack of clubs are removed from a deck of 52 playing cards and then well shuffled. Now one card is drawn at random from the remaining cards.  Determine the probability that the card is
\begin{enumerate}[label=(\roman*)]
\item a club
\item 10 of hearts
\end{enumerate}
\solution
%\input{exemplar/10/13/3/29/main.tex}
\item A team of medical students doing their internship have to assist during surgeries
at a city hospital. The probabilities of surgeries rated as very complex, complex,
routine, simple or very simple are respectively, 0.15, 0.20, 0.31, 0.26, .08. Find
the probabilities that a particular surgery will be rated
\begin{enumerate}
	\item complex or very complex;
	\item neither very complex nor very simple;
	\item routine or complex
	\item routine or simple
\end{enumerate}
\solution
%\input{exemplar/11/16/3/8(1)/main.tex}
\item A card is selected from a pack of 52 cards.
\begin{enumerate}[label=(\alph*)]
    \item How many points are there in the sample space?
    \item Calculate the probability that the card is an ace of spades.
    \item Calculate the probability that the card is (i) an ace and (ii) black card.
\end{enumerate}
\solution
%\input{exemplar/11/16/3/4/main2.tex}
\item The probability that a non leap year selected at random will contain 53 sundays.
\\
\solution
%\input{exemplar/10/13/1/19/main.tex}
\item One of the four persons John, Rita, Aslam or Gurpreet will be promoted next
month. Consequently the sample space consists of four elementary outcomes
S = {John promoted, Rita promoted, Aslam promoted, Gurpreet promoted}
You are told that the chances of John’s promotion is same as that of Gurpreet,
Rita’s chances of promotion are twice as likely as Johns. Aslam’s chances are
four times that of John.
\begin{enumerate}
	\item Determine
	\begin{enumerate}
		\item P (John promoted)
		\item P (Rita promoted)
		\item P (Aslam promoted)
		\item P (Gurpreet promoted)
	\end{enumerate}
	\item If A = {John promoted or Gurpreet promoted}, find P (A).
\end{enumerate}
\solution
%\input{exemplar/11/16/3/10/main.tex}
\item A card is drawn from a deck of 52 cards. Find the probability of getting a king or a heart or a red card.\\
\solution
%\input{exemplar/11/16/3/15/main.tex}
\item The probability that a student will pass his examination is 0.73, the probability of
the student getting a compartment is 0.13, and the probability that the student will
either pass or get compartment is 0.96. State True or False.\\
\solution
%\input{exemplar/11/16/3/31/main.tex}
\item A card is selected from a pack of 52 cards\\
\begin{enumerate}[label=(\alph*)]
\item How many points are there in the sample space?
\item Calculate the probability that the cards is an ace of spades.
\item Calculate the probability that the card is (i) an ace (ii)black card.\\
\end{enumerate}
%\input{ncert/11/16/3/4_1/Prob_4.tex}
\item In a non-leap year, the probability of having 53 tuesdays or 53 wednesdays is\\
\solution
%\input{exemplar/11/16/3/18/main.tex}
\item There are 1000 sealed envelopes in a box, 10 of them contain a cash prize of
Rs 100 each, 100 of them contain a cash prize of Rs 50 each and 200 of them
contain a cash prize of Rs 10 each and rest do not contain any cash prize. If they
are well shuffled and an envelope is picked up out, what is the probability that it
contains no cash prize?\\
\solution
%\input{exemplar/10/13/3/34/main.tex}
\item 
A die is thrown and a card is selected at random from a deck of 52 playing cards. The probability of getting an even number on the die and a spade card.\\
\solution
%\input{exemplar/12/13/3/78/main.tex}
\item
If 4-digit numbers greater than 5,000 are randomly formed from the digits 0, 1, 3, 5, and 7, what is the probability of forming a number divisible by 5 when:
\begin{enumerate}
    \item The digits are repeated?
    \item The repetition of digits is not allowed?
\end{enumerate}
\solution
%\input{ncert/11/16/4/9/main.tex}
\item Consider the probability space $\brak{\Omega, \mathcal{G}, P}$ where $\Omega = [0,2]$ and $\mathcal{G} = \cbrak{\phi, \Omega, [0,1], (1,2]}$. Let $X$ and $Y$ be two functions on $\Omega$ defined as
\begin{align*}
    X(\omega) = 
    \begin{cases}
        1 & \text{if }\omega \in [0, 1]\\
        2 & \text{if }\omega \in (1, 2]
    \end{cases}
\end{align*}
and
\begin{align*}
    Y(\omega) = 
    \begin{cases}
        2 & \text{if }\omega \in [0, 1.5]\\
        3 & \text{if }\omega \in (1.5, 2].
    \end{cases}
\end{align*}
Then which one of the following statements is true?
\begin{enumerate}
    \item [(A)] $X$ is a random variable with respect to $\mathcal{G}$, but $Y$ is not a random variable with respect to $\mathcal{G}$.
    \item [(B)] $Y$ is a random variable with respect to $\mathcal{G}$, but $X$ is not a random variable with respect to $\mathcal{G}$.
    \item [(C)] Neither $X$ nor $Y$ is a random variable with respect to $\mathcal{G}$.
    \item [(D)] Both $X$ and $Y$ are random variables with respect to $\mathcal{G}$.
\end{enumerate} \hfill (GATE ST 2023)\\
\solution
%\input{gate/ST/2023/14/main.tex}
	\item  A die is loaded in such a way that each odd number is twice as likely to occur as
each even number. Find $P(G)$, where $G$ is the event that a number greater than
3 occurs on a single roll of the die.
\\
\solution
		%\input{exemplar/11/16/3/5/main.tex}
	\item All the jacks, queens and kings are removed from a deck of 52 playing cards. The remaining cards are well shuffled and then one card is drawn at random. Giving ace a value 1 similar value for other cards, find the probability that the card has a value 
		\begin{enumerate}
			\item 7
			\item greater than 7
			\item less than 7
		\end{enumerate}
		%\input{exemplar/10/13/3/30/main.tex}
  \item A Lot consists of 48 mobile phones of which 42 are good, 3 have only minor defects and 3 have major defects.Varnika will buy a phone if it is good but the trader will only buy a mobile if it has no major defects. One phone is selected at random from the lot. What is the probability that it is
\begin{enumerate}
	\item acceptable to Varnika?
            \item acceptable to the trader?
\end{enumerate}
\solution
	%\input{exemplar/10/13/3/40/main.tex}
 \item A student says that if you throw a die, it will show up 1 or not 1. Therefore, the probability of getting 1 and the probability of getting 'not 1' each is equal to $\frac{1}{2}$. Is this correct? Give reasons.\\
 \solution
        %\input{exemplar/10/13/2/9/main.tex}
   \item Four candidates A, B, C, D have ap-
plied for the assignment to coach a school cricket
team. If A is twice as likely to be selected as B, and
B and C are given about the same chance of being
selected, while C is twice as likely to be selected
as D, what are the probabilities that
\begin{enumerate}
\item C will be selected?
\item A will not be selected?
\end{enumerate}
	%\input{exemplar/11/16/3/9/main.tex}
 \item A bag contain 24 balls of which $x$ balls are red, $2x$ are white and $3x$ are blue. A ball is selected at random, What is the probability that it is
\begin{enumerate}[label=\alph*)]
\item not red ?
\item white ?
\end{enumerate}
%\input{exemplar/10/13/3/41/main.tex}
If the letters of the word ASSASSINATION are arranged at random. Find the Probability that
\begin{enumerate}[label=(\alph*)]
\item Four $S's$ come consecutively in the word
\item Two  $I's$ and two $N's$ come together
\item All $A's$ are not coming together
\item No two $A's$ are coming together
\end{enumerate}
%\input{exemplar/11/16/3/14/main.tex}
	\item One urn contains two black balls (labelled B1 and B2) and one white ball. A
	second urn contains one black ball and two white balls (labelled W1 and W2).
	Suppose the following experiment is performed. One of the two urns is chosen
	at random. Next a ball is randomly chosen from the urn. Then a second ball is
	chosen at random from the same urn without replacing the first ball.
	
	\begin{enumerate}
	\item What is the probability that two black balls are chosen?
	
	\item What is the probability that two balls of opposite colour are chosen?
	\end{enumerate}
	\solution
	%\input{exemplar/11/16/3/12/main1.tex}
\end{enumerate}

	\item A bag contains $5$ red balls and some blue balls. If the probability of drawing a blue ball is double that if a red ball, determine the number of blue balls in the bag. 
		\\
\solution
		%\begin{enumerate}[label=\thesection.\arabic*,ref=\thesection.\theenumi]
	\item One card is drawn from a well-shuffled deck of 52 cards. Find the probability of getting
\begin{enumerate}
\item A king of red colour 
\item A face card 
\item A red face card
\item The jack of hearts
\item A spade
\item The queen of diamonds

\end{enumerate}
\solution
		%\input{ncert/10/15/1/14/main.tex}
	\item Five cards—the ten, jack, queen, king and ace of diamonds, are well-shuffled with their face downwards. One card is then picked up at random.
\begin{enumerate}
\item
What is the probability that the card is the queen? 
\item
If the queen is drawn and put aside, what is the probability that the second card picked up is (a) an ace? (b) a queen?\\
\end{enumerate}
\solution
		%\input{ncert/10/15/1/15/defs.tex}
	\item A bag contains $5$ red balls and some blue balls. If the probability of drawing a blue ball is double that if a red ball, determine the number of blue balls in the bag. 
		\\
\solution
		%\input{ncert/10/15/2/3/defs.tex}
	\item A card is selected from a pack of 52 cards.
 \begin{enumerate}[label=(\alph*)] 
                 \item How many points are there in the sample space?
                 \item Calculate the probability that the card is an ace of spades.
                 \item Calculate the probability that the card is (i) an ace and (ii) black card.
 \end{enumerate}
\solution
		%\input{ncert/11/16/3/4/main.tex}
\item Four cards are drawn from a well-shuffled deck of 52 cards. What is the probability of obtaining 3 diamonds and one spade.
\\
\solution
		%\input{ncert/11/16/4/2/defs.tex}
\item In a certain lottery 10,000 tickets are sold and ten equal prizes are awarded. What is the probability of not getting a prize if you buy (a) one ticket (b) two tickets (c) 10 tickets ?	
\\
\solution
		%\input{ncert/11/16/4/4/defs.tex}
		%
\item 
Out of 100 students, two sections of 40 and 60 are formed. If you and your friend are among the 100 students, what is the probability that
\begin{enumerate}
\item you both enter the same section?
\item you both enter the different sections?
\end{enumerate}
\solution
		%\input{ncert/11/16/4/5/defs.tex}
	\item 
The number lock of a suitcase has 4 wheels each labelled with ten digits i.e. from 0 to 9.The lock opens with a sequence of four digits with no repeats.What is the probability of a person getting the right sequence to open the suitcase.
\\
\solution
		%\input{ncert/11/16/4/10/defs.tex}
		%
\item 
Two cards are drawn at random and without replacement from a pack of 52 playing cards. Find the probability that both the cards are black.
\\
\solution
		%\input{ncert/12/13/2/2/defs.tex}
		\item A box of oranges is inspected by examining three randomly selected oranges drawn without replacement. If all the three oranges are good, the box is approved for sale, otherwise, it is rejected. Find the probability that a box containing 15 oranges out of which 12 are good and 3 are bad ones will be approved for sale.
		\label{ncert/12/13/2/3/defs.tex}
		\item Two balls are drawn at random with replacement from a box containing 10 black and 8 red balls. Find the probability that
		\label{ncert/12/13/2/12}
\begin{enumerate}
\item both balls are red.
\item first ball is black and second is red.
\item one of them is black and other is red.
\end{enumerate}

\item In a hostel, 60\% of the students read Hindi newspaper, 40\% read English newspaper and 20\% read both Hindi and English newspapers. A student is selected at random.
		\label{ncert/12/13/2/15}
\begin{enumerate}
\item Find the probability that she reads neither Hindi nor English newspapers.
\item If she reads Hindi newspaper, find the probability that she reads English newspaper.
\item If she reads English newspaper, find the probability that she reads Hindi newspaper.\\
\end{enumerate}
\item The probability of obtaining an even prime number on each die, when a pair of dice is rolled is 
\begin{enumerate}
    \item $0$ 
    
    \item $\frac{1}{3}$ 
    
    \item $\frac{1}{12}$ 
    
    \item $\frac{1}{36}$ 
\end{enumerate}
\solution
		%\input{ncert/12/13/2/17/defs.tex}
	\item A bag contains 4 red and 4 black balls, another bag contains 2 red and 6 black balls. One of the two bags is selected at random and a ball is drawn from the bag which is found to be red. Find the probability that the ball is drawn from the first bag.
\\
\solution
		%\input{ncert/12/13/3/2/main.tex}
  \item
  Cards with numbers 2 to 101 are placed in a box. A card is selected at random.Find the probability that the card has
\begin{enumerate}[label=(\roman*)]
	\item an even number 
	\item a square number
\end{enumerate}
\solution
%\input{exemplar/10/13/3/32/main.tex}
\item
The king, queen and jack of clubs are removed from a deck of 52 playing cards and then well shuffled. Now one card is drawn at random from the remaining cards.  Determine the probability that the card is
\begin{enumerate}[label=(\roman*)]
\item a club
\item 10 of hearts
\end{enumerate}
\solution
%\input{exemplar/10/13/3/29/main.tex}
\item A team of medical students doing their internship have to assist during surgeries
at a city hospital. The probabilities of surgeries rated as very complex, complex,
routine, simple or very simple are respectively, 0.15, 0.20, 0.31, 0.26, .08. Find
the probabilities that a particular surgery will be rated
\begin{enumerate}
	\item complex or very complex;
	\item neither very complex nor very simple;
	\item routine or complex
	\item routine or simple
\end{enumerate}
\solution
%\input{exemplar/11/16/3/8(1)/main.tex}
\item A card is selected from a pack of 52 cards.
\begin{enumerate}[label=(\alph*)]
    \item How many points are there in the sample space?
    \item Calculate the probability that the card is an ace of spades.
    \item Calculate the probability that the card is (i) an ace and (ii) black card.
\end{enumerate}
\solution
%\input{exemplar/11/16/3/4/main2.tex}
\item The probability that a non leap year selected at random will contain 53 sundays.
\\
\solution
%\input{exemplar/10/13/1/19/main.tex}
\item One of the four persons John, Rita, Aslam or Gurpreet will be promoted next
month. Consequently the sample space consists of four elementary outcomes
S = {John promoted, Rita promoted, Aslam promoted, Gurpreet promoted}
You are told that the chances of John’s promotion is same as that of Gurpreet,
Rita’s chances of promotion are twice as likely as Johns. Aslam’s chances are
four times that of John.
\begin{enumerate}
	\item Determine
	\begin{enumerate}
		\item P (John promoted)
		\item P (Rita promoted)
		\item P (Aslam promoted)
		\item P (Gurpreet promoted)
	\end{enumerate}
	\item If A = {John promoted or Gurpreet promoted}, find P (A).
\end{enumerate}
\solution
%\input{exemplar/11/16/3/10/main.tex}
\item A card is drawn from a deck of 52 cards. Find the probability of getting a king or a heart or a red card.\\
\solution
%\input{exemplar/11/16/3/15/main.tex}
\item The probability that a student will pass his examination is 0.73, the probability of
the student getting a compartment is 0.13, and the probability that the student will
either pass or get compartment is 0.96. State True or False.\\
\solution
%\input{exemplar/11/16/3/31/main.tex}
\item A card is selected from a pack of 52 cards\\
\begin{enumerate}[label=(\alph*)]
\item How many points are there in the sample space?
\item Calculate the probability that the cards is an ace of spades.
\item Calculate the probability that the card is (i) an ace (ii)black card.\\
\end{enumerate}
%\input{ncert/11/16/3/4_1/Prob_4.tex}
\item In a non-leap year, the probability of having 53 tuesdays or 53 wednesdays is\\
\solution
%\input{exemplar/11/16/3/18/main.tex}
\item There are 1000 sealed envelopes in a box, 10 of them contain a cash prize of
Rs 100 each, 100 of them contain a cash prize of Rs 50 each and 200 of them
contain a cash prize of Rs 10 each and rest do not contain any cash prize. If they
are well shuffled and an envelope is picked up out, what is the probability that it
contains no cash prize?\\
\solution
%\input{exemplar/10/13/3/34/main.tex}
\item 
A die is thrown and a card is selected at random from a deck of 52 playing cards. The probability of getting an even number on the die and a spade card.\\
\solution
%\input{exemplar/12/13/3/78/main.tex}
\item
If 4-digit numbers greater than 5,000 are randomly formed from the digits 0, 1, 3, 5, and 7, what is the probability of forming a number divisible by 5 when:
\begin{enumerate}
    \item The digits are repeated?
    \item The repetition of digits is not allowed?
\end{enumerate}
\solution
%\input{ncert/11/16/4/9/main.tex}
\item Consider the probability space $\brak{\Omega, \mathcal{G}, P}$ where $\Omega = [0,2]$ and $\mathcal{G} = \cbrak{\phi, \Omega, [0,1], (1,2]}$. Let $X$ and $Y$ be two functions on $\Omega$ defined as
\begin{align*}
    X(\omega) = 
    \begin{cases}
        1 & \text{if }\omega \in [0, 1]\\
        2 & \text{if }\omega \in (1, 2]
    \end{cases}
\end{align*}
and
\begin{align*}
    Y(\omega) = 
    \begin{cases}
        2 & \text{if }\omega \in [0, 1.5]\\
        3 & \text{if }\omega \in (1.5, 2].
    \end{cases}
\end{align*}
Then which one of the following statements is true?
\begin{enumerate}
    \item [(A)] $X$ is a random variable with respect to $\mathcal{G}$, but $Y$ is not a random variable with respect to $\mathcal{G}$.
    \item [(B)] $Y$ is a random variable with respect to $\mathcal{G}$, but $X$ is not a random variable with respect to $\mathcal{G}$.
    \item [(C)] Neither $X$ nor $Y$ is a random variable with respect to $\mathcal{G}$.
    \item [(D)] Both $X$ and $Y$ are random variables with respect to $\mathcal{G}$.
\end{enumerate} \hfill (GATE ST 2023)\\
\solution
%\input{gate/ST/2023/14/main.tex}
	\item  A die is loaded in such a way that each odd number is twice as likely to occur as
each even number. Find $P(G)$, where $G$ is the event that a number greater than
3 occurs on a single roll of the die.
\\
\solution
		%\input{exemplar/11/16/3/5/main.tex}
	\item All the jacks, queens and kings are removed from a deck of 52 playing cards. The remaining cards are well shuffled and then one card is drawn at random. Giving ace a value 1 similar value for other cards, find the probability that the card has a value 
		\begin{enumerate}
			\item 7
			\item greater than 7
			\item less than 7
		\end{enumerate}
		%\input{exemplar/10/13/3/30/main.tex}
  \item A Lot consists of 48 mobile phones of which 42 are good, 3 have only minor defects and 3 have major defects.Varnika will buy a phone if it is good but the trader will only buy a mobile if it has no major defects. One phone is selected at random from the lot. What is the probability that it is
\begin{enumerate}
	\item acceptable to Varnika?
            \item acceptable to the trader?
\end{enumerate}
\solution
	%\input{exemplar/10/13/3/40/main.tex}
 \item A student says that if you throw a die, it will show up 1 or not 1. Therefore, the probability of getting 1 and the probability of getting 'not 1' each is equal to $\frac{1}{2}$. Is this correct? Give reasons.\\
 \solution
        %\input{exemplar/10/13/2/9/main.tex}
   \item Four candidates A, B, C, D have ap-
plied for the assignment to coach a school cricket
team. If A is twice as likely to be selected as B, and
B and C are given about the same chance of being
selected, while C is twice as likely to be selected
as D, what are the probabilities that
\begin{enumerate}
\item C will be selected?
\item A will not be selected?
\end{enumerate}
	%\input{exemplar/11/16/3/9/main.tex}
 \item A bag contain 24 balls of which $x$ balls are red, $2x$ are white and $3x$ are blue. A ball is selected at random, What is the probability that it is
\begin{enumerate}[label=\alph*)]
\item not red ?
\item white ?
\end{enumerate}
%\input{exemplar/10/13/3/41/main.tex}
If the letters of the word ASSASSINATION are arranged at random. Find the Probability that
\begin{enumerate}[label=(\alph*)]
\item Four $S's$ come consecutively in the word
\item Two  $I's$ and two $N's$ come together
\item All $A's$ are not coming together
\item No two $A's$ are coming together
\end{enumerate}
%\input{exemplar/11/16/3/14/main.tex}
	\item One urn contains two black balls (labelled B1 and B2) and one white ball. A
	second urn contains one black ball and two white balls (labelled W1 and W2).
	Suppose the following experiment is performed. One of the two urns is chosen
	at random. Next a ball is randomly chosen from the urn. Then a second ball is
	chosen at random from the same urn without replacing the first ball.
	
	\begin{enumerate}
	\item What is the probability that two black balls are chosen?
	
	\item What is the probability that two balls of opposite colour are chosen?
	\end{enumerate}
	\solution
	%\input{exemplar/11/16/3/12/main1.tex}
\end{enumerate}

	\item A card is selected from a pack of 52 cards.
 \begin{enumerate}[label=(\alph*)] 
                 \item How many points are there in the sample space?
                 \item Calculate the probability that the card is an ace of spades.
                 \item Calculate the probability that the card is (i) an ace and (ii) black card.
 \end{enumerate}
\solution
		%\begin{table}[H]
	\centering
\begin{tabular}{|c|c|c|}
\hline
Random variable &Value &Definition\\ \hline
\multirow{3}{*}{X} &0 &Slips of Rs 1\\
&1 &Slips of Rs 5\\
&2 &Slips of Rs 13\\ \hline
\multirow{2}{*}{Y} &0 &Box A\\
&1 &Box B\\\hline
\end{tabular}
\caption{}
\label{tab:Distribution}
\end{table}
See \tabref{tab:Distribution}.
\begin{align}
p_{Y}\brak{k}= \begin{cases} 
      \frac{1}{3} & {k=0} \\
      \frac{2}{3 }& {k=1} 
   \end{cases}
   \\
p_{Y|X}\brak{0|0} = \frac{19}{25}\, 
p_{Y|X}\brak{0|1} = \frac{6}{25}\,
p_{Y|X}\brak{1|0} = \frac{45}{50}\,
p_{Y|X}\brak{1|2} = \frac{5}{50}
\end{align}
The desired probability is the probability that a slip drawn at random is marked other than Rs 1,
\begin{align}
&=1-p_X\brak{0}\\
&= p_X(1) + p_X(2)
\end{align}
Using Bayes theorem,
\begin{align}
&= p_Y\brak{0} \times \pr{Y=0 | X=1} + p_Y\brak{1} \times \pr{Y=1|X=2}\\
&=\frac{1}{3} \times \frac{6}{25} + \frac{2}{3} \times \frac{5}{50}\\
&=\frac{11}{75}
\end{align}

\newpage

%\tableofcontents

\bigskip

\renewcommand{\thefigure}{\theenumi}
\renewcommand{\thetable}{\theenumi}
%\renewcommand{\theequation}{\theenumi}

%\begin{abstract}
%%\boldmath
%In this letter, an algorithm for evaluating the exact analytical bit error rate  (BER)  for the piecewise linear (PL) combiner for  multiple relays is presented. Previous results were available only for upto three relays. The algorithm is unique in the sense that  the actual mathematical expressions, that are prohibitively large, need not be explicitly obtained. The diversity gain due to multiple relays is shown through plots of the analytical BER, well supported by simulations. 
%
%\end{abstract}
% IEEEtran.cls defaults to using nonbold math in the Abstract.
% This preserves the distinction between vectors and scalars. However,
% if the journal you are submitting to favors bold math in the abstract,
% then you can use LaTeX's standard command \boldmath at the very start
% of the abstract to achieve this. Many IEEE journals frown on math
% in the abstract anyway.

% Note that keywords are not normally used for peerreview papers.
%\begin{IEEEkeywords}
%Cooperative diversity, decode and forward, piecewise linear
%\end{IEEEkeywords}



% For peer review papers, you can put extra information on the cover
% page as needed:
% \ifCLASSOPTIONpeerreview
% \begin{center} \bfseries EDICS Category: 3-BBND \end{center}
% \fi
%
% For peerreview papers, this IEEEtran command inserts a page break and
% creates the second title. It will be ignored for other modes.
%\IEEEpeerreviewmaketitle




\item Four cards are drawn from a well-shuffled deck of 52 cards. What is the probability of obtaining 3 diamonds and one spade.
\\
\solution
		%\begin{enumerate}[label=\thesection.\arabic*,ref=\thesection.\theenumi]
	\item One card is drawn from a well-shuffled deck of 52 cards. Find the probability of getting
\begin{enumerate}
\item A king of red colour 
\item A face card 
\item A red face card
\item The jack of hearts
\item A spade
\item The queen of diamonds

\end{enumerate}
\solution
		%\input{ncert/10/15/1/14/main.tex}
	\item Five cards—the ten, jack, queen, king and ace of diamonds, are well-shuffled with their face downwards. One card is then picked up at random.
\begin{enumerate}
\item
What is the probability that the card is the queen? 
\item
If the queen is drawn and put aside, what is the probability that the second card picked up is (a) an ace? (b) a queen?\\
\end{enumerate}
\solution
		%\input{ncert/10/15/1/15/defs.tex}
	\item A bag contains $5$ red balls and some blue balls. If the probability of drawing a blue ball is double that if a red ball, determine the number of blue balls in the bag. 
		\\
\solution
		%\input{ncert/10/15/2/3/defs.tex}
	\item A card is selected from a pack of 52 cards.
 \begin{enumerate}[label=(\alph*)] 
                 \item How many points are there in the sample space?
                 \item Calculate the probability that the card is an ace of spades.
                 \item Calculate the probability that the card is (i) an ace and (ii) black card.
 \end{enumerate}
\solution
		%\input{ncert/11/16/3/4/main.tex}
\item Four cards are drawn from a well-shuffled deck of 52 cards. What is the probability of obtaining 3 diamonds and one spade.
\\
\solution
		%\input{ncert/11/16/4/2/defs.tex}
\item In a certain lottery 10,000 tickets are sold and ten equal prizes are awarded. What is the probability of not getting a prize if you buy (a) one ticket (b) two tickets (c) 10 tickets ?	
\\
\solution
		%\input{ncert/11/16/4/4/defs.tex}
		%
\item 
Out of 100 students, two sections of 40 and 60 are formed. If you and your friend are among the 100 students, what is the probability that
\begin{enumerate}
\item you both enter the same section?
\item you both enter the different sections?
\end{enumerate}
\solution
		%\input{ncert/11/16/4/5/defs.tex}
	\item 
The number lock of a suitcase has 4 wheels each labelled with ten digits i.e. from 0 to 9.The lock opens with a sequence of four digits with no repeats.What is the probability of a person getting the right sequence to open the suitcase.
\\
\solution
		%\input{ncert/11/16/4/10/defs.tex}
		%
\item 
Two cards are drawn at random and without replacement from a pack of 52 playing cards. Find the probability that both the cards are black.
\\
\solution
		%\input{ncert/12/13/2/2/defs.tex}
		\item A box of oranges is inspected by examining three randomly selected oranges drawn without replacement. If all the three oranges are good, the box is approved for sale, otherwise, it is rejected. Find the probability that a box containing 15 oranges out of which 12 are good and 3 are bad ones will be approved for sale.
		\label{ncert/12/13/2/3/defs.tex}
		\item Two balls are drawn at random with replacement from a box containing 10 black and 8 red balls. Find the probability that
		\label{ncert/12/13/2/12}
\begin{enumerate}
\item both balls are red.
\item first ball is black and second is red.
\item one of them is black and other is red.
\end{enumerate}

\item In a hostel, 60\% of the students read Hindi newspaper, 40\% read English newspaper and 20\% read both Hindi and English newspapers. A student is selected at random.
		\label{ncert/12/13/2/15}
\begin{enumerate}
\item Find the probability that she reads neither Hindi nor English newspapers.
\item If she reads Hindi newspaper, find the probability that she reads English newspaper.
\item If she reads English newspaper, find the probability that she reads Hindi newspaper.\\
\end{enumerate}
\item The probability of obtaining an even prime number on each die, when a pair of dice is rolled is 
\begin{enumerate}
    \item $0$ 
    
    \item $\frac{1}{3}$ 
    
    \item $\frac{1}{12}$ 
    
    \item $\frac{1}{36}$ 
\end{enumerate}
\solution
		%\input{ncert/12/13/2/17/defs.tex}
	\item A bag contains 4 red and 4 black balls, another bag contains 2 red and 6 black balls. One of the two bags is selected at random and a ball is drawn from the bag which is found to be red. Find the probability that the ball is drawn from the first bag.
\\
\solution
		%\input{ncert/12/13/3/2/main.tex}
  \item
  Cards with numbers 2 to 101 are placed in a box. A card is selected at random.Find the probability that the card has
\begin{enumerate}[label=(\roman*)]
	\item an even number 
	\item a square number
\end{enumerate}
\solution
%\input{exemplar/10/13/3/32/main.tex}
\item
The king, queen and jack of clubs are removed from a deck of 52 playing cards and then well shuffled. Now one card is drawn at random from the remaining cards.  Determine the probability that the card is
\begin{enumerate}[label=(\roman*)]
\item a club
\item 10 of hearts
\end{enumerate}
\solution
%\input{exemplar/10/13/3/29/main.tex}
\item A team of medical students doing their internship have to assist during surgeries
at a city hospital. The probabilities of surgeries rated as very complex, complex,
routine, simple or very simple are respectively, 0.15, 0.20, 0.31, 0.26, .08. Find
the probabilities that a particular surgery will be rated
\begin{enumerate}
	\item complex or very complex;
	\item neither very complex nor very simple;
	\item routine or complex
	\item routine or simple
\end{enumerate}
\solution
%\input{exemplar/11/16/3/8(1)/main.tex}
\item A card is selected from a pack of 52 cards.
\begin{enumerate}[label=(\alph*)]
    \item How many points are there in the sample space?
    \item Calculate the probability that the card is an ace of spades.
    \item Calculate the probability that the card is (i) an ace and (ii) black card.
\end{enumerate}
\solution
%\input{exemplar/11/16/3/4/main2.tex}
\item The probability that a non leap year selected at random will contain 53 sundays.
\\
\solution
%\input{exemplar/10/13/1/19/main.tex}
\item One of the four persons John, Rita, Aslam or Gurpreet will be promoted next
month. Consequently the sample space consists of four elementary outcomes
S = {John promoted, Rita promoted, Aslam promoted, Gurpreet promoted}
You are told that the chances of John’s promotion is same as that of Gurpreet,
Rita’s chances of promotion are twice as likely as Johns. Aslam’s chances are
four times that of John.
\begin{enumerate}
	\item Determine
	\begin{enumerate}
		\item P (John promoted)
		\item P (Rita promoted)
		\item P (Aslam promoted)
		\item P (Gurpreet promoted)
	\end{enumerate}
	\item If A = {John promoted or Gurpreet promoted}, find P (A).
\end{enumerate}
\solution
%\input{exemplar/11/16/3/10/main.tex}
\item A card is drawn from a deck of 52 cards. Find the probability of getting a king or a heart or a red card.\\
\solution
%\input{exemplar/11/16/3/15/main.tex}
\item The probability that a student will pass his examination is 0.73, the probability of
the student getting a compartment is 0.13, and the probability that the student will
either pass or get compartment is 0.96. State True or False.\\
\solution
%\input{exemplar/11/16/3/31/main.tex}
\item A card is selected from a pack of 52 cards\\
\begin{enumerate}[label=(\alph*)]
\item How many points are there in the sample space?
\item Calculate the probability that the cards is an ace of spades.
\item Calculate the probability that the card is (i) an ace (ii)black card.\\
\end{enumerate}
%\input{ncert/11/16/3/4_1/Prob_4.tex}
\item In a non-leap year, the probability of having 53 tuesdays or 53 wednesdays is\\
\solution
%\input{exemplar/11/16/3/18/main.tex}
\item There are 1000 sealed envelopes in a box, 10 of them contain a cash prize of
Rs 100 each, 100 of them contain a cash prize of Rs 50 each and 200 of them
contain a cash prize of Rs 10 each and rest do not contain any cash prize. If they
are well shuffled and an envelope is picked up out, what is the probability that it
contains no cash prize?\\
\solution
%\input{exemplar/10/13/3/34/main.tex}
\item 
A die is thrown and a card is selected at random from a deck of 52 playing cards. The probability of getting an even number on the die and a spade card.\\
\solution
%\input{exemplar/12/13/3/78/main.tex}
\item
If 4-digit numbers greater than 5,000 are randomly formed from the digits 0, 1, 3, 5, and 7, what is the probability of forming a number divisible by 5 when:
\begin{enumerate}
    \item The digits are repeated?
    \item The repetition of digits is not allowed?
\end{enumerate}
\solution
%\input{ncert/11/16/4/9/main.tex}
\item Consider the probability space $\brak{\Omega, \mathcal{G}, P}$ where $\Omega = [0,2]$ and $\mathcal{G} = \cbrak{\phi, \Omega, [0,1], (1,2]}$. Let $X$ and $Y$ be two functions on $\Omega$ defined as
\begin{align*}
    X(\omega) = 
    \begin{cases}
        1 & \text{if }\omega \in [0, 1]\\
        2 & \text{if }\omega \in (1, 2]
    \end{cases}
\end{align*}
and
\begin{align*}
    Y(\omega) = 
    \begin{cases}
        2 & \text{if }\omega \in [0, 1.5]\\
        3 & \text{if }\omega \in (1.5, 2].
    \end{cases}
\end{align*}
Then which one of the following statements is true?
\begin{enumerate}
    \item [(A)] $X$ is a random variable with respect to $\mathcal{G}$, but $Y$ is not a random variable with respect to $\mathcal{G}$.
    \item [(B)] $Y$ is a random variable with respect to $\mathcal{G}$, but $X$ is not a random variable with respect to $\mathcal{G}$.
    \item [(C)] Neither $X$ nor $Y$ is a random variable with respect to $\mathcal{G}$.
    \item [(D)] Both $X$ and $Y$ are random variables with respect to $\mathcal{G}$.
\end{enumerate} \hfill (GATE ST 2023)\\
\solution
%\input{gate/ST/2023/14/main.tex}
	\item  A die is loaded in such a way that each odd number is twice as likely to occur as
each even number. Find $P(G)$, where $G$ is the event that a number greater than
3 occurs on a single roll of the die.
\\
\solution
		%\input{exemplar/11/16/3/5/main.tex}
	\item All the jacks, queens and kings are removed from a deck of 52 playing cards. The remaining cards are well shuffled and then one card is drawn at random. Giving ace a value 1 similar value for other cards, find the probability that the card has a value 
		\begin{enumerate}
			\item 7
			\item greater than 7
			\item less than 7
		\end{enumerate}
		%\input{exemplar/10/13/3/30/main.tex}
  \item A Lot consists of 48 mobile phones of which 42 are good, 3 have only minor defects and 3 have major defects.Varnika will buy a phone if it is good but the trader will only buy a mobile if it has no major defects. One phone is selected at random from the lot. What is the probability that it is
\begin{enumerate}
	\item acceptable to Varnika?
            \item acceptable to the trader?
\end{enumerate}
\solution
	%\input{exemplar/10/13/3/40/main.tex}
 \item A student says that if you throw a die, it will show up 1 or not 1. Therefore, the probability of getting 1 and the probability of getting 'not 1' each is equal to $\frac{1}{2}$. Is this correct? Give reasons.\\
 \solution
        %\input{exemplar/10/13/2/9/main.tex}
   \item Four candidates A, B, C, D have ap-
plied for the assignment to coach a school cricket
team. If A is twice as likely to be selected as B, and
B and C are given about the same chance of being
selected, while C is twice as likely to be selected
as D, what are the probabilities that
\begin{enumerate}
\item C will be selected?
\item A will not be selected?
\end{enumerate}
	%\input{exemplar/11/16/3/9/main.tex}
 \item A bag contain 24 balls of which $x$ balls are red, $2x$ are white and $3x$ are blue. A ball is selected at random, What is the probability that it is
\begin{enumerate}[label=\alph*)]
\item not red ?
\item white ?
\end{enumerate}
%\input{exemplar/10/13/3/41/main.tex}
If the letters of the word ASSASSINATION are arranged at random. Find the Probability that
\begin{enumerate}[label=(\alph*)]
\item Four $S's$ come consecutively in the word
\item Two  $I's$ and two $N's$ come together
\item All $A's$ are not coming together
\item No two $A's$ are coming together
\end{enumerate}
%\input{exemplar/11/16/3/14/main.tex}
	\item One urn contains two black balls (labelled B1 and B2) and one white ball. A
	second urn contains one black ball and two white balls (labelled W1 and W2).
	Suppose the following experiment is performed. One of the two urns is chosen
	at random. Next a ball is randomly chosen from the urn. Then a second ball is
	chosen at random from the same urn without replacing the first ball.
	
	\begin{enumerate}
	\item What is the probability that two black balls are chosen?
	
	\item What is the probability that two balls of opposite colour are chosen?
	\end{enumerate}
	\solution
	%\input{exemplar/11/16/3/12/main1.tex}
\end{enumerate}

\item In a certain lottery 10,000 tickets are sold and ten equal prizes are awarded. What is the probability of not getting a prize if you buy (a) one ticket (b) two tickets (c) 10 tickets ?	
\\
\solution
		%\begin{enumerate}[label=\thesection.\arabic*,ref=\thesection.\theenumi]
	\item One card is drawn from a well-shuffled deck of 52 cards. Find the probability of getting
\begin{enumerate}
\item A king of red colour 
\item A face card 
\item A red face card
\item The jack of hearts
\item A spade
\item The queen of diamonds

\end{enumerate}
\solution
		%\input{ncert/10/15/1/14/main.tex}
	\item Five cards—the ten, jack, queen, king and ace of diamonds, are well-shuffled with their face downwards. One card is then picked up at random.
\begin{enumerate}
\item
What is the probability that the card is the queen? 
\item
If the queen is drawn and put aside, what is the probability that the second card picked up is (a) an ace? (b) a queen?\\
\end{enumerate}
\solution
		%\input{ncert/10/15/1/15/defs.tex}
	\item A bag contains $5$ red balls and some blue balls. If the probability of drawing a blue ball is double that if a red ball, determine the number of blue balls in the bag. 
		\\
\solution
		%\input{ncert/10/15/2/3/defs.tex}
	\item A card is selected from a pack of 52 cards.
 \begin{enumerate}[label=(\alph*)] 
                 \item How many points are there in the sample space?
                 \item Calculate the probability that the card is an ace of spades.
                 \item Calculate the probability that the card is (i) an ace and (ii) black card.
 \end{enumerate}
\solution
		%\input{ncert/11/16/3/4/main.tex}
\item Four cards are drawn from a well-shuffled deck of 52 cards. What is the probability of obtaining 3 diamonds and one spade.
\\
\solution
		%\input{ncert/11/16/4/2/defs.tex}
\item In a certain lottery 10,000 tickets are sold and ten equal prizes are awarded. What is the probability of not getting a prize if you buy (a) one ticket (b) two tickets (c) 10 tickets ?	
\\
\solution
		%\input{ncert/11/16/4/4/defs.tex}
		%
\item 
Out of 100 students, two sections of 40 and 60 are formed. If you and your friend are among the 100 students, what is the probability that
\begin{enumerate}
\item you both enter the same section?
\item you both enter the different sections?
\end{enumerate}
\solution
		%\input{ncert/11/16/4/5/defs.tex}
	\item 
The number lock of a suitcase has 4 wheels each labelled with ten digits i.e. from 0 to 9.The lock opens with a sequence of four digits with no repeats.What is the probability of a person getting the right sequence to open the suitcase.
\\
\solution
		%\input{ncert/11/16/4/10/defs.tex}
		%
\item 
Two cards are drawn at random and without replacement from a pack of 52 playing cards. Find the probability that both the cards are black.
\\
\solution
		%\input{ncert/12/13/2/2/defs.tex}
		\item A box of oranges is inspected by examining three randomly selected oranges drawn without replacement. If all the three oranges are good, the box is approved for sale, otherwise, it is rejected. Find the probability that a box containing 15 oranges out of which 12 are good and 3 are bad ones will be approved for sale.
		\label{ncert/12/13/2/3/defs.tex}
		\item Two balls are drawn at random with replacement from a box containing 10 black and 8 red balls. Find the probability that
		\label{ncert/12/13/2/12}
\begin{enumerate}
\item both balls are red.
\item first ball is black and second is red.
\item one of them is black and other is red.
\end{enumerate}

\item In a hostel, 60\% of the students read Hindi newspaper, 40\% read English newspaper and 20\% read both Hindi and English newspapers. A student is selected at random.
		\label{ncert/12/13/2/15}
\begin{enumerate}
\item Find the probability that she reads neither Hindi nor English newspapers.
\item If she reads Hindi newspaper, find the probability that she reads English newspaper.
\item If she reads English newspaper, find the probability that she reads Hindi newspaper.\\
\end{enumerate}
\item The probability of obtaining an even prime number on each die, when a pair of dice is rolled is 
\begin{enumerate}
    \item $0$ 
    
    \item $\frac{1}{3}$ 
    
    \item $\frac{1}{12}$ 
    
    \item $\frac{1}{36}$ 
\end{enumerate}
\solution
		%\input{ncert/12/13/2/17/defs.tex}
	\item A bag contains 4 red and 4 black balls, another bag contains 2 red and 6 black balls. One of the two bags is selected at random and a ball is drawn from the bag which is found to be red. Find the probability that the ball is drawn from the first bag.
\\
\solution
		%\input{ncert/12/13/3/2/main.tex}
  \item
  Cards with numbers 2 to 101 are placed in a box. A card is selected at random.Find the probability that the card has
\begin{enumerate}[label=(\roman*)]
	\item an even number 
	\item a square number
\end{enumerate}
\solution
%\input{exemplar/10/13/3/32/main.tex}
\item
The king, queen and jack of clubs are removed from a deck of 52 playing cards and then well shuffled. Now one card is drawn at random from the remaining cards.  Determine the probability that the card is
\begin{enumerate}[label=(\roman*)]
\item a club
\item 10 of hearts
\end{enumerate}
\solution
%\input{exemplar/10/13/3/29/main.tex}
\item A team of medical students doing their internship have to assist during surgeries
at a city hospital. The probabilities of surgeries rated as very complex, complex,
routine, simple or very simple are respectively, 0.15, 0.20, 0.31, 0.26, .08. Find
the probabilities that a particular surgery will be rated
\begin{enumerate}
	\item complex or very complex;
	\item neither very complex nor very simple;
	\item routine or complex
	\item routine or simple
\end{enumerate}
\solution
%\input{exemplar/11/16/3/8(1)/main.tex}
\item A card is selected from a pack of 52 cards.
\begin{enumerate}[label=(\alph*)]
    \item How many points are there in the sample space?
    \item Calculate the probability that the card is an ace of spades.
    \item Calculate the probability that the card is (i) an ace and (ii) black card.
\end{enumerate}
\solution
%\input{exemplar/11/16/3/4/main2.tex}
\item The probability that a non leap year selected at random will contain 53 sundays.
\\
\solution
%\input{exemplar/10/13/1/19/main.tex}
\item One of the four persons John, Rita, Aslam or Gurpreet will be promoted next
month. Consequently the sample space consists of four elementary outcomes
S = {John promoted, Rita promoted, Aslam promoted, Gurpreet promoted}
You are told that the chances of John’s promotion is same as that of Gurpreet,
Rita’s chances of promotion are twice as likely as Johns. Aslam’s chances are
four times that of John.
\begin{enumerate}
	\item Determine
	\begin{enumerate}
		\item P (John promoted)
		\item P (Rita promoted)
		\item P (Aslam promoted)
		\item P (Gurpreet promoted)
	\end{enumerate}
	\item If A = {John promoted or Gurpreet promoted}, find P (A).
\end{enumerate}
\solution
%\input{exemplar/11/16/3/10/main.tex}
\item A card is drawn from a deck of 52 cards. Find the probability of getting a king or a heart or a red card.\\
\solution
%\input{exemplar/11/16/3/15/main.tex}
\item The probability that a student will pass his examination is 0.73, the probability of
the student getting a compartment is 0.13, and the probability that the student will
either pass or get compartment is 0.96. State True or False.\\
\solution
%\input{exemplar/11/16/3/31/main.tex}
\item A card is selected from a pack of 52 cards\\
\begin{enumerate}[label=(\alph*)]
\item How many points are there in the sample space?
\item Calculate the probability that the cards is an ace of spades.
\item Calculate the probability that the card is (i) an ace (ii)black card.\\
\end{enumerate}
%\input{ncert/11/16/3/4_1/Prob_4.tex}
\item In a non-leap year, the probability of having 53 tuesdays or 53 wednesdays is\\
\solution
%\input{exemplar/11/16/3/18/main.tex}
\item There are 1000 sealed envelopes in a box, 10 of them contain a cash prize of
Rs 100 each, 100 of them contain a cash prize of Rs 50 each and 200 of them
contain a cash prize of Rs 10 each and rest do not contain any cash prize. If they
are well shuffled and an envelope is picked up out, what is the probability that it
contains no cash prize?\\
\solution
%\input{exemplar/10/13/3/34/main.tex}
\item 
A die is thrown and a card is selected at random from a deck of 52 playing cards. The probability of getting an even number on the die and a spade card.\\
\solution
%\input{exemplar/12/13/3/78/main.tex}
\item
If 4-digit numbers greater than 5,000 are randomly formed from the digits 0, 1, 3, 5, and 7, what is the probability of forming a number divisible by 5 when:
\begin{enumerate}
    \item The digits are repeated?
    \item The repetition of digits is not allowed?
\end{enumerate}
\solution
%\input{ncert/11/16/4/9/main.tex}
\item Consider the probability space $\brak{\Omega, \mathcal{G}, P}$ where $\Omega = [0,2]$ and $\mathcal{G} = \cbrak{\phi, \Omega, [0,1], (1,2]}$. Let $X$ and $Y$ be two functions on $\Omega$ defined as
\begin{align*}
    X(\omega) = 
    \begin{cases}
        1 & \text{if }\omega \in [0, 1]\\
        2 & \text{if }\omega \in (1, 2]
    \end{cases}
\end{align*}
and
\begin{align*}
    Y(\omega) = 
    \begin{cases}
        2 & \text{if }\omega \in [0, 1.5]\\
        3 & \text{if }\omega \in (1.5, 2].
    \end{cases}
\end{align*}
Then which one of the following statements is true?
\begin{enumerate}
    \item [(A)] $X$ is a random variable with respect to $\mathcal{G}$, but $Y$ is not a random variable with respect to $\mathcal{G}$.
    \item [(B)] $Y$ is a random variable with respect to $\mathcal{G}$, but $X$ is not a random variable with respect to $\mathcal{G}$.
    \item [(C)] Neither $X$ nor $Y$ is a random variable with respect to $\mathcal{G}$.
    \item [(D)] Both $X$ and $Y$ are random variables with respect to $\mathcal{G}$.
\end{enumerate} \hfill (GATE ST 2023)\\
\solution
%\input{gate/ST/2023/14/main.tex}
	\item  A die is loaded in such a way that each odd number is twice as likely to occur as
each even number. Find $P(G)$, where $G$ is the event that a number greater than
3 occurs on a single roll of the die.
\\
\solution
		%\input{exemplar/11/16/3/5/main.tex}
	\item All the jacks, queens and kings are removed from a deck of 52 playing cards. The remaining cards are well shuffled and then one card is drawn at random. Giving ace a value 1 similar value for other cards, find the probability that the card has a value 
		\begin{enumerate}
			\item 7
			\item greater than 7
			\item less than 7
		\end{enumerate}
		%\input{exemplar/10/13/3/30/main.tex}
  \item A Lot consists of 48 mobile phones of which 42 are good, 3 have only minor defects and 3 have major defects.Varnika will buy a phone if it is good but the trader will only buy a mobile if it has no major defects. One phone is selected at random from the lot. What is the probability that it is
\begin{enumerate}
	\item acceptable to Varnika?
            \item acceptable to the trader?
\end{enumerate}
\solution
	%\input{exemplar/10/13/3/40/main.tex}
 \item A student says that if you throw a die, it will show up 1 or not 1. Therefore, the probability of getting 1 and the probability of getting 'not 1' each is equal to $\frac{1}{2}$. Is this correct? Give reasons.\\
 \solution
        %\input{exemplar/10/13/2/9/main.tex}
   \item Four candidates A, B, C, D have ap-
plied for the assignment to coach a school cricket
team. If A is twice as likely to be selected as B, and
B and C are given about the same chance of being
selected, while C is twice as likely to be selected
as D, what are the probabilities that
\begin{enumerate}
\item C will be selected?
\item A will not be selected?
\end{enumerate}
	%\input{exemplar/11/16/3/9/main.tex}
 \item A bag contain 24 balls of which $x$ balls are red, $2x$ are white and $3x$ are blue. A ball is selected at random, What is the probability that it is
\begin{enumerate}[label=\alph*)]
\item not red ?
\item white ?
\end{enumerate}
%\input{exemplar/10/13/3/41/main.tex}
If the letters of the word ASSASSINATION are arranged at random. Find the Probability that
\begin{enumerate}[label=(\alph*)]
\item Four $S's$ come consecutively in the word
\item Two  $I's$ and two $N's$ come together
\item All $A's$ are not coming together
\item No two $A's$ are coming together
\end{enumerate}
%\input{exemplar/11/16/3/14/main.tex}
	\item One urn contains two black balls (labelled B1 and B2) and one white ball. A
	second urn contains one black ball and two white balls (labelled W1 and W2).
	Suppose the following experiment is performed. One of the two urns is chosen
	at random. Next a ball is randomly chosen from the urn. Then a second ball is
	chosen at random from the same urn without replacing the first ball.
	
	\begin{enumerate}
	\item What is the probability that two black balls are chosen?
	
	\item What is the probability that two balls of opposite colour are chosen?
	\end{enumerate}
	\solution
	%\input{exemplar/11/16/3/12/main1.tex}
\end{enumerate}

		%
\item 
Out of 100 students, two sections of 40 and 60 are formed. If you and your friend are among the 100 students, what is the probability that
\begin{enumerate}
\item you both enter the same section?
\item you both enter the different sections?
\end{enumerate}
\solution
		%\begin{enumerate}[label=\thesection.\arabic*,ref=\thesection.\theenumi]
	\item One card is drawn from a well-shuffled deck of 52 cards. Find the probability of getting
\begin{enumerate}
\item A king of red colour 
\item A face card 
\item A red face card
\item The jack of hearts
\item A spade
\item The queen of diamonds

\end{enumerate}
\solution
		%\input{ncert/10/15/1/14/main.tex}
	\item Five cards—the ten, jack, queen, king and ace of diamonds, are well-shuffled with their face downwards. One card is then picked up at random.
\begin{enumerate}
\item
What is the probability that the card is the queen? 
\item
If the queen is drawn and put aside, what is the probability that the second card picked up is (a) an ace? (b) a queen?\\
\end{enumerate}
\solution
		%\input{ncert/10/15/1/15/defs.tex}
	\item A bag contains $5$ red balls and some blue balls. If the probability of drawing a blue ball is double that if a red ball, determine the number of blue balls in the bag. 
		\\
\solution
		%\input{ncert/10/15/2/3/defs.tex}
	\item A card is selected from a pack of 52 cards.
 \begin{enumerate}[label=(\alph*)] 
                 \item How many points are there in the sample space?
                 \item Calculate the probability that the card is an ace of spades.
                 \item Calculate the probability that the card is (i) an ace and (ii) black card.
 \end{enumerate}
\solution
		%\input{ncert/11/16/3/4/main.tex}
\item Four cards are drawn from a well-shuffled deck of 52 cards. What is the probability of obtaining 3 diamonds and one spade.
\\
\solution
		%\input{ncert/11/16/4/2/defs.tex}
\item In a certain lottery 10,000 tickets are sold and ten equal prizes are awarded. What is the probability of not getting a prize if you buy (a) one ticket (b) two tickets (c) 10 tickets ?	
\\
\solution
		%\input{ncert/11/16/4/4/defs.tex}
		%
\item 
Out of 100 students, two sections of 40 and 60 are formed. If you and your friend are among the 100 students, what is the probability that
\begin{enumerate}
\item you both enter the same section?
\item you both enter the different sections?
\end{enumerate}
\solution
		%\input{ncert/11/16/4/5/defs.tex}
	\item 
The number lock of a suitcase has 4 wheels each labelled with ten digits i.e. from 0 to 9.The lock opens with a sequence of four digits with no repeats.What is the probability of a person getting the right sequence to open the suitcase.
\\
\solution
		%\input{ncert/11/16/4/10/defs.tex}
		%
\item 
Two cards are drawn at random and without replacement from a pack of 52 playing cards. Find the probability that both the cards are black.
\\
\solution
		%\input{ncert/12/13/2/2/defs.tex}
		\item A box of oranges is inspected by examining three randomly selected oranges drawn without replacement. If all the three oranges are good, the box is approved for sale, otherwise, it is rejected. Find the probability that a box containing 15 oranges out of which 12 are good and 3 are bad ones will be approved for sale.
		\label{ncert/12/13/2/3/defs.tex}
		\item Two balls are drawn at random with replacement from a box containing 10 black and 8 red balls. Find the probability that
		\label{ncert/12/13/2/12}
\begin{enumerate}
\item both balls are red.
\item first ball is black and second is red.
\item one of them is black and other is red.
\end{enumerate}

\item In a hostel, 60\% of the students read Hindi newspaper, 40\% read English newspaper and 20\% read both Hindi and English newspapers. A student is selected at random.
		\label{ncert/12/13/2/15}
\begin{enumerate}
\item Find the probability that she reads neither Hindi nor English newspapers.
\item If she reads Hindi newspaper, find the probability that she reads English newspaper.
\item If she reads English newspaper, find the probability that she reads Hindi newspaper.\\
\end{enumerate}
\item The probability of obtaining an even prime number on each die, when a pair of dice is rolled is 
\begin{enumerate}
    \item $0$ 
    
    \item $\frac{1}{3}$ 
    
    \item $\frac{1}{12}$ 
    
    \item $\frac{1}{36}$ 
\end{enumerate}
\solution
		%\input{ncert/12/13/2/17/defs.tex}
	\item A bag contains 4 red and 4 black balls, another bag contains 2 red and 6 black balls. One of the two bags is selected at random and a ball is drawn from the bag which is found to be red. Find the probability that the ball is drawn from the first bag.
\\
\solution
		%\input{ncert/12/13/3/2/main.tex}
  \item
  Cards with numbers 2 to 101 are placed in a box. A card is selected at random.Find the probability that the card has
\begin{enumerate}[label=(\roman*)]
	\item an even number 
	\item a square number
\end{enumerate}
\solution
%\input{exemplar/10/13/3/32/main.tex}
\item
The king, queen and jack of clubs are removed from a deck of 52 playing cards and then well shuffled. Now one card is drawn at random from the remaining cards.  Determine the probability that the card is
\begin{enumerate}[label=(\roman*)]
\item a club
\item 10 of hearts
\end{enumerate}
\solution
%\input{exemplar/10/13/3/29/main.tex}
\item A team of medical students doing their internship have to assist during surgeries
at a city hospital. The probabilities of surgeries rated as very complex, complex,
routine, simple or very simple are respectively, 0.15, 0.20, 0.31, 0.26, .08. Find
the probabilities that a particular surgery will be rated
\begin{enumerate}
	\item complex or very complex;
	\item neither very complex nor very simple;
	\item routine or complex
	\item routine or simple
\end{enumerate}
\solution
%\input{exemplar/11/16/3/8(1)/main.tex}
\item A card is selected from a pack of 52 cards.
\begin{enumerate}[label=(\alph*)]
    \item How many points are there in the sample space?
    \item Calculate the probability that the card is an ace of spades.
    \item Calculate the probability that the card is (i) an ace and (ii) black card.
\end{enumerate}
\solution
%\input{exemplar/11/16/3/4/main2.tex}
\item The probability that a non leap year selected at random will contain 53 sundays.
\\
\solution
%\input{exemplar/10/13/1/19/main.tex}
\item One of the four persons John, Rita, Aslam or Gurpreet will be promoted next
month. Consequently the sample space consists of four elementary outcomes
S = {John promoted, Rita promoted, Aslam promoted, Gurpreet promoted}
You are told that the chances of John’s promotion is same as that of Gurpreet,
Rita’s chances of promotion are twice as likely as Johns. Aslam’s chances are
four times that of John.
\begin{enumerate}
	\item Determine
	\begin{enumerate}
		\item P (John promoted)
		\item P (Rita promoted)
		\item P (Aslam promoted)
		\item P (Gurpreet promoted)
	\end{enumerate}
	\item If A = {John promoted or Gurpreet promoted}, find P (A).
\end{enumerate}
\solution
%\input{exemplar/11/16/3/10/main.tex}
\item A card is drawn from a deck of 52 cards. Find the probability of getting a king or a heart or a red card.\\
\solution
%\input{exemplar/11/16/3/15/main.tex}
\item The probability that a student will pass his examination is 0.73, the probability of
the student getting a compartment is 0.13, and the probability that the student will
either pass or get compartment is 0.96. State True or False.\\
\solution
%\input{exemplar/11/16/3/31/main.tex}
\item A card is selected from a pack of 52 cards\\
\begin{enumerate}[label=(\alph*)]
\item How many points are there in the sample space?
\item Calculate the probability that the cards is an ace of spades.
\item Calculate the probability that the card is (i) an ace (ii)black card.\\
\end{enumerate}
%\input{ncert/11/16/3/4_1/Prob_4.tex}
\item In a non-leap year, the probability of having 53 tuesdays or 53 wednesdays is\\
\solution
%\input{exemplar/11/16/3/18/main.tex}
\item There are 1000 sealed envelopes in a box, 10 of them contain a cash prize of
Rs 100 each, 100 of them contain a cash prize of Rs 50 each and 200 of them
contain a cash prize of Rs 10 each and rest do not contain any cash prize. If they
are well shuffled and an envelope is picked up out, what is the probability that it
contains no cash prize?\\
\solution
%\input{exemplar/10/13/3/34/main.tex}
\item 
A die is thrown and a card is selected at random from a deck of 52 playing cards. The probability of getting an even number on the die and a spade card.\\
\solution
%\input{exemplar/12/13/3/78/main.tex}
\item
If 4-digit numbers greater than 5,000 are randomly formed from the digits 0, 1, 3, 5, and 7, what is the probability of forming a number divisible by 5 when:
\begin{enumerate}
    \item The digits are repeated?
    \item The repetition of digits is not allowed?
\end{enumerate}
\solution
%\input{ncert/11/16/4/9/main.tex}
\item Consider the probability space $\brak{\Omega, \mathcal{G}, P}$ where $\Omega = [0,2]$ and $\mathcal{G} = \cbrak{\phi, \Omega, [0,1], (1,2]}$. Let $X$ and $Y$ be two functions on $\Omega$ defined as
\begin{align*}
    X(\omega) = 
    \begin{cases}
        1 & \text{if }\omega \in [0, 1]\\
        2 & \text{if }\omega \in (1, 2]
    \end{cases}
\end{align*}
and
\begin{align*}
    Y(\omega) = 
    \begin{cases}
        2 & \text{if }\omega \in [0, 1.5]\\
        3 & \text{if }\omega \in (1.5, 2].
    \end{cases}
\end{align*}
Then which one of the following statements is true?
\begin{enumerate}
    \item [(A)] $X$ is a random variable with respect to $\mathcal{G}$, but $Y$ is not a random variable with respect to $\mathcal{G}$.
    \item [(B)] $Y$ is a random variable with respect to $\mathcal{G}$, but $X$ is not a random variable with respect to $\mathcal{G}$.
    \item [(C)] Neither $X$ nor $Y$ is a random variable with respect to $\mathcal{G}$.
    \item [(D)] Both $X$ and $Y$ are random variables with respect to $\mathcal{G}$.
\end{enumerate} \hfill (GATE ST 2023)\\
\solution
%\input{gate/ST/2023/14/main.tex}
	\item  A die is loaded in such a way that each odd number is twice as likely to occur as
each even number. Find $P(G)$, where $G$ is the event that a number greater than
3 occurs on a single roll of the die.
\\
\solution
		%\input{exemplar/11/16/3/5/main.tex}
	\item All the jacks, queens and kings are removed from a deck of 52 playing cards. The remaining cards are well shuffled and then one card is drawn at random. Giving ace a value 1 similar value for other cards, find the probability that the card has a value 
		\begin{enumerate}
			\item 7
			\item greater than 7
			\item less than 7
		\end{enumerate}
		%\input{exemplar/10/13/3/30/main.tex}
  \item A Lot consists of 48 mobile phones of which 42 are good, 3 have only minor defects and 3 have major defects.Varnika will buy a phone if it is good but the trader will only buy a mobile if it has no major defects. One phone is selected at random from the lot. What is the probability that it is
\begin{enumerate}
	\item acceptable to Varnika?
            \item acceptable to the trader?
\end{enumerate}
\solution
	%\input{exemplar/10/13/3/40/main.tex}
 \item A student says that if you throw a die, it will show up 1 or not 1. Therefore, the probability of getting 1 and the probability of getting 'not 1' each is equal to $\frac{1}{2}$. Is this correct? Give reasons.\\
 \solution
        %\input{exemplar/10/13/2/9/main.tex}
   \item Four candidates A, B, C, D have ap-
plied for the assignment to coach a school cricket
team. If A is twice as likely to be selected as B, and
B and C are given about the same chance of being
selected, while C is twice as likely to be selected
as D, what are the probabilities that
\begin{enumerate}
\item C will be selected?
\item A will not be selected?
\end{enumerate}
	%\input{exemplar/11/16/3/9/main.tex}
 \item A bag contain 24 balls of which $x$ balls are red, $2x$ are white and $3x$ are blue. A ball is selected at random, What is the probability that it is
\begin{enumerate}[label=\alph*)]
\item not red ?
\item white ?
\end{enumerate}
%\input{exemplar/10/13/3/41/main.tex}
If the letters of the word ASSASSINATION are arranged at random. Find the Probability that
\begin{enumerate}[label=(\alph*)]
\item Four $S's$ come consecutively in the word
\item Two  $I's$ and two $N's$ come together
\item All $A's$ are not coming together
\item No two $A's$ are coming together
\end{enumerate}
%\input{exemplar/11/16/3/14/main.tex}
	\item One urn contains two black balls (labelled B1 and B2) and one white ball. A
	second urn contains one black ball and two white balls (labelled W1 and W2).
	Suppose the following experiment is performed. One of the two urns is chosen
	at random. Next a ball is randomly chosen from the urn. Then a second ball is
	chosen at random from the same urn without replacing the first ball.
	
	\begin{enumerate}
	\item What is the probability that two black balls are chosen?
	
	\item What is the probability that two balls of opposite colour are chosen?
	\end{enumerate}
	\solution
	%\input{exemplar/11/16/3/12/main1.tex}
\end{enumerate}

	\item 
The number lock of a suitcase has 4 wheels each labelled with ten digits i.e. from 0 to 9.The lock opens with a sequence of four digits with no repeats.What is the probability of a person getting the right sequence to open the suitcase.
\\
\solution
		%\begin{enumerate}[label=\thesection.\arabic*,ref=\thesection.\theenumi]
	\item One card is drawn from a well-shuffled deck of 52 cards. Find the probability of getting
\begin{enumerate}
\item A king of red colour 
\item A face card 
\item A red face card
\item The jack of hearts
\item A spade
\item The queen of diamonds

\end{enumerate}
\solution
		%\input{ncert/10/15/1/14/main.tex}
	\item Five cards—the ten, jack, queen, king and ace of diamonds, are well-shuffled with their face downwards. One card is then picked up at random.
\begin{enumerate}
\item
What is the probability that the card is the queen? 
\item
If the queen is drawn and put aside, what is the probability that the second card picked up is (a) an ace? (b) a queen?\\
\end{enumerate}
\solution
		%\input{ncert/10/15/1/15/defs.tex}
	\item A bag contains $5$ red balls and some blue balls. If the probability of drawing a blue ball is double that if a red ball, determine the number of blue balls in the bag. 
		\\
\solution
		%\input{ncert/10/15/2/3/defs.tex}
	\item A card is selected from a pack of 52 cards.
 \begin{enumerate}[label=(\alph*)] 
                 \item How many points are there in the sample space?
                 \item Calculate the probability that the card is an ace of spades.
                 \item Calculate the probability that the card is (i) an ace and (ii) black card.
 \end{enumerate}
\solution
		%\input{ncert/11/16/3/4/main.tex}
\item Four cards are drawn from a well-shuffled deck of 52 cards. What is the probability of obtaining 3 diamonds and one spade.
\\
\solution
		%\input{ncert/11/16/4/2/defs.tex}
\item In a certain lottery 10,000 tickets are sold and ten equal prizes are awarded. What is the probability of not getting a prize if you buy (a) one ticket (b) two tickets (c) 10 tickets ?	
\\
\solution
		%\input{ncert/11/16/4/4/defs.tex}
		%
\item 
Out of 100 students, two sections of 40 and 60 are formed. If you and your friend are among the 100 students, what is the probability that
\begin{enumerate}
\item you both enter the same section?
\item you both enter the different sections?
\end{enumerate}
\solution
		%\input{ncert/11/16/4/5/defs.tex}
	\item 
The number lock of a suitcase has 4 wheels each labelled with ten digits i.e. from 0 to 9.The lock opens with a sequence of four digits with no repeats.What is the probability of a person getting the right sequence to open the suitcase.
\\
\solution
		%\input{ncert/11/16/4/10/defs.tex}
		%
\item 
Two cards are drawn at random and without replacement from a pack of 52 playing cards. Find the probability that both the cards are black.
\\
\solution
		%\input{ncert/12/13/2/2/defs.tex}
		\item A box of oranges is inspected by examining three randomly selected oranges drawn without replacement. If all the three oranges are good, the box is approved for sale, otherwise, it is rejected. Find the probability that a box containing 15 oranges out of which 12 are good and 3 are bad ones will be approved for sale.
		\label{ncert/12/13/2/3/defs.tex}
		\item Two balls are drawn at random with replacement from a box containing 10 black and 8 red balls. Find the probability that
		\label{ncert/12/13/2/12}
\begin{enumerate}
\item both balls are red.
\item first ball is black and second is red.
\item one of them is black and other is red.
\end{enumerate}

\item In a hostel, 60\% of the students read Hindi newspaper, 40\% read English newspaper and 20\% read both Hindi and English newspapers. A student is selected at random.
		\label{ncert/12/13/2/15}
\begin{enumerate}
\item Find the probability that she reads neither Hindi nor English newspapers.
\item If she reads Hindi newspaper, find the probability that she reads English newspaper.
\item If she reads English newspaper, find the probability that she reads Hindi newspaper.\\
\end{enumerate}
\item The probability of obtaining an even prime number on each die, when a pair of dice is rolled is 
\begin{enumerate}
    \item $0$ 
    
    \item $\frac{1}{3}$ 
    
    \item $\frac{1}{12}$ 
    
    \item $\frac{1}{36}$ 
\end{enumerate}
\solution
		%\input{ncert/12/13/2/17/defs.tex}
	\item A bag contains 4 red and 4 black balls, another bag contains 2 red and 6 black balls. One of the two bags is selected at random and a ball is drawn from the bag which is found to be red. Find the probability that the ball is drawn from the first bag.
\\
\solution
		%\input{ncert/12/13/3/2/main.tex}
  \item
  Cards with numbers 2 to 101 are placed in a box. A card is selected at random.Find the probability that the card has
\begin{enumerate}[label=(\roman*)]
	\item an even number 
	\item a square number
\end{enumerate}
\solution
%\input{exemplar/10/13/3/32/main.tex}
\item
The king, queen and jack of clubs are removed from a deck of 52 playing cards and then well shuffled. Now one card is drawn at random from the remaining cards.  Determine the probability that the card is
\begin{enumerate}[label=(\roman*)]
\item a club
\item 10 of hearts
\end{enumerate}
\solution
%\input{exemplar/10/13/3/29/main.tex}
\item A team of medical students doing their internship have to assist during surgeries
at a city hospital. The probabilities of surgeries rated as very complex, complex,
routine, simple or very simple are respectively, 0.15, 0.20, 0.31, 0.26, .08. Find
the probabilities that a particular surgery will be rated
\begin{enumerate}
	\item complex or very complex;
	\item neither very complex nor very simple;
	\item routine or complex
	\item routine or simple
\end{enumerate}
\solution
%\input{exemplar/11/16/3/8(1)/main.tex}
\item A card is selected from a pack of 52 cards.
\begin{enumerate}[label=(\alph*)]
    \item How many points are there in the sample space?
    \item Calculate the probability that the card is an ace of spades.
    \item Calculate the probability that the card is (i) an ace and (ii) black card.
\end{enumerate}
\solution
%\input{exemplar/11/16/3/4/main2.tex}
\item The probability that a non leap year selected at random will contain 53 sundays.
\\
\solution
%\input{exemplar/10/13/1/19/main.tex}
\item One of the four persons John, Rita, Aslam or Gurpreet will be promoted next
month. Consequently the sample space consists of four elementary outcomes
S = {John promoted, Rita promoted, Aslam promoted, Gurpreet promoted}
You are told that the chances of John’s promotion is same as that of Gurpreet,
Rita’s chances of promotion are twice as likely as Johns. Aslam’s chances are
four times that of John.
\begin{enumerate}
	\item Determine
	\begin{enumerate}
		\item P (John promoted)
		\item P (Rita promoted)
		\item P (Aslam promoted)
		\item P (Gurpreet promoted)
	\end{enumerate}
	\item If A = {John promoted or Gurpreet promoted}, find P (A).
\end{enumerate}
\solution
%\input{exemplar/11/16/3/10/main.tex}
\item A card is drawn from a deck of 52 cards. Find the probability of getting a king or a heart or a red card.\\
\solution
%\input{exemplar/11/16/3/15/main.tex}
\item The probability that a student will pass his examination is 0.73, the probability of
the student getting a compartment is 0.13, and the probability that the student will
either pass or get compartment is 0.96. State True or False.\\
\solution
%\input{exemplar/11/16/3/31/main.tex}
\item A card is selected from a pack of 52 cards\\
\begin{enumerate}[label=(\alph*)]
\item How many points are there in the sample space?
\item Calculate the probability that the cards is an ace of spades.
\item Calculate the probability that the card is (i) an ace (ii)black card.\\
\end{enumerate}
%\input{ncert/11/16/3/4_1/Prob_4.tex}
\item In a non-leap year, the probability of having 53 tuesdays or 53 wednesdays is\\
\solution
%\input{exemplar/11/16/3/18/main.tex}
\item There are 1000 sealed envelopes in a box, 10 of them contain a cash prize of
Rs 100 each, 100 of them contain a cash prize of Rs 50 each and 200 of them
contain a cash prize of Rs 10 each and rest do not contain any cash prize. If they
are well shuffled and an envelope is picked up out, what is the probability that it
contains no cash prize?\\
\solution
%\input{exemplar/10/13/3/34/main.tex}
\item 
A die is thrown and a card is selected at random from a deck of 52 playing cards. The probability of getting an even number on the die and a spade card.\\
\solution
%\input{exemplar/12/13/3/78/main.tex}
\item
If 4-digit numbers greater than 5,000 are randomly formed from the digits 0, 1, 3, 5, and 7, what is the probability of forming a number divisible by 5 when:
\begin{enumerate}
    \item The digits are repeated?
    \item The repetition of digits is not allowed?
\end{enumerate}
\solution
%\input{ncert/11/16/4/9/main.tex}
\item Consider the probability space $\brak{\Omega, \mathcal{G}, P}$ where $\Omega = [0,2]$ and $\mathcal{G} = \cbrak{\phi, \Omega, [0,1], (1,2]}$. Let $X$ and $Y$ be two functions on $\Omega$ defined as
\begin{align*}
    X(\omega) = 
    \begin{cases}
        1 & \text{if }\omega \in [0, 1]\\
        2 & \text{if }\omega \in (1, 2]
    \end{cases}
\end{align*}
and
\begin{align*}
    Y(\omega) = 
    \begin{cases}
        2 & \text{if }\omega \in [0, 1.5]\\
        3 & \text{if }\omega \in (1.5, 2].
    \end{cases}
\end{align*}
Then which one of the following statements is true?
\begin{enumerate}
    \item [(A)] $X$ is a random variable with respect to $\mathcal{G}$, but $Y$ is not a random variable with respect to $\mathcal{G}$.
    \item [(B)] $Y$ is a random variable with respect to $\mathcal{G}$, but $X$ is not a random variable with respect to $\mathcal{G}$.
    \item [(C)] Neither $X$ nor $Y$ is a random variable with respect to $\mathcal{G}$.
    \item [(D)] Both $X$ and $Y$ are random variables with respect to $\mathcal{G}$.
\end{enumerate} \hfill (GATE ST 2023)\\
\solution
%\input{gate/ST/2023/14/main.tex}
	\item  A die is loaded in such a way that each odd number is twice as likely to occur as
each even number. Find $P(G)$, where $G$ is the event that a number greater than
3 occurs on a single roll of the die.
\\
\solution
		%\input{exemplar/11/16/3/5/main.tex}
	\item All the jacks, queens and kings are removed from a deck of 52 playing cards. The remaining cards are well shuffled and then one card is drawn at random. Giving ace a value 1 similar value for other cards, find the probability that the card has a value 
		\begin{enumerate}
			\item 7
			\item greater than 7
			\item less than 7
		\end{enumerate}
		%\input{exemplar/10/13/3/30/main.tex}
  \item A Lot consists of 48 mobile phones of which 42 are good, 3 have only minor defects and 3 have major defects.Varnika will buy a phone if it is good but the trader will only buy a mobile if it has no major defects. One phone is selected at random from the lot. What is the probability that it is
\begin{enumerate}
	\item acceptable to Varnika?
            \item acceptable to the trader?
\end{enumerate}
\solution
	%\input{exemplar/10/13/3/40/main.tex}
 \item A student says that if you throw a die, it will show up 1 or not 1. Therefore, the probability of getting 1 and the probability of getting 'not 1' each is equal to $\frac{1}{2}$. Is this correct? Give reasons.\\
 \solution
        %\input{exemplar/10/13/2/9/main.tex}
   \item Four candidates A, B, C, D have ap-
plied for the assignment to coach a school cricket
team. If A is twice as likely to be selected as B, and
B and C are given about the same chance of being
selected, while C is twice as likely to be selected
as D, what are the probabilities that
\begin{enumerate}
\item C will be selected?
\item A will not be selected?
\end{enumerate}
	%\input{exemplar/11/16/3/9/main.tex}
 \item A bag contain 24 balls of which $x$ balls are red, $2x$ are white and $3x$ are blue. A ball is selected at random, What is the probability that it is
\begin{enumerate}[label=\alph*)]
\item not red ?
\item white ?
\end{enumerate}
%\input{exemplar/10/13/3/41/main.tex}
If the letters of the word ASSASSINATION are arranged at random. Find the Probability that
\begin{enumerate}[label=(\alph*)]
\item Four $S's$ come consecutively in the word
\item Two  $I's$ and two $N's$ come together
\item All $A's$ are not coming together
\item No two $A's$ are coming together
\end{enumerate}
%\input{exemplar/11/16/3/14/main.tex}
	\item One urn contains two black balls (labelled B1 and B2) and one white ball. A
	second urn contains one black ball and two white balls (labelled W1 and W2).
	Suppose the following experiment is performed. One of the two urns is chosen
	at random. Next a ball is randomly chosen from the urn. Then a second ball is
	chosen at random from the same urn without replacing the first ball.
	
	\begin{enumerate}
	\item What is the probability that two black balls are chosen?
	
	\item What is the probability that two balls of opposite colour are chosen?
	\end{enumerate}
	\solution
	%\input{exemplar/11/16/3/12/main1.tex}
\end{enumerate}

		%
\item 
Two cards are drawn at random and without replacement from a pack of 52 playing cards. Find the probability that both the cards are black.
\\
\solution
		%\begin{enumerate}[label=\thesection.\arabic*,ref=\thesection.\theenumi]
	\item One card is drawn from a well-shuffled deck of 52 cards. Find the probability of getting
\begin{enumerate}
\item A king of red colour 
\item A face card 
\item A red face card
\item The jack of hearts
\item A spade
\item The queen of diamonds

\end{enumerate}
\solution
		%\input{ncert/10/15/1/14/main.tex}
	\item Five cards—the ten, jack, queen, king and ace of diamonds, are well-shuffled with their face downwards. One card is then picked up at random.
\begin{enumerate}
\item
What is the probability that the card is the queen? 
\item
If the queen is drawn and put aside, what is the probability that the second card picked up is (a) an ace? (b) a queen?\\
\end{enumerate}
\solution
		%\input{ncert/10/15/1/15/defs.tex}
	\item A bag contains $5$ red balls and some blue balls. If the probability of drawing a blue ball is double that if a red ball, determine the number of blue balls in the bag. 
		\\
\solution
		%\input{ncert/10/15/2/3/defs.tex}
	\item A card is selected from a pack of 52 cards.
 \begin{enumerate}[label=(\alph*)] 
                 \item How many points are there in the sample space?
                 \item Calculate the probability that the card is an ace of spades.
                 \item Calculate the probability that the card is (i) an ace and (ii) black card.
 \end{enumerate}
\solution
		%\input{ncert/11/16/3/4/main.tex}
\item Four cards are drawn from a well-shuffled deck of 52 cards. What is the probability of obtaining 3 diamonds and one spade.
\\
\solution
		%\input{ncert/11/16/4/2/defs.tex}
\item In a certain lottery 10,000 tickets are sold and ten equal prizes are awarded. What is the probability of not getting a prize if you buy (a) one ticket (b) two tickets (c) 10 tickets ?	
\\
\solution
		%\input{ncert/11/16/4/4/defs.tex}
		%
\item 
Out of 100 students, two sections of 40 and 60 are formed. If you and your friend are among the 100 students, what is the probability that
\begin{enumerate}
\item you both enter the same section?
\item you both enter the different sections?
\end{enumerate}
\solution
		%\input{ncert/11/16/4/5/defs.tex}
	\item 
The number lock of a suitcase has 4 wheels each labelled with ten digits i.e. from 0 to 9.The lock opens with a sequence of four digits with no repeats.What is the probability of a person getting the right sequence to open the suitcase.
\\
\solution
		%\input{ncert/11/16/4/10/defs.tex}
		%
\item 
Two cards are drawn at random and without replacement from a pack of 52 playing cards. Find the probability that both the cards are black.
\\
\solution
		%\input{ncert/12/13/2/2/defs.tex}
		\item A box of oranges is inspected by examining three randomly selected oranges drawn without replacement. If all the three oranges are good, the box is approved for sale, otherwise, it is rejected. Find the probability that a box containing 15 oranges out of which 12 are good and 3 are bad ones will be approved for sale.
		\label{ncert/12/13/2/3/defs.tex}
		\item Two balls are drawn at random with replacement from a box containing 10 black and 8 red balls. Find the probability that
		\label{ncert/12/13/2/12}
\begin{enumerate}
\item both balls are red.
\item first ball is black and second is red.
\item one of them is black and other is red.
\end{enumerate}

\item In a hostel, 60\% of the students read Hindi newspaper, 40\% read English newspaper and 20\% read both Hindi and English newspapers. A student is selected at random.
		\label{ncert/12/13/2/15}
\begin{enumerate}
\item Find the probability that she reads neither Hindi nor English newspapers.
\item If she reads Hindi newspaper, find the probability that she reads English newspaper.
\item If she reads English newspaper, find the probability that she reads Hindi newspaper.\\
\end{enumerate}
\item The probability of obtaining an even prime number on each die, when a pair of dice is rolled is 
\begin{enumerate}
    \item $0$ 
    
    \item $\frac{1}{3}$ 
    
    \item $\frac{1}{12}$ 
    
    \item $\frac{1}{36}$ 
\end{enumerate}
\solution
		%\input{ncert/12/13/2/17/defs.tex}
	\item A bag contains 4 red and 4 black balls, another bag contains 2 red and 6 black balls. One of the two bags is selected at random and a ball is drawn from the bag which is found to be red. Find the probability that the ball is drawn from the first bag.
\\
\solution
		%\input{ncert/12/13/3/2/main.tex}
  \item
  Cards with numbers 2 to 101 are placed in a box. A card is selected at random.Find the probability that the card has
\begin{enumerate}[label=(\roman*)]
	\item an even number 
	\item a square number
\end{enumerate}
\solution
%\input{exemplar/10/13/3/32/main.tex}
\item
The king, queen and jack of clubs are removed from a deck of 52 playing cards and then well shuffled. Now one card is drawn at random from the remaining cards.  Determine the probability that the card is
\begin{enumerate}[label=(\roman*)]
\item a club
\item 10 of hearts
\end{enumerate}
\solution
%\input{exemplar/10/13/3/29/main.tex}
\item A team of medical students doing their internship have to assist during surgeries
at a city hospital. The probabilities of surgeries rated as very complex, complex,
routine, simple or very simple are respectively, 0.15, 0.20, 0.31, 0.26, .08. Find
the probabilities that a particular surgery will be rated
\begin{enumerate}
	\item complex or very complex;
	\item neither very complex nor very simple;
	\item routine or complex
	\item routine or simple
\end{enumerate}
\solution
%\input{exemplar/11/16/3/8(1)/main.tex}
\item A card is selected from a pack of 52 cards.
\begin{enumerate}[label=(\alph*)]
    \item How many points are there in the sample space?
    \item Calculate the probability that the card is an ace of spades.
    \item Calculate the probability that the card is (i) an ace and (ii) black card.
\end{enumerate}
\solution
%\input{exemplar/11/16/3/4/main2.tex}
\item The probability that a non leap year selected at random will contain 53 sundays.
\\
\solution
%\input{exemplar/10/13/1/19/main.tex}
\item One of the four persons John, Rita, Aslam or Gurpreet will be promoted next
month. Consequently the sample space consists of four elementary outcomes
S = {John promoted, Rita promoted, Aslam promoted, Gurpreet promoted}
You are told that the chances of John’s promotion is same as that of Gurpreet,
Rita’s chances of promotion are twice as likely as Johns. Aslam’s chances are
four times that of John.
\begin{enumerate}
	\item Determine
	\begin{enumerate}
		\item P (John promoted)
		\item P (Rita promoted)
		\item P (Aslam promoted)
		\item P (Gurpreet promoted)
	\end{enumerate}
	\item If A = {John promoted or Gurpreet promoted}, find P (A).
\end{enumerate}
\solution
%\input{exemplar/11/16/3/10/main.tex}
\item A card is drawn from a deck of 52 cards. Find the probability of getting a king or a heart or a red card.\\
\solution
%\input{exemplar/11/16/3/15/main.tex}
\item The probability that a student will pass his examination is 0.73, the probability of
the student getting a compartment is 0.13, and the probability that the student will
either pass or get compartment is 0.96. State True or False.\\
\solution
%\input{exemplar/11/16/3/31/main.tex}
\item A card is selected from a pack of 52 cards\\
\begin{enumerate}[label=(\alph*)]
\item How many points are there in the sample space?
\item Calculate the probability that the cards is an ace of spades.
\item Calculate the probability that the card is (i) an ace (ii)black card.\\
\end{enumerate}
%\input{ncert/11/16/3/4_1/Prob_4.tex}
\item In a non-leap year, the probability of having 53 tuesdays or 53 wednesdays is\\
\solution
%\input{exemplar/11/16/3/18/main.tex}
\item There are 1000 sealed envelopes in a box, 10 of them contain a cash prize of
Rs 100 each, 100 of them contain a cash prize of Rs 50 each and 200 of them
contain a cash prize of Rs 10 each and rest do not contain any cash prize. If they
are well shuffled and an envelope is picked up out, what is the probability that it
contains no cash prize?\\
\solution
%\input{exemplar/10/13/3/34/main.tex}
\item 
A die is thrown and a card is selected at random from a deck of 52 playing cards. The probability of getting an even number on the die and a spade card.\\
\solution
%\input{exemplar/12/13/3/78/main.tex}
\item
If 4-digit numbers greater than 5,000 are randomly formed from the digits 0, 1, 3, 5, and 7, what is the probability of forming a number divisible by 5 when:
\begin{enumerate}
    \item The digits are repeated?
    \item The repetition of digits is not allowed?
\end{enumerate}
\solution
%\input{ncert/11/16/4/9/main.tex}
\item Consider the probability space $\brak{\Omega, \mathcal{G}, P}$ where $\Omega = [0,2]$ and $\mathcal{G} = \cbrak{\phi, \Omega, [0,1], (1,2]}$. Let $X$ and $Y$ be two functions on $\Omega$ defined as
\begin{align*}
    X(\omega) = 
    \begin{cases}
        1 & \text{if }\omega \in [0, 1]\\
        2 & \text{if }\omega \in (1, 2]
    \end{cases}
\end{align*}
and
\begin{align*}
    Y(\omega) = 
    \begin{cases}
        2 & \text{if }\omega \in [0, 1.5]\\
        3 & \text{if }\omega \in (1.5, 2].
    \end{cases}
\end{align*}
Then which one of the following statements is true?
\begin{enumerate}
    \item [(A)] $X$ is a random variable with respect to $\mathcal{G}$, but $Y$ is not a random variable with respect to $\mathcal{G}$.
    \item [(B)] $Y$ is a random variable with respect to $\mathcal{G}$, but $X$ is not a random variable with respect to $\mathcal{G}$.
    \item [(C)] Neither $X$ nor $Y$ is a random variable with respect to $\mathcal{G}$.
    \item [(D)] Both $X$ and $Y$ are random variables with respect to $\mathcal{G}$.
\end{enumerate} \hfill (GATE ST 2023)\\
\solution
%\input{gate/ST/2023/14/main.tex}
	\item  A die is loaded in such a way that each odd number is twice as likely to occur as
each even number. Find $P(G)$, where $G$ is the event that a number greater than
3 occurs on a single roll of the die.
\\
\solution
		%\input{exemplar/11/16/3/5/main.tex}
	\item All the jacks, queens and kings are removed from a deck of 52 playing cards. The remaining cards are well shuffled and then one card is drawn at random. Giving ace a value 1 similar value for other cards, find the probability that the card has a value 
		\begin{enumerate}
			\item 7
			\item greater than 7
			\item less than 7
		\end{enumerate}
		%\input{exemplar/10/13/3/30/main.tex}
  \item A Lot consists of 48 mobile phones of which 42 are good, 3 have only minor defects and 3 have major defects.Varnika will buy a phone if it is good but the trader will only buy a mobile if it has no major defects. One phone is selected at random from the lot. What is the probability that it is
\begin{enumerate}
	\item acceptable to Varnika?
            \item acceptable to the trader?
\end{enumerate}
\solution
	%\input{exemplar/10/13/3/40/main.tex}
 \item A student says that if you throw a die, it will show up 1 or not 1. Therefore, the probability of getting 1 and the probability of getting 'not 1' each is equal to $\frac{1}{2}$. Is this correct? Give reasons.\\
 \solution
        %\input{exemplar/10/13/2/9/main.tex}
   \item Four candidates A, B, C, D have ap-
plied for the assignment to coach a school cricket
team. If A is twice as likely to be selected as B, and
B and C are given about the same chance of being
selected, while C is twice as likely to be selected
as D, what are the probabilities that
\begin{enumerate}
\item C will be selected?
\item A will not be selected?
\end{enumerate}
	%\input{exemplar/11/16/3/9/main.tex}
 \item A bag contain 24 balls of which $x$ balls are red, $2x$ are white and $3x$ are blue. A ball is selected at random, What is the probability that it is
\begin{enumerate}[label=\alph*)]
\item not red ?
\item white ?
\end{enumerate}
%\input{exemplar/10/13/3/41/main.tex}
If the letters of the word ASSASSINATION are arranged at random. Find the Probability that
\begin{enumerate}[label=(\alph*)]
\item Four $S's$ come consecutively in the word
\item Two  $I's$ and two $N's$ come together
\item All $A's$ are not coming together
\item No two $A's$ are coming together
\end{enumerate}
%\input{exemplar/11/16/3/14/main.tex}
	\item One urn contains two black balls (labelled B1 and B2) and one white ball. A
	second urn contains one black ball and two white balls (labelled W1 and W2).
	Suppose the following experiment is performed. One of the two urns is chosen
	at random. Next a ball is randomly chosen from the urn. Then a second ball is
	chosen at random from the same urn without replacing the first ball.
	
	\begin{enumerate}
	\item What is the probability that two black balls are chosen?
	
	\item What is the probability that two balls of opposite colour are chosen?
	\end{enumerate}
	\solution
	%\input{exemplar/11/16/3/12/main1.tex}
\end{enumerate}

		\item A box of oranges is inspected by examining three randomly selected oranges drawn without replacement. If all the three oranges are good, the box is approved for sale, otherwise, it is rejected. Find the probability that a box containing 15 oranges out of which 12 are good and 3 are bad ones will be approved for sale.
		\label{ncert/12/13/2/3/defs.tex}
		\item Two balls are drawn at random with replacement from a box containing 10 black and 8 red balls. Find the probability that
		\label{ncert/12/13/2/12}
\begin{enumerate}
\item both balls are red.
\item first ball is black and second is red.
\item one of them is black and other is red.
\end{enumerate}

\item In a hostel, 60\% of the students read Hindi newspaper, 40\% read English newspaper and 20\% read both Hindi and English newspapers. A student is selected at random.
		\label{ncert/12/13/2/15}
\begin{enumerate}
\item Find the probability that she reads neither Hindi nor English newspapers.
\item If she reads Hindi newspaper, find the probability that she reads English newspaper.
\item If she reads English newspaper, find the probability that she reads Hindi newspaper.\\
\end{enumerate}
\item The probability of obtaining an even prime number on each die, when a pair of dice is rolled is 
\begin{enumerate}
    \item $0$ 
    
    \item $\frac{1}{3}$ 
    
    \item $\frac{1}{12}$ 
    
    \item $\frac{1}{36}$ 
\end{enumerate}
\solution
		%\begin{enumerate}[label=\thesection.\arabic*,ref=\thesection.\theenumi]
	\item One card is drawn from a well-shuffled deck of 52 cards. Find the probability of getting
\begin{enumerate}
\item A king of red colour 
\item A face card 
\item A red face card
\item The jack of hearts
\item A spade
\item The queen of diamonds

\end{enumerate}
\solution
		%\input{ncert/10/15/1/14/main.tex}
	\item Five cards—the ten, jack, queen, king and ace of diamonds, are well-shuffled with their face downwards. One card is then picked up at random.
\begin{enumerate}
\item
What is the probability that the card is the queen? 
\item
If the queen is drawn and put aside, what is the probability that the second card picked up is (a) an ace? (b) a queen?\\
\end{enumerate}
\solution
		%\input{ncert/10/15/1/15/defs.tex}
	\item A bag contains $5$ red balls and some blue balls. If the probability of drawing a blue ball is double that if a red ball, determine the number of blue balls in the bag. 
		\\
\solution
		%\input{ncert/10/15/2/3/defs.tex}
	\item A card is selected from a pack of 52 cards.
 \begin{enumerate}[label=(\alph*)] 
                 \item How many points are there in the sample space?
                 \item Calculate the probability that the card is an ace of spades.
                 \item Calculate the probability that the card is (i) an ace and (ii) black card.
 \end{enumerate}
\solution
		%\input{ncert/11/16/3/4/main.tex}
\item Four cards are drawn from a well-shuffled deck of 52 cards. What is the probability of obtaining 3 diamonds and one spade.
\\
\solution
		%\input{ncert/11/16/4/2/defs.tex}
\item In a certain lottery 10,000 tickets are sold and ten equal prizes are awarded. What is the probability of not getting a prize if you buy (a) one ticket (b) two tickets (c) 10 tickets ?	
\\
\solution
		%\input{ncert/11/16/4/4/defs.tex}
		%
\item 
Out of 100 students, two sections of 40 and 60 are formed. If you and your friend are among the 100 students, what is the probability that
\begin{enumerate}
\item you both enter the same section?
\item you both enter the different sections?
\end{enumerate}
\solution
		%\input{ncert/11/16/4/5/defs.tex}
	\item 
The number lock of a suitcase has 4 wheels each labelled with ten digits i.e. from 0 to 9.The lock opens with a sequence of four digits with no repeats.What is the probability of a person getting the right sequence to open the suitcase.
\\
\solution
		%\input{ncert/11/16/4/10/defs.tex}
		%
\item 
Two cards are drawn at random and without replacement from a pack of 52 playing cards. Find the probability that both the cards are black.
\\
\solution
		%\input{ncert/12/13/2/2/defs.tex}
		\item A box of oranges is inspected by examining three randomly selected oranges drawn without replacement. If all the three oranges are good, the box is approved for sale, otherwise, it is rejected. Find the probability that a box containing 15 oranges out of which 12 are good and 3 are bad ones will be approved for sale.
		\label{ncert/12/13/2/3/defs.tex}
		\item Two balls are drawn at random with replacement from a box containing 10 black and 8 red balls. Find the probability that
		\label{ncert/12/13/2/12}
\begin{enumerate}
\item both balls are red.
\item first ball is black and second is red.
\item one of them is black and other is red.
\end{enumerate}

\item In a hostel, 60\% of the students read Hindi newspaper, 40\% read English newspaper and 20\% read both Hindi and English newspapers. A student is selected at random.
		\label{ncert/12/13/2/15}
\begin{enumerate}
\item Find the probability that she reads neither Hindi nor English newspapers.
\item If she reads Hindi newspaper, find the probability that she reads English newspaper.
\item If she reads English newspaper, find the probability that she reads Hindi newspaper.\\
\end{enumerate}
\item The probability of obtaining an even prime number on each die, when a pair of dice is rolled is 
\begin{enumerate}
    \item $0$ 
    
    \item $\frac{1}{3}$ 
    
    \item $\frac{1}{12}$ 
    
    \item $\frac{1}{36}$ 
\end{enumerate}
\solution
		%\input{ncert/12/13/2/17/defs.tex}
	\item A bag contains 4 red and 4 black balls, another bag contains 2 red and 6 black balls. One of the two bags is selected at random and a ball is drawn from the bag which is found to be red. Find the probability that the ball is drawn from the first bag.
\\
\solution
		%\input{ncert/12/13/3/2/main.tex}
  \item
  Cards with numbers 2 to 101 are placed in a box. A card is selected at random.Find the probability that the card has
\begin{enumerate}[label=(\roman*)]
	\item an even number 
	\item a square number
\end{enumerate}
\solution
%\input{exemplar/10/13/3/32/main.tex}
\item
The king, queen and jack of clubs are removed from a deck of 52 playing cards and then well shuffled. Now one card is drawn at random from the remaining cards.  Determine the probability that the card is
\begin{enumerate}[label=(\roman*)]
\item a club
\item 10 of hearts
\end{enumerate}
\solution
%\input{exemplar/10/13/3/29/main.tex}
\item A team of medical students doing their internship have to assist during surgeries
at a city hospital. The probabilities of surgeries rated as very complex, complex,
routine, simple or very simple are respectively, 0.15, 0.20, 0.31, 0.26, .08. Find
the probabilities that a particular surgery will be rated
\begin{enumerate}
	\item complex or very complex;
	\item neither very complex nor very simple;
	\item routine or complex
	\item routine or simple
\end{enumerate}
\solution
%\input{exemplar/11/16/3/8(1)/main.tex}
\item A card is selected from a pack of 52 cards.
\begin{enumerate}[label=(\alph*)]
    \item How many points are there in the sample space?
    \item Calculate the probability that the card is an ace of spades.
    \item Calculate the probability that the card is (i) an ace and (ii) black card.
\end{enumerate}
\solution
%\input{exemplar/11/16/3/4/main2.tex}
\item The probability that a non leap year selected at random will contain 53 sundays.
\\
\solution
%\input{exemplar/10/13/1/19/main.tex}
\item One of the four persons John, Rita, Aslam or Gurpreet will be promoted next
month. Consequently the sample space consists of four elementary outcomes
S = {John promoted, Rita promoted, Aslam promoted, Gurpreet promoted}
You are told that the chances of John’s promotion is same as that of Gurpreet,
Rita’s chances of promotion are twice as likely as Johns. Aslam’s chances are
four times that of John.
\begin{enumerate}
	\item Determine
	\begin{enumerate}
		\item P (John promoted)
		\item P (Rita promoted)
		\item P (Aslam promoted)
		\item P (Gurpreet promoted)
	\end{enumerate}
	\item If A = {John promoted or Gurpreet promoted}, find P (A).
\end{enumerate}
\solution
%\input{exemplar/11/16/3/10/main.tex}
\item A card is drawn from a deck of 52 cards. Find the probability of getting a king or a heart or a red card.\\
\solution
%\input{exemplar/11/16/3/15/main.tex}
\item The probability that a student will pass his examination is 0.73, the probability of
the student getting a compartment is 0.13, and the probability that the student will
either pass or get compartment is 0.96. State True or False.\\
\solution
%\input{exemplar/11/16/3/31/main.tex}
\item A card is selected from a pack of 52 cards\\
\begin{enumerate}[label=(\alph*)]
\item How many points are there in the sample space?
\item Calculate the probability that the cards is an ace of spades.
\item Calculate the probability that the card is (i) an ace (ii)black card.\\
\end{enumerate}
%\input{ncert/11/16/3/4_1/Prob_4.tex}
\item In a non-leap year, the probability of having 53 tuesdays or 53 wednesdays is\\
\solution
%\input{exemplar/11/16/3/18/main.tex}
\item There are 1000 sealed envelopes in a box, 10 of them contain a cash prize of
Rs 100 each, 100 of them contain a cash prize of Rs 50 each and 200 of them
contain a cash prize of Rs 10 each and rest do not contain any cash prize. If they
are well shuffled and an envelope is picked up out, what is the probability that it
contains no cash prize?\\
\solution
%\input{exemplar/10/13/3/34/main.tex}
\item 
A die is thrown and a card is selected at random from a deck of 52 playing cards. The probability of getting an even number on the die and a spade card.\\
\solution
%\input{exemplar/12/13/3/78/main.tex}
\item
If 4-digit numbers greater than 5,000 are randomly formed from the digits 0, 1, 3, 5, and 7, what is the probability of forming a number divisible by 5 when:
\begin{enumerate}
    \item The digits are repeated?
    \item The repetition of digits is not allowed?
\end{enumerate}
\solution
%\input{ncert/11/16/4/9/main.tex}
\item Consider the probability space $\brak{\Omega, \mathcal{G}, P}$ where $\Omega = [0,2]$ and $\mathcal{G} = \cbrak{\phi, \Omega, [0,1], (1,2]}$. Let $X$ and $Y$ be two functions on $\Omega$ defined as
\begin{align*}
    X(\omega) = 
    \begin{cases}
        1 & \text{if }\omega \in [0, 1]\\
        2 & \text{if }\omega \in (1, 2]
    \end{cases}
\end{align*}
and
\begin{align*}
    Y(\omega) = 
    \begin{cases}
        2 & \text{if }\omega \in [0, 1.5]\\
        3 & \text{if }\omega \in (1.5, 2].
    \end{cases}
\end{align*}
Then which one of the following statements is true?
\begin{enumerate}
    \item [(A)] $X$ is a random variable with respect to $\mathcal{G}$, but $Y$ is not a random variable with respect to $\mathcal{G}$.
    \item [(B)] $Y$ is a random variable with respect to $\mathcal{G}$, but $X$ is not a random variable with respect to $\mathcal{G}$.
    \item [(C)] Neither $X$ nor $Y$ is a random variable with respect to $\mathcal{G}$.
    \item [(D)] Both $X$ and $Y$ are random variables with respect to $\mathcal{G}$.
\end{enumerate} \hfill (GATE ST 2023)\\
\solution
%\input{gate/ST/2023/14/main.tex}
	\item  A die is loaded in such a way that each odd number is twice as likely to occur as
each even number. Find $P(G)$, where $G$ is the event that a number greater than
3 occurs on a single roll of the die.
\\
\solution
		%\input{exemplar/11/16/3/5/main.tex}
	\item All the jacks, queens and kings are removed from a deck of 52 playing cards. The remaining cards are well shuffled and then one card is drawn at random. Giving ace a value 1 similar value for other cards, find the probability that the card has a value 
		\begin{enumerate}
			\item 7
			\item greater than 7
			\item less than 7
		\end{enumerate}
		%\input{exemplar/10/13/3/30/main.tex}
  \item A Lot consists of 48 mobile phones of which 42 are good, 3 have only minor defects and 3 have major defects.Varnika will buy a phone if it is good but the trader will only buy a mobile if it has no major defects. One phone is selected at random from the lot. What is the probability that it is
\begin{enumerate}
	\item acceptable to Varnika?
            \item acceptable to the trader?
\end{enumerate}
\solution
	%\input{exemplar/10/13/3/40/main.tex}
 \item A student says that if you throw a die, it will show up 1 or not 1. Therefore, the probability of getting 1 and the probability of getting 'not 1' each is equal to $\frac{1}{2}$. Is this correct? Give reasons.\\
 \solution
        %\input{exemplar/10/13/2/9/main.tex}
   \item Four candidates A, B, C, D have ap-
plied for the assignment to coach a school cricket
team. If A is twice as likely to be selected as B, and
B and C are given about the same chance of being
selected, while C is twice as likely to be selected
as D, what are the probabilities that
\begin{enumerate}
\item C will be selected?
\item A will not be selected?
\end{enumerate}
	%\input{exemplar/11/16/3/9/main.tex}
 \item A bag contain 24 balls of which $x$ balls are red, $2x$ are white and $3x$ are blue. A ball is selected at random, What is the probability that it is
\begin{enumerate}[label=\alph*)]
\item not red ?
\item white ?
\end{enumerate}
%\input{exemplar/10/13/3/41/main.tex}
If the letters of the word ASSASSINATION are arranged at random. Find the Probability that
\begin{enumerate}[label=(\alph*)]
\item Four $S's$ come consecutively in the word
\item Two  $I's$ and two $N's$ come together
\item All $A's$ are not coming together
\item No two $A's$ are coming together
\end{enumerate}
%\input{exemplar/11/16/3/14/main.tex}
	\item One urn contains two black balls (labelled B1 and B2) and one white ball. A
	second urn contains one black ball and two white balls (labelled W1 and W2).
	Suppose the following experiment is performed. One of the two urns is chosen
	at random. Next a ball is randomly chosen from the urn. Then a second ball is
	chosen at random from the same urn without replacing the first ball.
	
	\begin{enumerate}
	\item What is the probability that two black balls are chosen?
	
	\item What is the probability that two balls of opposite colour are chosen?
	\end{enumerate}
	\solution
	%\input{exemplar/11/16/3/12/main1.tex}
\end{enumerate}

	\item A bag contains 4 red and 4 black balls, another bag contains 2 red and 6 black balls. One of the two bags is selected at random and a ball is drawn from the bag which is found to be red. Find the probability that the ball is drawn from the first bag.
\\
\solution
		%\begin{table}[H]
	\centering
\begin{tabular}{|c|c|c|}
\hline
Random variable &Value &Definition\\ \hline
\multirow{3}{*}{X} &0 &Slips of Rs 1\\
&1 &Slips of Rs 5\\
&2 &Slips of Rs 13\\ \hline
\multirow{2}{*}{Y} &0 &Box A\\
&1 &Box B\\\hline
\end{tabular}
\caption{}
\label{tab:Distribution}
\end{table}
See \tabref{tab:Distribution}.
\begin{align}
p_{Y}\brak{k}= \begin{cases} 
      \frac{1}{3} & {k=0} \\
      \frac{2}{3 }& {k=1} 
   \end{cases}
   \\
p_{Y|X}\brak{0|0} = \frac{19}{25}\, 
p_{Y|X}\brak{0|1} = \frac{6}{25}\,
p_{Y|X}\brak{1|0} = \frac{45}{50}\,
p_{Y|X}\brak{1|2} = \frac{5}{50}
\end{align}
The desired probability is the probability that a slip drawn at random is marked other than Rs 1,
\begin{align}
&=1-p_X\brak{0}\\
&= p_X(1) + p_X(2)
\end{align}
Using Bayes theorem,
\begin{align}
&= p_Y\brak{0} \times \pr{Y=0 | X=1} + p_Y\brak{1} \times \pr{Y=1|X=2}\\
&=\frac{1}{3} \times \frac{6}{25} + \frac{2}{3} \times \frac{5}{50}\\
&=\frac{11}{75}
\end{align}

\newpage

%\tableofcontents

\bigskip

\renewcommand{\thefigure}{\theenumi}
\renewcommand{\thetable}{\theenumi}
%\renewcommand{\theequation}{\theenumi}

%\begin{abstract}
%%\boldmath
%In this letter, an algorithm for evaluating the exact analytical bit error rate  (BER)  for the piecewise linear (PL) combiner for  multiple relays is presented. Previous results were available only for upto three relays. The algorithm is unique in the sense that  the actual mathematical expressions, that are prohibitively large, need not be explicitly obtained. The diversity gain due to multiple relays is shown through plots of the analytical BER, well supported by simulations. 
%
%\end{abstract}
% IEEEtran.cls defaults to using nonbold math in the Abstract.
% This preserves the distinction between vectors and scalars. However,
% if the journal you are submitting to favors bold math in the abstract,
% then you can use LaTeX's standard command \boldmath at the very start
% of the abstract to achieve this. Many IEEE journals frown on math
% in the abstract anyway.

% Note that keywords are not normally used for peerreview papers.
%\begin{IEEEkeywords}
%Cooperative diversity, decode and forward, piecewise linear
%\end{IEEEkeywords}



% For peer review papers, you can put extra information on the cover
% page as needed:
% \ifCLASSOPTIONpeerreview
% \begin{center} \bfseries EDICS Category: 3-BBND \end{center}
% \fi
%
% For peerreview papers, this IEEEtran command inserts a page break and
% creates the second title. It will be ignored for other modes.
%\IEEEpeerreviewmaketitle




  \item
  Cards with numbers 2 to 101 are placed in a box. A card is selected at random.Find the probability that the card has
\begin{enumerate}[label=(\roman*)]
	\item an even number 
	\item a square number
\end{enumerate}
\solution
%\begin{table}[H]
	\centering
\begin{tabular}{|c|c|c|}
\hline
Random variable &Value &Definition\\ \hline
\multirow{3}{*}{X} &0 &Slips of Rs 1\\
&1 &Slips of Rs 5\\
&2 &Slips of Rs 13\\ \hline
\multirow{2}{*}{Y} &0 &Box A\\
&1 &Box B\\\hline
\end{tabular}
\caption{}
\label{tab:Distribution}
\end{table}
See \tabref{tab:Distribution}.
\begin{align}
p_{Y}\brak{k}= \begin{cases} 
      \frac{1}{3} & {k=0} \\
      \frac{2}{3 }& {k=1} 
   \end{cases}
   \\
p_{Y|X}\brak{0|0} = \frac{19}{25}\, 
p_{Y|X}\brak{0|1} = \frac{6}{25}\,
p_{Y|X}\brak{1|0} = \frac{45}{50}\,
p_{Y|X}\brak{1|2} = \frac{5}{50}
\end{align}
The desired probability is the probability that a slip drawn at random is marked other than Rs 1,
\begin{align}
&=1-p_X\brak{0}\\
&= p_X(1) + p_X(2)
\end{align}
Using Bayes theorem,
\begin{align}
&= p_Y\brak{0} \times \pr{Y=0 | X=1} + p_Y\brak{1} \times \pr{Y=1|X=2}\\
&=\frac{1}{3} \times \frac{6}{25} + \frac{2}{3} \times \frac{5}{50}\\
&=\frac{11}{75}
\end{align}

\newpage

%\tableofcontents

\bigskip

\renewcommand{\thefigure}{\theenumi}
\renewcommand{\thetable}{\theenumi}
%\renewcommand{\theequation}{\theenumi}

%\begin{abstract}
%%\boldmath
%In this letter, an algorithm for evaluating the exact analytical bit error rate  (BER)  for the piecewise linear (PL) combiner for  multiple relays is presented. Previous results were available only for upto three relays. The algorithm is unique in the sense that  the actual mathematical expressions, that are prohibitively large, need not be explicitly obtained. The diversity gain due to multiple relays is shown through plots of the analytical BER, well supported by simulations. 
%
%\end{abstract}
% IEEEtran.cls defaults to using nonbold math in the Abstract.
% This preserves the distinction between vectors and scalars. However,
% if the journal you are submitting to favors bold math in the abstract,
% then you can use LaTeX's standard command \boldmath at the very start
% of the abstract to achieve this. Many IEEE journals frown on math
% in the abstract anyway.

% Note that keywords are not normally used for peerreview papers.
%\begin{IEEEkeywords}
%Cooperative diversity, decode and forward, piecewise linear
%\end{IEEEkeywords}



% For peer review papers, you can put extra information on the cover
% page as needed:
% \ifCLASSOPTIONpeerreview
% \begin{center} \bfseries EDICS Category: 3-BBND \end{center}
% \fi
%
% For peerreview papers, this IEEEtran command inserts a page break and
% creates the second title. It will be ignored for other modes.
%\IEEEpeerreviewmaketitle




\item
The king, queen and jack of clubs are removed from a deck of 52 playing cards and then well shuffled. Now one card is drawn at random from the remaining cards.  Determine the probability that the card is
\begin{enumerate}[label=(\roman*)]
\item a club
\item 10 of hearts
\end{enumerate}
\solution
%\begin{table}[H]
	\centering
\begin{tabular}{|c|c|c|}
\hline
Random variable &Value &Definition\\ \hline
\multirow{3}{*}{X} &0 &Slips of Rs 1\\
&1 &Slips of Rs 5\\
&2 &Slips of Rs 13\\ \hline
\multirow{2}{*}{Y} &0 &Box A\\
&1 &Box B\\\hline
\end{tabular}
\caption{}
\label{tab:Distribution}
\end{table}
See \tabref{tab:Distribution}.
\begin{align}
p_{Y}\brak{k}= \begin{cases} 
      \frac{1}{3} & {k=0} \\
      \frac{2}{3 }& {k=1} 
   \end{cases}
   \\
p_{Y|X}\brak{0|0} = \frac{19}{25}\, 
p_{Y|X}\brak{0|1} = \frac{6}{25}\,
p_{Y|X}\brak{1|0} = \frac{45}{50}\,
p_{Y|X}\brak{1|2} = \frac{5}{50}
\end{align}
The desired probability is the probability that a slip drawn at random is marked other than Rs 1,
\begin{align}
&=1-p_X\brak{0}\\
&= p_X(1) + p_X(2)
\end{align}
Using Bayes theorem,
\begin{align}
&= p_Y\brak{0} \times \pr{Y=0 | X=1} + p_Y\brak{1} \times \pr{Y=1|X=2}\\
&=\frac{1}{3} \times \frac{6}{25} + \frac{2}{3} \times \frac{5}{50}\\
&=\frac{11}{75}
\end{align}

\newpage

%\tableofcontents

\bigskip

\renewcommand{\thefigure}{\theenumi}
\renewcommand{\thetable}{\theenumi}
%\renewcommand{\theequation}{\theenumi}

%\begin{abstract}
%%\boldmath
%In this letter, an algorithm for evaluating the exact analytical bit error rate  (BER)  for the piecewise linear (PL) combiner for  multiple relays is presented. Previous results were available only for upto three relays. The algorithm is unique in the sense that  the actual mathematical expressions, that are prohibitively large, need not be explicitly obtained. The diversity gain due to multiple relays is shown through plots of the analytical BER, well supported by simulations. 
%
%\end{abstract}
% IEEEtran.cls defaults to using nonbold math in the Abstract.
% This preserves the distinction between vectors and scalars. However,
% if the journal you are submitting to favors bold math in the abstract,
% then you can use LaTeX's standard command \boldmath at the very start
% of the abstract to achieve this. Many IEEE journals frown on math
% in the abstract anyway.

% Note that keywords are not normally used for peerreview papers.
%\begin{IEEEkeywords}
%Cooperative diversity, decode and forward, piecewise linear
%\end{IEEEkeywords}



% For peer review papers, you can put extra information on the cover
% page as needed:
% \ifCLASSOPTIONpeerreview
% \begin{center} \bfseries EDICS Category: 3-BBND \end{center}
% \fi
%
% For peerreview papers, this IEEEtran command inserts a page break and
% creates the second title. It will be ignored for other modes.
%\IEEEpeerreviewmaketitle




\item A team of medical students doing their internship have to assist during surgeries
at a city hospital. The probabilities of surgeries rated as very complex, complex,
routine, simple or very simple are respectively, 0.15, 0.20, 0.31, 0.26, .08. Find
the probabilities that a particular surgery will be rated
\begin{enumerate}
	\item complex or very complex;
	\item neither very complex nor very simple;
	\item routine or complex
	\item routine or simple
\end{enumerate}
\solution
%\begin{table}[H]
	\centering
\begin{tabular}{|c|c|c|}
\hline
Random variable &Value &Definition\\ \hline
\multirow{3}{*}{X} &0 &Slips of Rs 1\\
&1 &Slips of Rs 5\\
&2 &Slips of Rs 13\\ \hline
\multirow{2}{*}{Y} &0 &Box A\\
&1 &Box B\\\hline
\end{tabular}
\caption{}
\label{tab:Distribution}
\end{table}
See \tabref{tab:Distribution}.
\begin{align}
p_{Y}\brak{k}= \begin{cases} 
      \frac{1}{3} & {k=0} \\
      \frac{2}{3 }& {k=1} 
   \end{cases}
   \\
p_{Y|X}\brak{0|0} = \frac{19}{25}\, 
p_{Y|X}\brak{0|1} = \frac{6}{25}\,
p_{Y|X}\brak{1|0} = \frac{45}{50}\,
p_{Y|X}\brak{1|2} = \frac{5}{50}
\end{align}
The desired probability is the probability that a slip drawn at random is marked other than Rs 1,
\begin{align}
&=1-p_X\brak{0}\\
&= p_X(1) + p_X(2)
\end{align}
Using Bayes theorem,
\begin{align}
&= p_Y\brak{0} \times \pr{Y=0 | X=1} + p_Y\brak{1} \times \pr{Y=1|X=2}\\
&=\frac{1}{3} \times \frac{6}{25} + \frac{2}{3} \times \frac{5}{50}\\
&=\frac{11}{75}
\end{align}

\newpage

%\tableofcontents

\bigskip

\renewcommand{\thefigure}{\theenumi}
\renewcommand{\thetable}{\theenumi}
%\renewcommand{\theequation}{\theenumi}

%\begin{abstract}
%%\boldmath
%In this letter, an algorithm for evaluating the exact analytical bit error rate  (BER)  for the piecewise linear (PL) combiner for  multiple relays is presented. Previous results were available only for upto three relays. The algorithm is unique in the sense that  the actual mathematical expressions, that are prohibitively large, need not be explicitly obtained. The diversity gain due to multiple relays is shown through plots of the analytical BER, well supported by simulations. 
%
%\end{abstract}
% IEEEtran.cls defaults to using nonbold math in the Abstract.
% This preserves the distinction between vectors and scalars. However,
% if the journal you are submitting to favors bold math in the abstract,
% then you can use LaTeX's standard command \boldmath at the very start
% of the abstract to achieve this. Many IEEE journals frown on math
% in the abstract anyway.

% Note that keywords are not normally used for peerreview papers.
%\begin{IEEEkeywords}
%Cooperative diversity, decode and forward, piecewise linear
%\end{IEEEkeywords}



% For peer review papers, you can put extra information on the cover
% page as needed:
% \ifCLASSOPTIONpeerreview
% \begin{center} \bfseries EDICS Category: 3-BBND \end{center}
% \fi
%
% For peerreview papers, this IEEEtran command inserts a page break and
% creates the second title. It will be ignored for other modes.
%\IEEEpeerreviewmaketitle




\item A card is selected from a pack of 52 cards.
\begin{enumerate}[label=(\alph*)]
    \item How many points are there in the sample space?
    \item Calculate the probability that the card is an ace of spades.
    \item Calculate the probability that the card is (i) an ace and (ii) black card.
\end{enumerate}
\solution
%Let $X$ be an bernoulli rv defined as in \tabref{tab:exemplar/11/16/3/26}.  Then, 
\begin{equation}
    p =
        \frac{4}{11} 
\end{equation}
\begin{table}[H]
	\centering
	\input{exemplar/11/16/3/26/tables/Table2.tex}
	\caption{}
        \label{tab:exemplar/11/16/3/26}
\end{table}

\item The probability that a non leap year selected at random will contain 53 sundays.
\\
\solution
%\begin{table}[H]
	\centering
\begin{tabular}{|c|c|c|}
\hline
Random variable &Value &Definition\\ \hline
\multirow{3}{*}{X} &0 &Slips of Rs 1\\
&1 &Slips of Rs 5\\
&2 &Slips of Rs 13\\ \hline
\multirow{2}{*}{Y} &0 &Box A\\
&1 &Box B\\\hline
\end{tabular}
\caption{}
\label{tab:Distribution}
\end{table}
See \tabref{tab:Distribution}.
\begin{align}
p_{Y}\brak{k}= \begin{cases} 
      \frac{1}{3} & {k=0} \\
      \frac{2}{3 }& {k=1} 
   \end{cases}
   \\
p_{Y|X}\brak{0|0} = \frac{19}{25}\, 
p_{Y|X}\brak{0|1} = \frac{6}{25}\,
p_{Y|X}\brak{1|0} = \frac{45}{50}\,
p_{Y|X}\brak{1|2} = \frac{5}{50}
\end{align}
The desired probability is the probability that a slip drawn at random is marked other than Rs 1,
\begin{align}
&=1-p_X\brak{0}\\
&= p_X(1) + p_X(2)
\end{align}
Using Bayes theorem,
\begin{align}
&= p_Y\brak{0} \times \pr{Y=0 | X=1} + p_Y\brak{1} \times \pr{Y=1|X=2}\\
&=\frac{1}{3} \times \frac{6}{25} + \frac{2}{3} \times \frac{5}{50}\\
&=\frac{11}{75}
\end{align}

\newpage

%\tableofcontents

\bigskip

\renewcommand{\thefigure}{\theenumi}
\renewcommand{\thetable}{\theenumi}
%\renewcommand{\theequation}{\theenumi}

%\begin{abstract}
%%\boldmath
%In this letter, an algorithm for evaluating the exact analytical bit error rate  (BER)  for the piecewise linear (PL) combiner for  multiple relays is presented. Previous results were available only for upto three relays. The algorithm is unique in the sense that  the actual mathematical expressions, that are prohibitively large, need not be explicitly obtained. The diversity gain due to multiple relays is shown through plots of the analytical BER, well supported by simulations. 
%
%\end{abstract}
% IEEEtran.cls defaults to using nonbold math in the Abstract.
% This preserves the distinction between vectors and scalars. However,
% if the journal you are submitting to favors bold math in the abstract,
% then you can use LaTeX's standard command \boldmath at the very start
% of the abstract to achieve this. Many IEEE journals frown on math
% in the abstract anyway.

% Note that keywords are not normally used for peerreview papers.
%\begin{IEEEkeywords}
%Cooperative diversity, decode and forward, piecewise linear
%\end{IEEEkeywords}



% For peer review papers, you can put extra information on the cover
% page as needed:
% \ifCLASSOPTIONpeerreview
% \begin{center} \bfseries EDICS Category: 3-BBND \end{center}
% \fi
%
% For peerreview papers, this IEEEtran command inserts a page break and
% creates the second title. It will be ignored for other modes.
%\IEEEpeerreviewmaketitle




\item One of the four persons John, Rita, Aslam or Gurpreet will be promoted next
month. Consequently the sample space consists of four elementary outcomes
S = {John promoted, Rita promoted, Aslam promoted, Gurpreet promoted}
You are told that the chances of John’s promotion is same as that of Gurpreet,
Rita’s chances of promotion are twice as likely as Johns. Aslam’s chances are
four times that of John.
\begin{enumerate}
	\item Determine
	\begin{enumerate}
		\item P (John promoted)
		\item P (Rita promoted)
		\item P (Aslam promoted)
		\item P (Gurpreet promoted)
	\end{enumerate}
	\item If A = {John promoted or Gurpreet promoted}, find P (A).
\end{enumerate}
\solution
%\begin{table}[H]
	\centering
\begin{tabular}{|c|c|c|}
\hline
Random variable &Value &Definition\\ \hline
\multirow{3}{*}{X} &0 &Slips of Rs 1\\
&1 &Slips of Rs 5\\
&2 &Slips of Rs 13\\ \hline
\multirow{2}{*}{Y} &0 &Box A\\
&1 &Box B\\\hline
\end{tabular}
\caption{}
\label{tab:Distribution}
\end{table}
See \tabref{tab:Distribution}.
\begin{align}
p_{Y}\brak{k}= \begin{cases} 
      \frac{1}{3} & {k=0} \\
      \frac{2}{3 }& {k=1} 
   \end{cases}
   \\
p_{Y|X}\brak{0|0} = \frac{19}{25}\, 
p_{Y|X}\brak{0|1} = \frac{6}{25}\,
p_{Y|X}\brak{1|0} = \frac{45}{50}\,
p_{Y|X}\brak{1|2} = \frac{5}{50}
\end{align}
The desired probability is the probability that a slip drawn at random is marked other than Rs 1,
\begin{align}
&=1-p_X\brak{0}\\
&= p_X(1) + p_X(2)
\end{align}
Using Bayes theorem,
\begin{align}
&= p_Y\brak{0} \times \pr{Y=0 | X=1} + p_Y\brak{1} \times \pr{Y=1|X=2}\\
&=\frac{1}{3} \times \frac{6}{25} + \frac{2}{3} \times \frac{5}{50}\\
&=\frac{11}{75}
\end{align}

\newpage

%\tableofcontents

\bigskip

\renewcommand{\thefigure}{\theenumi}
\renewcommand{\thetable}{\theenumi}
%\renewcommand{\theequation}{\theenumi}

%\begin{abstract}
%%\boldmath
%In this letter, an algorithm for evaluating the exact analytical bit error rate  (BER)  for the piecewise linear (PL) combiner for  multiple relays is presented. Previous results were available only for upto three relays. The algorithm is unique in the sense that  the actual mathematical expressions, that are prohibitively large, need not be explicitly obtained. The diversity gain due to multiple relays is shown through plots of the analytical BER, well supported by simulations. 
%
%\end{abstract}
% IEEEtran.cls defaults to using nonbold math in the Abstract.
% This preserves the distinction between vectors and scalars. However,
% if the journal you are submitting to favors bold math in the abstract,
% then you can use LaTeX's standard command \boldmath at the very start
% of the abstract to achieve this. Many IEEE journals frown on math
% in the abstract anyway.

% Note that keywords are not normally used for peerreview papers.
%\begin{IEEEkeywords}
%Cooperative diversity, decode and forward, piecewise linear
%\end{IEEEkeywords}



% For peer review papers, you can put extra information on the cover
% page as needed:
% \ifCLASSOPTIONpeerreview
% \begin{center} \bfseries EDICS Category: 3-BBND \end{center}
% \fi
%
% For peerreview papers, this IEEEtran command inserts a page break and
% creates the second title. It will be ignored for other modes.
%\IEEEpeerreviewmaketitle




\item A card is drawn from a deck of 52 cards. Find the probability of getting a king or a heart or a red card.\\
\solution
%\begin{table}[H]
	\centering
\begin{tabular}{|c|c|c|}
\hline
Random variable &Value &Definition\\ \hline
\multirow{3}{*}{X} &0 &Slips of Rs 1\\
&1 &Slips of Rs 5\\
&2 &Slips of Rs 13\\ \hline
\multirow{2}{*}{Y} &0 &Box A\\
&1 &Box B\\\hline
\end{tabular}
\caption{}
\label{tab:Distribution}
\end{table}
See \tabref{tab:Distribution}.
\begin{align}
p_{Y}\brak{k}= \begin{cases} 
      \frac{1}{3} & {k=0} \\
      \frac{2}{3 }& {k=1} 
   \end{cases}
   \\
p_{Y|X}\brak{0|0} = \frac{19}{25}\, 
p_{Y|X}\brak{0|1} = \frac{6}{25}\,
p_{Y|X}\brak{1|0} = \frac{45}{50}\,
p_{Y|X}\brak{1|2} = \frac{5}{50}
\end{align}
The desired probability is the probability that a slip drawn at random is marked other than Rs 1,
\begin{align}
&=1-p_X\brak{0}\\
&= p_X(1) + p_X(2)
\end{align}
Using Bayes theorem,
\begin{align}
&= p_Y\brak{0} \times \pr{Y=0 | X=1} + p_Y\brak{1} \times \pr{Y=1|X=2}\\
&=\frac{1}{3} \times \frac{6}{25} + \frac{2}{3} \times \frac{5}{50}\\
&=\frac{11}{75}
\end{align}

\newpage

%\tableofcontents

\bigskip

\renewcommand{\thefigure}{\theenumi}
\renewcommand{\thetable}{\theenumi}
%\renewcommand{\theequation}{\theenumi}

%\begin{abstract}
%%\boldmath
%In this letter, an algorithm for evaluating the exact analytical bit error rate  (BER)  for the piecewise linear (PL) combiner for  multiple relays is presented. Previous results were available only for upto three relays. The algorithm is unique in the sense that  the actual mathematical expressions, that are prohibitively large, need not be explicitly obtained. The diversity gain due to multiple relays is shown through plots of the analytical BER, well supported by simulations. 
%
%\end{abstract}
% IEEEtran.cls defaults to using nonbold math in the Abstract.
% This preserves the distinction between vectors and scalars. However,
% if the journal you are submitting to favors bold math in the abstract,
% then you can use LaTeX's standard command \boldmath at the very start
% of the abstract to achieve this. Many IEEE journals frown on math
% in the abstract anyway.

% Note that keywords are not normally used for peerreview papers.
%\begin{IEEEkeywords}
%Cooperative diversity, decode and forward, piecewise linear
%\end{IEEEkeywords}



% For peer review papers, you can put extra information on the cover
% page as needed:
% \ifCLASSOPTIONpeerreview
% \begin{center} \bfseries EDICS Category: 3-BBND \end{center}
% \fi
%
% For peerreview papers, this IEEEtran command inserts a page break and
% creates the second title. It will be ignored for other modes.
%\IEEEpeerreviewmaketitle




\item The probability that a student will pass his examination is 0.73, the probability of
the student getting a compartment is 0.13, and the probability that the student will
either pass or get compartment is 0.96. State True or False.\\
\solution
%\begin{table}[H]
	\centering
\begin{tabular}{|c|c|c|}
\hline
Random variable &Value &Definition\\ \hline
\multirow{3}{*}{X} &0 &Slips of Rs 1\\
&1 &Slips of Rs 5\\
&2 &Slips of Rs 13\\ \hline
\multirow{2}{*}{Y} &0 &Box A\\
&1 &Box B\\\hline
\end{tabular}
\caption{}
\label{tab:Distribution}
\end{table}
See \tabref{tab:Distribution}.
\begin{align}
p_{Y}\brak{k}= \begin{cases} 
      \frac{1}{3} & {k=0} \\
      \frac{2}{3 }& {k=1} 
   \end{cases}
   \\
p_{Y|X}\brak{0|0} = \frac{19}{25}\, 
p_{Y|X}\brak{0|1} = \frac{6}{25}\,
p_{Y|X}\brak{1|0} = \frac{45}{50}\,
p_{Y|X}\brak{1|2} = \frac{5}{50}
\end{align}
The desired probability is the probability that a slip drawn at random is marked other than Rs 1,
\begin{align}
&=1-p_X\brak{0}\\
&= p_X(1) + p_X(2)
\end{align}
Using Bayes theorem,
\begin{align}
&= p_Y\brak{0} \times \pr{Y=0 | X=1} + p_Y\brak{1} \times \pr{Y=1|X=2}\\
&=\frac{1}{3} \times \frac{6}{25} + \frac{2}{3} \times \frac{5}{50}\\
&=\frac{11}{75}
\end{align}

\newpage

%\tableofcontents

\bigskip

\renewcommand{\thefigure}{\theenumi}
\renewcommand{\thetable}{\theenumi}
%\renewcommand{\theequation}{\theenumi}

%\begin{abstract}
%%\boldmath
%In this letter, an algorithm for evaluating the exact analytical bit error rate  (BER)  for the piecewise linear (PL) combiner for  multiple relays is presented. Previous results were available only for upto three relays. The algorithm is unique in the sense that  the actual mathematical expressions, that are prohibitively large, need not be explicitly obtained. The diversity gain due to multiple relays is shown through plots of the analytical BER, well supported by simulations. 
%
%\end{abstract}
% IEEEtran.cls defaults to using nonbold math in the Abstract.
% This preserves the distinction between vectors and scalars. However,
% if the journal you are submitting to favors bold math in the abstract,
% then you can use LaTeX's standard command \boldmath at the very start
% of the abstract to achieve this. Many IEEE journals frown on math
% in the abstract anyway.

% Note that keywords are not normally used for peerreview papers.
%\begin{IEEEkeywords}
%Cooperative diversity, decode and forward, piecewise linear
%\end{IEEEkeywords}



% For peer review papers, you can put extra information on the cover
% page as needed:
% \ifCLASSOPTIONpeerreview
% \begin{center} \bfseries EDICS Category: 3-BBND \end{center}
% \fi
%
% For peerreview papers, this IEEEtran command inserts a page break and
% creates the second title. It will be ignored for other modes.
%\IEEEpeerreviewmaketitle




\item A card is selected from a pack of 52 cards\\
\begin{enumerate}[label=(\alph*)]
\item How many points are there in the sample space?
\item Calculate the probability that the cards is an ace of spades.
\item Calculate the probability that the card is (i) an ace (ii)black card.\\
\end{enumerate}
%\input{ncert/11/16/3/4_1/Prob_4.tex}
\item In a non-leap year, the probability of having 53 tuesdays or 53 wednesdays is\\
\solution
%A non-leap year has a total of 365 days, and a week has 7 days.\\
So it can be expressed as 
\begin{align}
365\text{days} &=52\times 7+1 \text{day}
\end{align}
$\implies$ 52 tuesdays or wednesdays\\
Random variable X denotes the days of a week
\begin{align}
p_X\brak{k}&=\frac{1}{7}; \quad \brak{1<k<7}
\end{align}
So the probability of extra day being tuesday or wednesday is
\begin{align}
p_X\brak{3}+p_X\brak{4}&=\frac{1}{7}+\frac{1}{7}=\frac{2}{7}
\end{align}



\item There are 1000 sealed envelopes in a box, 10 of them contain a cash prize of
Rs 100 each, 100 of them contain a cash prize of Rs 50 each and 200 of them
contain a cash prize of Rs 10 each and rest do not contain any cash prize. If they
are well shuffled and an envelope is picked up out, what is the probability that it
contains no cash prize?\\
\solution
%\begin{table}[H]
	\centering
\begin{tabular}{|c|c|c|}
\hline
Random variable &Value &Definition\\ \hline
\multirow{3}{*}{X} &0 &Slips of Rs 1\\
&1 &Slips of Rs 5\\
&2 &Slips of Rs 13\\ \hline
\multirow{2}{*}{Y} &0 &Box A\\
&1 &Box B\\\hline
\end{tabular}
\caption{}
\label{tab:Distribution}
\end{table}
See \tabref{tab:Distribution}.
\begin{align}
p_{Y}\brak{k}= \begin{cases} 
      \frac{1}{3} & {k=0} \\
      \frac{2}{3 }& {k=1} 
   \end{cases}
   \\
p_{Y|X}\brak{0|0} = \frac{19}{25}\, 
p_{Y|X}\brak{0|1} = \frac{6}{25}\,
p_{Y|X}\brak{1|0} = \frac{45}{50}\,
p_{Y|X}\brak{1|2} = \frac{5}{50}
\end{align}
The desired probability is the probability that a slip drawn at random is marked other than Rs 1,
\begin{align}
&=1-p_X\brak{0}\\
&= p_X(1) + p_X(2)
\end{align}
Using Bayes theorem,
\begin{align}
&= p_Y\brak{0} \times \pr{Y=0 | X=1} + p_Y\brak{1} \times \pr{Y=1|X=2}\\
&=\frac{1}{3} \times \frac{6}{25} + \frac{2}{3} \times \frac{5}{50}\\
&=\frac{11}{75}
\end{align}

\newpage

%\tableofcontents

\bigskip

\renewcommand{\thefigure}{\theenumi}
\renewcommand{\thetable}{\theenumi}
%\renewcommand{\theequation}{\theenumi}

%\begin{abstract}
%%\boldmath
%In this letter, an algorithm for evaluating the exact analytical bit error rate  (BER)  for the piecewise linear (PL) combiner for  multiple relays is presented. Previous results were available only for upto three relays. The algorithm is unique in the sense that  the actual mathematical expressions, that are prohibitively large, need not be explicitly obtained. The diversity gain due to multiple relays is shown through plots of the analytical BER, well supported by simulations. 
%
%\end{abstract}
% IEEEtran.cls defaults to using nonbold math in the Abstract.
% This preserves the distinction between vectors and scalars. However,
% if the journal you are submitting to favors bold math in the abstract,
% then you can use LaTeX's standard command \boldmath at the very start
% of the abstract to achieve this. Many IEEE journals frown on math
% in the abstract anyway.

% Note that keywords are not normally used for peerreview papers.
%\begin{IEEEkeywords}
%Cooperative diversity, decode and forward, piecewise linear
%\end{IEEEkeywords}



% For peer review papers, you can put extra information on the cover
% page as needed:
% \ifCLASSOPTIONpeerreview
% \begin{center} \bfseries EDICS Category: 3-BBND \end{center}
% \fi
%
% For peerreview papers, this IEEEtran command inserts a page break and
% creates the second title. It will be ignored for other modes.
%\IEEEpeerreviewmaketitle




\item 
A die is thrown and a card is selected at random from a deck of 52 playing cards. The probability of getting an even number on the die and a spade card.\\
\solution
%\begin{table}[H]
	\centering
\begin{tabular}{|c|c|c|}
\hline
Random variable &Value &Definition\\ \hline
\multirow{3}{*}{X} &0 &Slips of Rs 1\\
&1 &Slips of Rs 5\\
&2 &Slips of Rs 13\\ \hline
\multirow{2}{*}{Y} &0 &Box A\\
&1 &Box B\\\hline
\end{tabular}
\caption{}
\label{tab:Distribution}
\end{table}
See \tabref{tab:Distribution}.
\begin{align}
p_{Y}\brak{k}= \begin{cases} 
      \frac{1}{3} & {k=0} \\
      \frac{2}{3 }& {k=1} 
   \end{cases}
   \\
p_{Y|X}\brak{0|0} = \frac{19}{25}\, 
p_{Y|X}\brak{0|1} = \frac{6}{25}\,
p_{Y|X}\brak{1|0} = \frac{45}{50}\,
p_{Y|X}\brak{1|2} = \frac{5}{50}
\end{align}
The desired probability is the probability that a slip drawn at random is marked other than Rs 1,
\begin{align}
&=1-p_X\brak{0}\\
&= p_X(1) + p_X(2)
\end{align}
Using Bayes theorem,
\begin{align}
&= p_Y\brak{0} \times \pr{Y=0 | X=1} + p_Y\brak{1} \times \pr{Y=1|X=2}\\
&=\frac{1}{3} \times \frac{6}{25} + \frac{2}{3} \times \frac{5}{50}\\
&=\frac{11}{75}
\end{align}

\newpage

%\tableofcontents

\bigskip

\renewcommand{\thefigure}{\theenumi}
\renewcommand{\thetable}{\theenumi}
%\renewcommand{\theequation}{\theenumi}

%\begin{abstract}
%%\boldmath
%In this letter, an algorithm for evaluating the exact analytical bit error rate  (BER)  for the piecewise linear (PL) combiner for  multiple relays is presented. Previous results were available only for upto three relays. The algorithm is unique in the sense that  the actual mathematical expressions, that are prohibitively large, need not be explicitly obtained. The diversity gain due to multiple relays is shown through plots of the analytical BER, well supported by simulations. 
%
%\end{abstract}
% IEEEtran.cls defaults to using nonbold math in the Abstract.
% This preserves the distinction between vectors and scalars. However,
% if the journal you are submitting to favors bold math in the abstract,
% then you can use LaTeX's standard command \boldmath at the very start
% of the abstract to achieve this. Many IEEE journals frown on math
% in the abstract anyway.

% Note that keywords are not normally used for peerreview papers.
%\begin{IEEEkeywords}
%Cooperative diversity, decode and forward, piecewise linear
%\end{IEEEkeywords}



% For peer review papers, you can put extra information on the cover
% page as needed:
% \ifCLASSOPTIONpeerreview
% \begin{center} \bfseries EDICS Category: 3-BBND \end{center}
% \fi
%
% For peerreview papers, this IEEEtran command inserts a page break and
% creates the second title. It will be ignored for other modes.
%\IEEEpeerreviewmaketitle




\item
If 4-digit numbers greater than 5,000 are randomly formed from the digits 0, 1, 3, 5, and 7, what is the probability of forming a number divisible by 5 when:
\begin{enumerate}
    \item The digits are repeated?
    \item The repetition of digits is not allowed?
\end{enumerate}
\solution
%\begin{table}[H]
	\centering
\begin{tabular}{|c|c|c|}
\hline
Random variable &Value &Definition\\ \hline
\multirow{3}{*}{X} &0 &Slips of Rs 1\\
&1 &Slips of Rs 5\\
&2 &Slips of Rs 13\\ \hline
\multirow{2}{*}{Y} &0 &Box A\\
&1 &Box B\\\hline
\end{tabular}
\caption{}
\label{tab:Distribution}
\end{table}
See \tabref{tab:Distribution}.
\begin{align}
p_{Y}\brak{k}= \begin{cases} 
      \frac{1}{3} & {k=0} \\
      \frac{2}{3 }& {k=1} 
   \end{cases}
   \\
p_{Y|X}\brak{0|0} = \frac{19}{25}\, 
p_{Y|X}\brak{0|1} = \frac{6}{25}\,
p_{Y|X}\brak{1|0} = \frac{45}{50}\,
p_{Y|X}\brak{1|2} = \frac{5}{50}
\end{align}
The desired probability is the probability that a slip drawn at random is marked other than Rs 1,
\begin{align}
&=1-p_X\brak{0}\\
&= p_X(1) + p_X(2)
\end{align}
Using Bayes theorem,
\begin{align}
&= p_Y\brak{0} \times \pr{Y=0 | X=1} + p_Y\brak{1} \times \pr{Y=1|X=2}\\
&=\frac{1}{3} \times \frac{6}{25} + \frac{2}{3} \times \frac{5}{50}\\
&=\frac{11}{75}
\end{align}

\newpage

%\tableofcontents

\bigskip

\renewcommand{\thefigure}{\theenumi}
\renewcommand{\thetable}{\theenumi}
%\renewcommand{\theequation}{\theenumi}

%\begin{abstract}
%%\boldmath
%In this letter, an algorithm for evaluating the exact analytical bit error rate  (BER)  for the piecewise linear (PL) combiner for  multiple relays is presented. Previous results were available only for upto three relays. The algorithm is unique in the sense that  the actual mathematical expressions, that are prohibitively large, need not be explicitly obtained. The diversity gain due to multiple relays is shown through plots of the analytical BER, well supported by simulations. 
%
%\end{abstract}
% IEEEtran.cls defaults to using nonbold math in the Abstract.
% This preserves the distinction between vectors and scalars. However,
% if the journal you are submitting to favors bold math in the abstract,
% then you can use LaTeX's standard command \boldmath at the very start
% of the abstract to achieve this. Many IEEE journals frown on math
% in the abstract anyway.

% Note that keywords are not normally used for peerreview papers.
%\begin{IEEEkeywords}
%Cooperative diversity, decode and forward, piecewise linear
%\end{IEEEkeywords}



% For peer review papers, you can put extra information on the cover
% page as needed:
% \ifCLASSOPTIONpeerreview
% \begin{center} \bfseries EDICS Category: 3-BBND \end{center}
% \fi
%
% For peerreview papers, this IEEEtran command inserts a page break and
% creates the second title. It will be ignored for other modes.
%\IEEEpeerreviewmaketitle




\item Consider the probability space $\brak{\Omega, \mathcal{G}, P}$ where $\Omega = [0,2]$ and $\mathcal{G} = \cbrak{\phi, \Omega, [0,1], (1,2]}$. Let $X$ and $Y$ be two functions on $\Omega$ defined as
\begin{align*}
    X(\omega) = 
    \begin{cases}
        1 & \text{if }\omega \in [0, 1]\\
        2 & \text{if }\omega \in (1, 2]
    \end{cases}
\end{align*}
and
\begin{align*}
    Y(\omega) = 
    \begin{cases}
        2 & \text{if }\omega \in [0, 1.5]\\
        3 & \text{if }\omega \in (1.5, 2].
    \end{cases}
\end{align*}
Then which one of the following statements is true?
\begin{enumerate}
    \item [(A)] $X$ is a random variable with respect to $\mathcal{G}$, but $Y$ is not a random variable with respect to $\mathcal{G}$.
    \item [(B)] $Y$ is a random variable with respect to $\mathcal{G}$, but $X$ is not a random variable with respect to $\mathcal{G}$.
    \item [(C)] Neither $X$ nor $Y$ is a random variable with respect to $\mathcal{G}$.
    \item [(D)] Both $X$ and $Y$ are random variables with respect to $\mathcal{G}$.
\end{enumerate} \hfill (GATE ST 2023)\\
\solution
%\begin{table}[H]
	\centering
\begin{tabular}{|c|c|c|}
\hline
Random variable &Value &Definition\\ \hline
\multirow{3}{*}{X} &0 &Slips of Rs 1\\
&1 &Slips of Rs 5\\
&2 &Slips of Rs 13\\ \hline
\multirow{2}{*}{Y} &0 &Box A\\
&1 &Box B\\\hline
\end{tabular}
\caption{}
\label{tab:Distribution}
\end{table}
See \tabref{tab:Distribution}.
\begin{align}
p_{Y}\brak{k}= \begin{cases} 
      \frac{1}{3} & {k=0} \\
      \frac{2}{3 }& {k=1} 
   \end{cases}
   \\
p_{Y|X}\brak{0|0} = \frac{19}{25}\, 
p_{Y|X}\brak{0|1} = \frac{6}{25}\,
p_{Y|X}\brak{1|0} = \frac{45}{50}\,
p_{Y|X}\brak{1|2} = \frac{5}{50}
\end{align}
The desired probability is the probability that a slip drawn at random is marked other than Rs 1,
\begin{align}
&=1-p_X\brak{0}\\
&= p_X(1) + p_X(2)
\end{align}
Using Bayes theorem,
\begin{align}
&= p_Y\brak{0} \times \pr{Y=0 | X=1} + p_Y\brak{1} \times \pr{Y=1|X=2}\\
&=\frac{1}{3} \times \frac{6}{25} + \frac{2}{3} \times \frac{5}{50}\\
&=\frac{11}{75}
\end{align}

\newpage

%\tableofcontents

\bigskip

\renewcommand{\thefigure}{\theenumi}
\renewcommand{\thetable}{\theenumi}
%\renewcommand{\theequation}{\theenumi}

%\begin{abstract}
%%\boldmath
%In this letter, an algorithm for evaluating the exact analytical bit error rate  (BER)  for the piecewise linear (PL) combiner for  multiple relays is presented. Previous results were available only for upto three relays. The algorithm is unique in the sense that  the actual mathematical expressions, that are prohibitively large, need not be explicitly obtained. The diversity gain due to multiple relays is shown through plots of the analytical BER, well supported by simulations. 
%
%\end{abstract}
% IEEEtran.cls defaults to using nonbold math in the Abstract.
% This preserves the distinction between vectors and scalars. However,
% if the journal you are submitting to favors bold math in the abstract,
% then you can use LaTeX's standard command \boldmath at the very start
% of the abstract to achieve this. Many IEEE journals frown on math
% in the abstract anyway.

% Note that keywords are not normally used for peerreview papers.
%\begin{IEEEkeywords}
%Cooperative diversity, decode and forward, piecewise linear
%\end{IEEEkeywords}



% For peer review papers, you can put extra information on the cover
% page as needed:
% \ifCLASSOPTIONpeerreview
% \begin{center} \bfseries EDICS Category: 3-BBND \end{center}
% \fi
%
% For peerreview papers, this IEEEtran command inserts a page break and
% creates the second title. It will be ignored for other modes.
%\IEEEpeerreviewmaketitle




	\item  A die is loaded in such a way that each odd number is twice as likely to occur as
each even number. Find $P(G)$, where $G$ is the event that a number greater than
3 occurs on a single roll of the die.
\\
\solution
		%\begin{table}[H]
	\centering
\begin{tabular}{|c|c|c|}
\hline
Random variable &Value &Definition\\ \hline
\multirow{3}{*}{X} &0 &Slips of Rs 1\\
&1 &Slips of Rs 5\\
&2 &Slips of Rs 13\\ \hline
\multirow{2}{*}{Y} &0 &Box A\\
&1 &Box B\\\hline
\end{tabular}
\caption{}
\label{tab:Distribution}
\end{table}
See \tabref{tab:Distribution}.
\begin{align}
p_{Y}\brak{k}= \begin{cases} 
      \frac{1}{3} & {k=0} \\
      \frac{2}{3 }& {k=1} 
   \end{cases}
   \\
p_{Y|X}\brak{0|0} = \frac{19}{25}\, 
p_{Y|X}\brak{0|1} = \frac{6}{25}\,
p_{Y|X}\brak{1|0} = \frac{45}{50}\,
p_{Y|X}\brak{1|2} = \frac{5}{50}
\end{align}
The desired probability is the probability that a slip drawn at random is marked other than Rs 1,
\begin{align}
&=1-p_X\brak{0}\\
&= p_X(1) + p_X(2)
\end{align}
Using Bayes theorem,
\begin{align}
&= p_Y\brak{0} \times \pr{Y=0 | X=1} + p_Y\brak{1} \times \pr{Y=1|X=2}\\
&=\frac{1}{3} \times \frac{6}{25} + \frac{2}{3} \times \frac{5}{50}\\
&=\frac{11}{75}
\end{align}

\newpage

%\tableofcontents

\bigskip

\renewcommand{\thefigure}{\theenumi}
\renewcommand{\thetable}{\theenumi}
%\renewcommand{\theequation}{\theenumi}

%\begin{abstract}
%%\boldmath
%In this letter, an algorithm for evaluating the exact analytical bit error rate  (BER)  for the piecewise linear (PL) combiner for  multiple relays is presented. Previous results were available only for upto three relays. The algorithm is unique in the sense that  the actual mathematical expressions, that are prohibitively large, need not be explicitly obtained. The diversity gain due to multiple relays is shown through plots of the analytical BER, well supported by simulations. 
%
%\end{abstract}
% IEEEtran.cls defaults to using nonbold math in the Abstract.
% This preserves the distinction between vectors and scalars. However,
% if the journal you are submitting to favors bold math in the abstract,
% then you can use LaTeX's standard command \boldmath at the very start
% of the abstract to achieve this. Many IEEE journals frown on math
% in the abstract anyway.

% Note that keywords are not normally used for peerreview papers.
%\begin{IEEEkeywords}
%Cooperative diversity, decode and forward, piecewise linear
%\end{IEEEkeywords}



% For peer review papers, you can put extra information on the cover
% page as needed:
% \ifCLASSOPTIONpeerreview
% \begin{center} \bfseries EDICS Category: 3-BBND \end{center}
% \fi
%
% For peerreview papers, this IEEEtran command inserts a page break and
% creates the second title. It will be ignored for other modes.
%\IEEEpeerreviewmaketitle




	\item All the jacks, queens and kings are removed from a deck of 52 playing cards. The remaining cards are well shuffled and then one card is drawn at random. Giving ace a value 1 similar value for other cards, find the probability that the card has a value 
		\begin{enumerate}
			\item 7
			\item greater than 7
			\item less than 7
		\end{enumerate}
		%Number of cards left after removing all jacks, queens and kings 
\begin{align}
N	= 52 - 4\times 3
	= 40
\end{align}
%\begin{table}[H]
%\def\arraystretch{1.2}
%\begin{tabular}{|c|c|c|}
%\hline
%	\textbf{Parameter} &\textbf{Value} &\textbf{Description}\\ \hline
%	$X$ &1-10 &Represents the value of the card picked \\ \hline
%\end{tabular}
%\end{table}
Let $1 \le X \le 10$ be the value of the card picked.  Then,
\begin{align}
	p_X(k) &= \Pr(X=k)\ \forall\ 1 \leq k \leq 10\\
	&= \frac{4\times 1}{40}\\
	&= \frac{1}{10}\\
	\therefore p_X(k) &= 
	\begin{cases}
		\frac{1}{10} & 1 \leq k \leq 10\\
		0 & \text{otherwise}
	\end{cases}
\end{align}
and
\begin{align}
	F_{X}(k) &= \sum_{m=0}^{k}p_{X}(m) \quad 1 \leq k \leq 10\\
	&= \frac{k}{10}\\
	\therefore F_{X}(k) &= 
	\begin{cases}
		0 & k \leq 0\\
		\frac{k}{10} & 1\leq k \leq 10\\
		1 & k > 10 
	\end{cases}
\end{align}
\begin{enumerate}
	\item Probability that card has value equal to 7 is
		\begin{align}
			 p_{X}(7)
			= \frac{1}{10}
		\end{align}
	\item Probability that card has value greater than 7 is
		\begin{align}
			1 - F_X(7)
			&= 1 - \frac{7}{10}
			\\
			&= \frac{3}{10}
		\end{align}
	\item Probability that card has value less than 7 is
		\begin{align}
			 F_{X}(6)
			=\frac{6}{10}
		\end{align}
\end{enumerate}

  \item A Lot consists of 48 mobile phones of which 42 are good, 3 have only minor defects and 3 have major defects.Varnika will buy a phone if it is good but the trader will only buy a mobile if it has no major defects. One phone is selected at random from the lot. What is the probability that it is
\begin{enumerate}
	\item acceptable to Varnika?
            \item acceptable to the trader?
\end{enumerate}
\solution
	%\begin{table}[H]
	\centering
\begin{tabular}{|c|c|c|}
\hline
Random variable &Value &Definition\\ \hline
\multirow{3}{*}{X} &0 &Slips of Rs 1\\
&1 &Slips of Rs 5\\
&2 &Slips of Rs 13\\ \hline
\multirow{2}{*}{Y} &0 &Box A\\
&1 &Box B\\\hline
\end{tabular}
\caption{}
\label{tab:Distribution}
\end{table}
See \tabref{tab:Distribution}.
\begin{align}
p_{Y}\brak{k}= \begin{cases} 
      \frac{1}{3} & {k=0} \\
      \frac{2}{3 }& {k=1} 
   \end{cases}
   \\
p_{Y|X}\brak{0|0} = \frac{19}{25}\, 
p_{Y|X}\brak{0|1} = \frac{6}{25}\,
p_{Y|X}\brak{1|0} = \frac{45}{50}\,
p_{Y|X}\brak{1|2} = \frac{5}{50}
\end{align}
The desired probability is the probability that a slip drawn at random is marked other than Rs 1,
\begin{align}
&=1-p_X\brak{0}\\
&= p_X(1) + p_X(2)
\end{align}
Using Bayes theorem,
\begin{align}
&= p_Y\brak{0} \times \pr{Y=0 | X=1} + p_Y\brak{1} \times \pr{Y=1|X=2}\\
&=\frac{1}{3} \times \frac{6}{25} + \frac{2}{3} \times \frac{5}{50}\\
&=\frac{11}{75}
\end{align}

\newpage

%\tableofcontents

\bigskip

\renewcommand{\thefigure}{\theenumi}
\renewcommand{\thetable}{\theenumi}
%\renewcommand{\theequation}{\theenumi}

%\begin{abstract}
%%\boldmath
%In this letter, an algorithm for evaluating the exact analytical bit error rate  (BER)  for the piecewise linear (PL) combiner for  multiple relays is presented. Previous results were available only for upto three relays. The algorithm is unique in the sense that  the actual mathematical expressions, that are prohibitively large, need not be explicitly obtained. The diversity gain due to multiple relays is shown through plots of the analytical BER, well supported by simulations. 
%
%\end{abstract}
% IEEEtran.cls defaults to using nonbold math in the Abstract.
% This preserves the distinction between vectors and scalars. However,
% if the journal you are submitting to favors bold math in the abstract,
% then you can use LaTeX's standard command \boldmath at the very start
% of the abstract to achieve this. Many IEEE journals frown on math
% in the abstract anyway.

% Note that keywords are not normally used for peerreview papers.
%\begin{IEEEkeywords}
%Cooperative diversity, decode and forward, piecewise linear
%\end{IEEEkeywords}



% For peer review papers, you can put extra information on the cover
% page as needed:
% \ifCLASSOPTIONpeerreview
% \begin{center} \bfseries EDICS Category: 3-BBND \end{center}
% \fi
%
% For peerreview papers, this IEEEtran command inserts a page break and
% creates the second title. It will be ignored for other modes.
%\IEEEpeerreviewmaketitle




 \item A student says that if you throw a die, it will show up 1 or not 1. Therefore, the probability of getting 1 and the probability of getting 'not 1' each is equal to $\frac{1}{2}$. Is this correct? Give reasons.\\
 \solution
        %\begin{table}[H]
	\centering
\begin{tabular}{|c|c|c|}
\hline
Random variable &Value &Definition\\ \hline
\multirow{3}{*}{X} &0 &Slips of Rs 1\\
&1 &Slips of Rs 5\\
&2 &Slips of Rs 13\\ \hline
\multirow{2}{*}{Y} &0 &Box A\\
&1 &Box B\\\hline
\end{tabular}
\caption{}
\label{tab:Distribution}
\end{table}
See \tabref{tab:Distribution}.
\begin{align}
p_{Y}\brak{k}= \begin{cases} 
      \frac{1}{3} & {k=0} \\
      \frac{2}{3 }& {k=1} 
   \end{cases}
   \\
p_{Y|X}\brak{0|0} = \frac{19}{25}\, 
p_{Y|X}\brak{0|1} = \frac{6}{25}\,
p_{Y|X}\brak{1|0} = \frac{45}{50}\,
p_{Y|X}\brak{1|2} = \frac{5}{50}
\end{align}
The desired probability is the probability that a slip drawn at random is marked other than Rs 1,
\begin{align}
&=1-p_X\brak{0}\\
&= p_X(1) + p_X(2)
\end{align}
Using Bayes theorem,
\begin{align}
&= p_Y\brak{0} \times \pr{Y=0 | X=1} + p_Y\brak{1} \times \pr{Y=1|X=2}\\
&=\frac{1}{3} \times \frac{6}{25} + \frac{2}{3} \times \frac{5}{50}\\
&=\frac{11}{75}
\end{align}

\newpage

%\tableofcontents

\bigskip

\renewcommand{\thefigure}{\theenumi}
\renewcommand{\thetable}{\theenumi}
%\renewcommand{\theequation}{\theenumi}

%\begin{abstract}
%%\boldmath
%In this letter, an algorithm for evaluating the exact analytical bit error rate  (BER)  for the piecewise linear (PL) combiner for  multiple relays is presented. Previous results were available only for upto three relays. The algorithm is unique in the sense that  the actual mathematical expressions, that are prohibitively large, need not be explicitly obtained. The diversity gain due to multiple relays is shown through plots of the analytical BER, well supported by simulations. 
%
%\end{abstract}
% IEEEtran.cls defaults to using nonbold math in the Abstract.
% This preserves the distinction between vectors and scalars. However,
% if the journal you are submitting to favors bold math in the abstract,
% then you can use LaTeX's standard command \boldmath at the very start
% of the abstract to achieve this. Many IEEE journals frown on math
% in the abstract anyway.

% Note that keywords are not normally used for peerreview papers.
%\begin{IEEEkeywords}
%Cooperative diversity, decode and forward, piecewise linear
%\end{IEEEkeywords}



% For peer review papers, you can put extra information on the cover
% page as needed:
% \ifCLASSOPTIONpeerreview
% \begin{center} \bfseries EDICS Category: 3-BBND \end{center}
% \fi
%
% For peerreview papers, this IEEEtran command inserts a page break and
% creates the second title. It will be ignored for other modes.
%\IEEEpeerreviewmaketitle




   \item Four candidates A, B, C, D have ap-
plied for the assignment to coach a school cricket
team. If A is twice as likely to be selected as B, and
B and C are given about the same chance of being
selected, while C is twice as likely to be selected
as D, what are the probabilities that
\begin{enumerate}
\item C will be selected?
\item A will not be selected?
\end{enumerate}
	%\begin{table}[H]
	\centering
\begin{tabular}{|c|c|c|}
\hline
Random variable &Value &Definition\\ \hline
\multirow{3}{*}{X} &0 &Slips of Rs 1\\
&1 &Slips of Rs 5\\
&2 &Slips of Rs 13\\ \hline
\multirow{2}{*}{Y} &0 &Box A\\
&1 &Box B\\\hline
\end{tabular}
\caption{}
\label{tab:Distribution}
\end{table}
See \tabref{tab:Distribution}.
\begin{align}
p_{Y}\brak{k}= \begin{cases} 
      \frac{1}{3} & {k=0} \\
      \frac{2}{3 }& {k=1} 
   \end{cases}
   \\
p_{Y|X}\brak{0|0} = \frac{19}{25}\, 
p_{Y|X}\brak{0|1} = \frac{6}{25}\,
p_{Y|X}\brak{1|0} = \frac{45}{50}\,
p_{Y|X}\brak{1|2} = \frac{5}{50}
\end{align}
The desired probability is the probability that a slip drawn at random is marked other than Rs 1,
\begin{align}
&=1-p_X\brak{0}\\
&= p_X(1) + p_X(2)
\end{align}
Using Bayes theorem,
\begin{align}
&= p_Y\brak{0} \times \pr{Y=0 | X=1} + p_Y\brak{1} \times \pr{Y=1|X=2}\\
&=\frac{1}{3} \times \frac{6}{25} + \frac{2}{3} \times \frac{5}{50}\\
&=\frac{11}{75}
\end{align}

\newpage

%\tableofcontents

\bigskip

\renewcommand{\thefigure}{\theenumi}
\renewcommand{\thetable}{\theenumi}
%\renewcommand{\theequation}{\theenumi}

%\begin{abstract}
%%\boldmath
%In this letter, an algorithm for evaluating the exact analytical bit error rate  (BER)  for the piecewise linear (PL) combiner for  multiple relays is presented. Previous results were available only for upto three relays. The algorithm is unique in the sense that  the actual mathematical expressions, that are prohibitively large, need not be explicitly obtained. The diversity gain due to multiple relays is shown through plots of the analytical BER, well supported by simulations. 
%
%\end{abstract}
% IEEEtran.cls defaults to using nonbold math in the Abstract.
% This preserves the distinction between vectors and scalars. However,
% if the journal you are submitting to favors bold math in the abstract,
% then you can use LaTeX's standard command \boldmath at the very start
% of the abstract to achieve this. Many IEEE journals frown on math
% in the abstract anyway.

% Note that keywords are not normally used for peerreview papers.
%\begin{IEEEkeywords}
%Cooperative diversity, decode and forward, piecewise linear
%\end{IEEEkeywords}



% For peer review papers, you can put extra information on the cover
% page as needed:
% \ifCLASSOPTIONpeerreview
% \begin{center} \bfseries EDICS Category: 3-BBND \end{center}
% \fi
%
% For peerreview papers, this IEEEtran command inserts a page break and
% creates the second title. It will be ignored for other modes.
%\IEEEpeerreviewmaketitle




 \item A bag contain 24 balls of which $x$ balls are red, $2x$ are white and $3x$ are blue. A ball is selected at random, What is the probability that it is
\begin{enumerate}[label=\alph*)]
\item not red ?
\item white ?
\end{enumerate}
%\begin{table}[H]
	\centering
\begin{tabular}{|c|c|c|}
\hline
Random variable &Value &Definition\\ \hline
\multirow{3}{*}{X} &0 &Slips of Rs 1\\
&1 &Slips of Rs 5\\
&2 &Slips of Rs 13\\ \hline
\multirow{2}{*}{Y} &0 &Box A\\
&1 &Box B\\\hline
\end{tabular}
\caption{}
\label{tab:Distribution}
\end{table}
See \tabref{tab:Distribution}.
\begin{align}
p_{Y}\brak{k}= \begin{cases} 
      \frac{1}{3} & {k=0} \\
      \frac{2}{3 }& {k=1} 
   \end{cases}
   \\
p_{Y|X}\brak{0|0} = \frac{19}{25}\, 
p_{Y|X}\brak{0|1} = \frac{6}{25}\,
p_{Y|X}\brak{1|0} = \frac{45}{50}\,
p_{Y|X}\brak{1|2} = \frac{5}{50}
\end{align}
The desired probability is the probability that a slip drawn at random is marked other than Rs 1,
\begin{align}
&=1-p_X\brak{0}\\
&= p_X(1) + p_X(2)
\end{align}
Using Bayes theorem,
\begin{align}
&= p_Y\brak{0} \times \pr{Y=0 | X=1} + p_Y\brak{1} \times \pr{Y=1|X=2}\\
&=\frac{1}{3} \times \frac{6}{25} + \frac{2}{3} \times \frac{5}{50}\\
&=\frac{11}{75}
\end{align}

\newpage

%\tableofcontents

\bigskip

\renewcommand{\thefigure}{\theenumi}
\renewcommand{\thetable}{\theenumi}
%\renewcommand{\theequation}{\theenumi}

%\begin{abstract}
%%\boldmath
%In this letter, an algorithm for evaluating the exact analytical bit error rate  (BER)  for the piecewise linear (PL) combiner for  multiple relays is presented. Previous results were available only for upto three relays. The algorithm is unique in the sense that  the actual mathematical expressions, that are prohibitively large, need not be explicitly obtained. The diversity gain due to multiple relays is shown through plots of the analytical BER, well supported by simulations. 
%
%\end{abstract}
% IEEEtran.cls defaults to using nonbold math in the Abstract.
% This preserves the distinction between vectors and scalars. However,
% if the journal you are submitting to favors bold math in the abstract,
% then you can use LaTeX's standard command \boldmath at the very start
% of the abstract to achieve this. Many IEEE journals frown on math
% in the abstract anyway.

% Note that keywords are not normally used for peerreview papers.
%\begin{IEEEkeywords}
%Cooperative diversity, decode and forward, piecewise linear
%\end{IEEEkeywords}



% For peer review papers, you can put extra information on the cover
% page as needed:
% \ifCLASSOPTIONpeerreview
% \begin{center} \bfseries EDICS Category: 3-BBND \end{center}
% \fi
%
% For peerreview papers, this IEEEtran command inserts a page break and
% creates the second title. It will be ignored for other modes.
%\IEEEpeerreviewmaketitle




If the letters of the word ASSASSINATION are arranged at random. Find the Probability that
\begin{enumerate}[label=(\alph*)]
\item Four $S's$ come consecutively in the word
\item Two  $I's$ and two $N's$ come together
\item All $A's$ are not coming together
\item No two $A's$ are coming together
\end{enumerate}
%\begin{table}[H]
	\centering
\begin{tabular}{|c|c|c|}
\hline
Random variable &Value &Definition\\ \hline
\multirow{3}{*}{X} &0 &Slips of Rs 1\\
&1 &Slips of Rs 5\\
&2 &Slips of Rs 13\\ \hline
\multirow{2}{*}{Y} &0 &Box A\\
&1 &Box B\\\hline
\end{tabular}
\caption{}
\label{tab:Distribution}
\end{table}
See \tabref{tab:Distribution}.
\begin{align}
p_{Y}\brak{k}= \begin{cases} 
      \frac{1}{3} & {k=0} \\
      \frac{2}{3 }& {k=1} 
   \end{cases}
   \\
p_{Y|X}\brak{0|0} = \frac{19}{25}\, 
p_{Y|X}\brak{0|1} = \frac{6}{25}\,
p_{Y|X}\brak{1|0} = \frac{45}{50}\,
p_{Y|X}\brak{1|2} = \frac{5}{50}
\end{align}
The desired probability is the probability that a slip drawn at random is marked other than Rs 1,
\begin{align}
&=1-p_X\brak{0}\\
&= p_X(1) + p_X(2)
\end{align}
Using Bayes theorem,
\begin{align}
&= p_Y\brak{0} \times \pr{Y=0 | X=1} + p_Y\brak{1} \times \pr{Y=1|X=2}\\
&=\frac{1}{3} \times \frac{6}{25} + \frac{2}{3} \times \frac{5}{50}\\
&=\frac{11}{75}
\end{align}

\newpage

%\tableofcontents

\bigskip

\renewcommand{\thefigure}{\theenumi}
\renewcommand{\thetable}{\theenumi}
%\renewcommand{\theequation}{\theenumi}

%\begin{abstract}
%%\boldmath
%In this letter, an algorithm for evaluating the exact analytical bit error rate  (BER)  for the piecewise linear (PL) combiner for  multiple relays is presented. Previous results were available only for upto three relays. The algorithm is unique in the sense that  the actual mathematical expressions, that are prohibitively large, need not be explicitly obtained. The diversity gain due to multiple relays is shown through plots of the analytical BER, well supported by simulations. 
%
%\end{abstract}
% IEEEtran.cls defaults to using nonbold math in the Abstract.
% This preserves the distinction between vectors and scalars. However,
% if the journal you are submitting to favors bold math in the abstract,
% then you can use LaTeX's standard command \boldmath at the very start
% of the abstract to achieve this. Many IEEE journals frown on math
% in the abstract anyway.

% Note that keywords are not normally used for peerreview papers.
%\begin{IEEEkeywords}
%Cooperative diversity, decode and forward, piecewise linear
%\end{IEEEkeywords}



% For peer review papers, you can put extra information on the cover
% page as needed:
% \ifCLASSOPTIONpeerreview
% \begin{center} \bfseries EDICS Category: 3-BBND \end{center}
% \fi
%
% For peerreview papers, this IEEEtran command inserts a page break and
% creates the second title. It will be ignored for other modes.
%\IEEEpeerreviewmaketitle




	\item One urn contains two black balls (labelled B1 and B2) and one white ball. A
	second urn contains one black ball and two white balls (labelled W1 and W2).
	Suppose the following experiment is performed. One of the two urns is chosen
	at random. Next a ball is randomly chosen from the urn. Then a second ball is
	chosen at random from the same urn without replacing the first ball.
	
	\begin{enumerate}
	\item What is the probability that two black balls are chosen?
	
	\item What is the probability that two balls of opposite colour are chosen?
	\end{enumerate}
	\solution
	%\begin{align}
    \label{eq:12.13.6.18.1}
	\because	\pr{A|B} &> \pr{A},\
\frac{\pr{AB}}{\pr{B}} > \pr{A}
\\
    \label{eq:12.13.6.18.2}
	\implies \pr{AB} &> \pr{A}\pr{B}
	\\
	\text{or, } \frac{\pr{AB}}{\pr{A}} &=\pr{B|A} > \pr{A}
\end{align}

\end{enumerate}

		\item A box of oranges is inspected by examining three randomly selected oranges drawn without replacement. If all the three oranges are good, the box is approved for sale, otherwise, it is rejected. Find the probability that a box containing 15 oranges out of which 12 are good and 3 are bad ones will be approved for sale.
		\label{ncert/12/13/2/3/defs.tex}
		\item Two balls are drawn at random with replacement from a box containing 10 black and 8 red balls. Find the probability that
		\label{ncert/12/13/2/12}
\begin{enumerate}
\item both balls are red.
\item first ball is black and second is red.
\item one of them is black and other is red.
\end{enumerate}

\item In a hostel, 60\% of the students read Hindi newspaper, 40\% read English newspaper and 20\% read both Hindi and English newspapers. A student is selected at random.
		\label{ncert/12/13/2/15}
\begin{enumerate}
\item Find the probability that she reads neither Hindi nor English newspapers.
\item If she reads Hindi newspaper, find the probability that she reads English newspaper.
\item If she reads English newspaper, find the probability that she reads Hindi newspaper.\\
\end{enumerate}
\item The probability of obtaining an even prime number on each die, when a pair of dice is rolled is 
\begin{enumerate}
    \item $0$ 
    
    \item $\frac{1}{3}$ 
    
    \item $\frac{1}{12}$ 
    
    \item $\frac{1}{36}$ 
\end{enumerate}
\solution
		%\begin{enumerate}[label=\thesection.\arabic*,ref=\thesection.\theenumi]
	\item One card is drawn from a well-shuffled deck of 52 cards. Find the probability of getting
\begin{enumerate}
\item A king of red colour 
\item A face card 
\item A red face card
\item The jack of hearts
\item A spade
\item The queen of diamonds

\end{enumerate}
\solution
		%\begin{table}[H]
	\centering
\begin{tabular}{|c|c|c|}
\hline
Random variable &Value &Definition\\ \hline
\multirow{3}{*}{X} &0 &Slips of Rs 1\\
&1 &Slips of Rs 5\\
&2 &Slips of Rs 13\\ \hline
\multirow{2}{*}{Y} &0 &Box A\\
&1 &Box B\\\hline
\end{tabular}
\caption{}
\label{tab:Distribution}
\end{table}
See \tabref{tab:Distribution}.
\begin{align}
p_{Y}\brak{k}= \begin{cases} 
      \frac{1}{3} & {k=0} \\
      \frac{2}{3 }& {k=1} 
   \end{cases}
   \\
p_{Y|X}\brak{0|0} = \frac{19}{25}\, 
p_{Y|X}\brak{0|1} = \frac{6}{25}\,
p_{Y|X}\brak{1|0} = \frac{45}{50}\,
p_{Y|X}\brak{1|2} = \frac{5}{50}
\end{align}
The desired probability is the probability that a slip drawn at random is marked other than Rs 1,
\begin{align}
&=1-p_X\brak{0}\\
&= p_X(1) + p_X(2)
\end{align}
Using Bayes theorem,
\begin{align}
&= p_Y\brak{0} \times \pr{Y=0 | X=1} + p_Y\brak{1} \times \pr{Y=1|X=2}\\
&=\frac{1}{3} \times \frac{6}{25} + \frac{2}{3} \times \frac{5}{50}\\
&=\frac{11}{75}
\end{align}

\newpage

%\tableofcontents

\bigskip

\renewcommand{\thefigure}{\theenumi}
\renewcommand{\thetable}{\theenumi}
%\renewcommand{\theequation}{\theenumi}

%\begin{abstract}
%%\boldmath
%In this letter, an algorithm for evaluating the exact analytical bit error rate  (BER)  for the piecewise linear (PL) combiner for  multiple relays is presented. Previous results were available only for upto three relays. The algorithm is unique in the sense that  the actual mathematical expressions, that are prohibitively large, need not be explicitly obtained. The diversity gain due to multiple relays is shown through plots of the analytical BER, well supported by simulations. 
%
%\end{abstract}
% IEEEtran.cls defaults to using nonbold math in the Abstract.
% This preserves the distinction between vectors and scalars. However,
% if the journal you are submitting to favors bold math in the abstract,
% then you can use LaTeX's standard command \boldmath at the very start
% of the abstract to achieve this. Many IEEE journals frown on math
% in the abstract anyway.

% Note that keywords are not normally used for peerreview papers.
%\begin{IEEEkeywords}
%Cooperative diversity, decode and forward, piecewise linear
%\end{IEEEkeywords}



% For peer review papers, you can put extra information on the cover
% page as needed:
% \ifCLASSOPTIONpeerreview
% \begin{center} \bfseries EDICS Category: 3-BBND \end{center}
% \fi
%
% For peerreview papers, this IEEEtran command inserts a page break and
% creates the second title. It will be ignored for other modes.
%\IEEEpeerreviewmaketitle




	\item Five cards—the ten, jack, queen, king and ace of diamonds, are well-shuffled with their face downwards. One card is then picked up at random.
\begin{enumerate}
\item
What is the probability that the card is the queen? 
\item
If the queen is drawn and put aside, what is the probability that the second card picked up is (a) an ace? (b) a queen?\\
\end{enumerate}
\solution
		%\begin{enumerate}[label=\thesection.\arabic*,ref=\thesection.\theenumi]
	\item One card is drawn from a well-shuffled deck of 52 cards. Find the probability of getting
\begin{enumerate}
\item A king of red colour 
\item A face card 
\item A red face card
\item The jack of hearts
\item A spade
\item The queen of diamonds

\end{enumerate}
\solution
		%\input{ncert/10/15/1/14/main.tex}
	\item Five cards—the ten, jack, queen, king and ace of diamonds, are well-shuffled with their face downwards. One card is then picked up at random.
\begin{enumerate}
\item
What is the probability that the card is the queen? 
\item
If the queen is drawn and put aside, what is the probability that the second card picked up is (a) an ace? (b) a queen?\\
\end{enumerate}
\solution
		%\input{ncert/10/15/1/15/defs.tex}
	\item A bag contains $5$ red balls and some blue balls. If the probability of drawing a blue ball is double that if a red ball, determine the number of blue balls in the bag. 
		\\
\solution
		%\input{ncert/10/15/2/3/defs.tex}
	\item A card is selected from a pack of 52 cards.
 \begin{enumerate}[label=(\alph*)] 
                 \item How many points are there in the sample space?
                 \item Calculate the probability that the card is an ace of spades.
                 \item Calculate the probability that the card is (i) an ace and (ii) black card.
 \end{enumerate}
\solution
		%\input{ncert/11/16/3/4/main.tex}
\item Four cards are drawn from a well-shuffled deck of 52 cards. What is the probability of obtaining 3 diamonds and one spade.
\\
\solution
		%\input{ncert/11/16/4/2/defs.tex}
\item In a certain lottery 10,000 tickets are sold and ten equal prizes are awarded. What is the probability of not getting a prize if you buy (a) one ticket (b) two tickets (c) 10 tickets ?	
\\
\solution
		%\input{ncert/11/16/4/4/defs.tex}
		%
\item 
Out of 100 students, two sections of 40 and 60 are formed. If you and your friend are among the 100 students, what is the probability that
\begin{enumerate}
\item you both enter the same section?
\item you both enter the different sections?
\end{enumerate}
\solution
		%\input{ncert/11/16/4/5/defs.tex}
	\item 
The number lock of a suitcase has 4 wheels each labelled with ten digits i.e. from 0 to 9.The lock opens with a sequence of four digits with no repeats.What is the probability of a person getting the right sequence to open the suitcase.
\\
\solution
		%\input{ncert/11/16/4/10/defs.tex}
		%
\item 
Two cards are drawn at random and without replacement from a pack of 52 playing cards. Find the probability that both the cards are black.
\\
\solution
		%\input{ncert/12/13/2/2/defs.tex}
		\item A box of oranges is inspected by examining three randomly selected oranges drawn without replacement. If all the three oranges are good, the box is approved for sale, otherwise, it is rejected. Find the probability that a box containing 15 oranges out of which 12 are good and 3 are bad ones will be approved for sale.
		\label{ncert/12/13/2/3/defs.tex}
		\item Two balls are drawn at random with replacement from a box containing 10 black and 8 red balls. Find the probability that
		\label{ncert/12/13/2/12}
\begin{enumerate}
\item both balls are red.
\item first ball is black and second is red.
\item one of them is black and other is red.
\end{enumerate}

\item In a hostel, 60\% of the students read Hindi newspaper, 40\% read English newspaper and 20\% read both Hindi and English newspapers. A student is selected at random.
		\label{ncert/12/13/2/15}
\begin{enumerate}
\item Find the probability that she reads neither Hindi nor English newspapers.
\item If she reads Hindi newspaper, find the probability that she reads English newspaper.
\item If she reads English newspaper, find the probability that she reads Hindi newspaper.\\
\end{enumerate}
\item The probability of obtaining an even prime number on each die, when a pair of dice is rolled is 
\begin{enumerate}
    \item $0$ 
    
    \item $\frac{1}{3}$ 
    
    \item $\frac{1}{12}$ 
    
    \item $\frac{1}{36}$ 
\end{enumerate}
\solution
		%\input{ncert/12/13/2/17/defs.tex}
	\item A bag contains 4 red and 4 black balls, another bag contains 2 red and 6 black balls. One of the two bags is selected at random and a ball is drawn from the bag which is found to be red. Find the probability that the ball is drawn from the first bag.
\\
\solution
		%\input{ncert/12/13/3/2/main.tex}
  \item
  Cards with numbers 2 to 101 are placed in a box. A card is selected at random.Find the probability that the card has
\begin{enumerate}[label=(\roman*)]
	\item an even number 
	\item a square number
\end{enumerate}
\solution
%\input{exemplar/10/13/3/32/main.tex}
\item
The king, queen and jack of clubs are removed from a deck of 52 playing cards and then well shuffled. Now one card is drawn at random from the remaining cards.  Determine the probability that the card is
\begin{enumerate}[label=(\roman*)]
\item a club
\item 10 of hearts
\end{enumerate}
\solution
%\input{exemplar/10/13/3/29/main.tex}
\item A team of medical students doing their internship have to assist during surgeries
at a city hospital. The probabilities of surgeries rated as very complex, complex,
routine, simple or very simple are respectively, 0.15, 0.20, 0.31, 0.26, .08. Find
the probabilities that a particular surgery will be rated
\begin{enumerate}
	\item complex or very complex;
	\item neither very complex nor very simple;
	\item routine or complex
	\item routine or simple
\end{enumerate}
\solution
%\input{exemplar/11/16/3/8(1)/main.tex}
\item A card is selected from a pack of 52 cards.
\begin{enumerate}[label=(\alph*)]
    \item How many points are there in the sample space?
    \item Calculate the probability that the card is an ace of spades.
    \item Calculate the probability that the card is (i) an ace and (ii) black card.
\end{enumerate}
\solution
%\input{exemplar/11/16/3/4/main2.tex}
\item The probability that a non leap year selected at random will contain 53 sundays.
\\
\solution
%\input{exemplar/10/13/1/19/main.tex}
\item One of the four persons John, Rita, Aslam or Gurpreet will be promoted next
month. Consequently the sample space consists of four elementary outcomes
S = {John promoted, Rita promoted, Aslam promoted, Gurpreet promoted}
You are told that the chances of John’s promotion is same as that of Gurpreet,
Rita’s chances of promotion are twice as likely as Johns. Aslam’s chances are
four times that of John.
\begin{enumerate}
	\item Determine
	\begin{enumerate}
		\item P (John promoted)
		\item P (Rita promoted)
		\item P (Aslam promoted)
		\item P (Gurpreet promoted)
	\end{enumerate}
	\item If A = {John promoted or Gurpreet promoted}, find P (A).
\end{enumerate}
\solution
%\input{exemplar/11/16/3/10/main.tex}
\item A card is drawn from a deck of 52 cards. Find the probability of getting a king or a heart or a red card.\\
\solution
%\input{exemplar/11/16/3/15/main.tex}
\item The probability that a student will pass his examination is 0.73, the probability of
the student getting a compartment is 0.13, and the probability that the student will
either pass or get compartment is 0.96. State True or False.\\
\solution
%\input{exemplar/11/16/3/31/main.tex}
\item A card is selected from a pack of 52 cards\\
\begin{enumerate}[label=(\alph*)]
\item How many points are there in the sample space?
\item Calculate the probability that the cards is an ace of spades.
\item Calculate the probability that the card is (i) an ace (ii)black card.\\
\end{enumerate}
%\input{ncert/11/16/3/4_1/Prob_4.tex}
\item In a non-leap year, the probability of having 53 tuesdays or 53 wednesdays is\\
\solution
%\input{exemplar/11/16/3/18/main.tex}
\item There are 1000 sealed envelopes in a box, 10 of them contain a cash prize of
Rs 100 each, 100 of them contain a cash prize of Rs 50 each and 200 of them
contain a cash prize of Rs 10 each and rest do not contain any cash prize. If they
are well shuffled and an envelope is picked up out, what is the probability that it
contains no cash prize?\\
\solution
%\input{exemplar/10/13/3/34/main.tex}
\item 
A die is thrown and a card is selected at random from a deck of 52 playing cards. The probability of getting an even number on the die and a spade card.\\
\solution
%\input{exemplar/12/13/3/78/main.tex}
\item
If 4-digit numbers greater than 5,000 are randomly formed from the digits 0, 1, 3, 5, and 7, what is the probability of forming a number divisible by 5 when:
\begin{enumerate}
    \item The digits are repeated?
    \item The repetition of digits is not allowed?
\end{enumerate}
\solution
%\input{ncert/11/16/4/9/main.tex}
\item Consider the probability space $\brak{\Omega, \mathcal{G}, P}$ where $\Omega = [0,2]$ and $\mathcal{G} = \cbrak{\phi, \Omega, [0,1], (1,2]}$. Let $X$ and $Y$ be two functions on $\Omega$ defined as
\begin{align*}
    X(\omega) = 
    \begin{cases}
        1 & \text{if }\omega \in [0, 1]\\
        2 & \text{if }\omega \in (1, 2]
    \end{cases}
\end{align*}
and
\begin{align*}
    Y(\omega) = 
    \begin{cases}
        2 & \text{if }\omega \in [0, 1.5]\\
        3 & \text{if }\omega \in (1.5, 2].
    \end{cases}
\end{align*}
Then which one of the following statements is true?
\begin{enumerate}
    \item [(A)] $X$ is a random variable with respect to $\mathcal{G}$, but $Y$ is not a random variable with respect to $\mathcal{G}$.
    \item [(B)] $Y$ is a random variable with respect to $\mathcal{G}$, but $X$ is not a random variable with respect to $\mathcal{G}$.
    \item [(C)] Neither $X$ nor $Y$ is a random variable with respect to $\mathcal{G}$.
    \item [(D)] Both $X$ and $Y$ are random variables with respect to $\mathcal{G}$.
\end{enumerate} \hfill (GATE ST 2023)\\
\solution
%\input{gate/ST/2023/14/main.tex}
	\item  A die is loaded in such a way that each odd number is twice as likely to occur as
each even number. Find $P(G)$, where $G$ is the event that a number greater than
3 occurs on a single roll of the die.
\\
\solution
		%\input{exemplar/11/16/3/5/main.tex}
	\item All the jacks, queens and kings are removed from a deck of 52 playing cards. The remaining cards are well shuffled and then one card is drawn at random. Giving ace a value 1 similar value for other cards, find the probability that the card has a value 
		\begin{enumerate}
			\item 7
			\item greater than 7
			\item less than 7
		\end{enumerate}
		%\input{exemplar/10/13/3/30/main.tex}
  \item A Lot consists of 48 mobile phones of which 42 are good, 3 have only minor defects and 3 have major defects.Varnika will buy a phone if it is good but the trader will only buy a mobile if it has no major defects. One phone is selected at random from the lot. What is the probability that it is
\begin{enumerate}
	\item acceptable to Varnika?
            \item acceptable to the trader?
\end{enumerate}
\solution
	%\input{exemplar/10/13/3/40/main.tex}
 \item A student says that if you throw a die, it will show up 1 or not 1. Therefore, the probability of getting 1 and the probability of getting 'not 1' each is equal to $\frac{1}{2}$. Is this correct? Give reasons.\\
 \solution
        %\input{exemplar/10/13/2/9/main.tex}
   \item Four candidates A, B, C, D have ap-
plied for the assignment to coach a school cricket
team. If A is twice as likely to be selected as B, and
B and C are given about the same chance of being
selected, while C is twice as likely to be selected
as D, what are the probabilities that
\begin{enumerate}
\item C will be selected?
\item A will not be selected?
\end{enumerate}
	%\input{exemplar/11/16/3/9/main.tex}
 \item A bag contain 24 balls of which $x$ balls are red, $2x$ are white and $3x$ are blue. A ball is selected at random, What is the probability that it is
\begin{enumerate}[label=\alph*)]
\item not red ?
\item white ?
\end{enumerate}
%\input{exemplar/10/13/3/41/main.tex}
If the letters of the word ASSASSINATION are arranged at random. Find the Probability that
\begin{enumerate}[label=(\alph*)]
\item Four $S's$ come consecutively in the word
\item Two  $I's$ and two $N's$ come together
\item All $A's$ are not coming together
\item No two $A's$ are coming together
\end{enumerate}
%\input{exemplar/11/16/3/14/main.tex}
	\item One urn contains two black balls (labelled B1 and B2) and one white ball. A
	second urn contains one black ball and two white balls (labelled W1 and W2).
	Suppose the following experiment is performed. One of the two urns is chosen
	at random. Next a ball is randomly chosen from the urn. Then a second ball is
	chosen at random from the same urn without replacing the first ball.
	
	\begin{enumerate}
	\item What is the probability that two black balls are chosen?
	
	\item What is the probability that two balls of opposite colour are chosen?
	\end{enumerate}
	\solution
	%\input{exemplar/11/16/3/12/main1.tex}
\end{enumerate}

	\item A bag contains $5$ red balls and some blue balls. If the probability of drawing a blue ball is double that if a red ball, determine the number of blue balls in the bag. 
		\\
\solution
		%\begin{enumerate}[label=\thesection.\arabic*,ref=\thesection.\theenumi]
	\item One card is drawn from a well-shuffled deck of 52 cards. Find the probability of getting
\begin{enumerate}
\item A king of red colour 
\item A face card 
\item A red face card
\item The jack of hearts
\item A spade
\item The queen of diamonds

\end{enumerate}
\solution
		%\input{ncert/10/15/1/14/main.tex}
	\item Five cards—the ten, jack, queen, king and ace of diamonds, are well-shuffled with their face downwards. One card is then picked up at random.
\begin{enumerate}
\item
What is the probability that the card is the queen? 
\item
If the queen is drawn and put aside, what is the probability that the second card picked up is (a) an ace? (b) a queen?\\
\end{enumerate}
\solution
		%\input{ncert/10/15/1/15/defs.tex}
	\item A bag contains $5$ red balls and some blue balls. If the probability of drawing a blue ball is double that if a red ball, determine the number of blue balls in the bag. 
		\\
\solution
		%\input{ncert/10/15/2/3/defs.tex}
	\item A card is selected from a pack of 52 cards.
 \begin{enumerate}[label=(\alph*)] 
                 \item How many points are there in the sample space?
                 \item Calculate the probability that the card is an ace of spades.
                 \item Calculate the probability that the card is (i) an ace and (ii) black card.
 \end{enumerate}
\solution
		%\input{ncert/11/16/3/4/main.tex}
\item Four cards are drawn from a well-shuffled deck of 52 cards. What is the probability of obtaining 3 diamonds and one spade.
\\
\solution
		%\input{ncert/11/16/4/2/defs.tex}
\item In a certain lottery 10,000 tickets are sold and ten equal prizes are awarded. What is the probability of not getting a prize if you buy (a) one ticket (b) two tickets (c) 10 tickets ?	
\\
\solution
		%\input{ncert/11/16/4/4/defs.tex}
		%
\item 
Out of 100 students, two sections of 40 and 60 are formed. If you and your friend are among the 100 students, what is the probability that
\begin{enumerate}
\item you both enter the same section?
\item you both enter the different sections?
\end{enumerate}
\solution
		%\input{ncert/11/16/4/5/defs.tex}
	\item 
The number lock of a suitcase has 4 wheels each labelled with ten digits i.e. from 0 to 9.The lock opens with a sequence of four digits with no repeats.What is the probability of a person getting the right sequence to open the suitcase.
\\
\solution
		%\input{ncert/11/16/4/10/defs.tex}
		%
\item 
Two cards are drawn at random and without replacement from a pack of 52 playing cards. Find the probability that both the cards are black.
\\
\solution
		%\input{ncert/12/13/2/2/defs.tex}
		\item A box of oranges is inspected by examining three randomly selected oranges drawn without replacement. If all the three oranges are good, the box is approved for sale, otherwise, it is rejected. Find the probability that a box containing 15 oranges out of which 12 are good and 3 are bad ones will be approved for sale.
		\label{ncert/12/13/2/3/defs.tex}
		\item Two balls are drawn at random with replacement from a box containing 10 black and 8 red balls. Find the probability that
		\label{ncert/12/13/2/12}
\begin{enumerate}
\item both balls are red.
\item first ball is black and second is red.
\item one of them is black and other is red.
\end{enumerate}

\item In a hostel, 60\% of the students read Hindi newspaper, 40\% read English newspaper and 20\% read both Hindi and English newspapers. A student is selected at random.
		\label{ncert/12/13/2/15}
\begin{enumerate}
\item Find the probability that she reads neither Hindi nor English newspapers.
\item If she reads Hindi newspaper, find the probability that she reads English newspaper.
\item If she reads English newspaper, find the probability that she reads Hindi newspaper.\\
\end{enumerate}
\item The probability of obtaining an even prime number on each die, when a pair of dice is rolled is 
\begin{enumerate}
    \item $0$ 
    
    \item $\frac{1}{3}$ 
    
    \item $\frac{1}{12}$ 
    
    \item $\frac{1}{36}$ 
\end{enumerate}
\solution
		%\input{ncert/12/13/2/17/defs.tex}
	\item A bag contains 4 red and 4 black balls, another bag contains 2 red and 6 black balls. One of the two bags is selected at random and a ball is drawn from the bag which is found to be red. Find the probability that the ball is drawn from the first bag.
\\
\solution
		%\input{ncert/12/13/3/2/main.tex}
  \item
  Cards with numbers 2 to 101 are placed in a box. A card is selected at random.Find the probability that the card has
\begin{enumerate}[label=(\roman*)]
	\item an even number 
	\item a square number
\end{enumerate}
\solution
%\input{exemplar/10/13/3/32/main.tex}
\item
The king, queen and jack of clubs are removed from a deck of 52 playing cards and then well shuffled. Now one card is drawn at random from the remaining cards.  Determine the probability that the card is
\begin{enumerate}[label=(\roman*)]
\item a club
\item 10 of hearts
\end{enumerate}
\solution
%\input{exemplar/10/13/3/29/main.tex}
\item A team of medical students doing their internship have to assist during surgeries
at a city hospital. The probabilities of surgeries rated as very complex, complex,
routine, simple or very simple are respectively, 0.15, 0.20, 0.31, 0.26, .08. Find
the probabilities that a particular surgery will be rated
\begin{enumerate}
	\item complex or very complex;
	\item neither very complex nor very simple;
	\item routine or complex
	\item routine or simple
\end{enumerate}
\solution
%\input{exemplar/11/16/3/8(1)/main.tex}
\item A card is selected from a pack of 52 cards.
\begin{enumerate}[label=(\alph*)]
    \item How many points are there in the sample space?
    \item Calculate the probability that the card is an ace of spades.
    \item Calculate the probability that the card is (i) an ace and (ii) black card.
\end{enumerate}
\solution
%\input{exemplar/11/16/3/4/main2.tex}
\item The probability that a non leap year selected at random will contain 53 sundays.
\\
\solution
%\input{exemplar/10/13/1/19/main.tex}
\item One of the four persons John, Rita, Aslam or Gurpreet will be promoted next
month. Consequently the sample space consists of four elementary outcomes
S = {John promoted, Rita promoted, Aslam promoted, Gurpreet promoted}
You are told that the chances of John’s promotion is same as that of Gurpreet,
Rita’s chances of promotion are twice as likely as Johns. Aslam’s chances are
four times that of John.
\begin{enumerate}
	\item Determine
	\begin{enumerate}
		\item P (John promoted)
		\item P (Rita promoted)
		\item P (Aslam promoted)
		\item P (Gurpreet promoted)
	\end{enumerate}
	\item If A = {John promoted or Gurpreet promoted}, find P (A).
\end{enumerate}
\solution
%\input{exemplar/11/16/3/10/main.tex}
\item A card is drawn from a deck of 52 cards. Find the probability of getting a king or a heart or a red card.\\
\solution
%\input{exemplar/11/16/3/15/main.tex}
\item The probability that a student will pass his examination is 0.73, the probability of
the student getting a compartment is 0.13, and the probability that the student will
either pass or get compartment is 0.96. State True or False.\\
\solution
%\input{exemplar/11/16/3/31/main.tex}
\item A card is selected from a pack of 52 cards\\
\begin{enumerate}[label=(\alph*)]
\item How many points are there in the sample space?
\item Calculate the probability that the cards is an ace of spades.
\item Calculate the probability that the card is (i) an ace (ii)black card.\\
\end{enumerate}
%\input{ncert/11/16/3/4_1/Prob_4.tex}
\item In a non-leap year, the probability of having 53 tuesdays or 53 wednesdays is\\
\solution
%\input{exemplar/11/16/3/18/main.tex}
\item There are 1000 sealed envelopes in a box, 10 of them contain a cash prize of
Rs 100 each, 100 of them contain a cash prize of Rs 50 each and 200 of them
contain a cash prize of Rs 10 each and rest do not contain any cash prize. If they
are well shuffled and an envelope is picked up out, what is the probability that it
contains no cash prize?\\
\solution
%\input{exemplar/10/13/3/34/main.tex}
\item 
A die is thrown and a card is selected at random from a deck of 52 playing cards. The probability of getting an even number on the die and a spade card.\\
\solution
%\input{exemplar/12/13/3/78/main.tex}
\item
If 4-digit numbers greater than 5,000 are randomly formed from the digits 0, 1, 3, 5, and 7, what is the probability of forming a number divisible by 5 when:
\begin{enumerate}
    \item The digits are repeated?
    \item The repetition of digits is not allowed?
\end{enumerate}
\solution
%\input{ncert/11/16/4/9/main.tex}
\item Consider the probability space $\brak{\Omega, \mathcal{G}, P}$ where $\Omega = [0,2]$ and $\mathcal{G} = \cbrak{\phi, \Omega, [0,1], (1,2]}$. Let $X$ and $Y$ be two functions on $\Omega$ defined as
\begin{align*}
    X(\omega) = 
    \begin{cases}
        1 & \text{if }\omega \in [0, 1]\\
        2 & \text{if }\omega \in (1, 2]
    \end{cases}
\end{align*}
and
\begin{align*}
    Y(\omega) = 
    \begin{cases}
        2 & \text{if }\omega \in [0, 1.5]\\
        3 & \text{if }\omega \in (1.5, 2].
    \end{cases}
\end{align*}
Then which one of the following statements is true?
\begin{enumerate}
    \item [(A)] $X$ is a random variable with respect to $\mathcal{G}$, but $Y$ is not a random variable with respect to $\mathcal{G}$.
    \item [(B)] $Y$ is a random variable with respect to $\mathcal{G}$, but $X$ is not a random variable with respect to $\mathcal{G}$.
    \item [(C)] Neither $X$ nor $Y$ is a random variable with respect to $\mathcal{G}$.
    \item [(D)] Both $X$ and $Y$ are random variables with respect to $\mathcal{G}$.
\end{enumerate} \hfill (GATE ST 2023)\\
\solution
%\input{gate/ST/2023/14/main.tex}
	\item  A die is loaded in such a way that each odd number is twice as likely to occur as
each even number. Find $P(G)$, where $G$ is the event that a number greater than
3 occurs on a single roll of the die.
\\
\solution
		%\input{exemplar/11/16/3/5/main.tex}
	\item All the jacks, queens and kings are removed from a deck of 52 playing cards. The remaining cards are well shuffled and then one card is drawn at random. Giving ace a value 1 similar value for other cards, find the probability that the card has a value 
		\begin{enumerate}
			\item 7
			\item greater than 7
			\item less than 7
		\end{enumerate}
		%\input{exemplar/10/13/3/30/main.tex}
  \item A Lot consists of 48 mobile phones of which 42 are good, 3 have only minor defects and 3 have major defects.Varnika will buy a phone if it is good but the trader will only buy a mobile if it has no major defects. One phone is selected at random from the lot. What is the probability that it is
\begin{enumerate}
	\item acceptable to Varnika?
            \item acceptable to the trader?
\end{enumerate}
\solution
	%\input{exemplar/10/13/3/40/main.tex}
 \item A student says that if you throw a die, it will show up 1 or not 1. Therefore, the probability of getting 1 and the probability of getting 'not 1' each is equal to $\frac{1}{2}$. Is this correct? Give reasons.\\
 \solution
        %\input{exemplar/10/13/2/9/main.tex}
   \item Four candidates A, B, C, D have ap-
plied for the assignment to coach a school cricket
team. If A is twice as likely to be selected as B, and
B and C are given about the same chance of being
selected, while C is twice as likely to be selected
as D, what are the probabilities that
\begin{enumerate}
\item C will be selected?
\item A will not be selected?
\end{enumerate}
	%\input{exemplar/11/16/3/9/main.tex}
 \item A bag contain 24 balls of which $x$ balls are red, $2x$ are white and $3x$ are blue. A ball is selected at random, What is the probability that it is
\begin{enumerate}[label=\alph*)]
\item not red ?
\item white ?
\end{enumerate}
%\input{exemplar/10/13/3/41/main.tex}
If the letters of the word ASSASSINATION are arranged at random. Find the Probability that
\begin{enumerate}[label=(\alph*)]
\item Four $S's$ come consecutively in the word
\item Two  $I's$ and two $N's$ come together
\item All $A's$ are not coming together
\item No two $A's$ are coming together
\end{enumerate}
%\input{exemplar/11/16/3/14/main.tex}
	\item One urn contains two black balls (labelled B1 and B2) and one white ball. A
	second urn contains one black ball and two white balls (labelled W1 and W2).
	Suppose the following experiment is performed. One of the two urns is chosen
	at random. Next a ball is randomly chosen from the urn. Then a second ball is
	chosen at random from the same urn without replacing the first ball.
	
	\begin{enumerate}
	\item What is the probability that two black balls are chosen?
	
	\item What is the probability that two balls of opposite colour are chosen?
	\end{enumerate}
	\solution
	%\input{exemplar/11/16/3/12/main1.tex}
\end{enumerate}

	\item A card is selected from a pack of 52 cards.
 \begin{enumerate}[label=(\alph*)] 
                 \item How many points are there in the sample space?
                 \item Calculate the probability that the card is an ace of spades.
                 \item Calculate the probability that the card is (i) an ace and (ii) black card.
 \end{enumerate}
\solution
		%\begin{table}[H]
	\centering
\begin{tabular}{|c|c|c|}
\hline
Random variable &Value &Definition\\ \hline
\multirow{3}{*}{X} &0 &Slips of Rs 1\\
&1 &Slips of Rs 5\\
&2 &Slips of Rs 13\\ \hline
\multirow{2}{*}{Y} &0 &Box A\\
&1 &Box B\\\hline
\end{tabular}
\caption{}
\label{tab:Distribution}
\end{table}
See \tabref{tab:Distribution}.
\begin{align}
p_{Y}\brak{k}= \begin{cases} 
      \frac{1}{3} & {k=0} \\
      \frac{2}{3 }& {k=1} 
   \end{cases}
   \\
p_{Y|X}\brak{0|0} = \frac{19}{25}\, 
p_{Y|X}\brak{0|1} = \frac{6}{25}\,
p_{Y|X}\brak{1|0} = \frac{45}{50}\,
p_{Y|X}\brak{1|2} = \frac{5}{50}
\end{align}
The desired probability is the probability that a slip drawn at random is marked other than Rs 1,
\begin{align}
&=1-p_X\brak{0}\\
&= p_X(1) + p_X(2)
\end{align}
Using Bayes theorem,
\begin{align}
&= p_Y\brak{0} \times \pr{Y=0 | X=1} + p_Y\brak{1} \times \pr{Y=1|X=2}\\
&=\frac{1}{3} \times \frac{6}{25} + \frac{2}{3} \times \frac{5}{50}\\
&=\frac{11}{75}
\end{align}

\newpage

%\tableofcontents

\bigskip

\renewcommand{\thefigure}{\theenumi}
\renewcommand{\thetable}{\theenumi}
%\renewcommand{\theequation}{\theenumi}

%\begin{abstract}
%%\boldmath
%In this letter, an algorithm for evaluating the exact analytical bit error rate  (BER)  for the piecewise linear (PL) combiner for  multiple relays is presented. Previous results were available only for upto three relays. The algorithm is unique in the sense that  the actual mathematical expressions, that are prohibitively large, need not be explicitly obtained. The diversity gain due to multiple relays is shown through plots of the analytical BER, well supported by simulations. 
%
%\end{abstract}
% IEEEtran.cls defaults to using nonbold math in the Abstract.
% This preserves the distinction between vectors and scalars. However,
% if the journal you are submitting to favors bold math in the abstract,
% then you can use LaTeX's standard command \boldmath at the very start
% of the abstract to achieve this. Many IEEE journals frown on math
% in the abstract anyway.

% Note that keywords are not normally used for peerreview papers.
%\begin{IEEEkeywords}
%Cooperative diversity, decode and forward, piecewise linear
%\end{IEEEkeywords}



% For peer review papers, you can put extra information on the cover
% page as needed:
% \ifCLASSOPTIONpeerreview
% \begin{center} \bfseries EDICS Category: 3-BBND \end{center}
% \fi
%
% For peerreview papers, this IEEEtran command inserts a page break and
% creates the second title. It will be ignored for other modes.
%\IEEEpeerreviewmaketitle




\item Four cards are drawn from a well-shuffled deck of 52 cards. What is the probability of obtaining 3 diamonds and one spade.
\\
\solution
		%\begin{enumerate}[label=\thesection.\arabic*,ref=\thesection.\theenumi]
	\item One card is drawn from a well-shuffled deck of 52 cards. Find the probability of getting
\begin{enumerate}
\item A king of red colour 
\item A face card 
\item A red face card
\item The jack of hearts
\item A spade
\item The queen of diamonds

\end{enumerate}
\solution
		%\input{ncert/10/15/1/14/main.tex}
	\item Five cards—the ten, jack, queen, king and ace of diamonds, are well-shuffled with their face downwards. One card is then picked up at random.
\begin{enumerate}
\item
What is the probability that the card is the queen? 
\item
If the queen is drawn and put aside, what is the probability that the second card picked up is (a) an ace? (b) a queen?\\
\end{enumerate}
\solution
		%\input{ncert/10/15/1/15/defs.tex}
	\item A bag contains $5$ red balls and some blue balls. If the probability of drawing a blue ball is double that if a red ball, determine the number of blue balls in the bag. 
		\\
\solution
		%\input{ncert/10/15/2/3/defs.tex}
	\item A card is selected from a pack of 52 cards.
 \begin{enumerate}[label=(\alph*)] 
                 \item How many points are there in the sample space?
                 \item Calculate the probability that the card is an ace of spades.
                 \item Calculate the probability that the card is (i) an ace and (ii) black card.
 \end{enumerate}
\solution
		%\input{ncert/11/16/3/4/main.tex}
\item Four cards are drawn from a well-shuffled deck of 52 cards. What is the probability of obtaining 3 diamonds and one spade.
\\
\solution
		%\input{ncert/11/16/4/2/defs.tex}
\item In a certain lottery 10,000 tickets are sold and ten equal prizes are awarded. What is the probability of not getting a prize if you buy (a) one ticket (b) two tickets (c) 10 tickets ?	
\\
\solution
		%\input{ncert/11/16/4/4/defs.tex}
		%
\item 
Out of 100 students, two sections of 40 and 60 are formed. If you and your friend are among the 100 students, what is the probability that
\begin{enumerate}
\item you both enter the same section?
\item you both enter the different sections?
\end{enumerate}
\solution
		%\input{ncert/11/16/4/5/defs.tex}
	\item 
The number lock of a suitcase has 4 wheels each labelled with ten digits i.e. from 0 to 9.The lock opens with a sequence of four digits with no repeats.What is the probability of a person getting the right sequence to open the suitcase.
\\
\solution
		%\input{ncert/11/16/4/10/defs.tex}
		%
\item 
Two cards are drawn at random and without replacement from a pack of 52 playing cards. Find the probability that both the cards are black.
\\
\solution
		%\input{ncert/12/13/2/2/defs.tex}
		\item A box of oranges is inspected by examining three randomly selected oranges drawn without replacement. If all the three oranges are good, the box is approved for sale, otherwise, it is rejected. Find the probability that a box containing 15 oranges out of which 12 are good and 3 are bad ones will be approved for sale.
		\label{ncert/12/13/2/3/defs.tex}
		\item Two balls are drawn at random with replacement from a box containing 10 black and 8 red balls. Find the probability that
		\label{ncert/12/13/2/12}
\begin{enumerate}
\item both balls are red.
\item first ball is black and second is red.
\item one of them is black and other is red.
\end{enumerate}

\item In a hostel, 60\% of the students read Hindi newspaper, 40\% read English newspaper and 20\% read both Hindi and English newspapers. A student is selected at random.
		\label{ncert/12/13/2/15}
\begin{enumerate}
\item Find the probability that she reads neither Hindi nor English newspapers.
\item If she reads Hindi newspaper, find the probability that she reads English newspaper.
\item If she reads English newspaper, find the probability that she reads Hindi newspaper.\\
\end{enumerate}
\item The probability of obtaining an even prime number on each die, when a pair of dice is rolled is 
\begin{enumerate}
    \item $0$ 
    
    \item $\frac{1}{3}$ 
    
    \item $\frac{1}{12}$ 
    
    \item $\frac{1}{36}$ 
\end{enumerate}
\solution
		%\input{ncert/12/13/2/17/defs.tex}
	\item A bag contains 4 red and 4 black balls, another bag contains 2 red and 6 black balls. One of the two bags is selected at random and a ball is drawn from the bag which is found to be red. Find the probability that the ball is drawn from the first bag.
\\
\solution
		%\input{ncert/12/13/3/2/main.tex}
  \item
  Cards with numbers 2 to 101 are placed in a box. A card is selected at random.Find the probability that the card has
\begin{enumerate}[label=(\roman*)]
	\item an even number 
	\item a square number
\end{enumerate}
\solution
%\input{exemplar/10/13/3/32/main.tex}
\item
The king, queen and jack of clubs are removed from a deck of 52 playing cards and then well shuffled. Now one card is drawn at random from the remaining cards.  Determine the probability that the card is
\begin{enumerate}[label=(\roman*)]
\item a club
\item 10 of hearts
\end{enumerate}
\solution
%\input{exemplar/10/13/3/29/main.tex}
\item A team of medical students doing their internship have to assist during surgeries
at a city hospital. The probabilities of surgeries rated as very complex, complex,
routine, simple or very simple are respectively, 0.15, 0.20, 0.31, 0.26, .08. Find
the probabilities that a particular surgery will be rated
\begin{enumerate}
	\item complex or very complex;
	\item neither very complex nor very simple;
	\item routine or complex
	\item routine or simple
\end{enumerate}
\solution
%\input{exemplar/11/16/3/8(1)/main.tex}
\item A card is selected from a pack of 52 cards.
\begin{enumerate}[label=(\alph*)]
    \item How many points are there in the sample space?
    \item Calculate the probability that the card is an ace of spades.
    \item Calculate the probability that the card is (i) an ace and (ii) black card.
\end{enumerate}
\solution
%\input{exemplar/11/16/3/4/main2.tex}
\item The probability that a non leap year selected at random will contain 53 sundays.
\\
\solution
%\input{exemplar/10/13/1/19/main.tex}
\item One of the four persons John, Rita, Aslam or Gurpreet will be promoted next
month. Consequently the sample space consists of four elementary outcomes
S = {John promoted, Rita promoted, Aslam promoted, Gurpreet promoted}
You are told that the chances of John’s promotion is same as that of Gurpreet,
Rita’s chances of promotion are twice as likely as Johns. Aslam’s chances are
four times that of John.
\begin{enumerate}
	\item Determine
	\begin{enumerate}
		\item P (John promoted)
		\item P (Rita promoted)
		\item P (Aslam promoted)
		\item P (Gurpreet promoted)
	\end{enumerate}
	\item If A = {John promoted or Gurpreet promoted}, find P (A).
\end{enumerate}
\solution
%\input{exemplar/11/16/3/10/main.tex}
\item A card is drawn from a deck of 52 cards. Find the probability of getting a king or a heart or a red card.\\
\solution
%\input{exemplar/11/16/3/15/main.tex}
\item The probability that a student will pass his examination is 0.73, the probability of
the student getting a compartment is 0.13, and the probability that the student will
either pass or get compartment is 0.96. State True or False.\\
\solution
%\input{exemplar/11/16/3/31/main.tex}
\item A card is selected from a pack of 52 cards\\
\begin{enumerate}[label=(\alph*)]
\item How many points are there in the sample space?
\item Calculate the probability that the cards is an ace of spades.
\item Calculate the probability that the card is (i) an ace (ii)black card.\\
\end{enumerate}
%\input{ncert/11/16/3/4_1/Prob_4.tex}
\item In a non-leap year, the probability of having 53 tuesdays or 53 wednesdays is\\
\solution
%\input{exemplar/11/16/3/18/main.tex}
\item There are 1000 sealed envelopes in a box, 10 of them contain a cash prize of
Rs 100 each, 100 of them contain a cash prize of Rs 50 each and 200 of them
contain a cash prize of Rs 10 each and rest do not contain any cash prize. If they
are well shuffled and an envelope is picked up out, what is the probability that it
contains no cash prize?\\
\solution
%\input{exemplar/10/13/3/34/main.tex}
\item 
A die is thrown and a card is selected at random from a deck of 52 playing cards. The probability of getting an even number on the die and a spade card.\\
\solution
%\input{exemplar/12/13/3/78/main.tex}
\item
If 4-digit numbers greater than 5,000 are randomly formed from the digits 0, 1, 3, 5, and 7, what is the probability of forming a number divisible by 5 when:
\begin{enumerate}
    \item The digits are repeated?
    \item The repetition of digits is not allowed?
\end{enumerate}
\solution
%\input{ncert/11/16/4/9/main.tex}
\item Consider the probability space $\brak{\Omega, \mathcal{G}, P}$ where $\Omega = [0,2]$ and $\mathcal{G} = \cbrak{\phi, \Omega, [0,1], (1,2]}$. Let $X$ and $Y$ be two functions on $\Omega$ defined as
\begin{align*}
    X(\omega) = 
    \begin{cases}
        1 & \text{if }\omega \in [0, 1]\\
        2 & \text{if }\omega \in (1, 2]
    \end{cases}
\end{align*}
and
\begin{align*}
    Y(\omega) = 
    \begin{cases}
        2 & \text{if }\omega \in [0, 1.5]\\
        3 & \text{if }\omega \in (1.5, 2].
    \end{cases}
\end{align*}
Then which one of the following statements is true?
\begin{enumerate}
    \item [(A)] $X$ is a random variable with respect to $\mathcal{G}$, but $Y$ is not a random variable with respect to $\mathcal{G}$.
    \item [(B)] $Y$ is a random variable with respect to $\mathcal{G}$, but $X$ is not a random variable with respect to $\mathcal{G}$.
    \item [(C)] Neither $X$ nor $Y$ is a random variable with respect to $\mathcal{G}$.
    \item [(D)] Both $X$ and $Y$ are random variables with respect to $\mathcal{G}$.
\end{enumerate} \hfill (GATE ST 2023)\\
\solution
%\input{gate/ST/2023/14/main.tex}
	\item  A die is loaded in such a way that each odd number is twice as likely to occur as
each even number. Find $P(G)$, where $G$ is the event that a number greater than
3 occurs on a single roll of the die.
\\
\solution
		%\input{exemplar/11/16/3/5/main.tex}
	\item All the jacks, queens and kings are removed from a deck of 52 playing cards. The remaining cards are well shuffled and then one card is drawn at random. Giving ace a value 1 similar value for other cards, find the probability that the card has a value 
		\begin{enumerate}
			\item 7
			\item greater than 7
			\item less than 7
		\end{enumerate}
		%\input{exemplar/10/13/3/30/main.tex}
  \item A Lot consists of 48 mobile phones of which 42 are good, 3 have only minor defects and 3 have major defects.Varnika will buy a phone if it is good but the trader will only buy a mobile if it has no major defects. One phone is selected at random from the lot. What is the probability that it is
\begin{enumerate}
	\item acceptable to Varnika?
            \item acceptable to the trader?
\end{enumerate}
\solution
	%\input{exemplar/10/13/3/40/main.tex}
 \item A student says that if you throw a die, it will show up 1 or not 1. Therefore, the probability of getting 1 and the probability of getting 'not 1' each is equal to $\frac{1}{2}$. Is this correct? Give reasons.\\
 \solution
        %\input{exemplar/10/13/2/9/main.tex}
   \item Four candidates A, B, C, D have ap-
plied for the assignment to coach a school cricket
team. If A is twice as likely to be selected as B, and
B and C are given about the same chance of being
selected, while C is twice as likely to be selected
as D, what are the probabilities that
\begin{enumerate}
\item C will be selected?
\item A will not be selected?
\end{enumerate}
	%\input{exemplar/11/16/3/9/main.tex}
 \item A bag contain 24 balls of which $x$ balls are red, $2x$ are white and $3x$ are blue. A ball is selected at random, What is the probability that it is
\begin{enumerate}[label=\alph*)]
\item not red ?
\item white ?
\end{enumerate}
%\input{exemplar/10/13/3/41/main.tex}
If the letters of the word ASSASSINATION are arranged at random. Find the Probability that
\begin{enumerate}[label=(\alph*)]
\item Four $S's$ come consecutively in the word
\item Two  $I's$ and two $N's$ come together
\item All $A's$ are not coming together
\item No two $A's$ are coming together
\end{enumerate}
%\input{exemplar/11/16/3/14/main.tex}
	\item One urn contains two black balls (labelled B1 and B2) and one white ball. A
	second urn contains one black ball and two white balls (labelled W1 and W2).
	Suppose the following experiment is performed. One of the two urns is chosen
	at random. Next a ball is randomly chosen from the urn. Then a second ball is
	chosen at random from the same urn without replacing the first ball.
	
	\begin{enumerate}
	\item What is the probability that two black balls are chosen?
	
	\item What is the probability that two balls of opposite colour are chosen?
	\end{enumerate}
	\solution
	%\input{exemplar/11/16/3/12/main1.tex}
\end{enumerate}

\item In a certain lottery 10,000 tickets are sold and ten equal prizes are awarded. What is the probability of not getting a prize if you buy (a) one ticket (b) two tickets (c) 10 tickets ?	
\\
\solution
		%\begin{enumerate}[label=\thesection.\arabic*,ref=\thesection.\theenumi]
	\item One card is drawn from a well-shuffled deck of 52 cards. Find the probability of getting
\begin{enumerate}
\item A king of red colour 
\item A face card 
\item A red face card
\item The jack of hearts
\item A spade
\item The queen of diamonds

\end{enumerate}
\solution
		%\input{ncert/10/15/1/14/main.tex}
	\item Five cards—the ten, jack, queen, king and ace of diamonds, are well-shuffled with their face downwards. One card is then picked up at random.
\begin{enumerate}
\item
What is the probability that the card is the queen? 
\item
If the queen is drawn and put aside, what is the probability that the second card picked up is (a) an ace? (b) a queen?\\
\end{enumerate}
\solution
		%\input{ncert/10/15/1/15/defs.tex}
	\item A bag contains $5$ red balls and some blue balls. If the probability of drawing a blue ball is double that if a red ball, determine the number of blue balls in the bag. 
		\\
\solution
		%\input{ncert/10/15/2/3/defs.tex}
	\item A card is selected from a pack of 52 cards.
 \begin{enumerate}[label=(\alph*)] 
                 \item How many points are there in the sample space?
                 \item Calculate the probability that the card is an ace of spades.
                 \item Calculate the probability that the card is (i) an ace and (ii) black card.
 \end{enumerate}
\solution
		%\input{ncert/11/16/3/4/main.tex}
\item Four cards are drawn from a well-shuffled deck of 52 cards. What is the probability of obtaining 3 diamonds and one spade.
\\
\solution
		%\input{ncert/11/16/4/2/defs.tex}
\item In a certain lottery 10,000 tickets are sold and ten equal prizes are awarded. What is the probability of not getting a prize if you buy (a) one ticket (b) two tickets (c) 10 tickets ?	
\\
\solution
		%\input{ncert/11/16/4/4/defs.tex}
		%
\item 
Out of 100 students, two sections of 40 and 60 are formed. If you and your friend are among the 100 students, what is the probability that
\begin{enumerate}
\item you both enter the same section?
\item you both enter the different sections?
\end{enumerate}
\solution
		%\input{ncert/11/16/4/5/defs.tex}
	\item 
The number lock of a suitcase has 4 wheels each labelled with ten digits i.e. from 0 to 9.The lock opens with a sequence of four digits with no repeats.What is the probability of a person getting the right sequence to open the suitcase.
\\
\solution
		%\input{ncert/11/16/4/10/defs.tex}
		%
\item 
Two cards are drawn at random and without replacement from a pack of 52 playing cards. Find the probability that both the cards are black.
\\
\solution
		%\input{ncert/12/13/2/2/defs.tex}
		\item A box of oranges is inspected by examining three randomly selected oranges drawn without replacement. If all the three oranges are good, the box is approved for sale, otherwise, it is rejected. Find the probability that a box containing 15 oranges out of which 12 are good and 3 are bad ones will be approved for sale.
		\label{ncert/12/13/2/3/defs.tex}
		\item Two balls are drawn at random with replacement from a box containing 10 black and 8 red balls. Find the probability that
		\label{ncert/12/13/2/12}
\begin{enumerate}
\item both balls are red.
\item first ball is black and second is red.
\item one of them is black and other is red.
\end{enumerate}

\item In a hostel, 60\% of the students read Hindi newspaper, 40\% read English newspaper and 20\% read both Hindi and English newspapers. A student is selected at random.
		\label{ncert/12/13/2/15}
\begin{enumerate}
\item Find the probability that she reads neither Hindi nor English newspapers.
\item If she reads Hindi newspaper, find the probability that she reads English newspaper.
\item If she reads English newspaper, find the probability that she reads Hindi newspaper.\\
\end{enumerate}
\item The probability of obtaining an even prime number on each die, when a pair of dice is rolled is 
\begin{enumerate}
    \item $0$ 
    
    \item $\frac{1}{3}$ 
    
    \item $\frac{1}{12}$ 
    
    \item $\frac{1}{36}$ 
\end{enumerate}
\solution
		%\input{ncert/12/13/2/17/defs.tex}
	\item A bag contains 4 red and 4 black balls, another bag contains 2 red and 6 black balls. One of the two bags is selected at random and a ball is drawn from the bag which is found to be red. Find the probability that the ball is drawn from the first bag.
\\
\solution
		%\input{ncert/12/13/3/2/main.tex}
  \item
  Cards with numbers 2 to 101 are placed in a box. A card is selected at random.Find the probability that the card has
\begin{enumerate}[label=(\roman*)]
	\item an even number 
	\item a square number
\end{enumerate}
\solution
%\input{exemplar/10/13/3/32/main.tex}
\item
The king, queen and jack of clubs are removed from a deck of 52 playing cards and then well shuffled. Now one card is drawn at random from the remaining cards.  Determine the probability that the card is
\begin{enumerate}[label=(\roman*)]
\item a club
\item 10 of hearts
\end{enumerate}
\solution
%\input{exemplar/10/13/3/29/main.tex}
\item A team of medical students doing their internship have to assist during surgeries
at a city hospital. The probabilities of surgeries rated as very complex, complex,
routine, simple or very simple are respectively, 0.15, 0.20, 0.31, 0.26, .08. Find
the probabilities that a particular surgery will be rated
\begin{enumerate}
	\item complex or very complex;
	\item neither very complex nor very simple;
	\item routine or complex
	\item routine or simple
\end{enumerate}
\solution
%\input{exemplar/11/16/3/8(1)/main.tex}
\item A card is selected from a pack of 52 cards.
\begin{enumerate}[label=(\alph*)]
    \item How many points are there in the sample space?
    \item Calculate the probability that the card is an ace of spades.
    \item Calculate the probability that the card is (i) an ace and (ii) black card.
\end{enumerate}
\solution
%\input{exemplar/11/16/3/4/main2.tex}
\item The probability that a non leap year selected at random will contain 53 sundays.
\\
\solution
%\input{exemplar/10/13/1/19/main.tex}
\item One of the four persons John, Rita, Aslam or Gurpreet will be promoted next
month. Consequently the sample space consists of four elementary outcomes
S = {John promoted, Rita promoted, Aslam promoted, Gurpreet promoted}
You are told that the chances of John’s promotion is same as that of Gurpreet,
Rita’s chances of promotion are twice as likely as Johns. Aslam’s chances are
four times that of John.
\begin{enumerate}
	\item Determine
	\begin{enumerate}
		\item P (John promoted)
		\item P (Rita promoted)
		\item P (Aslam promoted)
		\item P (Gurpreet promoted)
	\end{enumerate}
	\item If A = {John promoted or Gurpreet promoted}, find P (A).
\end{enumerate}
\solution
%\input{exemplar/11/16/3/10/main.tex}
\item A card is drawn from a deck of 52 cards. Find the probability of getting a king or a heart or a red card.\\
\solution
%\input{exemplar/11/16/3/15/main.tex}
\item The probability that a student will pass his examination is 0.73, the probability of
the student getting a compartment is 0.13, and the probability that the student will
either pass or get compartment is 0.96. State True or False.\\
\solution
%\input{exemplar/11/16/3/31/main.tex}
\item A card is selected from a pack of 52 cards\\
\begin{enumerate}[label=(\alph*)]
\item How many points are there in the sample space?
\item Calculate the probability that the cards is an ace of spades.
\item Calculate the probability that the card is (i) an ace (ii)black card.\\
\end{enumerate}
%\input{ncert/11/16/3/4_1/Prob_4.tex}
\item In a non-leap year, the probability of having 53 tuesdays or 53 wednesdays is\\
\solution
%\input{exemplar/11/16/3/18/main.tex}
\item There are 1000 sealed envelopes in a box, 10 of them contain a cash prize of
Rs 100 each, 100 of them contain a cash prize of Rs 50 each and 200 of them
contain a cash prize of Rs 10 each and rest do not contain any cash prize. If they
are well shuffled and an envelope is picked up out, what is the probability that it
contains no cash prize?\\
\solution
%\input{exemplar/10/13/3/34/main.tex}
\item 
A die is thrown and a card is selected at random from a deck of 52 playing cards. The probability of getting an even number on the die and a spade card.\\
\solution
%\input{exemplar/12/13/3/78/main.tex}
\item
If 4-digit numbers greater than 5,000 are randomly formed from the digits 0, 1, 3, 5, and 7, what is the probability of forming a number divisible by 5 when:
\begin{enumerate}
    \item The digits are repeated?
    \item The repetition of digits is not allowed?
\end{enumerate}
\solution
%\input{ncert/11/16/4/9/main.tex}
\item Consider the probability space $\brak{\Omega, \mathcal{G}, P}$ where $\Omega = [0,2]$ and $\mathcal{G} = \cbrak{\phi, \Omega, [0,1], (1,2]}$. Let $X$ and $Y$ be two functions on $\Omega$ defined as
\begin{align*}
    X(\omega) = 
    \begin{cases}
        1 & \text{if }\omega \in [0, 1]\\
        2 & \text{if }\omega \in (1, 2]
    \end{cases}
\end{align*}
and
\begin{align*}
    Y(\omega) = 
    \begin{cases}
        2 & \text{if }\omega \in [0, 1.5]\\
        3 & \text{if }\omega \in (1.5, 2].
    \end{cases}
\end{align*}
Then which one of the following statements is true?
\begin{enumerate}
    \item [(A)] $X$ is a random variable with respect to $\mathcal{G}$, but $Y$ is not a random variable with respect to $\mathcal{G}$.
    \item [(B)] $Y$ is a random variable with respect to $\mathcal{G}$, but $X$ is not a random variable with respect to $\mathcal{G}$.
    \item [(C)] Neither $X$ nor $Y$ is a random variable with respect to $\mathcal{G}$.
    \item [(D)] Both $X$ and $Y$ are random variables with respect to $\mathcal{G}$.
\end{enumerate} \hfill (GATE ST 2023)\\
\solution
%\input{gate/ST/2023/14/main.tex}
	\item  A die is loaded in such a way that each odd number is twice as likely to occur as
each even number. Find $P(G)$, where $G$ is the event that a number greater than
3 occurs on a single roll of the die.
\\
\solution
		%\input{exemplar/11/16/3/5/main.tex}
	\item All the jacks, queens and kings are removed from a deck of 52 playing cards. The remaining cards are well shuffled and then one card is drawn at random. Giving ace a value 1 similar value for other cards, find the probability that the card has a value 
		\begin{enumerate}
			\item 7
			\item greater than 7
			\item less than 7
		\end{enumerate}
		%\input{exemplar/10/13/3/30/main.tex}
  \item A Lot consists of 48 mobile phones of which 42 are good, 3 have only minor defects and 3 have major defects.Varnika will buy a phone if it is good but the trader will only buy a mobile if it has no major defects. One phone is selected at random from the lot. What is the probability that it is
\begin{enumerate}
	\item acceptable to Varnika?
            \item acceptable to the trader?
\end{enumerate}
\solution
	%\input{exemplar/10/13/3/40/main.tex}
 \item A student says that if you throw a die, it will show up 1 or not 1. Therefore, the probability of getting 1 and the probability of getting 'not 1' each is equal to $\frac{1}{2}$. Is this correct? Give reasons.\\
 \solution
        %\input{exemplar/10/13/2/9/main.tex}
   \item Four candidates A, B, C, D have ap-
plied for the assignment to coach a school cricket
team. If A is twice as likely to be selected as B, and
B and C are given about the same chance of being
selected, while C is twice as likely to be selected
as D, what are the probabilities that
\begin{enumerate}
\item C will be selected?
\item A will not be selected?
\end{enumerate}
	%\input{exemplar/11/16/3/9/main.tex}
 \item A bag contain 24 balls of which $x$ balls are red, $2x$ are white and $3x$ are blue. A ball is selected at random, What is the probability that it is
\begin{enumerate}[label=\alph*)]
\item not red ?
\item white ?
\end{enumerate}
%\input{exemplar/10/13/3/41/main.tex}
If the letters of the word ASSASSINATION are arranged at random. Find the Probability that
\begin{enumerate}[label=(\alph*)]
\item Four $S's$ come consecutively in the word
\item Two  $I's$ and two $N's$ come together
\item All $A's$ are not coming together
\item No two $A's$ are coming together
\end{enumerate}
%\input{exemplar/11/16/3/14/main.tex}
	\item One urn contains two black balls (labelled B1 and B2) and one white ball. A
	second urn contains one black ball and two white balls (labelled W1 and W2).
	Suppose the following experiment is performed. One of the two urns is chosen
	at random. Next a ball is randomly chosen from the urn. Then a second ball is
	chosen at random from the same urn without replacing the first ball.
	
	\begin{enumerate}
	\item What is the probability that two black balls are chosen?
	
	\item What is the probability that two balls of opposite colour are chosen?
	\end{enumerate}
	\solution
	%\input{exemplar/11/16/3/12/main1.tex}
\end{enumerate}

		%
\item 
Out of 100 students, two sections of 40 and 60 are formed. If you and your friend are among the 100 students, what is the probability that
\begin{enumerate}
\item you both enter the same section?
\item you both enter the different sections?
\end{enumerate}
\solution
		%\begin{enumerate}[label=\thesection.\arabic*,ref=\thesection.\theenumi]
	\item One card is drawn from a well-shuffled deck of 52 cards. Find the probability of getting
\begin{enumerate}
\item A king of red colour 
\item A face card 
\item A red face card
\item The jack of hearts
\item A spade
\item The queen of diamonds

\end{enumerate}
\solution
		%\input{ncert/10/15/1/14/main.tex}
	\item Five cards—the ten, jack, queen, king and ace of diamonds, are well-shuffled with their face downwards. One card is then picked up at random.
\begin{enumerate}
\item
What is the probability that the card is the queen? 
\item
If the queen is drawn and put aside, what is the probability that the second card picked up is (a) an ace? (b) a queen?\\
\end{enumerate}
\solution
		%\input{ncert/10/15/1/15/defs.tex}
	\item A bag contains $5$ red balls and some blue balls. If the probability of drawing a blue ball is double that if a red ball, determine the number of blue balls in the bag. 
		\\
\solution
		%\input{ncert/10/15/2/3/defs.tex}
	\item A card is selected from a pack of 52 cards.
 \begin{enumerate}[label=(\alph*)] 
                 \item How many points are there in the sample space?
                 \item Calculate the probability that the card is an ace of spades.
                 \item Calculate the probability that the card is (i) an ace and (ii) black card.
 \end{enumerate}
\solution
		%\input{ncert/11/16/3/4/main.tex}
\item Four cards are drawn from a well-shuffled deck of 52 cards. What is the probability of obtaining 3 diamonds and one spade.
\\
\solution
		%\input{ncert/11/16/4/2/defs.tex}
\item In a certain lottery 10,000 tickets are sold and ten equal prizes are awarded. What is the probability of not getting a prize if you buy (a) one ticket (b) two tickets (c) 10 tickets ?	
\\
\solution
		%\input{ncert/11/16/4/4/defs.tex}
		%
\item 
Out of 100 students, two sections of 40 and 60 are formed. If you and your friend are among the 100 students, what is the probability that
\begin{enumerate}
\item you both enter the same section?
\item you both enter the different sections?
\end{enumerate}
\solution
		%\input{ncert/11/16/4/5/defs.tex}
	\item 
The number lock of a suitcase has 4 wheels each labelled with ten digits i.e. from 0 to 9.The lock opens with a sequence of four digits with no repeats.What is the probability of a person getting the right sequence to open the suitcase.
\\
\solution
		%\input{ncert/11/16/4/10/defs.tex}
		%
\item 
Two cards are drawn at random and without replacement from a pack of 52 playing cards. Find the probability that both the cards are black.
\\
\solution
		%\input{ncert/12/13/2/2/defs.tex}
		\item A box of oranges is inspected by examining three randomly selected oranges drawn without replacement. If all the three oranges are good, the box is approved for sale, otherwise, it is rejected. Find the probability that a box containing 15 oranges out of which 12 are good and 3 are bad ones will be approved for sale.
		\label{ncert/12/13/2/3/defs.tex}
		\item Two balls are drawn at random with replacement from a box containing 10 black and 8 red balls. Find the probability that
		\label{ncert/12/13/2/12}
\begin{enumerate}
\item both balls are red.
\item first ball is black and second is red.
\item one of them is black and other is red.
\end{enumerate}

\item In a hostel, 60\% of the students read Hindi newspaper, 40\% read English newspaper and 20\% read both Hindi and English newspapers. A student is selected at random.
		\label{ncert/12/13/2/15}
\begin{enumerate}
\item Find the probability that she reads neither Hindi nor English newspapers.
\item If she reads Hindi newspaper, find the probability that she reads English newspaper.
\item If she reads English newspaper, find the probability that she reads Hindi newspaper.\\
\end{enumerate}
\item The probability of obtaining an even prime number on each die, when a pair of dice is rolled is 
\begin{enumerate}
    \item $0$ 
    
    \item $\frac{1}{3}$ 
    
    \item $\frac{1}{12}$ 
    
    \item $\frac{1}{36}$ 
\end{enumerate}
\solution
		%\input{ncert/12/13/2/17/defs.tex}
	\item A bag contains 4 red and 4 black balls, another bag contains 2 red and 6 black balls. One of the two bags is selected at random and a ball is drawn from the bag which is found to be red. Find the probability that the ball is drawn from the first bag.
\\
\solution
		%\input{ncert/12/13/3/2/main.tex}
  \item
  Cards with numbers 2 to 101 are placed in a box. A card is selected at random.Find the probability that the card has
\begin{enumerate}[label=(\roman*)]
	\item an even number 
	\item a square number
\end{enumerate}
\solution
%\input{exemplar/10/13/3/32/main.tex}
\item
The king, queen and jack of clubs are removed from a deck of 52 playing cards and then well shuffled. Now one card is drawn at random from the remaining cards.  Determine the probability that the card is
\begin{enumerate}[label=(\roman*)]
\item a club
\item 10 of hearts
\end{enumerate}
\solution
%\input{exemplar/10/13/3/29/main.tex}
\item A team of medical students doing their internship have to assist during surgeries
at a city hospital. The probabilities of surgeries rated as very complex, complex,
routine, simple or very simple are respectively, 0.15, 0.20, 0.31, 0.26, .08. Find
the probabilities that a particular surgery will be rated
\begin{enumerate}
	\item complex or very complex;
	\item neither very complex nor very simple;
	\item routine or complex
	\item routine or simple
\end{enumerate}
\solution
%\input{exemplar/11/16/3/8(1)/main.tex}
\item A card is selected from a pack of 52 cards.
\begin{enumerate}[label=(\alph*)]
    \item How many points are there in the sample space?
    \item Calculate the probability that the card is an ace of spades.
    \item Calculate the probability that the card is (i) an ace and (ii) black card.
\end{enumerate}
\solution
%\input{exemplar/11/16/3/4/main2.tex}
\item The probability that a non leap year selected at random will contain 53 sundays.
\\
\solution
%\input{exemplar/10/13/1/19/main.tex}
\item One of the four persons John, Rita, Aslam or Gurpreet will be promoted next
month. Consequently the sample space consists of four elementary outcomes
S = {John promoted, Rita promoted, Aslam promoted, Gurpreet promoted}
You are told that the chances of John’s promotion is same as that of Gurpreet,
Rita’s chances of promotion are twice as likely as Johns. Aslam’s chances are
four times that of John.
\begin{enumerate}
	\item Determine
	\begin{enumerate}
		\item P (John promoted)
		\item P (Rita promoted)
		\item P (Aslam promoted)
		\item P (Gurpreet promoted)
	\end{enumerate}
	\item If A = {John promoted or Gurpreet promoted}, find P (A).
\end{enumerate}
\solution
%\input{exemplar/11/16/3/10/main.tex}
\item A card is drawn from a deck of 52 cards. Find the probability of getting a king or a heart or a red card.\\
\solution
%\input{exemplar/11/16/3/15/main.tex}
\item The probability that a student will pass his examination is 0.73, the probability of
the student getting a compartment is 0.13, and the probability that the student will
either pass or get compartment is 0.96. State True or False.\\
\solution
%\input{exemplar/11/16/3/31/main.tex}
\item A card is selected from a pack of 52 cards\\
\begin{enumerate}[label=(\alph*)]
\item How many points are there in the sample space?
\item Calculate the probability that the cards is an ace of spades.
\item Calculate the probability that the card is (i) an ace (ii)black card.\\
\end{enumerate}
%\input{ncert/11/16/3/4_1/Prob_4.tex}
\item In a non-leap year, the probability of having 53 tuesdays or 53 wednesdays is\\
\solution
%\input{exemplar/11/16/3/18/main.tex}
\item There are 1000 sealed envelopes in a box, 10 of them contain a cash prize of
Rs 100 each, 100 of them contain a cash prize of Rs 50 each and 200 of them
contain a cash prize of Rs 10 each and rest do not contain any cash prize. If they
are well shuffled and an envelope is picked up out, what is the probability that it
contains no cash prize?\\
\solution
%\input{exemplar/10/13/3/34/main.tex}
\item 
A die is thrown and a card is selected at random from a deck of 52 playing cards. The probability of getting an even number on the die and a spade card.\\
\solution
%\input{exemplar/12/13/3/78/main.tex}
\item
If 4-digit numbers greater than 5,000 are randomly formed from the digits 0, 1, 3, 5, and 7, what is the probability of forming a number divisible by 5 when:
\begin{enumerate}
    \item The digits are repeated?
    \item The repetition of digits is not allowed?
\end{enumerate}
\solution
%\input{ncert/11/16/4/9/main.tex}
\item Consider the probability space $\brak{\Omega, \mathcal{G}, P}$ where $\Omega = [0,2]$ and $\mathcal{G} = \cbrak{\phi, \Omega, [0,1], (1,2]}$. Let $X$ and $Y$ be two functions on $\Omega$ defined as
\begin{align*}
    X(\omega) = 
    \begin{cases}
        1 & \text{if }\omega \in [0, 1]\\
        2 & \text{if }\omega \in (1, 2]
    \end{cases}
\end{align*}
and
\begin{align*}
    Y(\omega) = 
    \begin{cases}
        2 & \text{if }\omega \in [0, 1.5]\\
        3 & \text{if }\omega \in (1.5, 2].
    \end{cases}
\end{align*}
Then which one of the following statements is true?
\begin{enumerate}
    \item [(A)] $X$ is a random variable with respect to $\mathcal{G}$, but $Y$ is not a random variable with respect to $\mathcal{G}$.
    \item [(B)] $Y$ is a random variable with respect to $\mathcal{G}$, but $X$ is not a random variable with respect to $\mathcal{G}$.
    \item [(C)] Neither $X$ nor $Y$ is a random variable with respect to $\mathcal{G}$.
    \item [(D)] Both $X$ and $Y$ are random variables with respect to $\mathcal{G}$.
\end{enumerate} \hfill (GATE ST 2023)\\
\solution
%\input{gate/ST/2023/14/main.tex}
	\item  A die is loaded in such a way that each odd number is twice as likely to occur as
each even number. Find $P(G)$, where $G$ is the event that a number greater than
3 occurs on a single roll of the die.
\\
\solution
		%\input{exemplar/11/16/3/5/main.tex}
	\item All the jacks, queens and kings are removed from a deck of 52 playing cards. The remaining cards are well shuffled and then one card is drawn at random. Giving ace a value 1 similar value for other cards, find the probability that the card has a value 
		\begin{enumerate}
			\item 7
			\item greater than 7
			\item less than 7
		\end{enumerate}
		%\input{exemplar/10/13/3/30/main.tex}
  \item A Lot consists of 48 mobile phones of which 42 are good, 3 have only minor defects and 3 have major defects.Varnika will buy a phone if it is good but the trader will only buy a mobile if it has no major defects. One phone is selected at random from the lot. What is the probability that it is
\begin{enumerate}
	\item acceptable to Varnika?
            \item acceptable to the trader?
\end{enumerate}
\solution
	%\input{exemplar/10/13/3/40/main.tex}
 \item A student says that if you throw a die, it will show up 1 or not 1. Therefore, the probability of getting 1 and the probability of getting 'not 1' each is equal to $\frac{1}{2}$. Is this correct? Give reasons.\\
 \solution
        %\input{exemplar/10/13/2/9/main.tex}
   \item Four candidates A, B, C, D have ap-
plied for the assignment to coach a school cricket
team. If A is twice as likely to be selected as B, and
B and C are given about the same chance of being
selected, while C is twice as likely to be selected
as D, what are the probabilities that
\begin{enumerate}
\item C will be selected?
\item A will not be selected?
\end{enumerate}
	%\input{exemplar/11/16/3/9/main.tex}
 \item A bag contain 24 balls of which $x$ balls are red, $2x$ are white and $3x$ are blue. A ball is selected at random, What is the probability that it is
\begin{enumerate}[label=\alph*)]
\item not red ?
\item white ?
\end{enumerate}
%\input{exemplar/10/13/3/41/main.tex}
If the letters of the word ASSASSINATION are arranged at random. Find the Probability that
\begin{enumerate}[label=(\alph*)]
\item Four $S's$ come consecutively in the word
\item Two  $I's$ and two $N's$ come together
\item All $A's$ are not coming together
\item No two $A's$ are coming together
\end{enumerate}
%\input{exemplar/11/16/3/14/main.tex}
	\item One urn contains two black balls (labelled B1 and B2) and one white ball. A
	second urn contains one black ball and two white balls (labelled W1 and W2).
	Suppose the following experiment is performed. One of the two urns is chosen
	at random. Next a ball is randomly chosen from the urn. Then a second ball is
	chosen at random from the same urn without replacing the first ball.
	
	\begin{enumerate}
	\item What is the probability that two black balls are chosen?
	
	\item What is the probability that two balls of opposite colour are chosen?
	\end{enumerate}
	\solution
	%\input{exemplar/11/16/3/12/main1.tex}
\end{enumerate}

	\item 
The number lock of a suitcase has 4 wheels each labelled with ten digits i.e. from 0 to 9.The lock opens with a sequence of four digits with no repeats.What is the probability of a person getting the right sequence to open the suitcase.
\\
\solution
		%\begin{enumerate}[label=\thesection.\arabic*,ref=\thesection.\theenumi]
	\item One card is drawn from a well-shuffled deck of 52 cards. Find the probability of getting
\begin{enumerate}
\item A king of red colour 
\item A face card 
\item A red face card
\item The jack of hearts
\item A spade
\item The queen of diamonds

\end{enumerate}
\solution
		%\input{ncert/10/15/1/14/main.tex}
	\item Five cards—the ten, jack, queen, king and ace of diamonds, are well-shuffled with their face downwards. One card is then picked up at random.
\begin{enumerate}
\item
What is the probability that the card is the queen? 
\item
If the queen is drawn and put aside, what is the probability that the second card picked up is (a) an ace? (b) a queen?\\
\end{enumerate}
\solution
		%\input{ncert/10/15/1/15/defs.tex}
	\item A bag contains $5$ red balls and some blue balls. If the probability of drawing a blue ball is double that if a red ball, determine the number of blue balls in the bag. 
		\\
\solution
		%\input{ncert/10/15/2/3/defs.tex}
	\item A card is selected from a pack of 52 cards.
 \begin{enumerate}[label=(\alph*)] 
                 \item How many points are there in the sample space?
                 \item Calculate the probability that the card is an ace of spades.
                 \item Calculate the probability that the card is (i) an ace and (ii) black card.
 \end{enumerate}
\solution
		%\input{ncert/11/16/3/4/main.tex}
\item Four cards are drawn from a well-shuffled deck of 52 cards. What is the probability of obtaining 3 diamonds and one spade.
\\
\solution
		%\input{ncert/11/16/4/2/defs.tex}
\item In a certain lottery 10,000 tickets are sold and ten equal prizes are awarded. What is the probability of not getting a prize if you buy (a) one ticket (b) two tickets (c) 10 tickets ?	
\\
\solution
		%\input{ncert/11/16/4/4/defs.tex}
		%
\item 
Out of 100 students, two sections of 40 and 60 are formed. If you and your friend are among the 100 students, what is the probability that
\begin{enumerate}
\item you both enter the same section?
\item you both enter the different sections?
\end{enumerate}
\solution
		%\input{ncert/11/16/4/5/defs.tex}
	\item 
The number lock of a suitcase has 4 wheels each labelled with ten digits i.e. from 0 to 9.The lock opens with a sequence of four digits with no repeats.What is the probability of a person getting the right sequence to open the suitcase.
\\
\solution
		%\input{ncert/11/16/4/10/defs.tex}
		%
\item 
Two cards are drawn at random and without replacement from a pack of 52 playing cards. Find the probability that both the cards are black.
\\
\solution
		%\input{ncert/12/13/2/2/defs.tex}
		\item A box of oranges is inspected by examining three randomly selected oranges drawn without replacement. If all the three oranges are good, the box is approved for sale, otherwise, it is rejected. Find the probability that a box containing 15 oranges out of which 12 are good and 3 are bad ones will be approved for sale.
		\label{ncert/12/13/2/3/defs.tex}
		\item Two balls are drawn at random with replacement from a box containing 10 black and 8 red balls. Find the probability that
		\label{ncert/12/13/2/12}
\begin{enumerate}
\item both balls are red.
\item first ball is black and second is red.
\item one of them is black and other is red.
\end{enumerate}

\item In a hostel, 60\% of the students read Hindi newspaper, 40\% read English newspaper and 20\% read both Hindi and English newspapers. A student is selected at random.
		\label{ncert/12/13/2/15}
\begin{enumerate}
\item Find the probability that she reads neither Hindi nor English newspapers.
\item If she reads Hindi newspaper, find the probability that she reads English newspaper.
\item If she reads English newspaper, find the probability that she reads Hindi newspaper.\\
\end{enumerate}
\item The probability of obtaining an even prime number on each die, when a pair of dice is rolled is 
\begin{enumerate}
    \item $0$ 
    
    \item $\frac{1}{3}$ 
    
    \item $\frac{1}{12}$ 
    
    \item $\frac{1}{36}$ 
\end{enumerate}
\solution
		%\input{ncert/12/13/2/17/defs.tex}
	\item A bag contains 4 red and 4 black balls, another bag contains 2 red and 6 black balls. One of the two bags is selected at random and a ball is drawn from the bag which is found to be red. Find the probability that the ball is drawn from the first bag.
\\
\solution
		%\input{ncert/12/13/3/2/main.tex}
  \item
  Cards with numbers 2 to 101 are placed in a box. A card is selected at random.Find the probability that the card has
\begin{enumerate}[label=(\roman*)]
	\item an even number 
	\item a square number
\end{enumerate}
\solution
%\input{exemplar/10/13/3/32/main.tex}
\item
The king, queen and jack of clubs are removed from a deck of 52 playing cards and then well shuffled. Now one card is drawn at random from the remaining cards.  Determine the probability that the card is
\begin{enumerate}[label=(\roman*)]
\item a club
\item 10 of hearts
\end{enumerate}
\solution
%\input{exemplar/10/13/3/29/main.tex}
\item A team of medical students doing their internship have to assist during surgeries
at a city hospital. The probabilities of surgeries rated as very complex, complex,
routine, simple or very simple are respectively, 0.15, 0.20, 0.31, 0.26, .08. Find
the probabilities that a particular surgery will be rated
\begin{enumerate}
	\item complex or very complex;
	\item neither very complex nor very simple;
	\item routine or complex
	\item routine or simple
\end{enumerate}
\solution
%\input{exemplar/11/16/3/8(1)/main.tex}
\item A card is selected from a pack of 52 cards.
\begin{enumerate}[label=(\alph*)]
    \item How many points are there in the sample space?
    \item Calculate the probability that the card is an ace of spades.
    \item Calculate the probability that the card is (i) an ace and (ii) black card.
\end{enumerate}
\solution
%\input{exemplar/11/16/3/4/main2.tex}
\item The probability that a non leap year selected at random will contain 53 sundays.
\\
\solution
%\input{exemplar/10/13/1/19/main.tex}
\item One of the four persons John, Rita, Aslam or Gurpreet will be promoted next
month. Consequently the sample space consists of four elementary outcomes
S = {John promoted, Rita promoted, Aslam promoted, Gurpreet promoted}
You are told that the chances of John’s promotion is same as that of Gurpreet,
Rita’s chances of promotion are twice as likely as Johns. Aslam’s chances are
four times that of John.
\begin{enumerate}
	\item Determine
	\begin{enumerate}
		\item P (John promoted)
		\item P (Rita promoted)
		\item P (Aslam promoted)
		\item P (Gurpreet promoted)
	\end{enumerate}
	\item If A = {John promoted or Gurpreet promoted}, find P (A).
\end{enumerate}
\solution
%\input{exemplar/11/16/3/10/main.tex}
\item A card is drawn from a deck of 52 cards. Find the probability of getting a king or a heart or a red card.\\
\solution
%\input{exemplar/11/16/3/15/main.tex}
\item The probability that a student will pass his examination is 0.73, the probability of
the student getting a compartment is 0.13, and the probability that the student will
either pass or get compartment is 0.96. State True or False.\\
\solution
%\input{exemplar/11/16/3/31/main.tex}
\item A card is selected from a pack of 52 cards\\
\begin{enumerate}[label=(\alph*)]
\item How many points are there in the sample space?
\item Calculate the probability that the cards is an ace of spades.
\item Calculate the probability that the card is (i) an ace (ii)black card.\\
\end{enumerate}
%\input{ncert/11/16/3/4_1/Prob_4.tex}
\item In a non-leap year, the probability of having 53 tuesdays or 53 wednesdays is\\
\solution
%\input{exemplar/11/16/3/18/main.tex}
\item There are 1000 sealed envelopes in a box, 10 of them contain a cash prize of
Rs 100 each, 100 of them contain a cash prize of Rs 50 each and 200 of them
contain a cash prize of Rs 10 each and rest do not contain any cash prize. If they
are well shuffled and an envelope is picked up out, what is the probability that it
contains no cash prize?\\
\solution
%\input{exemplar/10/13/3/34/main.tex}
\item 
A die is thrown and a card is selected at random from a deck of 52 playing cards. The probability of getting an even number on the die and a spade card.\\
\solution
%\input{exemplar/12/13/3/78/main.tex}
\item
If 4-digit numbers greater than 5,000 are randomly formed from the digits 0, 1, 3, 5, and 7, what is the probability of forming a number divisible by 5 when:
\begin{enumerate}
    \item The digits are repeated?
    \item The repetition of digits is not allowed?
\end{enumerate}
\solution
%\input{ncert/11/16/4/9/main.tex}
\item Consider the probability space $\brak{\Omega, \mathcal{G}, P}$ where $\Omega = [0,2]$ and $\mathcal{G} = \cbrak{\phi, \Omega, [0,1], (1,2]}$. Let $X$ and $Y$ be two functions on $\Omega$ defined as
\begin{align*}
    X(\omega) = 
    \begin{cases}
        1 & \text{if }\omega \in [0, 1]\\
        2 & \text{if }\omega \in (1, 2]
    \end{cases}
\end{align*}
and
\begin{align*}
    Y(\omega) = 
    \begin{cases}
        2 & \text{if }\omega \in [0, 1.5]\\
        3 & \text{if }\omega \in (1.5, 2].
    \end{cases}
\end{align*}
Then which one of the following statements is true?
\begin{enumerate}
    \item [(A)] $X$ is a random variable with respect to $\mathcal{G}$, but $Y$ is not a random variable with respect to $\mathcal{G}$.
    \item [(B)] $Y$ is a random variable with respect to $\mathcal{G}$, but $X$ is not a random variable with respect to $\mathcal{G}$.
    \item [(C)] Neither $X$ nor $Y$ is a random variable with respect to $\mathcal{G}$.
    \item [(D)] Both $X$ and $Y$ are random variables with respect to $\mathcal{G}$.
\end{enumerate} \hfill (GATE ST 2023)\\
\solution
%\input{gate/ST/2023/14/main.tex}
	\item  A die is loaded in such a way that each odd number is twice as likely to occur as
each even number. Find $P(G)$, where $G$ is the event that a number greater than
3 occurs on a single roll of the die.
\\
\solution
		%\input{exemplar/11/16/3/5/main.tex}
	\item All the jacks, queens and kings are removed from a deck of 52 playing cards. The remaining cards are well shuffled and then one card is drawn at random. Giving ace a value 1 similar value for other cards, find the probability that the card has a value 
		\begin{enumerate}
			\item 7
			\item greater than 7
			\item less than 7
		\end{enumerate}
		%\input{exemplar/10/13/3/30/main.tex}
  \item A Lot consists of 48 mobile phones of which 42 are good, 3 have only minor defects and 3 have major defects.Varnika will buy a phone if it is good but the trader will only buy a mobile if it has no major defects. One phone is selected at random from the lot. What is the probability that it is
\begin{enumerate}
	\item acceptable to Varnika?
            \item acceptable to the trader?
\end{enumerate}
\solution
	%\input{exemplar/10/13/3/40/main.tex}
 \item A student says that if you throw a die, it will show up 1 or not 1. Therefore, the probability of getting 1 and the probability of getting 'not 1' each is equal to $\frac{1}{2}$. Is this correct? Give reasons.\\
 \solution
        %\input{exemplar/10/13/2/9/main.tex}
   \item Four candidates A, B, C, D have ap-
plied for the assignment to coach a school cricket
team. If A is twice as likely to be selected as B, and
B and C are given about the same chance of being
selected, while C is twice as likely to be selected
as D, what are the probabilities that
\begin{enumerate}
\item C will be selected?
\item A will not be selected?
\end{enumerate}
	%\input{exemplar/11/16/3/9/main.tex}
 \item A bag contain 24 balls of which $x$ balls are red, $2x$ are white and $3x$ are blue. A ball is selected at random, What is the probability that it is
\begin{enumerate}[label=\alph*)]
\item not red ?
\item white ?
\end{enumerate}
%\input{exemplar/10/13/3/41/main.tex}
If the letters of the word ASSASSINATION are arranged at random. Find the Probability that
\begin{enumerate}[label=(\alph*)]
\item Four $S's$ come consecutively in the word
\item Two  $I's$ and two $N's$ come together
\item All $A's$ are not coming together
\item No two $A's$ are coming together
\end{enumerate}
%\input{exemplar/11/16/3/14/main.tex}
	\item One urn contains two black balls (labelled B1 and B2) and one white ball. A
	second urn contains one black ball and two white balls (labelled W1 and W2).
	Suppose the following experiment is performed. One of the two urns is chosen
	at random. Next a ball is randomly chosen from the urn. Then a second ball is
	chosen at random from the same urn without replacing the first ball.
	
	\begin{enumerate}
	\item What is the probability that two black balls are chosen?
	
	\item What is the probability that two balls of opposite colour are chosen?
	\end{enumerate}
	\solution
	%\input{exemplar/11/16/3/12/main1.tex}
\end{enumerate}

		%
\item 
Two cards are drawn at random and without replacement from a pack of 52 playing cards. Find the probability that both the cards are black.
\\
\solution
		%\begin{enumerate}[label=\thesection.\arabic*,ref=\thesection.\theenumi]
	\item One card is drawn from a well-shuffled deck of 52 cards. Find the probability of getting
\begin{enumerate}
\item A king of red colour 
\item A face card 
\item A red face card
\item The jack of hearts
\item A spade
\item The queen of diamonds

\end{enumerate}
\solution
		%\input{ncert/10/15/1/14/main.tex}
	\item Five cards—the ten, jack, queen, king and ace of diamonds, are well-shuffled with their face downwards. One card is then picked up at random.
\begin{enumerate}
\item
What is the probability that the card is the queen? 
\item
If the queen is drawn and put aside, what is the probability that the second card picked up is (a) an ace? (b) a queen?\\
\end{enumerate}
\solution
		%\input{ncert/10/15/1/15/defs.tex}
	\item A bag contains $5$ red balls and some blue balls. If the probability of drawing a blue ball is double that if a red ball, determine the number of blue balls in the bag. 
		\\
\solution
		%\input{ncert/10/15/2/3/defs.tex}
	\item A card is selected from a pack of 52 cards.
 \begin{enumerate}[label=(\alph*)] 
                 \item How many points are there in the sample space?
                 \item Calculate the probability that the card is an ace of spades.
                 \item Calculate the probability that the card is (i) an ace and (ii) black card.
 \end{enumerate}
\solution
		%\input{ncert/11/16/3/4/main.tex}
\item Four cards are drawn from a well-shuffled deck of 52 cards. What is the probability of obtaining 3 diamonds and one spade.
\\
\solution
		%\input{ncert/11/16/4/2/defs.tex}
\item In a certain lottery 10,000 tickets are sold and ten equal prizes are awarded. What is the probability of not getting a prize if you buy (a) one ticket (b) two tickets (c) 10 tickets ?	
\\
\solution
		%\input{ncert/11/16/4/4/defs.tex}
		%
\item 
Out of 100 students, two sections of 40 and 60 are formed. If you and your friend are among the 100 students, what is the probability that
\begin{enumerate}
\item you both enter the same section?
\item you both enter the different sections?
\end{enumerate}
\solution
		%\input{ncert/11/16/4/5/defs.tex}
	\item 
The number lock of a suitcase has 4 wheels each labelled with ten digits i.e. from 0 to 9.The lock opens with a sequence of four digits with no repeats.What is the probability of a person getting the right sequence to open the suitcase.
\\
\solution
		%\input{ncert/11/16/4/10/defs.tex}
		%
\item 
Two cards are drawn at random and without replacement from a pack of 52 playing cards. Find the probability that both the cards are black.
\\
\solution
		%\input{ncert/12/13/2/2/defs.tex}
		\item A box of oranges is inspected by examining three randomly selected oranges drawn without replacement. If all the three oranges are good, the box is approved for sale, otherwise, it is rejected. Find the probability that a box containing 15 oranges out of which 12 are good and 3 are bad ones will be approved for sale.
		\label{ncert/12/13/2/3/defs.tex}
		\item Two balls are drawn at random with replacement from a box containing 10 black and 8 red balls. Find the probability that
		\label{ncert/12/13/2/12}
\begin{enumerate}
\item both balls are red.
\item first ball is black and second is red.
\item one of them is black and other is red.
\end{enumerate}

\item In a hostel, 60\% of the students read Hindi newspaper, 40\% read English newspaper and 20\% read both Hindi and English newspapers. A student is selected at random.
		\label{ncert/12/13/2/15}
\begin{enumerate}
\item Find the probability that she reads neither Hindi nor English newspapers.
\item If she reads Hindi newspaper, find the probability that she reads English newspaper.
\item If she reads English newspaper, find the probability that she reads Hindi newspaper.\\
\end{enumerate}
\item The probability of obtaining an even prime number on each die, when a pair of dice is rolled is 
\begin{enumerate}
    \item $0$ 
    
    \item $\frac{1}{3}$ 
    
    \item $\frac{1}{12}$ 
    
    \item $\frac{1}{36}$ 
\end{enumerate}
\solution
		%\input{ncert/12/13/2/17/defs.tex}
	\item A bag contains 4 red and 4 black balls, another bag contains 2 red and 6 black balls. One of the two bags is selected at random and a ball is drawn from the bag which is found to be red. Find the probability that the ball is drawn from the first bag.
\\
\solution
		%\input{ncert/12/13/3/2/main.tex}
  \item
  Cards with numbers 2 to 101 are placed in a box. A card is selected at random.Find the probability that the card has
\begin{enumerate}[label=(\roman*)]
	\item an even number 
	\item a square number
\end{enumerate}
\solution
%\input{exemplar/10/13/3/32/main.tex}
\item
The king, queen and jack of clubs are removed from a deck of 52 playing cards and then well shuffled. Now one card is drawn at random from the remaining cards.  Determine the probability that the card is
\begin{enumerate}[label=(\roman*)]
\item a club
\item 10 of hearts
\end{enumerate}
\solution
%\input{exemplar/10/13/3/29/main.tex}
\item A team of medical students doing their internship have to assist during surgeries
at a city hospital. The probabilities of surgeries rated as very complex, complex,
routine, simple or very simple are respectively, 0.15, 0.20, 0.31, 0.26, .08. Find
the probabilities that a particular surgery will be rated
\begin{enumerate}
	\item complex or very complex;
	\item neither very complex nor very simple;
	\item routine or complex
	\item routine or simple
\end{enumerate}
\solution
%\input{exemplar/11/16/3/8(1)/main.tex}
\item A card is selected from a pack of 52 cards.
\begin{enumerate}[label=(\alph*)]
    \item How many points are there in the sample space?
    \item Calculate the probability that the card is an ace of spades.
    \item Calculate the probability that the card is (i) an ace and (ii) black card.
\end{enumerate}
\solution
%\input{exemplar/11/16/3/4/main2.tex}
\item The probability that a non leap year selected at random will contain 53 sundays.
\\
\solution
%\input{exemplar/10/13/1/19/main.tex}
\item One of the four persons John, Rita, Aslam or Gurpreet will be promoted next
month. Consequently the sample space consists of four elementary outcomes
S = {John promoted, Rita promoted, Aslam promoted, Gurpreet promoted}
You are told that the chances of John’s promotion is same as that of Gurpreet,
Rita’s chances of promotion are twice as likely as Johns. Aslam’s chances are
four times that of John.
\begin{enumerate}
	\item Determine
	\begin{enumerate}
		\item P (John promoted)
		\item P (Rita promoted)
		\item P (Aslam promoted)
		\item P (Gurpreet promoted)
	\end{enumerate}
	\item If A = {John promoted or Gurpreet promoted}, find P (A).
\end{enumerate}
\solution
%\input{exemplar/11/16/3/10/main.tex}
\item A card is drawn from a deck of 52 cards. Find the probability of getting a king or a heart or a red card.\\
\solution
%\input{exemplar/11/16/3/15/main.tex}
\item The probability that a student will pass his examination is 0.73, the probability of
the student getting a compartment is 0.13, and the probability that the student will
either pass or get compartment is 0.96. State True or False.\\
\solution
%\input{exemplar/11/16/3/31/main.tex}
\item A card is selected from a pack of 52 cards\\
\begin{enumerate}[label=(\alph*)]
\item How many points are there in the sample space?
\item Calculate the probability that the cards is an ace of spades.
\item Calculate the probability that the card is (i) an ace (ii)black card.\\
\end{enumerate}
%\input{ncert/11/16/3/4_1/Prob_4.tex}
\item In a non-leap year, the probability of having 53 tuesdays or 53 wednesdays is\\
\solution
%\input{exemplar/11/16/3/18/main.tex}
\item There are 1000 sealed envelopes in a box, 10 of them contain a cash prize of
Rs 100 each, 100 of them contain a cash prize of Rs 50 each and 200 of them
contain a cash prize of Rs 10 each and rest do not contain any cash prize. If they
are well shuffled and an envelope is picked up out, what is the probability that it
contains no cash prize?\\
\solution
%\input{exemplar/10/13/3/34/main.tex}
\item 
A die is thrown and a card is selected at random from a deck of 52 playing cards. The probability of getting an even number on the die and a spade card.\\
\solution
%\input{exemplar/12/13/3/78/main.tex}
\item
If 4-digit numbers greater than 5,000 are randomly formed from the digits 0, 1, 3, 5, and 7, what is the probability of forming a number divisible by 5 when:
\begin{enumerate}
    \item The digits are repeated?
    \item The repetition of digits is not allowed?
\end{enumerate}
\solution
%\input{ncert/11/16/4/9/main.tex}
\item Consider the probability space $\brak{\Omega, \mathcal{G}, P}$ where $\Omega = [0,2]$ and $\mathcal{G} = \cbrak{\phi, \Omega, [0,1], (1,2]}$. Let $X$ and $Y$ be two functions on $\Omega$ defined as
\begin{align*}
    X(\omega) = 
    \begin{cases}
        1 & \text{if }\omega \in [0, 1]\\
        2 & \text{if }\omega \in (1, 2]
    \end{cases}
\end{align*}
and
\begin{align*}
    Y(\omega) = 
    \begin{cases}
        2 & \text{if }\omega \in [0, 1.5]\\
        3 & \text{if }\omega \in (1.5, 2].
    \end{cases}
\end{align*}
Then which one of the following statements is true?
\begin{enumerate}
    \item [(A)] $X$ is a random variable with respect to $\mathcal{G}$, but $Y$ is not a random variable with respect to $\mathcal{G}$.
    \item [(B)] $Y$ is a random variable with respect to $\mathcal{G}$, but $X$ is not a random variable with respect to $\mathcal{G}$.
    \item [(C)] Neither $X$ nor $Y$ is a random variable with respect to $\mathcal{G}$.
    \item [(D)] Both $X$ and $Y$ are random variables with respect to $\mathcal{G}$.
\end{enumerate} \hfill (GATE ST 2023)\\
\solution
%\input{gate/ST/2023/14/main.tex}
	\item  A die is loaded in such a way that each odd number is twice as likely to occur as
each even number. Find $P(G)$, where $G$ is the event that a number greater than
3 occurs on a single roll of the die.
\\
\solution
		%\input{exemplar/11/16/3/5/main.tex}
	\item All the jacks, queens and kings are removed from a deck of 52 playing cards. The remaining cards are well shuffled and then one card is drawn at random. Giving ace a value 1 similar value for other cards, find the probability that the card has a value 
		\begin{enumerate}
			\item 7
			\item greater than 7
			\item less than 7
		\end{enumerate}
		%\input{exemplar/10/13/3/30/main.tex}
  \item A Lot consists of 48 mobile phones of which 42 are good, 3 have only minor defects and 3 have major defects.Varnika will buy a phone if it is good but the trader will only buy a mobile if it has no major defects. One phone is selected at random from the lot. What is the probability that it is
\begin{enumerate}
	\item acceptable to Varnika?
            \item acceptable to the trader?
\end{enumerate}
\solution
	%\input{exemplar/10/13/3/40/main.tex}
 \item A student says that if you throw a die, it will show up 1 or not 1. Therefore, the probability of getting 1 and the probability of getting 'not 1' each is equal to $\frac{1}{2}$. Is this correct? Give reasons.\\
 \solution
        %\input{exemplar/10/13/2/9/main.tex}
   \item Four candidates A, B, C, D have ap-
plied for the assignment to coach a school cricket
team. If A is twice as likely to be selected as B, and
B and C are given about the same chance of being
selected, while C is twice as likely to be selected
as D, what are the probabilities that
\begin{enumerate}
\item C will be selected?
\item A will not be selected?
\end{enumerate}
	%\input{exemplar/11/16/3/9/main.tex}
 \item A bag contain 24 balls of which $x$ balls are red, $2x$ are white and $3x$ are blue. A ball is selected at random, What is the probability that it is
\begin{enumerate}[label=\alph*)]
\item not red ?
\item white ?
\end{enumerate}
%\input{exemplar/10/13/3/41/main.tex}
If the letters of the word ASSASSINATION are arranged at random. Find the Probability that
\begin{enumerate}[label=(\alph*)]
\item Four $S's$ come consecutively in the word
\item Two  $I's$ and two $N's$ come together
\item All $A's$ are not coming together
\item No two $A's$ are coming together
\end{enumerate}
%\input{exemplar/11/16/3/14/main.tex}
	\item One urn contains two black balls (labelled B1 and B2) and one white ball. A
	second urn contains one black ball and two white balls (labelled W1 and W2).
	Suppose the following experiment is performed. One of the two urns is chosen
	at random. Next a ball is randomly chosen from the urn. Then a second ball is
	chosen at random from the same urn without replacing the first ball.
	
	\begin{enumerate}
	\item What is the probability that two black balls are chosen?
	
	\item What is the probability that two balls of opposite colour are chosen?
	\end{enumerate}
	\solution
	%\input{exemplar/11/16/3/12/main1.tex}
\end{enumerate}

		\item A box of oranges is inspected by examining three randomly selected oranges drawn without replacement. If all the three oranges are good, the box is approved for sale, otherwise, it is rejected. Find the probability that a box containing 15 oranges out of which 12 are good and 3 are bad ones will be approved for sale.
		\label{ncert/12/13/2/3/defs.tex}
		\item Two balls are drawn at random with replacement from a box containing 10 black and 8 red balls. Find the probability that
		\label{ncert/12/13/2/12}
\begin{enumerate}
\item both balls are red.
\item first ball is black and second is red.
\item one of them is black and other is red.
\end{enumerate}

\item In a hostel, 60\% of the students read Hindi newspaper, 40\% read English newspaper and 20\% read both Hindi and English newspapers. A student is selected at random.
		\label{ncert/12/13/2/15}
\begin{enumerate}
\item Find the probability that she reads neither Hindi nor English newspapers.
\item If she reads Hindi newspaper, find the probability that she reads English newspaper.
\item If she reads English newspaper, find the probability that she reads Hindi newspaper.\\
\end{enumerate}
\item The probability of obtaining an even prime number on each die, when a pair of dice is rolled is 
\begin{enumerate}
    \item $0$ 
    
    \item $\frac{1}{3}$ 
    
    \item $\frac{1}{12}$ 
    
    \item $\frac{1}{36}$ 
\end{enumerate}
\solution
		%\begin{enumerate}[label=\thesection.\arabic*,ref=\thesection.\theenumi]
	\item One card is drawn from a well-shuffled deck of 52 cards. Find the probability of getting
\begin{enumerate}
\item A king of red colour 
\item A face card 
\item A red face card
\item The jack of hearts
\item A spade
\item The queen of diamonds

\end{enumerate}
\solution
		%\input{ncert/10/15/1/14/main.tex}
	\item Five cards—the ten, jack, queen, king and ace of diamonds, are well-shuffled with their face downwards. One card is then picked up at random.
\begin{enumerate}
\item
What is the probability that the card is the queen? 
\item
If the queen is drawn and put aside, what is the probability that the second card picked up is (a) an ace? (b) a queen?\\
\end{enumerate}
\solution
		%\input{ncert/10/15/1/15/defs.tex}
	\item A bag contains $5$ red balls and some blue balls. If the probability of drawing a blue ball is double that if a red ball, determine the number of blue balls in the bag. 
		\\
\solution
		%\input{ncert/10/15/2/3/defs.tex}
	\item A card is selected from a pack of 52 cards.
 \begin{enumerate}[label=(\alph*)] 
                 \item How many points are there in the sample space?
                 \item Calculate the probability that the card is an ace of spades.
                 \item Calculate the probability that the card is (i) an ace and (ii) black card.
 \end{enumerate}
\solution
		%\input{ncert/11/16/3/4/main.tex}
\item Four cards are drawn from a well-shuffled deck of 52 cards. What is the probability of obtaining 3 diamonds and one spade.
\\
\solution
		%\input{ncert/11/16/4/2/defs.tex}
\item In a certain lottery 10,000 tickets are sold and ten equal prizes are awarded. What is the probability of not getting a prize if you buy (a) one ticket (b) two tickets (c) 10 tickets ?	
\\
\solution
		%\input{ncert/11/16/4/4/defs.tex}
		%
\item 
Out of 100 students, two sections of 40 and 60 are formed. If you and your friend are among the 100 students, what is the probability that
\begin{enumerate}
\item you both enter the same section?
\item you both enter the different sections?
\end{enumerate}
\solution
		%\input{ncert/11/16/4/5/defs.tex}
	\item 
The number lock of a suitcase has 4 wheels each labelled with ten digits i.e. from 0 to 9.The lock opens with a sequence of four digits with no repeats.What is the probability of a person getting the right sequence to open the suitcase.
\\
\solution
		%\input{ncert/11/16/4/10/defs.tex}
		%
\item 
Two cards are drawn at random and without replacement from a pack of 52 playing cards. Find the probability that both the cards are black.
\\
\solution
		%\input{ncert/12/13/2/2/defs.tex}
		\item A box of oranges is inspected by examining three randomly selected oranges drawn without replacement. If all the three oranges are good, the box is approved for sale, otherwise, it is rejected. Find the probability that a box containing 15 oranges out of which 12 are good and 3 are bad ones will be approved for sale.
		\label{ncert/12/13/2/3/defs.tex}
		\item Two balls are drawn at random with replacement from a box containing 10 black and 8 red balls. Find the probability that
		\label{ncert/12/13/2/12}
\begin{enumerate}
\item both balls are red.
\item first ball is black and second is red.
\item one of them is black and other is red.
\end{enumerate}

\item In a hostel, 60\% of the students read Hindi newspaper, 40\% read English newspaper and 20\% read both Hindi and English newspapers. A student is selected at random.
		\label{ncert/12/13/2/15}
\begin{enumerate}
\item Find the probability that she reads neither Hindi nor English newspapers.
\item If she reads Hindi newspaper, find the probability that she reads English newspaper.
\item If she reads English newspaper, find the probability that she reads Hindi newspaper.\\
\end{enumerate}
\item The probability of obtaining an even prime number on each die, when a pair of dice is rolled is 
\begin{enumerate}
    \item $0$ 
    
    \item $\frac{1}{3}$ 
    
    \item $\frac{1}{12}$ 
    
    \item $\frac{1}{36}$ 
\end{enumerate}
\solution
		%\input{ncert/12/13/2/17/defs.tex}
	\item A bag contains 4 red and 4 black balls, another bag contains 2 red and 6 black balls. One of the two bags is selected at random and a ball is drawn from the bag which is found to be red. Find the probability that the ball is drawn from the first bag.
\\
\solution
		%\input{ncert/12/13/3/2/main.tex}
  \item
  Cards with numbers 2 to 101 are placed in a box. A card is selected at random.Find the probability that the card has
\begin{enumerate}[label=(\roman*)]
	\item an even number 
	\item a square number
\end{enumerate}
\solution
%\input{exemplar/10/13/3/32/main.tex}
\item
The king, queen and jack of clubs are removed from a deck of 52 playing cards and then well shuffled. Now one card is drawn at random from the remaining cards.  Determine the probability that the card is
\begin{enumerate}[label=(\roman*)]
\item a club
\item 10 of hearts
\end{enumerate}
\solution
%\input{exemplar/10/13/3/29/main.tex}
\item A team of medical students doing their internship have to assist during surgeries
at a city hospital. The probabilities of surgeries rated as very complex, complex,
routine, simple or very simple are respectively, 0.15, 0.20, 0.31, 0.26, .08. Find
the probabilities that a particular surgery will be rated
\begin{enumerate}
	\item complex or very complex;
	\item neither very complex nor very simple;
	\item routine or complex
	\item routine or simple
\end{enumerate}
\solution
%\input{exemplar/11/16/3/8(1)/main.tex}
\item A card is selected from a pack of 52 cards.
\begin{enumerate}[label=(\alph*)]
    \item How many points are there in the sample space?
    \item Calculate the probability that the card is an ace of spades.
    \item Calculate the probability that the card is (i) an ace and (ii) black card.
\end{enumerate}
\solution
%\input{exemplar/11/16/3/4/main2.tex}
\item The probability that a non leap year selected at random will contain 53 sundays.
\\
\solution
%\input{exemplar/10/13/1/19/main.tex}
\item One of the four persons John, Rita, Aslam or Gurpreet will be promoted next
month. Consequently the sample space consists of four elementary outcomes
S = {John promoted, Rita promoted, Aslam promoted, Gurpreet promoted}
You are told that the chances of John’s promotion is same as that of Gurpreet,
Rita’s chances of promotion are twice as likely as Johns. Aslam’s chances are
four times that of John.
\begin{enumerate}
	\item Determine
	\begin{enumerate}
		\item P (John promoted)
		\item P (Rita promoted)
		\item P (Aslam promoted)
		\item P (Gurpreet promoted)
	\end{enumerate}
	\item If A = {John promoted or Gurpreet promoted}, find P (A).
\end{enumerate}
\solution
%\input{exemplar/11/16/3/10/main.tex}
\item A card is drawn from a deck of 52 cards. Find the probability of getting a king or a heart or a red card.\\
\solution
%\input{exemplar/11/16/3/15/main.tex}
\item The probability that a student will pass his examination is 0.73, the probability of
the student getting a compartment is 0.13, and the probability that the student will
either pass or get compartment is 0.96. State True or False.\\
\solution
%\input{exemplar/11/16/3/31/main.tex}
\item A card is selected from a pack of 52 cards\\
\begin{enumerate}[label=(\alph*)]
\item How many points are there in the sample space?
\item Calculate the probability that the cards is an ace of spades.
\item Calculate the probability that the card is (i) an ace (ii)black card.\\
\end{enumerate}
%\input{ncert/11/16/3/4_1/Prob_4.tex}
\item In a non-leap year, the probability of having 53 tuesdays or 53 wednesdays is\\
\solution
%\input{exemplar/11/16/3/18/main.tex}
\item There are 1000 sealed envelopes in a box, 10 of them contain a cash prize of
Rs 100 each, 100 of them contain a cash prize of Rs 50 each and 200 of them
contain a cash prize of Rs 10 each and rest do not contain any cash prize. If they
are well shuffled and an envelope is picked up out, what is the probability that it
contains no cash prize?\\
\solution
%\input{exemplar/10/13/3/34/main.tex}
\item 
A die is thrown and a card is selected at random from a deck of 52 playing cards. The probability of getting an even number on the die and a spade card.\\
\solution
%\input{exemplar/12/13/3/78/main.tex}
\item
If 4-digit numbers greater than 5,000 are randomly formed from the digits 0, 1, 3, 5, and 7, what is the probability of forming a number divisible by 5 when:
\begin{enumerate}
    \item The digits are repeated?
    \item The repetition of digits is not allowed?
\end{enumerate}
\solution
%\input{ncert/11/16/4/9/main.tex}
\item Consider the probability space $\brak{\Omega, \mathcal{G}, P}$ where $\Omega = [0,2]$ and $\mathcal{G} = \cbrak{\phi, \Omega, [0,1], (1,2]}$. Let $X$ and $Y$ be two functions on $\Omega$ defined as
\begin{align*}
    X(\omega) = 
    \begin{cases}
        1 & \text{if }\omega \in [0, 1]\\
        2 & \text{if }\omega \in (1, 2]
    \end{cases}
\end{align*}
and
\begin{align*}
    Y(\omega) = 
    \begin{cases}
        2 & \text{if }\omega \in [0, 1.5]\\
        3 & \text{if }\omega \in (1.5, 2].
    \end{cases}
\end{align*}
Then which one of the following statements is true?
\begin{enumerate}
    \item [(A)] $X$ is a random variable with respect to $\mathcal{G}$, but $Y$ is not a random variable with respect to $\mathcal{G}$.
    \item [(B)] $Y$ is a random variable with respect to $\mathcal{G}$, but $X$ is not a random variable with respect to $\mathcal{G}$.
    \item [(C)] Neither $X$ nor $Y$ is a random variable with respect to $\mathcal{G}$.
    \item [(D)] Both $X$ and $Y$ are random variables with respect to $\mathcal{G}$.
\end{enumerate} \hfill (GATE ST 2023)\\
\solution
%\input{gate/ST/2023/14/main.tex}
	\item  A die is loaded in such a way that each odd number is twice as likely to occur as
each even number. Find $P(G)$, where $G$ is the event that a number greater than
3 occurs on a single roll of the die.
\\
\solution
		%\input{exemplar/11/16/3/5/main.tex}
	\item All the jacks, queens and kings are removed from a deck of 52 playing cards. The remaining cards are well shuffled and then one card is drawn at random. Giving ace a value 1 similar value for other cards, find the probability that the card has a value 
		\begin{enumerate}
			\item 7
			\item greater than 7
			\item less than 7
		\end{enumerate}
		%\input{exemplar/10/13/3/30/main.tex}
  \item A Lot consists of 48 mobile phones of which 42 are good, 3 have only minor defects and 3 have major defects.Varnika will buy a phone if it is good but the trader will only buy a mobile if it has no major defects. One phone is selected at random from the lot. What is the probability that it is
\begin{enumerate}
	\item acceptable to Varnika?
            \item acceptable to the trader?
\end{enumerate}
\solution
	%\input{exemplar/10/13/3/40/main.tex}
 \item A student says that if you throw a die, it will show up 1 or not 1. Therefore, the probability of getting 1 and the probability of getting 'not 1' each is equal to $\frac{1}{2}$. Is this correct? Give reasons.\\
 \solution
        %\input{exemplar/10/13/2/9/main.tex}
   \item Four candidates A, B, C, D have ap-
plied for the assignment to coach a school cricket
team. If A is twice as likely to be selected as B, and
B and C are given about the same chance of being
selected, while C is twice as likely to be selected
as D, what are the probabilities that
\begin{enumerate}
\item C will be selected?
\item A will not be selected?
\end{enumerate}
	%\input{exemplar/11/16/3/9/main.tex}
 \item A bag contain 24 balls of which $x$ balls are red, $2x$ are white and $3x$ are blue. A ball is selected at random, What is the probability that it is
\begin{enumerate}[label=\alph*)]
\item not red ?
\item white ?
\end{enumerate}
%\input{exemplar/10/13/3/41/main.tex}
If the letters of the word ASSASSINATION are arranged at random. Find the Probability that
\begin{enumerate}[label=(\alph*)]
\item Four $S's$ come consecutively in the word
\item Two  $I's$ and two $N's$ come together
\item All $A's$ are not coming together
\item No two $A's$ are coming together
\end{enumerate}
%\input{exemplar/11/16/3/14/main.tex}
	\item One urn contains two black balls (labelled B1 and B2) and one white ball. A
	second urn contains one black ball and two white balls (labelled W1 and W2).
	Suppose the following experiment is performed. One of the two urns is chosen
	at random. Next a ball is randomly chosen from the urn. Then a second ball is
	chosen at random from the same urn without replacing the first ball.
	
	\begin{enumerate}
	\item What is the probability that two black balls are chosen?
	
	\item What is the probability that two balls of opposite colour are chosen?
	\end{enumerate}
	\solution
	%\input{exemplar/11/16/3/12/main1.tex}
\end{enumerate}

	\item A bag contains 4 red and 4 black balls, another bag contains 2 red and 6 black balls. One of the two bags is selected at random and a ball is drawn from the bag which is found to be red. Find the probability that the ball is drawn from the first bag.
\\
\solution
		%\begin{table}[H]
	\centering
\begin{tabular}{|c|c|c|}
\hline
Random variable &Value &Definition\\ \hline
\multirow{3}{*}{X} &0 &Slips of Rs 1\\
&1 &Slips of Rs 5\\
&2 &Slips of Rs 13\\ \hline
\multirow{2}{*}{Y} &0 &Box A\\
&1 &Box B\\\hline
\end{tabular}
\caption{}
\label{tab:Distribution}
\end{table}
See \tabref{tab:Distribution}.
\begin{align}
p_{Y}\brak{k}= \begin{cases} 
      \frac{1}{3} & {k=0} \\
      \frac{2}{3 }& {k=1} 
   \end{cases}
   \\
p_{Y|X}\brak{0|0} = \frac{19}{25}\, 
p_{Y|X}\brak{0|1} = \frac{6}{25}\,
p_{Y|X}\brak{1|0} = \frac{45}{50}\,
p_{Y|X}\brak{1|2} = \frac{5}{50}
\end{align}
The desired probability is the probability that a slip drawn at random is marked other than Rs 1,
\begin{align}
&=1-p_X\brak{0}\\
&= p_X(1) + p_X(2)
\end{align}
Using Bayes theorem,
\begin{align}
&= p_Y\brak{0} \times \pr{Y=0 | X=1} + p_Y\brak{1} \times \pr{Y=1|X=2}\\
&=\frac{1}{3} \times \frac{6}{25} + \frac{2}{3} \times \frac{5}{50}\\
&=\frac{11}{75}
\end{align}

\newpage

%\tableofcontents

\bigskip

\renewcommand{\thefigure}{\theenumi}
\renewcommand{\thetable}{\theenumi}
%\renewcommand{\theequation}{\theenumi}

%\begin{abstract}
%%\boldmath
%In this letter, an algorithm for evaluating the exact analytical bit error rate  (BER)  for the piecewise linear (PL) combiner for  multiple relays is presented. Previous results were available only for upto three relays. The algorithm is unique in the sense that  the actual mathematical expressions, that are prohibitively large, need not be explicitly obtained. The diversity gain due to multiple relays is shown through plots of the analytical BER, well supported by simulations. 
%
%\end{abstract}
% IEEEtran.cls defaults to using nonbold math in the Abstract.
% This preserves the distinction between vectors and scalars. However,
% if the journal you are submitting to favors bold math in the abstract,
% then you can use LaTeX's standard command \boldmath at the very start
% of the abstract to achieve this. Many IEEE journals frown on math
% in the abstract anyway.

% Note that keywords are not normally used for peerreview papers.
%\begin{IEEEkeywords}
%Cooperative diversity, decode and forward, piecewise linear
%\end{IEEEkeywords}



% For peer review papers, you can put extra information on the cover
% page as needed:
% \ifCLASSOPTIONpeerreview
% \begin{center} \bfseries EDICS Category: 3-BBND \end{center}
% \fi
%
% For peerreview papers, this IEEEtran command inserts a page break and
% creates the second title. It will be ignored for other modes.
%\IEEEpeerreviewmaketitle




  \item
  Cards with numbers 2 to 101 are placed in a box. A card is selected at random.Find the probability that the card has
\begin{enumerate}[label=(\roman*)]
	\item an even number 
	\item a square number
\end{enumerate}
\solution
%\begin{table}[H]
	\centering
\begin{tabular}{|c|c|c|}
\hline
Random variable &Value &Definition\\ \hline
\multirow{3}{*}{X} &0 &Slips of Rs 1\\
&1 &Slips of Rs 5\\
&2 &Slips of Rs 13\\ \hline
\multirow{2}{*}{Y} &0 &Box A\\
&1 &Box B\\\hline
\end{tabular}
\caption{}
\label{tab:Distribution}
\end{table}
See \tabref{tab:Distribution}.
\begin{align}
p_{Y}\brak{k}= \begin{cases} 
      \frac{1}{3} & {k=0} \\
      \frac{2}{3 }& {k=1} 
   \end{cases}
   \\
p_{Y|X}\brak{0|0} = \frac{19}{25}\, 
p_{Y|X}\brak{0|1} = \frac{6}{25}\,
p_{Y|X}\brak{1|0} = \frac{45}{50}\,
p_{Y|X}\brak{1|2} = \frac{5}{50}
\end{align}
The desired probability is the probability that a slip drawn at random is marked other than Rs 1,
\begin{align}
&=1-p_X\brak{0}\\
&= p_X(1) + p_X(2)
\end{align}
Using Bayes theorem,
\begin{align}
&= p_Y\brak{0} \times \pr{Y=0 | X=1} + p_Y\brak{1} \times \pr{Y=1|X=2}\\
&=\frac{1}{3} \times \frac{6}{25} + \frac{2}{3} \times \frac{5}{50}\\
&=\frac{11}{75}
\end{align}

\newpage

%\tableofcontents

\bigskip

\renewcommand{\thefigure}{\theenumi}
\renewcommand{\thetable}{\theenumi}
%\renewcommand{\theequation}{\theenumi}

%\begin{abstract}
%%\boldmath
%In this letter, an algorithm for evaluating the exact analytical bit error rate  (BER)  for the piecewise linear (PL) combiner for  multiple relays is presented. Previous results were available only for upto three relays. The algorithm is unique in the sense that  the actual mathematical expressions, that are prohibitively large, need not be explicitly obtained. The diversity gain due to multiple relays is shown through plots of the analytical BER, well supported by simulations. 
%
%\end{abstract}
% IEEEtran.cls defaults to using nonbold math in the Abstract.
% This preserves the distinction between vectors and scalars. However,
% if the journal you are submitting to favors bold math in the abstract,
% then you can use LaTeX's standard command \boldmath at the very start
% of the abstract to achieve this. Many IEEE journals frown on math
% in the abstract anyway.

% Note that keywords are not normally used for peerreview papers.
%\begin{IEEEkeywords}
%Cooperative diversity, decode and forward, piecewise linear
%\end{IEEEkeywords}



% For peer review papers, you can put extra information on the cover
% page as needed:
% \ifCLASSOPTIONpeerreview
% \begin{center} \bfseries EDICS Category: 3-BBND \end{center}
% \fi
%
% For peerreview papers, this IEEEtran command inserts a page break and
% creates the second title. It will be ignored for other modes.
%\IEEEpeerreviewmaketitle




\item
The king, queen and jack of clubs are removed from a deck of 52 playing cards and then well shuffled. Now one card is drawn at random from the remaining cards.  Determine the probability that the card is
\begin{enumerate}[label=(\roman*)]
\item a club
\item 10 of hearts
\end{enumerate}
\solution
%\begin{table}[H]
	\centering
\begin{tabular}{|c|c|c|}
\hline
Random variable &Value &Definition\\ \hline
\multirow{3}{*}{X} &0 &Slips of Rs 1\\
&1 &Slips of Rs 5\\
&2 &Slips of Rs 13\\ \hline
\multirow{2}{*}{Y} &0 &Box A\\
&1 &Box B\\\hline
\end{tabular}
\caption{}
\label{tab:Distribution}
\end{table}
See \tabref{tab:Distribution}.
\begin{align}
p_{Y}\brak{k}= \begin{cases} 
      \frac{1}{3} & {k=0} \\
      \frac{2}{3 }& {k=1} 
   \end{cases}
   \\
p_{Y|X}\brak{0|0} = \frac{19}{25}\, 
p_{Y|X}\brak{0|1} = \frac{6}{25}\,
p_{Y|X}\brak{1|0} = \frac{45}{50}\,
p_{Y|X}\brak{1|2} = \frac{5}{50}
\end{align}
The desired probability is the probability that a slip drawn at random is marked other than Rs 1,
\begin{align}
&=1-p_X\brak{0}\\
&= p_X(1) + p_X(2)
\end{align}
Using Bayes theorem,
\begin{align}
&= p_Y\brak{0} \times \pr{Y=0 | X=1} + p_Y\brak{1} \times \pr{Y=1|X=2}\\
&=\frac{1}{3} \times \frac{6}{25} + \frac{2}{3} \times \frac{5}{50}\\
&=\frac{11}{75}
\end{align}

\newpage

%\tableofcontents

\bigskip

\renewcommand{\thefigure}{\theenumi}
\renewcommand{\thetable}{\theenumi}
%\renewcommand{\theequation}{\theenumi}

%\begin{abstract}
%%\boldmath
%In this letter, an algorithm for evaluating the exact analytical bit error rate  (BER)  for the piecewise linear (PL) combiner for  multiple relays is presented. Previous results were available only for upto three relays. The algorithm is unique in the sense that  the actual mathematical expressions, that are prohibitively large, need not be explicitly obtained. The diversity gain due to multiple relays is shown through plots of the analytical BER, well supported by simulations. 
%
%\end{abstract}
% IEEEtran.cls defaults to using nonbold math in the Abstract.
% This preserves the distinction between vectors and scalars. However,
% if the journal you are submitting to favors bold math in the abstract,
% then you can use LaTeX's standard command \boldmath at the very start
% of the abstract to achieve this. Many IEEE journals frown on math
% in the abstract anyway.

% Note that keywords are not normally used for peerreview papers.
%\begin{IEEEkeywords}
%Cooperative diversity, decode and forward, piecewise linear
%\end{IEEEkeywords}



% For peer review papers, you can put extra information on the cover
% page as needed:
% \ifCLASSOPTIONpeerreview
% \begin{center} \bfseries EDICS Category: 3-BBND \end{center}
% \fi
%
% For peerreview papers, this IEEEtran command inserts a page break and
% creates the second title. It will be ignored for other modes.
%\IEEEpeerreviewmaketitle




\item A team of medical students doing their internship have to assist during surgeries
at a city hospital. The probabilities of surgeries rated as very complex, complex,
routine, simple or very simple are respectively, 0.15, 0.20, 0.31, 0.26, .08. Find
the probabilities that a particular surgery will be rated
\begin{enumerate}
	\item complex or very complex;
	\item neither very complex nor very simple;
	\item routine or complex
	\item routine or simple
\end{enumerate}
\solution
%\begin{table}[H]
	\centering
\begin{tabular}{|c|c|c|}
\hline
Random variable &Value &Definition\\ \hline
\multirow{3}{*}{X} &0 &Slips of Rs 1\\
&1 &Slips of Rs 5\\
&2 &Slips of Rs 13\\ \hline
\multirow{2}{*}{Y} &0 &Box A\\
&1 &Box B\\\hline
\end{tabular}
\caption{}
\label{tab:Distribution}
\end{table}
See \tabref{tab:Distribution}.
\begin{align}
p_{Y}\brak{k}= \begin{cases} 
      \frac{1}{3} & {k=0} \\
      \frac{2}{3 }& {k=1} 
   \end{cases}
   \\
p_{Y|X}\brak{0|0} = \frac{19}{25}\, 
p_{Y|X}\brak{0|1} = \frac{6}{25}\,
p_{Y|X}\brak{1|0} = \frac{45}{50}\,
p_{Y|X}\brak{1|2} = \frac{5}{50}
\end{align}
The desired probability is the probability that a slip drawn at random is marked other than Rs 1,
\begin{align}
&=1-p_X\brak{0}\\
&= p_X(1) + p_X(2)
\end{align}
Using Bayes theorem,
\begin{align}
&= p_Y\brak{0} \times \pr{Y=0 | X=1} + p_Y\brak{1} \times \pr{Y=1|X=2}\\
&=\frac{1}{3} \times \frac{6}{25} + \frac{2}{3} \times \frac{5}{50}\\
&=\frac{11}{75}
\end{align}

\newpage

%\tableofcontents

\bigskip

\renewcommand{\thefigure}{\theenumi}
\renewcommand{\thetable}{\theenumi}
%\renewcommand{\theequation}{\theenumi}

%\begin{abstract}
%%\boldmath
%In this letter, an algorithm for evaluating the exact analytical bit error rate  (BER)  for the piecewise linear (PL) combiner for  multiple relays is presented. Previous results were available only for upto three relays. The algorithm is unique in the sense that  the actual mathematical expressions, that are prohibitively large, need not be explicitly obtained. The diversity gain due to multiple relays is shown through plots of the analytical BER, well supported by simulations. 
%
%\end{abstract}
% IEEEtran.cls defaults to using nonbold math in the Abstract.
% This preserves the distinction between vectors and scalars. However,
% if the journal you are submitting to favors bold math in the abstract,
% then you can use LaTeX's standard command \boldmath at the very start
% of the abstract to achieve this. Many IEEE journals frown on math
% in the abstract anyway.

% Note that keywords are not normally used for peerreview papers.
%\begin{IEEEkeywords}
%Cooperative diversity, decode and forward, piecewise linear
%\end{IEEEkeywords}



% For peer review papers, you can put extra information on the cover
% page as needed:
% \ifCLASSOPTIONpeerreview
% \begin{center} \bfseries EDICS Category: 3-BBND \end{center}
% \fi
%
% For peerreview papers, this IEEEtran command inserts a page break and
% creates the second title. It will be ignored for other modes.
%\IEEEpeerreviewmaketitle




\item A card is selected from a pack of 52 cards.
\begin{enumerate}[label=(\alph*)]
    \item How many points are there in the sample space?
    \item Calculate the probability that the card is an ace of spades.
    \item Calculate the probability that the card is (i) an ace and (ii) black card.
\end{enumerate}
\solution
%Let $X$ be an bernoulli rv defined as in \tabref{tab:exemplar/11/16/3/26}.  Then, 
\begin{equation}
    p =
        \frac{4}{11} 
\end{equation}
\begin{table}[H]
	\centering
	\input{exemplar/11/16/3/26/tables/Table2.tex}
	\caption{}
        \label{tab:exemplar/11/16/3/26}
\end{table}

\item The probability that a non leap year selected at random will contain 53 sundays.
\\
\solution
%\begin{table}[H]
	\centering
\begin{tabular}{|c|c|c|}
\hline
Random variable &Value &Definition\\ \hline
\multirow{3}{*}{X} &0 &Slips of Rs 1\\
&1 &Slips of Rs 5\\
&2 &Slips of Rs 13\\ \hline
\multirow{2}{*}{Y} &0 &Box A\\
&1 &Box B\\\hline
\end{tabular}
\caption{}
\label{tab:Distribution}
\end{table}
See \tabref{tab:Distribution}.
\begin{align}
p_{Y}\brak{k}= \begin{cases} 
      \frac{1}{3} & {k=0} \\
      \frac{2}{3 }& {k=1} 
   \end{cases}
   \\
p_{Y|X}\brak{0|0} = \frac{19}{25}\, 
p_{Y|X}\brak{0|1} = \frac{6}{25}\,
p_{Y|X}\brak{1|0} = \frac{45}{50}\,
p_{Y|X}\brak{1|2} = \frac{5}{50}
\end{align}
The desired probability is the probability that a slip drawn at random is marked other than Rs 1,
\begin{align}
&=1-p_X\brak{0}\\
&= p_X(1) + p_X(2)
\end{align}
Using Bayes theorem,
\begin{align}
&= p_Y\brak{0} \times \pr{Y=0 | X=1} + p_Y\brak{1} \times \pr{Y=1|X=2}\\
&=\frac{1}{3} \times \frac{6}{25} + \frac{2}{3} \times \frac{5}{50}\\
&=\frac{11}{75}
\end{align}

\newpage

%\tableofcontents

\bigskip

\renewcommand{\thefigure}{\theenumi}
\renewcommand{\thetable}{\theenumi}
%\renewcommand{\theequation}{\theenumi}

%\begin{abstract}
%%\boldmath
%In this letter, an algorithm for evaluating the exact analytical bit error rate  (BER)  for the piecewise linear (PL) combiner for  multiple relays is presented. Previous results were available only for upto three relays. The algorithm is unique in the sense that  the actual mathematical expressions, that are prohibitively large, need not be explicitly obtained. The diversity gain due to multiple relays is shown through plots of the analytical BER, well supported by simulations. 
%
%\end{abstract}
% IEEEtran.cls defaults to using nonbold math in the Abstract.
% This preserves the distinction between vectors and scalars. However,
% if the journal you are submitting to favors bold math in the abstract,
% then you can use LaTeX's standard command \boldmath at the very start
% of the abstract to achieve this. Many IEEE journals frown on math
% in the abstract anyway.

% Note that keywords are not normally used for peerreview papers.
%\begin{IEEEkeywords}
%Cooperative diversity, decode and forward, piecewise linear
%\end{IEEEkeywords}



% For peer review papers, you can put extra information on the cover
% page as needed:
% \ifCLASSOPTIONpeerreview
% \begin{center} \bfseries EDICS Category: 3-BBND \end{center}
% \fi
%
% For peerreview papers, this IEEEtran command inserts a page break and
% creates the second title. It will be ignored for other modes.
%\IEEEpeerreviewmaketitle




\item One of the four persons John, Rita, Aslam or Gurpreet will be promoted next
month. Consequently the sample space consists of four elementary outcomes
S = {John promoted, Rita promoted, Aslam promoted, Gurpreet promoted}
You are told that the chances of John’s promotion is same as that of Gurpreet,
Rita’s chances of promotion are twice as likely as Johns. Aslam’s chances are
four times that of John.
\begin{enumerate}
	\item Determine
	\begin{enumerate}
		\item P (John promoted)
		\item P (Rita promoted)
		\item P (Aslam promoted)
		\item P (Gurpreet promoted)
	\end{enumerate}
	\item If A = {John promoted or Gurpreet promoted}, find P (A).
\end{enumerate}
\solution
%\begin{table}[H]
	\centering
\begin{tabular}{|c|c|c|}
\hline
Random variable &Value &Definition\\ \hline
\multirow{3}{*}{X} &0 &Slips of Rs 1\\
&1 &Slips of Rs 5\\
&2 &Slips of Rs 13\\ \hline
\multirow{2}{*}{Y} &0 &Box A\\
&1 &Box B\\\hline
\end{tabular}
\caption{}
\label{tab:Distribution}
\end{table}
See \tabref{tab:Distribution}.
\begin{align}
p_{Y}\brak{k}= \begin{cases} 
      \frac{1}{3} & {k=0} \\
      \frac{2}{3 }& {k=1} 
   \end{cases}
   \\
p_{Y|X}\brak{0|0} = \frac{19}{25}\, 
p_{Y|X}\brak{0|1} = \frac{6}{25}\,
p_{Y|X}\brak{1|0} = \frac{45}{50}\,
p_{Y|X}\brak{1|2} = \frac{5}{50}
\end{align}
The desired probability is the probability that a slip drawn at random is marked other than Rs 1,
\begin{align}
&=1-p_X\brak{0}\\
&= p_X(1) + p_X(2)
\end{align}
Using Bayes theorem,
\begin{align}
&= p_Y\brak{0} \times \pr{Y=0 | X=1} + p_Y\brak{1} \times \pr{Y=1|X=2}\\
&=\frac{1}{3} \times \frac{6}{25} + \frac{2}{3} \times \frac{5}{50}\\
&=\frac{11}{75}
\end{align}

\newpage

%\tableofcontents

\bigskip

\renewcommand{\thefigure}{\theenumi}
\renewcommand{\thetable}{\theenumi}
%\renewcommand{\theequation}{\theenumi}

%\begin{abstract}
%%\boldmath
%In this letter, an algorithm for evaluating the exact analytical bit error rate  (BER)  for the piecewise linear (PL) combiner for  multiple relays is presented. Previous results were available only for upto three relays. The algorithm is unique in the sense that  the actual mathematical expressions, that are prohibitively large, need not be explicitly obtained. The diversity gain due to multiple relays is shown through plots of the analytical BER, well supported by simulations. 
%
%\end{abstract}
% IEEEtran.cls defaults to using nonbold math in the Abstract.
% This preserves the distinction between vectors and scalars. However,
% if the journal you are submitting to favors bold math in the abstract,
% then you can use LaTeX's standard command \boldmath at the very start
% of the abstract to achieve this. Many IEEE journals frown on math
% in the abstract anyway.

% Note that keywords are not normally used for peerreview papers.
%\begin{IEEEkeywords}
%Cooperative diversity, decode and forward, piecewise linear
%\end{IEEEkeywords}



% For peer review papers, you can put extra information on the cover
% page as needed:
% \ifCLASSOPTIONpeerreview
% \begin{center} \bfseries EDICS Category: 3-BBND \end{center}
% \fi
%
% For peerreview papers, this IEEEtran command inserts a page break and
% creates the second title. It will be ignored for other modes.
%\IEEEpeerreviewmaketitle




\item A card is drawn from a deck of 52 cards. Find the probability of getting a king or a heart or a red card.\\
\solution
%\begin{table}[H]
	\centering
\begin{tabular}{|c|c|c|}
\hline
Random variable &Value &Definition\\ \hline
\multirow{3}{*}{X} &0 &Slips of Rs 1\\
&1 &Slips of Rs 5\\
&2 &Slips of Rs 13\\ \hline
\multirow{2}{*}{Y} &0 &Box A\\
&1 &Box B\\\hline
\end{tabular}
\caption{}
\label{tab:Distribution}
\end{table}
See \tabref{tab:Distribution}.
\begin{align}
p_{Y}\brak{k}= \begin{cases} 
      \frac{1}{3} & {k=0} \\
      \frac{2}{3 }& {k=1} 
   \end{cases}
   \\
p_{Y|X}\brak{0|0} = \frac{19}{25}\, 
p_{Y|X}\brak{0|1} = \frac{6}{25}\,
p_{Y|X}\brak{1|0} = \frac{45}{50}\,
p_{Y|X}\brak{1|2} = \frac{5}{50}
\end{align}
The desired probability is the probability that a slip drawn at random is marked other than Rs 1,
\begin{align}
&=1-p_X\brak{0}\\
&= p_X(1) + p_X(2)
\end{align}
Using Bayes theorem,
\begin{align}
&= p_Y\brak{0} \times \pr{Y=0 | X=1} + p_Y\brak{1} \times \pr{Y=1|X=2}\\
&=\frac{1}{3} \times \frac{6}{25} + \frac{2}{3} \times \frac{5}{50}\\
&=\frac{11}{75}
\end{align}

\newpage

%\tableofcontents

\bigskip

\renewcommand{\thefigure}{\theenumi}
\renewcommand{\thetable}{\theenumi}
%\renewcommand{\theequation}{\theenumi}

%\begin{abstract}
%%\boldmath
%In this letter, an algorithm for evaluating the exact analytical bit error rate  (BER)  for the piecewise linear (PL) combiner for  multiple relays is presented. Previous results were available only for upto three relays. The algorithm is unique in the sense that  the actual mathematical expressions, that are prohibitively large, need not be explicitly obtained. The diversity gain due to multiple relays is shown through plots of the analytical BER, well supported by simulations. 
%
%\end{abstract}
% IEEEtran.cls defaults to using nonbold math in the Abstract.
% This preserves the distinction between vectors and scalars. However,
% if the journal you are submitting to favors bold math in the abstract,
% then you can use LaTeX's standard command \boldmath at the very start
% of the abstract to achieve this. Many IEEE journals frown on math
% in the abstract anyway.

% Note that keywords are not normally used for peerreview papers.
%\begin{IEEEkeywords}
%Cooperative diversity, decode and forward, piecewise linear
%\end{IEEEkeywords}



% For peer review papers, you can put extra information on the cover
% page as needed:
% \ifCLASSOPTIONpeerreview
% \begin{center} \bfseries EDICS Category: 3-BBND \end{center}
% \fi
%
% For peerreview papers, this IEEEtran command inserts a page break and
% creates the second title. It will be ignored for other modes.
%\IEEEpeerreviewmaketitle




\item The probability that a student will pass his examination is 0.73, the probability of
the student getting a compartment is 0.13, and the probability that the student will
either pass or get compartment is 0.96. State True or False.\\
\solution
%\begin{table}[H]
	\centering
\begin{tabular}{|c|c|c|}
\hline
Random variable &Value &Definition\\ \hline
\multirow{3}{*}{X} &0 &Slips of Rs 1\\
&1 &Slips of Rs 5\\
&2 &Slips of Rs 13\\ \hline
\multirow{2}{*}{Y} &0 &Box A\\
&1 &Box B\\\hline
\end{tabular}
\caption{}
\label{tab:Distribution}
\end{table}
See \tabref{tab:Distribution}.
\begin{align}
p_{Y}\brak{k}= \begin{cases} 
      \frac{1}{3} & {k=0} \\
      \frac{2}{3 }& {k=1} 
   \end{cases}
   \\
p_{Y|X}\brak{0|0} = \frac{19}{25}\, 
p_{Y|X}\brak{0|1} = \frac{6}{25}\,
p_{Y|X}\brak{1|0} = \frac{45}{50}\,
p_{Y|X}\brak{1|2} = \frac{5}{50}
\end{align}
The desired probability is the probability that a slip drawn at random is marked other than Rs 1,
\begin{align}
&=1-p_X\brak{0}\\
&= p_X(1) + p_X(2)
\end{align}
Using Bayes theorem,
\begin{align}
&= p_Y\brak{0} \times \pr{Y=0 | X=1} + p_Y\brak{1} \times \pr{Y=1|X=2}\\
&=\frac{1}{3} \times \frac{6}{25} + \frac{2}{3} \times \frac{5}{50}\\
&=\frac{11}{75}
\end{align}

\newpage

%\tableofcontents

\bigskip

\renewcommand{\thefigure}{\theenumi}
\renewcommand{\thetable}{\theenumi}
%\renewcommand{\theequation}{\theenumi}

%\begin{abstract}
%%\boldmath
%In this letter, an algorithm for evaluating the exact analytical bit error rate  (BER)  for the piecewise linear (PL) combiner for  multiple relays is presented. Previous results were available only for upto three relays. The algorithm is unique in the sense that  the actual mathematical expressions, that are prohibitively large, need not be explicitly obtained. The diversity gain due to multiple relays is shown through plots of the analytical BER, well supported by simulations. 
%
%\end{abstract}
% IEEEtran.cls defaults to using nonbold math in the Abstract.
% This preserves the distinction between vectors and scalars. However,
% if the journal you are submitting to favors bold math in the abstract,
% then you can use LaTeX's standard command \boldmath at the very start
% of the abstract to achieve this. Many IEEE journals frown on math
% in the abstract anyway.

% Note that keywords are not normally used for peerreview papers.
%\begin{IEEEkeywords}
%Cooperative diversity, decode and forward, piecewise linear
%\end{IEEEkeywords}



% For peer review papers, you can put extra information on the cover
% page as needed:
% \ifCLASSOPTIONpeerreview
% \begin{center} \bfseries EDICS Category: 3-BBND \end{center}
% \fi
%
% For peerreview papers, this IEEEtran command inserts a page break and
% creates the second title. It will be ignored for other modes.
%\IEEEpeerreviewmaketitle




\item A card is selected from a pack of 52 cards\\
\begin{enumerate}[label=(\alph*)]
\item How many points are there in the sample space?
\item Calculate the probability that the cards is an ace of spades.
\item Calculate the probability that the card is (i) an ace (ii)black card.\\
\end{enumerate}
%\input{ncert/11/16/3/4_1/Prob_4.tex}
\item In a non-leap year, the probability of having 53 tuesdays or 53 wednesdays is\\
\solution
%A non-leap year has a total of 365 days, and a week has 7 days.\\
So it can be expressed as 
\begin{align}
365\text{days} &=52\times 7+1 \text{day}
\end{align}
$\implies$ 52 tuesdays or wednesdays\\
Random variable X denotes the days of a week
\begin{align}
p_X\brak{k}&=\frac{1}{7}; \quad \brak{1<k<7}
\end{align}
So the probability of extra day being tuesday or wednesday is
\begin{align}
p_X\brak{3}+p_X\brak{4}&=\frac{1}{7}+\frac{1}{7}=\frac{2}{7}
\end{align}



\item There are 1000 sealed envelopes in a box, 10 of them contain a cash prize of
Rs 100 each, 100 of them contain a cash prize of Rs 50 each and 200 of them
contain a cash prize of Rs 10 each and rest do not contain any cash prize. If they
are well shuffled and an envelope is picked up out, what is the probability that it
contains no cash prize?\\
\solution
%\begin{table}[H]
	\centering
\begin{tabular}{|c|c|c|}
\hline
Random variable &Value &Definition\\ \hline
\multirow{3}{*}{X} &0 &Slips of Rs 1\\
&1 &Slips of Rs 5\\
&2 &Slips of Rs 13\\ \hline
\multirow{2}{*}{Y} &0 &Box A\\
&1 &Box B\\\hline
\end{tabular}
\caption{}
\label{tab:Distribution}
\end{table}
See \tabref{tab:Distribution}.
\begin{align}
p_{Y}\brak{k}= \begin{cases} 
      \frac{1}{3} & {k=0} \\
      \frac{2}{3 }& {k=1} 
   \end{cases}
   \\
p_{Y|X}\brak{0|0} = \frac{19}{25}\, 
p_{Y|X}\brak{0|1} = \frac{6}{25}\,
p_{Y|X}\brak{1|0} = \frac{45}{50}\,
p_{Y|X}\brak{1|2} = \frac{5}{50}
\end{align}
The desired probability is the probability that a slip drawn at random is marked other than Rs 1,
\begin{align}
&=1-p_X\brak{0}\\
&= p_X(1) + p_X(2)
\end{align}
Using Bayes theorem,
\begin{align}
&= p_Y\brak{0} \times \pr{Y=0 | X=1} + p_Y\brak{1} \times \pr{Y=1|X=2}\\
&=\frac{1}{3} \times \frac{6}{25} + \frac{2}{3} \times \frac{5}{50}\\
&=\frac{11}{75}
\end{align}

\newpage

%\tableofcontents

\bigskip

\renewcommand{\thefigure}{\theenumi}
\renewcommand{\thetable}{\theenumi}
%\renewcommand{\theequation}{\theenumi}

%\begin{abstract}
%%\boldmath
%In this letter, an algorithm for evaluating the exact analytical bit error rate  (BER)  for the piecewise linear (PL) combiner for  multiple relays is presented. Previous results were available only for upto three relays. The algorithm is unique in the sense that  the actual mathematical expressions, that are prohibitively large, need not be explicitly obtained. The diversity gain due to multiple relays is shown through plots of the analytical BER, well supported by simulations. 
%
%\end{abstract}
% IEEEtran.cls defaults to using nonbold math in the Abstract.
% This preserves the distinction between vectors and scalars. However,
% if the journal you are submitting to favors bold math in the abstract,
% then you can use LaTeX's standard command \boldmath at the very start
% of the abstract to achieve this. Many IEEE journals frown on math
% in the abstract anyway.

% Note that keywords are not normally used for peerreview papers.
%\begin{IEEEkeywords}
%Cooperative diversity, decode and forward, piecewise linear
%\end{IEEEkeywords}



% For peer review papers, you can put extra information on the cover
% page as needed:
% \ifCLASSOPTIONpeerreview
% \begin{center} \bfseries EDICS Category: 3-BBND \end{center}
% \fi
%
% For peerreview papers, this IEEEtran command inserts a page break and
% creates the second title. It will be ignored for other modes.
%\IEEEpeerreviewmaketitle




\item 
A die is thrown and a card is selected at random from a deck of 52 playing cards. The probability of getting an even number on the die and a spade card.\\
\solution
%\begin{table}[H]
	\centering
\begin{tabular}{|c|c|c|}
\hline
Random variable &Value &Definition\\ \hline
\multirow{3}{*}{X} &0 &Slips of Rs 1\\
&1 &Slips of Rs 5\\
&2 &Slips of Rs 13\\ \hline
\multirow{2}{*}{Y} &0 &Box A\\
&1 &Box B\\\hline
\end{tabular}
\caption{}
\label{tab:Distribution}
\end{table}
See \tabref{tab:Distribution}.
\begin{align}
p_{Y}\brak{k}= \begin{cases} 
      \frac{1}{3} & {k=0} \\
      \frac{2}{3 }& {k=1} 
   \end{cases}
   \\
p_{Y|X}\brak{0|0} = \frac{19}{25}\, 
p_{Y|X}\brak{0|1} = \frac{6}{25}\,
p_{Y|X}\brak{1|0} = \frac{45}{50}\,
p_{Y|X}\brak{1|2} = \frac{5}{50}
\end{align}
The desired probability is the probability that a slip drawn at random is marked other than Rs 1,
\begin{align}
&=1-p_X\brak{0}\\
&= p_X(1) + p_X(2)
\end{align}
Using Bayes theorem,
\begin{align}
&= p_Y\brak{0} \times \pr{Y=0 | X=1} + p_Y\brak{1} \times \pr{Y=1|X=2}\\
&=\frac{1}{3} \times \frac{6}{25} + \frac{2}{3} \times \frac{5}{50}\\
&=\frac{11}{75}
\end{align}

\newpage

%\tableofcontents

\bigskip

\renewcommand{\thefigure}{\theenumi}
\renewcommand{\thetable}{\theenumi}
%\renewcommand{\theequation}{\theenumi}

%\begin{abstract}
%%\boldmath
%In this letter, an algorithm for evaluating the exact analytical bit error rate  (BER)  for the piecewise linear (PL) combiner for  multiple relays is presented. Previous results were available only for upto three relays. The algorithm is unique in the sense that  the actual mathematical expressions, that are prohibitively large, need not be explicitly obtained. The diversity gain due to multiple relays is shown through plots of the analytical BER, well supported by simulations. 
%
%\end{abstract}
% IEEEtran.cls defaults to using nonbold math in the Abstract.
% This preserves the distinction between vectors and scalars. However,
% if the journal you are submitting to favors bold math in the abstract,
% then you can use LaTeX's standard command \boldmath at the very start
% of the abstract to achieve this. Many IEEE journals frown on math
% in the abstract anyway.

% Note that keywords are not normally used for peerreview papers.
%\begin{IEEEkeywords}
%Cooperative diversity, decode and forward, piecewise linear
%\end{IEEEkeywords}



% For peer review papers, you can put extra information on the cover
% page as needed:
% \ifCLASSOPTIONpeerreview
% \begin{center} \bfseries EDICS Category: 3-BBND \end{center}
% \fi
%
% For peerreview papers, this IEEEtran command inserts a page break and
% creates the second title. It will be ignored for other modes.
%\IEEEpeerreviewmaketitle




\item
If 4-digit numbers greater than 5,000 are randomly formed from the digits 0, 1, 3, 5, and 7, what is the probability of forming a number divisible by 5 when:
\begin{enumerate}
    \item The digits are repeated?
    \item The repetition of digits is not allowed?
\end{enumerate}
\solution
%\begin{table}[H]
	\centering
\begin{tabular}{|c|c|c|}
\hline
Random variable &Value &Definition\\ \hline
\multirow{3}{*}{X} &0 &Slips of Rs 1\\
&1 &Slips of Rs 5\\
&2 &Slips of Rs 13\\ \hline
\multirow{2}{*}{Y} &0 &Box A\\
&1 &Box B\\\hline
\end{tabular}
\caption{}
\label{tab:Distribution}
\end{table}
See \tabref{tab:Distribution}.
\begin{align}
p_{Y}\brak{k}= \begin{cases} 
      \frac{1}{3} & {k=0} \\
      \frac{2}{3 }& {k=1} 
   \end{cases}
   \\
p_{Y|X}\brak{0|0} = \frac{19}{25}\, 
p_{Y|X}\brak{0|1} = \frac{6}{25}\,
p_{Y|X}\brak{1|0} = \frac{45}{50}\,
p_{Y|X}\brak{1|2} = \frac{5}{50}
\end{align}
The desired probability is the probability that a slip drawn at random is marked other than Rs 1,
\begin{align}
&=1-p_X\brak{0}\\
&= p_X(1) + p_X(2)
\end{align}
Using Bayes theorem,
\begin{align}
&= p_Y\brak{0} \times \pr{Y=0 | X=1} + p_Y\brak{1} \times \pr{Y=1|X=2}\\
&=\frac{1}{3} \times \frac{6}{25} + \frac{2}{3} \times \frac{5}{50}\\
&=\frac{11}{75}
\end{align}

\newpage

%\tableofcontents

\bigskip

\renewcommand{\thefigure}{\theenumi}
\renewcommand{\thetable}{\theenumi}
%\renewcommand{\theequation}{\theenumi}

%\begin{abstract}
%%\boldmath
%In this letter, an algorithm for evaluating the exact analytical bit error rate  (BER)  for the piecewise linear (PL) combiner for  multiple relays is presented. Previous results were available only for upto three relays. The algorithm is unique in the sense that  the actual mathematical expressions, that are prohibitively large, need not be explicitly obtained. The diversity gain due to multiple relays is shown through plots of the analytical BER, well supported by simulations. 
%
%\end{abstract}
% IEEEtran.cls defaults to using nonbold math in the Abstract.
% This preserves the distinction between vectors and scalars. However,
% if the journal you are submitting to favors bold math in the abstract,
% then you can use LaTeX's standard command \boldmath at the very start
% of the abstract to achieve this. Many IEEE journals frown on math
% in the abstract anyway.

% Note that keywords are not normally used for peerreview papers.
%\begin{IEEEkeywords}
%Cooperative diversity, decode and forward, piecewise linear
%\end{IEEEkeywords}



% For peer review papers, you can put extra information on the cover
% page as needed:
% \ifCLASSOPTIONpeerreview
% \begin{center} \bfseries EDICS Category: 3-BBND \end{center}
% \fi
%
% For peerreview papers, this IEEEtran command inserts a page break and
% creates the second title. It will be ignored for other modes.
%\IEEEpeerreviewmaketitle




\item Consider the probability space $\brak{\Omega, \mathcal{G}, P}$ where $\Omega = [0,2]$ and $\mathcal{G} = \cbrak{\phi, \Omega, [0,1], (1,2]}$. Let $X$ and $Y$ be two functions on $\Omega$ defined as
\begin{align*}
    X(\omega) = 
    \begin{cases}
        1 & \text{if }\omega \in [0, 1]\\
        2 & \text{if }\omega \in (1, 2]
    \end{cases}
\end{align*}
and
\begin{align*}
    Y(\omega) = 
    \begin{cases}
        2 & \text{if }\omega \in [0, 1.5]\\
        3 & \text{if }\omega \in (1.5, 2].
    \end{cases}
\end{align*}
Then which one of the following statements is true?
\begin{enumerate}
    \item [(A)] $X$ is a random variable with respect to $\mathcal{G}$, but $Y$ is not a random variable with respect to $\mathcal{G}$.
    \item [(B)] $Y$ is a random variable with respect to $\mathcal{G}$, but $X$ is not a random variable with respect to $\mathcal{G}$.
    \item [(C)] Neither $X$ nor $Y$ is a random variable with respect to $\mathcal{G}$.
    \item [(D)] Both $X$ and $Y$ are random variables with respect to $\mathcal{G}$.
\end{enumerate} \hfill (GATE ST 2023)\\
\solution
%\begin{table}[H]
	\centering
\begin{tabular}{|c|c|c|}
\hline
Random variable &Value &Definition\\ \hline
\multirow{3}{*}{X} &0 &Slips of Rs 1\\
&1 &Slips of Rs 5\\
&2 &Slips of Rs 13\\ \hline
\multirow{2}{*}{Y} &0 &Box A\\
&1 &Box B\\\hline
\end{tabular}
\caption{}
\label{tab:Distribution}
\end{table}
See \tabref{tab:Distribution}.
\begin{align}
p_{Y}\brak{k}= \begin{cases} 
      \frac{1}{3} & {k=0} \\
      \frac{2}{3 }& {k=1} 
   \end{cases}
   \\
p_{Y|X}\brak{0|0} = \frac{19}{25}\, 
p_{Y|X}\brak{0|1} = \frac{6}{25}\,
p_{Y|X}\brak{1|0} = \frac{45}{50}\,
p_{Y|X}\brak{1|2} = \frac{5}{50}
\end{align}
The desired probability is the probability that a slip drawn at random is marked other than Rs 1,
\begin{align}
&=1-p_X\brak{0}\\
&= p_X(1) + p_X(2)
\end{align}
Using Bayes theorem,
\begin{align}
&= p_Y\brak{0} \times \pr{Y=0 | X=1} + p_Y\brak{1} \times \pr{Y=1|X=2}\\
&=\frac{1}{3} \times \frac{6}{25} + \frac{2}{3} \times \frac{5}{50}\\
&=\frac{11}{75}
\end{align}

\newpage

%\tableofcontents

\bigskip

\renewcommand{\thefigure}{\theenumi}
\renewcommand{\thetable}{\theenumi}
%\renewcommand{\theequation}{\theenumi}

%\begin{abstract}
%%\boldmath
%In this letter, an algorithm for evaluating the exact analytical bit error rate  (BER)  for the piecewise linear (PL) combiner for  multiple relays is presented. Previous results were available only for upto three relays. The algorithm is unique in the sense that  the actual mathematical expressions, that are prohibitively large, need not be explicitly obtained. The diversity gain due to multiple relays is shown through plots of the analytical BER, well supported by simulations. 
%
%\end{abstract}
% IEEEtran.cls defaults to using nonbold math in the Abstract.
% This preserves the distinction between vectors and scalars. However,
% if the journal you are submitting to favors bold math in the abstract,
% then you can use LaTeX's standard command \boldmath at the very start
% of the abstract to achieve this. Many IEEE journals frown on math
% in the abstract anyway.

% Note that keywords are not normally used for peerreview papers.
%\begin{IEEEkeywords}
%Cooperative diversity, decode and forward, piecewise linear
%\end{IEEEkeywords}



% For peer review papers, you can put extra information on the cover
% page as needed:
% \ifCLASSOPTIONpeerreview
% \begin{center} \bfseries EDICS Category: 3-BBND \end{center}
% \fi
%
% For peerreview papers, this IEEEtran command inserts a page break and
% creates the second title. It will be ignored for other modes.
%\IEEEpeerreviewmaketitle




	\item  A die is loaded in such a way that each odd number is twice as likely to occur as
each even number. Find $P(G)$, where $G$ is the event that a number greater than
3 occurs on a single roll of the die.
\\
\solution
		%\begin{table}[H]
	\centering
\begin{tabular}{|c|c|c|}
\hline
Random variable &Value &Definition\\ \hline
\multirow{3}{*}{X} &0 &Slips of Rs 1\\
&1 &Slips of Rs 5\\
&2 &Slips of Rs 13\\ \hline
\multirow{2}{*}{Y} &0 &Box A\\
&1 &Box B\\\hline
\end{tabular}
\caption{}
\label{tab:Distribution}
\end{table}
See \tabref{tab:Distribution}.
\begin{align}
p_{Y}\brak{k}= \begin{cases} 
      \frac{1}{3} & {k=0} \\
      \frac{2}{3 }& {k=1} 
   \end{cases}
   \\
p_{Y|X}\brak{0|0} = \frac{19}{25}\, 
p_{Y|X}\brak{0|1} = \frac{6}{25}\,
p_{Y|X}\brak{1|0} = \frac{45}{50}\,
p_{Y|X}\brak{1|2} = \frac{5}{50}
\end{align}
The desired probability is the probability that a slip drawn at random is marked other than Rs 1,
\begin{align}
&=1-p_X\brak{0}\\
&= p_X(1) + p_X(2)
\end{align}
Using Bayes theorem,
\begin{align}
&= p_Y\brak{0} \times \pr{Y=0 | X=1} + p_Y\brak{1} \times \pr{Y=1|X=2}\\
&=\frac{1}{3} \times \frac{6}{25} + \frac{2}{3} \times \frac{5}{50}\\
&=\frac{11}{75}
\end{align}

\newpage

%\tableofcontents

\bigskip

\renewcommand{\thefigure}{\theenumi}
\renewcommand{\thetable}{\theenumi}
%\renewcommand{\theequation}{\theenumi}

%\begin{abstract}
%%\boldmath
%In this letter, an algorithm for evaluating the exact analytical bit error rate  (BER)  for the piecewise linear (PL) combiner for  multiple relays is presented. Previous results were available only for upto three relays. The algorithm is unique in the sense that  the actual mathematical expressions, that are prohibitively large, need not be explicitly obtained. The diversity gain due to multiple relays is shown through plots of the analytical BER, well supported by simulations. 
%
%\end{abstract}
% IEEEtran.cls defaults to using nonbold math in the Abstract.
% This preserves the distinction between vectors and scalars. However,
% if the journal you are submitting to favors bold math in the abstract,
% then you can use LaTeX's standard command \boldmath at the very start
% of the abstract to achieve this. Many IEEE journals frown on math
% in the abstract anyway.

% Note that keywords are not normally used for peerreview papers.
%\begin{IEEEkeywords}
%Cooperative diversity, decode and forward, piecewise linear
%\end{IEEEkeywords}



% For peer review papers, you can put extra information on the cover
% page as needed:
% \ifCLASSOPTIONpeerreview
% \begin{center} \bfseries EDICS Category: 3-BBND \end{center}
% \fi
%
% For peerreview papers, this IEEEtran command inserts a page break and
% creates the second title. It will be ignored for other modes.
%\IEEEpeerreviewmaketitle




	\item All the jacks, queens and kings are removed from a deck of 52 playing cards. The remaining cards are well shuffled and then one card is drawn at random. Giving ace a value 1 similar value for other cards, find the probability that the card has a value 
		\begin{enumerate}
			\item 7
			\item greater than 7
			\item less than 7
		\end{enumerate}
		%Number of cards left after removing all jacks, queens and kings 
\begin{align}
N	= 52 - 4\times 3
	= 40
\end{align}
%\begin{table}[H]
%\def\arraystretch{1.2}
%\begin{tabular}{|c|c|c|}
%\hline
%	\textbf{Parameter} &\textbf{Value} &\textbf{Description}\\ \hline
%	$X$ &1-10 &Represents the value of the card picked \\ \hline
%\end{tabular}
%\end{table}
Let $1 \le X \le 10$ be the value of the card picked.  Then,
\begin{align}
	p_X(k) &= \Pr(X=k)\ \forall\ 1 \leq k \leq 10\\
	&= \frac{4\times 1}{40}\\
	&= \frac{1}{10}\\
	\therefore p_X(k) &= 
	\begin{cases}
		\frac{1}{10} & 1 \leq k \leq 10\\
		0 & \text{otherwise}
	\end{cases}
\end{align}
and
\begin{align}
	F_{X}(k) &= \sum_{m=0}^{k}p_{X}(m) \quad 1 \leq k \leq 10\\
	&= \frac{k}{10}\\
	\therefore F_{X}(k) &= 
	\begin{cases}
		0 & k \leq 0\\
		\frac{k}{10} & 1\leq k \leq 10\\
		1 & k > 10 
	\end{cases}
\end{align}
\begin{enumerate}
	\item Probability that card has value equal to 7 is
		\begin{align}
			 p_{X}(7)
			= \frac{1}{10}
		\end{align}
	\item Probability that card has value greater than 7 is
		\begin{align}
			1 - F_X(7)
			&= 1 - \frac{7}{10}
			\\
			&= \frac{3}{10}
		\end{align}
	\item Probability that card has value less than 7 is
		\begin{align}
			 F_{X}(6)
			=\frac{6}{10}
		\end{align}
\end{enumerate}

  \item A Lot consists of 48 mobile phones of which 42 are good, 3 have only minor defects and 3 have major defects.Varnika will buy a phone if it is good but the trader will only buy a mobile if it has no major defects. One phone is selected at random from the lot. What is the probability that it is
\begin{enumerate}
	\item acceptable to Varnika?
            \item acceptable to the trader?
\end{enumerate}
\solution
	%\begin{table}[H]
	\centering
\begin{tabular}{|c|c|c|}
\hline
Random variable &Value &Definition\\ \hline
\multirow{3}{*}{X} &0 &Slips of Rs 1\\
&1 &Slips of Rs 5\\
&2 &Slips of Rs 13\\ \hline
\multirow{2}{*}{Y} &0 &Box A\\
&1 &Box B\\\hline
\end{tabular}
\caption{}
\label{tab:Distribution}
\end{table}
See \tabref{tab:Distribution}.
\begin{align}
p_{Y}\brak{k}= \begin{cases} 
      \frac{1}{3} & {k=0} \\
      \frac{2}{3 }& {k=1} 
   \end{cases}
   \\
p_{Y|X}\brak{0|0} = \frac{19}{25}\, 
p_{Y|X}\brak{0|1} = \frac{6}{25}\,
p_{Y|X}\brak{1|0} = \frac{45}{50}\,
p_{Y|X}\brak{1|2} = \frac{5}{50}
\end{align}
The desired probability is the probability that a slip drawn at random is marked other than Rs 1,
\begin{align}
&=1-p_X\brak{0}\\
&= p_X(1) + p_X(2)
\end{align}
Using Bayes theorem,
\begin{align}
&= p_Y\brak{0} \times \pr{Y=0 | X=1} + p_Y\brak{1} \times \pr{Y=1|X=2}\\
&=\frac{1}{3} \times \frac{6}{25} + \frac{2}{3} \times \frac{5}{50}\\
&=\frac{11}{75}
\end{align}

\newpage

%\tableofcontents

\bigskip

\renewcommand{\thefigure}{\theenumi}
\renewcommand{\thetable}{\theenumi}
%\renewcommand{\theequation}{\theenumi}

%\begin{abstract}
%%\boldmath
%In this letter, an algorithm for evaluating the exact analytical bit error rate  (BER)  for the piecewise linear (PL) combiner for  multiple relays is presented. Previous results were available only for upto three relays. The algorithm is unique in the sense that  the actual mathematical expressions, that are prohibitively large, need not be explicitly obtained. The diversity gain due to multiple relays is shown through plots of the analytical BER, well supported by simulations. 
%
%\end{abstract}
% IEEEtran.cls defaults to using nonbold math in the Abstract.
% This preserves the distinction between vectors and scalars. However,
% if the journal you are submitting to favors bold math in the abstract,
% then you can use LaTeX's standard command \boldmath at the very start
% of the abstract to achieve this. Many IEEE journals frown on math
% in the abstract anyway.

% Note that keywords are not normally used for peerreview papers.
%\begin{IEEEkeywords}
%Cooperative diversity, decode and forward, piecewise linear
%\end{IEEEkeywords}



% For peer review papers, you can put extra information on the cover
% page as needed:
% \ifCLASSOPTIONpeerreview
% \begin{center} \bfseries EDICS Category: 3-BBND \end{center}
% \fi
%
% For peerreview papers, this IEEEtran command inserts a page break and
% creates the second title. It will be ignored for other modes.
%\IEEEpeerreviewmaketitle




 \item A student says that if you throw a die, it will show up 1 or not 1. Therefore, the probability of getting 1 and the probability of getting 'not 1' each is equal to $\frac{1}{2}$. Is this correct? Give reasons.\\
 \solution
        %\begin{table}[H]
	\centering
\begin{tabular}{|c|c|c|}
\hline
Random variable &Value &Definition\\ \hline
\multirow{3}{*}{X} &0 &Slips of Rs 1\\
&1 &Slips of Rs 5\\
&2 &Slips of Rs 13\\ \hline
\multirow{2}{*}{Y} &0 &Box A\\
&1 &Box B\\\hline
\end{tabular}
\caption{}
\label{tab:Distribution}
\end{table}
See \tabref{tab:Distribution}.
\begin{align}
p_{Y}\brak{k}= \begin{cases} 
      \frac{1}{3} & {k=0} \\
      \frac{2}{3 }& {k=1} 
   \end{cases}
   \\
p_{Y|X}\brak{0|0} = \frac{19}{25}\, 
p_{Y|X}\brak{0|1} = \frac{6}{25}\,
p_{Y|X}\brak{1|0} = \frac{45}{50}\,
p_{Y|X}\brak{1|2} = \frac{5}{50}
\end{align}
The desired probability is the probability that a slip drawn at random is marked other than Rs 1,
\begin{align}
&=1-p_X\brak{0}\\
&= p_X(1) + p_X(2)
\end{align}
Using Bayes theorem,
\begin{align}
&= p_Y\brak{0} \times \pr{Y=0 | X=1} + p_Y\brak{1} \times \pr{Y=1|X=2}\\
&=\frac{1}{3} \times \frac{6}{25} + \frac{2}{3} \times \frac{5}{50}\\
&=\frac{11}{75}
\end{align}

\newpage

%\tableofcontents

\bigskip

\renewcommand{\thefigure}{\theenumi}
\renewcommand{\thetable}{\theenumi}
%\renewcommand{\theequation}{\theenumi}

%\begin{abstract}
%%\boldmath
%In this letter, an algorithm for evaluating the exact analytical bit error rate  (BER)  for the piecewise linear (PL) combiner for  multiple relays is presented. Previous results were available only for upto three relays. The algorithm is unique in the sense that  the actual mathematical expressions, that are prohibitively large, need not be explicitly obtained. The diversity gain due to multiple relays is shown through plots of the analytical BER, well supported by simulations. 
%
%\end{abstract}
% IEEEtran.cls defaults to using nonbold math in the Abstract.
% This preserves the distinction between vectors and scalars. However,
% if the journal you are submitting to favors bold math in the abstract,
% then you can use LaTeX's standard command \boldmath at the very start
% of the abstract to achieve this. Many IEEE journals frown on math
% in the abstract anyway.

% Note that keywords are not normally used for peerreview papers.
%\begin{IEEEkeywords}
%Cooperative diversity, decode and forward, piecewise linear
%\end{IEEEkeywords}



% For peer review papers, you can put extra information on the cover
% page as needed:
% \ifCLASSOPTIONpeerreview
% \begin{center} \bfseries EDICS Category: 3-BBND \end{center}
% \fi
%
% For peerreview papers, this IEEEtran command inserts a page break and
% creates the second title. It will be ignored for other modes.
%\IEEEpeerreviewmaketitle




   \item Four candidates A, B, C, D have ap-
plied for the assignment to coach a school cricket
team. If A is twice as likely to be selected as B, and
B and C are given about the same chance of being
selected, while C is twice as likely to be selected
as D, what are the probabilities that
\begin{enumerate}
\item C will be selected?
\item A will not be selected?
\end{enumerate}
	%\begin{table}[H]
	\centering
\begin{tabular}{|c|c|c|}
\hline
Random variable &Value &Definition\\ \hline
\multirow{3}{*}{X} &0 &Slips of Rs 1\\
&1 &Slips of Rs 5\\
&2 &Slips of Rs 13\\ \hline
\multirow{2}{*}{Y} &0 &Box A\\
&1 &Box B\\\hline
\end{tabular}
\caption{}
\label{tab:Distribution}
\end{table}
See \tabref{tab:Distribution}.
\begin{align}
p_{Y}\brak{k}= \begin{cases} 
      \frac{1}{3} & {k=0} \\
      \frac{2}{3 }& {k=1} 
   \end{cases}
   \\
p_{Y|X}\brak{0|0} = \frac{19}{25}\, 
p_{Y|X}\brak{0|1} = \frac{6}{25}\,
p_{Y|X}\brak{1|0} = \frac{45}{50}\,
p_{Y|X}\brak{1|2} = \frac{5}{50}
\end{align}
The desired probability is the probability that a slip drawn at random is marked other than Rs 1,
\begin{align}
&=1-p_X\brak{0}\\
&= p_X(1) + p_X(2)
\end{align}
Using Bayes theorem,
\begin{align}
&= p_Y\brak{0} \times \pr{Y=0 | X=1} + p_Y\brak{1} \times \pr{Y=1|X=2}\\
&=\frac{1}{3} \times \frac{6}{25} + \frac{2}{3} \times \frac{5}{50}\\
&=\frac{11}{75}
\end{align}

\newpage

%\tableofcontents

\bigskip

\renewcommand{\thefigure}{\theenumi}
\renewcommand{\thetable}{\theenumi}
%\renewcommand{\theequation}{\theenumi}

%\begin{abstract}
%%\boldmath
%In this letter, an algorithm for evaluating the exact analytical bit error rate  (BER)  for the piecewise linear (PL) combiner for  multiple relays is presented. Previous results were available only for upto three relays. The algorithm is unique in the sense that  the actual mathematical expressions, that are prohibitively large, need not be explicitly obtained. The diversity gain due to multiple relays is shown through plots of the analytical BER, well supported by simulations. 
%
%\end{abstract}
% IEEEtran.cls defaults to using nonbold math in the Abstract.
% This preserves the distinction between vectors and scalars. However,
% if the journal you are submitting to favors bold math in the abstract,
% then you can use LaTeX's standard command \boldmath at the very start
% of the abstract to achieve this. Many IEEE journals frown on math
% in the abstract anyway.

% Note that keywords are not normally used for peerreview papers.
%\begin{IEEEkeywords}
%Cooperative diversity, decode and forward, piecewise linear
%\end{IEEEkeywords}



% For peer review papers, you can put extra information on the cover
% page as needed:
% \ifCLASSOPTIONpeerreview
% \begin{center} \bfseries EDICS Category: 3-BBND \end{center}
% \fi
%
% For peerreview papers, this IEEEtran command inserts a page break and
% creates the second title. It will be ignored for other modes.
%\IEEEpeerreviewmaketitle




 \item A bag contain 24 balls of which $x$ balls are red, $2x$ are white and $3x$ are blue. A ball is selected at random, What is the probability that it is
\begin{enumerate}[label=\alph*)]
\item not red ?
\item white ?
\end{enumerate}
%\begin{table}[H]
	\centering
\begin{tabular}{|c|c|c|}
\hline
Random variable &Value &Definition\\ \hline
\multirow{3}{*}{X} &0 &Slips of Rs 1\\
&1 &Slips of Rs 5\\
&2 &Slips of Rs 13\\ \hline
\multirow{2}{*}{Y} &0 &Box A\\
&1 &Box B\\\hline
\end{tabular}
\caption{}
\label{tab:Distribution}
\end{table}
See \tabref{tab:Distribution}.
\begin{align}
p_{Y}\brak{k}= \begin{cases} 
      \frac{1}{3} & {k=0} \\
      \frac{2}{3 }& {k=1} 
   \end{cases}
   \\
p_{Y|X}\brak{0|0} = \frac{19}{25}\, 
p_{Y|X}\brak{0|1} = \frac{6}{25}\,
p_{Y|X}\brak{1|0} = \frac{45}{50}\,
p_{Y|X}\brak{1|2} = \frac{5}{50}
\end{align}
The desired probability is the probability that a slip drawn at random is marked other than Rs 1,
\begin{align}
&=1-p_X\brak{0}\\
&= p_X(1) + p_X(2)
\end{align}
Using Bayes theorem,
\begin{align}
&= p_Y\brak{0} \times \pr{Y=0 | X=1} + p_Y\brak{1} \times \pr{Y=1|X=2}\\
&=\frac{1}{3} \times \frac{6}{25} + \frac{2}{3} \times \frac{5}{50}\\
&=\frac{11}{75}
\end{align}

\newpage

%\tableofcontents

\bigskip

\renewcommand{\thefigure}{\theenumi}
\renewcommand{\thetable}{\theenumi}
%\renewcommand{\theequation}{\theenumi}

%\begin{abstract}
%%\boldmath
%In this letter, an algorithm for evaluating the exact analytical bit error rate  (BER)  for the piecewise linear (PL) combiner for  multiple relays is presented. Previous results were available only for upto three relays. The algorithm is unique in the sense that  the actual mathematical expressions, that are prohibitively large, need not be explicitly obtained. The diversity gain due to multiple relays is shown through plots of the analytical BER, well supported by simulations. 
%
%\end{abstract}
% IEEEtran.cls defaults to using nonbold math in the Abstract.
% This preserves the distinction between vectors and scalars. However,
% if the journal you are submitting to favors bold math in the abstract,
% then you can use LaTeX's standard command \boldmath at the very start
% of the abstract to achieve this. Many IEEE journals frown on math
% in the abstract anyway.

% Note that keywords are not normally used for peerreview papers.
%\begin{IEEEkeywords}
%Cooperative diversity, decode and forward, piecewise linear
%\end{IEEEkeywords}



% For peer review papers, you can put extra information on the cover
% page as needed:
% \ifCLASSOPTIONpeerreview
% \begin{center} \bfseries EDICS Category: 3-BBND \end{center}
% \fi
%
% For peerreview papers, this IEEEtran command inserts a page break and
% creates the second title. It will be ignored for other modes.
%\IEEEpeerreviewmaketitle




If the letters of the word ASSASSINATION are arranged at random. Find the Probability that
\begin{enumerate}[label=(\alph*)]
\item Four $S's$ come consecutively in the word
\item Two  $I's$ and two $N's$ come together
\item All $A's$ are not coming together
\item No two $A's$ are coming together
\end{enumerate}
%\begin{table}[H]
	\centering
\begin{tabular}{|c|c|c|}
\hline
Random variable &Value &Definition\\ \hline
\multirow{3}{*}{X} &0 &Slips of Rs 1\\
&1 &Slips of Rs 5\\
&2 &Slips of Rs 13\\ \hline
\multirow{2}{*}{Y} &0 &Box A\\
&1 &Box B\\\hline
\end{tabular}
\caption{}
\label{tab:Distribution}
\end{table}
See \tabref{tab:Distribution}.
\begin{align}
p_{Y}\brak{k}= \begin{cases} 
      \frac{1}{3} & {k=0} \\
      \frac{2}{3 }& {k=1} 
   \end{cases}
   \\
p_{Y|X}\brak{0|0} = \frac{19}{25}\, 
p_{Y|X}\brak{0|1} = \frac{6}{25}\,
p_{Y|X}\brak{1|0} = \frac{45}{50}\,
p_{Y|X}\brak{1|2} = \frac{5}{50}
\end{align}
The desired probability is the probability that a slip drawn at random is marked other than Rs 1,
\begin{align}
&=1-p_X\brak{0}\\
&= p_X(1) + p_X(2)
\end{align}
Using Bayes theorem,
\begin{align}
&= p_Y\brak{0} \times \pr{Y=0 | X=1} + p_Y\brak{1} \times \pr{Y=1|X=2}\\
&=\frac{1}{3} \times \frac{6}{25} + \frac{2}{3} \times \frac{5}{50}\\
&=\frac{11}{75}
\end{align}

\newpage

%\tableofcontents

\bigskip

\renewcommand{\thefigure}{\theenumi}
\renewcommand{\thetable}{\theenumi}
%\renewcommand{\theequation}{\theenumi}

%\begin{abstract}
%%\boldmath
%In this letter, an algorithm for evaluating the exact analytical bit error rate  (BER)  for the piecewise linear (PL) combiner for  multiple relays is presented. Previous results were available only for upto three relays. The algorithm is unique in the sense that  the actual mathematical expressions, that are prohibitively large, need not be explicitly obtained. The diversity gain due to multiple relays is shown through plots of the analytical BER, well supported by simulations. 
%
%\end{abstract}
% IEEEtran.cls defaults to using nonbold math in the Abstract.
% This preserves the distinction between vectors and scalars. However,
% if the journal you are submitting to favors bold math in the abstract,
% then you can use LaTeX's standard command \boldmath at the very start
% of the abstract to achieve this. Many IEEE journals frown on math
% in the abstract anyway.

% Note that keywords are not normally used for peerreview papers.
%\begin{IEEEkeywords}
%Cooperative diversity, decode and forward, piecewise linear
%\end{IEEEkeywords}



% For peer review papers, you can put extra information on the cover
% page as needed:
% \ifCLASSOPTIONpeerreview
% \begin{center} \bfseries EDICS Category: 3-BBND \end{center}
% \fi
%
% For peerreview papers, this IEEEtran command inserts a page break and
% creates the second title. It will be ignored for other modes.
%\IEEEpeerreviewmaketitle




	\item One urn contains two black balls (labelled B1 and B2) and one white ball. A
	second urn contains one black ball and two white balls (labelled W1 and W2).
	Suppose the following experiment is performed. One of the two urns is chosen
	at random. Next a ball is randomly chosen from the urn. Then a second ball is
	chosen at random from the same urn without replacing the first ball.
	
	\begin{enumerate}
	\item What is the probability that two black balls are chosen?
	
	\item What is the probability that two balls of opposite colour are chosen?
	\end{enumerate}
	\solution
	%\begin{align}
    \label{eq:12.13.6.18.1}
	\because	\pr{A|B} &> \pr{A},\
\frac{\pr{AB}}{\pr{B}} > \pr{A}
\\
    \label{eq:12.13.6.18.2}
	\implies \pr{AB} &> \pr{A}\pr{B}
	\\
	\text{or, } \frac{\pr{AB}}{\pr{A}} &=\pr{B|A} > \pr{A}
\end{align}

\end{enumerate}

	\item A bag contains 4 red and 4 black balls, another bag contains 2 red and 6 black balls. One of the two bags is selected at random and a ball is drawn from the bag which is found to be red. Find the probability that the ball is drawn from the first bag.
\\
\solution
		%\begin{table}[H]
	\centering
\begin{tabular}{|c|c|c|}
\hline
Random variable &Value &Definition\\ \hline
\multirow{3}{*}{X} &0 &Slips of Rs 1\\
&1 &Slips of Rs 5\\
&2 &Slips of Rs 13\\ \hline
\multirow{2}{*}{Y} &0 &Box A\\
&1 &Box B\\\hline
\end{tabular}
\caption{}
\label{tab:Distribution}
\end{table}
See \tabref{tab:Distribution}.
\begin{align}
p_{Y}\brak{k}= \begin{cases} 
      \frac{1}{3} & {k=0} \\
      \frac{2}{3 }& {k=1} 
   \end{cases}
   \\
p_{Y|X}\brak{0|0} = \frac{19}{25}\, 
p_{Y|X}\brak{0|1} = \frac{6}{25}\,
p_{Y|X}\brak{1|0} = \frac{45}{50}\,
p_{Y|X}\brak{1|2} = \frac{5}{50}
\end{align}
The desired probability is the probability that a slip drawn at random is marked other than Rs 1,
\begin{align}
&=1-p_X\brak{0}\\
&= p_X(1) + p_X(2)
\end{align}
Using Bayes theorem,
\begin{align}
&= p_Y\brak{0} \times \pr{Y=0 | X=1} + p_Y\brak{1} \times \pr{Y=1|X=2}\\
&=\frac{1}{3} \times \frac{6}{25} + \frac{2}{3} \times \frac{5}{50}\\
&=\frac{11}{75}
\end{align}

\newpage

%\tableofcontents

\bigskip

\renewcommand{\thefigure}{\theenumi}
\renewcommand{\thetable}{\theenumi}
%\renewcommand{\theequation}{\theenumi}

%\begin{abstract}
%%\boldmath
%In this letter, an algorithm for evaluating the exact analytical bit error rate  (BER)  for the piecewise linear (PL) combiner for  multiple relays is presented. Previous results were available only for upto three relays. The algorithm is unique in the sense that  the actual mathematical expressions, that are prohibitively large, need not be explicitly obtained. The diversity gain due to multiple relays is shown through plots of the analytical BER, well supported by simulations. 
%
%\end{abstract}
% IEEEtran.cls defaults to using nonbold math in the Abstract.
% This preserves the distinction between vectors and scalars. However,
% if the journal you are submitting to favors bold math in the abstract,
% then you can use LaTeX's standard command \boldmath at the very start
% of the abstract to achieve this. Many IEEE journals frown on math
% in the abstract anyway.

% Note that keywords are not normally used for peerreview papers.
%\begin{IEEEkeywords}
%Cooperative diversity, decode and forward, piecewise linear
%\end{IEEEkeywords}



% For peer review papers, you can put extra information on the cover
% page as needed:
% \ifCLASSOPTIONpeerreview
% \begin{center} \bfseries EDICS Category: 3-BBND \end{center}
% \fi
%
% For peerreview papers, this IEEEtran command inserts a page break and
% creates the second title. It will be ignored for other modes.
%\IEEEpeerreviewmaketitle




  \item
  Cards with numbers 2 to 101 are placed in a box. A card is selected at random.Find the probability that the card has
\begin{enumerate}[label=(\roman*)]
	\item an even number 
	\item a square number
\end{enumerate}
\solution
%\begin{table}[H]
	\centering
\begin{tabular}{|c|c|c|}
\hline
Random variable &Value &Definition\\ \hline
\multirow{3}{*}{X} &0 &Slips of Rs 1\\
&1 &Slips of Rs 5\\
&2 &Slips of Rs 13\\ \hline
\multirow{2}{*}{Y} &0 &Box A\\
&1 &Box B\\\hline
\end{tabular}
\caption{}
\label{tab:Distribution}
\end{table}
See \tabref{tab:Distribution}.
\begin{align}
p_{Y}\brak{k}= \begin{cases} 
      \frac{1}{3} & {k=0} \\
      \frac{2}{3 }& {k=1} 
   \end{cases}
   \\
p_{Y|X}\brak{0|0} = \frac{19}{25}\, 
p_{Y|X}\brak{0|1} = \frac{6}{25}\,
p_{Y|X}\brak{1|0} = \frac{45}{50}\,
p_{Y|X}\brak{1|2} = \frac{5}{50}
\end{align}
The desired probability is the probability that a slip drawn at random is marked other than Rs 1,
\begin{align}
&=1-p_X\brak{0}\\
&= p_X(1) + p_X(2)
\end{align}
Using Bayes theorem,
\begin{align}
&= p_Y\brak{0} \times \pr{Y=0 | X=1} + p_Y\brak{1} \times \pr{Y=1|X=2}\\
&=\frac{1}{3} \times \frac{6}{25} + \frac{2}{3} \times \frac{5}{50}\\
&=\frac{11}{75}
\end{align}

\newpage

%\tableofcontents

\bigskip

\renewcommand{\thefigure}{\theenumi}
\renewcommand{\thetable}{\theenumi}
%\renewcommand{\theequation}{\theenumi}

%\begin{abstract}
%%\boldmath
%In this letter, an algorithm for evaluating the exact analytical bit error rate  (BER)  for the piecewise linear (PL) combiner for  multiple relays is presented. Previous results were available only for upto three relays. The algorithm is unique in the sense that  the actual mathematical expressions, that are prohibitively large, need not be explicitly obtained. The diversity gain due to multiple relays is shown through plots of the analytical BER, well supported by simulations. 
%
%\end{abstract}
% IEEEtran.cls defaults to using nonbold math in the Abstract.
% This preserves the distinction between vectors and scalars. However,
% if the journal you are submitting to favors bold math in the abstract,
% then you can use LaTeX's standard command \boldmath at the very start
% of the abstract to achieve this. Many IEEE journals frown on math
% in the abstract anyway.

% Note that keywords are not normally used for peerreview papers.
%\begin{IEEEkeywords}
%Cooperative diversity, decode and forward, piecewise linear
%\end{IEEEkeywords}



% For peer review papers, you can put extra information on the cover
% page as needed:
% \ifCLASSOPTIONpeerreview
% \begin{center} \bfseries EDICS Category: 3-BBND \end{center}
% \fi
%
% For peerreview papers, this IEEEtran command inserts a page break and
% creates the second title. It will be ignored for other modes.
%\IEEEpeerreviewmaketitle




\item
The king, queen and jack of clubs are removed from a deck of 52 playing cards and then well shuffled. Now one card is drawn at random from the remaining cards.  Determine the probability that the card is
\begin{enumerate}[label=(\roman*)]
\item a club
\item 10 of hearts
\end{enumerate}
\solution
%\begin{table}[H]
	\centering
\begin{tabular}{|c|c|c|}
\hline
Random variable &Value &Definition\\ \hline
\multirow{3}{*}{X} &0 &Slips of Rs 1\\
&1 &Slips of Rs 5\\
&2 &Slips of Rs 13\\ \hline
\multirow{2}{*}{Y} &0 &Box A\\
&1 &Box B\\\hline
\end{tabular}
\caption{}
\label{tab:Distribution}
\end{table}
See \tabref{tab:Distribution}.
\begin{align}
p_{Y}\brak{k}= \begin{cases} 
      \frac{1}{3} & {k=0} \\
      \frac{2}{3 }& {k=1} 
   \end{cases}
   \\
p_{Y|X}\brak{0|0} = \frac{19}{25}\, 
p_{Y|X}\brak{0|1} = \frac{6}{25}\,
p_{Y|X}\brak{1|0} = \frac{45}{50}\,
p_{Y|X}\brak{1|2} = \frac{5}{50}
\end{align}
The desired probability is the probability that a slip drawn at random is marked other than Rs 1,
\begin{align}
&=1-p_X\brak{0}\\
&= p_X(1) + p_X(2)
\end{align}
Using Bayes theorem,
\begin{align}
&= p_Y\brak{0} \times \pr{Y=0 | X=1} + p_Y\brak{1} \times \pr{Y=1|X=2}\\
&=\frac{1}{3} \times \frac{6}{25} + \frac{2}{3} \times \frac{5}{50}\\
&=\frac{11}{75}
\end{align}

\newpage

%\tableofcontents

\bigskip

\renewcommand{\thefigure}{\theenumi}
\renewcommand{\thetable}{\theenumi}
%\renewcommand{\theequation}{\theenumi}

%\begin{abstract}
%%\boldmath
%In this letter, an algorithm for evaluating the exact analytical bit error rate  (BER)  for the piecewise linear (PL) combiner for  multiple relays is presented. Previous results were available only for upto three relays. The algorithm is unique in the sense that  the actual mathematical expressions, that are prohibitively large, need not be explicitly obtained. The diversity gain due to multiple relays is shown through plots of the analytical BER, well supported by simulations. 
%
%\end{abstract}
% IEEEtran.cls defaults to using nonbold math in the Abstract.
% This preserves the distinction between vectors and scalars. However,
% if the journal you are submitting to favors bold math in the abstract,
% then you can use LaTeX's standard command \boldmath at the very start
% of the abstract to achieve this. Many IEEE journals frown on math
% in the abstract anyway.

% Note that keywords are not normally used for peerreview papers.
%\begin{IEEEkeywords}
%Cooperative diversity, decode and forward, piecewise linear
%\end{IEEEkeywords}



% For peer review papers, you can put extra information on the cover
% page as needed:
% \ifCLASSOPTIONpeerreview
% \begin{center} \bfseries EDICS Category: 3-BBND \end{center}
% \fi
%
% For peerreview papers, this IEEEtran command inserts a page break and
% creates the second title. It will be ignored for other modes.
%\IEEEpeerreviewmaketitle




\item A team of medical students doing their internship have to assist during surgeries
at a city hospital. The probabilities of surgeries rated as very complex, complex,
routine, simple or very simple are respectively, 0.15, 0.20, 0.31, 0.26, .08. Find
the probabilities that a particular surgery will be rated
\begin{enumerate}
	\item complex or very complex;
	\item neither very complex nor very simple;
	\item routine or complex
	\item routine or simple
\end{enumerate}
\solution
%\begin{table}[H]
	\centering
\begin{tabular}{|c|c|c|}
\hline
Random variable &Value &Definition\\ \hline
\multirow{3}{*}{X} &0 &Slips of Rs 1\\
&1 &Slips of Rs 5\\
&2 &Slips of Rs 13\\ \hline
\multirow{2}{*}{Y} &0 &Box A\\
&1 &Box B\\\hline
\end{tabular}
\caption{}
\label{tab:Distribution}
\end{table}
See \tabref{tab:Distribution}.
\begin{align}
p_{Y}\brak{k}= \begin{cases} 
      \frac{1}{3} & {k=0} \\
      \frac{2}{3 }& {k=1} 
   \end{cases}
   \\
p_{Y|X}\brak{0|0} = \frac{19}{25}\, 
p_{Y|X}\brak{0|1} = \frac{6}{25}\,
p_{Y|X}\brak{1|0} = \frac{45}{50}\,
p_{Y|X}\brak{1|2} = \frac{5}{50}
\end{align}
The desired probability is the probability that a slip drawn at random is marked other than Rs 1,
\begin{align}
&=1-p_X\brak{0}\\
&= p_X(1) + p_X(2)
\end{align}
Using Bayes theorem,
\begin{align}
&= p_Y\brak{0} \times \pr{Y=0 | X=1} + p_Y\brak{1} \times \pr{Y=1|X=2}\\
&=\frac{1}{3} \times \frac{6}{25} + \frac{2}{3} \times \frac{5}{50}\\
&=\frac{11}{75}
\end{align}

\newpage

%\tableofcontents

\bigskip

\renewcommand{\thefigure}{\theenumi}
\renewcommand{\thetable}{\theenumi}
%\renewcommand{\theequation}{\theenumi}

%\begin{abstract}
%%\boldmath
%In this letter, an algorithm for evaluating the exact analytical bit error rate  (BER)  for the piecewise linear (PL) combiner for  multiple relays is presented. Previous results were available only for upto three relays. The algorithm is unique in the sense that  the actual mathematical expressions, that are prohibitively large, need not be explicitly obtained. The diversity gain due to multiple relays is shown through plots of the analytical BER, well supported by simulations. 
%
%\end{abstract}
% IEEEtran.cls defaults to using nonbold math in the Abstract.
% This preserves the distinction between vectors and scalars. However,
% if the journal you are submitting to favors bold math in the abstract,
% then you can use LaTeX's standard command \boldmath at the very start
% of the abstract to achieve this. Many IEEE journals frown on math
% in the abstract anyway.

% Note that keywords are not normally used for peerreview papers.
%\begin{IEEEkeywords}
%Cooperative diversity, decode and forward, piecewise linear
%\end{IEEEkeywords}



% For peer review papers, you can put extra information on the cover
% page as needed:
% \ifCLASSOPTIONpeerreview
% \begin{center} \bfseries EDICS Category: 3-BBND \end{center}
% \fi
%
% For peerreview papers, this IEEEtran command inserts a page break and
% creates the second title. It will be ignored for other modes.
%\IEEEpeerreviewmaketitle




\item A card is selected from a pack of 52 cards.
\begin{enumerate}[label=(\alph*)]
    \item How many points are there in the sample space?
    \item Calculate the probability that the card is an ace of spades.
    \item Calculate the probability that the card is (i) an ace and (ii) black card.
\end{enumerate}
\solution
%Let $X$ be an bernoulli rv defined as in \tabref{tab:exemplar/11/16/3/26}.  Then, 
\begin{equation}
    p =
        \frac{4}{11} 
\end{equation}
\begin{table}[H]
	\centering
	\input{exemplar/11/16/3/26/tables/Table2.tex}
	\caption{}
        \label{tab:exemplar/11/16/3/26}
\end{table}

\item The probability that a non leap year selected at random will contain 53 sundays.
\\
\solution
%\begin{table}[H]
	\centering
\begin{tabular}{|c|c|c|}
\hline
Random variable &Value &Definition\\ \hline
\multirow{3}{*}{X} &0 &Slips of Rs 1\\
&1 &Slips of Rs 5\\
&2 &Slips of Rs 13\\ \hline
\multirow{2}{*}{Y} &0 &Box A\\
&1 &Box B\\\hline
\end{tabular}
\caption{}
\label{tab:Distribution}
\end{table}
See \tabref{tab:Distribution}.
\begin{align}
p_{Y}\brak{k}= \begin{cases} 
      \frac{1}{3} & {k=0} \\
      \frac{2}{3 }& {k=1} 
   \end{cases}
   \\
p_{Y|X}\brak{0|0} = \frac{19}{25}\, 
p_{Y|X}\brak{0|1} = \frac{6}{25}\,
p_{Y|X}\brak{1|0} = \frac{45}{50}\,
p_{Y|X}\brak{1|2} = \frac{5}{50}
\end{align}
The desired probability is the probability that a slip drawn at random is marked other than Rs 1,
\begin{align}
&=1-p_X\brak{0}\\
&= p_X(1) + p_X(2)
\end{align}
Using Bayes theorem,
\begin{align}
&= p_Y\brak{0} \times \pr{Y=0 | X=1} + p_Y\brak{1} \times \pr{Y=1|X=2}\\
&=\frac{1}{3} \times \frac{6}{25} + \frac{2}{3} \times \frac{5}{50}\\
&=\frac{11}{75}
\end{align}

\newpage

%\tableofcontents

\bigskip

\renewcommand{\thefigure}{\theenumi}
\renewcommand{\thetable}{\theenumi}
%\renewcommand{\theequation}{\theenumi}

%\begin{abstract}
%%\boldmath
%In this letter, an algorithm for evaluating the exact analytical bit error rate  (BER)  for the piecewise linear (PL) combiner for  multiple relays is presented. Previous results were available only for upto three relays. The algorithm is unique in the sense that  the actual mathematical expressions, that are prohibitively large, need not be explicitly obtained. The diversity gain due to multiple relays is shown through plots of the analytical BER, well supported by simulations. 
%
%\end{abstract}
% IEEEtran.cls defaults to using nonbold math in the Abstract.
% This preserves the distinction between vectors and scalars. However,
% if the journal you are submitting to favors bold math in the abstract,
% then you can use LaTeX's standard command \boldmath at the very start
% of the abstract to achieve this. Many IEEE journals frown on math
% in the abstract anyway.

% Note that keywords are not normally used for peerreview papers.
%\begin{IEEEkeywords}
%Cooperative diversity, decode and forward, piecewise linear
%\end{IEEEkeywords}



% For peer review papers, you can put extra information on the cover
% page as needed:
% \ifCLASSOPTIONpeerreview
% \begin{center} \bfseries EDICS Category: 3-BBND \end{center}
% \fi
%
% For peerreview papers, this IEEEtran command inserts a page break and
% creates the second title. It will be ignored for other modes.
%\IEEEpeerreviewmaketitle




\item One of the four persons John, Rita, Aslam or Gurpreet will be promoted next
month. Consequently the sample space consists of four elementary outcomes
S = {John promoted, Rita promoted, Aslam promoted, Gurpreet promoted}
You are told that the chances of John’s promotion is same as that of Gurpreet,
Rita’s chances of promotion are twice as likely as Johns. Aslam’s chances are
four times that of John.
\begin{enumerate}
	\item Determine
	\begin{enumerate}
		\item P (John promoted)
		\item P (Rita promoted)
		\item P (Aslam promoted)
		\item P (Gurpreet promoted)
	\end{enumerate}
	\item If A = {John promoted or Gurpreet promoted}, find P (A).
\end{enumerate}
\solution
%\begin{table}[H]
	\centering
\begin{tabular}{|c|c|c|}
\hline
Random variable &Value &Definition\\ \hline
\multirow{3}{*}{X} &0 &Slips of Rs 1\\
&1 &Slips of Rs 5\\
&2 &Slips of Rs 13\\ \hline
\multirow{2}{*}{Y} &0 &Box A\\
&1 &Box B\\\hline
\end{tabular}
\caption{}
\label{tab:Distribution}
\end{table}
See \tabref{tab:Distribution}.
\begin{align}
p_{Y}\brak{k}= \begin{cases} 
      \frac{1}{3} & {k=0} \\
      \frac{2}{3 }& {k=1} 
   \end{cases}
   \\
p_{Y|X}\brak{0|0} = \frac{19}{25}\, 
p_{Y|X}\brak{0|1} = \frac{6}{25}\,
p_{Y|X}\brak{1|0} = \frac{45}{50}\,
p_{Y|X}\brak{1|2} = \frac{5}{50}
\end{align}
The desired probability is the probability that a slip drawn at random is marked other than Rs 1,
\begin{align}
&=1-p_X\brak{0}\\
&= p_X(1) + p_X(2)
\end{align}
Using Bayes theorem,
\begin{align}
&= p_Y\brak{0} \times \pr{Y=0 | X=1} + p_Y\brak{1} \times \pr{Y=1|X=2}\\
&=\frac{1}{3} \times \frac{6}{25} + \frac{2}{3} \times \frac{5}{50}\\
&=\frac{11}{75}
\end{align}

\newpage

%\tableofcontents

\bigskip

\renewcommand{\thefigure}{\theenumi}
\renewcommand{\thetable}{\theenumi}
%\renewcommand{\theequation}{\theenumi}

%\begin{abstract}
%%\boldmath
%In this letter, an algorithm for evaluating the exact analytical bit error rate  (BER)  for the piecewise linear (PL) combiner for  multiple relays is presented. Previous results were available only for upto three relays. The algorithm is unique in the sense that  the actual mathematical expressions, that are prohibitively large, need not be explicitly obtained. The diversity gain due to multiple relays is shown through plots of the analytical BER, well supported by simulations. 
%
%\end{abstract}
% IEEEtran.cls defaults to using nonbold math in the Abstract.
% This preserves the distinction between vectors and scalars. However,
% if the journal you are submitting to favors bold math in the abstract,
% then you can use LaTeX's standard command \boldmath at the very start
% of the abstract to achieve this. Many IEEE journals frown on math
% in the abstract anyway.

% Note that keywords are not normally used for peerreview papers.
%\begin{IEEEkeywords}
%Cooperative diversity, decode and forward, piecewise linear
%\end{IEEEkeywords}



% For peer review papers, you can put extra information on the cover
% page as needed:
% \ifCLASSOPTIONpeerreview
% \begin{center} \bfseries EDICS Category: 3-BBND \end{center}
% \fi
%
% For peerreview papers, this IEEEtran command inserts a page break and
% creates the second title. It will be ignored for other modes.
%\IEEEpeerreviewmaketitle




\item A card is drawn from a deck of 52 cards. Find the probability of getting a king or a heart or a red card.\\
\solution
%\begin{table}[H]
	\centering
\begin{tabular}{|c|c|c|}
\hline
Random variable &Value &Definition\\ \hline
\multirow{3}{*}{X} &0 &Slips of Rs 1\\
&1 &Slips of Rs 5\\
&2 &Slips of Rs 13\\ \hline
\multirow{2}{*}{Y} &0 &Box A\\
&1 &Box B\\\hline
\end{tabular}
\caption{}
\label{tab:Distribution}
\end{table}
See \tabref{tab:Distribution}.
\begin{align}
p_{Y}\brak{k}= \begin{cases} 
      \frac{1}{3} & {k=0} \\
      \frac{2}{3 }& {k=1} 
   \end{cases}
   \\
p_{Y|X}\brak{0|0} = \frac{19}{25}\, 
p_{Y|X}\brak{0|1} = \frac{6}{25}\,
p_{Y|X}\brak{1|0} = \frac{45}{50}\,
p_{Y|X}\brak{1|2} = \frac{5}{50}
\end{align}
The desired probability is the probability that a slip drawn at random is marked other than Rs 1,
\begin{align}
&=1-p_X\brak{0}\\
&= p_X(1) + p_X(2)
\end{align}
Using Bayes theorem,
\begin{align}
&= p_Y\brak{0} \times \pr{Y=0 | X=1} + p_Y\brak{1} \times \pr{Y=1|X=2}\\
&=\frac{1}{3} \times \frac{6}{25} + \frac{2}{3} \times \frac{5}{50}\\
&=\frac{11}{75}
\end{align}

\newpage

%\tableofcontents

\bigskip

\renewcommand{\thefigure}{\theenumi}
\renewcommand{\thetable}{\theenumi}
%\renewcommand{\theequation}{\theenumi}

%\begin{abstract}
%%\boldmath
%In this letter, an algorithm for evaluating the exact analytical bit error rate  (BER)  for the piecewise linear (PL) combiner for  multiple relays is presented. Previous results were available only for upto three relays. The algorithm is unique in the sense that  the actual mathematical expressions, that are prohibitively large, need not be explicitly obtained. The diversity gain due to multiple relays is shown through plots of the analytical BER, well supported by simulations. 
%
%\end{abstract}
% IEEEtran.cls defaults to using nonbold math in the Abstract.
% This preserves the distinction between vectors and scalars. However,
% if the journal you are submitting to favors bold math in the abstract,
% then you can use LaTeX's standard command \boldmath at the very start
% of the abstract to achieve this. Many IEEE journals frown on math
% in the abstract anyway.

% Note that keywords are not normally used for peerreview papers.
%\begin{IEEEkeywords}
%Cooperative diversity, decode and forward, piecewise linear
%\end{IEEEkeywords}



% For peer review papers, you can put extra information on the cover
% page as needed:
% \ifCLASSOPTIONpeerreview
% \begin{center} \bfseries EDICS Category: 3-BBND \end{center}
% \fi
%
% For peerreview papers, this IEEEtran command inserts a page break and
% creates the second title. It will be ignored for other modes.
%\IEEEpeerreviewmaketitle




\item The probability that a student will pass his examination is 0.73, the probability of
the student getting a compartment is 0.13, and the probability that the student will
either pass or get compartment is 0.96. State True or False.\\
\solution
%\begin{table}[H]
	\centering
\begin{tabular}{|c|c|c|}
\hline
Random variable &Value &Definition\\ \hline
\multirow{3}{*}{X} &0 &Slips of Rs 1\\
&1 &Slips of Rs 5\\
&2 &Slips of Rs 13\\ \hline
\multirow{2}{*}{Y} &0 &Box A\\
&1 &Box B\\\hline
\end{tabular}
\caption{}
\label{tab:Distribution}
\end{table}
See \tabref{tab:Distribution}.
\begin{align}
p_{Y}\brak{k}= \begin{cases} 
      \frac{1}{3} & {k=0} \\
      \frac{2}{3 }& {k=1} 
   \end{cases}
   \\
p_{Y|X}\brak{0|0} = \frac{19}{25}\, 
p_{Y|X}\brak{0|1} = \frac{6}{25}\,
p_{Y|X}\brak{1|0} = \frac{45}{50}\,
p_{Y|X}\brak{1|2} = \frac{5}{50}
\end{align}
The desired probability is the probability that a slip drawn at random is marked other than Rs 1,
\begin{align}
&=1-p_X\brak{0}\\
&= p_X(1) + p_X(2)
\end{align}
Using Bayes theorem,
\begin{align}
&= p_Y\brak{0} \times \pr{Y=0 | X=1} + p_Y\brak{1} \times \pr{Y=1|X=2}\\
&=\frac{1}{3} \times \frac{6}{25} + \frac{2}{3} \times \frac{5}{50}\\
&=\frac{11}{75}
\end{align}

\newpage

%\tableofcontents

\bigskip

\renewcommand{\thefigure}{\theenumi}
\renewcommand{\thetable}{\theenumi}
%\renewcommand{\theequation}{\theenumi}

%\begin{abstract}
%%\boldmath
%In this letter, an algorithm for evaluating the exact analytical bit error rate  (BER)  for the piecewise linear (PL) combiner for  multiple relays is presented. Previous results were available only for upto three relays. The algorithm is unique in the sense that  the actual mathematical expressions, that are prohibitively large, need not be explicitly obtained. The diversity gain due to multiple relays is shown through plots of the analytical BER, well supported by simulations. 
%
%\end{abstract}
% IEEEtran.cls defaults to using nonbold math in the Abstract.
% This preserves the distinction between vectors and scalars. However,
% if the journal you are submitting to favors bold math in the abstract,
% then you can use LaTeX's standard command \boldmath at the very start
% of the abstract to achieve this. Many IEEE journals frown on math
% in the abstract anyway.

% Note that keywords are not normally used for peerreview papers.
%\begin{IEEEkeywords}
%Cooperative diversity, decode and forward, piecewise linear
%\end{IEEEkeywords}



% For peer review papers, you can put extra information on the cover
% page as needed:
% \ifCLASSOPTIONpeerreview
% \begin{center} \bfseries EDICS Category: 3-BBND \end{center}
% \fi
%
% For peerreview papers, this IEEEtran command inserts a page break and
% creates the second title. It will be ignored for other modes.
%\IEEEpeerreviewmaketitle




\item A card is selected from a pack of 52 cards\\
\begin{enumerate}[label=(\alph*)]
\item How many points are there in the sample space?
\item Calculate the probability that the cards is an ace of spades.
\item Calculate the probability that the card is (i) an ace (ii)black card.\\
\end{enumerate}
%\input{ncert/11/16/3/4_1/Prob_4.tex}
\item In a non-leap year, the probability of having 53 tuesdays or 53 wednesdays is\\
\solution
%A non-leap year has a total of 365 days, and a week has 7 days.\\
So it can be expressed as 
\begin{align}
365\text{days} &=52\times 7+1 \text{day}
\end{align}
$\implies$ 52 tuesdays or wednesdays\\
Random variable X denotes the days of a week
\begin{align}
p_X\brak{k}&=\frac{1}{7}; \quad \brak{1<k<7}
\end{align}
So the probability of extra day being tuesday or wednesday is
\begin{align}
p_X\brak{3}+p_X\brak{4}&=\frac{1}{7}+\frac{1}{7}=\frac{2}{7}
\end{align}



\item There are 1000 sealed envelopes in a box, 10 of them contain a cash prize of
Rs 100 each, 100 of them contain a cash prize of Rs 50 each and 200 of them
contain a cash prize of Rs 10 each and rest do not contain any cash prize. If they
are well shuffled and an envelope is picked up out, what is the probability that it
contains no cash prize?\\
\solution
%\begin{table}[H]
	\centering
\begin{tabular}{|c|c|c|}
\hline
Random variable &Value &Definition\\ \hline
\multirow{3}{*}{X} &0 &Slips of Rs 1\\
&1 &Slips of Rs 5\\
&2 &Slips of Rs 13\\ \hline
\multirow{2}{*}{Y} &0 &Box A\\
&1 &Box B\\\hline
\end{tabular}
\caption{}
\label{tab:Distribution}
\end{table}
See \tabref{tab:Distribution}.
\begin{align}
p_{Y}\brak{k}= \begin{cases} 
      \frac{1}{3} & {k=0} \\
      \frac{2}{3 }& {k=1} 
   \end{cases}
   \\
p_{Y|X}\brak{0|0} = \frac{19}{25}\, 
p_{Y|X}\brak{0|1} = \frac{6}{25}\,
p_{Y|X}\brak{1|0} = \frac{45}{50}\,
p_{Y|X}\brak{1|2} = \frac{5}{50}
\end{align}
The desired probability is the probability that a slip drawn at random is marked other than Rs 1,
\begin{align}
&=1-p_X\brak{0}\\
&= p_X(1) + p_X(2)
\end{align}
Using Bayes theorem,
\begin{align}
&= p_Y\brak{0} \times \pr{Y=0 | X=1} + p_Y\brak{1} \times \pr{Y=1|X=2}\\
&=\frac{1}{3} \times \frac{6}{25} + \frac{2}{3} \times \frac{5}{50}\\
&=\frac{11}{75}
\end{align}

\newpage

%\tableofcontents

\bigskip

\renewcommand{\thefigure}{\theenumi}
\renewcommand{\thetable}{\theenumi}
%\renewcommand{\theequation}{\theenumi}

%\begin{abstract}
%%\boldmath
%In this letter, an algorithm for evaluating the exact analytical bit error rate  (BER)  for the piecewise linear (PL) combiner for  multiple relays is presented. Previous results were available only for upto three relays. The algorithm is unique in the sense that  the actual mathematical expressions, that are prohibitively large, need not be explicitly obtained. The diversity gain due to multiple relays is shown through plots of the analytical BER, well supported by simulations. 
%
%\end{abstract}
% IEEEtran.cls defaults to using nonbold math in the Abstract.
% This preserves the distinction between vectors and scalars. However,
% if the journal you are submitting to favors bold math in the abstract,
% then you can use LaTeX's standard command \boldmath at the very start
% of the abstract to achieve this. Many IEEE journals frown on math
% in the abstract anyway.

% Note that keywords are not normally used for peerreview papers.
%\begin{IEEEkeywords}
%Cooperative diversity, decode and forward, piecewise linear
%\end{IEEEkeywords}



% For peer review papers, you can put extra information on the cover
% page as needed:
% \ifCLASSOPTIONpeerreview
% \begin{center} \bfseries EDICS Category: 3-BBND \end{center}
% \fi
%
% For peerreview papers, this IEEEtran command inserts a page break and
% creates the second title. It will be ignored for other modes.
%\IEEEpeerreviewmaketitle




\item 
A die is thrown and a card is selected at random from a deck of 52 playing cards. The probability of getting an even number on the die and a spade card.\\
\solution
%\begin{table}[H]
	\centering
\begin{tabular}{|c|c|c|}
\hline
Random variable &Value &Definition\\ \hline
\multirow{3}{*}{X} &0 &Slips of Rs 1\\
&1 &Slips of Rs 5\\
&2 &Slips of Rs 13\\ \hline
\multirow{2}{*}{Y} &0 &Box A\\
&1 &Box B\\\hline
\end{tabular}
\caption{}
\label{tab:Distribution}
\end{table}
See \tabref{tab:Distribution}.
\begin{align}
p_{Y}\brak{k}= \begin{cases} 
      \frac{1}{3} & {k=0} \\
      \frac{2}{3 }& {k=1} 
   \end{cases}
   \\
p_{Y|X}\brak{0|0} = \frac{19}{25}\, 
p_{Y|X}\brak{0|1} = \frac{6}{25}\,
p_{Y|X}\brak{1|0} = \frac{45}{50}\,
p_{Y|X}\brak{1|2} = \frac{5}{50}
\end{align}
The desired probability is the probability that a slip drawn at random is marked other than Rs 1,
\begin{align}
&=1-p_X\brak{0}\\
&= p_X(1) + p_X(2)
\end{align}
Using Bayes theorem,
\begin{align}
&= p_Y\brak{0} \times \pr{Y=0 | X=1} + p_Y\brak{1} \times \pr{Y=1|X=2}\\
&=\frac{1}{3} \times \frac{6}{25} + \frac{2}{3} \times \frac{5}{50}\\
&=\frac{11}{75}
\end{align}

\newpage

%\tableofcontents

\bigskip

\renewcommand{\thefigure}{\theenumi}
\renewcommand{\thetable}{\theenumi}
%\renewcommand{\theequation}{\theenumi}

%\begin{abstract}
%%\boldmath
%In this letter, an algorithm for evaluating the exact analytical bit error rate  (BER)  for the piecewise linear (PL) combiner for  multiple relays is presented. Previous results were available only for upto three relays. The algorithm is unique in the sense that  the actual mathematical expressions, that are prohibitively large, need not be explicitly obtained. The diversity gain due to multiple relays is shown through plots of the analytical BER, well supported by simulations. 
%
%\end{abstract}
% IEEEtran.cls defaults to using nonbold math in the Abstract.
% This preserves the distinction between vectors and scalars. However,
% if the journal you are submitting to favors bold math in the abstract,
% then you can use LaTeX's standard command \boldmath at the very start
% of the abstract to achieve this. Many IEEE journals frown on math
% in the abstract anyway.

% Note that keywords are not normally used for peerreview papers.
%\begin{IEEEkeywords}
%Cooperative diversity, decode and forward, piecewise linear
%\end{IEEEkeywords}



% For peer review papers, you can put extra information on the cover
% page as needed:
% \ifCLASSOPTIONpeerreview
% \begin{center} \bfseries EDICS Category: 3-BBND \end{center}
% \fi
%
% For peerreview papers, this IEEEtran command inserts a page break and
% creates the second title. It will be ignored for other modes.
%\IEEEpeerreviewmaketitle




\item
If 4-digit numbers greater than 5,000 are randomly formed from the digits 0, 1, 3, 5, and 7, what is the probability of forming a number divisible by 5 when:
\begin{enumerate}
    \item The digits are repeated?
    \item The repetition of digits is not allowed?
\end{enumerate}
\solution
%\begin{table}[H]
	\centering
\begin{tabular}{|c|c|c|}
\hline
Random variable &Value &Definition\\ \hline
\multirow{3}{*}{X} &0 &Slips of Rs 1\\
&1 &Slips of Rs 5\\
&2 &Slips of Rs 13\\ \hline
\multirow{2}{*}{Y} &0 &Box A\\
&1 &Box B\\\hline
\end{tabular}
\caption{}
\label{tab:Distribution}
\end{table}
See \tabref{tab:Distribution}.
\begin{align}
p_{Y}\brak{k}= \begin{cases} 
      \frac{1}{3} & {k=0} \\
      \frac{2}{3 }& {k=1} 
   \end{cases}
   \\
p_{Y|X}\brak{0|0} = \frac{19}{25}\, 
p_{Y|X}\brak{0|1} = \frac{6}{25}\,
p_{Y|X}\brak{1|0} = \frac{45}{50}\,
p_{Y|X}\brak{1|2} = \frac{5}{50}
\end{align}
The desired probability is the probability that a slip drawn at random is marked other than Rs 1,
\begin{align}
&=1-p_X\brak{0}\\
&= p_X(1) + p_X(2)
\end{align}
Using Bayes theorem,
\begin{align}
&= p_Y\brak{0} \times \pr{Y=0 | X=1} + p_Y\brak{1} \times \pr{Y=1|X=2}\\
&=\frac{1}{3} \times \frac{6}{25} + \frac{2}{3} \times \frac{5}{50}\\
&=\frac{11}{75}
\end{align}

\newpage

%\tableofcontents

\bigskip

\renewcommand{\thefigure}{\theenumi}
\renewcommand{\thetable}{\theenumi}
%\renewcommand{\theequation}{\theenumi}

%\begin{abstract}
%%\boldmath
%In this letter, an algorithm for evaluating the exact analytical bit error rate  (BER)  for the piecewise linear (PL) combiner for  multiple relays is presented. Previous results were available only for upto three relays. The algorithm is unique in the sense that  the actual mathematical expressions, that are prohibitively large, need not be explicitly obtained. The diversity gain due to multiple relays is shown through plots of the analytical BER, well supported by simulations. 
%
%\end{abstract}
% IEEEtran.cls defaults to using nonbold math in the Abstract.
% This preserves the distinction between vectors and scalars. However,
% if the journal you are submitting to favors bold math in the abstract,
% then you can use LaTeX's standard command \boldmath at the very start
% of the abstract to achieve this. Many IEEE journals frown on math
% in the abstract anyway.

% Note that keywords are not normally used for peerreview papers.
%\begin{IEEEkeywords}
%Cooperative diversity, decode and forward, piecewise linear
%\end{IEEEkeywords}



% For peer review papers, you can put extra information on the cover
% page as needed:
% \ifCLASSOPTIONpeerreview
% \begin{center} \bfseries EDICS Category: 3-BBND \end{center}
% \fi
%
% For peerreview papers, this IEEEtran command inserts a page break and
% creates the second title. It will be ignored for other modes.
%\IEEEpeerreviewmaketitle




\item Consider the probability space $\brak{\Omega, \mathcal{G}, P}$ where $\Omega = [0,2]$ and $\mathcal{G} = \cbrak{\phi, \Omega, [0,1], (1,2]}$. Let $X$ and $Y$ be two functions on $\Omega$ defined as
\begin{align*}
    X(\omega) = 
    \begin{cases}
        1 & \text{if }\omega \in [0, 1]\\
        2 & \text{if }\omega \in (1, 2]
    \end{cases}
\end{align*}
and
\begin{align*}
    Y(\omega) = 
    \begin{cases}
        2 & \text{if }\omega \in [0, 1.5]\\
        3 & \text{if }\omega \in (1.5, 2].
    \end{cases}
\end{align*}
Then which one of the following statements is true?
\begin{enumerate}
    \item [(A)] $X$ is a random variable with respect to $\mathcal{G}$, but $Y$ is not a random variable with respect to $\mathcal{G}$.
    \item [(B)] $Y$ is a random variable with respect to $\mathcal{G}$, but $X$ is not a random variable with respect to $\mathcal{G}$.
    \item [(C)] Neither $X$ nor $Y$ is a random variable with respect to $\mathcal{G}$.
    \item [(D)] Both $X$ and $Y$ are random variables with respect to $\mathcal{G}$.
\end{enumerate} \hfill (GATE ST 2023)\\
\solution
%\begin{table}[H]
	\centering
\begin{tabular}{|c|c|c|}
\hline
Random variable &Value &Definition\\ \hline
\multirow{3}{*}{X} &0 &Slips of Rs 1\\
&1 &Slips of Rs 5\\
&2 &Slips of Rs 13\\ \hline
\multirow{2}{*}{Y} &0 &Box A\\
&1 &Box B\\\hline
\end{tabular}
\caption{}
\label{tab:Distribution}
\end{table}
See \tabref{tab:Distribution}.
\begin{align}
p_{Y}\brak{k}= \begin{cases} 
      \frac{1}{3} & {k=0} \\
      \frac{2}{3 }& {k=1} 
   \end{cases}
   \\
p_{Y|X}\brak{0|0} = \frac{19}{25}\, 
p_{Y|X}\brak{0|1} = \frac{6}{25}\,
p_{Y|X}\brak{1|0} = \frac{45}{50}\,
p_{Y|X}\brak{1|2} = \frac{5}{50}
\end{align}
The desired probability is the probability that a slip drawn at random is marked other than Rs 1,
\begin{align}
&=1-p_X\brak{0}\\
&= p_X(1) + p_X(2)
\end{align}
Using Bayes theorem,
\begin{align}
&= p_Y\brak{0} \times \pr{Y=0 | X=1} + p_Y\brak{1} \times \pr{Y=1|X=2}\\
&=\frac{1}{3} \times \frac{6}{25} + \frac{2}{3} \times \frac{5}{50}\\
&=\frac{11}{75}
\end{align}

\newpage

%\tableofcontents

\bigskip

\renewcommand{\thefigure}{\theenumi}
\renewcommand{\thetable}{\theenumi}
%\renewcommand{\theequation}{\theenumi}

%\begin{abstract}
%%\boldmath
%In this letter, an algorithm for evaluating the exact analytical bit error rate  (BER)  for the piecewise linear (PL) combiner for  multiple relays is presented. Previous results were available only for upto three relays. The algorithm is unique in the sense that  the actual mathematical expressions, that are prohibitively large, need not be explicitly obtained. The diversity gain due to multiple relays is shown through plots of the analytical BER, well supported by simulations. 
%
%\end{abstract}
% IEEEtran.cls defaults to using nonbold math in the Abstract.
% This preserves the distinction between vectors and scalars. However,
% if the journal you are submitting to favors bold math in the abstract,
% then you can use LaTeX's standard command \boldmath at the very start
% of the abstract to achieve this. Many IEEE journals frown on math
% in the abstract anyway.

% Note that keywords are not normally used for peerreview papers.
%\begin{IEEEkeywords}
%Cooperative diversity, decode and forward, piecewise linear
%\end{IEEEkeywords}



% For peer review papers, you can put extra information on the cover
% page as needed:
% \ifCLASSOPTIONpeerreview
% \begin{center} \bfseries EDICS Category: 3-BBND \end{center}
% \fi
%
% For peerreview papers, this IEEEtran command inserts a page break and
% creates the second title. It will be ignored for other modes.
%\IEEEpeerreviewmaketitle




	\item  A die is loaded in such a way that each odd number is twice as likely to occur as
each even number. Find $P(G)$, where $G$ is the event that a number greater than
3 occurs on a single roll of the die.
\\
\solution
		%\begin{table}[H]
	\centering
\begin{tabular}{|c|c|c|}
\hline
Random variable &Value &Definition\\ \hline
\multirow{3}{*}{X} &0 &Slips of Rs 1\\
&1 &Slips of Rs 5\\
&2 &Slips of Rs 13\\ \hline
\multirow{2}{*}{Y} &0 &Box A\\
&1 &Box B\\\hline
\end{tabular}
\caption{}
\label{tab:Distribution}
\end{table}
See \tabref{tab:Distribution}.
\begin{align}
p_{Y}\brak{k}= \begin{cases} 
      \frac{1}{3} & {k=0} \\
      \frac{2}{3 }& {k=1} 
   \end{cases}
   \\
p_{Y|X}\brak{0|0} = \frac{19}{25}\, 
p_{Y|X}\brak{0|1} = \frac{6}{25}\,
p_{Y|X}\brak{1|0} = \frac{45}{50}\,
p_{Y|X}\brak{1|2} = \frac{5}{50}
\end{align}
The desired probability is the probability that a slip drawn at random is marked other than Rs 1,
\begin{align}
&=1-p_X\brak{0}\\
&= p_X(1) + p_X(2)
\end{align}
Using Bayes theorem,
\begin{align}
&= p_Y\brak{0} \times \pr{Y=0 | X=1} + p_Y\brak{1} \times \pr{Y=1|X=2}\\
&=\frac{1}{3} \times \frac{6}{25} + \frac{2}{3} \times \frac{5}{50}\\
&=\frac{11}{75}
\end{align}

\newpage

%\tableofcontents

\bigskip

\renewcommand{\thefigure}{\theenumi}
\renewcommand{\thetable}{\theenumi}
%\renewcommand{\theequation}{\theenumi}

%\begin{abstract}
%%\boldmath
%In this letter, an algorithm for evaluating the exact analytical bit error rate  (BER)  for the piecewise linear (PL) combiner for  multiple relays is presented. Previous results were available only for upto three relays. The algorithm is unique in the sense that  the actual mathematical expressions, that are prohibitively large, need not be explicitly obtained. The diversity gain due to multiple relays is shown through plots of the analytical BER, well supported by simulations. 
%
%\end{abstract}
% IEEEtran.cls defaults to using nonbold math in the Abstract.
% This preserves the distinction between vectors and scalars. However,
% if the journal you are submitting to favors bold math in the abstract,
% then you can use LaTeX's standard command \boldmath at the very start
% of the abstract to achieve this. Many IEEE journals frown on math
% in the abstract anyway.

% Note that keywords are not normally used for peerreview papers.
%\begin{IEEEkeywords}
%Cooperative diversity, decode and forward, piecewise linear
%\end{IEEEkeywords}



% For peer review papers, you can put extra information on the cover
% page as needed:
% \ifCLASSOPTIONpeerreview
% \begin{center} \bfseries EDICS Category: 3-BBND \end{center}
% \fi
%
% For peerreview papers, this IEEEtran command inserts a page break and
% creates the second title. It will be ignored for other modes.
%\IEEEpeerreviewmaketitle




	\item All the jacks, queens and kings are removed from a deck of 52 playing cards. The remaining cards are well shuffled and then one card is drawn at random. Giving ace a value 1 similar value for other cards, find the probability that the card has a value 
		\begin{enumerate}
			\item 7
			\item greater than 7
			\item less than 7
		\end{enumerate}
		%Number of cards left after removing all jacks, queens and kings 
\begin{align}
N	= 52 - 4\times 3
	= 40
\end{align}
%\begin{table}[H]
%\def\arraystretch{1.2}
%\begin{tabular}{|c|c|c|}
%\hline
%	\textbf{Parameter} &\textbf{Value} &\textbf{Description}\\ \hline
%	$X$ &1-10 &Represents the value of the card picked \\ \hline
%\end{tabular}
%\end{table}
Let $1 \le X \le 10$ be the value of the card picked.  Then,
\begin{align}
	p_X(k) &= \Pr(X=k)\ \forall\ 1 \leq k \leq 10\\
	&= \frac{4\times 1}{40}\\
	&= \frac{1}{10}\\
	\therefore p_X(k) &= 
	\begin{cases}
		\frac{1}{10} & 1 \leq k \leq 10\\
		0 & \text{otherwise}
	\end{cases}
\end{align}
and
\begin{align}
	F_{X}(k) &= \sum_{m=0}^{k}p_{X}(m) \quad 1 \leq k \leq 10\\
	&= \frac{k}{10}\\
	\therefore F_{X}(k) &= 
	\begin{cases}
		0 & k \leq 0\\
		\frac{k}{10} & 1\leq k \leq 10\\
		1 & k > 10 
	\end{cases}
\end{align}
\begin{enumerate}
	\item Probability that card has value equal to 7 is
		\begin{align}
			 p_{X}(7)
			= \frac{1}{10}
		\end{align}
	\item Probability that card has value greater than 7 is
		\begin{align}
			1 - F_X(7)
			&= 1 - \frac{7}{10}
			\\
			&= \frac{3}{10}
		\end{align}
	\item Probability that card has value less than 7 is
		\begin{align}
			 F_{X}(6)
			=\frac{6}{10}
		\end{align}
\end{enumerate}

  \item A Lot consists of 48 mobile phones of which 42 are good, 3 have only minor defects and 3 have major defects.Varnika will buy a phone if it is good but the trader will only buy a mobile if it has no major defects. One phone is selected at random from the lot. What is the probability that it is
\begin{enumerate}
	\item acceptable to Varnika?
            \item acceptable to the trader?
\end{enumerate}
\solution
	%\begin{table}[H]
	\centering
\begin{tabular}{|c|c|c|}
\hline
Random variable &Value &Definition\\ \hline
\multirow{3}{*}{X} &0 &Slips of Rs 1\\
&1 &Slips of Rs 5\\
&2 &Slips of Rs 13\\ \hline
\multirow{2}{*}{Y} &0 &Box A\\
&1 &Box B\\\hline
\end{tabular}
\caption{}
\label{tab:Distribution}
\end{table}
See \tabref{tab:Distribution}.
\begin{align}
p_{Y}\brak{k}= \begin{cases} 
      \frac{1}{3} & {k=0} \\
      \frac{2}{3 }& {k=1} 
   \end{cases}
   \\
p_{Y|X}\brak{0|0} = \frac{19}{25}\, 
p_{Y|X}\brak{0|1} = \frac{6}{25}\,
p_{Y|X}\brak{1|0} = \frac{45}{50}\,
p_{Y|X}\brak{1|2} = \frac{5}{50}
\end{align}
The desired probability is the probability that a slip drawn at random is marked other than Rs 1,
\begin{align}
&=1-p_X\brak{0}\\
&= p_X(1) + p_X(2)
\end{align}
Using Bayes theorem,
\begin{align}
&= p_Y\brak{0} \times \pr{Y=0 | X=1} + p_Y\brak{1} \times \pr{Y=1|X=2}\\
&=\frac{1}{3} \times \frac{6}{25} + \frac{2}{3} \times \frac{5}{50}\\
&=\frac{11}{75}
\end{align}

\newpage

%\tableofcontents

\bigskip

\renewcommand{\thefigure}{\theenumi}
\renewcommand{\thetable}{\theenumi}
%\renewcommand{\theequation}{\theenumi}

%\begin{abstract}
%%\boldmath
%In this letter, an algorithm for evaluating the exact analytical bit error rate  (BER)  for the piecewise linear (PL) combiner for  multiple relays is presented. Previous results were available only for upto three relays. The algorithm is unique in the sense that  the actual mathematical expressions, that are prohibitively large, need not be explicitly obtained. The diversity gain due to multiple relays is shown through plots of the analytical BER, well supported by simulations. 
%
%\end{abstract}
% IEEEtran.cls defaults to using nonbold math in the Abstract.
% This preserves the distinction between vectors and scalars. However,
% if the journal you are submitting to favors bold math in the abstract,
% then you can use LaTeX's standard command \boldmath at the very start
% of the abstract to achieve this. Many IEEE journals frown on math
% in the abstract anyway.

% Note that keywords are not normally used for peerreview papers.
%\begin{IEEEkeywords}
%Cooperative diversity, decode and forward, piecewise linear
%\end{IEEEkeywords}



% For peer review papers, you can put extra information on the cover
% page as needed:
% \ifCLASSOPTIONpeerreview
% \begin{center} \bfseries EDICS Category: 3-BBND \end{center}
% \fi
%
% For peerreview papers, this IEEEtran command inserts a page break and
% creates the second title. It will be ignored for other modes.
%\IEEEpeerreviewmaketitle




 \item A student says that if you throw a die, it will show up 1 or not 1. Therefore, the probability of getting 1 and the probability of getting 'not 1' each is equal to $\frac{1}{2}$. Is this correct? Give reasons.\\
 \solution
        %\begin{table}[H]
	\centering
\begin{tabular}{|c|c|c|}
\hline
Random variable &Value &Definition\\ \hline
\multirow{3}{*}{X} &0 &Slips of Rs 1\\
&1 &Slips of Rs 5\\
&2 &Slips of Rs 13\\ \hline
\multirow{2}{*}{Y} &0 &Box A\\
&1 &Box B\\\hline
\end{tabular}
\caption{}
\label{tab:Distribution}
\end{table}
See \tabref{tab:Distribution}.
\begin{align}
p_{Y}\brak{k}= \begin{cases} 
      \frac{1}{3} & {k=0} \\
      \frac{2}{3 }& {k=1} 
   \end{cases}
   \\
p_{Y|X}\brak{0|0} = \frac{19}{25}\, 
p_{Y|X}\brak{0|1} = \frac{6}{25}\,
p_{Y|X}\brak{1|0} = \frac{45}{50}\,
p_{Y|X}\brak{1|2} = \frac{5}{50}
\end{align}
The desired probability is the probability that a slip drawn at random is marked other than Rs 1,
\begin{align}
&=1-p_X\brak{0}\\
&= p_X(1) + p_X(2)
\end{align}
Using Bayes theorem,
\begin{align}
&= p_Y\brak{0} \times \pr{Y=0 | X=1} + p_Y\brak{1} \times \pr{Y=1|X=2}\\
&=\frac{1}{3} \times \frac{6}{25} + \frac{2}{3} \times \frac{5}{50}\\
&=\frac{11}{75}
\end{align}

\newpage

%\tableofcontents

\bigskip

\renewcommand{\thefigure}{\theenumi}
\renewcommand{\thetable}{\theenumi}
%\renewcommand{\theequation}{\theenumi}

%\begin{abstract}
%%\boldmath
%In this letter, an algorithm for evaluating the exact analytical bit error rate  (BER)  for the piecewise linear (PL) combiner for  multiple relays is presented. Previous results were available only for upto three relays. The algorithm is unique in the sense that  the actual mathematical expressions, that are prohibitively large, need not be explicitly obtained. The diversity gain due to multiple relays is shown through plots of the analytical BER, well supported by simulations. 
%
%\end{abstract}
% IEEEtran.cls defaults to using nonbold math in the Abstract.
% This preserves the distinction between vectors and scalars. However,
% if the journal you are submitting to favors bold math in the abstract,
% then you can use LaTeX's standard command \boldmath at the very start
% of the abstract to achieve this. Many IEEE journals frown on math
% in the abstract anyway.

% Note that keywords are not normally used for peerreview papers.
%\begin{IEEEkeywords}
%Cooperative diversity, decode and forward, piecewise linear
%\end{IEEEkeywords}



% For peer review papers, you can put extra information on the cover
% page as needed:
% \ifCLASSOPTIONpeerreview
% \begin{center} \bfseries EDICS Category: 3-BBND \end{center}
% \fi
%
% For peerreview papers, this IEEEtran command inserts a page break and
% creates the second title. It will be ignored for other modes.
%\IEEEpeerreviewmaketitle




   \item Four candidates A, B, C, D have ap-
plied for the assignment to coach a school cricket
team. If A is twice as likely to be selected as B, and
B and C are given about the same chance of being
selected, while C is twice as likely to be selected
as D, what are the probabilities that
\begin{enumerate}
\item C will be selected?
\item A will not be selected?
\end{enumerate}
	%\begin{table}[H]
	\centering
\begin{tabular}{|c|c|c|}
\hline
Random variable &Value &Definition\\ \hline
\multirow{3}{*}{X} &0 &Slips of Rs 1\\
&1 &Slips of Rs 5\\
&2 &Slips of Rs 13\\ \hline
\multirow{2}{*}{Y} &0 &Box A\\
&1 &Box B\\\hline
\end{tabular}
\caption{}
\label{tab:Distribution}
\end{table}
See \tabref{tab:Distribution}.
\begin{align}
p_{Y}\brak{k}= \begin{cases} 
      \frac{1}{3} & {k=0} \\
      \frac{2}{3 }& {k=1} 
   \end{cases}
   \\
p_{Y|X}\brak{0|0} = \frac{19}{25}\, 
p_{Y|X}\brak{0|1} = \frac{6}{25}\,
p_{Y|X}\brak{1|0} = \frac{45}{50}\,
p_{Y|X}\brak{1|2} = \frac{5}{50}
\end{align}
The desired probability is the probability that a slip drawn at random is marked other than Rs 1,
\begin{align}
&=1-p_X\brak{0}\\
&= p_X(1) + p_X(2)
\end{align}
Using Bayes theorem,
\begin{align}
&= p_Y\brak{0} \times \pr{Y=0 | X=1} + p_Y\brak{1} \times \pr{Y=1|X=2}\\
&=\frac{1}{3} \times \frac{6}{25} + \frac{2}{3} \times \frac{5}{50}\\
&=\frac{11}{75}
\end{align}

\newpage

%\tableofcontents

\bigskip

\renewcommand{\thefigure}{\theenumi}
\renewcommand{\thetable}{\theenumi}
%\renewcommand{\theequation}{\theenumi}

%\begin{abstract}
%%\boldmath
%In this letter, an algorithm for evaluating the exact analytical bit error rate  (BER)  for the piecewise linear (PL) combiner for  multiple relays is presented. Previous results were available only for upto three relays. The algorithm is unique in the sense that  the actual mathematical expressions, that are prohibitively large, need not be explicitly obtained. The diversity gain due to multiple relays is shown through plots of the analytical BER, well supported by simulations. 
%
%\end{abstract}
% IEEEtran.cls defaults to using nonbold math in the Abstract.
% This preserves the distinction between vectors and scalars. However,
% if the journal you are submitting to favors bold math in the abstract,
% then you can use LaTeX's standard command \boldmath at the very start
% of the abstract to achieve this. Many IEEE journals frown on math
% in the abstract anyway.

% Note that keywords are not normally used for peerreview papers.
%\begin{IEEEkeywords}
%Cooperative diversity, decode and forward, piecewise linear
%\end{IEEEkeywords}



% For peer review papers, you can put extra information on the cover
% page as needed:
% \ifCLASSOPTIONpeerreview
% \begin{center} \bfseries EDICS Category: 3-BBND \end{center}
% \fi
%
% For peerreview papers, this IEEEtran command inserts a page break and
% creates the second title. It will be ignored for other modes.
%\IEEEpeerreviewmaketitle




 \item A bag contain 24 balls of which $x$ balls are red, $2x$ are white and $3x$ are blue. A ball is selected at random, What is the probability that it is
\begin{enumerate}[label=\alph*)]
\item not red ?
\item white ?
\end{enumerate}
%\begin{table}[H]
	\centering
\begin{tabular}{|c|c|c|}
\hline
Random variable &Value &Definition\\ \hline
\multirow{3}{*}{X} &0 &Slips of Rs 1\\
&1 &Slips of Rs 5\\
&2 &Slips of Rs 13\\ \hline
\multirow{2}{*}{Y} &0 &Box A\\
&1 &Box B\\\hline
\end{tabular}
\caption{}
\label{tab:Distribution}
\end{table}
See \tabref{tab:Distribution}.
\begin{align}
p_{Y}\brak{k}= \begin{cases} 
      \frac{1}{3} & {k=0} \\
      \frac{2}{3 }& {k=1} 
   \end{cases}
   \\
p_{Y|X}\brak{0|0} = \frac{19}{25}\, 
p_{Y|X}\brak{0|1} = \frac{6}{25}\,
p_{Y|X}\brak{1|0} = \frac{45}{50}\,
p_{Y|X}\brak{1|2} = \frac{5}{50}
\end{align}
The desired probability is the probability that a slip drawn at random is marked other than Rs 1,
\begin{align}
&=1-p_X\brak{0}\\
&= p_X(1) + p_X(2)
\end{align}
Using Bayes theorem,
\begin{align}
&= p_Y\brak{0} \times \pr{Y=0 | X=1} + p_Y\brak{1} \times \pr{Y=1|X=2}\\
&=\frac{1}{3} \times \frac{6}{25} + \frac{2}{3} \times \frac{5}{50}\\
&=\frac{11}{75}
\end{align}

\newpage

%\tableofcontents

\bigskip

\renewcommand{\thefigure}{\theenumi}
\renewcommand{\thetable}{\theenumi}
%\renewcommand{\theequation}{\theenumi}

%\begin{abstract}
%%\boldmath
%In this letter, an algorithm for evaluating the exact analytical bit error rate  (BER)  for the piecewise linear (PL) combiner for  multiple relays is presented. Previous results were available only for upto three relays. The algorithm is unique in the sense that  the actual mathematical expressions, that are prohibitively large, need not be explicitly obtained. The diversity gain due to multiple relays is shown through plots of the analytical BER, well supported by simulations. 
%
%\end{abstract}
% IEEEtran.cls defaults to using nonbold math in the Abstract.
% This preserves the distinction between vectors and scalars. However,
% if the journal you are submitting to favors bold math in the abstract,
% then you can use LaTeX's standard command \boldmath at the very start
% of the abstract to achieve this. Many IEEE journals frown on math
% in the abstract anyway.

% Note that keywords are not normally used for peerreview papers.
%\begin{IEEEkeywords}
%Cooperative diversity, decode and forward, piecewise linear
%\end{IEEEkeywords}



% For peer review papers, you can put extra information on the cover
% page as needed:
% \ifCLASSOPTIONpeerreview
% \begin{center} \bfseries EDICS Category: 3-BBND \end{center}
% \fi
%
% For peerreview papers, this IEEEtran command inserts a page break and
% creates the second title. It will be ignored for other modes.
%\IEEEpeerreviewmaketitle




If the letters of the word ASSASSINATION are arranged at random. Find the Probability that
\begin{enumerate}[label=(\alph*)]
\item Four $S's$ come consecutively in the word
\item Two  $I's$ and two $N's$ come together
\item All $A's$ are not coming together
\item No two $A's$ are coming together
\end{enumerate}
%\begin{table}[H]
	\centering
\begin{tabular}{|c|c|c|}
\hline
Random variable &Value &Definition\\ \hline
\multirow{3}{*}{X} &0 &Slips of Rs 1\\
&1 &Slips of Rs 5\\
&2 &Slips of Rs 13\\ \hline
\multirow{2}{*}{Y} &0 &Box A\\
&1 &Box B\\\hline
\end{tabular}
\caption{}
\label{tab:Distribution}
\end{table}
See \tabref{tab:Distribution}.
\begin{align}
p_{Y}\brak{k}= \begin{cases} 
      \frac{1}{3} & {k=0} \\
      \frac{2}{3 }& {k=1} 
   \end{cases}
   \\
p_{Y|X}\brak{0|0} = \frac{19}{25}\, 
p_{Y|X}\brak{0|1} = \frac{6}{25}\,
p_{Y|X}\brak{1|0} = \frac{45}{50}\,
p_{Y|X}\brak{1|2} = \frac{5}{50}
\end{align}
The desired probability is the probability that a slip drawn at random is marked other than Rs 1,
\begin{align}
&=1-p_X\brak{0}\\
&= p_X(1) + p_X(2)
\end{align}
Using Bayes theorem,
\begin{align}
&= p_Y\brak{0} \times \pr{Y=0 | X=1} + p_Y\brak{1} \times \pr{Y=1|X=2}\\
&=\frac{1}{3} \times \frac{6}{25} + \frac{2}{3} \times \frac{5}{50}\\
&=\frac{11}{75}
\end{align}

\newpage

%\tableofcontents

\bigskip

\renewcommand{\thefigure}{\theenumi}
\renewcommand{\thetable}{\theenumi}
%\renewcommand{\theequation}{\theenumi}

%\begin{abstract}
%%\boldmath
%In this letter, an algorithm for evaluating the exact analytical bit error rate  (BER)  for the piecewise linear (PL) combiner for  multiple relays is presented. Previous results were available only for upto three relays. The algorithm is unique in the sense that  the actual mathematical expressions, that are prohibitively large, need not be explicitly obtained. The diversity gain due to multiple relays is shown through plots of the analytical BER, well supported by simulations. 
%
%\end{abstract}
% IEEEtran.cls defaults to using nonbold math in the Abstract.
% This preserves the distinction between vectors and scalars. However,
% if the journal you are submitting to favors bold math in the abstract,
% then you can use LaTeX's standard command \boldmath at the very start
% of the abstract to achieve this. Many IEEE journals frown on math
% in the abstract anyway.

% Note that keywords are not normally used for peerreview papers.
%\begin{IEEEkeywords}
%Cooperative diversity, decode and forward, piecewise linear
%\end{IEEEkeywords}



% For peer review papers, you can put extra information on the cover
% page as needed:
% \ifCLASSOPTIONpeerreview
% \begin{center} \bfseries EDICS Category: 3-BBND \end{center}
% \fi
%
% For peerreview papers, this IEEEtran command inserts a page break and
% creates the second title. It will be ignored for other modes.
%\IEEEpeerreviewmaketitle




	\item One urn contains two black balls (labelled B1 and B2) and one white ball. A
	second urn contains one black ball and two white balls (labelled W1 and W2).
	Suppose the following experiment is performed. One of the two urns is chosen
	at random. Next a ball is randomly chosen from the urn. Then a second ball is
	chosen at random from the same urn without replacing the first ball.
	
	\begin{enumerate}
	\item What is the probability that two black balls are chosen?
	
	\item What is the probability that two balls of opposite colour are chosen?
	\end{enumerate}
	\solution
	%\begin{align}
    \label{eq:12.13.6.18.1}
	\because	\pr{A|B} &> \pr{A},\
\frac{\pr{AB}}{\pr{B}} > \pr{A}
\\
    \label{eq:12.13.6.18.2}
	\implies \pr{AB} &> \pr{A}\pr{B}
	\\
	\text{or, } \frac{\pr{AB}}{\pr{A}} &=\pr{B|A} > \pr{A}
\end{align}

\end{enumerate}

		%
\item 
Two cards are drawn at random and without replacement from a pack of 52 playing cards. Find the probability that both the cards are black.
\\
\solution
		%\begin{enumerate}[label=\thesection.\arabic*,ref=\thesection.\theenumi]
	\item One card is drawn from a well-shuffled deck of 52 cards. Find the probability of getting
\begin{enumerate}
\item A king of red colour 
\item A face card 
\item A red face card
\item The jack of hearts
\item A spade
\item The queen of diamonds

\end{enumerate}
\solution
		%\begin{table}[H]
	\centering
\begin{tabular}{|c|c|c|}
\hline
Random variable &Value &Definition\\ \hline
\multirow{3}{*}{X} &0 &Slips of Rs 1\\
&1 &Slips of Rs 5\\
&2 &Slips of Rs 13\\ \hline
\multirow{2}{*}{Y} &0 &Box A\\
&1 &Box B\\\hline
\end{tabular}
\caption{}
\label{tab:Distribution}
\end{table}
See \tabref{tab:Distribution}.
\begin{align}
p_{Y}\brak{k}= \begin{cases} 
      \frac{1}{3} & {k=0} \\
      \frac{2}{3 }& {k=1} 
   \end{cases}
   \\
p_{Y|X}\brak{0|0} = \frac{19}{25}\, 
p_{Y|X}\brak{0|1} = \frac{6}{25}\,
p_{Y|X}\brak{1|0} = \frac{45}{50}\,
p_{Y|X}\brak{1|2} = \frac{5}{50}
\end{align}
The desired probability is the probability that a slip drawn at random is marked other than Rs 1,
\begin{align}
&=1-p_X\brak{0}\\
&= p_X(1) + p_X(2)
\end{align}
Using Bayes theorem,
\begin{align}
&= p_Y\brak{0} \times \pr{Y=0 | X=1} + p_Y\brak{1} \times \pr{Y=1|X=2}\\
&=\frac{1}{3} \times \frac{6}{25} + \frac{2}{3} \times \frac{5}{50}\\
&=\frac{11}{75}
\end{align}

\newpage

%\tableofcontents

\bigskip

\renewcommand{\thefigure}{\theenumi}
\renewcommand{\thetable}{\theenumi}
%\renewcommand{\theequation}{\theenumi}

%\begin{abstract}
%%\boldmath
%In this letter, an algorithm for evaluating the exact analytical bit error rate  (BER)  for the piecewise linear (PL) combiner for  multiple relays is presented. Previous results were available only for upto three relays. The algorithm is unique in the sense that  the actual mathematical expressions, that are prohibitively large, need not be explicitly obtained. The diversity gain due to multiple relays is shown through plots of the analytical BER, well supported by simulations. 
%
%\end{abstract}
% IEEEtran.cls defaults to using nonbold math in the Abstract.
% This preserves the distinction between vectors and scalars. However,
% if the journal you are submitting to favors bold math in the abstract,
% then you can use LaTeX's standard command \boldmath at the very start
% of the abstract to achieve this. Many IEEE journals frown on math
% in the abstract anyway.

% Note that keywords are not normally used for peerreview papers.
%\begin{IEEEkeywords}
%Cooperative diversity, decode and forward, piecewise linear
%\end{IEEEkeywords}



% For peer review papers, you can put extra information on the cover
% page as needed:
% \ifCLASSOPTIONpeerreview
% \begin{center} \bfseries EDICS Category: 3-BBND \end{center}
% \fi
%
% For peerreview papers, this IEEEtran command inserts a page break and
% creates the second title. It will be ignored for other modes.
%\IEEEpeerreviewmaketitle




	\item Five cards—the ten, jack, queen, king and ace of diamonds, are well-shuffled with their face downwards. One card is then picked up at random.
\begin{enumerate}
\item
What is the probability that the card is the queen? 
\item
If the queen is drawn and put aside, what is the probability that the second card picked up is (a) an ace? (b) a queen?\\
\end{enumerate}
\solution
		%\begin{enumerate}[label=\thesection.\arabic*,ref=\thesection.\theenumi]
	\item One card is drawn from a well-shuffled deck of 52 cards. Find the probability of getting
\begin{enumerate}
\item A king of red colour 
\item A face card 
\item A red face card
\item The jack of hearts
\item A spade
\item The queen of diamonds

\end{enumerate}
\solution
		%\begin{table}[H]
	\centering
\begin{tabular}{|c|c|c|}
\hline
Random variable &Value &Definition\\ \hline
\multirow{3}{*}{X} &0 &Slips of Rs 1\\
&1 &Slips of Rs 5\\
&2 &Slips of Rs 13\\ \hline
\multirow{2}{*}{Y} &0 &Box A\\
&1 &Box B\\\hline
\end{tabular}
\caption{}
\label{tab:Distribution}
\end{table}
See \tabref{tab:Distribution}.
\begin{align}
p_{Y}\brak{k}= \begin{cases} 
      \frac{1}{3} & {k=0} \\
      \frac{2}{3 }& {k=1} 
   \end{cases}
   \\
p_{Y|X}\brak{0|0} = \frac{19}{25}\, 
p_{Y|X}\brak{0|1} = \frac{6}{25}\,
p_{Y|X}\brak{1|0} = \frac{45}{50}\,
p_{Y|X}\brak{1|2} = \frac{5}{50}
\end{align}
The desired probability is the probability that a slip drawn at random is marked other than Rs 1,
\begin{align}
&=1-p_X\brak{0}\\
&= p_X(1) + p_X(2)
\end{align}
Using Bayes theorem,
\begin{align}
&= p_Y\brak{0} \times \pr{Y=0 | X=1} + p_Y\brak{1} \times \pr{Y=1|X=2}\\
&=\frac{1}{3} \times \frac{6}{25} + \frac{2}{3} \times \frac{5}{50}\\
&=\frac{11}{75}
\end{align}

\newpage

%\tableofcontents

\bigskip

\renewcommand{\thefigure}{\theenumi}
\renewcommand{\thetable}{\theenumi}
%\renewcommand{\theequation}{\theenumi}

%\begin{abstract}
%%\boldmath
%In this letter, an algorithm for evaluating the exact analytical bit error rate  (BER)  for the piecewise linear (PL) combiner for  multiple relays is presented. Previous results were available only for upto three relays. The algorithm is unique in the sense that  the actual mathematical expressions, that are prohibitively large, need not be explicitly obtained. The diversity gain due to multiple relays is shown through plots of the analytical BER, well supported by simulations. 
%
%\end{abstract}
% IEEEtran.cls defaults to using nonbold math in the Abstract.
% This preserves the distinction between vectors and scalars. However,
% if the journal you are submitting to favors bold math in the abstract,
% then you can use LaTeX's standard command \boldmath at the very start
% of the abstract to achieve this. Many IEEE journals frown on math
% in the abstract anyway.

% Note that keywords are not normally used for peerreview papers.
%\begin{IEEEkeywords}
%Cooperative diversity, decode and forward, piecewise linear
%\end{IEEEkeywords}



% For peer review papers, you can put extra information on the cover
% page as needed:
% \ifCLASSOPTIONpeerreview
% \begin{center} \bfseries EDICS Category: 3-BBND \end{center}
% \fi
%
% For peerreview papers, this IEEEtran command inserts a page break and
% creates the second title. It will be ignored for other modes.
%\IEEEpeerreviewmaketitle




	\item Five cards—the ten, jack, queen, king and ace of diamonds, are well-shuffled with their face downwards. One card is then picked up at random.
\begin{enumerate}
\item
What is the probability that the card is the queen? 
\item
If the queen is drawn and put aside, what is the probability that the second card picked up is (a) an ace? (b) a queen?\\
\end{enumerate}
\solution
		%\begin{enumerate}[label=\thesection.\arabic*,ref=\thesection.\theenumi]
	\item One card is drawn from a well-shuffled deck of 52 cards. Find the probability of getting
\begin{enumerate}
\item A king of red colour 
\item A face card 
\item A red face card
\item The jack of hearts
\item A spade
\item The queen of diamonds

\end{enumerate}
\solution
		%\input{ncert/10/15/1/14/main.tex}
	\item Five cards—the ten, jack, queen, king and ace of diamonds, are well-shuffled with their face downwards. One card is then picked up at random.
\begin{enumerate}
\item
What is the probability that the card is the queen? 
\item
If the queen is drawn and put aside, what is the probability that the second card picked up is (a) an ace? (b) a queen?\\
\end{enumerate}
\solution
		%\input{ncert/10/15/1/15/defs.tex}
	\item A bag contains $5$ red balls and some blue balls. If the probability of drawing a blue ball is double that if a red ball, determine the number of blue balls in the bag. 
		\\
\solution
		%\input{ncert/10/15/2/3/defs.tex}
	\item A card is selected from a pack of 52 cards.
 \begin{enumerate}[label=(\alph*)] 
                 \item How many points are there in the sample space?
                 \item Calculate the probability that the card is an ace of spades.
                 \item Calculate the probability that the card is (i) an ace and (ii) black card.
 \end{enumerate}
\solution
		%\input{ncert/11/16/3/4/main.tex}
\item Four cards are drawn from a well-shuffled deck of 52 cards. What is the probability of obtaining 3 diamonds and one spade.
\\
\solution
		%\input{ncert/11/16/4/2/defs.tex}
\item In a certain lottery 10,000 tickets are sold and ten equal prizes are awarded. What is the probability of not getting a prize if you buy (a) one ticket (b) two tickets (c) 10 tickets ?	
\\
\solution
		%\input{ncert/11/16/4/4/defs.tex}
		%
\item 
Out of 100 students, two sections of 40 and 60 are formed. If you and your friend are among the 100 students, what is the probability that
\begin{enumerate}
\item you both enter the same section?
\item you both enter the different sections?
\end{enumerate}
\solution
		%\input{ncert/11/16/4/5/defs.tex}
	\item 
The number lock of a suitcase has 4 wheels each labelled with ten digits i.e. from 0 to 9.The lock opens with a sequence of four digits with no repeats.What is the probability of a person getting the right sequence to open the suitcase.
\\
\solution
		%\input{ncert/11/16/4/10/defs.tex}
		%
\item 
Two cards are drawn at random and without replacement from a pack of 52 playing cards. Find the probability that both the cards are black.
\\
\solution
		%\input{ncert/12/13/2/2/defs.tex}
		\item A box of oranges is inspected by examining three randomly selected oranges drawn without replacement. If all the three oranges are good, the box is approved for sale, otherwise, it is rejected. Find the probability that a box containing 15 oranges out of which 12 are good and 3 are bad ones will be approved for sale.
		\label{ncert/12/13/2/3/defs.tex}
		\item Two balls are drawn at random with replacement from a box containing 10 black and 8 red balls. Find the probability that
		\label{ncert/12/13/2/12}
\begin{enumerate}
\item both balls are red.
\item first ball is black and second is red.
\item one of them is black and other is red.
\end{enumerate}

\item In a hostel, 60\% of the students read Hindi newspaper, 40\% read English newspaper and 20\% read both Hindi and English newspapers. A student is selected at random.
		\label{ncert/12/13/2/15}
\begin{enumerate}
\item Find the probability that she reads neither Hindi nor English newspapers.
\item If she reads Hindi newspaper, find the probability that she reads English newspaper.
\item If she reads English newspaper, find the probability that she reads Hindi newspaper.\\
\end{enumerate}
\item The probability of obtaining an even prime number on each die, when a pair of dice is rolled is 
\begin{enumerate}
    \item $0$ 
    
    \item $\frac{1}{3}$ 
    
    \item $\frac{1}{12}$ 
    
    \item $\frac{1}{36}$ 
\end{enumerate}
\solution
		%\input{ncert/12/13/2/17/defs.tex}
	\item A bag contains 4 red and 4 black balls, another bag contains 2 red and 6 black balls. One of the two bags is selected at random and a ball is drawn from the bag which is found to be red. Find the probability that the ball is drawn from the first bag.
\\
\solution
		%\input{ncert/12/13/3/2/main.tex}
  \item
  Cards with numbers 2 to 101 are placed in a box. A card is selected at random.Find the probability that the card has
\begin{enumerate}[label=(\roman*)]
	\item an even number 
	\item a square number
\end{enumerate}
\solution
%\input{exemplar/10/13/3/32/main.tex}
\item
The king, queen and jack of clubs are removed from a deck of 52 playing cards and then well shuffled. Now one card is drawn at random from the remaining cards.  Determine the probability that the card is
\begin{enumerate}[label=(\roman*)]
\item a club
\item 10 of hearts
\end{enumerate}
\solution
%\input{exemplar/10/13/3/29/main.tex}
\item A team of medical students doing their internship have to assist during surgeries
at a city hospital. The probabilities of surgeries rated as very complex, complex,
routine, simple or very simple are respectively, 0.15, 0.20, 0.31, 0.26, .08. Find
the probabilities that a particular surgery will be rated
\begin{enumerate}
	\item complex or very complex;
	\item neither very complex nor very simple;
	\item routine or complex
	\item routine or simple
\end{enumerate}
\solution
%\input{exemplar/11/16/3/8(1)/main.tex}
\item A card is selected from a pack of 52 cards.
\begin{enumerate}[label=(\alph*)]
    \item How many points are there in the sample space?
    \item Calculate the probability that the card is an ace of spades.
    \item Calculate the probability that the card is (i) an ace and (ii) black card.
\end{enumerate}
\solution
%\input{exemplar/11/16/3/4/main2.tex}
\item The probability that a non leap year selected at random will contain 53 sundays.
\\
\solution
%\input{exemplar/10/13/1/19/main.tex}
\item One of the four persons John, Rita, Aslam or Gurpreet will be promoted next
month. Consequently the sample space consists of four elementary outcomes
S = {John promoted, Rita promoted, Aslam promoted, Gurpreet promoted}
You are told that the chances of John’s promotion is same as that of Gurpreet,
Rita’s chances of promotion are twice as likely as Johns. Aslam’s chances are
four times that of John.
\begin{enumerate}
	\item Determine
	\begin{enumerate}
		\item P (John promoted)
		\item P (Rita promoted)
		\item P (Aslam promoted)
		\item P (Gurpreet promoted)
	\end{enumerate}
	\item If A = {John promoted or Gurpreet promoted}, find P (A).
\end{enumerate}
\solution
%\input{exemplar/11/16/3/10/main.tex}
\item A card is drawn from a deck of 52 cards. Find the probability of getting a king or a heart or a red card.\\
\solution
%\input{exemplar/11/16/3/15/main.tex}
\item The probability that a student will pass his examination is 0.73, the probability of
the student getting a compartment is 0.13, and the probability that the student will
either pass or get compartment is 0.96. State True or False.\\
\solution
%\input{exemplar/11/16/3/31/main.tex}
\item A card is selected from a pack of 52 cards\\
\begin{enumerate}[label=(\alph*)]
\item How many points are there in the sample space?
\item Calculate the probability that the cards is an ace of spades.
\item Calculate the probability that the card is (i) an ace (ii)black card.\\
\end{enumerate}
%\input{ncert/11/16/3/4_1/Prob_4.tex}
\item In a non-leap year, the probability of having 53 tuesdays or 53 wednesdays is\\
\solution
%\input{exemplar/11/16/3/18/main.tex}
\item There are 1000 sealed envelopes in a box, 10 of them contain a cash prize of
Rs 100 each, 100 of them contain a cash prize of Rs 50 each and 200 of them
contain a cash prize of Rs 10 each and rest do not contain any cash prize. If they
are well shuffled and an envelope is picked up out, what is the probability that it
contains no cash prize?\\
\solution
%\input{exemplar/10/13/3/34/main.tex}
\item 
A die is thrown and a card is selected at random from a deck of 52 playing cards. The probability of getting an even number on the die and a spade card.\\
\solution
%\input{exemplar/12/13/3/78/main.tex}
\item
If 4-digit numbers greater than 5,000 are randomly formed from the digits 0, 1, 3, 5, and 7, what is the probability of forming a number divisible by 5 when:
\begin{enumerate}
    \item The digits are repeated?
    \item The repetition of digits is not allowed?
\end{enumerate}
\solution
%\input{ncert/11/16/4/9/main.tex}
\item Consider the probability space $\brak{\Omega, \mathcal{G}, P}$ where $\Omega = [0,2]$ and $\mathcal{G} = \cbrak{\phi, \Omega, [0,1], (1,2]}$. Let $X$ and $Y$ be two functions on $\Omega$ defined as
\begin{align*}
    X(\omega) = 
    \begin{cases}
        1 & \text{if }\omega \in [0, 1]\\
        2 & \text{if }\omega \in (1, 2]
    \end{cases}
\end{align*}
and
\begin{align*}
    Y(\omega) = 
    \begin{cases}
        2 & \text{if }\omega \in [0, 1.5]\\
        3 & \text{if }\omega \in (1.5, 2].
    \end{cases}
\end{align*}
Then which one of the following statements is true?
\begin{enumerate}
    \item [(A)] $X$ is a random variable with respect to $\mathcal{G}$, but $Y$ is not a random variable with respect to $\mathcal{G}$.
    \item [(B)] $Y$ is a random variable with respect to $\mathcal{G}$, but $X$ is not a random variable with respect to $\mathcal{G}$.
    \item [(C)] Neither $X$ nor $Y$ is a random variable with respect to $\mathcal{G}$.
    \item [(D)] Both $X$ and $Y$ are random variables with respect to $\mathcal{G}$.
\end{enumerate} \hfill (GATE ST 2023)\\
\solution
%\input{gate/ST/2023/14/main.tex}
	\item  A die is loaded in such a way that each odd number is twice as likely to occur as
each even number. Find $P(G)$, where $G$ is the event that a number greater than
3 occurs on a single roll of the die.
\\
\solution
		%\input{exemplar/11/16/3/5/main.tex}
	\item All the jacks, queens and kings are removed from a deck of 52 playing cards. The remaining cards are well shuffled and then one card is drawn at random. Giving ace a value 1 similar value for other cards, find the probability that the card has a value 
		\begin{enumerate}
			\item 7
			\item greater than 7
			\item less than 7
		\end{enumerate}
		%\input{exemplar/10/13/3/30/main.tex}
  \item A Lot consists of 48 mobile phones of which 42 are good, 3 have only minor defects and 3 have major defects.Varnika will buy a phone if it is good but the trader will only buy a mobile if it has no major defects. One phone is selected at random from the lot. What is the probability that it is
\begin{enumerate}
	\item acceptable to Varnika?
            \item acceptable to the trader?
\end{enumerate}
\solution
	%\input{exemplar/10/13/3/40/main.tex}
 \item A student says that if you throw a die, it will show up 1 or not 1. Therefore, the probability of getting 1 and the probability of getting 'not 1' each is equal to $\frac{1}{2}$. Is this correct? Give reasons.\\
 \solution
        %\input{exemplar/10/13/2/9/main.tex}
   \item Four candidates A, B, C, D have ap-
plied for the assignment to coach a school cricket
team. If A is twice as likely to be selected as B, and
B and C are given about the same chance of being
selected, while C is twice as likely to be selected
as D, what are the probabilities that
\begin{enumerate}
\item C will be selected?
\item A will not be selected?
\end{enumerate}
	%\input{exemplar/11/16/3/9/main.tex}
 \item A bag contain 24 balls of which $x$ balls are red, $2x$ are white and $3x$ are blue. A ball is selected at random, What is the probability that it is
\begin{enumerate}[label=\alph*)]
\item not red ?
\item white ?
\end{enumerate}
%\input{exemplar/10/13/3/41/main.tex}
If the letters of the word ASSASSINATION are arranged at random. Find the Probability that
\begin{enumerate}[label=(\alph*)]
\item Four $S's$ come consecutively in the word
\item Two  $I's$ and two $N's$ come together
\item All $A's$ are not coming together
\item No two $A's$ are coming together
\end{enumerate}
%\input{exemplar/11/16/3/14/main.tex}
	\item One urn contains two black balls (labelled B1 and B2) and one white ball. A
	second urn contains one black ball and two white balls (labelled W1 and W2).
	Suppose the following experiment is performed. One of the two urns is chosen
	at random. Next a ball is randomly chosen from the urn. Then a second ball is
	chosen at random from the same urn without replacing the first ball.
	
	\begin{enumerate}
	\item What is the probability that two black balls are chosen?
	
	\item What is the probability that two balls of opposite colour are chosen?
	\end{enumerate}
	\solution
	%\input{exemplar/11/16/3/12/main1.tex}
\end{enumerate}

	\item A bag contains $5$ red balls and some blue balls. If the probability of drawing a blue ball is double that if a red ball, determine the number of blue balls in the bag. 
		\\
\solution
		%\begin{enumerate}[label=\thesection.\arabic*,ref=\thesection.\theenumi]
	\item One card is drawn from a well-shuffled deck of 52 cards. Find the probability of getting
\begin{enumerate}
\item A king of red colour 
\item A face card 
\item A red face card
\item The jack of hearts
\item A spade
\item The queen of diamonds

\end{enumerate}
\solution
		%\input{ncert/10/15/1/14/main.tex}
	\item Five cards—the ten, jack, queen, king and ace of diamonds, are well-shuffled with their face downwards. One card is then picked up at random.
\begin{enumerate}
\item
What is the probability that the card is the queen? 
\item
If the queen is drawn and put aside, what is the probability that the second card picked up is (a) an ace? (b) a queen?\\
\end{enumerate}
\solution
		%\input{ncert/10/15/1/15/defs.tex}
	\item A bag contains $5$ red balls and some blue balls. If the probability of drawing a blue ball is double that if a red ball, determine the number of blue balls in the bag. 
		\\
\solution
		%\input{ncert/10/15/2/3/defs.tex}
	\item A card is selected from a pack of 52 cards.
 \begin{enumerate}[label=(\alph*)] 
                 \item How many points are there in the sample space?
                 \item Calculate the probability that the card is an ace of spades.
                 \item Calculate the probability that the card is (i) an ace and (ii) black card.
 \end{enumerate}
\solution
		%\input{ncert/11/16/3/4/main.tex}
\item Four cards are drawn from a well-shuffled deck of 52 cards. What is the probability of obtaining 3 diamonds and one spade.
\\
\solution
		%\input{ncert/11/16/4/2/defs.tex}
\item In a certain lottery 10,000 tickets are sold and ten equal prizes are awarded. What is the probability of not getting a prize if you buy (a) one ticket (b) two tickets (c) 10 tickets ?	
\\
\solution
		%\input{ncert/11/16/4/4/defs.tex}
		%
\item 
Out of 100 students, two sections of 40 and 60 are formed. If you and your friend are among the 100 students, what is the probability that
\begin{enumerate}
\item you both enter the same section?
\item you both enter the different sections?
\end{enumerate}
\solution
		%\input{ncert/11/16/4/5/defs.tex}
	\item 
The number lock of a suitcase has 4 wheels each labelled with ten digits i.e. from 0 to 9.The lock opens with a sequence of four digits with no repeats.What is the probability of a person getting the right sequence to open the suitcase.
\\
\solution
		%\input{ncert/11/16/4/10/defs.tex}
		%
\item 
Two cards are drawn at random and without replacement from a pack of 52 playing cards. Find the probability that both the cards are black.
\\
\solution
		%\input{ncert/12/13/2/2/defs.tex}
		\item A box of oranges is inspected by examining three randomly selected oranges drawn without replacement. If all the three oranges are good, the box is approved for sale, otherwise, it is rejected. Find the probability that a box containing 15 oranges out of which 12 are good and 3 are bad ones will be approved for sale.
		\label{ncert/12/13/2/3/defs.tex}
		\item Two balls are drawn at random with replacement from a box containing 10 black and 8 red balls. Find the probability that
		\label{ncert/12/13/2/12}
\begin{enumerate}
\item both balls are red.
\item first ball is black and second is red.
\item one of them is black and other is red.
\end{enumerate}

\item In a hostel, 60\% of the students read Hindi newspaper, 40\% read English newspaper and 20\% read both Hindi and English newspapers. A student is selected at random.
		\label{ncert/12/13/2/15}
\begin{enumerate}
\item Find the probability that she reads neither Hindi nor English newspapers.
\item If she reads Hindi newspaper, find the probability that she reads English newspaper.
\item If she reads English newspaper, find the probability that she reads Hindi newspaper.\\
\end{enumerate}
\item The probability of obtaining an even prime number on each die, when a pair of dice is rolled is 
\begin{enumerate}
    \item $0$ 
    
    \item $\frac{1}{3}$ 
    
    \item $\frac{1}{12}$ 
    
    \item $\frac{1}{36}$ 
\end{enumerate}
\solution
		%\input{ncert/12/13/2/17/defs.tex}
	\item A bag contains 4 red and 4 black balls, another bag contains 2 red and 6 black balls. One of the two bags is selected at random and a ball is drawn from the bag which is found to be red. Find the probability that the ball is drawn from the first bag.
\\
\solution
		%\input{ncert/12/13/3/2/main.tex}
  \item
  Cards with numbers 2 to 101 are placed in a box. A card is selected at random.Find the probability that the card has
\begin{enumerate}[label=(\roman*)]
	\item an even number 
	\item a square number
\end{enumerate}
\solution
%\input{exemplar/10/13/3/32/main.tex}
\item
The king, queen and jack of clubs are removed from a deck of 52 playing cards and then well shuffled. Now one card is drawn at random from the remaining cards.  Determine the probability that the card is
\begin{enumerate}[label=(\roman*)]
\item a club
\item 10 of hearts
\end{enumerate}
\solution
%\input{exemplar/10/13/3/29/main.tex}
\item A team of medical students doing their internship have to assist during surgeries
at a city hospital. The probabilities of surgeries rated as very complex, complex,
routine, simple or very simple are respectively, 0.15, 0.20, 0.31, 0.26, .08. Find
the probabilities that a particular surgery will be rated
\begin{enumerate}
	\item complex or very complex;
	\item neither very complex nor very simple;
	\item routine or complex
	\item routine or simple
\end{enumerate}
\solution
%\input{exemplar/11/16/3/8(1)/main.tex}
\item A card is selected from a pack of 52 cards.
\begin{enumerate}[label=(\alph*)]
    \item How many points are there in the sample space?
    \item Calculate the probability that the card is an ace of spades.
    \item Calculate the probability that the card is (i) an ace and (ii) black card.
\end{enumerate}
\solution
%\input{exemplar/11/16/3/4/main2.tex}
\item The probability that a non leap year selected at random will contain 53 sundays.
\\
\solution
%\input{exemplar/10/13/1/19/main.tex}
\item One of the four persons John, Rita, Aslam or Gurpreet will be promoted next
month. Consequently the sample space consists of four elementary outcomes
S = {John promoted, Rita promoted, Aslam promoted, Gurpreet promoted}
You are told that the chances of John’s promotion is same as that of Gurpreet,
Rita’s chances of promotion are twice as likely as Johns. Aslam’s chances are
four times that of John.
\begin{enumerate}
	\item Determine
	\begin{enumerate}
		\item P (John promoted)
		\item P (Rita promoted)
		\item P (Aslam promoted)
		\item P (Gurpreet promoted)
	\end{enumerate}
	\item If A = {John promoted or Gurpreet promoted}, find P (A).
\end{enumerate}
\solution
%\input{exemplar/11/16/3/10/main.tex}
\item A card is drawn from a deck of 52 cards. Find the probability of getting a king or a heart or a red card.\\
\solution
%\input{exemplar/11/16/3/15/main.tex}
\item The probability that a student will pass his examination is 0.73, the probability of
the student getting a compartment is 0.13, and the probability that the student will
either pass or get compartment is 0.96. State True or False.\\
\solution
%\input{exemplar/11/16/3/31/main.tex}
\item A card is selected from a pack of 52 cards\\
\begin{enumerate}[label=(\alph*)]
\item How many points are there in the sample space?
\item Calculate the probability that the cards is an ace of spades.
\item Calculate the probability that the card is (i) an ace (ii)black card.\\
\end{enumerate}
%\input{ncert/11/16/3/4_1/Prob_4.tex}
\item In a non-leap year, the probability of having 53 tuesdays or 53 wednesdays is\\
\solution
%\input{exemplar/11/16/3/18/main.tex}
\item There are 1000 sealed envelopes in a box, 10 of them contain a cash prize of
Rs 100 each, 100 of them contain a cash prize of Rs 50 each and 200 of them
contain a cash prize of Rs 10 each and rest do not contain any cash prize. If they
are well shuffled and an envelope is picked up out, what is the probability that it
contains no cash prize?\\
\solution
%\input{exemplar/10/13/3/34/main.tex}
\item 
A die is thrown and a card is selected at random from a deck of 52 playing cards. The probability of getting an even number on the die and a spade card.\\
\solution
%\input{exemplar/12/13/3/78/main.tex}
\item
If 4-digit numbers greater than 5,000 are randomly formed from the digits 0, 1, 3, 5, and 7, what is the probability of forming a number divisible by 5 when:
\begin{enumerate}
    \item The digits are repeated?
    \item The repetition of digits is not allowed?
\end{enumerate}
\solution
%\input{ncert/11/16/4/9/main.tex}
\item Consider the probability space $\brak{\Omega, \mathcal{G}, P}$ where $\Omega = [0,2]$ and $\mathcal{G} = \cbrak{\phi, \Omega, [0,1], (1,2]}$. Let $X$ and $Y$ be two functions on $\Omega$ defined as
\begin{align*}
    X(\omega) = 
    \begin{cases}
        1 & \text{if }\omega \in [0, 1]\\
        2 & \text{if }\omega \in (1, 2]
    \end{cases}
\end{align*}
and
\begin{align*}
    Y(\omega) = 
    \begin{cases}
        2 & \text{if }\omega \in [0, 1.5]\\
        3 & \text{if }\omega \in (1.5, 2].
    \end{cases}
\end{align*}
Then which one of the following statements is true?
\begin{enumerate}
    \item [(A)] $X$ is a random variable with respect to $\mathcal{G}$, but $Y$ is not a random variable with respect to $\mathcal{G}$.
    \item [(B)] $Y$ is a random variable with respect to $\mathcal{G}$, but $X$ is not a random variable with respect to $\mathcal{G}$.
    \item [(C)] Neither $X$ nor $Y$ is a random variable with respect to $\mathcal{G}$.
    \item [(D)] Both $X$ and $Y$ are random variables with respect to $\mathcal{G}$.
\end{enumerate} \hfill (GATE ST 2023)\\
\solution
%\input{gate/ST/2023/14/main.tex}
	\item  A die is loaded in such a way that each odd number is twice as likely to occur as
each even number. Find $P(G)$, where $G$ is the event that a number greater than
3 occurs on a single roll of the die.
\\
\solution
		%\input{exemplar/11/16/3/5/main.tex}
	\item All the jacks, queens and kings are removed from a deck of 52 playing cards. The remaining cards are well shuffled and then one card is drawn at random. Giving ace a value 1 similar value for other cards, find the probability that the card has a value 
		\begin{enumerate}
			\item 7
			\item greater than 7
			\item less than 7
		\end{enumerate}
		%\input{exemplar/10/13/3/30/main.tex}
  \item A Lot consists of 48 mobile phones of which 42 are good, 3 have only minor defects and 3 have major defects.Varnika will buy a phone if it is good but the trader will only buy a mobile if it has no major defects. One phone is selected at random from the lot. What is the probability that it is
\begin{enumerate}
	\item acceptable to Varnika?
            \item acceptable to the trader?
\end{enumerate}
\solution
	%\input{exemplar/10/13/3/40/main.tex}
 \item A student says that if you throw a die, it will show up 1 or not 1. Therefore, the probability of getting 1 and the probability of getting 'not 1' each is equal to $\frac{1}{2}$. Is this correct? Give reasons.\\
 \solution
        %\input{exemplar/10/13/2/9/main.tex}
   \item Four candidates A, B, C, D have ap-
plied for the assignment to coach a school cricket
team. If A is twice as likely to be selected as B, and
B and C are given about the same chance of being
selected, while C is twice as likely to be selected
as D, what are the probabilities that
\begin{enumerate}
\item C will be selected?
\item A will not be selected?
\end{enumerate}
	%\input{exemplar/11/16/3/9/main.tex}
 \item A bag contain 24 balls of which $x$ balls are red, $2x$ are white and $3x$ are blue. A ball is selected at random, What is the probability that it is
\begin{enumerate}[label=\alph*)]
\item not red ?
\item white ?
\end{enumerate}
%\input{exemplar/10/13/3/41/main.tex}
If the letters of the word ASSASSINATION are arranged at random. Find the Probability that
\begin{enumerate}[label=(\alph*)]
\item Four $S's$ come consecutively in the word
\item Two  $I's$ and two $N's$ come together
\item All $A's$ are not coming together
\item No two $A's$ are coming together
\end{enumerate}
%\input{exemplar/11/16/3/14/main.tex}
	\item One urn contains two black balls (labelled B1 and B2) and one white ball. A
	second urn contains one black ball and two white balls (labelled W1 and W2).
	Suppose the following experiment is performed. One of the two urns is chosen
	at random. Next a ball is randomly chosen from the urn. Then a second ball is
	chosen at random from the same urn without replacing the first ball.
	
	\begin{enumerate}
	\item What is the probability that two black balls are chosen?
	
	\item What is the probability that two balls of opposite colour are chosen?
	\end{enumerate}
	\solution
	%\input{exemplar/11/16/3/12/main1.tex}
\end{enumerate}

	\item A card is selected from a pack of 52 cards.
 \begin{enumerate}[label=(\alph*)] 
                 \item How many points are there in the sample space?
                 \item Calculate the probability that the card is an ace of spades.
                 \item Calculate the probability that the card is (i) an ace and (ii) black card.
 \end{enumerate}
\solution
		%\begin{table}[H]
	\centering
\begin{tabular}{|c|c|c|}
\hline
Random variable &Value &Definition\\ \hline
\multirow{3}{*}{X} &0 &Slips of Rs 1\\
&1 &Slips of Rs 5\\
&2 &Slips of Rs 13\\ \hline
\multirow{2}{*}{Y} &0 &Box A\\
&1 &Box B\\\hline
\end{tabular}
\caption{}
\label{tab:Distribution}
\end{table}
See \tabref{tab:Distribution}.
\begin{align}
p_{Y}\brak{k}= \begin{cases} 
      \frac{1}{3} & {k=0} \\
      \frac{2}{3 }& {k=1} 
   \end{cases}
   \\
p_{Y|X}\brak{0|0} = \frac{19}{25}\, 
p_{Y|X}\brak{0|1} = \frac{6}{25}\,
p_{Y|X}\brak{1|0} = \frac{45}{50}\,
p_{Y|X}\brak{1|2} = \frac{5}{50}
\end{align}
The desired probability is the probability that a slip drawn at random is marked other than Rs 1,
\begin{align}
&=1-p_X\brak{0}\\
&= p_X(1) + p_X(2)
\end{align}
Using Bayes theorem,
\begin{align}
&= p_Y\brak{0} \times \pr{Y=0 | X=1} + p_Y\brak{1} \times \pr{Y=1|X=2}\\
&=\frac{1}{3} \times \frac{6}{25} + \frac{2}{3} \times \frac{5}{50}\\
&=\frac{11}{75}
\end{align}

\newpage

%\tableofcontents

\bigskip

\renewcommand{\thefigure}{\theenumi}
\renewcommand{\thetable}{\theenumi}
%\renewcommand{\theequation}{\theenumi}

%\begin{abstract}
%%\boldmath
%In this letter, an algorithm for evaluating the exact analytical bit error rate  (BER)  for the piecewise linear (PL) combiner for  multiple relays is presented. Previous results were available only for upto three relays. The algorithm is unique in the sense that  the actual mathematical expressions, that are prohibitively large, need not be explicitly obtained. The diversity gain due to multiple relays is shown through plots of the analytical BER, well supported by simulations. 
%
%\end{abstract}
% IEEEtran.cls defaults to using nonbold math in the Abstract.
% This preserves the distinction between vectors and scalars. However,
% if the journal you are submitting to favors bold math in the abstract,
% then you can use LaTeX's standard command \boldmath at the very start
% of the abstract to achieve this. Many IEEE journals frown on math
% in the abstract anyway.

% Note that keywords are not normally used for peerreview papers.
%\begin{IEEEkeywords}
%Cooperative diversity, decode and forward, piecewise linear
%\end{IEEEkeywords}



% For peer review papers, you can put extra information on the cover
% page as needed:
% \ifCLASSOPTIONpeerreview
% \begin{center} \bfseries EDICS Category: 3-BBND \end{center}
% \fi
%
% For peerreview papers, this IEEEtran command inserts a page break and
% creates the second title. It will be ignored for other modes.
%\IEEEpeerreviewmaketitle




\item Four cards are drawn from a well-shuffled deck of 52 cards. What is the probability of obtaining 3 diamonds and one spade.
\\
\solution
		%\begin{enumerate}[label=\thesection.\arabic*,ref=\thesection.\theenumi]
	\item One card is drawn from a well-shuffled deck of 52 cards. Find the probability of getting
\begin{enumerate}
\item A king of red colour 
\item A face card 
\item A red face card
\item The jack of hearts
\item A spade
\item The queen of diamonds

\end{enumerate}
\solution
		%\input{ncert/10/15/1/14/main.tex}
	\item Five cards—the ten, jack, queen, king and ace of diamonds, are well-shuffled with their face downwards. One card is then picked up at random.
\begin{enumerate}
\item
What is the probability that the card is the queen? 
\item
If the queen is drawn and put aside, what is the probability that the second card picked up is (a) an ace? (b) a queen?\\
\end{enumerate}
\solution
		%\input{ncert/10/15/1/15/defs.tex}
	\item A bag contains $5$ red balls and some blue balls. If the probability of drawing a blue ball is double that if a red ball, determine the number of blue balls in the bag. 
		\\
\solution
		%\input{ncert/10/15/2/3/defs.tex}
	\item A card is selected from a pack of 52 cards.
 \begin{enumerate}[label=(\alph*)] 
                 \item How many points are there in the sample space?
                 \item Calculate the probability that the card is an ace of spades.
                 \item Calculate the probability that the card is (i) an ace and (ii) black card.
 \end{enumerate}
\solution
		%\input{ncert/11/16/3/4/main.tex}
\item Four cards are drawn from a well-shuffled deck of 52 cards. What is the probability of obtaining 3 diamonds and one spade.
\\
\solution
		%\input{ncert/11/16/4/2/defs.tex}
\item In a certain lottery 10,000 tickets are sold and ten equal prizes are awarded. What is the probability of not getting a prize if you buy (a) one ticket (b) two tickets (c) 10 tickets ?	
\\
\solution
		%\input{ncert/11/16/4/4/defs.tex}
		%
\item 
Out of 100 students, two sections of 40 and 60 are formed. If you and your friend are among the 100 students, what is the probability that
\begin{enumerate}
\item you both enter the same section?
\item you both enter the different sections?
\end{enumerate}
\solution
		%\input{ncert/11/16/4/5/defs.tex}
	\item 
The number lock of a suitcase has 4 wheels each labelled with ten digits i.e. from 0 to 9.The lock opens with a sequence of four digits with no repeats.What is the probability of a person getting the right sequence to open the suitcase.
\\
\solution
		%\input{ncert/11/16/4/10/defs.tex}
		%
\item 
Two cards are drawn at random and without replacement from a pack of 52 playing cards. Find the probability that both the cards are black.
\\
\solution
		%\input{ncert/12/13/2/2/defs.tex}
		\item A box of oranges is inspected by examining three randomly selected oranges drawn without replacement. If all the three oranges are good, the box is approved for sale, otherwise, it is rejected. Find the probability that a box containing 15 oranges out of which 12 are good and 3 are bad ones will be approved for sale.
		\label{ncert/12/13/2/3/defs.tex}
		\item Two balls are drawn at random with replacement from a box containing 10 black and 8 red balls. Find the probability that
		\label{ncert/12/13/2/12}
\begin{enumerate}
\item both balls are red.
\item first ball is black and second is red.
\item one of them is black and other is red.
\end{enumerate}

\item In a hostel, 60\% of the students read Hindi newspaper, 40\% read English newspaper and 20\% read both Hindi and English newspapers. A student is selected at random.
		\label{ncert/12/13/2/15}
\begin{enumerate}
\item Find the probability that she reads neither Hindi nor English newspapers.
\item If she reads Hindi newspaper, find the probability that she reads English newspaper.
\item If she reads English newspaper, find the probability that she reads Hindi newspaper.\\
\end{enumerate}
\item The probability of obtaining an even prime number on each die, when a pair of dice is rolled is 
\begin{enumerate}
    \item $0$ 
    
    \item $\frac{1}{3}$ 
    
    \item $\frac{1}{12}$ 
    
    \item $\frac{1}{36}$ 
\end{enumerate}
\solution
		%\input{ncert/12/13/2/17/defs.tex}
	\item A bag contains 4 red and 4 black balls, another bag contains 2 red and 6 black balls. One of the two bags is selected at random and a ball is drawn from the bag which is found to be red. Find the probability that the ball is drawn from the first bag.
\\
\solution
		%\input{ncert/12/13/3/2/main.tex}
  \item
  Cards with numbers 2 to 101 are placed in a box. A card is selected at random.Find the probability that the card has
\begin{enumerate}[label=(\roman*)]
	\item an even number 
	\item a square number
\end{enumerate}
\solution
%\input{exemplar/10/13/3/32/main.tex}
\item
The king, queen and jack of clubs are removed from a deck of 52 playing cards and then well shuffled. Now one card is drawn at random from the remaining cards.  Determine the probability that the card is
\begin{enumerate}[label=(\roman*)]
\item a club
\item 10 of hearts
\end{enumerate}
\solution
%\input{exemplar/10/13/3/29/main.tex}
\item A team of medical students doing their internship have to assist during surgeries
at a city hospital. The probabilities of surgeries rated as very complex, complex,
routine, simple or very simple are respectively, 0.15, 0.20, 0.31, 0.26, .08. Find
the probabilities that a particular surgery will be rated
\begin{enumerate}
	\item complex or very complex;
	\item neither very complex nor very simple;
	\item routine or complex
	\item routine or simple
\end{enumerate}
\solution
%\input{exemplar/11/16/3/8(1)/main.tex}
\item A card is selected from a pack of 52 cards.
\begin{enumerate}[label=(\alph*)]
    \item How many points are there in the sample space?
    \item Calculate the probability that the card is an ace of spades.
    \item Calculate the probability that the card is (i) an ace and (ii) black card.
\end{enumerate}
\solution
%\input{exemplar/11/16/3/4/main2.tex}
\item The probability that a non leap year selected at random will contain 53 sundays.
\\
\solution
%\input{exemplar/10/13/1/19/main.tex}
\item One of the four persons John, Rita, Aslam or Gurpreet will be promoted next
month. Consequently the sample space consists of four elementary outcomes
S = {John promoted, Rita promoted, Aslam promoted, Gurpreet promoted}
You are told that the chances of John’s promotion is same as that of Gurpreet,
Rita’s chances of promotion are twice as likely as Johns. Aslam’s chances are
four times that of John.
\begin{enumerate}
	\item Determine
	\begin{enumerate}
		\item P (John promoted)
		\item P (Rita promoted)
		\item P (Aslam promoted)
		\item P (Gurpreet promoted)
	\end{enumerate}
	\item If A = {John promoted or Gurpreet promoted}, find P (A).
\end{enumerate}
\solution
%\input{exemplar/11/16/3/10/main.tex}
\item A card is drawn from a deck of 52 cards. Find the probability of getting a king or a heart or a red card.\\
\solution
%\input{exemplar/11/16/3/15/main.tex}
\item The probability that a student will pass his examination is 0.73, the probability of
the student getting a compartment is 0.13, and the probability that the student will
either pass or get compartment is 0.96. State True or False.\\
\solution
%\input{exemplar/11/16/3/31/main.tex}
\item A card is selected from a pack of 52 cards\\
\begin{enumerate}[label=(\alph*)]
\item How many points are there in the sample space?
\item Calculate the probability that the cards is an ace of spades.
\item Calculate the probability that the card is (i) an ace (ii)black card.\\
\end{enumerate}
%\input{ncert/11/16/3/4_1/Prob_4.tex}
\item In a non-leap year, the probability of having 53 tuesdays or 53 wednesdays is\\
\solution
%\input{exemplar/11/16/3/18/main.tex}
\item There are 1000 sealed envelopes in a box, 10 of them contain a cash prize of
Rs 100 each, 100 of them contain a cash prize of Rs 50 each and 200 of them
contain a cash prize of Rs 10 each and rest do not contain any cash prize. If they
are well shuffled and an envelope is picked up out, what is the probability that it
contains no cash prize?\\
\solution
%\input{exemplar/10/13/3/34/main.tex}
\item 
A die is thrown and a card is selected at random from a deck of 52 playing cards. The probability of getting an even number on the die and a spade card.\\
\solution
%\input{exemplar/12/13/3/78/main.tex}
\item
If 4-digit numbers greater than 5,000 are randomly formed from the digits 0, 1, 3, 5, and 7, what is the probability of forming a number divisible by 5 when:
\begin{enumerate}
    \item The digits are repeated?
    \item The repetition of digits is not allowed?
\end{enumerate}
\solution
%\input{ncert/11/16/4/9/main.tex}
\item Consider the probability space $\brak{\Omega, \mathcal{G}, P}$ where $\Omega = [0,2]$ and $\mathcal{G} = \cbrak{\phi, \Omega, [0,1], (1,2]}$. Let $X$ and $Y$ be two functions on $\Omega$ defined as
\begin{align*}
    X(\omega) = 
    \begin{cases}
        1 & \text{if }\omega \in [0, 1]\\
        2 & \text{if }\omega \in (1, 2]
    \end{cases}
\end{align*}
and
\begin{align*}
    Y(\omega) = 
    \begin{cases}
        2 & \text{if }\omega \in [0, 1.5]\\
        3 & \text{if }\omega \in (1.5, 2].
    \end{cases}
\end{align*}
Then which one of the following statements is true?
\begin{enumerate}
    \item [(A)] $X$ is a random variable with respect to $\mathcal{G}$, but $Y$ is not a random variable with respect to $\mathcal{G}$.
    \item [(B)] $Y$ is a random variable with respect to $\mathcal{G}$, but $X$ is not a random variable with respect to $\mathcal{G}$.
    \item [(C)] Neither $X$ nor $Y$ is a random variable with respect to $\mathcal{G}$.
    \item [(D)] Both $X$ and $Y$ are random variables with respect to $\mathcal{G}$.
\end{enumerate} \hfill (GATE ST 2023)\\
\solution
%\input{gate/ST/2023/14/main.tex}
	\item  A die is loaded in such a way that each odd number is twice as likely to occur as
each even number. Find $P(G)$, where $G$ is the event that a number greater than
3 occurs on a single roll of the die.
\\
\solution
		%\input{exemplar/11/16/3/5/main.tex}
	\item All the jacks, queens and kings are removed from a deck of 52 playing cards. The remaining cards are well shuffled and then one card is drawn at random. Giving ace a value 1 similar value for other cards, find the probability that the card has a value 
		\begin{enumerate}
			\item 7
			\item greater than 7
			\item less than 7
		\end{enumerate}
		%\input{exemplar/10/13/3/30/main.tex}
  \item A Lot consists of 48 mobile phones of which 42 are good, 3 have only minor defects and 3 have major defects.Varnika will buy a phone if it is good but the trader will only buy a mobile if it has no major defects. One phone is selected at random from the lot. What is the probability that it is
\begin{enumerate}
	\item acceptable to Varnika?
            \item acceptable to the trader?
\end{enumerate}
\solution
	%\input{exemplar/10/13/3/40/main.tex}
 \item A student says that if you throw a die, it will show up 1 or not 1. Therefore, the probability of getting 1 and the probability of getting 'not 1' each is equal to $\frac{1}{2}$. Is this correct? Give reasons.\\
 \solution
        %\input{exemplar/10/13/2/9/main.tex}
   \item Four candidates A, B, C, D have ap-
plied for the assignment to coach a school cricket
team. If A is twice as likely to be selected as B, and
B and C are given about the same chance of being
selected, while C is twice as likely to be selected
as D, what are the probabilities that
\begin{enumerate}
\item C will be selected?
\item A will not be selected?
\end{enumerate}
	%\input{exemplar/11/16/3/9/main.tex}
 \item A bag contain 24 balls of which $x$ balls are red, $2x$ are white and $3x$ are blue. A ball is selected at random, What is the probability that it is
\begin{enumerate}[label=\alph*)]
\item not red ?
\item white ?
\end{enumerate}
%\input{exemplar/10/13/3/41/main.tex}
If the letters of the word ASSASSINATION are arranged at random. Find the Probability that
\begin{enumerate}[label=(\alph*)]
\item Four $S's$ come consecutively in the word
\item Two  $I's$ and two $N's$ come together
\item All $A's$ are not coming together
\item No two $A's$ are coming together
\end{enumerate}
%\input{exemplar/11/16/3/14/main.tex}
	\item One urn contains two black balls (labelled B1 and B2) and one white ball. A
	second urn contains one black ball and two white balls (labelled W1 and W2).
	Suppose the following experiment is performed. One of the two urns is chosen
	at random. Next a ball is randomly chosen from the urn. Then a second ball is
	chosen at random from the same urn without replacing the first ball.
	
	\begin{enumerate}
	\item What is the probability that two black balls are chosen?
	
	\item What is the probability that two balls of opposite colour are chosen?
	\end{enumerate}
	\solution
	%\input{exemplar/11/16/3/12/main1.tex}
\end{enumerate}

\item In a certain lottery 10,000 tickets are sold and ten equal prizes are awarded. What is the probability of not getting a prize if you buy (a) one ticket (b) two tickets (c) 10 tickets ?	
\\
\solution
		%\begin{enumerate}[label=\thesection.\arabic*,ref=\thesection.\theenumi]
	\item One card is drawn from a well-shuffled deck of 52 cards. Find the probability of getting
\begin{enumerate}
\item A king of red colour 
\item A face card 
\item A red face card
\item The jack of hearts
\item A spade
\item The queen of diamonds

\end{enumerate}
\solution
		%\input{ncert/10/15/1/14/main.tex}
	\item Five cards—the ten, jack, queen, king and ace of diamonds, are well-shuffled with their face downwards. One card is then picked up at random.
\begin{enumerate}
\item
What is the probability that the card is the queen? 
\item
If the queen is drawn and put aside, what is the probability that the second card picked up is (a) an ace? (b) a queen?\\
\end{enumerate}
\solution
		%\input{ncert/10/15/1/15/defs.tex}
	\item A bag contains $5$ red balls and some blue balls. If the probability of drawing a blue ball is double that if a red ball, determine the number of blue balls in the bag. 
		\\
\solution
		%\input{ncert/10/15/2/3/defs.tex}
	\item A card is selected from a pack of 52 cards.
 \begin{enumerate}[label=(\alph*)] 
                 \item How many points are there in the sample space?
                 \item Calculate the probability that the card is an ace of spades.
                 \item Calculate the probability that the card is (i) an ace and (ii) black card.
 \end{enumerate}
\solution
		%\input{ncert/11/16/3/4/main.tex}
\item Four cards are drawn from a well-shuffled deck of 52 cards. What is the probability of obtaining 3 diamonds and one spade.
\\
\solution
		%\input{ncert/11/16/4/2/defs.tex}
\item In a certain lottery 10,000 tickets are sold and ten equal prizes are awarded. What is the probability of not getting a prize if you buy (a) one ticket (b) two tickets (c) 10 tickets ?	
\\
\solution
		%\input{ncert/11/16/4/4/defs.tex}
		%
\item 
Out of 100 students, two sections of 40 and 60 are formed. If you and your friend are among the 100 students, what is the probability that
\begin{enumerate}
\item you both enter the same section?
\item you both enter the different sections?
\end{enumerate}
\solution
		%\input{ncert/11/16/4/5/defs.tex}
	\item 
The number lock of a suitcase has 4 wheels each labelled with ten digits i.e. from 0 to 9.The lock opens with a sequence of four digits with no repeats.What is the probability of a person getting the right sequence to open the suitcase.
\\
\solution
		%\input{ncert/11/16/4/10/defs.tex}
		%
\item 
Two cards are drawn at random and without replacement from a pack of 52 playing cards. Find the probability that both the cards are black.
\\
\solution
		%\input{ncert/12/13/2/2/defs.tex}
		\item A box of oranges is inspected by examining three randomly selected oranges drawn without replacement. If all the three oranges are good, the box is approved for sale, otherwise, it is rejected. Find the probability that a box containing 15 oranges out of which 12 are good and 3 are bad ones will be approved for sale.
		\label{ncert/12/13/2/3/defs.tex}
		\item Two balls are drawn at random with replacement from a box containing 10 black and 8 red balls. Find the probability that
		\label{ncert/12/13/2/12}
\begin{enumerate}
\item both balls are red.
\item first ball is black and second is red.
\item one of them is black and other is red.
\end{enumerate}

\item In a hostel, 60\% of the students read Hindi newspaper, 40\% read English newspaper and 20\% read both Hindi and English newspapers. A student is selected at random.
		\label{ncert/12/13/2/15}
\begin{enumerate}
\item Find the probability that she reads neither Hindi nor English newspapers.
\item If she reads Hindi newspaper, find the probability that she reads English newspaper.
\item If she reads English newspaper, find the probability that she reads Hindi newspaper.\\
\end{enumerate}
\item The probability of obtaining an even prime number on each die, when a pair of dice is rolled is 
\begin{enumerate}
    \item $0$ 
    
    \item $\frac{1}{3}$ 
    
    \item $\frac{1}{12}$ 
    
    \item $\frac{1}{36}$ 
\end{enumerate}
\solution
		%\input{ncert/12/13/2/17/defs.tex}
	\item A bag contains 4 red and 4 black balls, another bag contains 2 red and 6 black balls. One of the two bags is selected at random and a ball is drawn from the bag which is found to be red. Find the probability that the ball is drawn from the first bag.
\\
\solution
		%\input{ncert/12/13/3/2/main.tex}
  \item
  Cards with numbers 2 to 101 are placed in a box. A card is selected at random.Find the probability that the card has
\begin{enumerate}[label=(\roman*)]
	\item an even number 
	\item a square number
\end{enumerate}
\solution
%\input{exemplar/10/13/3/32/main.tex}
\item
The king, queen and jack of clubs are removed from a deck of 52 playing cards and then well shuffled. Now one card is drawn at random from the remaining cards.  Determine the probability that the card is
\begin{enumerate}[label=(\roman*)]
\item a club
\item 10 of hearts
\end{enumerate}
\solution
%\input{exemplar/10/13/3/29/main.tex}
\item A team of medical students doing their internship have to assist during surgeries
at a city hospital. The probabilities of surgeries rated as very complex, complex,
routine, simple or very simple are respectively, 0.15, 0.20, 0.31, 0.26, .08. Find
the probabilities that a particular surgery will be rated
\begin{enumerate}
	\item complex or very complex;
	\item neither very complex nor very simple;
	\item routine or complex
	\item routine or simple
\end{enumerate}
\solution
%\input{exemplar/11/16/3/8(1)/main.tex}
\item A card is selected from a pack of 52 cards.
\begin{enumerate}[label=(\alph*)]
    \item How many points are there in the sample space?
    \item Calculate the probability that the card is an ace of spades.
    \item Calculate the probability that the card is (i) an ace and (ii) black card.
\end{enumerate}
\solution
%\input{exemplar/11/16/3/4/main2.tex}
\item The probability that a non leap year selected at random will contain 53 sundays.
\\
\solution
%\input{exemplar/10/13/1/19/main.tex}
\item One of the four persons John, Rita, Aslam or Gurpreet will be promoted next
month. Consequently the sample space consists of four elementary outcomes
S = {John promoted, Rita promoted, Aslam promoted, Gurpreet promoted}
You are told that the chances of John’s promotion is same as that of Gurpreet,
Rita’s chances of promotion are twice as likely as Johns. Aslam’s chances are
four times that of John.
\begin{enumerate}
	\item Determine
	\begin{enumerate}
		\item P (John promoted)
		\item P (Rita promoted)
		\item P (Aslam promoted)
		\item P (Gurpreet promoted)
	\end{enumerate}
	\item If A = {John promoted or Gurpreet promoted}, find P (A).
\end{enumerate}
\solution
%\input{exemplar/11/16/3/10/main.tex}
\item A card is drawn from a deck of 52 cards. Find the probability of getting a king or a heart or a red card.\\
\solution
%\input{exemplar/11/16/3/15/main.tex}
\item The probability that a student will pass his examination is 0.73, the probability of
the student getting a compartment is 0.13, and the probability that the student will
either pass or get compartment is 0.96. State True or False.\\
\solution
%\input{exemplar/11/16/3/31/main.tex}
\item A card is selected from a pack of 52 cards\\
\begin{enumerate}[label=(\alph*)]
\item How many points are there in the sample space?
\item Calculate the probability that the cards is an ace of spades.
\item Calculate the probability that the card is (i) an ace (ii)black card.\\
\end{enumerate}
%\input{ncert/11/16/3/4_1/Prob_4.tex}
\item In a non-leap year, the probability of having 53 tuesdays or 53 wednesdays is\\
\solution
%\input{exemplar/11/16/3/18/main.tex}
\item There are 1000 sealed envelopes in a box, 10 of them contain a cash prize of
Rs 100 each, 100 of them contain a cash prize of Rs 50 each and 200 of them
contain a cash prize of Rs 10 each and rest do not contain any cash prize. If they
are well shuffled and an envelope is picked up out, what is the probability that it
contains no cash prize?\\
\solution
%\input{exemplar/10/13/3/34/main.tex}
\item 
A die is thrown and a card is selected at random from a deck of 52 playing cards. The probability of getting an even number on the die and a spade card.\\
\solution
%\input{exemplar/12/13/3/78/main.tex}
\item
If 4-digit numbers greater than 5,000 are randomly formed from the digits 0, 1, 3, 5, and 7, what is the probability of forming a number divisible by 5 when:
\begin{enumerate}
    \item The digits are repeated?
    \item The repetition of digits is not allowed?
\end{enumerate}
\solution
%\input{ncert/11/16/4/9/main.tex}
\item Consider the probability space $\brak{\Omega, \mathcal{G}, P}$ where $\Omega = [0,2]$ and $\mathcal{G} = \cbrak{\phi, \Omega, [0,1], (1,2]}$. Let $X$ and $Y$ be two functions on $\Omega$ defined as
\begin{align*}
    X(\omega) = 
    \begin{cases}
        1 & \text{if }\omega \in [0, 1]\\
        2 & \text{if }\omega \in (1, 2]
    \end{cases}
\end{align*}
and
\begin{align*}
    Y(\omega) = 
    \begin{cases}
        2 & \text{if }\omega \in [0, 1.5]\\
        3 & \text{if }\omega \in (1.5, 2].
    \end{cases}
\end{align*}
Then which one of the following statements is true?
\begin{enumerate}
    \item [(A)] $X$ is a random variable with respect to $\mathcal{G}$, but $Y$ is not a random variable with respect to $\mathcal{G}$.
    \item [(B)] $Y$ is a random variable with respect to $\mathcal{G}$, but $X$ is not a random variable with respect to $\mathcal{G}$.
    \item [(C)] Neither $X$ nor $Y$ is a random variable with respect to $\mathcal{G}$.
    \item [(D)] Both $X$ and $Y$ are random variables with respect to $\mathcal{G}$.
\end{enumerate} \hfill (GATE ST 2023)\\
\solution
%\input{gate/ST/2023/14/main.tex}
	\item  A die is loaded in such a way that each odd number is twice as likely to occur as
each even number. Find $P(G)$, where $G$ is the event that a number greater than
3 occurs on a single roll of the die.
\\
\solution
		%\input{exemplar/11/16/3/5/main.tex}
	\item All the jacks, queens and kings are removed from a deck of 52 playing cards. The remaining cards are well shuffled and then one card is drawn at random. Giving ace a value 1 similar value for other cards, find the probability that the card has a value 
		\begin{enumerate}
			\item 7
			\item greater than 7
			\item less than 7
		\end{enumerate}
		%\input{exemplar/10/13/3/30/main.tex}
  \item A Lot consists of 48 mobile phones of which 42 are good, 3 have only minor defects and 3 have major defects.Varnika will buy a phone if it is good but the trader will only buy a mobile if it has no major defects. One phone is selected at random from the lot. What is the probability that it is
\begin{enumerate}
	\item acceptable to Varnika?
            \item acceptable to the trader?
\end{enumerate}
\solution
	%\input{exemplar/10/13/3/40/main.tex}
 \item A student says that if you throw a die, it will show up 1 or not 1. Therefore, the probability of getting 1 and the probability of getting 'not 1' each is equal to $\frac{1}{2}$. Is this correct? Give reasons.\\
 \solution
        %\input{exemplar/10/13/2/9/main.tex}
   \item Four candidates A, B, C, D have ap-
plied for the assignment to coach a school cricket
team. If A is twice as likely to be selected as B, and
B and C are given about the same chance of being
selected, while C is twice as likely to be selected
as D, what are the probabilities that
\begin{enumerate}
\item C will be selected?
\item A will not be selected?
\end{enumerate}
	%\input{exemplar/11/16/3/9/main.tex}
 \item A bag contain 24 balls of which $x$ balls are red, $2x$ are white and $3x$ are blue. A ball is selected at random, What is the probability that it is
\begin{enumerate}[label=\alph*)]
\item not red ?
\item white ?
\end{enumerate}
%\input{exemplar/10/13/3/41/main.tex}
If the letters of the word ASSASSINATION are arranged at random. Find the Probability that
\begin{enumerate}[label=(\alph*)]
\item Four $S's$ come consecutively in the word
\item Two  $I's$ and two $N's$ come together
\item All $A's$ are not coming together
\item No two $A's$ are coming together
\end{enumerate}
%\input{exemplar/11/16/3/14/main.tex}
	\item One urn contains two black balls (labelled B1 and B2) and one white ball. A
	second urn contains one black ball and two white balls (labelled W1 and W2).
	Suppose the following experiment is performed. One of the two urns is chosen
	at random. Next a ball is randomly chosen from the urn. Then a second ball is
	chosen at random from the same urn without replacing the first ball.
	
	\begin{enumerate}
	\item What is the probability that two black balls are chosen?
	
	\item What is the probability that two balls of opposite colour are chosen?
	\end{enumerate}
	\solution
	%\input{exemplar/11/16/3/12/main1.tex}
\end{enumerate}

		%
\item 
Out of 100 students, two sections of 40 and 60 are formed. If you and your friend are among the 100 students, what is the probability that
\begin{enumerate}
\item you both enter the same section?
\item you both enter the different sections?
\end{enumerate}
\solution
		%\begin{enumerate}[label=\thesection.\arabic*,ref=\thesection.\theenumi]
	\item One card is drawn from a well-shuffled deck of 52 cards. Find the probability of getting
\begin{enumerate}
\item A king of red colour 
\item A face card 
\item A red face card
\item The jack of hearts
\item A spade
\item The queen of diamonds

\end{enumerate}
\solution
		%\input{ncert/10/15/1/14/main.tex}
	\item Five cards—the ten, jack, queen, king and ace of diamonds, are well-shuffled with their face downwards. One card is then picked up at random.
\begin{enumerate}
\item
What is the probability that the card is the queen? 
\item
If the queen is drawn and put aside, what is the probability that the second card picked up is (a) an ace? (b) a queen?\\
\end{enumerate}
\solution
		%\input{ncert/10/15/1/15/defs.tex}
	\item A bag contains $5$ red balls and some blue balls. If the probability of drawing a blue ball is double that if a red ball, determine the number of blue balls in the bag. 
		\\
\solution
		%\input{ncert/10/15/2/3/defs.tex}
	\item A card is selected from a pack of 52 cards.
 \begin{enumerate}[label=(\alph*)] 
                 \item How many points are there in the sample space?
                 \item Calculate the probability that the card is an ace of spades.
                 \item Calculate the probability that the card is (i) an ace and (ii) black card.
 \end{enumerate}
\solution
		%\input{ncert/11/16/3/4/main.tex}
\item Four cards are drawn from a well-shuffled deck of 52 cards. What is the probability of obtaining 3 diamonds and one spade.
\\
\solution
		%\input{ncert/11/16/4/2/defs.tex}
\item In a certain lottery 10,000 tickets are sold and ten equal prizes are awarded. What is the probability of not getting a prize if you buy (a) one ticket (b) two tickets (c) 10 tickets ?	
\\
\solution
		%\input{ncert/11/16/4/4/defs.tex}
		%
\item 
Out of 100 students, two sections of 40 and 60 are formed. If you and your friend are among the 100 students, what is the probability that
\begin{enumerate}
\item you both enter the same section?
\item you both enter the different sections?
\end{enumerate}
\solution
		%\input{ncert/11/16/4/5/defs.tex}
	\item 
The number lock of a suitcase has 4 wheels each labelled with ten digits i.e. from 0 to 9.The lock opens with a sequence of four digits with no repeats.What is the probability of a person getting the right sequence to open the suitcase.
\\
\solution
		%\input{ncert/11/16/4/10/defs.tex}
		%
\item 
Two cards are drawn at random and without replacement from a pack of 52 playing cards. Find the probability that both the cards are black.
\\
\solution
		%\input{ncert/12/13/2/2/defs.tex}
		\item A box of oranges is inspected by examining three randomly selected oranges drawn without replacement. If all the three oranges are good, the box is approved for sale, otherwise, it is rejected. Find the probability that a box containing 15 oranges out of which 12 are good and 3 are bad ones will be approved for sale.
		\label{ncert/12/13/2/3/defs.tex}
		\item Two balls are drawn at random with replacement from a box containing 10 black and 8 red balls. Find the probability that
		\label{ncert/12/13/2/12}
\begin{enumerate}
\item both balls are red.
\item first ball is black and second is red.
\item one of them is black and other is red.
\end{enumerate}

\item In a hostel, 60\% of the students read Hindi newspaper, 40\% read English newspaper and 20\% read both Hindi and English newspapers. A student is selected at random.
		\label{ncert/12/13/2/15}
\begin{enumerate}
\item Find the probability that she reads neither Hindi nor English newspapers.
\item If she reads Hindi newspaper, find the probability that she reads English newspaper.
\item If she reads English newspaper, find the probability that she reads Hindi newspaper.\\
\end{enumerate}
\item The probability of obtaining an even prime number on each die, when a pair of dice is rolled is 
\begin{enumerate}
    \item $0$ 
    
    \item $\frac{1}{3}$ 
    
    \item $\frac{1}{12}$ 
    
    \item $\frac{1}{36}$ 
\end{enumerate}
\solution
		%\input{ncert/12/13/2/17/defs.tex}
	\item A bag contains 4 red and 4 black balls, another bag contains 2 red and 6 black balls. One of the two bags is selected at random and a ball is drawn from the bag which is found to be red. Find the probability that the ball is drawn from the first bag.
\\
\solution
		%\input{ncert/12/13/3/2/main.tex}
  \item
  Cards with numbers 2 to 101 are placed in a box. A card is selected at random.Find the probability that the card has
\begin{enumerate}[label=(\roman*)]
	\item an even number 
	\item a square number
\end{enumerate}
\solution
%\input{exemplar/10/13/3/32/main.tex}
\item
The king, queen and jack of clubs are removed from a deck of 52 playing cards and then well shuffled. Now one card is drawn at random from the remaining cards.  Determine the probability that the card is
\begin{enumerate}[label=(\roman*)]
\item a club
\item 10 of hearts
\end{enumerate}
\solution
%\input{exemplar/10/13/3/29/main.tex}
\item A team of medical students doing their internship have to assist during surgeries
at a city hospital. The probabilities of surgeries rated as very complex, complex,
routine, simple or very simple are respectively, 0.15, 0.20, 0.31, 0.26, .08. Find
the probabilities that a particular surgery will be rated
\begin{enumerate}
	\item complex or very complex;
	\item neither very complex nor very simple;
	\item routine or complex
	\item routine or simple
\end{enumerate}
\solution
%\input{exemplar/11/16/3/8(1)/main.tex}
\item A card is selected from a pack of 52 cards.
\begin{enumerate}[label=(\alph*)]
    \item How many points are there in the sample space?
    \item Calculate the probability that the card is an ace of spades.
    \item Calculate the probability that the card is (i) an ace and (ii) black card.
\end{enumerate}
\solution
%\input{exemplar/11/16/3/4/main2.tex}
\item The probability that a non leap year selected at random will contain 53 sundays.
\\
\solution
%\input{exemplar/10/13/1/19/main.tex}
\item One of the four persons John, Rita, Aslam or Gurpreet will be promoted next
month. Consequently the sample space consists of four elementary outcomes
S = {John promoted, Rita promoted, Aslam promoted, Gurpreet promoted}
You are told that the chances of John’s promotion is same as that of Gurpreet,
Rita’s chances of promotion are twice as likely as Johns. Aslam’s chances are
four times that of John.
\begin{enumerate}
	\item Determine
	\begin{enumerate}
		\item P (John promoted)
		\item P (Rita promoted)
		\item P (Aslam promoted)
		\item P (Gurpreet promoted)
	\end{enumerate}
	\item If A = {John promoted or Gurpreet promoted}, find P (A).
\end{enumerate}
\solution
%\input{exemplar/11/16/3/10/main.tex}
\item A card is drawn from a deck of 52 cards. Find the probability of getting a king or a heart or a red card.\\
\solution
%\input{exemplar/11/16/3/15/main.tex}
\item The probability that a student will pass his examination is 0.73, the probability of
the student getting a compartment is 0.13, and the probability that the student will
either pass or get compartment is 0.96. State True or False.\\
\solution
%\input{exemplar/11/16/3/31/main.tex}
\item A card is selected from a pack of 52 cards\\
\begin{enumerate}[label=(\alph*)]
\item How many points are there in the sample space?
\item Calculate the probability that the cards is an ace of spades.
\item Calculate the probability that the card is (i) an ace (ii)black card.\\
\end{enumerate}
%\input{ncert/11/16/3/4_1/Prob_4.tex}
\item In a non-leap year, the probability of having 53 tuesdays or 53 wednesdays is\\
\solution
%\input{exemplar/11/16/3/18/main.tex}
\item There are 1000 sealed envelopes in a box, 10 of them contain a cash prize of
Rs 100 each, 100 of them contain a cash prize of Rs 50 each and 200 of them
contain a cash prize of Rs 10 each and rest do not contain any cash prize. If they
are well shuffled and an envelope is picked up out, what is the probability that it
contains no cash prize?\\
\solution
%\input{exemplar/10/13/3/34/main.tex}
\item 
A die is thrown and a card is selected at random from a deck of 52 playing cards. The probability of getting an even number on the die and a spade card.\\
\solution
%\input{exemplar/12/13/3/78/main.tex}
\item
If 4-digit numbers greater than 5,000 are randomly formed from the digits 0, 1, 3, 5, and 7, what is the probability of forming a number divisible by 5 when:
\begin{enumerate}
    \item The digits are repeated?
    \item The repetition of digits is not allowed?
\end{enumerate}
\solution
%\input{ncert/11/16/4/9/main.tex}
\item Consider the probability space $\brak{\Omega, \mathcal{G}, P}$ where $\Omega = [0,2]$ and $\mathcal{G} = \cbrak{\phi, \Omega, [0,1], (1,2]}$. Let $X$ and $Y$ be two functions on $\Omega$ defined as
\begin{align*}
    X(\omega) = 
    \begin{cases}
        1 & \text{if }\omega \in [0, 1]\\
        2 & \text{if }\omega \in (1, 2]
    \end{cases}
\end{align*}
and
\begin{align*}
    Y(\omega) = 
    \begin{cases}
        2 & \text{if }\omega \in [0, 1.5]\\
        3 & \text{if }\omega \in (1.5, 2].
    \end{cases}
\end{align*}
Then which one of the following statements is true?
\begin{enumerate}
    \item [(A)] $X$ is a random variable with respect to $\mathcal{G}$, but $Y$ is not a random variable with respect to $\mathcal{G}$.
    \item [(B)] $Y$ is a random variable with respect to $\mathcal{G}$, but $X$ is not a random variable with respect to $\mathcal{G}$.
    \item [(C)] Neither $X$ nor $Y$ is a random variable with respect to $\mathcal{G}$.
    \item [(D)] Both $X$ and $Y$ are random variables with respect to $\mathcal{G}$.
\end{enumerate} \hfill (GATE ST 2023)\\
\solution
%\input{gate/ST/2023/14/main.tex}
	\item  A die is loaded in such a way that each odd number is twice as likely to occur as
each even number. Find $P(G)$, where $G$ is the event that a number greater than
3 occurs on a single roll of the die.
\\
\solution
		%\input{exemplar/11/16/3/5/main.tex}
	\item All the jacks, queens and kings are removed from a deck of 52 playing cards. The remaining cards are well shuffled and then one card is drawn at random. Giving ace a value 1 similar value for other cards, find the probability that the card has a value 
		\begin{enumerate}
			\item 7
			\item greater than 7
			\item less than 7
		\end{enumerate}
		%\input{exemplar/10/13/3/30/main.tex}
  \item A Lot consists of 48 mobile phones of which 42 are good, 3 have only minor defects and 3 have major defects.Varnika will buy a phone if it is good but the trader will only buy a mobile if it has no major defects. One phone is selected at random from the lot. What is the probability that it is
\begin{enumerate}
	\item acceptable to Varnika?
            \item acceptable to the trader?
\end{enumerate}
\solution
	%\input{exemplar/10/13/3/40/main.tex}
 \item A student says that if you throw a die, it will show up 1 or not 1. Therefore, the probability of getting 1 and the probability of getting 'not 1' each is equal to $\frac{1}{2}$. Is this correct? Give reasons.\\
 \solution
        %\input{exemplar/10/13/2/9/main.tex}
   \item Four candidates A, B, C, D have ap-
plied for the assignment to coach a school cricket
team. If A is twice as likely to be selected as B, and
B and C are given about the same chance of being
selected, while C is twice as likely to be selected
as D, what are the probabilities that
\begin{enumerate}
\item C will be selected?
\item A will not be selected?
\end{enumerate}
	%\input{exemplar/11/16/3/9/main.tex}
 \item A bag contain 24 balls of which $x$ balls are red, $2x$ are white and $3x$ are blue. A ball is selected at random, What is the probability that it is
\begin{enumerate}[label=\alph*)]
\item not red ?
\item white ?
\end{enumerate}
%\input{exemplar/10/13/3/41/main.tex}
If the letters of the word ASSASSINATION are arranged at random. Find the Probability that
\begin{enumerate}[label=(\alph*)]
\item Four $S's$ come consecutively in the word
\item Two  $I's$ and two $N's$ come together
\item All $A's$ are not coming together
\item No two $A's$ are coming together
\end{enumerate}
%\input{exemplar/11/16/3/14/main.tex}
	\item One urn contains two black balls (labelled B1 and B2) and one white ball. A
	second urn contains one black ball and two white balls (labelled W1 and W2).
	Suppose the following experiment is performed. One of the two urns is chosen
	at random. Next a ball is randomly chosen from the urn. Then a second ball is
	chosen at random from the same urn without replacing the first ball.
	
	\begin{enumerate}
	\item What is the probability that two black balls are chosen?
	
	\item What is the probability that two balls of opposite colour are chosen?
	\end{enumerate}
	\solution
	%\input{exemplar/11/16/3/12/main1.tex}
\end{enumerate}

	\item 
The number lock of a suitcase has 4 wheels each labelled with ten digits i.e. from 0 to 9.The lock opens with a sequence of four digits with no repeats.What is the probability of a person getting the right sequence to open the suitcase.
\\
\solution
		%\begin{enumerate}[label=\thesection.\arabic*,ref=\thesection.\theenumi]
	\item One card is drawn from a well-shuffled deck of 52 cards. Find the probability of getting
\begin{enumerate}
\item A king of red colour 
\item A face card 
\item A red face card
\item The jack of hearts
\item A spade
\item The queen of diamonds

\end{enumerate}
\solution
		%\input{ncert/10/15/1/14/main.tex}
	\item Five cards—the ten, jack, queen, king and ace of diamonds, are well-shuffled with their face downwards. One card is then picked up at random.
\begin{enumerate}
\item
What is the probability that the card is the queen? 
\item
If the queen is drawn and put aside, what is the probability that the second card picked up is (a) an ace? (b) a queen?\\
\end{enumerate}
\solution
		%\input{ncert/10/15/1/15/defs.tex}
	\item A bag contains $5$ red balls and some blue balls. If the probability of drawing a blue ball is double that if a red ball, determine the number of blue balls in the bag. 
		\\
\solution
		%\input{ncert/10/15/2/3/defs.tex}
	\item A card is selected from a pack of 52 cards.
 \begin{enumerate}[label=(\alph*)] 
                 \item How many points are there in the sample space?
                 \item Calculate the probability that the card is an ace of spades.
                 \item Calculate the probability that the card is (i) an ace and (ii) black card.
 \end{enumerate}
\solution
		%\input{ncert/11/16/3/4/main.tex}
\item Four cards are drawn from a well-shuffled deck of 52 cards. What is the probability of obtaining 3 diamonds and one spade.
\\
\solution
		%\input{ncert/11/16/4/2/defs.tex}
\item In a certain lottery 10,000 tickets are sold and ten equal prizes are awarded. What is the probability of not getting a prize if you buy (a) one ticket (b) two tickets (c) 10 tickets ?	
\\
\solution
		%\input{ncert/11/16/4/4/defs.tex}
		%
\item 
Out of 100 students, two sections of 40 and 60 are formed. If you and your friend are among the 100 students, what is the probability that
\begin{enumerate}
\item you both enter the same section?
\item you both enter the different sections?
\end{enumerate}
\solution
		%\input{ncert/11/16/4/5/defs.tex}
	\item 
The number lock of a suitcase has 4 wheels each labelled with ten digits i.e. from 0 to 9.The lock opens with a sequence of four digits with no repeats.What is the probability of a person getting the right sequence to open the suitcase.
\\
\solution
		%\input{ncert/11/16/4/10/defs.tex}
		%
\item 
Two cards are drawn at random and without replacement from a pack of 52 playing cards. Find the probability that both the cards are black.
\\
\solution
		%\input{ncert/12/13/2/2/defs.tex}
		\item A box of oranges is inspected by examining three randomly selected oranges drawn without replacement. If all the three oranges are good, the box is approved for sale, otherwise, it is rejected. Find the probability that a box containing 15 oranges out of which 12 are good and 3 are bad ones will be approved for sale.
		\label{ncert/12/13/2/3/defs.tex}
		\item Two balls are drawn at random with replacement from a box containing 10 black and 8 red balls. Find the probability that
		\label{ncert/12/13/2/12}
\begin{enumerate}
\item both balls are red.
\item first ball is black and second is red.
\item one of them is black and other is red.
\end{enumerate}

\item In a hostel, 60\% of the students read Hindi newspaper, 40\% read English newspaper and 20\% read both Hindi and English newspapers. A student is selected at random.
		\label{ncert/12/13/2/15}
\begin{enumerate}
\item Find the probability that she reads neither Hindi nor English newspapers.
\item If she reads Hindi newspaper, find the probability that she reads English newspaper.
\item If she reads English newspaper, find the probability that she reads Hindi newspaper.\\
\end{enumerate}
\item The probability of obtaining an even prime number on each die, when a pair of dice is rolled is 
\begin{enumerate}
    \item $0$ 
    
    \item $\frac{1}{3}$ 
    
    \item $\frac{1}{12}$ 
    
    \item $\frac{1}{36}$ 
\end{enumerate}
\solution
		%\input{ncert/12/13/2/17/defs.tex}
	\item A bag contains 4 red and 4 black balls, another bag contains 2 red and 6 black balls. One of the two bags is selected at random and a ball is drawn from the bag which is found to be red. Find the probability that the ball is drawn from the first bag.
\\
\solution
		%\input{ncert/12/13/3/2/main.tex}
  \item
  Cards with numbers 2 to 101 are placed in a box. A card is selected at random.Find the probability that the card has
\begin{enumerate}[label=(\roman*)]
	\item an even number 
	\item a square number
\end{enumerate}
\solution
%\input{exemplar/10/13/3/32/main.tex}
\item
The king, queen and jack of clubs are removed from a deck of 52 playing cards and then well shuffled. Now one card is drawn at random from the remaining cards.  Determine the probability that the card is
\begin{enumerate}[label=(\roman*)]
\item a club
\item 10 of hearts
\end{enumerate}
\solution
%\input{exemplar/10/13/3/29/main.tex}
\item A team of medical students doing their internship have to assist during surgeries
at a city hospital. The probabilities of surgeries rated as very complex, complex,
routine, simple or very simple are respectively, 0.15, 0.20, 0.31, 0.26, .08. Find
the probabilities that a particular surgery will be rated
\begin{enumerate}
	\item complex or very complex;
	\item neither very complex nor very simple;
	\item routine or complex
	\item routine or simple
\end{enumerate}
\solution
%\input{exemplar/11/16/3/8(1)/main.tex}
\item A card is selected from a pack of 52 cards.
\begin{enumerate}[label=(\alph*)]
    \item How many points are there in the sample space?
    \item Calculate the probability that the card is an ace of spades.
    \item Calculate the probability that the card is (i) an ace and (ii) black card.
\end{enumerate}
\solution
%\input{exemplar/11/16/3/4/main2.tex}
\item The probability that a non leap year selected at random will contain 53 sundays.
\\
\solution
%\input{exemplar/10/13/1/19/main.tex}
\item One of the four persons John, Rita, Aslam or Gurpreet will be promoted next
month. Consequently the sample space consists of four elementary outcomes
S = {John promoted, Rita promoted, Aslam promoted, Gurpreet promoted}
You are told that the chances of John’s promotion is same as that of Gurpreet,
Rita’s chances of promotion are twice as likely as Johns. Aslam’s chances are
four times that of John.
\begin{enumerate}
	\item Determine
	\begin{enumerate}
		\item P (John promoted)
		\item P (Rita promoted)
		\item P (Aslam promoted)
		\item P (Gurpreet promoted)
	\end{enumerate}
	\item If A = {John promoted or Gurpreet promoted}, find P (A).
\end{enumerate}
\solution
%\input{exemplar/11/16/3/10/main.tex}
\item A card is drawn from a deck of 52 cards. Find the probability of getting a king or a heart or a red card.\\
\solution
%\input{exemplar/11/16/3/15/main.tex}
\item The probability that a student will pass his examination is 0.73, the probability of
the student getting a compartment is 0.13, and the probability that the student will
either pass or get compartment is 0.96. State True or False.\\
\solution
%\input{exemplar/11/16/3/31/main.tex}
\item A card is selected from a pack of 52 cards\\
\begin{enumerate}[label=(\alph*)]
\item How many points are there in the sample space?
\item Calculate the probability that the cards is an ace of spades.
\item Calculate the probability that the card is (i) an ace (ii)black card.\\
\end{enumerate}
%\input{ncert/11/16/3/4_1/Prob_4.tex}
\item In a non-leap year, the probability of having 53 tuesdays or 53 wednesdays is\\
\solution
%\input{exemplar/11/16/3/18/main.tex}
\item There are 1000 sealed envelopes in a box, 10 of them contain a cash prize of
Rs 100 each, 100 of them contain a cash prize of Rs 50 each and 200 of them
contain a cash prize of Rs 10 each and rest do not contain any cash prize. If they
are well shuffled and an envelope is picked up out, what is the probability that it
contains no cash prize?\\
\solution
%\input{exemplar/10/13/3/34/main.tex}
\item 
A die is thrown and a card is selected at random from a deck of 52 playing cards. The probability of getting an even number on the die and a spade card.\\
\solution
%\input{exemplar/12/13/3/78/main.tex}
\item
If 4-digit numbers greater than 5,000 are randomly formed from the digits 0, 1, 3, 5, and 7, what is the probability of forming a number divisible by 5 when:
\begin{enumerate}
    \item The digits are repeated?
    \item The repetition of digits is not allowed?
\end{enumerate}
\solution
%\input{ncert/11/16/4/9/main.tex}
\item Consider the probability space $\brak{\Omega, \mathcal{G}, P}$ where $\Omega = [0,2]$ and $\mathcal{G} = \cbrak{\phi, \Omega, [0,1], (1,2]}$. Let $X$ and $Y$ be two functions on $\Omega$ defined as
\begin{align*}
    X(\omega) = 
    \begin{cases}
        1 & \text{if }\omega \in [0, 1]\\
        2 & \text{if }\omega \in (1, 2]
    \end{cases}
\end{align*}
and
\begin{align*}
    Y(\omega) = 
    \begin{cases}
        2 & \text{if }\omega \in [0, 1.5]\\
        3 & \text{if }\omega \in (1.5, 2].
    \end{cases}
\end{align*}
Then which one of the following statements is true?
\begin{enumerate}
    \item [(A)] $X$ is a random variable with respect to $\mathcal{G}$, but $Y$ is not a random variable with respect to $\mathcal{G}$.
    \item [(B)] $Y$ is a random variable with respect to $\mathcal{G}$, but $X$ is not a random variable with respect to $\mathcal{G}$.
    \item [(C)] Neither $X$ nor $Y$ is a random variable with respect to $\mathcal{G}$.
    \item [(D)] Both $X$ and $Y$ are random variables with respect to $\mathcal{G}$.
\end{enumerate} \hfill (GATE ST 2023)\\
\solution
%\input{gate/ST/2023/14/main.tex}
	\item  A die is loaded in such a way that each odd number is twice as likely to occur as
each even number. Find $P(G)$, where $G$ is the event that a number greater than
3 occurs on a single roll of the die.
\\
\solution
		%\input{exemplar/11/16/3/5/main.tex}
	\item All the jacks, queens and kings are removed from a deck of 52 playing cards. The remaining cards are well shuffled and then one card is drawn at random. Giving ace a value 1 similar value for other cards, find the probability that the card has a value 
		\begin{enumerate}
			\item 7
			\item greater than 7
			\item less than 7
		\end{enumerate}
		%\input{exemplar/10/13/3/30/main.tex}
  \item A Lot consists of 48 mobile phones of which 42 are good, 3 have only minor defects and 3 have major defects.Varnika will buy a phone if it is good but the trader will only buy a mobile if it has no major defects. One phone is selected at random from the lot. What is the probability that it is
\begin{enumerate}
	\item acceptable to Varnika?
            \item acceptable to the trader?
\end{enumerate}
\solution
	%\input{exemplar/10/13/3/40/main.tex}
 \item A student says that if you throw a die, it will show up 1 or not 1. Therefore, the probability of getting 1 and the probability of getting 'not 1' each is equal to $\frac{1}{2}$. Is this correct? Give reasons.\\
 \solution
        %\input{exemplar/10/13/2/9/main.tex}
   \item Four candidates A, B, C, D have ap-
plied for the assignment to coach a school cricket
team. If A is twice as likely to be selected as B, and
B and C are given about the same chance of being
selected, while C is twice as likely to be selected
as D, what are the probabilities that
\begin{enumerate}
\item C will be selected?
\item A will not be selected?
\end{enumerate}
	%\input{exemplar/11/16/3/9/main.tex}
 \item A bag contain 24 balls of which $x$ balls are red, $2x$ are white and $3x$ are blue. A ball is selected at random, What is the probability that it is
\begin{enumerate}[label=\alph*)]
\item not red ?
\item white ?
\end{enumerate}
%\input{exemplar/10/13/3/41/main.tex}
If the letters of the word ASSASSINATION are arranged at random. Find the Probability that
\begin{enumerate}[label=(\alph*)]
\item Four $S's$ come consecutively in the word
\item Two  $I's$ and two $N's$ come together
\item All $A's$ are not coming together
\item No two $A's$ are coming together
\end{enumerate}
%\input{exemplar/11/16/3/14/main.tex}
	\item One urn contains two black balls (labelled B1 and B2) and one white ball. A
	second urn contains one black ball and two white balls (labelled W1 and W2).
	Suppose the following experiment is performed. One of the two urns is chosen
	at random. Next a ball is randomly chosen from the urn. Then a second ball is
	chosen at random from the same urn without replacing the first ball.
	
	\begin{enumerate}
	\item What is the probability that two black balls are chosen?
	
	\item What is the probability that two balls of opposite colour are chosen?
	\end{enumerate}
	\solution
	%\input{exemplar/11/16/3/12/main1.tex}
\end{enumerate}

		%
\item 
Two cards are drawn at random and without replacement from a pack of 52 playing cards. Find the probability that both the cards are black.
\\
\solution
		%\begin{enumerate}[label=\thesection.\arabic*,ref=\thesection.\theenumi]
	\item One card is drawn from a well-shuffled deck of 52 cards. Find the probability of getting
\begin{enumerate}
\item A king of red colour 
\item A face card 
\item A red face card
\item The jack of hearts
\item A spade
\item The queen of diamonds

\end{enumerate}
\solution
		%\input{ncert/10/15/1/14/main.tex}
	\item Five cards—the ten, jack, queen, king and ace of diamonds, are well-shuffled with their face downwards. One card is then picked up at random.
\begin{enumerate}
\item
What is the probability that the card is the queen? 
\item
If the queen is drawn and put aside, what is the probability that the second card picked up is (a) an ace? (b) a queen?\\
\end{enumerate}
\solution
		%\input{ncert/10/15/1/15/defs.tex}
	\item A bag contains $5$ red balls and some blue balls. If the probability of drawing a blue ball is double that if a red ball, determine the number of blue balls in the bag. 
		\\
\solution
		%\input{ncert/10/15/2/3/defs.tex}
	\item A card is selected from a pack of 52 cards.
 \begin{enumerate}[label=(\alph*)] 
                 \item How many points are there in the sample space?
                 \item Calculate the probability that the card is an ace of spades.
                 \item Calculate the probability that the card is (i) an ace and (ii) black card.
 \end{enumerate}
\solution
		%\input{ncert/11/16/3/4/main.tex}
\item Four cards are drawn from a well-shuffled deck of 52 cards. What is the probability of obtaining 3 diamonds and one spade.
\\
\solution
		%\input{ncert/11/16/4/2/defs.tex}
\item In a certain lottery 10,000 tickets are sold and ten equal prizes are awarded. What is the probability of not getting a prize if you buy (a) one ticket (b) two tickets (c) 10 tickets ?	
\\
\solution
		%\input{ncert/11/16/4/4/defs.tex}
		%
\item 
Out of 100 students, two sections of 40 and 60 are formed. If you and your friend are among the 100 students, what is the probability that
\begin{enumerate}
\item you both enter the same section?
\item you both enter the different sections?
\end{enumerate}
\solution
		%\input{ncert/11/16/4/5/defs.tex}
	\item 
The number lock of a suitcase has 4 wheels each labelled with ten digits i.e. from 0 to 9.The lock opens with a sequence of four digits with no repeats.What is the probability of a person getting the right sequence to open the suitcase.
\\
\solution
		%\input{ncert/11/16/4/10/defs.tex}
		%
\item 
Two cards are drawn at random and without replacement from a pack of 52 playing cards. Find the probability that both the cards are black.
\\
\solution
		%\input{ncert/12/13/2/2/defs.tex}
		\item A box of oranges is inspected by examining three randomly selected oranges drawn without replacement. If all the three oranges are good, the box is approved for sale, otherwise, it is rejected. Find the probability that a box containing 15 oranges out of which 12 are good and 3 are bad ones will be approved for sale.
		\label{ncert/12/13/2/3/defs.tex}
		\item Two balls are drawn at random with replacement from a box containing 10 black and 8 red balls. Find the probability that
		\label{ncert/12/13/2/12}
\begin{enumerate}
\item both balls are red.
\item first ball is black and second is red.
\item one of them is black and other is red.
\end{enumerate}

\item In a hostel, 60\% of the students read Hindi newspaper, 40\% read English newspaper and 20\% read both Hindi and English newspapers. A student is selected at random.
		\label{ncert/12/13/2/15}
\begin{enumerate}
\item Find the probability that she reads neither Hindi nor English newspapers.
\item If she reads Hindi newspaper, find the probability that she reads English newspaper.
\item If she reads English newspaper, find the probability that she reads Hindi newspaper.\\
\end{enumerate}
\item The probability of obtaining an even prime number on each die, when a pair of dice is rolled is 
\begin{enumerate}
    \item $0$ 
    
    \item $\frac{1}{3}$ 
    
    \item $\frac{1}{12}$ 
    
    \item $\frac{1}{36}$ 
\end{enumerate}
\solution
		%\input{ncert/12/13/2/17/defs.tex}
	\item A bag contains 4 red and 4 black balls, another bag contains 2 red and 6 black balls. One of the two bags is selected at random and a ball is drawn from the bag which is found to be red. Find the probability that the ball is drawn from the first bag.
\\
\solution
		%\input{ncert/12/13/3/2/main.tex}
  \item
  Cards with numbers 2 to 101 are placed in a box. A card is selected at random.Find the probability that the card has
\begin{enumerate}[label=(\roman*)]
	\item an even number 
	\item a square number
\end{enumerate}
\solution
%\input{exemplar/10/13/3/32/main.tex}
\item
The king, queen and jack of clubs are removed from a deck of 52 playing cards and then well shuffled. Now one card is drawn at random from the remaining cards.  Determine the probability that the card is
\begin{enumerate}[label=(\roman*)]
\item a club
\item 10 of hearts
\end{enumerate}
\solution
%\input{exemplar/10/13/3/29/main.tex}
\item A team of medical students doing their internship have to assist during surgeries
at a city hospital. The probabilities of surgeries rated as very complex, complex,
routine, simple or very simple are respectively, 0.15, 0.20, 0.31, 0.26, .08. Find
the probabilities that a particular surgery will be rated
\begin{enumerate}
	\item complex or very complex;
	\item neither very complex nor very simple;
	\item routine or complex
	\item routine or simple
\end{enumerate}
\solution
%\input{exemplar/11/16/3/8(1)/main.tex}
\item A card is selected from a pack of 52 cards.
\begin{enumerate}[label=(\alph*)]
    \item How many points are there in the sample space?
    \item Calculate the probability that the card is an ace of spades.
    \item Calculate the probability that the card is (i) an ace and (ii) black card.
\end{enumerate}
\solution
%\input{exemplar/11/16/3/4/main2.tex}
\item The probability that a non leap year selected at random will contain 53 sundays.
\\
\solution
%\input{exemplar/10/13/1/19/main.tex}
\item One of the four persons John, Rita, Aslam or Gurpreet will be promoted next
month. Consequently the sample space consists of four elementary outcomes
S = {John promoted, Rita promoted, Aslam promoted, Gurpreet promoted}
You are told that the chances of John’s promotion is same as that of Gurpreet,
Rita’s chances of promotion are twice as likely as Johns. Aslam’s chances are
four times that of John.
\begin{enumerate}
	\item Determine
	\begin{enumerate}
		\item P (John promoted)
		\item P (Rita promoted)
		\item P (Aslam promoted)
		\item P (Gurpreet promoted)
	\end{enumerate}
	\item If A = {John promoted or Gurpreet promoted}, find P (A).
\end{enumerate}
\solution
%\input{exemplar/11/16/3/10/main.tex}
\item A card is drawn from a deck of 52 cards. Find the probability of getting a king or a heart or a red card.\\
\solution
%\input{exemplar/11/16/3/15/main.tex}
\item The probability that a student will pass his examination is 0.73, the probability of
the student getting a compartment is 0.13, and the probability that the student will
either pass or get compartment is 0.96. State True or False.\\
\solution
%\input{exemplar/11/16/3/31/main.tex}
\item A card is selected from a pack of 52 cards\\
\begin{enumerate}[label=(\alph*)]
\item How many points are there in the sample space?
\item Calculate the probability that the cards is an ace of spades.
\item Calculate the probability that the card is (i) an ace (ii)black card.\\
\end{enumerate}
%\input{ncert/11/16/3/4_1/Prob_4.tex}
\item In a non-leap year, the probability of having 53 tuesdays or 53 wednesdays is\\
\solution
%\input{exemplar/11/16/3/18/main.tex}
\item There are 1000 sealed envelopes in a box, 10 of them contain a cash prize of
Rs 100 each, 100 of them contain a cash prize of Rs 50 each and 200 of them
contain a cash prize of Rs 10 each and rest do not contain any cash prize. If they
are well shuffled and an envelope is picked up out, what is the probability that it
contains no cash prize?\\
\solution
%\input{exemplar/10/13/3/34/main.tex}
\item 
A die is thrown and a card is selected at random from a deck of 52 playing cards. The probability of getting an even number on the die and a spade card.\\
\solution
%\input{exemplar/12/13/3/78/main.tex}
\item
If 4-digit numbers greater than 5,000 are randomly formed from the digits 0, 1, 3, 5, and 7, what is the probability of forming a number divisible by 5 when:
\begin{enumerate}
    \item The digits are repeated?
    \item The repetition of digits is not allowed?
\end{enumerate}
\solution
%\input{ncert/11/16/4/9/main.tex}
\item Consider the probability space $\brak{\Omega, \mathcal{G}, P}$ where $\Omega = [0,2]$ and $\mathcal{G} = \cbrak{\phi, \Omega, [0,1], (1,2]}$. Let $X$ and $Y$ be two functions on $\Omega$ defined as
\begin{align*}
    X(\omega) = 
    \begin{cases}
        1 & \text{if }\omega \in [0, 1]\\
        2 & \text{if }\omega \in (1, 2]
    \end{cases}
\end{align*}
and
\begin{align*}
    Y(\omega) = 
    \begin{cases}
        2 & \text{if }\omega \in [0, 1.5]\\
        3 & \text{if }\omega \in (1.5, 2].
    \end{cases}
\end{align*}
Then which one of the following statements is true?
\begin{enumerate}
    \item [(A)] $X$ is a random variable with respect to $\mathcal{G}$, but $Y$ is not a random variable with respect to $\mathcal{G}$.
    \item [(B)] $Y$ is a random variable with respect to $\mathcal{G}$, but $X$ is not a random variable with respect to $\mathcal{G}$.
    \item [(C)] Neither $X$ nor $Y$ is a random variable with respect to $\mathcal{G}$.
    \item [(D)] Both $X$ and $Y$ are random variables with respect to $\mathcal{G}$.
\end{enumerate} \hfill (GATE ST 2023)\\
\solution
%\input{gate/ST/2023/14/main.tex}
	\item  A die is loaded in such a way that each odd number is twice as likely to occur as
each even number. Find $P(G)$, where $G$ is the event that a number greater than
3 occurs on a single roll of the die.
\\
\solution
		%\input{exemplar/11/16/3/5/main.tex}
	\item All the jacks, queens and kings are removed from a deck of 52 playing cards. The remaining cards are well shuffled and then one card is drawn at random. Giving ace a value 1 similar value for other cards, find the probability that the card has a value 
		\begin{enumerate}
			\item 7
			\item greater than 7
			\item less than 7
		\end{enumerate}
		%\input{exemplar/10/13/3/30/main.tex}
  \item A Lot consists of 48 mobile phones of which 42 are good, 3 have only minor defects and 3 have major defects.Varnika will buy a phone if it is good but the trader will only buy a mobile if it has no major defects. One phone is selected at random from the lot. What is the probability that it is
\begin{enumerate}
	\item acceptable to Varnika?
            \item acceptable to the trader?
\end{enumerate}
\solution
	%\input{exemplar/10/13/3/40/main.tex}
 \item A student says that if you throw a die, it will show up 1 or not 1. Therefore, the probability of getting 1 and the probability of getting 'not 1' each is equal to $\frac{1}{2}$. Is this correct? Give reasons.\\
 \solution
        %\input{exemplar/10/13/2/9/main.tex}
   \item Four candidates A, B, C, D have ap-
plied for the assignment to coach a school cricket
team. If A is twice as likely to be selected as B, and
B and C are given about the same chance of being
selected, while C is twice as likely to be selected
as D, what are the probabilities that
\begin{enumerate}
\item C will be selected?
\item A will not be selected?
\end{enumerate}
	%\input{exemplar/11/16/3/9/main.tex}
 \item A bag contain 24 balls of which $x$ balls are red, $2x$ are white and $3x$ are blue. A ball is selected at random, What is the probability that it is
\begin{enumerate}[label=\alph*)]
\item not red ?
\item white ?
\end{enumerate}
%\input{exemplar/10/13/3/41/main.tex}
If the letters of the word ASSASSINATION are arranged at random. Find the Probability that
\begin{enumerate}[label=(\alph*)]
\item Four $S's$ come consecutively in the word
\item Two  $I's$ and two $N's$ come together
\item All $A's$ are not coming together
\item No two $A's$ are coming together
\end{enumerate}
%\input{exemplar/11/16/3/14/main.tex}
	\item One urn contains two black balls (labelled B1 and B2) and one white ball. A
	second urn contains one black ball and two white balls (labelled W1 and W2).
	Suppose the following experiment is performed. One of the two urns is chosen
	at random. Next a ball is randomly chosen from the urn. Then a second ball is
	chosen at random from the same urn without replacing the first ball.
	
	\begin{enumerate}
	\item What is the probability that two black balls are chosen?
	
	\item What is the probability that two balls of opposite colour are chosen?
	\end{enumerate}
	\solution
	%\input{exemplar/11/16/3/12/main1.tex}
\end{enumerate}

		\item A box of oranges is inspected by examining three randomly selected oranges drawn without replacement. If all the three oranges are good, the box is approved for sale, otherwise, it is rejected. Find the probability that a box containing 15 oranges out of which 12 are good and 3 are bad ones will be approved for sale.
		\label{ncert/12/13/2/3/defs.tex}
		\item Two balls are drawn at random with replacement from a box containing 10 black and 8 red balls. Find the probability that
		\label{ncert/12/13/2/12}
\begin{enumerate}
\item both balls are red.
\item first ball is black and second is red.
\item one of them is black and other is red.
\end{enumerate}

\item In a hostel, 60\% of the students read Hindi newspaper, 40\% read English newspaper and 20\% read both Hindi and English newspapers. A student is selected at random.
		\label{ncert/12/13/2/15}
\begin{enumerate}
\item Find the probability that she reads neither Hindi nor English newspapers.
\item If she reads Hindi newspaper, find the probability that she reads English newspaper.
\item If she reads English newspaper, find the probability that she reads Hindi newspaper.\\
\end{enumerate}
\item The probability of obtaining an even prime number on each die, when a pair of dice is rolled is 
\begin{enumerate}
    \item $0$ 
    
    \item $\frac{1}{3}$ 
    
    \item $\frac{1}{12}$ 
    
    \item $\frac{1}{36}$ 
\end{enumerate}
\solution
		%\begin{enumerate}[label=\thesection.\arabic*,ref=\thesection.\theenumi]
	\item One card is drawn from a well-shuffled deck of 52 cards. Find the probability of getting
\begin{enumerate}
\item A king of red colour 
\item A face card 
\item A red face card
\item The jack of hearts
\item A spade
\item The queen of diamonds

\end{enumerate}
\solution
		%\input{ncert/10/15/1/14/main.tex}
	\item Five cards—the ten, jack, queen, king and ace of diamonds, are well-shuffled with their face downwards. One card is then picked up at random.
\begin{enumerate}
\item
What is the probability that the card is the queen? 
\item
If the queen is drawn and put aside, what is the probability that the second card picked up is (a) an ace? (b) a queen?\\
\end{enumerate}
\solution
		%\input{ncert/10/15/1/15/defs.tex}
	\item A bag contains $5$ red balls and some blue balls. If the probability of drawing a blue ball is double that if a red ball, determine the number of blue balls in the bag. 
		\\
\solution
		%\input{ncert/10/15/2/3/defs.tex}
	\item A card is selected from a pack of 52 cards.
 \begin{enumerate}[label=(\alph*)] 
                 \item How many points are there in the sample space?
                 \item Calculate the probability that the card is an ace of spades.
                 \item Calculate the probability that the card is (i) an ace and (ii) black card.
 \end{enumerate}
\solution
		%\input{ncert/11/16/3/4/main.tex}
\item Four cards are drawn from a well-shuffled deck of 52 cards. What is the probability of obtaining 3 diamonds and one spade.
\\
\solution
		%\input{ncert/11/16/4/2/defs.tex}
\item In a certain lottery 10,000 tickets are sold and ten equal prizes are awarded. What is the probability of not getting a prize if you buy (a) one ticket (b) two tickets (c) 10 tickets ?	
\\
\solution
		%\input{ncert/11/16/4/4/defs.tex}
		%
\item 
Out of 100 students, two sections of 40 and 60 are formed. If you and your friend are among the 100 students, what is the probability that
\begin{enumerate}
\item you both enter the same section?
\item you both enter the different sections?
\end{enumerate}
\solution
		%\input{ncert/11/16/4/5/defs.tex}
	\item 
The number lock of a suitcase has 4 wheels each labelled with ten digits i.e. from 0 to 9.The lock opens with a sequence of four digits with no repeats.What is the probability of a person getting the right sequence to open the suitcase.
\\
\solution
		%\input{ncert/11/16/4/10/defs.tex}
		%
\item 
Two cards are drawn at random and without replacement from a pack of 52 playing cards. Find the probability that both the cards are black.
\\
\solution
		%\input{ncert/12/13/2/2/defs.tex}
		\item A box of oranges is inspected by examining three randomly selected oranges drawn without replacement. If all the three oranges are good, the box is approved for sale, otherwise, it is rejected. Find the probability that a box containing 15 oranges out of which 12 are good and 3 are bad ones will be approved for sale.
		\label{ncert/12/13/2/3/defs.tex}
		\item Two balls are drawn at random with replacement from a box containing 10 black and 8 red balls. Find the probability that
		\label{ncert/12/13/2/12}
\begin{enumerate}
\item both balls are red.
\item first ball is black and second is red.
\item one of them is black and other is red.
\end{enumerate}

\item In a hostel, 60\% of the students read Hindi newspaper, 40\% read English newspaper and 20\% read both Hindi and English newspapers. A student is selected at random.
		\label{ncert/12/13/2/15}
\begin{enumerate}
\item Find the probability that she reads neither Hindi nor English newspapers.
\item If she reads Hindi newspaper, find the probability that she reads English newspaper.
\item If she reads English newspaper, find the probability that she reads Hindi newspaper.\\
\end{enumerate}
\item The probability of obtaining an even prime number on each die, when a pair of dice is rolled is 
\begin{enumerate}
    \item $0$ 
    
    \item $\frac{1}{3}$ 
    
    \item $\frac{1}{12}$ 
    
    \item $\frac{1}{36}$ 
\end{enumerate}
\solution
		%\input{ncert/12/13/2/17/defs.tex}
	\item A bag contains 4 red and 4 black balls, another bag contains 2 red and 6 black balls. One of the two bags is selected at random and a ball is drawn from the bag which is found to be red. Find the probability that the ball is drawn from the first bag.
\\
\solution
		%\input{ncert/12/13/3/2/main.tex}
  \item
  Cards with numbers 2 to 101 are placed in a box. A card is selected at random.Find the probability that the card has
\begin{enumerate}[label=(\roman*)]
	\item an even number 
	\item a square number
\end{enumerate}
\solution
%\input{exemplar/10/13/3/32/main.tex}
\item
The king, queen and jack of clubs are removed from a deck of 52 playing cards and then well shuffled. Now one card is drawn at random from the remaining cards.  Determine the probability that the card is
\begin{enumerate}[label=(\roman*)]
\item a club
\item 10 of hearts
\end{enumerate}
\solution
%\input{exemplar/10/13/3/29/main.tex}
\item A team of medical students doing their internship have to assist during surgeries
at a city hospital. The probabilities of surgeries rated as very complex, complex,
routine, simple or very simple are respectively, 0.15, 0.20, 0.31, 0.26, .08. Find
the probabilities that a particular surgery will be rated
\begin{enumerate}
	\item complex or very complex;
	\item neither very complex nor very simple;
	\item routine or complex
	\item routine or simple
\end{enumerate}
\solution
%\input{exemplar/11/16/3/8(1)/main.tex}
\item A card is selected from a pack of 52 cards.
\begin{enumerate}[label=(\alph*)]
    \item How many points are there in the sample space?
    \item Calculate the probability that the card is an ace of spades.
    \item Calculate the probability that the card is (i) an ace and (ii) black card.
\end{enumerate}
\solution
%\input{exemplar/11/16/3/4/main2.tex}
\item The probability that a non leap year selected at random will contain 53 sundays.
\\
\solution
%\input{exemplar/10/13/1/19/main.tex}
\item One of the four persons John, Rita, Aslam or Gurpreet will be promoted next
month. Consequently the sample space consists of four elementary outcomes
S = {John promoted, Rita promoted, Aslam promoted, Gurpreet promoted}
You are told that the chances of John’s promotion is same as that of Gurpreet,
Rita’s chances of promotion are twice as likely as Johns. Aslam’s chances are
four times that of John.
\begin{enumerate}
	\item Determine
	\begin{enumerate}
		\item P (John promoted)
		\item P (Rita promoted)
		\item P (Aslam promoted)
		\item P (Gurpreet promoted)
	\end{enumerate}
	\item If A = {John promoted or Gurpreet promoted}, find P (A).
\end{enumerate}
\solution
%\input{exemplar/11/16/3/10/main.tex}
\item A card is drawn from a deck of 52 cards. Find the probability of getting a king or a heart or a red card.\\
\solution
%\input{exemplar/11/16/3/15/main.tex}
\item The probability that a student will pass his examination is 0.73, the probability of
the student getting a compartment is 0.13, and the probability that the student will
either pass or get compartment is 0.96. State True or False.\\
\solution
%\input{exemplar/11/16/3/31/main.tex}
\item A card is selected from a pack of 52 cards\\
\begin{enumerate}[label=(\alph*)]
\item How many points are there in the sample space?
\item Calculate the probability that the cards is an ace of spades.
\item Calculate the probability that the card is (i) an ace (ii)black card.\\
\end{enumerate}
%\input{ncert/11/16/3/4_1/Prob_4.tex}
\item In a non-leap year, the probability of having 53 tuesdays or 53 wednesdays is\\
\solution
%\input{exemplar/11/16/3/18/main.tex}
\item There are 1000 sealed envelopes in a box, 10 of them contain a cash prize of
Rs 100 each, 100 of them contain a cash prize of Rs 50 each and 200 of them
contain a cash prize of Rs 10 each and rest do not contain any cash prize. If they
are well shuffled and an envelope is picked up out, what is the probability that it
contains no cash prize?\\
\solution
%\input{exemplar/10/13/3/34/main.tex}
\item 
A die is thrown and a card is selected at random from a deck of 52 playing cards. The probability of getting an even number on the die and a spade card.\\
\solution
%\input{exemplar/12/13/3/78/main.tex}
\item
If 4-digit numbers greater than 5,000 are randomly formed from the digits 0, 1, 3, 5, and 7, what is the probability of forming a number divisible by 5 when:
\begin{enumerate}
    \item The digits are repeated?
    \item The repetition of digits is not allowed?
\end{enumerate}
\solution
%\input{ncert/11/16/4/9/main.tex}
\item Consider the probability space $\brak{\Omega, \mathcal{G}, P}$ where $\Omega = [0,2]$ and $\mathcal{G} = \cbrak{\phi, \Omega, [0,1], (1,2]}$. Let $X$ and $Y$ be two functions on $\Omega$ defined as
\begin{align*}
    X(\omega) = 
    \begin{cases}
        1 & \text{if }\omega \in [0, 1]\\
        2 & \text{if }\omega \in (1, 2]
    \end{cases}
\end{align*}
and
\begin{align*}
    Y(\omega) = 
    \begin{cases}
        2 & \text{if }\omega \in [0, 1.5]\\
        3 & \text{if }\omega \in (1.5, 2].
    \end{cases}
\end{align*}
Then which one of the following statements is true?
\begin{enumerate}
    \item [(A)] $X$ is a random variable with respect to $\mathcal{G}$, but $Y$ is not a random variable with respect to $\mathcal{G}$.
    \item [(B)] $Y$ is a random variable with respect to $\mathcal{G}$, but $X$ is not a random variable with respect to $\mathcal{G}$.
    \item [(C)] Neither $X$ nor $Y$ is a random variable with respect to $\mathcal{G}$.
    \item [(D)] Both $X$ and $Y$ are random variables with respect to $\mathcal{G}$.
\end{enumerate} \hfill (GATE ST 2023)\\
\solution
%\input{gate/ST/2023/14/main.tex}
	\item  A die is loaded in such a way that each odd number is twice as likely to occur as
each even number. Find $P(G)$, where $G$ is the event that a number greater than
3 occurs on a single roll of the die.
\\
\solution
		%\input{exemplar/11/16/3/5/main.tex}
	\item All the jacks, queens and kings are removed from a deck of 52 playing cards. The remaining cards are well shuffled and then one card is drawn at random. Giving ace a value 1 similar value for other cards, find the probability that the card has a value 
		\begin{enumerate}
			\item 7
			\item greater than 7
			\item less than 7
		\end{enumerate}
		%\input{exemplar/10/13/3/30/main.tex}
  \item A Lot consists of 48 mobile phones of which 42 are good, 3 have only minor defects and 3 have major defects.Varnika will buy a phone if it is good but the trader will only buy a mobile if it has no major defects. One phone is selected at random from the lot. What is the probability that it is
\begin{enumerate}
	\item acceptable to Varnika?
            \item acceptable to the trader?
\end{enumerate}
\solution
	%\input{exemplar/10/13/3/40/main.tex}
 \item A student says that if you throw a die, it will show up 1 or not 1. Therefore, the probability of getting 1 and the probability of getting 'not 1' each is equal to $\frac{1}{2}$. Is this correct? Give reasons.\\
 \solution
        %\input{exemplar/10/13/2/9/main.tex}
   \item Four candidates A, B, C, D have ap-
plied for the assignment to coach a school cricket
team. If A is twice as likely to be selected as B, and
B and C are given about the same chance of being
selected, while C is twice as likely to be selected
as D, what are the probabilities that
\begin{enumerate}
\item C will be selected?
\item A will not be selected?
\end{enumerate}
	%\input{exemplar/11/16/3/9/main.tex}
 \item A bag contain 24 balls of which $x$ balls are red, $2x$ are white and $3x$ are blue. A ball is selected at random, What is the probability that it is
\begin{enumerate}[label=\alph*)]
\item not red ?
\item white ?
\end{enumerate}
%\input{exemplar/10/13/3/41/main.tex}
If the letters of the word ASSASSINATION are arranged at random. Find the Probability that
\begin{enumerate}[label=(\alph*)]
\item Four $S's$ come consecutively in the word
\item Two  $I's$ and two $N's$ come together
\item All $A's$ are not coming together
\item No two $A's$ are coming together
\end{enumerate}
%\input{exemplar/11/16/3/14/main.tex}
	\item One urn contains two black balls (labelled B1 and B2) and one white ball. A
	second urn contains one black ball and two white balls (labelled W1 and W2).
	Suppose the following experiment is performed. One of the two urns is chosen
	at random. Next a ball is randomly chosen from the urn. Then a second ball is
	chosen at random from the same urn without replacing the first ball.
	
	\begin{enumerate}
	\item What is the probability that two black balls are chosen?
	
	\item What is the probability that two balls of opposite colour are chosen?
	\end{enumerate}
	\solution
	%\input{exemplar/11/16/3/12/main1.tex}
\end{enumerate}

	\item A bag contains 4 red and 4 black balls, another bag contains 2 red and 6 black balls. One of the two bags is selected at random and a ball is drawn from the bag which is found to be red. Find the probability that the ball is drawn from the first bag.
\\
\solution
		%\begin{table}[H]
	\centering
\begin{tabular}{|c|c|c|}
\hline
Random variable &Value &Definition\\ \hline
\multirow{3}{*}{X} &0 &Slips of Rs 1\\
&1 &Slips of Rs 5\\
&2 &Slips of Rs 13\\ \hline
\multirow{2}{*}{Y} &0 &Box A\\
&1 &Box B\\\hline
\end{tabular}
\caption{}
\label{tab:Distribution}
\end{table}
See \tabref{tab:Distribution}.
\begin{align}
p_{Y}\brak{k}= \begin{cases} 
      \frac{1}{3} & {k=0} \\
      \frac{2}{3 }& {k=1} 
   \end{cases}
   \\
p_{Y|X}\brak{0|0} = \frac{19}{25}\, 
p_{Y|X}\brak{0|1} = \frac{6}{25}\,
p_{Y|X}\brak{1|0} = \frac{45}{50}\,
p_{Y|X}\brak{1|2} = \frac{5}{50}
\end{align}
The desired probability is the probability that a slip drawn at random is marked other than Rs 1,
\begin{align}
&=1-p_X\brak{0}\\
&= p_X(1) + p_X(2)
\end{align}
Using Bayes theorem,
\begin{align}
&= p_Y\brak{0} \times \pr{Y=0 | X=1} + p_Y\brak{1} \times \pr{Y=1|X=2}\\
&=\frac{1}{3} \times \frac{6}{25} + \frac{2}{3} \times \frac{5}{50}\\
&=\frac{11}{75}
\end{align}

\newpage

%\tableofcontents

\bigskip

\renewcommand{\thefigure}{\theenumi}
\renewcommand{\thetable}{\theenumi}
%\renewcommand{\theequation}{\theenumi}

%\begin{abstract}
%%\boldmath
%In this letter, an algorithm for evaluating the exact analytical bit error rate  (BER)  for the piecewise linear (PL) combiner for  multiple relays is presented. Previous results were available only for upto three relays. The algorithm is unique in the sense that  the actual mathematical expressions, that are prohibitively large, need not be explicitly obtained. The diversity gain due to multiple relays is shown through plots of the analytical BER, well supported by simulations. 
%
%\end{abstract}
% IEEEtran.cls defaults to using nonbold math in the Abstract.
% This preserves the distinction between vectors and scalars. However,
% if the journal you are submitting to favors bold math in the abstract,
% then you can use LaTeX's standard command \boldmath at the very start
% of the abstract to achieve this. Many IEEE journals frown on math
% in the abstract anyway.

% Note that keywords are not normally used for peerreview papers.
%\begin{IEEEkeywords}
%Cooperative diversity, decode and forward, piecewise linear
%\end{IEEEkeywords}



% For peer review papers, you can put extra information on the cover
% page as needed:
% \ifCLASSOPTIONpeerreview
% \begin{center} \bfseries EDICS Category: 3-BBND \end{center}
% \fi
%
% For peerreview papers, this IEEEtran command inserts a page break and
% creates the second title. It will be ignored for other modes.
%\IEEEpeerreviewmaketitle




  \item
  Cards with numbers 2 to 101 are placed in a box. A card is selected at random.Find the probability that the card has
\begin{enumerate}[label=(\roman*)]
	\item an even number 
	\item a square number
\end{enumerate}
\solution
%\begin{table}[H]
	\centering
\begin{tabular}{|c|c|c|}
\hline
Random variable &Value &Definition\\ \hline
\multirow{3}{*}{X} &0 &Slips of Rs 1\\
&1 &Slips of Rs 5\\
&2 &Slips of Rs 13\\ \hline
\multirow{2}{*}{Y} &0 &Box A\\
&1 &Box B\\\hline
\end{tabular}
\caption{}
\label{tab:Distribution}
\end{table}
See \tabref{tab:Distribution}.
\begin{align}
p_{Y}\brak{k}= \begin{cases} 
      \frac{1}{3} & {k=0} \\
      \frac{2}{3 }& {k=1} 
   \end{cases}
   \\
p_{Y|X}\brak{0|0} = \frac{19}{25}\, 
p_{Y|X}\brak{0|1} = \frac{6}{25}\,
p_{Y|X}\brak{1|0} = \frac{45}{50}\,
p_{Y|X}\brak{1|2} = \frac{5}{50}
\end{align}
The desired probability is the probability that a slip drawn at random is marked other than Rs 1,
\begin{align}
&=1-p_X\brak{0}\\
&= p_X(1) + p_X(2)
\end{align}
Using Bayes theorem,
\begin{align}
&= p_Y\brak{0} \times \pr{Y=0 | X=1} + p_Y\brak{1} \times \pr{Y=1|X=2}\\
&=\frac{1}{3} \times \frac{6}{25} + \frac{2}{3} \times \frac{5}{50}\\
&=\frac{11}{75}
\end{align}

\newpage

%\tableofcontents

\bigskip

\renewcommand{\thefigure}{\theenumi}
\renewcommand{\thetable}{\theenumi}
%\renewcommand{\theequation}{\theenumi}

%\begin{abstract}
%%\boldmath
%In this letter, an algorithm for evaluating the exact analytical bit error rate  (BER)  for the piecewise linear (PL) combiner for  multiple relays is presented. Previous results were available only for upto three relays. The algorithm is unique in the sense that  the actual mathematical expressions, that are prohibitively large, need not be explicitly obtained. The diversity gain due to multiple relays is shown through plots of the analytical BER, well supported by simulations. 
%
%\end{abstract}
% IEEEtran.cls defaults to using nonbold math in the Abstract.
% This preserves the distinction between vectors and scalars. However,
% if the journal you are submitting to favors bold math in the abstract,
% then you can use LaTeX's standard command \boldmath at the very start
% of the abstract to achieve this. Many IEEE journals frown on math
% in the abstract anyway.

% Note that keywords are not normally used for peerreview papers.
%\begin{IEEEkeywords}
%Cooperative diversity, decode and forward, piecewise linear
%\end{IEEEkeywords}



% For peer review papers, you can put extra information on the cover
% page as needed:
% \ifCLASSOPTIONpeerreview
% \begin{center} \bfseries EDICS Category: 3-BBND \end{center}
% \fi
%
% For peerreview papers, this IEEEtran command inserts a page break and
% creates the second title. It will be ignored for other modes.
%\IEEEpeerreviewmaketitle




\item
The king, queen and jack of clubs are removed from a deck of 52 playing cards and then well shuffled. Now one card is drawn at random from the remaining cards.  Determine the probability that the card is
\begin{enumerate}[label=(\roman*)]
\item a club
\item 10 of hearts
\end{enumerate}
\solution
%\begin{table}[H]
	\centering
\begin{tabular}{|c|c|c|}
\hline
Random variable &Value &Definition\\ \hline
\multirow{3}{*}{X} &0 &Slips of Rs 1\\
&1 &Slips of Rs 5\\
&2 &Slips of Rs 13\\ \hline
\multirow{2}{*}{Y} &0 &Box A\\
&1 &Box B\\\hline
\end{tabular}
\caption{}
\label{tab:Distribution}
\end{table}
See \tabref{tab:Distribution}.
\begin{align}
p_{Y}\brak{k}= \begin{cases} 
      \frac{1}{3} & {k=0} \\
      \frac{2}{3 }& {k=1} 
   \end{cases}
   \\
p_{Y|X}\brak{0|0} = \frac{19}{25}\, 
p_{Y|X}\brak{0|1} = \frac{6}{25}\,
p_{Y|X}\brak{1|0} = \frac{45}{50}\,
p_{Y|X}\brak{1|2} = \frac{5}{50}
\end{align}
The desired probability is the probability that a slip drawn at random is marked other than Rs 1,
\begin{align}
&=1-p_X\brak{0}\\
&= p_X(1) + p_X(2)
\end{align}
Using Bayes theorem,
\begin{align}
&= p_Y\brak{0} \times \pr{Y=0 | X=1} + p_Y\brak{1} \times \pr{Y=1|X=2}\\
&=\frac{1}{3} \times \frac{6}{25} + \frac{2}{3} \times \frac{5}{50}\\
&=\frac{11}{75}
\end{align}

\newpage

%\tableofcontents

\bigskip

\renewcommand{\thefigure}{\theenumi}
\renewcommand{\thetable}{\theenumi}
%\renewcommand{\theequation}{\theenumi}

%\begin{abstract}
%%\boldmath
%In this letter, an algorithm for evaluating the exact analytical bit error rate  (BER)  for the piecewise linear (PL) combiner for  multiple relays is presented. Previous results were available only for upto three relays. The algorithm is unique in the sense that  the actual mathematical expressions, that are prohibitively large, need not be explicitly obtained. The diversity gain due to multiple relays is shown through plots of the analytical BER, well supported by simulations. 
%
%\end{abstract}
% IEEEtran.cls defaults to using nonbold math in the Abstract.
% This preserves the distinction between vectors and scalars. However,
% if the journal you are submitting to favors bold math in the abstract,
% then you can use LaTeX's standard command \boldmath at the very start
% of the abstract to achieve this. Many IEEE journals frown on math
% in the abstract anyway.

% Note that keywords are not normally used for peerreview papers.
%\begin{IEEEkeywords}
%Cooperative diversity, decode and forward, piecewise linear
%\end{IEEEkeywords}



% For peer review papers, you can put extra information on the cover
% page as needed:
% \ifCLASSOPTIONpeerreview
% \begin{center} \bfseries EDICS Category: 3-BBND \end{center}
% \fi
%
% For peerreview papers, this IEEEtran command inserts a page break and
% creates the second title. It will be ignored for other modes.
%\IEEEpeerreviewmaketitle




\item A team of medical students doing their internship have to assist during surgeries
at a city hospital. The probabilities of surgeries rated as very complex, complex,
routine, simple or very simple are respectively, 0.15, 0.20, 0.31, 0.26, .08. Find
the probabilities that a particular surgery will be rated
\begin{enumerate}
	\item complex or very complex;
	\item neither very complex nor very simple;
	\item routine or complex
	\item routine or simple
\end{enumerate}
\solution
%\begin{table}[H]
	\centering
\begin{tabular}{|c|c|c|}
\hline
Random variable &Value &Definition\\ \hline
\multirow{3}{*}{X} &0 &Slips of Rs 1\\
&1 &Slips of Rs 5\\
&2 &Slips of Rs 13\\ \hline
\multirow{2}{*}{Y} &0 &Box A\\
&1 &Box B\\\hline
\end{tabular}
\caption{}
\label{tab:Distribution}
\end{table}
See \tabref{tab:Distribution}.
\begin{align}
p_{Y}\brak{k}= \begin{cases} 
      \frac{1}{3} & {k=0} \\
      \frac{2}{3 }& {k=1} 
   \end{cases}
   \\
p_{Y|X}\brak{0|0} = \frac{19}{25}\, 
p_{Y|X}\brak{0|1} = \frac{6}{25}\,
p_{Y|X}\brak{1|0} = \frac{45}{50}\,
p_{Y|X}\brak{1|2} = \frac{5}{50}
\end{align}
The desired probability is the probability that a slip drawn at random is marked other than Rs 1,
\begin{align}
&=1-p_X\brak{0}\\
&= p_X(1) + p_X(2)
\end{align}
Using Bayes theorem,
\begin{align}
&= p_Y\brak{0} \times \pr{Y=0 | X=1} + p_Y\brak{1} \times \pr{Y=1|X=2}\\
&=\frac{1}{3} \times \frac{6}{25} + \frac{2}{3} \times \frac{5}{50}\\
&=\frac{11}{75}
\end{align}

\newpage

%\tableofcontents

\bigskip

\renewcommand{\thefigure}{\theenumi}
\renewcommand{\thetable}{\theenumi}
%\renewcommand{\theequation}{\theenumi}

%\begin{abstract}
%%\boldmath
%In this letter, an algorithm for evaluating the exact analytical bit error rate  (BER)  for the piecewise linear (PL) combiner for  multiple relays is presented. Previous results were available only for upto three relays. The algorithm is unique in the sense that  the actual mathematical expressions, that are prohibitively large, need not be explicitly obtained. The diversity gain due to multiple relays is shown through plots of the analytical BER, well supported by simulations. 
%
%\end{abstract}
% IEEEtran.cls defaults to using nonbold math in the Abstract.
% This preserves the distinction between vectors and scalars. However,
% if the journal you are submitting to favors bold math in the abstract,
% then you can use LaTeX's standard command \boldmath at the very start
% of the abstract to achieve this. Many IEEE journals frown on math
% in the abstract anyway.

% Note that keywords are not normally used for peerreview papers.
%\begin{IEEEkeywords}
%Cooperative diversity, decode and forward, piecewise linear
%\end{IEEEkeywords}



% For peer review papers, you can put extra information on the cover
% page as needed:
% \ifCLASSOPTIONpeerreview
% \begin{center} \bfseries EDICS Category: 3-BBND \end{center}
% \fi
%
% For peerreview papers, this IEEEtran command inserts a page break and
% creates the second title. It will be ignored for other modes.
%\IEEEpeerreviewmaketitle




\item A card is selected from a pack of 52 cards.
\begin{enumerate}[label=(\alph*)]
    \item How many points are there in the sample space?
    \item Calculate the probability that the card is an ace of spades.
    \item Calculate the probability that the card is (i) an ace and (ii) black card.
\end{enumerate}
\solution
%Let $X$ be an bernoulli rv defined as in \tabref{tab:exemplar/11/16/3/26}.  Then, 
\begin{equation}
    p =
        \frac{4}{11} 
\end{equation}
\begin{table}[H]
	\centering
	\input{exemplar/11/16/3/26/tables/Table2.tex}
	\caption{}
        \label{tab:exemplar/11/16/3/26}
\end{table}

\item The probability that a non leap year selected at random will contain 53 sundays.
\\
\solution
%\begin{table}[H]
	\centering
\begin{tabular}{|c|c|c|}
\hline
Random variable &Value &Definition\\ \hline
\multirow{3}{*}{X} &0 &Slips of Rs 1\\
&1 &Slips of Rs 5\\
&2 &Slips of Rs 13\\ \hline
\multirow{2}{*}{Y} &0 &Box A\\
&1 &Box B\\\hline
\end{tabular}
\caption{}
\label{tab:Distribution}
\end{table}
See \tabref{tab:Distribution}.
\begin{align}
p_{Y}\brak{k}= \begin{cases} 
      \frac{1}{3} & {k=0} \\
      \frac{2}{3 }& {k=1} 
   \end{cases}
   \\
p_{Y|X}\brak{0|0} = \frac{19}{25}\, 
p_{Y|X}\brak{0|1} = \frac{6}{25}\,
p_{Y|X}\brak{1|0} = \frac{45}{50}\,
p_{Y|X}\brak{1|2} = \frac{5}{50}
\end{align}
The desired probability is the probability that a slip drawn at random is marked other than Rs 1,
\begin{align}
&=1-p_X\brak{0}\\
&= p_X(1) + p_X(2)
\end{align}
Using Bayes theorem,
\begin{align}
&= p_Y\brak{0} \times \pr{Y=0 | X=1} + p_Y\brak{1} \times \pr{Y=1|X=2}\\
&=\frac{1}{3} \times \frac{6}{25} + \frac{2}{3} \times \frac{5}{50}\\
&=\frac{11}{75}
\end{align}

\newpage

%\tableofcontents

\bigskip

\renewcommand{\thefigure}{\theenumi}
\renewcommand{\thetable}{\theenumi}
%\renewcommand{\theequation}{\theenumi}

%\begin{abstract}
%%\boldmath
%In this letter, an algorithm for evaluating the exact analytical bit error rate  (BER)  for the piecewise linear (PL) combiner for  multiple relays is presented. Previous results were available only for upto three relays. The algorithm is unique in the sense that  the actual mathematical expressions, that are prohibitively large, need not be explicitly obtained. The diversity gain due to multiple relays is shown through plots of the analytical BER, well supported by simulations. 
%
%\end{abstract}
% IEEEtran.cls defaults to using nonbold math in the Abstract.
% This preserves the distinction between vectors and scalars. However,
% if the journal you are submitting to favors bold math in the abstract,
% then you can use LaTeX's standard command \boldmath at the very start
% of the abstract to achieve this. Many IEEE journals frown on math
% in the abstract anyway.

% Note that keywords are not normally used for peerreview papers.
%\begin{IEEEkeywords}
%Cooperative diversity, decode and forward, piecewise linear
%\end{IEEEkeywords}



% For peer review papers, you can put extra information on the cover
% page as needed:
% \ifCLASSOPTIONpeerreview
% \begin{center} \bfseries EDICS Category: 3-BBND \end{center}
% \fi
%
% For peerreview papers, this IEEEtran command inserts a page break and
% creates the second title. It will be ignored for other modes.
%\IEEEpeerreviewmaketitle




\item One of the four persons John, Rita, Aslam or Gurpreet will be promoted next
month. Consequently the sample space consists of four elementary outcomes
S = {John promoted, Rita promoted, Aslam promoted, Gurpreet promoted}
You are told that the chances of John’s promotion is same as that of Gurpreet,
Rita’s chances of promotion are twice as likely as Johns. Aslam’s chances are
four times that of John.
\begin{enumerate}
	\item Determine
	\begin{enumerate}
		\item P (John promoted)
		\item P (Rita promoted)
		\item P (Aslam promoted)
		\item P (Gurpreet promoted)
	\end{enumerate}
	\item If A = {John promoted or Gurpreet promoted}, find P (A).
\end{enumerate}
\solution
%\begin{table}[H]
	\centering
\begin{tabular}{|c|c|c|}
\hline
Random variable &Value &Definition\\ \hline
\multirow{3}{*}{X} &0 &Slips of Rs 1\\
&1 &Slips of Rs 5\\
&2 &Slips of Rs 13\\ \hline
\multirow{2}{*}{Y} &0 &Box A\\
&1 &Box B\\\hline
\end{tabular}
\caption{}
\label{tab:Distribution}
\end{table}
See \tabref{tab:Distribution}.
\begin{align}
p_{Y}\brak{k}= \begin{cases} 
      \frac{1}{3} & {k=0} \\
      \frac{2}{3 }& {k=1} 
   \end{cases}
   \\
p_{Y|X}\brak{0|0} = \frac{19}{25}\, 
p_{Y|X}\brak{0|1} = \frac{6}{25}\,
p_{Y|X}\brak{1|0} = \frac{45}{50}\,
p_{Y|X}\brak{1|2} = \frac{5}{50}
\end{align}
The desired probability is the probability that a slip drawn at random is marked other than Rs 1,
\begin{align}
&=1-p_X\brak{0}\\
&= p_X(1) + p_X(2)
\end{align}
Using Bayes theorem,
\begin{align}
&= p_Y\brak{0} \times \pr{Y=0 | X=1} + p_Y\brak{1} \times \pr{Y=1|X=2}\\
&=\frac{1}{3} \times \frac{6}{25} + \frac{2}{3} \times \frac{5}{50}\\
&=\frac{11}{75}
\end{align}

\newpage

%\tableofcontents

\bigskip

\renewcommand{\thefigure}{\theenumi}
\renewcommand{\thetable}{\theenumi}
%\renewcommand{\theequation}{\theenumi}

%\begin{abstract}
%%\boldmath
%In this letter, an algorithm for evaluating the exact analytical bit error rate  (BER)  for the piecewise linear (PL) combiner for  multiple relays is presented. Previous results were available only for upto three relays. The algorithm is unique in the sense that  the actual mathematical expressions, that are prohibitively large, need not be explicitly obtained. The diversity gain due to multiple relays is shown through plots of the analytical BER, well supported by simulations. 
%
%\end{abstract}
% IEEEtran.cls defaults to using nonbold math in the Abstract.
% This preserves the distinction between vectors and scalars. However,
% if the journal you are submitting to favors bold math in the abstract,
% then you can use LaTeX's standard command \boldmath at the very start
% of the abstract to achieve this. Many IEEE journals frown on math
% in the abstract anyway.

% Note that keywords are not normally used for peerreview papers.
%\begin{IEEEkeywords}
%Cooperative diversity, decode and forward, piecewise linear
%\end{IEEEkeywords}



% For peer review papers, you can put extra information on the cover
% page as needed:
% \ifCLASSOPTIONpeerreview
% \begin{center} \bfseries EDICS Category: 3-BBND \end{center}
% \fi
%
% For peerreview papers, this IEEEtran command inserts a page break and
% creates the second title. It will be ignored for other modes.
%\IEEEpeerreviewmaketitle




\item A card is drawn from a deck of 52 cards. Find the probability of getting a king or a heart or a red card.\\
\solution
%\begin{table}[H]
	\centering
\begin{tabular}{|c|c|c|}
\hline
Random variable &Value &Definition\\ \hline
\multirow{3}{*}{X} &0 &Slips of Rs 1\\
&1 &Slips of Rs 5\\
&2 &Slips of Rs 13\\ \hline
\multirow{2}{*}{Y} &0 &Box A\\
&1 &Box B\\\hline
\end{tabular}
\caption{}
\label{tab:Distribution}
\end{table}
See \tabref{tab:Distribution}.
\begin{align}
p_{Y}\brak{k}= \begin{cases} 
      \frac{1}{3} & {k=0} \\
      \frac{2}{3 }& {k=1} 
   \end{cases}
   \\
p_{Y|X}\brak{0|0} = \frac{19}{25}\, 
p_{Y|X}\brak{0|1} = \frac{6}{25}\,
p_{Y|X}\brak{1|0} = \frac{45}{50}\,
p_{Y|X}\brak{1|2} = \frac{5}{50}
\end{align}
The desired probability is the probability that a slip drawn at random is marked other than Rs 1,
\begin{align}
&=1-p_X\brak{0}\\
&= p_X(1) + p_X(2)
\end{align}
Using Bayes theorem,
\begin{align}
&= p_Y\brak{0} \times \pr{Y=0 | X=1} + p_Y\brak{1} \times \pr{Y=1|X=2}\\
&=\frac{1}{3} \times \frac{6}{25} + \frac{2}{3} \times \frac{5}{50}\\
&=\frac{11}{75}
\end{align}

\newpage

%\tableofcontents

\bigskip

\renewcommand{\thefigure}{\theenumi}
\renewcommand{\thetable}{\theenumi}
%\renewcommand{\theequation}{\theenumi}

%\begin{abstract}
%%\boldmath
%In this letter, an algorithm for evaluating the exact analytical bit error rate  (BER)  for the piecewise linear (PL) combiner for  multiple relays is presented. Previous results were available only for upto three relays. The algorithm is unique in the sense that  the actual mathematical expressions, that are prohibitively large, need not be explicitly obtained. The diversity gain due to multiple relays is shown through plots of the analytical BER, well supported by simulations. 
%
%\end{abstract}
% IEEEtran.cls defaults to using nonbold math in the Abstract.
% This preserves the distinction between vectors and scalars. However,
% if the journal you are submitting to favors bold math in the abstract,
% then you can use LaTeX's standard command \boldmath at the very start
% of the abstract to achieve this. Many IEEE journals frown on math
% in the abstract anyway.

% Note that keywords are not normally used for peerreview papers.
%\begin{IEEEkeywords}
%Cooperative diversity, decode and forward, piecewise linear
%\end{IEEEkeywords}



% For peer review papers, you can put extra information on the cover
% page as needed:
% \ifCLASSOPTIONpeerreview
% \begin{center} \bfseries EDICS Category: 3-BBND \end{center}
% \fi
%
% For peerreview papers, this IEEEtran command inserts a page break and
% creates the second title. It will be ignored for other modes.
%\IEEEpeerreviewmaketitle




\item The probability that a student will pass his examination is 0.73, the probability of
the student getting a compartment is 0.13, and the probability that the student will
either pass or get compartment is 0.96. State True or False.\\
\solution
%\begin{table}[H]
	\centering
\begin{tabular}{|c|c|c|}
\hline
Random variable &Value &Definition\\ \hline
\multirow{3}{*}{X} &0 &Slips of Rs 1\\
&1 &Slips of Rs 5\\
&2 &Slips of Rs 13\\ \hline
\multirow{2}{*}{Y} &0 &Box A\\
&1 &Box B\\\hline
\end{tabular}
\caption{}
\label{tab:Distribution}
\end{table}
See \tabref{tab:Distribution}.
\begin{align}
p_{Y}\brak{k}= \begin{cases} 
      \frac{1}{3} & {k=0} \\
      \frac{2}{3 }& {k=1} 
   \end{cases}
   \\
p_{Y|X}\brak{0|0} = \frac{19}{25}\, 
p_{Y|X}\brak{0|1} = \frac{6}{25}\,
p_{Y|X}\brak{1|0} = \frac{45}{50}\,
p_{Y|X}\brak{1|2} = \frac{5}{50}
\end{align}
The desired probability is the probability that a slip drawn at random is marked other than Rs 1,
\begin{align}
&=1-p_X\brak{0}\\
&= p_X(1) + p_X(2)
\end{align}
Using Bayes theorem,
\begin{align}
&= p_Y\brak{0} \times \pr{Y=0 | X=1} + p_Y\brak{1} \times \pr{Y=1|X=2}\\
&=\frac{1}{3} \times \frac{6}{25} + \frac{2}{3} \times \frac{5}{50}\\
&=\frac{11}{75}
\end{align}

\newpage

%\tableofcontents

\bigskip

\renewcommand{\thefigure}{\theenumi}
\renewcommand{\thetable}{\theenumi}
%\renewcommand{\theequation}{\theenumi}

%\begin{abstract}
%%\boldmath
%In this letter, an algorithm for evaluating the exact analytical bit error rate  (BER)  for the piecewise linear (PL) combiner for  multiple relays is presented. Previous results were available only for upto three relays. The algorithm is unique in the sense that  the actual mathematical expressions, that are prohibitively large, need not be explicitly obtained. The diversity gain due to multiple relays is shown through plots of the analytical BER, well supported by simulations. 
%
%\end{abstract}
% IEEEtran.cls defaults to using nonbold math in the Abstract.
% This preserves the distinction between vectors and scalars. However,
% if the journal you are submitting to favors bold math in the abstract,
% then you can use LaTeX's standard command \boldmath at the very start
% of the abstract to achieve this. Many IEEE journals frown on math
% in the abstract anyway.

% Note that keywords are not normally used for peerreview papers.
%\begin{IEEEkeywords}
%Cooperative diversity, decode and forward, piecewise linear
%\end{IEEEkeywords}



% For peer review papers, you can put extra information on the cover
% page as needed:
% \ifCLASSOPTIONpeerreview
% \begin{center} \bfseries EDICS Category: 3-BBND \end{center}
% \fi
%
% For peerreview papers, this IEEEtran command inserts a page break and
% creates the second title. It will be ignored for other modes.
%\IEEEpeerreviewmaketitle




\item A card is selected from a pack of 52 cards\\
\begin{enumerate}[label=(\alph*)]
\item How many points are there in the sample space?
\item Calculate the probability that the cards is an ace of spades.
\item Calculate the probability that the card is (i) an ace (ii)black card.\\
\end{enumerate}
%\input{ncert/11/16/3/4_1/Prob_4.tex}
\item In a non-leap year, the probability of having 53 tuesdays or 53 wednesdays is\\
\solution
%A non-leap year has a total of 365 days, and a week has 7 days.\\
So it can be expressed as 
\begin{align}
365\text{days} &=52\times 7+1 \text{day}
\end{align}
$\implies$ 52 tuesdays or wednesdays\\
Random variable X denotes the days of a week
\begin{align}
p_X\brak{k}&=\frac{1}{7}; \quad \brak{1<k<7}
\end{align}
So the probability of extra day being tuesday or wednesday is
\begin{align}
p_X\brak{3}+p_X\brak{4}&=\frac{1}{7}+\frac{1}{7}=\frac{2}{7}
\end{align}



\item There are 1000 sealed envelopes in a box, 10 of them contain a cash prize of
Rs 100 each, 100 of them contain a cash prize of Rs 50 each and 200 of them
contain a cash prize of Rs 10 each and rest do not contain any cash prize. If they
are well shuffled and an envelope is picked up out, what is the probability that it
contains no cash prize?\\
\solution
%\begin{table}[H]
	\centering
\begin{tabular}{|c|c|c|}
\hline
Random variable &Value &Definition\\ \hline
\multirow{3}{*}{X} &0 &Slips of Rs 1\\
&1 &Slips of Rs 5\\
&2 &Slips of Rs 13\\ \hline
\multirow{2}{*}{Y} &0 &Box A\\
&1 &Box B\\\hline
\end{tabular}
\caption{}
\label{tab:Distribution}
\end{table}
See \tabref{tab:Distribution}.
\begin{align}
p_{Y}\brak{k}= \begin{cases} 
      \frac{1}{3} & {k=0} \\
      \frac{2}{3 }& {k=1} 
   \end{cases}
   \\
p_{Y|X}\brak{0|0} = \frac{19}{25}\, 
p_{Y|X}\brak{0|1} = \frac{6}{25}\,
p_{Y|X}\brak{1|0} = \frac{45}{50}\,
p_{Y|X}\brak{1|2} = \frac{5}{50}
\end{align}
The desired probability is the probability that a slip drawn at random is marked other than Rs 1,
\begin{align}
&=1-p_X\brak{0}\\
&= p_X(1) + p_X(2)
\end{align}
Using Bayes theorem,
\begin{align}
&= p_Y\brak{0} \times \pr{Y=0 | X=1} + p_Y\brak{1} \times \pr{Y=1|X=2}\\
&=\frac{1}{3} \times \frac{6}{25} + \frac{2}{3} \times \frac{5}{50}\\
&=\frac{11}{75}
\end{align}

\newpage

%\tableofcontents

\bigskip

\renewcommand{\thefigure}{\theenumi}
\renewcommand{\thetable}{\theenumi}
%\renewcommand{\theequation}{\theenumi}

%\begin{abstract}
%%\boldmath
%In this letter, an algorithm for evaluating the exact analytical bit error rate  (BER)  for the piecewise linear (PL) combiner for  multiple relays is presented. Previous results were available only for upto three relays. The algorithm is unique in the sense that  the actual mathematical expressions, that are prohibitively large, need not be explicitly obtained. The diversity gain due to multiple relays is shown through plots of the analytical BER, well supported by simulations. 
%
%\end{abstract}
% IEEEtran.cls defaults to using nonbold math in the Abstract.
% This preserves the distinction between vectors and scalars. However,
% if the journal you are submitting to favors bold math in the abstract,
% then you can use LaTeX's standard command \boldmath at the very start
% of the abstract to achieve this. Many IEEE journals frown on math
% in the abstract anyway.

% Note that keywords are not normally used for peerreview papers.
%\begin{IEEEkeywords}
%Cooperative diversity, decode and forward, piecewise linear
%\end{IEEEkeywords}



% For peer review papers, you can put extra information on the cover
% page as needed:
% \ifCLASSOPTIONpeerreview
% \begin{center} \bfseries EDICS Category: 3-BBND \end{center}
% \fi
%
% For peerreview papers, this IEEEtran command inserts a page break and
% creates the second title. It will be ignored for other modes.
%\IEEEpeerreviewmaketitle




\item 
A die is thrown and a card is selected at random from a deck of 52 playing cards. The probability of getting an even number on the die and a spade card.\\
\solution
%\begin{table}[H]
	\centering
\begin{tabular}{|c|c|c|}
\hline
Random variable &Value &Definition\\ \hline
\multirow{3}{*}{X} &0 &Slips of Rs 1\\
&1 &Slips of Rs 5\\
&2 &Slips of Rs 13\\ \hline
\multirow{2}{*}{Y} &0 &Box A\\
&1 &Box B\\\hline
\end{tabular}
\caption{}
\label{tab:Distribution}
\end{table}
See \tabref{tab:Distribution}.
\begin{align}
p_{Y}\brak{k}= \begin{cases} 
      \frac{1}{3} & {k=0} \\
      \frac{2}{3 }& {k=1} 
   \end{cases}
   \\
p_{Y|X}\brak{0|0} = \frac{19}{25}\, 
p_{Y|X}\brak{0|1} = \frac{6}{25}\,
p_{Y|X}\brak{1|0} = \frac{45}{50}\,
p_{Y|X}\brak{1|2} = \frac{5}{50}
\end{align}
The desired probability is the probability that a slip drawn at random is marked other than Rs 1,
\begin{align}
&=1-p_X\brak{0}\\
&= p_X(1) + p_X(2)
\end{align}
Using Bayes theorem,
\begin{align}
&= p_Y\brak{0} \times \pr{Y=0 | X=1} + p_Y\brak{1} \times \pr{Y=1|X=2}\\
&=\frac{1}{3} \times \frac{6}{25} + \frac{2}{3} \times \frac{5}{50}\\
&=\frac{11}{75}
\end{align}

\newpage

%\tableofcontents

\bigskip

\renewcommand{\thefigure}{\theenumi}
\renewcommand{\thetable}{\theenumi}
%\renewcommand{\theequation}{\theenumi}

%\begin{abstract}
%%\boldmath
%In this letter, an algorithm for evaluating the exact analytical bit error rate  (BER)  for the piecewise linear (PL) combiner for  multiple relays is presented. Previous results were available only for upto three relays. The algorithm is unique in the sense that  the actual mathematical expressions, that are prohibitively large, need not be explicitly obtained. The diversity gain due to multiple relays is shown through plots of the analytical BER, well supported by simulations. 
%
%\end{abstract}
% IEEEtran.cls defaults to using nonbold math in the Abstract.
% This preserves the distinction between vectors and scalars. However,
% if the journal you are submitting to favors bold math in the abstract,
% then you can use LaTeX's standard command \boldmath at the very start
% of the abstract to achieve this. Many IEEE journals frown on math
% in the abstract anyway.

% Note that keywords are not normally used for peerreview papers.
%\begin{IEEEkeywords}
%Cooperative diversity, decode and forward, piecewise linear
%\end{IEEEkeywords}



% For peer review papers, you can put extra information on the cover
% page as needed:
% \ifCLASSOPTIONpeerreview
% \begin{center} \bfseries EDICS Category: 3-BBND \end{center}
% \fi
%
% For peerreview papers, this IEEEtran command inserts a page break and
% creates the second title. It will be ignored for other modes.
%\IEEEpeerreviewmaketitle




\item
If 4-digit numbers greater than 5,000 are randomly formed from the digits 0, 1, 3, 5, and 7, what is the probability of forming a number divisible by 5 when:
\begin{enumerate}
    \item The digits are repeated?
    \item The repetition of digits is not allowed?
\end{enumerate}
\solution
%\begin{table}[H]
	\centering
\begin{tabular}{|c|c|c|}
\hline
Random variable &Value &Definition\\ \hline
\multirow{3}{*}{X} &0 &Slips of Rs 1\\
&1 &Slips of Rs 5\\
&2 &Slips of Rs 13\\ \hline
\multirow{2}{*}{Y} &0 &Box A\\
&1 &Box B\\\hline
\end{tabular}
\caption{}
\label{tab:Distribution}
\end{table}
See \tabref{tab:Distribution}.
\begin{align}
p_{Y}\brak{k}= \begin{cases} 
      \frac{1}{3} & {k=0} \\
      \frac{2}{3 }& {k=1} 
   \end{cases}
   \\
p_{Y|X}\brak{0|0} = \frac{19}{25}\, 
p_{Y|X}\brak{0|1} = \frac{6}{25}\,
p_{Y|X}\brak{1|0} = \frac{45}{50}\,
p_{Y|X}\brak{1|2} = \frac{5}{50}
\end{align}
The desired probability is the probability that a slip drawn at random is marked other than Rs 1,
\begin{align}
&=1-p_X\brak{0}\\
&= p_X(1) + p_X(2)
\end{align}
Using Bayes theorem,
\begin{align}
&= p_Y\brak{0} \times \pr{Y=0 | X=1} + p_Y\brak{1} \times \pr{Y=1|X=2}\\
&=\frac{1}{3} \times \frac{6}{25} + \frac{2}{3} \times \frac{5}{50}\\
&=\frac{11}{75}
\end{align}

\newpage

%\tableofcontents

\bigskip

\renewcommand{\thefigure}{\theenumi}
\renewcommand{\thetable}{\theenumi}
%\renewcommand{\theequation}{\theenumi}

%\begin{abstract}
%%\boldmath
%In this letter, an algorithm for evaluating the exact analytical bit error rate  (BER)  for the piecewise linear (PL) combiner for  multiple relays is presented. Previous results were available only for upto three relays. The algorithm is unique in the sense that  the actual mathematical expressions, that are prohibitively large, need not be explicitly obtained. The diversity gain due to multiple relays is shown through plots of the analytical BER, well supported by simulations. 
%
%\end{abstract}
% IEEEtran.cls defaults to using nonbold math in the Abstract.
% This preserves the distinction between vectors and scalars. However,
% if the journal you are submitting to favors bold math in the abstract,
% then you can use LaTeX's standard command \boldmath at the very start
% of the abstract to achieve this. Many IEEE journals frown on math
% in the abstract anyway.

% Note that keywords are not normally used for peerreview papers.
%\begin{IEEEkeywords}
%Cooperative diversity, decode and forward, piecewise linear
%\end{IEEEkeywords}



% For peer review papers, you can put extra information on the cover
% page as needed:
% \ifCLASSOPTIONpeerreview
% \begin{center} \bfseries EDICS Category: 3-BBND \end{center}
% \fi
%
% For peerreview papers, this IEEEtran command inserts a page break and
% creates the second title. It will be ignored for other modes.
%\IEEEpeerreviewmaketitle




\item Consider the probability space $\brak{\Omega, \mathcal{G}, P}$ where $\Omega = [0,2]$ and $\mathcal{G} = \cbrak{\phi, \Omega, [0,1], (1,2]}$. Let $X$ and $Y$ be two functions on $\Omega$ defined as
\begin{align*}
    X(\omega) = 
    \begin{cases}
        1 & \text{if }\omega \in [0, 1]\\
        2 & \text{if }\omega \in (1, 2]
    \end{cases}
\end{align*}
and
\begin{align*}
    Y(\omega) = 
    \begin{cases}
        2 & \text{if }\omega \in [0, 1.5]\\
        3 & \text{if }\omega \in (1.5, 2].
    \end{cases}
\end{align*}
Then which one of the following statements is true?
\begin{enumerate}
    \item [(A)] $X$ is a random variable with respect to $\mathcal{G}$, but $Y$ is not a random variable with respect to $\mathcal{G}$.
    \item [(B)] $Y$ is a random variable with respect to $\mathcal{G}$, but $X$ is not a random variable with respect to $\mathcal{G}$.
    \item [(C)] Neither $X$ nor $Y$ is a random variable with respect to $\mathcal{G}$.
    \item [(D)] Both $X$ and $Y$ are random variables with respect to $\mathcal{G}$.
\end{enumerate} \hfill (GATE ST 2023)\\
\solution
%\begin{table}[H]
	\centering
\begin{tabular}{|c|c|c|}
\hline
Random variable &Value &Definition\\ \hline
\multirow{3}{*}{X} &0 &Slips of Rs 1\\
&1 &Slips of Rs 5\\
&2 &Slips of Rs 13\\ \hline
\multirow{2}{*}{Y} &0 &Box A\\
&1 &Box B\\\hline
\end{tabular}
\caption{}
\label{tab:Distribution}
\end{table}
See \tabref{tab:Distribution}.
\begin{align}
p_{Y}\brak{k}= \begin{cases} 
      \frac{1}{3} & {k=0} \\
      \frac{2}{3 }& {k=1} 
   \end{cases}
   \\
p_{Y|X}\brak{0|0} = \frac{19}{25}\, 
p_{Y|X}\brak{0|1} = \frac{6}{25}\,
p_{Y|X}\brak{1|0} = \frac{45}{50}\,
p_{Y|X}\brak{1|2} = \frac{5}{50}
\end{align}
The desired probability is the probability that a slip drawn at random is marked other than Rs 1,
\begin{align}
&=1-p_X\brak{0}\\
&= p_X(1) + p_X(2)
\end{align}
Using Bayes theorem,
\begin{align}
&= p_Y\brak{0} \times \pr{Y=0 | X=1} + p_Y\brak{1} \times \pr{Y=1|X=2}\\
&=\frac{1}{3} \times \frac{6}{25} + \frac{2}{3} \times \frac{5}{50}\\
&=\frac{11}{75}
\end{align}

\newpage

%\tableofcontents

\bigskip

\renewcommand{\thefigure}{\theenumi}
\renewcommand{\thetable}{\theenumi}
%\renewcommand{\theequation}{\theenumi}

%\begin{abstract}
%%\boldmath
%In this letter, an algorithm for evaluating the exact analytical bit error rate  (BER)  for the piecewise linear (PL) combiner for  multiple relays is presented. Previous results were available only for upto three relays. The algorithm is unique in the sense that  the actual mathematical expressions, that are prohibitively large, need not be explicitly obtained. The diversity gain due to multiple relays is shown through plots of the analytical BER, well supported by simulations. 
%
%\end{abstract}
% IEEEtran.cls defaults to using nonbold math in the Abstract.
% This preserves the distinction between vectors and scalars. However,
% if the journal you are submitting to favors bold math in the abstract,
% then you can use LaTeX's standard command \boldmath at the very start
% of the abstract to achieve this. Many IEEE journals frown on math
% in the abstract anyway.

% Note that keywords are not normally used for peerreview papers.
%\begin{IEEEkeywords}
%Cooperative diversity, decode and forward, piecewise linear
%\end{IEEEkeywords}



% For peer review papers, you can put extra information on the cover
% page as needed:
% \ifCLASSOPTIONpeerreview
% \begin{center} \bfseries EDICS Category: 3-BBND \end{center}
% \fi
%
% For peerreview papers, this IEEEtran command inserts a page break and
% creates the second title. It will be ignored for other modes.
%\IEEEpeerreviewmaketitle




	\item  A die is loaded in such a way that each odd number is twice as likely to occur as
each even number. Find $P(G)$, where $G$ is the event that a number greater than
3 occurs on a single roll of the die.
\\
\solution
		%\begin{table}[H]
	\centering
\begin{tabular}{|c|c|c|}
\hline
Random variable &Value &Definition\\ \hline
\multirow{3}{*}{X} &0 &Slips of Rs 1\\
&1 &Slips of Rs 5\\
&2 &Slips of Rs 13\\ \hline
\multirow{2}{*}{Y} &0 &Box A\\
&1 &Box B\\\hline
\end{tabular}
\caption{}
\label{tab:Distribution}
\end{table}
See \tabref{tab:Distribution}.
\begin{align}
p_{Y}\brak{k}= \begin{cases} 
      \frac{1}{3} & {k=0} \\
      \frac{2}{3 }& {k=1} 
   \end{cases}
   \\
p_{Y|X}\brak{0|0} = \frac{19}{25}\, 
p_{Y|X}\brak{0|1} = \frac{6}{25}\,
p_{Y|X}\brak{1|0} = \frac{45}{50}\,
p_{Y|X}\brak{1|2} = \frac{5}{50}
\end{align}
The desired probability is the probability that a slip drawn at random is marked other than Rs 1,
\begin{align}
&=1-p_X\brak{0}\\
&= p_X(1) + p_X(2)
\end{align}
Using Bayes theorem,
\begin{align}
&= p_Y\brak{0} \times \pr{Y=0 | X=1} + p_Y\brak{1} \times \pr{Y=1|X=2}\\
&=\frac{1}{3} \times \frac{6}{25} + \frac{2}{3} \times \frac{5}{50}\\
&=\frac{11}{75}
\end{align}

\newpage

%\tableofcontents

\bigskip

\renewcommand{\thefigure}{\theenumi}
\renewcommand{\thetable}{\theenumi}
%\renewcommand{\theequation}{\theenumi}

%\begin{abstract}
%%\boldmath
%In this letter, an algorithm for evaluating the exact analytical bit error rate  (BER)  for the piecewise linear (PL) combiner for  multiple relays is presented. Previous results were available only for upto three relays. The algorithm is unique in the sense that  the actual mathematical expressions, that are prohibitively large, need not be explicitly obtained. The diversity gain due to multiple relays is shown through plots of the analytical BER, well supported by simulations. 
%
%\end{abstract}
% IEEEtran.cls defaults to using nonbold math in the Abstract.
% This preserves the distinction between vectors and scalars. However,
% if the journal you are submitting to favors bold math in the abstract,
% then you can use LaTeX's standard command \boldmath at the very start
% of the abstract to achieve this. Many IEEE journals frown on math
% in the abstract anyway.

% Note that keywords are not normally used for peerreview papers.
%\begin{IEEEkeywords}
%Cooperative diversity, decode and forward, piecewise linear
%\end{IEEEkeywords}



% For peer review papers, you can put extra information on the cover
% page as needed:
% \ifCLASSOPTIONpeerreview
% \begin{center} \bfseries EDICS Category: 3-BBND \end{center}
% \fi
%
% For peerreview papers, this IEEEtran command inserts a page break and
% creates the second title. It will be ignored for other modes.
%\IEEEpeerreviewmaketitle




	\item All the jacks, queens and kings are removed from a deck of 52 playing cards. The remaining cards are well shuffled and then one card is drawn at random. Giving ace a value 1 similar value for other cards, find the probability that the card has a value 
		\begin{enumerate}
			\item 7
			\item greater than 7
			\item less than 7
		\end{enumerate}
		%Number of cards left after removing all jacks, queens and kings 
\begin{align}
N	= 52 - 4\times 3
	= 40
\end{align}
%\begin{table}[H]
%\def\arraystretch{1.2}
%\begin{tabular}{|c|c|c|}
%\hline
%	\textbf{Parameter} &\textbf{Value} &\textbf{Description}\\ \hline
%	$X$ &1-10 &Represents the value of the card picked \\ \hline
%\end{tabular}
%\end{table}
Let $1 \le X \le 10$ be the value of the card picked.  Then,
\begin{align}
	p_X(k) &= \Pr(X=k)\ \forall\ 1 \leq k \leq 10\\
	&= \frac{4\times 1}{40}\\
	&= \frac{1}{10}\\
	\therefore p_X(k) &= 
	\begin{cases}
		\frac{1}{10} & 1 \leq k \leq 10\\
		0 & \text{otherwise}
	\end{cases}
\end{align}
and
\begin{align}
	F_{X}(k) &= \sum_{m=0}^{k}p_{X}(m) \quad 1 \leq k \leq 10\\
	&= \frac{k}{10}\\
	\therefore F_{X}(k) &= 
	\begin{cases}
		0 & k \leq 0\\
		\frac{k}{10} & 1\leq k \leq 10\\
		1 & k > 10 
	\end{cases}
\end{align}
\begin{enumerate}
	\item Probability that card has value equal to 7 is
		\begin{align}
			 p_{X}(7)
			= \frac{1}{10}
		\end{align}
	\item Probability that card has value greater than 7 is
		\begin{align}
			1 - F_X(7)
			&= 1 - \frac{7}{10}
			\\
			&= \frac{3}{10}
		\end{align}
	\item Probability that card has value less than 7 is
		\begin{align}
			 F_{X}(6)
			=\frac{6}{10}
		\end{align}
\end{enumerate}

  \item A Lot consists of 48 mobile phones of which 42 are good, 3 have only minor defects and 3 have major defects.Varnika will buy a phone if it is good but the trader will only buy a mobile if it has no major defects. One phone is selected at random from the lot. What is the probability that it is
\begin{enumerate}
	\item acceptable to Varnika?
            \item acceptable to the trader?
\end{enumerate}
\solution
	%\begin{table}[H]
	\centering
\begin{tabular}{|c|c|c|}
\hline
Random variable &Value &Definition\\ \hline
\multirow{3}{*}{X} &0 &Slips of Rs 1\\
&1 &Slips of Rs 5\\
&2 &Slips of Rs 13\\ \hline
\multirow{2}{*}{Y} &0 &Box A\\
&1 &Box B\\\hline
\end{tabular}
\caption{}
\label{tab:Distribution}
\end{table}
See \tabref{tab:Distribution}.
\begin{align}
p_{Y}\brak{k}= \begin{cases} 
      \frac{1}{3} & {k=0} \\
      \frac{2}{3 }& {k=1} 
   \end{cases}
   \\
p_{Y|X}\brak{0|0} = \frac{19}{25}\, 
p_{Y|X}\brak{0|1} = \frac{6}{25}\,
p_{Y|X}\brak{1|0} = \frac{45}{50}\,
p_{Y|X}\brak{1|2} = \frac{5}{50}
\end{align}
The desired probability is the probability that a slip drawn at random is marked other than Rs 1,
\begin{align}
&=1-p_X\brak{0}\\
&= p_X(1) + p_X(2)
\end{align}
Using Bayes theorem,
\begin{align}
&= p_Y\brak{0} \times \pr{Y=0 | X=1} + p_Y\brak{1} \times \pr{Y=1|X=2}\\
&=\frac{1}{3} \times \frac{6}{25} + \frac{2}{3} \times \frac{5}{50}\\
&=\frac{11}{75}
\end{align}

\newpage

%\tableofcontents

\bigskip

\renewcommand{\thefigure}{\theenumi}
\renewcommand{\thetable}{\theenumi}
%\renewcommand{\theequation}{\theenumi}

%\begin{abstract}
%%\boldmath
%In this letter, an algorithm for evaluating the exact analytical bit error rate  (BER)  for the piecewise linear (PL) combiner for  multiple relays is presented. Previous results were available only for upto three relays. The algorithm is unique in the sense that  the actual mathematical expressions, that are prohibitively large, need not be explicitly obtained. The diversity gain due to multiple relays is shown through plots of the analytical BER, well supported by simulations. 
%
%\end{abstract}
% IEEEtran.cls defaults to using nonbold math in the Abstract.
% This preserves the distinction between vectors and scalars. However,
% if the journal you are submitting to favors bold math in the abstract,
% then you can use LaTeX's standard command \boldmath at the very start
% of the abstract to achieve this. Many IEEE journals frown on math
% in the abstract anyway.

% Note that keywords are not normally used for peerreview papers.
%\begin{IEEEkeywords}
%Cooperative diversity, decode and forward, piecewise linear
%\end{IEEEkeywords}



% For peer review papers, you can put extra information on the cover
% page as needed:
% \ifCLASSOPTIONpeerreview
% \begin{center} \bfseries EDICS Category: 3-BBND \end{center}
% \fi
%
% For peerreview papers, this IEEEtran command inserts a page break and
% creates the second title. It will be ignored for other modes.
%\IEEEpeerreviewmaketitle




 \item A student says that if you throw a die, it will show up 1 or not 1. Therefore, the probability of getting 1 and the probability of getting 'not 1' each is equal to $\frac{1}{2}$. Is this correct? Give reasons.\\
 \solution
        %\begin{table}[H]
	\centering
\begin{tabular}{|c|c|c|}
\hline
Random variable &Value &Definition\\ \hline
\multirow{3}{*}{X} &0 &Slips of Rs 1\\
&1 &Slips of Rs 5\\
&2 &Slips of Rs 13\\ \hline
\multirow{2}{*}{Y} &0 &Box A\\
&1 &Box B\\\hline
\end{tabular}
\caption{}
\label{tab:Distribution}
\end{table}
See \tabref{tab:Distribution}.
\begin{align}
p_{Y}\brak{k}= \begin{cases} 
      \frac{1}{3} & {k=0} \\
      \frac{2}{3 }& {k=1} 
   \end{cases}
   \\
p_{Y|X}\brak{0|0} = \frac{19}{25}\, 
p_{Y|X}\brak{0|1} = \frac{6}{25}\,
p_{Y|X}\brak{1|0} = \frac{45}{50}\,
p_{Y|X}\brak{1|2} = \frac{5}{50}
\end{align}
The desired probability is the probability that a slip drawn at random is marked other than Rs 1,
\begin{align}
&=1-p_X\brak{0}\\
&= p_X(1) + p_X(2)
\end{align}
Using Bayes theorem,
\begin{align}
&= p_Y\brak{0} \times \pr{Y=0 | X=1} + p_Y\brak{1} \times \pr{Y=1|X=2}\\
&=\frac{1}{3} \times \frac{6}{25} + \frac{2}{3} \times \frac{5}{50}\\
&=\frac{11}{75}
\end{align}

\newpage

%\tableofcontents

\bigskip

\renewcommand{\thefigure}{\theenumi}
\renewcommand{\thetable}{\theenumi}
%\renewcommand{\theequation}{\theenumi}

%\begin{abstract}
%%\boldmath
%In this letter, an algorithm for evaluating the exact analytical bit error rate  (BER)  for the piecewise linear (PL) combiner for  multiple relays is presented. Previous results were available only for upto three relays. The algorithm is unique in the sense that  the actual mathematical expressions, that are prohibitively large, need not be explicitly obtained. The diversity gain due to multiple relays is shown through plots of the analytical BER, well supported by simulations. 
%
%\end{abstract}
% IEEEtran.cls defaults to using nonbold math in the Abstract.
% This preserves the distinction between vectors and scalars. However,
% if the journal you are submitting to favors bold math in the abstract,
% then you can use LaTeX's standard command \boldmath at the very start
% of the abstract to achieve this. Many IEEE journals frown on math
% in the abstract anyway.

% Note that keywords are not normally used for peerreview papers.
%\begin{IEEEkeywords}
%Cooperative diversity, decode and forward, piecewise linear
%\end{IEEEkeywords}



% For peer review papers, you can put extra information on the cover
% page as needed:
% \ifCLASSOPTIONpeerreview
% \begin{center} \bfseries EDICS Category: 3-BBND \end{center}
% \fi
%
% For peerreview papers, this IEEEtran command inserts a page break and
% creates the second title. It will be ignored for other modes.
%\IEEEpeerreviewmaketitle




   \item Four candidates A, B, C, D have ap-
plied for the assignment to coach a school cricket
team. If A is twice as likely to be selected as B, and
B and C are given about the same chance of being
selected, while C is twice as likely to be selected
as D, what are the probabilities that
\begin{enumerate}
\item C will be selected?
\item A will not be selected?
\end{enumerate}
	%\begin{table}[H]
	\centering
\begin{tabular}{|c|c|c|}
\hline
Random variable &Value &Definition\\ \hline
\multirow{3}{*}{X} &0 &Slips of Rs 1\\
&1 &Slips of Rs 5\\
&2 &Slips of Rs 13\\ \hline
\multirow{2}{*}{Y} &0 &Box A\\
&1 &Box B\\\hline
\end{tabular}
\caption{}
\label{tab:Distribution}
\end{table}
See \tabref{tab:Distribution}.
\begin{align}
p_{Y}\brak{k}= \begin{cases} 
      \frac{1}{3} & {k=0} \\
      \frac{2}{3 }& {k=1} 
   \end{cases}
   \\
p_{Y|X}\brak{0|0} = \frac{19}{25}\, 
p_{Y|X}\brak{0|1} = \frac{6}{25}\,
p_{Y|X}\brak{1|0} = \frac{45}{50}\,
p_{Y|X}\brak{1|2} = \frac{5}{50}
\end{align}
The desired probability is the probability that a slip drawn at random is marked other than Rs 1,
\begin{align}
&=1-p_X\brak{0}\\
&= p_X(1) + p_X(2)
\end{align}
Using Bayes theorem,
\begin{align}
&= p_Y\brak{0} \times \pr{Y=0 | X=1} + p_Y\brak{1} \times \pr{Y=1|X=2}\\
&=\frac{1}{3} \times \frac{6}{25} + \frac{2}{3} \times \frac{5}{50}\\
&=\frac{11}{75}
\end{align}

\newpage

%\tableofcontents

\bigskip

\renewcommand{\thefigure}{\theenumi}
\renewcommand{\thetable}{\theenumi}
%\renewcommand{\theequation}{\theenumi}

%\begin{abstract}
%%\boldmath
%In this letter, an algorithm for evaluating the exact analytical bit error rate  (BER)  for the piecewise linear (PL) combiner for  multiple relays is presented. Previous results were available only for upto three relays. The algorithm is unique in the sense that  the actual mathematical expressions, that are prohibitively large, need not be explicitly obtained. The diversity gain due to multiple relays is shown through plots of the analytical BER, well supported by simulations. 
%
%\end{abstract}
% IEEEtran.cls defaults to using nonbold math in the Abstract.
% This preserves the distinction between vectors and scalars. However,
% if the journal you are submitting to favors bold math in the abstract,
% then you can use LaTeX's standard command \boldmath at the very start
% of the abstract to achieve this. Many IEEE journals frown on math
% in the abstract anyway.

% Note that keywords are not normally used for peerreview papers.
%\begin{IEEEkeywords}
%Cooperative diversity, decode and forward, piecewise linear
%\end{IEEEkeywords}



% For peer review papers, you can put extra information on the cover
% page as needed:
% \ifCLASSOPTIONpeerreview
% \begin{center} \bfseries EDICS Category: 3-BBND \end{center}
% \fi
%
% For peerreview papers, this IEEEtran command inserts a page break and
% creates the second title. It will be ignored for other modes.
%\IEEEpeerreviewmaketitle




 \item A bag contain 24 balls of which $x$ balls are red, $2x$ are white and $3x$ are blue. A ball is selected at random, What is the probability that it is
\begin{enumerate}[label=\alph*)]
\item not red ?
\item white ?
\end{enumerate}
%\begin{table}[H]
	\centering
\begin{tabular}{|c|c|c|}
\hline
Random variable &Value &Definition\\ \hline
\multirow{3}{*}{X} &0 &Slips of Rs 1\\
&1 &Slips of Rs 5\\
&2 &Slips of Rs 13\\ \hline
\multirow{2}{*}{Y} &0 &Box A\\
&1 &Box B\\\hline
\end{tabular}
\caption{}
\label{tab:Distribution}
\end{table}
See \tabref{tab:Distribution}.
\begin{align}
p_{Y}\brak{k}= \begin{cases} 
      \frac{1}{3} & {k=0} \\
      \frac{2}{3 }& {k=1} 
   \end{cases}
   \\
p_{Y|X}\brak{0|0} = \frac{19}{25}\, 
p_{Y|X}\brak{0|1} = \frac{6}{25}\,
p_{Y|X}\brak{1|0} = \frac{45}{50}\,
p_{Y|X}\brak{1|2} = \frac{5}{50}
\end{align}
The desired probability is the probability that a slip drawn at random is marked other than Rs 1,
\begin{align}
&=1-p_X\brak{0}\\
&= p_X(1) + p_X(2)
\end{align}
Using Bayes theorem,
\begin{align}
&= p_Y\brak{0} \times \pr{Y=0 | X=1} + p_Y\brak{1} \times \pr{Y=1|X=2}\\
&=\frac{1}{3} \times \frac{6}{25} + \frac{2}{3} \times \frac{5}{50}\\
&=\frac{11}{75}
\end{align}

\newpage

%\tableofcontents

\bigskip

\renewcommand{\thefigure}{\theenumi}
\renewcommand{\thetable}{\theenumi}
%\renewcommand{\theequation}{\theenumi}

%\begin{abstract}
%%\boldmath
%In this letter, an algorithm for evaluating the exact analytical bit error rate  (BER)  for the piecewise linear (PL) combiner for  multiple relays is presented. Previous results were available only for upto three relays. The algorithm is unique in the sense that  the actual mathematical expressions, that are prohibitively large, need not be explicitly obtained. The diversity gain due to multiple relays is shown through plots of the analytical BER, well supported by simulations. 
%
%\end{abstract}
% IEEEtran.cls defaults to using nonbold math in the Abstract.
% This preserves the distinction between vectors and scalars. However,
% if the journal you are submitting to favors bold math in the abstract,
% then you can use LaTeX's standard command \boldmath at the very start
% of the abstract to achieve this. Many IEEE journals frown on math
% in the abstract anyway.

% Note that keywords are not normally used for peerreview papers.
%\begin{IEEEkeywords}
%Cooperative diversity, decode and forward, piecewise linear
%\end{IEEEkeywords}



% For peer review papers, you can put extra information on the cover
% page as needed:
% \ifCLASSOPTIONpeerreview
% \begin{center} \bfseries EDICS Category: 3-BBND \end{center}
% \fi
%
% For peerreview papers, this IEEEtran command inserts a page break and
% creates the second title. It will be ignored for other modes.
%\IEEEpeerreviewmaketitle




If the letters of the word ASSASSINATION are arranged at random. Find the Probability that
\begin{enumerate}[label=(\alph*)]
\item Four $S's$ come consecutively in the word
\item Two  $I's$ and two $N's$ come together
\item All $A's$ are not coming together
\item No two $A's$ are coming together
\end{enumerate}
%\begin{table}[H]
	\centering
\begin{tabular}{|c|c|c|}
\hline
Random variable &Value &Definition\\ \hline
\multirow{3}{*}{X} &0 &Slips of Rs 1\\
&1 &Slips of Rs 5\\
&2 &Slips of Rs 13\\ \hline
\multirow{2}{*}{Y} &0 &Box A\\
&1 &Box B\\\hline
\end{tabular}
\caption{}
\label{tab:Distribution}
\end{table}
See \tabref{tab:Distribution}.
\begin{align}
p_{Y}\brak{k}= \begin{cases} 
      \frac{1}{3} & {k=0} \\
      \frac{2}{3 }& {k=1} 
   \end{cases}
   \\
p_{Y|X}\brak{0|0} = \frac{19}{25}\, 
p_{Y|X}\brak{0|1} = \frac{6}{25}\,
p_{Y|X}\brak{1|0} = \frac{45}{50}\,
p_{Y|X}\brak{1|2} = \frac{5}{50}
\end{align}
The desired probability is the probability that a slip drawn at random is marked other than Rs 1,
\begin{align}
&=1-p_X\brak{0}\\
&= p_X(1) + p_X(2)
\end{align}
Using Bayes theorem,
\begin{align}
&= p_Y\brak{0} \times \pr{Y=0 | X=1} + p_Y\brak{1} \times \pr{Y=1|X=2}\\
&=\frac{1}{3} \times \frac{6}{25} + \frac{2}{3} \times \frac{5}{50}\\
&=\frac{11}{75}
\end{align}

\newpage

%\tableofcontents

\bigskip

\renewcommand{\thefigure}{\theenumi}
\renewcommand{\thetable}{\theenumi}
%\renewcommand{\theequation}{\theenumi}

%\begin{abstract}
%%\boldmath
%In this letter, an algorithm for evaluating the exact analytical bit error rate  (BER)  for the piecewise linear (PL) combiner for  multiple relays is presented. Previous results were available only for upto three relays. The algorithm is unique in the sense that  the actual mathematical expressions, that are prohibitively large, need not be explicitly obtained. The diversity gain due to multiple relays is shown through plots of the analytical BER, well supported by simulations. 
%
%\end{abstract}
% IEEEtran.cls defaults to using nonbold math in the Abstract.
% This preserves the distinction between vectors and scalars. However,
% if the journal you are submitting to favors bold math in the abstract,
% then you can use LaTeX's standard command \boldmath at the very start
% of the abstract to achieve this. Many IEEE journals frown on math
% in the abstract anyway.

% Note that keywords are not normally used for peerreview papers.
%\begin{IEEEkeywords}
%Cooperative diversity, decode and forward, piecewise linear
%\end{IEEEkeywords}



% For peer review papers, you can put extra information on the cover
% page as needed:
% \ifCLASSOPTIONpeerreview
% \begin{center} \bfseries EDICS Category: 3-BBND \end{center}
% \fi
%
% For peerreview papers, this IEEEtran command inserts a page break and
% creates the second title. It will be ignored for other modes.
%\IEEEpeerreviewmaketitle




	\item One urn contains two black balls (labelled B1 and B2) and one white ball. A
	second urn contains one black ball and two white balls (labelled W1 and W2).
	Suppose the following experiment is performed. One of the two urns is chosen
	at random. Next a ball is randomly chosen from the urn. Then a second ball is
	chosen at random from the same urn without replacing the first ball.
	
	\begin{enumerate}
	\item What is the probability that two black balls are chosen?
	
	\item What is the probability that two balls of opposite colour are chosen?
	\end{enumerate}
	\solution
	%\begin{align}
    \label{eq:12.13.6.18.1}
	\because	\pr{A|B} &> \pr{A},\
\frac{\pr{AB}}{\pr{B}} > \pr{A}
\\
    \label{eq:12.13.6.18.2}
	\implies \pr{AB} &> \pr{A}\pr{B}
	\\
	\text{or, } \frac{\pr{AB}}{\pr{A}} &=\pr{B|A} > \pr{A}
\end{align}

\end{enumerate}

	\item A bag contains $5$ red balls and some blue balls. If the probability of drawing a blue ball is double that if a red ball, determine the number of blue balls in the bag. 
		\\
\solution
		%\begin{enumerate}[label=\thesection.\arabic*,ref=\thesection.\theenumi]
	\item One card is drawn from a well-shuffled deck of 52 cards. Find the probability of getting
\begin{enumerate}
\item A king of red colour 
\item A face card 
\item A red face card
\item The jack of hearts
\item A spade
\item The queen of diamonds

\end{enumerate}
\solution
		%\begin{table}[H]
	\centering
\begin{tabular}{|c|c|c|}
\hline
Random variable &Value &Definition\\ \hline
\multirow{3}{*}{X} &0 &Slips of Rs 1\\
&1 &Slips of Rs 5\\
&2 &Slips of Rs 13\\ \hline
\multirow{2}{*}{Y} &0 &Box A\\
&1 &Box B\\\hline
\end{tabular}
\caption{}
\label{tab:Distribution}
\end{table}
See \tabref{tab:Distribution}.
\begin{align}
p_{Y}\brak{k}= \begin{cases} 
      \frac{1}{3} & {k=0} \\
      \frac{2}{3 }& {k=1} 
   \end{cases}
   \\
p_{Y|X}\brak{0|0} = \frac{19}{25}\, 
p_{Y|X}\brak{0|1} = \frac{6}{25}\,
p_{Y|X}\brak{1|0} = \frac{45}{50}\,
p_{Y|X}\brak{1|2} = \frac{5}{50}
\end{align}
The desired probability is the probability that a slip drawn at random is marked other than Rs 1,
\begin{align}
&=1-p_X\brak{0}\\
&= p_X(1) + p_X(2)
\end{align}
Using Bayes theorem,
\begin{align}
&= p_Y\brak{0} \times \pr{Y=0 | X=1} + p_Y\brak{1} \times \pr{Y=1|X=2}\\
&=\frac{1}{3} \times \frac{6}{25} + \frac{2}{3} \times \frac{5}{50}\\
&=\frac{11}{75}
\end{align}

\newpage

%\tableofcontents

\bigskip

\renewcommand{\thefigure}{\theenumi}
\renewcommand{\thetable}{\theenumi}
%\renewcommand{\theequation}{\theenumi}

%\begin{abstract}
%%\boldmath
%In this letter, an algorithm for evaluating the exact analytical bit error rate  (BER)  for the piecewise linear (PL) combiner for  multiple relays is presented. Previous results were available only for upto three relays. The algorithm is unique in the sense that  the actual mathematical expressions, that are prohibitively large, need not be explicitly obtained. The diversity gain due to multiple relays is shown through plots of the analytical BER, well supported by simulations. 
%
%\end{abstract}
% IEEEtran.cls defaults to using nonbold math in the Abstract.
% This preserves the distinction between vectors and scalars. However,
% if the journal you are submitting to favors bold math in the abstract,
% then you can use LaTeX's standard command \boldmath at the very start
% of the abstract to achieve this. Many IEEE journals frown on math
% in the abstract anyway.

% Note that keywords are not normally used for peerreview papers.
%\begin{IEEEkeywords}
%Cooperative diversity, decode and forward, piecewise linear
%\end{IEEEkeywords}



% For peer review papers, you can put extra information on the cover
% page as needed:
% \ifCLASSOPTIONpeerreview
% \begin{center} \bfseries EDICS Category: 3-BBND \end{center}
% \fi
%
% For peerreview papers, this IEEEtran command inserts a page break and
% creates the second title. It will be ignored for other modes.
%\IEEEpeerreviewmaketitle




	\item Five cards—the ten, jack, queen, king and ace of diamonds, are well-shuffled with their face downwards. One card is then picked up at random.
\begin{enumerate}
\item
What is the probability that the card is the queen? 
\item
If the queen is drawn and put aside, what is the probability that the second card picked up is (a) an ace? (b) a queen?\\
\end{enumerate}
\solution
		%\begin{enumerate}[label=\thesection.\arabic*,ref=\thesection.\theenumi]
	\item One card is drawn from a well-shuffled deck of 52 cards. Find the probability of getting
\begin{enumerate}
\item A king of red colour 
\item A face card 
\item A red face card
\item The jack of hearts
\item A spade
\item The queen of diamonds

\end{enumerate}
\solution
		%\input{ncert/10/15/1/14/main.tex}
	\item Five cards—the ten, jack, queen, king and ace of diamonds, are well-shuffled with their face downwards. One card is then picked up at random.
\begin{enumerate}
\item
What is the probability that the card is the queen? 
\item
If the queen is drawn and put aside, what is the probability that the second card picked up is (a) an ace? (b) a queen?\\
\end{enumerate}
\solution
		%\input{ncert/10/15/1/15/defs.tex}
	\item A bag contains $5$ red balls and some blue balls. If the probability of drawing a blue ball is double that if a red ball, determine the number of blue balls in the bag. 
		\\
\solution
		%\input{ncert/10/15/2/3/defs.tex}
	\item A card is selected from a pack of 52 cards.
 \begin{enumerate}[label=(\alph*)] 
                 \item How many points are there in the sample space?
                 \item Calculate the probability that the card is an ace of spades.
                 \item Calculate the probability that the card is (i) an ace and (ii) black card.
 \end{enumerate}
\solution
		%\input{ncert/11/16/3/4/main.tex}
\item Four cards are drawn from a well-shuffled deck of 52 cards. What is the probability of obtaining 3 diamonds and one spade.
\\
\solution
		%\input{ncert/11/16/4/2/defs.tex}
\item In a certain lottery 10,000 tickets are sold and ten equal prizes are awarded. What is the probability of not getting a prize if you buy (a) one ticket (b) two tickets (c) 10 tickets ?	
\\
\solution
		%\input{ncert/11/16/4/4/defs.tex}
		%
\item 
Out of 100 students, two sections of 40 and 60 are formed. If you and your friend are among the 100 students, what is the probability that
\begin{enumerate}
\item you both enter the same section?
\item you both enter the different sections?
\end{enumerate}
\solution
		%\input{ncert/11/16/4/5/defs.tex}
	\item 
The number lock of a suitcase has 4 wheels each labelled with ten digits i.e. from 0 to 9.The lock opens with a sequence of four digits with no repeats.What is the probability of a person getting the right sequence to open the suitcase.
\\
\solution
		%\input{ncert/11/16/4/10/defs.tex}
		%
\item 
Two cards are drawn at random and without replacement from a pack of 52 playing cards. Find the probability that both the cards are black.
\\
\solution
		%\input{ncert/12/13/2/2/defs.tex}
		\item A box of oranges is inspected by examining three randomly selected oranges drawn without replacement. If all the three oranges are good, the box is approved for sale, otherwise, it is rejected. Find the probability that a box containing 15 oranges out of which 12 are good and 3 are bad ones will be approved for sale.
		\label{ncert/12/13/2/3/defs.tex}
		\item Two balls are drawn at random with replacement from a box containing 10 black and 8 red balls. Find the probability that
		\label{ncert/12/13/2/12}
\begin{enumerate}
\item both balls are red.
\item first ball is black and second is red.
\item one of them is black and other is red.
\end{enumerate}

\item In a hostel, 60\% of the students read Hindi newspaper, 40\% read English newspaper and 20\% read both Hindi and English newspapers. A student is selected at random.
		\label{ncert/12/13/2/15}
\begin{enumerate}
\item Find the probability that she reads neither Hindi nor English newspapers.
\item If she reads Hindi newspaper, find the probability that she reads English newspaper.
\item If she reads English newspaper, find the probability that she reads Hindi newspaper.\\
\end{enumerate}
\item The probability of obtaining an even prime number on each die, when a pair of dice is rolled is 
\begin{enumerate}
    \item $0$ 
    
    \item $\frac{1}{3}$ 
    
    \item $\frac{1}{12}$ 
    
    \item $\frac{1}{36}$ 
\end{enumerate}
\solution
		%\input{ncert/12/13/2/17/defs.tex}
	\item A bag contains 4 red and 4 black balls, another bag contains 2 red and 6 black balls. One of the two bags is selected at random and a ball is drawn from the bag which is found to be red. Find the probability that the ball is drawn from the first bag.
\\
\solution
		%\input{ncert/12/13/3/2/main.tex}
  \item
  Cards with numbers 2 to 101 are placed in a box. A card is selected at random.Find the probability that the card has
\begin{enumerate}[label=(\roman*)]
	\item an even number 
	\item a square number
\end{enumerate}
\solution
%\input{exemplar/10/13/3/32/main.tex}
\item
The king, queen and jack of clubs are removed from a deck of 52 playing cards and then well shuffled. Now one card is drawn at random from the remaining cards.  Determine the probability that the card is
\begin{enumerate}[label=(\roman*)]
\item a club
\item 10 of hearts
\end{enumerate}
\solution
%\input{exemplar/10/13/3/29/main.tex}
\item A team of medical students doing their internship have to assist during surgeries
at a city hospital. The probabilities of surgeries rated as very complex, complex,
routine, simple or very simple are respectively, 0.15, 0.20, 0.31, 0.26, .08. Find
the probabilities that a particular surgery will be rated
\begin{enumerate}
	\item complex or very complex;
	\item neither very complex nor very simple;
	\item routine or complex
	\item routine or simple
\end{enumerate}
\solution
%\input{exemplar/11/16/3/8(1)/main.tex}
\item A card is selected from a pack of 52 cards.
\begin{enumerate}[label=(\alph*)]
    \item How many points are there in the sample space?
    \item Calculate the probability that the card is an ace of spades.
    \item Calculate the probability that the card is (i) an ace and (ii) black card.
\end{enumerate}
\solution
%\input{exemplar/11/16/3/4/main2.tex}
\item The probability that a non leap year selected at random will contain 53 sundays.
\\
\solution
%\input{exemplar/10/13/1/19/main.tex}
\item One of the four persons John, Rita, Aslam or Gurpreet will be promoted next
month. Consequently the sample space consists of four elementary outcomes
S = {John promoted, Rita promoted, Aslam promoted, Gurpreet promoted}
You are told that the chances of John’s promotion is same as that of Gurpreet,
Rita’s chances of promotion are twice as likely as Johns. Aslam’s chances are
four times that of John.
\begin{enumerate}
	\item Determine
	\begin{enumerate}
		\item P (John promoted)
		\item P (Rita promoted)
		\item P (Aslam promoted)
		\item P (Gurpreet promoted)
	\end{enumerate}
	\item If A = {John promoted or Gurpreet promoted}, find P (A).
\end{enumerate}
\solution
%\input{exemplar/11/16/3/10/main.tex}
\item A card is drawn from a deck of 52 cards. Find the probability of getting a king or a heart or a red card.\\
\solution
%\input{exemplar/11/16/3/15/main.tex}
\item The probability that a student will pass his examination is 0.73, the probability of
the student getting a compartment is 0.13, and the probability that the student will
either pass or get compartment is 0.96. State True or False.\\
\solution
%\input{exemplar/11/16/3/31/main.tex}
\item A card is selected from a pack of 52 cards\\
\begin{enumerate}[label=(\alph*)]
\item How many points are there in the sample space?
\item Calculate the probability that the cards is an ace of spades.
\item Calculate the probability that the card is (i) an ace (ii)black card.\\
\end{enumerate}
%\input{ncert/11/16/3/4_1/Prob_4.tex}
\item In a non-leap year, the probability of having 53 tuesdays or 53 wednesdays is\\
\solution
%\input{exemplar/11/16/3/18/main.tex}
\item There are 1000 sealed envelopes in a box, 10 of them contain a cash prize of
Rs 100 each, 100 of them contain a cash prize of Rs 50 each and 200 of them
contain a cash prize of Rs 10 each and rest do not contain any cash prize. If they
are well shuffled and an envelope is picked up out, what is the probability that it
contains no cash prize?\\
\solution
%\input{exemplar/10/13/3/34/main.tex}
\item 
A die is thrown and a card is selected at random from a deck of 52 playing cards. The probability of getting an even number on the die and a spade card.\\
\solution
%\input{exemplar/12/13/3/78/main.tex}
\item
If 4-digit numbers greater than 5,000 are randomly formed from the digits 0, 1, 3, 5, and 7, what is the probability of forming a number divisible by 5 when:
\begin{enumerate}
    \item The digits are repeated?
    \item The repetition of digits is not allowed?
\end{enumerate}
\solution
%\input{ncert/11/16/4/9/main.tex}
\item Consider the probability space $\brak{\Omega, \mathcal{G}, P}$ where $\Omega = [0,2]$ and $\mathcal{G} = \cbrak{\phi, \Omega, [0,1], (1,2]}$. Let $X$ and $Y$ be two functions on $\Omega$ defined as
\begin{align*}
    X(\omega) = 
    \begin{cases}
        1 & \text{if }\omega \in [0, 1]\\
        2 & \text{if }\omega \in (1, 2]
    \end{cases}
\end{align*}
and
\begin{align*}
    Y(\omega) = 
    \begin{cases}
        2 & \text{if }\omega \in [0, 1.5]\\
        3 & \text{if }\omega \in (1.5, 2].
    \end{cases}
\end{align*}
Then which one of the following statements is true?
\begin{enumerate}
    \item [(A)] $X$ is a random variable with respect to $\mathcal{G}$, but $Y$ is not a random variable with respect to $\mathcal{G}$.
    \item [(B)] $Y$ is a random variable with respect to $\mathcal{G}$, but $X$ is not a random variable with respect to $\mathcal{G}$.
    \item [(C)] Neither $X$ nor $Y$ is a random variable with respect to $\mathcal{G}$.
    \item [(D)] Both $X$ and $Y$ are random variables with respect to $\mathcal{G}$.
\end{enumerate} \hfill (GATE ST 2023)\\
\solution
%\input{gate/ST/2023/14/main.tex}
	\item  A die is loaded in such a way that each odd number is twice as likely to occur as
each even number. Find $P(G)$, where $G$ is the event that a number greater than
3 occurs on a single roll of the die.
\\
\solution
		%\input{exemplar/11/16/3/5/main.tex}
	\item All the jacks, queens and kings are removed from a deck of 52 playing cards. The remaining cards are well shuffled and then one card is drawn at random. Giving ace a value 1 similar value for other cards, find the probability that the card has a value 
		\begin{enumerate}
			\item 7
			\item greater than 7
			\item less than 7
		\end{enumerate}
		%\input{exemplar/10/13/3/30/main.tex}
  \item A Lot consists of 48 mobile phones of which 42 are good, 3 have only minor defects and 3 have major defects.Varnika will buy a phone if it is good but the trader will only buy a mobile if it has no major defects. One phone is selected at random from the lot. What is the probability that it is
\begin{enumerate}
	\item acceptable to Varnika?
            \item acceptable to the trader?
\end{enumerate}
\solution
	%\input{exemplar/10/13/3/40/main.tex}
 \item A student says that if you throw a die, it will show up 1 or not 1. Therefore, the probability of getting 1 and the probability of getting 'not 1' each is equal to $\frac{1}{2}$. Is this correct? Give reasons.\\
 \solution
        %\input{exemplar/10/13/2/9/main.tex}
   \item Four candidates A, B, C, D have ap-
plied for the assignment to coach a school cricket
team. If A is twice as likely to be selected as B, and
B and C are given about the same chance of being
selected, while C is twice as likely to be selected
as D, what are the probabilities that
\begin{enumerate}
\item C will be selected?
\item A will not be selected?
\end{enumerate}
	%\input{exemplar/11/16/3/9/main.tex}
 \item A bag contain 24 balls of which $x$ balls are red, $2x$ are white and $3x$ are blue. A ball is selected at random, What is the probability that it is
\begin{enumerate}[label=\alph*)]
\item not red ?
\item white ?
\end{enumerate}
%\input{exemplar/10/13/3/41/main.tex}
If the letters of the word ASSASSINATION are arranged at random. Find the Probability that
\begin{enumerate}[label=(\alph*)]
\item Four $S's$ come consecutively in the word
\item Two  $I's$ and two $N's$ come together
\item All $A's$ are not coming together
\item No two $A's$ are coming together
\end{enumerate}
%\input{exemplar/11/16/3/14/main.tex}
	\item One urn contains two black balls (labelled B1 and B2) and one white ball. A
	second urn contains one black ball and two white balls (labelled W1 and W2).
	Suppose the following experiment is performed. One of the two urns is chosen
	at random. Next a ball is randomly chosen from the urn. Then a second ball is
	chosen at random from the same urn without replacing the first ball.
	
	\begin{enumerate}
	\item What is the probability that two black balls are chosen?
	
	\item What is the probability that two balls of opposite colour are chosen?
	\end{enumerate}
	\solution
	%\input{exemplar/11/16/3/12/main1.tex}
\end{enumerate}

	\item A bag contains $5$ red balls and some blue balls. If the probability of drawing a blue ball is double that if a red ball, determine the number of blue balls in the bag. 
		\\
\solution
		%\begin{enumerate}[label=\thesection.\arabic*,ref=\thesection.\theenumi]
	\item One card is drawn from a well-shuffled deck of 52 cards. Find the probability of getting
\begin{enumerate}
\item A king of red colour 
\item A face card 
\item A red face card
\item The jack of hearts
\item A spade
\item The queen of diamonds

\end{enumerate}
\solution
		%\input{ncert/10/15/1/14/main.tex}
	\item Five cards—the ten, jack, queen, king and ace of diamonds, are well-shuffled with their face downwards. One card is then picked up at random.
\begin{enumerate}
\item
What is the probability that the card is the queen? 
\item
If the queen is drawn and put aside, what is the probability that the second card picked up is (a) an ace? (b) a queen?\\
\end{enumerate}
\solution
		%\input{ncert/10/15/1/15/defs.tex}
	\item A bag contains $5$ red balls and some blue balls. If the probability of drawing a blue ball is double that if a red ball, determine the number of blue balls in the bag. 
		\\
\solution
		%\input{ncert/10/15/2/3/defs.tex}
	\item A card is selected from a pack of 52 cards.
 \begin{enumerate}[label=(\alph*)] 
                 \item How many points are there in the sample space?
                 \item Calculate the probability that the card is an ace of spades.
                 \item Calculate the probability that the card is (i) an ace and (ii) black card.
 \end{enumerate}
\solution
		%\input{ncert/11/16/3/4/main.tex}
\item Four cards are drawn from a well-shuffled deck of 52 cards. What is the probability of obtaining 3 diamonds and one spade.
\\
\solution
		%\input{ncert/11/16/4/2/defs.tex}
\item In a certain lottery 10,000 tickets are sold and ten equal prizes are awarded. What is the probability of not getting a prize if you buy (a) one ticket (b) two tickets (c) 10 tickets ?	
\\
\solution
		%\input{ncert/11/16/4/4/defs.tex}
		%
\item 
Out of 100 students, two sections of 40 and 60 are formed. If you and your friend are among the 100 students, what is the probability that
\begin{enumerate}
\item you both enter the same section?
\item you both enter the different sections?
\end{enumerate}
\solution
		%\input{ncert/11/16/4/5/defs.tex}
	\item 
The number lock of a suitcase has 4 wheels each labelled with ten digits i.e. from 0 to 9.The lock opens with a sequence of four digits with no repeats.What is the probability of a person getting the right sequence to open the suitcase.
\\
\solution
		%\input{ncert/11/16/4/10/defs.tex}
		%
\item 
Two cards are drawn at random and without replacement from a pack of 52 playing cards. Find the probability that both the cards are black.
\\
\solution
		%\input{ncert/12/13/2/2/defs.tex}
		\item A box of oranges is inspected by examining three randomly selected oranges drawn without replacement. If all the three oranges are good, the box is approved for sale, otherwise, it is rejected. Find the probability that a box containing 15 oranges out of which 12 are good and 3 are bad ones will be approved for sale.
		\label{ncert/12/13/2/3/defs.tex}
		\item Two balls are drawn at random with replacement from a box containing 10 black and 8 red balls. Find the probability that
		\label{ncert/12/13/2/12}
\begin{enumerate}
\item both balls are red.
\item first ball is black and second is red.
\item one of them is black and other is red.
\end{enumerate}

\item In a hostel, 60\% of the students read Hindi newspaper, 40\% read English newspaper and 20\% read both Hindi and English newspapers. A student is selected at random.
		\label{ncert/12/13/2/15}
\begin{enumerate}
\item Find the probability that she reads neither Hindi nor English newspapers.
\item If she reads Hindi newspaper, find the probability that she reads English newspaper.
\item If she reads English newspaper, find the probability that she reads Hindi newspaper.\\
\end{enumerate}
\item The probability of obtaining an even prime number on each die, when a pair of dice is rolled is 
\begin{enumerate}
    \item $0$ 
    
    \item $\frac{1}{3}$ 
    
    \item $\frac{1}{12}$ 
    
    \item $\frac{1}{36}$ 
\end{enumerate}
\solution
		%\input{ncert/12/13/2/17/defs.tex}
	\item A bag contains 4 red and 4 black balls, another bag contains 2 red and 6 black balls. One of the two bags is selected at random and a ball is drawn from the bag which is found to be red. Find the probability that the ball is drawn from the first bag.
\\
\solution
		%\input{ncert/12/13/3/2/main.tex}
  \item
  Cards with numbers 2 to 101 are placed in a box. A card is selected at random.Find the probability that the card has
\begin{enumerate}[label=(\roman*)]
	\item an even number 
	\item a square number
\end{enumerate}
\solution
%\input{exemplar/10/13/3/32/main.tex}
\item
The king, queen and jack of clubs are removed from a deck of 52 playing cards and then well shuffled. Now one card is drawn at random from the remaining cards.  Determine the probability that the card is
\begin{enumerate}[label=(\roman*)]
\item a club
\item 10 of hearts
\end{enumerate}
\solution
%\input{exemplar/10/13/3/29/main.tex}
\item A team of medical students doing their internship have to assist during surgeries
at a city hospital. The probabilities of surgeries rated as very complex, complex,
routine, simple or very simple are respectively, 0.15, 0.20, 0.31, 0.26, .08. Find
the probabilities that a particular surgery will be rated
\begin{enumerate}
	\item complex or very complex;
	\item neither very complex nor very simple;
	\item routine or complex
	\item routine or simple
\end{enumerate}
\solution
%\input{exemplar/11/16/3/8(1)/main.tex}
\item A card is selected from a pack of 52 cards.
\begin{enumerate}[label=(\alph*)]
    \item How many points are there in the sample space?
    \item Calculate the probability that the card is an ace of spades.
    \item Calculate the probability that the card is (i) an ace and (ii) black card.
\end{enumerate}
\solution
%\input{exemplar/11/16/3/4/main2.tex}
\item The probability that a non leap year selected at random will contain 53 sundays.
\\
\solution
%\input{exemplar/10/13/1/19/main.tex}
\item One of the four persons John, Rita, Aslam or Gurpreet will be promoted next
month. Consequently the sample space consists of four elementary outcomes
S = {John promoted, Rita promoted, Aslam promoted, Gurpreet promoted}
You are told that the chances of John’s promotion is same as that of Gurpreet,
Rita’s chances of promotion are twice as likely as Johns. Aslam’s chances are
four times that of John.
\begin{enumerate}
	\item Determine
	\begin{enumerate}
		\item P (John promoted)
		\item P (Rita promoted)
		\item P (Aslam promoted)
		\item P (Gurpreet promoted)
	\end{enumerate}
	\item If A = {John promoted or Gurpreet promoted}, find P (A).
\end{enumerate}
\solution
%\input{exemplar/11/16/3/10/main.tex}
\item A card is drawn from a deck of 52 cards. Find the probability of getting a king or a heart or a red card.\\
\solution
%\input{exemplar/11/16/3/15/main.tex}
\item The probability that a student will pass his examination is 0.73, the probability of
the student getting a compartment is 0.13, and the probability that the student will
either pass or get compartment is 0.96. State True or False.\\
\solution
%\input{exemplar/11/16/3/31/main.tex}
\item A card is selected from a pack of 52 cards\\
\begin{enumerate}[label=(\alph*)]
\item How many points are there in the sample space?
\item Calculate the probability that the cards is an ace of spades.
\item Calculate the probability that the card is (i) an ace (ii)black card.\\
\end{enumerate}
%\input{ncert/11/16/3/4_1/Prob_4.tex}
\item In a non-leap year, the probability of having 53 tuesdays or 53 wednesdays is\\
\solution
%\input{exemplar/11/16/3/18/main.tex}
\item There are 1000 sealed envelopes in a box, 10 of them contain a cash prize of
Rs 100 each, 100 of them contain a cash prize of Rs 50 each and 200 of them
contain a cash prize of Rs 10 each and rest do not contain any cash prize. If they
are well shuffled and an envelope is picked up out, what is the probability that it
contains no cash prize?\\
\solution
%\input{exemplar/10/13/3/34/main.tex}
\item 
A die is thrown and a card is selected at random from a deck of 52 playing cards. The probability of getting an even number on the die and a spade card.\\
\solution
%\input{exemplar/12/13/3/78/main.tex}
\item
If 4-digit numbers greater than 5,000 are randomly formed from the digits 0, 1, 3, 5, and 7, what is the probability of forming a number divisible by 5 when:
\begin{enumerate}
    \item The digits are repeated?
    \item The repetition of digits is not allowed?
\end{enumerate}
\solution
%\input{ncert/11/16/4/9/main.tex}
\item Consider the probability space $\brak{\Omega, \mathcal{G}, P}$ where $\Omega = [0,2]$ and $\mathcal{G} = \cbrak{\phi, \Omega, [0,1], (1,2]}$. Let $X$ and $Y$ be two functions on $\Omega$ defined as
\begin{align*}
    X(\omega) = 
    \begin{cases}
        1 & \text{if }\omega \in [0, 1]\\
        2 & \text{if }\omega \in (1, 2]
    \end{cases}
\end{align*}
and
\begin{align*}
    Y(\omega) = 
    \begin{cases}
        2 & \text{if }\omega \in [0, 1.5]\\
        3 & \text{if }\omega \in (1.5, 2].
    \end{cases}
\end{align*}
Then which one of the following statements is true?
\begin{enumerate}
    \item [(A)] $X$ is a random variable with respect to $\mathcal{G}$, but $Y$ is not a random variable with respect to $\mathcal{G}$.
    \item [(B)] $Y$ is a random variable with respect to $\mathcal{G}$, but $X$ is not a random variable with respect to $\mathcal{G}$.
    \item [(C)] Neither $X$ nor $Y$ is a random variable with respect to $\mathcal{G}$.
    \item [(D)] Both $X$ and $Y$ are random variables with respect to $\mathcal{G}$.
\end{enumerate} \hfill (GATE ST 2023)\\
\solution
%\input{gate/ST/2023/14/main.tex}
	\item  A die is loaded in such a way that each odd number is twice as likely to occur as
each even number. Find $P(G)$, where $G$ is the event that a number greater than
3 occurs on a single roll of the die.
\\
\solution
		%\input{exemplar/11/16/3/5/main.tex}
	\item All the jacks, queens and kings are removed from a deck of 52 playing cards. The remaining cards are well shuffled and then one card is drawn at random. Giving ace a value 1 similar value for other cards, find the probability that the card has a value 
		\begin{enumerate}
			\item 7
			\item greater than 7
			\item less than 7
		\end{enumerate}
		%\input{exemplar/10/13/3/30/main.tex}
  \item A Lot consists of 48 mobile phones of which 42 are good, 3 have only minor defects and 3 have major defects.Varnika will buy a phone if it is good but the trader will only buy a mobile if it has no major defects. One phone is selected at random from the lot. What is the probability that it is
\begin{enumerate}
	\item acceptable to Varnika?
            \item acceptable to the trader?
\end{enumerate}
\solution
	%\input{exemplar/10/13/3/40/main.tex}
 \item A student says that if you throw a die, it will show up 1 or not 1. Therefore, the probability of getting 1 and the probability of getting 'not 1' each is equal to $\frac{1}{2}$. Is this correct? Give reasons.\\
 \solution
        %\input{exemplar/10/13/2/9/main.tex}
   \item Four candidates A, B, C, D have ap-
plied for the assignment to coach a school cricket
team. If A is twice as likely to be selected as B, and
B and C are given about the same chance of being
selected, while C is twice as likely to be selected
as D, what are the probabilities that
\begin{enumerate}
\item C will be selected?
\item A will not be selected?
\end{enumerate}
	%\input{exemplar/11/16/3/9/main.tex}
 \item A bag contain 24 balls of which $x$ balls are red, $2x$ are white and $3x$ are blue. A ball is selected at random, What is the probability that it is
\begin{enumerate}[label=\alph*)]
\item not red ?
\item white ?
\end{enumerate}
%\input{exemplar/10/13/3/41/main.tex}
If the letters of the word ASSASSINATION are arranged at random. Find the Probability that
\begin{enumerate}[label=(\alph*)]
\item Four $S's$ come consecutively in the word
\item Two  $I's$ and two $N's$ come together
\item All $A's$ are not coming together
\item No two $A's$ are coming together
\end{enumerate}
%\input{exemplar/11/16/3/14/main.tex}
	\item One urn contains two black balls (labelled B1 and B2) and one white ball. A
	second urn contains one black ball and two white balls (labelled W1 and W2).
	Suppose the following experiment is performed. One of the two urns is chosen
	at random. Next a ball is randomly chosen from the urn. Then a second ball is
	chosen at random from the same urn without replacing the first ball.
	
	\begin{enumerate}
	\item What is the probability that two black balls are chosen?
	
	\item What is the probability that two balls of opposite colour are chosen?
	\end{enumerate}
	\solution
	%\input{exemplar/11/16/3/12/main1.tex}
\end{enumerate}

	\item A card is selected from a pack of 52 cards.
 \begin{enumerate}[label=(\alph*)] 
                 \item How many points are there in the sample space?
                 \item Calculate the probability that the card is an ace of spades.
                 \item Calculate the probability that the card is (i) an ace and (ii) black card.
 \end{enumerate}
\solution
		%\begin{table}[H]
	\centering
\begin{tabular}{|c|c|c|}
\hline
Random variable &Value &Definition\\ \hline
\multirow{3}{*}{X} &0 &Slips of Rs 1\\
&1 &Slips of Rs 5\\
&2 &Slips of Rs 13\\ \hline
\multirow{2}{*}{Y} &0 &Box A\\
&1 &Box B\\\hline
\end{tabular}
\caption{}
\label{tab:Distribution}
\end{table}
See \tabref{tab:Distribution}.
\begin{align}
p_{Y}\brak{k}= \begin{cases} 
      \frac{1}{3} & {k=0} \\
      \frac{2}{3 }& {k=1} 
   \end{cases}
   \\
p_{Y|X}\brak{0|0} = \frac{19}{25}\, 
p_{Y|X}\brak{0|1} = \frac{6}{25}\,
p_{Y|X}\brak{1|0} = \frac{45}{50}\,
p_{Y|X}\brak{1|2} = \frac{5}{50}
\end{align}
The desired probability is the probability that a slip drawn at random is marked other than Rs 1,
\begin{align}
&=1-p_X\brak{0}\\
&= p_X(1) + p_X(2)
\end{align}
Using Bayes theorem,
\begin{align}
&= p_Y\brak{0} \times \pr{Y=0 | X=1} + p_Y\brak{1} \times \pr{Y=1|X=2}\\
&=\frac{1}{3} \times \frac{6}{25} + \frac{2}{3} \times \frac{5}{50}\\
&=\frac{11}{75}
\end{align}

\newpage

%\tableofcontents

\bigskip

\renewcommand{\thefigure}{\theenumi}
\renewcommand{\thetable}{\theenumi}
%\renewcommand{\theequation}{\theenumi}

%\begin{abstract}
%%\boldmath
%In this letter, an algorithm for evaluating the exact analytical bit error rate  (BER)  for the piecewise linear (PL) combiner for  multiple relays is presented. Previous results were available only for upto three relays. The algorithm is unique in the sense that  the actual mathematical expressions, that are prohibitively large, need not be explicitly obtained. The diversity gain due to multiple relays is shown through plots of the analytical BER, well supported by simulations. 
%
%\end{abstract}
% IEEEtran.cls defaults to using nonbold math in the Abstract.
% This preserves the distinction between vectors and scalars. However,
% if the journal you are submitting to favors bold math in the abstract,
% then you can use LaTeX's standard command \boldmath at the very start
% of the abstract to achieve this. Many IEEE journals frown on math
% in the abstract anyway.

% Note that keywords are not normally used for peerreview papers.
%\begin{IEEEkeywords}
%Cooperative diversity, decode and forward, piecewise linear
%\end{IEEEkeywords}



% For peer review papers, you can put extra information on the cover
% page as needed:
% \ifCLASSOPTIONpeerreview
% \begin{center} \bfseries EDICS Category: 3-BBND \end{center}
% \fi
%
% For peerreview papers, this IEEEtran command inserts a page break and
% creates the second title. It will be ignored for other modes.
%\IEEEpeerreviewmaketitle




\item Four cards are drawn from a well-shuffled deck of 52 cards. What is the probability of obtaining 3 diamonds and one spade.
\\
\solution
		%\begin{enumerate}[label=\thesection.\arabic*,ref=\thesection.\theenumi]
	\item One card is drawn from a well-shuffled deck of 52 cards. Find the probability of getting
\begin{enumerate}
\item A king of red colour 
\item A face card 
\item A red face card
\item The jack of hearts
\item A spade
\item The queen of diamonds

\end{enumerate}
\solution
		%\input{ncert/10/15/1/14/main.tex}
	\item Five cards—the ten, jack, queen, king and ace of diamonds, are well-shuffled with their face downwards. One card is then picked up at random.
\begin{enumerate}
\item
What is the probability that the card is the queen? 
\item
If the queen is drawn and put aside, what is the probability that the second card picked up is (a) an ace? (b) a queen?\\
\end{enumerate}
\solution
		%\input{ncert/10/15/1/15/defs.tex}
	\item A bag contains $5$ red balls and some blue balls. If the probability of drawing a blue ball is double that if a red ball, determine the number of blue balls in the bag. 
		\\
\solution
		%\input{ncert/10/15/2/3/defs.tex}
	\item A card is selected from a pack of 52 cards.
 \begin{enumerate}[label=(\alph*)] 
                 \item How many points are there in the sample space?
                 \item Calculate the probability that the card is an ace of spades.
                 \item Calculate the probability that the card is (i) an ace and (ii) black card.
 \end{enumerate}
\solution
		%\input{ncert/11/16/3/4/main.tex}
\item Four cards are drawn from a well-shuffled deck of 52 cards. What is the probability of obtaining 3 diamonds and one spade.
\\
\solution
		%\input{ncert/11/16/4/2/defs.tex}
\item In a certain lottery 10,000 tickets are sold and ten equal prizes are awarded. What is the probability of not getting a prize if you buy (a) one ticket (b) two tickets (c) 10 tickets ?	
\\
\solution
		%\input{ncert/11/16/4/4/defs.tex}
		%
\item 
Out of 100 students, two sections of 40 and 60 are formed. If you and your friend are among the 100 students, what is the probability that
\begin{enumerate}
\item you both enter the same section?
\item you both enter the different sections?
\end{enumerate}
\solution
		%\input{ncert/11/16/4/5/defs.tex}
	\item 
The number lock of a suitcase has 4 wheels each labelled with ten digits i.e. from 0 to 9.The lock opens with a sequence of four digits with no repeats.What is the probability of a person getting the right sequence to open the suitcase.
\\
\solution
		%\input{ncert/11/16/4/10/defs.tex}
		%
\item 
Two cards are drawn at random and without replacement from a pack of 52 playing cards. Find the probability that both the cards are black.
\\
\solution
		%\input{ncert/12/13/2/2/defs.tex}
		\item A box of oranges is inspected by examining three randomly selected oranges drawn without replacement. If all the three oranges are good, the box is approved for sale, otherwise, it is rejected. Find the probability that a box containing 15 oranges out of which 12 are good and 3 are bad ones will be approved for sale.
		\label{ncert/12/13/2/3/defs.tex}
		\item Two balls are drawn at random with replacement from a box containing 10 black and 8 red balls. Find the probability that
		\label{ncert/12/13/2/12}
\begin{enumerate}
\item both balls are red.
\item first ball is black and second is red.
\item one of them is black and other is red.
\end{enumerate}

\item In a hostel, 60\% of the students read Hindi newspaper, 40\% read English newspaper and 20\% read both Hindi and English newspapers. A student is selected at random.
		\label{ncert/12/13/2/15}
\begin{enumerate}
\item Find the probability that she reads neither Hindi nor English newspapers.
\item If she reads Hindi newspaper, find the probability that she reads English newspaper.
\item If she reads English newspaper, find the probability that she reads Hindi newspaper.\\
\end{enumerate}
\item The probability of obtaining an even prime number on each die, when a pair of dice is rolled is 
\begin{enumerate}
    \item $0$ 
    
    \item $\frac{1}{3}$ 
    
    \item $\frac{1}{12}$ 
    
    \item $\frac{1}{36}$ 
\end{enumerate}
\solution
		%\input{ncert/12/13/2/17/defs.tex}
	\item A bag contains 4 red and 4 black balls, another bag contains 2 red and 6 black balls. One of the two bags is selected at random and a ball is drawn from the bag which is found to be red. Find the probability that the ball is drawn from the first bag.
\\
\solution
		%\input{ncert/12/13/3/2/main.tex}
  \item
  Cards with numbers 2 to 101 are placed in a box. A card is selected at random.Find the probability that the card has
\begin{enumerate}[label=(\roman*)]
	\item an even number 
	\item a square number
\end{enumerate}
\solution
%\input{exemplar/10/13/3/32/main.tex}
\item
The king, queen and jack of clubs are removed from a deck of 52 playing cards and then well shuffled. Now one card is drawn at random from the remaining cards.  Determine the probability that the card is
\begin{enumerate}[label=(\roman*)]
\item a club
\item 10 of hearts
\end{enumerate}
\solution
%\input{exemplar/10/13/3/29/main.tex}
\item A team of medical students doing their internship have to assist during surgeries
at a city hospital. The probabilities of surgeries rated as very complex, complex,
routine, simple or very simple are respectively, 0.15, 0.20, 0.31, 0.26, .08. Find
the probabilities that a particular surgery will be rated
\begin{enumerate}
	\item complex or very complex;
	\item neither very complex nor very simple;
	\item routine or complex
	\item routine or simple
\end{enumerate}
\solution
%\input{exemplar/11/16/3/8(1)/main.tex}
\item A card is selected from a pack of 52 cards.
\begin{enumerate}[label=(\alph*)]
    \item How many points are there in the sample space?
    \item Calculate the probability that the card is an ace of spades.
    \item Calculate the probability that the card is (i) an ace and (ii) black card.
\end{enumerate}
\solution
%\input{exemplar/11/16/3/4/main2.tex}
\item The probability that a non leap year selected at random will contain 53 sundays.
\\
\solution
%\input{exemplar/10/13/1/19/main.tex}
\item One of the four persons John, Rita, Aslam or Gurpreet will be promoted next
month. Consequently the sample space consists of four elementary outcomes
S = {John promoted, Rita promoted, Aslam promoted, Gurpreet promoted}
You are told that the chances of John’s promotion is same as that of Gurpreet,
Rita’s chances of promotion are twice as likely as Johns. Aslam’s chances are
four times that of John.
\begin{enumerate}
	\item Determine
	\begin{enumerate}
		\item P (John promoted)
		\item P (Rita promoted)
		\item P (Aslam promoted)
		\item P (Gurpreet promoted)
	\end{enumerate}
	\item If A = {John promoted or Gurpreet promoted}, find P (A).
\end{enumerate}
\solution
%\input{exemplar/11/16/3/10/main.tex}
\item A card is drawn from a deck of 52 cards. Find the probability of getting a king or a heart or a red card.\\
\solution
%\input{exemplar/11/16/3/15/main.tex}
\item The probability that a student will pass his examination is 0.73, the probability of
the student getting a compartment is 0.13, and the probability that the student will
either pass or get compartment is 0.96. State True or False.\\
\solution
%\input{exemplar/11/16/3/31/main.tex}
\item A card is selected from a pack of 52 cards\\
\begin{enumerate}[label=(\alph*)]
\item How many points are there in the sample space?
\item Calculate the probability that the cards is an ace of spades.
\item Calculate the probability that the card is (i) an ace (ii)black card.\\
\end{enumerate}
%\input{ncert/11/16/3/4_1/Prob_4.tex}
\item In a non-leap year, the probability of having 53 tuesdays or 53 wednesdays is\\
\solution
%\input{exemplar/11/16/3/18/main.tex}
\item There are 1000 sealed envelopes in a box, 10 of them contain a cash prize of
Rs 100 each, 100 of them contain a cash prize of Rs 50 each and 200 of them
contain a cash prize of Rs 10 each and rest do not contain any cash prize. If they
are well shuffled and an envelope is picked up out, what is the probability that it
contains no cash prize?\\
\solution
%\input{exemplar/10/13/3/34/main.tex}
\item 
A die is thrown and a card is selected at random from a deck of 52 playing cards. The probability of getting an even number on the die and a spade card.\\
\solution
%\input{exemplar/12/13/3/78/main.tex}
\item
If 4-digit numbers greater than 5,000 are randomly formed from the digits 0, 1, 3, 5, and 7, what is the probability of forming a number divisible by 5 when:
\begin{enumerate}
    \item The digits are repeated?
    \item The repetition of digits is not allowed?
\end{enumerate}
\solution
%\input{ncert/11/16/4/9/main.tex}
\item Consider the probability space $\brak{\Omega, \mathcal{G}, P}$ where $\Omega = [0,2]$ and $\mathcal{G} = \cbrak{\phi, \Omega, [0,1], (1,2]}$. Let $X$ and $Y$ be two functions on $\Omega$ defined as
\begin{align*}
    X(\omega) = 
    \begin{cases}
        1 & \text{if }\omega \in [0, 1]\\
        2 & \text{if }\omega \in (1, 2]
    \end{cases}
\end{align*}
and
\begin{align*}
    Y(\omega) = 
    \begin{cases}
        2 & \text{if }\omega \in [0, 1.5]\\
        3 & \text{if }\omega \in (1.5, 2].
    \end{cases}
\end{align*}
Then which one of the following statements is true?
\begin{enumerate}
    \item [(A)] $X$ is a random variable with respect to $\mathcal{G}$, but $Y$ is not a random variable with respect to $\mathcal{G}$.
    \item [(B)] $Y$ is a random variable with respect to $\mathcal{G}$, but $X$ is not a random variable with respect to $\mathcal{G}$.
    \item [(C)] Neither $X$ nor $Y$ is a random variable with respect to $\mathcal{G}$.
    \item [(D)] Both $X$ and $Y$ are random variables with respect to $\mathcal{G}$.
\end{enumerate} \hfill (GATE ST 2023)\\
\solution
%\input{gate/ST/2023/14/main.tex}
	\item  A die is loaded in such a way that each odd number is twice as likely to occur as
each even number. Find $P(G)$, where $G$ is the event that a number greater than
3 occurs on a single roll of the die.
\\
\solution
		%\input{exemplar/11/16/3/5/main.tex}
	\item All the jacks, queens and kings are removed from a deck of 52 playing cards. The remaining cards are well shuffled and then one card is drawn at random. Giving ace a value 1 similar value for other cards, find the probability that the card has a value 
		\begin{enumerate}
			\item 7
			\item greater than 7
			\item less than 7
		\end{enumerate}
		%\input{exemplar/10/13/3/30/main.tex}
  \item A Lot consists of 48 mobile phones of which 42 are good, 3 have only minor defects and 3 have major defects.Varnika will buy a phone if it is good but the trader will only buy a mobile if it has no major defects. One phone is selected at random from the lot. What is the probability that it is
\begin{enumerate}
	\item acceptable to Varnika?
            \item acceptable to the trader?
\end{enumerate}
\solution
	%\input{exemplar/10/13/3/40/main.tex}
 \item A student says that if you throw a die, it will show up 1 or not 1. Therefore, the probability of getting 1 and the probability of getting 'not 1' each is equal to $\frac{1}{2}$. Is this correct? Give reasons.\\
 \solution
        %\input{exemplar/10/13/2/9/main.tex}
   \item Four candidates A, B, C, D have ap-
plied for the assignment to coach a school cricket
team. If A is twice as likely to be selected as B, and
B and C are given about the same chance of being
selected, while C is twice as likely to be selected
as D, what are the probabilities that
\begin{enumerate}
\item C will be selected?
\item A will not be selected?
\end{enumerate}
	%\input{exemplar/11/16/3/9/main.tex}
 \item A bag contain 24 balls of which $x$ balls are red, $2x$ are white and $3x$ are blue. A ball is selected at random, What is the probability that it is
\begin{enumerate}[label=\alph*)]
\item not red ?
\item white ?
\end{enumerate}
%\input{exemplar/10/13/3/41/main.tex}
If the letters of the word ASSASSINATION are arranged at random. Find the Probability that
\begin{enumerate}[label=(\alph*)]
\item Four $S's$ come consecutively in the word
\item Two  $I's$ and two $N's$ come together
\item All $A's$ are not coming together
\item No two $A's$ are coming together
\end{enumerate}
%\input{exemplar/11/16/3/14/main.tex}
	\item One urn contains two black balls (labelled B1 and B2) and one white ball. A
	second urn contains one black ball and two white balls (labelled W1 and W2).
	Suppose the following experiment is performed. One of the two urns is chosen
	at random. Next a ball is randomly chosen from the urn. Then a second ball is
	chosen at random from the same urn without replacing the first ball.
	
	\begin{enumerate}
	\item What is the probability that two black balls are chosen?
	
	\item What is the probability that two balls of opposite colour are chosen?
	\end{enumerate}
	\solution
	%\input{exemplar/11/16/3/12/main1.tex}
\end{enumerate}

\item In a certain lottery 10,000 tickets are sold and ten equal prizes are awarded. What is the probability of not getting a prize if you buy (a) one ticket (b) two tickets (c) 10 tickets ?	
\\
\solution
		%\begin{enumerate}[label=\thesection.\arabic*,ref=\thesection.\theenumi]
	\item One card is drawn from a well-shuffled deck of 52 cards. Find the probability of getting
\begin{enumerate}
\item A king of red colour 
\item A face card 
\item A red face card
\item The jack of hearts
\item A spade
\item The queen of diamonds

\end{enumerate}
\solution
		%\input{ncert/10/15/1/14/main.tex}
	\item Five cards—the ten, jack, queen, king and ace of diamonds, are well-shuffled with their face downwards. One card is then picked up at random.
\begin{enumerate}
\item
What is the probability that the card is the queen? 
\item
If the queen is drawn and put aside, what is the probability that the second card picked up is (a) an ace? (b) a queen?\\
\end{enumerate}
\solution
		%\input{ncert/10/15/1/15/defs.tex}
	\item A bag contains $5$ red balls and some blue balls. If the probability of drawing a blue ball is double that if a red ball, determine the number of blue balls in the bag. 
		\\
\solution
		%\input{ncert/10/15/2/3/defs.tex}
	\item A card is selected from a pack of 52 cards.
 \begin{enumerate}[label=(\alph*)] 
                 \item How many points are there in the sample space?
                 \item Calculate the probability that the card is an ace of spades.
                 \item Calculate the probability that the card is (i) an ace and (ii) black card.
 \end{enumerate}
\solution
		%\input{ncert/11/16/3/4/main.tex}
\item Four cards are drawn from a well-shuffled deck of 52 cards. What is the probability of obtaining 3 diamonds and one spade.
\\
\solution
		%\input{ncert/11/16/4/2/defs.tex}
\item In a certain lottery 10,000 tickets are sold and ten equal prizes are awarded. What is the probability of not getting a prize if you buy (a) one ticket (b) two tickets (c) 10 tickets ?	
\\
\solution
		%\input{ncert/11/16/4/4/defs.tex}
		%
\item 
Out of 100 students, two sections of 40 and 60 are formed. If you and your friend are among the 100 students, what is the probability that
\begin{enumerate}
\item you both enter the same section?
\item you both enter the different sections?
\end{enumerate}
\solution
		%\input{ncert/11/16/4/5/defs.tex}
	\item 
The number lock of a suitcase has 4 wheels each labelled with ten digits i.e. from 0 to 9.The lock opens with a sequence of four digits with no repeats.What is the probability of a person getting the right sequence to open the suitcase.
\\
\solution
		%\input{ncert/11/16/4/10/defs.tex}
		%
\item 
Two cards are drawn at random and without replacement from a pack of 52 playing cards. Find the probability that both the cards are black.
\\
\solution
		%\input{ncert/12/13/2/2/defs.tex}
		\item A box of oranges is inspected by examining three randomly selected oranges drawn without replacement. If all the three oranges are good, the box is approved for sale, otherwise, it is rejected. Find the probability that a box containing 15 oranges out of which 12 are good and 3 are bad ones will be approved for sale.
		\label{ncert/12/13/2/3/defs.tex}
		\item Two balls are drawn at random with replacement from a box containing 10 black and 8 red balls. Find the probability that
		\label{ncert/12/13/2/12}
\begin{enumerate}
\item both balls are red.
\item first ball is black and second is red.
\item one of them is black and other is red.
\end{enumerate}

\item In a hostel, 60\% of the students read Hindi newspaper, 40\% read English newspaper and 20\% read both Hindi and English newspapers. A student is selected at random.
		\label{ncert/12/13/2/15}
\begin{enumerate}
\item Find the probability that she reads neither Hindi nor English newspapers.
\item If she reads Hindi newspaper, find the probability that she reads English newspaper.
\item If she reads English newspaper, find the probability that she reads Hindi newspaper.\\
\end{enumerate}
\item The probability of obtaining an even prime number on each die, when a pair of dice is rolled is 
\begin{enumerate}
    \item $0$ 
    
    \item $\frac{1}{3}$ 
    
    \item $\frac{1}{12}$ 
    
    \item $\frac{1}{36}$ 
\end{enumerate}
\solution
		%\input{ncert/12/13/2/17/defs.tex}
	\item A bag contains 4 red and 4 black balls, another bag contains 2 red and 6 black balls. One of the two bags is selected at random and a ball is drawn from the bag which is found to be red. Find the probability that the ball is drawn from the first bag.
\\
\solution
		%\input{ncert/12/13/3/2/main.tex}
  \item
  Cards with numbers 2 to 101 are placed in a box. A card is selected at random.Find the probability that the card has
\begin{enumerate}[label=(\roman*)]
	\item an even number 
	\item a square number
\end{enumerate}
\solution
%\input{exemplar/10/13/3/32/main.tex}
\item
The king, queen and jack of clubs are removed from a deck of 52 playing cards and then well shuffled. Now one card is drawn at random from the remaining cards.  Determine the probability that the card is
\begin{enumerate}[label=(\roman*)]
\item a club
\item 10 of hearts
\end{enumerate}
\solution
%\input{exemplar/10/13/3/29/main.tex}
\item A team of medical students doing their internship have to assist during surgeries
at a city hospital. The probabilities of surgeries rated as very complex, complex,
routine, simple or very simple are respectively, 0.15, 0.20, 0.31, 0.26, .08. Find
the probabilities that a particular surgery will be rated
\begin{enumerate}
	\item complex or very complex;
	\item neither very complex nor very simple;
	\item routine or complex
	\item routine or simple
\end{enumerate}
\solution
%\input{exemplar/11/16/3/8(1)/main.tex}
\item A card is selected from a pack of 52 cards.
\begin{enumerate}[label=(\alph*)]
    \item How many points are there in the sample space?
    \item Calculate the probability that the card is an ace of spades.
    \item Calculate the probability that the card is (i) an ace and (ii) black card.
\end{enumerate}
\solution
%\input{exemplar/11/16/3/4/main2.tex}
\item The probability that a non leap year selected at random will contain 53 sundays.
\\
\solution
%\input{exemplar/10/13/1/19/main.tex}
\item One of the four persons John, Rita, Aslam or Gurpreet will be promoted next
month. Consequently the sample space consists of four elementary outcomes
S = {John promoted, Rita promoted, Aslam promoted, Gurpreet promoted}
You are told that the chances of John’s promotion is same as that of Gurpreet,
Rita’s chances of promotion are twice as likely as Johns. Aslam’s chances are
four times that of John.
\begin{enumerate}
	\item Determine
	\begin{enumerate}
		\item P (John promoted)
		\item P (Rita promoted)
		\item P (Aslam promoted)
		\item P (Gurpreet promoted)
	\end{enumerate}
	\item If A = {John promoted or Gurpreet promoted}, find P (A).
\end{enumerate}
\solution
%\input{exemplar/11/16/3/10/main.tex}
\item A card is drawn from a deck of 52 cards. Find the probability of getting a king or a heart or a red card.\\
\solution
%\input{exemplar/11/16/3/15/main.tex}
\item The probability that a student will pass his examination is 0.73, the probability of
the student getting a compartment is 0.13, and the probability that the student will
either pass or get compartment is 0.96. State True or False.\\
\solution
%\input{exemplar/11/16/3/31/main.tex}
\item A card is selected from a pack of 52 cards\\
\begin{enumerate}[label=(\alph*)]
\item How many points are there in the sample space?
\item Calculate the probability that the cards is an ace of spades.
\item Calculate the probability that the card is (i) an ace (ii)black card.\\
\end{enumerate}
%\input{ncert/11/16/3/4_1/Prob_4.tex}
\item In a non-leap year, the probability of having 53 tuesdays or 53 wednesdays is\\
\solution
%\input{exemplar/11/16/3/18/main.tex}
\item There are 1000 sealed envelopes in a box, 10 of them contain a cash prize of
Rs 100 each, 100 of them contain a cash prize of Rs 50 each and 200 of them
contain a cash prize of Rs 10 each and rest do not contain any cash prize. If they
are well shuffled and an envelope is picked up out, what is the probability that it
contains no cash prize?\\
\solution
%\input{exemplar/10/13/3/34/main.tex}
\item 
A die is thrown and a card is selected at random from a deck of 52 playing cards. The probability of getting an even number on the die and a spade card.\\
\solution
%\input{exemplar/12/13/3/78/main.tex}
\item
If 4-digit numbers greater than 5,000 are randomly formed from the digits 0, 1, 3, 5, and 7, what is the probability of forming a number divisible by 5 when:
\begin{enumerate}
    \item The digits are repeated?
    \item The repetition of digits is not allowed?
\end{enumerate}
\solution
%\input{ncert/11/16/4/9/main.tex}
\item Consider the probability space $\brak{\Omega, \mathcal{G}, P}$ where $\Omega = [0,2]$ and $\mathcal{G} = \cbrak{\phi, \Omega, [0,1], (1,2]}$. Let $X$ and $Y$ be two functions on $\Omega$ defined as
\begin{align*}
    X(\omega) = 
    \begin{cases}
        1 & \text{if }\omega \in [0, 1]\\
        2 & \text{if }\omega \in (1, 2]
    \end{cases}
\end{align*}
and
\begin{align*}
    Y(\omega) = 
    \begin{cases}
        2 & \text{if }\omega \in [0, 1.5]\\
        3 & \text{if }\omega \in (1.5, 2].
    \end{cases}
\end{align*}
Then which one of the following statements is true?
\begin{enumerate}
    \item [(A)] $X$ is a random variable with respect to $\mathcal{G}$, but $Y$ is not a random variable with respect to $\mathcal{G}$.
    \item [(B)] $Y$ is a random variable with respect to $\mathcal{G}$, but $X$ is not a random variable with respect to $\mathcal{G}$.
    \item [(C)] Neither $X$ nor $Y$ is a random variable with respect to $\mathcal{G}$.
    \item [(D)] Both $X$ and $Y$ are random variables with respect to $\mathcal{G}$.
\end{enumerate} \hfill (GATE ST 2023)\\
\solution
%\input{gate/ST/2023/14/main.tex}
	\item  A die is loaded in such a way that each odd number is twice as likely to occur as
each even number. Find $P(G)$, where $G$ is the event that a number greater than
3 occurs on a single roll of the die.
\\
\solution
		%\input{exemplar/11/16/3/5/main.tex}
	\item All the jacks, queens and kings are removed from a deck of 52 playing cards. The remaining cards are well shuffled and then one card is drawn at random. Giving ace a value 1 similar value for other cards, find the probability that the card has a value 
		\begin{enumerate}
			\item 7
			\item greater than 7
			\item less than 7
		\end{enumerate}
		%\input{exemplar/10/13/3/30/main.tex}
  \item A Lot consists of 48 mobile phones of which 42 are good, 3 have only minor defects and 3 have major defects.Varnika will buy a phone if it is good but the trader will only buy a mobile if it has no major defects. One phone is selected at random from the lot. What is the probability that it is
\begin{enumerate}
	\item acceptable to Varnika?
            \item acceptable to the trader?
\end{enumerate}
\solution
	%\input{exemplar/10/13/3/40/main.tex}
 \item A student says that if you throw a die, it will show up 1 or not 1. Therefore, the probability of getting 1 and the probability of getting 'not 1' each is equal to $\frac{1}{2}$. Is this correct? Give reasons.\\
 \solution
        %\input{exemplar/10/13/2/9/main.tex}
   \item Four candidates A, B, C, D have ap-
plied for the assignment to coach a school cricket
team. If A is twice as likely to be selected as B, and
B and C are given about the same chance of being
selected, while C is twice as likely to be selected
as D, what are the probabilities that
\begin{enumerate}
\item C will be selected?
\item A will not be selected?
\end{enumerate}
	%\input{exemplar/11/16/3/9/main.tex}
 \item A bag contain 24 balls of which $x$ balls are red, $2x$ are white and $3x$ are blue. A ball is selected at random, What is the probability that it is
\begin{enumerate}[label=\alph*)]
\item not red ?
\item white ?
\end{enumerate}
%\input{exemplar/10/13/3/41/main.tex}
If the letters of the word ASSASSINATION are arranged at random. Find the Probability that
\begin{enumerate}[label=(\alph*)]
\item Four $S's$ come consecutively in the word
\item Two  $I's$ and two $N's$ come together
\item All $A's$ are not coming together
\item No two $A's$ are coming together
\end{enumerate}
%\input{exemplar/11/16/3/14/main.tex}
	\item One urn contains two black balls (labelled B1 and B2) and one white ball. A
	second urn contains one black ball and two white balls (labelled W1 and W2).
	Suppose the following experiment is performed. One of the two urns is chosen
	at random. Next a ball is randomly chosen from the urn. Then a second ball is
	chosen at random from the same urn without replacing the first ball.
	
	\begin{enumerate}
	\item What is the probability that two black balls are chosen?
	
	\item What is the probability that two balls of opposite colour are chosen?
	\end{enumerate}
	\solution
	%\input{exemplar/11/16/3/12/main1.tex}
\end{enumerate}

		%
\item 
Out of 100 students, two sections of 40 and 60 are formed. If you and your friend are among the 100 students, what is the probability that
\begin{enumerate}
\item you both enter the same section?
\item you both enter the different sections?
\end{enumerate}
\solution
		%\begin{enumerate}[label=\thesection.\arabic*,ref=\thesection.\theenumi]
	\item One card is drawn from a well-shuffled deck of 52 cards. Find the probability of getting
\begin{enumerate}
\item A king of red colour 
\item A face card 
\item A red face card
\item The jack of hearts
\item A spade
\item The queen of diamonds

\end{enumerate}
\solution
		%\input{ncert/10/15/1/14/main.tex}
	\item Five cards—the ten, jack, queen, king and ace of diamonds, are well-shuffled with their face downwards. One card is then picked up at random.
\begin{enumerate}
\item
What is the probability that the card is the queen? 
\item
If the queen is drawn and put aside, what is the probability that the second card picked up is (a) an ace? (b) a queen?\\
\end{enumerate}
\solution
		%\input{ncert/10/15/1/15/defs.tex}
	\item A bag contains $5$ red balls and some blue balls. If the probability of drawing a blue ball is double that if a red ball, determine the number of blue balls in the bag. 
		\\
\solution
		%\input{ncert/10/15/2/3/defs.tex}
	\item A card is selected from a pack of 52 cards.
 \begin{enumerate}[label=(\alph*)] 
                 \item How many points are there in the sample space?
                 \item Calculate the probability that the card is an ace of spades.
                 \item Calculate the probability that the card is (i) an ace and (ii) black card.
 \end{enumerate}
\solution
		%\input{ncert/11/16/3/4/main.tex}
\item Four cards are drawn from a well-shuffled deck of 52 cards. What is the probability of obtaining 3 diamonds and one spade.
\\
\solution
		%\input{ncert/11/16/4/2/defs.tex}
\item In a certain lottery 10,000 tickets are sold and ten equal prizes are awarded. What is the probability of not getting a prize if you buy (a) one ticket (b) two tickets (c) 10 tickets ?	
\\
\solution
		%\input{ncert/11/16/4/4/defs.tex}
		%
\item 
Out of 100 students, two sections of 40 and 60 are formed. If you and your friend are among the 100 students, what is the probability that
\begin{enumerate}
\item you both enter the same section?
\item you both enter the different sections?
\end{enumerate}
\solution
		%\input{ncert/11/16/4/5/defs.tex}
	\item 
The number lock of a suitcase has 4 wheels each labelled with ten digits i.e. from 0 to 9.The lock opens with a sequence of four digits with no repeats.What is the probability of a person getting the right sequence to open the suitcase.
\\
\solution
		%\input{ncert/11/16/4/10/defs.tex}
		%
\item 
Two cards are drawn at random and without replacement from a pack of 52 playing cards. Find the probability that both the cards are black.
\\
\solution
		%\input{ncert/12/13/2/2/defs.tex}
		\item A box of oranges is inspected by examining three randomly selected oranges drawn without replacement. If all the three oranges are good, the box is approved for sale, otherwise, it is rejected. Find the probability that a box containing 15 oranges out of which 12 are good and 3 are bad ones will be approved for sale.
		\label{ncert/12/13/2/3/defs.tex}
		\item Two balls are drawn at random with replacement from a box containing 10 black and 8 red balls. Find the probability that
		\label{ncert/12/13/2/12}
\begin{enumerate}
\item both balls are red.
\item first ball is black and second is red.
\item one of them is black and other is red.
\end{enumerate}

\item In a hostel, 60\% of the students read Hindi newspaper, 40\% read English newspaper and 20\% read both Hindi and English newspapers. A student is selected at random.
		\label{ncert/12/13/2/15}
\begin{enumerate}
\item Find the probability that she reads neither Hindi nor English newspapers.
\item If she reads Hindi newspaper, find the probability that she reads English newspaper.
\item If she reads English newspaper, find the probability that she reads Hindi newspaper.\\
\end{enumerate}
\item The probability of obtaining an even prime number on each die, when a pair of dice is rolled is 
\begin{enumerate}
    \item $0$ 
    
    \item $\frac{1}{3}$ 
    
    \item $\frac{1}{12}$ 
    
    \item $\frac{1}{36}$ 
\end{enumerate}
\solution
		%\input{ncert/12/13/2/17/defs.tex}
	\item A bag contains 4 red and 4 black balls, another bag contains 2 red and 6 black balls. One of the two bags is selected at random and a ball is drawn from the bag which is found to be red. Find the probability that the ball is drawn from the first bag.
\\
\solution
		%\input{ncert/12/13/3/2/main.tex}
  \item
  Cards with numbers 2 to 101 are placed in a box. A card is selected at random.Find the probability that the card has
\begin{enumerate}[label=(\roman*)]
	\item an even number 
	\item a square number
\end{enumerate}
\solution
%\input{exemplar/10/13/3/32/main.tex}
\item
The king, queen and jack of clubs are removed from a deck of 52 playing cards and then well shuffled. Now one card is drawn at random from the remaining cards.  Determine the probability that the card is
\begin{enumerate}[label=(\roman*)]
\item a club
\item 10 of hearts
\end{enumerate}
\solution
%\input{exemplar/10/13/3/29/main.tex}
\item A team of medical students doing their internship have to assist during surgeries
at a city hospital. The probabilities of surgeries rated as very complex, complex,
routine, simple or very simple are respectively, 0.15, 0.20, 0.31, 0.26, .08. Find
the probabilities that a particular surgery will be rated
\begin{enumerate}
	\item complex or very complex;
	\item neither very complex nor very simple;
	\item routine or complex
	\item routine or simple
\end{enumerate}
\solution
%\input{exemplar/11/16/3/8(1)/main.tex}
\item A card is selected from a pack of 52 cards.
\begin{enumerate}[label=(\alph*)]
    \item How many points are there in the sample space?
    \item Calculate the probability that the card is an ace of spades.
    \item Calculate the probability that the card is (i) an ace and (ii) black card.
\end{enumerate}
\solution
%\input{exemplar/11/16/3/4/main2.tex}
\item The probability that a non leap year selected at random will contain 53 sundays.
\\
\solution
%\input{exemplar/10/13/1/19/main.tex}
\item One of the four persons John, Rita, Aslam or Gurpreet will be promoted next
month. Consequently the sample space consists of four elementary outcomes
S = {John promoted, Rita promoted, Aslam promoted, Gurpreet promoted}
You are told that the chances of John’s promotion is same as that of Gurpreet,
Rita’s chances of promotion are twice as likely as Johns. Aslam’s chances are
four times that of John.
\begin{enumerate}
	\item Determine
	\begin{enumerate}
		\item P (John promoted)
		\item P (Rita promoted)
		\item P (Aslam promoted)
		\item P (Gurpreet promoted)
	\end{enumerate}
	\item If A = {John promoted or Gurpreet promoted}, find P (A).
\end{enumerate}
\solution
%\input{exemplar/11/16/3/10/main.tex}
\item A card is drawn from a deck of 52 cards. Find the probability of getting a king or a heart or a red card.\\
\solution
%\input{exemplar/11/16/3/15/main.tex}
\item The probability that a student will pass his examination is 0.73, the probability of
the student getting a compartment is 0.13, and the probability that the student will
either pass or get compartment is 0.96. State True or False.\\
\solution
%\input{exemplar/11/16/3/31/main.tex}
\item A card is selected from a pack of 52 cards\\
\begin{enumerate}[label=(\alph*)]
\item How many points are there in the sample space?
\item Calculate the probability that the cards is an ace of spades.
\item Calculate the probability that the card is (i) an ace (ii)black card.\\
\end{enumerate}
%\input{ncert/11/16/3/4_1/Prob_4.tex}
\item In a non-leap year, the probability of having 53 tuesdays or 53 wednesdays is\\
\solution
%\input{exemplar/11/16/3/18/main.tex}
\item There are 1000 sealed envelopes in a box, 10 of them contain a cash prize of
Rs 100 each, 100 of them contain a cash prize of Rs 50 each and 200 of them
contain a cash prize of Rs 10 each and rest do not contain any cash prize. If they
are well shuffled and an envelope is picked up out, what is the probability that it
contains no cash prize?\\
\solution
%\input{exemplar/10/13/3/34/main.tex}
\item 
A die is thrown and a card is selected at random from a deck of 52 playing cards. The probability of getting an even number on the die and a spade card.\\
\solution
%\input{exemplar/12/13/3/78/main.tex}
\item
If 4-digit numbers greater than 5,000 are randomly formed from the digits 0, 1, 3, 5, and 7, what is the probability of forming a number divisible by 5 when:
\begin{enumerate}
    \item The digits are repeated?
    \item The repetition of digits is not allowed?
\end{enumerate}
\solution
%\input{ncert/11/16/4/9/main.tex}
\item Consider the probability space $\brak{\Omega, \mathcal{G}, P}$ where $\Omega = [0,2]$ and $\mathcal{G} = \cbrak{\phi, \Omega, [0,1], (1,2]}$. Let $X$ and $Y$ be two functions on $\Omega$ defined as
\begin{align*}
    X(\omega) = 
    \begin{cases}
        1 & \text{if }\omega \in [0, 1]\\
        2 & \text{if }\omega \in (1, 2]
    \end{cases}
\end{align*}
and
\begin{align*}
    Y(\omega) = 
    \begin{cases}
        2 & \text{if }\omega \in [0, 1.5]\\
        3 & \text{if }\omega \in (1.5, 2].
    \end{cases}
\end{align*}
Then which one of the following statements is true?
\begin{enumerate}
    \item [(A)] $X$ is a random variable with respect to $\mathcal{G}$, but $Y$ is not a random variable with respect to $\mathcal{G}$.
    \item [(B)] $Y$ is a random variable with respect to $\mathcal{G}$, but $X$ is not a random variable with respect to $\mathcal{G}$.
    \item [(C)] Neither $X$ nor $Y$ is a random variable with respect to $\mathcal{G}$.
    \item [(D)] Both $X$ and $Y$ are random variables with respect to $\mathcal{G}$.
\end{enumerate} \hfill (GATE ST 2023)\\
\solution
%\input{gate/ST/2023/14/main.tex}
	\item  A die is loaded in such a way that each odd number is twice as likely to occur as
each even number. Find $P(G)$, where $G$ is the event that a number greater than
3 occurs on a single roll of the die.
\\
\solution
		%\input{exemplar/11/16/3/5/main.tex}
	\item All the jacks, queens and kings are removed from a deck of 52 playing cards. The remaining cards are well shuffled and then one card is drawn at random. Giving ace a value 1 similar value for other cards, find the probability that the card has a value 
		\begin{enumerate}
			\item 7
			\item greater than 7
			\item less than 7
		\end{enumerate}
		%\input{exemplar/10/13/3/30/main.tex}
  \item A Lot consists of 48 mobile phones of which 42 are good, 3 have only minor defects and 3 have major defects.Varnika will buy a phone if it is good but the trader will only buy a mobile if it has no major defects. One phone is selected at random from the lot. What is the probability that it is
\begin{enumerate}
	\item acceptable to Varnika?
            \item acceptable to the trader?
\end{enumerate}
\solution
	%\input{exemplar/10/13/3/40/main.tex}
 \item A student says that if you throw a die, it will show up 1 or not 1. Therefore, the probability of getting 1 and the probability of getting 'not 1' each is equal to $\frac{1}{2}$. Is this correct? Give reasons.\\
 \solution
        %\input{exemplar/10/13/2/9/main.tex}
   \item Four candidates A, B, C, D have ap-
plied for the assignment to coach a school cricket
team. If A is twice as likely to be selected as B, and
B and C are given about the same chance of being
selected, while C is twice as likely to be selected
as D, what are the probabilities that
\begin{enumerate}
\item C will be selected?
\item A will not be selected?
\end{enumerate}
	%\input{exemplar/11/16/3/9/main.tex}
 \item A bag contain 24 balls of which $x$ balls are red, $2x$ are white and $3x$ are blue. A ball is selected at random, What is the probability that it is
\begin{enumerate}[label=\alph*)]
\item not red ?
\item white ?
\end{enumerate}
%\input{exemplar/10/13/3/41/main.tex}
If the letters of the word ASSASSINATION are arranged at random. Find the Probability that
\begin{enumerate}[label=(\alph*)]
\item Four $S's$ come consecutively in the word
\item Two  $I's$ and two $N's$ come together
\item All $A's$ are not coming together
\item No two $A's$ are coming together
\end{enumerate}
%\input{exemplar/11/16/3/14/main.tex}
	\item One urn contains two black balls (labelled B1 and B2) and one white ball. A
	second urn contains one black ball and two white balls (labelled W1 and W2).
	Suppose the following experiment is performed. One of the two urns is chosen
	at random. Next a ball is randomly chosen from the urn. Then a second ball is
	chosen at random from the same urn without replacing the first ball.
	
	\begin{enumerate}
	\item What is the probability that two black balls are chosen?
	
	\item What is the probability that two balls of opposite colour are chosen?
	\end{enumerate}
	\solution
	%\input{exemplar/11/16/3/12/main1.tex}
\end{enumerate}

	\item 
The number lock of a suitcase has 4 wheels each labelled with ten digits i.e. from 0 to 9.The lock opens with a sequence of four digits with no repeats.What is the probability of a person getting the right sequence to open the suitcase.
\\
\solution
		%\begin{enumerate}[label=\thesection.\arabic*,ref=\thesection.\theenumi]
	\item One card is drawn from a well-shuffled deck of 52 cards. Find the probability of getting
\begin{enumerate}
\item A king of red colour 
\item A face card 
\item A red face card
\item The jack of hearts
\item A spade
\item The queen of diamonds

\end{enumerate}
\solution
		%\input{ncert/10/15/1/14/main.tex}
	\item Five cards—the ten, jack, queen, king and ace of diamonds, are well-shuffled with their face downwards. One card is then picked up at random.
\begin{enumerate}
\item
What is the probability that the card is the queen? 
\item
If the queen is drawn and put aside, what is the probability that the second card picked up is (a) an ace? (b) a queen?\\
\end{enumerate}
\solution
		%\input{ncert/10/15/1/15/defs.tex}
	\item A bag contains $5$ red balls and some blue balls. If the probability of drawing a blue ball is double that if a red ball, determine the number of blue balls in the bag. 
		\\
\solution
		%\input{ncert/10/15/2/3/defs.tex}
	\item A card is selected from a pack of 52 cards.
 \begin{enumerate}[label=(\alph*)] 
                 \item How many points are there in the sample space?
                 \item Calculate the probability that the card is an ace of spades.
                 \item Calculate the probability that the card is (i) an ace and (ii) black card.
 \end{enumerate}
\solution
		%\input{ncert/11/16/3/4/main.tex}
\item Four cards are drawn from a well-shuffled deck of 52 cards. What is the probability of obtaining 3 diamonds and one spade.
\\
\solution
		%\input{ncert/11/16/4/2/defs.tex}
\item In a certain lottery 10,000 tickets are sold and ten equal prizes are awarded. What is the probability of not getting a prize if you buy (a) one ticket (b) two tickets (c) 10 tickets ?	
\\
\solution
		%\input{ncert/11/16/4/4/defs.tex}
		%
\item 
Out of 100 students, two sections of 40 and 60 are formed. If you and your friend are among the 100 students, what is the probability that
\begin{enumerate}
\item you both enter the same section?
\item you both enter the different sections?
\end{enumerate}
\solution
		%\input{ncert/11/16/4/5/defs.tex}
	\item 
The number lock of a suitcase has 4 wheels each labelled with ten digits i.e. from 0 to 9.The lock opens with a sequence of four digits with no repeats.What is the probability of a person getting the right sequence to open the suitcase.
\\
\solution
		%\input{ncert/11/16/4/10/defs.tex}
		%
\item 
Two cards are drawn at random and without replacement from a pack of 52 playing cards. Find the probability that both the cards are black.
\\
\solution
		%\input{ncert/12/13/2/2/defs.tex}
		\item A box of oranges is inspected by examining three randomly selected oranges drawn without replacement. If all the three oranges are good, the box is approved for sale, otherwise, it is rejected. Find the probability that a box containing 15 oranges out of which 12 are good and 3 are bad ones will be approved for sale.
		\label{ncert/12/13/2/3/defs.tex}
		\item Two balls are drawn at random with replacement from a box containing 10 black and 8 red balls. Find the probability that
		\label{ncert/12/13/2/12}
\begin{enumerate}
\item both balls are red.
\item first ball is black and second is red.
\item one of them is black and other is red.
\end{enumerate}

\item In a hostel, 60\% of the students read Hindi newspaper, 40\% read English newspaper and 20\% read both Hindi and English newspapers. A student is selected at random.
		\label{ncert/12/13/2/15}
\begin{enumerate}
\item Find the probability that she reads neither Hindi nor English newspapers.
\item If she reads Hindi newspaper, find the probability that she reads English newspaper.
\item If she reads English newspaper, find the probability that she reads Hindi newspaper.\\
\end{enumerate}
\item The probability of obtaining an even prime number on each die, when a pair of dice is rolled is 
\begin{enumerate}
    \item $0$ 
    
    \item $\frac{1}{3}$ 
    
    \item $\frac{1}{12}$ 
    
    \item $\frac{1}{36}$ 
\end{enumerate}
\solution
		%\input{ncert/12/13/2/17/defs.tex}
	\item A bag contains 4 red and 4 black balls, another bag contains 2 red and 6 black balls. One of the two bags is selected at random and a ball is drawn from the bag which is found to be red. Find the probability that the ball is drawn from the first bag.
\\
\solution
		%\input{ncert/12/13/3/2/main.tex}
  \item
  Cards with numbers 2 to 101 are placed in a box. A card is selected at random.Find the probability that the card has
\begin{enumerate}[label=(\roman*)]
	\item an even number 
	\item a square number
\end{enumerate}
\solution
%\input{exemplar/10/13/3/32/main.tex}
\item
The king, queen and jack of clubs are removed from a deck of 52 playing cards and then well shuffled. Now one card is drawn at random from the remaining cards.  Determine the probability that the card is
\begin{enumerate}[label=(\roman*)]
\item a club
\item 10 of hearts
\end{enumerate}
\solution
%\input{exemplar/10/13/3/29/main.tex}
\item A team of medical students doing their internship have to assist during surgeries
at a city hospital. The probabilities of surgeries rated as very complex, complex,
routine, simple or very simple are respectively, 0.15, 0.20, 0.31, 0.26, .08. Find
the probabilities that a particular surgery will be rated
\begin{enumerate}
	\item complex or very complex;
	\item neither very complex nor very simple;
	\item routine or complex
	\item routine or simple
\end{enumerate}
\solution
%\input{exemplar/11/16/3/8(1)/main.tex}
\item A card is selected from a pack of 52 cards.
\begin{enumerate}[label=(\alph*)]
    \item How many points are there in the sample space?
    \item Calculate the probability that the card is an ace of spades.
    \item Calculate the probability that the card is (i) an ace and (ii) black card.
\end{enumerate}
\solution
%\input{exemplar/11/16/3/4/main2.tex}
\item The probability that a non leap year selected at random will contain 53 sundays.
\\
\solution
%\input{exemplar/10/13/1/19/main.tex}
\item One of the four persons John, Rita, Aslam or Gurpreet will be promoted next
month. Consequently the sample space consists of four elementary outcomes
S = {John promoted, Rita promoted, Aslam promoted, Gurpreet promoted}
You are told that the chances of John’s promotion is same as that of Gurpreet,
Rita’s chances of promotion are twice as likely as Johns. Aslam’s chances are
four times that of John.
\begin{enumerate}
	\item Determine
	\begin{enumerate}
		\item P (John promoted)
		\item P (Rita promoted)
		\item P (Aslam promoted)
		\item P (Gurpreet promoted)
	\end{enumerate}
	\item If A = {John promoted or Gurpreet promoted}, find P (A).
\end{enumerate}
\solution
%\input{exemplar/11/16/3/10/main.tex}
\item A card is drawn from a deck of 52 cards. Find the probability of getting a king or a heart or a red card.\\
\solution
%\input{exemplar/11/16/3/15/main.tex}
\item The probability that a student will pass his examination is 0.73, the probability of
the student getting a compartment is 0.13, and the probability that the student will
either pass or get compartment is 0.96. State True or False.\\
\solution
%\input{exemplar/11/16/3/31/main.tex}
\item A card is selected from a pack of 52 cards\\
\begin{enumerate}[label=(\alph*)]
\item How many points are there in the sample space?
\item Calculate the probability that the cards is an ace of spades.
\item Calculate the probability that the card is (i) an ace (ii)black card.\\
\end{enumerate}
%\input{ncert/11/16/3/4_1/Prob_4.tex}
\item In a non-leap year, the probability of having 53 tuesdays or 53 wednesdays is\\
\solution
%\input{exemplar/11/16/3/18/main.tex}
\item There are 1000 sealed envelopes in a box, 10 of them contain a cash prize of
Rs 100 each, 100 of them contain a cash prize of Rs 50 each and 200 of them
contain a cash prize of Rs 10 each and rest do not contain any cash prize. If they
are well shuffled and an envelope is picked up out, what is the probability that it
contains no cash prize?\\
\solution
%\input{exemplar/10/13/3/34/main.tex}
\item 
A die is thrown and a card is selected at random from a deck of 52 playing cards. The probability of getting an even number on the die and a spade card.\\
\solution
%\input{exemplar/12/13/3/78/main.tex}
\item
If 4-digit numbers greater than 5,000 are randomly formed from the digits 0, 1, 3, 5, and 7, what is the probability of forming a number divisible by 5 when:
\begin{enumerate}
    \item The digits are repeated?
    \item The repetition of digits is not allowed?
\end{enumerate}
\solution
%\input{ncert/11/16/4/9/main.tex}
\item Consider the probability space $\brak{\Omega, \mathcal{G}, P}$ where $\Omega = [0,2]$ and $\mathcal{G} = \cbrak{\phi, \Omega, [0,1], (1,2]}$. Let $X$ and $Y$ be two functions on $\Omega$ defined as
\begin{align*}
    X(\omega) = 
    \begin{cases}
        1 & \text{if }\omega \in [0, 1]\\
        2 & \text{if }\omega \in (1, 2]
    \end{cases}
\end{align*}
and
\begin{align*}
    Y(\omega) = 
    \begin{cases}
        2 & \text{if }\omega \in [0, 1.5]\\
        3 & \text{if }\omega \in (1.5, 2].
    \end{cases}
\end{align*}
Then which one of the following statements is true?
\begin{enumerate}
    \item [(A)] $X$ is a random variable with respect to $\mathcal{G}$, but $Y$ is not a random variable with respect to $\mathcal{G}$.
    \item [(B)] $Y$ is a random variable with respect to $\mathcal{G}$, but $X$ is not a random variable with respect to $\mathcal{G}$.
    \item [(C)] Neither $X$ nor $Y$ is a random variable with respect to $\mathcal{G}$.
    \item [(D)] Both $X$ and $Y$ are random variables with respect to $\mathcal{G}$.
\end{enumerate} \hfill (GATE ST 2023)\\
\solution
%\input{gate/ST/2023/14/main.tex}
	\item  A die is loaded in such a way that each odd number is twice as likely to occur as
each even number. Find $P(G)$, where $G$ is the event that a number greater than
3 occurs on a single roll of the die.
\\
\solution
		%\input{exemplar/11/16/3/5/main.tex}
	\item All the jacks, queens and kings are removed from a deck of 52 playing cards. The remaining cards are well shuffled and then one card is drawn at random. Giving ace a value 1 similar value for other cards, find the probability that the card has a value 
		\begin{enumerate}
			\item 7
			\item greater than 7
			\item less than 7
		\end{enumerate}
		%\input{exemplar/10/13/3/30/main.tex}
  \item A Lot consists of 48 mobile phones of which 42 are good, 3 have only minor defects and 3 have major defects.Varnika will buy a phone if it is good but the trader will only buy a mobile if it has no major defects. One phone is selected at random from the lot. What is the probability that it is
\begin{enumerate}
	\item acceptable to Varnika?
            \item acceptable to the trader?
\end{enumerate}
\solution
	%\input{exemplar/10/13/3/40/main.tex}
 \item A student says that if you throw a die, it will show up 1 or not 1. Therefore, the probability of getting 1 and the probability of getting 'not 1' each is equal to $\frac{1}{2}$. Is this correct? Give reasons.\\
 \solution
        %\input{exemplar/10/13/2/9/main.tex}
   \item Four candidates A, B, C, D have ap-
plied for the assignment to coach a school cricket
team. If A is twice as likely to be selected as B, and
B and C are given about the same chance of being
selected, while C is twice as likely to be selected
as D, what are the probabilities that
\begin{enumerate}
\item C will be selected?
\item A will not be selected?
\end{enumerate}
	%\input{exemplar/11/16/3/9/main.tex}
 \item A bag contain 24 balls of which $x$ balls are red, $2x$ are white and $3x$ are blue. A ball is selected at random, What is the probability that it is
\begin{enumerate}[label=\alph*)]
\item not red ?
\item white ?
\end{enumerate}
%\input{exemplar/10/13/3/41/main.tex}
If the letters of the word ASSASSINATION are arranged at random. Find the Probability that
\begin{enumerate}[label=(\alph*)]
\item Four $S's$ come consecutively in the word
\item Two  $I's$ and two $N's$ come together
\item All $A's$ are not coming together
\item No two $A's$ are coming together
\end{enumerate}
%\input{exemplar/11/16/3/14/main.tex}
	\item One urn contains two black balls (labelled B1 and B2) and one white ball. A
	second urn contains one black ball and two white balls (labelled W1 and W2).
	Suppose the following experiment is performed. One of the two urns is chosen
	at random. Next a ball is randomly chosen from the urn. Then a second ball is
	chosen at random from the same urn without replacing the first ball.
	
	\begin{enumerate}
	\item What is the probability that two black balls are chosen?
	
	\item What is the probability that two balls of opposite colour are chosen?
	\end{enumerate}
	\solution
	%\input{exemplar/11/16/3/12/main1.tex}
\end{enumerate}

		%
\item 
Two cards are drawn at random and without replacement from a pack of 52 playing cards. Find the probability that both the cards are black.
\\
\solution
		%\begin{enumerate}[label=\thesection.\arabic*,ref=\thesection.\theenumi]
	\item One card is drawn from a well-shuffled deck of 52 cards. Find the probability of getting
\begin{enumerate}
\item A king of red colour 
\item A face card 
\item A red face card
\item The jack of hearts
\item A spade
\item The queen of diamonds

\end{enumerate}
\solution
		%\input{ncert/10/15/1/14/main.tex}
	\item Five cards—the ten, jack, queen, king and ace of diamonds, are well-shuffled with their face downwards. One card is then picked up at random.
\begin{enumerate}
\item
What is the probability that the card is the queen? 
\item
If the queen is drawn and put aside, what is the probability that the second card picked up is (a) an ace? (b) a queen?\\
\end{enumerate}
\solution
		%\input{ncert/10/15/1/15/defs.tex}
	\item A bag contains $5$ red balls and some blue balls. If the probability of drawing a blue ball is double that if a red ball, determine the number of blue balls in the bag. 
		\\
\solution
		%\input{ncert/10/15/2/3/defs.tex}
	\item A card is selected from a pack of 52 cards.
 \begin{enumerate}[label=(\alph*)] 
                 \item How many points are there in the sample space?
                 \item Calculate the probability that the card is an ace of spades.
                 \item Calculate the probability that the card is (i) an ace and (ii) black card.
 \end{enumerate}
\solution
		%\input{ncert/11/16/3/4/main.tex}
\item Four cards are drawn from a well-shuffled deck of 52 cards. What is the probability of obtaining 3 diamonds and one spade.
\\
\solution
		%\input{ncert/11/16/4/2/defs.tex}
\item In a certain lottery 10,000 tickets are sold and ten equal prizes are awarded. What is the probability of not getting a prize if you buy (a) one ticket (b) two tickets (c) 10 tickets ?	
\\
\solution
		%\input{ncert/11/16/4/4/defs.tex}
		%
\item 
Out of 100 students, two sections of 40 and 60 are formed. If you and your friend are among the 100 students, what is the probability that
\begin{enumerate}
\item you both enter the same section?
\item you both enter the different sections?
\end{enumerate}
\solution
		%\input{ncert/11/16/4/5/defs.tex}
	\item 
The number lock of a suitcase has 4 wheels each labelled with ten digits i.e. from 0 to 9.The lock opens with a sequence of four digits with no repeats.What is the probability of a person getting the right sequence to open the suitcase.
\\
\solution
		%\input{ncert/11/16/4/10/defs.tex}
		%
\item 
Two cards are drawn at random and without replacement from a pack of 52 playing cards. Find the probability that both the cards are black.
\\
\solution
		%\input{ncert/12/13/2/2/defs.tex}
		\item A box of oranges is inspected by examining three randomly selected oranges drawn without replacement. If all the three oranges are good, the box is approved for sale, otherwise, it is rejected. Find the probability that a box containing 15 oranges out of which 12 are good and 3 are bad ones will be approved for sale.
		\label{ncert/12/13/2/3/defs.tex}
		\item Two balls are drawn at random with replacement from a box containing 10 black and 8 red balls. Find the probability that
		\label{ncert/12/13/2/12}
\begin{enumerate}
\item both balls are red.
\item first ball is black and second is red.
\item one of them is black and other is red.
\end{enumerate}

\item In a hostel, 60\% of the students read Hindi newspaper, 40\% read English newspaper and 20\% read both Hindi and English newspapers. A student is selected at random.
		\label{ncert/12/13/2/15}
\begin{enumerate}
\item Find the probability that she reads neither Hindi nor English newspapers.
\item If she reads Hindi newspaper, find the probability that she reads English newspaper.
\item If she reads English newspaper, find the probability that she reads Hindi newspaper.\\
\end{enumerate}
\item The probability of obtaining an even prime number on each die, when a pair of dice is rolled is 
\begin{enumerate}
    \item $0$ 
    
    \item $\frac{1}{3}$ 
    
    \item $\frac{1}{12}$ 
    
    \item $\frac{1}{36}$ 
\end{enumerate}
\solution
		%\input{ncert/12/13/2/17/defs.tex}
	\item A bag contains 4 red and 4 black balls, another bag contains 2 red and 6 black balls. One of the two bags is selected at random and a ball is drawn from the bag which is found to be red. Find the probability that the ball is drawn from the first bag.
\\
\solution
		%\input{ncert/12/13/3/2/main.tex}
  \item
  Cards with numbers 2 to 101 are placed in a box. A card is selected at random.Find the probability that the card has
\begin{enumerate}[label=(\roman*)]
	\item an even number 
	\item a square number
\end{enumerate}
\solution
%\input{exemplar/10/13/3/32/main.tex}
\item
The king, queen and jack of clubs are removed from a deck of 52 playing cards and then well shuffled. Now one card is drawn at random from the remaining cards.  Determine the probability that the card is
\begin{enumerate}[label=(\roman*)]
\item a club
\item 10 of hearts
\end{enumerate}
\solution
%\input{exemplar/10/13/3/29/main.tex}
\item A team of medical students doing their internship have to assist during surgeries
at a city hospital. The probabilities of surgeries rated as very complex, complex,
routine, simple or very simple are respectively, 0.15, 0.20, 0.31, 0.26, .08. Find
the probabilities that a particular surgery will be rated
\begin{enumerate}
	\item complex or very complex;
	\item neither very complex nor very simple;
	\item routine or complex
	\item routine or simple
\end{enumerate}
\solution
%\input{exemplar/11/16/3/8(1)/main.tex}
\item A card is selected from a pack of 52 cards.
\begin{enumerate}[label=(\alph*)]
    \item How many points are there in the sample space?
    \item Calculate the probability that the card is an ace of spades.
    \item Calculate the probability that the card is (i) an ace and (ii) black card.
\end{enumerate}
\solution
%\input{exemplar/11/16/3/4/main2.tex}
\item The probability that a non leap year selected at random will contain 53 sundays.
\\
\solution
%\input{exemplar/10/13/1/19/main.tex}
\item One of the four persons John, Rita, Aslam or Gurpreet will be promoted next
month. Consequently the sample space consists of four elementary outcomes
S = {John promoted, Rita promoted, Aslam promoted, Gurpreet promoted}
You are told that the chances of John’s promotion is same as that of Gurpreet,
Rita’s chances of promotion are twice as likely as Johns. Aslam’s chances are
four times that of John.
\begin{enumerate}
	\item Determine
	\begin{enumerate}
		\item P (John promoted)
		\item P (Rita promoted)
		\item P (Aslam promoted)
		\item P (Gurpreet promoted)
	\end{enumerate}
	\item If A = {John promoted or Gurpreet promoted}, find P (A).
\end{enumerate}
\solution
%\input{exemplar/11/16/3/10/main.tex}
\item A card is drawn from a deck of 52 cards. Find the probability of getting a king or a heart or a red card.\\
\solution
%\input{exemplar/11/16/3/15/main.tex}
\item The probability that a student will pass his examination is 0.73, the probability of
the student getting a compartment is 0.13, and the probability that the student will
either pass or get compartment is 0.96. State True or False.\\
\solution
%\input{exemplar/11/16/3/31/main.tex}
\item A card is selected from a pack of 52 cards\\
\begin{enumerate}[label=(\alph*)]
\item How many points are there in the sample space?
\item Calculate the probability that the cards is an ace of spades.
\item Calculate the probability that the card is (i) an ace (ii)black card.\\
\end{enumerate}
%\input{ncert/11/16/3/4_1/Prob_4.tex}
\item In a non-leap year, the probability of having 53 tuesdays or 53 wednesdays is\\
\solution
%\input{exemplar/11/16/3/18/main.tex}
\item There are 1000 sealed envelopes in a box, 10 of them contain a cash prize of
Rs 100 each, 100 of them contain a cash prize of Rs 50 each and 200 of them
contain a cash prize of Rs 10 each and rest do not contain any cash prize. If they
are well shuffled and an envelope is picked up out, what is the probability that it
contains no cash prize?\\
\solution
%\input{exemplar/10/13/3/34/main.tex}
\item 
A die is thrown and a card is selected at random from a deck of 52 playing cards. The probability of getting an even number on the die and a spade card.\\
\solution
%\input{exemplar/12/13/3/78/main.tex}
\item
If 4-digit numbers greater than 5,000 are randomly formed from the digits 0, 1, 3, 5, and 7, what is the probability of forming a number divisible by 5 when:
\begin{enumerate}
    \item The digits are repeated?
    \item The repetition of digits is not allowed?
\end{enumerate}
\solution
%\input{ncert/11/16/4/9/main.tex}
\item Consider the probability space $\brak{\Omega, \mathcal{G}, P}$ where $\Omega = [0,2]$ and $\mathcal{G} = \cbrak{\phi, \Omega, [0,1], (1,2]}$. Let $X$ and $Y$ be two functions on $\Omega$ defined as
\begin{align*}
    X(\omega) = 
    \begin{cases}
        1 & \text{if }\omega \in [0, 1]\\
        2 & \text{if }\omega \in (1, 2]
    \end{cases}
\end{align*}
and
\begin{align*}
    Y(\omega) = 
    \begin{cases}
        2 & \text{if }\omega \in [0, 1.5]\\
        3 & \text{if }\omega \in (1.5, 2].
    \end{cases}
\end{align*}
Then which one of the following statements is true?
\begin{enumerate}
    \item [(A)] $X$ is a random variable with respect to $\mathcal{G}$, but $Y$ is not a random variable with respect to $\mathcal{G}$.
    \item [(B)] $Y$ is a random variable with respect to $\mathcal{G}$, but $X$ is not a random variable with respect to $\mathcal{G}$.
    \item [(C)] Neither $X$ nor $Y$ is a random variable with respect to $\mathcal{G}$.
    \item [(D)] Both $X$ and $Y$ are random variables with respect to $\mathcal{G}$.
\end{enumerate} \hfill (GATE ST 2023)\\
\solution
%\input{gate/ST/2023/14/main.tex}
	\item  A die is loaded in such a way that each odd number is twice as likely to occur as
each even number. Find $P(G)$, where $G$ is the event that a number greater than
3 occurs on a single roll of the die.
\\
\solution
		%\input{exemplar/11/16/3/5/main.tex}
	\item All the jacks, queens and kings are removed from a deck of 52 playing cards. The remaining cards are well shuffled and then one card is drawn at random. Giving ace a value 1 similar value for other cards, find the probability that the card has a value 
		\begin{enumerate}
			\item 7
			\item greater than 7
			\item less than 7
		\end{enumerate}
		%\input{exemplar/10/13/3/30/main.tex}
  \item A Lot consists of 48 mobile phones of which 42 are good, 3 have only minor defects and 3 have major defects.Varnika will buy a phone if it is good but the trader will only buy a mobile if it has no major defects. One phone is selected at random from the lot. What is the probability that it is
\begin{enumerate}
	\item acceptable to Varnika?
            \item acceptable to the trader?
\end{enumerate}
\solution
	%\input{exemplar/10/13/3/40/main.tex}
 \item A student says that if you throw a die, it will show up 1 or not 1. Therefore, the probability of getting 1 and the probability of getting 'not 1' each is equal to $\frac{1}{2}$. Is this correct? Give reasons.\\
 \solution
        %\input{exemplar/10/13/2/9/main.tex}
   \item Four candidates A, B, C, D have ap-
plied for the assignment to coach a school cricket
team. If A is twice as likely to be selected as B, and
B and C are given about the same chance of being
selected, while C is twice as likely to be selected
as D, what are the probabilities that
\begin{enumerate}
\item C will be selected?
\item A will not be selected?
\end{enumerate}
	%\input{exemplar/11/16/3/9/main.tex}
 \item A bag contain 24 balls of which $x$ balls are red, $2x$ are white and $3x$ are blue. A ball is selected at random, What is the probability that it is
\begin{enumerate}[label=\alph*)]
\item not red ?
\item white ?
\end{enumerate}
%\input{exemplar/10/13/3/41/main.tex}
If the letters of the word ASSASSINATION are arranged at random. Find the Probability that
\begin{enumerate}[label=(\alph*)]
\item Four $S's$ come consecutively in the word
\item Two  $I's$ and two $N's$ come together
\item All $A's$ are not coming together
\item No two $A's$ are coming together
\end{enumerate}
%\input{exemplar/11/16/3/14/main.tex}
	\item One urn contains two black balls (labelled B1 and B2) and one white ball. A
	second urn contains one black ball and two white balls (labelled W1 and W2).
	Suppose the following experiment is performed. One of the two urns is chosen
	at random. Next a ball is randomly chosen from the urn. Then a second ball is
	chosen at random from the same urn without replacing the first ball.
	
	\begin{enumerate}
	\item What is the probability that two black balls are chosen?
	
	\item What is the probability that two balls of opposite colour are chosen?
	\end{enumerate}
	\solution
	%\input{exemplar/11/16/3/12/main1.tex}
\end{enumerate}

		\item A box of oranges is inspected by examining three randomly selected oranges drawn without replacement. If all the three oranges are good, the box is approved for sale, otherwise, it is rejected. Find the probability that a box containing 15 oranges out of which 12 are good and 3 are bad ones will be approved for sale.
		\label{ncert/12/13/2/3/defs.tex}
		\item Two balls are drawn at random with replacement from a box containing 10 black and 8 red balls. Find the probability that
		\label{ncert/12/13/2/12}
\begin{enumerate}
\item both balls are red.
\item first ball is black and second is red.
\item one of them is black and other is red.
\end{enumerate}

\item In a hostel, 60\% of the students read Hindi newspaper, 40\% read English newspaper and 20\% read both Hindi and English newspapers. A student is selected at random.
		\label{ncert/12/13/2/15}
\begin{enumerate}
\item Find the probability that she reads neither Hindi nor English newspapers.
\item If she reads Hindi newspaper, find the probability that she reads English newspaper.
\item If she reads English newspaper, find the probability that she reads Hindi newspaper.\\
\end{enumerate}
\item The probability of obtaining an even prime number on each die, when a pair of dice is rolled is 
\begin{enumerate}
    \item $0$ 
    
    \item $\frac{1}{3}$ 
    
    \item $\frac{1}{12}$ 
    
    \item $\frac{1}{36}$ 
\end{enumerate}
\solution
		%\begin{enumerate}[label=\thesection.\arabic*,ref=\thesection.\theenumi]
	\item One card is drawn from a well-shuffled deck of 52 cards. Find the probability of getting
\begin{enumerate}
\item A king of red colour 
\item A face card 
\item A red face card
\item The jack of hearts
\item A spade
\item The queen of diamonds

\end{enumerate}
\solution
		%\input{ncert/10/15/1/14/main.tex}
	\item Five cards—the ten, jack, queen, king and ace of diamonds, are well-shuffled with their face downwards. One card is then picked up at random.
\begin{enumerate}
\item
What is the probability that the card is the queen? 
\item
If the queen is drawn and put aside, what is the probability that the second card picked up is (a) an ace? (b) a queen?\\
\end{enumerate}
\solution
		%\input{ncert/10/15/1/15/defs.tex}
	\item A bag contains $5$ red balls and some blue balls. If the probability of drawing a blue ball is double that if a red ball, determine the number of blue balls in the bag. 
		\\
\solution
		%\input{ncert/10/15/2/3/defs.tex}
	\item A card is selected from a pack of 52 cards.
 \begin{enumerate}[label=(\alph*)] 
                 \item How many points are there in the sample space?
                 \item Calculate the probability that the card is an ace of spades.
                 \item Calculate the probability that the card is (i) an ace and (ii) black card.
 \end{enumerate}
\solution
		%\input{ncert/11/16/3/4/main.tex}
\item Four cards are drawn from a well-shuffled deck of 52 cards. What is the probability of obtaining 3 diamonds and one spade.
\\
\solution
		%\input{ncert/11/16/4/2/defs.tex}
\item In a certain lottery 10,000 tickets are sold and ten equal prizes are awarded. What is the probability of not getting a prize if you buy (a) one ticket (b) two tickets (c) 10 tickets ?	
\\
\solution
		%\input{ncert/11/16/4/4/defs.tex}
		%
\item 
Out of 100 students, two sections of 40 and 60 are formed. If you and your friend are among the 100 students, what is the probability that
\begin{enumerate}
\item you both enter the same section?
\item you both enter the different sections?
\end{enumerate}
\solution
		%\input{ncert/11/16/4/5/defs.tex}
	\item 
The number lock of a suitcase has 4 wheels each labelled with ten digits i.e. from 0 to 9.The lock opens with a sequence of four digits with no repeats.What is the probability of a person getting the right sequence to open the suitcase.
\\
\solution
		%\input{ncert/11/16/4/10/defs.tex}
		%
\item 
Two cards are drawn at random and without replacement from a pack of 52 playing cards. Find the probability that both the cards are black.
\\
\solution
		%\input{ncert/12/13/2/2/defs.tex}
		\item A box of oranges is inspected by examining three randomly selected oranges drawn without replacement. If all the three oranges are good, the box is approved for sale, otherwise, it is rejected. Find the probability that a box containing 15 oranges out of which 12 are good and 3 are bad ones will be approved for sale.
		\label{ncert/12/13/2/3/defs.tex}
		\item Two balls are drawn at random with replacement from a box containing 10 black and 8 red balls. Find the probability that
		\label{ncert/12/13/2/12}
\begin{enumerate}
\item both balls are red.
\item first ball is black and second is red.
\item one of them is black and other is red.
\end{enumerate}

\item In a hostel, 60\% of the students read Hindi newspaper, 40\% read English newspaper and 20\% read both Hindi and English newspapers. A student is selected at random.
		\label{ncert/12/13/2/15}
\begin{enumerate}
\item Find the probability that she reads neither Hindi nor English newspapers.
\item If she reads Hindi newspaper, find the probability that she reads English newspaper.
\item If she reads English newspaper, find the probability that she reads Hindi newspaper.\\
\end{enumerate}
\item The probability of obtaining an even prime number on each die, when a pair of dice is rolled is 
\begin{enumerate}
    \item $0$ 
    
    \item $\frac{1}{3}$ 
    
    \item $\frac{1}{12}$ 
    
    \item $\frac{1}{36}$ 
\end{enumerate}
\solution
		%\input{ncert/12/13/2/17/defs.tex}
	\item A bag contains 4 red and 4 black balls, another bag contains 2 red and 6 black balls. One of the two bags is selected at random and a ball is drawn from the bag which is found to be red. Find the probability that the ball is drawn from the first bag.
\\
\solution
		%\input{ncert/12/13/3/2/main.tex}
  \item
  Cards with numbers 2 to 101 are placed in a box. A card is selected at random.Find the probability that the card has
\begin{enumerate}[label=(\roman*)]
	\item an even number 
	\item a square number
\end{enumerate}
\solution
%\input{exemplar/10/13/3/32/main.tex}
\item
The king, queen and jack of clubs are removed from a deck of 52 playing cards and then well shuffled. Now one card is drawn at random from the remaining cards.  Determine the probability that the card is
\begin{enumerate}[label=(\roman*)]
\item a club
\item 10 of hearts
\end{enumerate}
\solution
%\input{exemplar/10/13/3/29/main.tex}
\item A team of medical students doing their internship have to assist during surgeries
at a city hospital. The probabilities of surgeries rated as very complex, complex,
routine, simple or very simple are respectively, 0.15, 0.20, 0.31, 0.26, .08. Find
the probabilities that a particular surgery will be rated
\begin{enumerate}
	\item complex or very complex;
	\item neither very complex nor very simple;
	\item routine or complex
	\item routine or simple
\end{enumerate}
\solution
%\input{exemplar/11/16/3/8(1)/main.tex}
\item A card is selected from a pack of 52 cards.
\begin{enumerate}[label=(\alph*)]
    \item How many points are there in the sample space?
    \item Calculate the probability that the card is an ace of spades.
    \item Calculate the probability that the card is (i) an ace and (ii) black card.
\end{enumerate}
\solution
%\input{exemplar/11/16/3/4/main2.tex}
\item The probability that a non leap year selected at random will contain 53 sundays.
\\
\solution
%\input{exemplar/10/13/1/19/main.tex}
\item One of the four persons John, Rita, Aslam or Gurpreet will be promoted next
month. Consequently the sample space consists of four elementary outcomes
S = {John promoted, Rita promoted, Aslam promoted, Gurpreet promoted}
You are told that the chances of John’s promotion is same as that of Gurpreet,
Rita’s chances of promotion are twice as likely as Johns. Aslam’s chances are
four times that of John.
\begin{enumerate}
	\item Determine
	\begin{enumerate}
		\item P (John promoted)
		\item P (Rita promoted)
		\item P (Aslam promoted)
		\item P (Gurpreet promoted)
	\end{enumerate}
	\item If A = {John promoted or Gurpreet promoted}, find P (A).
\end{enumerate}
\solution
%\input{exemplar/11/16/3/10/main.tex}
\item A card is drawn from a deck of 52 cards. Find the probability of getting a king or a heart or a red card.\\
\solution
%\input{exemplar/11/16/3/15/main.tex}
\item The probability that a student will pass his examination is 0.73, the probability of
the student getting a compartment is 0.13, and the probability that the student will
either pass or get compartment is 0.96. State True or False.\\
\solution
%\input{exemplar/11/16/3/31/main.tex}
\item A card is selected from a pack of 52 cards\\
\begin{enumerate}[label=(\alph*)]
\item How many points are there in the sample space?
\item Calculate the probability that the cards is an ace of spades.
\item Calculate the probability that the card is (i) an ace (ii)black card.\\
\end{enumerate}
%\input{ncert/11/16/3/4_1/Prob_4.tex}
\item In a non-leap year, the probability of having 53 tuesdays or 53 wednesdays is\\
\solution
%\input{exemplar/11/16/3/18/main.tex}
\item There are 1000 sealed envelopes in a box, 10 of them contain a cash prize of
Rs 100 each, 100 of them contain a cash prize of Rs 50 each and 200 of them
contain a cash prize of Rs 10 each and rest do not contain any cash prize. If they
are well shuffled and an envelope is picked up out, what is the probability that it
contains no cash prize?\\
\solution
%\input{exemplar/10/13/3/34/main.tex}
\item 
A die is thrown and a card is selected at random from a deck of 52 playing cards. The probability of getting an even number on the die and a spade card.\\
\solution
%\input{exemplar/12/13/3/78/main.tex}
\item
If 4-digit numbers greater than 5,000 are randomly formed from the digits 0, 1, 3, 5, and 7, what is the probability of forming a number divisible by 5 when:
\begin{enumerate}
    \item The digits are repeated?
    \item The repetition of digits is not allowed?
\end{enumerate}
\solution
%\input{ncert/11/16/4/9/main.tex}
\item Consider the probability space $\brak{\Omega, \mathcal{G}, P}$ where $\Omega = [0,2]$ and $\mathcal{G} = \cbrak{\phi, \Omega, [0,1], (1,2]}$. Let $X$ and $Y$ be two functions on $\Omega$ defined as
\begin{align*}
    X(\omega) = 
    \begin{cases}
        1 & \text{if }\omega \in [0, 1]\\
        2 & \text{if }\omega \in (1, 2]
    \end{cases}
\end{align*}
and
\begin{align*}
    Y(\omega) = 
    \begin{cases}
        2 & \text{if }\omega \in [0, 1.5]\\
        3 & \text{if }\omega \in (1.5, 2].
    \end{cases}
\end{align*}
Then which one of the following statements is true?
\begin{enumerate}
    \item [(A)] $X$ is a random variable with respect to $\mathcal{G}$, but $Y$ is not a random variable with respect to $\mathcal{G}$.
    \item [(B)] $Y$ is a random variable with respect to $\mathcal{G}$, but $X$ is not a random variable with respect to $\mathcal{G}$.
    \item [(C)] Neither $X$ nor $Y$ is a random variable with respect to $\mathcal{G}$.
    \item [(D)] Both $X$ and $Y$ are random variables with respect to $\mathcal{G}$.
\end{enumerate} \hfill (GATE ST 2023)\\
\solution
%\input{gate/ST/2023/14/main.tex}
	\item  A die is loaded in such a way that each odd number is twice as likely to occur as
each even number. Find $P(G)$, where $G$ is the event that a number greater than
3 occurs on a single roll of the die.
\\
\solution
		%\input{exemplar/11/16/3/5/main.tex}
	\item All the jacks, queens and kings are removed from a deck of 52 playing cards. The remaining cards are well shuffled and then one card is drawn at random. Giving ace a value 1 similar value for other cards, find the probability that the card has a value 
		\begin{enumerate}
			\item 7
			\item greater than 7
			\item less than 7
		\end{enumerate}
		%\input{exemplar/10/13/3/30/main.tex}
  \item A Lot consists of 48 mobile phones of which 42 are good, 3 have only minor defects and 3 have major defects.Varnika will buy a phone if it is good but the trader will only buy a mobile if it has no major defects. One phone is selected at random from the lot. What is the probability that it is
\begin{enumerate}
	\item acceptable to Varnika?
            \item acceptable to the trader?
\end{enumerate}
\solution
	%\input{exemplar/10/13/3/40/main.tex}
 \item A student says that if you throw a die, it will show up 1 or not 1. Therefore, the probability of getting 1 and the probability of getting 'not 1' each is equal to $\frac{1}{2}$. Is this correct? Give reasons.\\
 \solution
        %\input{exemplar/10/13/2/9/main.tex}
   \item Four candidates A, B, C, D have ap-
plied for the assignment to coach a school cricket
team. If A is twice as likely to be selected as B, and
B and C are given about the same chance of being
selected, while C is twice as likely to be selected
as D, what are the probabilities that
\begin{enumerate}
\item C will be selected?
\item A will not be selected?
\end{enumerate}
	%\input{exemplar/11/16/3/9/main.tex}
 \item A bag contain 24 balls of which $x$ balls are red, $2x$ are white and $3x$ are blue. A ball is selected at random, What is the probability that it is
\begin{enumerate}[label=\alph*)]
\item not red ?
\item white ?
\end{enumerate}
%\input{exemplar/10/13/3/41/main.tex}
If the letters of the word ASSASSINATION are arranged at random. Find the Probability that
\begin{enumerate}[label=(\alph*)]
\item Four $S's$ come consecutively in the word
\item Two  $I's$ and two $N's$ come together
\item All $A's$ are not coming together
\item No two $A's$ are coming together
\end{enumerate}
%\input{exemplar/11/16/3/14/main.tex}
	\item One urn contains two black balls (labelled B1 and B2) and one white ball. A
	second urn contains one black ball and two white balls (labelled W1 and W2).
	Suppose the following experiment is performed. One of the two urns is chosen
	at random. Next a ball is randomly chosen from the urn. Then a second ball is
	chosen at random from the same urn without replacing the first ball.
	
	\begin{enumerate}
	\item What is the probability that two black balls are chosen?
	
	\item What is the probability that two balls of opposite colour are chosen?
	\end{enumerate}
	\solution
	%\input{exemplar/11/16/3/12/main1.tex}
\end{enumerate}

	\item A bag contains 4 red and 4 black balls, another bag contains 2 red and 6 black balls. One of the two bags is selected at random and a ball is drawn from the bag which is found to be red. Find the probability that the ball is drawn from the first bag.
\\
\solution
		%\begin{table}[H]
	\centering
\begin{tabular}{|c|c|c|}
\hline
Random variable &Value &Definition\\ \hline
\multirow{3}{*}{X} &0 &Slips of Rs 1\\
&1 &Slips of Rs 5\\
&2 &Slips of Rs 13\\ \hline
\multirow{2}{*}{Y} &0 &Box A\\
&1 &Box B\\\hline
\end{tabular}
\caption{}
\label{tab:Distribution}
\end{table}
See \tabref{tab:Distribution}.
\begin{align}
p_{Y}\brak{k}= \begin{cases} 
      \frac{1}{3} & {k=0} \\
      \frac{2}{3 }& {k=1} 
   \end{cases}
   \\
p_{Y|X}\brak{0|0} = \frac{19}{25}\, 
p_{Y|X}\brak{0|1} = \frac{6}{25}\,
p_{Y|X}\brak{1|0} = \frac{45}{50}\,
p_{Y|X}\brak{1|2} = \frac{5}{50}
\end{align}
The desired probability is the probability that a slip drawn at random is marked other than Rs 1,
\begin{align}
&=1-p_X\brak{0}\\
&= p_X(1) + p_X(2)
\end{align}
Using Bayes theorem,
\begin{align}
&= p_Y\brak{0} \times \pr{Y=0 | X=1} + p_Y\brak{1} \times \pr{Y=1|X=2}\\
&=\frac{1}{3} \times \frac{6}{25} + \frac{2}{3} \times \frac{5}{50}\\
&=\frac{11}{75}
\end{align}

\newpage

%\tableofcontents

\bigskip

\renewcommand{\thefigure}{\theenumi}
\renewcommand{\thetable}{\theenumi}
%\renewcommand{\theequation}{\theenumi}

%\begin{abstract}
%%\boldmath
%In this letter, an algorithm for evaluating the exact analytical bit error rate  (BER)  for the piecewise linear (PL) combiner for  multiple relays is presented. Previous results were available only for upto three relays. The algorithm is unique in the sense that  the actual mathematical expressions, that are prohibitively large, need not be explicitly obtained. The diversity gain due to multiple relays is shown through plots of the analytical BER, well supported by simulations. 
%
%\end{abstract}
% IEEEtran.cls defaults to using nonbold math in the Abstract.
% This preserves the distinction between vectors and scalars. However,
% if the journal you are submitting to favors bold math in the abstract,
% then you can use LaTeX's standard command \boldmath at the very start
% of the abstract to achieve this. Many IEEE journals frown on math
% in the abstract anyway.

% Note that keywords are not normally used for peerreview papers.
%\begin{IEEEkeywords}
%Cooperative diversity, decode and forward, piecewise linear
%\end{IEEEkeywords}



% For peer review papers, you can put extra information on the cover
% page as needed:
% \ifCLASSOPTIONpeerreview
% \begin{center} \bfseries EDICS Category: 3-BBND \end{center}
% \fi
%
% For peerreview papers, this IEEEtran command inserts a page break and
% creates the second title. It will be ignored for other modes.
%\IEEEpeerreviewmaketitle




  \item
  Cards with numbers 2 to 101 are placed in a box. A card is selected at random.Find the probability that the card has
\begin{enumerate}[label=(\roman*)]
	\item an even number 
	\item a square number
\end{enumerate}
\solution
%\begin{table}[H]
	\centering
\begin{tabular}{|c|c|c|}
\hline
Random variable &Value &Definition\\ \hline
\multirow{3}{*}{X} &0 &Slips of Rs 1\\
&1 &Slips of Rs 5\\
&2 &Slips of Rs 13\\ \hline
\multirow{2}{*}{Y} &0 &Box A\\
&1 &Box B\\\hline
\end{tabular}
\caption{}
\label{tab:Distribution}
\end{table}
See \tabref{tab:Distribution}.
\begin{align}
p_{Y}\brak{k}= \begin{cases} 
      \frac{1}{3} & {k=0} \\
      \frac{2}{3 }& {k=1} 
   \end{cases}
   \\
p_{Y|X}\brak{0|0} = \frac{19}{25}\, 
p_{Y|X}\brak{0|1} = \frac{6}{25}\,
p_{Y|X}\brak{1|0} = \frac{45}{50}\,
p_{Y|X}\brak{1|2} = \frac{5}{50}
\end{align}
The desired probability is the probability that a slip drawn at random is marked other than Rs 1,
\begin{align}
&=1-p_X\brak{0}\\
&= p_X(1) + p_X(2)
\end{align}
Using Bayes theorem,
\begin{align}
&= p_Y\brak{0} \times \pr{Y=0 | X=1} + p_Y\brak{1} \times \pr{Y=1|X=2}\\
&=\frac{1}{3} \times \frac{6}{25} + \frac{2}{3} \times \frac{5}{50}\\
&=\frac{11}{75}
\end{align}

\newpage

%\tableofcontents

\bigskip

\renewcommand{\thefigure}{\theenumi}
\renewcommand{\thetable}{\theenumi}
%\renewcommand{\theequation}{\theenumi}

%\begin{abstract}
%%\boldmath
%In this letter, an algorithm for evaluating the exact analytical bit error rate  (BER)  for the piecewise linear (PL) combiner for  multiple relays is presented. Previous results were available only for upto three relays. The algorithm is unique in the sense that  the actual mathematical expressions, that are prohibitively large, need not be explicitly obtained. The diversity gain due to multiple relays is shown through plots of the analytical BER, well supported by simulations. 
%
%\end{abstract}
% IEEEtran.cls defaults to using nonbold math in the Abstract.
% This preserves the distinction between vectors and scalars. However,
% if the journal you are submitting to favors bold math in the abstract,
% then you can use LaTeX's standard command \boldmath at the very start
% of the abstract to achieve this. Many IEEE journals frown on math
% in the abstract anyway.

% Note that keywords are not normally used for peerreview papers.
%\begin{IEEEkeywords}
%Cooperative diversity, decode and forward, piecewise linear
%\end{IEEEkeywords}



% For peer review papers, you can put extra information on the cover
% page as needed:
% \ifCLASSOPTIONpeerreview
% \begin{center} \bfseries EDICS Category: 3-BBND \end{center}
% \fi
%
% For peerreview papers, this IEEEtran command inserts a page break and
% creates the second title. It will be ignored for other modes.
%\IEEEpeerreviewmaketitle




\item
The king, queen and jack of clubs are removed from a deck of 52 playing cards and then well shuffled. Now one card is drawn at random from the remaining cards.  Determine the probability that the card is
\begin{enumerate}[label=(\roman*)]
\item a club
\item 10 of hearts
\end{enumerate}
\solution
%\begin{table}[H]
	\centering
\begin{tabular}{|c|c|c|}
\hline
Random variable &Value &Definition\\ \hline
\multirow{3}{*}{X} &0 &Slips of Rs 1\\
&1 &Slips of Rs 5\\
&2 &Slips of Rs 13\\ \hline
\multirow{2}{*}{Y} &0 &Box A\\
&1 &Box B\\\hline
\end{tabular}
\caption{}
\label{tab:Distribution}
\end{table}
See \tabref{tab:Distribution}.
\begin{align}
p_{Y}\brak{k}= \begin{cases} 
      \frac{1}{3} & {k=0} \\
      \frac{2}{3 }& {k=1} 
   \end{cases}
   \\
p_{Y|X}\brak{0|0} = \frac{19}{25}\, 
p_{Y|X}\brak{0|1} = \frac{6}{25}\,
p_{Y|X}\brak{1|0} = \frac{45}{50}\,
p_{Y|X}\brak{1|2} = \frac{5}{50}
\end{align}
The desired probability is the probability that a slip drawn at random is marked other than Rs 1,
\begin{align}
&=1-p_X\brak{0}\\
&= p_X(1) + p_X(2)
\end{align}
Using Bayes theorem,
\begin{align}
&= p_Y\brak{0} \times \pr{Y=0 | X=1} + p_Y\brak{1} \times \pr{Y=1|X=2}\\
&=\frac{1}{3} \times \frac{6}{25} + \frac{2}{3} \times \frac{5}{50}\\
&=\frac{11}{75}
\end{align}

\newpage

%\tableofcontents

\bigskip

\renewcommand{\thefigure}{\theenumi}
\renewcommand{\thetable}{\theenumi}
%\renewcommand{\theequation}{\theenumi}

%\begin{abstract}
%%\boldmath
%In this letter, an algorithm for evaluating the exact analytical bit error rate  (BER)  for the piecewise linear (PL) combiner for  multiple relays is presented. Previous results were available only for upto three relays. The algorithm is unique in the sense that  the actual mathematical expressions, that are prohibitively large, need not be explicitly obtained. The diversity gain due to multiple relays is shown through plots of the analytical BER, well supported by simulations. 
%
%\end{abstract}
% IEEEtran.cls defaults to using nonbold math in the Abstract.
% This preserves the distinction between vectors and scalars. However,
% if the journal you are submitting to favors bold math in the abstract,
% then you can use LaTeX's standard command \boldmath at the very start
% of the abstract to achieve this. Many IEEE journals frown on math
% in the abstract anyway.

% Note that keywords are not normally used for peerreview papers.
%\begin{IEEEkeywords}
%Cooperative diversity, decode and forward, piecewise linear
%\end{IEEEkeywords}



% For peer review papers, you can put extra information on the cover
% page as needed:
% \ifCLASSOPTIONpeerreview
% \begin{center} \bfseries EDICS Category: 3-BBND \end{center}
% \fi
%
% For peerreview papers, this IEEEtran command inserts a page break and
% creates the second title. It will be ignored for other modes.
%\IEEEpeerreviewmaketitle




\item A team of medical students doing their internship have to assist during surgeries
at a city hospital. The probabilities of surgeries rated as very complex, complex,
routine, simple or very simple are respectively, 0.15, 0.20, 0.31, 0.26, .08. Find
the probabilities that a particular surgery will be rated
\begin{enumerate}
	\item complex or very complex;
	\item neither very complex nor very simple;
	\item routine or complex
	\item routine or simple
\end{enumerate}
\solution
%\begin{table}[H]
	\centering
\begin{tabular}{|c|c|c|}
\hline
Random variable &Value &Definition\\ \hline
\multirow{3}{*}{X} &0 &Slips of Rs 1\\
&1 &Slips of Rs 5\\
&2 &Slips of Rs 13\\ \hline
\multirow{2}{*}{Y} &0 &Box A\\
&1 &Box B\\\hline
\end{tabular}
\caption{}
\label{tab:Distribution}
\end{table}
See \tabref{tab:Distribution}.
\begin{align}
p_{Y}\brak{k}= \begin{cases} 
      \frac{1}{3} & {k=0} \\
      \frac{2}{3 }& {k=1} 
   \end{cases}
   \\
p_{Y|X}\brak{0|0} = \frac{19}{25}\, 
p_{Y|X}\brak{0|1} = \frac{6}{25}\,
p_{Y|X}\brak{1|0} = \frac{45}{50}\,
p_{Y|X}\brak{1|2} = \frac{5}{50}
\end{align}
The desired probability is the probability that a slip drawn at random is marked other than Rs 1,
\begin{align}
&=1-p_X\brak{0}\\
&= p_X(1) + p_X(2)
\end{align}
Using Bayes theorem,
\begin{align}
&= p_Y\brak{0} \times \pr{Y=0 | X=1} + p_Y\brak{1} \times \pr{Y=1|X=2}\\
&=\frac{1}{3} \times \frac{6}{25} + \frac{2}{3} \times \frac{5}{50}\\
&=\frac{11}{75}
\end{align}

\newpage

%\tableofcontents

\bigskip

\renewcommand{\thefigure}{\theenumi}
\renewcommand{\thetable}{\theenumi}
%\renewcommand{\theequation}{\theenumi}

%\begin{abstract}
%%\boldmath
%In this letter, an algorithm for evaluating the exact analytical bit error rate  (BER)  for the piecewise linear (PL) combiner for  multiple relays is presented. Previous results were available only for upto three relays. The algorithm is unique in the sense that  the actual mathematical expressions, that are prohibitively large, need not be explicitly obtained. The diversity gain due to multiple relays is shown through plots of the analytical BER, well supported by simulations. 
%
%\end{abstract}
% IEEEtran.cls defaults to using nonbold math in the Abstract.
% This preserves the distinction between vectors and scalars. However,
% if the journal you are submitting to favors bold math in the abstract,
% then you can use LaTeX's standard command \boldmath at the very start
% of the abstract to achieve this. Many IEEE journals frown on math
% in the abstract anyway.

% Note that keywords are not normally used for peerreview papers.
%\begin{IEEEkeywords}
%Cooperative diversity, decode and forward, piecewise linear
%\end{IEEEkeywords}



% For peer review papers, you can put extra information on the cover
% page as needed:
% \ifCLASSOPTIONpeerreview
% \begin{center} \bfseries EDICS Category: 3-BBND \end{center}
% \fi
%
% For peerreview papers, this IEEEtran command inserts a page break and
% creates the second title. It will be ignored for other modes.
%\IEEEpeerreviewmaketitle




\item A card is selected from a pack of 52 cards.
\begin{enumerate}[label=(\alph*)]
    \item How many points are there in the sample space?
    \item Calculate the probability that the card is an ace of spades.
    \item Calculate the probability that the card is (i) an ace and (ii) black card.
\end{enumerate}
\solution
%Let $X$ be an bernoulli rv defined as in \tabref{tab:exemplar/11/16/3/26}.  Then, 
\begin{equation}
    p =
        \frac{4}{11} 
\end{equation}
\begin{table}[H]
	\centering
	\input{exemplar/11/16/3/26/tables/Table2.tex}
	\caption{}
        \label{tab:exemplar/11/16/3/26}
\end{table}

\item The probability that a non leap year selected at random will contain 53 sundays.
\\
\solution
%\begin{table}[H]
	\centering
\begin{tabular}{|c|c|c|}
\hline
Random variable &Value &Definition\\ \hline
\multirow{3}{*}{X} &0 &Slips of Rs 1\\
&1 &Slips of Rs 5\\
&2 &Slips of Rs 13\\ \hline
\multirow{2}{*}{Y} &0 &Box A\\
&1 &Box B\\\hline
\end{tabular}
\caption{}
\label{tab:Distribution}
\end{table}
See \tabref{tab:Distribution}.
\begin{align}
p_{Y}\brak{k}= \begin{cases} 
      \frac{1}{3} & {k=0} \\
      \frac{2}{3 }& {k=1} 
   \end{cases}
   \\
p_{Y|X}\brak{0|0} = \frac{19}{25}\, 
p_{Y|X}\brak{0|1} = \frac{6}{25}\,
p_{Y|X}\brak{1|0} = \frac{45}{50}\,
p_{Y|X}\brak{1|2} = \frac{5}{50}
\end{align}
The desired probability is the probability that a slip drawn at random is marked other than Rs 1,
\begin{align}
&=1-p_X\brak{0}\\
&= p_X(1) + p_X(2)
\end{align}
Using Bayes theorem,
\begin{align}
&= p_Y\brak{0} \times \pr{Y=0 | X=1} + p_Y\brak{1} \times \pr{Y=1|X=2}\\
&=\frac{1}{3} \times \frac{6}{25} + \frac{2}{3} \times \frac{5}{50}\\
&=\frac{11}{75}
\end{align}

\newpage

%\tableofcontents

\bigskip

\renewcommand{\thefigure}{\theenumi}
\renewcommand{\thetable}{\theenumi}
%\renewcommand{\theequation}{\theenumi}

%\begin{abstract}
%%\boldmath
%In this letter, an algorithm for evaluating the exact analytical bit error rate  (BER)  for the piecewise linear (PL) combiner for  multiple relays is presented. Previous results were available only for upto three relays. The algorithm is unique in the sense that  the actual mathematical expressions, that are prohibitively large, need not be explicitly obtained. The diversity gain due to multiple relays is shown through plots of the analytical BER, well supported by simulations. 
%
%\end{abstract}
% IEEEtran.cls defaults to using nonbold math in the Abstract.
% This preserves the distinction between vectors and scalars. However,
% if the journal you are submitting to favors bold math in the abstract,
% then you can use LaTeX's standard command \boldmath at the very start
% of the abstract to achieve this. Many IEEE journals frown on math
% in the abstract anyway.

% Note that keywords are not normally used for peerreview papers.
%\begin{IEEEkeywords}
%Cooperative diversity, decode and forward, piecewise linear
%\end{IEEEkeywords}



% For peer review papers, you can put extra information on the cover
% page as needed:
% \ifCLASSOPTIONpeerreview
% \begin{center} \bfseries EDICS Category: 3-BBND \end{center}
% \fi
%
% For peerreview papers, this IEEEtran command inserts a page break and
% creates the second title. It will be ignored for other modes.
%\IEEEpeerreviewmaketitle




\item One of the four persons John, Rita, Aslam or Gurpreet will be promoted next
month. Consequently the sample space consists of four elementary outcomes
S = {John promoted, Rita promoted, Aslam promoted, Gurpreet promoted}
You are told that the chances of John’s promotion is same as that of Gurpreet,
Rita’s chances of promotion are twice as likely as Johns. Aslam’s chances are
four times that of John.
\begin{enumerate}
	\item Determine
	\begin{enumerate}
		\item P (John promoted)
		\item P (Rita promoted)
		\item P (Aslam promoted)
		\item P (Gurpreet promoted)
	\end{enumerate}
	\item If A = {John promoted or Gurpreet promoted}, find P (A).
\end{enumerate}
\solution
%\begin{table}[H]
	\centering
\begin{tabular}{|c|c|c|}
\hline
Random variable &Value &Definition\\ \hline
\multirow{3}{*}{X} &0 &Slips of Rs 1\\
&1 &Slips of Rs 5\\
&2 &Slips of Rs 13\\ \hline
\multirow{2}{*}{Y} &0 &Box A\\
&1 &Box B\\\hline
\end{tabular}
\caption{}
\label{tab:Distribution}
\end{table}
See \tabref{tab:Distribution}.
\begin{align}
p_{Y}\brak{k}= \begin{cases} 
      \frac{1}{3} & {k=0} \\
      \frac{2}{3 }& {k=1} 
   \end{cases}
   \\
p_{Y|X}\brak{0|0} = \frac{19}{25}\, 
p_{Y|X}\brak{0|1} = \frac{6}{25}\,
p_{Y|X}\brak{1|0} = \frac{45}{50}\,
p_{Y|X}\brak{1|2} = \frac{5}{50}
\end{align}
The desired probability is the probability that a slip drawn at random is marked other than Rs 1,
\begin{align}
&=1-p_X\brak{0}\\
&= p_X(1) + p_X(2)
\end{align}
Using Bayes theorem,
\begin{align}
&= p_Y\brak{0} \times \pr{Y=0 | X=1} + p_Y\brak{1} \times \pr{Y=1|X=2}\\
&=\frac{1}{3} \times \frac{6}{25} + \frac{2}{3} \times \frac{5}{50}\\
&=\frac{11}{75}
\end{align}

\newpage

%\tableofcontents

\bigskip

\renewcommand{\thefigure}{\theenumi}
\renewcommand{\thetable}{\theenumi}
%\renewcommand{\theequation}{\theenumi}

%\begin{abstract}
%%\boldmath
%In this letter, an algorithm for evaluating the exact analytical bit error rate  (BER)  for the piecewise linear (PL) combiner for  multiple relays is presented. Previous results were available only for upto three relays. The algorithm is unique in the sense that  the actual mathematical expressions, that are prohibitively large, need not be explicitly obtained. The diversity gain due to multiple relays is shown through plots of the analytical BER, well supported by simulations. 
%
%\end{abstract}
% IEEEtran.cls defaults to using nonbold math in the Abstract.
% This preserves the distinction between vectors and scalars. However,
% if the journal you are submitting to favors bold math in the abstract,
% then you can use LaTeX's standard command \boldmath at the very start
% of the abstract to achieve this. Many IEEE journals frown on math
% in the abstract anyway.

% Note that keywords are not normally used for peerreview papers.
%\begin{IEEEkeywords}
%Cooperative diversity, decode and forward, piecewise linear
%\end{IEEEkeywords}



% For peer review papers, you can put extra information on the cover
% page as needed:
% \ifCLASSOPTIONpeerreview
% \begin{center} \bfseries EDICS Category: 3-BBND \end{center}
% \fi
%
% For peerreview papers, this IEEEtran command inserts a page break and
% creates the second title. It will be ignored for other modes.
%\IEEEpeerreviewmaketitle




\item A card is drawn from a deck of 52 cards. Find the probability of getting a king or a heart or a red card.\\
\solution
%\begin{table}[H]
	\centering
\begin{tabular}{|c|c|c|}
\hline
Random variable &Value &Definition\\ \hline
\multirow{3}{*}{X} &0 &Slips of Rs 1\\
&1 &Slips of Rs 5\\
&2 &Slips of Rs 13\\ \hline
\multirow{2}{*}{Y} &0 &Box A\\
&1 &Box B\\\hline
\end{tabular}
\caption{}
\label{tab:Distribution}
\end{table}
See \tabref{tab:Distribution}.
\begin{align}
p_{Y}\brak{k}= \begin{cases} 
      \frac{1}{3} & {k=0} \\
      \frac{2}{3 }& {k=1} 
   \end{cases}
   \\
p_{Y|X}\brak{0|0} = \frac{19}{25}\, 
p_{Y|X}\brak{0|1} = \frac{6}{25}\,
p_{Y|X}\brak{1|0} = \frac{45}{50}\,
p_{Y|X}\brak{1|2} = \frac{5}{50}
\end{align}
The desired probability is the probability that a slip drawn at random is marked other than Rs 1,
\begin{align}
&=1-p_X\brak{0}\\
&= p_X(1) + p_X(2)
\end{align}
Using Bayes theorem,
\begin{align}
&= p_Y\brak{0} \times \pr{Y=0 | X=1} + p_Y\brak{1} \times \pr{Y=1|X=2}\\
&=\frac{1}{3} \times \frac{6}{25} + \frac{2}{3} \times \frac{5}{50}\\
&=\frac{11}{75}
\end{align}

\newpage

%\tableofcontents

\bigskip

\renewcommand{\thefigure}{\theenumi}
\renewcommand{\thetable}{\theenumi}
%\renewcommand{\theequation}{\theenumi}

%\begin{abstract}
%%\boldmath
%In this letter, an algorithm for evaluating the exact analytical bit error rate  (BER)  for the piecewise linear (PL) combiner for  multiple relays is presented. Previous results were available only for upto three relays. The algorithm is unique in the sense that  the actual mathematical expressions, that are prohibitively large, need not be explicitly obtained. The diversity gain due to multiple relays is shown through plots of the analytical BER, well supported by simulations. 
%
%\end{abstract}
% IEEEtran.cls defaults to using nonbold math in the Abstract.
% This preserves the distinction between vectors and scalars. However,
% if the journal you are submitting to favors bold math in the abstract,
% then you can use LaTeX's standard command \boldmath at the very start
% of the abstract to achieve this. Many IEEE journals frown on math
% in the abstract anyway.

% Note that keywords are not normally used for peerreview papers.
%\begin{IEEEkeywords}
%Cooperative diversity, decode and forward, piecewise linear
%\end{IEEEkeywords}



% For peer review papers, you can put extra information on the cover
% page as needed:
% \ifCLASSOPTIONpeerreview
% \begin{center} \bfseries EDICS Category: 3-BBND \end{center}
% \fi
%
% For peerreview papers, this IEEEtran command inserts a page break and
% creates the second title. It will be ignored for other modes.
%\IEEEpeerreviewmaketitle




\item The probability that a student will pass his examination is 0.73, the probability of
the student getting a compartment is 0.13, and the probability that the student will
either pass or get compartment is 0.96. State True or False.\\
\solution
%\begin{table}[H]
	\centering
\begin{tabular}{|c|c|c|}
\hline
Random variable &Value &Definition\\ \hline
\multirow{3}{*}{X} &0 &Slips of Rs 1\\
&1 &Slips of Rs 5\\
&2 &Slips of Rs 13\\ \hline
\multirow{2}{*}{Y} &0 &Box A\\
&1 &Box B\\\hline
\end{tabular}
\caption{}
\label{tab:Distribution}
\end{table}
See \tabref{tab:Distribution}.
\begin{align}
p_{Y}\brak{k}= \begin{cases} 
      \frac{1}{3} & {k=0} \\
      \frac{2}{3 }& {k=1} 
   \end{cases}
   \\
p_{Y|X}\brak{0|0} = \frac{19}{25}\, 
p_{Y|X}\brak{0|1} = \frac{6}{25}\,
p_{Y|X}\brak{1|0} = \frac{45}{50}\,
p_{Y|X}\brak{1|2} = \frac{5}{50}
\end{align}
The desired probability is the probability that a slip drawn at random is marked other than Rs 1,
\begin{align}
&=1-p_X\brak{0}\\
&= p_X(1) + p_X(2)
\end{align}
Using Bayes theorem,
\begin{align}
&= p_Y\brak{0} \times \pr{Y=0 | X=1} + p_Y\brak{1} \times \pr{Y=1|X=2}\\
&=\frac{1}{3} \times \frac{6}{25} + \frac{2}{3} \times \frac{5}{50}\\
&=\frac{11}{75}
\end{align}

\newpage

%\tableofcontents

\bigskip

\renewcommand{\thefigure}{\theenumi}
\renewcommand{\thetable}{\theenumi}
%\renewcommand{\theequation}{\theenumi}

%\begin{abstract}
%%\boldmath
%In this letter, an algorithm for evaluating the exact analytical bit error rate  (BER)  for the piecewise linear (PL) combiner for  multiple relays is presented. Previous results were available only for upto three relays. The algorithm is unique in the sense that  the actual mathematical expressions, that are prohibitively large, need not be explicitly obtained. The diversity gain due to multiple relays is shown through plots of the analytical BER, well supported by simulations. 
%
%\end{abstract}
% IEEEtran.cls defaults to using nonbold math in the Abstract.
% This preserves the distinction between vectors and scalars. However,
% if the journal you are submitting to favors bold math in the abstract,
% then you can use LaTeX's standard command \boldmath at the very start
% of the abstract to achieve this. Many IEEE journals frown on math
% in the abstract anyway.

% Note that keywords are not normally used for peerreview papers.
%\begin{IEEEkeywords}
%Cooperative diversity, decode and forward, piecewise linear
%\end{IEEEkeywords}



% For peer review papers, you can put extra information on the cover
% page as needed:
% \ifCLASSOPTIONpeerreview
% \begin{center} \bfseries EDICS Category: 3-BBND \end{center}
% \fi
%
% For peerreview papers, this IEEEtran command inserts a page break and
% creates the second title. It will be ignored for other modes.
%\IEEEpeerreviewmaketitle




\item A card is selected from a pack of 52 cards\\
\begin{enumerate}[label=(\alph*)]
\item How many points are there in the sample space?
\item Calculate the probability that the cards is an ace of spades.
\item Calculate the probability that the card is (i) an ace (ii)black card.\\
\end{enumerate}
%\input{ncert/11/16/3/4_1/Prob_4.tex}
\item In a non-leap year, the probability of having 53 tuesdays or 53 wednesdays is\\
\solution
%A non-leap year has a total of 365 days, and a week has 7 days.\\
So it can be expressed as 
\begin{align}
365\text{days} &=52\times 7+1 \text{day}
\end{align}
$\implies$ 52 tuesdays or wednesdays\\
Random variable X denotes the days of a week
\begin{align}
p_X\brak{k}&=\frac{1}{7}; \quad \brak{1<k<7}
\end{align}
So the probability of extra day being tuesday or wednesday is
\begin{align}
p_X\brak{3}+p_X\brak{4}&=\frac{1}{7}+\frac{1}{7}=\frac{2}{7}
\end{align}



\item There are 1000 sealed envelopes in a box, 10 of them contain a cash prize of
Rs 100 each, 100 of them contain a cash prize of Rs 50 each and 200 of them
contain a cash prize of Rs 10 each and rest do not contain any cash prize. If they
are well shuffled and an envelope is picked up out, what is the probability that it
contains no cash prize?\\
\solution
%\begin{table}[H]
	\centering
\begin{tabular}{|c|c|c|}
\hline
Random variable &Value &Definition\\ \hline
\multirow{3}{*}{X} &0 &Slips of Rs 1\\
&1 &Slips of Rs 5\\
&2 &Slips of Rs 13\\ \hline
\multirow{2}{*}{Y} &0 &Box A\\
&1 &Box B\\\hline
\end{tabular}
\caption{}
\label{tab:Distribution}
\end{table}
See \tabref{tab:Distribution}.
\begin{align}
p_{Y}\brak{k}= \begin{cases} 
      \frac{1}{3} & {k=0} \\
      \frac{2}{3 }& {k=1} 
   \end{cases}
   \\
p_{Y|X}\brak{0|0} = \frac{19}{25}\, 
p_{Y|X}\brak{0|1} = \frac{6}{25}\,
p_{Y|X}\brak{1|0} = \frac{45}{50}\,
p_{Y|X}\brak{1|2} = \frac{5}{50}
\end{align}
The desired probability is the probability that a slip drawn at random is marked other than Rs 1,
\begin{align}
&=1-p_X\brak{0}\\
&= p_X(1) + p_X(2)
\end{align}
Using Bayes theorem,
\begin{align}
&= p_Y\brak{0} \times \pr{Y=0 | X=1} + p_Y\brak{1} \times \pr{Y=1|X=2}\\
&=\frac{1}{3} \times \frac{6}{25} + \frac{2}{3} \times \frac{5}{50}\\
&=\frac{11}{75}
\end{align}

\newpage

%\tableofcontents

\bigskip

\renewcommand{\thefigure}{\theenumi}
\renewcommand{\thetable}{\theenumi}
%\renewcommand{\theequation}{\theenumi}

%\begin{abstract}
%%\boldmath
%In this letter, an algorithm for evaluating the exact analytical bit error rate  (BER)  for the piecewise linear (PL) combiner for  multiple relays is presented. Previous results were available only for upto three relays. The algorithm is unique in the sense that  the actual mathematical expressions, that are prohibitively large, need not be explicitly obtained. The diversity gain due to multiple relays is shown through plots of the analytical BER, well supported by simulations. 
%
%\end{abstract}
% IEEEtran.cls defaults to using nonbold math in the Abstract.
% This preserves the distinction between vectors and scalars. However,
% if the journal you are submitting to favors bold math in the abstract,
% then you can use LaTeX's standard command \boldmath at the very start
% of the abstract to achieve this. Many IEEE journals frown on math
% in the abstract anyway.

% Note that keywords are not normally used for peerreview papers.
%\begin{IEEEkeywords}
%Cooperative diversity, decode and forward, piecewise linear
%\end{IEEEkeywords}



% For peer review papers, you can put extra information on the cover
% page as needed:
% \ifCLASSOPTIONpeerreview
% \begin{center} \bfseries EDICS Category: 3-BBND \end{center}
% \fi
%
% For peerreview papers, this IEEEtran command inserts a page break and
% creates the second title. It will be ignored for other modes.
%\IEEEpeerreviewmaketitle




\item 
A die is thrown and a card is selected at random from a deck of 52 playing cards. The probability of getting an even number on the die and a spade card.\\
\solution
%\begin{table}[H]
	\centering
\begin{tabular}{|c|c|c|}
\hline
Random variable &Value &Definition\\ \hline
\multirow{3}{*}{X} &0 &Slips of Rs 1\\
&1 &Slips of Rs 5\\
&2 &Slips of Rs 13\\ \hline
\multirow{2}{*}{Y} &0 &Box A\\
&1 &Box B\\\hline
\end{tabular}
\caption{}
\label{tab:Distribution}
\end{table}
See \tabref{tab:Distribution}.
\begin{align}
p_{Y}\brak{k}= \begin{cases} 
      \frac{1}{3} & {k=0} \\
      \frac{2}{3 }& {k=1} 
   \end{cases}
   \\
p_{Y|X}\brak{0|0} = \frac{19}{25}\, 
p_{Y|X}\brak{0|1} = \frac{6}{25}\,
p_{Y|X}\brak{1|0} = \frac{45}{50}\,
p_{Y|X}\brak{1|2} = \frac{5}{50}
\end{align}
The desired probability is the probability that a slip drawn at random is marked other than Rs 1,
\begin{align}
&=1-p_X\brak{0}\\
&= p_X(1) + p_X(2)
\end{align}
Using Bayes theorem,
\begin{align}
&= p_Y\brak{0} \times \pr{Y=0 | X=1} + p_Y\brak{1} \times \pr{Y=1|X=2}\\
&=\frac{1}{3} \times \frac{6}{25} + \frac{2}{3} \times \frac{5}{50}\\
&=\frac{11}{75}
\end{align}

\newpage

%\tableofcontents

\bigskip

\renewcommand{\thefigure}{\theenumi}
\renewcommand{\thetable}{\theenumi}
%\renewcommand{\theequation}{\theenumi}

%\begin{abstract}
%%\boldmath
%In this letter, an algorithm for evaluating the exact analytical bit error rate  (BER)  for the piecewise linear (PL) combiner for  multiple relays is presented. Previous results were available only for upto three relays. The algorithm is unique in the sense that  the actual mathematical expressions, that are prohibitively large, need not be explicitly obtained. The diversity gain due to multiple relays is shown through plots of the analytical BER, well supported by simulations. 
%
%\end{abstract}
% IEEEtran.cls defaults to using nonbold math in the Abstract.
% This preserves the distinction between vectors and scalars. However,
% if the journal you are submitting to favors bold math in the abstract,
% then you can use LaTeX's standard command \boldmath at the very start
% of the abstract to achieve this. Many IEEE journals frown on math
% in the abstract anyway.

% Note that keywords are not normally used for peerreview papers.
%\begin{IEEEkeywords}
%Cooperative diversity, decode and forward, piecewise linear
%\end{IEEEkeywords}



% For peer review papers, you can put extra information on the cover
% page as needed:
% \ifCLASSOPTIONpeerreview
% \begin{center} \bfseries EDICS Category: 3-BBND \end{center}
% \fi
%
% For peerreview papers, this IEEEtran command inserts a page break and
% creates the second title. It will be ignored for other modes.
%\IEEEpeerreviewmaketitle




\item
If 4-digit numbers greater than 5,000 are randomly formed from the digits 0, 1, 3, 5, and 7, what is the probability of forming a number divisible by 5 when:
\begin{enumerate}
    \item The digits are repeated?
    \item The repetition of digits is not allowed?
\end{enumerate}
\solution
%\begin{table}[H]
	\centering
\begin{tabular}{|c|c|c|}
\hline
Random variable &Value &Definition\\ \hline
\multirow{3}{*}{X} &0 &Slips of Rs 1\\
&1 &Slips of Rs 5\\
&2 &Slips of Rs 13\\ \hline
\multirow{2}{*}{Y} &0 &Box A\\
&1 &Box B\\\hline
\end{tabular}
\caption{}
\label{tab:Distribution}
\end{table}
See \tabref{tab:Distribution}.
\begin{align}
p_{Y}\brak{k}= \begin{cases} 
      \frac{1}{3} & {k=0} \\
      \frac{2}{3 }& {k=1} 
   \end{cases}
   \\
p_{Y|X}\brak{0|0} = \frac{19}{25}\, 
p_{Y|X}\brak{0|1} = \frac{6}{25}\,
p_{Y|X}\brak{1|0} = \frac{45}{50}\,
p_{Y|X}\brak{1|2} = \frac{5}{50}
\end{align}
The desired probability is the probability that a slip drawn at random is marked other than Rs 1,
\begin{align}
&=1-p_X\brak{0}\\
&= p_X(1) + p_X(2)
\end{align}
Using Bayes theorem,
\begin{align}
&= p_Y\brak{0} \times \pr{Y=0 | X=1} + p_Y\brak{1} \times \pr{Y=1|X=2}\\
&=\frac{1}{3} \times \frac{6}{25} + \frac{2}{3} \times \frac{5}{50}\\
&=\frac{11}{75}
\end{align}

\newpage

%\tableofcontents

\bigskip

\renewcommand{\thefigure}{\theenumi}
\renewcommand{\thetable}{\theenumi}
%\renewcommand{\theequation}{\theenumi}

%\begin{abstract}
%%\boldmath
%In this letter, an algorithm for evaluating the exact analytical bit error rate  (BER)  for the piecewise linear (PL) combiner for  multiple relays is presented. Previous results were available only for upto three relays. The algorithm is unique in the sense that  the actual mathematical expressions, that are prohibitively large, need not be explicitly obtained. The diversity gain due to multiple relays is shown through plots of the analytical BER, well supported by simulations. 
%
%\end{abstract}
% IEEEtran.cls defaults to using nonbold math in the Abstract.
% This preserves the distinction between vectors and scalars. However,
% if the journal you are submitting to favors bold math in the abstract,
% then you can use LaTeX's standard command \boldmath at the very start
% of the abstract to achieve this. Many IEEE journals frown on math
% in the abstract anyway.

% Note that keywords are not normally used for peerreview papers.
%\begin{IEEEkeywords}
%Cooperative diversity, decode and forward, piecewise linear
%\end{IEEEkeywords}



% For peer review papers, you can put extra information on the cover
% page as needed:
% \ifCLASSOPTIONpeerreview
% \begin{center} \bfseries EDICS Category: 3-BBND \end{center}
% \fi
%
% For peerreview papers, this IEEEtran command inserts a page break and
% creates the second title. It will be ignored for other modes.
%\IEEEpeerreviewmaketitle




\item Consider the probability space $\brak{\Omega, \mathcal{G}, P}$ where $\Omega = [0,2]$ and $\mathcal{G} = \cbrak{\phi, \Omega, [0,1], (1,2]}$. Let $X$ and $Y$ be two functions on $\Omega$ defined as
\begin{align*}
    X(\omega) = 
    \begin{cases}
        1 & \text{if }\omega \in [0, 1]\\
        2 & \text{if }\omega \in (1, 2]
    \end{cases}
\end{align*}
and
\begin{align*}
    Y(\omega) = 
    \begin{cases}
        2 & \text{if }\omega \in [0, 1.5]\\
        3 & \text{if }\omega \in (1.5, 2].
    \end{cases}
\end{align*}
Then which one of the following statements is true?
\begin{enumerate}
    \item [(A)] $X$ is a random variable with respect to $\mathcal{G}$, but $Y$ is not a random variable with respect to $\mathcal{G}$.
    \item [(B)] $Y$ is a random variable with respect to $\mathcal{G}$, but $X$ is not a random variable with respect to $\mathcal{G}$.
    \item [(C)] Neither $X$ nor $Y$ is a random variable with respect to $\mathcal{G}$.
    \item [(D)] Both $X$ and $Y$ are random variables with respect to $\mathcal{G}$.
\end{enumerate} \hfill (GATE ST 2023)\\
\solution
%\begin{table}[H]
	\centering
\begin{tabular}{|c|c|c|}
\hline
Random variable &Value &Definition\\ \hline
\multirow{3}{*}{X} &0 &Slips of Rs 1\\
&1 &Slips of Rs 5\\
&2 &Slips of Rs 13\\ \hline
\multirow{2}{*}{Y} &0 &Box A\\
&1 &Box B\\\hline
\end{tabular}
\caption{}
\label{tab:Distribution}
\end{table}
See \tabref{tab:Distribution}.
\begin{align}
p_{Y}\brak{k}= \begin{cases} 
      \frac{1}{3} & {k=0} \\
      \frac{2}{3 }& {k=1} 
   \end{cases}
   \\
p_{Y|X}\brak{0|0} = \frac{19}{25}\, 
p_{Y|X}\brak{0|1} = \frac{6}{25}\,
p_{Y|X}\brak{1|0} = \frac{45}{50}\,
p_{Y|X}\brak{1|2} = \frac{5}{50}
\end{align}
The desired probability is the probability that a slip drawn at random is marked other than Rs 1,
\begin{align}
&=1-p_X\brak{0}\\
&= p_X(1) + p_X(2)
\end{align}
Using Bayes theorem,
\begin{align}
&= p_Y\brak{0} \times \pr{Y=0 | X=1} + p_Y\brak{1} \times \pr{Y=1|X=2}\\
&=\frac{1}{3} \times \frac{6}{25} + \frac{2}{3} \times \frac{5}{50}\\
&=\frac{11}{75}
\end{align}

\newpage

%\tableofcontents

\bigskip

\renewcommand{\thefigure}{\theenumi}
\renewcommand{\thetable}{\theenumi}
%\renewcommand{\theequation}{\theenumi}

%\begin{abstract}
%%\boldmath
%In this letter, an algorithm for evaluating the exact analytical bit error rate  (BER)  for the piecewise linear (PL) combiner for  multiple relays is presented. Previous results were available only for upto three relays. The algorithm is unique in the sense that  the actual mathematical expressions, that are prohibitively large, need not be explicitly obtained. The diversity gain due to multiple relays is shown through plots of the analytical BER, well supported by simulations. 
%
%\end{abstract}
% IEEEtran.cls defaults to using nonbold math in the Abstract.
% This preserves the distinction between vectors and scalars. However,
% if the journal you are submitting to favors bold math in the abstract,
% then you can use LaTeX's standard command \boldmath at the very start
% of the abstract to achieve this. Many IEEE journals frown on math
% in the abstract anyway.

% Note that keywords are not normally used for peerreview papers.
%\begin{IEEEkeywords}
%Cooperative diversity, decode and forward, piecewise linear
%\end{IEEEkeywords}



% For peer review papers, you can put extra information on the cover
% page as needed:
% \ifCLASSOPTIONpeerreview
% \begin{center} \bfseries EDICS Category: 3-BBND \end{center}
% \fi
%
% For peerreview papers, this IEEEtran command inserts a page break and
% creates the second title. It will be ignored for other modes.
%\IEEEpeerreviewmaketitle




	\item  A die is loaded in such a way that each odd number is twice as likely to occur as
each even number. Find $P(G)$, where $G$ is the event that a number greater than
3 occurs on a single roll of the die.
\\
\solution
		%\begin{table}[H]
	\centering
\begin{tabular}{|c|c|c|}
\hline
Random variable &Value &Definition\\ \hline
\multirow{3}{*}{X} &0 &Slips of Rs 1\\
&1 &Slips of Rs 5\\
&2 &Slips of Rs 13\\ \hline
\multirow{2}{*}{Y} &0 &Box A\\
&1 &Box B\\\hline
\end{tabular}
\caption{}
\label{tab:Distribution}
\end{table}
See \tabref{tab:Distribution}.
\begin{align}
p_{Y}\brak{k}= \begin{cases} 
      \frac{1}{3} & {k=0} \\
      \frac{2}{3 }& {k=1} 
   \end{cases}
   \\
p_{Y|X}\brak{0|0} = \frac{19}{25}\, 
p_{Y|X}\brak{0|1} = \frac{6}{25}\,
p_{Y|X}\brak{1|0} = \frac{45}{50}\,
p_{Y|X}\brak{1|2} = \frac{5}{50}
\end{align}
The desired probability is the probability that a slip drawn at random is marked other than Rs 1,
\begin{align}
&=1-p_X\brak{0}\\
&= p_X(1) + p_X(2)
\end{align}
Using Bayes theorem,
\begin{align}
&= p_Y\brak{0} \times \pr{Y=0 | X=1} + p_Y\brak{1} \times \pr{Y=1|X=2}\\
&=\frac{1}{3} \times \frac{6}{25} + \frac{2}{3} \times \frac{5}{50}\\
&=\frac{11}{75}
\end{align}

\newpage

%\tableofcontents

\bigskip

\renewcommand{\thefigure}{\theenumi}
\renewcommand{\thetable}{\theenumi}
%\renewcommand{\theequation}{\theenumi}

%\begin{abstract}
%%\boldmath
%In this letter, an algorithm for evaluating the exact analytical bit error rate  (BER)  for the piecewise linear (PL) combiner for  multiple relays is presented. Previous results were available only for upto three relays. The algorithm is unique in the sense that  the actual mathematical expressions, that are prohibitively large, need not be explicitly obtained. The diversity gain due to multiple relays is shown through plots of the analytical BER, well supported by simulations. 
%
%\end{abstract}
% IEEEtran.cls defaults to using nonbold math in the Abstract.
% This preserves the distinction between vectors and scalars. However,
% if the journal you are submitting to favors bold math in the abstract,
% then you can use LaTeX's standard command \boldmath at the very start
% of the abstract to achieve this. Many IEEE journals frown on math
% in the abstract anyway.

% Note that keywords are not normally used for peerreview papers.
%\begin{IEEEkeywords}
%Cooperative diversity, decode and forward, piecewise linear
%\end{IEEEkeywords}



% For peer review papers, you can put extra information on the cover
% page as needed:
% \ifCLASSOPTIONpeerreview
% \begin{center} \bfseries EDICS Category: 3-BBND \end{center}
% \fi
%
% For peerreview papers, this IEEEtran command inserts a page break and
% creates the second title. It will be ignored for other modes.
%\IEEEpeerreviewmaketitle




	\item All the jacks, queens and kings are removed from a deck of 52 playing cards. The remaining cards are well shuffled and then one card is drawn at random. Giving ace a value 1 similar value for other cards, find the probability that the card has a value 
		\begin{enumerate}
			\item 7
			\item greater than 7
			\item less than 7
		\end{enumerate}
		%Number of cards left after removing all jacks, queens and kings 
\begin{align}
N	= 52 - 4\times 3
	= 40
\end{align}
%\begin{table}[H]
%\def\arraystretch{1.2}
%\begin{tabular}{|c|c|c|}
%\hline
%	\textbf{Parameter} &\textbf{Value} &\textbf{Description}\\ \hline
%	$X$ &1-10 &Represents the value of the card picked \\ \hline
%\end{tabular}
%\end{table}
Let $1 \le X \le 10$ be the value of the card picked.  Then,
\begin{align}
	p_X(k) &= \Pr(X=k)\ \forall\ 1 \leq k \leq 10\\
	&= \frac{4\times 1}{40}\\
	&= \frac{1}{10}\\
	\therefore p_X(k) &= 
	\begin{cases}
		\frac{1}{10} & 1 \leq k \leq 10\\
		0 & \text{otherwise}
	\end{cases}
\end{align}
and
\begin{align}
	F_{X}(k) &= \sum_{m=0}^{k}p_{X}(m) \quad 1 \leq k \leq 10\\
	&= \frac{k}{10}\\
	\therefore F_{X}(k) &= 
	\begin{cases}
		0 & k \leq 0\\
		\frac{k}{10} & 1\leq k \leq 10\\
		1 & k > 10 
	\end{cases}
\end{align}
\begin{enumerate}
	\item Probability that card has value equal to 7 is
		\begin{align}
			 p_{X}(7)
			= \frac{1}{10}
		\end{align}
	\item Probability that card has value greater than 7 is
		\begin{align}
			1 - F_X(7)
			&= 1 - \frac{7}{10}
			\\
			&= \frac{3}{10}
		\end{align}
	\item Probability that card has value less than 7 is
		\begin{align}
			 F_{X}(6)
			=\frac{6}{10}
		\end{align}
\end{enumerate}

  \item A Lot consists of 48 mobile phones of which 42 are good, 3 have only minor defects and 3 have major defects.Varnika will buy a phone if it is good but the trader will only buy a mobile if it has no major defects. One phone is selected at random from the lot. What is the probability that it is
\begin{enumerate}
	\item acceptable to Varnika?
            \item acceptable to the trader?
\end{enumerate}
\solution
	%\begin{table}[H]
	\centering
\begin{tabular}{|c|c|c|}
\hline
Random variable &Value &Definition\\ \hline
\multirow{3}{*}{X} &0 &Slips of Rs 1\\
&1 &Slips of Rs 5\\
&2 &Slips of Rs 13\\ \hline
\multirow{2}{*}{Y} &0 &Box A\\
&1 &Box B\\\hline
\end{tabular}
\caption{}
\label{tab:Distribution}
\end{table}
See \tabref{tab:Distribution}.
\begin{align}
p_{Y}\brak{k}= \begin{cases} 
      \frac{1}{3} & {k=0} \\
      \frac{2}{3 }& {k=1} 
   \end{cases}
   \\
p_{Y|X}\brak{0|0} = \frac{19}{25}\, 
p_{Y|X}\brak{0|1} = \frac{6}{25}\,
p_{Y|X}\brak{1|0} = \frac{45}{50}\,
p_{Y|X}\brak{1|2} = \frac{5}{50}
\end{align}
The desired probability is the probability that a slip drawn at random is marked other than Rs 1,
\begin{align}
&=1-p_X\brak{0}\\
&= p_X(1) + p_X(2)
\end{align}
Using Bayes theorem,
\begin{align}
&= p_Y\brak{0} \times \pr{Y=0 | X=1} + p_Y\brak{1} \times \pr{Y=1|X=2}\\
&=\frac{1}{3} \times \frac{6}{25} + \frac{2}{3} \times \frac{5}{50}\\
&=\frac{11}{75}
\end{align}

\newpage

%\tableofcontents

\bigskip

\renewcommand{\thefigure}{\theenumi}
\renewcommand{\thetable}{\theenumi}
%\renewcommand{\theequation}{\theenumi}

%\begin{abstract}
%%\boldmath
%In this letter, an algorithm for evaluating the exact analytical bit error rate  (BER)  for the piecewise linear (PL) combiner for  multiple relays is presented. Previous results were available only for upto three relays. The algorithm is unique in the sense that  the actual mathematical expressions, that are prohibitively large, need not be explicitly obtained. The diversity gain due to multiple relays is shown through plots of the analytical BER, well supported by simulations. 
%
%\end{abstract}
% IEEEtran.cls defaults to using nonbold math in the Abstract.
% This preserves the distinction between vectors and scalars. However,
% if the journal you are submitting to favors bold math in the abstract,
% then you can use LaTeX's standard command \boldmath at the very start
% of the abstract to achieve this. Many IEEE journals frown on math
% in the abstract anyway.

% Note that keywords are not normally used for peerreview papers.
%\begin{IEEEkeywords}
%Cooperative diversity, decode and forward, piecewise linear
%\end{IEEEkeywords}



% For peer review papers, you can put extra information on the cover
% page as needed:
% \ifCLASSOPTIONpeerreview
% \begin{center} \bfseries EDICS Category: 3-BBND \end{center}
% \fi
%
% For peerreview papers, this IEEEtran command inserts a page break and
% creates the second title. It will be ignored for other modes.
%\IEEEpeerreviewmaketitle




 \item A student says that if you throw a die, it will show up 1 or not 1. Therefore, the probability of getting 1 and the probability of getting 'not 1' each is equal to $\frac{1}{2}$. Is this correct? Give reasons.\\
 \solution
        %\begin{table}[H]
	\centering
\begin{tabular}{|c|c|c|}
\hline
Random variable &Value &Definition\\ \hline
\multirow{3}{*}{X} &0 &Slips of Rs 1\\
&1 &Slips of Rs 5\\
&2 &Slips of Rs 13\\ \hline
\multirow{2}{*}{Y} &0 &Box A\\
&1 &Box B\\\hline
\end{tabular}
\caption{}
\label{tab:Distribution}
\end{table}
See \tabref{tab:Distribution}.
\begin{align}
p_{Y}\brak{k}= \begin{cases} 
      \frac{1}{3} & {k=0} \\
      \frac{2}{3 }& {k=1} 
   \end{cases}
   \\
p_{Y|X}\brak{0|0} = \frac{19}{25}\, 
p_{Y|X}\brak{0|1} = \frac{6}{25}\,
p_{Y|X}\brak{1|0} = \frac{45}{50}\,
p_{Y|X}\brak{1|2} = \frac{5}{50}
\end{align}
The desired probability is the probability that a slip drawn at random is marked other than Rs 1,
\begin{align}
&=1-p_X\brak{0}\\
&= p_X(1) + p_X(2)
\end{align}
Using Bayes theorem,
\begin{align}
&= p_Y\brak{0} \times \pr{Y=0 | X=1} + p_Y\brak{1} \times \pr{Y=1|X=2}\\
&=\frac{1}{3} \times \frac{6}{25} + \frac{2}{3} \times \frac{5}{50}\\
&=\frac{11}{75}
\end{align}

\newpage

%\tableofcontents

\bigskip

\renewcommand{\thefigure}{\theenumi}
\renewcommand{\thetable}{\theenumi}
%\renewcommand{\theequation}{\theenumi}

%\begin{abstract}
%%\boldmath
%In this letter, an algorithm for evaluating the exact analytical bit error rate  (BER)  for the piecewise linear (PL) combiner for  multiple relays is presented. Previous results were available only for upto three relays. The algorithm is unique in the sense that  the actual mathematical expressions, that are prohibitively large, need not be explicitly obtained. The diversity gain due to multiple relays is shown through plots of the analytical BER, well supported by simulations. 
%
%\end{abstract}
% IEEEtran.cls defaults to using nonbold math in the Abstract.
% This preserves the distinction between vectors and scalars. However,
% if the journal you are submitting to favors bold math in the abstract,
% then you can use LaTeX's standard command \boldmath at the very start
% of the abstract to achieve this. Many IEEE journals frown on math
% in the abstract anyway.

% Note that keywords are not normally used for peerreview papers.
%\begin{IEEEkeywords}
%Cooperative diversity, decode and forward, piecewise linear
%\end{IEEEkeywords}



% For peer review papers, you can put extra information on the cover
% page as needed:
% \ifCLASSOPTIONpeerreview
% \begin{center} \bfseries EDICS Category: 3-BBND \end{center}
% \fi
%
% For peerreview papers, this IEEEtran command inserts a page break and
% creates the second title. It will be ignored for other modes.
%\IEEEpeerreviewmaketitle




   \item Four candidates A, B, C, D have ap-
plied for the assignment to coach a school cricket
team. If A is twice as likely to be selected as B, and
B and C are given about the same chance of being
selected, while C is twice as likely to be selected
as D, what are the probabilities that
\begin{enumerate}
\item C will be selected?
\item A will not be selected?
\end{enumerate}
	%\begin{table}[H]
	\centering
\begin{tabular}{|c|c|c|}
\hline
Random variable &Value &Definition\\ \hline
\multirow{3}{*}{X} &0 &Slips of Rs 1\\
&1 &Slips of Rs 5\\
&2 &Slips of Rs 13\\ \hline
\multirow{2}{*}{Y} &0 &Box A\\
&1 &Box B\\\hline
\end{tabular}
\caption{}
\label{tab:Distribution}
\end{table}
See \tabref{tab:Distribution}.
\begin{align}
p_{Y}\brak{k}= \begin{cases} 
      \frac{1}{3} & {k=0} \\
      \frac{2}{3 }& {k=1} 
   \end{cases}
   \\
p_{Y|X}\brak{0|0} = \frac{19}{25}\, 
p_{Y|X}\brak{0|1} = \frac{6}{25}\,
p_{Y|X}\brak{1|0} = \frac{45}{50}\,
p_{Y|X}\brak{1|2} = \frac{5}{50}
\end{align}
The desired probability is the probability that a slip drawn at random is marked other than Rs 1,
\begin{align}
&=1-p_X\brak{0}\\
&= p_X(1) + p_X(2)
\end{align}
Using Bayes theorem,
\begin{align}
&= p_Y\brak{0} \times \pr{Y=0 | X=1} + p_Y\brak{1} \times \pr{Y=1|X=2}\\
&=\frac{1}{3} \times \frac{6}{25} + \frac{2}{3} \times \frac{5}{50}\\
&=\frac{11}{75}
\end{align}

\newpage

%\tableofcontents

\bigskip

\renewcommand{\thefigure}{\theenumi}
\renewcommand{\thetable}{\theenumi}
%\renewcommand{\theequation}{\theenumi}

%\begin{abstract}
%%\boldmath
%In this letter, an algorithm for evaluating the exact analytical bit error rate  (BER)  for the piecewise linear (PL) combiner for  multiple relays is presented. Previous results were available only for upto three relays. The algorithm is unique in the sense that  the actual mathematical expressions, that are prohibitively large, need not be explicitly obtained. The diversity gain due to multiple relays is shown through plots of the analytical BER, well supported by simulations. 
%
%\end{abstract}
% IEEEtran.cls defaults to using nonbold math in the Abstract.
% This preserves the distinction between vectors and scalars. However,
% if the journal you are submitting to favors bold math in the abstract,
% then you can use LaTeX's standard command \boldmath at the very start
% of the abstract to achieve this. Many IEEE journals frown on math
% in the abstract anyway.

% Note that keywords are not normally used for peerreview papers.
%\begin{IEEEkeywords}
%Cooperative diversity, decode and forward, piecewise linear
%\end{IEEEkeywords}



% For peer review papers, you can put extra information on the cover
% page as needed:
% \ifCLASSOPTIONpeerreview
% \begin{center} \bfseries EDICS Category: 3-BBND \end{center}
% \fi
%
% For peerreview papers, this IEEEtran command inserts a page break and
% creates the second title. It will be ignored for other modes.
%\IEEEpeerreviewmaketitle




 \item A bag contain 24 balls of which $x$ balls are red, $2x$ are white and $3x$ are blue. A ball is selected at random, What is the probability that it is
\begin{enumerate}[label=\alph*)]
\item not red ?
\item white ?
\end{enumerate}
%\begin{table}[H]
	\centering
\begin{tabular}{|c|c|c|}
\hline
Random variable &Value &Definition\\ \hline
\multirow{3}{*}{X} &0 &Slips of Rs 1\\
&1 &Slips of Rs 5\\
&2 &Slips of Rs 13\\ \hline
\multirow{2}{*}{Y} &0 &Box A\\
&1 &Box B\\\hline
\end{tabular}
\caption{}
\label{tab:Distribution}
\end{table}
See \tabref{tab:Distribution}.
\begin{align}
p_{Y}\brak{k}= \begin{cases} 
      \frac{1}{3} & {k=0} \\
      \frac{2}{3 }& {k=1} 
   \end{cases}
   \\
p_{Y|X}\brak{0|0} = \frac{19}{25}\, 
p_{Y|X}\brak{0|1} = \frac{6}{25}\,
p_{Y|X}\brak{1|0} = \frac{45}{50}\,
p_{Y|X}\brak{1|2} = \frac{5}{50}
\end{align}
The desired probability is the probability that a slip drawn at random is marked other than Rs 1,
\begin{align}
&=1-p_X\brak{0}\\
&= p_X(1) + p_X(2)
\end{align}
Using Bayes theorem,
\begin{align}
&= p_Y\brak{0} \times \pr{Y=0 | X=1} + p_Y\brak{1} \times \pr{Y=1|X=2}\\
&=\frac{1}{3} \times \frac{6}{25} + \frac{2}{3} \times \frac{5}{50}\\
&=\frac{11}{75}
\end{align}

\newpage

%\tableofcontents

\bigskip

\renewcommand{\thefigure}{\theenumi}
\renewcommand{\thetable}{\theenumi}
%\renewcommand{\theequation}{\theenumi}

%\begin{abstract}
%%\boldmath
%In this letter, an algorithm for evaluating the exact analytical bit error rate  (BER)  for the piecewise linear (PL) combiner for  multiple relays is presented. Previous results were available only for upto three relays. The algorithm is unique in the sense that  the actual mathematical expressions, that are prohibitively large, need not be explicitly obtained. The diversity gain due to multiple relays is shown through plots of the analytical BER, well supported by simulations. 
%
%\end{abstract}
% IEEEtran.cls defaults to using nonbold math in the Abstract.
% This preserves the distinction between vectors and scalars. However,
% if the journal you are submitting to favors bold math in the abstract,
% then you can use LaTeX's standard command \boldmath at the very start
% of the abstract to achieve this. Many IEEE journals frown on math
% in the abstract anyway.

% Note that keywords are not normally used for peerreview papers.
%\begin{IEEEkeywords}
%Cooperative diversity, decode and forward, piecewise linear
%\end{IEEEkeywords}



% For peer review papers, you can put extra information on the cover
% page as needed:
% \ifCLASSOPTIONpeerreview
% \begin{center} \bfseries EDICS Category: 3-BBND \end{center}
% \fi
%
% For peerreview papers, this IEEEtran command inserts a page break and
% creates the second title. It will be ignored for other modes.
%\IEEEpeerreviewmaketitle




If the letters of the word ASSASSINATION are arranged at random. Find the Probability that
\begin{enumerate}[label=(\alph*)]
\item Four $S's$ come consecutively in the word
\item Two  $I's$ and two $N's$ come together
\item All $A's$ are not coming together
\item No two $A's$ are coming together
\end{enumerate}
%\begin{table}[H]
	\centering
\begin{tabular}{|c|c|c|}
\hline
Random variable &Value &Definition\\ \hline
\multirow{3}{*}{X} &0 &Slips of Rs 1\\
&1 &Slips of Rs 5\\
&2 &Slips of Rs 13\\ \hline
\multirow{2}{*}{Y} &0 &Box A\\
&1 &Box B\\\hline
\end{tabular}
\caption{}
\label{tab:Distribution}
\end{table}
See \tabref{tab:Distribution}.
\begin{align}
p_{Y}\brak{k}= \begin{cases} 
      \frac{1}{3} & {k=0} \\
      \frac{2}{3 }& {k=1} 
   \end{cases}
   \\
p_{Y|X}\brak{0|0} = \frac{19}{25}\, 
p_{Y|X}\brak{0|1} = \frac{6}{25}\,
p_{Y|X}\brak{1|0} = \frac{45}{50}\,
p_{Y|X}\brak{1|2} = \frac{5}{50}
\end{align}
The desired probability is the probability that a slip drawn at random is marked other than Rs 1,
\begin{align}
&=1-p_X\brak{0}\\
&= p_X(1) + p_X(2)
\end{align}
Using Bayes theorem,
\begin{align}
&= p_Y\brak{0} \times \pr{Y=0 | X=1} + p_Y\brak{1} \times \pr{Y=1|X=2}\\
&=\frac{1}{3} \times \frac{6}{25} + \frac{2}{3} \times \frac{5}{50}\\
&=\frac{11}{75}
\end{align}

\newpage

%\tableofcontents

\bigskip

\renewcommand{\thefigure}{\theenumi}
\renewcommand{\thetable}{\theenumi}
%\renewcommand{\theequation}{\theenumi}

%\begin{abstract}
%%\boldmath
%In this letter, an algorithm for evaluating the exact analytical bit error rate  (BER)  for the piecewise linear (PL) combiner for  multiple relays is presented. Previous results were available only for upto three relays. The algorithm is unique in the sense that  the actual mathematical expressions, that are prohibitively large, need not be explicitly obtained. The diversity gain due to multiple relays is shown through plots of the analytical BER, well supported by simulations. 
%
%\end{abstract}
% IEEEtran.cls defaults to using nonbold math in the Abstract.
% This preserves the distinction between vectors and scalars. However,
% if the journal you are submitting to favors bold math in the abstract,
% then you can use LaTeX's standard command \boldmath at the very start
% of the abstract to achieve this. Many IEEE journals frown on math
% in the abstract anyway.

% Note that keywords are not normally used for peerreview papers.
%\begin{IEEEkeywords}
%Cooperative diversity, decode and forward, piecewise linear
%\end{IEEEkeywords}



% For peer review papers, you can put extra information on the cover
% page as needed:
% \ifCLASSOPTIONpeerreview
% \begin{center} \bfseries EDICS Category: 3-BBND \end{center}
% \fi
%
% For peerreview papers, this IEEEtran command inserts a page break and
% creates the second title. It will be ignored for other modes.
%\IEEEpeerreviewmaketitle




	\item One urn contains two black balls (labelled B1 and B2) and one white ball. A
	second urn contains one black ball and two white balls (labelled W1 and W2).
	Suppose the following experiment is performed. One of the two urns is chosen
	at random. Next a ball is randomly chosen from the urn. Then a second ball is
	chosen at random from the same urn without replacing the first ball.
	
	\begin{enumerate}
	\item What is the probability that two black balls are chosen?
	
	\item What is the probability that two balls of opposite colour are chosen?
	\end{enumerate}
	\solution
	%\begin{align}
    \label{eq:12.13.6.18.1}
	\because	\pr{A|B} &> \pr{A},\
\frac{\pr{AB}}{\pr{B}} > \pr{A}
\\
    \label{eq:12.13.6.18.2}
	\implies \pr{AB} &> \pr{A}\pr{B}
	\\
	\text{or, } \frac{\pr{AB}}{\pr{A}} &=\pr{B|A} > \pr{A}
\end{align}

\end{enumerate}

	\item A card is selected from a pack of 52 cards.
 \begin{enumerate}[label=(\alph*)] 
                 \item How many points are there in the sample space?
                 \item Calculate the probability that the card is an ace of spades.
                 \item Calculate the probability that the card is (i) an ace and (ii) black card.
 \end{enumerate}
\solution
		%\begin{table}[H]
	\centering
\begin{tabular}{|c|c|c|}
\hline
Random variable &Value &Definition\\ \hline
\multirow{3}{*}{X} &0 &Slips of Rs 1\\
&1 &Slips of Rs 5\\
&2 &Slips of Rs 13\\ \hline
\multirow{2}{*}{Y} &0 &Box A\\
&1 &Box B\\\hline
\end{tabular}
\caption{}
\label{tab:Distribution}
\end{table}
See \tabref{tab:Distribution}.
\begin{align}
p_{Y}\brak{k}= \begin{cases} 
      \frac{1}{3} & {k=0} \\
      \frac{2}{3 }& {k=1} 
   \end{cases}
   \\
p_{Y|X}\brak{0|0} = \frac{19}{25}\, 
p_{Y|X}\brak{0|1} = \frac{6}{25}\,
p_{Y|X}\brak{1|0} = \frac{45}{50}\,
p_{Y|X}\brak{1|2} = \frac{5}{50}
\end{align}
The desired probability is the probability that a slip drawn at random is marked other than Rs 1,
\begin{align}
&=1-p_X\brak{0}\\
&= p_X(1) + p_X(2)
\end{align}
Using Bayes theorem,
\begin{align}
&= p_Y\brak{0} \times \pr{Y=0 | X=1} + p_Y\brak{1} \times \pr{Y=1|X=2}\\
&=\frac{1}{3} \times \frac{6}{25} + \frac{2}{3} \times \frac{5}{50}\\
&=\frac{11}{75}
\end{align}

\newpage

%\tableofcontents

\bigskip

\renewcommand{\thefigure}{\theenumi}
\renewcommand{\thetable}{\theenumi}
%\renewcommand{\theequation}{\theenumi}

%\begin{abstract}
%%\boldmath
%In this letter, an algorithm for evaluating the exact analytical bit error rate  (BER)  for the piecewise linear (PL) combiner for  multiple relays is presented. Previous results were available only for upto three relays. The algorithm is unique in the sense that  the actual mathematical expressions, that are prohibitively large, need not be explicitly obtained. The diversity gain due to multiple relays is shown through plots of the analytical BER, well supported by simulations. 
%
%\end{abstract}
% IEEEtran.cls defaults to using nonbold math in the Abstract.
% This preserves the distinction between vectors and scalars. However,
% if the journal you are submitting to favors bold math in the abstract,
% then you can use LaTeX's standard command \boldmath at the very start
% of the abstract to achieve this. Many IEEE journals frown on math
% in the abstract anyway.

% Note that keywords are not normally used for peerreview papers.
%\begin{IEEEkeywords}
%Cooperative diversity, decode and forward, piecewise linear
%\end{IEEEkeywords}



% For peer review papers, you can put extra information on the cover
% page as needed:
% \ifCLASSOPTIONpeerreview
% \begin{center} \bfseries EDICS Category: 3-BBND \end{center}
% \fi
%
% For peerreview papers, this IEEEtran command inserts a page break and
% creates the second title. It will be ignored for other modes.
%\IEEEpeerreviewmaketitle




\item Four cards are drawn from a well-shuffled deck of 52 cards. What is the probability of obtaining 3 diamonds and one spade.
\\
\solution
		%\begin{enumerate}[label=\thesection.\arabic*,ref=\thesection.\theenumi]
	\item One card is drawn from a well-shuffled deck of 52 cards. Find the probability of getting
\begin{enumerate}
\item A king of red colour 
\item A face card 
\item A red face card
\item The jack of hearts
\item A spade
\item The queen of diamonds

\end{enumerate}
\solution
		%\begin{table}[H]
	\centering
\begin{tabular}{|c|c|c|}
\hline
Random variable &Value &Definition\\ \hline
\multirow{3}{*}{X} &0 &Slips of Rs 1\\
&1 &Slips of Rs 5\\
&2 &Slips of Rs 13\\ \hline
\multirow{2}{*}{Y} &0 &Box A\\
&1 &Box B\\\hline
\end{tabular}
\caption{}
\label{tab:Distribution}
\end{table}
See \tabref{tab:Distribution}.
\begin{align}
p_{Y}\brak{k}= \begin{cases} 
      \frac{1}{3} & {k=0} \\
      \frac{2}{3 }& {k=1} 
   \end{cases}
   \\
p_{Y|X}\brak{0|0} = \frac{19}{25}\, 
p_{Y|X}\brak{0|1} = \frac{6}{25}\,
p_{Y|X}\brak{1|0} = \frac{45}{50}\,
p_{Y|X}\brak{1|2} = \frac{5}{50}
\end{align}
The desired probability is the probability that a slip drawn at random is marked other than Rs 1,
\begin{align}
&=1-p_X\brak{0}\\
&= p_X(1) + p_X(2)
\end{align}
Using Bayes theorem,
\begin{align}
&= p_Y\brak{0} \times \pr{Y=0 | X=1} + p_Y\brak{1} \times \pr{Y=1|X=2}\\
&=\frac{1}{3} \times \frac{6}{25} + \frac{2}{3} \times \frac{5}{50}\\
&=\frac{11}{75}
\end{align}

\newpage

%\tableofcontents

\bigskip

\renewcommand{\thefigure}{\theenumi}
\renewcommand{\thetable}{\theenumi}
%\renewcommand{\theequation}{\theenumi}

%\begin{abstract}
%%\boldmath
%In this letter, an algorithm for evaluating the exact analytical bit error rate  (BER)  for the piecewise linear (PL) combiner for  multiple relays is presented. Previous results were available only for upto three relays. The algorithm is unique in the sense that  the actual mathematical expressions, that are prohibitively large, need not be explicitly obtained. The diversity gain due to multiple relays is shown through plots of the analytical BER, well supported by simulations. 
%
%\end{abstract}
% IEEEtran.cls defaults to using nonbold math in the Abstract.
% This preserves the distinction between vectors and scalars. However,
% if the journal you are submitting to favors bold math in the abstract,
% then you can use LaTeX's standard command \boldmath at the very start
% of the abstract to achieve this. Many IEEE journals frown on math
% in the abstract anyway.

% Note that keywords are not normally used for peerreview papers.
%\begin{IEEEkeywords}
%Cooperative diversity, decode and forward, piecewise linear
%\end{IEEEkeywords}



% For peer review papers, you can put extra information on the cover
% page as needed:
% \ifCLASSOPTIONpeerreview
% \begin{center} \bfseries EDICS Category: 3-BBND \end{center}
% \fi
%
% For peerreview papers, this IEEEtran command inserts a page break and
% creates the second title. It will be ignored for other modes.
%\IEEEpeerreviewmaketitle




	\item Five cards—the ten, jack, queen, king and ace of diamonds, are well-shuffled with their face downwards. One card is then picked up at random.
\begin{enumerate}
\item
What is the probability that the card is the queen? 
\item
If the queen is drawn and put aside, what is the probability that the second card picked up is (a) an ace? (b) a queen?\\
\end{enumerate}
\solution
		%\begin{enumerate}[label=\thesection.\arabic*,ref=\thesection.\theenumi]
	\item One card is drawn from a well-shuffled deck of 52 cards. Find the probability of getting
\begin{enumerate}
\item A king of red colour 
\item A face card 
\item A red face card
\item The jack of hearts
\item A spade
\item The queen of diamonds

\end{enumerate}
\solution
		%\input{ncert/10/15/1/14/main.tex}
	\item Five cards—the ten, jack, queen, king and ace of diamonds, are well-shuffled with their face downwards. One card is then picked up at random.
\begin{enumerate}
\item
What is the probability that the card is the queen? 
\item
If the queen is drawn and put aside, what is the probability that the second card picked up is (a) an ace? (b) a queen?\\
\end{enumerate}
\solution
		%\input{ncert/10/15/1/15/defs.tex}
	\item A bag contains $5$ red balls and some blue balls. If the probability of drawing a blue ball is double that if a red ball, determine the number of blue balls in the bag. 
		\\
\solution
		%\input{ncert/10/15/2/3/defs.tex}
	\item A card is selected from a pack of 52 cards.
 \begin{enumerate}[label=(\alph*)] 
                 \item How many points are there in the sample space?
                 \item Calculate the probability that the card is an ace of spades.
                 \item Calculate the probability that the card is (i) an ace and (ii) black card.
 \end{enumerate}
\solution
		%\input{ncert/11/16/3/4/main.tex}
\item Four cards are drawn from a well-shuffled deck of 52 cards. What is the probability of obtaining 3 diamonds and one spade.
\\
\solution
		%\input{ncert/11/16/4/2/defs.tex}
\item In a certain lottery 10,000 tickets are sold and ten equal prizes are awarded. What is the probability of not getting a prize if you buy (a) one ticket (b) two tickets (c) 10 tickets ?	
\\
\solution
		%\input{ncert/11/16/4/4/defs.tex}
		%
\item 
Out of 100 students, two sections of 40 and 60 are formed. If you and your friend are among the 100 students, what is the probability that
\begin{enumerate}
\item you both enter the same section?
\item you both enter the different sections?
\end{enumerate}
\solution
		%\input{ncert/11/16/4/5/defs.tex}
	\item 
The number lock of a suitcase has 4 wheels each labelled with ten digits i.e. from 0 to 9.The lock opens with a sequence of four digits with no repeats.What is the probability of a person getting the right sequence to open the suitcase.
\\
\solution
		%\input{ncert/11/16/4/10/defs.tex}
		%
\item 
Two cards are drawn at random and without replacement from a pack of 52 playing cards. Find the probability that both the cards are black.
\\
\solution
		%\input{ncert/12/13/2/2/defs.tex}
		\item A box of oranges is inspected by examining three randomly selected oranges drawn without replacement. If all the three oranges are good, the box is approved for sale, otherwise, it is rejected. Find the probability that a box containing 15 oranges out of which 12 are good and 3 are bad ones will be approved for sale.
		\label{ncert/12/13/2/3/defs.tex}
		\item Two balls are drawn at random with replacement from a box containing 10 black and 8 red balls. Find the probability that
		\label{ncert/12/13/2/12}
\begin{enumerate}
\item both balls are red.
\item first ball is black and second is red.
\item one of them is black and other is red.
\end{enumerate}

\item In a hostel, 60\% of the students read Hindi newspaper, 40\% read English newspaper and 20\% read both Hindi and English newspapers. A student is selected at random.
		\label{ncert/12/13/2/15}
\begin{enumerate}
\item Find the probability that she reads neither Hindi nor English newspapers.
\item If she reads Hindi newspaper, find the probability that she reads English newspaper.
\item If she reads English newspaper, find the probability that she reads Hindi newspaper.\\
\end{enumerate}
\item The probability of obtaining an even prime number on each die, when a pair of dice is rolled is 
\begin{enumerate}
    \item $0$ 
    
    \item $\frac{1}{3}$ 
    
    \item $\frac{1}{12}$ 
    
    \item $\frac{1}{36}$ 
\end{enumerate}
\solution
		%\input{ncert/12/13/2/17/defs.tex}
	\item A bag contains 4 red and 4 black balls, another bag contains 2 red and 6 black balls. One of the two bags is selected at random and a ball is drawn from the bag which is found to be red. Find the probability that the ball is drawn from the first bag.
\\
\solution
		%\input{ncert/12/13/3/2/main.tex}
  \item
  Cards with numbers 2 to 101 are placed in a box. A card is selected at random.Find the probability that the card has
\begin{enumerate}[label=(\roman*)]
	\item an even number 
	\item a square number
\end{enumerate}
\solution
%\input{exemplar/10/13/3/32/main.tex}
\item
The king, queen and jack of clubs are removed from a deck of 52 playing cards and then well shuffled. Now one card is drawn at random from the remaining cards.  Determine the probability that the card is
\begin{enumerate}[label=(\roman*)]
\item a club
\item 10 of hearts
\end{enumerate}
\solution
%\input{exemplar/10/13/3/29/main.tex}
\item A team of medical students doing their internship have to assist during surgeries
at a city hospital. The probabilities of surgeries rated as very complex, complex,
routine, simple or very simple are respectively, 0.15, 0.20, 0.31, 0.26, .08. Find
the probabilities that a particular surgery will be rated
\begin{enumerate}
	\item complex or very complex;
	\item neither very complex nor very simple;
	\item routine or complex
	\item routine or simple
\end{enumerate}
\solution
%\input{exemplar/11/16/3/8(1)/main.tex}
\item A card is selected from a pack of 52 cards.
\begin{enumerate}[label=(\alph*)]
    \item How many points are there in the sample space?
    \item Calculate the probability that the card is an ace of spades.
    \item Calculate the probability that the card is (i) an ace and (ii) black card.
\end{enumerate}
\solution
%\input{exemplar/11/16/3/4/main2.tex}
\item The probability that a non leap year selected at random will contain 53 sundays.
\\
\solution
%\input{exemplar/10/13/1/19/main.tex}
\item One of the four persons John, Rita, Aslam or Gurpreet will be promoted next
month. Consequently the sample space consists of four elementary outcomes
S = {John promoted, Rita promoted, Aslam promoted, Gurpreet promoted}
You are told that the chances of John’s promotion is same as that of Gurpreet,
Rita’s chances of promotion are twice as likely as Johns. Aslam’s chances are
four times that of John.
\begin{enumerate}
	\item Determine
	\begin{enumerate}
		\item P (John promoted)
		\item P (Rita promoted)
		\item P (Aslam promoted)
		\item P (Gurpreet promoted)
	\end{enumerate}
	\item If A = {John promoted or Gurpreet promoted}, find P (A).
\end{enumerate}
\solution
%\input{exemplar/11/16/3/10/main.tex}
\item A card is drawn from a deck of 52 cards. Find the probability of getting a king or a heart or a red card.\\
\solution
%\input{exemplar/11/16/3/15/main.tex}
\item The probability that a student will pass his examination is 0.73, the probability of
the student getting a compartment is 0.13, and the probability that the student will
either pass or get compartment is 0.96. State True or False.\\
\solution
%\input{exemplar/11/16/3/31/main.tex}
\item A card is selected from a pack of 52 cards\\
\begin{enumerate}[label=(\alph*)]
\item How many points are there in the sample space?
\item Calculate the probability that the cards is an ace of spades.
\item Calculate the probability that the card is (i) an ace (ii)black card.\\
\end{enumerate}
%\input{ncert/11/16/3/4_1/Prob_4.tex}
\item In a non-leap year, the probability of having 53 tuesdays or 53 wednesdays is\\
\solution
%\input{exemplar/11/16/3/18/main.tex}
\item There are 1000 sealed envelopes in a box, 10 of them contain a cash prize of
Rs 100 each, 100 of them contain a cash prize of Rs 50 each and 200 of them
contain a cash prize of Rs 10 each and rest do not contain any cash prize. If they
are well shuffled and an envelope is picked up out, what is the probability that it
contains no cash prize?\\
\solution
%\input{exemplar/10/13/3/34/main.tex}
\item 
A die is thrown and a card is selected at random from a deck of 52 playing cards. The probability of getting an even number on the die and a spade card.\\
\solution
%\input{exemplar/12/13/3/78/main.tex}
\item
If 4-digit numbers greater than 5,000 are randomly formed from the digits 0, 1, 3, 5, and 7, what is the probability of forming a number divisible by 5 when:
\begin{enumerate}
    \item The digits are repeated?
    \item The repetition of digits is not allowed?
\end{enumerate}
\solution
%\input{ncert/11/16/4/9/main.tex}
\item Consider the probability space $\brak{\Omega, \mathcal{G}, P}$ where $\Omega = [0,2]$ and $\mathcal{G} = \cbrak{\phi, \Omega, [0,1], (1,2]}$. Let $X$ and $Y$ be two functions on $\Omega$ defined as
\begin{align*}
    X(\omega) = 
    \begin{cases}
        1 & \text{if }\omega \in [0, 1]\\
        2 & \text{if }\omega \in (1, 2]
    \end{cases}
\end{align*}
and
\begin{align*}
    Y(\omega) = 
    \begin{cases}
        2 & \text{if }\omega \in [0, 1.5]\\
        3 & \text{if }\omega \in (1.5, 2].
    \end{cases}
\end{align*}
Then which one of the following statements is true?
\begin{enumerate}
    \item [(A)] $X$ is a random variable with respect to $\mathcal{G}$, but $Y$ is not a random variable with respect to $\mathcal{G}$.
    \item [(B)] $Y$ is a random variable with respect to $\mathcal{G}$, but $X$ is not a random variable with respect to $\mathcal{G}$.
    \item [(C)] Neither $X$ nor $Y$ is a random variable with respect to $\mathcal{G}$.
    \item [(D)] Both $X$ and $Y$ are random variables with respect to $\mathcal{G}$.
\end{enumerate} \hfill (GATE ST 2023)\\
\solution
%\input{gate/ST/2023/14/main.tex}
	\item  A die is loaded in such a way that each odd number is twice as likely to occur as
each even number. Find $P(G)$, where $G$ is the event that a number greater than
3 occurs on a single roll of the die.
\\
\solution
		%\input{exemplar/11/16/3/5/main.tex}
	\item All the jacks, queens and kings are removed from a deck of 52 playing cards. The remaining cards are well shuffled and then one card is drawn at random. Giving ace a value 1 similar value for other cards, find the probability that the card has a value 
		\begin{enumerate}
			\item 7
			\item greater than 7
			\item less than 7
		\end{enumerate}
		%\input{exemplar/10/13/3/30/main.tex}
  \item A Lot consists of 48 mobile phones of which 42 are good, 3 have only minor defects and 3 have major defects.Varnika will buy a phone if it is good but the trader will only buy a mobile if it has no major defects. One phone is selected at random from the lot. What is the probability that it is
\begin{enumerate}
	\item acceptable to Varnika?
            \item acceptable to the trader?
\end{enumerate}
\solution
	%\input{exemplar/10/13/3/40/main.tex}
 \item A student says that if you throw a die, it will show up 1 or not 1. Therefore, the probability of getting 1 and the probability of getting 'not 1' each is equal to $\frac{1}{2}$. Is this correct? Give reasons.\\
 \solution
        %\input{exemplar/10/13/2/9/main.tex}
   \item Four candidates A, B, C, D have ap-
plied for the assignment to coach a school cricket
team. If A is twice as likely to be selected as B, and
B and C are given about the same chance of being
selected, while C is twice as likely to be selected
as D, what are the probabilities that
\begin{enumerate}
\item C will be selected?
\item A will not be selected?
\end{enumerate}
	%\input{exemplar/11/16/3/9/main.tex}
 \item A bag contain 24 balls of which $x$ balls are red, $2x$ are white and $3x$ are blue. A ball is selected at random, What is the probability that it is
\begin{enumerate}[label=\alph*)]
\item not red ?
\item white ?
\end{enumerate}
%\input{exemplar/10/13/3/41/main.tex}
If the letters of the word ASSASSINATION are arranged at random. Find the Probability that
\begin{enumerate}[label=(\alph*)]
\item Four $S's$ come consecutively in the word
\item Two  $I's$ and two $N's$ come together
\item All $A's$ are not coming together
\item No two $A's$ are coming together
\end{enumerate}
%\input{exemplar/11/16/3/14/main.tex}
	\item One urn contains two black balls (labelled B1 and B2) and one white ball. A
	second urn contains one black ball and two white balls (labelled W1 and W2).
	Suppose the following experiment is performed. One of the two urns is chosen
	at random. Next a ball is randomly chosen from the urn. Then a second ball is
	chosen at random from the same urn without replacing the first ball.
	
	\begin{enumerate}
	\item What is the probability that two black balls are chosen?
	
	\item What is the probability that two balls of opposite colour are chosen?
	\end{enumerate}
	\solution
	%\input{exemplar/11/16/3/12/main1.tex}
\end{enumerate}

	\item A bag contains $5$ red balls and some blue balls. If the probability of drawing a blue ball is double that if a red ball, determine the number of blue balls in the bag. 
		\\
\solution
		%\begin{enumerate}[label=\thesection.\arabic*,ref=\thesection.\theenumi]
	\item One card is drawn from a well-shuffled deck of 52 cards. Find the probability of getting
\begin{enumerate}
\item A king of red colour 
\item A face card 
\item A red face card
\item The jack of hearts
\item A spade
\item The queen of diamonds

\end{enumerate}
\solution
		%\input{ncert/10/15/1/14/main.tex}
	\item Five cards—the ten, jack, queen, king and ace of diamonds, are well-shuffled with their face downwards. One card is then picked up at random.
\begin{enumerate}
\item
What is the probability that the card is the queen? 
\item
If the queen is drawn and put aside, what is the probability that the second card picked up is (a) an ace? (b) a queen?\\
\end{enumerate}
\solution
		%\input{ncert/10/15/1/15/defs.tex}
	\item A bag contains $5$ red balls and some blue balls. If the probability of drawing a blue ball is double that if a red ball, determine the number of blue balls in the bag. 
		\\
\solution
		%\input{ncert/10/15/2/3/defs.tex}
	\item A card is selected from a pack of 52 cards.
 \begin{enumerate}[label=(\alph*)] 
                 \item How many points are there in the sample space?
                 \item Calculate the probability that the card is an ace of spades.
                 \item Calculate the probability that the card is (i) an ace and (ii) black card.
 \end{enumerate}
\solution
		%\input{ncert/11/16/3/4/main.tex}
\item Four cards are drawn from a well-shuffled deck of 52 cards. What is the probability of obtaining 3 diamonds and one spade.
\\
\solution
		%\input{ncert/11/16/4/2/defs.tex}
\item In a certain lottery 10,000 tickets are sold and ten equal prizes are awarded. What is the probability of not getting a prize if you buy (a) one ticket (b) two tickets (c) 10 tickets ?	
\\
\solution
		%\input{ncert/11/16/4/4/defs.tex}
		%
\item 
Out of 100 students, two sections of 40 and 60 are formed. If you and your friend are among the 100 students, what is the probability that
\begin{enumerate}
\item you both enter the same section?
\item you both enter the different sections?
\end{enumerate}
\solution
		%\input{ncert/11/16/4/5/defs.tex}
	\item 
The number lock of a suitcase has 4 wheels each labelled with ten digits i.e. from 0 to 9.The lock opens with a sequence of four digits with no repeats.What is the probability of a person getting the right sequence to open the suitcase.
\\
\solution
		%\input{ncert/11/16/4/10/defs.tex}
		%
\item 
Two cards are drawn at random and without replacement from a pack of 52 playing cards. Find the probability that both the cards are black.
\\
\solution
		%\input{ncert/12/13/2/2/defs.tex}
		\item A box of oranges is inspected by examining three randomly selected oranges drawn without replacement. If all the three oranges are good, the box is approved for sale, otherwise, it is rejected. Find the probability that a box containing 15 oranges out of which 12 are good and 3 are bad ones will be approved for sale.
		\label{ncert/12/13/2/3/defs.tex}
		\item Two balls are drawn at random with replacement from a box containing 10 black and 8 red balls. Find the probability that
		\label{ncert/12/13/2/12}
\begin{enumerate}
\item both balls are red.
\item first ball is black and second is red.
\item one of them is black and other is red.
\end{enumerate}

\item In a hostel, 60\% of the students read Hindi newspaper, 40\% read English newspaper and 20\% read both Hindi and English newspapers. A student is selected at random.
		\label{ncert/12/13/2/15}
\begin{enumerate}
\item Find the probability that she reads neither Hindi nor English newspapers.
\item If she reads Hindi newspaper, find the probability that she reads English newspaper.
\item If she reads English newspaper, find the probability that she reads Hindi newspaper.\\
\end{enumerate}
\item The probability of obtaining an even prime number on each die, when a pair of dice is rolled is 
\begin{enumerate}
    \item $0$ 
    
    \item $\frac{1}{3}$ 
    
    \item $\frac{1}{12}$ 
    
    \item $\frac{1}{36}$ 
\end{enumerate}
\solution
		%\input{ncert/12/13/2/17/defs.tex}
	\item A bag contains 4 red and 4 black balls, another bag contains 2 red and 6 black balls. One of the two bags is selected at random and a ball is drawn from the bag which is found to be red. Find the probability that the ball is drawn from the first bag.
\\
\solution
		%\input{ncert/12/13/3/2/main.tex}
  \item
  Cards with numbers 2 to 101 are placed in a box. A card is selected at random.Find the probability that the card has
\begin{enumerate}[label=(\roman*)]
	\item an even number 
	\item a square number
\end{enumerate}
\solution
%\input{exemplar/10/13/3/32/main.tex}
\item
The king, queen and jack of clubs are removed from a deck of 52 playing cards and then well shuffled. Now one card is drawn at random from the remaining cards.  Determine the probability that the card is
\begin{enumerate}[label=(\roman*)]
\item a club
\item 10 of hearts
\end{enumerate}
\solution
%\input{exemplar/10/13/3/29/main.tex}
\item A team of medical students doing their internship have to assist during surgeries
at a city hospital. The probabilities of surgeries rated as very complex, complex,
routine, simple or very simple are respectively, 0.15, 0.20, 0.31, 0.26, .08. Find
the probabilities that a particular surgery will be rated
\begin{enumerate}
	\item complex or very complex;
	\item neither very complex nor very simple;
	\item routine or complex
	\item routine or simple
\end{enumerate}
\solution
%\input{exemplar/11/16/3/8(1)/main.tex}
\item A card is selected from a pack of 52 cards.
\begin{enumerate}[label=(\alph*)]
    \item How many points are there in the sample space?
    \item Calculate the probability that the card is an ace of spades.
    \item Calculate the probability that the card is (i) an ace and (ii) black card.
\end{enumerate}
\solution
%\input{exemplar/11/16/3/4/main2.tex}
\item The probability that a non leap year selected at random will contain 53 sundays.
\\
\solution
%\input{exemplar/10/13/1/19/main.tex}
\item One of the four persons John, Rita, Aslam or Gurpreet will be promoted next
month. Consequently the sample space consists of four elementary outcomes
S = {John promoted, Rita promoted, Aslam promoted, Gurpreet promoted}
You are told that the chances of John’s promotion is same as that of Gurpreet,
Rita’s chances of promotion are twice as likely as Johns. Aslam’s chances are
four times that of John.
\begin{enumerate}
	\item Determine
	\begin{enumerate}
		\item P (John promoted)
		\item P (Rita promoted)
		\item P (Aslam promoted)
		\item P (Gurpreet promoted)
	\end{enumerate}
	\item If A = {John promoted or Gurpreet promoted}, find P (A).
\end{enumerate}
\solution
%\input{exemplar/11/16/3/10/main.tex}
\item A card is drawn from a deck of 52 cards. Find the probability of getting a king or a heart or a red card.\\
\solution
%\input{exemplar/11/16/3/15/main.tex}
\item The probability that a student will pass his examination is 0.73, the probability of
the student getting a compartment is 0.13, and the probability that the student will
either pass or get compartment is 0.96. State True or False.\\
\solution
%\input{exemplar/11/16/3/31/main.tex}
\item A card is selected from a pack of 52 cards\\
\begin{enumerate}[label=(\alph*)]
\item How many points are there in the sample space?
\item Calculate the probability that the cards is an ace of spades.
\item Calculate the probability that the card is (i) an ace (ii)black card.\\
\end{enumerate}
%\input{ncert/11/16/3/4_1/Prob_4.tex}
\item In a non-leap year, the probability of having 53 tuesdays or 53 wednesdays is\\
\solution
%\input{exemplar/11/16/3/18/main.tex}
\item There are 1000 sealed envelopes in a box, 10 of them contain a cash prize of
Rs 100 each, 100 of them contain a cash prize of Rs 50 each and 200 of them
contain a cash prize of Rs 10 each and rest do not contain any cash prize. If they
are well shuffled and an envelope is picked up out, what is the probability that it
contains no cash prize?\\
\solution
%\input{exemplar/10/13/3/34/main.tex}
\item 
A die is thrown and a card is selected at random from a deck of 52 playing cards. The probability of getting an even number on the die and a spade card.\\
\solution
%\input{exemplar/12/13/3/78/main.tex}
\item
If 4-digit numbers greater than 5,000 are randomly formed from the digits 0, 1, 3, 5, and 7, what is the probability of forming a number divisible by 5 when:
\begin{enumerate}
    \item The digits are repeated?
    \item The repetition of digits is not allowed?
\end{enumerate}
\solution
%\input{ncert/11/16/4/9/main.tex}
\item Consider the probability space $\brak{\Omega, \mathcal{G}, P}$ where $\Omega = [0,2]$ and $\mathcal{G} = \cbrak{\phi, \Omega, [0,1], (1,2]}$. Let $X$ and $Y$ be two functions on $\Omega$ defined as
\begin{align*}
    X(\omega) = 
    \begin{cases}
        1 & \text{if }\omega \in [0, 1]\\
        2 & \text{if }\omega \in (1, 2]
    \end{cases}
\end{align*}
and
\begin{align*}
    Y(\omega) = 
    \begin{cases}
        2 & \text{if }\omega \in [0, 1.5]\\
        3 & \text{if }\omega \in (1.5, 2].
    \end{cases}
\end{align*}
Then which one of the following statements is true?
\begin{enumerate}
    \item [(A)] $X$ is a random variable with respect to $\mathcal{G}$, but $Y$ is not a random variable with respect to $\mathcal{G}$.
    \item [(B)] $Y$ is a random variable with respect to $\mathcal{G}$, but $X$ is not a random variable with respect to $\mathcal{G}$.
    \item [(C)] Neither $X$ nor $Y$ is a random variable with respect to $\mathcal{G}$.
    \item [(D)] Both $X$ and $Y$ are random variables with respect to $\mathcal{G}$.
\end{enumerate} \hfill (GATE ST 2023)\\
\solution
%\input{gate/ST/2023/14/main.tex}
	\item  A die is loaded in such a way that each odd number is twice as likely to occur as
each even number. Find $P(G)$, where $G$ is the event that a number greater than
3 occurs on a single roll of the die.
\\
\solution
		%\input{exemplar/11/16/3/5/main.tex}
	\item All the jacks, queens and kings are removed from a deck of 52 playing cards. The remaining cards are well shuffled and then one card is drawn at random. Giving ace a value 1 similar value for other cards, find the probability that the card has a value 
		\begin{enumerate}
			\item 7
			\item greater than 7
			\item less than 7
		\end{enumerate}
		%\input{exemplar/10/13/3/30/main.tex}
  \item A Lot consists of 48 mobile phones of which 42 are good, 3 have only minor defects and 3 have major defects.Varnika will buy a phone if it is good but the trader will only buy a mobile if it has no major defects. One phone is selected at random from the lot. What is the probability that it is
\begin{enumerate}
	\item acceptable to Varnika?
            \item acceptable to the trader?
\end{enumerate}
\solution
	%\input{exemplar/10/13/3/40/main.tex}
 \item A student says that if you throw a die, it will show up 1 or not 1. Therefore, the probability of getting 1 and the probability of getting 'not 1' each is equal to $\frac{1}{2}$. Is this correct? Give reasons.\\
 \solution
        %\input{exemplar/10/13/2/9/main.tex}
   \item Four candidates A, B, C, D have ap-
plied for the assignment to coach a school cricket
team. If A is twice as likely to be selected as B, and
B and C are given about the same chance of being
selected, while C is twice as likely to be selected
as D, what are the probabilities that
\begin{enumerate}
\item C will be selected?
\item A will not be selected?
\end{enumerate}
	%\input{exemplar/11/16/3/9/main.tex}
 \item A bag contain 24 balls of which $x$ balls are red, $2x$ are white and $3x$ are blue. A ball is selected at random, What is the probability that it is
\begin{enumerate}[label=\alph*)]
\item not red ?
\item white ?
\end{enumerate}
%\input{exemplar/10/13/3/41/main.tex}
If the letters of the word ASSASSINATION are arranged at random. Find the Probability that
\begin{enumerate}[label=(\alph*)]
\item Four $S's$ come consecutively in the word
\item Two  $I's$ and two $N's$ come together
\item All $A's$ are not coming together
\item No two $A's$ are coming together
\end{enumerate}
%\input{exemplar/11/16/3/14/main.tex}
	\item One urn contains two black balls (labelled B1 and B2) and one white ball. A
	second urn contains one black ball and two white balls (labelled W1 and W2).
	Suppose the following experiment is performed. One of the two urns is chosen
	at random. Next a ball is randomly chosen from the urn. Then a second ball is
	chosen at random from the same urn without replacing the first ball.
	
	\begin{enumerate}
	\item What is the probability that two black balls are chosen?
	
	\item What is the probability that two balls of opposite colour are chosen?
	\end{enumerate}
	\solution
	%\input{exemplar/11/16/3/12/main1.tex}
\end{enumerate}

	\item A card is selected from a pack of 52 cards.
 \begin{enumerate}[label=(\alph*)] 
                 \item How many points are there in the sample space?
                 \item Calculate the probability that the card is an ace of spades.
                 \item Calculate the probability that the card is (i) an ace and (ii) black card.
 \end{enumerate}
\solution
		%\begin{table}[H]
	\centering
\begin{tabular}{|c|c|c|}
\hline
Random variable &Value &Definition\\ \hline
\multirow{3}{*}{X} &0 &Slips of Rs 1\\
&1 &Slips of Rs 5\\
&2 &Slips of Rs 13\\ \hline
\multirow{2}{*}{Y} &0 &Box A\\
&1 &Box B\\\hline
\end{tabular}
\caption{}
\label{tab:Distribution}
\end{table}
See \tabref{tab:Distribution}.
\begin{align}
p_{Y}\brak{k}= \begin{cases} 
      \frac{1}{3} & {k=0} \\
      \frac{2}{3 }& {k=1} 
   \end{cases}
   \\
p_{Y|X}\brak{0|0} = \frac{19}{25}\, 
p_{Y|X}\brak{0|1} = \frac{6}{25}\,
p_{Y|X}\brak{1|0} = \frac{45}{50}\,
p_{Y|X}\brak{1|2} = \frac{5}{50}
\end{align}
The desired probability is the probability that a slip drawn at random is marked other than Rs 1,
\begin{align}
&=1-p_X\brak{0}\\
&= p_X(1) + p_X(2)
\end{align}
Using Bayes theorem,
\begin{align}
&= p_Y\brak{0} \times \pr{Y=0 | X=1} + p_Y\brak{1} \times \pr{Y=1|X=2}\\
&=\frac{1}{3} \times \frac{6}{25} + \frac{2}{3} \times \frac{5}{50}\\
&=\frac{11}{75}
\end{align}

\newpage

%\tableofcontents

\bigskip

\renewcommand{\thefigure}{\theenumi}
\renewcommand{\thetable}{\theenumi}
%\renewcommand{\theequation}{\theenumi}

%\begin{abstract}
%%\boldmath
%In this letter, an algorithm for evaluating the exact analytical bit error rate  (BER)  for the piecewise linear (PL) combiner for  multiple relays is presented. Previous results were available only for upto three relays. The algorithm is unique in the sense that  the actual mathematical expressions, that are prohibitively large, need not be explicitly obtained. The diversity gain due to multiple relays is shown through plots of the analytical BER, well supported by simulations. 
%
%\end{abstract}
% IEEEtran.cls defaults to using nonbold math in the Abstract.
% This preserves the distinction between vectors and scalars. However,
% if the journal you are submitting to favors bold math in the abstract,
% then you can use LaTeX's standard command \boldmath at the very start
% of the abstract to achieve this. Many IEEE journals frown on math
% in the abstract anyway.

% Note that keywords are not normally used for peerreview papers.
%\begin{IEEEkeywords}
%Cooperative diversity, decode and forward, piecewise linear
%\end{IEEEkeywords}



% For peer review papers, you can put extra information on the cover
% page as needed:
% \ifCLASSOPTIONpeerreview
% \begin{center} \bfseries EDICS Category: 3-BBND \end{center}
% \fi
%
% For peerreview papers, this IEEEtran command inserts a page break and
% creates the second title. It will be ignored for other modes.
%\IEEEpeerreviewmaketitle




\item Four cards are drawn from a well-shuffled deck of 52 cards. What is the probability of obtaining 3 diamonds and one spade.
\\
\solution
		%\begin{enumerate}[label=\thesection.\arabic*,ref=\thesection.\theenumi]
	\item One card is drawn from a well-shuffled deck of 52 cards. Find the probability of getting
\begin{enumerate}
\item A king of red colour 
\item A face card 
\item A red face card
\item The jack of hearts
\item A spade
\item The queen of diamonds

\end{enumerate}
\solution
		%\input{ncert/10/15/1/14/main.tex}
	\item Five cards—the ten, jack, queen, king and ace of diamonds, are well-shuffled with their face downwards. One card is then picked up at random.
\begin{enumerate}
\item
What is the probability that the card is the queen? 
\item
If the queen is drawn and put aside, what is the probability that the second card picked up is (a) an ace? (b) a queen?\\
\end{enumerate}
\solution
		%\input{ncert/10/15/1/15/defs.tex}
	\item A bag contains $5$ red balls and some blue balls. If the probability of drawing a blue ball is double that if a red ball, determine the number of blue balls in the bag. 
		\\
\solution
		%\input{ncert/10/15/2/3/defs.tex}
	\item A card is selected from a pack of 52 cards.
 \begin{enumerate}[label=(\alph*)] 
                 \item How many points are there in the sample space?
                 \item Calculate the probability that the card is an ace of spades.
                 \item Calculate the probability that the card is (i) an ace and (ii) black card.
 \end{enumerate}
\solution
		%\input{ncert/11/16/3/4/main.tex}
\item Four cards are drawn from a well-shuffled deck of 52 cards. What is the probability of obtaining 3 diamonds and one spade.
\\
\solution
		%\input{ncert/11/16/4/2/defs.tex}
\item In a certain lottery 10,000 tickets are sold and ten equal prizes are awarded. What is the probability of not getting a prize if you buy (a) one ticket (b) two tickets (c) 10 tickets ?	
\\
\solution
		%\input{ncert/11/16/4/4/defs.tex}
		%
\item 
Out of 100 students, two sections of 40 and 60 are formed. If you and your friend are among the 100 students, what is the probability that
\begin{enumerate}
\item you both enter the same section?
\item you both enter the different sections?
\end{enumerate}
\solution
		%\input{ncert/11/16/4/5/defs.tex}
	\item 
The number lock of a suitcase has 4 wheels each labelled with ten digits i.e. from 0 to 9.The lock opens with a sequence of four digits with no repeats.What is the probability of a person getting the right sequence to open the suitcase.
\\
\solution
		%\input{ncert/11/16/4/10/defs.tex}
		%
\item 
Two cards are drawn at random and without replacement from a pack of 52 playing cards. Find the probability that both the cards are black.
\\
\solution
		%\input{ncert/12/13/2/2/defs.tex}
		\item A box of oranges is inspected by examining three randomly selected oranges drawn without replacement. If all the three oranges are good, the box is approved for sale, otherwise, it is rejected. Find the probability that a box containing 15 oranges out of which 12 are good and 3 are bad ones will be approved for sale.
		\label{ncert/12/13/2/3/defs.tex}
		\item Two balls are drawn at random with replacement from a box containing 10 black and 8 red balls. Find the probability that
		\label{ncert/12/13/2/12}
\begin{enumerate}
\item both balls are red.
\item first ball is black and second is red.
\item one of them is black and other is red.
\end{enumerate}

\item In a hostel, 60\% of the students read Hindi newspaper, 40\% read English newspaper and 20\% read both Hindi and English newspapers. A student is selected at random.
		\label{ncert/12/13/2/15}
\begin{enumerate}
\item Find the probability that she reads neither Hindi nor English newspapers.
\item If she reads Hindi newspaper, find the probability that she reads English newspaper.
\item If she reads English newspaper, find the probability that she reads Hindi newspaper.\\
\end{enumerate}
\item The probability of obtaining an even prime number on each die, when a pair of dice is rolled is 
\begin{enumerate}
    \item $0$ 
    
    \item $\frac{1}{3}$ 
    
    \item $\frac{1}{12}$ 
    
    \item $\frac{1}{36}$ 
\end{enumerate}
\solution
		%\input{ncert/12/13/2/17/defs.tex}
	\item A bag contains 4 red and 4 black balls, another bag contains 2 red and 6 black balls. One of the two bags is selected at random and a ball is drawn from the bag which is found to be red. Find the probability that the ball is drawn from the first bag.
\\
\solution
		%\input{ncert/12/13/3/2/main.tex}
  \item
  Cards with numbers 2 to 101 are placed in a box. A card is selected at random.Find the probability that the card has
\begin{enumerate}[label=(\roman*)]
	\item an even number 
	\item a square number
\end{enumerate}
\solution
%\input{exemplar/10/13/3/32/main.tex}
\item
The king, queen and jack of clubs are removed from a deck of 52 playing cards and then well shuffled. Now one card is drawn at random from the remaining cards.  Determine the probability that the card is
\begin{enumerate}[label=(\roman*)]
\item a club
\item 10 of hearts
\end{enumerate}
\solution
%\input{exemplar/10/13/3/29/main.tex}
\item A team of medical students doing their internship have to assist during surgeries
at a city hospital. The probabilities of surgeries rated as very complex, complex,
routine, simple or very simple are respectively, 0.15, 0.20, 0.31, 0.26, .08. Find
the probabilities that a particular surgery will be rated
\begin{enumerate}
	\item complex or very complex;
	\item neither very complex nor very simple;
	\item routine or complex
	\item routine or simple
\end{enumerate}
\solution
%\input{exemplar/11/16/3/8(1)/main.tex}
\item A card is selected from a pack of 52 cards.
\begin{enumerate}[label=(\alph*)]
    \item How many points are there in the sample space?
    \item Calculate the probability that the card is an ace of spades.
    \item Calculate the probability that the card is (i) an ace and (ii) black card.
\end{enumerate}
\solution
%\input{exemplar/11/16/3/4/main2.tex}
\item The probability that a non leap year selected at random will contain 53 sundays.
\\
\solution
%\input{exemplar/10/13/1/19/main.tex}
\item One of the four persons John, Rita, Aslam or Gurpreet will be promoted next
month. Consequently the sample space consists of four elementary outcomes
S = {John promoted, Rita promoted, Aslam promoted, Gurpreet promoted}
You are told that the chances of John’s promotion is same as that of Gurpreet,
Rita’s chances of promotion are twice as likely as Johns. Aslam’s chances are
four times that of John.
\begin{enumerate}
	\item Determine
	\begin{enumerate}
		\item P (John promoted)
		\item P (Rita promoted)
		\item P (Aslam promoted)
		\item P (Gurpreet promoted)
	\end{enumerate}
	\item If A = {John promoted or Gurpreet promoted}, find P (A).
\end{enumerate}
\solution
%\input{exemplar/11/16/3/10/main.tex}
\item A card is drawn from a deck of 52 cards. Find the probability of getting a king or a heart or a red card.\\
\solution
%\input{exemplar/11/16/3/15/main.tex}
\item The probability that a student will pass his examination is 0.73, the probability of
the student getting a compartment is 0.13, and the probability that the student will
either pass or get compartment is 0.96. State True or False.\\
\solution
%\input{exemplar/11/16/3/31/main.tex}
\item A card is selected from a pack of 52 cards\\
\begin{enumerate}[label=(\alph*)]
\item How many points are there in the sample space?
\item Calculate the probability that the cards is an ace of spades.
\item Calculate the probability that the card is (i) an ace (ii)black card.\\
\end{enumerate}
%\input{ncert/11/16/3/4_1/Prob_4.tex}
\item In a non-leap year, the probability of having 53 tuesdays or 53 wednesdays is\\
\solution
%\input{exemplar/11/16/3/18/main.tex}
\item There are 1000 sealed envelopes in a box, 10 of them contain a cash prize of
Rs 100 each, 100 of them contain a cash prize of Rs 50 each and 200 of them
contain a cash prize of Rs 10 each and rest do not contain any cash prize. If they
are well shuffled and an envelope is picked up out, what is the probability that it
contains no cash prize?\\
\solution
%\input{exemplar/10/13/3/34/main.tex}
\item 
A die is thrown and a card is selected at random from a deck of 52 playing cards. The probability of getting an even number on the die and a spade card.\\
\solution
%\input{exemplar/12/13/3/78/main.tex}
\item
If 4-digit numbers greater than 5,000 are randomly formed from the digits 0, 1, 3, 5, and 7, what is the probability of forming a number divisible by 5 when:
\begin{enumerate}
    \item The digits are repeated?
    \item The repetition of digits is not allowed?
\end{enumerate}
\solution
%\input{ncert/11/16/4/9/main.tex}
\item Consider the probability space $\brak{\Omega, \mathcal{G}, P}$ where $\Omega = [0,2]$ and $\mathcal{G} = \cbrak{\phi, \Omega, [0,1], (1,2]}$. Let $X$ and $Y$ be two functions on $\Omega$ defined as
\begin{align*}
    X(\omega) = 
    \begin{cases}
        1 & \text{if }\omega \in [0, 1]\\
        2 & \text{if }\omega \in (1, 2]
    \end{cases}
\end{align*}
and
\begin{align*}
    Y(\omega) = 
    \begin{cases}
        2 & \text{if }\omega \in [0, 1.5]\\
        3 & \text{if }\omega \in (1.5, 2].
    \end{cases}
\end{align*}
Then which one of the following statements is true?
\begin{enumerate}
    \item [(A)] $X$ is a random variable with respect to $\mathcal{G}$, but $Y$ is not a random variable with respect to $\mathcal{G}$.
    \item [(B)] $Y$ is a random variable with respect to $\mathcal{G}$, but $X$ is not a random variable with respect to $\mathcal{G}$.
    \item [(C)] Neither $X$ nor $Y$ is a random variable with respect to $\mathcal{G}$.
    \item [(D)] Both $X$ and $Y$ are random variables with respect to $\mathcal{G}$.
\end{enumerate} \hfill (GATE ST 2023)\\
\solution
%\input{gate/ST/2023/14/main.tex}
	\item  A die is loaded in such a way that each odd number is twice as likely to occur as
each even number. Find $P(G)$, where $G$ is the event that a number greater than
3 occurs on a single roll of the die.
\\
\solution
		%\input{exemplar/11/16/3/5/main.tex}
	\item All the jacks, queens and kings are removed from a deck of 52 playing cards. The remaining cards are well shuffled and then one card is drawn at random. Giving ace a value 1 similar value for other cards, find the probability that the card has a value 
		\begin{enumerate}
			\item 7
			\item greater than 7
			\item less than 7
		\end{enumerate}
		%\input{exemplar/10/13/3/30/main.tex}
  \item A Lot consists of 48 mobile phones of which 42 are good, 3 have only minor defects and 3 have major defects.Varnika will buy a phone if it is good but the trader will only buy a mobile if it has no major defects. One phone is selected at random from the lot. What is the probability that it is
\begin{enumerate}
	\item acceptable to Varnika?
            \item acceptable to the trader?
\end{enumerate}
\solution
	%\input{exemplar/10/13/3/40/main.tex}
 \item A student says that if you throw a die, it will show up 1 or not 1. Therefore, the probability of getting 1 and the probability of getting 'not 1' each is equal to $\frac{1}{2}$. Is this correct? Give reasons.\\
 \solution
        %\input{exemplar/10/13/2/9/main.tex}
   \item Four candidates A, B, C, D have ap-
plied for the assignment to coach a school cricket
team. If A is twice as likely to be selected as B, and
B and C are given about the same chance of being
selected, while C is twice as likely to be selected
as D, what are the probabilities that
\begin{enumerate}
\item C will be selected?
\item A will not be selected?
\end{enumerate}
	%\input{exemplar/11/16/3/9/main.tex}
 \item A bag contain 24 balls of which $x$ balls are red, $2x$ are white and $3x$ are blue. A ball is selected at random, What is the probability that it is
\begin{enumerate}[label=\alph*)]
\item not red ?
\item white ?
\end{enumerate}
%\input{exemplar/10/13/3/41/main.tex}
If the letters of the word ASSASSINATION are arranged at random. Find the Probability that
\begin{enumerate}[label=(\alph*)]
\item Four $S's$ come consecutively in the word
\item Two  $I's$ and two $N's$ come together
\item All $A's$ are not coming together
\item No two $A's$ are coming together
\end{enumerate}
%\input{exemplar/11/16/3/14/main.tex}
	\item One urn contains two black balls (labelled B1 and B2) and one white ball. A
	second urn contains one black ball and two white balls (labelled W1 and W2).
	Suppose the following experiment is performed. One of the two urns is chosen
	at random. Next a ball is randomly chosen from the urn. Then a second ball is
	chosen at random from the same urn without replacing the first ball.
	
	\begin{enumerate}
	\item What is the probability that two black balls are chosen?
	
	\item What is the probability that two balls of opposite colour are chosen?
	\end{enumerate}
	\solution
	%\input{exemplar/11/16/3/12/main1.tex}
\end{enumerate}

\item In a certain lottery 10,000 tickets are sold and ten equal prizes are awarded. What is the probability of not getting a prize if you buy (a) one ticket (b) two tickets (c) 10 tickets ?	
\\
\solution
		%\begin{enumerate}[label=\thesection.\arabic*,ref=\thesection.\theenumi]
	\item One card is drawn from a well-shuffled deck of 52 cards. Find the probability of getting
\begin{enumerate}
\item A king of red colour 
\item A face card 
\item A red face card
\item The jack of hearts
\item A spade
\item The queen of diamonds

\end{enumerate}
\solution
		%\input{ncert/10/15/1/14/main.tex}
	\item Five cards—the ten, jack, queen, king and ace of diamonds, are well-shuffled with their face downwards. One card is then picked up at random.
\begin{enumerate}
\item
What is the probability that the card is the queen? 
\item
If the queen is drawn and put aside, what is the probability that the second card picked up is (a) an ace? (b) a queen?\\
\end{enumerate}
\solution
		%\input{ncert/10/15/1/15/defs.tex}
	\item A bag contains $5$ red balls and some blue balls. If the probability of drawing a blue ball is double that if a red ball, determine the number of blue balls in the bag. 
		\\
\solution
		%\input{ncert/10/15/2/3/defs.tex}
	\item A card is selected from a pack of 52 cards.
 \begin{enumerate}[label=(\alph*)] 
                 \item How many points are there in the sample space?
                 \item Calculate the probability that the card is an ace of spades.
                 \item Calculate the probability that the card is (i) an ace and (ii) black card.
 \end{enumerate}
\solution
		%\input{ncert/11/16/3/4/main.tex}
\item Four cards are drawn from a well-shuffled deck of 52 cards. What is the probability of obtaining 3 diamonds and one spade.
\\
\solution
		%\input{ncert/11/16/4/2/defs.tex}
\item In a certain lottery 10,000 tickets are sold and ten equal prizes are awarded. What is the probability of not getting a prize if you buy (a) one ticket (b) two tickets (c) 10 tickets ?	
\\
\solution
		%\input{ncert/11/16/4/4/defs.tex}
		%
\item 
Out of 100 students, two sections of 40 and 60 are formed. If you and your friend are among the 100 students, what is the probability that
\begin{enumerate}
\item you both enter the same section?
\item you both enter the different sections?
\end{enumerate}
\solution
		%\input{ncert/11/16/4/5/defs.tex}
	\item 
The number lock of a suitcase has 4 wheels each labelled with ten digits i.e. from 0 to 9.The lock opens with a sequence of four digits with no repeats.What is the probability of a person getting the right sequence to open the suitcase.
\\
\solution
		%\input{ncert/11/16/4/10/defs.tex}
		%
\item 
Two cards are drawn at random and without replacement from a pack of 52 playing cards. Find the probability that both the cards are black.
\\
\solution
		%\input{ncert/12/13/2/2/defs.tex}
		\item A box of oranges is inspected by examining three randomly selected oranges drawn without replacement. If all the three oranges are good, the box is approved for sale, otherwise, it is rejected. Find the probability that a box containing 15 oranges out of which 12 are good and 3 are bad ones will be approved for sale.
		\label{ncert/12/13/2/3/defs.tex}
		\item Two balls are drawn at random with replacement from a box containing 10 black and 8 red balls. Find the probability that
		\label{ncert/12/13/2/12}
\begin{enumerate}
\item both balls are red.
\item first ball is black and second is red.
\item one of them is black and other is red.
\end{enumerate}

\item In a hostel, 60\% of the students read Hindi newspaper, 40\% read English newspaper and 20\% read both Hindi and English newspapers. A student is selected at random.
		\label{ncert/12/13/2/15}
\begin{enumerate}
\item Find the probability that she reads neither Hindi nor English newspapers.
\item If she reads Hindi newspaper, find the probability that she reads English newspaper.
\item If she reads English newspaper, find the probability that she reads Hindi newspaper.\\
\end{enumerate}
\item The probability of obtaining an even prime number on each die, when a pair of dice is rolled is 
\begin{enumerate}
    \item $0$ 
    
    \item $\frac{1}{3}$ 
    
    \item $\frac{1}{12}$ 
    
    \item $\frac{1}{36}$ 
\end{enumerate}
\solution
		%\input{ncert/12/13/2/17/defs.tex}
	\item A bag contains 4 red and 4 black balls, another bag contains 2 red and 6 black balls. One of the two bags is selected at random and a ball is drawn from the bag which is found to be red. Find the probability that the ball is drawn from the first bag.
\\
\solution
		%\input{ncert/12/13/3/2/main.tex}
  \item
  Cards with numbers 2 to 101 are placed in a box. A card is selected at random.Find the probability that the card has
\begin{enumerate}[label=(\roman*)]
	\item an even number 
	\item a square number
\end{enumerate}
\solution
%\input{exemplar/10/13/3/32/main.tex}
\item
The king, queen and jack of clubs are removed from a deck of 52 playing cards and then well shuffled. Now one card is drawn at random from the remaining cards.  Determine the probability that the card is
\begin{enumerate}[label=(\roman*)]
\item a club
\item 10 of hearts
\end{enumerate}
\solution
%\input{exemplar/10/13/3/29/main.tex}
\item A team of medical students doing their internship have to assist during surgeries
at a city hospital. The probabilities of surgeries rated as very complex, complex,
routine, simple or very simple are respectively, 0.15, 0.20, 0.31, 0.26, .08. Find
the probabilities that a particular surgery will be rated
\begin{enumerate}
	\item complex or very complex;
	\item neither very complex nor very simple;
	\item routine or complex
	\item routine or simple
\end{enumerate}
\solution
%\input{exemplar/11/16/3/8(1)/main.tex}
\item A card is selected from a pack of 52 cards.
\begin{enumerate}[label=(\alph*)]
    \item How many points are there in the sample space?
    \item Calculate the probability that the card is an ace of spades.
    \item Calculate the probability that the card is (i) an ace and (ii) black card.
\end{enumerate}
\solution
%\input{exemplar/11/16/3/4/main2.tex}
\item The probability that a non leap year selected at random will contain 53 sundays.
\\
\solution
%\input{exemplar/10/13/1/19/main.tex}
\item One of the four persons John, Rita, Aslam or Gurpreet will be promoted next
month. Consequently the sample space consists of four elementary outcomes
S = {John promoted, Rita promoted, Aslam promoted, Gurpreet promoted}
You are told that the chances of John’s promotion is same as that of Gurpreet,
Rita’s chances of promotion are twice as likely as Johns. Aslam’s chances are
four times that of John.
\begin{enumerate}
	\item Determine
	\begin{enumerate}
		\item P (John promoted)
		\item P (Rita promoted)
		\item P (Aslam promoted)
		\item P (Gurpreet promoted)
	\end{enumerate}
	\item If A = {John promoted or Gurpreet promoted}, find P (A).
\end{enumerate}
\solution
%\input{exemplar/11/16/3/10/main.tex}
\item A card is drawn from a deck of 52 cards. Find the probability of getting a king or a heart or a red card.\\
\solution
%\input{exemplar/11/16/3/15/main.tex}
\item The probability that a student will pass his examination is 0.73, the probability of
the student getting a compartment is 0.13, and the probability that the student will
either pass or get compartment is 0.96. State True or False.\\
\solution
%\input{exemplar/11/16/3/31/main.tex}
\item A card is selected from a pack of 52 cards\\
\begin{enumerate}[label=(\alph*)]
\item How many points are there in the sample space?
\item Calculate the probability that the cards is an ace of spades.
\item Calculate the probability that the card is (i) an ace (ii)black card.\\
\end{enumerate}
%\input{ncert/11/16/3/4_1/Prob_4.tex}
\item In a non-leap year, the probability of having 53 tuesdays or 53 wednesdays is\\
\solution
%\input{exemplar/11/16/3/18/main.tex}
\item There are 1000 sealed envelopes in a box, 10 of them contain a cash prize of
Rs 100 each, 100 of them contain a cash prize of Rs 50 each and 200 of them
contain a cash prize of Rs 10 each and rest do not contain any cash prize. If they
are well shuffled and an envelope is picked up out, what is the probability that it
contains no cash prize?\\
\solution
%\input{exemplar/10/13/3/34/main.tex}
\item 
A die is thrown and a card is selected at random from a deck of 52 playing cards. The probability of getting an even number on the die and a spade card.\\
\solution
%\input{exemplar/12/13/3/78/main.tex}
\item
If 4-digit numbers greater than 5,000 are randomly formed from the digits 0, 1, 3, 5, and 7, what is the probability of forming a number divisible by 5 when:
\begin{enumerate}
    \item The digits are repeated?
    \item The repetition of digits is not allowed?
\end{enumerate}
\solution
%\input{ncert/11/16/4/9/main.tex}
\item Consider the probability space $\brak{\Omega, \mathcal{G}, P}$ where $\Omega = [0,2]$ and $\mathcal{G} = \cbrak{\phi, \Omega, [0,1], (1,2]}$. Let $X$ and $Y$ be two functions on $\Omega$ defined as
\begin{align*}
    X(\omega) = 
    \begin{cases}
        1 & \text{if }\omega \in [0, 1]\\
        2 & \text{if }\omega \in (1, 2]
    \end{cases}
\end{align*}
and
\begin{align*}
    Y(\omega) = 
    \begin{cases}
        2 & \text{if }\omega \in [0, 1.5]\\
        3 & \text{if }\omega \in (1.5, 2].
    \end{cases}
\end{align*}
Then which one of the following statements is true?
\begin{enumerate}
    \item [(A)] $X$ is a random variable with respect to $\mathcal{G}$, but $Y$ is not a random variable with respect to $\mathcal{G}$.
    \item [(B)] $Y$ is a random variable with respect to $\mathcal{G}$, but $X$ is not a random variable with respect to $\mathcal{G}$.
    \item [(C)] Neither $X$ nor $Y$ is a random variable with respect to $\mathcal{G}$.
    \item [(D)] Both $X$ and $Y$ are random variables with respect to $\mathcal{G}$.
\end{enumerate} \hfill (GATE ST 2023)\\
\solution
%\input{gate/ST/2023/14/main.tex}
	\item  A die is loaded in such a way that each odd number is twice as likely to occur as
each even number. Find $P(G)$, where $G$ is the event that a number greater than
3 occurs on a single roll of the die.
\\
\solution
		%\input{exemplar/11/16/3/5/main.tex}
	\item All the jacks, queens and kings are removed from a deck of 52 playing cards. The remaining cards are well shuffled and then one card is drawn at random. Giving ace a value 1 similar value for other cards, find the probability that the card has a value 
		\begin{enumerate}
			\item 7
			\item greater than 7
			\item less than 7
		\end{enumerate}
		%\input{exemplar/10/13/3/30/main.tex}
  \item A Lot consists of 48 mobile phones of which 42 are good, 3 have only minor defects and 3 have major defects.Varnika will buy a phone if it is good but the trader will only buy a mobile if it has no major defects. One phone is selected at random from the lot. What is the probability that it is
\begin{enumerate}
	\item acceptable to Varnika?
            \item acceptable to the trader?
\end{enumerate}
\solution
	%\input{exemplar/10/13/3/40/main.tex}
 \item A student says that if you throw a die, it will show up 1 or not 1. Therefore, the probability of getting 1 and the probability of getting 'not 1' each is equal to $\frac{1}{2}$. Is this correct? Give reasons.\\
 \solution
        %\input{exemplar/10/13/2/9/main.tex}
   \item Four candidates A, B, C, D have ap-
plied for the assignment to coach a school cricket
team. If A is twice as likely to be selected as B, and
B and C are given about the same chance of being
selected, while C is twice as likely to be selected
as D, what are the probabilities that
\begin{enumerate}
\item C will be selected?
\item A will not be selected?
\end{enumerate}
	%\input{exemplar/11/16/3/9/main.tex}
 \item A bag contain 24 balls of which $x$ balls are red, $2x$ are white and $3x$ are blue. A ball is selected at random, What is the probability that it is
\begin{enumerate}[label=\alph*)]
\item not red ?
\item white ?
\end{enumerate}
%\input{exemplar/10/13/3/41/main.tex}
If the letters of the word ASSASSINATION are arranged at random. Find the Probability that
\begin{enumerate}[label=(\alph*)]
\item Four $S's$ come consecutively in the word
\item Two  $I's$ and two $N's$ come together
\item All $A's$ are not coming together
\item No two $A's$ are coming together
\end{enumerate}
%\input{exemplar/11/16/3/14/main.tex}
	\item One urn contains two black balls (labelled B1 and B2) and one white ball. A
	second urn contains one black ball and two white balls (labelled W1 and W2).
	Suppose the following experiment is performed. One of the two urns is chosen
	at random. Next a ball is randomly chosen from the urn. Then a second ball is
	chosen at random from the same urn without replacing the first ball.
	
	\begin{enumerate}
	\item What is the probability that two black balls are chosen?
	
	\item What is the probability that two balls of opposite colour are chosen?
	\end{enumerate}
	\solution
	%\input{exemplar/11/16/3/12/main1.tex}
\end{enumerate}

		%
\item 
Out of 100 students, two sections of 40 and 60 are formed. If you and your friend are among the 100 students, what is the probability that
\begin{enumerate}
\item you both enter the same section?
\item you both enter the different sections?
\end{enumerate}
\solution
		%\begin{enumerate}[label=\thesection.\arabic*,ref=\thesection.\theenumi]
	\item One card is drawn from a well-shuffled deck of 52 cards. Find the probability of getting
\begin{enumerate}
\item A king of red colour 
\item A face card 
\item A red face card
\item The jack of hearts
\item A spade
\item The queen of diamonds

\end{enumerate}
\solution
		%\input{ncert/10/15/1/14/main.tex}
	\item Five cards—the ten, jack, queen, king and ace of diamonds, are well-shuffled with their face downwards. One card is then picked up at random.
\begin{enumerate}
\item
What is the probability that the card is the queen? 
\item
If the queen is drawn and put aside, what is the probability that the second card picked up is (a) an ace? (b) a queen?\\
\end{enumerate}
\solution
		%\input{ncert/10/15/1/15/defs.tex}
	\item A bag contains $5$ red balls and some blue balls. If the probability of drawing a blue ball is double that if a red ball, determine the number of blue balls in the bag. 
		\\
\solution
		%\input{ncert/10/15/2/3/defs.tex}
	\item A card is selected from a pack of 52 cards.
 \begin{enumerate}[label=(\alph*)] 
                 \item How many points are there in the sample space?
                 \item Calculate the probability that the card is an ace of spades.
                 \item Calculate the probability that the card is (i) an ace and (ii) black card.
 \end{enumerate}
\solution
		%\input{ncert/11/16/3/4/main.tex}
\item Four cards are drawn from a well-shuffled deck of 52 cards. What is the probability of obtaining 3 diamonds and one spade.
\\
\solution
		%\input{ncert/11/16/4/2/defs.tex}
\item In a certain lottery 10,000 tickets are sold and ten equal prizes are awarded. What is the probability of not getting a prize if you buy (a) one ticket (b) two tickets (c) 10 tickets ?	
\\
\solution
		%\input{ncert/11/16/4/4/defs.tex}
		%
\item 
Out of 100 students, two sections of 40 and 60 are formed. If you and your friend are among the 100 students, what is the probability that
\begin{enumerate}
\item you both enter the same section?
\item you both enter the different sections?
\end{enumerate}
\solution
		%\input{ncert/11/16/4/5/defs.tex}
	\item 
The number lock of a suitcase has 4 wheels each labelled with ten digits i.e. from 0 to 9.The lock opens with a sequence of four digits with no repeats.What is the probability of a person getting the right sequence to open the suitcase.
\\
\solution
		%\input{ncert/11/16/4/10/defs.tex}
		%
\item 
Two cards are drawn at random and without replacement from a pack of 52 playing cards. Find the probability that both the cards are black.
\\
\solution
		%\input{ncert/12/13/2/2/defs.tex}
		\item A box of oranges is inspected by examining three randomly selected oranges drawn without replacement. If all the three oranges are good, the box is approved for sale, otherwise, it is rejected. Find the probability that a box containing 15 oranges out of which 12 are good and 3 are bad ones will be approved for sale.
		\label{ncert/12/13/2/3/defs.tex}
		\item Two balls are drawn at random with replacement from a box containing 10 black and 8 red balls. Find the probability that
		\label{ncert/12/13/2/12}
\begin{enumerate}
\item both balls are red.
\item first ball is black and second is red.
\item one of them is black and other is red.
\end{enumerate}

\item In a hostel, 60\% of the students read Hindi newspaper, 40\% read English newspaper and 20\% read both Hindi and English newspapers. A student is selected at random.
		\label{ncert/12/13/2/15}
\begin{enumerate}
\item Find the probability that she reads neither Hindi nor English newspapers.
\item If she reads Hindi newspaper, find the probability that she reads English newspaper.
\item If she reads English newspaper, find the probability that she reads Hindi newspaper.\\
\end{enumerate}
\item The probability of obtaining an even prime number on each die, when a pair of dice is rolled is 
\begin{enumerate}
    \item $0$ 
    
    \item $\frac{1}{3}$ 
    
    \item $\frac{1}{12}$ 
    
    \item $\frac{1}{36}$ 
\end{enumerate}
\solution
		%\input{ncert/12/13/2/17/defs.tex}
	\item A bag contains 4 red and 4 black balls, another bag contains 2 red and 6 black balls. One of the two bags is selected at random and a ball is drawn from the bag which is found to be red. Find the probability that the ball is drawn from the first bag.
\\
\solution
		%\input{ncert/12/13/3/2/main.tex}
  \item
  Cards with numbers 2 to 101 are placed in a box. A card is selected at random.Find the probability that the card has
\begin{enumerate}[label=(\roman*)]
	\item an even number 
	\item a square number
\end{enumerate}
\solution
%\input{exemplar/10/13/3/32/main.tex}
\item
The king, queen and jack of clubs are removed from a deck of 52 playing cards and then well shuffled. Now one card is drawn at random from the remaining cards.  Determine the probability that the card is
\begin{enumerate}[label=(\roman*)]
\item a club
\item 10 of hearts
\end{enumerate}
\solution
%\input{exemplar/10/13/3/29/main.tex}
\item A team of medical students doing their internship have to assist during surgeries
at a city hospital. The probabilities of surgeries rated as very complex, complex,
routine, simple or very simple are respectively, 0.15, 0.20, 0.31, 0.26, .08. Find
the probabilities that a particular surgery will be rated
\begin{enumerate}
	\item complex or very complex;
	\item neither very complex nor very simple;
	\item routine or complex
	\item routine or simple
\end{enumerate}
\solution
%\input{exemplar/11/16/3/8(1)/main.tex}
\item A card is selected from a pack of 52 cards.
\begin{enumerate}[label=(\alph*)]
    \item How many points are there in the sample space?
    \item Calculate the probability that the card is an ace of spades.
    \item Calculate the probability that the card is (i) an ace and (ii) black card.
\end{enumerate}
\solution
%\input{exemplar/11/16/3/4/main2.tex}
\item The probability that a non leap year selected at random will contain 53 sundays.
\\
\solution
%\input{exemplar/10/13/1/19/main.tex}
\item One of the four persons John, Rita, Aslam or Gurpreet will be promoted next
month. Consequently the sample space consists of four elementary outcomes
S = {John promoted, Rita promoted, Aslam promoted, Gurpreet promoted}
You are told that the chances of John’s promotion is same as that of Gurpreet,
Rita’s chances of promotion are twice as likely as Johns. Aslam’s chances are
four times that of John.
\begin{enumerate}
	\item Determine
	\begin{enumerate}
		\item P (John promoted)
		\item P (Rita promoted)
		\item P (Aslam promoted)
		\item P (Gurpreet promoted)
	\end{enumerate}
	\item If A = {John promoted or Gurpreet promoted}, find P (A).
\end{enumerate}
\solution
%\input{exemplar/11/16/3/10/main.tex}
\item A card is drawn from a deck of 52 cards. Find the probability of getting a king or a heart or a red card.\\
\solution
%\input{exemplar/11/16/3/15/main.tex}
\item The probability that a student will pass his examination is 0.73, the probability of
the student getting a compartment is 0.13, and the probability that the student will
either pass or get compartment is 0.96. State True or False.\\
\solution
%\input{exemplar/11/16/3/31/main.tex}
\item A card is selected from a pack of 52 cards\\
\begin{enumerate}[label=(\alph*)]
\item How many points are there in the sample space?
\item Calculate the probability that the cards is an ace of spades.
\item Calculate the probability that the card is (i) an ace (ii)black card.\\
\end{enumerate}
%\input{ncert/11/16/3/4_1/Prob_4.tex}
\item In a non-leap year, the probability of having 53 tuesdays or 53 wednesdays is\\
\solution
%\input{exemplar/11/16/3/18/main.tex}
\item There are 1000 sealed envelopes in a box, 10 of them contain a cash prize of
Rs 100 each, 100 of them contain a cash prize of Rs 50 each and 200 of them
contain a cash prize of Rs 10 each and rest do not contain any cash prize. If they
are well shuffled and an envelope is picked up out, what is the probability that it
contains no cash prize?\\
\solution
%\input{exemplar/10/13/3/34/main.tex}
\item 
A die is thrown and a card is selected at random from a deck of 52 playing cards. The probability of getting an even number on the die and a spade card.\\
\solution
%\input{exemplar/12/13/3/78/main.tex}
\item
If 4-digit numbers greater than 5,000 are randomly formed from the digits 0, 1, 3, 5, and 7, what is the probability of forming a number divisible by 5 when:
\begin{enumerate}
    \item The digits are repeated?
    \item The repetition of digits is not allowed?
\end{enumerate}
\solution
%\input{ncert/11/16/4/9/main.tex}
\item Consider the probability space $\brak{\Omega, \mathcal{G}, P}$ where $\Omega = [0,2]$ and $\mathcal{G} = \cbrak{\phi, \Omega, [0,1], (1,2]}$. Let $X$ and $Y$ be two functions on $\Omega$ defined as
\begin{align*}
    X(\omega) = 
    \begin{cases}
        1 & \text{if }\omega \in [0, 1]\\
        2 & \text{if }\omega \in (1, 2]
    \end{cases}
\end{align*}
and
\begin{align*}
    Y(\omega) = 
    \begin{cases}
        2 & \text{if }\omega \in [0, 1.5]\\
        3 & \text{if }\omega \in (1.5, 2].
    \end{cases}
\end{align*}
Then which one of the following statements is true?
\begin{enumerate}
    \item [(A)] $X$ is a random variable with respect to $\mathcal{G}$, but $Y$ is not a random variable with respect to $\mathcal{G}$.
    \item [(B)] $Y$ is a random variable with respect to $\mathcal{G}$, but $X$ is not a random variable with respect to $\mathcal{G}$.
    \item [(C)] Neither $X$ nor $Y$ is a random variable with respect to $\mathcal{G}$.
    \item [(D)] Both $X$ and $Y$ are random variables with respect to $\mathcal{G}$.
\end{enumerate} \hfill (GATE ST 2023)\\
\solution
%\input{gate/ST/2023/14/main.tex}
	\item  A die is loaded in such a way that each odd number is twice as likely to occur as
each even number. Find $P(G)$, where $G$ is the event that a number greater than
3 occurs on a single roll of the die.
\\
\solution
		%\input{exemplar/11/16/3/5/main.tex}
	\item All the jacks, queens and kings are removed from a deck of 52 playing cards. The remaining cards are well shuffled and then one card is drawn at random. Giving ace a value 1 similar value for other cards, find the probability that the card has a value 
		\begin{enumerate}
			\item 7
			\item greater than 7
			\item less than 7
		\end{enumerate}
		%\input{exemplar/10/13/3/30/main.tex}
  \item A Lot consists of 48 mobile phones of which 42 are good, 3 have only minor defects and 3 have major defects.Varnika will buy a phone if it is good but the trader will only buy a mobile if it has no major defects. One phone is selected at random from the lot. What is the probability that it is
\begin{enumerate}
	\item acceptable to Varnika?
            \item acceptable to the trader?
\end{enumerate}
\solution
	%\input{exemplar/10/13/3/40/main.tex}
 \item A student says that if you throw a die, it will show up 1 or not 1. Therefore, the probability of getting 1 and the probability of getting 'not 1' each is equal to $\frac{1}{2}$. Is this correct? Give reasons.\\
 \solution
        %\input{exemplar/10/13/2/9/main.tex}
   \item Four candidates A, B, C, D have ap-
plied for the assignment to coach a school cricket
team. If A is twice as likely to be selected as B, and
B and C are given about the same chance of being
selected, while C is twice as likely to be selected
as D, what are the probabilities that
\begin{enumerate}
\item C will be selected?
\item A will not be selected?
\end{enumerate}
	%\input{exemplar/11/16/3/9/main.tex}
 \item A bag contain 24 balls of which $x$ balls are red, $2x$ are white and $3x$ are blue. A ball is selected at random, What is the probability that it is
\begin{enumerate}[label=\alph*)]
\item not red ?
\item white ?
\end{enumerate}
%\input{exemplar/10/13/3/41/main.tex}
If the letters of the word ASSASSINATION are arranged at random. Find the Probability that
\begin{enumerate}[label=(\alph*)]
\item Four $S's$ come consecutively in the word
\item Two  $I's$ and two $N's$ come together
\item All $A's$ are not coming together
\item No two $A's$ are coming together
\end{enumerate}
%\input{exemplar/11/16/3/14/main.tex}
	\item One urn contains two black balls (labelled B1 and B2) and one white ball. A
	second urn contains one black ball and two white balls (labelled W1 and W2).
	Suppose the following experiment is performed. One of the two urns is chosen
	at random. Next a ball is randomly chosen from the urn. Then a second ball is
	chosen at random from the same urn without replacing the first ball.
	
	\begin{enumerate}
	\item What is the probability that two black balls are chosen?
	
	\item What is the probability that two balls of opposite colour are chosen?
	\end{enumerate}
	\solution
	%\input{exemplar/11/16/3/12/main1.tex}
\end{enumerate}

	\item 
The number lock of a suitcase has 4 wheels each labelled with ten digits i.e. from 0 to 9.The lock opens with a sequence of four digits with no repeats.What is the probability of a person getting the right sequence to open the suitcase.
\\
\solution
		%\begin{enumerate}[label=\thesection.\arabic*,ref=\thesection.\theenumi]
	\item One card is drawn from a well-shuffled deck of 52 cards. Find the probability of getting
\begin{enumerate}
\item A king of red colour 
\item A face card 
\item A red face card
\item The jack of hearts
\item A spade
\item The queen of diamonds

\end{enumerate}
\solution
		%\input{ncert/10/15/1/14/main.tex}
	\item Five cards—the ten, jack, queen, king and ace of diamonds, are well-shuffled with their face downwards. One card is then picked up at random.
\begin{enumerate}
\item
What is the probability that the card is the queen? 
\item
If the queen is drawn and put aside, what is the probability that the second card picked up is (a) an ace? (b) a queen?\\
\end{enumerate}
\solution
		%\input{ncert/10/15/1/15/defs.tex}
	\item A bag contains $5$ red balls and some blue balls. If the probability of drawing a blue ball is double that if a red ball, determine the number of blue balls in the bag. 
		\\
\solution
		%\input{ncert/10/15/2/3/defs.tex}
	\item A card is selected from a pack of 52 cards.
 \begin{enumerate}[label=(\alph*)] 
                 \item How many points are there in the sample space?
                 \item Calculate the probability that the card is an ace of spades.
                 \item Calculate the probability that the card is (i) an ace and (ii) black card.
 \end{enumerate}
\solution
		%\input{ncert/11/16/3/4/main.tex}
\item Four cards are drawn from a well-shuffled deck of 52 cards. What is the probability of obtaining 3 diamonds and one spade.
\\
\solution
		%\input{ncert/11/16/4/2/defs.tex}
\item In a certain lottery 10,000 tickets are sold and ten equal prizes are awarded. What is the probability of not getting a prize if you buy (a) one ticket (b) two tickets (c) 10 tickets ?	
\\
\solution
		%\input{ncert/11/16/4/4/defs.tex}
		%
\item 
Out of 100 students, two sections of 40 and 60 are formed. If you and your friend are among the 100 students, what is the probability that
\begin{enumerate}
\item you both enter the same section?
\item you both enter the different sections?
\end{enumerate}
\solution
		%\input{ncert/11/16/4/5/defs.tex}
	\item 
The number lock of a suitcase has 4 wheels each labelled with ten digits i.e. from 0 to 9.The lock opens with a sequence of four digits with no repeats.What is the probability of a person getting the right sequence to open the suitcase.
\\
\solution
		%\input{ncert/11/16/4/10/defs.tex}
		%
\item 
Two cards are drawn at random and without replacement from a pack of 52 playing cards. Find the probability that both the cards are black.
\\
\solution
		%\input{ncert/12/13/2/2/defs.tex}
		\item A box of oranges is inspected by examining three randomly selected oranges drawn without replacement. If all the three oranges are good, the box is approved for sale, otherwise, it is rejected. Find the probability that a box containing 15 oranges out of which 12 are good and 3 are bad ones will be approved for sale.
		\label{ncert/12/13/2/3/defs.tex}
		\item Two balls are drawn at random with replacement from a box containing 10 black and 8 red balls. Find the probability that
		\label{ncert/12/13/2/12}
\begin{enumerate}
\item both balls are red.
\item first ball is black and second is red.
\item one of them is black and other is red.
\end{enumerate}

\item In a hostel, 60\% of the students read Hindi newspaper, 40\% read English newspaper and 20\% read both Hindi and English newspapers. A student is selected at random.
		\label{ncert/12/13/2/15}
\begin{enumerate}
\item Find the probability that she reads neither Hindi nor English newspapers.
\item If she reads Hindi newspaper, find the probability that she reads English newspaper.
\item If she reads English newspaper, find the probability that she reads Hindi newspaper.\\
\end{enumerate}
\item The probability of obtaining an even prime number on each die, when a pair of dice is rolled is 
\begin{enumerate}
    \item $0$ 
    
    \item $\frac{1}{3}$ 
    
    \item $\frac{1}{12}$ 
    
    \item $\frac{1}{36}$ 
\end{enumerate}
\solution
		%\input{ncert/12/13/2/17/defs.tex}
	\item A bag contains 4 red and 4 black balls, another bag contains 2 red and 6 black balls. One of the two bags is selected at random and a ball is drawn from the bag which is found to be red. Find the probability that the ball is drawn from the first bag.
\\
\solution
		%\input{ncert/12/13/3/2/main.tex}
  \item
  Cards with numbers 2 to 101 are placed in a box. A card is selected at random.Find the probability that the card has
\begin{enumerate}[label=(\roman*)]
	\item an even number 
	\item a square number
\end{enumerate}
\solution
%\input{exemplar/10/13/3/32/main.tex}
\item
The king, queen and jack of clubs are removed from a deck of 52 playing cards and then well shuffled. Now one card is drawn at random from the remaining cards.  Determine the probability that the card is
\begin{enumerate}[label=(\roman*)]
\item a club
\item 10 of hearts
\end{enumerate}
\solution
%\input{exemplar/10/13/3/29/main.tex}
\item A team of medical students doing their internship have to assist during surgeries
at a city hospital. The probabilities of surgeries rated as very complex, complex,
routine, simple or very simple are respectively, 0.15, 0.20, 0.31, 0.26, .08. Find
the probabilities that a particular surgery will be rated
\begin{enumerate}
	\item complex or very complex;
	\item neither very complex nor very simple;
	\item routine or complex
	\item routine or simple
\end{enumerate}
\solution
%\input{exemplar/11/16/3/8(1)/main.tex}
\item A card is selected from a pack of 52 cards.
\begin{enumerate}[label=(\alph*)]
    \item How many points are there in the sample space?
    \item Calculate the probability that the card is an ace of spades.
    \item Calculate the probability that the card is (i) an ace and (ii) black card.
\end{enumerate}
\solution
%\input{exemplar/11/16/3/4/main2.tex}
\item The probability that a non leap year selected at random will contain 53 sundays.
\\
\solution
%\input{exemplar/10/13/1/19/main.tex}
\item One of the four persons John, Rita, Aslam or Gurpreet will be promoted next
month. Consequently the sample space consists of four elementary outcomes
S = {John promoted, Rita promoted, Aslam promoted, Gurpreet promoted}
You are told that the chances of John’s promotion is same as that of Gurpreet,
Rita’s chances of promotion are twice as likely as Johns. Aslam’s chances are
four times that of John.
\begin{enumerate}
	\item Determine
	\begin{enumerate}
		\item P (John promoted)
		\item P (Rita promoted)
		\item P (Aslam promoted)
		\item P (Gurpreet promoted)
	\end{enumerate}
	\item If A = {John promoted or Gurpreet promoted}, find P (A).
\end{enumerate}
\solution
%\input{exemplar/11/16/3/10/main.tex}
\item A card is drawn from a deck of 52 cards. Find the probability of getting a king or a heart or a red card.\\
\solution
%\input{exemplar/11/16/3/15/main.tex}
\item The probability that a student will pass his examination is 0.73, the probability of
the student getting a compartment is 0.13, and the probability that the student will
either pass or get compartment is 0.96. State True or False.\\
\solution
%\input{exemplar/11/16/3/31/main.tex}
\item A card is selected from a pack of 52 cards\\
\begin{enumerate}[label=(\alph*)]
\item How many points are there in the sample space?
\item Calculate the probability that the cards is an ace of spades.
\item Calculate the probability that the card is (i) an ace (ii)black card.\\
\end{enumerate}
%\input{ncert/11/16/3/4_1/Prob_4.tex}
\item In a non-leap year, the probability of having 53 tuesdays or 53 wednesdays is\\
\solution
%\input{exemplar/11/16/3/18/main.tex}
\item There are 1000 sealed envelopes in a box, 10 of them contain a cash prize of
Rs 100 each, 100 of them contain a cash prize of Rs 50 each and 200 of them
contain a cash prize of Rs 10 each and rest do not contain any cash prize. If they
are well shuffled and an envelope is picked up out, what is the probability that it
contains no cash prize?\\
\solution
%\input{exemplar/10/13/3/34/main.tex}
\item 
A die is thrown and a card is selected at random from a deck of 52 playing cards. The probability of getting an even number on the die and a spade card.\\
\solution
%\input{exemplar/12/13/3/78/main.tex}
\item
If 4-digit numbers greater than 5,000 are randomly formed from the digits 0, 1, 3, 5, and 7, what is the probability of forming a number divisible by 5 when:
\begin{enumerate}
    \item The digits are repeated?
    \item The repetition of digits is not allowed?
\end{enumerate}
\solution
%\input{ncert/11/16/4/9/main.tex}
\item Consider the probability space $\brak{\Omega, \mathcal{G}, P}$ where $\Omega = [0,2]$ and $\mathcal{G} = \cbrak{\phi, \Omega, [0,1], (1,2]}$. Let $X$ and $Y$ be two functions on $\Omega$ defined as
\begin{align*}
    X(\omega) = 
    \begin{cases}
        1 & \text{if }\omega \in [0, 1]\\
        2 & \text{if }\omega \in (1, 2]
    \end{cases}
\end{align*}
and
\begin{align*}
    Y(\omega) = 
    \begin{cases}
        2 & \text{if }\omega \in [0, 1.5]\\
        3 & \text{if }\omega \in (1.5, 2].
    \end{cases}
\end{align*}
Then which one of the following statements is true?
\begin{enumerate}
    \item [(A)] $X$ is a random variable with respect to $\mathcal{G}$, but $Y$ is not a random variable with respect to $\mathcal{G}$.
    \item [(B)] $Y$ is a random variable with respect to $\mathcal{G}$, but $X$ is not a random variable with respect to $\mathcal{G}$.
    \item [(C)] Neither $X$ nor $Y$ is a random variable with respect to $\mathcal{G}$.
    \item [(D)] Both $X$ and $Y$ are random variables with respect to $\mathcal{G}$.
\end{enumerate} \hfill (GATE ST 2023)\\
\solution
%\input{gate/ST/2023/14/main.tex}
	\item  A die is loaded in such a way that each odd number is twice as likely to occur as
each even number. Find $P(G)$, where $G$ is the event that a number greater than
3 occurs on a single roll of the die.
\\
\solution
		%\input{exemplar/11/16/3/5/main.tex}
	\item All the jacks, queens and kings are removed from a deck of 52 playing cards. The remaining cards are well shuffled and then one card is drawn at random. Giving ace a value 1 similar value for other cards, find the probability that the card has a value 
		\begin{enumerate}
			\item 7
			\item greater than 7
			\item less than 7
		\end{enumerate}
		%\input{exemplar/10/13/3/30/main.tex}
  \item A Lot consists of 48 mobile phones of which 42 are good, 3 have only minor defects and 3 have major defects.Varnika will buy a phone if it is good but the trader will only buy a mobile if it has no major defects. One phone is selected at random from the lot. What is the probability that it is
\begin{enumerate}
	\item acceptable to Varnika?
            \item acceptable to the trader?
\end{enumerate}
\solution
	%\input{exemplar/10/13/3/40/main.tex}
 \item A student says that if you throw a die, it will show up 1 or not 1. Therefore, the probability of getting 1 and the probability of getting 'not 1' each is equal to $\frac{1}{2}$. Is this correct? Give reasons.\\
 \solution
        %\input{exemplar/10/13/2/9/main.tex}
   \item Four candidates A, B, C, D have ap-
plied for the assignment to coach a school cricket
team. If A is twice as likely to be selected as B, and
B and C are given about the same chance of being
selected, while C is twice as likely to be selected
as D, what are the probabilities that
\begin{enumerate}
\item C will be selected?
\item A will not be selected?
\end{enumerate}
	%\input{exemplar/11/16/3/9/main.tex}
 \item A bag contain 24 balls of which $x$ balls are red, $2x$ are white and $3x$ are blue. A ball is selected at random, What is the probability that it is
\begin{enumerate}[label=\alph*)]
\item not red ?
\item white ?
\end{enumerate}
%\input{exemplar/10/13/3/41/main.tex}
If the letters of the word ASSASSINATION are arranged at random. Find the Probability that
\begin{enumerate}[label=(\alph*)]
\item Four $S's$ come consecutively in the word
\item Two  $I's$ and two $N's$ come together
\item All $A's$ are not coming together
\item No two $A's$ are coming together
\end{enumerate}
%\input{exemplar/11/16/3/14/main.tex}
	\item One urn contains two black balls (labelled B1 and B2) and one white ball. A
	second urn contains one black ball and two white balls (labelled W1 and W2).
	Suppose the following experiment is performed. One of the two urns is chosen
	at random. Next a ball is randomly chosen from the urn. Then a second ball is
	chosen at random from the same urn without replacing the first ball.
	
	\begin{enumerate}
	\item What is the probability that two black balls are chosen?
	
	\item What is the probability that two balls of opposite colour are chosen?
	\end{enumerate}
	\solution
	%\input{exemplar/11/16/3/12/main1.tex}
\end{enumerate}

		%
\item 
Two cards are drawn at random and without replacement from a pack of 52 playing cards. Find the probability that both the cards are black.
\\
\solution
		%\begin{enumerate}[label=\thesection.\arabic*,ref=\thesection.\theenumi]
	\item One card is drawn from a well-shuffled deck of 52 cards. Find the probability of getting
\begin{enumerate}
\item A king of red colour 
\item A face card 
\item A red face card
\item The jack of hearts
\item A spade
\item The queen of diamonds

\end{enumerate}
\solution
		%\input{ncert/10/15/1/14/main.tex}
	\item Five cards—the ten, jack, queen, king and ace of diamonds, are well-shuffled with their face downwards. One card is then picked up at random.
\begin{enumerate}
\item
What is the probability that the card is the queen? 
\item
If the queen is drawn and put aside, what is the probability that the second card picked up is (a) an ace? (b) a queen?\\
\end{enumerate}
\solution
		%\input{ncert/10/15/1/15/defs.tex}
	\item A bag contains $5$ red balls and some blue balls. If the probability of drawing a blue ball is double that if a red ball, determine the number of blue balls in the bag. 
		\\
\solution
		%\input{ncert/10/15/2/3/defs.tex}
	\item A card is selected from a pack of 52 cards.
 \begin{enumerate}[label=(\alph*)] 
                 \item How many points are there in the sample space?
                 \item Calculate the probability that the card is an ace of spades.
                 \item Calculate the probability that the card is (i) an ace and (ii) black card.
 \end{enumerate}
\solution
		%\input{ncert/11/16/3/4/main.tex}
\item Four cards are drawn from a well-shuffled deck of 52 cards. What is the probability of obtaining 3 diamonds and one spade.
\\
\solution
		%\input{ncert/11/16/4/2/defs.tex}
\item In a certain lottery 10,000 tickets are sold and ten equal prizes are awarded. What is the probability of not getting a prize if you buy (a) one ticket (b) two tickets (c) 10 tickets ?	
\\
\solution
		%\input{ncert/11/16/4/4/defs.tex}
		%
\item 
Out of 100 students, two sections of 40 and 60 are formed. If you and your friend are among the 100 students, what is the probability that
\begin{enumerate}
\item you both enter the same section?
\item you both enter the different sections?
\end{enumerate}
\solution
		%\input{ncert/11/16/4/5/defs.tex}
	\item 
The number lock of a suitcase has 4 wheels each labelled with ten digits i.e. from 0 to 9.The lock opens with a sequence of four digits with no repeats.What is the probability of a person getting the right sequence to open the suitcase.
\\
\solution
		%\input{ncert/11/16/4/10/defs.tex}
		%
\item 
Two cards are drawn at random and without replacement from a pack of 52 playing cards. Find the probability that both the cards are black.
\\
\solution
		%\input{ncert/12/13/2/2/defs.tex}
		\item A box of oranges is inspected by examining three randomly selected oranges drawn without replacement. If all the three oranges are good, the box is approved for sale, otherwise, it is rejected. Find the probability that a box containing 15 oranges out of which 12 are good and 3 are bad ones will be approved for sale.
		\label{ncert/12/13/2/3/defs.tex}
		\item Two balls are drawn at random with replacement from a box containing 10 black and 8 red balls. Find the probability that
		\label{ncert/12/13/2/12}
\begin{enumerate}
\item both balls are red.
\item first ball is black and second is red.
\item one of them is black and other is red.
\end{enumerate}

\item In a hostel, 60\% of the students read Hindi newspaper, 40\% read English newspaper and 20\% read both Hindi and English newspapers. A student is selected at random.
		\label{ncert/12/13/2/15}
\begin{enumerate}
\item Find the probability that she reads neither Hindi nor English newspapers.
\item If she reads Hindi newspaper, find the probability that she reads English newspaper.
\item If she reads English newspaper, find the probability that she reads Hindi newspaper.\\
\end{enumerate}
\item The probability of obtaining an even prime number on each die, when a pair of dice is rolled is 
\begin{enumerate}
    \item $0$ 
    
    \item $\frac{1}{3}$ 
    
    \item $\frac{1}{12}$ 
    
    \item $\frac{1}{36}$ 
\end{enumerate}
\solution
		%\input{ncert/12/13/2/17/defs.tex}
	\item A bag contains 4 red and 4 black balls, another bag contains 2 red and 6 black balls. One of the two bags is selected at random and a ball is drawn from the bag which is found to be red. Find the probability that the ball is drawn from the first bag.
\\
\solution
		%\input{ncert/12/13/3/2/main.tex}
  \item
  Cards with numbers 2 to 101 are placed in a box. A card is selected at random.Find the probability that the card has
\begin{enumerate}[label=(\roman*)]
	\item an even number 
	\item a square number
\end{enumerate}
\solution
%\input{exemplar/10/13/3/32/main.tex}
\item
The king, queen and jack of clubs are removed from a deck of 52 playing cards and then well shuffled. Now one card is drawn at random from the remaining cards.  Determine the probability that the card is
\begin{enumerate}[label=(\roman*)]
\item a club
\item 10 of hearts
\end{enumerate}
\solution
%\input{exemplar/10/13/3/29/main.tex}
\item A team of medical students doing their internship have to assist during surgeries
at a city hospital. The probabilities of surgeries rated as very complex, complex,
routine, simple or very simple are respectively, 0.15, 0.20, 0.31, 0.26, .08. Find
the probabilities that a particular surgery will be rated
\begin{enumerate}
	\item complex or very complex;
	\item neither very complex nor very simple;
	\item routine or complex
	\item routine or simple
\end{enumerate}
\solution
%\input{exemplar/11/16/3/8(1)/main.tex}
\item A card is selected from a pack of 52 cards.
\begin{enumerate}[label=(\alph*)]
    \item How many points are there in the sample space?
    \item Calculate the probability that the card is an ace of spades.
    \item Calculate the probability that the card is (i) an ace and (ii) black card.
\end{enumerate}
\solution
%\input{exemplar/11/16/3/4/main2.tex}
\item The probability that a non leap year selected at random will contain 53 sundays.
\\
\solution
%\input{exemplar/10/13/1/19/main.tex}
\item One of the four persons John, Rita, Aslam or Gurpreet will be promoted next
month. Consequently the sample space consists of four elementary outcomes
S = {John promoted, Rita promoted, Aslam promoted, Gurpreet promoted}
You are told that the chances of John’s promotion is same as that of Gurpreet,
Rita’s chances of promotion are twice as likely as Johns. Aslam’s chances are
four times that of John.
\begin{enumerate}
	\item Determine
	\begin{enumerate}
		\item P (John promoted)
		\item P (Rita promoted)
		\item P (Aslam promoted)
		\item P (Gurpreet promoted)
	\end{enumerate}
	\item If A = {John promoted or Gurpreet promoted}, find P (A).
\end{enumerate}
\solution
%\input{exemplar/11/16/3/10/main.tex}
\item A card is drawn from a deck of 52 cards. Find the probability of getting a king or a heart or a red card.\\
\solution
%\input{exemplar/11/16/3/15/main.tex}
\item The probability that a student will pass his examination is 0.73, the probability of
the student getting a compartment is 0.13, and the probability that the student will
either pass or get compartment is 0.96. State True or False.\\
\solution
%\input{exemplar/11/16/3/31/main.tex}
\item A card is selected from a pack of 52 cards\\
\begin{enumerate}[label=(\alph*)]
\item How many points are there in the sample space?
\item Calculate the probability that the cards is an ace of spades.
\item Calculate the probability that the card is (i) an ace (ii)black card.\\
\end{enumerate}
%\input{ncert/11/16/3/4_1/Prob_4.tex}
\item In a non-leap year, the probability of having 53 tuesdays or 53 wednesdays is\\
\solution
%\input{exemplar/11/16/3/18/main.tex}
\item There are 1000 sealed envelopes in a box, 10 of them contain a cash prize of
Rs 100 each, 100 of them contain a cash prize of Rs 50 each and 200 of them
contain a cash prize of Rs 10 each and rest do not contain any cash prize. If they
are well shuffled and an envelope is picked up out, what is the probability that it
contains no cash prize?\\
\solution
%\input{exemplar/10/13/3/34/main.tex}
\item 
A die is thrown and a card is selected at random from a deck of 52 playing cards. The probability of getting an even number on the die and a spade card.\\
\solution
%\input{exemplar/12/13/3/78/main.tex}
\item
If 4-digit numbers greater than 5,000 are randomly formed from the digits 0, 1, 3, 5, and 7, what is the probability of forming a number divisible by 5 when:
\begin{enumerate}
    \item The digits are repeated?
    \item The repetition of digits is not allowed?
\end{enumerate}
\solution
%\input{ncert/11/16/4/9/main.tex}
\item Consider the probability space $\brak{\Omega, \mathcal{G}, P}$ where $\Omega = [0,2]$ and $\mathcal{G} = \cbrak{\phi, \Omega, [0,1], (1,2]}$. Let $X$ and $Y$ be two functions on $\Omega$ defined as
\begin{align*}
    X(\omega) = 
    \begin{cases}
        1 & \text{if }\omega \in [0, 1]\\
        2 & \text{if }\omega \in (1, 2]
    \end{cases}
\end{align*}
and
\begin{align*}
    Y(\omega) = 
    \begin{cases}
        2 & \text{if }\omega \in [0, 1.5]\\
        3 & \text{if }\omega \in (1.5, 2].
    \end{cases}
\end{align*}
Then which one of the following statements is true?
\begin{enumerate}
    \item [(A)] $X$ is a random variable with respect to $\mathcal{G}$, but $Y$ is not a random variable with respect to $\mathcal{G}$.
    \item [(B)] $Y$ is a random variable with respect to $\mathcal{G}$, but $X$ is not a random variable with respect to $\mathcal{G}$.
    \item [(C)] Neither $X$ nor $Y$ is a random variable with respect to $\mathcal{G}$.
    \item [(D)] Both $X$ and $Y$ are random variables with respect to $\mathcal{G}$.
\end{enumerate} \hfill (GATE ST 2023)\\
\solution
%\input{gate/ST/2023/14/main.tex}
	\item  A die is loaded in such a way that each odd number is twice as likely to occur as
each even number. Find $P(G)$, where $G$ is the event that a number greater than
3 occurs on a single roll of the die.
\\
\solution
		%\input{exemplar/11/16/3/5/main.tex}
	\item All the jacks, queens and kings are removed from a deck of 52 playing cards. The remaining cards are well shuffled and then one card is drawn at random. Giving ace a value 1 similar value for other cards, find the probability that the card has a value 
		\begin{enumerate}
			\item 7
			\item greater than 7
			\item less than 7
		\end{enumerate}
		%\input{exemplar/10/13/3/30/main.tex}
  \item A Lot consists of 48 mobile phones of which 42 are good, 3 have only minor defects and 3 have major defects.Varnika will buy a phone if it is good but the trader will only buy a mobile if it has no major defects. One phone is selected at random from the lot. What is the probability that it is
\begin{enumerate}
	\item acceptable to Varnika?
            \item acceptable to the trader?
\end{enumerate}
\solution
	%\input{exemplar/10/13/3/40/main.tex}
 \item A student says that if you throw a die, it will show up 1 or not 1. Therefore, the probability of getting 1 and the probability of getting 'not 1' each is equal to $\frac{1}{2}$. Is this correct? Give reasons.\\
 \solution
        %\input{exemplar/10/13/2/9/main.tex}
   \item Four candidates A, B, C, D have ap-
plied for the assignment to coach a school cricket
team. If A is twice as likely to be selected as B, and
B and C are given about the same chance of being
selected, while C is twice as likely to be selected
as D, what are the probabilities that
\begin{enumerate}
\item C will be selected?
\item A will not be selected?
\end{enumerate}
	%\input{exemplar/11/16/3/9/main.tex}
 \item A bag contain 24 balls of which $x$ balls are red, $2x$ are white and $3x$ are blue. A ball is selected at random, What is the probability that it is
\begin{enumerate}[label=\alph*)]
\item not red ?
\item white ?
\end{enumerate}
%\input{exemplar/10/13/3/41/main.tex}
If the letters of the word ASSASSINATION are arranged at random. Find the Probability that
\begin{enumerate}[label=(\alph*)]
\item Four $S's$ come consecutively in the word
\item Two  $I's$ and two $N's$ come together
\item All $A's$ are not coming together
\item No two $A's$ are coming together
\end{enumerate}
%\input{exemplar/11/16/3/14/main.tex}
	\item One urn contains two black balls (labelled B1 and B2) and one white ball. A
	second urn contains one black ball and two white balls (labelled W1 and W2).
	Suppose the following experiment is performed. One of the two urns is chosen
	at random. Next a ball is randomly chosen from the urn. Then a second ball is
	chosen at random from the same urn without replacing the first ball.
	
	\begin{enumerate}
	\item What is the probability that two black balls are chosen?
	
	\item What is the probability that two balls of opposite colour are chosen?
	\end{enumerate}
	\solution
	%\input{exemplar/11/16/3/12/main1.tex}
\end{enumerate}

		\item A box of oranges is inspected by examining three randomly selected oranges drawn without replacement. If all the three oranges are good, the box is approved for sale, otherwise, it is rejected. Find the probability that a box containing 15 oranges out of which 12 are good and 3 are bad ones will be approved for sale.
		\label{ncert/12/13/2/3/defs.tex}
		\item Two balls are drawn at random with replacement from a box containing 10 black and 8 red balls. Find the probability that
		\label{ncert/12/13/2/12}
\begin{enumerate}
\item both balls are red.
\item first ball is black and second is red.
\item one of them is black and other is red.
\end{enumerate}

\item In a hostel, 60\% of the students read Hindi newspaper, 40\% read English newspaper and 20\% read both Hindi and English newspapers. A student is selected at random.
		\label{ncert/12/13/2/15}
\begin{enumerate}
\item Find the probability that she reads neither Hindi nor English newspapers.
\item If she reads Hindi newspaper, find the probability that she reads English newspaper.
\item If she reads English newspaper, find the probability that she reads Hindi newspaper.\\
\end{enumerate}
\item The probability of obtaining an even prime number on each die, when a pair of dice is rolled is 
\begin{enumerate}
    \item $0$ 
    
    \item $\frac{1}{3}$ 
    
    \item $\frac{1}{12}$ 
    
    \item $\frac{1}{36}$ 
\end{enumerate}
\solution
		%\begin{enumerate}[label=\thesection.\arabic*,ref=\thesection.\theenumi]
	\item One card is drawn from a well-shuffled deck of 52 cards. Find the probability of getting
\begin{enumerate}
\item A king of red colour 
\item A face card 
\item A red face card
\item The jack of hearts
\item A spade
\item The queen of diamonds

\end{enumerate}
\solution
		%\input{ncert/10/15/1/14/main.tex}
	\item Five cards—the ten, jack, queen, king and ace of diamonds, are well-shuffled with their face downwards. One card is then picked up at random.
\begin{enumerate}
\item
What is the probability that the card is the queen? 
\item
If the queen is drawn and put aside, what is the probability that the second card picked up is (a) an ace? (b) a queen?\\
\end{enumerate}
\solution
		%\input{ncert/10/15/1/15/defs.tex}
	\item A bag contains $5$ red balls and some blue balls. If the probability of drawing a blue ball is double that if a red ball, determine the number of blue balls in the bag. 
		\\
\solution
		%\input{ncert/10/15/2/3/defs.tex}
	\item A card is selected from a pack of 52 cards.
 \begin{enumerate}[label=(\alph*)] 
                 \item How many points are there in the sample space?
                 \item Calculate the probability that the card is an ace of spades.
                 \item Calculate the probability that the card is (i) an ace and (ii) black card.
 \end{enumerate}
\solution
		%\input{ncert/11/16/3/4/main.tex}
\item Four cards are drawn from a well-shuffled deck of 52 cards. What is the probability of obtaining 3 diamonds and one spade.
\\
\solution
		%\input{ncert/11/16/4/2/defs.tex}
\item In a certain lottery 10,000 tickets are sold and ten equal prizes are awarded. What is the probability of not getting a prize if you buy (a) one ticket (b) two tickets (c) 10 tickets ?	
\\
\solution
		%\input{ncert/11/16/4/4/defs.tex}
		%
\item 
Out of 100 students, two sections of 40 and 60 are formed. If you and your friend are among the 100 students, what is the probability that
\begin{enumerate}
\item you both enter the same section?
\item you both enter the different sections?
\end{enumerate}
\solution
		%\input{ncert/11/16/4/5/defs.tex}
	\item 
The number lock of a suitcase has 4 wheels each labelled with ten digits i.e. from 0 to 9.The lock opens with a sequence of four digits with no repeats.What is the probability of a person getting the right sequence to open the suitcase.
\\
\solution
		%\input{ncert/11/16/4/10/defs.tex}
		%
\item 
Two cards are drawn at random and without replacement from a pack of 52 playing cards. Find the probability that both the cards are black.
\\
\solution
		%\input{ncert/12/13/2/2/defs.tex}
		\item A box of oranges is inspected by examining three randomly selected oranges drawn without replacement. If all the three oranges are good, the box is approved for sale, otherwise, it is rejected. Find the probability that a box containing 15 oranges out of which 12 are good and 3 are bad ones will be approved for sale.
		\label{ncert/12/13/2/3/defs.tex}
		\item Two balls are drawn at random with replacement from a box containing 10 black and 8 red balls. Find the probability that
		\label{ncert/12/13/2/12}
\begin{enumerate}
\item both balls are red.
\item first ball is black and second is red.
\item one of them is black and other is red.
\end{enumerate}

\item In a hostel, 60\% of the students read Hindi newspaper, 40\% read English newspaper and 20\% read both Hindi and English newspapers. A student is selected at random.
		\label{ncert/12/13/2/15}
\begin{enumerate}
\item Find the probability that she reads neither Hindi nor English newspapers.
\item If she reads Hindi newspaper, find the probability that she reads English newspaper.
\item If she reads English newspaper, find the probability that she reads Hindi newspaper.\\
\end{enumerate}
\item The probability of obtaining an even prime number on each die, when a pair of dice is rolled is 
\begin{enumerate}
    \item $0$ 
    
    \item $\frac{1}{3}$ 
    
    \item $\frac{1}{12}$ 
    
    \item $\frac{1}{36}$ 
\end{enumerate}
\solution
		%\input{ncert/12/13/2/17/defs.tex}
	\item A bag contains 4 red and 4 black balls, another bag contains 2 red and 6 black balls. One of the two bags is selected at random and a ball is drawn from the bag which is found to be red. Find the probability that the ball is drawn from the first bag.
\\
\solution
		%\input{ncert/12/13/3/2/main.tex}
  \item
  Cards with numbers 2 to 101 are placed in a box. A card is selected at random.Find the probability that the card has
\begin{enumerate}[label=(\roman*)]
	\item an even number 
	\item a square number
\end{enumerate}
\solution
%\input{exemplar/10/13/3/32/main.tex}
\item
The king, queen and jack of clubs are removed from a deck of 52 playing cards and then well shuffled. Now one card is drawn at random from the remaining cards.  Determine the probability that the card is
\begin{enumerate}[label=(\roman*)]
\item a club
\item 10 of hearts
\end{enumerate}
\solution
%\input{exemplar/10/13/3/29/main.tex}
\item A team of medical students doing their internship have to assist during surgeries
at a city hospital. The probabilities of surgeries rated as very complex, complex,
routine, simple or very simple are respectively, 0.15, 0.20, 0.31, 0.26, .08. Find
the probabilities that a particular surgery will be rated
\begin{enumerate}
	\item complex or very complex;
	\item neither very complex nor very simple;
	\item routine or complex
	\item routine or simple
\end{enumerate}
\solution
%\input{exemplar/11/16/3/8(1)/main.tex}
\item A card is selected from a pack of 52 cards.
\begin{enumerate}[label=(\alph*)]
    \item How many points are there in the sample space?
    \item Calculate the probability that the card is an ace of spades.
    \item Calculate the probability that the card is (i) an ace and (ii) black card.
\end{enumerate}
\solution
%\input{exemplar/11/16/3/4/main2.tex}
\item The probability that a non leap year selected at random will contain 53 sundays.
\\
\solution
%\input{exemplar/10/13/1/19/main.tex}
\item One of the four persons John, Rita, Aslam or Gurpreet will be promoted next
month. Consequently the sample space consists of four elementary outcomes
S = {John promoted, Rita promoted, Aslam promoted, Gurpreet promoted}
You are told that the chances of John’s promotion is same as that of Gurpreet,
Rita’s chances of promotion are twice as likely as Johns. Aslam’s chances are
four times that of John.
\begin{enumerate}
	\item Determine
	\begin{enumerate}
		\item P (John promoted)
		\item P (Rita promoted)
		\item P (Aslam promoted)
		\item P (Gurpreet promoted)
	\end{enumerate}
	\item If A = {John promoted or Gurpreet promoted}, find P (A).
\end{enumerate}
\solution
%\input{exemplar/11/16/3/10/main.tex}
\item A card is drawn from a deck of 52 cards. Find the probability of getting a king or a heart or a red card.\\
\solution
%\input{exemplar/11/16/3/15/main.tex}
\item The probability that a student will pass his examination is 0.73, the probability of
the student getting a compartment is 0.13, and the probability that the student will
either pass or get compartment is 0.96. State True or False.\\
\solution
%\input{exemplar/11/16/3/31/main.tex}
\item A card is selected from a pack of 52 cards\\
\begin{enumerate}[label=(\alph*)]
\item How many points are there in the sample space?
\item Calculate the probability that the cards is an ace of spades.
\item Calculate the probability that the card is (i) an ace (ii)black card.\\
\end{enumerate}
%\input{ncert/11/16/3/4_1/Prob_4.tex}
\item In a non-leap year, the probability of having 53 tuesdays or 53 wednesdays is\\
\solution
%\input{exemplar/11/16/3/18/main.tex}
\item There are 1000 sealed envelopes in a box, 10 of them contain a cash prize of
Rs 100 each, 100 of them contain a cash prize of Rs 50 each and 200 of them
contain a cash prize of Rs 10 each and rest do not contain any cash prize. If they
are well shuffled and an envelope is picked up out, what is the probability that it
contains no cash prize?\\
\solution
%\input{exemplar/10/13/3/34/main.tex}
\item 
A die is thrown and a card is selected at random from a deck of 52 playing cards. The probability of getting an even number on the die and a spade card.\\
\solution
%\input{exemplar/12/13/3/78/main.tex}
\item
If 4-digit numbers greater than 5,000 are randomly formed from the digits 0, 1, 3, 5, and 7, what is the probability of forming a number divisible by 5 when:
\begin{enumerate}
    \item The digits are repeated?
    \item The repetition of digits is not allowed?
\end{enumerate}
\solution
%\input{ncert/11/16/4/9/main.tex}
\item Consider the probability space $\brak{\Omega, \mathcal{G}, P}$ where $\Omega = [0,2]$ and $\mathcal{G} = \cbrak{\phi, \Omega, [0,1], (1,2]}$. Let $X$ and $Y$ be two functions on $\Omega$ defined as
\begin{align*}
    X(\omega) = 
    \begin{cases}
        1 & \text{if }\omega \in [0, 1]\\
        2 & \text{if }\omega \in (1, 2]
    \end{cases}
\end{align*}
and
\begin{align*}
    Y(\omega) = 
    \begin{cases}
        2 & \text{if }\omega \in [0, 1.5]\\
        3 & \text{if }\omega \in (1.5, 2].
    \end{cases}
\end{align*}
Then which one of the following statements is true?
\begin{enumerate}
    \item [(A)] $X$ is a random variable with respect to $\mathcal{G}$, but $Y$ is not a random variable with respect to $\mathcal{G}$.
    \item [(B)] $Y$ is a random variable with respect to $\mathcal{G}$, but $X$ is not a random variable with respect to $\mathcal{G}$.
    \item [(C)] Neither $X$ nor $Y$ is a random variable with respect to $\mathcal{G}$.
    \item [(D)] Both $X$ and $Y$ are random variables with respect to $\mathcal{G}$.
\end{enumerate} \hfill (GATE ST 2023)\\
\solution
%\input{gate/ST/2023/14/main.tex}
	\item  A die is loaded in such a way that each odd number is twice as likely to occur as
each even number. Find $P(G)$, where $G$ is the event that a number greater than
3 occurs on a single roll of the die.
\\
\solution
		%\input{exemplar/11/16/3/5/main.tex}
	\item All the jacks, queens and kings are removed from a deck of 52 playing cards. The remaining cards are well shuffled and then one card is drawn at random. Giving ace a value 1 similar value for other cards, find the probability that the card has a value 
		\begin{enumerate}
			\item 7
			\item greater than 7
			\item less than 7
		\end{enumerate}
		%\input{exemplar/10/13/3/30/main.tex}
  \item A Lot consists of 48 mobile phones of which 42 are good, 3 have only minor defects and 3 have major defects.Varnika will buy a phone if it is good but the trader will only buy a mobile if it has no major defects. One phone is selected at random from the lot. What is the probability that it is
\begin{enumerate}
	\item acceptable to Varnika?
            \item acceptable to the trader?
\end{enumerate}
\solution
	%\input{exemplar/10/13/3/40/main.tex}
 \item A student says that if you throw a die, it will show up 1 or not 1. Therefore, the probability of getting 1 and the probability of getting 'not 1' each is equal to $\frac{1}{2}$. Is this correct? Give reasons.\\
 \solution
        %\input{exemplar/10/13/2/9/main.tex}
   \item Four candidates A, B, C, D have ap-
plied for the assignment to coach a school cricket
team. If A is twice as likely to be selected as B, and
B and C are given about the same chance of being
selected, while C is twice as likely to be selected
as D, what are the probabilities that
\begin{enumerate}
\item C will be selected?
\item A will not be selected?
\end{enumerate}
	%\input{exemplar/11/16/3/9/main.tex}
 \item A bag contain 24 balls of which $x$ balls are red, $2x$ are white and $3x$ are blue. A ball is selected at random, What is the probability that it is
\begin{enumerate}[label=\alph*)]
\item not red ?
\item white ?
\end{enumerate}
%\input{exemplar/10/13/3/41/main.tex}
If the letters of the word ASSASSINATION are arranged at random. Find the Probability that
\begin{enumerate}[label=(\alph*)]
\item Four $S's$ come consecutively in the word
\item Two  $I's$ and two $N's$ come together
\item All $A's$ are not coming together
\item No two $A's$ are coming together
\end{enumerate}
%\input{exemplar/11/16/3/14/main.tex}
	\item One urn contains two black balls (labelled B1 and B2) and one white ball. A
	second urn contains one black ball and two white balls (labelled W1 and W2).
	Suppose the following experiment is performed. One of the two urns is chosen
	at random. Next a ball is randomly chosen from the urn. Then a second ball is
	chosen at random from the same urn without replacing the first ball.
	
	\begin{enumerate}
	\item What is the probability that two black balls are chosen?
	
	\item What is the probability that two balls of opposite colour are chosen?
	\end{enumerate}
	\solution
	%\input{exemplar/11/16/3/12/main1.tex}
\end{enumerate}

	\item A bag contains 4 red and 4 black balls, another bag contains 2 red and 6 black balls. One of the two bags is selected at random and a ball is drawn from the bag which is found to be red. Find the probability that the ball is drawn from the first bag.
\\
\solution
		%\begin{table}[H]
	\centering
\begin{tabular}{|c|c|c|}
\hline
Random variable &Value &Definition\\ \hline
\multirow{3}{*}{X} &0 &Slips of Rs 1\\
&1 &Slips of Rs 5\\
&2 &Slips of Rs 13\\ \hline
\multirow{2}{*}{Y} &0 &Box A\\
&1 &Box B\\\hline
\end{tabular}
\caption{}
\label{tab:Distribution}
\end{table}
See \tabref{tab:Distribution}.
\begin{align}
p_{Y}\brak{k}= \begin{cases} 
      \frac{1}{3} & {k=0} \\
      \frac{2}{3 }& {k=1} 
   \end{cases}
   \\
p_{Y|X}\brak{0|0} = \frac{19}{25}\, 
p_{Y|X}\brak{0|1} = \frac{6}{25}\,
p_{Y|X}\brak{1|0} = \frac{45}{50}\,
p_{Y|X}\brak{1|2} = \frac{5}{50}
\end{align}
The desired probability is the probability that a slip drawn at random is marked other than Rs 1,
\begin{align}
&=1-p_X\brak{0}\\
&= p_X(1) + p_X(2)
\end{align}
Using Bayes theorem,
\begin{align}
&= p_Y\brak{0} \times \pr{Y=0 | X=1} + p_Y\brak{1} \times \pr{Y=1|X=2}\\
&=\frac{1}{3} \times \frac{6}{25} + \frac{2}{3} \times \frac{5}{50}\\
&=\frac{11}{75}
\end{align}

\newpage

%\tableofcontents

\bigskip

\renewcommand{\thefigure}{\theenumi}
\renewcommand{\thetable}{\theenumi}
%\renewcommand{\theequation}{\theenumi}

%\begin{abstract}
%%\boldmath
%In this letter, an algorithm for evaluating the exact analytical bit error rate  (BER)  for the piecewise linear (PL) combiner for  multiple relays is presented. Previous results were available only for upto three relays. The algorithm is unique in the sense that  the actual mathematical expressions, that are prohibitively large, need not be explicitly obtained. The diversity gain due to multiple relays is shown through plots of the analytical BER, well supported by simulations. 
%
%\end{abstract}
% IEEEtran.cls defaults to using nonbold math in the Abstract.
% This preserves the distinction between vectors and scalars. However,
% if the journal you are submitting to favors bold math in the abstract,
% then you can use LaTeX's standard command \boldmath at the very start
% of the abstract to achieve this. Many IEEE journals frown on math
% in the abstract anyway.

% Note that keywords are not normally used for peerreview papers.
%\begin{IEEEkeywords}
%Cooperative diversity, decode and forward, piecewise linear
%\end{IEEEkeywords}



% For peer review papers, you can put extra information on the cover
% page as needed:
% \ifCLASSOPTIONpeerreview
% \begin{center} \bfseries EDICS Category: 3-BBND \end{center}
% \fi
%
% For peerreview papers, this IEEEtran command inserts a page break and
% creates the second title. It will be ignored for other modes.
%\IEEEpeerreviewmaketitle




  \item
  Cards with numbers 2 to 101 are placed in a box. A card is selected at random.Find the probability that the card has
\begin{enumerate}[label=(\roman*)]
	\item an even number 
	\item a square number
\end{enumerate}
\solution
%\begin{table}[H]
	\centering
\begin{tabular}{|c|c|c|}
\hline
Random variable &Value &Definition\\ \hline
\multirow{3}{*}{X} &0 &Slips of Rs 1\\
&1 &Slips of Rs 5\\
&2 &Slips of Rs 13\\ \hline
\multirow{2}{*}{Y} &0 &Box A\\
&1 &Box B\\\hline
\end{tabular}
\caption{}
\label{tab:Distribution}
\end{table}
See \tabref{tab:Distribution}.
\begin{align}
p_{Y}\brak{k}= \begin{cases} 
      \frac{1}{3} & {k=0} \\
      \frac{2}{3 }& {k=1} 
   \end{cases}
   \\
p_{Y|X}\brak{0|0} = \frac{19}{25}\, 
p_{Y|X}\brak{0|1} = \frac{6}{25}\,
p_{Y|X}\brak{1|0} = \frac{45}{50}\,
p_{Y|X}\brak{1|2} = \frac{5}{50}
\end{align}
The desired probability is the probability that a slip drawn at random is marked other than Rs 1,
\begin{align}
&=1-p_X\brak{0}\\
&= p_X(1) + p_X(2)
\end{align}
Using Bayes theorem,
\begin{align}
&= p_Y\brak{0} \times \pr{Y=0 | X=1} + p_Y\brak{1} \times \pr{Y=1|X=2}\\
&=\frac{1}{3} \times \frac{6}{25} + \frac{2}{3} \times \frac{5}{50}\\
&=\frac{11}{75}
\end{align}

\newpage

%\tableofcontents

\bigskip

\renewcommand{\thefigure}{\theenumi}
\renewcommand{\thetable}{\theenumi}
%\renewcommand{\theequation}{\theenumi}

%\begin{abstract}
%%\boldmath
%In this letter, an algorithm for evaluating the exact analytical bit error rate  (BER)  for the piecewise linear (PL) combiner for  multiple relays is presented. Previous results were available only for upto three relays. The algorithm is unique in the sense that  the actual mathematical expressions, that are prohibitively large, need not be explicitly obtained. The diversity gain due to multiple relays is shown through plots of the analytical BER, well supported by simulations. 
%
%\end{abstract}
% IEEEtran.cls defaults to using nonbold math in the Abstract.
% This preserves the distinction between vectors and scalars. However,
% if the journal you are submitting to favors bold math in the abstract,
% then you can use LaTeX's standard command \boldmath at the very start
% of the abstract to achieve this. Many IEEE journals frown on math
% in the abstract anyway.

% Note that keywords are not normally used for peerreview papers.
%\begin{IEEEkeywords}
%Cooperative diversity, decode and forward, piecewise linear
%\end{IEEEkeywords}



% For peer review papers, you can put extra information on the cover
% page as needed:
% \ifCLASSOPTIONpeerreview
% \begin{center} \bfseries EDICS Category: 3-BBND \end{center}
% \fi
%
% For peerreview papers, this IEEEtran command inserts a page break and
% creates the second title. It will be ignored for other modes.
%\IEEEpeerreviewmaketitle




\item
The king, queen and jack of clubs are removed from a deck of 52 playing cards and then well shuffled. Now one card is drawn at random from the remaining cards.  Determine the probability that the card is
\begin{enumerate}[label=(\roman*)]
\item a club
\item 10 of hearts
\end{enumerate}
\solution
%\begin{table}[H]
	\centering
\begin{tabular}{|c|c|c|}
\hline
Random variable &Value &Definition\\ \hline
\multirow{3}{*}{X} &0 &Slips of Rs 1\\
&1 &Slips of Rs 5\\
&2 &Slips of Rs 13\\ \hline
\multirow{2}{*}{Y} &0 &Box A\\
&1 &Box B\\\hline
\end{tabular}
\caption{}
\label{tab:Distribution}
\end{table}
See \tabref{tab:Distribution}.
\begin{align}
p_{Y}\brak{k}= \begin{cases} 
      \frac{1}{3} & {k=0} \\
      \frac{2}{3 }& {k=1} 
   \end{cases}
   \\
p_{Y|X}\brak{0|0} = \frac{19}{25}\, 
p_{Y|X}\brak{0|1} = \frac{6}{25}\,
p_{Y|X}\brak{1|0} = \frac{45}{50}\,
p_{Y|X}\brak{1|2} = \frac{5}{50}
\end{align}
The desired probability is the probability that a slip drawn at random is marked other than Rs 1,
\begin{align}
&=1-p_X\brak{0}\\
&= p_X(1) + p_X(2)
\end{align}
Using Bayes theorem,
\begin{align}
&= p_Y\brak{0} \times \pr{Y=0 | X=1} + p_Y\brak{1} \times \pr{Y=1|X=2}\\
&=\frac{1}{3} \times \frac{6}{25} + \frac{2}{3} \times \frac{5}{50}\\
&=\frac{11}{75}
\end{align}

\newpage

%\tableofcontents

\bigskip

\renewcommand{\thefigure}{\theenumi}
\renewcommand{\thetable}{\theenumi}
%\renewcommand{\theequation}{\theenumi}

%\begin{abstract}
%%\boldmath
%In this letter, an algorithm for evaluating the exact analytical bit error rate  (BER)  for the piecewise linear (PL) combiner for  multiple relays is presented. Previous results were available only for upto three relays. The algorithm is unique in the sense that  the actual mathematical expressions, that are prohibitively large, need not be explicitly obtained. The diversity gain due to multiple relays is shown through plots of the analytical BER, well supported by simulations. 
%
%\end{abstract}
% IEEEtran.cls defaults to using nonbold math in the Abstract.
% This preserves the distinction between vectors and scalars. However,
% if the journal you are submitting to favors bold math in the abstract,
% then you can use LaTeX's standard command \boldmath at the very start
% of the abstract to achieve this. Many IEEE journals frown on math
% in the abstract anyway.

% Note that keywords are not normally used for peerreview papers.
%\begin{IEEEkeywords}
%Cooperative diversity, decode and forward, piecewise linear
%\end{IEEEkeywords}



% For peer review papers, you can put extra information on the cover
% page as needed:
% \ifCLASSOPTIONpeerreview
% \begin{center} \bfseries EDICS Category: 3-BBND \end{center}
% \fi
%
% For peerreview papers, this IEEEtran command inserts a page break and
% creates the second title. It will be ignored for other modes.
%\IEEEpeerreviewmaketitle




\item A team of medical students doing their internship have to assist during surgeries
at a city hospital. The probabilities of surgeries rated as very complex, complex,
routine, simple or very simple are respectively, 0.15, 0.20, 0.31, 0.26, .08. Find
the probabilities that a particular surgery will be rated
\begin{enumerate}
	\item complex or very complex;
	\item neither very complex nor very simple;
	\item routine or complex
	\item routine or simple
\end{enumerate}
\solution
%\begin{table}[H]
	\centering
\begin{tabular}{|c|c|c|}
\hline
Random variable &Value &Definition\\ \hline
\multirow{3}{*}{X} &0 &Slips of Rs 1\\
&1 &Slips of Rs 5\\
&2 &Slips of Rs 13\\ \hline
\multirow{2}{*}{Y} &0 &Box A\\
&1 &Box B\\\hline
\end{tabular}
\caption{}
\label{tab:Distribution}
\end{table}
See \tabref{tab:Distribution}.
\begin{align}
p_{Y}\brak{k}= \begin{cases} 
      \frac{1}{3} & {k=0} \\
      \frac{2}{3 }& {k=1} 
   \end{cases}
   \\
p_{Y|X}\brak{0|0} = \frac{19}{25}\, 
p_{Y|X}\brak{0|1} = \frac{6}{25}\,
p_{Y|X}\brak{1|0} = \frac{45}{50}\,
p_{Y|X}\brak{1|2} = \frac{5}{50}
\end{align}
The desired probability is the probability that a slip drawn at random is marked other than Rs 1,
\begin{align}
&=1-p_X\brak{0}\\
&= p_X(1) + p_X(2)
\end{align}
Using Bayes theorem,
\begin{align}
&= p_Y\brak{0} \times \pr{Y=0 | X=1} + p_Y\brak{1} \times \pr{Y=1|X=2}\\
&=\frac{1}{3} \times \frac{6}{25} + \frac{2}{3} \times \frac{5}{50}\\
&=\frac{11}{75}
\end{align}

\newpage

%\tableofcontents

\bigskip

\renewcommand{\thefigure}{\theenumi}
\renewcommand{\thetable}{\theenumi}
%\renewcommand{\theequation}{\theenumi}

%\begin{abstract}
%%\boldmath
%In this letter, an algorithm for evaluating the exact analytical bit error rate  (BER)  for the piecewise linear (PL) combiner for  multiple relays is presented. Previous results were available only for upto three relays. The algorithm is unique in the sense that  the actual mathematical expressions, that are prohibitively large, need not be explicitly obtained. The diversity gain due to multiple relays is shown through plots of the analytical BER, well supported by simulations. 
%
%\end{abstract}
% IEEEtran.cls defaults to using nonbold math in the Abstract.
% This preserves the distinction between vectors and scalars. However,
% if the journal you are submitting to favors bold math in the abstract,
% then you can use LaTeX's standard command \boldmath at the very start
% of the abstract to achieve this. Many IEEE journals frown on math
% in the abstract anyway.

% Note that keywords are not normally used for peerreview papers.
%\begin{IEEEkeywords}
%Cooperative diversity, decode and forward, piecewise linear
%\end{IEEEkeywords}



% For peer review papers, you can put extra information on the cover
% page as needed:
% \ifCLASSOPTIONpeerreview
% \begin{center} \bfseries EDICS Category: 3-BBND \end{center}
% \fi
%
% For peerreview papers, this IEEEtran command inserts a page break and
% creates the second title. It will be ignored for other modes.
%\IEEEpeerreviewmaketitle




\item A card is selected from a pack of 52 cards.
\begin{enumerate}[label=(\alph*)]
    \item How many points are there in the sample space?
    \item Calculate the probability that the card is an ace of spades.
    \item Calculate the probability that the card is (i) an ace and (ii) black card.
\end{enumerate}
\solution
%Let $X$ be an bernoulli rv defined as in \tabref{tab:exemplar/11/16/3/26}.  Then, 
\begin{equation}
    p =
        \frac{4}{11} 
\end{equation}
\begin{table}[H]
	\centering
	\input{exemplar/11/16/3/26/tables/Table2.tex}
	\caption{}
        \label{tab:exemplar/11/16/3/26}
\end{table}

\item The probability that a non leap year selected at random will contain 53 sundays.
\\
\solution
%\begin{table}[H]
	\centering
\begin{tabular}{|c|c|c|}
\hline
Random variable &Value &Definition\\ \hline
\multirow{3}{*}{X} &0 &Slips of Rs 1\\
&1 &Slips of Rs 5\\
&2 &Slips of Rs 13\\ \hline
\multirow{2}{*}{Y} &0 &Box A\\
&1 &Box B\\\hline
\end{tabular}
\caption{}
\label{tab:Distribution}
\end{table}
See \tabref{tab:Distribution}.
\begin{align}
p_{Y}\brak{k}= \begin{cases} 
      \frac{1}{3} & {k=0} \\
      \frac{2}{3 }& {k=1} 
   \end{cases}
   \\
p_{Y|X}\brak{0|0} = \frac{19}{25}\, 
p_{Y|X}\brak{0|1} = \frac{6}{25}\,
p_{Y|X}\brak{1|0} = \frac{45}{50}\,
p_{Y|X}\brak{1|2} = \frac{5}{50}
\end{align}
The desired probability is the probability that a slip drawn at random is marked other than Rs 1,
\begin{align}
&=1-p_X\brak{0}\\
&= p_X(1) + p_X(2)
\end{align}
Using Bayes theorem,
\begin{align}
&= p_Y\brak{0} \times \pr{Y=0 | X=1} + p_Y\brak{1} \times \pr{Y=1|X=2}\\
&=\frac{1}{3} \times \frac{6}{25} + \frac{2}{3} \times \frac{5}{50}\\
&=\frac{11}{75}
\end{align}

\newpage

%\tableofcontents

\bigskip

\renewcommand{\thefigure}{\theenumi}
\renewcommand{\thetable}{\theenumi}
%\renewcommand{\theequation}{\theenumi}

%\begin{abstract}
%%\boldmath
%In this letter, an algorithm for evaluating the exact analytical bit error rate  (BER)  for the piecewise linear (PL) combiner for  multiple relays is presented. Previous results were available only for upto three relays. The algorithm is unique in the sense that  the actual mathematical expressions, that are prohibitively large, need not be explicitly obtained. The diversity gain due to multiple relays is shown through plots of the analytical BER, well supported by simulations. 
%
%\end{abstract}
% IEEEtran.cls defaults to using nonbold math in the Abstract.
% This preserves the distinction between vectors and scalars. However,
% if the journal you are submitting to favors bold math in the abstract,
% then you can use LaTeX's standard command \boldmath at the very start
% of the abstract to achieve this. Many IEEE journals frown on math
% in the abstract anyway.

% Note that keywords are not normally used for peerreview papers.
%\begin{IEEEkeywords}
%Cooperative diversity, decode and forward, piecewise linear
%\end{IEEEkeywords}



% For peer review papers, you can put extra information on the cover
% page as needed:
% \ifCLASSOPTIONpeerreview
% \begin{center} \bfseries EDICS Category: 3-BBND \end{center}
% \fi
%
% For peerreview papers, this IEEEtran command inserts a page break and
% creates the second title. It will be ignored for other modes.
%\IEEEpeerreviewmaketitle




\item One of the four persons John, Rita, Aslam or Gurpreet will be promoted next
month. Consequently the sample space consists of four elementary outcomes
S = {John promoted, Rita promoted, Aslam promoted, Gurpreet promoted}
You are told that the chances of John’s promotion is same as that of Gurpreet,
Rita’s chances of promotion are twice as likely as Johns. Aslam’s chances are
four times that of John.
\begin{enumerate}
	\item Determine
	\begin{enumerate}
		\item P (John promoted)
		\item P (Rita promoted)
		\item P (Aslam promoted)
		\item P (Gurpreet promoted)
	\end{enumerate}
	\item If A = {John promoted or Gurpreet promoted}, find P (A).
\end{enumerate}
\solution
%\begin{table}[H]
	\centering
\begin{tabular}{|c|c|c|}
\hline
Random variable &Value &Definition\\ \hline
\multirow{3}{*}{X} &0 &Slips of Rs 1\\
&1 &Slips of Rs 5\\
&2 &Slips of Rs 13\\ \hline
\multirow{2}{*}{Y} &0 &Box A\\
&1 &Box B\\\hline
\end{tabular}
\caption{}
\label{tab:Distribution}
\end{table}
See \tabref{tab:Distribution}.
\begin{align}
p_{Y}\brak{k}= \begin{cases} 
      \frac{1}{3} & {k=0} \\
      \frac{2}{3 }& {k=1} 
   \end{cases}
   \\
p_{Y|X}\brak{0|0} = \frac{19}{25}\, 
p_{Y|X}\brak{0|1} = \frac{6}{25}\,
p_{Y|X}\brak{1|0} = \frac{45}{50}\,
p_{Y|X}\brak{1|2} = \frac{5}{50}
\end{align}
The desired probability is the probability that a slip drawn at random is marked other than Rs 1,
\begin{align}
&=1-p_X\brak{0}\\
&= p_X(1) + p_X(2)
\end{align}
Using Bayes theorem,
\begin{align}
&= p_Y\brak{0} \times \pr{Y=0 | X=1} + p_Y\brak{1} \times \pr{Y=1|X=2}\\
&=\frac{1}{3} \times \frac{6}{25} + \frac{2}{3} \times \frac{5}{50}\\
&=\frac{11}{75}
\end{align}

\newpage

%\tableofcontents

\bigskip

\renewcommand{\thefigure}{\theenumi}
\renewcommand{\thetable}{\theenumi}
%\renewcommand{\theequation}{\theenumi}

%\begin{abstract}
%%\boldmath
%In this letter, an algorithm for evaluating the exact analytical bit error rate  (BER)  for the piecewise linear (PL) combiner for  multiple relays is presented. Previous results were available only for upto three relays. The algorithm is unique in the sense that  the actual mathematical expressions, that are prohibitively large, need not be explicitly obtained. The diversity gain due to multiple relays is shown through plots of the analytical BER, well supported by simulations. 
%
%\end{abstract}
% IEEEtran.cls defaults to using nonbold math in the Abstract.
% This preserves the distinction between vectors and scalars. However,
% if the journal you are submitting to favors bold math in the abstract,
% then you can use LaTeX's standard command \boldmath at the very start
% of the abstract to achieve this. Many IEEE journals frown on math
% in the abstract anyway.

% Note that keywords are not normally used for peerreview papers.
%\begin{IEEEkeywords}
%Cooperative diversity, decode and forward, piecewise linear
%\end{IEEEkeywords}



% For peer review papers, you can put extra information on the cover
% page as needed:
% \ifCLASSOPTIONpeerreview
% \begin{center} \bfseries EDICS Category: 3-BBND \end{center}
% \fi
%
% For peerreview papers, this IEEEtran command inserts a page break and
% creates the second title. It will be ignored for other modes.
%\IEEEpeerreviewmaketitle




\item A card is drawn from a deck of 52 cards. Find the probability of getting a king or a heart or a red card.\\
\solution
%\begin{table}[H]
	\centering
\begin{tabular}{|c|c|c|}
\hline
Random variable &Value &Definition\\ \hline
\multirow{3}{*}{X} &0 &Slips of Rs 1\\
&1 &Slips of Rs 5\\
&2 &Slips of Rs 13\\ \hline
\multirow{2}{*}{Y} &0 &Box A\\
&1 &Box B\\\hline
\end{tabular}
\caption{}
\label{tab:Distribution}
\end{table}
See \tabref{tab:Distribution}.
\begin{align}
p_{Y}\brak{k}= \begin{cases} 
      \frac{1}{3} & {k=0} \\
      \frac{2}{3 }& {k=1} 
   \end{cases}
   \\
p_{Y|X}\brak{0|0} = \frac{19}{25}\, 
p_{Y|X}\brak{0|1} = \frac{6}{25}\,
p_{Y|X}\brak{1|0} = \frac{45}{50}\,
p_{Y|X}\brak{1|2} = \frac{5}{50}
\end{align}
The desired probability is the probability that a slip drawn at random is marked other than Rs 1,
\begin{align}
&=1-p_X\brak{0}\\
&= p_X(1) + p_X(2)
\end{align}
Using Bayes theorem,
\begin{align}
&= p_Y\brak{0} \times \pr{Y=0 | X=1} + p_Y\brak{1} \times \pr{Y=1|X=2}\\
&=\frac{1}{3} \times \frac{6}{25} + \frac{2}{3} \times \frac{5}{50}\\
&=\frac{11}{75}
\end{align}

\newpage

%\tableofcontents

\bigskip

\renewcommand{\thefigure}{\theenumi}
\renewcommand{\thetable}{\theenumi}
%\renewcommand{\theequation}{\theenumi}

%\begin{abstract}
%%\boldmath
%In this letter, an algorithm for evaluating the exact analytical bit error rate  (BER)  for the piecewise linear (PL) combiner for  multiple relays is presented. Previous results were available only for upto three relays. The algorithm is unique in the sense that  the actual mathematical expressions, that are prohibitively large, need not be explicitly obtained. The diversity gain due to multiple relays is shown through plots of the analytical BER, well supported by simulations. 
%
%\end{abstract}
% IEEEtran.cls defaults to using nonbold math in the Abstract.
% This preserves the distinction between vectors and scalars. However,
% if the journal you are submitting to favors bold math in the abstract,
% then you can use LaTeX's standard command \boldmath at the very start
% of the abstract to achieve this. Many IEEE journals frown on math
% in the abstract anyway.

% Note that keywords are not normally used for peerreview papers.
%\begin{IEEEkeywords}
%Cooperative diversity, decode and forward, piecewise linear
%\end{IEEEkeywords}



% For peer review papers, you can put extra information on the cover
% page as needed:
% \ifCLASSOPTIONpeerreview
% \begin{center} \bfseries EDICS Category: 3-BBND \end{center}
% \fi
%
% For peerreview papers, this IEEEtran command inserts a page break and
% creates the second title. It will be ignored for other modes.
%\IEEEpeerreviewmaketitle




\item The probability that a student will pass his examination is 0.73, the probability of
the student getting a compartment is 0.13, and the probability that the student will
either pass or get compartment is 0.96. State True or False.\\
\solution
%\begin{table}[H]
	\centering
\begin{tabular}{|c|c|c|}
\hline
Random variable &Value &Definition\\ \hline
\multirow{3}{*}{X} &0 &Slips of Rs 1\\
&1 &Slips of Rs 5\\
&2 &Slips of Rs 13\\ \hline
\multirow{2}{*}{Y} &0 &Box A\\
&1 &Box B\\\hline
\end{tabular}
\caption{}
\label{tab:Distribution}
\end{table}
See \tabref{tab:Distribution}.
\begin{align}
p_{Y}\brak{k}= \begin{cases} 
      \frac{1}{3} & {k=0} \\
      \frac{2}{3 }& {k=1} 
   \end{cases}
   \\
p_{Y|X}\brak{0|0} = \frac{19}{25}\, 
p_{Y|X}\brak{0|1} = \frac{6}{25}\,
p_{Y|X}\brak{1|0} = \frac{45}{50}\,
p_{Y|X}\brak{1|2} = \frac{5}{50}
\end{align}
The desired probability is the probability that a slip drawn at random is marked other than Rs 1,
\begin{align}
&=1-p_X\brak{0}\\
&= p_X(1) + p_X(2)
\end{align}
Using Bayes theorem,
\begin{align}
&= p_Y\brak{0} \times \pr{Y=0 | X=1} + p_Y\brak{1} \times \pr{Y=1|X=2}\\
&=\frac{1}{3} \times \frac{6}{25} + \frac{2}{3} \times \frac{5}{50}\\
&=\frac{11}{75}
\end{align}

\newpage

%\tableofcontents

\bigskip

\renewcommand{\thefigure}{\theenumi}
\renewcommand{\thetable}{\theenumi}
%\renewcommand{\theequation}{\theenumi}

%\begin{abstract}
%%\boldmath
%In this letter, an algorithm for evaluating the exact analytical bit error rate  (BER)  for the piecewise linear (PL) combiner for  multiple relays is presented. Previous results were available only for upto three relays. The algorithm is unique in the sense that  the actual mathematical expressions, that are prohibitively large, need not be explicitly obtained. The diversity gain due to multiple relays is shown through plots of the analytical BER, well supported by simulations. 
%
%\end{abstract}
% IEEEtran.cls defaults to using nonbold math in the Abstract.
% This preserves the distinction between vectors and scalars. However,
% if the journal you are submitting to favors bold math in the abstract,
% then you can use LaTeX's standard command \boldmath at the very start
% of the abstract to achieve this. Many IEEE journals frown on math
% in the abstract anyway.

% Note that keywords are not normally used for peerreview papers.
%\begin{IEEEkeywords}
%Cooperative diversity, decode and forward, piecewise linear
%\end{IEEEkeywords}



% For peer review papers, you can put extra information on the cover
% page as needed:
% \ifCLASSOPTIONpeerreview
% \begin{center} \bfseries EDICS Category: 3-BBND \end{center}
% \fi
%
% For peerreview papers, this IEEEtran command inserts a page break and
% creates the second title. It will be ignored for other modes.
%\IEEEpeerreviewmaketitle




\item A card is selected from a pack of 52 cards\\
\begin{enumerate}[label=(\alph*)]
\item How many points are there in the sample space?
\item Calculate the probability that the cards is an ace of spades.
\item Calculate the probability that the card is (i) an ace (ii)black card.\\
\end{enumerate}
%\input{ncert/11/16/3/4_1/Prob_4.tex}
\item In a non-leap year, the probability of having 53 tuesdays or 53 wednesdays is\\
\solution
%A non-leap year has a total of 365 days, and a week has 7 days.\\
So it can be expressed as 
\begin{align}
365\text{days} &=52\times 7+1 \text{day}
\end{align}
$\implies$ 52 tuesdays or wednesdays\\
Random variable X denotes the days of a week
\begin{align}
p_X\brak{k}&=\frac{1}{7}; \quad \brak{1<k<7}
\end{align}
So the probability of extra day being tuesday or wednesday is
\begin{align}
p_X\brak{3}+p_X\brak{4}&=\frac{1}{7}+\frac{1}{7}=\frac{2}{7}
\end{align}



\item There are 1000 sealed envelopes in a box, 10 of them contain a cash prize of
Rs 100 each, 100 of them contain a cash prize of Rs 50 each and 200 of them
contain a cash prize of Rs 10 each and rest do not contain any cash prize. If they
are well shuffled and an envelope is picked up out, what is the probability that it
contains no cash prize?\\
\solution
%\begin{table}[H]
	\centering
\begin{tabular}{|c|c|c|}
\hline
Random variable &Value &Definition\\ \hline
\multirow{3}{*}{X} &0 &Slips of Rs 1\\
&1 &Slips of Rs 5\\
&2 &Slips of Rs 13\\ \hline
\multirow{2}{*}{Y} &0 &Box A\\
&1 &Box B\\\hline
\end{tabular}
\caption{}
\label{tab:Distribution}
\end{table}
See \tabref{tab:Distribution}.
\begin{align}
p_{Y}\brak{k}= \begin{cases} 
      \frac{1}{3} & {k=0} \\
      \frac{2}{3 }& {k=1} 
   \end{cases}
   \\
p_{Y|X}\brak{0|0} = \frac{19}{25}\, 
p_{Y|X}\brak{0|1} = \frac{6}{25}\,
p_{Y|X}\brak{1|0} = \frac{45}{50}\,
p_{Y|X}\brak{1|2} = \frac{5}{50}
\end{align}
The desired probability is the probability that a slip drawn at random is marked other than Rs 1,
\begin{align}
&=1-p_X\brak{0}\\
&= p_X(1) + p_X(2)
\end{align}
Using Bayes theorem,
\begin{align}
&= p_Y\brak{0} \times \pr{Y=0 | X=1} + p_Y\brak{1} \times \pr{Y=1|X=2}\\
&=\frac{1}{3} \times \frac{6}{25} + \frac{2}{3} \times \frac{5}{50}\\
&=\frac{11}{75}
\end{align}

\newpage

%\tableofcontents

\bigskip

\renewcommand{\thefigure}{\theenumi}
\renewcommand{\thetable}{\theenumi}
%\renewcommand{\theequation}{\theenumi}

%\begin{abstract}
%%\boldmath
%In this letter, an algorithm for evaluating the exact analytical bit error rate  (BER)  for the piecewise linear (PL) combiner for  multiple relays is presented. Previous results were available only for upto three relays. The algorithm is unique in the sense that  the actual mathematical expressions, that are prohibitively large, need not be explicitly obtained. The diversity gain due to multiple relays is shown through plots of the analytical BER, well supported by simulations. 
%
%\end{abstract}
% IEEEtran.cls defaults to using nonbold math in the Abstract.
% This preserves the distinction between vectors and scalars. However,
% if the journal you are submitting to favors bold math in the abstract,
% then you can use LaTeX's standard command \boldmath at the very start
% of the abstract to achieve this. Many IEEE journals frown on math
% in the abstract anyway.

% Note that keywords are not normally used for peerreview papers.
%\begin{IEEEkeywords}
%Cooperative diversity, decode and forward, piecewise linear
%\end{IEEEkeywords}



% For peer review papers, you can put extra information on the cover
% page as needed:
% \ifCLASSOPTIONpeerreview
% \begin{center} \bfseries EDICS Category: 3-BBND \end{center}
% \fi
%
% For peerreview papers, this IEEEtran command inserts a page break and
% creates the second title. It will be ignored for other modes.
%\IEEEpeerreviewmaketitle




\item 
A die is thrown and a card is selected at random from a deck of 52 playing cards. The probability of getting an even number on the die and a spade card.\\
\solution
%\begin{table}[H]
	\centering
\begin{tabular}{|c|c|c|}
\hline
Random variable &Value &Definition\\ \hline
\multirow{3}{*}{X} &0 &Slips of Rs 1\\
&1 &Slips of Rs 5\\
&2 &Slips of Rs 13\\ \hline
\multirow{2}{*}{Y} &0 &Box A\\
&1 &Box B\\\hline
\end{tabular}
\caption{}
\label{tab:Distribution}
\end{table}
See \tabref{tab:Distribution}.
\begin{align}
p_{Y}\brak{k}= \begin{cases} 
      \frac{1}{3} & {k=0} \\
      \frac{2}{3 }& {k=1} 
   \end{cases}
   \\
p_{Y|X}\brak{0|0} = \frac{19}{25}\, 
p_{Y|X}\brak{0|1} = \frac{6}{25}\,
p_{Y|X}\brak{1|0} = \frac{45}{50}\,
p_{Y|X}\brak{1|2} = \frac{5}{50}
\end{align}
The desired probability is the probability that a slip drawn at random is marked other than Rs 1,
\begin{align}
&=1-p_X\brak{0}\\
&= p_X(1) + p_X(2)
\end{align}
Using Bayes theorem,
\begin{align}
&= p_Y\brak{0} \times \pr{Y=0 | X=1} + p_Y\brak{1} \times \pr{Y=1|X=2}\\
&=\frac{1}{3} \times \frac{6}{25} + \frac{2}{3} \times \frac{5}{50}\\
&=\frac{11}{75}
\end{align}

\newpage

%\tableofcontents

\bigskip

\renewcommand{\thefigure}{\theenumi}
\renewcommand{\thetable}{\theenumi}
%\renewcommand{\theequation}{\theenumi}

%\begin{abstract}
%%\boldmath
%In this letter, an algorithm for evaluating the exact analytical bit error rate  (BER)  for the piecewise linear (PL) combiner for  multiple relays is presented. Previous results were available only for upto three relays. The algorithm is unique in the sense that  the actual mathematical expressions, that are prohibitively large, need not be explicitly obtained. The diversity gain due to multiple relays is shown through plots of the analytical BER, well supported by simulations. 
%
%\end{abstract}
% IEEEtran.cls defaults to using nonbold math in the Abstract.
% This preserves the distinction between vectors and scalars. However,
% if the journal you are submitting to favors bold math in the abstract,
% then you can use LaTeX's standard command \boldmath at the very start
% of the abstract to achieve this. Many IEEE journals frown on math
% in the abstract anyway.

% Note that keywords are not normally used for peerreview papers.
%\begin{IEEEkeywords}
%Cooperative diversity, decode and forward, piecewise linear
%\end{IEEEkeywords}



% For peer review papers, you can put extra information on the cover
% page as needed:
% \ifCLASSOPTIONpeerreview
% \begin{center} \bfseries EDICS Category: 3-BBND \end{center}
% \fi
%
% For peerreview papers, this IEEEtran command inserts a page break and
% creates the second title. It will be ignored for other modes.
%\IEEEpeerreviewmaketitle




\item
If 4-digit numbers greater than 5,000 are randomly formed from the digits 0, 1, 3, 5, and 7, what is the probability of forming a number divisible by 5 when:
\begin{enumerate}
    \item The digits are repeated?
    \item The repetition of digits is not allowed?
\end{enumerate}
\solution
%\begin{table}[H]
	\centering
\begin{tabular}{|c|c|c|}
\hline
Random variable &Value &Definition\\ \hline
\multirow{3}{*}{X} &0 &Slips of Rs 1\\
&1 &Slips of Rs 5\\
&2 &Slips of Rs 13\\ \hline
\multirow{2}{*}{Y} &0 &Box A\\
&1 &Box B\\\hline
\end{tabular}
\caption{}
\label{tab:Distribution}
\end{table}
See \tabref{tab:Distribution}.
\begin{align}
p_{Y}\brak{k}= \begin{cases} 
      \frac{1}{3} & {k=0} \\
      \frac{2}{3 }& {k=1} 
   \end{cases}
   \\
p_{Y|X}\brak{0|0} = \frac{19}{25}\, 
p_{Y|X}\brak{0|1} = \frac{6}{25}\,
p_{Y|X}\brak{1|0} = \frac{45}{50}\,
p_{Y|X}\brak{1|2} = \frac{5}{50}
\end{align}
The desired probability is the probability that a slip drawn at random is marked other than Rs 1,
\begin{align}
&=1-p_X\brak{0}\\
&= p_X(1) + p_X(2)
\end{align}
Using Bayes theorem,
\begin{align}
&= p_Y\brak{0} \times \pr{Y=0 | X=1} + p_Y\brak{1} \times \pr{Y=1|X=2}\\
&=\frac{1}{3} \times \frac{6}{25} + \frac{2}{3} \times \frac{5}{50}\\
&=\frac{11}{75}
\end{align}

\newpage

%\tableofcontents

\bigskip

\renewcommand{\thefigure}{\theenumi}
\renewcommand{\thetable}{\theenumi}
%\renewcommand{\theequation}{\theenumi}

%\begin{abstract}
%%\boldmath
%In this letter, an algorithm for evaluating the exact analytical bit error rate  (BER)  for the piecewise linear (PL) combiner for  multiple relays is presented. Previous results were available only for upto three relays. The algorithm is unique in the sense that  the actual mathematical expressions, that are prohibitively large, need not be explicitly obtained. The diversity gain due to multiple relays is shown through plots of the analytical BER, well supported by simulations. 
%
%\end{abstract}
% IEEEtran.cls defaults to using nonbold math in the Abstract.
% This preserves the distinction between vectors and scalars. However,
% if the journal you are submitting to favors bold math in the abstract,
% then you can use LaTeX's standard command \boldmath at the very start
% of the abstract to achieve this. Many IEEE journals frown on math
% in the abstract anyway.

% Note that keywords are not normally used for peerreview papers.
%\begin{IEEEkeywords}
%Cooperative diversity, decode and forward, piecewise linear
%\end{IEEEkeywords}



% For peer review papers, you can put extra information on the cover
% page as needed:
% \ifCLASSOPTIONpeerreview
% \begin{center} \bfseries EDICS Category: 3-BBND \end{center}
% \fi
%
% For peerreview papers, this IEEEtran command inserts a page break and
% creates the second title. It will be ignored for other modes.
%\IEEEpeerreviewmaketitle




\item Consider the probability space $\brak{\Omega, \mathcal{G}, P}$ where $\Omega = [0,2]$ and $\mathcal{G} = \cbrak{\phi, \Omega, [0,1], (1,2]}$. Let $X$ and $Y$ be two functions on $\Omega$ defined as
\begin{align*}
    X(\omega) = 
    \begin{cases}
        1 & \text{if }\omega \in [0, 1]\\
        2 & \text{if }\omega \in (1, 2]
    \end{cases}
\end{align*}
and
\begin{align*}
    Y(\omega) = 
    \begin{cases}
        2 & \text{if }\omega \in [0, 1.5]\\
        3 & \text{if }\omega \in (1.5, 2].
    \end{cases}
\end{align*}
Then which one of the following statements is true?
\begin{enumerate}
    \item [(A)] $X$ is a random variable with respect to $\mathcal{G}$, but $Y$ is not a random variable with respect to $\mathcal{G}$.
    \item [(B)] $Y$ is a random variable with respect to $\mathcal{G}$, but $X$ is not a random variable with respect to $\mathcal{G}$.
    \item [(C)] Neither $X$ nor $Y$ is a random variable with respect to $\mathcal{G}$.
    \item [(D)] Both $X$ and $Y$ are random variables with respect to $\mathcal{G}$.
\end{enumerate} \hfill (GATE ST 2023)\\
\solution
%\begin{table}[H]
	\centering
\begin{tabular}{|c|c|c|}
\hline
Random variable &Value &Definition\\ \hline
\multirow{3}{*}{X} &0 &Slips of Rs 1\\
&1 &Slips of Rs 5\\
&2 &Slips of Rs 13\\ \hline
\multirow{2}{*}{Y} &0 &Box A\\
&1 &Box B\\\hline
\end{tabular}
\caption{}
\label{tab:Distribution}
\end{table}
See \tabref{tab:Distribution}.
\begin{align}
p_{Y}\brak{k}= \begin{cases} 
      \frac{1}{3} & {k=0} \\
      \frac{2}{3 }& {k=1} 
   \end{cases}
   \\
p_{Y|X}\brak{0|0} = \frac{19}{25}\, 
p_{Y|X}\brak{0|1} = \frac{6}{25}\,
p_{Y|X}\brak{1|0} = \frac{45}{50}\,
p_{Y|X}\brak{1|2} = \frac{5}{50}
\end{align}
The desired probability is the probability that a slip drawn at random is marked other than Rs 1,
\begin{align}
&=1-p_X\brak{0}\\
&= p_X(1) + p_X(2)
\end{align}
Using Bayes theorem,
\begin{align}
&= p_Y\brak{0} \times \pr{Y=0 | X=1} + p_Y\brak{1} \times \pr{Y=1|X=2}\\
&=\frac{1}{3} \times \frac{6}{25} + \frac{2}{3} \times \frac{5}{50}\\
&=\frac{11}{75}
\end{align}

\newpage

%\tableofcontents

\bigskip

\renewcommand{\thefigure}{\theenumi}
\renewcommand{\thetable}{\theenumi}
%\renewcommand{\theequation}{\theenumi}

%\begin{abstract}
%%\boldmath
%In this letter, an algorithm for evaluating the exact analytical bit error rate  (BER)  for the piecewise linear (PL) combiner for  multiple relays is presented. Previous results were available only for upto three relays. The algorithm is unique in the sense that  the actual mathematical expressions, that are prohibitively large, need not be explicitly obtained. The diversity gain due to multiple relays is shown through plots of the analytical BER, well supported by simulations. 
%
%\end{abstract}
% IEEEtran.cls defaults to using nonbold math in the Abstract.
% This preserves the distinction between vectors and scalars. However,
% if the journal you are submitting to favors bold math in the abstract,
% then you can use LaTeX's standard command \boldmath at the very start
% of the abstract to achieve this. Many IEEE journals frown on math
% in the abstract anyway.

% Note that keywords are not normally used for peerreview papers.
%\begin{IEEEkeywords}
%Cooperative diversity, decode and forward, piecewise linear
%\end{IEEEkeywords}



% For peer review papers, you can put extra information on the cover
% page as needed:
% \ifCLASSOPTIONpeerreview
% \begin{center} \bfseries EDICS Category: 3-BBND \end{center}
% \fi
%
% For peerreview papers, this IEEEtran command inserts a page break and
% creates the second title. It will be ignored for other modes.
%\IEEEpeerreviewmaketitle




	\item  A die is loaded in such a way that each odd number is twice as likely to occur as
each even number. Find $P(G)$, where $G$ is the event that a number greater than
3 occurs on a single roll of the die.
\\
\solution
		%\begin{table}[H]
	\centering
\begin{tabular}{|c|c|c|}
\hline
Random variable &Value &Definition\\ \hline
\multirow{3}{*}{X} &0 &Slips of Rs 1\\
&1 &Slips of Rs 5\\
&2 &Slips of Rs 13\\ \hline
\multirow{2}{*}{Y} &0 &Box A\\
&1 &Box B\\\hline
\end{tabular}
\caption{}
\label{tab:Distribution}
\end{table}
See \tabref{tab:Distribution}.
\begin{align}
p_{Y}\brak{k}= \begin{cases} 
      \frac{1}{3} & {k=0} \\
      \frac{2}{3 }& {k=1} 
   \end{cases}
   \\
p_{Y|X}\brak{0|0} = \frac{19}{25}\, 
p_{Y|X}\brak{0|1} = \frac{6}{25}\,
p_{Y|X}\brak{1|0} = \frac{45}{50}\,
p_{Y|X}\brak{1|2} = \frac{5}{50}
\end{align}
The desired probability is the probability that a slip drawn at random is marked other than Rs 1,
\begin{align}
&=1-p_X\brak{0}\\
&= p_X(1) + p_X(2)
\end{align}
Using Bayes theorem,
\begin{align}
&= p_Y\brak{0} \times \pr{Y=0 | X=1} + p_Y\brak{1} \times \pr{Y=1|X=2}\\
&=\frac{1}{3} \times \frac{6}{25} + \frac{2}{3} \times \frac{5}{50}\\
&=\frac{11}{75}
\end{align}

\newpage

%\tableofcontents

\bigskip

\renewcommand{\thefigure}{\theenumi}
\renewcommand{\thetable}{\theenumi}
%\renewcommand{\theequation}{\theenumi}

%\begin{abstract}
%%\boldmath
%In this letter, an algorithm for evaluating the exact analytical bit error rate  (BER)  for the piecewise linear (PL) combiner for  multiple relays is presented. Previous results were available only for upto three relays. The algorithm is unique in the sense that  the actual mathematical expressions, that are prohibitively large, need not be explicitly obtained. The diversity gain due to multiple relays is shown through plots of the analytical BER, well supported by simulations. 
%
%\end{abstract}
% IEEEtran.cls defaults to using nonbold math in the Abstract.
% This preserves the distinction between vectors and scalars. However,
% if the journal you are submitting to favors bold math in the abstract,
% then you can use LaTeX's standard command \boldmath at the very start
% of the abstract to achieve this. Many IEEE journals frown on math
% in the abstract anyway.

% Note that keywords are not normally used for peerreview papers.
%\begin{IEEEkeywords}
%Cooperative diversity, decode and forward, piecewise linear
%\end{IEEEkeywords}



% For peer review papers, you can put extra information on the cover
% page as needed:
% \ifCLASSOPTIONpeerreview
% \begin{center} \bfseries EDICS Category: 3-BBND \end{center}
% \fi
%
% For peerreview papers, this IEEEtran command inserts a page break and
% creates the second title. It will be ignored for other modes.
%\IEEEpeerreviewmaketitle




	\item All the jacks, queens and kings are removed from a deck of 52 playing cards. The remaining cards are well shuffled and then one card is drawn at random. Giving ace a value 1 similar value for other cards, find the probability that the card has a value 
		\begin{enumerate}
			\item 7
			\item greater than 7
			\item less than 7
		\end{enumerate}
		%Number of cards left after removing all jacks, queens and kings 
\begin{align}
N	= 52 - 4\times 3
	= 40
\end{align}
%\begin{table}[H]
%\def\arraystretch{1.2}
%\begin{tabular}{|c|c|c|}
%\hline
%	\textbf{Parameter} &\textbf{Value} &\textbf{Description}\\ \hline
%	$X$ &1-10 &Represents the value of the card picked \\ \hline
%\end{tabular}
%\end{table}
Let $1 \le X \le 10$ be the value of the card picked.  Then,
\begin{align}
	p_X(k) &= \Pr(X=k)\ \forall\ 1 \leq k \leq 10\\
	&= \frac{4\times 1}{40}\\
	&= \frac{1}{10}\\
	\therefore p_X(k) &= 
	\begin{cases}
		\frac{1}{10} & 1 \leq k \leq 10\\
		0 & \text{otherwise}
	\end{cases}
\end{align}
and
\begin{align}
	F_{X}(k) &= \sum_{m=0}^{k}p_{X}(m) \quad 1 \leq k \leq 10\\
	&= \frac{k}{10}\\
	\therefore F_{X}(k) &= 
	\begin{cases}
		0 & k \leq 0\\
		\frac{k}{10} & 1\leq k \leq 10\\
		1 & k > 10 
	\end{cases}
\end{align}
\begin{enumerate}
	\item Probability that card has value equal to 7 is
		\begin{align}
			 p_{X}(7)
			= \frac{1}{10}
		\end{align}
	\item Probability that card has value greater than 7 is
		\begin{align}
			1 - F_X(7)
			&= 1 - \frac{7}{10}
			\\
			&= \frac{3}{10}
		\end{align}
	\item Probability that card has value less than 7 is
		\begin{align}
			 F_{X}(6)
			=\frac{6}{10}
		\end{align}
\end{enumerate}

  \item A Lot consists of 48 mobile phones of which 42 are good, 3 have only minor defects and 3 have major defects.Varnika will buy a phone if it is good but the trader will only buy a mobile if it has no major defects. One phone is selected at random from the lot. What is the probability that it is
\begin{enumerate}
	\item acceptable to Varnika?
            \item acceptable to the trader?
\end{enumerate}
\solution
	%\begin{table}[H]
	\centering
\begin{tabular}{|c|c|c|}
\hline
Random variable &Value &Definition\\ \hline
\multirow{3}{*}{X} &0 &Slips of Rs 1\\
&1 &Slips of Rs 5\\
&2 &Slips of Rs 13\\ \hline
\multirow{2}{*}{Y} &0 &Box A\\
&1 &Box B\\\hline
\end{tabular}
\caption{}
\label{tab:Distribution}
\end{table}
See \tabref{tab:Distribution}.
\begin{align}
p_{Y}\brak{k}= \begin{cases} 
      \frac{1}{3} & {k=0} \\
      \frac{2}{3 }& {k=1} 
   \end{cases}
   \\
p_{Y|X}\brak{0|0} = \frac{19}{25}\, 
p_{Y|X}\brak{0|1} = \frac{6}{25}\,
p_{Y|X}\brak{1|0} = \frac{45}{50}\,
p_{Y|X}\brak{1|2} = \frac{5}{50}
\end{align}
The desired probability is the probability that a slip drawn at random is marked other than Rs 1,
\begin{align}
&=1-p_X\brak{0}\\
&= p_X(1) + p_X(2)
\end{align}
Using Bayes theorem,
\begin{align}
&= p_Y\brak{0} \times \pr{Y=0 | X=1} + p_Y\brak{1} \times \pr{Y=1|X=2}\\
&=\frac{1}{3} \times \frac{6}{25} + \frac{2}{3} \times \frac{5}{50}\\
&=\frac{11}{75}
\end{align}

\newpage

%\tableofcontents

\bigskip

\renewcommand{\thefigure}{\theenumi}
\renewcommand{\thetable}{\theenumi}
%\renewcommand{\theequation}{\theenumi}

%\begin{abstract}
%%\boldmath
%In this letter, an algorithm for evaluating the exact analytical bit error rate  (BER)  for the piecewise linear (PL) combiner for  multiple relays is presented. Previous results were available only for upto three relays. The algorithm is unique in the sense that  the actual mathematical expressions, that are prohibitively large, need not be explicitly obtained. The diversity gain due to multiple relays is shown through plots of the analytical BER, well supported by simulations. 
%
%\end{abstract}
% IEEEtran.cls defaults to using nonbold math in the Abstract.
% This preserves the distinction between vectors and scalars. However,
% if the journal you are submitting to favors bold math in the abstract,
% then you can use LaTeX's standard command \boldmath at the very start
% of the abstract to achieve this. Many IEEE journals frown on math
% in the abstract anyway.

% Note that keywords are not normally used for peerreview papers.
%\begin{IEEEkeywords}
%Cooperative diversity, decode and forward, piecewise linear
%\end{IEEEkeywords}



% For peer review papers, you can put extra information on the cover
% page as needed:
% \ifCLASSOPTIONpeerreview
% \begin{center} \bfseries EDICS Category: 3-BBND \end{center}
% \fi
%
% For peerreview papers, this IEEEtran command inserts a page break and
% creates the second title. It will be ignored for other modes.
%\IEEEpeerreviewmaketitle




 \item A student says that if you throw a die, it will show up 1 or not 1. Therefore, the probability of getting 1 and the probability of getting 'not 1' each is equal to $\frac{1}{2}$. Is this correct? Give reasons.\\
 \solution
        %\begin{table}[H]
	\centering
\begin{tabular}{|c|c|c|}
\hline
Random variable &Value &Definition\\ \hline
\multirow{3}{*}{X} &0 &Slips of Rs 1\\
&1 &Slips of Rs 5\\
&2 &Slips of Rs 13\\ \hline
\multirow{2}{*}{Y} &0 &Box A\\
&1 &Box B\\\hline
\end{tabular}
\caption{}
\label{tab:Distribution}
\end{table}
See \tabref{tab:Distribution}.
\begin{align}
p_{Y}\brak{k}= \begin{cases} 
      \frac{1}{3} & {k=0} \\
      \frac{2}{3 }& {k=1} 
   \end{cases}
   \\
p_{Y|X}\brak{0|0} = \frac{19}{25}\, 
p_{Y|X}\brak{0|1} = \frac{6}{25}\,
p_{Y|X}\brak{1|0} = \frac{45}{50}\,
p_{Y|X}\brak{1|2} = \frac{5}{50}
\end{align}
The desired probability is the probability that a slip drawn at random is marked other than Rs 1,
\begin{align}
&=1-p_X\brak{0}\\
&= p_X(1) + p_X(2)
\end{align}
Using Bayes theorem,
\begin{align}
&= p_Y\brak{0} \times \pr{Y=0 | X=1} + p_Y\brak{1} \times \pr{Y=1|X=2}\\
&=\frac{1}{3} \times \frac{6}{25} + \frac{2}{3} \times \frac{5}{50}\\
&=\frac{11}{75}
\end{align}

\newpage

%\tableofcontents

\bigskip

\renewcommand{\thefigure}{\theenumi}
\renewcommand{\thetable}{\theenumi}
%\renewcommand{\theequation}{\theenumi}

%\begin{abstract}
%%\boldmath
%In this letter, an algorithm for evaluating the exact analytical bit error rate  (BER)  for the piecewise linear (PL) combiner for  multiple relays is presented. Previous results were available only for upto three relays. The algorithm is unique in the sense that  the actual mathematical expressions, that are prohibitively large, need not be explicitly obtained. The diversity gain due to multiple relays is shown through plots of the analytical BER, well supported by simulations. 
%
%\end{abstract}
% IEEEtran.cls defaults to using nonbold math in the Abstract.
% This preserves the distinction between vectors and scalars. However,
% if the journal you are submitting to favors bold math in the abstract,
% then you can use LaTeX's standard command \boldmath at the very start
% of the abstract to achieve this. Many IEEE journals frown on math
% in the abstract anyway.

% Note that keywords are not normally used for peerreview papers.
%\begin{IEEEkeywords}
%Cooperative diversity, decode and forward, piecewise linear
%\end{IEEEkeywords}



% For peer review papers, you can put extra information on the cover
% page as needed:
% \ifCLASSOPTIONpeerreview
% \begin{center} \bfseries EDICS Category: 3-BBND \end{center}
% \fi
%
% For peerreview papers, this IEEEtran command inserts a page break and
% creates the second title. It will be ignored for other modes.
%\IEEEpeerreviewmaketitle




   \item Four candidates A, B, C, D have ap-
plied for the assignment to coach a school cricket
team. If A is twice as likely to be selected as B, and
B and C are given about the same chance of being
selected, while C is twice as likely to be selected
as D, what are the probabilities that
\begin{enumerate}
\item C will be selected?
\item A will not be selected?
\end{enumerate}
	%\begin{table}[H]
	\centering
\begin{tabular}{|c|c|c|}
\hline
Random variable &Value &Definition\\ \hline
\multirow{3}{*}{X} &0 &Slips of Rs 1\\
&1 &Slips of Rs 5\\
&2 &Slips of Rs 13\\ \hline
\multirow{2}{*}{Y} &0 &Box A\\
&1 &Box B\\\hline
\end{tabular}
\caption{}
\label{tab:Distribution}
\end{table}
See \tabref{tab:Distribution}.
\begin{align}
p_{Y}\brak{k}= \begin{cases} 
      \frac{1}{3} & {k=0} \\
      \frac{2}{3 }& {k=1} 
   \end{cases}
   \\
p_{Y|X}\brak{0|0} = \frac{19}{25}\, 
p_{Y|X}\brak{0|1} = \frac{6}{25}\,
p_{Y|X}\brak{1|0} = \frac{45}{50}\,
p_{Y|X}\brak{1|2} = \frac{5}{50}
\end{align}
The desired probability is the probability that a slip drawn at random is marked other than Rs 1,
\begin{align}
&=1-p_X\brak{0}\\
&= p_X(1) + p_X(2)
\end{align}
Using Bayes theorem,
\begin{align}
&= p_Y\brak{0} \times \pr{Y=0 | X=1} + p_Y\brak{1} \times \pr{Y=1|X=2}\\
&=\frac{1}{3} \times \frac{6}{25} + \frac{2}{3} \times \frac{5}{50}\\
&=\frac{11}{75}
\end{align}

\newpage

%\tableofcontents

\bigskip

\renewcommand{\thefigure}{\theenumi}
\renewcommand{\thetable}{\theenumi}
%\renewcommand{\theequation}{\theenumi}

%\begin{abstract}
%%\boldmath
%In this letter, an algorithm for evaluating the exact analytical bit error rate  (BER)  for the piecewise linear (PL) combiner for  multiple relays is presented. Previous results were available only for upto three relays. The algorithm is unique in the sense that  the actual mathematical expressions, that are prohibitively large, need not be explicitly obtained. The diversity gain due to multiple relays is shown through plots of the analytical BER, well supported by simulations. 
%
%\end{abstract}
% IEEEtran.cls defaults to using nonbold math in the Abstract.
% This preserves the distinction between vectors and scalars. However,
% if the journal you are submitting to favors bold math in the abstract,
% then you can use LaTeX's standard command \boldmath at the very start
% of the abstract to achieve this. Many IEEE journals frown on math
% in the abstract anyway.

% Note that keywords are not normally used for peerreview papers.
%\begin{IEEEkeywords}
%Cooperative diversity, decode and forward, piecewise linear
%\end{IEEEkeywords}



% For peer review papers, you can put extra information on the cover
% page as needed:
% \ifCLASSOPTIONpeerreview
% \begin{center} \bfseries EDICS Category: 3-BBND \end{center}
% \fi
%
% For peerreview papers, this IEEEtran command inserts a page break and
% creates the second title. It will be ignored for other modes.
%\IEEEpeerreviewmaketitle




 \item A bag contain 24 balls of which $x$ balls are red, $2x$ are white and $3x$ are blue. A ball is selected at random, What is the probability that it is
\begin{enumerate}[label=\alph*)]
\item not red ?
\item white ?
\end{enumerate}
%\begin{table}[H]
	\centering
\begin{tabular}{|c|c|c|}
\hline
Random variable &Value &Definition\\ \hline
\multirow{3}{*}{X} &0 &Slips of Rs 1\\
&1 &Slips of Rs 5\\
&2 &Slips of Rs 13\\ \hline
\multirow{2}{*}{Y} &0 &Box A\\
&1 &Box B\\\hline
\end{tabular}
\caption{}
\label{tab:Distribution}
\end{table}
See \tabref{tab:Distribution}.
\begin{align}
p_{Y}\brak{k}= \begin{cases} 
      \frac{1}{3} & {k=0} \\
      \frac{2}{3 }& {k=1} 
   \end{cases}
   \\
p_{Y|X}\brak{0|0} = \frac{19}{25}\, 
p_{Y|X}\brak{0|1} = \frac{6}{25}\,
p_{Y|X}\brak{1|0} = \frac{45}{50}\,
p_{Y|X}\brak{1|2} = \frac{5}{50}
\end{align}
The desired probability is the probability that a slip drawn at random is marked other than Rs 1,
\begin{align}
&=1-p_X\brak{0}\\
&= p_X(1) + p_X(2)
\end{align}
Using Bayes theorem,
\begin{align}
&= p_Y\brak{0} \times \pr{Y=0 | X=1} + p_Y\brak{1} \times \pr{Y=1|X=2}\\
&=\frac{1}{3} \times \frac{6}{25} + \frac{2}{3} \times \frac{5}{50}\\
&=\frac{11}{75}
\end{align}

\newpage

%\tableofcontents

\bigskip

\renewcommand{\thefigure}{\theenumi}
\renewcommand{\thetable}{\theenumi}
%\renewcommand{\theequation}{\theenumi}

%\begin{abstract}
%%\boldmath
%In this letter, an algorithm for evaluating the exact analytical bit error rate  (BER)  for the piecewise linear (PL) combiner for  multiple relays is presented. Previous results were available only for upto three relays. The algorithm is unique in the sense that  the actual mathematical expressions, that are prohibitively large, need not be explicitly obtained. The diversity gain due to multiple relays is shown through plots of the analytical BER, well supported by simulations. 
%
%\end{abstract}
% IEEEtran.cls defaults to using nonbold math in the Abstract.
% This preserves the distinction between vectors and scalars. However,
% if the journal you are submitting to favors bold math in the abstract,
% then you can use LaTeX's standard command \boldmath at the very start
% of the abstract to achieve this. Many IEEE journals frown on math
% in the abstract anyway.

% Note that keywords are not normally used for peerreview papers.
%\begin{IEEEkeywords}
%Cooperative diversity, decode and forward, piecewise linear
%\end{IEEEkeywords}



% For peer review papers, you can put extra information on the cover
% page as needed:
% \ifCLASSOPTIONpeerreview
% \begin{center} \bfseries EDICS Category: 3-BBND \end{center}
% \fi
%
% For peerreview papers, this IEEEtran command inserts a page break and
% creates the second title. It will be ignored for other modes.
%\IEEEpeerreviewmaketitle




If the letters of the word ASSASSINATION are arranged at random. Find the Probability that
\begin{enumerate}[label=(\alph*)]
\item Four $S's$ come consecutively in the word
\item Two  $I's$ and two $N's$ come together
\item All $A's$ are not coming together
\item No two $A's$ are coming together
\end{enumerate}
%\begin{table}[H]
	\centering
\begin{tabular}{|c|c|c|}
\hline
Random variable &Value &Definition\\ \hline
\multirow{3}{*}{X} &0 &Slips of Rs 1\\
&1 &Slips of Rs 5\\
&2 &Slips of Rs 13\\ \hline
\multirow{2}{*}{Y} &0 &Box A\\
&1 &Box B\\\hline
\end{tabular}
\caption{}
\label{tab:Distribution}
\end{table}
See \tabref{tab:Distribution}.
\begin{align}
p_{Y}\brak{k}= \begin{cases} 
      \frac{1}{3} & {k=0} \\
      \frac{2}{3 }& {k=1} 
   \end{cases}
   \\
p_{Y|X}\brak{0|0} = \frac{19}{25}\, 
p_{Y|X}\brak{0|1} = \frac{6}{25}\,
p_{Y|X}\brak{1|0} = \frac{45}{50}\,
p_{Y|X}\brak{1|2} = \frac{5}{50}
\end{align}
The desired probability is the probability that a slip drawn at random is marked other than Rs 1,
\begin{align}
&=1-p_X\brak{0}\\
&= p_X(1) + p_X(2)
\end{align}
Using Bayes theorem,
\begin{align}
&= p_Y\brak{0} \times \pr{Y=0 | X=1} + p_Y\brak{1} \times \pr{Y=1|X=2}\\
&=\frac{1}{3} \times \frac{6}{25} + \frac{2}{3} \times \frac{5}{50}\\
&=\frac{11}{75}
\end{align}

\newpage

%\tableofcontents

\bigskip

\renewcommand{\thefigure}{\theenumi}
\renewcommand{\thetable}{\theenumi}
%\renewcommand{\theequation}{\theenumi}

%\begin{abstract}
%%\boldmath
%In this letter, an algorithm for evaluating the exact analytical bit error rate  (BER)  for the piecewise linear (PL) combiner for  multiple relays is presented. Previous results were available only for upto three relays. The algorithm is unique in the sense that  the actual mathematical expressions, that are prohibitively large, need not be explicitly obtained. The diversity gain due to multiple relays is shown through plots of the analytical BER, well supported by simulations. 
%
%\end{abstract}
% IEEEtran.cls defaults to using nonbold math in the Abstract.
% This preserves the distinction between vectors and scalars. However,
% if the journal you are submitting to favors bold math in the abstract,
% then you can use LaTeX's standard command \boldmath at the very start
% of the abstract to achieve this. Many IEEE journals frown on math
% in the abstract anyway.

% Note that keywords are not normally used for peerreview papers.
%\begin{IEEEkeywords}
%Cooperative diversity, decode and forward, piecewise linear
%\end{IEEEkeywords}



% For peer review papers, you can put extra information on the cover
% page as needed:
% \ifCLASSOPTIONpeerreview
% \begin{center} \bfseries EDICS Category: 3-BBND \end{center}
% \fi
%
% For peerreview papers, this IEEEtran command inserts a page break and
% creates the second title. It will be ignored for other modes.
%\IEEEpeerreviewmaketitle




	\item One urn contains two black balls (labelled B1 and B2) and one white ball. A
	second urn contains one black ball and two white balls (labelled W1 and W2).
	Suppose the following experiment is performed. One of the two urns is chosen
	at random. Next a ball is randomly chosen from the urn. Then a second ball is
	chosen at random from the same urn without replacing the first ball.
	
	\begin{enumerate}
	\item What is the probability that two black balls are chosen?
	
	\item What is the probability that two balls of opposite colour are chosen?
	\end{enumerate}
	\solution
	%\begin{align}
    \label{eq:12.13.6.18.1}
	\because	\pr{A|B} &> \pr{A},\
\frac{\pr{AB}}{\pr{B}} > \pr{A}
\\
    \label{eq:12.13.6.18.2}
	\implies \pr{AB} &> \pr{A}\pr{B}
	\\
	\text{or, } \frac{\pr{AB}}{\pr{A}} &=\pr{B|A} > \pr{A}
\end{align}

\end{enumerate}

\item In a certain lottery 10,000 tickets are sold and ten equal prizes are awarded. What is the probability of not getting a prize if you buy (a) one ticket (b) two tickets (c) 10 tickets ?	
\\
\solution
		%\begin{enumerate}[label=\thesection.\arabic*,ref=\thesection.\theenumi]
	\item One card is drawn from a well-shuffled deck of 52 cards. Find the probability of getting
\begin{enumerate}
\item A king of red colour 
\item A face card 
\item A red face card
\item The jack of hearts
\item A spade
\item The queen of diamonds

\end{enumerate}
\solution
		%\begin{table}[H]
	\centering
\begin{tabular}{|c|c|c|}
\hline
Random variable &Value &Definition\\ \hline
\multirow{3}{*}{X} &0 &Slips of Rs 1\\
&1 &Slips of Rs 5\\
&2 &Slips of Rs 13\\ \hline
\multirow{2}{*}{Y} &0 &Box A\\
&1 &Box B\\\hline
\end{tabular}
\caption{}
\label{tab:Distribution}
\end{table}
See \tabref{tab:Distribution}.
\begin{align}
p_{Y}\brak{k}= \begin{cases} 
      \frac{1}{3} & {k=0} \\
      \frac{2}{3 }& {k=1} 
   \end{cases}
   \\
p_{Y|X}\brak{0|0} = \frac{19}{25}\, 
p_{Y|X}\brak{0|1} = \frac{6}{25}\,
p_{Y|X}\brak{1|0} = \frac{45}{50}\,
p_{Y|X}\brak{1|2} = \frac{5}{50}
\end{align}
The desired probability is the probability that a slip drawn at random is marked other than Rs 1,
\begin{align}
&=1-p_X\brak{0}\\
&= p_X(1) + p_X(2)
\end{align}
Using Bayes theorem,
\begin{align}
&= p_Y\brak{0} \times \pr{Y=0 | X=1} + p_Y\brak{1} \times \pr{Y=1|X=2}\\
&=\frac{1}{3} \times \frac{6}{25} + \frac{2}{3} \times \frac{5}{50}\\
&=\frac{11}{75}
\end{align}

\newpage

%\tableofcontents

\bigskip

\renewcommand{\thefigure}{\theenumi}
\renewcommand{\thetable}{\theenumi}
%\renewcommand{\theequation}{\theenumi}

%\begin{abstract}
%%\boldmath
%In this letter, an algorithm for evaluating the exact analytical bit error rate  (BER)  for the piecewise linear (PL) combiner for  multiple relays is presented. Previous results were available only for upto three relays. The algorithm is unique in the sense that  the actual mathematical expressions, that are prohibitively large, need not be explicitly obtained. The diversity gain due to multiple relays is shown through plots of the analytical BER, well supported by simulations. 
%
%\end{abstract}
% IEEEtran.cls defaults to using nonbold math in the Abstract.
% This preserves the distinction between vectors and scalars. However,
% if the journal you are submitting to favors bold math in the abstract,
% then you can use LaTeX's standard command \boldmath at the very start
% of the abstract to achieve this. Many IEEE journals frown on math
% in the abstract anyway.

% Note that keywords are not normally used for peerreview papers.
%\begin{IEEEkeywords}
%Cooperative diversity, decode and forward, piecewise linear
%\end{IEEEkeywords}



% For peer review papers, you can put extra information on the cover
% page as needed:
% \ifCLASSOPTIONpeerreview
% \begin{center} \bfseries EDICS Category: 3-BBND \end{center}
% \fi
%
% For peerreview papers, this IEEEtran command inserts a page break and
% creates the second title. It will be ignored for other modes.
%\IEEEpeerreviewmaketitle




	\item Five cards—the ten, jack, queen, king and ace of diamonds, are well-shuffled with their face downwards. One card is then picked up at random.
\begin{enumerate}
\item
What is the probability that the card is the queen? 
\item
If the queen is drawn and put aside, what is the probability that the second card picked up is (a) an ace? (b) a queen?\\
\end{enumerate}
\solution
		%\begin{enumerate}[label=\thesection.\arabic*,ref=\thesection.\theenumi]
	\item One card is drawn from a well-shuffled deck of 52 cards. Find the probability of getting
\begin{enumerate}
\item A king of red colour 
\item A face card 
\item A red face card
\item The jack of hearts
\item A spade
\item The queen of diamonds

\end{enumerate}
\solution
		%\input{ncert/10/15/1/14/main.tex}
	\item Five cards—the ten, jack, queen, king and ace of diamonds, are well-shuffled with their face downwards. One card is then picked up at random.
\begin{enumerate}
\item
What is the probability that the card is the queen? 
\item
If the queen is drawn and put aside, what is the probability that the second card picked up is (a) an ace? (b) a queen?\\
\end{enumerate}
\solution
		%\input{ncert/10/15/1/15/defs.tex}
	\item A bag contains $5$ red balls and some blue balls. If the probability of drawing a blue ball is double that if a red ball, determine the number of blue balls in the bag. 
		\\
\solution
		%\input{ncert/10/15/2/3/defs.tex}
	\item A card is selected from a pack of 52 cards.
 \begin{enumerate}[label=(\alph*)] 
                 \item How many points are there in the sample space?
                 \item Calculate the probability that the card is an ace of spades.
                 \item Calculate the probability that the card is (i) an ace and (ii) black card.
 \end{enumerate}
\solution
		%\input{ncert/11/16/3/4/main.tex}
\item Four cards are drawn from a well-shuffled deck of 52 cards. What is the probability of obtaining 3 diamonds and one spade.
\\
\solution
		%\input{ncert/11/16/4/2/defs.tex}
\item In a certain lottery 10,000 tickets are sold and ten equal prizes are awarded. What is the probability of not getting a prize if you buy (a) one ticket (b) two tickets (c) 10 tickets ?	
\\
\solution
		%\input{ncert/11/16/4/4/defs.tex}
		%
\item 
Out of 100 students, two sections of 40 and 60 are formed. If you and your friend are among the 100 students, what is the probability that
\begin{enumerate}
\item you both enter the same section?
\item you both enter the different sections?
\end{enumerate}
\solution
		%\input{ncert/11/16/4/5/defs.tex}
	\item 
The number lock of a suitcase has 4 wheels each labelled with ten digits i.e. from 0 to 9.The lock opens with a sequence of four digits with no repeats.What is the probability of a person getting the right sequence to open the suitcase.
\\
\solution
		%\input{ncert/11/16/4/10/defs.tex}
		%
\item 
Two cards are drawn at random and without replacement from a pack of 52 playing cards. Find the probability that both the cards are black.
\\
\solution
		%\input{ncert/12/13/2/2/defs.tex}
		\item A box of oranges is inspected by examining three randomly selected oranges drawn without replacement. If all the three oranges are good, the box is approved for sale, otherwise, it is rejected. Find the probability that a box containing 15 oranges out of which 12 are good and 3 are bad ones will be approved for sale.
		\label{ncert/12/13/2/3/defs.tex}
		\item Two balls are drawn at random with replacement from a box containing 10 black and 8 red balls. Find the probability that
		\label{ncert/12/13/2/12}
\begin{enumerate}
\item both balls are red.
\item first ball is black and second is red.
\item one of them is black and other is red.
\end{enumerate}

\item In a hostel, 60\% of the students read Hindi newspaper, 40\% read English newspaper and 20\% read both Hindi and English newspapers. A student is selected at random.
		\label{ncert/12/13/2/15}
\begin{enumerate}
\item Find the probability that she reads neither Hindi nor English newspapers.
\item If she reads Hindi newspaper, find the probability that she reads English newspaper.
\item If she reads English newspaper, find the probability that she reads Hindi newspaper.\\
\end{enumerate}
\item The probability of obtaining an even prime number on each die, when a pair of dice is rolled is 
\begin{enumerate}
    \item $0$ 
    
    \item $\frac{1}{3}$ 
    
    \item $\frac{1}{12}$ 
    
    \item $\frac{1}{36}$ 
\end{enumerate}
\solution
		%\input{ncert/12/13/2/17/defs.tex}
	\item A bag contains 4 red and 4 black balls, another bag contains 2 red and 6 black balls. One of the two bags is selected at random and a ball is drawn from the bag which is found to be red. Find the probability that the ball is drawn from the first bag.
\\
\solution
		%\input{ncert/12/13/3/2/main.tex}
  \item
  Cards with numbers 2 to 101 are placed in a box. A card is selected at random.Find the probability that the card has
\begin{enumerate}[label=(\roman*)]
	\item an even number 
	\item a square number
\end{enumerate}
\solution
%\input{exemplar/10/13/3/32/main.tex}
\item
The king, queen and jack of clubs are removed from a deck of 52 playing cards and then well shuffled. Now one card is drawn at random from the remaining cards.  Determine the probability that the card is
\begin{enumerate}[label=(\roman*)]
\item a club
\item 10 of hearts
\end{enumerate}
\solution
%\input{exemplar/10/13/3/29/main.tex}
\item A team of medical students doing their internship have to assist during surgeries
at a city hospital. The probabilities of surgeries rated as very complex, complex,
routine, simple or very simple are respectively, 0.15, 0.20, 0.31, 0.26, .08. Find
the probabilities that a particular surgery will be rated
\begin{enumerate}
	\item complex or very complex;
	\item neither very complex nor very simple;
	\item routine or complex
	\item routine or simple
\end{enumerate}
\solution
%\input{exemplar/11/16/3/8(1)/main.tex}
\item A card is selected from a pack of 52 cards.
\begin{enumerate}[label=(\alph*)]
    \item How many points are there in the sample space?
    \item Calculate the probability that the card is an ace of spades.
    \item Calculate the probability that the card is (i) an ace and (ii) black card.
\end{enumerate}
\solution
%\input{exemplar/11/16/3/4/main2.tex}
\item The probability that a non leap year selected at random will contain 53 sundays.
\\
\solution
%\input{exemplar/10/13/1/19/main.tex}
\item One of the four persons John, Rita, Aslam or Gurpreet will be promoted next
month. Consequently the sample space consists of four elementary outcomes
S = {John promoted, Rita promoted, Aslam promoted, Gurpreet promoted}
You are told that the chances of John’s promotion is same as that of Gurpreet,
Rita’s chances of promotion are twice as likely as Johns. Aslam’s chances are
four times that of John.
\begin{enumerate}
	\item Determine
	\begin{enumerate}
		\item P (John promoted)
		\item P (Rita promoted)
		\item P (Aslam promoted)
		\item P (Gurpreet promoted)
	\end{enumerate}
	\item If A = {John promoted or Gurpreet promoted}, find P (A).
\end{enumerate}
\solution
%\input{exemplar/11/16/3/10/main.tex}
\item A card is drawn from a deck of 52 cards. Find the probability of getting a king or a heart or a red card.\\
\solution
%\input{exemplar/11/16/3/15/main.tex}
\item The probability that a student will pass his examination is 0.73, the probability of
the student getting a compartment is 0.13, and the probability that the student will
either pass or get compartment is 0.96. State True or False.\\
\solution
%\input{exemplar/11/16/3/31/main.tex}
\item A card is selected from a pack of 52 cards\\
\begin{enumerate}[label=(\alph*)]
\item How many points are there in the sample space?
\item Calculate the probability that the cards is an ace of spades.
\item Calculate the probability that the card is (i) an ace (ii)black card.\\
\end{enumerate}
%\input{ncert/11/16/3/4_1/Prob_4.tex}
\item In a non-leap year, the probability of having 53 tuesdays or 53 wednesdays is\\
\solution
%\input{exemplar/11/16/3/18/main.tex}
\item There are 1000 sealed envelopes in a box, 10 of them contain a cash prize of
Rs 100 each, 100 of them contain a cash prize of Rs 50 each and 200 of them
contain a cash prize of Rs 10 each and rest do not contain any cash prize. If they
are well shuffled and an envelope is picked up out, what is the probability that it
contains no cash prize?\\
\solution
%\input{exemplar/10/13/3/34/main.tex}
\item 
A die is thrown and a card is selected at random from a deck of 52 playing cards. The probability of getting an even number on the die and a spade card.\\
\solution
%\input{exemplar/12/13/3/78/main.tex}
\item
If 4-digit numbers greater than 5,000 are randomly formed from the digits 0, 1, 3, 5, and 7, what is the probability of forming a number divisible by 5 when:
\begin{enumerate}
    \item The digits are repeated?
    \item The repetition of digits is not allowed?
\end{enumerate}
\solution
%\input{ncert/11/16/4/9/main.tex}
\item Consider the probability space $\brak{\Omega, \mathcal{G}, P}$ where $\Omega = [0,2]$ and $\mathcal{G} = \cbrak{\phi, \Omega, [0,1], (1,2]}$. Let $X$ and $Y$ be two functions on $\Omega$ defined as
\begin{align*}
    X(\omega) = 
    \begin{cases}
        1 & \text{if }\omega \in [0, 1]\\
        2 & \text{if }\omega \in (1, 2]
    \end{cases}
\end{align*}
and
\begin{align*}
    Y(\omega) = 
    \begin{cases}
        2 & \text{if }\omega \in [0, 1.5]\\
        3 & \text{if }\omega \in (1.5, 2].
    \end{cases}
\end{align*}
Then which one of the following statements is true?
\begin{enumerate}
    \item [(A)] $X$ is a random variable with respect to $\mathcal{G}$, but $Y$ is not a random variable with respect to $\mathcal{G}$.
    \item [(B)] $Y$ is a random variable with respect to $\mathcal{G}$, but $X$ is not a random variable with respect to $\mathcal{G}$.
    \item [(C)] Neither $X$ nor $Y$ is a random variable with respect to $\mathcal{G}$.
    \item [(D)] Both $X$ and $Y$ are random variables with respect to $\mathcal{G}$.
\end{enumerate} \hfill (GATE ST 2023)\\
\solution
%\input{gate/ST/2023/14/main.tex}
	\item  A die is loaded in such a way that each odd number is twice as likely to occur as
each even number. Find $P(G)$, where $G$ is the event that a number greater than
3 occurs on a single roll of the die.
\\
\solution
		%\input{exemplar/11/16/3/5/main.tex}
	\item All the jacks, queens and kings are removed from a deck of 52 playing cards. The remaining cards are well shuffled and then one card is drawn at random. Giving ace a value 1 similar value for other cards, find the probability that the card has a value 
		\begin{enumerate}
			\item 7
			\item greater than 7
			\item less than 7
		\end{enumerate}
		%\input{exemplar/10/13/3/30/main.tex}
  \item A Lot consists of 48 mobile phones of which 42 are good, 3 have only minor defects and 3 have major defects.Varnika will buy a phone if it is good but the trader will only buy a mobile if it has no major defects. One phone is selected at random from the lot. What is the probability that it is
\begin{enumerate}
	\item acceptable to Varnika?
            \item acceptable to the trader?
\end{enumerate}
\solution
	%\input{exemplar/10/13/3/40/main.tex}
 \item A student says that if you throw a die, it will show up 1 or not 1. Therefore, the probability of getting 1 and the probability of getting 'not 1' each is equal to $\frac{1}{2}$. Is this correct? Give reasons.\\
 \solution
        %\input{exemplar/10/13/2/9/main.tex}
   \item Four candidates A, B, C, D have ap-
plied for the assignment to coach a school cricket
team. If A is twice as likely to be selected as B, and
B and C are given about the same chance of being
selected, while C is twice as likely to be selected
as D, what are the probabilities that
\begin{enumerate}
\item C will be selected?
\item A will not be selected?
\end{enumerate}
	%\input{exemplar/11/16/3/9/main.tex}
 \item A bag contain 24 balls of which $x$ balls are red, $2x$ are white and $3x$ are blue. A ball is selected at random, What is the probability that it is
\begin{enumerate}[label=\alph*)]
\item not red ?
\item white ?
\end{enumerate}
%\input{exemplar/10/13/3/41/main.tex}
If the letters of the word ASSASSINATION are arranged at random. Find the Probability that
\begin{enumerate}[label=(\alph*)]
\item Four $S's$ come consecutively in the word
\item Two  $I's$ and two $N's$ come together
\item All $A's$ are not coming together
\item No two $A's$ are coming together
\end{enumerate}
%\input{exemplar/11/16/3/14/main.tex}
	\item One urn contains two black balls (labelled B1 and B2) and one white ball. A
	second urn contains one black ball and two white balls (labelled W1 and W2).
	Suppose the following experiment is performed. One of the two urns is chosen
	at random. Next a ball is randomly chosen from the urn. Then a second ball is
	chosen at random from the same urn without replacing the first ball.
	
	\begin{enumerate}
	\item What is the probability that two black balls are chosen?
	
	\item What is the probability that two balls of opposite colour are chosen?
	\end{enumerate}
	\solution
	%\input{exemplar/11/16/3/12/main1.tex}
\end{enumerate}

	\item A bag contains $5$ red balls and some blue balls. If the probability of drawing a blue ball is double that if a red ball, determine the number of blue balls in the bag. 
		\\
\solution
		%\begin{enumerate}[label=\thesection.\arabic*,ref=\thesection.\theenumi]
	\item One card is drawn from a well-shuffled deck of 52 cards. Find the probability of getting
\begin{enumerate}
\item A king of red colour 
\item A face card 
\item A red face card
\item The jack of hearts
\item A spade
\item The queen of diamonds

\end{enumerate}
\solution
		%\input{ncert/10/15/1/14/main.tex}
	\item Five cards—the ten, jack, queen, king and ace of diamonds, are well-shuffled with their face downwards. One card is then picked up at random.
\begin{enumerate}
\item
What is the probability that the card is the queen? 
\item
If the queen is drawn and put aside, what is the probability that the second card picked up is (a) an ace? (b) a queen?\\
\end{enumerate}
\solution
		%\input{ncert/10/15/1/15/defs.tex}
	\item A bag contains $5$ red balls and some blue balls. If the probability of drawing a blue ball is double that if a red ball, determine the number of blue balls in the bag. 
		\\
\solution
		%\input{ncert/10/15/2/3/defs.tex}
	\item A card is selected from a pack of 52 cards.
 \begin{enumerate}[label=(\alph*)] 
                 \item How many points are there in the sample space?
                 \item Calculate the probability that the card is an ace of spades.
                 \item Calculate the probability that the card is (i) an ace and (ii) black card.
 \end{enumerate}
\solution
		%\input{ncert/11/16/3/4/main.tex}
\item Four cards are drawn from a well-shuffled deck of 52 cards. What is the probability of obtaining 3 diamonds and one spade.
\\
\solution
		%\input{ncert/11/16/4/2/defs.tex}
\item In a certain lottery 10,000 tickets are sold and ten equal prizes are awarded. What is the probability of not getting a prize if you buy (a) one ticket (b) two tickets (c) 10 tickets ?	
\\
\solution
		%\input{ncert/11/16/4/4/defs.tex}
		%
\item 
Out of 100 students, two sections of 40 and 60 are formed. If you and your friend are among the 100 students, what is the probability that
\begin{enumerate}
\item you both enter the same section?
\item you both enter the different sections?
\end{enumerate}
\solution
		%\input{ncert/11/16/4/5/defs.tex}
	\item 
The number lock of a suitcase has 4 wheels each labelled with ten digits i.e. from 0 to 9.The lock opens with a sequence of four digits with no repeats.What is the probability of a person getting the right sequence to open the suitcase.
\\
\solution
		%\input{ncert/11/16/4/10/defs.tex}
		%
\item 
Two cards are drawn at random and without replacement from a pack of 52 playing cards. Find the probability that both the cards are black.
\\
\solution
		%\input{ncert/12/13/2/2/defs.tex}
		\item A box of oranges is inspected by examining three randomly selected oranges drawn without replacement. If all the three oranges are good, the box is approved for sale, otherwise, it is rejected. Find the probability that a box containing 15 oranges out of which 12 are good and 3 are bad ones will be approved for sale.
		\label{ncert/12/13/2/3/defs.tex}
		\item Two balls are drawn at random with replacement from a box containing 10 black and 8 red balls. Find the probability that
		\label{ncert/12/13/2/12}
\begin{enumerate}
\item both balls are red.
\item first ball is black and second is red.
\item one of them is black and other is red.
\end{enumerate}

\item In a hostel, 60\% of the students read Hindi newspaper, 40\% read English newspaper and 20\% read both Hindi and English newspapers. A student is selected at random.
		\label{ncert/12/13/2/15}
\begin{enumerate}
\item Find the probability that she reads neither Hindi nor English newspapers.
\item If she reads Hindi newspaper, find the probability that she reads English newspaper.
\item If she reads English newspaper, find the probability that she reads Hindi newspaper.\\
\end{enumerate}
\item The probability of obtaining an even prime number on each die, when a pair of dice is rolled is 
\begin{enumerate}
    \item $0$ 
    
    \item $\frac{1}{3}$ 
    
    \item $\frac{1}{12}$ 
    
    \item $\frac{1}{36}$ 
\end{enumerate}
\solution
		%\input{ncert/12/13/2/17/defs.tex}
	\item A bag contains 4 red and 4 black balls, another bag contains 2 red and 6 black balls. One of the two bags is selected at random and a ball is drawn from the bag which is found to be red. Find the probability that the ball is drawn from the first bag.
\\
\solution
		%\input{ncert/12/13/3/2/main.tex}
  \item
  Cards with numbers 2 to 101 are placed in a box. A card is selected at random.Find the probability that the card has
\begin{enumerate}[label=(\roman*)]
	\item an even number 
	\item a square number
\end{enumerate}
\solution
%\input{exemplar/10/13/3/32/main.tex}
\item
The king, queen and jack of clubs are removed from a deck of 52 playing cards and then well shuffled. Now one card is drawn at random from the remaining cards.  Determine the probability that the card is
\begin{enumerate}[label=(\roman*)]
\item a club
\item 10 of hearts
\end{enumerate}
\solution
%\input{exemplar/10/13/3/29/main.tex}
\item A team of medical students doing their internship have to assist during surgeries
at a city hospital. The probabilities of surgeries rated as very complex, complex,
routine, simple or very simple are respectively, 0.15, 0.20, 0.31, 0.26, .08. Find
the probabilities that a particular surgery will be rated
\begin{enumerate}
	\item complex or very complex;
	\item neither very complex nor very simple;
	\item routine or complex
	\item routine or simple
\end{enumerate}
\solution
%\input{exemplar/11/16/3/8(1)/main.tex}
\item A card is selected from a pack of 52 cards.
\begin{enumerate}[label=(\alph*)]
    \item How many points are there in the sample space?
    \item Calculate the probability that the card is an ace of spades.
    \item Calculate the probability that the card is (i) an ace and (ii) black card.
\end{enumerate}
\solution
%\input{exemplar/11/16/3/4/main2.tex}
\item The probability that a non leap year selected at random will contain 53 sundays.
\\
\solution
%\input{exemplar/10/13/1/19/main.tex}
\item One of the four persons John, Rita, Aslam or Gurpreet will be promoted next
month. Consequently the sample space consists of four elementary outcomes
S = {John promoted, Rita promoted, Aslam promoted, Gurpreet promoted}
You are told that the chances of John’s promotion is same as that of Gurpreet,
Rita’s chances of promotion are twice as likely as Johns. Aslam’s chances are
four times that of John.
\begin{enumerate}
	\item Determine
	\begin{enumerate}
		\item P (John promoted)
		\item P (Rita promoted)
		\item P (Aslam promoted)
		\item P (Gurpreet promoted)
	\end{enumerate}
	\item If A = {John promoted or Gurpreet promoted}, find P (A).
\end{enumerate}
\solution
%\input{exemplar/11/16/3/10/main.tex}
\item A card is drawn from a deck of 52 cards. Find the probability of getting a king or a heart or a red card.\\
\solution
%\input{exemplar/11/16/3/15/main.tex}
\item The probability that a student will pass his examination is 0.73, the probability of
the student getting a compartment is 0.13, and the probability that the student will
either pass or get compartment is 0.96. State True or False.\\
\solution
%\input{exemplar/11/16/3/31/main.tex}
\item A card is selected from a pack of 52 cards\\
\begin{enumerate}[label=(\alph*)]
\item How many points are there in the sample space?
\item Calculate the probability that the cards is an ace of spades.
\item Calculate the probability that the card is (i) an ace (ii)black card.\\
\end{enumerate}
%\input{ncert/11/16/3/4_1/Prob_4.tex}
\item In a non-leap year, the probability of having 53 tuesdays or 53 wednesdays is\\
\solution
%\input{exemplar/11/16/3/18/main.tex}
\item There are 1000 sealed envelopes in a box, 10 of them contain a cash prize of
Rs 100 each, 100 of them contain a cash prize of Rs 50 each and 200 of them
contain a cash prize of Rs 10 each and rest do not contain any cash prize. If they
are well shuffled and an envelope is picked up out, what is the probability that it
contains no cash prize?\\
\solution
%\input{exemplar/10/13/3/34/main.tex}
\item 
A die is thrown and a card is selected at random from a deck of 52 playing cards. The probability of getting an even number on the die and a spade card.\\
\solution
%\input{exemplar/12/13/3/78/main.tex}
\item
If 4-digit numbers greater than 5,000 are randomly formed from the digits 0, 1, 3, 5, and 7, what is the probability of forming a number divisible by 5 when:
\begin{enumerate}
    \item The digits are repeated?
    \item The repetition of digits is not allowed?
\end{enumerate}
\solution
%\input{ncert/11/16/4/9/main.tex}
\item Consider the probability space $\brak{\Omega, \mathcal{G}, P}$ where $\Omega = [0,2]$ and $\mathcal{G} = \cbrak{\phi, \Omega, [0,1], (1,2]}$. Let $X$ and $Y$ be two functions on $\Omega$ defined as
\begin{align*}
    X(\omega) = 
    \begin{cases}
        1 & \text{if }\omega \in [0, 1]\\
        2 & \text{if }\omega \in (1, 2]
    \end{cases}
\end{align*}
and
\begin{align*}
    Y(\omega) = 
    \begin{cases}
        2 & \text{if }\omega \in [0, 1.5]\\
        3 & \text{if }\omega \in (1.5, 2].
    \end{cases}
\end{align*}
Then which one of the following statements is true?
\begin{enumerate}
    \item [(A)] $X$ is a random variable with respect to $\mathcal{G}$, but $Y$ is not a random variable with respect to $\mathcal{G}$.
    \item [(B)] $Y$ is a random variable with respect to $\mathcal{G}$, but $X$ is not a random variable with respect to $\mathcal{G}$.
    \item [(C)] Neither $X$ nor $Y$ is a random variable with respect to $\mathcal{G}$.
    \item [(D)] Both $X$ and $Y$ are random variables with respect to $\mathcal{G}$.
\end{enumerate} \hfill (GATE ST 2023)\\
\solution
%\input{gate/ST/2023/14/main.tex}
	\item  A die is loaded in such a way that each odd number is twice as likely to occur as
each even number. Find $P(G)$, where $G$ is the event that a number greater than
3 occurs on a single roll of the die.
\\
\solution
		%\input{exemplar/11/16/3/5/main.tex}
	\item All the jacks, queens and kings are removed from a deck of 52 playing cards. The remaining cards are well shuffled and then one card is drawn at random. Giving ace a value 1 similar value for other cards, find the probability that the card has a value 
		\begin{enumerate}
			\item 7
			\item greater than 7
			\item less than 7
		\end{enumerate}
		%\input{exemplar/10/13/3/30/main.tex}
  \item A Lot consists of 48 mobile phones of which 42 are good, 3 have only minor defects and 3 have major defects.Varnika will buy a phone if it is good but the trader will only buy a mobile if it has no major defects. One phone is selected at random from the lot. What is the probability that it is
\begin{enumerate}
	\item acceptable to Varnika?
            \item acceptable to the trader?
\end{enumerate}
\solution
	%\input{exemplar/10/13/3/40/main.tex}
 \item A student says that if you throw a die, it will show up 1 or not 1. Therefore, the probability of getting 1 and the probability of getting 'not 1' each is equal to $\frac{1}{2}$. Is this correct? Give reasons.\\
 \solution
        %\input{exemplar/10/13/2/9/main.tex}
   \item Four candidates A, B, C, D have ap-
plied for the assignment to coach a school cricket
team. If A is twice as likely to be selected as B, and
B and C are given about the same chance of being
selected, while C is twice as likely to be selected
as D, what are the probabilities that
\begin{enumerate}
\item C will be selected?
\item A will not be selected?
\end{enumerate}
	%\input{exemplar/11/16/3/9/main.tex}
 \item A bag contain 24 balls of which $x$ balls are red, $2x$ are white and $3x$ are blue. A ball is selected at random, What is the probability that it is
\begin{enumerate}[label=\alph*)]
\item not red ?
\item white ?
\end{enumerate}
%\input{exemplar/10/13/3/41/main.tex}
If the letters of the word ASSASSINATION are arranged at random. Find the Probability that
\begin{enumerate}[label=(\alph*)]
\item Four $S's$ come consecutively in the word
\item Two  $I's$ and two $N's$ come together
\item All $A's$ are not coming together
\item No two $A's$ are coming together
\end{enumerate}
%\input{exemplar/11/16/3/14/main.tex}
	\item One urn contains two black balls (labelled B1 and B2) and one white ball. A
	second urn contains one black ball and two white balls (labelled W1 and W2).
	Suppose the following experiment is performed. One of the two urns is chosen
	at random. Next a ball is randomly chosen from the urn. Then a second ball is
	chosen at random from the same urn without replacing the first ball.
	
	\begin{enumerate}
	\item What is the probability that two black balls are chosen?
	
	\item What is the probability that two balls of opposite colour are chosen?
	\end{enumerate}
	\solution
	%\input{exemplar/11/16/3/12/main1.tex}
\end{enumerate}

	\item A card is selected from a pack of 52 cards.
 \begin{enumerate}[label=(\alph*)] 
                 \item How many points are there in the sample space?
                 \item Calculate the probability that the card is an ace of spades.
                 \item Calculate the probability that the card is (i) an ace and (ii) black card.
 \end{enumerate}
\solution
		%\begin{table}[H]
	\centering
\begin{tabular}{|c|c|c|}
\hline
Random variable &Value &Definition\\ \hline
\multirow{3}{*}{X} &0 &Slips of Rs 1\\
&1 &Slips of Rs 5\\
&2 &Slips of Rs 13\\ \hline
\multirow{2}{*}{Y} &0 &Box A\\
&1 &Box B\\\hline
\end{tabular}
\caption{}
\label{tab:Distribution}
\end{table}
See \tabref{tab:Distribution}.
\begin{align}
p_{Y}\brak{k}= \begin{cases} 
      \frac{1}{3} & {k=0} \\
      \frac{2}{3 }& {k=1} 
   \end{cases}
   \\
p_{Y|X}\brak{0|0} = \frac{19}{25}\, 
p_{Y|X}\brak{0|1} = \frac{6}{25}\,
p_{Y|X}\brak{1|0} = \frac{45}{50}\,
p_{Y|X}\brak{1|2} = \frac{5}{50}
\end{align}
The desired probability is the probability that a slip drawn at random is marked other than Rs 1,
\begin{align}
&=1-p_X\brak{0}\\
&= p_X(1) + p_X(2)
\end{align}
Using Bayes theorem,
\begin{align}
&= p_Y\brak{0} \times \pr{Y=0 | X=1} + p_Y\brak{1} \times \pr{Y=1|X=2}\\
&=\frac{1}{3} \times \frac{6}{25} + \frac{2}{3} \times \frac{5}{50}\\
&=\frac{11}{75}
\end{align}

\newpage

%\tableofcontents

\bigskip

\renewcommand{\thefigure}{\theenumi}
\renewcommand{\thetable}{\theenumi}
%\renewcommand{\theequation}{\theenumi}

%\begin{abstract}
%%\boldmath
%In this letter, an algorithm for evaluating the exact analytical bit error rate  (BER)  for the piecewise linear (PL) combiner for  multiple relays is presented. Previous results were available only for upto three relays. The algorithm is unique in the sense that  the actual mathematical expressions, that are prohibitively large, need not be explicitly obtained. The diversity gain due to multiple relays is shown through plots of the analytical BER, well supported by simulations. 
%
%\end{abstract}
% IEEEtran.cls defaults to using nonbold math in the Abstract.
% This preserves the distinction between vectors and scalars. However,
% if the journal you are submitting to favors bold math in the abstract,
% then you can use LaTeX's standard command \boldmath at the very start
% of the abstract to achieve this. Many IEEE journals frown on math
% in the abstract anyway.

% Note that keywords are not normally used for peerreview papers.
%\begin{IEEEkeywords}
%Cooperative diversity, decode and forward, piecewise linear
%\end{IEEEkeywords}



% For peer review papers, you can put extra information on the cover
% page as needed:
% \ifCLASSOPTIONpeerreview
% \begin{center} \bfseries EDICS Category: 3-BBND \end{center}
% \fi
%
% For peerreview papers, this IEEEtran command inserts a page break and
% creates the second title. It will be ignored for other modes.
%\IEEEpeerreviewmaketitle




\item Four cards are drawn from a well-shuffled deck of 52 cards. What is the probability of obtaining 3 diamonds and one spade.
\\
\solution
		%\begin{enumerate}[label=\thesection.\arabic*,ref=\thesection.\theenumi]
	\item One card is drawn from a well-shuffled deck of 52 cards. Find the probability of getting
\begin{enumerate}
\item A king of red colour 
\item A face card 
\item A red face card
\item The jack of hearts
\item A spade
\item The queen of diamonds

\end{enumerate}
\solution
		%\input{ncert/10/15/1/14/main.tex}
	\item Five cards—the ten, jack, queen, king and ace of diamonds, are well-shuffled with their face downwards. One card is then picked up at random.
\begin{enumerate}
\item
What is the probability that the card is the queen? 
\item
If the queen is drawn and put aside, what is the probability that the second card picked up is (a) an ace? (b) a queen?\\
\end{enumerate}
\solution
		%\input{ncert/10/15/1/15/defs.tex}
	\item A bag contains $5$ red balls and some blue balls. If the probability of drawing a blue ball is double that if a red ball, determine the number of blue balls in the bag. 
		\\
\solution
		%\input{ncert/10/15/2/3/defs.tex}
	\item A card is selected from a pack of 52 cards.
 \begin{enumerate}[label=(\alph*)] 
                 \item How many points are there in the sample space?
                 \item Calculate the probability that the card is an ace of spades.
                 \item Calculate the probability that the card is (i) an ace and (ii) black card.
 \end{enumerate}
\solution
		%\input{ncert/11/16/3/4/main.tex}
\item Four cards are drawn from a well-shuffled deck of 52 cards. What is the probability of obtaining 3 diamonds and one spade.
\\
\solution
		%\input{ncert/11/16/4/2/defs.tex}
\item In a certain lottery 10,000 tickets are sold and ten equal prizes are awarded. What is the probability of not getting a prize if you buy (a) one ticket (b) two tickets (c) 10 tickets ?	
\\
\solution
		%\input{ncert/11/16/4/4/defs.tex}
		%
\item 
Out of 100 students, two sections of 40 and 60 are formed. If you and your friend are among the 100 students, what is the probability that
\begin{enumerate}
\item you both enter the same section?
\item you both enter the different sections?
\end{enumerate}
\solution
		%\input{ncert/11/16/4/5/defs.tex}
	\item 
The number lock of a suitcase has 4 wheels each labelled with ten digits i.e. from 0 to 9.The lock opens with a sequence of four digits with no repeats.What is the probability of a person getting the right sequence to open the suitcase.
\\
\solution
		%\input{ncert/11/16/4/10/defs.tex}
		%
\item 
Two cards are drawn at random and without replacement from a pack of 52 playing cards. Find the probability that both the cards are black.
\\
\solution
		%\input{ncert/12/13/2/2/defs.tex}
		\item A box of oranges is inspected by examining three randomly selected oranges drawn without replacement. If all the three oranges are good, the box is approved for sale, otherwise, it is rejected. Find the probability that a box containing 15 oranges out of which 12 are good and 3 are bad ones will be approved for sale.
		\label{ncert/12/13/2/3/defs.tex}
		\item Two balls are drawn at random with replacement from a box containing 10 black and 8 red balls. Find the probability that
		\label{ncert/12/13/2/12}
\begin{enumerate}
\item both balls are red.
\item first ball is black and second is red.
\item one of them is black and other is red.
\end{enumerate}

\item In a hostel, 60\% of the students read Hindi newspaper, 40\% read English newspaper and 20\% read both Hindi and English newspapers. A student is selected at random.
		\label{ncert/12/13/2/15}
\begin{enumerate}
\item Find the probability that she reads neither Hindi nor English newspapers.
\item If she reads Hindi newspaper, find the probability that she reads English newspaper.
\item If she reads English newspaper, find the probability that she reads Hindi newspaper.\\
\end{enumerate}
\item The probability of obtaining an even prime number on each die, when a pair of dice is rolled is 
\begin{enumerate}
    \item $0$ 
    
    \item $\frac{1}{3}$ 
    
    \item $\frac{1}{12}$ 
    
    \item $\frac{1}{36}$ 
\end{enumerate}
\solution
		%\input{ncert/12/13/2/17/defs.tex}
	\item A bag contains 4 red and 4 black balls, another bag contains 2 red and 6 black balls. One of the two bags is selected at random and a ball is drawn from the bag which is found to be red. Find the probability that the ball is drawn from the first bag.
\\
\solution
		%\input{ncert/12/13/3/2/main.tex}
  \item
  Cards with numbers 2 to 101 are placed in a box. A card is selected at random.Find the probability that the card has
\begin{enumerate}[label=(\roman*)]
	\item an even number 
	\item a square number
\end{enumerate}
\solution
%\input{exemplar/10/13/3/32/main.tex}
\item
The king, queen and jack of clubs are removed from a deck of 52 playing cards and then well shuffled. Now one card is drawn at random from the remaining cards.  Determine the probability that the card is
\begin{enumerate}[label=(\roman*)]
\item a club
\item 10 of hearts
\end{enumerate}
\solution
%\input{exemplar/10/13/3/29/main.tex}
\item A team of medical students doing their internship have to assist during surgeries
at a city hospital. The probabilities of surgeries rated as very complex, complex,
routine, simple or very simple are respectively, 0.15, 0.20, 0.31, 0.26, .08. Find
the probabilities that a particular surgery will be rated
\begin{enumerate}
	\item complex or very complex;
	\item neither very complex nor very simple;
	\item routine or complex
	\item routine or simple
\end{enumerate}
\solution
%\input{exemplar/11/16/3/8(1)/main.tex}
\item A card is selected from a pack of 52 cards.
\begin{enumerate}[label=(\alph*)]
    \item How many points are there in the sample space?
    \item Calculate the probability that the card is an ace of spades.
    \item Calculate the probability that the card is (i) an ace and (ii) black card.
\end{enumerate}
\solution
%\input{exemplar/11/16/3/4/main2.tex}
\item The probability that a non leap year selected at random will contain 53 sundays.
\\
\solution
%\input{exemplar/10/13/1/19/main.tex}
\item One of the four persons John, Rita, Aslam or Gurpreet will be promoted next
month. Consequently the sample space consists of four elementary outcomes
S = {John promoted, Rita promoted, Aslam promoted, Gurpreet promoted}
You are told that the chances of John’s promotion is same as that of Gurpreet,
Rita’s chances of promotion are twice as likely as Johns. Aslam’s chances are
four times that of John.
\begin{enumerate}
	\item Determine
	\begin{enumerate}
		\item P (John promoted)
		\item P (Rita promoted)
		\item P (Aslam promoted)
		\item P (Gurpreet promoted)
	\end{enumerate}
	\item If A = {John promoted or Gurpreet promoted}, find P (A).
\end{enumerate}
\solution
%\input{exemplar/11/16/3/10/main.tex}
\item A card is drawn from a deck of 52 cards. Find the probability of getting a king or a heart or a red card.\\
\solution
%\input{exemplar/11/16/3/15/main.tex}
\item The probability that a student will pass his examination is 0.73, the probability of
the student getting a compartment is 0.13, and the probability that the student will
either pass or get compartment is 0.96. State True or False.\\
\solution
%\input{exemplar/11/16/3/31/main.tex}
\item A card is selected from a pack of 52 cards\\
\begin{enumerate}[label=(\alph*)]
\item How many points are there in the sample space?
\item Calculate the probability that the cards is an ace of spades.
\item Calculate the probability that the card is (i) an ace (ii)black card.\\
\end{enumerate}
%\input{ncert/11/16/3/4_1/Prob_4.tex}
\item In a non-leap year, the probability of having 53 tuesdays or 53 wednesdays is\\
\solution
%\input{exemplar/11/16/3/18/main.tex}
\item There are 1000 sealed envelopes in a box, 10 of them contain a cash prize of
Rs 100 each, 100 of them contain a cash prize of Rs 50 each and 200 of them
contain a cash prize of Rs 10 each and rest do not contain any cash prize. If they
are well shuffled and an envelope is picked up out, what is the probability that it
contains no cash prize?\\
\solution
%\input{exemplar/10/13/3/34/main.tex}
\item 
A die is thrown and a card is selected at random from a deck of 52 playing cards. The probability of getting an even number on the die and a spade card.\\
\solution
%\input{exemplar/12/13/3/78/main.tex}
\item
If 4-digit numbers greater than 5,000 are randomly formed from the digits 0, 1, 3, 5, and 7, what is the probability of forming a number divisible by 5 when:
\begin{enumerate}
    \item The digits are repeated?
    \item The repetition of digits is not allowed?
\end{enumerate}
\solution
%\input{ncert/11/16/4/9/main.tex}
\item Consider the probability space $\brak{\Omega, \mathcal{G}, P}$ where $\Omega = [0,2]$ and $\mathcal{G} = \cbrak{\phi, \Omega, [0,1], (1,2]}$. Let $X$ and $Y$ be two functions on $\Omega$ defined as
\begin{align*}
    X(\omega) = 
    \begin{cases}
        1 & \text{if }\omega \in [0, 1]\\
        2 & \text{if }\omega \in (1, 2]
    \end{cases}
\end{align*}
and
\begin{align*}
    Y(\omega) = 
    \begin{cases}
        2 & \text{if }\omega \in [0, 1.5]\\
        3 & \text{if }\omega \in (1.5, 2].
    \end{cases}
\end{align*}
Then which one of the following statements is true?
\begin{enumerate}
    \item [(A)] $X$ is a random variable with respect to $\mathcal{G}$, but $Y$ is not a random variable with respect to $\mathcal{G}$.
    \item [(B)] $Y$ is a random variable with respect to $\mathcal{G}$, but $X$ is not a random variable with respect to $\mathcal{G}$.
    \item [(C)] Neither $X$ nor $Y$ is a random variable with respect to $\mathcal{G}$.
    \item [(D)] Both $X$ and $Y$ are random variables with respect to $\mathcal{G}$.
\end{enumerate} \hfill (GATE ST 2023)\\
\solution
%\input{gate/ST/2023/14/main.tex}
	\item  A die is loaded in such a way that each odd number is twice as likely to occur as
each even number. Find $P(G)$, where $G$ is the event that a number greater than
3 occurs on a single roll of the die.
\\
\solution
		%\input{exemplar/11/16/3/5/main.tex}
	\item All the jacks, queens and kings are removed from a deck of 52 playing cards. The remaining cards are well shuffled and then one card is drawn at random. Giving ace a value 1 similar value for other cards, find the probability that the card has a value 
		\begin{enumerate}
			\item 7
			\item greater than 7
			\item less than 7
		\end{enumerate}
		%\input{exemplar/10/13/3/30/main.tex}
  \item A Lot consists of 48 mobile phones of which 42 are good, 3 have only minor defects and 3 have major defects.Varnika will buy a phone if it is good but the trader will only buy a mobile if it has no major defects. One phone is selected at random from the lot. What is the probability that it is
\begin{enumerate}
	\item acceptable to Varnika?
            \item acceptable to the trader?
\end{enumerate}
\solution
	%\input{exemplar/10/13/3/40/main.tex}
 \item A student says that if you throw a die, it will show up 1 or not 1. Therefore, the probability of getting 1 and the probability of getting 'not 1' each is equal to $\frac{1}{2}$. Is this correct? Give reasons.\\
 \solution
        %\input{exemplar/10/13/2/9/main.tex}
   \item Four candidates A, B, C, D have ap-
plied for the assignment to coach a school cricket
team. If A is twice as likely to be selected as B, and
B and C are given about the same chance of being
selected, while C is twice as likely to be selected
as D, what are the probabilities that
\begin{enumerate}
\item C will be selected?
\item A will not be selected?
\end{enumerate}
	%\input{exemplar/11/16/3/9/main.tex}
 \item A bag contain 24 balls of which $x$ balls are red, $2x$ are white and $3x$ are blue. A ball is selected at random, What is the probability that it is
\begin{enumerate}[label=\alph*)]
\item not red ?
\item white ?
\end{enumerate}
%\input{exemplar/10/13/3/41/main.tex}
If the letters of the word ASSASSINATION are arranged at random. Find the Probability that
\begin{enumerate}[label=(\alph*)]
\item Four $S's$ come consecutively in the word
\item Two  $I's$ and two $N's$ come together
\item All $A's$ are not coming together
\item No two $A's$ are coming together
\end{enumerate}
%\input{exemplar/11/16/3/14/main.tex}
	\item One urn contains two black balls (labelled B1 and B2) and one white ball. A
	second urn contains one black ball and two white balls (labelled W1 and W2).
	Suppose the following experiment is performed. One of the two urns is chosen
	at random. Next a ball is randomly chosen from the urn. Then a second ball is
	chosen at random from the same urn without replacing the first ball.
	
	\begin{enumerate}
	\item What is the probability that two black balls are chosen?
	
	\item What is the probability that two balls of opposite colour are chosen?
	\end{enumerate}
	\solution
	%\input{exemplar/11/16/3/12/main1.tex}
\end{enumerate}

\item In a certain lottery 10,000 tickets are sold and ten equal prizes are awarded. What is the probability of not getting a prize if you buy (a) one ticket (b) two tickets (c) 10 tickets ?	
\\
\solution
		%\begin{enumerate}[label=\thesection.\arabic*,ref=\thesection.\theenumi]
	\item One card is drawn from a well-shuffled deck of 52 cards. Find the probability of getting
\begin{enumerate}
\item A king of red colour 
\item A face card 
\item A red face card
\item The jack of hearts
\item A spade
\item The queen of diamonds

\end{enumerate}
\solution
		%\input{ncert/10/15/1/14/main.tex}
	\item Five cards—the ten, jack, queen, king and ace of diamonds, are well-shuffled with their face downwards. One card is then picked up at random.
\begin{enumerate}
\item
What is the probability that the card is the queen? 
\item
If the queen is drawn and put aside, what is the probability that the second card picked up is (a) an ace? (b) a queen?\\
\end{enumerate}
\solution
		%\input{ncert/10/15/1/15/defs.tex}
	\item A bag contains $5$ red balls and some blue balls. If the probability of drawing a blue ball is double that if a red ball, determine the number of blue balls in the bag. 
		\\
\solution
		%\input{ncert/10/15/2/3/defs.tex}
	\item A card is selected from a pack of 52 cards.
 \begin{enumerate}[label=(\alph*)] 
                 \item How many points are there in the sample space?
                 \item Calculate the probability that the card is an ace of spades.
                 \item Calculate the probability that the card is (i) an ace and (ii) black card.
 \end{enumerate}
\solution
		%\input{ncert/11/16/3/4/main.tex}
\item Four cards are drawn from a well-shuffled deck of 52 cards. What is the probability of obtaining 3 diamonds and one spade.
\\
\solution
		%\input{ncert/11/16/4/2/defs.tex}
\item In a certain lottery 10,000 tickets are sold and ten equal prizes are awarded. What is the probability of not getting a prize if you buy (a) one ticket (b) two tickets (c) 10 tickets ?	
\\
\solution
		%\input{ncert/11/16/4/4/defs.tex}
		%
\item 
Out of 100 students, two sections of 40 and 60 are formed. If you and your friend are among the 100 students, what is the probability that
\begin{enumerate}
\item you both enter the same section?
\item you both enter the different sections?
\end{enumerate}
\solution
		%\input{ncert/11/16/4/5/defs.tex}
	\item 
The number lock of a suitcase has 4 wheels each labelled with ten digits i.e. from 0 to 9.The lock opens with a sequence of four digits with no repeats.What is the probability of a person getting the right sequence to open the suitcase.
\\
\solution
		%\input{ncert/11/16/4/10/defs.tex}
		%
\item 
Two cards are drawn at random and without replacement from a pack of 52 playing cards. Find the probability that both the cards are black.
\\
\solution
		%\input{ncert/12/13/2/2/defs.tex}
		\item A box of oranges is inspected by examining three randomly selected oranges drawn without replacement. If all the three oranges are good, the box is approved for sale, otherwise, it is rejected. Find the probability that a box containing 15 oranges out of which 12 are good and 3 are bad ones will be approved for sale.
		\label{ncert/12/13/2/3/defs.tex}
		\item Two balls are drawn at random with replacement from a box containing 10 black and 8 red balls. Find the probability that
		\label{ncert/12/13/2/12}
\begin{enumerate}
\item both balls are red.
\item first ball is black and second is red.
\item one of them is black and other is red.
\end{enumerate}

\item In a hostel, 60\% of the students read Hindi newspaper, 40\% read English newspaper and 20\% read both Hindi and English newspapers. A student is selected at random.
		\label{ncert/12/13/2/15}
\begin{enumerate}
\item Find the probability that she reads neither Hindi nor English newspapers.
\item If she reads Hindi newspaper, find the probability that she reads English newspaper.
\item If she reads English newspaper, find the probability that she reads Hindi newspaper.\\
\end{enumerate}
\item The probability of obtaining an even prime number on each die, when a pair of dice is rolled is 
\begin{enumerate}
    \item $0$ 
    
    \item $\frac{1}{3}$ 
    
    \item $\frac{1}{12}$ 
    
    \item $\frac{1}{36}$ 
\end{enumerate}
\solution
		%\input{ncert/12/13/2/17/defs.tex}
	\item A bag contains 4 red and 4 black balls, another bag contains 2 red and 6 black balls. One of the two bags is selected at random and a ball is drawn from the bag which is found to be red. Find the probability that the ball is drawn from the first bag.
\\
\solution
		%\input{ncert/12/13/3/2/main.tex}
  \item
  Cards with numbers 2 to 101 are placed in a box. A card is selected at random.Find the probability that the card has
\begin{enumerate}[label=(\roman*)]
	\item an even number 
	\item a square number
\end{enumerate}
\solution
%\input{exemplar/10/13/3/32/main.tex}
\item
The king, queen and jack of clubs are removed from a deck of 52 playing cards and then well shuffled. Now one card is drawn at random from the remaining cards.  Determine the probability that the card is
\begin{enumerate}[label=(\roman*)]
\item a club
\item 10 of hearts
\end{enumerate}
\solution
%\input{exemplar/10/13/3/29/main.tex}
\item A team of medical students doing their internship have to assist during surgeries
at a city hospital. The probabilities of surgeries rated as very complex, complex,
routine, simple or very simple are respectively, 0.15, 0.20, 0.31, 0.26, .08. Find
the probabilities that a particular surgery will be rated
\begin{enumerate}
	\item complex or very complex;
	\item neither very complex nor very simple;
	\item routine or complex
	\item routine or simple
\end{enumerate}
\solution
%\input{exemplar/11/16/3/8(1)/main.tex}
\item A card is selected from a pack of 52 cards.
\begin{enumerate}[label=(\alph*)]
    \item How many points are there in the sample space?
    \item Calculate the probability that the card is an ace of spades.
    \item Calculate the probability that the card is (i) an ace and (ii) black card.
\end{enumerate}
\solution
%\input{exemplar/11/16/3/4/main2.tex}
\item The probability that a non leap year selected at random will contain 53 sundays.
\\
\solution
%\input{exemplar/10/13/1/19/main.tex}
\item One of the four persons John, Rita, Aslam or Gurpreet will be promoted next
month. Consequently the sample space consists of four elementary outcomes
S = {John promoted, Rita promoted, Aslam promoted, Gurpreet promoted}
You are told that the chances of John’s promotion is same as that of Gurpreet,
Rita’s chances of promotion are twice as likely as Johns. Aslam’s chances are
four times that of John.
\begin{enumerate}
	\item Determine
	\begin{enumerate}
		\item P (John promoted)
		\item P (Rita promoted)
		\item P (Aslam promoted)
		\item P (Gurpreet promoted)
	\end{enumerate}
	\item If A = {John promoted or Gurpreet promoted}, find P (A).
\end{enumerate}
\solution
%\input{exemplar/11/16/3/10/main.tex}
\item A card is drawn from a deck of 52 cards. Find the probability of getting a king or a heart or a red card.\\
\solution
%\input{exemplar/11/16/3/15/main.tex}
\item The probability that a student will pass his examination is 0.73, the probability of
the student getting a compartment is 0.13, and the probability that the student will
either pass or get compartment is 0.96. State True or False.\\
\solution
%\input{exemplar/11/16/3/31/main.tex}
\item A card is selected from a pack of 52 cards\\
\begin{enumerate}[label=(\alph*)]
\item How many points are there in the sample space?
\item Calculate the probability that the cards is an ace of spades.
\item Calculate the probability that the card is (i) an ace (ii)black card.\\
\end{enumerate}
%\input{ncert/11/16/3/4_1/Prob_4.tex}
\item In a non-leap year, the probability of having 53 tuesdays or 53 wednesdays is\\
\solution
%\input{exemplar/11/16/3/18/main.tex}
\item There are 1000 sealed envelopes in a box, 10 of them contain a cash prize of
Rs 100 each, 100 of them contain a cash prize of Rs 50 each and 200 of them
contain a cash prize of Rs 10 each and rest do not contain any cash prize. If they
are well shuffled and an envelope is picked up out, what is the probability that it
contains no cash prize?\\
\solution
%\input{exemplar/10/13/3/34/main.tex}
\item 
A die is thrown and a card is selected at random from a deck of 52 playing cards. The probability of getting an even number on the die and a spade card.\\
\solution
%\input{exemplar/12/13/3/78/main.tex}
\item
If 4-digit numbers greater than 5,000 are randomly formed from the digits 0, 1, 3, 5, and 7, what is the probability of forming a number divisible by 5 when:
\begin{enumerate}
    \item The digits are repeated?
    \item The repetition of digits is not allowed?
\end{enumerate}
\solution
%\input{ncert/11/16/4/9/main.tex}
\item Consider the probability space $\brak{\Omega, \mathcal{G}, P}$ where $\Omega = [0,2]$ and $\mathcal{G} = \cbrak{\phi, \Omega, [0,1], (1,2]}$. Let $X$ and $Y$ be two functions on $\Omega$ defined as
\begin{align*}
    X(\omega) = 
    \begin{cases}
        1 & \text{if }\omega \in [0, 1]\\
        2 & \text{if }\omega \in (1, 2]
    \end{cases}
\end{align*}
and
\begin{align*}
    Y(\omega) = 
    \begin{cases}
        2 & \text{if }\omega \in [0, 1.5]\\
        3 & \text{if }\omega \in (1.5, 2].
    \end{cases}
\end{align*}
Then which one of the following statements is true?
\begin{enumerate}
    \item [(A)] $X$ is a random variable with respect to $\mathcal{G}$, but $Y$ is not a random variable with respect to $\mathcal{G}$.
    \item [(B)] $Y$ is a random variable with respect to $\mathcal{G}$, but $X$ is not a random variable with respect to $\mathcal{G}$.
    \item [(C)] Neither $X$ nor $Y$ is a random variable with respect to $\mathcal{G}$.
    \item [(D)] Both $X$ and $Y$ are random variables with respect to $\mathcal{G}$.
\end{enumerate} \hfill (GATE ST 2023)\\
\solution
%\input{gate/ST/2023/14/main.tex}
	\item  A die is loaded in such a way that each odd number is twice as likely to occur as
each even number. Find $P(G)$, where $G$ is the event that a number greater than
3 occurs on a single roll of the die.
\\
\solution
		%\input{exemplar/11/16/3/5/main.tex}
	\item All the jacks, queens and kings are removed from a deck of 52 playing cards. The remaining cards are well shuffled and then one card is drawn at random. Giving ace a value 1 similar value for other cards, find the probability that the card has a value 
		\begin{enumerate}
			\item 7
			\item greater than 7
			\item less than 7
		\end{enumerate}
		%\input{exemplar/10/13/3/30/main.tex}
  \item A Lot consists of 48 mobile phones of which 42 are good, 3 have only minor defects and 3 have major defects.Varnika will buy a phone if it is good but the trader will only buy a mobile if it has no major defects. One phone is selected at random from the lot. What is the probability that it is
\begin{enumerate}
	\item acceptable to Varnika?
            \item acceptable to the trader?
\end{enumerate}
\solution
	%\input{exemplar/10/13/3/40/main.tex}
 \item A student says that if you throw a die, it will show up 1 or not 1. Therefore, the probability of getting 1 and the probability of getting 'not 1' each is equal to $\frac{1}{2}$. Is this correct? Give reasons.\\
 \solution
        %\input{exemplar/10/13/2/9/main.tex}
   \item Four candidates A, B, C, D have ap-
plied for the assignment to coach a school cricket
team. If A is twice as likely to be selected as B, and
B and C are given about the same chance of being
selected, while C is twice as likely to be selected
as D, what are the probabilities that
\begin{enumerate}
\item C will be selected?
\item A will not be selected?
\end{enumerate}
	%\input{exemplar/11/16/3/9/main.tex}
 \item A bag contain 24 balls of which $x$ balls are red, $2x$ are white and $3x$ are blue. A ball is selected at random, What is the probability that it is
\begin{enumerate}[label=\alph*)]
\item not red ?
\item white ?
\end{enumerate}
%\input{exemplar/10/13/3/41/main.tex}
If the letters of the word ASSASSINATION are arranged at random. Find the Probability that
\begin{enumerate}[label=(\alph*)]
\item Four $S's$ come consecutively in the word
\item Two  $I's$ and two $N's$ come together
\item All $A's$ are not coming together
\item No two $A's$ are coming together
\end{enumerate}
%\input{exemplar/11/16/3/14/main.tex}
	\item One urn contains two black balls (labelled B1 and B2) and one white ball. A
	second urn contains one black ball and two white balls (labelled W1 and W2).
	Suppose the following experiment is performed. One of the two urns is chosen
	at random. Next a ball is randomly chosen from the urn. Then a second ball is
	chosen at random from the same urn without replacing the first ball.
	
	\begin{enumerate}
	\item What is the probability that two black balls are chosen?
	
	\item What is the probability that two balls of opposite colour are chosen?
	\end{enumerate}
	\solution
	%\input{exemplar/11/16/3/12/main1.tex}
\end{enumerate}

		%
\item 
Out of 100 students, two sections of 40 and 60 are formed. If you and your friend are among the 100 students, what is the probability that
\begin{enumerate}
\item you both enter the same section?
\item you both enter the different sections?
\end{enumerate}
\solution
		%\begin{enumerate}[label=\thesection.\arabic*,ref=\thesection.\theenumi]
	\item One card is drawn from a well-shuffled deck of 52 cards. Find the probability of getting
\begin{enumerate}
\item A king of red colour 
\item A face card 
\item A red face card
\item The jack of hearts
\item A spade
\item The queen of diamonds

\end{enumerate}
\solution
		%\input{ncert/10/15/1/14/main.tex}
	\item Five cards—the ten, jack, queen, king and ace of diamonds, are well-shuffled with their face downwards. One card is then picked up at random.
\begin{enumerate}
\item
What is the probability that the card is the queen? 
\item
If the queen is drawn and put aside, what is the probability that the second card picked up is (a) an ace? (b) a queen?\\
\end{enumerate}
\solution
		%\input{ncert/10/15/1/15/defs.tex}
	\item A bag contains $5$ red balls and some blue balls. If the probability of drawing a blue ball is double that if a red ball, determine the number of blue balls in the bag. 
		\\
\solution
		%\input{ncert/10/15/2/3/defs.tex}
	\item A card is selected from a pack of 52 cards.
 \begin{enumerate}[label=(\alph*)] 
                 \item How many points are there in the sample space?
                 \item Calculate the probability that the card is an ace of spades.
                 \item Calculate the probability that the card is (i) an ace and (ii) black card.
 \end{enumerate}
\solution
		%\input{ncert/11/16/3/4/main.tex}
\item Four cards are drawn from a well-shuffled deck of 52 cards. What is the probability of obtaining 3 diamonds and one spade.
\\
\solution
		%\input{ncert/11/16/4/2/defs.tex}
\item In a certain lottery 10,000 tickets are sold and ten equal prizes are awarded. What is the probability of not getting a prize if you buy (a) one ticket (b) two tickets (c) 10 tickets ?	
\\
\solution
		%\input{ncert/11/16/4/4/defs.tex}
		%
\item 
Out of 100 students, two sections of 40 and 60 are formed. If you and your friend are among the 100 students, what is the probability that
\begin{enumerate}
\item you both enter the same section?
\item you both enter the different sections?
\end{enumerate}
\solution
		%\input{ncert/11/16/4/5/defs.tex}
	\item 
The number lock of a suitcase has 4 wheels each labelled with ten digits i.e. from 0 to 9.The lock opens with a sequence of four digits with no repeats.What is the probability of a person getting the right sequence to open the suitcase.
\\
\solution
		%\input{ncert/11/16/4/10/defs.tex}
		%
\item 
Two cards are drawn at random and without replacement from a pack of 52 playing cards. Find the probability that both the cards are black.
\\
\solution
		%\input{ncert/12/13/2/2/defs.tex}
		\item A box of oranges is inspected by examining three randomly selected oranges drawn without replacement. If all the three oranges are good, the box is approved for sale, otherwise, it is rejected. Find the probability that a box containing 15 oranges out of which 12 are good and 3 are bad ones will be approved for sale.
		\label{ncert/12/13/2/3/defs.tex}
		\item Two balls are drawn at random with replacement from a box containing 10 black and 8 red balls. Find the probability that
		\label{ncert/12/13/2/12}
\begin{enumerate}
\item both balls are red.
\item first ball is black and second is red.
\item one of them is black and other is red.
\end{enumerate}

\item In a hostel, 60\% of the students read Hindi newspaper, 40\% read English newspaper and 20\% read both Hindi and English newspapers. A student is selected at random.
		\label{ncert/12/13/2/15}
\begin{enumerate}
\item Find the probability that she reads neither Hindi nor English newspapers.
\item If she reads Hindi newspaper, find the probability that she reads English newspaper.
\item If she reads English newspaper, find the probability that she reads Hindi newspaper.\\
\end{enumerate}
\item The probability of obtaining an even prime number on each die, when a pair of dice is rolled is 
\begin{enumerate}
    \item $0$ 
    
    \item $\frac{1}{3}$ 
    
    \item $\frac{1}{12}$ 
    
    \item $\frac{1}{36}$ 
\end{enumerate}
\solution
		%\input{ncert/12/13/2/17/defs.tex}
	\item A bag contains 4 red and 4 black balls, another bag contains 2 red and 6 black balls. One of the two bags is selected at random and a ball is drawn from the bag which is found to be red. Find the probability that the ball is drawn from the first bag.
\\
\solution
		%\input{ncert/12/13/3/2/main.tex}
  \item
  Cards with numbers 2 to 101 are placed in a box. A card is selected at random.Find the probability that the card has
\begin{enumerate}[label=(\roman*)]
	\item an even number 
	\item a square number
\end{enumerate}
\solution
%\input{exemplar/10/13/3/32/main.tex}
\item
The king, queen and jack of clubs are removed from a deck of 52 playing cards and then well shuffled. Now one card is drawn at random from the remaining cards.  Determine the probability that the card is
\begin{enumerate}[label=(\roman*)]
\item a club
\item 10 of hearts
\end{enumerate}
\solution
%\input{exemplar/10/13/3/29/main.tex}
\item A team of medical students doing their internship have to assist during surgeries
at a city hospital. The probabilities of surgeries rated as very complex, complex,
routine, simple or very simple are respectively, 0.15, 0.20, 0.31, 0.26, .08. Find
the probabilities that a particular surgery will be rated
\begin{enumerate}
	\item complex or very complex;
	\item neither very complex nor very simple;
	\item routine or complex
	\item routine or simple
\end{enumerate}
\solution
%\input{exemplar/11/16/3/8(1)/main.tex}
\item A card is selected from a pack of 52 cards.
\begin{enumerate}[label=(\alph*)]
    \item How many points are there in the sample space?
    \item Calculate the probability that the card is an ace of spades.
    \item Calculate the probability that the card is (i) an ace and (ii) black card.
\end{enumerate}
\solution
%\input{exemplar/11/16/3/4/main2.tex}
\item The probability that a non leap year selected at random will contain 53 sundays.
\\
\solution
%\input{exemplar/10/13/1/19/main.tex}
\item One of the four persons John, Rita, Aslam or Gurpreet will be promoted next
month. Consequently the sample space consists of four elementary outcomes
S = {John promoted, Rita promoted, Aslam promoted, Gurpreet promoted}
You are told that the chances of John’s promotion is same as that of Gurpreet,
Rita’s chances of promotion are twice as likely as Johns. Aslam’s chances are
four times that of John.
\begin{enumerate}
	\item Determine
	\begin{enumerate}
		\item P (John promoted)
		\item P (Rita promoted)
		\item P (Aslam promoted)
		\item P (Gurpreet promoted)
	\end{enumerate}
	\item If A = {John promoted or Gurpreet promoted}, find P (A).
\end{enumerate}
\solution
%\input{exemplar/11/16/3/10/main.tex}
\item A card is drawn from a deck of 52 cards. Find the probability of getting a king or a heart or a red card.\\
\solution
%\input{exemplar/11/16/3/15/main.tex}
\item The probability that a student will pass his examination is 0.73, the probability of
the student getting a compartment is 0.13, and the probability that the student will
either pass or get compartment is 0.96. State True or False.\\
\solution
%\input{exemplar/11/16/3/31/main.tex}
\item A card is selected from a pack of 52 cards\\
\begin{enumerate}[label=(\alph*)]
\item How many points are there in the sample space?
\item Calculate the probability that the cards is an ace of spades.
\item Calculate the probability that the card is (i) an ace (ii)black card.\\
\end{enumerate}
%\input{ncert/11/16/3/4_1/Prob_4.tex}
\item In a non-leap year, the probability of having 53 tuesdays or 53 wednesdays is\\
\solution
%\input{exemplar/11/16/3/18/main.tex}
\item There are 1000 sealed envelopes in a box, 10 of them contain a cash prize of
Rs 100 each, 100 of them contain a cash prize of Rs 50 each and 200 of them
contain a cash prize of Rs 10 each and rest do not contain any cash prize. If they
are well shuffled and an envelope is picked up out, what is the probability that it
contains no cash prize?\\
\solution
%\input{exemplar/10/13/3/34/main.tex}
\item 
A die is thrown and a card is selected at random from a deck of 52 playing cards. The probability of getting an even number on the die and a spade card.\\
\solution
%\input{exemplar/12/13/3/78/main.tex}
\item
If 4-digit numbers greater than 5,000 are randomly formed from the digits 0, 1, 3, 5, and 7, what is the probability of forming a number divisible by 5 when:
\begin{enumerate}
    \item The digits are repeated?
    \item The repetition of digits is not allowed?
\end{enumerate}
\solution
%\input{ncert/11/16/4/9/main.tex}
\item Consider the probability space $\brak{\Omega, \mathcal{G}, P}$ where $\Omega = [0,2]$ and $\mathcal{G} = \cbrak{\phi, \Omega, [0,1], (1,2]}$. Let $X$ and $Y$ be two functions on $\Omega$ defined as
\begin{align*}
    X(\omega) = 
    \begin{cases}
        1 & \text{if }\omega \in [0, 1]\\
        2 & \text{if }\omega \in (1, 2]
    \end{cases}
\end{align*}
and
\begin{align*}
    Y(\omega) = 
    \begin{cases}
        2 & \text{if }\omega \in [0, 1.5]\\
        3 & \text{if }\omega \in (1.5, 2].
    \end{cases}
\end{align*}
Then which one of the following statements is true?
\begin{enumerate}
    \item [(A)] $X$ is a random variable with respect to $\mathcal{G}$, but $Y$ is not a random variable with respect to $\mathcal{G}$.
    \item [(B)] $Y$ is a random variable with respect to $\mathcal{G}$, but $X$ is not a random variable with respect to $\mathcal{G}$.
    \item [(C)] Neither $X$ nor $Y$ is a random variable with respect to $\mathcal{G}$.
    \item [(D)] Both $X$ and $Y$ are random variables with respect to $\mathcal{G}$.
\end{enumerate} \hfill (GATE ST 2023)\\
\solution
%\input{gate/ST/2023/14/main.tex}
	\item  A die is loaded in such a way that each odd number is twice as likely to occur as
each even number. Find $P(G)$, where $G$ is the event that a number greater than
3 occurs on a single roll of the die.
\\
\solution
		%\input{exemplar/11/16/3/5/main.tex}
	\item All the jacks, queens and kings are removed from a deck of 52 playing cards. The remaining cards are well shuffled and then one card is drawn at random. Giving ace a value 1 similar value for other cards, find the probability that the card has a value 
		\begin{enumerate}
			\item 7
			\item greater than 7
			\item less than 7
		\end{enumerate}
		%\input{exemplar/10/13/3/30/main.tex}
  \item A Lot consists of 48 mobile phones of which 42 are good, 3 have only minor defects and 3 have major defects.Varnika will buy a phone if it is good but the trader will only buy a mobile if it has no major defects. One phone is selected at random from the lot. What is the probability that it is
\begin{enumerate}
	\item acceptable to Varnika?
            \item acceptable to the trader?
\end{enumerate}
\solution
	%\input{exemplar/10/13/3/40/main.tex}
 \item A student says that if you throw a die, it will show up 1 or not 1. Therefore, the probability of getting 1 and the probability of getting 'not 1' each is equal to $\frac{1}{2}$. Is this correct? Give reasons.\\
 \solution
        %\input{exemplar/10/13/2/9/main.tex}
   \item Four candidates A, B, C, D have ap-
plied for the assignment to coach a school cricket
team. If A is twice as likely to be selected as B, and
B and C are given about the same chance of being
selected, while C is twice as likely to be selected
as D, what are the probabilities that
\begin{enumerate}
\item C will be selected?
\item A will not be selected?
\end{enumerate}
	%\input{exemplar/11/16/3/9/main.tex}
 \item A bag contain 24 balls of which $x$ balls are red, $2x$ are white and $3x$ are blue. A ball is selected at random, What is the probability that it is
\begin{enumerate}[label=\alph*)]
\item not red ?
\item white ?
\end{enumerate}
%\input{exemplar/10/13/3/41/main.tex}
If the letters of the word ASSASSINATION are arranged at random. Find the Probability that
\begin{enumerate}[label=(\alph*)]
\item Four $S's$ come consecutively in the word
\item Two  $I's$ and two $N's$ come together
\item All $A's$ are not coming together
\item No two $A's$ are coming together
\end{enumerate}
%\input{exemplar/11/16/3/14/main.tex}
	\item One urn contains two black balls (labelled B1 and B2) and one white ball. A
	second urn contains one black ball and two white balls (labelled W1 and W2).
	Suppose the following experiment is performed. One of the two urns is chosen
	at random. Next a ball is randomly chosen from the urn. Then a second ball is
	chosen at random from the same urn without replacing the first ball.
	
	\begin{enumerate}
	\item What is the probability that two black balls are chosen?
	
	\item What is the probability that two balls of opposite colour are chosen?
	\end{enumerate}
	\solution
	%\input{exemplar/11/16/3/12/main1.tex}
\end{enumerate}

	\item 
The number lock of a suitcase has 4 wheels each labelled with ten digits i.e. from 0 to 9.The lock opens with a sequence of four digits with no repeats.What is the probability of a person getting the right sequence to open the suitcase.
\\
\solution
		%\begin{enumerate}[label=\thesection.\arabic*,ref=\thesection.\theenumi]
	\item One card is drawn from a well-shuffled deck of 52 cards. Find the probability of getting
\begin{enumerate}
\item A king of red colour 
\item A face card 
\item A red face card
\item The jack of hearts
\item A spade
\item The queen of diamonds

\end{enumerate}
\solution
		%\input{ncert/10/15/1/14/main.tex}
	\item Five cards—the ten, jack, queen, king and ace of diamonds, are well-shuffled with their face downwards. One card is then picked up at random.
\begin{enumerate}
\item
What is the probability that the card is the queen? 
\item
If the queen is drawn and put aside, what is the probability that the second card picked up is (a) an ace? (b) a queen?\\
\end{enumerate}
\solution
		%\input{ncert/10/15/1/15/defs.tex}
	\item A bag contains $5$ red balls and some blue balls. If the probability of drawing a blue ball is double that if a red ball, determine the number of blue balls in the bag. 
		\\
\solution
		%\input{ncert/10/15/2/3/defs.tex}
	\item A card is selected from a pack of 52 cards.
 \begin{enumerate}[label=(\alph*)] 
                 \item How many points are there in the sample space?
                 \item Calculate the probability that the card is an ace of spades.
                 \item Calculate the probability that the card is (i) an ace and (ii) black card.
 \end{enumerate}
\solution
		%\input{ncert/11/16/3/4/main.tex}
\item Four cards are drawn from a well-shuffled deck of 52 cards. What is the probability of obtaining 3 diamonds and one spade.
\\
\solution
		%\input{ncert/11/16/4/2/defs.tex}
\item In a certain lottery 10,000 tickets are sold and ten equal prizes are awarded. What is the probability of not getting a prize if you buy (a) one ticket (b) two tickets (c) 10 tickets ?	
\\
\solution
		%\input{ncert/11/16/4/4/defs.tex}
		%
\item 
Out of 100 students, two sections of 40 and 60 are formed. If you and your friend are among the 100 students, what is the probability that
\begin{enumerate}
\item you both enter the same section?
\item you both enter the different sections?
\end{enumerate}
\solution
		%\input{ncert/11/16/4/5/defs.tex}
	\item 
The number lock of a suitcase has 4 wheels each labelled with ten digits i.e. from 0 to 9.The lock opens with a sequence of four digits with no repeats.What is the probability of a person getting the right sequence to open the suitcase.
\\
\solution
		%\input{ncert/11/16/4/10/defs.tex}
		%
\item 
Two cards are drawn at random and without replacement from a pack of 52 playing cards. Find the probability that both the cards are black.
\\
\solution
		%\input{ncert/12/13/2/2/defs.tex}
		\item A box of oranges is inspected by examining three randomly selected oranges drawn without replacement. If all the three oranges are good, the box is approved for sale, otherwise, it is rejected. Find the probability that a box containing 15 oranges out of which 12 are good and 3 are bad ones will be approved for sale.
		\label{ncert/12/13/2/3/defs.tex}
		\item Two balls are drawn at random with replacement from a box containing 10 black and 8 red balls. Find the probability that
		\label{ncert/12/13/2/12}
\begin{enumerate}
\item both balls are red.
\item first ball is black and second is red.
\item one of them is black and other is red.
\end{enumerate}

\item In a hostel, 60\% of the students read Hindi newspaper, 40\% read English newspaper and 20\% read both Hindi and English newspapers. A student is selected at random.
		\label{ncert/12/13/2/15}
\begin{enumerate}
\item Find the probability that she reads neither Hindi nor English newspapers.
\item If she reads Hindi newspaper, find the probability that she reads English newspaper.
\item If she reads English newspaper, find the probability that she reads Hindi newspaper.\\
\end{enumerate}
\item The probability of obtaining an even prime number on each die, when a pair of dice is rolled is 
\begin{enumerate}
    \item $0$ 
    
    \item $\frac{1}{3}$ 
    
    \item $\frac{1}{12}$ 
    
    \item $\frac{1}{36}$ 
\end{enumerate}
\solution
		%\input{ncert/12/13/2/17/defs.tex}
	\item A bag contains 4 red and 4 black balls, another bag contains 2 red and 6 black balls. One of the two bags is selected at random and a ball is drawn from the bag which is found to be red. Find the probability that the ball is drawn from the first bag.
\\
\solution
		%\input{ncert/12/13/3/2/main.tex}
  \item
  Cards with numbers 2 to 101 are placed in a box. A card is selected at random.Find the probability that the card has
\begin{enumerate}[label=(\roman*)]
	\item an even number 
	\item a square number
\end{enumerate}
\solution
%\input{exemplar/10/13/3/32/main.tex}
\item
The king, queen and jack of clubs are removed from a deck of 52 playing cards and then well shuffled. Now one card is drawn at random from the remaining cards.  Determine the probability that the card is
\begin{enumerate}[label=(\roman*)]
\item a club
\item 10 of hearts
\end{enumerate}
\solution
%\input{exemplar/10/13/3/29/main.tex}
\item A team of medical students doing their internship have to assist during surgeries
at a city hospital. The probabilities of surgeries rated as very complex, complex,
routine, simple or very simple are respectively, 0.15, 0.20, 0.31, 0.26, .08. Find
the probabilities that a particular surgery will be rated
\begin{enumerate}
	\item complex or very complex;
	\item neither very complex nor very simple;
	\item routine or complex
	\item routine or simple
\end{enumerate}
\solution
%\input{exemplar/11/16/3/8(1)/main.tex}
\item A card is selected from a pack of 52 cards.
\begin{enumerate}[label=(\alph*)]
    \item How many points are there in the sample space?
    \item Calculate the probability that the card is an ace of spades.
    \item Calculate the probability that the card is (i) an ace and (ii) black card.
\end{enumerate}
\solution
%\input{exemplar/11/16/3/4/main2.tex}
\item The probability that a non leap year selected at random will contain 53 sundays.
\\
\solution
%\input{exemplar/10/13/1/19/main.tex}
\item One of the four persons John, Rita, Aslam or Gurpreet will be promoted next
month. Consequently the sample space consists of four elementary outcomes
S = {John promoted, Rita promoted, Aslam promoted, Gurpreet promoted}
You are told that the chances of John’s promotion is same as that of Gurpreet,
Rita’s chances of promotion are twice as likely as Johns. Aslam’s chances are
four times that of John.
\begin{enumerate}
	\item Determine
	\begin{enumerate}
		\item P (John promoted)
		\item P (Rita promoted)
		\item P (Aslam promoted)
		\item P (Gurpreet promoted)
	\end{enumerate}
	\item If A = {John promoted or Gurpreet promoted}, find P (A).
\end{enumerate}
\solution
%\input{exemplar/11/16/3/10/main.tex}
\item A card is drawn from a deck of 52 cards. Find the probability of getting a king or a heart or a red card.\\
\solution
%\input{exemplar/11/16/3/15/main.tex}
\item The probability that a student will pass his examination is 0.73, the probability of
the student getting a compartment is 0.13, and the probability that the student will
either pass or get compartment is 0.96. State True or False.\\
\solution
%\input{exemplar/11/16/3/31/main.tex}
\item A card is selected from a pack of 52 cards\\
\begin{enumerate}[label=(\alph*)]
\item How many points are there in the sample space?
\item Calculate the probability that the cards is an ace of spades.
\item Calculate the probability that the card is (i) an ace (ii)black card.\\
\end{enumerate}
%\input{ncert/11/16/3/4_1/Prob_4.tex}
\item In a non-leap year, the probability of having 53 tuesdays or 53 wednesdays is\\
\solution
%\input{exemplar/11/16/3/18/main.tex}
\item There are 1000 sealed envelopes in a box, 10 of them contain a cash prize of
Rs 100 each, 100 of them contain a cash prize of Rs 50 each and 200 of them
contain a cash prize of Rs 10 each and rest do not contain any cash prize. If they
are well shuffled and an envelope is picked up out, what is the probability that it
contains no cash prize?\\
\solution
%\input{exemplar/10/13/3/34/main.tex}
\item 
A die is thrown and a card is selected at random from a deck of 52 playing cards. The probability of getting an even number on the die and a spade card.\\
\solution
%\input{exemplar/12/13/3/78/main.tex}
\item
If 4-digit numbers greater than 5,000 are randomly formed from the digits 0, 1, 3, 5, and 7, what is the probability of forming a number divisible by 5 when:
\begin{enumerate}
    \item The digits are repeated?
    \item The repetition of digits is not allowed?
\end{enumerate}
\solution
%\input{ncert/11/16/4/9/main.tex}
\item Consider the probability space $\brak{\Omega, \mathcal{G}, P}$ where $\Omega = [0,2]$ and $\mathcal{G} = \cbrak{\phi, \Omega, [0,1], (1,2]}$. Let $X$ and $Y$ be two functions on $\Omega$ defined as
\begin{align*}
    X(\omega) = 
    \begin{cases}
        1 & \text{if }\omega \in [0, 1]\\
        2 & \text{if }\omega \in (1, 2]
    \end{cases}
\end{align*}
and
\begin{align*}
    Y(\omega) = 
    \begin{cases}
        2 & \text{if }\omega \in [0, 1.5]\\
        3 & \text{if }\omega \in (1.5, 2].
    \end{cases}
\end{align*}
Then which one of the following statements is true?
\begin{enumerate}
    \item [(A)] $X$ is a random variable with respect to $\mathcal{G}$, but $Y$ is not a random variable with respect to $\mathcal{G}$.
    \item [(B)] $Y$ is a random variable with respect to $\mathcal{G}$, but $X$ is not a random variable with respect to $\mathcal{G}$.
    \item [(C)] Neither $X$ nor $Y$ is a random variable with respect to $\mathcal{G}$.
    \item [(D)] Both $X$ and $Y$ are random variables with respect to $\mathcal{G}$.
\end{enumerate} \hfill (GATE ST 2023)\\
\solution
%\input{gate/ST/2023/14/main.tex}
	\item  A die is loaded in such a way that each odd number is twice as likely to occur as
each even number. Find $P(G)$, where $G$ is the event that a number greater than
3 occurs on a single roll of the die.
\\
\solution
		%\input{exemplar/11/16/3/5/main.tex}
	\item All the jacks, queens and kings are removed from a deck of 52 playing cards. The remaining cards are well shuffled and then one card is drawn at random. Giving ace a value 1 similar value for other cards, find the probability that the card has a value 
		\begin{enumerate}
			\item 7
			\item greater than 7
			\item less than 7
		\end{enumerate}
		%\input{exemplar/10/13/3/30/main.tex}
  \item A Lot consists of 48 mobile phones of which 42 are good, 3 have only minor defects and 3 have major defects.Varnika will buy a phone if it is good but the trader will only buy a mobile if it has no major defects. One phone is selected at random from the lot. What is the probability that it is
\begin{enumerate}
	\item acceptable to Varnika?
            \item acceptable to the trader?
\end{enumerate}
\solution
	%\input{exemplar/10/13/3/40/main.tex}
 \item A student says that if you throw a die, it will show up 1 or not 1. Therefore, the probability of getting 1 and the probability of getting 'not 1' each is equal to $\frac{1}{2}$. Is this correct? Give reasons.\\
 \solution
        %\input{exemplar/10/13/2/9/main.tex}
   \item Four candidates A, B, C, D have ap-
plied for the assignment to coach a school cricket
team. If A is twice as likely to be selected as B, and
B and C are given about the same chance of being
selected, while C is twice as likely to be selected
as D, what are the probabilities that
\begin{enumerate}
\item C will be selected?
\item A will not be selected?
\end{enumerate}
	%\input{exemplar/11/16/3/9/main.tex}
 \item A bag contain 24 balls of which $x$ balls are red, $2x$ are white and $3x$ are blue. A ball is selected at random, What is the probability that it is
\begin{enumerate}[label=\alph*)]
\item not red ?
\item white ?
\end{enumerate}
%\input{exemplar/10/13/3/41/main.tex}
If the letters of the word ASSASSINATION are arranged at random. Find the Probability that
\begin{enumerate}[label=(\alph*)]
\item Four $S's$ come consecutively in the word
\item Two  $I's$ and two $N's$ come together
\item All $A's$ are not coming together
\item No two $A's$ are coming together
\end{enumerate}
%\input{exemplar/11/16/3/14/main.tex}
	\item One urn contains two black balls (labelled B1 and B2) and one white ball. A
	second urn contains one black ball and two white balls (labelled W1 and W2).
	Suppose the following experiment is performed. One of the two urns is chosen
	at random. Next a ball is randomly chosen from the urn. Then a second ball is
	chosen at random from the same urn without replacing the first ball.
	
	\begin{enumerate}
	\item What is the probability that two black balls are chosen?
	
	\item What is the probability that two balls of opposite colour are chosen?
	\end{enumerate}
	\solution
	%\input{exemplar/11/16/3/12/main1.tex}
\end{enumerate}

		%
\item 
Two cards are drawn at random and without replacement from a pack of 52 playing cards. Find the probability that both the cards are black.
\\
\solution
		%\begin{enumerate}[label=\thesection.\arabic*,ref=\thesection.\theenumi]
	\item One card is drawn from a well-shuffled deck of 52 cards. Find the probability of getting
\begin{enumerate}
\item A king of red colour 
\item A face card 
\item A red face card
\item The jack of hearts
\item A spade
\item The queen of diamonds

\end{enumerate}
\solution
		%\input{ncert/10/15/1/14/main.tex}
	\item Five cards—the ten, jack, queen, king and ace of diamonds, are well-shuffled with their face downwards. One card is then picked up at random.
\begin{enumerate}
\item
What is the probability that the card is the queen? 
\item
If the queen is drawn and put aside, what is the probability that the second card picked up is (a) an ace? (b) a queen?\\
\end{enumerate}
\solution
		%\input{ncert/10/15/1/15/defs.tex}
	\item A bag contains $5$ red balls and some blue balls. If the probability of drawing a blue ball is double that if a red ball, determine the number of blue balls in the bag. 
		\\
\solution
		%\input{ncert/10/15/2/3/defs.tex}
	\item A card is selected from a pack of 52 cards.
 \begin{enumerate}[label=(\alph*)] 
                 \item How many points are there in the sample space?
                 \item Calculate the probability that the card is an ace of spades.
                 \item Calculate the probability that the card is (i) an ace and (ii) black card.
 \end{enumerate}
\solution
		%\input{ncert/11/16/3/4/main.tex}
\item Four cards are drawn from a well-shuffled deck of 52 cards. What is the probability of obtaining 3 diamonds and one spade.
\\
\solution
		%\input{ncert/11/16/4/2/defs.tex}
\item In a certain lottery 10,000 tickets are sold and ten equal prizes are awarded. What is the probability of not getting a prize if you buy (a) one ticket (b) two tickets (c) 10 tickets ?	
\\
\solution
		%\input{ncert/11/16/4/4/defs.tex}
		%
\item 
Out of 100 students, two sections of 40 and 60 are formed. If you and your friend are among the 100 students, what is the probability that
\begin{enumerate}
\item you both enter the same section?
\item you both enter the different sections?
\end{enumerate}
\solution
		%\input{ncert/11/16/4/5/defs.tex}
	\item 
The number lock of a suitcase has 4 wheels each labelled with ten digits i.e. from 0 to 9.The lock opens with a sequence of four digits with no repeats.What is the probability of a person getting the right sequence to open the suitcase.
\\
\solution
		%\input{ncert/11/16/4/10/defs.tex}
		%
\item 
Two cards are drawn at random and without replacement from a pack of 52 playing cards. Find the probability that both the cards are black.
\\
\solution
		%\input{ncert/12/13/2/2/defs.tex}
		\item A box of oranges is inspected by examining three randomly selected oranges drawn without replacement. If all the three oranges are good, the box is approved for sale, otherwise, it is rejected. Find the probability that a box containing 15 oranges out of which 12 are good and 3 are bad ones will be approved for sale.
		\label{ncert/12/13/2/3/defs.tex}
		\item Two balls are drawn at random with replacement from a box containing 10 black and 8 red balls. Find the probability that
		\label{ncert/12/13/2/12}
\begin{enumerate}
\item both balls are red.
\item first ball is black and second is red.
\item one of them is black and other is red.
\end{enumerate}

\item In a hostel, 60\% of the students read Hindi newspaper, 40\% read English newspaper and 20\% read both Hindi and English newspapers. A student is selected at random.
		\label{ncert/12/13/2/15}
\begin{enumerate}
\item Find the probability that she reads neither Hindi nor English newspapers.
\item If she reads Hindi newspaper, find the probability that she reads English newspaper.
\item If she reads English newspaper, find the probability that she reads Hindi newspaper.\\
\end{enumerate}
\item The probability of obtaining an even prime number on each die, when a pair of dice is rolled is 
\begin{enumerate}
    \item $0$ 
    
    \item $\frac{1}{3}$ 
    
    \item $\frac{1}{12}$ 
    
    \item $\frac{1}{36}$ 
\end{enumerate}
\solution
		%\input{ncert/12/13/2/17/defs.tex}
	\item A bag contains 4 red and 4 black balls, another bag contains 2 red and 6 black balls. One of the two bags is selected at random and a ball is drawn from the bag which is found to be red. Find the probability that the ball is drawn from the first bag.
\\
\solution
		%\input{ncert/12/13/3/2/main.tex}
  \item
  Cards with numbers 2 to 101 are placed in a box. A card is selected at random.Find the probability that the card has
\begin{enumerate}[label=(\roman*)]
	\item an even number 
	\item a square number
\end{enumerate}
\solution
%\input{exemplar/10/13/3/32/main.tex}
\item
The king, queen and jack of clubs are removed from a deck of 52 playing cards and then well shuffled. Now one card is drawn at random from the remaining cards.  Determine the probability that the card is
\begin{enumerate}[label=(\roman*)]
\item a club
\item 10 of hearts
\end{enumerate}
\solution
%\input{exemplar/10/13/3/29/main.tex}
\item A team of medical students doing their internship have to assist during surgeries
at a city hospital. The probabilities of surgeries rated as very complex, complex,
routine, simple or very simple are respectively, 0.15, 0.20, 0.31, 0.26, .08. Find
the probabilities that a particular surgery will be rated
\begin{enumerate}
	\item complex or very complex;
	\item neither very complex nor very simple;
	\item routine or complex
	\item routine or simple
\end{enumerate}
\solution
%\input{exemplar/11/16/3/8(1)/main.tex}
\item A card is selected from a pack of 52 cards.
\begin{enumerate}[label=(\alph*)]
    \item How many points are there in the sample space?
    \item Calculate the probability that the card is an ace of spades.
    \item Calculate the probability that the card is (i) an ace and (ii) black card.
\end{enumerate}
\solution
%\input{exemplar/11/16/3/4/main2.tex}
\item The probability that a non leap year selected at random will contain 53 sundays.
\\
\solution
%\input{exemplar/10/13/1/19/main.tex}
\item One of the four persons John, Rita, Aslam or Gurpreet will be promoted next
month. Consequently the sample space consists of four elementary outcomes
S = {John promoted, Rita promoted, Aslam promoted, Gurpreet promoted}
You are told that the chances of John’s promotion is same as that of Gurpreet,
Rita’s chances of promotion are twice as likely as Johns. Aslam’s chances are
four times that of John.
\begin{enumerate}
	\item Determine
	\begin{enumerate}
		\item P (John promoted)
		\item P (Rita promoted)
		\item P (Aslam promoted)
		\item P (Gurpreet promoted)
	\end{enumerate}
	\item If A = {John promoted or Gurpreet promoted}, find P (A).
\end{enumerate}
\solution
%\input{exemplar/11/16/3/10/main.tex}
\item A card is drawn from a deck of 52 cards. Find the probability of getting a king or a heart or a red card.\\
\solution
%\input{exemplar/11/16/3/15/main.tex}
\item The probability that a student will pass his examination is 0.73, the probability of
the student getting a compartment is 0.13, and the probability that the student will
either pass or get compartment is 0.96. State True or False.\\
\solution
%\input{exemplar/11/16/3/31/main.tex}
\item A card is selected from a pack of 52 cards\\
\begin{enumerate}[label=(\alph*)]
\item How many points are there in the sample space?
\item Calculate the probability that the cards is an ace of spades.
\item Calculate the probability that the card is (i) an ace (ii)black card.\\
\end{enumerate}
%\input{ncert/11/16/3/4_1/Prob_4.tex}
\item In a non-leap year, the probability of having 53 tuesdays or 53 wednesdays is\\
\solution
%\input{exemplar/11/16/3/18/main.tex}
\item There are 1000 sealed envelopes in a box, 10 of them contain a cash prize of
Rs 100 each, 100 of them contain a cash prize of Rs 50 each and 200 of them
contain a cash prize of Rs 10 each and rest do not contain any cash prize. If they
are well shuffled and an envelope is picked up out, what is the probability that it
contains no cash prize?\\
\solution
%\input{exemplar/10/13/3/34/main.tex}
\item 
A die is thrown and a card is selected at random from a deck of 52 playing cards. The probability of getting an even number on the die and a spade card.\\
\solution
%\input{exemplar/12/13/3/78/main.tex}
\item
If 4-digit numbers greater than 5,000 are randomly formed from the digits 0, 1, 3, 5, and 7, what is the probability of forming a number divisible by 5 when:
\begin{enumerate}
    \item The digits are repeated?
    \item The repetition of digits is not allowed?
\end{enumerate}
\solution
%\input{ncert/11/16/4/9/main.tex}
\item Consider the probability space $\brak{\Omega, \mathcal{G}, P}$ where $\Omega = [0,2]$ and $\mathcal{G} = \cbrak{\phi, \Omega, [0,1], (1,2]}$. Let $X$ and $Y$ be two functions on $\Omega$ defined as
\begin{align*}
    X(\omega) = 
    \begin{cases}
        1 & \text{if }\omega \in [0, 1]\\
        2 & \text{if }\omega \in (1, 2]
    \end{cases}
\end{align*}
and
\begin{align*}
    Y(\omega) = 
    \begin{cases}
        2 & \text{if }\omega \in [0, 1.5]\\
        3 & \text{if }\omega \in (1.5, 2].
    \end{cases}
\end{align*}
Then which one of the following statements is true?
\begin{enumerate}
    \item [(A)] $X$ is a random variable with respect to $\mathcal{G}$, but $Y$ is not a random variable with respect to $\mathcal{G}$.
    \item [(B)] $Y$ is a random variable with respect to $\mathcal{G}$, but $X$ is not a random variable with respect to $\mathcal{G}$.
    \item [(C)] Neither $X$ nor $Y$ is a random variable with respect to $\mathcal{G}$.
    \item [(D)] Both $X$ and $Y$ are random variables with respect to $\mathcal{G}$.
\end{enumerate} \hfill (GATE ST 2023)\\
\solution
%\input{gate/ST/2023/14/main.tex}
	\item  A die is loaded in such a way that each odd number is twice as likely to occur as
each even number. Find $P(G)$, where $G$ is the event that a number greater than
3 occurs on a single roll of the die.
\\
\solution
		%\input{exemplar/11/16/3/5/main.tex}
	\item All the jacks, queens and kings are removed from a deck of 52 playing cards. The remaining cards are well shuffled and then one card is drawn at random. Giving ace a value 1 similar value for other cards, find the probability that the card has a value 
		\begin{enumerate}
			\item 7
			\item greater than 7
			\item less than 7
		\end{enumerate}
		%\input{exemplar/10/13/3/30/main.tex}
  \item A Lot consists of 48 mobile phones of which 42 are good, 3 have only minor defects and 3 have major defects.Varnika will buy a phone if it is good but the trader will only buy a mobile if it has no major defects. One phone is selected at random from the lot. What is the probability that it is
\begin{enumerate}
	\item acceptable to Varnika?
            \item acceptable to the trader?
\end{enumerate}
\solution
	%\input{exemplar/10/13/3/40/main.tex}
 \item A student says that if you throw a die, it will show up 1 or not 1. Therefore, the probability of getting 1 and the probability of getting 'not 1' each is equal to $\frac{1}{2}$. Is this correct? Give reasons.\\
 \solution
        %\input{exemplar/10/13/2/9/main.tex}
   \item Four candidates A, B, C, D have ap-
plied for the assignment to coach a school cricket
team. If A is twice as likely to be selected as B, and
B and C are given about the same chance of being
selected, while C is twice as likely to be selected
as D, what are the probabilities that
\begin{enumerate}
\item C will be selected?
\item A will not be selected?
\end{enumerate}
	%\input{exemplar/11/16/3/9/main.tex}
 \item A bag contain 24 balls of which $x$ balls are red, $2x$ are white and $3x$ are blue. A ball is selected at random, What is the probability that it is
\begin{enumerate}[label=\alph*)]
\item not red ?
\item white ?
\end{enumerate}
%\input{exemplar/10/13/3/41/main.tex}
If the letters of the word ASSASSINATION are arranged at random. Find the Probability that
\begin{enumerate}[label=(\alph*)]
\item Four $S's$ come consecutively in the word
\item Two  $I's$ and two $N's$ come together
\item All $A's$ are not coming together
\item No two $A's$ are coming together
\end{enumerate}
%\input{exemplar/11/16/3/14/main.tex}
	\item One urn contains two black balls (labelled B1 and B2) and one white ball. A
	second urn contains one black ball and two white balls (labelled W1 and W2).
	Suppose the following experiment is performed. One of the two urns is chosen
	at random. Next a ball is randomly chosen from the urn. Then a second ball is
	chosen at random from the same urn without replacing the first ball.
	
	\begin{enumerate}
	\item What is the probability that two black balls are chosen?
	
	\item What is the probability that two balls of opposite colour are chosen?
	\end{enumerate}
	\solution
	%\input{exemplar/11/16/3/12/main1.tex}
\end{enumerate}

		\item A box of oranges is inspected by examining three randomly selected oranges drawn without replacement. If all the three oranges are good, the box is approved for sale, otherwise, it is rejected. Find the probability that a box containing 15 oranges out of which 12 are good and 3 are bad ones will be approved for sale.
		\label{ncert/12/13/2/3/defs.tex}
		\item Two balls are drawn at random with replacement from a box containing 10 black and 8 red balls. Find the probability that
		\label{ncert/12/13/2/12}
\begin{enumerate}
\item both balls are red.
\item first ball is black and second is red.
\item one of them is black and other is red.
\end{enumerate}

\item In a hostel, 60\% of the students read Hindi newspaper, 40\% read English newspaper and 20\% read both Hindi and English newspapers. A student is selected at random.
		\label{ncert/12/13/2/15}
\begin{enumerate}
\item Find the probability that she reads neither Hindi nor English newspapers.
\item If she reads Hindi newspaper, find the probability that she reads English newspaper.
\item If she reads English newspaper, find the probability that she reads Hindi newspaper.\\
\end{enumerate}
\item The probability of obtaining an even prime number on each die, when a pair of dice is rolled is 
\begin{enumerate}
    \item $0$ 
    
    \item $\frac{1}{3}$ 
    
    \item $\frac{1}{12}$ 
    
    \item $\frac{1}{36}$ 
\end{enumerate}
\solution
		%\begin{enumerate}[label=\thesection.\arabic*,ref=\thesection.\theenumi]
	\item One card is drawn from a well-shuffled deck of 52 cards. Find the probability of getting
\begin{enumerate}
\item A king of red colour 
\item A face card 
\item A red face card
\item The jack of hearts
\item A spade
\item The queen of diamonds

\end{enumerate}
\solution
		%\input{ncert/10/15/1/14/main.tex}
	\item Five cards—the ten, jack, queen, king and ace of diamonds, are well-shuffled with their face downwards. One card is then picked up at random.
\begin{enumerate}
\item
What is the probability that the card is the queen? 
\item
If the queen is drawn and put aside, what is the probability that the second card picked up is (a) an ace? (b) a queen?\\
\end{enumerate}
\solution
		%\input{ncert/10/15/1/15/defs.tex}
	\item A bag contains $5$ red balls and some blue balls. If the probability of drawing a blue ball is double that if a red ball, determine the number of blue balls in the bag. 
		\\
\solution
		%\input{ncert/10/15/2/3/defs.tex}
	\item A card is selected from a pack of 52 cards.
 \begin{enumerate}[label=(\alph*)] 
                 \item How many points are there in the sample space?
                 \item Calculate the probability that the card is an ace of spades.
                 \item Calculate the probability that the card is (i) an ace and (ii) black card.
 \end{enumerate}
\solution
		%\input{ncert/11/16/3/4/main.tex}
\item Four cards are drawn from a well-shuffled deck of 52 cards. What is the probability of obtaining 3 diamonds and one spade.
\\
\solution
		%\input{ncert/11/16/4/2/defs.tex}
\item In a certain lottery 10,000 tickets are sold and ten equal prizes are awarded. What is the probability of not getting a prize if you buy (a) one ticket (b) two tickets (c) 10 tickets ?	
\\
\solution
		%\input{ncert/11/16/4/4/defs.tex}
		%
\item 
Out of 100 students, two sections of 40 and 60 are formed. If you and your friend are among the 100 students, what is the probability that
\begin{enumerate}
\item you both enter the same section?
\item you both enter the different sections?
\end{enumerate}
\solution
		%\input{ncert/11/16/4/5/defs.tex}
	\item 
The number lock of a suitcase has 4 wheels each labelled with ten digits i.e. from 0 to 9.The lock opens with a sequence of four digits with no repeats.What is the probability of a person getting the right sequence to open the suitcase.
\\
\solution
		%\input{ncert/11/16/4/10/defs.tex}
		%
\item 
Two cards are drawn at random and without replacement from a pack of 52 playing cards. Find the probability that both the cards are black.
\\
\solution
		%\input{ncert/12/13/2/2/defs.tex}
		\item A box of oranges is inspected by examining three randomly selected oranges drawn without replacement. If all the three oranges are good, the box is approved for sale, otherwise, it is rejected. Find the probability that a box containing 15 oranges out of which 12 are good and 3 are bad ones will be approved for sale.
		\label{ncert/12/13/2/3/defs.tex}
		\item Two balls are drawn at random with replacement from a box containing 10 black and 8 red balls. Find the probability that
		\label{ncert/12/13/2/12}
\begin{enumerate}
\item both balls are red.
\item first ball is black and second is red.
\item one of them is black and other is red.
\end{enumerate}

\item In a hostel, 60\% of the students read Hindi newspaper, 40\% read English newspaper and 20\% read both Hindi and English newspapers. A student is selected at random.
		\label{ncert/12/13/2/15}
\begin{enumerate}
\item Find the probability that she reads neither Hindi nor English newspapers.
\item If she reads Hindi newspaper, find the probability that she reads English newspaper.
\item If she reads English newspaper, find the probability that she reads Hindi newspaper.\\
\end{enumerate}
\item The probability of obtaining an even prime number on each die, when a pair of dice is rolled is 
\begin{enumerate}
    \item $0$ 
    
    \item $\frac{1}{3}$ 
    
    \item $\frac{1}{12}$ 
    
    \item $\frac{1}{36}$ 
\end{enumerate}
\solution
		%\input{ncert/12/13/2/17/defs.tex}
	\item A bag contains 4 red and 4 black balls, another bag contains 2 red and 6 black balls. One of the two bags is selected at random and a ball is drawn from the bag which is found to be red. Find the probability that the ball is drawn from the first bag.
\\
\solution
		%\input{ncert/12/13/3/2/main.tex}
  \item
  Cards with numbers 2 to 101 are placed in a box. A card is selected at random.Find the probability that the card has
\begin{enumerate}[label=(\roman*)]
	\item an even number 
	\item a square number
\end{enumerate}
\solution
%\input{exemplar/10/13/3/32/main.tex}
\item
The king, queen and jack of clubs are removed from a deck of 52 playing cards and then well shuffled. Now one card is drawn at random from the remaining cards.  Determine the probability that the card is
\begin{enumerate}[label=(\roman*)]
\item a club
\item 10 of hearts
\end{enumerate}
\solution
%\input{exemplar/10/13/3/29/main.tex}
\item A team of medical students doing their internship have to assist during surgeries
at a city hospital. The probabilities of surgeries rated as very complex, complex,
routine, simple or very simple are respectively, 0.15, 0.20, 0.31, 0.26, .08. Find
the probabilities that a particular surgery will be rated
\begin{enumerate}
	\item complex or very complex;
	\item neither very complex nor very simple;
	\item routine or complex
	\item routine or simple
\end{enumerate}
\solution
%\input{exemplar/11/16/3/8(1)/main.tex}
\item A card is selected from a pack of 52 cards.
\begin{enumerate}[label=(\alph*)]
    \item How many points are there in the sample space?
    \item Calculate the probability that the card is an ace of spades.
    \item Calculate the probability that the card is (i) an ace and (ii) black card.
\end{enumerate}
\solution
%\input{exemplar/11/16/3/4/main2.tex}
\item The probability that a non leap year selected at random will contain 53 sundays.
\\
\solution
%\input{exemplar/10/13/1/19/main.tex}
\item One of the four persons John, Rita, Aslam or Gurpreet will be promoted next
month. Consequently the sample space consists of four elementary outcomes
S = {John promoted, Rita promoted, Aslam promoted, Gurpreet promoted}
You are told that the chances of John’s promotion is same as that of Gurpreet,
Rita’s chances of promotion are twice as likely as Johns. Aslam’s chances are
four times that of John.
\begin{enumerate}
	\item Determine
	\begin{enumerate}
		\item P (John promoted)
		\item P (Rita promoted)
		\item P (Aslam promoted)
		\item P (Gurpreet promoted)
	\end{enumerate}
	\item If A = {John promoted or Gurpreet promoted}, find P (A).
\end{enumerate}
\solution
%\input{exemplar/11/16/3/10/main.tex}
\item A card is drawn from a deck of 52 cards. Find the probability of getting a king or a heart or a red card.\\
\solution
%\input{exemplar/11/16/3/15/main.tex}
\item The probability that a student will pass his examination is 0.73, the probability of
the student getting a compartment is 0.13, and the probability that the student will
either pass or get compartment is 0.96. State True or False.\\
\solution
%\input{exemplar/11/16/3/31/main.tex}
\item A card is selected from a pack of 52 cards\\
\begin{enumerate}[label=(\alph*)]
\item How many points are there in the sample space?
\item Calculate the probability that the cards is an ace of spades.
\item Calculate the probability that the card is (i) an ace (ii)black card.\\
\end{enumerate}
%\input{ncert/11/16/3/4_1/Prob_4.tex}
\item In a non-leap year, the probability of having 53 tuesdays or 53 wednesdays is\\
\solution
%\input{exemplar/11/16/3/18/main.tex}
\item There are 1000 sealed envelopes in a box, 10 of them contain a cash prize of
Rs 100 each, 100 of them contain a cash prize of Rs 50 each and 200 of them
contain a cash prize of Rs 10 each and rest do not contain any cash prize. If they
are well shuffled and an envelope is picked up out, what is the probability that it
contains no cash prize?\\
\solution
%\input{exemplar/10/13/3/34/main.tex}
\item 
A die is thrown and a card is selected at random from a deck of 52 playing cards. The probability of getting an even number on the die and a spade card.\\
\solution
%\input{exemplar/12/13/3/78/main.tex}
\item
If 4-digit numbers greater than 5,000 are randomly formed from the digits 0, 1, 3, 5, and 7, what is the probability of forming a number divisible by 5 when:
\begin{enumerate}
    \item The digits are repeated?
    \item The repetition of digits is not allowed?
\end{enumerate}
\solution
%\input{ncert/11/16/4/9/main.tex}
\item Consider the probability space $\brak{\Omega, \mathcal{G}, P}$ where $\Omega = [0,2]$ and $\mathcal{G} = \cbrak{\phi, \Omega, [0,1], (1,2]}$. Let $X$ and $Y$ be two functions on $\Omega$ defined as
\begin{align*}
    X(\omega) = 
    \begin{cases}
        1 & \text{if }\omega \in [0, 1]\\
        2 & \text{if }\omega \in (1, 2]
    \end{cases}
\end{align*}
and
\begin{align*}
    Y(\omega) = 
    \begin{cases}
        2 & \text{if }\omega \in [0, 1.5]\\
        3 & \text{if }\omega \in (1.5, 2].
    \end{cases}
\end{align*}
Then which one of the following statements is true?
\begin{enumerate}
    \item [(A)] $X$ is a random variable with respect to $\mathcal{G}$, but $Y$ is not a random variable with respect to $\mathcal{G}$.
    \item [(B)] $Y$ is a random variable with respect to $\mathcal{G}$, but $X$ is not a random variable with respect to $\mathcal{G}$.
    \item [(C)] Neither $X$ nor $Y$ is a random variable with respect to $\mathcal{G}$.
    \item [(D)] Both $X$ and $Y$ are random variables with respect to $\mathcal{G}$.
\end{enumerate} \hfill (GATE ST 2023)\\
\solution
%\input{gate/ST/2023/14/main.tex}
	\item  A die is loaded in such a way that each odd number is twice as likely to occur as
each even number. Find $P(G)$, where $G$ is the event that a number greater than
3 occurs on a single roll of the die.
\\
\solution
		%\input{exemplar/11/16/3/5/main.tex}
	\item All the jacks, queens and kings are removed from a deck of 52 playing cards. The remaining cards are well shuffled and then one card is drawn at random. Giving ace a value 1 similar value for other cards, find the probability that the card has a value 
		\begin{enumerate}
			\item 7
			\item greater than 7
			\item less than 7
		\end{enumerate}
		%\input{exemplar/10/13/3/30/main.tex}
  \item A Lot consists of 48 mobile phones of which 42 are good, 3 have only minor defects and 3 have major defects.Varnika will buy a phone if it is good but the trader will only buy a mobile if it has no major defects. One phone is selected at random from the lot. What is the probability that it is
\begin{enumerate}
	\item acceptable to Varnika?
            \item acceptable to the trader?
\end{enumerate}
\solution
	%\input{exemplar/10/13/3/40/main.tex}
 \item A student says that if you throw a die, it will show up 1 or not 1. Therefore, the probability of getting 1 and the probability of getting 'not 1' each is equal to $\frac{1}{2}$. Is this correct? Give reasons.\\
 \solution
        %\input{exemplar/10/13/2/9/main.tex}
   \item Four candidates A, B, C, D have ap-
plied for the assignment to coach a school cricket
team. If A is twice as likely to be selected as B, and
B and C are given about the same chance of being
selected, while C is twice as likely to be selected
as D, what are the probabilities that
\begin{enumerate}
\item C will be selected?
\item A will not be selected?
\end{enumerate}
	%\input{exemplar/11/16/3/9/main.tex}
 \item A bag contain 24 balls of which $x$ balls are red, $2x$ are white and $3x$ are blue. A ball is selected at random, What is the probability that it is
\begin{enumerate}[label=\alph*)]
\item not red ?
\item white ?
\end{enumerate}
%\input{exemplar/10/13/3/41/main.tex}
If the letters of the word ASSASSINATION are arranged at random. Find the Probability that
\begin{enumerate}[label=(\alph*)]
\item Four $S's$ come consecutively in the word
\item Two  $I's$ and two $N's$ come together
\item All $A's$ are not coming together
\item No two $A's$ are coming together
\end{enumerate}
%\input{exemplar/11/16/3/14/main.tex}
	\item One urn contains two black balls (labelled B1 and B2) and one white ball. A
	second urn contains one black ball and two white balls (labelled W1 and W2).
	Suppose the following experiment is performed. One of the two urns is chosen
	at random. Next a ball is randomly chosen from the urn. Then a second ball is
	chosen at random from the same urn without replacing the first ball.
	
	\begin{enumerate}
	\item What is the probability that two black balls are chosen?
	
	\item What is the probability that two balls of opposite colour are chosen?
	\end{enumerate}
	\solution
	%\input{exemplar/11/16/3/12/main1.tex}
\end{enumerate}

	\item A bag contains 4 red and 4 black balls, another bag contains 2 red and 6 black balls. One of the two bags is selected at random and a ball is drawn from the bag which is found to be red. Find the probability that the ball is drawn from the first bag.
\\
\solution
		%\begin{table}[H]
	\centering
\begin{tabular}{|c|c|c|}
\hline
Random variable &Value &Definition\\ \hline
\multirow{3}{*}{X} &0 &Slips of Rs 1\\
&1 &Slips of Rs 5\\
&2 &Slips of Rs 13\\ \hline
\multirow{2}{*}{Y} &0 &Box A\\
&1 &Box B\\\hline
\end{tabular}
\caption{}
\label{tab:Distribution}
\end{table}
See \tabref{tab:Distribution}.
\begin{align}
p_{Y}\brak{k}= \begin{cases} 
      \frac{1}{3} & {k=0} \\
      \frac{2}{3 }& {k=1} 
   \end{cases}
   \\
p_{Y|X}\brak{0|0} = \frac{19}{25}\, 
p_{Y|X}\brak{0|1} = \frac{6}{25}\,
p_{Y|X}\brak{1|0} = \frac{45}{50}\,
p_{Y|X}\brak{1|2} = \frac{5}{50}
\end{align}
The desired probability is the probability that a slip drawn at random is marked other than Rs 1,
\begin{align}
&=1-p_X\brak{0}\\
&= p_X(1) + p_X(2)
\end{align}
Using Bayes theorem,
\begin{align}
&= p_Y\brak{0} \times \pr{Y=0 | X=1} + p_Y\brak{1} \times \pr{Y=1|X=2}\\
&=\frac{1}{3} \times \frac{6}{25} + \frac{2}{3} \times \frac{5}{50}\\
&=\frac{11}{75}
\end{align}

\newpage

%\tableofcontents

\bigskip

\renewcommand{\thefigure}{\theenumi}
\renewcommand{\thetable}{\theenumi}
%\renewcommand{\theequation}{\theenumi}

%\begin{abstract}
%%\boldmath
%In this letter, an algorithm for evaluating the exact analytical bit error rate  (BER)  for the piecewise linear (PL) combiner for  multiple relays is presented. Previous results were available only for upto three relays. The algorithm is unique in the sense that  the actual mathematical expressions, that are prohibitively large, need not be explicitly obtained. The diversity gain due to multiple relays is shown through plots of the analytical BER, well supported by simulations. 
%
%\end{abstract}
% IEEEtran.cls defaults to using nonbold math in the Abstract.
% This preserves the distinction between vectors and scalars. However,
% if the journal you are submitting to favors bold math in the abstract,
% then you can use LaTeX's standard command \boldmath at the very start
% of the abstract to achieve this. Many IEEE journals frown on math
% in the abstract anyway.

% Note that keywords are not normally used for peerreview papers.
%\begin{IEEEkeywords}
%Cooperative diversity, decode and forward, piecewise linear
%\end{IEEEkeywords}



% For peer review papers, you can put extra information on the cover
% page as needed:
% \ifCLASSOPTIONpeerreview
% \begin{center} \bfseries EDICS Category: 3-BBND \end{center}
% \fi
%
% For peerreview papers, this IEEEtran command inserts a page break and
% creates the second title. It will be ignored for other modes.
%\IEEEpeerreviewmaketitle




  \item
  Cards with numbers 2 to 101 are placed in a box. A card is selected at random.Find the probability that the card has
\begin{enumerate}[label=(\roman*)]
	\item an even number 
	\item a square number
\end{enumerate}
\solution
%\begin{table}[H]
	\centering
\begin{tabular}{|c|c|c|}
\hline
Random variable &Value &Definition\\ \hline
\multirow{3}{*}{X} &0 &Slips of Rs 1\\
&1 &Slips of Rs 5\\
&2 &Slips of Rs 13\\ \hline
\multirow{2}{*}{Y} &0 &Box A\\
&1 &Box B\\\hline
\end{tabular}
\caption{}
\label{tab:Distribution}
\end{table}
See \tabref{tab:Distribution}.
\begin{align}
p_{Y}\brak{k}= \begin{cases} 
      \frac{1}{3} & {k=0} \\
      \frac{2}{3 }& {k=1} 
   \end{cases}
   \\
p_{Y|X}\brak{0|0} = \frac{19}{25}\, 
p_{Y|X}\brak{0|1} = \frac{6}{25}\,
p_{Y|X}\brak{1|0} = \frac{45}{50}\,
p_{Y|X}\brak{1|2} = \frac{5}{50}
\end{align}
The desired probability is the probability that a slip drawn at random is marked other than Rs 1,
\begin{align}
&=1-p_X\brak{0}\\
&= p_X(1) + p_X(2)
\end{align}
Using Bayes theorem,
\begin{align}
&= p_Y\brak{0} \times \pr{Y=0 | X=1} + p_Y\brak{1} \times \pr{Y=1|X=2}\\
&=\frac{1}{3} \times \frac{6}{25} + \frac{2}{3} \times \frac{5}{50}\\
&=\frac{11}{75}
\end{align}

\newpage

%\tableofcontents

\bigskip

\renewcommand{\thefigure}{\theenumi}
\renewcommand{\thetable}{\theenumi}
%\renewcommand{\theequation}{\theenumi}

%\begin{abstract}
%%\boldmath
%In this letter, an algorithm for evaluating the exact analytical bit error rate  (BER)  for the piecewise linear (PL) combiner for  multiple relays is presented. Previous results were available only for upto three relays. The algorithm is unique in the sense that  the actual mathematical expressions, that are prohibitively large, need not be explicitly obtained. The diversity gain due to multiple relays is shown through plots of the analytical BER, well supported by simulations. 
%
%\end{abstract}
% IEEEtran.cls defaults to using nonbold math in the Abstract.
% This preserves the distinction between vectors and scalars. However,
% if the journal you are submitting to favors bold math in the abstract,
% then you can use LaTeX's standard command \boldmath at the very start
% of the abstract to achieve this. Many IEEE journals frown on math
% in the abstract anyway.

% Note that keywords are not normally used for peerreview papers.
%\begin{IEEEkeywords}
%Cooperative diversity, decode and forward, piecewise linear
%\end{IEEEkeywords}



% For peer review papers, you can put extra information on the cover
% page as needed:
% \ifCLASSOPTIONpeerreview
% \begin{center} \bfseries EDICS Category: 3-BBND \end{center}
% \fi
%
% For peerreview papers, this IEEEtran command inserts a page break and
% creates the second title. It will be ignored for other modes.
%\IEEEpeerreviewmaketitle




\item
The king, queen and jack of clubs are removed from a deck of 52 playing cards and then well shuffled. Now one card is drawn at random from the remaining cards.  Determine the probability that the card is
\begin{enumerate}[label=(\roman*)]
\item a club
\item 10 of hearts
\end{enumerate}
\solution
%\begin{table}[H]
	\centering
\begin{tabular}{|c|c|c|}
\hline
Random variable &Value &Definition\\ \hline
\multirow{3}{*}{X} &0 &Slips of Rs 1\\
&1 &Slips of Rs 5\\
&2 &Slips of Rs 13\\ \hline
\multirow{2}{*}{Y} &0 &Box A\\
&1 &Box B\\\hline
\end{tabular}
\caption{}
\label{tab:Distribution}
\end{table}
See \tabref{tab:Distribution}.
\begin{align}
p_{Y}\brak{k}= \begin{cases} 
      \frac{1}{3} & {k=0} \\
      \frac{2}{3 }& {k=1} 
   \end{cases}
   \\
p_{Y|X}\brak{0|0} = \frac{19}{25}\, 
p_{Y|X}\brak{0|1} = \frac{6}{25}\,
p_{Y|X}\brak{1|0} = \frac{45}{50}\,
p_{Y|X}\brak{1|2} = \frac{5}{50}
\end{align}
The desired probability is the probability that a slip drawn at random is marked other than Rs 1,
\begin{align}
&=1-p_X\brak{0}\\
&= p_X(1) + p_X(2)
\end{align}
Using Bayes theorem,
\begin{align}
&= p_Y\brak{0} \times \pr{Y=0 | X=1} + p_Y\brak{1} \times \pr{Y=1|X=2}\\
&=\frac{1}{3} \times \frac{6}{25} + \frac{2}{3} \times \frac{5}{50}\\
&=\frac{11}{75}
\end{align}

\newpage

%\tableofcontents

\bigskip

\renewcommand{\thefigure}{\theenumi}
\renewcommand{\thetable}{\theenumi}
%\renewcommand{\theequation}{\theenumi}

%\begin{abstract}
%%\boldmath
%In this letter, an algorithm for evaluating the exact analytical bit error rate  (BER)  for the piecewise linear (PL) combiner for  multiple relays is presented. Previous results were available only for upto three relays. The algorithm is unique in the sense that  the actual mathematical expressions, that are prohibitively large, need not be explicitly obtained. The diversity gain due to multiple relays is shown through plots of the analytical BER, well supported by simulations. 
%
%\end{abstract}
% IEEEtran.cls defaults to using nonbold math in the Abstract.
% This preserves the distinction between vectors and scalars. However,
% if the journal you are submitting to favors bold math in the abstract,
% then you can use LaTeX's standard command \boldmath at the very start
% of the abstract to achieve this. Many IEEE journals frown on math
% in the abstract anyway.

% Note that keywords are not normally used for peerreview papers.
%\begin{IEEEkeywords}
%Cooperative diversity, decode and forward, piecewise linear
%\end{IEEEkeywords}



% For peer review papers, you can put extra information on the cover
% page as needed:
% \ifCLASSOPTIONpeerreview
% \begin{center} \bfseries EDICS Category: 3-BBND \end{center}
% \fi
%
% For peerreview papers, this IEEEtran command inserts a page break and
% creates the second title. It will be ignored for other modes.
%\IEEEpeerreviewmaketitle




\item A team of medical students doing their internship have to assist during surgeries
at a city hospital. The probabilities of surgeries rated as very complex, complex,
routine, simple or very simple are respectively, 0.15, 0.20, 0.31, 0.26, .08. Find
the probabilities that a particular surgery will be rated
\begin{enumerate}
	\item complex or very complex;
	\item neither very complex nor very simple;
	\item routine or complex
	\item routine or simple
\end{enumerate}
\solution
%\begin{table}[H]
	\centering
\begin{tabular}{|c|c|c|}
\hline
Random variable &Value &Definition\\ \hline
\multirow{3}{*}{X} &0 &Slips of Rs 1\\
&1 &Slips of Rs 5\\
&2 &Slips of Rs 13\\ \hline
\multirow{2}{*}{Y} &0 &Box A\\
&1 &Box B\\\hline
\end{tabular}
\caption{}
\label{tab:Distribution}
\end{table}
See \tabref{tab:Distribution}.
\begin{align}
p_{Y}\brak{k}= \begin{cases} 
      \frac{1}{3} & {k=0} \\
      \frac{2}{3 }& {k=1} 
   \end{cases}
   \\
p_{Y|X}\brak{0|0} = \frac{19}{25}\, 
p_{Y|X}\brak{0|1} = \frac{6}{25}\,
p_{Y|X}\brak{1|0} = \frac{45}{50}\,
p_{Y|X}\brak{1|2} = \frac{5}{50}
\end{align}
The desired probability is the probability that a slip drawn at random is marked other than Rs 1,
\begin{align}
&=1-p_X\brak{0}\\
&= p_X(1) + p_X(2)
\end{align}
Using Bayes theorem,
\begin{align}
&= p_Y\brak{0} \times \pr{Y=0 | X=1} + p_Y\brak{1} \times \pr{Y=1|X=2}\\
&=\frac{1}{3} \times \frac{6}{25} + \frac{2}{3} \times \frac{5}{50}\\
&=\frac{11}{75}
\end{align}

\newpage

%\tableofcontents

\bigskip

\renewcommand{\thefigure}{\theenumi}
\renewcommand{\thetable}{\theenumi}
%\renewcommand{\theequation}{\theenumi}

%\begin{abstract}
%%\boldmath
%In this letter, an algorithm for evaluating the exact analytical bit error rate  (BER)  for the piecewise linear (PL) combiner for  multiple relays is presented. Previous results were available only for upto three relays. The algorithm is unique in the sense that  the actual mathematical expressions, that are prohibitively large, need not be explicitly obtained. The diversity gain due to multiple relays is shown through plots of the analytical BER, well supported by simulations. 
%
%\end{abstract}
% IEEEtran.cls defaults to using nonbold math in the Abstract.
% This preserves the distinction between vectors and scalars. However,
% if the journal you are submitting to favors bold math in the abstract,
% then you can use LaTeX's standard command \boldmath at the very start
% of the abstract to achieve this. Many IEEE journals frown on math
% in the abstract anyway.

% Note that keywords are not normally used for peerreview papers.
%\begin{IEEEkeywords}
%Cooperative diversity, decode and forward, piecewise linear
%\end{IEEEkeywords}



% For peer review papers, you can put extra information on the cover
% page as needed:
% \ifCLASSOPTIONpeerreview
% \begin{center} \bfseries EDICS Category: 3-BBND \end{center}
% \fi
%
% For peerreview papers, this IEEEtran command inserts a page break and
% creates the second title. It will be ignored for other modes.
%\IEEEpeerreviewmaketitle




\item A card is selected from a pack of 52 cards.
\begin{enumerate}[label=(\alph*)]
    \item How many points are there in the sample space?
    \item Calculate the probability that the card is an ace of spades.
    \item Calculate the probability that the card is (i) an ace and (ii) black card.
\end{enumerate}
\solution
%Let $X$ be an bernoulli rv defined as in \tabref{tab:exemplar/11/16/3/26}.  Then, 
\begin{equation}
    p =
        \frac{4}{11} 
\end{equation}
\begin{table}[H]
	\centering
	\input{exemplar/11/16/3/26/tables/Table2.tex}
	\caption{}
        \label{tab:exemplar/11/16/3/26}
\end{table}

\item The probability that a non leap year selected at random will contain 53 sundays.
\\
\solution
%\begin{table}[H]
	\centering
\begin{tabular}{|c|c|c|}
\hline
Random variable &Value &Definition\\ \hline
\multirow{3}{*}{X} &0 &Slips of Rs 1\\
&1 &Slips of Rs 5\\
&2 &Slips of Rs 13\\ \hline
\multirow{2}{*}{Y} &0 &Box A\\
&1 &Box B\\\hline
\end{tabular}
\caption{}
\label{tab:Distribution}
\end{table}
See \tabref{tab:Distribution}.
\begin{align}
p_{Y}\brak{k}= \begin{cases} 
      \frac{1}{3} & {k=0} \\
      \frac{2}{3 }& {k=1} 
   \end{cases}
   \\
p_{Y|X}\brak{0|0} = \frac{19}{25}\, 
p_{Y|X}\brak{0|1} = \frac{6}{25}\,
p_{Y|X}\brak{1|0} = \frac{45}{50}\,
p_{Y|X}\brak{1|2} = \frac{5}{50}
\end{align}
The desired probability is the probability that a slip drawn at random is marked other than Rs 1,
\begin{align}
&=1-p_X\brak{0}\\
&= p_X(1) + p_X(2)
\end{align}
Using Bayes theorem,
\begin{align}
&= p_Y\brak{0} \times \pr{Y=0 | X=1} + p_Y\brak{1} \times \pr{Y=1|X=2}\\
&=\frac{1}{3} \times \frac{6}{25} + \frac{2}{3} \times \frac{5}{50}\\
&=\frac{11}{75}
\end{align}

\newpage

%\tableofcontents

\bigskip

\renewcommand{\thefigure}{\theenumi}
\renewcommand{\thetable}{\theenumi}
%\renewcommand{\theequation}{\theenumi}

%\begin{abstract}
%%\boldmath
%In this letter, an algorithm for evaluating the exact analytical bit error rate  (BER)  for the piecewise linear (PL) combiner for  multiple relays is presented. Previous results were available only for upto three relays. The algorithm is unique in the sense that  the actual mathematical expressions, that are prohibitively large, need not be explicitly obtained. The diversity gain due to multiple relays is shown through plots of the analytical BER, well supported by simulations. 
%
%\end{abstract}
% IEEEtran.cls defaults to using nonbold math in the Abstract.
% This preserves the distinction between vectors and scalars. However,
% if the journal you are submitting to favors bold math in the abstract,
% then you can use LaTeX's standard command \boldmath at the very start
% of the abstract to achieve this. Many IEEE journals frown on math
% in the abstract anyway.

% Note that keywords are not normally used for peerreview papers.
%\begin{IEEEkeywords}
%Cooperative diversity, decode and forward, piecewise linear
%\end{IEEEkeywords}



% For peer review papers, you can put extra information on the cover
% page as needed:
% \ifCLASSOPTIONpeerreview
% \begin{center} \bfseries EDICS Category: 3-BBND \end{center}
% \fi
%
% For peerreview papers, this IEEEtran command inserts a page break and
% creates the second title. It will be ignored for other modes.
%\IEEEpeerreviewmaketitle




\item One of the four persons John, Rita, Aslam or Gurpreet will be promoted next
month. Consequently the sample space consists of four elementary outcomes
S = {John promoted, Rita promoted, Aslam promoted, Gurpreet promoted}
You are told that the chances of John’s promotion is same as that of Gurpreet,
Rita’s chances of promotion are twice as likely as Johns. Aslam’s chances are
four times that of John.
\begin{enumerate}
	\item Determine
	\begin{enumerate}
		\item P (John promoted)
		\item P (Rita promoted)
		\item P (Aslam promoted)
		\item P (Gurpreet promoted)
	\end{enumerate}
	\item If A = {John promoted or Gurpreet promoted}, find P (A).
\end{enumerate}
\solution
%\begin{table}[H]
	\centering
\begin{tabular}{|c|c|c|}
\hline
Random variable &Value &Definition\\ \hline
\multirow{3}{*}{X} &0 &Slips of Rs 1\\
&1 &Slips of Rs 5\\
&2 &Slips of Rs 13\\ \hline
\multirow{2}{*}{Y} &0 &Box A\\
&1 &Box B\\\hline
\end{tabular}
\caption{}
\label{tab:Distribution}
\end{table}
See \tabref{tab:Distribution}.
\begin{align}
p_{Y}\brak{k}= \begin{cases} 
      \frac{1}{3} & {k=0} \\
      \frac{2}{3 }& {k=1} 
   \end{cases}
   \\
p_{Y|X}\brak{0|0} = \frac{19}{25}\, 
p_{Y|X}\brak{0|1} = \frac{6}{25}\,
p_{Y|X}\brak{1|0} = \frac{45}{50}\,
p_{Y|X}\brak{1|2} = \frac{5}{50}
\end{align}
The desired probability is the probability that a slip drawn at random is marked other than Rs 1,
\begin{align}
&=1-p_X\brak{0}\\
&= p_X(1) + p_X(2)
\end{align}
Using Bayes theorem,
\begin{align}
&= p_Y\brak{0} \times \pr{Y=0 | X=1} + p_Y\brak{1} \times \pr{Y=1|X=2}\\
&=\frac{1}{3} \times \frac{6}{25} + \frac{2}{3} \times \frac{5}{50}\\
&=\frac{11}{75}
\end{align}

\newpage

%\tableofcontents

\bigskip

\renewcommand{\thefigure}{\theenumi}
\renewcommand{\thetable}{\theenumi}
%\renewcommand{\theequation}{\theenumi}

%\begin{abstract}
%%\boldmath
%In this letter, an algorithm for evaluating the exact analytical bit error rate  (BER)  for the piecewise linear (PL) combiner for  multiple relays is presented. Previous results were available only for upto three relays. The algorithm is unique in the sense that  the actual mathematical expressions, that are prohibitively large, need not be explicitly obtained. The diversity gain due to multiple relays is shown through plots of the analytical BER, well supported by simulations. 
%
%\end{abstract}
% IEEEtran.cls defaults to using nonbold math in the Abstract.
% This preserves the distinction between vectors and scalars. However,
% if the journal you are submitting to favors bold math in the abstract,
% then you can use LaTeX's standard command \boldmath at the very start
% of the abstract to achieve this. Many IEEE journals frown on math
% in the abstract anyway.

% Note that keywords are not normally used for peerreview papers.
%\begin{IEEEkeywords}
%Cooperative diversity, decode and forward, piecewise linear
%\end{IEEEkeywords}



% For peer review papers, you can put extra information on the cover
% page as needed:
% \ifCLASSOPTIONpeerreview
% \begin{center} \bfseries EDICS Category: 3-BBND \end{center}
% \fi
%
% For peerreview papers, this IEEEtran command inserts a page break and
% creates the second title. It will be ignored for other modes.
%\IEEEpeerreviewmaketitle




\item A card is drawn from a deck of 52 cards. Find the probability of getting a king or a heart or a red card.\\
\solution
%\begin{table}[H]
	\centering
\begin{tabular}{|c|c|c|}
\hline
Random variable &Value &Definition\\ \hline
\multirow{3}{*}{X} &0 &Slips of Rs 1\\
&1 &Slips of Rs 5\\
&2 &Slips of Rs 13\\ \hline
\multirow{2}{*}{Y} &0 &Box A\\
&1 &Box B\\\hline
\end{tabular}
\caption{}
\label{tab:Distribution}
\end{table}
See \tabref{tab:Distribution}.
\begin{align}
p_{Y}\brak{k}= \begin{cases} 
      \frac{1}{3} & {k=0} \\
      \frac{2}{3 }& {k=1} 
   \end{cases}
   \\
p_{Y|X}\brak{0|0} = \frac{19}{25}\, 
p_{Y|X}\brak{0|1} = \frac{6}{25}\,
p_{Y|X}\brak{1|0} = \frac{45}{50}\,
p_{Y|X}\brak{1|2} = \frac{5}{50}
\end{align}
The desired probability is the probability that a slip drawn at random is marked other than Rs 1,
\begin{align}
&=1-p_X\brak{0}\\
&= p_X(1) + p_X(2)
\end{align}
Using Bayes theorem,
\begin{align}
&= p_Y\brak{0} \times \pr{Y=0 | X=1} + p_Y\brak{1} \times \pr{Y=1|X=2}\\
&=\frac{1}{3} \times \frac{6}{25} + \frac{2}{3} \times \frac{5}{50}\\
&=\frac{11}{75}
\end{align}

\newpage

%\tableofcontents

\bigskip

\renewcommand{\thefigure}{\theenumi}
\renewcommand{\thetable}{\theenumi}
%\renewcommand{\theequation}{\theenumi}

%\begin{abstract}
%%\boldmath
%In this letter, an algorithm for evaluating the exact analytical bit error rate  (BER)  for the piecewise linear (PL) combiner for  multiple relays is presented. Previous results were available only for upto three relays. The algorithm is unique in the sense that  the actual mathematical expressions, that are prohibitively large, need not be explicitly obtained. The diversity gain due to multiple relays is shown through plots of the analytical BER, well supported by simulations. 
%
%\end{abstract}
% IEEEtran.cls defaults to using nonbold math in the Abstract.
% This preserves the distinction between vectors and scalars. However,
% if the journal you are submitting to favors bold math in the abstract,
% then you can use LaTeX's standard command \boldmath at the very start
% of the abstract to achieve this. Many IEEE journals frown on math
% in the abstract anyway.

% Note that keywords are not normally used for peerreview papers.
%\begin{IEEEkeywords}
%Cooperative diversity, decode and forward, piecewise linear
%\end{IEEEkeywords}



% For peer review papers, you can put extra information on the cover
% page as needed:
% \ifCLASSOPTIONpeerreview
% \begin{center} \bfseries EDICS Category: 3-BBND \end{center}
% \fi
%
% For peerreview papers, this IEEEtran command inserts a page break and
% creates the second title. It will be ignored for other modes.
%\IEEEpeerreviewmaketitle




\item The probability that a student will pass his examination is 0.73, the probability of
the student getting a compartment is 0.13, and the probability that the student will
either pass or get compartment is 0.96. State True or False.\\
\solution
%\begin{table}[H]
	\centering
\begin{tabular}{|c|c|c|}
\hline
Random variable &Value &Definition\\ \hline
\multirow{3}{*}{X} &0 &Slips of Rs 1\\
&1 &Slips of Rs 5\\
&2 &Slips of Rs 13\\ \hline
\multirow{2}{*}{Y} &0 &Box A\\
&1 &Box B\\\hline
\end{tabular}
\caption{}
\label{tab:Distribution}
\end{table}
See \tabref{tab:Distribution}.
\begin{align}
p_{Y}\brak{k}= \begin{cases} 
      \frac{1}{3} & {k=0} \\
      \frac{2}{3 }& {k=1} 
   \end{cases}
   \\
p_{Y|X}\brak{0|0} = \frac{19}{25}\, 
p_{Y|X}\brak{0|1} = \frac{6}{25}\,
p_{Y|X}\brak{1|0} = \frac{45}{50}\,
p_{Y|X}\brak{1|2} = \frac{5}{50}
\end{align}
The desired probability is the probability that a slip drawn at random is marked other than Rs 1,
\begin{align}
&=1-p_X\brak{0}\\
&= p_X(1) + p_X(2)
\end{align}
Using Bayes theorem,
\begin{align}
&= p_Y\brak{0} \times \pr{Y=0 | X=1} + p_Y\brak{1} \times \pr{Y=1|X=2}\\
&=\frac{1}{3} \times \frac{6}{25} + \frac{2}{3} \times \frac{5}{50}\\
&=\frac{11}{75}
\end{align}

\newpage

%\tableofcontents

\bigskip

\renewcommand{\thefigure}{\theenumi}
\renewcommand{\thetable}{\theenumi}
%\renewcommand{\theequation}{\theenumi}

%\begin{abstract}
%%\boldmath
%In this letter, an algorithm for evaluating the exact analytical bit error rate  (BER)  for the piecewise linear (PL) combiner for  multiple relays is presented. Previous results were available only for upto three relays. The algorithm is unique in the sense that  the actual mathematical expressions, that are prohibitively large, need not be explicitly obtained. The diversity gain due to multiple relays is shown through plots of the analytical BER, well supported by simulations. 
%
%\end{abstract}
% IEEEtran.cls defaults to using nonbold math in the Abstract.
% This preserves the distinction between vectors and scalars. However,
% if the journal you are submitting to favors bold math in the abstract,
% then you can use LaTeX's standard command \boldmath at the very start
% of the abstract to achieve this. Many IEEE journals frown on math
% in the abstract anyway.

% Note that keywords are not normally used for peerreview papers.
%\begin{IEEEkeywords}
%Cooperative diversity, decode and forward, piecewise linear
%\end{IEEEkeywords}



% For peer review papers, you can put extra information on the cover
% page as needed:
% \ifCLASSOPTIONpeerreview
% \begin{center} \bfseries EDICS Category: 3-BBND \end{center}
% \fi
%
% For peerreview papers, this IEEEtran command inserts a page break and
% creates the second title. It will be ignored for other modes.
%\IEEEpeerreviewmaketitle




\item A card is selected from a pack of 52 cards\\
\begin{enumerate}[label=(\alph*)]
\item How many points are there in the sample space?
\item Calculate the probability that the cards is an ace of spades.
\item Calculate the probability that the card is (i) an ace (ii)black card.\\
\end{enumerate}
%\input{ncert/11/16/3/4_1/Prob_4.tex}
\item In a non-leap year, the probability of having 53 tuesdays or 53 wednesdays is\\
\solution
%A non-leap year has a total of 365 days, and a week has 7 days.\\
So it can be expressed as 
\begin{align}
365\text{days} &=52\times 7+1 \text{day}
\end{align}
$\implies$ 52 tuesdays or wednesdays\\
Random variable X denotes the days of a week
\begin{align}
p_X\brak{k}&=\frac{1}{7}; \quad \brak{1<k<7}
\end{align}
So the probability of extra day being tuesday or wednesday is
\begin{align}
p_X\brak{3}+p_X\brak{4}&=\frac{1}{7}+\frac{1}{7}=\frac{2}{7}
\end{align}



\item There are 1000 sealed envelopes in a box, 10 of them contain a cash prize of
Rs 100 each, 100 of them contain a cash prize of Rs 50 each and 200 of them
contain a cash prize of Rs 10 each and rest do not contain any cash prize. If they
are well shuffled and an envelope is picked up out, what is the probability that it
contains no cash prize?\\
\solution
%\begin{table}[H]
	\centering
\begin{tabular}{|c|c|c|}
\hline
Random variable &Value &Definition\\ \hline
\multirow{3}{*}{X} &0 &Slips of Rs 1\\
&1 &Slips of Rs 5\\
&2 &Slips of Rs 13\\ \hline
\multirow{2}{*}{Y} &0 &Box A\\
&1 &Box B\\\hline
\end{tabular}
\caption{}
\label{tab:Distribution}
\end{table}
See \tabref{tab:Distribution}.
\begin{align}
p_{Y}\brak{k}= \begin{cases} 
      \frac{1}{3} & {k=0} \\
      \frac{2}{3 }& {k=1} 
   \end{cases}
   \\
p_{Y|X}\brak{0|0} = \frac{19}{25}\, 
p_{Y|X}\brak{0|1} = \frac{6}{25}\,
p_{Y|X}\brak{1|0} = \frac{45}{50}\,
p_{Y|X}\brak{1|2} = \frac{5}{50}
\end{align}
The desired probability is the probability that a slip drawn at random is marked other than Rs 1,
\begin{align}
&=1-p_X\brak{0}\\
&= p_X(1) + p_X(2)
\end{align}
Using Bayes theorem,
\begin{align}
&= p_Y\brak{0} \times \pr{Y=0 | X=1} + p_Y\brak{1} \times \pr{Y=1|X=2}\\
&=\frac{1}{3} \times \frac{6}{25} + \frac{2}{3} \times \frac{5}{50}\\
&=\frac{11}{75}
\end{align}

\newpage

%\tableofcontents

\bigskip

\renewcommand{\thefigure}{\theenumi}
\renewcommand{\thetable}{\theenumi}
%\renewcommand{\theequation}{\theenumi}

%\begin{abstract}
%%\boldmath
%In this letter, an algorithm for evaluating the exact analytical bit error rate  (BER)  for the piecewise linear (PL) combiner for  multiple relays is presented. Previous results were available only for upto three relays. The algorithm is unique in the sense that  the actual mathematical expressions, that are prohibitively large, need not be explicitly obtained. The diversity gain due to multiple relays is shown through plots of the analytical BER, well supported by simulations. 
%
%\end{abstract}
% IEEEtran.cls defaults to using nonbold math in the Abstract.
% This preserves the distinction between vectors and scalars. However,
% if the journal you are submitting to favors bold math in the abstract,
% then you can use LaTeX's standard command \boldmath at the very start
% of the abstract to achieve this. Many IEEE journals frown on math
% in the abstract anyway.

% Note that keywords are not normally used for peerreview papers.
%\begin{IEEEkeywords}
%Cooperative diversity, decode and forward, piecewise linear
%\end{IEEEkeywords}



% For peer review papers, you can put extra information on the cover
% page as needed:
% \ifCLASSOPTIONpeerreview
% \begin{center} \bfseries EDICS Category: 3-BBND \end{center}
% \fi
%
% For peerreview papers, this IEEEtran command inserts a page break and
% creates the second title. It will be ignored for other modes.
%\IEEEpeerreviewmaketitle




\item 
A die is thrown and a card is selected at random from a deck of 52 playing cards. The probability of getting an even number on the die and a spade card.\\
\solution
%\begin{table}[H]
	\centering
\begin{tabular}{|c|c|c|}
\hline
Random variable &Value &Definition\\ \hline
\multirow{3}{*}{X} &0 &Slips of Rs 1\\
&1 &Slips of Rs 5\\
&2 &Slips of Rs 13\\ \hline
\multirow{2}{*}{Y} &0 &Box A\\
&1 &Box B\\\hline
\end{tabular}
\caption{}
\label{tab:Distribution}
\end{table}
See \tabref{tab:Distribution}.
\begin{align}
p_{Y}\brak{k}= \begin{cases} 
      \frac{1}{3} & {k=0} \\
      \frac{2}{3 }& {k=1} 
   \end{cases}
   \\
p_{Y|X}\brak{0|0} = \frac{19}{25}\, 
p_{Y|X}\brak{0|1} = \frac{6}{25}\,
p_{Y|X}\brak{1|0} = \frac{45}{50}\,
p_{Y|X}\brak{1|2} = \frac{5}{50}
\end{align}
The desired probability is the probability that a slip drawn at random is marked other than Rs 1,
\begin{align}
&=1-p_X\brak{0}\\
&= p_X(1) + p_X(2)
\end{align}
Using Bayes theorem,
\begin{align}
&= p_Y\brak{0} \times \pr{Y=0 | X=1} + p_Y\brak{1} \times \pr{Y=1|X=2}\\
&=\frac{1}{3} \times \frac{6}{25} + \frac{2}{3} \times \frac{5}{50}\\
&=\frac{11}{75}
\end{align}

\newpage

%\tableofcontents

\bigskip

\renewcommand{\thefigure}{\theenumi}
\renewcommand{\thetable}{\theenumi}
%\renewcommand{\theequation}{\theenumi}

%\begin{abstract}
%%\boldmath
%In this letter, an algorithm for evaluating the exact analytical bit error rate  (BER)  for the piecewise linear (PL) combiner for  multiple relays is presented. Previous results were available only for upto three relays. The algorithm is unique in the sense that  the actual mathematical expressions, that are prohibitively large, need not be explicitly obtained. The diversity gain due to multiple relays is shown through plots of the analytical BER, well supported by simulations. 
%
%\end{abstract}
% IEEEtran.cls defaults to using nonbold math in the Abstract.
% This preserves the distinction between vectors and scalars. However,
% if the journal you are submitting to favors bold math in the abstract,
% then you can use LaTeX's standard command \boldmath at the very start
% of the abstract to achieve this. Many IEEE journals frown on math
% in the abstract anyway.

% Note that keywords are not normally used for peerreview papers.
%\begin{IEEEkeywords}
%Cooperative diversity, decode and forward, piecewise linear
%\end{IEEEkeywords}



% For peer review papers, you can put extra information on the cover
% page as needed:
% \ifCLASSOPTIONpeerreview
% \begin{center} \bfseries EDICS Category: 3-BBND \end{center}
% \fi
%
% For peerreview papers, this IEEEtran command inserts a page break and
% creates the second title. It will be ignored for other modes.
%\IEEEpeerreviewmaketitle




\item
If 4-digit numbers greater than 5,000 are randomly formed from the digits 0, 1, 3, 5, and 7, what is the probability of forming a number divisible by 5 when:
\begin{enumerate}
    \item The digits are repeated?
    \item The repetition of digits is not allowed?
\end{enumerate}
\solution
%\begin{table}[H]
	\centering
\begin{tabular}{|c|c|c|}
\hline
Random variable &Value &Definition\\ \hline
\multirow{3}{*}{X} &0 &Slips of Rs 1\\
&1 &Slips of Rs 5\\
&2 &Slips of Rs 13\\ \hline
\multirow{2}{*}{Y} &0 &Box A\\
&1 &Box B\\\hline
\end{tabular}
\caption{}
\label{tab:Distribution}
\end{table}
See \tabref{tab:Distribution}.
\begin{align}
p_{Y}\brak{k}= \begin{cases} 
      \frac{1}{3} & {k=0} \\
      \frac{2}{3 }& {k=1} 
   \end{cases}
   \\
p_{Y|X}\brak{0|0} = \frac{19}{25}\, 
p_{Y|X}\brak{0|1} = \frac{6}{25}\,
p_{Y|X}\brak{1|0} = \frac{45}{50}\,
p_{Y|X}\brak{1|2} = \frac{5}{50}
\end{align}
The desired probability is the probability that a slip drawn at random is marked other than Rs 1,
\begin{align}
&=1-p_X\brak{0}\\
&= p_X(1) + p_X(2)
\end{align}
Using Bayes theorem,
\begin{align}
&= p_Y\brak{0} \times \pr{Y=0 | X=1} + p_Y\brak{1} \times \pr{Y=1|X=2}\\
&=\frac{1}{3} \times \frac{6}{25} + \frac{2}{3} \times \frac{5}{50}\\
&=\frac{11}{75}
\end{align}

\newpage

%\tableofcontents

\bigskip

\renewcommand{\thefigure}{\theenumi}
\renewcommand{\thetable}{\theenumi}
%\renewcommand{\theequation}{\theenumi}

%\begin{abstract}
%%\boldmath
%In this letter, an algorithm for evaluating the exact analytical bit error rate  (BER)  for the piecewise linear (PL) combiner for  multiple relays is presented. Previous results were available only for upto three relays. The algorithm is unique in the sense that  the actual mathematical expressions, that are prohibitively large, need not be explicitly obtained. The diversity gain due to multiple relays is shown through plots of the analytical BER, well supported by simulations. 
%
%\end{abstract}
% IEEEtran.cls defaults to using nonbold math in the Abstract.
% This preserves the distinction between vectors and scalars. However,
% if the journal you are submitting to favors bold math in the abstract,
% then you can use LaTeX's standard command \boldmath at the very start
% of the abstract to achieve this. Many IEEE journals frown on math
% in the abstract anyway.

% Note that keywords are not normally used for peerreview papers.
%\begin{IEEEkeywords}
%Cooperative diversity, decode and forward, piecewise linear
%\end{IEEEkeywords}



% For peer review papers, you can put extra information on the cover
% page as needed:
% \ifCLASSOPTIONpeerreview
% \begin{center} \bfseries EDICS Category: 3-BBND \end{center}
% \fi
%
% For peerreview papers, this IEEEtran command inserts a page break and
% creates the second title. It will be ignored for other modes.
%\IEEEpeerreviewmaketitle




\item Consider the probability space $\brak{\Omega, \mathcal{G}, P}$ where $\Omega = [0,2]$ and $\mathcal{G} = \cbrak{\phi, \Omega, [0,1], (1,2]}$. Let $X$ and $Y$ be two functions on $\Omega$ defined as
\begin{align*}
    X(\omega) = 
    \begin{cases}
        1 & \text{if }\omega \in [0, 1]\\
        2 & \text{if }\omega \in (1, 2]
    \end{cases}
\end{align*}
and
\begin{align*}
    Y(\omega) = 
    \begin{cases}
        2 & \text{if }\omega \in [0, 1.5]\\
        3 & \text{if }\omega \in (1.5, 2].
    \end{cases}
\end{align*}
Then which one of the following statements is true?
\begin{enumerate}
    \item [(A)] $X$ is a random variable with respect to $\mathcal{G}$, but $Y$ is not a random variable with respect to $\mathcal{G}$.
    \item [(B)] $Y$ is a random variable with respect to $\mathcal{G}$, but $X$ is not a random variable with respect to $\mathcal{G}$.
    \item [(C)] Neither $X$ nor $Y$ is a random variable with respect to $\mathcal{G}$.
    \item [(D)] Both $X$ and $Y$ are random variables with respect to $\mathcal{G}$.
\end{enumerate} \hfill (GATE ST 2023)\\
\solution
%\begin{table}[H]
	\centering
\begin{tabular}{|c|c|c|}
\hline
Random variable &Value &Definition\\ \hline
\multirow{3}{*}{X} &0 &Slips of Rs 1\\
&1 &Slips of Rs 5\\
&2 &Slips of Rs 13\\ \hline
\multirow{2}{*}{Y} &0 &Box A\\
&1 &Box B\\\hline
\end{tabular}
\caption{}
\label{tab:Distribution}
\end{table}
See \tabref{tab:Distribution}.
\begin{align}
p_{Y}\brak{k}= \begin{cases} 
      \frac{1}{3} & {k=0} \\
      \frac{2}{3 }& {k=1} 
   \end{cases}
   \\
p_{Y|X}\brak{0|0} = \frac{19}{25}\, 
p_{Y|X}\brak{0|1} = \frac{6}{25}\,
p_{Y|X}\brak{1|0} = \frac{45}{50}\,
p_{Y|X}\brak{1|2} = \frac{5}{50}
\end{align}
The desired probability is the probability that a slip drawn at random is marked other than Rs 1,
\begin{align}
&=1-p_X\brak{0}\\
&= p_X(1) + p_X(2)
\end{align}
Using Bayes theorem,
\begin{align}
&= p_Y\brak{0} \times \pr{Y=0 | X=1} + p_Y\brak{1} \times \pr{Y=1|X=2}\\
&=\frac{1}{3} \times \frac{6}{25} + \frac{2}{3} \times \frac{5}{50}\\
&=\frac{11}{75}
\end{align}

\newpage

%\tableofcontents

\bigskip

\renewcommand{\thefigure}{\theenumi}
\renewcommand{\thetable}{\theenumi}
%\renewcommand{\theequation}{\theenumi}

%\begin{abstract}
%%\boldmath
%In this letter, an algorithm for evaluating the exact analytical bit error rate  (BER)  for the piecewise linear (PL) combiner for  multiple relays is presented. Previous results were available only for upto three relays. The algorithm is unique in the sense that  the actual mathematical expressions, that are prohibitively large, need not be explicitly obtained. The diversity gain due to multiple relays is shown through plots of the analytical BER, well supported by simulations. 
%
%\end{abstract}
% IEEEtran.cls defaults to using nonbold math in the Abstract.
% This preserves the distinction between vectors and scalars. However,
% if the journal you are submitting to favors bold math in the abstract,
% then you can use LaTeX's standard command \boldmath at the very start
% of the abstract to achieve this. Many IEEE journals frown on math
% in the abstract anyway.

% Note that keywords are not normally used for peerreview papers.
%\begin{IEEEkeywords}
%Cooperative diversity, decode and forward, piecewise linear
%\end{IEEEkeywords}



% For peer review papers, you can put extra information on the cover
% page as needed:
% \ifCLASSOPTIONpeerreview
% \begin{center} \bfseries EDICS Category: 3-BBND \end{center}
% \fi
%
% For peerreview papers, this IEEEtran command inserts a page break and
% creates the second title. It will be ignored for other modes.
%\IEEEpeerreviewmaketitle




	\item  A die is loaded in such a way that each odd number is twice as likely to occur as
each even number. Find $P(G)$, where $G$ is the event that a number greater than
3 occurs on a single roll of the die.
\\
\solution
		%\begin{table}[H]
	\centering
\begin{tabular}{|c|c|c|}
\hline
Random variable &Value &Definition\\ \hline
\multirow{3}{*}{X} &0 &Slips of Rs 1\\
&1 &Slips of Rs 5\\
&2 &Slips of Rs 13\\ \hline
\multirow{2}{*}{Y} &0 &Box A\\
&1 &Box B\\\hline
\end{tabular}
\caption{}
\label{tab:Distribution}
\end{table}
See \tabref{tab:Distribution}.
\begin{align}
p_{Y}\brak{k}= \begin{cases} 
      \frac{1}{3} & {k=0} \\
      \frac{2}{3 }& {k=1} 
   \end{cases}
   \\
p_{Y|X}\brak{0|0} = \frac{19}{25}\, 
p_{Y|X}\brak{0|1} = \frac{6}{25}\,
p_{Y|X}\brak{1|0} = \frac{45}{50}\,
p_{Y|X}\brak{1|2} = \frac{5}{50}
\end{align}
The desired probability is the probability that a slip drawn at random is marked other than Rs 1,
\begin{align}
&=1-p_X\brak{0}\\
&= p_X(1) + p_X(2)
\end{align}
Using Bayes theorem,
\begin{align}
&= p_Y\brak{0} \times \pr{Y=0 | X=1} + p_Y\brak{1} \times \pr{Y=1|X=2}\\
&=\frac{1}{3} \times \frac{6}{25} + \frac{2}{3} \times \frac{5}{50}\\
&=\frac{11}{75}
\end{align}

\newpage

%\tableofcontents

\bigskip

\renewcommand{\thefigure}{\theenumi}
\renewcommand{\thetable}{\theenumi}
%\renewcommand{\theequation}{\theenumi}

%\begin{abstract}
%%\boldmath
%In this letter, an algorithm for evaluating the exact analytical bit error rate  (BER)  for the piecewise linear (PL) combiner for  multiple relays is presented. Previous results were available only for upto three relays. The algorithm is unique in the sense that  the actual mathematical expressions, that are prohibitively large, need not be explicitly obtained. The diversity gain due to multiple relays is shown through plots of the analytical BER, well supported by simulations. 
%
%\end{abstract}
% IEEEtran.cls defaults to using nonbold math in the Abstract.
% This preserves the distinction between vectors and scalars. However,
% if the journal you are submitting to favors bold math in the abstract,
% then you can use LaTeX's standard command \boldmath at the very start
% of the abstract to achieve this. Many IEEE journals frown on math
% in the abstract anyway.

% Note that keywords are not normally used for peerreview papers.
%\begin{IEEEkeywords}
%Cooperative diversity, decode and forward, piecewise linear
%\end{IEEEkeywords}



% For peer review papers, you can put extra information on the cover
% page as needed:
% \ifCLASSOPTIONpeerreview
% \begin{center} \bfseries EDICS Category: 3-BBND \end{center}
% \fi
%
% For peerreview papers, this IEEEtran command inserts a page break and
% creates the second title. It will be ignored for other modes.
%\IEEEpeerreviewmaketitle




	\item All the jacks, queens and kings are removed from a deck of 52 playing cards. The remaining cards are well shuffled and then one card is drawn at random. Giving ace a value 1 similar value for other cards, find the probability that the card has a value 
		\begin{enumerate}
			\item 7
			\item greater than 7
			\item less than 7
		\end{enumerate}
		%Number of cards left after removing all jacks, queens and kings 
\begin{align}
N	= 52 - 4\times 3
	= 40
\end{align}
%\begin{table}[H]
%\def\arraystretch{1.2}
%\begin{tabular}{|c|c|c|}
%\hline
%	\textbf{Parameter} &\textbf{Value} &\textbf{Description}\\ \hline
%	$X$ &1-10 &Represents the value of the card picked \\ \hline
%\end{tabular}
%\end{table}
Let $1 \le X \le 10$ be the value of the card picked.  Then,
\begin{align}
	p_X(k) &= \Pr(X=k)\ \forall\ 1 \leq k \leq 10\\
	&= \frac{4\times 1}{40}\\
	&= \frac{1}{10}\\
	\therefore p_X(k) &= 
	\begin{cases}
		\frac{1}{10} & 1 \leq k \leq 10\\
		0 & \text{otherwise}
	\end{cases}
\end{align}
and
\begin{align}
	F_{X}(k) &= \sum_{m=0}^{k}p_{X}(m) \quad 1 \leq k \leq 10\\
	&= \frac{k}{10}\\
	\therefore F_{X}(k) &= 
	\begin{cases}
		0 & k \leq 0\\
		\frac{k}{10} & 1\leq k \leq 10\\
		1 & k > 10 
	\end{cases}
\end{align}
\begin{enumerate}
	\item Probability that card has value equal to 7 is
		\begin{align}
			 p_{X}(7)
			= \frac{1}{10}
		\end{align}
	\item Probability that card has value greater than 7 is
		\begin{align}
			1 - F_X(7)
			&= 1 - \frac{7}{10}
			\\
			&= \frac{3}{10}
		\end{align}
	\item Probability that card has value less than 7 is
		\begin{align}
			 F_{X}(6)
			=\frac{6}{10}
		\end{align}
\end{enumerate}

  \item A Lot consists of 48 mobile phones of which 42 are good, 3 have only minor defects and 3 have major defects.Varnika will buy a phone if it is good but the trader will only buy a mobile if it has no major defects. One phone is selected at random from the lot. What is the probability that it is
\begin{enumerate}
	\item acceptable to Varnika?
            \item acceptable to the trader?
\end{enumerate}
\solution
	%\begin{table}[H]
	\centering
\begin{tabular}{|c|c|c|}
\hline
Random variable &Value &Definition\\ \hline
\multirow{3}{*}{X} &0 &Slips of Rs 1\\
&1 &Slips of Rs 5\\
&2 &Slips of Rs 13\\ \hline
\multirow{2}{*}{Y} &0 &Box A\\
&1 &Box B\\\hline
\end{tabular}
\caption{}
\label{tab:Distribution}
\end{table}
See \tabref{tab:Distribution}.
\begin{align}
p_{Y}\brak{k}= \begin{cases} 
      \frac{1}{3} & {k=0} \\
      \frac{2}{3 }& {k=1} 
   \end{cases}
   \\
p_{Y|X}\brak{0|0} = \frac{19}{25}\, 
p_{Y|X}\brak{0|1} = \frac{6}{25}\,
p_{Y|X}\brak{1|0} = \frac{45}{50}\,
p_{Y|X}\brak{1|2} = \frac{5}{50}
\end{align}
The desired probability is the probability that a slip drawn at random is marked other than Rs 1,
\begin{align}
&=1-p_X\brak{0}\\
&= p_X(1) + p_X(2)
\end{align}
Using Bayes theorem,
\begin{align}
&= p_Y\brak{0} \times \pr{Y=0 | X=1} + p_Y\brak{1} \times \pr{Y=1|X=2}\\
&=\frac{1}{3} \times \frac{6}{25} + \frac{2}{3} \times \frac{5}{50}\\
&=\frac{11}{75}
\end{align}

\newpage

%\tableofcontents

\bigskip

\renewcommand{\thefigure}{\theenumi}
\renewcommand{\thetable}{\theenumi}
%\renewcommand{\theequation}{\theenumi}

%\begin{abstract}
%%\boldmath
%In this letter, an algorithm for evaluating the exact analytical bit error rate  (BER)  for the piecewise linear (PL) combiner for  multiple relays is presented. Previous results were available only for upto three relays. The algorithm is unique in the sense that  the actual mathematical expressions, that are prohibitively large, need not be explicitly obtained. The diversity gain due to multiple relays is shown through plots of the analytical BER, well supported by simulations. 
%
%\end{abstract}
% IEEEtran.cls defaults to using nonbold math in the Abstract.
% This preserves the distinction between vectors and scalars. However,
% if the journal you are submitting to favors bold math in the abstract,
% then you can use LaTeX's standard command \boldmath at the very start
% of the abstract to achieve this. Many IEEE journals frown on math
% in the abstract anyway.

% Note that keywords are not normally used for peerreview papers.
%\begin{IEEEkeywords}
%Cooperative diversity, decode and forward, piecewise linear
%\end{IEEEkeywords}



% For peer review papers, you can put extra information on the cover
% page as needed:
% \ifCLASSOPTIONpeerreview
% \begin{center} \bfseries EDICS Category: 3-BBND \end{center}
% \fi
%
% For peerreview papers, this IEEEtran command inserts a page break and
% creates the second title. It will be ignored for other modes.
%\IEEEpeerreviewmaketitle




 \item A student says that if you throw a die, it will show up 1 or not 1. Therefore, the probability of getting 1 and the probability of getting 'not 1' each is equal to $\frac{1}{2}$. Is this correct? Give reasons.\\
 \solution
        %\begin{table}[H]
	\centering
\begin{tabular}{|c|c|c|}
\hline
Random variable &Value &Definition\\ \hline
\multirow{3}{*}{X} &0 &Slips of Rs 1\\
&1 &Slips of Rs 5\\
&2 &Slips of Rs 13\\ \hline
\multirow{2}{*}{Y} &0 &Box A\\
&1 &Box B\\\hline
\end{tabular}
\caption{}
\label{tab:Distribution}
\end{table}
See \tabref{tab:Distribution}.
\begin{align}
p_{Y}\brak{k}= \begin{cases} 
      \frac{1}{3} & {k=0} \\
      \frac{2}{3 }& {k=1} 
   \end{cases}
   \\
p_{Y|X}\brak{0|0} = \frac{19}{25}\, 
p_{Y|X}\brak{0|1} = \frac{6}{25}\,
p_{Y|X}\brak{1|0} = \frac{45}{50}\,
p_{Y|X}\brak{1|2} = \frac{5}{50}
\end{align}
The desired probability is the probability that a slip drawn at random is marked other than Rs 1,
\begin{align}
&=1-p_X\brak{0}\\
&= p_X(1) + p_X(2)
\end{align}
Using Bayes theorem,
\begin{align}
&= p_Y\brak{0} \times \pr{Y=0 | X=1} + p_Y\brak{1} \times \pr{Y=1|X=2}\\
&=\frac{1}{3} \times \frac{6}{25} + \frac{2}{3} \times \frac{5}{50}\\
&=\frac{11}{75}
\end{align}

\newpage

%\tableofcontents

\bigskip

\renewcommand{\thefigure}{\theenumi}
\renewcommand{\thetable}{\theenumi}
%\renewcommand{\theequation}{\theenumi}

%\begin{abstract}
%%\boldmath
%In this letter, an algorithm for evaluating the exact analytical bit error rate  (BER)  for the piecewise linear (PL) combiner for  multiple relays is presented. Previous results were available only for upto three relays. The algorithm is unique in the sense that  the actual mathematical expressions, that are prohibitively large, need not be explicitly obtained. The diversity gain due to multiple relays is shown through plots of the analytical BER, well supported by simulations. 
%
%\end{abstract}
% IEEEtran.cls defaults to using nonbold math in the Abstract.
% This preserves the distinction between vectors and scalars. However,
% if the journal you are submitting to favors bold math in the abstract,
% then you can use LaTeX's standard command \boldmath at the very start
% of the abstract to achieve this. Many IEEE journals frown on math
% in the abstract anyway.

% Note that keywords are not normally used for peerreview papers.
%\begin{IEEEkeywords}
%Cooperative diversity, decode and forward, piecewise linear
%\end{IEEEkeywords}



% For peer review papers, you can put extra information on the cover
% page as needed:
% \ifCLASSOPTIONpeerreview
% \begin{center} \bfseries EDICS Category: 3-BBND \end{center}
% \fi
%
% For peerreview papers, this IEEEtran command inserts a page break and
% creates the second title. It will be ignored for other modes.
%\IEEEpeerreviewmaketitle




   \item Four candidates A, B, C, D have ap-
plied for the assignment to coach a school cricket
team. If A is twice as likely to be selected as B, and
B and C are given about the same chance of being
selected, while C is twice as likely to be selected
as D, what are the probabilities that
\begin{enumerate}
\item C will be selected?
\item A will not be selected?
\end{enumerate}
	%\begin{table}[H]
	\centering
\begin{tabular}{|c|c|c|}
\hline
Random variable &Value &Definition\\ \hline
\multirow{3}{*}{X} &0 &Slips of Rs 1\\
&1 &Slips of Rs 5\\
&2 &Slips of Rs 13\\ \hline
\multirow{2}{*}{Y} &0 &Box A\\
&1 &Box B\\\hline
\end{tabular}
\caption{}
\label{tab:Distribution}
\end{table}
See \tabref{tab:Distribution}.
\begin{align}
p_{Y}\brak{k}= \begin{cases} 
      \frac{1}{3} & {k=0} \\
      \frac{2}{3 }& {k=1} 
   \end{cases}
   \\
p_{Y|X}\brak{0|0} = \frac{19}{25}\, 
p_{Y|X}\brak{0|1} = \frac{6}{25}\,
p_{Y|X}\brak{1|0} = \frac{45}{50}\,
p_{Y|X}\brak{1|2} = \frac{5}{50}
\end{align}
The desired probability is the probability that a slip drawn at random is marked other than Rs 1,
\begin{align}
&=1-p_X\brak{0}\\
&= p_X(1) + p_X(2)
\end{align}
Using Bayes theorem,
\begin{align}
&= p_Y\brak{0} \times \pr{Y=0 | X=1} + p_Y\brak{1} \times \pr{Y=1|X=2}\\
&=\frac{1}{3} \times \frac{6}{25} + \frac{2}{3} \times \frac{5}{50}\\
&=\frac{11}{75}
\end{align}

\newpage

%\tableofcontents

\bigskip

\renewcommand{\thefigure}{\theenumi}
\renewcommand{\thetable}{\theenumi}
%\renewcommand{\theequation}{\theenumi}

%\begin{abstract}
%%\boldmath
%In this letter, an algorithm for evaluating the exact analytical bit error rate  (BER)  for the piecewise linear (PL) combiner for  multiple relays is presented. Previous results were available only for upto three relays. The algorithm is unique in the sense that  the actual mathematical expressions, that are prohibitively large, need not be explicitly obtained. The diversity gain due to multiple relays is shown through plots of the analytical BER, well supported by simulations. 
%
%\end{abstract}
% IEEEtran.cls defaults to using nonbold math in the Abstract.
% This preserves the distinction between vectors and scalars. However,
% if the journal you are submitting to favors bold math in the abstract,
% then you can use LaTeX's standard command \boldmath at the very start
% of the abstract to achieve this. Many IEEE journals frown on math
% in the abstract anyway.

% Note that keywords are not normally used for peerreview papers.
%\begin{IEEEkeywords}
%Cooperative diversity, decode and forward, piecewise linear
%\end{IEEEkeywords}



% For peer review papers, you can put extra information on the cover
% page as needed:
% \ifCLASSOPTIONpeerreview
% \begin{center} \bfseries EDICS Category: 3-BBND \end{center}
% \fi
%
% For peerreview papers, this IEEEtran command inserts a page break and
% creates the second title. It will be ignored for other modes.
%\IEEEpeerreviewmaketitle




 \item A bag contain 24 balls of which $x$ balls are red, $2x$ are white and $3x$ are blue. A ball is selected at random, What is the probability that it is
\begin{enumerate}[label=\alph*)]
\item not red ?
\item white ?
\end{enumerate}
%\begin{table}[H]
	\centering
\begin{tabular}{|c|c|c|}
\hline
Random variable &Value &Definition\\ \hline
\multirow{3}{*}{X} &0 &Slips of Rs 1\\
&1 &Slips of Rs 5\\
&2 &Slips of Rs 13\\ \hline
\multirow{2}{*}{Y} &0 &Box A\\
&1 &Box B\\\hline
\end{tabular}
\caption{}
\label{tab:Distribution}
\end{table}
See \tabref{tab:Distribution}.
\begin{align}
p_{Y}\brak{k}= \begin{cases} 
      \frac{1}{3} & {k=0} \\
      \frac{2}{3 }& {k=1} 
   \end{cases}
   \\
p_{Y|X}\brak{0|0} = \frac{19}{25}\, 
p_{Y|X}\brak{0|1} = \frac{6}{25}\,
p_{Y|X}\brak{1|0} = \frac{45}{50}\,
p_{Y|X}\brak{1|2} = \frac{5}{50}
\end{align}
The desired probability is the probability that a slip drawn at random is marked other than Rs 1,
\begin{align}
&=1-p_X\brak{0}\\
&= p_X(1) + p_X(2)
\end{align}
Using Bayes theorem,
\begin{align}
&= p_Y\brak{0} \times \pr{Y=0 | X=1} + p_Y\brak{1} \times \pr{Y=1|X=2}\\
&=\frac{1}{3} \times \frac{6}{25} + \frac{2}{3} \times \frac{5}{50}\\
&=\frac{11}{75}
\end{align}

\newpage

%\tableofcontents

\bigskip

\renewcommand{\thefigure}{\theenumi}
\renewcommand{\thetable}{\theenumi}
%\renewcommand{\theequation}{\theenumi}

%\begin{abstract}
%%\boldmath
%In this letter, an algorithm for evaluating the exact analytical bit error rate  (BER)  for the piecewise linear (PL) combiner for  multiple relays is presented. Previous results were available only for upto three relays. The algorithm is unique in the sense that  the actual mathematical expressions, that are prohibitively large, need not be explicitly obtained. The diversity gain due to multiple relays is shown through plots of the analytical BER, well supported by simulations. 
%
%\end{abstract}
% IEEEtran.cls defaults to using nonbold math in the Abstract.
% This preserves the distinction between vectors and scalars. However,
% if the journal you are submitting to favors bold math in the abstract,
% then you can use LaTeX's standard command \boldmath at the very start
% of the abstract to achieve this. Many IEEE journals frown on math
% in the abstract anyway.

% Note that keywords are not normally used for peerreview papers.
%\begin{IEEEkeywords}
%Cooperative diversity, decode and forward, piecewise linear
%\end{IEEEkeywords}



% For peer review papers, you can put extra information on the cover
% page as needed:
% \ifCLASSOPTIONpeerreview
% \begin{center} \bfseries EDICS Category: 3-BBND \end{center}
% \fi
%
% For peerreview papers, this IEEEtran command inserts a page break and
% creates the second title. It will be ignored for other modes.
%\IEEEpeerreviewmaketitle




If the letters of the word ASSASSINATION are arranged at random. Find the Probability that
\begin{enumerate}[label=(\alph*)]
\item Four $S's$ come consecutively in the word
\item Two  $I's$ and two $N's$ come together
\item All $A's$ are not coming together
\item No two $A's$ are coming together
\end{enumerate}
%\begin{table}[H]
	\centering
\begin{tabular}{|c|c|c|}
\hline
Random variable &Value &Definition\\ \hline
\multirow{3}{*}{X} &0 &Slips of Rs 1\\
&1 &Slips of Rs 5\\
&2 &Slips of Rs 13\\ \hline
\multirow{2}{*}{Y} &0 &Box A\\
&1 &Box B\\\hline
\end{tabular}
\caption{}
\label{tab:Distribution}
\end{table}
See \tabref{tab:Distribution}.
\begin{align}
p_{Y}\brak{k}= \begin{cases} 
      \frac{1}{3} & {k=0} \\
      \frac{2}{3 }& {k=1} 
   \end{cases}
   \\
p_{Y|X}\brak{0|0} = \frac{19}{25}\, 
p_{Y|X}\brak{0|1} = \frac{6}{25}\,
p_{Y|X}\brak{1|0} = \frac{45}{50}\,
p_{Y|X}\brak{1|2} = \frac{5}{50}
\end{align}
The desired probability is the probability that a slip drawn at random is marked other than Rs 1,
\begin{align}
&=1-p_X\brak{0}\\
&= p_X(1) + p_X(2)
\end{align}
Using Bayes theorem,
\begin{align}
&= p_Y\brak{0} \times \pr{Y=0 | X=1} + p_Y\brak{1} \times \pr{Y=1|X=2}\\
&=\frac{1}{3} \times \frac{6}{25} + \frac{2}{3} \times \frac{5}{50}\\
&=\frac{11}{75}
\end{align}

\newpage

%\tableofcontents

\bigskip

\renewcommand{\thefigure}{\theenumi}
\renewcommand{\thetable}{\theenumi}
%\renewcommand{\theequation}{\theenumi}

%\begin{abstract}
%%\boldmath
%In this letter, an algorithm for evaluating the exact analytical bit error rate  (BER)  for the piecewise linear (PL) combiner for  multiple relays is presented. Previous results were available only for upto three relays. The algorithm is unique in the sense that  the actual mathematical expressions, that are prohibitively large, need not be explicitly obtained. The diversity gain due to multiple relays is shown through plots of the analytical BER, well supported by simulations. 
%
%\end{abstract}
% IEEEtran.cls defaults to using nonbold math in the Abstract.
% This preserves the distinction between vectors and scalars. However,
% if the journal you are submitting to favors bold math in the abstract,
% then you can use LaTeX's standard command \boldmath at the very start
% of the abstract to achieve this. Many IEEE journals frown on math
% in the abstract anyway.

% Note that keywords are not normally used for peerreview papers.
%\begin{IEEEkeywords}
%Cooperative diversity, decode and forward, piecewise linear
%\end{IEEEkeywords}



% For peer review papers, you can put extra information on the cover
% page as needed:
% \ifCLASSOPTIONpeerreview
% \begin{center} \bfseries EDICS Category: 3-BBND \end{center}
% \fi
%
% For peerreview papers, this IEEEtran command inserts a page break and
% creates the second title. It will be ignored for other modes.
%\IEEEpeerreviewmaketitle




	\item One urn contains two black balls (labelled B1 and B2) and one white ball. A
	second urn contains one black ball and two white balls (labelled W1 and W2).
	Suppose the following experiment is performed. One of the two urns is chosen
	at random. Next a ball is randomly chosen from the urn. Then a second ball is
	chosen at random from the same urn without replacing the first ball.
	
	\begin{enumerate}
	\item What is the probability that two black balls are chosen?
	
	\item What is the probability that two balls of opposite colour are chosen?
	\end{enumerate}
	\solution
	%\begin{align}
    \label{eq:12.13.6.18.1}
	\because	\pr{A|B} &> \pr{A},\
\frac{\pr{AB}}{\pr{B}} > \pr{A}
\\
    \label{eq:12.13.6.18.2}
	\implies \pr{AB} &> \pr{A}\pr{B}
	\\
	\text{or, } \frac{\pr{AB}}{\pr{A}} &=\pr{B|A} > \pr{A}
\end{align}

\end{enumerate}

		%
\item 
Out of 100 students, two sections of 40 and 60 are formed. If you and your friend are among the 100 students, what is the probability that
\begin{enumerate}
\item you both enter the same section?
\item you both enter the different sections?
\end{enumerate}
\solution
		%\begin{enumerate}[label=\thesection.\arabic*,ref=\thesection.\theenumi]
	\item One card is drawn from a well-shuffled deck of 52 cards. Find the probability of getting
\begin{enumerate}
\item A king of red colour 
\item A face card 
\item A red face card
\item The jack of hearts
\item A spade
\item The queen of diamonds

\end{enumerate}
\solution
		%\begin{table}[H]
	\centering
\begin{tabular}{|c|c|c|}
\hline
Random variable &Value &Definition\\ \hline
\multirow{3}{*}{X} &0 &Slips of Rs 1\\
&1 &Slips of Rs 5\\
&2 &Slips of Rs 13\\ \hline
\multirow{2}{*}{Y} &0 &Box A\\
&1 &Box B\\\hline
\end{tabular}
\caption{}
\label{tab:Distribution}
\end{table}
See \tabref{tab:Distribution}.
\begin{align}
p_{Y}\brak{k}= \begin{cases} 
      \frac{1}{3} & {k=0} \\
      \frac{2}{3 }& {k=1} 
   \end{cases}
   \\
p_{Y|X}\brak{0|0} = \frac{19}{25}\, 
p_{Y|X}\brak{0|1} = \frac{6}{25}\,
p_{Y|X}\brak{1|0} = \frac{45}{50}\,
p_{Y|X}\brak{1|2} = \frac{5}{50}
\end{align}
The desired probability is the probability that a slip drawn at random is marked other than Rs 1,
\begin{align}
&=1-p_X\brak{0}\\
&= p_X(1) + p_X(2)
\end{align}
Using Bayes theorem,
\begin{align}
&= p_Y\brak{0} \times \pr{Y=0 | X=1} + p_Y\brak{1} \times \pr{Y=1|X=2}\\
&=\frac{1}{3} \times \frac{6}{25} + \frac{2}{3} \times \frac{5}{50}\\
&=\frac{11}{75}
\end{align}

\newpage

%\tableofcontents

\bigskip

\renewcommand{\thefigure}{\theenumi}
\renewcommand{\thetable}{\theenumi}
%\renewcommand{\theequation}{\theenumi}

%\begin{abstract}
%%\boldmath
%In this letter, an algorithm for evaluating the exact analytical bit error rate  (BER)  for the piecewise linear (PL) combiner for  multiple relays is presented. Previous results were available only for upto three relays. The algorithm is unique in the sense that  the actual mathematical expressions, that are prohibitively large, need not be explicitly obtained. The diversity gain due to multiple relays is shown through plots of the analytical BER, well supported by simulations. 
%
%\end{abstract}
% IEEEtran.cls defaults to using nonbold math in the Abstract.
% This preserves the distinction between vectors and scalars. However,
% if the journal you are submitting to favors bold math in the abstract,
% then you can use LaTeX's standard command \boldmath at the very start
% of the abstract to achieve this. Many IEEE journals frown on math
% in the abstract anyway.

% Note that keywords are not normally used for peerreview papers.
%\begin{IEEEkeywords}
%Cooperative diversity, decode and forward, piecewise linear
%\end{IEEEkeywords}



% For peer review papers, you can put extra information on the cover
% page as needed:
% \ifCLASSOPTIONpeerreview
% \begin{center} \bfseries EDICS Category: 3-BBND \end{center}
% \fi
%
% For peerreview papers, this IEEEtran command inserts a page break and
% creates the second title. It will be ignored for other modes.
%\IEEEpeerreviewmaketitle




	\item Five cards—the ten, jack, queen, king and ace of diamonds, are well-shuffled with their face downwards. One card is then picked up at random.
\begin{enumerate}
\item
What is the probability that the card is the queen? 
\item
If the queen is drawn and put aside, what is the probability that the second card picked up is (a) an ace? (b) a queen?\\
\end{enumerate}
\solution
		%\begin{enumerate}[label=\thesection.\arabic*,ref=\thesection.\theenumi]
	\item One card is drawn from a well-shuffled deck of 52 cards. Find the probability of getting
\begin{enumerate}
\item A king of red colour 
\item A face card 
\item A red face card
\item The jack of hearts
\item A spade
\item The queen of diamonds

\end{enumerate}
\solution
		%\input{ncert/10/15/1/14/main.tex}
	\item Five cards—the ten, jack, queen, king and ace of diamonds, are well-shuffled with their face downwards. One card is then picked up at random.
\begin{enumerate}
\item
What is the probability that the card is the queen? 
\item
If the queen is drawn and put aside, what is the probability that the second card picked up is (a) an ace? (b) a queen?\\
\end{enumerate}
\solution
		%\input{ncert/10/15/1/15/defs.tex}
	\item A bag contains $5$ red balls and some blue balls. If the probability of drawing a blue ball is double that if a red ball, determine the number of blue balls in the bag. 
		\\
\solution
		%\input{ncert/10/15/2/3/defs.tex}
	\item A card is selected from a pack of 52 cards.
 \begin{enumerate}[label=(\alph*)] 
                 \item How many points are there in the sample space?
                 \item Calculate the probability that the card is an ace of spades.
                 \item Calculate the probability that the card is (i) an ace and (ii) black card.
 \end{enumerate}
\solution
		%\input{ncert/11/16/3/4/main.tex}
\item Four cards are drawn from a well-shuffled deck of 52 cards. What is the probability of obtaining 3 diamonds and one spade.
\\
\solution
		%\input{ncert/11/16/4/2/defs.tex}
\item In a certain lottery 10,000 tickets are sold and ten equal prizes are awarded. What is the probability of not getting a prize if you buy (a) one ticket (b) two tickets (c) 10 tickets ?	
\\
\solution
		%\input{ncert/11/16/4/4/defs.tex}
		%
\item 
Out of 100 students, two sections of 40 and 60 are formed. If you and your friend are among the 100 students, what is the probability that
\begin{enumerate}
\item you both enter the same section?
\item you both enter the different sections?
\end{enumerate}
\solution
		%\input{ncert/11/16/4/5/defs.tex}
	\item 
The number lock of a suitcase has 4 wheels each labelled with ten digits i.e. from 0 to 9.The lock opens with a sequence of four digits with no repeats.What is the probability of a person getting the right sequence to open the suitcase.
\\
\solution
		%\input{ncert/11/16/4/10/defs.tex}
		%
\item 
Two cards are drawn at random and without replacement from a pack of 52 playing cards. Find the probability that both the cards are black.
\\
\solution
		%\input{ncert/12/13/2/2/defs.tex}
		\item A box of oranges is inspected by examining three randomly selected oranges drawn without replacement. If all the three oranges are good, the box is approved for sale, otherwise, it is rejected. Find the probability that a box containing 15 oranges out of which 12 are good and 3 are bad ones will be approved for sale.
		\label{ncert/12/13/2/3/defs.tex}
		\item Two balls are drawn at random with replacement from a box containing 10 black and 8 red balls. Find the probability that
		\label{ncert/12/13/2/12}
\begin{enumerate}
\item both balls are red.
\item first ball is black and second is red.
\item one of them is black and other is red.
\end{enumerate}

\item In a hostel, 60\% of the students read Hindi newspaper, 40\% read English newspaper and 20\% read both Hindi and English newspapers. A student is selected at random.
		\label{ncert/12/13/2/15}
\begin{enumerate}
\item Find the probability that she reads neither Hindi nor English newspapers.
\item If she reads Hindi newspaper, find the probability that she reads English newspaper.
\item If she reads English newspaper, find the probability that she reads Hindi newspaper.\\
\end{enumerate}
\item The probability of obtaining an even prime number on each die, when a pair of dice is rolled is 
\begin{enumerate}
    \item $0$ 
    
    \item $\frac{1}{3}$ 
    
    \item $\frac{1}{12}$ 
    
    \item $\frac{1}{36}$ 
\end{enumerate}
\solution
		%\input{ncert/12/13/2/17/defs.tex}
	\item A bag contains 4 red and 4 black balls, another bag contains 2 red and 6 black balls. One of the two bags is selected at random and a ball is drawn from the bag which is found to be red. Find the probability that the ball is drawn from the first bag.
\\
\solution
		%\input{ncert/12/13/3/2/main.tex}
  \item
  Cards with numbers 2 to 101 are placed in a box. A card is selected at random.Find the probability that the card has
\begin{enumerate}[label=(\roman*)]
	\item an even number 
	\item a square number
\end{enumerate}
\solution
%\input{exemplar/10/13/3/32/main.tex}
\item
The king, queen and jack of clubs are removed from a deck of 52 playing cards and then well shuffled. Now one card is drawn at random from the remaining cards.  Determine the probability that the card is
\begin{enumerate}[label=(\roman*)]
\item a club
\item 10 of hearts
\end{enumerate}
\solution
%\input{exemplar/10/13/3/29/main.tex}
\item A team of medical students doing their internship have to assist during surgeries
at a city hospital. The probabilities of surgeries rated as very complex, complex,
routine, simple or very simple are respectively, 0.15, 0.20, 0.31, 0.26, .08. Find
the probabilities that a particular surgery will be rated
\begin{enumerate}
	\item complex or very complex;
	\item neither very complex nor very simple;
	\item routine or complex
	\item routine or simple
\end{enumerate}
\solution
%\input{exemplar/11/16/3/8(1)/main.tex}
\item A card is selected from a pack of 52 cards.
\begin{enumerate}[label=(\alph*)]
    \item How many points are there in the sample space?
    \item Calculate the probability that the card is an ace of spades.
    \item Calculate the probability that the card is (i) an ace and (ii) black card.
\end{enumerate}
\solution
%\input{exemplar/11/16/3/4/main2.tex}
\item The probability that a non leap year selected at random will contain 53 sundays.
\\
\solution
%\input{exemplar/10/13/1/19/main.tex}
\item One of the four persons John, Rita, Aslam or Gurpreet will be promoted next
month. Consequently the sample space consists of four elementary outcomes
S = {John promoted, Rita promoted, Aslam promoted, Gurpreet promoted}
You are told that the chances of John’s promotion is same as that of Gurpreet,
Rita’s chances of promotion are twice as likely as Johns. Aslam’s chances are
four times that of John.
\begin{enumerate}
	\item Determine
	\begin{enumerate}
		\item P (John promoted)
		\item P (Rita promoted)
		\item P (Aslam promoted)
		\item P (Gurpreet promoted)
	\end{enumerate}
	\item If A = {John promoted or Gurpreet promoted}, find P (A).
\end{enumerate}
\solution
%\input{exemplar/11/16/3/10/main.tex}
\item A card is drawn from a deck of 52 cards. Find the probability of getting a king or a heart or a red card.\\
\solution
%\input{exemplar/11/16/3/15/main.tex}
\item The probability that a student will pass his examination is 0.73, the probability of
the student getting a compartment is 0.13, and the probability that the student will
either pass or get compartment is 0.96. State True or False.\\
\solution
%\input{exemplar/11/16/3/31/main.tex}
\item A card is selected from a pack of 52 cards\\
\begin{enumerate}[label=(\alph*)]
\item How many points are there in the sample space?
\item Calculate the probability that the cards is an ace of spades.
\item Calculate the probability that the card is (i) an ace (ii)black card.\\
\end{enumerate}
%\input{ncert/11/16/3/4_1/Prob_4.tex}
\item In a non-leap year, the probability of having 53 tuesdays or 53 wednesdays is\\
\solution
%\input{exemplar/11/16/3/18/main.tex}
\item There are 1000 sealed envelopes in a box, 10 of them contain a cash prize of
Rs 100 each, 100 of them contain a cash prize of Rs 50 each and 200 of them
contain a cash prize of Rs 10 each and rest do not contain any cash prize. If they
are well shuffled and an envelope is picked up out, what is the probability that it
contains no cash prize?\\
\solution
%\input{exemplar/10/13/3/34/main.tex}
\item 
A die is thrown and a card is selected at random from a deck of 52 playing cards. The probability of getting an even number on the die and a spade card.\\
\solution
%\input{exemplar/12/13/3/78/main.tex}
\item
If 4-digit numbers greater than 5,000 are randomly formed from the digits 0, 1, 3, 5, and 7, what is the probability of forming a number divisible by 5 when:
\begin{enumerate}
    \item The digits are repeated?
    \item The repetition of digits is not allowed?
\end{enumerate}
\solution
%\input{ncert/11/16/4/9/main.tex}
\item Consider the probability space $\brak{\Omega, \mathcal{G}, P}$ where $\Omega = [0,2]$ and $\mathcal{G} = \cbrak{\phi, \Omega, [0,1], (1,2]}$. Let $X$ and $Y$ be two functions on $\Omega$ defined as
\begin{align*}
    X(\omega) = 
    \begin{cases}
        1 & \text{if }\omega \in [0, 1]\\
        2 & \text{if }\omega \in (1, 2]
    \end{cases}
\end{align*}
and
\begin{align*}
    Y(\omega) = 
    \begin{cases}
        2 & \text{if }\omega \in [0, 1.5]\\
        3 & \text{if }\omega \in (1.5, 2].
    \end{cases}
\end{align*}
Then which one of the following statements is true?
\begin{enumerate}
    \item [(A)] $X$ is a random variable with respect to $\mathcal{G}$, but $Y$ is not a random variable with respect to $\mathcal{G}$.
    \item [(B)] $Y$ is a random variable with respect to $\mathcal{G}$, but $X$ is not a random variable with respect to $\mathcal{G}$.
    \item [(C)] Neither $X$ nor $Y$ is a random variable with respect to $\mathcal{G}$.
    \item [(D)] Both $X$ and $Y$ are random variables with respect to $\mathcal{G}$.
\end{enumerate} \hfill (GATE ST 2023)\\
\solution
%\input{gate/ST/2023/14/main.tex}
	\item  A die is loaded in such a way that each odd number is twice as likely to occur as
each even number. Find $P(G)$, where $G$ is the event that a number greater than
3 occurs on a single roll of the die.
\\
\solution
		%\input{exemplar/11/16/3/5/main.tex}
	\item All the jacks, queens and kings are removed from a deck of 52 playing cards. The remaining cards are well shuffled and then one card is drawn at random. Giving ace a value 1 similar value for other cards, find the probability that the card has a value 
		\begin{enumerate}
			\item 7
			\item greater than 7
			\item less than 7
		\end{enumerate}
		%\input{exemplar/10/13/3/30/main.tex}
  \item A Lot consists of 48 mobile phones of which 42 are good, 3 have only minor defects and 3 have major defects.Varnika will buy a phone if it is good but the trader will only buy a mobile if it has no major defects. One phone is selected at random from the lot. What is the probability that it is
\begin{enumerate}
	\item acceptable to Varnika?
            \item acceptable to the trader?
\end{enumerate}
\solution
	%\input{exemplar/10/13/3/40/main.tex}
 \item A student says that if you throw a die, it will show up 1 or not 1. Therefore, the probability of getting 1 and the probability of getting 'not 1' each is equal to $\frac{1}{2}$. Is this correct? Give reasons.\\
 \solution
        %\input{exemplar/10/13/2/9/main.tex}
   \item Four candidates A, B, C, D have ap-
plied for the assignment to coach a school cricket
team. If A is twice as likely to be selected as B, and
B and C are given about the same chance of being
selected, while C is twice as likely to be selected
as D, what are the probabilities that
\begin{enumerate}
\item C will be selected?
\item A will not be selected?
\end{enumerate}
	%\input{exemplar/11/16/3/9/main.tex}
 \item A bag contain 24 balls of which $x$ balls are red, $2x$ are white and $3x$ are blue. A ball is selected at random, What is the probability that it is
\begin{enumerate}[label=\alph*)]
\item not red ?
\item white ?
\end{enumerate}
%\input{exemplar/10/13/3/41/main.tex}
If the letters of the word ASSASSINATION are arranged at random. Find the Probability that
\begin{enumerate}[label=(\alph*)]
\item Four $S's$ come consecutively in the word
\item Two  $I's$ and two $N's$ come together
\item All $A's$ are not coming together
\item No two $A's$ are coming together
\end{enumerate}
%\input{exemplar/11/16/3/14/main.tex}
	\item One urn contains two black balls (labelled B1 and B2) and one white ball. A
	second urn contains one black ball and two white balls (labelled W1 and W2).
	Suppose the following experiment is performed. One of the two urns is chosen
	at random. Next a ball is randomly chosen from the urn. Then a second ball is
	chosen at random from the same urn without replacing the first ball.
	
	\begin{enumerate}
	\item What is the probability that two black balls are chosen?
	
	\item What is the probability that two balls of opposite colour are chosen?
	\end{enumerate}
	\solution
	%\input{exemplar/11/16/3/12/main1.tex}
\end{enumerate}

	\item A bag contains $5$ red balls and some blue balls. If the probability of drawing a blue ball is double that if a red ball, determine the number of blue balls in the bag. 
		\\
\solution
		%\begin{enumerate}[label=\thesection.\arabic*,ref=\thesection.\theenumi]
	\item One card is drawn from a well-shuffled deck of 52 cards. Find the probability of getting
\begin{enumerate}
\item A king of red colour 
\item A face card 
\item A red face card
\item The jack of hearts
\item A spade
\item The queen of diamonds

\end{enumerate}
\solution
		%\input{ncert/10/15/1/14/main.tex}
	\item Five cards—the ten, jack, queen, king and ace of diamonds, are well-shuffled with their face downwards. One card is then picked up at random.
\begin{enumerate}
\item
What is the probability that the card is the queen? 
\item
If the queen is drawn and put aside, what is the probability that the second card picked up is (a) an ace? (b) a queen?\\
\end{enumerate}
\solution
		%\input{ncert/10/15/1/15/defs.tex}
	\item A bag contains $5$ red balls and some blue balls. If the probability of drawing a blue ball is double that if a red ball, determine the number of blue balls in the bag. 
		\\
\solution
		%\input{ncert/10/15/2/3/defs.tex}
	\item A card is selected from a pack of 52 cards.
 \begin{enumerate}[label=(\alph*)] 
                 \item How many points are there in the sample space?
                 \item Calculate the probability that the card is an ace of spades.
                 \item Calculate the probability that the card is (i) an ace and (ii) black card.
 \end{enumerate}
\solution
		%\input{ncert/11/16/3/4/main.tex}
\item Four cards are drawn from a well-shuffled deck of 52 cards. What is the probability of obtaining 3 diamonds and one spade.
\\
\solution
		%\input{ncert/11/16/4/2/defs.tex}
\item In a certain lottery 10,000 tickets are sold and ten equal prizes are awarded. What is the probability of not getting a prize if you buy (a) one ticket (b) two tickets (c) 10 tickets ?	
\\
\solution
		%\input{ncert/11/16/4/4/defs.tex}
		%
\item 
Out of 100 students, two sections of 40 and 60 are formed. If you and your friend are among the 100 students, what is the probability that
\begin{enumerate}
\item you both enter the same section?
\item you both enter the different sections?
\end{enumerate}
\solution
		%\input{ncert/11/16/4/5/defs.tex}
	\item 
The number lock of a suitcase has 4 wheels each labelled with ten digits i.e. from 0 to 9.The lock opens with a sequence of four digits with no repeats.What is the probability of a person getting the right sequence to open the suitcase.
\\
\solution
		%\input{ncert/11/16/4/10/defs.tex}
		%
\item 
Two cards are drawn at random and without replacement from a pack of 52 playing cards. Find the probability that both the cards are black.
\\
\solution
		%\input{ncert/12/13/2/2/defs.tex}
		\item A box of oranges is inspected by examining three randomly selected oranges drawn without replacement. If all the three oranges are good, the box is approved for sale, otherwise, it is rejected. Find the probability that a box containing 15 oranges out of which 12 are good and 3 are bad ones will be approved for sale.
		\label{ncert/12/13/2/3/defs.tex}
		\item Two balls are drawn at random with replacement from a box containing 10 black and 8 red balls. Find the probability that
		\label{ncert/12/13/2/12}
\begin{enumerate}
\item both balls are red.
\item first ball is black and second is red.
\item one of them is black and other is red.
\end{enumerate}

\item In a hostel, 60\% of the students read Hindi newspaper, 40\% read English newspaper and 20\% read both Hindi and English newspapers. A student is selected at random.
		\label{ncert/12/13/2/15}
\begin{enumerate}
\item Find the probability that she reads neither Hindi nor English newspapers.
\item If she reads Hindi newspaper, find the probability that she reads English newspaper.
\item If she reads English newspaper, find the probability that she reads Hindi newspaper.\\
\end{enumerate}
\item The probability of obtaining an even prime number on each die, when a pair of dice is rolled is 
\begin{enumerate}
    \item $0$ 
    
    \item $\frac{1}{3}$ 
    
    \item $\frac{1}{12}$ 
    
    \item $\frac{1}{36}$ 
\end{enumerate}
\solution
		%\input{ncert/12/13/2/17/defs.tex}
	\item A bag contains 4 red and 4 black balls, another bag contains 2 red and 6 black balls. One of the two bags is selected at random and a ball is drawn from the bag which is found to be red. Find the probability that the ball is drawn from the first bag.
\\
\solution
		%\input{ncert/12/13/3/2/main.tex}
  \item
  Cards with numbers 2 to 101 are placed in a box. A card is selected at random.Find the probability that the card has
\begin{enumerate}[label=(\roman*)]
	\item an even number 
	\item a square number
\end{enumerate}
\solution
%\input{exemplar/10/13/3/32/main.tex}
\item
The king, queen and jack of clubs are removed from a deck of 52 playing cards and then well shuffled. Now one card is drawn at random from the remaining cards.  Determine the probability that the card is
\begin{enumerate}[label=(\roman*)]
\item a club
\item 10 of hearts
\end{enumerate}
\solution
%\input{exemplar/10/13/3/29/main.tex}
\item A team of medical students doing their internship have to assist during surgeries
at a city hospital. The probabilities of surgeries rated as very complex, complex,
routine, simple or very simple are respectively, 0.15, 0.20, 0.31, 0.26, .08. Find
the probabilities that a particular surgery will be rated
\begin{enumerate}
	\item complex or very complex;
	\item neither very complex nor very simple;
	\item routine or complex
	\item routine or simple
\end{enumerate}
\solution
%\input{exemplar/11/16/3/8(1)/main.tex}
\item A card is selected from a pack of 52 cards.
\begin{enumerate}[label=(\alph*)]
    \item How many points are there in the sample space?
    \item Calculate the probability that the card is an ace of spades.
    \item Calculate the probability that the card is (i) an ace and (ii) black card.
\end{enumerate}
\solution
%\input{exemplar/11/16/3/4/main2.tex}
\item The probability that a non leap year selected at random will contain 53 sundays.
\\
\solution
%\input{exemplar/10/13/1/19/main.tex}
\item One of the four persons John, Rita, Aslam or Gurpreet will be promoted next
month. Consequently the sample space consists of four elementary outcomes
S = {John promoted, Rita promoted, Aslam promoted, Gurpreet promoted}
You are told that the chances of John’s promotion is same as that of Gurpreet,
Rita’s chances of promotion are twice as likely as Johns. Aslam’s chances are
four times that of John.
\begin{enumerate}
	\item Determine
	\begin{enumerate}
		\item P (John promoted)
		\item P (Rita promoted)
		\item P (Aslam promoted)
		\item P (Gurpreet promoted)
	\end{enumerate}
	\item If A = {John promoted or Gurpreet promoted}, find P (A).
\end{enumerate}
\solution
%\input{exemplar/11/16/3/10/main.tex}
\item A card is drawn from a deck of 52 cards. Find the probability of getting a king or a heart or a red card.\\
\solution
%\input{exemplar/11/16/3/15/main.tex}
\item The probability that a student will pass his examination is 0.73, the probability of
the student getting a compartment is 0.13, and the probability that the student will
either pass or get compartment is 0.96. State True or False.\\
\solution
%\input{exemplar/11/16/3/31/main.tex}
\item A card is selected from a pack of 52 cards\\
\begin{enumerate}[label=(\alph*)]
\item How many points are there in the sample space?
\item Calculate the probability that the cards is an ace of spades.
\item Calculate the probability that the card is (i) an ace (ii)black card.\\
\end{enumerate}
%\input{ncert/11/16/3/4_1/Prob_4.tex}
\item In a non-leap year, the probability of having 53 tuesdays or 53 wednesdays is\\
\solution
%\input{exemplar/11/16/3/18/main.tex}
\item There are 1000 sealed envelopes in a box, 10 of them contain a cash prize of
Rs 100 each, 100 of them contain a cash prize of Rs 50 each and 200 of them
contain a cash prize of Rs 10 each and rest do not contain any cash prize. If they
are well shuffled and an envelope is picked up out, what is the probability that it
contains no cash prize?\\
\solution
%\input{exemplar/10/13/3/34/main.tex}
\item 
A die is thrown and a card is selected at random from a deck of 52 playing cards. The probability of getting an even number on the die and a spade card.\\
\solution
%\input{exemplar/12/13/3/78/main.tex}
\item
If 4-digit numbers greater than 5,000 are randomly formed from the digits 0, 1, 3, 5, and 7, what is the probability of forming a number divisible by 5 when:
\begin{enumerate}
    \item The digits are repeated?
    \item The repetition of digits is not allowed?
\end{enumerate}
\solution
%\input{ncert/11/16/4/9/main.tex}
\item Consider the probability space $\brak{\Omega, \mathcal{G}, P}$ where $\Omega = [0,2]$ and $\mathcal{G} = \cbrak{\phi, \Omega, [0,1], (1,2]}$. Let $X$ and $Y$ be two functions on $\Omega$ defined as
\begin{align*}
    X(\omega) = 
    \begin{cases}
        1 & \text{if }\omega \in [0, 1]\\
        2 & \text{if }\omega \in (1, 2]
    \end{cases}
\end{align*}
and
\begin{align*}
    Y(\omega) = 
    \begin{cases}
        2 & \text{if }\omega \in [0, 1.5]\\
        3 & \text{if }\omega \in (1.5, 2].
    \end{cases}
\end{align*}
Then which one of the following statements is true?
\begin{enumerate}
    \item [(A)] $X$ is a random variable with respect to $\mathcal{G}$, but $Y$ is not a random variable with respect to $\mathcal{G}$.
    \item [(B)] $Y$ is a random variable with respect to $\mathcal{G}$, but $X$ is not a random variable with respect to $\mathcal{G}$.
    \item [(C)] Neither $X$ nor $Y$ is a random variable with respect to $\mathcal{G}$.
    \item [(D)] Both $X$ and $Y$ are random variables with respect to $\mathcal{G}$.
\end{enumerate} \hfill (GATE ST 2023)\\
\solution
%\input{gate/ST/2023/14/main.tex}
	\item  A die is loaded in such a way that each odd number is twice as likely to occur as
each even number. Find $P(G)$, where $G$ is the event that a number greater than
3 occurs on a single roll of the die.
\\
\solution
		%\input{exemplar/11/16/3/5/main.tex}
	\item All the jacks, queens and kings are removed from a deck of 52 playing cards. The remaining cards are well shuffled and then one card is drawn at random. Giving ace a value 1 similar value for other cards, find the probability that the card has a value 
		\begin{enumerate}
			\item 7
			\item greater than 7
			\item less than 7
		\end{enumerate}
		%\input{exemplar/10/13/3/30/main.tex}
  \item A Lot consists of 48 mobile phones of which 42 are good, 3 have only minor defects and 3 have major defects.Varnika will buy a phone if it is good but the trader will only buy a mobile if it has no major defects. One phone is selected at random from the lot. What is the probability that it is
\begin{enumerate}
	\item acceptable to Varnika?
            \item acceptable to the trader?
\end{enumerate}
\solution
	%\input{exemplar/10/13/3/40/main.tex}
 \item A student says that if you throw a die, it will show up 1 or not 1. Therefore, the probability of getting 1 and the probability of getting 'not 1' each is equal to $\frac{1}{2}$. Is this correct? Give reasons.\\
 \solution
        %\input{exemplar/10/13/2/9/main.tex}
   \item Four candidates A, B, C, D have ap-
plied for the assignment to coach a school cricket
team. If A is twice as likely to be selected as B, and
B and C are given about the same chance of being
selected, while C is twice as likely to be selected
as D, what are the probabilities that
\begin{enumerate}
\item C will be selected?
\item A will not be selected?
\end{enumerate}
	%\input{exemplar/11/16/3/9/main.tex}
 \item A bag contain 24 balls of which $x$ balls are red, $2x$ are white and $3x$ are blue. A ball is selected at random, What is the probability that it is
\begin{enumerate}[label=\alph*)]
\item not red ?
\item white ?
\end{enumerate}
%\input{exemplar/10/13/3/41/main.tex}
If the letters of the word ASSASSINATION are arranged at random. Find the Probability that
\begin{enumerate}[label=(\alph*)]
\item Four $S's$ come consecutively in the word
\item Two  $I's$ and two $N's$ come together
\item All $A's$ are not coming together
\item No two $A's$ are coming together
\end{enumerate}
%\input{exemplar/11/16/3/14/main.tex}
	\item One urn contains two black balls (labelled B1 and B2) and one white ball. A
	second urn contains one black ball and two white balls (labelled W1 and W2).
	Suppose the following experiment is performed. One of the two urns is chosen
	at random. Next a ball is randomly chosen from the urn. Then a second ball is
	chosen at random from the same urn without replacing the first ball.
	
	\begin{enumerate}
	\item What is the probability that two black balls are chosen?
	
	\item What is the probability that two balls of opposite colour are chosen?
	\end{enumerate}
	\solution
	%\input{exemplar/11/16/3/12/main1.tex}
\end{enumerate}

	\item A card is selected from a pack of 52 cards.
 \begin{enumerate}[label=(\alph*)] 
                 \item How many points are there in the sample space?
                 \item Calculate the probability that the card is an ace of spades.
                 \item Calculate the probability that the card is (i) an ace and (ii) black card.
 \end{enumerate}
\solution
		%\begin{table}[H]
	\centering
\begin{tabular}{|c|c|c|}
\hline
Random variable &Value &Definition\\ \hline
\multirow{3}{*}{X} &0 &Slips of Rs 1\\
&1 &Slips of Rs 5\\
&2 &Slips of Rs 13\\ \hline
\multirow{2}{*}{Y} &0 &Box A\\
&1 &Box B\\\hline
\end{tabular}
\caption{}
\label{tab:Distribution}
\end{table}
See \tabref{tab:Distribution}.
\begin{align}
p_{Y}\brak{k}= \begin{cases} 
      \frac{1}{3} & {k=0} \\
      \frac{2}{3 }& {k=1} 
   \end{cases}
   \\
p_{Y|X}\brak{0|0} = \frac{19}{25}\, 
p_{Y|X}\brak{0|1} = \frac{6}{25}\,
p_{Y|X}\brak{1|0} = \frac{45}{50}\,
p_{Y|X}\brak{1|2} = \frac{5}{50}
\end{align}
The desired probability is the probability that a slip drawn at random is marked other than Rs 1,
\begin{align}
&=1-p_X\brak{0}\\
&= p_X(1) + p_X(2)
\end{align}
Using Bayes theorem,
\begin{align}
&= p_Y\brak{0} \times \pr{Y=0 | X=1} + p_Y\brak{1} \times \pr{Y=1|X=2}\\
&=\frac{1}{3} \times \frac{6}{25} + \frac{2}{3} \times \frac{5}{50}\\
&=\frac{11}{75}
\end{align}

\newpage

%\tableofcontents

\bigskip

\renewcommand{\thefigure}{\theenumi}
\renewcommand{\thetable}{\theenumi}
%\renewcommand{\theequation}{\theenumi}

%\begin{abstract}
%%\boldmath
%In this letter, an algorithm for evaluating the exact analytical bit error rate  (BER)  for the piecewise linear (PL) combiner for  multiple relays is presented. Previous results were available only for upto three relays. The algorithm is unique in the sense that  the actual mathematical expressions, that are prohibitively large, need not be explicitly obtained. The diversity gain due to multiple relays is shown through plots of the analytical BER, well supported by simulations. 
%
%\end{abstract}
% IEEEtran.cls defaults to using nonbold math in the Abstract.
% This preserves the distinction between vectors and scalars. However,
% if the journal you are submitting to favors bold math in the abstract,
% then you can use LaTeX's standard command \boldmath at the very start
% of the abstract to achieve this. Many IEEE journals frown on math
% in the abstract anyway.

% Note that keywords are not normally used for peerreview papers.
%\begin{IEEEkeywords}
%Cooperative diversity, decode and forward, piecewise linear
%\end{IEEEkeywords}



% For peer review papers, you can put extra information on the cover
% page as needed:
% \ifCLASSOPTIONpeerreview
% \begin{center} \bfseries EDICS Category: 3-BBND \end{center}
% \fi
%
% For peerreview papers, this IEEEtran command inserts a page break and
% creates the second title. It will be ignored for other modes.
%\IEEEpeerreviewmaketitle




\item Four cards are drawn from a well-shuffled deck of 52 cards. What is the probability of obtaining 3 diamonds and one spade.
\\
\solution
		%\begin{enumerate}[label=\thesection.\arabic*,ref=\thesection.\theenumi]
	\item One card is drawn from a well-shuffled deck of 52 cards. Find the probability of getting
\begin{enumerate}
\item A king of red colour 
\item A face card 
\item A red face card
\item The jack of hearts
\item A spade
\item The queen of diamonds

\end{enumerate}
\solution
		%\input{ncert/10/15/1/14/main.tex}
	\item Five cards—the ten, jack, queen, king and ace of diamonds, are well-shuffled with their face downwards. One card is then picked up at random.
\begin{enumerate}
\item
What is the probability that the card is the queen? 
\item
If the queen is drawn and put aside, what is the probability that the second card picked up is (a) an ace? (b) a queen?\\
\end{enumerate}
\solution
		%\input{ncert/10/15/1/15/defs.tex}
	\item A bag contains $5$ red balls and some blue balls. If the probability of drawing a blue ball is double that if a red ball, determine the number of blue balls in the bag. 
		\\
\solution
		%\input{ncert/10/15/2/3/defs.tex}
	\item A card is selected from a pack of 52 cards.
 \begin{enumerate}[label=(\alph*)] 
                 \item How many points are there in the sample space?
                 \item Calculate the probability that the card is an ace of spades.
                 \item Calculate the probability that the card is (i) an ace and (ii) black card.
 \end{enumerate}
\solution
		%\input{ncert/11/16/3/4/main.tex}
\item Four cards are drawn from a well-shuffled deck of 52 cards. What is the probability of obtaining 3 diamonds and one spade.
\\
\solution
		%\input{ncert/11/16/4/2/defs.tex}
\item In a certain lottery 10,000 tickets are sold and ten equal prizes are awarded. What is the probability of not getting a prize if you buy (a) one ticket (b) two tickets (c) 10 tickets ?	
\\
\solution
		%\input{ncert/11/16/4/4/defs.tex}
		%
\item 
Out of 100 students, two sections of 40 and 60 are formed. If you and your friend are among the 100 students, what is the probability that
\begin{enumerate}
\item you both enter the same section?
\item you both enter the different sections?
\end{enumerate}
\solution
		%\input{ncert/11/16/4/5/defs.tex}
	\item 
The number lock of a suitcase has 4 wheels each labelled with ten digits i.e. from 0 to 9.The lock opens with a sequence of four digits with no repeats.What is the probability of a person getting the right sequence to open the suitcase.
\\
\solution
		%\input{ncert/11/16/4/10/defs.tex}
		%
\item 
Two cards are drawn at random and without replacement from a pack of 52 playing cards. Find the probability that both the cards are black.
\\
\solution
		%\input{ncert/12/13/2/2/defs.tex}
		\item A box of oranges is inspected by examining three randomly selected oranges drawn without replacement. If all the three oranges are good, the box is approved for sale, otherwise, it is rejected. Find the probability that a box containing 15 oranges out of which 12 are good and 3 are bad ones will be approved for sale.
		\label{ncert/12/13/2/3/defs.tex}
		\item Two balls are drawn at random with replacement from a box containing 10 black and 8 red balls. Find the probability that
		\label{ncert/12/13/2/12}
\begin{enumerate}
\item both balls are red.
\item first ball is black and second is red.
\item one of them is black and other is red.
\end{enumerate}

\item In a hostel, 60\% of the students read Hindi newspaper, 40\% read English newspaper and 20\% read both Hindi and English newspapers. A student is selected at random.
		\label{ncert/12/13/2/15}
\begin{enumerate}
\item Find the probability that she reads neither Hindi nor English newspapers.
\item If she reads Hindi newspaper, find the probability that she reads English newspaper.
\item If she reads English newspaper, find the probability that she reads Hindi newspaper.\\
\end{enumerate}
\item The probability of obtaining an even prime number on each die, when a pair of dice is rolled is 
\begin{enumerate}
    \item $0$ 
    
    \item $\frac{1}{3}$ 
    
    \item $\frac{1}{12}$ 
    
    \item $\frac{1}{36}$ 
\end{enumerate}
\solution
		%\input{ncert/12/13/2/17/defs.tex}
	\item A bag contains 4 red and 4 black balls, another bag contains 2 red and 6 black balls. One of the two bags is selected at random and a ball is drawn from the bag which is found to be red. Find the probability that the ball is drawn from the first bag.
\\
\solution
		%\input{ncert/12/13/3/2/main.tex}
  \item
  Cards with numbers 2 to 101 are placed in a box. A card is selected at random.Find the probability that the card has
\begin{enumerate}[label=(\roman*)]
	\item an even number 
	\item a square number
\end{enumerate}
\solution
%\input{exemplar/10/13/3/32/main.tex}
\item
The king, queen and jack of clubs are removed from a deck of 52 playing cards and then well shuffled. Now one card is drawn at random from the remaining cards.  Determine the probability that the card is
\begin{enumerate}[label=(\roman*)]
\item a club
\item 10 of hearts
\end{enumerate}
\solution
%\input{exemplar/10/13/3/29/main.tex}
\item A team of medical students doing their internship have to assist during surgeries
at a city hospital. The probabilities of surgeries rated as very complex, complex,
routine, simple or very simple are respectively, 0.15, 0.20, 0.31, 0.26, .08. Find
the probabilities that a particular surgery will be rated
\begin{enumerate}
	\item complex or very complex;
	\item neither very complex nor very simple;
	\item routine or complex
	\item routine or simple
\end{enumerate}
\solution
%\input{exemplar/11/16/3/8(1)/main.tex}
\item A card is selected from a pack of 52 cards.
\begin{enumerate}[label=(\alph*)]
    \item How many points are there in the sample space?
    \item Calculate the probability that the card is an ace of spades.
    \item Calculate the probability that the card is (i) an ace and (ii) black card.
\end{enumerate}
\solution
%\input{exemplar/11/16/3/4/main2.tex}
\item The probability that a non leap year selected at random will contain 53 sundays.
\\
\solution
%\input{exemplar/10/13/1/19/main.tex}
\item One of the four persons John, Rita, Aslam or Gurpreet will be promoted next
month. Consequently the sample space consists of four elementary outcomes
S = {John promoted, Rita promoted, Aslam promoted, Gurpreet promoted}
You are told that the chances of John’s promotion is same as that of Gurpreet,
Rita’s chances of promotion are twice as likely as Johns. Aslam’s chances are
four times that of John.
\begin{enumerate}
	\item Determine
	\begin{enumerate}
		\item P (John promoted)
		\item P (Rita promoted)
		\item P (Aslam promoted)
		\item P (Gurpreet promoted)
	\end{enumerate}
	\item If A = {John promoted or Gurpreet promoted}, find P (A).
\end{enumerate}
\solution
%\input{exemplar/11/16/3/10/main.tex}
\item A card is drawn from a deck of 52 cards. Find the probability of getting a king or a heart or a red card.\\
\solution
%\input{exemplar/11/16/3/15/main.tex}
\item The probability that a student will pass his examination is 0.73, the probability of
the student getting a compartment is 0.13, and the probability that the student will
either pass or get compartment is 0.96. State True or False.\\
\solution
%\input{exemplar/11/16/3/31/main.tex}
\item A card is selected from a pack of 52 cards\\
\begin{enumerate}[label=(\alph*)]
\item How many points are there in the sample space?
\item Calculate the probability that the cards is an ace of spades.
\item Calculate the probability that the card is (i) an ace (ii)black card.\\
\end{enumerate}
%\input{ncert/11/16/3/4_1/Prob_4.tex}
\item In a non-leap year, the probability of having 53 tuesdays or 53 wednesdays is\\
\solution
%\input{exemplar/11/16/3/18/main.tex}
\item There are 1000 sealed envelopes in a box, 10 of them contain a cash prize of
Rs 100 each, 100 of them contain a cash prize of Rs 50 each and 200 of them
contain a cash prize of Rs 10 each and rest do not contain any cash prize. If they
are well shuffled and an envelope is picked up out, what is the probability that it
contains no cash prize?\\
\solution
%\input{exemplar/10/13/3/34/main.tex}
\item 
A die is thrown and a card is selected at random from a deck of 52 playing cards. The probability of getting an even number on the die and a spade card.\\
\solution
%\input{exemplar/12/13/3/78/main.tex}
\item
If 4-digit numbers greater than 5,000 are randomly formed from the digits 0, 1, 3, 5, and 7, what is the probability of forming a number divisible by 5 when:
\begin{enumerate}
    \item The digits are repeated?
    \item The repetition of digits is not allowed?
\end{enumerate}
\solution
%\input{ncert/11/16/4/9/main.tex}
\item Consider the probability space $\brak{\Omega, \mathcal{G}, P}$ where $\Omega = [0,2]$ and $\mathcal{G} = \cbrak{\phi, \Omega, [0,1], (1,2]}$. Let $X$ and $Y$ be two functions on $\Omega$ defined as
\begin{align*}
    X(\omega) = 
    \begin{cases}
        1 & \text{if }\omega \in [0, 1]\\
        2 & \text{if }\omega \in (1, 2]
    \end{cases}
\end{align*}
and
\begin{align*}
    Y(\omega) = 
    \begin{cases}
        2 & \text{if }\omega \in [0, 1.5]\\
        3 & \text{if }\omega \in (1.5, 2].
    \end{cases}
\end{align*}
Then which one of the following statements is true?
\begin{enumerate}
    \item [(A)] $X$ is a random variable with respect to $\mathcal{G}$, but $Y$ is not a random variable with respect to $\mathcal{G}$.
    \item [(B)] $Y$ is a random variable with respect to $\mathcal{G}$, but $X$ is not a random variable with respect to $\mathcal{G}$.
    \item [(C)] Neither $X$ nor $Y$ is a random variable with respect to $\mathcal{G}$.
    \item [(D)] Both $X$ and $Y$ are random variables with respect to $\mathcal{G}$.
\end{enumerate} \hfill (GATE ST 2023)\\
\solution
%\input{gate/ST/2023/14/main.tex}
	\item  A die is loaded in such a way that each odd number is twice as likely to occur as
each even number. Find $P(G)$, where $G$ is the event that a number greater than
3 occurs on a single roll of the die.
\\
\solution
		%\input{exemplar/11/16/3/5/main.tex}
	\item All the jacks, queens and kings are removed from a deck of 52 playing cards. The remaining cards are well shuffled and then one card is drawn at random. Giving ace a value 1 similar value for other cards, find the probability that the card has a value 
		\begin{enumerate}
			\item 7
			\item greater than 7
			\item less than 7
		\end{enumerate}
		%\input{exemplar/10/13/3/30/main.tex}
  \item A Lot consists of 48 mobile phones of which 42 are good, 3 have only minor defects and 3 have major defects.Varnika will buy a phone if it is good but the trader will only buy a mobile if it has no major defects. One phone is selected at random from the lot. What is the probability that it is
\begin{enumerate}
	\item acceptable to Varnika?
            \item acceptable to the trader?
\end{enumerate}
\solution
	%\input{exemplar/10/13/3/40/main.tex}
 \item A student says that if you throw a die, it will show up 1 or not 1. Therefore, the probability of getting 1 and the probability of getting 'not 1' each is equal to $\frac{1}{2}$. Is this correct? Give reasons.\\
 \solution
        %\input{exemplar/10/13/2/9/main.tex}
   \item Four candidates A, B, C, D have ap-
plied for the assignment to coach a school cricket
team. If A is twice as likely to be selected as B, and
B and C are given about the same chance of being
selected, while C is twice as likely to be selected
as D, what are the probabilities that
\begin{enumerate}
\item C will be selected?
\item A will not be selected?
\end{enumerate}
	%\input{exemplar/11/16/3/9/main.tex}
 \item A bag contain 24 balls of which $x$ balls are red, $2x$ are white and $3x$ are blue. A ball is selected at random, What is the probability that it is
\begin{enumerate}[label=\alph*)]
\item not red ?
\item white ?
\end{enumerate}
%\input{exemplar/10/13/3/41/main.tex}
If the letters of the word ASSASSINATION are arranged at random. Find the Probability that
\begin{enumerate}[label=(\alph*)]
\item Four $S's$ come consecutively in the word
\item Two  $I's$ and two $N's$ come together
\item All $A's$ are not coming together
\item No two $A's$ are coming together
\end{enumerate}
%\input{exemplar/11/16/3/14/main.tex}
	\item One urn contains two black balls (labelled B1 and B2) and one white ball. A
	second urn contains one black ball and two white balls (labelled W1 and W2).
	Suppose the following experiment is performed. One of the two urns is chosen
	at random. Next a ball is randomly chosen from the urn. Then a second ball is
	chosen at random from the same urn without replacing the first ball.
	
	\begin{enumerate}
	\item What is the probability that two black balls are chosen?
	
	\item What is the probability that two balls of opposite colour are chosen?
	\end{enumerate}
	\solution
	%\input{exemplar/11/16/3/12/main1.tex}
\end{enumerate}

\item In a certain lottery 10,000 tickets are sold and ten equal prizes are awarded. What is the probability of not getting a prize if you buy (a) one ticket (b) two tickets (c) 10 tickets ?	
\\
\solution
		%\begin{enumerate}[label=\thesection.\arabic*,ref=\thesection.\theenumi]
	\item One card is drawn from a well-shuffled deck of 52 cards. Find the probability of getting
\begin{enumerate}
\item A king of red colour 
\item A face card 
\item A red face card
\item The jack of hearts
\item A spade
\item The queen of diamonds

\end{enumerate}
\solution
		%\input{ncert/10/15/1/14/main.tex}
	\item Five cards—the ten, jack, queen, king and ace of diamonds, are well-shuffled with their face downwards. One card is then picked up at random.
\begin{enumerate}
\item
What is the probability that the card is the queen? 
\item
If the queen is drawn and put aside, what is the probability that the second card picked up is (a) an ace? (b) a queen?\\
\end{enumerate}
\solution
		%\input{ncert/10/15/1/15/defs.tex}
	\item A bag contains $5$ red balls and some blue balls. If the probability of drawing a blue ball is double that if a red ball, determine the number of blue balls in the bag. 
		\\
\solution
		%\input{ncert/10/15/2/3/defs.tex}
	\item A card is selected from a pack of 52 cards.
 \begin{enumerate}[label=(\alph*)] 
                 \item How many points are there in the sample space?
                 \item Calculate the probability that the card is an ace of spades.
                 \item Calculate the probability that the card is (i) an ace and (ii) black card.
 \end{enumerate}
\solution
		%\input{ncert/11/16/3/4/main.tex}
\item Four cards are drawn from a well-shuffled deck of 52 cards. What is the probability of obtaining 3 diamonds and one spade.
\\
\solution
		%\input{ncert/11/16/4/2/defs.tex}
\item In a certain lottery 10,000 tickets are sold and ten equal prizes are awarded. What is the probability of not getting a prize if you buy (a) one ticket (b) two tickets (c) 10 tickets ?	
\\
\solution
		%\input{ncert/11/16/4/4/defs.tex}
		%
\item 
Out of 100 students, two sections of 40 and 60 are formed. If you and your friend are among the 100 students, what is the probability that
\begin{enumerate}
\item you both enter the same section?
\item you both enter the different sections?
\end{enumerate}
\solution
		%\input{ncert/11/16/4/5/defs.tex}
	\item 
The number lock of a suitcase has 4 wheels each labelled with ten digits i.e. from 0 to 9.The lock opens with a sequence of four digits with no repeats.What is the probability of a person getting the right sequence to open the suitcase.
\\
\solution
		%\input{ncert/11/16/4/10/defs.tex}
		%
\item 
Two cards are drawn at random and without replacement from a pack of 52 playing cards. Find the probability that both the cards are black.
\\
\solution
		%\input{ncert/12/13/2/2/defs.tex}
		\item A box of oranges is inspected by examining three randomly selected oranges drawn without replacement. If all the three oranges are good, the box is approved for sale, otherwise, it is rejected. Find the probability that a box containing 15 oranges out of which 12 are good and 3 are bad ones will be approved for sale.
		\label{ncert/12/13/2/3/defs.tex}
		\item Two balls are drawn at random with replacement from a box containing 10 black and 8 red balls. Find the probability that
		\label{ncert/12/13/2/12}
\begin{enumerate}
\item both balls are red.
\item first ball is black and second is red.
\item one of them is black and other is red.
\end{enumerate}

\item In a hostel, 60\% of the students read Hindi newspaper, 40\% read English newspaper and 20\% read both Hindi and English newspapers. A student is selected at random.
		\label{ncert/12/13/2/15}
\begin{enumerate}
\item Find the probability that she reads neither Hindi nor English newspapers.
\item If she reads Hindi newspaper, find the probability that she reads English newspaper.
\item If she reads English newspaper, find the probability that she reads Hindi newspaper.\\
\end{enumerate}
\item The probability of obtaining an even prime number on each die, when a pair of dice is rolled is 
\begin{enumerate}
    \item $0$ 
    
    \item $\frac{1}{3}$ 
    
    \item $\frac{1}{12}$ 
    
    \item $\frac{1}{36}$ 
\end{enumerate}
\solution
		%\input{ncert/12/13/2/17/defs.tex}
	\item A bag contains 4 red and 4 black balls, another bag contains 2 red and 6 black balls. One of the two bags is selected at random and a ball is drawn from the bag which is found to be red. Find the probability that the ball is drawn from the first bag.
\\
\solution
		%\input{ncert/12/13/3/2/main.tex}
  \item
  Cards with numbers 2 to 101 are placed in a box. A card is selected at random.Find the probability that the card has
\begin{enumerate}[label=(\roman*)]
	\item an even number 
	\item a square number
\end{enumerate}
\solution
%\input{exemplar/10/13/3/32/main.tex}
\item
The king, queen and jack of clubs are removed from a deck of 52 playing cards and then well shuffled. Now one card is drawn at random from the remaining cards.  Determine the probability that the card is
\begin{enumerate}[label=(\roman*)]
\item a club
\item 10 of hearts
\end{enumerate}
\solution
%\input{exemplar/10/13/3/29/main.tex}
\item A team of medical students doing their internship have to assist during surgeries
at a city hospital. The probabilities of surgeries rated as very complex, complex,
routine, simple or very simple are respectively, 0.15, 0.20, 0.31, 0.26, .08. Find
the probabilities that a particular surgery will be rated
\begin{enumerate}
	\item complex or very complex;
	\item neither very complex nor very simple;
	\item routine or complex
	\item routine or simple
\end{enumerate}
\solution
%\input{exemplar/11/16/3/8(1)/main.tex}
\item A card is selected from a pack of 52 cards.
\begin{enumerate}[label=(\alph*)]
    \item How many points are there in the sample space?
    \item Calculate the probability that the card is an ace of spades.
    \item Calculate the probability that the card is (i) an ace and (ii) black card.
\end{enumerate}
\solution
%\input{exemplar/11/16/3/4/main2.tex}
\item The probability that a non leap year selected at random will contain 53 sundays.
\\
\solution
%\input{exemplar/10/13/1/19/main.tex}
\item One of the four persons John, Rita, Aslam or Gurpreet will be promoted next
month. Consequently the sample space consists of four elementary outcomes
S = {John promoted, Rita promoted, Aslam promoted, Gurpreet promoted}
You are told that the chances of John’s promotion is same as that of Gurpreet,
Rita’s chances of promotion are twice as likely as Johns. Aslam’s chances are
four times that of John.
\begin{enumerate}
	\item Determine
	\begin{enumerate}
		\item P (John promoted)
		\item P (Rita promoted)
		\item P (Aslam promoted)
		\item P (Gurpreet promoted)
	\end{enumerate}
	\item If A = {John promoted or Gurpreet promoted}, find P (A).
\end{enumerate}
\solution
%\input{exemplar/11/16/3/10/main.tex}
\item A card is drawn from a deck of 52 cards. Find the probability of getting a king or a heart or a red card.\\
\solution
%\input{exemplar/11/16/3/15/main.tex}
\item The probability that a student will pass his examination is 0.73, the probability of
the student getting a compartment is 0.13, and the probability that the student will
either pass or get compartment is 0.96. State True or False.\\
\solution
%\input{exemplar/11/16/3/31/main.tex}
\item A card is selected from a pack of 52 cards\\
\begin{enumerate}[label=(\alph*)]
\item How many points are there in the sample space?
\item Calculate the probability that the cards is an ace of spades.
\item Calculate the probability that the card is (i) an ace (ii)black card.\\
\end{enumerate}
%\input{ncert/11/16/3/4_1/Prob_4.tex}
\item In a non-leap year, the probability of having 53 tuesdays or 53 wednesdays is\\
\solution
%\input{exemplar/11/16/3/18/main.tex}
\item There are 1000 sealed envelopes in a box, 10 of them contain a cash prize of
Rs 100 each, 100 of them contain a cash prize of Rs 50 each and 200 of them
contain a cash prize of Rs 10 each and rest do not contain any cash prize. If they
are well shuffled and an envelope is picked up out, what is the probability that it
contains no cash prize?\\
\solution
%\input{exemplar/10/13/3/34/main.tex}
\item 
A die is thrown and a card is selected at random from a deck of 52 playing cards. The probability of getting an even number on the die and a spade card.\\
\solution
%\input{exemplar/12/13/3/78/main.tex}
\item
If 4-digit numbers greater than 5,000 are randomly formed from the digits 0, 1, 3, 5, and 7, what is the probability of forming a number divisible by 5 when:
\begin{enumerate}
    \item The digits are repeated?
    \item The repetition of digits is not allowed?
\end{enumerate}
\solution
%\input{ncert/11/16/4/9/main.tex}
\item Consider the probability space $\brak{\Omega, \mathcal{G}, P}$ where $\Omega = [0,2]$ and $\mathcal{G} = \cbrak{\phi, \Omega, [0,1], (1,2]}$. Let $X$ and $Y$ be two functions on $\Omega$ defined as
\begin{align*}
    X(\omega) = 
    \begin{cases}
        1 & \text{if }\omega \in [0, 1]\\
        2 & \text{if }\omega \in (1, 2]
    \end{cases}
\end{align*}
and
\begin{align*}
    Y(\omega) = 
    \begin{cases}
        2 & \text{if }\omega \in [0, 1.5]\\
        3 & \text{if }\omega \in (1.5, 2].
    \end{cases}
\end{align*}
Then which one of the following statements is true?
\begin{enumerate}
    \item [(A)] $X$ is a random variable with respect to $\mathcal{G}$, but $Y$ is not a random variable with respect to $\mathcal{G}$.
    \item [(B)] $Y$ is a random variable with respect to $\mathcal{G}$, but $X$ is not a random variable with respect to $\mathcal{G}$.
    \item [(C)] Neither $X$ nor $Y$ is a random variable with respect to $\mathcal{G}$.
    \item [(D)] Both $X$ and $Y$ are random variables with respect to $\mathcal{G}$.
\end{enumerate} \hfill (GATE ST 2023)\\
\solution
%\input{gate/ST/2023/14/main.tex}
	\item  A die is loaded in such a way that each odd number is twice as likely to occur as
each even number. Find $P(G)$, where $G$ is the event that a number greater than
3 occurs on a single roll of the die.
\\
\solution
		%\input{exemplar/11/16/3/5/main.tex}
	\item All the jacks, queens and kings are removed from a deck of 52 playing cards. The remaining cards are well shuffled and then one card is drawn at random. Giving ace a value 1 similar value for other cards, find the probability that the card has a value 
		\begin{enumerate}
			\item 7
			\item greater than 7
			\item less than 7
		\end{enumerate}
		%\input{exemplar/10/13/3/30/main.tex}
  \item A Lot consists of 48 mobile phones of which 42 are good, 3 have only minor defects and 3 have major defects.Varnika will buy a phone if it is good but the trader will only buy a mobile if it has no major defects. One phone is selected at random from the lot. What is the probability that it is
\begin{enumerate}
	\item acceptable to Varnika?
            \item acceptable to the trader?
\end{enumerate}
\solution
	%\input{exemplar/10/13/3/40/main.tex}
 \item A student says that if you throw a die, it will show up 1 or not 1. Therefore, the probability of getting 1 and the probability of getting 'not 1' each is equal to $\frac{1}{2}$. Is this correct? Give reasons.\\
 \solution
        %\input{exemplar/10/13/2/9/main.tex}
   \item Four candidates A, B, C, D have ap-
plied for the assignment to coach a school cricket
team. If A is twice as likely to be selected as B, and
B and C are given about the same chance of being
selected, while C is twice as likely to be selected
as D, what are the probabilities that
\begin{enumerate}
\item C will be selected?
\item A will not be selected?
\end{enumerate}
	%\input{exemplar/11/16/3/9/main.tex}
 \item A bag contain 24 balls of which $x$ balls are red, $2x$ are white and $3x$ are blue. A ball is selected at random, What is the probability that it is
\begin{enumerate}[label=\alph*)]
\item not red ?
\item white ?
\end{enumerate}
%\input{exemplar/10/13/3/41/main.tex}
If the letters of the word ASSASSINATION are arranged at random. Find the Probability that
\begin{enumerate}[label=(\alph*)]
\item Four $S's$ come consecutively in the word
\item Two  $I's$ and two $N's$ come together
\item All $A's$ are not coming together
\item No two $A's$ are coming together
\end{enumerate}
%\input{exemplar/11/16/3/14/main.tex}
	\item One urn contains two black balls (labelled B1 and B2) and one white ball. A
	second urn contains one black ball and two white balls (labelled W1 and W2).
	Suppose the following experiment is performed. One of the two urns is chosen
	at random. Next a ball is randomly chosen from the urn. Then a second ball is
	chosen at random from the same urn without replacing the first ball.
	
	\begin{enumerate}
	\item What is the probability that two black balls are chosen?
	
	\item What is the probability that two balls of opposite colour are chosen?
	\end{enumerate}
	\solution
	%\input{exemplar/11/16/3/12/main1.tex}
\end{enumerate}

		%
\item 
Out of 100 students, two sections of 40 and 60 are formed. If you and your friend are among the 100 students, what is the probability that
\begin{enumerate}
\item you both enter the same section?
\item you both enter the different sections?
\end{enumerate}
\solution
		%\begin{enumerate}[label=\thesection.\arabic*,ref=\thesection.\theenumi]
	\item One card is drawn from a well-shuffled deck of 52 cards. Find the probability of getting
\begin{enumerate}
\item A king of red colour 
\item A face card 
\item A red face card
\item The jack of hearts
\item A spade
\item The queen of diamonds

\end{enumerate}
\solution
		%\input{ncert/10/15/1/14/main.tex}
	\item Five cards—the ten, jack, queen, king and ace of diamonds, are well-shuffled with their face downwards. One card is then picked up at random.
\begin{enumerate}
\item
What is the probability that the card is the queen? 
\item
If the queen is drawn and put aside, what is the probability that the second card picked up is (a) an ace? (b) a queen?\\
\end{enumerate}
\solution
		%\input{ncert/10/15/1/15/defs.tex}
	\item A bag contains $5$ red balls and some blue balls. If the probability of drawing a blue ball is double that if a red ball, determine the number of blue balls in the bag. 
		\\
\solution
		%\input{ncert/10/15/2/3/defs.tex}
	\item A card is selected from a pack of 52 cards.
 \begin{enumerate}[label=(\alph*)] 
                 \item How many points are there in the sample space?
                 \item Calculate the probability that the card is an ace of spades.
                 \item Calculate the probability that the card is (i) an ace and (ii) black card.
 \end{enumerate}
\solution
		%\input{ncert/11/16/3/4/main.tex}
\item Four cards are drawn from a well-shuffled deck of 52 cards. What is the probability of obtaining 3 diamonds and one spade.
\\
\solution
		%\input{ncert/11/16/4/2/defs.tex}
\item In a certain lottery 10,000 tickets are sold and ten equal prizes are awarded. What is the probability of not getting a prize if you buy (a) one ticket (b) two tickets (c) 10 tickets ?	
\\
\solution
		%\input{ncert/11/16/4/4/defs.tex}
		%
\item 
Out of 100 students, two sections of 40 and 60 are formed. If you and your friend are among the 100 students, what is the probability that
\begin{enumerate}
\item you both enter the same section?
\item you both enter the different sections?
\end{enumerate}
\solution
		%\input{ncert/11/16/4/5/defs.tex}
	\item 
The number lock of a suitcase has 4 wheels each labelled with ten digits i.e. from 0 to 9.The lock opens with a sequence of four digits with no repeats.What is the probability of a person getting the right sequence to open the suitcase.
\\
\solution
		%\input{ncert/11/16/4/10/defs.tex}
		%
\item 
Two cards are drawn at random and without replacement from a pack of 52 playing cards. Find the probability that both the cards are black.
\\
\solution
		%\input{ncert/12/13/2/2/defs.tex}
		\item A box of oranges is inspected by examining three randomly selected oranges drawn without replacement. If all the three oranges are good, the box is approved for sale, otherwise, it is rejected. Find the probability that a box containing 15 oranges out of which 12 are good and 3 are bad ones will be approved for sale.
		\label{ncert/12/13/2/3/defs.tex}
		\item Two balls are drawn at random with replacement from a box containing 10 black and 8 red balls. Find the probability that
		\label{ncert/12/13/2/12}
\begin{enumerate}
\item both balls are red.
\item first ball is black and second is red.
\item one of them is black and other is red.
\end{enumerate}

\item In a hostel, 60\% of the students read Hindi newspaper, 40\% read English newspaper and 20\% read both Hindi and English newspapers. A student is selected at random.
		\label{ncert/12/13/2/15}
\begin{enumerate}
\item Find the probability that she reads neither Hindi nor English newspapers.
\item If she reads Hindi newspaper, find the probability that she reads English newspaper.
\item If she reads English newspaper, find the probability that she reads Hindi newspaper.\\
\end{enumerate}
\item The probability of obtaining an even prime number on each die, when a pair of dice is rolled is 
\begin{enumerate}
    \item $0$ 
    
    \item $\frac{1}{3}$ 
    
    \item $\frac{1}{12}$ 
    
    \item $\frac{1}{36}$ 
\end{enumerate}
\solution
		%\input{ncert/12/13/2/17/defs.tex}
	\item A bag contains 4 red and 4 black balls, another bag contains 2 red and 6 black balls. One of the two bags is selected at random and a ball is drawn from the bag which is found to be red. Find the probability that the ball is drawn from the first bag.
\\
\solution
		%\input{ncert/12/13/3/2/main.tex}
  \item
  Cards with numbers 2 to 101 are placed in a box. A card is selected at random.Find the probability that the card has
\begin{enumerate}[label=(\roman*)]
	\item an even number 
	\item a square number
\end{enumerate}
\solution
%\input{exemplar/10/13/3/32/main.tex}
\item
The king, queen and jack of clubs are removed from a deck of 52 playing cards and then well shuffled. Now one card is drawn at random from the remaining cards.  Determine the probability that the card is
\begin{enumerate}[label=(\roman*)]
\item a club
\item 10 of hearts
\end{enumerate}
\solution
%\input{exemplar/10/13/3/29/main.tex}
\item A team of medical students doing their internship have to assist during surgeries
at a city hospital. The probabilities of surgeries rated as very complex, complex,
routine, simple or very simple are respectively, 0.15, 0.20, 0.31, 0.26, .08. Find
the probabilities that a particular surgery will be rated
\begin{enumerate}
	\item complex or very complex;
	\item neither very complex nor very simple;
	\item routine or complex
	\item routine or simple
\end{enumerate}
\solution
%\input{exemplar/11/16/3/8(1)/main.tex}
\item A card is selected from a pack of 52 cards.
\begin{enumerate}[label=(\alph*)]
    \item How many points are there in the sample space?
    \item Calculate the probability that the card is an ace of spades.
    \item Calculate the probability that the card is (i) an ace and (ii) black card.
\end{enumerate}
\solution
%\input{exemplar/11/16/3/4/main2.tex}
\item The probability that a non leap year selected at random will contain 53 sundays.
\\
\solution
%\input{exemplar/10/13/1/19/main.tex}
\item One of the four persons John, Rita, Aslam or Gurpreet will be promoted next
month. Consequently the sample space consists of four elementary outcomes
S = {John promoted, Rita promoted, Aslam promoted, Gurpreet promoted}
You are told that the chances of John’s promotion is same as that of Gurpreet,
Rita’s chances of promotion are twice as likely as Johns. Aslam’s chances are
four times that of John.
\begin{enumerate}
	\item Determine
	\begin{enumerate}
		\item P (John promoted)
		\item P (Rita promoted)
		\item P (Aslam promoted)
		\item P (Gurpreet promoted)
	\end{enumerate}
	\item If A = {John promoted or Gurpreet promoted}, find P (A).
\end{enumerate}
\solution
%\input{exemplar/11/16/3/10/main.tex}
\item A card is drawn from a deck of 52 cards. Find the probability of getting a king or a heart or a red card.\\
\solution
%\input{exemplar/11/16/3/15/main.tex}
\item The probability that a student will pass his examination is 0.73, the probability of
the student getting a compartment is 0.13, and the probability that the student will
either pass or get compartment is 0.96. State True or False.\\
\solution
%\input{exemplar/11/16/3/31/main.tex}
\item A card is selected from a pack of 52 cards\\
\begin{enumerate}[label=(\alph*)]
\item How many points are there in the sample space?
\item Calculate the probability that the cards is an ace of spades.
\item Calculate the probability that the card is (i) an ace (ii)black card.\\
\end{enumerate}
%\input{ncert/11/16/3/4_1/Prob_4.tex}
\item In a non-leap year, the probability of having 53 tuesdays or 53 wednesdays is\\
\solution
%\input{exemplar/11/16/3/18/main.tex}
\item There are 1000 sealed envelopes in a box, 10 of them contain a cash prize of
Rs 100 each, 100 of them contain a cash prize of Rs 50 each and 200 of them
contain a cash prize of Rs 10 each and rest do not contain any cash prize. If they
are well shuffled and an envelope is picked up out, what is the probability that it
contains no cash prize?\\
\solution
%\input{exemplar/10/13/3/34/main.tex}
\item 
A die is thrown and a card is selected at random from a deck of 52 playing cards. The probability of getting an even number on the die and a spade card.\\
\solution
%\input{exemplar/12/13/3/78/main.tex}
\item
If 4-digit numbers greater than 5,000 are randomly formed from the digits 0, 1, 3, 5, and 7, what is the probability of forming a number divisible by 5 when:
\begin{enumerate}
    \item The digits are repeated?
    \item The repetition of digits is not allowed?
\end{enumerate}
\solution
%\input{ncert/11/16/4/9/main.tex}
\item Consider the probability space $\brak{\Omega, \mathcal{G}, P}$ where $\Omega = [0,2]$ and $\mathcal{G} = \cbrak{\phi, \Omega, [0,1], (1,2]}$. Let $X$ and $Y$ be two functions on $\Omega$ defined as
\begin{align*}
    X(\omega) = 
    \begin{cases}
        1 & \text{if }\omega \in [0, 1]\\
        2 & \text{if }\omega \in (1, 2]
    \end{cases}
\end{align*}
and
\begin{align*}
    Y(\omega) = 
    \begin{cases}
        2 & \text{if }\omega \in [0, 1.5]\\
        3 & \text{if }\omega \in (1.5, 2].
    \end{cases}
\end{align*}
Then which one of the following statements is true?
\begin{enumerate}
    \item [(A)] $X$ is a random variable with respect to $\mathcal{G}$, but $Y$ is not a random variable with respect to $\mathcal{G}$.
    \item [(B)] $Y$ is a random variable with respect to $\mathcal{G}$, but $X$ is not a random variable with respect to $\mathcal{G}$.
    \item [(C)] Neither $X$ nor $Y$ is a random variable with respect to $\mathcal{G}$.
    \item [(D)] Both $X$ and $Y$ are random variables with respect to $\mathcal{G}$.
\end{enumerate} \hfill (GATE ST 2023)\\
\solution
%\input{gate/ST/2023/14/main.tex}
	\item  A die is loaded in such a way that each odd number is twice as likely to occur as
each even number. Find $P(G)$, where $G$ is the event that a number greater than
3 occurs on a single roll of the die.
\\
\solution
		%\input{exemplar/11/16/3/5/main.tex}
	\item All the jacks, queens and kings are removed from a deck of 52 playing cards. The remaining cards are well shuffled and then one card is drawn at random. Giving ace a value 1 similar value for other cards, find the probability that the card has a value 
		\begin{enumerate}
			\item 7
			\item greater than 7
			\item less than 7
		\end{enumerate}
		%\input{exemplar/10/13/3/30/main.tex}
  \item A Lot consists of 48 mobile phones of which 42 are good, 3 have only minor defects and 3 have major defects.Varnika will buy a phone if it is good but the trader will only buy a mobile if it has no major defects. One phone is selected at random from the lot. What is the probability that it is
\begin{enumerate}
	\item acceptable to Varnika?
            \item acceptable to the trader?
\end{enumerate}
\solution
	%\input{exemplar/10/13/3/40/main.tex}
 \item A student says that if you throw a die, it will show up 1 or not 1. Therefore, the probability of getting 1 and the probability of getting 'not 1' each is equal to $\frac{1}{2}$. Is this correct? Give reasons.\\
 \solution
        %\input{exemplar/10/13/2/9/main.tex}
   \item Four candidates A, B, C, D have ap-
plied for the assignment to coach a school cricket
team. If A is twice as likely to be selected as B, and
B and C are given about the same chance of being
selected, while C is twice as likely to be selected
as D, what are the probabilities that
\begin{enumerate}
\item C will be selected?
\item A will not be selected?
\end{enumerate}
	%\input{exemplar/11/16/3/9/main.tex}
 \item A bag contain 24 balls of which $x$ balls are red, $2x$ are white and $3x$ are blue. A ball is selected at random, What is the probability that it is
\begin{enumerate}[label=\alph*)]
\item not red ?
\item white ?
\end{enumerate}
%\input{exemplar/10/13/3/41/main.tex}
If the letters of the word ASSASSINATION are arranged at random. Find the Probability that
\begin{enumerate}[label=(\alph*)]
\item Four $S's$ come consecutively in the word
\item Two  $I's$ and two $N's$ come together
\item All $A's$ are not coming together
\item No two $A's$ are coming together
\end{enumerate}
%\input{exemplar/11/16/3/14/main.tex}
	\item One urn contains two black balls (labelled B1 and B2) and one white ball. A
	second urn contains one black ball and two white balls (labelled W1 and W2).
	Suppose the following experiment is performed. One of the two urns is chosen
	at random. Next a ball is randomly chosen from the urn. Then a second ball is
	chosen at random from the same urn without replacing the first ball.
	
	\begin{enumerate}
	\item What is the probability that two black balls are chosen?
	
	\item What is the probability that two balls of opposite colour are chosen?
	\end{enumerate}
	\solution
	%\input{exemplar/11/16/3/12/main1.tex}
\end{enumerate}

	\item 
The number lock of a suitcase has 4 wheels each labelled with ten digits i.e. from 0 to 9.The lock opens with a sequence of four digits with no repeats.What is the probability of a person getting the right sequence to open the suitcase.
\\
\solution
		%\begin{enumerate}[label=\thesection.\arabic*,ref=\thesection.\theenumi]
	\item One card is drawn from a well-shuffled deck of 52 cards. Find the probability of getting
\begin{enumerate}
\item A king of red colour 
\item A face card 
\item A red face card
\item The jack of hearts
\item A spade
\item The queen of diamonds

\end{enumerate}
\solution
		%\input{ncert/10/15/1/14/main.tex}
	\item Five cards—the ten, jack, queen, king and ace of diamonds, are well-shuffled with their face downwards. One card is then picked up at random.
\begin{enumerate}
\item
What is the probability that the card is the queen? 
\item
If the queen is drawn and put aside, what is the probability that the second card picked up is (a) an ace? (b) a queen?\\
\end{enumerate}
\solution
		%\input{ncert/10/15/1/15/defs.tex}
	\item A bag contains $5$ red balls and some blue balls. If the probability of drawing a blue ball is double that if a red ball, determine the number of blue balls in the bag. 
		\\
\solution
		%\input{ncert/10/15/2/3/defs.tex}
	\item A card is selected from a pack of 52 cards.
 \begin{enumerate}[label=(\alph*)] 
                 \item How many points are there in the sample space?
                 \item Calculate the probability that the card is an ace of spades.
                 \item Calculate the probability that the card is (i) an ace and (ii) black card.
 \end{enumerate}
\solution
		%\input{ncert/11/16/3/4/main.tex}
\item Four cards are drawn from a well-shuffled deck of 52 cards. What is the probability of obtaining 3 diamonds and one spade.
\\
\solution
		%\input{ncert/11/16/4/2/defs.tex}
\item In a certain lottery 10,000 tickets are sold and ten equal prizes are awarded. What is the probability of not getting a prize if you buy (a) one ticket (b) two tickets (c) 10 tickets ?	
\\
\solution
		%\input{ncert/11/16/4/4/defs.tex}
		%
\item 
Out of 100 students, two sections of 40 and 60 are formed. If you and your friend are among the 100 students, what is the probability that
\begin{enumerate}
\item you both enter the same section?
\item you both enter the different sections?
\end{enumerate}
\solution
		%\input{ncert/11/16/4/5/defs.tex}
	\item 
The number lock of a suitcase has 4 wheels each labelled with ten digits i.e. from 0 to 9.The lock opens with a sequence of four digits with no repeats.What is the probability of a person getting the right sequence to open the suitcase.
\\
\solution
		%\input{ncert/11/16/4/10/defs.tex}
		%
\item 
Two cards are drawn at random and without replacement from a pack of 52 playing cards. Find the probability that both the cards are black.
\\
\solution
		%\input{ncert/12/13/2/2/defs.tex}
		\item A box of oranges is inspected by examining three randomly selected oranges drawn without replacement. If all the three oranges are good, the box is approved for sale, otherwise, it is rejected. Find the probability that a box containing 15 oranges out of which 12 are good and 3 are bad ones will be approved for sale.
		\label{ncert/12/13/2/3/defs.tex}
		\item Two balls are drawn at random with replacement from a box containing 10 black and 8 red balls. Find the probability that
		\label{ncert/12/13/2/12}
\begin{enumerate}
\item both balls are red.
\item first ball is black and second is red.
\item one of them is black and other is red.
\end{enumerate}

\item In a hostel, 60\% of the students read Hindi newspaper, 40\% read English newspaper and 20\% read both Hindi and English newspapers. A student is selected at random.
		\label{ncert/12/13/2/15}
\begin{enumerate}
\item Find the probability that she reads neither Hindi nor English newspapers.
\item If she reads Hindi newspaper, find the probability that she reads English newspaper.
\item If she reads English newspaper, find the probability that she reads Hindi newspaper.\\
\end{enumerate}
\item The probability of obtaining an even prime number on each die, when a pair of dice is rolled is 
\begin{enumerate}
    \item $0$ 
    
    \item $\frac{1}{3}$ 
    
    \item $\frac{1}{12}$ 
    
    \item $\frac{1}{36}$ 
\end{enumerate}
\solution
		%\input{ncert/12/13/2/17/defs.tex}
	\item A bag contains 4 red and 4 black balls, another bag contains 2 red and 6 black balls. One of the two bags is selected at random and a ball is drawn from the bag which is found to be red. Find the probability that the ball is drawn from the first bag.
\\
\solution
		%\input{ncert/12/13/3/2/main.tex}
  \item
  Cards with numbers 2 to 101 are placed in a box. A card is selected at random.Find the probability that the card has
\begin{enumerate}[label=(\roman*)]
	\item an even number 
	\item a square number
\end{enumerate}
\solution
%\input{exemplar/10/13/3/32/main.tex}
\item
The king, queen and jack of clubs are removed from a deck of 52 playing cards and then well shuffled. Now one card is drawn at random from the remaining cards.  Determine the probability that the card is
\begin{enumerate}[label=(\roman*)]
\item a club
\item 10 of hearts
\end{enumerate}
\solution
%\input{exemplar/10/13/3/29/main.tex}
\item A team of medical students doing their internship have to assist during surgeries
at a city hospital. The probabilities of surgeries rated as very complex, complex,
routine, simple or very simple are respectively, 0.15, 0.20, 0.31, 0.26, .08. Find
the probabilities that a particular surgery will be rated
\begin{enumerate}
	\item complex or very complex;
	\item neither very complex nor very simple;
	\item routine or complex
	\item routine or simple
\end{enumerate}
\solution
%\input{exemplar/11/16/3/8(1)/main.tex}
\item A card is selected from a pack of 52 cards.
\begin{enumerate}[label=(\alph*)]
    \item How many points are there in the sample space?
    \item Calculate the probability that the card is an ace of spades.
    \item Calculate the probability that the card is (i) an ace and (ii) black card.
\end{enumerate}
\solution
%\input{exemplar/11/16/3/4/main2.tex}
\item The probability that a non leap year selected at random will contain 53 sundays.
\\
\solution
%\input{exemplar/10/13/1/19/main.tex}
\item One of the four persons John, Rita, Aslam or Gurpreet will be promoted next
month. Consequently the sample space consists of four elementary outcomes
S = {John promoted, Rita promoted, Aslam promoted, Gurpreet promoted}
You are told that the chances of John’s promotion is same as that of Gurpreet,
Rita’s chances of promotion are twice as likely as Johns. Aslam’s chances are
four times that of John.
\begin{enumerate}
	\item Determine
	\begin{enumerate}
		\item P (John promoted)
		\item P (Rita promoted)
		\item P (Aslam promoted)
		\item P (Gurpreet promoted)
	\end{enumerate}
	\item If A = {John promoted or Gurpreet promoted}, find P (A).
\end{enumerate}
\solution
%\input{exemplar/11/16/3/10/main.tex}
\item A card is drawn from a deck of 52 cards. Find the probability of getting a king or a heart or a red card.\\
\solution
%\input{exemplar/11/16/3/15/main.tex}
\item The probability that a student will pass his examination is 0.73, the probability of
the student getting a compartment is 0.13, and the probability that the student will
either pass or get compartment is 0.96. State True or False.\\
\solution
%\input{exemplar/11/16/3/31/main.tex}
\item A card is selected from a pack of 52 cards\\
\begin{enumerate}[label=(\alph*)]
\item How many points are there in the sample space?
\item Calculate the probability that the cards is an ace of spades.
\item Calculate the probability that the card is (i) an ace (ii)black card.\\
\end{enumerate}
%\input{ncert/11/16/3/4_1/Prob_4.tex}
\item In a non-leap year, the probability of having 53 tuesdays or 53 wednesdays is\\
\solution
%\input{exemplar/11/16/3/18/main.tex}
\item There are 1000 sealed envelopes in a box, 10 of them contain a cash prize of
Rs 100 each, 100 of them contain a cash prize of Rs 50 each and 200 of them
contain a cash prize of Rs 10 each and rest do not contain any cash prize. If they
are well shuffled and an envelope is picked up out, what is the probability that it
contains no cash prize?\\
\solution
%\input{exemplar/10/13/3/34/main.tex}
\item 
A die is thrown and a card is selected at random from a deck of 52 playing cards. The probability of getting an even number on the die and a spade card.\\
\solution
%\input{exemplar/12/13/3/78/main.tex}
\item
If 4-digit numbers greater than 5,000 are randomly formed from the digits 0, 1, 3, 5, and 7, what is the probability of forming a number divisible by 5 when:
\begin{enumerate}
    \item The digits are repeated?
    \item The repetition of digits is not allowed?
\end{enumerate}
\solution
%\input{ncert/11/16/4/9/main.tex}
\item Consider the probability space $\brak{\Omega, \mathcal{G}, P}$ where $\Omega = [0,2]$ and $\mathcal{G} = \cbrak{\phi, \Omega, [0,1], (1,2]}$. Let $X$ and $Y$ be two functions on $\Omega$ defined as
\begin{align*}
    X(\omega) = 
    \begin{cases}
        1 & \text{if }\omega \in [0, 1]\\
        2 & \text{if }\omega \in (1, 2]
    \end{cases}
\end{align*}
and
\begin{align*}
    Y(\omega) = 
    \begin{cases}
        2 & \text{if }\omega \in [0, 1.5]\\
        3 & \text{if }\omega \in (1.5, 2].
    \end{cases}
\end{align*}
Then which one of the following statements is true?
\begin{enumerate}
    \item [(A)] $X$ is a random variable with respect to $\mathcal{G}$, but $Y$ is not a random variable with respect to $\mathcal{G}$.
    \item [(B)] $Y$ is a random variable with respect to $\mathcal{G}$, but $X$ is not a random variable with respect to $\mathcal{G}$.
    \item [(C)] Neither $X$ nor $Y$ is a random variable with respect to $\mathcal{G}$.
    \item [(D)] Both $X$ and $Y$ are random variables with respect to $\mathcal{G}$.
\end{enumerate} \hfill (GATE ST 2023)\\
\solution
%\input{gate/ST/2023/14/main.tex}
	\item  A die is loaded in such a way that each odd number is twice as likely to occur as
each even number. Find $P(G)$, where $G$ is the event that a number greater than
3 occurs on a single roll of the die.
\\
\solution
		%\input{exemplar/11/16/3/5/main.tex}
	\item All the jacks, queens and kings are removed from a deck of 52 playing cards. The remaining cards are well shuffled and then one card is drawn at random. Giving ace a value 1 similar value for other cards, find the probability that the card has a value 
		\begin{enumerate}
			\item 7
			\item greater than 7
			\item less than 7
		\end{enumerate}
		%\input{exemplar/10/13/3/30/main.tex}
  \item A Lot consists of 48 mobile phones of which 42 are good, 3 have only minor defects and 3 have major defects.Varnika will buy a phone if it is good but the trader will only buy a mobile if it has no major defects. One phone is selected at random from the lot. What is the probability that it is
\begin{enumerate}
	\item acceptable to Varnika?
            \item acceptable to the trader?
\end{enumerate}
\solution
	%\input{exemplar/10/13/3/40/main.tex}
 \item A student says that if you throw a die, it will show up 1 or not 1. Therefore, the probability of getting 1 and the probability of getting 'not 1' each is equal to $\frac{1}{2}$. Is this correct? Give reasons.\\
 \solution
        %\input{exemplar/10/13/2/9/main.tex}
   \item Four candidates A, B, C, D have ap-
plied for the assignment to coach a school cricket
team. If A is twice as likely to be selected as B, and
B and C are given about the same chance of being
selected, while C is twice as likely to be selected
as D, what are the probabilities that
\begin{enumerate}
\item C will be selected?
\item A will not be selected?
\end{enumerate}
	%\input{exemplar/11/16/3/9/main.tex}
 \item A bag contain 24 balls of which $x$ balls are red, $2x$ are white and $3x$ are blue. A ball is selected at random, What is the probability that it is
\begin{enumerate}[label=\alph*)]
\item not red ?
\item white ?
\end{enumerate}
%\input{exemplar/10/13/3/41/main.tex}
If the letters of the word ASSASSINATION are arranged at random. Find the Probability that
\begin{enumerate}[label=(\alph*)]
\item Four $S's$ come consecutively in the word
\item Two  $I's$ and two $N's$ come together
\item All $A's$ are not coming together
\item No two $A's$ are coming together
\end{enumerate}
%\input{exemplar/11/16/3/14/main.tex}
	\item One urn contains two black balls (labelled B1 and B2) and one white ball. A
	second urn contains one black ball and two white balls (labelled W1 and W2).
	Suppose the following experiment is performed. One of the two urns is chosen
	at random. Next a ball is randomly chosen from the urn. Then a second ball is
	chosen at random from the same urn without replacing the first ball.
	
	\begin{enumerate}
	\item What is the probability that two black balls are chosen?
	
	\item What is the probability that two balls of opposite colour are chosen?
	\end{enumerate}
	\solution
	%\input{exemplar/11/16/3/12/main1.tex}
\end{enumerate}

		%
\item 
Two cards are drawn at random and without replacement from a pack of 52 playing cards. Find the probability that both the cards are black.
\\
\solution
		%\begin{enumerate}[label=\thesection.\arabic*,ref=\thesection.\theenumi]
	\item One card is drawn from a well-shuffled deck of 52 cards. Find the probability of getting
\begin{enumerate}
\item A king of red colour 
\item A face card 
\item A red face card
\item The jack of hearts
\item A spade
\item The queen of diamonds

\end{enumerate}
\solution
		%\input{ncert/10/15/1/14/main.tex}
	\item Five cards—the ten, jack, queen, king and ace of diamonds, are well-shuffled with their face downwards. One card is then picked up at random.
\begin{enumerate}
\item
What is the probability that the card is the queen? 
\item
If the queen is drawn and put aside, what is the probability that the second card picked up is (a) an ace? (b) a queen?\\
\end{enumerate}
\solution
		%\input{ncert/10/15/1/15/defs.tex}
	\item A bag contains $5$ red balls and some blue balls. If the probability of drawing a blue ball is double that if a red ball, determine the number of blue balls in the bag. 
		\\
\solution
		%\input{ncert/10/15/2/3/defs.tex}
	\item A card is selected from a pack of 52 cards.
 \begin{enumerate}[label=(\alph*)] 
                 \item How many points are there in the sample space?
                 \item Calculate the probability that the card is an ace of spades.
                 \item Calculate the probability that the card is (i) an ace and (ii) black card.
 \end{enumerate}
\solution
		%\input{ncert/11/16/3/4/main.tex}
\item Four cards are drawn from a well-shuffled deck of 52 cards. What is the probability of obtaining 3 diamonds and one spade.
\\
\solution
		%\input{ncert/11/16/4/2/defs.tex}
\item In a certain lottery 10,000 tickets are sold and ten equal prizes are awarded. What is the probability of not getting a prize if you buy (a) one ticket (b) two tickets (c) 10 tickets ?	
\\
\solution
		%\input{ncert/11/16/4/4/defs.tex}
		%
\item 
Out of 100 students, two sections of 40 and 60 are formed. If you and your friend are among the 100 students, what is the probability that
\begin{enumerate}
\item you both enter the same section?
\item you both enter the different sections?
\end{enumerate}
\solution
		%\input{ncert/11/16/4/5/defs.tex}
	\item 
The number lock of a suitcase has 4 wheels each labelled with ten digits i.e. from 0 to 9.The lock opens with a sequence of four digits with no repeats.What is the probability of a person getting the right sequence to open the suitcase.
\\
\solution
		%\input{ncert/11/16/4/10/defs.tex}
		%
\item 
Two cards are drawn at random and without replacement from a pack of 52 playing cards. Find the probability that both the cards are black.
\\
\solution
		%\input{ncert/12/13/2/2/defs.tex}
		\item A box of oranges is inspected by examining three randomly selected oranges drawn without replacement. If all the three oranges are good, the box is approved for sale, otherwise, it is rejected. Find the probability that a box containing 15 oranges out of which 12 are good and 3 are bad ones will be approved for sale.
		\label{ncert/12/13/2/3/defs.tex}
		\item Two balls are drawn at random with replacement from a box containing 10 black and 8 red balls. Find the probability that
		\label{ncert/12/13/2/12}
\begin{enumerate}
\item both balls are red.
\item first ball is black and second is red.
\item one of them is black and other is red.
\end{enumerate}

\item In a hostel, 60\% of the students read Hindi newspaper, 40\% read English newspaper and 20\% read both Hindi and English newspapers. A student is selected at random.
		\label{ncert/12/13/2/15}
\begin{enumerate}
\item Find the probability that she reads neither Hindi nor English newspapers.
\item If she reads Hindi newspaper, find the probability that she reads English newspaper.
\item If she reads English newspaper, find the probability that she reads Hindi newspaper.\\
\end{enumerate}
\item The probability of obtaining an even prime number on each die, when a pair of dice is rolled is 
\begin{enumerate}
    \item $0$ 
    
    \item $\frac{1}{3}$ 
    
    \item $\frac{1}{12}$ 
    
    \item $\frac{1}{36}$ 
\end{enumerate}
\solution
		%\input{ncert/12/13/2/17/defs.tex}
	\item A bag contains 4 red and 4 black balls, another bag contains 2 red and 6 black balls. One of the two bags is selected at random and a ball is drawn from the bag which is found to be red. Find the probability that the ball is drawn from the first bag.
\\
\solution
		%\input{ncert/12/13/3/2/main.tex}
  \item
  Cards with numbers 2 to 101 are placed in a box. A card is selected at random.Find the probability that the card has
\begin{enumerate}[label=(\roman*)]
	\item an even number 
	\item a square number
\end{enumerate}
\solution
%\input{exemplar/10/13/3/32/main.tex}
\item
The king, queen and jack of clubs are removed from a deck of 52 playing cards and then well shuffled. Now one card is drawn at random from the remaining cards.  Determine the probability that the card is
\begin{enumerate}[label=(\roman*)]
\item a club
\item 10 of hearts
\end{enumerate}
\solution
%\input{exemplar/10/13/3/29/main.tex}
\item A team of medical students doing their internship have to assist during surgeries
at a city hospital. The probabilities of surgeries rated as very complex, complex,
routine, simple or very simple are respectively, 0.15, 0.20, 0.31, 0.26, .08. Find
the probabilities that a particular surgery will be rated
\begin{enumerate}
	\item complex or very complex;
	\item neither very complex nor very simple;
	\item routine or complex
	\item routine or simple
\end{enumerate}
\solution
%\input{exemplar/11/16/3/8(1)/main.tex}
\item A card is selected from a pack of 52 cards.
\begin{enumerate}[label=(\alph*)]
    \item How many points are there in the sample space?
    \item Calculate the probability that the card is an ace of spades.
    \item Calculate the probability that the card is (i) an ace and (ii) black card.
\end{enumerate}
\solution
%\input{exemplar/11/16/3/4/main2.tex}
\item The probability that a non leap year selected at random will contain 53 sundays.
\\
\solution
%\input{exemplar/10/13/1/19/main.tex}
\item One of the four persons John, Rita, Aslam or Gurpreet will be promoted next
month. Consequently the sample space consists of four elementary outcomes
S = {John promoted, Rita promoted, Aslam promoted, Gurpreet promoted}
You are told that the chances of John’s promotion is same as that of Gurpreet,
Rita’s chances of promotion are twice as likely as Johns. Aslam’s chances are
four times that of John.
\begin{enumerate}
	\item Determine
	\begin{enumerate}
		\item P (John promoted)
		\item P (Rita promoted)
		\item P (Aslam promoted)
		\item P (Gurpreet promoted)
	\end{enumerate}
	\item If A = {John promoted or Gurpreet promoted}, find P (A).
\end{enumerate}
\solution
%\input{exemplar/11/16/3/10/main.tex}
\item A card is drawn from a deck of 52 cards. Find the probability of getting a king or a heart or a red card.\\
\solution
%\input{exemplar/11/16/3/15/main.tex}
\item The probability that a student will pass his examination is 0.73, the probability of
the student getting a compartment is 0.13, and the probability that the student will
either pass or get compartment is 0.96. State True or False.\\
\solution
%\input{exemplar/11/16/3/31/main.tex}
\item A card is selected from a pack of 52 cards\\
\begin{enumerate}[label=(\alph*)]
\item How many points are there in the sample space?
\item Calculate the probability that the cards is an ace of spades.
\item Calculate the probability that the card is (i) an ace (ii)black card.\\
\end{enumerate}
%\input{ncert/11/16/3/4_1/Prob_4.tex}
\item In a non-leap year, the probability of having 53 tuesdays or 53 wednesdays is\\
\solution
%\input{exemplar/11/16/3/18/main.tex}
\item There are 1000 sealed envelopes in a box, 10 of them contain a cash prize of
Rs 100 each, 100 of them contain a cash prize of Rs 50 each and 200 of them
contain a cash prize of Rs 10 each and rest do not contain any cash prize. If they
are well shuffled and an envelope is picked up out, what is the probability that it
contains no cash prize?\\
\solution
%\input{exemplar/10/13/3/34/main.tex}
\item 
A die is thrown and a card is selected at random from a deck of 52 playing cards. The probability of getting an even number on the die and a spade card.\\
\solution
%\input{exemplar/12/13/3/78/main.tex}
\item
If 4-digit numbers greater than 5,000 are randomly formed from the digits 0, 1, 3, 5, and 7, what is the probability of forming a number divisible by 5 when:
\begin{enumerate}
    \item The digits are repeated?
    \item The repetition of digits is not allowed?
\end{enumerate}
\solution
%\input{ncert/11/16/4/9/main.tex}
\item Consider the probability space $\brak{\Omega, \mathcal{G}, P}$ where $\Omega = [0,2]$ and $\mathcal{G} = \cbrak{\phi, \Omega, [0,1], (1,2]}$. Let $X$ and $Y$ be two functions on $\Omega$ defined as
\begin{align*}
    X(\omega) = 
    \begin{cases}
        1 & \text{if }\omega \in [0, 1]\\
        2 & \text{if }\omega \in (1, 2]
    \end{cases}
\end{align*}
and
\begin{align*}
    Y(\omega) = 
    \begin{cases}
        2 & \text{if }\omega \in [0, 1.5]\\
        3 & \text{if }\omega \in (1.5, 2].
    \end{cases}
\end{align*}
Then which one of the following statements is true?
\begin{enumerate}
    \item [(A)] $X$ is a random variable with respect to $\mathcal{G}$, but $Y$ is not a random variable with respect to $\mathcal{G}$.
    \item [(B)] $Y$ is a random variable with respect to $\mathcal{G}$, but $X$ is not a random variable with respect to $\mathcal{G}$.
    \item [(C)] Neither $X$ nor $Y$ is a random variable with respect to $\mathcal{G}$.
    \item [(D)] Both $X$ and $Y$ are random variables with respect to $\mathcal{G}$.
\end{enumerate} \hfill (GATE ST 2023)\\
\solution
%\input{gate/ST/2023/14/main.tex}
	\item  A die is loaded in such a way that each odd number is twice as likely to occur as
each even number. Find $P(G)$, where $G$ is the event that a number greater than
3 occurs on a single roll of the die.
\\
\solution
		%\input{exemplar/11/16/3/5/main.tex}
	\item All the jacks, queens and kings are removed from a deck of 52 playing cards. The remaining cards are well shuffled and then one card is drawn at random. Giving ace a value 1 similar value for other cards, find the probability that the card has a value 
		\begin{enumerate}
			\item 7
			\item greater than 7
			\item less than 7
		\end{enumerate}
		%\input{exemplar/10/13/3/30/main.tex}
  \item A Lot consists of 48 mobile phones of which 42 are good, 3 have only minor defects and 3 have major defects.Varnika will buy a phone if it is good but the trader will only buy a mobile if it has no major defects. One phone is selected at random from the lot. What is the probability that it is
\begin{enumerate}
	\item acceptable to Varnika?
            \item acceptable to the trader?
\end{enumerate}
\solution
	%\input{exemplar/10/13/3/40/main.tex}
 \item A student says that if you throw a die, it will show up 1 or not 1. Therefore, the probability of getting 1 and the probability of getting 'not 1' each is equal to $\frac{1}{2}$. Is this correct? Give reasons.\\
 \solution
        %\input{exemplar/10/13/2/9/main.tex}
   \item Four candidates A, B, C, D have ap-
plied for the assignment to coach a school cricket
team. If A is twice as likely to be selected as B, and
B and C are given about the same chance of being
selected, while C is twice as likely to be selected
as D, what are the probabilities that
\begin{enumerate}
\item C will be selected?
\item A will not be selected?
\end{enumerate}
	%\input{exemplar/11/16/3/9/main.tex}
 \item A bag contain 24 balls of which $x$ balls are red, $2x$ are white and $3x$ are blue. A ball is selected at random, What is the probability that it is
\begin{enumerate}[label=\alph*)]
\item not red ?
\item white ?
\end{enumerate}
%\input{exemplar/10/13/3/41/main.tex}
If the letters of the word ASSASSINATION are arranged at random. Find the Probability that
\begin{enumerate}[label=(\alph*)]
\item Four $S's$ come consecutively in the word
\item Two  $I's$ and two $N's$ come together
\item All $A's$ are not coming together
\item No two $A's$ are coming together
\end{enumerate}
%\input{exemplar/11/16/3/14/main.tex}
	\item One urn contains two black balls (labelled B1 and B2) and one white ball. A
	second urn contains one black ball and two white balls (labelled W1 and W2).
	Suppose the following experiment is performed. One of the two urns is chosen
	at random. Next a ball is randomly chosen from the urn. Then a second ball is
	chosen at random from the same urn without replacing the first ball.
	
	\begin{enumerate}
	\item What is the probability that two black balls are chosen?
	
	\item What is the probability that two balls of opposite colour are chosen?
	\end{enumerate}
	\solution
	%\input{exemplar/11/16/3/12/main1.tex}
\end{enumerate}

		\item A box of oranges is inspected by examining three randomly selected oranges drawn without replacement. If all the three oranges are good, the box is approved for sale, otherwise, it is rejected. Find the probability that a box containing 15 oranges out of which 12 are good and 3 are bad ones will be approved for sale.
		\label{ncert/12/13/2/3/defs.tex}
		\item Two balls are drawn at random with replacement from a box containing 10 black and 8 red balls. Find the probability that
		\label{ncert/12/13/2/12}
\begin{enumerate}
\item both balls are red.
\item first ball is black and second is red.
\item one of them is black and other is red.
\end{enumerate}

\item In a hostel, 60\% of the students read Hindi newspaper, 40\% read English newspaper and 20\% read both Hindi and English newspapers. A student is selected at random.
		\label{ncert/12/13/2/15}
\begin{enumerate}
\item Find the probability that she reads neither Hindi nor English newspapers.
\item If she reads Hindi newspaper, find the probability that she reads English newspaper.
\item If she reads English newspaper, find the probability that she reads Hindi newspaper.\\
\end{enumerate}
\item The probability of obtaining an even prime number on each die, when a pair of dice is rolled is 
\begin{enumerate}
    \item $0$ 
    
    \item $\frac{1}{3}$ 
    
    \item $\frac{1}{12}$ 
    
    \item $\frac{1}{36}$ 
\end{enumerate}
\solution
		%\begin{enumerate}[label=\thesection.\arabic*,ref=\thesection.\theenumi]
	\item One card is drawn from a well-shuffled deck of 52 cards. Find the probability of getting
\begin{enumerate}
\item A king of red colour 
\item A face card 
\item A red face card
\item The jack of hearts
\item A spade
\item The queen of diamonds

\end{enumerate}
\solution
		%\input{ncert/10/15/1/14/main.tex}
	\item Five cards—the ten, jack, queen, king and ace of diamonds, are well-shuffled with their face downwards. One card is then picked up at random.
\begin{enumerate}
\item
What is the probability that the card is the queen? 
\item
If the queen is drawn and put aside, what is the probability that the second card picked up is (a) an ace? (b) a queen?\\
\end{enumerate}
\solution
		%\input{ncert/10/15/1/15/defs.tex}
	\item A bag contains $5$ red balls and some blue balls. If the probability of drawing a blue ball is double that if a red ball, determine the number of blue balls in the bag. 
		\\
\solution
		%\input{ncert/10/15/2/3/defs.tex}
	\item A card is selected from a pack of 52 cards.
 \begin{enumerate}[label=(\alph*)] 
                 \item How many points are there in the sample space?
                 \item Calculate the probability that the card is an ace of spades.
                 \item Calculate the probability that the card is (i) an ace and (ii) black card.
 \end{enumerate}
\solution
		%\input{ncert/11/16/3/4/main.tex}
\item Four cards are drawn from a well-shuffled deck of 52 cards. What is the probability of obtaining 3 diamonds and one spade.
\\
\solution
		%\input{ncert/11/16/4/2/defs.tex}
\item In a certain lottery 10,000 tickets are sold and ten equal prizes are awarded. What is the probability of not getting a prize if you buy (a) one ticket (b) two tickets (c) 10 tickets ?	
\\
\solution
		%\input{ncert/11/16/4/4/defs.tex}
		%
\item 
Out of 100 students, two sections of 40 and 60 are formed. If you and your friend are among the 100 students, what is the probability that
\begin{enumerate}
\item you both enter the same section?
\item you both enter the different sections?
\end{enumerate}
\solution
		%\input{ncert/11/16/4/5/defs.tex}
	\item 
The number lock of a suitcase has 4 wheels each labelled with ten digits i.e. from 0 to 9.The lock opens with a sequence of four digits with no repeats.What is the probability of a person getting the right sequence to open the suitcase.
\\
\solution
		%\input{ncert/11/16/4/10/defs.tex}
		%
\item 
Two cards are drawn at random and without replacement from a pack of 52 playing cards. Find the probability that both the cards are black.
\\
\solution
		%\input{ncert/12/13/2/2/defs.tex}
		\item A box of oranges is inspected by examining three randomly selected oranges drawn without replacement. If all the three oranges are good, the box is approved for sale, otherwise, it is rejected. Find the probability that a box containing 15 oranges out of which 12 are good and 3 are bad ones will be approved for sale.
		\label{ncert/12/13/2/3/defs.tex}
		\item Two balls are drawn at random with replacement from a box containing 10 black and 8 red balls. Find the probability that
		\label{ncert/12/13/2/12}
\begin{enumerate}
\item both balls are red.
\item first ball is black and second is red.
\item one of them is black and other is red.
\end{enumerate}

\item In a hostel, 60\% of the students read Hindi newspaper, 40\% read English newspaper and 20\% read both Hindi and English newspapers. A student is selected at random.
		\label{ncert/12/13/2/15}
\begin{enumerate}
\item Find the probability that she reads neither Hindi nor English newspapers.
\item If she reads Hindi newspaper, find the probability that she reads English newspaper.
\item If she reads English newspaper, find the probability that she reads Hindi newspaper.\\
\end{enumerate}
\item The probability of obtaining an even prime number on each die, when a pair of dice is rolled is 
\begin{enumerate}
    \item $0$ 
    
    \item $\frac{1}{3}$ 
    
    \item $\frac{1}{12}$ 
    
    \item $\frac{1}{36}$ 
\end{enumerate}
\solution
		%\input{ncert/12/13/2/17/defs.tex}
	\item A bag contains 4 red and 4 black balls, another bag contains 2 red and 6 black balls. One of the two bags is selected at random and a ball is drawn from the bag which is found to be red. Find the probability that the ball is drawn from the first bag.
\\
\solution
		%\input{ncert/12/13/3/2/main.tex}
  \item
  Cards with numbers 2 to 101 are placed in a box. A card is selected at random.Find the probability that the card has
\begin{enumerate}[label=(\roman*)]
	\item an even number 
	\item a square number
\end{enumerate}
\solution
%\input{exemplar/10/13/3/32/main.tex}
\item
The king, queen and jack of clubs are removed from a deck of 52 playing cards and then well shuffled. Now one card is drawn at random from the remaining cards.  Determine the probability that the card is
\begin{enumerate}[label=(\roman*)]
\item a club
\item 10 of hearts
\end{enumerate}
\solution
%\input{exemplar/10/13/3/29/main.tex}
\item A team of medical students doing their internship have to assist during surgeries
at a city hospital. The probabilities of surgeries rated as very complex, complex,
routine, simple or very simple are respectively, 0.15, 0.20, 0.31, 0.26, .08. Find
the probabilities that a particular surgery will be rated
\begin{enumerate}
	\item complex or very complex;
	\item neither very complex nor very simple;
	\item routine or complex
	\item routine or simple
\end{enumerate}
\solution
%\input{exemplar/11/16/3/8(1)/main.tex}
\item A card is selected from a pack of 52 cards.
\begin{enumerate}[label=(\alph*)]
    \item How many points are there in the sample space?
    \item Calculate the probability that the card is an ace of spades.
    \item Calculate the probability that the card is (i) an ace and (ii) black card.
\end{enumerate}
\solution
%\input{exemplar/11/16/3/4/main2.tex}
\item The probability that a non leap year selected at random will contain 53 sundays.
\\
\solution
%\input{exemplar/10/13/1/19/main.tex}
\item One of the four persons John, Rita, Aslam or Gurpreet will be promoted next
month. Consequently the sample space consists of four elementary outcomes
S = {John promoted, Rita promoted, Aslam promoted, Gurpreet promoted}
You are told that the chances of John’s promotion is same as that of Gurpreet,
Rita’s chances of promotion are twice as likely as Johns. Aslam’s chances are
four times that of John.
\begin{enumerate}
	\item Determine
	\begin{enumerate}
		\item P (John promoted)
		\item P (Rita promoted)
		\item P (Aslam promoted)
		\item P (Gurpreet promoted)
	\end{enumerate}
	\item If A = {John promoted or Gurpreet promoted}, find P (A).
\end{enumerate}
\solution
%\input{exemplar/11/16/3/10/main.tex}
\item A card is drawn from a deck of 52 cards. Find the probability of getting a king or a heart or a red card.\\
\solution
%\input{exemplar/11/16/3/15/main.tex}
\item The probability that a student will pass his examination is 0.73, the probability of
the student getting a compartment is 0.13, and the probability that the student will
either pass or get compartment is 0.96. State True or False.\\
\solution
%\input{exemplar/11/16/3/31/main.tex}
\item A card is selected from a pack of 52 cards\\
\begin{enumerate}[label=(\alph*)]
\item How many points are there in the sample space?
\item Calculate the probability that the cards is an ace of spades.
\item Calculate the probability that the card is (i) an ace (ii)black card.\\
\end{enumerate}
%\input{ncert/11/16/3/4_1/Prob_4.tex}
\item In a non-leap year, the probability of having 53 tuesdays or 53 wednesdays is\\
\solution
%\input{exemplar/11/16/3/18/main.tex}
\item There are 1000 sealed envelopes in a box, 10 of them contain a cash prize of
Rs 100 each, 100 of them contain a cash prize of Rs 50 each and 200 of them
contain a cash prize of Rs 10 each and rest do not contain any cash prize. If they
are well shuffled and an envelope is picked up out, what is the probability that it
contains no cash prize?\\
\solution
%\input{exemplar/10/13/3/34/main.tex}
\item 
A die is thrown and a card is selected at random from a deck of 52 playing cards. The probability of getting an even number on the die and a spade card.\\
\solution
%\input{exemplar/12/13/3/78/main.tex}
\item
If 4-digit numbers greater than 5,000 are randomly formed from the digits 0, 1, 3, 5, and 7, what is the probability of forming a number divisible by 5 when:
\begin{enumerate}
    \item The digits are repeated?
    \item The repetition of digits is not allowed?
\end{enumerate}
\solution
%\input{ncert/11/16/4/9/main.tex}
\item Consider the probability space $\brak{\Omega, \mathcal{G}, P}$ where $\Omega = [0,2]$ and $\mathcal{G} = \cbrak{\phi, \Omega, [0,1], (1,2]}$. Let $X$ and $Y$ be two functions on $\Omega$ defined as
\begin{align*}
    X(\omega) = 
    \begin{cases}
        1 & \text{if }\omega \in [0, 1]\\
        2 & \text{if }\omega \in (1, 2]
    \end{cases}
\end{align*}
and
\begin{align*}
    Y(\omega) = 
    \begin{cases}
        2 & \text{if }\omega \in [0, 1.5]\\
        3 & \text{if }\omega \in (1.5, 2].
    \end{cases}
\end{align*}
Then which one of the following statements is true?
\begin{enumerate}
    \item [(A)] $X$ is a random variable with respect to $\mathcal{G}$, but $Y$ is not a random variable with respect to $\mathcal{G}$.
    \item [(B)] $Y$ is a random variable with respect to $\mathcal{G}$, but $X$ is not a random variable with respect to $\mathcal{G}$.
    \item [(C)] Neither $X$ nor $Y$ is a random variable with respect to $\mathcal{G}$.
    \item [(D)] Both $X$ and $Y$ are random variables with respect to $\mathcal{G}$.
\end{enumerate} \hfill (GATE ST 2023)\\
\solution
%\input{gate/ST/2023/14/main.tex}
	\item  A die is loaded in such a way that each odd number is twice as likely to occur as
each even number. Find $P(G)$, where $G$ is the event that a number greater than
3 occurs on a single roll of the die.
\\
\solution
		%\input{exemplar/11/16/3/5/main.tex}
	\item All the jacks, queens and kings are removed from a deck of 52 playing cards. The remaining cards are well shuffled and then one card is drawn at random. Giving ace a value 1 similar value for other cards, find the probability that the card has a value 
		\begin{enumerate}
			\item 7
			\item greater than 7
			\item less than 7
		\end{enumerate}
		%\input{exemplar/10/13/3/30/main.tex}
  \item A Lot consists of 48 mobile phones of which 42 are good, 3 have only minor defects and 3 have major defects.Varnika will buy a phone if it is good but the trader will only buy a mobile if it has no major defects. One phone is selected at random from the lot. What is the probability that it is
\begin{enumerate}
	\item acceptable to Varnika?
            \item acceptable to the trader?
\end{enumerate}
\solution
	%\input{exemplar/10/13/3/40/main.tex}
 \item A student says that if you throw a die, it will show up 1 or not 1. Therefore, the probability of getting 1 and the probability of getting 'not 1' each is equal to $\frac{1}{2}$. Is this correct? Give reasons.\\
 \solution
        %\input{exemplar/10/13/2/9/main.tex}
   \item Four candidates A, B, C, D have ap-
plied for the assignment to coach a school cricket
team. If A is twice as likely to be selected as B, and
B and C are given about the same chance of being
selected, while C is twice as likely to be selected
as D, what are the probabilities that
\begin{enumerate}
\item C will be selected?
\item A will not be selected?
\end{enumerate}
	%\input{exemplar/11/16/3/9/main.tex}
 \item A bag contain 24 balls of which $x$ balls are red, $2x$ are white and $3x$ are blue. A ball is selected at random, What is the probability that it is
\begin{enumerate}[label=\alph*)]
\item not red ?
\item white ?
\end{enumerate}
%\input{exemplar/10/13/3/41/main.tex}
If the letters of the word ASSASSINATION are arranged at random. Find the Probability that
\begin{enumerate}[label=(\alph*)]
\item Four $S's$ come consecutively in the word
\item Two  $I's$ and two $N's$ come together
\item All $A's$ are not coming together
\item No two $A's$ are coming together
\end{enumerate}
%\input{exemplar/11/16/3/14/main.tex}
	\item One urn contains two black balls (labelled B1 and B2) and one white ball. A
	second urn contains one black ball and two white balls (labelled W1 and W2).
	Suppose the following experiment is performed. One of the two urns is chosen
	at random. Next a ball is randomly chosen from the urn. Then a second ball is
	chosen at random from the same urn without replacing the first ball.
	
	\begin{enumerate}
	\item What is the probability that two black balls are chosen?
	
	\item What is the probability that two balls of opposite colour are chosen?
	\end{enumerate}
	\solution
	%\input{exemplar/11/16/3/12/main1.tex}
\end{enumerate}

	\item A bag contains 4 red and 4 black balls, another bag contains 2 red and 6 black balls. One of the two bags is selected at random and a ball is drawn from the bag which is found to be red. Find the probability that the ball is drawn from the first bag.
\\
\solution
		%\begin{table}[H]
	\centering
\begin{tabular}{|c|c|c|}
\hline
Random variable &Value &Definition\\ \hline
\multirow{3}{*}{X} &0 &Slips of Rs 1\\
&1 &Slips of Rs 5\\
&2 &Slips of Rs 13\\ \hline
\multirow{2}{*}{Y} &0 &Box A\\
&1 &Box B\\\hline
\end{tabular}
\caption{}
\label{tab:Distribution}
\end{table}
See \tabref{tab:Distribution}.
\begin{align}
p_{Y}\brak{k}= \begin{cases} 
      \frac{1}{3} & {k=0} \\
      \frac{2}{3 }& {k=1} 
   \end{cases}
   \\
p_{Y|X}\brak{0|0} = \frac{19}{25}\, 
p_{Y|X}\brak{0|1} = \frac{6}{25}\,
p_{Y|X}\brak{1|0} = \frac{45}{50}\,
p_{Y|X}\brak{1|2} = \frac{5}{50}
\end{align}
The desired probability is the probability that a slip drawn at random is marked other than Rs 1,
\begin{align}
&=1-p_X\brak{0}\\
&= p_X(1) + p_X(2)
\end{align}
Using Bayes theorem,
\begin{align}
&= p_Y\brak{0} \times \pr{Y=0 | X=1} + p_Y\brak{1} \times \pr{Y=1|X=2}\\
&=\frac{1}{3} \times \frac{6}{25} + \frac{2}{3} \times \frac{5}{50}\\
&=\frac{11}{75}
\end{align}

\newpage

%\tableofcontents

\bigskip

\renewcommand{\thefigure}{\theenumi}
\renewcommand{\thetable}{\theenumi}
%\renewcommand{\theequation}{\theenumi}

%\begin{abstract}
%%\boldmath
%In this letter, an algorithm for evaluating the exact analytical bit error rate  (BER)  for the piecewise linear (PL) combiner for  multiple relays is presented. Previous results were available only for upto three relays. The algorithm is unique in the sense that  the actual mathematical expressions, that are prohibitively large, need not be explicitly obtained. The diversity gain due to multiple relays is shown through plots of the analytical BER, well supported by simulations. 
%
%\end{abstract}
% IEEEtran.cls defaults to using nonbold math in the Abstract.
% This preserves the distinction between vectors and scalars. However,
% if the journal you are submitting to favors bold math in the abstract,
% then you can use LaTeX's standard command \boldmath at the very start
% of the abstract to achieve this. Many IEEE journals frown on math
% in the abstract anyway.

% Note that keywords are not normally used for peerreview papers.
%\begin{IEEEkeywords}
%Cooperative diversity, decode and forward, piecewise linear
%\end{IEEEkeywords}



% For peer review papers, you can put extra information on the cover
% page as needed:
% \ifCLASSOPTIONpeerreview
% \begin{center} \bfseries EDICS Category: 3-BBND \end{center}
% \fi
%
% For peerreview papers, this IEEEtran command inserts a page break and
% creates the second title. It will be ignored for other modes.
%\IEEEpeerreviewmaketitle




  \item
  Cards with numbers 2 to 101 are placed in a box. A card is selected at random.Find the probability that the card has
\begin{enumerate}[label=(\roman*)]
	\item an even number 
	\item a square number
\end{enumerate}
\solution
%\begin{table}[H]
	\centering
\begin{tabular}{|c|c|c|}
\hline
Random variable &Value &Definition\\ \hline
\multirow{3}{*}{X} &0 &Slips of Rs 1\\
&1 &Slips of Rs 5\\
&2 &Slips of Rs 13\\ \hline
\multirow{2}{*}{Y} &0 &Box A\\
&1 &Box B\\\hline
\end{tabular}
\caption{}
\label{tab:Distribution}
\end{table}
See \tabref{tab:Distribution}.
\begin{align}
p_{Y}\brak{k}= \begin{cases} 
      \frac{1}{3} & {k=0} \\
      \frac{2}{3 }& {k=1} 
   \end{cases}
   \\
p_{Y|X}\brak{0|0} = \frac{19}{25}\, 
p_{Y|X}\brak{0|1} = \frac{6}{25}\,
p_{Y|X}\brak{1|0} = \frac{45}{50}\,
p_{Y|X}\brak{1|2} = \frac{5}{50}
\end{align}
The desired probability is the probability that a slip drawn at random is marked other than Rs 1,
\begin{align}
&=1-p_X\brak{0}\\
&= p_X(1) + p_X(2)
\end{align}
Using Bayes theorem,
\begin{align}
&= p_Y\brak{0} \times \pr{Y=0 | X=1} + p_Y\brak{1} \times \pr{Y=1|X=2}\\
&=\frac{1}{3} \times \frac{6}{25} + \frac{2}{3} \times \frac{5}{50}\\
&=\frac{11}{75}
\end{align}

\newpage

%\tableofcontents

\bigskip

\renewcommand{\thefigure}{\theenumi}
\renewcommand{\thetable}{\theenumi}
%\renewcommand{\theequation}{\theenumi}

%\begin{abstract}
%%\boldmath
%In this letter, an algorithm for evaluating the exact analytical bit error rate  (BER)  for the piecewise linear (PL) combiner for  multiple relays is presented. Previous results were available only for upto three relays. The algorithm is unique in the sense that  the actual mathematical expressions, that are prohibitively large, need not be explicitly obtained. The diversity gain due to multiple relays is shown through plots of the analytical BER, well supported by simulations. 
%
%\end{abstract}
% IEEEtran.cls defaults to using nonbold math in the Abstract.
% This preserves the distinction between vectors and scalars. However,
% if the journal you are submitting to favors bold math in the abstract,
% then you can use LaTeX's standard command \boldmath at the very start
% of the abstract to achieve this. Many IEEE journals frown on math
% in the abstract anyway.

% Note that keywords are not normally used for peerreview papers.
%\begin{IEEEkeywords}
%Cooperative diversity, decode and forward, piecewise linear
%\end{IEEEkeywords}



% For peer review papers, you can put extra information on the cover
% page as needed:
% \ifCLASSOPTIONpeerreview
% \begin{center} \bfseries EDICS Category: 3-BBND \end{center}
% \fi
%
% For peerreview papers, this IEEEtran command inserts a page break and
% creates the second title. It will be ignored for other modes.
%\IEEEpeerreviewmaketitle




\item
The king, queen and jack of clubs are removed from a deck of 52 playing cards and then well shuffled. Now one card is drawn at random from the remaining cards.  Determine the probability that the card is
\begin{enumerate}[label=(\roman*)]
\item a club
\item 10 of hearts
\end{enumerate}
\solution
%\begin{table}[H]
	\centering
\begin{tabular}{|c|c|c|}
\hline
Random variable &Value &Definition\\ \hline
\multirow{3}{*}{X} &0 &Slips of Rs 1\\
&1 &Slips of Rs 5\\
&2 &Slips of Rs 13\\ \hline
\multirow{2}{*}{Y} &0 &Box A\\
&1 &Box B\\\hline
\end{tabular}
\caption{}
\label{tab:Distribution}
\end{table}
See \tabref{tab:Distribution}.
\begin{align}
p_{Y}\brak{k}= \begin{cases} 
      \frac{1}{3} & {k=0} \\
      \frac{2}{3 }& {k=1} 
   \end{cases}
   \\
p_{Y|X}\brak{0|0} = \frac{19}{25}\, 
p_{Y|X}\brak{0|1} = \frac{6}{25}\,
p_{Y|X}\brak{1|0} = \frac{45}{50}\,
p_{Y|X}\brak{1|2} = \frac{5}{50}
\end{align}
The desired probability is the probability that a slip drawn at random is marked other than Rs 1,
\begin{align}
&=1-p_X\brak{0}\\
&= p_X(1) + p_X(2)
\end{align}
Using Bayes theorem,
\begin{align}
&= p_Y\brak{0} \times \pr{Y=0 | X=1} + p_Y\brak{1} \times \pr{Y=1|X=2}\\
&=\frac{1}{3} \times \frac{6}{25} + \frac{2}{3} \times \frac{5}{50}\\
&=\frac{11}{75}
\end{align}

\newpage

%\tableofcontents

\bigskip

\renewcommand{\thefigure}{\theenumi}
\renewcommand{\thetable}{\theenumi}
%\renewcommand{\theequation}{\theenumi}

%\begin{abstract}
%%\boldmath
%In this letter, an algorithm for evaluating the exact analytical bit error rate  (BER)  for the piecewise linear (PL) combiner for  multiple relays is presented. Previous results were available only for upto three relays. The algorithm is unique in the sense that  the actual mathematical expressions, that are prohibitively large, need not be explicitly obtained. The diversity gain due to multiple relays is shown through plots of the analytical BER, well supported by simulations. 
%
%\end{abstract}
% IEEEtran.cls defaults to using nonbold math in the Abstract.
% This preserves the distinction between vectors and scalars. However,
% if the journal you are submitting to favors bold math in the abstract,
% then you can use LaTeX's standard command \boldmath at the very start
% of the abstract to achieve this. Many IEEE journals frown on math
% in the abstract anyway.

% Note that keywords are not normally used for peerreview papers.
%\begin{IEEEkeywords}
%Cooperative diversity, decode and forward, piecewise linear
%\end{IEEEkeywords}



% For peer review papers, you can put extra information on the cover
% page as needed:
% \ifCLASSOPTIONpeerreview
% \begin{center} \bfseries EDICS Category: 3-BBND \end{center}
% \fi
%
% For peerreview papers, this IEEEtran command inserts a page break and
% creates the second title. It will be ignored for other modes.
%\IEEEpeerreviewmaketitle




\item A team of medical students doing their internship have to assist during surgeries
at a city hospital. The probabilities of surgeries rated as very complex, complex,
routine, simple or very simple are respectively, 0.15, 0.20, 0.31, 0.26, .08. Find
the probabilities that a particular surgery will be rated
\begin{enumerate}
	\item complex or very complex;
	\item neither very complex nor very simple;
	\item routine or complex
	\item routine or simple
\end{enumerate}
\solution
%\begin{table}[H]
	\centering
\begin{tabular}{|c|c|c|}
\hline
Random variable &Value &Definition\\ \hline
\multirow{3}{*}{X} &0 &Slips of Rs 1\\
&1 &Slips of Rs 5\\
&2 &Slips of Rs 13\\ \hline
\multirow{2}{*}{Y} &0 &Box A\\
&1 &Box B\\\hline
\end{tabular}
\caption{}
\label{tab:Distribution}
\end{table}
See \tabref{tab:Distribution}.
\begin{align}
p_{Y}\brak{k}= \begin{cases} 
      \frac{1}{3} & {k=0} \\
      \frac{2}{3 }& {k=1} 
   \end{cases}
   \\
p_{Y|X}\brak{0|0} = \frac{19}{25}\, 
p_{Y|X}\brak{0|1} = \frac{6}{25}\,
p_{Y|X}\brak{1|0} = \frac{45}{50}\,
p_{Y|X}\brak{1|2} = \frac{5}{50}
\end{align}
The desired probability is the probability that a slip drawn at random is marked other than Rs 1,
\begin{align}
&=1-p_X\brak{0}\\
&= p_X(1) + p_X(2)
\end{align}
Using Bayes theorem,
\begin{align}
&= p_Y\brak{0} \times \pr{Y=0 | X=1} + p_Y\brak{1} \times \pr{Y=1|X=2}\\
&=\frac{1}{3} \times \frac{6}{25} + \frac{2}{3} \times \frac{5}{50}\\
&=\frac{11}{75}
\end{align}

\newpage

%\tableofcontents

\bigskip

\renewcommand{\thefigure}{\theenumi}
\renewcommand{\thetable}{\theenumi}
%\renewcommand{\theequation}{\theenumi}

%\begin{abstract}
%%\boldmath
%In this letter, an algorithm for evaluating the exact analytical bit error rate  (BER)  for the piecewise linear (PL) combiner for  multiple relays is presented. Previous results were available only for upto three relays. The algorithm is unique in the sense that  the actual mathematical expressions, that are prohibitively large, need not be explicitly obtained. The diversity gain due to multiple relays is shown through plots of the analytical BER, well supported by simulations. 
%
%\end{abstract}
% IEEEtran.cls defaults to using nonbold math in the Abstract.
% This preserves the distinction between vectors and scalars. However,
% if the journal you are submitting to favors bold math in the abstract,
% then you can use LaTeX's standard command \boldmath at the very start
% of the abstract to achieve this. Many IEEE journals frown on math
% in the abstract anyway.

% Note that keywords are not normally used for peerreview papers.
%\begin{IEEEkeywords}
%Cooperative diversity, decode and forward, piecewise linear
%\end{IEEEkeywords}



% For peer review papers, you can put extra information on the cover
% page as needed:
% \ifCLASSOPTIONpeerreview
% \begin{center} \bfseries EDICS Category: 3-BBND \end{center}
% \fi
%
% For peerreview papers, this IEEEtran command inserts a page break and
% creates the second title. It will be ignored for other modes.
%\IEEEpeerreviewmaketitle




\item A card is selected from a pack of 52 cards.
\begin{enumerate}[label=(\alph*)]
    \item How many points are there in the sample space?
    \item Calculate the probability that the card is an ace of spades.
    \item Calculate the probability that the card is (i) an ace and (ii) black card.
\end{enumerate}
\solution
%Let $X$ be an bernoulli rv defined as in \tabref{tab:exemplar/11/16/3/26}.  Then, 
\begin{equation}
    p =
        \frac{4}{11} 
\end{equation}
\begin{table}[H]
	\centering
	\input{exemplar/11/16/3/26/tables/Table2.tex}
	\caption{}
        \label{tab:exemplar/11/16/3/26}
\end{table}

\item The probability that a non leap year selected at random will contain 53 sundays.
\\
\solution
%\begin{table}[H]
	\centering
\begin{tabular}{|c|c|c|}
\hline
Random variable &Value &Definition\\ \hline
\multirow{3}{*}{X} &0 &Slips of Rs 1\\
&1 &Slips of Rs 5\\
&2 &Slips of Rs 13\\ \hline
\multirow{2}{*}{Y} &0 &Box A\\
&1 &Box B\\\hline
\end{tabular}
\caption{}
\label{tab:Distribution}
\end{table}
See \tabref{tab:Distribution}.
\begin{align}
p_{Y}\brak{k}= \begin{cases} 
      \frac{1}{3} & {k=0} \\
      \frac{2}{3 }& {k=1} 
   \end{cases}
   \\
p_{Y|X}\brak{0|0} = \frac{19}{25}\, 
p_{Y|X}\brak{0|1} = \frac{6}{25}\,
p_{Y|X}\brak{1|0} = \frac{45}{50}\,
p_{Y|X}\brak{1|2} = \frac{5}{50}
\end{align}
The desired probability is the probability that a slip drawn at random is marked other than Rs 1,
\begin{align}
&=1-p_X\brak{0}\\
&= p_X(1) + p_X(2)
\end{align}
Using Bayes theorem,
\begin{align}
&= p_Y\brak{0} \times \pr{Y=0 | X=1} + p_Y\brak{1} \times \pr{Y=1|X=2}\\
&=\frac{1}{3} \times \frac{6}{25} + \frac{2}{3} \times \frac{5}{50}\\
&=\frac{11}{75}
\end{align}

\newpage

%\tableofcontents

\bigskip

\renewcommand{\thefigure}{\theenumi}
\renewcommand{\thetable}{\theenumi}
%\renewcommand{\theequation}{\theenumi}

%\begin{abstract}
%%\boldmath
%In this letter, an algorithm for evaluating the exact analytical bit error rate  (BER)  for the piecewise linear (PL) combiner for  multiple relays is presented. Previous results were available only for upto three relays. The algorithm is unique in the sense that  the actual mathematical expressions, that are prohibitively large, need not be explicitly obtained. The diversity gain due to multiple relays is shown through plots of the analytical BER, well supported by simulations. 
%
%\end{abstract}
% IEEEtran.cls defaults to using nonbold math in the Abstract.
% This preserves the distinction between vectors and scalars. However,
% if the journal you are submitting to favors bold math in the abstract,
% then you can use LaTeX's standard command \boldmath at the very start
% of the abstract to achieve this. Many IEEE journals frown on math
% in the abstract anyway.

% Note that keywords are not normally used for peerreview papers.
%\begin{IEEEkeywords}
%Cooperative diversity, decode and forward, piecewise linear
%\end{IEEEkeywords}



% For peer review papers, you can put extra information on the cover
% page as needed:
% \ifCLASSOPTIONpeerreview
% \begin{center} \bfseries EDICS Category: 3-BBND \end{center}
% \fi
%
% For peerreview papers, this IEEEtran command inserts a page break and
% creates the second title. It will be ignored for other modes.
%\IEEEpeerreviewmaketitle




\item One of the four persons John, Rita, Aslam or Gurpreet will be promoted next
month. Consequently the sample space consists of four elementary outcomes
S = {John promoted, Rita promoted, Aslam promoted, Gurpreet promoted}
You are told that the chances of John’s promotion is same as that of Gurpreet,
Rita’s chances of promotion are twice as likely as Johns. Aslam’s chances are
four times that of John.
\begin{enumerate}
	\item Determine
	\begin{enumerate}
		\item P (John promoted)
		\item P (Rita promoted)
		\item P (Aslam promoted)
		\item P (Gurpreet promoted)
	\end{enumerate}
	\item If A = {John promoted or Gurpreet promoted}, find P (A).
\end{enumerate}
\solution
%\begin{table}[H]
	\centering
\begin{tabular}{|c|c|c|}
\hline
Random variable &Value &Definition\\ \hline
\multirow{3}{*}{X} &0 &Slips of Rs 1\\
&1 &Slips of Rs 5\\
&2 &Slips of Rs 13\\ \hline
\multirow{2}{*}{Y} &0 &Box A\\
&1 &Box B\\\hline
\end{tabular}
\caption{}
\label{tab:Distribution}
\end{table}
See \tabref{tab:Distribution}.
\begin{align}
p_{Y}\brak{k}= \begin{cases} 
      \frac{1}{3} & {k=0} \\
      \frac{2}{3 }& {k=1} 
   \end{cases}
   \\
p_{Y|X}\brak{0|0} = \frac{19}{25}\, 
p_{Y|X}\brak{0|1} = \frac{6}{25}\,
p_{Y|X}\brak{1|0} = \frac{45}{50}\,
p_{Y|X}\brak{1|2} = \frac{5}{50}
\end{align}
The desired probability is the probability that a slip drawn at random is marked other than Rs 1,
\begin{align}
&=1-p_X\brak{0}\\
&= p_X(1) + p_X(2)
\end{align}
Using Bayes theorem,
\begin{align}
&= p_Y\brak{0} \times \pr{Y=0 | X=1} + p_Y\brak{1} \times \pr{Y=1|X=2}\\
&=\frac{1}{3} \times \frac{6}{25} + \frac{2}{3} \times \frac{5}{50}\\
&=\frac{11}{75}
\end{align}

\newpage

%\tableofcontents

\bigskip

\renewcommand{\thefigure}{\theenumi}
\renewcommand{\thetable}{\theenumi}
%\renewcommand{\theequation}{\theenumi}

%\begin{abstract}
%%\boldmath
%In this letter, an algorithm for evaluating the exact analytical bit error rate  (BER)  for the piecewise linear (PL) combiner for  multiple relays is presented. Previous results were available only for upto three relays. The algorithm is unique in the sense that  the actual mathematical expressions, that are prohibitively large, need not be explicitly obtained. The diversity gain due to multiple relays is shown through plots of the analytical BER, well supported by simulations. 
%
%\end{abstract}
% IEEEtran.cls defaults to using nonbold math in the Abstract.
% This preserves the distinction between vectors and scalars. However,
% if the journal you are submitting to favors bold math in the abstract,
% then you can use LaTeX's standard command \boldmath at the very start
% of the abstract to achieve this. Many IEEE journals frown on math
% in the abstract anyway.

% Note that keywords are not normally used for peerreview papers.
%\begin{IEEEkeywords}
%Cooperative diversity, decode and forward, piecewise linear
%\end{IEEEkeywords}



% For peer review papers, you can put extra information on the cover
% page as needed:
% \ifCLASSOPTIONpeerreview
% \begin{center} \bfseries EDICS Category: 3-BBND \end{center}
% \fi
%
% For peerreview papers, this IEEEtran command inserts a page break and
% creates the second title. It will be ignored for other modes.
%\IEEEpeerreviewmaketitle




\item A card is drawn from a deck of 52 cards. Find the probability of getting a king or a heart or a red card.\\
\solution
%\begin{table}[H]
	\centering
\begin{tabular}{|c|c|c|}
\hline
Random variable &Value &Definition\\ \hline
\multirow{3}{*}{X} &0 &Slips of Rs 1\\
&1 &Slips of Rs 5\\
&2 &Slips of Rs 13\\ \hline
\multirow{2}{*}{Y} &0 &Box A\\
&1 &Box B\\\hline
\end{tabular}
\caption{}
\label{tab:Distribution}
\end{table}
See \tabref{tab:Distribution}.
\begin{align}
p_{Y}\brak{k}= \begin{cases} 
      \frac{1}{3} & {k=0} \\
      \frac{2}{3 }& {k=1} 
   \end{cases}
   \\
p_{Y|X}\brak{0|0} = \frac{19}{25}\, 
p_{Y|X}\brak{0|1} = \frac{6}{25}\,
p_{Y|X}\brak{1|0} = \frac{45}{50}\,
p_{Y|X}\brak{1|2} = \frac{5}{50}
\end{align}
The desired probability is the probability that a slip drawn at random is marked other than Rs 1,
\begin{align}
&=1-p_X\brak{0}\\
&= p_X(1) + p_X(2)
\end{align}
Using Bayes theorem,
\begin{align}
&= p_Y\brak{0} \times \pr{Y=0 | X=1} + p_Y\brak{1} \times \pr{Y=1|X=2}\\
&=\frac{1}{3} \times \frac{6}{25} + \frac{2}{3} \times \frac{5}{50}\\
&=\frac{11}{75}
\end{align}

\newpage

%\tableofcontents

\bigskip

\renewcommand{\thefigure}{\theenumi}
\renewcommand{\thetable}{\theenumi}
%\renewcommand{\theequation}{\theenumi}

%\begin{abstract}
%%\boldmath
%In this letter, an algorithm for evaluating the exact analytical bit error rate  (BER)  for the piecewise linear (PL) combiner for  multiple relays is presented. Previous results were available only for upto three relays. The algorithm is unique in the sense that  the actual mathematical expressions, that are prohibitively large, need not be explicitly obtained. The diversity gain due to multiple relays is shown through plots of the analytical BER, well supported by simulations. 
%
%\end{abstract}
% IEEEtran.cls defaults to using nonbold math in the Abstract.
% This preserves the distinction between vectors and scalars. However,
% if the journal you are submitting to favors bold math in the abstract,
% then you can use LaTeX's standard command \boldmath at the very start
% of the abstract to achieve this. Many IEEE journals frown on math
% in the abstract anyway.

% Note that keywords are not normally used for peerreview papers.
%\begin{IEEEkeywords}
%Cooperative diversity, decode and forward, piecewise linear
%\end{IEEEkeywords}



% For peer review papers, you can put extra information on the cover
% page as needed:
% \ifCLASSOPTIONpeerreview
% \begin{center} \bfseries EDICS Category: 3-BBND \end{center}
% \fi
%
% For peerreview papers, this IEEEtran command inserts a page break and
% creates the second title. It will be ignored for other modes.
%\IEEEpeerreviewmaketitle




\item The probability that a student will pass his examination is 0.73, the probability of
the student getting a compartment is 0.13, and the probability that the student will
either pass or get compartment is 0.96. State True or False.\\
\solution
%\begin{table}[H]
	\centering
\begin{tabular}{|c|c|c|}
\hline
Random variable &Value &Definition\\ \hline
\multirow{3}{*}{X} &0 &Slips of Rs 1\\
&1 &Slips of Rs 5\\
&2 &Slips of Rs 13\\ \hline
\multirow{2}{*}{Y} &0 &Box A\\
&1 &Box B\\\hline
\end{tabular}
\caption{}
\label{tab:Distribution}
\end{table}
See \tabref{tab:Distribution}.
\begin{align}
p_{Y}\brak{k}= \begin{cases} 
      \frac{1}{3} & {k=0} \\
      \frac{2}{3 }& {k=1} 
   \end{cases}
   \\
p_{Y|X}\brak{0|0} = \frac{19}{25}\, 
p_{Y|X}\brak{0|1} = \frac{6}{25}\,
p_{Y|X}\brak{1|0} = \frac{45}{50}\,
p_{Y|X}\brak{1|2} = \frac{5}{50}
\end{align}
The desired probability is the probability that a slip drawn at random is marked other than Rs 1,
\begin{align}
&=1-p_X\brak{0}\\
&= p_X(1) + p_X(2)
\end{align}
Using Bayes theorem,
\begin{align}
&= p_Y\brak{0} \times \pr{Y=0 | X=1} + p_Y\brak{1} \times \pr{Y=1|X=2}\\
&=\frac{1}{3} \times \frac{6}{25} + \frac{2}{3} \times \frac{5}{50}\\
&=\frac{11}{75}
\end{align}

\newpage

%\tableofcontents

\bigskip

\renewcommand{\thefigure}{\theenumi}
\renewcommand{\thetable}{\theenumi}
%\renewcommand{\theequation}{\theenumi}

%\begin{abstract}
%%\boldmath
%In this letter, an algorithm for evaluating the exact analytical bit error rate  (BER)  for the piecewise linear (PL) combiner for  multiple relays is presented. Previous results were available only for upto three relays. The algorithm is unique in the sense that  the actual mathematical expressions, that are prohibitively large, need not be explicitly obtained. The diversity gain due to multiple relays is shown through plots of the analytical BER, well supported by simulations. 
%
%\end{abstract}
% IEEEtran.cls defaults to using nonbold math in the Abstract.
% This preserves the distinction between vectors and scalars. However,
% if the journal you are submitting to favors bold math in the abstract,
% then you can use LaTeX's standard command \boldmath at the very start
% of the abstract to achieve this. Many IEEE journals frown on math
% in the abstract anyway.

% Note that keywords are not normally used for peerreview papers.
%\begin{IEEEkeywords}
%Cooperative diversity, decode and forward, piecewise linear
%\end{IEEEkeywords}



% For peer review papers, you can put extra information on the cover
% page as needed:
% \ifCLASSOPTIONpeerreview
% \begin{center} \bfseries EDICS Category: 3-BBND \end{center}
% \fi
%
% For peerreview papers, this IEEEtran command inserts a page break and
% creates the second title. It will be ignored for other modes.
%\IEEEpeerreviewmaketitle




\item A card is selected from a pack of 52 cards\\
\begin{enumerate}[label=(\alph*)]
\item How many points are there in the sample space?
\item Calculate the probability that the cards is an ace of spades.
\item Calculate the probability that the card is (i) an ace (ii)black card.\\
\end{enumerate}
%\input{ncert/11/16/3/4_1/Prob_4.tex}
\item In a non-leap year, the probability of having 53 tuesdays or 53 wednesdays is\\
\solution
%A non-leap year has a total of 365 days, and a week has 7 days.\\
So it can be expressed as 
\begin{align}
365\text{days} &=52\times 7+1 \text{day}
\end{align}
$\implies$ 52 tuesdays or wednesdays\\
Random variable X denotes the days of a week
\begin{align}
p_X\brak{k}&=\frac{1}{7}; \quad \brak{1<k<7}
\end{align}
So the probability of extra day being tuesday or wednesday is
\begin{align}
p_X\brak{3}+p_X\brak{4}&=\frac{1}{7}+\frac{1}{7}=\frac{2}{7}
\end{align}



\item There are 1000 sealed envelopes in a box, 10 of them contain a cash prize of
Rs 100 each, 100 of them contain a cash prize of Rs 50 each and 200 of them
contain a cash prize of Rs 10 each and rest do not contain any cash prize. If they
are well shuffled and an envelope is picked up out, what is the probability that it
contains no cash prize?\\
\solution
%\begin{table}[H]
	\centering
\begin{tabular}{|c|c|c|}
\hline
Random variable &Value &Definition\\ \hline
\multirow{3}{*}{X} &0 &Slips of Rs 1\\
&1 &Slips of Rs 5\\
&2 &Slips of Rs 13\\ \hline
\multirow{2}{*}{Y} &0 &Box A\\
&1 &Box B\\\hline
\end{tabular}
\caption{}
\label{tab:Distribution}
\end{table}
See \tabref{tab:Distribution}.
\begin{align}
p_{Y}\brak{k}= \begin{cases} 
      \frac{1}{3} & {k=0} \\
      \frac{2}{3 }& {k=1} 
   \end{cases}
   \\
p_{Y|X}\brak{0|0} = \frac{19}{25}\, 
p_{Y|X}\brak{0|1} = \frac{6}{25}\,
p_{Y|X}\brak{1|0} = \frac{45}{50}\,
p_{Y|X}\brak{1|2} = \frac{5}{50}
\end{align}
The desired probability is the probability that a slip drawn at random is marked other than Rs 1,
\begin{align}
&=1-p_X\brak{0}\\
&= p_X(1) + p_X(2)
\end{align}
Using Bayes theorem,
\begin{align}
&= p_Y\brak{0} \times \pr{Y=0 | X=1} + p_Y\brak{1} \times \pr{Y=1|X=2}\\
&=\frac{1}{3} \times \frac{6}{25} + \frac{2}{3} \times \frac{5}{50}\\
&=\frac{11}{75}
\end{align}

\newpage

%\tableofcontents

\bigskip

\renewcommand{\thefigure}{\theenumi}
\renewcommand{\thetable}{\theenumi}
%\renewcommand{\theequation}{\theenumi}

%\begin{abstract}
%%\boldmath
%In this letter, an algorithm for evaluating the exact analytical bit error rate  (BER)  for the piecewise linear (PL) combiner for  multiple relays is presented. Previous results were available only for upto three relays. The algorithm is unique in the sense that  the actual mathematical expressions, that are prohibitively large, need not be explicitly obtained. The diversity gain due to multiple relays is shown through plots of the analytical BER, well supported by simulations. 
%
%\end{abstract}
% IEEEtran.cls defaults to using nonbold math in the Abstract.
% This preserves the distinction between vectors and scalars. However,
% if the journal you are submitting to favors bold math in the abstract,
% then you can use LaTeX's standard command \boldmath at the very start
% of the abstract to achieve this. Many IEEE journals frown on math
% in the abstract anyway.

% Note that keywords are not normally used for peerreview papers.
%\begin{IEEEkeywords}
%Cooperative diversity, decode and forward, piecewise linear
%\end{IEEEkeywords}



% For peer review papers, you can put extra information on the cover
% page as needed:
% \ifCLASSOPTIONpeerreview
% \begin{center} \bfseries EDICS Category: 3-BBND \end{center}
% \fi
%
% For peerreview papers, this IEEEtran command inserts a page break and
% creates the second title. It will be ignored for other modes.
%\IEEEpeerreviewmaketitle




\item 
A die is thrown and a card is selected at random from a deck of 52 playing cards. The probability of getting an even number on the die and a spade card.\\
\solution
%\begin{table}[H]
	\centering
\begin{tabular}{|c|c|c|}
\hline
Random variable &Value &Definition\\ \hline
\multirow{3}{*}{X} &0 &Slips of Rs 1\\
&1 &Slips of Rs 5\\
&2 &Slips of Rs 13\\ \hline
\multirow{2}{*}{Y} &0 &Box A\\
&1 &Box B\\\hline
\end{tabular}
\caption{}
\label{tab:Distribution}
\end{table}
See \tabref{tab:Distribution}.
\begin{align}
p_{Y}\brak{k}= \begin{cases} 
      \frac{1}{3} & {k=0} \\
      \frac{2}{3 }& {k=1} 
   \end{cases}
   \\
p_{Y|X}\brak{0|0} = \frac{19}{25}\, 
p_{Y|X}\brak{0|1} = \frac{6}{25}\,
p_{Y|X}\brak{1|0} = \frac{45}{50}\,
p_{Y|X}\brak{1|2} = \frac{5}{50}
\end{align}
The desired probability is the probability that a slip drawn at random is marked other than Rs 1,
\begin{align}
&=1-p_X\brak{0}\\
&= p_X(1) + p_X(2)
\end{align}
Using Bayes theorem,
\begin{align}
&= p_Y\brak{0} \times \pr{Y=0 | X=1} + p_Y\brak{1} \times \pr{Y=1|X=2}\\
&=\frac{1}{3} \times \frac{6}{25} + \frac{2}{3} \times \frac{5}{50}\\
&=\frac{11}{75}
\end{align}

\newpage

%\tableofcontents

\bigskip

\renewcommand{\thefigure}{\theenumi}
\renewcommand{\thetable}{\theenumi}
%\renewcommand{\theequation}{\theenumi}

%\begin{abstract}
%%\boldmath
%In this letter, an algorithm for evaluating the exact analytical bit error rate  (BER)  for the piecewise linear (PL) combiner for  multiple relays is presented. Previous results were available only for upto three relays. The algorithm is unique in the sense that  the actual mathematical expressions, that are prohibitively large, need not be explicitly obtained. The diversity gain due to multiple relays is shown through plots of the analytical BER, well supported by simulations. 
%
%\end{abstract}
% IEEEtran.cls defaults to using nonbold math in the Abstract.
% This preserves the distinction between vectors and scalars. However,
% if the journal you are submitting to favors bold math in the abstract,
% then you can use LaTeX's standard command \boldmath at the very start
% of the abstract to achieve this. Many IEEE journals frown on math
% in the abstract anyway.

% Note that keywords are not normally used for peerreview papers.
%\begin{IEEEkeywords}
%Cooperative diversity, decode and forward, piecewise linear
%\end{IEEEkeywords}



% For peer review papers, you can put extra information on the cover
% page as needed:
% \ifCLASSOPTIONpeerreview
% \begin{center} \bfseries EDICS Category: 3-BBND \end{center}
% \fi
%
% For peerreview papers, this IEEEtran command inserts a page break and
% creates the second title. It will be ignored for other modes.
%\IEEEpeerreviewmaketitle




\item
If 4-digit numbers greater than 5,000 are randomly formed from the digits 0, 1, 3, 5, and 7, what is the probability of forming a number divisible by 5 when:
\begin{enumerate}
    \item The digits are repeated?
    \item The repetition of digits is not allowed?
\end{enumerate}
\solution
%\begin{table}[H]
	\centering
\begin{tabular}{|c|c|c|}
\hline
Random variable &Value &Definition\\ \hline
\multirow{3}{*}{X} &0 &Slips of Rs 1\\
&1 &Slips of Rs 5\\
&2 &Slips of Rs 13\\ \hline
\multirow{2}{*}{Y} &0 &Box A\\
&1 &Box B\\\hline
\end{tabular}
\caption{}
\label{tab:Distribution}
\end{table}
See \tabref{tab:Distribution}.
\begin{align}
p_{Y}\brak{k}= \begin{cases} 
      \frac{1}{3} & {k=0} \\
      \frac{2}{3 }& {k=1} 
   \end{cases}
   \\
p_{Y|X}\brak{0|0} = \frac{19}{25}\, 
p_{Y|X}\brak{0|1} = \frac{6}{25}\,
p_{Y|X}\brak{1|0} = \frac{45}{50}\,
p_{Y|X}\brak{1|2} = \frac{5}{50}
\end{align}
The desired probability is the probability that a slip drawn at random is marked other than Rs 1,
\begin{align}
&=1-p_X\brak{0}\\
&= p_X(1) + p_X(2)
\end{align}
Using Bayes theorem,
\begin{align}
&= p_Y\brak{0} \times \pr{Y=0 | X=1} + p_Y\brak{1} \times \pr{Y=1|X=2}\\
&=\frac{1}{3} \times \frac{6}{25} + \frac{2}{3} \times \frac{5}{50}\\
&=\frac{11}{75}
\end{align}

\newpage

%\tableofcontents

\bigskip

\renewcommand{\thefigure}{\theenumi}
\renewcommand{\thetable}{\theenumi}
%\renewcommand{\theequation}{\theenumi}

%\begin{abstract}
%%\boldmath
%In this letter, an algorithm for evaluating the exact analytical bit error rate  (BER)  for the piecewise linear (PL) combiner for  multiple relays is presented. Previous results were available only for upto three relays. The algorithm is unique in the sense that  the actual mathematical expressions, that are prohibitively large, need not be explicitly obtained. The diversity gain due to multiple relays is shown through plots of the analytical BER, well supported by simulations. 
%
%\end{abstract}
% IEEEtran.cls defaults to using nonbold math in the Abstract.
% This preserves the distinction between vectors and scalars. However,
% if the journal you are submitting to favors bold math in the abstract,
% then you can use LaTeX's standard command \boldmath at the very start
% of the abstract to achieve this. Many IEEE journals frown on math
% in the abstract anyway.

% Note that keywords are not normally used for peerreview papers.
%\begin{IEEEkeywords}
%Cooperative diversity, decode and forward, piecewise linear
%\end{IEEEkeywords}



% For peer review papers, you can put extra information on the cover
% page as needed:
% \ifCLASSOPTIONpeerreview
% \begin{center} \bfseries EDICS Category: 3-BBND \end{center}
% \fi
%
% For peerreview papers, this IEEEtran command inserts a page break and
% creates the second title. It will be ignored for other modes.
%\IEEEpeerreviewmaketitle




\item Consider the probability space $\brak{\Omega, \mathcal{G}, P}$ where $\Omega = [0,2]$ and $\mathcal{G} = \cbrak{\phi, \Omega, [0,1], (1,2]}$. Let $X$ and $Y$ be two functions on $\Omega$ defined as
\begin{align*}
    X(\omega) = 
    \begin{cases}
        1 & \text{if }\omega \in [0, 1]\\
        2 & \text{if }\omega \in (1, 2]
    \end{cases}
\end{align*}
and
\begin{align*}
    Y(\omega) = 
    \begin{cases}
        2 & \text{if }\omega \in [0, 1.5]\\
        3 & \text{if }\omega \in (1.5, 2].
    \end{cases}
\end{align*}
Then which one of the following statements is true?
\begin{enumerate}
    \item [(A)] $X$ is a random variable with respect to $\mathcal{G}$, but $Y$ is not a random variable with respect to $\mathcal{G}$.
    \item [(B)] $Y$ is a random variable with respect to $\mathcal{G}$, but $X$ is not a random variable with respect to $\mathcal{G}$.
    \item [(C)] Neither $X$ nor $Y$ is a random variable with respect to $\mathcal{G}$.
    \item [(D)] Both $X$ and $Y$ are random variables with respect to $\mathcal{G}$.
\end{enumerate} \hfill (GATE ST 2023)\\
\solution
%\begin{table}[H]
	\centering
\begin{tabular}{|c|c|c|}
\hline
Random variable &Value &Definition\\ \hline
\multirow{3}{*}{X} &0 &Slips of Rs 1\\
&1 &Slips of Rs 5\\
&2 &Slips of Rs 13\\ \hline
\multirow{2}{*}{Y} &0 &Box A\\
&1 &Box B\\\hline
\end{tabular}
\caption{}
\label{tab:Distribution}
\end{table}
See \tabref{tab:Distribution}.
\begin{align}
p_{Y}\brak{k}= \begin{cases} 
      \frac{1}{3} & {k=0} \\
      \frac{2}{3 }& {k=1} 
   \end{cases}
   \\
p_{Y|X}\brak{0|0} = \frac{19}{25}\, 
p_{Y|X}\brak{0|1} = \frac{6}{25}\,
p_{Y|X}\brak{1|0} = \frac{45}{50}\,
p_{Y|X}\brak{1|2} = \frac{5}{50}
\end{align}
The desired probability is the probability that a slip drawn at random is marked other than Rs 1,
\begin{align}
&=1-p_X\brak{0}\\
&= p_X(1) + p_X(2)
\end{align}
Using Bayes theorem,
\begin{align}
&= p_Y\brak{0} \times \pr{Y=0 | X=1} + p_Y\brak{1} \times \pr{Y=1|X=2}\\
&=\frac{1}{3} \times \frac{6}{25} + \frac{2}{3} \times \frac{5}{50}\\
&=\frac{11}{75}
\end{align}

\newpage

%\tableofcontents

\bigskip

\renewcommand{\thefigure}{\theenumi}
\renewcommand{\thetable}{\theenumi}
%\renewcommand{\theequation}{\theenumi}

%\begin{abstract}
%%\boldmath
%In this letter, an algorithm for evaluating the exact analytical bit error rate  (BER)  for the piecewise linear (PL) combiner for  multiple relays is presented. Previous results were available only for upto three relays. The algorithm is unique in the sense that  the actual mathematical expressions, that are prohibitively large, need not be explicitly obtained. The diversity gain due to multiple relays is shown through plots of the analytical BER, well supported by simulations. 
%
%\end{abstract}
% IEEEtran.cls defaults to using nonbold math in the Abstract.
% This preserves the distinction between vectors and scalars. However,
% if the journal you are submitting to favors bold math in the abstract,
% then you can use LaTeX's standard command \boldmath at the very start
% of the abstract to achieve this. Many IEEE journals frown on math
% in the abstract anyway.

% Note that keywords are not normally used for peerreview papers.
%\begin{IEEEkeywords}
%Cooperative diversity, decode and forward, piecewise linear
%\end{IEEEkeywords}



% For peer review papers, you can put extra information on the cover
% page as needed:
% \ifCLASSOPTIONpeerreview
% \begin{center} \bfseries EDICS Category: 3-BBND \end{center}
% \fi
%
% For peerreview papers, this IEEEtran command inserts a page break and
% creates the second title. It will be ignored for other modes.
%\IEEEpeerreviewmaketitle




	\item  A die is loaded in such a way that each odd number is twice as likely to occur as
each even number. Find $P(G)$, where $G$ is the event that a number greater than
3 occurs on a single roll of the die.
\\
\solution
		%\begin{table}[H]
	\centering
\begin{tabular}{|c|c|c|}
\hline
Random variable &Value &Definition\\ \hline
\multirow{3}{*}{X} &0 &Slips of Rs 1\\
&1 &Slips of Rs 5\\
&2 &Slips of Rs 13\\ \hline
\multirow{2}{*}{Y} &0 &Box A\\
&1 &Box B\\\hline
\end{tabular}
\caption{}
\label{tab:Distribution}
\end{table}
See \tabref{tab:Distribution}.
\begin{align}
p_{Y}\brak{k}= \begin{cases} 
      \frac{1}{3} & {k=0} \\
      \frac{2}{3 }& {k=1} 
   \end{cases}
   \\
p_{Y|X}\brak{0|0} = \frac{19}{25}\, 
p_{Y|X}\brak{0|1} = \frac{6}{25}\,
p_{Y|X}\brak{1|0} = \frac{45}{50}\,
p_{Y|X}\brak{1|2} = \frac{5}{50}
\end{align}
The desired probability is the probability that a slip drawn at random is marked other than Rs 1,
\begin{align}
&=1-p_X\brak{0}\\
&= p_X(1) + p_X(2)
\end{align}
Using Bayes theorem,
\begin{align}
&= p_Y\brak{0} \times \pr{Y=0 | X=1} + p_Y\brak{1} \times \pr{Y=1|X=2}\\
&=\frac{1}{3} \times \frac{6}{25} + \frac{2}{3} \times \frac{5}{50}\\
&=\frac{11}{75}
\end{align}

\newpage

%\tableofcontents

\bigskip

\renewcommand{\thefigure}{\theenumi}
\renewcommand{\thetable}{\theenumi}
%\renewcommand{\theequation}{\theenumi}

%\begin{abstract}
%%\boldmath
%In this letter, an algorithm for evaluating the exact analytical bit error rate  (BER)  for the piecewise linear (PL) combiner for  multiple relays is presented. Previous results were available only for upto three relays. The algorithm is unique in the sense that  the actual mathematical expressions, that are prohibitively large, need not be explicitly obtained. The diversity gain due to multiple relays is shown through plots of the analytical BER, well supported by simulations. 
%
%\end{abstract}
% IEEEtran.cls defaults to using nonbold math in the Abstract.
% This preserves the distinction between vectors and scalars. However,
% if the journal you are submitting to favors bold math in the abstract,
% then you can use LaTeX's standard command \boldmath at the very start
% of the abstract to achieve this. Many IEEE journals frown on math
% in the abstract anyway.

% Note that keywords are not normally used for peerreview papers.
%\begin{IEEEkeywords}
%Cooperative diversity, decode and forward, piecewise linear
%\end{IEEEkeywords}



% For peer review papers, you can put extra information on the cover
% page as needed:
% \ifCLASSOPTIONpeerreview
% \begin{center} \bfseries EDICS Category: 3-BBND \end{center}
% \fi
%
% For peerreview papers, this IEEEtran command inserts a page break and
% creates the second title. It will be ignored for other modes.
%\IEEEpeerreviewmaketitle




	\item All the jacks, queens and kings are removed from a deck of 52 playing cards. The remaining cards are well shuffled and then one card is drawn at random. Giving ace a value 1 similar value for other cards, find the probability that the card has a value 
		\begin{enumerate}
			\item 7
			\item greater than 7
			\item less than 7
		\end{enumerate}
		%Number of cards left after removing all jacks, queens and kings 
\begin{align}
N	= 52 - 4\times 3
	= 40
\end{align}
%\begin{table}[H]
%\def\arraystretch{1.2}
%\begin{tabular}{|c|c|c|}
%\hline
%	\textbf{Parameter} &\textbf{Value} &\textbf{Description}\\ \hline
%	$X$ &1-10 &Represents the value of the card picked \\ \hline
%\end{tabular}
%\end{table}
Let $1 \le X \le 10$ be the value of the card picked.  Then,
\begin{align}
	p_X(k) &= \Pr(X=k)\ \forall\ 1 \leq k \leq 10\\
	&= \frac{4\times 1}{40}\\
	&= \frac{1}{10}\\
	\therefore p_X(k) &= 
	\begin{cases}
		\frac{1}{10} & 1 \leq k \leq 10\\
		0 & \text{otherwise}
	\end{cases}
\end{align}
and
\begin{align}
	F_{X}(k) &= \sum_{m=0}^{k}p_{X}(m) \quad 1 \leq k \leq 10\\
	&= \frac{k}{10}\\
	\therefore F_{X}(k) &= 
	\begin{cases}
		0 & k \leq 0\\
		\frac{k}{10} & 1\leq k \leq 10\\
		1 & k > 10 
	\end{cases}
\end{align}
\begin{enumerate}
	\item Probability that card has value equal to 7 is
		\begin{align}
			 p_{X}(7)
			= \frac{1}{10}
		\end{align}
	\item Probability that card has value greater than 7 is
		\begin{align}
			1 - F_X(7)
			&= 1 - \frac{7}{10}
			\\
			&= \frac{3}{10}
		\end{align}
	\item Probability that card has value less than 7 is
		\begin{align}
			 F_{X}(6)
			=\frac{6}{10}
		\end{align}
\end{enumerate}

  \item A Lot consists of 48 mobile phones of which 42 are good, 3 have only minor defects and 3 have major defects.Varnika will buy a phone if it is good but the trader will only buy a mobile if it has no major defects. One phone is selected at random from the lot. What is the probability that it is
\begin{enumerate}
	\item acceptable to Varnika?
            \item acceptable to the trader?
\end{enumerate}
\solution
	%\begin{table}[H]
	\centering
\begin{tabular}{|c|c|c|}
\hline
Random variable &Value &Definition\\ \hline
\multirow{3}{*}{X} &0 &Slips of Rs 1\\
&1 &Slips of Rs 5\\
&2 &Slips of Rs 13\\ \hline
\multirow{2}{*}{Y} &0 &Box A\\
&1 &Box B\\\hline
\end{tabular}
\caption{}
\label{tab:Distribution}
\end{table}
See \tabref{tab:Distribution}.
\begin{align}
p_{Y}\brak{k}= \begin{cases} 
      \frac{1}{3} & {k=0} \\
      \frac{2}{3 }& {k=1} 
   \end{cases}
   \\
p_{Y|X}\brak{0|0} = \frac{19}{25}\, 
p_{Y|X}\brak{0|1} = \frac{6}{25}\,
p_{Y|X}\brak{1|0} = \frac{45}{50}\,
p_{Y|X}\brak{1|2} = \frac{5}{50}
\end{align}
The desired probability is the probability that a slip drawn at random is marked other than Rs 1,
\begin{align}
&=1-p_X\brak{0}\\
&= p_X(1) + p_X(2)
\end{align}
Using Bayes theorem,
\begin{align}
&= p_Y\brak{0} \times \pr{Y=0 | X=1} + p_Y\brak{1} \times \pr{Y=1|X=2}\\
&=\frac{1}{3} \times \frac{6}{25} + \frac{2}{3} \times \frac{5}{50}\\
&=\frac{11}{75}
\end{align}

\newpage

%\tableofcontents

\bigskip

\renewcommand{\thefigure}{\theenumi}
\renewcommand{\thetable}{\theenumi}
%\renewcommand{\theequation}{\theenumi}

%\begin{abstract}
%%\boldmath
%In this letter, an algorithm for evaluating the exact analytical bit error rate  (BER)  for the piecewise linear (PL) combiner for  multiple relays is presented. Previous results were available only for upto three relays. The algorithm is unique in the sense that  the actual mathematical expressions, that are prohibitively large, need not be explicitly obtained. The diversity gain due to multiple relays is shown through plots of the analytical BER, well supported by simulations. 
%
%\end{abstract}
% IEEEtran.cls defaults to using nonbold math in the Abstract.
% This preserves the distinction between vectors and scalars. However,
% if the journal you are submitting to favors bold math in the abstract,
% then you can use LaTeX's standard command \boldmath at the very start
% of the abstract to achieve this. Many IEEE journals frown on math
% in the abstract anyway.

% Note that keywords are not normally used for peerreview papers.
%\begin{IEEEkeywords}
%Cooperative diversity, decode and forward, piecewise linear
%\end{IEEEkeywords}



% For peer review papers, you can put extra information on the cover
% page as needed:
% \ifCLASSOPTIONpeerreview
% \begin{center} \bfseries EDICS Category: 3-BBND \end{center}
% \fi
%
% For peerreview papers, this IEEEtran command inserts a page break and
% creates the second title. It will be ignored for other modes.
%\IEEEpeerreviewmaketitle




 \item A student says that if you throw a die, it will show up 1 or not 1. Therefore, the probability of getting 1 and the probability of getting 'not 1' each is equal to $\frac{1}{2}$. Is this correct? Give reasons.\\
 \solution
        %\begin{table}[H]
	\centering
\begin{tabular}{|c|c|c|}
\hline
Random variable &Value &Definition\\ \hline
\multirow{3}{*}{X} &0 &Slips of Rs 1\\
&1 &Slips of Rs 5\\
&2 &Slips of Rs 13\\ \hline
\multirow{2}{*}{Y} &0 &Box A\\
&1 &Box B\\\hline
\end{tabular}
\caption{}
\label{tab:Distribution}
\end{table}
See \tabref{tab:Distribution}.
\begin{align}
p_{Y}\brak{k}= \begin{cases} 
      \frac{1}{3} & {k=0} \\
      \frac{2}{3 }& {k=1} 
   \end{cases}
   \\
p_{Y|X}\brak{0|0} = \frac{19}{25}\, 
p_{Y|X}\brak{0|1} = \frac{6}{25}\,
p_{Y|X}\brak{1|0} = \frac{45}{50}\,
p_{Y|X}\brak{1|2} = \frac{5}{50}
\end{align}
The desired probability is the probability that a slip drawn at random is marked other than Rs 1,
\begin{align}
&=1-p_X\brak{0}\\
&= p_X(1) + p_X(2)
\end{align}
Using Bayes theorem,
\begin{align}
&= p_Y\brak{0} \times \pr{Y=0 | X=1} + p_Y\brak{1} \times \pr{Y=1|X=2}\\
&=\frac{1}{3} \times \frac{6}{25} + \frac{2}{3} \times \frac{5}{50}\\
&=\frac{11}{75}
\end{align}

\newpage

%\tableofcontents

\bigskip

\renewcommand{\thefigure}{\theenumi}
\renewcommand{\thetable}{\theenumi}
%\renewcommand{\theequation}{\theenumi}

%\begin{abstract}
%%\boldmath
%In this letter, an algorithm for evaluating the exact analytical bit error rate  (BER)  for the piecewise linear (PL) combiner for  multiple relays is presented. Previous results were available only for upto three relays. The algorithm is unique in the sense that  the actual mathematical expressions, that are prohibitively large, need not be explicitly obtained. The diversity gain due to multiple relays is shown through plots of the analytical BER, well supported by simulations. 
%
%\end{abstract}
% IEEEtran.cls defaults to using nonbold math in the Abstract.
% This preserves the distinction between vectors and scalars. However,
% if the journal you are submitting to favors bold math in the abstract,
% then you can use LaTeX's standard command \boldmath at the very start
% of the abstract to achieve this. Many IEEE journals frown on math
% in the abstract anyway.

% Note that keywords are not normally used for peerreview papers.
%\begin{IEEEkeywords}
%Cooperative diversity, decode and forward, piecewise linear
%\end{IEEEkeywords}



% For peer review papers, you can put extra information on the cover
% page as needed:
% \ifCLASSOPTIONpeerreview
% \begin{center} \bfseries EDICS Category: 3-BBND \end{center}
% \fi
%
% For peerreview papers, this IEEEtran command inserts a page break and
% creates the second title. It will be ignored for other modes.
%\IEEEpeerreviewmaketitle




   \item Four candidates A, B, C, D have ap-
plied for the assignment to coach a school cricket
team. If A is twice as likely to be selected as B, and
B and C are given about the same chance of being
selected, while C is twice as likely to be selected
as D, what are the probabilities that
\begin{enumerate}
\item C will be selected?
\item A will not be selected?
\end{enumerate}
	%\begin{table}[H]
	\centering
\begin{tabular}{|c|c|c|}
\hline
Random variable &Value &Definition\\ \hline
\multirow{3}{*}{X} &0 &Slips of Rs 1\\
&1 &Slips of Rs 5\\
&2 &Slips of Rs 13\\ \hline
\multirow{2}{*}{Y} &0 &Box A\\
&1 &Box B\\\hline
\end{tabular}
\caption{}
\label{tab:Distribution}
\end{table}
See \tabref{tab:Distribution}.
\begin{align}
p_{Y}\brak{k}= \begin{cases} 
      \frac{1}{3} & {k=0} \\
      \frac{2}{3 }& {k=1} 
   \end{cases}
   \\
p_{Y|X}\brak{0|0} = \frac{19}{25}\, 
p_{Y|X}\brak{0|1} = \frac{6}{25}\,
p_{Y|X}\brak{1|0} = \frac{45}{50}\,
p_{Y|X}\brak{1|2} = \frac{5}{50}
\end{align}
The desired probability is the probability that a slip drawn at random is marked other than Rs 1,
\begin{align}
&=1-p_X\brak{0}\\
&= p_X(1) + p_X(2)
\end{align}
Using Bayes theorem,
\begin{align}
&= p_Y\brak{0} \times \pr{Y=0 | X=1} + p_Y\brak{1} \times \pr{Y=1|X=2}\\
&=\frac{1}{3} \times \frac{6}{25} + \frac{2}{3} \times \frac{5}{50}\\
&=\frac{11}{75}
\end{align}

\newpage

%\tableofcontents

\bigskip

\renewcommand{\thefigure}{\theenumi}
\renewcommand{\thetable}{\theenumi}
%\renewcommand{\theequation}{\theenumi}

%\begin{abstract}
%%\boldmath
%In this letter, an algorithm for evaluating the exact analytical bit error rate  (BER)  for the piecewise linear (PL) combiner for  multiple relays is presented. Previous results were available only for upto three relays. The algorithm is unique in the sense that  the actual mathematical expressions, that are prohibitively large, need not be explicitly obtained. The diversity gain due to multiple relays is shown through plots of the analytical BER, well supported by simulations. 
%
%\end{abstract}
% IEEEtran.cls defaults to using nonbold math in the Abstract.
% This preserves the distinction between vectors and scalars. However,
% if the journal you are submitting to favors bold math in the abstract,
% then you can use LaTeX's standard command \boldmath at the very start
% of the abstract to achieve this. Many IEEE journals frown on math
% in the abstract anyway.

% Note that keywords are not normally used for peerreview papers.
%\begin{IEEEkeywords}
%Cooperative diversity, decode and forward, piecewise linear
%\end{IEEEkeywords}



% For peer review papers, you can put extra information on the cover
% page as needed:
% \ifCLASSOPTIONpeerreview
% \begin{center} \bfseries EDICS Category: 3-BBND \end{center}
% \fi
%
% For peerreview papers, this IEEEtran command inserts a page break and
% creates the second title. It will be ignored for other modes.
%\IEEEpeerreviewmaketitle




 \item A bag contain 24 balls of which $x$ balls are red, $2x$ are white and $3x$ are blue. A ball is selected at random, What is the probability that it is
\begin{enumerate}[label=\alph*)]
\item not red ?
\item white ?
\end{enumerate}
%\begin{table}[H]
	\centering
\begin{tabular}{|c|c|c|}
\hline
Random variable &Value &Definition\\ \hline
\multirow{3}{*}{X} &0 &Slips of Rs 1\\
&1 &Slips of Rs 5\\
&2 &Slips of Rs 13\\ \hline
\multirow{2}{*}{Y} &0 &Box A\\
&1 &Box B\\\hline
\end{tabular}
\caption{}
\label{tab:Distribution}
\end{table}
See \tabref{tab:Distribution}.
\begin{align}
p_{Y}\brak{k}= \begin{cases} 
      \frac{1}{3} & {k=0} \\
      \frac{2}{3 }& {k=1} 
   \end{cases}
   \\
p_{Y|X}\brak{0|0} = \frac{19}{25}\, 
p_{Y|X}\brak{0|1} = \frac{6}{25}\,
p_{Y|X}\brak{1|0} = \frac{45}{50}\,
p_{Y|X}\brak{1|2} = \frac{5}{50}
\end{align}
The desired probability is the probability that a slip drawn at random is marked other than Rs 1,
\begin{align}
&=1-p_X\brak{0}\\
&= p_X(1) + p_X(2)
\end{align}
Using Bayes theorem,
\begin{align}
&= p_Y\brak{0} \times \pr{Y=0 | X=1} + p_Y\brak{1} \times \pr{Y=1|X=2}\\
&=\frac{1}{3} \times \frac{6}{25} + \frac{2}{3} \times \frac{5}{50}\\
&=\frac{11}{75}
\end{align}

\newpage

%\tableofcontents

\bigskip

\renewcommand{\thefigure}{\theenumi}
\renewcommand{\thetable}{\theenumi}
%\renewcommand{\theequation}{\theenumi}

%\begin{abstract}
%%\boldmath
%In this letter, an algorithm for evaluating the exact analytical bit error rate  (BER)  for the piecewise linear (PL) combiner for  multiple relays is presented. Previous results were available only for upto three relays. The algorithm is unique in the sense that  the actual mathematical expressions, that are prohibitively large, need not be explicitly obtained. The diversity gain due to multiple relays is shown through plots of the analytical BER, well supported by simulations. 
%
%\end{abstract}
% IEEEtran.cls defaults to using nonbold math in the Abstract.
% This preserves the distinction between vectors and scalars. However,
% if the journal you are submitting to favors bold math in the abstract,
% then you can use LaTeX's standard command \boldmath at the very start
% of the abstract to achieve this. Many IEEE journals frown on math
% in the abstract anyway.

% Note that keywords are not normally used for peerreview papers.
%\begin{IEEEkeywords}
%Cooperative diversity, decode and forward, piecewise linear
%\end{IEEEkeywords}



% For peer review papers, you can put extra information on the cover
% page as needed:
% \ifCLASSOPTIONpeerreview
% \begin{center} \bfseries EDICS Category: 3-BBND \end{center}
% \fi
%
% For peerreview papers, this IEEEtran command inserts a page break and
% creates the second title. It will be ignored for other modes.
%\IEEEpeerreviewmaketitle




If the letters of the word ASSASSINATION are arranged at random. Find the Probability that
\begin{enumerate}[label=(\alph*)]
\item Four $S's$ come consecutively in the word
\item Two  $I's$ and two $N's$ come together
\item All $A's$ are not coming together
\item No two $A's$ are coming together
\end{enumerate}
%\begin{table}[H]
	\centering
\begin{tabular}{|c|c|c|}
\hline
Random variable &Value &Definition\\ \hline
\multirow{3}{*}{X} &0 &Slips of Rs 1\\
&1 &Slips of Rs 5\\
&2 &Slips of Rs 13\\ \hline
\multirow{2}{*}{Y} &0 &Box A\\
&1 &Box B\\\hline
\end{tabular}
\caption{}
\label{tab:Distribution}
\end{table}
See \tabref{tab:Distribution}.
\begin{align}
p_{Y}\brak{k}= \begin{cases} 
      \frac{1}{3} & {k=0} \\
      \frac{2}{3 }& {k=1} 
   \end{cases}
   \\
p_{Y|X}\brak{0|0} = \frac{19}{25}\, 
p_{Y|X}\brak{0|1} = \frac{6}{25}\,
p_{Y|X}\brak{1|0} = \frac{45}{50}\,
p_{Y|X}\brak{1|2} = \frac{5}{50}
\end{align}
The desired probability is the probability that a slip drawn at random is marked other than Rs 1,
\begin{align}
&=1-p_X\brak{0}\\
&= p_X(1) + p_X(2)
\end{align}
Using Bayes theorem,
\begin{align}
&= p_Y\brak{0} \times \pr{Y=0 | X=1} + p_Y\brak{1} \times \pr{Y=1|X=2}\\
&=\frac{1}{3} \times \frac{6}{25} + \frac{2}{3} \times \frac{5}{50}\\
&=\frac{11}{75}
\end{align}

\newpage

%\tableofcontents

\bigskip

\renewcommand{\thefigure}{\theenumi}
\renewcommand{\thetable}{\theenumi}
%\renewcommand{\theequation}{\theenumi}

%\begin{abstract}
%%\boldmath
%In this letter, an algorithm for evaluating the exact analytical bit error rate  (BER)  for the piecewise linear (PL) combiner for  multiple relays is presented. Previous results were available only for upto three relays. The algorithm is unique in the sense that  the actual mathematical expressions, that are prohibitively large, need not be explicitly obtained. The diversity gain due to multiple relays is shown through plots of the analytical BER, well supported by simulations. 
%
%\end{abstract}
% IEEEtran.cls defaults to using nonbold math in the Abstract.
% This preserves the distinction between vectors and scalars. However,
% if the journal you are submitting to favors bold math in the abstract,
% then you can use LaTeX's standard command \boldmath at the very start
% of the abstract to achieve this. Many IEEE journals frown on math
% in the abstract anyway.

% Note that keywords are not normally used for peerreview papers.
%\begin{IEEEkeywords}
%Cooperative diversity, decode and forward, piecewise linear
%\end{IEEEkeywords}



% For peer review papers, you can put extra information on the cover
% page as needed:
% \ifCLASSOPTIONpeerreview
% \begin{center} \bfseries EDICS Category: 3-BBND \end{center}
% \fi
%
% For peerreview papers, this IEEEtran command inserts a page break and
% creates the second title. It will be ignored for other modes.
%\IEEEpeerreviewmaketitle




	\item One urn contains two black balls (labelled B1 and B2) and one white ball. A
	second urn contains one black ball and two white balls (labelled W1 and W2).
	Suppose the following experiment is performed. One of the two urns is chosen
	at random. Next a ball is randomly chosen from the urn. Then a second ball is
	chosen at random from the same urn without replacing the first ball.
	
	\begin{enumerate}
	\item What is the probability that two black balls are chosen?
	
	\item What is the probability that two balls of opposite colour are chosen?
	\end{enumerate}
	\solution
	%\begin{align}
    \label{eq:12.13.6.18.1}
	\because	\pr{A|B} &> \pr{A},\
\frac{\pr{AB}}{\pr{B}} > \pr{A}
\\
    \label{eq:12.13.6.18.2}
	\implies \pr{AB} &> \pr{A}\pr{B}
	\\
	\text{or, } \frac{\pr{AB}}{\pr{A}} &=\pr{B|A} > \pr{A}
\end{align}

\end{enumerate}

	\item 
The number lock of a suitcase has 4 wheels each labelled with ten digits i.e. from 0 to 9.The lock opens with a sequence of four digits with no repeats.What is the probability of a person getting the right sequence to open the suitcase.
\\
\solution
		%\begin{enumerate}[label=\thesection.\arabic*,ref=\thesection.\theenumi]
	\item One card is drawn from a well-shuffled deck of 52 cards. Find the probability of getting
\begin{enumerate}
\item A king of red colour 
\item A face card 
\item A red face card
\item The jack of hearts
\item A spade
\item The queen of diamonds

\end{enumerate}
\solution
		%\begin{table}[H]
	\centering
\begin{tabular}{|c|c|c|}
\hline
Random variable &Value &Definition\\ \hline
\multirow{3}{*}{X} &0 &Slips of Rs 1\\
&1 &Slips of Rs 5\\
&2 &Slips of Rs 13\\ \hline
\multirow{2}{*}{Y} &0 &Box A\\
&1 &Box B\\\hline
\end{tabular}
\caption{}
\label{tab:Distribution}
\end{table}
See \tabref{tab:Distribution}.
\begin{align}
p_{Y}\brak{k}= \begin{cases} 
      \frac{1}{3} & {k=0} \\
      \frac{2}{3 }& {k=1} 
   \end{cases}
   \\
p_{Y|X}\brak{0|0} = \frac{19}{25}\, 
p_{Y|X}\brak{0|1} = \frac{6}{25}\,
p_{Y|X}\brak{1|0} = \frac{45}{50}\,
p_{Y|X}\brak{1|2} = \frac{5}{50}
\end{align}
The desired probability is the probability that a slip drawn at random is marked other than Rs 1,
\begin{align}
&=1-p_X\brak{0}\\
&= p_X(1) + p_X(2)
\end{align}
Using Bayes theorem,
\begin{align}
&= p_Y\brak{0} \times \pr{Y=0 | X=1} + p_Y\brak{1} \times \pr{Y=1|X=2}\\
&=\frac{1}{3} \times \frac{6}{25} + \frac{2}{3} \times \frac{5}{50}\\
&=\frac{11}{75}
\end{align}

\newpage

%\tableofcontents

\bigskip

\renewcommand{\thefigure}{\theenumi}
\renewcommand{\thetable}{\theenumi}
%\renewcommand{\theequation}{\theenumi}

%\begin{abstract}
%%\boldmath
%In this letter, an algorithm for evaluating the exact analytical bit error rate  (BER)  for the piecewise linear (PL) combiner for  multiple relays is presented. Previous results were available only for upto three relays. The algorithm is unique in the sense that  the actual mathematical expressions, that are prohibitively large, need not be explicitly obtained. The diversity gain due to multiple relays is shown through plots of the analytical BER, well supported by simulations. 
%
%\end{abstract}
% IEEEtran.cls defaults to using nonbold math in the Abstract.
% This preserves the distinction between vectors and scalars. However,
% if the journal you are submitting to favors bold math in the abstract,
% then you can use LaTeX's standard command \boldmath at the very start
% of the abstract to achieve this. Many IEEE journals frown on math
% in the abstract anyway.

% Note that keywords are not normally used for peerreview papers.
%\begin{IEEEkeywords}
%Cooperative diversity, decode and forward, piecewise linear
%\end{IEEEkeywords}



% For peer review papers, you can put extra information on the cover
% page as needed:
% \ifCLASSOPTIONpeerreview
% \begin{center} \bfseries EDICS Category: 3-BBND \end{center}
% \fi
%
% For peerreview papers, this IEEEtran command inserts a page break and
% creates the second title. It will be ignored for other modes.
%\IEEEpeerreviewmaketitle




	\item Five cards—the ten, jack, queen, king and ace of diamonds, are well-shuffled with their face downwards. One card is then picked up at random.
\begin{enumerate}
\item
What is the probability that the card is the queen? 
\item
If the queen is drawn and put aside, what is the probability that the second card picked up is (a) an ace? (b) a queen?\\
\end{enumerate}
\solution
		%\begin{enumerate}[label=\thesection.\arabic*,ref=\thesection.\theenumi]
	\item One card is drawn from a well-shuffled deck of 52 cards. Find the probability of getting
\begin{enumerate}
\item A king of red colour 
\item A face card 
\item A red face card
\item The jack of hearts
\item A spade
\item The queen of diamonds

\end{enumerate}
\solution
		%\input{ncert/10/15/1/14/main.tex}
	\item Five cards—the ten, jack, queen, king and ace of diamonds, are well-shuffled with their face downwards. One card is then picked up at random.
\begin{enumerate}
\item
What is the probability that the card is the queen? 
\item
If the queen is drawn and put aside, what is the probability that the second card picked up is (a) an ace? (b) a queen?\\
\end{enumerate}
\solution
		%\input{ncert/10/15/1/15/defs.tex}
	\item A bag contains $5$ red balls and some blue balls. If the probability of drawing a blue ball is double that if a red ball, determine the number of blue balls in the bag. 
		\\
\solution
		%\input{ncert/10/15/2/3/defs.tex}
	\item A card is selected from a pack of 52 cards.
 \begin{enumerate}[label=(\alph*)] 
                 \item How many points are there in the sample space?
                 \item Calculate the probability that the card is an ace of spades.
                 \item Calculate the probability that the card is (i) an ace and (ii) black card.
 \end{enumerate}
\solution
		%\input{ncert/11/16/3/4/main.tex}
\item Four cards are drawn from a well-shuffled deck of 52 cards. What is the probability of obtaining 3 diamonds and one spade.
\\
\solution
		%\input{ncert/11/16/4/2/defs.tex}
\item In a certain lottery 10,000 tickets are sold and ten equal prizes are awarded. What is the probability of not getting a prize if you buy (a) one ticket (b) two tickets (c) 10 tickets ?	
\\
\solution
		%\input{ncert/11/16/4/4/defs.tex}
		%
\item 
Out of 100 students, two sections of 40 and 60 are formed. If you and your friend are among the 100 students, what is the probability that
\begin{enumerate}
\item you both enter the same section?
\item you both enter the different sections?
\end{enumerate}
\solution
		%\input{ncert/11/16/4/5/defs.tex}
	\item 
The number lock of a suitcase has 4 wheels each labelled with ten digits i.e. from 0 to 9.The lock opens with a sequence of four digits with no repeats.What is the probability of a person getting the right sequence to open the suitcase.
\\
\solution
		%\input{ncert/11/16/4/10/defs.tex}
		%
\item 
Two cards are drawn at random and without replacement from a pack of 52 playing cards. Find the probability that both the cards are black.
\\
\solution
		%\input{ncert/12/13/2/2/defs.tex}
		\item A box of oranges is inspected by examining three randomly selected oranges drawn without replacement. If all the three oranges are good, the box is approved for sale, otherwise, it is rejected. Find the probability that a box containing 15 oranges out of which 12 are good and 3 are bad ones will be approved for sale.
		\label{ncert/12/13/2/3/defs.tex}
		\item Two balls are drawn at random with replacement from a box containing 10 black and 8 red balls. Find the probability that
		\label{ncert/12/13/2/12}
\begin{enumerate}
\item both balls are red.
\item first ball is black and second is red.
\item one of them is black and other is red.
\end{enumerate}

\item In a hostel, 60\% of the students read Hindi newspaper, 40\% read English newspaper and 20\% read both Hindi and English newspapers. A student is selected at random.
		\label{ncert/12/13/2/15}
\begin{enumerate}
\item Find the probability that she reads neither Hindi nor English newspapers.
\item If she reads Hindi newspaper, find the probability that she reads English newspaper.
\item If she reads English newspaper, find the probability that she reads Hindi newspaper.\\
\end{enumerate}
\item The probability of obtaining an even prime number on each die, when a pair of dice is rolled is 
\begin{enumerate}
    \item $0$ 
    
    \item $\frac{1}{3}$ 
    
    \item $\frac{1}{12}$ 
    
    \item $\frac{1}{36}$ 
\end{enumerate}
\solution
		%\input{ncert/12/13/2/17/defs.tex}
	\item A bag contains 4 red and 4 black balls, another bag contains 2 red and 6 black balls. One of the two bags is selected at random and a ball is drawn from the bag which is found to be red. Find the probability that the ball is drawn from the first bag.
\\
\solution
		%\input{ncert/12/13/3/2/main.tex}
  \item
  Cards with numbers 2 to 101 are placed in a box. A card is selected at random.Find the probability that the card has
\begin{enumerate}[label=(\roman*)]
	\item an even number 
	\item a square number
\end{enumerate}
\solution
%\input{exemplar/10/13/3/32/main.tex}
\item
The king, queen and jack of clubs are removed from a deck of 52 playing cards and then well shuffled. Now one card is drawn at random from the remaining cards.  Determine the probability that the card is
\begin{enumerate}[label=(\roman*)]
\item a club
\item 10 of hearts
\end{enumerate}
\solution
%\input{exemplar/10/13/3/29/main.tex}
\item A team of medical students doing their internship have to assist during surgeries
at a city hospital. The probabilities of surgeries rated as very complex, complex,
routine, simple or very simple are respectively, 0.15, 0.20, 0.31, 0.26, .08. Find
the probabilities that a particular surgery will be rated
\begin{enumerate}
	\item complex or very complex;
	\item neither very complex nor very simple;
	\item routine or complex
	\item routine or simple
\end{enumerate}
\solution
%\input{exemplar/11/16/3/8(1)/main.tex}
\item A card is selected from a pack of 52 cards.
\begin{enumerate}[label=(\alph*)]
    \item How many points are there in the sample space?
    \item Calculate the probability that the card is an ace of spades.
    \item Calculate the probability that the card is (i) an ace and (ii) black card.
\end{enumerate}
\solution
%\input{exemplar/11/16/3/4/main2.tex}
\item The probability that a non leap year selected at random will contain 53 sundays.
\\
\solution
%\input{exemplar/10/13/1/19/main.tex}
\item One of the four persons John, Rita, Aslam or Gurpreet will be promoted next
month. Consequently the sample space consists of four elementary outcomes
S = {John promoted, Rita promoted, Aslam promoted, Gurpreet promoted}
You are told that the chances of John’s promotion is same as that of Gurpreet,
Rita’s chances of promotion are twice as likely as Johns. Aslam’s chances are
four times that of John.
\begin{enumerate}
	\item Determine
	\begin{enumerate}
		\item P (John promoted)
		\item P (Rita promoted)
		\item P (Aslam promoted)
		\item P (Gurpreet promoted)
	\end{enumerate}
	\item If A = {John promoted or Gurpreet promoted}, find P (A).
\end{enumerate}
\solution
%\input{exemplar/11/16/3/10/main.tex}
\item A card is drawn from a deck of 52 cards. Find the probability of getting a king or a heart or a red card.\\
\solution
%\input{exemplar/11/16/3/15/main.tex}
\item The probability that a student will pass his examination is 0.73, the probability of
the student getting a compartment is 0.13, and the probability that the student will
either pass or get compartment is 0.96. State True or False.\\
\solution
%\input{exemplar/11/16/3/31/main.tex}
\item A card is selected from a pack of 52 cards\\
\begin{enumerate}[label=(\alph*)]
\item How many points are there in the sample space?
\item Calculate the probability that the cards is an ace of spades.
\item Calculate the probability that the card is (i) an ace (ii)black card.\\
\end{enumerate}
%\input{ncert/11/16/3/4_1/Prob_4.tex}
\item In a non-leap year, the probability of having 53 tuesdays or 53 wednesdays is\\
\solution
%\input{exemplar/11/16/3/18/main.tex}
\item There are 1000 sealed envelopes in a box, 10 of them contain a cash prize of
Rs 100 each, 100 of them contain a cash prize of Rs 50 each and 200 of them
contain a cash prize of Rs 10 each and rest do not contain any cash prize. If they
are well shuffled and an envelope is picked up out, what is the probability that it
contains no cash prize?\\
\solution
%\input{exemplar/10/13/3/34/main.tex}
\item 
A die is thrown and a card is selected at random from a deck of 52 playing cards. The probability of getting an even number on the die and a spade card.\\
\solution
%\input{exemplar/12/13/3/78/main.tex}
\item
If 4-digit numbers greater than 5,000 are randomly formed from the digits 0, 1, 3, 5, and 7, what is the probability of forming a number divisible by 5 when:
\begin{enumerate}
    \item The digits are repeated?
    \item The repetition of digits is not allowed?
\end{enumerate}
\solution
%\input{ncert/11/16/4/9/main.tex}
\item Consider the probability space $\brak{\Omega, \mathcal{G}, P}$ where $\Omega = [0,2]$ and $\mathcal{G} = \cbrak{\phi, \Omega, [0,1], (1,2]}$. Let $X$ and $Y$ be two functions on $\Omega$ defined as
\begin{align*}
    X(\omega) = 
    \begin{cases}
        1 & \text{if }\omega \in [0, 1]\\
        2 & \text{if }\omega \in (1, 2]
    \end{cases}
\end{align*}
and
\begin{align*}
    Y(\omega) = 
    \begin{cases}
        2 & \text{if }\omega \in [0, 1.5]\\
        3 & \text{if }\omega \in (1.5, 2].
    \end{cases}
\end{align*}
Then which one of the following statements is true?
\begin{enumerate}
    \item [(A)] $X$ is a random variable with respect to $\mathcal{G}$, but $Y$ is not a random variable with respect to $\mathcal{G}$.
    \item [(B)] $Y$ is a random variable with respect to $\mathcal{G}$, but $X$ is not a random variable with respect to $\mathcal{G}$.
    \item [(C)] Neither $X$ nor $Y$ is a random variable with respect to $\mathcal{G}$.
    \item [(D)] Both $X$ and $Y$ are random variables with respect to $\mathcal{G}$.
\end{enumerate} \hfill (GATE ST 2023)\\
\solution
%\input{gate/ST/2023/14/main.tex}
	\item  A die is loaded in such a way that each odd number is twice as likely to occur as
each even number. Find $P(G)$, where $G$ is the event that a number greater than
3 occurs on a single roll of the die.
\\
\solution
		%\input{exemplar/11/16/3/5/main.tex}
	\item All the jacks, queens and kings are removed from a deck of 52 playing cards. The remaining cards are well shuffled and then one card is drawn at random. Giving ace a value 1 similar value for other cards, find the probability that the card has a value 
		\begin{enumerate}
			\item 7
			\item greater than 7
			\item less than 7
		\end{enumerate}
		%\input{exemplar/10/13/3/30/main.tex}
  \item A Lot consists of 48 mobile phones of which 42 are good, 3 have only minor defects and 3 have major defects.Varnika will buy a phone if it is good but the trader will only buy a mobile if it has no major defects. One phone is selected at random from the lot. What is the probability that it is
\begin{enumerate}
	\item acceptable to Varnika?
            \item acceptable to the trader?
\end{enumerate}
\solution
	%\input{exemplar/10/13/3/40/main.tex}
 \item A student says that if you throw a die, it will show up 1 or not 1. Therefore, the probability of getting 1 and the probability of getting 'not 1' each is equal to $\frac{1}{2}$. Is this correct? Give reasons.\\
 \solution
        %\input{exemplar/10/13/2/9/main.tex}
   \item Four candidates A, B, C, D have ap-
plied for the assignment to coach a school cricket
team. If A is twice as likely to be selected as B, and
B and C are given about the same chance of being
selected, while C is twice as likely to be selected
as D, what are the probabilities that
\begin{enumerate}
\item C will be selected?
\item A will not be selected?
\end{enumerate}
	%\input{exemplar/11/16/3/9/main.tex}
 \item A bag contain 24 balls of which $x$ balls are red, $2x$ are white and $3x$ are blue. A ball is selected at random, What is the probability that it is
\begin{enumerate}[label=\alph*)]
\item not red ?
\item white ?
\end{enumerate}
%\input{exemplar/10/13/3/41/main.tex}
If the letters of the word ASSASSINATION are arranged at random. Find the Probability that
\begin{enumerate}[label=(\alph*)]
\item Four $S's$ come consecutively in the word
\item Two  $I's$ and two $N's$ come together
\item All $A's$ are not coming together
\item No two $A's$ are coming together
\end{enumerate}
%\input{exemplar/11/16/3/14/main.tex}
	\item One urn contains two black balls (labelled B1 and B2) and one white ball. A
	second urn contains one black ball and two white balls (labelled W1 and W2).
	Suppose the following experiment is performed. One of the two urns is chosen
	at random. Next a ball is randomly chosen from the urn. Then a second ball is
	chosen at random from the same urn without replacing the first ball.
	
	\begin{enumerate}
	\item What is the probability that two black balls are chosen?
	
	\item What is the probability that two balls of opposite colour are chosen?
	\end{enumerate}
	\solution
	%\input{exemplar/11/16/3/12/main1.tex}
\end{enumerate}

	\item A bag contains $5$ red balls and some blue balls. If the probability of drawing a blue ball is double that if a red ball, determine the number of blue balls in the bag. 
		\\
\solution
		%\begin{enumerate}[label=\thesection.\arabic*,ref=\thesection.\theenumi]
	\item One card is drawn from a well-shuffled deck of 52 cards. Find the probability of getting
\begin{enumerate}
\item A king of red colour 
\item A face card 
\item A red face card
\item The jack of hearts
\item A spade
\item The queen of diamonds

\end{enumerate}
\solution
		%\input{ncert/10/15/1/14/main.tex}
	\item Five cards—the ten, jack, queen, king and ace of diamonds, are well-shuffled with their face downwards. One card is then picked up at random.
\begin{enumerate}
\item
What is the probability that the card is the queen? 
\item
If the queen is drawn and put aside, what is the probability that the second card picked up is (a) an ace? (b) a queen?\\
\end{enumerate}
\solution
		%\input{ncert/10/15/1/15/defs.tex}
	\item A bag contains $5$ red balls and some blue balls. If the probability of drawing a blue ball is double that if a red ball, determine the number of blue balls in the bag. 
		\\
\solution
		%\input{ncert/10/15/2/3/defs.tex}
	\item A card is selected from a pack of 52 cards.
 \begin{enumerate}[label=(\alph*)] 
                 \item How many points are there in the sample space?
                 \item Calculate the probability that the card is an ace of spades.
                 \item Calculate the probability that the card is (i) an ace and (ii) black card.
 \end{enumerate}
\solution
		%\input{ncert/11/16/3/4/main.tex}
\item Four cards are drawn from a well-shuffled deck of 52 cards. What is the probability of obtaining 3 diamonds and one spade.
\\
\solution
		%\input{ncert/11/16/4/2/defs.tex}
\item In a certain lottery 10,000 tickets are sold and ten equal prizes are awarded. What is the probability of not getting a prize if you buy (a) one ticket (b) two tickets (c) 10 tickets ?	
\\
\solution
		%\input{ncert/11/16/4/4/defs.tex}
		%
\item 
Out of 100 students, two sections of 40 and 60 are formed. If you and your friend are among the 100 students, what is the probability that
\begin{enumerate}
\item you both enter the same section?
\item you both enter the different sections?
\end{enumerate}
\solution
		%\input{ncert/11/16/4/5/defs.tex}
	\item 
The number lock of a suitcase has 4 wheels each labelled with ten digits i.e. from 0 to 9.The lock opens with a sequence of four digits with no repeats.What is the probability of a person getting the right sequence to open the suitcase.
\\
\solution
		%\input{ncert/11/16/4/10/defs.tex}
		%
\item 
Two cards are drawn at random and without replacement from a pack of 52 playing cards. Find the probability that both the cards are black.
\\
\solution
		%\input{ncert/12/13/2/2/defs.tex}
		\item A box of oranges is inspected by examining three randomly selected oranges drawn without replacement. If all the three oranges are good, the box is approved for sale, otherwise, it is rejected. Find the probability that a box containing 15 oranges out of which 12 are good and 3 are bad ones will be approved for sale.
		\label{ncert/12/13/2/3/defs.tex}
		\item Two balls are drawn at random with replacement from a box containing 10 black and 8 red balls. Find the probability that
		\label{ncert/12/13/2/12}
\begin{enumerate}
\item both balls are red.
\item first ball is black and second is red.
\item one of them is black and other is red.
\end{enumerate}

\item In a hostel, 60\% of the students read Hindi newspaper, 40\% read English newspaper and 20\% read both Hindi and English newspapers. A student is selected at random.
		\label{ncert/12/13/2/15}
\begin{enumerate}
\item Find the probability that she reads neither Hindi nor English newspapers.
\item If she reads Hindi newspaper, find the probability that she reads English newspaper.
\item If she reads English newspaper, find the probability that she reads Hindi newspaper.\\
\end{enumerate}
\item The probability of obtaining an even prime number on each die, when a pair of dice is rolled is 
\begin{enumerate}
    \item $0$ 
    
    \item $\frac{1}{3}$ 
    
    \item $\frac{1}{12}$ 
    
    \item $\frac{1}{36}$ 
\end{enumerate}
\solution
		%\input{ncert/12/13/2/17/defs.tex}
	\item A bag contains 4 red and 4 black balls, another bag contains 2 red and 6 black balls. One of the two bags is selected at random and a ball is drawn from the bag which is found to be red. Find the probability that the ball is drawn from the first bag.
\\
\solution
		%\input{ncert/12/13/3/2/main.tex}
  \item
  Cards with numbers 2 to 101 are placed in a box. A card is selected at random.Find the probability that the card has
\begin{enumerate}[label=(\roman*)]
	\item an even number 
	\item a square number
\end{enumerate}
\solution
%\input{exemplar/10/13/3/32/main.tex}
\item
The king, queen and jack of clubs are removed from a deck of 52 playing cards and then well shuffled. Now one card is drawn at random from the remaining cards.  Determine the probability that the card is
\begin{enumerate}[label=(\roman*)]
\item a club
\item 10 of hearts
\end{enumerate}
\solution
%\input{exemplar/10/13/3/29/main.tex}
\item A team of medical students doing their internship have to assist during surgeries
at a city hospital. The probabilities of surgeries rated as very complex, complex,
routine, simple or very simple are respectively, 0.15, 0.20, 0.31, 0.26, .08. Find
the probabilities that a particular surgery will be rated
\begin{enumerate}
	\item complex or very complex;
	\item neither very complex nor very simple;
	\item routine or complex
	\item routine or simple
\end{enumerate}
\solution
%\input{exemplar/11/16/3/8(1)/main.tex}
\item A card is selected from a pack of 52 cards.
\begin{enumerate}[label=(\alph*)]
    \item How many points are there in the sample space?
    \item Calculate the probability that the card is an ace of spades.
    \item Calculate the probability that the card is (i) an ace and (ii) black card.
\end{enumerate}
\solution
%\input{exemplar/11/16/3/4/main2.tex}
\item The probability that a non leap year selected at random will contain 53 sundays.
\\
\solution
%\input{exemplar/10/13/1/19/main.tex}
\item One of the four persons John, Rita, Aslam or Gurpreet will be promoted next
month. Consequently the sample space consists of four elementary outcomes
S = {John promoted, Rita promoted, Aslam promoted, Gurpreet promoted}
You are told that the chances of John’s promotion is same as that of Gurpreet,
Rita’s chances of promotion are twice as likely as Johns. Aslam’s chances are
four times that of John.
\begin{enumerate}
	\item Determine
	\begin{enumerate}
		\item P (John promoted)
		\item P (Rita promoted)
		\item P (Aslam promoted)
		\item P (Gurpreet promoted)
	\end{enumerate}
	\item If A = {John promoted or Gurpreet promoted}, find P (A).
\end{enumerate}
\solution
%\input{exemplar/11/16/3/10/main.tex}
\item A card is drawn from a deck of 52 cards. Find the probability of getting a king or a heart or a red card.\\
\solution
%\input{exemplar/11/16/3/15/main.tex}
\item The probability that a student will pass his examination is 0.73, the probability of
the student getting a compartment is 0.13, and the probability that the student will
either pass or get compartment is 0.96. State True or False.\\
\solution
%\input{exemplar/11/16/3/31/main.tex}
\item A card is selected from a pack of 52 cards\\
\begin{enumerate}[label=(\alph*)]
\item How many points are there in the sample space?
\item Calculate the probability that the cards is an ace of spades.
\item Calculate the probability that the card is (i) an ace (ii)black card.\\
\end{enumerate}
%\input{ncert/11/16/3/4_1/Prob_4.tex}
\item In a non-leap year, the probability of having 53 tuesdays or 53 wednesdays is\\
\solution
%\input{exemplar/11/16/3/18/main.tex}
\item There are 1000 sealed envelopes in a box, 10 of them contain a cash prize of
Rs 100 each, 100 of them contain a cash prize of Rs 50 each and 200 of them
contain a cash prize of Rs 10 each and rest do not contain any cash prize. If they
are well shuffled and an envelope is picked up out, what is the probability that it
contains no cash prize?\\
\solution
%\input{exemplar/10/13/3/34/main.tex}
\item 
A die is thrown and a card is selected at random from a deck of 52 playing cards. The probability of getting an even number on the die and a spade card.\\
\solution
%\input{exemplar/12/13/3/78/main.tex}
\item
If 4-digit numbers greater than 5,000 are randomly formed from the digits 0, 1, 3, 5, and 7, what is the probability of forming a number divisible by 5 when:
\begin{enumerate}
    \item The digits are repeated?
    \item The repetition of digits is not allowed?
\end{enumerate}
\solution
%\input{ncert/11/16/4/9/main.tex}
\item Consider the probability space $\brak{\Omega, \mathcal{G}, P}$ where $\Omega = [0,2]$ and $\mathcal{G} = \cbrak{\phi, \Omega, [0,1], (1,2]}$. Let $X$ and $Y$ be two functions on $\Omega$ defined as
\begin{align*}
    X(\omega) = 
    \begin{cases}
        1 & \text{if }\omega \in [0, 1]\\
        2 & \text{if }\omega \in (1, 2]
    \end{cases}
\end{align*}
and
\begin{align*}
    Y(\omega) = 
    \begin{cases}
        2 & \text{if }\omega \in [0, 1.5]\\
        3 & \text{if }\omega \in (1.5, 2].
    \end{cases}
\end{align*}
Then which one of the following statements is true?
\begin{enumerate}
    \item [(A)] $X$ is a random variable with respect to $\mathcal{G}$, but $Y$ is not a random variable with respect to $\mathcal{G}$.
    \item [(B)] $Y$ is a random variable with respect to $\mathcal{G}$, but $X$ is not a random variable with respect to $\mathcal{G}$.
    \item [(C)] Neither $X$ nor $Y$ is a random variable with respect to $\mathcal{G}$.
    \item [(D)] Both $X$ and $Y$ are random variables with respect to $\mathcal{G}$.
\end{enumerate} \hfill (GATE ST 2023)\\
\solution
%\input{gate/ST/2023/14/main.tex}
	\item  A die is loaded in such a way that each odd number is twice as likely to occur as
each even number. Find $P(G)$, where $G$ is the event that a number greater than
3 occurs on a single roll of the die.
\\
\solution
		%\input{exemplar/11/16/3/5/main.tex}
	\item All the jacks, queens and kings are removed from a deck of 52 playing cards. The remaining cards are well shuffled and then one card is drawn at random. Giving ace a value 1 similar value for other cards, find the probability that the card has a value 
		\begin{enumerate}
			\item 7
			\item greater than 7
			\item less than 7
		\end{enumerate}
		%\input{exemplar/10/13/3/30/main.tex}
  \item A Lot consists of 48 mobile phones of which 42 are good, 3 have only minor defects and 3 have major defects.Varnika will buy a phone if it is good but the trader will only buy a mobile if it has no major defects. One phone is selected at random from the lot. What is the probability that it is
\begin{enumerate}
	\item acceptable to Varnika?
            \item acceptable to the trader?
\end{enumerate}
\solution
	%\input{exemplar/10/13/3/40/main.tex}
 \item A student says that if you throw a die, it will show up 1 or not 1. Therefore, the probability of getting 1 and the probability of getting 'not 1' each is equal to $\frac{1}{2}$. Is this correct? Give reasons.\\
 \solution
        %\input{exemplar/10/13/2/9/main.tex}
   \item Four candidates A, B, C, D have ap-
plied for the assignment to coach a school cricket
team. If A is twice as likely to be selected as B, and
B and C are given about the same chance of being
selected, while C is twice as likely to be selected
as D, what are the probabilities that
\begin{enumerate}
\item C will be selected?
\item A will not be selected?
\end{enumerate}
	%\input{exemplar/11/16/3/9/main.tex}
 \item A bag contain 24 balls of which $x$ balls are red, $2x$ are white and $3x$ are blue. A ball is selected at random, What is the probability that it is
\begin{enumerate}[label=\alph*)]
\item not red ?
\item white ?
\end{enumerate}
%\input{exemplar/10/13/3/41/main.tex}
If the letters of the word ASSASSINATION are arranged at random. Find the Probability that
\begin{enumerate}[label=(\alph*)]
\item Four $S's$ come consecutively in the word
\item Two  $I's$ and two $N's$ come together
\item All $A's$ are not coming together
\item No two $A's$ are coming together
\end{enumerate}
%\input{exemplar/11/16/3/14/main.tex}
	\item One urn contains two black balls (labelled B1 and B2) and one white ball. A
	second urn contains one black ball and two white balls (labelled W1 and W2).
	Suppose the following experiment is performed. One of the two urns is chosen
	at random. Next a ball is randomly chosen from the urn. Then a second ball is
	chosen at random from the same urn without replacing the first ball.
	
	\begin{enumerate}
	\item What is the probability that two black balls are chosen?
	
	\item What is the probability that two balls of opposite colour are chosen?
	\end{enumerate}
	\solution
	%\input{exemplar/11/16/3/12/main1.tex}
\end{enumerate}

	\item A card is selected from a pack of 52 cards.
 \begin{enumerate}[label=(\alph*)] 
                 \item How many points are there in the sample space?
                 \item Calculate the probability that the card is an ace of spades.
                 \item Calculate the probability that the card is (i) an ace and (ii) black card.
 \end{enumerate}
\solution
		%\begin{table}[H]
	\centering
\begin{tabular}{|c|c|c|}
\hline
Random variable &Value &Definition\\ \hline
\multirow{3}{*}{X} &0 &Slips of Rs 1\\
&1 &Slips of Rs 5\\
&2 &Slips of Rs 13\\ \hline
\multirow{2}{*}{Y} &0 &Box A\\
&1 &Box B\\\hline
\end{tabular}
\caption{}
\label{tab:Distribution}
\end{table}
See \tabref{tab:Distribution}.
\begin{align}
p_{Y}\brak{k}= \begin{cases} 
      \frac{1}{3} & {k=0} \\
      \frac{2}{3 }& {k=1} 
   \end{cases}
   \\
p_{Y|X}\brak{0|0} = \frac{19}{25}\, 
p_{Y|X}\brak{0|1} = \frac{6}{25}\,
p_{Y|X}\brak{1|0} = \frac{45}{50}\,
p_{Y|X}\brak{1|2} = \frac{5}{50}
\end{align}
The desired probability is the probability that a slip drawn at random is marked other than Rs 1,
\begin{align}
&=1-p_X\brak{0}\\
&= p_X(1) + p_X(2)
\end{align}
Using Bayes theorem,
\begin{align}
&= p_Y\brak{0} \times \pr{Y=0 | X=1} + p_Y\brak{1} \times \pr{Y=1|X=2}\\
&=\frac{1}{3} \times \frac{6}{25} + \frac{2}{3} \times \frac{5}{50}\\
&=\frac{11}{75}
\end{align}

\newpage

%\tableofcontents

\bigskip

\renewcommand{\thefigure}{\theenumi}
\renewcommand{\thetable}{\theenumi}
%\renewcommand{\theequation}{\theenumi}

%\begin{abstract}
%%\boldmath
%In this letter, an algorithm for evaluating the exact analytical bit error rate  (BER)  for the piecewise linear (PL) combiner for  multiple relays is presented. Previous results were available only for upto three relays. The algorithm is unique in the sense that  the actual mathematical expressions, that are prohibitively large, need not be explicitly obtained. The diversity gain due to multiple relays is shown through plots of the analytical BER, well supported by simulations. 
%
%\end{abstract}
% IEEEtran.cls defaults to using nonbold math in the Abstract.
% This preserves the distinction between vectors and scalars. However,
% if the journal you are submitting to favors bold math in the abstract,
% then you can use LaTeX's standard command \boldmath at the very start
% of the abstract to achieve this. Many IEEE journals frown on math
% in the abstract anyway.

% Note that keywords are not normally used for peerreview papers.
%\begin{IEEEkeywords}
%Cooperative diversity, decode and forward, piecewise linear
%\end{IEEEkeywords}



% For peer review papers, you can put extra information on the cover
% page as needed:
% \ifCLASSOPTIONpeerreview
% \begin{center} \bfseries EDICS Category: 3-BBND \end{center}
% \fi
%
% For peerreview papers, this IEEEtran command inserts a page break and
% creates the second title. It will be ignored for other modes.
%\IEEEpeerreviewmaketitle




\item Four cards are drawn from a well-shuffled deck of 52 cards. What is the probability of obtaining 3 diamonds and one spade.
\\
\solution
		%\begin{enumerate}[label=\thesection.\arabic*,ref=\thesection.\theenumi]
	\item One card is drawn from a well-shuffled deck of 52 cards. Find the probability of getting
\begin{enumerate}
\item A king of red colour 
\item A face card 
\item A red face card
\item The jack of hearts
\item A spade
\item The queen of diamonds

\end{enumerate}
\solution
		%\input{ncert/10/15/1/14/main.tex}
	\item Five cards—the ten, jack, queen, king and ace of diamonds, are well-shuffled with their face downwards. One card is then picked up at random.
\begin{enumerate}
\item
What is the probability that the card is the queen? 
\item
If the queen is drawn and put aside, what is the probability that the second card picked up is (a) an ace? (b) a queen?\\
\end{enumerate}
\solution
		%\input{ncert/10/15/1/15/defs.tex}
	\item A bag contains $5$ red balls and some blue balls. If the probability of drawing a blue ball is double that if a red ball, determine the number of blue balls in the bag. 
		\\
\solution
		%\input{ncert/10/15/2/3/defs.tex}
	\item A card is selected from a pack of 52 cards.
 \begin{enumerate}[label=(\alph*)] 
                 \item How many points are there in the sample space?
                 \item Calculate the probability that the card is an ace of spades.
                 \item Calculate the probability that the card is (i) an ace and (ii) black card.
 \end{enumerate}
\solution
		%\input{ncert/11/16/3/4/main.tex}
\item Four cards are drawn from a well-shuffled deck of 52 cards. What is the probability of obtaining 3 diamonds and one spade.
\\
\solution
		%\input{ncert/11/16/4/2/defs.tex}
\item In a certain lottery 10,000 tickets are sold and ten equal prizes are awarded. What is the probability of not getting a prize if you buy (a) one ticket (b) two tickets (c) 10 tickets ?	
\\
\solution
		%\input{ncert/11/16/4/4/defs.tex}
		%
\item 
Out of 100 students, two sections of 40 and 60 are formed. If you and your friend are among the 100 students, what is the probability that
\begin{enumerate}
\item you both enter the same section?
\item you both enter the different sections?
\end{enumerate}
\solution
		%\input{ncert/11/16/4/5/defs.tex}
	\item 
The number lock of a suitcase has 4 wheels each labelled with ten digits i.e. from 0 to 9.The lock opens with a sequence of four digits with no repeats.What is the probability of a person getting the right sequence to open the suitcase.
\\
\solution
		%\input{ncert/11/16/4/10/defs.tex}
		%
\item 
Two cards are drawn at random and without replacement from a pack of 52 playing cards. Find the probability that both the cards are black.
\\
\solution
		%\input{ncert/12/13/2/2/defs.tex}
		\item A box of oranges is inspected by examining three randomly selected oranges drawn without replacement. If all the three oranges are good, the box is approved for sale, otherwise, it is rejected. Find the probability that a box containing 15 oranges out of which 12 are good and 3 are bad ones will be approved for sale.
		\label{ncert/12/13/2/3/defs.tex}
		\item Two balls are drawn at random with replacement from a box containing 10 black and 8 red balls. Find the probability that
		\label{ncert/12/13/2/12}
\begin{enumerate}
\item both balls are red.
\item first ball is black and second is red.
\item one of them is black and other is red.
\end{enumerate}

\item In a hostel, 60\% of the students read Hindi newspaper, 40\% read English newspaper and 20\% read both Hindi and English newspapers. A student is selected at random.
		\label{ncert/12/13/2/15}
\begin{enumerate}
\item Find the probability that she reads neither Hindi nor English newspapers.
\item If she reads Hindi newspaper, find the probability that she reads English newspaper.
\item If she reads English newspaper, find the probability that she reads Hindi newspaper.\\
\end{enumerate}
\item The probability of obtaining an even prime number on each die, when a pair of dice is rolled is 
\begin{enumerate}
    \item $0$ 
    
    \item $\frac{1}{3}$ 
    
    \item $\frac{1}{12}$ 
    
    \item $\frac{1}{36}$ 
\end{enumerate}
\solution
		%\input{ncert/12/13/2/17/defs.tex}
	\item A bag contains 4 red and 4 black balls, another bag contains 2 red and 6 black balls. One of the two bags is selected at random and a ball is drawn from the bag which is found to be red. Find the probability that the ball is drawn from the first bag.
\\
\solution
		%\input{ncert/12/13/3/2/main.tex}
  \item
  Cards with numbers 2 to 101 are placed in a box. A card is selected at random.Find the probability that the card has
\begin{enumerate}[label=(\roman*)]
	\item an even number 
	\item a square number
\end{enumerate}
\solution
%\input{exemplar/10/13/3/32/main.tex}
\item
The king, queen and jack of clubs are removed from a deck of 52 playing cards and then well shuffled. Now one card is drawn at random from the remaining cards.  Determine the probability that the card is
\begin{enumerate}[label=(\roman*)]
\item a club
\item 10 of hearts
\end{enumerate}
\solution
%\input{exemplar/10/13/3/29/main.tex}
\item A team of medical students doing their internship have to assist during surgeries
at a city hospital. The probabilities of surgeries rated as very complex, complex,
routine, simple or very simple are respectively, 0.15, 0.20, 0.31, 0.26, .08. Find
the probabilities that a particular surgery will be rated
\begin{enumerate}
	\item complex or very complex;
	\item neither very complex nor very simple;
	\item routine or complex
	\item routine or simple
\end{enumerate}
\solution
%\input{exemplar/11/16/3/8(1)/main.tex}
\item A card is selected from a pack of 52 cards.
\begin{enumerate}[label=(\alph*)]
    \item How many points are there in the sample space?
    \item Calculate the probability that the card is an ace of spades.
    \item Calculate the probability that the card is (i) an ace and (ii) black card.
\end{enumerate}
\solution
%\input{exemplar/11/16/3/4/main2.tex}
\item The probability that a non leap year selected at random will contain 53 sundays.
\\
\solution
%\input{exemplar/10/13/1/19/main.tex}
\item One of the four persons John, Rita, Aslam or Gurpreet will be promoted next
month. Consequently the sample space consists of four elementary outcomes
S = {John promoted, Rita promoted, Aslam promoted, Gurpreet promoted}
You are told that the chances of John’s promotion is same as that of Gurpreet,
Rita’s chances of promotion are twice as likely as Johns. Aslam’s chances are
four times that of John.
\begin{enumerate}
	\item Determine
	\begin{enumerate}
		\item P (John promoted)
		\item P (Rita promoted)
		\item P (Aslam promoted)
		\item P (Gurpreet promoted)
	\end{enumerate}
	\item If A = {John promoted or Gurpreet promoted}, find P (A).
\end{enumerate}
\solution
%\input{exemplar/11/16/3/10/main.tex}
\item A card is drawn from a deck of 52 cards. Find the probability of getting a king or a heart or a red card.\\
\solution
%\input{exemplar/11/16/3/15/main.tex}
\item The probability that a student will pass his examination is 0.73, the probability of
the student getting a compartment is 0.13, and the probability that the student will
either pass or get compartment is 0.96. State True or False.\\
\solution
%\input{exemplar/11/16/3/31/main.tex}
\item A card is selected from a pack of 52 cards\\
\begin{enumerate}[label=(\alph*)]
\item How many points are there in the sample space?
\item Calculate the probability that the cards is an ace of spades.
\item Calculate the probability that the card is (i) an ace (ii)black card.\\
\end{enumerate}
%\input{ncert/11/16/3/4_1/Prob_4.tex}
\item In a non-leap year, the probability of having 53 tuesdays or 53 wednesdays is\\
\solution
%\input{exemplar/11/16/3/18/main.tex}
\item There are 1000 sealed envelopes in a box, 10 of them contain a cash prize of
Rs 100 each, 100 of them contain a cash prize of Rs 50 each and 200 of them
contain a cash prize of Rs 10 each and rest do not contain any cash prize. If they
are well shuffled and an envelope is picked up out, what is the probability that it
contains no cash prize?\\
\solution
%\input{exemplar/10/13/3/34/main.tex}
\item 
A die is thrown and a card is selected at random from a deck of 52 playing cards. The probability of getting an even number on the die and a spade card.\\
\solution
%\input{exemplar/12/13/3/78/main.tex}
\item
If 4-digit numbers greater than 5,000 are randomly formed from the digits 0, 1, 3, 5, and 7, what is the probability of forming a number divisible by 5 when:
\begin{enumerate}
    \item The digits are repeated?
    \item The repetition of digits is not allowed?
\end{enumerate}
\solution
%\input{ncert/11/16/4/9/main.tex}
\item Consider the probability space $\brak{\Omega, \mathcal{G}, P}$ where $\Omega = [0,2]$ and $\mathcal{G} = \cbrak{\phi, \Omega, [0,1], (1,2]}$. Let $X$ and $Y$ be two functions on $\Omega$ defined as
\begin{align*}
    X(\omega) = 
    \begin{cases}
        1 & \text{if }\omega \in [0, 1]\\
        2 & \text{if }\omega \in (1, 2]
    \end{cases}
\end{align*}
and
\begin{align*}
    Y(\omega) = 
    \begin{cases}
        2 & \text{if }\omega \in [0, 1.5]\\
        3 & \text{if }\omega \in (1.5, 2].
    \end{cases}
\end{align*}
Then which one of the following statements is true?
\begin{enumerate}
    \item [(A)] $X$ is a random variable with respect to $\mathcal{G}$, but $Y$ is not a random variable with respect to $\mathcal{G}$.
    \item [(B)] $Y$ is a random variable with respect to $\mathcal{G}$, but $X$ is not a random variable with respect to $\mathcal{G}$.
    \item [(C)] Neither $X$ nor $Y$ is a random variable with respect to $\mathcal{G}$.
    \item [(D)] Both $X$ and $Y$ are random variables with respect to $\mathcal{G}$.
\end{enumerate} \hfill (GATE ST 2023)\\
\solution
%\input{gate/ST/2023/14/main.tex}
	\item  A die is loaded in such a way that each odd number is twice as likely to occur as
each even number. Find $P(G)$, where $G$ is the event that a number greater than
3 occurs on a single roll of the die.
\\
\solution
		%\input{exemplar/11/16/3/5/main.tex}
	\item All the jacks, queens and kings are removed from a deck of 52 playing cards. The remaining cards are well shuffled and then one card is drawn at random. Giving ace a value 1 similar value for other cards, find the probability that the card has a value 
		\begin{enumerate}
			\item 7
			\item greater than 7
			\item less than 7
		\end{enumerate}
		%\input{exemplar/10/13/3/30/main.tex}
  \item A Lot consists of 48 mobile phones of which 42 are good, 3 have only minor defects and 3 have major defects.Varnika will buy a phone if it is good but the trader will only buy a mobile if it has no major defects. One phone is selected at random from the lot. What is the probability that it is
\begin{enumerate}
	\item acceptable to Varnika?
            \item acceptable to the trader?
\end{enumerate}
\solution
	%\input{exemplar/10/13/3/40/main.tex}
 \item A student says that if you throw a die, it will show up 1 or not 1. Therefore, the probability of getting 1 and the probability of getting 'not 1' each is equal to $\frac{1}{2}$. Is this correct? Give reasons.\\
 \solution
        %\input{exemplar/10/13/2/9/main.tex}
   \item Four candidates A, B, C, D have ap-
plied for the assignment to coach a school cricket
team. If A is twice as likely to be selected as B, and
B and C are given about the same chance of being
selected, while C is twice as likely to be selected
as D, what are the probabilities that
\begin{enumerate}
\item C will be selected?
\item A will not be selected?
\end{enumerate}
	%\input{exemplar/11/16/3/9/main.tex}
 \item A bag contain 24 balls of which $x$ balls are red, $2x$ are white and $3x$ are blue. A ball is selected at random, What is the probability that it is
\begin{enumerate}[label=\alph*)]
\item not red ?
\item white ?
\end{enumerate}
%\input{exemplar/10/13/3/41/main.tex}
If the letters of the word ASSASSINATION are arranged at random. Find the Probability that
\begin{enumerate}[label=(\alph*)]
\item Four $S's$ come consecutively in the word
\item Two  $I's$ and two $N's$ come together
\item All $A's$ are not coming together
\item No two $A's$ are coming together
\end{enumerate}
%\input{exemplar/11/16/3/14/main.tex}
	\item One urn contains two black balls (labelled B1 and B2) and one white ball. A
	second urn contains one black ball and two white balls (labelled W1 and W2).
	Suppose the following experiment is performed. One of the two urns is chosen
	at random. Next a ball is randomly chosen from the urn. Then a second ball is
	chosen at random from the same urn without replacing the first ball.
	
	\begin{enumerate}
	\item What is the probability that two black balls are chosen?
	
	\item What is the probability that two balls of opposite colour are chosen?
	\end{enumerate}
	\solution
	%\input{exemplar/11/16/3/12/main1.tex}
\end{enumerate}

\item In a certain lottery 10,000 tickets are sold and ten equal prizes are awarded. What is the probability of not getting a prize if you buy (a) one ticket (b) two tickets (c) 10 tickets ?	
\\
\solution
		%\begin{enumerate}[label=\thesection.\arabic*,ref=\thesection.\theenumi]
	\item One card is drawn from a well-shuffled deck of 52 cards. Find the probability of getting
\begin{enumerate}
\item A king of red colour 
\item A face card 
\item A red face card
\item The jack of hearts
\item A spade
\item The queen of diamonds

\end{enumerate}
\solution
		%\input{ncert/10/15/1/14/main.tex}
	\item Five cards—the ten, jack, queen, king and ace of diamonds, are well-shuffled with their face downwards. One card is then picked up at random.
\begin{enumerate}
\item
What is the probability that the card is the queen? 
\item
If the queen is drawn and put aside, what is the probability that the second card picked up is (a) an ace? (b) a queen?\\
\end{enumerate}
\solution
		%\input{ncert/10/15/1/15/defs.tex}
	\item A bag contains $5$ red balls and some blue balls. If the probability of drawing a blue ball is double that if a red ball, determine the number of blue balls in the bag. 
		\\
\solution
		%\input{ncert/10/15/2/3/defs.tex}
	\item A card is selected from a pack of 52 cards.
 \begin{enumerate}[label=(\alph*)] 
                 \item How many points are there in the sample space?
                 \item Calculate the probability that the card is an ace of spades.
                 \item Calculate the probability that the card is (i) an ace and (ii) black card.
 \end{enumerate}
\solution
		%\input{ncert/11/16/3/4/main.tex}
\item Four cards are drawn from a well-shuffled deck of 52 cards. What is the probability of obtaining 3 diamonds and one spade.
\\
\solution
		%\input{ncert/11/16/4/2/defs.tex}
\item In a certain lottery 10,000 tickets are sold and ten equal prizes are awarded. What is the probability of not getting a prize if you buy (a) one ticket (b) two tickets (c) 10 tickets ?	
\\
\solution
		%\input{ncert/11/16/4/4/defs.tex}
		%
\item 
Out of 100 students, two sections of 40 and 60 are formed. If you and your friend are among the 100 students, what is the probability that
\begin{enumerate}
\item you both enter the same section?
\item you both enter the different sections?
\end{enumerate}
\solution
		%\input{ncert/11/16/4/5/defs.tex}
	\item 
The number lock of a suitcase has 4 wheels each labelled with ten digits i.e. from 0 to 9.The lock opens with a sequence of four digits with no repeats.What is the probability of a person getting the right sequence to open the suitcase.
\\
\solution
		%\input{ncert/11/16/4/10/defs.tex}
		%
\item 
Two cards are drawn at random and without replacement from a pack of 52 playing cards. Find the probability that both the cards are black.
\\
\solution
		%\input{ncert/12/13/2/2/defs.tex}
		\item A box of oranges is inspected by examining three randomly selected oranges drawn without replacement. If all the three oranges are good, the box is approved for sale, otherwise, it is rejected. Find the probability that a box containing 15 oranges out of which 12 are good and 3 are bad ones will be approved for sale.
		\label{ncert/12/13/2/3/defs.tex}
		\item Two balls are drawn at random with replacement from a box containing 10 black and 8 red balls. Find the probability that
		\label{ncert/12/13/2/12}
\begin{enumerate}
\item both balls are red.
\item first ball is black and second is red.
\item one of them is black and other is red.
\end{enumerate}

\item In a hostel, 60\% of the students read Hindi newspaper, 40\% read English newspaper and 20\% read both Hindi and English newspapers. A student is selected at random.
		\label{ncert/12/13/2/15}
\begin{enumerate}
\item Find the probability that she reads neither Hindi nor English newspapers.
\item If she reads Hindi newspaper, find the probability that she reads English newspaper.
\item If she reads English newspaper, find the probability that she reads Hindi newspaper.\\
\end{enumerate}
\item The probability of obtaining an even prime number on each die, when a pair of dice is rolled is 
\begin{enumerate}
    \item $0$ 
    
    \item $\frac{1}{3}$ 
    
    \item $\frac{1}{12}$ 
    
    \item $\frac{1}{36}$ 
\end{enumerate}
\solution
		%\input{ncert/12/13/2/17/defs.tex}
	\item A bag contains 4 red and 4 black balls, another bag contains 2 red and 6 black balls. One of the two bags is selected at random and a ball is drawn from the bag which is found to be red. Find the probability that the ball is drawn from the first bag.
\\
\solution
		%\input{ncert/12/13/3/2/main.tex}
  \item
  Cards with numbers 2 to 101 are placed in a box. A card is selected at random.Find the probability that the card has
\begin{enumerate}[label=(\roman*)]
	\item an even number 
	\item a square number
\end{enumerate}
\solution
%\input{exemplar/10/13/3/32/main.tex}
\item
The king, queen and jack of clubs are removed from a deck of 52 playing cards and then well shuffled. Now one card is drawn at random from the remaining cards.  Determine the probability that the card is
\begin{enumerate}[label=(\roman*)]
\item a club
\item 10 of hearts
\end{enumerate}
\solution
%\input{exemplar/10/13/3/29/main.tex}
\item A team of medical students doing their internship have to assist during surgeries
at a city hospital. The probabilities of surgeries rated as very complex, complex,
routine, simple or very simple are respectively, 0.15, 0.20, 0.31, 0.26, .08. Find
the probabilities that a particular surgery will be rated
\begin{enumerate}
	\item complex or very complex;
	\item neither very complex nor very simple;
	\item routine or complex
	\item routine or simple
\end{enumerate}
\solution
%\input{exemplar/11/16/3/8(1)/main.tex}
\item A card is selected from a pack of 52 cards.
\begin{enumerate}[label=(\alph*)]
    \item How many points are there in the sample space?
    \item Calculate the probability that the card is an ace of spades.
    \item Calculate the probability that the card is (i) an ace and (ii) black card.
\end{enumerate}
\solution
%\input{exemplar/11/16/3/4/main2.tex}
\item The probability that a non leap year selected at random will contain 53 sundays.
\\
\solution
%\input{exemplar/10/13/1/19/main.tex}
\item One of the four persons John, Rita, Aslam or Gurpreet will be promoted next
month. Consequently the sample space consists of four elementary outcomes
S = {John promoted, Rita promoted, Aslam promoted, Gurpreet promoted}
You are told that the chances of John’s promotion is same as that of Gurpreet,
Rita’s chances of promotion are twice as likely as Johns. Aslam’s chances are
four times that of John.
\begin{enumerate}
	\item Determine
	\begin{enumerate}
		\item P (John promoted)
		\item P (Rita promoted)
		\item P (Aslam promoted)
		\item P (Gurpreet promoted)
	\end{enumerate}
	\item If A = {John promoted or Gurpreet promoted}, find P (A).
\end{enumerate}
\solution
%\input{exemplar/11/16/3/10/main.tex}
\item A card is drawn from a deck of 52 cards. Find the probability of getting a king or a heart or a red card.\\
\solution
%\input{exemplar/11/16/3/15/main.tex}
\item The probability that a student will pass his examination is 0.73, the probability of
the student getting a compartment is 0.13, and the probability that the student will
either pass or get compartment is 0.96. State True or False.\\
\solution
%\input{exemplar/11/16/3/31/main.tex}
\item A card is selected from a pack of 52 cards\\
\begin{enumerate}[label=(\alph*)]
\item How many points are there in the sample space?
\item Calculate the probability that the cards is an ace of spades.
\item Calculate the probability that the card is (i) an ace (ii)black card.\\
\end{enumerate}
%\input{ncert/11/16/3/4_1/Prob_4.tex}
\item In a non-leap year, the probability of having 53 tuesdays or 53 wednesdays is\\
\solution
%\input{exemplar/11/16/3/18/main.tex}
\item There are 1000 sealed envelopes in a box, 10 of them contain a cash prize of
Rs 100 each, 100 of them contain a cash prize of Rs 50 each and 200 of them
contain a cash prize of Rs 10 each and rest do not contain any cash prize. If they
are well shuffled and an envelope is picked up out, what is the probability that it
contains no cash prize?\\
\solution
%\input{exemplar/10/13/3/34/main.tex}
\item 
A die is thrown and a card is selected at random from a deck of 52 playing cards. The probability of getting an even number on the die and a spade card.\\
\solution
%\input{exemplar/12/13/3/78/main.tex}
\item
If 4-digit numbers greater than 5,000 are randomly formed from the digits 0, 1, 3, 5, and 7, what is the probability of forming a number divisible by 5 when:
\begin{enumerate}
    \item The digits are repeated?
    \item The repetition of digits is not allowed?
\end{enumerate}
\solution
%\input{ncert/11/16/4/9/main.tex}
\item Consider the probability space $\brak{\Omega, \mathcal{G}, P}$ where $\Omega = [0,2]$ and $\mathcal{G} = \cbrak{\phi, \Omega, [0,1], (1,2]}$. Let $X$ and $Y$ be two functions on $\Omega$ defined as
\begin{align*}
    X(\omega) = 
    \begin{cases}
        1 & \text{if }\omega \in [0, 1]\\
        2 & \text{if }\omega \in (1, 2]
    \end{cases}
\end{align*}
and
\begin{align*}
    Y(\omega) = 
    \begin{cases}
        2 & \text{if }\omega \in [0, 1.5]\\
        3 & \text{if }\omega \in (1.5, 2].
    \end{cases}
\end{align*}
Then which one of the following statements is true?
\begin{enumerate}
    \item [(A)] $X$ is a random variable with respect to $\mathcal{G}$, but $Y$ is not a random variable with respect to $\mathcal{G}$.
    \item [(B)] $Y$ is a random variable with respect to $\mathcal{G}$, but $X$ is not a random variable with respect to $\mathcal{G}$.
    \item [(C)] Neither $X$ nor $Y$ is a random variable with respect to $\mathcal{G}$.
    \item [(D)] Both $X$ and $Y$ are random variables with respect to $\mathcal{G}$.
\end{enumerate} \hfill (GATE ST 2023)\\
\solution
%\input{gate/ST/2023/14/main.tex}
	\item  A die is loaded in such a way that each odd number is twice as likely to occur as
each even number. Find $P(G)$, where $G$ is the event that a number greater than
3 occurs on a single roll of the die.
\\
\solution
		%\input{exemplar/11/16/3/5/main.tex}
	\item All the jacks, queens and kings are removed from a deck of 52 playing cards. The remaining cards are well shuffled and then one card is drawn at random. Giving ace a value 1 similar value for other cards, find the probability that the card has a value 
		\begin{enumerate}
			\item 7
			\item greater than 7
			\item less than 7
		\end{enumerate}
		%\input{exemplar/10/13/3/30/main.tex}
  \item A Lot consists of 48 mobile phones of which 42 are good, 3 have only minor defects and 3 have major defects.Varnika will buy a phone if it is good but the trader will only buy a mobile if it has no major defects. One phone is selected at random from the lot. What is the probability that it is
\begin{enumerate}
	\item acceptable to Varnika?
            \item acceptable to the trader?
\end{enumerate}
\solution
	%\input{exemplar/10/13/3/40/main.tex}
 \item A student says that if you throw a die, it will show up 1 or not 1. Therefore, the probability of getting 1 and the probability of getting 'not 1' each is equal to $\frac{1}{2}$. Is this correct? Give reasons.\\
 \solution
        %\input{exemplar/10/13/2/9/main.tex}
   \item Four candidates A, B, C, D have ap-
plied for the assignment to coach a school cricket
team. If A is twice as likely to be selected as B, and
B and C are given about the same chance of being
selected, while C is twice as likely to be selected
as D, what are the probabilities that
\begin{enumerate}
\item C will be selected?
\item A will not be selected?
\end{enumerate}
	%\input{exemplar/11/16/3/9/main.tex}
 \item A bag contain 24 balls of which $x$ balls are red, $2x$ are white and $3x$ are blue. A ball is selected at random, What is the probability that it is
\begin{enumerate}[label=\alph*)]
\item not red ?
\item white ?
\end{enumerate}
%\input{exemplar/10/13/3/41/main.tex}
If the letters of the word ASSASSINATION are arranged at random. Find the Probability that
\begin{enumerate}[label=(\alph*)]
\item Four $S's$ come consecutively in the word
\item Two  $I's$ and two $N's$ come together
\item All $A's$ are not coming together
\item No two $A's$ are coming together
\end{enumerate}
%\input{exemplar/11/16/3/14/main.tex}
	\item One urn contains two black balls (labelled B1 and B2) and one white ball. A
	second urn contains one black ball and two white balls (labelled W1 and W2).
	Suppose the following experiment is performed. One of the two urns is chosen
	at random. Next a ball is randomly chosen from the urn. Then a second ball is
	chosen at random from the same urn without replacing the first ball.
	
	\begin{enumerate}
	\item What is the probability that two black balls are chosen?
	
	\item What is the probability that two balls of opposite colour are chosen?
	\end{enumerate}
	\solution
	%\input{exemplar/11/16/3/12/main1.tex}
\end{enumerate}

		%
\item 
Out of 100 students, two sections of 40 and 60 are formed. If you and your friend are among the 100 students, what is the probability that
\begin{enumerate}
\item you both enter the same section?
\item you both enter the different sections?
\end{enumerate}
\solution
		%\begin{enumerate}[label=\thesection.\arabic*,ref=\thesection.\theenumi]
	\item One card is drawn from a well-shuffled deck of 52 cards. Find the probability of getting
\begin{enumerate}
\item A king of red colour 
\item A face card 
\item A red face card
\item The jack of hearts
\item A spade
\item The queen of diamonds

\end{enumerate}
\solution
		%\input{ncert/10/15/1/14/main.tex}
	\item Five cards—the ten, jack, queen, king and ace of diamonds, are well-shuffled with their face downwards. One card is then picked up at random.
\begin{enumerate}
\item
What is the probability that the card is the queen? 
\item
If the queen is drawn and put aside, what is the probability that the second card picked up is (a) an ace? (b) a queen?\\
\end{enumerate}
\solution
		%\input{ncert/10/15/1/15/defs.tex}
	\item A bag contains $5$ red balls and some blue balls. If the probability of drawing a blue ball is double that if a red ball, determine the number of blue balls in the bag. 
		\\
\solution
		%\input{ncert/10/15/2/3/defs.tex}
	\item A card is selected from a pack of 52 cards.
 \begin{enumerate}[label=(\alph*)] 
                 \item How many points are there in the sample space?
                 \item Calculate the probability that the card is an ace of spades.
                 \item Calculate the probability that the card is (i) an ace and (ii) black card.
 \end{enumerate}
\solution
		%\input{ncert/11/16/3/4/main.tex}
\item Four cards are drawn from a well-shuffled deck of 52 cards. What is the probability of obtaining 3 diamonds and one spade.
\\
\solution
		%\input{ncert/11/16/4/2/defs.tex}
\item In a certain lottery 10,000 tickets are sold and ten equal prizes are awarded. What is the probability of not getting a prize if you buy (a) one ticket (b) two tickets (c) 10 tickets ?	
\\
\solution
		%\input{ncert/11/16/4/4/defs.tex}
		%
\item 
Out of 100 students, two sections of 40 and 60 are formed. If you and your friend are among the 100 students, what is the probability that
\begin{enumerate}
\item you both enter the same section?
\item you both enter the different sections?
\end{enumerate}
\solution
		%\input{ncert/11/16/4/5/defs.tex}
	\item 
The number lock of a suitcase has 4 wheels each labelled with ten digits i.e. from 0 to 9.The lock opens with a sequence of four digits with no repeats.What is the probability of a person getting the right sequence to open the suitcase.
\\
\solution
		%\input{ncert/11/16/4/10/defs.tex}
		%
\item 
Two cards are drawn at random and without replacement from a pack of 52 playing cards. Find the probability that both the cards are black.
\\
\solution
		%\input{ncert/12/13/2/2/defs.tex}
		\item A box of oranges is inspected by examining three randomly selected oranges drawn without replacement. If all the three oranges are good, the box is approved for sale, otherwise, it is rejected. Find the probability that a box containing 15 oranges out of which 12 are good and 3 are bad ones will be approved for sale.
		\label{ncert/12/13/2/3/defs.tex}
		\item Two balls are drawn at random with replacement from a box containing 10 black and 8 red balls. Find the probability that
		\label{ncert/12/13/2/12}
\begin{enumerate}
\item both balls are red.
\item first ball is black and second is red.
\item one of them is black and other is red.
\end{enumerate}

\item In a hostel, 60\% of the students read Hindi newspaper, 40\% read English newspaper and 20\% read both Hindi and English newspapers. A student is selected at random.
		\label{ncert/12/13/2/15}
\begin{enumerate}
\item Find the probability that she reads neither Hindi nor English newspapers.
\item If she reads Hindi newspaper, find the probability that she reads English newspaper.
\item If she reads English newspaper, find the probability that she reads Hindi newspaper.\\
\end{enumerate}
\item The probability of obtaining an even prime number on each die, when a pair of dice is rolled is 
\begin{enumerate}
    \item $0$ 
    
    \item $\frac{1}{3}$ 
    
    \item $\frac{1}{12}$ 
    
    \item $\frac{1}{36}$ 
\end{enumerate}
\solution
		%\input{ncert/12/13/2/17/defs.tex}
	\item A bag contains 4 red and 4 black balls, another bag contains 2 red and 6 black balls. One of the two bags is selected at random and a ball is drawn from the bag which is found to be red. Find the probability that the ball is drawn from the first bag.
\\
\solution
		%\input{ncert/12/13/3/2/main.tex}
  \item
  Cards with numbers 2 to 101 are placed in a box. A card is selected at random.Find the probability that the card has
\begin{enumerate}[label=(\roman*)]
	\item an even number 
	\item a square number
\end{enumerate}
\solution
%\input{exemplar/10/13/3/32/main.tex}
\item
The king, queen and jack of clubs are removed from a deck of 52 playing cards and then well shuffled. Now one card is drawn at random from the remaining cards.  Determine the probability that the card is
\begin{enumerate}[label=(\roman*)]
\item a club
\item 10 of hearts
\end{enumerate}
\solution
%\input{exemplar/10/13/3/29/main.tex}
\item A team of medical students doing their internship have to assist during surgeries
at a city hospital. The probabilities of surgeries rated as very complex, complex,
routine, simple or very simple are respectively, 0.15, 0.20, 0.31, 0.26, .08. Find
the probabilities that a particular surgery will be rated
\begin{enumerate}
	\item complex or very complex;
	\item neither very complex nor very simple;
	\item routine or complex
	\item routine or simple
\end{enumerate}
\solution
%\input{exemplar/11/16/3/8(1)/main.tex}
\item A card is selected from a pack of 52 cards.
\begin{enumerate}[label=(\alph*)]
    \item How many points are there in the sample space?
    \item Calculate the probability that the card is an ace of spades.
    \item Calculate the probability that the card is (i) an ace and (ii) black card.
\end{enumerate}
\solution
%\input{exemplar/11/16/3/4/main2.tex}
\item The probability that a non leap year selected at random will contain 53 sundays.
\\
\solution
%\input{exemplar/10/13/1/19/main.tex}
\item One of the four persons John, Rita, Aslam or Gurpreet will be promoted next
month. Consequently the sample space consists of four elementary outcomes
S = {John promoted, Rita promoted, Aslam promoted, Gurpreet promoted}
You are told that the chances of John’s promotion is same as that of Gurpreet,
Rita’s chances of promotion are twice as likely as Johns. Aslam’s chances are
four times that of John.
\begin{enumerate}
	\item Determine
	\begin{enumerate}
		\item P (John promoted)
		\item P (Rita promoted)
		\item P (Aslam promoted)
		\item P (Gurpreet promoted)
	\end{enumerate}
	\item If A = {John promoted or Gurpreet promoted}, find P (A).
\end{enumerate}
\solution
%\input{exemplar/11/16/3/10/main.tex}
\item A card is drawn from a deck of 52 cards. Find the probability of getting a king or a heart or a red card.\\
\solution
%\input{exemplar/11/16/3/15/main.tex}
\item The probability that a student will pass his examination is 0.73, the probability of
the student getting a compartment is 0.13, and the probability that the student will
either pass or get compartment is 0.96. State True or False.\\
\solution
%\input{exemplar/11/16/3/31/main.tex}
\item A card is selected from a pack of 52 cards\\
\begin{enumerate}[label=(\alph*)]
\item How many points are there in the sample space?
\item Calculate the probability that the cards is an ace of spades.
\item Calculate the probability that the card is (i) an ace (ii)black card.\\
\end{enumerate}
%\input{ncert/11/16/3/4_1/Prob_4.tex}
\item In a non-leap year, the probability of having 53 tuesdays or 53 wednesdays is\\
\solution
%\input{exemplar/11/16/3/18/main.tex}
\item There are 1000 sealed envelopes in a box, 10 of them contain a cash prize of
Rs 100 each, 100 of them contain a cash prize of Rs 50 each and 200 of them
contain a cash prize of Rs 10 each and rest do not contain any cash prize. If they
are well shuffled and an envelope is picked up out, what is the probability that it
contains no cash prize?\\
\solution
%\input{exemplar/10/13/3/34/main.tex}
\item 
A die is thrown and a card is selected at random from a deck of 52 playing cards. The probability of getting an even number on the die and a spade card.\\
\solution
%\input{exemplar/12/13/3/78/main.tex}
\item
If 4-digit numbers greater than 5,000 are randomly formed from the digits 0, 1, 3, 5, and 7, what is the probability of forming a number divisible by 5 when:
\begin{enumerate}
    \item The digits are repeated?
    \item The repetition of digits is not allowed?
\end{enumerate}
\solution
%\input{ncert/11/16/4/9/main.tex}
\item Consider the probability space $\brak{\Omega, \mathcal{G}, P}$ where $\Omega = [0,2]$ and $\mathcal{G} = \cbrak{\phi, \Omega, [0,1], (1,2]}$. Let $X$ and $Y$ be two functions on $\Omega$ defined as
\begin{align*}
    X(\omega) = 
    \begin{cases}
        1 & \text{if }\omega \in [0, 1]\\
        2 & \text{if }\omega \in (1, 2]
    \end{cases}
\end{align*}
and
\begin{align*}
    Y(\omega) = 
    \begin{cases}
        2 & \text{if }\omega \in [0, 1.5]\\
        3 & \text{if }\omega \in (1.5, 2].
    \end{cases}
\end{align*}
Then which one of the following statements is true?
\begin{enumerate}
    \item [(A)] $X$ is a random variable with respect to $\mathcal{G}$, but $Y$ is not a random variable with respect to $\mathcal{G}$.
    \item [(B)] $Y$ is a random variable with respect to $\mathcal{G}$, but $X$ is not a random variable with respect to $\mathcal{G}$.
    \item [(C)] Neither $X$ nor $Y$ is a random variable with respect to $\mathcal{G}$.
    \item [(D)] Both $X$ and $Y$ are random variables with respect to $\mathcal{G}$.
\end{enumerate} \hfill (GATE ST 2023)\\
\solution
%\input{gate/ST/2023/14/main.tex}
	\item  A die is loaded in such a way that each odd number is twice as likely to occur as
each even number. Find $P(G)$, where $G$ is the event that a number greater than
3 occurs on a single roll of the die.
\\
\solution
		%\input{exemplar/11/16/3/5/main.tex}
	\item All the jacks, queens and kings are removed from a deck of 52 playing cards. The remaining cards are well shuffled and then one card is drawn at random. Giving ace a value 1 similar value for other cards, find the probability that the card has a value 
		\begin{enumerate}
			\item 7
			\item greater than 7
			\item less than 7
		\end{enumerate}
		%\input{exemplar/10/13/3/30/main.tex}
  \item A Lot consists of 48 mobile phones of which 42 are good, 3 have only minor defects and 3 have major defects.Varnika will buy a phone if it is good but the trader will only buy a mobile if it has no major defects. One phone is selected at random from the lot. What is the probability that it is
\begin{enumerate}
	\item acceptable to Varnika?
            \item acceptable to the trader?
\end{enumerate}
\solution
	%\input{exemplar/10/13/3/40/main.tex}
 \item A student says that if you throw a die, it will show up 1 or not 1. Therefore, the probability of getting 1 and the probability of getting 'not 1' each is equal to $\frac{1}{2}$. Is this correct? Give reasons.\\
 \solution
        %\input{exemplar/10/13/2/9/main.tex}
   \item Four candidates A, B, C, D have ap-
plied for the assignment to coach a school cricket
team. If A is twice as likely to be selected as B, and
B and C are given about the same chance of being
selected, while C is twice as likely to be selected
as D, what are the probabilities that
\begin{enumerate}
\item C will be selected?
\item A will not be selected?
\end{enumerate}
	%\input{exemplar/11/16/3/9/main.tex}
 \item A bag contain 24 balls of which $x$ balls are red, $2x$ are white and $3x$ are blue. A ball is selected at random, What is the probability that it is
\begin{enumerate}[label=\alph*)]
\item not red ?
\item white ?
\end{enumerate}
%\input{exemplar/10/13/3/41/main.tex}
If the letters of the word ASSASSINATION are arranged at random. Find the Probability that
\begin{enumerate}[label=(\alph*)]
\item Four $S's$ come consecutively in the word
\item Two  $I's$ and two $N's$ come together
\item All $A's$ are not coming together
\item No two $A's$ are coming together
\end{enumerate}
%\input{exemplar/11/16/3/14/main.tex}
	\item One urn contains two black balls (labelled B1 and B2) and one white ball. A
	second urn contains one black ball and two white balls (labelled W1 and W2).
	Suppose the following experiment is performed. One of the two urns is chosen
	at random. Next a ball is randomly chosen from the urn. Then a second ball is
	chosen at random from the same urn without replacing the first ball.
	
	\begin{enumerate}
	\item What is the probability that two black balls are chosen?
	
	\item What is the probability that two balls of opposite colour are chosen?
	\end{enumerate}
	\solution
	%\input{exemplar/11/16/3/12/main1.tex}
\end{enumerate}

	\item 
The number lock of a suitcase has 4 wheels each labelled with ten digits i.e. from 0 to 9.The lock opens with a sequence of four digits with no repeats.What is the probability of a person getting the right sequence to open the suitcase.
\\
\solution
		%\begin{enumerate}[label=\thesection.\arabic*,ref=\thesection.\theenumi]
	\item One card is drawn from a well-shuffled deck of 52 cards. Find the probability of getting
\begin{enumerate}
\item A king of red colour 
\item A face card 
\item A red face card
\item The jack of hearts
\item A spade
\item The queen of diamonds

\end{enumerate}
\solution
		%\input{ncert/10/15/1/14/main.tex}
	\item Five cards—the ten, jack, queen, king and ace of diamonds, are well-shuffled with their face downwards. One card is then picked up at random.
\begin{enumerate}
\item
What is the probability that the card is the queen? 
\item
If the queen is drawn and put aside, what is the probability that the second card picked up is (a) an ace? (b) a queen?\\
\end{enumerate}
\solution
		%\input{ncert/10/15/1/15/defs.tex}
	\item A bag contains $5$ red balls and some blue balls. If the probability of drawing a blue ball is double that if a red ball, determine the number of blue balls in the bag. 
		\\
\solution
		%\input{ncert/10/15/2/3/defs.tex}
	\item A card is selected from a pack of 52 cards.
 \begin{enumerate}[label=(\alph*)] 
                 \item How many points are there in the sample space?
                 \item Calculate the probability that the card is an ace of spades.
                 \item Calculate the probability that the card is (i) an ace and (ii) black card.
 \end{enumerate}
\solution
		%\input{ncert/11/16/3/4/main.tex}
\item Four cards are drawn from a well-shuffled deck of 52 cards. What is the probability of obtaining 3 diamonds and one spade.
\\
\solution
		%\input{ncert/11/16/4/2/defs.tex}
\item In a certain lottery 10,000 tickets are sold and ten equal prizes are awarded. What is the probability of not getting a prize if you buy (a) one ticket (b) two tickets (c) 10 tickets ?	
\\
\solution
		%\input{ncert/11/16/4/4/defs.tex}
		%
\item 
Out of 100 students, two sections of 40 and 60 are formed. If you and your friend are among the 100 students, what is the probability that
\begin{enumerate}
\item you both enter the same section?
\item you both enter the different sections?
\end{enumerate}
\solution
		%\input{ncert/11/16/4/5/defs.tex}
	\item 
The number lock of a suitcase has 4 wheels each labelled with ten digits i.e. from 0 to 9.The lock opens with a sequence of four digits with no repeats.What is the probability of a person getting the right sequence to open the suitcase.
\\
\solution
		%\input{ncert/11/16/4/10/defs.tex}
		%
\item 
Two cards are drawn at random and without replacement from a pack of 52 playing cards. Find the probability that both the cards are black.
\\
\solution
		%\input{ncert/12/13/2/2/defs.tex}
		\item A box of oranges is inspected by examining three randomly selected oranges drawn without replacement. If all the three oranges are good, the box is approved for sale, otherwise, it is rejected. Find the probability that a box containing 15 oranges out of which 12 are good and 3 are bad ones will be approved for sale.
		\label{ncert/12/13/2/3/defs.tex}
		\item Two balls are drawn at random with replacement from a box containing 10 black and 8 red balls. Find the probability that
		\label{ncert/12/13/2/12}
\begin{enumerate}
\item both balls are red.
\item first ball is black and second is red.
\item one of them is black and other is red.
\end{enumerate}

\item In a hostel, 60\% of the students read Hindi newspaper, 40\% read English newspaper and 20\% read both Hindi and English newspapers. A student is selected at random.
		\label{ncert/12/13/2/15}
\begin{enumerate}
\item Find the probability that she reads neither Hindi nor English newspapers.
\item If she reads Hindi newspaper, find the probability that she reads English newspaper.
\item If she reads English newspaper, find the probability that she reads Hindi newspaper.\\
\end{enumerate}
\item The probability of obtaining an even prime number on each die, when a pair of dice is rolled is 
\begin{enumerate}
    \item $0$ 
    
    \item $\frac{1}{3}$ 
    
    \item $\frac{1}{12}$ 
    
    \item $\frac{1}{36}$ 
\end{enumerate}
\solution
		%\input{ncert/12/13/2/17/defs.tex}
	\item A bag contains 4 red and 4 black balls, another bag contains 2 red and 6 black balls. One of the two bags is selected at random and a ball is drawn from the bag which is found to be red. Find the probability that the ball is drawn from the first bag.
\\
\solution
		%\input{ncert/12/13/3/2/main.tex}
  \item
  Cards with numbers 2 to 101 are placed in a box. A card is selected at random.Find the probability that the card has
\begin{enumerate}[label=(\roman*)]
	\item an even number 
	\item a square number
\end{enumerate}
\solution
%\input{exemplar/10/13/3/32/main.tex}
\item
The king, queen and jack of clubs are removed from a deck of 52 playing cards and then well shuffled. Now one card is drawn at random from the remaining cards.  Determine the probability that the card is
\begin{enumerate}[label=(\roman*)]
\item a club
\item 10 of hearts
\end{enumerate}
\solution
%\input{exemplar/10/13/3/29/main.tex}
\item A team of medical students doing their internship have to assist during surgeries
at a city hospital. The probabilities of surgeries rated as very complex, complex,
routine, simple or very simple are respectively, 0.15, 0.20, 0.31, 0.26, .08. Find
the probabilities that a particular surgery will be rated
\begin{enumerate}
	\item complex or very complex;
	\item neither very complex nor very simple;
	\item routine or complex
	\item routine or simple
\end{enumerate}
\solution
%\input{exemplar/11/16/3/8(1)/main.tex}
\item A card is selected from a pack of 52 cards.
\begin{enumerate}[label=(\alph*)]
    \item How many points are there in the sample space?
    \item Calculate the probability that the card is an ace of spades.
    \item Calculate the probability that the card is (i) an ace and (ii) black card.
\end{enumerate}
\solution
%\input{exemplar/11/16/3/4/main2.tex}
\item The probability that a non leap year selected at random will contain 53 sundays.
\\
\solution
%\input{exemplar/10/13/1/19/main.tex}
\item One of the four persons John, Rita, Aslam or Gurpreet will be promoted next
month. Consequently the sample space consists of four elementary outcomes
S = {John promoted, Rita promoted, Aslam promoted, Gurpreet promoted}
You are told that the chances of John’s promotion is same as that of Gurpreet,
Rita’s chances of promotion are twice as likely as Johns. Aslam’s chances are
four times that of John.
\begin{enumerate}
	\item Determine
	\begin{enumerate}
		\item P (John promoted)
		\item P (Rita promoted)
		\item P (Aslam promoted)
		\item P (Gurpreet promoted)
	\end{enumerate}
	\item If A = {John promoted or Gurpreet promoted}, find P (A).
\end{enumerate}
\solution
%\input{exemplar/11/16/3/10/main.tex}
\item A card is drawn from a deck of 52 cards. Find the probability of getting a king or a heart or a red card.\\
\solution
%\input{exemplar/11/16/3/15/main.tex}
\item The probability that a student will pass his examination is 0.73, the probability of
the student getting a compartment is 0.13, and the probability that the student will
either pass or get compartment is 0.96. State True or False.\\
\solution
%\input{exemplar/11/16/3/31/main.tex}
\item A card is selected from a pack of 52 cards\\
\begin{enumerate}[label=(\alph*)]
\item How many points are there in the sample space?
\item Calculate the probability that the cards is an ace of spades.
\item Calculate the probability that the card is (i) an ace (ii)black card.\\
\end{enumerate}
%\input{ncert/11/16/3/4_1/Prob_4.tex}
\item In a non-leap year, the probability of having 53 tuesdays or 53 wednesdays is\\
\solution
%\input{exemplar/11/16/3/18/main.tex}
\item There are 1000 sealed envelopes in a box, 10 of them contain a cash prize of
Rs 100 each, 100 of them contain a cash prize of Rs 50 each and 200 of them
contain a cash prize of Rs 10 each and rest do not contain any cash prize. If they
are well shuffled and an envelope is picked up out, what is the probability that it
contains no cash prize?\\
\solution
%\input{exemplar/10/13/3/34/main.tex}
\item 
A die is thrown and a card is selected at random from a deck of 52 playing cards. The probability of getting an even number on the die and a spade card.\\
\solution
%\input{exemplar/12/13/3/78/main.tex}
\item
If 4-digit numbers greater than 5,000 are randomly formed from the digits 0, 1, 3, 5, and 7, what is the probability of forming a number divisible by 5 when:
\begin{enumerate}
    \item The digits are repeated?
    \item The repetition of digits is not allowed?
\end{enumerate}
\solution
%\input{ncert/11/16/4/9/main.tex}
\item Consider the probability space $\brak{\Omega, \mathcal{G}, P}$ where $\Omega = [0,2]$ and $\mathcal{G} = \cbrak{\phi, \Omega, [0,1], (1,2]}$. Let $X$ and $Y$ be two functions on $\Omega$ defined as
\begin{align*}
    X(\omega) = 
    \begin{cases}
        1 & \text{if }\omega \in [0, 1]\\
        2 & \text{if }\omega \in (1, 2]
    \end{cases}
\end{align*}
and
\begin{align*}
    Y(\omega) = 
    \begin{cases}
        2 & \text{if }\omega \in [0, 1.5]\\
        3 & \text{if }\omega \in (1.5, 2].
    \end{cases}
\end{align*}
Then which one of the following statements is true?
\begin{enumerate}
    \item [(A)] $X$ is a random variable with respect to $\mathcal{G}$, but $Y$ is not a random variable with respect to $\mathcal{G}$.
    \item [(B)] $Y$ is a random variable with respect to $\mathcal{G}$, but $X$ is not a random variable with respect to $\mathcal{G}$.
    \item [(C)] Neither $X$ nor $Y$ is a random variable with respect to $\mathcal{G}$.
    \item [(D)] Both $X$ and $Y$ are random variables with respect to $\mathcal{G}$.
\end{enumerate} \hfill (GATE ST 2023)\\
\solution
%\input{gate/ST/2023/14/main.tex}
	\item  A die is loaded in such a way that each odd number is twice as likely to occur as
each even number. Find $P(G)$, where $G$ is the event that a number greater than
3 occurs on a single roll of the die.
\\
\solution
		%\input{exemplar/11/16/3/5/main.tex}
	\item All the jacks, queens and kings are removed from a deck of 52 playing cards. The remaining cards are well shuffled and then one card is drawn at random. Giving ace a value 1 similar value for other cards, find the probability that the card has a value 
		\begin{enumerate}
			\item 7
			\item greater than 7
			\item less than 7
		\end{enumerate}
		%\input{exemplar/10/13/3/30/main.tex}
  \item A Lot consists of 48 mobile phones of which 42 are good, 3 have only minor defects and 3 have major defects.Varnika will buy a phone if it is good but the trader will only buy a mobile if it has no major defects. One phone is selected at random from the lot. What is the probability that it is
\begin{enumerate}
	\item acceptable to Varnika?
            \item acceptable to the trader?
\end{enumerate}
\solution
	%\input{exemplar/10/13/3/40/main.tex}
 \item A student says that if you throw a die, it will show up 1 or not 1. Therefore, the probability of getting 1 and the probability of getting 'not 1' each is equal to $\frac{1}{2}$. Is this correct? Give reasons.\\
 \solution
        %\input{exemplar/10/13/2/9/main.tex}
   \item Four candidates A, B, C, D have ap-
plied for the assignment to coach a school cricket
team. If A is twice as likely to be selected as B, and
B and C are given about the same chance of being
selected, while C is twice as likely to be selected
as D, what are the probabilities that
\begin{enumerate}
\item C will be selected?
\item A will not be selected?
\end{enumerate}
	%\input{exemplar/11/16/3/9/main.tex}
 \item A bag contain 24 balls of which $x$ balls are red, $2x$ are white and $3x$ are blue. A ball is selected at random, What is the probability that it is
\begin{enumerate}[label=\alph*)]
\item not red ?
\item white ?
\end{enumerate}
%\input{exemplar/10/13/3/41/main.tex}
If the letters of the word ASSASSINATION are arranged at random. Find the Probability that
\begin{enumerate}[label=(\alph*)]
\item Four $S's$ come consecutively in the word
\item Two  $I's$ and two $N's$ come together
\item All $A's$ are not coming together
\item No two $A's$ are coming together
\end{enumerate}
%\input{exemplar/11/16/3/14/main.tex}
	\item One urn contains two black balls (labelled B1 and B2) and one white ball. A
	second urn contains one black ball and two white balls (labelled W1 and W2).
	Suppose the following experiment is performed. One of the two urns is chosen
	at random. Next a ball is randomly chosen from the urn. Then a second ball is
	chosen at random from the same urn without replacing the first ball.
	
	\begin{enumerate}
	\item What is the probability that two black balls are chosen?
	
	\item What is the probability that two balls of opposite colour are chosen?
	\end{enumerate}
	\solution
	%\input{exemplar/11/16/3/12/main1.tex}
\end{enumerate}

		%
\item 
Two cards are drawn at random and without replacement from a pack of 52 playing cards. Find the probability that both the cards are black.
\\
\solution
		%\begin{enumerate}[label=\thesection.\arabic*,ref=\thesection.\theenumi]
	\item One card is drawn from a well-shuffled deck of 52 cards. Find the probability of getting
\begin{enumerate}
\item A king of red colour 
\item A face card 
\item A red face card
\item The jack of hearts
\item A spade
\item The queen of diamonds

\end{enumerate}
\solution
		%\input{ncert/10/15/1/14/main.tex}
	\item Five cards—the ten, jack, queen, king and ace of diamonds, are well-shuffled with their face downwards. One card is then picked up at random.
\begin{enumerate}
\item
What is the probability that the card is the queen? 
\item
If the queen is drawn and put aside, what is the probability that the second card picked up is (a) an ace? (b) a queen?\\
\end{enumerate}
\solution
		%\input{ncert/10/15/1/15/defs.tex}
	\item A bag contains $5$ red balls and some blue balls. If the probability of drawing a blue ball is double that if a red ball, determine the number of blue balls in the bag. 
		\\
\solution
		%\input{ncert/10/15/2/3/defs.tex}
	\item A card is selected from a pack of 52 cards.
 \begin{enumerate}[label=(\alph*)] 
                 \item How many points are there in the sample space?
                 \item Calculate the probability that the card is an ace of spades.
                 \item Calculate the probability that the card is (i) an ace and (ii) black card.
 \end{enumerate}
\solution
		%\input{ncert/11/16/3/4/main.tex}
\item Four cards are drawn from a well-shuffled deck of 52 cards. What is the probability of obtaining 3 diamonds and one spade.
\\
\solution
		%\input{ncert/11/16/4/2/defs.tex}
\item In a certain lottery 10,000 tickets are sold and ten equal prizes are awarded. What is the probability of not getting a prize if you buy (a) one ticket (b) two tickets (c) 10 tickets ?	
\\
\solution
		%\input{ncert/11/16/4/4/defs.tex}
		%
\item 
Out of 100 students, two sections of 40 and 60 are formed. If you and your friend are among the 100 students, what is the probability that
\begin{enumerate}
\item you both enter the same section?
\item you both enter the different sections?
\end{enumerate}
\solution
		%\input{ncert/11/16/4/5/defs.tex}
	\item 
The number lock of a suitcase has 4 wheels each labelled with ten digits i.e. from 0 to 9.The lock opens with a sequence of four digits with no repeats.What is the probability of a person getting the right sequence to open the suitcase.
\\
\solution
		%\input{ncert/11/16/4/10/defs.tex}
		%
\item 
Two cards are drawn at random and without replacement from a pack of 52 playing cards. Find the probability that both the cards are black.
\\
\solution
		%\input{ncert/12/13/2/2/defs.tex}
		\item A box of oranges is inspected by examining three randomly selected oranges drawn without replacement. If all the three oranges are good, the box is approved for sale, otherwise, it is rejected. Find the probability that a box containing 15 oranges out of which 12 are good and 3 are bad ones will be approved for sale.
		\label{ncert/12/13/2/3/defs.tex}
		\item Two balls are drawn at random with replacement from a box containing 10 black and 8 red balls. Find the probability that
		\label{ncert/12/13/2/12}
\begin{enumerate}
\item both balls are red.
\item first ball is black and second is red.
\item one of them is black and other is red.
\end{enumerate}

\item In a hostel, 60\% of the students read Hindi newspaper, 40\% read English newspaper and 20\% read both Hindi and English newspapers. A student is selected at random.
		\label{ncert/12/13/2/15}
\begin{enumerate}
\item Find the probability that she reads neither Hindi nor English newspapers.
\item If she reads Hindi newspaper, find the probability that she reads English newspaper.
\item If she reads English newspaper, find the probability that she reads Hindi newspaper.\\
\end{enumerate}
\item The probability of obtaining an even prime number on each die, when a pair of dice is rolled is 
\begin{enumerate}
    \item $0$ 
    
    \item $\frac{1}{3}$ 
    
    \item $\frac{1}{12}$ 
    
    \item $\frac{1}{36}$ 
\end{enumerate}
\solution
		%\input{ncert/12/13/2/17/defs.tex}
	\item A bag contains 4 red and 4 black balls, another bag contains 2 red and 6 black balls. One of the two bags is selected at random and a ball is drawn from the bag which is found to be red. Find the probability that the ball is drawn from the first bag.
\\
\solution
		%\input{ncert/12/13/3/2/main.tex}
  \item
  Cards with numbers 2 to 101 are placed in a box. A card is selected at random.Find the probability that the card has
\begin{enumerate}[label=(\roman*)]
	\item an even number 
	\item a square number
\end{enumerate}
\solution
%\input{exemplar/10/13/3/32/main.tex}
\item
The king, queen and jack of clubs are removed from a deck of 52 playing cards and then well shuffled. Now one card is drawn at random from the remaining cards.  Determine the probability that the card is
\begin{enumerate}[label=(\roman*)]
\item a club
\item 10 of hearts
\end{enumerate}
\solution
%\input{exemplar/10/13/3/29/main.tex}
\item A team of medical students doing their internship have to assist during surgeries
at a city hospital. The probabilities of surgeries rated as very complex, complex,
routine, simple or very simple are respectively, 0.15, 0.20, 0.31, 0.26, .08. Find
the probabilities that a particular surgery will be rated
\begin{enumerate}
	\item complex or very complex;
	\item neither very complex nor very simple;
	\item routine or complex
	\item routine or simple
\end{enumerate}
\solution
%\input{exemplar/11/16/3/8(1)/main.tex}
\item A card is selected from a pack of 52 cards.
\begin{enumerate}[label=(\alph*)]
    \item How many points are there in the sample space?
    \item Calculate the probability that the card is an ace of spades.
    \item Calculate the probability that the card is (i) an ace and (ii) black card.
\end{enumerate}
\solution
%\input{exemplar/11/16/3/4/main2.tex}
\item The probability that a non leap year selected at random will contain 53 sundays.
\\
\solution
%\input{exemplar/10/13/1/19/main.tex}
\item One of the four persons John, Rita, Aslam or Gurpreet will be promoted next
month. Consequently the sample space consists of four elementary outcomes
S = {John promoted, Rita promoted, Aslam promoted, Gurpreet promoted}
You are told that the chances of John’s promotion is same as that of Gurpreet,
Rita’s chances of promotion are twice as likely as Johns. Aslam’s chances are
four times that of John.
\begin{enumerate}
	\item Determine
	\begin{enumerate}
		\item P (John promoted)
		\item P (Rita promoted)
		\item P (Aslam promoted)
		\item P (Gurpreet promoted)
	\end{enumerate}
	\item If A = {John promoted or Gurpreet promoted}, find P (A).
\end{enumerate}
\solution
%\input{exemplar/11/16/3/10/main.tex}
\item A card is drawn from a deck of 52 cards. Find the probability of getting a king or a heart or a red card.\\
\solution
%\input{exemplar/11/16/3/15/main.tex}
\item The probability that a student will pass his examination is 0.73, the probability of
the student getting a compartment is 0.13, and the probability that the student will
either pass or get compartment is 0.96. State True or False.\\
\solution
%\input{exemplar/11/16/3/31/main.tex}
\item A card is selected from a pack of 52 cards\\
\begin{enumerate}[label=(\alph*)]
\item How many points are there in the sample space?
\item Calculate the probability that the cards is an ace of spades.
\item Calculate the probability that the card is (i) an ace (ii)black card.\\
\end{enumerate}
%\input{ncert/11/16/3/4_1/Prob_4.tex}
\item In a non-leap year, the probability of having 53 tuesdays or 53 wednesdays is\\
\solution
%\input{exemplar/11/16/3/18/main.tex}
\item There are 1000 sealed envelopes in a box, 10 of them contain a cash prize of
Rs 100 each, 100 of them contain a cash prize of Rs 50 each and 200 of them
contain a cash prize of Rs 10 each and rest do not contain any cash prize. If they
are well shuffled and an envelope is picked up out, what is the probability that it
contains no cash prize?\\
\solution
%\input{exemplar/10/13/3/34/main.tex}
\item 
A die is thrown and a card is selected at random from a deck of 52 playing cards. The probability of getting an even number on the die and a spade card.\\
\solution
%\input{exemplar/12/13/3/78/main.tex}
\item
If 4-digit numbers greater than 5,000 are randomly formed from the digits 0, 1, 3, 5, and 7, what is the probability of forming a number divisible by 5 when:
\begin{enumerate}
    \item The digits are repeated?
    \item The repetition of digits is not allowed?
\end{enumerate}
\solution
%\input{ncert/11/16/4/9/main.tex}
\item Consider the probability space $\brak{\Omega, \mathcal{G}, P}$ where $\Omega = [0,2]$ and $\mathcal{G} = \cbrak{\phi, \Omega, [0,1], (1,2]}$. Let $X$ and $Y$ be two functions on $\Omega$ defined as
\begin{align*}
    X(\omega) = 
    \begin{cases}
        1 & \text{if }\omega \in [0, 1]\\
        2 & \text{if }\omega \in (1, 2]
    \end{cases}
\end{align*}
and
\begin{align*}
    Y(\omega) = 
    \begin{cases}
        2 & \text{if }\omega \in [0, 1.5]\\
        3 & \text{if }\omega \in (1.5, 2].
    \end{cases}
\end{align*}
Then which one of the following statements is true?
\begin{enumerate}
    \item [(A)] $X$ is a random variable with respect to $\mathcal{G}$, but $Y$ is not a random variable with respect to $\mathcal{G}$.
    \item [(B)] $Y$ is a random variable with respect to $\mathcal{G}$, but $X$ is not a random variable with respect to $\mathcal{G}$.
    \item [(C)] Neither $X$ nor $Y$ is a random variable with respect to $\mathcal{G}$.
    \item [(D)] Both $X$ and $Y$ are random variables with respect to $\mathcal{G}$.
\end{enumerate} \hfill (GATE ST 2023)\\
\solution
%\input{gate/ST/2023/14/main.tex}
	\item  A die is loaded in such a way that each odd number is twice as likely to occur as
each even number. Find $P(G)$, where $G$ is the event that a number greater than
3 occurs on a single roll of the die.
\\
\solution
		%\input{exemplar/11/16/3/5/main.tex}
	\item All the jacks, queens and kings are removed from a deck of 52 playing cards. The remaining cards are well shuffled and then one card is drawn at random. Giving ace a value 1 similar value for other cards, find the probability that the card has a value 
		\begin{enumerate}
			\item 7
			\item greater than 7
			\item less than 7
		\end{enumerate}
		%\input{exemplar/10/13/3/30/main.tex}
  \item A Lot consists of 48 mobile phones of which 42 are good, 3 have only minor defects and 3 have major defects.Varnika will buy a phone if it is good but the trader will only buy a mobile if it has no major defects. One phone is selected at random from the lot. What is the probability that it is
\begin{enumerate}
	\item acceptable to Varnika?
            \item acceptable to the trader?
\end{enumerate}
\solution
	%\input{exemplar/10/13/3/40/main.tex}
 \item A student says that if you throw a die, it will show up 1 or not 1. Therefore, the probability of getting 1 and the probability of getting 'not 1' each is equal to $\frac{1}{2}$. Is this correct? Give reasons.\\
 \solution
        %\input{exemplar/10/13/2/9/main.tex}
   \item Four candidates A, B, C, D have ap-
plied for the assignment to coach a school cricket
team. If A is twice as likely to be selected as B, and
B and C are given about the same chance of being
selected, while C is twice as likely to be selected
as D, what are the probabilities that
\begin{enumerate}
\item C will be selected?
\item A will not be selected?
\end{enumerate}
	%\input{exemplar/11/16/3/9/main.tex}
 \item A bag contain 24 balls of which $x$ balls are red, $2x$ are white and $3x$ are blue. A ball is selected at random, What is the probability that it is
\begin{enumerate}[label=\alph*)]
\item not red ?
\item white ?
\end{enumerate}
%\input{exemplar/10/13/3/41/main.tex}
If the letters of the word ASSASSINATION are arranged at random. Find the Probability that
\begin{enumerate}[label=(\alph*)]
\item Four $S's$ come consecutively in the word
\item Two  $I's$ and two $N's$ come together
\item All $A's$ are not coming together
\item No two $A's$ are coming together
\end{enumerate}
%\input{exemplar/11/16/3/14/main.tex}
	\item One urn contains two black balls (labelled B1 and B2) and one white ball. A
	second urn contains one black ball and two white balls (labelled W1 and W2).
	Suppose the following experiment is performed. One of the two urns is chosen
	at random. Next a ball is randomly chosen from the urn. Then a second ball is
	chosen at random from the same urn without replacing the first ball.
	
	\begin{enumerate}
	\item What is the probability that two black balls are chosen?
	
	\item What is the probability that two balls of opposite colour are chosen?
	\end{enumerate}
	\solution
	%\input{exemplar/11/16/3/12/main1.tex}
\end{enumerate}

		\item A box of oranges is inspected by examining three randomly selected oranges drawn without replacement. If all the three oranges are good, the box is approved for sale, otherwise, it is rejected. Find the probability that a box containing 15 oranges out of which 12 are good and 3 are bad ones will be approved for sale.
		\label{ncert/12/13/2/3/defs.tex}
		\item Two balls are drawn at random with replacement from a box containing 10 black and 8 red balls. Find the probability that
		\label{ncert/12/13/2/12}
\begin{enumerate}
\item both balls are red.
\item first ball is black and second is red.
\item one of them is black and other is red.
\end{enumerate}

\item In a hostel, 60\% of the students read Hindi newspaper, 40\% read English newspaper and 20\% read both Hindi and English newspapers. A student is selected at random.
		\label{ncert/12/13/2/15}
\begin{enumerate}
\item Find the probability that she reads neither Hindi nor English newspapers.
\item If she reads Hindi newspaper, find the probability that she reads English newspaper.
\item If she reads English newspaper, find the probability that she reads Hindi newspaper.\\
\end{enumerate}
\item The probability of obtaining an even prime number on each die, when a pair of dice is rolled is 
\begin{enumerate}
    \item $0$ 
    
    \item $\frac{1}{3}$ 
    
    \item $\frac{1}{12}$ 
    
    \item $\frac{1}{36}$ 
\end{enumerate}
\solution
		%\begin{enumerate}[label=\thesection.\arabic*,ref=\thesection.\theenumi]
	\item One card is drawn from a well-shuffled deck of 52 cards. Find the probability of getting
\begin{enumerate}
\item A king of red colour 
\item A face card 
\item A red face card
\item The jack of hearts
\item A spade
\item The queen of diamonds

\end{enumerate}
\solution
		%\input{ncert/10/15/1/14/main.tex}
	\item Five cards—the ten, jack, queen, king and ace of diamonds, are well-shuffled with their face downwards. One card is then picked up at random.
\begin{enumerate}
\item
What is the probability that the card is the queen? 
\item
If the queen is drawn and put aside, what is the probability that the second card picked up is (a) an ace? (b) a queen?\\
\end{enumerate}
\solution
		%\input{ncert/10/15/1/15/defs.tex}
	\item A bag contains $5$ red balls and some blue balls. If the probability of drawing a blue ball is double that if a red ball, determine the number of blue balls in the bag. 
		\\
\solution
		%\input{ncert/10/15/2/3/defs.tex}
	\item A card is selected from a pack of 52 cards.
 \begin{enumerate}[label=(\alph*)] 
                 \item How many points are there in the sample space?
                 \item Calculate the probability that the card is an ace of spades.
                 \item Calculate the probability that the card is (i) an ace and (ii) black card.
 \end{enumerate}
\solution
		%\input{ncert/11/16/3/4/main.tex}
\item Four cards are drawn from a well-shuffled deck of 52 cards. What is the probability of obtaining 3 diamonds and one spade.
\\
\solution
		%\input{ncert/11/16/4/2/defs.tex}
\item In a certain lottery 10,000 tickets are sold and ten equal prizes are awarded. What is the probability of not getting a prize if you buy (a) one ticket (b) two tickets (c) 10 tickets ?	
\\
\solution
		%\input{ncert/11/16/4/4/defs.tex}
		%
\item 
Out of 100 students, two sections of 40 and 60 are formed. If you and your friend are among the 100 students, what is the probability that
\begin{enumerate}
\item you both enter the same section?
\item you both enter the different sections?
\end{enumerate}
\solution
		%\input{ncert/11/16/4/5/defs.tex}
	\item 
The number lock of a suitcase has 4 wheels each labelled with ten digits i.e. from 0 to 9.The lock opens with a sequence of four digits with no repeats.What is the probability of a person getting the right sequence to open the suitcase.
\\
\solution
		%\input{ncert/11/16/4/10/defs.tex}
		%
\item 
Two cards are drawn at random and without replacement from a pack of 52 playing cards. Find the probability that both the cards are black.
\\
\solution
		%\input{ncert/12/13/2/2/defs.tex}
		\item A box of oranges is inspected by examining three randomly selected oranges drawn without replacement. If all the three oranges are good, the box is approved for sale, otherwise, it is rejected. Find the probability that a box containing 15 oranges out of which 12 are good and 3 are bad ones will be approved for sale.
		\label{ncert/12/13/2/3/defs.tex}
		\item Two balls are drawn at random with replacement from a box containing 10 black and 8 red balls. Find the probability that
		\label{ncert/12/13/2/12}
\begin{enumerate}
\item both balls are red.
\item first ball is black and second is red.
\item one of them is black and other is red.
\end{enumerate}

\item In a hostel, 60\% of the students read Hindi newspaper, 40\% read English newspaper and 20\% read both Hindi and English newspapers. A student is selected at random.
		\label{ncert/12/13/2/15}
\begin{enumerate}
\item Find the probability that she reads neither Hindi nor English newspapers.
\item If she reads Hindi newspaper, find the probability that she reads English newspaper.
\item If she reads English newspaper, find the probability that she reads Hindi newspaper.\\
\end{enumerate}
\item The probability of obtaining an even prime number on each die, when a pair of dice is rolled is 
\begin{enumerate}
    \item $0$ 
    
    \item $\frac{1}{3}$ 
    
    \item $\frac{1}{12}$ 
    
    \item $\frac{1}{36}$ 
\end{enumerate}
\solution
		%\input{ncert/12/13/2/17/defs.tex}
	\item A bag contains 4 red and 4 black balls, another bag contains 2 red and 6 black balls. One of the two bags is selected at random and a ball is drawn from the bag which is found to be red. Find the probability that the ball is drawn from the first bag.
\\
\solution
		%\input{ncert/12/13/3/2/main.tex}
  \item
  Cards with numbers 2 to 101 are placed in a box. A card is selected at random.Find the probability that the card has
\begin{enumerate}[label=(\roman*)]
	\item an even number 
	\item a square number
\end{enumerate}
\solution
%\input{exemplar/10/13/3/32/main.tex}
\item
The king, queen and jack of clubs are removed from a deck of 52 playing cards and then well shuffled. Now one card is drawn at random from the remaining cards.  Determine the probability that the card is
\begin{enumerate}[label=(\roman*)]
\item a club
\item 10 of hearts
\end{enumerate}
\solution
%\input{exemplar/10/13/3/29/main.tex}
\item A team of medical students doing their internship have to assist during surgeries
at a city hospital. The probabilities of surgeries rated as very complex, complex,
routine, simple or very simple are respectively, 0.15, 0.20, 0.31, 0.26, .08. Find
the probabilities that a particular surgery will be rated
\begin{enumerate}
	\item complex or very complex;
	\item neither very complex nor very simple;
	\item routine or complex
	\item routine or simple
\end{enumerate}
\solution
%\input{exemplar/11/16/3/8(1)/main.tex}
\item A card is selected from a pack of 52 cards.
\begin{enumerate}[label=(\alph*)]
    \item How many points are there in the sample space?
    \item Calculate the probability that the card is an ace of spades.
    \item Calculate the probability that the card is (i) an ace and (ii) black card.
\end{enumerate}
\solution
%\input{exemplar/11/16/3/4/main2.tex}
\item The probability that a non leap year selected at random will contain 53 sundays.
\\
\solution
%\input{exemplar/10/13/1/19/main.tex}
\item One of the four persons John, Rita, Aslam or Gurpreet will be promoted next
month. Consequently the sample space consists of four elementary outcomes
S = {John promoted, Rita promoted, Aslam promoted, Gurpreet promoted}
You are told that the chances of John’s promotion is same as that of Gurpreet,
Rita’s chances of promotion are twice as likely as Johns. Aslam’s chances are
four times that of John.
\begin{enumerate}
	\item Determine
	\begin{enumerate}
		\item P (John promoted)
		\item P (Rita promoted)
		\item P (Aslam promoted)
		\item P (Gurpreet promoted)
	\end{enumerate}
	\item If A = {John promoted or Gurpreet promoted}, find P (A).
\end{enumerate}
\solution
%\input{exemplar/11/16/3/10/main.tex}
\item A card is drawn from a deck of 52 cards. Find the probability of getting a king or a heart or a red card.\\
\solution
%\input{exemplar/11/16/3/15/main.tex}
\item The probability that a student will pass his examination is 0.73, the probability of
the student getting a compartment is 0.13, and the probability that the student will
either pass or get compartment is 0.96. State True or False.\\
\solution
%\input{exemplar/11/16/3/31/main.tex}
\item A card is selected from a pack of 52 cards\\
\begin{enumerate}[label=(\alph*)]
\item How many points are there in the sample space?
\item Calculate the probability that the cards is an ace of spades.
\item Calculate the probability that the card is (i) an ace (ii)black card.\\
\end{enumerate}
%\input{ncert/11/16/3/4_1/Prob_4.tex}
\item In a non-leap year, the probability of having 53 tuesdays or 53 wednesdays is\\
\solution
%\input{exemplar/11/16/3/18/main.tex}
\item There are 1000 sealed envelopes in a box, 10 of them contain a cash prize of
Rs 100 each, 100 of them contain a cash prize of Rs 50 each and 200 of them
contain a cash prize of Rs 10 each and rest do not contain any cash prize. If they
are well shuffled and an envelope is picked up out, what is the probability that it
contains no cash prize?\\
\solution
%\input{exemplar/10/13/3/34/main.tex}
\item 
A die is thrown and a card is selected at random from a deck of 52 playing cards. The probability of getting an even number on the die and a spade card.\\
\solution
%\input{exemplar/12/13/3/78/main.tex}
\item
If 4-digit numbers greater than 5,000 are randomly formed from the digits 0, 1, 3, 5, and 7, what is the probability of forming a number divisible by 5 when:
\begin{enumerate}
    \item The digits are repeated?
    \item The repetition of digits is not allowed?
\end{enumerate}
\solution
%\input{ncert/11/16/4/9/main.tex}
\item Consider the probability space $\brak{\Omega, \mathcal{G}, P}$ where $\Omega = [0,2]$ and $\mathcal{G} = \cbrak{\phi, \Omega, [0,1], (1,2]}$. Let $X$ and $Y$ be two functions on $\Omega$ defined as
\begin{align*}
    X(\omega) = 
    \begin{cases}
        1 & \text{if }\omega \in [0, 1]\\
        2 & \text{if }\omega \in (1, 2]
    \end{cases}
\end{align*}
and
\begin{align*}
    Y(\omega) = 
    \begin{cases}
        2 & \text{if }\omega \in [0, 1.5]\\
        3 & \text{if }\omega \in (1.5, 2].
    \end{cases}
\end{align*}
Then which one of the following statements is true?
\begin{enumerate}
    \item [(A)] $X$ is a random variable with respect to $\mathcal{G}$, but $Y$ is not a random variable with respect to $\mathcal{G}$.
    \item [(B)] $Y$ is a random variable with respect to $\mathcal{G}$, but $X$ is not a random variable with respect to $\mathcal{G}$.
    \item [(C)] Neither $X$ nor $Y$ is a random variable with respect to $\mathcal{G}$.
    \item [(D)] Both $X$ and $Y$ are random variables with respect to $\mathcal{G}$.
\end{enumerate} \hfill (GATE ST 2023)\\
\solution
%\input{gate/ST/2023/14/main.tex}
	\item  A die is loaded in such a way that each odd number is twice as likely to occur as
each even number. Find $P(G)$, where $G$ is the event that a number greater than
3 occurs on a single roll of the die.
\\
\solution
		%\input{exemplar/11/16/3/5/main.tex}
	\item All the jacks, queens and kings are removed from a deck of 52 playing cards. The remaining cards are well shuffled and then one card is drawn at random. Giving ace a value 1 similar value for other cards, find the probability that the card has a value 
		\begin{enumerate}
			\item 7
			\item greater than 7
			\item less than 7
		\end{enumerate}
		%\input{exemplar/10/13/3/30/main.tex}
  \item A Lot consists of 48 mobile phones of which 42 are good, 3 have only minor defects and 3 have major defects.Varnika will buy a phone if it is good but the trader will only buy a mobile if it has no major defects. One phone is selected at random from the lot. What is the probability that it is
\begin{enumerate}
	\item acceptable to Varnika?
            \item acceptable to the trader?
\end{enumerate}
\solution
	%\input{exemplar/10/13/3/40/main.tex}
 \item A student says that if you throw a die, it will show up 1 or not 1. Therefore, the probability of getting 1 and the probability of getting 'not 1' each is equal to $\frac{1}{2}$. Is this correct? Give reasons.\\
 \solution
        %\input{exemplar/10/13/2/9/main.tex}
   \item Four candidates A, B, C, D have ap-
plied for the assignment to coach a school cricket
team. If A is twice as likely to be selected as B, and
B and C are given about the same chance of being
selected, while C is twice as likely to be selected
as D, what are the probabilities that
\begin{enumerate}
\item C will be selected?
\item A will not be selected?
\end{enumerate}
	%\input{exemplar/11/16/3/9/main.tex}
 \item A bag contain 24 balls of which $x$ balls are red, $2x$ are white and $3x$ are blue. A ball is selected at random, What is the probability that it is
\begin{enumerate}[label=\alph*)]
\item not red ?
\item white ?
\end{enumerate}
%\input{exemplar/10/13/3/41/main.tex}
If the letters of the word ASSASSINATION are arranged at random. Find the Probability that
\begin{enumerate}[label=(\alph*)]
\item Four $S's$ come consecutively in the word
\item Two  $I's$ and two $N's$ come together
\item All $A's$ are not coming together
\item No two $A's$ are coming together
\end{enumerate}
%\input{exemplar/11/16/3/14/main.tex}
	\item One urn contains two black balls (labelled B1 and B2) and one white ball. A
	second urn contains one black ball and two white balls (labelled W1 and W2).
	Suppose the following experiment is performed. One of the two urns is chosen
	at random. Next a ball is randomly chosen from the urn. Then a second ball is
	chosen at random from the same urn without replacing the first ball.
	
	\begin{enumerate}
	\item What is the probability that two black balls are chosen?
	
	\item What is the probability that two balls of opposite colour are chosen?
	\end{enumerate}
	\solution
	%\input{exemplar/11/16/3/12/main1.tex}
\end{enumerate}

	\item A bag contains 4 red and 4 black balls, another bag contains 2 red and 6 black balls. One of the two bags is selected at random and a ball is drawn from the bag which is found to be red. Find the probability that the ball is drawn from the first bag.
\\
\solution
		%\begin{table}[H]
	\centering
\begin{tabular}{|c|c|c|}
\hline
Random variable &Value &Definition\\ \hline
\multirow{3}{*}{X} &0 &Slips of Rs 1\\
&1 &Slips of Rs 5\\
&2 &Slips of Rs 13\\ \hline
\multirow{2}{*}{Y} &0 &Box A\\
&1 &Box B\\\hline
\end{tabular}
\caption{}
\label{tab:Distribution}
\end{table}
See \tabref{tab:Distribution}.
\begin{align}
p_{Y}\brak{k}= \begin{cases} 
      \frac{1}{3} & {k=0} \\
      \frac{2}{3 }& {k=1} 
   \end{cases}
   \\
p_{Y|X}\brak{0|0} = \frac{19}{25}\, 
p_{Y|X}\brak{0|1} = \frac{6}{25}\,
p_{Y|X}\brak{1|0} = \frac{45}{50}\,
p_{Y|X}\brak{1|2} = \frac{5}{50}
\end{align}
The desired probability is the probability that a slip drawn at random is marked other than Rs 1,
\begin{align}
&=1-p_X\brak{0}\\
&= p_X(1) + p_X(2)
\end{align}
Using Bayes theorem,
\begin{align}
&= p_Y\brak{0} \times \pr{Y=0 | X=1} + p_Y\brak{1} \times \pr{Y=1|X=2}\\
&=\frac{1}{3} \times \frac{6}{25} + \frac{2}{3} \times \frac{5}{50}\\
&=\frac{11}{75}
\end{align}

\newpage

%\tableofcontents

\bigskip

\renewcommand{\thefigure}{\theenumi}
\renewcommand{\thetable}{\theenumi}
%\renewcommand{\theequation}{\theenumi}

%\begin{abstract}
%%\boldmath
%In this letter, an algorithm for evaluating the exact analytical bit error rate  (BER)  for the piecewise linear (PL) combiner for  multiple relays is presented. Previous results were available only for upto three relays. The algorithm is unique in the sense that  the actual mathematical expressions, that are prohibitively large, need not be explicitly obtained. The diversity gain due to multiple relays is shown through plots of the analytical BER, well supported by simulations. 
%
%\end{abstract}
% IEEEtran.cls defaults to using nonbold math in the Abstract.
% This preserves the distinction between vectors and scalars. However,
% if the journal you are submitting to favors bold math in the abstract,
% then you can use LaTeX's standard command \boldmath at the very start
% of the abstract to achieve this. Many IEEE journals frown on math
% in the abstract anyway.

% Note that keywords are not normally used for peerreview papers.
%\begin{IEEEkeywords}
%Cooperative diversity, decode and forward, piecewise linear
%\end{IEEEkeywords}



% For peer review papers, you can put extra information on the cover
% page as needed:
% \ifCLASSOPTIONpeerreview
% \begin{center} \bfseries EDICS Category: 3-BBND \end{center}
% \fi
%
% For peerreview papers, this IEEEtran command inserts a page break and
% creates the second title. It will be ignored for other modes.
%\IEEEpeerreviewmaketitle




  \item
  Cards with numbers 2 to 101 are placed in a box. A card is selected at random.Find the probability that the card has
\begin{enumerate}[label=(\roman*)]
	\item an even number 
	\item a square number
\end{enumerate}
\solution
%\begin{table}[H]
	\centering
\begin{tabular}{|c|c|c|}
\hline
Random variable &Value &Definition\\ \hline
\multirow{3}{*}{X} &0 &Slips of Rs 1\\
&1 &Slips of Rs 5\\
&2 &Slips of Rs 13\\ \hline
\multirow{2}{*}{Y} &0 &Box A\\
&1 &Box B\\\hline
\end{tabular}
\caption{}
\label{tab:Distribution}
\end{table}
See \tabref{tab:Distribution}.
\begin{align}
p_{Y}\brak{k}= \begin{cases} 
      \frac{1}{3} & {k=0} \\
      \frac{2}{3 }& {k=1} 
   \end{cases}
   \\
p_{Y|X}\brak{0|0} = \frac{19}{25}\, 
p_{Y|X}\brak{0|1} = \frac{6}{25}\,
p_{Y|X}\brak{1|0} = \frac{45}{50}\,
p_{Y|X}\brak{1|2} = \frac{5}{50}
\end{align}
The desired probability is the probability that a slip drawn at random is marked other than Rs 1,
\begin{align}
&=1-p_X\brak{0}\\
&= p_X(1) + p_X(2)
\end{align}
Using Bayes theorem,
\begin{align}
&= p_Y\brak{0} \times \pr{Y=0 | X=1} + p_Y\brak{1} \times \pr{Y=1|X=2}\\
&=\frac{1}{3} \times \frac{6}{25} + \frac{2}{3} \times \frac{5}{50}\\
&=\frac{11}{75}
\end{align}

\newpage

%\tableofcontents

\bigskip

\renewcommand{\thefigure}{\theenumi}
\renewcommand{\thetable}{\theenumi}
%\renewcommand{\theequation}{\theenumi}

%\begin{abstract}
%%\boldmath
%In this letter, an algorithm for evaluating the exact analytical bit error rate  (BER)  for the piecewise linear (PL) combiner for  multiple relays is presented. Previous results were available only for upto three relays. The algorithm is unique in the sense that  the actual mathematical expressions, that are prohibitively large, need not be explicitly obtained. The diversity gain due to multiple relays is shown through plots of the analytical BER, well supported by simulations. 
%
%\end{abstract}
% IEEEtran.cls defaults to using nonbold math in the Abstract.
% This preserves the distinction between vectors and scalars. However,
% if the journal you are submitting to favors bold math in the abstract,
% then you can use LaTeX's standard command \boldmath at the very start
% of the abstract to achieve this. Many IEEE journals frown on math
% in the abstract anyway.

% Note that keywords are not normally used for peerreview papers.
%\begin{IEEEkeywords}
%Cooperative diversity, decode and forward, piecewise linear
%\end{IEEEkeywords}



% For peer review papers, you can put extra information on the cover
% page as needed:
% \ifCLASSOPTIONpeerreview
% \begin{center} \bfseries EDICS Category: 3-BBND \end{center}
% \fi
%
% For peerreview papers, this IEEEtran command inserts a page break and
% creates the second title. It will be ignored for other modes.
%\IEEEpeerreviewmaketitle




\item
The king, queen and jack of clubs are removed from a deck of 52 playing cards and then well shuffled. Now one card is drawn at random from the remaining cards.  Determine the probability that the card is
\begin{enumerate}[label=(\roman*)]
\item a club
\item 10 of hearts
\end{enumerate}
\solution
%\begin{table}[H]
	\centering
\begin{tabular}{|c|c|c|}
\hline
Random variable &Value &Definition\\ \hline
\multirow{3}{*}{X} &0 &Slips of Rs 1\\
&1 &Slips of Rs 5\\
&2 &Slips of Rs 13\\ \hline
\multirow{2}{*}{Y} &0 &Box A\\
&1 &Box B\\\hline
\end{tabular}
\caption{}
\label{tab:Distribution}
\end{table}
See \tabref{tab:Distribution}.
\begin{align}
p_{Y}\brak{k}= \begin{cases} 
      \frac{1}{3} & {k=0} \\
      \frac{2}{3 }& {k=1} 
   \end{cases}
   \\
p_{Y|X}\brak{0|0} = \frac{19}{25}\, 
p_{Y|X}\brak{0|1} = \frac{6}{25}\,
p_{Y|X}\brak{1|0} = \frac{45}{50}\,
p_{Y|X}\brak{1|2} = \frac{5}{50}
\end{align}
The desired probability is the probability that a slip drawn at random is marked other than Rs 1,
\begin{align}
&=1-p_X\brak{0}\\
&= p_X(1) + p_X(2)
\end{align}
Using Bayes theorem,
\begin{align}
&= p_Y\brak{0} \times \pr{Y=0 | X=1} + p_Y\brak{1} \times \pr{Y=1|X=2}\\
&=\frac{1}{3} \times \frac{6}{25} + \frac{2}{3} \times \frac{5}{50}\\
&=\frac{11}{75}
\end{align}

\newpage

%\tableofcontents

\bigskip

\renewcommand{\thefigure}{\theenumi}
\renewcommand{\thetable}{\theenumi}
%\renewcommand{\theequation}{\theenumi}

%\begin{abstract}
%%\boldmath
%In this letter, an algorithm for evaluating the exact analytical bit error rate  (BER)  for the piecewise linear (PL) combiner for  multiple relays is presented. Previous results were available only for upto three relays. The algorithm is unique in the sense that  the actual mathematical expressions, that are prohibitively large, need not be explicitly obtained. The diversity gain due to multiple relays is shown through plots of the analytical BER, well supported by simulations. 
%
%\end{abstract}
% IEEEtran.cls defaults to using nonbold math in the Abstract.
% This preserves the distinction between vectors and scalars. However,
% if the journal you are submitting to favors bold math in the abstract,
% then you can use LaTeX's standard command \boldmath at the very start
% of the abstract to achieve this. Many IEEE journals frown on math
% in the abstract anyway.

% Note that keywords are not normally used for peerreview papers.
%\begin{IEEEkeywords}
%Cooperative diversity, decode and forward, piecewise linear
%\end{IEEEkeywords}



% For peer review papers, you can put extra information on the cover
% page as needed:
% \ifCLASSOPTIONpeerreview
% \begin{center} \bfseries EDICS Category: 3-BBND \end{center}
% \fi
%
% For peerreview papers, this IEEEtran command inserts a page break and
% creates the second title. It will be ignored for other modes.
%\IEEEpeerreviewmaketitle




\item A team of medical students doing their internship have to assist during surgeries
at a city hospital. The probabilities of surgeries rated as very complex, complex,
routine, simple or very simple are respectively, 0.15, 0.20, 0.31, 0.26, .08. Find
the probabilities that a particular surgery will be rated
\begin{enumerate}
	\item complex or very complex;
	\item neither very complex nor very simple;
	\item routine or complex
	\item routine or simple
\end{enumerate}
\solution
%\begin{table}[H]
	\centering
\begin{tabular}{|c|c|c|}
\hline
Random variable &Value &Definition\\ \hline
\multirow{3}{*}{X} &0 &Slips of Rs 1\\
&1 &Slips of Rs 5\\
&2 &Slips of Rs 13\\ \hline
\multirow{2}{*}{Y} &0 &Box A\\
&1 &Box B\\\hline
\end{tabular}
\caption{}
\label{tab:Distribution}
\end{table}
See \tabref{tab:Distribution}.
\begin{align}
p_{Y}\brak{k}= \begin{cases} 
      \frac{1}{3} & {k=0} \\
      \frac{2}{3 }& {k=1} 
   \end{cases}
   \\
p_{Y|X}\brak{0|0} = \frac{19}{25}\, 
p_{Y|X}\brak{0|1} = \frac{6}{25}\,
p_{Y|X}\brak{1|0} = \frac{45}{50}\,
p_{Y|X}\brak{1|2} = \frac{5}{50}
\end{align}
The desired probability is the probability that a slip drawn at random is marked other than Rs 1,
\begin{align}
&=1-p_X\brak{0}\\
&= p_X(1) + p_X(2)
\end{align}
Using Bayes theorem,
\begin{align}
&= p_Y\brak{0} \times \pr{Y=0 | X=1} + p_Y\brak{1} \times \pr{Y=1|X=2}\\
&=\frac{1}{3} \times \frac{6}{25} + \frac{2}{3} \times \frac{5}{50}\\
&=\frac{11}{75}
\end{align}

\newpage

%\tableofcontents

\bigskip

\renewcommand{\thefigure}{\theenumi}
\renewcommand{\thetable}{\theenumi}
%\renewcommand{\theequation}{\theenumi}

%\begin{abstract}
%%\boldmath
%In this letter, an algorithm for evaluating the exact analytical bit error rate  (BER)  for the piecewise linear (PL) combiner for  multiple relays is presented. Previous results were available only for upto three relays. The algorithm is unique in the sense that  the actual mathematical expressions, that are prohibitively large, need not be explicitly obtained. The diversity gain due to multiple relays is shown through plots of the analytical BER, well supported by simulations. 
%
%\end{abstract}
% IEEEtran.cls defaults to using nonbold math in the Abstract.
% This preserves the distinction between vectors and scalars. However,
% if the journal you are submitting to favors bold math in the abstract,
% then you can use LaTeX's standard command \boldmath at the very start
% of the abstract to achieve this. Many IEEE journals frown on math
% in the abstract anyway.

% Note that keywords are not normally used for peerreview papers.
%\begin{IEEEkeywords}
%Cooperative diversity, decode and forward, piecewise linear
%\end{IEEEkeywords}



% For peer review papers, you can put extra information on the cover
% page as needed:
% \ifCLASSOPTIONpeerreview
% \begin{center} \bfseries EDICS Category: 3-BBND \end{center}
% \fi
%
% For peerreview papers, this IEEEtran command inserts a page break and
% creates the second title. It will be ignored for other modes.
%\IEEEpeerreviewmaketitle




\item A card is selected from a pack of 52 cards.
\begin{enumerate}[label=(\alph*)]
    \item How many points are there in the sample space?
    \item Calculate the probability that the card is an ace of spades.
    \item Calculate the probability that the card is (i) an ace and (ii) black card.
\end{enumerate}
\solution
%Let $X$ be an bernoulli rv defined as in \tabref{tab:exemplar/11/16/3/26}.  Then, 
\begin{equation}
    p =
        \frac{4}{11} 
\end{equation}
\begin{table}[H]
	\centering
	\input{exemplar/11/16/3/26/tables/Table2.tex}
	\caption{}
        \label{tab:exemplar/11/16/3/26}
\end{table}

\item The probability that a non leap year selected at random will contain 53 sundays.
\\
\solution
%\begin{table}[H]
	\centering
\begin{tabular}{|c|c|c|}
\hline
Random variable &Value &Definition\\ \hline
\multirow{3}{*}{X} &0 &Slips of Rs 1\\
&1 &Slips of Rs 5\\
&2 &Slips of Rs 13\\ \hline
\multirow{2}{*}{Y} &0 &Box A\\
&1 &Box B\\\hline
\end{tabular}
\caption{}
\label{tab:Distribution}
\end{table}
See \tabref{tab:Distribution}.
\begin{align}
p_{Y}\brak{k}= \begin{cases} 
      \frac{1}{3} & {k=0} \\
      \frac{2}{3 }& {k=1} 
   \end{cases}
   \\
p_{Y|X}\brak{0|0} = \frac{19}{25}\, 
p_{Y|X}\brak{0|1} = \frac{6}{25}\,
p_{Y|X}\brak{1|0} = \frac{45}{50}\,
p_{Y|X}\brak{1|2} = \frac{5}{50}
\end{align}
The desired probability is the probability that a slip drawn at random is marked other than Rs 1,
\begin{align}
&=1-p_X\brak{0}\\
&= p_X(1) + p_X(2)
\end{align}
Using Bayes theorem,
\begin{align}
&= p_Y\brak{0} \times \pr{Y=0 | X=1} + p_Y\brak{1} \times \pr{Y=1|X=2}\\
&=\frac{1}{3} \times \frac{6}{25} + \frac{2}{3} \times \frac{5}{50}\\
&=\frac{11}{75}
\end{align}

\newpage

%\tableofcontents

\bigskip

\renewcommand{\thefigure}{\theenumi}
\renewcommand{\thetable}{\theenumi}
%\renewcommand{\theequation}{\theenumi}

%\begin{abstract}
%%\boldmath
%In this letter, an algorithm for evaluating the exact analytical bit error rate  (BER)  for the piecewise linear (PL) combiner for  multiple relays is presented. Previous results were available only for upto three relays. The algorithm is unique in the sense that  the actual mathematical expressions, that are prohibitively large, need not be explicitly obtained. The diversity gain due to multiple relays is shown through plots of the analytical BER, well supported by simulations. 
%
%\end{abstract}
% IEEEtran.cls defaults to using nonbold math in the Abstract.
% This preserves the distinction between vectors and scalars. However,
% if the journal you are submitting to favors bold math in the abstract,
% then you can use LaTeX's standard command \boldmath at the very start
% of the abstract to achieve this. Many IEEE journals frown on math
% in the abstract anyway.

% Note that keywords are not normally used for peerreview papers.
%\begin{IEEEkeywords}
%Cooperative diversity, decode and forward, piecewise linear
%\end{IEEEkeywords}



% For peer review papers, you can put extra information on the cover
% page as needed:
% \ifCLASSOPTIONpeerreview
% \begin{center} \bfseries EDICS Category: 3-BBND \end{center}
% \fi
%
% For peerreview papers, this IEEEtran command inserts a page break and
% creates the second title. It will be ignored for other modes.
%\IEEEpeerreviewmaketitle




\item One of the four persons John, Rita, Aslam or Gurpreet will be promoted next
month. Consequently the sample space consists of four elementary outcomes
S = {John promoted, Rita promoted, Aslam promoted, Gurpreet promoted}
You are told that the chances of John’s promotion is same as that of Gurpreet,
Rita’s chances of promotion are twice as likely as Johns. Aslam’s chances are
four times that of John.
\begin{enumerate}
	\item Determine
	\begin{enumerate}
		\item P (John promoted)
		\item P (Rita promoted)
		\item P (Aslam promoted)
		\item P (Gurpreet promoted)
	\end{enumerate}
	\item If A = {John promoted or Gurpreet promoted}, find P (A).
\end{enumerate}
\solution
%\begin{table}[H]
	\centering
\begin{tabular}{|c|c|c|}
\hline
Random variable &Value &Definition\\ \hline
\multirow{3}{*}{X} &0 &Slips of Rs 1\\
&1 &Slips of Rs 5\\
&2 &Slips of Rs 13\\ \hline
\multirow{2}{*}{Y} &0 &Box A\\
&1 &Box B\\\hline
\end{tabular}
\caption{}
\label{tab:Distribution}
\end{table}
See \tabref{tab:Distribution}.
\begin{align}
p_{Y}\brak{k}= \begin{cases} 
      \frac{1}{3} & {k=0} \\
      \frac{2}{3 }& {k=1} 
   \end{cases}
   \\
p_{Y|X}\brak{0|0} = \frac{19}{25}\, 
p_{Y|X}\brak{0|1} = \frac{6}{25}\,
p_{Y|X}\brak{1|0} = \frac{45}{50}\,
p_{Y|X}\brak{1|2} = \frac{5}{50}
\end{align}
The desired probability is the probability that a slip drawn at random is marked other than Rs 1,
\begin{align}
&=1-p_X\brak{0}\\
&= p_X(1) + p_X(2)
\end{align}
Using Bayes theorem,
\begin{align}
&= p_Y\brak{0} \times \pr{Y=0 | X=1} + p_Y\brak{1} \times \pr{Y=1|X=2}\\
&=\frac{1}{3} \times \frac{6}{25} + \frac{2}{3} \times \frac{5}{50}\\
&=\frac{11}{75}
\end{align}

\newpage

%\tableofcontents

\bigskip

\renewcommand{\thefigure}{\theenumi}
\renewcommand{\thetable}{\theenumi}
%\renewcommand{\theequation}{\theenumi}

%\begin{abstract}
%%\boldmath
%In this letter, an algorithm for evaluating the exact analytical bit error rate  (BER)  for the piecewise linear (PL) combiner for  multiple relays is presented. Previous results were available only for upto three relays. The algorithm is unique in the sense that  the actual mathematical expressions, that are prohibitively large, need not be explicitly obtained. The diversity gain due to multiple relays is shown through plots of the analytical BER, well supported by simulations. 
%
%\end{abstract}
% IEEEtran.cls defaults to using nonbold math in the Abstract.
% This preserves the distinction between vectors and scalars. However,
% if the journal you are submitting to favors bold math in the abstract,
% then you can use LaTeX's standard command \boldmath at the very start
% of the abstract to achieve this. Many IEEE journals frown on math
% in the abstract anyway.

% Note that keywords are not normally used for peerreview papers.
%\begin{IEEEkeywords}
%Cooperative diversity, decode and forward, piecewise linear
%\end{IEEEkeywords}



% For peer review papers, you can put extra information on the cover
% page as needed:
% \ifCLASSOPTIONpeerreview
% \begin{center} \bfseries EDICS Category: 3-BBND \end{center}
% \fi
%
% For peerreview papers, this IEEEtran command inserts a page break and
% creates the second title. It will be ignored for other modes.
%\IEEEpeerreviewmaketitle




\item A card is drawn from a deck of 52 cards. Find the probability of getting a king or a heart or a red card.\\
\solution
%\begin{table}[H]
	\centering
\begin{tabular}{|c|c|c|}
\hline
Random variable &Value &Definition\\ \hline
\multirow{3}{*}{X} &0 &Slips of Rs 1\\
&1 &Slips of Rs 5\\
&2 &Slips of Rs 13\\ \hline
\multirow{2}{*}{Y} &0 &Box A\\
&1 &Box B\\\hline
\end{tabular}
\caption{}
\label{tab:Distribution}
\end{table}
See \tabref{tab:Distribution}.
\begin{align}
p_{Y}\brak{k}= \begin{cases} 
      \frac{1}{3} & {k=0} \\
      \frac{2}{3 }& {k=1} 
   \end{cases}
   \\
p_{Y|X}\brak{0|0} = \frac{19}{25}\, 
p_{Y|X}\brak{0|1} = \frac{6}{25}\,
p_{Y|X}\brak{1|0} = \frac{45}{50}\,
p_{Y|X}\brak{1|2} = \frac{5}{50}
\end{align}
The desired probability is the probability that a slip drawn at random is marked other than Rs 1,
\begin{align}
&=1-p_X\brak{0}\\
&= p_X(1) + p_X(2)
\end{align}
Using Bayes theorem,
\begin{align}
&= p_Y\brak{0} \times \pr{Y=0 | X=1} + p_Y\brak{1} \times \pr{Y=1|X=2}\\
&=\frac{1}{3} \times \frac{6}{25} + \frac{2}{3} \times \frac{5}{50}\\
&=\frac{11}{75}
\end{align}

\newpage

%\tableofcontents

\bigskip

\renewcommand{\thefigure}{\theenumi}
\renewcommand{\thetable}{\theenumi}
%\renewcommand{\theequation}{\theenumi}

%\begin{abstract}
%%\boldmath
%In this letter, an algorithm for evaluating the exact analytical bit error rate  (BER)  for the piecewise linear (PL) combiner for  multiple relays is presented. Previous results were available only for upto three relays. The algorithm is unique in the sense that  the actual mathematical expressions, that are prohibitively large, need not be explicitly obtained. The diversity gain due to multiple relays is shown through plots of the analytical BER, well supported by simulations. 
%
%\end{abstract}
% IEEEtran.cls defaults to using nonbold math in the Abstract.
% This preserves the distinction between vectors and scalars. However,
% if the journal you are submitting to favors bold math in the abstract,
% then you can use LaTeX's standard command \boldmath at the very start
% of the abstract to achieve this. Many IEEE journals frown on math
% in the abstract anyway.

% Note that keywords are not normally used for peerreview papers.
%\begin{IEEEkeywords}
%Cooperative diversity, decode and forward, piecewise linear
%\end{IEEEkeywords}



% For peer review papers, you can put extra information on the cover
% page as needed:
% \ifCLASSOPTIONpeerreview
% \begin{center} \bfseries EDICS Category: 3-BBND \end{center}
% \fi
%
% For peerreview papers, this IEEEtran command inserts a page break and
% creates the second title. It will be ignored for other modes.
%\IEEEpeerreviewmaketitle




\item The probability that a student will pass his examination is 0.73, the probability of
the student getting a compartment is 0.13, and the probability that the student will
either pass or get compartment is 0.96. State True or False.\\
\solution
%\begin{table}[H]
	\centering
\begin{tabular}{|c|c|c|}
\hline
Random variable &Value &Definition\\ \hline
\multirow{3}{*}{X} &0 &Slips of Rs 1\\
&1 &Slips of Rs 5\\
&2 &Slips of Rs 13\\ \hline
\multirow{2}{*}{Y} &0 &Box A\\
&1 &Box B\\\hline
\end{tabular}
\caption{}
\label{tab:Distribution}
\end{table}
See \tabref{tab:Distribution}.
\begin{align}
p_{Y}\brak{k}= \begin{cases} 
      \frac{1}{3} & {k=0} \\
      \frac{2}{3 }& {k=1} 
   \end{cases}
   \\
p_{Y|X}\brak{0|0} = \frac{19}{25}\, 
p_{Y|X}\brak{0|1} = \frac{6}{25}\,
p_{Y|X}\brak{1|0} = \frac{45}{50}\,
p_{Y|X}\brak{1|2} = \frac{5}{50}
\end{align}
The desired probability is the probability that a slip drawn at random is marked other than Rs 1,
\begin{align}
&=1-p_X\brak{0}\\
&= p_X(1) + p_X(2)
\end{align}
Using Bayes theorem,
\begin{align}
&= p_Y\brak{0} \times \pr{Y=0 | X=1} + p_Y\brak{1} \times \pr{Y=1|X=2}\\
&=\frac{1}{3} \times \frac{6}{25} + \frac{2}{3} \times \frac{5}{50}\\
&=\frac{11}{75}
\end{align}

\newpage

%\tableofcontents

\bigskip

\renewcommand{\thefigure}{\theenumi}
\renewcommand{\thetable}{\theenumi}
%\renewcommand{\theequation}{\theenumi}

%\begin{abstract}
%%\boldmath
%In this letter, an algorithm for evaluating the exact analytical bit error rate  (BER)  for the piecewise linear (PL) combiner for  multiple relays is presented. Previous results were available only for upto three relays. The algorithm is unique in the sense that  the actual mathematical expressions, that are prohibitively large, need not be explicitly obtained. The diversity gain due to multiple relays is shown through plots of the analytical BER, well supported by simulations. 
%
%\end{abstract}
% IEEEtran.cls defaults to using nonbold math in the Abstract.
% This preserves the distinction between vectors and scalars. However,
% if the journal you are submitting to favors bold math in the abstract,
% then you can use LaTeX's standard command \boldmath at the very start
% of the abstract to achieve this. Many IEEE journals frown on math
% in the abstract anyway.

% Note that keywords are not normally used for peerreview papers.
%\begin{IEEEkeywords}
%Cooperative diversity, decode and forward, piecewise linear
%\end{IEEEkeywords}



% For peer review papers, you can put extra information on the cover
% page as needed:
% \ifCLASSOPTIONpeerreview
% \begin{center} \bfseries EDICS Category: 3-BBND \end{center}
% \fi
%
% For peerreview papers, this IEEEtran command inserts a page break and
% creates the second title. It will be ignored for other modes.
%\IEEEpeerreviewmaketitle




\item A card is selected from a pack of 52 cards\\
\begin{enumerate}[label=(\alph*)]
\item How many points are there in the sample space?
\item Calculate the probability that the cards is an ace of spades.
\item Calculate the probability that the card is (i) an ace (ii)black card.\\
\end{enumerate}
%\input{ncert/11/16/3/4_1/Prob_4.tex}
\item In a non-leap year, the probability of having 53 tuesdays or 53 wednesdays is\\
\solution
%A non-leap year has a total of 365 days, and a week has 7 days.\\
So it can be expressed as 
\begin{align}
365\text{days} &=52\times 7+1 \text{day}
\end{align}
$\implies$ 52 tuesdays or wednesdays\\
Random variable X denotes the days of a week
\begin{align}
p_X\brak{k}&=\frac{1}{7}; \quad \brak{1<k<7}
\end{align}
So the probability of extra day being tuesday or wednesday is
\begin{align}
p_X\brak{3}+p_X\brak{4}&=\frac{1}{7}+\frac{1}{7}=\frac{2}{7}
\end{align}



\item There are 1000 sealed envelopes in a box, 10 of them contain a cash prize of
Rs 100 each, 100 of them contain a cash prize of Rs 50 each and 200 of them
contain a cash prize of Rs 10 each and rest do not contain any cash prize. If they
are well shuffled and an envelope is picked up out, what is the probability that it
contains no cash prize?\\
\solution
%\begin{table}[H]
	\centering
\begin{tabular}{|c|c|c|}
\hline
Random variable &Value &Definition\\ \hline
\multirow{3}{*}{X} &0 &Slips of Rs 1\\
&1 &Slips of Rs 5\\
&2 &Slips of Rs 13\\ \hline
\multirow{2}{*}{Y} &0 &Box A\\
&1 &Box B\\\hline
\end{tabular}
\caption{}
\label{tab:Distribution}
\end{table}
See \tabref{tab:Distribution}.
\begin{align}
p_{Y}\brak{k}= \begin{cases} 
      \frac{1}{3} & {k=0} \\
      \frac{2}{3 }& {k=1} 
   \end{cases}
   \\
p_{Y|X}\brak{0|0} = \frac{19}{25}\, 
p_{Y|X}\brak{0|1} = \frac{6}{25}\,
p_{Y|X}\brak{1|0} = \frac{45}{50}\,
p_{Y|X}\brak{1|2} = \frac{5}{50}
\end{align}
The desired probability is the probability that a slip drawn at random is marked other than Rs 1,
\begin{align}
&=1-p_X\brak{0}\\
&= p_X(1) + p_X(2)
\end{align}
Using Bayes theorem,
\begin{align}
&= p_Y\brak{0} \times \pr{Y=0 | X=1} + p_Y\brak{1} \times \pr{Y=1|X=2}\\
&=\frac{1}{3} \times \frac{6}{25} + \frac{2}{3} \times \frac{5}{50}\\
&=\frac{11}{75}
\end{align}

\newpage

%\tableofcontents

\bigskip

\renewcommand{\thefigure}{\theenumi}
\renewcommand{\thetable}{\theenumi}
%\renewcommand{\theequation}{\theenumi}

%\begin{abstract}
%%\boldmath
%In this letter, an algorithm for evaluating the exact analytical bit error rate  (BER)  for the piecewise linear (PL) combiner for  multiple relays is presented. Previous results were available only for upto three relays. The algorithm is unique in the sense that  the actual mathematical expressions, that are prohibitively large, need not be explicitly obtained. The diversity gain due to multiple relays is shown through plots of the analytical BER, well supported by simulations. 
%
%\end{abstract}
% IEEEtran.cls defaults to using nonbold math in the Abstract.
% This preserves the distinction between vectors and scalars. However,
% if the journal you are submitting to favors bold math in the abstract,
% then you can use LaTeX's standard command \boldmath at the very start
% of the abstract to achieve this. Many IEEE journals frown on math
% in the abstract anyway.

% Note that keywords are not normally used for peerreview papers.
%\begin{IEEEkeywords}
%Cooperative diversity, decode and forward, piecewise linear
%\end{IEEEkeywords}



% For peer review papers, you can put extra information on the cover
% page as needed:
% \ifCLASSOPTIONpeerreview
% \begin{center} \bfseries EDICS Category: 3-BBND \end{center}
% \fi
%
% For peerreview papers, this IEEEtran command inserts a page break and
% creates the second title. It will be ignored for other modes.
%\IEEEpeerreviewmaketitle




\item 
A die is thrown and a card is selected at random from a deck of 52 playing cards. The probability of getting an even number on the die and a spade card.\\
\solution
%\begin{table}[H]
	\centering
\begin{tabular}{|c|c|c|}
\hline
Random variable &Value &Definition\\ \hline
\multirow{3}{*}{X} &0 &Slips of Rs 1\\
&1 &Slips of Rs 5\\
&2 &Slips of Rs 13\\ \hline
\multirow{2}{*}{Y} &0 &Box A\\
&1 &Box B\\\hline
\end{tabular}
\caption{}
\label{tab:Distribution}
\end{table}
See \tabref{tab:Distribution}.
\begin{align}
p_{Y}\brak{k}= \begin{cases} 
      \frac{1}{3} & {k=0} \\
      \frac{2}{3 }& {k=1} 
   \end{cases}
   \\
p_{Y|X}\brak{0|0} = \frac{19}{25}\, 
p_{Y|X}\brak{0|1} = \frac{6}{25}\,
p_{Y|X}\brak{1|0} = \frac{45}{50}\,
p_{Y|X}\brak{1|2} = \frac{5}{50}
\end{align}
The desired probability is the probability that a slip drawn at random is marked other than Rs 1,
\begin{align}
&=1-p_X\brak{0}\\
&= p_X(1) + p_X(2)
\end{align}
Using Bayes theorem,
\begin{align}
&= p_Y\brak{0} \times \pr{Y=0 | X=1} + p_Y\brak{1} \times \pr{Y=1|X=2}\\
&=\frac{1}{3} \times \frac{6}{25} + \frac{2}{3} \times \frac{5}{50}\\
&=\frac{11}{75}
\end{align}

\newpage

%\tableofcontents

\bigskip

\renewcommand{\thefigure}{\theenumi}
\renewcommand{\thetable}{\theenumi}
%\renewcommand{\theequation}{\theenumi}

%\begin{abstract}
%%\boldmath
%In this letter, an algorithm for evaluating the exact analytical bit error rate  (BER)  for the piecewise linear (PL) combiner for  multiple relays is presented. Previous results were available only for upto three relays. The algorithm is unique in the sense that  the actual mathematical expressions, that are prohibitively large, need not be explicitly obtained. The diversity gain due to multiple relays is shown through plots of the analytical BER, well supported by simulations. 
%
%\end{abstract}
% IEEEtran.cls defaults to using nonbold math in the Abstract.
% This preserves the distinction between vectors and scalars. However,
% if the journal you are submitting to favors bold math in the abstract,
% then you can use LaTeX's standard command \boldmath at the very start
% of the abstract to achieve this. Many IEEE journals frown on math
% in the abstract anyway.

% Note that keywords are not normally used for peerreview papers.
%\begin{IEEEkeywords}
%Cooperative diversity, decode and forward, piecewise linear
%\end{IEEEkeywords}



% For peer review papers, you can put extra information on the cover
% page as needed:
% \ifCLASSOPTIONpeerreview
% \begin{center} \bfseries EDICS Category: 3-BBND \end{center}
% \fi
%
% For peerreview papers, this IEEEtran command inserts a page break and
% creates the second title. It will be ignored for other modes.
%\IEEEpeerreviewmaketitle




\item
If 4-digit numbers greater than 5,000 are randomly formed from the digits 0, 1, 3, 5, and 7, what is the probability of forming a number divisible by 5 when:
\begin{enumerate}
    \item The digits are repeated?
    \item The repetition of digits is not allowed?
\end{enumerate}
\solution
%\begin{table}[H]
	\centering
\begin{tabular}{|c|c|c|}
\hline
Random variable &Value &Definition\\ \hline
\multirow{3}{*}{X} &0 &Slips of Rs 1\\
&1 &Slips of Rs 5\\
&2 &Slips of Rs 13\\ \hline
\multirow{2}{*}{Y} &0 &Box A\\
&1 &Box B\\\hline
\end{tabular}
\caption{}
\label{tab:Distribution}
\end{table}
See \tabref{tab:Distribution}.
\begin{align}
p_{Y}\brak{k}= \begin{cases} 
      \frac{1}{3} & {k=0} \\
      \frac{2}{3 }& {k=1} 
   \end{cases}
   \\
p_{Y|X}\brak{0|0} = \frac{19}{25}\, 
p_{Y|X}\brak{0|1} = \frac{6}{25}\,
p_{Y|X}\brak{1|0} = \frac{45}{50}\,
p_{Y|X}\brak{1|2} = \frac{5}{50}
\end{align}
The desired probability is the probability that a slip drawn at random is marked other than Rs 1,
\begin{align}
&=1-p_X\brak{0}\\
&= p_X(1) + p_X(2)
\end{align}
Using Bayes theorem,
\begin{align}
&= p_Y\brak{0} \times \pr{Y=0 | X=1} + p_Y\brak{1} \times \pr{Y=1|X=2}\\
&=\frac{1}{3} \times \frac{6}{25} + \frac{2}{3} \times \frac{5}{50}\\
&=\frac{11}{75}
\end{align}

\newpage

%\tableofcontents

\bigskip

\renewcommand{\thefigure}{\theenumi}
\renewcommand{\thetable}{\theenumi}
%\renewcommand{\theequation}{\theenumi}

%\begin{abstract}
%%\boldmath
%In this letter, an algorithm for evaluating the exact analytical bit error rate  (BER)  for the piecewise linear (PL) combiner for  multiple relays is presented. Previous results were available only for upto three relays. The algorithm is unique in the sense that  the actual mathematical expressions, that are prohibitively large, need not be explicitly obtained. The diversity gain due to multiple relays is shown through plots of the analytical BER, well supported by simulations. 
%
%\end{abstract}
% IEEEtran.cls defaults to using nonbold math in the Abstract.
% This preserves the distinction between vectors and scalars. However,
% if the journal you are submitting to favors bold math in the abstract,
% then you can use LaTeX's standard command \boldmath at the very start
% of the abstract to achieve this. Many IEEE journals frown on math
% in the abstract anyway.

% Note that keywords are not normally used for peerreview papers.
%\begin{IEEEkeywords}
%Cooperative diversity, decode and forward, piecewise linear
%\end{IEEEkeywords}



% For peer review papers, you can put extra information on the cover
% page as needed:
% \ifCLASSOPTIONpeerreview
% \begin{center} \bfseries EDICS Category: 3-BBND \end{center}
% \fi
%
% For peerreview papers, this IEEEtran command inserts a page break and
% creates the second title. It will be ignored for other modes.
%\IEEEpeerreviewmaketitle




\item Consider the probability space $\brak{\Omega, \mathcal{G}, P}$ where $\Omega = [0,2]$ and $\mathcal{G} = \cbrak{\phi, \Omega, [0,1], (1,2]}$. Let $X$ and $Y$ be two functions on $\Omega$ defined as
\begin{align*}
    X(\omega) = 
    \begin{cases}
        1 & \text{if }\omega \in [0, 1]\\
        2 & \text{if }\omega \in (1, 2]
    \end{cases}
\end{align*}
and
\begin{align*}
    Y(\omega) = 
    \begin{cases}
        2 & \text{if }\omega \in [0, 1.5]\\
        3 & \text{if }\omega \in (1.5, 2].
    \end{cases}
\end{align*}
Then which one of the following statements is true?
\begin{enumerate}
    \item [(A)] $X$ is a random variable with respect to $\mathcal{G}$, but $Y$ is not a random variable with respect to $\mathcal{G}$.
    \item [(B)] $Y$ is a random variable with respect to $\mathcal{G}$, but $X$ is not a random variable with respect to $\mathcal{G}$.
    \item [(C)] Neither $X$ nor $Y$ is a random variable with respect to $\mathcal{G}$.
    \item [(D)] Both $X$ and $Y$ are random variables with respect to $\mathcal{G}$.
\end{enumerate} \hfill (GATE ST 2023)\\
\solution
%\begin{table}[H]
	\centering
\begin{tabular}{|c|c|c|}
\hline
Random variable &Value &Definition\\ \hline
\multirow{3}{*}{X} &0 &Slips of Rs 1\\
&1 &Slips of Rs 5\\
&2 &Slips of Rs 13\\ \hline
\multirow{2}{*}{Y} &0 &Box A\\
&1 &Box B\\\hline
\end{tabular}
\caption{}
\label{tab:Distribution}
\end{table}
See \tabref{tab:Distribution}.
\begin{align}
p_{Y}\brak{k}= \begin{cases} 
      \frac{1}{3} & {k=0} \\
      \frac{2}{3 }& {k=1} 
   \end{cases}
   \\
p_{Y|X}\brak{0|0} = \frac{19}{25}\, 
p_{Y|X}\brak{0|1} = \frac{6}{25}\,
p_{Y|X}\brak{1|0} = \frac{45}{50}\,
p_{Y|X}\brak{1|2} = \frac{5}{50}
\end{align}
The desired probability is the probability that a slip drawn at random is marked other than Rs 1,
\begin{align}
&=1-p_X\brak{0}\\
&= p_X(1) + p_X(2)
\end{align}
Using Bayes theorem,
\begin{align}
&= p_Y\brak{0} \times \pr{Y=0 | X=1} + p_Y\brak{1} \times \pr{Y=1|X=2}\\
&=\frac{1}{3} \times \frac{6}{25} + \frac{2}{3} \times \frac{5}{50}\\
&=\frac{11}{75}
\end{align}

\newpage

%\tableofcontents

\bigskip

\renewcommand{\thefigure}{\theenumi}
\renewcommand{\thetable}{\theenumi}
%\renewcommand{\theequation}{\theenumi}

%\begin{abstract}
%%\boldmath
%In this letter, an algorithm for evaluating the exact analytical bit error rate  (BER)  for the piecewise linear (PL) combiner for  multiple relays is presented. Previous results were available only for upto three relays. The algorithm is unique in the sense that  the actual mathematical expressions, that are prohibitively large, need not be explicitly obtained. The diversity gain due to multiple relays is shown through plots of the analytical BER, well supported by simulations. 
%
%\end{abstract}
% IEEEtran.cls defaults to using nonbold math in the Abstract.
% This preserves the distinction between vectors and scalars. However,
% if the journal you are submitting to favors bold math in the abstract,
% then you can use LaTeX's standard command \boldmath at the very start
% of the abstract to achieve this. Many IEEE journals frown on math
% in the abstract anyway.

% Note that keywords are not normally used for peerreview papers.
%\begin{IEEEkeywords}
%Cooperative diversity, decode and forward, piecewise linear
%\end{IEEEkeywords}



% For peer review papers, you can put extra information on the cover
% page as needed:
% \ifCLASSOPTIONpeerreview
% \begin{center} \bfseries EDICS Category: 3-BBND \end{center}
% \fi
%
% For peerreview papers, this IEEEtran command inserts a page break and
% creates the second title. It will be ignored for other modes.
%\IEEEpeerreviewmaketitle




	\item  A die is loaded in such a way that each odd number is twice as likely to occur as
each even number. Find $P(G)$, where $G$ is the event that a number greater than
3 occurs on a single roll of the die.
\\
\solution
		%\begin{table}[H]
	\centering
\begin{tabular}{|c|c|c|}
\hline
Random variable &Value &Definition\\ \hline
\multirow{3}{*}{X} &0 &Slips of Rs 1\\
&1 &Slips of Rs 5\\
&2 &Slips of Rs 13\\ \hline
\multirow{2}{*}{Y} &0 &Box A\\
&1 &Box B\\\hline
\end{tabular}
\caption{}
\label{tab:Distribution}
\end{table}
See \tabref{tab:Distribution}.
\begin{align}
p_{Y}\brak{k}= \begin{cases} 
      \frac{1}{3} & {k=0} \\
      \frac{2}{3 }& {k=1} 
   \end{cases}
   \\
p_{Y|X}\brak{0|0} = \frac{19}{25}\, 
p_{Y|X}\brak{0|1} = \frac{6}{25}\,
p_{Y|X}\brak{1|0} = \frac{45}{50}\,
p_{Y|X}\brak{1|2} = \frac{5}{50}
\end{align}
The desired probability is the probability that a slip drawn at random is marked other than Rs 1,
\begin{align}
&=1-p_X\brak{0}\\
&= p_X(1) + p_X(2)
\end{align}
Using Bayes theorem,
\begin{align}
&= p_Y\brak{0} \times \pr{Y=0 | X=1} + p_Y\brak{1} \times \pr{Y=1|X=2}\\
&=\frac{1}{3} \times \frac{6}{25} + \frac{2}{3} \times \frac{5}{50}\\
&=\frac{11}{75}
\end{align}

\newpage

%\tableofcontents

\bigskip

\renewcommand{\thefigure}{\theenumi}
\renewcommand{\thetable}{\theenumi}
%\renewcommand{\theequation}{\theenumi}

%\begin{abstract}
%%\boldmath
%In this letter, an algorithm for evaluating the exact analytical bit error rate  (BER)  for the piecewise linear (PL) combiner for  multiple relays is presented. Previous results were available only for upto three relays. The algorithm is unique in the sense that  the actual mathematical expressions, that are prohibitively large, need not be explicitly obtained. The diversity gain due to multiple relays is shown through plots of the analytical BER, well supported by simulations. 
%
%\end{abstract}
% IEEEtran.cls defaults to using nonbold math in the Abstract.
% This preserves the distinction between vectors and scalars. However,
% if the journal you are submitting to favors bold math in the abstract,
% then you can use LaTeX's standard command \boldmath at the very start
% of the abstract to achieve this. Many IEEE journals frown on math
% in the abstract anyway.

% Note that keywords are not normally used for peerreview papers.
%\begin{IEEEkeywords}
%Cooperative diversity, decode and forward, piecewise linear
%\end{IEEEkeywords}



% For peer review papers, you can put extra information on the cover
% page as needed:
% \ifCLASSOPTIONpeerreview
% \begin{center} \bfseries EDICS Category: 3-BBND \end{center}
% \fi
%
% For peerreview papers, this IEEEtran command inserts a page break and
% creates the second title. It will be ignored for other modes.
%\IEEEpeerreviewmaketitle




	\item All the jacks, queens and kings are removed from a deck of 52 playing cards. The remaining cards are well shuffled and then one card is drawn at random. Giving ace a value 1 similar value for other cards, find the probability that the card has a value 
		\begin{enumerate}
			\item 7
			\item greater than 7
			\item less than 7
		\end{enumerate}
		%Number of cards left after removing all jacks, queens and kings 
\begin{align}
N	= 52 - 4\times 3
	= 40
\end{align}
%\begin{table}[H]
%\def\arraystretch{1.2}
%\begin{tabular}{|c|c|c|}
%\hline
%	\textbf{Parameter} &\textbf{Value} &\textbf{Description}\\ \hline
%	$X$ &1-10 &Represents the value of the card picked \\ \hline
%\end{tabular}
%\end{table}
Let $1 \le X \le 10$ be the value of the card picked.  Then,
\begin{align}
	p_X(k) &= \Pr(X=k)\ \forall\ 1 \leq k \leq 10\\
	&= \frac{4\times 1}{40}\\
	&= \frac{1}{10}\\
	\therefore p_X(k) &= 
	\begin{cases}
		\frac{1}{10} & 1 \leq k \leq 10\\
		0 & \text{otherwise}
	\end{cases}
\end{align}
and
\begin{align}
	F_{X}(k) &= \sum_{m=0}^{k}p_{X}(m) \quad 1 \leq k \leq 10\\
	&= \frac{k}{10}\\
	\therefore F_{X}(k) &= 
	\begin{cases}
		0 & k \leq 0\\
		\frac{k}{10} & 1\leq k \leq 10\\
		1 & k > 10 
	\end{cases}
\end{align}
\begin{enumerate}
	\item Probability that card has value equal to 7 is
		\begin{align}
			 p_{X}(7)
			= \frac{1}{10}
		\end{align}
	\item Probability that card has value greater than 7 is
		\begin{align}
			1 - F_X(7)
			&= 1 - \frac{7}{10}
			\\
			&= \frac{3}{10}
		\end{align}
	\item Probability that card has value less than 7 is
		\begin{align}
			 F_{X}(6)
			=\frac{6}{10}
		\end{align}
\end{enumerate}

  \item A Lot consists of 48 mobile phones of which 42 are good, 3 have only minor defects and 3 have major defects.Varnika will buy a phone if it is good but the trader will only buy a mobile if it has no major defects. One phone is selected at random from the lot. What is the probability that it is
\begin{enumerate}
	\item acceptable to Varnika?
            \item acceptable to the trader?
\end{enumerate}
\solution
	%\begin{table}[H]
	\centering
\begin{tabular}{|c|c|c|}
\hline
Random variable &Value &Definition\\ \hline
\multirow{3}{*}{X} &0 &Slips of Rs 1\\
&1 &Slips of Rs 5\\
&2 &Slips of Rs 13\\ \hline
\multirow{2}{*}{Y} &0 &Box A\\
&1 &Box B\\\hline
\end{tabular}
\caption{}
\label{tab:Distribution}
\end{table}
See \tabref{tab:Distribution}.
\begin{align}
p_{Y}\brak{k}= \begin{cases} 
      \frac{1}{3} & {k=0} \\
      \frac{2}{3 }& {k=1} 
   \end{cases}
   \\
p_{Y|X}\brak{0|0} = \frac{19}{25}\, 
p_{Y|X}\brak{0|1} = \frac{6}{25}\,
p_{Y|X}\brak{1|0} = \frac{45}{50}\,
p_{Y|X}\brak{1|2} = \frac{5}{50}
\end{align}
The desired probability is the probability that a slip drawn at random is marked other than Rs 1,
\begin{align}
&=1-p_X\brak{0}\\
&= p_X(1) + p_X(2)
\end{align}
Using Bayes theorem,
\begin{align}
&= p_Y\brak{0} \times \pr{Y=0 | X=1} + p_Y\brak{1} \times \pr{Y=1|X=2}\\
&=\frac{1}{3} \times \frac{6}{25} + \frac{2}{3} \times \frac{5}{50}\\
&=\frac{11}{75}
\end{align}

\newpage

%\tableofcontents

\bigskip

\renewcommand{\thefigure}{\theenumi}
\renewcommand{\thetable}{\theenumi}
%\renewcommand{\theequation}{\theenumi}

%\begin{abstract}
%%\boldmath
%In this letter, an algorithm for evaluating the exact analytical bit error rate  (BER)  for the piecewise linear (PL) combiner for  multiple relays is presented. Previous results were available only for upto three relays. The algorithm is unique in the sense that  the actual mathematical expressions, that are prohibitively large, need not be explicitly obtained. The diversity gain due to multiple relays is shown through plots of the analytical BER, well supported by simulations. 
%
%\end{abstract}
% IEEEtran.cls defaults to using nonbold math in the Abstract.
% This preserves the distinction between vectors and scalars. However,
% if the journal you are submitting to favors bold math in the abstract,
% then you can use LaTeX's standard command \boldmath at the very start
% of the abstract to achieve this. Many IEEE journals frown on math
% in the abstract anyway.

% Note that keywords are not normally used for peerreview papers.
%\begin{IEEEkeywords}
%Cooperative diversity, decode and forward, piecewise linear
%\end{IEEEkeywords}



% For peer review papers, you can put extra information on the cover
% page as needed:
% \ifCLASSOPTIONpeerreview
% \begin{center} \bfseries EDICS Category: 3-BBND \end{center}
% \fi
%
% For peerreview papers, this IEEEtran command inserts a page break and
% creates the second title. It will be ignored for other modes.
%\IEEEpeerreviewmaketitle




 \item A student says that if you throw a die, it will show up 1 or not 1. Therefore, the probability of getting 1 and the probability of getting 'not 1' each is equal to $\frac{1}{2}$. Is this correct? Give reasons.\\
 \solution
        %\begin{table}[H]
	\centering
\begin{tabular}{|c|c|c|}
\hline
Random variable &Value &Definition\\ \hline
\multirow{3}{*}{X} &0 &Slips of Rs 1\\
&1 &Slips of Rs 5\\
&2 &Slips of Rs 13\\ \hline
\multirow{2}{*}{Y} &0 &Box A\\
&1 &Box B\\\hline
\end{tabular}
\caption{}
\label{tab:Distribution}
\end{table}
See \tabref{tab:Distribution}.
\begin{align}
p_{Y}\brak{k}= \begin{cases} 
      \frac{1}{3} & {k=0} \\
      \frac{2}{3 }& {k=1} 
   \end{cases}
   \\
p_{Y|X}\brak{0|0} = \frac{19}{25}\, 
p_{Y|X}\brak{0|1} = \frac{6}{25}\,
p_{Y|X}\brak{1|0} = \frac{45}{50}\,
p_{Y|X}\brak{1|2} = \frac{5}{50}
\end{align}
The desired probability is the probability that a slip drawn at random is marked other than Rs 1,
\begin{align}
&=1-p_X\brak{0}\\
&= p_X(1) + p_X(2)
\end{align}
Using Bayes theorem,
\begin{align}
&= p_Y\brak{0} \times \pr{Y=0 | X=1} + p_Y\brak{1} \times \pr{Y=1|X=2}\\
&=\frac{1}{3} \times \frac{6}{25} + \frac{2}{3} \times \frac{5}{50}\\
&=\frac{11}{75}
\end{align}

\newpage

%\tableofcontents

\bigskip

\renewcommand{\thefigure}{\theenumi}
\renewcommand{\thetable}{\theenumi}
%\renewcommand{\theequation}{\theenumi}

%\begin{abstract}
%%\boldmath
%In this letter, an algorithm for evaluating the exact analytical bit error rate  (BER)  for the piecewise linear (PL) combiner for  multiple relays is presented. Previous results were available only for upto three relays. The algorithm is unique in the sense that  the actual mathematical expressions, that are prohibitively large, need not be explicitly obtained. The diversity gain due to multiple relays is shown through plots of the analytical BER, well supported by simulations. 
%
%\end{abstract}
% IEEEtran.cls defaults to using nonbold math in the Abstract.
% This preserves the distinction between vectors and scalars. However,
% if the journal you are submitting to favors bold math in the abstract,
% then you can use LaTeX's standard command \boldmath at the very start
% of the abstract to achieve this. Many IEEE journals frown on math
% in the abstract anyway.

% Note that keywords are not normally used for peerreview papers.
%\begin{IEEEkeywords}
%Cooperative diversity, decode and forward, piecewise linear
%\end{IEEEkeywords}



% For peer review papers, you can put extra information on the cover
% page as needed:
% \ifCLASSOPTIONpeerreview
% \begin{center} \bfseries EDICS Category: 3-BBND \end{center}
% \fi
%
% For peerreview papers, this IEEEtran command inserts a page break and
% creates the second title. It will be ignored for other modes.
%\IEEEpeerreviewmaketitle




   \item Four candidates A, B, C, D have ap-
plied for the assignment to coach a school cricket
team. If A is twice as likely to be selected as B, and
B and C are given about the same chance of being
selected, while C is twice as likely to be selected
as D, what are the probabilities that
\begin{enumerate}
\item C will be selected?
\item A will not be selected?
\end{enumerate}
	%\begin{table}[H]
	\centering
\begin{tabular}{|c|c|c|}
\hline
Random variable &Value &Definition\\ \hline
\multirow{3}{*}{X} &0 &Slips of Rs 1\\
&1 &Slips of Rs 5\\
&2 &Slips of Rs 13\\ \hline
\multirow{2}{*}{Y} &0 &Box A\\
&1 &Box B\\\hline
\end{tabular}
\caption{}
\label{tab:Distribution}
\end{table}
See \tabref{tab:Distribution}.
\begin{align}
p_{Y}\brak{k}= \begin{cases} 
      \frac{1}{3} & {k=0} \\
      \frac{2}{3 }& {k=1} 
   \end{cases}
   \\
p_{Y|X}\brak{0|0} = \frac{19}{25}\, 
p_{Y|X}\brak{0|1} = \frac{6}{25}\,
p_{Y|X}\brak{1|0} = \frac{45}{50}\,
p_{Y|X}\brak{1|2} = \frac{5}{50}
\end{align}
The desired probability is the probability that a slip drawn at random is marked other than Rs 1,
\begin{align}
&=1-p_X\brak{0}\\
&= p_X(1) + p_X(2)
\end{align}
Using Bayes theorem,
\begin{align}
&= p_Y\brak{0} \times \pr{Y=0 | X=1} + p_Y\brak{1} \times \pr{Y=1|X=2}\\
&=\frac{1}{3} \times \frac{6}{25} + \frac{2}{3} \times \frac{5}{50}\\
&=\frac{11}{75}
\end{align}

\newpage

%\tableofcontents

\bigskip

\renewcommand{\thefigure}{\theenumi}
\renewcommand{\thetable}{\theenumi}
%\renewcommand{\theequation}{\theenumi}

%\begin{abstract}
%%\boldmath
%In this letter, an algorithm for evaluating the exact analytical bit error rate  (BER)  for the piecewise linear (PL) combiner for  multiple relays is presented. Previous results were available only for upto three relays. The algorithm is unique in the sense that  the actual mathematical expressions, that are prohibitively large, need not be explicitly obtained. The diversity gain due to multiple relays is shown through plots of the analytical BER, well supported by simulations. 
%
%\end{abstract}
% IEEEtran.cls defaults to using nonbold math in the Abstract.
% This preserves the distinction between vectors and scalars. However,
% if the journal you are submitting to favors bold math in the abstract,
% then you can use LaTeX's standard command \boldmath at the very start
% of the abstract to achieve this. Many IEEE journals frown on math
% in the abstract anyway.

% Note that keywords are not normally used for peerreview papers.
%\begin{IEEEkeywords}
%Cooperative diversity, decode and forward, piecewise linear
%\end{IEEEkeywords}



% For peer review papers, you can put extra information on the cover
% page as needed:
% \ifCLASSOPTIONpeerreview
% \begin{center} \bfseries EDICS Category: 3-BBND \end{center}
% \fi
%
% For peerreview papers, this IEEEtran command inserts a page break and
% creates the second title. It will be ignored for other modes.
%\IEEEpeerreviewmaketitle




 \item A bag contain 24 balls of which $x$ balls are red, $2x$ are white and $3x$ are blue. A ball is selected at random, What is the probability that it is
\begin{enumerate}[label=\alph*)]
\item not red ?
\item white ?
\end{enumerate}
%\begin{table}[H]
	\centering
\begin{tabular}{|c|c|c|}
\hline
Random variable &Value &Definition\\ \hline
\multirow{3}{*}{X} &0 &Slips of Rs 1\\
&1 &Slips of Rs 5\\
&2 &Slips of Rs 13\\ \hline
\multirow{2}{*}{Y} &0 &Box A\\
&1 &Box B\\\hline
\end{tabular}
\caption{}
\label{tab:Distribution}
\end{table}
See \tabref{tab:Distribution}.
\begin{align}
p_{Y}\brak{k}= \begin{cases} 
      \frac{1}{3} & {k=0} \\
      \frac{2}{3 }& {k=1} 
   \end{cases}
   \\
p_{Y|X}\brak{0|0} = \frac{19}{25}\, 
p_{Y|X}\brak{0|1} = \frac{6}{25}\,
p_{Y|X}\brak{1|0} = \frac{45}{50}\,
p_{Y|X}\brak{1|2} = \frac{5}{50}
\end{align}
The desired probability is the probability that a slip drawn at random is marked other than Rs 1,
\begin{align}
&=1-p_X\brak{0}\\
&= p_X(1) + p_X(2)
\end{align}
Using Bayes theorem,
\begin{align}
&= p_Y\brak{0} \times \pr{Y=0 | X=1} + p_Y\brak{1} \times \pr{Y=1|X=2}\\
&=\frac{1}{3} \times \frac{6}{25} + \frac{2}{3} \times \frac{5}{50}\\
&=\frac{11}{75}
\end{align}

\newpage

%\tableofcontents

\bigskip

\renewcommand{\thefigure}{\theenumi}
\renewcommand{\thetable}{\theenumi}
%\renewcommand{\theequation}{\theenumi}

%\begin{abstract}
%%\boldmath
%In this letter, an algorithm for evaluating the exact analytical bit error rate  (BER)  for the piecewise linear (PL) combiner for  multiple relays is presented. Previous results were available only for upto three relays. The algorithm is unique in the sense that  the actual mathematical expressions, that are prohibitively large, need not be explicitly obtained. The diversity gain due to multiple relays is shown through plots of the analytical BER, well supported by simulations. 
%
%\end{abstract}
% IEEEtran.cls defaults to using nonbold math in the Abstract.
% This preserves the distinction between vectors and scalars. However,
% if the journal you are submitting to favors bold math in the abstract,
% then you can use LaTeX's standard command \boldmath at the very start
% of the abstract to achieve this. Many IEEE journals frown on math
% in the abstract anyway.

% Note that keywords are not normally used for peerreview papers.
%\begin{IEEEkeywords}
%Cooperative diversity, decode and forward, piecewise linear
%\end{IEEEkeywords}



% For peer review papers, you can put extra information on the cover
% page as needed:
% \ifCLASSOPTIONpeerreview
% \begin{center} \bfseries EDICS Category: 3-BBND \end{center}
% \fi
%
% For peerreview papers, this IEEEtran command inserts a page break and
% creates the second title. It will be ignored for other modes.
%\IEEEpeerreviewmaketitle




If the letters of the word ASSASSINATION are arranged at random. Find the Probability that
\begin{enumerate}[label=(\alph*)]
\item Four $S's$ come consecutively in the word
\item Two  $I's$ and two $N's$ come together
\item All $A's$ are not coming together
\item No two $A's$ are coming together
\end{enumerate}
%\begin{table}[H]
	\centering
\begin{tabular}{|c|c|c|}
\hline
Random variable &Value &Definition\\ \hline
\multirow{3}{*}{X} &0 &Slips of Rs 1\\
&1 &Slips of Rs 5\\
&2 &Slips of Rs 13\\ \hline
\multirow{2}{*}{Y} &0 &Box A\\
&1 &Box B\\\hline
\end{tabular}
\caption{}
\label{tab:Distribution}
\end{table}
See \tabref{tab:Distribution}.
\begin{align}
p_{Y}\brak{k}= \begin{cases} 
      \frac{1}{3} & {k=0} \\
      \frac{2}{3 }& {k=1} 
   \end{cases}
   \\
p_{Y|X}\brak{0|0} = \frac{19}{25}\, 
p_{Y|X}\brak{0|1} = \frac{6}{25}\,
p_{Y|X}\brak{1|0} = \frac{45}{50}\,
p_{Y|X}\brak{1|2} = \frac{5}{50}
\end{align}
The desired probability is the probability that a slip drawn at random is marked other than Rs 1,
\begin{align}
&=1-p_X\brak{0}\\
&= p_X(1) + p_X(2)
\end{align}
Using Bayes theorem,
\begin{align}
&= p_Y\brak{0} \times \pr{Y=0 | X=1} + p_Y\brak{1} \times \pr{Y=1|X=2}\\
&=\frac{1}{3} \times \frac{6}{25} + \frac{2}{3} \times \frac{5}{50}\\
&=\frac{11}{75}
\end{align}

\newpage

%\tableofcontents

\bigskip

\renewcommand{\thefigure}{\theenumi}
\renewcommand{\thetable}{\theenumi}
%\renewcommand{\theequation}{\theenumi}

%\begin{abstract}
%%\boldmath
%In this letter, an algorithm for evaluating the exact analytical bit error rate  (BER)  for the piecewise linear (PL) combiner for  multiple relays is presented. Previous results were available only for upto three relays. The algorithm is unique in the sense that  the actual mathematical expressions, that are prohibitively large, need not be explicitly obtained. The diversity gain due to multiple relays is shown through plots of the analytical BER, well supported by simulations. 
%
%\end{abstract}
% IEEEtran.cls defaults to using nonbold math in the Abstract.
% This preserves the distinction between vectors and scalars. However,
% if the journal you are submitting to favors bold math in the abstract,
% then you can use LaTeX's standard command \boldmath at the very start
% of the abstract to achieve this. Many IEEE journals frown on math
% in the abstract anyway.

% Note that keywords are not normally used for peerreview papers.
%\begin{IEEEkeywords}
%Cooperative diversity, decode and forward, piecewise linear
%\end{IEEEkeywords}



% For peer review papers, you can put extra information on the cover
% page as needed:
% \ifCLASSOPTIONpeerreview
% \begin{center} \bfseries EDICS Category: 3-BBND \end{center}
% \fi
%
% For peerreview papers, this IEEEtran command inserts a page break and
% creates the second title. It will be ignored for other modes.
%\IEEEpeerreviewmaketitle




	\item One urn contains two black balls (labelled B1 and B2) and one white ball. A
	second urn contains one black ball and two white balls (labelled W1 and W2).
	Suppose the following experiment is performed. One of the two urns is chosen
	at random. Next a ball is randomly chosen from the urn. Then a second ball is
	chosen at random from the same urn without replacing the first ball.
	
	\begin{enumerate}
	\item What is the probability that two black balls are chosen?
	
	\item What is the probability that two balls of opposite colour are chosen?
	\end{enumerate}
	\solution
	%\begin{align}
    \label{eq:12.13.6.18.1}
	\because	\pr{A|B} &> \pr{A},\
\frac{\pr{AB}}{\pr{B}} > \pr{A}
\\
    \label{eq:12.13.6.18.2}
	\implies \pr{AB} &> \pr{A}\pr{B}
	\\
	\text{or, } \frac{\pr{AB}}{\pr{A}} &=\pr{B|A} > \pr{A}
\end{align}

\end{enumerate}

		%
\item 
Two cards are drawn at random and without replacement from a pack of 52 playing cards. Find the probability that both the cards are black.
\\
\solution
		%\begin{enumerate}[label=\thesection.\arabic*,ref=\thesection.\theenumi]
	\item One card is drawn from a well-shuffled deck of 52 cards. Find the probability of getting
\begin{enumerate}
\item A king of red colour 
\item A face card 
\item A red face card
\item The jack of hearts
\item A spade
\item The queen of diamonds

\end{enumerate}
\solution
		%\begin{table}[H]
	\centering
\begin{tabular}{|c|c|c|}
\hline
Random variable &Value &Definition\\ \hline
\multirow{3}{*}{X} &0 &Slips of Rs 1\\
&1 &Slips of Rs 5\\
&2 &Slips of Rs 13\\ \hline
\multirow{2}{*}{Y} &0 &Box A\\
&1 &Box B\\\hline
\end{tabular}
\caption{}
\label{tab:Distribution}
\end{table}
See \tabref{tab:Distribution}.
\begin{align}
p_{Y}\brak{k}= \begin{cases} 
      \frac{1}{3} & {k=0} \\
      \frac{2}{3 }& {k=1} 
   \end{cases}
   \\
p_{Y|X}\brak{0|0} = \frac{19}{25}\, 
p_{Y|X}\brak{0|1} = \frac{6}{25}\,
p_{Y|X}\brak{1|0} = \frac{45}{50}\,
p_{Y|X}\brak{1|2} = \frac{5}{50}
\end{align}
The desired probability is the probability that a slip drawn at random is marked other than Rs 1,
\begin{align}
&=1-p_X\brak{0}\\
&= p_X(1) + p_X(2)
\end{align}
Using Bayes theorem,
\begin{align}
&= p_Y\brak{0} \times \pr{Y=0 | X=1} + p_Y\brak{1} \times \pr{Y=1|X=2}\\
&=\frac{1}{3} \times \frac{6}{25} + \frac{2}{3} \times \frac{5}{50}\\
&=\frac{11}{75}
\end{align}

\newpage

%\tableofcontents

\bigskip

\renewcommand{\thefigure}{\theenumi}
\renewcommand{\thetable}{\theenumi}
%\renewcommand{\theequation}{\theenumi}

%\begin{abstract}
%%\boldmath
%In this letter, an algorithm for evaluating the exact analytical bit error rate  (BER)  for the piecewise linear (PL) combiner for  multiple relays is presented. Previous results were available only for upto three relays. The algorithm is unique in the sense that  the actual mathematical expressions, that are prohibitively large, need not be explicitly obtained. The diversity gain due to multiple relays is shown through plots of the analytical BER, well supported by simulations. 
%
%\end{abstract}
% IEEEtran.cls defaults to using nonbold math in the Abstract.
% This preserves the distinction between vectors and scalars. However,
% if the journal you are submitting to favors bold math in the abstract,
% then you can use LaTeX's standard command \boldmath at the very start
% of the abstract to achieve this. Many IEEE journals frown on math
% in the abstract anyway.

% Note that keywords are not normally used for peerreview papers.
%\begin{IEEEkeywords}
%Cooperative diversity, decode and forward, piecewise linear
%\end{IEEEkeywords}



% For peer review papers, you can put extra information on the cover
% page as needed:
% \ifCLASSOPTIONpeerreview
% \begin{center} \bfseries EDICS Category: 3-BBND \end{center}
% \fi
%
% For peerreview papers, this IEEEtran command inserts a page break and
% creates the second title. It will be ignored for other modes.
%\IEEEpeerreviewmaketitle




	\item Five cards—the ten, jack, queen, king and ace of diamonds, are well-shuffled with their face downwards. One card is then picked up at random.
\begin{enumerate}
\item
What is the probability that the card is the queen? 
\item
If the queen is drawn and put aside, what is the probability that the second card picked up is (a) an ace? (b) a queen?\\
\end{enumerate}
\solution
		%\begin{enumerate}[label=\thesection.\arabic*,ref=\thesection.\theenumi]
	\item One card is drawn from a well-shuffled deck of 52 cards. Find the probability of getting
\begin{enumerate}
\item A king of red colour 
\item A face card 
\item A red face card
\item The jack of hearts
\item A spade
\item The queen of diamonds

\end{enumerate}
\solution
		%\input{ncert/10/15/1/14/main.tex}
	\item Five cards—the ten, jack, queen, king and ace of diamonds, are well-shuffled with their face downwards. One card is then picked up at random.
\begin{enumerate}
\item
What is the probability that the card is the queen? 
\item
If the queen is drawn and put aside, what is the probability that the second card picked up is (a) an ace? (b) a queen?\\
\end{enumerate}
\solution
		%\input{ncert/10/15/1/15/defs.tex}
	\item A bag contains $5$ red balls and some blue balls. If the probability of drawing a blue ball is double that if a red ball, determine the number of blue balls in the bag. 
		\\
\solution
		%\input{ncert/10/15/2/3/defs.tex}
	\item A card is selected from a pack of 52 cards.
 \begin{enumerate}[label=(\alph*)] 
                 \item How many points are there in the sample space?
                 \item Calculate the probability that the card is an ace of spades.
                 \item Calculate the probability that the card is (i) an ace and (ii) black card.
 \end{enumerate}
\solution
		%\input{ncert/11/16/3/4/main.tex}
\item Four cards are drawn from a well-shuffled deck of 52 cards. What is the probability of obtaining 3 diamonds and one spade.
\\
\solution
		%\input{ncert/11/16/4/2/defs.tex}
\item In a certain lottery 10,000 tickets are sold and ten equal prizes are awarded. What is the probability of not getting a prize if you buy (a) one ticket (b) two tickets (c) 10 tickets ?	
\\
\solution
		%\input{ncert/11/16/4/4/defs.tex}
		%
\item 
Out of 100 students, two sections of 40 and 60 are formed. If you and your friend are among the 100 students, what is the probability that
\begin{enumerate}
\item you both enter the same section?
\item you both enter the different sections?
\end{enumerate}
\solution
		%\input{ncert/11/16/4/5/defs.tex}
	\item 
The number lock of a suitcase has 4 wheels each labelled with ten digits i.e. from 0 to 9.The lock opens with a sequence of four digits with no repeats.What is the probability of a person getting the right sequence to open the suitcase.
\\
\solution
		%\input{ncert/11/16/4/10/defs.tex}
		%
\item 
Two cards are drawn at random and without replacement from a pack of 52 playing cards. Find the probability that both the cards are black.
\\
\solution
		%\input{ncert/12/13/2/2/defs.tex}
		\item A box of oranges is inspected by examining three randomly selected oranges drawn without replacement. If all the three oranges are good, the box is approved for sale, otherwise, it is rejected. Find the probability that a box containing 15 oranges out of which 12 are good and 3 are bad ones will be approved for sale.
		\label{ncert/12/13/2/3/defs.tex}
		\item Two balls are drawn at random with replacement from a box containing 10 black and 8 red balls. Find the probability that
		\label{ncert/12/13/2/12}
\begin{enumerate}
\item both balls are red.
\item first ball is black and second is red.
\item one of them is black and other is red.
\end{enumerate}

\item In a hostel, 60\% of the students read Hindi newspaper, 40\% read English newspaper and 20\% read both Hindi and English newspapers. A student is selected at random.
		\label{ncert/12/13/2/15}
\begin{enumerate}
\item Find the probability that she reads neither Hindi nor English newspapers.
\item If she reads Hindi newspaper, find the probability that she reads English newspaper.
\item If she reads English newspaper, find the probability that she reads Hindi newspaper.\\
\end{enumerate}
\item The probability of obtaining an even prime number on each die, when a pair of dice is rolled is 
\begin{enumerate}
    \item $0$ 
    
    \item $\frac{1}{3}$ 
    
    \item $\frac{1}{12}$ 
    
    \item $\frac{1}{36}$ 
\end{enumerate}
\solution
		%\input{ncert/12/13/2/17/defs.tex}
	\item A bag contains 4 red and 4 black balls, another bag contains 2 red and 6 black balls. One of the two bags is selected at random and a ball is drawn from the bag which is found to be red. Find the probability that the ball is drawn from the first bag.
\\
\solution
		%\input{ncert/12/13/3/2/main.tex}
  \item
  Cards with numbers 2 to 101 are placed in a box. A card is selected at random.Find the probability that the card has
\begin{enumerate}[label=(\roman*)]
	\item an even number 
	\item a square number
\end{enumerate}
\solution
%\input{exemplar/10/13/3/32/main.tex}
\item
The king, queen and jack of clubs are removed from a deck of 52 playing cards and then well shuffled. Now one card is drawn at random from the remaining cards.  Determine the probability that the card is
\begin{enumerate}[label=(\roman*)]
\item a club
\item 10 of hearts
\end{enumerate}
\solution
%\input{exemplar/10/13/3/29/main.tex}
\item A team of medical students doing their internship have to assist during surgeries
at a city hospital. The probabilities of surgeries rated as very complex, complex,
routine, simple or very simple are respectively, 0.15, 0.20, 0.31, 0.26, .08. Find
the probabilities that a particular surgery will be rated
\begin{enumerate}
	\item complex or very complex;
	\item neither very complex nor very simple;
	\item routine or complex
	\item routine or simple
\end{enumerate}
\solution
%\input{exemplar/11/16/3/8(1)/main.tex}
\item A card is selected from a pack of 52 cards.
\begin{enumerate}[label=(\alph*)]
    \item How many points are there in the sample space?
    \item Calculate the probability that the card is an ace of spades.
    \item Calculate the probability that the card is (i) an ace and (ii) black card.
\end{enumerate}
\solution
%\input{exemplar/11/16/3/4/main2.tex}
\item The probability that a non leap year selected at random will contain 53 sundays.
\\
\solution
%\input{exemplar/10/13/1/19/main.tex}
\item One of the four persons John, Rita, Aslam or Gurpreet will be promoted next
month. Consequently the sample space consists of four elementary outcomes
S = {John promoted, Rita promoted, Aslam promoted, Gurpreet promoted}
You are told that the chances of John’s promotion is same as that of Gurpreet,
Rita’s chances of promotion are twice as likely as Johns. Aslam’s chances are
four times that of John.
\begin{enumerate}
	\item Determine
	\begin{enumerate}
		\item P (John promoted)
		\item P (Rita promoted)
		\item P (Aslam promoted)
		\item P (Gurpreet promoted)
	\end{enumerate}
	\item If A = {John promoted or Gurpreet promoted}, find P (A).
\end{enumerate}
\solution
%\input{exemplar/11/16/3/10/main.tex}
\item A card is drawn from a deck of 52 cards. Find the probability of getting a king or a heart or a red card.\\
\solution
%\input{exemplar/11/16/3/15/main.tex}
\item The probability that a student will pass his examination is 0.73, the probability of
the student getting a compartment is 0.13, and the probability that the student will
either pass or get compartment is 0.96. State True or False.\\
\solution
%\input{exemplar/11/16/3/31/main.tex}
\item A card is selected from a pack of 52 cards\\
\begin{enumerate}[label=(\alph*)]
\item How many points are there in the sample space?
\item Calculate the probability that the cards is an ace of spades.
\item Calculate the probability that the card is (i) an ace (ii)black card.\\
\end{enumerate}
%\input{ncert/11/16/3/4_1/Prob_4.tex}
\item In a non-leap year, the probability of having 53 tuesdays or 53 wednesdays is\\
\solution
%\input{exemplar/11/16/3/18/main.tex}
\item There are 1000 sealed envelopes in a box, 10 of them contain a cash prize of
Rs 100 each, 100 of them contain a cash prize of Rs 50 each and 200 of them
contain a cash prize of Rs 10 each and rest do not contain any cash prize. If they
are well shuffled and an envelope is picked up out, what is the probability that it
contains no cash prize?\\
\solution
%\input{exemplar/10/13/3/34/main.tex}
\item 
A die is thrown and a card is selected at random from a deck of 52 playing cards. The probability of getting an even number on the die and a spade card.\\
\solution
%\input{exemplar/12/13/3/78/main.tex}
\item
If 4-digit numbers greater than 5,000 are randomly formed from the digits 0, 1, 3, 5, and 7, what is the probability of forming a number divisible by 5 when:
\begin{enumerate}
    \item The digits are repeated?
    \item The repetition of digits is not allowed?
\end{enumerate}
\solution
%\input{ncert/11/16/4/9/main.tex}
\item Consider the probability space $\brak{\Omega, \mathcal{G}, P}$ where $\Omega = [0,2]$ and $\mathcal{G} = \cbrak{\phi, \Omega, [0,1], (1,2]}$. Let $X$ and $Y$ be two functions on $\Omega$ defined as
\begin{align*}
    X(\omega) = 
    \begin{cases}
        1 & \text{if }\omega \in [0, 1]\\
        2 & \text{if }\omega \in (1, 2]
    \end{cases}
\end{align*}
and
\begin{align*}
    Y(\omega) = 
    \begin{cases}
        2 & \text{if }\omega \in [0, 1.5]\\
        3 & \text{if }\omega \in (1.5, 2].
    \end{cases}
\end{align*}
Then which one of the following statements is true?
\begin{enumerate}
    \item [(A)] $X$ is a random variable with respect to $\mathcal{G}$, but $Y$ is not a random variable with respect to $\mathcal{G}$.
    \item [(B)] $Y$ is a random variable with respect to $\mathcal{G}$, but $X$ is not a random variable with respect to $\mathcal{G}$.
    \item [(C)] Neither $X$ nor $Y$ is a random variable with respect to $\mathcal{G}$.
    \item [(D)] Both $X$ and $Y$ are random variables with respect to $\mathcal{G}$.
\end{enumerate} \hfill (GATE ST 2023)\\
\solution
%\input{gate/ST/2023/14/main.tex}
	\item  A die is loaded in such a way that each odd number is twice as likely to occur as
each even number. Find $P(G)$, where $G$ is the event that a number greater than
3 occurs on a single roll of the die.
\\
\solution
		%\input{exemplar/11/16/3/5/main.tex}
	\item All the jacks, queens and kings are removed from a deck of 52 playing cards. The remaining cards are well shuffled and then one card is drawn at random. Giving ace a value 1 similar value for other cards, find the probability that the card has a value 
		\begin{enumerate}
			\item 7
			\item greater than 7
			\item less than 7
		\end{enumerate}
		%\input{exemplar/10/13/3/30/main.tex}
  \item A Lot consists of 48 mobile phones of which 42 are good, 3 have only minor defects and 3 have major defects.Varnika will buy a phone if it is good but the trader will only buy a mobile if it has no major defects. One phone is selected at random from the lot. What is the probability that it is
\begin{enumerate}
	\item acceptable to Varnika?
            \item acceptable to the trader?
\end{enumerate}
\solution
	%\input{exemplar/10/13/3/40/main.tex}
 \item A student says that if you throw a die, it will show up 1 or not 1. Therefore, the probability of getting 1 and the probability of getting 'not 1' each is equal to $\frac{1}{2}$. Is this correct? Give reasons.\\
 \solution
        %\input{exemplar/10/13/2/9/main.tex}
   \item Four candidates A, B, C, D have ap-
plied for the assignment to coach a school cricket
team. If A is twice as likely to be selected as B, and
B and C are given about the same chance of being
selected, while C is twice as likely to be selected
as D, what are the probabilities that
\begin{enumerate}
\item C will be selected?
\item A will not be selected?
\end{enumerate}
	%\input{exemplar/11/16/3/9/main.tex}
 \item A bag contain 24 balls of which $x$ balls are red, $2x$ are white and $3x$ are blue. A ball is selected at random, What is the probability that it is
\begin{enumerate}[label=\alph*)]
\item not red ?
\item white ?
\end{enumerate}
%\input{exemplar/10/13/3/41/main.tex}
If the letters of the word ASSASSINATION are arranged at random. Find the Probability that
\begin{enumerate}[label=(\alph*)]
\item Four $S's$ come consecutively in the word
\item Two  $I's$ and two $N's$ come together
\item All $A's$ are not coming together
\item No two $A's$ are coming together
\end{enumerate}
%\input{exemplar/11/16/3/14/main.tex}
	\item One urn contains two black balls (labelled B1 and B2) and one white ball. A
	second urn contains one black ball and two white balls (labelled W1 and W2).
	Suppose the following experiment is performed. One of the two urns is chosen
	at random. Next a ball is randomly chosen from the urn. Then a second ball is
	chosen at random from the same urn without replacing the first ball.
	
	\begin{enumerate}
	\item What is the probability that two black balls are chosen?
	
	\item What is the probability that two balls of opposite colour are chosen?
	\end{enumerate}
	\solution
	%\input{exemplar/11/16/3/12/main1.tex}
\end{enumerate}

	\item A bag contains $5$ red balls and some blue balls. If the probability of drawing a blue ball is double that if a red ball, determine the number of blue balls in the bag. 
		\\
\solution
		%\begin{enumerate}[label=\thesection.\arabic*,ref=\thesection.\theenumi]
	\item One card is drawn from a well-shuffled deck of 52 cards. Find the probability of getting
\begin{enumerate}
\item A king of red colour 
\item A face card 
\item A red face card
\item The jack of hearts
\item A spade
\item The queen of diamonds

\end{enumerate}
\solution
		%\input{ncert/10/15/1/14/main.tex}
	\item Five cards—the ten, jack, queen, king and ace of diamonds, are well-shuffled with their face downwards. One card is then picked up at random.
\begin{enumerate}
\item
What is the probability that the card is the queen? 
\item
If the queen is drawn and put aside, what is the probability that the second card picked up is (a) an ace? (b) a queen?\\
\end{enumerate}
\solution
		%\input{ncert/10/15/1/15/defs.tex}
	\item A bag contains $5$ red balls and some blue balls. If the probability of drawing a blue ball is double that if a red ball, determine the number of blue balls in the bag. 
		\\
\solution
		%\input{ncert/10/15/2/3/defs.tex}
	\item A card is selected from a pack of 52 cards.
 \begin{enumerate}[label=(\alph*)] 
                 \item How many points are there in the sample space?
                 \item Calculate the probability that the card is an ace of spades.
                 \item Calculate the probability that the card is (i) an ace and (ii) black card.
 \end{enumerate}
\solution
		%\input{ncert/11/16/3/4/main.tex}
\item Four cards are drawn from a well-shuffled deck of 52 cards. What is the probability of obtaining 3 diamonds and one spade.
\\
\solution
		%\input{ncert/11/16/4/2/defs.tex}
\item In a certain lottery 10,000 tickets are sold and ten equal prizes are awarded. What is the probability of not getting a prize if you buy (a) one ticket (b) two tickets (c) 10 tickets ?	
\\
\solution
		%\input{ncert/11/16/4/4/defs.tex}
		%
\item 
Out of 100 students, two sections of 40 and 60 are formed. If you and your friend are among the 100 students, what is the probability that
\begin{enumerate}
\item you both enter the same section?
\item you both enter the different sections?
\end{enumerate}
\solution
		%\input{ncert/11/16/4/5/defs.tex}
	\item 
The number lock of a suitcase has 4 wheels each labelled with ten digits i.e. from 0 to 9.The lock opens with a sequence of four digits with no repeats.What is the probability of a person getting the right sequence to open the suitcase.
\\
\solution
		%\input{ncert/11/16/4/10/defs.tex}
		%
\item 
Two cards are drawn at random and without replacement from a pack of 52 playing cards. Find the probability that both the cards are black.
\\
\solution
		%\input{ncert/12/13/2/2/defs.tex}
		\item A box of oranges is inspected by examining three randomly selected oranges drawn without replacement. If all the three oranges are good, the box is approved for sale, otherwise, it is rejected. Find the probability that a box containing 15 oranges out of which 12 are good and 3 are bad ones will be approved for sale.
		\label{ncert/12/13/2/3/defs.tex}
		\item Two balls are drawn at random with replacement from a box containing 10 black and 8 red balls. Find the probability that
		\label{ncert/12/13/2/12}
\begin{enumerate}
\item both balls are red.
\item first ball is black and second is red.
\item one of them is black and other is red.
\end{enumerate}

\item In a hostel, 60\% of the students read Hindi newspaper, 40\% read English newspaper and 20\% read both Hindi and English newspapers. A student is selected at random.
		\label{ncert/12/13/2/15}
\begin{enumerate}
\item Find the probability that she reads neither Hindi nor English newspapers.
\item If she reads Hindi newspaper, find the probability that she reads English newspaper.
\item If she reads English newspaper, find the probability that she reads Hindi newspaper.\\
\end{enumerate}
\item The probability of obtaining an even prime number on each die, when a pair of dice is rolled is 
\begin{enumerate}
    \item $0$ 
    
    \item $\frac{1}{3}$ 
    
    \item $\frac{1}{12}$ 
    
    \item $\frac{1}{36}$ 
\end{enumerate}
\solution
		%\input{ncert/12/13/2/17/defs.tex}
	\item A bag contains 4 red and 4 black balls, another bag contains 2 red and 6 black balls. One of the two bags is selected at random and a ball is drawn from the bag which is found to be red. Find the probability that the ball is drawn from the first bag.
\\
\solution
		%\input{ncert/12/13/3/2/main.tex}
  \item
  Cards with numbers 2 to 101 are placed in a box. A card is selected at random.Find the probability that the card has
\begin{enumerate}[label=(\roman*)]
	\item an even number 
	\item a square number
\end{enumerate}
\solution
%\input{exemplar/10/13/3/32/main.tex}
\item
The king, queen and jack of clubs are removed from a deck of 52 playing cards and then well shuffled. Now one card is drawn at random from the remaining cards.  Determine the probability that the card is
\begin{enumerate}[label=(\roman*)]
\item a club
\item 10 of hearts
\end{enumerate}
\solution
%\input{exemplar/10/13/3/29/main.tex}
\item A team of medical students doing their internship have to assist during surgeries
at a city hospital. The probabilities of surgeries rated as very complex, complex,
routine, simple or very simple are respectively, 0.15, 0.20, 0.31, 0.26, .08. Find
the probabilities that a particular surgery will be rated
\begin{enumerate}
	\item complex or very complex;
	\item neither very complex nor very simple;
	\item routine or complex
	\item routine or simple
\end{enumerate}
\solution
%\input{exemplar/11/16/3/8(1)/main.tex}
\item A card is selected from a pack of 52 cards.
\begin{enumerate}[label=(\alph*)]
    \item How many points are there in the sample space?
    \item Calculate the probability that the card is an ace of spades.
    \item Calculate the probability that the card is (i) an ace and (ii) black card.
\end{enumerate}
\solution
%\input{exemplar/11/16/3/4/main2.tex}
\item The probability that a non leap year selected at random will contain 53 sundays.
\\
\solution
%\input{exemplar/10/13/1/19/main.tex}
\item One of the four persons John, Rita, Aslam or Gurpreet will be promoted next
month. Consequently the sample space consists of four elementary outcomes
S = {John promoted, Rita promoted, Aslam promoted, Gurpreet promoted}
You are told that the chances of John’s promotion is same as that of Gurpreet,
Rita’s chances of promotion are twice as likely as Johns. Aslam’s chances are
four times that of John.
\begin{enumerate}
	\item Determine
	\begin{enumerate}
		\item P (John promoted)
		\item P (Rita promoted)
		\item P (Aslam promoted)
		\item P (Gurpreet promoted)
	\end{enumerate}
	\item If A = {John promoted or Gurpreet promoted}, find P (A).
\end{enumerate}
\solution
%\input{exemplar/11/16/3/10/main.tex}
\item A card is drawn from a deck of 52 cards. Find the probability of getting a king or a heart or a red card.\\
\solution
%\input{exemplar/11/16/3/15/main.tex}
\item The probability that a student will pass his examination is 0.73, the probability of
the student getting a compartment is 0.13, and the probability that the student will
either pass or get compartment is 0.96. State True or False.\\
\solution
%\input{exemplar/11/16/3/31/main.tex}
\item A card is selected from a pack of 52 cards\\
\begin{enumerate}[label=(\alph*)]
\item How many points are there in the sample space?
\item Calculate the probability that the cards is an ace of spades.
\item Calculate the probability that the card is (i) an ace (ii)black card.\\
\end{enumerate}
%\input{ncert/11/16/3/4_1/Prob_4.tex}
\item In a non-leap year, the probability of having 53 tuesdays or 53 wednesdays is\\
\solution
%\input{exemplar/11/16/3/18/main.tex}
\item There are 1000 sealed envelopes in a box, 10 of them contain a cash prize of
Rs 100 each, 100 of them contain a cash prize of Rs 50 each and 200 of them
contain a cash prize of Rs 10 each and rest do not contain any cash prize. If they
are well shuffled and an envelope is picked up out, what is the probability that it
contains no cash prize?\\
\solution
%\input{exemplar/10/13/3/34/main.tex}
\item 
A die is thrown and a card is selected at random from a deck of 52 playing cards. The probability of getting an even number on the die and a spade card.\\
\solution
%\input{exemplar/12/13/3/78/main.tex}
\item
If 4-digit numbers greater than 5,000 are randomly formed from the digits 0, 1, 3, 5, and 7, what is the probability of forming a number divisible by 5 when:
\begin{enumerate}
    \item The digits are repeated?
    \item The repetition of digits is not allowed?
\end{enumerate}
\solution
%\input{ncert/11/16/4/9/main.tex}
\item Consider the probability space $\brak{\Omega, \mathcal{G}, P}$ where $\Omega = [0,2]$ and $\mathcal{G} = \cbrak{\phi, \Omega, [0,1], (1,2]}$. Let $X$ and $Y$ be two functions on $\Omega$ defined as
\begin{align*}
    X(\omega) = 
    \begin{cases}
        1 & \text{if }\omega \in [0, 1]\\
        2 & \text{if }\omega \in (1, 2]
    \end{cases}
\end{align*}
and
\begin{align*}
    Y(\omega) = 
    \begin{cases}
        2 & \text{if }\omega \in [0, 1.5]\\
        3 & \text{if }\omega \in (1.5, 2].
    \end{cases}
\end{align*}
Then which one of the following statements is true?
\begin{enumerate}
    \item [(A)] $X$ is a random variable with respect to $\mathcal{G}$, but $Y$ is not a random variable with respect to $\mathcal{G}$.
    \item [(B)] $Y$ is a random variable with respect to $\mathcal{G}$, but $X$ is not a random variable with respect to $\mathcal{G}$.
    \item [(C)] Neither $X$ nor $Y$ is a random variable with respect to $\mathcal{G}$.
    \item [(D)] Both $X$ and $Y$ are random variables with respect to $\mathcal{G}$.
\end{enumerate} \hfill (GATE ST 2023)\\
\solution
%\input{gate/ST/2023/14/main.tex}
	\item  A die is loaded in such a way that each odd number is twice as likely to occur as
each even number. Find $P(G)$, where $G$ is the event that a number greater than
3 occurs on a single roll of the die.
\\
\solution
		%\input{exemplar/11/16/3/5/main.tex}
	\item All the jacks, queens and kings are removed from a deck of 52 playing cards. The remaining cards are well shuffled and then one card is drawn at random. Giving ace a value 1 similar value for other cards, find the probability that the card has a value 
		\begin{enumerate}
			\item 7
			\item greater than 7
			\item less than 7
		\end{enumerate}
		%\input{exemplar/10/13/3/30/main.tex}
  \item A Lot consists of 48 mobile phones of which 42 are good, 3 have only minor defects and 3 have major defects.Varnika will buy a phone if it is good but the trader will only buy a mobile if it has no major defects. One phone is selected at random from the lot. What is the probability that it is
\begin{enumerate}
	\item acceptable to Varnika?
            \item acceptable to the trader?
\end{enumerate}
\solution
	%\input{exemplar/10/13/3/40/main.tex}
 \item A student says that if you throw a die, it will show up 1 or not 1. Therefore, the probability of getting 1 and the probability of getting 'not 1' each is equal to $\frac{1}{2}$. Is this correct? Give reasons.\\
 \solution
        %\input{exemplar/10/13/2/9/main.tex}
   \item Four candidates A, B, C, D have ap-
plied for the assignment to coach a school cricket
team. If A is twice as likely to be selected as B, and
B and C are given about the same chance of being
selected, while C is twice as likely to be selected
as D, what are the probabilities that
\begin{enumerate}
\item C will be selected?
\item A will not be selected?
\end{enumerate}
	%\input{exemplar/11/16/3/9/main.tex}
 \item A bag contain 24 balls of which $x$ balls are red, $2x$ are white and $3x$ are blue. A ball is selected at random, What is the probability that it is
\begin{enumerate}[label=\alph*)]
\item not red ?
\item white ?
\end{enumerate}
%\input{exemplar/10/13/3/41/main.tex}
If the letters of the word ASSASSINATION are arranged at random. Find the Probability that
\begin{enumerate}[label=(\alph*)]
\item Four $S's$ come consecutively in the word
\item Two  $I's$ and two $N's$ come together
\item All $A's$ are not coming together
\item No two $A's$ are coming together
\end{enumerate}
%\input{exemplar/11/16/3/14/main.tex}
	\item One urn contains two black balls (labelled B1 and B2) and one white ball. A
	second urn contains one black ball and two white balls (labelled W1 and W2).
	Suppose the following experiment is performed. One of the two urns is chosen
	at random. Next a ball is randomly chosen from the urn. Then a second ball is
	chosen at random from the same urn without replacing the first ball.
	
	\begin{enumerate}
	\item What is the probability that two black balls are chosen?
	
	\item What is the probability that two balls of opposite colour are chosen?
	\end{enumerate}
	\solution
	%\input{exemplar/11/16/3/12/main1.tex}
\end{enumerate}

	\item A card is selected from a pack of 52 cards.
 \begin{enumerate}[label=(\alph*)] 
                 \item How many points are there in the sample space?
                 \item Calculate the probability that the card is an ace of spades.
                 \item Calculate the probability that the card is (i) an ace and (ii) black card.
 \end{enumerate}
\solution
		%\begin{table}[H]
	\centering
\begin{tabular}{|c|c|c|}
\hline
Random variable &Value &Definition\\ \hline
\multirow{3}{*}{X} &0 &Slips of Rs 1\\
&1 &Slips of Rs 5\\
&2 &Slips of Rs 13\\ \hline
\multirow{2}{*}{Y} &0 &Box A\\
&1 &Box B\\\hline
\end{tabular}
\caption{}
\label{tab:Distribution}
\end{table}
See \tabref{tab:Distribution}.
\begin{align}
p_{Y}\brak{k}= \begin{cases} 
      \frac{1}{3} & {k=0} \\
      \frac{2}{3 }& {k=1} 
   \end{cases}
   \\
p_{Y|X}\brak{0|0} = \frac{19}{25}\, 
p_{Y|X}\brak{0|1} = \frac{6}{25}\,
p_{Y|X}\brak{1|0} = \frac{45}{50}\,
p_{Y|X}\brak{1|2} = \frac{5}{50}
\end{align}
The desired probability is the probability that a slip drawn at random is marked other than Rs 1,
\begin{align}
&=1-p_X\brak{0}\\
&= p_X(1) + p_X(2)
\end{align}
Using Bayes theorem,
\begin{align}
&= p_Y\brak{0} \times \pr{Y=0 | X=1} + p_Y\brak{1} \times \pr{Y=1|X=2}\\
&=\frac{1}{3} \times \frac{6}{25} + \frac{2}{3} \times \frac{5}{50}\\
&=\frac{11}{75}
\end{align}

\newpage

%\tableofcontents

\bigskip

\renewcommand{\thefigure}{\theenumi}
\renewcommand{\thetable}{\theenumi}
%\renewcommand{\theequation}{\theenumi}

%\begin{abstract}
%%\boldmath
%In this letter, an algorithm for evaluating the exact analytical bit error rate  (BER)  for the piecewise linear (PL) combiner for  multiple relays is presented. Previous results were available only for upto three relays. The algorithm is unique in the sense that  the actual mathematical expressions, that are prohibitively large, need not be explicitly obtained. The diversity gain due to multiple relays is shown through plots of the analytical BER, well supported by simulations. 
%
%\end{abstract}
% IEEEtran.cls defaults to using nonbold math in the Abstract.
% This preserves the distinction between vectors and scalars. However,
% if the journal you are submitting to favors bold math in the abstract,
% then you can use LaTeX's standard command \boldmath at the very start
% of the abstract to achieve this. Many IEEE journals frown on math
% in the abstract anyway.

% Note that keywords are not normally used for peerreview papers.
%\begin{IEEEkeywords}
%Cooperative diversity, decode and forward, piecewise linear
%\end{IEEEkeywords}



% For peer review papers, you can put extra information on the cover
% page as needed:
% \ifCLASSOPTIONpeerreview
% \begin{center} \bfseries EDICS Category: 3-BBND \end{center}
% \fi
%
% For peerreview papers, this IEEEtran command inserts a page break and
% creates the second title. It will be ignored for other modes.
%\IEEEpeerreviewmaketitle




\item Four cards are drawn from a well-shuffled deck of 52 cards. What is the probability of obtaining 3 diamonds and one spade.
\\
\solution
		%\begin{enumerate}[label=\thesection.\arabic*,ref=\thesection.\theenumi]
	\item One card is drawn from a well-shuffled deck of 52 cards. Find the probability of getting
\begin{enumerate}
\item A king of red colour 
\item A face card 
\item A red face card
\item The jack of hearts
\item A spade
\item The queen of diamonds

\end{enumerate}
\solution
		%\input{ncert/10/15/1/14/main.tex}
	\item Five cards—the ten, jack, queen, king and ace of diamonds, are well-shuffled with their face downwards. One card is then picked up at random.
\begin{enumerate}
\item
What is the probability that the card is the queen? 
\item
If the queen is drawn and put aside, what is the probability that the second card picked up is (a) an ace? (b) a queen?\\
\end{enumerate}
\solution
		%\input{ncert/10/15/1/15/defs.tex}
	\item A bag contains $5$ red balls and some blue balls. If the probability of drawing a blue ball is double that if a red ball, determine the number of blue balls in the bag. 
		\\
\solution
		%\input{ncert/10/15/2/3/defs.tex}
	\item A card is selected from a pack of 52 cards.
 \begin{enumerate}[label=(\alph*)] 
                 \item How many points are there in the sample space?
                 \item Calculate the probability that the card is an ace of spades.
                 \item Calculate the probability that the card is (i) an ace and (ii) black card.
 \end{enumerate}
\solution
		%\input{ncert/11/16/3/4/main.tex}
\item Four cards are drawn from a well-shuffled deck of 52 cards. What is the probability of obtaining 3 diamonds and one spade.
\\
\solution
		%\input{ncert/11/16/4/2/defs.tex}
\item In a certain lottery 10,000 tickets are sold and ten equal prizes are awarded. What is the probability of not getting a prize if you buy (a) one ticket (b) two tickets (c) 10 tickets ?	
\\
\solution
		%\input{ncert/11/16/4/4/defs.tex}
		%
\item 
Out of 100 students, two sections of 40 and 60 are formed. If you and your friend are among the 100 students, what is the probability that
\begin{enumerate}
\item you both enter the same section?
\item you both enter the different sections?
\end{enumerate}
\solution
		%\input{ncert/11/16/4/5/defs.tex}
	\item 
The number lock of a suitcase has 4 wheels each labelled with ten digits i.e. from 0 to 9.The lock opens with a sequence of four digits with no repeats.What is the probability of a person getting the right sequence to open the suitcase.
\\
\solution
		%\input{ncert/11/16/4/10/defs.tex}
		%
\item 
Two cards are drawn at random and without replacement from a pack of 52 playing cards. Find the probability that both the cards are black.
\\
\solution
		%\input{ncert/12/13/2/2/defs.tex}
		\item A box of oranges is inspected by examining three randomly selected oranges drawn without replacement. If all the three oranges are good, the box is approved for sale, otherwise, it is rejected. Find the probability that a box containing 15 oranges out of which 12 are good and 3 are bad ones will be approved for sale.
		\label{ncert/12/13/2/3/defs.tex}
		\item Two balls are drawn at random with replacement from a box containing 10 black and 8 red balls. Find the probability that
		\label{ncert/12/13/2/12}
\begin{enumerate}
\item both balls are red.
\item first ball is black and second is red.
\item one of them is black and other is red.
\end{enumerate}

\item In a hostel, 60\% of the students read Hindi newspaper, 40\% read English newspaper and 20\% read both Hindi and English newspapers. A student is selected at random.
		\label{ncert/12/13/2/15}
\begin{enumerate}
\item Find the probability that she reads neither Hindi nor English newspapers.
\item If she reads Hindi newspaper, find the probability that she reads English newspaper.
\item If she reads English newspaper, find the probability that she reads Hindi newspaper.\\
\end{enumerate}
\item The probability of obtaining an even prime number on each die, when a pair of dice is rolled is 
\begin{enumerate}
    \item $0$ 
    
    \item $\frac{1}{3}$ 
    
    \item $\frac{1}{12}$ 
    
    \item $\frac{1}{36}$ 
\end{enumerate}
\solution
		%\input{ncert/12/13/2/17/defs.tex}
	\item A bag contains 4 red and 4 black balls, another bag contains 2 red and 6 black balls. One of the two bags is selected at random and a ball is drawn from the bag which is found to be red. Find the probability that the ball is drawn from the first bag.
\\
\solution
		%\input{ncert/12/13/3/2/main.tex}
  \item
  Cards with numbers 2 to 101 are placed in a box. A card is selected at random.Find the probability that the card has
\begin{enumerate}[label=(\roman*)]
	\item an even number 
	\item a square number
\end{enumerate}
\solution
%\input{exemplar/10/13/3/32/main.tex}
\item
The king, queen and jack of clubs are removed from a deck of 52 playing cards and then well shuffled. Now one card is drawn at random from the remaining cards.  Determine the probability that the card is
\begin{enumerate}[label=(\roman*)]
\item a club
\item 10 of hearts
\end{enumerate}
\solution
%\input{exemplar/10/13/3/29/main.tex}
\item A team of medical students doing their internship have to assist during surgeries
at a city hospital. The probabilities of surgeries rated as very complex, complex,
routine, simple or very simple are respectively, 0.15, 0.20, 0.31, 0.26, .08. Find
the probabilities that a particular surgery will be rated
\begin{enumerate}
	\item complex or very complex;
	\item neither very complex nor very simple;
	\item routine or complex
	\item routine or simple
\end{enumerate}
\solution
%\input{exemplar/11/16/3/8(1)/main.tex}
\item A card is selected from a pack of 52 cards.
\begin{enumerate}[label=(\alph*)]
    \item How many points are there in the sample space?
    \item Calculate the probability that the card is an ace of spades.
    \item Calculate the probability that the card is (i) an ace and (ii) black card.
\end{enumerate}
\solution
%\input{exemplar/11/16/3/4/main2.tex}
\item The probability that a non leap year selected at random will contain 53 sundays.
\\
\solution
%\input{exemplar/10/13/1/19/main.tex}
\item One of the four persons John, Rita, Aslam or Gurpreet will be promoted next
month. Consequently the sample space consists of four elementary outcomes
S = {John promoted, Rita promoted, Aslam promoted, Gurpreet promoted}
You are told that the chances of John’s promotion is same as that of Gurpreet,
Rita’s chances of promotion are twice as likely as Johns. Aslam’s chances are
four times that of John.
\begin{enumerate}
	\item Determine
	\begin{enumerate}
		\item P (John promoted)
		\item P (Rita promoted)
		\item P (Aslam promoted)
		\item P (Gurpreet promoted)
	\end{enumerate}
	\item If A = {John promoted or Gurpreet promoted}, find P (A).
\end{enumerate}
\solution
%\input{exemplar/11/16/3/10/main.tex}
\item A card is drawn from a deck of 52 cards. Find the probability of getting a king or a heart or a red card.\\
\solution
%\input{exemplar/11/16/3/15/main.tex}
\item The probability that a student will pass his examination is 0.73, the probability of
the student getting a compartment is 0.13, and the probability that the student will
either pass or get compartment is 0.96. State True or False.\\
\solution
%\input{exemplar/11/16/3/31/main.tex}
\item A card is selected from a pack of 52 cards\\
\begin{enumerate}[label=(\alph*)]
\item How many points are there in the sample space?
\item Calculate the probability that the cards is an ace of spades.
\item Calculate the probability that the card is (i) an ace (ii)black card.\\
\end{enumerate}
%\input{ncert/11/16/3/4_1/Prob_4.tex}
\item In a non-leap year, the probability of having 53 tuesdays or 53 wednesdays is\\
\solution
%\input{exemplar/11/16/3/18/main.tex}
\item There are 1000 sealed envelopes in a box, 10 of them contain a cash prize of
Rs 100 each, 100 of them contain a cash prize of Rs 50 each and 200 of them
contain a cash prize of Rs 10 each and rest do not contain any cash prize. If they
are well shuffled and an envelope is picked up out, what is the probability that it
contains no cash prize?\\
\solution
%\input{exemplar/10/13/3/34/main.tex}
\item 
A die is thrown and a card is selected at random from a deck of 52 playing cards. The probability of getting an even number on the die and a spade card.\\
\solution
%\input{exemplar/12/13/3/78/main.tex}
\item
If 4-digit numbers greater than 5,000 are randomly formed from the digits 0, 1, 3, 5, and 7, what is the probability of forming a number divisible by 5 when:
\begin{enumerate}
    \item The digits are repeated?
    \item The repetition of digits is not allowed?
\end{enumerate}
\solution
%\input{ncert/11/16/4/9/main.tex}
\item Consider the probability space $\brak{\Omega, \mathcal{G}, P}$ where $\Omega = [0,2]$ and $\mathcal{G} = \cbrak{\phi, \Omega, [0,1], (1,2]}$. Let $X$ and $Y$ be two functions on $\Omega$ defined as
\begin{align*}
    X(\omega) = 
    \begin{cases}
        1 & \text{if }\omega \in [0, 1]\\
        2 & \text{if }\omega \in (1, 2]
    \end{cases}
\end{align*}
and
\begin{align*}
    Y(\omega) = 
    \begin{cases}
        2 & \text{if }\omega \in [0, 1.5]\\
        3 & \text{if }\omega \in (1.5, 2].
    \end{cases}
\end{align*}
Then which one of the following statements is true?
\begin{enumerate}
    \item [(A)] $X$ is a random variable with respect to $\mathcal{G}$, but $Y$ is not a random variable with respect to $\mathcal{G}$.
    \item [(B)] $Y$ is a random variable with respect to $\mathcal{G}$, but $X$ is not a random variable with respect to $\mathcal{G}$.
    \item [(C)] Neither $X$ nor $Y$ is a random variable with respect to $\mathcal{G}$.
    \item [(D)] Both $X$ and $Y$ are random variables with respect to $\mathcal{G}$.
\end{enumerate} \hfill (GATE ST 2023)\\
\solution
%\input{gate/ST/2023/14/main.tex}
	\item  A die is loaded in such a way that each odd number is twice as likely to occur as
each even number. Find $P(G)$, where $G$ is the event that a number greater than
3 occurs on a single roll of the die.
\\
\solution
		%\input{exemplar/11/16/3/5/main.tex}
	\item All the jacks, queens and kings are removed from a deck of 52 playing cards. The remaining cards are well shuffled and then one card is drawn at random. Giving ace a value 1 similar value for other cards, find the probability that the card has a value 
		\begin{enumerate}
			\item 7
			\item greater than 7
			\item less than 7
		\end{enumerate}
		%\input{exemplar/10/13/3/30/main.tex}
  \item A Lot consists of 48 mobile phones of which 42 are good, 3 have only minor defects and 3 have major defects.Varnika will buy a phone if it is good but the trader will only buy a mobile if it has no major defects. One phone is selected at random from the lot. What is the probability that it is
\begin{enumerate}
	\item acceptable to Varnika?
            \item acceptable to the trader?
\end{enumerate}
\solution
	%\input{exemplar/10/13/3/40/main.tex}
 \item A student says that if you throw a die, it will show up 1 or not 1. Therefore, the probability of getting 1 and the probability of getting 'not 1' each is equal to $\frac{1}{2}$. Is this correct? Give reasons.\\
 \solution
        %\input{exemplar/10/13/2/9/main.tex}
   \item Four candidates A, B, C, D have ap-
plied for the assignment to coach a school cricket
team. If A is twice as likely to be selected as B, and
B and C are given about the same chance of being
selected, while C is twice as likely to be selected
as D, what are the probabilities that
\begin{enumerate}
\item C will be selected?
\item A will not be selected?
\end{enumerate}
	%\input{exemplar/11/16/3/9/main.tex}
 \item A bag contain 24 balls of which $x$ balls are red, $2x$ are white and $3x$ are blue. A ball is selected at random, What is the probability that it is
\begin{enumerate}[label=\alph*)]
\item not red ?
\item white ?
\end{enumerate}
%\input{exemplar/10/13/3/41/main.tex}
If the letters of the word ASSASSINATION are arranged at random. Find the Probability that
\begin{enumerate}[label=(\alph*)]
\item Four $S's$ come consecutively in the word
\item Two  $I's$ and two $N's$ come together
\item All $A's$ are not coming together
\item No two $A's$ are coming together
\end{enumerate}
%\input{exemplar/11/16/3/14/main.tex}
	\item One urn contains two black balls (labelled B1 and B2) and one white ball. A
	second urn contains one black ball and two white balls (labelled W1 and W2).
	Suppose the following experiment is performed. One of the two urns is chosen
	at random. Next a ball is randomly chosen from the urn. Then a second ball is
	chosen at random from the same urn without replacing the first ball.
	
	\begin{enumerate}
	\item What is the probability that two black balls are chosen?
	
	\item What is the probability that two balls of opposite colour are chosen?
	\end{enumerate}
	\solution
	%\input{exemplar/11/16/3/12/main1.tex}
\end{enumerate}

\item In a certain lottery 10,000 tickets are sold and ten equal prizes are awarded. What is the probability of not getting a prize if you buy (a) one ticket (b) two tickets (c) 10 tickets ?	
\\
\solution
		%\begin{enumerate}[label=\thesection.\arabic*,ref=\thesection.\theenumi]
	\item One card is drawn from a well-shuffled deck of 52 cards. Find the probability of getting
\begin{enumerate}
\item A king of red colour 
\item A face card 
\item A red face card
\item The jack of hearts
\item A spade
\item The queen of diamonds

\end{enumerate}
\solution
		%\input{ncert/10/15/1/14/main.tex}
	\item Five cards—the ten, jack, queen, king and ace of diamonds, are well-shuffled with their face downwards. One card is then picked up at random.
\begin{enumerate}
\item
What is the probability that the card is the queen? 
\item
If the queen is drawn and put aside, what is the probability that the second card picked up is (a) an ace? (b) a queen?\\
\end{enumerate}
\solution
		%\input{ncert/10/15/1/15/defs.tex}
	\item A bag contains $5$ red balls and some blue balls. If the probability of drawing a blue ball is double that if a red ball, determine the number of blue balls in the bag. 
		\\
\solution
		%\input{ncert/10/15/2/3/defs.tex}
	\item A card is selected from a pack of 52 cards.
 \begin{enumerate}[label=(\alph*)] 
                 \item How many points are there in the sample space?
                 \item Calculate the probability that the card is an ace of spades.
                 \item Calculate the probability that the card is (i) an ace and (ii) black card.
 \end{enumerate}
\solution
		%\input{ncert/11/16/3/4/main.tex}
\item Four cards are drawn from a well-shuffled deck of 52 cards. What is the probability of obtaining 3 diamonds and one spade.
\\
\solution
		%\input{ncert/11/16/4/2/defs.tex}
\item In a certain lottery 10,000 tickets are sold and ten equal prizes are awarded. What is the probability of not getting a prize if you buy (a) one ticket (b) two tickets (c) 10 tickets ?	
\\
\solution
		%\input{ncert/11/16/4/4/defs.tex}
		%
\item 
Out of 100 students, two sections of 40 and 60 are formed. If you and your friend are among the 100 students, what is the probability that
\begin{enumerate}
\item you both enter the same section?
\item you both enter the different sections?
\end{enumerate}
\solution
		%\input{ncert/11/16/4/5/defs.tex}
	\item 
The number lock of a suitcase has 4 wheels each labelled with ten digits i.e. from 0 to 9.The lock opens with a sequence of four digits with no repeats.What is the probability of a person getting the right sequence to open the suitcase.
\\
\solution
		%\input{ncert/11/16/4/10/defs.tex}
		%
\item 
Two cards are drawn at random and without replacement from a pack of 52 playing cards. Find the probability that both the cards are black.
\\
\solution
		%\input{ncert/12/13/2/2/defs.tex}
		\item A box of oranges is inspected by examining three randomly selected oranges drawn without replacement. If all the three oranges are good, the box is approved for sale, otherwise, it is rejected. Find the probability that a box containing 15 oranges out of which 12 are good and 3 are bad ones will be approved for sale.
		\label{ncert/12/13/2/3/defs.tex}
		\item Two balls are drawn at random with replacement from a box containing 10 black and 8 red balls. Find the probability that
		\label{ncert/12/13/2/12}
\begin{enumerate}
\item both balls are red.
\item first ball is black and second is red.
\item one of them is black and other is red.
\end{enumerate}

\item In a hostel, 60\% of the students read Hindi newspaper, 40\% read English newspaper and 20\% read both Hindi and English newspapers. A student is selected at random.
		\label{ncert/12/13/2/15}
\begin{enumerate}
\item Find the probability that she reads neither Hindi nor English newspapers.
\item If she reads Hindi newspaper, find the probability that she reads English newspaper.
\item If she reads English newspaper, find the probability that she reads Hindi newspaper.\\
\end{enumerate}
\item The probability of obtaining an even prime number on each die, when a pair of dice is rolled is 
\begin{enumerate}
    \item $0$ 
    
    \item $\frac{1}{3}$ 
    
    \item $\frac{1}{12}$ 
    
    \item $\frac{1}{36}$ 
\end{enumerate}
\solution
		%\input{ncert/12/13/2/17/defs.tex}
	\item A bag contains 4 red and 4 black balls, another bag contains 2 red and 6 black balls. One of the two bags is selected at random and a ball is drawn from the bag which is found to be red. Find the probability that the ball is drawn from the first bag.
\\
\solution
		%\input{ncert/12/13/3/2/main.tex}
  \item
  Cards with numbers 2 to 101 are placed in a box. A card is selected at random.Find the probability that the card has
\begin{enumerate}[label=(\roman*)]
	\item an even number 
	\item a square number
\end{enumerate}
\solution
%\input{exemplar/10/13/3/32/main.tex}
\item
The king, queen and jack of clubs are removed from a deck of 52 playing cards and then well shuffled. Now one card is drawn at random from the remaining cards.  Determine the probability that the card is
\begin{enumerate}[label=(\roman*)]
\item a club
\item 10 of hearts
\end{enumerate}
\solution
%\input{exemplar/10/13/3/29/main.tex}
\item A team of medical students doing their internship have to assist during surgeries
at a city hospital. The probabilities of surgeries rated as very complex, complex,
routine, simple or very simple are respectively, 0.15, 0.20, 0.31, 0.26, .08. Find
the probabilities that a particular surgery will be rated
\begin{enumerate}
	\item complex or very complex;
	\item neither very complex nor very simple;
	\item routine or complex
	\item routine or simple
\end{enumerate}
\solution
%\input{exemplar/11/16/3/8(1)/main.tex}
\item A card is selected from a pack of 52 cards.
\begin{enumerate}[label=(\alph*)]
    \item How many points are there in the sample space?
    \item Calculate the probability that the card is an ace of spades.
    \item Calculate the probability that the card is (i) an ace and (ii) black card.
\end{enumerate}
\solution
%\input{exemplar/11/16/3/4/main2.tex}
\item The probability that a non leap year selected at random will contain 53 sundays.
\\
\solution
%\input{exemplar/10/13/1/19/main.tex}
\item One of the four persons John, Rita, Aslam or Gurpreet will be promoted next
month. Consequently the sample space consists of four elementary outcomes
S = {John promoted, Rita promoted, Aslam promoted, Gurpreet promoted}
You are told that the chances of John’s promotion is same as that of Gurpreet,
Rita’s chances of promotion are twice as likely as Johns. Aslam’s chances are
four times that of John.
\begin{enumerate}
	\item Determine
	\begin{enumerate}
		\item P (John promoted)
		\item P (Rita promoted)
		\item P (Aslam promoted)
		\item P (Gurpreet promoted)
	\end{enumerate}
	\item If A = {John promoted or Gurpreet promoted}, find P (A).
\end{enumerate}
\solution
%\input{exemplar/11/16/3/10/main.tex}
\item A card is drawn from a deck of 52 cards. Find the probability of getting a king or a heart or a red card.\\
\solution
%\input{exemplar/11/16/3/15/main.tex}
\item The probability that a student will pass his examination is 0.73, the probability of
the student getting a compartment is 0.13, and the probability that the student will
either pass or get compartment is 0.96. State True or False.\\
\solution
%\input{exemplar/11/16/3/31/main.tex}
\item A card is selected from a pack of 52 cards\\
\begin{enumerate}[label=(\alph*)]
\item How many points are there in the sample space?
\item Calculate the probability that the cards is an ace of spades.
\item Calculate the probability that the card is (i) an ace (ii)black card.\\
\end{enumerate}
%\input{ncert/11/16/3/4_1/Prob_4.tex}
\item In a non-leap year, the probability of having 53 tuesdays or 53 wednesdays is\\
\solution
%\input{exemplar/11/16/3/18/main.tex}
\item There are 1000 sealed envelopes in a box, 10 of them contain a cash prize of
Rs 100 each, 100 of them contain a cash prize of Rs 50 each and 200 of them
contain a cash prize of Rs 10 each and rest do not contain any cash prize. If they
are well shuffled and an envelope is picked up out, what is the probability that it
contains no cash prize?\\
\solution
%\input{exemplar/10/13/3/34/main.tex}
\item 
A die is thrown and a card is selected at random from a deck of 52 playing cards. The probability of getting an even number on the die and a spade card.\\
\solution
%\input{exemplar/12/13/3/78/main.tex}
\item
If 4-digit numbers greater than 5,000 are randomly formed from the digits 0, 1, 3, 5, and 7, what is the probability of forming a number divisible by 5 when:
\begin{enumerate}
    \item The digits are repeated?
    \item The repetition of digits is not allowed?
\end{enumerate}
\solution
%\input{ncert/11/16/4/9/main.tex}
\item Consider the probability space $\brak{\Omega, \mathcal{G}, P}$ where $\Omega = [0,2]$ and $\mathcal{G} = \cbrak{\phi, \Omega, [0,1], (1,2]}$. Let $X$ and $Y$ be two functions on $\Omega$ defined as
\begin{align*}
    X(\omega) = 
    \begin{cases}
        1 & \text{if }\omega \in [0, 1]\\
        2 & \text{if }\omega \in (1, 2]
    \end{cases}
\end{align*}
and
\begin{align*}
    Y(\omega) = 
    \begin{cases}
        2 & \text{if }\omega \in [0, 1.5]\\
        3 & \text{if }\omega \in (1.5, 2].
    \end{cases}
\end{align*}
Then which one of the following statements is true?
\begin{enumerate}
    \item [(A)] $X$ is a random variable with respect to $\mathcal{G}$, but $Y$ is not a random variable with respect to $\mathcal{G}$.
    \item [(B)] $Y$ is a random variable with respect to $\mathcal{G}$, but $X$ is not a random variable with respect to $\mathcal{G}$.
    \item [(C)] Neither $X$ nor $Y$ is a random variable with respect to $\mathcal{G}$.
    \item [(D)] Both $X$ and $Y$ are random variables with respect to $\mathcal{G}$.
\end{enumerate} \hfill (GATE ST 2023)\\
\solution
%\input{gate/ST/2023/14/main.tex}
	\item  A die is loaded in such a way that each odd number is twice as likely to occur as
each even number. Find $P(G)$, where $G$ is the event that a number greater than
3 occurs on a single roll of the die.
\\
\solution
		%\input{exemplar/11/16/3/5/main.tex}
	\item All the jacks, queens and kings are removed from a deck of 52 playing cards. The remaining cards are well shuffled and then one card is drawn at random. Giving ace a value 1 similar value for other cards, find the probability that the card has a value 
		\begin{enumerate}
			\item 7
			\item greater than 7
			\item less than 7
		\end{enumerate}
		%\input{exemplar/10/13/3/30/main.tex}
  \item A Lot consists of 48 mobile phones of which 42 are good, 3 have only minor defects and 3 have major defects.Varnika will buy a phone if it is good but the trader will only buy a mobile if it has no major defects. One phone is selected at random from the lot. What is the probability that it is
\begin{enumerate}
	\item acceptable to Varnika?
            \item acceptable to the trader?
\end{enumerate}
\solution
	%\input{exemplar/10/13/3/40/main.tex}
 \item A student says that if you throw a die, it will show up 1 or not 1. Therefore, the probability of getting 1 and the probability of getting 'not 1' each is equal to $\frac{1}{2}$. Is this correct? Give reasons.\\
 \solution
        %\input{exemplar/10/13/2/9/main.tex}
   \item Four candidates A, B, C, D have ap-
plied for the assignment to coach a school cricket
team. If A is twice as likely to be selected as B, and
B and C are given about the same chance of being
selected, while C is twice as likely to be selected
as D, what are the probabilities that
\begin{enumerate}
\item C will be selected?
\item A will not be selected?
\end{enumerate}
	%\input{exemplar/11/16/3/9/main.tex}
 \item A bag contain 24 balls of which $x$ balls are red, $2x$ are white and $3x$ are blue. A ball is selected at random, What is the probability that it is
\begin{enumerate}[label=\alph*)]
\item not red ?
\item white ?
\end{enumerate}
%\input{exemplar/10/13/3/41/main.tex}
If the letters of the word ASSASSINATION are arranged at random. Find the Probability that
\begin{enumerate}[label=(\alph*)]
\item Four $S's$ come consecutively in the word
\item Two  $I's$ and two $N's$ come together
\item All $A's$ are not coming together
\item No two $A's$ are coming together
\end{enumerate}
%\input{exemplar/11/16/3/14/main.tex}
	\item One urn contains two black balls (labelled B1 and B2) and one white ball. A
	second urn contains one black ball and two white balls (labelled W1 and W2).
	Suppose the following experiment is performed. One of the two urns is chosen
	at random. Next a ball is randomly chosen from the urn. Then a second ball is
	chosen at random from the same urn without replacing the first ball.
	
	\begin{enumerate}
	\item What is the probability that two black balls are chosen?
	
	\item What is the probability that two balls of opposite colour are chosen?
	\end{enumerate}
	\solution
	%\input{exemplar/11/16/3/12/main1.tex}
\end{enumerate}

		%
\item 
Out of 100 students, two sections of 40 and 60 are formed. If you and your friend are among the 100 students, what is the probability that
\begin{enumerate}
\item you both enter the same section?
\item you both enter the different sections?
\end{enumerate}
\solution
		%\begin{enumerate}[label=\thesection.\arabic*,ref=\thesection.\theenumi]
	\item One card is drawn from a well-shuffled deck of 52 cards. Find the probability of getting
\begin{enumerate}
\item A king of red colour 
\item A face card 
\item A red face card
\item The jack of hearts
\item A spade
\item The queen of diamonds

\end{enumerate}
\solution
		%\input{ncert/10/15/1/14/main.tex}
	\item Five cards—the ten, jack, queen, king and ace of diamonds, are well-shuffled with their face downwards. One card is then picked up at random.
\begin{enumerate}
\item
What is the probability that the card is the queen? 
\item
If the queen is drawn and put aside, what is the probability that the second card picked up is (a) an ace? (b) a queen?\\
\end{enumerate}
\solution
		%\input{ncert/10/15/1/15/defs.tex}
	\item A bag contains $5$ red balls and some blue balls. If the probability of drawing a blue ball is double that if a red ball, determine the number of blue balls in the bag. 
		\\
\solution
		%\input{ncert/10/15/2/3/defs.tex}
	\item A card is selected from a pack of 52 cards.
 \begin{enumerate}[label=(\alph*)] 
                 \item How many points are there in the sample space?
                 \item Calculate the probability that the card is an ace of spades.
                 \item Calculate the probability that the card is (i) an ace and (ii) black card.
 \end{enumerate}
\solution
		%\input{ncert/11/16/3/4/main.tex}
\item Four cards are drawn from a well-shuffled deck of 52 cards. What is the probability of obtaining 3 diamonds and one spade.
\\
\solution
		%\input{ncert/11/16/4/2/defs.tex}
\item In a certain lottery 10,000 tickets are sold and ten equal prizes are awarded. What is the probability of not getting a prize if you buy (a) one ticket (b) two tickets (c) 10 tickets ?	
\\
\solution
		%\input{ncert/11/16/4/4/defs.tex}
		%
\item 
Out of 100 students, two sections of 40 and 60 are formed. If you and your friend are among the 100 students, what is the probability that
\begin{enumerate}
\item you both enter the same section?
\item you both enter the different sections?
\end{enumerate}
\solution
		%\input{ncert/11/16/4/5/defs.tex}
	\item 
The number lock of a suitcase has 4 wheels each labelled with ten digits i.e. from 0 to 9.The lock opens with a sequence of four digits with no repeats.What is the probability of a person getting the right sequence to open the suitcase.
\\
\solution
		%\input{ncert/11/16/4/10/defs.tex}
		%
\item 
Two cards are drawn at random and without replacement from a pack of 52 playing cards. Find the probability that both the cards are black.
\\
\solution
		%\input{ncert/12/13/2/2/defs.tex}
		\item A box of oranges is inspected by examining three randomly selected oranges drawn without replacement. If all the three oranges are good, the box is approved for sale, otherwise, it is rejected. Find the probability that a box containing 15 oranges out of which 12 are good and 3 are bad ones will be approved for sale.
		\label{ncert/12/13/2/3/defs.tex}
		\item Two balls are drawn at random with replacement from a box containing 10 black and 8 red balls. Find the probability that
		\label{ncert/12/13/2/12}
\begin{enumerate}
\item both balls are red.
\item first ball is black and second is red.
\item one of them is black and other is red.
\end{enumerate}

\item In a hostel, 60\% of the students read Hindi newspaper, 40\% read English newspaper and 20\% read both Hindi and English newspapers. A student is selected at random.
		\label{ncert/12/13/2/15}
\begin{enumerate}
\item Find the probability that she reads neither Hindi nor English newspapers.
\item If she reads Hindi newspaper, find the probability that she reads English newspaper.
\item If she reads English newspaper, find the probability that she reads Hindi newspaper.\\
\end{enumerate}
\item The probability of obtaining an even prime number on each die, when a pair of dice is rolled is 
\begin{enumerate}
    \item $0$ 
    
    \item $\frac{1}{3}$ 
    
    \item $\frac{1}{12}$ 
    
    \item $\frac{1}{36}$ 
\end{enumerate}
\solution
		%\input{ncert/12/13/2/17/defs.tex}
	\item A bag contains 4 red and 4 black balls, another bag contains 2 red and 6 black balls. One of the two bags is selected at random and a ball is drawn from the bag which is found to be red. Find the probability that the ball is drawn from the first bag.
\\
\solution
		%\input{ncert/12/13/3/2/main.tex}
  \item
  Cards with numbers 2 to 101 are placed in a box. A card is selected at random.Find the probability that the card has
\begin{enumerate}[label=(\roman*)]
	\item an even number 
	\item a square number
\end{enumerate}
\solution
%\input{exemplar/10/13/3/32/main.tex}
\item
The king, queen and jack of clubs are removed from a deck of 52 playing cards and then well shuffled. Now one card is drawn at random from the remaining cards.  Determine the probability that the card is
\begin{enumerate}[label=(\roman*)]
\item a club
\item 10 of hearts
\end{enumerate}
\solution
%\input{exemplar/10/13/3/29/main.tex}
\item A team of medical students doing their internship have to assist during surgeries
at a city hospital. The probabilities of surgeries rated as very complex, complex,
routine, simple or very simple are respectively, 0.15, 0.20, 0.31, 0.26, .08. Find
the probabilities that a particular surgery will be rated
\begin{enumerate}
	\item complex or very complex;
	\item neither very complex nor very simple;
	\item routine or complex
	\item routine or simple
\end{enumerate}
\solution
%\input{exemplar/11/16/3/8(1)/main.tex}
\item A card is selected from a pack of 52 cards.
\begin{enumerate}[label=(\alph*)]
    \item How many points are there in the sample space?
    \item Calculate the probability that the card is an ace of spades.
    \item Calculate the probability that the card is (i) an ace and (ii) black card.
\end{enumerate}
\solution
%\input{exemplar/11/16/3/4/main2.tex}
\item The probability that a non leap year selected at random will contain 53 sundays.
\\
\solution
%\input{exemplar/10/13/1/19/main.tex}
\item One of the four persons John, Rita, Aslam or Gurpreet will be promoted next
month. Consequently the sample space consists of four elementary outcomes
S = {John promoted, Rita promoted, Aslam promoted, Gurpreet promoted}
You are told that the chances of John’s promotion is same as that of Gurpreet,
Rita’s chances of promotion are twice as likely as Johns. Aslam’s chances are
four times that of John.
\begin{enumerate}
	\item Determine
	\begin{enumerate}
		\item P (John promoted)
		\item P (Rita promoted)
		\item P (Aslam promoted)
		\item P (Gurpreet promoted)
	\end{enumerate}
	\item If A = {John promoted or Gurpreet promoted}, find P (A).
\end{enumerate}
\solution
%\input{exemplar/11/16/3/10/main.tex}
\item A card is drawn from a deck of 52 cards. Find the probability of getting a king or a heart or a red card.\\
\solution
%\input{exemplar/11/16/3/15/main.tex}
\item The probability that a student will pass his examination is 0.73, the probability of
the student getting a compartment is 0.13, and the probability that the student will
either pass or get compartment is 0.96. State True or False.\\
\solution
%\input{exemplar/11/16/3/31/main.tex}
\item A card is selected from a pack of 52 cards\\
\begin{enumerate}[label=(\alph*)]
\item How many points are there in the sample space?
\item Calculate the probability that the cards is an ace of spades.
\item Calculate the probability that the card is (i) an ace (ii)black card.\\
\end{enumerate}
%\input{ncert/11/16/3/4_1/Prob_4.tex}
\item In a non-leap year, the probability of having 53 tuesdays or 53 wednesdays is\\
\solution
%\input{exemplar/11/16/3/18/main.tex}
\item There are 1000 sealed envelopes in a box, 10 of them contain a cash prize of
Rs 100 each, 100 of them contain a cash prize of Rs 50 each and 200 of them
contain a cash prize of Rs 10 each and rest do not contain any cash prize. If they
are well shuffled and an envelope is picked up out, what is the probability that it
contains no cash prize?\\
\solution
%\input{exemplar/10/13/3/34/main.tex}
\item 
A die is thrown and a card is selected at random from a deck of 52 playing cards. The probability of getting an even number on the die and a spade card.\\
\solution
%\input{exemplar/12/13/3/78/main.tex}
\item
If 4-digit numbers greater than 5,000 are randomly formed from the digits 0, 1, 3, 5, and 7, what is the probability of forming a number divisible by 5 when:
\begin{enumerate}
    \item The digits are repeated?
    \item The repetition of digits is not allowed?
\end{enumerate}
\solution
%\input{ncert/11/16/4/9/main.tex}
\item Consider the probability space $\brak{\Omega, \mathcal{G}, P}$ where $\Omega = [0,2]$ and $\mathcal{G} = \cbrak{\phi, \Omega, [0,1], (1,2]}$. Let $X$ and $Y$ be two functions on $\Omega$ defined as
\begin{align*}
    X(\omega) = 
    \begin{cases}
        1 & \text{if }\omega \in [0, 1]\\
        2 & \text{if }\omega \in (1, 2]
    \end{cases}
\end{align*}
and
\begin{align*}
    Y(\omega) = 
    \begin{cases}
        2 & \text{if }\omega \in [0, 1.5]\\
        3 & \text{if }\omega \in (1.5, 2].
    \end{cases}
\end{align*}
Then which one of the following statements is true?
\begin{enumerate}
    \item [(A)] $X$ is a random variable with respect to $\mathcal{G}$, but $Y$ is not a random variable with respect to $\mathcal{G}$.
    \item [(B)] $Y$ is a random variable with respect to $\mathcal{G}$, but $X$ is not a random variable with respect to $\mathcal{G}$.
    \item [(C)] Neither $X$ nor $Y$ is a random variable with respect to $\mathcal{G}$.
    \item [(D)] Both $X$ and $Y$ are random variables with respect to $\mathcal{G}$.
\end{enumerate} \hfill (GATE ST 2023)\\
\solution
%\input{gate/ST/2023/14/main.tex}
	\item  A die is loaded in such a way that each odd number is twice as likely to occur as
each even number. Find $P(G)$, where $G$ is the event that a number greater than
3 occurs on a single roll of the die.
\\
\solution
		%\input{exemplar/11/16/3/5/main.tex}
	\item All the jacks, queens and kings are removed from a deck of 52 playing cards. The remaining cards are well shuffled and then one card is drawn at random. Giving ace a value 1 similar value for other cards, find the probability that the card has a value 
		\begin{enumerate}
			\item 7
			\item greater than 7
			\item less than 7
		\end{enumerate}
		%\input{exemplar/10/13/3/30/main.tex}
  \item A Lot consists of 48 mobile phones of which 42 are good, 3 have only minor defects and 3 have major defects.Varnika will buy a phone if it is good but the trader will only buy a mobile if it has no major defects. One phone is selected at random from the lot. What is the probability that it is
\begin{enumerate}
	\item acceptable to Varnika?
            \item acceptable to the trader?
\end{enumerate}
\solution
	%\input{exemplar/10/13/3/40/main.tex}
 \item A student says that if you throw a die, it will show up 1 or not 1. Therefore, the probability of getting 1 and the probability of getting 'not 1' each is equal to $\frac{1}{2}$. Is this correct? Give reasons.\\
 \solution
        %\input{exemplar/10/13/2/9/main.tex}
   \item Four candidates A, B, C, D have ap-
plied for the assignment to coach a school cricket
team. If A is twice as likely to be selected as B, and
B and C are given about the same chance of being
selected, while C is twice as likely to be selected
as D, what are the probabilities that
\begin{enumerate}
\item C will be selected?
\item A will not be selected?
\end{enumerate}
	%\input{exemplar/11/16/3/9/main.tex}
 \item A bag contain 24 balls of which $x$ balls are red, $2x$ are white and $3x$ are blue. A ball is selected at random, What is the probability that it is
\begin{enumerate}[label=\alph*)]
\item not red ?
\item white ?
\end{enumerate}
%\input{exemplar/10/13/3/41/main.tex}
If the letters of the word ASSASSINATION are arranged at random. Find the Probability that
\begin{enumerate}[label=(\alph*)]
\item Four $S's$ come consecutively in the word
\item Two  $I's$ and two $N's$ come together
\item All $A's$ are not coming together
\item No two $A's$ are coming together
\end{enumerate}
%\input{exemplar/11/16/3/14/main.tex}
	\item One urn contains two black balls (labelled B1 and B2) and one white ball. A
	second urn contains one black ball and two white balls (labelled W1 and W2).
	Suppose the following experiment is performed. One of the two urns is chosen
	at random. Next a ball is randomly chosen from the urn. Then a second ball is
	chosen at random from the same urn without replacing the first ball.
	
	\begin{enumerate}
	\item What is the probability that two black balls are chosen?
	
	\item What is the probability that two balls of opposite colour are chosen?
	\end{enumerate}
	\solution
	%\input{exemplar/11/16/3/12/main1.tex}
\end{enumerate}

	\item 
The number lock of a suitcase has 4 wheels each labelled with ten digits i.e. from 0 to 9.The lock opens with a sequence of four digits with no repeats.What is the probability of a person getting the right sequence to open the suitcase.
\\
\solution
		%\begin{enumerate}[label=\thesection.\arabic*,ref=\thesection.\theenumi]
	\item One card is drawn from a well-shuffled deck of 52 cards. Find the probability of getting
\begin{enumerate}
\item A king of red colour 
\item A face card 
\item A red face card
\item The jack of hearts
\item A spade
\item The queen of diamonds

\end{enumerate}
\solution
		%\input{ncert/10/15/1/14/main.tex}
	\item Five cards—the ten, jack, queen, king and ace of diamonds, are well-shuffled with their face downwards. One card is then picked up at random.
\begin{enumerate}
\item
What is the probability that the card is the queen? 
\item
If the queen is drawn and put aside, what is the probability that the second card picked up is (a) an ace? (b) a queen?\\
\end{enumerate}
\solution
		%\input{ncert/10/15/1/15/defs.tex}
	\item A bag contains $5$ red balls and some blue balls. If the probability of drawing a blue ball is double that if a red ball, determine the number of blue balls in the bag. 
		\\
\solution
		%\input{ncert/10/15/2/3/defs.tex}
	\item A card is selected from a pack of 52 cards.
 \begin{enumerate}[label=(\alph*)] 
                 \item How many points are there in the sample space?
                 \item Calculate the probability that the card is an ace of spades.
                 \item Calculate the probability that the card is (i) an ace and (ii) black card.
 \end{enumerate}
\solution
		%\input{ncert/11/16/3/4/main.tex}
\item Four cards are drawn from a well-shuffled deck of 52 cards. What is the probability of obtaining 3 diamonds and one spade.
\\
\solution
		%\input{ncert/11/16/4/2/defs.tex}
\item In a certain lottery 10,000 tickets are sold and ten equal prizes are awarded. What is the probability of not getting a prize if you buy (a) one ticket (b) two tickets (c) 10 tickets ?	
\\
\solution
		%\input{ncert/11/16/4/4/defs.tex}
		%
\item 
Out of 100 students, two sections of 40 and 60 are formed. If you and your friend are among the 100 students, what is the probability that
\begin{enumerate}
\item you both enter the same section?
\item you both enter the different sections?
\end{enumerate}
\solution
		%\input{ncert/11/16/4/5/defs.tex}
	\item 
The number lock of a suitcase has 4 wheels each labelled with ten digits i.e. from 0 to 9.The lock opens with a sequence of four digits with no repeats.What is the probability of a person getting the right sequence to open the suitcase.
\\
\solution
		%\input{ncert/11/16/4/10/defs.tex}
		%
\item 
Two cards are drawn at random and without replacement from a pack of 52 playing cards. Find the probability that both the cards are black.
\\
\solution
		%\input{ncert/12/13/2/2/defs.tex}
		\item A box of oranges is inspected by examining three randomly selected oranges drawn without replacement. If all the three oranges are good, the box is approved for sale, otherwise, it is rejected. Find the probability that a box containing 15 oranges out of which 12 are good and 3 are bad ones will be approved for sale.
		\label{ncert/12/13/2/3/defs.tex}
		\item Two balls are drawn at random with replacement from a box containing 10 black and 8 red balls. Find the probability that
		\label{ncert/12/13/2/12}
\begin{enumerate}
\item both balls are red.
\item first ball is black and second is red.
\item one of them is black and other is red.
\end{enumerate}

\item In a hostel, 60\% of the students read Hindi newspaper, 40\% read English newspaper and 20\% read both Hindi and English newspapers. A student is selected at random.
		\label{ncert/12/13/2/15}
\begin{enumerate}
\item Find the probability that she reads neither Hindi nor English newspapers.
\item If she reads Hindi newspaper, find the probability that she reads English newspaper.
\item If she reads English newspaper, find the probability that she reads Hindi newspaper.\\
\end{enumerate}
\item The probability of obtaining an even prime number on each die, when a pair of dice is rolled is 
\begin{enumerate}
    \item $0$ 
    
    \item $\frac{1}{3}$ 
    
    \item $\frac{1}{12}$ 
    
    \item $\frac{1}{36}$ 
\end{enumerate}
\solution
		%\input{ncert/12/13/2/17/defs.tex}
	\item A bag contains 4 red and 4 black balls, another bag contains 2 red and 6 black balls. One of the two bags is selected at random and a ball is drawn from the bag which is found to be red. Find the probability that the ball is drawn from the first bag.
\\
\solution
		%\input{ncert/12/13/3/2/main.tex}
  \item
  Cards with numbers 2 to 101 are placed in a box. A card is selected at random.Find the probability that the card has
\begin{enumerate}[label=(\roman*)]
	\item an even number 
	\item a square number
\end{enumerate}
\solution
%\input{exemplar/10/13/3/32/main.tex}
\item
The king, queen and jack of clubs are removed from a deck of 52 playing cards and then well shuffled. Now one card is drawn at random from the remaining cards.  Determine the probability that the card is
\begin{enumerate}[label=(\roman*)]
\item a club
\item 10 of hearts
\end{enumerate}
\solution
%\input{exemplar/10/13/3/29/main.tex}
\item A team of medical students doing their internship have to assist during surgeries
at a city hospital. The probabilities of surgeries rated as very complex, complex,
routine, simple or very simple are respectively, 0.15, 0.20, 0.31, 0.26, .08. Find
the probabilities that a particular surgery will be rated
\begin{enumerate}
	\item complex or very complex;
	\item neither very complex nor very simple;
	\item routine or complex
	\item routine or simple
\end{enumerate}
\solution
%\input{exemplar/11/16/3/8(1)/main.tex}
\item A card is selected from a pack of 52 cards.
\begin{enumerate}[label=(\alph*)]
    \item How many points are there in the sample space?
    \item Calculate the probability that the card is an ace of spades.
    \item Calculate the probability that the card is (i) an ace and (ii) black card.
\end{enumerate}
\solution
%\input{exemplar/11/16/3/4/main2.tex}
\item The probability that a non leap year selected at random will contain 53 sundays.
\\
\solution
%\input{exemplar/10/13/1/19/main.tex}
\item One of the four persons John, Rita, Aslam or Gurpreet will be promoted next
month. Consequently the sample space consists of four elementary outcomes
S = {John promoted, Rita promoted, Aslam promoted, Gurpreet promoted}
You are told that the chances of John’s promotion is same as that of Gurpreet,
Rita’s chances of promotion are twice as likely as Johns. Aslam’s chances are
four times that of John.
\begin{enumerate}
	\item Determine
	\begin{enumerate}
		\item P (John promoted)
		\item P (Rita promoted)
		\item P (Aslam promoted)
		\item P (Gurpreet promoted)
	\end{enumerate}
	\item If A = {John promoted or Gurpreet promoted}, find P (A).
\end{enumerate}
\solution
%\input{exemplar/11/16/3/10/main.tex}
\item A card is drawn from a deck of 52 cards. Find the probability of getting a king or a heart or a red card.\\
\solution
%\input{exemplar/11/16/3/15/main.tex}
\item The probability that a student will pass his examination is 0.73, the probability of
the student getting a compartment is 0.13, and the probability that the student will
either pass or get compartment is 0.96. State True or False.\\
\solution
%\input{exemplar/11/16/3/31/main.tex}
\item A card is selected from a pack of 52 cards\\
\begin{enumerate}[label=(\alph*)]
\item How many points are there in the sample space?
\item Calculate the probability that the cards is an ace of spades.
\item Calculate the probability that the card is (i) an ace (ii)black card.\\
\end{enumerate}
%\input{ncert/11/16/3/4_1/Prob_4.tex}
\item In a non-leap year, the probability of having 53 tuesdays or 53 wednesdays is\\
\solution
%\input{exemplar/11/16/3/18/main.tex}
\item There are 1000 sealed envelopes in a box, 10 of them contain a cash prize of
Rs 100 each, 100 of them contain a cash prize of Rs 50 each and 200 of them
contain a cash prize of Rs 10 each and rest do not contain any cash prize. If they
are well shuffled and an envelope is picked up out, what is the probability that it
contains no cash prize?\\
\solution
%\input{exemplar/10/13/3/34/main.tex}
\item 
A die is thrown and a card is selected at random from a deck of 52 playing cards. The probability of getting an even number on the die and a spade card.\\
\solution
%\input{exemplar/12/13/3/78/main.tex}
\item
If 4-digit numbers greater than 5,000 are randomly formed from the digits 0, 1, 3, 5, and 7, what is the probability of forming a number divisible by 5 when:
\begin{enumerate}
    \item The digits are repeated?
    \item The repetition of digits is not allowed?
\end{enumerate}
\solution
%\input{ncert/11/16/4/9/main.tex}
\item Consider the probability space $\brak{\Omega, \mathcal{G}, P}$ where $\Omega = [0,2]$ and $\mathcal{G} = \cbrak{\phi, \Omega, [0,1], (1,2]}$. Let $X$ and $Y$ be two functions on $\Omega$ defined as
\begin{align*}
    X(\omega) = 
    \begin{cases}
        1 & \text{if }\omega \in [0, 1]\\
        2 & \text{if }\omega \in (1, 2]
    \end{cases}
\end{align*}
and
\begin{align*}
    Y(\omega) = 
    \begin{cases}
        2 & \text{if }\omega \in [0, 1.5]\\
        3 & \text{if }\omega \in (1.5, 2].
    \end{cases}
\end{align*}
Then which one of the following statements is true?
\begin{enumerate}
    \item [(A)] $X$ is a random variable with respect to $\mathcal{G}$, but $Y$ is not a random variable with respect to $\mathcal{G}$.
    \item [(B)] $Y$ is a random variable with respect to $\mathcal{G}$, but $X$ is not a random variable with respect to $\mathcal{G}$.
    \item [(C)] Neither $X$ nor $Y$ is a random variable with respect to $\mathcal{G}$.
    \item [(D)] Both $X$ and $Y$ are random variables with respect to $\mathcal{G}$.
\end{enumerate} \hfill (GATE ST 2023)\\
\solution
%\input{gate/ST/2023/14/main.tex}
	\item  A die is loaded in such a way that each odd number is twice as likely to occur as
each even number. Find $P(G)$, where $G$ is the event that a number greater than
3 occurs on a single roll of the die.
\\
\solution
		%\input{exemplar/11/16/3/5/main.tex}
	\item All the jacks, queens and kings are removed from a deck of 52 playing cards. The remaining cards are well shuffled and then one card is drawn at random. Giving ace a value 1 similar value for other cards, find the probability that the card has a value 
		\begin{enumerate}
			\item 7
			\item greater than 7
			\item less than 7
		\end{enumerate}
		%\input{exemplar/10/13/3/30/main.tex}
  \item A Lot consists of 48 mobile phones of which 42 are good, 3 have only minor defects and 3 have major defects.Varnika will buy a phone if it is good but the trader will only buy a mobile if it has no major defects. One phone is selected at random from the lot. What is the probability that it is
\begin{enumerate}
	\item acceptable to Varnika?
            \item acceptable to the trader?
\end{enumerate}
\solution
	%\input{exemplar/10/13/3/40/main.tex}
 \item A student says that if you throw a die, it will show up 1 or not 1. Therefore, the probability of getting 1 and the probability of getting 'not 1' each is equal to $\frac{1}{2}$. Is this correct? Give reasons.\\
 \solution
        %\input{exemplar/10/13/2/9/main.tex}
   \item Four candidates A, B, C, D have ap-
plied for the assignment to coach a school cricket
team. If A is twice as likely to be selected as B, and
B and C are given about the same chance of being
selected, while C is twice as likely to be selected
as D, what are the probabilities that
\begin{enumerate}
\item C will be selected?
\item A will not be selected?
\end{enumerate}
	%\input{exemplar/11/16/3/9/main.tex}
 \item A bag contain 24 balls of which $x$ balls are red, $2x$ are white and $3x$ are blue. A ball is selected at random, What is the probability that it is
\begin{enumerate}[label=\alph*)]
\item not red ?
\item white ?
\end{enumerate}
%\input{exemplar/10/13/3/41/main.tex}
If the letters of the word ASSASSINATION are arranged at random. Find the Probability that
\begin{enumerate}[label=(\alph*)]
\item Four $S's$ come consecutively in the word
\item Two  $I's$ and two $N's$ come together
\item All $A's$ are not coming together
\item No two $A's$ are coming together
\end{enumerate}
%\input{exemplar/11/16/3/14/main.tex}
	\item One urn contains two black balls (labelled B1 and B2) and one white ball. A
	second urn contains one black ball and two white balls (labelled W1 and W2).
	Suppose the following experiment is performed. One of the two urns is chosen
	at random. Next a ball is randomly chosen from the urn. Then a second ball is
	chosen at random from the same urn without replacing the first ball.
	
	\begin{enumerate}
	\item What is the probability that two black balls are chosen?
	
	\item What is the probability that two balls of opposite colour are chosen?
	\end{enumerate}
	\solution
	%\input{exemplar/11/16/3/12/main1.tex}
\end{enumerate}

		%
\item 
Two cards are drawn at random and without replacement from a pack of 52 playing cards. Find the probability that both the cards are black.
\\
\solution
		%\begin{enumerate}[label=\thesection.\arabic*,ref=\thesection.\theenumi]
	\item One card is drawn from a well-shuffled deck of 52 cards. Find the probability of getting
\begin{enumerate}
\item A king of red colour 
\item A face card 
\item A red face card
\item The jack of hearts
\item A spade
\item The queen of diamonds

\end{enumerate}
\solution
		%\input{ncert/10/15/1/14/main.tex}
	\item Five cards—the ten, jack, queen, king and ace of diamonds, are well-shuffled with their face downwards. One card is then picked up at random.
\begin{enumerate}
\item
What is the probability that the card is the queen? 
\item
If the queen is drawn and put aside, what is the probability that the second card picked up is (a) an ace? (b) a queen?\\
\end{enumerate}
\solution
		%\input{ncert/10/15/1/15/defs.tex}
	\item A bag contains $5$ red balls and some blue balls. If the probability of drawing a blue ball is double that if a red ball, determine the number of blue balls in the bag. 
		\\
\solution
		%\input{ncert/10/15/2/3/defs.tex}
	\item A card is selected from a pack of 52 cards.
 \begin{enumerate}[label=(\alph*)] 
                 \item How many points are there in the sample space?
                 \item Calculate the probability that the card is an ace of spades.
                 \item Calculate the probability that the card is (i) an ace and (ii) black card.
 \end{enumerate}
\solution
		%\input{ncert/11/16/3/4/main.tex}
\item Four cards are drawn from a well-shuffled deck of 52 cards. What is the probability of obtaining 3 diamonds and one spade.
\\
\solution
		%\input{ncert/11/16/4/2/defs.tex}
\item In a certain lottery 10,000 tickets are sold and ten equal prizes are awarded. What is the probability of not getting a prize if you buy (a) one ticket (b) two tickets (c) 10 tickets ?	
\\
\solution
		%\input{ncert/11/16/4/4/defs.tex}
		%
\item 
Out of 100 students, two sections of 40 and 60 are formed. If you and your friend are among the 100 students, what is the probability that
\begin{enumerate}
\item you both enter the same section?
\item you both enter the different sections?
\end{enumerate}
\solution
		%\input{ncert/11/16/4/5/defs.tex}
	\item 
The number lock of a suitcase has 4 wheels each labelled with ten digits i.e. from 0 to 9.The lock opens with a sequence of four digits with no repeats.What is the probability of a person getting the right sequence to open the suitcase.
\\
\solution
		%\input{ncert/11/16/4/10/defs.tex}
		%
\item 
Two cards are drawn at random and without replacement from a pack of 52 playing cards. Find the probability that both the cards are black.
\\
\solution
		%\input{ncert/12/13/2/2/defs.tex}
		\item A box of oranges is inspected by examining three randomly selected oranges drawn without replacement. If all the three oranges are good, the box is approved for sale, otherwise, it is rejected. Find the probability that a box containing 15 oranges out of which 12 are good and 3 are bad ones will be approved for sale.
		\label{ncert/12/13/2/3/defs.tex}
		\item Two balls are drawn at random with replacement from a box containing 10 black and 8 red balls. Find the probability that
		\label{ncert/12/13/2/12}
\begin{enumerate}
\item both balls are red.
\item first ball is black and second is red.
\item one of them is black and other is red.
\end{enumerate}

\item In a hostel, 60\% of the students read Hindi newspaper, 40\% read English newspaper and 20\% read both Hindi and English newspapers. A student is selected at random.
		\label{ncert/12/13/2/15}
\begin{enumerate}
\item Find the probability that she reads neither Hindi nor English newspapers.
\item If she reads Hindi newspaper, find the probability that she reads English newspaper.
\item If she reads English newspaper, find the probability that she reads Hindi newspaper.\\
\end{enumerate}
\item The probability of obtaining an even prime number on each die, when a pair of dice is rolled is 
\begin{enumerate}
    \item $0$ 
    
    \item $\frac{1}{3}$ 
    
    \item $\frac{1}{12}$ 
    
    \item $\frac{1}{36}$ 
\end{enumerate}
\solution
		%\input{ncert/12/13/2/17/defs.tex}
	\item A bag contains 4 red and 4 black balls, another bag contains 2 red and 6 black balls. One of the two bags is selected at random and a ball is drawn from the bag which is found to be red. Find the probability that the ball is drawn from the first bag.
\\
\solution
		%\input{ncert/12/13/3/2/main.tex}
  \item
  Cards with numbers 2 to 101 are placed in a box. A card is selected at random.Find the probability that the card has
\begin{enumerate}[label=(\roman*)]
	\item an even number 
	\item a square number
\end{enumerate}
\solution
%\input{exemplar/10/13/3/32/main.tex}
\item
The king, queen and jack of clubs are removed from a deck of 52 playing cards and then well shuffled. Now one card is drawn at random from the remaining cards.  Determine the probability that the card is
\begin{enumerate}[label=(\roman*)]
\item a club
\item 10 of hearts
\end{enumerate}
\solution
%\input{exemplar/10/13/3/29/main.tex}
\item A team of medical students doing their internship have to assist during surgeries
at a city hospital. The probabilities of surgeries rated as very complex, complex,
routine, simple or very simple are respectively, 0.15, 0.20, 0.31, 0.26, .08. Find
the probabilities that a particular surgery will be rated
\begin{enumerate}
	\item complex or very complex;
	\item neither very complex nor very simple;
	\item routine or complex
	\item routine or simple
\end{enumerate}
\solution
%\input{exemplar/11/16/3/8(1)/main.tex}
\item A card is selected from a pack of 52 cards.
\begin{enumerate}[label=(\alph*)]
    \item How many points are there in the sample space?
    \item Calculate the probability that the card is an ace of spades.
    \item Calculate the probability that the card is (i) an ace and (ii) black card.
\end{enumerate}
\solution
%\input{exemplar/11/16/3/4/main2.tex}
\item The probability that a non leap year selected at random will contain 53 sundays.
\\
\solution
%\input{exemplar/10/13/1/19/main.tex}
\item One of the four persons John, Rita, Aslam or Gurpreet will be promoted next
month. Consequently the sample space consists of four elementary outcomes
S = {John promoted, Rita promoted, Aslam promoted, Gurpreet promoted}
You are told that the chances of John’s promotion is same as that of Gurpreet,
Rita’s chances of promotion are twice as likely as Johns. Aslam’s chances are
four times that of John.
\begin{enumerate}
	\item Determine
	\begin{enumerate}
		\item P (John promoted)
		\item P (Rita promoted)
		\item P (Aslam promoted)
		\item P (Gurpreet promoted)
	\end{enumerate}
	\item If A = {John promoted or Gurpreet promoted}, find P (A).
\end{enumerate}
\solution
%\input{exemplar/11/16/3/10/main.tex}
\item A card is drawn from a deck of 52 cards. Find the probability of getting a king or a heart or a red card.\\
\solution
%\input{exemplar/11/16/3/15/main.tex}
\item The probability that a student will pass his examination is 0.73, the probability of
the student getting a compartment is 0.13, and the probability that the student will
either pass or get compartment is 0.96. State True or False.\\
\solution
%\input{exemplar/11/16/3/31/main.tex}
\item A card is selected from a pack of 52 cards\\
\begin{enumerate}[label=(\alph*)]
\item How many points are there in the sample space?
\item Calculate the probability that the cards is an ace of spades.
\item Calculate the probability that the card is (i) an ace (ii)black card.\\
\end{enumerate}
%\input{ncert/11/16/3/4_1/Prob_4.tex}
\item In a non-leap year, the probability of having 53 tuesdays or 53 wednesdays is\\
\solution
%\input{exemplar/11/16/3/18/main.tex}
\item There are 1000 sealed envelopes in a box, 10 of them contain a cash prize of
Rs 100 each, 100 of them contain a cash prize of Rs 50 each and 200 of them
contain a cash prize of Rs 10 each and rest do not contain any cash prize. If they
are well shuffled and an envelope is picked up out, what is the probability that it
contains no cash prize?\\
\solution
%\input{exemplar/10/13/3/34/main.tex}
\item 
A die is thrown and a card is selected at random from a deck of 52 playing cards. The probability of getting an even number on the die and a spade card.\\
\solution
%\input{exemplar/12/13/3/78/main.tex}
\item
If 4-digit numbers greater than 5,000 are randomly formed from the digits 0, 1, 3, 5, and 7, what is the probability of forming a number divisible by 5 when:
\begin{enumerate}
    \item The digits are repeated?
    \item The repetition of digits is not allowed?
\end{enumerate}
\solution
%\input{ncert/11/16/4/9/main.tex}
\item Consider the probability space $\brak{\Omega, \mathcal{G}, P}$ where $\Omega = [0,2]$ and $\mathcal{G} = \cbrak{\phi, \Omega, [0,1], (1,2]}$. Let $X$ and $Y$ be two functions on $\Omega$ defined as
\begin{align*}
    X(\omega) = 
    \begin{cases}
        1 & \text{if }\omega \in [0, 1]\\
        2 & \text{if }\omega \in (1, 2]
    \end{cases}
\end{align*}
and
\begin{align*}
    Y(\omega) = 
    \begin{cases}
        2 & \text{if }\omega \in [0, 1.5]\\
        3 & \text{if }\omega \in (1.5, 2].
    \end{cases}
\end{align*}
Then which one of the following statements is true?
\begin{enumerate}
    \item [(A)] $X$ is a random variable with respect to $\mathcal{G}$, but $Y$ is not a random variable with respect to $\mathcal{G}$.
    \item [(B)] $Y$ is a random variable with respect to $\mathcal{G}$, but $X$ is not a random variable with respect to $\mathcal{G}$.
    \item [(C)] Neither $X$ nor $Y$ is a random variable with respect to $\mathcal{G}$.
    \item [(D)] Both $X$ and $Y$ are random variables with respect to $\mathcal{G}$.
\end{enumerate} \hfill (GATE ST 2023)\\
\solution
%\input{gate/ST/2023/14/main.tex}
	\item  A die is loaded in such a way that each odd number is twice as likely to occur as
each even number. Find $P(G)$, where $G$ is the event that a number greater than
3 occurs on a single roll of the die.
\\
\solution
		%\input{exemplar/11/16/3/5/main.tex}
	\item All the jacks, queens and kings are removed from a deck of 52 playing cards. The remaining cards are well shuffled and then one card is drawn at random. Giving ace a value 1 similar value for other cards, find the probability that the card has a value 
		\begin{enumerate}
			\item 7
			\item greater than 7
			\item less than 7
		\end{enumerate}
		%\input{exemplar/10/13/3/30/main.tex}
  \item A Lot consists of 48 mobile phones of which 42 are good, 3 have only minor defects and 3 have major defects.Varnika will buy a phone if it is good but the trader will only buy a mobile if it has no major defects. One phone is selected at random from the lot. What is the probability that it is
\begin{enumerate}
	\item acceptable to Varnika?
            \item acceptable to the trader?
\end{enumerate}
\solution
	%\input{exemplar/10/13/3/40/main.tex}
 \item A student says that if you throw a die, it will show up 1 or not 1. Therefore, the probability of getting 1 and the probability of getting 'not 1' each is equal to $\frac{1}{2}$. Is this correct? Give reasons.\\
 \solution
        %\input{exemplar/10/13/2/9/main.tex}
   \item Four candidates A, B, C, D have ap-
plied for the assignment to coach a school cricket
team. If A is twice as likely to be selected as B, and
B and C are given about the same chance of being
selected, while C is twice as likely to be selected
as D, what are the probabilities that
\begin{enumerate}
\item C will be selected?
\item A will not be selected?
\end{enumerate}
	%\input{exemplar/11/16/3/9/main.tex}
 \item A bag contain 24 balls of which $x$ balls are red, $2x$ are white and $3x$ are blue. A ball is selected at random, What is the probability that it is
\begin{enumerate}[label=\alph*)]
\item not red ?
\item white ?
\end{enumerate}
%\input{exemplar/10/13/3/41/main.tex}
If the letters of the word ASSASSINATION are arranged at random. Find the Probability that
\begin{enumerate}[label=(\alph*)]
\item Four $S's$ come consecutively in the word
\item Two  $I's$ and two $N's$ come together
\item All $A's$ are not coming together
\item No two $A's$ are coming together
\end{enumerate}
%\input{exemplar/11/16/3/14/main.tex}
	\item One urn contains two black balls (labelled B1 and B2) and one white ball. A
	second urn contains one black ball and two white balls (labelled W1 and W2).
	Suppose the following experiment is performed. One of the two urns is chosen
	at random. Next a ball is randomly chosen from the urn. Then a second ball is
	chosen at random from the same urn without replacing the first ball.
	
	\begin{enumerate}
	\item What is the probability that two black balls are chosen?
	
	\item What is the probability that two balls of opposite colour are chosen?
	\end{enumerate}
	\solution
	%\input{exemplar/11/16/3/12/main1.tex}
\end{enumerate}

		\item A box of oranges is inspected by examining three randomly selected oranges drawn without replacement. If all the three oranges are good, the box is approved for sale, otherwise, it is rejected. Find the probability that a box containing 15 oranges out of which 12 are good and 3 are bad ones will be approved for sale.
		\label{ncert/12/13/2/3/defs.tex}
		\item Two balls are drawn at random with replacement from a box containing 10 black and 8 red balls. Find the probability that
		\label{ncert/12/13/2/12}
\begin{enumerate}
\item both balls are red.
\item first ball is black and second is red.
\item one of them is black and other is red.
\end{enumerate}

\item In a hostel, 60\% of the students read Hindi newspaper, 40\% read English newspaper and 20\% read both Hindi and English newspapers. A student is selected at random.
		\label{ncert/12/13/2/15}
\begin{enumerate}
\item Find the probability that she reads neither Hindi nor English newspapers.
\item If she reads Hindi newspaper, find the probability that she reads English newspaper.
\item If she reads English newspaper, find the probability that she reads Hindi newspaper.\\
\end{enumerate}
\item The probability of obtaining an even prime number on each die, when a pair of dice is rolled is 
\begin{enumerate}
    \item $0$ 
    
    \item $\frac{1}{3}$ 
    
    \item $\frac{1}{12}$ 
    
    \item $\frac{1}{36}$ 
\end{enumerate}
\solution
		%\begin{enumerate}[label=\thesection.\arabic*,ref=\thesection.\theenumi]
	\item One card is drawn from a well-shuffled deck of 52 cards. Find the probability of getting
\begin{enumerate}
\item A king of red colour 
\item A face card 
\item A red face card
\item The jack of hearts
\item A spade
\item The queen of diamonds

\end{enumerate}
\solution
		%\input{ncert/10/15/1/14/main.tex}
	\item Five cards—the ten, jack, queen, king and ace of diamonds, are well-shuffled with their face downwards. One card is then picked up at random.
\begin{enumerate}
\item
What is the probability that the card is the queen? 
\item
If the queen is drawn and put aside, what is the probability that the second card picked up is (a) an ace? (b) a queen?\\
\end{enumerate}
\solution
		%\input{ncert/10/15/1/15/defs.tex}
	\item A bag contains $5$ red balls and some blue balls. If the probability of drawing a blue ball is double that if a red ball, determine the number of blue balls in the bag. 
		\\
\solution
		%\input{ncert/10/15/2/3/defs.tex}
	\item A card is selected from a pack of 52 cards.
 \begin{enumerate}[label=(\alph*)] 
                 \item How many points are there in the sample space?
                 \item Calculate the probability that the card is an ace of spades.
                 \item Calculate the probability that the card is (i) an ace and (ii) black card.
 \end{enumerate}
\solution
		%\input{ncert/11/16/3/4/main.tex}
\item Four cards are drawn from a well-shuffled deck of 52 cards. What is the probability of obtaining 3 diamonds and one spade.
\\
\solution
		%\input{ncert/11/16/4/2/defs.tex}
\item In a certain lottery 10,000 tickets are sold and ten equal prizes are awarded. What is the probability of not getting a prize if you buy (a) one ticket (b) two tickets (c) 10 tickets ?	
\\
\solution
		%\input{ncert/11/16/4/4/defs.tex}
		%
\item 
Out of 100 students, two sections of 40 and 60 are formed. If you and your friend are among the 100 students, what is the probability that
\begin{enumerate}
\item you both enter the same section?
\item you both enter the different sections?
\end{enumerate}
\solution
		%\input{ncert/11/16/4/5/defs.tex}
	\item 
The number lock of a suitcase has 4 wheels each labelled with ten digits i.e. from 0 to 9.The lock opens with a sequence of four digits with no repeats.What is the probability of a person getting the right sequence to open the suitcase.
\\
\solution
		%\input{ncert/11/16/4/10/defs.tex}
		%
\item 
Two cards are drawn at random and without replacement from a pack of 52 playing cards. Find the probability that both the cards are black.
\\
\solution
		%\input{ncert/12/13/2/2/defs.tex}
		\item A box of oranges is inspected by examining three randomly selected oranges drawn without replacement. If all the three oranges are good, the box is approved for sale, otherwise, it is rejected. Find the probability that a box containing 15 oranges out of which 12 are good and 3 are bad ones will be approved for sale.
		\label{ncert/12/13/2/3/defs.tex}
		\item Two balls are drawn at random with replacement from a box containing 10 black and 8 red balls. Find the probability that
		\label{ncert/12/13/2/12}
\begin{enumerate}
\item both balls are red.
\item first ball is black and second is red.
\item one of them is black and other is red.
\end{enumerate}

\item In a hostel, 60\% of the students read Hindi newspaper, 40\% read English newspaper and 20\% read both Hindi and English newspapers. A student is selected at random.
		\label{ncert/12/13/2/15}
\begin{enumerate}
\item Find the probability that she reads neither Hindi nor English newspapers.
\item If she reads Hindi newspaper, find the probability that she reads English newspaper.
\item If she reads English newspaper, find the probability that she reads Hindi newspaper.\\
\end{enumerate}
\item The probability of obtaining an even prime number on each die, when a pair of dice is rolled is 
\begin{enumerate}
    \item $0$ 
    
    \item $\frac{1}{3}$ 
    
    \item $\frac{1}{12}$ 
    
    \item $\frac{1}{36}$ 
\end{enumerate}
\solution
		%\input{ncert/12/13/2/17/defs.tex}
	\item A bag contains 4 red and 4 black balls, another bag contains 2 red and 6 black balls. One of the two bags is selected at random and a ball is drawn from the bag which is found to be red. Find the probability that the ball is drawn from the first bag.
\\
\solution
		%\input{ncert/12/13/3/2/main.tex}
  \item
  Cards with numbers 2 to 101 are placed in a box. A card is selected at random.Find the probability that the card has
\begin{enumerate}[label=(\roman*)]
	\item an even number 
	\item a square number
\end{enumerate}
\solution
%\input{exemplar/10/13/3/32/main.tex}
\item
The king, queen and jack of clubs are removed from a deck of 52 playing cards and then well shuffled. Now one card is drawn at random from the remaining cards.  Determine the probability that the card is
\begin{enumerate}[label=(\roman*)]
\item a club
\item 10 of hearts
\end{enumerate}
\solution
%\input{exemplar/10/13/3/29/main.tex}
\item A team of medical students doing their internship have to assist during surgeries
at a city hospital. The probabilities of surgeries rated as very complex, complex,
routine, simple or very simple are respectively, 0.15, 0.20, 0.31, 0.26, .08. Find
the probabilities that a particular surgery will be rated
\begin{enumerate}
	\item complex or very complex;
	\item neither very complex nor very simple;
	\item routine or complex
	\item routine or simple
\end{enumerate}
\solution
%\input{exemplar/11/16/3/8(1)/main.tex}
\item A card is selected from a pack of 52 cards.
\begin{enumerate}[label=(\alph*)]
    \item How many points are there in the sample space?
    \item Calculate the probability that the card is an ace of spades.
    \item Calculate the probability that the card is (i) an ace and (ii) black card.
\end{enumerate}
\solution
%\input{exemplar/11/16/3/4/main2.tex}
\item The probability that a non leap year selected at random will contain 53 sundays.
\\
\solution
%\input{exemplar/10/13/1/19/main.tex}
\item One of the four persons John, Rita, Aslam or Gurpreet will be promoted next
month. Consequently the sample space consists of four elementary outcomes
S = {John promoted, Rita promoted, Aslam promoted, Gurpreet promoted}
You are told that the chances of John’s promotion is same as that of Gurpreet,
Rita’s chances of promotion are twice as likely as Johns. Aslam’s chances are
four times that of John.
\begin{enumerate}
	\item Determine
	\begin{enumerate}
		\item P (John promoted)
		\item P (Rita promoted)
		\item P (Aslam promoted)
		\item P (Gurpreet promoted)
	\end{enumerate}
	\item If A = {John promoted or Gurpreet promoted}, find P (A).
\end{enumerate}
\solution
%\input{exemplar/11/16/3/10/main.tex}
\item A card is drawn from a deck of 52 cards. Find the probability of getting a king or a heart or a red card.\\
\solution
%\input{exemplar/11/16/3/15/main.tex}
\item The probability that a student will pass his examination is 0.73, the probability of
the student getting a compartment is 0.13, and the probability that the student will
either pass or get compartment is 0.96. State True or False.\\
\solution
%\input{exemplar/11/16/3/31/main.tex}
\item A card is selected from a pack of 52 cards\\
\begin{enumerate}[label=(\alph*)]
\item How many points are there in the sample space?
\item Calculate the probability that the cards is an ace of spades.
\item Calculate the probability that the card is (i) an ace (ii)black card.\\
\end{enumerate}
%\input{ncert/11/16/3/4_1/Prob_4.tex}
\item In a non-leap year, the probability of having 53 tuesdays or 53 wednesdays is\\
\solution
%\input{exemplar/11/16/3/18/main.tex}
\item There are 1000 sealed envelopes in a box, 10 of them contain a cash prize of
Rs 100 each, 100 of them contain a cash prize of Rs 50 each and 200 of them
contain a cash prize of Rs 10 each and rest do not contain any cash prize. If they
are well shuffled and an envelope is picked up out, what is the probability that it
contains no cash prize?\\
\solution
%\input{exemplar/10/13/3/34/main.tex}
\item 
A die is thrown and a card is selected at random from a deck of 52 playing cards. The probability of getting an even number on the die and a spade card.\\
\solution
%\input{exemplar/12/13/3/78/main.tex}
\item
If 4-digit numbers greater than 5,000 are randomly formed from the digits 0, 1, 3, 5, and 7, what is the probability of forming a number divisible by 5 when:
\begin{enumerate}
    \item The digits are repeated?
    \item The repetition of digits is not allowed?
\end{enumerate}
\solution
%\input{ncert/11/16/4/9/main.tex}
\item Consider the probability space $\brak{\Omega, \mathcal{G}, P}$ where $\Omega = [0,2]$ and $\mathcal{G} = \cbrak{\phi, \Omega, [0,1], (1,2]}$. Let $X$ and $Y$ be two functions on $\Omega$ defined as
\begin{align*}
    X(\omega) = 
    \begin{cases}
        1 & \text{if }\omega \in [0, 1]\\
        2 & \text{if }\omega \in (1, 2]
    \end{cases}
\end{align*}
and
\begin{align*}
    Y(\omega) = 
    \begin{cases}
        2 & \text{if }\omega \in [0, 1.5]\\
        3 & \text{if }\omega \in (1.5, 2].
    \end{cases}
\end{align*}
Then which one of the following statements is true?
\begin{enumerate}
    \item [(A)] $X$ is a random variable with respect to $\mathcal{G}$, but $Y$ is not a random variable with respect to $\mathcal{G}$.
    \item [(B)] $Y$ is a random variable with respect to $\mathcal{G}$, but $X$ is not a random variable with respect to $\mathcal{G}$.
    \item [(C)] Neither $X$ nor $Y$ is a random variable with respect to $\mathcal{G}$.
    \item [(D)] Both $X$ and $Y$ are random variables with respect to $\mathcal{G}$.
\end{enumerate} \hfill (GATE ST 2023)\\
\solution
%\input{gate/ST/2023/14/main.tex}
	\item  A die is loaded in such a way that each odd number is twice as likely to occur as
each even number. Find $P(G)$, where $G$ is the event that a number greater than
3 occurs on a single roll of the die.
\\
\solution
		%\input{exemplar/11/16/3/5/main.tex}
	\item All the jacks, queens and kings are removed from a deck of 52 playing cards. The remaining cards are well shuffled and then one card is drawn at random. Giving ace a value 1 similar value for other cards, find the probability that the card has a value 
		\begin{enumerate}
			\item 7
			\item greater than 7
			\item less than 7
		\end{enumerate}
		%\input{exemplar/10/13/3/30/main.tex}
  \item A Lot consists of 48 mobile phones of which 42 are good, 3 have only minor defects and 3 have major defects.Varnika will buy a phone if it is good but the trader will only buy a mobile if it has no major defects. One phone is selected at random from the lot. What is the probability that it is
\begin{enumerate}
	\item acceptable to Varnika?
            \item acceptable to the trader?
\end{enumerate}
\solution
	%\input{exemplar/10/13/3/40/main.tex}
 \item A student says that if you throw a die, it will show up 1 or not 1. Therefore, the probability of getting 1 and the probability of getting 'not 1' each is equal to $\frac{1}{2}$. Is this correct? Give reasons.\\
 \solution
        %\input{exemplar/10/13/2/9/main.tex}
   \item Four candidates A, B, C, D have ap-
plied for the assignment to coach a school cricket
team. If A is twice as likely to be selected as B, and
B and C are given about the same chance of being
selected, while C is twice as likely to be selected
as D, what are the probabilities that
\begin{enumerate}
\item C will be selected?
\item A will not be selected?
\end{enumerate}
	%\input{exemplar/11/16/3/9/main.tex}
 \item A bag contain 24 balls of which $x$ balls are red, $2x$ are white and $3x$ are blue. A ball is selected at random, What is the probability that it is
\begin{enumerate}[label=\alph*)]
\item not red ?
\item white ?
\end{enumerate}
%\input{exemplar/10/13/3/41/main.tex}
If the letters of the word ASSASSINATION are arranged at random. Find the Probability that
\begin{enumerate}[label=(\alph*)]
\item Four $S's$ come consecutively in the word
\item Two  $I's$ and two $N's$ come together
\item All $A's$ are not coming together
\item No two $A's$ are coming together
\end{enumerate}
%\input{exemplar/11/16/3/14/main.tex}
	\item One urn contains two black balls (labelled B1 and B2) and one white ball. A
	second urn contains one black ball and two white balls (labelled W1 and W2).
	Suppose the following experiment is performed. One of the two urns is chosen
	at random. Next a ball is randomly chosen from the urn. Then a second ball is
	chosen at random from the same urn without replacing the first ball.
	
	\begin{enumerate}
	\item What is the probability that two black balls are chosen?
	
	\item What is the probability that two balls of opposite colour are chosen?
	\end{enumerate}
	\solution
	%\input{exemplar/11/16/3/12/main1.tex}
\end{enumerate}

	\item A bag contains 4 red and 4 black balls, another bag contains 2 red and 6 black balls. One of the two bags is selected at random and a ball is drawn from the bag which is found to be red. Find the probability that the ball is drawn from the first bag.
\\
\solution
		%\begin{table}[H]
	\centering
\begin{tabular}{|c|c|c|}
\hline
Random variable &Value &Definition\\ \hline
\multirow{3}{*}{X} &0 &Slips of Rs 1\\
&1 &Slips of Rs 5\\
&2 &Slips of Rs 13\\ \hline
\multirow{2}{*}{Y} &0 &Box A\\
&1 &Box B\\\hline
\end{tabular}
\caption{}
\label{tab:Distribution}
\end{table}
See \tabref{tab:Distribution}.
\begin{align}
p_{Y}\brak{k}= \begin{cases} 
      \frac{1}{3} & {k=0} \\
      \frac{2}{3 }& {k=1} 
   \end{cases}
   \\
p_{Y|X}\brak{0|0} = \frac{19}{25}\, 
p_{Y|X}\brak{0|1} = \frac{6}{25}\,
p_{Y|X}\brak{1|0} = \frac{45}{50}\,
p_{Y|X}\brak{1|2} = \frac{5}{50}
\end{align}
The desired probability is the probability that a slip drawn at random is marked other than Rs 1,
\begin{align}
&=1-p_X\brak{0}\\
&= p_X(1) + p_X(2)
\end{align}
Using Bayes theorem,
\begin{align}
&= p_Y\brak{0} \times \pr{Y=0 | X=1} + p_Y\brak{1} \times \pr{Y=1|X=2}\\
&=\frac{1}{3} \times \frac{6}{25} + \frac{2}{3} \times \frac{5}{50}\\
&=\frac{11}{75}
\end{align}

\newpage

%\tableofcontents

\bigskip

\renewcommand{\thefigure}{\theenumi}
\renewcommand{\thetable}{\theenumi}
%\renewcommand{\theequation}{\theenumi}

%\begin{abstract}
%%\boldmath
%In this letter, an algorithm for evaluating the exact analytical bit error rate  (BER)  for the piecewise linear (PL) combiner for  multiple relays is presented. Previous results were available only for upto three relays. The algorithm is unique in the sense that  the actual mathematical expressions, that are prohibitively large, need not be explicitly obtained. The diversity gain due to multiple relays is shown through plots of the analytical BER, well supported by simulations. 
%
%\end{abstract}
% IEEEtran.cls defaults to using nonbold math in the Abstract.
% This preserves the distinction between vectors and scalars. However,
% if the journal you are submitting to favors bold math in the abstract,
% then you can use LaTeX's standard command \boldmath at the very start
% of the abstract to achieve this. Many IEEE journals frown on math
% in the abstract anyway.

% Note that keywords are not normally used for peerreview papers.
%\begin{IEEEkeywords}
%Cooperative diversity, decode and forward, piecewise linear
%\end{IEEEkeywords}



% For peer review papers, you can put extra information on the cover
% page as needed:
% \ifCLASSOPTIONpeerreview
% \begin{center} \bfseries EDICS Category: 3-BBND \end{center}
% \fi
%
% For peerreview papers, this IEEEtran command inserts a page break and
% creates the second title. It will be ignored for other modes.
%\IEEEpeerreviewmaketitle




  \item
  Cards with numbers 2 to 101 are placed in a box. A card is selected at random.Find the probability that the card has
\begin{enumerate}[label=(\roman*)]
	\item an even number 
	\item a square number
\end{enumerate}
\solution
%\begin{table}[H]
	\centering
\begin{tabular}{|c|c|c|}
\hline
Random variable &Value &Definition\\ \hline
\multirow{3}{*}{X} &0 &Slips of Rs 1\\
&1 &Slips of Rs 5\\
&2 &Slips of Rs 13\\ \hline
\multirow{2}{*}{Y} &0 &Box A\\
&1 &Box B\\\hline
\end{tabular}
\caption{}
\label{tab:Distribution}
\end{table}
See \tabref{tab:Distribution}.
\begin{align}
p_{Y}\brak{k}= \begin{cases} 
      \frac{1}{3} & {k=0} \\
      \frac{2}{3 }& {k=1} 
   \end{cases}
   \\
p_{Y|X}\brak{0|0} = \frac{19}{25}\, 
p_{Y|X}\brak{0|1} = \frac{6}{25}\,
p_{Y|X}\brak{1|0} = \frac{45}{50}\,
p_{Y|X}\brak{1|2} = \frac{5}{50}
\end{align}
The desired probability is the probability that a slip drawn at random is marked other than Rs 1,
\begin{align}
&=1-p_X\brak{0}\\
&= p_X(1) + p_X(2)
\end{align}
Using Bayes theorem,
\begin{align}
&= p_Y\brak{0} \times \pr{Y=0 | X=1} + p_Y\brak{1} \times \pr{Y=1|X=2}\\
&=\frac{1}{3} \times \frac{6}{25} + \frac{2}{3} \times \frac{5}{50}\\
&=\frac{11}{75}
\end{align}

\newpage

%\tableofcontents

\bigskip

\renewcommand{\thefigure}{\theenumi}
\renewcommand{\thetable}{\theenumi}
%\renewcommand{\theequation}{\theenumi}

%\begin{abstract}
%%\boldmath
%In this letter, an algorithm for evaluating the exact analytical bit error rate  (BER)  for the piecewise linear (PL) combiner for  multiple relays is presented. Previous results were available only for upto three relays. The algorithm is unique in the sense that  the actual mathematical expressions, that are prohibitively large, need not be explicitly obtained. The diversity gain due to multiple relays is shown through plots of the analytical BER, well supported by simulations. 
%
%\end{abstract}
% IEEEtran.cls defaults to using nonbold math in the Abstract.
% This preserves the distinction between vectors and scalars. However,
% if the journal you are submitting to favors bold math in the abstract,
% then you can use LaTeX's standard command \boldmath at the very start
% of the abstract to achieve this. Many IEEE journals frown on math
% in the abstract anyway.

% Note that keywords are not normally used for peerreview papers.
%\begin{IEEEkeywords}
%Cooperative diversity, decode and forward, piecewise linear
%\end{IEEEkeywords}



% For peer review papers, you can put extra information on the cover
% page as needed:
% \ifCLASSOPTIONpeerreview
% \begin{center} \bfseries EDICS Category: 3-BBND \end{center}
% \fi
%
% For peerreview papers, this IEEEtran command inserts a page break and
% creates the second title. It will be ignored for other modes.
%\IEEEpeerreviewmaketitle




\item
The king, queen and jack of clubs are removed from a deck of 52 playing cards and then well shuffled. Now one card is drawn at random from the remaining cards.  Determine the probability that the card is
\begin{enumerate}[label=(\roman*)]
\item a club
\item 10 of hearts
\end{enumerate}
\solution
%\begin{table}[H]
	\centering
\begin{tabular}{|c|c|c|}
\hline
Random variable &Value &Definition\\ \hline
\multirow{3}{*}{X} &0 &Slips of Rs 1\\
&1 &Slips of Rs 5\\
&2 &Slips of Rs 13\\ \hline
\multirow{2}{*}{Y} &0 &Box A\\
&1 &Box B\\\hline
\end{tabular}
\caption{}
\label{tab:Distribution}
\end{table}
See \tabref{tab:Distribution}.
\begin{align}
p_{Y}\brak{k}= \begin{cases} 
      \frac{1}{3} & {k=0} \\
      \frac{2}{3 }& {k=1} 
   \end{cases}
   \\
p_{Y|X}\brak{0|0} = \frac{19}{25}\, 
p_{Y|X}\brak{0|1} = \frac{6}{25}\,
p_{Y|X}\brak{1|0} = \frac{45}{50}\,
p_{Y|X}\brak{1|2} = \frac{5}{50}
\end{align}
The desired probability is the probability that a slip drawn at random is marked other than Rs 1,
\begin{align}
&=1-p_X\brak{0}\\
&= p_X(1) + p_X(2)
\end{align}
Using Bayes theorem,
\begin{align}
&= p_Y\brak{0} \times \pr{Y=0 | X=1} + p_Y\brak{1} \times \pr{Y=1|X=2}\\
&=\frac{1}{3} \times \frac{6}{25} + \frac{2}{3} \times \frac{5}{50}\\
&=\frac{11}{75}
\end{align}

\newpage

%\tableofcontents

\bigskip

\renewcommand{\thefigure}{\theenumi}
\renewcommand{\thetable}{\theenumi}
%\renewcommand{\theequation}{\theenumi}

%\begin{abstract}
%%\boldmath
%In this letter, an algorithm for evaluating the exact analytical bit error rate  (BER)  for the piecewise linear (PL) combiner for  multiple relays is presented. Previous results were available only for upto three relays. The algorithm is unique in the sense that  the actual mathematical expressions, that are prohibitively large, need not be explicitly obtained. The diversity gain due to multiple relays is shown through plots of the analytical BER, well supported by simulations. 
%
%\end{abstract}
% IEEEtran.cls defaults to using nonbold math in the Abstract.
% This preserves the distinction between vectors and scalars. However,
% if the journal you are submitting to favors bold math in the abstract,
% then you can use LaTeX's standard command \boldmath at the very start
% of the abstract to achieve this. Many IEEE journals frown on math
% in the abstract anyway.

% Note that keywords are not normally used for peerreview papers.
%\begin{IEEEkeywords}
%Cooperative diversity, decode and forward, piecewise linear
%\end{IEEEkeywords}



% For peer review papers, you can put extra information on the cover
% page as needed:
% \ifCLASSOPTIONpeerreview
% \begin{center} \bfseries EDICS Category: 3-BBND \end{center}
% \fi
%
% For peerreview papers, this IEEEtran command inserts a page break and
% creates the second title. It will be ignored for other modes.
%\IEEEpeerreviewmaketitle




\item A team of medical students doing their internship have to assist during surgeries
at a city hospital. The probabilities of surgeries rated as very complex, complex,
routine, simple or very simple are respectively, 0.15, 0.20, 0.31, 0.26, .08. Find
the probabilities that a particular surgery will be rated
\begin{enumerate}
	\item complex or very complex;
	\item neither very complex nor very simple;
	\item routine or complex
	\item routine or simple
\end{enumerate}
\solution
%\begin{table}[H]
	\centering
\begin{tabular}{|c|c|c|}
\hline
Random variable &Value &Definition\\ \hline
\multirow{3}{*}{X} &0 &Slips of Rs 1\\
&1 &Slips of Rs 5\\
&2 &Slips of Rs 13\\ \hline
\multirow{2}{*}{Y} &0 &Box A\\
&1 &Box B\\\hline
\end{tabular}
\caption{}
\label{tab:Distribution}
\end{table}
See \tabref{tab:Distribution}.
\begin{align}
p_{Y}\brak{k}= \begin{cases} 
      \frac{1}{3} & {k=0} \\
      \frac{2}{3 }& {k=1} 
   \end{cases}
   \\
p_{Y|X}\brak{0|0} = \frac{19}{25}\, 
p_{Y|X}\brak{0|1} = \frac{6}{25}\,
p_{Y|X}\brak{1|0} = \frac{45}{50}\,
p_{Y|X}\brak{1|2} = \frac{5}{50}
\end{align}
The desired probability is the probability that a slip drawn at random is marked other than Rs 1,
\begin{align}
&=1-p_X\brak{0}\\
&= p_X(1) + p_X(2)
\end{align}
Using Bayes theorem,
\begin{align}
&= p_Y\brak{0} \times \pr{Y=0 | X=1} + p_Y\brak{1} \times \pr{Y=1|X=2}\\
&=\frac{1}{3} \times \frac{6}{25} + \frac{2}{3} \times \frac{5}{50}\\
&=\frac{11}{75}
\end{align}

\newpage

%\tableofcontents

\bigskip

\renewcommand{\thefigure}{\theenumi}
\renewcommand{\thetable}{\theenumi}
%\renewcommand{\theequation}{\theenumi}

%\begin{abstract}
%%\boldmath
%In this letter, an algorithm for evaluating the exact analytical bit error rate  (BER)  for the piecewise linear (PL) combiner for  multiple relays is presented. Previous results were available only for upto three relays. The algorithm is unique in the sense that  the actual mathematical expressions, that are prohibitively large, need not be explicitly obtained. The diversity gain due to multiple relays is shown through plots of the analytical BER, well supported by simulations. 
%
%\end{abstract}
% IEEEtran.cls defaults to using nonbold math in the Abstract.
% This preserves the distinction between vectors and scalars. However,
% if the journal you are submitting to favors bold math in the abstract,
% then you can use LaTeX's standard command \boldmath at the very start
% of the abstract to achieve this. Many IEEE journals frown on math
% in the abstract anyway.

% Note that keywords are not normally used for peerreview papers.
%\begin{IEEEkeywords}
%Cooperative diversity, decode and forward, piecewise linear
%\end{IEEEkeywords}



% For peer review papers, you can put extra information on the cover
% page as needed:
% \ifCLASSOPTIONpeerreview
% \begin{center} \bfseries EDICS Category: 3-BBND \end{center}
% \fi
%
% For peerreview papers, this IEEEtran command inserts a page break and
% creates the second title. It will be ignored for other modes.
%\IEEEpeerreviewmaketitle




\item A card is selected from a pack of 52 cards.
\begin{enumerate}[label=(\alph*)]
    \item How many points are there in the sample space?
    \item Calculate the probability that the card is an ace of spades.
    \item Calculate the probability that the card is (i) an ace and (ii) black card.
\end{enumerate}
\solution
%Let $X$ be an bernoulli rv defined as in \tabref{tab:exemplar/11/16/3/26}.  Then, 
\begin{equation}
    p =
        \frac{4}{11} 
\end{equation}
\begin{table}[H]
	\centering
	\input{exemplar/11/16/3/26/tables/Table2.tex}
	\caption{}
        \label{tab:exemplar/11/16/3/26}
\end{table}

\item The probability that a non leap year selected at random will contain 53 sundays.
\\
\solution
%\begin{table}[H]
	\centering
\begin{tabular}{|c|c|c|}
\hline
Random variable &Value &Definition\\ \hline
\multirow{3}{*}{X} &0 &Slips of Rs 1\\
&1 &Slips of Rs 5\\
&2 &Slips of Rs 13\\ \hline
\multirow{2}{*}{Y} &0 &Box A\\
&1 &Box B\\\hline
\end{tabular}
\caption{}
\label{tab:Distribution}
\end{table}
See \tabref{tab:Distribution}.
\begin{align}
p_{Y}\brak{k}= \begin{cases} 
      \frac{1}{3} & {k=0} \\
      \frac{2}{3 }& {k=1} 
   \end{cases}
   \\
p_{Y|X}\brak{0|0} = \frac{19}{25}\, 
p_{Y|X}\brak{0|1} = \frac{6}{25}\,
p_{Y|X}\brak{1|0} = \frac{45}{50}\,
p_{Y|X}\brak{1|2} = \frac{5}{50}
\end{align}
The desired probability is the probability that a slip drawn at random is marked other than Rs 1,
\begin{align}
&=1-p_X\brak{0}\\
&= p_X(1) + p_X(2)
\end{align}
Using Bayes theorem,
\begin{align}
&= p_Y\brak{0} \times \pr{Y=0 | X=1} + p_Y\brak{1} \times \pr{Y=1|X=2}\\
&=\frac{1}{3} \times \frac{6}{25} + \frac{2}{3} \times \frac{5}{50}\\
&=\frac{11}{75}
\end{align}

\newpage

%\tableofcontents

\bigskip

\renewcommand{\thefigure}{\theenumi}
\renewcommand{\thetable}{\theenumi}
%\renewcommand{\theequation}{\theenumi}

%\begin{abstract}
%%\boldmath
%In this letter, an algorithm for evaluating the exact analytical bit error rate  (BER)  for the piecewise linear (PL) combiner for  multiple relays is presented. Previous results were available only for upto three relays. The algorithm is unique in the sense that  the actual mathematical expressions, that are prohibitively large, need not be explicitly obtained. The diversity gain due to multiple relays is shown through plots of the analytical BER, well supported by simulations. 
%
%\end{abstract}
% IEEEtran.cls defaults to using nonbold math in the Abstract.
% This preserves the distinction between vectors and scalars. However,
% if the journal you are submitting to favors bold math in the abstract,
% then you can use LaTeX's standard command \boldmath at the very start
% of the abstract to achieve this. Many IEEE journals frown on math
% in the abstract anyway.

% Note that keywords are not normally used for peerreview papers.
%\begin{IEEEkeywords}
%Cooperative diversity, decode and forward, piecewise linear
%\end{IEEEkeywords}



% For peer review papers, you can put extra information on the cover
% page as needed:
% \ifCLASSOPTIONpeerreview
% \begin{center} \bfseries EDICS Category: 3-BBND \end{center}
% \fi
%
% For peerreview papers, this IEEEtran command inserts a page break and
% creates the second title. It will be ignored for other modes.
%\IEEEpeerreviewmaketitle




\item One of the four persons John, Rita, Aslam or Gurpreet will be promoted next
month. Consequently the sample space consists of four elementary outcomes
S = {John promoted, Rita promoted, Aslam promoted, Gurpreet promoted}
You are told that the chances of John’s promotion is same as that of Gurpreet,
Rita’s chances of promotion are twice as likely as Johns. Aslam’s chances are
four times that of John.
\begin{enumerate}
	\item Determine
	\begin{enumerate}
		\item P (John promoted)
		\item P (Rita promoted)
		\item P (Aslam promoted)
		\item P (Gurpreet promoted)
	\end{enumerate}
	\item If A = {John promoted or Gurpreet promoted}, find P (A).
\end{enumerate}
\solution
%\begin{table}[H]
	\centering
\begin{tabular}{|c|c|c|}
\hline
Random variable &Value &Definition\\ \hline
\multirow{3}{*}{X} &0 &Slips of Rs 1\\
&1 &Slips of Rs 5\\
&2 &Slips of Rs 13\\ \hline
\multirow{2}{*}{Y} &0 &Box A\\
&1 &Box B\\\hline
\end{tabular}
\caption{}
\label{tab:Distribution}
\end{table}
See \tabref{tab:Distribution}.
\begin{align}
p_{Y}\brak{k}= \begin{cases} 
      \frac{1}{3} & {k=0} \\
      \frac{2}{3 }& {k=1} 
   \end{cases}
   \\
p_{Y|X}\brak{0|0} = \frac{19}{25}\, 
p_{Y|X}\brak{0|1} = \frac{6}{25}\,
p_{Y|X}\brak{1|0} = \frac{45}{50}\,
p_{Y|X}\brak{1|2} = \frac{5}{50}
\end{align}
The desired probability is the probability that a slip drawn at random is marked other than Rs 1,
\begin{align}
&=1-p_X\brak{0}\\
&= p_X(1) + p_X(2)
\end{align}
Using Bayes theorem,
\begin{align}
&= p_Y\brak{0} \times \pr{Y=0 | X=1} + p_Y\brak{1} \times \pr{Y=1|X=2}\\
&=\frac{1}{3} \times \frac{6}{25} + \frac{2}{3} \times \frac{5}{50}\\
&=\frac{11}{75}
\end{align}

\newpage

%\tableofcontents

\bigskip

\renewcommand{\thefigure}{\theenumi}
\renewcommand{\thetable}{\theenumi}
%\renewcommand{\theequation}{\theenumi}

%\begin{abstract}
%%\boldmath
%In this letter, an algorithm for evaluating the exact analytical bit error rate  (BER)  for the piecewise linear (PL) combiner for  multiple relays is presented. Previous results were available only for upto three relays. The algorithm is unique in the sense that  the actual mathematical expressions, that are prohibitively large, need not be explicitly obtained. The diversity gain due to multiple relays is shown through plots of the analytical BER, well supported by simulations. 
%
%\end{abstract}
% IEEEtran.cls defaults to using nonbold math in the Abstract.
% This preserves the distinction between vectors and scalars. However,
% if the journal you are submitting to favors bold math in the abstract,
% then you can use LaTeX's standard command \boldmath at the very start
% of the abstract to achieve this. Many IEEE journals frown on math
% in the abstract anyway.

% Note that keywords are not normally used for peerreview papers.
%\begin{IEEEkeywords}
%Cooperative diversity, decode and forward, piecewise linear
%\end{IEEEkeywords}



% For peer review papers, you can put extra information on the cover
% page as needed:
% \ifCLASSOPTIONpeerreview
% \begin{center} \bfseries EDICS Category: 3-BBND \end{center}
% \fi
%
% For peerreview papers, this IEEEtran command inserts a page break and
% creates the second title. It will be ignored for other modes.
%\IEEEpeerreviewmaketitle




\item A card is drawn from a deck of 52 cards. Find the probability of getting a king or a heart or a red card.\\
\solution
%\begin{table}[H]
	\centering
\begin{tabular}{|c|c|c|}
\hline
Random variable &Value &Definition\\ \hline
\multirow{3}{*}{X} &0 &Slips of Rs 1\\
&1 &Slips of Rs 5\\
&2 &Slips of Rs 13\\ \hline
\multirow{2}{*}{Y} &0 &Box A\\
&1 &Box B\\\hline
\end{tabular}
\caption{}
\label{tab:Distribution}
\end{table}
See \tabref{tab:Distribution}.
\begin{align}
p_{Y}\brak{k}= \begin{cases} 
      \frac{1}{3} & {k=0} \\
      \frac{2}{3 }& {k=1} 
   \end{cases}
   \\
p_{Y|X}\brak{0|0} = \frac{19}{25}\, 
p_{Y|X}\brak{0|1} = \frac{6}{25}\,
p_{Y|X}\brak{1|0} = \frac{45}{50}\,
p_{Y|X}\brak{1|2} = \frac{5}{50}
\end{align}
The desired probability is the probability that a slip drawn at random is marked other than Rs 1,
\begin{align}
&=1-p_X\brak{0}\\
&= p_X(1) + p_X(2)
\end{align}
Using Bayes theorem,
\begin{align}
&= p_Y\brak{0} \times \pr{Y=0 | X=1} + p_Y\brak{1} \times \pr{Y=1|X=2}\\
&=\frac{1}{3} \times \frac{6}{25} + \frac{2}{3} \times \frac{5}{50}\\
&=\frac{11}{75}
\end{align}

\newpage

%\tableofcontents

\bigskip

\renewcommand{\thefigure}{\theenumi}
\renewcommand{\thetable}{\theenumi}
%\renewcommand{\theequation}{\theenumi}

%\begin{abstract}
%%\boldmath
%In this letter, an algorithm for evaluating the exact analytical bit error rate  (BER)  for the piecewise linear (PL) combiner for  multiple relays is presented. Previous results were available only for upto three relays. The algorithm is unique in the sense that  the actual mathematical expressions, that are prohibitively large, need not be explicitly obtained. The diversity gain due to multiple relays is shown through plots of the analytical BER, well supported by simulations. 
%
%\end{abstract}
% IEEEtran.cls defaults to using nonbold math in the Abstract.
% This preserves the distinction between vectors and scalars. However,
% if the journal you are submitting to favors bold math in the abstract,
% then you can use LaTeX's standard command \boldmath at the very start
% of the abstract to achieve this. Many IEEE journals frown on math
% in the abstract anyway.

% Note that keywords are not normally used for peerreview papers.
%\begin{IEEEkeywords}
%Cooperative diversity, decode and forward, piecewise linear
%\end{IEEEkeywords}



% For peer review papers, you can put extra information on the cover
% page as needed:
% \ifCLASSOPTIONpeerreview
% \begin{center} \bfseries EDICS Category: 3-BBND \end{center}
% \fi
%
% For peerreview papers, this IEEEtran command inserts a page break and
% creates the second title. It will be ignored for other modes.
%\IEEEpeerreviewmaketitle




\item The probability that a student will pass his examination is 0.73, the probability of
the student getting a compartment is 0.13, and the probability that the student will
either pass or get compartment is 0.96. State True or False.\\
\solution
%\begin{table}[H]
	\centering
\begin{tabular}{|c|c|c|}
\hline
Random variable &Value &Definition\\ \hline
\multirow{3}{*}{X} &0 &Slips of Rs 1\\
&1 &Slips of Rs 5\\
&2 &Slips of Rs 13\\ \hline
\multirow{2}{*}{Y} &0 &Box A\\
&1 &Box B\\\hline
\end{tabular}
\caption{}
\label{tab:Distribution}
\end{table}
See \tabref{tab:Distribution}.
\begin{align}
p_{Y}\brak{k}= \begin{cases} 
      \frac{1}{3} & {k=0} \\
      \frac{2}{3 }& {k=1} 
   \end{cases}
   \\
p_{Y|X}\brak{0|0} = \frac{19}{25}\, 
p_{Y|X}\brak{0|1} = \frac{6}{25}\,
p_{Y|X}\brak{1|0} = \frac{45}{50}\,
p_{Y|X}\brak{1|2} = \frac{5}{50}
\end{align}
The desired probability is the probability that a slip drawn at random is marked other than Rs 1,
\begin{align}
&=1-p_X\brak{0}\\
&= p_X(1) + p_X(2)
\end{align}
Using Bayes theorem,
\begin{align}
&= p_Y\brak{0} \times \pr{Y=0 | X=1} + p_Y\brak{1} \times \pr{Y=1|X=2}\\
&=\frac{1}{3} \times \frac{6}{25} + \frac{2}{3} \times \frac{5}{50}\\
&=\frac{11}{75}
\end{align}

\newpage

%\tableofcontents

\bigskip

\renewcommand{\thefigure}{\theenumi}
\renewcommand{\thetable}{\theenumi}
%\renewcommand{\theequation}{\theenumi}

%\begin{abstract}
%%\boldmath
%In this letter, an algorithm for evaluating the exact analytical bit error rate  (BER)  for the piecewise linear (PL) combiner for  multiple relays is presented. Previous results were available only for upto three relays. The algorithm is unique in the sense that  the actual mathematical expressions, that are prohibitively large, need not be explicitly obtained. The diversity gain due to multiple relays is shown through plots of the analytical BER, well supported by simulations. 
%
%\end{abstract}
% IEEEtran.cls defaults to using nonbold math in the Abstract.
% This preserves the distinction between vectors and scalars. However,
% if the journal you are submitting to favors bold math in the abstract,
% then you can use LaTeX's standard command \boldmath at the very start
% of the abstract to achieve this. Many IEEE journals frown on math
% in the abstract anyway.

% Note that keywords are not normally used for peerreview papers.
%\begin{IEEEkeywords}
%Cooperative diversity, decode and forward, piecewise linear
%\end{IEEEkeywords}



% For peer review papers, you can put extra information on the cover
% page as needed:
% \ifCLASSOPTIONpeerreview
% \begin{center} \bfseries EDICS Category: 3-BBND \end{center}
% \fi
%
% For peerreview papers, this IEEEtran command inserts a page break and
% creates the second title. It will be ignored for other modes.
%\IEEEpeerreviewmaketitle




\item A card is selected from a pack of 52 cards\\
\begin{enumerate}[label=(\alph*)]
\item How many points are there in the sample space?
\item Calculate the probability that the cards is an ace of spades.
\item Calculate the probability that the card is (i) an ace (ii)black card.\\
\end{enumerate}
%\input{ncert/11/16/3/4_1/Prob_4.tex}
\item In a non-leap year, the probability of having 53 tuesdays or 53 wednesdays is\\
\solution
%A non-leap year has a total of 365 days, and a week has 7 days.\\
So it can be expressed as 
\begin{align}
365\text{days} &=52\times 7+1 \text{day}
\end{align}
$\implies$ 52 tuesdays or wednesdays\\
Random variable X denotes the days of a week
\begin{align}
p_X\brak{k}&=\frac{1}{7}; \quad \brak{1<k<7}
\end{align}
So the probability of extra day being tuesday or wednesday is
\begin{align}
p_X\brak{3}+p_X\brak{4}&=\frac{1}{7}+\frac{1}{7}=\frac{2}{7}
\end{align}



\item There are 1000 sealed envelopes in a box, 10 of them contain a cash prize of
Rs 100 each, 100 of them contain a cash prize of Rs 50 each and 200 of them
contain a cash prize of Rs 10 each and rest do not contain any cash prize. If they
are well shuffled and an envelope is picked up out, what is the probability that it
contains no cash prize?\\
\solution
%\begin{table}[H]
	\centering
\begin{tabular}{|c|c|c|}
\hline
Random variable &Value &Definition\\ \hline
\multirow{3}{*}{X} &0 &Slips of Rs 1\\
&1 &Slips of Rs 5\\
&2 &Slips of Rs 13\\ \hline
\multirow{2}{*}{Y} &0 &Box A\\
&1 &Box B\\\hline
\end{tabular}
\caption{}
\label{tab:Distribution}
\end{table}
See \tabref{tab:Distribution}.
\begin{align}
p_{Y}\brak{k}= \begin{cases} 
      \frac{1}{3} & {k=0} \\
      \frac{2}{3 }& {k=1} 
   \end{cases}
   \\
p_{Y|X}\brak{0|0} = \frac{19}{25}\, 
p_{Y|X}\brak{0|1} = \frac{6}{25}\,
p_{Y|X}\brak{1|0} = \frac{45}{50}\,
p_{Y|X}\brak{1|2} = \frac{5}{50}
\end{align}
The desired probability is the probability that a slip drawn at random is marked other than Rs 1,
\begin{align}
&=1-p_X\brak{0}\\
&= p_X(1) + p_X(2)
\end{align}
Using Bayes theorem,
\begin{align}
&= p_Y\brak{0} \times \pr{Y=0 | X=1} + p_Y\brak{1} \times \pr{Y=1|X=2}\\
&=\frac{1}{3} \times \frac{6}{25} + \frac{2}{3} \times \frac{5}{50}\\
&=\frac{11}{75}
\end{align}

\newpage

%\tableofcontents

\bigskip

\renewcommand{\thefigure}{\theenumi}
\renewcommand{\thetable}{\theenumi}
%\renewcommand{\theequation}{\theenumi}

%\begin{abstract}
%%\boldmath
%In this letter, an algorithm for evaluating the exact analytical bit error rate  (BER)  for the piecewise linear (PL) combiner for  multiple relays is presented. Previous results were available only for upto three relays. The algorithm is unique in the sense that  the actual mathematical expressions, that are prohibitively large, need not be explicitly obtained. The diversity gain due to multiple relays is shown through plots of the analytical BER, well supported by simulations. 
%
%\end{abstract}
% IEEEtran.cls defaults to using nonbold math in the Abstract.
% This preserves the distinction between vectors and scalars. However,
% if the journal you are submitting to favors bold math in the abstract,
% then you can use LaTeX's standard command \boldmath at the very start
% of the abstract to achieve this. Many IEEE journals frown on math
% in the abstract anyway.

% Note that keywords are not normally used for peerreview papers.
%\begin{IEEEkeywords}
%Cooperative diversity, decode and forward, piecewise linear
%\end{IEEEkeywords}



% For peer review papers, you can put extra information on the cover
% page as needed:
% \ifCLASSOPTIONpeerreview
% \begin{center} \bfseries EDICS Category: 3-BBND \end{center}
% \fi
%
% For peerreview papers, this IEEEtran command inserts a page break and
% creates the second title. It will be ignored for other modes.
%\IEEEpeerreviewmaketitle




\item 
A die is thrown and a card is selected at random from a deck of 52 playing cards. The probability of getting an even number on the die and a spade card.\\
\solution
%\begin{table}[H]
	\centering
\begin{tabular}{|c|c|c|}
\hline
Random variable &Value &Definition\\ \hline
\multirow{3}{*}{X} &0 &Slips of Rs 1\\
&1 &Slips of Rs 5\\
&2 &Slips of Rs 13\\ \hline
\multirow{2}{*}{Y} &0 &Box A\\
&1 &Box B\\\hline
\end{tabular}
\caption{}
\label{tab:Distribution}
\end{table}
See \tabref{tab:Distribution}.
\begin{align}
p_{Y}\brak{k}= \begin{cases} 
      \frac{1}{3} & {k=0} \\
      \frac{2}{3 }& {k=1} 
   \end{cases}
   \\
p_{Y|X}\brak{0|0} = \frac{19}{25}\, 
p_{Y|X}\brak{0|1} = \frac{6}{25}\,
p_{Y|X}\brak{1|0} = \frac{45}{50}\,
p_{Y|X}\brak{1|2} = \frac{5}{50}
\end{align}
The desired probability is the probability that a slip drawn at random is marked other than Rs 1,
\begin{align}
&=1-p_X\brak{0}\\
&= p_X(1) + p_X(2)
\end{align}
Using Bayes theorem,
\begin{align}
&= p_Y\brak{0} \times \pr{Y=0 | X=1} + p_Y\brak{1} \times \pr{Y=1|X=2}\\
&=\frac{1}{3} \times \frac{6}{25} + \frac{2}{3} \times \frac{5}{50}\\
&=\frac{11}{75}
\end{align}

\newpage

%\tableofcontents

\bigskip

\renewcommand{\thefigure}{\theenumi}
\renewcommand{\thetable}{\theenumi}
%\renewcommand{\theequation}{\theenumi}

%\begin{abstract}
%%\boldmath
%In this letter, an algorithm for evaluating the exact analytical bit error rate  (BER)  for the piecewise linear (PL) combiner for  multiple relays is presented. Previous results were available only for upto three relays. The algorithm is unique in the sense that  the actual mathematical expressions, that are prohibitively large, need not be explicitly obtained. The diversity gain due to multiple relays is shown through plots of the analytical BER, well supported by simulations. 
%
%\end{abstract}
% IEEEtran.cls defaults to using nonbold math in the Abstract.
% This preserves the distinction between vectors and scalars. However,
% if the journal you are submitting to favors bold math in the abstract,
% then you can use LaTeX's standard command \boldmath at the very start
% of the abstract to achieve this. Many IEEE journals frown on math
% in the abstract anyway.

% Note that keywords are not normally used for peerreview papers.
%\begin{IEEEkeywords}
%Cooperative diversity, decode and forward, piecewise linear
%\end{IEEEkeywords}



% For peer review papers, you can put extra information on the cover
% page as needed:
% \ifCLASSOPTIONpeerreview
% \begin{center} \bfseries EDICS Category: 3-BBND \end{center}
% \fi
%
% For peerreview papers, this IEEEtran command inserts a page break and
% creates the second title. It will be ignored for other modes.
%\IEEEpeerreviewmaketitle




\item
If 4-digit numbers greater than 5,000 are randomly formed from the digits 0, 1, 3, 5, and 7, what is the probability of forming a number divisible by 5 when:
\begin{enumerate}
    \item The digits are repeated?
    \item The repetition of digits is not allowed?
\end{enumerate}
\solution
%\begin{table}[H]
	\centering
\begin{tabular}{|c|c|c|}
\hline
Random variable &Value &Definition\\ \hline
\multirow{3}{*}{X} &0 &Slips of Rs 1\\
&1 &Slips of Rs 5\\
&2 &Slips of Rs 13\\ \hline
\multirow{2}{*}{Y} &0 &Box A\\
&1 &Box B\\\hline
\end{tabular}
\caption{}
\label{tab:Distribution}
\end{table}
See \tabref{tab:Distribution}.
\begin{align}
p_{Y}\brak{k}= \begin{cases} 
      \frac{1}{3} & {k=0} \\
      \frac{2}{3 }& {k=1} 
   \end{cases}
   \\
p_{Y|X}\brak{0|0} = \frac{19}{25}\, 
p_{Y|X}\brak{0|1} = \frac{6}{25}\,
p_{Y|X}\brak{1|0} = \frac{45}{50}\,
p_{Y|X}\brak{1|2} = \frac{5}{50}
\end{align}
The desired probability is the probability that a slip drawn at random is marked other than Rs 1,
\begin{align}
&=1-p_X\brak{0}\\
&= p_X(1) + p_X(2)
\end{align}
Using Bayes theorem,
\begin{align}
&= p_Y\brak{0} \times \pr{Y=0 | X=1} + p_Y\brak{1} \times \pr{Y=1|X=2}\\
&=\frac{1}{3} \times \frac{6}{25} + \frac{2}{3} \times \frac{5}{50}\\
&=\frac{11}{75}
\end{align}

\newpage

%\tableofcontents

\bigskip

\renewcommand{\thefigure}{\theenumi}
\renewcommand{\thetable}{\theenumi}
%\renewcommand{\theequation}{\theenumi}

%\begin{abstract}
%%\boldmath
%In this letter, an algorithm for evaluating the exact analytical bit error rate  (BER)  for the piecewise linear (PL) combiner for  multiple relays is presented. Previous results were available only for upto three relays. The algorithm is unique in the sense that  the actual mathematical expressions, that are prohibitively large, need not be explicitly obtained. The diversity gain due to multiple relays is shown through plots of the analytical BER, well supported by simulations. 
%
%\end{abstract}
% IEEEtran.cls defaults to using nonbold math in the Abstract.
% This preserves the distinction between vectors and scalars. However,
% if the journal you are submitting to favors bold math in the abstract,
% then you can use LaTeX's standard command \boldmath at the very start
% of the abstract to achieve this. Many IEEE journals frown on math
% in the abstract anyway.

% Note that keywords are not normally used for peerreview papers.
%\begin{IEEEkeywords}
%Cooperative diversity, decode and forward, piecewise linear
%\end{IEEEkeywords}



% For peer review papers, you can put extra information on the cover
% page as needed:
% \ifCLASSOPTIONpeerreview
% \begin{center} \bfseries EDICS Category: 3-BBND \end{center}
% \fi
%
% For peerreview papers, this IEEEtran command inserts a page break and
% creates the second title. It will be ignored for other modes.
%\IEEEpeerreviewmaketitle




\item Consider the probability space $\brak{\Omega, \mathcal{G}, P}$ where $\Omega = [0,2]$ and $\mathcal{G} = \cbrak{\phi, \Omega, [0,1], (1,2]}$. Let $X$ and $Y$ be two functions on $\Omega$ defined as
\begin{align*}
    X(\omega) = 
    \begin{cases}
        1 & \text{if }\omega \in [0, 1]\\
        2 & \text{if }\omega \in (1, 2]
    \end{cases}
\end{align*}
and
\begin{align*}
    Y(\omega) = 
    \begin{cases}
        2 & \text{if }\omega \in [0, 1.5]\\
        3 & \text{if }\omega \in (1.5, 2].
    \end{cases}
\end{align*}
Then which one of the following statements is true?
\begin{enumerate}
    \item [(A)] $X$ is a random variable with respect to $\mathcal{G}$, but $Y$ is not a random variable with respect to $\mathcal{G}$.
    \item [(B)] $Y$ is a random variable with respect to $\mathcal{G}$, but $X$ is not a random variable with respect to $\mathcal{G}$.
    \item [(C)] Neither $X$ nor $Y$ is a random variable with respect to $\mathcal{G}$.
    \item [(D)] Both $X$ and $Y$ are random variables with respect to $\mathcal{G}$.
\end{enumerate} \hfill (GATE ST 2023)\\
\solution
%\begin{table}[H]
	\centering
\begin{tabular}{|c|c|c|}
\hline
Random variable &Value &Definition\\ \hline
\multirow{3}{*}{X} &0 &Slips of Rs 1\\
&1 &Slips of Rs 5\\
&2 &Slips of Rs 13\\ \hline
\multirow{2}{*}{Y} &0 &Box A\\
&1 &Box B\\\hline
\end{tabular}
\caption{}
\label{tab:Distribution}
\end{table}
See \tabref{tab:Distribution}.
\begin{align}
p_{Y}\brak{k}= \begin{cases} 
      \frac{1}{3} & {k=0} \\
      \frac{2}{3 }& {k=1} 
   \end{cases}
   \\
p_{Y|X}\brak{0|0} = \frac{19}{25}\, 
p_{Y|X}\brak{0|1} = \frac{6}{25}\,
p_{Y|X}\brak{1|0} = \frac{45}{50}\,
p_{Y|X}\brak{1|2} = \frac{5}{50}
\end{align}
The desired probability is the probability that a slip drawn at random is marked other than Rs 1,
\begin{align}
&=1-p_X\brak{0}\\
&= p_X(1) + p_X(2)
\end{align}
Using Bayes theorem,
\begin{align}
&= p_Y\brak{0} \times \pr{Y=0 | X=1} + p_Y\brak{1} \times \pr{Y=1|X=2}\\
&=\frac{1}{3} \times \frac{6}{25} + \frac{2}{3} \times \frac{5}{50}\\
&=\frac{11}{75}
\end{align}

\newpage

%\tableofcontents

\bigskip

\renewcommand{\thefigure}{\theenumi}
\renewcommand{\thetable}{\theenumi}
%\renewcommand{\theequation}{\theenumi}

%\begin{abstract}
%%\boldmath
%In this letter, an algorithm for evaluating the exact analytical bit error rate  (BER)  for the piecewise linear (PL) combiner for  multiple relays is presented. Previous results were available only for upto three relays. The algorithm is unique in the sense that  the actual mathematical expressions, that are prohibitively large, need not be explicitly obtained. The diversity gain due to multiple relays is shown through plots of the analytical BER, well supported by simulations. 
%
%\end{abstract}
% IEEEtran.cls defaults to using nonbold math in the Abstract.
% This preserves the distinction between vectors and scalars. However,
% if the journal you are submitting to favors bold math in the abstract,
% then you can use LaTeX's standard command \boldmath at the very start
% of the abstract to achieve this. Many IEEE journals frown on math
% in the abstract anyway.

% Note that keywords are not normally used for peerreview papers.
%\begin{IEEEkeywords}
%Cooperative diversity, decode and forward, piecewise linear
%\end{IEEEkeywords}



% For peer review papers, you can put extra information on the cover
% page as needed:
% \ifCLASSOPTIONpeerreview
% \begin{center} \bfseries EDICS Category: 3-BBND \end{center}
% \fi
%
% For peerreview papers, this IEEEtran command inserts a page break and
% creates the second title. It will be ignored for other modes.
%\IEEEpeerreviewmaketitle




	\item  A die is loaded in such a way that each odd number is twice as likely to occur as
each even number. Find $P(G)$, where $G$ is the event that a number greater than
3 occurs on a single roll of the die.
\\
\solution
		%\begin{table}[H]
	\centering
\begin{tabular}{|c|c|c|}
\hline
Random variable &Value &Definition\\ \hline
\multirow{3}{*}{X} &0 &Slips of Rs 1\\
&1 &Slips of Rs 5\\
&2 &Slips of Rs 13\\ \hline
\multirow{2}{*}{Y} &0 &Box A\\
&1 &Box B\\\hline
\end{tabular}
\caption{}
\label{tab:Distribution}
\end{table}
See \tabref{tab:Distribution}.
\begin{align}
p_{Y}\brak{k}= \begin{cases} 
      \frac{1}{3} & {k=0} \\
      \frac{2}{3 }& {k=1} 
   \end{cases}
   \\
p_{Y|X}\brak{0|0} = \frac{19}{25}\, 
p_{Y|X}\brak{0|1} = \frac{6}{25}\,
p_{Y|X}\brak{1|0} = \frac{45}{50}\,
p_{Y|X}\brak{1|2} = \frac{5}{50}
\end{align}
The desired probability is the probability that a slip drawn at random is marked other than Rs 1,
\begin{align}
&=1-p_X\brak{0}\\
&= p_X(1) + p_X(2)
\end{align}
Using Bayes theorem,
\begin{align}
&= p_Y\brak{0} \times \pr{Y=0 | X=1} + p_Y\brak{1} \times \pr{Y=1|X=2}\\
&=\frac{1}{3} \times \frac{6}{25} + \frac{2}{3} \times \frac{5}{50}\\
&=\frac{11}{75}
\end{align}

\newpage

%\tableofcontents

\bigskip

\renewcommand{\thefigure}{\theenumi}
\renewcommand{\thetable}{\theenumi}
%\renewcommand{\theequation}{\theenumi}

%\begin{abstract}
%%\boldmath
%In this letter, an algorithm for evaluating the exact analytical bit error rate  (BER)  for the piecewise linear (PL) combiner for  multiple relays is presented. Previous results were available only for upto three relays. The algorithm is unique in the sense that  the actual mathematical expressions, that are prohibitively large, need not be explicitly obtained. The diversity gain due to multiple relays is shown through plots of the analytical BER, well supported by simulations. 
%
%\end{abstract}
% IEEEtran.cls defaults to using nonbold math in the Abstract.
% This preserves the distinction between vectors and scalars. However,
% if the journal you are submitting to favors bold math in the abstract,
% then you can use LaTeX's standard command \boldmath at the very start
% of the abstract to achieve this. Many IEEE journals frown on math
% in the abstract anyway.

% Note that keywords are not normally used for peerreview papers.
%\begin{IEEEkeywords}
%Cooperative diversity, decode and forward, piecewise linear
%\end{IEEEkeywords}



% For peer review papers, you can put extra information on the cover
% page as needed:
% \ifCLASSOPTIONpeerreview
% \begin{center} \bfseries EDICS Category: 3-BBND \end{center}
% \fi
%
% For peerreview papers, this IEEEtran command inserts a page break and
% creates the second title. It will be ignored for other modes.
%\IEEEpeerreviewmaketitle




	\item All the jacks, queens and kings are removed from a deck of 52 playing cards. The remaining cards are well shuffled and then one card is drawn at random. Giving ace a value 1 similar value for other cards, find the probability that the card has a value 
		\begin{enumerate}
			\item 7
			\item greater than 7
			\item less than 7
		\end{enumerate}
		%Number of cards left after removing all jacks, queens and kings 
\begin{align}
N	= 52 - 4\times 3
	= 40
\end{align}
%\begin{table}[H]
%\def\arraystretch{1.2}
%\begin{tabular}{|c|c|c|}
%\hline
%	\textbf{Parameter} &\textbf{Value} &\textbf{Description}\\ \hline
%	$X$ &1-10 &Represents the value of the card picked \\ \hline
%\end{tabular}
%\end{table}
Let $1 \le X \le 10$ be the value of the card picked.  Then,
\begin{align}
	p_X(k) &= \Pr(X=k)\ \forall\ 1 \leq k \leq 10\\
	&= \frac{4\times 1}{40}\\
	&= \frac{1}{10}\\
	\therefore p_X(k) &= 
	\begin{cases}
		\frac{1}{10} & 1 \leq k \leq 10\\
		0 & \text{otherwise}
	\end{cases}
\end{align}
and
\begin{align}
	F_{X}(k) &= \sum_{m=0}^{k}p_{X}(m) \quad 1 \leq k \leq 10\\
	&= \frac{k}{10}\\
	\therefore F_{X}(k) &= 
	\begin{cases}
		0 & k \leq 0\\
		\frac{k}{10} & 1\leq k \leq 10\\
		1 & k > 10 
	\end{cases}
\end{align}
\begin{enumerate}
	\item Probability that card has value equal to 7 is
		\begin{align}
			 p_{X}(7)
			= \frac{1}{10}
		\end{align}
	\item Probability that card has value greater than 7 is
		\begin{align}
			1 - F_X(7)
			&= 1 - \frac{7}{10}
			\\
			&= \frac{3}{10}
		\end{align}
	\item Probability that card has value less than 7 is
		\begin{align}
			 F_{X}(6)
			=\frac{6}{10}
		\end{align}
\end{enumerate}

  \item A Lot consists of 48 mobile phones of which 42 are good, 3 have only minor defects and 3 have major defects.Varnika will buy a phone if it is good but the trader will only buy a mobile if it has no major defects. One phone is selected at random from the lot. What is the probability that it is
\begin{enumerate}
	\item acceptable to Varnika?
            \item acceptable to the trader?
\end{enumerate}
\solution
	%\begin{table}[H]
	\centering
\begin{tabular}{|c|c|c|}
\hline
Random variable &Value &Definition\\ \hline
\multirow{3}{*}{X} &0 &Slips of Rs 1\\
&1 &Slips of Rs 5\\
&2 &Slips of Rs 13\\ \hline
\multirow{2}{*}{Y} &0 &Box A\\
&1 &Box B\\\hline
\end{tabular}
\caption{}
\label{tab:Distribution}
\end{table}
See \tabref{tab:Distribution}.
\begin{align}
p_{Y}\brak{k}= \begin{cases} 
      \frac{1}{3} & {k=0} \\
      \frac{2}{3 }& {k=1} 
   \end{cases}
   \\
p_{Y|X}\brak{0|0} = \frac{19}{25}\, 
p_{Y|X}\brak{0|1} = \frac{6}{25}\,
p_{Y|X}\brak{1|0} = \frac{45}{50}\,
p_{Y|X}\brak{1|2} = \frac{5}{50}
\end{align}
The desired probability is the probability that a slip drawn at random is marked other than Rs 1,
\begin{align}
&=1-p_X\brak{0}\\
&= p_X(1) + p_X(2)
\end{align}
Using Bayes theorem,
\begin{align}
&= p_Y\brak{0} \times \pr{Y=0 | X=1} + p_Y\brak{1} \times \pr{Y=1|X=2}\\
&=\frac{1}{3} \times \frac{6}{25} + \frac{2}{3} \times \frac{5}{50}\\
&=\frac{11}{75}
\end{align}

\newpage

%\tableofcontents

\bigskip

\renewcommand{\thefigure}{\theenumi}
\renewcommand{\thetable}{\theenumi}
%\renewcommand{\theequation}{\theenumi}

%\begin{abstract}
%%\boldmath
%In this letter, an algorithm for evaluating the exact analytical bit error rate  (BER)  for the piecewise linear (PL) combiner for  multiple relays is presented. Previous results were available only for upto three relays. The algorithm is unique in the sense that  the actual mathematical expressions, that are prohibitively large, need not be explicitly obtained. The diversity gain due to multiple relays is shown through plots of the analytical BER, well supported by simulations. 
%
%\end{abstract}
% IEEEtran.cls defaults to using nonbold math in the Abstract.
% This preserves the distinction between vectors and scalars. However,
% if the journal you are submitting to favors bold math in the abstract,
% then you can use LaTeX's standard command \boldmath at the very start
% of the abstract to achieve this. Many IEEE journals frown on math
% in the abstract anyway.

% Note that keywords are not normally used for peerreview papers.
%\begin{IEEEkeywords}
%Cooperative diversity, decode and forward, piecewise linear
%\end{IEEEkeywords}



% For peer review papers, you can put extra information on the cover
% page as needed:
% \ifCLASSOPTIONpeerreview
% \begin{center} \bfseries EDICS Category: 3-BBND \end{center}
% \fi
%
% For peerreview papers, this IEEEtran command inserts a page break and
% creates the second title. It will be ignored for other modes.
%\IEEEpeerreviewmaketitle




 \item A student says that if you throw a die, it will show up 1 or not 1. Therefore, the probability of getting 1 and the probability of getting 'not 1' each is equal to $\frac{1}{2}$. Is this correct? Give reasons.\\
 \solution
        %\begin{table}[H]
	\centering
\begin{tabular}{|c|c|c|}
\hline
Random variable &Value &Definition\\ \hline
\multirow{3}{*}{X} &0 &Slips of Rs 1\\
&1 &Slips of Rs 5\\
&2 &Slips of Rs 13\\ \hline
\multirow{2}{*}{Y} &0 &Box A\\
&1 &Box B\\\hline
\end{tabular}
\caption{}
\label{tab:Distribution}
\end{table}
See \tabref{tab:Distribution}.
\begin{align}
p_{Y}\brak{k}= \begin{cases} 
      \frac{1}{3} & {k=0} \\
      \frac{2}{3 }& {k=1} 
   \end{cases}
   \\
p_{Y|X}\brak{0|0} = \frac{19}{25}\, 
p_{Y|X}\brak{0|1} = \frac{6}{25}\,
p_{Y|X}\brak{1|0} = \frac{45}{50}\,
p_{Y|X}\brak{1|2} = \frac{5}{50}
\end{align}
The desired probability is the probability that a slip drawn at random is marked other than Rs 1,
\begin{align}
&=1-p_X\brak{0}\\
&= p_X(1) + p_X(2)
\end{align}
Using Bayes theorem,
\begin{align}
&= p_Y\brak{0} \times \pr{Y=0 | X=1} + p_Y\brak{1} \times \pr{Y=1|X=2}\\
&=\frac{1}{3} \times \frac{6}{25} + \frac{2}{3} \times \frac{5}{50}\\
&=\frac{11}{75}
\end{align}

\newpage

%\tableofcontents

\bigskip

\renewcommand{\thefigure}{\theenumi}
\renewcommand{\thetable}{\theenumi}
%\renewcommand{\theequation}{\theenumi}

%\begin{abstract}
%%\boldmath
%In this letter, an algorithm for evaluating the exact analytical bit error rate  (BER)  for the piecewise linear (PL) combiner for  multiple relays is presented. Previous results were available only for upto three relays. The algorithm is unique in the sense that  the actual mathematical expressions, that are prohibitively large, need not be explicitly obtained. The diversity gain due to multiple relays is shown through plots of the analytical BER, well supported by simulations. 
%
%\end{abstract}
% IEEEtran.cls defaults to using nonbold math in the Abstract.
% This preserves the distinction between vectors and scalars. However,
% if the journal you are submitting to favors bold math in the abstract,
% then you can use LaTeX's standard command \boldmath at the very start
% of the abstract to achieve this. Many IEEE journals frown on math
% in the abstract anyway.

% Note that keywords are not normally used for peerreview papers.
%\begin{IEEEkeywords}
%Cooperative diversity, decode and forward, piecewise linear
%\end{IEEEkeywords}



% For peer review papers, you can put extra information on the cover
% page as needed:
% \ifCLASSOPTIONpeerreview
% \begin{center} \bfseries EDICS Category: 3-BBND \end{center}
% \fi
%
% For peerreview papers, this IEEEtran command inserts a page break and
% creates the second title. It will be ignored for other modes.
%\IEEEpeerreviewmaketitle




   \item Four candidates A, B, C, D have ap-
plied for the assignment to coach a school cricket
team. If A is twice as likely to be selected as B, and
B and C are given about the same chance of being
selected, while C is twice as likely to be selected
as D, what are the probabilities that
\begin{enumerate}
\item C will be selected?
\item A will not be selected?
\end{enumerate}
	%\begin{table}[H]
	\centering
\begin{tabular}{|c|c|c|}
\hline
Random variable &Value &Definition\\ \hline
\multirow{3}{*}{X} &0 &Slips of Rs 1\\
&1 &Slips of Rs 5\\
&2 &Slips of Rs 13\\ \hline
\multirow{2}{*}{Y} &0 &Box A\\
&1 &Box B\\\hline
\end{tabular}
\caption{}
\label{tab:Distribution}
\end{table}
See \tabref{tab:Distribution}.
\begin{align}
p_{Y}\brak{k}= \begin{cases} 
      \frac{1}{3} & {k=0} \\
      \frac{2}{3 }& {k=1} 
   \end{cases}
   \\
p_{Y|X}\brak{0|0} = \frac{19}{25}\, 
p_{Y|X}\brak{0|1} = \frac{6}{25}\,
p_{Y|X}\brak{1|0} = \frac{45}{50}\,
p_{Y|X}\brak{1|2} = \frac{5}{50}
\end{align}
The desired probability is the probability that a slip drawn at random is marked other than Rs 1,
\begin{align}
&=1-p_X\brak{0}\\
&= p_X(1) + p_X(2)
\end{align}
Using Bayes theorem,
\begin{align}
&= p_Y\brak{0} \times \pr{Y=0 | X=1} + p_Y\brak{1} \times \pr{Y=1|X=2}\\
&=\frac{1}{3} \times \frac{6}{25} + \frac{2}{3} \times \frac{5}{50}\\
&=\frac{11}{75}
\end{align}

\newpage

%\tableofcontents

\bigskip

\renewcommand{\thefigure}{\theenumi}
\renewcommand{\thetable}{\theenumi}
%\renewcommand{\theequation}{\theenumi}

%\begin{abstract}
%%\boldmath
%In this letter, an algorithm for evaluating the exact analytical bit error rate  (BER)  for the piecewise linear (PL) combiner for  multiple relays is presented. Previous results were available only for upto three relays. The algorithm is unique in the sense that  the actual mathematical expressions, that are prohibitively large, need not be explicitly obtained. The diversity gain due to multiple relays is shown through plots of the analytical BER, well supported by simulations. 
%
%\end{abstract}
% IEEEtran.cls defaults to using nonbold math in the Abstract.
% This preserves the distinction between vectors and scalars. However,
% if the journal you are submitting to favors bold math in the abstract,
% then you can use LaTeX's standard command \boldmath at the very start
% of the abstract to achieve this. Many IEEE journals frown on math
% in the abstract anyway.

% Note that keywords are not normally used for peerreview papers.
%\begin{IEEEkeywords}
%Cooperative diversity, decode and forward, piecewise linear
%\end{IEEEkeywords}



% For peer review papers, you can put extra information on the cover
% page as needed:
% \ifCLASSOPTIONpeerreview
% \begin{center} \bfseries EDICS Category: 3-BBND \end{center}
% \fi
%
% For peerreview papers, this IEEEtran command inserts a page break and
% creates the second title. It will be ignored for other modes.
%\IEEEpeerreviewmaketitle




 \item A bag contain 24 balls of which $x$ balls are red, $2x$ are white and $3x$ are blue. A ball is selected at random, What is the probability that it is
\begin{enumerate}[label=\alph*)]
\item not red ?
\item white ?
\end{enumerate}
%\begin{table}[H]
	\centering
\begin{tabular}{|c|c|c|}
\hline
Random variable &Value &Definition\\ \hline
\multirow{3}{*}{X} &0 &Slips of Rs 1\\
&1 &Slips of Rs 5\\
&2 &Slips of Rs 13\\ \hline
\multirow{2}{*}{Y} &0 &Box A\\
&1 &Box B\\\hline
\end{tabular}
\caption{}
\label{tab:Distribution}
\end{table}
See \tabref{tab:Distribution}.
\begin{align}
p_{Y}\brak{k}= \begin{cases} 
      \frac{1}{3} & {k=0} \\
      \frac{2}{3 }& {k=1} 
   \end{cases}
   \\
p_{Y|X}\brak{0|0} = \frac{19}{25}\, 
p_{Y|X}\brak{0|1} = \frac{6}{25}\,
p_{Y|X}\brak{1|0} = \frac{45}{50}\,
p_{Y|X}\brak{1|2} = \frac{5}{50}
\end{align}
The desired probability is the probability that a slip drawn at random is marked other than Rs 1,
\begin{align}
&=1-p_X\brak{0}\\
&= p_X(1) + p_X(2)
\end{align}
Using Bayes theorem,
\begin{align}
&= p_Y\brak{0} \times \pr{Y=0 | X=1} + p_Y\brak{1} \times \pr{Y=1|X=2}\\
&=\frac{1}{3} \times \frac{6}{25} + \frac{2}{3} \times \frac{5}{50}\\
&=\frac{11}{75}
\end{align}

\newpage

%\tableofcontents

\bigskip

\renewcommand{\thefigure}{\theenumi}
\renewcommand{\thetable}{\theenumi}
%\renewcommand{\theequation}{\theenumi}

%\begin{abstract}
%%\boldmath
%In this letter, an algorithm for evaluating the exact analytical bit error rate  (BER)  for the piecewise linear (PL) combiner for  multiple relays is presented. Previous results were available only for upto three relays. The algorithm is unique in the sense that  the actual mathematical expressions, that are prohibitively large, need not be explicitly obtained. The diversity gain due to multiple relays is shown through plots of the analytical BER, well supported by simulations. 
%
%\end{abstract}
% IEEEtran.cls defaults to using nonbold math in the Abstract.
% This preserves the distinction between vectors and scalars. However,
% if the journal you are submitting to favors bold math in the abstract,
% then you can use LaTeX's standard command \boldmath at the very start
% of the abstract to achieve this. Many IEEE journals frown on math
% in the abstract anyway.

% Note that keywords are not normally used for peerreview papers.
%\begin{IEEEkeywords}
%Cooperative diversity, decode and forward, piecewise linear
%\end{IEEEkeywords}



% For peer review papers, you can put extra information on the cover
% page as needed:
% \ifCLASSOPTIONpeerreview
% \begin{center} \bfseries EDICS Category: 3-BBND \end{center}
% \fi
%
% For peerreview papers, this IEEEtran command inserts a page break and
% creates the second title. It will be ignored for other modes.
%\IEEEpeerreviewmaketitle




If the letters of the word ASSASSINATION are arranged at random. Find the Probability that
\begin{enumerate}[label=(\alph*)]
\item Four $S's$ come consecutively in the word
\item Two  $I's$ and two $N's$ come together
\item All $A's$ are not coming together
\item No two $A's$ are coming together
\end{enumerate}
%\begin{table}[H]
	\centering
\begin{tabular}{|c|c|c|}
\hline
Random variable &Value &Definition\\ \hline
\multirow{3}{*}{X} &0 &Slips of Rs 1\\
&1 &Slips of Rs 5\\
&2 &Slips of Rs 13\\ \hline
\multirow{2}{*}{Y} &0 &Box A\\
&1 &Box B\\\hline
\end{tabular}
\caption{}
\label{tab:Distribution}
\end{table}
See \tabref{tab:Distribution}.
\begin{align}
p_{Y}\brak{k}= \begin{cases} 
      \frac{1}{3} & {k=0} \\
      \frac{2}{3 }& {k=1} 
   \end{cases}
   \\
p_{Y|X}\brak{0|0} = \frac{19}{25}\, 
p_{Y|X}\brak{0|1} = \frac{6}{25}\,
p_{Y|X}\brak{1|0} = \frac{45}{50}\,
p_{Y|X}\brak{1|2} = \frac{5}{50}
\end{align}
The desired probability is the probability that a slip drawn at random is marked other than Rs 1,
\begin{align}
&=1-p_X\brak{0}\\
&= p_X(1) + p_X(2)
\end{align}
Using Bayes theorem,
\begin{align}
&= p_Y\brak{0} \times \pr{Y=0 | X=1} + p_Y\brak{1} \times \pr{Y=1|X=2}\\
&=\frac{1}{3} \times \frac{6}{25} + \frac{2}{3} \times \frac{5}{50}\\
&=\frac{11}{75}
\end{align}

\newpage

%\tableofcontents

\bigskip

\renewcommand{\thefigure}{\theenumi}
\renewcommand{\thetable}{\theenumi}
%\renewcommand{\theequation}{\theenumi}

%\begin{abstract}
%%\boldmath
%In this letter, an algorithm for evaluating the exact analytical bit error rate  (BER)  for the piecewise linear (PL) combiner for  multiple relays is presented. Previous results were available only for upto three relays. The algorithm is unique in the sense that  the actual mathematical expressions, that are prohibitively large, need not be explicitly obtained. The diversity gain due to multiple relays is shown through plots of the analytical BER, well supported by simulations. 
%
%\end{abstract}
% IEEEtran.cls defaults to using nonbold math in the Abstract.
% This preserves the distinction between vectors and scalars. However,
% if the journal you are submitting to favors bold math in the abstract,
% then you can use LaTeX's standard command \boldmath at the very start
% of the abstract to achieve this. Many IEEE journals frown on math
% in the abstract anyway.

% Note that keywords are not normally used for peerreview papers.
%\begin{IEEEkeywords}
%Cooperative diversity, decode and forward, piecewise linear
%\end{IEEEkeywords}



% For peer review papers, you can put extra information on the cover
% page as needed:
% \ifCLASSOPTIONpeerreview
% \begin{center} \bfseries EDICS Category: 3-BBND \end{center}
% \fi
%
% For peerreview papers, this IEEEtran command inserts a page break and
% creates the second title. It will be ignored for other modes.
%\IEEEpeerreviewmaketitle




	\item One urn contains two black balls (labelled B1 and B2) and one white ball. A
	second urn contains one black ball and two white balls (labelled W1 and W2).
	Suppose the following experiment is performed. One of the two urns is chosen
	at random. Next a ball is randomly chosen from the urn. Then a second ball is
	chosen at random from the same urn without replacing the first ball.
	
	\begin{enumerate}
	\item What is the probability that two black balls are chosen?
	
	\item What is the probability that two balls of opposite colour are chosen?
	\end{enumerate}
	\solution
	%\begin{align}
    \label{eq:12.13.6.18.1}
	\because	\pr{A|B} &> \pr{A},\
\frac{\pr{AB}}{\pr{B}} > \pr{A}
\\
    \label{eq:12.13.6.18.2}
	\implies \pr{AB} &> \pr{A}\pr{B}
	\\
	\text{or, } \frac{\pr{AB}}{\pr{A}} &=\pr{B|A} > \pr{A}
\end{align}

\end{enumerate}

		\item A box of oranges is inspected by examining three randomly selected oranges drawn without replacement. If all the three oranges are good, the box is approved for sale, otherwise, it is rejected. Find the probability that a box containing 15 oranges out of which 12 are good and 3 are bad ones will be approved for sale.
		\label{ncert/12/13/2/3/defs.tex}
		\item Two balls are drawn at random with replacement from a box containing 10 black and 8 red balls. Find the probability that
		\label{ncert/12/13/2/12}
\begin{enumerate}
\item both balls are red.
\item first ball is black and second is red.
\item one of them is black and other is red.
\end{enumerate}

\item In a hostel, 60\% of the students read Hindi newspaper, 40\% read English newspaper and 20\% read both Hindi and English newspapers. A student is selected at random.
		\label{ncert/12/13/2/15}
\begin{enumerate}
\item Find the probability that she reads neither Hindi nor English newspapers.
\item If she reads Hindi newspaper, find the probability that she reads English newspaper.
\item If she reads English newspaper, find the probability that she reads Hindi newspaper.\\
\end{enumerate}
\item The probability of obtaining an even prime number on each die, when a pair of dice is rolled is 
\begin{enumerate}
    \item $0$ 
    
    \item $\frac{1}{3}$ 
    
    \item $\frac{1}{12}$ 
    
    \item $\frac{1}{36}$ 
\end{enumerate}
\solution
		%\begin{enumerate}[label=\thesection.\arabic*,ref=\thesection.\theenumi]
	\item One card is drawn from a well-shuffled deck of 52 cards. Find the probability of getting
\begin{enumerate}
\item A king of red colour 
\item A face card 
\item A red face card
\item The jack of hearts
\item A spade
\item The queen of diamonds

\end{enumerate}
\solution
		%\begin{table}[H]
	\centering
\begin{tabular}{|c|c|c|}
\hline
Random variable &Value &Definition\\ \hline
\multirow{3}{*}{X} &0 &Slips of Rs 1\\
&1 &Slips of Rs 5\\
&2 &Slips of Rs 13\\ \hline
\multirow{2}{*}{Y} &0 &Box A\\
&1 &Box B\\\hline
\end{tabular}
\caption{}
\label{tab:Distribution}
\end{table}
See \tabref{tab:Distribution}.
\begin{align}
p_{Y}\brak{k}= \begin{cases} 
      \frac{1}{3} & {k=0} \\
      \frac{2}{3 }& {k=1} 
   \end{cases}
   \\
p_{Y|X}\brak{0|0} = \frac{19}{25}\, 
p_{Y|X}\brak{0|1} = \frac{6}{25}\,
p_{Y|X}\brak{1|0} = \frac{45}{50}\,
p_{Y|X}\brak{1|2} = \frac{5}{50}
\end{align}
The desired probability is the probability that a slip drawn at random is marked other than Rs 1,
\begin{align}
&=1-p_X\brak{0}\\
&= p_X(1) + p_X(2)
\end{align}
Using Bayes theorem,
\begin{align}
&= p_Y\brak{0} \times \pr{Y=0 | X=1} + p_Y\brak{1} \times \pr{Y=1|X=2}\\
&=\frac{1}{3} \times \frac{6}{25} + \frac{2}{3} \times \frac{5}{50}\\
&=\frac{11}{75}
\end{align}

\newpage

%\tableofcontents

\bigskip

\renewcommand{\thefigure}{\theenumi}
\renewcommand{\thetable}{\theenumi}
%\renewcommand{\theequation}{\theenumi}

%\begin{abstract}
%%\boldmath
%In this letter, an algorithm for evaluating the exact analytical bit error rate  (BER)  for the piecewise linear (PL) combiner for  multiple relays is presented. Previous results were available only for upto three relays. The algorithm is unique in the sense that  the actual mathematical expressions, that are prohibitively large, need not be explicitly obtained. The diversity gain due to multiple relays is shown through plots of the analytical BER, well supported by simulations. 
%
%\end{abstract}
% IEEEtran.cls defaults to using nonbold math in the Abstract.
% This preserves the distinction between vectors and scalars. However,
% if the journal you are submitting to favors bold math in the abstract,
% then you can use LaTeX's standard command \boldmath at the very start
% of the abstract to achieve this. Many IEEE journals frown on math
% in the abstract anyway.

% Note that keywords are not normally used for peerreview papers.
%\begin{IEEEkeywords}
%Cooperative diversity, decode and forward, piecewise linear
%\end{IEEEkeywords}



% For peer review papers, you can put extra information on the cover
% page as needed:
% \ifCLASSOPTIONpeerreview
% \begin{center} \bfseries EDICS Category: 3-BBND \end{center}
% \fi
%
% For peerreview papers, this IEEEtran command inserts a page break and
% creates the second title. It will be ignored for other modes.
%\IEEEpeerreviewmaketitle




	\item Five cards—the ten, jack, queen, king and ace of diamonds, are well-shuffled with their face downwards. One card is then picked up at random.
\begin{enumerate}
\item
What is the probability that the card is the queen? 
\item
If the queen is drawn and put aside, what is the probability that the second card picked up is (a) an ace? (b) a queen?\\
\end{enumerate}
\solution
		%\begin{enumerate}[label=\thesection.\arabic*,ref=\thesection.\theenumi]
	\item One card is drawn from a well-shuffled deck of 52 cards. Find the probability of getting
\begin{enumerate}
\item A king of red colour 
\item A face card 
\item A red face card
\item The jack of hearts
\item A spade
\item The queen of diamonds

\end{enumerate}
\solution
		%\input{ncert/10/15/1/14/main.tex}
	\item Five cards—the ten, jack, queen, king and ace of diamonds, are well-shuffled with their face downwards. One card is then picked up at random.
\begin{enumerate}
\item
What is the probability that the card is the queen? 
\item
If the queen is drawn and put aside, what is the probability that the second card picked up is (a) an ace? (b) a queen?\\
\end{enumerate}
\solution
		%\input{ncert/10/15/1/15/defs.tex}
	\item A bag contains $5$ red balls and some blue balls. If the probability of drawing a blue ball is double that if a red ball, determine the number of blue balls in the bag. 
		\\
\solution
		%\input{ncert/10/15/2/3/defs.tex}
	\item A card is selected from a pack of 52 cards.
 \begin{enumerate}[label=(\alph*)] 
                 \item How many points are there in the sample space?
                 \item Calculate the probability that the card is an ace of spades.
                 \item Calculate the probability that the card is (i) an ace and (ii) black card.
 \end{enumerate}
\solution
		%\input{ncert/11/16/3/4/main.tex}
\item Four cards are drawn from a well-shuffled deck of 52 cards. What is the probability of obtaining 3 diamonds and one spade.
\\
\solution
		%\input{ncert/11/16/4/2/defs.tex}
\item In a certain lottery 10,000 tickets are sold and ten equal prizes are awarded. What is the probability of not getting a prize if you buy (a) one ticket (b) two tickets (c) 10 tickets ?	
\\
\solution
		%\input{ncert/11/16/4/4/defs.tex}
		%
\item 
Out of 100 students, two sections of 40 and 60 are formed. If you and your friend are among the 100 students, what is the probability that
\begin{enumerate}
\item you both enter the same section?
\item you both enter the different sections?
\end{enumerate}
\solution
		%\input{ncert/11/16/4/5/defs.tex}
	\item 
The number lock of a suitcase has 4 wheels each labelled with ten digits i.e. from 0 to 9.The lock opens with a sequence of four digits with no repeats.What is the probability of a person getting the right sequence to open the suitcase.
\\
\solution
		%\input{ncert/11/16/4/10/defs.tex}
		%
\item 
Two cards are drawn at random and without replacement from a pack of 52 playing cards. Find the probability that both the cards are black.
\\
\solution
		%\input{ncert/12/13/2/2/defs.tex}
		\item A box of oranges is inspected by examining three randomly selected oranges drawn without replacement. If all the three oranges are good, the box is approved for sale, otherwise, it is rejected. Find the probability that a box containing 15 oranges out of which 12 are good and 3 are bad ones will be approved for sale.
		\label{ncert/12/13/2/3/defs.tex}
		\item Two balls are drawn at random with replacement from a box containing 10 black and 8 red balls. Find the probability that
		\label{ncert/12/13/2/12}
\begin{enumerate}
\item both balls are red.
\item first ball is black and second is red.
\item one of them is black and other is red.
\end{enumerate}

\item In a hostel, 60\% of the students read Hindi newspaper, 40\% read English newspaper and 20\% read both Hindi and English newspapers. A student is selected at random.
		\label{ncert/12/13/2/15}
\begin{enumerate}
\item Find the probability that she reads neither Hindi nor English newspapers.
\item If she reads Hindi newspaper, find the probability that she reads English newspaper.
\item If she reads English newspaper, find the probability that she reads Hindi newspaper.\\
\end{enumerate}
\item The probability of obtaining an even prime number on each die, when a pair of dice is rolled is 
\begin{enumerate}
    \item $0$ 
    
    \item $\frac{1}{3}$ 
    
    \item $\frac{1}{12}$ 
    
    \item $\frac{1}{36}$ 
\end{enumerate}
\solution
		%\input{ncert/12/13/2/17/defs.tex}
	\item A bag contains 4 red and 4 black balls, another bag contains 2 red and 6 black balls. One of the two bags is selected at random and a ball is drawn from the bag which is found to be red. Find the probability that the ball is drawn from the first bag.
\\
\solution
		%\input{ncert/12/13/3/2/main.tex}
  \item
  Cards with numbers 2 to 101 are placed in a box. A card is selected at random.Find the probability that the card has
\begin{enumerate}[label=(\roman*)]
	\item an even number 
	\item a square number
\end{enumerate}
\solution
%\input{exemplar/10/13/3/32/main.tex}
\item
The king, queen and jack of clubs are removed from a deck of 52 playing cards and then well shuffled. Now one card is drawn at random from the remaining cards.  Determine the probability that the card is
\begin{enumerate}[label=(\roman*)]
\item a club
\item 10 of hearts
\end{enumerate}
\solution
%\input{exemplar/10/13/3/29/main.tex}
\item A team of medical students doing their internship have to assist during surgeries
at a city hospital. The probabilities of surgeries rated as very complex, complex,
routine, simple or very simple are respectively, 0.15, 0.20, 0.31, 0.26, .08. Find
the probabilities that a particular surgery will be rated
\begin{enumerate}
	\item complex or very complex;
	\item neither very complex nor very simple;
	\item routine or complex
	\item routine or simple
\end{enumerate}
\solution
%\input{exemplar/11/16/3/8(1)/main.tex}
\item A card is selected from a pack of 52 cards.
\begin{enumerate}[label=(\alph*)]
    \item How many points are there in the sample space?
    \item Calculate the probability that the card is an ace of spades.
    \item Calculate the probability that the card is (i) an ace and (ii) black card.
\end{enumerate}
\solution
%\input{exemplar/11/16/3/4/main2.tex}
\item The probability that a non leap year selected at random will contain 53 sundays.
\\
\solution
%\input{exemplar/10/13/1/19/main.tex}
\item One of the four persons John, Rita, Aslam or Gurpreet will be promoted next
month. Consequently the sample space consists of four elementary outcomes
S = {John promoted, Rita promoted, Aslam promoted, Gurpreet promoted}
You are told that the chances of John’s promotion is same as that of Gurpreet,
Rita’s chances of promotion are twice as likely as Johns. Aslam’s chances are
four times that of John.
\begin{enumerate}
	\item Determine
	\begin{enumerate}
		\item P (John promoted)
		\item P (Rita promoted)
		\item P (Aslam promoted)
		\item P (Gurpreet promoted)
	\end{enumerate}
	\item If A = {John promoted or Gurpreet promoted}, find P (A).
\end{enumerate}
\solution
%\input{exemplar/11/16/3/10/main.tex}
\item A card is drawn from a deck of 52 cards. Find the probability of getting a king or a heart or a red card.\\
\solution
%\input{exemplar/11/16/3/15/main.tex}
\item The probability that a student will pass his examination is 0.73, the probability of
the student getting a compartment is 0.13, and the probability that the student will
either pass or get compartment is 0.96. State True or False.\\
\solution
%\input{exemplar/11/16/3/31/main.tex}
\item A card is selected from a pack of 52 cards\\
\begin{enumerate}[label=(\alph*)]
\item How many points are there in the sample space?
\item Calculate the probability that the cards is an ace of spades.
\item Calculate the probability that the card is (i) an ace (ii)black card.\\
\end{enumerate}
%\input{ncert/11/16/3/4_1/Prob_4.tex}
\item In a non-leap year, the probability of having 53 tuesdays or 53 wednesdays is\\
\solution
%\input{exemplar/11/16/3/18/main.tex}
\item There are 1000 sealed envelopes in a box, 10 of them contain a cash prize of
Rs 100 each, 100 of them contain a cash prize of Rs 50 each and 200 of them
contain a cash prize of Rs 10 each and rest do not contain any cash prize. If they
are well shuffled and an envelope is picked up out, what is the probability that it
contains no cash prize?\\
\solution
%\input{exemplar/10/13/3/34/main.tex}
\item 
A die is thrown and a card is selected at random from a deck of 52 playing cards. The probability of getting an even number on the die and a spade card.\\
\solution
%\input{exemplar/12/13/3/78/main.tex}
\item
If 4-digit numbers greater than 5,000 are randomly formed from the digits 0, 1, 3, 5, and 7, what is the probability of forming a number divisible by 5 when:
\begin{enumerate}
    \item The digits are repeated?
    \item The repetition of digits is not allowed?
\end{enumerate}
\solution
%\input{ncert/11/16/4/9/main.tex}
\item Consider the probability space $\brak{\Omega, \mathcal{G}, P}$ where $\Omega = [0,2]$ and $\mathcal{G} = \cbrak{\phi, \Omega, [0,1], (1,2]}$. Let $X$ and $Y$ be two functions on $\Omega$ defined as
\begin{align*}
    X(\omega) = 
    \begin{cases}
        1 & \text{if }\omega \in [0, 1]\\
        2 & \text{if }\omega \in (1, 2]
    \end{cases}
\end{align*}
and
\begin{align*}
    Y(\omega) = 
    \begin{cases}
        2 & \text{if }\omega \in [0, 1.5]\\
        3 & \text{if }\omega \in (1.5, 2].
    \end{cases}
\end{align*}
Then which one of the following statements is true?
\begin{enumerate}
    \item [(A)] $X$ is a random variable with respect to $\mathcal{G}$, but $Y$ is not a random variable with respect to $\mathcal{G}$.
    \item [(B)] $Y$ is a random variable with respect to $\mathcal{G}$, but $X$ is not a random variable with respect to $\mathcal{G}$.
    \item [(C)] Neither $X$ nor $Y$ is a random variable with respect to $\mathcal{G}$.
    \item [(D)] Both $X$ and $Y$ are random variables with respect to $\mathcal{G}$.
\end{enumerate} \hfill (GATE ST 2023)\\
\solution
%\input{gate/ST/2023/14/main.tex}
	\item  A die is loaded in such a way that each odd number is twice as likely to occur as
each even number. Find $P(G)$, where $G$ is the event that a number greater than
3 occurs on a single roll of the die.
\\
\solution
		%\input{exemplar/11/16/3/5/main.tex}
	\item All the jacks, queens and kings are removed from a deck of 52 playing cards. The remaining cards are well shuffled and then one card is drawn at random. Giving ace a value 1 similar value for other cards, find the probability that the card has a value 
		\begin{enumerate}
			\item 7
			\item greater than 7
			\item less than 7
		\end{enumerate}
		%\input{exemplar/10/13/3/30/main.tex}
  \item A Lot consists of 48 mobile phones of which 42 are good, 3 have only minor defects and 3 have major defects.Varnika will buy a phone if it is good but the trader will only buy a mobile if it has no major defects. One phone is selected at random from the lot. What is the probability that it is
\begin{enumerate}
	\item acceptable to Varnika?
            \item acceptable to the trader?
\end{enumerate}
\solution
	%\input{exemplar/10/13/3/40/main.tex}
 \item A student says that if you throw a die, it will show up 1 or not 1. Therefore, the probability of getting 1 and the probability of getting 'not 1' each is equal to $\frac{1}{2}$. Is this correct? Give reasons.\\
 \solution
        %\input{exemplar/10/13/2/9/main.tex}
   \item Four candidates A, B, C, D have ap-
plied for the assignment to coach a school cricket
team. If A is twice as likely to be selected as B, and
B and C are given about the same chance of being
selected, while C is twice as likely to be selected
as D, what are the probabilities that
\begin{enumerate}
\item C will be selected?
\item A will not be selected?
\end{enumerate}
	%\input{exemplar/11/16/3/9/main.tex}
 \item A bag contain 24 balls of which $x$ balls are red, $2x$ are white and $3x$ are blue. A ball is selected at random, What is the probability that it is
\begin{enumerate}[label=\alph*)]
\item not red ?
\item white ?
\end{enumerate}
%\input{exemplar/10/13/3/41/main.tex}
If the letters of the word ASSASSINATION are arranged at random. Find the Probability that
\begin{enumerate}[label=(\alph*)]
\item Four $S's$ come consecutively in the word
\item Two  $I's$ and two $N's$ come together
\item All $A's$ are not coming together
\item No two $A's$ are coming together
\end{enumerate}
%\input{exemplar/11/16/3/14/main.tex}
	\item One urn contains two black balls (labelled B1 and B2) and one white ball. A
	second urn contains one black ball and two white balls (labelled W1 and W2).
	Suppose the following experiment is performed. One of the two urns is chosen
	at random. Next a ball is randomly chosen from the urn. Then a second ball is
	chosen at random from the same urn without replacing the first ball.
	
	\begin{enumerate}
	\item What is the probability that two black balls are chosen?
	
	\item What is the probability that two balls of opposite colour are chosen?
	\end{enumerate}
	\solution
	%\input{exemplar/11/16/3/12/main1.tex}
\end{enumerate}

	\item A bag contains $5$ red balls and some blue balls. If the probability of drawing a blue ball is double that if a red ball, determine the number of blue balls in the bag. 
		\\
\solution
		%\begin{enumerate}[label=\thesection.\arabic*,ref=\thesection.\theenumi]
	\item One card is drawn from a well-shuffled deck of 52 cards. Find the probability of getting
\begin{enumerate}
\item A king of red colour 
\item A face card 
\item A red face card
\item The jack of hearts
\item A spade
\item The queen of diamonds

\end{enumerate}
\solution
		%\input{ncert/10/15/1/14/main.tex}
	\item Five cards—the ten, jack, queen, king and ace of diamonds, are well-shuffled with their face downwards. One card is then picked up at random.
\begin{enumerate}
\item
What is the probability that the card is the queen? 
\item
If the queen is drawn and put aside, what is the probability that the second card picked up is (a) an ace? (b) a queen?\\
\end{enumerate}
\solution
		%\input{ncert/10/15/1/15/defs.tex}
	\item A bag contains $5$ red balls and some blue balls. If the probability of drawing a blue ball is double that if a red ball, determine the number of blue balls in the bag. 
		\\
\solution
		%\input{ncert/10/15/2/3/defs.tex}
	\item A card is selected from a pack of 52 cards.
 \begin{enumerate}[label=(\alph*)] 
                 \item How many points are there in the sample space?
                 \item Calculate the probability that the card is an ace of spades.
                 \item Calculate the probability that the card is (i) an ace and (ii) black card.
 \end{enumerate}
\solution
		%\input{ncert/11/16/3/4/main.tex}
\item Four cards are drawn from a well-shuffled deck of 52 cards. What is the probability of obtaining 3 diamonds and one spade.
\\
\solution
		%\input{ncert/11/16/4/2/defs.tex}
\item In a certain lottery 10,000 tickets are sold and ten equal prizes are awarded. What is the probability of not getting a prize if you buy (a) one ticket (b) two tickets (c) 10 tickets ?	
\\
\solution
		%\input{ncert/11/16/4/4/defs.tex}
		%
\item 
Out of 100 students, two sections of 40 and 60 are formed. If you and your friend are among the 100 students, what is the probability that
\begin{enumerate}
\item you both enter the same section?
\item you both enter the different sections?
\end{enumerate}
\solution
		%\input{ncert/11/16/4/5/defs.tex}
	\item 
The number lock of a suitcase has 4 wheels each labelled with ten digits i.e. from 0 to 9.The lock opens with a sequence of four digits with no repeats.What is the probability of a person getting the right sequence to open the suitcase.
\\
\solution
		%\input{ncert/11/16/4/10/defs.tex}
		%
\item 
Two cards are drawn at random and without replacement from a pack of 52 playing cards. Find the probability that both the cards are black.
\\
\solution
		%\input{ncert/12/13/2/2/defs.tex}
		\item A box of oranges is inspected by examining three randomly selected oranges drawn without replacement. If all the three oranges are good, the box is approved for sale, otherwise, it is rejected. Find the probability that a box containing 15 oranges out of which 12 are good and 3 are bad ones will be approved for sale.
		\label{ncert/12/13/2/3/defs.tex}
		\item Two balls are drawn at random with replacement from a box containing 10 black and 8 red balls. Find the probability that
		\label{ncert/12/13/2/12}
\begin{enumerate}
\item both balls are red.
\item first ball is black and second is red.
\item one of them is black and other is red.
\end{enumerate}

\item In a hostel, 60\% of the students read Hindi newspaper, 40\% read English newspaper and 20\% read both Hindi and English newspapers. A student is selected at random.
		\label{ncert/12/13/2/15}
\begin{enumerate}
\item Find the probability that she reads neither Hindi nor English newspapers.
\item If she reads Hindi newspaper, find the probability that she reads English newspaper.
\item If she reads English newspaper, find the probability that she reads Hindi newspaper.\\
\end{enumerate}
\item The probability of obtaining an even prime number on each die, when a pair of dice is rolled is 
\begin{enumerate}
    \item $0$ 
    
    \item $\frac{1}{3}$ 
    
    \item $\frac{1}{12}$ 
    
    \item $\frac{1}{36}$ 
\end{enumerate}
\solution
		%\input{ncert/12/13/2/17/defs.tex}
	\item A bag contains 4 red and 4 black balls, another bag contains 2 red and 6 black balls. One of the two bags is selected at random and a ball is drawn from the bag which is found to be red. Find the probability that the ball is drawn from the first bag.
\\
\solution
		%\input{ncert/12/13/3/2/main.tex}
  \item
  Cards with numbers 2 to 101 are placed in a box. A card is selected at random.Find the probability that the card has
\begin{enumerate}[label=(\roman*)]
	\item an even number 
	\item a square number
\end{enumerate}
\solution
%\input{exemplar/10/13/3/32/main.tex}
\item
The king, queen and jack of clubs are removed from a deck of 52 playing cards and then well shuffled. Now one card is drawn at random from the remaining cards.  Determine the probability that the card is
\begin{enumerate}[label=(\roman*)]
\item a club
\item 10 of hearts
\end{enumerate}
\solution
%\input{exemplar/10/13/3/29/main.tex}
\item A team of medical students doing their internship have to assist during surgeries
at a city hospital. The probabilities of surgeries rated as very complex, complex,
routine, simple or very simple are respectively, 0.15, 0.20, 0.31, 0.26, .08. Find
the probabilities that a particular surgery will be rated
\begin{enumerate}
	\item complex or very complex;
	\item neither very complex nor very simple;
	\item routine or complex
	\item routine or simple
\end{enumerate}
\solution
%\input{exemplar/11/16/3/8(1)/main.tex}
\item A card is selected from a pack of 52 cards.
\begin{enumerate}[label=(\alph*)]
    \item How many points are there in the sample space?
    \item Calculate the probability that the card is an ace of spades.
    \item Calculate the probability that the card is (i) an ace and (ii) black card.
\end{enumerate}
\solution
%\input{exemplar/11/16/3/4/main2.tex}
\item The probability that a non leap year selected at random will contain 53 sundays.
\\
\solution
%\input{exemplar/10/13/1/19/main.tex}
\item One of the four persons John, Rita, Aslam or Gurpreet will be promoted next
month. Consequently the sample space consists of four elementary outcomes
S = {John promoted, Rita promoted, Aslam promoted, Gurpreet promoted}
You are told that the chances of John’s promotion is same as that of Gurpreet,
Rita’s chances of promotion are twice as likely as Johns. Aslam’s chances are
four times that of John.
\begin{enumerate}
	\item Determine
	\begin{enumerate}
		\item P (John promoted)
		\item P (Rita promoted)
		\item P (Aslam promoted)
		\item P (Gurpreet promoted)
	\end{enumerate}
	\item If A = {John promoted or Gurpreet promoted}, find P (A).
\end{enumerate}
\solution
%\input{exemplar/11/16/3/10/main.tex}
\item A card is drawn from a deck of 52 cards. Find the probability of getting a king or a heart or a red card.\\
\solution
%\input{exemplar/11/16/3/15/main.tex}
\item The probability that a student will pass his examination is 0.73, the probability of
the student getting a compartment is 0.13, and the probability that the student will
either pass or get compartment is 0.96. State True or False.\\
\solution
%\input{exemplar/11/16/3/31/main.tex}
\item A card is selected from a pack of 52 cards\\
\begin{enumerate}[label=(\alph*)]
\item How many points are there in the sample space?
\item Calculate the probability that the cards is an ace of spades.
\item Calculate the probability that the card is (i) an ace (ii)black card.\\
\end{enumerate}
%\input{ncert/11/16/3/4_1/Prob_4.tex}
\item In a non-leap year, the probability of having 53 tuesdays or 53 wednesdays is\\
\solution
%\input{exemplar/11/16/3/18/main.tex}
\item There are 1000 sealed envelopes in a box, 10 of them contain a cash prize of
Rs 100 each, 100 of them contain a cash prize of Rs 50 each and 200 of them
contain a cash prize of Rs 10 each and rest do not contain any cash prize. If they
are well shuffled and an envelope is picked up out, what is the probability that it
contains no cash prize?\\
\solution
%\input{exemplar/10/13/3/34/main.tex}
\item 
A die is thrown and a card is selected at random from a deck of 52 playing cards. The probability of getting an even number on the die and a spade card.\\
\solution
%\input{exemplar/12/13/3/78/main.tex}
\item
If 4-digit numbers greater than 5,000 are randomly formed from the digits 0, 1, 3, 5, and 7, what is the probability of forming a number divisible by 5 when:
\begin{enumerate}
    \item The digits are repeated?
    \item The repetition of digits is not allowed?
\end{enumerate}
\solution
%\input{ncert/11/16/4/9/main.tex}
\item Consider the probability space $\brak{\Omega, \mathcal{G}, P}$ where $\Omega = [0,2]$ and $\mathcal{G} = \cbrak{\phi, \Omega, [0,1], (1,2]}$. Let $X$ and $Y$ be two functions on $\Omega$ defined as
\begin{align*}
    X(\omega) = 
    \begin{cases}
        1 & \text{if }\omega \in [0, 1]\\
        2 & \text{if }\omega \in (1, 2]
    \end{cases}
\end{align*}
and
\begin{align*}
    Y(\omega) = 
    \begin{cases}
        2 & \text{if }\omega \in [0, 1.5]\\
        3 & \text{if }\omega \in (1.5, 2].
    \end{cases}
\end{align*}
Then which one of the following statements is true?
\begin{enumerate}
    \item [(A)] $X$ is a random variable with respect to $\mathcal{G}$, but $Y$ is not a random variable with respect to $\mathcal{G}$.
    \item [(B)] $Y$ is a random variable with respect to $\mathcal{G}$, but $X$ is not a random variable with respect to $\mathcal{G}$.
    \item [(C)] Neither $X$ nor $Y$ is a random variable with respect to $\mathcal{G}$.
    \item [(D)] Both $X$ and $Y$ are random variables with respect to $\mathcal{G}$.
\end{enumerate} \hfill (GATE ST 2023)\\
\solution
%\input{gate/ST/2023/14/main.tex}
	\item  A die is loaded in such a way that each odd number is twice as likely to occur as
each even number. Find $P(G)$, where $G$ is the event that a number greater than
3 occurs on a single roll of the die.
\\
\solution
		%\input{exemplar/11/16/3/5/main.tex}
	\item All the jacks, queens and kings are removed from a deck of 52 playing cards. The remaining cards are well shuffled and then one card is drawn at random. Giving ace a value 1 similar value for other cards, find the probability that the card has a value 
		\begin{enumerate}
			\item 7
			\item greater than 7
			\item less than 7
		\end{enumerate}
		%\input{exemplar/10/13/3/30/main.tex}
  \item A Lot consists of 48 mobile phones of which 42 are good, 3 have only minor defects and 3 have major defects.Varnika will buy a phone if it is good but the trader will only buy a mobile if it has no major defects. One phone is selected at random from the lot. What is the probability that it is
\begin{enumerate}
	\item acceptable to Varnika?
            \item acceptable to the trader?
\end{enumerate}
\solution
	%\input{exemplar/10/13/3/40/main.tex}
 \item A student says that if you throw a die, it will show up 1 or not 1. Therefore, the probability of getting 1 and the probability of getting 'not 1' each is equal to $\frac{1}{2}$. Is this correct? Give reasons.\\
 \solution
        %\input{exemplar/10/13/2/9/main.tex}
   \item Four candidates A, B, C, D have ap-
plied for the assignment to coach a school cricket
team. If A is twice as likely to be selected as B, and
B and C are given about the same chance of being
selected, while C is twice as likely to be selected
as D, what are the probabilities that
\begin{enumerate}
\item C will be selected?
\item A will not be selected?
\end{enumerate}
	%\input{exemplar/11/16/3/9/main.tex}
 \item A bag contain 24 balls of which $x$ balls are red, $2x$ are white and $3x$ are blue. A ball is selected at random, What is the probability that it is
\begin{enumerate}[label=\alph*)]
\item not red ?
\item white ?
\end{enumerate}
%\input{exemplar/10/13/3/41/main.tex}
If the letters of the word ASSASSINATION are arranged at random. Find the Probability that
\begin{enumerate}[label=(\alph*)]
\item Four $S's$ come consecutively in the word
\item Two  $I's$ and two $N's$ come together
\item All $A's$ are not coming together
\item No two $A's$ are coming together
\end{enumerate}
%\input{exemplar/11/16/3/14/main.tex}
	\item One urn contains two black balls (labelled B1 and B2) and one white ball. A
	second urn contains one black ball and two white balls (labelled W1 and W2).
	Suppose the following experiment is performed. One of the two urns is chosen
	at random. Next a ball is randomly chosen from the urn. Then a second ball is
	chosen at random from the same urn without replacing the first ball.
	
	\begin{enumerate}
	\item What is the probability that two black balls are chosen?
	
	\item What is the probability that two balls of opposite colour are chosen?
	\end{enumerate}
	\solution
	%\input{exemplar/11/16/3/12/main1.tex}
\end{enumerate}

	\item A card is selected from a pack of 52 cards.
 \begin{enumerate}[label=(\alph*)] 
                 \item How many points are there in the sample space?
                 \item Calculate the probability that the card is an ace of spades.
                 \item Calculate the probability that the card is (i) an ace and (ii) black card.
 \end{enumerate}
\solution
		%\begin{table}[H]
	\centering
\begin{tabular}{|c|c|c|}
\hline
Random variable &Value &Definition\\ \hline
\multirow{3}{*}{X} &0 &Slips of Rs 1\\
&1 &Slips of Rs 5\\
&2 &Slips of Rs 13\\ \hline
\multirow{2}{*}{Y} &0 &Box A\\
&1 &Box B\\\hline
\end{tabular}
\caption{}
\label{tab:Distribution}
\end{table}
See \tabref{tab:Distribution}.
\begin{align}
p_{Y}\brak{k}= \begin{cases} 
      \frac{1}{3} & {k=0} \\
      \frac{2}{3 }& {k=1} 
   \end{cases}
   \\
p_{Y|X}\brak{0|0} = \frac{19}{25}\, 
p_{Y|X}\brak{0|1} = \frac{6}{25}\,
p_{Y|X}\brak{1|0} = \frac{45}{50}\,
p_{Y|X}\brak{1|2} = \frac{5}{50}
\end{align}
The desired probability is the probability that a slip drawn at random is marked other than Rs 1,
\begin{align}
&=1-p_X\brak{0}\\
&= p_X(1) + p_X(2)
\end{align}
Using Bayes theorem,
\begin{align}
&= p_Y\brak{0} \times \pr{Y=0 | X=1} + p_Y\brak{1} \times \pr{Y=1|X=2}\\
&=\frac{1}{3} \times \frac{6}{25} + \frac{2}{3} \times \frac{5}{50}\\
&=\frac{11}{75}
\end{align}

\newpage

%\tableofcontents

\bigskip

\renewcommand{\thefigure}{\theenumi}
\renewcommand{\thetable}{\theenumi}
%\renewcommand{\theequation}{\theenumi}

%\begin{abstract}
%%\boldmath
%In this letter, an algorithm for evaluating the exact analytical bit error rate  (BER)  for the piecewise linear (PL) combiner for  multiple relays is presented. Previous results were available only for upto three relays. The algorithm is unique in the sense that  the actual mathematical expressions, that are prohibitively large, need not be explicitly obtained. The diversity gain due to multiple relays is shown through plots of the analytical BER, well supported by simulations. 
%
%\end{abstract}
% IEEEtran.cls defaults to using nonbold math in the Abstract.
% This preserves the distinction between vectors and scalars. However,
% if the journal you are submitting to favors bold math in the abstract,
% then you can use LaTeX's standard command \boldmath at the very start
% of the abstract to achieve this. Many IEEE journals frown on math
% in the abstract anyway.

% Note that keywords are not normally used for peerreview papers.
%\begin{IEEEkeywords}
%Cooperative diversity, decode and forward, piecewise linear
%\end{IEEEkeywords}



% For peer review papers, you can put extra information on the cover
% page as needed:
% \ifCLASSOPTIONpeerreview
% \begin{center} \bfseries EDICS Category: 3-BBND \end{center}
% \fi
%
% For peerreview papers, this IEEEtran command inserts a page break and
% creates the second title. It will be ignored for other modes.
%\IEEEpeerreviewmaketitle




\item Four cards are drawn from a well-shuffled deck of 52 cards. What is the probability of obtaining 3 diamonds and one spade.
\\
\solution
		%\begin{enumerate}[label=\thesection.\arabic*,ref=\thesection.\theenumi]
	\item One card is drawn from a well-shuffled deck of 52 cards. Find the probability of getting
\begin{enumerate}
\item A king of red colour 
\item A face card 
\item A red face card
\item The jack of hearts
\item A spade
\item The queen of diamonds

\end{enumerate}
\solution
		%\input{ncert/10/15/1/14/main.tex}
	\item Five cards—the ten, jack, queen, king and ace of diamonds, are well-shuffled with their face downwards. One card is then picked up at random.
\begin{enumerate}
\item
What is the probability that the card is the queen? 
\item
If the queen is drawn and put aside, what is the probability that the second card picked up is (a) an ace? (b) a queen?\\
\end{enumerate}
\solution
		%\input{ncert/10/15/1/15/defs.tex}
	\item A bag contains $5$ red balls and some blue balls. If the probability of drawing a blue ball is double that if a red ball, determine the number of blue balls in the bag. 
		\\
\solution
		%\input{ncert/10/15/2/3/defs.tex}
	\item A card is selected from a pack of 52 cards.
 \begin{enumerate}[label=(\alph*)] 
                 \item How many points are there in the sample space?
                 \item Calculate the probability that the card is an ace of spades.
                 \item Calculate the probability that the card is (i) an ace and (ii) black card.
 \end{enumerate}
\solution
		%\input{ncert/11/16/3/4/main.tex}
\item Four cards are drawn from a well-shuffled deck of 52 cards. What is the probability of obtaining 3 diamonds and one spade.
\\
\solution
		%\input{ncert/11/16/4/2/defs.tex}
\item In a certain lottery 10,000 tickets are sold and ten equal prizes are awarded. What is the probability of not getting a prize if you buy (a) one ticket (b) two tickets (c) 10 tickets ?	
\\
\solution
		%\input{ncert/11/16/4/4/defs.tex}
		%
\item 
Out of 100 students, two sections of 40 and 60 are formed. If you and your friend are among the 100 students, what is the probability that
\begin{enumerate}
\item you both enter the same section?
\item you both enter the different sections?
\end{enumerate}
\solution
		%\input{ncert/11/16/4/5/defs.tex}
	\item 
The number lock of a suitcase has 4 wheels each labelled with ten digits i.e. from 0 to 9.The lock opens with a sequence of four digits with no repeats.What is the probability of a person getting the right sequence to open the suitcase.
\\
\solution
		%\input{ncert/11/16/4/10/defs.tex}
		%
\item 
Two cards are drawn at random and without replacement from a pack of 52 playing cards. Find the probability that both the cards are black.
\\
\solution
		%\input{ncert/12/13/2/2/defs.tex}
		\item A box of oranges is inspected by examining three randomly selected oranges drawn without replacement. If all the three oranges are good, the box is approved for sale, otherwise, it is rejected. Find the probability that a box containing 15 oranges out of which 12 are good and 3 are bad ones will be approved for sale.
		\label{ncert/12/13/2/3/defs.tex}
		\item Two balls are drawn at random with replacement from a box containing 10 black and 8 red balls. Find the probability that
		\label{ncert/12/13/2/12}
\begin{enumerate}
\item both balls are red.
\item first ball is black and second is red.
\item one of them is black and other is red.
\end{enumerate}

\item In a hostel, 60\% of the students read Hindi newspaper, 40\% read English newspaper and 20\% read both Hindi and English newspapers. A student is selected at random.
		\label{ncert/12/13/2/15}
\begin{enumerate}
\item Find the probability that she reads neither Hindi nor English newspapers.
\item If she reads Hindi newspaper, find the probability that she reads English newspaper.
\item If she reads English newspaper, find the probability that she reads Hindi newspaper.\\
\end{enumerate}
\item The probability of obtaining an even prime number on each die, when a pair of dice is rolled is 
\begin{enumerate}
    \item $0$ 
    
    \item $\frac{1}{3}$ 
    
    \item $\frac{1}{12}$ 
    
    \item $\frac{1}{36}$ 
\end{enumerate}
\solution
		%\input{ncert/12/13/2/17/defs.tex}
	\item A bag contains 4 red and 4 black balls, another bag contains 2 red and 6 black balls. One of the two bags is selected at random and a ball is drawn from the bag which is found to be red. Find the probability that the ball is drawn from the first bag.
\\
\solution
		%\input{ncert/12/13/3/2/main.tex}
  \item
  Cards with numbers 2 to 101 are placed in a box. A card is selected at random.Find the probability that the card has
\begin{enumerate}[label=(\roman*)]
	\item an even number 
	\item a square number
\end{enumerate}
\solution
%\input{exemplar/10/13/3/32/main.tex}
\item
The king, queen and jack of clubs are removed from a deck of 52 playing cards and then well shuffled. Now one card is drawn at random from the remaining cards.  Determine the probability that the card is
\begin{enumerate}[label=(\roman*)]
\item a club
\item 10 of hearts
\end{enumerate}
\solution
%\input{exemplar/10/13/3/29/main.tex}
\item A team of medical students doing their internship have to assist during surgeries
at a city hospital. The probabilities of surgeries rated as very complex, complex,
routine, simple or very simple are respectively, 0.15, 0.20, 0.31, 0.26, .08. Find
the probabilities that a particular surgery will be rated
\begin{enumerate}
	\item complex or very complex;
	\item neither very complex nor very simple;
	\item routine or complex
	\item routine or simple
\end{enumerate}
\solution
%\input{exemplar/11/16/3/8(1)/main.tex}
\item A card is selected from a pack of 52 cards.
\begin{enumerate}[label=(\alph*)]
    \item How many points are there in the sample space?
    \item Calculate the probability that the card is an ace of spades.
    \item Calculate the probability that the card is (i) an ace and (ii) black card.
\end{enumerate}
\solution
%\input{exemplar/11/16/3/4/main2.tex}
\item The probability that a non leap year selected at random will contain 53 sundays.
\\
\solution
%\input{exemplar/10/13/1/19/main.tex}
\item One of the four persons John, Rita, Aslam or Gurpreet will be promoted next
month. Consequently the sample space consists of four elementary outcomes
S = {John promoted, Rita promoted, Aslam promoted, Gurpreet promoted}
You are told that the chances of John’s promotion is same as that of Gurpreet,
Rita’s chances of promotion are twice as likely as Johns. Aslam’s chances are
four times that of John.
\begin{enumerate}
	\item Determine
	\begin{enumerate}
		\item P (John promoted)
		\item P (Rita promoted)
		\item P (Aslam promoted)
		\item P (Gurpreet promoted)
	\end{enumerate}
	\item If A = {John promoted or Gurpreet promoted}, find P (A).
\end{enumerate}
\solution
%\input{exemplar/11/16/3/10/main.tex}
\item A card is drawn from a deck of 52 cards. Find the probability of getting a king or a heart or a red card.\\
\solution
%\input{exemplar/11/16/3/15/main.tex}
\item The probability that a student will pass his examination is 0.73, the probability of
the student getting a compartment is 0.13, and the probability that the student will
either pass or get compartment is 0.96. State True or False.\\
\solution
%\input{exemplar/11/16/3/31/main.tex}
\item A card is selected from a pack of 52 cards\\
\begin{enumerate}[label=(\alph*)]
\item How many points are there in the sample space?
\item Calculate the probability that the cards is an ace of spades.
\item Calculate the probability that the card is (i) an ace (ii)black card.\\
\end{enumerate}
%\input{ncert/11/16/3/4_1/Prob_4.tex}
\item In a non-leap year, the probability of having 53 tuesdays or 53 wednesdays is\\
\solution
%\input{exemplar/11/16/3/18/main.tex}
\item There are 1000 sealed envelopes in a box, 10 of them contain a cash prize of
Rs 100 each, 100 of them contain a cash prize of Rs 50 each and 200 of them
contain a cash prize of Rs 10 each and rest do not contain any cash prize. If they
are well shuffled and an envelope is picked up out, what is the probability that it
contains no cash prize?\\
\solution
%\input{exemplar/10/13/3/34/main.tex}
\item 
A die is thrown and a card is selected at random from a deck of 52 playing cards. The probability of getting an even number on the die and a spade card.\\
\solution
%\input{exemplar/12/13/3/78/main.tex}
\item
If 4-digit numbers greater than 5,000 are randomly formed from the digits 0, 1, 3, 5, and 7, what is the probability of forming a number divisible by 5 when:
\begin{enumerate}
    \item The digits are repeated?
    \item The repetition of digits is not allowed?
\end{enumerate}
\solution
%\input{ncert/11/16/4/9/main.tex}
\item Consider the probability space $\brak{\Omega, \mathcal{G}, P}$ where $\Omega = [0,2]$ and $\mathcal{G} = \cbrak{\phi, \Omega, [0,1], (1,2]}$. Let $X$ and $Y$ be two functions on $\Omega$ defined as
\begin{align*}
    X(\omega) = 
    \begin{cases}
        1 & \text{if }\omega \in [0, 1]\\
        2 & \text{if }\omega \in (1, 2]
    \end{cases}
\end{align*}
and
\begin{align*}
    Y(\omega) = 
    \begin{cases}
        2 & \text{if }\omega \in [0, 1.5]\\
        3 & \text{if }\omega \in (1.5, 2].
    \end{cases}
\end{align*}
Then which one of the following statements is true?
\begin{enumerate}
    \item [(A)] $X$ is a random variable with respect to $\mathcal{G}$, but $Y$ is not a random variable with respect to $\mathcal{G}$.
    \item [(B)] $Y$ is a random variable with respect to $\mathcal{G}$, but $X$ is not a random variable with respect to $\mathcal{G}$.
    \item [(C)] Neither $X$ nor $Y$ is a random variable with respect to $\mathcal{G}$.
    \item [(D)] Both $X$ and $Y$ are random variables with respect to $\mathcal{G}$.
\end{enumerate} \hfill (GATE ST 2023)\\
\solution
%\input{gate/ST/2023/14/main.tex}
	\item  A die is loaded in such a way that each odd number is twice as likely to occur as
each even number. Find $P(G)$, where $G$ is the event that a number greater than
3 occurs on a single roll of the die.
\\
\solution
		%\input{exemplar/11/16/3/5/main.tex}
	\item All the jacks, queens and kings are removed from a deck of 52 playing cards. The remaining cards are well shuffled and then one card is drawn at random. Giving ace a value 1 similar value for other cards, find the probability that the card has a value 
		\begin{enumerate}
			\item 7
			\item greater than 7
			\item less than 7
		\end{enumerate}
		%\input{exemplar/10/13/3/30/main.tex}
  \item A Lot consists of 48 mobile phones of which 42 are good, 3 have only minor defects and 3 have major defects.Varnika will buy a phone if it is good but the trader will only buy a mobile if it has no major defects. One phone is selected at random from the lot. What is the probability that it is
\begin{enumerate}
	\item acceptable to Varnika?
            \item acceptable to the trader?
\end{enumerate}
\solution
	%\input{exemplar/10/13/3/40/main.tex}
 \item A student says that if you throw a die, it will show up 1 or not 1. Therefore, the probability of getting 1 and the probability of getting 'not 1' each is equal to $\frac{1}{2}$. Is this correct? Give reasons.\\
 \solution
        %\input{exemplar/10/13/2/9/main.tex}
   \item Four candidates A, B, C, D have ap-
plied for the assignment to coach a school cricket
team. If A is twice as likely to be selected as B, and
B and C are given about the same chance of being
selected, while C is twice as likely to be selected
as D, what are the probabilities that
\begin{enumerate}
\item C will be selected?
\item A will not be selected?
\end{enumerate}
	%\input{exemplar/11/16/3/9/main.tex}
 \item A bag contain 24 balls of which $x$ balls are red, $2x$ are white and $3x$ are blue. A ball is selected at random, What is the probability that it is
\begin{enumerate}[label=\alph*)]
\item not red ?
\item white ?
\end{enumerate}
%\input{exemplar/10/13/3/41/main.tex}
If the letters of the word ASSASSINATION are arranged at random. Find the Probability that
\begin{enumerate}[label=(\alph*)]
\item Four $S's$ come consecutively in the word
\item Two  $I's$ and two $N's$ come together
\item All $A's$ are not coming together
\item No two $A's$ are coming together
\end{enumerate}
%\input{exemplar/11/16/3/14/main.tex}
	\item One urn contains two black balls (labelled B1 and B2) and one white ball. A
	second urn contains one black ball and two white balls (labelled W1 and W2).
	Suppose the following experiment is performed. One of the two urns is chosen
	at random. Next a ball is randomly chosen from the urn. Then a second ball is
	chosen at random from the same urn without replacing the first ball.
	
	\begin{enumerate}
	\item What is the probability that two black balls are chosen?
	
	\item What is the probability that two balls of opposite colour are chosen?
	\end{enumerate}
	\solution
	%\input{exemplar/11/16/3/12/main1.tex}
\end{enumerate}

\item In a certain lottery 10,000 tickets are sold and ten equal prizes are awarded. What is the probability of not getting a prize if you buy (a) one ticket (b) two tickets (c) 10 tickets ?	
\\
\solution
		%\begin{enumerate}[label=\thesection.\arabic*,ref=\thesection.\theenumi]
	\item One card is drawn from a well-shuffled deck of 52 cards. Find the probability of getting
\begin{enumerate}
\item A king of red colour 
\item A face card 
\item A red face card
\item The jack of hearts
\item A spade
\item The queen of diamonds

\end{enumerate}
\solution
		%\input{ncert/10/15/1/14/main.tex}
	\item Five cards—the ten, jack, queen, king and ace of diamonds, are well-shuffled with their face downwards. One card is then picked up at random.
\begin{enumerate}
\item
What is the probability that the card is the queen? 
\item
If the queen is drawn and put aside, what is the probability that the second card picked up is (a) an ace? (b) a queen?\\
\end{enumerate}
\solution
		%\input{ncert/10/15/1/15/defs.tex}
	\item A bag contains $5$ red balls and some blue balls. If the probability of drawing a blue ball is double that if a red ball, determine the number of blue balls in the bag. 
		\\
\solution
		%\input{ncert/10/15/2/3/defs.tex}
	\item A card is selected from a pack of 52 cards.
 \begin{enumerate}[label=(\alph*)] 
                 \item How many points are there in the sample space?
                 \item Calculate the probability that the card is an ace of spades.
                 \item Calculate the probability that the card is (i) an ace and (ii) black card.
 \end{enumerate}
\solution
		%\input{ncert/11/16/3/4/main.tex}
\item Four cards are drawn from a well-shuffled deck of 52 cards. What is the probability of obtaining 3 diamonds and one spade.
\\
\solution
		%\input{ncert/11/16/4/2/defs.tex}
\item In a certain lottery 10,000 tickets are sold and ten equal prizes are awarded. What is the probability of not getting a prize if you buy (a) one ticket (b) two tickets (c) 10 tickets ?	
\\
\solution
		%\input{ncert/11/16/4/4/defs.tex}
		%
\item 
Out of 100 students, two sections of 40 and 60 are formed. If you and your friend are among the 100 students, what is the probability that
\begin{enumerate}
\item you both enter the same section?
\item you both enter the different sections?
\end{enumerate}
\solution
		%\input{ncert/11/16/4/5/defs.tex}
	\item 
The number lock of a suitcase has 4 wheels each labelled with ten digits i.e. from 0 to 9.The lock opens with a sequence of four digits with no repeats.What is the probability of a person getting the right sequence to open the suitcase.
\\
\solution
		%\input{ncert/11/16/4/10/defs.tex}
		%
\item 
Two cards are drawn at random and without replacement from a pack of 52 playing cards. Find the probability that both the cards are black.
\\
\solution
		%\input{ncert/12/13/2/2/defs.tex}
		\item A box of oranges is inspected by examining three randomly selected oranges drawn without replacement. If all the three oranges are good, the box is approved for sale, otherwise, it is rejected. Find the probability that a box containing 15 oranges out of which 12 are good and 3 are bad ones will be approved for sale.
		\label{ncert/12/13/2/3/defs.tex}
		\item Two balls are drawn at random with replacement from a box containing 10 black and 8 red balls. Find the probability that
		\label{ncert/12/13/2/12}
\begin{enumerate}
\item both balls are red.
\item first ball is black and second is red.
\item one of them is black and other is red.
\end{enumerate}

\item In a hostel, 60\% of the students read Hindi newspaper, 40\% read English newspaper and 20\% read both Hindi and English newspapers. A student is selected at random.
		\label{ncert/12/13/2/15}
\begin{enumerate}
\item Find the probability that she reads neither Hindi nor English newspapers.
\item If she reads Hindi newspaper, find the probability that she reads English newspaper.
\item If she reads English newspaper, find the probability that she reads Hindi newspaper.\\
\end{enumerate}
\item The probability of obtaining an even prime number on each die, when a pair of dice is rolled is 
\begin{enumerate}
    \item $0$ 
    
    \item $\frac{1}{3}$ 
    
    \item $\frac{1}{12}$ 
    
    \item $\frac{1}{36}$ 
\end{enumerate}
\solution
		%\input{ncert/12/13/2/17/defs.tex}
	\item A bag contains 4 red and 4 black balls, another bag contains 2 red and 6 black balls. One of the two bags is selected at random and a ball is drawn from the bag which is found to be red. Find the probability that the ball is drawn from the first bag.
\\
\solution
		%\input{ncert/12/13/3/2/main.tex}
  \item
  Cards with numbers 2 to 101 are placed in a box. A card is selected at random.Find the probability that the card has
\begin{enumerate}[label=(\roman*)]
	\item an even number 
	\item a square number
\end{enumerate}
\solution
%\input{exemplar/10/13/3/32/main.tex}
\item
The king, queen and jack of clubs are removed from a deck of 52 playing cards and then well shuffled. Now one card is drawn at random from the remaining cards.  Determine the probability that the card is
\begin{enumerate}[label=(\roman*)]
\item a club
\item 10 of hearts
\end{enumerate}
\solution
%\input{exemplar/10/13/3/29/main.tex}
\item A team of medical students doing their internship have to assist during surgeries
at a city hospital. The probabilities of surgeries rated as very complex, complex,
routine, simple or very simple are respectively, 0.15, 0.20, 0.31, 0.26, .08. Find
the probabilities that a particular surgery will be rated
\begin{enumerate}
	\item complex or very complex;
	\item neither very complex nor very simple;
	\item routine or complex
	\item routine or simple
\end{enumerate}
\solution
%\input{exemplar/11/16/3/8(1)/main.tex}
\item A card is selected from a pack of 52 cards.
\begin{enumerate}[label=(\alph*)]
    \item How many points are there in the sample space?
    \item Calculate the probability that the card is an ace of spades.
    \item Calculate the probability that the card is (i) an ace and (ii) black card.
\end{enumerate}
\solution
%\input{exemplar/11/16/3/4/main2.tex}
\item The probability that a non leap year selected at random will contain 53 sundays.
\\
\solution
%\input{exemplar/10/13/1/19/main.tex}
\item One of the four persons John, Rita, Aslam or Gurpreet will be promoted next
month. Consequently the sample space consists of four elementary outcomes
S = {John promoted, Rita promoted, Aslam promoted, Gurpreet promoted}
You are told that the chances of John’s promotion is same as that of Gurpreet,
Rita’s chances of promotion are twice as likely as Johns. Aslam’s chances are
four times that of John.
\begin{enumerate}
	\item Determine
	\begin{enumerate}
		\item P (John promoted)
		\item P (Rita promoted)
		\item P (Aslam promoted)
		\item P (Gurpreet promoted)
	\end{enumerate}
	\item If A = {John promoted or Gurpreet promoted}, find P (A).
\end{enumerate}
\solution
%\input{exemplar/11/16/3/10/main.tex}
\item A card is drawn from a deck of 52 cards. Find the probability of getting a king or a heart or a red card.\\
\solution
%\input{exemplar/11/16/3/15/main.tex}
\item The probability that a student will pass his examination is 0.73, the probability of
the student getting a compartment is 0.13, and the probability that the student will
either pass or get compartment is 0.96. State True or False.\\
\solution
%\input{exemplar/11/16/3/31/main.tex}
\item A card is selected from a pack of 52 cards\\
\begin{enumerate}[label=(\alph*)]
\item How many points are there in the sample space?
\item Calculate the probability that the cards is an ace of spades.
\item Calculate the probability that the card is (i) an ace (ii)black card.\\
\end{enumerate}
%\input{ncert/11/16/3/4_1/Prob_4.tex}
\item In a non-leap year, the probability of having 53 tuesdays or 53 wednesdays is\\
\solution
%\input{exemplar/11/16/3/18/main.tex}
\item There are 1000 sealed envelopes in a box, 10 of them contain a cash prize of
Rs 100 each, 100 of them contain a cash prize of Rs 50 each and 200 of them
contain a cash prize of Rs 10 each and rest do not contain any cash prize. If they
are well shuffled and an envelope is picked up out, what is the probability that it
contains no cash prize?\\
\solution
%\input{exemplar/10/13/3/34/main.tex}
\item 
A die is thrown and a card is selected at random from a deck of 52 playing cards. The probability of getting an even number on the die and a spade card.\\
\solution
%\input{exemplar/12/13/3/78/main.tex}
\item
If 4-digit numbers greater than 5,000 are randomly formed from the digits 0, 1, 3, 5, and 7, what is the probability of forming a number divisible by 5 when:
\begin{enumerate}
    \item The digits are repeated?
    \item The repetition of digits is not allowed?
\end{enumerate}
\solution
%\input{ncert/11/16/4/9/main.tex}
\item Consider the probability space $\brak{\Omega, \mathcal{G}, P}$ where $\Omega = [0,2]$ and $\mathcal{G} = \cbrak{\phi, \Omega, [0,1], (1,2]}$. Let $X$ and $Y$ be two functions on $\Omega$ defined as
\begin{align*}
    X(\omega) = 
    \begin{cases}
        1 & \text{if }\omega \in [0, 1]\\
        2 & \text{if }\omega \in (1, 2]
    \end{cases}
\end{align*}
and
\begin{align*}
    Y(\omega) = 
    \begin{cases}
        2 & \text{if }\omega \in [0, 1.5]\\
        3 & \text{if }\omega \in (1.5, 2].
    \end{cases}
\end{align*}
Then which one of the following statements is true?
\begin{enumerate}
    \item [(A)] $X$ is a random variable with respect to $\mathcal{G}$, but $Y$ is not a random variable with respect to $\mathcal{G}$.
    \item [(B)] $Y$ is a random variable with respect to $\mathcal{G}$, but $X$ is not a random variable with respect to $\mathcal{G}$.
    \item [(C)] Neither $X$ nor $Y$ is a random variable with respect to $\mathcal{G}$.
    \item [(D)] Both $X$ and $Y$ are random variables with respect to $\mathcal{G}$.
\end{enumerate} \hfill (GATE ST 2023)\\
\solution
%\input{gate/ST/2023/14/main.tex}
	\item  A die is loaded in such a way that each odd number is twice as likely to occur as
each even number. Find $P(G)$, where $G$ is the event that a number greater than
3 occurs on a single roll of the die.
\\
\solution
		%\input{exemplar/11/16/3/5/main.tex}
	\item All the jacks, queens and kings are removed from a deck of 52 playing cards. The remaining cards are well shuffled and then one card is drawn at random. Giving ace a value 1 similar value for other cards, find the probability that the card has a value 
		\begin{enumerate}
			\item 7
			\item greater than 7
			\item less than 7
		\end{enumerate}
		%\input{exemplar/10/13/3/30/main.tex}
  \item A Lot consists of 48 mobile phones of which 42 are good, 3 have only minor defects and 3 have major defects.Varnika will buy a phone if it is good but the trader will only buy a mobile if it has no major defects. One phone is selected at random from the lot. What is the probability that it is
\begin{enumerate}
	\item acceptable to Varnika?
            \item acceptable to the trader?
\end{enumerate}
\solution
	%\input{exemplar/10/13/3/40/main.tex}
 \item A student says that if you throw a die, it will show up 1 or not 1. Therefore, the probability of getting 1 and the probability of getting 'not 1' each is equal to $\frac{1}{2}$. Is this correct? Give reasons.\\
 \solution
        %\input{exemplar/10/13/2/9/main.tex}
   \item Four candidates A, B, C, D have ap-
plied for the assignment to coach a school cricket
team. If A is twice as likely to be selected as B, and
B and C are given about the same chance of being
selected, while C is twice as likely to be selected
as D, what are the probabilities that
\begin{enumerate}
\item C will be selected?
\item A will not be selected?
\end{enumerate}
	%\input{exemplar/11/16/3/9/main.tex}
 \item A bag contain 24 balls of which $x$ balls are red, $2x$ are white and $3x$ are blue. A ball is selected at random, What is the probability that it is
\begin{enumerate}[label=\alph*)]
\item not red ?
\item white ?
\end{enumerate}
%\input{exemplar/10/13/3/41/main.tex}
If the letters of the word ASSASSINATION are arranged at random. Find the Probability that
\begin{enumerate}[label=(\alph*)]
\item Four $S's$ come consecutively in the word
\item Two  $I's$ and two $N's$ come together
\item All $A's$ are not coming together
\item No two $A's$ are coming together
\end{enumerate}
%\input{exemplar/11/16/3/14/main.tex}
	\item One urn contains two black balls (labelled B1 and B2) and one white ball. A
	second urn contains one black ball and two white balls (labelled W1 and W2).
	Suppose the following experiment is performed. One of the two urns is chosen
	at random. Next a ball is randomly chosen from the urn. Then a second ball is
	chosen at random from the same urn without replacing the first ball.
	
	\begin{enumerate}
	\item What is the probability that two black balls are chosen?
	
	\item What is the probability that two balls of opposite colour are chosen?
	\end{enumerate}
	\solution
	%\input{exemplar/11/16/3/12/main1.tex}
\end{enumerate}

		%
\item 
Out of 100 students, two sections of 40 and 60 are formed. If you and your friend are among the 100 students, what is the probability that
\begin{enumerate}
\item you both enter the same section?
\item you both enter the different sections?
\end{enumerate}
\solution
		%\begin{enumerate}[label=\thesection.\arabic*,ref=\thesection.\theenumi]
	\item One card is drawn from a well-shuffled deck of 52 cards. Find the probability of getting
\begin{enumerate}
\item A king of red colour 
\item A face card 
\item A red face card
\item The jack of hearts
\item A spade
\item The queen of diamonds

\end{enumerate}
\solution
		%\input{ncert/10/15/1/14/main.tex}
	\item Five cards—the ten, jack, queen, king and ace of diamonds, are well-shuffled with their face downwards. One card is then picked up at random.
\begin{enumerate}
\item
What is the probability that the card is the queen? 
\item
If the queen is drawn and put aside, what is the probability that the second card picked up is (a) an ace? (b) a queen?\\
\end{enumerate}
\solution
		%\input{ncert/10/15/1/15/defs.tex}
	\item A bag contains $5$ red balls and some blue balls. If the probability of drawing a blue ball is double that if a red ball, determine the number of blue balls in the bag. 
		\\
\solution
		%\input{ncert/10/15/2/3/defs.tex}
	\item A card is selected from a pack of 52 cards.
 \begin{enumerate}[label=(\alph*)] 
                 \item How many points are there in the sample space?
                 \item Calculate the probability that the card is an ace of spades.
                 \item Calculate the probability that the card is (i) an ace and (ii) black card.
 \end{enumerate}
\solution
		%\input{ncert/11/16/3/4/main.tex}
\item Four cards are drawn from a well-shuffled deck of 52 cards. What is the probability of obtaining 3 diamonds and one spade.
\\
\solution
		%\input{ncert/11/16/4/2/defs.tex}
\item In a certain lottery 10,000 tickets are sold and ten equal prizes are awarded. What is the probability of not getting a prize if you buy (a) one ticket (b) two tickets (c) 10 tickets ?	
\\
\solution
		%\input{ncert/11/16/4/4/defs.tex}
		%
\item 
Out of 100 students, two sections of 40 and 60 are formed. If you and your friend are among the 100 students, what is the probability that
\begin{enumerate}
\item you both enter the same section?
\item you both enter the different sections?
\end{enumerate}
\solution
		%\input{ncert/11/16/4/5/defs.tex}
	\item 
The number lock of a suitcase has 4 wheels each labelled with ten digits i.e. from 0 to 9.The lock opens with a sequence of four digits with no repeats.What is the probability of a person getting the right sequence to open the suitcase.
\\
\solution
		%\input{ncert/11/16/4/10/defs.tex}
		%
\item 
Two cards are drawn at random and without replacement from a pack of 52 playing cards. Find the probability that both the cards are black.
\\
\solution
		%\input{ncert/12/13/2/2/defs.tex}
		\item A box of oranges is inspected by examining three randomly selected oranges drawn without replacement. If all the three oranges are good, the box is approved for sale, otherwise, it is rejected. Find the probability that a box containing 15 oranges out of which 12 are good and 3 are bad ones will be approved for sale.
		\label{ncert/12/13/2/3/defs.tex}
		\item Two balls are drawn at random with replacement from a box containing 10 black and 8 red balls. Find the probability that
		\label{ncert/12/13/2/12}
\begin{enumerate}
\item both balls are red.
\item first ball is black and second is red.
\item one of them is black and other is red.
\end{enumerate}

\item In a hostel, 60\% of the students read Hindi newspaper, 40\% read English newspaper and 20\% read both Hindi and English newspapers. A student is selected at random.
		\label{ncert/12/13/2/15}
\begin{enumerate}
\item Find the probability that she reads neither Hindi nor English newspapers.
\item If she reads Hindi newspaper, find the probability that she reads English newspaper.
\item If she reads English newspaper, find the probability that she reads Hindi newspaper.\\
\end{enumerate}
\item The probability of obtaining an even prime number on each die, when a pair of dice is rolled is 
\begin{enumerate}
    \item $0$ 
    
    \item $\frac{1}{3}$ 
    
    \item $\frac{1}{12}$ 
    
    \item $\frac{1}{36}$ 
\end{enumerate}
\solution
		%\input{ncert/12/13/2/17/defs.tex}
	\item A bag contains 4 red and 4 black balls, another bag contains 2 red and 6 black balls. One of the two bags is selected at random and a ball is drawn from the bag which is found to be red. Find the probability that the ball is drawn from the first bag.
\\
\solution
		%\input{ncert/12/13/3/2/main.tex}
  \item
  Cards with numbers 2 to 101 are placed in a box. A card is selected at random.Find the probability that the card has
\begin{enumerate}[label=(\roman*)]
	\item an even number 
	\item a square number
\end{enumerate}
\solution
%\input{exemplar/10/13/3/32/main.tex}
\item
The king, queen and jack of clubs are removed from a deck of 52 playing cards and then well shuffled. Now one card is drawn at random from the remaining cards.  Determine the probability that the card is
\begin{enumerate}[label=(\roman*)]
\item a club
\item 10 of hearts
\end{enumerate}
\solution
%\input{exemplar/10/13/3/29/main.tex}
\item A team of medical students doing their internship have to assist during surgeries
at a city hospital. The probabilities of surgeries rated as very complex, complex,
routine, simple or very simple are respectively, 0.15, 0.20, 0.31, 0.26, .08. Find
the probabilities that a particular surgery will be rated
\begin{enumerate}
	\item complex or very complex;
	\item neither very complex nor very simple;
	\item routine or complex
	\item routine or simple
\end{enumerate}
\solution
%\input{exemplar/11/16/3/8(1)/main.tex}
\item A card is selected from a pack of 52 cards.
\begin{enumerate}[label=(\alph*)]
    \item How many points are there in the sample space?
    \item Calculate the probability that the card is an ace of spades.
    \item Calculate the probability that the card is (i) an ace and (ii) black card.
\end{enumerate}
\solution
%\input{exemplar/11/16/3/4/main2.tex}
\item The probability that a non leap year selected at random will contain 53 sundays.
\\
\solution
%\input{exemplar/10/13/1/19/main.tex}
\item One of the four persons John, Rita, Aslam or Gurpreet will be promoted next
month. Consequently the sample space consists of four elementary outcomes
S = {John promoted, Rita promoted, Aslam promoted, Gurpreet promoted}
You are told that the chances of John’s promotion is same as that of Gurpreet,
Rita’s chances of promotion are twice as likely as Johns. Aslam’s chances are
four times that of John.
\begin{enumerate}
	\item Determine
	\begin{enumerate}
		\item P (John promoted)
		\item P (Rita promoted)
		\item P (Aslam promoted)
		\item P (Gurpreet promoted)
	\end{enumerate}
	\item If A = {John promoted or Gurpreet promoted}, find P (A).
\end{enumerate}
\solution
%\input{exemplar/11/16/3/10/main.tex}
\item A card is drawn from a deck of 52 cards. Find the probability of getting a king or a heart or a red card.\\
\solution
%\input{exemplar/11/16/3/15/main.tex}
\item The probability that a student will pass his examination is 0.73, the probability of
the student getting a compartment is 0.13, and the probability that the student will
either pass or get compartment is 0.96. State True or False.\\
\solution
%\input{exemplar/11/16/3/31/main.tex}
\item A card is selected from a pack of 52 cards\\
\begin{enumerate}[label=(\alph*)]
\item How many points are there in the sample space?
\item Calculate the probability that the cards is an ace of spades.
\item Calculate the probability that the card is (i) an ace (ii)black card.\\
\end{enumerate}
%\input{ncert/11/16/3/4_1/Prob_4.tex}
\item In a non-leap year, the probability of having 53 tuesdays or 53 wednesdays is\\
\solution
%\input{exemplar/11/16/3/18/main.tex}
\item There are 1000 sealed envelopes in a box, 10 of them contain a cash prize of
Rs 100 each, 100 of them contain a cash prize of Rs 50 each and 200 of them
contain a cash prize of Rs 10 each and rest do not contain any cash prize. If they
are well shuffled and an envelope is picked up out, what is the probability that it
contains no cash prize?\\
\solution
%\input{exemplar/10/13/3/34/main.tex}
\item 
A die is thrown and a card is selected at random from a deck of 52 playing cards. The probability of getting an even number on the die and a spade card.\\
\solution
%\input{exemplar/12/13/3/78/main.tex}
\item
If 4-digit numbers greater than 5,000 are randomly formed from the digits 0, 1, 3, 5, and 7, what is the probability of forming a number divisible by 5 when:
\begin{enumerate}
    \item The digits are repeated?
    \item The repetition of digits is not allowed?
\end{enumerate}
\solution
%\input{ncert/11/16/4/9/main.tex}
\item Consider the probability space $\brak{\Omega, \mathcal{G}, P}$ where $\Omega = [0,2]$ and $\mathcal{G} = \cbrak{\phi, \Omega, [0,1], (1,2]}$. Let $X$ and $Y$ be two functions on $\Omega$ defined as
\begin{align*}
    X(\omega) = 
    \begin{cases}
        1 & \text{if }\omega \in [0, 1]\\
        2 & \text{if }\omega \in (1, 2]
    \end{cases}
\end{align*}
and
\begin{align*}
    Y(\omega) = 
    \begin{cases}
        2 & \text{if }\omega \in [0, 1.5]\\
        3 & \text{if }\omega \in (1.5, 2].
    \end{cases}
\end{align*}
Then which one of the following statements is true?
\begin{enumerate}
    \item [(A)] $X$ is a random variable with respect to $\mathcal{G}$, but $Y$ is not a random variable with respect to $\mathcal{G}$.
    \item [(B)] $Y$ is a random variable with respect to $\mathcal{G}$, but $X$ is not a random variable with respect to $\mathcal{G}$.
    \item [(C)] Neither $X$ nor $Y$ is a random variable with respect to $\mathcal{G}$.
    \item [(D)] Both $X$ and $Y$ are random variables with respect to $\mathcal{G}$.
\end{enumerate} \hfill (GATE ST 2023)\\
\solution
%\input{gate/ST/2023/14/main.tex}
	\item  A die is loaded in such a way that each odd number is twice as likely to occur as
each even number. Find $P(G)$, where $G$ is the event that a number greater than
3 occurs on a single roll of the die.
\\
\solution
		%\input{exemplar/11/16/3/5/main.tex}
	\item All the jacks, queens and kings are removed from a deck of 52 playing cards. The remaining cards are well shuffled and then one card is drawn at random. Giving ace a value 1 similar value for other cards, find the probability that the card has a value 
		\begin{enumerate}
			\item 7
			\item greater than 7
			\item less than 7
		\end{enumerate}
		%\input{exemplar/10/13/3/30/main.tex}
  \item A Lot consists of 48 mobile phones of which 42 are good, 3 have only minor defects and 3 have major defects.Varnika will buy a phone if it is good but the trader will only buy a mobile if it has no major defects. One phone is selected at random from the lot. What is the probability that it is
\begin{enumerate}
	\item acceptable to Varnika?
            \item acceptable to the trader?
\end{enumerate}
\solution
	%\input{exemplar/10/13/3/40/main.tex}
 \item A student says that if you throw a die, it will show up 1 or not 1. Therefore, the probability of getting 1 and the probability of getting 'not 1' each is equal to $\frac{1}{2}$. Is this correct? Give reasons.\\
 \solution
        %\input{exemplar/10/13/2/9/main.tex}
   \item Four candidates A, B, C, D have ap-
plied for the assignment to coach a school cricket
team. If A is twice as likely to be selected as B, and
B and C are given about the same chance of being
selected, while C is twice as likely to be selected
as D, what are the probabilities that
\begin{enumerate}
\item C will be selected?
\item A will not be selected?
\end{enumerate}
	%\input{exemplar/11/16/3/9/main.tex}
 \item A bag contain 24 balls of which $x$ balls are red, $2x$ are white and $3x$ are blue. A ball is selected at random, What is the probability that it is
\begin{enumerate}[label=\alph*)]
\item not red ?
\item white ?
\end{enumerate}
%\input{exemplar/10/13/3/41/main.tex}
If the letters of the word ASSASSINATION are arranged at random. Find the Probability that
\begin{enumerate}[label=(\alph*)]
\item Four $S's$ come consecutively in the word
\item Two  $I's$ and two $N's$ come together
\item All $A's$ are not coming together
\item No two $A's$ are coming together
\end{enumerate}
%\input{exemplar/11/16/3/14/main.tex}
	\item One urn contains two black balls (labelled B1 and B2) and one white ball. A
	second urn contains one black ball and two white balls (labelled W1 and W2).
	Suppose the following experiment is performed. One of the two urns is chosen
	at random. Next a ball is randomly chosen from the urn. Then a second ball is
	chosen at random from the same urn without replacing the first ball.
	
	\begin{enumerate}
	\item What is the probability that two black balls are chosen?
	
	\item What is the probability that two balls of opposite colour are chosen?
	\end{enumerate}
	\solution
	%\input{exemplar/11/16/3/12/main1.tex}
\end{enumerate}

	\item 
The number lock of a suitcase has 4 wheels each labelled with ten digits i.e. from 0 to 9.The lock opens with a sequence of four digits with no repeats.What is the probability of a person getting the right sequence to open the suitcase.
\\
\solution
		%\begin{enumerate}[label=\thesection.\arabic*,ref=\thesection.\theenumi]
	\item One card is drawn from a well-shuffled deck of 52 cards. Find the probability of getting
\begin{enumerate}
\item A king of red colour 
\item A face card 
\item A red face card
\item The jack of hearts
\item A spade
\item The queen of diamonds

\end{enumerate}
\solution
		%\input{ncert/10/15/1/14/main.tex}
	\item Five cards—the ten, jack, queen, king and ace of diamonds, are well-shuffled with their face downwards. One card is then picked up at random.
\begin{enumerate}
\item
What is the probability that the card is the queen? 
\item
If the queen is drawn and put aside, what is the probability that the second card picked up is (a) an ace? (b) a queen?\\
\end{enumerate}
\solution
		%\input{ncert/10/15/1/15/defs.tex}
	\item A bag contains $5$ red balls and some blue balls. If the probability of drawing a blue ball is double that if a red ball, determine the number of blue balls in the bag. 
		\\
\solution
		%\input{ncert/10/15/2/3/defs.tex}
	\item A card is selected from a pack of 52 cards.
 \begin{enumerate}[label=(\alph*)] 
                 \item How many points are there in the sample space?
                 \item Calculate the probability that the card is an ace of spades.
                 \item Calculate the probability that the card is (i) an ace and (ii) black card.
 \end{enumerate}
\solution
		%\input{ncert/11/16/3/4/main.tex}
\item Four cards are drawn from a well-shuffled deck of 52 cards. What is the probability of obtaining 3 diamonds and one spade.
\\
\solution
		%\input{ncert/11/16/4/2/defs.tex}
\item In a certain lottery 10,000 tickets are sold and ten equal prizes are awarded. What is the probability of not getting a prize if you buy (a) one ticket (b) two tickets (c) 10 tickets ?	
\\
\solution
		%\input{ncert/11/16/4/4/defs.tex}
		%
\item 
Out of 100 students, two sections of 40 and 60 are formed. If you and your friend are among the 100 students, what is the probability that
\begin{enumerate}
\item you both enter the same section?
\item you both enter the different sections?
\end{enumerate}
\solution
		%\input{ncert/11/16/4/5/defs.tex}
	\item 
The number lock of a suitcase has 4 wheels each labelled with ten digits i.e. from 0 to 9.The lock opens with a sequence of four digits with no repeats.What is the probability of a person getting the right sequence to open the suitcase.
\\
\solution
		%\input{ncert/11/16/4/10/defs.tex}
		%
\item 
Two cards are drawn at random and without replacement from a pack of 52 playing cards. Find the probability that both the cards are black.
\\
\solution
		%\input{ncert/12/13/2/2/defs.tex}
		\item A box of oranges is inspected by examining three randomly selected oranges drawn without replacement. If all the three oranges are good, the box is approved for sale, otherwise, it is rejected. Find the probability that a box containing 15 oranges out of which 12 are good and 3 are bad ones will be approved for sale.
		\label{ncert/12/13/2/3/defs.tex}
		\item Two balls are drawn at random with replacement from a box containing 10 black and 8 red balls. Find the probability that
		\label{ncert/12/13/2/12}
\begin{enumerate}
\item both balls are red.
\item first ball is black and second is red.
\item one of them is black and other is red.
\end{enumerate}

\item In a hostel, 60\% of the students read Hindi newspaper, 40\% read English newspaper and 20\% read both Hindi and English newspapers. A student is selected at random.
		\label{ncert/12/13/2/15}
\begin{enumerate}
\item Find the probability that she reads neither Hindi nor English newspapers.
\item If she reads Hindi newspaper, find the probability that she reads English newspaper.
\item If she reads English newspaper, find the probability that she reads Hindi newspaper.\\
\end{enumerate}
\item The probability of obtaining an even prime number on each die, when a pair of dice is rolled is 
\begin{enumerate}
    \item $0$ 
    
    \item $\frac{1}{3}$ 
    
    \item $\frac{1}{12}$ 
    
    \item $\frac{1}{36}$ 
\end{enumerate}
\solution
		%\input{ncert/12/13/2/17/defs.tex}
	\item A bag contains 4 red and 4 black balls, another bag contains 2 red and 6 black balls. One of the two bags is selected at random and a ball is drawn from the bag which is found to be red. Find the probability that the ball is drawn from the first bag.
\\
\solution
		%\input{ncert/12/13/3/2/main.tex}
  \item
  Cards with numbers 2 to 101 are placed in a box. A card is selected at random.Find the probability that the card has
\begin{enumerate}[label=(\roman*)]
	\item an even number 
	\item a square number
\end{enumerate}
\solution
%\input{exemplar/10/13/3/32/main.tex}
\item
The king, queen and jack of clubs are removed from a deck of 52 playing cards and then well shuffled. Now one card is drawn at random from the remaining cards.  Determine the probability that the card is
\begin{enumerate}[label=(\roman*)]
\item a club
\item 10 of hearts
\end{enumerate}
\solution
%\input{exemplar/10/13/3/29/main.tex}
\item A team of medical students doing their internship have to assist during surgeries
at a city hospital. The probabilities of surgeries rated as very complex, complex,
routine, simple or very simple are respectively, 0.15, 0.20, 0.31, 0.26, .08. Find
the probabilities that a particular surgery will be rated
\begin{enumerate}
	\item complex or very complex;
	\item neither very complex nor very simple;
	\item routine or complex
	\item routine or simple
\end{enumerate}
\solution
%\input{exemplar/11/16/3/8(1)/main.tex}
\item A card is selected from a pack of 52 cards.
\begin{enumerate}[label=(\alph*)]
    \item How many points are there in the sample space?
    \item Calculate the probability that the card is an ace of spades.
    \item Calculate the probability that the card is (i) an ace and (ii) black card.
\end{enumerate}
\solution
%\input{exemplar/11/16/3/4/main2.tex}
\item The probability that a non leap year selected at random will contain 53 sundays.
\\
\solution
%\input{exemplar/10/13/1/19/main.tex}
\item One of the four persons John, Rita, Aslam or Gurpreet will be promoted next
month. Consequently the sample space consists of four elementary outcomes
S = {John promoted, Rita promoted, Aslam promoted, Gurpreet promoted}
You are told that the chances of John’s promotion is same as that of Gurpreet,
Rita’s chances of promotion are twice as likely as Johns. Aslam’s chances are
four times that of John.
\begin{enumerate}
	\item Determine
	\begin{enumerate}
		\item P (John promoted)
		\item P (Rita promoted)
		\item P (Aslam promoted)
		\item P (Gurpreet promoted)
	\end{enumerate}
	\item If A = {John promoted or Gurpreet promoted}, find P (A).
\end{enumerate}
\solution
%\input{exemplar/11/16/3/10/main.tex}
\item A card is drawn from a deck of 52 cards. Find the probability of getting a king or a heart or a red card.\\
\solution
%\input{exemplar/11/16/3/15/main.tex}
\item The probability that a student will pass his examination is 0.73, the probability of
the student getting a compartment is 0.13, and the probability that the student will
either pass or get compartment is 0.96. State True or False.\\
\solution
%\input{exemplar/11/16/3/31/main.tex}
\item A card is selected from a pack of 52 cards\\
\begin{enumerate}[label=(\alph*)]
\item How many points are there in the sample space?
\item Calculate the probability that the cards is an ace of spades.
\item Calculate the probability that the card is (i) an ace (ii)black card.\\
\end{enumerate}
%\input{ncert/11/16/3/4_1/Prob_4.tex}
\item In a non-leap year, the probability of having 53 tuesdays or 53 wednesdays is\\
\solution
%\input{exemplar/11/16/3/18/main.tex}
\item There are 1000 sealed envelopes in a box, 10 of them contain a cash prize of
Rs 100 each, 100 of them contain a cash prize of Rs 50 each and 200 of them
contain a cash prize of Rs 10 each and rest do not contain any cash prize. If they
are well shuffled and an envelope is picked up out, what is the probability that it
contains no cash prize?\\
\solution
%\input{exemplar/10/13/3/34/main.tex}
\item 
A die is thrown and a card is selected at random from a deck of 52 playing cards. The probability of getting an even number on the die and a spade card.\\
\solution
%\input{exemplar/12/13/3/78/main.tex}
\item
If 4-digit numbers greater than 5,000 are randomly formed from the digits 0, 1, 3, 5, and 7, what is the probability of forming a number divisible by 5 when:
\begin{enumerate}
    \item The digits are repeated?
    \item The repetition of digits is not allowed?
\end{enumerate}
\solution
%\input{ncert/11/16/4/9/main.tex}
\item Consider the probability space $\brak{\Omega, \mathcal{G}, P}$ where $\Omega = [0,2]$ and $\mathcal{G} = \cbrak{\phi, \Omega, [0,1], (1,2]}$. Let $X$ and $Y$ be two functions on $\Omega$ defined as
\begin{align*}
    X(\omega) = 
    \begin{cases}
        1 & \text{if }\omega \in [0, 1]\\
        2 & \text{if }\omega \in (1, 2]
    \end{cases}
\end{align*}
and
\begin{align*}
    Y(\omega) = 
    \begin{cases}
        2 & \text{if }\omega \in [0, 1.5]\\
        3 & \text{if }\omega \in (1.5, 2].
    \end{cases}
\end{align*}
Then which one of the following statements is true?
\begin{enumerate}
    \item [(A)] $X$ is a random variable with respect to $\mathcal{G}$, but $Y$ is not a random variable with respect to $\mathcal{G}$.
    \item [(B)] $Y$ is a random variable with respect to $\mathcal{G}$, but $X$ is not a random variable with respect to $\mathcal{G}$.
    \item [(C)] Neither $X$ nor $Y$ is a random variable with respect to $\mathcal{G}$.
    \item [(D)] Both $X$ and $Y$ are random variables with respect to $\mathcal{G}$.
\end{enumerate} \hfill (GATE ST 2023)\\
\solution
%\input{gate/ST/2023/14/main.tex}
	\item  A die is loaded in such a way that each odd number is twice as likely to occur as
each even number. Find $P(G)$, where $G$ is the event that a number greater than
3 occurs on a single roll of the die.
\\
\solution
		%\input{exemplar/11/16/3/5/main.tex}
	\item All the jacks, queens and kings are removed from a deck of 52 playing cards. The remaining cards are well shuffled and then one card is drawn at random. Giving ace a value 1 similar value for other cards, find the probability that the card has a value 
		\begin{enumerate}
			\item 7
			\item greater than 7
			\item less than 7
		\end{enumerate}
		%\input{exemplar/10/13/3/30/main.tex}
  \item A Lot consists of 48 mobile phones of which 42 are good, 3 have only minor defects and 3 have major defects.Varnika will buy a phone if it is good but the trader will only buy a mobile if it has no major defects. One phone is selected at random from the lot. What is the probability that it is
\begin{enumerate}
	\item acceptable to Varnika?
            \item acceptable to the trader?
\end{enumerate}
\solution
	%\input{exemplar/10/13/3/40/main.tex}
 \item A student says that if you throw a die, it will show up 1 or not 1. Therefore, the probability of getting 1 and the probability of getting 'not 1' each is equal to $\frac{1}{2}$. Is this correct? Give reasons.\\
 \solution
        %\input{exemplar/10/13/2/9/main.tex}
   \item Four candidates A, B, C, D have ap-
plied for the assignment to coach a school cricket
team. If A is twice as likely to be selected as B, and
B and C are given about the same chance of being
selected, while C is twice as likely to be selected
as D, what are the probabilities that
\begin{enumerate}
\item C will be selected?
\item A will not be selected?
\end{enumerate}
	%\input{exemplar/11/16/3/9/main.tex}
 \item A bag contain 24 balls of which $x$ balls are red, $2x$ are white and $3x$ are blue. A ball is selected at random, What is the probability that it is
\begin{enumerate}[label=\alph*)]
\item not red ?
\item white ?
\end{enumerate}
%\input{exemplar/10/13/3/41/main.tex}
If the letters of the word ASSASSINATION are arranged at random. Find the Probability that
\begin{enumerate}[label=(\alph*)]
\item Four $S's$ come consecutively in the word
\item Two  $I's$ and two $N's$ come together
\item All $A's$ are not coming together
\item No two $A's$ are coming together
\end{enumerate}
%\input{exemplar/11/16/3/14/main.tex}
	\item One urn contains two black balls (labelled B1 and B2) and one white ball. A
	second urn contains one black ball and two white balls (labelled W1 and W2).
	Suppose the following experiment is performed. One of the two urns is chosen
	at random. Next a ball is randomly chosen from the urn. Then a second ball is
	chosen at random from the same urn without replacing the first ball.
	
	\begin{enumerate}
	\item What is the probability that two black balls are chosen?
	
	\item What is the probability that two balls of opposite colour are chosen?
	\end{enumerate}
	\solution
	%\input{exemplar/11/16/3/12/main1.tex}
\end{enumerate}

		%
\item 
Two cards are drawn at random and without replacement from a pack of 52 playing cards. Find the probability that both the cards are black.
\\
\solution
		%\begin{enumerate}[label=\thesection.\arabic*,ref=\thesection.\theenumi]
	\item One card is drawn from a well-shuffled deck of 52 cards. Find the probability of getting
\begin{enumerate}
\item A king of red colour 
\item A face card 
\item A red face card
\item The jack of hearts
\item A spade
\item The queen of diamonds

\end{enumerate}
\solution
		%\input{ncert/10/15/1/14/main.tex}
	\item Five cards—the ten, jack, queen, king and ace of diamonds, are well-shuffled with their face downwards. One card is then picked up at random.
\begin{enumerate}
\item
What is the probability that the card is the queen? 
\item
If the queen is drawn and put aside, what is the probability that the second card picked up is (a) an ace? (b) a queen?\\
\end{enumerate}
\solution
		%\input{ncert/10/15/1/15/defs.tex}
	\item A bag contains $5$ red balls and some blue balls. If the probability of drawing a blue ball is double that if a red ball, determine the number of blue balls in the bag. 
		\\
\solution
		%\input{ncert/10/15/2/3/defs.tex}
	\item A card is selected from a pack of 52 cards.
 \begin{enumerate}[label=(\alph*)] 
                 \item How many points are there in the sample space?
                 \item Calculate the probability that the card is an ace of spades.
                 \item Calculate the probability that the card is (i) an ace and (ii) black card.
 \end{enumerate}
\solution
		%\input{ncert/11/16/3/4/main.tex}
\item Four cards are drawn from a well-shuffled deck of 52 cards. What is the probability of obtaining 3 diamonds and one spade.
\\
\solution
		%\input{ncert/11/16/4/2/defs.tex}
\item In a certain lottery 10,000 tickets are sold and ten equal prizes are awarded. What is the probability of not getting a prize if you buy (a) one ticket (b) two tickets (c) 10 tickets ?	
\\
\solution
		%\input{ncert/11/16/4/4/defs.tex}
		%
\item 
Out of 100 students, two sections of 40 and 60 are formed. If you and your friend are among the 100 students, what is the probability that
\begin{enumerate}
\item you both enter the same section?
\item you both enter the different sections?
\end{enumerate}
\solution
		%\input{ncert/11/16/4/5/defs.tex}
	\item 
The number lock of a suitcase has 4 wheels each labelled with ten digits i.e. from 0 to 9.The lock opens with a sequence of four digits with no repeats.What is the probability of a person getting the right sequence to open the suitcase.
\\
\solution
		%\input{ncert/11/16/4/10/defs.tex}
		%
\item 
Two cards are drawn at random and without replacement from a pack of 52 playing cards. Find the probability that both the cards are black.
\\
\solution
		%\input{ncert/12/13/2/2/defs.tex}
		\item A box of oranges is inspected by examining three randomly selected oranges drawn without replacement. If all the three oranges are good, the box is approved for sale, otherwise, it is rejected. Find the probability that a box containing 15 oranges out of which 12 are good and 3 are bad ones will be approved for sale.
		\label{ncert/12/13/2/3/defs.tex}
		\item Two balls are drawn at random with replacement from a box containing 10 black and 8 red balls. Find the probability that
		\label{ncert/12/13/2/12}
\begin{enumerate}
\item both balls are red.
\item first ball is black and second is red.
\item one of them is black and other is red.
\end{enumerate}

\item In a hostel, 60\% of the students read Hindi newspaper, 40\% read English newspaper and 20\% read both Hindi and English newspapers. A student is selected at random.
		\label{ncert/12/13/2/15}
\begin{enumerate}
\item Find the probability that she reads neither Hindi nor English newspapers.
\item If she reads Hindi newspaper, find the probability that she reads English newspaper.
\item If she reads English newspaper, find the probability that she reads Hindi newspaper.\\
\end{enumerate}
\item The probability of obtaining an even prime number on each die, when a pair of dice is rolled is 
\begin{enumerate}
    \item $0$ 
    
    \item $\frac{1}{3}$ 
    
    \item $\frac{1}{12}$ 
    
    \item $\frac{1}{36}$ 
\end{enumerate}
\solution
		%\input{ncert/12/13/2/17/defs.tex}
	\item A bag contains 4 red and 4 black balls, another bag contains 2 red and 6 black balls. One of the two bags is selected at random and a ball is drawn from the bag which is found to be red. Find the probability that the ball is drawn from the first bag.
\\
\solution
		%\input{ncert/12/13/3/2/main.tex}
  \item
  Cards with numbers 2 to 101 are placed in a box. A card is selected at random.Find the probability that the card has
\begin{enumerate}[label=(\roman*)]
	\item an even number 
	\item a square number
\end{enumerate}
\solution
%\input{exemplar/10/13/3/32/main.tex}
\item
The king, queen and jack of clubs are removed from a deck of 52 playing cards and then well shuffled. Now one card is drawn at random from the remaining cards.  Determine the probability that the card is
\begin{enumerate}[label=(\roman*)]
\item a club
\item 10 of hearts
\end{enumerate}
\solution
%\input{exemplar/10/13/3/29/main.tex}
\item A team of medical students doing their internship have to assist during surgeries
at a city hospital. The probabilities of surgeries rated as very complex, complex,
routine, simple or very simple are respectively, 0.15, 0.20, 0.31, 0.26, .08. Find
the probabilities that a particular surgery will be rated
\begin{enumerate}
	\item complex or very complex;
	\item neither very complex nor very simple;
	\item routine or complex
	\item routine or simple
\end{enumerate}
\solution
%\input{exemplar/11/16/3/8(1)/main.tex}
\item A card is selected from a pack of 52 cards.
\begin{enumerate}[label=(\alph*)]
    \item How many points are there in the sample space?
    \item Calculate the probability that the card is an ace of spades.
    \item Calculate the probability that the card is (i) an ace and (ii) black card.
\end{enumerate}
\solution
%\input{exemplar/11/16/3/4/main2.tex}
\item The probability that a non leap year selected at random will contain 53 sundays.
\\
\solution
%\input{exemplar/10/13/1/19/main.tex}
\item One of the four persons John, Rita, Aslam or Gurpreet will be promoted next
month. Consequently the sample space consists of four elementary outcomes
S = {John promoted, Rita promoted, Aslam promoted, Gurpreet promoted}
You are told that the chances of John’s promotion is same as that of Gurpreet,
Rita’s chances of promotion are twice as likely as Johns. Aslam’s chances are
four times that of John.
\begin{enumerate}
	\item Determine
	\begin{enumerate}
		\item P (John promoted)
		\item P (Rita promoted)
		\item P (Aslam promoted)
		\item P (Gurpreet promoted)
	\end{enumerate}
	\item If A = {John promoted or Gurpreet promoted}, find P (A).
\end{enumerate}
\solution
%\input{exemplar/11/16/3/10/main.tex}
\item A card is drawn from a deck of 52 cards. Find the probability of getting a king or a heart or a red card.\\
\solution
%\input{exemplar/11/16/3/15/main.tex}
\item The probability that a student will pass his examination is 0.73, the probability of
the student getting a compartment is 0.13, and the probability that the student will
either pass or get compartment is 0.96. State True or False.\\
\solution
%\input{exemplar/11/16/3/31/main.tex}
\item A card is selected from a pack of 52 cards\\
\begin{enumerate}[label=(\alph*)]
\item How many points are there in the sample space?
\item Calculate the probability that the cards is an ace of spades.
\item Calculate the probability that the card is (i) an ace (ii)black card.\\
\end{enumerate}
%\input{ncert/11/16/3/4_1/Prob_4.tex}
\item In a non-leap year, the probability of having 53 tuesdays or 53 wednesdays is\\
\solution
%\input{exemplar/11/16/3/18/main.tex}
\item There are 1000 sealed envelopes in a box, 10 of them contain a cash prize of
Rs 100 each, 100 of them contain a cash prize of Rs 50 each and 200 of them
contain a cash prize of Rs 10 each and rest do not contain any cash prize. If they
are well shuffled and an envelope is picked up out, what is the probability that it
contains no cash prize?\\
\solution
%\input{exemplar/10/13/3/34/main.tex}
\item 
A die is thrown and a card is selected at random from a deck of 52 playing cards. The probability of getting an even number on the die and a spade card.\\
\solution
%\input{exemplar/12/13/3/78/main.tex}
\item
If 4-digit numbers greater than 5,000 are randomly formed from the digits 0, 1, 3, 5, and 7, what is the probability of forming a number divisible by 5 when:
\begin{enumerate}
    \item The digits are repeated?
    \item The repetition of digits is not allowed?
\end{enumerate}
\solution
%\input{ncert/11/16/4/9/main.tex}
\item Consider the probability space $\brak{\Omega, \mathcal{G}, P}$ where $\Omega = [0,2]$ and $\mathcal{G} = \cbrak{\phi, \Omega, [0,1], (1,2]}$. Let $X$ and $Y$ be two functions on $\Omega$ defined as
\begin{align*}
    X(\omega) = 
    \begin{cases}
        1 & \text{if }\omega \in [0, 1]\\
        2 & \text{if }\omega \in (1, 2]
    \end{cases}
\end{align*}
and
\begin{align*}
    Y(\omega) = 
    \begin{cases}
        2 & \text{if }\omega \in [0, 1.5]\\
        3 & \text{if }\omega \in (1.5, 2].
    \end{cases}
\end{align*}
Then which one of the following statements is true?
\begin{enumerate}
    \item [(A)] $X$ is a random variable with respect to $\mathcal{G}$, but $Y$ is not a random variable with respect to $\mathcal{G}$.
    \item [(B)] $Y$ is a random variable with respect to $\mathcal{G}$, but $X$ is not a random variable with respect to $\mathcal{G}$.
    \item [(C)] Neither $X$ nor $Y$ is a random variable with respect to $\mathcal{G}$.
    \item [(D)] Both $X$ and $Y$ are random variables with respect to $\mathcal{G}$.
\end{enumerate} \hfill (GATE ST 2023)\\
\solution
%\input{gate/ST/2023/14/main.tex}
	\item  A die is loaded in such a way that each odd number is twice as likely to occur as
each even number. Find $P(G)$, where $G$ is the event that a number greater than
3 occurs on a single roll of the die.
\\
\solution
		%\input{exemplar/11/16/3/5/main.tex}
	\item All the jacks, queens and kings are removed from a deck of 52 playing cards. The remaining cards are well shuffled and then one card is drawn at random. Giving ace a value 1 similar value for other cards, find the probability that the card has a value 
		\begin{enumerate}
			\item 7
			\item greater than 7
			\item less than 7
		\end{enumerate}
		%\input{exemplar/10/13/3/30/main.tex}
  \item A Lot consists of 48 mobile phones of which 42 are good, 3 have only minor defects and 3 have major defects.Varnika will buy a phone if it is good but the trader will only buy a mobile if it has no major defects. One phone is selected at random from the lot. What is the probability that it is
\begin{enumerate}
	\item acceptable to Varnika?
            \item acceptable to the trader?
\end{enumerate}
\solution
	%\input{exemplar/10/13/3/40/main.tex}
 \item A student says that if you throw a die, it will show up 1 or not 1. Therefore, the probability of getting 1 and the probability of getting 'not 1' each is equal to $\frac{1}{2}$. Is this correct? Give reasons.\\
 \solution
        %\input{exemplar/10/13/2/9/main.tex}
   \item Four candidates A, B, C, D have ap-
plied for the assignment to coach a school cricket
team. If A is twice as likely to be selected as B, and
B and C are given about the same chance of being
selected, while C is twice as likely to be selected
as D, what are the probabilities that
\begin{enumerate}
\item C will be selected?
\item A will not be selected?
\end{enumerate}
	%\input{exemplar/11/16/3/9/main.tex}
 \item A bag contain 24 balls of which $x$ balls are red, $2x$ are white and $3x$ are blue. A ball is selected at random, What is the probability that it is
\begin{enumerate}[label=\alph*)]
\item not red ?
\item white ?
\end{enumerate}
%\input{exemplar/10/13/3/41/main.tex}
If the letters of the word ASSASSINATION are arranged at random. Find the Probability that
\begin{enumerate}[label=(\alph*)]
\item Four $S's$ come consecutively in the word
\item Two  $I's$ and two $N's$ come together
\item All $A's$ are not coming together
\item No two $A's$ are coming together
\end{enumerate}
%\input{exemplar/11/16/3/14/main.tex}
	\item One urn contains two black balls (labelled B1 and B2) and one white ball. A
	second urn contains one black ball and two white balls (labelled W1 and W2).
	Suppose the following experiment is performed. One of the two urns is chosen
	at random. Next a ball is randomly chosen from the urn. Then a second ball is
	chosen at random from the same urn without replacing the first ball.
	
	\begin{enumerate}
	\item What is the probability that two black balls are chosen?
	
	\item What is the probability that two balls of opposite colour are chosen?
	\end{enumerate}
	\solution
	%\input{exemplar/11/16/3/12/main1.tex}
\end{enumerate}

		\item A box of oranges is inspected by examining three randomly selected oranges drawn without replacement. If all the three oranges are good, the box is approved for sale, otherwise, it is rejected. Find the probability that a box containing 15 oranges out of which 12 are good and 3 are bad ones will be approved for sale.
		\label{ncert/12/13/2/3/defs.tex}
		\item Two balls are drawn at random with replacement from a box containing 10 black and 8 red balls. Find the probability that
		\label{ncert/12/13/2/12}
\begin{enumerate}
\item both balls are red.
\item first ball is black and second is red.
\item one of them is black and other is red.
\end{enumerate}

\item In a hostel, 60\% of the students read Hindi newspaper, 40\% read English newspaper and 20\% read both Hindi and English newspapers. A student is selected at random.
		\label{ncert/12/13/2/15}
\begin{enumerate}
\item Find the probability that she reads neither Hindi nor English newspapers.
\item If she reads Hindi newspaper, find the probability that she reads English newspaper.
\item If she reads English newspaper, find the probability that she reads Hindi newspaper.\\
\end{enumerate}
\item The probability of obtaining an even prime number on each die, when a pair of dice is rolled is 
\begin{enumerate}
    \item $0$ 
    
    \item $\frac{1}{3}$ 
    
    \item $\frac{1}{12}$ 
    
    \item $\frac{1}{36}$ 
\end{enumerate}
\solution
		%\begin{enumerate}[label=\thesection.\arabic*,ref=\thesection.\theenumi]
	\item One card is drawn from a well-shuffled deck of 52 cards. Find the probability of getting
\begin{enumerate}
\item A king of red colour 
\item A face card 
\item A red face card
\item The jack of hearts
\item A spade
\item The queen of diamonds

\end{enumerate}
\solution
		%\input{ncert/10/15/1/14/main.tex}
	\item Five cards—the ten, jack, queen, king and ace of diamonds, are well-shuffled with their face downwards. One card is then picked up at random.
\begin{enumerate}
\item
What is the probability that the card is the queen? 
\item
If the queen is drawn and put aside, what is the probability that the second card picked up is (a) an ace? (b) a queen?\\
\end{enumerate}
\solution
		%\input{ncert/10/15/1/15/defs.tex}
	\item A bag contains $5$ red balls and some blue balls. If the probability of drawing a blue ball is double that if a red ball, determine the number of blue balls in the bag. 
		\\
\solution
		%\input{ncert/10/15/2/3/defs.tex}
	\item A card is selected from a pack of 52 cards.
 \begin{enumerate}[label=(\alph*)] 
                 \item How many points are there in the sample space?
                 \item Calculate the probability that the card is an ace of spades.
                 \item Calculate the probability that the card is (i) an ace and (ii) black card.
 \end{enumerate}
\solution
		%\input{ncert/11/16/3/4/main.tex}
\item Four cards are drawn from a well-shuffled deck of 52 cards. What is the probability of obtaining 3 diamonds and one spade.
\\
\solution
		%\input{ncert/11/16/4/2/defs.tex}
\item In a certain lottery 10,000 tickets are sold and ten equal prizes are awarded. What is the probability of not getting a prize if you buy (a) one ticket (b) two tickets (c) 10 tickets ?	
\\
\solution
		%\input{ncert/11/16/4/4/defs.tex}
		%
\item 
Out of 100 students, two sections of 40 and 60 are formed. If you and your friend are among the 100 students, what is the probability that
\begin{enumerate}
\item you both enter the same section?
\item you both enter the different sections?
\end{enumerate}
\solution
		%\input{ncert/11/16/4/5/defs.tex}
	\item 
The number lock of a suitcase has 4 wheels each labelled with ten digits i.e. from 0 to 9.The lock opens with a sequence of four digits with no repeats.What is the probability of a person getting the right sequence to open the suitcase.
\\
\solution
		%\input{ncert/11/16/4/10/defs.tex}
		%
\item 
Two cards are drawn at random and without replacement from a pack of 52 playing cards. Find the probability that both the cards are black.
\\
\solution
		%\input{ncert/12/13/2/2/defs.tex}
		\item A box of oranges is inspected by examining three randomly selected oranges drawn without replacement. If all the three oranges are good, the box is approved for sale, otherwise, it is rejected. Find the probability that a box containing 15 oranges out of which 12 are good and 3 are bad ones will be approved for sale.
		\label{ncert/12/13/2/3/defs.tex}
		\item Two balls are drawn at random with replacement from a box containing 10 black and 8 red balls. Find the probability that
		\label{ncert/12/13/2/12}
\begin{enumerate}
\item both balls are red.
\item first ball is black and second is red.
\item one of them is black and other is red.
\end{enumerate}

\item In a hostel, 60\% of the students read Hindi newspaper, 40\% read English newspaper and 20\% read both Hindi and English newspapers. A student is selected at random.
		\label{ncert/12/13/2/15}
\begin{enumerate}
\item Find the probability that she reads neither Hindi nor English newspapers.
\item If she reads Hindi newspaper, find the probability that she reads English newspaper.
\item If she reads English newspaper, find the probability that she reads Hindi newspaper.\\
\end{enumerate}
\item The probability of obtaining an even prime number on each die, when a pair of dice is rolled is 
\begin{enumerate}
    \item $0$ 
    
    \item $\frac{1}{3}$ 
    
    \item $\frac{1}{12}$ 
    
    \item $\frac{1}{36}$ 
\end{enumerate}
\solution
		%\input{ncert/12/13/2/17/defs.tex}
	\item A bag contains 4 red and 4 black balls, another bag contains 2 red and 6 black balls. One of the two bags is selected at random and a ball is drawn from the bag which is found to be red. Find the probability that the ball is drawn from the first bag.
\\
\solution
		%\input{ncert/12/13/3/2/main.tex}
  \item
  Cards with numbers 2 to 101 are placed in a box. A card is selected at random.Find the probability that the card has
\begin{enumerate}[label=(\roman*)]
	\item an even number 
	\item a square number
\end{enumerate}
\solution
%\input{exemplar/10/13/3/32/main.tex}
\item
The king, queen and jack of clubs are removed from a deck of 52 playing cards and then well shuffled. Now one card is drawn at random from the remaining cards.  Determine the probability that the card is
\begin{enumerate}[label=(\roman*)]
\item a club
\item 10 of hearts
\end{enumerate}
\solution
%\input{exemplar/10/13/3/29/main.tex}
\item A team of medical students doing their internship have to assist during surgeries
at a city hospital. The probabilities of surgeries rated as very complex, complex,
routine, simple or very simple are respectively, 0.15, 0.20, 0.31, 0.26, .08. Find
the probabilities that a particular surgery will be rated
\begin{enumerate}
	\item complex or very complex;
	\item neither very complex nor very simple;
	\item routine or complex
	\item routine or simple
\end{enumerate}
\solution
%\input{exemplar/11/16/3/8(1)/main.tex}
\item A card is selected from a pack of 52 cards.
\begin{enumerate}[label=(\alph*)]
    \item How many points are there in the sample space?
    \item Calculate the probability that the card is an ace of spades.
    \item Calculate the probability that the card is (i) an ace and (ii) black card.
\end{enumerate}
\solution
%\input{exemplar/11/16/3/4/main2.tex}
\item The probability that a non leap year selected at random will contain 53 sundays.
\\
\solution
%\input{exemplar/10/13/1/19/main.tex}
\item One of the four persons John, Rita, Aslam or Gurpreet will be promoted next
month. Consequently the sample space consists of four elementary outcomes
S = {John promoted, Rita promoted, Aslam promoted, Gurpreet promoted}
You are told that the chances of John’s promotion is same as that of Gurpreet,
Rita’s chances of promotion are twice as likely as Johns. Aslam’s chances are
four times that of John.
\begin{enumerate}
	\item Determine
	\begin{enumerate}
		\item P (John promoted)
		\item P (Rita promoted)
		\item P (Aslam promoted)
		\item P (Gurpreet promoted)
	\end{enumerate}
	\item If A = {John promoted or Gurpreet promoted}, find P (A).
\end{enumerate}
\solution
%\input{exemplar/11/16/3/10/main.tex}
\item A card is drawn from a deck of 52 cards. Find the probability of getting a king or a heart or a red card.\\
\solution
%\input{exemplar/11/16/3/15/main.tex}
\item The probability that a student will pass his examination is 0.73, the probability of
the student getting a compartment is 0.13, and the probability that the student will
either pass or get compartment is 0.96. State True or False.\\
\solution
%\input{exemplar/11/16/3/31/main.tex}
\item A card is selected from a pack of 52 cards\\
\begin{enumerate}[label=(\alph*)]
\item How many points are there in the sample space?
\item Calculate the probability that the cards is an ace of spades.
\item Calculate the probability that the card is (i) an ace (ii)black card.\\
\end{enumerate}
%\input{ncert/11/16/3/4_1/Prob_4.tex}
\item In a non-leap year, the probability of having 53 tuesdays or 53 wednesdays is\\
\solution
%\input{exemplar/11/16/3/18/main.tex}
\item There are 1000 sealed envelopes in a box, 10 of them contain a cash prize of
Rs 100 each, 100 of them contain a cash prize of Rs 50 each and 200 of them
contain a cash prize of Rs 10 each and rest do not contain any cash prize. If they
are well shuffled and an envelope is picked up out, what is the probability that it
contains no cash prize?\\
\solution
%\input{exemplar/10/13/3/34/main.tex}
\item 
A die is thrown and a card is selected at random from a deck of 52 playing cards. The probability of getting an even number on the die and a spade card.\\
\solution
%\input{exemplar/12/13/3/78/main.tex}
\item
If 4-digit numbers greater than 5,000 are randomly formed from the digits 0, 1, 3, 5, and 7, what is the probability of forming a number divisible by 5 when:
\begin{enumerate}
    \item The digits are repeated?
    \item The repetition of digits is not allowed?
\end{enumerate}
\solution
%\input{ncert/11/16/4/9/main.tex}
\item Consider the probability space $\brak{\Omega, \mathcal{G}, P}$ where $\Omega = [0,2]$ and $\mathcal{G} = \cbrak{\phi, \Omega, [0,1], (1,2]}$. Let $X$ and $Y$ be two functions on $\Omega$ defined as
\begin{align*}
    X(\omega) = 
    \begin{cases}
        1 & \text{if }\omega \in [0, 1]\\
        2 & \text{if }\omega \in (1, 2]
    \end{cases}
\end{align*}
and
\begin{align*}
    Y(\omega) = 
    \begin{cases}
        2 & \text{if }\omega \in [0, 1.5]\\
        3 & \text{if }\omega \in (1.5, 2].
    \end{cases}
\end{align*}
Then which one of the following statements is true?
\begin{enumerate}
    \item [(A)] $X$ is a random variable with respect to $\mathcal{G}$, but $Y$ is not a random variable with respect to $\mathcal{G}$.
    \item [(B)] $Y$ is a random variable with respect to $\mathcal{G}$, but $X$ is not a random variable with respect to $\mathcal{G}$.
    \item [(C)] Neither $X$ nor $Y$ is a random variable with respect to $\mathcal{G}$.
    \item [(D)] Both $X$ and $Y$ are random variables with respect to $\mathcal{G}$.
\end{enumerate} \hfill (GATE ST 2023)\\
\solution
%\input{gate/ST/2023/14/main.tex}
	\item  A die is loaded in such a way that each odd number is twice as likely to occur as
each even number. Find $P(G)$, where $G$ is the event that a number greater than
3 occurs on a single roll of the die.
\\
\solution
		%\input{exemplar/11/16/3/5/main.tex}
	\item All the jacks, queens and kings are removed from a deck of 52 playing cards. The remaining cards are well shuffled and then one card is drawn at random. Giving ace a value 1 similar value for other cards, find the probability that the card has a value 
		\begin{enumerate}
			\item 7
			\item greater than 7
			\item less than 7
		\end{enumerate}
		%\input{exemplar/10/13/3/30/main.tex}
  \item A Lot consists of 48 mobile phones of which 42 are good, 3 have only minor defects and 3 have major defects.Varnika will buy a phone if it is good but the trader will only buy a mobile if it has no major defects. One phone is selected at random from the lot. What is the probability that it is
\begin{enumerate}
	\item acceptable to Varnika?
            \item acceptable to the trader?
\end{enumerate}
\solution
	%\input{exemplar/10/13/3/40/main.tex}
 \item A student says that if you throw a die, it will show up 1 or not 1. Therefore, the probability of getting 1 and the probability of getting 'not 1' each is equal to $\frac{1}{2}$. Is this correct? Give reasons.\\
 \solution
        %\input{exemplar/10/13/2/9/main.tex}
   \item Four candidates A, B, C, D have ap-
plied for the assignment to coach a school cricket
team. If A is twice as likely to be selected as B, and
B and C are given about the same chance of being
selected, while C is twice as likely to be selected
as D, what are the probabilities that
\begin{enumerate}
\item C will be selected?
\item A will not be selected?
\end{enumerate}
	%\input{exemplar/11/16/3/9/main.tex}
 \item A bag contain 24 balls of which $x$ balls are red, $2x$ are white and $3x$ are blue. A ball is selected at random, What is the probability that it is
\begin{enumerate}[label=\alph*)]
\item not red ?
\item white ?
\end{enumerate}
%\input{exemplar/10/13/3/41/main.tex}
If the letters of the word ASSASSINATION are arranged at random. Find the Probability that
\begin{enumerate}[label=(\alph*)]
\item Four $S's$ come consecutively in the word
\item Two  $I's$ and two $N's$ come together
\item All $A's$ are not coming together
\item No two $A's$ are coming together
\end{enumerate}
%\input{exemplar/11/16/3/14/main.tex}
	\item One urn contains two black balls (labelled B1 and B2) and one white ball. A
	second urn contains one black ball and two white balls (labelled W1 and W2).
	Suppose the following experiment is performed. One of the two urns is chosen
	at random. Next a ball is randomly chosen from the urn. Then a second ball is
	chosen at random from the same urn without replacing the first ball.
	
	\begin{enumerate}
	\item What is the probability that two black balls are chosen?
	
	\item What is the probability that two balls of opposite colour are chosen?
	\end{enumerate}
	\solution
	%\input{exemplar/11/16/3/12/main1.tex}
\end{enumerate}

	\item A bag contains 4 red and 4 black balls, another bag contains 2 red and 6 black balls. One of the two bags is selected at random and a ball is drawn from the bag which is found to be red. Find the probability that the ball is drawn from the first bag.
\\
\solution
		%\begin{table}[H]
	\centering
\begin{tabular}{|c|c|c|}
\hline
Random variable &Value &Definition\\ \hline
\multirow{3}{*}{X} &0 &Slips of Rs 1\\
&1 &Slips of Rs 5\\
&2 &Slips of Rs 13\\ \hline
\multirow{2}{*}{Y} &0 &Box A\\
&1 &Box B\\\hline
\end{tabular}
\caption{}
\label{tab:Distribution}
\end{table}
See \tabref{tab:Distribution}.
\begin{align}
p_{Y}\brak{k}= \begin{cases} 
      \frac{1}{3} & {k=0} \\
      \frac{2}{3 }& {k=1} 
   \end{cases}
   \\
p_{Y|X}\brak{0|0} = \frac{19}{25}\, 
p_{Y|X}\brak{0|1} = \frac{6}{25}\,
p_{Y|X}\brak{1|0} = \frac{45}{50}\,
p_{Y|X}\brak{1|2} = \frac{5}{50}
\end{align}
The desired probability is the probability that a slip drawn at random is marked other than Rs 1,
\begin{align}
&=1-p_X\brak{0}\\
&= p_X(1) + p_X(2)
\end{align}
Using Bayes theorem,
\begin{align}
&= p_Y\brak{0} \times \pr{Y=0 | X=1} + p_Y\brak{1} \times \pr{Y=1|X=2}\\
&=\frac{1}{3} \times \frac{6}{25} + \frac{2}{3} \times \frac{5}{50}\\
&=\frac{11}{75}
\end{align}

\newpage

%\tableofcontents

\bigskip

\renewcommand{\thefigure}{\theenumi}
\renewcommand{\thetable}{\theenumi}
%\renewcommand{\theequation}{\theenumi}

%\begin{abstract}
%%\boldmath
%In this letter, an algorithm for evaluating the exact analytical bit error rate  (BER)  for the piecewise linear (PL) combiner for  multiple relays is presented. Previous results were available only for upto three relays. The algorithm is unique in the sense that  the actual mathematical expressions, that are prohibitively large, need not be explicitly obtained. The diversity gain due to multiple relays is shown through plots of the analytical BER, well supported by simulations. 
%
%\end{abstract}
% IEEEtran.cls defaults to using nonbold math in the Abstract.
% This preserves the distinction between vectors and scalars. However,
% if the journal you are submitting to favors bold math in the abstract,
% then you can use LaTeX's standard command \boldmath at the very start
% of the abstract to achieve this. Many IEEE journals frown on math
% in the abstract anyway.

% Note that keywords are not normally used for peerreview papers.
%\begin{IEEEkeywords}
%Cooperative diversity, decode and forward, piecewise linear
%\end{IEEEkeywords}



% For peer review papers, you can put extra information on the cover
% page as needed:
% \ifCLASSOPTIONpeerreview
% \begin{center} \bfseries EDICS Category: 3-BBND \end{center}
% \fi
%
% For peerreview papers, this IEEEtran command inserts a page break and
% creates the second title. It will be ignored for other modes.
%\IEEEpeerreviewmaketitle




  \item
  Cards with numbers 2 to 101 are placed in a box. A card is selected at random.Find the probability that the card has
\begin{enumerate}[label=(\roman*)]
	\item an even number 
	\item a square number
\end{enumerate}
\solution
%\begin{table}[H]
	\centering
\begin{tabular}{|c|c|c|}
\hline
Random variable &Value &Definition\\ \hline
\multirow{3}{*}{X} &0 &Slips of Rs 1\\
&1 &Slips of Rs 5\\
&2 &Slips of Rs 13\\ \hline
\multirow{2}{*}{Y} &0 &Box A\\
&1 &Box B\\\hline
\end{tabular}
\caption{}
\label{tab:Distribution}
\end{table}
See \tabref{tab:Distribution}.
\begin{align}
p_{Y}\brak{k}= \begin{cases} 
      \frac{1}{3} & {k=0} \\
      \frac{2}{3 }& {k=1} 
   \end{cases}
   \\
p_{Y|X}\brak{0|0} = \frac{19}{25}\, 
p_{Y|X}\brak{0|1} = \frac{6}{25}\,
p_{Y|X}\brak{1|0} = \frac{45}{50}\,
p_{Y|X}\brak{1|2} = \frac{5}{50}
\end{align}
The desired probability is the probability that a slip drawn at random is marked other than Rs 1,
\begin{align}
&=1-p_X\brak{0}\\
&= p_X(1) + p_X(2)
\end{align}
Using Bayes theorem,
\begin{align}
&= p_Y\brak{0} \times \pr{Y=0 | X=1} + p_Y\brak{1} \times \pr{Y=1|X=2}\\
&=\frac{1}{3} \times \frac{6}{25} + \frac{2}{3} \times \frac{5}{50}\\
&=\frac{11}{75}
\end{align}

\newpage

%\tableofcontents

\bigskip

\renewcommand{\thefigure}{\theenumi}
\renewcommand{\thetable}{\theenumi}
%\renewcommand{\theequation}{\theenumi}

%\begin{abstract}
%%\boldmath
%In this letter, an algorithm for evaluating the exact analytical bit error rate  (BER)  for the piecewise linear (PL) combiner for  multiple relays is presented. Previous results were available only for upto three relays. The algorithm is unique in the sense that  the actual mathematical expressions, that are prohibitively large, need not be explicitly obtained. The diversity gain due to multiple relays is shown through plots of the analytical BER, well supported by simulations. 
%
%\end{abstract}
% IEEEtran.cls defaults to using nonbold math in the Abstract.
% This preserves the distinction between vectors and scalars. However,
% if the journal you are submitting to favors bold math in the abstract,
% then you can use LaTeX's standard command \boldmath at the very start
% of the abstract to achieve this. Many IEEE journals frown on math
% in the abstract anyway.

% Note that keywords are not normally used for peerreview papers.
%\begin{IEEEkeywords}
%Cooperative diversity, decode and forward, piecewise linear
%\end{IEEEkeywords}



% For peer review papers, you can put extra information on the cover
% page as needed:
% \ifCLASSOPTIONpeerreview
% \begin{center} \bfseries EDICS Category: 3-BBND \end{center}
% \fi
%
% For peerreview papers, this IEEEtran command inserts a page break and
% creates the second title. It will be ignored for other modes.
%\IEEEpeerreviewmaketitle




\item
The king, queen and jack of clubs are removed from a deck of 52 playing cards and then well shuffled. Now one card is drawn at random from the remaining cards.  Determine the probability that the card is
\begin{enumerate}[label=(\roman*)]
\item a club
\item 10 of hearts
\end{enumerate}
\solution
%\begin{table}[H]
	\centering
\begin{tabular}{|c|c|c|}
\hline
Random variable &Value &Definition\\ \hline
\multirow{3}{*}{X} &0 &Slips of Rs 1\\
&1 &Slips of Rs 5\\
&2 &Slips of Rs 13\\ \hline
\multirow{2}{*}{Y} &0 &Box A\\
&1 &Box B\\\hline
\end{tabular}
\caption{}
\label{tab:Distribution}
\end{table}
See \tabref{tab:Distribution}.
\begin{align}
p_{Y}\brak{k}= \begin{cases} 
      \frac{1}{3} & {k=0} \\
      \frac{2}{3 }& {k=1} 
   \end{cases}
   \\
p_{Y|X}\brak{0|0} = \frac{19}{25}\, 
p_{Y|X}\brak{0|1} = \frac{6}{25}\,
p_{Y|X}\brak{1|0} = \frac{45}{50}\,
p_{Y|X}\brak{1|2} = \frac{5}{50}
\end{align}
The desired probability is the probability that a slip drawn at random is marked other than Rs 1,
\begin{align}
&=1-p_X\brak{0}\\
&= p_X(1) + p_X(2)
\end{align}
Using Bayes theorem,
\begin{align}
&= p_Y\brak{0} \times \pr{Y=0 | X=1} + p_Y\brak{1} \times \pr{Y=1|X=2}\\
&=\frac{1}{3} \times \frac{6}{25} + \frac{2}{3} \times \frac{5}{50}\\
&=\frac{11}{75}
\end{align}

\newpage

%\tableofcontents

\bigskip

\renewcommand{\thefigure}{\theenumi}
\renewcommand{\thetable}{\theenumi}
%\renewcommand{\theequation}{\theenumi}

%\begin{abstract}
%%\boldmath
%In this letter, an algorithm for evaluating the exact analytical bit error rate  (BER)  for the piecewise linear (PL) combiner for  multiple relays is presented. Previous results were available only for upto three relays. The algorithm is unique in the sense that  the actual mathematical expressions, that are prohibitively large, need not be explicitly obtained. The diversity gain due to multiple relays is shown through plots of the analytical BER, well supported by simulations. 
%
%\end{abstract}
% IEEEtran.cls defaults to using nonbold math in the Abstract.
% This preserves the distinction between vectors and scalars. However,
% if the journal you are submitting to favors bold math in the abstract,
% then you can use LaTeX's standard command \boldmath at the very start
% of the abstract to achieve this. Many IEEE journals frown on math
% in the abstract anyway.

% Note that keywords are not normally used for peerreview papers.
%\begin{IEEEkeywords}
%Cooperative diversity, decode and forward, piecewise linear
%\end{IEEEkeywords}



% For peer review papers, you can put extra information on the cover
% page as needed:
% \ifCLASSOPTIONpeerreview
% \begin{center} \bfseries EDICS Category: 3-BBND \end{center}
% \fi
%
% For peerreview papers, this IEEEtran command inserts a page break and
% creates the second title. It will be ignored for other modes.
%\IEEEpeerreviewmaketitle




\item A team of medical students doing their internship have to assist during surgeries
at a city hospital. The probabilities of surgeries rated as very complex, complex,
routine, simple or very simple are respectively, 0.15, 0.20, 0.31, 0.26, .08. Find
the probabilities that a particular surgery will be rated
\begin{enumerate}
	\item complex or very complex;
	\item neither very complex nor very simple;
	\item routine or complex
	\item routine or simple
\end{enumerate}
\solution
%\begin{table}[H]
	\centering
\begin{tabular}{|c|c|c|}
\hline
Random variable &Value &Definition\\ \hline
\multirow{3}{*}{X} &0 &Slips of Rs 1\\
&1 &Slips of Rs 5\\
&2 &Slips of Rs 13\\ \hline
\multirow{2}{*}{Y} &0 &Box A\\
&1 &Box B\\\hline
\end{tabular}
\caption{}
\label{tab:Distribution}
\end{table}
See \tabref{tab:Distribution}.
\begin{align}
p_{Y}\brak{k}= \begin{cases} 
      \frac{1}{3} & {k=0} \\
      \frac{2}{3 }& {k=1} 
   \end{cases}
   \\
p_{Y|X}\brak{0|0} = \frac{19}{25}\, 
p_{Y|X}\brak{0|1} = \frac{6}{25}\,
p_{Y|X}\brak{1|0} = \frac{45}{50}\,
p_{Y|X}\brak{1|2} = \frac{5}{50}
\end{align}
The desired probability is the probability that a slip drawn at random is marked other than Rs 1,
\begin{align}
&=1-p_X\brak{0}\\
&= p_X(1) + p_X(2)
\end{align}
Using Bayes theorem,
\begin{align}
&= p_Y\brak{0} \times \pr{Y=0 | X=1} + p_Y\brak{1} \times \pr{Y=1|X=2}\\
&=\frac{1}{3} \times \frac{6}{25} + \frac{2}{3} \times \frac{5}{50}\\
&=\frac{11}{75}
\end{align}

\newpage

%\tableofcontents

\bigskip

\renewcommand{\thefigure}{\theenumi}
\renewcommand{\thetable}{\theenumi}
%\renewcommand{\theequation}{\theenumi}

%\begin{abstract}
%%\boldmath
%In this letter, an algorithm for evaluating the exact analytical bit error rate  (BER)  for the piecewise linear (PL) combiner for  multiple relays is presented. Previous results were available only for upto three relays. The algorithm is unique in the sense that  the actual mathematical expressions, that are prohibitively large, need not be explicitly obtained. The diversity gain due to multiple relays is shown through plots of the analytical BER, well supported by simulations. 
%
%\end{abstract}
% IEEEtran.cls defaults to using nonbold math in the Abstract.
% This preserves the distinction between vectors and scalars. However,
% if the journal you are submitting to favors bold math in the abstract,
% then you can use LaTeX's standard command \boldmath at the very start
% of the abstract to achieve this. Many IEEE journals frown on math
% in the abstract anyway.

% Note that keywords are not normally used for peerreview papers.
%\begin{IEEEkeywords}
%Cooperative diversity, decode and forward, piecewise linear
%\end{IEEEkeywords}



% For peer review papers, you can put extra information on the cover
% page as needed:
% \ifCLASSOPTIONpeerreview
% \begin{center} \bfseries EDICS Category: 3-BBND \end{center}
% \fi
%
% For peerreview papers, this IEEEtran command inserts a page break and
% creates the second title. It will be ignored for other modes.
%\IEEEpeerreviewmaketitle




\item A card is selected from a pack of 52 cards.
\begin{enumerate}[label=(\alph*)]
    \item How many points are there in the sample space?
    \item Calculate the probability that the card is an ace of spades.
    \item Calculate the probability that the card is (i) an ace and (ii) black card.
\end{enumerate}
\solution
%Let $X$ be an bernoulli rv defined as in \tabref{tab:exemplar/11/16/3/26}.  Then, 
\begin{equation}
    p =
        \frac{4}{11} 
\end{equation}
\begin{table}[H]
	\centering
	\input{exemplar/11/16/3/26/tables/Table2.tex}
	\caption{}
        \label{tab:exemplar/11/16/3/26}
\end{table}

\item The probability that a non leap year selected at random will contain 53 sundays.
\\
\solution
%\begin{table}[H]
	\centering
\begin{tabular}{|c|c|c|}
\hline
Random variable &Value &Definition\\ \hline
\multirow{3}{*}{X} &0 &Slips of Rs 1\\
&1 &Slips of Rs 5\\
&2 &Slips of Rs 13\\ \hline
\multirow{2}{*}{Y} &0 &Box A\\
&1 &Box B\\\hline
\end{tabular}
\caption{}
\label{tab:Distribution}
\end{table}
See \tabref{tab:Distribution}.
\begin{align}
p_{Y}\brak{k}= \begin{cases} 
      \frac{1}{3} & {k=0} \\
      \frac{2}{3 }& {k=1} 
   \end{cases}
   \\
p_{Y|X}\brak{0|0} = \frac{19}{25}\, 
p_{Y|X}\brak{0|1} = \frac{6}{25}\,
p_{Y|X}\brak{1|0} = \frac{45}{50}\,
p_{Y|X}\brak{1|2} = \frac{5}{50}
\end{align}
The desired probability is the probability that a slip drawn at random is marked other than Rs 1,
\begin{align}
&=1-p_X\brak{0}\\
&= p_X(1) + p_X(2)
\end{align}
Using Bayes theorem,
\begin{align}
&= p_Y\brak{0} \times \pr{Y=0 | X=1} + p_Y\brak{1} \times \pr{Y=1|X=2}\\
&=\frac{1}{3} \times \frac{6}{25} + \frac{2}{3} \times \frac{5}{50}\\
&=\frac{11}{75}
\end{align}

\newpage

%\tableofcontents

\bigskip

\renewcommand{\thefigure}{\theenumi}
\renewcommand{\thetable}{\theenumi}
%\renewcommand{\theequation}{\theenumi}

%\begin{abstract}
%%\boldmath
%In this letter, an algorithm for evaluating the exact analytical bit error rate  (BER)  for the piecewise linear (PL) combiner for  multiple relays is presented. Previous results were available only for upto three relays. The algorithm is unique in the sense that  the actual mathematical expressions, that are prohibitively large, need not be explicitly obtained. The diversity gain due to multiple relays is shown through plots of the analytical BER, well supported by simulations. 
%
%\end{abstract}
% IEEEtran.cls defaults to using nonbold math in the Abstract.
% This preserves the distinction between vectors and scalars. However,
% if the journal you are submitting to favors bold math in the abstract,
% then you can use LaTeX's standard command \boldmath at the very start
% of the abstract to achieve this. Many IEEE journals frown on math
% in the abstract anyway.

% Note that keywords are not normally used for peerreview papers.
%\begin{IEEEkeywords}
%Cooperative diversity, decode and forward, piecewise linear
%\end{IEEEkeywords}



% For peer review papers, you can put extra information on the cover
% page as needed:
% \ifCLASSOPTIONpeerreview
% \begin{center} \bfseries EDICS Category: 3-BBND \end{center}
% \fi
%
% For peerreview papers, this IEEEtran command inserts a page break and
% creates the second title. It will be ignored for other modes.
%\IEEEpeerreviewmaketitle




\item One of the four persons John, Rita, Aslam or Gurpreet will be promoted next
month. Consequently the sample space consists of four elementary outcomes
S = {John promoted, Rita promoted, Aslam promoted, Gurpreet promoted}
You are told that the chances of John’s promotion is same as that of Gurpreet,
Rita’s chances of promotion are twice as likely as Johns. Aslam’s chances are
four times that of John.
\begin{enumerate}
	\item Determine
	\begin{enumerate}
		\item P (John promoted)
		\item P (Rita promoted)
		\item P (Aslam promoted)
		\item P (Gurpreet promoted)
	\end{enumerate}
	\item If A = {John promoted or Gurpreet promoted}, find P (A).
\end{enumerate}
\solution
%\begin{table}[H]
	\centering
\begin{tabular}{|c|c|c|}
\hline
Random variable &Value &Definition\\ \hline
\multirow{3}{*}{X} &0 &Slips of Rs 1\\
&1 &Slips of Rs 5\\
&2 &Slips of Rs 13\\ \hline
\multirow{2}{*}{Y} &0 &Box A\\
&1 &Box B\\\hline
\end{tabular}
\caption{}
\label{tab:Distribution}
\end{table}
See \tabref{tab:Distribution}.
\begin{align}
p_{Y}\brak{k}= \begin{cases} 
      \frac{1}{3} & {k=0} \\
      \frac{2}{3 }& {k=1} 
   \end{cases}
   \\
p_{Y|X}\brak{0|0} = \frac{19}{25}\, 
p_{Y|X}\brak{0|1} = \frac{6}{25}\,
p_{Y|X}\brak{1|0} = \frac{45}{50}\,
p_{Y|X}\brak{1|2} = \frac{5}{50}
\end{align}
The desired probability is the probability that a slip drawn at random is marked other than Rs 1,
\begin{align}
&=1-p_X\brak{0}\\
&= p_X(1) + p_X(2)
\end{align}
Using Bayes theorem,
\begin{align}
&= p_Y\brak{0} \times \pr{Y=0 | X=1} + p_Y\brak{1} \times \pr{Y=1|X=2}\\
&=\frac{1}{3} \times \frac{6}{25} + \frac{2}{3} \times \frac{5}{50}\\
&=\frac{11}{75}
\end{align}

\newpage

%\tableofcontents

\bigskip

\renewcommand{\thefigure}{\theenumi}
\renewcommand{\thetable}{\theenumi}
%\renewcommand{\theequation}{\theenumi}

%\begin{abstract}
%%\boldmath
%In this letter, an algorithm for evaluating the exact analytical bit error rate  (BER)  for the piecewise linear (PL) combiner for  multiple relays is presented. Previous results were available only for upto three relays. The algorithm is unique in the sense that  the actual mathematical expressions, that are prohibitively large, need not be explicitly obtained. The diversity gain due to multiple relays is shown through plots of the analytical BER, well supported by simulations. 
%
%\end{abstract}
% IEEEtran.cls defaults to using nonbold math in the Abstract.
% This preserves the distinction between vectors and scalars. However,
% if the journal you are submitting to favors bold math in the abstract,
% then you can use LaTeX's standard command \boldmath at the very start
% of the abstract to achieve this. Many IEEE journals frown on math
% in the abstract anyway.

% Note that keywords are not normally used for peerreview papers.
%\begin{IEEEkeywords}
%Cooperative diversity, decode and forward, piecewise linear
%\end{IEEEkeywords}



% For peer review papers, you can put extra information on the cover
% page as needed:
% \ifCLASSOPTIONpeerreview
% \begin{center} \bfseries EDICS Category: 3-BBND \end{center}
% \fi
%
% For peerreview papers, this IEEEtran command inserts a page break and
% creates the second title. It will be ignored for other modes.
%\IEEEpeerreviewmaketitle




\item A card is drawn from a deck of 52 cards. Find the probability of getting a king or a heart or a red card.\\
\solution
%\begin{table}[H]
	\centering
\begin{tabular}{|c|c|c|}
\hline
Random variable &Value &Definition\\ \hline
\multirow{3}{*}{X} &0 &Slips of Rs 1\\
&1 &Slips of Rs 5\\
&2 &Slips of Rs 13\\ \hline
\multirow{2}{*}{Y} &0 &Box A\\
&1 &Box B\\\hline
\end{tabular}
\caption{}
\label{tab:Distribution}
\end{table}
See \tabref{tab:Distribution}.
\begin{align}
p_{Y}\brak{k}= \begin{cases} 
      \frac{1}{3} & {k=0} \\
      \frac{2}{3 }& {k=1} 
   \end{cases}
   \\
p_{Y|X}\brak{0|0} = \frac{19}{25}\, 
p_{Y|X}\brak{0|1} = \frac{6}{25}\,
p_{Y|X}\brak{1|0} = \frac{45}{50}\,
p_{Y|X}\brak{1|2} = \frac{5}{50}
\end{align}
The desired probability is the probability that a slip drawn at random is marked other than Rs 1,
\begin{align}
&=1-p_X\brak{0}\\
&= p_X(1) + p_X(2)
\end{align}
Using Bayes theorem,
\begin{align}
&= p_Y\brak{0} \times \pr{Y=0 | X=1} + p_Y\brak{1} \times \pr{Y=1|X=2}\\
&=\frac{1}{3} \times \frac{6}{25} + \frac{2}{3} \times \frac{5}{50}\\
&=\frac{11}{75}
\end{align}

\newpage

%\tableofcontents

\bigskip

\renewcommand{\thefigure}{\theenumi}
\renewcommand{\thetable}{\theenumi}
%\renewcommand{\theequation}{\theenumi}

%\begin{abstract}
%%\boldmath
%In this letter, an algorithm for evaluating the exact analytical bit error rate  (BER)  for the piecewise linear (PL) combiner for  multiple relays is presented. Previous results were available only for upto three relays. The algorithm is unique in the sense that  the actual mathematical expressions, that are prohibitively large, need not be explicitly obtained. The diversity gain due to multiple relays is shown through plots of the analytical BER, well supported by simulations. 
%
%\end{abstract}
% IEEEtran.cls defaults to using nonbold math in the Abstract.
% This preserves the distinction between vectors and scalars. However,
% if the journal you are submitting to favors bold math in the abstract,
% then you can use LaTeX's standard command \boldmath at the very start
% of the abstract to achieve this. Many IEEE journals frown on math
% in the abstract anyway.

% Note that keywords are not normally used for peerreview papers.
%\begin{IEEEkeywords}
%Cooperative diversity, decode and forward, piecewise linear
%\end{IEEEkeywords}



% For peer review papers, you can put extra information on the cover
% page as needed:
% \ifCLASSOPTIONpeerreview
% \begin{center} \bfseries EDICS Category: 3-BBND \end{center}
% \fi
%
% For peerreview papers, this IEEEtran command inserts a page break and
% creates the second title. It will be ignored for other modes.
%\IEEEpeerreviewmaketitle




\item The probability that a student will pass his examination is 0.73, the probability of
the student getting a compartment is 0.13, and the probability that the student will
either pass or get compartment is 0.96. State True or False.\\
\solution
%\begin{table}[H]
	\centering
\begin{tabular}{|c|c|c|}
\hline
Random variable &Value &Definition\\ \hline
\multirow{3}{*}{X} &0 &Slips of Rs 1\\
&1 &Slips of Rs 5\\
&2 &Slips of Rs 13\\ \hline
\multirow{2}{*}{Y} &0 &Box A\\
&1 &Box B\\\hline
\end{tabular}
\caption{}
\label{tab:Distribution}
\end{table}
See \tabref{tab:Distribution}.
\begin{align}
p_{Y}\brak{k}= \begin{cases} 
      \frac{1}{3} & {k=0} \\
      \frac{2}{3 }& {k=1} 
   \end{cases}
   \\
p_{Y|X}\brak{0|0} = \frac{19}{25}\, 
p_{Y|X}\brak{0|1} = \frac{6}{25}\,
p_{Y|X}\brak{1|0} = \frac{45}{50}\,
p_{Y|X}\brak{1|2} = \frac{5}{50}
\end{align}
The desired probability is the probability that a slip drawn at random is marked other than Rs 1,
\begin{align}
&=1-p_X\brak{0}\\
&= p_X(1) + p_X(2)
\end{align}
Using Bayes theorem,
\begin{align}
&= p_Y\brak{0} \times \pr{Y=0 | X=1} + p_Y\brak{1} \times \pr{Y=1|X=2}\\
&=\frac{1}{3} \times \frac{6}{25} + \frac{2}{3} \times \frac{5}{50}\\
&=\frac{11}{75}
\end{align}

\newpage

%\tableofcontents

\bigskip

\renewcommand{\thefigure}{\theenumi}
\renewcommand{\thetable}{\theenumi}
%\renewcommand{\theequation}{\theenumi}

%\begin{abstract}
%%\boldmath
%In this letter, an algorithm for evaluating the exact analytical bit error rate  (BER)  for the piecewise linear (PL) combiner for  multiple relays is presented. Previous results were available only for upto three relays. The algorithm is unique in the sense that  the actual mathematical expressions, that are prohibitively large, need not be explicitly obtained. The diversity gain due to multiple relays is shown through plots of the analytical BER, well supported by simulations. 
%
%\end{abstract}
% IEEEtran.cls defaults to using nonbold math in the Abstract.
% This preserves the distinction between vectors and scalars. However,
% if the journal you are submitting to favors bold math in the abstract,
% then you can use LaTeX's standard command \boldmath at the very start
% of the abstract to achieve this. Many IEEE journals frown on math
% in the abstract anyway.

% Note that keywords are not normally used for peerreview papers.
%\begin{IEEEkeywords}
%Cooperative diversity, decode and forward, piecewise linear
%\end{IEEEkeywords}



% For peer review papers, you can put extra information on the cover
% page as needed:
% \ifCLASSOPTIONpeerreview
% \begin{center} \bfseries EDICS Category: 3-BBND \end{center}
% \fi
%
% For peerreview papers, this IEEEtran command inserts a page break and
% creates the second title. It will be ignored for other modes.
%\IEEEpeerreviewmaketitle




\item A card is selected from a pack of 52 cards\\
\begin{enumerate}[label=(\alph*)]
\item How many points are there in the sample space?
\item Calculate the probability that the cards is an ace of spades.
\item Calculate the probability that the card is (i) an ace (ii)black card.\\
\end{enumerate}
%\input{ncert/11/16/3/4_1/Prob_4.tex}
\item In a non-leap year, the probability of having 53 tuesdays or 53 wednesdays is\\
\solution
%A non-leap year has a total of 365 days, and a week has 7 days.\\
So it can be expressed as 
\begin{align}
365\text{days} &=52\times 7+1 \text{day}
\end{align}
$\implies$ 52 tuesdays or wednesdays\\
Random variable X denotes the days of a week
\begin{align}
p_X\brak{k}&=\frac{1}{7}; \quad \brak{1<k<7}
\end{align}
So the probability of extra day being tuesday or wednesday is
\begin{align}
p_X\brak{3}+p_X\brak{4}&=\frac{1}{7}+\frac{1}{7}=\frac{2}{7}
\end{align}



\item There are 1000 sealed envelopes in a box, 10 of them contain a cash prize of
Rs 100 each, 100 of them contain a cash prize of Rs 50 each and 200 of them
contain a cash prize of Rs 10 each and rest do not contain any cash prize. If they
are well shuffled and an envelope is picked up out, what is the probability that it
contains no cash prize?\\
\solution
%\begin{table}[H]
	\centering
\begin{tabular}{|c|c|c|}
\hline
Random variable &Value &Definition\\ \hline
\multirow{3}{*}{X} &0 &Slips of Rs 1\\
&1 &Slips of Rs 5\\
&2 &Slips of Rs 13\\ \hline
\multirow{2}{*}{Y} &0 &Box A\\
&1 &Box B\\\hline
\end{tabular}
\caption{}
\label{tab:Distribution}
\end{table}
See \tabref{tab:Distribution}.
\begin{align}
p_{Y}\brak{k}= \begin{cases} 
      \frac{1}{3} & {k=0} \\
      \frac{2}{3 }& {k=1} 
   \end{cases}
   \\
p_{Y|X}\brak{0|0} = \frac{19}{25}\, 
p_{Y|X}\brak{0|1} = \frac{6}{25}\,
p_{Y|X}\brak{1|0} = \frac{45}{50}\,
p_{Y|X}\brak{1|2} = \frac{5}{50}
\end{align}
The desired probability is the probability that a slip drawn at random is marked other than Rs 1,
\begin{align}
&=1-p_X\brak{0}\\
&= p_X(1) + p_X(2)
\end{align}
Using Bayes theorem,
\begin{align}
&= p_Y\brak{0} \times \pr{Y=0 | X=1} + p_Y\brak{1} \times \pr{Y=1|X=2}\\
&=\frac{1}{3} \times \frac{6}{25} + \frac{2}{3} \times \frac{5}{50}\\
&=\frac{11}{75}
\end{align}

\newpage

%\tableofcontents

\bigskip

\renewcommand{\thefigure}{\theenumi}
\renewcommand{\thetable}{\theenumi}
%\renewcommand{\theequation}{\theenumi}

%\begin{abstract}
%%\boldmath
%In this letter, an algorithm for evaluating the exact analytical bit error rate  (BER)  for the piecewise linear (PL) combiner for  multiple relays is presented. Previous results were available only for upto three relays. The algorithm is unique in the sense that  the actual mathematical expressions, that are prohibitively large, need not be explicitly obtained. The diversity gain due to multiple relays is shown through plots of the analytical BER, well supported by simulations. 
%
%\end{abstract}
% IEEEtran.cls defaults to using nonbold math in the Abstract.
% This preserves the distinction between vectors and scalars. However,
% if the journal you are submitting to favors bold math in the abstract,
% then you can use LaTeX's standard command \boldmath at the very start
% of the abstract to achieve this. Many IEEE journals frown on math
% in the abstract anyway.

% Note that keywords are not normally used for peerreview papers.
%\begin{IEEEkeywords}
%Cooperative diversity, decode and forward, piecewise linear
%\end{IEEEkeywords}



% For peer review papers, you can put extra information on the cover
% page as needed:
% \ifCLASSOPTIONpeerreview
% \begin{center} \bfseries EDICS Category: 3-BBND \end{center}
% \fi
%
% For peerreview papers, this IEEEtran command inserts a page break and
% creates the second title. It will be ignored for other modes.
%\IEEEpeerreviewmaketitle




\item 
A die is thrown and a card is selected at random from a deck of 52 playing cards. The probability of getting an even number on the die and a spade card.\\
\solution
%\begin{table}[H]
	\centering
\begin{tabular}{|c|c|c|}
\hline
Random variable &Value &Definition\\ \hline
\multirow{3}{*}{X} &0 &Slips of Rs 1\\
&1 &Slips of Rs 5\\
&2 &Slips of Rs 13\\ \hline
\multirow{2}{*}{Y} &0 &Box A\\
&1 &Box B\\\hline
\end{tabular}
\caption{}
\label{tab:Distribution}
\end{table}
See \tabref{tab:Distribution}.
\begin{align}
p_{Y}\brak{k}= \begin{cases} 
      \frac{1}{3} & {k=0} \\
      \frac{2}{3 }& {k=1} 
   \end{cases}
   \\
p_{Y|X}\brak{0|0} = \frac{19}{25}\, 
p_{Y|X}\brak{0|1} = \frac{6}{25}\,
p_{Y|X}\brak{1|0} = \frac{45}{50}\,
p_{Y|X}\brak{1|2} = \frac{5}{50}
\end{align}
The desired probability is the probability that a slip drawn at random is marked other than Rs 1,
\begin{align}
&=1-p_X\brak{0}\\
&= p_X(1) + p_X(2)
\end{align}
Using Bayes theorem,
\begin{align}
&= p_Y\brak{0} \times \pr{Y=0 | X=1} + p_Y\brak{1} \times \pr{Y=1|X=2}\\
&=\frac{1}{3} \times \frac{6}{25} + \frac{2}{3} \times \frac{5}{50}\\
&=\frac{11}{75}
\end{align}

\newpage

%\tableofcontents

\bigskip

\renewcommand{\thefigure}{\theenumi}
\renewcommand{\thetable}{\theenumi}
%\renewcommand{\theequation}{\theenumi}

%\begin{abstract}
%%\boldmath
%In this letter, an algorithm for evaluating the exact analytical bit error rate  (BER)  for the piecewise linear (PL) combiner for  multiple relays is presented. Previous results were available only for upto three relays. The algorithm is unique in the sense that  the actual mathematical expressions, that are prohibitively large, need not be explicitly obtained. The diversity gain due to multiple relays is shown through plots of the analytical BER, well supported by simulations. 
%
%\end{abstract}
% IEEEtran.cls defaults to using nonbold math in the Abstract.
% This preserves the distinction between vectors and scalars. However,
% if the journal you are submitting to favors bold math in the abstract,
% then you can use LaTeX's standard command \boldmath at the very start
% of the abstract to achieve this. Many IEEE journals frown on math
% in the abstract anyway.

% Note that keywords are not normally used for peerreview papers.
%\begin{IEEEkeywords}
%Cooperative diversity, decode and forward, piecewise linear
%\end{IEEEkeywords}



% For peer review papers, you can put extra information on the cover
% page as needed:
% \ifCLASSOPTIONpeerreview
% \begin{center} \bfseries EDICS Category: 3-BBND \end{center}
% \fi
%
% For peerreview papers, this IEEEtran command inserts a page break and
% creates the second title. It will be ignored for other modes.
%\IEEEpeerreviewmaketitle




\item
If 4-digit numbers greater than 5,000 are randomly formed from the digits 0, 1, 3, 5, and 7, what is the probability of forming a number divisible by 5 when:
\begin{enumerate}
    \item The digits are repeated?
    \item The repetition of digits is not allowed?
\end{enumerate}
\solution
%\begin{table}[H]
	\centering
\begin{tabular}{|c|c|c|}
\hline
Random variable &Value &Definition\\ \hline
\multirow{3}{*}{X} &0 &Slips of Rs 1\\
&1 &Slips of Rs 5\\
&2 &Slips of Rs 13\\ \hline
\multirow{2}{*}{Y} &0 &Box A\\
&1 &Box B\\\hline
\end{tabular}
\caption{}
\label{tab:Distribution}
\end{table}
See \tabref{tab:Distribution}.
\begin{align}
p_{Y}\brak{k}= \begin{cases} 
      \frac{1}{3} & {k=0} \\
      \frac{2}{3 }& {k=1} 
   \end{cases}
   \\
p_{Y|X}\brak{0|0} = \frac{19}{25}\, 
p_{Y|X}\brak{0|1} = \frac{6}{25}\,
p_{Y|X}\brak{1|0} = \frac{45}{50}\,
p_{Y|X}\brak{1|2} = \frac{5}{50}
\end{align}
The desired probability is the probability that a slip drawn at random is marked other than Rs 1,
\begin{align}
&=1-p_X\brak{0}\\
&= p_X(1) + p_X(2)
\end{align}
Using Bayes theorem,
\begin{align}
&= p_Y\brak{0} \times \pr{Y=0 | X=1} + p_Y\brak{1} \times \pr{Y=1|X=2}\\
&=\frac{1}{3} \times \frac{6}{25} + \frac{2}{3} \times \frac{5}{50}\\
&=\frac{11}{75}
\end{align}

\newpage

%\tableofcontents

\bigskip

\renewcommand{\thefigure}{\theenumi}
\renewcommand{\thetable}{\theenumi}
%\renewcommand{\theequation}{\theenumi}

%\begin{abstract}
%%\boldmath
%In this letter, an algorithm for evaluating the exact analytical bit error rate  (BER)  for the piecewise linear (PL) combiner for  multiple relays is presented. Previous results were available only for upto three relays. The algorithm is unique in the sense that  the actual mathematical expressions, that are prohibitively large, need not be explicitly obtained. The diversity gain due to multiple relays is shown through plots of the analytical BER, well supported by simulations. 
%
%\end{abstract}
% IEEEtran.cls defaults to using nonbold math in the Abstract.
% This preserves the distinction between vectors and scalars. However,
% if the journal you are submitting to favors bold math in the abstract,
% then you can use LaTeX's standard command \boldmath at the very start
% of the abstract to achieve this. Many IEEE journals frown on math
% in the abstract anyway.

% Note that keywords are not normally used for peerreview papers.
%\begin{IEEEkeywords}
%Cooperative diversity, decode and forward, piecewise linear
%\end{IEEEkeywords}



% For peer review papers, you can put extra information on the cover
% page as needed:
% \ifCLASSOPTIONpeerreview
% \begin{center} \bfseries EDICS Category: 3-BBND \end{center}
% \fi
%
% For peerreview papers, this IEEEtran command inserts a page break and
% creates the second title. It will be ignored for other modes.
%\IEEEpeerreviewmaketitle




\item Consider the probability space $\brak{\Omega, \mathcal{G}, P}$ where $\Omega = [0,2]$ and $\mathcal{G} = \cbrak{\phi, \Omega, [0,1], (1,2]}$. Let $X$ and $Y$ be two functions on $\Omega$ defined as
\begin{align*}
    X(\omega) = 
    \begin{cases}
        1 & \text{if }\omega \in [0, 1]\\
        2 & \text{if }\omega \in (1, 2]
    \end{cases}
\end{align*}
and
\begin{align*}
    Y(\omega) = 
    \begin{cases}
        2 & \text{if }\omega \in [0, 1.5]\\
        3 & \text{if }\omega \in (1.5, 2].
    \end{cases}
\end{align*}
Then which one of the following statements is true?
\begin{enumerate}
    \item [(A)] $X$ is a random variable with respect to $\mathcal{G}$, but $Y$ is not a random variable with respect to $\mathcal{G}$.
    \item [(B)] $Y$ is a random variable with respect to $\mathcal{G}$, but $X$ is not a random variable with respect to $\mathcal{G}$.
    \item [(C)] Neither $X$ nor $Y$ is a random variable with respect to $\mathcal{G}$.
    \item [(D)] Both $X$ and $Y$ are random variables with respect to $\mathcal{G}$.
\end{enumerate} \hfill (GATE ST 2023)\\
\solution
%\begin{table}[H]
	\centering
\begin{tabular}{|c|c|c|}
\hline
Random variable &Value &Definition\\ \hline
\multirow{3}{*}{X} &0 &Slips of Rs 1\\
&1 &Slips of Rs 5\\
&2 &Slips of Rs 13\\ \hline
\multirow{2}{*}{Y} &0 &Box A\\
&1 &Box B\\\hline
\end{tabular}
\caption{}
\label{tab:Distribution}
\end{table}
See \tabref{tab:Distribution}.
\begin{align}
p_{Y}\brak{k}= \begin{cases} 
      \frac{1}{3} & {k=0} \\
      \frac{2}{3 }& {k=1} 
   \end{cases}
   \\
p_{Y|X}\brak{0|0} = \frac{19}{25}\, 
p_{Y|X}\brak{0|1} = \frac{6}{25}\,
p_{Y|X}\brak{1|0} = \frac{45}{50}\,
p_{Y|X}\brak{1|2} = \frac{5}{50}
\end{align}
The desired probability is the probability that a slip drawn at random is marked other than Rs 1,
\begin{align}
&=1-p_X\brak{0}\\
&= p_X(1) + p_X(2)
\end{align}
Using Bayes theorem,
\begin{align}
&= p_Y\brak{0} \times \pr{Y=0 | X=1} + p_Y\brak{1} \times \pr{Y=1|X=2}\\
&=\frac{1}{3} \times \frac{6}{25} + \frac{2}{3} \times \frac{5}{50}\\
&=\frac{11}{75}
\end{align}

\newpage

%\tableofcontents

\bigskip

\renewcommand{\thefigure}{\theenumi}
\renewcommand{\thetable}{\theenumi}
%\renewcommand{\theequation}{\theenumi}

%\begin{abstract}
%%\boldmath
%In this letter, an algorithm for evaluating the exact analytical bit error rate  (BER)  for the piecewise linear (PL) combiner for  multiple relays is presented. Previous results were available only for upto three relays. The algorithm is unique in the sense that  the actual mathematical expressions, that are prohibitively large, need not be explicitly obtained. The diversity gain due to multiple relays is shown through plots of the analytical BER, well supported by simulations. 
%
%\end{abstract}
% IEEEtran.cls defaults to using nonbold math in the Abstract.
% This preserves the distinction between vectors and scalars. However,
% if the journal you are submitting to favors bold math in the abstract,
% then you can use LaTeX's standard command \boldmath at the very start
% of the abstract to achieve this. Many IEEE journals frown on math
% in the abstract anyway.

% Note that keywords are not normally used for peerreview papers.
%\begin{IEEEkeywords}
%Cooperative diversity, decode and forward, piecewise linear
%\end{IEEEkeywords}



% For peer review papers, you can put extra information on the cover
% page as needed:
% \ifCLASSOPTIONpeerreview
% \begin{center} \bfseries EDICS Category: 3-BBND \end{center}
% \fi
%
% For peerreview papers, this IEEEtran command inserts a page break and
% creates the second title. It will be ignored for other modes.
%\IEEEpeerreviewmaketitle




	\item  A die is loaded in such a way that each odd number is twice as likely to occur as
each even number. Find $P(G)$, where $G$ is the event that a number greater than
3 occurs on a single roll of the die.
\\
\solution
		%\begin{table}[H]
	\centering
\begin{tabular}{|c|c|c|}
\hline
Random variable &Value &Definition\\ \hline
\multirow{3}{*}{X} &0 &Slips of Rs 1\\
&1 &Slips of Rs 5\\
&2 &Slips of Rs 13\\ \hline
\multirow{2}{*}{Y} &0 &Box A\\
&1 &Box B\\\hline
\end{tabular}
\caption{}
\label{tab:Distribution}
\end{table}
See \tabref{tab:Distribution}.
\begin{align}
p_{Y}\brak{k}= \begin{cases} 
      \frac{1}{3} & {k=0} \\
      \frac{2}{3 }& {k=1} 
   \end{cases}
   \\
p_{Y|X}\brak{0|0} = \frac{19}{25}\, 
p_{Y|X}\brak{0|1} = \frac{6}{25}\,
p_{Y|X}\brak{1|0} = \frac{45}{50}\,
p_{Y|X}\brak{1|2} = \frac{5}{50}
\end{align}
The desired probability is the probability that a slip drawn at random is marked other than Rs 1,
\begin{align}
&=1-p_X\brak{0}\\
&= p_X(1) + p_X(2)
\end{align}
Using Bayes theorem,
\begin{align}
&= p_Y\brak{0} \times \pr{Y=0 | X=1} + p_Y\brak{1} \times \pr{Y=1|X=2}\\
&=\frac{1}{3} \times \frac{6}{25} + \frac{2}{3} \times \frac{5}{50}\\
&=\frac{11}{75}
\end{align}

\newpage

%\tableofcontents

\bigskip

\renewcommand{\thefigure}{\theenumi}
\renewcommand{\thetable}{\theenumi}
%\renewcommand{\theequation}{\theenumi}

%\begin{abstract}
%%\boldmath
%In this letter, an algorithm for evaluating the exact analytical bit error rate  (BER)  for the piecewise linear (PL) combiner for  multiple relays is presented. Previous results were available only for upto three relays. The algorithm is unique in the sense that  the actual mathematical expressions, that are prohibitively large, need not be explicitly obtained. The diversity gain due to multiple relays is shown through plots of the analytical BER, well supported by simulations. 
%
%\end{abstract}
% IEEEtran.cls defaults to using nonbold math in the Abstract.
% This preserves the distinction between vectors and scalars. However,
% if the journal you are submitting to favors bold math in the abstract,
% then you can use LaTeX's standard command \boldmath at the very start
% of the abstract to achieve this. Many IEEE journals frown on math
% in the abstract anyway.

% Note that keywords are not normally used for peerreview papers.
%\begin{IEEEkeywords}
%Cooperative diversity, decode and forward, piecewise linear
%\end{IEEEkeywords}



% For peer review papers, you can put extra information on the cover
% page as needed:
% \ifCLASSOPTIONpeerreview
% \begin{center} \bfseries EDICS Category: 3-BBND \end{center}
% \fi
%
% For peerreview papers, this IEEEtran command inserts a page break and
% creates the second title. It will be ignored for other modes.
%\IEEEpeerreviewmaketitle




	\item All the jacks, queens and kings are removed from a deck of 52 playing cards. The remaining cards are well shuffled and then one card is drawn at random. Giving ace a value 1 similar value for other cards, find the probability that the card has a value 
		\begin{enumerate}
			\item 7
			\item greater than 7
			\item less than 7
		\end{enumerate}
		%Number of cards left after removing all jacks, queens and kings 
\begin{align}
N	= 52 - 4\times 3
	= 40
\end{align}
%\begin{table}[H]
%\def\arraystretch{1.2}
%\begin{tabular}{|c|c|c|}
%\hline
%	\textbf{Parameter} &\textbf{Value} &\textbf{Description}\\ \hline
%	$X$ &1-10 &Represents the value of the card picked \\ \hline
%\end{tabular}
%\end{table}
Let $1 \le X \le 10$ be the value of the card picked.  Then,
\begin{align}
	p_X(k) &= \Pr(X=k)\ \forall\ 1 \leq k \leq 10\\
	&= \frac{4\times 1}{40}\\
	&= \frac{1}{10}\\
	\therefore p_X(k) &= 
	\begin{cases}
		\frac{1}{10} & 1 \leq k \leq 10\\
		0 & \text{otherwise}
	\end{cases}
\end{align}
and
\begin{align}
	F_{X}(k) &= \sum_{m=0}^{k}p_{X}(m) \quad 1 \leq k \leq 10\\
	&= \frac{k}{10}\\
	\therefore F_{X}(k) &= 
	\begin{cases}
		0 & k \leq 0\\
		\frac{k}{10} & 1\leq k \leq 10\\
		1 & k > 10 
	\end{cases}
\end{align}
\begin{enumerate}
	\item Probability that card has value equal to 7 is
		\begin{align}
			 p_{X}(7)
			= \frac{1}{10}
		\end{align}
	\item Probability that card has value greater than 7 is
		\begin{align}
			1 - F_X(7)
			&= 1 - \frac{7}{10}
			\\
			&= \frac{3}{10}
		\end{align}
	\item Probability that card has value less than 7 is
		\begin{align}
			 F_{X}(6)
			=\frac{6}{10}
		\end{align}
\end{enumerate}

  \item A Lot consists of 48 mobile phones of which 42 are good, 3 have only minor defects and 3 have major defects.Varnika will buy a phone if it is good but the trader will only buy a mobile if it has no major defects. One phone is selected at random from the lot. What is the probability that it is
\begin{enumerate}
	\item acceptable to Varnika?
            \item acceptable to the trader?
\end{enumerate}
\solution
	%\begin{table}[H]
	\centering
\begin{tabular}{|c|c|c|}
\hline
Random variable &Value &Definition\\ \hline
\multirow{3}{*}{X} &0 &Slips of Rs 1\\
&1 &Slips of Rs 5\\
&2 &Slips of Rs 13\\ \hline
\multirow{2}{*}{Y} &0 &Box A\\
&1 &Box B\\\hline
\end{tabular}
\caption{}
\label{tab:Distribution}
\end{table}
See \tabref{tab:Distribution}.
\begin{align}
p_{Y}\brak{k}= \begin{cases} 
      \frac{1}{3} & {k=0} \\
      \frac{2}{3 }& {k=1} 
   \end{cases}
   \\
p_{Y|X}\brak{0|0} = \frac{19}{25}\, 
p_{Y|X}\brak{0|1} = \frac{6}{25}\,
p_{Y|X}\brak{1|0} = \frac{45}{50}\,
p_{Y|X}\brak{1|2} = \frac{5}{50}
\end{align}
The desired probability is the probability that a slip drawn at random is marked other than Rs 1,
\begin{align}
&=1-p_X\brak{0}\\
&= p_X(1) + p_X(2)
\end{align}
Using Bayes theorem,
\begin{align}
&= p_Y\brak{0} \times \pr{Y=0 | X=1} + p_Y\brak{1} \times \pr{Y=1|X=2}\\
&=\frac{1}{3} \times \frac{6}{25} + \frac{2}{3} \times \frac{5}{50}\\
&=\frac{11}{75}
\end{align}

\newpage

%\tableofcontents

\bigskip

\renewcommand{\thefigure}{\theenumi}
\renewcommand{\thetable}{\theenumi}
%\renewcommand{\theequation}{\theenumi}

%\begin{abstract}
%%\boldmath
%In this letter, an algorithm for evaluating the exact analytical bit error rate  (BER)  for the piecewise linear (PL) combiner for  multiple relays is presented. Previous results were available only for upto three relays. The algorithm is unique in the sense that  the actual mathematical expressions, that are prohibitively large, need not be explicitly obtained. The diversity gain due to multiple relays is shown through plots of the analytical BER, well supported by simulations. 
%
%\end{abstract}
% IEEEtran.cls defaults to using nonbold math in the Abstract.
% This preserves the distinction between vectors and scalars. However,
% if the journal you are submitting to favors bold math in the abstract,
% then you can use LaTeX's standard command \boldmath at the very start
% of the abstract to achieve this. Many IEEE journals frown on math
% in the abstract anyway.

% Note that keywords are not normally used for peerreview papers.
%\begin{IEEEkeywords}
%Cooperative diversity, decode and forward, piecewise linear
%\end{IEEEkeywords}



% For peer review papers, you can put extra information on the cover
% page as needed:
% \ifCLASSOPTIONpeerreview
% \begin{center} \bfseries EDICS Category: 3-BBND \end{center}
% \fi
%
% For peerreview papers, this IEEEtran command inserts a page break and
% creates the second title. It will be ignored for other modes.
%\IEEEpeerreviewmaketitle




 \item A student says that if you throw a die, it will show up 1 or not 1. Therefore, the probability of getting 1 and the probability of getting 'not 1' each is equal to $\frac{1}{2}$. Is this correct? Give reasons.\\
 \solution
        %\begin{table}[H]
	\centering
\begin{tabular}{|c|c|c|}
\hline
Random variable &Value &Definition\\ \hline
\multirow{3}{*}{X} &0 &Slips of Rs 1\\
&1 &Slips of Rs 5\\
&2 &Slips of Rs 13\\ \hline
\multirow{2}{*}{Y} &0 &Box A\\
&1 &Box B\\\hline
\end{tabular}
\caption{}
\label{tab:Distribution}
\end{table}
See \tabref{tab:Distribution}.
\begin{align}
p_{Y}\brak{k}= \begin{cases} 
      \frac{1}{3} & {k=0} \\
      \frac{2}{3 }& {k=1} 
   \end{cases}
   \\
p_{Y|X}\brak{0|0} = \frac{19}{25}\, 
p_{Y|X}\brak{0|1} = \frac{6}{25}\,
p_{Y|X}\brak{1|0} = \frac{45}{50}\,
p_{Y|X}\brak{1|2} = \frac{5}{50}
\end{align}
The desired probability is the probability that a slip drawn at random is marked other than Rs 1,
\begin{align}
&=1-p_X\brak{0}\\
&= p_X(1) + p_X(2)
\end{align}
Using Bayes theorem,
\begin{align}
&= p_Y\brak{0} \times \pr{Y=0 | X=1} + p_Y\brak{1} \times \pr{Y=1|X=2}\\
&=\frac{1}{3} \times \frac{6}{25} + \frac{2}{3} \times \frac{5}{50}\\
&=\frac{11}{75}
\end{align}

\newpage

%\tableofcontents

\bigskip

\renewcommand{\thefigure}{\theenumi}
\renewcommand{\thetable}{\theenumi}
%\renewcommand{\theequation}{\theenumi}

%\begin{abstract}
%%\boldmath
%In this letter, an algorithm for evaluating the exact analytical bit error rate  (BER)  for the piecewise linear (PL) combiner for  multiple relays is presented. Previous results were available only for upto three relays. The algorithm is unique in the sense that  the actual mathematical expressions, that are prohibitively large, need not be explicitly obtained. The diversity gain due to multiple relays is shown through plots of the analytical BER, well supported by simulations. 
%
%\end{abstract}
% IEEEtran.cls defaults to using nonbold math in the Abstract.
% This preserves the distinction between vectors and scalars. However,
% if the journal you are submitting to favors bold math in the abstract,
% then you can use LaTeX's standard command \boldmath at the very start
% of the abstract to achieve this. Many IEEE journals frown on math
% in the abstract anyway.

% Note that keywords are not normally used for peerreview papers.
%\begin{IEEEkeywords}
%Cooperative diversity, decode and forward, piecewise linear
%\end{IEEEkeywords}



% For peer review papers, you can put extra information on the cover
% page as needed:
% \ifCLASSOPTIONpeerreview
% \begin{center} \bfseries EDICS Category: 3-BBND \end{center}
% \fi
%
% For peerreview papers, this IEEEtran command inserts a page break and
% creates the second title. It will be ignored for other modes.
%\IEEEpeerreviewmaketitle




   \item Four candidates A, B, C, D have ap-
plied for the assignment to coach a school cricket
team. If A is twice as likely to be selected as B, and
B and C are given about the same chance of being
selected, while C is twice as likely to be selected
as D, what are the probabilities that
\begin{enumerate}
\item C will be selected?
\item A will not be selected?
\end{enumerate}
	%\begin{table}[H]
	\centering
\begin{tabular}{|c|c|c|}
\hline
Random variable &Value &Definition\\ \hline
\multirow{3}{*}{X} &0 &Slips of Rs 1\\
&1 &Slips of Rs 5\\
&2 &Slips of Rs 13\\ \hline
\multirow{2}{*}{Y} &0 &Box A\\
&1 &Box B\\\hline
\end{tabular}
\caption{}
\label{tab:Distribution}
\end{table}
See \tabref{tab:Distribution}.
\begin{align}
p_{Y}\brak{k}= \begin{cases} 
      \frac{1}{3} & {k=0} \\
      \frac{2}{3 }& {k=1} 
   \end{cases}
   \\
p_{Y|X}\brak{0|0} = \frac{19}{25}\, 
p_{Y|X}\brak{0|1} = \frac{6}{25}\,
p_{Y|X}\brak{1|0} = \frac{45}{50}\,
p_{Y|X}\brak{1|2} = \frac{5}{50}
\end{align}
The desired probability is the probability that a slip drawn at random is marked other than Rs 1,
\begin{align}
&=1-p_X\brak{0}\\
&= p_X(1) + p_X(2)
\end{align}
Using Bayes theorem,
\begin{align}
&= p_Y\brak{0} \times \pr{Y=0 | X=1} + p_Y\brak{1} \times \pr{Y=1|X=2}\\
&=\frac{1}{3} \times \frac{6}{25} + \frac{2}{3} \times \frac{5}{50}\\
&=\frac{11}{75}
\end{align}

\newpage

%\tableofcontents

\bigskip

\renewcommand{\thefigure}{\theenumi}
\renewcommand{\thetable}{\theenumi}
%\renewcommand{\theequation}{\theenumi}

%\begin{abstract}
%%\boldmath
%In this letter, an algorithm for evaluating the exact analytical bit error rate  (BER)  for the piecewise linear (PL) combiner for  multiple relays is presented. Previous results were available only for upto three relays. The algorithm is unique in the sense that  the actual mathematical expressions, that are prohibitively large, need not be explicitly obtained. The diversity gain due to multiple relays is shown through plots of the analytical BER, well supported by simulations. 
%
%\end{abstract}
% IEEEtran.cls defaults to using nonbold math in the Abstract.
% This preserves the distinction between vectors and scalars. However,
% if the journal you are submitting to favors bold math in the abstract,
% then you can use LaTeX's standard command \boldmath at the very start
% of the abstract to achieve this. Many IEEE journals frown on math
% in the abstract anyway.

% Note that keywords are not normally used for peerreview papers.
%\begin{IEEEkeywords}
%Cooperative diversity, decode and forward, piecewise linear
%\end{IEEEkeywords}



% For peer review papers, you can put extra information on the cover
% page as needed:
% \ifCLASSOPTIONpeerreview
% \begin{center} \bfseries EDICS Category: 3-BBND \end{center}
% \fi
%
% For peerreview papers, this IEEEtran command inserts a page break and
% creates the second title. It will be ignored for other modes.
%\IEEEpeerreviewmaketitle




 \item A bag contain 24 balls of which $x$ balls are red, $2x$ are white and $3x$ are blue. A ball is selected at random, What is the probability that it is
\begin{enumerate}[label=\alph*)]
\item not red ?
\item white ?
\end{enumerate}
%\begin{table}[H]
	\centering
\begin{tabular}{|c|c|c|}
\hline
Random variable &Value &Definition\\ \hline
\multirow{3}{*}{X} &0 &Slips of Rs 1\\
&1 &Slips of Rs 5\\
&2 &Slips of Rs 13\\ \hline
\multirow{2}{*}{Y} &0 &Box A\\
&1 &Box B\\\hline
\end{tabular}
\caption{}
\label{tab:Distribution}
\end{table}
See \tabref{tab:Distribution}.
\begin{align}
p_{Y}\brak{k}= \begin{cases} 
      \frac{1}{3} & {k=0} \\
      \frac{2}{3 }& {k=1} 
   \end{cases}
   \\
p_{Y|X}\brak{0|0} = \frac{19}{25}\, 
p_{Y|X}\brak{0|1} = \frac{6}{25}\,
p_{Y|X}\brak{1|0} = \frac{45}{50}\,
p_{Y|X}\brak{1|2} = \frac{5}{50}
\end{align}
The desired probability is the probability that a slip drawn at random is marked other than Rs 1,
\begin{align}
&=1-p_X\brak{0}\\
&= p_X(1) + p_X(2)
\end{align}
Using Bayes theorem,
\begin{align}
&= p_Y\brak{0} \times \pr{Y=0 | X=1} + p_Y\brak{1} \times \pr{Y=1|X=2}\\
&=\frac{1}{3} \times \frac{6}{25} + \frac{2}{3} \times \frac{5}{50}\\
&=\frac{11}{75}
\end{align}

\newpage

%\tableofcontents

\bigskip

\renewcommand{\thefigure}{\theenumi}
\renewcommand{\thetable}{\theenumi}
%\renewcommand{\theequation}{\theenumi}

%\begin{abstract}
%%\boldmath
%In this letter, an algorithm for evaluating the exact analytical bit error rate  (BER)  for the piecewise linear (PL) combiner for  multiple relays is presented. Previous results were available only for upto three relays. The algorithm is unique in the sense that  the actual mathematical expressions, that are prohibitively large, need not be explicitly obtained. The diversity gain due to multiple relays is shown through plots of the analytical BER, well supported by simulations. 
%
%\end{abstract}
% IEEEtran.cls defaults to using nonbold math in the Abstract.
% This preserves the distinction between vectors and scalars. However,
% if the journal you are submitting to favors bold math in the abstract,
% then you can use LaTeX's standard command \boldmath at the very start
% of the abstract to achieve this. Many IEEE journals frown on math
% in the abstract anyway.

% Note that keywords are not normally used for peerreview papers.
%\begin{IEEEkeywords}
%Cooperative diversity, decode and forward, piecewise linear
%\end{IEEEkeywords}



% For peer review papers, you can put extra information on the cover
% page as needed:
% \ifCLASSOPTIONpeerreview
% \begin{center} \bfseries EDICS Category: 3-BBND \end{center}
% \fi
%
% For peerreview papers, this IEEEtran command inserts a page break and
% creates the second title. It will be ignored for other modes.
%\IEEEpeerreviewmaketitle




If the letters of the word ASSASSINATION are arranged at random. Find the Probability that
\begin{enumerate}[label=(\alph*)]
\item Four $S's$ come consecutively in the word
\item Two  $I's$ and two $N's$ come together
\item All $A's$ are not coming together
\item No two $A's$ are coming together
\end{enumerate}
%\begin{table}[H]
	\centering
\begin{tabular}{|c|c|c|}
\hline
Random variable &Value &Definition\\ \hline
\multirow{3}{*}{X} &0 &Slips of Rs 1\\
&1 &Slips of Rs 5\\
&2 &Slips of Rs 13\\ \hline
\multirow{2}{*}{Y} &0 &Box A\\
&1 &Box B\\\hline
\end{tabular}
\caption{}
\label{tab:Distribution}
\end{table}
See \tabref{tab:Distribution}.
\begin{align}
p_{Y}\brak{k}= \begin{cases} 
      \frac{1}{3} & {k=0} \\
      \frac{2}{3 }& {k=1} 
   \end{cases}
   \\
p_{Y|X}\brak{0|0} = \frac{19}{25}\, 
p_{Y|X}\brak{0|1} = \frac{6}{25}\,
p_{Y|X}\brak{1|0} = \frac{45}{50}\,
p_{Y|X}\brak{1|2} = \frac{5}{50}
\end{align}
The desired probability is the probability that a slip drawn at random is marked other than Rs 1,
\begin{align}
&=1-p_X\brak{0}\\
&= p_X(1) + p_X(2)
\end{align}
Using Bayes theorem,
\begin{align}
&= p_Y\brak{0} \times \pr{Y=0 | X=1} + p_Y\brak{1} \times \pr{Y=1|X=2}\\
&=\frac{1}{3} \times \frac{6}{25} + \frac{2}{3} \times \frac{5}{50}\\
&=\frac{11}{75}
\end{align}

\newpage

%\tableofcontents

\bigskip

\renewcommand{\thefigure}{\theenumi}
\renewcommand{\thetable}{\theenumi}
%\renewcommand{\theequation}{\theenumi}

%\begin{abstract}
%%\boldmath
%In this letter, an algorithm for evaluating the exact analytical bit error rate  (BER)  for the piecewise linear (PL) combiner for  multiple relays is presented. Previous results were available only for upto three relays. The algorithm is unique in the sense that  the actual mathematical expressions, that are prohibitively large, need not be explicitly obtained. The diversity gain due to multiple relays is shown through plots of the analytical BER, well supported by simulations. 
%
%\end{abstract}
% IEEEtran.cls defaults to using nonbold math in the Abstract.
% This preserves the distinction between vectors and scalars. However,
% if the journal you are submitting to favors bold math in the abstract,
% then you can use LaTeX's standard command \boldmath at the very start
% of the abstract to achieve this. Many IEEE journals frown on math
% in the abstract anyway.

% Note that keywords are not normally used for peerreview papers.
%\begin{IEEEkeywords}
%Cooperative diversity, decode and forward, piecewise linear
%\end{IEEEkeywords}



% For peer review papers, you can put extra information on the cover
% page as needed:
% \ifCLASSOPTIONpeerreview
% \begin{center} \bfseries EDICS Category: 3-BBND \end{center}
% \fi
%
% For peerreview papers, this IEEEtran command inserts a page break and
% creates the second title. It will be ignored for other modes.
%\IEEEpeerreviewmaketitle




	\item One urn contains two black balls (labelled B1 and B2) and one white ball. A
	second urn contains one black ball and two white balls (labelled W1 and W2).
	Suppose the following experiment is performed. One of the two urns is chosen
	at random. Next a ball is randomly chosen from the urn. Then a second ball is
	chosen at random from the same urn without replacing the first ball.
	
	\begin{enumerate}
	\item What is the probability that two black balls are chosen?
	
	\item What is the probability that two balls of opposite colour are chosen?
	\end{enumerate}
	\solution
	%\begin{align}
    \label{eq:12.13.6.18.1}
	\because	\pr{A|B} &> \pr{A},\
\frac{\pr{AB}}{\pr{B}} > \pr{A}
\\
    \label{eq:12.13.6.18.2}
	\implies \pr{AB} &> \pr{A}\pr{B}
	\\
	\text{or, } \frac{\pr{AB}}{\pr{A}} &=\pr{B|A} > \pr{A}
\end{align}

\end{enumerate}

	\item A bag contains 4 red and 4 black balls, another bag contains 2 red and 6 black balls. One of the two bags is selected at random and a ball is drawn from the bag which is found to be red. Find the probability that the ball is drawn from the first bag.
\\
\solution
		%\begin{table}[H]
	\centering
\begin{tabular}{|c|c|c|}
\hline
Random variable &Value &Definition\\ \hline
\multirow{3}{*}{X} &0 &Slips of Rs 1\\
&1 &Slips of Rs 5\\
&2 &Slips of Rs 13\\ \hline
\multirow{2}{*}{Y} &0 &Box A\\
&1 &Box B\\\hline
\end{tabular}
\caption{}
\label{tab:Distribution}
\end{table}
See \tabref{tab:Distribution}.
\begin{align}
p_{Y}\brak{k}= \begin{cases} 
      \frac{1}{3} & {k=0} \\
      \frac{2}{3 }& {k=1} 
   \end{cases}
   \\
p_{Y|X}\brak{0|0} = \frac{19}{25}\, 
p_{Y|X}\brak{0|1} = \frac{6}{25}\,
p_{Y|X}\brak{1|0} = \frac{45}{50}\,
p_{Y|X}\brak{1|2} = \frac{5}{50}
\end{align}
The desired probability is the probability that a slip drawn at random is marked other than Rs 1,
\begin{align}
&=1-p_X\brak{0}\\
&= p_X(1) + p_X(2)
\end{align}
Using Bayes theorem,
\begin{align}
&= p_Y\brak{0} \times \pr{Y=0 | X=1} + p_Y\brak{1} \times \pr{Y=1|X=2}\\
&=\frac{1}{3} \times \frac{6}{25} + \frac{2}{3} \times \frac{5}{50}\\
&=\frac{11}{75}
\end{align}

\newpage

%\tableofcontents

\bigskip

\renewcommand{\thefigure}{\theenumi}
\renewcommand{\thetable}{\theenumi}
%\renewcommand{\theequation}{\theenumi}

%\begin{abstract}
%%\boldmath
%In this letter, an algorithm for evaluating the exact analytical bit error rate  (BER)  for the piecewise linear (PL) combiner for  multiple relays is presented. Previous results were available only for upto three relays. The algorithm is unique in the sense that  the actual mathematical expressions, that are prohibitively large, need not be explicitly obtained. The diversity gain due to multiple relays is shown through plots of the analytical BER, well supported by simulations. 
%
%\end{abstract}
% IEEEtran.cls defaults to using nonbold math in the Abstract.
% This preserves the distinction between vectors and scalars. However,
% if the journal you are submitting to favors bold math in the abstract,
% then you can use LaTeX's standard command \boldmath at the very start
% of the abstract to achieve this. Many IEEE journals frown on math
% in the abstract anyway.

% Note that keywords are not normally used for peerreview papers.
%\begin{IEEEkeywords}
%Cooperative diversity, decode and forward, piecewise linear
%\end{IEEEkeywords}



% For peer review papers, you can put extra information on the cover
% page as needed:
% \ifCLASSOPTIONpeerreview
% \begin{center} \bfseries EDICS Category: 3-BBND \end{center}
% \fi
%
% For peerreview papers, this IEEEtran command inserts a page break and
% creates the second title. It will be ignored for other modes.
%\IEEEpeerreviewmaketitle




  \item
  Cards with numbers 2 to 101 are placed in a box. A card is selected at random.Find the probability that the card has
\begin{enumerate}[label=(\roman*)]
	\item an even number 
	\item a square number
\end{enumerate}
\solution
%\begin{table}[H]
	\centering
\begin{tabular}{|c|c|c|}
\hline
Random variable &Value &Definition\\ \hline
\multirow{3}{*}{X} &0 &Slips of Rs 1\\
&1 &Slips of Rs 5\\
&2 &Slips of Rs 13\\ \hline
\multirow{2}{*}{Y} &0 &Box A\\
&1 &Box B\\\hline
\end{tabular}
\caption{}
\label{tab:Distribution}
\end{table}
See \tabref{tab:Distribution}.
\begin{align}
p_{Y}\brak{k}= \begin{cases} 
      \frac{1}{3} & {k=0} \\
      \frac{2}{3 }& {k=1} 
   \end{cases}
   \\
p_{Y|X}\brak{0|0} = \frac{19}{25}\, 
p_{Y|X}\brak{0|1} = \frac{6}{25}\,
p_{Y|X}\brak{1|0} = \frac{45}{50}\,
p_{Y|X}\brak{1|2} = \frac{5}{50}
\end{align}
The desired probability is the probability that a slip drawn at random is marked other than Rs 1,
\begin{align}
&=1-p_X\brak{0}\\
&= p_X(1) + p_X(2)
\end{align}
Using Bayes theorem,
\begin{align}
&= p_Y\brak{0} \times \pr{Y=0 | X=1} + p_Y\brak{1} \times \pr{Y=1|X=2}\\
&=\frac{1}{3} \times \frac{6}{25} + \frac{2}{3} \times \frac{5}{50}\\
&=\frac{11}{75}
\end{align}

\newpage

%\tableofcontents

\bigskip

\renewcommand{\thefigure}{\theenumi}
\renewcommand{\thetable}{\theenumi}
%\renewcommand{\theequation}{\theenumi}

%\begin{abstract}
%%\boldmath
%In this letter, an algorithm for evaluating the exact analytical bit error rate  (BER)  for the piecewise linear (PL) combiner for  multiple relays is presented. Previous results were available only for upto three relays. The algorithm is unique in the sense that  the actual mathematical expressions, that are prohibitively large, need not be explicitly obtained. The diversity gain due to multiple relays is shown through plots of the analytical BER, well supported by simulations. 
%
%\end{abstract}
% IEEEtran.cls defaults to using nonbold math in the Abstract.
% This preserves the distinction between vectors and scalars. However,
% if the journal you are submitting to favors bold math in the abstract,
% then you can use LaTeX's standard command \boldmath at the very start
% of the abstract to achieve this. Many IEEE journals frown on math
% in the abstract anyway.

% Note that keywords are not normally used for peerreview papers.
%\begin{IEEEkeywords}
%Cooperative diversity, decode and forward, piecewise linear
%\end{IEEEkeywords}



% For peer review papers, you can put extra information on the cover
% page as needed:
% \ifCLASSOPTIONpeerreview
% \begin{center} \bfseries EDICS Category: 3-BBND \end{center}
% \fi
%
% For peerreview papers, this IEEEtran command inserts a page break and
% creates the second title. It will be ignored for other modes.
%\IEEEpeerreviewmaketitle




\item
The king, queen and jack of clubs are removed from a deck of 52 playing cards and then well shuffled. Now one card is drawn at random from the remaining cards.  Determine the probability that the card is
\begin{enumerate}[label=(\roman*)]
\item a club
\item 10 of hearts
\end{enumerate}
\solution
%\begin{table}[H]
	\centering
\begin{tabular}{|c|c|c|}
\hline
Random variable &Value &Definition\\ \hline
\multirow{3}{*}{X} &0 &Slips of Rs 1\\
&1 &Slips of Rs 5\\
&2 &Slips of Rs 13\\ \hline
\multirow{2}{*}{Y} &0 &Box A\\
&1 &Box B\\\hline
\end{tabular}
\caption{}
\label{tab:Distribution}
\end{table}
See \tabref{tab:Distribution}.
\begin{align}
p_{Y}\brak{k}= \begin{cases} 
      \frac{1}{3} & {k=0} \\
      \frac{2}{3 }& {k=1} 
   \end{cases}
   \\
p_{Y|X}\brak{0|0} = \frac{19}{25}\, 
p_{Y|X}\brak{0|1} = \frac{6}{25}\,
p_{Y|X}\brak{1|0} = \frac{45}{50}\,
p_{Y|X}\brak{1|2} = \frac{5}{50}
\end{align}
The desired probability is the probability that a slip drawn at random is marked other than Rs 1,
\begin{align}
&=1-p_X\brak{0}\\
&= p_X(1) + p_X(2)
\end{align}
Using Bayes theorem,
\begin{align}
&= p_Y\brak{0} \times \pr{Y=0 | X=1} + p_Y\brak{1} \times \pr{Y=1|X=2}\\
&=\frac{1}{3} \times \frac{6}{25} + \frac{2}{3} \times \frac{5}{50}\\
&=\frac{11}{75}
\end{align}

\newpage

%\tableofcontents

\bigskip

\renewcommand{\thefigure}{\theenumi}
\renewcommand{\thetable}{\theenumi}
%\renewcommand{\theequation}{\theenumi}

%\begin{abstract}
%%\boldmath
%In this letter, an algorithm for evaluating the exact analytical bit error rate  (BER)  for the piecewise linear (PL) combiner for  multiple relays is presented. Previous results were available only for upto three relays. The algorithm is unique in the sense that  the actual mathematical expressions, that are prohibitively large, need not be explicitly obtained. The diversity gain due to multiple relays is shown through plots of the analytical BER, well supported by simulations. 
%
%\end{abstract}
% IEEEtran.cls defaults to using nonbold math in the Abstract.
% This preserves the distinction between vectors and scalars. However,
% if the journal you are submitting to favors bold math in the abstract,
% then you can use LaTeX's standard command \boldmath at the very start
% of the abstract to achieve this. Many IEEE journals frown on math
% in the abstract anyway.

% Note that keywords are not normally used for peerreview papers.
%\begin{IEEEkeywords}
%Cooperative diversity, decode and forward, piecewise linear
%\end{IEEEkeywords}



% For peer review papers, you can put extra information on the cover
% page as needed:
% \ifCLASSOPTIONpeerreview
% \begin{center} \bfseries EDICS Category: 3-BBND \end{center}
% \fi
%
% For peerreview papers, this IEEEtran command inserts a page break and
% creates the second title. It will be ignored for other modes.
%\IEEEpeerreviewmaketitle




\item A team of medical students doing their internship have to assist during surgeries
at a city hospital. The probabilities of surgeries rated as very complex, complex,
routine, simple or very simple are respectively, 0.15, 0.20, 0.31, 0.26, .08. Find
the probabilities that a particular surgery will be rated
\begin{enumerate}
	\item complex or very complex;
	\item neither very complex nor very simple;
	\item routine or complex
	\item routine or simple
\end{enumerate}
\solution
%\begin{table}[H]
	\centering
\begin{tabular}{|c|c|c|}
\hline
Random variable &Value &Definition\\ \hline
\multirow{3}{*}{X} &0 &Slips of Rs 1\\
&1 &Slips of Rs 5\\
&2 &Slips of Rs 13\\ \hline
\multirow{2}{*}{Y} &0 &Box A\\
&1 &Box B\\\hline
\end{tabular}
\caption{}
\label{tab:Distribution}
\end{table}
See \tabref{tab:Distribution}.
\begin{align}
p_{Y}\brak{k}= \begin{cases} 
      \frac{1}{3} & {k=0} \\
      \frac{2}{3 }& {k=1} 
   \end{cases}
   \\
p_{Y|X}\brak{0|0} = \frac{19}{25}\, 
p_{Y|X}\brak{0|1} = \frac{6}{25}\,
p_{Y|X}\brak{1|0} = \frac{45}{50}\,
p_{Y|X}\brak{1|2} = \frac{5}{50}
\end{align}
The desired probability is the probability that a slip drawn at random is marked other than Rs 1,
\begin{align}
&=1-p_X\brak{0}\\
&= p_X(1) + p_X(2)
\end{align}
Using Bayes theorem,
\begin{align}
&= p_Y\brak{0} \times \pr{Y=0 | X=1} + p_Y\brak{1} \times \pr{Y=1|X=2}\\
&=\frac{1}{3} \times \frac{6}{25} + \frac{2}{3} \times \frac{5}{50}\\
&=\frac{11}{75}
\end{align}

\newpage

%\tableofcontents

\bigskip

\renewcommand{\thefigure}{\theenumi}
\renewcommand{\thetable}{\theenumi}
%\renewcommand{\theequation}{\theenumi}

%\begin{abstract}
%%\boldmath
%In this letter, an algorithm for evaluating the exact analytical bit error rate  (BER)  for the piecewise linear (PL) combiner for  multiple relays is presented. Previous results were available only for upto three relays. The algorithm is unique in the sense that  the actual mathematical expressions, that are prohibitively large, need not be explicitly obtained. The diversity gain due to multiple relays is shown through plots of the analytical BER, well supported by simulations. 
%
%\end{abstract}
% IEEEtran.cls defaults to using nonbold math in the Abstract.
% This preserves the distinction between vectors and scalars. However,
% if the journal you are submitting to favors bold math in the abstract,
% then you can use LaTeX's standard command \boldmath at the very start
% of the abstract to achieve this. Many IEEE journals frown on math
% in the abstract anyway.

% Note that keywords are not normally used for peerreview papers.
%\begin{IEEEkeywords}
%Cooperative diversity, decode and forward, piecewise linear
%\end{IEEEkeywords}



% For peer review papers, you can put extra information on the cover
% page as needed:
% \ifCLASSOPTIONpeerreview
% \begin{center} \bfseries EDICS Category: 3-BBND \end{center}
% \fi
%
% For peerreview papers, this IEEEtran command inserts a page break and
% creates the second title. It will be ignored for other modes.
%\IEEEpeerreviewmaketitle




\item A card is selected from a pack of 52 cards.
\begin{enumerate}[label=(\alph*)]
    \item How many points are there in the sample space?
    \item Calculate the probability that the card is an ace of spades.
    \item Calculate the probability that the card is (i) an ace and (ii) black card.
\end{enumerate}
\solution
%Let $X$ be an bernoulli rv defined as in \tabref{tab:exemplar/11/16/3/26}.  Then, 
\begin{equation}
    p =
        \frac{4}{11} 
\end{equation}
\begin{table}[H]
	\centering
	\input{exemplar/11/16/3/26/tables/Table2.tex}
	\caption{}
        \label{tab:exemplar/11/16/3/26}
\end{table}

\item The probability that a non leap year selected at random will contain 53 sundays.
\\
\solution
%\begin{table}[H]
	\centering
\begin{tabular}{|c|c|c|}
\hline
Random variable &Value &Definition\\ \hline
\multirow{3}{*}{X} &0 &Slips of Rs 1\\
&1 &Slips of Rs 5\\
&2 &Slips of Rs 13\\ \hline
\multirow{2}{*}{Y} &0 &Box A\\
&1 &Box B\\\hline
\end{tabular}
\caption{}
\label{tab:Distribution}
\end{table}
See \tabref{tab:Distribution}.
\begin{align}
p_{Y}\brak{k}= \begin{cases} 
      \frac{1}{3} & {k=0} \\
      \frac{2}{3 }& {k=1} 
   \end{cases}
   \\
p_{Y|X}\brak{0|0} = \frac{19}{25}\, 
p_{Y|X}\brak{0|1} = \frac{6}{25}\,
p_{Y|X}\brak{1|0} = \frac{45}{50}\,
p_{Y|X}\brak{1|2} = \frac{5}{50}
\end{align}
The desired probability is the probability that a slip drawn at random is marked other than Rs 1,
\begin{align}
&=1-p_X\brak{0}\\
&= p_X(1) + p_X(2)
\end{align}
Using Bayes theorem,
\begin{align}
&= p_Y\brak{0} \times \pr{Y=0 | X=1} + p_Y\brak{1} \times \pr{Y=1|X=2}\\
&=\frac{1}{3} \times \frac{6}{25} + \frac{2}{3} \times \frac{5}{50}\\
&=\frac{11}{75}
\end{align}

\newpage

%\tableofcontents

\bigskip

\renewcommand{\thefigure}{\theenumi}
\renewcommand{\thetable}{\theenumi}
%\renewcommand{\theequation}{\theenumi}

%\begin{abstract}
%%\boldmath
%In this letter, an algorithm for evaluating the exact analytical bit error rate  (BER)  for the piecewise linear (PL) combiner for  multiple relays is presented. Previous results were available only for upto three relays. The algorithm is unique in the sense that  the actual mathematical expressions, that are prohibitively large, need not be explicitly obtained. The diversity gain due to multiple relays is shown through plots of the analytical BER, well supported by simulations. 
%
%\end{abstract}
% IEEEtran.cls defaults to using nonbold math in the Abstract.
% This preserves the distinction between vectors and scalars. However,
% if the journal you are submitting to favors bold math in the abstract,
% then you can use LaTeX's standard command \boldmath at the very start
% of the abstract to achieve this. Many IEEE journals frown on math
% in the abstract anyway.

% Note that keywords are not normally used for peerreview papers.
%\begin{IEEEkeywords}
%Cooperative diversity, decode and forward, piecewise linear
%\end{IEEEkeywords}



% For peer review papers, you can put extra information on the cover
% page as needed:
% \ifCLASSOPTIONpeerreview
% \begin{center} \bfseries EDICS Category: 3-BBND \end{center}
% \fi
%
% For peerreview papers, this IEEEtran command inserts a page break and
% creates the second title. It will be ignored for other modes.
%\IEEEpeerreviewmaketitle




\item One of the four persons John, Rita, Aslam or Gurpreet will be promoted next
month. Consequently the sample space consists of four elementary outcomes
S = {John promoted, Rita promoted, Aslam promoted, Gurpreet promoted}
You are told that the chances of John’s promotion is same as that of Gurpreet,
Rita’s chances of promotion are twice as likely as Johns. Aslam’s chances are
four times that of John.
\begin{enumerate}
	\item Determine
	\begin{enumerate}
		\item P (John promoted)
		\item P (Rita promoted)
		\item P (Aslam promoted)
		\item P (Gurpreet promoted)
	\end{enumerate}
	\item If A = {John promoted or Gurpreet promoted}, find P (A).
\end{enumerate}
\solution
%\begin{table}[H]
	\centering
\begin{tabular}{|c|c|c|}
\hline
Random variable &Value &Definition\\ \hline
\multirow{3}{*}{X} &0 &Slips of Rs 1\\
&1 &Slips of Rs 5\\
&2 &Slips of Rs 13\\ \hline
\multirow{2}{*}{Y} &0 &Box A\\
&1 &Box B\\\hline
\end{tabular}
\caption{}
\label{tab:Distribution}
\end{table}
See \tabref{tab:Distribution}.
\begin{align}
p_{Y}\brak{k}= \begin{cases} 
      \frac{1}{3} & {k=0} \\
      \frac{2}{3 }& {k=1} 
   \end{cases}
   \\
p_{Y|X}\brak{0|0} = \frac{19}{25}\, 
p_{Y|X}\brak{0|1} = \frac{6}{25}\,
p_{Y|X}\brak{1|0} = \frac{45}{50}\,
p_{Y|X}\brak{1|2} = \frac{5}{50}
\end{align}
The desired probability is the probability that a slip drawn at random is marked other than Rs 1,
\begin{align}
&=1-p_X\brak{0}\\
&= p_X(1) + p_X(2)
\end{align}
Using Bayes theorem,
\begin{align}
&= p_Y\brak{0} \times \pr{Y=0 | X=1} + p_Y\brak{1} \times \pr{Y=1|X=2}\\
&=\frac{1}{3} \times \frac{6}{25} + \frac{2}{3} \times \frac{5}{50}\\
&=\frac{11}{75}
\end{align}

\newpage

%\tableofcontents

\bigskip

\renewcommand{\thefigure}{\theenumi}
\renewcommand{\thetable}{\theenumi}
%\renewcommand{\theequation}{\theenumi}

%\begin{abstract}
%%\boldmath
%In this letter, an algorithm for evaluating the exact analytical bit error rate  (BER)  for the piecewise linear (PL) combiner for  multiple relays is presented. Previous results were available only for upto three relays. The algorithm is unique in the sense that  the actual mathematical expressions, that are prohibitively large, need not be explicitly obtained. The diversity gain due to multiple relays is shown through plots of the analytical BER, well supported by simulations. 
%
%\end{abstract}
% IEEEtran.cls defaults to using nonbold math in the Abstract.
% This preserves the distinction between vectors and scalars. However,
% if the journal you are submitting to favors bold math in the abstract,
% then you can use LaTeX's standard command \boldmath at the very start
% of the abstract to achieve this. Many IEEE journals frown on math
% in the abstract anyway.

% Note that keywords are not normally used for peerreview papers.
%\begin{IEEEkeywords}
%Cooperative diversity, decode and forward, piecewise linear
%\end{IEEEkeywords}



% For peer review papers, you can put extra information on the cover
% page as needed:
% \ifCLASSOPTIONpeerreview
% \begin{center} \bfseries EDICS Category: 3-BBND \end{center}
% \fi
%
% For peerreview papers, this IEEEtran command inserts a page break and
% creates the second title. It will be ignored for other modes.
%\IEEEpeerreviewmaketitle




\item A card is drawn from a deck of 52 cards. Find the probability of getting a king or a heart or a red card.\\
\solution
%\begin{table}[H]
	\centering
\begin{tabular}{|c|c|c|}
\hline
Random variable &Value &Definition\\ \hline
\multirow{3}{*}{X} &0 &Slips of Rs 1\\
&1 &Slips of Rs 5\\
&2 &Slips of Rs 13\\ \hline
\multirow{2}{*}{Y} &0 &Box A\\
&1 &Box B\\\hline
\end{tabular}
\caption{}
\label{tab:Distribution}
\end{table}
See \tabref{tab:Distribution}.
\begin{align}
p_{Y}\brak{k}= \begin{cases} 
      \frac{1}{3} & {k=0} \\
      \frac{2}{3 }& {k=1} 
   \end{cases}
   \\
p_{Y|X}\brak{0|0} = \frac{19}{25}\, 
p_{Y|X}\brak{0|1} = \frac{6}{25}\,
p_{Y|X}\brak{1|0} = \frac{45}{50}\,
p_{Y|X}\brak{1|2} = \frac{5}{50}
\end{align}
The desired probability is the probability that a slip drawn at random is marked other than Rs 1,
\begin{align}
&=1-p_X\brak{0}\\
&= p_X(1) + p_X(2)
\end{align}
Using Bayes theorem,
\begin{align}
&= p_Y\brak{0} \times \pr{Y=0 | X=1} + p_Y\brak{1} \times \pr{Y=1|X=2}\\
&=\frac{1}{3} \times \frac{6}{25} + \frac{2}{3} \times \frac{5}{50}\\
&=\frac{11}{75}
\end{align}

\newpage

%\tableofcontents

\bigskip

\renewcommand{\thefigure}{\theenumi}
\renewcommand{\thetable}{\theenumi}
%\renewcommand{\theequation}{\theenumi}

%\begin{abstract}
%%\boldmath
%In this letter, an algorithm for evaluating the exact analytical bit error rate  (BER)  for the piecewise linear (PL) combiner for  multiple relays is presented. Previous results were available only for upto three relays. The algorithm is unique in the sense that  the actual mathematical expressions, that are prohibitively large, need not be explicitly obtained. The diversity gain due to multiple relays is shown through plots of the analytical BER, well supported by simulations. 
%
%\end{abstract}
% IEEEtran.cls defaults to using nonbold math in the Abstract.
% This preserves the distinction between vectors and scalars. However,
% if the journal you are submitting to favors bold math in the abstract,
% then you can use LaTeX's standard command \boldmath at the very start
% of the abstract to achieve this. Many IEEE journals frown on math
% in the abstract anyway.

% Note that keywords are not normally used for peerreview papers.
%\begin{IEEEkeywords}
%Cooperative diversity, decode and forward, piecewise linear
%\end{IEEEkeywords}



% For peer review papers, you can put extra information on the cover
% page as needed:
% \ifCLASSOPTIONpeerreview
% \begin{center} \bfseries EDICS Category: 3-BBND \end{center}
% \fi
%
% For peerreview papers, this IEEEtran command inserts a page break and
% creates the second title. It will be ignored for other modes.
%\IEEEpeerreviewmaketitle




\item The probability that a student will pass his examination is 0.73, the probability of
the student getting a compartment is 0.13, and the probability that the student will
either pass or get compartment is 0.96. State True or False.\\
\solution
%\begin{table}[H]
	\centering
\begin{tabular}{|c|c|c|}
\hline
Random variable &Value &Definition\\ \hline
\multirow{3}{*}{X} &0 &Slips of Rs 1\\
&1 &Slips of Rs 5\\
&2 &Slips of Rs 13\\ \hline
\multirow{2}{*}{Y} &0 &Box A\\
&1 &Box B\\\hline
\end{tabular}
\caption{}
\label{tab:Distribution}
\end{table}
See \tabref{tab:Distribution}.
\begin{align}
p_{Y}\brak{k}= \begin{cases} 
      \frac{1}{3} & {k=0} \\
      \frac{2}{3 }& {k=1} 
   \end{cases}
   \\
p_{Y|X}\brak{0|0} = \frac{19}{25}\, 
p_{Y|X}\brak{0|1} = \frac{6}{25}\,
p_{Y|X}\brak{1|0} = \frac{45}{50}\,
p_{Y|X}\brak{1|2} = \frac{5}{50}
\end{align}
The desired probability is the probability that a slip drawn at random is marked other than Rs 1,
\begin{align}
&=1-p_X\brak{0}\\
&= p_X(1) + p_X(2)
\end{align}
Using Bayes theorem,
\begin{align}
&= p_Y\brak{0} \times \pr{Y=0 | X=1} + p_Y\brak{1} \times \pr{Y=1|X=2}\\
&=\frac{1}{3} \times \frac{6}{25} + \frac{2}{3} \times \frac{5}{50}\\
&=\frac{11}{75}
\end{align}

\newpage

%\tableofcontents

\bigskip

\renewcommand{\thefigure}{\theenumi}
\renewcommand{\thetable}{\theenumi}
%\renewcommand{\theequation}{\theenumi}

%\begin{abstract}
%%\boldmath
%In this letter, an algorithm for evaluating the exact analytical bit error rate  (BER)  for the piecewise linear (PL) combiner for  multiple relays is presented. Previous results were available only for upto three relays. The algorithm is unique in the sense that  the actual mathematical expressions, that are prohibitively large, need not be explicitly obtained. The diversity gain due to multiple relays is shown through plots of the analytical BER, well supported by simulations. 
%
%\end{abstract}
% IEEEtran.cls defaults to using nonbold math in the Abstract.
% This preserves the distinction between vectors and scalars. However,
% if the journal you are submitting to favors bold math in the abstract,
% then you can use LaTeX's standard command \boldmath at the very start
% of the abstract to achieve this. Many IEEE journals frown on math
% in the abstract anyway.

% Note that keywords are not normally used for peerreview papers.
%\begin{IEEEkeywords}
%Cooperative diversity, decode and forward, piecewise linear
%\end{IEEEkeywords}



% For peer review papers, you can put extra information on the cover
% page as needed:
% \ifCLASSOPTIONpeerreview
% \begin{center} \bfseries EDICS Category: 3-BBND \end{center}
% \fi
%
% For peerreview papers, this IEEEtran command inserts a page break and
% creates the second title. It will be ignored for other modes.
%\IEEEpeerreviewmaketitle




\item A card is selected from a pack of 52 cards\\
\begin{enumerate}[label=(\alph*)]
\item How many points are there in the sample space?
\item Calculate the probability that the cards is an ace of spades.
\item Calculate the probability that the card is (i) an ace (ii)black card.\\
\end{enumerate}
%\input{ncert/11/16/3/4_1/Prob_4.tex}
\item In a non-leap year, the probability of having 53 tuesdays or 53 wednesdays is\\
\solution
%A non-leap year has a total of 365 days, and a week has 7 days.\\
So it can be expressed as 
\begin{align}
365\text{days} &=52\times 7+1 \text{day}
\end{align}
$\implies$ 52 tuesdays or wednesdays\\
Random variable X denotes the days of a week
\begin{align}
p_X\brak{k}&=\frac{1}{7}; \quad \brak{1<k<7}
\end{align}
So the probability of extra day being tuesday or wednesday is
\begin{align}
p_X\brak{3}+p_X\brak{4}&=\frac{1}{7}+\frac{1}{7}=\frac{2}{7}
\end{align}



\item There are 1000 sealed envelopes in a box, 10 of them contain a cash prize of
Rs 100 each, 100 of them contain a cash prize of Rs 50 each and 200 of them
contain a cash prize of Rs 10 each and rest do not contain any cash prize. If they
are well shuffled and an envelope is picked up out, what is the probability that it
contains no cash prize?\\
\solution
%\begin{table}[H]
	\centering
\begin{tabular}{|c|c|c|}
\hline
Random variable &Value &Definition\\ \hline
\multirow{3}{*}{X} &0 &Slips of Rs 1\\
&1 &Slips of Rs 5\\
&2 &Slips of Rs 13\\ \hline
\multirow{2}{*}{Y} &0 &Box A\\
&1 &Box B\\\hline
\end{tabular}
\caption{}
\label{tab:Distribution}
\end{table}
See \tabref{tab:Distribution}.
\begin{align}
p_{Y}\brak{k}= \begin{cases} 
      \frac{1}{3} & {k=0} \\
      \frac{2}{3 }& {k=1} 
   \end{cases}
   \\
p_{Y|X}\brak{0|0} = \frac{19}{25}\, 
p_{Y|X}\brak{0|1} = \frac{6}{25}\,
p_{Y|X}\brak{1|0} = \frac{45}{50}\,
p_{Y|X}\brak{1|2} = \frac{5}{50}
\end{align}
The desired probability is the probability that a slip drawn at random is marked other than Rs 1,
\begin{align}
&=1-p_X\brak{0}\\
&= p_X(1) + p_X(2)
\end{align}
Using Bayes theorem,
\begin{align}
&= p_Y\brak{0} \times \pr{Y=0 | X=1} + p_Y\brak{1} \times \pr{Y=1|X=2}\\
&=\frac{1}{3} \times \frac{6}{25} + \frac{2}{3} \times \frac{5}{50}\\
&=\frac{11}{75}
\end{align}

\newpage

%\tableofcontents

\bigskip

\renewcommand{\thefigure}{\theenumi}
\renewcommand{\thetable}{\theenumi}
%\renewcommand{\theequation}{\theenumi}

%\begin{abstract}
%%\boldmath
%In this letter, an algorithm for evaluating the exact analytical bit error rate  (BER)  for the piecewise linear (PL) combiner for  multiple relays is presented. Previous results were available only for upto three relays. The algorithm is unique in the sense that  the actual mathematical expressions, that are prohibitively large, need not be explicitly obtained. The diversity gain due to multiple relays is shown through plots of the analytical BER, well supported by simulations. 
%
%\end{abstract}
% IEEEtran.cls defaults to using nonbold math in the Abstract.
% This preserves the distinction between vectors and scalars. However,
% if the journal you are submitting to favors bold math in the abstract,
% then you can use LaTeX's standard command \boldmath at the very start
% of the abstract to achieve this. Many IEEE journals frown on math
% in the abstract anyway.

% Note that keywords are not normally used for peerreview papers.
%\begin{IEEEkeywords}
%Cooperative diversity, decode and forward, piecewise linear
%\end{IEEEkeywords}



% For peer review papers, you can put extra information on the cover
% page as needed:
% \ifCLASSOPTIONpeerreview
% \begin{center} \bfseries EDICS Category: 3-BBND \end{center}
% \fi
%
% For peerreview papers, this IEEEtran command inserts a page break and
% creates the second title. It will be ignored for other modes.
%\IEEEpeerreviewmaketitle




\item 
A die is thrown and a card is selected at random from a deck of 52 playing cards. The probability of getting an even number on the die and a spade card.\\
\solution
%\begin{table}[H]
	\centering
\begin{tabular}{|c|c|c|}
\hline
Random variable &Value &Definition\\ \hline
\multirow{3}{*}{X} &0 &Slips of Rs 1\\
&1 &Slips of Rs 5\\
&2 &Slips of Rs 13\\ \hline
\multirow{2}{*}{Y} &0 &Box A\\
&1 &Box B\\\hline
\end{tabular}
\caption{}
\label{tab:Distribution}
\end{table}
See \tabref{tab:Distribution}.
\begin{align}
p_{Y}\brak{k}= \begin{cases} 
      \frac{1}{3} & {k=0} \\
      \frac{2}{3 }& {k=1} 
   \end{cases}
   \\
p_{Y|X}\brak{0|0} = \frac{19}{25}\, 
p_{Y|X}\brak{0|1} = \frac{6}{25}\,
p_{Y|X}\brak{1|0} = \frac{45}{50}\,
p_{Y|X}\brak{1|2} = \frac{5}{50}
\end{align}
The desired probability is the probability that a slip drawn at random is marked other than Rs 1,
\begin{align}
&=1-p_X\brak{0}\\
&= p_X(1) + p_X(2)
\end{align}
Using Bayes theorem,
\begin{align}
&= p_Y\brak{0} \times \pr{Y=0 | X=1} + p_Y\brak{1} \times \pr{Y=1|X=2}\\
&=\frac{1}{3} \times \frac{6}{25} + \frac{2}{3} \times \frac{5}{50}\\
&=\frac{11}{75}
\end{align}

\newpage

%\tableofcontents

\bigskip

\renewcommand{\thefigure}{\theenumi}
\renewcommand{\thetable}{\theenumi}
%\renewcommand{\theequation}{\theenumi}

%\begin{abstract}
%%\boldmath
%In this letter, an algorithm for evaluating the exact analytical bit error rate  (BER)  for the piecewise linear (PL) combiner for  multiple relays is presented. Previous results were available only for upto three relays. The algorithm is unique in the sense that  the actual mathematical expressions, that are prohibitively large, need not be explicitly obtained. The diversity gain due to multiple relays is shown through plots of the analytical BER, well supported by simulations. 
%
%\end{abstract}
% IEEEtran.cls defaults to using nonbold math in the Abstract.
% This preserves the distinction between vectors and scalars. However,
% if the journal you are submitting to favors bold math in the abstract,
% then you can use LaTeX's standard command \boldmath at the very start
% of the abstract to achieve this. Many IEEE journals frown on math
% in the abstract anyway.

% Note that keywords are not normally used for peerreview papers.
%\begin{IEEEkeywords}
%Cooperative diversity, decode and forward, piecewise linear
%\end{IEEEkeywords}



% For peer review papers, you can put extra information on the cover
% page as needed:
% \ifCLASSOPTIONpeerreview
% \begin{center} \bfseries EDICS Category: 3-BBND \end{center}
% \fi
%
% For peerreview papers, this IEEEtran command inserts a page break and
% creates the second title. It will be ignored for other modes.
%\IEEEpeerreviewmaketitle




\item
If 4-digit numbers greater than 5,000 are randomly formed from the digits 0, 1, 3, 5, and 7, what is the probability of forming a number divisible by 5 when:
\begin{enumerate}
    \item The digits are repeated?
    \item The repetition of digits is not allowed?
\end{enumerate}
\solution
%\begin{table}[H]
	\centering
\begin{tabular}{|c|c|c|}
\hline
Random variable &Value &Definition\\ \hline
\multirow{3}{*}{X} &0 &Slips of Rs 1\\
&1 &Slips of Rs 5\\
&2 &Slips of Rs 13\\ \hline
\multirow{2}{*}{Y} &0 &Box A\\
&1 &Box B\\\hline
\end{tabular}
\caption{}
\label{tab:Distribution}
\end{table}
See \tabref{tab:Distribution}.
\begin{align}
p_{Y}\brak{k}= \begin{cases} 
      \frac{1}{3} & {k=0} \\
      \frac{2}{3 }& {k=1} 
   \end{cases}
   \\
p_{Y|X}\brak{0|0} = \frac{19}{25}\, 
p_{Y|X}\brak{0|1} = \frac{6}{25}\,
p_{Y|X}\brak{1|0} = \frac{45}{50}\,
p_{Y|X}\brak{1|2} = \frac{5}{50}
\end{align}
The desired probability is the probability that a slip drawn at random is marked other than Rs 1,
\begin{align}
&=1-p_X\brak{0}\\
&= p_X(1) + p_X(2)
\end{align}
Using Bayes theorem,
\begin{align}
&= p_Y\brak{0} \times \pr{Y=0 | X=1} + p_Y\brak{1} \times \pr{Y=1|X=2}\\
&=\frac{1}{3} \times \frac{6}{25} + \frac{2}{3} \times \frac{5}{50}\\
&=\frac{11}{75}
\end{align}

\newpage

%\tableofcontents

\bigskip

\renewcommand{\thefigure}{\theenumi}
\renewcommand{\thetable}{\theenumi}
%\renewcommand{\theequation}{\theenumi}

%\begin{abstract}
%%\boldmath
%In this letter, an algorithm for evaluating the exact analytical bit error rate  (BER)  for the piecewise linear (PL) combiner for  multiple relays is presented. Previous results were available only for upto three relays. The algorithm is unique in the sense that  the actual mathematical expressions, that are prohibitively large, need not be explicitly obtained. The diversity gain due to multiple relays is shown through plots of the analytical BER, well supported by simulations. 
%
%\end{abstract}
% IEEEtran.cls defaults to using nonbold math in the Abstract.
% This preserves the distinction between vectors and scalars. However,
% if the journal you are submitting to favors bold math in the abstract,
% then you can use LaTeX's standard command \boldmath at the very start
% of the abstract to achieve this. Many IEEE journals frown on math
% in the abstract anyway.

% Note that keywords are not normally used for peerreview papers.
%\begin{IEEEkeywords}
%Cooperative diversity, decode and forward, piecewise linear
%\end{IEEEkeywords}



% For peer review papers, you can put extra information on the cover
% page as needed:
% \ifCLASSOPTIONpeerreview
% \begin{center} \bfseries EDICS Category: 3-BBND \end{center}
% \fi
%
% For peerreview papers, this IEEEtran command inserts a page break and
% creates the second title. It will be ignored for other modes.
%\IEEEpeerreviewmaketitle




\item Consider the probability space $\brak{\Omega, \mathcal{G}, P}$ where $\Omega = [0,2]$ and $\mathcal{G} = \cbrak{\phi, \Omega, [0,1], (1,2]}$. Let $X$ and $Y$ be two functions on $\Omega$ defined as
\begin{align*}
    X(\omega) = 
    \begin{cases}
        1 & \text{if }\omega \in [0, 1]\\
        2 & \text{if }\omega \in (1, 2]
    \end{cases}
\end{align*}
and
\begin{align*}
    Y(\omega) = 
    \begin{cases}
        2 & \text{if }\omega \in [0, 1.5]\\
        3 & \text{if }\omega \in (1.5, 2].
    \end{cases}
\end{align*}
Then which one of the following statements is true?
\begin{enumerate}
    \item [(A)] $X$ is a random variable with respect to $\mathcal{G}$, but $Y$ is not a random variable with respect to $\mathcal{G}$.
    \item [(B)] $Y$ is a random variable with respect to $\mathcal{G}$, but $X$ is not a random variable with respect to $\mathcal{G}$.
    \item [(C)] Neither $X$ nor $Y$ is a random variable with respect to $\mathcal{G}$.
    \item [(D)] Both $X$ and $Y$ are random variables with respect to $\mathcal{G}$.
\end{enumerate} \hfill (GATE ST 2023)\\
\solution
%\begin{table}[H]
	\centering
\begin{tabular}{|c|c|c|}
\hline
Random variable &Value &Definition\\ \hline
\multirow{3}{*}{X} &0 &Slips of Rs 1\\
&1 &Slips of Rs 5\\
&2 &Slips of Rs 13\\ \hline
\multirow{2}{*}{Y} &0 &Box A\\
&1 &Box B\\\hline
\end{tabular}
\caption{}
\label{tab:Distribution}
\end{table}
See \tabref{tab:Distribution}.
\begin{align}
p_{Y}\brak{k}= \begin{cases} 
      \frac{1}{3} & {k=0} \\
      \frac{2}{3 }& {k=1} 
   \end{cases}
   \\
p_{Y|X}\brak{0|0} = \frac{19}{25}\, 
p_{Y|X}\brak{0|1} = \frac{6}{25}\,
p_{Y|X}\brak{1|0} = \frac{45}{50}\,
p_{Y|X}\brak{1|2} = \frac{5}{50}
\end{align}
The desired probability is the probability that a slip drawn at random is marked other than Rs 1,
\begin{align}
&=1-p_X\brak{0}\\
&= p_X(1) + p_X(2)
\end{align}
Using Bayes theorem,
\begin{align}
&= p_Y\brak{0} \times \pr{Y=0 | X=1} + p_Y\brak{1} \times \pr{Y=1|X=2}\\
&=\frac{1}{3} \times \frac{6}{25} + \frac{2}{3} \times \frac{5}{50}\\
&=\frac{11}{75}
\end{align}

\newpage

%\tableofcontents

\bigskip

\renewcommand{\thefigure}{\theenumi}
\renewcommand{\thetable}{\theenumi}
%\renewcommand{\theequation}{\theenumi}

%\begin{abstract}
%%\boldmath
%In this letter, an algorithm for evaluating the exact analytical bit error rate  (BER)  for the piecewise linear (PL) combiner for  multiple relays is presented. Previous results were available only for upto three relays. The algorithm is unique in the sense that  the actual mathematical expressions, that are prohibitively large, need not be explicitly obtained. The diversity gain due to multiple relays is shown through plots of the analytical BER, well supported by simulations. 
%
%\end{abstract}
% IEEEtran.cls defaults to using nonbold math in the Abstract.
% This preserves the distinction between vectors and scalars. However,
% if the journal you are submitting to favors bold math in the abstract,
% then you can use LaTeX's standard command \boldmath at the very start
% of the abstract to achieve this. Many IEEE journals frown on math
% in the abstract anyway.

% Note that keywords are not normally used for peerreview papers.
%\begin{IEEEkeywords}
%Cooperative diversity, decode and forward, piecewise linear
%\end{IEEEkeywords}



% For peer review papers, you can put extra information on the cover
% page as needed:
% \ifCLASSOPTIONpeerreview
% \begin{center} \bfseries EDICS Category: 3-BBND \end{center}
% \fi
%
% For peerreview papers, this IEEEtran command inserts a page break and
% creates the second title. It will be ignored for other modes.
%\IEEEpeerreviewmaketitle




	\item  A die is loaded in such a way that each odd number is twice as likely to occur as
each even number. Find $P(G)$, where $G$ is the event that a number greater than
3 occurs on a single roll of the die.
\\
\solution
		%\begin{table}[H]
	\centering
\begin{tabular}{|c|c|c|}
\hline
Random variable &Value &Definition\\ \hline
\multirow{3}{*}{X} &0 &Slips of Rs 1\\
&1 &Slips of Rs 5\\
&2 &Slips of Rs 13\\ \hline
\multirow{2}{*}{Y} &0 &Box A\\
&1 &Box B\\\hline
\end{tabular}
\caption{}
\label{tab:Distribution}
\end{table}
See \tabref{tab:Distribution}.
\begin{align}
p_{Y}\brak{k}= \begin{cases} 
      \frac{1}{3} & {k=0} \\
      \frac{2}{3 }& {k=1} 
   \end{cases}
   \\
p_{Y|X}\brak{0|0} = \frac{19}{25}\, 
p_{Y|X}\brak{0|1} = \frac{6}{25}\,
p_{Y|X}\brak{1|0} = \frac{45}{50}\,
p_{Y|X}\brak{1|2} = \frac{5}{50}
\end{align}
The desired probability is the probability that a slip drawn at random is marked other than Rs 1,
\begin{align}
&=1-p_X\brak{0}\\
&= p_X(1) + p_X(2)
\end{align}
Using Bayes theorem,
\begin{align}
&= p_Y\brak{0} \times \pr{Y=0 | X=1} + p_Y\brak{1} \times \pr{Y=1|X=2}\\
&=\frac{1}{3} \times \frac{6}{25} + \frac{2}{3} \times \frac{5}{50}\\
&=\frac{11}{75}
\end{align}

\newpage

%\tableofcontents

\bigskip

\renewcommand{\thefigure}{\theenumi}
\renewcommand{\thetable}{\theenumi}
%\renewcommand{\theequation}{\theenumi}

%\begin{abstract}
%%\boldmath
%In this letter, an algorithm for evaluating the exact analytical bit error rate  (BER)  for the piecewise linear (PL) combiner for  multiple relays is presented. Previous results were available only for upto three relays. The algorithm is unique in the sense that  the actual mathematical expressions, that are prohibitively large, need not be explicitly obtained. The diversity gain due to multiple relays is shown through plots of the analytical BER, well supported by simulations. 
%
%\end{abstract}
% IEEEtran.cls defaults to using nonbold math in the Abstract.
% This preserves the distinction between vectors and scalars. However,
% if the journal you are submitting to favors bold math in the abstract,
% then you can use LaTeX's standard command \boldmath at the very start
% of the abstract to achieve this. Many IEEE journals frown on math
% in the abstract anyway.

% Note that keywords are not normally used for peerreview papers.
%\begin{IEEEkeywords}
%Cooperative diversity, decode and forward, piecewise linear
%\end{IEEEkeywords}



% For peer review papers, you can put extra information on the cover
% page as needed:
% \ifCLASSOPTIONpeerreview
% \begin{center} \bfseries EDICS Category: 3-BBND \end{center}
% \fi
%
% For peerreview papers, this IEEEtran command inserts a page break and
% creates the second title. It will be ignored for other modes.
%\IEEEpeerreviewmaketitle




	\item All the jacks, queens and kings are removed from a deck of 52 playing cards. The remaining cards are well shuffled and then one card is drawn at random. Giving ace a value 1 similar value for other cards, find the probability that the card has a value 
		\begin{enumerate}
			\item 7
			\item greater than 7
			\item less than 7
		\end{enumerate}
		%Number of cards left after removing all jacks, queens and kings 
\begin{align}
N	= 52 - 4\times 3
	= 40
\end{align}
%\begin{table}[H]
%\def\arraystretch{1.2}
%\begin{tabular}{|c|c|c|}
%\hline
%	\textbf{Parameter} &\textbf{Value} &\textbf{Description}\\ \hline
%	$X$ &1-10 &Represents the value of the card picked \\ \hline
%\end{tabular}
%\end{table}
Let $1 \le X \le 10$ be the value of the card picked.  Then,
\begin{align}
	p_X(k) &= \Pr(X=k)\ \forall\ 1 \leq k \leq 10\\
	&= \frac{4\times 1}{40}\\
	&= \frac{1}{10}\\
	\therefore p_X(k) &= 
	\begin{cases}
		\frac{1}{10} & 1 \leq k \leq 10\\
		0 & \text{otherwise}
	\end{cases}
\end{align}
and
\begin{align}
	F_{X}(k) &= \sum_{m=0}^{k}p_{X}(m) \quad 1 \leq k \leq 10\\
	&= \frac{k}{10}\\
	\therefore F_{X}(k) &= 
	\begin{cases}
		0 & k \leq 0\\
		\frac{k}{10} & 1\leq k \leq 10\\
		1 & k > 10 
	\end{cases}
\end{align}
\begin{enumerate}
	\item Probability that card has value equal to 7 is
		\begin{align}
			 p_{X}(7)
			= \frac{1}{10}
		\end{align}
	\item Probability that card has value greater than 7 is
		\begin{align}
			1 - F_X(7)
			&= 1 - \frac{7}{10}
			\\
			&= \frac{3}{10}
		\end{align}
	\item Probability that card has value less than 7 is
		\begin{align}
			 F_{X}(6)
			=\frac{6}{10}
		\end{align}
\end{enumerate}

  \item A Lot consists of 48 mobile phones of which 42 are good, 3 have only minor defects and 3 have major defects.Varnika will buy a phone if it is good but the trader will only buy a mobile if it has no major defects. One phone is selected at random from the lot. What is the probability that it is
\begin{enumerate}
	\item acceptable to Varnika?
            \item acceptable to the trader?
\end{enumerate}
\solution
	%\begin{table}[H]
	\centering
\begin{tabular}{|c|c|c|}
\hline
Random variable &Value &Definition\\ \hline
\multirow{3}{*}{X} &0 &Slips of Rs 1\\
&1 &Slips of Rs 5\\
&2 &Slips of Rs 13\\ \hline
\multirow{2}{*}{Y} &0 &Box A\\
&1 &Box B\\\hline
\end{tabular}
\caption{}
\label{tab:Distribution}
\end{table}
See \tabref{tab:Distribution}.
\begin{align}
p_{Y}\brak{k}= \begin{cases} 
      \frac{1}{3} & {k=0} \\
      \frac{2}{3 }& {k=1} 
   \end{cases}
   \\
p_{Y|X}\brak{0|0} = \frac{19}{25}\, 
p_{Y|X}\brak{0|1} = \frac{6}{25}\,
p_{Y|X}\brak{1|0} = \frac{45}{50}\,
p_{Y|X}\brak{1|2} = \frac{5}{50}
\end{align}
The desired probability is the probability that a slip drawn at random is marked other than Rs 1,
\begin{align}
&=1-p_X\brak{0}\\
&= p_X(1) + p_X(2)
\end{align}
Using Bayes theorem,
\begin{align}
&= p_Y\brak{0} \times \pr{Y=0 | X=1} + p_Y\brak{1} \times \pr{Y=1|X=2}\\
&=\frac{1}{3} \times \frac{6}{25} + \frac{2}{3} \times \frac{5}{50}\\
&=\frac{11}{75}
\end{align}

\newpage

%\tableofcontents

\bigskip

\renewcommand{\thefigure}{\theenumi}
\renewcommand{\thetable}{\theenumi}
%\renewcommand{\theequation}{\theenumi}

%\begin{abstract}
%%\boldmath
%In this letter, an algorithm for evaluating the exact analytical bit error rate  (BER)  for the piecewise linear (PL) combiner for  multiple relays is presented. Previous results were available only for upto three relays. The algorithm is unique in the sense that  the actual mathematical expressions, that are prohibitively large, need not be explicitly obtained. The diversity gain due to multiple relays is shown through plots of the analytical BER, well supported by simulations. 
%
%\end{abstract}
% IEEEtran.cls defaults to using nonbold math in the Abstract.
% This preserves the distinction between vectors and scalars. However,
% if the journal you are submitting to favors bold math in the abstract,
% then you can use LaTeX's standard command \boldmath at the very start
% of the abstract to achieve this. Many IEEE journals frown on math
% in the abstract anyway.

% Note that keywords are not normally used for peerreview papers.
%\begin{IEEEkeywords}
%Cooperative diversity, decode and forward, piecewise linear
%\end{IEEEkeywords}



% For peer review papers, you can put extra information on the cover
% page as needed:
% \ifCLASSOPTIONpeerreview
% \begin{center} \bfseries EDICS Category: 3-BBND \end{center}
% \fi
%
% For peerreview papers, this IEEEtran command inserts a page break and
% creates the second title. It will be ignored for other modes.
%\IEEEpeerreviewmaketitle




 \item A student says that if you throw a die, it will show up 1 or not 1. Therefore, the probability of getting 1 and the probability of getting 'not 1' each is equal to $\frac{1}{2}$. Is this correct? Give reasons.\\
 \solution
        %\begin{table}[H]
	\centering
\begin{tabular}{|c|c|c|}
\hline
Random variable &Value &Definition\\ \hline
\multirow{3}{*}{X} &0 &Slips of Rs 1\\
&1 &Slips of Rs 5\\
&2 &Slips of Rs 13\\ \hline
\multirow{2}{*}{Y} &0 &Box A\\
&1 &Box B\\\hline
\end{tabular}
\caption{}
\label{tab:Distribution}
\end{table}
See \tabref{tab:Distribution}.
\begin{align}
p_{Y}\brak{k}= \begin{cases} 
      \frac{1}{3} & {k=0} \\
      \frac{2}{3 }& {k=1} 
   \end{cases}
   \\
p_{Y|X}\brak{0|0} = \frac{19}{25}\, 
p_{Y|X}\brak{0|1} = \frac{6}{25}\,
p_{Y|X}\brak{1|0} = \frac{45}{50}\,
p_{Y|X}\brak{1|2} = \frac{5}{50}
\end{align}
The desired probability is the probability that a slip drawn at random is marked other than Rs 1,
\begin{align}
&=1-p_X\brak{0}\\
&= p_X(1) + p_X(2)
\end{align}
Using Bayes theorem,
\begin{align}
&= p_Y\brak{0} \times \pr{Y=0 | X=1} + p_Y\brak{1} \times \pr{Y=1|X=2}\\
&=\frac{1}{3} \times \frac{6}{25} + \frac{2}{3} \times \frac{5}{50}\\
&=\frac{11}{75}
\end{align}

\newpage

%\tableofcontents

\bigskip

\renewcommand{\thefigure}{\theenumi}
\renewcommand{\thetable}{\theenumi}
%\renewcommand{\theequation}{\theenumi}

%\begin{abstract}
%%\boldmath
%In this letter, an algorithm for evaluating the exact analytical bit error rate  (BER)  for the piecewise linear (PL) combiner for  multiple relays is presented. Previous results were available only for upto three relays. The algorithm is unique in the sense that  the actual mathematical expressions, that are prohibitively large, need not be explicitly obtained. The diversity gain due to multiple relays is shown through plots of the analytical BER, well supported by simulations. 
%
%\end{abstract}
% IEEEtran.cls defaults to using nonbold math in the Abstract.
% This preserves the distinction between vectors and scalars. However,
% if the journal you are submitting to favors bold math in the abstract,
% then you can use LaTeX's standard command \boldmath at the very start
% of the abstract to achieve this. Many IEEE journals frown on math
% in the abstract anyway.

% Note that keywords are not normally used for peerreview papers.
%\begin{IEEEkeywords}
%Cooperative diversity, decode and forward, piecewise linear
%\end{IEEEkeywords}



% For peer review papers, you can put extra information on the cover
% page as needed:
% \ifCLASSOPTIONpeerreview
% \begin{center} \bfseries EDICS Category: 3-BBND \end{center}
% \fi
%
% For peerreview papers, this IEEEtran command inserts a page break and
% creates the second title. It will be ignored for other modes.
%\IEEEpeerreviewmaketitle




   \item Four candidates A, B, C, D have ap-
plied for the assignment to coach a school cricket
team. If A is twice as likely to be selected as B, and
B and C are given about the same chance of being
selected, while C is twice as likely to be selected
as D, what are the probabilities that
\begin{enumerate}
\item C will be selected?
\item A will not be selected?
\end{enumerate}
	%\begin{table}[H]
	\centering
\begin{tabular}{|c|c|c|}
\hline
Random variable &Value &Definition\\ \hline
\multirow{3}{*}{X} &0 &Slips of Rs 1\\
&1 &Slips of Rs 5\\
&2 &Slips of Rs 13\\ \hline
\multirow{2}{*}{Y} &0 &Box A\\
&1 &Box B\\\hline
\end{tabular}
\caption{}
\label{tab:Distribution}
\end{table}
See \tabref{tab:Distribution}.
\begin{align}
p_{Y}\brak{k}= \begin{cases} 
      \frac{1}{3} & {k=0} \\
      \frac{2}{3 }& {k=1} 
   \end{cases}
   \\
p_{Y|X}\brak{0|0} = \frac{19}{25}\, 
p_{Y|X}\brak{0|1} = \frac{6}{25}\,
p_{Y|X}\brak{1|0} = \frac{45}{50}\,
p_{Y|X}\brak{1|2} = \frac{5}{50}
\end{align}
The desired probability is the probability that a slip drawn at random is marked other than Rs 1,
\begin{align}
&=1-p_X\brak{0}\\
&= p_X(1) + p_X(2)
\end{align}
Using Bayes theorem,
\begin{align}
&= p_Y\brak{0} \times \pr{Y=0 | X=1} + p_Y\brak{1} \times \pr{Y=1|X=2}\\
&=\frac{1}{3} \times \frac{6}{25} + \frac{2}{3} \times \frac{5}{50}\\
&=\frac{11}{75}
\end{align}

\newpage

%\tableofcontents

\bigskip

\renewcommand{\thefigure}{\theenumi}
\renewcommand{\thetable}{\theenumi}
%\renewcommand{\theequation}{\theenumi}

%\begin{abstract}
%%\boldmath
%In this letter, an algorithm for evaluating the exact analytical bit error rate  (BER)  for the piecewise linear (PL) combiner for  multiple relays is presented. Previous results were available only for upto three relays. The algorithm is unique in the sense that  the actual mathematical expressions, that are prohibitively large, need not be explicitly obtained. The diversity gain due to multiple relays is shown through plots of the analytical BER, well supported by simulations. 
%
%\end{abstract}
% IEEEtran.cls defaults to using nonbold math in the Abstract.
% This preserves the distinction between vectors and scalars. However,
% if the journal you are submitting to favors bold math in the abstract,
% then you can use LaTeX's standard command \boldmath at the very start
% of the abstract to achieve this. Many IEEE journals frown on math
% in the abstract anyway.

% Note that keywords are not normally used for peerreview papers.
%\begin{IEEEkeywords}
%Cooperative diversity, decode and forward, piecewise linear
%\end{IEEEkeywords}



% For peer review papers, you can put extra information on the cover
% page as needed:
% \ifCLASSOPTIONpeerreview
% \begin{center} \bfseries EDICS Category: 3-BBND \end{center}
% \fi
%
% For peerreview papers, this IEEEtran command inserts a page break and
% creates the second title. It will be ignored for other modes.
%\IEEEpeerreviewmaketitle




 \item A bag contain 24 balls of which $x$ balls are red, $2x$ are white and $3x$ are blue. A ball is selected at random, What is the probability that it is
\begin{enumerate}[label=\alph*)]
\item not red ?
\item white ?
\end{enumerate}
%\begin{table}[H]
	\centering
\begin{tabular}{|c|c|c|}
\hline
Random variable &Value &Definition\\ \hline
\multirow{3}{*}{X} &0 &Slips of Rs 1\\
&1 &Slips of Rs 5\\
&2 &Slips of Rs 13\\ \hline
\multirow{2}{*}{Y} &0 &Box A\\
&1 &Box B\\\hline
\end{tabular}
\caption{}
\label{tab:Distribution}
\end{table}
See \tabref{tab:Distribution}.
\begin{align}
p_{Y}\brak{k}= \begin{cases} 
      \frac{1}{3} & {k=0} \\
      \frac{2}{3 }& {k=1} 
   \end{cases}
   \\
p_{Y|X}\brak{0|0} = \frac{19}{25}\, 
p_{Y|X}\brak{0|1} = \frac{6}{25}\,
p_{Y|X}\brak{1|0} = \frac{45}{50}\,
p_{Y|X}\brak{1|2} = \frac{5}{50}
\end{align}
The desired probability is the probability that a slip drawn at random is marked other than Rs 1,
\begin{align}
&=1-p_X\brak{0}\\
&= p_X(1) + p_X(2)
\end{align}
Using Bayes theorem,
\begin{align}
&= p_Y\brak{0} \times \pr{Y=0 | X=1} + p_Y\brak{1} \times \pr{Y=1|X=2}\\
&=\frac{1}{3} \times \frac{6}{25} + \frac{2}{3} \times \frac{5}{50}\\
&=\frac{11}{75}
\end{align}

\newpage

%\tableofcontents

\bigskip

\renewcommand{\thefigure}{\theenumi}
\renewcommand{\thetable}{\theenumi}
%\renewcommand{\theequation}{\theenumi}

%\begin{abstract}
%%\boldmath
%In this letter, an algorithm for evaluating the exact analytical bit error rate  (BER)  for the piecewise linear (PL) combiner for  multiple relays is presented. Previous results were available only for upto three relays. The algorithm is unique in the sense that  the actual mathematical expressions, that are prohibitively large, need not be explicitly obtained. The diversity gain due to multiple relays is shown through plots of the analytical BER, well supported by simulations. 
%
%\end{abstract}
% IEEEtran.cls defaults to using nonbold math in the Abstract.
% This preserves the distinction between vectors and scalars. However,
% if the journal you are submitting to favors bold math in the abstract,
% then you can use LaTeX's standard command \boldmath at the very start
% of the abstract to achieve this. Many IEEE journals frown on math
% in the abstract anyway.

% Note that keywords are not normally used for peerreview papers.
%\begin{IEEEkeywords}
%Cooperative diversity, decode and forward, piecewise linear
%\end{IEEEkeywords}



% For peer review papers, you can put extra information on the cover
% page as needed:
% \ifCLASSOPTIONpeerreview
% \begin{center} \bfseries EDICS Category: 3-BBND \end{center}
% \fi
%
% For peerreview papers, this IEEEtran command inserts a page break and
% creates the second title. It will be ignored for other modes.
%\IEEEpeerreviewmaketitle




If the letters of the word ASSASSINATION are arranged at random. Find the Probability that
\begin{enumerate}[label=(\alph*)]
\item Four $S's$ come consecutively in the word
\item Two  $I's$ and two $N's$ come together
\item All $A's$ are not coming together
\item No two $A's$ are coming together
\end{enumerate}
%\begin{table}[H]
	\centering
\begin{tabular}{|c|c|c|}
\hline
Random variable &Value &Definition\\ \hline
\multirow{3}{*}{X} &0 &Slips of Rs 1\\
&1 &Slips of Rs 5\\
&2 &Slips of Rs 13\\ \hline
\multirow{2}{*}{Y} &0 &Box A\\
&1 &Box B\\\hline
\end{tabular}
\caption{}
\label{tab:Distribution}
\end{table}
See \tabref{tab:Distribution}.
\begin{align}
p_{Y}\brak{k}= \begin{cases} 
      \frac{1}{3} & {k=0} \\
      \frac{2}{3 }& {k=1} 
   \end{cases}
   \\
p_{Y|X}\brak{0|0} = \frac{19}{25}\, 
p_{Y|X}\brak{0|1} = \frac{6}{25}\,
p_{Y|X}\brak{1|0} = \frac{45}{50}\,
p_{Y|X}\brak{1|2} = \frac{5}{50}
\end{align}
The desired probability is the probability that a slip drawn at random is marked other than Rs 1,
\begin{align}
&=1-p_X\brak{0}\\
&= p_X(1) + p_X(2)
\end{align}
Using Bayes theorem,
\begin{align}
&= p_Y\brak{0} \times \pr{Y=0 | X=1} + p_Y\brak{1} \times \pr{Y=1|X=2}\\
&=\frac{1}{3} \times \frac{6}{25} + \frac{2}{3} \times \frac{5}{50}\\
&=\frac{11}{75}
\end{align}

\newpage

%\tableofcontents

\bigskip

\renewcommand{\thefigure}{\theenumi}
\renewcommand{\thetable}{\theenumi}
%\renewcommand{\theequation}{\theenumi}

%\begin{abstract}
%%\boldmath
%In this letter, an algorithm for evaluating the exact analytical bit error rate  (BER)  for the piecewise linear (PL) combiner for  multiple relays is presented. Previous results were available only for upto three relays. The algorithm is unique in the sense that  the actual mathematical expressions, that are prohibitively large, need not be explicitly obtained. The diversity gain due to multiple relays is shown through plots of the analytical BER, well supported by simulations. 
%
%\end{abstract}
% IEEEtran.cls defaults to using nonbold math in the Abstract.
% This preserves the distinction between vectors and scalars. However,
% if the journal you are submitting to favors bold math in the abstract,
% then you can use LaTeX's standard command \boldmath at the very start
% of the abstract to achieve this. Many IEEE journals frown on math
% in the abstract anyway.

% Note that keywords are not normally used for peerreview papers.
%\begin{IEEEkeywords}
%Cooperative diversity, decode and forward, piecewise linear
%\end{IEEEkeywords}



% For peer review papers, you can put extra information on the cover
% page as needed:
% \ifCLASSOPTIONpeerreview
% \begin{center} \bfseries EDICS Category: 3-BBND \end{center}
% \fi
%
% For peerreview papers, this IEEEtran command inserts a page break and
% creates the second title. It will be ignored for other modes.
%\IEEEpeerreviewmaketitle




	\item One urn contains two black balls (labelled B1 and B2) and one white ball. A
	second urn contains one black ball and two white balls (labelled W1 and W2).
	Suppose the following experiment is performed. One of the two urns is chosen
	at random. Next a ball is randomly chosen from the urn. Then a second ball is
	chosen at random from the same urn without replacing the first ball.
	
	\begin{enumerate}
	\item What is the probability that two black balls are chosen?
	
	\item What is the probability that two balls of opposite colour are chosen?
	\end{enumerate}
	\solution
	%\begin{align}
    \label{eq:12.13.6.18.1}
	\because	\pr{A|B} &> \pr{A},\
\frac{\pr{AB}}{\pr{B}} > \pr{A}
\\
    \label{eq:12.13.6.18.2}
	\implies \pr{AB} &> \pr{A}\pr{B}
	\\
	\text{or, } \frac{\pr{AB}}{\pr{A}} &=\pr{B|A} > \pr{A}
\end{align}

\end{enumerate}

		\item A box of oranges is inspected by examining three randomly selected oranges drawn without replacement. If all the three oranges are good, the box is approved for sale, otherwise, it is rejected. Find the probability that a box containing 15 oranges out of which 12 are good and 3 are bad ones will be approved for sale.
		\label{ncert/12/13/2/3/defs.tex}
		\item Two balls are drawn at random with replacement from a box containing 10 black and 8 red balls. Find the probability that
		\label{ncert/12/13/2/12}
\begin{enumerate}
\item both balls are red.
\item first ball is black and second is red.
\item one of them is black and other is red.
\end{enumerate}

\item In a hostel, 60\% of the students read Hindi newspaper, 40\% read English newspaper and 20\% read both Hindi and English newspapers. A student is selected at random.
		\label{ncert/12/13/2/15}
\begin{enumerate}
\item Find the probability that she reads neither Hindi nor English newspapers.
\item If she reads Hindi newspaper, find the probability that she reads English newspaper.
\item If she reads English newspaper, find the probability that she reads Hindi newspaper.\\
\end{enumerate}
\item The probability of obtaining an even prime number on each die, when a pair of dice is rolled is 
\begin{enumerate}
    \item $0$ 
    
    \item $\frac{1}{3}$ 
    
    \item $\frac{1}{12}$ 
    
    \item $\frac{1}{36}$ 
\end{enumerate}
\solution
		%\begin{enumerate}[label=\thesection.\arabic*,ref=\thesection.\theenumi]
	\item One card is drawn from a well-shuffled deck of 52 cards. Find the probability of getting
\begin{enumerate}
\item A king of red colour 
\item A face card 
\item A red face card
\item The jack of hearts
\item A spade
\item The queen of diamonds

\end{enumerate}
\solution
		%\begin{table}[H]
	\centering
\begin{tabular}{|c|c|c|}
\hline
Random variable &Value &Definition\\ \hline
\multirow{3}{*}{X} &0 &Slips of Rs 1\\
&1 &Slips of Rs 5\\
&2 &Slips of Rs 13\\ \hline
\multirow{2}{*}{Y} &0 &Box A\\
&1 &Box B\\\hline
\end{tabular}
\caption{}
\label{tab:Distribution}
\end{table}
See \tabref{tab:Distribution}.
\begin{align}
p_{Y}\brak{k}= \begin{cases} 
      \frac{1}{3} & {k=0} \\
      \frac{2}{3 }& {k=1} 
   \end{cases}
   \\
p_{Y|X}\brak{0|0} = \frac{19}{25}\, 
p_{Y|X}\brak{0|1} = \frac{6}{25}\,
p_{Y|X}\brak{1|0} = \frac{45}{50}\,
p_{Y|X}\brak{1|2} = \frac{5}{50}
\end{align}
The desired probability is the probability that a slip drawn at random is marked other than Rs 1,
\begin{align}
&=1-p_X\brak{0}\\
&= p_X(1) + p_X(2)
\end{align}
Using Bayes theorem,
\begin{align}
&= p_Y\brak{0} \times \pr{Y=0 | X=1} + p_Y\brak{1} \times \pr{Y=1|X=2}\\
&=\frac{1}{3} \times \frac{6}{25} + \frac{2}{3} \times \frac{5}{50}\\
&=\frac{11}{75}
\end{align}

\newpage

%\tableofcontents

\bigskip

\renewcommand{\thefigure}{\theenumi}
\renewcommand{\thetable}{\theenumi}
%\renewcommand{\theequation}{\theenumi}

%\begin{abstract}
%%\boldmath
%In this letter, an algorithm for evaluating the exact analytical bit error rate  (BER)  for the piecewise linear (PL) combiner for  multiple relays is presented. Previous results were available only for upto three relays. The algorithm is unique in the sense that  the actual mathematical expressions, that are prohibitively large, need not be explicitly obtained. The diversity gain due to multiple relays is shown through plots of the analytical BER, well supported by simulations. 
%
%\end{abstract}
% IEEEtran.cls defaults to using nonbold math in the Abstract.
% This preserves the distinction between vectors and scalars. However,
% if the journal you are submitting to favors bold math in the abstract,
% then you can use LaTeX's standard command \boldmath at the very start
% of the abstract to achieve this. Many IEEE journals frown on math
% in the abstract anyway.

% Note that keywords are not normally used for peerreview papers.
%\begin{IEEEkeywords}
%Cooperative diversity, decode and forward, piecewise linear
%\end{IEEEkeywords}



% For peer review papers, you can put extra information on the cover
% page as needed:
% \ifCLASSOPTIONpeerreview
% \begin{center} \bfseries EDICS Category: 3-BBND \end{center}
% \fi
%
% For peerreview papers, this IEEEtran command inserts a page break and
% creates the second title. It will be ignored for other modes.
%\IEEEpeerreviewmaketitle




	\item Five cards—the ten, jack, queen, king and ace of diamonds, are well-shuffled with their face downwards. One card is then picked up at random.
\begin{enumerate}
\item
What is the probability that the card is the queen? 
\item
If the queen is drawn and put aside, what is the probability that the second card picked up is (a) an ace? (b) a queen?\\
\end{enumerate}
\solution
		%\begin{enumerate}[label=\thesection.\arabic*,ref=\thesection.\theenumi]
	\item One card is drawn from a well-shuffled deck of 52 cards. Find the probability of getting
\begin{enumerate}
\item A king of red colour 
\item A face card 
\item A red face card
\item The jack of hearts
\item A spade
\item The queen of diamonds

\end{enumerate}
\solution
		%\begin{table}[H]
	\centering
\begin{tabular}{|c|c|c|}
\hline
Random variable &Value &Definition\\ \hline
\multirow{3}{*}{X} &0 &Slips of Rs 1\\
&1 &Slips of Rs 5\\
&2 &Slips of Rs 13\\ \hline
\multirow{2}{*}{Y} &0 &Box A\\
&1 &Box B\\\hline
\end{tabular}
\caption{}
\label{tab:Distribution}
\end{table}
See \tabref{tab:Distribution}.
\begin{align}
p_{Y}\brak{k}= \begin{cases} 
      \frac{1}{3} & {k=0} \\
      \frac{2}{3 }& {k=1} 
   \end{cases}
   \\
p_{Y|X}\brak{0|0} = \frac{19}{25}\, 
p_{Y|X}\brak{0|1} = \frac{6}{25}\,
p_{Y|X}\brak{1|0} = \frac{45}{50}\,
p_{Y|X}\brak{1|2} = \frac{5}{50}
\end{align}
The desired probability is the probability that a slip drawn at random is marked other than Rs 1,
\begin{align}
&=1-p_X\brak{0}\\
&= p_X(1) + p_X(2)
\end{align}
Using Bayes theorem,
\begin{align}
&= p_Y\brak{0} \times \pr{Y=0 | X=1} + p_Y\brak{1} \times \pr{Y=1|X=2}\\
&=\frac{1}{3} \times \frac{6}{25} + \frac{2}{3} \times \frac{5}{50}\\
&=\frac{11}{75}
\end{align}

\newpage

%\tableofcontents

\bigskip

\renewcommand{\thefigure}{\theenumi}
\renewcommand{\thetable}{\theenumi}
%\renewcommand{\theequation}{\theenumi}

%\begin{abstract}
%%\boldmath
%In this letter, an algorithm for evaluating the exact analytical bit error rate  (BER)  for the piecewise linear (PL) combiner for  multiple relays is presented. Previous results were available only for upto three relays. The algorithm is unique in the sense that  the actual mathematical expressions, that are prohibitively large, need not be explicitly obtained. The diversity gain due to multiple relays is shown through plots of the analytical BER, well supported by simulations. 
%
%\end{abstract}
% IEEEtran.cls defaults to using nonbold math in the Abstract.
% This preserves the distinction between vectors and scalars. However,
% if the journal you are submitting to favors bold math in the abstract,
% then you can use LaTeX's standard command \boldmath at the very start
% of the abstract to achieve this. Many IEEE journals frown on math
% in the abstract anyway.

% Note that keywords are not normally used for peerreview papers.
%\begin{IEEEkeywords}
%Cooperative diversity, decode and forward, piecewise linear
%\end{IEEEkeywords}



% For peer review papers, you can put extra information on the cover
% page as needed:
% \ifCLASSOPTIONpeerreview
% \begin{center} \bfseries EDICS Category: 3-BBND \end{center}
% \fi
%
% For peerreview papers, this IEEEtran command inserts a page break and
% creates the second title. It will be ignored for other modes.
%\IEEEpeerreviewmaketitle




	\item Five cards—the ten, jack, queen, king and ace of diamonds, are well-shuffled with their face downwards. One card is then picked up at random.
\begin{enumerate}
\item
What is the probability that the card is the queen? 
\item
If the queen is drawn and put aside, what is the probability that the second card picked up is (a) an ace? (b) a queen?\\
\end{enumerate}
\solution
		%\begin{enumerate}[label=\thesection.\arabic*,ref=\thesection.\theenumi]
	\item One card is drawn from a well-shuffled deck of 52 cards. Find the probability of getting
\begin{enumerate}
\item A king of red colour 
\item A face card 
\item A red face card
\item The jack of hearts
\item A spade
\item The queen of diamonds

\end{enumerate}
\solution
		%\input{ncert/10/15/1/14/main.tex}
	\item Five cards—the ten, jack, queen, king and ace of diamonds, are well-shuffled with their face downwards. One card is then picked up at random.
\begin{enumerate}
\item
What is the probability that the card is the queen? 
\item
If the queen is drawn and put aside, what is the probability that the second card picked up is (a) an ace? (b) a queen?\\
\end{enumerate}
\solution
		%\input{ncert/10/15/1/15/defs.tex}
	\item A bag contains $5$ red balls and some blue balls. If the probability of drawing a blue ball is double that if a red ball, determine the number of blue balls in the bag. 
		\\
\solution
		%\input{ncert/10/15/2/3/defs.tex}
	\item A card is selected from a pack of 52 cards.
 \begin{enumerate}[label=(\alph*)] 
                 \item How many points are there in the sample space?
                 \item Calculate the probability that the card is an ace of spades.
                 \item Calculate the probability that the card is (i) an ace and (ii) black card.
 \end{enumerate}
\solution
		%\input{ncert/11/16/3/4/main.tex}
\item Four cards are drawn from a well-shuffled deck of 52 cards. What is the probability of obtaining 3 diamonds and one spade.
\\
\solution
		%\input{ncert/11/16/4/2/defs.tex}
\item In a certain lottery 10,000 tickets are sold and ten equal prizes are awarded. What is the probability of not getting a prize if you buy (a) one ticket (b) two tickets (c) 10 tickets ?	
\\
\solution
		%\input{ncert/11/16/4/4/defs.tex}
		%
\item 
Out of 100 students, two sections of 40 and 60 are formed. If you and your friend are among the 100 students, what is the probability that
\begin{enumerate}
\item you both enter the same section?
\item you both enter the different sections?
\end{enumerate}
\solution
		%\input{ncert/11/16/4/5/defs.tex}
	\item 
The number lock of a suitcase has 4 wheels each labelled with ten digits i.e. from 0 to 9.The lock opens with a sequence of four digits with no repeats.What is the probability of a person getting the right sequence to open the suitcase.
\\
\solution
		%\input{ncert/11/16/4/10/defs.tex}
		%
\item 
Two cards are drawn at random and without replacement from a pack of 52 playing cards. Find the probability that both the cards are black.
\\
\solution
		%\input{ncert/12/13/2/2/defs.tex}
		\item A box of oranges is inspected by examining three randomly selected oranges drawn without replacement. If all the three oranges are good, the box is approved for sale, otherwise, it is rejected. Find the probability that a box containing 15 oranges out of which 12 are good and 3 are bad ones will be approved for sale.
		\label{ncert/12/13/2/3/defs.tex}
		\item Two balls are drawn at random with replacement from a box containing 10 black and 8 red balls. Find the probability that
		\label{ncert/12/13/2/12}
\begin{enumerate}
\item both balls are red.
\item first ball is black and second is red.
\item one of them is black and other is red.
\end{enumerate}

\item In a hostel, 60\% of the students read Hindi newspaper, 40\% read English newspaper and 20\% read both Hindi and English newspapers. A student is selected at random.
		\label{ncert/12/13/2/15}
\begin{enumerate}
\item Find the probability that she reads neither Hindi nor English newspapers.
\item If she reads Hindi newspaper, find the probability that she reads English newspaper.
\item If she reads English newspaper, find the probability that she reads Hindi newspaper.\\
\end{enumerate}
\item The probability of obtaining an even prime number on each die, when a pair of dice is rolled is 
\begin{enumerate}
    \item $0$ 
    
    \item $\frac{1}{3}$ 
    
    \item $\frac{1}{12}$ 
    
    \item $\frac{1}{36}$ 
\end{enumerate}
\solution
		%\input{ncert/12/13/2/17/defs.tex}
	\item A bag contains 4 red and 4 black balls, another bag contains 2 red and 6 black balls. One of the two bags is selected at random and a ball is drawn from the bag which is found to be red. Find the probability that the ball is drawn from the first bag.
\\
\solution
		%\input{ncert/12/13/3/2/main.tex}
  \item
  Cards with numbers 2 to 101 are placed in a box. A card is selected at random.Find the probability that the card has
\begin{enumerate}[label=(\roman*)]
	\item an even number 
	\item a square number
\end{enumerate}
\solution
%\input{exemplar/10/13/3/32/main.tex}
\item
The king, queen and jack of clubs are removed from a deck of 52 playing cards and then well shuffled. Now one card is drawn at random from the remaining cards.  Determine the probability that the card is
\begin{enumerate}[label=(\roman*)]
\item a club
\item 10 of hearts
\end{enumerate}
\solution
%\input{exemplar/10/13/3/29/main.tex}
\item A team of medical students doing their internship have to assist during surgeries
at a city hospital. The probabilities of surgeries rated as very complex, complex,
routine, simple or very simple are respectively, 0.15, 0.20, 0.31, 0.26, .08. Find
the probabilities that a particular surgery will be rated
\begin{enumerate}
	\item complex or very complex;
	\item neither very complex nor very simple;
	\item routine or complex
	\item routine or simple
\end{enumerate}
\solution
%\input{exemplar/11/16/3/8(1)/main.tex}
\item A card is selected from a pack of 52 cards.
\begin{enumerate}[label=(\alph*)]
    \item How many points are there in the sample space?
    \item Calculate the probability that the card is an ace of spades.
    \item Calculate the probability that the card is (i) an ace and (ii) black card.
\end{enumerate}
\solution
%\input{exemplar/11/16/3/4/main2.tex}
\item The probability that a non leap year selected at random will contain 53 sundays.
\\
\solution
%\input{exemplar/10/13/1/19/main.tex}
\item One of the four persons John, Rita, Aslam or Gurpreet will be promoted next
month. Consequently the sample space consists of four elementary outcomes
S = {John promoted, Rita promoted, Aslam promoted, Gurpreet promoted}
You are told that the chances of John’s promotion is same as that of Gurpreet,
Rita’s chances of promotion are twice as likely as Johns. Aslam’s chances are
four times that of John.
\begin{enumerate}
	\item Determine
	\begin{enumerate}
		\item P (John promoted)
		\item P (Rita promoted)
		\item P (Aslam promoted)
		\item P (Gurpreet promoted)
	\end{enumerate}
	\item If A = {John promoted or Gurpreet promoted}, find P (A).
\end{enumerate}
\solution
%\input{exemplar/11/16/3/10/main.tex}
\item A card is drawn from a deck of 52 cards. Find the probability of getting a king or a heart or a red card.\\
\solution
%\input{exemplar/11/16/3/15/main.tex}
\item The probability that a student will pass his examination is 0.73, the probability of
the student getting a compartment is 0.13, and the probability that the student will
either pass or get compartment is 0.96. State True or False.\\
\solution
%\input{exemplar/11/16/3/31/main.tex}
\item A card is selected from a pack of 52 cards\\
\begin{enumerate}[label=(\alph*)]
\item How many points are there in the sample space?
\item Calculate the probability that the cards is an ace of spades.
\item Calculate the probability that the card is (i) an ace (ii)black card.\\
\end{enumerate}
%\input{ncert/11/16/3/4_1/Prob_4.tex}
\item In a non-leap year, the probability of having 53 tuesdays or 53 wednesdays is\\
\solution
%\input{exemplar/11/16/3/18/main.tex}
\item There are 1000 sealed envelopes in a box, 10 of them contain a cash prize of
Rs 100 each, 100 of them contain a cash prize of Rs 50 each and 200 of them
contain a cash prize of Rs 10 each and rest do not contain any cash prize. If they
are well shuffled and an envelope is picked up out, what is the probability that it
contains no cash prize?\\
\solution
%\input{exemplar/10/13/3/34/main.tex}
\item 
A die is thrown and a card is selected at random from a deck of 52 playing cards. The probability of getting an even number on the die and a spade card.\\
\solution
%\input{exemplar/12/13/3/78/main.tex}
\item
If 4-digit numbers greater than 5,000 are randomly formed from the digits 0, 1, 3, 5, and 7, what is the probability of forming a number divisible by 5 when:
\begin{enumerate}
    \item The digits are repeated?
    \item The repetition of digits is not allowed?
\end{enumerate}
\solution
%\input{ncert/11/16/4/9/main.tex}
\item Consider the probability space $\brak{\Omega, \mathcal{G}, P}$ where $\Omega = [0,2]$ and $\mathcal{G} = \cbrak{\phi, \Omega, [0,1], (1,2]}$. Let $X$ and $Y$ be two functions on $\Omega$ defined as
\begin{align*}
    X(\omega) = 
    \begin{cases}
        1 & \text{if }\omega \in [0, 1]\\
        2 & \text{if }\omega \in (1, 2]
    \end{cases}
\end{align*}
and
\begin{align*}
    Y(\omega) = 
    \begin{cases}
        2 & \text{if }\omega \in [0, 1.5]\\
        3 & \text{if }\omega \in (1.5, 2].
    \end{cases}
\end{align*}
Then which one of the following statements is true?
\begin{enumerate}
    \item [(A)] $X$ is a random variable with respect to $\mathcal{G}$, but $Y$ is not a random variable with respect to $\mathcal{G}$.
    \item [(B)] $Y$ is a random variable with respect to $\mathcal{G}$, but $X$ is not a random variable with respect to $\mathcal{G}$.
    \item [(C)] Neither $X$ nor $Y$ is a random variable with respect to $\mathcal{G}$.
    \item [(D)] Both $X$ and $Y$ are random variables with respect to $\mathcal{G}$.
\end{enumerate} \hfill (GATE ST 2023)\\
\solution
%\input{gate/ST/2023/14/main.tex}
	\item  A die is loaded in such a way that each odd number is twice as likely to occur as
each even number. Find $P(G)$, where $G$ is the event that a number greater than
3 occurs on a single roll of the die.
\\
\solution
		%\input{exemplar/11/16/3/5/main.tex}
	\item All the jacks, queens and kings are removed from a deck of 52 playing cards. The remaining cards are well shuffled and then one card is drawn at random. Giving ace a value 1 similar value for other cards, find the probability that the card has a value 
		\begin{enumerate}
			\item 7
			\item greater than 7
			\item less than 7
		\end{enumerate}
		%\input{exemplar/10/13/3/30/main.tex}
  \item A Lot consists of 48 mobile phones of which 42 are good, 3 have only minor defects and 3 have major defects.Varnika will buy a phone if it is good but the trader will only buy a mobile if it has no major defects. One phone is selected at random from the lot. What is the probability that it is
\begin{enumerate}
	\item acceptable to Varnika?
            \item acceptable to the trader?
\end{enumerate}
\solution
	%\input{exemplar/10/13/3/40/main.tex}
 \item A student says that if you throw a die, it will show up 1 or not 1. Therefore, the probability of getting 1 and the probability of getting 'not 1' each is equal to $\frac{1}{2}$. Is this correct? Give reasons.\\
 \solution
        %\input{exemplar/10/13/2/9/main.tex}
   \item Four candidates A, B, C, D have ap-
plied for the assignment to coach a school cricket
team. If A is twice as likely to be selected as B, and
B and C are given about the same chance of being
selected, while C is twice as likely to be selected
as D, what are the probabilities that
\begin{enumerate}
\item C will be selected?
\item A will not be selected?
\end{enumerate}
	%\input{exemplar/11/16/3/9/main.tex}
 \item A bag contain 24 balls of which $x$ balls are red, $2x$ are white and $3x$ are blue. A ball is selected at random, What is the probability that it is
\begin{enumerate}[label=\alph*)]
\item not red ?
\item white ?
\end{enumerate}
%\input{exemplar/10/13/3/41/main.tex}
If the letters of the word ASSASSINATION are arranged at random. Find the Probability that
\begin{enumerate}[label=(\alph*)]
\item Four $S's$ come consecutively in the word
\item Two  $I's$ and two $N's$ come together
\item All $A's$ are not coming together
\item No two $A's$ are coming together
\end{enumerate}
%\input{exemplar/11/16/3/14/main.tex}
	\item One urn contains two black balls (labelled B1 and B2) and one white ball. A
	second urn contains one black ball and two white balls (labelled W1 and W2).
	Suppose the following experiment is performed. One of the two urns is chosen
	at random. Next a ball is randomly chosen from the urn. Then a second ball is
	chosen at random from the same urn without replacing the first ball.
	
	\begin{enumerate}
	\item What is the probability that two black balls are chosen?
	
	\item What is the probability that two balls of opposite colour are chosen?
	\end{enumerate}
	\solution
	%\input{exemplar/11/16/3/12/main1.tex}
\end{enumerate}

	\item A bag contains $5$ red balls and some blue balls. If the probability of drawing a blue ball is double that if a red ball, determine the number of blue balls in the bag. 
		\\
\solution
		%\begin{enumerate}[label=\thesection.\arabic*,ref=\thesection.\theenumi]
	\item One card is drawn from a well-shuffled deck of 52 cards. Find the probability of getting
\begin{enumerate}
\item A king of red colour 
\item A face card 
\item A red face card
\item The jack of hearts
\item A spade
\item The queen of diamonds

\end{enumerate}
\solution
		%\input{ncert/10/15/1/14/main.tex}
	\item Five cards—the ten, jack, queen, king and ace of diamonds, are well-shuffled with their face downwards. One card is then picked up at random.
\begin{enumerate}
\item
What is the probability that the card is the queen? 
\item
If the queen is drawn and put aside, what is the probability that the second card picked up is (a) an ace? (b) a queen?\\
\end{enumerate}
\solution
		%\input{ncert/10/15/1/15/defs.tex}
	\item A bag contains $5$ red balls and some blue balls. If the probability of drawing a blue ball is double that if a red ball, determine the number of blue balls in the bag. 
		\\
\solution
		%\input{ncert/10/15/2/3/defs.tex}
	\item A card is selected from a pack of 52 cards.
 \begin{enumerate}[label=(\alph*)] 
                 \item How many points are there in the sample space?
                 \item Calculate the probability that the card is an ace of spades.
                 \item Calculate the probability that the card is (i) an ace and (ii) black card.
 \end{enumerate}
\solution
		%\input{ncert/11/16/3/4/main.tex}
\item Four cards are drawn from a well-shuffled deck of 52 cards. What is the probability of obtaining 3 diamonds and one spade.
\\
\solution
		%\input{ncert/11/16/4/2/defs.tex}
\item In a certain lottery 10,000 tickets are sold and ten equal prizes are awarded. What is the probability of not getting a prize if you buy (a) one ticket (b) two tickets (c) 10 tickets ?	
\\
\solution
		%\input{ncert/11/16/4/4/defs.tex}
		%
\item 
Out of 100 students, two sections of 40 and 60 are formed. If you and your friend are among the 100 students, what is the probability that
\begin{enumerate}
\item you both enter the same section?
\item you both enter the different sections?
\end{enumerate}
\solution
		%\input{ncert/11/16/4/5/defs.tex}
	\item 
The number lock of a suitcase has 4 wheels each labelled with ten digits i.e. from 0 to 9.The lock opens with a sequence of four digits with no repeats.What is the probability of a person getting the right sequence to open the suitcase.
\\
\solution
		%\input{ncert/11/16/4/10/defs.tex}
		%
\item 
Two cards are drawn at random and without replacement from a pack of 52 playing cards. Find the probability that both the cards are black.
\\
\solution
		%\input{ncert/12/13/2/2/defs.tex}
		\item A box of oranges is inspected by examining three randomly selected oranges drawn without replacement. If all the three oranges are good, the box is approved for sale, otherwise, it is rejected. Find the probability that a box containing 15 oranges out of which 12 are good and 3 are bad ones will be approved for sale.
		\label{ncert/12/13/2/3/defs.tex}
		\item Two balls are drawn at random with replacement from a box containing 10 black and 8 red balls. Find the probability that
		\label{ncert/12/13/2/12}
\begin{enumerate}
\item both balls are red.
\item first ball is black and second is red.
\item one of them is black and other is red.
\end{enumerate}

\item In a hostel, 60\% of the students read Hindi newspaper, 40\% read English newspaper and 20\% read both Hindi and English newspapers. A student is selected at random.
		\label{ncert/12/13/2/15}
\begin{enumerate}
\item Find the probability that she reads neither Hindi nor English newspapers.
\item If she reads Hindi newspaper, find the probability that she reads English newspaper.
\item If she reads English newspaper, find the probability that she reads Hindi newspaper.\\
\end{enumerate}
\item The probability of obtaining an even prime number on each die, when a pair of dice is rolled is 
\begin{enumerate}
    \item $0$ 
    
    \item $\frac{1}{3}$ 
    
    \item $\frac{1}{12}$ 
    
    \item $\frac{1}{36}$ 
\end{enumerate}
\solution
		%\input{ncert/12/13/2/17/defs.tex}
	\item A bag contains 4 red and 4 black balls, another bag contains 2 red and 6 black balls. One of the two bags is selected at random and a ball is drawn from the bag which is found to be red. Find the probability that the ball is drawn from the first bag.
\\
\solution
		%\input{ncert/12/13/3/2/main.tex}
  \item
  Cards with numbers 2 to 101 are placed in a box. A card is selected at random.Find the probability that the card has
\begin{enumerate}[label=(\roman*)]
	\item an even number 
	\item a square number
\end{enumerate}
\solution
%\input{exemplar/10/13/3/32/main.tex}
\item
The king, queen and jack of clubs are removed from a deck of 52 playing cards and then well shuffled. Now one card is drawn at random from the remaining cards.  Determine the probability that the card is
\begin{enumerate}[label=(\roman*)]
\item a club
\item 10 of hearts
\end{enumerate}
\solution
%\input{exemplar/10/13/3/29/main.tex}
\item A team of medical students doing their internship have to assist during surgeries
at a city hospital. The probabilities of surgeries rated as very complex, complex,
routine, simple or very simple are respectively, 0.15, 0.20, 0.31, 0.26, .08. Find
the probabilities that a particular surgery will be rated
\begin{enumerate}
	\item complex or very complex;
	\item neither very complex nor very simple;
	\item routine or complex
	\item routine or simple
\end{enumerate}
\solution
%\input{exemplar/11/16/3/8(1)/main.tex}
\item A card is selected from a pack of 52 cards.
\begin{enumerate}[label=(\alph*)]
    \item How many points are there in the sample space?
    \item Calculate the probability that the card is an ace of spades.
    \item Calculate the probability that the card is (i) an ace and (ii) black card.
\end{enumerate}
\solution
%\input{exemplar/11/16/3/4/main2.tex}
\item The probability that a non leap year selected at random will contain 53 sundays.
\\
\solution
%\input{exemplar/10/13/1/19/main.tex}
\item One of the four persons John, Rita, Aslam or Gurpreet will be promoted next
month. Consequently the sample space consists of four elementary outcomes
S = {John promoted, Rita promoted, Aslam promoted, Gurpreet promoted}
You are told that the chances of John’s promotion is same as that of Gurpreet,
Rita’s chances of promotion are twice as likely as Johns. Aslam’s chances are
four times that of John.
\begin{enumerate}
	\item Determine
	\begin{enumerate}
		\item P (John promoted)
		\item P (Rita promoted)
		\item P (Aslam promoted)
		\item P (Gurpreet promoted)
	\end{enumerate}
	\item If A = {John promoted or Gurpreet promoted}, find P (A).
\end{enumerate}
\solution
%\input{exemplar/11/16/3/10/main.tex}
\item A card is drawn from a deck of 52 cards. Find the probability of getting a king or a heart or a red card.\\
\solution
%\input{exemplar/11/16/3/15/main.tex}
\item The probability that a student will pass his examination is 0.73, the probability of
the student getting a compartment is 0.13, and the probability that the student will
either pass or get compartment is 0.96. State True or False.\\
\solution
%\input{exemplar/11/16/3/31/main.tex}
\item A card is selected from a pack of 52 cards\\
\begin{enumerate}[label=(\alph*)]
\item How many points are there in the sample space?
\item Calculate the probability that the cards is an ace of spades.
\item Calculate the probability that the card is (i) an ace (ii)black card.\\
\end{enumerate}
%\input{ncert/11/16/3/4_1/Prob_4.tex}
\item In a non-leap year, the probability of having 53 tuesdays or 53 wednesdays is\\
\solution
%\input{exemplar/11/16/3/18/main.tex}
\item There are 1000 sealed envelopes in a box, 10 of them contain a cash prize of
Rs 100 each, 100 of them contain a cash prize of Rs 50 each and 200 of them
contain a cash prize of Rs 10 each and rest do not contain any cash prize. If they
are well shuffled and an envelope is picked up out, what is the probability that it
contains no cash prize?\\
\solution
%\input{exemplar/10/13/3/34/main.tex}
\item 
A die is thrown and a card is selected at random from a deck of 52 playing cards. The probability of getting an even number on the die and a spade card.\\
\solution
%\input{exemplar/12/13/3/78/main.tex}
\item
If 4-digit numbers greater than 5,000 are randomly formed from the digits 0, 1, 3, 5, and 7, what is the probability of forming a number divisible by 5 when:
\begin{enumerate}
    \item The digits are repeated?
    \item The repetition of digits is not allowed?
\end{enumerate}
\solution
%\input{ncert/11/16/4/9/main.tex}
\item Consider the probability space $\brak{\Omega, \mathcal{G}, P}$ where $\Omega = [0,2]$ and $\mathcal{G} = \cbrak{\phi, \Omega, [0,1], (1,2]}$. Let $X$ and $Y$ be two functions on $\Omega$ defined as
\begin{align*}
    X(\omega) = 
    \begin{cases}
        1 & \text{if }\omega \in [0, 1]\\
        2 & \text{if }\omega \in (1, 2]
    \end{cases}
\end{align*}
and
\begin{align*}
    Y(\omega) = 
    \begin{cases}
        2 & \text{if }\omega \in [0, 1.5]\\
        3 & \text{if }\omega \in (1.5, 2].
    \end{cases}
\end{align*}
Then which one of the following statements is true?
\begin{enumerate}
    \item [(A)] $X$ is a random variable with respect to $\mathcal{G}$, but $Y$ is not a random variable with respect to $\mathcal{G}$.
    \item [(B)] $Y$ is a random variable with respect to $\mathcal{G}$, but $X$ is not a random variable with respect to $\mathcal{G}$.
    \item [(C)] Neither $X$ nor $Y$ is a random variable with respect to $\mathcal{G}$.
    \item [(D)] Both $X$ and $Y$ are random variables with respect to $\mathcal{G}$.
\end{enumerate} \hfill (GATE ST 2023)\\
\solution
%\input{gate/ST/2023/14/main.tex}
	\item  A die is loaded in such a way that each odd number is twice as likely to occur as
each even number. Find $P(G)$, where $G$ is the event that a number greater than
3 occurs on a single roll of the die.
\\
\solution
		%\input{exemplar/11/16/3/5/main.tex}
	\item All the jacks, queens and kings are removed from a deck of 52 playing cards. The remaining cards are well shuffled and then one card is drawn at random. Giving ace a value 1 similar value for other cards, find the probability that the card has a value 
		\begin{enumerate}
			\item 7
			\item greater than 7
			\item less than 7
		\end{enumerate}
		%\input{exemplar/10/13/3/30/main.tex}
  \item A Lot consists of 48 mobile phones of which 42 are good, 3 have only minor defects and 3 have major defects.Varnika will buy a phone if it is good but the trader will only buy a mobile if it has no major defects. One phone is selected at random from the lot. What is the probability that it is
\begin{enumerate}
	\item acceptable to Varnika?
            \item acceptable to the trader?
\end{enumerate}
\solution
	%\input{exemplar/10/13/3/40/main.tex}
 \item A student says that if you throw a die, it will show up 1 or not 1. Therefore, the probability of getting 1 and the probability of getting 'not 1' each is equal to $\frac{1}{2}$. Is this correct? Give reasons.\\
 \solution
        %\input{exemplar/10/13/2/9/main.tex}
   \item Four candidates A, B, C, D have ap-
plied for the assignment to coach a school cricket
team. If A is twice as likely to be selected as B, and
B and C are given about the same chance of being
selected, while C is twice as likely to be selected
as D, what are the probabilities that
\begin{enumerate}
\item C will be selected?
\item A will not be selected?
\end{enumerate}
	%\input{exemplar/11/16/3/9/main.tex}
 \item A bag contain 24 balls of which $x$ balls are red, $2x$ are white and $3x$ are blue. A ball is selected at random, What is the probability that it is
\begin{enumerate}[label=\alph*)]
\item not red ?
\item white ?
\end{enumerate}
%\input{exemplar/10/13/3/41/main.tex}
If the letters of the word ASSASSINATION are arranged at random. Find the Probability that
\begin{enumerate}[label=(\alph*)]
\item Four $S's$ come consecutively in the word
\item Two  $I's$ and two $N's$ come together
\item All $A's$ are not coming together
\item No two $A's$ are coming together
\end{enumerate}
%\input{exemplar/11/16/3/14/main.tex}
	\item One urn contains two black balls (labelled B1 and B2) and one white ball. A
	second urn contains one black ball and two white balls (labelled W1 and W2).
	Suppose the following experiment is performed. One of the two urns is chosen
	at random. Next a ball is randomly chosen from the urn. Then a second ball is
	chosen at random from the same urn without replacing the first ball.
	
	\begin{enumerate}
	\item What is the probability that two black balls are chosen?
	
	\item What is the probability that two balls of opposite colour are chosen?
	\end{enumerate}
	\solution
	%\input{exemplar/11/16/3/12/main1.tex}
\end{enumerate}

	\item A card is selected from a pack of 52 cards.
 \begin{enumerate}[label=(\alph*)] 
                 \item How many points are there in the sample space?
                 \item Calculate the probability that the card is an ace of spades.
                 \item Calculate the probability that the card is (i) an ace and (ii) black card.
 \end{enumerate}
\solution
		%\begin{table}[H]
	\centering
\begin{tabular}{|c|c|c|}
\hline
Random variable &Value &Definition\\ \hline
\multirow{3}{*}{X} &0 &Slips of Rs 1\\
&1 &Slips of Rs 5\\
&2 &Slips of Rs 13\\ \hline
\multirow{2}{*}{Y} &0 &Box A\\
&1 &Box B\\\hline
\end{tabular}
\caption{}
\label{tab:Distribution}
\end{table}
See \tabref{tab:Distribution}.
\begin{align}
p_{Y}\brak{k}= \begin{cases} 
      \frac{1}{3} & {k=0} \\
      \frac{2}{3 }& {k=1} 
   \end{cases}
   \\
p_{Y|X}\brak{0|0} = \frac{19}{25}\, 
p_{Y|X}\brak{0|1} = \frac{6}{25}\,
p_{Y|X}\brak{1|0} = \frac{45}{50}\,
p_{Y|X}\brak{1|2} = \frac{5}{50}
\end{align}
The desired probability is the probability that a slip drawn at random is marked other than Rs 1,
\begin{align}
&=1-p_X\brak{0}\\
&= p_X(1) + p_X(2)
\end{align}
Using Bayes theorem,
\begin{align}
&= p_Y\brak{0} \times \pr{Y=0 | X=1} + p_Y\brak{1} \times \pr{Y=1|X=2}\\
&=\frac{1}{3} \times \frac{6}{25} + \frac{2}{3} \times \frac{5}{50}\\
&=\frac{11}{75}
\end{align}

\newpage

%\tableofcontents

\bigskip

\renewcommand{\thefigure}{\theenumi}
\renewcommand{\thetable}{\theenumi}
%\renewcommand{\theequation}{\theenumi}

%\begin{abstract}
%%\boldmath
%In this letter, an algorithm for evaluating the exact analytical bit error rate  (BER)  for the piecewise linear (PL) combiner for  multiple relays is presented. Previous results were available only for upto three relays. The algorithm is unique in the sense that  the actual mathematical expressions, that are prohibitively large, need not be explicitly obtained. The diversity gain due to multiple relays is shown through plots of the analytical BER, well supported by simulations. 
%
%\end{abstract}
% IEEEtran.cls defaults to using nonbold math in the Abstract.
% This preserves the distinction between vectors and scalars. However,
% if the journal you are submitting to favors bold math in the abstract,
% then you can use LaTeX's standard command \boldmath at the very start
% of the abstract to achieve this. Many IEEE journals frown on math
% in the abstract anyway.

% Note that keywords are not normally used for peerreview papers.
%\begin{IEEEkeywords}
%Cooperative diversity, decode and forward, piecewise linear
%\end{IEEEkeywords}



% For peer review papers, you can put extra information on the cover
% page as needed:
% \ifCLASSOPTIONpeerreview
% \begin{center} \bfseries EDICS Category: 3-BBND \end{center}
% \fi
%
% For peerreview papers, this IEEEtran command inserts a page break and
% creates the second title. It will be ignored for other modes.
%\IEEEpeerreviewmaketitle




\item Four cards are drawn from a well-shuffled deck of 52 cards. What is the probability of obtaining 3 diamonds and one spade.
\\
\solution
		%\begin{enumerate}[label=\thesection.\arabic*,ref=\thesection.\theenumi]
	\item One card is drawn from a well-shuffled deck of 52 cards. Find the probability of getting
\begin{enumerate}
\item A king of red colour 
\item A face card 
\item A red face card
\item The jack of hearts
\item A spade
\item The queen of diamonds

\end{enumerate}
\solution
		%\input{ncert/10/15/1/14/main.tex}
	\item Five cards—the ten, jack, queen, king and ace of diamonds, are well-shuffled with their face downwards. One card is then picked up at random.
\begin{enumerate}
\item
What is the probability that the card is the queen? 
\item
If the queen is drawn and put aside, what is the probability that the second card picked up is (a) an ace? (b) a queen?\\
\end{enumerate}
\solution
		%\input{ncert/10/15/1/15/defs.tex}
	\item A bag contains $5$ red balls and some blue balls. If the probability of drawing a blue ball is double that if a red ball, determine the number of blue balls in the bag. 
		\\
\solution
		%\input{ncert/10/15/2/3/defs.tex}
	\item A card is selected from a pack of 52 cards.
 \begin{enumerate}[label=(\alph*)] 
                 \item How many points are there in the sample space?
                 \item Calculate the probability that the card is an ace of spades.
                 \item Calculate the probability that the card is (i) an ace and (ii) black card.
 \end{enumerate}
\solution
		%\input{ncert/11/16/3/4/main.tex}
\item Four cards are drawn from a well-shuffled deck of 52 cards. What is the probability of obtaining 3 diamonds and one spade.
\\
\solution
		%\input{ncert/11/16/4/2/defs.tex}
\item In a certain lottery 10,000 tickets are sold and ten equal prizes are awarded. What is the probability of not getting a prize if you buy (a) one ticket (b) two tickets (c) 10 tickets ?	
\\
\solution
		%\input{ncert/11/16/4/4/defs.tex}
		%
\item 
Out of 100 students, two sections of 40 and 60 are formed. If you and your friend are among the 100 students, what is the probability that
\begin{enumerate}
\item you both enter the same section?
\item you both enter the different sections?
\end{enumerate}
\solution
		%\input{ncert/11/16/4/5/defs.tex}
	\item 
The number lock of a suitcase has 4 wheels each labelled with ten digits i.e. from 0 to 9.The lock opens with a sequence of four digits with no repeats.What is the probability of a person getting the right sequence to open the suitcase.
\\
\solution
		%\input{ncert/11/16/4/10/defs.tex}
		%
\item 
Two cards are drawn at random and without replacement from a pack of 52 playing cards. Find the probability that both the cards are black.
\\
\solution
		%\input{ncert/12/13/2/2/defs.tex}
		\item A box of oranges is inspected by examining three randomly selected oranges drawn without replacement. If all the three oranges are good, the box is approved for sale, otherwise, it is rejected. Find the probability that a box containing 15 oranges out of which 12 are good and 3 are bad ones will be approved for sale.
		\label{ncert/12/13/2/3/defs.tex}
		\item Two balls are drawn at random with replacement from a box containing 10 black and 8 red balls. Find the probability that
		\label{ncert/12/13/2/12}
\begin{enumerate}
\item both balls are red.
\item first ball is black and second is red.
\item one of them is black and other is red.
\end{enumerate}

\item In a hostel, 60\% of the students read Hindi newspaper, 40\% read English newspaper and 20\% read both Hindi and English newspapers. A student is selected at random.
		\label{ncert/12/13/2/15}
\begin{enumerate}
\item Find the probability that she reads neither Hindi nor English newspapers.
\item If she reads Hindi newspaper, find the probability that she reads English newspaper.
\item If she reads English newspaper, find the probability that she reads Hindi newspaper.\\
\end{enumerate}
\item The probability of obtaining an even prime number on each die, when a pair of dice is rolled is 
\begin{enumerate}
    \item $0$ 
    
    \item $\frac{1}{3}$ 
    
    \item $\frac{1}{12}$ 
    
    \item $\frac{1}{36}$ 
\end{enumerate}
\solution
		%\input{ncert/12/13/2/17/defs.tex}
	\item A bag contains 4 red and 4 black balls, another bag contains 2 red and 6 black balls. One of the two bags is selected at random and a ball is drawn from the bag which is found to be red. Find the probability that the ball is drawn from the first bag.
\\
\solution
		%\input{ncert/12/13/3/2/main.tex}
  \item
  Cards with numbers 2 to 101 are placed in a box. A card is selected at random.Find the probability that the card has
\begin{enumerate}[label=(\roman*)]
	\item an even number 
	\item a square number
\end{enumerate}
\solution
%\input{exemplar/10/13/3/32/main.tex}
\item
The king, queen and jack of clubs are removed from a deck of 52 playing cards and then well shuffled. Now one card is drawn at random from the remaining cards.  Determine the probability that the card is
\begin{enumerate}[label=(\roman*)]
\item a club
\item 10 of hearts
\end{enumerate}
\solution
%\input{exemplar/10/13/3/29/main.tex}
\item A team of medical students doing their internship have to assist during surgeries
at a city hospital. The probabilities of surgeries rated as very complex, complex,
routine, simple or very simple are respectively, 0.15, 0.20, 0.31, 0.26, .08. Find
the probabilities that a particular surgery will be rated
\begin{enumerate}
	\item complex or very complex;
	\item neither very complex nor very simple;
	\item routine or complex
	\item routine or simple
\end{enumerate}
\solution
%\input{exemplar/11/16/3/8(1)/main.tex}
\item A card is selected from a pack of 52 cards.
\begin{enumerate}[label=(\alph*)]
    \item How many points are there in the sample space?
    \item Calculate the probability that the card is an ace of spades.
    \item Calculate the probability that the card is (i) an ace and (ii) black card.
\end{enumerate}
\solution
%\input{exemplar/11/16/3/4/main2.tex}
\item The probability that a non leap year selected at random will contain 53 sundays.
\\
\solution
%\input{exemplar/10/13/1/19/main.tex}
\item One of the four persons John, Rita, Aslam or Gurpreet will be promoted next
month. Consequently the sample space consists of four elementary outcomes
S = {John promoted, Rita promoted, Aslam promoted, Gurpreet promoted}
You are told that the chances of John’s promotion is same as that of Gurpreet,
Rita’s chances of promotion are twice as likely as Johns. Aslam’s chances are
four times that of John.
\begin{enumerate}
	\item Determine
	\begin{enumerate}
		\item P (John promoted)
		\item P (Rita promoted)
		\item P (Aslam promoted)
		\item P (Gurpreet promoted)
	\end{enumerate}
	\item If A = {John promoted or Gurpreet promoted}, find P (A).
\end{enumerate}
\solution
%\input{exemplar/11/16/3/10/main.tex}
\item A card is drawn from a deck of 52 cards. Find the probability of getting a king or a heart or a red card.\\
\solution
%\input{exemplar/11/16/3/15/main.tex}
\item The probability that a student will pass his examination is 0.73, the probability of
the student getting a compartment is 0.13, and the probability that the student will
either pass or get compartment is 0.96. State True or False.\\
\solution
%\input{exemplar/11/16/3/31/main.tex}
\item A card is selected from a pack of 52 cards\\
\begin{enumerate}[label=(\alph*)]
\item How many points are there in the sample space?
\item Calculate the probability that the cards is an ace of spades.
\item Calculate the probability that the card is (i) an ace (ii)black card.\\
\end{enumerate}
%\input{ncert/11/16/3/4_1/Prob_4.tex}
\item In a non-leap year, the probability of having 53 tuesdays or 53 wednesdays is\\
\solution
%\input{exemplar/11/16/3/18/main.tex}
\item There are 1000 sealed envelopes in a box, 10 of them contain a cash prize of
Rs 100 each, 100 of them contain a cash prize of Rs 50 each and 200 of them
contain a cash prize of Rs 10 each and rest do not contain any cash prize. If they
are well shuffled and an envelope is picked up out, what is the probability that it
contains no cash prize?\\
\solution
%\input{exemplar/10/13/3/34/main.tex}
\item 
A die is thrown and a card is selected at random from a deck of 52 playing cards. The probability of getting an even number on the die and a spade card.\\
\solution
%\input{exemplar/12/13/3/78/main.tex}
\item
If 4-digit numbers greater than 5,000 are randomly formed from the digits 0, 1, 3, 5, and 7, what is the probability of forming a number divisible by 5 when:
\begin{enumerate}
    \item The digits are repeated?
    \item The repetition of digits is not allowed?
\end{enumerate}
\solution
%\input{ncert/11/16/4/9/main.tex}
\item Consider the probability space $\brak{\Omega, \mathcal{G}, P}$ where $\Omega = [0,2]$ and $\mathcal{G} = \cbrak{\phi, \Omega, [0,1], (1,2]}$. Let $X$ and $Y$ be two functions on $\Omega$ defined as
\begin{align*}
    X(\omega) = 
    \begin{cases}
        1 & \text{if }\omega \in [0, 1]\\
        2 & \text{if }\omega \in (1, 2]
    \end{cases}
\end{align*}
and
\begin{align*}
    Y(\omega) = 
    \begin{cases}
        2 & \text{if }\omega \in [0, 1.5]\\
        3 & \text{if }\omega \in (1.5, 2].
    \end{cases}
\end{align*}
Then which one of the following statements is true?
\begin{enumerate}
    \item [(A)] $X$ is a random variable with respect to $\mathcal{G}$, but $Y$ is not a random variable with respect to $\mathcal{G}$.
    \item [(B)] $Y$ is a random variable with respect to $\mathcal{G}$, but $X$ is not a random variable with respect to $\mathcal{G}$.
    \item [(C)] Neither $X$ nor $Y$ is a random variable with respect to $\mathcal{G}$.
    \item [(D)] Both $X$ and $Y$ are random variables with respect to $\mathcal{G}$.
\end{enumerate} \hfill (GATE ST 2023)\\
\solution
%\input{gate/ST/2023/14/main.tex}
	\item  A die is loaded in such a way that each odd number is twice as likely to occur as
each even number. Find $P(G)$, where $G$ is the event that a number greater than
3 occurs on a single roll of the die.
\\
\solution
		%\input{exemplar/11/16/3/5/main.tex}
	\item All the jacks, queens and kings are removed from a deck of 52 playing cards. The remaining cards are well shuffled and then one card is drawn at random. Giving ace a value 1 similar value for other cards, find the probability that the card has a value 
		\begin{enumerate}
			\item 7
			\item greater than 7
			\item less than 7
		\end{enumerate}
		%\input{exemplar/10/13/3/30/main.tex}
  \item A Lot consists of 48 mobile phones of which 42 are good, 3 have only minor defects and 3 have major defects.Varnika will buy a phone if it is good but the trader will only buy a mobile if it has no major defects. One phone is selected at random from the lot. What is the probability that it is
\begin{enumerate}
	\item acceptable to Varnika?
            \item acceptable to the trader?
\end{enumerate}
\solution
	%\input{exemplar/10/13/3/40/main.tex}
 \item A student says that if you throw a die, it will show up 1 or not 1. Therefore, the probability of getting 1 and the probability of getting 'not 1' each is equal to $\frac{1}{2}$. Is this correct? Give reasons.\\
 \solution
        %\input{exemplar/10/13/2/9/main.tex}
   \item Four candidates A, B, C, D have ap-
plied for the assignment to coach a school cricket
team. If A is twice as likely to be selected as B, and
B and C are given about the same chance of being
selected, while C is twice as likely to be selected
as D, what are the probabilities that
\begin{enumerate}
\item C will be selected?
\item A will not be selected?
\end{enumerate}
	%\input{exemplar/11/16/3/9/main.tex}
 \item A bag contain 24 balls of which $x$ balls are red, $2x$ are white and $3x$ are blue. A ball is selected at random, What is the probability that it is
\begin{enumerate}[label=\alph*)]
\item not red ?
\item white ?
\end{enumerate}
%\input{exemplar/10/13/3/41/main.tex}
If the letters of the word ASSASSINATION are arranged at random. Find the Probability that
\begin{enumerate}[label=(\alph*)]
\item Four $S's$ come consecutively in the word
\item Two  $I's$ and two $N's$ come together
\item All $A's$ are not coming together
\item No two $A's$ are coming together
\end{enumerate}
%\input{exemplar/11/16/3/14/main.tex}
	\item One urn contains two black balls (labelled B1 and B2) and one white ball. A
	second urn contains one black ball and two white balls (labelled W1 and W2).
	Suppose the following experiment is performed. One of the two urns is chosen
	at random. Next a ball is randomly chosen from the urn. Then a second ball is
	chosen at random from the same urn without replacing the first ball.
	
	\begin{enumerate}
	\item What is the probability that two black balls are chosen?
	
	\item What is the probability that two balls of opposite colour are chosen?
	\end{enumerate}
	\solution
	%\input{exemplar/11/16/3/12/main1.tex}
\end{enumerate}

\item In a certain lottery 10,000 tickets are sold and ten equal prizes are awarded. What is the probability of not getting a prize if you buy (a) one ticket (b) two tickets (c) 10 tickets ?	
\\
\solution
		%\begin{enumerate}[label=\thesection.\arabic*,ref=\thesection.\theenumi]
	\item One card is drawn from a well-shuffled deck of 52 cards. Find the probability of getting
\begin{enumerate}
\item A king of red colour 
\item A face card 
\item A red face card
\item The jack of hearts
\item A spade
\item The queen of diamonds

\end{enumerate}
\solution
		%\input{ncert/10/15/1/14/main.tex}
	\item Five cards—the ten, jack, queen, king and ace of diamonds, are well-shuffled with their face downwards. One card is then picked up at random.
\begin{enumerate}
\item
What is the probability that the card is the queen? 
\item
If the queen is drawn and put aside, what is the probability that the second card picked up is (a) an ace? (b) a queen?\\
\end{enumerate}
\solution
		%\input{ncert/10/15/1/15/defs.tex}
	\item A bag contains $5$ red balls and some blue balls. If the probability of drawing a blue ball is double that if a red ball, determine the number of blue balls in the bag. 
		\\
\solution
		%\input{ncert/10/15/2/3/defs.tex}
	\item A card is selected from a pack of 52 cards.
 \begin{enumerate}[label=(\alph*)] 
                 \item How many points are there in the sample space?
                 \item Calculate the probability that the card is an ace of spades.
                 \item Calculate the probability that the card is (i) an ace and (ii) black card.
 \end{enumerate}
\solution
		%\input{ncert/11/16/3/4/main.tex}
\item Four cards are drawn from a well-shuffled deck of 52 cards. What is the probability of obtaining 3 diamonds and one spade.
\\
\solution
		%\input{ncert/11/16/4/2/defs.tex}
\item In a certain lottery 10,000 tickets are sold and ten equal prizes are awarded. What is the probability of not getting a prize if you buy (a) one ticket (b) two tickets (c) 10 tickets ?	
\\
\solution
		%\input{ncert/11/16/4/4/defs.tex}
		%
\item 
Out of 100 students, two sections of 40 and 60 are formed. If you and your friend are among the 100 students, what is the probability that
\begin{enumerate}
\item you both enter the same section?
\item you both enter the different sections?
\end{enumerate}
\solution
		%\input{ncert/11/16/4/5/defs.tex}
	\item 
The number lock of a suitcase has 4 wheels each labelled with ten digits i.e. from 0 to 9.The lock opens with a sequence of four digits with no repeats.What is the probability of a person getting the right sequence to open the suitcase.
\\
\solution
		%\input{ncert/11/16/4/10/defs.tex}
		%
\item 
Two cards are drawn at random and without replacement from a pack of 52 playing cards. Find the probability that both the cards are black.
\\
\solution
		%\input{ncert/12/13/2/2/defs.tex}
		\item A box of oranges is inspected by examining three randomly selected oranges drawn without replacement. If all the three oranges are good, the box is approved for sale, otherwise, it is rejected. Find the probability that a box containing 15 oranges out of which 12 are good and 3 are bad ones will be approved for sale.
		\label{ncert/12/13/2/3/defs.tex}
		\item Two balls are drawn at random with replacement from a box containing 10 black and 8 red balls. Find the probability that
		\label{ncert/12/13/2/12}
\begin{enumerate}
\item both balls are red.
\item first ball is black and second is red.
\item one of them is black and other is red.
\end{enumerate}

\item In a hostel, 60\% of the students read Hindi newspaper, 40\% read English newspaper and 20\% read both Hindi and English newspapers. A student is selected at random.
		\label{ncert/12/13/2/15}
\begin{enumerate}
\item Find the probability that she reads neither Hindi nor English newspapers.
\item If she reads Hindi newspaper, find the probability that she reads English newspaper.
\item If she reads English newspaper, find the probability that she reads Hindi newspaper.\\
\end{enumerate}
\item The probability of obtaining an even prime number on each die, when a pair of dice is rolled is 
\begin{enumerate}
    \item $0$ 
    
    \item $\frac{1}{3}$ 
    
    \item $\frac{1}{12}$ 
    
    \item $\frac{1}{36}$ 
\end{enumerate}
\solution
		%\input{ncert/12/13/2/17/defs.tex}
	\item A bag contains 4 red and 4 black balls, another bag contains 2 red and 6 black balls. One of the two bags is selected at random and a ball is drawn from the bag which is found to be red. Find the probability that the ball is drawn from the first bag.
\\
\solution
		%\input{ncert/12/13/3/2/main.tex}
  \item
  Cards with numbers 2 to 101 are placed in a box. A card is selected at random.Find the probability that the card has
\begin{enumerate}[label=(\roman*)]
	\item an even number 
	\item a square number
\end{enumerate}
\solution
%\input{exemplar/10/13/3/32/main.tex}
\item
The king, queen and jack of clubs are removed from a deck of 52 playing cards and then well shuffled. Now one card is drawn at random from the remaining cards.  Determine the probability that the card is
\begin{enumerate}[label=(\roman*)]
\item a club
\item 10 of hearts
\end{enumerate}
\solution
%\input{exemplar/10/13/3/29/main.tex}
\item A team of medical students doing their internship have to assist during surgeries
at a city hospital. The probabilities of surgeries rated as very complex, complex,
routine, simple or very simple are respectively, 0.15, 0.20, 0.31, 0.26, .08. Find
the probabilities that a particular surgery will be rated
\begin{enumerate}
	\item complex or very complex;
	\item neither very complex nor very simple;
	\item routine or complex
	\item routine or simple
\end{enumerate}
\solution
%\input{exemplar/11/16/3/8(1)/main.tex}
\item A card is selected from a pack of 52 cards.
\begin{enumerate}[label=(\alph*)]
    \item How many points are there in the sample space?
    \item Calculate the probability that the card is an ace of spades.
    \item Calculate the probability that the card is (i) an ace and (ii) black card.
\end{enumerate}
\solution
%\input{exemplar/11/16/3/4/main2.tex}
\item The probability that a non leap year selected at random will contain 53 sundays.
\\
\solution
%\input{exemplar/10/13/1/19/main.tex}
\item One of the four persons John, Rita, Aslam or Gurpreet will be promoted next
month. Consequently the sample space consists of four elementary outcomes
S = {John promoted, Rita promoted, Aslam promoted, Gurpreet promoted}
You are told that the chances of John’s promotion is same as that of Gurpreet,
Rita’s chances of promotion are twice as likely as Johns. Aslam’s chances are
four times that of John.
\begin{enumerate}
	\item Determine
	\begin{enumerate}
		\item P (John promoted)
		\item P (Rita promoted)
		\item P (Aslam promoted)
		\item P (Gurpreet promoted)
	\end{enumerate}
	\item If A = {John promoted or Gurpreet promoted}, find P (A).
\end{enumerate}
\solution
%\input{exemplar/11/16/3/10/main.tex}
\item A card is drawn from a deck of 52 cards. Find the probability of getting a king or a heart or a red card.\\
\solution
%\input{exemplar/11/16/3/15/main.tex}
\item The probability that a student will pass his examination is 0.73, the probability of
the student getting a compartment is 0.13, and the probability that the student will
either pass or get compartment is 0.96. State True or False.\\
\solution
%\input{exemplar/11/16/3/31/main.tex}
\item A card is selected from a pack of 52 cards\\
\begin{enumerate}[label=(\alph*)]
\item How many points are there in the sample space?
\item Calculate the probability that the cards is an ace of spades.
\item Calculate the probability that the card is (i) an ace (ii)black card.\\
\end{enumerate}
%\input{ncert/11/16/3/4_1/Prob_4.tex}
\item In a non-leap year, the probability of having 53 tuesdays or 53 wednesdays is\\
\solution
%\input{exemplar/11/16/3/18/main.tex}
\item There are 1000 sealed envelopes in a box, 10 of them contain a cash prize of
Rs 100 each, 100 of them contain a cash prize of Rs 50 each and 200 of them
contain a cash prize of Rs 10 each and rest do not contain any cash prize. If they
are well shuffled and an envelope is picked up out, what is the probability that it
contains no cash prize?\\
\solution
%\input{exemplar/10/13/3/34/main.tex}
\item 
A die is thrown and a card is selected at random from a deck of 52 playing cards. The probability of getting an even number on the die and a spade card.\\
\solution
%\input{exemplar/12/13/3/78/main.tex}
\item
If 4-digit numbers greater than 5,000 are randomly formed from the digits 0, 1, 3, 5, and 7, what is the probability of forming a number divisible by 5 when:
\begin{enumerate}
    \item The digits are repeated?
    \item The repetition of digits is not allowed?
\end{enumerate}
\solution
%\input{ncert/11/16/4/9/main.tex}
\item Consider the probability space $\brak{\Omega, \mathcal{G}, P}$ where $\Omega = [0,2]$ and $\mathcal{G} = \cbrak{\phi, \Omega, [0,1], (1,2]}$. Let $X$ and $Y$ be two functions on $\Omega$ defined as
\begin{align*}
    X(\omega) = 
    \begin{cases}
        1 & \text{if }\omega \in [0, 1]\\
        2 & \text{if }\omega \in (1, 2]
    \end{cases}
\end{align*}
and
\begin{align*}
    Y(\omega) = 
    \begin{cases}
        2 & \text{if }\omega \in [0, 1.5]\\
        3 & \text{if }\omega \in (1.5, 2].
    \end{cases}
\end{align*}
Then which one of the following statements is true?
\begin{enumerate}
    \item [(A)] $X$ is a random variable with respect to $\mathcal{G}$, but $Y$ is not a random variable with respect to $\mathcal{G}$.
    \item [(B)] $Y$ is a random variable with respect to $\mathcal{G}$, but $X$ is not a random variable with respect to $\mathcal{G}$.
    \item [(C)] Neither $X$ nor $Y$ is a random variable with respect to $\mathcal{G}$.
    \item [(D)] Both $X$ and $Y$ are random variables with respect to $\mathcal{G}$.
\end{enumerate} \hfill (GATE ST 2023)\\
\solution
%\input{gate/ST/2023/14/main.tex}
	\item  A die is loaded in such a way that each odd number is twice as likely to occur as
each even number. Find $P(G)$, where $G$ is the event that a number greater than
3 occurs on a single roll of the die.
\\
\solution
		%\input{exemplar/11/16/3/5/main.tex}
	\item All the jacks, queens and kings are removed from a deck of 52 playing cards. The remaining cards are well shuffled and then one card is drawn at random. Giving ace a value 1 similar value for other cards, find the probability that the card has a value 
		\begin{enumerate}
			\item 7
			\item greater than 7
			\item less than 7
		\end{enumerate}
		%\input{exemplar/10/13/3/30/main.tex}
  \item A Lot consists of 48 mobile phones of which 42 are good, 3 have only minor defects and 3 have major defects.Varnika will buy a phone if it is good but the trader will only buy a mobile if it has no major defects. One phone is selected at random from the lot. What is the probability that it is
\begin{enumerate}
	\item acceptable to Varnika?
            \item acceptable to the trader?
\end{enumerate}
\solution
	%\input{exemplar/10/13/3/40/main.tex}
 \item A student says that if you throw a die, it will show up 1 or not 1. Therefore, the probability of getting 1 and the probability of getting 'not 1' each is equal to $\frac{1}{2}$. Is this correct? Give reasons.\\
 \solution
        %\input{exemplar/10/13/2/9/main.tex}
   \item Four candidates A, B, C, D have ap-
plied for the assignment to coach a school cricket
team. If A is twice as likely to be selected as B, and
B and C are given about the same chance of being
selected, while C is twice as likely to be selected
as D, what are the probabilities that
\begin{enumerate}
\item C will be selected?
\item A will not be selected?
\end{enumerate}
	%\input{exemplar/11/16/3/9/main.tex}
 \item A bag contain 24 balls of which $x$ balls are red, $2x$ are white and $3x$ are blue. A ball is selected at random, What is the probability that it is
\begin{enumerate}[label=\alph*)]
\item not red ?
\item white ?
\end{enumerate}
%\input{exemplar/10/13/3/41/main.tex}
If the letters of the word ASSASSINATION are arranged at random. Find the Probability that
\begin{enumerate}[label=(\alph*)]
\item Four $S's$ come consecutively in the word
\item Two  $I's$ and two $N's$ come together
\item All $A's$ are not coming together
\item No two $A's$ are coming together
\end{enumerate}
%\input{exemplar/11/16/3/14/main.tex}
	\item One urn contains two black balls (labelled B1 and B2) and one white ball. A
	second urn contains one black ball and two white balls (labelled W1 and W2).
	Suppose the following experiment is performed. One of the two urns is chosen
	at random. Next a ball is randomly chosen from the urn. Then a second ball is
	chosen at random from the same urn without replacing the first ball.
	
	\begin{enumerate}
	\item What is the probability that two black balls are chosen?
	
	\item What is the probability that two balls of opposite colour are chosen?
	\end{enumerate}
	\solution
	%\input{exemplar/11/16/3/12/main1.tex}
\end{enumerate}

		%
\item 
Out of 100 students, two sections of 40 and 60 are formed. If you and your friend are among the 100 students, what is the probability that
\begin{enumerate}
\item you both enter the same section?
\item you both enter the different sections?
\end{enumerate}
\solution
		%\begin{enumerate}[label=\thesection.\arabic*,ref=\thesection.\theenumi]
	\item One card is drawn from a well-shuffled deck of 52 cards. Find the probability of getting
\begin{enumerate}
\item A king of red colour 
\item A face card 
\item A red face card
\item The jack of hearts
\item A spade
\item The queen of diamonds

\end{enumerate}
\solution
		%\input{ncert/10/15/1/14/main.tex}
	\item Five cards—the ten, jack, queen, king and ace of diamonds, are well-shuffled with their face downwards. One card is then picked up at random.
\begin{enumerate}
\item
What is the probability that the card is the queen? 
\item
If the queen is drawn and put aside, what is the probability that the second card picked up is (a) an ace? (b) a queen?\\
\end{enumerate}
\solution
		%\input{ncert/10/15/1/15/defs.tex}
	\item A bag contains $5$ red balls and some blue balls. If the probability of drawing a blue ball is double that if a red ball, determine the number of blue balls in the bag. 
		\\
\solution
		%\input{ncert/10/15/2/3/defs.tex}
	\item A card is selected from a pack of 52 cards.
 \begin{enumerate}[label=(\alph*)] 
                 \item How many points are there in the sample space?
                 \item Calculate the probability that the card is an ace of spades.
                 \item Calculate the probability that the card is (i) an ace and (ii) black card.
 \end{enumerate}
\solution
		%\input{ncert/11/16/3/4/main.tex}
\item Four cards are drawn from a well-shuffled deck of 52 cards. What is the probability of obtaining 3 diamonds and one spade.
\\
\solution
		%\input{ncert/11/16/4/2/defs.tex}
\item In a certain lottery 10,000 tickets are sold and ten equal prizes are awarded. What is the probability of not getting a prize if you buy (a) one ticket (b) two tickets (c) 10 tickets ?	
\\
\solution
		%\input{ncert/11/16/4/4/defs.tex}
		%
\item 
Out of 100 students, two sections of 40 and 60 are formed. If you and your friend are among the 100 students, what is the probability that
\begin{enumerate}
\item you both enter the same section?
\item you both enter the different sections?
\end{enumerate}
\solution
		%\input{ncert/11/16/4/5/defs.tex}
	\item 
The number lock of a suitcase has 4 wheels each labelled with ten digits i.e. from 0 to 9.The lock opens with a sequence of four digits with no repeats.What is the probability of a person getting the right sequence to open the suitcase.
\\
\solution
		%\input{ncert/11/16/4/10/defs.tex}
		%
\item 
Two cards are drawn at random and without replacement from a pack of 52 playing cards. Find the probability that both the cards are black.
\\
\solution
		%\input{ncert/12/13/2/2/defs.tex}
		\item A box of oranges is inspected by examining three randomly selected oranges drawn without replacement. If all the three oranges are good, the box is approved for sale, otherwise, it is rejected. Find the probability that a box containing 15 oranges out of which 12 are good and 3 are bad ones will be approved for sale.
		\label{ncert/12/13/2/3/defs.tex}
		\item Two balls are drawn at random with replacement from a box containing 10 black and 8 red balls. Find the probability that
		\label{ncert/12/13/2/12}
\begin{enumerate}
\item both balls are red.
\item first ball is black and second is red.
\item one of them is black and other is red.
\end{enumerate}

\item In a hostel, 60\% of the students read Hindi newspaper, 40\% read English newspaper and 20\% read both Hindi and English newspapers. A student is selected at random.
		\label{ncert/12/13/2/15}
\begin{enumerate}
\item Find the probability that she reads neither Hindi nor English newspapers.
\item If she reads Hindi newspaper, find the probability that she reads English newspaper.
\item If she reads English newspaper, find the probability that she reads Hindi newspaper.\\
\end{enumerate}
\item The probability of obtaining an even prime number on each die, when a pair of dice is rolled is 
\begin{enumerate}
    \item $0$ 
    
    \item $\frac{1}{3}$ 
    
    \item $\frac{1}{12}$ 
    
    \item $\frac{1}{36}$ 
\end{enumerate}
\solution
		%\input{ncert/12/13/2/17/defs.tex}
	\item A bag contains 4 red and 4 black balls, another bag contains 2 red and 6 black balls. One of the two bags is selected at random and a ball is drawn from the bag which is found to be red. Find the probability that the ball is drawn from the first bag.
\\
\solution
		%\input{ncert/12/13/3/2/main.tex}
  \item
  Cards with numbers 2 to 101 are placed in a box. A card is selected at random.Find the probability that the card has
\begin{enumerate}[label=(\roman*)]
	\item an even number 
	\item a square number
\end{enumerate}
\solution
%\input{exemplar/10/13/3/32/main.tex}
\item
The king, queen and jack of clubs are removed from a deck of 52 playing cards and then well shuffled. Now one card is drawn at random from the remaining cards.  Determine the probability that the card is
\begin{enumerate}[label=(\roman*)]
\item a club
\item 10 of hearts
\end{enumerate}
\solution
%\input{exemplar/10/13/3/29/main.tex}
\item A team of medical students doing their internship have to assist during surgeries
at a city hospital. The probabilities of surgeries rated as very complex, complex,
routine, simple or very simple are respectively, 0.15, 0.20, 0.31, 0.26, .08. Find
the probabilities that a particular surgery will be rated
\begin{enumerate}
	\item complex or very complex;
	\item neither very complex nor very simple;
	\item routine or complex
	\item routine or simple
\end{enumerate}
\solution
%\input{exemplar/11/16/3/8(1)/main.tex}
\item A card is selected from a pack of 52 cards.
\begin{enumerate}[label=(\alph*)]
    \item How many points are there in the sample space?
    \item Calculate the probability that the card is an ace of spades.
    \item Calculate the probability that the card is (i) an ace and (ii) black card.
\end{enumerate}
\solution
%\input{exemplar/11/16/3/4/main2.tex}
\item The probability that a non leap year selected at random will contain 53 sundays.
\\
\solution
%\input{exemplar/10/13/1/19/main.tex}
\item One of the four persons John, Rita, Aslam or Gurpreet will be promoted next
month. Consequently the sample space consists of four elementary outcomes
S = {John promoted, Rita promoted, Aslam promoted, Gurpreet promoted}
You are told that the chances of John’s promotion is same as that of Gurpreet,
Rita’s chances of promotion are twice as likely as Johns. Aslam’s chances are
four times that of John.
\begin{enumerate}
	\item Determine
	\begin{enumerate}
		\item P (John promoted)
		\item P (Rita promoted)
		\item P (Aslam promoted)
		\item P (Gurpreet promoted)
	\end{enumerate}
	\item If A = {John promoted or Gurpreet promoted}, find P (A).
\end{enumerate}
\solution
%\input{exemplar/11/16/3/10/main.tex}
\item A card is drawn from a deck of 52 cards. Find the probability of getting a king or a heart or a red card.\\
\solution
%\input{exemplar/11/16/3/15/main.tex}
\item The probability that a student will pass his examination is 0.73, the probability of
the student getting a compartment is 0.13, and the probability that the student will
either pass or get compartment is 0.96. State True or False.\\
\solution
%\input{exemplar/11/16/3/31/main.tex}
\item A card is selected from a pack of 52 cards\\
\begin{enumerate}[label=(\alph*)]
\item How many points are there in the sample space?
\item Calculate the probability that the cards is an ace of spades.
\item Calculate the probability that the card is (i) an ace (ii)black card.\\
\end{enumerate}
%\input{ncert/11/16/3/4_1/Prob_4.tex}
\item In a non-leap year, the probability of having 53 tuesdays or 53 wednesdays is\\
\solution
%\input{exemplar/11/16/3/18/main.tex}
\item There are 1000 sealed envelopes in a box, 10 of them contain a cash prize of
Rs 100 each, 100 of them contain a cash prize of Rs 50 each and 200 of them
contain a cash prize of Rs 10 each and rest do not contain any cash prize. If they
are well shuffled and an envelope is picked up out, what is the probability that it
contains no cash prize?\\
\solution
%\input{exemplar/10/13/3/34/main.tex}
\item 
A die is thrown and a card is selected at random from a deck of 52 playing cards. The probability of getting an even number on the die and a spade card.\\
\solution
%\input{exemplar/12/13/3/78/main.tex}
\item
If 4-digit numbers greater than 5,000 are randomly formed from the digits 0, 1, 3, 5, and 7, what is the probability of forming a number divisible by 5 when:
\begin{enumerate}
    \item The digits are repeated?
    \item The repetition of digits is not allowed?
\end{enumerate}
\solution
%\input{ncert/11/16/4/9/main.tex}
\item Consider the probability space $\brak{\Omega, \mathcal{G}, P}$ where $\Omega = [0,2]$ and $\mathcal{G} = \cbrak{\phi, \Omega, [0,1], (1,2]}$. Let $X$ and $Y$ be two functions on $\Omega$ defined as
\begin{align*}
    X(\omega) = 
    \begin{cases}
        1 & \text{if }\omega \in [0, 1]\\
        2 & \text{if }\omega \in (1, 2]
    \end{cases}
\end{align*}
and
\begin{align*}
    Y(\omega) = 
    \begin{cases}
        2 & \text{if }\omega \in [0, 1.5]\\
        3 & \text{if }\omega \in (1.5, 2].
    \end{cases}
\end{align*}
Then which one of the following statements is true?
\begin{enumerate}
    \item [(A)] $X$ is a random variable with respect to $\mathcal{G}$, but $Y$ is not a random variable with respect to $\mathcal{G}$.
    \item [(B)] $Y$ is a random variable with respect to $\mathcal{G}$, but $X$ is not a random variable with respect to $\mathcal{G}$.
    \item [(C)] Neither $X$ nor $Y$ is a random variable with respect to $\mathcal{G}$.
    \item [(D)] Both $X$ and $Y$ are random variables with respect to $\mathcal{G}$.
\end{enumerate} \hfill (GATE ST 2023)\\
\solution
%\input{gate/ST/2023/14/main.tex}
	\item  A die is loaded in such a way that each odd number is twice as likely to occur as
each even number. Find $P(G)$, where $G$ is the event that a number greater than
3 occurs on a single roll of the die.
\\
\solution
		%\input{exemplar/11/16/3/5/main.tex}
	\item All the jacks, queens and kings are removed from a deck of 52 playing cards. The remaining cards are well shuffled and then one card is drawn at random. Giving ace a value 1 similar value for other cards, find the probability that the card has a value 
		\begin{enumerate}
			\item 7
			\item greater than 7
			\item less than 7
		\end{enumerate}
		%\input{exemplar/10/13/3/30/main.tex}
  \item A Lot consists of 48 mobile phones of which 42 are good, 3 have only minor defects and 3 have major defects.Varnika will buy a phone if it is good but the trader will only buy a mobile if it has no major defects. One phone is selected at random from the lot. What is the probability that it is
\begin{enumerate}
	\item acceptable to Varnika?
            \item acceptable to the trader?
\end{enumerate}
\solution
	%\input{exemplar/10/13/3/40/main.tex}
 \item A student says that if you throw a die, it will show up 1 or not 1. Therefore, the probability of getting 1 and the probability of getting 'not 1' each is equal to $\frac{1}{2}$. Is this correct? Give reasons.\\
 \solution
        %\input{exemplar/10/13/2/9/main.tex}
   \item Four candidates A, B, C, D have ap-
plied for the assignment to coach a school cricket
team. If A is twice as likely to be selected as B, and
B and C are given about the same chance of being
selected, while C is twice as likely to be selected
as D, what are the probabilities that
\begin{enumerate}
\item C will be selected?
\item A will not be selected?
\end{enumerate}
	%\input{exemplar/11/16/3/9/main.tex}
 \item A bag contain 24 balls of which $x$ balls are red, $2x$ are white and $3x$ are blue. A ball is selected at random, What is the probability that it is
\begin{enumerate}[label=\alph*)]
\item not red ?
\item white ?
\end{enumerate}
%\input{exemplar/10/13/3/41/main.tex}
If the letters of the word ASSASSINATION are arranged at random. Find the Probability that
\begin{enumerate}[label=(\alph*)]
\item Four $S's$ come consecutively in the word
\item Two  $I's$ and two $N's$ come together
\item All $A's$ are not coming together
\item No two $A's$ are coming together
\end{enumerate}
%\input{exemplar/11/16/3/14/main.tex}
	\item One urn contains two black balls (labelled B1 and B2) and one white ball. A
	second urn contains one black ball and two white balls (labelled W1 and W2).
	Suppose the following experiment is performed. One of the two urns is chosen
	at random. Next a ball is randomly chosen from the urn. Then a second ball is
	chosen at random from the same urn without replacing the first ball.
	
	\begin{enumerate}
	\item What is the probability that two black balls are chosen?
	
	\item What is the probability that two balls of opposite colour are chosen?
	\end{enumerate}
	\solution
	%\input{exemplar/11/16/3/12/main1.tex}
\end{enumerate}

	\item 
The number lock of a suitcase has 4 wheels each labelled with ten digits i.e. from 0 to 9.The lock opens with a sequence of four digits with no repeats.What is the probability of a person getting the right sequence to open the suitcase.
\\
\solution
		%\begin{enumerate}[label=\thesection.\arabic*,ref=\thesection.\theenumi]
	\item One card is drawn from a well-shuffled deck of 52 cards. Find the probability of getting
\begin{enumerate}
\item A king of red colour 
\item A face card 
\item A red face card
\item The jack of hearts
\item A spade
\item The queen of diamonds

\end{enumerate}
\solution
		%\input{ncert/10/15/1/14/main.tex}
	\item Five cards—the ten, jack, queen, king and ace of diamonds, are well-shuffled with their face downwards. One card is then picked up at random.
\begin{enumerate}
\item
What is the probability that the card is the queen? 
\item
If the queen is drawn and put aside, what is the probability that the second card picked up is (a) an ace? (b) a queen?\\
\end{enumerate}
\solution
		%\input{ncert/10/15/1/15/defs.tex}
	\item A bag contains $5$ red balls and some blue balls. If the probability of drawing a blue ball is double that if a red ball, determine the number of blue balls in the bag. 
		\\
\solution
		%\input{ncert/10/15/2/3/defs.tex}
	\item A card is selected from a pack of 52 cards.
 \begin{enumerate}[label=(\alph*)] 
                 \item How many points are there in the sample space?
                 \item Calculate the probability that the card is an ace of spades.
                 \item Calculate the probability that the card is (i) an ace and (ii) black card.
 \end{enumerate}
\solution
		%\input{ncert/11/16/3/4/main.tex}
\item Four cards are drawn from a well-shuffled deck of 52 cards. What is the probability of obtaining 3 diamonds and one spade.
\\
\solution
		%\input{ncert/11/16/4/2/defs.tex}
\item In a certain lottery 10,000 tickets are sold and ten equal prizes are awarded. What is the probability of not getting a prize if you buy (a) one ticket (b) two tickets (c) 10 tickets ?	
\\
\solution
		%\input{ncert/11/16/4/4/defs.tex}
		%
\item 
Out of 100 students, two sections of 40 and 60 are formed. If you and your friend are among the 100 students, what is the probability that
\begin{enumerate}
\item you both enter the same section?
\item you both enter the different sections?
\end{enumerate}
\solution
		%\input{ncert/11/16/4/5/defs.tex}
	\item 
The number lock of a suitcase has 4 wheels each labelled with ten digits i.e. from 0 to 9.The lock opens with a sequence of four digits with no repeats.What is the probability of a person getting the right sequence to open the suitcase.
\\
\solution
		%\input{ncert/11/16/4/10/defs.tex}
		%
\item 
Two cards are drawn at random and without replacement from a pack of 52 playing cards. Find the probability that both the cards are black.
\\
\solution
		%\input{ncert/12/13/2/2/defs.tex}
		\item A box of oranges is inspected by examining three randomly selected oranges drawn without replacement. If all the three oranges are good, the box is approved for sale, otherwise, it is rejected. Find the probability that a box containing 15 oranges out of which 12 are good and 3 are bad ones will be approved for sale.
		\label{ncert/12/13/2/3/defs.tex}
		\item Two balls are drawn at random with replacement from a box containing 10 black and 8 red balls. Find the probability that
		\label{ncert/12/13/2/12}
\begin{enumerate}
\item both balls are red.
\item first ball is black and second is red.
\item one of them is black and other is red.
\end{enumerate}

\item In a hostel, 60\% of the students read Hindi newspaper, 40\% read English newspaper and 20\% read both Hindi and English newspapers. A student is selected at random.
		\label{ncert/12/13/2/15}
\begin{enumerate}
\item Find the probability that she reads neither Hindi nor English newspapers.
\item If she reads Hindi newspaper, find the probability that she reads English newspaper.
\item If she reads English newspaper, find the probability that she reads Hindi newspaper.\\
\end{enumerate}
\item The probability of obtaining an even prime number on each die, when a pair of dice is rolled is 
\begin{enumerate}
    \item $0$ 
    
    \item $\frac{1}{3}$ 
    
    \item $\frac{1}{12}$ 
    
    \item $\frac{1}{36}$ 
\end{enumerate}
\solution
		%\input{ncert/12/13/2/17/defs.tex}
	\item A bag contains 4 red and 4 black balls, another bag contains 2 red and 6 black balls. One of the two bags is selected at random and a ball is drawn from the bag which is found to be red. Find the probability that the ball is drawn from the first bag.
\\
\solution
		%\input{ncert/12/13/3/2/main.tex}
  \item
  Cards with numbers 2 to 101 are placed in a box. A card is selected at random.Find the probability that the card has
\begin{enumerate}[label=(\roman*)]
	\item an even number 
	\item a square number
\end{enumerate}
\solution
%\input{exemplar/10/13/3/32/main.tex}
\item
The king, queen and jack of clubs are removed from a deck of 52 playing cards and then well shuffled. Now one card is drawn at random from the remaining cards.  Determine the probability that the card is
\begin{enumerate}[label=(\roman*)]
\item a club
\item 10 of hearts
\end{enumerate}
\solution
%\input{exemplar/10/13/3/29/main.tex}
\item A team of medical students doing their internship have to assist during surgeries
at a city hospital. The probabilities of surgeries rated as very complex, complex,
routine, simple or very simple are respectively, 0.15, 0.20, 0.31, 0.26, .08. Find
the probabilities that a particular surgery will be rated
\begin{enumerate}
	\item complex or very complex;
	\item neither very complex nor very simple;
	\item routine or complex
	\item routine or simple
\end{enumerate}
\solution
%\input{exemplar/11/16/3/8(1)/main.tex}
\item A card is selected from a pack of 52 cards.
\begin{enumerate}[label=(\alph*)]
    \item How many points are there in the sample space?
    \item Calculate the probability that the card is an ace of spades.
    \item Calculate the probability that the card is (i) an ace and (ii) black card.
\end{enumerate}
\solution
%\input{exemplar/11/16/3/4/main2.tex}
\item The probability that a non leap year selected at random will contain 53 sundays.
\\
\solution
%\input{exemplar/10/13/1/19/main.tex}
\item One of the four persons John, Rita, Aslam or Gurpreet will be promoted next
month. Consequently the sample space consists of four elementary outcomes
S = {John promoted, Rita promoted, Aslam promoted, Gurpreet promoted}
You are told that the chances of John’s promotion is same as that of Gurpreet,
Rita’s chances of promotion are twice as likely as Johns. Aslam’s chances are
four times that of John.
\begin{enumerate}
	\item Determine
	\begin{enumerate}
		\item P (John promoted)
		\item P (Rita promoted)
		\item P (Aslam promoted)
		\item P (Gurpreet promoted)
	\end{enumerate}
	\item If A = {John promoted or Gurpreet promoted}, find P (A).
\end{enumerate}
\solution
%\input{exemplar/11/16/3/10/main.tex}
\item A card is drawn from a deck of 52 cards. Find the probability of getting a king or a heart or a red card.\\
\solution
%\input{exemplar/11/16/3/15/main.tex}
\item The probability that a student will pass his examination is 0.73, the probability of
the student getting a compartment is 0.13, and the probability that the student will
either pass or get compartment is 0.96. State True or False.\\
\solution
%\input{exemplar/11/16/3/31/main.tex}
\item A card is selected from a pack of 52 cards\\
\begin{enumerate}[label=(\alph*)]
\item How many points are there in the sample space?
\item Calculate the probability that the cards is an ace of spades.
\item Calculate the probability that the card is (i) an ace (ii)black card.\\
\end{enumerate}
%\input{ncert/11/16/3/4_1/Prob_4.tex}
\item In a non-leap year, the probability of having 53 tuesdays or 53 wednesdays is\\
\solution
%\input{exemplar/11/16/3/18/main.tex}
\item There are 1000 sealed envelopes in a box, 10 of them contain a cash prize of
Rs 100 each, 100 of them contain a cash prize of Rs 50 each and 200 of them
contain a cash prize of Rs 10 each and rest do not contain any cash prize. If they
are well shuffled and an envelope is picked up out, what is the probability that it
contains no cash prize?\\
\solution
%\input{exemplar/10/13/3/34/main.tex}
\item 
A die is thrown and a card is selected at random from a deck of 52 playing cards. The probability of getting an even number on the die and a spade card.\\
\solution
%\input{exemplar/12/13/3/78/main.tex}
\item
If 4-digit numbers greater than 5,000 are randomly formed from the digits 0, 1, 3, 5, and 7, what is the probability of forming a number divisible by 5 when:
\begin{enumerate}
    \item The digits are repeated?
    \item The repetition of digits is not allowed?
\end{enumerate}
\solution
%\input{ncert/11/16/4/9/main.tex}
\item Consider the probability space $\brak{\Omega, \mathcal{G}, P}$ where $\Omega = [0,2]$ and $\mathcal{G} = \cbrak{\phi, \Omega, [0,1], (1,2]}$. Let $X$ and $Y$ be two functions on $\Omega$ defined as
\begin{align*}
    X(\omega) = 
    \begin{cases}
        1 & \text{if }\omega \in [0, 1]\\
        2 & \text{if }\omega \in (1, 2]
    \end{cases}
\end{align*}
and
\begin{align*}
    Y(\omega) = 
    \begin{cases}
        2 & \text{if }\omega \in [0, 1.5]\\
        3 & \text{if }\omega \in (1.5, 2].
    \end{cases}
\end{align*}
Then which one of the following statements is true?
\begin{enumerate}
    \item [(A)] $X$ is a random variable with respect to $\mathcal{G}$, but $Y$ is not a random variable with respect to $\mathcal{G}$.
    \item [(B)] $Y$ is a random variable with respect to $\mathcal{G}$, but $X$ is not a random variable with respect to $\mathcal{G}$.
    \item [(C)] Neither $X$ nor $Y$ is a random variable with respect to $\mathcal{G}$.
    \item [(D)] Both $X$ and $Y$ are random variables with respect to $\mathcal{G}$.
\end{enumerate} \hfill (GATE ST 2023)\\
\solution
%\input{gate/ST/2023/14/main.tex}
	\item  A die is loaded in such a way that each odd number is twice as likely to occur as
each even number. Find $P(G)$, where $G$ is the event that a number greater than
3 occurs on a single roll of the die.
\\
\solution
		%\input{exemplar/11/16/3/5/main.tex}
	\item All the jacks, queens and kings are removed from a deck of 52 playing cards. The remaining cards are well shuffled and then one card is drawn at random. Giving ace a value 1 similar value for other cards, find the probability that the card has a value 
		\begin{enumerate}
			\item 7
			\item greater than 7
			\item less than 7
		\end{enumerate}
		%\input{exemplar/10/13/3/30/main.tex}
  \item A Lot consists of 48 mobile phones of which 42 are good, 3 have only minor defects and 3 have major defects.Varnika will buy a phone if it is good but the trader will only buy a mobile if it has no major defects. One phone is selected at random from the lot. What is the probability that it is
\begin{enumerate}
	\item acceptable to Varnika?
            \item acceptable to the trader?
\end{enumerate}
\solution
	%\input{exemplar/10/13/3/40/main.tex}
 \item A student says that if you throw a die, it will show up 1 or not 1. Therefore, the probability of getting 1 and the probability of getting 'not 1' each is equal to $\frac{1}{2}$. Is this correct? Give reasons.\\
 \solution
        %\input{exemplar/10/13/2/9/main.tex}
   \item Four candidates A, B, C, D have ap-
plied for the assignment to coach a school cricket
team. If A is twice as likely to be selected as B, and
B and C are given about the same chance of being
selected, while C is twice as likely to be selected
as D, what are the probabilities that
\begin{enumerate}
\item C will be selected?
\item A will not be selected?
\end{enumerate}
	%\input{exemplar/11/16/3/9/main.tex}
 \item A bag contain 24 balls of which $x$ balls are red, $2x$ are white and $3x$ are blue. A ball is selected at random, What is the probability that it is
\begin{enumerate}[label=\alph*)]
\item not red ?
\item white ?
\end{enumerate}
%\input{exemplar/10/13/3/41/main.tex}
If the letters of the word ASSASSINATION are arranged at random. Find the Probability that
\begin{enumerate}[label=(\alph*)]
\item Four $S's$ come consecutively in the word
\item Two  $I's$ and two $N's$ come together
\item All $A's$ are not coming together
\item No two $A's$ are coming together
\end{enumerate}
%\input{exemplar/11/16/3/14/main.tex}
	\item One urn contains two black balls (labelled B1 and B2) and one white ball. A
	second urn contains one black ball and two white balls (labelled W1 and W2).
	Suppose the following experiment is performed. One of the two urns is chosen
	at random. Next a ball is randomly chosen from the urn. Then a second ball is
	chosen at random from the same urn without replacing the first ball.
	
	\begin{enumerate}
	\item What is the probability that two black balls are chosen?
	
	\item What is the probability that two balls of opposite colour are chosen?
	\end{enumerate}
	\solution
	%\input{exemplar/11/16/3/12/main1.tex}
\end{enumerate}

		%
\item 
Two cards are drawn at random and without replacement from a pack of 52 playing cards. Find the probability that both the cards are black.
\\
\solution
		%\begin{enumerate}[label=\thesection.\arabic*,ref=\thesection.\theenumi]
	\item One card is drawn from a well-shuffled deck of 52 cards. Find the probability of getting
\begin{enumerate}
\item A king of red colour 
\item A face card 
\item A red face card
\item The jack of hearts
\item A spade
\item The queen of diamonds

\end{enumerate}
\solution
		%\input{ncert/10/15/1/14/main.tex}
	\item Five cards—the ten, jack, queen, king and ace of diamonds, are well-shuffled with their face downwards. One card is then picked up at random.
\begin{enumerate}
\item
What is the probability that the card is the queen? 
\item
If the queen is drawn and put aside, what is the probability that the second card picked up is (a) an ace? (b) a queen?\\
\end{enumerate}
\solution
		%\input{ncert/10/15/1/15/defs.tex}
	\item A bag contains $5$ red balls and some blue balls. If the probability of drawing a blue ball is double that if a red ball, determine the number of blue balls in the bag. 
		\\
\solution
		%\input{ncert/10/15/2/3/defs.tex}
	\item A card is selected from a pack of 52 cards.
 \begin{enumerate}[label=(\alph*)] 
                 \item How many points are there in the sample space?
                 \item Calculate the probability that the card is an ace of spades.
                 \item Calculate the probability that the card is (i) an ace and (ii) black card.
 \end{enumerate}
\solution
		%\input{ncert/11/16/3/4/main.tex}
\item Four cards are drawn from a well-shuffled deck of 52 cards. What is the probability of obtaining 3 diamonds and one spade.
\\
\solution
		%\input{ncert/11/16/4/2/defs.tex}
\item In a certain lottery 10,000 tickets are sold and ten equal prizes are awarded. What is the probability of not getting a prize if you buy (a) one ticket (b) two tickets (c) 10 tickets ?	
\\
\solution
		%\input{ncert/11/16/4/4/defs.tex}
		%
\item 
Out of 100 students, two sections of 40 and 60 are formed. If you and your friend are among the 100 students, what is the probability that
\begin{enumerate}
\item you both enter the same section?
\item you both enter the different sections?
\end{enumerate}
\solution
		%\input{ncert/11/16/4/5/defs.tex}
	\item 
The number lock of a suitcase has 4 wheels each labelled with ten digits i.e. from 0 to 9.The lock opens with a sequence of four digits with no repeats.What is the probability of a person getting the right sequence to open the suitcase.
\\
\solution
		%\input{ncert/11/16/4/10/defs.tex}
		%
\item 
Two cards are drawn at random and without replacement from a pack of 52 playing cards. Find the probability that both the cards are black.
\\
\solution
		%\input{ncert/12/13/2/2/defs.tex}
		\item A box of oranges is inspected by examining three randomly selected oranges drawn without replacement. If all the three oranges are good, the box is approved for sale, otherwise, it is rejected. Find the probability that a box containing 15 oranges out of which 12 are good and 3 are bad ones will be approved for sale.
		\label{ncert/12/13/2/3/defs.tex}
		\item Two balls are drawn at random with replacement from a box containing 10 black and 8 red balls. Find the probability that
		\label{ncert/12/13/2/12}
\begin{enumerate}
\item both balls are red.
\item first ball is black and second is red.
\item one of them is black and other is red.
\end{enumerate}

\item In a hostel, 60\% of the students read Hindi newspaper, 40\% read English newspaper and 20\% read both Hindi and English newspapers. A student is selected at random.
		\label{ncert/12/13/2/15}
\begin{enumerate}
\item Find the probability that she reads neither Hindi nor English newspapers.
\item If she reads Hindi newspaper, find the probability that she reads English newspaper.
\item If she reads English newspaper, find the probability that she reads Hindi newspaper.\\
\end{enumerate}
\item The probability of obtaining an even prime number on each die, when a pair of dice is rolled is 
\begin{enumerate}
    \item $0$ 
    
    \item $\frac{1}{3}$ 
    
    \item $\frac{1}{12}$ 
    
    \item $\frac{1}{36}$ 
\end{enumerate}
\solution
		%\input{ncert/12/13/2/17/defs.tex}
	\item A bag contains 4 red and 4 black balls, another bag contains 2 red and 6 black balls. One of the two bags is selected at random and a ball is drawn from the bag which is found to be red. Find the probability that the ball is drawn from the first bag.
\\
\solution
		%\input{ncert/12/13/3/2/main.tex}
  \item
  Cards with numbers 2 to 101 are placed in a box. A card is selected at random.Find the probability that the card has
\begin{enumerate}[label=(\roman*)]
	\item an even number 
	\item a square number
\end{enumerate}
\solution
%\input{exemplar/10/13/3/32/main.tex}
\item
The king, queen and jack of clubs are removed from a deck of 52 playing cards and then well shuffled. Now one card is drawn at random from the remaining cards.  Determine the probability that the card is
\begin{enumerate}[label=(\roman*)]
\item a club
\item 10 of hearts
\end{enumerate}
\solution
%\input{exemplar/10/13/3/29/main.tex}
\item A team of medical students doing their internship have to assist during surgeries
at a city hospital. The probabilities of surgeries rated as very complex, complex,
routine, simple or very simple are respectively, 0.15, 0.20, 0.31, 0.26, .08. Find
the probabilities that a particular surgery will be rated
\begin{enumerate}
	\item complex or very complex;
	\item neither very complex nor very simple;
	\item routine or complex
	\item routine or simple
\end{enumerate}
\solution
%\input{exemplar/11/16/3/8(1)/main.tex}
\item A card is selected from a pack of 52 cards.
\begin{enumerate}[label=(\alph*)]
    \item How many points are there in the sample space?
    \item Calculate the probability that the card is an ace of spades.
    \item Calculate the probability that the card is (i) an ace and (ii) black card.
\end{enumerate}
\solution
%\input{exemplar/11/16/3/4/main2.tex}
\item The probability that a non leap year selected at random will contain 53 sundays.
\\
\solution
%\input{exemplar/10/13/1/19/main.tex}
\item One of the four persons John, Rita, Aslam or Gurpreet will be promoted next
month. Consequently the sample space consists of four elementary outcomes
S = {John promoted, Rita promoted, Aslam promoted, Gurpreet promoted}
You are told that the chances of John’s promotion is same as that of Gurpreet,
Rita’s chances of promotion are twice as likely as Johns. Aslam’s chances are
four times that of John.
\begin{enumerate}
	\item Determine
	\begin{enumerate}
		\item P (John promoted)
		\item P (Rita promoted)
		\item P (Aslam promoted)
		\item P (Gurpreet promoted)
	\end{enumerate}
	\item If A = {John promoted or Gurpreet promoted}, find P (A).
\end{enumerate}
\solution
%\input{exemplar/11/16/3/10/main.tex}
\item A card is drawn from a deck of 52 cards. Find the probability of getting a king or a heart or a red card.\\
\solution
%\input{exemplar/11/16/3/15/main.tex}
\item The probability that a student will pass his examination is 0.73, the probability of
the student getting a compartment is 0.13, and the probability that the student will
either pass or get compartment is 0.96. State True or False.\\
\solution
%\input{exemplar/11/16/3/31/main.tex}
\item A card is selected from a pack of 52 cards\\
\begin{enumerate}[label=(\alph*)]
\item How many points are there in the sample space?
\item Calculate the probability that the cards is an ace of spades.
\item Calculate the probability that the card is (i) an ace (ii)black card.\\
\end{enumerate}
%\input{ncert/11/16/3/4_1/Prob_4.tex}
\item In a non-leap year, the probability of having 53 tuesdays or 53 wednesdays is\\
\solution
%\input{exemplar/11/16/3/18/main.tex}
\item There are 1000 sealed envelopes in a box, 10 of them contain a cash prize of
Rs 100 each, 100 of them contain a cash prize of Rs 50 each and 200 of them
contain a cash prize of Rs 10 each and rest do not contain any cash prize. If they
are well shuffled and an envelope is picked up out, what is the probability that it
contains no cash prize?\\
\solution
%\input{exemplar/10/13/3/34/main.tex}
\item 
A die is thrown and a card is selected at random from a deck of 52 playing cards. The probability of getting an even number on the die and a spade card.\\
\solution
%\input{exemplar/12/13/3/78/main.tex}
\item
If 4-digit numbers greater than 5,000 are randomly formed from the digits 0, 1, 3, 5, and 7, what is the probability of forming a number divisible by 5 when:
\begin{enumerate}
    \item The digits are repeated?
    \item The repetition of digits is not allowed?
\end{enumerate}
\solution
%\input{ncert/11/16/4/9/main.tex}
\item Consider the probability space $\brak{\Omega, \mathcal{G}, P}$ where $\Omega = [0,2]$ and $\mathcal{G} = \cbrak{\phi, \Omega, [0,1], (1,2]}$. Let $X$ and $Y$ be two functions on $\Omega$ defined as
\begin{align*}
    X(\omega) = 
    \begin{cases}
        1 & \text{if }\omega \in [0, 1]\\
        2 & \text{if }\omega \in (1, 2]
    \end{cases}
\end{align*}
and
\begin{align*}
    Y(\omega) = 
    \begin{cases}
        2 & \text{if }\omega \in [0, 1.5]\\
        3 & \text{if }\omega \in (1.5, 2].
    \end{cases}
\end{align*}
Then which one of the following statements is true?
\begin{enumerate}
    \item [(A)] $X$ is a random variable with respect to $\mathcal{G}$, but $Y$ is not a random variable with respect to $\mathcal{G}$.
    \item [(B)] $Y$ is a random variable with respect to $\mathcal{G}$, but $X$ is not a random variable with respect to $\mathcal{G}$.
    \item [(C)] Neither $X$ nor $Y$ is a random variable with respect to $\mathcal{G}$.
    \item [(D)] Both $X$ and $Y$ are random variables with respect to $\mathcal{G}$.
\end{enumerate} \hfill (GATE ST 2023)\\
\solution
%\input{gate/ST/2023/14/main.tex}
	\item  A die is loaded in such a way that each odd number is twice as likely to occur as
each even number. Find $P(G)$, where $G$ is the event that a number greater than
3 occurs on a single roll of the die.
\\
\solution
		%\input{exemplar/11/16/3/5/main.tex}
	\item All the jacks, queens and kings are removed from a deck of 52 playing cards. The remaining cards are well shuffled and then one card is drawn at random. Giving ace a value 1 similar value for other cards, find the probability that the card has a value 
		\begin{enumerate}
			\item 7
			\item greater than 7
			\item less than 7
		\end{enumerate}
		%\input{exemplar/10/13/3/30/main.tex}
  \item A Lot consists of 48 mobile phones of which 42 are good, 3 have only minor defects and 3 have major defects.Varnika will buy a phone if it is good but the trader will only buy a mobile if it has no major defects. One phone is selected at random from the lot. What is the probability that it is
\begin{enumerate}
	\item acceptable to Varnika?
            \item acceptable to the trader?
\end{enumerate}
\solution
	%\input{exemplar/10/13/3/40/main.tex}
 \item A student says that if you throw a die, it will show up 1 or not 1. Therefore, the probability of getting 1 and the probability of getting 'not 1' each is equal to $\frac{1}{2}$. Is this correct? Give reasons.\\
 \solution
        %\input{exemplar/10/13/2/9/main.tex}
   \item Four candidates A, B, C, D have ap-
plied for the assignment to coach a school cricket
team. If A is twice as likely to be selected as B, and
B and C are given about the same chance of being
selected, while C is twice as likely to be selected
as D, what are the probabilities that
\begin{enumerate}
\item C will be selected?
\item A will not be selected?
\end{enumerate}
	%\input{exemplar/11/16/3/9/main.tex}
 \item A bag contain 24 balls of which $x$ balls are red, $2x$ are white and $3x$ are blue. A ball is selected at random, What is the probability that it is
\begin{enumerate}[label=\alph*)]
\item not red ?
\item white ?
\end{enumerate}
%\input{exemplar/10/13/3/41/main.tex}
If the letters of the word ASSASSINATION are arranged at random. Find the Probability that
\begin{enumerate}[label=(\alph*)]
\item Four $S's$ come consecutively in the word
\item Two  $I's$ and two $N's$ come together
\item All $A's$ are not coming together
\item No two $A's$ are coming together
\end{enumerate}
%\input{exemplar/11/16/3/14/main.tex}
	\item One urn contains two black balls (labelled B1 and B2) and one white ball. A
	second urn contains one black ball and two white balls (labelled W1 and W2).
	Suppose the following experiment is performed. One of the two urns is chosen
	at random. Next a ball is randomly chosen from the urn. Then a second ball is
	chosen at random from the same urn without replacing the first ball.
	
	\begin{enumerate}
	\item What is the probability that two black balls are chosen?
	
	\item What is the probability that two balls of opposite colour are chosen?
	\end{enumerate}
	\solution
	%\input{exemplar/11/16/3/12/main1.tex}
\end{enumerate}

		\item A box of oranges is inspected by examining three randomly selected oranges drawn without replacement. If all the three oranges are good, the box is approved for sale, otherwise, it is rejected. Find the probability that a box containing 15 oranges out of which 12 are good and 3 are bad ones will be approved for sale.
		\label{ncert/12/13/2/3/defs.tex}
		\item Two balls are drawn at random with replacement from a box containing 10 black and 8 red balls. Find the probability that
		\label{ncert/12/13/2/12}
\begin{enumerate}
\item both balls are red.
\item first ball is black and second is red.
\item one of them is black and other is red.
\end{enumerate}

\item In a hostel, 60\% of the students read Hindi newspaper, 40\% read English newspaper and 20\% read both Hindi and English newspapers. A student is selected at random.
		\label{ncert/12/13/2/15}
\begin{enumerate}
\item Find the probability that she reads neither Hindi nor English newspapers.
\item If she reads Hindi newspaper, find the probability that she reads English newspaper.
\item If she reads English newspaper, find the probability that she reads Hindi newspaper.\\
\end{enumerate}
\item The probability of obtaining an even prime number on each die, when a pair of dice is rolled is 
\begin{enumerate}
    \item $0$ 
    
    \item $\frac{1}{3}$ 
    
    \item $\frac{1}{12}$ 
    
    \item $\frac{1}{36}$ 
\end{enumerate}
\solution
		%\begin{enumerate}[label=\thesection.\arabic*,ref=\thesection.\theenumi]
	\item One card is drawn from a well-shuffled deck of 52 cards. Find the probability of getting
\begin{enumerate}
\item A king of red colour 
\item A face card 
\item A red face card
\item The jack of hearts
\item A spade
\item The queen of diamonds

\end{enumerate}
\solution
		%\input{ncert/10/15/1/14/main.tex}
	\item Five cards—the ten, jack, queen, king and ace of diamonds, are well-shuffled with their face downwards. One card is then picked up at random.
\begin{enumerate}
\item
What is the probability that the card is the queen? 
\item
If the queen is drawn and put aside, what is the probability that the second card picked up is (a) an ace? (b) a queen?\\
\end{enumerate}
\solution
		%\input{ncert/10/15/1/15/defs.tex}
	\item A bag contains $5$ red balls and some blue balls. If the probability of drawing a blue ball is double that if a red ball, determine the number of blue balls in the bag. 
		\\
\solution
		%\input{ncert/10/15/2/3/defs.tex}
	\item A card is selected from a pack of 52 cards.
 \begin{enumerate}[label=(\alph*)] 
                 \item How many points are there in the sample space?
                 \item Calculate the probability that the card is an ace of spades.
                 \item Calculate the probability that the card is (i) an ace and (ii) black card.
 \end{enumerate}
\solution
		%\input{ncert/11/16/3/4/main.tex}
\item Four cards are drawn from a well-shuffled deck of 52 cards. What is the probability of obtaining 3 diamonds and one spade.
\\
\solution
		%\input{ncert/11/16/4/2/defs.tex}
\item In a certain lottery 10,000 tickets are sold and ten equal prizes are awarded. What is the probability of not getting a prize if you buy (a) one ticket (b) two tickets (c) 10 tickets ?	
\\
\solution
		%\input{ncert/11/16/4/4/defs.tex}
		%
\item 
Out of 100 students, two sections of 40 and 60 are formed. If you and your friend are among the 100 students, what is the probability that
\begin{enumerate}
\item you both enter the same section?
\item you both enter the different sections?
\end{enumerate}
\solution
		%\input{ncert/11/16/4/5/defs.tex}
	\item 
The number lock of a suitcase has 4 wheels each labelled with ten digits i.e. from 0 to 9.The lock opens with a sequence of four digits with no repeats.What is the probability of a person getting the right sequence to open the suitcase.
\\
\solution
		%\input{ncert/11/16/4/10/defs.tex}
		%
\item 
Two cards are drawn at random and without replacement from a pack of 52 playing cards. Find the probability that both the cards are black.
\\
\solution
		%\input{ncert/12/13/2/2/defs.tex}
		\item A box of oranges is inspected by examining three randomly selected oranges drawn without replacement. If all the three oranges are good, the box is approved for sale, otherwise, it is rejected. Find the probability that a box containing 15 oranges out of which 12 are good and 3 are bad ones will be approved for sale.
		\label{ncert/12/13/2/3/defs.tex}
		\item Two balls are drawn at random with replacement from a box containing 10 black and 8 red balls. Find the probability that
		\label{ncert/12/13/2/12}
\begin{enumerate}
\item both balls are red.
\item first ball is black and second is red.
\item one of them is black and other is red.
\end{enumerate}

\item In a hostel, 60\% of the students read Hindi newspaper, 40\% read English newspaper and 20\% read both Hindi and English newspapers. A student is selected at random.
		\label{ncert/12/13/2/15}
\begin{enumerate}
\item Find the probability that she reads neither Hindi nor English newspapers.
\item If she reads Hindi newspaper, find the probability that she reads English newspaper.
\item If she reads English newspaper, find the probability that she reads Hindi newspaper.\\
\end{enumerate}
\item The probability of obtaining an even prime number on each die, when a pair of dice is rolled is 
\begin{enumerate}
    \item $0$ 
    
    \item $\frac{1}{3}$ 
    
    \item $\frac{1}{12}$ 
    
    \item $\frac{1}{36}$ 
\end{enumerate}
\solution
		%\input{ncert/12/13/2/17/defs.tex}
	\item A bag contains 4 red and 4 black balls, another bag contains 2 red and 6 black balls. One of the two bags is selected at random and a ball is drawn from the bag which is found to be red. Find the probability that the ball is drawn from the first bag.
\\
\solution
		%\input{ncert/12/13/3/2/main.tex}
  \item
  Cards with numbers 2 to 101 are placed in a box. A card is selected at random.Find the probability that the card has
\begin{enumerate}[label=(\roman*)]
	\item an even number 
	\item a square number
\end{enumerate}
\solution
%\input{exemplar/10/13/3/32/main.tex}
\item
The king, queen and jack of clubs are removed from a deck of 52 playing cards and then well shuffled. Now one card is drawn at random from the remaining cards.  Determine the probability that the card is
\begin{enumerate}[label=(\roman*)]
\item a club
\item 10 of hearts
\end{enumerate}
\solution
%\input{exemplar/10/13/3/29/main.tex}
\item A team of medical students doing their internship have to assist during surgeries
at a city hospital. The probabilities of surgeries rated as very complex, complex,
routine, simple or very simple are respectively, 0.15, 0.20, 0.31, 0.26, .08. Find
the probabilities that a particular surgery will be rated
\begin{enumerate}
	\item complex or very complex;
	\item neither very complex nor very simple;
	\item routine or complex
	\item routine or simple
\end{enumerate}
\solution
%\input{exemplar/11/16/3/8(1)/main.tex}
\item A card is selected from a pack of 52 cards.
\begin{enumerate}[label=(\alph*)]
    \item How many points are there in the sample space?
    \item Calculate the probability that the card is an ace of spades.
    \item Calculate the probability that the card is (i) an ace and (ii) black card.
\end{enumerate}
\solution
%\input{exemplar/11/16/3/4/main2.tex}
\item The probability that a non leap year selected at random will contain 53 sundays.
\\
\solution
%\input{exemplar/10/13/1/19/main.tex}
\item One of the four persons John, Rita, Aslam or Gurpreet will be promoted next
month. Consequently the sample space consists of four elementary outcomes
S = {John promoted, Rita promoted, Aslam promoted, Gurpreet promoted}
You are told that the chances of John’s promotion is same as that of Gurpreet,
Rita’s chances of promotion are twice as likely as Johns. Aslam’s chances are
four times that of John.
\begin{enumerate}
	\item Determine
	\begin{enumerate}
		\item P (John promoted)
		\item P (Rita promoted)
		\item P (Aslam promoted)
		\item P (Gurpreet promoted)
	\end{enumerate}
	\item If A = {John promoted or Gurpreet promoted}, find P (A).
\end{enumerate}
\solution
%\input{exemplar/11/16/3/10/main.tex}
\item A card is drawn from a deck of 52 cards. Find the probability of getting a king or a heart or a red card.\\
\solution
%\input{exemplar/11/16/3/15/main.tex}
\item The probability that a student will pass his examination is 0.73, the probability of
the student getting a compartment is 0.13, and the probability that the student will
either pass or get compartment is 0.96. State True or False.\\
\solution
%\input{exemplar/11/16/3/31/main.tex}
\item A card is selected from a pack of 52 cards\\
\begin{enumerate}[label=(\alph*)]
\item How many points are there in the sample space?
\item Calculate the probability that the cards is an ace of spades.
\item Calculate the probability that the card is (i) an ace (ii)black card.\\
\end{enumerate}
%\input{ncert/11/16/3/4_1/Prob_4.tex}
\item In a non-leap year, the probability of having 53 tuesdays or 53 wednesdays is\\
\solution
%\input{exemplar/11/16/3/18/main.tex}
\item There are 1000 sealed envelopes in a box, 10 of them contain a cash prize of
Rs 100 each, 100 of them contain a cash prize of Rs 50 each and 200 of them
contain a cash prize of Rs 10 each and rest do not contain any cash prize. If they
are well shuffled and an envelope is picked up out, what is the probability that it
contains no cash prize?\\
\solution
%\input{exemplar/10/13/3/34/main.tex}
\item 
A die is thrown and a card is selected at random from a deck of 52 playing cards. The probability of getting an even number on the die and a spade card.\\
\solution
%\input{exemplar/12/13/3/78/main.tex}
\item
If 4-digit numbers greater than 5,000 are randomly formed from the digits 0, 1, 3, 5, and 7, what is the probability of forming a number divisible by 5 when:
\begin{enumerate}
    \item The digits are repeated?
    \item The repetition of digits is not allowed?
\end{enumerate}
\solution
%\input{ncert/11/16/4/9/main.tex}
\item Consider the probability space $\brak{\Omega, \mathcal{G}, P}$ where $\Omega = [0,2]$ and $\mathcal{G} = \cbrak{\phi, \Omega, [0,1], (1,2]}$. Let $X$ and $Y$ be two functions on $\Omega$ defined as
\begin{align*}
    X(\omega) = 
    \begin{cases}
        1 & \text{if }\omega \in [0, 1]\\
        2 & \text{if }\omega \in (1, 2]
    \end{cases}
\end{align*}
and
\begin{align*}
    Y(\omega) = 
    \begin{cases}
        2 & \text{if }\omega \in [0, 1.5]\\
        3 & \text{if }\omega \in (1.5, 2].
    \end{cases}
\end{align*}
Then which one of the following statements is true?
\begin{enumerate}
    \item [(A)] $X$ is a random variable with respect to $\mathcal{G}$, but $Y$ is not a random variable with respect to $\mathcal{G}$.
    \item [(B)] $Y$ is a random variable with respect to $\mathcal{G}$, but $X$ is not a random variable with respect to $\mathcal{G}$.
    \item [(C)] Neither $X$ nor $Y$ is a random variable with respect to $\mathcal{G}$.
    \item [(D)] Both $X$ and $Y$ are random variables with respect to $\mathcal{G}$.
\end{enumerate} \hfill (GATE ST 2023)\\
\solution
%\input{gate/ST/2023/14/main.tex}
	\item  A die is loaded in such a way that each odd number is twice as likely to occur as
each even number. Find $P(G)$, where $G$ is the event that a number greater than
3 occurs on a single roll of the die.
\\
\solution
		%\input{exemplar/11/16/3/5/main.tex}
	\item All the jacks, queens and kings are removed from a deck of 52 playing cards. The remaining cards are well shuffled and then one card is drawn at random. Giving ace a value 1 similar value for other cards, find the probability that the card has a value 
		\begin{enumerate}
			\item 7
			\item greater than 7
			\item less than 7
		\end{enumerate}
		%\input{exemplar/10/13/3/30/main.tex}
  \item A Lot consists of 48 mobile phones of which 42 are good, 3 have only minor defects and 3 have major defects.Varnika will buy a phone if it is good but the trader will only buy a mobile if it has no major defects. One phone is selected at random from the lot. What is the probability that it is
\begin{enumerate}
	\item acceptable to Varnika?
            \item acceptable to the trader?
\end{enumerate}
\solution
	%\input{exemplar/10/13/3/40/main.tex}
 \item A student says that if you throw a die, it will show up 1 or not 1. Therefore, the probability of getting 1 and the probability of getting 'not 1' each is equal to $\frac{1}{2}$. Is this correct? Give reasons.\\
 \solution
        %\input{exemplar/10/13/2/9/main.tex}
   \item Four candidates A, B, C, D have ap-
plied for the assignment to coach a school cricket
team. If A is twice as likely to be selected as B, and
B and C are given about the same chance of being
selected, while C is twice as likely to be selected
as D, what are the probabilities that
\begin{enumerate}
\item C will be selected?
\item A will not be selected?
\end{enumerate}
	%\input{exemplar/11/16/3/9/main.tex}
 \item A bag contain 24 balls of which $x$ balls are red, $2x$ are white and $3x$ are blue. A ball is selected at random, What is the probability that it is
\begin{enumerate}[label=\alph*)]
\item not red ?
\item white ?
\end{enumerate}
%\input{exemplar/10/13/3/41/main.tex}
If the letters of the word ASSASSINATION are arranged at random. Find the Probability that
\begin{enumerate}[label=(\alph*)]
\item Four $S's$ come consecutively in the word
\item Two  $I's$ and two $N's$ come together
\item All $A's$ are not coming together
\item No two $A's$ are coming together
\end{enumerate}
%\input{exemplar/11/16/3/14/main.tex}
	\item One urn contains two black balls (labelled B1 and B2) and one white ball. A
	second urn contains one black ball and two white balls (labelled W1 and W2).
	Suppose the following experiment is performed. One of the two urns is chosen
	at random. Next a ball is randomly chosen from the urn. Then a second ball is
	chosen at random from the same urn without replacing the first ball.
	
	\begin{enumerate}
	\item What is the probability that two black balls are chosen?
	
	\item What is the probability that two balls of opposite colour are chosen?
	\end{enumerate}
	\solution
	%\input{exemplar/11/16/3/12/main1.tex}
\end{enumerate}

	\item A bag contains 4 red and 4 black balls, another bag contains 2 red and 6 black balls. One of the two bags is selected at random and a ball is drawn from the bag which is found to be red. Find the probability that the ball is drawn from the first bag.
\\
\solution
		%\begin{table}[H]
	\centering
\begin{tabular}{|c|c|c|}
\hline
Random variable &Value &Definition\\ \hline
\multirow{3}{*}{X} &0 &Slips of Rs 1\\
&1 &Slips of Rs 5\\
&2 &Slips of Rs 13\\ \hline
\multirow{2}{*}{Y} &0 &Box A\\
&1 &Box B\\\hline
\end{tabular}
\caption{}
\label{tab:Distribution}
\end{table}
See \tabref{tab:Distribution}.
\begin{align}
p_{Y}\brak{k}= \begin{cases} 
      \frac{1}{3} & {k=0} \\
      \frac{2}{3 }& {k=1} 
   \end{cases}
   \\
p_{Y|X}\brak{0|0} = \frac{19}{25}\, 
p_{Y|X}\brak{0|1} = \frac{6}{25}\,
p_{Y|X}\brak{1|0} = \frac{45}{50}\,
p_{Y|X}\brak{1|2} = \frac{5}{50}
\end{align}
The desired probability is the probability that a slip drawn at random is marked other than Rs 1,
\begin{align}
&=1-p_X\brak{0}\\
&= p_X(1) + p_X(2)
\end{align}
Using Bayes theorem,
\begin{align}
&= p_Y\brak{0} \times \pr{Y=0 | X=1} + p_Y\brak{1} \times \pr{Y=1|X=2}\\
&=\frac{1}{3} \times \frac{6}{25} + \frac{2}{3} \times \frac{5}{50}\\
&=\frac{11}{75}
\end{align}

\newpage

%\tableofcontents

\bigskip

\renewcommand{\thefigure}{\theenumi}
\renewcommand{\thetable}{\theenumi}
%\renewcommand{\theequation}{\theenumi}

%\begin{abstract}
%%\boldmath
%In this letter, an algorithm for evaluating the exact analytical bit error rate  (BER)  for the piecewise linear (PL) combiner for  multiple relays is presented. Previous results were available only for upto three relays. The algorithm is unique in the sense that  the actual mathematical expressions, that are prohibitively large, need not be explicitly obtained. The diversity gain due to multiple relays is shown through plots of the analytical BER, well supported by simulations. 
%
%\end{abstract}
% IEEEtran.cls defaults to using nonbold math in the Abstract.
% This preserves the distinction between vectors and scalars. However,
% if the journal you are submitting to favors bold math in the abstract,
% then you can use LaTeX's standard command \boldmath at the very start
% of the abstract to achieve this. Many IEEE journals frown on math
% in the abstract anyway.

% Note that keywords are not normally used for peerreview papers.
%\begin{IEEEkeywords}
%Cooperative diversity, decode and forward, piecewise linear
%\end{IEEEkeywords}



% For peer review papers, you can put extra information on the cover
% page as needed:
% \ifCLASSOPTIONpeerreview
% \begin{center} \bfseries EDICS Category: 3-BBND \end{center}
% \fi
%
% For peerreview papers, this IEEEtran command inserts a page break and
% creates the second title. It will be ignored for other modes.
%\IEEEpeerreviewmaketitle




  \item
  Cards with numbers 2 to 101 are placed in a box. A card is selected at random.Find the probability that the card has
\begin{enumerate}[label=(\roman*)]
	\item an even number 
	\item a square number
\end{enumerate}
\solution
%\begin{table}[H]
	\centering
\begin{tabular}{|c|c|c|}
\hline
Random variable &Value &Definition\\ \hline
\multirow{3}{*}{X} &0 &Slips of Rs 1\\
&1 &Slips of Rs 5\\
&2 &Slips of Rs 13\\ \hline
\multirow{2}{*}{Y} &0 &Box A\\
&1 &Box B\\\hline
\end{tabular}
\caption{}
\label{tab:Distribution}
\end{table}
See \tabref{tab:Distribution}.
\begin{align}
p_{Y}\brak{k}= \begin{cases} 
      \frac{1}{3} & {k=0} \\
      \frac{2}{3 }& {k=1} 
   \end{cases}
   \\
p_{Y|X}\brak{0|0} = \frac{19}{25}\, 
p_{Y|X}\brak{0|1} = \frac{6}{25}\,
p_{Y|X}\brak{1|0} = \frac{45}{50}\,
p_{Y|X}\brak{1|2} = \frac{5}{50}
\end{align}
The desired probability is the probability that a slip drawn at random is marked other than Rs 1,
\begin{align}
&=1-p_X\brak{0}\\
&= p_X(1) + p_X(2)
\end{align}
Using Bayes theorem,
\begin{align}
&= p_Y\brak{0} \times \pr{Y=0 | X=1} + p_Y\brak{1} \times \pr{Y=1|X=2}\\
&=\frac{1}{3} \times \frac{6}{25} + \frac{2}{3} \times \frac{5}{50}\\
&=\frac{11}{75}
\end{align}

\newpage

%\tableofcontents

\bigskip

\renewcommand{\thefigure}{\theenumi}
\renewcommand{\thetable}{\theenumi}
%\renewcommand{\theequation}{\theenumi}

%\begin{abstract}
%%\boldmath
%In this letter, an algorithm for evaluating the exact analytical bit error rate  (BER)  for the piecewise linear (PL) combiner for  multiple relays is presented. Previous results were available only for upto three relays. The algorithm is unique in the sense that  the actual mathematical expressions, that are prohibitively large, need not be explicitly obtained. The diversity gain due to multiple relays is shown through plots of the analytical BER, well supported by simulations. 
%
%\end{abstract}
% IEEEtran.cls defaults to using nonbold math in the Abstract.
% This preserves the distinction between vectors and scalars. However,
% if the journal you are submitting to favors bold math in the abstract,
% then you can use LaTeX's standard command \boldmath at the very start
% of the abstract to achieve this. Many IEEE journals frown on math
% in the abstract anyway.

% Note that keywords are not normally used for peerreview papers.
%\begin{IEEEkeywords}
%Cooperative diversity, decode and forward, piecewise linear
%\end{IEEEkeywords}



% For peer review papers, you can put extra information on the cover
% page as needed:
% \ifCLASSOPTIONpeerreview
% \begin{center} \bfseries EDICS Category: 3-BBND \end{center}
% \fi
%
% For peerreview papers, this IEEEtran command inserts a page break and
% creates the second title. It will be ignored for other modes.
%\IEEEpeerreviewmaketitle




\item
The king, queen and jack of clubs are removed from a deck of 52 playing cards and then well shuffled. Now one card is drawn at random from the remaining cards.  Determine the probability that the card is
\begin{enumerate}[label=(\roman*)]
\item a club
\item 10 of hearts
\end{enumerate}
\solution
%\begin{table}[H]
	\centering
\begin{tabular}{|c|c|c|}
\hline
Random variable &Value &Definition\\ \hline
\multirow{3}{*}{X} &0 &Slips of Rs 1\\
&1 &Slips of Rs 5\\
&2 &Slips of Rs 13\\ \hline
\multirow{2}{*}{Y} &0 &Box A\\
&1 &Box B\\\hline
\end{tabular}
\caption{}
\label{tab:Distribution}
\end{table}
See \tabref{tab:Distribution}.
\begin{align}
p_{Y}\brak{k}= \begin{cases} 
      \frac{1}{3} & {k=0} \\
      \frac{2}{3 }& {k=1} 
   \end{cases}
   \\
p_{Y|X}\brak{0|0} = \frac{19}{25}\, 
p_{Y|X}\brak{0|1} = \frac{6}{25}\,
p_{Y|X}\brak{1|0} = \frac{45}{50}\,
p_{Y|X}\brak{1|2} = \frac{5}{50}
\end{align}
The desired probability is the probability that a slip drawn at random is marked other than Rs 1,
\begin{align}
&=1-p_X\brak{0}\\
&= p_X(1) + p_X(2)
\end{align}
Using Bayes theorem,
\begin{align}
&= p_Y\brak{0} \times \pr{Y=0 | X=1} + p_Y\brak{1} \times \pr{Y=1|X=2}\\
&=\frac{1}{3} \times \frac{6}{25} + \frac{2}{3} \times \frac{5}{50}\\
&=\frac{11}{75}
\end{align}

\newpage

%\tableofcontents

\bigskip

\renewcommand{\thefigure}{\theenumi}
\renewcommand{\thetable}{\theenumi}
%\renewcommand{\theequation}{\theenumi}

%\begin{abstract}
%%\boldmath
%In this letter, an algorithm for evaluating the exact analytical bit error rate  (BER)  for the piecewise linear (PL) combiner for  multiple relays is presented. Previous results were available only for upto three relays. The algorithm is unique in the sense that  the actual mathematical expressions, that are prohibitively large, need not be explicitly obtained. The diversity gain due to multiple relays is shown through plots of the analytical BER, well supported by simulations. 
%
%\end{abstract}
% IEEEtran.cls defaults to using nonbold math in the Abstract.
% This preserves the distinction between vectors and scalars. However,
% if the journal you are submitting to favors bold math in the abstract,
% then you can use LaTeX's standard command \boldmath at the very start
% of the abstract to achieve this. Many IEEE journals frown on math
% in the abstract anyway.

% Note that keywords are not normally used for peerreview papers.
%\begin{IEEEkeywords}
%Cooperative diversity, decode and forward, piecewise linear
%\end{IEEEkeywords}



% For peer review papers, you can put extra information on the cover
% page as needed:
% \ifCLASSOPTIONpeerreview
% \begin{center} \bfseries EDICS Category: 3-BBND \end{center}
% \fi
%
% For peerreview papers, this IEEEtran command inserts a page break and
% creates the second title. It will be ignored for other modes.
%\IEEEpeerreviewmaketitle




\item A team of medical students doing their internship have to assist during surgeries
at a city hospital. The probabilities of surgeries rated as very complex, complex,
routine, simple or very simple are respectively, 0.15, 0.20, 0.31, 0.26, .08. Find
the probabilities that a particular surgery will be rated
\begin{enumerate}
	\item complex or very complex;
	\item neither very complex nor very simple;
	\item routine or complex
	\item routine or simple
\end{enumerate}
\solution
%\begin{table}[H]
	\centering
\begin{tabular}{|c|c|c|}
\hline
Random variable &Value &Definition\\ \hline
\multirow{3}{*}{X} &0 &Slips of Rs 1\\
&1 &Slips of Rs 5\\
&2 &Slips of Rs 13\\ \hline
\multirow{2}{*}{Y} &0 &Box A\\
&1 &Box B\\\hline
\end{tabular}
\caption{}
\label{tab:Distribution}
\end{table}
See \tabref{tab:Distribution}.
\begin{align}
p_{Y}\brak{k}= \begin{cases} 
      \frac{1}{3} & {k=0} \\
      \frac{2}{3 }& {k=1} 
   \end{cases}
   \\
p_{Y|X}\brak{0|0} = \frac{19}{25}\, 
p_{Y|X}\brak{0|1} = \frac{6}{25}\,
p_{Y|X}\brak{1|0} = \frac{45}{50}\,
p_{Y|X}\brak{1|2} = \frac{5}{50}
\end{align}
The desired probability is the probability that a slip drawn at random is marked other than Rs 1,
\begin{align}
&=1-p_X\brak{0}\\
&= p_X(1) + p_X(2)
\end{align}
Using Bayes theorem,
\begin{align}
&= p_Y\brak{0} \times \pr{Y=0 | X=1} + p_Y\brak{1} \times \pr{Y=1|X=2}\\
&=\frac{1}{3} \times \frac{6}{25} + \frac{2}{3} \times \frac{5}{50}\\
&=\frac{11}{75}
\end{align}

\newpage

%\tableofcontents

\bigskip

\renewcommand{\thefigure}{\theenumi}
\renewcommand{\thetable}{\theenumi}
%\renewcommand{\theequation}{\theenumi}

%\begin{abstract}
%%\boldmath
%In this letter, an algorithm for evaluating the exact analytical bit error rate  (BER)  for the piecewise linear (PL) combiner for  multiple relays is presented. Previous results were available only for upto three relays. The algorithm is unique in the sense that  the actual mathematical expressions, that are prohibitively large, need not be explicitly obtained. The diversity gain due to multiple relays is shown through plots of the analytical BER, well supported by simulations. 
%
%\end{abstract}
% IEEEtran.cls defaults to using nonbold math in the Abstract.
% This preserves the distinction between vectors and scalars. However,
% if the journal you are submitting to favors bold math in the abstract,
% then you can use LaTeX's standard command \boldmath at the very start
% of the abstract to achieve this. Many IEEE journals frown on math
% in the abstract anyway.

% Note that keywords are not normally used for peerreview papers.
%\begin{IEEEkeywords}
%Cooperative diversity, decode and forward, piecewise linear
%\end{IEEEkeywords}



% For peer review papers, you can put extra information on the cover
% page as needed:
% \ifCLASSOPTIONpeerreview
% \begin{center} \bfseries EDICS Category: 3-BBND \end{center}
% \fi
%
% For peerreview papers, this IEEEtran command inserts a page break and
% creates the second title. It will be ignored for other modes.
%\IEEEpeerreviewmaketitle




\item A card is selected from a pack of 52 cards.
\begin{enumerate}[label=(\alph*)]
    \item How many points are there in the sample space?
    \item Calculate the probability that the card is an ace of spades.
    \item Calculate the probability that the card is (i) an ace and (ii) black card.
\end{enumerate}
\solution
%Let $X$ be an bernoulli rv defined as in \tabref{tab:exemplar/11/16/3/26}.  Then, 
\begin{equation}
    p =
        \frac{4}{11} 
\end{equation}
\begin{table}[H]
	\centering
	\input{exemplar/11/16/3/26/tables/Table2.tex}
	\caption{}
        \label{tab:exemplar/11/16/3/26}
\end{table}

\item The probability that a non leap year selected at random will contain 53 sundays.
\\
\solution
%\begin{table}[H]
	\centering
\begin{tabular}{|c|c|c|}
\hline
Random variable &Value &Definition\\ \hline
\multirow{3}{*}{X} &0 &Slips of Rs 1\\
&1 &Slips of Rs 5\\
&2 &Slips of Rs 13\\ \hline
\multirow{2}{*}{Y} &0 &Box A\\
&1 &Box B\\\hline
\end{tabular}
\caption{}
\label{tab:Distribution}
\end{table}
See \tabref{tab:Distribution}.
\begin{align}
p_{Y}\brak{k}= \begin{cases} 
      \frac{1}{3} & {k=0} \\
      \frac{2}{3 }& {k=1} 
   \end{cases}
   \\
p_{Y|X}\brak{0|0} = \frac{19}{25}\, 
p_{Y|X}\brak{0|1} = \frac{6}{25}\,
p_{Y|X}\brak{1|0} = \frac{45}{50}\,
p_{Y|X}\brak{1|2} = \frac{5}{50}
\end{align}
The desired probability is the probability that a slip drawn at random is marked other than Rs 1,
\begin{align}
&=1-p_X\brak{0}\\
&= p_X(1) + p_X(2)
\end{align}
Using Bayes theorem,
\begin{align}
&= p_Y\brak{0} \times \pr{Y=0 | X=1} + p_Y\brak{1} \times \pr{Y=1|X=2}\\
&=\frac{1}{3} \times \frac{6}{25} + \frac{2}{3} \times \frac{5}{50}\\
&=\frac{11}{75}
\end{align}

\newpage

%\tableofcontents

\bigskip

\renewcommand{\thefigure}{\theenumi}
\renewcommand{\thetable}{\theenumi}
%\renewcommand{\theequation}{\theenumi}

%\begin{abstract}
%%\boldmath
%In this letter, an algorithm for evaluating the exact analytical bit error rate  (BER)  for the piecewise linear (PL) combiner for  multiple relays is presented. Previous results were available only for upto three relays. The algorithm is unique in the sense that  the actual mathematical expressions, that are prohibitively large, need not be explicitly obtained. The diversity gain due to multiple relays is shown through plots of the analytical BER, well supported by simulations. 
%
%\end{abstract}
% IEEEtran.cls defaults to using nonbold math in the Abstract.
% This preserves the distinction between vectors and scalars. However,
% if the journal you are submitting to favors bold math in the abstract,
% then you can use LaTeX's standard command \boldmath at the very start
% of the abstract to achieve this. Many IEEE journals frown on math
% in the abstract anyway.

% Note that keywords are not normally used for peerreview papers.
%\begin{IEEEkeywords}
%Cooperative diversity, decode and forward, piecewise linear
%\end{IEEEkeywords}



% For peer review papers, you can put extra information on the cover
% page as needed:
% \ifCLASSOPTIONpeerreview
% \begin{center} \bfseries EDICS Category: 3-BBND \end{center}
% \fi
%
% For peerreview papers, this IEEEtran command inserts a page break and
% creates the second title. It will be ignored for other modes.
%\IEEEpeerreviewmaketitle




\item One of the four persons John, Rita, Aslam or Gurpreet will be promoted next
month. Consequently the sample space consists of four elementary outcomes
S = {John promoted, Rita promoted, Aslam promoted, Gurpreet promoted}
You are told that the chances of John’s promotion is same as that of Gurpreet,
Rita’s chances of promotion are twice as likely as Johns. Aslam’s chances are
four times that of John.
\begin{enumerate}
	\item Determine
	\begin{enumerate}
		\item P (John promoted)
		\item P (Rita promoted)
		\item P (Aslam promoted)
		\item P (Gurpreet promoted)
	\end{enumerate}
	\item If A = {John promoted or Gurpreet promoted}, find P (A).
\end{enumerate}
\solution
%\begin{table}[H]
	\centering
\begin{tabular}{|c|c|c|}
\hline
Random variable &Value &Definition\\ \hline
\multirow{3}{*}{X} &0 &Slips of Rs 1\\
&1 &Slips of Rs 5\\
&2 &Slips of Rs 13\\ \hline
\multirow{2}{*}{Y} &0 &Box A\\
&1 &Box B\\\hline
\end{tabular}
\caption{}
\label{tab:Distribution}
\end{table}
See \tabref{tab:Distribution}.
\begin{align}
p_{Y}\brak{k}= \begin{cases} 
      \frac{1}{3} & {k=0} \\
      \frac{2}{3 }& {k=1} 
   \end{cases}
   \\
p_{Y|X}\brak{0|0} = \frac{19}{25}\, 
p_{Y|X}\brak{0|1} = \frac{6}{25}\,
p_{Y|X}\brak{1|0} = \frac{45}{50}\,
p_{Y|X}\brak{1|2} = \frac{5}{50}
\end{align}
The desired probability is the probability that a slip drawn at random is marked other than Rs 1,
\begin{align}
&=1-p_X\brak{0}\\
&= p_X(1) + p_X(2)
\end{align}
Using Bayes theorem,
\begin{align}
&= p_Y\brak{0} \times \pr{Y=0 | X=1} + p_Y\brak{1} \times \pr{Y=1|X=2}\\
&=\frac{1}{3} \times \frac{6}{25} + \frac{2}{3} \times \frac{5}{50}\\
&=\frac{11}{75}
\end{align}

\newpage

%\tableofcontents

\bigskip

\renewcommand{\thefigure}{\theenumi}
\renewcommand{\thetable}{\theenumi}
%\renewcommand{\theequation}{\theenumi}

%\begin{abstract}
%%\boldmath
%In this letter, an algorithm for evaluating the exact analytical bit error rate  (BER)  for the piecewise linear (PL) combiner for  multiple relays is presented. Previous results were available only for upto three relays. The algorithm is unique in the sense that  the actual mathematical expressions, that are prohibitively large, need not be explicitly obtained. The diversity gain due to multiple relays is shown through plots of the analytical BER, well supported by simulations. 
%
%\end{abstract}
% IEEEtran.cls defaults to using nonbold math in the Abstract.
% This preserves the distinction between vectors and scalars. However,
% if the journal you are submitting to favors bold math in the abstract,
% then you can use LaTeX's standard command \boldmath at the very start
% of the abstract to achieve this. Many IEEE journals frown on math
% in the abstract anyway.

% Note that keywords are not normally used for peerreview papers.
%\begin{IEEEkeywords}
%Cooperative diversity, decode and forward, piecewise linear
%\end{IEEEkeywords}



% For peer review papers, you can put extra information on the cover
% page as needed:
% \ifCLASSOPTIONpeerreview
% \begin{center} \bfseries EDICS Category: 3-BBND \end{center}
% \fi
%
% For peerreview papers, this IEEEtran command inserts a page break and
% creates the second title. It will be ignored for other modes.
%\IEEEpeerreviewmaketitle




\item A card is drawn from a deck of 52 cards. Find the probability of getting a king or a heart or a red card.\\
\solution
%\begin{table}[H]
	\centering
\begin{tabular}{|c|c|c|}
\hline
Random variable &Value &Definition\\ \hline
\multirow{3}{*}{X} &0 &Slips of Rs 1\\
&1 &Slips of Rs 5\\
&2 &Slips of Rs 13\\ \hline
\multirow{2}{*}{Y} &0 &Box A\\
&1 &Box B\\\hline
\end{tabular}
\caption{}
\label{tab:Distribution}
\end{table}
See \tabref{tab:Distribution}.
\begin{align}
p_{Y}\brak{k}= \begin{cases} 
      \frac{1}{3} & {k=0} \\
      \frac{2}{3 }& {k=1} 
   \end{cases}
   \\
p_{Y|X}\brak{0|0} = \frac{19}{25}\, 
p_{Y|X}\brak{0|1} = \frac{6}{25}\,
p_{Y|X}\brak{1|0} = \frac{45}{50}\,
p_{Y|X}\brak{1|2} = \frac{5}{50}
\end{align}
The desired probability is the probability that a slip drawn at random is marked other than Rs 1,
\begin{align}
&=1-p_X\brak{0}\\
&= p_X(1) + p_X(2)
\end{align}
Using Bayes theorem,
\begin{align}
&= p_Y\brak{0} \times \pr{Y=0 | X=1} + p_Y\brak{1} \times \pr{Y=1|X=2}\\
&=\frac{1}{3} \times \frac{6}{25} + \frac{2}{3} \times \frac{5}{50}\\
&=\frac{11}{75}
\end{align}

\newpage

%\tableofcontents

\bigskip

\renewcommand{\thefigure}{\theenumi}
\renewcommand{\thetable}{\theenumi}
%\renewcommand{\theequation}{\theenumi}

%\begin{abstract}
%%\boldmath
%In this letter, an algorithm for evaluating the exact analytical bit error rate  (BER)  for the piecewise linear (PL) combiner for  multiple relays is presented. Previous results were available only for upto three relays. The algorithm is unique in the sense that  the actual mathematical expressions, that are prohibitively large, need not be explicitly obtained. The diversity gain due to multiple relays is shown through plots of the analytical BER, well supported by simulations. 
%
%\end{abstract}
% IEEEtran.cls defaults to using nonbold math in the Abstract.
% This preserves the distinction between vectors and scalars. However,
% if the journal you are submitting to favors bold math in the abstract,
% then you can use LaTeX's standard command \boldmath at the very start
% of the abstract to achieve this. Many IEEE journals frown on math
% in the abstract anyway.

% Note that keywords are not normally used for peerreview papers.
%\begin{IEEEkeywords}
%Cooperative diversity, decode and forward, piecewise linear
%\end{IEEEkeywords}



% For peer review papers, you can put extra information on the cover
% page as needed:
% \ifCLASSOPTIONpeerreview
% \begin{center} \bfseries EDICS Category: 3-BBND \end{center}
% \fi
%
% For peerreview papers, this IEEEtran command inserts a page break and
% creates the second title. It will be ignored for other modes.
%\IEEEpeerreviewmaketitle




\item The probability that a student will pass his examination is 0.73, the probability of
the student getting a compartment is 0.13, and the probability that the student will
either pass or get compartment is 0.96. State True or False.\\
\solution
%\begin{table}[H]
	\centering
\begin{tabular}{|c|c|c|}
\hline
Random variable &Value &Definition\\ \hline
\multirow{3}{*}{X} &0 &Slips of Rs 1\\
&1 &Slips of Rs 5\\
&2 &Slips of Rs 13\\ \hline
\multirow{2}{*}{Y} &0 &Box A\\
&1 &Box B\\\hline
\end{tabular}
\caption{}
\label{tab:Distribution}
\end{table}
See \tabref{tab:Distribution}.
\begin{align}
p_{Y}\brak{k}= \begin{cases} 
      \frac{1}{3} & {k=0} \\
      \frac{2}{3 }& {k=1} 
   \end{cases}
   \\
p_{Y|X}\brak{0|0} = \frac{19}{25}\, 
p_{Y|X}\brak{0|1} = \frac{6}{25}\,
p_{Y|X}\brak{1|0} = \frac{45}{50}\,
p_{Y|X}\brak{1|2} = \frac{5}{50}
\end{align}
The desired probability is the probability that a slip drawn at random is marked other than Rs 1,
\begin{align}
&=1-p_X\brak{0}\\
&= p_X(1) + p_X(2)
\end{align}
Using Bayes theorem,
\begin{align}
&= p_Y\brak{0} \times \pr{Y=0 | X=1} + p_Y\brak{1} \times \pr{Y=1|X=2}\\
&=\frac{1}{3} \times \frac{6}{25} + \frac{2}{3} \times \frac{5}{50}\\
&=\frac{11}{75}
\end{align}

\newpage

%\tableofcontents

\bigskip

\renewcommand{\thefigure}{\theenumi}
\renewcommand{\thetable}{\theenumi}
%\renewcommand{\theequation}{\theenumi}

%\begin{abstract}
%%\boldmath
%In this letter, an algorithm for evaluating the exact analytical bit error rate  (BER)  for the piecewise linear (PL) combiner for  multiple relays is presented. Previous results were available only for upto three relays. The algorithm is unique in the sense that  the actual mathematical expressions, that are prohibitively large, need not be explicitly obtained. The diversity gain due to multiple relays is shown through plots of the analytical BER, well supported by simulations. 
%
%\end{abstract}
% IEEEtran.cls defaults to using nonbold math in the Abstract.
% This preserves the distinction between vectors and scalars. However,
% if the journal you are submitting to favors bold math in the abstract,
% then you can use LaTeX's standard command \boldmath at the very start
% of the abstract to achieve this. Many IEEE journals frown on math
% in the abstract anyway.

% Note that keywords are not normally used for peerreview papers.
%\begin{IEEEkeywords}
%Cooperative diversity, decode and forward, piecewise linear
%\end{IEEEkeywords}



% For peer review papers, you can put extra information on the cover
% page as needed:
% \ifCLASSOPTIONpeerreview
% \begin{center} \bfseries EDICS Category: 3-BBND \end{center}
% \fi
%
% For peerreview papers, this IEEEtran command inserts a page break and
% creates the second title. It will be ignored for other modes.
%\IEEEpeerreviewmaketitle




\item A card is selected from a pack of 52 cards\\
\begin{enumerate}[label=(\alph*)]
\item How many points are there in the sample space?
\item Calculate the probability that the cards is an ace of spades.
\item Calculate the probability that the card is (i) an ace (ii)black card.\\
\end{enumerate}
%\input{ncert/11/16/3/4_1/Prob_4.tex}
\item In a non-leap year, the probability of having 53 tuesdays or 53 wednesdays is\\
\solution
%A non-leap year has a total of 365 days, and a week has 7 days.\\
So it can be expressed as 
\begin{align}
365\text{days} &=52\times 7+1 \text{day}
\end{align}
$\implies$ 52 tuesdays or wednesdays\\
Random variable X denotes the days of a week
\begin{align}
p_X\brak{k}&=\frac{1}{7}; \quad \brak{1<k<7}
\end{align}
So the probability of extra day being tuesday or wednesday is
\begin{align}
p_X\brak{3}+p_X\brak{4}&=\frac{1}{7}+\frac{1}{7}=\frac{2}{7}
\end{align}



\item There are 1000 sealed envelopes in a box, 10 of them contain a cash prize of
Rs 100 each, 100 of them contain a cash prize of Rs 50 each and 200 of them
contain a cash prize of Rs 10 each and rest do not contain any cash prize. If they
are well shuffled and an envelope is picked up out, what is the probability that it
contains no cash prize?\\
\solution
%\begin{table}[H]
	\centering
\begin{tabular}{|c|c|c|}
\hline
Random variable &Value &Definition\\ \hline
\multirow{3}{*}{X} &0 &Slips of Rs 1\\
&1 &Slips of Rs 5\\
&2 &Slips of Rs 13\\ \hline
\multirow{2}{*}{Y} &0 &Box A\\
&1 &Box B\\\hline
\end{tabular}
\caption{}
\label{tab:Distribution}
\end{table}
See \tabref{tab:Distribution}.
\begin{align}
p_{Y}\brak{k}= \begin{cases} 
      \frac{1}{3} & {k=0} \\
      \frac{2}{3 }& {k=1} 
   \end{cases}
   \\
p_{Y|X}\brak{0|0} = \frac{19}{25}\, 
p_{Y|X}\brak{0|1} = \frac{6}{25}\,
p_{Y|X}\brak{1|0} = \frac{45}{50}\,
p_{Y|X}\brak{1|2} = \frac{5}{50}
\end{align}
The desired probability is the probability that a slip drawn at random is marked other than Rs 1,
\begin{align}
&=1-p_X\brak{0}\\
&= p_X(1) + p_X(2)
\end{align}
Using Bayes theorem,
\begin{align}
&= p_Y\brak{0} \times \pr{Y=0 | X=1} + p_Y\brak{1} \times \pr{Y=1|X=2}\\
&=\frac{1}{3} \times \frac{6}{25} + \frac{2}{3} \times \frac{5}{50}\\
&=\frac{11}{75}
\end{align}

\newpage

%\tableofcontents

\bigskip

\renewcommand{\thefigure}{\theenumi}
\renewcommand{\thetable}{\theenumi}
%\renewcommand{\theequation}{\theenumi}

%\begin{abstract}
%%\boldmath
%In this letter, an algorithm for evaluating the exact analytical bit error rate  (BER)  for the piecewise linear (PL) combiner for  multiple relays is presented. Previous results were available only for upto three relays. The algorithm is unique in the sense that  the actual mathematical expressions, that are prohibitively large, need not be explicitly obtained. The diversity gain due to multiple relays is shown through plots of the analytical BER, well supported by simulations. 
%
%\end{abstract}
% IEEEtran.cls defaults to using nonbold math in the Abstract.
% This preserves the distinction between vectors and scalars. However,
% if the journal you are submitting to favors bold math in the abstract,
% then you can use LaTeX's standard command \boldmath at the very start
% of the abstract to achieve this. Many IEEE journals frown on math
% in the abstract anyway.

% Note that keywords are not normally used for peerreview papers.
%\begin{IEEEkeywords}
%Cooperative diversity, decode and forward, piecewise linear
%\end{IEEEkeywords}



% For peer review papers, you can put extra information on the cover
% page as needed:
% \ifCLASSOPTIONpeerreview
% \begin{center} \bfseries EDICS Category: 3-BBND \end{center}
% \fi
%
% For peerreview papers, this IEEEtran command inserts a page break and
% creates the second title. It will be ignored for other modes.
%\IEEEpeerreviewmaketitle




\item 
A die is thrown and a card is selected at random from a deck of 52 playing cards. The probability of getting an even number on the die and a spade card.\\
\solution
%\begin{table}[H]
	\centering
\begin{tabular}{|c|c|c|}
\hline
Random variable &Value &Definition\\ \hline
\multirow{3}{*}{X} &0 &Slips of Rs 1\\
&1 &Slips of Rs 5\\
&2 &Slips of Rs 13\\ \hline
\multirow{2}{*}{Y} &0 &Box A\\
&1 &Box B\\\hline
\end{tabular}
\caption{}
\label{tab:Distribution}
\end{table}
See \tabref{tab:Distribution}.
\begin{align}
p_{Y}\brak{k}= \begin{cases} 
      \frac{1}{3} & {k=0} \\
      \frac{2}{3 }& {k=1} 
   \end{cases}
   \\
p_{Y|X}\brak{0|0} = \frac{19}{25}\, 
p_{Y|X}\brak{0|1} = \frac{6}{25}\,
p_{Y|X}\brak{1|0} = \frac{45}{50}\,
p_{Y|X}\brak{1|2} = \frac{5}{50}
\end{align}
The desired probability is the probability that a slip drawn at random is marked other than Rs 1,
\begin{align}
&=1-p_X\brak{0}\\
&= p_X(1) + p_X(2)
\end{align}
Using Bayes theorem,
\begin{align}
&= p_Y\brak{0} \times \pr{Y=0 | X=1} + p_Y\brak{1} \times \pr{Y=1|X=2}\\
&=\frac{1}{3} \times \frac{6}{25} + \frac{2}{3} \times \frac{5}{50}\\
&=\frac{11}{75}
\end{align}

\newpage

%\tableofcontents

\bigskip

\renewcommand{\thefigure}{\theenumi}
\renewcommand{\thetable}{\theenumi}
%\renewcommand{\theequation}{\theenumi}

%\begin{abstract}
%%\boldmath
%In this letter, an algorithm for evaluating the exact analytical bit error rate  (BER)  for the piecewise linear (PL) combiner for  multiple relays is presented. Previous results were available only for upto three relays. The algorithm is unique in the sense that  the actual mathematical expressions, that are prohibitively large, need not be explicitly obtained. The diversity gain due to multiple relays is shown through plots of the analytical BER, well supported by simulations. 
%
%\end{abstract}
% IEEEtran.cls defaults to using nonbold math in the Abstract.
% This preserves the distinction between vectors and scalars. However,
% if the journal you are submitting to favors bold math in the abstract,
% then you can use LaTeX's standard command \boldmath at the very start
% of the abstract to achieve this. Many IEEE journals frown on math
% in the abstract anyway.

% Note that keywords are not normally used for peerreview papers.
%\begin{IEEEkeywords}
%Cooperative diversity, decode and forward, piecewise linear
%\end{IEEEkeywords}



% For peer review papers, you can put extra information on the cover
% page as needed:
% \ifCLASSOPTIONpeerreview
% \begin{center} \bfseries EDICS Category: 3-BBND \end{center}
% \fi
%
% For peerreview papers, this IEEEtran command inserts a page break and
% creates the second title. It will be ignored for other modes.
%\IEEEpeerreviewmaketitle




\item
If 4-digit numbers greater than 5,000 are randomly formed from the digits 0, 1, 3, 5, and 7, what is the probability of forming a number divisible by 5 when:
\begin{enumerate}
    \item The digits are repeated?
    \item The repetition of digits is not allowed?
\end{enumerate}
\solution
%\begin{table}[H]
	\centering
\begin{tabular}{|c|c|c|}
\hline
Random variable &Value &Definition\\ \hline
\multirow{3}{*}{X} &0 &Slips of Rs 1\\
&1 &Slips of Rs 5\\
&2 &Slips of Rs 13\\ \hline
\multirow{2}{*}{Y} &0 &Box A\\
&1 &Box B\\\hline
\end{tabular}
\caption{}
\label{tab:Distribution}
\end{table}
See \tabref{tab:Distribution}.
\begin{align}
p_{Y}\brak{k}= \begin{cases} 
      \frac{1}{3} & {k=0} \\
      \frac{2}{3 }& {k=1} 
   \end{cases}
   \\
p_{Y|X}\brak{0|0} = \frac{19}{25}\, 
p_{Y|X}\brak{0|1} = \frac{6}{25}\,
p_{Y|X}\brak{1|0} = \frac{45}{50}\,
p_{Y|X}\brak{1|2} = \frac{5}{50}
\end{align}
The desired probability is the probability that a slip drawn at random is marked other than Rs 1,
\begin{align}
&=1-p_X\brak{0}\\
&= p_X(1) + p_X(2)
\end{align}
Using Bayes theorem,
\begin{align}
&= p_Y\brak{0} \times \pr{Y=0 | X=1} + p_Y\brak{1} \times \pr{Y=1|X=2}\\
&=\frac{1}{3} \times \frac{6}{25} + \frac{2}{3} \times \frac{5}{50}\\
&=\frac{11}{75}
\end{align}

\newpage

%\tableofcontents

\bigskip

\renewcommand{\thefigure}{\theenumi}
\renewcommand{\thetable}{\theenumi}
%\renewcommand{\theequation}{\theenumi}

%\begin{abstract}
%%\boldmath
%In this letter, an algorithm for evaluating the exact analytical bit error rate  (BER)  for the piecewise linear (PL) combiner for  multiple relays is presented. Previous results were available only for upto three relays. The algorithm is unique in the sense that  the actual mathematical expressions, that are prohibitively large, need not be explicitly obtained. The diversity gain due to multiple relays is shown through plots of the analytical BER, well supported by simulations. 
%
%\end{abstract}
% IEEEtran.cls defaults to using nonbold math in the Abstract.
% This preserves the distinction between vectors and scalars. However,
% if the journal you are submitting to favors bold math in the abstract,
% then you can use LaTeX's standard command \boldmath at the very start
% of the abstract to achieve this. Many IEEE journals frown on math
% in the abstract anyway.

% Note that keywords are not normally used for peerreview papers.
%\begin{IEEEkeywords}
%Cooperative diversity, decode and forward, piecewise linear
%\end{IEEEkeywords}



% For peer review papers, you can put extra information on the cover
% page as needed:
% \ifCLASSOPTIONpeerreview
% \begin{center} \bfseries EDICS Category: 3-BBND \end{center}
% \fi
%
% For peerreview papers, this IEEEtran command inserts a page break and
% creates the second title. It will be ignored for other modes.
%\IEEEpeerreviewmaketitle




\item Consider the probability space $\brak{\Omega, \mathcal{G}, P}$ where $\Omega = [0,2]$ and $\mathcal{G} = \cbrak{\phi, \Omega, [0,1], (1,2]}$. Let $X$ and $Y$ be two functions on $\Omega$ defined as
\begin{align*}
    X(\omega) = 
    \begin{cases}
        1 & \text{if }\omega \in [0, 1]\\
        2 & \text{if }\omega \in (1, 2]
    \end{cases}
\end{align*}
and
\begin{align*}
    Y(\omega) = 
    \begin{cases}
        2 & \text{if }\omega \in [0, 1.5]\\
        3 & \text{if }\omega \in (1.5, 2].
    \end{cases}
\end{align*}
Then which one of the following statements is true?
\begin{enumerate}
    \item [(A)] $X$ is a random variable with respect to $\mathcal{G}$, but $Y$ is not a random variable with respect to $\mathcal{G}$.
    \item [(B)] $Y$ is a random variable with respect to $\mathcal{G}$, but $X$ is not a random variable with respect to $\mathcal{G}$.
    \item [(C)] Neither $X$ nor $Y$ is a random variable with respect to $\mathcal{G}$.
    \item [(D)] Both $X$ and $Y$ are random variables with respect to $\mathcal{G}$.
\end{enumerate} \hfill (GATE ST 2023)\\
\solution
%\begin{table}[H]
	\centering
\begin{tabular}{|c|c|c|}
\hline
Random variable &Value &Definition\\ \hline
\multirow{3}{*}{X} &0 &Slips of Rs 1\\
&1 &Slips of Rs 5\\
&2 &Slips of Rs 13\\ \hline
\multirow{2}{*}{Y} &0 &Box A\\
&1 &Box B\\\hline
\end{tabular}
\caption{}
\label{tab:Distribution}
\end{table}
See \tabref{tab:Distribution}.
\begin{align}
p_{Y}\brak{k}= \begin{cases} 
      \frac{1}{3} & {k=0} \\
      \frac{2}{3 }& {k=1} 
   \end{cases}
   \\
p_{Y|X}\brak{0|0} = \frac{19}{25}\, 
p_{Y|X}\brak{0|1} = \frac{6}{25}\,
p_{Y|X}\brak{1|0} = \frac{45}{50}\,
p_{Y|X}\brak{1|2} = \frac{5}{50}
\end{align}
The desired probability is the probability that a slip drawn at random is marked other than Rs 1,
\begin{align}
&=1-p_X\brak{0}\\
&= p_X(1) + p_X(2)
\end{align}
Using Bayes theorem,
\begin{align}
&= p_Y\brak{0} \times \pr{Y=0 | X=1} + p_Y\brak{1} \times \pr{Y=1|X=2}\\
&=\frac{1}{3} \times \frac{6}{25} + \frac{2}{3} \times \frac{5}{50}\\
&=\frac{11}{75}
\end{align}

\newpage

%\tableofcontents

\bigskip

\renewcommand{\thefigure}{\theenumi}
\renewcommand{\thetable}{\theenumi}
%\renewcommand{\theequation}{\theenumi}

%\begin{abstract}
%%\boldmath
%In this letter, an algorithm for evaluating the exact analytical bit error rate  (BER)  for the piecewise linear (PL) combiner for  multiple relays is presented. Previous results were available only for upto three relays. The algorithm is unique in the sense that  the actual mathematical expressions, that are prohibitively large, need not be explicitly obtained. The diversity gain due to multiple relays is shown through plots of the analytical BER, well supported by simulations. 
%
%\end{abstract}
% IEEEtran.cls defaults to using nonbold math in the Abstract.
% This preserves the distinction between vectors and scalars. However,
% if the journal you are submitting to favors bold math in the abstract,
% then you can use LaTeX's standard command \boldmath at the very start
% of the abstract to achieve this. Many IEEE journals frown on math
% in the abstract anyway.

% Note that keywords are not normally used for peerreview papers.
%\begin{IEEEkeywords}
%Cooperative diversity, decode and forward, piecewise linear
%\end{IEEEkeywords}



% For peer review papers, you can put extra information on the cover
% page as needed:
% \ifCLASSOPTIONpeerreview
% \begin{center} \bfseries EDICS Category: 3-BBND \end{center}
% \fi
%
% For peerreview papers, this IEEEtran command inserts a page break and
% creates the second title. It will be ignored for other modes.
%\IEEEpeerreviewmaketitle




	\item  A die is loaded in such a way that each odd number is twice as likely to occur as
each even number. Find $P(G)$, where $G$ is the event that a number greater than
3 occurs on a single roll of the die.
\\
\solution
		%\begin{table}[H]
	\centering
\begin{tabular}{|c|c|c|}
\hline
Random variable &Value &Definition\\ \hline
\multirow{3}{*}{X} &0 &Slips of Rs 1\\
&1 &Slips of Rs 5\\
&2 &Slips of Rs 13\\ \hline
\multirow{2}{*}{Y} &0 &Box A\\
&1 &Box B\\\hline
\end{tabular}
\caption{}
\label{tab:Distribution}
\end{table}
See \tabref{tab:Distribution}.
\begin{align}
p_{Y}\brak{k}= \begin{cases} 
      \frac{1}{3} & {k=0} \\
      \frac{2}{3 }& {k=1} 
   \end{cases}
   \\
p_{Y|X}\brak{0|0} = \frac{19}{25}\, 
p_{Y|X}\brak{0|1} = \frac{6}{25}\,
p_{Y|X}\brak{1|0} = \frac{45}{50}\,
p_{Y|X}\brak{1|2} = \frac{5}{50}
\end{align}
The desired probability is the probability that a slip drawn at random is marked other than Rs 1,
\begin{align}
&=1-p_X\brak{0}\\
&= p_X(1) + p_X(2)
\end{align}
Using Bayes theorem,
\begin{align}
&= p_Y\brak{0} \times \pr{Y=0 | X=1} + p_Y\brak{1} \times \pr{Y=1|X=2}\\
&=\frac{1}{3} \times \frac{6}{25} + \frac{2}{3} \times \frac{5}{50}\\
&=\frac{11}{75}
\end{align}

\newpage

%\tableofcontents

\bigskip

\renewcommand{\thefigure}{\theenumi}
\renewcommand{\thetable}{\theenumi}
%\renewcommand{\theequation}{\theenumi}

%\begin{abstract}
%%\boldmath
%In this letter, an algorithm for evaluating the exact analytical bit error rate  (BER)  for the piecewise linear (PL) combiner for  multiple relays is presented. Previous results were available only for upto three relays. The algorithm is unique in the sense that  the actual mathematical expressions, that are prohibitively large, need not be explicitly obtained. The diversity gain due to multiple relays is shown through plots of the analytical BER, well supported by simulations. 
%
%\end{abstract}
% IEEEtran.cls defaults to using nonbold math in the Abstract.
% This preserves the distinction between vectors and scalars. However,
% if the journal you are submitting to favors bold math in the abstract,
% then you can use LaTeX's standard command \boldmath at the very start
% of the abstract to achieve this. Many IEEE journals frown on math
% in the abstract anyway.

% Note that keywords are not normally used for peerreview papers.
%\begin{IEEEkeywords}
%Cooperative diversity, decode and forward, piecewise linear
%\end{IEEEkeywords}



% For peer review papers, you can put extra information on the cover
% page as needed:
% \ifCLASSOPTIONpeerreview
% \begin{center} \bfseries EDICS Category: 3-BBND \end{center}
% \fi
%
% For peerreview papers, this IEEEtran command inserts a page break and
% creates the second title. It will be ignored for other modes.
%\IEEEpeerreviewmaketitle




	\item All the jacks, queens and kings are removed from a deck of 52 playing cards. The remaining cards are well shuffled and then one card is drawn at random. Giving ace a value 1 similar value for other cards, find the probability that the card has a value 
		\begin{enumerate}
			\item 7
			\item greater than 7
			\item less than 7
		\end{enumerate}
		%Number of cards left after removing all jacks, queens and kings 
\begin{align}
N	= 52 - 4\times 3
	= 40
\end{align}
%\begin{table}[H]
%\def\arraystretch{1.2}
%\begin{tabular}{|c|c|c|}
%\hline
%	\textbf{Parameter} &\textbf{Value} &\textbf{Description}\\ \hline
%	$X$ &1-10 &Represents the value of the card picked \\ \hline
%\end{tabular}
%\end{table}
Let $1 \le X \le 10$ be the value of the card picked.  Then,
\begin{align}
	p_X(k) &= \Pr(X=k)\ \forall\ 1 \leq k \leq 10\\
	&= \frac{4\times 1}{40}\\
	&= \frac{1}{10}\\
	\therefore p_X(k) &= 
	\begin{cases}
		\frac{1}{10} & 1 \leq k \leq 10\\
		0 & \text{otherwise}
	\end{cases}
\end{align}
and
\begin{align}
	F_{X}(k) &= \sum_{m=0}^{k}p_{X}(m) \quad 1 \leq k \leq 10\\
	&= \frac{k}{10}\\
	\therefore F_{X}(k) &= 
	\begin{cases}
		0 & k \leq 0\\
		\frac{k}{10} & 1\leq k \leq 10\\
		1 & k > 10 
	\end{cases}
\end{align}
\begin{enumerate}
	\item Probability that card has value equal to 7 is
		\begin{align}
			 p_{X}(7)
			= \frac{1}{10}
		\end{align}
	\item Probability that card has value greater than 7 is
		\begin{align}
			1 - F_X(7)
			&= 1 - \frac{7}{10}
			\\
			&= \frac{3}{10}
		\end{align}
	\item Probability that card has value less than 7 is
		\begin{align}
			 F_{X}(6)
			=\frac{6}{10}
		\end{align}
\end{enumerate}

  \item A Lot consists of 48 mobile phones of which 42 are good, 3 have only minor defects and 3 have major defects.Varnika will buy a phone if it is good but the trader will only buy a mobile if it has no major defects. One phone is selected at random from the lot. What is the probability that it is
\begin{enumerate}
	\item acceptable to Varnika?
            \item acceptable to the trader?
\end{enumerate}
\solution
	%\begin{table}[H]
	\centering
\begin{tabular}{|c|c|c|}
\hline
Random variable &Value &Definition\\ \hline
\multirow{3}{*}{X} &0 &Slips of Rs 1\\
&1 &Slips of Rs 5\\
&2 &Slips of Rs 13\\ \hline
\multirow{2}{*}{Y} &0 &Box A\\
&1 &Box B\\\hline
\end{tabular}
\caption{}
\label{tab:Distribution}
\end{table}
See \tabref{tab:Distribution}.
\begin{align}
p_{Y}\brak{k}= \begin{cases} 
      \frac{1}{3} & {k=0} \\
      \frac{2}{3 }& {k=1} 
   \end{cases}
   \\
p_{Y|X}\brak{0|0} = \frac{19}{25}\, 
p_{Y|X}\brak{0|1} = \frac{6}{25}\,
p_{Y|X}\brak{1|0} = \frac{45}{50}\,
p_{Y|X}\brak{1|2} = \frac{5}{50}
\end{align}
The desired probability is the probability that a slip drawn at random is marked other than Rs 1,
\begin{align}
&=1-p_X\brak{0}\\
&= p_X(1) + p_X(2)
\end{align}
Using Bayes theorem,
\begin{align}
&= p_Y\brak{0} \times \pr{Y=0 | X=1} + p_Y\brak{1} \times \pr{Y=1|X=2}\\
&=\frac{1}{3} \times \frac{6}{25} + \frac{2}{3} \times \frac{5}{50}\\
&=\frac{11}{75}
\end{align}

\newpage

%\tableofcontents

\bigskip

\renewcommand{\thefigure}{\theenumi}
\renewcommand{\thetable}{\theenumi}
%\renewcommand{\theequation}{\theenumi}

%\begin{abstract}
%%\boldmath
%In this letter, an algorithm for evaluating the exact analytical bit error rate  (BER)  for the piecewise linear (PL) combiner for  multiple relays is presented. Previous results were available only for upto three relays. The algorithm is unique in the sense that  the actual mathematical expressions, that are prohibitively large, need not be explicitly obtained. The diversity gain due to multiple relays is shown through plots of the analytical BER, well supported by simulations. 
%
%\end{abstract}
% IEEEtran.cls defaults to using nonbold math in the Abstract.
% This preserves the distinction between vectors and scalars. However,
% if the journal you are submitting to favors bold math in the abstract,
% then you can use LaTeX's standard command \boldmath at the very start
% of the abstract to achieve this. Many IEEE journals frown on math
% in the abstract anyway.

% Note that keywords are not normally used for peerreview papers.
%\begin{IEEEkeywords}
%Cooperative diversity, decode and forward, piecewise linear
%\end{IEEEkeywords}



% For peer review papers, you can put extra information on the cover
% page as needed:
% \ifCLASSOPTIONpeerreview
% \begin{center} \bfseries EDICS Category: 3-BBND \end{center}
% \fi
%
% For peerreview papers, this IEEEtran command inserts a page break and
% creates the second title. It will be ignored for other modes.
%\IEEEpeerreviewmaketitle




 \item A student says that if you throw a die, it will show up 1 or not 1. Therefore, the probability of getting 1 and the probability of getting 'not 1' each is equal to $\frac{1}{2}$. Is this correct? Give reasons.\\
 \solution
        %\begin{table}[H]
	\centering
\begin{tabular}{|c|c|c|}
\hline
Random variable &Value &Definition\\ \hline
\multirow{3}{*}{X} &0 &Slips of Rs 1\\
&1 &Slips of Rs 5\\
&2 &Slips of Rs 13\\ \hline
\multirow{2}{*}{Y} &0 &Box A\\
&1 &Box B\\\hline
\end{tabular}
\caption{}
\label{tab:Distribution}
\end{table}
See \tabref{tab:Distribution}.
\begin{align}
p_{Y}\brak{k}= \begin{cases} 
      \frac{1}{3} & {k=0} \\
      \frac{2}{3 }& {k=1} 
   \end{cases}
   \\
p_{Y|X}\brak{0|0} = \frac{19}{25}\, 
p_{Y|X}\brak{0|1} = \frac{6}{25}\,
p_{Y|X}\brak{1|0} = \frac{45}{50}\,
p_{Y|X}\brak{1|2} = \frac{5}{50}
\end{align}
The desired probability is the probability that a slip drawn at random is marked other than Rs 1,
\begin{align}
&=1-p_X\brak{0}\\
&= p_X(1) + p_X(2)
\end{align}
Using Bayes theorem,
\begin{align}
&= p_Y\brak{0} \times \pr{Y=0 | X=1} + p_Y\brak{1} \times \pr{Y=1|X=2}\\
&=\frac{1}{3} \times \frac{6}{25} + \frac{2}{3} \times \frac{5}{50}\\
&=\frac{11}{75}
\end{align}

\newpage

%\tableofcontents

\bigskip

\renewcommand{\thefigure}{\theenumi}
\renewcommand{\thetable}{\theenumi}
%\renewcommand{\theequation}{\theenumi}

%\begin{abstract}
%%\boldmath
%In this letter, an algorithm for evaluating the exact analytical bit error rate  (BER)  for the piecewise linear (PL) combiner for  multiple relays is presented. Previous results were available only for upto three relays. The algorithm is unique in the sense that  the actual mathematical expressions, that are prohibitively large, need not be explicitly obtained. The diversity gain due to multiple relays is shown through plots of the analytical BER, well supported by simulations. 
%
%\end{abstract}
% IEEEtran.cls defaults to using nonbold math in the Abstract.
% This preserves the distinction between vectors and scalars. However,
% if the journal you are submitting to favors bold math in the abstract,
% then you can use LaTeX's standard command \boldmath at the very start
% of the abstract to achieve this. Many IEEE journals frown on math
% in the abstract anyway.

% Note that keywords are not normally used for peerreview papers.
%\begin{IEEEkeywords}
%Cooperative diversity, decode and forward, piecewise linear
%\end{IEEEkeywords}



% For peer review papers, you can put extra information on the cover
% page as needed:
% \ifCLASSOPTIONpeerreview
% \begin{center} \bfseries EDICS Category: 3-BBND \end{center}
% \fi
%
% For peerreview papers, this IEEEtran command inserts a page break and
% creates the second title. It will be ignored for other modes.
%\IEEEpeerreviewmaketitle




   \item Four candidates A, B, C, D have ap-
plied for the assignment to coach a school cricket
team. If A is twice as likely to be selected as B, and
B and C are given about the same chance of being
selected, while C is twice as likely to be selected
as D, what are the probabilities that
\begin{enumerate}
\item C will be selected?
\item A will not be selected?
\end{enumerate}
	%\begin{table}[H]
	\centering
\begin{tabular}{|c|c|c|}
\hline
Random variable &Value &Definition\\ \hline
\multirow{3}{*}{X} &0 &Slips of Rs 1\\
&1 &Slips of Rs 5\\
&2 &Slips of Rs 13\\ \hline
\multirow{2}{*}{Y} &0 &Box A\\
&1 &Box B\\\hline
\end{tabular}
\caption{}
\label{tab:Distribution}
\end{table}
See \tabref{tab:Distribution}.
\begin{align}
p_{Y}\brak{k}= \begin{cases} 
      \frac{1}{3} & {k=0} \\
      \frac{2}{3 }& {k=1} 
   \end{cases}
   \\
p_{Y|X}\brak{0|0} = \frac{19}{25}\, 
p_{Y|X}\brak{0|1} = \frac{6}{25}\,
p_{Y|X}\brak{1|0} = \frac{45}{50}\,
p_{Y|X}\brak{1|2} = \frac{5}{50}
\end{align}
The desired probability is the probability that a slip drawn at random is marked other than Rs 1,
\begin{align}
&=1-p_X\brak{0}\\
&= p_X(1) + p_X(2)
\end{align}
Using Bayes theorem,
\begin{align}
&= p_Y\brak{0} \times \pr{Y=0 | X=1} + p_Y\brak{1} \times \pr{Y=1|X=2}\\
&=\frac{1}{3} \times \frac{6}{25} + \frac{2}{3} \times \frac{5}{50}\\
&=\frac{11}{75}
\end{align}

\newpage

%\tableofcontents

\bigskip

\renewcommand{\thefigure}{\theenumi}
\renewcommand{\thetable}{\theenumi}
%\renewcommand{\theequation}{\theenumi}

%\begin{abstract}
%%\boldmath
%In this letter, an algorithm for evaluating the exact analytical bit error rate  (BER)  for the piecewise linear (PL) combiner for  multiple relays is presented. Previous results were available only for upto three relays. The algorithm is unique in the sense that  the actual mathematical expressions, that are prohibitively large, need not be explicitly obtained. The diversity gain due to multiple relays is shown through plots of the analytical BER, well supported by simulations. 
%
%\end{abstract}
% IEEEtran.cls defaults to using nonbold math in the Abstract.
% This preserves the distinction between vectors and scalars. However,
% if the journal you are submitting to favors bold math in the abstract,
% then you can use LaTeX's standard command \boldmath at the very start
% of the abstract to achieve this. Many IEEE journals frown on math
% in the abstract anyway.

% Note that keywords are not normally used for peerreview papers.
%\begin{IEEEkeywords}
%Cooperative diversity, decode and forward, piecewise linear
%\end{IEEEkeywords}



% For peer review papers, you can put extra information on the cover
% page as needed:
% \ifCLASSOPTIONpeerreview
% \begin{center} \bfseries EDICS Category: 3-BBND \end{center}
% \fi
%
% For peerreview papers, this IEEEtran command inserts a page break and
% creates the second title. It will be ignored for other modes.
%\IEEEpeerreviewmaketitle




 \item A bag contain 24 balls of which $x$ balls are red, $2x$ are white and $3x$ are blue. A ball is selected at random, What is the probability that it is
\begin{enumerate}[label=\alph*)]
\item not red ?
\item white ?
\end{enumerate}
%\begin{table}[H]
	\centering
\begin{tabular}{|c|c|c|}
\hline
Random variable &Value &Definition\\ \hline
\multirow{3}{*}{X} &0 &Slips of Rs 1\\
&1 &Slips of Rs 5\\
&2 &Slips of Rs 13\\ \hline
\multirow{2}{*}{Y} &0 &Box A\\
&1 &Box B\\\hline
\end{tabular}
\caption{}
\label{tab:Distribution}
\end{table}
See \tabref{tab:Distribution}.
\begin{align}
p_{Y}\brak{k}= \begin{cases} 
      \frac{1}{3} & {k=0} \\
      \frac{2}{3 }& {k=1} 
   \end{cases}
   \\
p_{Y|X}\brak{0|0} = \frac{19}{25}\, 
p_{Y|X}\brak{0|1} = \frac{6}{25}\,
p_{Y|X}\brak{1|0} = \frac{45}{50}\,
p_{Y|X}\brak{1|2} = \frac{5}{50}
\end{align}
The desired probability is the probability that a slip drawn at random is marked other than Rs 1,
\begin{align}
&=1-p_X\brak{0}\\
&= p_X(1) + p_X(2)
\end{align}
Using Bayes theorem,
\begin{align}
&= p_Y\brak{0} \times \pr{Y=0 | X=1} + p_Y\brak{1} \times \pr{Y=1|X=2}\\
&=\frac{1}{3} \times \frac{6}{25} + \frac{2}{3} \times \frac{5}{50}\\
&=\frac{11}{75}
\end{align}

\newpage

%\tableofcontents

\bigskip

\renewcommand{\thefigure}{\theenumi}
\renewcommand{\thetable}{\theenumi}
%\renewcommand{\theequation}{\theenumi}

%\begin{abstract}
%%\boldmath
%In this letter, an algorithm for evaluating the exact analytical bit error rate  (BER)  for the piecewise linear (PL) combiner for  multiple relays is presented. Previous results were available only for upto three relays. The algorithm is unique in the sense that  the actual mathematical expressions, that are prohibitively large, need not be explicitly obtained. The diversity gain due to multiple relays is shown through plots of the analytical BER, well supported by simulations. 
%
%\end{abstract}
% IEEEtran.cls defaults to using nonbold math in the Abstract.
% This preserves the distinction between vectors and scalars. However,
% if the journal you are submitting to favors bold math in the abstract,
% then you can use LaTeX's standard command \boldmath at the very start
% of the abstract to achieve this. Many IEEE journals frown on math
% in the abstract anyway.

% Note that keywords are not normally used for peerreview papers.
%\begin{IEEEkeywords}
%Cooperative diversity, decode and forward, piecewise linear
%\end{IEEEkeywords}



% For peer review papers, you can put extra information on the cover
% page as needed:
% \ifCLASSOPTIONpeerreview
% \begin{center} \bfseries EDICS Category: 3-BBND \end{center}
% \fi
%
% For peerreview papers, this IEEEtran command inserts a page break and
% creates the second title. It will be ignored for other modes.
%\IEEEpeerreviewmaketitle




If the letters of the word ASSASSINATION are arranged at random. Find the Probability that
\begin{enumerate}[label=(\alph*)]
\item Four $S's$ come consecutively in the word
\item Two  $I's$ and two $N's$ come together
\item All $A's$ are not coming together
\item No two $A's$ are coming together
\end{enumerate}
%\begin{table}[H]
	\centering
\begin{tabular}{|c|c|c|}
\hline
Random variable &Value &Definition\\ \hline
\multirow{3}{*}{X} &0 &Slips of Rs 1\\
&1 &Slips of Rs 5\\
&2 &Slips of Rs 13\\ \hline
\multirow{2}{*}{Y} &0 &Box A\\
&1 &Box B\\\hline
\end{tabular}
\caption{}
\label{tab:Distribution}
\end{table}
See \tabref{tab:Distribution}.
\begin{align}
p_{Y}\brak{k}= \begin{cases} 
      \frac{1}{3} & {k=0} \\
      \frac{2}{3 }& {k=1} 
   \end{cases}
   \\
p_{Y|X}\brak{0|0} = \frac{19}{25}\, 
p_{Y|X}\brak{0|1} = \frac{6}{25}\,
p_{Y|X}\brak{1|0} = \frac{45}{50}\,
p_{Y|X}\brak{1|2} = \frac{5}{50}
\end{align}
The desired probability is the probability that a slip drawn at random is marked other than Rs 1,
\begin{align}
&=1-p_X\brak{0}\\
&= p_X(1) + p_X(2)
\end{align}
Using Bayes theorem,
\begin{align}
&= p_Y\brak{0} \times \pr{Y=0 | X=1} + p_Y\brak{1} \times \pr{Y=1|X=2}\\
&=\frac{1}{3} \times \frac{6}{25} + \frac{2}{3} \times \frac{5}{50}\\
&=\frac{11}{75}
\end{align}

\newpage

%\tableofcontents

\bigskip

\renewcommand{\thefigure}{\theenumi}
\renewcommand{\thetable}{\theenumi}
%\renewcommand{\theequation}{\theenumi}

%\begin{abstract}
%%\boldmath
%In this letter, an algorithm for evaluating the exact analytical bit error rate  (BER)  for the piecewise linear (PL) combiner for  multiple relays is presented. Previous results were available only for upto three relays. The algorithm is unique in the sense that  the actual mathematical expressions, that are prohibitively large, need not be explicitly obtained. The diversity gain due to multiple relays is shown through plots of the analytical BER, well supported by simulations. 
%
%\end{abstract}
% IEEEtran.cls defaults to using nonbold math in the Abstract.
% This preserves the distinction between vectors and scalars. However,
% if the journal you are submitting to favors bold math in the abstract,
% then you can use LaTeX's standard command \boldmath at the very start
% of the abstract to achieve this. Many IEEE journals frown on math
% in the abstract anyway.

% Note that keywords are not normally used for peerreview papers.
%\begin{IEEEkeywords}
%Cooperative diversity, decode and forward, piecewise linear
%\end{IEEEkeywords}



% For peer review papers, you can put extra information on the cover
% page as needed:
% \ifCLASSOPTIONpeerreview
% \begin{center} \bfseries EDICS Category: 3-BBND \end{center}
% \fi
%
% For peerreview papers, this IEEEtran command inserts a page break and
% creates the second title. It will be ignored for other modes.
%\IEEEpeerreviewmaketitle




	\item One urn contains two black balls (labelled B1 and B2) and one white ball. A
	second urn contains one black ball and two white balls (labelled W1 and W2).
	Suppose the following experiment is performed. One of the two urns is chosen
	at random. Next a ball is randomly chosen from the urn. Then a second ball is
	chosen at random from the same urn without replacing the first ball.
	
	\begin{enumerate}
	\item What is the probability that two black balls are chosen?
	
	\item What is the probability that two balls of opposite colour are chosen?
	\end{enumerate}
	\solution
	%\begin{align}
    \label{eq:12.13.6.18.1}
	\because	\pr{A|B} &> \pr{A},\
\frac{\pr{AB}}{\pr{B}} > \pr{A}
\\
    \label{eq:12.13.6.18.2}
	\implies \pr{AB} &> \pr{A}\pr{B}
	\\
	\text{or, } \frac{\pr{AB}}{\pr{A}} &=\pr{B|A} > \pr{A}
\end{align}

\end{enumerate}

	\item A bag contains $5$ red balls and some blue balls. If the probability of drawing a blue ball is double that if a red ball, determine the number of blue balls in the bag. 
		\\
\solution
		%\begin{enumerate}[label=\thesection.\arabic*,ref=\thesection.\theenumi]
	\item One card is drawn from a well-shuffled deck of 52 cards. Find the probability of getting
\begin{enumerate}
\item A king of red colour 
\item A face card 
\item A red face card
\item The jack of hearts
\item A spade
\item The queen of diamonds

\end{enumerate}
\solution
		%\begin{table}[H]
	\centering
\begin{tabular}{|c|c|c|}
\hline
Random variable &Value &Definition\\ \hline
\multirow{3}{*}{X} &0 &Slips of Rs 1\\
&1 &Slips of Rs 5\\
&2 &Slips of Rs 13\\ \hline
\multirow{2}{*}{Y} &0 &Box A\\
&1 &Box B\\\hline
\end{tabular}
\caption{}
\label{tab:Distribution}
\end{table}
See \tabref{tab:Distribution}.
\begin{align}
p_{Y}\brak{k}= \begin{cases} 
      \frac{1}{3} & {k=0} \\
      \frac{2}{3 }& {k=1} 
   \end{cases}
   \\
p_{Y|X}\brak{0|0} = \frac{19}{25}\, 
p_{Y|X}\brak{0|1} = \frac{6}{25}\,
p_{Y|X}\brak{1|0} = \frac{45}{50}\,
p_{Y|X}\brak{1|2} = \frac{5}{50}
\end{align}
The desired probability is the probability that a slip drawn at random is marked other than Rs 1,
\begin{align}
&=1-p_X\brak{0}\\
&= p_X(1) + p_X(2)
\end{align}
Using Bayes theorem,
\begin{align}
&= p_Y\brak{0} \times \pr{Y=0 | X=1} + p_Y\brak{1} \times \pr{Y=1|X=2}\\
&=\frac{1}{3} \times \frac{6}{25} + \frac{2}{3} \times \frac{5}{50}\\
&=\frac{11}{75}
\end{align}

\newpage

%\tableofcontents

\bigskip

\renewcommand{\thefigure}{\theenumi}
\renewcommand{\thetable}{\theenumi}
%\renewcommand{\theequation}{\theenumi}

%\begin{abstract}
%%\boldmath
%In this letter, an algorithm for evaluating the exact analytical bit error rate  (BER)  for the piecewise linear (PL) combiner for  multiple relays is presented. Previous results were available only for upto three relays. The algorithm is unique in the sense that  the actual mathematical expressions, that are prohibitively large, need not be explicitly obtained. The diversity gain due to multiple relays is shown through plots of the analytical BER, well supported by simulations. 
%
%\end{abstract}
% IEEEtran.cls defaults to using nonbold math in the Abstract.
% This preserves the distinction between vectors and scalars. However,
% if the journal you are submitting to favors bold math in the abstract,
% then you can use LaTeX's standard command \boldmath at the very start
% of the abstract to achieve this. Many IEEE journals frown on math
% in the abstract anyway.

% Note that keywords are not normally used for peerreview papers.
%\begin{IEEEkeywords}
%Cooperative diversity, decode and forward, piecewise linear
%\end{IEEEkeywords}



% For peer review papers, you can put extra information on the cover
% page as needed:
% \ifCLASSOPTIONpeerreview
% \begin{center} \bfseries EDICS Category: 3-BBND \end{center}
% \fi
%
% For peerreview papers, this IEEEtran command inserts a page break and
% creates the second title. It will be ignored for other modes.
%\IEEEpeerreviewmaketitle




	\item Five cards—the ten, jack, queen, king and ace of diamonds, are well-shuffled with their face downwards. One card is then picked up at random.
\begin{enumerate}
\item
What is the probability that the card is the queen? 
\item
If the queen is drawn and put aside, what is the probability that the second card picked up is (a) an ace? (b) a queen?\\
\end{enumerate}
\solution
		%\begin{enumerate}[label=\thesection.\arabic*,ref=\thesection.\theenumi]
	\item One card is drawn from a well-shuffled deck of 52 cards. Find the probability of getting
\begin{enumerate}
\item A king of red colour 
\item A face card 
\item A red face card
\item The jack of hearts
\item A spade
\item The queen of diamonds

\end{enumerate}
\solution
		%\input{ncert/10/15/1/14/main.tex}
	\item Five cards—the ten, jack, queen, king and ace of diamonds, are well-shuffled with their face downwards. One card is then picked up at random.
\begin{enumerate}
\item
What is the probability that the card is the queen? 
\item
If the queen is drawn and put aside, what is the probability that the second card picked up is (a) an ace? (b) a queen?\\
\end{enumerate}
\solution
		%\input{ncert/10/15/1/15/defs.tex}
	\item A bag contains $5$ red balls and some blue balls. If the probability of drawing a blue ball is double that if a red ball, determine the number of blue balls in the bag. 
		\\
\solution
		%\input{ncert/10/15/2/3/defs.tex}
	\item A card is selected from a pack of 52 cards.
 \begin{enumerate}[label=(\alph*)] 
                 \item How many points are there in the sample space?
                 \item Calculate the probability that the card is an ace of spades.
                 \item Calculate the probability that the card is (i) an ace and (ii) black card.
 \end{enumerate}
\solution
		%\input{ncert/11/16/3/4/main.tex}
\item Four cards are drawn from a well-shuffled deck of 52 cards. What is the probability of obtaining 3 diamonds and one spade.
\\
\solution
		%\input{ncert/11/16/4/2/defs.tex}
\item In a certain lottery 10,000 tickets are sold and ten equal prizes are awarded. What is the probability of not getting a prize if you buy (a) one ticket (b) two tickets (c) 10 tickets ?	
\\
\solution
		%\input{ncert/11/16/4/4/defs.tex}
		%
\item 
Out of 100 students, two sections of 40 and 60 are formed. If you and your friend are among the 100 students, what is the probability that
\begin{enumerate}
\item you both enter the same section?
\item you both enter the different sections?
\end{enumerate}
\solution
		%\input{ncert/11/16/4/5/defs.tex}
	\item 
The number lock of a suitcase has 4 wheels each labelled with ten digits i.e. from 0 to 9.The lock opens with a sequence of four digits with no repeats.What is the probability of a person getting the right sequence to open the suitcase.
\\
\solution
		%\input{ncert/11/16/4/10/defs.tex}
		%
\item 
Two cards are drawn at random and without replacement from a pack of 52 playing cards. Find the probability that both the cards are black.
\\
\solution
		%\input{ncert/12/13/2/2/defs.tex}
		\item A box of oranges is inspected by examining three randomly selected oranges drawn without replacement. If all the three oranges are good, the box is approved for sale, otherwise, it is rejected. Find the probability that a box containing 15 oranges out of which 12 are good and 3 are bad ones will be approved for sale.
		\label{ncert/12/13/2/3/defs.tex}
		\item Two balls are drawn at random with replacement from a box containing 10 black and 8 red balls. Find the probability that
		\label{ncert/12/13/2/12}
\begin{enumerate}
\item both balls are red.
\item first ball is black and second is red.
\item one of them is black and other is red.
\end{enumerate}

\item In a hostel, 60\% of the students read Hindi newspaper, 40\% read English newspaper and 20\% read both Hindi and English newspapers. A student is selected at random.
		\label{ncert/12/13/2/15}
\begin{enumerate}
\item Find the probability that she reads neither Hindi nor English newspapers.
\item If she reads Hindi newspaper, find the probability that she reads English newspaper.
\item If she reads English newspaper, find the probability that she reads Hindi newspaper.\\
\end{enumerate}
\item The probability of obtaining an even prime number on each die, when a pair of dice is rolled is 
\begin{enumerate}
    \item $0$ 
    
    \item $\frac{1}{3}$ 
    
    \item $\frac{1}{12}$ 
    
    \item $\frac{1}{36}$ 
\end{enumerate}
\solution
		%\input{ncert/12/13/2/17/defs.tex}
	\item A bag contains 4 red and 4 black balls, another bag contains 2 red and 6 black balls. One of the two bags is selected at random and a ball is drawn from the bag which is found to be red. Find the probability that the ball is drawn from the first bag.
\\
\solution
		%\input{ncert/12/13/3/2/main.tex}
  \item
  Cards with numbers 2 to 101 are placed in a box. A card is selected at random.Find the probability that the card has
\begin{enumerate}[label=(\roman*)]
	\item an even number 
	\item a square number
\end{enumerate}
\solution
%\input{exemplar/10/13/3/32/main.tex}
\item
The king, queen and jack of clubs are removed from a deck of 52 playing cards and then well shuffled. Now one card is drawn at random from the remaining cards.  Determine the probability that the card is
\begin{enumerate}[label=(\roman*)]
\item a club
\item 10 of hearts
\end{enumerate}
\solution
%\input{exemplar/10/13/3/29/main.tex}
\item A team of medical students doing their internship have to assist during surgeries
at a city hospital. The probabilities of surgeries rated as very complex, complex,
routine, simple or very simple are respectively, 0.15, 0.20, 0.31, 0.26, .08. Find
the probabilities that a particular surgery will be rated
\begin{enumerate}
	\item complex or very complex;
	\item neither very complex nor very simple;
	\item routine or complex
	\item routine or simple
\end{enumerate}
\solution
%\input{exemplar/11/16/3/8(1)/main.tex}
\item A card is selected from a pack of 52 cards.
\begin{enumerate}[label=(\alph*)]
    \item How many points are there in the sample space?
    \item Calculate the probability that the card is an ace of spades.
    \item Calculate the probability that the card is (i) an ace and (ii) black card.
\end{enumerate}
\solution
%\input{exemplar/11/16/3/4/main2.tex}
\item The probability that a non leap year selected at random will contain 53 sundays.
\\
\solution
%\input{exemplar/10/13/1/19/main.tex}
\item One of the four persons John, Rita, Aslam or Gurpreet will be promoted next
month. Consequently the sample space consists of four elementary outcomes
S = {John promoted, Rita promoted, Aslam promoted, Gurpreet promoted}
You are told that the chances of John’s promotion is same as that of Gurpreet,
Rita’s chances of promotion are twice as likely as Johns. Aslam’s chances are
four times that of John.
\begin{enumerate}
	\item Determine
	\begin{enumerate}
		\item P (John promoted)
		\item P (Rita promoted)
		\item P (Aslam promoted)
		\item P (Gurpreet promoted)
	\end{enumerate}
	\item If A = {John promoted or Gurpreet promoted}, find P (A).
\end{enumerate}
\solution
%\input{exemplar/11/16/3/10/main.tex}
\item A card is drawn from a deck of 52 cards. Find the probability of getting a king or a heart or a red card.\\
\solution
%\input{exemplar/11/16/3/15/main.tex}
\item The probability that a student will pass his examination is 0.73, the probability of
the student getting a compartment is 0.13, and the probability that the student will
either pass or get compartment is 0.96. State True or False.\\
\solution
%\input{exemplar/11/16/3/31/main.tex}
\item A card is selected from a pack of 52 cards\\
\begin{enumerate}[label=(\alph*)]
\item How many points are there in the sample space?
\item Calculate the probability that the cards is an ace of spades.
\item Calculate the probability that the card is (i) an ace (ii)black card.\\
\end{enumerate}
%\input{ncert/11/16/3/4_1/Prob_4.tex}
\item In a non-leap year, the probability of having 53 tuesdays or 53 wednesdays is\\
\solution
%\input{exemplar/11/16/3/18/main.tex}
\item There are 1000 sealed envelopes in a box, 10 of them contain a cash prize of
Rs 100 each, 100 of them contain a cash prize of Rs 50 each and 200 of them
contain a cash prize of Rs 10 each and rest do not contain any cash prize. If they
are well shuffled and an envelope is picked up out, what is the probability that it
contains no cash prize?\\
\solution
%\input{exemplar/10/13/3/34/main.tex}
\item 
A die is thrown and a card is selected at random from a deck of 52 playing cards. The probability of getting an even number on the die and a spade card.\\
\solution
%\input{exemplar/12/13/3/78/main.tex}
\item
If 4-digit numbers greater than 5,000 are randomly formed from the digits 0, 1, 3, 5, and 7, what is the probability of forming a number divisible by 5 when:
\begin{enumerate}
    \item The digits are repeated?
    \item The repetition of digits is not allowed?
\end{enumerate}
\solution
%\input{ncert/11/16/4/9/main.tex}
\item Consider the probability space $\brak{\Omega, \mathcal{G}, P}$ where $\Omega = [0,2]$ and $\mathcal{G} = \cbrak{\phi, \Omega, [0,1], (1,2]}$. Let $X$ and $Y$ be two functions on $\Omega$ defined as
\begin{align*}
    X(\omega) = 
    \begin{cases}
        1 & \text{if }\omega \in [0, 1]\\
        2 & \text{if }\omega \in (1, 2]
    \end{cases}
\end{align*}
and
\begin{align*}
    Y(\omega) = 
    \begin{cases}
        2 & \text{if }\omega \in [0, 1.5]\\
        3 & \text{if }\omega \in (1.5, 2].
    \end{cases}
\end{align*}
Then which one of the following statements is true?
\begin{enumerate}
    \item [(A)] $X$ is a random variable with respect to $\mathcal{G}$, but $Y$ is not a random variable with respect to $\mathcal{G}$.
    \item [(B)] $Y$ is a random variable with respect to $\mathcal{G}$, but $X$ is not a random variable with respect to $\mathcal{G}$.
    \item [(C)] Neither $X$ nor $Y$ is a random variable with respect to $\mathcal{G}$.
    \item [(D)] Both $X$ and $Y$ are random variables with respect to $\mathcal{G}$.
\end{enumerate} \hfill (GATE ST 2023)\\
\solution
%\input{gate/ST/2023/14/main.tex}
	\item  A die is loaded in such a way that each odd number is twice as likely to occur as
each even number. Find $P(G)$, where $G$ is the event that a number greater than
3 occurs on a single roll of the die.
\\
\solution
		%\input{exemplar/11/16/3/5/main.tex}
	\item All the jacks, queens and kings are removed from a deck of 52 playing cards. The remaining cards are well shuffled and then one card is drawn at random. Giving ace a value 1 similar value for other cards, find the probability that the card has a value 
		\begin{enumerate}
			\item 7
			\item greater than 7
			\item less than 7
		\end{enumerate}
		%\input{exemplar/10/13/3/30/main.tex}
  \item A Lot consists of 48 mobile phones of which 42 are good, 3 have only minor defects and 3 have major defects.Varnika will buy a phone if it is good but the trader will only buy a mobile if it has no major defects. One phone is selected at random from the lot. What is the probability that it is
\begin{enumerate}
	\item acceptable to Varnika?
            \item acceptable to the trader?
\end{enumerate}
\solution
	%\input{exemplar/10/13/3/40/main.tex}
 \item A student says that if you throw a die, it will show up 1 or not 1. Therefore, the probability of getting 1 and the probability of getting 'not 1' each is equal to $\frac{1}{2}$. Is this correct? Give reasons.\\
 \solution
        %\input{exemplar/10/13/2/9/main.tex}
   \item Four candidates A, B, C, D have ap-
plied for the assignment to coach a school cricket
team. If A is twice as likely to be selected as B, and
B and C are given about the same chance of being
selected, while C is twice as likely to be selected
as D, what are the probabilities that
\begin{enumerate}
\item C will be selected?
\item A will not be selected?
\end{enumerate}
	%\input{exemplar/11/16/3/9/main.tex}
 \item A bag contain 24 balls of which $x$ balls are red, $2x$ are white and $3x$ are blue. A ball is selected at random, What is the probability that it is
\begin{enumerate}[label=\alph*)]
\item not red ?
\item white ?
\end{enumerate}
%\input{exemplar/10/13/3/41/main.tex}
If the letters of the word ASSASSINATION are arranged at random. Find the Probability that
\begin{enumerate}[label=(\alph*)]
\item Four $S's$ come consecutively in the word
\item Two  $I's$ and two $N's$ come together
\item All $A's$ are not coming together
\item No two $A's$ are coming together
\end{enumerate}
%\input{exemplar/11/16/3/14/main.tex}
	\item One urn contains two black balls (labelled B1 and B2) and one white ball. A
	second urn contains one black ball and two white balls (labelled W1 and W2).
	Suppose the following experiment is performed. One of the two urns is chosen
	at random. Next a ball is randomly chosen from the urn. Then a second ball is
	chosen at random from the same urn without replacing the first ball.
	
	\begin{enumerate}
	\item What is the probability that two black balls are chosen?
	
	\item What is the probability that two balls of opposite colour are chosen?
	\end{enumerate}
	\solution
	%\input{exemplar/11/16/3/12/main1.tex}
\end{enumerate}

	\item A bag contains $5$ red balls and some blue balls. If the probability of drawing a blue ball is double that if a red ball, determine the number of blue balls in the bag. 
		\\
\solution
		%\begin{enumerate}[label=\thesection.\arabic*,ref=\thesection.\theenumi]
	\item One card is drawn from a well-shuffled deck of 52 cards. Find the probability of getting
\begin{enumerate}
\item A king of red colour 
\item A face card 
\item A red face card
\item The jack of hearts
\item A spade
\item The queen of diamonds

\end{enumerate}
\solution
		%\input{ncert/10/15/1/14/main.tex}
	\item Five cards—the ten, jack, queen, king and ace of diamonds, are well-shuffled with their face downwards. One card is then picked up at random.
\begin{enumerate}
\item
What is the probability that the card is the queen? 
\item
If the queen is drawn and put aside, what is the probability that the second card picked up is (a) an ace? (b) a queen?\\
\end{enumerate}
\solution
		%\input{ncert/10/15/1/15/defs.tex}
	\item A bag contains $5$ red balls and some blue balls. If the probability of drawing a blue ball is double that if a red ball, determine the number of blue balls in the bag. 
		\\
\solution
		%\input{ncert/10/15/2/3/defs.tex}
	\item A card is selected from a pack of 52 cards.
 \begin{enumerate}[label=(\alph*)] 
                 \item How many points are there in the sample space?
                 \item Calculate the probability that the card is an ace of spades.
                 \item Calculate the probability that the card is (i) an ace and (ii) black card.
 \end{enumerate}
\solution
		%\input{ncert/11/16/3/4/main.tex}
\item Four cards are drawn from a well-shuffled deck of 52 cards. What is the probability of obtaining 3 diamonds and one spade.
\\
\solution
		%\input{ncert/11/16/4/2/defs.tex}
\item In a certain lottery 10,000 tickets are sold and ten equal prizes are awarded. What is the probability of not getting a prize if you buy (a) one ticket (b) two tickets (c) 10 tickets ?	
\\
\solution
		%\input{ncert/11/16/4/4/defs.tex}
		%
\item 
Out of 100 students, two sections of 40 and 60 are formed. If you and your friend are among the 100 students, what is the probability that
\begin{enumerate}
\item you both enter the same section?
\item you both enter the different sections?
\end{enumerate}
\solution
		%\input{ncert/11/16/4/5/defs.tex}
	\item 
The number lock of a suitcase has 4 wheels each labelled with ten digits i.e. from 0 to 9.The lock opens with a sequence of four digits with no repeats.What is the probability of a person getting the right sequence to open the suitcase.
\\
\solution
		%\input{ncert/11/16/4/10/defs.tex}
		%
\item 
Two cards are drawn at random and without replacement from a pack of 52 playing cards. Find the probability that both the cards are black.
\\
\solution
		%\input{ncert/12/13/2/2/defs.tex}
		\item A box of oranges is inspected by examining three randomly selected oranges drawn without replacement. If all the three oranges are good, the box is approved for sale, otherwise, it is rejected. Find the probability that a box containing 15 oranges out of which 12 are good and 3 are bad ones will be approved for sale.
		\label{ncert/12/13/2/3/defs.tex}
		\item Two balls are drawn at random with replacement from a box containing 10 black and 8 red balls. Find the probability that
		\label{ncert/12/13/2/12}
\begin{enumerate}
\item both balls are red.
\item first ball is black and second is red.
\item one of them is black and other is red.
\end{enumerate}

\item In a hostel, 60\% of the students read Hindi newspaper, 40\% read English newspaper and 20\% read both Hindi and English newspapers. A student is selected at random.
		\label{ncert/12/13/2/15}
\begin{enumerate}
\item Find the probability that she reads neither Hindi nor English newspapers.
\item If she reads Hindi newspaper, find the probability that she reads English newspaper.
\item If she reads English newspaper, find the probability that she reads Hindi newspaper.\\
\end{enumerate}
\item The probability of obtaining an even prime number on each die, when a pair of dice is rolled is 
\begin{enumerate}
    \item $0$ 
    
    \item $\frac{1}{3}$ 
    
    \item $\frac{1}{12}$ 
    
    \item $\frac{1}{36}$ 
\end{enumerate}
\solution
		%\input{ncert/12/13/2/17/defs.tex}
	\item A bag contains 4 red and 4 black balls, another bag contains 2 red and 6 black balls. One of the two bags is selected at random and a ball is drawn from the bag which is found to be red. Find the probability that the ball is drawn from the first bag.
\\
\solution
		%\input{ncert/12/13/3/2/main.tex}
  \item
  Cards with numbers 2 to 101 are placed in a box. A card is selected at random.Find the probability that the card has
\begin{enumerate}[label=(\roman*)]
	\item an even number 
	\item a square number
\end{enumerate}
\solution
%\input{exemplar/10/13/3/32/main.tex}
\item
The king, queen and jack of clubs are removed from a deck of 52 playing cards and then well shuffled. Now one card is drawn at random from the remaining cards.  Determine the probability that the card is
\begin{enumerate}[label=(\roman*)]
\item a club
\item 10 of hearts
\end{enumerate}
\solution
%\input{exemplar/10/13/3/29/main.tex}
\item A team of medical students doing their internship have to assist during surgeries
at a city hospital. The probabilities of surgeries rated as very complex, complex,
routine, simple or very simple are respectively, 0.15, 0.20, 0.31, 0.26, .08. Find
the probabilities that a particular surgery will be rated
\begin{enumerate}
	\item complex or very complex;
	\item neither very complex nor very simple;
	\item routine or complex
	\item routine or simple
\end{enumerate}
\solution
%\input{exemplar/11/16/3/8(1)/main.tex}
\item A card is selected from a pack of 52 cards.
\begin{enumerate}[label=(\alph*)]
    \item How many points are there in the sample space?
    \item Calculate the probability that the card is an ace of spades.
    \item Calculate the probability that the card is (i) an ace and (ii) black card.
\end{enumerate}
\solution
%\input{exemplar/11/16/3/4/main2.tex}
\item The probability that a non leap year selected at random will contain 53 sundays.
\\
\solution
%\input{exemplar/10/13/1/19/main.tex}
\item One of the four persons John, Rita, Aslam or Gurpreet will be promoted next
month. Consequently the sample space consists of four elementary outcomes
S = {John promoted, Rita promoted, Aslam promoted, Gurpreet promoted}
You are told that the chances of John’s promotion is same as that of Gurpreet,
Rita’s chances of promotion are twice as likely as Johns. Aslam’s chances are
four times that of John.
\begin{enumerate}
	\item Determine
	\begin{enumerate}
		\item P (John promoted)
		\item P (Rita promoted)
		\item P (Aslam promoted)
		\item P (Gurpreet promoted)
	\end{enumerate}
	\item If A = {John promoted or Gurpreet promoted}, find P (A).
\end{enumerate}
\solution
%\input{exemplar/11/16/3/10/main.tex}
\item A card is drawn from a deck of 52 cards. Find the probability of getting a king or a heart or a red card.\\
\solution
%\input{exemplar/11/16/3/15/main.tex}
\item The probability that a student will pass his examination is 0.73, the probability of
the student getting a compartment is 0.13, and the probability that the student will
either pass or get compartment is 0.96. State True or False.\\
\solution
%\input{exemplar/11/16/3/31/main.tex}
\item A card is selected from a pack of 52 cards\\
\begin{enumerate}[label=(\alph*)]
\item How many points are there in the sample space?
\item Calculate the probability that the cards is an ace of spades.
\item Calculate the probability that the card is (i) an ace (ii)black card.\\
\end{enumerate}
%\input{ncert/11/16/3/4_1/Prob_4.tex}
\item In a non-leap year, the probability of having 53 tuesdays or 53 wednesdays is\\
\solution
%\input{exemplar/11/16/3/18/main.tex}
\item There are 1000 sealed envelopes in a box, 10 of them contain a cash prize of
Rs 100 each, 100 of them contain a cash prize of Rs 50 each and 200 of them
contain a cash prize of Rs 10 each and rest do not contain any cash prize. If they
are well shuffled and an envelope is picked up out, what is the probability that it
contains no cash prize?\\
\solution
%\input{exemplar/10/13/3/34/main.tex}
\item 
A die is thrown and a card is selected at random from a deck of 52 playing cards. The probability of getting an even number on the die and a spade card.\\
\solution
%\input{exemplar/12/13/3/78/main.tex}
\item
If 4-digit numbers greater than 5,000 are randomly formed from the digits 0, 1, 3, 5, and 7, what is the probability of forming a number divisible by 5 when:
\begin{enumerate}
    \item The digits are repeated?
    \item The repetition of digits is not allowed?
\end{enumerate}
\solution
%\input{ncert/11/16/4/9/main.tex}
\item Consider the probability space $\brak{\Omega, \mathcal{G}, P}$ where $\Omega = [0,2]$ and $\mathcal{G} = \cbrak{\phi, \Omega, [0,1], (1,2]}$. Let $X$ and $Y$ be two functions on $\Omega$ defined as
\begin{align*}
    X(\omega) = 
    \begin{cases}
        1 & \text{if }\omega \in [0, 1]\\
        2 & \text{if }\omega \in (1, 2]
    \end{cases}
\end{align*}
and
\begin{align*}
    Y(\omega) = 
    \begin{cases}
        2 & \text{if }\omega \in [0, 1.5]\\
        3 & \text{if }\omega \in (1.5, 2].
    \end{cases}
\end{align*}
Then which one of the following statements is true?
\begin{enumerate}
    \item [(A)] $X$ is a random variable with respect to $\mathcal{G}$, but $Y$ is not a random variable with respect to $\mathcal{G}$.
    \item [(B)] $Y$ is a random variable with respect to $\mathcal{G}$, but $X$ is not a random variable with respect to $\mathcal{G}$.
    \item [(C)] Neither $X$ nor $Y$ is a random variable with respect to $\mathcal{G}$.
    \item [(D)] Both $X$ and $Y$ are random variables with respect to $\mathcal{G}$.
\end{enumerate} \hfill (GATE ST 2023)\\
\solution
%\input{gate/ST/2023/14/main.tex}
	\item  A die is loaded in such a way that each odd number is twice as likely to occur as
each even number. Find $P(G)$, where $G$ is the event that a number greater than
3 occurs on a single roll of the die.
\\
\solution
		%\input{exemplar/11/16/3/5/main.tex}
	\item All the jacks, queens and kings are removed from a deck of 52 playing cards. The remaining cards are well shuffled and then one card is drawn at random. Giving ace a value 1 similar value for other cards, find the probability that the card has a value 
		\begin{enumerate}
			\item 7
			\item greater than 7
			\item less than 7
		\end{enumerate}
		%\input{exemplar/10/13/3/30/main.tex}
  \item A Lot consists of 48 mobile phones of which 42 are good, 3 have only minor defects and 3 have major defects.Varnika will buy a phone if it is good but the trader will only buy a mobile if it has no major defects. One phone is selected at random from the lot. What is the probability that it is
\begin{enumerate}
	\item acceptable to Varnika?
            \item acceptable to the trader?
\end{enumerate}
\solution
	%\input{exemplar/10/13/3/40/main.tex}
 \item A student says that if you throw a die, it will show up 1 or not 1. Therefore, the probability of getting 1 and the probability of getting 'not 1' each is equal to $\frac{1}{2}$. Is this correct? Give reasons.\\
 \solution
        %\input{exemplar/10/13/2/9/main.tex}
   \item Four candidates A, B, C, D have ap-
plied for the assignment to coach a school cricket
team. If A is twice as likely to be selected as B, and
B and C are given about the same chance of being
selected, while C is twice as likely to be selected
as D, what are the probabilities that
\begin{enumerate}
\item C will be selected?
\item A will not be selected?
\end{enumerate}
	%\input{exemplar/11/16/3/9/main.tex}
 \item A bag contain 24 balls of which $x$ balls are red, $2x$ are white and $3x$ are blue. A ball is selected at random, What is the probability that it is
\begin{enumerate}[label=\alph*)]
\item not red ?
\item white ?
\end{enumerate}
%\input{exemplar/10/13/3/41/main.tex}
If the letters of the word ASSASSINATION are arranged at random. Find the Probability that
\begin{enumerate}[label=(\alph*)]
\item Four $S's$ come consecutively in the word
\item Two  $I's$ and two $N's$ come together
\item All $A's$ are not coming together
\item No two $A's$ are coming together
\end{enumerate}
%\input{exemplar/11/16/3/14/main.tex}
	\item One urn contains two black balls (labelled B1 and B2) and one white ball. A
	second urn contains one black ball and two white balls (labelled W1 and W2).
	Suppose the following experiment is performed. One of the two urns is chosen
	at random. Next a ball is randomly chosen from the urn. Then a second ball is
	chosen at random from the same urn without replacing the first ball.
	
	\begin{enumerate}
	\item What is the probability that two black balls are chosen?
	
	\item What is the probability that two balls of opposite colour are chosen?
	\end{enumerate}
	\solution
	%\input{exemplar/11/16/3/12/main1.tex}
\end{enumerate}

	\item A card is selected from a pack of 52 cards.
 \begin{enumerate}[label=(\alph*)] 
                 \item How many points are there in the sample space?
                 \item Calculate the probability that the card is an ace of spades.
                 \item Calculate the probability that the card is (i) an ace and (ii) black card.
 \end{enumerate}
\solution
		%\begin{table}[H]
	\centering
\begin{tabular}{|c|c|c|}
\hline
Random variable &Value &Definition\\ \hline
\multirow{3}{*}{X} &0 &Slips of Rs 1\\
&1 &Slips of Rs 5\\
&2 &Slips of Rs 13\\ \hline
\multirow{2}{*}{Y} &0 &Box A\\
&1 &Box B\\\hline
\end{tabular}
\caption{}
\label{tab:Distribution}
\end{table}
See \tabref{tab:Distribution}.
\begin{align}
p_{Y}\brak{k}= \begin{cases} 
      \frac{1}{3} & {k=0} \\
      \frac{2}{3 }& {k=1} 
   \end{cases}
   \\
p_{Y|X}\brak{0|0} = \frac{19}{25}\, 
p_{Y|X}\brak{0|1} = \frac{6}{25}\,
p_{Y|X}\brak{1|0} = \frac{45}{50}\,
p_{Y|X}\brak{1|2} = \frac{5}{50}
\end{align}
The desired probability is the probability that a slip drawn at random is marked other than Rs 1,
\begin{align}
&=1-p_X\brak{0}\\
&= p_X(1) + p_X(2)
\end{align}
Using Bayes theorem,
\begin{align}
&= p_Y\brak{0} \times \pr{Y=0 | X=1} + p_Y\brak{1} \times \pr{Y=1|X=2}\\
&=\frac{1}{3} \times \frac{6}{25} + \frac{2}{3} \times \frac{5}{50}\\
&=\frac{11}{75}
\end{align}

\newpage

%\tableofcontents

\bigskip

\renewcommand{\thefigure}{\theenumi}
\renewcommand{\thetable}{\theenumi}
%\renewcommand{\theequation}{\theenumi}

%\begin{abstract}
%%\boldmath
%In this letter, an algorithm for evaluating the exact analytical bit error rate  (BER)  for the piecewise linear (PL) combiner for  multiple relays is presented. Previous results were available only for upto three relays. The algorithm is unique in the sense that  the actual mathematical expressions, that are prohibitively large, need not be explicitly obtained. The diversity gain due to multiple relays is shown through plots of the analytical BER, well supported by simulations. 
%
%\end{abstract}
% IEEEtran.cls defaults to using nonbold math in the Abstract.
% This preserves the distinction between vectors and scalars. However,
% if the journal you are submitting to favors bold math in the abstract,
% then you can use LaTeX's standard command \boldmath at the very start
% of the abstract to achieve this. Many IEEE journals frown on math
% in the abstract anyway.

% Note that keywords are not normally used for peerreview papers.
%\begin{IEEEkeywords}
%Cooperative diversity, decode and forward, piecewise linear
%\end{IEEEkeywords}



% For peer review papers, you can put extra information on the cover
% page as needed:
% \ifCLASSOPTIONpeerreview
% \begin{center} \bfseries EDICS Category: 3-BBND \end{center}
% \fi
%
% For peerreview papers, this IEEEtran command inserts a page break and
% creates the second title. It will be ignored for other modes.
%\IEEEpeerreviewmaketitle




\item Four cards are drawn from a well-shuffled deck of 52 cards. What is the probability of obtaining 3 diamonds and one spade.
\\
\solution
		%\begin{enumerate}[label=\thesection.\arabic*,ref=\thesection.\theenumi]
	\item One card is drawn from a well-shuffled deck of 52 cards. Find the probability of getting
\begin{enumerate}
\item A king of red colour 
\item A face card 
\item A red face card
\item The jack of hearts
\item A spade
\item The queen of diamonds

\end{enumerate}
\solution
		%\input{ncert/10/15/1/14/main.tex}
	\item Five cards—the ten, jack, queen, king and ace of diamonds, are well-shuffled with their face downwards. One card is then picked up at random.
\begin{enumerate}
\item
What is the probability that the card is the queen? 
\item
If the queen is drawn and put aside, what is the probability that the second card picked up is (a) an ace? (b) a queen?\\
\end{enumerate}
\solution
		%\input{ncert/10/15/1/15/defs.tex}
	\item A bag contains $5$ red balls and some blue balls. If the probability of drawing a blue ball is double that if a red ball, determine the number of blue balls in the bag. 
		\\
\solution
		%\input{ncert/10/15/2/3/defs.tex}
	\item A card is selected from a pack of 52 cards.
 \begin{enumerate}[label=(\alph*)] 
                 \item How many points are there in the sample space?
                 \item Calculate the probability that the card is an ace of spades.
                 \item Calculate the probability that the card is (i) an ace and (ii) black card.
 \end{enumerate}
\solution
		%\input{ncert/11/16/3/4/main.tex}
\item Four cards are drawn from a well-shuffled deck of 52 cards. What is the probability of obtaining 3 diamonds and one spade.
\\
\solution
		%\input{ncert/11/16/4/2/defs.tex}
\item In a certain lottery 10,000 tickets are sold and ten equal prizes are awarded. What is the probability of not getting a prize if you buy (a) one ticket (b) two tickets (c) 10 tickets ?	
\\
\solution
		%\input{ncert/11/16/4/4/defs.tex}
		%
\item 
Out of 100 students, two sections of 40 and 60 are formed. If you and your friend are among the 100 students, what is the probability that
\begin{enumerate}
\item you both enter the same section?
\item you both enter the different sections?
\end{enumerate}
\solution
		%\input{ncert/11/16/4/5/defs.tex}
	\item 
The number lock of a suitcase has 4 wheels each labelled with ten digits i.e. from 0 to 9.The lock opens with a sequence of four digits with no repeats.What is the probability of a person getting the right sequence to open the suitcase.
\\
\solution
		%\input{ncert/11/16/4/10/defs.tex}
		%
\item 
Two cards are drawn at random and without replacement from a pack of 52 playing cards. Find the probability that both the cards are black.
\\
\solution
		%\input{ncert/12/13/2/2/defs.tex}
		\item A box of oranges is inspected by examining three randomly selected oranges drawn without replacement. If all the three oranges are good, the box is approved for sale, otherwise, it is rejected. Find the probability that a box containing 15 oranges out of which 12 are good and 3 are bad ones will be approved for sale.
		\label{ncert/12/13/2/3/defs.tex}
		\item Two balls are drawn at random with replacement from a box containing 10 black and 8 red balls. Find the probability that
		\label{ncert/12/13/2/12}
\begin{enumerate}
\item both balls are red.
\item first ball is black and second is red.
\item one of them is black and other is red.
\end{enumerate}

\item In a hostel, 60\% of the students read Hindi newspaper, 40\% read English newspaper and 20\% read both Hindi and English newspapers. A student is selected at random.
		\label{ncert/12/13/2/15}
\begin{enumerate}
\item Find the probability that she reads neither Hindi nor English newspapers.
\item If she reads Hindi newspaper, find the probability that she reads English newspaper.
\item If she reads English newspaper, find the probability that she reads Hindi newspaper.\\
\end{enumerate}
\item The probability of obtaining an even prime number on each die, when a pair of dice is rolled is 
\begin{enumerate}
    \item $0$ 
    
    \item $\frac{1}{3}$ 
    
    \item $\frac{1}{12}$ 
    
    \item $\frac{1}{36}$ 
\end{enumerate}
\solution
		%\input{ncert/12/13/2/17/defs.tex}
	\item A bag contains 4 red and 4 black balls, another bag contains 2 red and 6 black balls. One of the two bags is selected at random and a ball is drawn from the bag which is found to be red. Find the probability that the ball is drawn from the first bag.
\\
\solution
		%\input{ncert/12/13/3/2/main.tex}
  \item
  Cards with numbers 2 to 101 are placed in a box. A card is selected at random.Find the probability that the card has
\begin{enumerate}[label=(\roman*)]
	\item an even number 
	\item a square number
\end{enumerate}
\solution
%\input{exemplar/10/13/3/32/main.tex}
\item
The king, queen and jack of clubs are removed from a deck of 52 playing cards and then well shuffled. Now one card is drawn at random from the remaining cards.  Determine the probability that the card is
\begin{enumerate}[label=(\roman*)]
\item a club
\item 10 of hearts
\end{enumerate}
\solution
%\input{exemplar/10/13/3/29/main.tex}
\item A team of medical students doing their internship have to assist during surgeries
at a city hospital. The probabilities of surgeries rated as very complex, complex,
routine, simple or very simple are respectively, 0.15, 0.20, 0.31, 0.26, .08. Find
the probabilities that a particular surgery will be rated
\begin{enumerate}
	\item complex or very complex;
	\item neither very complex nor very simple;
	\item routine or complex
	\item routine or simple
\end{enumerate}
\solution
%\input{exemplar/11/16/3/8(1)/main.tex}
\item A card is selected from a pack of 52 cards.
\begin{enumerate}[label=(\alph*)]
    \item How many points are there in the sample space?
    \item Calculate the probability that the card is an ace of spades.
    \item Calculate the probability that the card is (i) an ace and (ii) black card.
\end{enumerate}
\solution
%\input{exemplar/11/16/3/4/main2.tex}
\item The probability that a non leap year selected at random will contain 53 sundays.
\\
\solution
%\input{exemplar/10/13/1/19/main.tex}
\item One of the four persons John, Rita, Aslam or Gurpreet will be promoted next
month. Consequently the sample space consists of four elementary outcomes
S = {John promoted, Rita promoted, Aslam promoted, Gurpreet promoted}
You are told that the chances of John’s promotion is same as that of Gurpreet,
Rita’s chances of promotion are twice as likely as Johns. Aslam’s chances are
four times that of John.
\begin{enumerate}
	\item Determine
	\begin{enumerate}
		\item P (John promoted)
		\item P (Rita promoted)
		\item P (Aslam promoted)
		\item P (Gurpreet promoted)
	\end{enumerate}
	\item If A = {John promoted or Gurpreet promoted}, find P (A).
\end{enumerate}
\solution
%\input{exemplar/11/16/3/10/main.tex}
\item A card is drawn from a deck of 52 cards. Find the probability of getting a king or a heart or a red card.\\
\solution
%\input{exemplar/11/16/3/15/main.tex}
\item The probability that a student will pass his examination is 0.73, the probability of
the student getting a compartment is 0.13, and the probability that the student will
either pass or get compartment is 0.96. State True or False.\\
\solution
%\input{exemplar/11/16/3/31/main.tex}
\item A card is selected from a pack of 52 cards\\
\begin{enumerate}[label=(\alph*)]
\item How many points are there in the sample space?
\item Calculate the probability that the cards is an ace of spades.
\item Calculate the probability that the card is (i) an ace (ii)black card.\\
\end{enumerate}
%\input{ncert/11/16/3/4_1/Prob_4.tex}
\item In a non-leap year, the probability of having 53 tuesdays or 53 wednesdays is\\
\solution
%\input{exemplar/11/16/3/18/main.tex}
\item There are 1000 sealed envelopes in a box, 10 of them contain a cash prize of
Rs 100 each, 100 of them contain a cash prize of Rs 50 each and 200 of them
contain a cash prize of Rs 10 each and rest do not contain any cash prize. If they
are well shuffled and an envelope is picked up out, what is the probability that it
contains no cash prize?\\
\solution
%\input{exemplar/10/13/3/34/main.tex}
\item 
A die is thrown and a card is selected at random from a deck of 52 playing cards. The probability of getting an even number on the die and a spade card.\\
\solution
%\input{exemplar/12/13/3/78/main.tex}
\item
If 4-digit numbers greater than 5,000 are randomly formed from the digits 0, 1, 3, 5, and 7, what is the probability of forming a number divisible by 5 when:
\begin{enumerate}
    \item The digits are repeated?
    \item The repetition of digits is not allowed?
\end{enumerate}
\solution
%\input{ncert/11/16/4/9/main.tex}
\item Consider the probability space $\brak{\Omega, \mathcal{G}, P}$ where $\Omega = [0,2]$ and $\mathcal{G} = \cbrak{\phi, \Omega, [0,1], (1,2]}$. Let $X$ and $Y$ be two functions on $\Omega$ defined as
\begin{align*}
    X(\omega) = 
    \begin{cases}
        1 & \text{if }\omega \in [0, 1]\\
        2 & \text{if }\omega \in (1, 2]
    \end{cases}
\end{align*}
and
\begin{align*}
    Y(\omega) = 
    \begin{cases}
        2 & \text{if }\omega \in [0, 1.5]\\
        3 & \text{if }\omega \in (1.5, 2].
    \end{cases}
\end{align*}
Then which one of the following statements is true?
\begin{enumerate}
    \item [(A)] $X$ is a random variable with respect to $\mathcal{G}$, but $Y$ is not a random variable with respect to $\mathcal{G}$.
    \item [(B)] $Y$ is a random variable with respect to $\mathcal{G}$, but $X$ is not a random variable with respect to $\mathcal{G}$.
    \item [(C)] Neither $X$ nor $Y$ is a random variable with respect to $\mathcal{G}$.
    \item [(D)] Both $X$ and $Y$ are random variables with respect to $\mathcal{G}$.
\end{enumerate} \hfill (GATE ST 2023)\\
\solution
%\input{gate/ST/2023/14/main.tex}
	\item  A die is loaded in such a way that each odd number is twice as likely to occur as
each even number. Find $P(G)$, where $G$ is the event that a number greater than
3 occurs on a single roll of the die.
\\
\solution
		%\input{exemplar/11/16/3/5/main.tex}
	\item All the jacks, queens and kings are removed from a deck of 52 playing cards. The remaining cards are well shuffled and then one card is drawn at random. Giving ace a value 1 similar value for other cards, find the probability that the card has a value 
		\begin{enumerate}
			\item 7
			\item greater than 7
			\item less than 7
		\end{enumerate}
		%\input{exemplar/10/13/3/30/main.tex}
  \item A Lot consists of 48 mobile phones of which 42 are good, 3 have only minor defects and 3 have major defects.Varnika will buy a phone if it is good but the trader will only buy a mobile if it has no major defects. One phone is selected at random from the lot. What is the probability that it is
\begin{enumerate}
	\item acceptable to Varnika?
            \item acceptable to the trader?
\end{enumerate}
\solution
	%\input{exemplar/10/13/3/40/main.tex}
 \item A student says that if you throw a die, it will show up 1 or not 1. Therefore, the probability of getting 1 and the probability of getting 'not 1' each is equal to $\frac{1}{2}$. Is this correct? Give reasons.\\
 \solution
        %\input{exemplar/10/13/2/9/main.tex}
   \item Four candidates A, B, C, D have ap-
plied for the assignment to coach a school cricket
team. If A is twice as likely to be selected as B, and
B and C are given about the same chance of being
selected, while C is twice as likely to be selected
as D, what are the probabilities that
\begin{enumerate}
\item C will be selected?
\item A will not be selected?
\end{enumerate}
	%\input{exemplar/11/16/3/9/main.tex}
 \item A bag contain 24 balls of which $x$ balls are red, $2x$ are white and $3x$ are blue. A ball is selected at random, What is the probability that it is
\begin{enumerate}[label=\alph*)]
\item not red ?
\item white ?
\end{enumerate}
%\input{exemplar/10/13/3/41/main.tex}
If the letters of the word ASSASSINATION are arranged at random. Find the Probability that
\begin{enumerate}[label=(\alph*)]
\item Four $S's$ come consecutively in the word
\item Two  $I's$ and two $N's$ come together
\item All $A's$ are not coming together
\item No two $A's$ are coming together
\end{enumerate}
%\input{exemplar/11/16/3/14/main.tex}
	\item One urn contains two black balls (labelled B1 and B2) and one white ball. A
	second urn contains one black ball and two white balls (labelled W1 and W2).
	Suppose the following experiment is performed. One of the two urns is chosen
	at random. Next a ball is randomly chosen from the urn. Then a second ball is
	chosen at random from the same urn without replacing the first ball.
	
	\begin{enumerate}
	\item What is the probability that two black balls are chosen?
	
	\item What is the probability that two balls of opposite colour are chosen?
	\end{enumerate}
	\solution
	%\input{exemplar/11/16/3/12/main1.tex}
\end{enumerate}

\item In a certain lottery 10,000 tickets are sold and ten equal prizes are awarded. What is the probability of not getting a prize if you buy (a) one ticket (b) two tickets (c) 10 tickets ?	
\\
\solution
		%\begin{enumerate}[label=\thesection.\arabic*,ref=\thesection.\theenumi]
	\item One card is drawn from a well-shuffled deck of 52 cards. Find the probability of getting
\begin{enumerate}
\item A king of red colour 
\item A face card 
\item A red face card
\item The jack of hearts
\item A spade
\item The queen of diamonds

\end{enumerate}
\solution
		%\input{ncert/10/15/1/14/main.tex}
	\item Five cards—the ten, jack, queen, king and ace of diamonds, are well-shuffled with their face downwards. One card is then picked up at random.
\begin{enumerate}
\item
What is the probability that the card is the queen? 
\item
If the queen is drawn and put aside, what is the probability that the second card picked up is (a) an ace? (b) a queen?\\
\end{enumerate}
\solution
		%\input{ncert/10/15/1/15/defs.tex}
	\item A bag contains $5$ red balls and some blue balls. If the probability of drawing a blue ball is double that if a red ball, determine the number of blue balls in the bag. 
		\\
\solution
		%\input{ncert/10/15/2/3/defs.tex}
	\item A card is selected from a pack of 52 cards.
 \begin{enumerate}[label=(\alph*)] 
                 \item How many points are there in the sample space?
                 \item Calculate the probability that the card is an ace of spades.
                 \item Calculate the probability that the card is (i) an ace and (ii) black card.
 \end{enumerate}
\solution
		%\input{ncert/11/16/3/4/main.tex}
\item Four cards are drawn from a well-shuffled deck of 52 cards. What is the probability of obtaining 3 diamonds and one spade.
\\
\solution
		%\input{ncert/11/16/4/2/defs.tex}
\item In a certain lottery 10,000 tickets are sold and ten equal prizes are awarded. What is the probability of not getting a prize if you buy (a) one ticket (b) two tickets (c) 10 tickets ?	
\\
\solution
		%\input{ncert/11/16/4/4/defs.tex}
		%
\item 
Out of 100 students, two sections of 40 and 60 are formed. If you and your friend are among the 100 students, what is the probability that
\begin{enumerate}
\item you both enter the same section?
\item you both enter the different sections?
\end{enumerate}
\solution
		%\input{ncert/11/16/4/5/defs.tex}
	\item 
The number lock of a suitcase has 4 wheels each labelled with ten digits i.e. from 0 to 9.The lock opens with a sequence of four digits with no repeats.What is the probability of a person getting the right sequence to open the suitcase.
\\
\solution
		%\input{ncert/11/16/4/10/defs.tex}
		%
\item 
Two cards are drawn at random and without replacement from a pack of 52 playing cards. Find the probability that both the cards are black.
\\
\solution
		%\input{ncert/12/13/2/2/defs.tex}
		\item A box of oranges is inspected by examining three randomly selected oranges drawn without replacement. If all the three oranges are good, the box is approved for sale, otherwise, it is rejected. Find the probability that a box containing 15 oranges out of which 12 are good and 3 are bad ones will be approved for sale.
		\label{ncert/12/13/2/3/defs.tex}
		\item Two balls are drawn at random with replacement from a box containing 10 black and 8 red balls. Find the probability that
		\label{ncert/12/13/2/12}
\begin{enumerate}
\item both balls are red.
\item first ball is black and second is red.
\item one of them is black and other is red.
\end{enumerate}

\item In a hostel, 60\% of the students read Hindi newspaper, 40\% read English newspaper and 20\% read both Hindi and English newspapers. A student is selected at random.
		\label{ncert/12/13/2/15}
\begin{enumerate}
\item Find the probability that she reads neither Hindi nor English newspapers.
\item If she reads Hindi newspaper, find the probability that she reads English newspaper.
\item If she reads English newspaper, find the probability that she reads Hindi newspaper.\\
\end{enumerate}
\item The probability of obtaining an even prime number on each die, when a pair of dice is rolled is 
\begin{enumerate}
    \item $0$ 
    
    \item $\frac{1}{3}$ 
    
    \item $\frac{1}{12}$ 
    
    \item $\frac{1}{36}$ 
\end{enumerate}
\solution
		%\input{ncert/12/13/2/17/defs.tex}
	\item A bag contains 4 red and 4 black balls, another bag contains 2 red and 6 black balls. One of the two bags is selected at random and a ball is drawn from the bag which is found to be red. Find the probability that the ball is drawn from the first bag.
\\
\solution
		%\input{ncert/12/13/3/2/main.tex}
  \item
  Cards with numbers 2 to 101 are placed in a box. A card is selected at random.Find the probability that the card has
\begin{enumerate}[label=(\roman*)]
	\item an even number 
	\item a square number
\end{enumerate}
\solution
%\input{exemplar/10/13/3/32/main.tex}
\item
The king, queen and jack of clubs are removed from a deck of 52 playing cards and then well shuffled. Now one card is drawn at random from the remaining cards.  Determine the probability that the card is
\begin{enumerate}[label=(\roman*)]
\item a club
\item 10 of hearts
\end{enumerate}
\solution
%\input{exemplar/10/13/3/29/main.tex}
\item A team of medical students doing their internship have to assist during surgeries
at a city hospital. The probabilities of surgeries rated as very complex, complex,
routine, simple or very simple are respectively, 0.15, 0.20, 0.31, 0.26, .08. Find
the probabilities that a particular surgery will be rated
\begin{enumerate}
	\item complex or very complex;
	\item neither very complex nor very simple;
	\item routine or complex
	\item routine or simple
\end{enumerate}
\solution
%\input{exemplar/11/16/3/8(1)/main.tex}
\item A card is selected from a pack of 52 cards.
\begin{enumerate}[label=(\alph*)]
    \item How many points are there in the sample space?
    \item Calculate the probability that the card is an ace of spades.
    \item Calculate the probability that the card is (i) an ace and (ii) black card.
\end{enumerate}
\solution
%\input{exemplar/11/16/3/4/main2.tex}
\item The probability that a non leap year selected at random will contain 53 sundays.
\\
\solution
%\input{exemplar/10/13/1/19/main.tex}
\item One of the four persons John, Rita, Aslam or Gurpreet will be promoted next
month. Consequently the sample space consists of four elementary outcomes
S = {John promoted, Rita promoted, Aslam promoted, Gurpreet promoted}
You are told that the chances of John’s promotion is same as that of Gurpreet,
Rita’s chances of promotion are twice as likely as Johns. Aslam’s chances are
four times that of John.
\begin{enumerate}
	\item Determine
	\begin{enumerate}
		\item P (John promoted)
		\item P (Rita promoted)
		\item P (Aslam promoted)
		\item P (Gurpreet promoted)
	\end{enumerate}
	\item If A = {John promoted or Gurpreet promoted}, find P (A).
\end{enumerate}
\solution
%\input{exemplar/11/16/3/10/main.tex}
\item A card is drawn from a deck of 52 cards. Find the probability of getting a king or a heart or a red card.\\
\solution
%\input{exemplar/11/16/3/15/main.tex}
\item The probability that a student will pass his examination is 0.73, the probability of
the student getting a compartment is 0.13, and the probability that the student will
either pass or get compartment is 0.96. State True or False.\\
\solution
%\input{exemplar/11/16/3/31/main.tex}
\item A card is selected from a pack of 52 cards\\
\begin{enumerate}[label=(\alph*)]
\item How many points are there in the sample space?
\item Calculate the probability that the cards is an ace of spades.
\item Calculate the probability that the card is (i) an ace (ii)black card.\\
\end{enumerate}
%\input{ncert/11/16/3/4_1/Prob_4.tex}
\item In a non-leap year, the probability of having 53 tuesdays or 53 wednesdays is\\
\solution
%\input{exemplar/11/16/3/18/main.tex}
\item There are 1000 sealed envelopes in a box, 10 of them contain a cash prize of
Rs 100 each, 100 of them contain a cash prize of Rs 50 each and 200 of them
contain a cash prize of Rs 10 each and rest do not contain any cash prize. If they
are well shuffled and an envelope is picked up out, what is the probability that it
contains no cash prize?\\
\solution
%\input{exemplar/10/13/3/34/main.tex}
\item 
A die is thrown and a card is selected at random from a deck of 52 playing cards. The probability of getting an even number on the die and a spade card.\\
\solution
%\input{exemplar/12/13/3/78/main.tex}
\item
If 4-digit numbers greater than 5,000 are randomly formed from the digits 0, 1, 3, 5, and 7, what is the probability of forming a number divisible by 5 when:
\begin{enumerate}
    \item The digits are repeated?
    \item The repetition of digits is not allowed?
\end{enumerate}
\solution
%\input{ncert/11/16/4/9/main.tex}
\item Consider the probability space $\brak{\Omega, \mathcal{G}, P}$ where $\Omega = [0,2]$ and $\mathcal{G} = \cbrak{\phi, \Omega, [0,1], (1,2]}$. Let $X$ and $Y$ be two functions on $\Omega$ defined as
\begin{align*}
    X(\omega) = 
    \begin{cases}
        1 & \text{if }\omega \in [0, 1]\\
        2 & \text{if }\omega \in (1, 2]
    \end{cases}
\end{align*}
and
\begin{align*}
    Y(\omega) = 
    \begin{cases}
        2 & \text{if }\omega \in [0, 1.5]\\
        3 & \text{if }\omega \in (1.5, 2].
    \end{cases}
\end{align*}
Then which one of the following statements is true?
\begin{enumerate}
    \item [(A)] $X$ is a random variable with respect to $\mathcal{G}$, but $Y$ is not a random variable with respect to $\mathcal{G}$.
    \item [(B)] $Y$ is a random variable with respect to $\mathcal{G}$, but $X$ is not a random variable with respect to $\mathcal{G}$.
    \item [(C)] Neither $X$ nor $Y$ is a random variable with respect to $\mathcal{G}$.
    \item [(D)] Both $X$ and $Y$ are random variables with respect to $\mathcal{G}$.
\end{enumerate} \hfill (GATE ST 2023)\\
\solution
%\input{gate/ST/2023/14/main.tex}
	\item  A die is loaded in such a way that each odd number is twice as likely to occur as
each even number. Find $P(G)$, where $G$ is the event that a number greater than
3 occurs on a single roll of the die.
\\
\solution
		%\input{exemplar/11/16/3/5/main.tex}
	\item All the jacks, queens and kings are removed from a deck of 52 playing cards. The remaining cards are well shuffled and then one card is drawn at random. Giving ace a value 1 similar value for other cards, find the probability that the card has a value 
		\begin{enumerate}
			\item 7
			\item greater than 7
			\item less than 7
		\end{enumerate}
		%\input{exemplar/10/13/3/30/main.tex}
  \item A Lot consists of 48 mobile phones of which 42 are good, 3 have only minor defects and 3 have major defects.Varnika will buy a phone if it is good but the trader will only buy a mobile if it has no major defects. One phone is selected at random from the lot. What is the probability that it is
\begin{enumerate}
	\item acceptable to Varnika?
            \item acceptable to the trader?
\end{enumerate}
\solution
	%\input{exemplar/10/13/3/40/main.tex}
 \item A student says that if you throw a die, it will show up 1 or not 1. Therefore, the probability of getting 1 and the probability of getting 'not 1' each is equal to $\frac{1}{2}$. Is this correct? Give reasons.\\
 \solution
        %\input{exemplar/10/13/2/9/main.tex}
   \item Four candidates A, B, C, D have ap-
plied for the assignment to coach a school cricket
team. If A is twice as likely to be selected as B, and
B and C are given about the same chance of being
selected, while C is twice as likely to be selected
as D, what are the probabilities that
\begin{enumerate}
\item C will be selected?
\item A will not be selected?
\end{enumerate}
	%\input{exemplar/11/16/3/9/main.tex}
 \item A bag contain 24 balls of which $x$ balls are red, $2x$ are white and $3x$ are blue. A ball is selected at random, What is the probability that it is
\begin{enumerate}[label=\alph*)]
\item not red ?
\item white ?
\end{enumerate}
%\input{exemplar/10/13/3/41/main.tex}
If the letters of the word ASSASSINATION are arranged at random. Find the Probability that
\begin{enumerate}[label=(\alph*)]
\item Four $S's$ come consecutively in the word
\item Two  $I's$ and two $N's$ come together
\item All $A's$ are not coming together
\item No two $A's$ are coming together
\end{enumerate}
%\input{exemplar/11/16/3/14/main.tex}
	\item One urn contains two black balls (labelled B1 and B2) and one white ball. A
	second urn contains one black ball and two white balls (labelled W1 and W2).
	Suppose the following experiment is performed. One of the two urns is chosen
	at random. Next a ball is randomly chosen from the urn. Then a second ball is
	chosen at random from the same urn without replacing the first ball.
	
	\begin{enumerate}
	\item What is the probability that two black balls are chosen?
	
	\item What is the probability that two balls of opposite colour are chosen?
	\end{enumerate}
	\solution
	%\input{exemplar/11/16/3/12/main1.tex}
\end{enumerate}

		%
\item 
Out of 100 students, two sections of 40 and 60 are formed. If you and your friend are among the 100 students, what is the probability that
\begin{enumerate}
\item you both enter the same section?
\item you both enter the different sections?
\end{enumerate}
\solution
		%\begin{enumerate}[label=\thesection.\arabic*,ref=\thesection.\theenumi]
	\item One card is drawn from a well-shuffled deck of 52 cards. Find the probability of getting
\begin{enumerate}
\item A king of red colour 
\item A face card 
\item A red face card
\item The jack of hearts
\item A spade
\item The queen of diamonds

\end{enumerate}
\solution
		%\input{ncert/10/15/1/14/main.tex}
	\item Five cards—the ten, jack, queen, king and ace of diamonds, are well-shuffled with their face downwards. One card is then picked up at random.
\begin{enumerate}
\item
What is the probability that the card is the queen? 
\item
If the queen is drawn and put aside, what is the probability that the second card picked up is (a) an ace? (b) a queen?\\
\end{enumerate}
\solution
		%\input{ncert/10/15/1/15/defs.tex}
	\item A bag contains $5$ red balls and some blue balls. If the probability of drawing a blue ball is double that if a red ball, determine the number of blue balls in the bag. 
		\\
\solution
		%\input{ncert/10/15/2/3/defs.tex}
	\item A card is selected from a pack of 52 cards.
 \begin{enumerate}[label=(\alph*)] 
                 \item How many points are there in the sample space?
                 \item Calculate the probability that the card is an ace of spades.
                 \item Calculate the probability that the card is (i) an ace and (ii) black card.
 \end{enumerate}
\solution
		%\input{ncert/11/16/3/4/main.tex}
\item Four cards are drawn from a well-shuffled deck of 52 cards. What is the probability of obtaining 3 diamonds and one spade.
\\
\solution
		%\input{ncert/11/16/4/2/defs.tex}
\item In a certain lottery 10,000 tickets are sold and ten equal prizes are awarded. What is the probability of not getting a prize if you buy (a) one ticket (b) two tickets (c) 10 tickets ?	
\\
\solution
		%\input{ncert/11/16/4/4/defs.tex}
		%
\item 
Out of 100 students, two sections of 40 and 60 are formed. If you and your friend are among the 100 students, what is the probability that
\begin{enumerate}
\item you both enter the same section?
\item you both enter the different sections?
\end{enumerate}
\solution
		%\input{ncert/11/16/4/5/defs.tex}
	\item 
The number lock of a suitcase has 4 wheels each labelled with ten digits i.e. from 0 to 9.The lock opens with a sequence of four digits with no repeats.What is the probability of a person getting the right sequence to open the suitcase.
\\
\solution
		%\input{ncert/11/16/4/10/defs.tex}
		%
\item 
Two cards are drawn at random and without replacement from a pack of 52 playing cards. Find the probability that both the cards are black.
\\
\solution
		%\input{ncert/12/13/2/2/defs.tex}
		\item A box of oranges is inspected by examining three randomly selected oranges drawn without replacement. If all the three oranges are good, the box is approved for sale, otherwise, it is rejected. Find the probability that a box containing 15 oranges out of which 12 are good and 3 are bad ones will be approved for sale.
		\label{ncert/12/13/2/3/defs.tex}
		\item Two balls are drawn at random with replacement from a box containing 10 black and 8 red balls. Find the probability that
		\label{ncert/12/13/2/12}
\begin{enumerate}
\item both balls are red.
\item first ball is black and second is red.
\item one of them is black and other is red.
\end{enumerate}

\item In a hostel, 60\% of the students read Hindi newspaper, 40\% read English newspaper and 20\% read both Hindi and English newspapers. A student is selected at random.
		\label{ncert/12/13/2/15}
\begin{enumerate}
\item Find the probability that she reads neither Hindi nor English newspapers.
\item If she reads Hindi newspaper, find the probability that she reads English newspaper.
\item If she reads English newspaper, find the probability that she reads Hindi newspaper.\\
\end{enumerate}
\item The probability of obtaining an even prime number on each die, when a pair of dice is rolled is 
\begin{enumerate}
    \item $0$ 
    
    \item $\frac{1}{3}$ 
    
    \item $\frac{1}{12}$ 
    
    \item $\frac{1}{36}$ 
\end{enumerate}
\solution
		%\input{ncert/12/13/2/17/defs.tex}
	\item A bag contains 4 red and 4 black balls, another bag contains 2 red and 6 black balls. One of the two bags is selected at random and a ball is drawn from the bag which is found to be red. Find the probability that the ball is drawn from the first bag.
\\
\solution
		%\input{ncert/12/13/3/2/main.tex}
  \item
  Cards with numbers 2 to 101 are placed in a box. A card is selected at random.Find the probability that the card has
\begin{enumerate}[label=(\roman*)]
	\item an even number 
	\item a square number
\end{enumerate}
\solution
%\input{exemplar/10/13/3/32/main.tex}
\item
The king, queen and jack of clubs are removed from a deck of 52 playing cards and then well shuffled. Now one card is drawn at random from the remaining cards.  Determine the probability that the card is
\begin{enumerate}[label=(\roman*)]
\item a club
\item 10 of hearts
\end{enumerate}
\solution
%\input{exemplar/10/13/3/29/main.tex}
\item A team of medical students doing their internship have to assist during surgeries
at a city hospital. The probabilities of surgeries rated as very complex, complex,
routine, simple or very simple are respectively, 0.15, 0.20, 0.31, 0.26, .08. Find
the probabilities that a particular surgery will be rated
\begin{enumerate}
	\item complex or very complex;
	\item neither very complex nor very simple;
	\item routine or complex
	\item routine or simple
\end{enumerate}
\solution
%\input{exemplar/11/16/3/8(1)/main.tex}
\item A card is selected from a pack of 52 cards.
\begin{enumerate}[label=(\alph*)]
    \item How many points are there in the sample space?
    \item Calculate the probability that the card is an ace of spades.
    \item Calculate the probability that the card is (i) an ace and (ii) black card.
\end{enumerate}
\solution
%\input{exemplar/11/16/3/4/main2.tex}
\item The probability that a non leap year selected at random will contain 53 sundays.
\\
\solution
%\input{exemplar/10/13/1/19/main.tex}
\item One of the four persons John, Rita, Aslam or Gurpreet will be promoted next
month. Consequently the sample space consists of four elementary outcomes
S = {John promoted, Rita promoted, Aslam promoted, Gurpreet promoted}
You are told that the chances of John’s promotion is same as that of Gurpreet,
Rita’s chances of promotion are twice as likely as Johns. Aslam’s chances are
four times that of John.
\begin{enumerate}
	\item Determine
	\begin{enumerate}
		\item P (John promoted)
		\item P (Rita promoted)
		\item P (Aslam promoted)
		\item P (Gurpreet promoted)
	\end{enumerate}
	\item If A = {John promoted or Gurpreet promoted}, find P (A).
\end{enumerate}
\solution
%\input{exemplar/11/16/3/10/main.tex}
\item A card is drawn from a deck of 52 cards. Find the probability of getting a king or a heart or a red card.\\
\solution
%\input{exemplar/11/16/3/15/main.tex}
\item The probability that a student will pass his examination is 0.73, the probability of
the student getting a compartment is 0.13, and the probability that the student will
either pass or get compartment is 0.96. State True or False.\\
\solution
%\input{exemplar/11/16/3/31/main.tex}
\item A card is selected from a pack of 52 cards\\
\begin{enumerate}[label=(\alph*)]
\item How many points are there in the sample space?
\item Calculate the probability that the cards is an ace of spades.
\item Calculate the probability that the card is (i) an ace (ii)black card.\\
\end{enumerate}
%\input{ncert/11/16/3/4_1/Prob_4.tex}
\item In a non-leap year, the probability of having 53 tuesdays or 53 wednesdays is\\
\solution
%\input{exemplar/11/16/3/18/main.tex}
\item There are 1000 sealed envelopes in a box, 10 of them contain a cash prize of
Rs 100 each, 100 of them contain a cash prize of Rs 50 each and 200 of them
contain a cash prize of Rs 10 each and rest do not contain any cash prize. If they
are well shuffled and an envelope is picked up out, what is the probability that it
contains no cash prize?\\
\solution
%\input{exemplar/10/13/3/34/main.tex}
\item 
A die is thrown and a card is selected at random from a deck of 52 playing cards. The probability of getting an even number on the die and a spade card.\\
\solution
%\input{exemplar/12/13/3/78/main.tex}
\item
If 4-digit numbers greater than 5,000 are randomly formed from the digits 0, 1, 3, 5, and 7, what is the probability of forming a number divisible by 5 when:
\begin{enumerate}
    \item The digits are repeated?
    \item The repetition of digits is not allowed?
\end{enumerate}
\solution
%\input{ncert/11/16/4/9/main.tex}
\item Consider the probability space $\brak{\Omega, \mathcal{G}, P}$ where $\Omega = [0,2]$ and $\mathcal{G} = \cbrak{\phi, \Omega, [0,1], (1,2]}$. Let $X$ and $Y$ be two functions on $\Omega$ defined as
\begin{align*}
    X(\omega) = 
    \begin{cases}
        1 & \text{if }\omega \in [0, 1]\\
        2 & \text{if }\omega \in (1, 2]
    \end{cases}
\end{align*}
and
\begin{align*}
    Y(\omega) = 
    \begin{cases}
        2 & \text{if }\omega \in [0, 1.5]\\
        3 & \text{if }\omega \in (1.5, 2].
    \end{cases}
\end{align*}
Then which one of the following statements is true?
\begin{enumerate}
    \item [(A)] $X$ is a random variable with respect to $\mathcal{G}$, but $Y$ is not a random variable with respect to $\mathcal{G}$.
    \item [(B)] $Y$ is a random variable with respect to $\mathcal{G}$, but $X$ is not a random variable with respect to $\mathcal{G}$.
    \item [(C)] Neither $X$ nor $Y$ is a random variable with respect to $\mathcal{G}$.
    \item [(D)] Both $X$ and $Y$ are random variables with respect to $\mathcal{G}$.
\end{enumerate} \hfill (GATE ST 2023)\\
\solution
%\input{gate/ST/2023/14/main.tex}
	\item  A die is loaded in such a way that each odd number is twice as likely to occur as
each even number. Find $P(G)$, where $G$ is the event that a number greater than
3 occurs on a single roll of the die.
\\
\solution
		%\input{exemplar/11/16/3/5/main.tex}
	\item All the jacks, queens and kings are removed from a deck of 52 playing cards. The remaining cards are well shuffled and then one card is drawn at random. Giving ace a value 1 similar value for other cards, find the probability that the card has a value 
		\begin{enumerate}
			\item 7
			\item greater than 7
			\item less than 7
		\end{enumerate}
		%\input{exemplar/10/13/3/30/main.tex}
  \item A Lot consists of 48 mobile phones of which 42 are good, 3 have only minor defects and 3 have major defects.Varnika will buy a phone if it is good but the trader will only buy a mobile if it has no major defects. One phone is selected at random from the lot. What is the probability that it is
\begin{enumerate}
	\item acceptable to Varnika?
            \item acceptable to the trader?
\end{enumerate}
\solution
	%\input{exemplar/10/13/3/40/main.tex}
 \item A student says that if you throw a die, it will show up 1 or not 1. Therefore, the probability of getting 1 and the probability of getting 'not 1' each is equal to $\frac{1}{2}$. Is this correct? Give reasons.\\
 \solution
        %\input{exemplar/10/13/2/9/main.tex}
   \item Four candidates A, B, C, D have ap-
plied for the assignment to coach a school cricket
team. If A is twice as likely to be selected as B, and
B and C are given about the same chance of being
selected, while C is twice as likely to be selected
as D, what are the probabilities that
\begin{enumerate}
\item C will be selected?
\item A will not be selected?
\end{enumerate}
	%\input{exemplar/11/16/3/9/main.tex}
 \item A bag contain 24 balls of which $x$ balls are red, $2x$ are white and $3x$ are blue. A ball is selected at random, What is the probability that it is
\begin{enumerate}[label=\alph*)]
\item not red ?
\item white ?
\end{enumerate}
%\input{exemplar/10/13/3/41/main.tex}
If the letters of the word ASSASSINATION are arranged at random. Find the Probability that
\begin{enumerate}[label=(\alph*)]
\item Four $S's$ come consecutively in the word
\item Two  $I's$ and two $N's$ come together
\item All $A's$ are not coming together
\item No two $A's$ are coming together
\end{enumerate}
%\input{exemplar/11/16/3/14/main.tex}
	\item One urn contains two black balls (labelled B1 and B2) and one white ball. A
	second urn contains one black ball and two white balls (labelled W1 and W2).
	Suppose the following experiment is performed. One of the two urns is chosen
	at random. Next a ball is randomly chosen from the urn. Then a second ball is
	chosen at random from the same urn without replacing the first ball.
	
	\begin{enumerate}
	\item What is the probability that two black balls are chosen?
	
	\item What is the probability that two balls of opposite colour are chosen?
	\end{enumerate}
	\solution
	%\input{exemplar/11/16/3/12/main1.tex}
\end{enumerate}

	\item 
The number lock of a suitcase has 4 wheels each labelled with ten digits i.e. from 0 to 9.The lock opens with a sequence of four digits with no repeats.What is the probability of a person getting the right sequence to open the suitcase.
\\
\solution
		%\begin{enumerate}[label=\thesection.\arabic*,ref=\thesection.\theenumi]
	\item One card is drawn from a well-shuffled deck of 52 cards. Find the probability of getting
\begin{enumerate}
\item A king of red colour 
\item A face card 
\item A red face card
\item The jack of hearts
\item A spade
\item The queen of diamonds

\end{enumerate}
\solution
		%\input{ncert/10/15/1/14/main.tex}
	\item Five cards—the ten, jack, queen, king and ace of diamonds, are well-shuffled with their face downwards. One card is then picked up at random.
\begin{enumerate}
\item
What is the probability that the card is the queen? 
\item
If the queen is drawn and put aside, what is the probability that the second card picked up is (a) an ace? (b) a queen?\\
\end{enumerate}
\solution
		%\input{ncert/10/15/1/15/defs.tex}
	\item A bag contains $5$ red balls and some blue balls. If the probability of drawing a blue ball is double that if a red ball, determine the number of blue balls in the bag. 
		\\
\solution
		%\input{ncert/10/15/2/3/defs.tex}
	\item A card is selected from a pack of 52 cards.
 \begin{enumerate}[label=(\alph*)] 
                 \item How many points are there in the sample space?
                 \item Calculate the probability that the card is an ace of spades.
                 \item Calculate the probability that the card is (i) an ace and (ii) black card.
 \end{enumerate}
\solution
		%\input{ncert/11/16/3/4/main.tex}
\item Four cards are drawn from a well-shuffled deck of 52 cards. What is the probability of obtaining 3 diamonds and one spade.
\\
\solution
		%\input{ncert/11/16/4/2/defs.tex}
\item In a certain lottery 10,000 tickets are sold and ten equal prizes are awarded. What is the probability of not getting a prize if you buy (a) one ticket (b) two tickets (c) 10 tickets ?	
\\
\solution
		%\input{ncert/11/16/4/4/defs.tex}
		%
\item 
Out of 100 students, two sections of 40 and 60 are formed. If you and your friend are among the 100 students, what is the probability that
\begin{enumerate}
\item you both enter the same section?
\item you both enter the different sections?
\end{enumerate}
\solution
		%\input{ncert/11/16/4/5/defs.tex}
	\item 
The number lock of a suitcase has 4 wheels each labelled with ten digits i.e. from 0 to 9.The lock opens with a sequence of four digits with no repeats.What is the probability of a person getting the right sequence to open the suitcase.
\\
\solution
		%\input{ncert/11/16/4/10/defs.tex}
		%
\item 
Two cards are drawn at random and without replacement from a pack of 52 playing cards. Find the probability that both the cards are black.
\\
\solution
		%\input{ncert/12/13/2/2/defs.tex}
		\item A box of oranges is inspected by examining three randomly selected oranges drawn without replacement. If all the three oranges are good, the box is approved for sale, otherwise, it is rejected. Find the probability that a box containing 15 oranges out of which 12 are good and 3 are bad ones will be approved for sale.
		\label{ncert/12/13/2/3/defs.tex}
		\item Two balls are drawn at random with replacement from a box containing 10 black and 8 red balls. Find the probability that
		\label{ncert/12/13/2/12}
\begin{enumerate}
\item both balls are red.
\item first ball is black and second is red.
\item one of them is black and other is red.
\end{enumerate}

\item In a hostel, 60\% of the students read Hindi newspaper, 40\% read English newspaper and 20\% read both Hindi and English newspapers. A student is selected at random.
		\label{ncert/12/13/2/15}
\begin{enumerate}
\item Find the probability that she reads neither Hindi nor English newspapers.
\item If she reads Hindi newspaper, find the probability that she reads English newspaper.
\item If she reads English newspaper, find the probability that she reads Hindi newspaper.\\
\end{enumerate}
\item The probability of obtaining an even prime number on each die, when a pair of dice is rolled is 
\begin{enumerate}
    \item $0$ 
    
    \item $\frac{1}{3}$ 
    
    \item $\frac{1}{12}$ 
    
    \item $\frac{1}{36}$ 
\end{enumerate}
\solution
		%\input{ncert/12/13/2/17/defs.tex}
	\item A bag contains 4 red and 4 black balls, another bag contains 2 red and 6 black balls. One of the two bags is selected at random and a ball is drawn from the bag which is found to be red. Find the probability that the ball is drawn from the first bag.
\\
\solution
		%\input{ncert/12/13/3/2/main.tex}
  \item
  Cards with numbers 2 to 101 are placed in a box. A card is selected at random.Find the probability that the card has
\begin{enumerate}[label=(\roman*)]
	\item an even number 
	\item a square number
\end{enumerate}
\solution
%\input{exemplar/10/13/3/32/main.tex}
\item
The king, queen and jack of clubs are removed from a deck of 52 playing cards and then well shuffled. Now one card is drawn at random from the remaining cards.  Determine the probability that the card is
\begin{enumerate}[label=(\roman*)]
\item a club
\item 10 of hearts
\end{enumerate}
\solution
%\input{exemplar/10/13/3/29/main.tex}
\item A team of medical students doing their internship have to assist during surgeries
at a city hospital. The probabilities of surgeries rated as very complex, complex,
routine, simple or very simple are respectively, 0.15, 0.20, 0.31, 0.26, .08. Find
the probabilities that a particular surgery will be rated
\begin{enumerate}
	\item complex or very complex;
	\item neither very complex nor very simple;
	\item routine or complex
	\item routine or simple
\end{enumerate}
\solution
%\input{exemplar/11/16/3/8(1)/main.tex}
\item A card is selected from a pack of 52 cards.
\begin{enumerate}[label=(\alph*)]
    \item How many points are there in the sample space?
    \item Calculate the probability that the card is an ace of spades.
    \item Calculate the probability that the card is (i) an ace and (ii) black card.
\end{enumerate}
\solution
%\input{exemplar/11/16/3/4/main2.tex}
\item The probability that a non leap year selected at random will contain 53 sundays.
\\
\solution
%\input{exemplar/10/13/1/19/main.tex}
\item One of the four persons John, Rita, Aslam or Gurpreet will be promoted next
month. Consequently the sample space consists of four elementary outcomes
S = {John promoted, Rita promoted, Aslam promoted, Gurpreet promoted}
You are told that the chances of John’s promotion is same as that of Gurpreet,
Rita’s chances of promotion are twice as likely as Johns. Aslam’s chances are
four times that of John.
\begin{enumerate}
	\item Determine
	\begin{enumerate}
		\item P (John promoted)
		\item P (Rita promoted)
		\item P (Aslam promoted)
		\item P (Gurpreet promoted)
	\end{enumerate}
	\item If A = {John promoted or Gurpreet promoted}, find P (A).
\end{enumerate}
\solution
%\input{exemplar/11/16/3/10/main.tex}
\item A card is drawn from a deck of 52 cards. Find the probability of getting a king or a heart or a red card.\\
\solution
%\input{exemplar/11/16/3/15/main.tex}
\item The probability that a student will pass his examination is 0.73, the probability of
the student getting a compartment is 0.13, and the probability that the student will
either pass or get compartment is 0.96. State True or False.\\
\solution
%\input{exemplar/11/16/3/31/main.tex}
\item A card is selected from a pack of 52 cards\\
\begin{enumerate}[label=(\alph*)]
\item How many points are there in the sample space?
\item Calculate the probability that the cards is an ace of spades.
\item Calculate the probability that the card is (i) an ace (ii)black card.\\
\end{enumerate}
%\input{ncert/11/16/3/4_1/Prob_4.tex}
\item In a non-leap year, the probability of having 53 tuesdays or 53 wednesdays is\\
\solution
%\input{exemplar/11/16/3/18/main.tex}
\item There are 1000 sealed envelopes in a box, 10 of them contain a cash prize of
Rs 100 each, 100 of them contain a cash prize of Rs 50 each and 200 of them
contain a cash prize of Rs 10 each and rest do not contain any cash prize. If they
are well shuffled and an envelope is picked up out, what is the probability that it
contains no cash prize?\\
\solution
%\input{exemplar/10/13/3/34/main.tex}
\item 
A die is thrown and a card is selected at random from a deck of 52 playing cards. The probability of getting an even number on the die and a spade card.\\
\solution
%\input{exemplar/12/13/3/78/main.tex}
\item
If 4-digit numbers greater than 5,000 are randomly formed from the digits 0, 1, 3, 5, and 7, what is the probability of forming a number divisible by 5 when:
\begin{enumerate}
    \item The digits are repeated?
    \item The repetition of digits is not allowed?
\end{enumerate}
\solution
%\input{ncert/11/16/4/9/main.tex}
\item Consider the probability space $\brak{\Omega, \mathcal{G}, P}$ where $\Omega = [0,2]$ and $\mathcal{G} = \cbrak{\phi, \Omega, [0,1], (1,2]}$. Let $X$ and $Y$ be two functions on $\Omega$ defined as
\begin{align*}
    X(\omega) = 
    \begin{cases}
        1 & \text{if }\omega \in [0, 1]\\
        2 & \text{if }\omega \in (1, 2]
    \end{cases}
\end{align*}
and
\begin{align*}
    Y(\omega) = 
    \begin{cases}
        2 & \text{if }\omega \in [0, 1.5]\\
        3 & \text{if }\omega \in (1.5, 2].
    \end{cases}
\end{align*}
Then which one of the following statements is true?
\begin{enumerate}
    \item [(A)] $X$ is a random variable with respect to $\mathcal{G}$, but $Y$ is not a random variable with respect to $\mathcal{G}$.
    \item [(B)] $Y$ is a random variable with respect to $\mathcal{G}$, but $X$ is not a random variable with respect to $\mathcal{G}$.
    \item [(C)] Neither $X$ nor $Y$ is a random variable with respect to $\mathcal{G}$.
    \item [(D)] Both $X$ and $Y$ are random variables with respect to $\mathcal{G}$.
\end{enumerate} \hfill (GATE ST 2023)\\
\solution
%\input{gate/ST/2023/14/main.tex}
	\item  A die is loaded in such a way that each odd number is twice as likely to occur as
each even number. Find $P(G)$, where $G$ is the event that a number greater than
3 occurs on a single roll of the die.
\\
\solution
		%\input{exemplar/11/16/3/5/main.tex}
	\item All the jacks, queens and kings are removed from a deck of 52 playing cards. The remaining cards are well shuffled and then one card is drawn at random. Giving ace a value 1 similar value for other cards, find the probability that the card has a value 
		\begin{enumerate}
			\item 7
			\item greater than 7
			\item less than 7
		\end{enumerate}
		%\input{exemplar/10/13/3/30/main.tex}
  \item A Lot consists of 48 mobile phones of which 42 are good, 3 have only minor defects and 3 have major defects.Varnika will buy a phone if it is good but the trader will only buy a mobile if it has no major defects. One phone is selected at random from the lot. What is the probability that it is
\begin{enumerate}
	\item acceptable to Varnika?
            \item acceptable to the trader?
\end{enumerate}
\solution
	%\input{exemplar/10/13/3/40/main.tex}
 \item A student says that if you throw a die, it will show up 1 or not 1. Therefore, the probability of getting 1 and the probability of getting 'not 1' each is equal to $\frac{1}{2}$. Is this correct? Give reasons.\\
 \solution
        %\input{exemplar/10/13/2/9/main.tex}
   \item Four candidates A, B, C, D have ap-
plied for the assignment to coach a school cricket
team. If A is twice as likely to be selected as B, and
B and C are given about the same chance of being
selected, while C is twice as likely to be selected
as D, what are the probabilities that
\begin{enumerate}
\item C will be selected?
\item A will not be selected?
\end{enumerate}
	%\input{exemplar/11/16/3/9/main.tex}
 \item A bag contain 24 balls of which $x$ balls are red, $2x$ are white and $3x$ are blue. A ball is selected at random, What is the probability that it is
\begin{enumerate}[label=\alph*)]
\item not red ?
\item white ?
\end{enumerate}
%\input{exemplar/10/13/3/41/main.tex}
If the letters of the word ASSASSINATION are arranged at random. Find the Probability that
\begin{enumerate}[label=(\alph*)]
\item Four $S's$ come consecutively in the word
\item Two  $I's$ and two $N's$ come together
\item All $A's$ are not coming together
\item No two $A's$ are coming together
\end{enumerate}
%\input{exemplar/11/16/3/14/main.tex}
	\item One urn contains two black balls (labelled B1 and B2) and one white ball. A
	second urn contains one black ball and two white balls (labelled W1 and W2).
	Suppose the following experiment is performed. One of the two urns is chosen
	at random. Next a ball is randomly chosen from the urn. Then a second ball is
	chosen at random from the same urn without replacing the first ball.
	
	\begin{enumerate}
	\item What is the probability that two black balls are chosen?
	
	\item What is the probability that two balls of opposite colour are chosen?
	\end{enumerate}
	\solution
	%\input{exemplar/11/16/3/12/main1.tex}
\end{enumerate}

		%
\item 
Two cards are drawn at random and without replacement from a pack of 52 playing cards. Find the probability that both the cards are black.
\\
\solution
		%\begin{enumerate}[label=\thesection.\arabic*,ref=\thesection.\theenumi]
	\item One card is drawn from a well-shuffled deck of 52 cards. Find the probability of getting
\begin{enumerate}
\item A king of red colour 
\item A face card 
\item A red face card
\item The jack of hearts
\item A spade
\item The queen of diamonds

\end{enumerate}
\solution
		%\input{ncert/10/15/1/14/main.tex}
	\item Five cards—the ten, jack, queen, king and ace of diamonds, are well-shuffled with their face downwards. One card is then picked up at random.
\begin{enumerate}
\item
What is the probability that the card is the queen? 
\item
If the queen is drawn and put aside, what is the probability that the second card picked up is (a) an ace? (b) a queen?\\
\end{enumerate}
\solution
		%\input{ncert/10/15/1/15/defs.tex}
	\item A bag contains $5$ red balls and some blue balls. If the probability of drawing a blue ball is double that if a red ball, determine the number of blue balls in the bag. 
		\\
\solution
		%\input{ncert/10/15/2/3/defs.tex}
	\item A card is selected from a pack of 52 cards.
 \begin{enumerate}[label=(\alph*)] 
                 \item How many points are there in the sample space?
                 \item Calculate the probability that the card is an ace of spades.
                 \item Calculate the probability that the card is (i) an ace and (ii) black card.
 \end{enumerate}
\solution
		%\input{ncert/11/16/3/4/main.tex}
\item Four cards are drawn from a well-shuffled deck of 52 cards. What is the probability of obtaining 3 diamonds and one spade.
\\
\solution
		%\input{ncert/11/16/4/2/defs.tex}
\item In a certain lottery 10,000 tickets are sold and ten equal prizes are awarded. What is the probability of not getting a prize if you buy (a) one ticket (b) two tickets (c) 10 tickets ?	
\\
\solution
		%\input{ncert/11/16/4/4/defs.tex}
		%
\item 
Out of 100 students, two sections of 40 and 60 are formed. If you and your friend are among the 100 students, what is the probability that
\begin{enumerate}
\item you both enter the same section?
\item you both enter the different sections?
\end{enumerate}
\solution
		%\input{ncert/11/16/4/5/defs.tex}
	\item 
The number lock of a suitcase has 4 wheels each labelled with ten digits i.e. from 0 to 9.The lock opens with a sequence of four digits with no repeats.What is the probability of a person getting the right sequence to open the suitcase.
\\
\solution
		%\input{ncert/11/16/4/10/defs.tex}
		%
\item 
Two cards are drawn at random and without replacement from a pack of 52 playing cards. Find the probability that both the cards are black.
\\
\solution
		%\input{ncert/12/13/2/2/defs.tex}
		\item A box of oranges is inspected by examining three randomly selected oranges drawn without replacement. If all the three oranges are good, the box is approved for sale, otherwise, it is rejected. Find the probability that a box containing 15 oranges out of which 12 are good and 3 are bad ones will be approved for sale.
		\label{ncert/12/13/2/3/defs.tex}
		\item Two balls are drawn at random with replacement from a box containing 10 black and 8 red balls. Find the probability that
		\label{ncert/12/13/2/12}
\begin{enumerate}
\item both balls are red.
\item first ball is black and second is red.
\item one of them is black and other is red.
\end{enumerate}

\item In a hostel, 60\% of the students read Hindi newspaper, 40\% read English newspaper and 20\% read both Hindi and English newspapers. A student is selected at random.
		\label{ncert/12/13/2/15}
\begin{enumerate}
\item Find the probability that she reads neither Hindi nor English newspapers.
\item If she reads Hindi newspaper, find the probability that she reads English newspaper.
\item If she reads English newspaper, find the probability that she reads Hindi newspaper.\\
\end{enumerate}
\item The probability of obtaining an even prime number on each die, when a pair of dice is rolled is 
\begin{enumerate}
    \item $0$ 
    
    \item $\frac{1}{3}$ 
    
    \item $\frac{1}{12}$ 
    
    \item $\frac{1}{36}$ 
\end{enumerate}
\solution
		%\input{ncert/12/13/2/17/defs.tex}
	\item A bag contains 4 red and 4 black balls, another bag contains 2 red and 6 black balls. One of the two bags is selected at random and a ball is drawn from the bag which is found to be red. Find the probability that the ball is drawn from the first bag.
\\
\solution
		%\input{ncert/12/13/3/2/main.tex}
  \item
  Cards with numbers 2 to 101 are placed in a box. A card is selected at random.Find the probability that the card has
\begin{enumerate}[label=(\roman*)]
	\item an even number 
	\item a square number
\end{enumerate}
\solution
%\input{exemplar/10/13/3/32/main.tex}
\item
The king, queen and jack of clubs are removed from a deck of 52 playing cards and then well shuffled. Now one card is drawn at random from the remaining cards.  Determine the probability that the card is
\begin{enumerate}[label=(\roman*)]
\item a club
\item 10 of hearts
\end{enumerate}
\solution
%\input{exemplar/10/13/3/29/main.tex}
\item A team of medical students doing their internship have to assist during surgeries
at a city hospital. The probabilities of surgeries rated as very complex, complex,
routine, simple or very simple are respectively, 0.15, 0.20, 0.31, 0.26, .08. Find
the probabilities that a particular surgery will be rated
\begin{enumerate}
	\item complex or very complex;
	\item neither very complex nor very simple;
	\item routine or complex
	\item routine or simple
\end{enumerate}
\solution
%\input{exemplar/11/16/3/8(1)/main.tex}
\item A card is selected from a pack of 52 cards.
\begin{enumerate}[label=(\alph*)]
    \item How many points are there in the sample space?
    \item Calculate the probability that the card is an ace of spades.
    \item Calculate the probability that the card is (i) an ace and (ii) black card.
\end{enumerate}
\solution
%\input{exemplar/11/16/3/4/main2.tex}
\item The probability that a non leap year selected at random will contain 53 sundays.
\\
\solution
%\input{exemplar/10/13/1/19/main.tex}
\item One of the four persons John, Rita, Aslam or Gurpreet will be promoted next
month. Consequently the sample space consists of four elementary outcomes
S = {John promoted, Rita promoted, Aslam promoted, Gurpreet promoted}
You are told that the chances of John’s promotion is same as that of Gurpreet,
Rita’s chances of promotion are twice as likely as Johns. Aslam’s chances are
four times that of John.
\begin{enumerate}
	\item Determine
	\begin{enumerate}
		\item P (John promoted)
		\item P (Rita promoted)
		\item P (Aslam promoted)
		\item P (Gurpreet promoted)
	\end{enumerate}
	\item If A = {John promoted or Gurpreet promoted}, find P (A).
\end{enumerate}
\solution
%\input{exemplar/11/16/3/10/main.tex}
\item A card is drawn from a deck of 52 cards. Find the probability of getting a king or a heart or a red card.\\
\solution
%\input{exemplar/11/16/3/15/main.tex}
\item The probability that a student will pass his examination is 0.73, the probability of
the student getting a compartment is 0.13, and the probability that the student will
either pass or get compartment is 0.96. State True or False.\\
\solution
%\input{exemplar/11/16/3/31/main.tex}
\item A card is selected from a pack of 52 cards\\
\begin{enumerate}[label=(\alph*)]
\item How many points are there in the sample space?
\item Calculate the probability that the cards is an ace of spades.
\item Calculate the probability that the card is (i) an ace (ii)black card.\\
\end{enumerate}
%\input{ncert/11/16/3/4_1/Prob_4.tex}
\item In a non-leap year, the probability of having 53 tuesdays or 53 wednesdays is\\
\solution
%\input{exemplar/11/16/3/18/main.tex}
\item There are 1000 sealed envelopes in a box, 10 of them contain a cash prize of
Rs 100 each, 100 of them contain a cash prize of Rs 50 each and 200 of them
contain a cash prize of Rs 10 each and rest do not contain any cash prize. If they
are well shuffled and an envelope is picked up out, what is the probability that it
contains no cash prize?\\
\solution
%\input{exemplar/10/13/3/34/main.tex}
\item 
A die is thrown and a card is selected at random from a deck of 52 playing cards. The probability of getting an even number on the die and a spade card.\\
\solution
%\input{exemplar/12/13/3/78/main.tex}
\item
If 4-digit numbers greater than 5,000 are randomly formed from the digits 0, 1, 3, 5, and 7, what is the probability of forming a number divisible by 5 when:
\begin{enumerate}
    \item The digits are repeated?
    \item The repetition of digits is not allowed?
\end{enumerate}
\solution
%\input{ncert/11/16/4/9/main.tex}
\item Consider the probability space $\brak{\Omega, \mathcal{G}, P}$ where $\Omega = [0,2]$ and $\mathcal{G} = \cbrak{\phi, \Omega, [0,1], (1,2]}$. Let $X$ and $Y$ be two functions on $\Omega$ defined as
\begin{align*}
    X(\omega) = 
    \begin{cases}
        1 & \text{if }\omega \in [0, 1]\\
        2 & \text{if }\omega \in (1, 2]
    \end{cases}
\end{align*}
and
\begin{align*}
    Y(\omega) = 
    \begin{cases}
        2 & \text{if }\omega \in [0, 1.5]\\
        3 & \text{if }\omega \in (1.5, 2].
    \end{cases}
\end{align*}
Then which one of the following statements is true?
\begin{enumerate}
    \item [(A)] $X$ is a random variable with respect to $\mathcal{G}$, but $Y$ is not a random variable with respect to $\mathcal{G}$.
    \item [(B)] $Y$ is a random variable with respect to $\mathcal{G}$, but $X$ is not a random variable with respect to $\mathcal{G}$.
    \item [(C)] Neither $X$ nor $Y$ is a random variable with respect to $\mathcal{G}$.
    \item [(D)] Both $X$ and $Y$ are random variables with respect to $\mathcal{G}$.
\end{enumerate} \hfill (GATE ST 2023)\\
\solution
%\input{gate/ST/2023/14/main.tex}
	\item  A die is loaded in such a way that each odd number is twice as likely to occur as
each even number. Find $P(G)$, where $G$ is the event that a number greater than
3 occurs on a single roll of the die.
\\
\solution
		%\input{exemplar/11/16/3/5/main.tex}
	\item All the jacks, queens and kings are removed from a deck of 52 playing cards. The remaining cards are well shuffled and then one card is drawn at random. Giving ace a value 1 similar value for other cards, find the probability that the card has a value 
		\begin{enumerate}
			\item 7
			\item greater than 7
			\item less than 7
		\end{enumerate}
		%\input{exemplar/10/13/3/30/main.tex}
  \item A Lot consists of 48 mobile phones of which 42 are good, 3 have only minor defects and 3 have major defects.Varnika will buy a phone if it is good but the trader will only buy a mobile if it has no major defects. One phone is selected at random from the lot. What is the probability that it is
\begin{enumerate}
	\item acceptable to Varnika?
            \item acceptable to the trader?
\end{enumerate}
\solution
	%\input{exemplar/10/13/3/40/main.tex}
 \item A student says that if you throw a die, it will show up 1 or not 1. Therefore, the probability of getting 1 and the probability of getting 'not 1' each is equal to $\frac{1}{2}$. Is this correct? Give reasons.\\
 \solution
        %\input{exemplar/10/13/2/9/main.tex}
   \item Four candidates A, B, C, D have ap-
plied for the assignment to coach a school cricket
team. If A is twice as likely to be selected as B, and
B and C are given about the same chance of being
selected, while C is twice as likely to be selected
as D, what are the probabilities that
\begin{enumerate}
\item C will be selected?
\item A will not be selected?
\end{enumerate}
	%\input{exemplar/11/16/3/9/main.tex}
 \item A bag contain 24 balls of which $x$ balls are red, $2x$ are white and $3x$ are blue. A ball is selected at random, What is the probability that it is
\begin{enumerate}[label=\alph*)]
\item not red ?
\item white ?
\end{enumerate}
%\input{exemplar/10/13/3/41/main.tex}
If the letters of the word ASSASSINATION are arranged at random. Find the Probability that
\begin{enumerate}[label=(\alph*)]
\item Four $S's$ come consecutively in the word
\item Two  $I's$ and two $N's$ come together
\item All $A's$ are not coming together
\item No two $A's$ are coming together
\end{enumerate}
%\input{exemplar/11/16/3/14/main.tex}
	\item One urn contains two black balls (labelled B1 and B2) and one white ball. A
	second urn contains one black ball and two white balls (labelled W1 and W2).
	Suppose the following experiment is performed. One of the two urns is chosen
	at random. Next a ball is randomly chosen from the urn. Then a second ball is
	chosen at random from the same urn without replacing the first ball.
	
	\begin{enumerate}
	\item What is the probability that two black balls are chosen?
	
	\item What is the probability that two balls of opposite colour are chosen?
	\end{enumerate}
	\solution
	%\input{exemplar/11/16/3/12/main1.tex}
\end{enumerate}

		\item A box of oranges is inspected by examining three randomly selected oranges drawn without replacement. If all the three oranges are good, the box is approved for sale, otherwise, it is rejected. Find the probability that a box containing 15 oranges out of which 12 are good and 3 are bad ones will be approved for sale.
		\label{ncert/12/13/2/3/defs.tex}
		\item Two balls are drawn at random with replacement from a box containing 10 black and 8 red balls. Find the probability that
		\label{ncert/12/13/2/12}
\begin{enumerate}
\item both balls are red.
\item first ball is black and second is red.
\item one of them is black and other is red.
\end{enumerate}

\item In a hostel, 60\% of the students read Hindi newspaper, 40\% read English newspaper and 20\% read both Hindi and English newspapers. A student is selected at random.
		\label{ncert/12/13/2/15}
\begin{enumerate}
\item Find the probability that she reads neither Hindi nor English newspapers.
\item If she reads Hindi newspaper, find the probability that she reads English newspaper.
\item If she reads English newspaper, find the probability that she reads Hindi newspaper.\\
\end{enumerate}
\item The probability of obtaining an even prime number on each die, when a pair of dice is rolled is 
\begin{enumerate}
    \item $0$ 
    
    \item $\frac{1}{3}$ 
    
    \item $\frac{1}{12}$ 
    
    \item $\frac{1}{36}$ 
\end{enumerate}
\solution
		%\begin{enumerate}[label=\thesection.\arabic*,ref=\thesection.\theenumi]
	\item One card is drawn from a well-shuffled deck of 52 cards. Find the probability of getting
\begin{enumerate}
\item A king of red colour 
\item A face card 
\item A red face card
\item The jack of hearts
\item A spade
\item The queen of diamonds

\end{enumerate}
\solution
		%\input{ncert/10/15/1/14/main.tex}
	\item Five cards—the ten, jack, queen, king and ace of diamonds, are well-shuffled with their face downwards. One card is then picked up at random.
\begin{enumerate}
\item
What is the probability that the card is the queen? 
\item
If the queen is drawn and put aside, what is the probability that the second card picked up is (a) an ace? (b) a queen?\\
\end{enumerate}
\solution
		%\input{ncert/10/15/1/15/defs.tex}
	\item A bag contains $5$ red balls and some blue balls. If the probability of drawing a blue ball is double that if a red ball, determine the number of blue balls in the bag. 
		\\
\solution
		%\input{ncert/10/15/2/3/defs.tex}
	\item A card is selected from a pack of 52 cards.
 \begin{enumerate}[label=(\alph*)] 
                 \item How many points are there in the sample space?
                 \item Calculate the probability that the card is an ace of spades.
                 \item Calculate the probability that the card is (i) an ace and (ii) black card.
 \end{enumerate}
\solution
		%\input{ncert/11/16/3/4/main.tex}
\item Four cards are drawn from a well-shuffled deck of 52 cards. What is the probability of obtaining 3 diamonds and one spade.
\\
\solution
		%\input{ncert/11/16/4/2/defs.tex}
\item In a certain lottery 10,000 tickets are sold and ten equal prizes are awarded. What is the probability of not getting a prize if you buy (a) one ticket (b) two tickets (c) 10 tickets ?	
\\
\solution
		%\input{ncert/11/16/4/4/defs.tex}
		%
\item 
Out of 100 students, two sections of 40 and 60 are formed. If you and your friend are among the 100 students, what is the probability that
\begin{enumerate}
\item you both enter the same section?
\item you both enter the different sections?
\end{enumerate}
\solution
		%\input{ncert/11/16/4/5/defs.tex}
	\item 
The number lock of a suitcase has 4 wheels each labelled with ten digits i.e. from 0 to 9.The lock opens with a sequence of four digits with no repeats.What is the probability of a person getting the right sequence to open the suitcase.
\\
\solution
		%\input{ncert/11/16/4/10/defs.tex}
		%
\item 
Two cards are drawn at random and without replacement from a pack of 52 playing cards. Find the probability that both the cards are black.
\\
\solution
		%\input{ncert/12/13/2/2/defs.tex}
		\item A box of oranges is inspected by examining three randomly selected oranges drawn without replacement. If all the three oranges are good, the box is approved for sale, otherwise, it is rejected. Find the probability that a box containing 15 oranges out of which 12 are good and 3 are bad ones will be approved for sale.
		\label{ncert/12/13/2/3/defs.tex}
		\item Two balls are drawn at random with replacement from a box containing 10 black and 8 red balls. Find the probability that
		\label{ncert/12/13/2/12}
\begin{enumerate}
\item both balls are red.
\item first ball is black and second is red.
\item one of them is black and other is red.
\end{enumerate}

\item In a hostel, 60\% of the students read Hindi newspaper, 40\% read English newspaper and 20\% read both Hindi and English newspapers. A student is selected at random.
		\label{ncert/12/13/2/15}
\begin{enumerate}
\item Find the probability that she reads neither Hindi nor English newspapers.
\item If she reads Hindi newspaper, find the probability that she reads English newspaper.
\item If she reads English newspaper, find the probability that she reads Hindi newspaper.\\
\end{enumerate}
\item The probability of obtaining an even prime number on each die, when a pair of dice is rolled is 
\begin{enumerate}
    \item $0$ 
    
    \item $\frac{1}{3}$ 
    
    \item $\frac{1}{12}$ 
    
    \item $\frac{1}{36}$ 
\end{enumerate}
\solution
		%\input{ncert/12/13/2/17/defs.tex}
	\item A bag contains 4 red and 4 black balls, another bag contains 2 red and 6 black balls. One of the two bags is selected at random and a ball is drawn from the bag which is found to be red. Find the probability that the ball is drawn from the first bag.
\\
\solution
		%\input{ncert/12/13/3/2/main.tex}
  \item
  Cards with numbers 2 to 101 are placed in a box. A card is selected at random.Find the probability that the card has
\begin{enumerate}[label=(\roman*)]
	\item an even number 
	\item a square number
\end{enumerate}
\solution
%\input{exemplar/10/13/3/32/main.tex}
\item
The king, queen and jack of clubs are removed from a deck of 52 playing cards and then well shuffled. Now one card is drawn at random from the remaining cards.  Determine the probability that the card is
\begin{enumerate}[label=(\roman*)]
\item a club
\item 10 of hearts
\end{enumerate}
\solution
%\input{exemplar/10/13/3/29/main.tex}
\item A team of medical students doing their internship have to assist during surgeries
at a city hospital. The probabilities of surgeries rated as very complex, complex,
routine, simple or very simple are respectively, 0.15, 0.20, 0.31, 0.26, .08. Find
the probabilities that a particular surgery will be rated
\begin{enumerate}
	\item complex or very complex;
	\item neither very complex nor very simple;
	\item routine or complex
	\item routine or simple
\end{enumerate}
\solution
%\input{exemplar/11/16/3/8(1)/main.tex}
\item A card is selected from a pack of 52 cards.
\begin{enumerate}[label=(\alph*)]
    \item How many points are there in the sample space?
    \item Calculate the probability that the card is an ace of spades.
    \item Calculate the probability that the card is (i) an ace and (ii) black card.
\end{enumerate}
\solution
%\input{exemplar/11/16/3/4/main2.tex}
\item The probability that a non leap year selected at random will contain 53 sundays.
\\
\solution
%\input{exemplar/10/13/1/19/main.tex}
\item One of the four persons John, Rita, Aslam or Gurpreet will be promoted next
month. Consequently the sample space consists of four elementary outcomes
S = {John promoted, Rita promoted, Aslam promoted, Gurpreet promoted}
You are told that the chances of John’s promotion is same as that of Gurpreet,
Rita’s chances of promotion are twice as likely as Johns. Aslam’s chances are
four times that of John.
\begin{enumerate}
	\item Determine
	\begin{enumerate}
		\item P (John promoted)
		\item P (Rita promoted)
		\item P (Aslam promoted)
		\item P (Gurpreet promoted)
	\end{enumerate}
	\item If A = {John promoted or Gurpreet promoted}, find P (A).
\end{enumerate}
\solution
%\input{exemplar/11/16/3/10/main.tex}
\item A card is drawn from a deck of 52 cards. Find the probability of getting a king or a heart or a red card.\\
\solution
%\input{exemplar/11/16/3/15/main.tex}
\item The probability that a student will pass his examination is 0.73, the probability of
the student getting a compartment is 0.13, and the probability that the student will
either pass or get compartment is 0.96. State True or False.\\
\solution
%\input{exemplar/11/16/3/31/main.tex}
\item A card is selected from a pack of 52 cards\\
\begin{enumerate}[label=(\alph*)]
\item How many points are there in the sample space?
\item Calculate the probability that the cards is an ace of spades.
\item Calculate the probability that the card is (i) an ace (ii)black card.\\
\end{enumerate}
%\input{ncert/11/16/3/4_1/Prob_4.tex}
\item In a non-leap year, the probability of having 53 tuesdays or 53 wednesdays is\\
\solution
%\input{exemplar/11/16/3/18/main.tex}
\item There are 1000 sealed envelopes in a box, 10 of them contain a cash prize of
Rs 100 each, 100 of them contain a cash prize of Rs 50 each and 200 of them
contain a cash prize of Rs 10 each and rest do not contain any cash prize. If they
are well shuffled and an envelope is picked up out, what is the probability that it
contains no cash prize?\\
\solution
%\input{exemplar/10/13/3/34/main.tex}
\item 
A die is thrown and a card is selected at random from a deck of 52 playing cards. The probability of getting an even number on the die and a spade card.\\
\solution
%\input{exemplar/12/13/3/78/main.tex}
\item
If 4-digit numbers greater than 5,000 are randomly formed from the digits 0, 1, 3, 5, and 7, what is the probability of forming a number divisible by 5 when:
\begin{enumerate}
    \item The digits are repeated?
    \item The repetition of digits is not allowed?
\end{enumerate}
\solution
%\input{ncert/11/16/4/9/main.tex}
\item Consider the probability space $\brak{\Omega, \mathcal{G}, P}$ where $\Omega = [0,2]$ and $\mathcal{G} = \cbrak{\phi, \Omega, [0,1], (1,2]}$. Let $X$ and $Y$ be two functions on $\Omega$ defined as
\begin{align*}
    X(\omega) = 
    \begin{cases}
        1 & \text{if }\omega \in [0, 1]\\
        2 & \text{if }\omega \in (1, 2]
    \end{cases}
\end{align*}
and
\begin{align*}
    Y(\omega) = 
    \begin{cases}
        2 & \text{if }\omega \in [0, 1.5]\\
        3 & \text{if }\omega \in (1.5, 2].
    \end{cases}
\end{align*}
Then which one of the following statements is true?
\begin{enumerate}
    \item [(A)] $X$ is a random variable with respect to $\mathcal{G}$, but $Y$ is not a random variable with respect to $\mathcal{G}$.
    \item [(B)] $Y$ is a random variable with respect to $\mathcal{G}$, but $X$ is not a random variable with respect to $\mathcal{G}$.
    \item [(C)] Neither $X$ nor $Y$ is a random variable with respect to $\mathcal{G}$.
    \item [(D)] Both $X$ and $Y$ are random variables with respect to $\mathcal{G}$.
\end{enumerate} \hfill (GATE ST 2023)\\
\solution
%\input{gate/ST/2023/14/main.tex}
	\item  A die is loaded in such a way that each odd number is twice as likely to occur as
each even number. Find $P(G)$, where $G$ is the event that a number greater than
3 occurs on a single roll of the die.
\\
\solution
		%\input{exemplar/11/16/3/5/main.tex}
	\item All the jacks, queens and kings are removed from a deck of 52 playing cards. The remaining cards are well shuffled and then one card is drawn at random. Giving ace a value 1 similar value for other cards, find the probability that the card has a value 
		\begin{enumerate}
			\item 7
			\item greater than 7
			\item less than 7
		\end{enumerate}
		%\input{exemplar/10/13/3/30/main.tex}
  \item A Lot consists of 48 mobile phones of which 42 are good, 3 have only minor defects and 3 have major defects.Varnika will buy a phone if it is good but the trader will only buy a mobile if it has no major defects. One phone is selected at random from the lot. What is the probability that it is
\begin{enumerate}
	\item acceptable to Varnika?
            \item acceptable to the trader?
\end{enumerate}
\solution
	%\input{exemplar/10/13/3/40/main.tex}
 \item A student says that if you throw a die, it will show up 1 or not 1. Therefore, the probability of getting 1 and the probability of getting 'not 1' each is equal to $\frac{1}{2}$. Is this correct? Give reasons.\\
 \solution
        %\input{exemplar/10/13/2/9/main.tex}
   \item Four candidates A, B, C, D have ap-
plied for the assignment to coach a school cricket
team. If A is twice as likely to be selected as B, and
B and C are given about the same chance of being
selected, while C is twice as likely to be selected
as D, what are the probabilities that
\begin{enumerate}
\item C will be selected?
\item A will not be selected?
\end{enumerate}
	%\input{exemplar/11/16/3/9/main.tex}
 \item A bag contain 24 balls of which $x$ balls are red, $2x$ are white and $3x$ are blue. A ball is selected at random, What is the probability that it is
\begin{enumerate}[label=\alph*)]
\item not red ?
\item white ?
\end{enumerate}
%\input{exemplar/10/13/3/41/main.tex}
If the letters of the word ASSASSINATION are arranged at random. Find the Probability that
\begin{enumerate}[label=(\alph*)]
\item Four $S's$ come consecutively in the word
\item Two  $I's$ and two $N's$ come together
\item All $A's$ are not coming together
\item No two $A's$ are coming together
\end{enumerate}
%\input{exemplar/11/16/3/14/main.tex}
	\item One urn contains two black balls (labelled B1 and B2) and one white ball. A
	second urn contains one black ball and two white balls (labelled W1 and W2).
	Suppose the following experiment is performed. One of the two urns is chosen
	at random. Next a ball is randomly chosen from the urn. Then a second ball is
	chosen at random from the same urn without replacing the first ball.
	
	\begin{enumerate}
	\item What is the probability that two black balls are chosen?
	
	\item What is the probability that two balls of opposite colour are chosen?
	\end{enumerate}
	\solution
	%\input{exemplar/11/16/3/12/main1.tex}
\end{enumerate}

	\item A bag contains 4 red and 4 black balls, another bag contains 2 red and 6 black balls. One of the two bags is selected at random and a ball is drawn from the bag which is found to be red. Find the probability that the ball is drawn from the first bag.
\\
\solution
		%\begin{table}[H]
	\centering
\begin{tabular}{|c|c|c|}
\hline
Random variable &Value &Definition\\ \hline
\multirow{3}{*}{X} &0 &Slips of Rs 1\\
&1 &Slips of Rs 5\\
&2 &Slips of Rs 13\\ \hline
\multirow{2}{*}{Y} &0 &Box A\\
&1 &Box B\\\hline
\end{tabular}
\caption{}
\label{tab:Distribution}
\end{table}
See \tabref{tab:Distribution}.
\begin{align}
p_{Y}\brak{k}= \begin{cases} 
      \frac{1}{3} & {k=0} \\
      \frac{2}{3 }& {k=1} 
   \end{cases}
   \\
p_{Y|X}\brak{0|0} = \frac{19}{25}\, 
p_{Y|X}\brak{0|1} = \frac{6}{25}\,
p_{Y|X}\brak{1|0} = \frac{45}{50}\,
p_{Y|X}\brak{1|2} = \frac{5}{50}
\end{align}
The desired probability is the probability that a slip drawn at random is marked other than Rs 1,
\begin{align}
&=1-p_X\brak{0}\\
&= p_X(1) + p_X(2)
\end{align}
Using Bayes theorem,
\begin{align}
&= p_Y\brak{0} \times \pr{Y=0 | X=1} + p_Y\brak{1} \times \pr{Y=1|X=2}\\
&=\frac{1}{3} \times \frac{6}{25} + \frac{2}{3} \times \frac{5}{50}\\
&=\frac{11}{75}
\end{align}

\newpage

%\tableofcontents

\bigskip

\renewcommand{\thefigure}{\theenumi}
\renewcommand{\thetable}{\theenumi}
%\renewcommand{\theequation}{\theenumi}

%\begin{abstract}
%%\boldmath
%In this letter, an algorithm for evaluating the exact analytical bit error rate  (BER)  for the piecewise linear (PL) combiner for  multiple relays is presented. Previous results were available only for upto three relays. The algorithm is unique in the sense that  the actual mathematical expressions, that are prohibitively large, need not be explicitly obtained. The diversity gain due to multiple relays is shown through plots of the analytical BER, well supported by simulations. 
%
%\end{abstract}
% IEEEtran.cls defaults to using nonbold math in the Abstract.
% This preserves the distinction between vectors and scalars. However,
% if the journal you are submitting to favors bold math in the abstract,
% then you can use LaTeX's standard command \boldmath at the very start
% of the abstract to achieve this. Many IEEE journals frown on math
% in the abstract anyway.

% Note that keywords are not normally used for peerreview papers.
%\begin{IEEEkeywords}
%Cooperative diversity, decode and forward, piecewise linear
%\end{IEEEkeywords}



% For peer review papers, you can put extra information on the cover
% page as needed:
% \ifCLASSOPTIONpeerreview
% \begin{center} \bfseries EDICS Category: 3-BBND \end{center}
% \fi
%
% For peerreview papers, this IEEEtran command inserts a page break and
% creates the second title. It will be ignored for other modes.
%\IEEEpeerreviewmaketitle




  \item
  Cards with numbers 2 to 101 are placed in a box. A card is selected at random.Find the probability that the card has
\begin{enumerate}[label=(\roman*)]
	\item an even number 
	\item a square number
\end{enumerate}
\solution
%\begin{table}[H]
	\centering
\begin{tabular}{|c|c|c|}
\hline
Random variable &Value &Definition\\ \hline
\multirow{3}{*}{X} &0 &Slips of Rs 1\\
&1 &Slips of Rs 5\\
&2 &Slips of Rs 13\\ \hline
\multirow{2}{*}{Y} &0 &Box A\\
&1 &Box B\\\hline
\end{tabular}
\caption{}
\label{tab:Distribution}
\end{table}
See \tabref{tab:Distribution}.
\begin{align}
p_{Y}\brak{k}= \begin{cases} 
      \frac{1}{3} & {k=0} \\
      \frac{2}{3 }& {k=1} 
   \end{cases}
   \\
p_{Y|X}\brak{0|0} = \frac{19}{25}\, 
p_{Y|X}\brak{0|1} = \frac{6}{25}\,
p_{Y|X}\brak{1|0} = \frac{45}{50}\,
p_{Y|X}\brak{1|2} = \frac{5}{50}
\end{align}
The desired probability is the probability that a slip drawn at random is marked other than Rs 1,
\begin{align}
&=1-p_X\brak{0}\\
&= p_X(1) + p_X(2)
\end{align}
Using Bayes theorem,
\begin{align}
&= p_Y\brak{0} \times \pr{Y=0 | X=1} + p_Y\brak{1} \times \pr{Y=1|X=2}\\
&=\frac{1}{3} \times \frac{6}{25} + \frac{2}{3} \times \frac{5}{50}\\
&=\frac{11}{75}
\end{align}

\newpage

%\tableofcontents

\bigskip

\renewcommand{\thefigure}{\theenumi}
\renewcommand{\thetable}{\theenumi}
%\renewcommand{\theequation}{\theenumi}

%\begin{abstract}
%%\boldmath
%In this letter, an algorithm for evaluating the exact analytical bit error rate  (BER)  for the piecewise linear (PL) combiner for  multiple relays is presented. Previous results were available only for upto three relays. The algorithm is unique in the sense that  the actual mathematical expressions, that are prohibitively large, need not be explicitly obtained. The diversity gain due to multiple relays is shown through plots of the analytical BER, well supported by simulations. 
%
%\end{abstract}
% IEEEtran.cls defaults to using nonbold math in the Abstract.
% This preserves the distinction between vectors and scalars. However,
% if the journal you are submitting to favors bold math in the abstract,
% then you can use LaTeX's standard command \boldmath at the very start
% of the abstract to achieve this. Many IEEE journals frown on math
% in the abstract anyway.

% Note that keywords are not normally used for peerreview papers.
%\begin{IEEEkeywords}
%Cooperative diversity, decode and forward, piecewise linear
%\end{IEEEkeywords}



% For peer review papers, you can put extra information on the cover
% page as needed:
% \ifCLASSOPTIONpeerreview
% \begin{center} \bfseries EDICS Category: 3-BBND \end{center}
% \fi
%
% For peerreview papers, this IEEEtran command inserts a page break and
% creates the second title. It will be ignored for other modes.
%\IEEEpeerreviewmaketitle




\item
The king, queen and jack of clubs are removed from a deck of 52 playing cards and then well shuffled. Now one card is drawn at random from the remaining cards.  Determine the probability that the card is
\begin{enumerate}[label=(\roman*)]
\item a club
\item 10 of hearts
\end{enumerate}
\solution
%\begin{table}[H]
	\centering
\begin{tabular}{|c|c|c|}
\hline
Random variable &Value &Definition\\ \hline
\multirow{3}{*}{X} &0 &Slips of Rs 1\\
&1 &Slips of Rs 5\\
&2 &Slips of Rs 13\\ \hline
\multirow{2}{*}{Y} &0 &Box A\\
&1 &Box B\\\hline
\end{tabular}
\caption{}
\label{tab:Distribution}
\end{table}
See \tabref{tab:Distribution}.
\begin{align}
p_{Y}\brak{k}= \begin{cases} 
      \frac{1}{3} & {k=0} \\
      \frac{2}{3 }& {k=1} 
   \end{cases}
   \\
p_{Y|X}\brak{0|0} = \frac{19}{25}\, 
p_{Y|X}\brak{0|1} = \frac{6}{25}\,
p_{Y|X}\brak{1|0} = \frac{45}{50}\,
p_{Y|X}\brak{1|2} = \frac{5}{50}
\end{align}
The desired probability is the probability that a slip drawn at random is marked other than Rs 1,
\begin{align}
&=1-p_X\brak{0}\\
&= p_X(1) + p_X(2)
\end{align}
Using Bayes theorem,
\begin{align}
&= p_Y\brak{0} \times \pr{Y=0 | X=1} + p_Y\brak{1} \times \pr{Y=1|X=2}\\
&=\frac{1}{3} \times \frac{6}{25} + \frac{2}{3} \times \frac{5}{50}\\
&=\frac{11}{75}
\end{align}

\newpage

%\tableofcontents

\bigskip

\renewcommand{\thefigure}{\theenumi}
\renewcommand{\thetable}{\theenumi}
%\renewcommand{\theequation}{\theenumi}

%\begin{abstract}
%%\boldmath
%In this letter, an algorithm for evaluating the exact analytical bit error rate  (BER)  for the piecewise linear (PL) combiner for  multiple relays is presented. Previous results were available only for upto three relays. The algorithm is unique in the sense that  the actual mathematical expressions, that are prohibitively large, need not be explicitly obtained. The diversity gain due to multiple relays is shown through plots of the analytical BER, well supported by simulations. 
%
%\end{abstract}
% IEEEtran.cls defaults to using nonbold math in the Abstract.
% This preserves the distinction between vectors and scalars. However,
% if the journal you are submitting to favors bold math in the abstract,
% then you can use LaTeX's standard command \boldmath at the very start
% of the abstract to achieve this. Many IEEE journals frown on math
% in the abstract anyway.

% Note that keywords are not normally used for peerreview papers.
%\begin{IEEEkeywords}
%Cooperative diversity, decode and forward, piecewise linear
%\end{IEEEkeywords}



% For peer review papers, you can put extra information on the cover
% page as needed:
% \ifCLASSOPTIONpeerreview
% \begin{center} \bfseries EDICS Category: 3-BBND \end{center}
% \fi
%
% For peerreview papers, this IEEEtran command inserts a page break and
% creates the second title. It will be ignored for other modes.
%\IEEEpeerreviewmaketitle




\item A team of medical students doing their internship have to assist during surgeries
at a city hospital. The probabilities of surgeries rated as very complex, complex,
routine, simple or very simple are respectively, 0.15, 0.20, 0.31, 0.26, .08. Find
the probabilities that a particular surgery will be rated
\begin{enumerate}
	\item complex or very complex;
	\item neither very complex nor very simple;
	\item routine or complex
	\item routine or simple
\end{enumerate}
\solution
%\begin{table}[H]
	\centering
\begin{tabular}{|c|c|c|}
\hline
Random variable &Value &Definition\\ \hline
\multirow{3}{*}{X} &0 &Slips of Rs 1\\
&1 &Slips of Rs 5\\
&2 &Slips of Rs 13\\ \hline
\multirow{2}{*}{Y} &0 &Box A\\
&1 &Box B\\\hline
\end{tabular}
\caption{}
\label{tab:Distribution}
\end{table}
See \tabref{tab:Distribution}.
\begin{align}
p_{Y}\brak{k}= \begin{cases} 
      \frac{1}{3} & {k=0} \\
      \frac{2}{3 }& {k=1} 
   \end{cases}
   \\
p_{Y|X}\brak{0|0} = \frac{19}{25}\, 
p_{Y|X}\brak{0|1} = \frac{6}{25}\,
p_{Y|X}\brak{1|0} = \frac{45}{50}\,
p_{Y|X}\brak{1|2} = \frac{5}{50}
\end{align}
The desired probability is the probability that a slip drawn at random is marked other than Rs 1,
\begin{align}
&=1-p_X\brak{0}\\
&= p_X(1) + p_X(2)
\end{align}
Using Bayes theorem,
\begin{align}
&= p_Y\brak{0} \times \pr{Y=0 | X=1} + p_Y\brak{1} \times \pr{Y=1|X=2}\\
&=\frac{1}{3} \times \frac{6}{25} + \frac{2}{3} \times \frac{5}{50}\\
&=\frac{11}{75}
\end{align}

\newpage

%\tableofcontents

\bigskip

\renewcommand{\thefigure}{\theenumi}
\renewcommand{\thetable}{\theenumi}
%\renewcommand{\theequation}{\theenumi}

%\begin{abstract}
%%\boldmath
%In this letter, an algorithm for evaluating the exact analytical bit error rate  (BER)  for the piecewise linear (PL) combiner for  multiple relays is presented. Previous results were available only for upto three relays. The algorithm is unique in the sense that  the actual mathematical expressions, that are prohibitively large, need not be explicitly obtained. The diversity gain due to multiple relays is shown through plots of the analytical BER, well supported by simulations. 
%
%\end{abstract}
% IEEEtran.cls defaults to using nonbold math in the Abstract.
% This preserves the distinction between vectors and scalars. However,
% if the journal you are submitting to favors bold math in the abstract,
% then you can use LaTeX's standard command \boldmath at the very start
% of the abstract to achieve this. Many IEEE journals frown on math
% in the abstract anyway.

% Note that keywords are not normally used for peerreview papers.
%\begin{IEEEkeywords}
%Cooperative diversity, decode and forward, piecewise linear
%\end{IEEEkeywords}



% For peer review papers, you can put extra information on the cover
% page as needed:
% \ifCLASSOPTIONpeerreview
% \begin{center} \bfseries EDICS Category: 3-BBND \end{center}
% \fi
%
% For peerreview papers, this IEEEtran command inserts a page break and
% creates the second title. It will be ignored for other modes.
%\IEEEpeerreviewmaketitle




\item A card is selected from a pack of 52 cards.
\begin{enumerate}[label=(\alph*)]
    \item How many points are there in the sample space?
    \item Calculate the probability that the card is an ace of spades.
    \item Calculate the probability that the card is (i) an ace and (ii) black card.
\end{enumerate}
\solution
%Let $X$ be an bernoulli rv defined as in \tabref{tab:exemplar/11/16/3/26}.  Then, 
\begin{equation}
    p =
        \frac{4}{11} 
\end{equation}
\begin{table}[H]
	\centering
	\input{exemplar/11/16/3/26/tables/Table2.tex}
	\caption{}
        \label{tab:exemplar/11/16/3/26}
\end{table}

\item The probability that a non leap year selected at random will contain 53 sundays.
\\
\solution
%\begin{table}[H]
	\centering
\begin{tabular}{|c|c|c|}
\hline
Random variable &Value &Definition\\ \hline
\multirow{3}{*}{X} &0 &Slips of Rs 1\\
&1 &Slips of Rs 5\\
&2 &Slips of Rs 13\\ \hline
\multirow{2}{*}{Y} &0 &Box A\\
&1 &Box B\\\hline
\end{tabular}
\caption{}
\label{tab:Distribution}
\end{table}
See \tabref{tab:Distribution}.
\begin{align}
p_{Y}\brak{k}= \begin{cases} 
      \frac{1}{3} & {k=0} \\
      \frac{2}{3 }& {k=1} 
   \end{cases}
   \\
p_{Y|X}\brak{0|0} = \frac{19}{25}\, 
p_{Y|X}\brak{0|1} = \frac{6}{25}\,
p_{Y|X}\brak{1|0} = \frac{45}{50}\,
p_{Y|X}\brak{1|2} = \frac{5}{50}
\end{align}
The desired probability is the probability that a slip drawn at random is marked other than Rs 1,
\begin{align}
&=1-p_X\brak{0}\\
&= p_X(1) + p_X(2)
\end{align}
Using Bayes theorem,
\begin{align}
&= p_Y\brak{0} \times \pr{Y=0 | X=1} + p_Y\brak{1} \times \pr{Y=1|X=2}\\
&=\frac{1}{3} \times \frac{6}{25} + \frac{2}{3} \times \frac{5}{50}\\
&=\frac{11}{75}
\end{align}

\newpage

%\tableofcontents

\bigskip

\renewcommand{\thefigure}{\theenumi}
\renewcommand{\thetable}{\theenumi}
%\renewcommand{\theequation}{\theenumi}

%\begin{abstract}
%%\boldmath
%In this letter, an algorithm for evaluating the exact analytical bit error rate  (BER)  for the piecewise linear (PL) combiner for  multiple relays is presented. Previous results were available only for upto three relays. The algorithm is unique in the sense that  the actual mathematical expressions, that are prohibitively large, need not be explicitly obtained. The diversity gain due to multiple relays is shown through plots of the analytical BER, well supported by simulations. 
%
%\end{abstract}
% IEEEtran.cls defaults to using nonbold math in the Abstract.
% This preserves the distinction between vectors and scalars. However,
% if the journal you are submitting to favors bold math in the abstract,
% then you can use LaTeX's standard command \boldmath at the very start
% of the abstract to achieve this. Many IEEE journals frown on math
% in the abstract anyway.

% Note that keywords are not normally used for peerreview papers.
%\begin{IEEEkeywords}
%Cooperative diversity, decode and forward, piecewise linear
%\end{IEEEkeywords}



% For peer review papers, you can put extra information on the cover
% page as needed:
% \ifCLASSOPTIONpeerreview
% \begin{center} \bfseries EDICS Category: 3-BBND \end{center}
% \fi
%
% For peerreview papers, this IEEEtran command inserts a page break and
% creates the second title. It will be ignored for other modes.
%\IEEEpeerreviewmaketitle




\item One of the four persons John, Rita, Aslam or Gurpreet will be promoted next
month. Consequently the sample space consists of four elementary outcomes
S = {John promoted, Rita promoted, Aslam promoted, Gurpreet promoted}
You are told that the chances of John’s promotion is same as that of Gurpreet,
Rita’s chances of promotion are twice as likely as Johns. Aslam’s chances are
four times that of John.
\begin{enumerate}
	\item Determine
	\begin{enumerate}
		\item P (John promoted)
		\item P (Rita promoted)
		\item P (Aslam promoted)
		\item P (Gurpreet promoted)
	\end{enumerate}
	\item If A = {John promoted or Gurpreet promoted}, find P (A).
\end{enumerate}
\solution
%\begin{table}[H]
	\centering
\begin{tabular}{|c|c|c|}
\hline
Random variable &Value &Definition\\ \hline
\multirow{3}{*}{X} &0 &Slips of Rs 1\\
&1 &Slips of Rs 5\\
&2 &Slips of Rs 13\\ \hline
\multirow{2}{*}{Y} &0 &Box A\\
&1 &Box B\\\hline
\end{tabular}
\caption{}
\label{tab:Distribution}
\end{table}
See \tabref{tab:Distribution}.
\begin{align}
p_{Y}\brak{k}= \begin{cases} 
      \frac{1}{3} & {k=0} \\
      \frac{2}{3 }& {k=1} 
   \end{cases}
   \\
p_{Y|X}\brak{0|0} = \frac{19}{25}\, 
p_{Y|X}\brak{0|1} = \frac{6}{25}\,
p_{Y|X}\brak{1|0} = \frac{45}{50}\,
p_{Y|X}\brak{1|2} = \frac{5}{50}
\end{align}
The desired probability is the probability that a slip drawn at random is marked other than Rs 1,
\begin{align}
&=1-p_X\brak{0}\\
&= p_X(1) + p_X(2)
\end{align}
Using Bayes theorem,
\begin{align}
&= p_Y\brak{0} \times \pr{Y=0 | X=1} + p_Y\brak{1} \times \pr{Y=1|X=2}\\
&=\frac{1}{3} \times \frac{6}{25} + \frac{2}{3} \times \frac{5}{50}\\
&=\frac{11}{75}
\end{align}

\newpage

%\tableofcontents

\bigskip

\renewcommand{\thefigure}{\theenumi}
\renewcommand{\thetable}{\theenumi}
%\renewcommand{\theequation}{\theenumi}

%\begin{abstract}
%%\boldmath
%In this letter, an algorithm for evaluating the exact analytical bit error rate  (BER)  for the piecewise linear (PL) combiner for  multiple relays is presented. Previous results were available only for upto three relays. The algorithm is unique in the sense that  the actual mathematical expressions, that are prohibitively large, need not be explicitly obtained. The diversity gain due to multiple relays is shown through plots of the analytical BER, well supported by simulations. 
%
%\end{abstract}
% IEEEtran.cls defaults to using nonbold math in the Abstract.
% This preserves the distinction between vectors and scalars. However,
% if the journal you are submitting to favors bold math in the abstract,
% then you can use LaTeX's standard command \boldmath at the very start
% of the abstract to achieve this. Many IEEE journals frown on math
% in the abstract anyway.

% Note that keywords are not normally used for peerreview papers.
%\begin{IEEEkeywords}
%Cooperative diversity, decode and forward, piecewise linear
%\end{IEEEkeywords}



% For peer review papers, you can put extra information on the cover
% page as needed:
% \ifCLASSOPTIONpeerreview
% \begin{center} \bfseries EDICS Category: 3-BBND \end{center}
% \fi
%
% For peerreview papers, this IEEEtran command inserts a page break and
% creates the second title. It will be ignored for other modes.
%\IEEEpeerreviewmaketitle




\item A card is drawn from a deck of 52 cards. Find the probability of getting a king or a heart or a red card.\\
\solution
%\begin{table}[H]
	\centering
\begin{tabular}{|c|c|c|}
\hline
Random variable &Value &Definition\\ \hline
\multirow{3}{*}{X} &0 &Slips of Rs 1\\
&1 &Slips of Rs 5\\
&2 &Slips of Rs 13\\ \hline
\multirow{2}{*}{Y} &0 &Box A\\
&1 &Box B\\\hline
\end{tabular}
\caption{}
\label{tab:Distribution}
\end{table}
See \tabref{tab:Distribution}.
\begin{align}
p_{Y}\brak{k}= \begin{cases} 
      \frac{1}{3} & {k=0} \\
      \frac{2}{3 }& {k=1} 
   \end{cases}
   \\
p_{Y|X}\brak{0|0} = \frac{19}{25}\, 
p_{Y|X}\brak{0|1} = \frac{6}{25}\,
p_{Y|X}\brak{1|0} = \frac{45}{50}\,
p_{Y|X}\brak{1|2} = \frac{5}{50}
\end{align}
The desired probability is the probability that a slip drawn at random is marked other than Rs 1,
\begin{align}
&=1-p_X\brak{0}\\
&= p_X(1) + p_X(2)
\end{align}
Using Bayes theorem,
\begin{align}
&= p_Y\brak{0} \times \pr{Y=0 | X=1} + p_Y\brak{1} \times \pr{Y=1|X=2}\\
&=\frac{1}{3} \times \frac{6}{25} + \frac{2}{3} \times \frac{5}{50}\\
&=\frac{11}{75}
\end{align}

\newpage

%\tableofcontents

\bigskip

\renewcommand{\thefigure}{\theenumi}
\renewcommand{\thetable}{\theenumi}
%\renewcommand{\theequation}{\theenumi}

%\begin{abstract}
%%\boldmath
%In this letter, an algorithm for evaluating the exact analytical bit error rate  (BER)  for the piecewise linear (PL) combiner for  multiple relays is presented. Previous results were available only for upto three relays. The algorithm is unique in the sense that  the actual mathematical expressions, that are prohibitively large, need not be explicitly obtained. The diversity gain due to multiple relays is shown through plots of the analytical BER, well supported by simulations. 
%
%\end{abstract}
% IEEEtran.cls defaults to using nonbold math in the Abstract.
% This preserves the distinction between vectors and scalars. However,
% if the journal you are submitting to favors bold math in the abstract,
% then you can use LaTeX's standard command \boldmath at the very start
% of the abstract to achieve this. Many IEEE journals frown on math
% in the abstract anyway.

% Note that keywords are not normally used for peerreview papers.
%\begin{IEEEkeywords}
%Cooperative diversity, decode and forward, piecewise linear
%\end{IEEEkeywords}



% For peer review papers, you can put extra information on the cover
% page as needed:
% \ifCLASSOPTIONpeerreview
% \begin{center} \bfseries EDICS Category: 3-BBND \end{center}
% \fi
%
% For peerreview papers, this IEEEtran command inserts a page break and
% creates the second title. It will be ignored for other modes.
%\IEEEpeerreviewmaketitle




\item The probability that a student will pass his examination is 0.73, the probability of
the student getting a compartment is 0.13, and the probability that the student will
either pass or get compartment is 0.96. State True or False.\\
\solution
%\begin{table}[H]
	\centering
\begin{tabular}{|c|c|c|}
\hline
Random variable &Value &Definition\\ \hline
\multirow{3}{*}{X} &0 &Slips of Rs 1\\
&1 &Slips of Rs 5\\
&2 &Slips of Rs 13\\ \hline
\multirow{2}{*}{Y} &0 &Box A\\
&1 &Box B\\\hline
\end{tabular}
\caption{}
\label{tab:Distribution}
\end{table}
See \tabref{tab:Distribution}.
\begin{align}
p_{Y}\brak{k}= \begin{cases} 
      \frac{1}{3} & {k=0} \\
      \frac{2}{3 }& {k=1} 
   \end{cases}
   \\
p_{Y|X}\brak{0|0} = \frac{19}{25}\, 
p_{Y|X}\brak{0|1} = \frac{6}{25}\,
p_{Y|X}\brak{1|0} = \frac{45}{50}\,
p_{Y|X}\brak{1|2} = \frac{5}{50}
\end{align}
The desired probability is the probability that a slip drawn at random is marked other than Rs 1,
\begin{align}
&=1-p_X\brak{0}\\
&= p_X(1) + p_X(2)
\end{align}
Using Bayes theorem,
\begin{align}
&= p_Y\brak{0} \times \pr{Y=0 | X=1} + p_Y\brak{1} \times \pr{Y=1|X=2}\\
&=\frac{1}{3} \times \frac{6}{25} + \frac{2}{3} \times \frac{5}{50}\\
&=\frac{11}{75}
\end{align}

\newpage

%\tableofcontents

\bigskip

\renewcommand{\thefigure}{\theenumi}
\renewcommand{\thetable}{\theenumi}
%\renewcommand{\theequation}{\theenumi}

%\begin{abstract}
%%\boldmath
%In this letter, an algorithm for evaluating the exact analytical bit error rate  (BER)  for the piecewise linear (PL) combiner for  multiple relays is presented. Previous results were available only for upto three relays. The algorithm is unique in the sense that  the actual mathematical expressions, that are prohibitively large, need not be explicitly obtained. The diversity gain due to multiple relays is shown through plots of the analytical BER, well supported by simulations. 
%
%\end{abstract}
% IEEEtran.cls defaults to using nonbold math in the Abstract.
% This preserves the distinction between vectors and scalars. However,
% if the journal you are submitting to favors bold math in the abstract,
% then you can use LaTeX's standard command \boldmath at the very start
% of the abstract to achieve this. Many IEEE journals frown on math
% in the abstract anyway.

% Note that keywords are not normally used for peerreview papers.
%\begin{IEEEkeywords}
%Cooperative diversity, decode and forward, piecewise linear
%\end{IEEEkeywords}



% For peer review papers, you can put extra information on the cover
% page as needed:
% \ifCLASSOPTIONpeerreview
% \begin{center} \bfseries EDICS Category: 3-BBND \end{center}
% \fi
%
% For peerreview papers, this IEEEtran command inserts a page break and
% creates the second title. It will be ignored for other modes.
%\IEEEpeerreviewmaketitle




\item A card is selected from a pack of 52 cards\\
\begin{enumerate}[label=(\alph*)]
\item How many points are there in the sample space?
\item Calculate the probability that the cards is an ace of spades.
\item Calculate the probability that the card is (i) an ace (ii)black card.\\
\end{enumerate}
%\input{ncert/11/16/3/4_1/Prob_4.tex}
\item In a non-leap year, the probability of having 53 tuesdays or 53 wednesdays is\\
\solution
%A non-leap year has a total of 365 days, and a week has 7 days.\\
So it can be expressed as 
\begin{align}
365\text{days} &=52\times 7+1 \text{day}
\end{align}
$\implies$ 52 tuesdays or wednesdays\\
Random variable X denotes the days of a week
\begin{align}
p_X\brak{k}&=\frac{1}{7}; \quad \brak{1<k<7}
\end{align}
So the probability of extra day being tuesday or wednesday is
\begin{align}
p_X\brak{3}+p_X\brak{4}&=\frac{1}{7}+\frac{1}{7}=\frac{2}{7}
\end{align}



\item There are 1000 sealed envelopes in a box, 10 of them contain a cash prize of
Rs 100 each, 100 of them contain a cash prize of Rs 50 each and 200 of them
contain a cash prize of Rs 10 each and rest do not contain any cash prize. If they
are well shuffled and an envelope is picked up out, what is the probability that it
contains no cash prize?\\
\solution
%\begin{table}[H]
	\centering
\begin{tabular}{|c|c|c|}
\hline
Random variable &Value &Definition\\ \hline
\multirow{3}{*}{X} &0 &Slips of Rs 1\\
&1 &Slips of Rs 5\\
&2 &Slips of Rs 13\\ \hline
\multirow{2}{*}{Y} &0 &Box A\\
&1 &Box B\\\hline
\end{tabular}
\caption{}
\label{tab:Distribution}
\end{table}
See \tabref{tab:Distribution}.
\begin{align}
p_{Y}\brak{k}= \begin{cases} 
      \frac{1}{3} & {k=0} \\
      \frac{2}{3 }& {k=1} 
   \end{cases}
   \\
p_{Y|X}\brak{0|0} = \frac{19}{25}\, 
p_{Y|X}\brak{0|1} = \frac{6}{25}\,
p_{Y|X}\brak{1|0} = \frac{45}{50}\,
p_{Y|X}\brak{1|2} = \frac{5}{50}
\end{align}
The desired probability is the probability that a slip drawn at random is marked other than Rs 1,
\begin{align}
&=1-p_X\brak{0}\\
&= p_X(1) + p_X(2)
\end{align}
Using Bayes theorem,
\begin{align}
&= p_Y\brak{0} \times \pr{Y=0 | X=1} + p_Y\brak{1} \times \pr{Y=1|X=2}\\
&=\frac{1}{3} \times \frac{6}{25} + \frac{2}{3} \times \frac{5}{50}\\
&=\frac{11}{75}
\end{align}

\newpage

%\tableofcontents

\bigskip

\renewcommand{\thefigure}{\theenumi}
\renewcommand{\thetable}{\theenumi}
%\renewcommand{\theequation}{\theenumi}

%\begin{abstract}
%%\boldmath
%In this letter, an algorithm for evaluating the exact analytical bit error rate  (BER)  for the piecewise linear (PL) combiner for  multiple relays is presented. Previous results were available only for upto three relays. The algorithm is unique in the sense that  the actual mathematical expressions, that are prohibitively large, need not be explicitly obtained. The diversity gain due to multiple relays is shown through plots of the analytical BER, well supported by simulations. 
%
%\end{abstract}
% IEEEtran.cls defaults to using nonbold math in the Abstract.
% This preserves the distinction between vectors and scalars. However,
% if the journal you are submitting to favors bold math in the abstract,
% then you can use LaTeX's standard command \boldmath at the very start
% of the abstract to achieve this. Many IEEE journals frown on math
% in the abstract anyway.

% Note that keywords are not normally used for peerreview papers.
%\begin{IEEEkeywords}
%Cooperative diversity, decode and forward, piecewise linear
%\end{IEEEkeywords}



% For peer review papers, you can put extra information on the cover
% page as needed:
% \ifCLASSOPTIONpeerreview
% \begin{center} \bfseries EDICS Category: 3-BBND \end{center}
% \fi
%
% For peerreview papers, this IEEEtran command inserts a page break and
% creates the second title. It will be ignored for other modes.
%\IEEEpeerreviewmaketitle




\item 
A die is thrown and a card is selected at random from a deck of 52 playing cards. The probability of getting an even number on the die and a spade card.\\
\solution
%\begin{table}[H]
	\centering
\begin{tabular}{|c|c|c|}
\hline
Random variable &Value &Definition\\ \hline
\multirow{3}{*}{X} &0 &Slips of Rs 1\\
&1 &Slips of Rs 5\\
&2 &Slips of Rs 13\\ \hline
\multirow{2}{*}{Y} &0 &Box A\\
&1 &Box B\\\hline
\end{tabular}
\caption{}
\label{tab:Distribution}
\end{table}
See \tabref{tab:Distribution}.
\begin{align}
p_{Y}\brak{k}= \begin{cases} 
      \frac{1}{3} & {k=0} \\
      \frac{2}{3 }& {k=1} 
   \end{cases}
   \\
p_{Y|X}\brak{0|0} = \frac{19}{25}\, 
p_{Y|X}\brak{0|1} = \frac{6}{25}\,
p_{Y|X}\brak{1|0} = \frac{45}{50}\,
p_{Y|X}\brak{1|2} = \frac{5}{50}
\end{align}
The desired probability is the probability that a slip drawn at random is marked other than Rs 1,
\begin{align}
&=1-p_X\brak{0}\\
&= p_X(1) + p_X(2)
\end{align}
Using Bayes theorem,
\begin{align}
&= p_Y\brak{0} \times \pr{Y=0 | X=1} + p_Y\brak{1} \times \pr{Y=1|X=2}\\
&=\frac{1}{3} \times \frac{6}{25} + \frac{2}{3} \times \frac{5}{50}\\
&=\frac{11}{75}
\end{align}

\newpage

%\tableofcontents

\bigskip

\renewcommand{\thefigure}{\theenumi}
\renewcommand{\thetable}{\theenumi}
%\renewcommand{\theequation}{\theenumi}

%\begin{abstract}
%%\boldmath
%In this letter, an algorithm for evaluating the exact analytical bit error rate  (BER)  for the piecewise linear (PL) combiner for  multiple relays is presented. Previous results were available only for upto three relays. The algorithm is unique in the sense that  the actual mathematical expressions, that are prohibitively large, need not be explicitly obtained. The diversity gain due to multiple relays is shown through plots of the analytical BER, well supported by simulations. 
%
%\end{abstract}
% IEEEtran.cls defaults to using nonbold math in the Abstract.
% This preserves the distinction between vectors and scalars. However,
% if the journal you are submitting to favors bold math in the abstract,
% then you can use LaTeX's standard command \boldmath at the very start
% of the abstract to achieve this. Many IEEE journals frown on math
% in the abstract anyway.

% Note that keywords are not normally used for peerreview papers.
%\begin{IEEEkeywords}
%Cooperative diversity, decode and forward, piecewise linear
%\end{IEEEkeywords}



% For peer review papers, you can put extra information on the cover
% page as needed:
% \ifCLASSOPTIONpeerreview
% \begin{center} \bfseries EDICS Category: 3-BBND \end{center}
% \fi
%
% For peerreview papers, this IEEEtran command inserts a page break and
% creates the second title. It will be ignored for other modes.
%\IEEEpeerreviewmaketitle




\item
If 4-digit numbers greater than 5,000 are randomly formed from the digits 0, 1, 3, 5, and 7, what is the probability of forming a number divisible by 5 when:
\begin{enumerate}
    \item The digits are repeated?
    \item The repetition of digits is not allowed?
\end{enumerate}
\solution
%\begin{table}[H]
	\centering
\begin{tabular}{|c|c|c|}
\hline
Random variable &Value &Definition\\ \hline
\multirow{3}{*}{X} &0 &Slips of Rs 1\\
&1 &Slips of Rs 5\\
&2 &Slips of Rs 13\\ \hline
\multirow{2}{*}{Y} &0 &Box A\\
&1 &Box B\\\hline
\end{tabular}
\caption{}
\label{tab:Distribution}
\end{table}
See \tabref{tab:Distribution}.
\begin{align}
p_{Y}\brak{k}= \begin{cases} 
      \frac{1}{3} & {k=0} \\
      \frac{2}{3 }& {k=1} 
   \end{cases}
   \\
p_{Y|X}\brak{0|0} = \frac{19}{25}\, 
p_{Y|X}\brak{0|1} = \frac{6}{25}\,
p_{Y|X}\brak{1|0} = \frac{45}{50}\,
p_{Y|X}\brak{1|2} = \frac{5}{50}
\end{align}
The desired probability is the probability that a slip drawn at random is marked other than Rs 1,
\begin{align}
&=1-p_X\brak{0}\\
&= p_X(1) + p_X(2)
\end{align}
Using Bayes theorem,
\begin{align}
&= p_Y\brak{0} \times \pr{Y=0 | X=1} + p_Y\brak{1} \times \pr{Y=1|X=2}\\
&=\frac{1}{3} \times \frac{6}{25} + \frac{2}{3} \times \frac{5}{50}\\
&=\frac{11}{75}
\end{align}

\newpage

%\tableofcontents

\bigskip

\renewcommand{\thefigure}{\theenumi}
\renewcommand{\thetable}{\theenumi}
%\renewcommand{\theequation}{\theenumi}

%\begin{abstract}
%%\boldmath
%In this letter, an algorithm for evaluating the exact analytical bit error rate  (BER)  for the piecewise linear (PL) combiner for  multiple relays is presented. Previous results were available only for upto three relays. The algorithm is unique in the sense that  the actual mathematical expressions, that are prohibitively large, need not be explicitly obtained. The diversity gain due to multiple relays is shown through plots of the analytical BER, well supported by simulations. 
%
%\end{abstract}
% IEEEtran.cls defaults to using nonbold math in the Abstract.
% This preserves the distinction between vectors and scalars. However,
% if the journal you are submitting to favors bold math in the abstract,
% then you can use LaTeX's standard command \boldmath at the very start
% of the abstract to achieve this. Many IEEE journals frown on math
% in the abstract anyway.

% Note that keywords are not normally used for peerreview papers.
%\begin{IEEEkeywords}
%Cooperative diversity, decode and forward, piecewise linear
%\end{IEEEkeywords}



% For peer review papers, you can put extra information on the cover
% page as needed:
% \ifCLASSOPTIONpeerreview
% \begin{center} \bfseries EDICS Category: 3-BBND \end{center}
% \fi
%
% For peerreview papers, this IEEEtran command inserts a page break and
% creates the second title. It will be ignored for other modes.
%\IEEEpeerreviewmaketitle




\item Consider the probability space $\brak{\Omega, \mathcal{G}, P}$ where $\Omega = [0,2]$ and $\mathcal{G} = \cbrak{\phi, \Omega, [0,1], (1,2]}$. Let $X$ and $Y$ be two functions on $\Omega$ defined as
\begin{align*}
    X(\omega) = 
    \begin{cases}
        1 & \text{if }\omega \in [0, 1]\\
        2 & \text{if }\omega \in (1, 2]
    \end{cases}
\end{align*}
and
\begin{align*}
    Y(\omega) = 
    \begin{cases}
        2 & \text{if }\omega \in [0, 1.5]\\
        3 & \text{if }\omega \in (1.5, 2].
    \end{cases}
\end{align*}
Then which one of the following statements is true?
\begin{enumerate}
    \item [(A)] $X$ is a random variable with respect to $\mathcal{G}$, but $Y$ is not a random variable with respect to $\mathcal{G}$.
    \item [(B)] $Y$ is a random variable with respect to $\mathcal{G}$, but $X$ is not a random variable with respect to $\mathcal{G}$.
    \item [(C)] Neither $X$ nor $Y$ is a random variable with respect to $\mathcal{G}$.
    \item [(D)] Both $X$ and $Y$ are random variables with respect to $\mathcal{G}$.
\end{enumerate} \hfill (GATE ST 2023)\\
\solution
%\begin{table}[H]
	\centering
\begin{tabular}{|c|c|c|}
\hline
Random variable &Value &Definition\\ \hline
\multirow{3}{*}{X} &0 &Slips of Rs 1\\
&1 &Slips of Rs 5\\
&2 &Slips of Rs 13\\ \hline
\multirow{2}{*}{Y} &0 &Box A\\
&1 &Box B\\\hline
\end{tabular}
\caption{}
\label{tab:Distribution}
\end{table}
See \tabref{tab:Distribution}.
\begin{align}
p_{Y}\brak{k}= \begin{cases} 
      \frac{1}{3} & {k=0} \\
      \frac{2}{3 }& {k=1} 
   \end{cases}
   \\
p_{Y|X}\brak{0|0} = \frac{19}{25}\, 
p_{Y|X}\brak{0|1} = \frac{6}{25}\,
p_{Y|X}\brak{1|0} = \frac{45}{50}\,
p_{Y|X}\brak{1|2} = \frac{5}{50}
\end{align}
The desired probability is the probability that a slip drawn at random is marked other than Rs 1,
\begin{align}
&=1-p_X\brak{0}\\
&= p_X(1) + p_X(2)
\end{align}
Using Bayes theorem,
\begin{align}
&= p_Y\brak{0} \times \pr{Y=0 | X=1} + p_Y\brak{1} \times \pr{Y=1|X=2}\\
&=\frac{1}{3} \times \frac{6}{25} + \frac{2}{3} \times \frac{5}{50}\\
&=\frac{11}{75}
\end{align}

\newpage

%\tableofcontents

\bigskip

\renewcommand{\thefigure}{\theenumi}
\renewcommand{\thetable}{\theenumi}
%\renewcommand{\theequation}{\theenumi}

%\begin{abstract}
%%\boldmath
%In this letter, an algorithm for evaluating the exact analytical bit error rate  (BER)  for the piecewise linear (PL) combiner for  multiple relays is presented. Previous results were available only for upto three relays. The algorithm is unique in the sense that  the actual mathematical expressions, that are prohibitively large, need not be explicitly obtained. The diversity gain due to multiple relays is shown through plots of the analytical BER, well supported by simulations. 
%
%\end{abstract}
% IEEEtran.cls defaults to using nonbold math in the Abstract.
% This preserves the distinction between vectors and scalars. However,
% if the journal you are submitting to favors bold math in the abstract,
% then you can use LaTeX's standard command \boldmath at the very start
% of the abstract to achieve this. Many IEEE journals frown on math
% in the abstract anyway.

% Note that keywords are not normally used for peerreview papers.
%\begin{IEEEkeywords}
%Cooperative diversity, decode and forward, piecewise linear
%\end{IEEEkeywords}



% For peer review papers, you can put extra information on the cover
% page as needed:
% \ifCLASSOPTIONpeerreview
% \begin{center} \bfseries EDICS Category: 3-BBND \end{center}
% \fi
%
% For peerreview papers, this IEEEtran command inserts a page break and
% creates the second title. It will be ignored for other modes.
%\IEEEpeerreviewmaketitle




	\item  A die is loaded in such a way that each odd number is twice as likely to occur as
each even number. Find $P(G)$, where $G$ is the event that a number greater than
3 occurs on a single roll of the die.
\\
\solution
		%\begin{table}[H]
	\centering
\begin{tabular}{|c|c|c|}
\hline
Random variable &Value &Definition\\ \hline
\multirow{3}{*}{X} &0 &Slips of Rs 1\\
&1 &Slips of Rs 5\\
&2 &Slips of Rs 13\\ \hline
\multirow{2}{*}{Y} &0 &Box A\\
&1 &Box B\\\hline
\end{tabular}
\caption{}
\label{tab:Distribution}
\end{table}
See \tabref{tab:Distribution}.
\begin{align}
p_{Y}\brak{k}= \begin{cases} 
      \frac{1}{3} & {k=0} \\
      \frac{2}{3 }& {k=1} 
   \end{cases}
   \\
p_{Y|X}\brak{0|0} = \frac{19}{25}\, 
p_{Y|X}\brak{0|1} = \frac{6}{25}\,
p_{Y|X}\brak{1|0} = \frac{45}{50}\,
p_{Y|X}\brak{1|2} = \frac{5}{50}
\end{align}
The desired probability is the probability that a slip drawn at random is marked other than Rs 1,
\begin{align}
&=1-p_X\brak{0}\\
&= p_X(1) + p_X(2)
\end{align}
Using Bayes theorem,
\begin{align}
&= p_Y\brak{0} \times \pr{Y=0 | X=1} + p_Y\brak{1} \times \pr{Y=1|X=2}\\
&=\frac{1}{3} \times \frac{6}{25} + \frac{2}{3} \times \frac{5}{50}\\
&=\frac{11}{75}
\end{align}

\newpage

%\tableofcontents

\bigskip

\renewcommand{\thefigure}{\theenumi}
\renewcommand{\thetable}{\theenumi}
%\renewcommand{\theequation}{\theenumi}

%\begin{abstract}
%%\boldmath
%In this letter, an algorithm for evaluating the exact analytical bit error rate  (BER)  for the piecewise linear (PL) combiner for  multiple relays is presented. Previous results were available only for upto three relays. The algorithm is unique in the sense that  the actual mathematical expressions, that are prohibitively large, need not be explicitly obtained. The diversity gain due to multiple relays is shown through plots of the analytical BER, well supported by simulations. 
%
%\end{abstract}
% IEEEtran.cls defaults to using nonbold math in the Abstract.
% This preserves the distinction between vectors and scalars. However,
% if the journal you are submitting to favors bold math in the abstract,
% then you can use LaTeX's standard command \boldmath at the very start
% of the abstract to achieve this. Many IEEE journals frown on math
% in the abstract anyway.

% Note that keywords are not normally used for peerreview papers.
%\begin{IEEEkeywords}
%Cooperative diversity, decode and forward, piecewise linear
%\end{IEEEkeywords}



% For peer review papers, you can put extra information on the cover
% page as needed:
% \ifCLASSOPTIONpeerreview
% \begin{center} \bfseries EDICS Category: 3-BBND \end{center}
% \fi
%
% For peerreview papers, this IEEEtran command inserts a page break and
% creates the second title. It will be ignored for other modes.
%\IEEEpeerreviewmaketitle




	\item All the jacks, queens and kings are removed from a deck of 52 playing cards. The remaining cards are well shuffled and then one card is drawn at random. Giving ace a value 1 similar value for other cards, find the probability that the card has a value 
		\begin{enumerate}
			\item 7
			\item greater than 7
			\item less than 7
		\end{enumerate}
		%Number of cards left after removing all jacks, queens and kings 
\begin{align}
N	= 52 - 4\times 3
	= 40
\end{align}
%\begin{table}[H]
%\def\arraystretch{1.2}
%\begin{tabular}{|c|c|c|}
%\hline
%	\textbf{Parameter} &\textbf{Value} &\textbf{Description}\\ \hline
%	$X$ &1-10 &Represents the value of the card picked \\ \hline
%\end{tabular}
%\end{table}
Let $1 \le X \le 10$ be the value of the card picked.  Then,
\begin{align}
	p_X(k) &= \Pr(X=k)\ \forall\ 1 \leq k \leq 10\\
	&= \frac{4\times 1}{40}\\
	&= \frac{1}{10}\\
	\therefore p_X(k) &= 
	\begin{cases}
		\frac{1}{10} & 1 \leq k \leq 10\\
		0 & \text{otherwise}
	\end{cases}
\end{align}
and
\begin{align}
	F_{X}(k) &= \sum_{m=0}^{k}p_{X}(m) \quad 1 \leq k \leq 10\\
	&= \frac{k}{10}\\
	\therefore F_{X}(k) &= 
	\begin{cases}
		0 & k \leq 0\\
		\frac{k}{10} & 1\leq k \leq 10\\
		1 & k > 10 
	\end{cases}
\end{align}
\begin{enumerate}
	\item Probability that card has value equal to 7 is
		\begin{align}
			 p_{X}(7)
			= \frac{1}{10}
		\end{align}
	\item Probability that card has value greater than 7 is
		\begin{align}
			1 - F_X(7)
			&= 1 - \frac{7}{10}
			\\
			&= \frac{3}{10}
		\end{align}
	\item Probability that card has value less than 7 is
		\begin{align}
			 F_{X}(6)
			=\frac{6}{10}
		\end{align}
\end{enumerate}

  \item A Lot consists of 48 mobile phones of which 42 are good, 3 have only minor defects and 3 have major defects.Varnika will buy a phone if it is good but the trader will only buy a mobile if it has no major defects. One phone is selected at random from the lot. What is the probability that it is
\begin{enumerate}
	\item acceptable to Varnika?
            \item acceptable to the trader?
\end{enumerate}
\solution
	%\begin{table}[H]
	\centering
\begin{tabular}{|c|c|c|}
\hline
Random variable &Value &Definition\\ \hline
\multirow{3}{*}{X} &0 &Slips of Rs 1\\
&1 &Slips of Rs 5\\
&2 &Slips of Rs 13\\ \hline
\multirow{2}{*}{Y} &0 &Box A\\
&1 &Box B\\\hline
\end{tabular}
\caption{}
\label{tab:Distribution}
\end{table}
See \tabref{tab:Distribution}.
\begin{align}
p_{Y}\brak{k}= \begin{cases} 
      \frac{1}{3} & {k=0} \\
      \frac{2}{3 }& {k=1} 
   \end{cases}
   \\
p_{Y|X}\brak{0|0} = \frac{19}{25}\, 
p_{Y|X}\brak{0|1} = \frac{6}{25}\,
p_{Y|X}\brak{1|0} = \frac{45}{50}\,
p_{Y|X}\brak{1|2} = \frac{5}{50}
\end{align}
The desired probability is the probability that a slip drawn at random is marked other than Rs 1,
\begin{align}
&=1-p_X\brak{0}\\
&= p_X(1) + p_X(2)
\end{align}
Using Bayes theorem,
\begin{align}
&= p_Y\brak{0} \times \pr{Y=0 | X=1} + p_Y\brak{1} \times \pr{Y=1|X=2}\\
&=\frac{1}{3} \times \frac{6}{25} + \frac{2}{3} \times \frac{5}{50}\\
&=\frac{11}{75}
\end{align}

\newpage

%\tableofcontents

\bigskip

\renewcommand{\thefigure}{\theenumi}
\renewcommand{\thetable}{\theenumi}
%\renewcommand{\theequation}{\theenumi}

%\begin{abstract}
%%\boldmath
%In this letter, an algorithm for evaluating the exact analytical bit error rate  (BER)  for the piecewise linear (PL) combiner for  multiple relays is presented. Previous results were available only for upto three relays. The algorithm is unique in the sense that  the actual mathematical expressions, that are prohibitively large, need not be explicitly obtained. The diversity gain due to multiple relays is shown through plots of the analytical BER, well supported by simulations. 
%
%\end{abstract}
% IEEEtran.cls defaults to using nonbold math in the Abstract.
% This preserves the distinction between vectors and scalars. However,
% if the journal you are submitting to favors bold math in the abstract,
% then you can use LaTeX's standard command \boldmath at the very start
% of the abstract to achieve this. Many IEEE journals frown on math
% in the abstract anyway.

% Note that keywords are not normally used for peerreview papers.
%\begin{IEEEkeywords}
%Cooperative diversity, decode and forward, piecewise linear
%\end{IEEEkeywords}



% For peer review papers, you can put extra information on the cover
% page as needed:
% \ifCLASSOPTIONpeerreview
% \begin{center} \bfseries EDICS Category: 3-BBND \end{center}
% \fi
%
% For peerreview papers, this IEEEtran command inserts a page break and
% creates the second title. It will be ignored for other modes.
%\IEEEpeerreviewmaketitle




 \item A student says that if you throw a die, it will show up 1 or not 1. Therefore, the probability of getting 1 and the probability of getting 'not 1' each is equal to $\frac{1}{2}$. Is this correct? Give reasons.\\
 \solution
        %\begin{table}[H]
	\centering
\begin{tabular}{|c|c|c|}
\hline
Random variable &Value &Definition\\ \hline
\multirow{3}{*}{X} &0 &Slips of Rs 1\\
&1 &Slips of Rs 5\\
&2 &Slips of Rs 13\\ \hline
\multirow{2}{*}{Y} &0 &Box A\\
&1 &Box B\\\hline
\end{tabular}
\caption{}
\label{tab:Distribution}
\end{table}
See \tabref{tab:Distribution}.
\begin{align}
p_{Y}\brak{k}= \begin{cases} 
      \frac{1}{3} & {k=0} \\
      \frac{2}{3 }& {k=1} 
   \end{cases}
   \\
p_{Y|X}\brak{0|0} = \frac{19}{25}\, 
p_{Y|X}\brak{0|1} = \frac{6}{25}\,
p_{Y|X}\brak{1|0} = \frac{45}{50}\,
p_{Y|X}\brak{1|2} = \frac{5}{50}
\end{align}
The desired probability is the probability that a slip drawn at random is marked other than Rs 1,
\begin{align}
&=1-p_X\brak{0}\\
&= p_X(1) + p_X(2)
\end{align}
Using Bayes theorem,
\begin{align}
&= p_Y\brak{0} \times \pr{Y=0 | X=1} + p_Y\brak{1} \times \pr{Y=1|X=2}\\
&=\frac{1}{3} \times \frac{6}{25} + \frac{2}{3} \times \frac{5}{50}\\
&=\frac{11}{75}
\end{align}

\newpage

%\tableofcontents

\bigskip

\renewcommand{\thefigure}{\theenumi}
\renewcommand{\thetable}{\theenumi}
%\renewcommand{\theequation}{\theenumi}

%\begin{abstract}
%%\boldmath
%In this letter, an algorithm for evaluating the exact analytical bit error rate  (BER)  for the piecewise linear (PL) combiner for  multiple relays is presented. Previous results were available only for upto three relays. The algorithm is unique in the sense that  the actual mathematical expressions, that are prohibitively large, need not be explicitly obtained. The diversity gain due to multiple relays is shown through plots of the analytical BER, well supported by simulations. 
%
%\end{abstract}
% IEEEtran.cls defaults to using nonbold math in the Abstract.
% This preserves the distinction between vectors and scalars. However,
% if the journal you are submitting to favors bold math in the abstract,
% then you can use LaTeX's standard command \boldmath at the very start
% of the abstract to achieve this. Many IEEE journals frown on math
% in the abstract anyway.

% Note that keywords are not normally used for peerreview papers.
%\begin{IEEEkeywords}
%Cooperative diversity, decode and forward, piecewise linear
%\end{IEEEkeywords}



% For peer review papers, you can put extra information on the cover
% page as needed:
% \ifCLASSOPTIONpeerreview
% \begin{center} \bfseries EDICS Category: 3-BBND \end{center}
% \fi
%
% For peerreview papers, this IEEEtran command inserts a page break and
% creates the second title. It will be ignored for other modes.
%\IEEEpeerreviewmaketitle




   \item Four candidates A, B, C, D have ap-
plied for the assignment to coach a school cricket
team. If A is twice as likely to be selected as B, and
B and C are given about the same chance of being
selected, while C is twice as likely to be selected
as D, what are the probabilities that
\begin{enumerate}
\item C will be selected?
\item A will not be selected?
\end{enumerate}
	%\begin{table}[H]
	\centering
\begin{tabular}{|c|c|c|}
\hline
Random variable &Value &Definition\\ \hline
\multirow{3}{*}{X} &0 &Slips of Rs 1\\
&1 &Slips of Rs 5\\
&2 &Slips of Rs 13\\ \hline
\multirow{2}{*}{Y} &0 &Box A\\
&1 &Box B\\\hline
\end{tabular}
\caption{}
\label{tab:Distribution}
\end{table}
See \tabref{tab:Distribution}.
\begin{align}
p_{Y}\brak{k}= \begin{cases} 
      \frac{1}{3} & {k=0} \\
      \frac{2}{3 }& {k=1} 
   \end{cases}
   \\
p_{Y|X}\brak{0|0} = \frac{19}{25}\, 
p_{Y|X}\brak{0|1} = \frac{6}{25}\,
p_{Y|X}\brak{1|0} = \frac{45}{50}\,
p_{Y|X}\brak{1|2} = \frac{5}{50}
\end{align}
The desired probability is the probability that a slip drawn at random is marked other than Rs 1,
\begin{align}
&=1-p_X\brak{0}\\
&= p_X(1) + p_X(2)
\end{align}
Using Bayes theorem,
\begin{align}
&= p_Y\brak{0} \times \pr{Y=0 | X=1} + p_Y\brak{1} \times \pr{Y=1|X=2}\\
&=\frac{1}{3} \times \frac{6}{25} + \frac{2}{3} \times \frac{5}{50}\\
&=\frac{11}{75}
\end{align}

\newpage

%\tableofcontents

\bigskip

\renewcommand{\thefigure}{\theenumi}
\renewcommand{\thetable}{\theenumi}
%\renewcommand{\theequation}{\theenumi}

%\begin{abstract}
%%\boldmath
%In this letter, an algorithm for evaluating the exact analytical bit error rate  (BER)  for the piecewise linear (PL) combiner for  multiple relays is presented. Previous results were available only for upto three relays. The algorithm is unique in the sense that  the actual mathematical expressions, that are prohibitively large, need not be explicitly obtained. The diversity gain due to multiple relays is shown through plots of the analytical BER, well supported by simulations. 
%
%\end{abstract}
% IEEEtran.cls defaults to using nonbold math in the Abstract.
% This preserves the distinction between vectors and scalars. However,
% if the journal you are submitting to favors bold math in the abstract,
% then you can use LaTeX's standard command \boldmath at the very start
% of the abstract to achieve this. Many IEEE journals frown on math
% in the abstract anyway.

% Note that keywords are not normally used for peerreview papers.
%\begin{IEEEkeywords}
%Cooperative diversity, decode and forward, piecewise linear
%\end{IEEEkeywords}



% For peer review papers, you can put extra information on the cover
% page as needed:
% \ifCLASSOPTIONpeerreview
% \begin{center} \bfseries EDICS Category: 3-BBND \end{center}
% \fi
%
% For peerreview papers, this IEEEtran command inserts a page break and
% creates the second title. It will be ignored for other modes.
%\IEEEpeerreviewmaketitle




 \item A bag contain 24 balls of which $x$ balls are red, $2x$ are white and $3x$ are blue. A ball is selected at random, What is the probability that it is
\begin{enumerate}[label=\alph*)]
\item not red ?
\item white ?
\end{enumerate}
%\begin{table}[H]
	\centering
\begin{tabular}{|c|c|c|}
\hline
Random variable &Value &Definition\\ \hline
\multirow{3}{*}{X} &0 &Slips of Rs 1\\
&1 &Slips of Rs 5\\
&2 &Slips of Rs 13\\ \hline
\multirow{2}{*}{Y} &0 &Box A\\
&1 &Box B\\\hline
\end{tabular}
\caption{}
\label{tab:Distribution}
\end{table}
See \tabref{tab:Distribution}.
\begin{align}
p_{Y}\brak{k}= \begin{cases} 
      \frac{1}{3} & {k=0} \\
      \frac{2}{3 }& {k=1} 
   \end{cases}
   \\
p_{Y|X}\brak{0|0} = \frac{19}{25}\, 
p_{Y|X}\brak{0|1} = \frac{6}{25}\,
p_{Y|X}\brak{1|0} = \frac{45}{50}\,
p_{Y|X}\brak{1|2} = \frac{5}{50}
\end{align}
The desired probability is the probability that a slip drawn at random is marked other than Rs 1,
\begin{align}
&=1-p_X\brak{0}\\
&= p_X(1) + p_X(2)
\end{align}
Using Bayes theorem,
\begin{align}
&= p_Y\brak{0} \times \pr{Y=0 | X=1} + p_Y\brak{1} \times \pr{Y=1|X=2}\\
&=\frac{1}{3} \times \frac{6}{25} + \frac{2}{3} \times \frac{5}{50}\\
&=\frac{11}{75}
\end{align}

\newpage

%\tableofcontents

\bigskip

\renewcommand{\thefigure}{\theenumi}
\renewcommand{\thetable}{\theenumi}
%\renewcommand{\theequation}{\theenumi}

%\begin{abstract}
%%\boldmath
%In this letter, an algorithm for evaluating the exact analytical bit error rate  (BER)  for the piecewise linear (PL) combiner for  multiple relays is presented. Previous results were available only for upto three relays. The algorithm is unique in the sense that  the actual mathematical expressions, that are prohibitively large, need not be explicitly obtained. The diversity gain due to multiple relays is shown through plots of the analytical BER, well supported by simulations. 
%
%\end{abstract}
% IEEEtran.cls defaults to using nonbold math in the Abstract.
% This preserves the distinction between vectors and scalars. However,
% if the journal you are submitting to favors bold math in the abstract,
% then you can use LaTeX's standard command \boldmath at the very start
% of the abstract to achieve this. Many IEEE journals frown on math
% in the abstract anyway.

% Note that keywords are not normally used for peerreview papers.
%\begin{IEEEkeywords}
%Cooperative diversity, decode and forward, piecewise linear
%\end{IEEEkeywords}



% For peer review papers, you can put extra information on the cover
% page as needed:
% \ifCLASSOPTIONpeerreview
% \begin{center} \bfseries EDICS Category: 3-BBND \end{center}
% \fi
%
% For peerreview papers, this IEEEtran command inserts a page break and
% creates the second title. It will be ignored for other modes.
%\IEEEpeerreviewmaketitle




If the letters of the word ASSASSINATION are arranged at random. Find the Probability that
\begin{enumerate}[label=(\alph*)]
\item Four $S's$ come consecutively in the word
\item Two  $I's$ and two $N's$ come together
\item All $A's$ are not coming together
\item No two $A's$ are coming together
\end{enumerate}
%\begin{table}[H]
	\centering
\begin{tabular}{|c|c|c|}
\hline
Random variable &Value &Definition\\ \hline
\multirow{3}{*}{X} &0 &Slips of Rs 1\\
&1 &Slips of Rs 5\\
&2 &Slips of Rs 13\\ \hline
\multirow{2}{*}{Y} &0 &Box A\\
&1 &Box B\\\hline
\end{tabular}
\caption{}
\label{tab:Distribution}
\end{table}
See \tabref{tab:Distribution}.
\begin{align}
p_{Y}\brak{k}= \begin{cases} 
      \frac{1}{3} & {k=0} \\
      \frac{2}{3 }& {k=1} 
   \end{cases}
   \\
p_{Y|X}\brak{0|0} = \frac{19}{25}\, 
p_{Y|X}\brak{0|1} = \frac{6}{25}\,
p_{Y|X}\brak{1|0} = \frac{45}{50}\,
p_{Y|X}\brak{1|2} = \frac{5}{50}
\end{align}
The desired probability is the probability that a slip drawn at random is marked other than Rs 1,
\begin{align}
&=1-p_X\brak{0}\\
&= p_X(1) + p_X(2)
\end{align}
Using Bayes theorem,
\begin{align}
&= p_Y\brak{0} \times \pr{Y=0 | X=1} + p_Y\brak{1} \times \pr{Y=1|X=2}\\
&=\frac{1}{3} \times \frac{6}{25} + \frac{2}{3} \times \frac{5}{50}\\
&=\frac{11}{75}
\end{align}

\newpage

%\tableofcontents

\bigskip

\renewcommand{\thefigure}{\theenumi}
\renewcommand{\thetable}{\theenumi}
%\renewcommand{\theequation}{\theenumi}

%\begin{abstract}
%%\boldmath
%In this letter, an algorithm for evaluating the exact analytical bit error rate  (BER)  for the piecewise linear (PL) combiner for  multiple relays is presented. Previous results were available only for upto three relays. The algorithm is unique in the sense that  the actual mathematical expressions, that are prohibitively large, need not be explicitly obtained. The diversity gain due to multiple relays is shown through plots of the analytical BER, well supported by simulations. 
%
%\end{abstract}
% IEEEtran.cls defaults to using nonbold math in the Abstract.
% This preserves the distinction between vectors and scalars. However,
% if the journal you are submitting to favors bold math in the abstract,
% then you can use LaTeX's standard command \boldmath at the very start
% of the abstract to achieve this. Many IEEE journals frown on math
% in the abstract anyway.

% Note that keywords are not normally used for peerreview papers.
%\begin{IEEEkeywords}
%Cooperative diversity, decode and forward, piecewise linear
%\end{IEEEkeywords}



% For peer review papers, you can put extra information on the cover
% page as needed:
% \ifCLASSOPTIONpeerreview
% \begin{center} \bfseries EDICS Category: 3-BBND \end{center}
% \fi
%
% For peerreview papers, this IEEEtran command inserts a page break and
% creates the second title. It will be ignored for other modes.
%\IEEEpeerreviewmaketitle




	\item One urn contains two black balls (labelled B1 and B2) and one white ball. A
	second urn contains one black ball and two white balls (labelled W1 and W2).
	Suppose the following experiment is performed. One of the two urns is chosen
	at random. Next a ball is randomly chosen from the urn. Then a second ball is
	chosen at random from the same urn without replacing the first ball.
	
	\begin{enumerate}
	\item What is the probability that two black balls are chosen?
	
	\item What is the probability that two balls of opposite colour are chosen?
	\end{enumerate}
	\solution
	%\begin{align}
    \label{eq:12.13.6.18.1}
	\because	\pr{A|B} &> \pr{A},\
\frac{\pr{AB}}{\pr{B}} > \pr{A}
\\
    \label{eq:12.13.6.18.2}
	\implies \pr{AB} &> \pr{A}\pr{B}
	\\
	\text{or, } \frac{\pr{AB}}{\pr{A}} &=\pr{B|A} > \pr{A}
\end{align}

\end{enumerate}

	\item A card is selected from a pack of 52 cards.
 \begin{enumerate}[label=(\alph*)] 
                 \item How many points are there in the sample space?
                 \item Calculate the probability that the card is an ace of spades.
                 \item Calculate the probability that the card is (i) an ace and (ii) black card.
 \end{enumerate}
\solution
		%\begin{table}[H]
	\centering
\begin{tabular}{|c|c|c|}
\hline
Random variable &Value &Definition\\ \hline
\multirow{3}{*}{X} &0 &Slips of Rs 1\\
&1 &Slips of Rs 5\\
&2 &Slips of Rs 13\\ \hline
\multirow{2}{*}{Y} &0 &Box A\\
&1 &Box B\\\hline
\end{tabular}
\caption{}
\label{tab:Distribution}
\end{table}
See \tabref{tab:Distribution}.
\begin{align}
p_{Y}\brak{k}= \begin{cases} 
      \frac{1}{3} & {k=0} \\
      \frac{2}{3 }& {k=1} 
   \end{cases}
   \\
p_{Y|X}\brak{0|0} = \frac{19}{25}\, 
p_{Y|X}\brak{0|1} = \frac{6}{25}\,
p_{Y|X}\brak{1|0} = \frac{45}{50}\,
p_{Y|X}\brak{1|2} = \frac{5}{50}
\end{align}
The desired probability is the probability that a slip drawn at random is marked other than Rs 1,
\begin{align}
&=1-p_X\brak{0}\\
&= p_X(1) + p_X(2)
\end{align}
Using Bayes theorem,
\begin{align}
&= p_Y\brak{0} \times \pr{Y=0 | X=1} + p_Y\brak{1} \times \pr{Y=1|X=2}\\
&=\frac{1}{3} \times \frac{6}{25} + \frac{2}{3} \times \frac{5}{50}\\
&=\frac{11}{75}
\end{align}

\newpage

%\tableofcontents

\bigskip

\renewcommand{\thefigure}{\theenumi}
\renewcommand{\thetable}{\theenumi}
%\renewcommand{\theequation}{\theenumi}

%\begin{abstract}
%%\boldmath
%In this letter, an algorithm for evaluating the exact analytical bit error rate  (BER)  for the piecewise linear (PL) combiner for  multiple relays is presented. Previous results were available only for upto three relays. The algorithm is unique in the sense that  the actual mathematical expressions, that are prohibitively large, need not be explicitly obtained. The diversity gain due to multiple relays is shown through plots of the analytical BER, well supported by simulations. 
%
%\end{abstract}
% IEEEtran.cls defaults to using nonbold math in the Abstract.
% This preserves the distinction between vectors and scalars. However,
% if the journal you are submitting to favors bold math in the abstract,
% then you can use LaTeX's standard command \boldmath at the very start
% of the abstract to achieve this. Many IEEE journals frown on math
% in the abstract anyway.

% Note that keywords are not normally used for peerreview papers.
%\begin{IEEEkeywords}
%Cooperative diversity, decode and forward, piecewise linear
%\end{IEEEkeywords}



% For peer review papers, you can put extra information on the cover
% page as needed:
% \ifCLASSOPTIONpeerreview
% \begin{center} \bfseries EDICS Category: 3-BBND \end{center}
% \fi
%
% For peerreview papers, this IEEEtran command inserts a page break and
% creates the second title. It will be ignored for other modes.
%\IEEEpeerreviewmaketitle




\item Four cards are drawn from a well-shuffled deck of 52 cards. What is the probability of obtaining 3 diamonds and one spade.
\\
\solution
		%\begin{enumerate}[label=\thesection.\arabic*,ref=\thesection.\theenumi]
	\item One card is drawn from a well-shuffled deck of 52 cards. Find the probability of getting
\begin{enumerate}
\item A king of red colour 
\item A face card 
\item A red face card
\item The jack of hearts
\item A spade
\item The queen of diamonds

\end{enumerate}
\solution
		%\begin{table}[H]
	\centering
\begin{tabular}{|c|c|c|}
\hline
Random variable &Value &Definition\\ \hline
\multirow{3}{*}{X} &0 &Slips of Rs 1\\
&1 &Slips of Rs 5\\
&2 &Slips of Rs 13\\ \hline
\multirow{2}{*}{Y} &0 &Box A\\
&1 &Box B\\\hline
\end{tabular}
\caption{}
\label{tab:Distribution}
\end{table}
See \tabref{tab:Distribution}.
\begin{align}
p_{Y}\brak{k}= \begin{cases} 
      \frac{1}{3} & {k=0} \\
      \frac{2}{3 }& {k=1} 
   \end{cases}
   \\
p_{Y|X}\brak{0|0} = \frac{19}{25}\, 
p_{Y|X}\brak{0|1} = \frac{6}{25}\,
p_{Y|X}\brak{1|0} = \frac{45}{50}\,
p_{Y|X}\brak{1|2} = \frac{5}{50}
\end{align}
The desired probability is the probability that a slip drawn at random is marked other than Rs 1,
\begin{align}
&=1-p_X\brak{0}\\
&= p_X(1) + p_X(2)
\end{align}
Using Bayes theorem,
\begin{align}
&= p_Y\brak{0} \times \pr{Y=0 | X=1} + p_Y\brak{1} \times \pr{Y=1|X=2}\\
&=\frac{1}{3} \times \frac{6}{25} + \frac{2}{3} \times \frac{5}{50}\\
&=\frac{11}{75}
\end{align}

\newpage

%\tableofcontents

\bigskip

\renewcommand{\thefigure}{\theenumi}
\renewcommand{\thetable}{\theenumi}
%\renewcommand{\theequation}{\theenumi}

%\begin{abstract}
%%\boldmath
%In this letter, an algorithm for evaluating the exact analytical bit error rate  (BER)  for the piecewise linear (PL) combiner for  multiple relays is presented. Previous results were available only for upto three relays. The algorithm is unique in the sense that  the actual mathematical expressions, that are prohibitively large, need not be explicitly obtained. The diversity gain due to multiple relays is shown through plots of the analytical BER, well supported by simulations. 
%
%\end{abstract}
% IEEEtran.cls defaults to using nonbold math in the Abstract.
% This preserves the distinction between vectors and scalars. However,
% if the journal you are submitting to favors bold math in the abstract,
% then you can use LaTeX's standard command \boldmath at the very start
% of the abstract to achieve this. Many IEEE journals frown on math
% in the abstract anyway.

% Note that keywords are not normally used for peerreview papers.
%\begin{IEEEkeywords}
%Cooperative diversity, decode and forward, piecewise linear
%\end{IEEEkeywords}



% For peer review papers, you can put extra information on the cover
% page as needed:
% \ifCLASSOPTIONpeerreview
% \begin{center} \bfseries EDICS Category: 3-BBND \end{center}
% \fi
%
% For peerreview papers, this IEEEtran command inserts a page break and
% creates the second title. It will be ignored for other modes.
%\IEEEpeerreviewmaketitle




	\item Five cards—the ten, jack, queen, king and ace of diamonds, are well-shuffled with their face downwards. One card is then picked up at random.
\begin{enumerate}
\item
What is the probability that the card is the queen? 
\item
If the queen is drawn and put aside, what is the probability that the second card picked up is (a) an ace? (b) a queen?\\
\end{enumerate}
\solution
		%\begin{enumerate}[label=\thesection.\arabic*,ref=\thesection.\theenumi]
	\item One card is drawn from a well-shuffled deck of 52 cards. Find the probability of getting
\begin{enumerate}
\item A king of red colour 
\item A face card 
\item A red face card
\item The jack of hearts
\item A spade
\item The queen of diamonds

\end{enumerate}
\solution
		%\input{ncert/10/15/1/14/main.tex}
	\item Five cards—the ten, jack, queen, king and ace of diamonds, are well-shuffled with their face downwards. One card is then picked up at random.
\begin{enumerate}
\item
What is the probability that the card is the queen? 
\item
If the queen is drawn and put aside, what is the probability that the second card picked up is (a) an ace? (b) a queen?\\
\end{enumerate}
\solution
		%\input{ncert/10/15/1/15/defs.tex}
	\item A bag contains $5$ red balls and some blue balls. If the probability of drawing a blue ball is double that if a red ball, determine the number of blue balls in the bag. 
		\\
\solution
		%\input{ncert/10/15/2/3/defs.tex}
	\item A card is selected from a pack of 52 cards.
 \begin{enumerate}[label=(\alph*)] 
                 \item How many points are there in the sample space?
                 \item Calculate the probability that the card is an ace of spades.
                 \item Calculate the probability that the card is (i) an ace and (ii) black card.
 \end{enumerate}
\solution
		%\input{ncert/11/16/3/4/main.tex}
\item Four cards are drawn from a well-shuffled deck of 52 cards. What is the probability of obtaining 3 diamonds and one spade.
\\
\solution
		%\input{ncert/11/16/4/2/defs.tex}
\item In a certain lottery 10,000 tickets are sold and ten equal prizes are awarded. What is the probability of not getting a prize if you buy (a) one ticket (b) two tickets (c) 10 tickets ?	
\\
\solution
		%\input{ncert/11/16/4/4/defs.tex}
		%
\item 
Out of 100 students, two sections of 40 and 60 are formed. If you and your friend are among the 100 students, what is the probability that
\begin{enumerate}
\item you both enter the same section?
\item you both enter the different sections?
\end{enumerate}
\solution
		%\input{ncert/11/16/4/5/defs.tex}
	\item 
The number lock of a suitcase has 4 wheels each labelled with ten digits i.e. from 0 to 9.The lock opens with a sequence of four digits with no repeats.What is the probability of a person getting the right sequence to open the suitcase.
\\
\solution
		%\input{ncert/11/16/4/10/defs.tex}
		%
\item 
Two cards are drawn at random and without replacement from a pack of 52 playing cards. Find the probability that both the cards are black.
\\
\solution
		%\input{ncert/12/13/2/2/defs.tex}
		\item A box of oranges is inspected by examining three randomly selected oranges drawn without replacement. If all the three oranges are good, the box is approved for sale, otherwise, it is rejected. Find the probability that a box containing 15 oranges out of which 12 are good and 3 are bad ones will be approved for sale.
		\label{ncert/12/13/2/3/defs.tex}
		\item Two balls are drawn at random with replacement from a box containing 10 black and 8 red balls. Find the probability that
		\label{ncert/12/13/2/12}
\begin{enumerate}
\item both balls are red.
\item first ball is black and second is red.
\item one of them is black and other is red.
\end{enumerate}

\item In a hostel, 60\% of the students read Hindi newspaper, 40\% read English newspaper and 20\% read both Hindi and English newspapers. A student is selected at random.
		\label{ncert/12/13/2/15}
\begin{enumerate}
\item Find the probability that she reads neither Hindi nor English newspapers.
\item If she reads Hindi newspaper, find the probability that she reads English newspaper.
\item If she reads English newspaper, find the probability that she reads Hindi newspaper.\\
\end{enumerate}
\item The probability of obtaining an even prime number on each die, when a pair of dice is rolled is 
\begin{enumerate}
    \item $0$ 
    
    \item $\frac{1}{3}$ 
    
    \item $\frac{1}{12}$ 
    
    \item $\frac{1}{36}$ 
\end{enumerate}
\solution
		%\input{ncert/12/13/2/17/defs.tex}
	\item A bag contains 4 red and 4 black balls, another bag contains 2 red and 6 black balls. One of the two bags is selected at random and a ball is drawn from the bag which is found to be red. Find the probability that the ball is drawn from the first bag.
\\
\solution
		%\input{ncert/12/13/3/2/main.tex}
  \item
  Cards with numbers 2 to 101 are placed in a box. A card is selected at random.Find the probability that the card has
\begin{enumerate}[label=(\roman*)]
	\item an even number 
	\item a square number
\end{enumerate}
\solution
%\input{exemplar/10/13/3/32/main.tex}
\item
The king, queen and jack of clubs are removed from a deck of 52 playing cards and then well shuffled. Now one card is drawn at random from the remaining cards.  Determine the probability that the card is
\begin{enumerate}[label=(\roman*)]
\item a club
\item 10 of hearts
\end{enumerate}
\solution
%\input{exemplar/10/13/3/29/main.tex}
\item A team of medical students doing their internship have to assist during surgeries
at a city hospital. The probabilities of surgeries rated as very complex, complex,
routine, simple or very simple are respectively, 0.15, 0.20, 0.31, 0.26, .08. Find
the probabilities that a particular surgery will be rated
\begin{enumerate}
	\item complex or very complex;
	\item neither very complex nor very simple;
	\item routine or complex
	\item routine or simple
\end{enumerate}
\solution
%\input{exemplar/11/16/3/8(1)/main.tex}
\item A card is selected from a pack of 52 cards.
\begin{enumerate}[label=(\alph*)]
    \item How many points are there in the sample space?
    \item Calculate the probability that the card is an ace of spades.
    \item Calculate the probability that the card is (i) an ace and (ii) black card.
\end{enumerate}
\solution
%\input{exemplar/11/16/3/4/main2.tex}
\item The probability that a non leap year selected at random will contain 53 sundays.
\\
\solution
%\input{exemplar/10/13/1/19/main.tex}
\item One of the four persons John, Rita, Aslam or Gurpreet will be promoted next
month. Consequently the sample space consists of four elementary outcomes
S = {John promoted, Rita promoted, Aslam promoted, Gurpreet promoted}
You are told that the chances of John’s promotion is same as that of Gurpreet,
Rita’s chances of promotion are twice as likely as Johns. Aslam’s chances are
four times that of John.
\begin{enumerate}
	\item Determine
	\begin{enumerate}
		\item P (John promoted)
		\item P (Rita promoted)
		\item P (Aslam promoted)
		\item P (Gurpreet promoted)
	\end{enumerate}
	\item If A = {John promoted or Gurpreet promoted}, find P (A).
\end{enumerate}
\solution
%\input{exemplar/11/16/3/10/main.tex}
\item A card is drawn from a deck of 52 cards. Find the probability of getting a king or a heart or a red card.\\
\solution
%\input{exemplar/11/16/3/15/main.tex}
\item The probability that a student will pass his examination is 0.73, the probability of
the student getting a compartment is 0.13, and the probability that the student will
either pass or get compartment is 0.96. State True or False.\\
\solution
%\input{exemplar/11/16/3/31/main.tex}
\item A card is selected from a pack of 52 cards\\
\begin{enumerate}[label=(\alph*)]
\item How many points are there in the sample space?
\item Calculate the probability that the cards is an ace of spades.
\item Calculate the probability that the card is (i) an ace (ii)black card.\\
\end{enumerate}
%\input{ncert/11/16/3/4_1/Prob_4.tex}
\item In a non-leap year, the probability of having 53 tuesdays or 53 wednesdays is\\
\solution
%\input{exemplar/11/16/3/18/main.tex}
\item There are 1000 sealed envelopes in a box, 10 of them contain a cash prize of
Rs 100 each, 100 of them contain a cash prize of Rs 50 each and 200 of them
contain a cash prize of Rs 10 each and rest do not contain any cash prize. If they
are well shuffled and an envelope is picked up out, what is the probability that it
contains no cash prize?\\
\solution
%\input{exemplar/10/13/3/34/main.tex}
\item 
A die is thrown and a card is selected at random from a deck of 52 playing cards. The probability of getting an even number on the die and a spade card.\\
\solution
%\input{exemplar/12/13/3/78/main.tex}
\item
If 4-digit numbers greater than 5,000 are randomly formed from the digits 0, 1, 3, 5, and 7, what is the probability of forming a number divisible by 5 when:
\begin{enumerate}
    \item The digits are repeated?
    \item The repetition of digits is not allowed?
\end{enumerate}
\solution
%\input{ncert/11/16/4/9/main.tex}
\item Consider the probability space $\brak{\Omega, \mathcal{G}, P}$ where $\Omega = [0,2]$ and $\mathcal{G} = \cbrak{\phi, \Omega, [0,1], (1,2]}$. Let $X$ and $Y$ be two functions on $\Omega$ defined as
\begin{align*}
    X(\omega) = 
    \begin{cases}
        1 & \text{if }\omega \in [0, 1]\\
        2 & \text{if }\omega \in (1, 2]
    \end{cases}
\end{align*}
and
\begin{align*}
    Y(\omega) = 
    \begin{cases}
        2 & \text{if }\omega \in [0, 1.5]\\
        3 & \text{if }\omega \in (1.5, 2].
    \end{cases}
\end{align*}
Then which one of the following statements is true?
\begin{enumerate}
    \item [(A)] $X$ is a random variable with respect to $\mathcal{G}$, but $Y$ is not a random variable with respect to $\mathcal{G}$.
    \item [(B)] $Y$ is a random variable with respect to $\mathcal{G}$, but $X$ is not a random variable with respect to $\mathcal{G}$.
    \item [(C)] Neither $X$ nor $Y$ is a random variable with respect to $\mathcal{G}$.
    \item [(D)] Both $X$ and $Y$ are random variables with respect to $\mathcal{G}$.
\end{enumerate} \hfill (GATE ST 2023)\\
\solution
%\input{gate/ST/2023/14/main.tex}
	\item  A die is loaded in such a way that each odd number is twice as likely to occur as
each even number. Find $P(G)$, where $G$ is the event that a number greater than
3 occurs on a single roll of the die.
\\
\solution
		%\input{exemplar/11/16/3/5/main.tex}
	\item All the jacks, queens and kings are removed from a deck of 52 playing cards. The remaining cards are well shuffled and then one card is drawn at random. Giving ace a value 1 similar value for other cards, find the probability that the card has a value 
		\begin{enumerate}
			\item 7
			\item greater than 7
			\item less than 7
		\end{enumerate}
		%\input{exemplar/10/13/3/30/main.tex}
  \item A Lot consists of 48 mobile phones of which 42 are good, 3 have only minor defects and 3 have major defects.Varnika will buy a phone if it is good but the trader will only buy a mobile if it has no major defects. One phone is selected at random from the lot. What is the probability that it is
\begin{enumerate}
	\item acceptable to Varnika?
            \item acceptable to the trader?
\end{enumerate}
\solution
	%\input{exemplar/10/13/3/40/main.tex}
 \item A student says that if you throw a die, it will show up 1 or not 1. Therefore, the probability of getting 1 and the probability of getting 'not 1' each is equal to $\frac{1}{2}$. Is this correct? Give reasons.\\
 \solution
        %\input{exemplar/10/13/2/9/main.tex}
   \item Four candidates A, B, C, D have ap-
plied for the assignment to coach a school cricket
team. If A is twice as likely to be selected as B, and
B and C are given about the same chance of being
selected, while C is twice as likely to be selected
as D, what are the probabilities that
\begin{enumerate}
\item C will be selected?
\item A will not be selected?
\end{enumerate}
	%\input{exemplar/11/16/3/9/main.tex}
 \item A bag contain 24 balls of which $x$ balls are red, $2x$ are white and $3x$ are blue. A ball is selected at random, What is the probability that it is
\begin{enumerate}[label=\alph*)]
\item not red ?
\item white ?
\end{enumerate}
%\input{exemplar/10/13/3/41/main.tex}
If the letters of the word ASSASSINATION are arranged at random. Find the Probability that
\begin{enumerate}[label=(\alph*)]
\item Four $S's$ come consecutively in the word
\item Two  $I's$ and two $N's$ come together
\item All $A's$ are not coming together
\item No two $A's$ are coming together
\end{enumerate}
%\input{exemplar/11/16/3/14/main.tex}
	\item One urn contains two black balls (labelled B1 and B2) and one white ball. A
	second urn contains one black ball and two white balls (labelled W1 and W2).
	Suppose the following experiment is performed. One of the two urns is chosen
	at random. Next a ball is randomly chosen from the urn. Then a second ball is
	chosen at random from the same urn without replacing the first ball.
	
	\begin{enumerate}
	\item What is the probability that two black balls are chosen?
	
	\item What is the probability that two balls of opposite colour are chosen?
	\end{enumerate}
	\solution
	%\input{exemplar/11/16/3/12/main1.tex}
\end{enumerate}

	\item A bag contains $5$ red balls and some blue balls. If the probability of drawing a blue ball is double that if a red ball, determine the number of blue balls in the bag. 
		\\
\solution
		%\begin{enumerate}[label=\thesection.\arabic*,ref=\thesection.\theenumi]
	\item One card is drawn from a well-shuffled deck of 52 cards. Find the probability of getting
\begin{enumerate}
\item A king of red colour 
\item A face card 
\item A red face card
\item The jack of hearts
\item A spade
\item The queen of diamonds

\end{enumerate}
\solution
		%\input{ncert/10/15/1/14/main.tex}
	\item Five cards—the ten, jack, queen, king and ace of diamonds, are well-shuffled with their face downwards. One card is then picked up at random.
\begin{enumerate}
\item
What is the probability that the card is the queen? 
\item
If the queen is drawn and put aside, what is the probability that the second card picked up is (a) an ace? (b) a queen?\\
\end{enumerate}
\solution
		%\input{ncert/10/15/1/15/defs.tex}
	\item A bag contains $5$ red balls and some blue balls. If the probability of drawing a blue ball is double that if a red ball, determine the number of blue balls in the bag. 
		\\
\solution
		%\input{ncert/10/15/2/3/defs.tex}
	\item A card is selected from a pack of 52 cards.
 \begin{enumerate}[label=(\alph*)] 
                 \item How many points are there in the sample space?
                 \item Calculate the probability that the card is an ace of spades.
                 \item Calculate the probability that the card is (i) an ace and (ii) black card.
 \end{enumerate}
\solution
		%\input{ncert/11/16/3/4/main.tex}
\item Four cards are drawn from a well-shuffled deck of 52 cards. What is the probability of obtaining 3 diamonds and one spade.
\\
\solution
		%\input{ncert/11/16/4/2/defs.tex}
\item In a certain lottery 10,000 tickets are sold and ten equal prizes are awarded. What is the probability of not getting a prize if you buy (a) one ticket (b) two tickets (c) 10 tickets ?	
\\
\solution
		%\input{ncert/11/16/4/4/defs.tex}
		%
\item 
Out of 100 students, two sections of 40 and 60 are formed. If you and your friend are among the 100 students, what is the probability that
\begin{enumerate}
\item you both enter the same section?
\item you both enter the different sections?
\end{enumerate}
\solution
		%\input{ncert/11/16/4/5/defs.tex}
	\item 
The number lock of a suitcase has 4 wheels each labelled with ten digits i.e. from 0 to 9.The lock opens with a sequence of four digits with no repeats.What is the probability of a person getting the right sequence to open the suitcase.
\\
\solution
		%\input{ncert/11/16/4/10/defs.tex}
		%
\item 
Two cards are drawn at random and without replacement from a pack of 52 playing cards. Find the probability that both the cards are black.
\\
\solution
		%\input{ncert/12/13/2/2/defs.tex}
		\item A box of oranges is inspected by examining three randomly selected oranges drawn without replacement. If all the three oranges are good, the box is approved for sale, otherwise, it is rejected. Find the probability that a box containing 15 oranges out of which 12 are good and 3 are bad ones will be approved for sale.
		\label{ncert/12/13/2/3/defs.tex}
		\item Two balls are drawn at random with replacement from a box containing 10 black and 8 red balls. Find the probability that
		\label{ncert/12/13/2/12}
\begin{enumerate}
\item both balls are red.
\item first ball is black and second is red.
\item one of them is black and other is red.
\end{enumerate}

\item In a hostel, 60\% of the students read Hindi newspaper, 40\% read English newspaper and 20\% read both Hindi and English newspapers. A student is selected at random.
		\label{ncert/12/13/2/15}
\begin{enumerate}
\item Find the probability that she reads neither Hindi nor English newspapers.
\item If she reads Hindi newspaper, find the probability that she reads English newspaper.
\item If she reads English newspaper, find the probability that she reads Hindi newspaper.\\
\end{enumerate}
\item The probability of obtaining an even prime number on each die, when a pair of dice is rolled is 
\begin{enumerate}
    \item $0$ 
    
    \item $\frac{1}{3}$ 
    
    \item $\frac{1}{12}$ 
    
    \item $\frac{1}{36}$ 
\end{enumerate}
\solution
		%\input{ncert/12/13/2/17/defs.tex}
	\item A bag contains 4 red and 4 black balls, another bag contains 2 red and 6 black balls. One of the two bags is selected at random and a ball is drawn from the bag which is found to be red. Find the probability that the ball is drawn from the first bag.
\\
\solution
		%\input{ncert/12/13/3/2/main.tex}
  \item
  Cards with numbers 2 to 101 are placed in a box. A card is selected at random.Find the probability that the card has
\begin{enumerate}[label=(\roman*)]
	\item an even number 
	\item a square number
\end{enumerate}
\solution
%\input{exemplar/10/13/3/32/main.tex}
\item
The king, queen and jack of clubs are removed from a deck of 52 playing cards and then well shuffled. Now one card is drawn at random from the remaining cards.  Determine the probability that the card is
\begin{enumerate}[label=(\roman*)]
\item a club
\item 10 of hearts
\end{enumerate}
\solution
%\input{exemplar/10/13/3/29/main.tex}
\item A team of medical students doing their internship have to assist during surgeries
at a city hospital. The probabilities of surgeries rated as very complex, complex,
routine, simple or very simple are respectively, 0.15, 0.20, 0.31, 0.26, .08. Find
the probabilities that a particular surgery will be rated
\begin{enumerate}
	\item complex or very complex;
	\item neither very complex nor very simple;
	\item routine or complex
	\item routine or simple
\end{enumerate}
\solution
%\input{exemplar/11/16/3/8(1)/main.tex}
\item A card is selected from a pack of 52 cards.
\begin{enumerate}[label=(\alph*)]
    \item How many points are there in the sample space?
    \item Calculate the probability that the card is an ace of spades.
    \item Calculate the probability that the card is (i) an ace and (ii) black card.
\end{enumerate}
\solution
%\input{exemplar/11/16/3/4/main2.tex}
\item The probability that a non leap year selected at random will contain 53 sundays.
\\
\solution
%\input{exemplar/10/13/1/19/main.tex}
\item One of the four persons John, Rita, Aslam or Gurpreet will be promoted next
month. Consequently the sample space consists of four elementary outcomes
S = {John promoted, Rita promoted, Aslam promoted, Gurpreet promoted}
You are told that the chances of John’s promotion is same as that of Gurpreet,
Rita’s chances of promotion are twice as likely as Johns. Aslam’s chances are
four times that of John.
\begin{enumerate}
	\item Determine
	\begin{enumerate}
		\item P (John promoted)
		\item P (Rita promoted)
		\item P (Aslam promoted)
		\item P (Gurpreet promoted)
	\end{enumerate}
	\item If A = {John promoted or Gurpreet promoted}, find P (A).
\end{enumerate}
\solution
%\input{exemplar/11/16/3/10/main.tex}
\item A card is drawn from a deck of 52 cards. Find the probability of getting a king or a heart or a red card.\\
\solution
%\input{exemplar/11/16/3/15/main.tex}
\item The probability that a student will pass his examination is 0.73, the probability of
the student getting a compartment is 0.13, and the probability that the student will
either pass or get compartment is 0.96. State True or False.\\
\solution
%\input{exemplar/11/16/3/31/main.tex}
\item A card is selected from a pack of 52 cards\\
\begin{enumerate}[label=(\alph*)]
\item How many points are there in the sample space?
\item Calculate the probability that the cards is an ace of spades.
\item Calculate the probability that the card is (i) an ace (ii)black card.\\
\end{enumerate}
%\input{ncert/11/16/3/4_1/Prob_4.tex}
\item In a non-leap year, the probability of having 53 tuesdays or 53 wednesdays is\\
\solution
%\input{exemplar/11/16/3/18/main.tex}
\item There are 1000 sealed envelopes in a box, 10 of them contain a cash prize of
Rs 100 each, 100 of them contain a cash prize of Rs 50 each and 200 of them
contain a cash prize of Rs 10 each and rest do not contain any cash prize. If they
are well shuffled and an envelope is picked up out, what is the probability that it
contains no cash prize?\\
\solution
%\input{exemplar/10/13/3/34/main.tex}
\item 
A die is thrown and a card is selected at random from a deck of 52 playing cards. The probability of getting an even number on the die and a spade card.\\
\solution
%\input{exemplar/12/13/3/78/main.tex}
\item
If 4-digit numbers greater than 5,000 are randomly formed from the digits 0, 1, 3, 5, and 7, what is the probability of forming a number divisible by 5 when:
\begin{enumerate}
    \item The digits are repeated?
    \item The repetition of digits is not allowed?
\end{enumerate}
\solution
%\input{ncert/11/16/4/9/main.tex}
\item Consider the probability space $\brak{\Omega, \mathcal{G}, P}$ where $\Omega = [0,2]$ and $\mathcal{G} = \cbrak{\phi, \Omega, [0,1], (1,2]}$. Let $X$ and $Y$ be two functions on $\Omega$ defined as
\begin{align*}
    X(\omega) = 
    \begin{cases}
        1 & \text{if }\omega \in [0, 1]\\
        2 & \text{if }\omega \in (1, 2]
    \end{cases}
\end{align*}
and
\begin{align*}
    Y(\omega) = 
    \begin{cases}
        2 & \text{if }\omega \in [0, 1.5]\\
        3 & \text{if }\omega \in (1.5, 2].
    \end{cases}
\end{align*}
Then which one of the following statements is true?
\begin{enumerate}
    \item [(A)] $X$ is a random variable with respect to $\mathcal{G}$, but $Y$ is not a random variable with respect to $\mathcal{G}$.
    \item [(B)] $Y$ is a random variable with respect to $\mathcal{G}$, but $X$ is not a random variable with respect to $\mathcal{G}$.
    \item [(C)] Neither $X$ nor $Y$ is a random variable with respect to $\mathcal{G}$.
    \item [(D)] Both $X$ and $Y$ are random variables with respect to $\mathcal{G}$.
\end{enumerate} \hfill (GATE ST 2023)\\
\solution
%\input{gate/ST/2023/14/main.tex}
	\item  A die is loaded in such a way that each odd number is twice as likely to occur as
each even number. Find $P(G)$, where $G$ is the event that a number greater than
3 occurs on a single roll of the die.
\\
\solution
		%\input{exemplar/11/16/3/5/main.tex}
	\item All the jacks, queens and kings are removed from a deck of 52 playing cards. The remaining cards are well shuffled and then one card is drawn at random. Giving ace a value 1 similar value for other cards, find the probability that the card has a value 
		\begin{enumerate}
			\item 7
			\item greater than 7
			\item less than 7
		\end{enumerate}
		%\input{exemplar/10/13/3/30/main.tex}
  \item A Lot consists of 48 mobile phones of which 42 are good, 3 have only minor defects and 3 have major defects.Varnika will buy a phone if it is good but the trader will only buy a mobile if it has no major defects. One phone is selected at random from the lot. What is the probability that it is
\begin{enumerate}
	\item acceptable to Varnika?
            \item acceptable to the trader?
\end{enumerate}
\solution
	%\input{exemplar/10/13/3/40/main.tex}
 \item A student says that if you throw a die, it will show up 1 or not 1. Therefore, the probability of getting 1 and the probability of getting 'not 1' each is equal to $\frac{1}{2}$. Is this correct? Give reasons.\\
 \solution
        %\input{exemplar/10/13/2/9/main.tex}
   \item Four candidates A, B, C, D have ap-
plied for the assignment to coach a school cricket
team. If A is twice as likely to be selected as B, and
B and C are given about the same chance of being
selected, while C is twice as likely to be selected
as D, what are the probabilities that
\begin{enumerate}
\item C will be selected?
\item A will not be selected?
\end{enumerate}
	%\input{exemplar/11/16/3/9/main.tex}
 \item A bag contain 24 balls of which $x$ balls are red, $2x$ are white and $3x$ are blue. A ball is selected at random, What is the probability that it is
\begin{enumerate}[label=\alph*)]
\item not red ?
\item white ?
\end{enumerate}
%\input{exemplar/10/13/3/41/main.tex}
If the letters of the word ASSASSINATION are arranged at random. Find the Probability that
\begin{enumerate}[label=(\alph*)]
\item Four $S's$ come consecutively in the word
\item Two  $I's$ and two $N's$ come together
\item All $A's$ are not coming together
\item No two $A's$ are coming together
\end{enumerate}
%\input{exemplar/11/16/3/14/main.tex}
	\item One urn contains two black balls (labelled B1 and B2) and one white ball. A
	second urn contains one black ball and two white balls (labelled W1 and W2).
	Suppose the following experiment is performed. One of the two urns is chosen
	at random. Next a ball is randomly chosen from the urn. Then a second ball is
	chosen at random from the same urn without replacing the first ball.
	
	\begin{enumerate}
	\item What is the probability that two black balls are chosen?
	
	\item What is the probability that two balls of opposite colour are chosen?
	\end{enumerate}
	\solution
	%\input{exemplar/11/16/3/12/main1.tex}
\end{enumerate}

	\item A card is selected from a pack of 52 cards.
 \begin{enumerate}[label=(\alph*)] 
                 \item How many points are there in the sample space?
                 \item Calculate the probability that the card is an ace of spades.
                 \item Calculate the probability that the card is (i) an ace and (ii) black card.
 \end{enumerate}
\solution
		%\begin{table}[H]
	\centering
\begin{tabular}{|c|c|c|}
\hline
Random variable &Value &Definition\\ \hline
\multirow{3}{*}{X} &0 &Slips of Rs 1\\
&1 &Slips of Rs 5\\
&2 &Slips of Rs 13\\ \hline
\multirow{2}{*}{Y} &0 &Box A\\
&1 &Box B\\\hline
\end{tabular}
\caption{}
\label{tab:Distribution}
\end{table}
See \tabref{tab:Distribution}.
\begin{align}
p_{Y}\brak{k}= \begin{cases} 
      \frac{1}{3} & {k=0} \\
      \frac{2}{3 }& {k=1} 
   \end{cases}
   \\
p_{Y|X}\brak{0|0} = \frac{19}{25}\, 
p_{Y|X}\brak{0|1} = \frac{6}{25}\,
p_{Y|X}\brak{1|0} = \frac{45}{50}\,
p_{Y|X}\brak{1|2} = \frac{5}{50}
\end{align}
The desired probability is the probability that a slip drawn at random is marked other than Rs 1,
\begin{align}
&=1-p_X\brak{0}\\
&= p_X(1) + p_X(2)
\end{align}
Using Bayes theorem,
\begin{align}
&= p_Y\brak{0} \times \pr{Y=0 | X=1} + p_Y\brak{1} \times \pr{Y=1|X=2}\\
&=\frac{1}{3} \times \frac{6}{25} + \frac{2}{3} \times \frac{5}{50}\\
&=\frac{11}{75}
\end{align}

\newpage

%\tableofcontents

\bigskip

\renewcommand{\thefigure}{\theenumi}
\renewcommand{\thetable}{\theenumi}
%\renewcommand{\theequation}{\theenumi}

%\begin{abstract}
%%\boldmath
%In this letter, an algorithm for evaluating the exact analytical bit error rate  (BER)  for the piecewise linear (PL) combiner for  multiple relays is presented. Previous results were available only for upto three relays. The algorithm is unique in the sense that  the actual mathematical expressions, that are prohibitively large, need not be explicitly obtained. The diversity gain due to multiple relays is shown through plots of the analytical BER, well supported by simulations. 
%
%\end{abstract}
% IEEEtran.cls defaults to using nonbold math in the Abstract.
% This preserves the distinction between vectors and scalars. However,
% if the journal you are submitting to favors bold math in the abstract,
% then you can use LaTeX's standard command \boldmath at the very start
% of the abstract to achieve this. Many IEEE journals frown on math
% in the abstract anyway.

% Note that keywords are not normally used for peerreview papers.
%\begin{IEEEkeywords}
%Cooperative diversity, decode and forward, piecewise linear
%\end{IEEEkeywords}



% For peer review papers, you can put extra information on the cover
% page as needed:
% \ifCLASSOPTIONpeerreview
% \begin{center} \bfseries EDICS Category: 3-BBND \end{center}
% \fi
%
% For peerreview papers, this IEEEtran command inserts a page break and
% creates the second title. It will be ignored for other modes.
%\IEEEpeerreviewmaketitle




\item Four cards are drawn from a well-shuffled deck of 52 cards. What is the probability of obtaining 3 diamonds and one spade.
\\
\solution
		%\begin{enumerate}[label=\thesection.\arabic*,ref=\thesection.\theenumi]
	\item One card is drawn from a well-shuffled deck of 52 cards. Find the probability of getting
\begin{enumerate}
\item A king of red colour 
\item A face card 
\item A red face card
\item The jack of hearts
\item A spade
\item The queen of diamonds

\end{enumerate}
\solution
		%\input{ncert/10/15/1/14/main.tex}
	\item Five cards—the ten, jack, queen, king and ace of diamonds, are well-shuffled with their face downwards. One card is then picked up at random.
\begin{enumerate}
\item
What is the probability that the card is the queen? 
\item
If the queen is drawn and put aside, what is the probability that the second card picked up is (a) an ace? (b) a queen?\\
\end{enumerate}
\solution
		%\input{ncert/10/15/1/15/defs.tex}
	\item A bag contains $5$ red balls and some blue balls. If the probability of drawing a blue ball is double that if a red ball, determine the number of blue balls in the bag. 
		\\
\solution
		%\input{ncert/10/15/2/3/defs.tex}
	\item A card is selected from a pack of 52 cards.
 \begin{enumerate}[label=(\alph*)] 
                 \item How many points are there in the sample space?
                 \item Calculate the probability that the card is an ace of spades.
                 \item Calculate the probability that the card is (i) an ace and (ii) black card.
 \end{enumerate}
\solution
		%\input{ncert/11/16/3/4/main.tex}
\item Four cards are drawn from a well-shuffled deck of 52 cards. What is the probability of obtaining 3 diamonds and one spade.
\\
\solution
		%\input{ncert/11/16/4/2/defs.tex}
\item In a certain lottery 10,000 tickets are sold and ten equal prizes are awarded. What is the probability of not getting a prize if you buy (a) one ticket (b) two tickets (c) 10 tickets ?	
\\
\solution
		%\input{ncert/11/16/4/4/defs.tex}
		%
\item 
Out of 100 students, two sections of 40 and 60 are formed. If you and your friend are among the 100 students, what is the probability that
\begin{enumerate}
\item you both enter the same section?
\item you both enter the different sections?
\end{enumerate}
\solution
		%\input{ncert/11/16/4/5/defs.tex}
	\item 
The number lock of a suitcase has 4 wheels each labelled with ten digits i.e. from 0 to 9.The lock opens with a sequence of four digits with no repeats.What is the probability of a person getting the right sequence to open the suitcase.
\\
\solution
		%\input{ncert/11/16/4/10/defs.tex}
		%
\item 
Two cards are drawn at random and without replacement from a pack of 52 playing cards. Find the probability that both the cards are black.
\\
\solution
		%\input{ncert/12/13/2/2/defs.tex}
		\item A box of oranges is inspected by examining three randomly selected oranges drawn without replacement. If all the three oranges are good, the box is approved for sale, otherwise, it is rejected. Find the probability that a box containing 15 oranges out of which 12 are good and 3 are bad ones will be approved for sale.
		\label{ncert/12/13/2/3/defs.tex}
		\item Two balls are drawn at random with replacement from a box containing 10 black and 8 red balls. Find the probability that
		\label{ncert/12/13/2/12}
\begin{enumerate}
\item both balls are red.
\item first ball is black and second is red.
\item one of them is black and other is red.
\end{enumerate}

\item In a hostel, 60\% of the students read Hindi newspaper, 40\% read English newspaper and 20\% read both Hindi and English newspapers. A student is selected at random.
		\label{ncert/12/13/2/15}
\begin{enumerate}
\item Find the probability that she reads neither Hindi nor English newspapers.
\item If she reads Hindi newspaper, find the probability that she reads English newspaper.
\item If she reads English newspaper, find the probability that she reads Hindi newspaper.\\
\end{enumerate}
\item The probability of obtaining an even prime number on each die, when a pair of dice is rolled is 
\begin{enumerate}
    \item $0$ 
    
    \item $\frac{1}{3}$ 
    
    \item $\frac{1}{12}$ 
    
    \item $\frac{1}{36}$ 
\end{enumerate}
\solution
		%\input{ncert/12/13/2/17/defs.tex}
	\item A bag contains 4 red and 4 black balls, another bag contains 2 red and 6 black balls. One of the two bags is selected at random and a ball is drawn from the bag which is found to be red. Find the probability that the ball is drawn from the first bag.
\\
\solution
		%\input{ncert/12/13/3/2/main.tex}
  \item
  Cards with numbers 2 to 101 are placed in a box. A card is selected at random.Find the probability that the card has
\begin{enumerate}[label=(\roman*)]
	\item an even number 
	\item a square number
\end{enumerate}
\solution
%\input{exemplar/10/13/3/32/main.tex}
\item
The king, queen and jack of clubs are removed from a deck of 52 playing cards and then well shuffled. Now one card is drawn at random from the remaining cards.  Determine the probability that the card is
\begin{enumerate}[label=(\roman*)]
\item a club
\item 10 of hearts
\end{enumerate}
\solution
%\input{exemplar/10/13/3/29/main.tex}
\item A team of medical students doing their internship have to assist during surgeries
at a city hospital. The probabilities of surgeries rated as very complex, complex,
routine, simple or very simple are respectively, 0.15, 0.20, 0.31, 0.26, .08. Find
the probabilities that a particular surgery will be rated
\begin{enumerate}
	\item complex or very complex;
	\item neither very complex nor very simple;
	\item routine or complex
	\item routine or simple
\end{enumerate}
\solution
%\input{exemplar/11/16/3/8(1)/main.tex}
\item A card is selected from a pack of 52 cards.
\begin{enumerate}[label=(\alph*)]
    \item How many points are there in the sample space?
    \item Calculate the probability that the card is an ace of spades.
    \item Calculate the probability that the card is (i) an ace and (ii) black card.
\end{enumerate}
\solution
%\input{exemplar/11/16/3/4/main2.tex}
\item The probability that a non leap year selected at random will contain 53 sundays.
\\
\solution
%\input{exemplar/10/13/1/19/main.tex}
\item One of the four persons John, Rita, Aslam or Gurpreet will be promoted next
month. Consequently the sample space consists of four elementary outcomes
S = {John promoted, Rita promoted, Aslam promoted, Gurpreet promoted}
You are told that the chances of John’s promotion is same as that of Gurpreet,
Rita’s chances of promotion are twice as likely as Johns. Aslam’s chances are
four times that of John.
\begin{enumerate}
	\item Determine
	\begin{enumerate}
		\item P (John promoted)
		\item P (Rita promoted)
		\item P (Aslam promoted)
		\item P (Gurpreet promoted)
	\end{enumerate}
	\item If A = {John promoted or Gurpreet promoted}, find P (A).
\end{enumerate}
\solution
%\input{exemplar/11/16/3/10/main.tex}
\item A card is drawn from a deck of 52 cards. Find the probability of getting a king or a heart or a red card.\\
\solution
%\input{exemplar/11/16/3/15/main.tex}
\item The probability that a student will pass his examination is 0.73, the probability of
the student getting a compartment is 0.13, and the probability that the student will
either pass or get compartment is 0.96. State True or False.\\
\solution
%\input{exemplar/11/16/3/31/main.tex}
\item A card is selected from a pack of 52 cards\\
\begin{enumerate}[label=(\alph*)]
\item How many points are there in the sample space?
\item Calculate the probability that the cards is an ace of spades.
\item Calculate the probability that the card is (i) an ace (ii)black card.\\
\end{enumerate}
%\input{ncert/11/16/3/4_1/Prob_4.tex}
\item In a non-leap year, the probability of having 53 tuesdays or 53 wednesdays is\\
\solution
%\input{exemplar/11/16/3/18/main.tex}
\item There are 1000 sealed envelopes in a box, 10 of them contain a cash prize of
Rs 100 each, 100 of them contain a cash prize of Rs 50 each and 200 of them
contain a cash prize of Rs 10 each and rest do not contain any cash prize. If they
are well shuffled and an envelope is picked up out, what is the probability that it
contains no cash prize?\\
\solution
%\input{exemplar/10/13/3/34/main.tex}
\item 
A die is thrown and a card is selected at random from a deck of 52 playing cards. The probability of getting an even number on the die and a spade card.\\
\solution
%\input{exemplar/12/13/3/78/main.tex}
\item
If 4-digit numbers greater than 5,000 are randomly formed from the digits 0, 1, 3, 5, and 7, what is the probability of forming a number divisible by 5 when:
\begin{enumerate}
    \item The digits are repeated?
    \item The repetition of digits is not allowed?
\end{enumerate}
\solution
%\input{ncert/11/16/4/9/main.tex}
\item Consider the probability space $\brak{\Omega, \mathcal{G}, P}$ where $\Omega = [0,2]$ and $\mathcal{G} = \cbrak{\phi, \Omega, [0,1], (1,2]}$. Let $X$ and $Y$ be two functions on $\Omega$ defined as
\begin{align*}
    X(\omega) = 
    \begin{cases}
        1 & \text{if }\omega \in [0, 1]\\
        2 & \text{if }\omega \in (1, 2]
    \end{cases}
\end{align*}
and
\begin{align*}
    Y(\omega) = 
    \begin{cases}
        2 & \text{if }\omega \in [0, 1.5]\\
        3 & \text{if }\omega \in (1.5, 2].
    \end{cases}
\end{align*}
Then which one of the following statements is true?
\begin{enumerate}
    \item [(A)] $X$ is a random variable with respect to $\mathcal{G}$, but $Y$ is not a random variable with respect to $\mathcal{G}$.
    \item [(B)] $Y$ is a random variable with respect to $\mathcal{G}$, but $X$ is not a random variable with respect to $\mathcal{G}$.
    \item [(C)] Neither $X$ nor $Y$ is a random variable with respect to $\mathcal{G}$.
    \item [(D)] Both $X$ and $Y$ are random variables with respect to $\mathcal{G}$.
\end{enumerate} \hfill (GATE ST 2023)\\
\solution
%\input{gate/ST/2023/14/main.tex}
	\item  A die is loaded in such a way that each odd number is twice as likely to occur as
each even number. Find $P(G)$, where $G$ is the event that a number greater than
3 occurs on a single roll of the die.
\\
\solution
		%\input{exemplar/11/16/3/5/main.tex}
	\item All the jacks, queens and kings are removed from a deck of 52 playing cards. The remaining cards are well shuffled and then one card is drawn at random. Giving ace a value 1 similar value for other cards, find the probability that the card has a value 
		\begin{enumerate}
			\item 7
			\item greater than 7
			\item less than 7
		\end{enumerate}
		%\input{exemplar/10/13/3/30/main.tex}
  \item A Lot consists of 48 mobile phones of which 42 are good, 3 have only minor defects and 3 have major defects.Varnika will buy a phone if it is good but the trader will only buy a mobile if it has no major defects. One phone is selected at random from the lot. What is the probability that it is
\begin{enumerate}
	\item acceptable to Varnika?
            \item acceptable to the trader?
\end{enumerate}
\solution
	%\input{exemplar/10/13/3/40/main.tex}
 \item A student says that if you throw a die, it will show up 1 or not 1. Therefore, the probability of getting 1 and the probability of getting 'not 1' each is equal to $\frac{1}{2}$. Is this correct? Give reasons.\\
 \solution
        %\input{exemplar/10/13/2/9/main.tex}
   \item Four candidates A, B, C, D have ap-
plied for the assignment to coach a school cricket
team. If A is twice as likely to be selected as B, and
B and C are given about the same chance of being
selected, while C is twice as likely to be selected
as D, what are the probabilities that
\begin{enumerate}
\item C will be selected?
\item A will not be selected?
\end{enumerate}
	%\input{exemplar/11/16/3/9/main.tex}
 \item A bag contain 24 balls of which $x$ balls are red, $2x$ are white and $3x$ are blue. A ball is selected at random, What is the probability that it is
\begin{enumerate}[label=\alph*)]
\item not red ?
\item white ?
\end{enumerate}
%\input{exemplar/10/13/3/41/main.tex}
If the letters of the word ASSASSINATION are arranged at random. Find the Probability that
\begin{enumerate}[label=(\alph*)]
\item Four $S's$ come consecutively in the word
\item Two  $I's$ and two $N's$ come together
\item All $A's$ are not coming together
\item No two $A's$ are coming together
\end{enumerate}
%\input{exemplar/11/16/3/14/main.tex}
	\item One urn contains two black balls (labelled B1 and B2) and one white ball. A
	second urn contains one black ball and two white balls (labelled W1 and W2).
	Suppose the following experiment is performed. One of the two urns is chosen
	at random. Next a ball is randomly chosen from the urn. Then a second ball is
	chosen at random from the same urn without replacing the first ball.
	
	\begin{enumerate}
	\item What is the probability that two black balls are chosen?
	
	\item What is the probability that two balls of opposite colour are chosen?
	\end{enumerate}
	\solution
	%\input{exemplar/11/16/3/12/main1.tex}
\end{enumerate}

\item In a certain lottery 10,000 tickets are sold and ten equal prizes are awarded. What is the probability of not getting a prize if you buy (a) one ticket (b) two tickets (c) 10 tickets ?	
\\
\solution
		%\begin{enumerate}[label=\thesection.\arabic*,ref=\thesection.\theenumi]
	\item One card is drawn from a well-shuffled deck of 52 cards. Find the probability of getting
\begin{enumerate}
\item A king of red colour 
\item A face card 
\item A red face card
\item The jack of hearts
\item A spade
\item The queen of diamonds

\end{enumerate}
\solution
		%\input{ncert/10/15/1/14/main.tex}
	\item Five cards—the ten, jack, queen, king and ace of diamonds, are well-shuffled with their face downwards. One card is then picked up at random.
\begin{enumerate}
\item
What is the probability that the card is the queen? 
\item
If the queen is drawn and put aside, what is the probability that the second card picked up is (a) an ace? (b) a queen?\\
\end{enumerate}
\solution
		%\input{ncert/10/15/1/15/defs.tex}
	\item A bag contains $5$ red balls and some blue balls. If the probability of drawing a blue ball is double that if a red ball, determine the number of blue balls in the bag. 
		\\
\solution
		%\input{ncert/10/15/2/3/defs.tex}
	\item A card is selected from a pack of 52 cards.
 \begin{enumerate}[label=(\alph*)] 
                 \item How many points are there in the sample space?
                 \item Calculate the probability that the card is an ace of spades.
                 \item Calculate the probability that the card is (i) an ace and (ii) black card.
 \end{enumerate}
\solution
		%\input{ncert/11/16/3/4/main.tex}
\item Four cards are drawn from a well-shuffled deck of 52 cards. What is the probability of obtaining 3 diamonds and one spade.
\\
\solution
		%\input{ncert/11/16/4/2/defs.tex}
\item In a certain lottery 10,000 tickets are sold and ten equal prizes are awarded. What is the probability of not getting a prize if you buy (a) one ticket (b) two tickets (c) 10 tickets ?	
\\
\solution
		%\input{ncert/11/16/4/4/defs.tex}
		%
\item 
Out of 100 students, two sections of 40 and 60 are formed. If you and your friend are among the 100 students, what is the probability that
\begin{enumerate}
\item you both enter the same section?
\item you both enter the different sections?
\end{enumerate}
\solution
		%\input{ncert/11/16/4/5/defs.tex}
	\item 
The number lock of a suitcase has 4 wheels each labelled with ten digits i.e. from 0 to 9.The lock opens with a sequence of four digits with no repeats.What is the probability of a person getting the right sequence to open the suitcase.
\\
\solution
		%\input{ncert/11/16/4/10/defs.tex}
		%
\item 
Two cards are drawn at random and without replacement from a pack of 52 playing cards. Find the probability that both the cards are black.
\\
\solution
		%\input{ncert/12/13/2/2/defs.tex}
		\item A box of oranges is inspected by examining three randomly selected oranges drawn without replacement. If all the three oranges are good, the box is approved for sale, otherwise, it is rejected. Find the probability that a box containing 15 oranges out of which 12 are good and 3 are bad ones will be approved for sale.
		\label{ncert/12/13/2/3/defs.tex}
		\item Two balls are drawn at random with replacement from a box containing 10 black and 8 red balls. Find the probability that
		\label{ncert/12/13/2/12}
\begin{enumerate}
\item both balls are red.
\item first ball is black and second is red.
\item one of them is black and other is red.
\end{enumerate}

\item In a hostel, 60\% of the students read Hindi newspaper, 40\% read English newspaper and 20\% read both Hindi and English newspapers. A student is selected at random.
		\label{ncert/12/13/2/15}
\begin{enumerate}
\item Find the probability that she reads neither Hindi nor English newspapers.
\item If she reads Hindi newspaper, find the probability that she reads English newspaper.
\item If she reads English newspaper, find the probability that she reads Hindi newspaper.\\
\end{enumerate}
\item The probability of obtaining an even prime number on each die, when a pair of dice is rolled is 
\begin{enumerate}
    \item $0$ 
    
    \item $\frac{1}{3}$ 
    
    \item $\frac{1}{12}$ 
    
    \item $\frac{1}{36}$ 
\end{enumerate}
\solution
		%\input{ncert/12/13/2/17/defs.tex}
	\item A bag contains 4 red and 4 black balls, another bag contains 2 red and 6 black balls. One of the two bags is selected at random and a ball is drawn from the bag which is found to be red. Find the probability that the ball is drawn from the first bag.
\\
\solution
		%\input{ncert/12/13/3/2/main.tex}
  \item
  Cards with numbers 2 to 101 are placed in a box. A card is selected at random.Find the probability that the card has
\begin{enumerate}[label=(\roman*)]
	\item an even number 
	\item a square number
\end{enumerate}
\solution
%\input{exemplar/10/13/3/32/main.tex}
\item
The king, queen and jack of clubs are removed from a deck of 52 playing cards and then well shuffled. Now one card is drawn at random from the remaining cards.  Determine the probability that the card is
\begin{enumerate}[label=(\roman*)]
\item a club
\item 10 of hearts
\end{enumerate}
\solution
%\input{exemplar/10/13/3/29/main.tex}
\item A team of medical students doing their internship have to assist during surgeries
at a city hospital. The probabilities of surgeries rated as very complex, complex,
routine, simple or very simple are respectively, 0.15, 0.20, 0.31, 0.26, .08. Find
the probabilities that a particular surgery will be rated
\begin{enumerate}
	\item complex or very complex;
	\item neither very complex nor very simple;
	\item routine or complex
	\item routine or simple
\end{enumerate}
\solution
%\input{exemplar/11/16/3/8(1)/main.tex}
\item A card is selected from a pack of 52 cards.
\begin{enumerate}[label=(\alph*)]
    \item How many points are there in the sample space?
    \item Calculate the probability that the card is an ace of spades.
    \item Calculate the probability that the card is (i) an ace and (ii) black card.
\end{enumerate}
\solution
%\input{exemplar/11/16/3/4/main2.tex}
\item The probability that a non leap year selected at random will contain 53 sundays.
\\
\solution
%\input{exemplar/10/13/1/19/main.tex}
\item One of the four persons John, Rita, Aslam or Gurpreet will be promoted next
month. Consequently the sample space consists of four elementary outcomes
S = {John promoted, Rita promoted, Aslam promoted, Gurpreet promoted}
You are told that the chances of John’s promotion is same as that of Gurpreet,
Rita’s chances of promotion are twice as likely as Johns. Aslam’s chances are
four times that of John.
\begin{enumerate}
	\item Determine
	\begin{enumerate}
		\item P (John promoted)
		\item P (Rita promoted)
		\item P (Aslam promoted)
		\item P (Gurpreet promoted)
	\end{enumerate}
	\item If A = {John promoted or Gurpreet promoted}, find P (A).
\end{enumerate}
\solution
%\input{exemplar/11/16/3/10/main.tex}
\item A card is drawn from a deck of 52 cards. Find the probability of getting a king or a heart or a red card.\\
\solution
%\input{exemplar/11/16/3/15/main.tex}
\item The probability that a student will pass his examination is 0.73, the probability of
the student getting a compartment is 0.13, and the probability that the student will
either pass or get compartment is 0.96. State True or False.\\
\solution
%\input{exemplar/11/16/3/31/main.tex}
\item A card is selected from a pack of 52 cards\\
\begin{enumerate}[label=(\alph*)]
\item How many points are there in the sample space?
\item Calculate the probability that the cards is an ace of spades.
\item Calculate the probability that the card is (i) an ace (ii)black card.\\
\end{enumerate}
%\input{ncert/11/16/3/4_1/Prob_4.tex}
\item In a non-leap year, the probability of having 53 tuesdays or 53 wednesdays is\\
\solution
%\input{exemplar/11/16/3/18/main.tex}
\item There are 1000 sealed envelopes in a box, 10 of them contain a cash prize of
Rs 100 each, 100 of them contain a cash prize of Rs 50 each and 200 of them
contain a cash prize of Rs 10 each and rest do not contain any cash prize. If they
are well shuffled and an envelope is picked up out, what is the probability that it
contains no cash prize?\\
\solution
%\input{exemplar/10/13/3/34/main.tex}
\item 
A die is thrown and a card is selected at random from a deck of 52 playing cards. The probability of getting an even number on the die and a spade card.\\
\solution
%\input{exemplar/12/13/3/78/main.tex}
\item
If 4-digit numbers greater than 5,000 are randomly formed from the digits 0, 1, 3, 5, and 7, what is the probability of forming a number divisible by 5 when:
\begin{enumerate}
    \item The digits are repeated?
    \item The repetition of digits is not allowed?
\end{enumerate}
\solution
%\input{ncert/11/16/4/9/main.tex}
\item Consider the probability space $\brak{\Omega, \mathcal{G}, P}$ where $\Omega = [0,2]$ and $\mathcal{G} = \cbrak{\phi, \Omega, [0,1], (1,2]}$. Let $X$ and $Y$ be two functions on $\Omega$ defined as
\begin{align*}
    X(\omega) = 
    \begin{cases}
        1 & \text{if }\omega \in [0, 1]\\
        2 & \text{if }\omega \in (1, 2]
    \end{cases}
\end{align*}
and
\begin{align*}
    Y(\omega) = 
    \begin{cases}
        2 & \text{if }\omega \in [0, 1.5]\\
        3 & \text{if }\omega \in (1.5, 2].
    \end{cases}
\end{align*}
Then which one of the following statements is true?
\begin{enumerate}
    \item [(A)] $X$ is a random variable with respect to $\mathcal{G}$, but $Y$ is not a random variable with respect to $\mathcal{G}$.
    \item [(B)] $Y$ is a random variable with respect to $\mathcal{G}$, but $X$ is not a random variable with respect to $\mathcal{G}$.
    \item [(C)] Neither $X$ nor $Y$ is a random variable with respect to $\mathcal{G}$.
    \item [(D)] Both $X$ and $Y$ are random variables with respect to $\mathcal{G}$.
\end{enumerate} \hfill (GATE ST 2023)\\
\solution
%\input{gate/ST/2023/14/main.tex}
	\item  A die is loaded in such a way that each odd number is twice as likely to occur as
each even number. Find $P(G)$, where $G$ is the event that a number greater than
3 occurs on a single roll of the die.
\\
\solution
		%\input{exemplar/11/16/3/5/main.tex}
	\item All the jacks, queens and kings are removed from a deck of 52 playing cards. The remaining cards are well shuffled and then one card is drawn at random. Giving ace a value 1 similar value for other cards, find the probability that the card has a value 
		\begin{enumerate}
			\item 7
			\item greater than 7
			\item less than 7
		\end{enumerate}
		%\input{exemplar/10/13/3/30/main.tex}
  \item A Lot consists of 48 mobile phones of which 42 are good, 3 have only minor defects and 3 have major defects.Varnika will buy a phone if it is good but the trader will only buy a mobile if it has no major defects. One phone is selected at random from the lot. What is the probability that it is
\begin{enumerate}
	\item acceptable to Varnika?
            \item acceptable to the trader?
\end{enumerate}
\solution
	%\input{exemplar/10/13/3/40/main.tex}
 \item A student says that if you throw a die, it will show up 1 or not 1. Therefore, the probability of getting 1 and the probability of getting 'not 1' each is equal to $\frac{1}{2}$. Is this correct? Give reasons.\\
 \solution
        %\input{exemplar/10/13/2/9/main.tex}
   \item Four candidates A, B, C, D have ap-
plied for the assignment to coach a school cricket
team. If A is twice as likely to be selected as B, and
B and C are given about the same chance of being
selected, while C is twice as likely to be selected
as D, what are the probabilities that
\begin{enumerate}
\item C will be selected?
\item A will not be selected?
\end{enumerate}
	%\input{exemplar/11/16/3/9/main.tex}
 \item A bag contain 24 balls of which $x$ balls are red, $2x$ are white and $3x$ are blue. A ball is selected at random, What is the probability that it is
\begin{enumerate}[label=\alph*)]
\item not red ?
\item white ?
\end{enumerate}
%\input{exemplar/10/13/3/41/main.tex}
If the letters of the word ASSASSINATION are arranged at random. Find the Probability that
\begin{enumerate}[label=(\alph*)]
\item Four $S's$ come consecutively in the word
\item Two  $I's$ and two $N's$ come together
\item All $A's$ are not coming together
\item No two $A's$ are coming together
\end{enumerate}
%\input{exemplar/11/16/3/14/main.tex}
	\item One urn contains two black balls (labelled B1 and B2) and one white ball. A
	second urn contains one black ball and two white balls (labelled W1 and W2).
	Suppose the following experiment is performed. One of the two urns is chosen
	at random. Next a ball is randomly chosen from the urn. Then a second ball is
	chosen at random from the same urn without replacing the first ball.
	
	\begin{enumerate}
	\item What is the probability that two black balls are chosen?
	
	\item What is the probability that two balls of opposite colour are chosen?
	\end{enumerate}
	\solution
	%\input{exemplar/11/16/3/12/main1.tex}
\end{enumerate}

		%
\item 
Out of 100 students, two sections of 40 and 60 are formed. If you and your friend are among the 100 students, what is the probability that
\begin{enumerate}
\item you both enter the same section?
\item you both enter the different sections?
\end{enumerate}
\solution
		%\begin{enumerate}[label=\thesection.\arabic*,ref=\thesection.\theenumi]
	\item One card is drawn from a well-shuffled deck of 52 cards. Find the probability of getting
\begin{enumerate}
\item A king of red colour 
\item A face card 
\item A red face card
\item The jack of hearts
\item A spade
\item The queen of diamonds

\end{enumerate}
\solution
		%\input{ncert/10/15/1/14/main.tex}
	\item Five cards—the ten, jack, queen, king and ace of diamonds, are well-shuffled with their face downwards. One card is then picked up at random.
\begin{enumerate}
\item
What is the probability that the card is the queen? 
\item
If the queen is drawn and put aside, what is the probability that the second card picked up is (a) an ace? (b) a queen?\\
\end{enumerate}
\solution
		%\input{ncert/10/15/1/15/defs.tex}
	\item A bag contains $5$ red balls and some blue balls. If the probability of drawing a blue ball is double that if a red ball, determine the number of blue balls in the bag. 
		\\
\solution
		%\input{ncert/10/15/2/3/defs.tex}
	\item A card is selected from a pack of 52 cards.
 \begin{enumerate}[label=(\alph*)] 
                 \item How many points are there in the sample space?
                 \item Calculate the probability that the card is an ace of spades.
                 \item Calculate the probability that the card is (i) an ace and (ii) black card.
 \end{enumerate}
\solution
		%\input{ncert/11/16/3/4/main.tex}
\item Four cards are drawn from a well-shuffled deck of 52 cards. What is the probability of obtaining 3 diamonds and one spade.
\\
\solution
		%\input{ncert/11/16/4/2/defs.tex}
\item In a certain lottery 10,000 tickets are sold and ten equal prizes are awarded. What is the probability of not getting a prize if you buy (a) one ticket (b) two tickets (c) 10 tickets ?	
\\
\solution
		%\input{ncert/11/16/4/4/defs.tex}
		%
\item 
Out of 100 students, two sections of 40 and 60 are formed. If you and your friend are among the 100 students, what is the probability that
\begin{enumerate}
\item you both enter the same section?
\item you both enter the different sections?
\end{enumerate}
\solution
		%\input{ncert/11/16/4/5/defs.tex}
	\item 
The number lock of a suitcase has 4 wheels each labelled with ten digits i.e. from 0 to 9.The lock opens with a sequence of four digits with no repeats.What is the probability of a person getting the right sequence to open the suitcase.
\\
\solution
		%\input{ncert/11/16/4/10/defs.tex}
		%
\item 
Two cards are drawn at random and without replacement from a pack of 52 playing cards. Find the probability that both the cards are black.
\\
\solution
		%\input{ncert/12/13/2/2/defs.tex}
		\item A box of oranges is inspected by examining three randomly selected oranges drawn without replacement. If all the three oranges are good, the box is approved for sale, otherwise, it is rejected. Find the probability that a box containing 15 oranges out of which 12 are good and 3 are bad ones will be approved for sale.
		\label{ncert/12/13/2/3/defs.tex}
		\item Two balls are drawn at random with replacement from a box containing 10 black and 8 red balls. Find the probability that
		\label{ncert/12/13/2/12}
\begin{enumerate}
\item both balls are red.
\item first ball is black and second is red.
\item one of them is black and other is red.
\end{enumerate}

\item In a hostel, 60\% of the students read Hindi newspaper, 40\% read English newspaper and 20\% read both Hindi and English newspapers. A student is selected at random.
		\label{ncert/12/13/2/15}
\begin{enumerate}
\item Find the probability that she reads neither Hindi nor English newspapers.
\item If she reads Hindi newspaper, find the probability that she reads English newspaper.
\item If she reads English newspaper, find the probability that she reads Hindi newspaper.\\
\end{enumerate}
\item The probability of obtaining an even prime number on each die, when a pair of dice is rolled is 
\begin{enumerate}
    \item $0$ 
    
    \item $\frac{1}{3}$ 
    
    \item $\frac{1}{12}$ 
    
    \item $\frac{1}{36}$ 
\end{enumerate}
\solution
		%\input{ncert/12/13/2/17/defs.tex}
	\item A bag contains 4 red and 4 black balls, another bag contains 2 red and 6 black balls. One of the two bags is selected at random and a ball is drawn from the bag which is found to be red. Find the probability that the ball is drawn from the first bag.
\\
\solution
		%\input{ncert/12/13/3/2/main.tex}
  \item
  Cards with numbers 2 to 101 are placed in a box. A card is selected at random.Find the probability that the card has
\begin{enumerate}[label=(\roman*)]
	\item an even number 
	\item a square number
\end{enumerate}
\solution
%\input{exemplar/10/13/3/32/main.tex}
\item
The king, queen and jack of clubs are removed from a deck of 52 playing cards and then well shuffled. Now one card is drawn at random from the remaining cards.  Determine the probability that the card is
\begin{enumerate}[label=(\roman*)]
\item a club
\item 10 of hearts
\end{enumerate}
\solution
%\input{exemplar/10/13/3/29/main.tex}
\item A team of medical students doing their internship have to assist during surgeries
at a city hospital. The probabilities of surgeries rated as very complex, complex,
routine, simple or very simple are respectively, 0.15, 0.20, 0.31, 0.26, .08. Find
the probabilities that a particular surgery will be rated
\begin{enumerate}
	\item complex or very complex;
	\item neither very complex nor very simple;
	\item routine or complex
	\item routine or simple
\end{enumerate}
\solution
%\input{exemplar/11/16/3/8(1)/main.tex}
\item A card is selected from a pack of 52 cards.
\begin{enumerate}[label=(\alph*)]
    \item How many points are there in the sample space?
    \item Calculate the probability that the card is an ace of spades.
    \item Calculate the probability that the card is (i) an ace and (ii) black card.
\end{enumerate}
\solution
%\input{exemplar/11/16/3/4/main2.tex}
\item The probability that a non leap year selected at random will contain 53 sundays.
\\
\solution
%\input{exemplar/10/13/1/19/main.tex}
\item One of the four persons John, Rita, Aslam or Gurpreet will be promoted next
month. Consequently the sample space consists of four elementary outcomes
S = {John promoted, Rita promoted, Aslam promoted, Gurpreet promoted}
You are told that the chances of John’s promotion is same as that of Gurpreet,
Rita’s chances of promotion are twice as likely as Johns. Aslam’s chances are
four times that of John.
\begin{enumerate}
	\item Determine
	\begin{enumerate}
		\item P (John promoted)
		\item P (Rita promoted)
		\item P (Aslam promoted)
		\item P (Gurpreet promoted)
	\end{enumerate}
	\item If A = {John promoted or Gurpreet promoted}, find P (A).
\end{enumerate}
\solution
%\input{exemplar/11/16/3/10/main.tex}
\item A card is drawn from a deck of 52 cards. Find the probability of getting a king or a heart or a red card.\\
\solution
%\input{exemplar/11/16/3/15/main.tex}
\item The probability that a student will pass his examination is 0.73, the probability of
the student getting a compartment is 0.13, and the probability that the student will
either pass or get compartment is 0.96. State True or False.\\
\solution
%\input{exemplar/11/16/3/31/main.tex}
\item A card is selected from a pack of 52 cards\\
\begin{enumerate}[label=(\alph*)]
\item How many points are there in the sample space?
\item Calculate the probability that the cards is an ace of spades.
\item Calculate the probability that the card is (i) an ace (ii)black card.\\
\end{enumerate}
%\input{ncert/11/16/3/4_1/Prob_4.tex}
\item In a non-leap year, the probability of having 53 tuesdays or 53 wednesdays is\\
\solution
%\input{exemplar/11/16/3/18/main.tex}
\item There are 1000 sealed envelopes in a box, 10 of them contain a cash prize of
Rs 100 each, 100 of them contain a cash prize of Rs 50 each and 200 of them
contain a cash prize of Rs 10 each and rest do not contain any cash prize. If they
are well shuffled and an envelope is picked up out, what is the probability that it
contains no cash prize?\\
\solution
%\input{exemplar/10/13/3/34/main.tex}
\item 
A die is thrown and a card is selected at random from a deck of 52 playing cards. The probability of getting an even number on the die and a spade card.\\
\solution
%\input{exemplar/12/13/3/78/main.tex}
\item
If 4-digit numbers greater than 5,000 are randomly formed from the digits 0, 1, 3, 5, and 7, what is the probability of forming a number divisible by 5 when:
\begin{enumerate}
    \item The digits are repeated?
    \item The repetition of digits is not allowed?
\end{enumerate}
\solution
%\input{ncert/11/16/4/9/main.tex}
\item Consider the probability space $\brak{\Omega, \mathcal{G}, P}$ where $\Omega = [0,2]$ and $\mathcal{G} = \cbrak{\phi, \Omega, [0,1], (1,2]}$. Let $X$ and $Y$ be two functions on $\Omega$ defined as
\begin{align*}
    X(\omega) = 
    \begin{cases}
        1 & \text{if }\omega \in [0, 1]\\
        2 & \text{if }\omega \in (1, 2]
    \end{cases}
\end{align*}
and
\begin{align*}
    Y(\omega) = 
    \begin{cases}
        2 & \text{if }\omega \in [0, 1.5]\\
        3 & \text{if }\omega \in (1.5, 2].
    \end{cases}
\end{align*}
Then which one of the following statements is true?
\begin{enumerate}
    \item [(A)] $X$ is a random variable with respect to $\mathcal{G}$, but $Y$ is not a random variable with respect to $\mathcal{G}$.
    \item [(B)] $Y$ is a random variable with respect to $\mathcal{G}$, but $X$ is not a random variable with respect to $\mathcal{G}$.
    \item [(C)] Neither $X$ nor $Y$ is a random variable with respect to $\mathcal{G}$.
    \item [(D)] Both $X$ and $Y$ are random variables with respect to $\mathcal{G}$.
\end{enumerate} \hfill (GATE ST 2023)\\
\solution
%\input{gate/ST/2023/14/main.tex}
	\item  A die is loaded in such a way that each odd number is twice as likely to occur as
each even number. Find $P(G)$, where $G$ is the event that a number greater than
3 occurs on a single roll of the die.
\\
\solution
		%\input{exemplar/11/16/3/5/main.tex}
	\item All the jacks, queens and kings are removed from a deck of 52 playing cards. The remaining cards are well shuffled and then one card is drawn at random. Giving ace a value 1 similar value for other cards, find the probability that the card has a value 
		\begin{enumerate}
			\item 7
			\item greater than 7
			\item less than 7
		\end{enumerate}
		%\input{exemplar/10/13/3/30/main.tex}
  \item A Lot consists of 48 mobile phones of which 42 are good, 3 have only minor defects and 3 have major defects.Varnika will buy a phone if it is good but the trader will only buy a mobile if it has no major defects. One phone is selected at random from the lot. What is the probability that it is
\begin{enumerate}
	\item acceptable to Varnika?
            \item acceptable to the trader?
\end{enumerate}
\solution
	%\input{exemplar/10/13/3/40/main.tex}
 \item A student says that if you throw a die, it will show up 1 or not 1. Therefore, the probability of getting 1 and the probability of getting 'not 1' each is equal to $\frac{1}{2}$. Is this correct? Give reasons.\\
 \solution
        %\input{exemplar/10/13/2/9/main.tex}
   \item Four candidates A, B, C, D have ap-
plied for the assignment to coach a school cricket
team. If A is twice as likely to be selected as B, and
B and C are given about the same chance of being
selected, while C is twice as likely to be selected
as D, what are the probabilities that
\begin{enumerate}
\item C will be selected?
\item A will not be selected?
\end{enumerate}
	%\input{exemplar/11/16/3/9/main.tex}
 \item A bag contain 24 balls of which $x$ balls are red, $2x$ are white and $3x$ are blue. A ball is selected at random, What is the probability that it is
\begin{enumerate}[label=\alph*)]
\item not red ?
\item white ?
\end{enumerate}
%\input{exemplar/10/13/3/41/main.tex}
If the letters of the word ASSASSINATION are arranged at random. Find the Probability that
\begin{enumerate}[label=(\alph*)]
\item Four $S's$ come consecutively in the word
\item Two  $I's$ and two $N's$ come together
\item All $A's$ are not coming together
\item No two $A's$ are coming together
\end{enumerate}
%\input{exemplar/11/16/3/14/main.tex}
	\item One urn contains two black balls (labelled B1 and B2) and one white ball. A
	second urn contains one black ball and two white balls (labelled W1 and W2).
	Suppose the following experiment is performed. One of the two urns is chosen
	at random. Next a ball is randomly chosen from the urn. Then a second ball is
	chosen at random from the same urn without replacing the first ball.
	
	\begin{enumerate}
	\item What is the probability that two black balls are chosen?
	
	\item What is the probability that two balls of opposite colour are chosen?
	\end{enumerate}
	\solution
	%\input{exemplar/11/16/3/12/main1.tex}
\end{enumerate}

	\item 
The number lock of a suitcase has 4 wheels each labelled with ten digits i.e. from 0 to 9.The lock opens with a sequence of four digits with no repeats.What is the probability of a person getting the right sequence to open the suitcase.
\\
\solution
		%\begin{enumerate}[label=\thesection.\arabic*,ref=\thesection.\theenumi]
	\item One card is drawn from a well-shuffled deck of 52 cards. Find the probability of getting
\begin{enumerate}
\item A king of red colour 
\item A face card 
\item A red face card
\item The jack of hearts
\item A spade
\item The queen of diamonds

\end{enumerate}
\solution
		%\input{ncert/10/15/1/14/main.tex}
	\item Five cards—the ten, jack, queen, king and ace of diamonds, are well-shuffled with their face downwards. One card is then picked up at random.
\begin{enumerate}
\item
What is the probability that the card is the queen? 
\item
If the queen is drawn and put aside, what is the probability that the second card picked up is (a) an ace? (b) a queen?\\
\end{enumerate}
\solution
		%\input{ncert/10/15/1/15/defs.tex}
	\item A bag contains $5$ red balls and some blue balls. If the probability of drawing a blue ball is double that if a red ball, determine the number of blue balls in the bag. 
		\\
\solution
		%\input{ncert/10/15/2/3/defs.tex}
	\item A card is selected from a pack of 52 cards.
 \begin{enumerate}[label=(\alph*)] 
                 \item How many points are there in the sample space?
                 \item Calculate the probability that the card is an ace of spades.
                 \item Calculate the probability that the card is (i) an ace and (ii) black card.
 \end{enumerate}
\solution
		%\input{ncert/11/16/3/4/main.tex}
\item Four cards are drawn from a well-shuffled deck of 52 cards. What is the probability of obtaining 3 diamonds and one spade.
\\
\solution
		%\input{ncert/11/16/4/2/defs.tex}
\item In a certain lottery 10,000 tickets are sold and ten equal prizes are awarded. What is the probability of not getting a prize if you buy (a) one ticket (b) two tickets (c) 10 tickets ?	
\\
\solution
		%\input{ncert/11/16/4/4/defs.tex}
		%
\item 
Out of 100 students, two sections of 40 and 60 are formed. If you and your friend are among the 100 students, what is the probability that
\begin{enumerate}
\item you both enter the same section?
\item you both enter the different sections?
\end{enumerate}
\solution
		%\input{ncert/11/16/4/5/defs.tex}
	\item 
The number lock of a suitcase has 4 wheels each labelled with ten digits i.e. from 0 to 9.The lock opens with a sequence of four digits with no repeats.What is the probability of a person getting the right sequence to open the suitcase.
\\
\solution
		%\input{ncert/11/16/4/10/defs.tex}
		%
\item 
Two cards are drawn at random and without replacement from a pack of 52 playing cards. Find the probability that both the cards are black.
\\
\solution
		%\input{ncert/12/13/2/2/defs.tex}
		\item A box of oranges is inspected by examining three randomly selected oranges drawn without replacement. If all the three oranges are good, the box is approved for sale, otherwise, it is rejected. Find the probability that a box containing 15 oranges out of which 12 are good and 3 are bad ones will be approved for sale.
		\label{ncert/12/13/2/3/defs.tex}
		\item Two balls are drawn at random with replacement from a box containing 10 black and 8 red balls. Find the probability that
		\label{ncert/12/13/2/12}
\begin{enumerate}
\item both balls are red.
\item first ball is black and second is red.
\item one of them is black and other is red.
\end{enumerate}

\item In a hostel, 60\% of the students read Hindi newspaper, 40\% read English newspaper and 20\% read both Hindi and English newspapers. A student is selected at random.
		\label{ncert/12/13/2/15}
\begin{enumerate}
\item Find the probability that she reads neither Hindi nor English newspapers.
\item If she reads Hindi newspaper, find the probability that she reads English newspaper.
\item If she reads English newspaper, find the probability that she reads Hindi newspaper.\\
\end{enumerate}
\item The probability of obtaining an even prime number on each die, when a pair of dice is rolled is 
\begin{enumerate}
    \item $0$ 
    
    \item $\frac{1}{3}$ 
    
    \item $\frac{1}{12}$ 
    
    \item $\frac{1}{36}$ 
\end{enumerate}
\solution
		%\input{ncert/12/13/2/17/defs.tex}
	\item A bag contains 4 red and 4 black balls, another bag contains 2 red and 6 black balls. One of the two bags is selected at random and a ball is drawn from the bag which is found to be red. Find the probability that the ball is drawn from the first bag.
\\
\solution
		%\input{ncert/12/13/3/2/main.tex}
  \item
  Cards with numbers 2 to 101 are placed in a box. A card is selected at random.Find the probability that the card has
\begin{enumerate}[label=(\roman*)]
	\item an even number 
	\item a square number
\end{enumerate}
\solution
%\input{exemplar/10/13/3/32/main.tex}
\item
The king, queen and jack of clubs are removed from a deck of 52 playing cards and then well shuffled. Now one card is drawn at random from the remaining cards.  Determine the probability that the card is
\begin{enumerate}[label=(\roman*)]
\item a club
\item 10 of hearts
\end{enumerate}
\solution
%\input{exemplar/10/13/3/29/main.tex}
\item A team of medical students doing their internship have to assist during surgeries
at a city hospital. The probabilities of surgeries rated as very complex, complex,
routine, simple or very simple are respectively, 0.15, 0.20, 0.31, 0.26, .08. Find
the probabilities that a particular surgery will be rated
\begin{enumerate}
	\item complex or very complex;
	\item neither very complex nor very simple;
	\item routine or complex
	\item routine or simple
\end{enumerate}
\solution
%\input{exemplar/11/16/3/8(1)/main.tex}
\item A card is selected from a pack of 52 cards.
\begin{enumerate}[label=(\alph*)]
    \item How many points are there in the sample space?
    \item Calculate the probability that the card is an ace of spades.
    \item Calculate the probability that the card is (i) an ace and (ii) black card.
\end{enumerate}
\solution
%\input{exemplar/11/16/3/4/main2.tex}
\item The probability that a non leap year selected at random will contain 53 sundays.
\\
\solution
%\input{exemplar/10/13/1/19/main.tex}
\item One of the four persons John, Rita, Aslam or Gurpreet will be promoted next
month. Consequently the sample space consists of four elementary outcomes
S = {John promoted, Rita promoted, Aslam promoted, Gurpreet promoted}
You are told that the chances of John’s promotion is same as that of Gurpreet,
Rita’s chances of promotion are twice as likely as Johns. Aslam’s chances are
four times that of John.
\begin{enumerate}
	\item Determine
	\begin{enumerate}
		\item P (John promoted)
		\item P (Rita promoted)
		\item P (Aslam promoted)
		\item P (Gurpreet promoted)
	\end{enumerate}
	\item If A = {John promoted or Gurpreet promoted}, find P (A).
\end{enumerate}
\solution
%\input{exemplar/11/16/3/10/main.tex}
\item A card is drawn from a deck of 52 cards. Find the probability of getting a king or a heart or a red card.\\
\solution
%\input{exemplar/11/16/3/15/main.tex}
\item The probability that a student will pass his examination is 0.73, the probability of
the student getting a compartment is 0.13, and the probability that the student will
either pass or get compartment is 0.96. State True or False.\\
\solution
%\input{exemplar/11/16/3/31/main.tex}
\item A card is selected from a pack of 52 cards\\
\begin{enumerate}[label=(\alph*)]
\item How many points are there in the sample space?
\item Calculate the probability that the cards is an ace of spades.
\item Calculate the probability that the card is (i) an ace (ii)black card.\\
\end{enumerate}
%\input{ncert/11/16/3/4_1/Prob_4.tex}
\item In a non-leap year, the probability of having 53 tuesdays or 53 wednesdays is\\
\solution
%\input{exemplar/11/16/3/18/main.tex}
\item There are 1000 sealed envelopes in a box, 10 of them contain a cash prize of
Rs 100 each, 100 of them contain a cash prize of Rs 50 each and 200 of them
contain a cash prize of Rs 10 each and rest do not contain any cash prize. If they
are well shuffled and an envelope is picked up out, what is the probability that it
contains no cash prize?\\
\solution
%\input{exemplar/10/13/3/34/main.tex}
\item 
A die is thrown and a card is selected at random from a deck of 52 playing cards. The probability of getting an even number on the die and a spade card.\\
\solution
%\input{exemplar/12/13/3/78/main.tex}
\item
If 4-digit numbers greater than 5,000 are randomly formed from the digits 0, 1, 3, 5, and 7, what is the probability of forming a number divisible by 5 when:
\begin{enumerate}
    \item The digits are repeated?
    \item The repetition of digits is not allowed?
\end{enumerate}
\solution
%\input{ncert/11/16/4/9/main.tex}
\item Consider the probability space $\brak{\Omega, \mathcal{G}, P}$ where $\Omega = [0,2]$ and $\mathcal{G} = \cbrak{\phi, \Omega, [0,1], (1,2]}$. Let $X$ and $Y$ be two functions on $\Omega$ defined as
\begin{align*}
    X(\omega) = 
    \begin{cases}
        1 & \text{if }\omega \in [0, 1]\\
        2 & \text{if }\omega \in (1, 2]
    \end{cases}
\end{align*}
and
\begin{align*}
    Y(\omega) = 
    \begin{cases}
        2 & \text{if }\omega \in [0, 1.5]\\
        3 & \text{if }\omega \in (1.5, 2].
    \end{cases}
\end{align*}
Then which one of the following statements is true?
\begin{enumerate}
    \item [(A)] $X$ is a random variable with respect to $\mathcal{G}$, but $Y$ is not a random variable with respect to $\mathcal{G}$.
    \item [(B)] $Y$ is a random variable with respect to $\mathcal{G}$, but $X$ is not a random variable with respect to $\mathcal{G}$.
    \item [(C)] Neither $X$ nor $Y$ is a random variable with respect to $\mathcal{G}$.
    \item [(D)] Both $X$ and $Y$ are random variables with respect to $\mathcal{G}$.
\end{enumerate} \hfill (GATE ST 2023)\\
\solution
%\input{gate/ST/2023/14/main.tex}
	\item  A die is loaded in such a way that each odd number is twice as likely to occur as
each even number. Find $P(G)$, where $G$ is the event that a number greater than
3 occurs on a single roll of the die.
\\
\solution
		%\input{exemplar/11/16/3/5/main.tex}
	\item All the jacks, queens and kings are removed from a deck of 52 playing cards. The remaining cards are well shuffled and then one card is drawn at random. Giving ace a value 1 similar value for other cards, find the probability that the card has a value 
		\begin{enumerate}
			\item 7
			\item greater than 7
			\item less than 7
		\end{enumerate}
		%\input{exemplar/10/13/3/30/main.tex}
  \item A Lot consists of 48 mobile phones of which 42 are good, 3 have only minor defects and 3 have major defects.Varnika will buy a phone if it is good but the trader will only buy a mobile if it has no major defects. One phone is selected at random from the lot. What is the probability that it is
\begin{enumerate}
	\item acceptable to Varnika?
            \item acceptable to the trader?
\end{enumerate}
\solution
	%\input{exemplar/10/13/3/40/main.tex}
 \item A student says that if you throw a die, it will show up 1 or not 1. Therefore, the probability of getting 1 and the probability of getting 'not 1' each is equal to $\frac{1}{2}$. Is this correct? Give reasons.\\
 \solution
        %\input{exemplar/10/13/2/9/main.tex}
   \item Four candidates A, B, C, D have ap-
plied for the assignment to coach a school cricket
team. If A is twice as likely to be selected as B, and
B and C are given about the same chance of being
selected, while C is twice as likely to be selected
as D, what are the probabilities that
\begin{enumerate}
\item C will be selected?
\item A will not be selected?
\end{enumerate}
	%\input{exemplar/11/16/3/9/main.tex}
 \item A bag contain 24 balls of which $x$ balls are red, $2x$ are white and $3x$ are blue. A ball is selected at random, What is the probability that it is
\begin{enumerate}[label=\alph*)]
\item not red ?
\item white ?
\end{enumerate}
%\input{exemplar/10/13/3/41/main.tex}
If the letters of the word ASSASSINATION are arranged at random. Find the Probability that
\begin{enumerate}[label=(\alph*)]
\item Four $S's$ come consecutively in the word
\item Two  $I's$ and two $N's$ come together
\item All $A's$ are not coming together
\item No two $A's$ are coming together
\end{enumerate}
%\input{exemplar/11/16/3/14/main.tex}
	\item One urn contains two black balls (labelled B1 and B2) and one white ball. A
	second urn contains one black ball and two white balls (labelled W1 and W2).
	Suppose the following experiment is performed. One of the two urns is chosen
	at random. Next a ball is randomly chosen from the urn. Then a second ball is
	chosen at random from the same urn without replacing the first ball.
	
	\begin{enumerate}
	\item What is the probability that two black balls are chosen?
	
	\item What is the probability that two balls of opposite colour are chosen?
	\end{enumerate}
	\solution
	%\input{exemplar/11/16/3/12/main1.tex}
\end{enumerate}

		%
\item 
Two cards are drawn at random and without replacement from a pack of 52 playing cards. Find the probability that both the cards are black.
\\
\solution
		%\begin{enumerate}[label=\thesection.\arabic*,ref=\thesection.\theenumi]
	\item One card is drawn from a well-shuffled deck of 52 cards. Find the probability of getting
\begin{enumerate}
\item A king of red colour 
\item A face card 
\item A red face card
\item The jack of hearts
\item A spade
\item The queen of diamonds

\end{enumerate}
\solution
		%\input{ncert/10/15/1/14/main.tex}
	\item Five cards—the ten, jack, queen, king and ace of diamonds, are well-shuffled with their face downwards. One card is then picked up at random.
\begin{enumerate}
\item
What is the probability that the card is the queen? 
\item
If the queen is drawn and put aside, what is the probability that the second card picked up is (a) an ace? (b) a queen?\\
\end{enumerate}
\solution
		%\input{ncert/10/15/1/15/defs.tex}
	\item A bag contains $5$ red balls and some blue balls. If the probability of drawing a blue ball is double that if a red ball, determine the number of blue balls in the bag. 
		\\
\solution
		%\input{ncert/10/15/2/3/defs.tex}
	\item A card is selected from a pack of 52 cards.
 \begin{enumerate}[label=(\alph*)] 
                 \item How many points are there in the sample space?
                 \item Calculate the probability that the card is an ace of spades.
                 \item Calculate the probability that the card is (i) an ace and (ii) black card.
 \end{enumerate}
\solution
		%\input{ncert/11/16/3/4/main.tex}
\item Four cards are drawn from a well-shuffled deck of 52 cards. What is the probability of obtaining 3 diamonds and one spade.
\\
\solution
		%\input{ncert/11/16/4/2/defs.tex}
\item In a certain lottery 10,000 tickets are sold and ten equal prizes are awarded. What is the probability of not getting a prize if you buy (a) one ticket (b) two tickets (c) 10 tickets ?	
\\
\solution
		%\input{ncert/11/16/4/4/defs.tex}
		%
\item 
Out of 100 students, two sections of 40 and 60 are formed. If you and your friend are among the 100 students, what is the probability that
\begin{enumerate}
\item you both enter the same section?
\item you both enter the different sections?
\end{enumerate}
\solution
		%\input{ncert/11/16/4/5/defs.tex}
	\item 
The number lock of a suitcase has 4 wheels each labelled with ten digits i.e. from 0 to 9.The lock opens with a sequence of four digits with no repeats.What is the probability of a person getting the right sequence to open the suitcase.
\\
\solution
		%\input{ncert/11/16/4/10/defs.tex}
		%
\item 
Two cards are drawn at random and without replacement from a pack of 52 playing cards. Find the probability that both the cards are black.
\\
\solution
		%\input{ncert/12/13/2/2/defs.tex}
		\item A box of oranges is inspected by examining three randomly selected oranges drawn without replacement. If all the three oranges are good, the box is approved for sale, otherwise, it is rejected. Find the probability that a box containing 15 oranges out of which 12 are good and 3 are bad ones will be approved for sale.
		\label{ncert/12/13/2/3/defs.tex}
		\item Two balls are drawn at random with replacement from a box containing 10 black and 8 red balls. Find the probability that
		\label{ncert/12/13/2/12}
\begin{enumerate}
\item both balls are red.
\item first ball is black and second is red.
\item one of them is black and other is red.
\end{enumerate}

\item In a hostel, 60\% of the students read Hindi newspaper, 40\% read English newspaper and 20\% read both Hindi and English newspapers. A student is selected at random.
		\label{ncert/12/13/2/15}
\begin{enumerate}
\item Find the probability that she reads neither Hindi nor English newspapers.
\item If she reads Hindi newspaper, find the probability that she reads English newspaper.
\item If she reads English newspaper, find the probability that she reads Hindi newspaper.\\
\end{enumerate}
\item The probability of obtaining an even prime number on each die, when a pair of dice is rolled is 
\begin{enumerate}
    \item $0$ 
    
    \item $\frac{1}{3}$ 
    
    \item $\frac{1}{12}$ 
    
    \item $\frac{1}{36}$ 
\end{enumerate}
\solution
		%\input{ncert/12/13/2/17/defs.tex}
	\item A bag contains 4 red and 4 black balls, another bag contains 2 red and 6 black balls. One of the two bags is selected at random and a ball is drawn from the bag which is found to be red. Find the probability that the ball is drawn from the first bag.
\\
\solution
		%\input{ncert/12/13/3/2/main.tex}
  \item
  Cards with numbers 2 to 101 are placed in a box. A card is selected at random.Find the probability that the card has
\begin{enumerate}[label=(\roman*)]
	\item an even number 
	\item a square number
\end{enumerate}
\solution
%\input{exemplar/10/13/3/32/main.tex}
\item
The king, queen and jack of clubs are removed from a deck of 52 playing cards and then well shuffled. Now one card is drawn at random from the remaining cards.  Determine the probability that the card is
\begin{enumerate}[label=(\roman*)]
\item a club
\item 10 of hearts
\end{enumerate}
\solution
%\input{exemplar/10/13/3/29/main.tex}
\item A team of medical students doing their internship have to assist during surgeries
at a city hospital. The probabilities of surgeries rated as very complex, complex,
routine, simple or very simple are respectively, 0.15, 0.20, 0.31, 0.26, .08. Find
the probabilities that a particular surgery will be rated
\begin{enumerate}
	\item complex or very complex;
	\item neither very complex nor very simple;
	\item routine or complex
	\item routine or simple
\end{enumerate}
\solution
%\input{exemplar/11/16/3/8(1)/main.tex}
\item A card is selected from a pack of 52 cards.
\begin{enumerate}[label=(\alph*)]
    \item How many points are there in the sample space?
    \item Calculate the probability that the card is an ace of spades.
    \item Calculate the probability that the card is (i) an ace and (ii) black card.
\end{enumerate}
\solution
%\input{exemplar/11/16/3/4/main2.tex}
\item The probability that a non leap year selected at random will contain 53 sundays.
\\
\solution
%\input{exemplar/10/13/1/19/main.tex}
\item One of the four persons John, Rita, Aslam or Gurpreet will be promoted next
month. Consequently the sample space consists of four elementary outcomes
S = {John promoted, Rita promoted, Aslam promoted, Gurpreet promoted}
You are told that the chances of John’s promotion is same as that of Gurpreet,
Rita’s chances of promotion are twice as likely as Johns. Aslam’s chances are
four times that of John.
\begin{enumerate}
	\item Determine
	\begin{enumerate}
		\item P (John promoted)
		\item P (Rita promoted)
		\item P (Aslam promoted)
		\item P (Gurpreet promoted)
	\end{enumerate}
	\item If A = {John promoted or Gurpreet promoted}, find P (A).
\end{enumerate}
\solution
%\input{exemplar/11/16/3/10/main.tex}
\item A card is drawn from a deck of 52 cards. Find the probability of getting a king or a heart or a red card.\\
\solution
%\input{exemplar/11/16/3/15/main.tex}
\item The probability that a student will pass his examination is 0.73, the probability of
the student getting a compartment is 0.13, and the probability that the student will
either pass or get compartment is 0.96. State True or False.\\
\solution
%\input{exemplar/11/16/3/31/main.tex}
\item A card is selected from a pack of 52 cards\\
\begin{enumerate}[label=(\alph*)]
\item How many points are there in the sample space?
\item Calculate the probability that the cards is an ace of spades.
\item Calculate the probability that the card is (i) an ace (ii)black card.\\
\end{enumerate}
%\input{ncert/11/16/3/4_1/Prob_4.tex}
\item In a non-leap year, the probability of having 53 tuesdays or 53 wednesdays is\\
\solution
%\input{exemplar/11/16/3/18/main.tex}
\item There are 1000 sealed envelopes in a box, 10 of them contain a cash prize of
Rs 100 each, 100 of them contain a cash prize of Rs 50 each and 200 of them
contain a cash prize of Rs 10 each and rest do not contain any cash prize. If they
are well shuffled and an envelope is picked up out, what is the probability that it
contains no cash prize?\\
\solution
%\input{exemplar/10/13/3/34/main.tex}
\item 
A die is thrown and a card is selected at random from a deck of 52 playing cards. The probability of getting an even number on the die and a spade card.\\
\solution
%\input{exemplar/12/13/3/78/main.tex}
\item
If 4-digit numbers greater than 5,000 are randomly formed from the digits 0, 1, 3, 5, and 7, what is the probability of forming a number divisible by 5 when:
\begin{enumerate}
    \item The digits are repeated?
    \item The repetition of digits is not allowed?
\end{enumerate}
\solution
%\input{ncert/11/16/4/9/main.tex}
\item Consider the probability space $\brak{\Omega, \mathcal{G}, P}$ where $\Omega = [0,2]$ and $\mathcal{G} = \cbrak{\phi, \Omega, [0,1], (1,2]}$. Let $X$ and $Y$ be two functions on $\Omega$ defined as
\begin{align*}
    X(\omega) = 
    \begin{cases}
        1 & \text{if }\omega \in [0, 1]\\
        2 & \text{if }\omega \in (1, 2]
    \end{cases}
\end{align*}
and
\begin{align*}
    Y(\omega) = 
    \begin{cases}
        2 & \text{if }\omega \in [0, 1.5]\\
        3 & \text{if }\omega \in (1.5, 2].
    \end{cases}
\end{align*}
Then which one of the following statements is true?
\begin{enumerate}
    \item [(A)] $X$ is a random variable with respect to $\mathcal{G}$, but $Y$ is not a random variable with respect to $\mathcal{G}$.
    \item [(B)] $Y$ is a random variable with respect to $\mathcal{G}$, but $X$ is not a random variable with respect to $\mathcal{G}$.
    \item [(C)] Neither $X$ nor $Y$ is a random variable with respect to $\mathcal{G}$.
    \item [(D)] Both $X$ and $Y$ are random variables with respect to $\mathcal{G}$.
\end{enumerate} \hfill (GATE ST 2023)\\
\solution
%\input{gate/ST/2023/14/main.tex}
	\item  A die is loaded in such a way that each odd number is twice as likely to occur as
each even number. Find $P(G)$, where $G$ is the event that a number greater than
3 occurs on a single roll of the die.
\\
\solution
		%\input{exemplar/11/16/3/5/main.tex}
	\item All the jacks, queens and kings are removed from a deck of 52 playing cards. The remaining cards are well shuffled and then one card is drawn at random. Giving ace a value 1 similar value for other cards, find the probability that the card has a value 
		\begin{enumerate}
			\item 7
			\item greater than 7
			\item less than 7
		\end{enumerate}
		%\input{exemplar/10/13/3/30/main.tex}
  \item A Lot consists of 48 mobile phones of which 42 are good, 3 have only minor defects and 3 have major defects.Varnika will buy a phone if it is good but the trader will only buy a mobile if it has no major defects. One phone is selected at random from the lot. What is the probability that it is
\begin{enumerate}
	\item acceptable to Varnika?
            \item acceptable to the trader?
\end{enumerate}
\solution
	%\input{exemplar/10/13/3/40/main.tex}
 \item A student says that if you throw a die, it will show up 1 or not 1. Therefore, the probability of getting 1 and the probability of getting 'not 1' each is equal to $\frac{1}{2}$. Is this correct? Give reasons.\\
 \solution
        %\input{exemplar/10/13/2/9/main.tex}
   \item Four candidates A, B, C, D have ap-
plied for the assignment to coach a school cricket
team. If A is twice as likely to be selected as B, and
B and C are given about the same chance of being
selected, while C is twice as likely to be selected
as D, what are the probabilities that
\begin{enumerate}
\item C will be selected?
\item A will not be selected?
\end{enumerate}
	%\input{exemplar/11/16/3/9/main.tex}
 \item A bag contain 24 balls of which $x$ balls are red, $2x$ are white and $3x$ are blue. A ball is selected at random, What is the probability that it is
\begin{enumerate}[label=\alph*)]
\item not red ?
\item white ?
\end{enumerate}
%\input{exemplar/10/13/3/41/main.tex}
If the letters of the word ASSASSINATION are arranged at random. Find the Probability that
\begin{enumerate}[label=(\alph*)]
\item Four $S's$ come consecutively in the word
\item Two  $I's$ and two $N's$ come together
\item All $A's$ are not coming together
\item No two $A's$ are coming together
\end{enumerate}
%\input{exemplar/11/16/3/14/main.tex}
	\item One urn contains two black balls (labelled B1 and B2) and one white ball. A
	second urn contains one black ball and two white balls (labelled W1 and W2).
	Suppose the following experiment is performed. One of the two urns is chosen
	at random. Next a ball is randomly chosen from the urn. Then a second ball is
	chosen at random from the same urn without replacing the first ball.
	
	\begin{enumerate}
	\item What is the probability that two black balls are chosen?
	
	\item What is the probability that two balls of opposite colour are chosen?
	\end{enumerate}
	\solution
	%\input{exemplar/11/16/3/12/main1.tex}
\end{enumerate}

		\item A box of oranges is inspected by examining three randomly selected oranges drawn without replacement. If all the three oranges are good, the box is approved for sale, otherwise, it is rejected. Find the probability that a box containing 15 oranges out of which 12 are good and 3 are bad ones will be approved for sale.
		\label{ncert/12/13/2/3/defs.tex}
		\item Two balls are drawn at random with replacement from a box containing 10 black and 8 red balls. Find the probability that
		\label{ncert/12/13/2/12}
\begin{enumerate}
\item both balls are red.
\item first ball is black and second is red.
\item one of them is black and other is red.
\end{enumerate}

\item In a hostel, 60\% of the students read Hindi newspaper, 40\% read English newspaper and 20\% read both Hindi and English newspapers. A student is selected at random.
		\label{ncert/12/13/2/15}
\begin{enumerate}
\item Find the probability that she reads neither Hindi nor English newspapers.
\item If she reads Hindi newspaper, find the probability that she reads English newspaper.
\item If she reads English newspaper, find the probability that she reads Hindi newspaper.\\
\end{enumerate}
\item The probability of obtaining an even prime number on each die, when a pair of dice is rolled is 
\begin{enumerate}
    \item $0$ 
    
    \item $\frac{1}{3}$ 
    
    \item $\frac{1}{12}$ 
    
    \item $\frac{1}{36}$ 
\end{enumerate}
\solution
		%\begin{enumerate}[label=\thesection.\arabic*,ref=\thesection.\theenumi]
	\item One card is drawn from a well-shuffled deck of 52 cards. Find the probability of getting
\begin{enumerate}
\item A king of red colour 
\item A face card 
\item A red face card
\item The jack of hearts
\item A spade
\item The queen of diamonds

\end{enumerate}
\solution
		%\input{ncert/10/15/1/14/main.tex}
	\item Five cards—the ten, jack, queen, king and ace of diamonds, are well-shuffled with their face downwards. One card is then picked up at random.
\begin{enumerate}
\item
What is the probability that the card is the queen? 
\item
If the queen is drawn and put aside, what is the probability that the second card picked up is (a) an ace? (b) a queen?\\
\end{enumerate}
\solution
		%\input{ncert/10/15/1/15/defs.tex}
	\item A bag contains $5$ red balls and some blue balls. If the probability of drawing a blue ball is double that if a red ball, determine the number of blue balls in the bag. 
		\\
\solution
		%\input{ncert/10/15/2/3/defs.tex}
	\item A card is selected from a pack of 52 cards.
 \begin{enumerate}[label=(\alph*)] 
                 \item How many points are there in the sample space?
                 \item Calculate the probability that the card is an ace of spades.
                 \item Calculate the probability that the card is (i) an ace and (ii) black card.
 \end{enumerate}
\solution
		%\input{ncert/11/16/3/4/main.tex}
\item Four cards are drawn from a well-shuffled deck of 52 cards. What is the probability of obtaining 3 diamonds and one spade.
\\
\solution
		%\input{ncert/11/16/4/2/defs.tex}
\item In a certain lottery 10,000 tickets are sold and ten equal prizes are awarded. What is the probability of not getting a prize if you buy (a) one ticket (b) two tickets (c) 10 tickets ?	
\\
\solution
		%\input{ncert/11/16/4/4/defs.tex}
		%
\item 
Out of 100 students, two sections of 40 and 60 are formed. If you and your friend are among the 100 students, what is the probability that
\begin{enumerate}
\item you both enter the same section?
\item you both enter the different sections?
\end{enumerate}
\solution
		%\input{ncert/11/16/4/5/defs.tex}
	\item 
The number lock of a suitcase has 4 wheels each labelled with ten digits i.e. from 0 to 9.The lock opens with a sequence of four digits with no repeats.What is the probability of a person getting the right sequence to open the suitcase.
\\
\solution
		%\input{ncert/11/16/4/10/defs.tex}
		%
\item 
Two cards are drawn at random and without replacement from a pack of 52 playing cards. Find the probability that both the cards are black.
\\
\solution
		%\input{ncert/12/13/2/2/defs.tex}
		\item A box of oranges is inspected by examining three randomly selected oranges drawn without replacement. If all the three oranges are good, the box is approved for sale, otherwise, it is rejected. Find the probability that a box containing 15 oranges out of which 12 are good and 3 are bad ones will be approved for sale.
		\label{ncert/12/13/2/3/defs.tex}
		\item Two balls are drawn at random with replacement from a box containing 10 black and 8 red balls. Find the probability that
		\label{ncert/12/13/2/12}
\begin{enumerate}
\item both balls are red.
\item first ball is black and second is red.
\item one of them is black and other is red.
\end{enumerate}

\item In a hostel, 60\% of the students read Hindi newspaper, 40\% read English newspaper and 20\% read both Hindi and English newspapers. A student is selected at random.
		\label{ncert/12/13/2/15}
\begin{enumerate}
\item Find the probability that she reads neither Hindi nor English newspapers.
\item If she reads Hindi newspaper, find the probability that she reads English newspaper.
\item If she reads English newspaper, find the probability that she reads Hindi newspaper.\\
\end{enumerate}
\item The probability of obtaining an even prime number on each die, when a pair of dice is rolled is 
\begin{enumerate}
    \item $0$ 
    
    \item $\frac{1}{3}$ 
    
    \item $\frac{1}{12}$ 
    
    \item $\frac{1}{36}$ 
\end{enumerate}
\solution
		%\input{ncert/12/13/2/17/defs.tex}
	\item A bag contains 4 red and 4 black balls, another bag contains 2 red and 6 black balls. One of the two bags is selected at random and a ball is drawn from the bag which is found to be red. Find the probability that the ball is drawn from the first bag.
\\
\solution
		%\input{ncert/12/13/3/2/main.tex}
  \item
  Cards with numbers 2 to 101 are placed in a box. A card is selected at random.Find the probability that the card has
\begin{enumerate}[label=(\roman*)]
	\item an even number 
	\item a square number
\end{enumerate}
\solution
%\input{exemplar/10/13/3/32/main.tex}
\item
The king, queen and jack of clubs are removed from a deck of 52 playing cards and then well shuffled. Now one card is drawn at random from the remaining cards.  Determine the probability that the card is
\begin{enumerate}[label=(\roman*)]
\item a club
\item 10 of hearts
\end{enumerate}
\solution
%\input{exemplar/10/13/3/29/main.tex}
\item A team of medical students doing their internship have to assist during surgeries
at a city hospital. The probabilities of surgeries rated as very complex, complex,
routine, simple or very simple are respectively, 0.15, 0.20, 0.31, 0.26, .08. Find
the probabilities that a particular surgery will be rated
\begin{enumerate}
	\item complex or very complex;
	\item neither very complex nor very simple;
	\item routine or complex
	\item routine or simple
\end{enumerate}
\solution
%\input{exemplar/11/16/3/8(1)/main.tex}
\item A card is selected from a pack of 52 cards.
\begin{enumerate}[label=(\alph*)]
    \item How many points are there in the sample space?
    \item Calculate the probability that the card is an ace of spades.
    \item Calculate the probability that the card is (i) an ace and (ii) black card.
\end{enumerate}
\solution
%\input{exemplar/11/16/3/4/main2.tex}
\item The probability that a non leap year selected at random will contain 53 sundays.
\\
\solution
%\input{exemplar/10/13/1/19/main.tex}
\item One of the four persons John, Rita, Aslam or Gurpreet will be promoted next
month. Consequently the sample space consists of four elementary outcomes
S = {John promoted, Rita promoted, Aslam promoted, Gurpreet promoted}
You are told that the chances of John’s promotion is same as that of Gurpreet,
Rita’s chances of promotion are twice as likely as Johns. Aslam’s chances are
four times that of John.
\begin{enumerate}
	\item Determine
	\begin{enumerate}
		\item P (John promoted)
		\item P (Rita promoted)
		\item P (Aslam promoted)
		\item P (Gurpreet promoted)
	\end{enumerate}
	\item If A = {John promoted or Gurpreet promoted}, find P (A).
\end{enumerate}
\solution
%\input{exemplar/11/16/3/10/main.tex}
\item A card is drawn from a deck of 52 cards. Find the probability of getting a king or a heart or a red card.\\
\solution
%\input{exemplar/11/16/3/15/main.tex}
\item The probability that a student will pass his examination is 0.73, the probability of
the student getting a compartment is 0.13, and the probability that the student will
either pass or get compartment is 0.96. State True or False.\\
\solution
%\input{exemplar/11/16/3/31/main.tex}
\item A card is selected from a pack of 52 cards\\
\begin{enumerate}[label=(\alph*)]
\item How many points are there in the sample space?
\item Calculate the probability that the cards is an ace of spades.
\item Calculate the probability that the card is (i) an ace (ii)black card.\\
\end{enumerate}
%\input{ncert/11/16/3/4_1/Prob_4.tex}
\item In a non-leap year, the probability of having 53 tuesdays or 53 wednesdays is\\
\solution
%\input{exemplar/11/16/3/18/main.tex}
\item There are 1000 sealed envelopes in a box, 10 of them contain a cash prize of
Rs 100 each, 100 of them contain a cash prize of Rs 50 each and 200 of them
contain a cash prize of Rs 10 each and rest do not contain any cash prize. If they
are well shuffled and an envelope is picked up out, what is the probability that it
contains no cash prize?\\
\solution
%\input{exemplar/10/13/3/34/main.tex}
\item 
A die is thrown and a card is selected at random from a deck of 52 playing cards. The probability of getting an even number on the die and a spade card.\\
\solution
%\input{exemplar/12/13/3/78/main.tex}
\item
If 4-digit numbers greater than 5,000 are randomly formed from the digits 0, 1, 3, 5, and 7, what is the probability of forming a number divisible by 5 when:
\begin{enumerate}
    \item The digits are repeated?
    \item The repetition of digits is not allowed?
\end{enumerate}
\solution
%\input{ncert/11/16/4/9/main.tex}
\item Consider the probability space $\brak{\Omega, \mathcal{G}, P}$ where $\Omega = [0,2]$ and $\mathcal{G} = \cbrak{\phi, \Omega, [0,1], (1,2]}$. Let $X$ and $Y$ be two functions on $\Omega$ defined as
\begin{align*}
    X(\omega) = 
    \begin{cases}
        1 & \text{if }\omega \in [0, 1]\\
        2 & \text{if }\omega \in (1, 2]
    \end{cases}
\end{align*}
and
\begin{align*}
    Y(\omega) = 
    \begin{cases}
        2 & \text{if }\omega \in [0, 1.5]\\
        3 & \text{if }\omega \in (1.5, 2].
    \end{cases}
\end{align*}
Then which one of the following statements is true?
\begin{enumerate}
    \item [(A)] $X$ is a random variable with respect to $\mathcal{G}$, but $Y$ is not a random variable with respect to $\mathcal{G}$.
    \item [(B)] $Y$ is a random variable with respect to $\mathcal{G}$, but $X$ is not a random variable with respect to $\mathcal{G}$.
    \item [(C)] Neither $X$ nor $Y$ is a random variable with respect to $\mathcal{G}$.
    \item [(D)] Both $X$ and $Y$ are random variables with respect to $\mathcal{G}$.
\end{enumerate} \hfill (GATE ST 2023)\\
\solution
%\input{gate/ST/2023/14/main.tex}
	\item  A die is loaded in such a way that each odd number is twice as likely to occur as
each even number. Find $P(G)$, where $G$ is the event that a number greater than
3 occurs on a single roll of the die.
\\
\solution
		%\input{exemplar/11/16/3/5/main.tex}
	\item All the jacks, queens and kings are removed from a deck of 52 playing cards. The remaining cards are well shuffled and then one card is drawn at random. Giving ace a value 1 similar value for other cards, find the probability that the card has a value 
		\begin{enumerate}
			\item 7
			\item greater than 7
			\item less than 7
		\end{enumerate}
		%\input{exemplar/10/13/3/30/main.tex}
  \item A Lot consists of 48 mobile phones of which 42 are good, 3 have only minor defects and 3 have major defects.Varnika will buy a phone if it is good but the trader will only buy a mobile if it has no major defects. One phone is selected at random from the lot. What is the probability that it is
\begin{enumerate}
	\item acceptable to Varnika?
            \item acceptable to the trader?
\end{enumerate}
\solution
	%\input{exemplar/10/13/3/40/main.tex}
 \item A student says that if you throw a die, it will show up 1 or not 1. Therefore, the probability of getting 1 and the probability of getting 'not 1' each is equal to $\frac{1}{2}$. Is this correct? Give reasons.\\
 \solution
        %\input{exemplar/10/13/2/9/main.tex}
   \item Four candidates A, B, C, D have ap-
plied for the assignment to coach a school cricket
team. If A is twice as likely to be selected as B, and
B and C are given about the same chance of being
selected, while C is twice as likely to be selected
as D, what are the probabilities that
\begin{enumerate}
\item C will be selected?
\item A will not be selected?
\end{enumerate}
	%\input{exemplar/11/16/3/9/main.tex}
 \item A bag contain 24 balls of which $x$ balls are red, $2x$ are white and $3x$ are blue. A ball is selected at random, What is the probability that it is
\begin{enumerate}[label=\alph*)]
\item not red ?
\item white ?
\end{enumerate}
%\input{exemplar/10/13/3/41/main.tex}
If the letters of the word ASSASSINATION are arranged at random. Find the Probability that
\begin{enumerate}[label=(\alph*)]
\item Four $S's$ come consecutively in the word
\item Two  $I's$ and two $N's$ come together
\item All $A's$ are not coming together
\item No two $A's$ are coming together
\end{enumerate}
%\input{exemplar/11/16/3/14/main.tex}
	\item One urn contains two black balls (labelled B1 and B2) and one white ball. A
	second urn contains one black ball and two white balls (labelled W1 and W2).
	Suppose the following experiment is performed. One of the two urns is chosen
	at random. Next a ball is randomly chosen from the urn. Then a second ball is
	chosen at random from the same urn without replacing the first ball.
	
	\begin{enumerate}
	\item What is the probability that two black balls are chosen?
	
	\item What is the probability that two balls of opposite colour are chosen?
	\end{enumerate}
	\solution
	%\input{exemplar/11/16/3/12/main1.tex}
\end{enumerate}

	\item A bag contains 4 red and 4 black balls, another bag contains 2 red and 6 black balls. One of the two bags is selected at random and a ball is drawn from the bag which is found to be red. Find the probability that the ball is drawn from the first bag.
\\
\solution
		%\begin{table}[H]
	\centering
\begin{tabular}{|c|c|c|}
\hline
Random variable &Value &Definition\\ \hline
\multirow{3}{*}{X} &0 &Slips of Rs 1\\
&1 &Slips of Rs 5\\
&2 &Slips of Rs 13\\ \hline
\multirow{2}{*}{Y} &0 &Box A\\
&1 &Box B\\\hline
\end{tabular}
\caption{}
\label{tab:Distribution}
\end{table}
See \tabref{tab:Distribution}.
\begin{align}
p_{Y}\brak{k}= \begin{cases} 
      \frac{1}{3} & {k=0} \\
      \frac{2}{3 }& {k=1} 
   \end{cases}
   \\
p_{Y|X}\brak{0|0} = \frac{19}{25}\, 
p_{Y|X}\brak{0|1} = \frac{6}{25}\,
p_{Y|X}\brak{1|0} = \frac{45}{50}\,
p_{Y|X}\brak{1|2} = \frac{5}{50}
\end{align}
The desired probability is the probability that a slip drawn at random is marked other than Rs 1,
\begin{align}
&=1-p_X\brak{0}\\
&= p_X(1) + p_X(2)
\end{align}
Using Bayes theorem,
\begin{align}
&= p_Y\brak{0} \times \pr{Y=0 | X=1} + p_Y\brak{1} \times \pr{Y=1|X=2}\\
&=\frac{1}{3} \times \frac{6}{25} + \frac{2}{3} \times \frac{5}{50}\\
&=\frac{11}{75}
\end{align}

\newpage

%\tableofcontents

\bigskip

\renewcommand{\thefigure}{\theenumi}
\renewcommand{\thetable}{\theenumi}
%\renewcommand{\theequation}{\theenumi}

%\begin{abstract}
%%\boldmath
%In this letter, an algorithm for evaluating the exact analytical bit error rate  (BER)  for the piecewise linear (PL) combiner for  multiple relays is presented. Previous results were available only for upto three relays. The algorithm is unique in the sense that  the actual mathematical expressions, that are prohibitively large, need not be explicitly obtained. The diversity gain due to multiple relays is shown through plots of the analytical BER, well supported by simulations. 
%
%\end{abstract}
% IEEEtran.cls defaults to using nonbold math in the Abstract.
% This preserves the distinction between vectors and scalars. However,
% if the journal you are submitting to favors bold math in the abstract,
% then you can use LaTeX's standard command \boldmath at the very start
% of the abstract to achieve this. Many IEEE journals frown on math
% in the abstract anyway.

% Note that keywords are not normally used for peerreview papers.
%\begin{IEEEkeywords}
%Cooperative diversity, decode and forward, piecewise linear
%\end{IEEEkeywords}



% For peer review papers, you can put extra information on the cover
% page as needed:
% \ifCLASSOPTIONpeerreview
% \begin{center} \bfseries EDICS Category: 3-BBND \end{center}
% \fi
%
% For peerreview papers, this IEEEtran command inserts a page break and
% creates the second title. It will be ignored for other modes.
%\IEEEpeerreviewmaketitle




  \item
  Cards with numbers 2 to 101 are placed in a box. A card is selected at random.Find the probability that the card has
\begin{enumerate}[label=(\roman*)]
	\item an even number 
	\item a square number
\end{enumerate}
\solution
%\begin{table}[H]
	\centering
\begin{tabular}{|c|c|c|}
\hline
Random variable &Value &Definition\\ \hline
\multirow{3}{*}{X} &0 &Slips of Rs 1\\
&1 &Slips of Rs 5\\
&2 &Slips of Rs 13\\ \hline
\multirow{2}{*}{Y} &0 &Box A\\
&1 &Box B\\\hline
\end{tabular}
\caption{}
\label{tab:Distribution}
\end{table}
See \tabref{tab:Distribution}.
\begin{align}
p_{Y}\brak{k}= \begin{cases} 
      \frac{1}{3} & {k=0} \\
      \frac{2}{3 }& {k=1} 
   \end{cases}
   \\
p_{Y|X}\brak{0|0} = \frac{19}{25}\, 
p_{Y|X}\brak{0|1} = \frac{6}{25}\,
p_{Y|X}\brak{1|0} = \frac{45}{50}\,
p_{Y|X}\brak{1|2} = \frac{5}{50}
\end{align}
The desired probability is the probability that a slip drawn at random is marked other than Rs 1,
\begin{align}
&=1-p_X\brak{0}\\
&= p_X(1) + p_X(2)
\end{align}
Using Bayes theorem,
\begin{align}
&= p_Y\brak{0} \times \pr{Y=0 | X=1} + p_Y\brak{1} \times \pr{Y=1|X=2}\\
&=\frac{1}{3} \times \frac{6}{25} + \frac{2}{3} \times \frac{5}{50}\\
&=\frac{11}{75}
\end{align}

\newpage

%\tableofcontents

\bigskip

\renewcommand{\thefigure}{\theenumi}
\renewcommand{\thetable}{\theenumi}
%\renewcommand{\theequation}{\theenumi}

%\begin{abstract}
%%\boldmath
%In this letter, an algorithm for evaluating the exact analytical bit error rate  (BER)  for the piecewise linear (PL) combiner for  multiple relays is presented. Previous results were available only for upto three relays. The algorithm is unique in the sense that  the actual mathematical expressions, that are prohibitively large, need not be explicitly obtained. The diversity gain due to multiple relays is shown through plots of the analytical BER, well supported by simulations. 
%
%\end{abstract}
% IEEEtran.cls defaults to using nonbold math in the Abstract.
% This preserves the distinction between vectors and scalars. However,
% if the journal you are submitting to favors bold math in the abstract,
% then you can use LaTeX's standard command \boldmath at the very start
% of the abstract to achieve this. Many IEEE journals frown on math
% in the abstract anyway.

% Note that keywords are not normally used for peerreview papers.
%\begin{IEEEkeywords}
%Cooperative diversity, decode and forward, piecewise linear
%\end{IEEEkeywords}



% For peer review papers, you can put extra information on the cover
% page as needed:
% \ifCLASSOPTIONpeerreview
% \begin{center} \bfseries EDICS Category: 3-BBND \end{center}
% \fi
%
% For peerreview papers, this IEEEtran command inserts a page break and
% creates the second title. It will be ignored for other modes.
%\IEEEpeerreviewmaketitle




\item
The king, queen and jack of clubs are removed from a deck of 52 playing cards and then well shuffled. Now one card is drawn at random from the remaining cards.  Determine the probability that the card is
\begin{enumerate}[label=(\roman*)]
\item a club
\item 10 of hearts
\end{enumerate}
\solution
%\begin{table}[H]
	\centering
\begin{tabular}{|c|c|c|}
\hline
Random variable &Value &Definition\\ \hline
\multirow{3}{*}{X} &0 &Slips of Rs 1\\
&1 &Slips of Rs 5\\
&2 &Slips of Rs 13\\ \hline
\multirow{2}{*}{Y} &0 &Box A\\
&1 &Box B\\\hline
\end{tabular}
\caption{}
\label{tab:Distribution}
\end{table}
See \tabref{tab:Distribution}.
\begin{align}
p_{Y}\brak{k}= \begin{cases} 
      \frac{1}{3} & {k=0} \\
      \frac{2}{3 }& {k=1} 
   \end{cases}
   \\
p_{Y|X}\brak{0|0} = \frac{19}{25}\, 
p_{Y|X}\brak{0|1} = \frac{6}{25}\,
p_{Y|X}\brak{1|0} = \frac{45}{50}\,
p_{Y|X}\brak{1|2} = \frac{5}{50}
\end{align}
The desired probability is the probability that a slip drawn at random is marked other than Rs 1,
\begin{align}
&=1-p_X\brak{0}\\
&= p_X(1) + p_X(2)
\end{align}
Using Bayes theorem,
\begin{align}
&= p_Y\brak{0} \times \pr{Y=0 | X=1} + p_Y\brak{1} \times \pr{Y=1|X=2}\\
&=\frac{1}{3} \times \frac{6}{25} + \frac{2}{3} \times \frac{5}{50}\\
&=\frac{11}{75}
\end{align}

\newpage

%\tableofcontents

\bigskip

\renewcommand{\thefigure}{\theenumi}
\renewcommand{\thetable}{\theenumi}
%\renewcommand{\theequation}{\theenumi}

%\begin{abstract}
%%\boldmath
%In this letter, an algorithm for evaluating the exact analytical bit error rate  (BER)  for the piecewise linear (PL) combiner for  multiple relays is presented. Previous results were available only for upto three relays. The algorithm is unique in the sense that  the actual mathematical expressions, that are prohibitively large, need not be explicitly obtained. The diversity gain due to multiple relays is shown through plots of the analytical BER, well supported by simulations. 
%
%\end{abstract}
% IEEEtran.cls defaults to using nonbold math in the Abstract.
% This preserves the distinction between vectors and scalars. However,
% if the journal you are submitting to favors bold math in the abstract,
% then you can use LaTeX's standard command \boldmath at the very start
% of the abstract to achieve this. Many IEEE journals frown on math
% in the abstract anyway.

% Note that keywords are not normally used for peerreview papers.
%\begin{IEEEkeywords}
%Cooperative diversity, decode and forward, piecewise linear
%\end{IEEEkeywords}



% For peer review papers, you can put extra information on the cover
% page as needed:
% \ifCLASSOPTIONpeerreview
% \begin{center} \bfseries EDICS Category: 3-BBND \end{center}
% \fi
%
% For peerreview papers, this IEEEtran command inserts a page break and
% creates the second title. It will be ignored for other modes.
%\IEEEpeerreviewmaketitle




\item A team of medical students doing their internship have to assist during surgeries
at a city hospital. The probabilities of surgeries rated as very complex, complex,
routine, simple or very simple are respectively, 0.15, 0.20, 0.31, 0.26, .08. Find
the probabilities that a particular surgery will be rated
\begin{enumerate}
	\item complex or very complex;
	\item neither very complex nor very simple;
	\item routine or complex
	\item routine or simple
\end{enumerate}
\solution
%\begin{table}[H]
	\centering
\begin{tabular}{|c|c|c|}
\hline
Random variable &Value &Definition\\ \hline
\multirow{3}{*}{X} &0 &Slips of Rs 1\\
&1 &Slips of Rs 5\\
&2 &Slips of Rs 13\\ \hline
\multirow{2}{*}{Y} &0 &Box A\\
&1 &Box B\\\hline
\end{tabular}
\caption{}
\label{tab:Distribution}
\end{table}
See \tabref{tab:Distribution}.
\begin{align}
p_{Y}\brak{k}= \begin{cases} 
      \frac{1}{3} & {k=0} \\
      \frac{2}{3 }& {k=1} 
   \end{cases}
   \\
p_{Y|X}\brak{0|0} = \frac{19}{25}\, 
p_{Y|X}\brak{0|1} = \frac{6}{25}\,
p_{Y|X}\brak{1|0} = \frac{45}{50}\,
p_{Y|X}\brak{1|2} = \frac{5}{50}
\end{align}
The desired probability is the probability that a slip drawn at random is marked other than Rs 1,
\begin{align}
&=1-p_X\brak{0}\\
&= p_X(1) + p_X(2)
\end{align}
Using Bayes theorem,
\begin{align}
&= p_Y\brak{0} \times \pr{Y=0 | X=1} + p_Y\brak{1} \times \pr{Y=1|X=2}\\
&=\frac{1}{3} \times \frac{6}{25} + \frac{2}{3} \times \frac{5}{50}\\
&=\frac{11}{75}
\end{align}

\newpage

%\tableofcontents

\bigskip

\renewcommand{\thefigure}{\theenumi}
\renewcommand{\thetable}{\theenumi}
%\renewcommand{\theequation}{\theenumi}

%\begin{abstract}
%%\boldmath
%In this letter, an algorithm for evaluating the exact analytical bit error rate  (BER)  for the piecewise linear (PL) combiner for  multiple relays is presented. Previous results were available only for upto three relays. The algorithm is unique in the sense that  the actual mathematical expressions, that are prohibitively large, need not be explicitly obtained. The diversity gain due to multiple relays is shown through plots of the analytical BER, well supported by simulations. 
%
%\end{abstract}
% IEEEtran.cls defaults to using nonbold math in the Abstract.
% This preserves the distinction between vectors and scalars. However,
% if the journal you are submitting to favors bold math in the abstract,
% then you can use LaTeX's standard command \boldmath at the very start
% of the abstract to achieve this. Many IEEE journals frown on math
% in the abstract anyway.

% Note that keywords are not normally used for peerreview papers.
%\begin{IEEEkeywords}
%Cooperative diversity, decode and forward, piecewise linear
%\end{IEEEkeywords}



% For peer review papers, you can put extra information on the cover
% page as needed:
% \ifCLASSOPTIONpeerreview
% \begin{center} \bfseries EDICS Category: 3-BBND \end{center}
% \fi
%
% For peerreview papers, this IEEEtran command inserts a page break and
% creates the second title. It will be ignored for other modes.
%\IEEEpeerreviewmaketitle




\item A card is selected from a pack of 52 cards.
\begin{enumerate}[label=(\alph*)]
    \item How many points are there in the sample space?
    \item Calculate the probability that the card is an ace of spades.
    \item Calculate the probability that the card is (i) an ace and (ii) black card.
\end{enumerate}
\solution
%Let $X$ be an bernoulli rv defined as in \tabref{tab:exemplar/11/16/3/26}.  Then, 
\begin{equation}
    p =
        \frac{4}{11} 
\end{equation}
\begin{table}[H]
	\centering
	\input{exemplar/11/16/3/26/tables/Table2.tex}
	\caption{}
        \label{tab:exemplar/11/16/3/26}
\end{table}

\item The probability that a non leap year selected at random will contain 53 sundays.
\\
\solution
%\begin{table}[H]
	\centering
\begin{tabular}{|c|c|c|}
\hline
Random variable &Value &Definition\\ \hline
\multirow{3}{*}{X} &0 &Slips of Rs 1\\
&1 &Slips of Rs 5\\
&2 &Slips of Rs 13\\ \hline
\multirow{2}{*}{Y} &0 &Box A\\
&1 &Box B\\\hline
\end{tabular}
\caption{}
\label{tab:Distribution}
\end{table}
See \tabref{tab:Distribution}.
\begin{align}
p_{Y}\brak{k}= \begin{cases} 
      \frac{1}{3} & {k=0} \\
      \frac{2}{3 }& {k=1} 
   \end{cases}
   \\
p_{Y|X}\brak{0|0} = \frac{19}{25}\, 
p_{Y|X}\brak{0|1} = \frac{6}{25}\,
p_{Y|X}\brak{1|0} = \frac{45}{50}\,
p_{Y|X}\brak{1|2} = \frac{5}{50}
\end{align}
The desired probability is the probability that a slip drawn at random is marked other than Rs 1,
\begin{align}
&=1-p_X\brak{0}\\
&= p_X(1) + p_X(2)
\end{align}
Using Bayes theorem,
\begin{align}
&= p_Y\brak{0} \times \pr{Y=0 | X=1} + p_Y\brak{1} \times \pr{Y=1|X=2}\\
&=\frac{1}{3} \times \frac{6}{25} + \frac{2}{3} \times \frac{5}{50}\\
&=\frac{11}{75}
\end{align}

\newpage

%\tableofcontents

\bigskip

\renewcommand{\thefigure}{\theenumi}
\renewcommand{\thetable}{\theenumi}
%\renewcommand{\theequation}{\theenumi}

%\begin{abstract}
%%\boldmath
%In this letter, an algorithm for evaluating the exact analytical bit error rate  (BER)  for the piecewise linear (PL) combiner for  multiple relays is presented. Previous results were available only for upto three relays. The algorithm is unique in the sense that  the actual mathematical expressions, that are prohibitively large, need not be explicitly obtained. The diversity gain due to multiple relays is shown through plots of the analytical BER, well supported by simulations. 
%
%\end{abstract}
% IEEEtran.cls defaults to using nonbold math in the Abstract.
% This preserves the distinction between vectors and scalars. However,
% if the journal you are submitting to favors bold math in the abstract,
% then you can use LaTeX's standard command \boldmath at the very start
% of the abstract to achieve this. Many IEEE journals frown on math
% in the abstract anyway.

% Note that keywords are not normally used for peerreview papers.
%\begin{IEEEkeywords}
%Cooperative diversity, decode and forward, piecewise linear
%\end{IEEEkeywords}



% For peer review papers, you can put extra information on the cover
% page as needed:
% \ifCLASSOPTIONpeerreview
% \begin{center} \bfseries EDICS Category: 3-BBND \end{center}
% \fi
%
% For peerreview papers, this IEEEtran command inserts a page break and
% creates the second title. It will be ignored for other modes.
%\IEEEpeerreviewmaketitle




\item One of the four persons John, Rita, Aslam or Gurpreet will be promoted next
month. Consequently the sample space consists of four elementary outcomes
S = {John promoted, Rita promoted, Aslam promoted, Gurpreet promoted}
You are told that the chances of John’s promotion is same as that of Gurpreet,
Rita’s chances of promotion are twice as likely as Johns. Aslam’s chances are
four times that of John.
\begin{enumerate}
	\item Determine
	\begin{enumerate}
		\item P (John promoted)
		\item P (Rita promoted)
		\item P (Aslam promoted)
		\item P (Gurpreet promoted)
	\end{enumerate}
	\item If A = {John promoted or Gurpreet promoted}, find P (A).
\end{enumerate}
\solution
%\begin{table}[H]
	\centering
\begin{tabular}{|c|c|c|}
\hline
Random variable &Value &Definition\\ \hline
\multirow{3}{*}{X} &0 &Slips of Rs 1\\
&1 &Slips of Rs 5\\
&2 &Slips of Rs 13\\ \hline
\multirow{2}{*}{Y} &0 &Box A\\
&1 &Box B\\\hline
\end{tabular}
\caption{}
\label{tab:Distribution}
\end{table}
See \tabref{tab:Distribution}.
\begin{align}
p_{Y}\brak{k}= \begin{cases} 
      \frac{1}{3} & {k=0} \\
      \frac{2}{3 }& {k=1} 
   \end{cases}
   \\
p_{Y|X}\brak{0|0} = \frac{19}{25}\, 
p_{Y|X}\brak{0|1} = \frac{6}{25}\,
p_{Y|X}\brak{1|0} = \frac{45}{50}\,
p_{Y|X}\brak{1|2} = \frac{5}{50}
\end{align}
The desired probability is the probability that a slip drawn at random is marked other than Rs 1,
\begin{align}
&=1-p_X\brak{0}\\
&= p_X(1) + p_X(2)
\end{align}
Using Bayes theorem,
\begin{align}
&= p_Y\brak{0} \times \pr{Y=0 | X=1} + p_Y\brak{1} \times \pr{Y=1|X=2}\\
&=\frac{1}{3} \times \frac{6}{25} + \frac{2}{3} \times \frac{5}{50}\\
&=\frac{11}{75}
\end{align}

\newpage

%\tableofcontents

\bigskip

\renewcommand{\thefigure}{\theenumi}
\renewcommand{\thetable}{\theenumi}
%\renewcommand{\theequation}{\theenumi}

%\begin{abstract}
%%\boldmath
%In this letter, an algorithm for evaluating the exact analytical bit error rate  (BER)  for the piecewise linear (PL) combiner for  multiple relays is presented. Previous results were available only for upto three relays. The algorithm is unique in the sense that  the actual mathematical expressions, that are prohibitively large, need not be explicitly obtained. The diversity gain due to multiple relays is shown through plots of the analytical BER, well supported by simulations. 
%
%\end{abstract}
% IEEEtran.cls defaults to using nonbold math in the Abstract.
% This preserves the distinction between vectors and scalars. However,
% if the journal you are submitting to favors bold math in the abstract,
% then you can use LaTeX's standard command \boldmath at the very start
% of the abstract to achieve this. Many IEEE journals frown on math
% in the abstract anyway.

% Note that keywords are not normally used for peerreview papers.
%\begin{IEEEkeywords}
%Cooperative diversity, decode and forward, piecewise linear
%\end{IEEEkeywords}



% For peer review papers, you can put extra information on the cover
% page as needed:
% \ifCLASSOPTIONpeerreview
% \begin{center} \bfseries EDICS Category: 3-BBND \end{center}
% \fi
%
% For peerreview papers, this IEEEtran command inserts a page break and
% creates the second title. It will be ignored for other modes.
%\IEEEpeerreviewmaketitle




\item A card is drawn from a deck of 52 cards. Find the probability of getting a king or a heart or a red card.\\
\solution
%\begin{table}[H]
	\centering
\begin{tabular}{|c|c|c|}
\hline
Random variable &Value &Definition\\ \hline
\multirow{3}{*}{X} &0 &Slips of Rs 1\\
&1 &Slips of Rs 5\\
&2 &Slips of Rs 13\\ \hline
\multirow{2}{*}{Y} &0 &Box A\\
&1 &Box B\\\hline
\end{tabular}
\caption{}
\label{tab:Distribution}
\end{table}
See \tabref{tab:Distribution}.
\begin{align}
p_{Y}\brak{k}= \begin{cases} 
      \frac{1}{3} & {k=0} \\
      \frac{2}{3 }& {k=1} 
   \end{cases}
   \\
p_{Y|X}\brak{0|0} = \frac{19}{25}\, 
p_{Y|X}\brak{0|1} = \frac{6}{25}\,
p_{Y|X}\brak{1|0} = \frac{45}{50}\,
p_{Y|X}\brak{1|2} = \frac{5}{50}
\end{align}
The desired probability is the probability that a slip drawn at random is marked other than Rs 1,
\begin{align}
&=1-p_X\brak{0}\\
&= p_X(1) + p_X(2)
\end{align}
Using Bayes theorem,
\begin{align}
&= p_Y\brak{0} \times \pr{Y=0 | X=1} + p_Y\brak{1} \times \pr{Y=1|X=2}\\
&=\frac{1}{3} \times \frac{6}{25} + \frac{2}{3} \times \frac{5}{50}\\
&=\frac{11}{75}
\end{align}

\newpage

%\tableofcontents

\bigskip

\renewcommand{\thefigure}{\theenumi}
\renewcommand{\thetable}{\theenumi}
%\renewcommand{\theequation}{\theenumi}

%\begin{abstract}
%%\boldmath
%In this letter, an algorithm for evaluating the exact analytical bit error rate  (BER)  for the piecewise linear (PL) combiner for  multiple relays is presented. Previous results were available only for upto three relays. The algorithm is unique in the sense that  the actual mathematical expressions, that are prohibitively large, need not be explicitly obtained. The diversity gain due to multiple relays is shown through plots of the analytical BER, well supported by simulations. 
%
%\end{abstract}
% IEEEtran.cls defaults to using nonbold math in the Abstract.
% This preserves the distinction between vectors and scalars. However,
% if the journal you are submitting to favors bold math in the abstract,
% then you can use LaTeX's standard command \boldmath at the very start
% of the abstract to achieve this. Many IEEE journals frown on math
% in the abstract anyway.

% Note that keywords are not normally used for peerreview papers.
%\begin{IEEEkeywords}
%Cooperative diversity, decode and forward, piecewise linear
%\end{IEEEkeywords}



% For peer review papers, you can put extra information on the cover
% page as needed:
% \ifCLASSOPTIONpeerreview
% \begin{center} \bfseries EDICS Category: 3-BBND \end{center}
% \fi
%
% For peerreview papers, this IEEEtran command inserts a page break and
% creates the second title. It will be ignored for other modes.
%\IEEEpeerreviewmaketitle




\item The probability that a student will pass his examination is 0.73, the probability of
the student getting a compartment is 0.13, and the probability that the student will
either pass or get compartment is 0.96. State True or False.\\
\solution
%\begin{table}[H]
	\centering
\begin{tabular}{|c|c|c|}
\hline
Random variable &Value &Definition\\ \hline
\multirow{3}{*}{X} &0 &Slips of Rs 1\\
&1 &Slips of Rs 5\\
&2 &Slips of Rs 13\\ \hline
\multirow{2}{*}{Y} &0 &Box A\\
&1 &Box B\\\hline
\end{tabular}
\caption{}
\label{tab:Distribution}
\end{table}
See \tabref{tab:Distribution}.
\begin{align}
p_{Y}\brak{k}= \begin{cases} 
      \frac{1}{3} & {k=0} \\
      \frac{2}{3 }& {k=1} 
   \end{cases}
   \\
p_{Y|X}\brak{0|0} = \frac{19}{25}\, 
p_{Y|X}\brak{0|1} = \frac{6}{25}\,
p_{Y|X}\brak{1|0} = \frac{45}{50}\,
p_{Y|X}\brak{1|2} = \frac{5}{50}
\end{align}
The desired probability is the probability that a slip drawn at random is marked other than Rs 1,
\begin{align}
&=1-p_X\brak{0}\\
&= p_X(1) + p_X(2)
\end{align}
Using Bayes theorem,
\begin{align}
&= p_Y\brak{0} \times \pr{Y=0 | X=1} + p_Y\brak{1} \times \pr{Y=1|X=2}\\
&=\frac{1}{3} \times \frac{6}{25} + \frac{2}{3} \times \frac{5}{50}\\
&=\frac{11}{75}
\end{align}

\newpage

%\tableofcontents

\bigskip

\renewcommand{\thefigure}{\theenumi}
\renewcommand{\thetable}{\theenumi}
%\renewcommand{\theequation}{\theenumi}

%\begin{abstract}
%%\boldmath
%In this letter, an algorithm for evaluating the exact analytical bit error rate  (BER)  for the piecewise linear (PL) combiner for  multiple relays is presented. Previous results were available only for upto three relays. The algorithm is unique in the sense that  the actual mathematical expressions, that are prohibitively large, need not be explicitly obtained. The diversity gain due to multiple relays is shown through plots of the analytical BER, well supported by simulations. 
%
%\end{abstract}
% IEEEtran.cls defaults to using nonbold math in the Abstract.
% This preserves the distinction between vectors and scalars. However,
% if the journal you are submitting to favors bold math in the abstract,
% then you can use LaTeX's standard command \boldmath at the very start
% of the abstract to achieve this. Many IEEE journals frown on math
% in the abstract anyway.

% Note that keywords are not normally used for peerreview papers.
%\begin{IEEEkeywords}
%Cooperative diversity, decode and forward, piecewise linear
%\end{IEEEkeywords}



% For peer review papers, you can put extra information on the cover
% page as needed:
% \ifCLASSOPTIONpeerreview
% \begin{center} \bfseries EDICS Category: 3-BBND \end{center}
% \fi
%
% For peerreview papers, this IEEEtran command inserts a page break and
% creates the second title. It will be ignored for other modes.
%\IEEEpeerreviewmaketitle




\item A card is selected from a pack of 52 cards\\
\begin{enumerate}[label=(\alph*)]
\item How many points are there in the sample space?
\item Calculate the probability that the cards is an ace of spades.
\item Calculate the probability that the card is (i) an ace (ii)black card.\\
\end{enumerate}
%\input{ncert/11/16/3/4_1/Prob_4.tex}
\item In a non-leap year, the probability of having 53 tuesdays or 53 wednesdays is\\
\solution
%A non-leap year has a total of 365 days, and a week has 7 days.\\
So it can be expressed as 
\begin{align}
365\text{days} &=52\times 7+1 \text{day}
\end{align}
$\implies$ 52 tuesdays or wednesdays\\
Random variable X denotes the days of a week
\begin{align}
p_X\brak{k}&=\frac{1}{7}; \quad \brak{1<k<7}
\end{align}
So the probability of extra day being tuesday or wednesday is
\begin{align}
p_X\brak{3}+p_X\brak{4}&=\frac{1}{7}+\frac{1}{7}=\frac{2}{7}
\end{align}



\item There are 1000 sealed envelopes in a box, 10 of them contain a cash prize of
Rs 100 each, 100 of them contain a cash prize of Rs 50 each and 200 of them
contain a cash prize of Rs 10 each and rest do not contain any cash prize. If they
are well shuffled and an envelope is picked up out, what is the probability that it
contains no cash prize?\\
\solution
%\begin{table}[H]
	\centering
\begin{tabular}{|c|c|c|}
\hline
Random variable &Value &Definition\\ \hline
\multirow{3}{*}{X} &0 &Slips of Rs 1\\
&1 &Slips of Rs 5\\
&2 &Slips of Rs 13\\ \hline
\multirow{2}{*}{Y} &0 &Box A\\
&1 &Box B\\\hline
\end{tabular}
\caption{}
\label{tab:Distribution}
\end{table}
See \tabref{tab:Distribution}.
\begin{align}
p_{Y}\brak{k}= \begin{cases} 
      \frac{1}{3} & {k=0} \\
      \frac{2}{3 }& {k=1} 
   \end{cases}
   \\
p_{Y|X}\brak{0|0} = \frac{19}{25}\, 
p_{Y|X}\brak{0|1} = \frac{6}{25}\,
p_{Y|X}\brak{1|0} = \frac{45}{50}\,
p_{Y|X}\brak{1|2} = \frac{5}{50}
\end{align}
The desired probability is the probability that a slip drawn at random is marked other than Rs 1,
\begin{align}
&=1-p_X\brak{0}\\
&= p_X(1) + p_X(2)
\end{align}
Using Bayes theorem,
\begin{align}
&= p_Y\brak{0} \times \pr{Y=0 | X=1} + p_Y\brak{1} \times \pr{Y=1|X=2}\\
&=\frac{1}{3} \times \frac{6}{25} + \frac{2}{3} \times \frac{5}{50}\\
&=\frac{11}{75}
\end{align}

\newpage

%\tableofcontents

\bigskip

\renewcommand{\thefigure}{\theenumi}
\renewcommand{\thetable}{\theenumi}
%\renewcommand{\theequation}{\theenumi}

%\begin{abstract}
%%\boldmath
%In this letter, an algorithm for evaluating the exact analytical bit error rate  (BER)  for the piecewise linear (PL) combiner for  multiple relays is presented. Previous results were available only for upto three relays. The algorithm is unique in the sense that  the actual mathematical expressions, that are prohibitively large, need not be explicitly obtained. The diversity gain due to multiple relays is shown through plots of the analytical BER, well supported by simulations. 
%
%\end{abstract}
% IEEEtran.cls defaults to using nonbold math in the Abstract.
% This preserves the distinction between vectors and scalars. However,
% if the journal you are submitting to favors bold math in the abstract,
% then you can use LaTeX's standard command \boldmath at the very start
% of the abstract to achieve this. Many IEEE journals frown on math
% in the abstract anyway.

% Note that keywords are not normally used for peerreview papers.
%\begin{IEEEkeywords}
%Cooperative diversity, decode and forward, piecewise linear
%\end{IEEEkeywords}



% For peer review papers, you can put extra information on the cover
% page as needed:
% \ifCLASSOPTIONpeerreview
% \begin{center} \bfseries EDICS Category: 3-BBND \end{center}
% \fi
%
% For peerreview papers, this IEEEtran command inserts a page break and
% creates the second title. It will be ignored for other modes.
%\IEEEpeerreviewmaketitle




\item 
A die is thrown and a card is selected at random from a deck of 52 playing cards. The probability of getting an even number on the die and a spade card.\\
\solution
%\begin{table}[H]
	\centering
\begin{tabular}{|c|c|c|}
\hline
Random variable &Value &Definition\\ \hline
\multirow{3}{*}{X} &0 &Slips of Rs 1\\
&1 &Slips of Rs 5\\
&2 &Slips of Rs 13\\ \hline
\multirow{2}{*}{Y} &0 &Box A\\
&1 &Box B\\\hline
\end{tabular}
\caption{}
\label{tab:Distribution}
\end{table}
See \tabref{tab:Distribution}.
\begin{align}
p_{Y}\brak{k}= \begin{cases} 
      \frac{1}{3} & {k=0} \\
      \frac{2}{3 }& {k=1} 
   \end{cases}
   \\
p_{Y|X}\brak{0|0} = \frac{19}{25}\, 
p_{Y|X}\brak{0|1} = \frac{6}{25}\,
p_{Y|X}\brak{1|0} = \frac{45}{50}\,
p_{Y|X}\brak{1|2} = \frac{5}{50}
\end{align}
The desired probability is the probability that a slip drawn at random is marked other than Rs 1,
\begin{align}
&=1-p_X\brak{0}\\
&= p_X(1) + p_X(2)
\end{align}
Using Bayes theorem,
\begin{align}
&= p_Y\brak{0} \times \pr{Y=0 | X=1} + p_Y\brak{1} \times \pr{Y=1|X=2}\\
&=\frac{1}{3} \times \frac{6}{25} + \frac{2}{3} \times \frac{5}{50}\\
&=\frac{11}{75}
\end{align}

\newpage

%\tableofcontents

\bigskip

\renewcommand{\thefigure}{\theenumi}
\renewcommand{\thetable}{\theenumi}
%\renewcommand{\theequation}{\theenumi}

%\begin{abstract}
%%\boldmath
%In this letter, an algorithm for evaluating the exact analytical bit error rate  (BER)  for the piecewise linear (PL) combiner for  multiple relays is presented. Previous results were available only for upto three relays. The algorithm is unique in the sense that  the actual mathematical expressions, that are prohibitively large, need not be explicitly obtained. The diversity gain due to multiple relays is shown through plots of the analytical BER, well supported by simulations. 
%
%\end{abstract}
% IEEEtran.cls defaults to using nonbold math in the Abstract.
% This preserves the distinction between vectors and scalars. However,
% if the journal you are submitting to favors bold math in the abstract,
% then you can use LaTeX's standard command \boldmath at the very start
% of the abstract to achieve this. Many IEEE journals frown on math
% in the abstract anyway.

% Note that keywords are not normally used for peerreview papers.
%\begin{IEEEkeywords}
%Cooperative diversity, decode and forward, piecewise linear
%\end{IEEEkeywords}



% For peer review papers, you can put extra information on the cover
% page as needed:
% \ifCLASSOPTIONpeerreview
% \begin{center} \bfseries EDICS Category: 3-BBND \end{center}
% \fi
%
% For peerreview papers, this IEEEtran command inserts a page break and
% creates the second title. It will be ignored for other modes.
%\IEEEpeerreviewmaketitle




\item
If 4-digit numbers greater than 5,000 are randomly formed from the digits 0, 1, 3, 5, and 7, what is the probability of forming a number divisible by 5 when:
\begin{enumerate}
    \item The digits are repeated?
    \item The repetition of digits is not allowed?
\end{enumerate}
\solution
%\begin{table}[H]
	\centering
\begin{tabular}{|c|c|c|}
\hline
Random variable &Value &Definition\\ \hline
\multirow{3}{*}{X} &0 &Slips of Rs 1\\
&1 &Slips of Rs 5\\
&2 &Slips of Rs 13\\ \hline
\multirow{2}{*}{Y} &0 &Box A\\
&1 &Box B\\\hline
\end{tabular}
\caption{}
\label{tab:Distribution}
\end{table}
See \tabref{tab:Distribution}.
\begin{align}
p_{Y}\brak{k}= \begin{cases} 
      \frac{1}{3} & {k=0} \\
      \frac{2}{3 }& {k=1} 
   \end{cases}
   \\
p_{Y|X}\brak{0|0} = \frac{19}{25}\, 
p_{Y|X}\brak{0|1} = \frac{6}{25}\,
p_{Y|X}\brak{1|0} = \frac{45}{50}\,
p_{Y|X}\brak{1|2} = \frac{5}{50}
\end{align}
The desired probability is the probability that a slip drawn at random is marked other than Rs 1,
\begin{align}
&=1-p_X\brak{0}\\
&= p_X(1) + p_X(2)
\end{align}
Using Bayes theorem,
\begin{align}
&= p_Y\brak{0} \times \pr{Y=0 | X=1} + p_Y\brak{1} \times \pr{Y=1|X=2}\\
&=\frac{1}{3} \times \frac{6}{25} + \frac{2}{3} \times \frac{5}{50}\\
&=\frac{11}{75}
\end{align}

\newpage

%\tableofcontents

\bigskip

\renewcommand{\thefigure}{\theenumi}
\renewcommand{\thetable}{\theenumi}
%\renewcommand{\theequation}{\theenumi}

%\begin{abstract}
%%\boldmath
%In this letter, an algorithm for evaluating the exact analytical bit error rate  (BER)  for the piecewise linear (PL) combiner for  multiple relays is presented. Previous results were available only for upto three relays. The algorithm is unique in the sense that  the actual mathematical expressions, that are prohibitively large, need not be explicitly obtained. The diversity gain due to multiple relays is shown through plots of the analytical BER, well supported by simulations. 
%
%\end{abstract}
% IEEEtran.cls defaults to using nonbold math in the Abstract.
% This preserves the distinction between vectors and scalars. However,
% if the journal you are submitting to favors bold math in the abstract,
% then you can use LaTeX's standard command \boldmath at the very start
% of the abstract to achieve this. Many IEEE journals frown on math
% in the abstract anyway.

% Note that keywords are not normally used for peerreview papers.
%\begin{IEEEkeywords}
%Cooperative diversity, decode and forward, piecewise linear
%\end{IEEEkeywords}



% For peer review papers, you can put extra information on the cover
% page as needed:
% \ifCLASSOPTIONpeerreview
% \begin{center} \bfseries EDICS Category: 3-BBND \end{center}
% \fi
%
% For peerreview papers, this IEEEtran command inserts a page break and
% creates the second title. It will be ignored for other modes.
%\IEEEpeerreviewmaketitle




\item Consider the probability space $\brak{\Omega, \mathcal{G}, P}$ where $\Omega = [0,2]$ and $\mathcal{G} = \cbrak{\phi, \Omega, [0,1], (1,2]}$. Let $X$ and $Y$ be two functions on $\Omega$ defined as
\begin{align*}
    X(\omega) = 
    \begin{cases}
        1 & \text{if }\omega \in [0, 1]\\
        2 & \text{if }\omega \in (1, 2]
    \end{cases}
\end{align*}
and
\begin{align*}
    Y(\omega) = 
    \begin{cases}
        2 & \text{if }\omega \in [0, 1.5]\\
        3 & \text{if }\omega \in (1.5, 2].
    \end{cases}
\end{align*}
Then which one of the following statements is true?
\begin{enumerate}
    \item [(A)] $X$ is a random variable with respect to $\mathcal{G}$, but $Y$ is not a random variable with respect to $\mathcal{G}$.
    \item [(B)] $Y$ is a random variable with respect to $\mathcal{G}$, but $X$ is not a random variable with respect to $\mathcal{G}$.
    \item [(C)] Neither $X$ nor $Y$ is a random variable with respect to $\mathcal{G}$.
    \item [(D)] Both $X$ and $Y$ are random variables with respect to $\mathcal{G}$.
\end{enumerate} \hfill (GATE ST 2023)\\
\solution
%\begin{table}[H]
	\centering
\begin{tabular}{|c|c|c|}
\hline
Random variable &Value &Definition\\ \hline
\multirow{3}{*}{X} &0 &Slips of Rs 1\\
&1 &Slips of Rs 5\\
&2 &Slips of Rs 13\\ \hline
\multirow{2}{*}{Y} &0 &Box A\\
&1 &Box B\\\hline
\end{tabular}
\caption{}
\label{tab:Distribution}
\end{table}
See \tabref{tab:Distribution}.
\begin{align}
p_{Y}\brak{k}= \begin{cases} 
      \frac{1}{3} & {k=0} \\
      \frac{2}{3 }& {k=1} 
   \end{cases}
   \\
p_{Y|X}\brak{0|0} = \frac{19}{25}\, 
p_{Y|X}\brak{0|1} = \frac{6}{25}\,
p_{Y|X}\brak{1|0} = \frac{45}{50}\,
p_{Y|X}\brak{1|2} = \frac{5}{50}
\end{align}
The desired probability is the probability that a slip drawn at random is marked other than Rs 1,
\begin{align}
&=1-p_X\brak{0}\\
&= p_X(1) + p_X(2)
\end{align}
Using Bayes theorem,
\begin{align}
&= p_Y\brak{0} \times \pr{Y=0 | X=1} + p_Y\brak{1} \times \pr{Y=1|X=2}\\
&=\frac{1}{3} \times \frac{6}{25} + \frac{2}{3} \times \frac{5}{50}\\
&=\frac{11}{75}
\end{align}

\newpage

%\tableofcontents

\bigskip

\renewcommand{\thefigure}{\theenumi}
\renewcommand{\thetable}{\theenumi}
%\renewcommand{\theequation}{\theenumi}

%\begin{abstract}
%%\boldmath
%In this letter, an algorithm for evaluating the exact analytical bit error rate  (BER)  for the piecewise linear (PL) combiner for  multiple relays is presented. Previous results were available only for upto three relays. The algorithm is unique in the sense that  the actual mathematical expressions, that are prohibitively large, need not be explicitly obtained. The diversity gain due to multiple relays is shown through plots of the analytical BER, well supported by simulations. 
%
%\end{abstract}
% IEEEtran.cls defaults to using nonbold math in the Abstract.
% This preserves the distinction between vectors and scalars. However,
% if the journal you are submitting to favors bold math in the abstract,
% then you can use LaTeX's standard command \boldmath at the very start
% of the abstract to achieve this. Many IEEE journals frown on math
% in the abstract anyway.

% Note that keywords are not normally used for peerreview papers.
%\begin{IEEEkeywords}
%Cooperative diversity, decode and forward, piecewise linear
%\end{IEEEkeywords}



% For peer review papers, you can put extra information on the cover
% page as needed:
% \ifCLASSOPTIONpeerreview
% \begin{center} \bfseries EDICS Category: 3-BBND \end{center}
% \fi
%
% For peerreview papers, this IEEEtran command inserts a page break and
% creates the second title. It will be ignored for other modes.
%\IEEEpeerreviewmaketitle




	\item  A die is loaded in such a way that each odd number is twice as likely to occur as
each even number. Find $P(G)$, where $G$ is the event that a number greater than
3 occurs on a single roll of the die.
\\
\solution
		%\begin{table}[H]
	\centering
\begin{tabular}{|c|c|c|}
\hline
Random variable &Value &Definition\\ \hline
\multirow{3}{*}{X} &0 &Slips of Rs 1\\
&1 &Slips of Rs 5\\
&2 &Slips of Rs 13\\ \hline
\multirow{2}{*}{Y} &0 &Box A\\
&1 &Box B\\\hline
\end{tabular}
\caption{}
\label{tab:Distribution}
\end{table}
See \tabref{tab:Distribution}.
\begin{align}
p_{Y}\brak{k}= \begin{cases} 
      \frac{1}{3} & {k=0} \\
      \frac{2}{3 }& {k=1} 
   \end{cases}
   \\
p_{Y|X}\brak{0|0} = \frac{19}{25}\, 
p_{Y|X}\brak{0|1} = \frac{6}{25}\,
p_{Y|X}\brak{1|0} = \frac{45}{50}\,
p_{Y|X}\brak{1|2} = \frac{5}{50}
\end{align}
The desired probability is the probability that a slip drawn at random is marked other than Rs 1,
\begin{align}
&=1-p_X\brak{0}\\
&= p_X(1) + p_X(2)
\end{align}
Using Bayes theorem,
\begin{align}
&= p_Y\brak{0} \times \pr{Y=0 | X=1} + p_Y\brak{1} \times \pr{Y=1|X=2}\\
&=\frac{1}{3} \times \frac{6}{25} + \frac{2}{3} \times \frac{5}{50}\\
&=\frac{11}{75}
\end{align}

\newpage

%\tableofcontents

\bigskip

\renewcommand{\thefigure}{\theenumi}
\renewcommand{\thetable}{\theenumi}
%\renewcommand{\theequation}{\theenumi}

%\begin{abstract}
%%\boldmath
%In this letter, an algorithm for evaluating the exact analytical bit error rate  (BER)  for the piecewise linear (PL) combiner for  multiple relays is presented. Previous results were available only for upto three relays. The algorithm is unique in the sense that  the actual mathematical expressions, that are prohibitively large, need not be explicitly obtained. The diversity gain due to multiple relays is shown through plots of the analytical BER, well supported by simulations. 
%
%\end{abstract}
% IEEEtran.cls defaults to using nonbold math in the Abstract.
% This preserves the distinction between vectors and scalars. However,
% if the journal you are submitting to favors bold math in the abstract,
% then you can use LaTeX's standard command \boldmath at the very start
% of the abstract to achieve this. Many IEEE journals frown on math
% in the abstract anyway.

% Note that keywords are not normally used for peerreview papers.
%\begin{IEEEkeywords}
%Cooperative diversity, decode and forward, piecewise linear
%\end{IEEEkeywords}



% For peer review papers, you can put extra information on the cover
% page as needed:
% \ifCLASSOPTIONpeerreview
% \begin{center} \bfseries EDICS Category: 3-BBND \end{center}
% \fi
%
% For peerreview papers, this IEEEtran command inserts a page break and
% creates the second title. It will be ignored for other modes.
%\IEEEpeerreviewmaketitle




	\item All the jacks, queens and kings are removed from a deck of 52 playing cards. The remaining cards are well shuffled and then one card is drawn at random. Giving ace a value 1 similar value for other cards, find the probability that the card has a value 
		\begin{enumerate}
			\item 7
			\item greater than 7
			\item less than 7
		\end{enumerate}
		%Number of cards left after removing all jacks, queens and kings 
\begin{align}
N	= 52 - 4\times 3
	= 40
\end{align}
%\begin{table}[H]
%\def\arraystretch{1.2}
%\begin{tabular}{|c|c|c|}
%\hline
%	\textbf{Parameter} &\textbf{Value} &\textbf{Description}\\ \hline
%	$X$ &1-10 &Represents the value of the card picked \\ \hline
%\end{tabular}
%\end{table}
Let $1 \le X \le 10$ be the value of the card picked.  Then,
\begin{align}
	p_X(k) &= \Pr(X=k)\ \forall\ 1 \leq k \leq 10\\
	&= \frac{4\times 1}{40}\\
	&= \frac{1}{10}\\
	\therefore p_X(k) &= 
	\begin{cases}
		\frac{1}{10} & 1 \leq k \leq 10\\
		0 & \text{otherwise}
	\end{cases}
\end{align}
and
\begin{align}
	F_{X}(k) &= \sum_{m=0}^{k}p_{X}(m) \quad 1 \leq k \leq 10\\
	&= \frac{k}{10}\\
	\therefore F_{X}(k) &= 
	\begin{cases}
		0 & k \leq 0\\
		\frac{k}{10} & 1\leq k \leq 10\\
		1 & k > 10 
	\end{cases}
\end{align}
\begin{enumerate}
	\item Probability that card has value equal to 7 is
		\begin{align}
			 p_{X}(7)
			= \frac{1}{10}
		\end{align}
	\item Probability that card has value greater than 7 is
		\begin{align}
			1 - F_X(7)
			&= 1 - \frac{7}{10}
			\\
			&= \frac{3}{10}
		\end{align}
	\item Probability that card has value less than 7 is
		\begin{align}
			 F_{X}(6)
			=\frac{6}{10}
		\end{align}
\end{enumerate}

  \item A Lot consists of 48 mobile phones of which 42 are good, 3 have only minor defects and 3 have major defects.Varnika will buy a phone if it is good but the trader will only buy a mobile if it has no major defects. One phone is selected at random from the lot. What is the probability that it is
\begin{enumerate}
	\item acceptable to Varnika?
            \item acceptable to the trader?
\end{enumerate}
\solution
	%\begin{table}[H]
	\centering
\begin{tabular}{|c|c|c|}
\hline
Random variable &Value &Definition\\ \hline
\multirow{3}{*}{X} &0 &Slips of Rs 1\\
&1 &Slips of Rs 5\\
&2 &Slips of Rs 13\\ \hline
\multirow{2}{*}{Y} &0 &Box A\\
&1 &Box B\\\hline
\end{tabular}
\caption{}
\label{tab:Distribution}
\end{table}
See \tabref{tab:Distribution}.
\begin{align}
p_{Y}\brak{k}= \begin{cases} 
      \frac{1}{3} & {k=0} \\
      \frac{2}{3 }& {k=1} 
   \end{cases}
   \\
p_{Y|X}\brak{0|0} = \frac{19}{25}\, 
p_{Y|X}\brak{0|1} = \frac{6}{25}\,
p_{Y|X}\brak{1|0} = \frac{45}{50}\,
p_{Y|X}\brak{1|2} = \frac{5}{50}
\end{align}
The desired probability is the probability that a slip drawn at random is marked other than Rs 1,
\begin{align}
&=1-p_X\brak{0}\\
&= p_X(1) + p_X(2)
\end{align}
Using Bayes theorem,
\begin{align}
&= p_Y\brak{0} \times \pr{Y=0 | X=1} + p_Y\brak{1} \times \pr{Y=1|X=2}\\
&=\frac{1}{3} \times \frac{6}{25} + \frac{2}{3} \times \frac{5}{50}\\
&=\frac{11}{75}
\end{align}

\newpage

%\tableofcontents

\bigskip

\renewcommand{\thefigure}{\theenumi}
\renewcommand{\thetable}{\theenumi}
%\renewcommand{\theequation}{\theenumi}

%\begin{abstract}
%%\boldmath
%In this letter, an algorithm for evaluating the exact analytical bit error rate  (BER)  for the piecewise linear (PL) combiner for  multiple relays is presented. Previous results were available only for upto three relays. The algorithm is unique in the sense that  the actual mathematical expressions, that are prohibitively large, need not be explicitly obtained. The diversity gain due to multiple relays is shown through plots of the analytical BER, well supported by simulations. 
%
%\end{abstract}
% IEEEtran.cls defaults to using nonbold math in the Abstract.
% This preserves the distinction between vectors and scalars. However,
% if the journal you are submitting to favors bold math in the abstract,
% then you can use LaTeX's standard command \boldmath at the very start
% of the abstract to achieve this. Many IEEE journals frown on math
% in the abstract anyway.

% Note that keywords are not normally used for peerreview papers.
%\begin{IEEEkeywords}
%Cooperative diversity, decode and forward, piecewise linear
%\end{IEEEkeywords}



% For peer review papers, you can put extra information on the cover
% page as needed:
% \ifCLASSOPTIONpeerreview
% \begin{center} \bfseries EDICS Category: 3-BBND \end{center}
% \fi
%
% For peerreview papers, this IEEEtran command inserts a page break and
% creates the second title. It will be ignored for other modes.
%\IEEEpeerreviewmaketitle




 \item A student says that if you throw a die, it will show up 1 or not 1. Therefore, the probability of getting 1 and the probability of getting 'not 1' each is equal to $\frac{1}{2}$. Is this correct? Give reasons.\\
 \solution
        %\begin{table}[H]
	\centering
\begin{tabular}{|c|c|c|}
\hline
Random variable &Value &Definition\\ \hline
\multirow{3}{*}{X} &0 &Slips of Rs 1\\
&1 &Slips of Rs 5\\
&2 &Slips of Rs 13\\ \hline
\multirow{2}{*}{Y} &0 &Box A\\
&1 &Box B\\\hline
\end{tabular}
\caption{}
\label{tab:Distribution}
\end{table}
See \tabref{tab:Distribution}.
\begin{align}
p_{Y}\brak{k}= \begin{cases} 
      \frac{1}{3} & {k=0} \\
      \frac{2}{3 }& {k=1} 
   \end{cases}
   \\
p_{Y|X}\brak{0|0} = \frac{19}{25}\, 
p_{Y|X}\brak{0|1} = \frac{6}{25}\,
p_{Y|X}\brak{1|0} = \frac{45}{50}\,
p_{Y|X}\brak{1|2} = \frac{5}{50}
\end{align}
The desired probability is the probability that a slip drawn at random is marked other than Rs 1,
\begin{align}
&=1-p_X\brak{0}\\
&= p_X(1) + p_X(2)
\end{align}
Using Bayes theorem,
\begin{align}
&= p_Y\brak{0} \times \pr{Y=0 | X=1} + p_Y\brak{1} \times \pr{Y=1|X=2}\\
&=\frac{1}{3} \times \frac{6}{25} + \frac{2}{3} \times \frac{5}{50}\\
&=\frac{11}{75}
\end{align}

\newpage

%\tableofcontents

\bigskip

\renewcommand{\thefigure}{\theenumi}
\renewcommand{\thetable}{\theenumi}
%\renewcommand{\theequation}{\theenumi}

%\begin{abstract}
%%\boldmath
%In this letter, an algorithm for evaluating the exact analytical bit error rate  (BER)  for the piecewise linear (PL) combiner for  multiple relays is presented. Previous results were available only for upto three relays. The algorithm is unique in the sense that  the actual mathematical expressions, that are prohibitively large, need not be explicitly obtained. The diversity gain due to multiple relays is shown through plots of the analytical BER, well supported by simulations. 
%
%\end{abstract}
% IEEEtran.cls defaults to using nonbold math in the Abstract.
% This preserves the distinction between vectors and scalars. However,
% if the journal you are submitting to favors bold math in the abstract,
% then you can use LaTeX's standard command \boldmath at the very start
% of the abstract to achieve this. Many IEEE journals frown on math
% in the abstract anyway.

% Note that keywords are not normally used for peerreview papers.
%\begin{IEEEkeywords}
%Cooperative diversity, decode and forward, piecewise linear
%\end{IEEEkeywords}



% For peer review papers, you can put extra information on the cover
% page as needed:
% \ifCLASSOPTIONpeerreview
% \begin{center} \bfseries EDICS Category: 3-BBND \end{center}
% \fi
%
% For peerreview papers, this IEEEtran command inserts a page break and
% creates the second title. It will be ignored for other modes.
%\IEEEpeerreviewmaketitle




   \item Four candidates A, B, C, D have ap-
plied for the assignment to coach a school cricket
team. If A is twice as likely to be selected as B, and
B and C are given about the same chance of being
selected, while C is twice as likely to be selected
as D, what are the probabilities that
\begin{enumerate}
\item C will be selected?
\item A will not be selected?
\end{enumerate}
	%\begin{table}[H]
	\centering
\begin{tabular}{|c|c|c|}
\hline
Random variable &Value &Definition\\ \hline
\multirow{3}{*}{X} &0 &Slips of Rs 1\\
&1 &Slips of Rs 5\\
&2 &Slips of Rs 13\\ \hline
\multirow{2}{*}{Y} &0 &Box A\\
&1 &Box B\\\hline
\end{tabular}
\caption{}
\label{tab:Distribution}
\end{table}
See \tabref{tab:Distribution}.
\begin{align}
p_{Y}\brak{k}= \begin{cases} 
      \frac{1}{3} & {k=0} \\
      \frac{2}{3 }& {k=1} 
   \end{cases}
   \\
p_{Y|X}\brak{0|0} = \frac{19}{25}\, 
p_{Y|X}\brak{0|1} = \frac{6}{25}\,
p_{Y|X}\brak{1|0} = \frac{45}{50}\,
p_{Y|X}\brak{1|2} = \frac{5}{50}
\end{align}
The desired probability is the probability that a slip drawn at random is marked other than Rs 1,
\begin{align}
&=1-p_X\brak{0}\\
&= p_X(1) + p_X(2)
\end{align}
Using Bayes theorem,
\begin{align}
&= p_Y\brak{0} \times \pr{Y=0 | X=1} + p_Y\brak{1} \times \pr{Y=1|X=2}\\
&=\frac{1}{3} \times \frac{6}{25} + \frac{2}{3} \times \frac{5}{50}\\
&=\frac{11}{75}
\end{align}

\newpage

%\tableofcontents

\bigskip

\renewcommand{\thefigure}{\theenumi}
\renewcommand{\thetable}{\theenumi}
%\renewcommand{\theequation}{\theenumi}

%\begin{abstract}
%%\boldmath
%In this letter, an algorithm for evaluating the exact analytical bit error rate  (BER)  for the piecewise linear (PL) combiner for  multiple relays is presented. Previous results were available only for upto three relays. The algorithm is unique in the sense that  the actual mathematical expressions, that are prohibitively large, need not be explicitly obtained. The diversity gain due to multiple relays is shown through plots of the analytical BER, well supported by simulations. 
%
%\end{abstract}
% IEEEtran.cls defaults to using nonbold math in the Abstract.
% This preserves the distinction between vectors and scalars. However,
% if the journal you are submitting to favors bold math in the abstract,
% then you can use LaTeX's standard command \boldmath at the very start
% of the abstract to achieve this. Many IEEE journals frown on math
% in the abstract anyway.

% Note that keywords are not normally used for peerreview papers.
%\begin{IEEEkeywords}
%Cooperative diversity, decode and forward, piecewise linear
%\end{IEEEkeywords}



% For peer review papers, you can put extra information on the cover
% page as needed:
% \ifCLASSOPTIONpeerreview
% \begin{center} \bfseries EDICS Category: 3-BBND \end{center}
% \fi
%
% For peerreview papers, this IEEEtran command inserts a page break and
% creates the second title. It will be ignored for other modes.
%\IEEEpeerreviewmaketitle




 \item A bag contain 24 balls of which $x$ balls are red, $2x$ are white and $3x$ are blue. A ball is selected at random, What is the probability that it is
\begin{enumerate}[label=\alph*)]
\item not red ?
\item white ?
\end{enumerate}
%\begin{table}[H]
	\centering
\begin{tabular}{|c|c|c|}
\hline
Random variable &Value &Definition\\ \hline
\multirow{3}{*}{X} &0 &Slips of Rs 1\\
&1 &Slips of Rs 5\\
&2 &Slips of Rs 13\\ \hline
\multirow{2}{*}{Y} &0 &Box A\\
&1 &Box B\\\hline
\end{tabular}
\caption{}
\label{tab:Distribution}
\end{table}
See \tabref{tab:Distribution}.
\begin{align}
p_{Y}\brak{k}= \begin{cases} 
      \frac{1}{3} & {k=0} \\
      \frac{2}{3 }& {k=1} 
   \end{cases}
   \\
p_{Y|X}\brak{0|0} = \frac{19}{25}\, 
p_{Y|X}\brak{0|1} = \frac{6}{25}\,
p_{Y|X}\brak{1|0} = \frac{45}{50}\,
p_{Y|X}\brak{1|2} = \frac{5}{50}
\end{align}
The desired probability is the probability that a slip drawn at random is marked other than Rs 1,
\begin{align}
&=1-p_X\brak{0}\\
&= p_X(1) + p_X(2)
\end{align}
Using Bayes theorem,
\begin{align}
&= p_Y\brak{0} \times \pr{Y=0 | X=1} + p_Y\brak{1} \times \pr{Y=1|X=2}\\
&=\frac{1}{3} \times \frac{6}{25} + \frac{2}{3} \times \frac{5}{50}\\
&=\frac{11}{75}
\end{align}

\newpage

%\tableofcontents

\bigskip

\renewcommand{\thefigure}{\theenumi}
\renewcommand{\thetable}{\theenumi}
%\renewcommand{\theequation}{\theenumi}

%\begin{abstract}
%%\boldmath
%In this letter, an algorithm for evaluating the exact analytical bit error rate  (BER)  for the piecewise linear (PL) combiner for  multiple relays is presented. Previous results were available only for upto three relays. The algorithm is unique in the sense that  the actual mathematical expressions, that are prohibitively large, need not be explicitly obtained. The diversity gain due to multiple relays is shown through plots of the analytical BER, well supported by simulations. 
%
%\end{abstract}
% IEEEtran.cls defaults to using nonbold math in the Abstract.
% This preserves the distinction between vectors and scalars. However,
% if the journal you are submitting to favors bold math in the abstract,
% then you can use LaTeX's standard command \boldmath at the very start
% of the abstract to achieve this. Many IEEE journals frown on math
% in the abstract anyway.

% Note that keywords are not normally used for peerreview papers.
%\begin{IEEEkeywords}
%Cooperative diversity, decode and forward, piecewise linear
%\end{IEEEkeywords}



% For peer review papers, you can put extra information on the cover
% page as needed:
% \ifCLASSOPTIONpeerreview
% \begin{center} \bfseries EDICS Category: 3-BBND \end{center}
% \fi
%
% For peerreview papers, this IEEEtran command inserts a page break and
% creates the second title. It will be ignored for other modes.
%\IEEEpeerreviewmaketitle




If the letters of the word ASSASSINATION are arranged at random. Find the Probability that
\begin{enumerate}[label=(\alph*)]
\item Four $S's$ come consecutively in the word
\item Two  $I's$ and two $N's$ come together
\item All $A's$ are not coming together
\item No two $A's$ are coming together
\end{enumerate}
%\begin{table}[H]
	\centering
\begin{tabular}{|c|c|c|}
\hline
Random variable &Value &Definition\\ \hline
\multirow{3}{*}{X} &0 &Slips of Rs 1\\
&1 &Slips of Rs 5\\
&2 &Slips of Rs 13\\ \hline
\multirow{2}{*}{Y} &0 &Box A\\
&1 &Box B\\\hline
\end{tabular}
\caption{}
\label{tab:Distribution}
\end{table}
See \tabref{tab:Distribution}.
\begin{align}
p_{Y}\brak{k}= \begin{cases} 
      \frac{1}{3} & {k=0} \\
      \frac{2}{3 }& {k=1} 
   \end{cases}
   \\
p_{Y|X}\brak{0|0} = \frac{19}{25}\, 
p_{Y|X}\brak{0|1} = \frac{6}{25}\,
p_{Y|X}\brak{1|0} = \frac{45}{50}\,
p_{Y|X}\brak{1|2} = \frac{5}{50}
\end{align}
The desired probability is the probability that a slip drawn at random is marked other than Rs 1,
\begin{align}
&=1-p_X\brak{0}\\
&= p_X(1) + p_X(2)
\end{align}
Using Bayes theorem,
\begin{align}
&= p_Y\brak{0} \times \pr{Y=0 | X=1} + p_Y\brak{1} \times \pr{Y=1|X=2}\\
&=\frac{1}{3} \times \frac{6}{25} + \frac{2}{3} \times \frac{5}{50}\\
&=\frac{11}{75}
\end{align}

\newpage

%\tableofcontents

\bigskip

\renewcommand{\thefigure}{\theenumi}
\renewcommand{\thetable}{\theenumi}
%\renewcommand{\theequation}{\theenumi}

%\begin{abstract}
%%\boldmath
%In this letter, an algorithm for evaluating the exact analytical bit error rate  (BER)  for the piecewise linear (PL) combiner for  multiple relays is presented. Previous results were available only for upto three relays. The algorithm is unique in the sense that  the actual mathematical expressions, that are prohibitively large, need not be explicitly obtained. The diversity gain due to multiple relays is shown through plots of the analytical BER, well supported by simulations. 
%
%\end{abstract}
% IEEEtran.cls defaults to using nonbold math in the Abstract.
% This preserves the distinction between vectors and scalars. However,
% if the journal you are submitting to favors bold math in the abstract,
% then you can use LaTeX's standard command \boldmath at the very start
% of the abstract to achieve this. Many IEEE journals frown on math
% in the abstract anyway.

% Note that keywords are not normally used for peerreview papers.
%\begin{IEEEkeywords}
%Cooperative diversity, decode and forward, piecewise linear
%\end{IEEEkeywords}



% For peer review papers, you can put extra information on the cover
% page as needed:
% \ifCLASSOPTIONpeerreview
% \begin{center} \bfseries EDICS Category: 3-BBND \end{center}
% \fi
%
% For peerreview papers, this IEEEtran command inserts a page break and
% creates the second title. It will be ignored for other modes.
%\IEEEpeerreviewmaketitle




	\item One urn contains two black balls (labelled B1 and B2) and one white ball. A
	second urn contains one black ball and two white balls (labelled W1 and W2).
	Suppose the following experiment is performed. One of the two urns is chosen
	at random. Next a ball is randomly chosen from the urn. Then a second ball is
	chosen at random from the same urn without replacing the first ball.
	
	\begin{enumerate}
	\item What is the probability that two black balls are chosen?
	
	\item What is the probability that two balls of opposite colour are chosen?
	\end{enumerate}
	\solution
	%\begin{align}
    \label{eq:12.13.6.18.1}
	\because	\pr{A|B} &> \pr{A},\
\frac{\pr{AB}}{\pr{B}} > \pr{A}
\\
    \label{eq:12.13.6.18.2}
	\implies \pr{AB} &> \pr{A}\pr{B}
	\\
	\text{or, } \frac{\pr{AB}}{\pr{A}} &=\pr{B|A} > \pr{A}
\end{align}

\end{enumerate}

\item In a certain lottery 10,000 tickets are sold and ten equal prizes are awarded. What is the probability of not getting a prize if you buy (a) one ticket (b) two tickets (c) 10 tickets ?	
\\
\solution
		%\begin{enumerate}[label=\thesection.\arabic*,ref=\thesection.\theenumi]
	\item One card is drawn from a well-shuffled deck of 52 cards. Find the probability of getting
\begin{enumerate}
\item A king of red colour 
\item A face card 
\item A red face card
\item The jack of hearts
\item A spade
\item The queen of diamonds

\end{enumerate}
\solution
		%\begin{table}[H]
	\centering
\begin{tabular}{|c|c|c|}
\hline
Random variable &Value &Definition\\ \hline
\multirow{3}{*}{X} &0 &Slips of Rs 1\\
&1 &Slips of Rs 5\\
&2 &Slips of Rs 13\\ \hline
\multirow{2}{*}{Y} &0 &Box A\\
&1 &Box B\\\hline
\end{tabular}
\caption{}
\label{tab:Distribution}
\end{table}
See \tabref{tab:Distribution}.
\begin{align}
p_{Y}\brak{k}= \begin{cases} 
      \frac{1}{3} & {k=0} \\
      \frac{2}{3 }& {k=1} 
   \end{cases}
   \\
p_{Y|X}\brak{0|0} = \frac{19}{25}\, 
p_{Y|X}\brak{0|1} = \frac{6}{25}\,
p_{Y|X}\brak{1|0} = \frac{45}{50}\,
p_{Y|X}\brak{1|2} = \frac{5}{50}
\end{align}
The desired probability is the probability that a slip drawn at random is marked other than Rs 1,
\begin{align}
&=1-p_X\brak{0}\\
&= p_X(1) + p_X(2)
\end{align}
Using Bayes theorem,
\begin{align}
&= p_Y\brak{0} \times \pr{Y=0 | X=1} + p_Y\brak{1} \times \pr{Y=1|X=2}\\
&=\frac{1}{3} \times \frac{6}{25} + \frac{2}{3} \times \frac{5}{50}\\
&=\frac{11}{75}
\end{align}

\newpage

%\tableofcontents

\bigskip

\renewcommand{\thefigure}{\theenumi}
\renewcommand{\thetable}{\theenumi}
%\renewcommand{\theequation}{\theenumi}

%\begin{abstract}
%%\boldmath
%In this letter, an algorithm for evaluating the exact analytical bit error rate  (BER)  for the piecewise linear (PL) combiner for  multiple relays is presented. Previous results were available only for upto three relays. The algorithm is unique in the sense that  the actual mathematical expressions, that are prohibitively large, need not be explicitly obtained. The diversity gain due to multiple relays is shown through plots of the analytical BER, well supported by simulations. 
%
%\end{abstract}
% IEEEtran.cls defaults to using nonbold math in the Abstract.
% This preserves the distinction between vectors and scalars. However,
% if the journal you are submitting to favors bold math in the abstract,
% then you can use LaTeX's standard command \boldmath at the very start
% of the abstract to achieve this. Many IEEE journals frown on math
% in the abstract anyway.

% Note that keywords are not normally used for peerreview papers.
%\begin{IEEEkeywords}
%Cooperative diversity, decode and forward, piecewise linear
%\end{IEEEkeywords}



% For peer review papers, you can put extra information on the cover
% page as needed:
% \ifCLASSOPTIONpeerreview
% \begin{center} \bfseries EDICS Category: 3-BBND \end{center}
% \fi
%
% For peerreview papers, this IEEEtran command inserts a page break and
% creates the second title. It will be ignored for other modes.
%\IEEEpeerreviewmaketitle




	\item Five cards—the ten, jack, queen, king and ace of diamonds, are well-shuffled with their face downwards. One card is then picked up at random.
\begin{enumerate}
\item
What is the probability that the card is the queen? 
\item
If the queen is drawn and put aside, what is the probability that the second card picked up is (a) an ace? (b) a queen?\\
\end{enumerate}
\solution
		%\begin{enumerate}[label=\thesection.\arabic*,ref=\thesection.\theenumi]
	\item One card is drawn from a well-shuffled deck of 52 cards. Find the probability of getting
\begin{enumerate}
\item A king of red colour 
\item A face card 
\item A red face card
\item The jack of hearts
\item A spade
\item The queen of diamonds

\end{enumerate}
\solution
		%\input{ncert/10/15/1/14/main.tex}
	\item Five cards—the ten, jack, queen, king and ace of diamonds, are well-shuffled with their face downwards. One card is then picked up at random.
\begin{enumerate}
\item
What is the probability that the card is the queen? 
\item
If the queen is drawn and put aside, what is the probability that the second card picked up is (a) an ace? (b) a queen?\\
\end{enumerate}
\solution
		%\input{ncert/10/15/1/15/defs.tex}
	\item A bag contains $5$ red balls and some blue balls. If the probability of drawing a blue ball is double that if a red ball, determine the number of blue balls in the bag. 
		\\
\solution
		%\input{ncert/10/15/2/3/defs.tex}
	\item A card is selected from a pack of 52 cards.
 \begin{enumerate}[label=(\alph*)] 
                 \item How many points are there in the sample space?
                 \item Calculate the probability that the card is an ace of spades.
                 \item Calculate the probability that the card is (i) an ace and (ii) black card.
 \end{enumerate}
\solution
		%\input{ncert/11/16/3/4/main.tex}
\item Four cards are drawn from a well-shuffled deck of 52 cards. What is the probability of obtaining 3 diamonds and one spade.
\\
\solution
		%\input{ncert/11/16/4/2/defs.tex}
\item In a certain lottery 10,000 tickets are sold and ten equal prizes are awarded. What is the probability of not getting a prize if you buy (a) one ticket (b) two tickets (c) 10 tickets ?	
\\
\solution
		%\input{ncert/11/16/4/4/defs.tex}
		%
\item 
Out of 100 students, two sections of 40 and 60 are formed. If you and your friend are among the 100 students, what is the probability that
\begin{enumerate}
\item you both enter the same section?
\item you both enter the different sections?
\end{enumerate}
\solution
		%\input{ncert/11/16/4/5/defs.tex}
	\item 
The number lock of a suitcase has 4 wheels each labelled with ten digits i.e. from 0 to 9.The lock opens with a sequence of four digits with no repeats.What is the probability of a person getting the right sequence to open the suitcase.
\\
\solution
		%\input{ncert/11/16/4/10/defs.tex}
		%
\item 
Two cards are drawn at random and without replacement from a pack of 52 playing cards. Find the probability that both the cards are black.
\\
\solution
		%\input{ncert/12/13/2/2/defs.tex}
		\item A box of oranges is inspected by examining three randomly selected oranges drawn without replacement. If all the three oranges are good, the box is approved for sale, otherwise, it is rejected. Find the probability that a box containing 15 oranges out of which 12 are good and 3 are bad ones will be approved for sale.
		\label{ncert/12/13/2/3/defs.tex}
		\item Two balls are drawn at random with replacement from a box containing 10 black and 8 red balls. Find the probability that
		\label{ncert/12/13/2/12}
\begin{enumerate}
\item both balls are red.
\item first ball is black and second is red.
\item one of them is black and other is red.
\end{enumerate}

\item In a hostel, 60\% of the students read Hindi newspaper, 40\% read English newspaper and 20\% read both Hindi and English newspapers. A student is selected at random.
		\label{ncert/12/13/2/15}
\begin{enumerate}
\item Find the probability that she reads neither Hindi nor English newspapers.
\item If she reads Hindi newspaper, find the probability that she reads English newspaper.
\item If she reads English newspaper, find the probability that she reads Hindi newspaper.\\
\end{enumerate}
\item The probability of obtaining an even prime number on each die, when a pair of dice is rolled is 
\begin{enumerate}
    \item $0$ 
    
    \item $\frac{1}{3}$ 
    
    \item $\frac{1}{12}$ 
    
    \item $\frac{1}{36}$ 
\end{enumerate}
\solution
		%\input{ncert/12/13/2/17/defs.tex}
	\item A bag contains 4 red and 4 black balls, another bag contains 2 red and 6 black balls. One of the two bags is selected at random and a ball is drawn from the bag which is found to be red. Find the probability that the ball is drawn from the first bag.
\\
\solution
		%\input{ncert/12/13/3/2/main.tex}
  \item
  Cards with numbers 2 to 101 are placed in a box. A card is selected at random.Find the probability that the card has
\begin{enumerate}[label=(\roman*)]
	\item an even number 
	\item a square number
\end{enumerate}
\solution
%\input{exemplar/10/13/3/32/main.tex}
\item
The king, queen and jack of clubs are removed from a deck of 52 playing cards and then well shuffled. Now one card is drawn at random from the remaining cards.  Determine the probability that the card is
\begin{enumerate}[label=(\roman*)]
\item a club
\item 10 of hearts
\end{enumerate}
\solution
%\input{exemplar/10/13/3/29/main.tex}
\item A team of medical students doing their internship have to assist during surgeries
at a city hospital. The probabilities of surgeries rated as very complex, complex,
routine, simple or very simple are respectively, 0.15, 0.20, 0.31, 0.26, .08. Find
the probabilities that a particular surgery will be rated
\begin{enumerate}
	\item complex or very complex;
	\item neither very complex nor very simple;
	\item routine or complex
	\item routine or simple
\end{enumerate}
\solution
%\input{exemplar/11/16/3/8(1)/main.tex}
\item A card is selected from a pack of 52 cards.
\begin{enumerate}[label=(\alph*)]
    \item How many points are there in the sample space?
    \item Calculate the probability that the card is an ace of spades.
    \item Calculate the probability that the card is (i) an ace and (ii) black card.
\end{enumerate}
\solution
%\input{exemplar/11/16/3/4/main2.tex}
\item The probability that a non leap year selected at random will contain 53 sundays.
\\
\solution
%\input{exemplar/10/13/1/19/main.tex}
\item One of the four persons John, Rita, Aslam or Gurpreet will be promoted next
month. Consequently the sample space consists of four elementary outcomes
S = {John promoted, Rita promoted, Aslam promoted, Gurpreet promoted}
You are told that the chances of John’s promotion is same as that of Gurpreet,
Rita’s chances of promotion are twice as likely as Johns. Aslam’s chances are
four times that of John.
\begin{enumerate}
	\item Determine
	\begin{enumerate}
		\item P (John promoted)
		\item P (Rita promoted)
		\item P (Aslam promoted)
		\item P (Gurpreet promoted)
	\end{enumerate}
	\item If A = {John promoted or Gurpreet promoted}, find P (A).
\end{enumerate}
\solution
%\input{exemplar/11/16/3/10/main.tex}
\item A card is drawn from a deck of 52 cards. Find the probability of getting a king or a heart or a red card.\\
\solution
%\input{exemplar/11/16/3/15/main.tex}
\item The probability that a student will pass his examination is 0.73, the probability of
the student getting a compartment is 0.13, and the probability that the student will
either pass or get compartment is 0.96. State True or False.\\
\solution
%\input{exemplar/11/16/3/31/main.tex}
\item A card is selected from a pack of 52 cards\\
\begin{enumerate}[label=(\alph*)]
\item How many points are there in the sample space?
\item Calculate the probability that the cards is an ace of spades.
\item Calculate the probability that the card is (i) an ace (ii)black card.\\
\end{enumerate}
%\input{ncert/11/16/3/4_1/Prob_4.tex}
\item In a non-leap year, the probability of having 53 tuesdays or 53 wednesdays is\\
\solution
%\input{exemplar/11/16/3/18/main.tex}
\item There are 1000 sealed envelopes in a box, 10 of them contain a cash prize of
Rs 100 each, 100 of them contain a cash prize of Rs 50 each and 200 of them
contain a cash prize of Rs 10 each and rest do not contain any cash prize. If they
are well shuffled and an envelope is picked up out, what is the probability that it
contains no cash prize?\\
\solution
%\input{exemplar/10/13/3/34/main.tex}
\item 
A die is thrown and a card is selected at random from a deck of 52 playing cards. The probability of getting an even number on the die and a spade card.\\
\solution
%\input{exemplar/12/13/3/78/main.tex}
\item
If 4-digit numbers greater than 5,000 are randomly formed from the digits 0, 1, 3, 5, and 7, what is the probability of forming a number divisible by 5 when:
\begin{enumerate}
    \item The digits are repeated?
    \item The repetition of digits is not allowed?
\end{enumerate}
\solution
%\input{ncert/11/16/4/9/main.tex}
\item Consider the probability space $\brak{\Omega, \mathcal{G}, P}$ where $\Omega = [0,2]$ and $\mathcal{G} = \cbrak{\phi, \Omega, [0,1], (1,2]}$. Let $X$ and $Y$ be two functions on $\Omega$ defined as
\begin{align*}
    X(\omega) = 
    \begin{cases}
        1 & \text{if }\omega \in [0, 1]\\
        2 & \text{if }\omega \in (1, 2]
    \end{cases}
\end{align*}
and
\begin{align*}
    Y(\omega) = 
    \begin{cases}
        2 & \text{if }\omega \in [0, 1.5]\\
        3 & \text{if }\omega \in (1.5, 2].
    \end{cases}
\end{align*}
Then which one of the following statements is true?
\begin{enumerate}
    \item [(A)] $X$ is a random variable with respect to $\mathcal{G}$, but $Y$ is not a random variable with respect to $\mathcal{G}$.
    \item [(B)] $Y$ is a random variable with respect to $\mathcal{G}$, but $X$ is not a random variable with respect to $\mathcal{G}$.
    \item [(C)] Neither $X$ nor $Y$ is a random variable with respect to $\mathcal{G}$.
    \item [(D)] Both $X$ and $Y$ are random variables with respect to $\mathcal{G}$.
\end{enumerate} \hfill (GATE ST 2023)\\
\solution
%\input{gate/ST/2023/14/main.tex}
	\item  A die is loaded in such a way that each odd number is twice as likely to occur as
each even number. Find $P(G)$, where $G$ is the event that a number greater than
3 occurs on a single roll of the die.
\\
\solution
		%\input{exemplar/11/16/3/5/main.tex}
	\item All the jacks, queens and kings are removed from a deck of 52 playing cards. The remaining cards are well shuffled and then one card is drawn at random. Giving ace a value 1 similar value for other cards, find the probability that the card has a value 
		\begin{enumerate}
			\item 7
			\item greater than 7
			\item less than 7
		\end{enumerate}
		%\input{exemplar/10/13/3/30/main.tex}
  \item A Lot consists of 48 mobile phones of which 42 are good, 3 have only minor defects and 3 have major defects.Varnika will buy a phone if it is good but the trader will only buy a mobile if it has no major defects. One phone is selected at random from the lot. What is the probability that it is
\begin{enumerate}
	\item acceptable to Varnika?
            \item acceptable to the trader?
\end{enumerate}
\solution
	%\input{exemplar/10/13/3/40/main.tex}
 \item A student says that if you throw a die, it will show up 1 or not 1. Therefore, the probability of getting 1 and the probability of getting 'not 1' each is equal to $\frac{1}{2}$. Is this correct? Give reasons.\\
 \solution
        %\input{exemplar/10/13/2/9/main.tex}
   \item Four candidates A, B, C, D have ap-
plied for the assignment to coach a school cricket
team. If A is twice as likely to be selected as B, and
B and C are given about the same chance of being
selected, while C is twice as likely to be selected
as D, what are the probabilities that
\begin{enumerate}
\item C will be selected?
\item A will not be selected?
\end{enumerate}
	%\input{exemplar/11/16/3/9/main.tex}
 \item A bag contain 24 balls of which $x$ balls are red, $2x$ are white and $3x$ are blue. A ball is selected at random, What is the probability that it is
\begin{enumerate}[label=\alph*)]
\item not red ?
\item white ?
\end{enumerate}
%\input{exemplar/10/13/3/41/main.tex}
If the letters of the word ASSASSINATION are arranged at random. Find the Probability that
\begin{enumerate}[label=(\alph*)]
\item Four $S's$ come consecutively in the word
\item Two  $I's$ and two $N's$ come together
\item All $A's$ are not coming together
\item No two $A's$ are coming together
\end{enumerate}
%\input{exemplar/11/16/3/14/main.tex}
	\item One urn contains two black balls (labelled B1 and B2) and one white ball. A
	second urn contains one black ball and two white balls (labelled W1 and W2).
	Suppose the following experiment is performed. One of the two urns is chosen
	at random. Next a ball is randomly chosen from the urn. Then a second ball is
	chosen at random from the same urn without replacing the first ball.
	
	\begin{enumerate}
	\item What is the probability that two black balls are chosen?
	
	\item What is the probability that two balls of opposite colour are chosen?
	\end{enumerate}
	\solution
	%\input{exemplar/11/16/3/12/main1.tex}
\end{enumerate}

	\item A bag contains $5$ red balls and some blue balls. If the probability of drawing a blue ball is double that if a red ball, determine the number of blue balls in the bag. 
		\\
\solution
		%\begin{enumerate}[label=\thesection.\arabic*,ref=\thesection.\theenumi]
	\item One card is drawn from a well-shuffled deck of 52 cards. Find the probability of getting
\begin{enumerate}
\item A king of red colour 
\item A face card 
\item A red face card
\item The jack of hearts
\item A spade
\item The queen of diamonds

\end{enumerate}
\solution
		%\input{ncert/10/15/1/14/main.tex}
	\item Five cards—the ten, jack, queen, king and ace of diamonds, are well-shuffled with their face downwards. One card is then picked up at random.
\begin{enumerate}
\item
What is the probability that the card is the queen? 
\item
If the queen is drawn and put aside, what is the probability that the second card picked up is (a) an ace? (b) a queen?\\
\end{enumerate}
\solution
		%\input{ncert/10/15/1/15/defs.tex}
	\item A bag contains $5$ red balls and some blue balls. If the probability of drawing a blue ball is double that if a red ball, determine the number of blue balls in the bag. 
		\\
\solution
		%\input{ncert/10/15/2/3/defs.tex}
	\item A card is selected from a pack of 52 cards.
 \begin{enumerate}[label=(\alph*)] 
                 \item How many points are there in the sample space?
                 \item Calculate the probability that the card is an ace of spades.
                 \item Calculate the probability that the card is (i) an ace and (ii) black card.
 \end{enumerate}
\solution
		%\input{ncert/11/16/3/4/main.tex}
\item Four cards are drawn from a well-shuffled deck of 52 cards. What is the probability of obtaining 3 diamonds and one spade.
\\
\solution
		%\input{ncert/11/16/4/2/defs.tex}
\item In a certain lottery 10,000 tickets are sold and ten equal prizes are awarded. What is the probability of not getting a prize if you buy (a) one ticket (b) two tickets (c) 10 tickets ?	
\\
\solution
		%\input{ncert/11/16/4/4/defs.tex}
		%
\item 
Out of 100 students, two sections of 40 and 60 are formed. If you and your friend are among the 100 students, what is the probability that
\begin{enumerate}
\item you both enter the same section?
\item you both enter the different sections?
\end{enumerate}
\solution
		%\input{ncert/11/16/4/5/defs.tex}
	\item 
The number lock of a suitcase has 4 wheels each labelled with ten digits i.e. from 0 to 9.The lock opens with a sequence of four digits with no repeats.What is the probability of a person getting the right sequence to open the suitcase.
\\
\solution
		%\input{ncert/11/16/4/10/defs.tex}
		%
\item 
Two cards are drawn at random and without replacement from a pack of 52 playing cards. Find the probability that both the cards are black.
\\
\solution
		%\input{ncert/12/13/2/2/defs.tex}
		\item A box of oranges is inspected by examining three randomly selected oranges drawn without replacement. If all the three oranges are good, the box is approved for sale, otherwise, it is rejected. Find the probability that a box containing 15 oranges out of which 12 are good and 3 are bad ones will be approved for sale.
		\label{ncert/12/13/2/3/defs.tex}
		\item Two balls are drawn at random with replacement from a box containing 10 black and 8 red balls. Find the probability that
		\label{ncert/12/13/2/12}
\begin{enumerate}
\item both balls are red.
\item first ball is black and second is red.
\item one of them is black and other is red.
\end{enumerate}

\item In a hostel, 60\% of the students read Hindi newspaper, 40\% read English newspaper and 20\% read both Hindi and English newspapers. A student is selected at random.
		\label{ncert/12/13/2/15}
\begin{enumerate}
\item Find the probability that she reads neither Hindi nor English newspapers.
\item If she reads Hindi newspaper, find the probability that she reads English newspaper.
\item If she reads English newspaper, find the probability that she reads Hindi newspaper.\\
\end{enumerate}
\item The probability of obtaining an even prime number on each die, when a pair of dice is rolled is 
\begin{enumerate}
    \item $0$ 
    
    \item $\frac{1}{3}$ 
    
    \item $\frac{1}{12}$ 
    
    \item $\frac{1}{36}$ 
\end{enumerate}
\solution
		%\input{ncert/12/13/2/17/defs.tex}
	\item A bag contains 4 red and 4 black balls, another bag contains 2 red and 6 black balls. One of the two bags is selected at random and a ball is drawn from the bag which is found to be red. Find the probability that the ball is drawn from the first bag.
\\
\solution
		%\input{ncert/12/13/3/2/main.tex}
  \item
  Cards with numbers 2 to 101 are placed in a box. A card is selected at random.Find the probability that the card has
\begin{enumerate}[label=(\roman*)]
	\item an even number 
	\item a square number
\end{enumerate}
\solution
%\input{exemplar/10/13/3/32/main.tex}
\item
The king, queen and jack of clubs are removed from a deck of 52 playing cards and then well shuffled. Now one card is drawn at random from the remaining cards.  Determine the probability that the card is
\begin{enumerate}[label=(\roman*)]
\item a club
\item 10 of hearts
\end{enumerate}
\solution
%\input{exemplar/10/13/3/29/main.tex}
\item A team of medical students doing their internship have to assist during surgeries
at a city hospital. The probabilities of surgeries rated as very complex, complex,
routine, simple or very simple are respectively, 0.15, 0.20, 0.31, 0.26, .08. Find
the probabilities that a particular surgery will be rated
\begin{enumerate}
	\item complex or very complex;
	\item neither very complex nor very simple;
	\item routine or complex
	\item routine or simple
\end{enumerate}
\solution
%\input{exemplar/11/16/3/8(1)/main.tex}
\item A card is selected from a pack of 52 cards.
\begin{enumerate}[label=(\alph*)]
    \item How many points are there in the sample space?
    \item Calculate the probability that the card is an ace of spades.
    \item Calculate the probability that the card is (i) an ace and (ii) black card.
\end{enumerate}
\solution
%\input{exemplar/11/16/3/4/main2.tex}
\item The probability that a non leap year selected at random will contain 53 sundays.
\\
\solution
%\input{exemplar/10/13/1/19/main.tex}
\item One of the four persons John, Rita, Aslam or Gurpreet will be promoted next
month. Consequently the sample space consists of four elementary outcomes
S = {John promoted, Rita promoted, Aslam promoted, Gurpreet promoted}
You are told that the chances of John’s promotion is same as that of Gurpreet,
Rita’s chances of promotion are twice as likely as Johns. Aslam’s chances are
four times that of John.
\begin{enumerate}
	\item Determine
	\begin{enumerate}
		\item P (John promoted)
		\item P (Rita promoted)
		\item P (Aslam promoted)
		\item P (Gurpreet promoted)
	\end{enumerate}
	\item If A = {John promoted or Gurpreet promoted}, find P (A).
\end{enumerate}
\solution
%\input{exemplar/11/16/3/10/main.tex}
\item A card is drawn from a deck of 52 cards. Find the probability of getting a king or a heart or a red card.\\
\solution
%\input{exemplar/11/16/3/15/main.tex}
\item The probability that a student will pass his examination is 0.73, the probability of
the student getting a compartment is 0.13, and the probability that the student will
either pass or get compartment is 0.96. State True or False.\\
\solution
%\input{exemplar/11/16/3/31/main.tex}
\item A card is selected from a pack of 52 cards\\
\begin{enumerate}[label=(\alph*)]
\item How many points are there in the sample space?
\item Calculate the probability that the cards is an ace of spades.
\item Calculate the probability that the card is (i) an ace (ii)black card.\\
\end{enumerate}
%\input{ncert/11/16/3/4_1/Prob_4.tex}
\item In a non-leap year, the probability of having 53 tuesdays or 53 wednesdays is\\
\solution
%\input{exemplar/11/16/3/18/main.tex}
\item There are 1000 sealed envelopes in a box, 10 of them contain a cash prize of
Rs 100 each, 100 of them contain a cash prize of Rs 50 each and 200 of them
contain a cash prize of Rs 10 each and rest do not contain any cash prize. If they
are well shuffled and an envelope is picked up out, what is the probability that it
contains no cash prize?\\
\solution
%\input{exemplar/10/13/3/34/main.tex}
\item 
A die is thrown and a card is selected at random from a deck of 52 playing cards. The probability of getting an even number on the die and a spade card.\\
\solution
%\input{exemplar/12/13/3/78/main.tex}
\item
If 4-digit numbers greater than 5,000 are randomly formed from the digits 0, 1, 3, 5, and 7, what is the probability of forming a number divisible by 5 when:
\begin{enumerate}
    \item The digits are repeated?
    \item The repetition of digits is not allowed?
\end{enumerate}
\solution
%\input{ncert/11/16/4/9/main.tex}
\item Consider the probability space $\brak{\Omega, \mathcal{G}, P}$ where $\Omega = [0,2]$ and $\mathcal{G} = \cbrak{\phi, \Omega, [0,1], (1,2]}$. Let $X$ and $Y$ be two functions on $\Omega$ defined as
\begin{align*}
    X(\omega) = 
    \begin{cases}
        1 & \text{if }\omega \in [0, 1]\\
        2 & \text{if }\omega \in (1, 2]
    \end{cases}
\end{align*}
and
\begin{align*}
    Y(\omega) = 
    \begin{cases}
        2 & \text{if }\omega \in [0, 1.5]\\
        3 & \text{if }\omega \in (1.5, 2].
    \end{cases}
\end{align*}
Then which one of the following statements is true?
\begin{enumerate}
    \item [(A)] $X$ is a random variable with respect to $\mathcal{G}$, but $Y$ is not a random variable with respect to $\mathcal{G}$.
    \item [(B)] $Y$ is a random variable with respect to $\mathcal{G}$, but $X$ is not a random variable with respect to $\mathcal{G}$.
    \item [(C)] Neither $X$ nor $Y$ is a random variable with respect to $\mathcal{G}$.
    \item [(D)] Both $X$ and $Y$ are random variables with respect to $\mathcal{G}$.
\end{enumerate} \hfill (GATE ST 2023)\\
\solution
%\input{gate/ST/2023/14/main.tex}
	\item  A die is loaded in such a way that each odd number is twice as likely to occur as
each even number. Find $P(G)$, where $G$ is the event that a number greater than
3 occurs on a single roll of the die.
\\
\solution
		%\input{exemplar/11/16/3/5/main.tex}
	\item All the jacks, queens and kings are removed from a deck of 52 playing cards. The remaining cards are well shuffled and then one card is drawn at random. Giving ace a value 1 similar value for other cards, find the probability that the card has a value 
		\begin{enumerate}
			\item 7
			\item greater than 7
			\item less than 7
		\end{enumerate}
		%\input{exemplar/10/13/3/30/main.tex}
  \item A Lot consists of 48 mobile phones of which 42 are good, 3 have only minor defects and 3 have major defects.Varnika will buy a phone if it is good but the trader will only buy a mobile if it has no major defects. One phone is selected at random from the lot. What is the probability that it is
\begin{enumerate}
	\item acceptable to Varnika?
            \item acceptable to the trader?
\end{enumerate}
\solution
	%\input{exemplar/10/13/3/40/main.tex}
 \item A student says that if you throw a die, it will show up 1 or not 1. Therefore, the probability of getting 1 and the probability of getting 'not 1' each is equal to $\frac{1}{2}$. Is this correct? Give reasons.\\
 \solution
        %\input{exemplar/10/13/2/9/main.tex}
   \item Four candidates A, B, C, D have ap-
plied for the assignment to coach a school cricket
team. If A is twice as likely to be selected as B, and
B and C are given about the same chance of being
selected, while C is twice as likely to be selected
as D, what are the probabilities that
\begin{enumerate}
\item C will be selected?
\item A will not be selected?
\end{enumerate}
	%\input{exemplar/11/16/3/9/main.tex}
 \item A bag contain 24 balls of which $x$ balls are red, $2x$ are white and $3x$ are blue. A ball is selected at random, What is the probability that it is
\begin{enumerate}[label=\alph*)]
\item not red ?
\item white ?
\end{enumerate}
%\input{exemplar/10/13/3/41/main.tex}
If the letters of the word ASSASSINATION are arranged at random. Find the Probability that
\begin{enumerate}[label=(\alph*)]
\item Four $S's$ come consecutively in the word
\item Two  $I's$ and two $N's$ come together
\item All $A's$ are not coming together
\item No two $A's$ are coming together
\end{enumerate}
%\input{exemplar/11/16/3/14/main.tex}
	\item One urn contains two black balls (labelled B1 and B2) and one white ball. A
	second urn contains one black ball and two white balls (labelled W1 and W2).
	Suppose the following experiment is performed. One of the two urns is chosen
	at random. Next a ball is randomly chosen from the urn. Then a second ball is
	chosen at random from the same urn without replacing the first ball.
	
	\begin{enumerate}
	\item What is the probability that two black balls are chosen?
	
	\item What is the probability that two balls of opposite colour are chosen?
	\end{enumerate}
	\solution
	%\input{exemplar/11/16/3/12/main1.tex}
\end{enumerate}

	\item A card is selected from a pack of 52 cards.
 \begin{enumerate}[label=(\alph*)] 
                 \item How many points are there in the sample space?
                 \item Calculate the probability that the card is an ace of spades.
                 \item Calculate the probability that the card is (i) an ace and (ii) black card.
 \end{enumerate}
\solution
		%\begin{table}[H]
	\centering
\begin{tabular}{|c|c|c|}
\hline
Random variable &Value &Definition\\ \hline
\multirow{3}{*}{X} &0 &Slips of Rs 1\\
&1 &Slips of Rs 5\\
&2 &Slips of Rs 13\\ \hline
\multirow{2}{*}{Y} &0 &Box A\\
&1 &Box B\\\hline
\end{tabular}
\caption{}
\label{tab:Distribution}
\end{table}
See \tabref{tab:Distribution}.
\begin{align}
p_{Y}\brak{k}= \begin{cases} 
      \frac{1}{3} & {k=0} \\
      \frac{2}{3 }& {k=1} 
   \end{cases}
   \\
p_{Y|X}\brak{0|0} = \frac{19}{25}\, 
p_{Y|X}\brak{0|1} = \frac{6}{25}\,
p_{Y|X}\brak{1|0} = \frac{45}{50}\,
p_{Y|X}\brak{1|2} = \frac{5}{50}
\end{align}
The desired probability is the probability that a slip drawn at random is marked other than Rs 1,
\begin{align}
&=1-p_X\brak{0}\\
&= p_X(1) + p_X(2)
\end{align}
Using Bayes theorem,
\begin{align}
&= p_Y\brak{0} \times \pr{Y=0 | X=1} + p_Y\brak{1} \times \pr{Y=1|X=2}\\
&=\frac{1}{3} \times \frac{6}{25} + \frac{2}{3} \times \frac{5}{50}\\
&=\frac{11}{75}
\end{align}

\newpage

%\tableofcontents

\bigskip

\renewcommand{\thefigure}{\theenumi}
\renewcommand{\thetable}{\theenumi}
%\renewcommand{\theequation}{\theenumi}

%\begin{abstract}
%%\boldmath
%In this letter, an algorithm for evaluating the exact analytical bit error rate  (BER)  for the piecewise linear (PL) combiner for  multiple relays is presented. Previous results were available only for upto three relays. The algorithm is unique in the sense that  the actual mathematical expressions, that are prohibitively large, need not be explicitly obtained. The diversity gain due to multiple relays is shown through plots of the analytical BER, well supported by simulations. 
%
%\end{abstract}
% IEEEtran.cls defaults to using nonbold math in the Abstract.
% This preserves the distinction between vectors and scalars. However,
% if the journal you are submitting to favors bold math in the abstract,
% then you can use LaTeX's standard command \boldmath at the very start
% of the abstract to achieve this. Many IEEE journals frown on math
% in the abstract anyway.

% Note that keywords are not normally used for peerreview papers.
%\begin{IEEEkeywords}
%Cooperative diversity, decode and forward, piecewise linear
%\end{IEEEkeywords}



% For peer review papers, you can put extra information on the cover
% page as needed:
% \ifCLASSOPTIONpeerreview
% \begin{center} \bfseries EDICS Category: 3-BBND \end{center}
% \fi
%
% For peerreview papers, this IEEEtran command inserts a page break and
% creates the second title. It will be ignored for other modes.
%\IEEEpeerreviewmaketitle




\item Four cards are drawn from a well-shuffled deck of 52 cards. What is the probability of obtaining 3 diamonds and one spade.
\\
\solution
		%\begin{enumerate}[label=\thesection.\arabic*,ref=\thesection.\theenumi]
	\item One card is drawn from a well-shuffled deck of 52 cards. Find the probability of getting
\begin{enumerate}
\item A king of red colour 
\item A face card 
\item A red face card
\item The jack of hearts
\item A spade
\item The queen of diamonds

\end{enumerate}
\solution
		%\input{ncert/10/15/1/14/main.tex}
	\item Five cards—the ten, jack, queen, king and ace of diamonds, are well-shuffled with their face downwards. One card is then picked up at random.
\begin{enumerate}
\item
What is the probability that the card is the queen? 
\item
If the queen is drawn and put aside, what is the probability that the second card picked up is (a) an ace? (b) a queen?\\
\end{enumerate}
\solution
		%\input{ncert/10/15/1/15/defs.tex}
	\item A bag contains $5$ red balls and some blue balls. If the probability of drawing a blue ball is double that if a red ball, determine the number of blue balls in the bag. 
		\\
\solution
		%\input{ncert/10/15/2/3/defs.tex}
	\item A card is selected from a pack of 52 cards.
 \begin{enumerate}[label=(\alph*)] 
                 \item How many points are there in the sample space?
                 \item Calculate the probability that the card is an ace of spades.
                 \item Calculate the probability that the card is (i) an ace and (ii) black card.
 \end{enumerate}
\solution
		%\input{ncert/11/16/3/4/main.tex}
\item Four cards are drawn from a well-shuffled deck of 52 cards. What is the probability of obtaining 3 diamonds and one spade.
\\
\solution
		%\input{ncert/11/16/4/2/defs.tex}
\item In a certain lottery 10,000 tickets are sold and ten equal prizes are awarded. What is the probability of not getting a prize if you buy (a) one ticket (b) two tickets (c) 10 tickets ?	
\\
\solution
		%\input{ncert/11/16/4/4/defs.tex}
		%
\item 
Out of 100 students, two sections of 40 and 60 are formed. If you and your friend are among the 100 students, what is the probability that
\begin{enumerate}
\item you both enter the same section?
\item you both enter the different sections?
\end{enumerate}
\solution
		%\input{ncert/11/16/4/5/defs.tex}
	\item 
The number lock of a suitcase has 4 wheels each labelled with ten digits i.e. from 0 to 9.The lock opens with a sequence of four digits with no repeats.What is the probability of a person getting the right sequence to open the suitcase.
\\
\solution
		%\input{ncert/11/16/4/10/defs.tex}
		%
\item 
Two cards are drawn at random and without replacement from a pack of 52 playing cards. Find the probability that both the cards are black.
\\
\solution
		%\input{ncert/12/13/2/2/defs.tex}
		\item A box of oranges is inspected by examining three randomly selected oranges drawn without replacement. If all the three oranges are good, the box is approved for sale, otherwise, it is rejected. Find the probability that a box containing 15 oranges out of which 12 are good and 3 are bad ones will be approved for sale.
		\label{ncert/12/13/2/3/defs.tex}
		\item Two balls are drawn at random with replacement from a box containing 10 black and 8 red balls. Find the probability that
		\label{ncert/12/13/2/12}
\begin{enumerate}
\item both balls are red.
\item first ball is black and second is red.
\item one of them is black and other is red.
\end{enumerate}

\item In a hostel, 60\% of the students read Hindi newspaper, 40\% read English newspaper and 20\% read both Hindi and English newspapers. A student is selected at random.
		\label{ncert/12/13/2/15}
\begin{enumerate}
\item Find the probability that she reads neither Hindi nor English newspapers.
\item If she reads Hindi newspaper, find the probability that she reads English newspaper.
\item If she reads English newspaper, find the probability that she reads Hindi newspaper.\\
\end{enumerate}
\item The probability of obtaining an even prime number on each die, when a pair of dice is rolled is 
\begin{enumerate}
    \item $0$ 
    
    \item $\frac{1}{3}$ 
    
    \item $\frac{1}{12}$ 
    
    \item $\frac{1}{36}$ 
\end{enumerate}
\solution
		%\input{ncert/12/13/2/17/defs.tex}
	\item A bag contains 4 red and 4 black balls, another bag contains 2 red and 6 black balls. One of the two bags is selected at random and a ball is drawn from the bag which is found to be red. Find the probability that the ball is drawn from the first bag.
\\
\solution
		%\input{ncert/12/13/3/2/main.tex}
  \item
  Cards with numbers 2 to 101 are placed in a box. A card is selected at random.Find the probability that the card has
\begin{enumerate}[label=(\roman*)]
	\item an even number 
	\item a square number
\end{enumerate}
\solution
%\input{exemplar/10/13/3/32/main.tex}
\item
The king, queen and jack of clubs are removed from a deck of 52 playing cards and then well shuffled. Now one card is drawn at random from the remaining cards.  Determine the probability that the card is
\begin{enumerate}[label=(\roman*)]
\item a club
\item 10 of hearts
\end{enumerate}
\solution
%\input{exemplar/10/13/3/29/main.tex}
\item A team of medical students doing their internship have to assist during surgeries
at a city hospital. The probabilities of surgeries rated as very complex, complex,
routine, simple or very simple are respectively, 0.15, 0.20, 0.31, 0.26, .08. Find
the probabilities that a particular surgery will be rated
\begin{enumerate}
	\item complex or very complex;
	\item neither very complex nor very simple;
	\item routine or complex
	\item routine or simple
\end{enumerate}
\solution
%\input{exemplar/11/16/3/8(1)/main.tex}
\item A card is selected from a pack of 52 cards.
\begin{enumerate}[label=(\alph*)]
    \item How many points are there in the sample space?
    \item Calculate the probability that the card is an ace of spades.
    \item Calculate the probability that the card is (i) an ace and (ii) black card.
\end{enumerate}
\solution
%\input{exemplar/11/16/3/4/main2.tex}
\item The probability that a non leap year selected at random will contain 53 sundays.
\\
\solution
%\input{exemplar/10/13/1/19/main.tex}
\item One of the four persons John, Rita, Aslam or Gurpreet will be promoted next
month. Consequently the sample space consists of four elementary outcomes
S = {John promoted, Rita promoted, Aslam promoted, Gurpreet promoted}
You are told that the chances of John’s promotion is same as that of Gurpreet,
Rita’s chances of promotion are twice as likely as Johns. Aslam’s chances are
four times that of John.
\begin{enumerate}
	\item Determine
	\begin{enumerate}
		\item P (John promoted)
		\item P (Rita promoted)
		\item P (Aslam promoted)
		\item P (Gurpreet promoted)
	\end{enumerate}
	\item If A = {John promoted or Gurpreet promoted}, find P (A).
\end{enumerate}
\solution
%\input{exemplar/11/16/3/10/main.tex}
\item A card is drawn from a deck of 52 cards. Find the probability of getting a king or a heart or a red card.\\
\solution
%\input{exemplar/11/16/3/15/main.tex}
\item The probability that a student will pass his examination is 0.73, the probability of
the student getting a compartment is 0.13, and the probability that the student will
either pass or get compartment is 0.96. State True or False.\\
\solution
%\input{exemplar/11/16/3/31/main.tex}
\item A card is selected from a pack of 52 cards\\
\begin{enumerate}[label=(\alph*)]
\item How many points are there in the sample space?
\item Calculate the probability that the cards is an ace of spades.
\item Calculate the probability that the card is (i) an ace (ii)black card.\\
\end{enumerate}
%\input{ncert/11/16/3/4_1/Prob_4.tex}
\item In a non-leap year, the probability of having 53 tuesdays or 53 wednesdays is\\
\solution
%\input{exemplar/11/16/3/18/main.tex}
\item There are 1000 sealed envelopes in a box, 10 of them contain a cash prize of
Rs 100 each, 100 of them contain a cash prize of Rs 50 each and 200 of them
contain a cash prize of Rs 10 each and rest do not contain any cash prize. If they
are well shuffled and an envelope is picked up out, what is the probability that it
contains no cash prize?\\
\solution
%\input{exemplar/10/13/3/34/main.tex}
\item 
A die is thrown and a card is selected at random from a deck of 52 playing cards. The probability of getting an even number on the die and a spade card.\\
\solution
%\input{exemplar/12/13/3/78/main.tex}
\item
If 4-digit numbers greater than 5,000 are randomly formed from the digits 0, 1, 3, 5, and 7, what is the probability of forming a number divisible by 5 when:
\begin{enumerate}
    \item The digits are repeated?
    \item The repetition of digits is not allowed?
\end{enumerate}
\solution
%\input{ncert/11/16/4/9/main.tex}
\item Consider the probability space $\brak{\Omega, \mathcal{G}, P}$ where $\Omega = [0,2]$ and $\mathcal{G} = \cbrak{\phi, \Omega, [0,1], (1,2]}$. Let $X$ and $Y$ be two functions on $\Omega$ defined as
\begin{align*}
    X(\omega) = 
    \begin{cases}
        1 & \text{if }\omega \in [0, 1]\\
        2 & \text{if }\omega \in (1, 2]
    \end{cases}
\end{align*}
and
\begin{align*}
    Y(\omega) = 
    \begin{cases}
        2 & \text{if }\omega \in [0, 1.5]\\
        3 & \text{if }\omega \in (1.5, 2].
    \end{cases}
\end{align*}
Then which one of the following statements is true?
\begin{enumerate}
    \item [(A)] $X$ is a random variable with respect to $\mathcal{G}$, but $Y$ is not a random variable with respect to $\mathcal{G}$.
    \item [(B)] $Y$ is a random variable with respect to $\mathcal{G}$, but $X$ is not a random variable with respect to $\mathcal{G}$.
    \item [(C)] Neither $X$ nor $Y$ is a random variable with respect to $\mathcal{G}$.
    \item [(D)] Both $X$ and $Y$ are random variables with respect to $\mathcal{G}$.
\end{enumerate} \hfill (GATE ST 2023)\\
\solution
%\input{gate/ST/2023/14/main.tex}
	\item  A die is loaded in such a way that each odd number is twice as likely to occur as
each even number. Find $P(G)$, where $G$ is the event that a number greater than
3 occurs on a single roll of the die.
\\
\solution
		%\input{exemplar/11/16/3/5/main.tex}
	\item All the jacks, queens and kings are removed from a deck of 52 playing cards. The remaining cards are well shuffled and then one card is drawn at random. Giving ace a value 1 similar value for other cards, find the probability that the card has a value 
		\begin{enumerate}
			\item 7
			\item greater than 7
			\item less than 7
		\end{enumerate}
		%\input{exemplar/10/13/3/30/main.tex}
  \item A Lot consists of 48 mobile phones of which 42 are good, 3 have only minor defects and 3 have major defects.Varnika will buy a phone if it is good but the trader will only buy a mobile if it has no major defects. One phone is selected at random from the lot. What is the probability that it is
\begin{enumerate}
	\item acceptable to Varnika?
            \item acceptable to the trader?
\end{enumerate}
\solution
	%\input{exemplar/10/13/3/40/main.tex}
 \item A student says that if you throw a die, it will show up 1 or not 1. Therefore, the probability of getting 1 and the probability of getting 'not 1' each is equal to $\frac{1}{2}$. Is this correct? Give reasons.\\
 \solution
        %\input{exemplar/10/13/2/9/main.tex}
   \item Four candidates A, B, C, D have ap-
plied for the assignment to coach a school cricket
team. If A is twice as likely to be selected as B, and
B and C are given about the same chance of being
selected, while C is twice as likely to be selected
as D, what are the probabilities that
\begin{enumerate}
\item C will be selected?
\item A will not be selected?
\end{enumerate}
	%\input{exemplar/11/16/3/9/main.tex}
 \item A bag contain 24 balls of which $x$ balls are red, $2x$ are white and $3x$ are blue. A ball is selected at random, What is the probability that it is
\begin{enumerate}[label=\alph*)]
\item not red ?
\item white ?
\end{enumerate}
%\input{exemplar/10/13/3/41/main.tex}
If the letters of the word ASSASSINATION are arranged at random. Find the Probability that
\begin{enumerate}[label=(\alph*)]
\item Four $S's$ come consecutively in the word
\item Two  $I's$ and two $N's$ come together
\item All $A's$ are not coming together
\item No two $A's$ are coming together
\end{enumerate}
%\input{exemplar/11/16/3/14/main.tex}
	\item One urn contains two black balls (labelled B1 and B2) and one white ball. A
	second urn contains one black ball and two white balls (labelled W1 and W2).
	Suppose the following experiment is performed. One of the two urns is chosen
	at random. Next a ball is randomly chosen from the urn. Then a second ball is
	chosen at random from the same urn without replacing the first ball.
	
	\begin{enumerate}
	\item What is the probability that two black balls are chosen?
	
	\item What is the probability that two balls of opposite colour are chosen?
	\end{enumerate}
	\solution
	%\input{exemplar/11/16/3/12/main1.tex}
\end{enumerate}

\item In a certain lottery 10,000 tickets are sold and ten equal prizes are awarded. What is the probability of not getting a prize if you buy (a) one ticket (b) two tickets (c) 10 tickets ?	
\\
\solution
		%\begin{enumerate}[label=\thesection.\arabic*,ref=\thesection.\theenumi]
	\item One card is drawn from a well-shuffled deck of 52 cards. Find the probability of getting
\begin{enumerate}
\item A king of red colour 
\item A face card 
\item A red face card
\item The jack of hearts
\item A spade
\item The queen of diamonds

\end{enumerate}
\solution
		%\input{ncert/10/15/1/14/main.tex}
	\item Five cards—the ten, jack, queen, king and ace of diamonds, are well-shuffled with their face downwards. One card is then picked up at random.
\begin{enumerate}
\item
What is the probability that the card is the queen? 
\item
If the queen is drawn and put aside, what is the probability that the second card picked up is (a) an ace? (b) a queen?\\
\end{enumerate}
\solution
		%\input{ncert/10/15/1/15/defs.tex}
	\item A bag contains $5$ red balls and some blue balls. If the probability of drawing a blue ball is double that if a red ball, determine the number of blue balls in the bag. 
		\\
\solution
		%\input{ncert/10/15/2/3/defs.tex}
	\item A card is selected from a pack of 52 cards.
 \begin{enumerate}[label=(\alph*)] 
                 \item How many points are there in the sample space?
                 \item Calculate the probability that the card is an ace of spades.
                 \item Calculate the probability that the card is (i) an ace and (ii) black card.
 \end{enumerate}
\solution
		%\input{ncert/11/16/3/4/main.tex}
\item Four cards are drawn from a well-shuffled deck of 52 cards. What is the probability of obtaining 3 diamonds and one spade.
\\
\solution
		%\input{ncert/11/16/4/2/defs.tex}
\item In a certain lottery 10,000 tickets are sold and ten equal prizes are awarded. What is the probability of not getting a prize if you buy (a) one ticket (b) two tickets (c) 10 tickets ?	
\\
\solution
		%\input{ncert/11/16/4/4/defs.tex}
		%
\item 
Out of 100 students, two sections of 40 and 60 are formed. If you and your friend are among the 100 students, what is the probability that
\begin{enumerate}
\item you both enter the same section?
\item you both enter the different sections?
\end{enumerate}
\solution
		%\input{ncert/11/16/4/5/defs.tex}
	\item 
The number lock of a suitcase has 4 wheels each labelled with ten digits i.e. from 0 to 9.The lock opens with a sequence of four digits with no repeats.What is the probability of a person getting the right sequence to open the suitcase.
\\
\solution
		%\input{ncert/11/16/4/10/defs.tex}
		%
\item 
Two cards are drawn at random and without replacement from a pack of 52 playing cards. Find the probability that both the cards are black.
\\
\solution
		%\input{ncert/12/13/2/2/defs.tex}
		\item A box of oranges is inspected by examining three randomly selected oranges drawn without replacement. If all the three oranges are good, the box is approved for sale, otherwise, it is rejected. Find the probability that a box containing 15 oranges out of which 12 are good and 3 are bad ones will be approved for sale.
		\label{ncert/12/13/2/3/defs.tex}
		\item Two balls are drawn at random with replacement from a box containing 10 black and 8 red balls. Find the probability that
		\label{ncert/12/13/2/12}
\begin{enumerate}
\item both balls are red.
\item first ball is black and second is red.
\item one of them is black and other is red.
\end{enumerate}

\item In a hostel, 60\% of the students read Hindi newspaper, 40\% read English newspaper and 20\% read both Hindi and English newspapers. A student is selected at random.
		\label{ncert/12/13/2/15}
\begin{enumerate}
\item Find the probability that she reads neither Hindi nor English newspapers.
\item If she reads Hindi newspaper, find the probability that she reads English newspaper.
\item If she reads English newspaper, find the probability that she reads Hindi newspaper.\\
\end{enumerate}
\item The probability of obtaining an even prime number on each die, when a pair of dice is rolled is 
\begin{enumerate}
    \item $0$ 
    
    \item $\frac{1}{3}$ 
    
    \item $\frac{1}{12}$ 
    
    \item $\frac{1}{36}$ 
\end{enumerate}
\solution
		%\input{ncert/12/13/2/17/defs.tex}
	\item A bag contains 4 red and 4 black balls, another bag contains 2 red and 6 black balls. One of the two bags is selected at random and a ball is drawn from the bag which is found to be red. Find the probability that the ball is drawn from the first bag.
\\
\solution
		%\input{ncert/12/13/3/2/main.tex}
  \item
  Cards with numbers 2 to 101 are placed in a box. A card is selected at random.Find the probability that the card has
\begin{enumerate}[label=(\roman*)]
	\item an even number 
	\item a square number
\end{enumerate}
\solution
%\input{exemplar/10/13/3/32/main.tex}
\item
The king, queen and jack of clubs are removed from a deck of 52 playing cards and then well shuffled. Now one card is drawn at random from the remaining cards.  Determine the probability that the card is
\begin{enumerate}[label=(\roman*)]
\item a club
\item 10 of hearts
\end{enumerate}
\solution
%\input{exemplar/10/13/3/29/main.tex}
\item A team of medical students doing their internship have to assist during surgeries
at a city hospital. The probabilities of surgeries rated as very complex, complex,
routine, simple or very simple are respectively, 0.15, 0.20, 0.31, 0.26, .08. Find
the probabilities that a particular surgery will be rated
\begin{enumerate}
	\item complex or very complex;
	\item neither very complex nor very simple;
	\item routine or complex
	\item routine or simple
\end{enumerate}
\solution
%\input{exemplar/11/16/3/8(1)/main.tex}
\item A card is selected from a pack of 52 cards.
\begin{enumerate}[label=(\alph*)]
    \item How many points are there in the sample space?
    \item Calculate the probability that the card is an ace of spades.
    \item Calculate the probability that the card is (i) an ace and (ii) black card.
\end{enumerate}
\solution
%\input{exemplar/11/16/3/4/main2.tex}
\item The probability that a non leap year selected at random will contain 53 sundays.
\\
\solution
%\input{exemplar/10/13/1/19/main.tex}
\item One of the four persons John, Rita, Aslam or Gurpreet will be promoted next
month. Consequently the sample space consists of four elementary outcomes
S = {John promoted, Rita promoted, Aslam promoted, Gurpreet promoted}
You are told that the chances of John’s promotion is same as that of Gurpreet,
Rita’s chances of promotion are twice as likely as Johns. Aslam’s chances are
four times that of John.
\begin{enumerate}
	\item Determine
	\begin{enumerate}
		\item P (John promoted)
		\item P (Rita promoted)
		\item P (Aslam promoted)
		\item P (Gurpreet promoted)
	\end{enumerate}
	\item If A = {John promoted or Gurpreet promoted}, find P (A).
\end{enumerate}
\solution
%\input{exemplar/11/16/3/10/main.tex}
\item A card is drawn from a deck of 52 cards. Find the probability of getting a king or a heart or a red card.\\
\solution
%\input{exemplar/11/16/3/15/main.tex}
\item The probability that a student will pass his examination is 0.73, the probability of
the student getting a compartment is 0.13, and the probability that the student will
either pass or get compartment is 0.96. State True or False.\\
\solution
%\input{exemplar/11/16/3/31/main.tex}
\item A card is selected from a pack of 52 cards\\
\begin{enumerate}[label=(\alph*)]
\item How many points are there in the sample space?
\item Calculate the probability that the cards is an ace of spades.
\item Calculate the probability that the card is (i) an ace (ii)black card.\\
\end{enumerate}
%\input{ncert/11/16/3/4_1/Prob_4.tex}
\item In a non-leap year, the probability of having 53 tuesdays or 53 wednesdays is\\
\solution
%\input{exemplar/11/16/3/18/main.tex}
\item There are 1000 sealed envelopes in a box, 10 of them contain a cash prize of
Rs 100 each, 100 of them contain a cash prize of Rs 50 each and 200 of them
contain a cash prize of Rs 10 each and rest do not contain any cash prize. If they
are well shuffled and an envelope is picked up out, what is the probability that it
contains no cash prize?\\
\solution
%\input{exemplar/10/13/3/34/main.tex}
\item 
A die is thrown and a card is selected at random from a deck of 52 playing cards. The probability of getting an even number on the die and a spade card.\\
\solution
%\input{exemplar/12/13/3/78/main.tex}
\item
If 4-digit numbers greater than 5,000 are randomly formed from the digits 0, 1, 3, 5, and 7, what is the probability of forming a number divisible by 5 when:
\begin{enumerate}
    \item The digits are repeated?
    \item The repetition of digits is not allowed?
\end{enumerate}
\solution
%\input{ncert/11/16/4/9/main.tex}
\item Consider the probability space $\brak{\Omega, \mathcal{G}, P}$ where $\Omega = [0,2]$ and $\mathcal{G} = \cbrak{\phi, \Omega, [0,1], (1,2]}$. Let $X$ and $Y$ be two functions on $\Omega$ defined as
\begin{align*}
    X(\omega) = 
    \begin{cases}
        1 & \text{if }\omega \in [0, 1]\\
        2 & \text{if }\omega \in (1, 2]
    \end{cases}
\end{align*}
and
\begin{align*}
    Y(\omega) = 
    \begin{cases}
        2 & \text{if }\omega \in [0, 1.5]\\
        3 & \text{if }\omega \in (1.5, 2].
    \end{cases}
\end{align*}
Then which one of the following statements is true?
\begin{enumerate}
    \item [(A)] $X$ is a random variable with respect to $\mathcal{G}$, but $Y$ is not a random variable with respect to $\mathcal{G}$.
    \item [(B)] $Y$ is a random variable with respect to $\mathcal{G}$, but $X$ is not a random variable with respect to $\mathcal{G}$.
    \item [(C)] Neither $X$ nor $Y$ is a random variable with respect to $\mathcal{G}$.
    \item [(D)] Both $X$ and $Y$ are random variables with respect to $\mathcal{G}$.
\end{enumerate} \hfill (GATE ST 2023)\\
\solution
%\input{gate/ST/2023/14/main.tex}
	\item  A die is loaded in such a way that each odd number is twice as likely to occur as
each even number. Find $P(G)$, where $G$ is the event that a number greater than
3 occurs on a single roll of the die.
\\
\solution
		%\input{exemplar/11/16/3/5/main.tex}
	\item All the jacks, queens and kings are removed from a deck of 52 playing cards. The remaining cards are well shuffled and then one card is drawn at random. Giving ace a value 1 similar value for other cards, find the probability that the card has a value 
		\begin{enumerate}
			\item 7
			\item greater than 7
			\item less than 7
		\end{enumerate}
		%\input{exemplar/10/13/3/30/main.tex}
  \item A Lot consists of 48 mobile phones of which 42 are good, 3 have only minor defects and 3 have major defects.Varnika will buy a phone if it is good but the trader will only buy a mobile if it has no major defects. One phone is selected at random from the lot. What is the probability that it is
\begin{enumerate}
	\item acceptable to Varnika?
            \item acceptable to the trader?
\end{enumerate}
\solution
	%\input{exemplar/10/13/3/40/main.tex}
 \item A student says that if you throw a die, it will show up 1 or not 1. Therefore, the probability of getting 1 and the probability of getting 'not 1' each is equal to $\frac{1}{2}$. Is this correct? Give reasons.\\
 \solution
        %\input{exemplar/10/13/2/9/main.tex}
   \item Four candidates A, B, C, D have ap-
plied for the assignment to coach a school cricket
team. If A is twice as likely to be selected as B, and
B and C are given about the same chance of being
selected, while C is twice as likely to be selected
as D, what are the probabilities that
\begin{enumerate}
\item C will be selected?
\item A will not be selected?
\end{enumerate}
	%\input{exemplar/11/16/3/9/main.tex}
 \item A bag contain 24 balls of which $x$ balls are red, $2x$ are white and $3x$ are blue. A ball is selected at random, What is the probability that it is
\begin{enumerate}[label=\alph*)]
\item not red ?
\item white ?
\end{enumerate}
%\input{exemplar/10/13/3/41/main.tex}
If the letters of the word ASSASSINATION are arranged at random. Find the Probability that
\begin{enumerate}[label=(\alph*)]
\item Four $S's$ come consecutively in the word
\item Two  $I's$ and two $N's$ come together
\item All $A's$ are not coming together
\item No two $A's$ are coming together
\end{enumerate}
%\input{exemplar/11/16/3/14/main.tex}
	\item One urn contains two black balls (labelled B1 and B2) and one white ball. A
	second urn contains one black ball and two white balls (labelled W1 and W2).
	Suppose the following experiment is performed. One of the two urns is chosen
	at random. Next a ball is randomly chosen from the urn. Then a second ball is
	chosen at random from the same urn without replacing the first ball.
	
	\begin{enumerate}
	\item What is the probability that two black balls are chosen?
	
	\item What is the probability that two balls of opposite colour are chosen?
	\end{enumerate}
	\solution
	%\input{exemplar/11/16/3/12/main1.tex}
\end{enumerate}

		%
\item 
Out of 100 students, two sections of 40 and 60 are formed. If you and your friend are among the 100 students, what is the probability that
\begin{enumerate}
\item you both enter the same section?
\item you both enter the different sections?
\end{enumerate}
\solution
		%\begin{enumerate}[label=\thesection.\arabic*,ref=\thesection.\theenumi]
	\item One card is drawn from a well-shuffled deck of 52 cards. Find the probability of getting
\begin{enumerate}
\item A king of red colour 
\item A face card 
\item A red face card
\item The jack of hearts
\item A spade
\item The queen of diamonds

\end{enumerate}
\solution
		%\input{ncert/10/15/1/14/main.tex}
	\item Five cards—the ten, jack, queen, king and ace of diamonds, are well-shuffled with their face downwards. One card is then picked up at random.
\begin{enumerate}
\item
What is the probability that the card is the queen? 
\item
If the queen is drawn and put aside, what is the probability that the second card picked up is (a) an ace? (b) a queen?\\
\end{enumerate}
\solution
		%\input{ncert/10/15/1/15/defs.tex}
	\item A bag contains $5$ red balls and some blue balls. If the probability of drawing a blue ball is double that if a red ball, determine the number of blue balls in the bag. 
		\\
\solution
		%\input{ncert/10/15/2/3/defs.tex}
	\item A card is selected from a pack of 52 cards.
 \begin{enumerate}[label=(\alph*)] 
                 \item How many points are there in the sample space?
                 \item Calculate the probability that the card is an ace of spades.
                 \item Calculate the probability that the card is (i) an ace and (ii) black card.
 \end{enumerate}
\solution
		%\input{ncert/11/16/3/4/main.tex}
\item Four cards are drawn from a well-shuffled deck of 52 cards. What is the probability of obtaining 3 diamonds and one spade.
\\
\solution
		%\input{ncert/11/16/4/2/defs.tex}
\item In a certain lottery 10,000 tickets are sold and ten equal prizes are awarded. What is the probability of not getting a prize if you buy (a) one ticket (b) two tickets (c) 10 tickets ?	
\\
\solution
		%\input{ncert/11/16/4/4/defs.tex}
		%
\item 
Out of 100 students, two sections of 40 and 60 are formed. If you and your friend are among the 100 students, what is the probability that
\begin{enumerate}
\item you both enter the same section?
\item you both enter the different sections?
\end{enumerate}
\solution
		%\input{ncert/11/16/4/5/defs.tex}
	\item 
The number lock of a suitcase has 4 wheels each labelled with ten digits i.e. from 0 to 9.The lock opens with a sequence of four digits with no repeats.What is the probability of a person getting the right sequence to open the suitcase.
\\
\solution
		%\input{ncert/11/16/4/10/defs.tex}
		%
\item 
Two cards are drawn at random and without replacement from a pack of 52 playing cards. Find the probability that both the cards are black.
\\
\solution
		%\input{ncert/12/13/2/2/defs.tex}
		\item A box of oranges is inspected by examining three randomly selected oranges drawn without replacement. If all the three oranges are good, the box is approved for sale, otherwise, it is rejected. Find the probability that a box containing 15 oranges out of which 12 are good and 3 are bad ones will be approved for sale.
		\label{ncert/12/13/2/3/defs.tex}
		\item Two balls are drawn at random with replacement from a box containing 10 black and 8 red balls. Find the probability that
		\label{ncert/12/13/2/12}
\begin{enumerate}
\item both balls are red.
\item first ball is black and second is red.
\item one of them is black and other is red.
\end{enumerate}

\item In a hostel, 60\% of the students read Hindi newspaper, 40\% read English newspaper and 20\% read both Hindi and English newspapers. A student is selected at random.
		\label{ncert/12/13/2/15}
\begin{enumerate}
\item Find the probability that she reads neither Hindi nor English newspapers.
\item If she reads Hindi newspaper, find the probability that she reads English newspaper.
\item If she reads English newspaper, find the probability that she reads Hindi newspaper.\\
\end{enumerate}
\item The probability of obtaining an even prime number on each die, when a pair of dice is rolled is 
\begin{enumerate}
    \item $0$ 
    
    \item $\frac{1}{3}$ 
    
    \item $\frac{1}{12}$ 
    
    \item $\frac{1}{36}$ 
\end{enumerate}
\solution
		%\input{ncert/12/13/2/17/defs.tex}
	\item A bag contains 4 red and 4 black balls, another bag contains 2 red and 6 black balls. One of the two bags is selected at random and a ball is drawn from the bag which is found to be red. Find the probability that the ball is drawn from the first bag.
\\
\solution
		%\input{ncert/12/13/3/2/main.tex}
  \item
  Cards with numbers 2 to 101 are placed in a box. A card is selected at random.Find the probability that the card has
\begin{enumerate}[label=(\roman*)]
	\item an even number 
	\item a square number
\end{enumerate}
\solution
%\input{exemplar/10/13/3/32/main.tex}
\item
The king, queen and jack of clubs are removed from a deck of 52 playing cards and then well shuffled. Now one card is drawn at random from the remaining cards.  Determine the probability that the card is
\begin{enumerate}[label=(\roman*)]
\item a club
\item 10 of hearts
\end{enumerate}
\solution
%\input{exemplar/10/13/3/29/main.tex}
\item A team of medical students doing their internship have to assist during surgeries
at a city hospital. The probabilities of surgeries rated as very complex, complex,
routine, simple or very simple are respectively, 0.15, 0.20, 0.31, 0.26, .08. Find
the probabilities that a particular surgery will be rated
\begin{enumerate}
	\item complex or very complex;
	\item neither very complex nor very simple;
	\item routine or complex
	\item routine or simple
\end{enumerate}
\solution
%\input{exemplar/11/16/3/8(1)/main.tex}
\item A card is selected from a pack of 52 cards.
\begin{enumerate}[label=(\alph*)]
    \item How many points are there in the sample space?
    \item Calculate the probability that the card is an ace of spades.
    \item Calculate the probability that the card is (i) an ace and (ii) black card.
\end{enumerate}
\solution
%\input{exemplar/11/16/3/4/main2.tex}
\item The probability that a non leap year selected at random will contain 53 sundays.
\\
\solution
%\input{exemplar/10/13/1/19/main.tex}
\item One of the four persons John, Rita, Aslam or Gurpreet will be promoted next
month. Consequently the sample space consists of four elementary outcomes
S = {John promoted, Rita promoted, Aslam promoted, Gurpreet promoted}
You are told that the chances of John’s promotion is same as that of Gurpreet,
Rita’s chances of promotion are twice as likely as Johns. Aslam’s chances are
four times that of John.
\begin{enumerate}
	\item Determine
	\begin{enumerate}
		\item P (John promoted)
		\item P (Rita promoted)
		\item P (Aslam promoted)
		\item P (Gurpreet promoted)
	\end{enumerate}
	\item If A = {John promoted or Gurpreet promoted}, find P (A).
\end{enumerate}
\solution
%\input{exemplar/11/16/3/10/main.tex}
\item A card is drawn from a deck of 52 cards. Find the probability of getting a king or a heart or a red card.\\
\solution
%\input{exemplar/11/16/3/15/main.tex}
\item The probability that a student will pass his examination is 0.73, the probability of
the student getting a compartment is 0.13, and the probability that the student will
either pass or get compartment is 0.96. State True or False.\\
\solution
%\input{exemplar/11/16/3/31/main.tex}
\item A card is selected from a pack of 52 cards\\
\begin{enumerate}[label=(\alph*)]
\item How many points are there in the sample space?
\item Calculate the probability that the cards is an ace of spades.
\item Calculate the probability that the card is (i) an ace (ii)black card.\\
\end{enumerate}
%\input{ncert/11/16/3/4_1/Prob_4.tex}
\item In a non-leap year, the probability of having 53 tuesdays or 53 wednesdays is\\
\solution
%\input{exemplar/11/16/3/18/main.tex}
\item There are 1000 sealed envelopes in a box, 10 of them contain a cash prize of
Rs 100 each, 100 of them contain a cash prize of Rs 50 each and 200 of them
contain a cash prize of Rs 10 each and rest do not contain any cash prize. If they
are well shuffled and an envelope is picked up out, what is the probability that it
contains no cash prize?\\
\solution
%\input{exemplar/10/13/3/34/main.tex}
\item 
A die is thrown and a card is selected at random from a deck of 52 playing cards. The probability of getting an even number on the die and a spade card.\\
\solution
%\input{exemplar/12/13/3/78/main.tex}
\item
If 4-digit numbers greater than 5,000 are randomly formed from the digits 0, 1, 3, 5, and 7, what is the probability of forming a number divisible by 5 when:
\begin{enumerate}
    \item The digits are repeated?
    \item The repetition of digits is not allowed?
\end{enumerate}
\solution
%\input{ncert/11/16/4/9/main.tex}
\item Consider the probability space $\brak{\Omega, \mathcal{G}, P}$ where $\Omega = [0,2]$ and $\mathcal{G} = \cbrak{\phi, \Omega, [0,1], (1,2]}$. Let $X$ and $Y$ be two functions on $\Omega$ defined as
\begin{align*}
    X(\omega) = 
    \begin{cases}
        1 & \text{if }\omega \in [0, 1]\\
        2 & \text{if }\omega \in (1, 2]
    \end{cases}
\end{align*}
and
\begin{align*}
    Y(\omega) = 
    \begin{cases}
        2 & \text{if }\omega \in [0, 1.5]\\
        3 & \text{if }\omega \in (1.5, 2].
    \end{cases}
\end{align*}
Then which one of the following statements is true?
\begin{enumerate}
    \item [(A)] $X$ is a random variable with respect to $\mathcal{G}$, but $Y$ is not a random variable with respect to $\mathcal{G}$.
    \item [(B)] $Y$ is a random variable with respect to $\mathcal{G}$, but $X$ is not a random variable with respect to $\mathcal{G}$.
    \item [(C)] Neither $X$ nor $Y$ is a random variable with respect to $\mathcal{G}$.
    \item [(D)] Both $X$ and $Y$ are random variables with respect to $\mathcal{G}$.
\end{enumerate} \hfill (GATE ST 2023)\\
\solution
%\input{gate/ST/2023/14/main.tex}
	\item  A die is loaded in such a way that each odd number is twice as likely to occur as
each even number. Find $P(G)$, where $G$ is the event that a number greater than
3 occurs on a single roll of the die.
\\
\solution
		%\input{exemplar/11/16/3/5/main.tex}
	\item All the jacks, queens and kings are removed from a deck of 52 playing cards. The remaining cards are well shuffled and then one card is drawn at random. Giving ace a value 1 similar value for other cards, find the probability that the card has a value 
		\begin{enumerate}
			\item 7
			\item greater than 7
			\item less than 7
		\end{enumerate}
		%\input{exemplar/10/13/3/30/main.tex}
  \item A Lot consists of 48 mobile phones of which 42 are good, 3 have only minor defects and 3 have major defects.Varnika will buy a phone if it is good but the trader will only buy a mobile if it has no major defects. One phone is selected at random from the lot. What is the probability that it is
\begin{enumerate}
	\item acceptable to Varnika?
            \item acceptable to the trader?
\end{enumerate}
\solution
	%\input{exemplar/10/13/3/40/main.tex}
 \item A student says that if you throw a die, it will show up 1 or not 1. Therefore, the probability of getting 1 and the probability of getting 'not 1' each is equal to $\frac{1}{2}$. Is this correct? Give reasons.\\
 \solution
        %\input{exemplar/10/13/2/9/main.tex}
   \item Four candidates A, B, C, D have ap-
plied for the assignment to coach a school cricket
team. If A is twice as likely to be selected as B, and
B and C are given about the same chance of being
selected, while C is twice as likely to be selected
as D, what are the probabilities that
\begin{enumerate}
\item C will be selected?
\item A will not be selected?
\end{enumerate}
	%\input{exemplar/11/16/3/9/main.tex}
 \item A bag contain 24 balls of which $x$ balls are red, $2x$ are white and $3x$ are blue. A ball is selected at random, What is the probability that it is
\begin{enumerate}[label=\alph*)]
\item not red ?
\item white ?
\end{enumerate}
%\input{exemplar/10/13/3/41/main.tex}
If the letters of the word ASSASSINATION are arranged at random. Find the Probability that
\begin{enumerate}[label=(\alph*)]
\item Four $S's$ come consecutively in the word
\item Two  $I's$ and two $N's$ come together
\item All $A's$ are not coming together
\item No two $A's$ are coming together
\end{enumerate}
%\input{exemplar/11/16/3/14/main.tex}
	\item One urn contains two black balls (labelled B1 and B2) and one white ball. A
	second urn contains one black ball and two white balls (labelled W1 and W2).
	Suppose the following experiment is performed. One of the two urns is chosen
	at random. Next a ball is randomly chosen from the urn. Then a second ball is
	chosen at random from the same urn without replacing the first ball.
	
	\begin{enumerate}
	\item What is the probability that two black balls are chosen?
	
	\item What is the probability that two balls of opposite colour are chosen?
	\end{enumerate}
	\solution
	%\input{exemplar/11/16/3/12/main1.tex}
\end{enumerate}

	\item 
The number lock of a suitcase has 4 wheels each labelled with ten digits i.e. from 0 to 9.The lock opens with a sequence of four digits with no repeats.What is the probability of a person getting the right sequence to open the suitcase.
\\
\solution
		%\begin{enumerate}[label=\thesection.\arabic*,ref=\thesection.\theenumi]
	\item One card is drawn from a well-shuffled deck of 52 cards. Find the probability of getting
\begin{enumerate}
\item A king of red colour 
\item A face card 
\item A red face card
\item The jack of hearts
\item A spade
\item The queen of diamonds

\end{enumerate}
\solution
		%\input{ncert/10/15/1/14/main.tex}
	\item Five cards—the ten, jack, queen, king and ace of diamonds, are well-shuffled with their face downwards. One card is then picked up at random.
\begin{enumerate}
\item
What is the probability that the card is the queen? 
\item
If the queen is drawn and put aside, what is the probability that the second card picked up is (a) an ace? (b) a queen?\\
\end{enumerate}
\solution
		%\input{ncert/10/15/1/15/defs.tex}
	\item A bag contains $5$ red balls and some blue balls. If the probability of drawing a blue ball is double that if a red ball, determine the number of blue balls in the bag. 
		\\
\solution
		%\input{ncert/10/15/2/3/defs.tex}
	\item A card is selected from a pack of 52 cards.
 \begin{enumerate}[label=(\alph*)] 
                 \item How many points are there in the sample space?
                 \item Calculate the probability that the card is an ace of spades.
                 \item Calculate the probability that the card is (i) an ace and (ii) black card.
 \end{enumerate}
\solution
		%\input{ncert/11/16/3/4/main.tex}
\item Four cards are drawn from a well-shuffled deck of 52 cards. What is the probability of obtaining 3 diamonds and one spade.
\\
\solution
		%\input{ncert/11/16/4/2/defs.tex}
\item In a certain lottery 10,000 tickets are sold and ten equal prizes are awarded. What is the probability of not getting a prize if you buy (a) one ticket (b) two tickets (c) 10 tickets ?	
\\
\solution
		%\input{ncert/11/16/4/4/defs.tex}
		%
\item 
Out of 100 students, two sections of 40 and 60 are formed. If you and your friend are among the 100 students, what is the probability that
\begin{enumerate}
\item you both enter the same section?
\item you both enter the different sections?
\end{enumerate}
\solution
		%\input{ncert/11/16/4/5/defs.tex}
	\item 
The number lock of a suitcase has 4 wheels each labelled with ten digits i.e. from 0 to 9.The lock opens with a sequence of four digits with no repeats.What is the probability of a person getting the right sequence to open the suitcase.
\\
\solution
		%\input{ncert/11/16/4/10/defs.tex}
		%
\item 
Two cards are drawn at random and without replacement from a pack of 52 playing cards. Find the probability that both the cards are black.
\\
\solution
		%\input{ncert/12/13/2/2/defs.tex}
		\item A box of oranges is inspected by examining three randomly selected oranges drawn without replacement. If all the three oranges are good, the box is approved for sale, otherwise, it is rejected. Find the probability that a box containing 15 oranges out of which 12 are good and 3 are bad ones will be approved for sale.
		\label{ncert/12/13/2/3/defs.tex}
		\item Two balls are drawn at random with replacement from a box containing 10 black and 8 red balls. Find the probability that
		\label{ncert/12/13/2/12}
\begin{enumerate}
\item both balls are red.
\item first ball is black and second is red.
\item one of them is black and other is red.
\end{enumerate}

\item In a hostel, 60\% of the students read Hindi newspaper, 40\% read English newspaper and 20\% read both Hindi and English newspapers. A student is selected at random.
		\label{ncert/12/13/2/15}
\begin{enumerate}
\item Find the probability that she reads neither Hindi nor English newspapers.
\item If she reads Hindi newspaper, find the probability that she reads English newspaper.
\item If she reads English newspaper, find the probability that she reads Hindi newspaper.\\
\end{enumerate}
\item The probability of obtaining an even prime number on each die, when a pair of dice is rolled is 
\begin{enumerate}
    \item $0$ 
    
    \item $\frac{1}{3}$ 
    
    \item $\frac{1}{12}$ 
    
    \item $\frac{1}{36}$ 
\end{enumerate}
\solution
		%\input{ncert/12/13/2/17/defs.tex}
	\item A bag contains 4 red and 4 black balls, another bag contains 2 red and 6 black balls. One of the two bags is selected at random and a ball is drawn from the bag which is found to be red. Find the probability that the ball is drawn from the first bag.
\\
\solution
		%\input{ncert/12/13/3/2/main.tex}
  \item
  Cards with numbers 2 to 101 are placed in a box. A card is selected at random.Find the probability that the card has
\begin{enumerate}[label=(\roman*)]
	\item an even number 
	\item a square number
\end{enumerate}
\solution
%\input{exemplar/10/13/3/32/main.tex}
\item
The king, queen and jack of clubs are removed from a deck of 52 playing cards and then well shuffled. Now one card is drawn at random from the remaining cards.  Determine the probability that the card is
\begin{enumerate}[label=(\roman*)]
\item a club
\item 10 of hearts
\end{enumerate}
\solution
%\input{exemplar/10/13/3/29/main.tex}
\item A team of medical students doing their internship have to assist during surgeries
at a city hospital. The probabilities of surgeries rated as very complex, complex,
routine, simple or very simple are respectively, 0.15, 0.20, 0.31, 0.26, .08. Find
the probabilities that a particular surgery will be rated
\begin{enumerate}
	\item complex or very complex;
	\item neither very complex nor very simple;
	\item routine or complex
	\item routine or simple
\end{enumerate}
\solution
%\input{exemplar/11/16/3/8(1)/main.tex}
\item A card is selected from a pack of 52 cards.
\begin{enumerate}[label=(\alph*)]
    \item How many points are there in the sample space?
    \item Calculate the probability that the card is an ace of spades.
    \item Calculate the probability that the card is (i) an ace and (ii) black card.
\end{enumerate}
\solution
%\input{exemplar/11/16/3/4/main2.tex}
\item The probability that a non leap year selected at random will contain 53 sundays.
\\
\solution
%\input{exemplar/10/13/1/19/main.tex}
\item One of the four persons John, Rita, Aslam or Gurpreet will be promoted next
month. Consequently the sample space consists of four elementary outcomes
S = {John promoted, Rita promoted, Aslam promoted, Gurpreet promoted}
You are told that the chances of John’s promotion is same as that of Gurpreet,
Rita’s chances of promotion are twice as likely as Johns. Aslam’s chances are
four times that of John.
\begin{enumerate}
	\item Determine
	\begin{enumerate}
		\item P (John promoted)
		\item P (Rita promoted)
		\item P (Aslam promoted)
		\item P (Gurpreet promoted)
	\end{enumerate}
	\item If A = {John promoted or Gurpreet promoted}, find P (A).
\end{enumerate}
\solution
%\input{exemplar/11/16/3/10/main.tex}
\item A card is drawn from a deck of 52 cards. Find the probability of getting a king or a heart or a red card.\\
\solution
%\input{exemplar/11/16/3/15/main.tex}
\item The probability that a student will pass his examination is 0.73, the probability of
the student getting a compartment is 0.13, and the probability that the student will
either pass or get compartment is 0.96. State True or False.\\
\solution
%\input{exemplar/11/16/3/31/main.tex}
\item A card is selected from a pack of 52 cards\\
\begin{enumerate}[label=(\alph*)]
\item How many points are there in the sample space?
\item Calculate the probability that the cards is an ace of spades.
\item Calculate the probability that the card is (i) an ace (ii)black card.\\
\end{enumerate}
%\input{ncert/11/16/3/4_1/Prob_4.tex}
\item In a non-leap year, the probability of having 53 tuesdays or 53 wednesdays is\\
\solution
%\input{exemplar/11/16/3/18/main.tex}
\item There are 1000 sealed envelopes in a box, 10 of them contain a cash prize of
Rs 100 each, 100 of them contain a cash prize of Rs 50 each and 200 of them
contain a cash prize of Rs 10 each and rest do not contain any cash prize. If they
are well shuffled and an envelope is picked up out, what is the probability that it
contains no cash prize?\\
\solution
%\input{exemplar/10/13/3/34/main.tex}
\item 
A die is thrown and a card is selected at random from a deck of 52 playing cards. The probability of getting an even number on the die and a spade card.\\
\solution
%\input{exemplar/12/13/3/78/main.tex}
\item
If 4-digit numbers greater than 5,000 are randomly formed from the digits 0, 1, 3, 5, and 7, what is the probability of forming a number divisible by 5 when:
\begin{enumerate}
    \item The digits are repeated?
    \item The repetition of digits is not allowed?
\end{enumerate}
\solution
%\input{ncert/11/16/4/9/main.tex}
\item Consider the probability space $\brak{\Omega, \mathcal{G}, P}$ where $\Omega = [0,2]$ and $\mathcal{G} = \cbrak{\phi, \Omega, [0,1], (1,2]}$. Let $X$ and $Y$ be two functions on $\Omega$ defined as
\begin{align*}
    X(\omega) = 
    \begin{cases}
        1 & \text{if }\omega \in [0, 1]\\
        2 & \text{if }\omega \in (1, 2]
    \end{cases}
\end{align*}
and
\begin{align*}
    Y(\omega) = 
    \begin{cases}
        2 & \text{if }\omega \in [0, 1.5]\\
        3 & \text{if }\omega \in (1.5, 2].
    \end{cases}
\end{align*}
Then which one of the following statements is true?
\begin{enumerate}
    \item [(A)] $X$ is a random variable with respect to $\mathcal{G}$, but $Y$ is not a random variable with respect to $\mathcal{G}$.
    \item [(B)] $Y$ is a random variable with respect to $\mathcal{G}$, but $X$ is not a random variable with respect to $\mathcal{G}$.
    \item [(C)] Neither $X$ nor $Y$ is a random variable with respect to $\mathcal{G}$.
    \item [(D)] Both $X$ and $Y$ are random variables with respect to $\mathcal{G}$.
\end{enumerate} \hfill (GATE ST 2023)\\
\solution
%\input{gate/ST/2023/14/main.tex}
	\item  A die is loaded in such a way that each odd number is twice as likely to occur as
each even number. Find $P(G)$, where $G$ is the event that a number greater than
3 occurs on a single roll of the die.
\\
\solution
		%\input{exemplar/11/16/3/5/main.tex}
	\item All the jacks, queens and kings are removed from a deck of 52 playing cards. The remaining cards are well shuffled and then one card is drawn at random. Giving ace a value 1 similar value for other cards, find the probability that the card has a value 
		\begin{enumerate}
			\item 7
			\item greater than 7
			\item less than 7
		\end{enumerate}
		%\input{exemplar/10/13/3/30/main.tex}
  \item A Lot consists of 48 mobile phones of which 42 are good, 3 have only minor defects and 3 have major defects.Varnika will buy a phone if it is good but the trader will only buy a mobile if it has no major defects. One phone is selected at random from the lot. What is the probability that it is
\begin{enumerate}
	\item acceptable to Varnika?
            \item acceptable to the trader?
\end{enumerate}
\solution
	%\input{exemplar/10/13/3/40/main.tex}
 \item A student says that if you throw a die, it will show up 1 or not 1. Therefore, the probability of getting 1 and the probability of getting 'not 1' each is equal to $\frac{1}{2}$. Is this correct? Give reasons.\\
 \solution
        %\input{exemplar/10/13/2/9/main.tex}
   \item Four candidates A, B, C, D have ap-
plied for the assignment to coach a school cricket
team. If A is twice as likely to be selected as B, and
B and C are given about the same chance of being
selected, while C is twice as likely to be selected
as D, what are the probabilities that
\begin{enumerate}
\item C will be selected?
\item A will not be selected?
\end{enumerate}
	%\input{exemplar/11/16/3/9/main.tex}
 \item A bag contain 24 balls of which $x$ balls are red, $2x$ are white and $3x$ are blue. A ball is selected at random, What is the probability that it is
\begin{enumerate}[label=\alph*)]
\item not red ?
\item white ?
\end{enumerate}
%\input{exemplar/10/13/3/41/main.tex}
If the letters of the word ASSASSINATION are arranged at random. Find the Probability that
\begin{enumerate}[label=(\alph*)]
\item Four $S's$ come consecutively in the word
\item Two  $I's$ and two $N's$ come together
\item All $A's$ are not coming together
\item No two $A's$ are coming together
\end{enumerate}
%\input{exemplar/11/16/3/14/main.tex}
	\item One urn contains two black balls (labelled B1 and B2) and one white ball. A
	second urn contains one black ball and two white balls (labelled W1 and W2).
	Suppose the following experiment is performed. One of the two urns is chosen
	at random. Next a ball is randomly chosen from the urn. Then a second ball is
	chosen at random from the same urn without replacing the first ball.
	
	\begin{enumerate}
	\item What is the probability that two black balls are chosen?
	
	\item What is the probability that two balls of opposite colour are chosen?
	\end{enumerate}
	\solution
	%\input{exemplar/11/16/3/12/main1.tex}
\end{enumerate}

		%
\item 
Two cards are drawn at random and without replacement from a pack of 52 playing cards. Find the probability that both the cards are black.
\\
\solution
		%\begin{enumerate}[label=\thesection.\arabic*,ref=\thesection.\theenumi]
	\item One card is drawn from a well-shuffled deck of 52 cards. Find the probability of getting
\begin{enumerate}
\item A king of red colour 
\item A face card 
\item A red face card
\item The jack of hearts
\item A spade
\item The queen of diamonds

\end{enumerate}
\solution
		%\input{ncert/10/15/1/14/main.tex}
	\item Five cards—the ten, jack, queen, king and ace of diamonds, are well-shuffled with their face downwards. One card is then picked up at random.
\begin{enumerate}
\item
What is the probability that the card is the queen? 
\item
If the queen is drawn and put aside, what is the probability that the second card picked up is (a) an ace? (b) a queen?\\
\end{enumerate}
\solution
		%\input{ncert/10/15/1/15/defs.tex}
	\item A bag contains $5$ red balls and some blue balls. If the probability of drawing a blue ball is double that if a red ball, determine the number of blue balls in the bag. 
		\\
\solution
		%\input{ncert/10/15/2/3/defs.tex}
	\item A card is selected from a pack of 52 cards.
 \begin{enumerate}[label=(\alph*)] 
                 \item How many points are there in the sample space?
                 \item Calculate the probability that the card is an ace of spades.
                 \item Calculate the probability that the card is (i) an ace and (ii) black card.
 \end{enumerate}
\solution
		%\input{ncert/11/16/3/4/main.tex}
\item Four cards are drawn from a well-shuffled deck of 52 cards. What is the probability of obtaining 3 diamonds and one spade.
\\
\solution
		%\input{ncert/11/16/4/2/defs.tex}
\item In a certain lottery 10,000 tickets are sold and ten equal prizes are awarded. What is the probability of not getting a prize if you buy (a) one ticket (b) two tickets (c) 10 tickets ?	
\\
\solution
		%\input{ncert/11/16/4/4/defs.tex}
		%
\item 
Out of 100 students, two sections of 40 and 60 are formed. If you and your friend are among the 100 students, what is the probability that
\begin{enumerate}
\item you both enter the same section?
\item you both enter the different sections?
\end{enumerate}
\solution
		%\input{ncert/11/16/4/5/defs.tex}
	\item 
The number lock of a suitcase has 4 wheels each labelled with ten digits i.e. from 0 to 9.The lock opens with a sequence of four digits with no repeats.What is the probability of a person getting the right sequence to open the suitcase.
\\
\solution
		%\input{ncert/11/16/4/10/defs.tex}
		%
\item 
Two cards are drawn at random and without replacement from a pack of 52 playing cards. Find the probability that both the cards are black.
\\
\solution
		%\input{ncert/12/13/2/2/defs.tex}
		\item A box of oranges is inspected by examining three randomly selected oranges drawn without replacement. If all the three oranges are good, the box is approved for sale, otherwise, it is rejected. Find the probability that a box containing 15 oranges out of which 12 are good and 3 are bad ones will be approved for sale.
		\label{ncert/12/13/2/3/defs.tex}
		\item Two balls are drawn at random with replacement from a box containing 10 black and 8 red balls. Find the probability that
		\label{ncert/12/13/2/12}
\begin{enumerate}
\item both balls are red.
\item first ball is black and second is red.
\item one of them is black and other is red.
\end{enumerate}

\item In a hostel, 60\% of the students read Hindi newspaper, 40\% read English newspaper and 20\% read both Hindi and English newspapers. A student is selected at random.
		\label{ncert/12/13/2/15}
\begin{enumerate}
\item Find the probability that she reads neither Hindi nor English newspapers.
\item If she reads Hindi newspaper, find the probability that she reads English newspaper.
\item If she reads English newspaper, find the probability that she reads Hindi newspaper.\\
\end{enumerate}
\item The probability of obtaining an even prime number on each die, when a pair of dice is rolled is 
\begin{enumerate}
    \item $0$ 
    
    \item $\frac{1}{3}$ 
    
    \item $\frac{1}{12}$ 
    
    \item $\frac{1}{36}$ 
\end{enumerate}
\solution
		%\input{ncert/12/13/2/17/defs.tex}
	\item A bag contains 4 red and 4 black balls, another bag contains 2 red and 6 black balls. One of the two bags is selected at random and a ball is drawn from the bag which is found to be red. Find the probability that the ball is drawn from the first bag.
\\
\solution
		%\input{ncert/12/13/3/2/main.tex}
  \item
  Cards with numbers 2 to 101 are placed in a box. A card is selected at random.Find the probability that the card has
\begin{enumerate}[label=(\roman*)]
	\item an even number 
	\item a square number
\end{enumerate}
\solution
%\input{exemplar/10/13/3/32/main.tex}
\item
The king, queen and jack of clubs are removed from a deck of 52 playing cards and then well shuffled. Now one card is drawn at random from the remaining cards.  Determine the probability that the card is
\begin{enumerate}[label=(\roman*)]
\item a club
\item 10 of hearts
\end{enumerate}
\solution
%\input{exemplar/10/13/3/29/main.tex}
\item A team of medical students doing their internship have to assist during surgeries
at a city hospital. The probabilities of surgeries rated as very complex, complex,
routine, simple or very simple are respectively, 0.15, 0.20, 0.31, 0.26, .08. Find
the probabilities that a particular surgery will be rated
\begin{enumerate}
	\item complex or very complex;
	\item neither very complex nor very simple;
	\item routine or complex
	\item routine or simple
\end{enumerate}
\solution
%\input{exemplar/11/16/3/8(1)/main.tex}
\item A card is selected from a pack of 52 cards.
\begin{enumerate}[label=(\alph*)]
    \item How many points are there in the sample space?
    \item Calculate the probability that the card is an ace of spades.
    \item Calculate the probability that the card is (i) an ace and (ii) black card.
\end{enumerate}
\solution
%\input{exemplar/11/16/3/4/main2.tex}
\item The probability that a non leap year selected at random will contain 53 sundays.
\\
\solution
%\input{exemplar/10/13/1/19/main.tex}
\item One of the four persons John, Rita, Aslam or Gurpreet will be promoted next
month. Consequently the sample space consists of four elementary outcomes
S = {John promoted, Rita promoted, Aslam promoted, Gurpreet promoted}
You are told that the chances of John’s promotion is same as that of Gurpreet,
Rita’s chances of promotion are twice as likely as Johns. Aslam’s chances are
four times that of John.
\begin{enumerate}
	\item Determine
	\begin{enumerate}
		\item P (John promoted)
		\item P (Rita promoted)
		\item P (Aslam promoted)
		\item P (Gurpreet promoted)
	\end{enumerate}
	\item If A = {John promoted or Gurpreet promoted}, find P (A).
\end{enumerate}
\solution
%\input{exemplar/11/16/3/10/main.tex}
\item A card is drawn from a deck of 52 cards. Find the probability of getting a king or a heart or a red card.\\
\solution
%\input{exemplar/11/16/3/15/main.tex}
\item The probability that a student will pass his examination is 0.73, the probability of
the student getting a compartment is 0.13, and the probability that the student will
either pass or get compartment is 0.96. State True or False.\\
\solution
%\input{exemplar/11/16/3/31/main.tex}
\item A card is selected from a pack of 52 cards\\
\begin{enumerate}[label=(\alph*)]
\item How many points are there in the sample space?
\item Calculate the probability that the cards is an ace of spades.
\item Calculate the probability that the card is (i) an ace (ii)black card.\\
\end{enumerate}
%\input{ncert/11/16/3/4_1/Prob_4.tex}
\item In a non-leap year, the probability of having 53 tuesdays or 53 wednesdays is\\
\solution
%\input{exemplar/11/16/3/18/main.tex}
\item There are 1000 sealed envelopes in a box, 10 of them contain a cash prize of
Rs 100 each, 100 of them contain a cash prize of Rs 50 each and 200 of them
contain a cash prize of Rs 10 each and rest do not contain any cash prize. If they
are well shuffled and an envelope is picked up out, what is the probability that it
contains no cash prize?\\
\solution
%\input{exemplar/10/13/3/34/main.tex}
\item 
A die is thrown and a card is selected at random from a deck of 52 playing cards. The probability of getting an even number on the die and a spade card.\\
\solution
%\input{exemplar/12/13/3/78/main.tex}
\item
If 4-digit numbers greater than 5,000 are randomly formed from the digits 0, 1, 3, 5, and 7, what is the probability of forming a number divisible by 5 when:
\begin{enumerate}
    \item The digits are repeated?
    \item The repetition of digits is not allowed?
\end{enumerate}
\solution
%\input{ncert/11/16/4/9/main.tex}
\item Consider the probability space $\brak{\Omega, \mathcal{G}, P}$ where $\Omega = [0,2]$ and $\mathcal{G} = \cbrak{\phi, \Omega, [0,1], (1,2]}$. Let $X$ and $Y$ be two functions on $\Omega$ defined as
\begin{align*}
    X(\omega) = 
    \begin{cases}
        1 & \text{if }\omega \in [0, 1]\\
        2 & \text{if }\omega \in (1, 2]
    \end{cases}
\end{align*}
and
\begin{align*}
    Y(\omega) = 
    \begin{cases}
        2 & \text{if }\omega \in [0, 1.5]\\
        3 & \text{if }\omega \in (1.5, 2].
    \end{cases}
\end{align*}
Then which one of the following statements is true?
\begin{enumerate}
    \item [(A)] $X$ is a random variable with respect to $\mathcal{G}$, but $Y$ is not a random variable with respect to $\mathcal{G}$.
    \item [(B)] $Y$ is a random variable with respect to $\mathcal{G}$, but $X$ is not a random variable with respect to $\mathcal{G}$.
    \item [(C)] Neither $X$ nor $Y$ is a random variable with respect to $\mathcal{G}$.
    \item [(D)] Both $X$ and $Y$ are random variables with respect to $\mathcal{G}$.
\end{enumerate} \hfill (GATE ST 2023)\\
\solution
%\input{gate/ST/2023/14/main.tex}
	\item  A die is loaded in such a way that each odd number is twice as likely to occur as
each even number. Find $P(G)$, where $G$ is the event that a number greater than
3 occurs on a single roll of the die.
\\
\solution
		%\input{exemplar/11/16/3/5/main.tex}
	\item All the jacks, queens and kings are removed from a deck of 52 playing cards. The remaining cards are well shuffled and then one card is drawn at random. Giving ace a value 1 similar value for other cards, find the probability that the card has a value 
		\begin{enumerate}
			\item 7
			\item greater than 7
			\item less than 7
		\end{enumerate}
		%\input{exemplar/10/13/3/30/main.tex}
  \item A Lot consists of 48 mobile phones of which 42 are good, 3 have only minor defects and 3 have major defects.Varnika will buy a phone if it is good but the trader will only buy a mobile if it has no major defects. One phone is selected at random from the lot. What is the probability that it is
\begin{enumerate}
	\item acceptable to Varnika?
            \item acceptable to the trader?
\end{enumerate}
\solution
	%\input{exemplar/10/13/3/40/main.tex}
 \item A student says that if you throw a die, it will show up 1 or not 1. Therefore, the probability of getting 1 and the probability of getting 'not 1' each is equal to $\frac{1}{2}$. Is this correct? Give reasons.\\
 \solution
        %\input{exemplar/10/13/2/9/main.tex}
   \item Four candidates A, B, C, D have ap-
plied for the assignment to coach a school cricket
team. If A is twice as likely to be selected as B, and
B and C are given about the same chance of being
selected, while C is twice as likely to be selected
as D, what are the probabilities that
\begin{enumerate}
\item C will be selected?
\item A will not be selected?
\end{enumerate}
	%\input{exemplar/11/16/3/9/main.tex}
 \item A bag contain 24 balls of which $x$ balls are red, $2x$ are white and $3x$ are blue. A ball is selected at random, What is the probability that it is
\begin{enumerate}[label=\alph*)]
\item not red ?
\item white ?
\end{enumerate}
%\input{exemplar/10/13/3/41/main.tex}
If the letters of the word ASSASSINATION are arranged at random. Find the Probability that
\begin{enumerate}[label=(\alph*)]
\item Four $S's$ come consecutively in the word
\item Two  $I's$ and two $N's$ come together
\item All $A's$ are not coming together
\item No two $A's$ are coming together
\end{enumerate}
%\input{exemplar/11/16/3/14/main.tex}
	\item One urn contains two black balls (labelled B1 and B2) and one white ball. A
	second urn contains one black ball and two white balls (labelled W1 and W2).
	Suppose the following experiment is performed. One of the two urns is chosen
	at random. Next a ball is randomly chosen from the urn. Then a second ball is
	chosen at random from the same urn without replacing the first ball.
	
	\begin{enumerate}
	\item What is the probability that two black balls are chosen?
	
	\item What is the probability that two balls of opposite colour are chosen?
	\end{enumerate}
	\solution
	%\input{exemplar/11/16/3/12/main1.tex}
\end{enumerate}

		\item A box of oranges is inspected by examining three randomly selected oranges drawn without replacement. If all the three oranges are good, the box is approved for sale, otherwise, it is rejected. Find the probability that a box containing 15 oranges out of which 12 are good and 3 are bad ones will be approved for sale.
		\label{ncert/12/13/2/3/defs.tex}
		\item Two balls are drawn at random with replacement from a box containing 10 black and 8 red balls. Find the probability that
		\label{ncert/12/13/2/12}
\begin{enumerate}
\item both balls are red.
\item first ball is black and second is red.
\item one of them is black and other is red.
\end{enumerate}

\item In a hostel, 60\% of the students read Hindi newspaper, 40\% read English newspaper and 20\% read both Hindi and English newspapers. A student is selected at random.
		\label{ncert/12/13/2/15}
\begin{enumerate}
\item Find the probability that she reads neither Hindi nor English newspapers.
\item If she reads Hindi newspaper, find the probability that she reads English newspaper.
\item If she reads English newspaper, find the probability that she reads Hindi newspaper.\\
\end{enumerate}
\item The probability of obtaining an even prime number on each die, when a pair of dice is rolled is 
\begin{enumerate}
    \item $0$ 
    
    \item $\frac{1}{3}$ 
    
    \item $\frac{1}{12}$ 
    
    \item $\frac{1}{36}$ 
\end{enumerate}
\solution
		%\begin{enumerate}[label=\thesection.\arabic*,ref=\thesection.\theenumi]
	\item One card is drawn from a well-shuffled deck of 52 cards. Find the probability of getting
\begin{enumerate}
\item A king of red colour 
\item A face card 
\item A red face card
\item The jack of hearts
\item A spade
\item The queen of diamonds

\end{enumerate}
\solution
		%\input{ncert/10/15/1/14/main.tex}
	\item Five cards—the ten, jack, queen, king and ace of diamonds, are well-shuffled with their face downwards. One card is then picked up at random.
\begin{enumerate}
\item
What is the probability that the card is the queen? 
\item
If the queen is drawn and put aside, what is the probability that the second card picked up is (a) an ace? (b) a queen?\\
\end{enumerate}
\solution
		%\input{ncert/10/15/1/15/defs.tex}
	\item A bag contains $5$ red balls and some blue balls. If the probability of drawing a blue ball is double that if a red ball, determine the number of blue balls in the bag. 
		\\
\solution
		%\input{ncert/10/15/2/3/defs.tex}
	\item A card is selected from a pack of 52 cards.
 \begin{enumerate}[label=(\alph*)] 
                 \item How many points are there in the sample space?
                 \item Calculate the probability that the card is an ace of spades.
                 \item Calculate the probability that the card is (i) an ace and (ii) black card.
 \end{enumerate}
\solution
		%\input{ncert/11/16/3/4/main.tex}
\item Four cards are drawn from a well-shuffled deck of 52 cards. What is the probability of obtaining 3 diamonds and one spade.
\\
\solution
		%\input{ncert/11/16/4/2/defs.tex}
\item In a certain lottery 10,000 tickets are sold and ten equal prizes are awarded. What is the probability of not getting a prize if you buy (a) one ticket (b) two tickets (c) 10 tickets ?	
\\
\solution
		%\input{ncert/11/16/4/4/defs.tex}
		%
\item 
Out of 100 students, two sections of 40 and 60 are formed. If you and your friend are among the 100 students, what is the probability that
\begin{enumerate}
\item you both enter the same section?
\item you both enter the different sections?
\end{enumerate}
\solution
		%\input{ncert/11/16/4/5/defs.tex}
	\item 
The number lock of a suitcase has 4 wheels each labelled with ten digits i.e. from 0 to 9.The lock opens with a sequence of four digits with no repeats.What is the probability of a person getting the right sequence to open the suitcase.
\\
\solution
		%\input{ncert/11/16/4/10/defs.tex}
		%
\item 
Two cards are drawn at random and without replacement from a pack of 52 playing cards. Find the probability that both the cards are black.
\\
\solution
		%\input{ncert/12/13/2/2/defs.tex}
		\item A box of oranges is inspected by examining three randomly selected oranges drawn without replacement. If all the three oranges are good, the box is approved for sale, otherwise, it is rejected. Find the probability that a box containing 15 oranges out of which 12 are good and 3 are bad ones will be approved for sale.
		\label{ncert/12/13/2/3/defs.tex}
		\item Two balls are drawn at random with replacement from a box containing 10 black and 8 red balls. Find the probability that
		\label{ncert/12/13/2/12}
\begin{enumerate}
\item both balls are red.
\item first ball is black and second is red.
\item one of them is black and other is red.
\end{enumerate}

\item In a hostel, 60\% of the students read Hindi newspaper, 40\% read English newspaper and 20\% read both Hindi and English newspapers. A student is selected at random.
		\label{ncert/12/13/2/15}
\begin{enumerate}
\item Find the probability that she reads neither Hindi nor English newspapers.
\item If she reads Hindi newspaper, find the probability that she reads English newspaper.
\item If she reads English newspaper, find the probability that she reads Hindi newspaper.\\
\end{enumerate}
\item The probability of obtaining an even prime number on each die, when a pair of dice is rolled is 
\begin{enumerate}
    \item $0$ 
    
    \item $\frac{1}{3}$ 
    
    \item $\frac{1}{12}$ 
    
    \item $\frac{1}{36}$ 
\end{enumerate}
\solution
		%\input{ncert/12/13/2/17/defs.tex}
	\item A bag contains 4 red and 4 black balls, another bag contains 2 red and 6 black balls. One of the two bags is selected at random and a ball is drawn from the bag which is found to be red. Find the probability that the ball is drawn from the first bag.
\\
\solution
		%\input{ncert/12/13/3/2/main.tex}
  \item
  Cards with numbers 2 to 101 are placed in a box. A card is selected at random.Find the probability that the card has
\begin{enumerate}[label=(\roman*)]
	\item an even number 
	\item a square number
\end{enumerate}
\solution
%\input{exemplar/10/13/3/32/main.tex}
\item
The king, queen and jack of clubs are removed from a deck of 52 playing cards and then well shuffled. Now one card is drawn at random from the remaining cards.  Determine the probability that the card is
\begin{enumerate}[label=(\roman*)]
\item a club
\item 10 of hearts
\end{enumerate}
\solution
%\input{exemplar/10/13/3/29/main.tex}
\item A team of medical students doing their internship have to assist during surgeries
at a city hospital. The probabilities of surgeries rated as very complex, complex,
routine, simple or very simple are respectively, 0.15, 0.20, 0.31, 0.26, .08. Find
the probabilities that a particular surgery will be rated
\begin{enumerate}
	\item complex or very complex;
	\item neither very complex nor very simple;
	\item routine or complex
	\item routine or simple
\end{enumerate}
\solution
%\input{exemplar/11/16/3/8(1)/main.tex}
\item A card is selected from a pack of 52 cards.
\begin{enumerate}[label=(\alph*)]
    \item How many points are there in the sample space?
    \item Calculate the probability that the card is an ace of spades.
    \item Calculate the probability that the card is (i) an ace and (ii) black card.
\end{enumerate}
\solution
%\input{exemplar/11/16/3/4/main2.tex}
\item The probability that a non leap year selected at random will contain 53 sundays.
\\
\solution
%\input{exemplar/10/13/1/19/main.tex}
\item One of the four persons John, Rita, Aslam or Gurpreet will be promoted next
month. Consequently the sample space consists of four elementary outcomes
S = {John promoted, Rita promoted, Aslam promoted, Gurpreet promoted}
You are told that the chances of John’s promotion is same as that of Gurpreet,
Rita’s chances of promotion are twice as likely as Johns. Aslam’s chances are
four times that of John.
\begin{enumerate}
	\item Determine
	\begin{enumerate}
		\item P (John promoted)
		\item P (Rita promoted)
		\item P (Aslam promoted)
		\item P (Gurpreet promoted)
	\end{enumerate}
	\item If A = {John promoted or Gurpreet promoted}, find P (A).
\end{enumerate}
\solution
%\input{exemplar/11/16/3/10/main.tex}
\item A card is drawn from a deck of 52 cards. Find the probability of getting a king or a heart or a red card.\\
\solution
%\input{exemplar/11/16/3/15/main.tex}
\item The probability that a student will pass his examination is 0.73, the probability of
the student getting a compartment is 0.13, and the probability that the student will
either pass or get compartment is 0.96. State True or False.\\
\solution
%\input{exemplar/11/16/3/31/main.tex}
\item A card is selected from a pack of 52 cards\\
\begin{enumerate}[label=(\alph*)]
\item How many points are there in the sample space?
\item Calculate the probability that the cards is an ace of spades.
\item Calculate the probability that the card is (i) an ace (ii)black card.\\
\end{enumerate}
%\input{ncert/11/16/3/4_1/Prob_4.tex}
\item In a non-leap year, the probability of having 53 tuesdays or 53 wednesdays is\\
\solution
%\input{exemplar/11/16/3/18/main.tex}
\item There are 1000 sealed envelopes in a box, 10 of them contain a cash prize of
Rs 100 each, 100 of them contain a cash prize of Rs 50 each and 200 of them
contain a cash prize of Rs 10 each and rest do not contain any cash prize. If they
are well shuffled and an envelope is picked up out, what is the probability that it
contains no cash prize?\\
\solution
%\input{exemplar/10/13/3/34/main.tex}
\item 
A die is thrown and a card is selected at random from a deck of 52 playing cards. The probability of getting an even number on the die and a spade card.\\
\solution
%\input{exemplar/12/13/3/78/main.tex}
\item
If 4-digit numbers greater than 5,000 are randomly formed from the digits 0, 1, 3, 5, and 7, what is the probability of forming a number divisible by 5 when:
\begin{enumerate}
    \item The digits are repeated?
    \item The repetition of digits is not allowed?
\end{enumerate}
\solution
%\input{ncert/11/16/4/9/main.tex}
\item Consider the probability space $\brak{\Omega, \mathcal{G}, P}$ where $\Omega = [0,2]$ and $\mathcal{G} = \cbrak{\phi, \Omega, [0,1], (1,2]}$. Let $X$ and $Y$ be two functions on $\Omega$ defined as
\begin{align*}
    X(\omega) = 
    \begin{cases}
        1 & \text{if }\omega \in [0, 1]\\
        2 & \text{if }\omega \in (1, 2]
    \end{cases}
\end{align*}
and
\begin{align*}
    Y(\omega) = 
    \begin{cases}
        2 & \text{if }\omega \in [0, 1.5]\\
        3 & \text{if }\omega \in (1.5, 2].
    \end{cases}
\end{align*}
Then which one of the following statements is true?
\begin{enumerate}
    \item [(A)] $X$ is a random variable with respect to $\mathcal{G}$, but $Y$ is not a random variable with respect to $\mathcal{G}$.
    \item [(B)] $Y$ is a random variable with respect to $\mathcal{G}$, but $X$ is not a random variable with respect to $\mathcal{G}$.
    \item [(C)] Neither $X$ nor $Y$ is a random variable with respect to $\mathcal{G}$.
    \item [(D)] Both $X$ and $Y$ are random variables with respect to $\mathcal{G}$.
\end{enumerate} \hfill (GATE ST 2023)\\
\solution
%\input{gate/ST/2023/14/main.tex}
	\item  A die is loaded in such a way that each odd number is twice as likely to occur as
each even number. Find $P(G)$, where $G$ is the event that a number greater than
3 occurs on a single roll of the die.
\\
\solution
		%\input{exemplar/11/16/3/5/main.tex}
	\item All the jacks, queens and kings are removed from a deck of 52 playing cards. The remaining cards are well shuffled and then one card is drawn at random. Giving ace a value 1 similar value for other cards, find the probability that the card has a value 
		\begin{enumerate}
			\item 7
			\item greater than 7
			\item less than 7
		\end{enumerate}
		%\input{exemplar/10/13/3/30/main.tex}
  \item A Lot consists of 48 mobile phones of which 42 are good, 3 have only minor defects and 3 have major defects.Varnika will buy a phone if it is good but the trader will only buy a mobile if it has no major defects. One phone is selected at random from the lot. What is the probability that it is
\begin{enumerate}
	\item acceptable to Varnika?
            \item acceptable to the trader?
\end{enumerate}
\solution
	%\input{exemplar/10/13/3/40/main.tex}
 \item A student says that if you throw a die, it will show up 1 or not 1. Therefore, the probability of getting 1 and the probability of getting 'not 1' each is equal to $\frac{1}{2}$. Is this correct? Give reasons.\\
 \solution
        %\input{exemplar/10/13/2/9/main.tex}
   \item Four candidates A, B, C, D have ap-
plied for the assignment to coach a school cricket
team. If A is twice as likely to be selected as B, and
B and C are given about the same chance of being
selected, while C is twice as likely to be selected
as D, what are the probabilities that
\begin{enumerate}
\item C will be selected?
\item A will not be selected?
\end{enumerate}
	%\input{exemplar/11/16/3/9/main.tex}
 \item A bag contain 24 balls of which $x$ balls are red, $2x$ are white and $3x$ are blue. A ball is selected at random, What is the probability that it is
\begin{enumerate}[label=\alph*)]
\item not red ?
\item white ?
\end{enumerate}
%\input{exemplar/10/13/3/41/main.tex}
If the letters of the word ASSASSINATION are arranged at random. Find the Probability that
\begin{enumerate}[label=(\alph*)]
\item Four $S's$ come consecutively in the word
\item Two  $I's$ and two $N's$ come together
\item All $A's$ are not coming together
\item No two $A's$ are coming together
\end{enumerate}
%\input{exemplar/11/16/3/14/main.tex}
	\item One urn contains two black balls (labelled B1 and B2) and one white ball. A
	second urn contains one black ball and two white balls (labelled W1 and W2).
	Suppose the following experiment is performed. One of the two urns is chosen
	at random. Next a ball is randomly chosen from the urn. Then a second ball is
	chosen at random from the same urn without replacing the first ball.
	
	\begin{enumerate}
	\item What is the probability that two black balls are chosen?
	
	\item What is the probability that two balls of opposite colour are chosen?
	\end{enumerate}
	\solution
	%\input{exemplar/11/16/3/12/main1.tex}
\end{enumerate}

	\item A bag contains 4 red and 4 black balls, another bag contains 2 red and 6 black balls. One of the two bags is selected at random and a ball is drawn from the bag which is found to be red. Find the probability that the ball is drawn from the first bag.
\\
\solution
		%\begin{table}[H]
	\centering
\begin{tabular}{|c|c|c|}
\hline
Random variable &Value &Definition\\ \hline
\multirow{3}{*}{X} &0 &Slips of Rs 1\\
&1 &Slips of Rs 5\\
&2 &Slips of Rs 13\\ \hline
\multirow{2}{*}{Y} &0 &Box A\\
&1 &Box B\\\hline
\end{tabular}
\caption{}
\label{tab:Distribution}
\end{table}
See \tabref{tab:Distribution}.
\begin{align}
p_{Y}\brak{k}= \begin{cases} 
      \frac{1}{3} & {k=0} \\
      \frac{2}{3 }& {k=1} 
   \end{cases}
   \\
p_{Y|X}\brak{0|0} = \frac{19}{25}\, 
p_{Y|X}\brak{0|1} = \frac{6}{25}\,
p_{Y|X}\brak{1|0} = \frac{45}{50}\,
p_{Y|X}\brak{1|2} = \frac{5}{50}
\end{align}
The desired probability is the probability that a slip drawn at random is marked other than Rs 1,
\begin{align}
&=1-p_X\brak{0}\\
&= p_X(1) + p_X(2)
\end{align}
Using Bayes theorem,
\begin{align}
&= p_Y\brak{0} \times \pr{Y=0 | X=1} + p_Y\brak{1} \times \pr{Y=1|X=2}\\
&=\frac{1}{3} \times \frac{6}{25} + \frac{2}{3} \times \frac{5}{50}\\
&=\frac{11}{75}
\end{align}

\newpage

%\tableofcontents

\bigskip

\renewcommand{\thefigure}{\theenumi}
\renewcommand{\thetable}{\theenumi}
%\renewcommand{\theequation}{\theenumi}

%\begin{abstract}
%%\boldmath
%In this letter, an algorithm for evaluating the exact analytical bit error rate  (BER)  for the piecewise linear (PL) combiner for  multiple relays is presented. Previous results were available only for upto three relays. The algorithm is unique in the sense that  the actual mathematical expressions, that are prohibitively large, need not be explicitly obtained. The diversity gain due to multiple relays is shown through plots of the analytical BER, well supported by simulations. 
%
%\end{abstract}
% IEEEtran.cls defaults to using nonbold math in the Abstract.
% This preserves the distinction between vectors and scalars. However,
% if the journal you are submitting to favors bold math in the abstract,
% then you can use LaTeX's standard command \boldmath at the very start
% of the abstract to achieve this. Many IEEE journals frown on math
% in the abstract anyway.

% Note that keywords are not normally used for peerreview papers.
%\begin{IEEEkeywords}
%Cooperative diversity, decode and forward, piecewise linear
%\end{IEEEkeywords}



% For peer review papers, you can put extra information on the cover
% page as needed:
% \ifCLASSOPTIONpeerreview
% \begin{center} \bfseries EDICS Category: 3-BBND \end{center}
% \fi
%
% For peerreview papers, this IEEEtran command inserts a page break and
% creates the second title. It will be ignored for other modes.
%\IEEEpeerreviewmaketitle




  \item
  Cards with numbers 2 to 101 are placed in a box. A card is selected at random.Find the probability that the card has
\begin{enumerate}[label=(\roman*)]
	\item an even number 
	\item a square number
\end{enumerate}
\solution
%\begin{table}[H]
	\centering
\begin{tabular}{|c|c|c|}
\hline
Random variable &Value &Definition\\ \hline
\multirow{3}{*}{X} &0 &Slips of Rs 1\\
&1 &Slips of Rs 5\\
&2 &Slips of Rs 13\\ \hline
\multirow{2}{*}{Y} &0 &Box A\\
&1 &Box B\\\hline
\end{tabular}
\caption{}
\label{tab:Distribution}
\end{table}
See \tabref{tab:Distribution}.
\begin{align}
p_{Y}\brak{k}= \begin{cases} 
      \frac{1}{3} & {k=0} \\
      \frac{2}{3 }& {k=1} 
   \end{cases}
   \\
p_{Y|X}\brak{0|0} = \frac{19}{25}\, 
p_{Y|X}\brak{0|1} = \frac{6}{25}\,
p_{Y|X}\brak{1|0} = \frac{45}{50}\,
p_{Y|X}\brak{1|2} = \frac{5}{50}
\end{align}
The desired probability is the probability that a slip drawn at random is marked other than Rs 1,
\begin{align}
&=1-p_X\brak{0}\\
&= p_X(1) + p_X(2)
\end{align}
Using Bayes theorem,
\begin{align}
&= p_Y\brak{0} \times \pr{Y=0 | X=1} + p_Y\brak{1} \times \pr{Y=1|X=2}\\
&=\frac{1}{3} \times \frac{6}{25} + \frac{2}{3} \times \frac{5}{50}\\
&=\frac{11}{75}
\end{align}

\newpage

%\tableofcontents

\bigskip

\renewcommand{\thefigure}{\theenumi}
\renewcommand{\thetable}{\theenumi}
%\renewcommand{\theequation}{\theenumi}

%\begin{abstract}
%%\boldmath
%In this letter, an algorithm for evaluating the exact analytical bit error rate  (BER)  for the piecewise linear (PL) combiner for  multiple relays is presented. Previous results were available only for upto three relays. The algorithm is unique in the sense that  the actual mathematical expressions, that are prohibitively large, need not be explicitly obtained. The diversity gain due to multiple relays is shown through plots of the analytical BER, well supported by simulations. 
%
%\end{abstract}
% IEEEtran.cls defaults to using nonbold math in the Abstract.
% This preserves the distinction between vectors and scalars. However,
% if the journal you are submitting to favors bold math in the abstract,
% then you can use LaTeX's standard command \boldmath at the very start
% of the abstract to achieve this. Many IEEE journals frown on math
% in the abstract anyway.

% Note that keywords are not normally used for peerreview papers.
%\begin{IEEEkeywords}
%Cooperative diversity, decode and forward, piecewise linear
%\end{IEEEkeywords}



% For peer review papers, you can put extra information on the cover
% page as needed:
% \ifCLASSOPTIONpeerreview
% \begin{center} \bfseries EDICS Category: 3-BBND \end{center}
% \fi
%
% For peerreview papers, this IEEEtran command inserts a page break and
% creates the second title. It will be ignored for other modes.
%\IEEEpeerreviewmaketitle




\item
The king, queen and jack of clubs are removed from a deck of 52 playing cards and then well shuffled. Now one card is drawn at random from the remaining cards.  Determine the probability that the card is
\begin{enumerate}[label=(\roman*)]
\item a club
\item 10 of hearts
\end{enumerate}
\solution
%\begin{table}[H]
	\centering
\begin{tabular}{|c|c|c|}
\hline
Random variable &Value &Definition\\ \hline
\multirow{3}{*}{X} &0 &Slips of Rs 1\\
&1 &Slips of Rs 5\\
&2 &Slips of Rs 13\\ \hline
\multirow{2}{*}{Y} &0 &Box A\\
&1 &Box B\\\hline
\end{tabular}
\caption{}
\label{tab:Distribution}
\end{table}
See \tabref{tab:Distribution}.
\begin{align}
p_{Y}\brak{k}= \begin{cases} 
      \frac{1}{3} & {k=0} \\
      \frac{2}{3 }& {k=1} 
   \end{cases}
   \\
p_{Y|X}\brak{0|0} = \frac{19}{25}\, 
p_{Y|X}\brak{0|1} = \frac{6}{25}\,
p_{Y|X}\brak{1|0} = \frac{45}{50}\,
p_{Y|X}\brak{1|2} = \frac{5}{50}
\end{align}
The desired probability is the probability that a slip drawn at random is marked other than Rs 1,
\begin{align}
&=1-p_X\brak{0}\\
&= p_X(1) + p_X(2)
\end{align}
Using Bayes theorem,
\begin{align}
&= p_Y\brak{0} \times \pr{Y=0 | X=1} + p_Y\brak{1} \times \pr{Y=1|X=2}\\
&=\frac{1}{3} \times \frac{6}{25} + \frac{2}{3} \times \frac{5}{50}\\
&=\frac{11}{75}
\end{align}

\newpage

%\tableofcontents

\bigskip

\renewcommand{\thefigure}{\theenumi}
\renewcommand{\thetable}{\theenumi}
%\renewcommand{\theequation}{\theenumi}

%\begin{abstract}
%%\boldmath
%In this letter, an algorithm for evaluating the exact analytical bit error rate  (BER)  for the piecewise linear (PL) combiner for  multiple relays is presented. Previous results were available only for upto three relays. The algorithm is unique in the sense that  the actual mathematical expressions, that are prohibitively large, need not be explicitly obtained. The diversity gain due to multiple relays is shown through plots of the analytical BER, well supported by simulations. 
%
%\end{abstract}
% IEEEtran.cls defaults to using nonbold math in the Abstract.
% This preserves the distinction between vectors and scalars. However,
% if the journal you are submitting to favors bold math in the abstract,
% then you can use LaTeX's standard command \boldmath at the very start
% of the abstract to achieve this. Many IEEE journals frown on math
% in the abstract anyway.

% Note that keywords are not normally used for peerreview papers.
%\begin{IEEEkeywords}
%Cooperative diversity, decode and forward, piecewise linear
%\end{IEEEkeywords}



% For peer review papers, you can put extra information on the cover
% page as needed:
% \ifCLASSOPTIONpeerreview
% \begin{center} \bfseries EDICS Category: 3-BBND \end{center}
% \fi
%
% For peerreview papers, this IEEEtran command inserts a page break and
% creates the second title. It will be ignored for other modes.
%\IEEEpeerreviewmaketitle




\item A team of medical students doing their internship have to assist during surgeries
at a city hospital. The probabilities of surgeries rated as very complex, complex,
routine, simple or very simple are respectively, 0.15, 0.20, 0.31, 0.26, .08. Find
the probabilities that a particular surgery will be rated
\begin{enumerate}
	\item complex or very complex;
	\item neither very complex nor very simple;
	\item routine or complex
	\item routine or simple
\end{enumerate}
\solution
%\begin{table}[H]
	\centering
\begin{tabular}{|c|c|c|}
\hline
Random variable &Value &Definition\\ \hline
\multirow{3}{*}{X} &0 &Slips of Rs 1\\
&1 &Slips of Rs 5\\
&2 &Slips of Rs 13\\ \hline
\multirow{2}{*}{Y} &0 &Box A\\
&1 &Box B\\\hline
\end{tabular}
\caption{}
\label{tab:Distribution}
\end{table}
See \tabref{tab:Distribution}.
\begin{align}
p_{Y}\brak{k}= \begin{cases} 
      \frac{1}{3} & {k=0} \\
      \frac{2}{3 }& {k=1} 
   \end{cases}
   \\
p_{Y|X}\brak{0|0} = \frac{19}{25}\, 
p_{Y|X}\brak{0|1} = \frac{6}{25}\,
p_{Y|X}\brak{1|0} = \frac{45}{50}\,
p_{Y|X}\brak{1|2} = \frac{5}{50}
\end{align}
The desired probability is the probability that a slip drawn at random is marked other than Rs 1,
\begin{align}
&=1-p_X\brak{0}\\
&= p_X(1) + p_X(2)
\end{align}
Using Bayes theorem,
\begin{align}
&= p_Y\brak{0} \times \pr{Y=0 | X=1} + p_Y\brak{1} \times \pr{Y=1|X=2}\\
&=\frac{1}{3} \times \frac{6}{25} + \frac{2}{3} \times \frac{5}{50}\\
&=\frac{11}{75}
\end{align}

\newpage

%\tableofcontents

\bigskip

\renewcommand{\thefigure}{\theenumi}
\renewcommand{\thetable}{\theenumi}
%\renewcommand{\theequation}{\theenumi}

%\begin{abstract}
%%\boldmath
%In this letter, an algorithm for evaluating the exact analytical bit error rate  (BER)  for the piecewise linear (PL) combiner for  multiple relays is presented. Previous results were available only for upto three relays. The algorithm is unique in the sense that  the actual mathematical expressions, that are prohibitively large, need not be explicitly obtained. The diversity gain due to multiple relays is shown through plots of the analytical BER, well supported by simulations. 
%
%\end{abstract}
% IEEEtran.cls defaults to using nonbold math in the Abstract.
% This preserves the distinction between vectors and scalars. However,
% if the journal you are submitting to favors bold math in the abstract,
% then you can use LaTeX's standard command \boldmath at the very start
% of the abstract to achieve this. Many IEEE journals frown on math
% in the abstract anyway.

% Note that keywords are not normally used for peerreview papers.
%\begin{IEEEkeywords}
%Cooperative diversity, decode and forward, piecewise linear
%\end{IEEEkeywords}



% For peer review papers, you can put extra information on the cover
% page as needed:
% \ifCLASSOPTIONpeerreview
% \begin{center} \bfseries EDICS Category: 3-BBND \end{center}
% \fi
%
% For peerreview papers, this IEEEtran command inserts a page break and
% creates the second title. It will be ignored for other modes.
%\IEEEpeerreviewmaketitle




\item A card is selected from a pack of 52 cards.
\begin{enumerate}[label=(\alph*)]
    \item How many points are there in the sample space?
    \item Calculate the probability that the card is an ace of spades.
    \item Calculate the probability that the card is (i) an ace and (ii) black card.
\end{enumerate}
\solution
%Let $X$ be an bernoulli rv defined as in \tabref{tab:exemplar/11/16/3/26}.  Then, 
\begin{equation}
    p =
        \frac{4}{11} 
\end{equation}
\begin{table}[H]
	\centering
	\input{exemplar/11/16/3/26/tables/Table2.tex}
	\caption{}
        \label{tab:exemplar/11/16/3/26}
\end{table}

\item The probability that a non leap year selected at random will contain 53 sundays.
\\
\solution
%\begin{table}[H]
	\centering
\begin{tabular}{|c|c|c|}
\hline
Random variable &Value &Definition\\ \hline
\multirow{3}{*}{X} &0 &Slips of Rs 1\\
&1 &Slips of Rs 5\\
&2 &Slips of Rs 13\\ \hline
\multirow{2}{*}{Y} &0 &Box A\\
&1 &Box B\\\hline
\end{tabular}
\caption{}
\label{tab:Distribution}
\end{table}
See \tabref{tab:Distribution}.
\begin{align}
p_{Y}\brak{k}= \begin{cases} 
      \frac{1}{3} & {k=0} \\
      \frac{2}{3 }& {k=1} 
   \end{cases}
   \\
p_{Y|X}\brak{0|0} = \frac{19}{25}\, 
p_{Y|X}\brak{0|1} = \frac{6}{25}\,
p_{Y|X}\brak{1|0} = \frac{45}{50}\,
p_{Y|X}\brak{1|2} = \frac{5}{50}
\end{align}
The desired probability is the probability that a slip drawn at random is marked other than Rs 1,
\begin{align}
&=1-p_X\brak{0}\\
&= p_X(1) + p_X(2)
\end{align}
Using Bayes theorem,
\begin{align}
&= p_Y\brak{0} \times \pr{Y=0 | X=1} + p_Y\brak{1} \times \pr{Y=1|X=2}\\
&=\frac{1}{3} \times \frac{6}{25} + \frac{2}{3} \times \frac{5}{50}\\
&=\frac{11}{75}
\end{align}

\newpage

%\tableofcontents

\bigskip

\renewcommand{\thefigure}{\theenumi}
\renewcommand{\thetable}{\theenumi}
%\renewcommand{\theequation}{\theenumi}

%\begin{abstract}
%%\boldmath
%In this letter, an algorithm for evaluating the exact analytical bit error rate  (BER)  for the piecewise linear (PL) combiner for  multiple relays is presented. Previous results were available only for upto three relays. The algorithm is unique in the sense that  the actual mathematical expressions, that are prohibitively large, need not be explicitly obtained. The diversity gain due to multiple relays is shown through plots of the analytical BER, well supported by simulations. 
%
%\end{abstract}
% IEEEtran.cls defaults to using nonbold math in the Abstract.
% This preserves the distinction between vectors and scalars. However,
% if the journal you are submitting to favors bold math in the abstract,
% then you can use LaTeX's standard command \boldmath at the very start
% of the abstract to achieve this. Many IEEE journals frown on math
% in the abstract anyway.

% Note that keywords are not normally used for peerreview papers.
%\begin{IEEEkeywords}
%Cooperative diversity, decode and forward, piecewise linear
%\end{IEEEkeywords}



% For peer review papers, you can put extra information on the cover
% page as needed:
% \ifCLASSOPTIONpeerreview
% \begin{center} \bfseries EDICS Category: 3-BBND \end{center}
% \fi
%
% For peerreview papers, this IEEEtran command inserts a page break and
% creates the second title. It will be ignored for other modes.
%\IEEEpeerreviewmaketitle




\item One of the four persons John, Rita, Aslam or Gurpreet will be promoted next
month. Consequently the sample space consists of four elementary outcomes
S = {John promoted, Rita promoted, Aslam promoted, Gurpreet promoted}
You are told that the chances of John’s promotion is same as that of Gurpreet,
Rita’s chances of promotion are twice as likely as Johns. Aslam’s chances are
four times that of John.
\begin{enumerate}
	\item Determine
	\begin{enumerate}
		\item P (John promoted)
		\item P (Rita promoted)
		\item P (Aslam promoted)
		\item P (Gurpreet promoted)
	\end{enumerate}
	\item If A = {John promoted or Gurpreet promoted}, find P (A).
\end{enumerate}
\solution
%\begin{table}[H]
	\centering
\begin{tabular}{|c|c|c|}
\hline
Random variable &Value &Definition\\ \hline
\multirow{3}{*}{X} &0 &Slips of Rs 1\\
&1 &Slips of Rs 5\\
&2 &Slips of Rs 13\\ \hline
\multirow{2}{*}{Y} &0 &Box A\\
&1 &Box B\\\hline
\end{tabular}
\caption{}
\label{tab:Distribution}
\end{table}
See \tabref{tab:Distribution}.
\begin{align}
p_{Y}\brak{k}= \begin{cases} 
      \frac{1}{3} & {k=0} \\
      \frac{2}{3 }& {k=1} 
   \end{cases}
   \\
p_{Y|X}\brak{0|0} = \frac{19}{25}\, 
p_{Y|X}\brak{0|1} = \frac{6}{25}\,
p_{Y|X}\brak{1|0} = \frac{45}{50}\,
p_{Y|X}\brak{1|2} = \frac{5}{50}
\end{align}
The desired probability is the probability that a slip drawn at random is marked other than Rs 1,
\begin{align}
&=1-p_X\brak{0}\\
&= p_X(1) + p_X(2)
\end{align}
Using Bayes theorem,
\begin{align}
&= p_Y\brak{0} \times \pr{Y=0 | X=1} + p_Y\brak{1} \times \pr{Y=1|X=2}\\
&=\frac{1}{3} \times \frac{6}{25} + \frac{2}{3} \times \frac{5}{50}\\
&=\frac{11}{75}
\end{align}

\newpage

%\tableofcontents

\bigskip

\renewcommand{\thefigure}{\theenumi}
\renewcommand{\thetable}{\theenumi}
%\renewcommand{\theequation}{\theenumi}

%\begin{abstract}
%%\boldmath
%In this letter, an algorithm for evaluating the exact analytical bit error rate  (BER)  for the piecewise linear (PL) combiner for  multiple relays is presented. Previous results were available only for upto three relays. The algorithm is unique in the sense that  the actual mathematical expressions, that are prohibitively large, need not be explicitly obtained. The diversity gain due to multiple relays is shown through plots of the analytical BER, well supported by simulations. 
%
%\end{abstract}
% IEEEtran.cls defaults to using nonbold math in the Abstract.
% This preserves the distinction between vectors and scalars. However,
% if the journal you are submitting to favors bold math in the abstract,
% then you can use LaTeX's standard command \boldmath at the very start
% of the abstract to achieve this. Many IEEE journals frown on math
% in the abstract anyway.

% Note that keywords are not normally used for peerreview papers.
%\begin{IEEEkeywords}
%Cooperative diversity, decode and forward, piecewise linear
%\end{IEEEkeywords}



% For peer review papers, you can put extra information on the cover
% page as needed:
% \ifCLASSOPTIONpeerreview
% \begin{center} \bfseries EDICS Category: 3-BBND \end{center}
% \fi
%
% For peerreview papers, this IEEEtran command inserts a page break and
% creates the second title. It will be ignored for other modes.
%\IEEEpeerreviewmaketitle




\item A card is drawn from a deck of 52 cards. Find the probability of getting a king or a heart or a red card.\\
\solution
%\begin{table}[H]
	\centering
\begin{tabular}{|c|c|c|}
\hline
Random variable &Value &Definition\\ \hline
\multirow{3}{*}{X} &0 &Slips of Rs 1\\
&1 &Slips of Rs 5\\
&2 &Slips of Rs 13\\ \hline
\multirow{2}{*}{Y} &0 &Box A\\
&1 &Box B\\\hline
\end{tabular}
\caption{}
\label{tab:Distribution}
\end{table}
See \tabref{tab:Distribution}.
\begin{align}
p_{Y}\brak{k}= \begin{cases} 
      \frac{1}{3} & {k=0} \\
      \frac{2}{3 }& {k=1} 
   \end{cases}
   \\
p_{Y|X}\brak{0|0} = \frac{19}{25}\, 
p_{Y|X}\brak{0|1} = \frac{6}{25}\,
p_{Y|X}\brak{1|0} = \frac{45}{50}\,
p_{Y|X}\brak{1|2} = \frac{5}{50}
\end{align}
The desired probability is the probability that a slip drawn at random is marked other than Rs 1,
\begin{align}
&=1-p_X\brak{0}\\
&= p_X(1) + p_X(2)
\end{align}
Using Bayes theorem,
\begin{align}
&= p_Y\brak{0} \times \pr{Y=0 | X=1} + p_Y\brak{1} \times \pr{Y=1|X=2}\\
&=\frac{1}{3} \times \frac{6}{25} + \frac{2}{3} \times \frac{5}{50}\\
&=\frac{11}{75}
\end{align}

\newpage

%\tableofcontents

\bigskip

\renewcommand{\thefigure}{\theenumi}
\renewcommand{\thetable}{\theenumi}
%\renewcommand{\theequation}{\theenumi}

%\begin{abstract}
%%\boldmath
%In this letter, an algorithm for evaluating the exact analytical bit error rate  (BER)  for the piecewise linear (PL) combiner for  multiple relays is presented. Previous results were available only for upto three relays. The algorithm is unique in the sense that  the actual mathematical expressions, that are prohibitively large, need not be explicitly obtained. The diversity gain due to multiple relays is shown through plots of the analytical BER, well supported by simulations. 
%
%\end{abstract}
% IEEEtran.cls defaults to using nonbold math in the Abstract.
% This preserves the distinction between vectors and scalars. However,
% if the journal you are submitting to favors bold math in the abstract,
% then you can use LaTeX's standard command \boldmath at the very start
% of the abstract to achieve this. Many IEEE journals frown on math
% in the abstract anyway.

% Note that keywords are not normally used for peerreview papers.
%\begin{IEEEkeywords}
%Cooperative diversity, decode and forward, piecewise linear
%\end{IEEEkeywords}



% For peer review papers, you can put extra information on the cover
% page as needed:
% \ifCLASSOPTIONpeerreview
% \begin{center} \bfseries EDICS Category: 3-BBND \end{center}
% \fi
%
% For peerreview papers, this IEEEtran command inserts a page break and
% creates the second title. It will be ignored for other modes.
%\IEEEpeerreviewmaketitle




\item The probability that a student will pass his examination is 0.73, the probability of
the student getting a compartment is 0.13, and the probability that the student will
either pass or get compartment is 0.96. State True or False.\\
\solution
%\begin{table}[H]
	\centering
\begin{tabular}{|c|c|c|}
\hline
Random variable &Value &Definition\\ \hline
\multirow{3}{*}{X} &0 &Slips of Rs 1\\
&1 &Slips of Rs 5\\
&2 &Slips of Rs 13\\ \hline
\multirow{2}{*}{Y} &0 &Box A\\
&1 &Box B\\\hline
\end{tabular}
\caption{}
\label{tab:Distribution}
\end{table}
See \tabref{tab:Distribution}.
\begin{align}
p_{Y}\brak{k}= \begin{cases} 
      \frac{1}{3} & {k=0} \\
      \frac{2}{3 }& {k=1} 
   \end{cases}
   \\
p_{Y|X}\brak{0|0} = \frac{19}{25}\, 
p_{Y|X}\brak{0|1} = \frac{6}{25}\,
p_{Y|X}\brak{1|0} = \frac{45}{50}\,
p_{Y|X}\brak{1|2} = \frac{5}{50}
\end{align}
The desired probability is the probability that a slip drawn at random is marked other than Rs 1,
\begin{align}
&=1-p_X\brak{0}\\
&= p_X(1) + p_X(2)
\end{align}
Using Bayes theorem,
\begin{align}
&= p_Y\brak{0} \times \pr{Y=0 | X=1} + p_Y\brak{1} \times \pr{Y=1|X=2}\\
&=\frac{1}{3} \times \frac{6}{25} + \frac{2}{3} \times \frac{5}{50}\\
&=\frac{11}{75}
\end{align}

\newpage

%\tableofcontents

\bigskip

\renewcommand{\thefigure}{\theenumi}
\renewcommand{\thetable}{\theenumi}
%\renewcommand{\theequation}{\theenumi}

%\begin{abstract}
%%\boldmath
%In this letter, an algorithm for evaluating the exact analytical bit error rate  (BER)  for the piecewise linear (PL) combiner for  multiple relays is presented. Previous results were available only for upto three relays. The algorithm is unique in the sense that  the actual mathematical expressions, that are prohibitively large, need not be explicitly obtained. The diversity gain due to multiple relays is shown through plots of the analytical BER, well supported by simulations. 
%
%\end{abstract}
% IEEEtran.cls defaults to using nonbold math in the Abstract.
% This preserves the distinction between vectors and scalars. However,
% if the journal you are submitting to favors bold math in the abstract,
% then you can use LaTeX's standard command \boldmath at the very start
% of the abstract to achieve this. Many IEEE journals frown on math
% in the abstract anyway.

% Note that keywords are not normally used for peerreview papers.
%\begin{IEEEkeywords}
%Cooperative diversity, decode and forward, piecewise linear
%\end{IEEEkeywords}



% For peer review papers, you can put extra information on the cover
% page as needed:
% \ifCLASSOPTIONpeerreview
% \begin{center} \bfseries EDICS Category: 3-BBND \end{center}
% \fi
%
% For peerreview papers, this IEEEtran command inserts a page break and
% creates the second title. It will be ignored for other modes.
%\IEEEpeerreviewmaketitle




\item A card is selected from a pack of 52 cards\\
\begin{enumerate}[label=(\alph*)]
\item How many points are there in the sample space?
\item Calculate the probability that the cards is an ace of spades.
\item Calculate the probability that the card is (i) an ace (ii)black card.\\
\end{enumerate}
%\input{ncert/11/16/3/4_1/Prob_4.tex}
\item In a non-leap year, the probability of having 53 tuesdays or 53 wednesdays is\\
\solution
%A non-leap year has a total of 365 days, and a week has 7 days.\\
So it can be expressed as 
\begin{align}
365\text{days} &=52\times 7+1 \text{day}
\end{align}
$\implies$ 52 tuesdays or wednesdays\\
Random variable X denotes the days of a week
\begin{align}
p_X\brak{k}&=\frac{1}{7}; \quad \brak{1<k<7}
\end{align}
So the probability of extra day being tuesday or wednesday is
\begin{align}
p_X\brak{3}+p_X\brak{4}&=\frac{1}{7}+\frac{1}{7}=\frac{2}{7}
\end{align}



\item There are 1000 sealed envelopes in a box, 10 of them contain a cash prize of
Rs 100 each, 100 of them contain a cash prize of Rs 50 each and 200 of them
contain a cash prize of Rs 10 each and rest do not contain any cash prize. If they
are well shuffled and an envelope is picked up out, what is the probability that it
contains no cash prize?\\
\solution
%\begin{table}[H]
	\centering
\begin{tabular}{|c|c|c|}
\hline
Random variable &Value &Definition\\ \hline
\multirow{3}{*}{X} &0 &Slips of Rs 1\\
&1 &Slips of Rs 5\\
&2 &Slips of Rs 13\\ \hline
\multirow{2}{*}{Y} &0 &Box A\\
&1 &Box B\\\hline
\end{tabular}
\caption{}
\label{tab:Distribution}
\end{table}
See \tabref{tab:Distribution}.
\begin{align}
p_{Y}\brak{k}= \begin{cases} 
      \frac{1}{3} & {k=0} \\
      \frac{2}{3 }& {k=1} 
   \end{cases}
   \\
p_{Y|X}\brak{0|0} = \frac{19}{25}\, 
p_{Y|X}\brak{0|1} = \frac{6}{25}\,
p_{Y|X}\brak{1|0} = \frac{45}{50}\,
p_{Y|X}\brak{1|2} = \frac{5}{50}
\end{align}
The desired probability is the probability that a slip drawn at random is marked other than Rs 1,
\begin{align}
&=1-p_X\brak{0}\\
&= p_X(1) + p_X(2)
\end{align}
Using Bayes theorem,
\begin{align}
&= p_Y\brak{0} \times \pr{Y=0 | X=1} + p_Y\brak{1} \times \pr{Y=1|X=2}\\
&=\frac{1}{3} \times \frac{6}{25} + \frac{2}{3} \times \frac{5}{50}\\
&=\frac{11}{75}
\end{align}

\newpage

%\tableofcontents

\bigskip

\renewcommand{\thefigure}{\theenumi}
\renewcommand{\thetable}{\theenumi}
%\renewcommand{\theequation}{\theenumi}

%\begin{abstract}
%%\boldmath
%In this letter, an algorithm for evaluating the exact analytical bit error rate  (BER)  for the piecewise linear (PL) combiner for  multiple relays is presented. Previous results were available only for upto three relays. The algorithm is unique in the sense that  the actual mathematical expressions, that are prohibitively large, need not be explicitly obtained. The diversity gain due to multiple relays is shown through plots of the analytical BER, well supported by simulations. 
%
%\end{abstract}
% IEEEtran.cls defaults to using nonbold math in the Abstract.
% This preserves the distinction between vectors and scalars. However,
% if the journal you are submitting to favors bold math in the abstract,
% then you can use LaTeX's standard command \boldmath at the very start
% of the abstract to achieve this. Many IEEE journals frown on math
% in the abstract anyway.

% Note that keywords are not normally used for peerreview papers.
%\begin{IEEEkeywords}
%Cooperative diversity, decode and forward, piecewise linear
%\end{IEEEkeywords}



% For peer review papers, you can put extra information on the cover
% page as needed:
% \ifCLASSOPTIONpeerreview
% \begin{center} \bfseries EDICS Category: 3-BBND \end{center}
% \fi
%
% For peerreview papers, this IEEEtran command inserts a page break and
% creates the second title. It will be ignored for other modes.
%\IEEEpeerreviewmaketitle




\item 
A die is thrown and a card is selected at random from a deck of 52 playing cards. The probability of getting an even number on the die and a spade card.\\
\solution
%\begin{table}[H]
	\centering
\begin{tabular}{|c|c|c|}
\hline
Random variable &Value &Definition\\ \hline
\multirow{3}{*}{X} &0 &Slips of Rs 1\\
&1 &Slips of Rs 5\\
&2 &Slips of Rs 13\\ \hline
\multirow{2}{*}{Y} &0 &Box A\\
&1 &Box B\\\hline
\end{tabular}
\caption{}
\label{tab:Distribution}
\end{table}
See \tabref{tab:Distribution}.
\begin{align}
p_{Y}\brak{k}= \begin{cases} 
      \frac{1}{3} & {k=0} \\
      \frac{2}{3 }& {k=1} 
   \end{cases}
   \\
p_{Y|X}\brak{0|0} = \frac{19}{25}\, 
p_{Y|X}\brak{0|1} = \frac{6}{25}\,
p_{Y|X}\brak{1|0} = \frac{45}{50}\,
p_{Y|X}\brak{1|2} = \frac{5}{50}
\end{align}
The desired probability is the probability that a slip drawn at random is marked other than Rs 1,
\begin{align}
&=1-p_X\brak{0}\\
&= p_X(1) + p_X(2)
\end{align}
Using Bayes theorem,
\begin{align}
&= p_Y\brak{0} \times \pr{Y=0 | X=1} + p_Y\brak{1} \times \pr{Y=1|X=2}\\
&=\frac{1}{3} \times \frac{6}{25} + \frac{2}{3} \times \frac{5}{50}\\
&=\frac{11}{75}
\end{align}

\newpage

%\tableofcontents

\bigskip

\renewcommand{\thefigure}{\theenumi}
\renewcommand{\thetable}{\theenumi}
%\renewcommand{\theequation}{\theenumi}

%\begin{abstract}
%%\boldmath
%In this letter, an algorithm for evaluating the exact analytical bit error rate  (BER)  for the piecewise linear (PL) combiner for  multiple relays is presented. Previous results were available only for upto three relays. The algorithm is unique in the sense that  the actual mathematical expressions, that are prohibitively large, need not be explicitly obtained. The diversity gain due to multiple relays is shown through plots of the analytical BER, well supported by simulations. 
%
%\end{abstract}
% IEEEtran.cls defaults to using nonbold math in the Abstract.
% This preserves the distinction between vectors and scalars. However,
% if the journal you are submitting to favors bold math in the abstract,
% then you can use LaTeX's standard command \boldmath at the very start
% of the abstract to achieve this. Many IEEE journals frown on math
% in the abstract anyway.

% Note that keywords are not normally used for peerreview papers.
%\begin{IEEEkeywords}
%Cooperative diversity, decode and forward, piecewise linear
%\end{IEEEkeywords}



% For peer review papers, you can put extra information on the cover
% page as needed:
% \ifCLASSOPTIONpeerreview
% \begin{center} \bfseries EDICS Category: 3-BBND \end{center}
% \fi
%
% For peerreview papers, this IEEEtran command inserts a page break and
% creates the second title. It will be ignored for other modes.
%\IEEEpeerreviewmaketitle




\item
If 4-digit numbers greater than 5,000 are randomly formed from the digits 0, 1, 3, 5, and 7, what is the probability of forming a number divisible by 5 when:
\begin{enumerate}
    \item The digits are repeated?
    \item The repetition of digits is not allowed?
\end{enumerate}
\solution
%\begin{table}[H]
	\centering
\begin{tabular}{|c|c|c|}
\hline
Random variable &Value &Definition\\ \hline
\multirow{3}{*}{X} &0 &Slips of Rs 1\\
&1 &Slips of Rs 5\\
&2 &Slips of Rs 13\\ \hline
\multirow{2}{*}{Y} &0 &Box A\\
&1 &Box B\\\hline
\end{tabular}
\caption{}
\label{tab:Distribution}
\end{table}
See \tabref{tab:Distribution}.
\begin{align}
p_{Y}\brak{k}= \begin{cases} 
      \frac{1}{3} & {k=0} \\
      \frac{2}{3 }& {k=1} 
   \end{cases}
   \\
p_{Y|X}\brak{0|0} = \frac{19}{25}\, 
p_{Y|X}\brak{0|1} = \frac{6}{25}\,
p_{Y|X}\brak{1|0} = \frac{45}{50}\,
p_{Y|X}\brak{1|2} = \frac{5}{50}
\end{align}
The desired probability is the probability that a slip drawn at random is marked other than Rs 1,
\begin{align}
&=1-p_X\brak{0}\\
&= p_X(1) + p_X(2)
\end{align}
Using Bayes theorem,
\begin{align}
&= p_Y\brak{0} \times \pr{Y=0 | X=1} + p_Y\brak{1} \times \pr{Y=1|X=2}\\
&=\frac{1}{3} \times \frac{6}{25} + \frac{2}{3} \times \frac{5}{50}\\
&=\frac{11}{75}
\end{align}

\newpage

%\tableofcontents

\bigskip

\renewcommand{\thefigure}{\theenumi}
\renewcommand{\thetable}{\theenumi}
%\renewcommand{\theequation}{\theenumi}

%\begin{abstract}
%%\boldmath
%In this letter, an algorithm for evaluating the exact analytical bit error rate  (BER)  for the piecewise linear (PL) combiner for  multiple relays is presented. Previous results were available only for upto three relays. The algorithm is unique in the sense that  the actual mathematical expressions, that are prohibitively large, need not be explicitly obtained. The diversity gain due to multiple relays is shown through plots of the analytical BER, well supported by simulations. 
%
%\end{abstract}
% IEEEtran.cls defaults to using nonbold math in the Abstract.
% This preserves the distinction between vectors and scalars. However,
% if the journal you are submitting to favors bold math in the abstract,
% then you can use LaTeX's standard command \boldmath at the very start
% of the abstract to achieve this. Many IEEE journals frown on math
% in the abstract anyway.

% Note that keywords are not normally used for peerreview papers.
%\begin{IEEEkeywords}
%Cooperative diversity, decode and forward, piecewise linear
%\end{IEEEkeywords}



% For peer review papers, you can put extra information on the cover
% page as needed:
% \ifCLASSOPTIONpeerreview
% \begin{center} \bfseries EDICS Category: 3-BBND \end{center}
% \fi
%
% For peerreview papers, this IEEEtran command inserts a page break and
% creates the second title. It will be ignored for other modes.
%\IEEEpeerreviewmaketitle




\item Consider the probability space $\brak{\Omega, \mathcal{G}, P}$ where $\Omega = [0,2]$ and $\mathcal{G} = \cbrak{\phi, \Omega, [0,1], (1,2]}$. Let $X$ and $Y$ be two functions on $\Omega$ defined as
\begin{align*}
    X(\omega) = 
    \begin{cases}
        1 & \text{if }\omega \in [0, 1]\\
        2 & \text{if }\omega \in (1, 2]
    \end{cases}
\end{align*}
and
\begin{align*}
    Y(\omega) = 
    \begin{cases}
        2 & \text{if }\omega \in [0, 1.5]\\
        3 & \text{if }\omega \in (1.5, 2].
    \end{cases}
\end{align*}
Then which one of the following statements is true?
\begin{enumerate}
    \item [(A)] $X$ is a random variable with respect to $\mathcal{G}$, but $Y$ is not a random variable with respect to $\mathcal{G}$.
    \item [(B)] $Y$ is a random variable with respect to $\mathcal{G}$, but $X$ is not a random variable with respect to $\mathcal{G}$.
    \item [(C)] Neither $X$ nor $Y$ is a random variable with respect to $\mathcal{G}$.
    \item [(D)] Both $X$ and $Y$ are random variables with respect to $\mathcal{G}$.
\end{enumerate} \hfill (GATE ST 2023)\\
\solution
%\begin{table}[H]
	\centering
\begin{tabular}{|c|c|c|}
\hline
Random variable &Value &Definition\\ \hline
\multirow{3}{*}{X} &0 &Slips of Rs 1\\
&1 &Slips of Rs 5\\
&2 &Slips of Rs 13\\ \hline
\multirow{2}{*}{Y} &0 &Box A\\
&1 &Box B\\\hline
\end{tabular}
\caption{}
\label{tab:Distribution}
\end{table}
See \tabref{tab:Distribution}.
\begin{align}
p_{Y}\brak{k}= \begin{cases} 
      \frac{1}{3} & {k=0} \\
      \frac{2}{3 }& {k=1} 
   \end{cases}
   \\
p_{Y|X}\brak{0|0} = \frac{19}{25}\, 
p_{Y|X}\brak{0|1} = \frac{6}{25}\,
p_{Y|X}\brak{1|0} = \frac{45}{50}\,
p_{Y|X}\brak{1|2} = \frac{5}{50}
\end{align}
The desired probability is the probability that a slip drawn at random is marked other than Rs 1,
\begin{align}
&=1-p_X\brak{0}\\
&= p_X(1) + p_X(2)
\end{align}
Using Bayes theorem,
\begin{align}
&= p_Y\brak{0} \times \pr{Y=0 | X=1} + p_Y\brak{1} \times \pr{Y=1|X=2}\\
&=\frac{1}{3} \times \frac{6}{25} + \frac{2}{3} \times \frac{5}{50}\\
&=\frac{11}{75}
\end{align}

\newpage

%\tableofcontents

\bigskip

\renewcommand{\thefigure}{\theenumi}
\renewcommand{\thetable}{\theenumi}
%\renewcommand{\theequation}{\theenumi}

%\begin{abstract}
%%\boldmath
%In this letter, an algorithm for evaluating the exact analytical bit error rate  (BER)  for the piecewise linear (PL) combiner for  multiple relays is presented. Previous results were available only for upto three relays. The algorithm is unique in the sense that  the actual mathematical expressions, that are prohibitively large, need not be explicitly obtained. The diversity gain due to multiple relays is shown through plots of the analytical BER, well supported by simulations. 
%
%\end{abstract}
% IEEEtran.cls defaults to using nonbold math in the Abstract.
% This preserves the distinction between vectors and scalars. However,
% if the journal you are submitting to favors bold math in the abstract,
% then you can use LaTeX's standard command \boldmath at the very start
% of the abstract to achieve this. Many IEEE journals frown on math
% in the abstract anyway.

% Note that keywords are not normally used for peerreview papers.
%\begin{IEEEkeywords}
%Cooperative diversity, decode and forward, piecewise linear
%\end{IEEEkeywords}



% For peer review papers, you can put extra information on the cover
% page as needed:
% \ifCLASSOPTIONpeerreview
% \begin{center} \bfseries EDICS Category: 3-BBND \end{center}
% \fi
%
% For peerreview papers, this IEEEtran command inserts a page break and
% creates the second title. It will be ignored for other modes.
%\IEEEpeerreviewmaketitle




	\item  A die is loaded in such a way that each odd number is twice as likely to occur as
each even number. Find $P(G)$, where $G$ is the event that a number greater than
3 occurs on a single roll of the die.
\\
\solution
		%\begin{table}[H]
	\centering
\begin{tabular}{|c|c|c|}
\hline
Random variable &Value &Definition\\ \hline
\multirow{3}{*}{X} &0 &Slips of Rs 1\\
&1 &Slips of Rs 5\\
&2 &Slips of Rs 13\\ \hline
\multirow{2}{*}{Y} &0 &Box A\\
&1 &Box B\\\hline
\end{tabular}
\caption{}
\label{tab:Distribution}
\end{table}
See \tabref{tab:Distribution}.
\begin{align}
p_{Y}\brak{k}= \begin{cases} 
      \frac{1}{3} & {k=0} \\
      \frac{2}{3 }& {k=1} 
   \end{cases}
   \\
p_{Y|X}\brak{0|0} = \frac{19}{25}\, 
p_{Y|X}\brak{0|1} = \frac{6}{25}\,
p_{Y|X}\brak{1|0} = \frac{45}{50}\,
p_{Y|X}\brak{1|2} = \frac{5}{50}
\end{align}
The desired probability is the probability that a slip drawn at random is marked other than Rs 1,
\begin{align}
&=1-p_X\brak{0}\\
&= p_X(1) + p_X(2)
\end{align}
Using Bayes theorem,
\begin{align}
&= p_Y\brak{0} \times \pr{Y=0 | X=1} + p_Y\brak{1} \times \pr{Y=1|X=2}\\
&=\frac{1}{3} \times \frac{6}{25} + \frac{2}{3} \times \frac{5}{50}\\
&=\frac{11}{75}
\end{align}

\newpage

%\tableofcontents

\bigskip

\renewcommand{\thefigure}{\theenumi}
\renewcommand{\thetable}{\theenumi}
%\renewcommand{\theequation}{\theenumi}

%\begin{abstract}
%%\boldmath
%In this letter, an algorithm for evaluating the exact analytical bit error rate  (BER)  for the piecewise linear (PL) combiner for  multiple relays is presented. Previous results were available only for upto three relays. The algorithm is unique in the sense that  the actual mathematical expressions, that are prohibitively large, need not be explicitly obtained. The diversity gain due to multiple relays is shown through plots of the analytical BER, well supported by simulations. 
%
%\end{abstract}
% IEEEtran.cls defaults to using nonbold math in the Abstract.
% This preserves the distinction between vectors and scalars. However,
% if the journal you are submitting to favors bold math in the abstract,
% then you can use LaTeX's standard command \boldmath at the very start
% of the abstract to achieve this. Many IEEE journals frown on math
% in the abstract anyway.

% Note that keywords are not normally used for peerreview papers.
%\begin{IEEEkeywords}
%Cooperative diversity, decode and forward, piecewise linear
%\end{IEEEkeywords}



% For peer review papers, you can put extra information on the cover
% page as needed:
% \ifCLASSOPTIONpeerreview
% \begin{center} \bfseries EDICS Category: 3-BBND \end{center}
% \fi
%
% For peerreview papers, this IEEEtran command inserts a page break and
% creates the second title. It will be ignored for other modes.
%\IEEEpeerreviewmaketitle




	\item All the jacks, queens and kings are removed from a deck of 52 playing cards. The remaining cards are well shuffled and then one card is drawn at random. Giving ace a value 1 similar value for other cards, find the probability that the card has a value 
		\begin{enumerate}
			\item 7
			\item greater than 7
			\item less than 7
		\end{enumerate}
		%Number of cards left after removing all jacks, queens and kings 
\begin{align}
N	= 52 - 4\times 3
	= 40
\end{align}
%\begin{table}[H]
%\def\arraystretch{1.2}
%\begin{tabular}{|c|c|c|}
%\hline
%	\textbf{Parameter} &\textbf{Value} &\textbf{Description}\\ \hline
%	$X$ &1-10 &Represents the value of the card picked \\ \hline
%\end{tabular}
%\end{table}
Let $1 \le X \le 10$ be the value of the card picked.  Then,
\begin{align}
	p_X(k) &= \Pr(X=k)\ \forall\ 1 \leq k \leq 10\\
	&= \frac{4\times 1}{40}\\
	&= \frac{1}{10}\\
	\therefore p_X(k) &= 
	\begin{cases}
		\frac{1}{10} & 1 \leq k \leq 10\\
		0 & \text{otherwise}
	\end{cases}
\end{align}
and
\begin{align}
	F_{X}(k) &= \sum_{m=0}^{k}p_{X}(m) \quad 1 \leq k \leq 10\\
	&= \frac{k}{10}\\
	\therefore F_{X}(k) &= 
	\begin{cases}
		0 & k \leq 0\\
		\frac{k}{10} & 1\leq k \leq 10\\
		1 & k > 10 
	\end{cases}
\end{align}
\begin{enumerate}
	\item Probability that card has value equal to 7 is
		\begin{align}
			 p_{X}(7)
			= \frac{1}{10}
		\end{align}
	\item Probability that card has value greater than 7 is
		\begin{align}
			1 - F_X(7)
			&= 1 - \frac{7}{10}
			\\
			&= \frac{3}{10}
		\end{align}
	\item Probability that card has value less than 7 is
		\begin{align}
			 F_{X}(6)
			=\frac{6}{10}
		\end{align}
\end{enumerate}

  \item A Lot consists of 48 mobile phones of which 42 are good, 3 have only minor defects and 3 have major defects.Varnika will buy a phone if it is good but the trader will only buy a mobile if it has no major defects. One phone is selected at random from the lot. What is the probability that it is
\begin{enumerate}
	\item acceptable to Varnika?
            \item acceptable to the trader?
\end{enumerate}
\solution
	%\begin{table}[H]
	\centering
\begin{tabular}{|c|c|c|}
\hline
Random variable &Value &Definition\\ \hline
\multirow{3}{*}{X} &0 &Slips of Rs 1\\
&1 &Slips of Rs 5\\
&2 &Slips of Rs 13\\ \hline
\multirow{2}{*}{Y} &0 &Box A\\
&1 &Box B\\\hline
\end{tabular}
\caption{}
\label{tab:Distribution}
\end{table}
See \tabref{tab:Distribution}.
\begin{align}
p_{Y}\brak{k}= \begin{cases} 
      \frac{1}{3} & {k=0} \\
      \frac{2}{3 }& {k=1} 
   \end{cases}
   \\
p_{Y|X}\brak{0|0} = \frac{19}{25}\, 
p_{Y|X}\brak{0|1} = \frac{6}{25}\,
p_{Y|X}\brak{1|0} = \frac{45}{50}\,
p_{Y|X}\brak{1|2} = \frac{5}{50}
\end{align}
The desired probability is the probability that a slip drawn at random is marked other than Rs 1,
\begin{align}
&=1-p_X\brak{0}\\
&= p_X(1) + p_X(2)
\end{align}
Using Bayes theorem,
\begin{align}
&= p_Y\brak{0} \times \pr{Y=0 | X=1} + p_Y\brak{1} \times \pr{Y=1|X=2}\\
&=\frac{1}{3} \times \frac{6}{25} + \frac{2}{3} \times \frac{5}{50}\\
&=\frac{11}{75}
\end{align}

\newpage

%\tableofcontents

\bigskip

\renewcommand{\thefigure}{\theenumi}
\renewcommand{\thetable}{\theenumi}
%\renewcommand{\theequation}{\theenumi}

%\begin{abstract}
%%\boldmath
%In this letter, an algorithm for evaluating the exact analytical bit error rate  (BER)  for the piecewise linear (PL) combiner for  multiple relays is presented. Previous results were available only for upto three relays. The algorithm is unique in the sense that  the actual mathematical expressions, that are prohibitively large, need not be explicitly obtained. The diversity gain due to multiple relays is shown through plots of the analytical BER, well supported by simulations. 
%
%\end{abstract}
% IEEEtran.cls defaults to using nonbold math in the Abstract.
% This preserves the distinction between vectors and scalars. However,
% if the journal you are submitting to favors bold math in the abstract,
% then you can use LaTeX's standard command \boldmath at the very start
% of the abstract to achieve this. Many IEEE journals frown on math
% in the abstract anyway.

% Note that keywords are not normally used for peerreview papers.
%\begin{IEEEkeywords}
%Cooperative diversity, decode and forward, piecewise linear
%\end{IEEEkeywords}



% For peer review papers, you can put extra information on the cover
% page as needed:
% \ifCLASSOPTIONpeerreview
% \begin{center} \bfseries EDICS Category: 3-BBND \end{center}
% \fi
%
% For peerreview papers, this IEEEtran command inserts a page break and
% creates the second title. It will be ignored for other modes.
%\IEEEpeerreviewmaketitle




 \item A student says that if you throw a die, it will show up 1 or not 1. Therefore, the probability of getting 1 and the probability of getting 'not 1' each is equal to $\frac{1}{2}$. Is this correct? Give reasons.\\
 \solution
        %\begin{table}[H]
	\centering
\begin{tabular}{|c|c|c|}
\hline
Random variable &Value &Definition\\ \hline
\multirow{3}{*}{X} &0 &Slips of Rs 1\\
&1 &Slips of Rs 5\\
&2 &Slips of Rs 13\\ \hline
\multirow{2}{*}{Y} &0 &Box A\\
&1 &Box B\\\hline
\end{tabular}
\caption{}
\label{tab:Distribution}
\end{table}
See \tabref{tab:Distribution}.
\begin{align}
p_{Y}\brak{k}= \begin{cases} 
      \frac{1}{3} & {k=0} \\
      \frac{2}{3 }& {k=1} 
   \end{cases}
   \\
p_{Y|X}\brak{0|0} = \frac{19}{25}\, 
p_{Y|X}\brak{0|1} = \frac{6}{25}\,
p_{Y|X}\brak{1|0} = \frac{45}{50}\,
p_{Y|X}\brak{1|2} = \frac{5}{50}
\end{align}
The desired probability is the probability that a slip drawn at random is marked other than Rs 1,
\begin{align}
&=1-p_X\brak{0}\\
&= p_X(1) + p_X(2)
\end{align}
Using Bayes theorem,
\begin{align}
&= p_Y\brak{0} \times \pr{Y=0 | X=1} + p_Y\brak{1} \times \pr{Y=1|X=2}\\
&=\frac{1}{3} \times \frac{6}{25} + \frac{2}{3} \times \frac{5}{50}\\
&=\frac{11}{75}
\end{align}

\newpage

%\tableofcontents

\bigskip

\renewcommand{\thefigure}{\theenumi}
\renewcommand{\thetable}{\theenumi}
%\renewcommand{\theequation}{\theenumi}

%\begin{abstract}
%%\boldmath
%In this letter, an algorithm for evaluating the exact analytical bit error rate  (BER)  for the piecewise linear (PL) combiner for  multiple relays is presented. Previous results were available only for upto three relays. The algorithm is unique in the sense that  the actual mathematical expressions, that are prohibitively large, need not be explicitly obtained. The diversity gain due to multiple relays is shown through plots of the analytical BER, well supported by simulations. 
%
%\end{abstract}
% IEEEtran.cls defaults to using nonbold math in the Abstract.
% This preserves the distinction between vectors and scalars. However,
% if the journal you are submitting to favors bold math in the abstract,
% then you can use LaTeX's standard command \boldmath at the very start
% of the abstract to achieve this. Many IEEE journals frown on math
% in the abstract anyway.

% Note that keywords are not normally used for peerreview papers.
%\begin{IEEEkeywords}
%Cooperative diversity, decode and forward, piecewise linear
%\end{IEEEkeywords}



% For peer review papers, you can put extra information on the cover
% page as needed:
% \ifCLASSOPTIONpeerreview
% \begin{center} \bfseries EDICS Category: 3-BBND \end{center}
% \fi
%
% For peerreview papers, this IEEEtran command inserts a page break and
% creates the second title. It will be ignored for other modes.
%\IEEEpeerreviewmaketitle




   \item Four candidates A, B, C, D have ap-
plied for the assignment to coach a school cricket
team. If A is twice as likely to be selected as B, and
B and C are given about the same chance of being
selected, while C is twice as likely to be selected
as D, what are the probabilities that
\begin{enumerate}
\item C will be selected?
\item A will not be selected?
\end{enumerate}
	%\begin{table}[H]
	\centering
\begin{tabular}{|c|c|c|}
\hline
Random variable &Value &Definition\\ \hline
\multirow{3}{*}{X} &0 &Slips of Rs 1\\
&1 &Slips of Rs 5\\
&2 &Slips of Rs 13\\ \hline
\multirow{2}{*}{Y} &0 &Box A\\
&1 &Box B\\\hline
\end{tabular}
\caption{}
\label{tab:Distribution}
\end{table}
See \tabref{tab:Distribution}.
\begin{align}
p_{Y}\brak{k}= \begin{cases} 
      \frac{1}{3} & {k=0} \\
      \frac{2}{3 }& {k=1} 
   \end{cases}
   \\
p_{Y|X}\brak{0|0} = \frac{19}{25}\, 
p_{Y|X}\brak{0|1} = \frac{6}{25}\,
p_{Y|X}\brak{1|0} = \frac{45}{50}\,
p_{Y|X}\brak{1|2} = \frac{5}{50}
\end{align}
The desired probability is the probability that a slip drawn at random is marked other than Rs 1,
\begin{align}
&=1-p_X\brak{0}\\
&= p_X(1) + p_X(2)
\end{align}
Using Bayes theorem,
\begin{align}
&= p_Y\brak{0} \times \pr{Y=0 | X=1} + p_Y\brak{1} \times \pr{Y=1|X=2}\\
&=\frac{1}{3} \times \frac{6}{25} + \frac{2}{3} \times \frac{5}{50}\\
&=\frac{11}{75}
\end{align}

\newpage

%\tableofcontents

\bigskip

\renewcommand{\thefigure}{\theenumi}
\renewcommand{\thetable}{\theenumi}
%\renewcommand{\theequation}{\theenumi}

%\begin{abstract}
%%\boldmath
%In this letter, an algorithm for evaluating the exact analytical bit error rate  (BER)  for the piecewise linear (PL) combiner for  multiple relays is presented. Previous results were available only for upto three relays. The algorithm is unique in the sense that  the actual mathematical expressions, that are prohibitively large, need not be explicitly obtained. The diversity gain due to multiple relays is shown through plots of the analytical BER, well supported by simulations. 
%
%\end{abstract}
% IEEEtran.cls defaults to using nonbold math in the Abstract.
% This preserves the distinction between vectors and scalars. However,
% if the journal you are submitting to favors bold math in the abstract,
% then you can use LaTeX's standard command \boldmath at the very start
% of the abstract to achieve this. Many IEEE journals frown on math
% in the abstract anyway.

% Note that keywords are not normally used for peerreview papers.
%\begin{IEEEkeywords}
%Cooperative diversity, decode and forward, piecewise linear
%\end{IEEEkeywords}



% For peer review papers, you can put extra information on the cover
% page as needed:
% \ifCLASSOPTIONpeerreview
% \begin{center} \bfseries EDICS Category: 3-BBND \end{center}
% \fi
%
% For peerreview papers, this IEEEtran command inserts a page break and
% creates the second title. It will be ignored for other modes.
%\IEEEpeerreviewmaketitle




 \item A bag contain 24 balls of which $x$ balls are red, $2x$ are white and $3x$ are blue. A ball is selected at random, What is the probability that it is
\begin{enumerate}[label=\alph*)]
\item not red ?
\item white ?
\end{enumerate}
%\begin{table}[H]
	\centering
\begin{tabular}{|c|c|c|}
\hline
Random variable &Value &Definition\\ \hline
\multirow{3}{*}{X} &0 &Slips of Rs 1\\
&1 &Slips of Rs 5\\
&2 &Slips of Rs 13\\ \hline
\multirow{2}{*}{Y} &0 &Box A\\
&1 &Box B\\\hline
\end{tabular}
\caption{}
\label{tab:Distribution}
\end{table}
See \tabref{tab:Distribution}.
\begin{align}
p_{Y}\brak{k}= \begin{cases} 
      \frac{1}{3} & {k=0} \\
      \frac{2}{3 }& {k=1} 
   \end{cases}
   \\
p_{Y|X}\brak{0|0} = \frac{19}{25}\, 
p_{Y|X}\brak{0|1} = \frac{6}{25}\,
p_{Y|X}\brak{1|0} = \frac{45}{50}\,
p_{Y|X}\brak{1|2} = \frac{5}{50}
\end{align}
The desired probability is the probability that a slip drawn at random is marked other than Rs 1,
\begin{align}
&=1-p_X\brak{0}\\
&= p_X(1) + p_X(2)
\end{align}
Using Bayes theorem,
\begin{align}
&= p_Y\brak{0} \times \pr{Y=0 | X=1} + p_Y\brak{1} \times \pr{Y=1|X=2}\\
&=\frac{1}{3} \times \frac{6}{25} + \frac{2}{3} \times \frac{5}{50}\\
&=\frac{11}{75}
\end{align}

\newpage

%\tableofcontents

\bigskip

\renewcommand{\thefigure}{\theenumi}
\renewcommand{\thetable}{\theenumi}
%\renewcommand{\theequation}{\theenumi}

%\begin{abstract}
%%\boldmath
%In this letter, an algorithm for evaluating the exact analytical bit error rate  (BER)  for the piecewise linear (PL) combiner for  multiple relays is presented. Previous results were available only for upto three relays. The algorithm is unique in the sense that  the actual mathematical expressions, that are prohibitively large, need not be explicitly obtained. The diversity gain due to multiple relays is shown through plots of the analytical BER, well supported by simulations. 
%
%\end{abstract}
% IEEEtran.cls defaults to using nonbold math in the Abstract.
% This preserves the distinction between vectors and scalars. However,
% if the journal you are submitting to favors bold math in the abstract,
% then you can use LaTeX's standard command \boldmath at the very start
% of the abstract to achieve this. Many IEEE journals frown on math
% in the abstract anyway.

% Note that keywords are not normally used for peerreview papers.
%\begin{IEEEkeywords}
%Cooperative diversity, decode and forward, piecewise linear
%\end{IEEEkeywords}



% For peer review papers, you can put extra information on the cover
% page as needed:
% \ifCLASSOPTIONpeerreview
% \begin{center} \bfseries EDICS Category: 3-BBND \end{center}
% \fi
%
% For peerreview papers, this IEEEtran command inserts a page break and
% creates the second title. It will be ignored for other modes.
%\IEEEpeerreviewmaketitle




If the letters of the word ASSASSINATION are arranged at random. Find the Probability that
\begin{enumerate}[label=(\alph*)]
\item Four $S's$ come consecutively in the word
\item Two  $I's$ and two $N's$ come together
\item All $A's$ are not coming together
\item No two $A's$ are coming together
\end{enumerate}
%\begin{table}[H]
	\centering
\begin{tabular}{|c|c|c|}
\hline
Random variable &Value &Definition\\ \hline
\multirow{3}{*}{X} &0 &Slips of Rs 1\\
&1 &Slips of Rs 5\\
&2 &Slips of Rs 13\\ \hline
\multirow{2}{*}{Y} &0 &Box A\\
&1 &Box B\\\hline
\end{tabular}
\caption{}
\label{tab:Distribution}
\end{table}
See \tabref{tab:Distribution}.
\begin{align}
p_{Y}\brak{k}= \begin{cases} 
      \frac{1}{3} & {k=0} \\
      \frac{2}{3 }& {k=1} 
   \end{cases}
   \\
p_{Y|X}\brak{0|0} = \frac{19}{25}\, 
p_{Y|X}\brak{0|1} = \frac{6}{25}\,
p_{Y|X}\brak{1|0} = \frac{45}{50}\,
p_{Y|X}\brak{1|2} = \frac{5}{50}
\end{align}
The desired probability is the probability that a slip drawn at random is marked other than Rs 1,
\begin{align}
&=1-p_X\brak{0}\\
&= p_X(1) + p_X(2)
\end{align}
Using Bayes theorem,
\begin{align}
&= p_Y\brak{0} \times \pr{Y=0 | X=1} + p_Y\brak{1} \times \pr{Y=1|X=2}\\
&=\frac{1}{3} \times \frac{6}{25} + \frac{2}{3} \times \frac{5}{50}\\
&=\frac{11}{75}
\end{align}

\newpage

%\tableofcontents

\bigskip

\renewcommand{\thefigure}{\theenumi}
\renewcommand{\thetable}{\theenumi}
%\renewcommand{\theequation}{\theenumi}

%\begin{abstract}
%%\boldmath
%In this letter, an algorithm for evaluating the exact analytical bit error rate  (BER)  for the piecewise linear (PL) combiner for  multiple relays is presented. Previous results were available only for upto three relays. The algorithm is unique in the sense that  the actual mathematical expressions, that are prohibitively large, need not be explicitly obtained. The diversity gain due to multiple relays is shown through plots of the analytical BER, well supported by simulations. 
%
%\end{abstract}
% IEEEtran.cls defaults to using nonbold math in the Abstract.
% This preserves the distinction between vectors and scalars. However,
% if the journal you are submitting to favors bold math in the abstract,
% then you can use LaTeX's standard command \boldmath at the very start
% of the abstract to achieve this. Many IEEE journals frown on math
% in the abstract anyway.

% Note that keywords are not normally used for peerreview papers.
%\begin{IEEEkeywords}
%Cooperative diversity, decode and forward, piecewise linear
%\end{IEEEkeywords}



% For peer review papers, you can put extra information on the cover
% page as needed:
% \ifCLASSOPTIONpeerreview
% \begin{center} \bfseries EDICS Category: 3-BBND \end{center}
% \fi
%
% For peerreview papers, this IEEEtran command inserts a page break and
% creates the second title. It will be ignored for other modes.
%\IEEEpeerreviewmaketitle




	\item One urn contains two black balls (labelled B1 and B2) and one white ball. A
	second urn contains one black ball and two white balls (labelled W1 and W2).
	Suppose the following experiment is performed. One of the two urns is chosen
	at random. Next a ball is randomly chosen from the urn. Then a second ball is
	chosen at random from the same urn without replacing the first ball.
	
	\begin{enumerate}
	\item What is the probability that two black balls are chosen?
	
	\item What is the probability that two balls of opposite colour are chosen?
	\end{enumerate}
	\solution
	%\begin{align}
    \label{eq:12.13.6.18.1}
	\because	\pr{A|B} &> \pr{A},\
\frac{\pr{AB}}{\pr{B}} > \pr{A}
\\
    \label{eq:12.13.6.18.2}
	\implies \pr{AB} &> \pr{A}\pr{B}
	\\
	\text{or, } \frac{\pr{AB}}{\pr{A}} &=\pr{B|A} > \pr{A}
\end{align}

\end{enumerate}

		%
\item 
Out of 100 students, two sections of 40 and 60 are formed. If you and your friend are among the 100 students, what is the probability that
\begin{enumerate}
\item you both enter the same section?
\item you both enter the different sections?
\end{enumerate}
\solution
		%\begin{enumerate}[label=\thesection.\arabic*,ref=\thesection.\theenumi]
	\item One card is drawn from a well-shuffled deck of 52 cards. Find the probability of getting
\begin{enumerate}
\item A king of red colour 
\item A face card 
\item A red face card
\item The jack of hearts
\item A spade
\item The queen of diamonds

\end{enumerate}
\solution
		%\begin{table}[H]
	\centering
\begin{tabular}{|c|c|c|}
\hline
Random variable &Value &Definition\\ \hline
\multirow{3}{*}{X} &0 &Slips of Rs 1\\
&1 &Slips of Rs 5\\
&2 &Slips of Rs 13\\ \hline
\multirow{2}{*}{Y} &0 &Box A\\
&1 &Box B\\\hline
\end{tabular}
\caption{}
\label{tab:Distribution}
\end{table}
See \tabref{tab:Distribution}.
\begin{align}
p_{Y}\brak{k}= \begin{cases} 
      \frac{1}{3} & {k=0} \\
      \frac{2}{3 }& {k=1} 
   \end{cases}
   \\
p_{Y|X}\brak{0|0} = \frac{19}{25}\, 
p_{Y|X}\brak{0|1} = \frac{6}{25}\,
p_{Y|X}\brak{1|0} = \frac{45}{50}\,
p_{Y|X}\brak{1|2} = \frac{5}{50}
\end{align}
The desired probability is the probability that a slip drawn at random is marked other than Rs 1,
\begin{align}
&=1-p_X\brak{0}\\
&= p_X(1) + p_X(2)
\end{align}
Using Bayes theorem,
\begin{align}
&= p_Y\brak{0} \times \pr{Y=0 | X=1} + p_Y\brak{1} \times \pr{Y=1|X=2}\\
&=\frac{1}{3} \times \frac{6}{25} + \frac{2}{3} \times \frac{5}{50}\\
&=\frac{11}{75}
\end{align}

\newpage

%\tableofcontents

\bigskip

\renewcommand{\thefigure}{\theenumi}
\renewcommand{\thetable}{\theenumi}
%\renewcommand{\theequation}{\theenumi}

%\begin{abstract}
%%\boldmath
%In this letter, an algorithm for evaluating the exact analytical bit error rate  (BER)  for the piecewise linear (PL) combiner for  multiple relays is presented. Previous results were available only for upto three relays. The algorithm is unique in the sense that  the actual mathematical expressions, that are prohibitively large, need not be explicitly obtained. The diversity gain due to multiple relays is shown through plots of the analytical BER, well supported by simulations. 
%
%\end{abstract}
% IEEEtran.cls defaults to using nonbold math in the Abstract.
% This preserves the distinction between vectors and scalars. However,
% if the journal you are submitting to favors bold math in the abstract,
% then you can use LaTeX's standard command \boldmath at the very start
% of the abstract to achieve this. Many IEEE journals frown on math
% in the abstract anyway.

% Note that keywords are not normally used for peerreview papers.
%\begin{IEEEkeywords}
%Cooperative diversity, decode and forward, piecewise linear
%\end{IEEEkeywords}



% For peer review papers, you can put extra information on the cover
% page as needed:
% \ifCLASSOPTIONpeerreview
% \begin{center} \bfseries EDICS Category: 3-BBND \end{center}
% \fi
%
% For peerreview papers, this IEEEtran command inserts a page break and
% creates the second title. It will be ignored for other modes.
%\IEEEpeerreviewmaketitle




	\item Five cards—the ten, jack, queen, king and ace of diamonds, are well-shuffled with their face downwards. One card is then picked up at random.
\begin{enumerate}
\item
What is the probability that the card is the queen? 
\item
If the queen is drawn and put aside, what is the probability that the second card picked up is (a) an ace? (b) a queen?\\
\end{enumerate}
\solution
		%\begin{enumerate}[label=\thesection.\arabic*,ref=\thesection.\theenumi]
	\item One card is drawn from a well-shuffled deck of 52 cards. Find the probability of getting
\begin{enumerate}
\item A king of red colour 
\item A face card 
\item A red face card
\item The jack of hearts
\item A spade
\item The queen of diamonds

\end{enumerate}
\solution
		%\input{ncert/10/15/1/14/main.tex}
	\item Five cards—the ten, jack, queen, king and ace of diamonds, are well-shuffled with their face downwards. One card is then picked up at random.
\begin{enumerate}
\item
What is the probability that the card is the queen? 
\item
If the queen is drawn and put aside, what is the probability that the second card picked up is (a) an ace? (b) a queen?\\
\end{enumerate}
\solution
		%\input{ncert/10/15/1/15/defs.tex}
	\item A bag contains $5$ red balls and some blue balls. If the probability of drawing a blue ball is double that if a red ball, determine the number of blue balls in the bag. 
		\\
\solution
		%\input{ncert/10/15/2/3/defs.tex}
	\item A card is selected from a pack of 52 cards.
 \begin{enumerate}[label=(\alph*)] 
                 \item How many points are there in the sample space?
                 \item Calculate the probability that the card is an ace of spades.
                 \item Calculate the probability that the card is (i) an ace and (ii) black card.
 \end{enumerate}
\solution
		%\input{ncert/11/16/3/4/main.tex}
\item Four cards are drawn from a well-shuffled deck of 52 cards. What is the probability of obtaining 3 diamonds and one spade.
\\
\solution
		%\input{ncert/11/16/4/2/defs.tex}
\item In a certain lottery 10,000 tickets are sold and ten equal prizes are awarded. What is the probability of not getting a prize if you buy (a) one ticket (b) two tickets (c) 10 tickets ?	
\\
\solution
		%\input{ncert/11/16/4/4/defs.tex}
		%
\item 
Out of 100 students, two sections of 40 and 60 are formed. If you and your friend are among the 100 students, what is the probability that
\begin{enumerate}
\item you both enter the same section?
\item you both enter the different sections?
\end{enumerate}
\solution
		%\input{ncert/11/16/4/5/defs.tex}
	\item 
The number lock of a suitcase has 4 wheels each labelled with ten digits i.e. from 0 to 9.The lock opens with a sequence of four digits with no repeats.What is the probability of a person getting the right sequence to open the suitcase.
\\
\solution
		%\input{ncert/11/16/4/10/defs.tex}
		%
\item 
Two cards are drawn at random and without replacement from a pack of 52 playing cards. Find the probability that both the cards are black.
\\
\solution
		%\input{ncert/12/13/2/2/defs.tex}
		\item A box of oranges is inspected by examining three randomly selected oranges drawn without replacement. If all the three oranges are good, the box is approved for sale, otherwise, it is rejected. Find the probability that a box containing 15 oranges out of which 12 are good and 3 are bad ones will be approved for sale.
		\label{ncert/12/13/2/3/defs.tex}
		\item Two balls are drawn at random with replacement from a box containing 10 black and 8 red balls. Find the probability that
		\label{ncert/12/13/2/12}
\begin{enumerate}
\item both balls are red.
\item first ball is black and second is red.
\item one of them is black and other is red.
\end{enumerate}

\item In a hostel, 60\% of the students read Hindi newspaper, 40\% read English newspaper and 20\% read both Hindi and English newspapers. A student is selected at random.
		\label{ncert/12/13/2/15}
\begin{enumerate}
\item Find the probability that she reads neither Hindi nor English newspapers.
\item If she reads Hindi newspaper, find the probability that she reads English newspaper.
\item If she reads English newspaper, find the probability that she reads Hindi newspaper.\\
\end{enumerate}
\item The probability of obtaining an even prime number on each die, when a pair of dice is rolled is 
\begin{enumerate}
    \item $0$ 
    
    \item $\frac{1}{3}$ 
    
    \item $\frac{1}{12}$ 
    
    \item $\frac{1}{36}$ 
\end{enumerate}
\solution
		%\input{ncert/12/13/2/17/defs.tex}
	\item A bag contains 4 red and 4 black balls, another bag contains 2 red and 6 black balls. One of the two bags is selected at random and a ball is drawn from the bag which is found to be red. Find the probability that the ball is drawn from the first bag.
\\
\solution
		%\input{ncert/12/13/3/2/main.tex}
  \item
  Cards with numbers 2 to 101 are placed in a box. A card is selected at random.Find the probability that the card has
\begin{enumerate}[label=(\roman*)]
	\item an even number 
	\item a square number
\end{enumerate}
\solution
%\input{exemplar/10/13/3/32/main.tex}
\item
The king, queen and jack of clubs are removed from a deck of 52 playing cards and then well shuffled. Now one card is drawn at random from the remaining cards.  Determine the probability that the card is
\begin{enumerate}[label=(\roman*)]
\item a club
\item 10 of hearts
\end{enumerate}
\solution
%\input{exemplar/10/13/3/29/main.tex}
\item A team of medical students doing their internship have to assist during surgeries
at a city hospital. The probabilities of surgeries rated as very complex, complex,
routine, simple or very simple are respectively, 0.15, 0.20, 0.31, 0.26, .08. Find
the probabilities that a particular surgery will be rated
\begin{enumerate}
	\item complex or very complex;
	\item neither very complex nor very simple;
	\item routine or complex
	\item routine or simple
\end{enumerate}
\solution
%\input{exemplar/11/16/3/8(1)/main.tex}
\item A card is selected from a pack of 52 cards.
\begin{enumerate}[label=(\alph*)]
    \item How many points are there in the sample space?
    \item Calculate the probability that the card is an ace of spades.
    \item Calculate the probability that the card is (i) an ace and (ii) black card.
\end{enumerate}
\solution
%\input{exemplar/11/16/3/4/main2.tex}
\item The probability that a non leap year selected at random will contain 53 sundays.
\\
\solution
%\input{exemplar/10/13/1/19/main.tex}
\item One of the four persons John, Rita, Aslam or Gurpreet will be promoted next
month. Consequently the sample space consists of four elementary outcomes
S = {John promoted, Rita promoted, Aslam promoted, Gurpreet promoted}
You are told that the chances of John’s promotion is same as that of Gurpreet,
Rita’s chances of promotion are twice as likely as Johns. Aslam’s chances are
four times that of John.
\begin{enumerate}
	\item Determine
	\begin{enumerate}
		\item P (John promoted)
		\item P (Rita promoted)
		\item P (Aslam promoted)
		\item P (Gurpreet promoted)
	\end{enumerate}
	\item If A = {John promoted or Gurpreet promoted}, find P (A).
\end{enumerate}
\solution
%\input{exemplar/11/16/3/10/main.tex}
\item A card is drawn from a deck of 52 cards. Find the probability of getting a king or a heart or a red card.\\
\solution
%\input{exemplar/11/16/3/15/main.tex}
\item The probability that a student will pass his examination is 0.73, the probability of
the student getting a compartment is 0.13, and the probability that the student will
either pass or get compartment is 0.96. State True or False.\\
\solution
%\input{exemplar/11/16/3/31/main.tex}
\item A card is selected from a pack of 52 cards\\
\begin{enumerate}[label=(\alph*)]
\item How many points are there in the sample space?
\item Calculate the probability that the cards is an ace of spades.
\item Calculate the probability that the card is (i) an ace (ii)black card.\\
\end{enumerate}
%\input{ncert/11/16/3/4_1/Prob_4.tex}
\item In a non-leap year, the probability of having 53 tuesdays or 53 wednesdays is\\
\solution
%\input{exemplar/11/16/3/18/main.tex}
\item There are 1000 sealed envelopes in a box, 10 of them contain a cash prize of
Rs 100 each, 100 of them contain a cash prize of Rs 50 each and 200 of them
contain a cash prize of Rs 10 each and rest do not contain any cash prize. If they
are well shuffled and an envelope is picked up out, what is the probability that it
contains no cash prize?\\
\solution
%\input{exemplar/10/13/3/34/main.tex}
\item 
A die is thrown and a card is selected at random from a deck of 52 playing cards. The probability of getting an even number on the die and a spade card.\\
\solution
%\input{exemplar/12/13/3/78/main.tex}
\item
If 4-digit numbers greater than 5,000 are randomly formed from the digits 0, 1, 3, 5, and 7, what is the probability of forming a number divisible by 5 when:
\begin{enumerate}
    \item The digits are repeated?
    \item The repetition of digits is not allowed?
\end{enumerate}
\solution
%\input{ncert/11/16/4/9/main.tex}
\item Consider the probability space $\brak{\Omega, \mathcal{G}, P}$ where $\Omega = [0,2]$ and $\mathcal{G} = \cbrak{\phi, \Omega, [0,1], (1,2]}$. Let $X$ and $Y$ be two functions on $\Omega$ defined as
\begin{align*}
    X(\omega) = 
    \begin{cases}
        1 & \text{if }\omega \in [0, 1]\\
        2 & \text{if }\omega \in (1, 2]
    \end{cases}
\end{align*}
and
\begin{align*}
    Y(\omega) = 
    \begin{cases}
        2 & \text{if }\omega \in [0, 1.5]\\
        3 & \text{if }\omega \in (1.5, 2].
    \end{cases}
\end{align*}
Then which one of the following statements is true?
\begin{enumerate}
    \item [(A)] $X$ is a random variable with respect to $\mathcal{G}$, but $Y$ is not a random variable with respect to $\mathcal{G}$.
    \item [(B)] $Y$ is a random variable with respect to $\mathcal{G}$, but $X$ is not a random variable with respect to $\mathcal{G}$.
    \item [(C)] Neither $X$ nor $Y$ is a random variable with respect to $\mathcal{G}$.
    \item [(D)] Both $X$ and $Y$ are random variables with respect to $\mathcal{G}$.
\end{enumerate} \hfill (GATE ST 2023)\\
\solution
%\input{gate/ST/2023/14/main.tex}
	\item  A die is loaded in such a way that each odd number is twice as likely to occur as
each even number. Find $P(G)$, where $G$ is the event that a number greater than
3 occurs on a single roll of the die.
\\
\solution
		%\input{exemplar/11/16/3/5/main.tex}
	\item All the jacks, queens and kings are removed from a deck of 52 playing cards. The remaining cards are well shuffled and then one card is drawn at random. Giving ace a value 1 similar value for other cards, find the probability that the card has a value 
		\begin{enumerate}
			\item 7
			\item greater than 7
			\item less than 7
		\end{enumerate}
		%\input{exemplar/10/13/3/30/main.tex}
  \item A Lot consists of 48 mobile phones of which 42 are good, 3 have only minor defects and 3 have major defects.Varnika will buy a phone if it is good but the trader will only buy a mobile if it has no major defects. One phone is selected at random from the lot. What is the probability that it is
\begin{enumerate}
	\item acceptable to Varnika?
            \item acceptable to the trader?
\end{enumerate}
\solution
	%\input{exemplar/10/13/3/40/main.tex}
 \item A student says that if you throw a die, it will show up 1 or not 1. Therefore, the probability of getting 1 and the probability of getting 'not 1' each is equal to $\frac{1}{2}$. Is this correct? Give reasons.\\
 \solution
        %\input{exemplar/10/13/2/9/main.tex}
   \item Four candidates A, B, C, D have ap-
plied for the assignment to coach a school cricket
team. If A is twice as likely to be selected as B, and
B and C are given about the same chance of being
selected, while C is twice as likely to be selected
as D, what are the probabilities that
\begin{enumerate}
\item C will be selected?
\item A will not be selected?
\end{enumerate}
	%\input{exemplar/11/16/3/9/main.tex}
 \item A bag contain 24 balls of which $x$ balls are red, $2x$ are white and $3x$ are blue. A ball is selected at random, What is the probability that it is
\begin{enumerate}[label=\alph*)]
\item not red ?
\item white ?
\end{enumerate}
%\input{exemplar/10/13/3/41/main.tex}
If the letters of the word ASSASSINATION are arranged at random. Find the Probability that
\begin{enumerate}[label=(\alph*)]
\item Four $S's$ come consecutively in the word
\item Two  $I's$ and two $N's$ come together
\item All $A's$ are not coming together
\item No two $A's$ are coming together
\end{enumerate}
%\input{exemplar/11/16/3/14/main.tex}
	\item One urn contains two black balls (labelled B1 and B2) and one white ball. A
	second urn contains one black ball and two white balls (labelled W1 and W2).
	Suppose the following experiment is performed. One of the two urns is chosen
	at random. Next a ball is randomly chosen from the urn. Then a second ball is
	chosen at random from the same urn without replacing the first ball.
	
	\begin{enumerate}
	\item What is the probability that two black balls are chosen?
	
	\item What is the probability that two balls of opposite colour are chosen?
	\end{enumerate}
	\solution
	%\input{exemplar/11/16/3/12/main1.tex}
\end{enumerate}

	\item A bag contains $5$ red balls and some blue balls. If the probability of drawing a blue ball is double that if a red ball, determine the number of blue balls in the bag. 
		\\
\solution
		%\begin{enumerate}[label=\thesection.\arabic*,ref=\thesection.\theenumi]
	\item One card is drawn from a well-shuffled deck of 52 cards. Find the probability of getting
\begin{enumerate}
\item A king of red colour 
\item A face card 
\item A red face card
\item The jack of hearts
\item A spade
\item The queen of diamonds

\end{enumerate}
\solution
		%\input{ncert/10/15/1/14/main.tex}
	\item Five cards—the ten, jack, queen, king and ace of diamonds, are well-shuffled with their face downwards. One card is then picked up at random.
\begin{enumerate}
\item
What is the probability that the card is the queen? 
\item
If the queen is drawn and put aside, what is the probability that the second card picked up is (a) an ace? (b) a queen?\\
\end{enumerate}
\solution
		%\input{ncert/10/15/1/15/defs.tex}
	\item A bag contains $5$ red balls and some blue balls. If the probability of drawing a blue ball is double that if a red ball, determine the number of blue balls in the bag. 
		\\
\solution
		%\input{ncert/10/15/2/3/defs.tex}
	\item A card is selected from a pack of 52 cards.
 \begin{enumerate}[label=(\alph*)] 
                 \item How many points are there in the sample space?
                 \item Calculate the probability that the card is an ace of spades.
                 \item Calculate the probability that the card is (i) an ace and (ii) black card.
 \end{enumerate}
\solution
		%\input{ncert/11/16/3/4/main.tex}
\item Four cards are drawn from a well-shuffled deck of 52 cards. What is the probability of obtaining 3 diamonds and one spade.
\\
\solution
		%\input{ncert/11/16/4/2/defs.tex}
\item In a certain lottery 10,000 tickets are sold and ten equal prizes are awarded. What is the probability of not getting a prize if you buy (a) one ticket (b) two tickets (c) 10 tickets ?	
\\
\solution
		%\input{ncert/11/16/4/4/defs.tex}
		%
\item 
Out of 100 students, two sections of 40 and 60 are formed. If you and your friend are among the 100 students, what is the probability that
\begin{enumerate}
\item you both enter the same section?
\item you both enter the different sections?
\end{enumerate}
\solution
		%\input{ncert/11/16/4/5/defs.tex}
	\item 
The number lock of a suitcase has 4 wheels each labelled with ten digits i.e. from 0 to 9.The lock opens with a sequence of four digits with no repeats.What is the probability of a person getting the right sequence to open the suitcase.
\\
\solution
		%\input{ncert/11/16/4/10/defs.tex}
		%
\item 
Two cards are drawn at random and without replacement from a pack of 52 playing cards. Find the probability that both the cards are black.
\\
\solution
		%\input{ncert/12/13/2/2/defs.tex}
		\item A box of oranges is inspected by examining three randomly selected oranges drawn without replacement. If all the three oranges are good, the box is approved for sale, otherwise, it is rejected. Find the probability that a box containing 15 oranges out of which 12 are good and 3 are bad ones will be approved for sale.
		\label{ncert/12/13/2/3/defs.tex}
		\item Two balls are drawn at random with replacement from a box containing 10 black and 8 red balls. Find the probability that
		\label{ncert/12/13/2/12}
\begin{enumerate}
\item both balls are red.
\item first ball is black and second is red.
\item one of them is black and other is red.
\end{enumerate}

\item In a hostel, 60\% of the students read Hindi newspaper, 40\% read English newspaper and 20\% read both Hindi and English newspapers. A student is selected at random.
		\label{ncert/12/13/2/15}
\begin{enumerate}
\item Find the probability that she reads neither Hindi nor English newspapers.
\item If she reads Hindi newspaper, find the probability that she reads English newspaper.
\item If she reads English newspaper, find the probability that she reads Hindi newspaper.\\
\end{enumerate}
\item The probability of obtaining an even prime number on each die, when a pair of dice is rolled is 
\begin{enumerate}
    \item $0$ 
    
    \item $\frac{1}{3}$ 
    
    \item $\frac{1}{12}$ 
    
    \item $\frac{1}{36}$ 
\end{enumerate}
\solution
		%\input{ncert/12/13/2/17/defs.tex}
	\item A bag contains 4 red and 4 black balls, another bag contains 2 red and 6 black balls. One of the two bags is selected at random and a ball is drawn from the bag which is found to be red. Find the probability that the ball is drawn from the first bag.
\\
\solution
		%\input{ncert/12/13/3/2/main.tex}
  \item
  Cards with numbers 2 to 101 are placed in a box. A card is selected at random.Find the probability that the card has
\begin{enumerate}[label=(\roman*)]
	\item an even number 
	\item a square number
\end{enumerate}
\solution
%\input{exemplar/10/13/3/32/main.tex}
\item
The king, queen and jack of clubs are removed from a deck of 52 playing cards and then well shuffled. Now one card is drawn at random from the remaining cards.  Determine the probability that the card is
\begin{enumerate}[label=(\roman*)]
\item a club
\item 10 of hearts
\end{enumerate}
\solution
%\input{exemplar/10/13/3/29/main.tex}
\item A team of medical students doing their internship have to assist during surgeries
at a city hospital. The probabilities of surgeries rated as very complex, complex,
routine, simple or very simple are respectively, 0.15, 0.20, 0.31, 0.26, .08. Find
the probabilities that a particular surgery will be rated
\begin{enumerate}
	\item complex or very complex;
	\item neither very complex nor very simple;
	\item routine or complex
	\item routine or simple
\end{enumerate}
\solution
%\input{exemplar/11/16/3/8(1)/main.tex}
\item A card is selected from a pack of 52 cards.
\begin{enumerate}[label=(\alph*)]
    \item How many points are there in the sample space?
    \item Calculate the probability that the card is an ace of spades.
    \item Calculate the probability that the card is (i) an ace and (ii) black card.
\end{enumerate}
\solution
%\input{exemplar/11/16/3/4/main2.tex}
\item The probability that a non leap year selected at random will contain 53 sundays.
\\
\solution
%\input{exemplar/10/13/1/19/main.tex}
\item One of the four persons John, Rita, Aslam or Gurpreet will be promoted next
month. Consequently the sample space consists of four elementary outcomes
S = {John promoted, Rita promoted, Aslam promoted, Gurpreet promoted}
You are told that the chances of John’s promotion is same as that of Gurpreet,
Rita’s chances of promotion are twice as likely as Johns. Aslam’s chances are
four times that of John.
\begin{enumerate}
	\item Determine
	\begin{enumerate}
		\item P (John promoted)
		\item P (Rita promoted)
		\item P (Aslam promoted)
		\item P (Gurpreet promoted)
	\end{enumerate}
	\item If A = {John promoted or Gurpreet promoted}, find P (A).
\end{enumerate}
\solution
%\input{exemplar/11/16/3/10/main.tex}
\item A card is drawn from a deck of 52 cards. Find the probability of getting a king or a heart or a red card.\\
\solution
%\input{exemplar/11/16/3/15/main.tex}
\item The probability that a student will pass his examination is 0.73, the probability of
the student getting a compartment is 0.13, and the probability that the student will
either pass or get compartment is 0.96. State True or False.\\
\solution
%\input{exemplar/11/16/3/31/main.tex}
\item A card is selected from a pack of 52 cards\\
\begin{enumerate}[label=(\alph*)]
\item How many points are there in the sample space?
\item Calculate the probability that the cards is an ace of spades.
\item Calculate the probability that the card is (i) an ace (ii)black card.\\
\end{enumerate}
%\input{ncert/11/16/3/4_1/Prob_4.tex}
\item In a non-leap year, the probability of having 53 tuesdays or 53 wednesdays is\\
\solution
%\input{exemplar/11/16/3/18/main.tex}
\item There are 1000 sealed envelopes in a box, 10 of them contain a cash prize of
Rs 100 each, 100 of them contain a cash prize of Rs 50 each and 200 of them
contain a cash prize of Rs 10 each and rest do not contain any cash prize. If they
are well shuffled and an envelope is picked up out, what is the probability that it
contains no cash prize?\\
\solution
%\input{exemplar/10/13/3/34/main.tex}
\item 
A die is thrown and a card is selected at random from a deck of 52 playing cards. The probability of getting an even number on the die and a spade card.\\
\solution
%\input{exemplar/12/13/3/78/main.tex}
\item
If 4-digit numbers greater than 5,000 are randomly formed from the digits 0, 1, 3, 5, and 7, what is the probability of forming a number divisible by 5 when:
\begin{enumerate}
    \item The digits are repeated?
    \item The repetition of digits is not allowed?
\end{enumerate}
\solution
%\input{ncert/11/16/4/9/main.tex}
\item Consider the probability space $\brak{\Omega, \mathcal{G}, P}$ where $\Omega = [0,2]$ and $\mathcal{G} = \cbrak{\phi, \Omega, [0,1], (1,2]}$. Let $X$ and $Y$ be two functions on $\Omega$ defined as
\begin{align*}
    X(\omega) = 
    \begin{cases}
        1 & \text{if }\omega \in [0, 1]\\
        2 & \text{if }\omega \in (1, 2]
    \end{cases}
\end{align*}
and
\begin{align*}
    Y(\omega) = 
    \begin{cases}
        2 & \text{if }\omega \in [0, 1.5]\\
        3 & \text{if }\omega \in (1.5, 2].
    \end{cases}
\end{align*}
Then which one of the following statements is true?
\begin{enumerate}
    \item [(A)] $X$ is a random variable with respect to $\mathcal{G}$, but $Y$ is not a random variable with respect to $\mathcal{G}$.
    \item [(B)] $Y$ is a random variable with respect to $\mathcal{G}$, but $X$ is not a random variable with respect to $\mathcal{G}$.
    \item [(C)] Neither $X$ nor $Y$ is a random variable with respect to $\mathcal{G}$.
    \item [(D)] Both $X$ and $Y$ are random variables with respect to $\mathcal{G}$.
\end{enumerate} \hfill (GATE ST 2023)\\
\solution
%\input{gate/ST/2023/14/main.tex}
	\item  A die is loaded in such a way that each odd number is twice as likely to occur as
each even number. Find $P(G)$, where $G$ is the event that a number greater than
3 occurs on a single roll of the die.
\\
\solution
		%\input{exemplar/11/16/3/5/main.tex}
	\item All the jacks, queens and kings are removed from a deck of 52 playing cards. The remaining cards are well shuffled and then one card is drawn at random. Giving ace a value 1 similar value for other cards, find the probability that the card has a value 
		\begin{enumerate}
			\item 7
			\item greater than 7
			\item less than 7
		\end{enumerate}
		%\input{exemplar/10/13/3/30/main.tex}
  \item A Lot consists of 48 mobile phones of which 42 are good, 3 have only minor defects and 3 have major defects.Varnika will buy a phone if it is good but the trader will only buy a mobile if it has no major defects. One phone is selected at random from the lot. What is the probability that it is
\begin{enumerate}
	\item acceptable to Varnika?
            \item acceptable to the trader?
\end{enumerate}
\solution
	%\input{exemplar/10/13/3/40/main.tex}
 \item A student says that if you throw a die, it will show up 1 or not 1. Therefore, the probability of getting 1 and the probability of getting 'not 1' each is equal to $\frac{1}{2}$. Is this correct? Give reasons.\\
 \solution
        %\input{exemplar/10/13/2/9/main.tex}
   \item Four candidates A, B, C, D have ap-
plied for the assignment to coach a school cricket
team. If A is twice as likely to be selected as B, and
B and C are given about the same chance of being
selected, while C is twice as likely to be selected
as D, what are the probabilities that
\begin{enumerate}
\item C will be selected?
\item A will not be selected?
\end{enumerate}
	%\input{exemplar/11/16/3/9/main.tex}
 \item A bag contain 24 balls of which $x$ balls are red, $2x$ are white and $3x$ are blue. A ball is selected at random, What is the probability that it is
\begin{enumerate}[label=\alph*)]
\item not red ?
\item white ?
\end{enumerate}
%\input{exemplar/10/13/3/41/main.tex}
If the letters of the word ASSASSINATION are arranged at random. Find the Probability that
\begin{enumerate}[label=(\alph*)]
\item Four $S's$ come consecutively in the word
\item Two  $I's$ and two $N's$ come together
\item All $A's$ are not coming together
\item No two $A's$ are coming together
\end{enumerate}
%\input{exemplar/11/16/3/14/main.tex}
	\item One urn contains two black balls (labelled B1 and B2) and one white ball. A
	second urn contains one black ball and two white balls (labelled W1 and W2).
	Suppose the following experiment is performed. One of the two urns is chosen
	at random. Next a ball is randomly chosen from the urn. Then a second ball is
	chosen at random from the same urn without replacing the first ball.
	
	\begin{enumerate}
	\item What is the probability that two black balls are chosen?
	
	\item What is the probability that two balls of opposite colour are chosen?
	\end{enumerate}
	\solution
	%\input{exemplar/11/16/3/12/main1.tex}
\end{enumerate}

	\item A card is selected from a pack of 52 cards.
 \begin{enumerate}[label=(\alph*)] 
                 \item How many points are there in the sample space?
                 \item Calculate the probability that the card is an ace of spades.
                 \item Calculate the probability that the card is (i) an ace and (ii) black card.
 \end{enumerate}
\solution
		%\begin{table}[H]
	\centering
\begin{tabular}{|c|c|c|}
\hline
Random variable &Value &Definition\\ \hline
\multirow{3}{*}{X} &0 &Slips of Rs 1\\
&1 &Slips of Rs 5\\
&2 &Slips of Rs 13\\ \hline
\multirow{2}{*}{Y} &0 &Box A\\
&1 &Box B\\\hline
\end{tabular}
\caption{}
\label{tab:Distribution}
\end{table}
See \tabref{tab:Distribution}.
\begin{align}
p_{Y}\brak{k}= \begin{cases} 
      \frac{1}{3} & {k=0} \\
      \frac{2}{3 }& {k=1} 
   \end{cases}
   \\
p_{Y|X}\brak{0|0} = \frac{19}{25}\, 
p_{Y|X}\brak{0|1} = \frac{6}{25}\,
p_{Y|X}\brak{1|0} = \frac{45}{50}\,
p_{Y|X}\brak{1|2} = \frac{5}{50}
\end{align}
The desired probability is the probability that a slip drawn at random is marked other than Rs 1,
\begin{align}
&=1-p_X\brak{0}\\
&= p_X(1) + p_X(2)
\end{align}
Using Bayes theorem,
\begin{align}
&= p_Y\brak{0} \times \pr{Y=0 | X=1} + p_Y\brak{1} \times \pr{Y=1|X=2}\\
&=\frac{1}{3} \times \frac{6}{25} + \frac{2}{3} \times \frac{5}{50}\\
&=\frac{11}{75}
\end{align}

\newpage

%\tableofcontents

\bigskip

\renewcommand{\thefigure}{\theenumi}
\renewcommand{\thetable}{\theenumi}
%\renewcommand{\theequation}{\theenumi}

%\begin{abstract}
%%\boldmath
%In this letter, an algorithm for evaluating the exact analytical bit error rate  (BER)  for the piecewise linear (PL) combiner for  multiple relays is presented. Previous results were available only for upto three relays. The algorithm is unique in the sense that  the actual mathematical expressions, that are prohibitively large, need not be explicitly obtained. The diversity gain due to multiple relays is shown through plots of the analytical BER, well supported by simulations. 
%
%\end{abstract}
% IEEEtran.cls defaults to using nonbold math in the Abstract.
% This preserves the distinction between vectors and scalars. However,
% if the journal you are submitting to favors bold math in the abstract,
% then you can use LaTeX's standard command \boldmath at the very start
% of the abstract to achieve this. Many IEEE journals frown on math
% in the abstract anyway.

% Note that keywords are not normally used for peerreview papers.
%\begin{IEEEkeywords}
%Cooperative diversity, decode and forward, piecewise linear
%\end{IEEEkeywords}



% For peer review papers, you can put extra information on the cover
% page as needed:
% \ifCLASSOPTIONpeerreview
% \begin{center} \bfseries EDICS Category: 3-BBND \end{center}
% \fi
%
% For peerreview papers, this IEEEtran command inserts a page break and
% creates the second title. It will be ignored for other modes.
%\IEEEpeerreviewmaketitle




\item Four cards are drawn from a well-shuffled deck of 52 cards. What is the probability of obtaining 3 diamonds and one spade.
\\
\solution
		%\begin{enumerate}[label=\thesection.\arabic*,ref=\thesection.\theenumi]
	\item One card is drawn from a well-shuffled deck of 52 cards. Find the probability of getting
\begin{enumerate}
\item A king of red colour 
\item A face card 
\item A red face card
\item The jack of hearts
\item A spade
\item The queen of diamonds

\end{enumerate}
\solution
		%\input{ncert/10/15/1/14/main.tex}
	\item Five cards—the ten, jack, queen, king and ace of diamonds, are well-shuffled with their face downwards. One card is then picked up at random.
\begin{enumerate}
\item
What is the probability that the card is the queen? 
\item
If the queen is drawn and put aside, what is the probability that the second card picked up is (a) an ace? (b) a queen?\\
\end{enumerate}
\solution
		%\input{ncert/10/15/1/15/defs.tex}
	\item A bag contains $5$ red balls and some blue balls. If the probability of drawing a blue ball is double that if a red ball, determine the number of blue balls in the bag. 
		\\
\solution
		%\input{ncert/10/15/2/3/defs.tex}
	\item A card is selected from a pack of 52 cards.
 \begin{enumerate}[label=(\alph*)] 
                 \item How many points are there in the sample space?
                 \item Calculate the probability that the card is an ace of spades.
                 \item Calculate the probability that the card is (i) an ace and (ii) black card.
 \end{enumerate}
\solution
		%\input{ncert/11/16/3/4/main.tex}
\item Four cards are drawn from a well-shuffled deck of 52 cards. What is the probability of obtaining 3 diamonds and one spade.
\\
\solution
		%\input{ncert/11/16/4/2/defs.tex}
\item In a certain lottery 10,000 tickets are sold and ten equal prizes are awarded. What is the probability of not getting a prize if you buy (a) one ticket (b) two tickets (c) 10 tickets ?	
\\
\solution
		%\input{ncert/11/16/4/4/defs.tex}
		%
\item 
Out of 100 students, two sections of 40 and 60 are formed. If you and your friend are among the 100 students, what is the probability that
\begin{enumerate}
\item you both enter the same section?
\item you both enter the different sections?
\end{enumerate}
\solution
		%\input{ncert/11/16/4/5/defs.tex}
	\item 
The number lock of a suitcase has 4 wheels each labelled with ten digits i.e. from 0 to 9.The lock opens with a sequence of four digits with no repeats.What is the probability of a person getting the right sequence to open the suitcase.
\\
\solution
		%\input{ncert/11/16/4/10/defs.tex}
		%
\item 
Two cards are drawn at random and without replacement from a pack of 52 playing cards. Find the probability that both the cards are black.
\\
\solution
		%\input{ncert/12/13/2/2/defs.tex}
		\item A box of oranges is inspected by examining three randomly selected oranges drawn without replacement. If all the three oranges are good, the box is approved for sale, otherwise, it is rejected. Find the probability that a box containing 15 oranges out of which 12 are good and 3 are bad ones will be approved for sale.
		\label{ncert/12/13/2/3/defs.tex}
		\item Two balls are drawn at random with replacement from a box containing 10 black and 8 red balls. Find the probability that
		\label{ncert/12/13/2/12}
\begin{enumerate}
\item both balls are red.
\item first ball is black and second is red.
\item one of them is black and other is red.
\end{enumerate}

\item In a hostel, 60\% of the students read Hindi newspaper, 40\% read English newspaper and 20\% read both Hindi and English newspapers. A student is selected at random.
		\label{ncert/12/13/2/15}
\begin{enumerate}
\item Find the probability that she reads neither Hindi nor English newspapers.
\item If she reads Hindi newspaper, find the probability that she reads English newspaper.
\item If she reads English newspaper, find the probability that she reads Hindi newspaper.\\
\end{enumerate}
\item The probability of obtaining an even prime number on each die, when a pair of dice is rolled is 
\begin{enumerate}
    \item $0$ 
    
    \item $\frac{1}{3}$ 
    
    \item $\frac{1}{12}$ 
    
    \item $\frac{1}{36}$ 
\end{enumerate}
\solution
		%\input{ncert/12/13/2/17/defs.tex}
	\item A bag contains 4 red and 4 black balls, another bag contains 2 red and 6 black balls. One of the two bags is selected at random and a ball is drawn from the bag which is found to be red. Find the probability that the ball is drawn from the first bag.
\\
\solution
		%\input{ncert/12/13/3/2/main.tex}
  \item
  Cards with numbers 2 to 101 are placed in a box. A card is selected at random.Find the probability that the card has
\begin{enumerate}[label=(\roman*)]
	\item an even number 
	\item a square number
\end{enumerate}
\solution
%\input{exemplar/10/13/3/32/main.tex}
\item
The king, queen and jack of clubs are removed from a deck of 52 playing cards and then well shuffled. Now one card is drawn at random from the remaining cards.  Determine the probability that the card is
\begin{enumerate}[label=(\roman*)]
\item a club
\item 10 of hearts
\end{enumerate}
\solution
%\input{exemplar/10/13/3/29/main.tex}
\item A team of medical students doing their internship have to assist during surgeries
at a city hospital. The probabilities of surgeries rated as very complex, complex,
routine, simple or very simple are respectively, 0.15, 0.20, 0.31, 0.26, .08. Find
the probabilities that a particular surgery will be rated
\begin{enumerate}
	\item complex or very complex;
	\item neither very complex nor very simple;
	\item routine or complex
	\item routine or simple
\end{enumerate}
\solution
%\input{exemplar/11/16/3/8(1)/main.tex}
\item A card is selected from a pack of 52 cards.
\begin{enumerate}[label=(\alph*)]
    \item How many points are there in the sample space?
    \item Calculate the probability that the card is an ace of spades.
    \item Calculate the probability that the card is (i) an ace and (ii) black card.
\end{enumerate}
\solution
%\input{exemplar/11/16/3/4/main2.tex}
\item The probability that a non leap year selected at random will contain 53 sundays.
\\
\solution
%\input{exemplar/10/13/1/19/main.tex}
\item One of the four persons John, Rita, Aslam or Gurpreet will be promoted next
month. Consequently the sample space consists of four elementary outcomes
S = {John promoted, Rita promoted, Aslam promoted, Gurpreet promoted}
You are told that the chances of John’s promotion is same as that of Gurpreet,
Rita’s chances of promotion are twice as likely as Johns. Aslam’s chances are
four times that of John.
\begin{enumerate}
	\item Determine
	\begin{enumerate}
		\item P (John promoted)
		\item P (Rita promoted)
		\item P (Aslam promoted)
		\item P (Gurpreet promoted)
	\end{enumerate}
	\item If A = {John promoted or Gurpreet promoted}, find P (A).
\end{enumerate}
\solution
%\input{exemplar/11/16/3/10/main.tex}
\item A card is drawn from a deck of 52 cards. Find the probability of getting a king or a heart or a red card.\\
\solution
%\input{exemplar/11/16/3/15/main.tex}
\item The probability that a student will pass his examination is 0.73, the probability of
the student getting a compartment is 0.13, and the probability that the student will
either pass or get compartment is 0.96. State True or False.\\
\solution
%\input{exemplar/11/16/3/31/main.tex}
\item A card is selected from a pack of 52 cards\\
\begin{enumerate}[label=(\alph*)]
\item How many points are there in the sample space?
\item Calculate the probability that the cards is an ace of spades.
\item Calculate the probability that the card is (i) an ace (ii)black card.\\
\end{enumerate}
%\input{ncert/11/16/3/4_1/Prob_4.tex}
\item In a non-leap year, the probability of having 53 tuesdays or 53 wednesdays is\\
\solution
%\input{exemplar/11/16/3/18/main.tex}
\item There are 1000 sealed envelopes in a box, 10 of them contain a cash prize of
Rs 100 each, 100 of them contain a cash prize of Rs 50 each and 200 of them
contain a cash prize of Rs 10 each and rest do not contain any cash prize. If they
are well shuffled and an envelope is picked up out, what is the probability that it
contains no cash prize?\\
\solution
%\input{exemplar/10/13/3/34/main.tex}
\item 
A die is thrown and a card is selected at random from a deck of 52 playing cards. The probability of getting an even number on the die and a spade card.\\
\solution
%\input{exemplar/12/13/3/78/main.tex}
\item
If 4-digit numbers greater than 5,000 are randomly formed from the digits 0, 1, 3, 5, and 7, what is the probability of forming a number divisible by 5 when:
\begin{enumerate}
    \item The digits are repeated?
    \item The repetition of digits is not allowed?
\end{enumerate}
\solution
%\input{ncert/11/16/4/9/main.tex}
\item Consider the probability space $\brak{\Omega, \mathcal{G}, P}$ where $\Omega = [0,2]$ and $\mathcal{G} = \cbrak{\phi, \Omega, [0,1], (1,2]}$. Let $X$ and $Y$ be two functions on $\Omega$ defined as
\begin{align*}
    X(\omega) = 
    \begin{cases}
        1 & \text{if }\omega \in [0, 1]\\
        2 & \text{if }\omega \in (1, 2]
    \end{cases}
\end{align*}
and
\begin{align*}
    Y(\omega) = 
    \begin{cases}
        2 & \text{if }\omega \in [0, 1.5]\\
        3 & \text{if }\omega \in (1.5, 2].
    \end{cases}
\end{align*}
Then which one of the following statements is true?
\begin{enumerate}
    \item [(A)] $X$ is a random variable with respect to $\mathcal{G}$, but $Y$ is not a random variable with respect to $\mathcal{G}$.
    \item [(B)] $Y$ is a random variable with respect to $\mathcal{G}$, but $X$ is not a random variable with respect to $\mathcal{G}$.
    \item [(C)] Neither $X$ nor $Y$ is a random variable with respect to $\mathcal{G}$.
    \item [(D)] Both $X$ and $Y$ are random variables with respect to $\mathcal{G}$.
\end{enumerate} \hfill (GATE ST 2023)\\
\solution
%\input{gate/ST/2023/14/main.tex}
	\item  A die is loaded in such a way that each odd number is twice as likely to occur as
each even number. Find $P(G)$, where $G$ is the event that a number greater than
3 occurs on a single roll of the die.
\\
\solution
		%\input{exemplar/11/16/3/5/main.tex}
	\item All the jacks, queens and kings are removed from a deck of 52 playing cards. The remaining cards are well shuffled and then one card is drawn at random. Giving ace a value 1 similar value for other cards, find the probability that the card has a value 
		\begin{enumerate}
			\item 7
			\item greater than 7
			\item less than 7
		\end{enumerate}
		%\input{exemplar/10/13/3/30/main.tex}
  \item A Lot consists of 48 mobile phones of which 42 are good, 3 have only minor defects and 3 have major defects.Varnika will buy a phone if it is good but the trader will only buy a mobile if it has no major defects. One phone is selected at random from the lot. What is the probability that it is
\begin{enumerate}
	\item acceptable to Varnika?
            \item acceptable to the trader?
\end{enumerate}
\solution
	%\input{exemplar/10/13/3/40/main.tex}
 \item A student says that if you throw a die, it will show up 1 or not 1. Therefore, the probability of getting 1 and the probability of getting 'not 1' each is equal to $\frac{1}{2}$. Is this correct? Give reasons.\\
 \solution
        %\input{exemplar/10/13/2/9/main.tex}
   \item Four candidates A, B, C, D have ap-
plied for the assignment to coach a school cricket
team. If A is twice as likely to be selected as B, and
B and C are given about the same chance of being
selected, while C is twice as likely to be selected
as D, what are the probabilities that
\begin{enumerate}
\item C will be selected?
\item A will not be selected?
\end{enumerate}
	%\input{exemplar/11/16/3/9/main.tex}
 \item A bag contain 24 balls of which $x$ balls are red, $2x$ are white and $3x$ are blue. A ball is selected at random, What is the probability that it is
\begin{enumerate}[label=\alph*)]
\item not red ?
\item white ?
\end{enumerate}
%\input{exemplar/10/13/3/41/main.tex}
If the letters of the word ASSASSINATION are arranged at random. Find the Probability that
\begin{enumerate}[label=(\alph*)]
\item Four $S's$ come consecutively in the word
\item Two  $I's$ and two $N's$ come together
\item All $A's$ are not coming together
\item No two $A's$ are coming together
\end{enumerate}
%\input{exemplar/11/16/3/14/main.tex}
	\item One urn contains two black balls (labelled B1 and B2) and one white ball. A
	second urn contains one black ball and two white balls (labelled W1 and W2).
	Suppose the following experiment is performed. One of the two urns is chosen
	at random. Next a ball is randomly chosen from the urn. Then a second ball is
	chosen at random from the same urn without replacing the first ball.
	
	\begin{enumerate}
	\item What is the probability that two black balls are chosen?
	
	\item What is the probability that two balls of opposite colour are chosen?
	\end{enumerate}
	\solution
	%\input{exemplar/11/16/3/12/main1.tex}
\end{enumerate}

\item In a certain lottery 10,000 tickets are sold and ten equal prizes are awarded. What is the probability of not getting a prize if you buy (a) one ticket (b) two tickets (c) 10 tickets ?	
\\
\solution
		%\begin{enumerate}[label=\thesection.\arabic*,ref=\thesection.\theenumi]
	\item One card is drawn from a well-shuffled deck of 52 cards. Find the probability of getting
\begin{enumerate}
\item A king of red colour 
\item A face card 
\item A red face card
\item The jack of hearts
\item A spade
\item The queen of diamonds

\end{enumerate}
\solution
		%\input{ncert/10/15/1/14/main.tex}
	\item Five cards—the ten, jack, queen, king and ace of diamonds, are well-shuffled with their face downwards. One card is then picked up at random.
\begin{enumerate}
\item
What is the probability that the card is the queen? 
\item
If the queen is drawn and put aside, what is the probability that the second card picked up is (a) an ace? (b) a queen?\\
\end{enumerate}
\solution
		%\input{ncert/10/15/1/15/defs.tex}
	\item A bag contains $5$ red balls and some blue balls. If the probability of drawing a blue ball is double that if a red ball, determine the number of blue balls in the bag. 
		\\
\solution
		%\input{ncert/10/15/2/3/defs.tex}
	\item A card is selected from a pack of 52 cards.
 \begin{enumerate}[label=(\alph*)] 
                 \item How many points are there in the sample space?
                 \item Calculate the probability that the card is an ace of spades.
                 \item Calculate the probability that the card is (i) an ace and (ii) black card.
 \end{enumerate}
\solution
		%\input{ncert/11/16/3/4/main.tex}
\item Four cards are drawn from a well-shuffled deck of 52 cards. What is the probability of obtaining 3 diamonds and one spade.
\\
\solution
		%\input{ncert/11/16/4/2/defs.tex}
\item In a certain lottery 10,000 tickets are sold and ten equal prizes are awarded. What is the probability of not getting a prize if you buy (a) one ticket (b) two tickets (c) 10 tickets ?	
\\
\solution
		%\input{ncert/11/16/4/4/defs.tex}
		%
\item 
Out of 100 students, two sections of 40 and 60 are formed. If you and your friend are among the 100 students, what is the probability that
\begin{enumerate}
\item you both enter the same section?
\item you both enter the different sections?
\end{enumerate}
\solution
		%\input{ncert/11/16/4/5/defs.tex}
	\item 
The number lock of a suitcase has 4 wheels each labelled with ten digits i.e. from 0 to 9.The lock opens with a sequence of four digits with no repeats.What is the probability of a person getting the right sequence to open the suitcase.
\\
\solution
		%\input{ncert/11/16/4/10/defs.tex}
		%
\item 
Two cards are drawn at random and without replacement from a pack of 52 playing cards. Find the probability that both the cards are black.
\\
\solution
		%\input{ncert/12/13/2/2/defs.tex}
		\item A box of oranges is inspected by examining three randomly selected oranges drawn without replacement. If all the three oranges are good, the box is approved for sale, otherwise, it is rejected. Find the probability that a box containing 15 oranges out of which 12 are good and 3 are bad ones will be approved for sale.
		\label{ncert/12/13/2/3/defs.tex}
		\item Two balls are drawn at random with replacement from a box containing 10 black and 8 red balls. Find the probability that
		\label{ncert/12/13/2/12}
\begin{enumerate}
\item both balls are red.
\item first ball is black and second is red.
\item one of them is black and other is red.
\end{enumerate}

\item In a hostel, 60\% of the students read Hindi newspaper, 40\% read English newspaper and 20\% read both Hindi and English newspapers. A student is selected at random.
		\label{ncert/12/13/2/15}
\begin{enumerate}
\item Find the probability that she reads neither Hindi nor English newspapers.
\item If she reads Hindi newspaper, find the probability that she reads English newspaper.
\item If she reads English newspaper, find the probability that she reads Hindi newspaper.\\
\end{enumerate}
\item The probability of obtaining an even prime number on each die, when a pair of dice is rolled is 
\begin{enumerate}
    \item $0$ 
    
    \item $\frac{1}{3}$ 
    
    \item $\frac{1}{12}$ 
    
    \item $\frac{1}{36}$ 
\end{enumerate}
\solution
		%\input{ncert/12/13/2/17/defs.tex}
	\item A bag contains 4 red and 4 black balls, another bag contains 2 red and 6 black balls. One of the two bags is selected at random and a ball is drawn from the bag which is found to be red. Find the probability that the ball is drawn from the first bag.
\\
\solution
		%\input{ncert/12/13/3/2/main.tex}
  \item
  Cards with numbers 2 to 101 are placed in a box. A card is selected at random.Find the probability that the card has
\begin{enumerate}[label=(\roman*)]
	\item an even number 
	\item a square number
\end{enumerate}
\solution
%\input{exemplar/10/13/3/32/main.tex}
\item
The king, queen and jack of clubs are removed from a deck of 52 playing cards and then well shuffled. Now one card is drawn at random from the remaining cards.  Determine the probability that the card is
\begin{enumerate}[label=(\roman*)]
\item a club
\item 10 of hearts
\end{enumerate}
\solution
%\input{exemplar/10/13/3/29/main.tex}
\item A team of medical students doing their internship have to assist during surgeries
at a city hospital. The probabilities of surgeries rated as very complex, complex,
routine, simple or very simple are respectively, 0.15, 0.20, 0.31, 0.26, .08. Find
the probabilities that a particular surgery will be rated
\begin{enumerate}
	\item complex or very complex;
	\item neither very complex nor very simple;
	\item routine or complex
	\item routine or simple
\end{enumerate}
\solution
%\input{exemplar/11/16/3/8(1)/main.tex}
\item A card is selected from a pack of 52 cards.
\begin{enumerate}[label=(\alph*)]
    \item How many points are there in the sample space?
    \item Calculate the probability that the card is an ace of spades.
    \item Calculate the probability that the card is (i) an ace and (ii) black card.
\end{enumerate}
\solution
%\input{exemplar/11/16/3/4/main2.tex}
\item The probability that a non leap year selected at random will contain 53 sundays.
\\
\solution
%\input{exemplar/10/13/1/19/main.tex}
\item One of the four persons John, Rita, Aslam or Gurpreet will be promoted next
month. Consequently the sample space consists of four elementary outcomes
S = {John promoted, Rita promoted, Aslam promoted, Gurpreet promoted}
You are told that the chances of John’s promotion is same as that of Gurpreet,
Rita’s chances of promotion are twice as likely as Johns. Aslam’s chances are
four times that of John.
\begin{enumerate}
	\item Determine
	\begin{enumerate}
		\item P (John promoted)
		\item P (Rita promoted)
		\item P (Aslam promoted)
		\item P (Gurpreet promoted)
	\end{enumerate}
	\item If A = {John promoted or Gurpreet promoted}, find P (A).
\end{enumerate}
\solution
%\input{exemplar/11/16/3/10/main.tex}
\item A card is drawn from a deck of 52 cards. Find the probability of getting a king or a heart or a red card.\\
\solution
%\input{exemplar/11/16/3/15/main.tex}
\item The probability that a student will pass his examination is 0.73, the probability of
the student getting a compartment is 0.13, and the probability that the student will
either pass or get compartment is 0.96. State True or False.\\
\solution
%\input{exemplar/11/16/3/31/main.tex}
\item A card is selected from a pack of 52 cards\\
\begin{enumerate}[label=(\alph*)]
\item How many points are there in the sample space?
\item Calculate the probability that the cards is an ace of spades.
\item Calculate the probability that the card is (i) an ace (ii)black card.\\
\end{enumerate}
%\input{ncert/11/16/3/4_1/Prob_4.tex}
\item In a non-leap year, the probability of having 53 tuesdays or 53 wednesdays is\\
\solution
%\input{exemplar/11/16/3/18/main.tex}
\item There are 1000 sealed envelopes in a box, 10 of them contain a cash prize of
Rs 100 each, 100 of them contain a cash prize of Rs 50 each and 200 of them
contain a cash prize of Rs 10 each and rest do not contain any cash prize. If they
are well shuffled and an envelope is picked up out, what is the probability that it
contains no cash prize?\\
\solution
%\input{exemplar/10/13/3/34/main.tex}
\item 
A die is thrown and a card is selected at random from a deck of 52 playing cards. The probability of getting an even number on the die and a spade card.\\
\solution
%\input{exemplar/12/13/3/78/main.tex}
\item
If 4-digit numbers greater than 5,000 are randomly formed from the digits 0, 1, 3, 5, and 7, what is the probability of forming a number divisible by 5 when:
\begin{enumerate}
    \item The digits are repeated?
    \item The repetition of digits is not allowed?
\end{enumerate}
\solution
%\input{ncert/11/16/4/9/main.tex}
\item Consider the probability space $\brak{\Omega, \mathcal{G}, P}$ where $\Omega = [0,2]$ and $\mathcal{G} = \cbrak{\phi, \Omega, [0,1], (1,2]}$. Let $X$ and $Y$ be two functions on $\Omega$ defined as
\begin{align*}
    X(\omega) = 
    \begin{cases}
        1 & \text{if }\omega \in [0, 1]\\
        2 & \text{if }\omega \in (1, 2]
    \end{cases}
\end{align*}
and
\begin{align*}
    Y(\omega) = 
    \begin{cases}
        2 & \text{if }\omega \in [0, 1.5]\\
        3 & \text{if }\omega \in (1.5, 2].
    \end{cases}
\end{align*}
Then which one of the following statements is true?
\begin{enumerate}
    \item [(A)] $X$ is a random variable with respect to $\mathcal{G}$, but $Y$ is not a random variable with respect to $\mathcal{G}$.
    \item [(B)] $Y$ is a random variable with respect to $\mathcal{G}$, but $X$ is not a random variable with respect to $\mathcal{G}$.
    \item [(C)] Neither $X$ nor $Y$ is a random variable with respect to $\mathcal{G}$.
    \item [(D)] Both $X$ and $Y$ are random variables with respect to $\mathcal{G}$.
\end{enumerate} \hfill (GATE ST 2023)\\
\solution
%\input{gate/ST/2023/14/main.tex}
	\item  A die is loaded in such a way that each odd number is twice as likely to occur as
each even number. Find $P(G)$, where $G$ is the event that a number greater than
3 occurs on a single roll of the die.
\\
\solution
		%\input{exemplar/11/16/3/5/main.tex}
	\item All the jacks, queens and kings are removed from a deck of 52 playing cards. The remaining cards are well shuffled and then one card is drawn at random. Giving ace a value 1 similar value for other cards, find the probability that the card has a value 
		\begin{enumerate}
			\item 7
			\item greater than 7
			\item less than 7
		\end{enumerate}
		%\input{exemplar/10/13/3/30/main.tex}
  \item A Lot consists of 48 mobile phones of which 42 are good, 3 have only minor defects and 3 have major defects.Varnika will buy a phone if it is good but the trader will only buy a mobile if it has no major defects. One phone is selected at random from the lot. What is the probability that it is
\begin{enumerate}
	\item acceptable to Varnika?
            \item acceptable to the trader?
\end{enumerate}
\solution
	%\input{exemplar/10/13/3/40/main.tex}
 \item A student says that if you throw a die, it will show up 1 or not 1. Therefore, the probability of getting 1 and the probability of getting 'not 1' each is equal to $\frac{1}{2}$. Is this correct? Give reasons.\\
 \solution
        %\input{exemplar/10/13/2/9/main.tex}
   \item Four candidates A, B, C, D have ap-
plied for the assignment to coach a school cricket
team. If A is twice as likely to be selected as B, and
B and C are given about the same chance of being
selected, while C is twice as likely to be selected
as D, what are the probabilities that
\begin{enumerate}
\item C will be selected?
\item A will not be selected?
\end{enumerate}
	%\input{exemplar/11/16/3/9/main.tex}
 \item A bag contain 24 balls of which $x$ balls are red, $2x$ are white and $3x$ are blue. A ball is selected at random, What is the probability that it is
\begin{enumerate}[label=\alph*)]
\item not red ?
\item white ?
\end{enumerate}
%\input{exemplar/10/13/3/41/main.tex}
If the letters of the word ASSASSINATION are arranged at random. Find the Probability that
\begin{enumerate}[label=(\alph*)]
\item Four $S's$ come consecutively in the word
\item Two  $I's$ and two $N's$ come together
\item All $A's$ are not coming together
\item No two $A's$ are coming together
\end{enumerate}
%\input{exemplar/11/16/3/14/main.tex}
	\item One urn contains two black balls (labelled B1 and B2) and one white ball. A
	second urn contains one black ball and two white balls (labelled W1 and W2).
	Suppose the following experiment is performed. One of the two urns is chosen
	at random. Next a ball is randomly chosen from the urn. Then a second ball is
	chosen at random from the same urn without replacing the first ball.
	
	\begin{enumerate}
	\item What is the probability that two black balls are chosen?
	
	\item What is the probability that two balls of opposite colour are chosen?
	\end{enumerate}
	\solution
	%\input{exemplar/11/16/3/12/main1.tex}
\end{enumerate}

		%
\item 
Out of 100 students, two sections of 40 and 60 are formed. If you and your friend are among the 100 students, what is the probability that
\begin{enumerate}
\item you both enter the same section?
\item you both enter the different sections?
\end{enumerate}
\solution
		%\begin{enumerate}[label=\thesection.\arabic*,ref=\thesection.\theenumi]
	\item One card is drawn from a well-shuffled deck of 52 cards. Find the probability of getting
\begin{enumerate}
\item A king of red colour 
\item A face card 
\item A red face card
\item The jack of hearts
\item A spade
\item The queen of diamonds

\end{enumerate}
\solution
		%\input{ncert/10/15/1/14/main.tex}
	\item Five cards—the ten, jack, queen, king and ace of diamonds, are well-shuffled with their face downwards. One card is then picked up at random.
\begin{enumerate}
\item
What is the probability that the card is the queen? 
\item
If the queen is drawn and put aside, what is the probability that the second card picked up is (a) an ace? (b) a queen?\\
\end{enumerate}
\solution
		%\input{ncert/10/15/1/15/defs.tex}
	\item A bag contains $5$ red balls and some blue balls. If the probability of drawing a blue ball is double that if a red ball, determine the number of blue balls in the bag. 
		\\
\solution
		%\input{ncert/10/15/2/3/defs.tex}
	\item A card is selected from a pack of 52 cards.
 \begin{enumerate}[label=(\alph*)] 
                 \item How many points are there in the sample space?
                 \item Calculate the probability that the card is an ace of spades.
                 \item Calculate the probability that the card is (i) an ace and (ii) black card.
 \end{enumerate}
\solution
		%\input{ncert/11/16/3/4/main.tex}
\item Four cards are drawn from a well-shuffled deck of 52 cards. What is the probability of obtaining 3 diamonds and one spade.
\\
\solution
		%\input{ncert/11/16/4/2/defs.tex}
\item In a certain lottery 10,000 tickets are sold and ten equal prizes are awarded. What is the probability of not getting a prize if you buy (a) one ticket (b) two tickets (c) 10 tickets ?	
\\
\solution
		%\input{ncert/11/16/4/4/defs.tex}
		%
\item 
Out of 100 students, two sections of 40 and 60 are formed. If you and your friend are among the 100 students, what is the probability that
\begin{enumerate}
\item you both enter the same section?
\item you both enter the different sections?
\end{enumerate}
\solution
		%\input{ncert/11/16/4/5/defs.tex}
	\item 
The number lock of a suitcase has 4 wheels each labelled with ten digits i.e. from 0 to 9.The lock opens with a sequence of four digits with no repeats.What is the probability of a person getting the right sequence to open the suitcase.
\\
\solution
		%\input{ncert/11/16/4/10/defs.tex}
		%
\item 
Two cards are drawn at random and without replacement from a pack of 52 playing cards. Find the probability that both the cards are black.
\\
\solution
		%\input{ncert/12/13/2/2/defs.tex}
		\item A box of oranges is inspected by examining three randomly selected oranges drawn without replacement. If all the three oranges are good, the box is approved for sale, otherwise, it is rejected. Find the probability that a box containing 15 oranges out of which 12 are good and 3 are bad ones will be approved for sale.
		\label{ncert/12/13/2/3/defs.tex}
		\item Two balls are drawn at random with replacement from a box containing 10 black and 8 red balls. Find the probability that
		\label{ncert/12/13/2/12}
\begin{enumerate}
\item both balls are red.
\item first ball is black and second is red.
\item one of them is black and other is red.
\end{enumerate}

\item In a hostel, 60\% of the students read Hindi newspaper, 40\% read English newspaper and 20\% read both Hindi and English newspapers. A student is selected at random.
		\label{ncert/12/13/2/15}
\begin{enumerate}
\item Find the probability that she reads neither Hindi nor English newspapers.
\item If she reads Hindi newspaper, find the probability that she reads English newspaper.
\item If she reads English newspaper, find the probability that she reads Hindi newspaper.\\
\end{enumerate}
\item The probability of obtaining an even prime number on each die, when a pair of dice is rolled is 
\begin{enumerate}
    \item $0$ 
    
    \item $\frac{1}{3}$ 
    
    \item $\frac{1}{12}$ 
    
    \item $\frac{1}{36}$ 
\end{enumerate}
\solution
		%\input{ncert/12/13/2/17/defs.tex}
	\item A bag contains 4 red and 4 black balls, another bag contains 2 red and 6 black balls. One of the two bags is selected at random and a ball is drawn from the bag which is found to be red. Find the probability that the ball is drawn from the first bag.
\\
\solution
		%\input{ncert/12/13/3/2/main.tex}
  \item
  Cards with numbers 2 to 101 are placed in a box. A card is selected at random.Find the probability that the card has
\begin{enumerate}[label=(\roman*)]
	\item an even number 
	\item a square number
\end{enumerate}
\solution
%\input{exemplar/10/13/3/32/main.tex}
\item
The king, queen and jack of clubs are removed from a deck of 52 playing cards and then well shuffled. Now one card is drawn at random from the remaining cards.  Determine the probability that the card is
\begin{enumerate}[label=(\roman*)]
\item a club
\item 10 of hearts
\end{enumerate}
\solution
%\input{exemplar/10/13/3/29/main.tex}
\item A team of medical students doing their internship have to assist during surgeries
at a city hospital. The probabilities of surgeries rated as very complex, complex,
routine, simple or very simple are respectively, 0.15, 0.20, 0.31, 0.26, .08. Find
the probabilities that a particular surgery will be rated
\begin{enumerate}
	\item complex or very complex;
	\item neither very complex nor very simple;
	\item routine or complex
	\item routine or simple
\end{enumerate}
\solution
%\input{exemplar/11/16/3/8(1)/main.tex}
\item A card is selected from a pack of 52 cards.
\begin{enumerate}[label=(\alph*)]
    \item How many points are there in the sample space?
    \item Calculate the probability that the card is an ace of spades.
    \item Calculate the probability that the card is (i) an ace and (ii) black card.
\end{enumerate}
\solution
%\input{exemplar/11/16/3/4/main2.tex}
\item The probability that a non leap year selected at random will contain 53 sundays.
\\
\solution
%\input{exemplar/10/13/1/19/main.tex}
\item One of the four persons John, Rita, Aslam or Gurpreet will be promoted next
month. Consequently the sample space consists of four elementary outcomes
S = {John promoted, Rita promoted, Aslam promoted, Gurpreet promoted}
You are told that the chances of John’s promotion is same as that of Gurpreet,
Rita’s chances of promotion are twice as likely as Johns. Aslam’s chances are
four times that of John.
\begin{enumerate}
	\item Determine
	\begin{enumerate}
		\item P (John promoted)
		\item P (Rita promoted)
		\item P (Aslam promoted)
		\item P (Gurpreet promoted)
	\end{enumerate}
	\item If A = {John promoted or Gurpreet promoted}, find P (A).
\end{enumerate}
\solution
%\input{exemplar/11/16/3/10/main.tex}
\item A card is drawn from a deck of 52 cards. Find the probability of getting a king or a heart or a red card.\\
\solution
%\input{exemplar/11/16/3/15/main.tex}
\item The probability that a student will pass his examination is 0.73, the probability of
the student getting a compartment is 0.13, and the probability that the student will
either pass or get compartment is 0.96. State True or False.\\
\solution
%\input{exemplar/11/16/3/31/main.tex}
\item A card is selected from a pack of 52 cards\\
\begin{enumerate}[label=(\alph*)]
\item How many points are there in the sample space?
\item Calculate the probability that the cards is an ace of spades.
\item Calculate the probability that the card is (i) an ace (ii)black card.\\
\end{enumerate}
%\input{ncert/11/16/3/4_1/Prob_4.tex}
\item In a non-leap year, the probability of having 53 tuesdays or 53 wednesdays is\\
\solution
%\input{exemplar/11/16/3/18/main.tex}
\item There are 1000 sealed envelopes in a box, 10 of them contain a cash prize of
Rs 100 each, 100 of them contain a cash prize of Rs 50 each and 200 of them
contain a cash prize of Rs 10 each and rest do not contain any cash prize. If they
are well shuffled and an envelope is picked up out, what is the probability that it
contains no cash prize?\\
\solution
%\input{exemplar/10/13/3/34/main.tex}
\item 
A die is thrown and a card is selected at random from a deck of 52 playing cards. The probability of getting an even number on the die and a spade card.\\
\solution
%\input{exemplar/12/13/3/78/main.tex}
\item
If 4-digit numbers greater than 5,000 are randomly formed from the digits 0, 1, 3, 5, and 7, what is the probability of forming a number divisible by 5 when:
\begin{enumerate}
    \item The digits are repeated?
    \item The repetition of digits is not allowed?
\end{enumerate}
\solution
%\input{ncert/11/16/4/9/main.tex}
\item Consider the probability space $\brak{\Omega, \mathcal{G}, P}$ where $\Omega = [0,2]$ and $\mathcal{G} = \cbrak{\phi, \Omega, [0,1], (1,2]}$. Let $X$ and $Y$ be two functions on $\Omega$ defined as
\begin{align*}
    X(\omega) = 
    \begin{cases}
        1 & \text{if }\omega \in [0, 1]\\
        2 & \text{if }\omega \in (1, 2]
    \end{cases}
\end{align*}
and
\begin{align*}
    Y(\omega) = 
    \begin{cases}
        2 & \text{if }\omega \in [0, 1.5]\\
        3 & \text{if }\omega \in (1.5, 2].
    \end{cases}
\end{align*}
Then which one of the following statements is true?
\begin{enumerate}
    \item [(A)] $X$ is a random variable with respect to $\mathcal{G}$, but $Y$ is not a random variable with respect to $\mathcal{G}$.
    \item [(B)] $Y$ is a random variable with respect to $\mathcal{G}$, but $X$ is not a random variable with respect to $\mathcal{G}$.
    \item [(C)] Neither $X$ nor $Y$ is a random variable with respect to $\mathcal{G}$.
    \item [(D)] Both $X$ and $Y$ are random variables with respect to $\mathcal{G}$.
\end{enumerate} \hfill (GATE ST 2023)\\
\solution
%\input{gate/ST/2023/14/main.tex}
	\item  A die is loaded in such a way that each odd number is twice as likely to occur as
each even number. Find $P(G)$, where $G$ is the event that a number greater than
3 occurs on a single roll of the die.
\\
\solution
		%\input{exemplar/11/16/3/5/main.tex}
	\item All the jacks, queens and kings are removed from a deck of 52 playing cards. The remaining cards are well shuffled and then one card is drawn at random. Giving ace a value 1 similar value for other cards, find the probability that the card has a value 
		\begin{enumerate}
			\item 7
			\item greater than 7
			\item less than 7
		\end{enumerate}
		%\input{exemplar/10/13/3/30/main.tex}
  \item A Lot consists of 48 mobile phones of which 42 are good, 3 have only minor defects and 3 have major defects.Varnika will buy a phone if it is good but the trader will only buy a mobile if it has no major defects. One phone is selected at random from the lot. What is the probability that it is
\begin{enumerate}
	\item acceptable to Varnika?
            \item acceptable to the trader?
\end{enumerate}
\solution
	%\input{exemplar/10/13/3/40/main.tex}
 \item A student says that if you throw a die, it will show up 1 or not 1. Therefore, the probability of getting 1 and the probability of getting 'not 1' each is equal to $\frac{1}{2}$. Is this correct? Give reasons.\\
 \solution
        %\input{exemplar/10/13/2/9/main.tex}
   \item Four candidates A, B, C, D have ap-
plied for the assignment to coach a school cricket
team. If A is twice as likely to be selected as B, and
B and C are given about the same chance of being
selected, while C is twice as likely to be selected
as D, what are the probabilities that
\begin{enumerate}
\item C will be selected?
\item A will not be selected?
\end{enumerate}
	%\input{exemplar/11/16/3/9/main.tex}
 \item A bag contain 24 balls of which $x$ balls are red, $2x$ are white and $3x$ are blue. A ball is selected at random, What is the probability that it is
\begin{enumerate}[label=\alph*)]
\item not red ?
\item white ?
\end{enumerate}
%\input{exemplar/10/13/3/41/main.tex}
If the letters of the word ASSASSINATION are arranged at random. Find the Probability that
\begin{enumerate}[label=(\alph*)]
\item Four $S's$ come consecutively in the word
\item Two  $I's$ and two $N's$ come together
\item All $A's$ are not coming together
\item No two $A's$ are coming together
\end{enumerate}
%\input{exemplar/11/16/3/14/main.tex}
	\item One urn contains two black balls (labelled B1 and B2) and one white ball. A
	second urn contains one black ball and two white balls (labelled W1 and W2).
	Suppose the following experiment is performed. One of the two urns is chosen
	at random. Next a ball is randomly chosen from the urn. Then a second ball is
	chosen at random from the same urn without replacing the first ball.
	
	\begin{enumerate}
	\item What is the probability that two black balls are chosen?
	
	\item What is the probability that two balls of opposite colour are chosen?
	\end{enumerate}
	\solution
	%\input{exemplar/11/16/3/12/main1.tex}
\end{enumerate}

	\item 
The number lock of a suitcase has 4 wheels each labelled with ten digits i.e. from 0 to 9.The lock opens with a sequence of four digits with no repeats.What is the probability of a person getting the right sequence to open the suitcase.
\\
\solution
		%\begin{enumerate}[label=\thesection.\arabic*,ref=\thesection.\theenumi]
	\item One card is drawn from a well-shuffled deck of 52 cards. Find the probability of getting
\begin{enumerate}
\item A king of red colour 
\item A face card 
\item A red face card
\item The jack of hearts
\item A spade
\item The queen of diamonds

\end{enumerate}
\solution
		%\input{ncert/10/15/1/14/main.tex}
	\item Five cards—the ten, jack, queen, king and ace of diamonds, are well-shuffled with their face downwards. One card is then picked up at random.
\begin{enumerate}
\item
What is the probability that the card is the queen? 
\item
If the queen is drawn and put aside, what is the probability that the second card picked up is (a) an ace? (b) a queen?\\
\end{enumerate}
\solution
		%\input{ncert/10/15/1/15/defs.tex}
	\item A bag contains $5$ red balls and some blue balls. If the probability of drawing a blue ball is double that if a red ball, determine the number of blue balls in the bag. 
		\\
\solution
		%\input{ncert/10/15/2/3/defs.tex}
	\item A card is selected from a pack of 52 cards.
 \begin{enumerate}[label=(\alph*)] 
                 \item How many points are there in the sample space?
                 \item Calculate the probability that the card is an ace of spades.
                 \item Calculate the probability that the card is (i) an ace and (ii) black card.
 \end{enumerate}
\solution
		%\input{ncert/11/16/3/4/main.tex}
\item Four cards are drawn from a well-shuffled deck of 52 cards. What is the probability of obtaining 3 diamonds and one spade.
\\
\solution
		%\input{ncert/11/16/4/2/defs.tex}
\item In a certain lottery 10,000 tickets are sold and ten equal prizes are awarded. What is the probability of not getting a prize if you buy (a) one ticket (b) two tickets (c) 10 tickets ?	
\\
\solution
		%\input{ncert/11/16/4/4/defs.tex}
		%
\item 
Out of 100 students, two sections of 40 and 60 are formed. If you and your friend are among the 100 students, what is the probability that
\begin{enumerate}
\item you both enter the same section?
\item you both enter the different sections?
\end{enumerate}
\solution
		%\input{ncert/11/16/4/5/defs.tex}
	\item 
The number lock of a suitcase has 4 wheels each labelled with ten digits i.e. from 0 to 9.The lock opens with a sequence of four digits with no repeats.What is the probability of a person getting the right sequence to open the suitcase.
\\
\solution
		%\input{ncert/11/16/4/10/defs.tex}
		%
\item 
Two cards are drawn at random and without replacement from a pack of 52 playing cards. Find the probability that both the cards are black.
\\
\solution
		%\input{ncert/12/13/2/2/defs.tex}
		\item A box of oranges is inspected by examining three randomly selected oranges drawn without replacement. If all the three oranges are good, the box is approved for sale, otherwise, it is rejected. Find the probability that a box containing 15 oranges out of which 12 are good and 3 are bad ones will be approved for sale.
		\label{ncert/12/13/2/3/defs.tex}
		\item Two balls are drawn at random with replacement from a box containing 10 black and 8 red balls. Find the probability that
		\label{ncert/12/13/2/12}
\begin{enumerate}
\item both balls are red.
\item first ball is black and second is red.
\item one of them is black and other is red.
\end{enumerate}

\item In a hostel, 60\% of the students read Hindi newspaper, 40\% read English newspaper and 20\% read both Hindi and English newspapers. A student is selected at random.
		\label{ncert/12/13/2/15}
\begin{enumerate}
\item Find the probability that she reads neither Hindi nor English newspapers.
\item If she reads Hindi newspaper, find the probability that she reads English newspaper.
\item If she reads English newspaper, find the probability that she reads Hindi newspaper.\\
\end{enumerate}
\item The probability of obtaining an even prime number on each die, when a pair of dice is rolled is 
\begin{enumerate}
    \item $0$ 
    
    \item $\frac{1}{3}$ 
    
    \item $\frac{1}{12}$ 
    
    \item $\frac{1}{36}$ 
\end{enumerate}
\solution
		%\input{ncert/12/13/2/17/defs.tex}
	\item A bag contains 4 red and 4 black balls, another bag contains 2 red and 6 black balls. One of the two bags is selected at random and a ball is drawn from the bag which is found to be red. Find the probability that the ball is drawn from the first bag.
\\
\solution
		%\input{ncert/12/13/3/2/main.tex}
  \item
  Cards with numbers 2 to 101 are placed in a box. A card is selected at random.Find the probability that the card has
\begin{enumerate}[label=(\roman*)]
	\item an even number 
	\item a square number
\end{enumerate}
\solution
%\input{exemplar/10/13/3/32/main.tex}
\item
The king, queen and jack of clubs are removed from a deck of 52 playing cards and then well shuffled. Now one card is drawn at random from the remaining cards.  Determine the probability that the card is
\begin{enumerate}[label=(\roman*)]
\item a club
\item 10 of hearts
\end{enumerate}
\solution
%\input{exemplar/10/13/3/29/main.tex}
\item A team of medical students doing their internship have to assist during surgeries
at a city hospital. The probabilities of surgeries rated as very complex, complex,
routine, simple or very simple are respectively, 0.15, 0.20, 0.31, 0.26, .08. Find
the probabilities that a particular surgery will be rated
\begin{enumerate}
	\item complex or very complex;
	\item neither very complex nor very simple;
	\item routine or complex
	\item routine or simple
\end{enumerate}
\solution
%\input{exemplar/11/16/3/8(1)/main.tex}
\item A card is selected from a pack of 52 cards.
\begin{enumerate}[label=(\alph*)]
    \item How many points are there in the sample space?
    \item Calculate the probability that the card is an ace of spades.
    \item Calculate the probability that the card is (i) an ace and (ii) black card.
\end{enumerate}
\solution
%\input{exemplar/11/16/3/4/main2.tex}
\item The probability that a non leap year selected at random will contain 53 sundays.
\\
\solution
%\input{exemplar/10/13/1/19/main.tex}
\item One of the four persons John, Rita, Aslam or Gurpreet will be promoted next
month. Consequently the sample space consists of four elementary outcomes
S = {John promoted, Rita promoted, Aslam promoted, Gurpreet promoted}
You are told that the chances of John’s promotion is same as that of Gurpreet,
Rita’s chances of promotion are twice as likely as Johns. Aslam’s chances are
four times that of John.
\begin{enumerate}
	\item Determine
	\begin{enumerate}
		\item P (John promoted)
		\item P (Rita promoted)
		\item P (Aslam promoted)
		\item P (Gurpreet promoted)
	\end{enumerate}
	\item If A = {John promoted or Gurpreet promoted}, find P (A).
\end{enumerate}
\solution
%\input{exemplar/11/16/3/10/main.tex}
\item A card is drawn from a deck of 52 cards. Find the probability of getting a king or a heart or a red card.\\
\solution
%\input{exemplar/11/16/3/15/main.tex}
\item The probability that a student will pass his examination is 0.73, the probability of
the student getting a compartment is 0.13, and the probability that the student will
either pass or get compartment is 0.96. State True or False.\\
\solution
%\input{exemplar/11/16/3/31/main.tex}
\item A card is selected from a pack of 52 cards\\
\begin{enumerate}[label=(\alph*)]
\item How many points are there in the sample space?
\item Calculate the probability that the cards is an ace of spades.
\item Calculate the probability that the card is (i) an ace (ii)black card.\\
\end{enumerate}
%\input{ncert/11/16/3/4_1/Prob_4.tex}
\item In a non-leap year, the probability of having 53 tuesdays or 53 wednesdays is\\
\solution
%\input{exemplar/11/16/3/18/main.tex}
\item There are 1000 sealed envelopes in a box, 10 of them contain a cash prize of
Rs 100 each, 100 of them contain a cash prize of Rs 50 each and 200 of them
contain a cash prize of Rs 10 each and rest do not contain any cash prize. If they
are well shuffled and an envelope is picked up out, what is the probability that it
contains no cash prize?\\
\solution
%\input{exemplar/10/13/3/34/main.tex}
\item 
A die is thrown and a card is selected at random from a deck of 52 playing cards. The probability of getting an even number on the die and a spade card.\\
\solution
%\input{exemplar/12/13/3/78/main.tex}
\item
If 4-digit numbers greater than 5,000 are randomly formed from the digits 0, 1, 3, 5, and 7, what is the probability of forming a number divisible by 5 when:
\begin{enumerate}
    \item The digits are repeated?
    \item The repetition of digits is not allowed?
\end{enumerate}
\solution
%\input{ncert/11/16/4/9/main.tex}
\item Consider the probability space $\brak{\Omega, \mathcal{G}, P}$ where $\Omega = [0,2]$ and $\mathcal{G} = \cbrak{\phi, \Omega, [0,1], (1,2]}$. Let $X$ and $Y$ be two functions on $\Omega$ defined as
\begin{align*}
    X(\omega) = 
    \begin{cases}
        1 & \text{if }\omega \in [0, 1]\\
        2 & \text{if }\omega \in (1, 2]
    \end{cases}
\end{align*}
and
\begin{align*}
    Y(\omega) = 
    \begin{cases}
        2 & \text{if }\omega \in [0, 1.5]\\
        3 & \text{if }\omega \in (1.5, 2].
    \end{cases}
\end{align*}
Then which one of the following statements is true?
\begin{enumerate}
    \item [(A)] $X$ is a random variable with respect to $\mathcal{G}$, but $Y$ is not a random variable with respect to $\mathcal{G}$.
    \item [(B)] $Y$ is a random variable with respect to $\mathcal{G}$, but $X$ is not a random variable with respect to $\mathcal{G}$.
    \item [(C)] Neither $X$ nor $Y$ is a random variable with respect to $\mathcal{G}$.
    \item [(D)] Both $X$ and $Y$ are random variables with respect to $\mathcal{G}$.
\end{enumerate} \hfill (GATE ST 2023)\\
\solution
%\input{gate/ST/2023/14/main.tex}
	\item  A die is loaded in such a way that each odd number is twice as likely to occur as
each even number. Find $P(G)$, where $G$ is the event that a number greater than
3 occurs on a single roll of the die.
\\
\solution
		%\input{exemplar/11/16/3/5/main.tex}
	\item All the jacks, queens and kings are removed from a deck of 52 playing cards. The remaining cards are well shuffled and then one card is drawn at random. Giving ace a value 1 similar value for other cards, find the probability that the card has a value 
		\begin{enumerate}
			\item 7
			\item greater than 7
			\item less than 7
		\end{enumerate}
		%\input{exemplar/10/13/3/30/main.tex}
  \item A Lot consists of 48 mobile phones of which 42 are good, 3 have only minor defects and 3 have major defects.Varnika will buy a phone if it is good but the trader will only buy a mobile if it has no major defects. One phone is selected at random from the lot. What is the probability that it is
\begin{enumerate}
	\item acceptable to Varnika?
            \item acceptable to the trader?
\end{enumerate}
\solution
	%\input{exemplar/10/13/3/40/main.tex}
 \item A student says that if you throw a die, it will show up 1 or not 1. Therefore, the probability of getting 1 and the probability of getting 'not 1' each is equal to $\frac{1}{2}$. Is this correct? Give reasons.\\
 \solution
        %\input{exemplar/10/13/2/9/main.tex}
   \item Four candidates A, B, C, D have ap-
plied for the assignment to coach a school cricket
team. If A is twice as likely to be selected as B, and
B and C are given about the same chance of being
selected, while C is twice as likely to be selected
as D, what are the probabilities that
\begin{enumerate}
\item C will be selected?
\item A will not be selected?
\end{enumerate}
	%\input{exemplar/11/16/3/9/main.tex}
 \item A bag contain 24 balls of which $x$ balls are red, $2x$ are white and $3x$ are blue. A ball is selected at random, What is the probability that it is
\begin{enumerate}[label=\alph*)]
\item not red ?
\item white ?
\end{enumerate}
%\input{exemplar/10/13/3/41/main.tex}
If the letters of the word ASSASSINATION are arranged at random. Find the Probability that
\begin{enumerate}[label=(\alph*)]
\item Four $S's$ come consecutively in the word
\item Two  $I's$ and two $N's$ come together
\item All $A's$ are not coming together
\item No two $A's$ are coming together
\end{enumerate}
%\input{exemplar/11/16/3/14/main.tex}
	\item One urn contains two black balls (labelled B1 and B2) and one white ball. A
	second urn contains one black ball and two white balls (labelled W1 and W2).
	Suppose the following experiment is performed. One of the two urns is chosen
	at random. Next a ball is randomly chosen from the urn. Then a second ball is
	chosen at random from the same urn without replacing the first ball.
	
	\begin{enumerate}
	\item What is the probability that two black balls are chosen?
	
	\item What is the probability that two balls of opposite colour are chosen?
	\end{enumerate}
	\solution
	%\input{exemplar/11/16/3/12/main1.tex}
\end{enumerate}

		%
\item 
Two cards are drawn at random and without replacement from a pack of 52 playing cards. Find the probability that both the cards are black.
\\
\solution
		%\begin{enumerate}[label=\thesection.\arabic*,ref=\thesection.\theenumi]
	\item One card is drawn from a well-shuffled deck of 52 cards. Find the probability of getting
\begin{enumerate}
\item A king of red colour 
\item A face card 
\item A red face card
\item The jack of hearts
\item A spade
\item The queen of diamonds

\end{enumerate}
\solution
		%\input{ncert/10/15/1/14/main.tex}
	\item Five cards—the ten, jack, queen, king and ace of diamonds, are well-shuffled with their face downwards. One card is then picked up at random.
\begin{enumerate}
\item
What is the probability that the card is the queen? 
\item
If the queen is drawn and put aside, what is the probability that the second card picked up is (a) an ace? (b) a queen?\\
\end{enumerate}
\solution
		%\input{ncert/10/15/1/15/defs.tex}
	\item A bag contains $5$ red balls and some blue balls. If the probability of drawing a blue ball is double that if a red ball, determine the number of blue balls in the bag. 
		\\
\solution
		%\input{ncert/10/15/2/3/defs.tex}
	\item A card is selected from a pack of 52 cards.
 \begin{enumerate}[label=(\alph*)] 
                 \item How many points are there in the sample space?
                 \item Calculate the probability that the card is an ace of spades.
                 \item Calculate the probability that the card is (i) an ace and (ii) black card.
 \end{enumerate}
\solution
		%\input{ncert/11/16/3/4/main.tex}
\item Four cards are drawn from a well-shuffled deck of 52 cards. What is the probability of obtaining 3 diamonds and one spade.
\\
\solution
		%\input{ncert/11/16/4/2/defs.tex}
\item In a certain lottery 10,000 tickets are sold and ten equal prizes are awarded. What is the probability of not getting a prize if you buy (a) one ticket (b) two tickets (c) 10 tickets ?	
\\
\solution
		%\input{ncert/11/16/4/4/defs.tex}
		%
\item 
Out of 100 students, two sections of 40 and 60 are formed. If you and your friend are among the 100 students, what is the probability that
\begin{enumerate}
\item you both enter the same section?
\item you both enter the different sections?
\end{enumerate}
\solution
		%\input{ncert/11/16/4/5/defs.tex}
	\item 
The number lock of a suitcase has 4 wheels each labelled with ten digits i.e. from 0 to 9.The lock opens with a sequence of four digits with no repeats.What is the probability of a person getting the right sequence to open the suitcase.
\\
\solution
		%\input{ncert/11/16/4/10/defs.tex}
		%
\item 
Two cards are drawn at random and without replacement from a pack of 52 playing cards. Find the probability that both the cards are black.
\\
\solution
		%\input{ncert/12/13/2/2/defs.tex}
		\item A box of oranges is inspected by examining three randomly selected oranges drawn without replacement. If all the three oranges are good, the box is approved for sale, otherwise, it is rejected. Find the probability that a box containing 15 oranges out of which 12 are good and 3 are bad ones will be approved for sale.
		\label{ncert/12/13/2/3/defs.tex}
		\item Two balls are drawn at random with replacement from a box containing 10 black and 8 red balls. Find the probability that
		\label{ncert/12/13/2/12}
\begin{enumerate}
\item both balls are red.
\item first ball is black and second is red.
\item one of them is black and other is red.
\end{enumerate}

\item In a hostel, 60\% of the students read Hindi newspaper, 40\% read English newspaper and 20\% read both Hindi and English newspapers. A student is selected at random.
		\label{ncert/12/13/2/15}
\begin{enumerate}
\item Find the probability that she reads neither Hindi nor English newspapers.
\item If she reads Hindi newspaper, find the probability that she reads English newspaper.
\item If she reads English newspaper, find the probability that she reads Hindi newspaper.\\
\end{enumerate}
\item The probability of obtaining an even prime number on each die, when a pair of dice is rolled is 
\begin{enumerate}
    \item $0$ 
    
    \item $\frac{1}{3}$ 
    
    \item $\frac{1}{12}$ 
    
    \item $\frac{1}{36}$ 
\end{enumerate}
\solution
		%\input{ncert/12/13/2/17/defs.tex}
	\item A bag contains 4 red and 4 black balls, another bag contains 2 red and 6 black balls. One of the two bags is selected at random and a ball is drawn from the bag which is found to be red. Find the probability that the ball is drawn from the first bag.
\\
\solution
		%\input{ncert/12/13/3/2/main.tex}
  \item
  Cards with numbers 2 to 101 are placed in a box. A card is selected at random.Find the probability that the card has
\begin{enumerate}[label=(\roman*)]
	\item an even number 
	\item a square number
\end{enumerate}
\solution
%\input{exemplar/10/13/3/32/main.tex}
\item
The king, queen and jack of clubs are removed from a deck of 52 playing cards and then well shuffled. Now one card is drawn at random from the remaining cards.  Determine the probability that the card is
\begin{enumerate}[label=(\roman*)]
\item a club
\item 10 of hearts
\end{enumerate}
\solution
%\input{exemplar/10/13/3/29/main.tex}
\item A team of medical students doing their internship have to assist during surgeries
at a city hospital. The probabilities of surgeries rated as very complex, complex,
routine, simple or very simple are respectively, 0.15, 0.20, 0.31, 0.26, .08. Find
the probabilities that a particular surgery will be rated
\begin{enumerate}
	\item complex or very complex;
	\item neither very complex nor very simple;
	\item routine or complex
	\item routine or simple
\end{enumerate}
\solution
%\input{exemplar/11/16/3/8(1)/main.tex}
\item A card is selected from a pack of 52 cards.
\begin{enumerate}[label=(\alph*)]
    \item How many points are there in the sample space?
    \item Calculate the probability that the card is an ace of spades.
    \item Calculate the probability that the card is (i) an ace and (ii) black card.
\end{enumerate}
\solution
%\input{exemplar/11/16/3/4/main2.tex}
\item The probability that a non leap year selected at random will contain 53 sundays.
\\
\solution
%\input{exemplar/10/13/1/19/main.tex}
\item One of the four persons John, Rita, Aslam or Gurpreet will be promoted next
month. Consequently the sample space consists of four elementary outcomes
S = {John promoted, Rita promoted, Aslam promoted, Gurpreet promoted}
You are told that the chances of John’s promotion is same as that of Gurpreet,
Rita’s chances of promotion are twice as likely as Johns. Aslam’s chances are
four times that of John.
\begin{enumerate}
	\item Determine
	\begin{enumerate}
		\item P (John promoted)
		\item P (Rita promoted)
		\item P (Aslam promoted)
		\item P (Gurpreet promoted)
	\end{enumerate}
	\item If A = {John promoted or Gurpreet promoted}, find P (A).
\end{enumerate}
\solution
%\input{exemplar/11/16/3/10/main.tex}
\item A card is drawn from a deck of 52 cards. Find the probability of getting a king or a heart or a red card.\\
\solution
%\input{exemplar/11/16/3/15/main.tex}
\item The probability that a student will pass his examination is 0.73, the probability of
the student getting a compartment is 0.13, and the probability that the student will
either pass or get compartment is 0.96. State True or False.\\
\solution
%\input{exemplar/11/16/3/31/main.tex}
\item A card is selected from a pack of 52 cards\\
\begin{enumerate}[label=(\alph*)]
\item How many points are there in the sample space?
\item Calculate the probability that the cards is an ace of spades.
\item Calculate the probability that the card is (i) an ace (ii)black card.\\
\end{enumerate}
%\input{ncert/11/16/3/4_1/Prob_4.tex}
\item In a non-leap year, the probability of having 53 tuesdays or 53 wednesdays is\\
\solution
%\input{exemplar/11/16/3/18/main.tex}
\item There are 1000 sealed envelopes in a box, 10 of them contain a cash prize of
Rs 100 each, 100 of them contain a cash prize of Rs 50 each and 200 of them
contain a cash prize of Rs 10 each and rest do not contain any cash prize. If they
are well shuffled and an envelope is picked up out, what is the probability that it
contains no cash prize?\\
\solution
%\input{exemplar/10/13/3/34/main.tex}
\item 
A die is thrown and a card is selected at random from a deck of 52 playing cards. The probability of getting an even number on the die and a spade card.\\
\solution
%\input{exemplar/12/13/3/78/main.tex}
\item
If 4-digit numbers greater than 5,000 are randomly formed from the digits 0, 1, 3, 5, and 7, what is the probability of forming a number divisible by 5 when:
\begin{enumerate}
    \item The digits are repeated?
    \item The repetition of digits is not allowed?
\end{enumerate}
\solution
%\input{ncert/11/16/4/9/main.tex}
\item Consider the probability space $\brak{\Omega, \mathcal{G}, P}$ where $\Omega = [0,2]$ and $\mathcal{G} = \cbrak{\phi, \Omega, [0,1], (1,2]}$. Let $X$ and $Y$ be two functions on $\Omega$ defined as
\begin{align*}
    X(\omega) = 
    \begin{cases}
        1 & \text{if }\omega \in [0, 1]\\
        2 & \text{if }\omega \in (1, 2]
    \end{cases}
\end{align*}
and
\begin{align*}
    Y(\omega) = 
    \begin{cases}
        2 & \text{if }\omega \in [0, 1.5]\\
        3 & \text{if }\omega \in (1.5, 2].
    \end{cases}
\end{align*}
Then which one of the following statements is true?
\begin{enumerate}
    \item [(A)] $X$ is a random variable with respect to $\mathcal{G}$, but $Y$ is not a random variable with respect to $\mathcal{G}$.
    \item [(B)] $Y$ is a random variable with respect to $\mathcal{G}$, but $X$ is not a random variable with respect to $\mathcal{G}$.
    \item [(C)] Neither $X$ nor $Y$ is a random variable with respect to $\mathcal{G}$.
    \item [(D)] Both $X$ and $Y$ are random variables with respect to $\mathcal{G}$.
\end{enumerate} \hfill (GATE ST 2023)\\
\solution
%\input{gate/ST/2023/14/main.tex}
	\item  A die is loaded in such a way that each odd number is twice as likely to occur as
each even number. Find $P(G)$, where $G$ is the event that a number greater than
3 occurs on a single roll of the die.
\\
\solution
		%\input{exemplar/11/16/3/5/main.tex}
	\item All the jacks, queens and kings are removed from a deck of 52 playing cards. The remaining cards are well shuffled and then one card is drawn at random. Giving ace a value 1 similar value for other cards, find the probability that the card has a value 
		\begin{enumerate}
			\item 7
			\item greater than 7
			\item less than 7
		\end{enumerate}
		%\input{exemplar/10/13/3/30/main.tex}
  \item A Lot consists of 48 mobile phones of which 42 are good, 3 have only minor defects and 3 have major defects.Varnika will buy a phone if it is good but the trader will only buy a mobile if it has no major defects. One phone is selected at random from the lot. What is the probability that it is
\begin{enumerate}
	\item acceptable to Varnika?
            \item acceptable to the trader?
\end{enumerate}
\solution
	%\input{exemplar/10/13/3/40/main.tex}
 \item A student says that if you throw a die, it will show up 1 or not 1. Therefore, the probability of getting 1 and the probability of getting 'not 1' each is equal to $\frac{1}{2}$. Is this correct? Give reasons.\\
 \solution
        %\input{exemplar/10/13/2/9/main.tex}
   \item Four candidates A, B, C, D have ap-
plied for the assignment to coach a school cricket
team. If A is twice as likely to be selected as B, and
B and C are given about the same chance of being
selected, while C is twice as likely to be selected
as D, what are the probabilities that
\begin{enumerate}
\item C will be selected?
\item A will not be selected?
\end{enumerate}
	%\input{exemplar/11/16/3/9/main.tex}
 \item A bag contain 24 balls of which $x$ balls are red, $2x$ are white and $3x$ are blue. A ball is selected at random, What is the probability that it is
\begin{enumerate}[label=\alph*)]
\item not red ?
\item white ?
\end{enumerate}
%\input{exemplar/10/13/3/41/main.tex}
If the letters of the word ASSASSINATION are arranged at random. Find the Probability that
\begin{enumerate}[label=(\alph*)]
\item Four $S's$ come consecutively in the word
\item Two  $I's$ and two $N's$ come together
\item All $A's$ are not coming together
\item No two $A's$ are coming together
\end{enumerate}
%\input{exemplar/11/16/3/14/main.tex}
	\item One urn contains two black balls (labelled B1 and B2) and one white ball. A
	second urn contains one black ball and two white balls (labelled W1 and W2).
	Suppose the following experiment is performed. One of the two urns is chosen
	at random. Next a ball is randomly chosen from the urn. Then a second ball is
	chosen at random from the same urn without replacing the first ball.
	
	\begin{enumerate}
	\item What is the probability that two black balls are chosen?
	
	\item What is the probability that two balls of opposite colour are chosen?
	\end{enumerate}
	\solution
	%\input{exemplar/11/16/3/12/main1.tex}
\end{enumerate}

		\item A box of oranges is inspected by examining three randomly selected oranges drawn without replacement. If all the three oranges are good, the box is approved for sale, otherwise, it is rejected. Find the probability that a box containing 15 oranges out of which 12 are good and 3 are bad ones will be approved for sale.
		\label{ncert/12/13/2/3/defs.tex}
		\item Two balls are drawn at random with replacement from a box containing 10 black and 8 red balls. Find the probability that
		\label{ncert/12/13/2/12}
\begin{enumerate}
\item both balls are red.
\item first ball is black and second is red.
\item one of them is black and other is red.
\end{enumerate}

\item In a hostel, 60\% of the students read Hindi newspaper, 40\% read English newspaper and 20\% read both Hindi and English newspapers. A student is selected at random.
		\label{ncert/12/13/2/15}
\begin{enumerate}
\item Find the probability that she reads neither Hindi nor English newspapers.
\item If she reads Hindi newspaper, find the probability that she reads English newspaper.
\item If she reads English newspaper, find the probability that she reads Hindi newspaper.\\
\end{enumerate}
\item The probability of obtaining an even prime number on each die, when a pair of dice is rolled is 
\begin{enumerate}
    \item $0$ 
    
    \item $\frac{1}{3}$ 
    
    \item $\frac{1}{12}$ 
    
    \item $\frac{1}{36}$ 
\end{enumerate}
\solution
		%\begin{enumerate}[label=\thesection.\arabic*,ref=\thesection.\theenumi]
	\item One card is drawn from a well-shuffled deck of 52 cards. Find the probability of getting
\begin{enumerate}
\item A king of red colour 
\item A face card 
\item A red face card
\item The jack of hearts
\item A spade
\item The queen of diamonds

\end{enumerate}
\solution
		%\input{ncert/10/15/1/14/main.tex}
	\item Five cards—the ten, jack, queen, king and ace of diamonds, are well-shuffled with their face downwards. One card is then picked up at random.
\begin{enumerate}
\item
What is the probability that the card is the queen? 
\item
If the queen is drawn and put aside, what is the probability that the second card picked up is (a) an ace? (b) a queen?\\
\end{enumerate}
\solution
		%\input{ncert/10/15/1/15/defs.tex}
	\item A bag contains $5$ red balls and some blue balls. If the probability of drawing a blue ball is double that if a red ball, determine the number of blue balls in the bag. 
		\\
\solution
		%\input{ncert/10/15/2/3/defs.tex}
	\item A card is selected from a pack of 52 cards.
 \begin{enumerate}[label=(\alph*)] 
                 \item How many points are there in the sample space?
                 \item Calculate the probability that the card is an ace of spades.
                 \item Calculate the probability that the card is (i) an ace and (ii) black card.
 \end{enumerate}
\solution
		%\input{ncert/11/16/3/4/main.tex}
\item Four cards are drawn from a well-shuffled deck of 52 cards. What is the probability of obtaining 3 diamonds and one spade.
\\
\solution
		%\input{ncert/11/16/4/2/defs.tex}
\item In a certain lottery 10,000 tickets are sold and ten equal prizes are awarded. What is the probability of not getting a prize if you buy (a) one ticket (b) two tickets (c) 10 tickets ?	
\\
\solution
		%\input{ncert/11/16/4/4/defs.tex}
		%
\item 
Out of 100 students, two sections of 40 and 60 are formed. If you and your friend are among the 100 students, what is the probability that
\begin{enumerate}
\item you both enter the same section?
\item you both enter the different sections?
\end{enumerate}
\solution
		%\input{ncert/11/16/4/5/defs.tex}
	\item 
The number lock of a suitcase has 4 wheels each labelled with ten digits i.e. from 0 to 9.The lock opens with a sequence of four digits with no repeats.What is the probability of a person getting the right sequence to open the suitcase.
\\
\solution
		%\input{ncert/11/16/4/10/defs.tex}
		%
\item 
Two cards are drawn at random and without replacement from a pack of 52 playing cards. Find the probability that both the cards are black.
\\
\solution
		%\input{ncert/12/13/2/2/defs.tex}
		\item A box of oranges is inspected by examining three randomly selected oranges drawn without replacement. If all the three oranges are good, the box is approved for sale, otherwise, it is rejected. Find the probability that a box containing 15 oranges out of which 12 are good and 3 are bad ones will be approved for sale.
		\label{ncert/12/13/2/3/defs.tex}
		\item Two balls are drawn at random with replacement from a box containing 10 black and 8 red balls. Find the probability that
		\label{ncert/12/13/2/12}
\begin{enumerate}
\item both balls are red.
\item first ball is black and second is red.
\item one of them is black and other is red.
\end{enumerate}

\item In a hostel, 60\% of the students read Hindi newspaper, 40\% read English newspaper and 20\% read both Hindi and English newspapers. A student is selected at random.
		\label{ncert/12/13/2/15}
\begin{enumerate}
\item Find the probability that she reads neither Hindi nor English newspapers.
\item If she reads Hindi newspaper, find the probability that she reads English newspaper.
\item If she reads English newspaper, find the probability that she reads Hindi newspaper.\\
\end{enumerate}
\item The probability of obtaining an even prime number on each die, when a pair of dice is rolled is 
\begin{enumerate}
    \item $0$ 
    
    \item $\frac{1}{3}$ 
    
    \item $\frac{1}{12}$ 
    
    \item $\frac{1}{36}$ 
\end{enumerate}
\solution
		%\input{ncert/12/13/2/17/defs.tex}
	\item A bag contains 4 red and 4 black balls, another bag contains 2 red and 6 black balls. One of the two bags is selected at random and a ball is drawn from the bag which is found to be red. Find the probability that the ball is drawn from the first bag.
\\
\solution
		%\input{ncert/12/13/3/2/main.tex}
  \item
  Cards with numbers 2 to 101 are placed in a box. A card is selected at random.Find the probability that the card has
\begin{enumerate}[label=(\roman*)]
	\item an even number 
	\item a square number
\end{enumerate}
\solution
%\input{exemplar/10/13/3/32/main.tex}
\item
The king, queen and jack of clubs are removed from a deck of 52 playing cards and then well shuffled. Now one card is drawn at random from the remaining cards.  Determine the probability that the card is
\begin{enumerate}[label=(\roman*)]
\item a club
\item 10 of hearts
\end{enumerate}
\solution
%\input{exemplar/10/13/3/29/main.tex}
\item A team of medical students doing their internship have to assist during surgeries
at a city hospital. The probabilities of surgeries rated as very complex, complex,
routine, simple or very simple are respectively, 0.15, 0.20, 0.31, 0.26, .08. Find
the probabilities that a particular surgery will be rated
\begin{enumerate}
	\item complex or very complex;
	\item neither very complex nor very simple;
	\item routine or complex
	\item routine or simple
\end{enumerate}
\solution
%\input{exemplar/11/16/3/8(1)/main.tex}
\item A card is selected from a pack of 52 cards.
\begin{enumerate}[label=(\alph*)]
    \item How many points are there in the sample space?
    \item Calculate the probability that the card is an ace of spades.
    \item Calculate the probability that the card is (i) an ace and (ii) black card.
\end{enumerate}
\solution
%\input{exemplar/11/16/3/4/main2.tex}
\item The probability that a non leap year selected at random will contain 53 sundays.
\\
\solution
%\input{exemplar/10/13/1/19/main.tex}
\item One of the four persons John, Rita, Aslam or Gurpreet will be promoted next
month. Consequently the sample space consists of four elementary outcomes
S = {John promoted, Rita promoted, Aslam promoted, Gurpreet promoted}
You are told that the chances of John’s promotion is same as that of Gurpreet,
Rita’s chances of promotion are twice as likely as Johns. Aslam’s chances are
four times that of John.
\begin{enumerate}
	\item Determine
	\begin{enumerate}
		\item P (John promoted)
		\item P (Rita promoted)
		\item P (Aslam promoted)
		\item P (Gurpreet promoted)
	\end{enumerate}
	\item If A = {John promoted or Gurpreet promoted}, find P (A).
\end{enumerate}
\solution
%\input{exemplar/11/16/3/10/main.tex}
\item A card is drawn from a deck of 52 cards. Find the probability of getting a king or a heart or a red card.\\
\solution
%\input{exemplar/11/16/3/15/main.tex}
\item The probability that a student will pass his examination is 0.73, the probability of
the student getting a compartment is 0.13, and the probability that the student will
either pass or get compartment is 0.96. State True or False.\\
\solution
%\input{exemplar/11/16/3/31/main.tex}
\item A card is selected from a pack of 52 cards\\
\begin{enumerate}[label=(\alph*)]
\item How many points are there in the sample space?
\item Calculate the probability that the cards is an ace of spades.
\item Calculate the probability that the card is (i) an ace (ii)black card.\\
\end{enumerate}
%\input{ncert/11/16/3/4_1/Prob_4.tex}
\item In a non-leap year, the probability of having 53 tuesdays or 53 wednesdays is\\
\solution
%\input{exemplar/11/16/3/18/main.tex}
\item There are 1000 sealed envelopes in a box, 10 of them contain a cash prize of
Rs 100 each, 100 of them contain a cash prize of Rs 50 each and 200 of them
contain a cash prize of Rs 10 each and rest do not contain any cash prize. If they
are well shuffled and an envelope is picked up out, what is the probability that it
contains no cash prize?\\
\solution
%\input{exemplar/10/13/3/34/main.tex}
\item 
A die is thrown and a card is selected at random from a deck of 52 playing cards. The probability of getting an even number on the die and a spade card.\\
\solution
%\input{exemplar/12/13/3/78/main.tex}
\item
If 4-digit numbers greater than 5,000 are randomly formed from the digits 0, 1, 3, 5, and 7, what is the probability of forming a number divisible by 5 when:
\begin{enumerate}
    \item The digits are repeated?
    \item The repetition of digits is not allowed?
\end{enumerate}
\solution
%\input{ncert/11/16/4/9/main.tex}
\item Consider the probability space $\brak{\Omega, \mathcal{G}, P}$ where $\Omega = [0,2]$ and $\mathcal{G} = \cbrak{\phi, \Omega, [0,1], (1,2]}$. Let $X$ and $Y$ be two functions on $\Omega$ defined as
\begin{align*}
    X(\omega) = 
    \begin{cases}
        1 & \text{if }\omega \in [0, 1]\\
        2 & \text{if }\omega \in (1, 2]
    \end{cases}
\end{align*}
and
\begin{align*}
    Y(\omega) = 
    \begin{cases}
        2 & \text{if }\omega \in [0, 1.5]\\
        3 & \text{if }\omega \in (1.5, 2].
    \end{cases}
\end{align*}
Then which one of the following statements is true?
\begin{enumerate}
    \item [(A)] $X$ is a random variable with respect to $\mathcal{G}$, but $Y$ is not a random variable with respect to $\mathcal{G}$.
    \item [(B)] $Y$ is a random variable with respect to $\mathcal{G}$, but $X$ is not a random variable with respect to $\mathcal{G}$.
    \item [(C)] Neither $X$ nor $Y$ is a random variable with respect to $\mathcal{G}$.
    \item [(D)] Both $X$ and $Y$ are random variables with respect to $\mathcal{G}$.
\end{enumerate} \hfill (GATE ST 2023)\\
\solution
%\input{gate/ST/2023/14/main.tex}
	\item  A die is loaded in such a way that each odd number is twice as likely to occur as
each even number. Find $P(G)$, where $G$ is the event that a number greater than
3 occurs on a single roll of the die.
\\
\solution
		%\input{exemplar/11/16/3/5/main.tex}
	\item All the jacks, queens and kings are removed from a deck of 52 playing cards. The remaining cards are well shuffled and then one card is drawn at random. Giving ace a value 1 similar value for other cards, find the probability that the card has a value 
		\begin{enumerate}
			\item 7
			\item greater than 7
			\item less than 7
		\end{enumerate}
		%\input{exemplar/10/13/3/30/main.tex}
  \item A Lot consists of 48 mobile phones of which 42 are good, 3 have only minor defects and 3 have major defects.Varnika will buy a phone if it is good but the trader will only buy a mobile if it has no major defects. One phone is selected at random from the lot. What is the probability that it is
\begin{enumerate}
	\item acceptable to Varnika?
            \item acceptable to the trader?
\end{enumerate}
\solution
	%\input{exemplar/10/13/3/40/main.tex}
 \item A student says that if you throw a die, it will show up 1 or not 1. Therefore, the probability of getting 1 and the probability of getting 'not 1' each is equal to $\frac{1}{2}$. Is this correct? Give reasons.\\
 \solution
        %\input{exemplar/10/13/2/9/main.tex}
   \item Four candidates A, B, C, D have ap-
plied for the assignment to coach a school cricket
team. If A is twice as likely to be selected as B, and
B and C are given about the same chance of being
selected, while C is twice as likely to be selected
as D, what are the probabilities that
\begin{enumerate}
\item C will be selected?
\item A will not be selected?
\end{enumerate}
	%\input{exemplar/11/16/3/9/main.tex}
 \item A bag contain 24 balls of which $x$ balls are red, $2x$ are white and $3x$ are blue. A ball is selected at random, What is the probability that it is
\begin{enumerate}[label=\alph*)]
\item not red ?
\item white ?
\end{enumerate}
%\input{exemplar/10/13/3/41/main.tex}
If the letters of the word ASSASSINATION are arranged at random. Find the Probability that
\begin{enumerate}[label=(\alph*)]
\item Four $S's$ come consecutively in the word
\item Two  $I's$ and two $N's$ come together
\item All $A's$ are not coming together
\item No two $A's$ are coming together
\end{enumerate}
%\input{exemplar/11/16/3/14/main.tex}
	\item One urn contains two black balls (labelled B1 and B2) and one white ball. A
	second urn contains one black ball and two white balls (labelled W1 and W2).
	Suppose the following experiment is performed. One of the two urns is chosen
	at random. Next a ball is randomly chosen from the urn. Then a second ball is
	chosen at random from the same urn without replacing the first ball.
	
	\begin{enumerate}
	\item What is the probability that two black balls are chosen?
	
	\item What is the probability that two balls of opposite colour are chosen?
	\end{enumerate}
	\solution
	%\input{exemplar/11/16/3/12/main1.tex}
\end{enumerate}

	\item A bag contains 4 red and 4 black balls, another bag contains 2 red and 6 black balls. One of the two bags is selected at random and a ball is drawn from the bag which is found to be red. Find the probability that the ball is drawn from the first bag.
\\
\solution
		%\begin{table}[H]
	\centering
\begin{tabular}{|c|c|c|}
\hline
Random variable &Value &Definition\\ \hline
\multirow{3}{*}{X} &0 &Slips of Rs 1\\
&1 &Slips of Rs 5\\
&2 &Slips of Rs 13\\ \hline
\multirow{2}{*}{Y} &0 &Box A\\
&1 &Box B\\\hline
\end{tabular}
\caption{}
\label{tab:Distribution}
\end{table}
See \tabref{tab:Distribution}.
\begin{align}
p_{Y}\brak{k}= \begin{cases} 
      \frac{1}{3} & {k=0} \\
      \frac{2}{3 }& {k=1} 
   \end{cases}
   \\
p_{Y|X}\brak{0|0} = \frac{19}{25}\, 
p_{Y|X}\brak{0|1} = \frac{6}{25}\,
p_{Y|X}\brak{1|0} = \frac{45}{50}\,
p_{Y|X}\brak{1|2} = \frac{5}{50}
\end{align}
The desired probability is the probability that a slip drawn at random is marked other than Rs 1,
\begin{align}
&=1-p_X\brak{0}\\
&= p_X(1) + p_X(2)
\end{align}
Using Bayes theorem,
\begin{align}
&= p_Y\brak{0} \times \pr{Y=0 | X=1} + p_Y\brak{1} \times \pr{Y=1|X=2}\\
&=\frac{1}{3} \times \frac{6}{25} + \frac{2}{3} \times \frac{5}{50}\\
&=\frac{11}{75}
\end{align}

\newpage

%\tableofcontents

\bigskip

\renewcommand{\thefigure}{\theenumi}
\renewcommand{\thetable}{\theenumi}
%\renewcommand{\theequation}{\theenumi}

%\begin{abstract}
%%\boldmath
%In this letter, an algorithm for evaluating the exact analytical bit error rate  (BER)  for the piecewise linear (PL) combiner for  multiple relays is presented. Previous results were available only for upto three relays. The algorithm is unique in the sense that  the actual mathematical expressions, that are prohibitively large, need not be explicitly obtained. The diversity gain due to multiple relays is shown through plots of the analytical BER, well supported by simulations. 
%
%\end{abstract}
% IEEEtran.cls defaults to using nonbold math in the Abstract.
% This preserves the distinction between vectors and scalars. However,
% if the journal you are submitting to favors bold math in the abstract,
% then you can use LaTeX's standard command \boldmath at the very start
% of the abstract to achieve this. Many IEEE journals frown on math
% in the abstract anyway.

% Note that keywords are not normally used for peerreview papers.
%\begin{IEEEkeywords}
%Cooperative diversity, decode and forward, piecewise linear
%\end{IEEEkeywords}



% For peer review papers, you can put extra information on the cover
% page as needed:
% \ifCLASSOPTIONpeerreview
% \begin{center} \bfseries EDICS Category: 3-BBND \end{center}
% \fi
%
% For peerreview papers, this IEEEtran command inserts a page break and
% creates the second title. It will be ignored for other modes.
%\IEEEpeerreviewmaketitle




  \item
  Cards with numbers 2 to 101 are placed in a box. A card is selected at random.Find the probability that the card has
\begin{enumerate}[label=(\roman*)]
	\item an even number 
	\item a square number
\end{enumerate}
\solution
%\begin{table}[H]
	\centering
\begin{tabular}{|c|c|c|}
\hline
Random variable &Value &Definition\\ \hline
\multirow{3}{*}{X} &0 &Slips of Rs 1\\
&1 &Slips of Rs 5\\
&2 &Slips of Rs 13\\ \hline
\multirow{2}{*}{Y} &0 &Box A\\
&1 &Box B\\\hline
\end{tabular}
\caption{}
\label{tab:Distribution}
\end{table}
See \tabref{tab:Distribution}.
\begin{align}
p_{Y}\brak{k}= \begin{cases} 
      \frac{1}{3} & {k=0} \\
      \frac{2}{3 }& {k=1} 
   \end{cases}
   \\
p_{Y|X}\brak{0|0} = \frac{19}{25}\, 
p_{Y|X}\brak{0|1} = \frac{6}{25}\,
p_{Y|X}\brak{1|0} = \frac{45}{50}\,
p_{Y|X}\brak{1|2} = \frac{5}{50}
\end{align}
The desired probability is the probability that a slip drawn at random is marked other than Rs 1,
\begin{align}
&=1-p_X\brak{0}\\
&= p_X(1) + p_X(2)
\end{align}
Using Bayes theorem,
\begin{align}
&= p_Y\brak{0} \times \pr{Y=0 | X=1} + p_Y\brak{1} \times \pr{Y=1|X=2}\\
&=\frac{1}{3} \times \frac{6}{25} + \frac{2}{3} \times \frac{5}{50}\\
&=\frac{11}{75}
\end{align}

\newpage

%\tableofcontents

\bigskip

\renewcommand{\thefigure}{\theenumi}
\renewcommand{\thetable}{\theenumi}
%\renewcommand{\theequation}{\theenumi}

%\begin{abstract}
%%\boldmath
%In this letter, an algorithm for evaluating the exact analytical bit error rate  (BER)  for the piecewise linear (PL) combiner for  multiple relays is presented. Previous results were available only for upto three relays. The algorithm is unique in the sense that  the actual mathematical expressions, that are prohibitively large, need not be explicitly obtained. The diversity gain due to multiple relays is shown through plots of the analytical BER, well supported by simulations. 
%
%\end{abstract}
% IEEEtran.cls defaults to using nonbold math in the Abstract.
% This preserves the distinction between vectors and scalars. However,
% if the journal you are submitting to favors bold math in the abstract,
% then you can use LaTeX's standard command \boldmath at the very start
% of the abstract to achieve this. Many IEEE journals frown on math
% in the abstract anyway.

% Note that keywords are not normally used for peerreview papers.
%\begin{IEEEkeywords}
%Cooperative diversity, decode and forward, piecewise linear
%\end{IEEEkeywords}



% For peer review papers, you can put extra information on the cover
% page as needed:
% \ifCLASSOPTIONpeerreview
% \begin{center} \bfseries EDICS Category: 3-BBND \end{center}
% \fi
%
% For peerreview papers, this IEEEtran command inserts a page break and
% creates the second title. It will be ignored for other modes.
%\IEEEpeerreviewmaketitle




\item
The king, queen and jack of clubs are removed from a deck of 52 playing cards and then well shuffled. Now one card is drawn at random from the remaining cards.  Determine the probability that the card is
\begin{enumerate}[label=(\roman*)]
\item a club
\item 10 of hearts
\end{enumerate}
\solution
%\begin{table}[H]
	\centering
\begin{tabular}{|c|c|c|}
\hline
Random variable &Value &Definition\\ \hline
\multirow{3}{*}{X} &0 &Slips of Rs 1\\
&1 &Slips of Rs 5\\
&2 &Slips of Rs 13\\ \hline
\multirow{2}{*}{Y} &0 &Box A\\
&1 &Box B\\\hline
\end{tabular}
\caption{}
\label{tab:Distribution}
\end{table}
See \tabref{tab:Distribution}.
\begin{align}
p_{Y}\brak{k}= \begin{cases} 
      \frac{1}{3} & {k=0} \\
      \frac{2}{3 }& {k=1} 
   \end{cases}
   \\
p_{Y|X}\brak{0|0} = \frac{19}{25}\, 
p_{Y|X}\brak{0|1} = \frac{6}{25}\,
p_{Y|X}\brak{1|0} = \frac{45}{50}\,
p_{Y|X}\brak{1|2} = \frac{5}{50}
\end{align}
The desired probability is the probability that a slip drawn at random is marked other than Rs 1,
\begin{align}
&=1-p_X\brak{0}\\
&= p_X(1) + p_X(2)
\end{align}
Using Bayes theorem,
\begin{align}
&= p_Y\brak{0} \times \pr{Y=0 | X=1} + p_Y\brak{1} \times \pr{Y=1|X=2}\\
&=\frac{1}{3} \times \frac{6}{25} + \frac{2}{3} \times \frac{5}{50}\\
&=\frac{11}{75}
\end{align}

\newpage

%\tableofcontents

\bigskip

\renewcommand{\thefigure}{\theenumi}
\renewcommand{\thetable}{\theenumi}
%\renewcommand{\theequation}{\theenumi}

%\begin{abstract}
%%\boldmath
%In this letter, an algorithm for evaluating the exact analytical bit error rate  (BER)  for the piecewise linear (PL) combiner for  multiple relays is presented. Previous results were available only for upto three relays. The algorithm is unique in the sense that  the actual mathematical expressions, that are prohibitively large, need not be explicitly obtained. The diversity gain due to multiple relays is shown through plots of the analytical BER, well supported by simulations. 
%
%\end{abstract}
% IEEEtran.cls defaults to using nonbold math in the Abstract.
% This preserves the distinction between vectors and scalars. However,
% if the journal you are submitting to favors bold math in the abstract,
% then you can use LaTeX's standard command \boldmath at the very start
% of the abstract to achieve this. Many IEEE journals frown on math
% in the abstract anyway.

% Note that keywords are not normally used for peerreview papers.
%\begin{IEEEkeywords}
%Cooperative diversity, decode and forward, piecewise linear
%\end{IEEEkeywords}



% For peer review papers, you can put extra information on the cover
% page as needed:
% \ifCLASSOPTIONpeerreview
% \begin{center} \bfseries EDICS Category: 3-BBND \end{center}
% \fi
%
% For peerreview papers, this IEEEtran command inserts a page break and
% creates the second title. It will be ignored for other modes.
%\IEEEpeerreviewmaketitle




\item A team of medical students doing their internship have to assist during surgeries
at a city hospital. The probabilities of surgeries rated as very complex, complex,
routine, simple or very simple are respectively, 0.15, 0.20, 0.31, 0.26, .08. Find
the probabilities that a particular surgery will be rated
\begin{enumerate}
	\item complex or very complex;
	\item neither very complex nor very simple;
	\item routine or complex
	\item routine or simple
\end{enumerate}
\solution
%\begin{table}[H]
	\centering
\begin{tabular}{|c|c|c|}
\hline
Random variable &Value &Definition\\ \hline
\multirow{3}{*}{X} &0 &Slips of Rs 1\\
&1 &Slips of Rs 5\\
&2 &Slips of Rs 13\\ \hline
\multirow{2}{*}{Y} &0 &Box A\\
&1 &Box B\\\hline
\end{tabular}
\caption{}
\label{tab:Distribution}
\end{table}
See \tabref{tab:Distribution}.
\begin{align}
p_{Y}\brak{k}= \begin{cases} 
      \frac{1}{3} & {k=0} \\
      \frac{2}{3 }& {k=1} 
   \end{cases}
   \\
p_{Y|X}\brak{0|0} = \frac{19}{25}\, 
p_{Y|X}\brak{0|1} = \frac{6}{25}\,
p_{Y|X}\brak{1|0} = \frac{45}{50}\,
p_{Y|X}\brak{1|2} = \frac{5}{50}
\end{align}
The desired probability is the probability that a slip drawn at random is marked other than Rs 1,
\begin{align}
&=1-p_X\brak{0}\\
&= p_X(1) + p_X(2)
\end{align}
Using Bayes theorem,
\begin{align}
&= p_Y\brak{0} \times \pr{Y=0 | X=1} + p_Y\brak{1} \times \pr{Y=1|X=2}\\
&=\frac{1}{3} \times \frac{6}{25} + \frac{2}{3} \times \frac{5}{50}\\
&=\frac{11}{75}
\end{align}

\newpage

%\tableofcontents

\bigskip

\renewcommand{\thefigure}{\theenumi}
\renewcommand{\thetable}{\theenumi}
%\renewcommand{\theequation}{\theenumi}

%\begin{abstract}
%%\boldmath
%In this letter, an algorithm for evaluating the exact analytical bit error rate  (BER)  for the piecewise linear (PL) combiner for  multiple relays is presented. Previous results were available only for upto three relays. The algorithm is unique in the sense that  the actual mathematical expressions, that are prohibitively large, need not be explicitly obtained. The diversity gain due to multiple relays is shown through plots of the analytical BER, well supported by simulations. 
%
%\end{abstract}
% IEEEtran.cls defaults to using nonbold math in the Abstract.
% This preserves the distinction between vectors and scalars. However,
% if the journal you are submitting to favors bold math in the abstract,
% then you can use LaTeX's standard command \boldmath at the very start
% of the abstract to achieve this. Many IEEE journals frown on math
% in the abstract anyway.

% Note that keywords are not normally used for peerreview papers.
%\begin{IEEEkeywords}
%Cooperative diversity, decode and forward, piecewise linear
%\end{IEEEkeywords}



% For peer review papers, you can put extra information on the cover
% page as needed:
% \ifCLASSOPTIONpeerreview
% \begin{center} \bfseries EDICS Category: 3-BBND \end{center}
% \fi
%
% For peerreview papers, this IEEEtran command inserts a page break and
% creates the second title. It will be ignored for other modes.
%\IEEEpeerreviewmaketitle




\item A card is selected from a pack of 52 cards.
\begin{enumerate}[label=(\alph*)]
    \item How many points are there in the sample space?
    \item Calculate the probability that the card is an ace of spades.
    \item Calculate the probability that the card is (i) an ace and (ii) black card.
\end{enumerate}
\solution
%Let $X$ be an bernoulli rv defined as in \tabref{tab:exemplar/11/16/3/26}.  Then, 
\begin{equation}
    p =
        \frac{4}{11} 
\end{equation}
\begin{table}[H]
	\centering
	\input{exemplar/11/16/3/26/tables/Table2.tex}
	\caption{}
        \label{tab:exemplar/11/16/3/26}
\end{table}

\item The probability that a non leap year selected at random will contain 53 sundays.
\\
\solution
%\begin{table}[H]
	\centering
\begin{tabular}{|c|c|c|}
\hline
Random variable &Value &Definition\\ \hline
\multirow{3}{*}{X} &0 &Slips of Rs 1\\
&1 &Slips of Rs 5\\
&2 &Slips of Rs 13\\ \hline
\multirow{2}{*}{Y} &0 &Box A\\
&1 &Box B\\\hline
\end{tabular}
\caption{}
\label{tab:Distribution}
\end{table}
See \tabref{tab:Distribution}.
\begin{align}
p_{Y}\brak{k}= \begin{cases} 
      \frac{1}{3} & {k=0} \\
      \frac{2}{3 }& {k=1} 
   \end{cases}
   \\
p_{Y|X}\brak{0|0} = \frac{19}{25}\, 
p_{Y|X}\brak{0|1} = \frac{6}{25}\,
p_{Y|X}\brak{1|0} = \frac{45}{50}\,
p_{Y|X}\brak{1|2} = \frac{5}{50}
\end{align}
The desired probability is the probability that a slip drawn at random is marked other than Rs 1,
\begin{align}
&=1-p_X\brak{0}\\
&= p_X(1) + p_X(2)
\end{align}
Using Bayes theorem,
\begin{align}
&= p_Y\brak{0} \times \pr{Y=0 | X=1} + p_Y\brak{1} \times \pr{Y=1|X=2}\\
&=\frac{1}{3} \times \frac{6}{25} + \frac{2}{3} \times \frac{5}{50}\\
&=\frac{11}{75}
\end{align}

\newpage

%\tableofcontents

\bigskip

\renewcommand{\thefigure}{\theenumi}
\renewcommand{\thetable}{\theenumi}
%\renewcommand{\theequation}{\theenumi}

%\begin{abstract}
%%\boldmath
%In this letter, an algorithm for evaluating the exact analytical bit error rate  (BER)  for the piecewise linear (PL) combiner for  multiple relays is presented. Previous results were available only for upto three relays. The algorithm is unique in the sense that  the actual mathematical expressions, that are prohibitively large, need not be explicitly obtained. The diversity gain due to multiple relays is shown through plots of the analytical BER, well supported by simulations. 
%
%\end{abstract}
% IEEEtran.cls defaults to using nonbold math in the Abstract.
% This preserves the distinction between vectors and scalars. However,
% if the journal you are submitting to favors bold math in the abstract,
% then you can use LaTeX's standard command \boldmath at the very start
% of the abstract to achieve this. Many IEEE journals frown on math
% in the abstract anyway.

% Note that keywords are not normally used for peerreview papers.
%\begin{IEEEkeywords}
%Cooperative diversity, decode and forward, piecewise linear
%\end{IEEEkeywords}



% For peer review papers, you can put extra information on the cover
% page as needed:
% \ifCLASSOPTIONpeerreview
% \begin{center} \bfseries EDICS Category: 3-BBND \end{center}
% \fi
%
% For peerreview papers, this IEEEtran command inserts a page break and
% creates the second title. It will be ignored for other modes.
%\IEEEpeerreviewmaketitle




\item One of the four persons John, Rita, Aslam or Gurpreet will be promoted next
month. Consequently the sample space consists of four elementary outcomes
S = {John promoted, Rita promoted, Aslam promoted, Gurpreet promoted}
You are told that the chances of John’s promotion is same as that of Gurpreet,
Rita’s chances of promotion are twice as likely as Johns. Aslam’s chances are
four times that of John.
\begin{enumerate}
	\item Determine
	\begin{enumerate}
		\item P (John promoted)
		\item P (Rita promoted)
		\item P (Aslam promoted)
		\item P (Gurpreet promoted)
	\end{enumerate}
	\item If A = {John promoted or Gurpreet promoted}, find P (A).
\end{enumerate}
\solution
%\begin{table}[H]
	\centering
\begin{tabular}{|c|c|c|}
\hline
Random variable &Value &Definition\\ \hline
\multirow{3}{*}{X} &0 &Slips of Rs 1\\
&1 &Slips of Rs 5\\
&2 &Slips of Rs 13\\ \hline
\multirow{2}{*}{Y} &0 &Box A\\
&1 &Box B\\\hline
\end{tabular}
\caption{}
\label{tab:Distribution}
\end{table}
See \tabref{tab:Distribution}.
\begin{align}
p_{Y}\brak{k}= \begin{cases} 
      \frac{1}{3} & {k=0} \\
      \frac{2}{3 }& {k=1} 
   \end{cases}
   \\
p_{Y|X}\brak{0|0} = \frac{19}{25}\, 
p_{Y|X}\brak{0|1} = \frac{6}{25}\,
p_{Y|X}\brak{1|0} = \frac{45}{50}\,
p_{Y|X}\brak{1|2} = \frac{5}{50}
\end{align}
The desired probability is the probability that a slip drawn at random is marked other than Rs 1,
\begin{align}
&=1-p_X\brak{0}\\
&= p_X(1) + p_X(2)
\end{align}
Using Bayes theorem,
\begin{align}
&= p_Y\brak{0} \times \pr{Y=0 | X=1} + p_Y\brak{1} \times \pr{Y=1|X=2}\\
&=\frac{1}{3} \times \frac{6}{25} + \frac{2}{3} \times \frac{5}{50}\\
&=\frac{11}{75}
\end{align}

\newpage

%\tableofcontents

\bigskip

\renewcommand{\thefigure}{\theenumi}
\renewcommand{\thetable}{\theenumi}
%\renewcommand{\theequation}{\theenumi}

%\begin{abstract}
%%\boldmath
%In this letter, an algorithm for evaluating the exact analytical bit error rate  (BER)  for the piecewise linear (PL) combiner for  multiple relays is presented. Previous results were available only for upto three relays. The algorithm is unique in the sense that  the actual mathematical expressions, that are prohibitively large, need not be explicitly obtained. The diversity gain due to multiple relays is shown through plots of the analytical BER, well supported by simulations. 
%
%\end{abstract}
% IEEEtran.cls defaults to using nonbold math in the Abstract.
% This preserves the distinction between vectors and scalars. However,
% if the journal you are submitting to favors bold math in the abstract,
% then you can use LaTeX's standard command \boldmath at the very start
% of the abstract to achieve this. Many IEEE journals frown on math
% in the abstract anyway.

% Note that keywords are not normally used for peerreview papers.
%\begin{IEEEkeywords}
%Cooperative diversity, decode and forward, piecewise linear
%\end{IEEEkeywords}



% For peer review papers, you can put extra information on the cover
% page as needed:
% \ifCLASSOPTIONpeerreview
% \begin{center} \bfseries EDICS Category: 3-BBND \end{center}
% \fi
%
% For peerreview papers, this IEEEtran command inserts a page break and
% creates the second title. It will be ignored for other modes.
%\IEEEpeerreviewmaketitle




\item A card is drawn from a deck of 52 cards. Find the probability of getting a king or a heart or a red card.\\
\solution
%\begin{table}[H]
	\centering
\begin{tabular}{|c|c|c|}
\hline
Random variable &Value &Definition\\ \hline
\multirow{3}{*}{X} &0 &Slips of Rs 1\\
&1 &Slips of Rs 5\\
&2 &Slips of Rs 13\\ \hline
\multirow{2}{*}{Y} &0 &Box A\\
&1 &Box B\\\hline
\end{tabular}
\caption{}
\label{tab:Distribution}
\end{table}
See \tabref{tab:Distribution}.
\begin{align}
p_{Y}\brak{k}= \begin{cases} 
      \frac{1}{3} & {k=0} \\
      \frac{2}{3 }& {k=1} 
   \end{cases}
   \\
p_{Y|X}\brak{0|0} = \frac{19}{25}\, 
p_{Y|X}\brak{0|1} = \frac{6}{25}\,
p_{Y|X}\brak{1|0} = \frac{45}{50}\,
p_{Y|X}\brak{1|2} = \frac{5}{50}
\end{align}
The desired probability is the probability that a slip drawn at random is marked other than Rs 1,
\begin{align}
&=1-p_X\brak{0}\\
&= p_X(1) + p_X(2)
\end{align}
Using Bayes theorem,
\begin{align}
&= p_Y\brak{0} \times \pr{Y=0 | X=1} + p_Y\brak{1} \times \pr{Y=1|X=2}\\
&=\frac{1}{3} \times \frac{6}{25} + \frac{2}{3} \times \frac{5}{50}\\
&=\frac{11}{75}
\end{align}

\newpage

%\tableofcontents

\bigskip

\renewcommand{\thefigure}{\theenumi}
\renewcommand{\thetable}{\theenumi}
%\renewcommand{\theequation}{\theenumi}

%\begin{abstract}
%%\boldmath
%In this letter, an algorithm for evaluating the exact analytical bit error rate  (BER)  for the piecewise linear (PL) combiner for  multiple relays is presented. Previous results were available only for upto three relays. The algorithm is unique in the sense that  the actual mathematical expressions, that are prohibitively large, need not be explicitly obtained. The diversity gain due to multiple relays is shown through plots of the analytical BER, well supported by simulations. 
%
%\end{abstract}
% IEEEtran.cls defaults to using nonbold math in the Abstract.
% This preserves the distinction between vectors and scalars. However,
% if the journal you are submitting to favors bold math in the abstract,
% then you can use LaTeX's standard command \boldmath at the very start
% of the abstract to achieve this. Many IEEE journals frown on math
% in the abstract anyway.

% Note that keywords are not normally used for peerreview papers.
%\begin{IEEEkeywords}
%Cooperative diversity, decode and forward, piecewise linear
%\end{IEEEkeywords}



% For peer review papers, you can put extra information on the cover
% page as needed:
% \ifCLASSOPTIONpeerreview
% \begin{center} \bfseries EDICS Category: 3-BBND \end{center}
% \fi
%
% For peerreview papers, this IEEEtran command inserts a page break and
% creates the second title. It will be ignored for other modes.
%\IEEEpeerreviewmaketitle




\item The probability that a student will pass his examination is 0.73, the probability of
the student getting a compartment is 0.13, and the probability that the student will
either pass or get compartment is 0.96. State True or False.\\
\solution
%\begin{table}[H]
	\centering
\begin{tabular}{|c|c|c|}
\hline
Random variable &Value &Definition\\ \hline
\multirow{3}{*}{X} &0 &Slips of Rs 1\\
&1 &Slips of Rs 5\\
&2 &Slips of Rs 13\\ \hline
\multirow{2}{*}{Y} &0 &Box A\\
&1 &Box B\\\hline
\end{tabular}
\caption{}
\label{tab:Distribution}
\end{table}
See \tabref{tab:Distribution}.
\begin{align}
p_{Y}\brak{k}= \begin{cases} 
      \frac{1}{3} & {k=0} \\
      \frac{2}{3 }& {k=1} 
   \end{cases}
   \\
p_{Y|X}\brak{0|0} = \frac{19}{25}\, 
p_{Y|X}\brak{0|1} = \frac{6}{25}\,
p_{Y|X}\brak{1|0} = \frac{45}{50}\,
p_{Y|X}\brak{1|2} = \frac{5}{50}
\end{align}
The desired probability is the probability that a slip drawn at random is marked other than Rs 1,
\begin{align}
&=1-p_X\brak{0}\\
&= p_X(1) + p_X(2)
\end{align}
Using Bayes theorem,
\begin{align}
&= p_Y\brak{0} \times \pr{Y=0 | X=1} + p_Y\brak{1} \times \pr{Y=1|X=2}\\
&=\frac{1}{3} \times \frac{6}{25} + \frac{2}{3} \times \frac{5}{50}\\
&=\frac{11}{75}
\end{align}

\newpage

%\tableofcontents

\bigskip

\renewcommand{\thefigure}{\theenumi}
\renewcommand{\thetable}{\theenumi}
%\renewcommand{\theequation}{\theenumi}

%\begin{abstract}
%%\boldmath
%In this letter, an algorithm for evaluating the exact analytical bit error rate  (BER)  for the piecewise linear (PL) combiner for  multiple relays is presented. Previous results were available only for upto three relays. The algorithm is unique in the sense that  the actual mathematical expressions, that are prohibitively large, need not be explicitly obtained. The diversity gain due to multiple relays is shown through plots of the analytical BER, well supported by simulations. 
%
%\end{abstract}
% IEEEtran.cls defaults to using nonbold math in the Abstract.
% This preserves the distinction between vectors and scalars. However,
% if the journal you are submitting to favors bold math in the abstract,
% then you can use LaTeX's standard command \boldmath at the very start
% of the abstract to achieve this. Many IEEE journals frown on math
% in the abstract anyway.

% Note that keywords are not normally used for peerreview papers.
%\begin{IEEEkeywords}
%Cooperative diversity, decode and forward, piecewise linear
%\end{IEEEkeywords}



% For peer review papers, you can put extra information on the cover
% page as needed:
% \ifCLASSOPTIONpeerreview
% \begin{center} \bfseries EDICS Category: 3-BBND \end{center}
% \fi
%
% For peerreview papers, this IEEEtran command inserts a page break and
% creates the second title. It will be ignored for other modes.
%\IEEEpeerreviewmaketitle




\item A card is selected from a pack of 52 cards\\
\begin{enumerate}[label=(\alph*)]
\item How many points are there in the sample space?
\item Calculate the probability that the cards is an ace of spades.
\item Calculate the probability that the card is (i) an ace (ii)black card.\\
\end{enumerate}
%\input{ncert/11/16/3/4_1/Prob_4.tex}
\item In a non-leap year, the probability of having 53 tuesdays or 53 wednesdays is\\
\solution
%A non-leap year has a total of 365 days, and a week has 7 days.\\
So it can be expressed as 
\begin{align}
365\text{days} &=52\times 7+1 \text{day}
\end{align}
$\implies$ 52 tuesdays or wednesdays\\
Random variable X denotes the days of a week
\begin{align}
p_X\brak{k}&=\frac{1}{7}; \quad \brak{1<k<7}
\end{align}
So the probability of extra day being tuesday or wednesday is
\begin{align}
p_X\brak{3}+p_X\brak{4}&=\frac{1}{7}+\frac{1}{7}=\frac{2}{7}
\end{align}



\item There are 1000 sealed envelopes in a box, 10 of them contain a cash prize of
Rs 100 each, 100 of them contain a cash prize of Rs 50 each and 200 of them
contain a cash prize of Rs 10 each and rest do not contain any cash prize. If they
are well shuffled and an envelope is picked up out, what is the probability that it
contains no cash prize?\\
\solution
%\begin{table}[H]
	\centering
\begin{tabular}{|c|c|c|}
\hline
Random variable &Value &Definition\\ \hline
\multirow{3}{*}{X} &0 &Slips of Rs 1\\
&1 &Slips of Rs 5\\
&2 &Slips of Rs 13\\ \hline
\multirow{2}{*}{Y} &0 &Box A\\
&1 &Box B\\\hline
\end{tabular}
\caption{}
\label{tab:Distribution}
\end{table}
See \tabref{tab:Distribution}.
\begin{align}
p_{Y}\brak{k}= \begin{cases} 
      \frac{1}{3} & {k=0} \\
      \frac{2}{3 }& {k=1} 
   \end{cases}
   \\
p_{Y|X}\brak{0|0} = \frac{19}{25}\, 
p_{Y|X}\brak{0|1} = \frac{6}{25}\,
p_{Y|X}\brak{1|0} = \frac{45}{50}\,
p_{Y|X}\brak{1|2} = \frac{5}{50}
\end{align}
The desired probability is the probability that a slip drawn at random is marked other than Rs 1,
\begin{align}
&=1-p_X\brak{0}\\
&= p_X(1) + p_X(2)
\end{align}
Using Bayes theorem,
\begin{align}
&= p_Y\brak{0} \times \pr{Y=0 | X=1} + p_Y\brak{1} \times \pr{Y=1|X=2}\\
&=\frac{1}{3} \times \frac{6}{25} + \frac{2}{3} \times \frac{5}{50}\\
&=\frac{11}{75}
\end{align}

\newpage

%\tableofcontents

\bigskip

\renewcommand{\thefigure}{\theenumi}
\renewcommand{\thetable}{\theenumi}
%\renewcommand{\theequation}{\theenumi}

%\begin{abstract}
%%\boldmath
%In this letter, an algorithm for evaluating the exact analytical bit error rate  (BER)  for the piecewise linear (PL) combiner for  multiple relays is presented. Previous results were available only for upto three relays. The algorithm is unique in the sense that  the actual mathematical expressions, that are prohibitively large, need not be explicitly obtained. The diversity gain due to multiple relays is shown through plots of the analytical BER, well supported by simulations. 
%
%\end{abstract}
% IEEEtran.cls defaults to using nonbold math in the Abstract.
% This preserves the distinction between vectors and scalars. However,
% if the journal you are submitting to favors bold math in the abstract,
% then you can use LaTeX's standard command \boldmath at the very start
% of the abstract to achieve this. Many IEEE journals frown on math
% in the abstract anyway.

% Note that keywords are not normally used for peerreview papers.
%\begin{IEEEkeywords}
%Cooperative diversity, decode and forward, piecewise linear
%\end{IEEEkeywords}



% For peer review papers, you can put extra information on the cover
% page as needed:
% \ifCLASSOPTIONpeerreview
% \begin{center} \bfseries EDICS Category: 3-BBND \end{center}
% \fi
%
% For peerreview papers, this IEEEtran command inserts a page break and
% creates the second title. It will be ignored for other modes.
%\IEEEpeerreviewmaketitle




\item 
A die is thrown and a card is selected at random from a deck of 52 playing cards. The probability of getting an even number on the die and a spade card.\\
\solution
%\begin{table}[H]
	\centering
\begin{tabular}{|c|c|c|}
\hline
Random variable &Value &Definition\\ \hline
\multirow{3}{*}{X} &0 &Slips of Rs 1\\
&1 &Slips of Rs 5\\
&2 &Slips of Rs 13\\ \hline
\multirow{2}{*}{Y} &0 &Box A\\
&1 &Box B\\\hline
\end{tabular}
\caption{}
\label{tab:Distribution}
\end{table}
See \tabref{tab:Distribution}.
\begin{align}
p_{Y}\brak{k}= \begin{cases} 
      \frac{1}{3} & {k=0} \\
      \frac{2}{3 }& {k=1} 
   \end{cases}
   \\
p_{Y|X}\brak{0|0} = \frac{19}{25}\, 
p_{Y|X}\brak{0|1} = \frac{6}{25}\,
p_{Y|X}\brak{1|0} = \frac{45}{50}\,
p_{Y|X}\brak{1|2} = \frac{5}{50}
\end{align}
The desired probability is the probability that a slip drawn at random is marked other than Rs 1,
\begin{align}
&=1-p_X\brak{0}\\
&= p_X(1) + p_X(2)
\end{align}
Using Bayes theorem,
\begin{align}
&= p_Y\brak{0} \times \pr{Y=0 | X=1} + p_Y\brak{1} \times \pr{Y=1|X=2}\\
&=\frac{1}{3} \times \frac{6}{25} + \frac{2}{3} \times \frac{5}{50}\\
&=\frac{11}{75}
\end{align}

\newpage

%\tableofcontents

\bigskip

\renewcommand{\thefigure}{\theenumi}
\renewcommand{\thetable}{\theenumi}
%\renewcommand{\theequation}{\theenumi}

%\begin{abstract}
%%\boldmath
%In this letter, an algorithm for evaluating the exact analytical bit error rate  (BER)  for the piecewise linear (PL) combiner for  multiple relays is presented. Previous results were available only for upto three relays. The algorithm is unique in the sense that  the actual mathematical expressions, that are prohibitively large, need not be explicitly obtained. The diversity gain due to multiple relays is shown through plots of the analytical BER, well supported by simulations. 
%
%\end{abstract}
% IEEEtran.cls defaults to using nonbold math in the Abstract.
% This preserves the distinction between vectors and scalars. However,
% if the journal you are submitting to favors bold math in the abstract,
% then you can use LaTeX's standard command \boldmath at the very start
% of the abstract to achieve this. Many IEEE journals frown on math
% in the abstract anyway.

% Note that keywords are not normally used for peerreview papers.
%\begin{IEEEkeywords}
%Cooperative diversity, decode and forward, piecewise linear
%\end{IEEEkeywords}



% For peer review papers, you can put extra information on the cover
% page as needed:
% \ifCLASSOPTIONpeerreview
% \begin{center} \bfseries EDICS Category: 3-BBND \end{center}
% \fi
%
% For peerreview papers, this IEEEtran command inserts a page break and
% creates the second title. It will be ignored for other modes.
%\IEEEpeerreviewmaketitle




\item
If 4-digit numbers greater than 5,000 are randomly formed from the digits 0, 1, 3, 5, and 7, what is the probability of forming a number divisible by 5 when:
\begin{enumerate}
    \item The digits are repeated?
    \item The repetition of digits is not allowed?
\end{enumerate}
\solution
%\begin{table}[H]
	\centering
\begin{tabular}{|c|c|c|}
\hline
Random variable &Value &Definition\\ \hline
\multirow{3}{*}{X} &0 &Slips of Rs 1\\
&1 &Slips of Rs 5\\
&2 &Slips of Rs 13\\ \hline
\multirow{2}{*}{Y} &0 &Box A\\
&1 &Box B\\\hline
\end{tabular}
\caption{}
\label{tab:Distribution}
\end{table}
See \tabref{tab:Distribution}.
\begin{align}
p_{Y}\brak{k}= \begin{cases} 
      \frac{1}{3} & {k=0} \\
      \frac{2}{3 }& {k=1} 
   \end{cases}
   \\
p_{Y|X}\brak{0|0} = \frac{19}{25}\, 
p_{Y|X}\brak{0|1} = \frac{6}{25}\,
p_{Y|X}\brak{1|0} = \frac{45}{50}\,
p_{Y|X}\brak{1|2} = \frac{5}{50}
\end{align}
The desired probability is the probability that a slip drawn at random is marked other than Rs 1,
\begin{align}
&=1-p_X\brak{0}\\
&= p_X(1) + p_X(2)
\end{align}
Using Bayes theorem,
\begin{align}
&= p_Y\brak{0} \times \pr{Y=0 | X=1} + p_Y\brak{1} \times \pr{Y=1|X=2}\\
&=\frac{1}{3} \times \frac{6}{25} + \frac{2}{3} \times \frac{5}{50}\\
&=\frac{11}{75}
\end{align}

\newpage

%\tableofcontents

\bigskip

\renewcommand{\thefigure}{\theenumi}
\renewcommand{\thetable}{\theenumi}
%\renewcommand{\theequation}{\theenumi}

%\begin{abstract}
%%\boldmath
%In this letter, an algorithm for evaluating the exact analytical bit error rate  (BER)  for the piecewise linear (PL) combiner for  multiple relays is presented. Previous results were available only for upto three relays. The algorithm is unique in the sense that  the actual mathematical expressions, that are prohibitively large, need not be explicitly obtained. The diversity gain due to multiple relays is shown through plots of the analytical BER, well supported by simulations. 
%
%\end{abstract}
% IEEEtran.cls defaults to using nonbold math in the Abstract.
% This preserves the distinction between vectors and scalars. However,
% if the journal you are submitting to favors bold math in the abstract,
% then you can use LaTeX's standard command \boldmath at the very start
% of the abstract to achieve this. Many IEEE journals frown on math
% in the abstract anyway.

% Note that keywords are not normally used for peerreview papers.
%\begin{IEEEkeywords}
%Cooperative diversity, decode and forward, piecewise linear
%\end{IEEEkeywords}



% For peer review papers, you can put extra information on the cover
% page as needed:
% \ifCLASSOPTIONpeerreview
% \begin{center} \bfseries EDICS Category: 3-BBND \end{center}
% \fi
%
% For peerreview papers, this IEEEtran command inserts a page break and
% creates the second title. It will be ignored for other modes.
%\IEEEpeerreviewmaketitle




\item Consider the probability space $\brak{\Omega, \mathcal{G}, P}$ where $\Omega = [0,2]$ and $\mathcal{G} = \cbrak{\phi, \Omega, [0,1], (1,2]}$. Let $X$ and $Y$ be two functions on $\Omega$ defined as
\begin{align*}
    X(\omega) = 
    \begin{cases}
        1 & \text{if }\omega \in [0, 1]\\
        2 & \text{if }\omega \in (1, 2]
    \end{cases}
\end{align*}
and
\begin{align*}
    Y(\omega) = 
    \begin{cases}
        2 & \text{if }\omega \in [0, 1.5]\\
        3 & \text{if }\omega \in (1.5, 2].
    \end{cases}
\end{align*}
Then which one of the following statements is true?
\begin{enumerate}
    \item [(A)] $X$ is a random variable with respect to $\mathcal{G}$, but $Y$ is not a random variable with respect to $\mathcal{G}$.
    \item [(B)] $Y$ is a random variable with respect to $\mathcal{G}$, but $X$ is not a random variable with respect to $\mathcal{G}$.
    \item [(C)] Neither $X$ nor $Y$ is a random variable with respect to $\mathcal{G}$.
    \item [(D)] Both $X$ and $Y$ are random variables with respect to $\mathcal{G}$.
\end{enumerate} \hfill (GATE ST 2023)\\
\solution
%\begin{table}[H]
	\centering
\begin{tabular}{|c|c|c|}
\hline
Random variable &Value &Definition\\ \hline
\multirow{3}{*}{X} &0 &Slips of Rs 1\\
&1 &Slips of Rs 5\\
&2 &Slips of Rs 13\\ \hline
\multirow{2}{*}{Y} &0 &Box A\\
&1 &Box B\\\hline
\end{tabular}
\caption{}
\label{tab:Distribution}
\end{table}
See \tabref{tab:Distribution}.
\begin{align}
p_{Y}\brak{k}= \begin{cases} 
      \frac{1}{3} & {k=0} \\
      \frac{2}{3 }& {k=1} 
   \end{cases}
   \\
p_{Y|X}\brak{0|0} = \frac{19}{25}\, 
p_{Y|X}\brak{0|1} = \frac{6}{25}\,
p_{Y|X}\brak{1|0} = \frac{45}{50}\,
p_{Y|X}\brak{1|2} = \frac{5}{50}
\end{align}
The desired probability is the probability that a slip drawn at random is marked other than Rs 1,
\begin{align}
&=1-p_X\brak{0}\\
&= p_X(1) + p_X(2)
\end{align}
Using Bayes theorem,
\begin{align}
&= p_Y\brak{0} \times \pr{Y=0 | X=1} + p_Y\brak{1} \times \pr{Y=1|X=2}\\
&=\frac{1}{3} \times \frac{6}{25} + \frac{2}{3} \times \frac{5}{50}\\
&=\frac{11}{75}
\end{align}

\newpage

%\tableofcontents

\bigskip

\renewcommand{\thefigure}{\theenumi}
\renewcommand{\thetable}{\theenumi}
%\renewcommand{\theequation}{\theenumi}

%\begin{abstract}
%%\boldmath
%In this letter, an algorithm for evaluating the exact analytical bit error rate  (BER)  for the piecewise linear (PL) combiner for  multiple relays is presented. Previous results were available only for upto three relays. The algorithm is unique in the sense that  the actual mathematical expressions, that are prohibitively large, need not be explicitly obtained. The diversity gain due to multiple relays is shown through plots of the analytical BER, well supported by simulations. 
%
%\end{abstract}
% IEEEtran.cls defaults to using nonbold math in the Abstract.
% This preserves the distinction between vectors and scalars. However,
% if the journal you are submitting to favors bold math in the abstract,
% then you can use LaTeX's standard command \boldmath at the very start
% of the abstract to achieve this. Many IEEE journals frown on math
% in the abstract anyway.

% Note that keywords are not normally used for peerreview papers.
%\begin{IEEEkeywords}
%Cooperative diversity, decode and forward, piecewise linear
%\end{IEEEkeywords}



% For peer review papers, you can put extra information on the cover
% page as needed:
% \ifCLASSOPTIONpeerreview
% \begin{center} \bfseries EDICS Category: 3-BBND \end{center}
% \fi
%
% For peerreview papers, this IEEEtran command inserts a page break and
% creates the second title. It will be ignored for other modes.
%\IEEEpeerreviewmaketitle




	\item  A die is loaded in such a way that each odd number is twice as likely to occur as
each even number. Find $P(G)$, where $G$ is the event that a number greater than
3 occurs on a single roll of the die.
\\
\solution
		%\begin{table}[H]
	\centering
\begin{tabular}{|c|c|c|}
\hline
Random variable &Value &Definition\\ \hline
\multirow{3}{*}{X} &0 &Slips of Rs 1\\
&1 &Slips of Rs 5\\
&2 &Slips of Rs 13\\ \hline
\multirow{2}{*}{Y} &0 &Box A\\
&1 &Box B\\\hline
\end{tabular}
\caption{}
\label{tab:Distribution}
\end{table}
See \tabref{tab:Distribution}.
\begin{align}
p_{Y}\brak{k}= \begin{cases} 
      \frac{1}{3} & {k=0} \\
      \frac{2}{3 }& {k=1} 
   \end{cases}
   \\
p_{Y|X}\brak{0|0} = \frac{19}{25}\, 
p_{Y|X}\brak{0|1} = \frac{6}{25}\,
p_{Y|X}\brak{1|0} = \frac{45}{50}\,
p_{Y|X}\brak{1|2} = \frac{5}{50}
\end{align}
The desired probability is the probability that a slip drawn at random is marked other than Rs 1,
\begin{align}
&=1-p_X\brak{0}\\
&= p_X(1) + p_X(2)
\end{align}
Using Bayes theorem,
\begin{align}
&= p_Y\brak{0} \times \pr{Y=0 | X=1} + p_Y\brak{1} \times \pr{Y=1|X=2}\\
&=\frac{1}{3} \times \frac{6}{25} + \frac{2}{3} \times \frac{5}{50}\\
&=\frac{11}{75}
\end{align}

\newpage

%\tableofcontents

\bigskip

\renewcommand{\thefigure}{\theenumi}
\renewcommand{\thetable}{\theenumi}
%\renewcommand{\theequation}{\theenumi}

%\begin{abstract}
%%\boldmath
%In this letter, an algorithm for evaluating the exact analytical bit error rate  (BER)  for the piecewise linear (PL) combiner for  multiple relays is presented. Previous results were available only for upto three relays. The algorithm is unique in the sense that  the actual mathematical expressions, that are prohibitively large, need not be explicitly obtained. The diversity gain due to multiple relays is shown through plots of the analytical BER, well supported by simulations. 
%
%\end{abstract}
% IEEEtran.cls defaults to using nonbold math in the Abstract.
% This preserves the distinction between vectors and scalars. However,
% if the journal you are submitting to favors bold math in the abstract,
% then you can use LaTeX's standard command \boldmath at the very start
% of the abstract to achieve this. Many IEEE journals frown on math
% in the abstract anyway.

% Note that keywords are not normally used for peerreview papers.
%\begin{IEEEkeywords}
%Cooperative diversity, decode and forward, piecewise linear
%\end{IEEEkeywords}



% For peer review papers, you can put extra information on the cover
% page as needed:
% \ifCLASSOPTIONpeerreview
% \begin{center} \bfseries EDICS Category: 3-BBND \end{center}
% \fi
%
% For peerreview papers, this IEEEtran command inserts a page break and
% creates the second title. It will be ignored for other modes.
%\IEEEpeerreviewmaketitle




	\item All the jacks, queens and kings are removed from a deck of 52 playing cards. The remaining cards are well shuffled and then one card is drawn at random. Giving ace a value 1 similar value for other cards, find the probability that the card has a value 
		\begin{enumerate}
			\item 7
			\item greater than 7
			\item less than 7
		\end{enumerate}
		%Number of cards left after removing all jacks, queens and kings 
\begin{align}
N	= 52 - 4\times 3
	= 40
\end{align}
%\begin{table}[H]
%\def\arraystretch{1.2}
%\begin{tabular}{|c|c|c|}
%\hline
%	\textbf{Parameter} &\textbf{Value} &\textbf{Description}\\ \hline
%	$X$ &1-10 &Represents the value of the card picked \\ \hline
%\end{tabular}
%\end{table}
Let $1 \le X \le 10$ be the value of the card picked.  Then,
\begin{align}
	p_X(k) &= \Pr(X=k)\ \forall\ 1 \leq k \leq 10\\
	&= \frac{4\times 1}{40}\\
	&= \frac{1}{10}\\
	\therefore p_X(k) &= 
	\begin{cases}
		\frac{1}{10} & 1 \leq k \leq 10\\
		0 & \text{otherwise}
	\end{cases}
\end{align}
and
\begin{align}
	F_{X}(k) &= \sum_{m=0}^{k}p_{X}(m) \quad 1 \leq k \leq 10\\
	&= \frac{k}{10}\\
	\therefore F_{X}(k) &= 
	\begin{cases}
		0 & k \leq 0\\
		\frac{k}{10} & 1\leq k \leq 10\\
		1 & k > 10 
	\end{cases}
\end{align}
\begin{enumerate}
	\item Probability that card has value equal to 7 is
		\begin{align}
			 p_{X}(7)
			= \frac{1}{10}
		\end{align}
	\item Probability that card has value greater than 7 is
		\begin{align}
			1 - F_X(7)
			&= 1 - \frac{7}{10}
			\\
			&= \frac{3}{10}
		\end{align}
	\item Probability that card has value less than 7 is
		\begin{align}
			 F_{X}(6)
			=\frac{6}{10}
		\end{align}
\end{enumerate}

  \item A Lot consists of 48 mobile phones of which 42 are good, 3 have only minor defects and 3 have major defects.Varnika will buy a phone if it is good but the trader will only buy a mobile if it has no major defects. One phone is selected at random from the lot. What is the probability that it is
\begin{enumerate}
	\item acceptable to Varnika?
            \item acceptable to the trader?
\end{enumerate}
\solution
	%\begin{table}[H]
	\centering
\begin{tabular}{|c|c|c|}
\hline
Random variable &Value &Definition\\ \hline
\multirow{3}{*}{X} &0 &Slips of Rs 1\\
&1 &Slips of Rs 5\\
&2 &Slips of Rs 13\\ \hline
\multirow{2}{*}{Y} &0 &Box A\\
&1 &Box B\\\hline
\end{tabular}
\caption{}
\label{tab:Distribution}
\end{table}
See \tabref{tab:Distribution}.
\begin{align}
p_{Y}\brak{k}= \begin{cases} 
      \frac{1}{3} & {k=0} \\
      \frac{2}{3 }& {k=1} 
   \end{cases}
   \\
p_{Y|X}\brak{0|0} = \frac{19}{25}\, 
p_{Y|X}\brak{0|1} = \frac{6}{25}\,
p_{Y|X}\brak{1|0} = \frac{45}{50}\,
p_{Y|X}\brak{1|2} = \frac{5}{50}
\end{align}
The desired probability is the probability that a slip drawn at random is marked other than Rs 1,
\begin{align}
&=1-p_X\brak{0}\\
&= p_X(1) + p_X(2)
\end{align}
Using Bayes theorem,
\begin{align}
&= p_Y\brak{0} \times \pr{Y=0 | X=1} + p_Y\brak{1} \times \pr{Y=1|X=2}\\
&=\frac{1}{3} \times \frac{6}{25} + \frac{2}{3} \times \frac{5}{50}\\
&=\frac{11}{75}
\end{align}

\newpage

%\tableofcontents

\bigskip

\renewcommand{\thefigure}{\theenumi}
\renewcommand{\thetable}{\theenumi}
%\renewcommand{\theequation}{\theenumi}

%\begin{abstract}
%%\boldmath
%In this letter, an algorithm for evaluating the exact analytical bit error rate  (BER)  for the piecewise linear (PL) combiner for  multiple relays is presented. Previous results were available only for upto three relays. The algorithm is unique in the sense that  the actual mathematical expressions, that are prohibitively large, need not be explicitly obtained. The diversity gain due to multiple relays is shown through plots of the analytical BER, well supported by simulations. 
%
%\end{abstract}
% IEEEtran.cls defaults to using nonbold math in the Abstract.
% This preserves the distinction between vectors and scalars. However,
% if the journal you are submitting to favors bold math in the abstract,
% then you can use LaTeX's standard command \boldmath at the very start
% of the abstract to achieve this. Many IEEE journals frown on math
% in the abstract anyway.

% Note that keywords are not normally used for peerreview papers.
%\begin{IEEEkeywords}
%Cooperative diversity, decode and forward, piecewise linear
%\end{IEEEkeywords}



% For peer review papers, you can put extra information on the cover
% page as needed:
% \ifCLASSOPTIONpeerreview
% \begin{center} \bfseries EDICS Category: 3-BBND \end{center}
% \fi
%
% For peerreview papers, this IEEEtran command inserts a page break and
% creates the second title. It will be ignored for other modes.
%\IEEEpeerreviewmaketitle




 \item A student says that if you throw a die, it will show up 1 or not 1. Therefore, the probability of getting 1 and the probability of getting 'not 1' each is equal to $\frac{1}{2}$. Is this correct? Give reasons.\\
 \solution
        %\begin{table}[H]
	\centering
\begin{tabular}{|c|c|c|}
\hline
Random variable &Value &Definition\\ \hline
\multirow{3}{*}{X} &0 &Slips of Rs 1\\
&1 &Slips of Rs 5\\
&2 &Slips of Rs 13\\ \hline
\multirow{2}{*}{Y} &0 &Box A\\
&1 &Box B\\\hline
\end{tabular}
\caption{}
\label{tab:Distribution}
\end{table}
See \tabref{tab:Distribution}.
\begin{align}
p_{Y}\brak{k}= \begin{cases} 
      \frac{1}{3} & {k=0} \\
      \frac{2}{3 }& {k=1} 
   \end{cases}
   \\
p_{Y|X}\brak{0|0} = \frac{19}{25}\, 
p_{Y|X}\brak{0|1} = \frac{6}{25}\,
p_{Y|X}\brak{1|0} = \frac{45}{50}\,
p_{Y|X}\brak{1|2} = \frac{5}{50}
\end{align}
The desired probability is the probability that a slip drawn at random is marked other than Rs 1,
\begin{align}
&=1-p_X\brak{0}\\
&= p_X(1) + p_X(2)
\end{align}
Using Bayes theorem,
\begin{align}
&= p_Y\brak{0} \times \pr{Y=0 | X=1} + p_Y\brak{1} \times \pr{Y=1|X=2}\\
&=\frac{1}{3} \times \frac{6}{25} + \frac{2}{3} \times \frac{5}{50}\\
&=\frac{11}{75}
\end{align}

\newpage

%\tableofcontents

\bigskip

\renewcommand{\thefigure}{\theenumi}
\renewcommand{\thetable}{\theenumi}
%\renewcommand{\theequation}{\theenumi}

%\begin{abstract}
%%\boldmath
%In this letter, an algorithm for evaluating the exact analytical bit error rate  (BER)  for the piecewise linear (PL) combiner for  multiple relays is presented. Previous results were available only for upto three relays. The algorithm is unique in the sense that  the actual mathematical expressions, that are prohibitively large, need not be explicitly obtained. The diversity gain due to multiple relays is shown through plots of the analytical BER, well supported by simulations. 
%
%\end{abstract}
% IEEEtran.cls defaults to using nonbold math in the Abstract.
% This preserves the distinction between vectors and scalars. However,
% if the journal you are submitting to favors bold math in the abstract,
% then you can use LaTeX's standard command \boldmath at the very start
% of the abstract to achieve this. Many IEEE journals frown on math
% in the abstract anyway.

% Note that keywords are not normally used for peerreview papers.
%\begin{IEEEkeywords}
%Cooperative diversity, decode and forward, piecewise linear
%\end{IEEEkeywords}



% For peer review papers, you can put extra information on the cover
% page as needed:
% \ifCLASSOPTIONpeerreview
% \begin{center} \bfseries EDICS Category: 3-BBND \end{center}
% \fi
%
% For peerreview papers, this IEEEtran command inserts a page break and
% creates the second title. It will be ignored for other modes.
%\IEEEpeerreviewmaketitle




   \item Four candidates A, B, C, D have ap-
plied for the assignment to coach a school cricket
team. If A is twice as likely to be selected as B, and
B and C are given about the same chance of being
selected, while C is twice as likely to be selected
as D, what are the probabilities that
\begin{enumerate}
\item C will be selected?
\item A will not be selected?
\end{enumerate}
	%\begin{table}[H]
	\centering
\begin{tabular}{|c|c|c|}
\hline
Random variable &Value &Definition\\ \hline
\multirow{3}{*}{X} &0 &Slips of Rs 1\\
&1 &Slips of Rs 5\\
&2 &Slips of Rs 13\\ \hline
\multirow{2}{*}{Y} &0 &Box A\\
&1 &Box B\\\hline
\end{tabular}
\caption{}
\label{tab:Distribution}
\end{table}
See \tabref{tab:Distribution}.
\begin{align}
p_{Y}\brak{k}= \begin{cases} 
      \frac{1}{3} & {k=0} \\
      \frac{2}{3 }& {k=1} 
   \end{cases}
   \\
p_{Y|X}\brak{0|0} = \frac{19}{25}\, 
p_{Y|X}\brak{0|1} = \frac{6}{25}\,
p_{Y|X}\brak{1|0} = \frac{45}{50}\,
p_{Y|X}\brak{1|2} = \frac{5}{50}
\end{align}
The desired probability is the probability that a slip drawn at random is marked other than Rs 1,
\begin{align}
&=1-p_X\brak{0}\\
&= p_X(1) + p_X(2)
\end{align}
Using Bayes theorem,
\begin{align}
&= p_Y\brak{0} \times \pr{Y=0 | X=1} + p_Y\brak{1} \times \pr{Y=1|X=2}\\
&=\frac{1}{3} \times \frac{6}{25} + \frac{2}{3} \times \frac{5}{50}\\
&=\frac{11}{75}
\end{align}

\newpage

%\tableofcontents

\bigskip

\renewcommand{\thefigure}{\theenumi}
\renewcommand{\thetable}{\theenumi}
%\renewcommand{\theequation}{\theenumi}

%\begin{abstract}
%%\boldmath
%In this letter, an algorithm for evaluating the exact analytical bit error rate  (BER)  for the piecewise linear (PL) combiner for  multiple relays is presented. Previous results were available only for upto three relays. The algorithm is unique in the sense that  the actual mathematical expressions, that are prohibitively large, need not be explicitly obtained. The diversity gain due to multiple relays is shown through plots of the analytical BER, well supported by simulations. 
%
%\end{abstract}
% IEEEtran.cls defaults to using nonbold math in the Abstract.
% This preserves the distinction between vectors and scalars. However,
% if the journal you are submitting to favors bold math in the abstract,
% then you can use LaTeX's standard command \boldmath at the very start
% of the abstract to achieve this. Many IEEE journals frown on math
% in the abstract anyway.

% Note that keywords are not normally used for peerreview papers.
%\begin{IEEEkeywords}
%Cooperative diversity, decode and forward, piecewise linear
%\end{IEEEkeywords}



% For peer review papers, you can put extra information on the cover
% page as needed:
% \ifCLASSOPTIONpeerreview
% \begin{center} \bfseries EDICS Category: 3-BBND \end{center}
% \fi
%
% For peerreview papers, this IEEEtran command inserts a page break and
% creates the second title. It will be ignored for other modes.
%\IEEEpeerreviewmaketitle




 \item A bag contain 24 balls of which $x$ balls are red, $2x$ are white and $3x$ are blue. A ball is selected at random, What is the probability that it is
\begin{enumerate}[label=\alph*)]
\item not red ?
\item white ?
\end{enumerate}
%\begin{table}[H]
	\centering
\begin{tabular}{|c|c|c|}
\hline
Random variable &Value &Definition\\ \hline
\multirow{3}{*}{X} &0 &Slips of Rs 1\\
&1 &Slips of Rs 5\\
&2 &Slips of Rs 13\\ \hline
\multirow{2}{*}{Y} &0 &Box A\\
&1 &Box B\\\hline
\end{tabular}
\caption{}
\label{tab:Distribution}
\end{table}
See \tabref{tab:Distribution}.
\begin{align}
p_{Y}\brak{k}= \begin{cases} 
      \frac{1}{3} & {k=0} \\
      \frac{2}{3 }& {k=1} 
   \end{cases}
   \\
p_{Y|X}\brak{0|0} = \frac{19}{25}\, 
p_{Y|X}\brak{0|1} = \frac{6}{25}\,
p_{Y|X}\brak{1|0} = \frac{45}{50}\,
p_{Y|X}\brak{1|2} = \frac{5}{50}
\end{align}
The desired probability is the probability that a slip drawn at random is marked other than Rs 1,
\begin{align}
&=1-p_X\brak{0}\\
&= p_X(1) + p_X(2)
\end{align}
Using Bayes theorem,
\begin{align}
&= p_Y\brak{0} \times \pr{Y=0 | X=1} + p_Y\brak{1} \times \pr{Y=1|X=2}\\
&=\frac{1}{3} \times \frac{6}{25} + \frac{2}{3} \times \frac{5}{50}\\
&=\frac{11}{75}
\end{align}

\newpage

%\tableofcontents

\bigskip

\renewcommand{\thefigure}{\theenumi}
\renewcommand{\thetable}{\theenumi}
%\renewcommand{\theequation}{\theenumi}

%\begin{abstract}
%%\boldmath
%In this letter, an algorithm for evaluating the exact analytical bit error rate  (BER)  for the piecewise linear (PL) combiner for  multiple relays is presented. Previous results were available only for upto three relays. The algorithm is unique in the sense that  the actual mathematical expressions, that are prohibitively large, need not be explicitly obtained. The diversity gain due to multiple relays is shown through plots of the analytical BER, well supported by simulations. 
%
%\end{abstract}
% IEEEtran.cls defaults to using nonbold math in the Abstract.
% This preserves the distinction between vectors and scalars. However,
% if the journal you are submitting to favors bold math in the abstract,
% then you can use LaTeX's standard command \boldmath at the very start
% of the abstract to achieve this. Many IEEE journals frown on math
% in the abstract anyway.

% Note that keywords are not normally used for peerreview papers.
%\begin{IEEEkeywords}
%Cooperative diversity, decode and forward, piecewise linear
%\end{IEEEkeywords}



% For peer review papers, you can put extra information on the cover
% page as needed:
% \ifCLASSOPTIONpeerreview
% \begin{center} \bfseries EDICS Category: 3-BBND \end{center}
% \fi
%
% For peerreview papers, this IEEEtran command inserts a page break and
% creates the second title. It will be ignored for other modes.
%\IEEEpeerreviewmaketitle




If the letters of the word ASSASSINATION are arranged at random. Find the Probability that
\begin{enumerate}[label=(\alph*)]
\item Four $S's$ come consecutively in the word
\item Two  $I's$ and two $N's$ come together
\item All $A's$ are not coming together
\item No two $A's$ are coming together
\end{enumerate}
%\begin{table}[H]
	\centering
\begin{tabular}{|c|c|c|}
\hline
Random variable &Value &Definition\\ \hline
\multirow{3}{*}{X} &0 &Slips of Rs 1\\
&1 &Slips of Rs 5\\
&2 &Slips of Rs 13\\ \hline
\multirow{2}{*}{Y} &0 &Box A\\
&1 &Box B\\\hline
\end{tabular}
\caption{}
\label{tab:Distribution}
\end{table}
See \tabref{tab:Distribution}.
\begin{align}
p_{Y}\brak{k}= \begin{cases} 
      \frac{1}{3} & {k=0} \\
      \frac{2}{3 }& {k=1} 
   \end{cases}
   \\
p_{Y|X}\brak{0|0} = \frac{19}{25}\, 
p_{Y|X}\brak{0|1} = \frac{6}{25}\,
p_{Y|X}\brak{1|0} = \frac{45}{50}\,
p_{Y|X}\brak{1|2} = \frac{5}{50}
\end{align}
The desired probability is the probability that a slip drawn at random is marked other than Rs 1,
\begin{align}
&=1-p_X\brak{0}\\
&= p_X(1) + p_X(2)
\end{align}
Using Bayes theorem,
\begin{align}
&= p_Y\brak{0} \times \pr{Y=0 | X=1} + p_Y\brak{1} \times \pr{Y=1|X=2}\\
&=\frac{1}{3} \times \frac{6}{25} + \frac{2}{3} \times \frac{5}{50}\\
&=\frac{11}{75}
\end{align}

\newpage

%\tableofcontents

\bigskip

\renewcommand{\thefigure}{\theenumi}
\renewcommand{\thetable}{\theenumi}
%\renewcommand{\theequation}{\theenumi}

%\begin{abstract}
%%\boldmath
%In this letter, an algorithm for evaluating the exact analytical bit error rate  (BER)  for the piecewise linear (PL) combiner for  multiple relays is presented. Previous results were available only for upto three relays. The algorithm is unique in the sense that  the actual mathematical expressions, that are prohibitively large, need not be explicitly obtained. The diversity gain due to multiple relays is shown through plots of the analytical BER, well supported by simulations. 
%
%\end{abstract}
% IEEEtran.cls defaults to using nonbold math in the Abstract.
% This preserves the distinction between vectors and scalars. However,
% if the journal you are submitting to favors bold math in the abstract,
% then you can use LaTeX's standard command \boldmath at the very start
% of the abstract to achieve this. Many IEEE journals frown on math
% in the abstract anyway.

% Note that keywords are not normally used for peerreview papers.
%\begin{IEEEkeywords}
%Cooperative diversity, decode and forward, piecewise linear
%\end{IEEEkeywords}



% For peer review papers, you can put extra information on the cover
% page as needed:
% \ifCLASSOPTIONpeerreview
% \begin{center} \bfseries EDICS Category: 3-BBND \end{center}
% \fi
%
% For peerreview papers, this IEEEtran command inserts a page break and
% creates the second title. It will be ignored for other modes.
%\IEEEpeerreviewmaketitle




	\item One urn contains two black balls (labelled B1 and B2) and one white ball. A
	second urn contains one black ball and two white balls (labelled W1 and W2).
	Suppose the following experiment is performed. One of the two urns is chosen
	at random. Next a ball is randomly chosen from the urn. Then a second ball is
	chosen at random from the same urn without replacing the first ball.
	
	\begin{enumerate}
	\item What is the probability that two black balls are chosen?
	
	\item What is the probability that two balls of opposite colour are chosen?
	\end{enumerate}
	\solution
	%\begin{align}
    \label{eq:12.13.6.18.1}
	\because	\pr{A|B} &> \pr{A},\
\frac{\pr{AB}}{\pr{B}} > \pr{A}
\\
    \label{eq:12.13.6.18.2}
	\implies \pr{AB} &> \pr{A}\pr{B}
	\\
	\text{or, } \frac{\pr{AB}}{\pr{A}} &=\pr{B|A} > \pr{A}
\end{align}

\end{enumerate}

	\item 
The number lock of a suitcase has 4 wheels each labelled with ten digits i.e. from 0 to 9.The lock opens with a sequence of four digits with no repeats.What is the probability of a person getting the right sequence to open the suitcase.
\\
\solution
		%\begin{enumerate}[label=\thesection.\arabic*,ref=\thesection.\theenumi]
	\item One card is drawn from a well-shuffled deck of 52 cards. Find the probability of getting
\begin{enumerate}
\item A king of red colour 
\item A face card 
\item A red face card
\item The jack of hearts
\item A spade
\item The queen of diamonds

\end{enumerate}
\solution
		%\begin{table}[H]
	\centering
\begin{tabular}{|c|c|c|}
\hline
Random variable &Value &Definition\\ \hline
\multirow{3}{*}{X} &0 &Slips of Rs 1\\
&1 &Slips of Rs 5\\
&2 &Slips of Rs 13\\ \hline
\multirow{2}{*}{Y} &0 &Box A\\
&1 &Box B\\\hline
\end{tabular}
\caption{}
\label{tab:Distribution}
\end{table}
See \tabref{tab:Distribution}.
\begin{align}
p_{Y}\brak{k}= \begin{cases} 
      \frac{1}{3} & {k=0} \\
      \frac{2}{3 }& {k=1} 
   \end{cases}
   \\
p_{Y|X}\brak{0|0} = \frac{19}{25}\, 
p_{Y|X}\brak{0|1} = \frac{6}{25}\,
p_{Y|X}\brak{1|0} = \frac{45}{50}\,
p_{Y|X}\brak{1|2} = \frac{5}{50}
\end{align}
The desired probability is the probability that a slip drawn at random is marked other than Rs 1,
\begin{align}
&=1-p_X\brak{0}\\
&= p_X(1) + p_X(2)
\end{align}
Using Bayes theorem,
\begin{align}
&= p_Y\brak{0} \times \pr{Y=0 | X=1} + p_Y\brak{1} \times \pr{Y=1|X=2}\\
&=\frac{1}{3} \times \frac{6}{25} + \frac{2}{3} \times \frac{5}{50}\\
&=\frac{11}{75}
\end{align}

\newpage

%\tableofcontents

\bigskip

\renewcommand{\thefigure}{\theenumi}
\renewcommand{\thetable}{\theenumi}
%\renewcommand{\theequation}{\theenumi}

%\begin{abstract}
%%\boldmath
%In this letter, an algorithm for evaluating the exact analytical bit error rate  (BER)  for the piecewise linear (PL) combiner for  multiple relays is presented. Previous results were available only for upto three relays. The algorithm is unique in the sense that  the actual mathematical expressions, that are prohibitively large, need not be explicitly obtained. The diversity gain due to multiple relays is shown through plots of the analytical BER, well supported by simulations. 
%
%\end{abstract}
% IEEEtran.cls defaults to using nonbold math in the Abstract.
% This preserves the distinction between vectors and scalars. However,
% if the journal you are submitting to favors bold math in the abstract,
% then you can use LaTeX's standard command \boldmath at the very start
% of the abstract to achieve this. Many IEEE journals frown on math
% in the abstract anyway.

% Note that keywords are not normally used for peerreview papers.
%\begin{IEEEkeywords}
%Cooperative diversity, decode and forward, piecewise linear
%\end{IEEEkeywords}



% For peer review papers, you can put extra information on the cover
% page as needed:
% \ifCLASSOPTIONpeerreview
% \begin{center} \bfseries EDICS Category: 3-BBND \end{center}
% \fi
%
% For peerreview papers, this IEEEtran command inserts a page break and
% creates the second title. It will be ignored for other modes.
%\IEEEpeerreviewmaketitle




	\item Five cards—the ten, jack, queen, king and ace of diamonds, are well-shuffled with their face downwards. One card is then picked up at random.
\begin{enumerate}
\item
What is the probability that the card is the queen? 
\item
If the queen is drawn and put aside, what is the probability that the second card picked up is (a) an ace? (b) a queen?\\
\end{enumerate}
\solution
		%\begin{enumerate}[label=\thesection.\arabic*,ref=\thesection.\theenumi]
	\item One card is drawn from a well-shuffled deck of 52 cards. Find the probability of getting
\begin{enumerate}
\item A king of red colour 
\item A face card 
\item A red face card
\item The jack of hearts
\item A spade
\item The queen of diamonds

\end{enumerate}
\solution
		%\input{ncert/10/15/1/14/main.tex}
	\item Five cards—the ten, jack, queen, king and ace of diamonds, are well-shuffled with their face downwards. One card is then picked up at random.
\begin{enumerate}
\item
What is the probability that the card is the queen? 
\item
If the queen is drawn and put aside, what is the probability that the second card picked up is (a) an ace? (b) a queen?\\
\end{enumerate}
\solution
		%\input{ncert/10/15/1/15/defs.tex}
	\item A bag contains $5$ red balls and some blue balls. If the probability of drawing a blue ball is double that if a red ball, determine the number of blue balls in the bag. 
		\\
\solution
		%\input{ncert/10/15/2/3/defs.tex}
	\item A card is selected from a pack of 52 cards.
 \begin{enumerate}[label=(\alph*)] 
                 \item How many points are there in the sample space?
                 \item Calculate the probability that the card is an ace of spades.
                 \item Calculate the probability that the card is (i) an ace and (ii) black card.
 \end{enumerate}
\solution
		%\input{ncert/11/16/3/4/main.tex}
\item Four cards are drawn from a well-shuffled deck of 52 cards. What is the probability of obtaining 3 diamonds and one spade.
\\
\solution
		%\input{ncert/11/16/4/2/defs.tex}
\item In a certain lottery 10,000 tickets are sold and ten equal prizes are awarded. What is the probability of not getting a prize if you buy (a) one ticket (b) two tickets (c) 10 tickets ?	
\\
\solution
		%\input{ncert/11/16/4/4/defs.tex}
		%
\item 
Out of 100 students, two sections of 40 and 60 are formed. If you and your friend are among the 100 students, what is the probability that
\begin{enumerate}
\item you both enter the same section?
\item you both enter the different sections?
\end{enumerate}
\solution
		%\input{ncert/11/16/4/5/defs.tex}
	\item 
The number lock of a suitcase has 4 wheels each labelled with ten digits i.e. from 0 to 9.The lock opens with a sequence of four digits with no repeats.What is the probability of a person getting the right sequence to open the suitcase.
\\
\solution
		%\input{ncert/11/16/4/10/defs.tex}
		%
\item 
Two cards are drawn at random and without replacement from a pack of 52 playing cards. Find the probability that both the cards are black.
\\
\solution
		%\input{ncert/12/13/2/2/defs.tex}
		\item A box of oranges is inspected by examining three randomly selected oranges drawn without replacement. If all the three oranges are good, the box is approved for sale, otherwise, it is rejected. Find the probability that a box containing 15 oranges out of which 12 are good and 3 are bad ones will be approved for sale.
		\label{ncert/12/13/2/3/defs.tex}
		\item Two balls are drawn at random with replacement from a box containing 10 black and 8 red balls. Find the probability that
		\label{ncert/12/13/2/12}
\begin{enumerate}
\item both balls are red.
\item first ball is black and second is red.
\item one of them is black and other is red.
\end{enumerate}

\item In a hostel, 60\% of the students read Hindi newspaper, 40\% read English newspaper and 20\% read both Hindi and English newspapers. A student is selected at random.
		\label{ncert/12/13/2/15}
\begin{enumerate}
\item Find the probability that she reads neither Hindi nor English newspapers.
\item If she reads Hindi newspaper, find the probability that she reads English newspaper.
\item If she reads English newspaper, find the probability that she reads Hindi newspaper.\\
\end{enumerate}
\item The probability of obtaining an even prime number on each die, when a pair of dice is rolled is 
\begin{enumerate}
    \item $0$ 
    
    \item $\frac{1}{3}$ 
    
    \item $\frac{1}{12}$ 
    
    \item $\frac{1}{36}$ 
\end{enumerate}
\solution
		%\input{ncert/12/13/2/17/defs.tex}
	\item A bag contains 4 red and 4 black balls, another bag contains 2 red and 6 black balls. One of the two bags is selected at random and a ball is drawn from the bag which is found to be red. Find the probability that the ball is drawn from the first bag.
\\
\solution
		%\input{ncert/12/13/3/2/main.tex}
  \item
  Cards with numbers 2 to 101 are placed in a box. A card is selected at random.Find the probability that the card has
\begin{enumerate}[label=(\roman*)]
	\item an even number 
	\item a square number
\end{enumerate}
\solution
%\input{exemplar/10/13/3/32/main.tex}
\item
The king, queen and jack of clubs are removed from a deck of 52 playing cards and then well shuffled. Now one card is drawn at random from the remaining cards.  Determine the probability that the card is
\begin{enumerate}[label=(\roman*)]
\item a club
\item 10 of hearts
\end{enumerate}
\solution
%\input{exemplar/10/13/3/29/main.tex}
\item A team of medical students doing their internship have to assist during surgeries
at a city hospital. The probabilities of surgeries rated as very complex, complex,
routine, simple or very simple are respectively, 0.15, 0.20, 0.31, 0.26, .08. Find
the probabilities that a particular surgery will be rated
\begin{enumerate}
	\item complex or very complex;
	\item neither very complex nor very simple;
	\item routine or complex
	\item routine or simple
\end{enumerate}
\solution
%\input{exemplar/11/16/3/8(1)/main.tex}
\item A card is selected from a pack of 52 cards.
\begin{enumerate}[label=(\alph*)]
    \item How many points are there in the sample space?
    \item Calculate the probability that the card is an ace of spades.
    \item Calculate the probability that the card is (i) an ace and (ii) black card.
\end{enumerate}
\solution
%\input{exemplar/11/16/3/4/main2.tex}
\item The probability that a non leap year selected at random will contain 53 sundays.
\\
\solution
%\input{exemplar/10/13/1/19/main.tex}
\item One of the four persons John, Rita, Aslam or Gurpreet will be promoted next
month. Consequently the sample space consists of four elementary outcomes
S = {John promoted, Rita promoted, Aslam promoted, Gurpreet promoted}
You are told that the chances of John’s promotion is same as that of Gurpreet,
Rita’s chances of promotion are twice as likely as Johns. Aslam’s chances are
four times that of John.
\begin{enumerate}
	\item Determine
	\begin{enumerate}
		\item P (John promoted)
		\item P (Rita promoted)
		\item P (Aslam promoted)
		\item P (Gurpreet promoted)
	\end{enumerate}
	\item If A = {John promoted or Gurpreet promoted}, find P (A).
\end{enumerate}
\solution
%\input{exemplar/11/16/3/10/main.tex}
\item A card is drawn from a deck of 52 cards. Find the probability of getting a king or a heart or a red card.\\
\solution
%\input{exemplar/11/16/3/15/main.tex}
\item The probability that a student will pass his examination is 0.73, the probability of
the student getting a compartment is 0.13, and the probability that the student will
either pass or get compartment is 0.96. State True or False.\\
\solution
%\input{exemplar/11/16/3/31/main.tex}
\item A card is selected from a pack of 52 cards\\
\begin{enumerate}[label=(\alph*)]
\item How many points are there in the sample space?
\item Calculate the probability that the cards is an ace of spades.
\item Calculate the probability that the card is (i) an ace (ii)black card.\\
\end{enumerate}
%\input{ncert/11/16/3/4_1/Prob_4.tex}
\item In a non-leap year, the probability of having 53 tuesdays or 53 wednesdays is\\
\solution
%\input{exemplar/11/16/3/18/main.tex}
\item There are 1000 sealed envelopes in a box, 10 of them contain a cash prize of
Rs 100 each, 100 of them contain a cash prize of Rs 50 each and 200 of them
contain a cash prize of Rs 10 each and rest do not contain any cash prize. If they
are well shuffled and an envelope is picked up out, what is the probability that it
contains no cash prize?\\
\solution
%\input{exemplar/10/13/3/34/main.tex}
\item 
A die is thrown and a card is selected at random from a deck of 52 playing cards. The probability of getting an even number on the die and a spade card.\\
\solution
%\input{exemplar/12/13/3/78/main.tex}
\item
If 4-digit numbers greater than 5,000 are randomly formed from the digits 0, 1, 3, 5, and 7, what is the probability of forming a number divisible by 5 when:
\begin{enumerate}
    \item The digits are repeated?
    \item The repetition of digits is not allowed?
\end{enumerate}
\solution
%\input{ncert/11/16/4/9/main.tex}
\item Consider the probability space $\brak{\Omega, \mathcal{G}, P}$ where $\Omega = [0,2]$ and $\mathcal{G} = \cbrak{\phi, \Omega, [0,1], (1,2]}$. Let $X$ and $Y$ be two functions on $\Omega$ defined as
\begin{align*}
    X(\omega) = 
    \begin{cases}
        1 & \text{if }\omega \in [0, 1]\\
        2 & \text{if }\omega \in (1, 2]
    \end{cases}
\end{align*}
and
\begin{align*}
    Y(\omega) = 
    \begin{cases}
        2 & \text{if }\omega \in [0, 1.5]\\
        3 & \text{if }\omega \in (1.5, 2].
    \end{cases}
\end{align*}
Then which one of the following statements is true?
\begin{enumerate}
    \item [(A)] $X$ is a random variable with respect to $\mathcal{G}$, but $Y$ is not a random variable with respect to $\mathcal{G}$.
    \item [(B)] $Y$ is a random variable with respect to $\mathcal{G}$, but $X$ is not a random variable with respect to $\mathcal{G}$.
    \item [(C)] Neither $X$ nor $Y$ is a random variable with respect to $\mathcal{G}$.
    \item [(D)] Both $X$ and $Y$ are random variables with respect to $\mathcal{G}$.
\end{enumerate} \hfill (GATE ST 2023)\\
\solution
%\input{gate/ST/2023/14/main.tex}
	\item  A die is loaded in such a way that each odd number is twice as likely to occur as
each even number. Find $P(G)$, where $G$ is the event that a number greater than
3 occurs on a single roll of the die.
\\
\solution
		%\input{exemplar/11/16/3/5/main.tex}
	\item All the jacks, queens and kings are removed from a deck of 52 playing cards. The remaining cards are well shuffled and then one card is drawn at random. Giving ace a value 1 similar value for other cards, find the probability that the card has a value 
		\begin{enumerate}
			\item 7
			\item greater than 7
			\item less than 7
		\end{enumerate}
		%\input{exemplar/10/13/3/30/main.tex}
  \item A Lot consists of 48 mobile phones of which 42 are good, 3 have only minor defects and 3 have major defects.Varnika will buy a phone if it is good but the trader will only buy a mobile if it has no major defects. One phone is selected at random from the lot. What is the probability that it is
\begin{enumerate}
	\item acceptable to Varnika?
            \item acceptable to the trader?
\end{enumerate}
\solution
	%\input{exemplar/10/13/3/40/main.tex}
 \item A student says that if you throw a die, it will show up 1 or not 1. Therefore, the probability of getting 1 and the probability of getting 'not 1' each is equal to $\frac{1}{2}$. Is this correct? Give reasons.\\
 \solution
        %\input{exemplar/10/13/2/9/main.tex}
   \item Four candidates A, B, C, D have ap-
plied for the assignment to coach a school cricket
team. If A is twice as likely to be selected as B, and
B and C are given about the same chance of being
selected, while C is twice as likely to be selected
as D, what are the probabilities that
\begin{enumerate}
\item C will be selected?
\item A will not be selected?
\end{enumerate}
	%\input{exemplar/11/16/3/9/main.tex}
 \item A bag contain 24 balls of which $x$ balls are red, $2x$ are white and $3x$ are blue. A ball is selected at random, What is the probability that it is
\begin{enumerate}[label=\alph*)]
\item not red ?
\item white ?
\end{enumerate}
%\input{exemplar/10/13/3/41/main.tex}
If the letters of the word ASSASSINATION are arranged at random. Find the Probability that
\begin{enumerate}[label=(\alph*)]
\item Four $S's$ come consecutively in the word
\item Two  $I's$ and two $N's$ come together
\item All $A's$ are not coming together
\item No two $A's$ are coming together
\end{enumerate}
%\input{exemplar/11/16/3/14/main.tex}
	\item One urn contains two black balls (labelled B1 and B2) and one white ball. A
	second urn contains one black ball and two white balls (labelled W1 and W2).
	Suppose the following experiment is performed. One of the two urns is chosen
	at random. Next a ball is randomly chosen from the urn. Then a second ball is
	chosen at random from the same urn without replacing the first ball.
	
	\begin{enumerate}
	\item What is the probability that two black balls are chosen?
	
	\item What is the probability that two balls of opposite colour are chosen?
	\end{enumerate}
	\solution
	%\input{exemplar/11/16/3/12/main1.tex}
\end{enumerate}

	\item A bag contains $5$ red balls and some blue balls. If the probability of drawing a blue ball is double that if a red ball, determine the number of blue balls in the bag. 
		\\
\solution
		%\begin{enumerate}[label=\thesection.\arabic*,ref=\thesection.\theenumi]
	\item One card is drawn from a well-shuffled deck of 52 cards. Find the probability of getting
\begin{enumerate}
\item A king of red colour 
\item A face card 
\item A red face card
\item The jack of hearts
\item A spade
\item The queen of diamonds

\end{enumerate}
\solution
		%\input{ncert/10/15/1/14/main.tex}
	\item Five cards—the ten, jack, queen, king and ace of diamonds, are well-shuffled with their face downwards. One card is then picked up at random.
\begin{enumerate}
\item
What is the probability that the card is the queen? 
\item
If the queen is drawn and put aside, what is the probability that the second card picked up is (a) an ace? (b) a queen?\\
\end{enumerate}
\solution
		%\input{ncert/10/15/1/15/defs.tex}
	\item A bag contains $5$ red balls and some blue balls. If the probability of drawing a blue ball is double that if a red ball, determine the number of blue balls in the bag. 
		\\
\solution
		%\input{ncert/10/15/2/3/defs.tex}
	\item A card is selected from a pack of 52 cards.
 \begin{enumerate}[label=(\alph*)] 
                 \item How many points are there in the sample space?
                 \item Calculate the probability that the card is an ace of spades.
                 \item Calculate the probability that the card is (i) an ace and (ii) black card.
 \end{enumerate}
\solution
		%\input{ncert/11/16/3/4/main.tex}
\item Four cards are drawn from a well-shuffled deck of 52 cards. What is the probability of obtaining 3 diamonds and one spade.
\\
\solution
		%\input{ncert/11/16/4/2/defs.tex}
\item In a certain lottery 10,000 tickets are sold and ten equal prizes are awarded. What is the probability of not getting a prize if you buy (a) one ticket (b) two tickets (c) 10 tickets ?	
\\
\solution
		%\input{ncert/11/16/4/4/defs.tex}
		%
\item 
Out of 100 students, two sections of 40 and 60 are formed. If you and your friend are among the 100 students, what is the probability that
\begin{enumerate}
\item you both enter the same section?
\item you both enter the different sections?
\end{enumerate}
\solution
		%\input{ncert/11/16/4/5/defs.tex}
	\item 
The number lock of a suitcase has 4 wheels each labelled with ten digits i.e. from 0 to 9.The lock opens with a sequence of four digits with no repeats.What is the probability of a person getting the right sequence to open the suitcase.
\\
\solution
		%\input{ncert/11/16/4/10/defs.tex}
		%
\item 
Two cards are drawn at random and without replacement from a pack of 52 playing cards. Find the probability that both the cards are black.
\\
\solution
		%\input{ncert/12/13/2/2/defs.tex}
		\item A box of oranges is inspected by examining three randomly selected oranges drawn without replacement. If all the three oranges are good, the box is approved for sale, otherwise, it is rejected. Find the probability that a box containing 15 oranges out of which 12 are good and 3 are bad ones will be approved for sale.
		\label{ncert/12/13/2/3/defs.tex}
		\item Two balls are drawn at random with replacement from a box containing 10 black and 8 red balls. Find the probability that
		\label{ncert/12/13/2/12}
\begin{enumerate}
\item both balls are red.
\item first ball is black and second is red.
\item one of them is black and other is red.
\end{enumerate}

\item In a hostel, 60\% of the students read Hindi newspaper, 40\% read English newspaper and 20\% read both Hindi and English newspapers. A student is selected at random.
		\label{ncert/12/13/2/15}
\begin{enumerate}
\item Find the probability that she reads neither Hindi nor English newspapers.
\item If she reads Hindi newspaper, find the probability that she reads English newspaper.
\item If she reads English newspaper, find the probability that she reads Hindi newspaper.\\
\end{enumerate}
\item The probability of obtaining an even prime number on each die, when a pair of dice is rolled is 
\begin{enumerate}
    \item $0$ 
    
    \item $\frac{1}{3}$ 
    
    \item $\frac{1}{12}$ 
    
    \item $\frac{1}{36}$ 
\end{enumerate}
\solution
		%\input{ncert/12/13/2/17/defs.tex}
	\item A bag contains 4 red and 4 black balls, another bag contains 2 red and 6 black balls. One of the two bags is selected at random and a ball is drawn from the bag which is found to be red. Find the probability that the ball is drawn from the first bag.
\\
\solution
		%\input{ncert/12/13/3/2/main.tex}
  \item
  Cards with numbers 2 to 101 are placed in a box. A card is selected at random.Find the probability that the card has
\begin{enumerate}[label=(\roman*)]
	\item an even number 
	\item a square number
\end{enumerate}
\solution
%\input{exemplar/10/13/3/32/main.tex}
\item
The king, queen and jack of clubs are removed from a deck of 52 playing cards and then well shuffled. Now one card is drawn at random from the remaining cards.  Determine the probability that the card is
\begin{enumerate}[label=(\roman*)]
\item a club
\item 10 of hearts
\end{enumerate}
\solution
%\input{exemplar/10/13/3/29/main.tex}
\item A team of medical students doing their internship have to assist during surgeries
at a city hospital. The probabilities of surgeries rated as very complex, complex,
routine, simple or very simple are respectively, 0.15, 0.20, 0.31, 0.26, .08. Find
the probabilities that a particular surgery will be rated
\begin{enumerate}
	\item complex or very complex;
	\item neither very complex nor very simple;
	\item routine or complex
	\item routine or simple
\end{enumerate}
\solution
%\input{exemplar/11/16/3/8(1)/main.tex}
\item A card is selected from a pack of 52 cards.
\begin{enumerate}[label=(\alph*)]
    \item How many points are there in the sample space?
    \item Calculate the probability that the card is an ace of spades.
    \item Calculate the probability that the card is (i) an ace and (ii) black card.
\end{enumerate}
\solution
%\input{exemplar/11/16/3/4/main2.tex}
\item The probability that a non leap year selected at random will contain 53 sundays.
\\
\solution
%\input{exemplar/10/13/1/19/main.tex}
\item One of the four persons John, Rita, Aslam or Gurpreet will be promoted next
month. Consequently the sample space consists of four elementary outcomes
S = {John promoted, Rita promoted, Aslam promoted, Gurpreet promoted}
You are told that the chances of John’s promotion is same as that of Gurpreet,
Rita’s chances of promotion are twice as likely as Johns. Aslam’s chances are
four times that of John.
\begin{enumerate}
	\item Determine
	\begin{enumerate}
		\item P (John promoted)
		\item P (Rita promoted)
		\item P (Aslam promoted)
		\item P (Gurpreet promoted)
	\end{enumerate}
	\item If A = {John promoted or Gurpreet promoted}, find P (A).
\end{enumerate}
\solution
%\input{exemplar/11/16/3/10/main.tex}
\item A card is drawn from a deck of 52 cards. Find the probability of getting a king or a heart or a red card.\\
\solution
%\input{exemplar/11/16/3/15/main.tex}
\item The probability that a student will pass his examination is 0.73, the probability of
the student getting a compartment is 0.13, and the probability that the student will
either pass or get compartment is 0.96. State True or False.\\
\solution
%\input{exemplar/11/16/3/31/main.tex}
\item A card is selected from a pack of 52 cards\\
\begin{enumerate}[label=(\alph*)]
\item How many points are there in the sample space?
\item Calculate the probability that the cards is an ace of spades.
\item Calculate the probability that the card is (i) an ace (ii)black card.\\
\end{enumerate}
%\input{ncert/11/16/3/4_1/Prob_4.tex}
\item In a non-leap year, the probability of having 53 tuesdays or 53 wednesdays is\\
\solution
%\input{exemplar/11/16/3/18/main.tex}
\item There are 1000 sealed envelopes in a box, 10 of them contain a cash prize of
Rs 100 each, 100 of them contain a cash prize of Rs 50 each and 200 of them
contain a cash prize of Rs 10 each and rest do not contain any cash prize. If they
are well shuffled and an envelope is picked up out, what is the probability that it
contains no cash prize?\\
\solution
%\input{exemplar/10/13/3/34/main.tex}
\item 
A die is thrown and a card is selected at random from a deck of 52 playing cards. The probability of getting an even number on the die and a spade card.\\
\solution
%\input{exemplar/12/13/3/78/main.tex}
\item
If 4-digit numbers greater than 5,000 are randomly formed from the digits 0, 1, 3, 5, and 7, what is the probability of forming a number divisible by 5 when:
\begin{enumerate}
    \item The digits are repeated?
    \item The repetition of digits is not allowed?
\end{enumerate}
\solution
%\input{ncert/11/16/4/9/main.tex}
\item Consider the probability space $\brak{\Omega, \mathcal{G}, P}$ where $\Omega = [0,2]$ and $\mathcal{G} = \cbrak{\phi, \Omega, [0,1], (1,2]}$. Let $X$ and $Y$ be two functions on $\Omega$ defined as
\begin{align*}
    X(\omega) = 
    \begin{cases}
        1 & \text{if }\omega \in [0, 1]\\
        2 & \text{if }\omega \in (1, 2]
    \end{cases}
\end{align*}
and
\begin{align*}
    Y(\omega) = 
    \begin{cases}
        2 & \text{if }\omega \in [0, 1.5]\\
        3 & \text{if }\omega \in (1.5, 2].
    \end{cases}
\end{align*}
Then which one of the following statements is true?
\begin{enumerate}
    \item [(A)] $X$ is a random variable with respect to $\mathcal{G}$, but $Y$ is not a random variable with respect to $\mathcal{G}$.
    \item [(B)] $Y$ is a random variable with respect to $\mathcal{G}$, but $X$ is not a random variable with respect to $\mathcal{G}$.
    \item [(C)] Neither $X$ nor $Y$ is a random variable with respect to $\mathcal{G}$.
    \item [(D)] Both $X$ and $Y$ are random variables with respect to $\mathcal{G}$.
\end{enumerate} \hfill (GATE ST 2023)\\
\solution
%\input{gate/ST/2023/14/main.tex}
	\item  A die is loaded in such a way that each odd number is twice as likely to occur as
each even number. Find $P(G)$, where $G$ is the event that a number greater than
3 occurs on a single roll of the die.
\\
\solution
		%\input{exemplar/11/16/3/5/main.tex}
	\item All the jacks, queens and kings are removed from a deck of 52 playing cards. The remaining cards are well shuffled and then one card is drawn at random. Giving ace a value 1 similar value for other cards, find the probability that the card has a value 
		\begin{enumerate}
			\item 7
			\item greater than 7
			\item less than 7
		\end{enumerate}
		%\input{exemplar/10/13/3/30/main.tex}
  \item A Lot consists of 48 mobile phones of which 42 are good, 3 have only minor defects and 3 have major defects.Varnika will buy a phone if it is good but the trader will only buy a mobile if it has no major defects. One phone is selected at random from the lot. What is the probability that it is
\begin{enumerate}
	\item acceptable to Varnika?
            \item acceptable to the trader?
\end{enumerate}
\solution
	%\input{exemplar/10/13/3/40/main.tex}
 \item A student says that if you throw a die, it will show up 1 or not 1. Therefore, the probability of getting 1 and the probability of getting 'not 1' each is equal to $\frac{1}{2}$. Is this correct? Give reasons.\\
 \solution
        %\input{exemplar/10/13/2/9/main.tex}
   \item Four candidates A, B, C, D have ap-
plied for the assignment to coach a school cricket
team. If A is twice as likely to be selected as B, and
B and C are given about the same chance of being
selected, while C is twice as likely to be selected
as D, what are the probabilities that
\begin{enumerate}
\item C will be selected?
\item A will not be selected?
\end{enumerate}
	%\input{exemplar/11/16/3/9/main.tex}
 \item A bag contain 24 balls of which $x$ balls are red, $2x$ are white and $3x$ are blue. A ball is selected at random, What is the probability that it is
\begin{enumerate}[label=\alph*)]
\item not red ?
\item white ?
\end{enumerate}
%\input{exemplar/10/13/3/41/main.tex}
If the letters of the word ASSASSINATION are arranged at random. Find the Probability that
\begin{enumerate}[label=(\alph*)]
\item Four $S's$ come consecutively in the word
\item Two  $I's$ and two $N's$ come together
\item All $A's$ are not coming together
\item No two $A's$ are coming together
\end{enumerate}
%\input{exemplar/11/16/3/14/main.tex}
	\item One urn contains two black balls (labelled B1 and B2) and one white ball. A
	second urn contains one black ball and two white balls (labelled W1 and W2).
	Suppose the following experiment is performed. One of the two urns is chosen
	at random. Next a ball is randomly chosen from the urn. Then a second ball is
	chosen at random from the same urn without replacing the first ball.
	
	\begin{enumerate}
	\item What is the probability that two black balls are chosen?
	
	\item What is the probability that two balls of opposite colour are chosen?
	\end{enumerate}
	\solution
	%\input{exemplar/11/16/3/12/main1.tex}
\end{enumerate}

	\item A card is selected from a pack of 52 cards.
 \begin{enumerate}[label=(\alph*)] 
                 \item How many points are there in the sample space?
                 \item Calculate the probability that the card is an ace of spades.
                 \item Calculate the probability that the card is (i) an ace and (ii) black card.
 \end{enumerate}
\solution
		%\begin{table}[H]
	\centering
\begin{tabular}{|c|c|c|}
\hline
Random variable &Value &Definition\\ \hline
\multirow{3}{*}{X} &0 &Slips of Rs 1\\
&1 &Slips of Rs 5\\
&2 &Slips of Rs 13\\ \hline
\multirow{2}{*}{Y} &0 &Box A\\
&1 &Box B\\\hline
\end{tabular}
\caption{}
\label{tab:Distribution}
\end{table}
See \tabref{tab:Distribution}.
\begin{align}
p_{Y}\brak{k}= \begin{cases} 
      \frac{1}{3} & {k=0} \\
      \frac{2}{3 }& {k=1} 
   \end{cases}
   \\
p_{Y|X}\brak{0|0} = \frac{19}{25}\, 
p_{Y|X}\brak{0|1} = \frac{6}{25}\,
p_{Y|X}\brak{1|0} = \frac{45}{50}\,
p_{Y|X}\brak{1|2} = \frac{5}{50}
\end{align}
The desired probability is the probability that a slip drawn at random is marked other than Rs 1,
\begin{align}
&=1-p_X\brak{0}\\
&= p_X(1) + p_X(2)
\end{align}
Using Bayes theorem,
\begin{align}
&= p_Y\brak{0} \times \pr{Y=0 | X=1} + p_Y\brak{1} \times \pr{Y=1|X=2}\\
&=\frac{1}{3} \times \frac{6}{25} + \frac{2}{3} \times \frac{5}{50}\\
&=\frac{11}{75}
\end{align}

\newpage

%\tableofcontents

\bigskip

\renewcommand{\thefigure}{\theenumi}
\renewcommand{\thetable}{\theenumi}
%\renewcommand{\theequation}{\theenumi}

%\begin{abstract}
%%\boldmath
%In this letter, an algorithm for evaluating the exact analytical bit error rate  (BER)  for the piecewise linear (PL) combiner for  multiple relays is presented. Previous results were available only for upto three relays. The algorithm is unique in the sense that  the actual mathematical expressions, that are prohibitively large, need not be explicitly obtained. The diversity gain due to multiple relays is shown through plots of the analytical BER, well supported by simulations. 
%
%\end{abstract}
% IEEEtran.cls defaults to using nonbold math in the Abstract.
% This preserves the distinction between vectors and scalars. However,
% if the journal you are submitting to favors bold math in the abstract,
% then you can use LaTeX's standard command \boldmath at the very start
% of the abstract to achieve this. Many IEEE journals frown on math
% in the abstract anyway.

% Note that keywords are not normally used for peerreview papers.
%\begin{IEEEkeywords}
%Cooperative diversity, decode and forward, piecewise linear
%\end{IEEEkeywords}



% For peer review papers, you can put extra information on the cover
% page as needed:
% \ifCLASSOPTIONpeerreview
% \begin{center} \bfseries EDICS Category: 3-BBND \end{center}
% \fi
%
% For peerreview papers, this IEEEtran command inserts a page break and
% creates the second title. It will be ignored for other modes.
%\IEEEpeerreviewmaketitle




\item Four cards are drawn from a well-shuffled deck of 52 cards. What is the probability of obtaining 3 diamonds and one spade.
\\
\solution
		%\begin{enumerate}[label=\thesection.\arabic*,ref=\thesection.\theenumi]
	\item One card is drawn from a well-shuffled deck of 52 cards. Find the probability of getting
\begin{enumerate}
\item A king of red colour 
\item A face card 
\item A red face card
\item The jack of hearts
\item A spade
\item The queen of diamonds

\end{enumerate}
\solution
		%\input{ncert/10/15/1/14/main.tex}
	\item Five cards—the ten, jack, queen, king and ace of diamonds, are well-shuffled with their face downwards. One card is then picked up at random.
\begin{enumerate}
\item
What is the probability that the card is the queen? 
\item
If the queen is drawn and put aside, what is the probability that the second card picked up is (a) an ace? (b) a queen?\\
\end{enumerate}
\solution
		%\input{ncert/10/15/1/15/defs.tex}
	\item A bag contains $5$ red balls and some blue balls. If the probability of drawing a blue ball is double that if a red ball, determine the number of blue balls in the bag. 
		\\
\solution
		%\input{ncert/10/15/2/3/defs.tex}
	\item A card is selected from a pack of 52 cards.
 \begin{enumerate}[label=(\alph*)] 
                 \item How many points are there in the sample space?
                 \item Calculate the probability that the card is an ace of spades.
                 \item Calculate the probability that the card is (i) an ace and (ii) black card.
 \end{enumerate}
\solution
		%\input{ncert/11/16/3/4/main.tex}
\item Four cards are drawn from a well-shuffled deck of 52 cards. What is the probability of obtaining 3 diamonds and one spade.
\\
\solution
		%\input{ncert/11/16/4/2/defs.tex}
\item In a certain lottery 10,000 tickets are sold and ten equal prizes are awarded. What is the probability of not getting a prize if you buy (a) one ticket (b) two tickets (c) 10 tickets ?	
\\
\solution
		%\input{ncert/11/16/4/4/defs.tex}
		%
\item 
Out of 100 students, two sections of 40 and 60 are formed. If you and your friend are among the 100 students, what is the probability that
\begin{enumerate}
\item you both enter the same section?
\item you both enter the different sections?
\end{enumerate}
\solution
		%\input{ncert/11/16/4/5/defs.tex}
	\item 
The number lock of a suitcase has 4 wheels each labelled with ten digits i.e. from 0 to 9.The lock opens with a sequence of four digits with no repeats.What is the probability of a person getting the right sequence to open the suitcase.
\\
\solution
		%\input{ncert/11/16/4/10/defs.tex}
		%
\item 
Two cards are drawn at random and without replacement from a pack of 52 playing cards. Find the probability that both the cards are black.
\\
\solution
		%\input{ncert/12/13/2/2/defs.tex}
		\item A box of oranges is inspected by examining three randomly selected oranges drawn without replacement. If all the three oranges are good, the box is approved for sale, otherwise, it is rejected. Find the probability that a box containing 15 oranges out of which 12 are good and 3 are bad ones will be approved for sale.
		\label{ncert/12/13/2/3/defs.tex}
		\item Two balls are drawn at random with replacement from a box containing 10 black and 8 red balls. Find the probability that
		\label{ncert/12/13/2/12}
\begin{enumerate}
\item both balls are red.
\item first ball is black and second is red.
\item one of them is black and other is red.
\end{enumerate}

\item In a hostel, 60\% of the students read Hindi newspaper, 40\% read English newspaper and 20\% read both Hindi and English newspapers. A student is selected at random.
		\label{ncert/12/13/2/15}
\begin{enumerate}
\item Find the probability that she reads neither Hindi nor English newspapers.
\item If she reads Hindi newspaper, find the probability that she reads English newspaper.
\item If she reads English newspaper, find the probability that she reads Hindi newspaper.\\
\end{enumerate}
\item The probability of obtaining an even prime number on each die, when a pair of dice is rolled is 
\begin{enumerate}
    \item $0$ 
    
    \item $\frac{1}{3}$ 
    
    \item $\frac{1}{12}$ 
    
    \item $\frac{1}{36}$ 
\end{enumerate}
\solution
		%\input{ncert/12/13/2/17/defs.tex}
	\item A bag contains 4 red and 4 black balls, another bag contains 2 red and 6 black balls. One of the two bags is selected at random and a ball is drawn from the bag which is found to be red. Find the probability that the ball is drawn from the first bag.
\\
\solution
		%\input{ncert/12/13/3/2/main.tex}
  \item
  Cards with numbers 2 to 101 are placed in a box. A card is selected at random.Find the probability that the card has
\begin{enumerate}[label=(\roman*)]
	\item an even number 
	\item a square number
\end{enumerate}
\solution
%\input{exemplar/10/13/3/32/main.tex}
\item
The king, queen and jack of clubs are removed from a deck of 52 playing cards and then well shuffled. Now one card is drawn at random from the remaining cards.  Determine the probability that the card is
\begin{enumerate}[label=(\roman*)]
\item a club
\item 10 of hearts
\end{enumerate}
\solution
%\input{exemplar/10/13/3/29/main.tex}
\item A team of medical students doing their internship have to assist during surgeries
at a city hospital. The probabilities of surgeries rated as very complex, complex,
routine, simple or very simple are respectively, 0.15, 0.20, 0.31, 0.26, .08. Find
the probabilities that a particular surgery will be rated
\begin{enumerate}
	\item complex or very complex;
	\item neither very complex nor very simple;
	\item routine or complex
	\item routine or simple
\end{enumerate}
\solution
%\input{exemplar/11/16/3/8(1)/main.tex}
\item A card is selected from a pack of 52 cards.
\begin{enumerate}[label=(\alph*)]
    \item How many points are there in the sample space?
    \item Calculate the probability that the card is an ace of spades.
    \item Calculate the probability that the card is (i) an ace and (ii) black card.
\end{enumerate}
\solution
%\input{exemplar/11/16/3/4/main2.tex}
\item The probability that a non leap year selected at random will contain 53 sundays.
\\
\solution
%\input{exemplar/10/13/1/19/main.tex}
\item One of the four persons John, Rita, Aslam or Gurpreet will be promoted next
month. Consequently the sample space consists of four elementary outcomes
S = {John promoted, Rita promoted, Aslam promoted, Gurpreet promoted}
You are told that the chances of John’s promotion is same as that of Gurpreet,
Rita’s chances of promotion are twice as likely as Johns. Aslam’s chances are
four times that of John.
\begin{enumerate}
	\item Determine
	\begin{enumerate}
		\item P (John promoted)
		\item P (Rita promoted)
		\item P (Aslam promoted)
		\item P (Gurpreet promoted)
	\end{enumerate}
	\item If A = {John promoted or Gurpreet promoted}, find P (A).
\end{enumerate}
\solution
%\input{exemplar/11/16/3/10/main.tex}
\item A card is drawn from a deck of 52 cards. Find the probability of getting a king or a heart or a red card.\\
\solution
%\input{exemplar/11/16/3/15/main.tex}
\item The probability that a student will pass his examination is 0.73, the probability of
the student getting a compartment is 0.13, and the probability that the student will
either pass or get compartment is 0.96. State True or False.\\
\solution
%\input{exemplar/11/16/3/31/main.tex}
\item A card is selected from a pack of 52 cards\\
\begin{enumerate}[label=(\alph*)]
\item How many points are there in the sample space?
\item Calculate the probability that the cards is an ace of spades.
\item Calculate the probability that the card is (i) an ace (ii)black card.\\
\end{enumerate}
%\input{ncert/11/16/3/4_1/Prob_4.tex}
\item In a non-leap year, the probability of having 53 tuesdays or 53 wednesdays is\\
\solution
%\input{exemplar/11/16/3/18/main.tex}
\item There are 1000 sealed envelopes in a box, 10 of them contain a cash prize of
Rs 100 each, 100 of them contain a cash prize of Rs 50 each and 200 of them
contain a cash prize of Rs 10 each and rest do not contain any cash prize. If they
are well shuffled and an envelope is picked up out, what is the probability that it
contains no cash prize?\\
\solution
%\input{exemplar/10/13/3/34/main.tex}
\item 
A die is thrown and a card is selected at random from a deck of 52 playing cards. The probability of getting an even number on the die and a spade card.\\
\solution
%\input{exemplar/12/13/3/78/main.tex}
\item
If 4-digit numbers greater than 5,000 are randomly formed from the digits 0, 1, 3, 5, and 7, what is the probability of forming a number divisible by 5 when:
\begin{enumerate}
    \item The digits are repeated?
    \item The repetition of digits is not allowed?
\end{enumerate}
\solution
%\input{ncert/11/16/4/9/main.tex}
\item Consider the probability space $\brak{\Omega, \mathcal{G}, P}$ where $\Omega = [0,2]$ and $\mathcal{G} = \cbrak{\phi, \Omega, [0,1], (1,2]}$. Let $X$ and $Y$ be two functions on $\Omega$ defined as
\begin{align*}
    X(\omega) = 
    \begin{cases}
        1 & \text{if }\omega \in [0, 1]\\
        2 & \text{if }\omega \in (1, 2]
    \end{cases}
\end{align*}
and
\begin{align*}
    Y(\omega) = 
    \begin{cases}
        2 & \text{if }\omega \in [0, 1.5]\\
        3 & \text{if }\omega \in (1.5, 2].
    \end{cases}
\end{align*}
Then which one of the following statements is true?
\begin{enumerate}
    \item [(A)] $X$ is a random variable with respect to $\mathcal{G}$, but $Y$ is not a random variable with respect to $\mathcal{G}$.
    \item [(B)] $Y$ is a random variable with respect to $\mathcal{G}$, but $X$ is not a random variable with respect to $\mathcal{G}$.
    \item [(C)] Neither $X$ nor $Y$ is a random variable with respect to $\mathcal{G}$.
    \item [(D)] Both $X$ and $Y$ are random variables with respect to $\mathcal{G}$.
\end{enumerate} \hfill (GATE ST 2023)\\
\solution
%\input{gate/ST/2023/14/main.tex}
	\item  A die is loaded in such a way that each odd number is twice as likely to occur as
each even number. Find $P(G)$, where $G$ is the event that a number greater than
3 occurs on a single roll of the die.
\\
\solution
		%\input{exemplar/11/16/3/5/main.tex}
	\item All the jacks, queens and kings are removed from a deck of 52 playing cards. The remaining cards are well shuffled and then one card is drawn at random. Giving ace a value 1 similar value for other cards, find the probability that the card has a value 
		\begin{enumerate}
			\item 7
			\item greater than 7
			\item less than 7
		\end{enumerate}
		%\input{exemplar/10/13/3/30/main.tex}
  \item A Lot consists of 48 mobile phones of which 42 are good, 3 have only minor defects and 3 have major defects.Varnika will buy a phone if it is good but the trader will only buy a mobile if it has no major defects. One phone is selected at random from the lot. What is the probability that it is
\begin{enumerate}
	\item acceptable to Varnika?
            \item acceptable to the trader?
\end{enumerate}
\solution
	%\input{exemplar/10/13/3/40/main.tex}
 \item A student says that if you throw a die, it will show up 1 or not 1. Therefore, the probability of getting 1 and the probability of getting 'not 1' each is equal to $\frac{1}{2}$. Is this correct? Give reasons.\\
 \solution
        %\input{exemplar/10/13/2/9/main.tex}
   \item Four candidates A, B, C, D have ap-
plied for the assignment to coach a school cricket
team. If A is twice as likely to be selected as B, and
B and C are given about the same chance of being
selected, while C is twice as likely to be selected
as D, what are the probabilities that
\begin{enumerate}
\item C will be selected?
\item A will not be selected?
\end{enumerate}
	%\input{exemplar/11/16/3/9/main.tex}
 \item A bag contain 24 balls of which $x$ balls are red, $2x$ are white and $3x$ are blue. A ball is selected at random, What is the probability that it is
\begin{enumerate}[label=\alph*)]
\item not red ?
\item white ?
\end{enumerate}
%\input{exemplar/10/13/3/41/main.tex}
If the letters of the word ASSASSINATION are arranged at random. Find the Probability that
\begin{enumerate}[label=(\alph*)]
\item Four $S's$ come consecutively in the word
\item Two  $I's$ and two $N's$ come together
\item All $A's$ are not coming together
\item No two $A's$ are coming together
\end{enumerate}
%\input{exemplar/11/16/3/14/main.tex}
	\item One urn contains two black balls (labelled B1 and B2) and one white ball. A
	second urn contains one black ball and two white balls (labelled W1 and W2).
	Suppose the following experiment is performed. One of the two urns is chosen
	at random. Next a ball is randomly chosen from the urn. Then a second ball is
	chosen at random from the same urn without replacing the first ball.
	
	\begin{enumerate}
	\item What is the probability that two black balls are chosen?
	
	\item What is the probability that two balls of opposite colour are chosen?
	\end{enumerate}
	\solution
	%\input{exemplar/11/16/3/12/main1.tex}
\end{enumerate}

\item In a certain lottery 10,000 tickets are sold and ten equal prizes are awarded. What is the probability of not getting a prize if you buy (a) one ticket (b) two tickets (c) 10 tickets ?	
\\
\solution
		%\begin{enumerate}[label=\thesection.\arabic*,ref=\thesection.\theenumi]
	\item One card is drawn from a well-shuffled deck of 52 cards. Find the probability of getting
\begin{enumerate}
\item A king of red colour 
\item A face card 
\item A red face card
\item The jack of hearts
\item A spade
\item The queen of diamonds

\end{enumerate}
\solution
		%\input{ncert/10/15/1/14/main.tex}
	\item Five cards—the ten, jack, queen, king and ace of diamonds, are well-shuffled with their face downwards. One card is then picked up at random.
\begin{enumerate}
\item
What is the probability that the card is the queen? 
\item
If the queen is drawn and put aside, what is the probability that the second card picked up is (a) an ace? (b) a queen?\\
\end{enumerate}
\solution
		%\input{ncert/10/15/1/15/defs.tex}
	\item A bag contains $5$ red balls and some blue balls. If the probability of drawing a blue ball is double that if a red ball, determine the number of blue balls in the bag. 
		\\
\solution
		%\input{ncert/10/15/2/3/defs.tex}
	\item A card is selected from a pack of 52 cards.
 \begin{enumerate}[label=(\alph*)] 
                 \item How many points are there in the sample space?
                 \item Calculate the probability that the card is an ace of spades.
                 \item Calculate the probability that the card is (i) an ace and (ii) black card.
 \end{enumerate}
\solution
		%\input{ncert/11/16/3/4/main.tex}
\item Four cards are drawn from a well-shuffled deck of 52 cards. What is the probability of obtaining 3 diamonds and one spade.
\\
\solution
		%\input{ncert/11/16/4/2/defs.tex}
\item In a certain lottery 10,000 tickets are sold and ten equal prizes are awarded. What is the probability of not getting a prize if you buy (a) one ticket (b) two tickets (c) 10 tickets ?	
\\
\solution
		%\input{ncert/11/16/4/4/defs.tex}
		%
\item 
Out of 100 students, two sections of 40 and 60 are formed. If you and your friend are among the 100 students, what is the probability that
\begin{enumerate}
\item you both enter the same section?
\item you both enter the different sections?
\end{enumerate}
\solution
		%\input{ncert/11/16/4/5/defs.tex}
	\item 
The number lock of a suitcase has 4 wheels each labelled with ten digits i.e. from 0 to 9.The lock opens with a sequence of four digits with no repeats.What is the probability of a person getting the right sequence to open the suitcase.
\\
\solution
		%\input{ncert/11/16/4/10/defs.tex}
		%
\item 
Two cards are drawn at random and without replacement from a pack of 52 playing cards. Find the probability that both the cards are black.
\\
\solution
		%\input{ncert/12/13/2/2/defs.tex}
		\item A box of oranges is inspected by examining three randomly selected oranges drawn without replacement. If all the three oranges are good, the box is approved for sale, otherwise, it is rejected. Find the probability that a box containing 15 oranges out of which 12 are good and 3 are bad ones will be approved for sale.
		\label{ncert/12/13/2/3/defs.tex}
		\item Two balls are drawn at random with replacement from a box containing 10 black and 8 red balls. Find the probability that
		\label{ncert/12/13/2/12}
\begin{enumerate}
\item both balls are red.
\item first ball is black and second is red.
\item one of them is black and other is red.
\end{enumerate}

\item In a hostel, 60\% of the students read Hindi newspaper, 40\% read English newspaper and 20\% read both Hindi and English newspapers. A student is selected at random.
		\label{ncert/12/13/2/15}
\begin{enumerate}
\item Find the probability that she reads neither Hindi nor English newspapers.
\item If she reads Hindi newspaper, find the probability that she reads English newspaper.
\item If she reads English newspaper, find the probability that she reads Hindi newspaper.\\
\end{enumerate}
\item The probability of obtaining an even prime number on each die, when a pair of dice is rolled is 
\begin{enumerate}
    \item $0$ 
    
    \item $\frac{1}{3}$ 
    
    \item $\frac{1}{12}$ 
    
    \item $\frac{1}{36}$ 
\end{enumerate}
\solution
		%\input{ncert/12/13/2/17/defs.tex}
	\item A bag contains 4 red and 4 black balls, another bag contains 2 red and 6 black balls. One of the two bags is selected at random and a ball is drawn from the bag which is found to be red. Find the probability that the ball is drawn from the first bag.
\\
\solution
		%\input{ncert/12/13/3/2/main.tex}
  \item
  Cards with numbers 2 to 101 are placed in a box. A card is selected at random.Find the probability that the card has
\begin{enumerate}[label=(\roman*)]
	\item an even number 
	\item a square number
\end{enumerate}
\solution
%\input{exemplar/10/13/3/32/main.tex}
\item
The king, queen and jack of clubs are removed from a deck of 52 playing cards and then well shuffled. Now one card is drawn at random from the remaining cards.  Determine the probability that the card is
\begin{enumerate}[label=(\roman*)]
\item a club
\item 10 of hearts
\end{enumerate}
\solution
%\input{exemplar/10/13/3/29/main.tex}
\item A team of medical students doing their internship have to assist during surgeries
at a city hospital. The probabilities of surgeries rated as very complex, complex,
routine, simple or very simple are respectively, 0.15, 0.20, 0.31, 0.26, .08. Find
the probabilities that a particular surgery will be rated
\begin{enumerate}
	\item complex or very complex;
	\item neither very complex nor very simple;
	\item routine or complex
	\item routine or simple
\end{enumerate}
\solution
%\input{exemplar/11/16/3/8(1)/main.tex}
\item A card is selected from a pack of 52 cards.
\begin{enumerate}[label=(\alph*)]
    \item How many points are there in the sample space?
    \item Calculate the probability that the card is an ace of spades.
    \item Calculate the probability that the card is (i) an ace and (ii) black card.
\end{enumerate}
\solution
%\input{exemplar/11/16/3/4/main2.tex}
\item The probability that a non leap year selected at random will contain 53 sundays.
\\
\solution
%\input{exemplar/10/13/1/19/main.tex}
\item One of the four persons John, Rita, Aslam or Gurpreet will be promoted next
month. Consequently the sample space consists of four elementary outcomes
S = {John promoted, Rita promoted, Aslam promoted, Gurpreet promoted}
You are told that the chances of John’s promotion is same as that of Gurpreet,
Rita’s chances of promotion are twice as likely as Johns. Aslam’s chances are
four times that of John.
\begin{enumerate}
	\item Determine
	\begin{enumerate}
		\item P (John promoted)
		\item P (Rita promoted)
		\item P (Aslam promoted)
		\item P (Gurpreet promoted)
	\end{enumerate}
	\item If A = {John promoted or Gurpreet promoted}, find P (A).
\end{enumerate}
\solution
%\input{exemplar/11/16/3/10/main.tex}
\item A card is drawn from a deck of 52 cards. Find the probability of getting a king or a heart or a red card.\\
\solution
%\input{exemplar/11/16/3/15/main.tex}
\item The probability that a student will pass his examination is 0.73, the probability of
the student getting a compartment is 0.13, and the probability that the student will
either pass or get compartment is 0.96. State True or False.\\
\solution
%\input{exemplar/11/16/3/31/main.tex}
\item A card is selected from a pack of 52 cards\\
\begin{enumerate}[label=(\alph*)]
\item How many points are there in the sample space?
\item Calculate the probability that the cards is an ace of spades.
\item Calculate the probability that the card is (i) an ace (ii)black card.\\
\end{enumerate}
%\input{ncert/11/16/3/4_1/Prob_4.tex}
\item In a non-leap year, the probability of having 53 tuesdays or 53 wednesdays is\\
\solution
%\input{exemplar/11/16/3/18/main.tex}
\item There are 1000 sealed envelopes in a box, 10 of them contain a cash prize of
Rs 100 each, 100 of them contain a cash prize of Rs 50 each and 200 of them
contain a cash prize of Rs 10 each and rest do not contain any cash prize. If they
are well shuffled and an envelope is picked up out, what is the probability that it
contains no cash prize?\\
\solution
%\input{exemplar/10/13/3/34/main.tex}
\item 
A die is thrown and a card is selected at random from a deck of 52 playing cards. The probability of getting an even number on the die and a spade card.\\
\solution
%\input{exemplar/12/13/3/78/main.tex}
\item
If 4-digit numbers greater than 5,000 are randomly formed from the digits 0, 1, 3, 5, and 7, what is the probability of forming a number divisible by 5 when:
\begin{enumerate}
    \item The digits are repeated?
    \item The repetition of digits is not allowed?
\end{enumerate}
\solution
%\input{ncert/11/16/4/9/main.tex}
\item Consider the probability space $\brak{\Omega, \mathcal{G}, P}$ where $\Omega = [0,2]$ and $\mathcal{G} = \cbrak{\phi, \Omega, [0,1], (1,2]}$. Let $X$ and $Y$ be two functions on $\Omega$ defined as
\begin{align*}
    X(\omega) = 
    \begin{cases}
        1 & \text{if }\omega \in [0, 1]\\
        2 & \text{if }\omega \in (1, 2]
    \end{cases}
\end{align*}
and
\begin{align*}
    Y(\omega) = 
    \begin{cases}
        2 & \text{if }\omega \in [0, 1.5]\\
        3 & \text{if }\omega \in (1.5, 2].
    \end{cases}
\end{align*}
Then which one of the following statements is true?
\begin{enumerate}
    \item [(A)] $X$ is a random variable with respect to $\mathcal{G}$, but $Y$ is not a random variable with respect to $\mathcal{G}$.
    \item [(B)] $Y$ is a random variable with respect to $\mathcal{G}$, but $X$ is not a random variable with respect to $\mathcal{G}$.
    \item [(C)] Neither $X$ nor $Y$ is a random variable with respect to $\mathcal{G}$.
    \item [(D)] Both $X$ and $Y$ are random variables with respect to $\mathcal{G}$.
\end{enumerate} \hfill (GATE ST 2023)\\
\solution
%\input{gate/ST/2023/14/main.tex}
	\item  A die is loaded in such a way that each odd number is twice as likely to occur as
each even number. Find $P(G)$, where $G$ is the event that a number greater than
3 occurs on a single roll of the die.
\\
\solution
		%\input{exemplar/11/16/3/5/main.tex}
	\item All the jacks, queens and kings are removed from a deck of 52 playing cards. The remaining cards are well shuffled and then one card is drawn at random. Giving ace a value 1 similar value for other cards, find the probability that the card has a value 
		\begin{enumerate}
			\item 7
			\item greater than 7
			\item less than 7
		\end{enumerate}
		%\input{exemplar/10/13/3/30/main.tex}
  \item A Lot consists of 48 mobile phones of which 42 are good, 3 have only minor defects and 3 have major defects.Varnika will buy a phone if it is good but the trader will only buy a mobile if it has no major defects. One phone is selected at random from the lot. What is the probability that it is
\begin{enumerate}
	\item acceptable to Varnika?
            \item acceptable to the trader?
\end{enumerate}
\solution
	%\input{exemplar/10/13/3/40/main.tex}
 \item A student says that if you throw a die, it will show up 1 or not 1. Therefore, the probability of getting 1 and the probability of getting 'not 1' each is equal to $\frac{1}{2}$. Is this correct? Give reasons.\\
 \solution
        %\input{exemplar/10/13/2/9/main.tex}
   \item Four candidates A, B, C, D have ap-
plied for the assignment to coach a school cricket
team. If A is twice as likely to be selected as B, and
B and C are given about the same chance of being
selected, while C is twice as likely to be selected
as D, what are the probabilities that
\begin{enumerate}
\item C will be selected?
\item A will not be selected?
\end{enumerate}
	%\input{exemplar/11/16/3/9/main.tex}
 \item A bag contain 24 balls of which $x$ balls are red, $2x$ are white and $3x$ are blue. A ball is selected at random, What is the probability that it is
\begin{enumerate}[label=\alph*)]
\item not red ?
\item white ?
\end{enumerate}
%\input{exemplar/10/13/3/41/main.tex}
If the letters of the word ASSASSINATION are arranged at random. Find the Probability that
\begin{enumerate}[label=(\alph*)]
\item Four $S's$ come consecutively in the word
\item Two  $I's$ and two $N's$ come together
\item All $A's$ are not coming together
\item No two $A's$ are coming together
\end{enumerate}
%\input{exemplar/11/16/3/14/main.tex}
	\item One urn contains two black balls (labelled B1 and B2) and one white ball. A
	second urn contains one black ball and two white balls (labelled W1 and W2).
	Suppose the following experiment is performed. One of the two urns is chosen
	at random. Next a ball is randomly chosen from the urn. Then a second ball is
	chosen at random from the same urn without replacing the first ball.
	
	\begin{enumerate}
	\item What is the probability that two black balls are chosen?
	
	\item What is the probability that two balls of opposite colour are chosen?
	\end{enumerate}
	\solution
	%\input{exemplar/11/16/3/12/main1.tex}
\end{enumerate}

		%
\item 
Out of 100 students, two sections of 40 and 60 are formed. If you and your friend are among the 100 students, what is the probability that
\begin{enumerate}
\item you both enter the same section?
\item you both enter the different sections?
\end{enumerate}
\solution
		%\begin{enumerate}[label=\thesection.\arabic*,ref=\thesection.\theenumi]
	\item One card is drawn from a well-shuffled deck of 52 cards. Find the probability of getting
\begin{enumerate}
\item A king of red colour 
\item A face card 
\item A red face card
\item The jack of hearts
\item A spade
\item The queen of diamonds

\end{enumerate}
\solution
		%\input{ncert/10/15/1/14/main.tex}
	\item Five cards—the ten, jack, queen, king and ace of diamonds, are well-shuffled with their face downwards. One card is then picked up at random.
\begin{enumerate}
\item
What is the probability that the card is the queen? 
\item
If the queen is drawn and put aside, what is the probability that the second card picked up is (a) an ace? (b) a queen?\\
\end{enumerate}
\solution
		%\input{ncert/10/15/1/15/defs.tex}
	\item A bag contains $5$ red balls and some blue balls. If the probability of drawing a blue ball is double that if a red ball, determine the number of blue balls in the bag. 
		\\
\solution
		%\input{ncert/10/15/2/3/defs.tex}
	\item A card is selected from a pack of 52 cards.
 \begin{enumerate}[label=(\alph*)] 
                 \item How many points are there in the sample space?
                 \item Calculate the probability that the card is an ace of spades.
                 \item Calculate the probability that the card is (i) an ace and (ii) black card.
 \end{enumerate}
\solution
		%\input{ncert/11/16/3/4/main.tex}
\item Four cards are drawn from a well-shuffled deck of 52 cards. What is the probability of obtaining 3 diamonds and one spade.
\\
\solution
		%\input{ncert/11/16/4/2/defs.tex}
\item In a certain lottery 10,000 tickets are sold and ten equal prizes are awarded. What is the probability of not getting a prize if you buy (a) one ticket (b) two tickets (c) 10 tickets ?	
\\
\solution
		%\input{ncert/11/16/4/4/defs.tex}
		%
\item 
Out of 100 students, two sections of 40 and 60 are formed. If you and your friend are among the 100 students, what is the probability that
\begin{enumerate}
\item you both enter the same section?
\item you both enter the different sections?
\end{enumerate}
\solution
		%\input{ncert/11/16/4/5/defs.tex}
	\item 
The number lock of a suitcase has 4 wheels each labelled with ten digits i.e. from 0 to 9.The lock opens with a sequence of four digits with no repeats.What is the probability of a person getting the right sequence to open the suitcase.
\\
\solution
		%\input{ncert/11/16/4/10/defs.tex}
		%
\item 
Two cards are drawn at random and without replacement from a pack of 52 playing cards. Find the probability that both the cards are black.
\\
\solution
		%\input{ncert/12/13/2/2/defs.tex}
		\item A box of oranges is inspected by examining three randomly selected oranges drawn without replacement. If all the three oranges are good, the box is approved for sale, otherwise, it is rejected. Find the probability that a box containing 15 oranges out of which 12 are good and 3 are bad ones will be approved for sale.
		\label{ncert/12/13/2/3/defs.tex}
		\item Two balls are drawn at random with replacement from a box containing 10 black and 8 red balls. Find the probability that
		\label{ncert/12/13/2/12}
\begin{enumerate}
\item both balls are red.
\item first ball is black and second is red.
\item one of them is black and other is red.
\end{enumerate}

\item In a hostel, 60\% of the students read Hindi newspaper, 40\% read English newspaper and 20\% read both Hindi and English newspapers. A student is selected at random.
		\label{ncert/12/13/2/15}
\begin{enumerate}
\item Find the probability that she reads neither Hindi nor English newspapers.
\item If she reads Hindi newspaper, find the probability that she reads English newspaper.
\item If she reads English newspaper, find the probability that she reads Hindi newspaper.\\
\end{enumerate}
\item The probability of obtaining an even prime number on each die, when a pair of dice is rolled is 
\begin{enumerate}
    \item $0$ 
    
    \item $\frac{1}{3}$ 
    
    \item $\frac{1}{12}$ 
    
    \item $\frac{1}{36}$ 
\end{enumerate}
\solution
		%\input{ncert/12/13/2/17/defs.tex}
	\item A bag contains 4 red and 4 black balls, another bag contains 2 red and 6 black balls. One of the two bags is selected at random and a ball is drawn from the bag which is found to be red. Find the probability that the ball is drawn from the first bag.
\\
\solution
		%\input{ncert/12/13/3/2/main.tex}
  \item
  Cards with numbers 2 to 101 are placed in a box. A card is selected at random.Find the probability that the card has
\begin{enumerate}[label=(\roman*)]
	\item an even number 
	\item a square number
\end{enumerate}
\solution
%\input{exemplar/10/13/3/32/main.tex}
\item
The king, queen and jack of clubs are removed from a deck of 52 playing cards and then well shuffled. Now one card is drawn at random from the remaining cards.  Determine the probability that the card is
\begin{enumerate}[label=(\roman*)]
\item a club
\item 10 of hearts
\end{enumerate}
\solution
%\input{exemplar/10/13/3/29/main.tex}
\item A team of medical students doing their internship have to assist during surgeries
at a city hospital. The probabilities of surgeries rated as very complex, complex,
routine, simple or very simple are respectively, 0.15, 0.20, 0.31, 0.26, .08. Find
the probabilities that a particular surgery will be rated
\begin{enumerate}
	\item complex or very complex;
	\item neither very complex nor very simple;
	\item routine or complex
	\item routine or simple
\end{enumerate}
\solution
%\input{exemplar/11/16/3/8(1)/main.tex}
\item A card is selected from a pack of 52 cards.
\begin{enumerate}[label=(\alph*)]
    \item How many points are there in the sample space?
    \item Calculate the probability that the card is an ace of spades.
    \item Calculate the probability that the card is (i) an ace and (ii) black card.
\end{enumerate}
\solution
%\input{exemplar/11/16/3/4/main2.tex}
\item The probability that a non leap year selected at random will contain 53 sundays.
\\
\solution
%\input{exemplar/10/13/1/19/main.tex}
\item One of the four persons John, Rita, Aslam or Gurpreet will be promoted next
month. Consequently the sample space consists of four elementary outcomes
S = {John promoted, Rita promoted, Aslam promoted, Gurpreet promoted}
You are told that the chances of John’s promotion is same as that of Gurpreet,
Rita’s chances of promotion are twice as likely as Johns. Aslam’s chances are
four times that of John.
\begin{enumerate}
	\item Determine
	\begin{enumerate}
		\item P (John promoted)
		\item P (Rita promoted)
		\item P (Aslam promoted)
		\item P (Gurpreet promoted)
	\end{enumerate}
	\item If A = {John promoted or Gurpreet promoted}, find P (A).
\end{enumerate}
\solution
%\input{exemplar/11/16/3/10/main.tex}
\item A card is drawn from a deck of 52 cards. Find the probability of getting a king or a heart or a red card.\\
\solution
%\input{exemplar/11/16/3/15/main.tex}
\item The probability that a student will pass his examination is 0.73, the probability of
the student getting a compartment is 0.13, and the probability that the student will
either pass or get compartment is 0.96. State True or False.\\
\solution
%\input{exemplar/11/16/3/31/main.tex}
\item A card is selected from a pack of 52 cards\\
\begin{enumerate}[label=(\alph*)]
\item How many points are there in the sample space?
\item Calculate the probability that the cards is an ace of spades.
\item Calculate the probability that the card is (i) an ace (ii)black card.\\
\end{enumerate}
%\input{ncert/11/16/3/4_1/Prob_4.tex}
\item In a non-leap year, the probability of having 53 tuesdays or 53 wednesdays is\\
\solution
%\input{exemplar/11/16/3/18/main.tex}
\item There are 1000 sealed envelopes in a box, 10 of them contain a cash prize of
Rs 100 each, 100 of them contain a cash prize of Rs 50 each and 200 of them
contain a cash prize of Rs 10 each and rest do not contain any cash prize. If they
are well shuffled and an envelope is picked up out, what is the probability that it
contains no cash prize?\\
\solution
%\input{exemplar/10/13/3/34/main.tex}
\item 
A die is thrown and a card is selected at random from a deck of 52 playing cards. The probability of getting an even number on the die and a spade card.\\
\solution
%\input{exemplar/12/13/3/78/main.tex}
\item
If 4-digit numbers greater than 5,000 are randomly formed from the digits 0, 1, 3, 5, and 7, what is the probability of forming a number divisible by 5 when:
\begin{enumerate}
    \item The digits are repeated?
    \item The repetition of digits is not allowed?
\end{enumerate}
\solution
%\input{ncert/11/16/4/9/main.tex}
\item Consider the probability space $\brak{\Omega, \mathcal{G}, P}$ where $\Omega = [0,2]$ and $\mathcal{G} = \cbrak{\phi, \Omega, [0,1], (1,2]}$. Let $X$ and $Y$ be two functions on $\Omega$ defined as
\begin{align*}
    X(\omega) = 
    \begin{cases}
        1 & \text{if }\omega \in [0, 1]\\
        2 & \text{if }\omega \in (1, 2]
    \end{cases}
\end{align*}
and
\begin{align*}
    Y(\omega) = 
    \begin{cases}
        2 & \text{if }\omega \in [0, 1.5]\\
        3 & \text{if }\omega \in (1.5, 2].
    \end{cases}
\end{align*}
Then which one of the following statements is true?
\begin{enumerate}
    \item [(A)] $X$ is a random variable with respect to $\mathcal{G}$, but $Y$ is not a random variable with respect to $\mathcal{G}$.
    \item [(B)] $Y$ is a random variable with respect to $\mathcal{G}$, but $X$ is not a random variable with respect to $\mathcal{G}$.
    \item [(C)] Neither $X$ nor $Y$ is a random variable with respect to $\mathcal{G}$.
    \item [(D)] Both $X$ and $Y$ are random variables with respect to $\mathcal{G}$.
\end{enumerate} \hfill (GATE ST 2023)\\
\solution
%\input{gate/ST/2023/14/main.tex}
	\item  A die is loaded in such a way that each odd number is twice as likely to occur as
each even number. Find $P(G)$, where $G$ is the event that a number greater than
3 occurs on a single roll of the die.
\\
\solution
		%\input{exemplar/11/16/3/5/main.tex}
	\item All the jacks, queens and kings are removed from a deck of 52 playing cards. The remaining cards are well shuffled and then one card is drawn at random. Giving ace a value 1 similar value for other cards, find the probability that the card has a value 
		\begin{enumerate}
			\item 7
			\item greater than 7
			\item less than 7
		\end{enumerate}
		%\input{exemplar/10/13/3/30/main.tex}
  \item A Lot consists of 48 mobile phones of which 42 are good, 3 have only minor defects and 3 have major defects.Varnika will buy a phone if it is good but the trader will only buy a mobile if it has no major defects. One phone is selected at random from the lot. What is the probability that it is
\begin{enumerate}
	\item acceptable to Varnika?
            \item acceptable to the trader?
\end{enumerate}
\solution
	%\input{exemplar/10/13/3/40/main.tex}
 \item A student says that if you throw a die, it will show up 1 or not 1. Therefore, the probability of getting 1 and the probability of getting 'not 1' each is equal to $\frac{1}{2}$. Is this correct? Give reasons.\\
 \solution
        %\input{exemplar/10/13/2/9/main.tex}
   \item Four candidates A, B, C, D have ap-
plied for the assignment to coach a school cricket
team. If A is twice as likely to be selected as B, and
B and C are given about the same chance of being
selected, while C is twice as likely to be selected
as D, what are the probabilities that
\begin{enumerate}
\item C will be selected?
\item A will not be selected?
\end{enumerate}
	%\input{exemplar/11/16/3/9/main.tex}
 \item A bag contain 24 balls of which $x$ balls are red, $2x$ are white and $3x$ are blue. A ball is selected at random, What is the probability that it is
\begin{enumerate}[label=\alph*)]
\item not red ?
\item white ?
\end{enumerate}
%\input{exemplar/10/13/3/41/main.tex}
If the letters of the word ASSASSINATION are arranged at random. Find the Probability that
\begin{enumerate}[label=(\alph*)]
\item Four $S's$ come consecutively in the word
\item Two  $I's$ and two $N's$ come together
\item All $A's$ are not coming together
\item No two $A's$ are coming together
\end{enumerate}
%\input{exemplar/11/16/3/14/main.tex}
	\item One urn contains two black balls (labelled B1 and B2) and one white ball. A
	second urn contains one black ball and two white balls (labelled W1 and W2).
	Suppose the following experiment is performed. One of the two urns is chosen
	at random. Next a ball is randomly chosen from the urn. Then a second ball is
	chosen at random from the same urn without replacing the first ball.
	
	\begin{enumerate}
	\item What is the probability that two black balls are chosen?
	
	\item What is the probability that two balls of opposite colour are chosen?
	\end{enumerate}
	\solution
	%\input{exemplar/11/16/3/12/main1.tex}
\end{enumerate}

	\item 
The number lock of a suitcase has 4 wheels each labelled with ten digits i.e. from 0 to 9.The lock opens with a sequence of four digits with no repeats.What is the probability of a person getting the right sequence to open the suitcase.
\\
\solution
		%\begin{enumerate}[label=\thesection.\arabic*,ref=\thesection.\theenumi]
	\item One card is drawn from a well-shuffled deck of 52 cards. Find the probability of getting
\begin{enumerate}
\item A king of red colour 
\item A face card 
\item A red face card
\item The jack of hearts
\item A spade
\item The queen of diamonds

\end{enumerate}
\solution
		%\input{ncert/10/15/1/14/main.tex}
	\item Five cards—the ten, jack, queen, king and ace of diamonds, are well-shuffled with their face downwards. One card is then picked up at random.
\begin{enumerate}
\item
What is the probability that the card is the queen? 
\item
If the queen is drawn and put aside, what is the probability that the second card picked up is (a) an ace? (b) a queen?\\
\end{enumerate}
\solution
		%\input{ncert/10/15/1/15/defs.tex}
	\item A bag contains $5$ red balls and some blue balls. If the probability of drawing a blue ball is double that if a red ball, determine the number of blue balls in the bag. 
		\\
\solution
		%\input{ncert/10/15/2/3/defs.tex}
	\item A card is selected from a pack of 52 cards.
 \begin{enumerate}[label=(\alph*)] 
                 \item How many points are there in the sample space?
                 \item Calculate the probability that the card is an ace of spades.
                 \item Calculate the probability that the card is (i) an ace and (ii) black card.
 \end{enumerate}
\solution
		%\input{ncert/11/16/3/4/main.tex}
\item Four cards are drawn from a well-shuffled deck of 52 cards. What is the probability of obtaining 3 diamonds and one spade.
\\
\solution
		%\input{ncert/11/16/4/2/defs.tex}
\item In a certain lottery 10,000 tickets are sold and ten equal prizes are awarded. What is the probability of not getting a prize if you buy (a) one ticket (b) two tickets (c) 10 tickets ?	
\\
\solution
		%\input{ncert/11/16/4/4/defs.tex}
		%
\item 
Out of 100 students, two sections of 40 and 60 are formed. If you and your friend are among the 100 students, what is the probability that
\begin{enumerate}
\item you both enter the same section?
\item you both enter the different sections?
\end{enumerate}
\solution
		%\input{ncert/11/16/4/5/defs.tex}
	\item 
The number lock of a suitcase has 4 wheels each labelled with ten digits i.e. from 0 to 9.The lock opens with a sequence of four digits with no repeats.What is the probability of a person getting the right sequence to open the suitcase.
\\
\solution
		%\input{ncert/11/16/4/10/defs.tex}
		%
\item 
Two cards are drawn at random and without replacement from a pack of 52 playing cards. Find the probability that both the cards are black.
\\
\solution
		%\input{ncert/12/13/2/2/defs.tex}
		\item A box of oranges is inspected by examining three randomly selected oranges drawn without replacement. If all the three oranges are good, the box is approved for sale, otherwise, it is rejected. Find the probability that a box containing 15 oranges out of which 12 are good and 3 are bad ones will be approved for sale.
		\label{ncert/12/13/2/3/defs.tex}
		\item Two balls are drawn at random with replacement from a box containing 10 black and 8 red balls. Find the probability that
		\label{ncert/12/13/2/12}
\begin{enumerate}
\item both balls are red.
\item first ball is black and second is red.
\item one of them is black and other is red.
\end{enumerate}

\item In a hostel, 60\% of the students read Hindi newspaper, 40\% read English newspaper and 20\% read both Hindi and English newspapers. A student is selected at random.
		\label{ncert/12/13/2/15}
\begin{enumerate}
\item Find the probability that she reads neither Hindi nor English newspapers.
\item If she reads Hindi newspaper, find the probability that she reads English newspaper.
\item If she reads English newspaper, find the probability that she reads Hindi newspaper.\\
\end{enumerate}
\item The probability of obtaining an even prime number on each die, when a pair of dice is rolled is 
\begin{enumerate}
    \item $0$ 
    
    \item $\frac{1}{3}$ 
    
    \item $\frac{1}{12}$ 
    
    \item $\frac{1}{36}$ 
\end{enumerate}
\solution
		%\input{ncert/12/13/2/17/defs.tex}
	\item A bag contains 4 red and 4 black balls, another bag contains 2 red and 6 black balls. One of the two bags is selected at random and a ball is drawn from the bag which is found to be red. Find the probability that the ball is drawn from the first bag.
\\
\solution
		%\input{ncert/12/13/3/2/main.tex}
  \item
  Cards with numbers 2 to 101 are placed in a box. A card is selected at random.Find the probability that the card has
\begin{enumerate}[label=(\roman*)]
	\item an even number 
	\item a square number
\end{enumerate}
\solution
%\input{exemplar/10/13/3/32/main.tex}
\item
The king, queen and jack of clubs are removed from a deck of 52 playing cards and then well shuffled. Now one card is drawn at random from the remaining cards.  Determine the probability that the card is
\begin{enumerate}[label=(\roman*)]
\item a club
\item 10 of hearts
\end{enumerate}
\solution
%\input{exemplar/10/13/3/29/main.tex}
\item A team of medical students doing their internship have to assist during surgeries
at a city hospital. The probabilities of surgeries rated as very complex, complex,
routine, simple or very simple are respectively, 0.15, 0.20, 0.31, 0.26, .08. Find
the probabilities that a particular surgery will be rated
\begin{enumerate}
	\item complex or very complex;
	\item neither very complex nor very simple;
	\item routine or complex
	\item routine or simple
\end{enumerate}
\solution
%\input{exemplar/11/16/3/8(1)/main.tex}
\item A card is selected from a pack of 52 cards.
\begin{enumerate}[label=(\alph*)]
    \item How many points are there in the sample space?
    \item Calculate the probability that the card is an ace of spades.
    \item Calculate the probability that the card is (i) an ace and (ii) black card.
\end{enumerate}
\solution
%\input{exemplar/11/16/3/4/main2.tex}
\item The probability that a non leap year selected at random will contain 53 sundays.
\\
\solution
%\input{exemplar/10/13/1/19/main.tex}
\item One of the four persons John, Rita, Aslam or Gurpreet will be promoted next
month. Consequently the sample space consists of four elementary outcomes
S = {John promoted, Rita promoted, Aslam promoted, Gurpreet promoted}
You are told that the chances of John’s promotion is same as that of Gurpreet,
Rita’s chances of promotion are twice as likely as Johns. Aslam’s chances are
four times that of John.
\begin{enumerate}
	\item Determine
	\begin{enumerate}
		\item P (John promoted)
		\item P (Rita promoted)
		\item P (Aslam promoted)
		\item P (Gurpreet promoted)
	\end{enumerate}
	\item If A = {John promoted or Gurpreet promoted}, find P (A).
\end{enumerate}
\solution
%\input{exemplar/11/16/3/10/main.tex}
\item A card is drawn from a deck of 52 cards. Find the probability of getting a king or a heart or a red card.\\
\solution
%\input{exemplar/11/16/3/15/main.tex}
\item The probability that a student will pass his examination is 0.73, the probability of
the student getting a compartment is 0.13, and the probability that the student will
either pass or get compartment is 0.96. State True or False.\\
\solution
%\input{exemplar/11/16/3/31/main.tex}
\item A card is selected from a pack of 52 cards\\
\begin{enumerate}[label=(\alph*)]
\item How many points are there in the sample space?
\item Calculate the probability that the cards is an ace of spades.
\item Calculate the probability that the card is (i) an ace (ii)black card.\\
\end{enumerate}
%\input{ncert/11/16/3/4_1/Prob_4.tex}
\item In a non-leap year, the probability of having 53 tuesdays or 53 wednesdays is\\
\solution
%\input{exemplar/11/16/3/18/main.tex}
\item There are 1000 sealed envelopes in a box, 10 of them contain a cash prize of
Rs 100 each, 100 of them contain a cash prize of Rs 50 each and 200 of them
contain a cash prize of Rs 10 each and rest do not contain any cash prize. If they
are well shuffled and an envelope is picked up out, what is the probability that it
contains no cash prize?\\
\solution
%\input{exemplar/10/13/3/34/main.tex}
\item 
A die is thrown and a card is selected at random from a deck of 52 playing cards. The probability of getting an even number on the die and a spade card.\\
\solution
%\input{exemplar/12/13/3/78/main.tex}
\item
If 4-digit numbers greater than 5,000 are randomly formed from the digits 0, 1, 3, 5, and 7, what is the probability of forming a number divisible by 5 when:
\begin{enumerate}
    \item The digits are repeated?
    \item The repetition of digits is not allowed?
\end{enumerate}
\solution
%\input{ncert/11/16/4/9/main.tex}
\item Consider the probability space $\brak{\Omega, \mathcal{G}, P}$ where $\Omega = [0,2]$ and $\mathcal{G} = \cbrak{\phi, \Omega, [0,1], (1,2]}$. Let $X$ and $Y$ be two functions on $\Omega$ defined as
\begin{align*}
    X(\omega) = 
    \begin{cases}
        1 & \text{if }\omega \in [0, 1]\\
        2 & \text{if }\omega \in (1, 2]
    \end{cases}
\end{align*}
and
\begin{align*}
    Y(\omega) = 
    \begin{cases}
        2 & \text{if }\omega \in [0, 1.5]\\
        3 & \text{if }\omega \in (1.5, 2].
    \end{cases}
\end{align*}
Then which one of the following statements is true?
\begin{enumerate}
    \item [(A)] $X$ is a random variable with respect to $\mathcal{G}$, but $Y$ is not a random variable with respect to $\mathcal{G}$.
    \item [(B)] $Y$ is a random variable with respect to $\mathcal{G}$, but $X$ is not a random variable with respect to $\mathcal{G}$.
    \item [(C)] Neither $X$ nor $Y$ is a random variable with respect to $\mathcal{G}$.
    \item [(D)] Both $X$ and $Y$ are random variables with respect to $\mathcal{G}$.
\end{enumerate} \hfill (GATE ST 2023)\\
\solution
%\input{gate/ST/2023/14/main.tex}
	\item  A die is loaded in such a way that each odd number is twice as likely to occur as
each even number. Find $P(G)$, where $G$ is the event that a number greater than
3 occurs on a single roll of the die.
\\
\solution
		%\input{exemplar/11/16/3/5/main.tex}
	\item All the jacks, queens and kings are removed from a deck of 52 playing cards. The remaining cards are well shuffled and then one card is drawn at random. Giving ace a value 1 similar value for other cards, find the probability that the card has a value 
		\begin{enumerate}
			\item 7
			\item greater than 7
			\item less than 7
		\end{enumerate}
		%\input{exemplar/10/13/3/30/main.tex}
  \item A Lot consists of 48 mobile phones of which 42 are good, 3 have only minor defects and 3 have major defects.Varnika will buy a phone if it is good but the trader will only buy a mobile if it has no major defects. One phone is selected at random from the lot. What is the probability that it is
\begin{enumerate}
	\item acceptable to Varnika?
            \item acceptable to the trader?
\end{enumerate}
\solution
	%\input{exemplar/10/13/3/40/main.tex}
 \item A student says that if you throw a die, it will show up 1 or not 1. Therefore, the probability of getting 1 and the probability of getting 'not 1' each is equal to $\frac{1}{2}$. Is this correct? Give reasons.\\
 \solution
        %\input{exemplar/10/13/2/9/main.tex}
   \item Four candidates A, B, C, D have ap-
plied for the assignment to coach a school cricket
team. If A is twice as likely to be selected as B, and
B and C are given about the same chance of being
selected, while C is twice as likely to be selected
as D, what are the probabilities that
\begin{enumerate}
\item C will be selected?
\item A will not be selected?
\end{enumerate}
	%\input{exemplar/11/16/3/9/main.tex}
 \item A bag contain 24 balls of which $x$ balls are red, $2x$ are white and $3x$ are blue. A ball is selected at random, What is the probability that it is
\begin{enumerate}[label=\alph*)]
\item not red ?
\item white ?
\end{enumerate}
%\input{exemplar/10/13/3/41/main.tex}
If the letters of the word ASSASSINATION are arranged at random. Find the Probability that
\begin{enumerate}[label=(\alph*)]
\item Four $S's$ come consecutively in the word
\item Two  $I's$ and two $N's$ come together
\item All $A's$ are not coming together
\item No two $A's$ are coming together
\end{enumerate}
%\input{exemplar/11/16/3/14/main.tex}
	\item One urn contains two black balls (labelled B1 and B2) and one white ball. A
	second urn contains one black ball and two white balls (labelled W1 and W2).
	Suppose the following experiment is performed. One of the two urns is chosen
	at random. Next a ball is randomly chosen from the urn. Then a second ball is
	chosen at random from the same urn without replacing the first ball.
	
	\begin{enumerate}
	\item What is the probability that two black balls are chosen?
	
	\item What is the probability that two balls of opposite colour are chosen?
	\end{enumerate}
	\solution
	%\input{exemplar/11/16/3/12/main1.tex}
\end{enumerate}

		%
\item 
Two cards are drawn at random and without replacement from a pack of 52 playing cards. Find the probability that both the cards are black.
\\
\solution
		%\begin{enumerate}[label=\thesection.\arabic*,ref=\thesection.\theenumi]
	\item One card is drawn from a well-shuffled deck of 52 cards. Find the probability of getting
\begin{enumerate}
\item A king of red colour 
\item A face card 
\item A red face card
\item The jack of hearts
\item A spade
\item The queen of diamonds

\end{enumerate}
\solution
		%\input{ncert/10/15/1/14/main.tex}
	\item Five cards—the ten, jack, queen, king and ace of diamonds, are well-shuffled with their face downwards. One card is then picked up at random.
\begin{enumerate}
\item
What is the probability that the card is the queen? 
\item
If the queen is drawn and put aside, what is the probability that the second card picked up is (a) an ace? (b) a queen?\\
\end{enumerate}
\solution
		%\input{ncert/10/15/1/15/defs.tex}
	\item A bag contains $5$ red balls and some blue balls. If the probability of drawing a blue ball is double that if a red ball, determine the number of blue balls in the bag. 
		\\
\solution
		%\input{ncert/10/15/2/3/defs.tex}
	\item A card is selected from a pack of 52 cards.
 \begin{enumerate}[label=(\alph*)] 
                 \item How many points are there in the sample space?
                 \item Calculate the probability that the card is an ace of spades.
                 \item Calculate the probability that the card is (i) an ace and (ii) black card.
 \end{enumerate}
\solution
		%\input{ncert/11/16/3/4/main.tex}
\item Four cards are drawn from a well-shuffled deck of 52 cards. What is the probability of obtaining 3 diamonds and one spade.
\\
\solution
		%\input{ncert/11/16/4/2/defs.tex}
\item In a certain lottery 10,000 tickets are sold and ten equal prizes are awarded. What is the probability of not getting a prize if you buy (a) one ticket (b) two tickets (c) 10 tickets ?	
\\
\solution
		%\input{ncert/11/16/4/4/defs.tex}
		%
\item 
Out of 100 students, two sections of 40 and 60 are formed. If you and your friend are among the 100 students, what is the probability that
\begin{enumerate}
\item you both enter the same section?
\item you both enter the different sections?
\end{enumerate}
\solution
		%\input{ncert/11/16/4/5/defs.tex}
	\item 
The number lock of a suitcase has 4 wheels each labelled with ten digits i.e. from 0 to 9.The lock opens with a sequence of four digits with no repeats.What is the probability of a person getting the right sequence to open the suitcase.
\\
\solution
		%\input{ncert/11/16/4/10/defs.tex}
		%
\item 
Two cards are drawn at random and without replacement from a pack of 52 playing cards. Find the probability that both the cards are black.
\\
\solution
		%\input{ncert/12/13/2/2/defs.tex}
		\item A box of oranges is inspected by examining three randomly selected oranges drawn without replacement. If all the three oranges are good, the box is approved for sale, otherwise, it is rejected. Find the probability that a box containing 15 oranges out of which 12 are good and 3 are bad ones will be approved for sale.
		\label{ncert/12/13/2/3/defs.tex}
		\item Two balls are drawn at random with replacement from a box containing 10 black and 8 red balls. Find the probability that
		\label{ncert/12/13/2/12}
\begin{enumerate}
\item both balls are red.
\item first ball is black and second is red.
\item one of them is black and other is red.
\end{enumerate}

\item In a hostel, 60\% of the students read Hindi newspaper, 40\% read English newspaper and 20\% read both Hindi and English newspapers. A student is selected at random.
		\label{ncert/12/13/2/15}
\begin{enumerate}
\item Find the probability that she reads neither Hindi nor English newspapers.
\item If she reads Hindi newspaper, find the probability that she reads English newspaper.
\item If she reads English newspaper, find the probability that she reads Hindi newspaper.\\
\end{enumerate}
\item The probability of obtaining an even prime number on each die, when a pair of dice is rolled is 
\begin{enumerate}
    \item $0$ 
    
    \item $\frac{1}{3}$ 
    
    \item $\frac{1}{12}$ 
    
    \item $\frac{1}{36}$ 
\end{enumerate}
\solution
		%\input{ncert/12/13/2/17/defs.tex}
	\item A bag contains 4 red and 4 black balls, another bag contains 2 red and 6 black balls. One of the two bags is selected at random and a ball is drawn from the bag which is found to be red. Find the probability that the ball is drawn from the first bag.
\\
\solution
		%\input{ncert/12/13/3/2/main.tex}
  \item
  Cards with numbers 2 to 101 are placed in a box. A card is selected at random.Find the probability that the card has
\begin{enumerate}[label=(\roman*)]
	\item an even number 
	\item a square number
\end{enumerate}
\solution
%\input{exemplar/10/13/3/32/main.tex}
\item
The king, queen and jack of clubs are removed from a deck of 52 playing cards and then well shuffled. Now one card is drawn at random from the remaining cards.  Determine the probability that the card is
\begin{enumerate}[label=(\roman*)]
\item a club
\item 10 of hearts
\end{enumerate}
\solution
%\input{exemplar/10/13/3/29/main.tex}
\item A team of medical students doing their internship have to assist during surgeries
at a city hospital. The probabilities of surgeries rated as very complex, complex,
routine, simple or very simple are respectively, 0.15, 0.20, 0.31, 0.26, .08. Find
the probabilities that a particular surgery will be rated
\begin{enumerate}
	\item complex or very complex;
	\item neither very complex nor very simple;
	\item routine or complex
	\item routine or simple
\end{enumerate}
\solution
%\input{exemplar/11/16/3/8(1)/main.tex}
\item A card is selected from a pack of 52 cards.
\begin{enumerate}[label=(\alph*)]
    \item How many points are there in the sample space?
    \item Calculate the probability that the card is an ace of spades.
    \item Calculate the probability that the card is (i) an ace and (ii) black card.
\end{enumerate}
\solution
%\input{exemplar/11/16/3/4/main2.tex}
\item The probability that a non leap year selected at random will contain 53 sundays.
\\
\solution
%\input{exemplar/10/13/1/19/main.tex}
\item One of the four persons John, Rita, Aslam or Gurpreet will be promoted next
month. Consequently the sample space consists of four elementary outcomes
S = {John promoted, Rita promoted, Aslam promoted, Gurpreet promoted}
You are told that the chances of John’s promotion is same as that of Gurpreet,
Rita’s chances of promotion are twice as likely as Johns. Aslam’s chances are
four times that of John.
\begin{enumerate}
	\item Determine
	\begin{enumerate}
		\item P (John promoted)
		\item P (Rita promoted)
		\item P (Aslam promoted)
		\item P (Gurpreet promoted)
	\end{enumerate}
	\item If A = {John promoted or Gurpreet promoted}, find P (A).
\end{enumerate}
\solution
%\input{exemplar/11/16/3/10/main.tex}
\item A card is drawn from a deck of 52 cards. Find the probability of getting a king or a heart or a red card.\\
\solution
%\input{exemplar/11/16/3/15/main.tex}
\item The probability that a student will pass his examination is 0.73, the probability of
the student getting a compartment is 0.13, and the probability that the student will
either pass or get compartment is 0.96. State True or False.\\
\solution
%\input{exemplar/11/16/3/31/main.tex}
\item A card is selected from a pack of 52 cards\\
\begin{enumerate}[label=(\alph*)]
\item How many points are there in the sample space?
\item Calculate the probability that the cards is an ace of spades.
\item Calculate the probability that the card is (i) an ace (ii)black card.\\
\end{enumerate}
%\input{ncert/11/16/3/4_1/Prob_4.tex}
\item In a non-leap year, the probability of having 53 tuesdays or 53 wednesdays is\\
\solution
%\input{exemplar/11/16/3/18/main.tex}
\item There are 1000 sealed envelopes in a box, 10 of them contain a cash prize of
Rs 100 each, 100 of them contain a cash prize of Rs 50 each and 200 of them
contain a cash prize of Rs 10 each and rest do not contain any cash prize. If they
are well shuffled and an envelope is picked up out, what is the probability that it
contains no cash prize?\\
\solution
%\input{exemplar/10/13/3/34/main.tex}
\item 
A die is thrown and a card is selected at random from a deck of 52 playing cards. The probability of getting an even number on the die and a spade card.\\
\solution
%\input{exemplar/12/13/3/78/main.tex}
\item
If 4-digit numbers greater than 5,000 are randomly formed from the digits 0, 1, 3, 5, and 7, what is the probability of forming a number divisible by 5 when:
\begin{enumerate}
    \item The digits are repeated?
    \item The repetition of digits is not allowed?
\end{enumerate}
\solution
%\input{ncert/11/16/4/9/main.tex}
\item Consider the probability space $\brak{\Omega, \mathcal{G}, P}$ where $\Omega = [0,2]$ and $\mathcal{G} = \cbrak{\phi, \Omega, [0,1], (1,2]}$. Let $X$ and $Y$ be two functions on $\Omega$ defined as
\begin{align*}
    X(\omega) = 
    \begin{cases}
        1 & \text{if }\omega \in [0, 1]\\
        2 & \text{if }\omega \in (1, 2]
    \end{cases}
\end{align*}
and
\begin{align*}
    Y(\omega) = 
    \begin{cases}
        2 & \text{if }\omega \in [0, 1.5]\\
        3 & \text{if }\omega \in (1.5, 2].
    \end{cases}
\end{align*}
Then which one of the following statements is true?
\begin{enumerate}
    \item [(A)] $X$ is a random variable with respect to $\mathcal{G}$, but $Y$ is not a random variable with respect to $\mathcal{G}$.
    \item [(B)] $Y$ is a random variable with respect to $\mathcal{G}$, but $X$ is not a random variable with respect to $\mathcal{G}$.
    \item [(C)] Neither $X$ nor $Y$ is a random variable with respect to $\mathcal{G}$.
    \item [(D)] Both $X$ and $Y$ are random variables with respect to $\mathcal{G}$.
\end{enumerate} \hfill (GATE ST 2023)\\
\solution
%\input{gate/ST/2023/14/main.tex}
	\item  A die is loaded in such a way that each odd number is twice as likely to occur as
each even number. Find $P(G)$, where $G$ is the event that a number greater than
3 occurs on a single roll of the die.
\\
\solution
		%\input{exemplar/11/16/3/5/main.tex}
	\item All the jacks, queens and kings are removed from a deck of 52 playing cards. The remaining cards are well shuffled and then one card is drawn at random. Giving ace a value 1 similar value for other cards, find the probability that the card has a value 
		\begin{enumerate}
			\item 7
			\item greater than 7
			\item less than 7
		\end{enumerate}
		%\input{exemplar/10/13/3/30/main.tex}
  \item A Lot consists of 48 mobile phones of which 42 are good, 3 have only minor defects and 3 have major defects.Varnika will buy a phone if it is good but the trader will only buy a mobile if it has no major defects. One phone is selected at random from the lot. What is the probability that it is
\begin{enumerate}
	\item acceptable to Varnika?
            \item acceptable to the trader?
\end{enumerate}
\solution
	%\input{exemplar/10/13/3/40/main.tex}
 \item A student says that if you throw a die, it will show up 1 or not 1. Therefore, the probability of getting 1 and the probability of getting 'not 1' each is equal to $\frac{1}{2}$. Is this correct? Give reasons.\\
 \solution
        %\input{exemplar/10/13/2/9/main.tex}
   \item Four candidates A, B, C, D have ap-
plied for the assignment to coach a school cricket
team. If A is twice as likely to be selected as B, and
B and C are given about the same chance of being
selected, while C is twice as likely to be selected
as D, what are the probabilities that
\begin{enumerate}
\item C will be selected?
\item A will not be selected?
\end{enumerate}
	%\input{exemplar/11/16/3/9/main.tex}
 \item A bag contain 24 balls of which $x$ balls are red, $2x$ are white and $3x$ are blue. A ball is selected at random, What is the probability that it is
\begin{enumerate}[label=\alph*)]
\item not red ?
\item white ?
\end{enumerate}
%\input{exemplar/10/13/3/41/main.tex}
If the letters of the word ASSASSINATION are arranged at random. Find the Probability that
\begin{enumerate}[label=(\alph*)]
\item Four $S's$ come consecutively in the word
\item Two  $I's$ and two $N's$ come together
\item All $A's$ are not coming together
\item No two $A's$ are coming together
\end{enumerate}
%\input{exemplar/11/16/3/14/main.tex}
	\item One urn contains two black balls (labelled B1 and B2) and one white ball. A
	second urn contains one black ball and two white balls (labelled W1 and W2).
	Suppose the following experiment is performed. One of the two urns is chosen
	at random. Next a ball is randomly chosen from the urn. Then a second ball is
	chosen at random from the same urn without replacing the first ball.
	
	\begin{enumerate}
	\item What is the probability that two black balls are chosen?
	
	\item What is the probability that two balls of opposite colour are chosen?
	\end{enumerate}
	\solution
	%\input{exemplar/11/16/3/12/main1.tex}
\end{enumerate}

		\item A box of oranges is inspected by examining three randomly selected oranges drawn without replacement. If all the three oranges are good, the box is approved for sale, otherwise, it is rejected. Find the probability that a box containing 15 oranges out of which 12 are good and 3 are bad ones will be approved for sale.
		\label{ncert/12/13/2/3/defs.tex}
		\item Two balls are drawn at random with replacement from a box containing 10 black and 8 red balls. Find the probability that
		\label{ncert/12/13/2/12}
\begin{enumerate}
\item both balls are red.
\item first ball is black and second is red.
\item one of them is black and other is red.
\end{enumerate}

\item In a hostel, 60\% of the students read Hindi newspaper, 40\% read English newspaper and 20\% read both Hindi and English newspapers. A student is selected at random.
		\label{ncert/12/13/2/15}
\begin{enumerate}
\item Find the probability that she reads neither Hindi nor English newspapers.
\item If she reads Hindi newspaper, find the probability that she reads English newspaper.
\item If she reads English newspaper, find the probability that she reads Hindi newspaper.\\
\end{enumerate}
\item The probability of obtaining an even prime number on each die, when a pair of dice is rolled is 
\begin{enumerate}
    \item $0$ 
    
    \item $\frac{1}{3}$ 
    
    \item $\frac{1}{12}$ 
    
    \item $\frac{1}{36}$ 
\end{enumerate}
\solution
		%\begin{enumerate}[label=\thesection.\arabic*,ref=\thesection.\theenumi]
	\item One card is drawn from a well-shuffled deck of 52 cards. Find the probability of getting
\begin{enumerate}
\item A king of red colour 
\item A face card 
\item A red face card
\item The jack of hearts
\item A spade
\item The queen of diamonds

\end{enumerate}
\solution
		%\input{ncert/10/15/1/14/main.tex}
	\item Five cards—the ten, jack, queen, king and ace of diamonds, are well-shuffled with their face downwards. One card is then picked up at random.
\begin{enumerate}
\item
What is the probability that the card is the queen? 
\item
If the queen is drawn and put aside, what is the probability that the second card picked up is (a) an ace? (b) a queen?\\
\end{enumerate}
\solution
		%\input{ncert/10/15/1/15/defs.tex}
	\item A bag contains $5$ red balls and some blue balls. If the probability of drawing a blue ball is double that if a red ball, determine the number of blue balls in the bag. 
		\\
\solution
		%\input{ncert/10/15/2/3/defs.tex}
	\item A card is selected from a pack of 52 cards.
 \begin{enumerate}[label=(\alph*)] 
                 \item How many points are there in the sample space?
                 \item Calculate the probability that the card is an ace of spades.
                 \item Calculate the probability that the card is (i) an ace and (ii) black card.
 \end{enumerate}
\solution
		%\input{ncert/11/16/3/4/main.tex}
\item Four cards are drawn from a well-shuffled deck of 52 cards. What is the probability of obtaining 3 diamonds and one spade.
\\
\solution
		%\input{ncert/11/16/4/2/defs.tex}
\item In a certain lottery 10,000 tickets are sold and ten equal prizes are awarded. What is the probability of not getting a prize if you buy (a) one ticket (b) two tickets (c) 10 tickets ?	
\\
\solution
		%\input{ncert/11/16/4/4/defs.tex}
		%
\item 
Out of 100 students, two sections of 40 and 60 are formed. If you and your friend are among the 100 students, what is the probability that
\begin{enumerate}
\item you both enter the same section?
\item you both enter the different sections?
\end{enumerate}
\solution
		%\input{ncert/11/16/4/5/defs.tex}
	\item 
The number lock of a suitcase has 4 wheels each labelled with ten digits i.e. from 0 to 9.The lock opens with a sequence of four digits with no repeats.What is the probability of a person getting the right sequence to open the suitcase.
\\
\solution
		%\input{ncert/11/16/4/10/defs.tex}
		%
\item 
Two cards are drawn at random and without replacement from a pack of 52 playing cards. Find the probability that both the cards are black.
\\
\solution
		%\input{ncert/12/13/2/2/defs.tex}
		\item A box of oranges is inspected by examining three randomly selected oranges drawn without replacement. If all the three oranges are good, the box is approved for sale, otherwise, it is rejected. Find the probability that a box containing 15 oranges out of which 12 are good and 3 are bad ones will be approved for sale.
		\label{ncert/12/13/2/3/defs.tex}
		\item Two balls are drawn at random with replacement from a box containing 10 black and 8 red balls. Find the probability that
		\label{ncert/12/13/2/12}
\begin{enumerate}
\item both balls are red.
\item first ball is black and second is red.
\item one of them is black and other is red.
\end{enumerate}

\item In a hostel, 60\% of the students read Hindi newspaper, 40\% read English newspaper and 20\% read both Hindi and English newspapers. A student is selected at random.
		\label{ncert/12/13/2/15}
\begin{enumerate}
\item Find the probability that she reads neither Hindi nor English newspapers.
\item If she reads Hindi newspaper, find the probability that she reads English newspaper.
\item If she reads English newspaper, find the probability that she reads Hindi newspaper.\\
\end{enumerate}
\item The probability of obtaining an even prime number on each die, when a pair of dice is rolled is 
\begin{enumerate}
    \item $0$ 
    
    \item $\frac{1}{3}$ 
    
    \item $\frac{1}{12}$ 
    
    \item $\frac{1}{36}$ 
\end{enumerate}
\solution
		%\input{ncert/12/13/2/17/defs.tex}
	\item A bag contains 4 red and 4 black balls, another bag contains 2 red and 6 black balls. One of the two bags is selected at random and a ball is drawn from the bag which is found to be red. Find the probability that the ball is drawn from the first bag.
\\
\solution
		%\input{ncert/12/13/3/2/main.tex}
  \item
  Cards with numbers 2 to 101 are placed in a box. A card is selected at random.Find the probability that the card has
\begin{enumerate}[label=(\roman*)]
	\item an even number 
	\item a square number
\end{enumerate}
\solution
%\input{exemplar/10/13/3/32/main.tex}
\item
The king, queen and jack of clubs are removed from a deck of 52 playing cards and then well shuffled. Now one card is drawn at random from the remaining cards.  Determine the probability that the card is
\begin{enumerate}[label=(\roman*)]
\item a club
\item 10 of hearts
\end{enumerate}
\solution
%\input{exemplar/10/13/3/29/main.tex}
\item A team of medical students doing their internship have to assist during surgeries
at a city hospital. The probabilities of surgeries rated as very complex, complex,
routine, simple or very simple are respectively, 0.15, 0.20, 0.31, 0.26, .08. Find
the probabilities that a particular surgery will be rated
\begin{enumerate}
	\item complex or very complex;
	\item neither very complex nor very simple;
	\item routine or complex
	\item routine or simple
\end{enumerate}
\solution
%\input{exemplar/11/16/3/8(1)/main.tex}
\item A card is selected from a pack of 52 cards.
\begin{enumerate}[label=(\alph*)]
    \item How many points are there in the sample space?
    \item Calculate the probability that the card is an ace of spades.
    \item Calculate the probability that the card is (i) an ace and (ii) black card.
\end{enumerate}
\solution
%\input{exemplar/11/16/3/4/main2.tex}
\item The probability that a non leap year selected at random will contain 53 sundays.
\\
\solution
%\input{exemplar/10/13/1/19/main.tex}
\item One of the four persons John, Rita, Aslam or Gurpreet will be promoted next
month. Consequently the sample space consists of four elementary outcomes
S = {John promoted, Rita promoted, Aslam promoted, Gurpreet promoted}
You are told that the chances of John’s promotion is same as that of Gurpreet,
Rita’s chances of promotion are twice as likely as Johns. Aslam’s chances are
four times that of John.
\begin{enumerate}
	\item Determine
	\begin{enumerate}
		\item P (John promoted)
		\item P (Rita promoted)
		\item P (Aslam promoted)
		\item P (Gurpreet promoted)
	\end{enumerate}
	\item If A = {John promoted or Gurpreet promoted}, find P (A).
\end{enumerate}
\solution
%\input{exemplar/11/16/3/10/main.tex}
\item A card is drawn from a deck of 52 cards. Find the probability of getting a king or a heart or a red card.\\
\solution
%\input{exemplar/11/16/3/15/main.tex}
\item The probability that a student will pass his examination is 0.73, the probability of
the student getting a compartment is 0.13, and the probability that the student will
either pass or get compartment is 0.96. State True or False.\\
\solution
%\input{exemplar/11/16/3/31/main.tex}
\item A card is selected from a pack of 52 cards\\
\begin{enumerate}[label=(\alph*)]
\item How many points are there in the sample space?
\item Calculate the probability that the cards is an ace of spades.
\item Calculate the probability that the card is (i) an ace (ii)black card.\\
\end{enumerate}
%\input{ncert/11/16/3/4_1/Prob_4.tex}
\item In a non-leap year, the probability of having 53 tuesdays or 53 wednesdays is\\
\solution
%\input{exemplar/11/16/3/18/main.tex}
\item There are 1000 sealed envelopes in a box, 10 of them contain a cash prize of
Rs 100 each, 100 of them contain a cash prize of Rs 50 each and 200 of them
contain a cash prize of Rs 10 each and rest do not contain any cash prize. If they
are well shuffled and an envelope is picked up out, what is the probability that it
contains no cash prize?\\
\solution
%\input{exemplar/10/13/3/34/main.tex}
\item 
A die is thrown and a card is selected at random from a deck of 52 playing cards. The probability of getting an even number on the die and a spade card.\\
\solution
%\input{exemplar/12/13/3/78/main.tex}
\item
If 4-digit numbers greater than 5,000 are randomly formed from the digits 0, 1, 3, 5, and 7, what is the probability of forming a number divisible by 5 when:
\begin{enumerate}
    \item The digits are repeated?
    \item The repetition of digits is not allowed?
\end{enumerate}
\solution
%\input{ncert/11/16/4/9/main.tex}
\item Consider the probability space $\brak{\Omega, \mathcal{G}, P}$ where $\Omega = [0,2]$ and $\mathcal{G} = \cbrak{\phi, \Omega, [0,1], (1,2]}$. Let $X$ and $Y$ be two functions on $\Omega$ defined as
\begin{align*}
    X(\omega) = 
    \begin{cases}
        1 & \text{if }\omega \in [0, 1]\\
        2 & \text{if }\omega \in (1, 2]
    \end{cases}
\end{align*}
and
\begin{align*}
    Y(\omega) = 
    \begin{cases}
        2 & \text{if }\omega \in [0, 1.5]\\
        3 & \text{if }\omega \in (1.5, 2].
    \end{cases}
\end{align*}
Then which one of the following statements is true?
\begin{enumerate}
    \item [(A)] $X$ is a random variable with respect to $\mathcal{G}$, but $Y$ is not a random variable with respect to $\mathcal{G}$.
    \item [(B)] $Y$ is a random variable with respect to $\mathcal{G}$, but $X$ is not a random variable with respect to $\mathcal{G}$.
    \item [(C)] Neither $X$ nor $Y$ is a random variable with respect to $\mathcal{G}$.
    \item [(D)] Both $X$ and $Y$ are random variables with respect to $\mathcal{G}$.
\end{enumerate} \hfill (GATE ST 2023)\\
\solution
%\input{gate/ST/2023/14/main.tex}
	\item  A die is loaded in such a way that each odd number is twice as likely to occur as
each even number. Find $P(G)$, where $G$ is the event that a number greater than
3 occurs on a single roll of the die.
\\
\solution
		%\input{exemplar/11/16/3/5/main.tex}
	\item All the jacks, queens and kings are removed from a deck of 52 playing cards. The remaining cards are well shuffled and then one card is drawn at random. Giving ace a value 1 similar value for other cards, find the probability that the card has a value 
		\begin{enumerate}
			\item 7
			\item greater than 7
			\item less than 7
		\end{enumerate}
		%\input{exemplar/10/13/3/30/main.tex}
  \item A Lot consists of 48 mobile phones of which 42 are good, 3 have only minor defects and 3 have major defects.Varnika will buy a phone if it is good but the trader will only buy a mobile if it has no major defects. One phone is selected at random from the lot. What is the probability that it is
\begin{enumerate}
	\item acceptable to Varnika?
            \item acceptable to the trader?
\end{enumerate}
\solution
	%\input{exemplar/10/13/3/40/main.tex}
 \item A student says that if you throw a die, it will show up 1 or not 1. Therefore, the probability of getting 1 and the probability of getting 'not 1' each is equal to $\frac{1}{2}$. Is this correct? Give reasons.\\
 \solution
        %\input{exemplar/10/13/2/9/main.tex}
   \item Four candidates A, B, C, D have ap-
plied for the assignment to coach a school cricket
team. If A is twice as likely to be selected as B, and
B and C are given about the same chance of being
selected, while C is twice as likely to be selected
as D, what are the probabilities that
\begin{enumerate}
\item C will be selected?
\item A will not be selected?
\end{enumerate}
	%\input{exemplar/11/16/3/9/main.tex}
 \item A bag contain 24 balls of which $x$ balls are red, $2x$ are white and $3x$ are blue. A ball is selected at random, What is the probability that it is
\begin{enumerate}[label=\alph*)]
\item not red ?
\item white ?
\end{enumerate}
%\input{exemplar/10/13/3/41/main.tex}
If the letters of the word ASSASSINATION are arranged at random. Find the Probability that
\begin{enumerate}[label=(\alph*)]
\item Four $S's$ come consecutively in the word
\item Two  $I's$ and two $N's$ come together
\item All $A's$ are not coming together
\item No two $A's$ are coming together
\end{enumerate}
%\input{exemplar/11/16/3/14/main.tex}
	\item One urn contains two black balls (labelled B1 and B2) and one white ball. A
	second urn contains one black ball and two white balls (labelled W1 and W2).
	Suppose the following experiment is performed. One of the two urns is chosen
	at random. Next a ball is randomly chosen from the urn. Then a second ball is
	chosen at random from the same urn without replacing the first ball.
	
	\begin{enumerate}
	\item What is the probability that two black balls are chosen?
	
	\item What is the probability that two balls of opposite colour are chosen?
	\end{enumerate}
	\solution
	%\input{exemplar/11/16/3/12/main1.tex}
\end{enumerate}

	\item A bag contains 4 red and 4 black balls, another bag contains 2 red and 6 black balls. One of the two bags is selected at random and a ball is drawn from the bag which is found to be red. Find the probability that the ball is drawn from the first bag.
\\
\solution
		%\begin{table}[H]
	\centering
\begin{tabular}{|c|c|c|}
\hline
Random variable &Value &Definition\\ \hline
\multirow{3}{*}{X} &0 &Slips of Rs 1\\
&1 &Slips of Rs 5\\
&2 &Slips of Rs 13\\ \hline
\multirow{2}{*}{Y} &0 &Box A\\
&1 &Box B\\\hline
\end{tabular}
\caption{}
\label{tab:Distribution}
\end{table}
See \tabref{tab:Distribution}.
\begin{align}
p_{Y}\brak{k}= \begin{cases} 
      \frac{1}{3} & {k=0} \\
      \frac{2}{3 }& {k=1} 
   \end{cases}
   \\
p_{Y|X}\brak{0|0} = \frac{19}{25}\, 
p_{Y|X}\brak{0|1} = \frac{6}{25}\,
p_{Y|X}\brak{1|0} = \frac{45}{50}\,
p_{Y|X}\brak{1|2} = \frac{5}{50}
\end{align}
The desired probability is the probability that a slip drawn at random is marked other than Rs 1,
\begin{align}
&=1-p_X\brak{0}\\
&= p_X(1) + p_X(2)
\end{align}
Using Bayes theorem,
\begin{align}
&= p_Y\brak{0} \times \pr{Y=0 | X=1} + p_Y\brak{1} \times \pr{Y=1|X=2}\\
&=\frac{1}{3} \times \frac{6}{25} + \frac{2}{3} \times \frac{5}{50}\\
&=\frac{11}{75}
\end{align}

\newpage

%\tableofcontents

\bigskip

\renewcommand{\thefigure}{\theenumi}
\renewcommand{\thetable}{\theenumi}
%\renewcommand{\theequation}{\theenumi}

%\begin{abstract}
%%\boldmath
%In this letter, an algorithm for evaluating the exact analytical bit error rate  (BER)  for the piecewise linear (PL) combiner for  multiple relays is presented. Previous results were available only for upto three relays. The algorithm is unique in the sense that  the actual mathematical expressions, that are prohibitively large, need not be explicitly obtained. The diversity gain due to multiple relays is shown through plots of the analytical BER, well supported by simulations. 
%
%\end{abstract}
% IEEEtran.cls defaults to using nonbold math in the Abstract.
% This preserves the distinction between vectors and scalars. However,
% if the journal you are submitting to favors bold math in the abstract,
% then you can use LaTeX's standard command \boldmath at the very start
% of the abstract to achieve this. Many IEEE journals frown on math
% in the abstract anyway.

% Note that keywords are not normally used for peerreview papers.
%\begin{IEEEkeywords}
%Cooperative diversity, decode and forward, piecewise linear
%\end{IEEEkeywords}



% For peer review papers, you can put extra information on the cover
% page as needed:
% \ifCLASSOPTIONpeerreview
% \begin{center} \bfseries EDICS Category: 3-BBND \end{center}
% \fi
%
% For peerreview papers, this IEEEtran command inserts a page break and
% creates the second title. It will be ignored for other modes.
%\IEEEpeerreviewmaketitle




  \item
  Cards with numbers 2 to 101 are placed in a box. A card is selected at random.Find the probability that the card has
\begin{enumerate}[label=(\roman*)]
	\item an even number 
	\item a square number
\end{enumerate}
\solution
%\begin{table}[H]
	\centering
\begin{tabular}{|c|c|c|}
\hline
Random variable &Value &Definition\\ \hline
\multirow{3}{*}{X} &0 &Slips of Rs 1\\
&1 &Slips of Rs 5\\
&2 &Slips of Rs 13\\ \hline
\multirow{2}{*}{Y} &0 &Box A\\
&1 &Box B\\\hline
\end{tabular}
\caption{}
\label{tab:Distribution}
\end{table}
See \tabref{tab:Distribution}.
\begin{align}
p_{Y}\brak{k}= \begin{cases} 
      \frac{1}{3} & {k=0} \\
      \frac{2}{3 }& {k=1} 
   \end{cases}
   \\
p_{Y|X}\brak{0|0} = \frac{19}{25}\, 
p_{Y|X}\brak{0|1} = \frac{6}{25}\,
p_{Y|X}\brak{1|0} = \frac{45}{50}\,
p_{Y|X}\brak{1|2} = \frac{5}{50}
\end{align}
The desired probability is the probability that a slip drawn at random is marked other than Rs 1,
\begin{align}
&=1-p_X\brak{0}\\
&= p_X(1) + p_X(2)
\end{align}
Using Bayes theorem,
\begin{align}
&= p_Y\brak{0} \times \pr{Y=0 | X=1} + p_Y\brak{1} \times \pr{Y=1|X=2}\\
&=\frac{1}{3} \times \frac{6}{25} + \frac{2}{3} \times \frac{5}{50}\\
&=\frac{11}{75}
\end{align}

\newpage

%\tableofcontents

\bigskip

\renewcommand{\thefigure}{\theenumi}
\renewcommand{\thetable}{\theenumi}
%\renewcommand{\theequation}{\theenumi}

%\begin{abstract}
%%\boldmath
%In this letter, an algorithm for evaluating the exact analytical bit error rate  (BER)  for the piecewise linear (PL) combiner for  multiple relays is presented. Previous results were available only for upto three relays. The algorithm is unique in the sense that  the actual mathematical expressions, that are prohibitively large, need not be explicitly obtained. The diversity gain due to multiple relays is shown through plots of the analytical BER, well supported by simulations. 
%
%\end{abstract}
% IEEEtran.cls defaults to using nonbold math in the Abstract.
% This preserves the distinction between vectors and scalars. However,
% if the journal you are submitting to favors bold math in the abstract,
% then you can use LaTeX's standard command \boldmath at the very start
% of the abstract to achieve this. Many IEEE journals frown on math
% in the abstract anyway.

% Note that keywords are not normally used for peerreview papers.
%\begin{IEEEkeywords}
%Cooperative diversity, decode and forward, piecewise linear
%\end{IEEEkeywords}



% For peer review papers, you can put extra information on the cover
% page as needed:
% \ifCLASSOPTIONpeerreview
% \begin{center} \bfseries EDICS Category: 3-BBND \end{center}
% \fi
%
% For peerreview papers, this IEEEtran command inserts a page break and
% creates the second title. It will be ignored for other modes.
%\IEEEpeerreviewmaketitle




\item
The king, queen and jack of clubs are removed from a deck of 52 playing cards and then well shuffled. Now one card is drawn at random from the remaining cards.  Determine the probability that the card is
\begin{enumerate}[label=(\roman*)]
\item a club
\item 10 of hearts
\end{enumerate}
\solution
%\begin{table}[H]
	\centering
\begin{tabular}{|c|c|c|}
\hline
Random variable &Value &Definition\\ \hline
\multirow{3}{*}{X} &0 &Slips of Rs 1\\
&1 &Slips of Rs 5\\
&2 &Slips of Rs 13\\ \hline
\multirow{2}{*}{Y} &0 &Box A\\
&1 &Box B\\\hline
\end{tabular}
\caption{}
\label{tab:Distribution}
\end{table}
See \tabref{tab:Distribution}.
\begin{align}
p_{Y}\brak{k}= \begin{cases} 
      \frac{1}{3} & {k=0} \\
      \frac{2}{3 }& {k=1} 
   \end{cases}
   \\
p_{Y|X}\brak{0|0} = \frac{19}{25}\, 
p_{Y|X}\brak{0|1} = \frac{6}{25}\,
p_{Y|X}\brak{1|0} = \frac{45}{50}\,
p_{Y|X}\brak{1|2} = \frac{5}{50}
\end{align}
The desired probability is the probability that a slip drawn at random is marked other than Rs 1,
\begin{align}
&=1-p_X\brak{0}\\
&= p_X(1) + p_X(2)
\end{align}
Using Bayes theorem,
\begin{align}
&= p_Y\brak{0} \times \pr{Y=0 | X=1} + p_Y\brak{1} \times \pr{Y=1|X=2}\\
&=\frac{1}{3} \times \frac{6}{25} + \frac{2}{3} \times \frac{5}{50}\\
&=\frac{11}{75}
\end{align}

\newpage

%\tableofcontents

\bigskip

\renewcommand{\thefigure}{\theenumi}
\renewcommand{\thetable}{\theenumi}
%\renewcommand{\theequation}{\theenumi}

%\begin{abstract}
%%\boldmath
%In this letter, an algorithm for evaluating the exact analytical bit error rate  (BER)  for the piecewise linear (PL) combiner for  multiple relays is presented. Previous results were available only for upto three relays. The algorithm is unique in the sense that  the actual mathematical expressions, that are prohibitively large, need not be explicitly obtained. The diversity gain due to multiple relays is shown through plots of the analytical BER, well supported by simulations. 
%
%\end{abstract}
% IEEEtran.cls defaults to using nonbold math in the Abstract.
% This preserves the distinction between vectors and scalars. However,
% if the journal you are submitting to favors bold math in the abstract,
% then you can use LaTeX's standard command \boldmath at the very start
% of the abstract to achieve this. Many IEEE journals frown on math
% in the abstract anyway.

% Note that keywords are not normally used for peerreview papers.
%\begin{IEEEkeywords}
%Cooperative diversity, decode and forward, piecewise linear
%\end{IEEEkeywords}



% For peer review papers, you can put extra information on the cover
% page as needed:
% \ifCLASSOPTIONpeerreview
% \begin{center} \bfseries EDICS Category: 3-BBND \end{center}
% \fi
%
% For peerreview papers, this IEEEtran command inserts a page break and
% creates the second title. It will be ignored for other modes.
%\IEEEpeerreviewmaketitle




\item A team of medical students doing their internship have to assist during surgeries
at a city hospital. The probabilities of surgeries rated as very complex, complex,
routine, simple or very simple are respectively, 0.15, 0.20, 0.31, 0.26, .08. Find
the probabilities that a particular surgery will be rated
\begin{enumerate}
	\item complex or very complex;
	\item neither very complex nor very simple;
	\item routine or complex
	\item routine or simple
\end{enumerate}
\solution
%\begin{table}[H]
	\centering
\begin{tabular}{|c|c|c|}
\hline
Random variable &Value &Definition\\ \hline
\multirow{3}{*}{X} &0 &Slips of Rs 1\\
&1 &Slips of Rs 5\\
&2 &Slips of Rs 13\\ \hline
\multirow{2}{*}{Y} &0 &Box A\\
&1 &Box B\\\hline
\end{tabular}
\caption{}
\label{tab:Distribution}
\end{table}
See \tabref{tab:Distribution}.
\begin{align}
p_{Y}\brak{k}= \begin{cases} 
      \frac{1}{3} & {k=0} \\
      \frac{2}{3 }& {k=1} 
   \end{cases}
   \\
p_{Y|X}\brak{0|0} = \frac{19}{25}\, 
p_{Y|X}\brak{0|1} = \frac{6}{25}\,
p_{Y|X}\brak{1|0} = \frac{45}{50}\,
p_{Y|X}\brak{1|2} = \frac{5}{50}
\end{align}
The desired probability is the probability that a slip drawn at random is marked other than Rs 1,
\begin{align}
&=1-p_X\brak{0}\\
&= p_X(1) + p_X(2)
\end{align}
Using Bayes theorem,
\begin{align}
&= p_Y\brak{0} \times \pr{Y=0 | X=1} + p_Y\brak{1} \times \pr{Y=1|X=2}\\
&=\frac{1}{3} \times \frac{6}{25} + \frac{2}{3} \times \frac{5}{50}\\
&=\frac{11}{75}
\end{align}

\newpage

%\tableofcontents

\bigskip

\renewcommand{\thefigure}{\theenumi}
\renewcommand{\thetable}{\theenumi}
%\renewcommand{\theequation}{\theenumi}

%\begin{abstract}
%%\boldmath
%In this letter, an algorithm for evaluating the exact analytical bit error rate  (BER)  for the piecewise linear (PL) combiner for  multiple relays is presented. Previous results were available only for upto three relays. The algorithm is unique in the sense that  the actual mathematical expressions, that are prohibitively large, need not be explicitly obtained. The diversity gain due to multiple relays is shown through plots of the analytical BER, well supported by simulations. 
%
%\end{abstract}
% IEEEtran.cls defaults to using nonbold math in the Abstract.
% This preserves the distinction between vectors and scalars. However,
% if the journal you are submitting to favors bold math in the abstract,
% then you can use LaTeX's standard command \boldmath at the very start
% of the abstract to achieve this. Many IEEE journals frown on math
% in the abstract anyway.

% Note that keywords are not normally used for peerreview papers.
%\begin{IEEEkeywords}
%Cooperative diversity, decode and forward, piecewise linear
%\end{IEEEkeywords}



% For peer review papers, you can put extra information on the cover
% page as needed:
% \ifCLASSOPTIONpeerreview
% \begin{center} \bfseries EDICS Category: 3-BBND \end{center}
% \fi
%
% For peerreview papers, this IEEEtran command inserts a page break and
% creates the second title. It will be ignored for other modes.
%\IEEEpeerreviewmaketitle




\item A card is selected from a pack of 52 cards.
\begin{enumerate}[label=(\alph*)]
    \item How many points are there in the sample space?
    \item Calculate the probability that the card is an ace of spades.
    \item Calculate the probability that the card is (i) an ace and (ii) black card.
\end{enumerate}
\solution
%Let $X$ be an bernoulli rv defined as in \tabref{tab:exemplar/11/16/3/26}.  Then, 
\begin{equation}
    p =
        \frac{4}{11} 
\end{equation}
\begin{table}[H]
	\centering
	\input{exemplar/11/16/3/26/tables/Table2.tex}
	\caption{}
        \label{tab:exemplar/11/16/3/26}
\end{table}

\item The probability that a non leap year selected at random will contain 53 sundays.
\\
\solution
%\begin{table}[H]
	\centering
\begin{tabular}{|c|c|c|}
\hline
Random variable &Value &Definition\\ \hline
\multirow{3}{*}{X} &0 &Slips of Rs 1\\
&1 &Slips of Rs 5\\
&2 &Slips of Rs 13\\ \hline
\multirow{2}{*}{Y} &0 &Box A\\
&1 &Box B\\\hline
\end{tabular}
\caption{}
\label{tab:Distribution}
\end{table}
See \tabref{tab:Distribution}.
\begin{align}
p_{Y}\brak{k}= \begin{cases} 
      \frac{1}{3} & {k=0} \\
      \frac{2}{3 }& {k=1} 
   \end{cases}
   \\
p_{Y|X}\brak{0|0} = \frac{19}{25}\, 
p_{Y|X}\brak{0|1} = \frac{6}{25}\,
p_{Y|X}\brak{1|0} = \frac{45}{50}\,
p_{Y|X}\brak{1|2} = \frac{5}{50}
\end{align}
The desired probability is the probability that a slip drawn at random is marked other than Rs 1,
\begin{align}
&=1-p_X\brak{0}\\
&= p_X(1) + p_X(2)
\end{align}
Using Bayes theorem,
\begin{align}
&= p_Y\brak{0} \times \pr{Y=0 | X=1} + p_Y\brak{1} \times \pr{Y=1|X=2}\\
&=\frac{1}{3} \times \frac{6}{25} + \frac{2}{3} \times \frac{5}{50}\\
&=\frac{11}{75}
\end{align}

\newpage

%\tableofcontents

\bigskip

\renewcommand{\thefigure}{\theenumi}
\renewcommand{\thetable}{\theenumi}
%\renewcommand{\theequation}{\theenumi}

%\begin{abstract}
%%\boldmath
%In this letter, an algorithm for evaluating the exact analytical bit error rate  (BER)  for the piecewise linear (PL) combiner for  multiple relays is presented. Previous results were available only for upto three relays. The algorithm is unique in the sense that  the actual mathematical expressions, that are prohibitively large, need not be explicitly obtained. The diversity gain due to multiple relays is shown through plots of the analytical BER, well supported by simulations. 
%
%\end{abstract}
% IEEEtran.cls defaults to using nonbold math in the Abstract.
% This preserves the distinction between vectors and scalars. However,
% if the journal you are submitting to favors bold math in the abstract,
% then you can use LaTeX's standard command \boldmath at the very start
% of the abstract to achieve this. Many IEEE journals frown on math
% in the abstract anyway.

% Note that keywords are not normally used for peerreview papers.
%\begin{IEEEkeywords}
%Cooperative diversity, decode and forward, piecewise linear
%\end{IEEEkeywords}



% For peer review papers, you can put extra information on the cover
% page as needed:
% \ifCLASSOPTIONpeerreview
% \begin{center} \bfseries EDICS Category: 3-BBND \end{center}
% \fi
%
% For peerreview papers, this IEEEtran command inserts a page break and
% creates the second title. It will be ignored for other modes.
%\IEEEpeerreviewmaketitle




\item One of the four persons John, Rita, Aslam or Gurpreet will be promoted next
month. Consequently the sample space consists of four elementary outcomes
S = {John promoted, Rita promoted, Aslam promoted, Gurpreet promoted}
You are told that the chances of John’s promotion is same as that of Gurpreet,
Rita’s chances of promotion are twice as likely as Johns. Aslam’s chances are
four times that of John.
\begin{enumerate}
	\item Determine
	\begin{enumerate}
		\item P (John promoted)
		\item P (Rita promoted)
		\item P (Aslam promoted)
		\item P (Gurpreet promoted)
	\end{enumerate}
	\item If A = {John promoted or Gurpreet promoted}, find P (A).
\end{enumerate}
\solution
%\begin{table}[H]
	\centering
\begin{tabular}{|c|c|c|}
\hline
Random variable &Value &Definition\\ \hline
\multirow{3}{*}{X} &0 &Slips of Rs 1\\
&1 &Slips of Rs 5\\
&2 &Slips of Rs 13\\ \hline
\multirow{2}{*}{Y} &0 &Box A\\
&1 &Box B\\\hline
\end{tabular}
\caption{}
\label{tab:Distribution}
\end{table}
See \tabref{tab:Distribution}.
\begin{align}
p_{Y}\brak{k}= \begin{cases} 
      \frac{1}{3} & {k=0} \\
      \frac{2}{3 }& {k=1} 
   \end{cases}
   \\
p_{Y|X}\brak{0|0} = \frac{19}{25}\, 
p_{Y|X}\brak{0|1} = \frac{6}{25}\,
p_{Y|X}\brak{1|0} = \frac{45}{50}\,
p_{Y|X}\brak{1|2} = \frac{5}{50}
\end{align}
The desired probability is the probability that a slip drawn at random is marked other than Rs 1,
\begin{align}
&=1-p_X\brak{0}\\
&= p_X(1) + p_X(2)
\end{align}
Using Bayes theorem,
\begin{align}
&= p_Y\brak{0} \times \pr{Y=0 | X=1} + p_Y\brak{1} \times \pr{Y=1|X=2}\\
&=\frac{1}{3} \times \frac{6}{25} + \frac{2}{3} \times \frac{5}{50}\\
&=\frac{11}{75}
\end{align}

\newpage

%\tableofcontents

\bigskip

\renewcommand{\thefigure}{\theenumi}
\renewcommand{\thetable}{\theenumi}
%\renewcommand{\theequation}{\theenumi}

%\begin{abstract}
%%\boldmath
%In this letter, an algorithm for evaluating the exact analytical bit error rate  (BER)  for the piecewise linear (PL) combiner for  multiple relays is presented. Previous results were available only for upto three relays. The algorithm is unique in the sense that  the actual mathematical expressions, that are prohibitively large, need not be explicitly obtained. The diversity gain due to multiple relays is shown through plots of the analytical BER, well supported by simulations. 
%
%\end{abstract}
% IEEEtran.cls defaults to using nonbold math in the Abstract.
% This preserves the distinction between vectors and scalars. However,
% if the journal you are submitting to favors bold math in the abstract,
% then you can use LaTeX's standard command \boldmath at the very start
% of the abstract to achieve this. Many IEEE journals frown on math
% in the abstract anyway.

% Note that keywords are not normally used for peerreview papers.
%\begin{IEEEkeywords}
%Cooperative diversity, decode and forward, piecewise linear
%\end{IEEEkeywords}



% For peer review papers, you can put extra information on the cover
% page as needed:
% \ifCLASSOPTIONpeerreview
% \begin{center} \bfseries EDICS Category: 3-BBND \end{center}
% \fi
%
% For peerreview papers, this IEEEtran command inserts a page break and
% creates the second title. It will be ignored for other modes.
%\IEEEpeerreviewmaketitle




\item A card is drawn from a deck of 52 cards. Find the probability of getting a king or a heart or a red card.\\
\solution
%\begin{table}[H]
	\centering
\begin{tabular}{|c|c|c|}
\hline
Random variable &Value &Definition\\ \hline
\multirow{3}{*}{X} &0 &Slips of Rs 1\\
&1 &Slips of Rs 5\\
&2 &Slips of Rs 13\\ \hline
\multirow{2}{*}{Y} &0 &Box A\\
&1 &Box B\\\hline
\end{tabular}
\caption{}
\label{tab:Distribution}
\end{table}
See \tabref{tab:Distribution}.
\begin{align}
p_{Y}\brak{k}= \begin{cases} 
      \frac{1}{3} & {k=0} \\
      \frac{2}{3 }& {k=1} 
   \end{cases}
   \\
p_{Y|X}\brak{0|0} = \frac{19}{25}\, 
p_{Y|X}\brak{0|1} = \frac{6}{25}\,
p_{Y|X}\brak{1|0} = \frac{45}{50}\,
p_{Y|X}\brak{1|2} = \frac{5}{50}
\end{align}
The desired probability is the probability that a slip drawn at random is marked other than Rs 1,
\begin{align}
&=1-p_X\brak{0}\\
&= p_X(1) + p_X(2)
\end{align}
Using Bayes theorem,
\begin{align}
&= p_Y\brak{0} \times \pr{Y=0 | X=1} + p_Y\brak{1} \times \pr{Y=1|X=2}\\
&=\frac{1}{3} \times \frac{6}{25} + \frac{2}{3} \times \frac{5}{50}\\
&=\frac{11}{75}
\end{align}

\newpage

%\tableofcontents

\bigskip

\renewcommand{\thefigure}{\theenumi}
\renewcommand{\thetable}{\theenumi}
%\renewcommand{\theequation}{\theenumi}

%\begin{abstract}
%%\boldmath
%In this letter, an algorithm for evaluating the exact analytical bit error rate  (BER)  for the piecewise linear (PL) combiner for  multiple relays is presented. Previous results were available only for upto three relays. The algorithm is unique in the sense that  the actual mathematical expressions, that are prohibitively large, need not be explicitly obtained. The diversity gain due to multiple relays is shown through plots of the analytical BER, well supported by simulations. 
%
%\end{abstract}
% IEEEtran.cls defaults to using nonbold math in the Abstract.
% This preserves the distinction between vectors and scalars. However,
% if the journal you are submitting to favors bold math in the abstract,
% then you can use LaTeX's standard command \boldmath at the very start
% of the abstract to achieve this. Many IEEE journals frown on math
% in the abstract anyway.

% Note that keywords are not normally used for peerreview papers.
%\begin{IEEEkeywords}
%Cooperative diversity, decode and forward, piecewise linear
%\end{IEEEkeywords}



% For peer review papers, you can put extra information on the cover
% page as needed:
% \ifCLASSOPTIONpeerreview
% \begin{center} \bfseries EDICS Category: 3-BBND \end{center}
% \fi
%
% For peerreview papers, this IEEEtran command inserts a page break and
% creates the second title. It will be ignored for other modes.
%\IEEEpeerreviewmaketitle




\item The probability that a student will pass his examination is 0.73, the probability of
the student getting a compartment is 0.13, and the probability that the student will
either pass or get compartment is 0.96. State True or False.\\
\solution
%\begin{table}[H]
	\centering
\begin{tabular}{|c|c|c|}
\hline
Random variable &Value &Definition\\ \hline
\multirow{3}{*}{X} &0 &Slips of Rs 1\\
&1 &Slips of Rs 5\\
&2 &Slips of Rs 13\\ \hline
\multirow{2}{*}{Y} &0 &Box A\\
&1 &Box B\\\hline
\end{tabular}
\caption{}
\label{tab:Distribution}
\end{table}
See \tabref{tab:Distribution}.
\begin{align}
p_{Y}\brak{k}= \begin{cases} 
      \frac{1}{3} & {k=0} \\
      \frac{2}{3 }& {k=1} 
   \end{cases}
   \\
p_{Y|X}\brak{0|0} = \frac{19}{25}\, 
p_{Y|X}\brak{0|1} = \frac{6}{25}\,
p_{Y|X}\brak{1|0} = \frac{45}{50}\,
p_{Y|X}\brak{1|2} = \frac{5}{50}
\end{align}
The desired probability is the probability that a slip drawn at random is marked other than Rs 1,
\begin{align}
&=1-p_X\brak{0}\\
&= p_X(1) + p_X(2)
\end{align}
Using Bayes theorem,
\begin{align}
&= p_Y\brak{0} \times \pr{Y=0 | X=1} + p_Y\brak{1} \times \pr{Y=1|X=2}\\
&=\frac{1}{3} \times \frac{6}{25} + \frac{2}{3} \times \frac{5}{50}\\
&=\frac{11}{75}
\end{align}

\newpage

%\tableofcontents

\bigskip

\renewcommand{\thefigure}{\theenumi}
\renewcommand{\thetable}{\theenumi}
%\renewcommand{\theequation}{\theenumi}

%\begin{abstract}
%%\boldmath
%In this letter, an algorithm for evaluating the exact analytical bit error rate  (BER)  for the piecewise linear (PL) combiner for  multiple relays is presented. Previous results were available only for upto three relays. The algorithm is unique in the sense that  the actual mathematical expressions, that are prohibitively large, need not be explicitly obtained. The diversity gain due to multiple relays is shown through plots of the analytical BER, well supported by simulations. 
%
%\end{abstract}
% IEEEtran.cls defaults to using nonbold math in the Abstract.
% This preserves the distinction between vectors and scalars. However,
% if the journal you are submitting to favors bold math in the abstract,
% then you can use LaTeX's standard command \boldmath at the very start
% of the abstract to achieve this. Many IEEE journals frown on math
% in the abstract anyway.

% Note that keywords are not normally used for peerreview papers.
%\begin{IEEEkeywords}
%Cooperative diversity, decode and forward, piecewise linear
%\end{IEEEkeywords}



% For peer review papers, you can put extra information on the cover
% page as needed:
% \ifCLASSOPTIONpeerreview
% \begin{center} \bfseries EDICS Category: 3-BBND \end{center}
% \fi
%
% For peerreview papers, this IEEEtran command inserts a page break and
% creates the second title. It will be ignored for other modes.
%\IEEEpeerreviewmaketitle




\item A card is selected from a pack of 52 cards\\
\begin{enumerate}[label=(\alph*)]
\item How many points are there in the sample space?
\item Calculate the probability that the cards is an ace of spades.
\item Calculate the probability that the card is (i) an ace (ii)black card.\\
\end{enumerate}
%\input{ncert/11/16/3/4_1/Prob_4.tex}
\item In a non-leap year, the probability of having 53 tuesdays or 53 wednesdays is\\
\solution
%A non-leap year has a total of 365 days, and a week has 7 days.\\
So it can be expressed as 
\begin{align}
365\text{days} &=52\times 7+1 \text{day}
\end{align}
$\implies$ 52 tuesdays or wednesdays\\
Random variable X denotes the days of a week
\begin{align}
p_X\brak{k}&=\frac{1}{7}; \quad \brak{1<k<7}
\end{align}
So the probability of extra day being tuesday or wednesday is
\begin{align}
p_X\brak{3}+p_X\brak{4}&=\frac{1}{7}+\frac{1}{7}=\frac{2}{7}
\end{align}



\item There are 1000 sealed envelopes in a box, 10 of them contain a cash prize of
Rs 100 each, 100 of them contain a cash prize of Rs 50 each and 200 of them
contain a cash prize of Rs 10 each and rest do not contain any cash prize. If they
are well shuffled and an envelope is picked up out, what is the probability that it
contains no cash prize?\\
\solution
%\begin{table}[H]
	\centering
\begin{tabular}{|c|c|c|}
\hline
Random variable &Value &Definition\\ \hline
\multirow{3}{*}{X} &0 &Slips of Rs 1\\
&1 &Slips of Rs 5\\
&2 &Slips of Rs 13\\ \hline
\multirow{2}{*}{Y} &0 &Box A\\
&1 &Box B\\\hline
\end{tabular}
\caption{}
\label{tab:Distribution}
\end{table}
See \tabref{tab:Distribution}.
\begin{align}
p_{Y}\brak{k}= \begin{cases} 
      \frac{1}{3} & {k=0} \\
      \frac{2}{3 }& {k=1} 
   \end{cases}
   \\
p_{Y|X}\brak{0|0} = \frac{19}{25}\, 
p_{Y|X}\brak{0|1} = \frac{6}{25}\,
p_{Y|X}\brak{1|0} = \frac{45}{50}\,
p_{Y|X}\brak{1|2} = \frac{5}{50}
\end{align}
The desired probability is the probability that a slip drawn at random is marked other than Rs 1,
\begin{align}
&=1-p_X\brak{0}\\
&= p_X(1) + p_X(2)
\end{align}
Using Bayes theorem,
\begin{align}
&= p_Y\brak{0} \times \pr{Y=0 | X=1} + p_Y\brak{1} \times \pr{Y=1|X=2}\\
&=\frac{1}{3} \times \frac{6}{25} + \frac{2}{3} \times \frac{5}{50}\\
&=\frac{11}{75}
\end{align}

\newpage

%\tableofcontents

\bigskip

\renewcommand{\thefigure}{\theenumi}
\renewcommand{\thetable}{\theenumi}
%\renewcommand{\theequation}{\theenumi}

%\begin{abstract}
%%\boldmath
%In this letter, an algorithm for evaluating the exact analytical bit error rate  (BER)  for the piecewise linear (PL) combiner for  multiple relays is presented. Previous results were available only for upto three relays. The algorithm is unique in the sense that  the actual mathematical expressions, that are prohibitively large, need not be explicitly obtained. The diversity gain due to multiple relays is shown through plots of the analytical BER, well supported by simulations. 
%
%\end{abstract}
% IEEEtran.cls defaults to using nonbold math in the Abstract.
% This preserves the distinction between vectors and scalars. However,
% if the journal you are submitting to favors bold math in the abstract,
% then you can use LaTeX's standard command \boldmath at the very start
% of the abstract to achieve this. Many IEEE journals frown on math
% in the abstract anyway.

% Note that keywords are not normally used for peerreview papers.
%\begin{IEEEkeywords}
%Cooperative diversity, decode and forward, piecewise linear
%\end{IEEEkeywords}



% For peer review papers, you can put extra information on the cover
% page as needed:
% \ifCLASSOPTIONpeerreview
% \begin{center} \bfseries EDICS Category: 3-BBND \end{center}
% \fi
%
% For peerreview papers, this IEEEtran command inserts a page break and
% creates the second title. It will be ignored for other modes.
%\IEEEpeerreviewmaketitle




\item 
A die is thrown and a card is selected at random from a deck of 52 playing cards. The probability of getting an even number on the die and a spade card.\\
\solution
%\begin{table}[H]
	\centering
\begin{tabular}{|c|c|c|}
\hline
Random variable &Value &Definition\\ \hline
\multirow{3}{*}{X} &0 &Slips of Rs 1\\
&1 &Slips of Rs 5\\
&2 &Slips of Rs 13\\ \hline
\multirow{2}{*}{Y} &0 &Box A\\
&1 &Box B\\\hline
\end{tabular}
\caption{}
\label{tab:Distribution}
\end{table}
See \tabref{tab:Distribution}.
\begin{align}
p_{Y}\brak{k}= \begin{cases} 
      \frac{1}{3} & {k=0} \\
      \frac{2}{3 }& {k=1} 
   \end{cases}
   \\
p_{Y|X}\brak{0|0} = \frac{19}{25}\, 
p_{Y|X}\brak{0|1} = \frac{6}{25}\,
p_{Y|X}\brak{1|0} = \frac{45}{50}\,
p_{Y|X}\brak{1|2} = \frac{5}{50}
\end{align}
The desired probability is the probability that a slip drawn at random is marked other than Rs 1,
\begin{align}
&=1-p_X\brak{0}\\
&= p_X(1) + p_X(2)
\end{align}
Using Bayes theorem,
\begin{align}
&= p_Y\brak{0} \times \pr{Y=0 | X=1} + p_Y\brak{1} \times \pr{Y=1|X=2}\\
&=\frac{1}{3} \times \frac{6}{25} + \frac{2}{3} \times \frac{5}{50}\\
&=\frac{11}{75}
\end{align}

\newpage

%\tableofcontents

\bigskip

\renewcommand{\thefigure}{\theenumi}
\renewcommand{\thetable}{\theenumi}
%\renewcommand{\theequation}{\theenumi}

%\begin{abstract}
%%\boldmath
%In this letter, an algorithm for evaluating the exact analytical bit error rate  (BER)  for the piecewise linear (PL) combiner for  multiple relays is presented. Previous results were available only for upto three relays. The algorithm is unique in the sense that  the actual mathematical expressions, that are prohibitively large, need not be explicitly obtained. The diversity gain due to multiple relays is shown through plots of the analytical BER, well supported by simulations. 
%
%\end{abstract}
% IEEEtran.cls defaults to using nonbold math in the Abstract.
% This preserves the distinction between vectors and scalars. However,
% if the journal you are submitting to favors bold math in the abstract,
% then you can use LaTeX's standard command \boldmath at the very start
% of the abstract to achieve this. Many IEEE journals frown on math
% in the abstract anyway.

% Note that keywords are not normally used for peerreview papers.
%\begin{IEEEkeywords}
%Cooperative diversity, decode and forward, piecewise linear
%\end{IEEEkeywords}



% For peer review papers, you can put extra information on the cover
% page as needed:
% \ifCLASSOPTIONpeerreview
% \begin{center} \bfseries EDICS Category: 3-BBND \end{center}
% \fi
%
% For peerreview papers, this IEEEtran command inserts a page break and
% creates the second title. It will be ignored for other modes.
%\IEEEpeerreviewmaketitle




\item
If 4-digit numbers greater than 5,000 are randomly formed from the digits 0, 1, 3, 5, and 7, what is the probability of forming a number divisible by 5 when:
\begin{enumerate}
    \item The digits are repeated?
    \item The repetition of digits is not allowed?
\end{enumerate}
\solution
%\begin{table}[H]
	\centering
\begin{tabular}{|c|c|c|}
\hline
Random variable &Value &Definition\\ \hline
\multirow{3}{*}{X} &0 &Slips of Rs 1\\
&1 &Slips of Rs 5\\
&2 &Slips of Rs 13\\ \hline
\multirow{2}{*}{Y} &0 &Box A\\
&1 &Box B\\\hline
\end{tabular}
\caption{}
\label{tab:Distribution}
\end{table}
See \tabref{tab:Distribution}.
\begin{align}
p_{Y}\brak{k}= \begin{cases} 
      \frac{1}{3} & {k=0} \\
      \frac{2}{3 }& {k=1} 
   \end{cases}
   \\
p_{Y|X}\brak{0|0} = \frac{19}{25}\, 
p_{Y|X}\brak{0|1} = \frac{6}{25}\,
p_{Y|X}\brak{1|0} = \frac{45}{50}\,
p_{Y|X}\brak{1|2} = \frac{5}{50}
\end{align}
The desired probability is the probability that a slip drawn at random is marked other than Rs 1,
\begin{align}
&=1-p_X\brak{0}\\
&= p_X(1) + p_X(2)
\end{align}
Using Bayes theorem,
\begin{align}
&= p_Y\brak{0} \times \pr{Y=0 | X=1} + p_Y\brak{1} \times \pr{Y=1|X=2}\\
&=\frac{1}{3} \times \frac{6}{25} + \frac{2}{3} \times \frac{5}{50}\\
&=\frac{11}{75}
\end{align}

\newpage

%\tableofcontents

\bigskip

\renewcommand{\thefigure}{\theenumi}
\renewcommand{\thetable}{\theenumi}
%\renewcommand{\theequation}{\theenumi}

%\begin{abstract}
%%\boldmath
%In this letter, an algorithm for evaluating the exact analytical bit error rate  (BER)  for the piecewise linear (PL) combiner for  multiple relays is presented. Previous results were available only for upto three relays. The algorithm is unique in the sense that  the actual mathematical expressions, that are prohibitively large, need not be explicitly obtained. The diversity gain due to multiple relays is shown through plots of the analytical BER, well supported by simulations. 
%
%\end{abstract}
% IEEEtran.cls defaults to using nonbold math in the Abstract.
% This preserves the distinction between vectors and scalars. However,
% if the journal you are submitting to favors bold math in the abstract,
% then you can use LaTeX's standard command \boldmath at the very start
% of the abstract to achieve this. Many IEEE journals frown on math
% in the abstract anyway.

% Note that keywords are not normally used for peerreview papers.
%\begin{IEEEkeywords}
%Cooperative diversity, decode and forward, piecewise linear
%\end{IEEEkeywords}



% For peer review papers, you can put extra information on the cover
% page as needed:
% \ifCLASSOPTIONpeerreview
% \begin{center} \bfseries EDICS Category: 3-BBND \end{center}
% \fi
%
% For peerreview papers, this IEEEtran command inserts a page break and
% creates the second title. It will be ignored for other modes.
%\IEEEpeerreviewmaketitle




\item Consider the probability space $\brak{\Omega, \mathcal{G}, P}$ where $\Omega = [0,2]$ and $\mathcal{G} = \cbrak{\phi, \Omega, [0,1], (1,2]}$. Let $X$ and $Y$ be two functions on $\Omega$ defined as
\begin{align*}
    X(\omega) = 
    \begin{cases}
        1 & \text{if }\omega \in [0, 1]\\
        2 & \text{if }\omega \in (1, 2]
    \end{cases}
\end{align*}
and
\begin{align*}
    Y(\omega) = 
    \begin{cases}
        2 & \text{if }\omega \in [0, 1.5]\\
        3 & \text{if }\omega \in (1.5, 2].
    \end{cases}
\end{align*}
Then which one of the following statements is true?
\begin{enumerate}
    \item [(A)] $X$ is a random variable with respect to $\mathcal{G}$, but $Y$ is not a random variable with respect to $\mathcal{G}$.
    \item [(B)] $Y$ is a random variable with respect to $\mathcal{G}$, but $X$ is not a random variable with respect to $\mathcal{G}$.
    \item [(C)] Neither $X$ nor $Y$ is a random variable with respect to $\mathcal{G}$.
    \item [(D)] Both $X$ and $Y$ are random variables with respect to $\mathcal{G}$.
\end{enumerate} \hfill (GATE ST 2023)\\
\solution
%\begin{table}[H]
	\centering
\begin{tabular}{|c|c|c|}
\hline
Random variable &Value &Definition\\ \hline
\multirow{3}{*}{X} &0 &Slips of Rs 1\\
&1 &Slips of Rs 5\\
&2 &Slips of Rs 13\\ \hline
\multirow{2}{*}{Y} &0 &Box A\\
&1 &Box B\\\hline
\end{tabular}
\caption{}
\label{tab:Distribution}
\end{table}
See \tabref{tab:Distribution}.
\begin{align}
p_{Y}\brak{k}= \begin{cases} 
      \frac{1}{3} & {k=0} \\
      \frac{2}{3 }& {k=1} 
   \end{cases}
   \\
p_{Y|X}\brak{0|0} = \frac{19}{25}\, 
p_{Y|X}\brak{0|1} = \frac{6}{25}\,
p_{Y|X}\brak{1|0} = \frac{45}{50}\,
p_{Y|X}\brak{1|2} = \frac{5}{50}
\end{align}
The desired probability is the probability that a slip drawn at random is marked other than Rs 1,
\begin{align}
&=1-p_X\brak{0}\\
&= p_X(1) + p_X(2)
\end{align}
Using Bayes theorem,
\begin{align}
&= p_Y\brak{0} \times \pr{Y=0 | X=1} + p_Y\brak{1} \times \pr{Y=1|X=2}\\
&=\frac{1}{3} \times \frac{6}{25} + \frac{2}{3} \times \frac{5}{50}\\
&=\frac{11}{75}
\end{align}

\newpage

%\tableofcontents

\bigskip

\renewcommand{\thefigure}{\theenumi}
\renewcommand{\thetable}{\theenumi}
%\renewcommand{\theequation}{\theenumi}

%\begin{abstract}
%%\boldmath
%In this letter, an algorithm for evaluating the exact analytical bit error rate  (BER)  for the piecewise linear (PL) combiner for  multiple relays is presented. Previous results were available only for upto three relays. The algorithm is unique in the sense that  the actual mathematical expressions, that are prohibitively large, need not be explicitly obtained. The diversity gain due to multiple relays is shown through plots of the analytical BER, well supported by simulations. 
%
%\end{abstract}
% IEEEtran.cls defaults to using nonbold math in the Abstract.
% This preserves the distinction between vectors and scalars. However,
% if the journal you are submitting to favors bold math in the abstract,
% then you can use LaTeX's standard command \boldmath at the very start
% of the abstract to achieve this. Many IEEE journals frown on math
% in the abstract anyway.

% Note that keywords are not normally used for peerreview papers.
%\begin{IEEEkeywords}
%Cooperative diversity, decode and forward, piecewise linear
%\end{IEEEkeywords}



% For peer review papers, you can put extra information on the cover
% page as needed:
% \ifCLASSOPTIONpeerreview
% \begin{center} \bfseries EDICS Category: 3-BBND \end{center}
% \fi
%
% For peerreview papers, this IEEEtran command inserts a page break and
% creates the second title. It will be ignored for other modes.
%\IEEEpeerreviewmaketitle




	\item  A die is loaded in such a way that each odd number is twice as likely to occur as
each even number. Find $P(G)$, where $G$ is the event that a number greater than
3 occurs on a single roll of the die.
\\
\solution
		%\begin{table}[H]
	\centering
\begin{tabular}{|c|c|c|}
\hline
Random variable &Value &Definition\\ \hline
\multirow{3}{*}{X} &0 &Slips of Rs 1\\
&1 &Slips of Rs 5\\
&2 &Slips of Rs 13\\ \hline
\multirow{2}{*}{Y} &0 &Box A\\
&1 &Box B\\\hline
\end{tabular}
\caption{}
\label{tab:Distribution}
\end{table}
See \tabref{tab:Distribution}.
\begin{align}
p_{Y}\brak{k}= \begin{cases} 
      \frac{1}{3} & {k=0} \\
      \frac{2}{3 }& {k=1} 
   \end{cases}
   \\
p_{Y|X}\brak{0|0} = \frac{19}{25}\, 
p_{Y|X}\brak{0|1} = \frac{6}{25}\,
p_{Y|X}\brak{1|0} = \frac{45}{50}\,
p_{Y|X}\brak{1|2} = \frac{5}{50}
\end{align}
The desired probability is the probability that a slip drawn at random is marked other than Rs 1,
\begin{align}
&=1-p_X\brak{0}\\
&= p_X(1) + p_X(2)
\end{align}
Using Bayes theorem,
\begin{align}
&= p_Y\brak{0} \times \pr{Y=0 | X=1} + p_Y\brak{1} \times \pr{Y=1|X=2}\\
&=\frac{1}{3} \times \frac{6}{25} + \frac{2}{3} \times \frac{5}{50}\\
&=\frac{11}{75}
\end{align}

\newpage

%\tableofcontents

\bigskip

\renewcommand{\thefigure}{\theenumi}
\renewcommand{\thetable}{\theenumi}
%\renewcommand{\theequation}{\theenumi}

%\begin{abstract}
%%\boldmath
%In this letter, an algorithm for evaluating the exact analytical bit error rate  (BER)  for the piecewise linear (PL) combiner for  multiple relays is presented. Previous results were available only for upto three relays. The algorithm is unique in the sense that  the actual mathematical expressions, that are prohibitively large, need not be explicitly obtained. The diversity gain due to multiple relays is shown through plots of the analytical BER, well supported by simulations. 
%
%\end{abstract}
% IEEEtran.cls defaults to using nonbold math in the Abstract.
% This preserves the distinction between vectors and scalars. However,
% if the journal you are submitting to favors bold math in the abstract,
% then you can use LaTeX's standard command \boldmath at the very start
% of the abstract to achieve this. Many IEEE journals frown on math
% in the abstract anyway.

% Note that keywords are not normally used for peerreview papers.
%\begin{IEEEkeywords}
%Cooperative diversity, decode and forward, piecewise linear
%\end{IEEEkeywords}



% For peer review papers, you can put extra information on the cover
% page as needed:
% \ifCLASSOPTIONpeerreview
% \begin{center} \bfseries EDICS Category: 3-BBND \end{center}
% \fi
%
% For peerreview papers, this IEEEtran command inserts a page break and
% creates the second title. It will be ignored for other modes.
%\IEEEpeerreviewmaketitle




	\item All the jacks, queens and kings are removed from a deck of 52 playing cards. The remaining cards are well shuffled and then one card is drawn at random. Giving ace a value 1 similar value for other cards, find the probability that the card has a value 
		\begin{enumerate}
			\item 7
			\item greater than 7
			\item less than 7
		\end{enumerate}
		%Number of cards left after removing all jacks, queens and kings 
\begin{align}
N	= 52 - 4\times 3
	= 40
\end{align}
%\begin{table}[H]
%\def\arraystretch{1.2}
%\begin{tabular}{|c|c|c|}
%\hline
%	\textbf{Parameter} &\textbf{Value} &\textbf{Description}\\ \hline
%	$X$ &1-10 &Represents the value of the card picked \\ \hline
%\end{tabular}
%\end{table}
Let $1 \le X \le 10$ be the value of the card picked.  Then,
\begin{align}
	p_X(k) &= \Pr(X=k)\ \forall\ 1 \leq k \leq 10\\
	&= \frac{4\times 1}{40}\\
	&= \frac{1}{10}\\
	\therefore p_X(k) &= 
	\begin{cases}
		\frac{1}{10} & 1 \leq k \leq 10\\
		0 & \text{otherwise}
	\end{cases}
\end{align}
and
\begin{align}
	F_{X}(k) &= \sum_{m=0}^{k}p_{X}(m) \quad 1 \leq k \leq 10\\
	&= \frac{k}{10}\\
	\therefore F_{X}(k) &= 
	\begin{cases}
		0 & k \leq 0\\
		\frac{k}{10} & 1\leq k \leq 10\\
		1 & k > 10 
	\end{cases}
\end{align}
\begin{enumerate}
	\item Probability that card has value equal to 7 is
		\begin{align}
			 p_{X}(7)
			= \frac{1}{10}
		\end{align}
	\item Probability that card has value greater than 7 is
		\begin{align}
			1 - F_X(7)
			&= 1 - \frac{7}{10}
			\\
			&= \frac{3}{10}
		\end{align}
	\item Probability that card has value less than 7 is
		\begin{align}
			 F_{X}(6)
			=\frac{6}{10}
		\end{align}
\end{enumerate}

  \item A Lot consists of 48 mobile phones of which 42 are good, 3 have only minor defects and 3 have major defects.Varnika will buy a phone if it is good but the trader will only buy a mobile if it has no major defects. One phone is selected at random from the lot. What is the probability that it is
\begin{enumerate}
	\item acceptable to Varnika?
            \item acceptable to the trader?
\end{enumerate}
\solution
	%\begin{table}[H]
	\centering
\begin{tabular}{|c|c|c|}
\hline
Random variable &Value &Definition\\ \hline
\multirow{3}{*}{X} &0 &Slips of Rs 1\\
&1 &Slips of Rs 5\\
&2 &Slips of Rs 13\\ \hline
\multirow{2}{*}{Y} &0 &Box A\\
&1 &Box B\\\hline
\end{tabular}
\caption{}
\label{tab:Distribution}
\end{table}
See \tabref{tab:Distribution}.
\begin{align}
p_{Y}\brak{k}= \begin{cases} 
      \frac{1}{3} & {k=0} \\
      \frac{2}{3 }& {k=1} 
   \end{cases}
   \\
p_{Y|X}\brak{0|0} = \frac{19}{25}\, 
p_{Y|X}\brak{0|1} = \frac{6}{25}\,
p_{Y|X}\brak{1|0} = \frac{45}{50}\,
p_{Y|X}\brak{1|2} = \frac{5}{50}
\end{align}
The desired probability is the probability that a slip drawn at random is marked other than Rs 1,
\begin{align}
&=1-p_X\brak{0}\\
&= p_X(1) + p_X(2)
\end{align}
Using Bayes theorem,
\begin{align}
&= p_Y\brak{0} \times \pr{Y=0 | X=1} + p_Y\brak{1} \times \pr{Y=1|X=2}\\
&=\frac{1}{3} \times \frac{6}{25} + \frac{2}{3} \times \frac{5}{50}\\
&=\frac{11}{75}
\end{align}

\newpage

%\tableofcontents

\bigskip

\renewcommand{\thefigure}{\theenumi}
\renewcommand{\thetable}{\theenumi}
%\renewcommand{\theequation}{\theenumi}

%\begin{abstract}
%%\boldmath
%In this letter, an algorithm for evaluating the exact analytical bit error rate  (BER)  for the piecewise linear (PL) combiner for  multiple relays is presented. Previous results were available only for upto three relays. The algorithm is unique in the sense that  the actual mathematical expressions, that are prohibitively large, need not be explicitly obtained. The diversity gain due to multiple relays is shown through plots of the analytical BER, well supported by simulations. 
%
%\end{abstract}
% IEEEtran.cls defaults to using nonbold math in the Abstract.
% This preserves the distinction between vectors and scalars. However,
% if the journal you are submitting to favors bold math in the abstract,
% then you can use LaTeX's standard command \boldmath at the very start
% of the abstract to achieve this. Many IEEE journals frown on math
% in the abstract anyway.

% Note that keywords are not normally used for peerreview papers.
%\begin{IEEEkeywords}
%Cooperative diversity, decode and forward, piecewise linear
%\end{IEEEkeywords}



% For peer review papers, you can put extra information on the cover
% page as needed:
% \ifCLASSOPTIONpeerreview
% \begin{center} \bfseries EDICS Category: 3-BBND \end{center}
% \fi
%
% For peerreview papers, this IEEEtran command inserts a page break and
% creates the second title. It will be ignored for other modes.
%\IEEEpeerreviewmaketitle




 \item A student says that if you throw a die, it will show up 1 or not 1. Therefore, the probability of getting 1 and the probability of getting 'not 1' each is equal to $\frac{1}{2}$. Is this correct? Give reasons.\\
 \solution
        %\begin{table}[H]
	\centering
\begin{tabular}{|c|c|c|}
\hline
Random variable &Value &Definition\\ \hline
\multirow{3}{*}{X} &0 &Slips of Rs 1\\
&1 &Slips of Rs 5\\
&2 &Slips of Rs 13\\ \hline
\multirow{2}{*}{Y} &0 &Box A\\
&1 &Box B\\\hline
\end{tabular}
\caption{}
\label{tab:Distribution}
\end{table}
See \tabref{tab:Distribution}.
\begin{align}
p_{Y}\brak{k}= \begin{cases} 
      \frac{1}{3} & {k=0} \\
      \frac{2}{3 }& {k=1} 
   \end{cases}
   \\
p_{Y|X}\brak{0|0} = \frac{19}{25}\, 
p_{Y|X}\brak{0|1} = \frac{6}{25}\,
p_{Y|X}\brak{1|0} = \frac{45}{50}\,
p_{Y|X}\brak{1|2} = \frac{5}{50}
\end{align}
The desired probability is the probability that a slip drawn at random is marked other than Rs 1,
\begin{align}
&=1-p_X\brak{0}\\
&= p_X(1) + p_X(2)
\end{align}
Using Bayes theorem,
\begin{align}
&= p_Y\brak{0} \times \pr{Y=0 | X=1} + p_Y\brak{1} \times \pr{Y=1|X=2}\\
&=\frac{1}{3} \times \frac{6}{25} + \frac{2}{3} \times \frac{5}{50}\\
&=\frac{11}{75}
\end{align}

\newpage

%\tableofcontents

\bigskip

\renewcommand{\thefigure}{\theenumi}
\renewcommand{\thetable}{\theenumi}
%\renewcommand{\theequation}{\theenumi}

%\begin{abstract}
%%\boldmath
%In this letter, an algorithm for evaluating the exact analytical bit error rate  (BER)  for the piecewise linear (PL) combiner for  multiple relays is presented. Previous results were available only for upto three relays. The algorithm is unique in the sense that  the actual mathematical expressions, that are prohibitively large, need not be explicitly obtained. The diversity gain due to multiple relays is shown through plots of the analytical BER, well supported by simulations. 
%
%\end{abstract}
% IEEEtran.cls defaults to using nonbold math in the Abstract.
% This preserves the distinction between vectors and scalars. However,
% if the journal you are submitting to favors bold math in the abstract,
% then you can use LaTeX's standard command \boldmath at the very start
% of the abstract to achieve this. Many IEEE journals frown on math
% in the abstract anyway.

% Note that keywords are not normally used for peerreview papers.
%\begin{IEEEkeywords}
%Cooperative diversity, decode and forward, piecewise linear
%\end{IEEEkeywords}



% For peer review papers, you can put extra information on the cover
% page as needed:
% \ifCLASSOPTIONpeerreview
% \begin{center} \bfseries EDICS Category: 3-BBND \end{center}
% \fi
%
% For peerreview papers, this IEEEtran command inserts a page break and
% creates the second title. It will be ignored for other modes.
%\IEEEpeerreviewmaketitle




   \item Four candidates A, B, C, D have ap-
plied for the assignment to coach a school cricket
team. If A is twice as likely to be selected as B, and
B and C are given about the same chance of being
selected, while C is twice as likely to be selected
as D, what are the probabilities that
\begin{enumerate}
\item C will be selected?
\item A will not be selected?
\end{enumerate}
	%\begin{table}[H]
	\centering
\begin{tabular}{|c|c|c|}
\hline
Random variable &Value &Definition\\ \hline
\multirow{3}{*}{X} &0 &Slips of Rs 1\\
&1 &Slips of Rs 5\\
&2 &Slips of Rs 13\\ \hline
\multirow{2}{*}{Y} &0 &Box A\\
&1 &Box B\\\hline
\end{tabular}
\caption{}
\label{tab:Distribution}
\end{table}
See \tabref{tab:Distribution}.
\begin{align}
p_{Y}\brak{k}= \begin{cases} 
      \frac{1}{3} & {k=0} \\
      \frac{2}{3 }& {k=1} 
   \end{cases}
   \\
p_{Y|X}\brak{0|0} = \frac{19}{25}\, 
p_{Y|X}\brak{0|1} = \frac{6}{25}\,
p_{Y|X}\brak{1|0} = \frac{45}{50}\,
p_{Y|X}\brak{1|2} = \frac{5}{50}
\end{align}
The desired probability is the probability that a slip drawn at random is marked other than Rs 1,
\begin{align}
&=1-p_X\brak{0}\\
&= p_X(1) + p_X(2)
\end{align}
Using Bayes theorem,
\begin{align}
&= p_Y\brak{0} \times \pr{Y=0 | X=1} + p_Y\brak{1} \times \pr{Y=1|X=2}\\
&=\frac{1}{3} \times \frac{6}{25} + \frac{2}{3} \times \frac{5}{50}\\
&=\frac{11}{75}
\end{align}

\newpage

%\tableofcontents

\bigskip

\renewcommand{\thefigure}{\theenumi}
\renewcommand{\thetable}{\theenumi}
%\renewcommand{\theequation}{\theenumi}

%\begin{abstract}
%%\boldmath
%In this letter, an algorithm for evaluating the exact analytical bit error rate  (BER)  for the piecewise linear (PL) combiner for  multiple relays is presented. Previous results were available only for upto three relays. The algorithm is unique in the sense that  the actual mathematical expressions, that are prohibitively large, need not be explicitly obtained. The diversity gain due to multiple relays is shown through plots of the analytical BER, well supported by simulations. 
%
%\end{abstract}
% IEEEtran.cls defaults to using nonbold math in the Abstract.
% This preserves the distinction between vectors and scalars. However,
% if the journal you are submitting to favors bold math in the abstract,
% then you can use LaTeX's standard command \boldmath at the very start
% of the abstract to achieve this. Many IEEE journals frown on math
% in the abstract anyway.

% Note that keywords are not normally used for peerreview papers.
%\begin{IEEEkeywords}
%Cooperative diversity, decode and forward, piecewise linear
%\end{IEEEkeywords}



% For peer review papers, you can put extra information on the cover
% page as needed:
% \ifCLASSOPTIONpeerreview
% \begin{center} \bfseries EDICS Category: 3-BBND \end{center}
% \fi
%
% For peerreview papers, this IEEEtran command inserts a page break and
% creates the second title. It will be ignored for other modes.
%\IEEEpeerreviewmaketitle




 \item A bag contain 24 balls of which $x$ balls are red, $2x$ are white and $3x$ are blue. A ball is selected at random, What is the probability that it is
\begin{enumerate}[label=\alph*)]
\item not red ?
\item white ?
\end{enumerate}
%\begin{table}[H]
	\centering
\begin{tabular}{|c|c|c|}
\hline
Random variable &Value &Definition\\ \hline
\multirow{3}{*}{X} &0 &Slips of Rs 1\\
&1 &Slips of Rs 5\\
&2 &Slips of Rs 13\\ \hline
\multirow{2}{*}{Y} &0 &Box A\\
&1 &Box B\\\hline
\end{tabular}
\caption{}
\label{tab:Distribution}
\end{table}
See \tabref{tab:Distribution}.
\begin{align}
p_{Y}\brak{k}= \begin{cases} 
      \frac{1}{3} & {k=0} \\
      \frac{2}{3 }& {k=1} 
   \end{cases}
   \\
p_{Y|X}\brak{0|0} = \frac{19}{25}\, 
p_{Y|X}\brak{0|1} = \frac{6}{25}\,
p_{Y|X}\brak{1|0} = \frac{45}{50}\,
p_{Y|X}\brak{1|2} = \frac{5}{50}
\end{align}
The desired probability is the probability that a slip drawn at random is marked other than Rs 1,
\begin{align}
&=1-p_X\brak{0}\\
&= p_X(1) + p_X(2)
\end{align}
Using Bayes theorem,
\begin{align}
&= p_Y\brak{0} \times \pr{Y=0 | X=1} + p_Y\brak{1} \times \pr{Y=1|X=2}\\
&=\frac{1}{3} \times \frac{6}{25} + \frac{2}{3} \times \frac{5}{50}\\
&=\frac{11}{75}
\end{align}

\newpage

%\tableofcontents

\bigskip

\renewcommand{\thefigure}{\theenumi}
\renewcommand{\thetable}{\theenumi}
%\renewcommand{\theequation}{\theenumi}

%\begin{abstract}
%%\boldmath
%In this letter, an algorithm for evaluating the exact analytical bit error rate  (BER)  for the piecewise linear (PL) combiner for  multiple relays is presented. Previous results were available only for upto three relays. The algorithm is unique in the sense that  the actual mathematical expressions, that are prohibitively large, need not be explicitly obtained. The diversity gain due to multiple relays is shown through plots of the analytical BER, well supported by simulations. 
%
%\end{abstract}
% IEEEtran.cls defaults to using nonbold math in the Abstract.
% This preserves the distinction between vectors and scalars. However,
% if the journal you are submitting to favors bold math in the abstract,
% then you can use LaTeX's standard command \boldmath at the very start
% of the abstract to achieve this. Many IEEE journals frown on math
% in the abstract anyway.

% Note that keywords are not normally used for peerreview papers.
%\begin{IEEEkeywords}
%Cooperative diversity, decode and forward, piecewise linear
%\end{IEEEkeywords}



% For peer review papers, you can put extra information on the cover
% page as needed:
% \ifCLASSOPTIONpeerreview
% \begin{center} \bfseries EDICS Category: 3-BBND \end{center}
% \fi
%
% For peerreview papers, this IEEEtran command inserts a page break and
% creates the second title. It will be ignored for other modes.
%\IEEEpeerreviewmaketitle




If the letters of the word ASSASSINATION are arranged at random. Find the Probability that
\begin{enumerate}[label=(\alph*)]
\item Four $S's$ come consecutively in the word
\item Two  $I's$ and two $N's$ come together
\item All $A's$ are not coming together
\item No two $A's$ are coming together
\end{enumerate}
%\begin{table}[H]
	\centering
\begin{tabular}{|c|c|c|}
\hline
Random variable &Value &Definition\\ \hline
\multirow{3}{*}{X} &0 &Slips of Rs 1\\
&1 &Slips of Rs 5\\
&2 &Slips of Rs 13\\ \hline
\multirow{2}{*}{Y} &0 &Box A\\
&1 &Box B\\\hline
\end{tabular}
\caption{}
\label{tab:Distribution}
\end{table}
See \tabref{tab:Distribution}.
\begin{align}
p_{Y}\brak{k}= \begin{cases} 
      \frac{1}{3} & {k=0} \\
      \frac{2}{3 }& {k=1} 
   \end{cases}
   \\
p_{Y|X}\brak{0|0} = \frac{19}{25}\, 
p_{Y|X}\brak{0|1} = \frac{6}{25}\,
p_{Y|X}\brak{1|0} = \frac{45}{50}\,
p_{Y|X}\brak{1|2} = \frac{5}{50}
\end{align}
The desired probability is the probability that a slip drawn at random is marked other than Rs 1,
\begin{align}
&=1-p_X\brak{0}\\
&= p_X(1) + p_X(2)
\end{align}
Using Bayes theorem,
\begin{align}
&= p_Y\brak{0} \times \pr{Y=0 | X=1} + p_Y\brak{1} \times \pr{Y=1|X=2}\\
&=\frac{1}{3} \times \frac{6}{25} + \frac{2}{3} \times \frac{5}{50}\\
&=\frac{11}{75}
\end{align}

\newpage

%\tableofcontents

\bigskip

\renewcommand{\thefigure}{\theenumi}
\renewcommand{\thetable}{\theenumi}
%\renewcommand{\theequation}{\theenumi}

%\begin{abstract}
%%\boldmath
%In this letter, an algorithm for evaluating the exact analytical bit error rate  (BER)  for the piecewise linear (PL) combiner for  multiple relays is presented. Previous results were available only for upto three relays. The algorithm is unique in the sense that  the actual mathematical expressions, that are prohibitively large, need not be explicitly obtained. The diversity gain due to multiple relays is shown through plots of the analytical BER, well supported by simulations. 
%
%\end{abstract}
% IEEEtran.cls defaults to using nonbold math in the Abstract.
% This preserves the distinction between vectors and scalars. However,
% if the journal you are submitting to favors bold math in the abstract,
% then you can use LaTeX's standard command \boldmath at the very start
% of the abstract to achieve this. Many IEEE journals frown on math
% in the abstract anyway.

% Note that keywords are not normally used for peerreview papers.
%\begin{IEEEkeywords}
%Cooperative diversity, decode and forward, piecewise linear
%\end{IEEEkeywords}



% For peer review papers, you can put extra information on the cover
% page as needed:
% \ifCLASSOPTIONpeerreview
% \begin{center} \bfseries EDICS Category: 3-BBND \end{center}
% \fi
%
% For peerreview papers, this IEEEtran command inserts a page break and
% creates the second title. It will be ignored for other modes.
%\IEEEpeerreviewmaketitle




	\item One urn contains two black balls (labelled B1 and B2) and one white ball. A
	second urn contains one black ball and two white balls (labelled W1 and W2).
	Suppose the following experiment is performed. One of the two urns is chosen
	at random. Next a ball is randomly chosen from the urn. Then a second ball is
	chosen at random from the same urn without replacing the first ball.
	
	\begin{enumerate}
	\item What is the probability that two black balls are chosen?
	
	\item What is the probability that two balls of opposite colour are chosen?
	\end{enumerate}
	\solution
	%\begin{align}
    \label{eq:12.13.6.18.1}
	\because	\pr{A|B} &> \pr{A},\
\frac{\pr{AB}}{\pr{B}} > \pr{A}
\\
    \label{eq:12.13.6.18.2}
	\implies \pr{AB} &> \pr{A}\pr{B}
	\\
	\text{or, } \frac{\pr{AB}}{\pr{A}} &=\pr{B|A} > \pr{A}
\end{align}

\end{enumerate}

		%
\item 
Two cards are drawn at random and without replacement from a pack of 52 playing cards. Find the probability that both the cards are black.
\\
\solution
		%\begin{enumerate}[label=\thesection.\arabic*,ref=\thesection.\theenumi]
	\item One card is drawn from a well-shuffled deck of 52 cards. Find the probability of getting
\begin{enumerate}
\item A king of red colour 
\item A face card 
\item A red face card
\item The jack of hearts
\item A spade
\item The queen of diamonds

\end{enumerate}
\solution
		%\begin{table}[H]
	\centering
\begin{tabular}{|c|c|c|}
\hline
Random variable &Value &Definition\\ \hline
\multirow{3}{*}{X} &0 &Slips of Rs 1\\
&1 &Slips of Rs 5\\
&2 &Slips of Rs 13\\ \hline
\multirow{2}{*}{Y} &0 &Box A\\
&1 &Box B\\\hline
\end{tabular}
\caption{}
\label{tab:Distribution}
\end{table}
See \tabref{tab:Distribution}.
\begin{align}
p_{Y}\brak{k}= \begin{cases} 
      \frac{1}{3} & {k=0} \\
      \frac{2}{3 }& {k=1} 
   \end{cases}
   \\
p_{Y|X}\brak{0|0} = \frac{19}{25}\, 
p_{Y|X}\brak{0|1} = \frac{6}{25}\,
p_{Y|X}\brak{1|0} = \frac{45}{50}\,
p_{Y|X}\brak{1|2} = \frac{5}{50}
\end{align}
The desired probability is the probability that a slip drawn at random is marked other than Rs 1,
\begin{align}
&=1-p_X\brak{0}\\
&= p_X(1) + p_X(2)
\end{align}
Using Bayes theorem,
\begin{align}
&= p_Y\brak{0} \times \pr{Y=0 | X=1} + p_Y\brak{1} \times \pr{Y=1|X=2}\\
&=\frac{1}{3} \times \frac{6}{25} + \frac{2}{3} \times \frac{5}{50}\\
&=\frac{11}{75}
\end{align}

\newpage

%\tableofcontents

\bigskip

\renewcommand{\thefigure}{\theenumi}
\renewcommand{\thetable}{\theenumi}
%\renewcommand{\theequation}{\theenumi}

%\begin{abstract}
%%\boldmath
%In this letter, an algorithm for evaluating the exact analytical bit error rate  (BER)  for the piecewise linear (PL) combiner for  multiple relays is presented. Previous results were available only for upto three relays. The algorithm is unique in the sense that  the actual mathematical expressions, that are prohibitively large, need not be explicitly obtained. The diversity gain due to multiple relays is shown through plots of the analytical BER, well supported by simulations. 
%
%\end{abstract}
% IEEEtran.cls defaults to using nonbold math in the Abstract.
% This preserves the distinction between vectors and scalars. However,
% if the journal you are submitting to favors bold math in the abstract,
% then you can use LaTeX's standard command \boldmath at the very start
% of the abstract to achieve this. Many IEEE journals frown on math
% in the abstract anyway.

% Note that keywords are not normally used for peerreview papers.
%\begin{IEEEkeywords}
%Cooperative diversity, decode and forward, piecewise linear
%\end{IEEEkeywords}



% For peer review papers, you can put extra information on the cover
% page as needed:
% \ifCLASSOPTIONpeerreview
% \begin{center} \bfseries EDICS Category: 3-BBND \end{center}
% \fi
%
% For peerreview papers, this IEEEtran command inserts a page break and
% creates the second title. It will be ignored for other modes.
%\IEEEpeerreviewmaketitle




	\item Five cards—the ten, jack, queen, king and ace of diamonds, are well-shuffled with their face downwards. One card is then picked up at random.
\begin{enumerate}
\item
What is the probability that the card is the queen? 
\item
If the queen is drawn and put aside, what is the probability that the second card picked up is (a) an ace? (b) a queen?\\
\end{enumerate}
\solution
		%\begin{enumerate}[label=\thesection.\arabic*,ref=\thesection.\theenumi]
	\item One card is drawn from a well-shuffled deck of 52 cards. Find the probability of getting
\begin{enumerate}
\item A king of red colour 
\item A face card 
\item A red face card
\item The jack of hearts
\item A spade
\item The queen of diamonds

\end{enumerate}
\solution
		%\input{ncert/10/15/1/14/main.tex}
	\item Five cards—the ten, jack, queen, king and ace of diamonds, are well-shuffled with their face downwards. One card is then picked up at random.
\begin{enumerate}
\item
What is the probability that the card is the queen? 
\item
If the queen is drawn and put aside, what is the probability that the second card picked up is (a) an ace? (b) a queen?\\
\end{enumerate}
\solution
		%\input{ncert/10/15/1/15/defs.tex}
	\item A bag contains $5$ red balls and some blue balls. If the probability of drawing a blue ball is double that if a red ball, determine the number of blue balls in the bag. 
		\\
\solution
		%\input{ncert/10/15/2/3/defs.tex}
	\item A card is selected from a pack of 52 cards.
 \begin{enumerate}[label=(\alph*)] 
                 \item How many points are there in the sample space?
                 \item Calculate the probability that the card is an ace of spades.
                 \item Calculate the probability that the card is (i) an ace and (ii) black card.
 \end{enumerate}
\solution
		%\input{ncert/11/16/3/4/main.tex}
\item Four cards are drawn from a well-shuffled deck of 52 cards. What is the probability of obtaining 3 diamonds and one spade.
\\
\solution
		%\input{ncert/11/16/4/2/defs.tex}
\item In a certain lottery 10,000 tickets are sold and ten equal prizes are awarded. What is the probability of not getting a prize if you buy (a) one ticket (b) two tickets (c) 10 tickets ?	
\\
\solution
		%\input{ncert/11/16/4/4/defs.tex}
		%
\item 
Out of 100 students, two sections of 40 and 60 are formed. If you and your friend are among the 100 students, what is the probability that
\begin{enumerate}
\item you both enter the same section?
\item you both enter the different sections?
\end{enumerate}
\solution
		%\input{ncert/11/16/4/5/defs.tex}
	\item 
The number lock of a suitcase has 4 wheels each labelled with ten digits i.e. from 0 to 9.The lock opens with a sequence of four digits with no repeats.What is the probability of a person getting the right sequence to open the suitcase.
\\
\solution
		%\input{ncert/11/16/4/10/defs.tex}
		%
\item 
Two cards are drawn at random and without replacement from a pack of 52 playing cards. Find the probability that both the cards are black.
\\
\solution
		%\input{ncert/12/13/2/2/defs.tex}
		\item A box of oranges is inspected by examining three randomly selected oranges drawn without replacement. If all the three oranges are good, the box is approved for sale, otherwise, it is rejected. Find the probability that a box containing 15 oranges out of which 12 are good and 3 are bad ones will be approved for sale.
		\label{ncert/12/13/2/3/defs.tex}
		\item Two balls are drawn at random with replacement from a box containing 10 black and 8 red balls. Find the probability that
		\label{ncert/12/13/2/12}
\begin{enumerate}
\item both balls are red.
\item first ball is black and second is red.
\item one of them is black and other is red.
\end{enumerate}

\item In a hostel, 60\% of the students read Hindi newspaper, 40\% read English newspaper and 20\% read both Hindi and English newspapers. A student is selected at random.
		\label{ncert/12/13/2/15}
\begin{enumerate}
\item Find the probability that she reads neither Hindi nor English newspapers.
\item If she reads Hindi newspaper, find the probability that she reads English newspaper.
\item If she reads English newspaper, find the probability that she reads Hindi newspaper.\\
\end{enumerate}
\item The probability of obtaining an even prime number on each die, when a pair of dice is rolled is 
\begin{enumerate}
    \item $0$ 
    
    \item $\frac{1}{3}$ 
    
    \item $\frac{1}{12}$ 
    
    \item $\frac{1}{36}$ 
\end{enumerate}
\solution
		%\input{ncert/12/13/2/17/defs.tex}
	\item A bag contains 4 red and 4 black balls, another bag contains 2 red and 6 black balls. One of the two bags is selected at random and a ball is drawn from the bag which is found to be red. Find the probability that the ball is drawn from the first bag.
\\
\solution
		%\input{ncert/12/13/3/2/main.tex}
  \item
  Cards with numbers 2 to 101 are placed in a box. A card is selected at random.Find the probability that the card has
\begin{enumerate}[label=(\roman*)]
	\item an even number 
	\item a square number
\end{enumerate}
\solution
%\input{exemplar/10/13/3/32/main.tex}
\item
The king, queen and jack of clubs are removed from a deck of 52 playing cards and then well shuffled. Now one card is drawn at random from the remaining cards.  Determine the probability that the card is
\begin{enumerate}[label=(\roman*)]
\item a club
\item 10 of hearts
\end{enumerate}
\solution
%\input{exemplar/10/13/3/29/main.tex}
\item A team of medical students doing their internship have to assist during surgeries
at a city hospital. The probabilities of surgeries rated as very complex, complex,
routine, simple or very simple are respectively, 0.15, 0.20, 0.31, 0.26, .08. Find
the probabilities that a particular surgery will be rated
\begin{enumerate}
	\item complex or very complex;
	\item neither very complex nor very simple;
	\item routine or complex
	\item routine or simple
\end{enumerate}
\solution
%\input{exemplar/11/16/3/8(1)/main.tex}
\item A card is selected from a pack of 52 cards.
\begin{enumerate}[label=(\alph*)]
    \item How many points are there in the sample space?
    \item Calculate the probability that the card is an ace of spades.
    \item Calculate the probability that the card is (i) an ace and (ii) black card.
\end{enumerate}
\solution
%\input{exemplar/11/16/3/4/main2.tex}
\item The probability that a non leap year selected at random will contain 53 sundays.
\\
\solution
%\input{exemplar/10/13/1/19/main.tex}
\item One of the four persons John, Rita, Aslam or Gurpreet will be promoted next
month. Consequently the sample space consists of four elementary outcomes
S = {John promoted, Rita promoted, Aslam promoted, Gurpreet promoted}
You are told that the chances of John’s promotion is same as that of Gurpreet,
Rita’s chances of promotion are twice as likely as Johns. Aslam’s chances are
four times that of John.
\begin{enumerate}
	\item Determine
	\begin{enumerate}
		\item P (John promoted)
		\item P (Rita promoted)
		\item P (Aslam promoted)
		\item P (Gurpreet promoted)
	\end{enumerate}
	\item If A = {John promoted or Gurpreet promoted}, find P (A).
\end{enumerate}
\solution
%\input{exemplar/11/16/3/10/main.tex}
\item A card is drawn from a deck of 52 cards. Find the probability of getting a king or a heart or a red card.\\
\solution
%\input{exemplar/11/16/3/15/main.tex}
\item The probability that a student will pass his examination is 0.73, the probability of
the student getting a compartment is 0.13, and the probability that the student will
either pass or get compartment is 0.96. State True or False.\\
\solution
%\input{exemplar/11/16/3/31/main.tex}
\item A card is selected from a pack of 52 cards\\
\begin{enumerate}[label=(\alph*)]
\item How many points are there in the sample space?
\item Calculate the probability that the cards is an ace of spades.
\item Calculate the probability that the card is (i) an ace (ii)black card.\\
\end{enumerate}
%\input{ncert/11/16/3/4_1/Prob_4.tex}
\item In a non-leap year, the probability of having 53 tuesdays or 53 wednesdays is\\
\solution
%\input{exemplar/11/16/3/18/main.tex}
\item There are 1000 sealed envelopes in a box, 10 of them contain a cash prize of
Rs 100 each, 100 of them contain a cash prize of Rs 50 each and 200 of them
contain a cash prize of Rs 10 each and rest do not contain any cash prize. If they
are well shuffled and an envelope is picked up out, what is the probability that it
contains no cash prize?\\
\solution
%\input{exemplar/10/13/3/34/main.tex}
\item 
A die is thrown and a card is selected at random from a deck of 52 playing cards. The probability of getting an even number on the die and a spade card.\\
\solution
%\input{exemplar/12/13/3/78/main.tex}
\item
If 4-digit numbers greater than 5,000 are randomly formed from the digits 0, 1, 3, 5, and 7, what is the probability of forming a number divisible by 5 when:
\begin{enumerate}
    \item The digits are repeated?
    \item The repetition of digits is not allowed?
\end{enumerate}
\solution
%\input{ncert/11/16/4/9/main.tex}
\item Consider the probability space $\brak{\Omega, \mathcal{G}, P}$ where $\Omega = [0,2]$ and $\mathcal{G} = \cbrak{\phi, \Omega, [0,1], (1,2]}$. Let $X$ and $Y$ be two functions on $\Omega$ defined as
\begin{align*}
    X(\omega) = 
    \begin{cases}
        1 & \text{if }\omega \in [0, 1]\\
        2 & \text{if }\omega \in (1, 2]
    \end{cases}
\end{align*}
and
\begin{align*}
    Y(\omega) = 
    \begin{cases}
        2 & \text{if }\omega \in [0, 1.5]\\
        3 & \text{if }\omega \in (1.5, 2].
    \end{cases}
\end{align*}
Then which one of the following statements is true?
\begin{enumerate}
    \item [(A)] $X$ is a random variable with respect to $\mathcal{G}$, but $Y$ is not a random variable with respect to $\mathcal{G}$.
    \item [(B)] $Y$ is a random variable with respect to $\mathcal{G}$, but $X$ is not a random variable with respect to $\mathcal{G}$.
    \item [(C)] Neither $X$ nor $Y$ is a random variable with respect to $\mathcal{G}$.
    \item [(D)] Both $X$ and $Y$ are random variables with respect to $\mathcal{G}$.
\end{enumerate} \hfill (GATE ST 2023)\\
\solution
%\input{gate/ST/2023/14/main.tex}
	\item  A die is loaded in such a way that each odd number is twice as likely to occur as
each even number. Find $P(G)$, where $G$ is the event that a number greater than
3 occurs on a single roll of the die.
\\
\solution
		%\input{exemplar/11/16/3/5/main.tex}
	\item All the jacks, queens and kings are removed from a deck of 52 playing cards. The remaining cards are well shuffled and then one card is drawn at random. Giving ace a value 1 similar value for other cards, find the probability that the card has a value 
		\begin{enumerate}
			\item 7
			\item greater than 7
			\item less than 7
		\end{enumerate}
		%\input{exemplar/10/13/3/30/main.tex}
  \item A Lot consists of 48 mobile phones of which 42 are good, 3 have only minor defects and 3 have major defects.Varnika will buy a phone if it is good but the trader will only buy a mobile if it has no major defects. One phone is selected at random from the lot. What is the probability that it is
\begin{enumerate}
	\item acceptable to Varnika?
            \item acceptable to the trader?
\end{enumerate}
\solution
	%\input{exemplar/10/13/3/40/main.tex}
 \item A student says that if you throw a die, it will show up 1 or not 1. Therefore, the probability of getting 1 and the probability of getting 'not 1' each is equal to $\frac{1}{2}$. Is this correct? Give reasons.\\
 \solution
        %\input{exemplar/10/13/2/9/main.tex}
   \item Four candidates A, B, C, D have ap-
plied for the assignment to coach a school cricket
team. If A is twice as likely to be selected as B, and
B and C are given about the same chance of being
selected, while C is twice as likely to be selected
as D, what are the probabilities that
\begin{enumerate}
\item C will be selected?
\item A will not be selected?
\end{enumerate}
	%\input{exemplar/11/16/3/9/main.tex}
 \item A bag contain 24 balls of which $x$ balls are red, $2x$ are white and $3x$ are blue. A ball is selected at random, What is the probability that it is
\begin{enumerate}[label=\alph*)]
\item not red ?
\item white ?
\end{enumerate}
%\input{exemplar/10/13/3/41/main.tex}
If the letters of the word ASSASSINATION are arranged at random. Find the Probability that
\begin{enumerate}[label=(\alph*)]
\item Four $S's$ come consecutively in the word
\item Two  $I's$ and two $N's$ come together
\item All $A's$ are not coming together
\item No two $A's$ are coming together
\end{enumerate}
%\input{exemplar/11/16/3/14/main.tex}
	\item One urn contains two black balls (labelled B1 and B2) and one white ball. A
	second urn contains one black ball and two white balls (labelled W1 and W2).
	Suppose the following experiment is performed. One of the two urns is chosen
	at random. Next a ball is randomly chosen from the urn. Then a second ball is
	chosen at random from the same urn without replacing the first ball.
	
	\begin{enumerate}
	\item What is the probability that two black balls are chosen?
	
	\item What is the probability that two balls of opposite colour are chosen?
	\end{enumerate}
	\solution
	%\input{exemplar/11/16/3/12/main1.tex}
\end{enumerate}

	\item A bag contains $5$ red balls and some blue balls. If the probability of drawing a blue ball is double that if a red ball, determine the number of blue balls in the bag. 
		\\
\solution
		%\begin{enumerate}[label=\thesection.\arabic*,ref=\thesection.\theenumi]
	\item One card is drawn from a well-shuffled deck of 52 cards. Find the probability of getting
\begin{enumerate}
\item A king of red colour 
\item A face card 
\item A red face card
\item The jack of hearts
\item A spade
\item The queen of diamonds

\end{enumerate}
\solution
		%\input{ncert/10/15/1/14/main.tex}
	\item Five cards—the ten, jack, queen, king and ace of diamonds, are well-shuffled with their face downwards. One card is then picked up at random.
\begin{enumerate}
\item
What is the probability that the card is the queen? 
\item
If the queen is drawn and put aside, what is the probability that the second card picked up is (a) an ace? (b) a queen?\\
\end{enumerate}
\solution
		%\input{ncert/10/15/1/15/defs.tex}
	\item A bag contains $5$ red balls and some blue balls. If the probability of drawing a blue ball is double that if a red ball, determine the number of blue balls in the bag. 
		\\
\solution
		%\input{ncert/10/15/2/3/defs.tex}
	\item A card is selected from a pack of 52 cards.
 \begin{enumerate}[label=(\alph*)] 
                 \item How many points are there in the sample space?
                 \item Calculate the probability that the card is an ace of spades.
                 \item Calculate the probability that the card is (i) an ace and (ii) black card.
 \end{enumerate}
\solution
		%\input{ncert/11/16/3/4/main.tex}
\item Four cards are drawn from a well-shuffled deck of 52 cards. What is the probability of obtaining 3 diamonds and one spade.
\\
\solution
		%\input{ncert/11/16/4/2/defs.tex}
\item In a certain lottery 10,000 tickets are sold and ten equal prizes are awarded. What is the probability of not getting a prize if you buy (a) one ticket (b) two tickets (c) 10 tickets ?	
\\
\solution
		%\input{ncert/11/16/4/4/defs.tex}
		%
\item 
Out of 100 students, two sections of 40 and 60 are formed. If you and your friend are among the 100 students, what is the probability that
\begin{enumerate}
\item you both enter the same section?
\item you both enter the different sections?
\end{enumerate}
\solution
		%\input{ncert/11/16/4/5/defs.tex}
	\item 
The number lock of a suitcase has 4 wheels each labelled with ten digits i.e. from 0 to 9.The lock opens with a sequence of four digits with no repeats.What is the probability of a person getting the right sequence to open the suitcase.
\\
\solution
		%\input{ncert/11/16/4/10/defs.tex}
		%
\item 
Two cards are drawn at random and without replacement from a pack of 52 playing cards. Find the probability that both the cards are black.
\\
\solution
		%\input{ncert/12/13/2/2/defs.tex}
		\item A box of oranges is inspected by examining three randomly selected oranges drawn without replacement. If all the three oranges are good, the box is approved for sale, otherwise, it is rejected. Find the probability that a box containing 15 oranges out of which 12 are good and 3 are bad ones will be approved for sale.
		\label{ncert/12/13/2/3/defs.tex}
		\item Two balls are drawn at random with replacement from a box containing 10 black and 8 red balls. Find the probability that
		\label{ncert/12/13/2/12}
\begin{enumerate}
\item both balls are red.
\item first ball is black and second is red.
\item one of them is black and other is red.
\end{enumerate}

\item In a hostel, 60\% of the students read Hindi newspaper, 40\% read English newspaper and 20\% read both Hindi and English newspapers. A student is selected at random.
		\label{ncert/12/13/2/15}
\begin{enumerate}
\item Find the probability that she reads neither Hindi nor English newspapers.
\item If she reads Hindi newspaper, find the probability that she reads English newspaper.
\item If she reads English newspaper, find the probability that she reads Hindi newspaper.\\
\end{enumerate}
\item The probability of obtaining an even prime number on each die, when a pair of dice is rolled is 
\begin{enumerate}
    \item $0$ 
    
    \item $\frac{1}{3}$ 
    
    \item $\frac{1}{12}$ 
    
    \item $\frac{1}{36}$ 
\end{enumerate}
\solution
		%\input{ncert/12/13/2/17/defs.tex}
	\item A bag contains 4 red and 4 black balls, another bag contains 2 red and 6 black balls. One of the two bags is selected at random and a ball is drawn from the bag which is found to be red. Find the probability that the ball is drawn from the first bag.
\\
\solution
		%\input{ncert/12/13/3/2/main.tex}
  \item
  Cards with numbers 2 to 101 are placed in a box. A card is selected at random.Find the probability that the card has
\begin{enumerate}[label=(\roman*)]
	\item an even number 
	\item a square number
\end{enumerate}
\solution
%\input{exemplar/10/13/3/32/main.tex}
\item
The king, queen and jack of clubs are removed from a deck of 52 playing cards and then well shuffled. Now one card is drawn at random from the remaining cards.  Determine the probability that the card is
\begin{enumerate}[label=(\roman*)]
\item a club
\item 10 of hearts
\end{enumerate}
\solution
%\input{exemplar/10/13/3/29/main.tex}
\item A team of medical students doing their internship have to assist during surgeries
at a city hospital. The probabilities of surgeries rated as very complex, complex,
routine, simple or very simple are respectively, 0.15, 0.20, 0.31, 0.26, .08. Find
the probabilities that a particular surgery will be rated
\begin{enumerate}
	\item complex or very complex;
	\item neither very complex nor very simple;
	\item routine or complex
	\item routine or simple
\end{enumerate}
\solution
%\input{exemplar/11/16/3/8(1)/main.tex}
\item A card is selected from a pack of 52 cards.
\begin{enumerate}[label=(\alph*)]
    \item How many points are there in the sample space?
    \item Calculate the probability that the card is an ace of spades.
    \item Calculate the probability that the card is (i) an ace and (ii) black card.
\end{enumerate}
\solution
%\input{exemplar/11/16/3/4/main2.tex}
\item The probability that a non leap year selected at random will contain 53 sundays.
\\
\solution
%\input{exemplar/10/13/1/19/main.tex}
\item One of the four persons John, Rita, Aslam or Gurpreet will be promoted next
month. Consequently the sample space consists of four elementary outcomes
S = {John promoted, Rita promoted, Aslam promoted, Gurpreet promoted}
You are told that the chances of John’s promotion is same as that of Gurpreet,
Rita’s chances of promotion are twice as likely as Johns. Aslam’s chances are
four times that of John.
\begin{enumerate}
	\item Determine
	\begin{enumerate}
		\item P (John promoted)
		\item P (Rita promoted)
		\item P (Aslam promoted)
		\item P (Gurpreet promoted)
	\end{enumerate}
	\item If A = {John promoted or Gurpreet promoted}, find P (A).
\end{enumerate}
\solution
%\input{exemplar/11/16/3/10/main.tex}
\item A card is drawn from a deck of 52 cards. Find the probability of getting a king or a heart or a red card.\\
\solution
%\input{exemplar/11/16/3/15/main.tex}
\item The probability that a student will pass his examination is 0.73, the probability of
the student getting a compartment is 0.13, and the probability that the student will
either pass or get compartment is 0.96. State True or False.\\
\solution
%\input{exemplar/11/16/3/31/main.tex}
\item A card is selected from a pack of 52 cards\\
\begin{enumerate}[label=(\alph*)]
\item How many points are there in the sample space?
\item Calculate the probability that the cards is an ace of spades.
\item Calculate the probability that the card is (i) an ace (ii)black card.\\
\end{enumerate}
%\input{ncert/11/16/3/4_1/Prob_4.tex}
\item In a non-leap year, the probability of having 53 tuesdays or 53 wednesdays is\\
\solution
%\input{exemplar/11/16/3/18/main.tex}
\item There are 1000 sealed envelopes in a box, 10 of them contain a cash prize of
Rs 100 each, 100 of them contain a cash prize of Rs 50 each and 200 of them
contain a cash prize of Rs 10 each and rest do not contain any cash prize. If they
are well shuffled and an envelope is picked up out, what is the probability that it
contains no cash prize?\\
\solution
%\input{exemplar/10/13/3/34/main.tex}
\item 
A die is thrown and a card is selected at random from a deck of 52 playing cards. The probability of getting an even number on the die and a spade card.\\
\solution
%\input{exemplar/12/13/3/78/main.tex}
\item
If 4-digit numbers greater than 5,000 are randomly formed from the digits 0, 1, 3, 5, and 7, what is the probability of forming a number divisible by 5 when:
\begin{enumerate}
    \item The digits are repeated?
    \item The repetition of digits is not allowed?
\end{enumerate}
\solution
%\input{ncert/11/16/4/9/main.tex}
\item Consider the probability space $\brak{\Omega, \mathcal{G}, P}$ where $\Omega = [0,2]$ and $\mathcal{G} = \cbrak{\phi, \Omega, [0,1], (1,2]}$. Let $X$ and $Y$ be two functions on $\Omega$ defined as
\begin{align*}
    X(\omega) = 
    \begin{cases}
        1 & \text{if }\omega \in [0, 1]\\
        2 & \text{if }\omega \in (1, 2]
    \end{cases}
\end{align*}
and
\begin{align*}
    Y(\omega) = 
    \begin{cases}
        2 & \text{if }\omega \in [0, 1.5]\\
        3 & \text{if }\omega \in (1.5, 2].
    \end{cases}
\end{align*}
Then which one of the following statements is true?
\begin{enumerate}
    \item [(A)] $X$ is a random variable with respect to $\mathcal{G}$, but $Y$ is not a random variable with respect to $\mathcal{G}$.
    \item [(B)] $Y$ is a random variable with respect to $\mathcal{G}$, but $X$ is not a random variable with respect to $\mathcal{G}$.
    \item [(C)] Neither $X$ nor $Y$ is a random variable with respect to $\mathcal{G}$.
    \item [(D)] Both $X$ and $Y$ are random variables with respect to $\mathcal{G}$.
\end{enumerate} \hfill (GATE ST 2023)\\
\solution
%\input{gate/ST/2023/14/main.tex}
	\item  A die is loaded in such a way that each odd number is twice as likely to occur as
each even number. Find $P(G)$, where $G$ is the event that a number greater than
3 occurs on a single roll of the die.
\\
\solution
		%\input{exemplar/11/16/3/5/main.tex}
	\item All the jacks, queens and kings are removed from a deck of 52 playing cards. The remaining cards are well shuffled and then one card is drawn at random. Giving ace a value 1 similar value for other cards, find the probability that the card has a value 
		\begin{enumerate}
			\item 7
			\item greater than 7
			\item less than 7
		\end{enumerate}
		%\input{exemplar/10/13/3/30/main.tex}
  \item A Lot consists of 48 mobile phones of which 42 are good, 3 have only minor defects and 3 have major defects.Varnika will buy a phone if it is good but the trader will only buy a mobile if it has no major defects. One phone is selected at random from the lot. What is the probability that it is
\begin{enumerate}
	\item acceptable to Varnika?
            \item acceptable to the trader?
\end{enumerate}
\solution
	%\input{exemplar/10/13/3/40/main.tex}
 \item A student says that if you throw a die, it will show up 1 or not 1. Therefore, the probability of getting 1 and the probability of getting 'not 1' each is equal to $\frac{1}{2}$. Is this correct? Give reasons.\\
 \solution
        %\input{exemplar/10/13/2/9/main.tex}
   \item Four candidates A, B, C, D have ap-
plied for the assignment to coach a school cricket
team. If A is twice as likely to be selected as B, and
B and C are given about the same chance of being
selected, while C is twice as likely to be selected
as D, what are the probabilities that
\begin{enumerate}
\item C will be selected?
\item A will not be selected?
\end{enumerate}
	%\input{exemplar/11/16/3/9/main.tex}
 \item A bag contain 24 balls of which $x$ balls are red, $2x$ are white and $3x$ are blue. A ball is selected at random, What is the probability that it is
\begin{enumerate}[label=\alph*)]
\item not red ?
\item white ?
\end{enumerate}
%\input{exemplar/10/13/3/41/main.tex}
If the letters of the word ASSASSINATION are arranged at random. Find the Probability that
\begin{enumerate}[label=(\alph*)]
\item Four $S's$ come consecutively in the word
\item Two  $I's$ and two $N's$ come together
\item All $A's$ are not coming together
\item No two $A's$ are coming together
\end{enumerate}
%\input{exemplar/11/16/3/14/main.tex}
	\item One urn contains two black balls (labelled B1 and B2) and one white ball. A
	second urn contains one black ball and two white balls (labelled W1 and W2).
	Suppose the following experiment is performed. One of the two urns is chosen
	at random. Next a ball is randomly chosen from the urn. Then a second ball is
	chosen at random from the same urn without replacing the first ball.
	
	\begin{enumerate}
	\item What is the probability that two black balls are chosen?
	
	\item What is the probability that two balls of opposite colour are chosen?
	\end{enumerate}
	\solution
	%\input{exemplar/11/16/3/12/main1.tex}
\end{enumerate}

	\item A card is selected from a pack of 52 cards.
 \begin{enumerate}[label=(\alph*)] 
                 \item How many points are there in the sample space?
                 \item Calculate the probability that the card is an ace of spades.
                 \item Calculate the probability that the card is (i) an ace and (ii) black card.
 \end{enumerate}
\solution
		%\begin{table}[H]
	\centering
\begin{tabular}{|c|c|c|}
\hline
Random variable &Value &Definition\\ \hline
\multirow{3}{*}{X} &0 &Slips of Rs 1\\
&1 &Slips of Rs 5\\
&2 &Slips of Rs 13\\ \hline
\multirow{2}{*}{Y} &0 &Box A\\
&1 &Box B\\\hline
\end{tabular}
\caption{}
\label{tab:Distribution}
\end{table}
See \tabref{tab:Distribution}.
\begin{align}
p_{Y}\brak{k}= \begin{cases} 
      \frac{1}{3} & {k=0} \\
      \frac{2}{3 }& {k=1} 
   \end{cases}
   \\
p_{Y|X}\brak{0|0} = \frac{19}{25}\, 
p_{Y|X}\brak{0|1} = \frac{6}{25}\,
p_{Y|X}\brak{1|0} = \frac{45}{50}\,
p_{Y|X}\brak{1|2} = \frac{5}{50}
\end{align}
The desired probability is the probability that a slip drawn at random is marked other than Rs 1,
\begin{align}
&=1-p_X\brak{0}\\
&= p_X(1) + p_X(2)
\end{align}
Using Bayes theorem,
\begin{align}
&= p_Y\brak{0} \times \pr{Y=0 | X=1} + p_Y\brak{1} \times \pr{Y=1|X=2}\\
&=\frac{1}{3} \times \frac{6}{25} + \frac{2}{3} \times \frac{5}{50}\\
&=\frac{11}{75}
\end{align}

\newpage

%\tableofcontents

\bigskip

\renewcommand{\thefigure}{\theenumi}
\renewcommand{\thetable}{\theenumi}
%\renewcommand{\theequation}{\theenumi}

%\begin{abstract}
%%\boldmath
%In this letter, an algorithm for evaluating the exact analytical bit error rate  (BER)  for the piecewise linear (PL) combiner for  multiple relays is presented. Previous results were available only for upto three relays. The algorithm is unique in the sense that  the actual mathematical expressions, that are prohibitively large, need not be explicitly obtained. The diversity gain due to multiple relays is shown through plots of the analytical BER, well supported by simulations. 
%
%\end{abstract}
% IEEEtran.cls defaults to using nonbold math in the Abstract.
% This preserves the distinction between vectors and scalars. However,
% if the journal you are submitting to favors bold math in the abstract,
% then you can use LaTeX's standard command \boldmath at the very start
% of the abstract to achieve this. Many IEEE journals frown on math
% in the abstract anyway.

% Note that keywords are not normally used for peerreview papers.
%\begin{IEEEkeywords}
%Cooperative diversity, decode and forward, piecewise linear
%\end{IEEEkeywords}



% For peer review papers, you can put extra information on the cover
% page as needed:
% \ifCLASSOPTIONpeerreview
% \begin{center} \bfseries EDICS Category: 3-BBND \end{center}
% \fi
%
% For peerreview papers, this IEEEtran command inserts a page break and
% creates the second title. It will be ignored for other modes.
%\IEEEpeerreviewmaketitle




\item Four cards are drawn from a well-shuffled deck of 52 cards. What is the probability of obtaining 3 diamonds and one spade.
\\
\solution
		%\begin{enumerate}[label=\thesection.\arabic*,ref=\thesection.\theenumi]
	\item One card is drawn from a well-shuffled deck of 52 cards. Find the probability of getting
\begin{enumerate}
\item A king of red colour 
\item A face card 
\item A red face card
\item The jack of hearts
\item A spade
\item The queen of diamonds

\end{enumerate}
\solution
		%\input{ncert/10/15/1/14/main.tex}
	\item Five cards—the ten, jack, queen, king and ace of diamonds, are well-shuffled with their face downwards. One card is then picked up at random.
\begin{enumerate}
\item
What is the probability that the card is the queen? 
\item
If the queen is drawn and put aside, what is the probability that the second card picked up is (a) an ace? (b) a queen?\\
\end{enumerate}
\solution
		%\input{ncert/10/15/1/15/defs.tex}
	\item A bag contains $5$ red balls and some blue balls. If the probability of drawing a blue ball is double that if a red ball, determine the number of blue balls in the bag. 
		\\
\solution
		%\input{ncert/10/15/2/3/defs.tex}
	\item A card is selected from a pack of 52 cards.
 \begin{enumerate}[label=(\alph*)] 
                 \item How many points are there in the sample space?
                 \item Calculate the probability that the card is an ace of spades.
                 \item Calculate the probability that the card is (i) an ace and (ii) black card.
 \end{enumerate}
\solution
		%\input{ncert/11/16/3/4/main.tex}
\item Four cards are drawn from a well-shuffled deck of 52 cards. What is the probability of obtaining 3 diamonds and one spade.
\\
\solution
		%\input{ncert/11/16/4/2/defs.tex}
\item In a certain lottery 10,000 tickets are sold and ten equal prizes are awarded. What is the probability of not getting a prize if you buy (a) one ticket (b) two tickets (c) 10 tickets ?	
\\
\solution
		%\input{ncert/11/16/4/4/defs.tex}
		%
\item 
Out of 100 students, two sections of 40 and 60 are formed. If you and your friend are among the 100 students, what is the probability that
\begin{enumerate}
\item you both enter the same section?
\item you both enter the different sections?
\end{enumerate}
\solution
		%\input{ncert/11/16/4/5/defs.tex}
	\item 
The number lock of a suitcase has 4 wheels each labelled with ten digits i.e. from 0 to 9.The lock opens with a sequence of four digits with no repeats.What is the probability of a person getting the right sequence to open the suitcase.
\\
\solution
		%\input{ncert/11/16/4/10/defs.tex}
		%
\item 
Two cards are drawn at random and without replacement from a pack of 52 playing cards. Find the probability that both the cards are black.
\\
\solution
		%\input{ncert/12/13/2/2/defs.tex}
		\item A box of oranges is inspected by examining three randomly selected oranges drawn without replacement. If all the three oranges are good, the box is approved for sale, otherwise, it is rejected. Find the probability that a box containing 15 oranges out of which 12 are good and 3 are bad ones will be approved for sale.
		\label{ncert/12/13/2/3/defs.tex}
		\item Two balls are drawn at random with replacement from a box containing 10 black and 8 red balls. Find the probability that
		\label{ncert/12/13/2/12}
\begin{enumerate}
\item both balls are red.
\item first ball is black and second is red.
\item one of them is black and other is red.
\end{enumerate}

\item In a hostel, 60\% of the students read Hindi newspaper, 40\% read English newspaper and 20\% read both Hindi and English newspapers. A student is selected at random.
		\label{ncert/12/13/2/15}
\begin{enumerate}
\item Find the probability that she reads neither Hindi nor English newspapers.
\item If she reads Hindi newspaper, find the probability that she reads English newspaper.
\item If she reads English newspaper, find the probability that she reads Hindi newspaper.\\
\end{enumerate}
\item The probability of obtaining an even prime number on each die, when a pair of dice is rolled is 
\begin{enumerate}
    \item $0$ 
    
    \item $\frac{1}{3}$ 
    
    \item $\frac{1}{12}$ 
    
    \item $\frac{1}{36}$ 
\end{enumerate}
\solution
		%\input{ncert/12/13/2/17/defs.tex}
	\item A bag contains 4 red and 4 black balls, another bag contains 2 red and 6 black balls. One of the two bags is selected at random and a ball is drawn from the bag which is found to be red. Find the probability that the ball is drawn from the first bag.
\\
\solution
		%\input{ncert/12/13/3/2/main.tex}
  \item
  Cards with numbers 2 to 101 are placed in a box. A card is selected at random.Find the probability that the card has
\begin{enumerate}[label=(\roman*)]
	\item an even number 
	\item a square number
\end{enumerate}
\solution
%\input{exemplar/10/13/3/32/main.tex}
\item
The king, queen and jack of clubs are removed from a deck of 52 playing cards and then well shuffled. Now one card is drawn at random from the remaining cards.  Determine the probability that the card is
\begin{enumerate}[label=(\roman*)]
\item a club
\item 10 of hearts
\end{enumerate}
\solution
%\input{exemplar/10/13/3/29/main.tex}
\item A team of medical students doing their internship have to assist during surgeries
at a city hospital. The probabilities of surgeries rated as very complex, complex,
routine, simple or very simple are respectively, 0.15, 0.20, 0.31, 0.26, .08. Find
the probabilities that a particular surgery will be rated
\begin{enumerate}
	\item complex or very complex;
	\item neither very complex nor very simple;
	\item routine or complex
	\item routine or simple
\end{enumerate}
\solution
%\input{exemplar/11/16/3/8(1)/main.tex}
\item A card is selected from a pack of 52 cards.
\begin{enumerate}[label=(\alph*)]
    \item How many points are there in the sample space?
    \item Calculate the probability that the card is an ace of spades.
    \item Calculate the probability that the card is (i) an ace and (ii) black card.
\end{enumerate}
\solution
%\input{exemplar/11/16/3/4/main2.tex}
\item The probability that a non leap year selected at random will contain 53 sundays.
\\
\solution
%\input{exemplar/10/13/1/19/main.tex}
\item One of the four persons John, Rita, Aslam or Gurpreet will be promoted next
month. Consequently the sample space consists of four elementary outcomes
S = {John promoted, Rita promoted, Aslam promoted, Gurpreet promoted}
You are told that the chances of John’s promotion is same as that of Gurpreet,
Rita’s chances of promotion are twice as likely as Johns. Aslam’s chances are
four times that of John.
\begin{enumerate}
	\item Determine
	\begin{enumerate}
		\item P (John promoted)
		\item P (Rita promoted)
		\item P (Aslam promoted)
		\item P (Gurpreet promoted)
	\end{enumerate}
	\item If A = {John promoted or Gurpreet promoted}, find P (A).
\end{enumerate}
\solution
%\input{exemplar/11/16/3/10/main.tex}
\item A card is drawn from a deck of 52 cards. Find the probability of getting a king or a heart or a red card.\\
\solution
%\input{exemplar/11/16/3/15/main.tex}
\item The probability that a student will pass his examination is 0.73, the probability of
the student getting a compartment is 0.13, and the probability that the student will
either pass or get compartment is 0.96. State True or False.\\
\solution
%\input{exemplar/11/16/3/31/main.tex}
\item A card is selected from a pack of 52 cards\\
\begin{enumerate}[label=(\alph*)]
\item How many points are there in the sample space?
\item Calculate the probability that the cards is an ace of spades.
\item Calculate the probability that the card is (i) an ace (ii)black card.\\
\end{enumerate}
%\input{ncert/11/16/3/4_1/Prob_4.tex}
\item In a non-leap year, the probability of having 53 tuesdays or 53 wednesdays is\\
\solution
%\input{exemplar/11/16/3/18/main.tex}
\item There are 1000 sealed envelopes in a box, 10 of them contain a cash prize of
Rs 100 each, 100 of them contain a cash prize of Rs 50 each and 200 of them
contain a cash prize of Rs 10 each and rest do not contain any cash prize. If they
are well shuffled and an envelope is picked up out, what is the probability that it
contains no cash prize?\\
\solution
%\input{exemplar/10/13/3/34/main.tex}
\item 
A die is thrown and a card is selected at random from a deck of 52 playing cards. The probability of getting an even number on the die and a spade card.\\
\solution
%\input{exemplar/12/13/3/78/main.tex}
\item
If 4-digit numbers greater than 5,000 are randomly formed from the digits 0, 1, 3, 5, and 7, what is the probability of forming a number divisible by 5 when:
\begin{enumerate}
    \item The digits are repeated?
    \item The repetition of digits is not allowed?
\end{enumerate}
\solution
%\input{ncert/11/16/4/9/main.tex}
\item Consider the probability space $\brak{\Omega, \mathcal{G}, P}$ where $\Omega = [0,2]$ and $\mathcal{G} = \cbrak{\phi, \Omega, [0,1], (1,2]}$. Let $X$ and $Y$ be two functions on $\Omega$ defined as
\begin{align*}
    X(\omega) = 
    \begin{cases}
        1 & \text{if }\omega \in [0, 1]\\
        2 & \text{if }\omega \in (1, 2]
    \end{cases}
\end{align*}
and
\begin{align*}
    Y(\omega) = 
    \begin{cases}
        2 & \text{if }\omega \in [0, 1.5]\\
        3 & \text{if }\omega \in (1.5, 2].
    \end{cases}
\end{align*}
Then which one of the following statements is true?
\begin{enumerate}
    \item [(A)] $X$ is a random variable with respect to $\mathcal{G}$, but $Y$ is not a random variable with respect to $\mathcal{G}$.
    \item [(B)] $Y$ is a random variable with respect to $\mathcal{G}$, but $X$ is not a random variable with respect to $\mathcal{G}$.
    \item [(C)] Neither $X$ nor $Y$ is a random variable with respect to $\mathcal{G}$.
    \item [(D)] Both $X$ and $Y$ are random variables with respect to $\mathcal{G}$.
\end{enumerate} \hfill (GATE ST 2023)\\
\solution
%\input{gate/ST/2023/14/main.tex}
	\item  A die is loaded in such a way that each odd number is twice as likely to occur as
each even number. Find $P(G)$, where $G$ is the event that a number greater than
3 occurs on a single roll of the die.
\\
\solution
		%\input{exemplar/11/16/3/5/main.tex}
	\item All the jacks, queens and kings are removed from a deck of 52 playing cards. The remaining cards are well shuffled and then one card is drawn at random. Giving ace a value 1 similar value for other cards, find the probability that the card has a value 
		\begin{enumerate}
			\item 7
			\item greater than 7
			\item less than 7
		\end{enumerate}
		%\input{exemplar/10/13/3/30/main.tex}
  \item A Lot consists of 48 mobile phones of which 42 are good, 3 have only minor defects and 3 have major defects.Varnika will buy a phone if it is good but the trader will only buy a mobile if it has no major defects. One phone is selected at random from the lot. What is the probability that it is
\begin{enumerate}
	\item acceptable to Varnika?
            \item acceptable to the trader?
\end{enumerate}
\solution
	%\input{exemplar/10/13/3/40/main.tex}
 \item A student says that if you throw a die, it will show up 1 or not 1. Therefore, the probability of getting 1 and the probability of getting 'not 1' each is equal to $\frac{1}{2}$. Is this correct? Give reasons.\\
 \solution
        %\input{exemplar/10/13/2/9/main.tex}
   \item Four candidates A, B, C, D have ap-
plied for the assignment to coach a school cricket
team. If A is twice as likely to be selected as B, and
B and C are given about the same chance of being
selected, while C is twice as likely to be selected
as D, what are the probabilities that
\begin{enumerate}
\item C will be selected?
\item A will not be selected?
\end{enumerate}
	%\input{exemplar/11/16/3/9/main.tex}
 \item A bag contain 24 balls of which $x$ balls are red, $2x$ are white and $3x$ are blue. A ball is selected at random, What is the probability that it is
\begin{enumerate}[label=\alph*)]
\item not red ?
\item white ?
\end{enumerate}
%\input{exemplar/10/13/3/41/main.tex}
If the letters of the word ASSASSINATION are arranged at random. Find the Probability that
\begin{enumerate}[label=(\alph*)]
\item Four $S's$ come consecutively in the word
\item Two  $I's$ and two $N's$ come together
\item All $A's$ are not coming together
\item No two $A's$ are coming together
\end{enumerate}
%\input{exemplar/11/16/3/14/main.tex}
	\item One urn contains two black balls (labelled B1 and B2) and one white ball. A
	second urn contains one black ball and two white balls (labelled W1 and W2).
	Suppose the following experiment is performed. One of the two urns is chosen
	at random. Next a ball is randomly chosen from the urn. Then a second ball is
	chosen at random from the same urn without replacing the first ball.
	
	\begin{enumerate}
	\item What is the probability that two black balls are chosen?
	
	\item What is the probability that two balls of opposite colour are chosen?
	\end{enumerate}
	\solution
	%\input{exemplar/11/16/3/12/main1.tex}
\end{enumerate}

\item In a certain lottery 10,000 tickets are sold and ten equal prizes are awarded. What is the probability of not getting a prize if you buy (a) one ticket (b) two tickets (c) 10 tickets ?	
\\
\solution
		%\begin{enumerate}[label=\thesection.\arabic*,ref=\thesection.\theenumi]
	\item One card is drawn from a well-shuffled deck of 52 cards. Find the probability of getting
\begin{enumerate}
\item A king of red colour 
\item A face card 
\item A red face card
\item The jack of hearts
\item A spade
\item The queen of diamonds

\end{enumerate}
\solution
		%\input{ncert/10/15/1/14/main.tex}
	\item Five cards—the ten, jack, queen, king and ace of diamonds, are well-shuffled with their face downwards. One card is then picked up at random.
\begin{enumerate}
\item
What is the probability that the card is the queen? 
\item
If the queen is drawn and put aside, what is the probability that the second card picked up is (a) an ace? (b) a queen?\\
\end{enumerate}
\solution
		%\input{ncert/10/15/1/15/defs.tex}
	\item A bag contains $5$ red balls and some blue balls. If the probability of drawing a blue ball is double that if a red ball, determine the number of blue balls in the bag. 
		\\
\solution
		%\input{ncert/10/15/2/3/defs.tex}
	\item A card is selected from a pack of 52 cards.
 \begin{enumerate}[label=(\alph*)] 
                 \item How many points are there in the sample space?
                 \item Calculate the probability that the card is an ace of spades.
                 \item Calculate the probability that the card is (i) an ace and (ii) black card.
 \end{enumerate}
\solution
		%\input{ncert/11/16/3/4/main.tex}
\item Four cards are drawn from a well-shuffled deck of 52 cards. What is the probability of obtaining 3 diamonds and one spade.
\\
\solution
		%\input{ncert/11/16/4/2/defs.tex}
\item In a certain lottery 10,000 tickets are sold and ten equal prizes are awarded. What is the probability of not getting a prize if you buy (a) one ticket (b) two tickets (c) 10 tickets ?	
\\
\solution
		%\input{ncert/11/16/4/4/defs.tex}
		%
\item 
Out of 100 students, two sections of 40 and 60 are formed. If you and your friend are among the 100 students, what is the probability that
\begin{enumerate}
\item you both enter the same section?
\item you both enter the different sections?
\end{enumerate}
\solution
		%\input{ncert/11/16/4/5/defs.tex}
	\item 
The number lock of a suitcase has 4 wheels each labelled with ten digits i.e. from 0 to 9.The lock opens with a sequence of four digits with no repeats.What is the probability of a person getting the right sequence to open the suitcase.
\\
\solution
		%\input{ncert/11/16/4/10/defs.tex}
		%
\item 
Two cards are drawn at random and without replacement from a pack of 52 playing cards. Find the probability that both the cards are black.
\\
\solution
		%\input{ncert/12/13/2/2/defs.tex}
		\item A box of oranges is inspected by examining three randomly selected oranges drawn without replacement. If all the three oranges are good, the box is approved for sale, otherwise, it is rejected. Find the probability that a box containing 15 oranges out of which 12 are good and 3 are bad ones will be approved for sale.
		\label{ncert/12/13/2/3/defs.tex}
		\item Two balls are drawn at random with replacement from a box containing 10 black and 8 red balls. Find the probability that
		\label{ncert/12/13/2/12}
\begin{enumerate}
\item both balls are red.
\item first ball is black and second is red.
\item one of them is black and other is red.
\end{enumerate}

\item In a hostel, 60\% of the students read Hindi newspaper, 40\% read English newspaper and 20\% read both Hindi and English newspapers. A student is selected at random.
		\label{ncert/12/13/2/15}
\begin{enumerate}
\item Find the probability that she reads neither Hindi nor English newspapers.
\item If she reads Hindi newspaper, find the probability that she reads English newspaper.
\item If she reads English newspaper, find the probability that she reads Hindi newspaper.\\
\end{enumerate}
\item The probability of obtaining an even prime number on each die, when a pair of dice is rolled is 
\begin{enumerate}
    \item $0$ 
    
    \item $\frac{1}{3}$ 
    
    \item $\frac{1}{12}$ 
    
    \item $\frac{1}{36}$ 
\end{enumerate}
\solution
		%\input{ncert/12/13/2/17/defs.tex}
	\item A bag contains 4 red and 4 black balls, another bag contains 2 red and 6 black balls. One of the two bags is selected at random and a ball is drawn from the bag which is found to be red. Find the probability that the ball is drawn from the first bag.
\\
\solution
		%\input{ncert/12/13/3/2/main.tex}
  \item
  Cards with numbers 2 to 101 are placed in a box. A card is selected at random.Find the probability that the card has
\begin{enumerate}[label=(\roman*)]
	\item an even number 
	\item a square number
\end{enumerate}
\solution
%\input{exemplar/10/13/3/32/main.tex}
\item
The king, queen and jack of clubs are removed from a deck of 52 playing cards and then well shuffled. Now one card is drawn at random from the remaining cards.  Determine the probability that the card is
\begin{enumerate}[label=(\roman*)]
\item a club
\item 10 of hearts
\end{enumerate}
\solution
%\input{exemplar/10/13/3/29/main.tex}
\item A team of medical students doing their internship have to assist during surgeries
at a city hospital. The probabilities of surgeries rated as very complex, complex,
routine, simple or very simple are respectively, 0.15, 0.20, 0.31, 0.26, .08. Find
the probabilities that a particular surgery will be rated
\begin{enumerate}
	\item complex or very complex;
	\item neither very complex nor very simple;
	\item routine or complex
	\item routine or simple
\end{enumerate}
\solution
%\input{exemplar/11/16/3/8(1)/main.tex}
\item A card is selected from a pack of 52 cards.
\begin{enumerate}[label=(\alph*)]
    \item How many points are there in the sample space?
    \item Calculate the probability that the card is an ace of spades.
    \item Calculate the probability that the card is (i) an ace and (ii) black card.
\end{enumerate}
\solution
%\input{exemplar/11/16/3/4/main2.tex}
\item The probability that a non leap year selected at random will contain 53 sundays.
\\
\solution
%\input{exemplar/10/13/1/19/main.tex}
\item One of the four persons John, Rita, Aslam or Gurpreet will be promoted next
month. Consequently the sample space consists of four elementary outcomes
S = {John promoted, Rita promoted, Aslam promoted, Gurpreet promoted}
You are told that the chances of John’s promotion is same as that of Gurpreet,
Rita’s chances of promotion are twice as likely as Johns. Aslam’s chances are
four times that of John.
\begin{enumerate}
	\item Determine
	\begin{enumerate}
		\item P (John promoted)
		\item P (Rita promoted)
		\item P (Aslam promoted)
		\item P (Gurpreet promoted)
	\end{enumerate}
	\item If A = {John promoted or Gurpreet promoted}, find P (A).
\end{enumerate}
\solution
%\input{exemplar/11/16/3/10/main.tex}
\item A card is drawn from a deck of 52 cards. Find the probability of getting a king or a heart or a red card.\\
\solution
%\input{exemplar/11/16/3/15/main.tex}
\item The probability that a student will pass his examination is 0.73, the probability of
the student getting a compartment is 0.13, and the probability that the student will
either pass or get compartment is 0.96. State True or False.\\
\solution
%\input{exemplar/11/16/3/31/main.tex}
\item A card is selected from a pack of 52 cards\\
\begin{enumerate}[label=(\alph*)]
\item How many points are there in the sample space?
\item Calculate the probability that the cards is an ace of spades.
\item Calculate the probability that the card is (i) an ace (ii)black card.\\
\end{enumerate}
%\input{ncert/11/16/3/4_1/Prob_4.tex}
\item In a non-leap year, the probability of having 53 tuesdays or 53 wednesdays is\\
\solution
%\input{exemplar/11/16/3/18/main.tex}
\item There are 1000 sealed envelopes in a box, 10 of them contain a cash prize of
Rs 100 each, 100 of them contain a cash prize of Rs 50 each and 200 of them
contain a cash prize of Rs 10 each and rest do not contain any cash prize. If they
are well shuffled and an envelope is picked up out, what is the probability that it
contains no cash prize?\\
\solution
%\input{exemplar/10/13/3/34/main.tex}
\item 
A die is thrown and a card is selected at random from a deck of 52 playing cards. The probability of getting an even number on the die and a spade card.\\
\solution
%\input{exemplar/12/13/3/78/main.tex}
\item
If 4-digit numbers greater than 5,000 are randomly formed from the digits 0, 1, 3, 5, and 7, what is the probability of forming a number divisible by 5 when:
\begin{enumerate}
    \item The digits are repeated?
    \item The repetition of digits is not allowed?
\end{enumerate}
\solution
%\input{ncert/11/16/4/9/main.tex}
\item Consider the probability space $\brak{\Omega, \mathcal{G}, P}$ where $\Omega = [0,2]$ and $\mathcal{G} = \cbrak{\phi, \Omega, [0,1], (1,2]}$. Let $X$ and $Y$ be two functions on $\Omega$ defined as
\begin{align*}
    X(\omega) = 
    \begin{cases}
        1 & \text{if }\omega \in [0, 1]\\
        2 & \text{if }\omega \in (1, 2]
    \end{cases}
\end{align*}
and
\begin{align*}
    Y(\omega) = 
    \begin{cases}
        2 & \text{if }\omega \in [0, 1.5]\\
        3 & \text{if }\omega \in (1.5, 2].
    \end{cases}
\end{align*}
Then which one of the following statements is true?
\begin{enumerate}
    \item [(A)] $X$ is a random variable with respect to $\mathcal{G}$, but $Y$ is not a random variable with respect to $\mathcal{G}$.
    \item [(B)] $Y$ is a random variable with respect to $\mathcal{G}$, but $X$ is not a random variable with respect to $\mathcal{G}$.
    \item [(C)] Neither $X$ nor $Y$ is a random variable with respect to $\mathcal{G}$.
    \item [(D)] Both $X$ and $Y$ are random variables with respect to $\mathcal{G}$.
\end{enumerate} \hfill (GATE ST 2023)\\
\solution
%\input{gate/ST/2023/14/main.tex}
	\item  A die is loaded in such a way that each odd number is twice as likely to occur as
each even number. Find $P(G)$, where $G$ is the event that a number greater than
3 occurs on a single roll of the die.
\\
\solution
		%\input{exemplar/11/16/3/5/main.tex}
	\item All the jacks, queens and kings are removed from a deck of 52 playing cards. The remaining cards are well shuffled and then one card is drawn at random. Giving ace a value 1 similar value for other cards, find the probability that the card has a value 
		\begin{enumerate}
			\item 7
			\item greater than 7
			\item less than 7
		\end{enumerate}
		%\input{exemplar/10/13/3/30/main.tex}
  \item A Lot consists of 48 mobile phones of which 42 are good, 3 have only minor defects and 3 have major defects.Varnika will buy a phone if it is good but the trader will only buy a mobile if it has no major defects. One phone is selected at random from the lot. What is the probability that it is
\begin{enumerate}
	\item acceptable to Varnika?
            \item acceptable to the trader?
\end{enumerate}
\solution
	%\input{exemplar/10/13/3/40/main.tex}
 \item A student says that if you throw a die, it will show up 1 or not 1. Therefore, the probability of getting 1 and the probability of getting 'not 1' each is equal to $\frac{1}{2}$. Is this correct? Give reasons.\\
 \solution
        %\input{exemplar/10/13/2/9/main.tex}
   \item Four candidates A, B, C, D have ap-
plied for the assignment to coach a school cricket
team. If A is twice as likely to be selected as B, and
B and C are given about the same chance of being
selected, while C is twice as likely to be selected
as D, what are the probabilities that
\begin{enumerate}
\item C will be selected?
\item A will not be selected?
\end{enumerate}
	%\input{exemplar/11/16/3/9/main.tex}
 \item A bag contain 24 balls of which $x$ balls are red, $2x$ are white and $3x$ are blue. A ball is selected at random, What is the probability that it is
\begin{enumerate}[label=\alph*)]
\item not red ?
\item white ?
\end{enumerate}
%\input{exemplar/10/13/3/41/main.tex}
If the letters of the word ASSASSINATION are arranged at random. Find the Probability that
\begin{enumerate}[label=(\alph*)]
\item Four $S's$ come consecutively in the word
\item Two  $I's$ and two $N's$ come together
\item All $A's$ are not coming together
\item No two $A's$ are coming together
\end{enumerate}
%\input{exemplar/11/16/3/14/main.tex}
	\item One urn contains two black balls (labelled B1 and B2) and one white ball. A
	second urn contains one black ball and two white balls (labelled W1 and W2).
	Suppose the following experiment is performed. One of the two urns is chosen
	at random. Next a ball is randomly chosen from the urn. Then a second ball is
	chosen at random from the same urn without replacing the first ball.
	
	\begin{enumerate}
	\item What is the probability that two black balls are chosen?
	
	\item What is the probability that two balls of opposite colour are chosen?
	\end{enumerate}
	\solution
	%\input{exemplar/11/16/3/12/main1.tex}
\end{enumerate}

		%
\item 
Out of 100 students, two sections of 40 and 60 are formed. If you and your friend are among the 100 students, what is the probability that
\begin{enumerate}
\item you both enter the same section?
\item you both enter the different sections?
\end{enumerate}
\solution
		%\begin{enumerate}[label=\thesection.\arabic*,ref=\thesection.\theenumi]
	\item One card is drawn from a well-shuffled deck of 52 cards. Find the probability of getting
\begin{enumerate}
\item A king of red colour 
\item A face card 
\item A red face card
\item The jack of hearts
\item A spade
\item The queen of diamonds

\end{enumerate}
\solution
		%\input{ncert/10/15/1/14/main.tex}
	\item Five cards—the ten, jack, queen, king and ace of diamonds, are well-shuffled with their face downwards. One card is then picked up at random.
\begin{enumerate}
\item
What is the probability that the card is the queen? 
\item
If the queen is drawn and put aside, what is the probability that the second card picked up is (a) an ace? (b) a queen?\\
\end{enumerate}
\solution
		%\input{ncert/10/15/1/15/defs.tex}
	\item A bag contains $5$ red balls and some blue balls. If the probability of drawing a blue ball is double that if a red ball, determine the number of blue balls in the bag. 
		\\
\solution
		%\input{ncert/10/15/2/3/defs.tex}
	\item A card is selected from a pack of 52 cards.
 \begin{enumerate}[label=(\alph*)] 
                 \item How many points are there in the sample space?
                 \item Calculate the probability that the card is an ace of spades.
                 \item Calculate the probability that the card is (i) an ace and (ii) black card.
 \end{enumerate}
\solution
		%\input{ncert/11/16/3/4/main.tex}
\item Four cards are drawn from a well-shuffled deck of 52 cards. What is the probability of obtaining 3 diamonds and one spade.
\\
\solution
		%\input{ncert/11/16/4/2/defs.tex}
\item In a certain lottery 10,000 tickets are sold and ten equal prizes are awarded. What is the probability of not getting a prize if you buy (a) one ticket (b) two tickets (c) 10 tickets ?	
\\
\solution
		%\input{ncert/11/16/4/4/defs.tex}
		%
\item 
Out of 100 students, two sections of 40 and 60 are formed. If you and your friend are among the 100 students, what is the probability that
\begin{enumerate}
\item you both enter the same section?
\item you both enter the different sections?
\end{enumerate}
\solution
		%\input{ncert/11/16/4/5/defs.tex}
	\item 
The number lock of a suitcase has 4 wheels each labelled with ten digits i.e. from 0 to 9.The lock opens with a sequence of four digits with no repeats.What is the probability of a person getting the right sequence to open the suitcase.
\\
\solution
		%\input{ncert/11/16/4/10/defs.tex}
		%
\item 
Two cards are drawn at random and without replacement from a pack of 52 playing cards. Find the probability that both the cards are black.
\\
\solution
		%\input{ncert/12/13/2/2/defs.tex}
		\item A box of oranges is inspected by examining three randomly selected oranges drawn without replacement. If all the three oranges are good, the box is approved for sale, otherwise, it is rejected. Find the probability that a box containing 15 oranges out of which 12 are good and 3 are bad ones will be approved for sale.
		\label{ncert/12/13/2/3/defs.tex}
		\item Two balls are drawn at random with replacement from a box containing 10 black and 8 red balls. Find the probability that
		\label{ncert/12/13/2/12}
\begin{enumerate}
\item both balls are red.
\item first ball is black and second is red.
\item one of them is black and other is red.
\end{enumerate}

\item In a hostel, 60\% of the students read Hindi newspaper, 40\% read English newspaper and 20\% read both Hindi and English newspapers. A student is selected at random.
		\label{ncert/12/13/2/15}
\begin{enumerate}
\item Find the probability that she reads neither Hindi nor English newspapers.
\item If she reads Hindi newspaper, find the probability that she reads English newspaper.
\item If she reads English newspaper, find the probability that she reads Hindi newspaper.\\
\end{enumerate}
\item The probability of obtaining an even prime number on each die, when a pair of dice is rolled is 
\begin{enumerate}
    \item $0$ 
    
    \item $\frac{1}{3}$ 
    
    \item $\frac{1}{12}$ 
    
    \item $\frac{1}{36}$ 
\end{enumerate}
\solution
		%\input{ncert/12/13/2/17/defs.tex}
	\item A bag contains 4 red and 4 black balls, another bag contains 2 red and 6 black balls. One of the two bags is selected at random and a ball is drawn from the bag which is found to be red. Find the probability that the ball is drawn from the first bag.
\\
\solution
		%\input{ncert/12/13/3/2/main.tex}
  \item
  Cards with numbers 2 to 101 are placed in a box. A card is selected at random.Find the probability that the card has
\begin{enumerate}[label=(\roman*)]
	\item an even number 
	\item a square number
\end{enumerate}
\solution
%\input{exemplar/10/13/3/32/main.tex}
\item
The king, queen and jack of clubs are removed from a deck of 52 playing cards and then well shuffled. Now one card is drawn at random from the remaining cards.  Determine the probability that the card is
\begin{enumerate}[label=(\roman*)]
\item a club
\item 10 of hearts
\end{enumerate}
\solution
%\input{exemplar/10/13/3/29/main.tex}
\item A team of medical students doing their internship have to assist during surgeries
at a city hospital. The probabilities of surgeries rated as very complex, complex,
routine, simple or very simple are respectively, 0.15, 0.20, 0.31, 0.26, .08. Find
the probabilities that a particular surgery will be rated
\begin{enumerate}
	\item complex or very complex;
	\item neither very complex nor very simple;
	\item routine or complex
	\item routine or simple
\end{enumerate}
\solution
%\input{exemplar/11/16/3/8(1)/main.tex}
\item A card is selected from a pack of 52 cards.
\begin{enumerate}[label=(\alph*)]
    \item How many points are there in the sample space?
    \item Calculate the probability that the card is an ace of spades.
    \item Calculate the probability that the card is (i) an ace and (ii) black card.
\end{enumerate}
\solution
%\input{exemplar/11/16/3/4/main2.tex}
\item The probability that a non leap year selected at random will contain 53 sundays.
\\
\solution
%\input{exemplar/10/13/1/19/main.tex}
\item One of the four persons John, Rita, Aslam or Gurpreet will be promoted next
month. Consequently the sample space consists of four elementary outcomes
S = {John promoted, Rita promoted, Aslam promoted, Gurpreet promoted}
You are told that the chances of John’s promotion is same as that of Gurpreet,
Rita’s chances of promotion are twice as likely as Johns. Aslam’s chances are
four times that of John.
\begin{enumerate}
	\item Determine
	\begin{enumerate}
		\item P (John promoted)
		\item P (Rita promoted)
		\item P (Aslam promoted)
		\item P (Gurpreet promoted)
	\end{enumerate}
	\item If A = {John promoted or Gurpreet promoted}, find P (A).
\end{enumerate}
\solution
%\input{exemplar/11/16/3/10/main.tex}
\item A card is drawn from a deck of 52 cards. Find the probability of getting a king or a heart or a red card.\\
\solution
%\input{exemplar/11/16/3/15/main.tex}
\item The probability that a student will pass his examination is 0.73, the probability of
the student getting a compartment is 0.13, and the probability that the student will
either pass or get compartment is 0.96. State True or False.\\
\solution
%\input{exemplar/11/16/3/31/main.tex}
\item A card is selected from a pack of 52 cards\\
\begin{enumerate}[label=(\alph*)]
\item How many points are there in the sample space?
\item Calculate the probability that the cards is an ace of spades.
\item Calculate the probability that the card is (i) an ace (ii)black card.\\
\end{enumerate}
%\input{ncert/11/16/3/4_1/Prob_4.tex}
\item In a non-leap year, the probability of having 53 tuesdays or 53 wednesdays is\\
\solution
%\input{exemplar/11/16/3/18/main.tex}
\item There are 1000 sealed envelopes in a box, 10 of them contain a cash prize of
Rs 100 each, 100 of them contain a cash prize of Rs 50 each and 200 of them
contain a cash prize of Rs 10 each and rest do not contain any cash prize. If they
are well shuffled and an envelope is picked up out, what is the probability that it
contains no cash prize?\\
\solution
%\input{exemplar/10/13/3/34/main.tex}
\item 
A die is thrown and a card is selected at random from a deck of 52 playing cards. The probability of getting an even number on the die and a spade card.\\
\solution
%\input{exemplar/12/13/3/78/main.tex}
\item
If 4-digit numbers greater than 5,000 are randomly formed from the digits 0, 1, 3, 5, and 7, what is the probability of forming a number divisible by 5 when:
\begin{enumerate}
    \item The digits are repeated?
    \item The repetition of digits is not allowed?
\end{enumerate}
\solution
%\input{ncert/11/16/4/9/main.tex}
\item Consider the probability space $\brak{\Omega, \mathcal{G}, P}$ where $\Omega = [0,2]$ and $\mathcal{G} = \cbrak{\phi, \Omega, [0,1], (1,2]}$. Let $X$ and $Y$ be two functions on $\Omega$ defined as
\begin{align*}
    X(\omega) = 
    \begin{cases}
        1 & \text{if }\omega \in [0, 1]\\
        2 & \text{if }\omega \in (1, 2]
    \end{cases}
\end{align*}
and
\begin{align*}
    Y(\omega) = 
    \begin{cases}
        2 & \text{if }\omega \in [0, 1.5]\\
        3 & \text{if }\omega \in (1.5, 2].
    \end{cases}
\end{align*}
Then which one of the following statements is true?
\begin{enumerate}
    \item [(A)] $X$ is a random variable with respect to $\mathcal{G}$, but $Y$ is not a random variable with respect to $\mathcal{G}$.
    \item [(B)] $Y$ is a random variable with respect to $\mathcal{G}$, but $X$ is not a random variable with respect to $\mathcal{G}$.
    \item [(C)] Neither $X$ nor $Y$ is a random variable with respect to $\mathcal{G}$.
    \item [(D)] Both $X$ and $Y$ are random variables with respect to $\mathcal{G}$.
\end{enumerate} \hfill (GATE ST 2023)\\
\solution
%\input{gate/ST/2023/14/main.tex}
	\item  A die is loaded in such a way that each odd number is twice as likely to occur as
each even number. Find $P(G)$, where $G$ is the event that a number greater than
3 occurs on a single roll of the die.
\\
\solution
		%\input{exemplar/11/16/3/5/main.tex}
	\item All the jacks, queens and kings are removed from a deck of 52 playing cards. The remaining cards are well shuffled and then one card is drawn at random. Giving ace a value 1 similar value for other cards, find the probability that the card has a value 
		\begin{enumerate}
			\item 7
			\item greater than 7
			\item less than 7
		\end{enumerate}
		%\input{exemplar/10/13/3/30/main.tex}
  \item A Lot consists of 48 mobile phones of which 42 are good, 3 have only minor defects and 3 have major defects.Varnika will buy a phone if it is good but the trader will only buy a mobile if it has no major defects. One phone is selected at random from the lot. What is the probability that it is
\begin{enumerate}
	\item acceptable to Varnika?
            \item acceptable to the trader?
\end{enumerate}
\solution
	%\input{exemplar/10/13/3/40/main.tex}
 \item A student says that if you throw a die, it will show up 1 or not 1. Therefore, the probability of getting 1 and the probability of getting 'not 1' each is equal to $\frac{1}{2}$. Is this correct? Give reasons.\\
 \solution
        %\input{exemplar/10/13/2/9/main.tex}
   \item Four candidates A, B, C, D have ap-
plied for the assignment to coach a school cricket
team. If A is twice as likely to be selected as B, and
B and C are given about the same chance of being
selected, while C is twice as likely to be selected
as D, what are the probabilities that
\begin{enumerate}
\item C will be selected?
\item A will not be selected?
\end{enumerate}
	%\input{exemplar/11/16/3/9/main.tex}
 \item A bag contain 24 balls of which $x$ balls are red, $2x$ are white and $3x$ are blue. A ball is selected at random, What is the probability that it is
\begin{enumerate}[label=\alph*)]
\item not red ?
\item white ?
\end{enumerate}
%\input{exemplar/10/13/3/41/main.tex}
If the letters of the word ASSASSINATION are arranged at random. Find the Probability that
\begin{enumerate}[label=(\alph*)]
\item Four $S's$ come consecutively in the word
\item Two  $I's$ and two $N's$ come together
\item All $A's$ are not coming together
\item No two $A's$ are coming together
\end{enumerate}
%\input{exemplar/11/16/3/14/main.tex}
	\item One urn contains two black balls (labelled B1 and B2) and one white ball. A
	second urn contains one black ball and two white balls (labelled W1 and W2).
	Suppose the following experiment is performed. One of the two urns is chosen
	at random. Next a ball is randomly chosen from the urn. Then a second ball is
	chosen at random from the same urn without replacing the first ball.
	
	\begin{enumerate}
	\item What is the probability that two black balls are chosen?
	
	\item What is the probability that two balls of opposite colour are chosen?
	\end{enumerate}
	\solution
	%\input{exemplar/11/16/3/12/main1.tex}
\end{enumerate}

	\item 
The number lock of a suitcase has 4 wheels each labelled with ten digits i.e. from 0 to 9.The lock opens with a sequence of four digits with no repeats.What is the probability of a person getting the right sequence to open the suitcase.
\\
\solution
		%\begin{enumerate}[label=\thesection.\arabic*,ref=\thesection.\theenumi]
	\item One card is drawn from a well-shuffled deck of 52 cards. Find the probability of getting
\begin{enumerate}
\item A king of red colour 
\item A face card 
\item A red face card
\item The jack of hearts
\item A spade
\item The queen of diamonds

\end{enumerate}
\solution
		%\input{ncert/10/15/1/14/main.tex}
	\item Five cards—the ten, jack, queen, king and ace of diamonds, are well-shuffled with their face downwards. One card is then picked up at random.
\begin{enumerate}
\item
What is the probability that the card is the queen? 
\item
If the queen is drawn and put aside, what is the probability that the second card picked up is (a) an ace? (b) a queen?\\
\end{enumerate}
\solution
		%\input{ncert/10/15/1/15/defs.tex}
	\item A bag contains $5$ red balls and some blue balls. If the probability of drawing a blue ball is double that if a red ball, determine the number of blue balls in the bag. 
		\\
\solution
		%\input{ncert/10/15/2/3/defs.tex}
	\item A card is selected from a pack of 52 cards.
 \begin{enumerate}[label=(\alph*)] 
                 \item How many points are there in the sample space?
                 \item Calculate the probability that the card is an ace of spades.
                 \item Calculate the probability that the card is (i) an ace and (ii) black card.
 \end{enumerate}
\solution
		%\input{ncert/11/16/3/4/main.tex}
\item Four cards are drawn from a well-shuffled deck of 52 cards. What is the probability of obtaining 3 diamonds and one spade.
\\
\solution
		%\input{ncert/11/16/4/2/defs.tex}
\item In a certain lottery 10,000 tickets are sold and ten equal prizes are awarded. What is the probability of not getting a prize if you buy (a) one ticket (b) two tickets (c) 10 tickets ?	
\\
\solution
		%\input{ncert/11/16/4/4/defs.tex}
		%
\item 
Out of 100 students, two sections of 40 and 60 are formed. If you and your friend are among the 100 students, what is the probability that
\begin{enumerate}
\item you both enter the same section?
\item you both enter the different sections?
\end{enumerate}
\solution
		%\input{ncert/11/16/4/5/defs.tex}
	\item 
The number lock of a suitcase has 4 wheels each labelled with ten digits i.e. from 0 to 9.The lock opens with a sequence of four digits with no repeats.What is the probability of a person getting the right sequence to open the suitcase.
\\
\solution
		%\input{ncert/11/16/4/10/defs.tex}
		%
\item 
Two cards are drawn at random and without replacement from a pack of 52 playing cards. Find the probability that both the cards are black.
\\
\solution
		%\input{ncert/12/13/2/2/defs.tex}
		\item A box of oranges is inspected by examining three randomly selected oranges drawn without replacement. If all the three oranges are good, the box is approved for sale, otherwise, it is rejected. Find the probability that a box containing 15 oranges out of which 12 are good and 3 are bad ones will be approved for sale.
		\label{ncert/12/13/2/3/defs.tex}
		\item Two balls are drawn at random with replacement from a box containing 10 black and 8 red balls. Find the probability that
		\label{ncert/12/13/2/12}
\begin{enumerate}
\item both balls are red.
\item first ball is black and second is red.
\item one of them is black and other is red.
\end{enumerate}

\item In a hostel, 60\% of the students read Hindi newspaper, 40\% read English newspaper and 20\% read both Hindi and English newspapers. A student is selected at random.
		\label{ncert/12/13/2/15}
\begin{enumerate}
\item Find the probability that she reads neither Hindi nor English newspapers.
\item If she reads Hindi newspaper, find the probability that she reads English newspaper.
\item If she reads English newspaper, find the probability that she reads Hindi newspaper.\\
\end{enumerate}
\item The probability of obtaining an even prime number on each die, when a pair of dice is rolled is 
\begin{enumerate}
    \item $0$ 
    
    \item $\frac{1}{3}$ 
    
    \item $\frac{1}{12}$ 
    
    \item $\frac{1}{36}$ 
\end{enumerate}
\solution
		%\input{ncert/12/13/2/17/defs.tex}
	\item A bag contains 4 red and 4 black balls, another bag contains 2 red and 6 black balls. One of the two bags is selected at random and a ball is drawn from the bag which is found to be red. Find the probability that the ball is drawn from the first bag.
\\
\solution
		%\input{ncert/12/13/3/2/main.tex}
  \item
  Cards with numbers 2 to 101 are placed in a box. A card is selected at random.Find the probability that the card has
\begin{enumerate}[label=(\roman*)]
	\item an even number 
	\item a square number
\end{enumerate}
\solution
%\input{exemplar/10/13/3/32/main.tex}
\item
The king, queen and jack of clubs are removed from a deck of 52 playing cards and then well shuffled. Now one card is drawn at random from the remaining cards.  Determine the probability that the card is
\begin{enumerate}[label=(\roman*)]
\item a club
\item 10 of hearts
\end{enumerate}
\solution
%\input{exemplar/10/13/3/29/main.tex}
\item A team of medical students doing their internship have to assist during surgeries
at a city hospital. The probabilities of surgeries rated as very complex, complex,
routine, simple or very simple are respectively, 0.15, 0.20, 0.31, 0.26, .08. Find
the probabilities that a particular surgery will be rated
\begin{enumerate}
	\item complex or very complex;
	\item neither very complex nor very simple;
	\item routine or complex
	\item routine or simple
\end{enumerate}
\solution
%\input{exemplar/11/16/3/8(1)/main.tex}
\item A card is selected from a pack of 52 cards.
\begin{enumerate}[label=(\alph*)]
    \item How many points are there in the sample space?
    \item Calculate the probability that the card is an ace of spades.
    \item Calculate the probability that the card is (i) an ace and (ii) black card.
\end{enumerate}
\solution
%\input{exemplar/11/16/3/4/main2.tex}
\item The probability that a non leap year selected at random will contain 53 sundays.
\\
\solution
%\input{exemplar/10/13/1/19/main.tex}
\item One of the four persons John, Rita, Aslam or Gurpreet will be promoted next
month. Consequently the sample space consists of four elementary outcomes
S = {John promoted, Rita promoted, Aslam promoted, Gurpreet promoted}
You are told that the chances of John’s promotion is same as that of Gurpreet,
Rita’s chances of promotion are twice as likely as Johns. Aslam’s chances are
four times that of John.
\begin{enumerate}
	\item Determine
	\begin{enumerate}
		\item P (John promoted)
		\item P (Rita promoted)
		\item P (Aslam promoted)
		\item P (Gurpreet promoted)
	\end{enumerate}
	\item If A = {John promoted or Gurpreet promoted}, find P (A).
\end{enumerate}
\solution
%\input{exemplar/11/16/3/10/main.tex}
\item A card is drawn from a deck of 52 cards. Find the probability of getting a king or a heart or a red card.\\
\solution
%\input{exemplar/11/16/3/15/main.tex}
\item The probability that a student will pass his examination is 0.73, the probability of
the student getting a compartment is 0.13, and the probability that the student will
either pass or get compartment is 0.96. State True or False.\\
\solution
%\input{exemplar/11/16/3/31/main.tex}
\item A card is selected from a pack of 52 cards\\
\begin{enumerate}[label=(\alph*)]
\item How many points are there in the sample space?
\item Calculate the probability that the cards is an ace of spades.
\item Calculate the probability that the card is (i) an ace (ii)black card.\\
\end{enumerate}
%\input{ncert/11/16/3/4_1/Prob_4.tex}
\item In a non-leap year, the probability of having 53 tuesdays or 53 wednesdays is\\
\solution
%\input{exemplar/11/16/3/18/main.tex}
\item There are 1000 sealed envelopes in a box, 10 of them contain a cash prize of
Rs 100 each, 100 of them contain a cash prize of Rs 50 each and 200 of them
contain a cash prize of Rs 10 each and rest do not contain any cash prize. If they
are well shuffled and an envelope is picked up out, what is the probability that it
contains no cash prize?\\
\solution
%\input{exemplar/10/13/3/34/main.tex}
\item 
A die is thrown and a card is selected at random from a deck of 52 playing cards. The probability of getting an even number on the die and a spade card.\\
\solution
%\input{exemplar/12/13/3/78/main.tex}
\item
If 4-digit numbers greater than 5,000 are randomly formed from the digits 0, 1, 3, 5, and 7, what is the probability of forming a number divisible by 5 when:
\begin{enumerate}
    \item The digits are repeated?
    \item The repetition of digits is not allowed?
\end{enumerate}
\solution
%\input{ncert/11/16/4/9/main.tex}
\item Consider the probability space $\brak{\Omega, \mathcal{G}, P}$ where $\Omega = [0,2]$ and $\mathcal{G} = \cbrak{\phi, \Omega, [0,1], (1,2]}$. Let $X$ and $Y$ be two functions on $\Omega$ defined as
\begin{align*}
    X(\omega) = 
    \begin{cases}
        1 & \text{if }\omega \in [0, 1]\\
        2 & \text{if }\omega \in (1, 2]
    \end{cases}
\end{align*}
and
\begin{align*}
    Y(\omega) = 
    \begin{cases}
        2 & \text{if }\omega \in [0, 1.5]\\
        3 & \text{if }\omega \in (1.5, 2].
    \end{cases}
\end{align*}
Then which one of the following statements is true?
\begin{enumerate}
    \item [(A)] $X$ is a random variable with respect to $\mathcal{G}$, but $Y$ is not a random variable with respect to $\mathcal{G}$.
    \item [(B)] $Y$ is a random variable with respect to $\mathcal{G}$, but $X$ is not a random variable with respect to $\mathcal{G}$.
    \item [(C)] Neither $X$ nor $Y$ is a random variable with respect to $\mathcal{G}$.
    \item [(D)] Both $X$ and $Y$ are random variables with respect to $\mathcal{G}$.
\end{enumerate} \hfill (GATE ST 2023)\\
\solution
%\input{gate/ST/2023/14/main.tex}
	\item  A die is loaded in such a way that each odd number is twice as likely to occur as
each even number. Find $P(G)$, where $G$ is the event that a number greater than
3 occurs on a single roll of the die.
\\
\solution
		%\input{exemplar/11/16/3/5/main.tex}
	\item All the jacks, queens and kings are removed from a deck of 52 playing cards. The remaining cards are well shuffled and then one card is drawn at random. Giving ace a value 1 similar value for other cards, find the probability that the card has a value 
		\begin{enumerate}
			\item 7
			\item greater than 7
			\item less than 7
		\end{enumerate}
		%\input{exemplar/10/13/3/30/main.tex}
  \item A Lot consists of 48 mobile phones of which 42 are good, 3 have only minor defects and 3 have major defects.Varnika will buy a phone if it is good but the trader will only buy a mobile if it has no major defects. One phone is selected at random from the lot. What is the probability that it is
\begin{enumerate}
	\item acceptable to Varnika?
            \item acceptable to the trader?
\end{enumerate}
\solution
	%\input{exemplar/10/13/3/40/main.tex}
 \item A student says that if you throw a die, it will show up 1 or not 1. Therefore, the probability of getting 1 and the probability of getting 'not 1' each is equal to $\frac{1}{2}$. Is this correct? Give reasons.\\
 \solution
        %\input{exemplar/10/13/2/9/main.tex}
   \item Four candidates A, B, C, D have ap-
plied for the assignment to coach a school cricket
team. If A is twice as likely to be selected as B, and
B and C are given about the same chance of being
selected, while C is twice as likely to be selected
as D, what are the probabilities that
\begin{enumerate}
\item C will be selected?
\item A will not be selected?
\end{enumerate}
	%\input{exemplar/11/16/3/9/main.tex}
 \item A bag contain 24 balls of which $x$ balls are red, $2x$ are white and $3x$ are blue. A ball is selected at random, What is the probability that it is
\begin{enumerate}[label=\alph*)]
\item not red ?
\item white ?
\end{enumerate}
%\input{exemplar/10/13/3/41/main.tex}
If the letters of the word ASSASSINATION are arranged at random. Find the Probability that
\begin{enumerate}[label=(\alph*)]
\item Four $S's$ come consecutively in the word
\item Two  $I's$ and two $N's$ come together
\item All $A's$ are not coming together
\item No two $A's$ are coming together
\end{enumerate}
%\input{exemplar/11/16/3/14/main.tex}
	\item One urn contains two black balls (labelled B1 and B2) and one white ball. A
	second urn contains one black ball and two white balls (labelled W1 and W2).
	Suppose the following experiment is performed. One of the two urns is chosen
	at random. Next a ball is randomly chosen from the urn. Then a second ball is
	chosen at random from the same urn without replacing the first ball.
	
	\begin{enumerate}
	\item What is the probability that two black balls are chosen?
	
	\item What is the probability that two balls of opposite colour are chosen?
	\end{enumerate}
	\solution
	%\input{exemplar/11/16/3/12/main1.tex}
\end{enumerate}

		%
\item 
Two cards are drawn at random and without replacement from a pack of 52 playing cards. Find the probability that both the cards are black.
\\
\solution
		%\begin{enumerate}[label=\thesection.\arabic*,ref=\thesection.\theenumi]
	\item One card is drawn from a well-shuffled deck of 52 cards. Find the probability of getting
\begin{enumerate}
\item A king of red colour 
\item A face card 
\item A red face card
\item The jack of hearts
\item A spade
\item The queen of diamonds

\end{enumerate}
\solution
		%\input{ncert/10/15/1/14/main.tex}
	\item Five cards—the ten, jack, queen, king and ace of diamonds, are well-shuffled with their face downwards. One card is then picked up at random.
\begin{enumerate}
\item
What is the probability that the card is the queen? 
\item
If the queen is drawn and put aside, what is the probability that the second card picked up is (a) an ace? (b) a queen?\\
\end{enumerate}
\solution
		%\input{ncert/10/15/1/15/defs.tex}
	\item A bag contains $5$ red balls and some blue balls. If the probability of drawing a blue ball is double that if a red ball, determine the number of blue balls in the bag. 
		\\
\solution
		%\input{ncert/10/15/2/3/defs.tex}
	\item A card is selected from a pack of 52 cards.
 \begin{enumerate}[label=(\alph*)] 
                 \item How many points are there in the sample space?
                 \item Calculate the probability that the card is an ace of spades.
                 \item Calculate the probability that the card is (i) an ace and (ii) black card.
 \end{enumerate}
\solution
		%\input{ncert/11/16/3/4/main.tex}
\item Four cards are drawn from a well-shuffled deck of 52 cards. What is the probability of obtaining 3 diamonds and one spade.
\\
\solution
		%\input{ncert/11/16/4/2/defs.tex}
\item In a certain lottery 10,000 tickets are sold and ten equal prizes are awarded. What is the probability of not getting a prize if you buy (a) one ticket (b) two tickets (c) 10 tickets ?	
\\
\solution
		%\input{ncert/11/16/4/4/defs.tex}
		%
\item 
Out of 100 students, two sections of 40 and 60 are formed. If you and your friend are among the 100 students, what is the probability that
\begin{enumerate}
\item you both enter the same section?
\item you both enter the different sections?
\end{enumerate}
\solution
		%\input{ncert/11/16/4/5/defs.tex}
	\item 
The number lock of a suitcase has 4 wheels each labelled with ten digits i.e. from 0 to 9.The lock opens with a sequence of four digits with no repeats.What is the probability of a person getting the right sequence to open the suitcase.
\\
\solution
		%\input{ncert/11/16/4/10/defs.tex}
		%
\item 
Two cards are drawn at random and without replacement from a pack of 52 playing cards. Find the probability that both the cards are black.
\\
\solution
		%\input{ncert/12/13/2/2/defs.tex}
		\item A box of oranges is inspected by examining three randomly selected oranges drawn without replacement. If all the three oranges are good, the box is approved for sale, otherwise, it is rejected. Find the probability that a box containing 15 oranges out of which 12 are good and 3 are bad ones will be approved for sale.
		\label{ncert/12/13/2/3/defs.tex}
		\item Two balls are drawn at random with replacement from a box containing 10 black and 8 red balls. Find the probability that
		\label{ncert/12/13/2/12}
\begin{enumerate}
\item both balls are red.
\item first ball is black and second is red.
\item one of them is black and other is red.
\end{enumerate}

\item In a hostel, 60\% of the students read Hindi newspaper, 40\% read English newspaper and 20\% read both Hindi and English newspapers. A student is selected at random.
		\label{ncert/12/13/2/15}
\begin{enumerate}
\item Find the probability that she reads neither Hindi nor English newspapers.
\item If she reads Hindi newspaper, find the probability that she reads English newspaper.
\item If she reads English newspaper, find the probability that she reads Hindi newspaper.\\
\end{enumerate}
\item The probability of obtaining an even prime number on each die, when a pair of dice is rolled is 
\begin{enumerate}
    \item $0$ 
    
    \item $\frac{1}{3}$ 
    
    \item $\frac{1}{12}$ 
    
    \item $\frac{1}{36}$ 
\end{enumerate}
\solution
		%\input{ncert/12/13/2/17/defs.tex}
	\item A bag contains 4 red and 4 black balls, another bag contains 2 red and 6 black balls. One of the two bags is selected at random and a ball is drawn from the bag which is found to be red. Find the probability that the ball is drawn from the first bag.
\\
\solution
		%\input{ncert/12/13/3/2/main.tex}
  \item
  Cards with numbers 2 to 101 are placed in a box. A card is selected at random.Find the probability that the card has
\begin{enumerate}[label=(\roman*)]
	\item an even number 
	\item a square number
\end{enumerate}
\solution
%\input{exemplar/10/13/3/32/main.tex}
\item
The king, queen and jack of clubs are removed from a deck of 52 playing cards and then well shuffled. Now one card is drawn at random from the remaining cards.  Determine the probability that the card is
\begin{enumerate}[label=(\roman*)]
\item a club
\item 10 of hearts
\end{enumerate}
\solution
%\input{exemplar/10/13/3/29/main.tex}
\item A team of medical students doing their internship have to assist during surgeries
at a city hospital. The probabilities of surgeries rated as very complex, complex,
routine, simple or very simple are respectively, 0.15, 0.20, 0.31, 0.26, .08. Find
the probabilities that a particular surgery will be rated
\begin{enumerate}
	\item complex or very complex;
	\item neither very complex nor very simple;
	\item routine or complex
	\item routine or simple
\end{enumerate}
\solution
%\input{exemplar/11/16/3/8(1)/main.tex}
\item A card is selected from a pack of 52 cards.
\begin{enumerate}[label=(\alph*)]
    \item How many points are there in the sample space?
    \item Calculate the probability that the card is an ace of spades.
    \item Calculate the probability that the card is (i) an ace and (ii) black card.
\end{enumerate}
\solution
%\input{exemplar/11/16/3/4/main2.tex}
\item The probability that a non leap year selected at random will contain 53 sundays.
\\
\solution
%\input{exemplar/10/13/1/19/main.tex}
\item One of the four persons John, Rita, Aslam or Gurpreet will be promoted next
month. Consequently the sample space consists of four elementary outcomes
S = {John promoted, Rita promoted, Aslam promoted, Gurpreet promoted}
You are told that the chances of John’s promotion is same as that of Gurpreet,
Rita’s chances of promotion are twice as likely as Johns. Aslam’s chances are
four times that of John.
\begin{enumerate}
	\item Determine
	\begin{enumerate}
		\item P (John promoted)
		\item P (Rita promoted)
		\item P (Aslam promoted)
		\item P (Gurpreet promoted)
	\end{enumerate}
	\item If A = {John promoted or Gurpreet promoted}, find P (A).
\end{enumerate}
\solution
%\input{exemplar/11/16/3/10/main.tex}
\item A card is drawn from a deck of 52 cards. Find the probability of getting a king or a heart or a red card.\\
\solution
%\input{exemplar/11/16/3/15/main.tex}
\item The probability that a student will pass his examination is 0.73, the probability of
the student getting a compartment is 0.13, and the probability that the student will
either pass or get compartment is 0.96. State True or False.\\
\solution
%\input{exemplar/11/16/3/31/main.tex}
\item A card is selected from a pack of 52 cards\\
\begin{enumerate}[label=(\alph*)]
\item How many points are there in the sample space?
\item Calculate the probability that the cards is an ace of spades.
\item Calculate the probability that the card is (i) an ace (ii)black card.\\
\end{enumerate}
%\input{ncert/11/16/3/4_1/Prob_4.tex}
\item In a non-leap year, the probability of having 53 tuesdays or 53 wednesdays is\\
\solution
%\input{exemplar/11/16/3/18/main.tex}
\item There are 1000 sealed envelopes in a box, 10 of them contain a cash prize of
Rs 100 each, 100 of them contain a cash prize of Rs 50 each and 200 of them
contain a cash prize of Rs 10 each and rest do not contain any cash prize. If they
are well shuffled and an envelope is picked up out, what is the probability that it
contains no cash prize?\\
\solution
%\input{exemplar/10/13/3/34/main.tex}
\item 
A die is thrown and a card is selected at random from a deck of 52 playing cards. The probability of getting an even number on the die and a spade card.\\
\solution
%\input{exemplar/12/13/3/78/main.tex}
\item
If 4-digit numbers greater than 5,000 are randomly formed from the digits 0, 1, 3, 5, and 7, what is the probability of forming a number divisible by 5 when:
\begin{enumerate}
    \item The digits are repeated?
    \item The repetition of digits is not allowed?
\end{enumerate}
\solution
%\input{ncert/11/16/4/9/main.tex}
\item Consider the probability space $\brak{\Omega, \mathcal{G}, P}$ where $\Omega = [0,2]$ and $\mathcal{G} = \cbrak{\phi, \Omega, [0,1], (1,2]}$. Let $X$ and $Y$ be two functions on $\Omega$ defined as
\begin{align*}
    X(\omega) = 
    \begin{cases}
        1 & \text{if }\omega \in [0, 1]\\
        2 & \text{if }\omega \in (1, 2]
    \end{cases}
\end{align*}
and
\begin{align*}
    Y(\omega) = 
    \begin{cases}
        2 & \text{if }\omega \in [0, 1.5]\\
        3 & \text{if }\omega \in (1.5, 2].
    \end{cases}
\end{align*}
Then which one of the following statements is true?
\begin{enumerate}
    \item [(A)] $X$ is a random variable with respect to $\mathcal{G}$, but $Y$ is not a random variable with respect to $\mathcal{G}$.
    \item [(B)] $Y$ is a random variable with respect to $\mathcal{G}$, but $X$ is not a random variable with respect to $\mathcal{G}$.
    \item [(C)] Neither $X$ nor $Y$ is a random variable with respect to $\mathcal{G}$.
    \item [(D)] Both $X$ and $Y$ are random variables with respect to $\mathcal{G}$.
\end{enumerate} \hfill (GATE ST 2023)\\
\solution
%\input{gate/ST/2023/14/main.tex}
	\item  A die is loaded in such a way that each odd number is twice as likely to occur as
each even number. Find $P(G)$, where $G$ is the event that a number greater than
3 occurs on a single roll of the die.
\\
\solution
		%\input{exemplar/11/16/3/5/main.tex}
	\item All the jacks, queens and kings are removed from a deck of 52 playing cards. The remaining cards are well shuffled and then one card is drawn at random. Giving ace a value 1 similar value for other cards, find the probability that the card has a value 
		\begin{enumerate}
			\item 7
			\item greater than 7
			\item less than 7
		\end{enumerate}
		%\input{exemplar/10/13/3/30/main.tex}
  \item A Lot consists of 48 mobile phones of which 42 are good, 3 have only minor defects and 3 have major defects.Varnika will buy a phone if it is good but the trader will only buy a mobile if it has no major defects. One phone is selected at random from the lot. What is the probability that it is
\begin{enumerate}
	\item acceptable to Varnika?
            \item acceptable to the trader?
\end{enumerate}
\solution
	%\input{exemplar/10/13/3/40/main.tex}
 \item A student says that if you throw a die, it will show up 1 or not 1. Therefore, the probability of getting 1 and the probability of getting 'not 1' each is equal to $\frac{1}{2}$. Is this correct? Give reasons.\\
 \solution
        %\input{exemplar/10/13/2/9/main.tex}
   \item Four candidates A, B, C, D have ap-
plied for the assignment to coach a school cricket
team. If A is twice as likely to be selected as B, and
B and C are given about the same chance of being
selected, while C is twice as likely to be selected
as D, what are the probabilities that
\begin{enumerate}
\item C will be selected?
\item A will not be selected?
\end{enumerate}
	%\input{exemplar/11/16/3/9/main.tex}
 \item A bag contain 24 balls of which $x$ balls are red, $2x$ are white and $3x$ are blue. A ball is selected at random, What is the probability that it is
\begin{enumerate}[label=\alph*)]
\item not red ?
\item white ?
\end{enumerate}
%\input{exemplar/10/13/3/41/main.tex}
If the letters of the word ASSASSINATION are arranged at random. Find the Probability that
\begin{enumerate}[label=(\alph*)]
\item Four $S's$ come consecutively in the word
\item Two  $I's$ and two $N's$ come together
\item All $A's$ are not coming together
\item No two $A's$ are coming together
\end{enumerate}
%\input{exemplar/11/16/3/14/main.tex}
	\item One urn contains two black balls (labelled B1 and B2) and one white ball. A
	second urn contains one black ball and two white balls (labelled W1 and W2).
	Suppose the following experiment is performed. One of the two urns is chosen
	at random. Next a ball is randomly chosen from the urn. Then a second ball is
	chosen at random from the same urn without replacing the first ball.
	
	\begin{enumerate}
	\item What is the probability that two black balls are chosen?
	
	\item What is the probability that two balls of opposite colour are chosen?
	\end{enumerate}
	\solution
	%\input{exemplar/11/16/3/12/main1.tex}
\end{enumerate}

		\item A box of oranges is inspected by examining three randomly selected oranges drawn without replacement. If all the three oranges are good, the box is approved for sale, otherwise, it is rejected. Find the probability that a box containing 15 oranges out of which 12 are good and 3 are bad ones will be approved for sale.
		\label{ncert/12/13/2/3/defs.tex}
		\item Two balls are drawn at random with replacement from a box containing 10 black and 8 red balls. Find the probability that
		\label{ncert/12/13/2/12}
\begin{enumerate}
\item both balls are red.
\item first ball is black and second is red.
\item one of them is black and other is red.
\end{enumerate}

\item In a hostel, 60\% of the students read Hindi newspaper, 40\% read English newspaper and 20\% read both Hindi and English newspapers. A student is selected at random.
		\label{ncert/12/13/2/15}
\begin{enumerate}
\item Find the probability that she reads neither Hindi nor English newspapers.
\item If she reads Hindi newspaper, find the probability that she reads English newspaper.
\item If she reads English newspaper, find the probability that she reads Hindi newspaper.\\
\end{enumerate}
\item The probability of obtaining an even prime number on each die, when a pair of dice is rolled is 
\begin{enumerate}
    \item $0$ 
    
    \item $\frac{1}{3}$ 
    
    \item $\frac{1}{12}$ 
    
    \item $\frac{1}{36}$ 
\end{enumerate}
\solution
		%\begin{enumerate}[label=\thesection.\arabic*,ref=\thesection.\theenumi]
	\item One card is drawn from a well-shuffled deck of 52 cards. Find the probability of getting
\begin{enumerate}
\item A king of red colour 
\item A face card 
\item A red face card
\item The jack of hearts
\item A spade
\item The queen of diamonds

\end{enumerate}
\solution
		%\input{ncert/10/15/1/14/main.tex}
	\item Five cards—the ten, jack, queen, king and ace of diamonds, are well-shuffled with their face downwards. One card is then picked up at random.
\begin{enumerate}
\item
What is the probability that the card is the queen? 
\item
If the queen is drawn and put aside, what is the probability that the second card picked up is (a) an ace? (b) a queen?\\
\end{enumerate}
\solution
		%\input{ncert/10/15/1/15/defs.tex}
	\item A bag contains $5$ red balls and some blue balls. If the probability of drawing a blue ball is double that if a red ball, determine the number of blue balls in the bag. 
		\\
\solution
		%\input{ncert/10/15/2/3/defs.tex}
	\item A card is selected from a pack of 52 cards.
 \begin{enumerate}[label=(\alph*)] 
                 \item How many points are there in the sample space?
                 \item Calculate the probability that the card is an ace of spades.
                 \item Calculate the probability that the card is (i) an ace and (ii) black card.
 \end{enumerate}
\solution
		%\input{ncert/11/16/3/4/main.tex}
\item Four cards are drawn from a well-shuffled deck of 52 cards. What is the probability of obtaining 3 diamonds and one spade.
\\
\solution
		%\input{ncert/11/16/4/2/defs.tex}
\item In a certain lottery 10,000 tickets are sold and ten equal prizes are awarded. What is the probability of not getting a prize if you buy (a) one ticket (b) two tickets (c) 10 tickets ?	
\\
\solution
		%\input{ncert/11/16/4/4/defs.tex}
		%
\item 
Out of 100 students, two sections of 40 and 60 are formed. If you and your friend are among the 100 students, what is the probability that
\begin{enumerate}
\item you both enter the same section?
\item you both enter the different sections?
\end{enumerate}
\solution
		%\input{ncert/11/16/4/5/defs.tex}
	\item 
The number lock of a suitcase has 4 wheels each labelled with ten digits i.e. from 0 to 9.The lock opens with a sequence of four digits with no repeats.What is the probability of a person getting the right sequence to open the suitcase.
\\
\solution
		%\input{ncert/11/16/4/10/defs.tex}
		%
\item 
Two cards are drawn at random and without replacement from a pack of 52 playing cards. Find the probability that both the cards are black.
\\
\solution
		%\input{ncert/12/13/2/2/defs.tex}
		\item A box of oranges is inspected by examining three randomly selected oranges drawn without replacement. If all the three oranges are good, the box is approved for sale, otherwise, it is rejected. Find the probability that a box containing 15 oranges out of which 12 are good and 3 are bad ones will be approved for sale.
		\label{ncert/12/13/2/3/defs.tex}
		\item Two balls are drawn at random with replacement from a box containing 10 black and 8 red balls. Find the probability that
		\label{ncert/12/13/2/12}
\begin{enumerate}
\item both balls are red.
\item first ball is black and second is red.
\item one of them is black and other is red.
\end{enumerate}

\item In a hostel, 60\% of the students read Hindi newspaper, 40\% read English newspaper and 20\% read both Hindi and English newspapers. A student is selected at random.
		\label{ncert/12/13/2/15}
\begin{enumerate}
\item Find the probability that she reads neither Hindi nor English newspapers.
\item If she reads Hindi newspaper, find the probability that she reads English newspaper.
\item If she reads English newspaper, find the probability that she reads Hindi newspaper.\\
\end{enumerate}
\item The probability of obtaining an even prime number on each die, when a pair of dice is rolled is 
\begin{enumerate}
    \item $0$ 
    
    \item $\frac{1}{3}$ 
    
    \item $\frac{1}{12}$ 
    
    \item $\frac{1}{36}$ 
\end{enumerate}
\solution
		%\input{ncert/12/13/2/17/defs.tex}
	\item A bag contains 4 red and 4 black balls, another bag contains 2 red and 6 black balls. One of the two bags is selected at random and a ball is drawn from the bag which is found to be red. Find the probability that the ball is drawn from the first bag.
\\
\solution
		%\input{ncert/12/13/3/2/main.tex}
  \item
  Cards with numbers 2 to 101 are placed in a box. A card is selected at random.Find the probability that the card has
\begin{enumerate}[label=(\roman*)]
	\item an even number 
	\item a square number
\end{enumerate}
\solution
%\input{exemplar/10/13/3/32/main.tex}
\item
The king, queen and jack of clubs are removed from a deck of 52 playing cards and then well shuffled. Now one card is drawn at random from the remaining cards.  Determine the probability that the card is
\begin{enumerate}[label=(\roman*)]
\item a club
\item 10 of hearts
\end{enumerate}
\solution
%\input{exemplar/10/13/3/29/main.tex}
\item A team of medical students doing their internship have to assist during surgeries
at a city hospital. The probabilities of surgeries rated as very complex, complex,
routine, simple or very simple are respectively, 0.15, 0.20, 0.31, 0.26, .08. Find
the probabilities that a particular surgery will be rated
\begin{enumerate}
	\item complex or very complex;
	\item neither very complex nor very simple;
	\item routine or complex
	\item routine or simple
\end{enumerate}
\solution
%\input{exemplar/11/16/3/8(1)/main.tex}
\item A card is selected from a pack of 52 cards.
\begin{enumerate}[label=(\alph*)]
    \item How many points are there in the sample space?
    \item Calculate the probability that the card is an ace of spades.
    \item Calculate the probability that the card is (i) an ace and (ii) black card.
\end{enumerate}
\solution
%\input{exemplar/11/16/3/4/main2.tex}
\item The probability that a non leap year selected at random will contain 53 sundays.
\\
\solution
%\input{exemplar/10/13/1/19/main.tex}
\item One of the four persons John, Rita, Aslam or Gurpreet will be promoted next
month. Consequently the sample space consists of four elementary outcomes
S = {John promoted, Rita promoted, Aslam promoted, Gurpreet promoted}
You are told that the chances of John’s promotion is same as that of Gurpreet,
Rita’s chances of promotion are twice as likely as Johns. Aslam’s chances are
four times that of John.
\begin{enumerate}
	\item Determine
	\begin{enumerate}
		\item P (John promoted)
		\item P (Rita promoted)
		\item P (Aslam promoted)
		\item P (Gurpreet promoted)
	\end{enumerate}
	\item If A = {John promoted or Gurpreet promoted}, find P (A).
\end{enumerate}
\solution
%\input{exemplar/11/16/3/10/main.tex}
\item A card is drawn from a deck of 52 cards. Find the probability of getting a king or a heart or a red card.\\
\solution
%\input{exemplar/11/16/3/15/main.tex}
\item The probability that a student will pass his examination is 0.73, the probability of
the student getting a compartment is 0.13, and the probability that the student will
either pass or get compartment is 0.96. State True or False.\\
\solution
%\input{exemplar/11/16/3/31/main.tex}
\item A card is selected from a pack of 52 cards\\
\begin{enumerate}[label=(\alph*)]
\item How many points are there in the sample space?
\item Calculate the probability that the cards is an ace of spades.
\item Calculate the probability that the card is (i) an ace (ii)black card.\\
\end{enumerate}
%\input{ncert/11/16/3/4_1/Prob_4.tex}
\item In a non-leap year, the probability of having 53 tuesdays or 53 wednesdays is\\
\solution
%\input{exemplar/11/16/3/18/main.tex}
\item There are 1000 sealed envelopes in a box, 10 of them contain a cash prize of
Rs 100 each, 100 of them contain a cash prize of Rs 50 each and 200 of them
contain a cash prize of Rs 10 each and rest do not contain any cash prize. If they
are well shuffled and an envelope is picked up out, what is the probability that it
contains no cash prize?\\
\solution
%\input{exemplar/10/13/3/34/main.tex}
\item 
A die is thrown and a card is selected at random from a deck of 52 playing cards. The probability of getting an even number on the die and a spade card.\\
\solution
%\input{exemplar/12/13/3/78/main.tex}
\item
If 4-digit numbers greater than 5,000 are randomly formed from the digits 0, 1, 3, 5, and 7, what is the probability of forming a number divisible by 5 when:
\begin{enumerate}
    \item The digits are repeated?
    \item The repetition of digits is not allowed?
\end{enumerate}
\solution
%\input{ncert/11/16/4/9/main.tex}
\item Consider the probability space $\brak{\Omega, \mathcal{G}, P}$ where $\Omega = [0,2]$ and $\mathcal{G} = \cbrak{\phi, \Omega, [0,1], (1,2]}$. Let $X$ and $Y$ be two functions on $\Omega$ defined as
\begin{align*}
    X(\omega) = 
    \begin{cases}
        1 & \text{if }\omega \in [0, 1]\\
        2 & \text{if }\omega \in (1, 2]
    \end{cases}
\end{align*}
and
\begin{align*}
    Y(\omega) = 
    \begin{cases}
        2 & \text{if }\omega \in [0, 1.5]\\
        3 & \text{if }\omega \in (1.5, 2].
    \end{cases}
\end{align*}
Then which one of the following statements is true?
\begin{enumerate}
    \item [(A)] $X$ is a random variable with respect to $\mathcal{G}$, but $Y$ is not a random variable with respect to $\mathcal{G}$.
    \item [(B)] $Y$ is a random variable with respect to $\mathcal{G}$, but $X$ is not a random variable with respect to $\mathcal{G}$.
    \item [(C)] Neither $X$ nor $Y$ is a random variable with respect to $\mathcal{G}$.
    \item [(D)] Both $X$ and $Y$ are random variables with respect to $\mathcal{G}$.
\end{enumerate} \hfill (GATE ST 2023)\\
\solution
%\input{gate/ST/2023/14/main.tex}
	\item  A die is loaded in such a way that each odd number is twice as likely to occur as
each even number. Find $P(G)$, where $G$ is the event that a number greater than
3 occurs on a single roll of the die.
\\
\solution
		%\input{exemplar/11/16/3/5/main.tex}
	\item All the jacks, queens and kings are removed from a deck of 52 playing cards. The remaining cards are well shuffled and then one card is drawn at random. Giving ace a value 1 similar value for other cards, find the probability that the card has a value 
		\begin{enumerate}
			\item 7
			\item greater than 7
			\item less than 7
		\end{enumerate}
		%\input{exemplar/10/13/3/30/main.tex}
  \item A Lot consists of 48 mobile phones of which 42 are good, 3 have only minor defects and 3 have major defects.Varnika will buy a phone if it is good but the trader will only buy a mobile if it has no major defects. One phone is selected at random from the lot. What is the probability that it is
\begin{enumerate}
	\item acceptable to Varnika?
            \item acceptable to the trader?
\end{enumerate}
\solution
	%\input{exemplar/10/13/3/40/main.tex}
 \item A student says that if you throw a die, it will show up 1 or not 1. Therefore, the probability of getting 1 and the probability of getting 'not 1' each is equal to $\frac{1}{2}$. Is this correct? Give reasons.\\
 \solution
        %\input{exemplar/10/13/2/9/main.tex}
   \item Four candidates A, B, C, D have ap-
plied for the assignment to coach a school cricket
team. If A is twice as likely to be selected as B, and
B and C are given about the same chance of being
selected, while C is twice as likely to be selected
as D, what are the probabilities that
\begin{enumerate}
\item C will be selected?
\item A will not be selected?
\end{enumerate}
	%\input{exemplar/11/16/3/9/main.tex}
 \item A bag contain 24 balls of which $x$ balls are red, $2x$ are white and $3x$ are blue. A ball is selected at random, What is the probability that it is
\begin{enumerate}[label=\alph*)]
\item not red ?
\item white ?
\end{enumerate}
%\input{exemplar/10/13/3/41/main.tex}
If the letters of the word ASSASSINATION are arranged at random. Find the Probability that
\begin{enumerate}[label=(\alph*)]
\item Four $S's$ come consecutively in the word
\item Two  $I's$ and two $N's$ come together
\item All $A's$ are not coming together
\item No two $A's$ are coming together
\end{enumerate}
%\input{exemplar/11/16/3/14/main.tex}
	\item One urn contains two black balls (labelled B1 and B2) and one white ball. A
	second urn contains one black ball and two white balls (labelled W1 and W2).
	Suppose the following experiment is performed. One of the two urns is chosen
	at random. Next a ball is randomly chosen from the urn. Then a second ball is
	chosen at random from the same urn without replacing the first ball.
	
	\begin{enumerate}
	\item What is the probability that two black balls are chosen?
	
	\item What is the probability that two balls of opposite colour are chosen?
	\end{enumerate}
	\solution
	%\input{exemplar/11/16/3/12/main1.tex}
\end{enumerate}

	\item A bag contains 4 red and 4 black balls, another bag contains 2 red and 6 black balls. One of the two bags is selected at random and a ball is drawn from the bag which is found to be red. Find the probability that the ball is drawn from the first bag.
\\
\solution
		%\begin{table}[H]
	\centering
\begin{tabular}{|c|c|c|}
\hline
Random variable &Value &Definition\\ \hline
\multirow{3}{*}{X} &0 &Slips of Rs 1\\
&1 &Slips of Rs 5\\
&2 &Slips of Rs 13\\ \hline
\multirow{2}{*}{Y} &0 &Box A\\
&1 &Box B\\\hline
\end{tabular}
\caption{}
\label{tab:Distribution}
\end{table}
See \tabref{tab:Distribution}.
\begin{align}
p_{Y}\brak{k}= \begin{cases} 
      \frac{1}{3} & {k=0} \\
      \frac{2}{3 }& {k=1} 
   \end{cases}
   \\
p_{Y|X}\brak{0|0} = \frac{19}{25}\, 
p_{Y|X}\brak{0|1} = \frac{6}{25}\,
p_{Y|X}\brak{1|0} = \frac{45}{50}\,
p_{Y|X}\brak{1|2} = \frac{5}{50}
\end{align}
The desired probability is the probability that a slip drawn at random is marked other than Rs 1,
\begin{align}
&=1-p_X\brak{0}\\
&= p_X(1) + p_X(2)
\end{align}
Using Bayes theorem,
\begin{align}
&= p_Y\brak{0} \times \pr{Y=0 | X=1} + p_Y\brak{1} \times \pr{Y=1|X=2}\\
&=\frac{1}{3} \times \frac{6}{25} + \frac{2}{3} \times \frac{5}{50}\\
&=\frac{11}{75}
\end{align}

\newpage

%\tableofcontents

\bigskip

\renewcommand{\thefigure}{\theenumi}
\renewcommand{\thetable}{\theenumi}
%\renewcommand{\theequation}{\theenumi}

%\begin{abstract}
%%\boldmath
%In this letter, an algorithm for evaluating the exact analytical bit error rate  (BER)  for the piecewise linear (PL) combiner for  multiple relays is presented. Previous results were available only for upto three relays. The algorithm is unique in the sense that  the actual mathematical expressions, that are prohibitively large, need not be explicitly obtained. The diversity gain due to multiple relays is shown through plots of the analytical BER, well supported by simulations. 
%
%\end{abstract}
% IEEEtran.cls defaults to using nonbold math in the Abstract.
% This preserves the distinction between vectors and scalars. However,
% if the journal you are submitting to favors bold math in the abstract,
% then you can use LaTeX's standard command \boldmath at the very start
% of the abstract to achieve this. Many IEEE journals frown on math
% in the abstract anyway.

% Note that keywords are not normally used for peerreview papers.
%\begin{IEEEkeywords}
%Cooperative diversity, decode and forward, piecewise linear
%\end{IEEEkeywords}



% For peer review papers, you can put extra information on the cover
% page as needed:
% \ifCLASSOPTIONpeerreview
% \begin{center} \bfseries EDICS Category: 3-BBND \end{center}
% \fi
%
% For peerreview papers, this IEEEtran command inserts a page break and
% creates the second title. It will be ignored for other modes.
%\IEEEpeerreviewmaketitle




  \item
  Cards with numbers 2 to 101 are placed in a box. A card is selected at random.Find the probability that the card has
\begin{enumerate}[label=(\roman*)]
	\item an even number 
	\item a square number
\end{enumerate}
\solution
%\begin{table}[H]
	\centering
\begin{tabular}{|c|c|c|}
\hline
Random variable &Value &Definition\\ \hline
\multirow{3}{*}{X} &0 &Slips of Rs 1\\
&1 &Slips of Rs 5\\
&2 &Slips of Rs 13\\ \hline
\multirow{2}{*}{Y} &0 &Box A\\
&1 &Box B\\\hline
\end{tabular}
\caption{}
\label{tab:Distribution}
\end{table}
See \tabref{tab:Distribution}.
\begin{align}
p_{Y}\brak{k}= \begin{cases} 
      \frac{1}{3} & {k=0} \\
      \frac{2}{3 }& {k=1} 
   \end{cases}
   \\
p_{Y|X}\brak{0|0} = \frac{19}{25}\, 
p_{Y|X}\brak{0|1} = \frac{6}{25}\,
p_{Y|X}\brak{1|0} = \frac{45}{50}\,
p_{Y|X}\brak{1|2} = \frac{5}{50}
\end{align}
The desired probability is the probability that a slip drawn at random is marked other than Rs 1,
\begin{align}
&=1-p_X\brak{0}\\
&= p_X(1) + p_X(2)
\end{align}
Using Bayes theorem,
\begin{align}
&= p_Y\brak{0} \times \pr{Y=0 | X=1} + p_Y\brak{1} \times \pr{Y=1|X=2}\\
&=\frac{1}{3} \times \frac{6}{25} + \frac{2}{3} \times \frac{5}{50}\\
&=\frac{11}{75}
\end{align}

\newpage

%\tableofcontents

\bigskip

\renewcommand{\thefigure}{\theenumi}
\renewcommand{\thetable}{\theenumi}
%\renewcommand{\theequation}{\theenumi}

%\begin{abstract}
%%\boldmath
%In this letter, an algorithm for evaluating the exact analytical bit error rate  (BER)  for the piecewise linear (PL) combiner for  multiple relays is presented. Previous results were available only for upto three relays. The algorithm is unique in the sense that  the actual mathematical expressions, that are prohibitively large, need not be explicitly obtained. The diversity gain due to multiple relays is shown through plots of the analytical BER, well supported by simulations. 
%
%\end{abstract}
% IEEEtran.cls defaults to using nonbold math in the Abstract.
% This preserves the distinction between vectors and scalars. However,
% if the journal you are submitting to favors bold math in the abstract,
% then you can use LaTeX's standard command \boldmath at the very start
% of the abstract to achieve this. Many IEEE journals frown on math
% in the abstract anyway.

% Note that keywords are not normally used for peerreview papers.
%\begin{IEEEkeywords}
%Cooperative diversity, decode and forward, piecewise linear
%\end{IEEEkeywords}



% For peer review papers, you can put extra information on the cover
% page as needed:
% \ifCLASSOPTIONpeerreview
% \begin{center} \bfseries EDICS Category: 3-BBND \end{center}
% \fi
%
% For peerreview papers, this IEEEtran command inserts a page break and
% creates the second title. It will be ignored for other modes.
%\IEEEpeerreviewmaketitle




\item
The king, queen and jack of clubs are removed from a deck of 52 playing cards and then well shuffled. Now one card is drawn at random from the remaining cards.  Determine the probability that the card is
\begin{enumerate}[label=(\roman*)]
\item a club
\item 10 of hearts
\end{enumerate}
\solution
%\begin{table}[H]
	\centering
\begin{tabular}{|c|c|c|}
\hline
Random variable &Value &Definition\\ \hline
\multirow{3}{*}{X} &0 &Slips of Rs 1\\
&1 &Slips of Rs 5\\
&2 &Slips of Rs 13\\ \hline
\multirow{2}{*}{Y} &0 &Box A\\
&1 &Box B\\\hline
\end{tabular}
\caption{}
\label{tab:Distribution}
\end{table}
See \tabref{tab:Distribution}.
\begin{align}
p_{Y}\brak{k}= \begin{cases} 
      \frac{1}{3} & {k=0} \\
      \frac{2}{3 }& {k=1} 
   \end{cases}
   \\
p_{Y|X}\brak{0|0} = \frac{19}{25}\, 
p_{Y|X}\brak{0|1} = \frac{6}{25}\,
p_{Y|X}\brak{1|0} = \frac{45}{50}\,
p_{Y|X}\brak{1|2} = \frac{5}{50}
\end{align}
The desired probability is the probability that a slip drawn at random is marked other than Rs 1,
\begin{align}
&=1-p_X\brak{0}\\
&= p_X(1) + p_X(2)
\end{align}
Using Bayes theorem,
\begin{align}
&= p_Y\brak{0} \times \pr{Y=0 | X=1} + p_Y\brak{1} \times \pr{Y=1|X=2}\\
&=\frac{1}{3} \times \frac{6}{25} + \frac{2}{3} \times \frac{5}{50}\\
&=\frac{11}{75}
\end{align}

\newpage

%\tableofcontents

\bigskip

\renewcommand{\thefigure}{\theenumi}
\renewcommand{\thetable}{\theenumi}
%\renewcommand{\theequation}{\theenumi}

%\begin{abstract}
%%\boldmath
%In this letter, an algorithm for evaluating the exact analytical bit error rate  (BER)  for the piecewise linear (PL) combiner for  multiple relays is presented. Previous results were available only for upto three relays. The algorithm is unique in the sense that  the actual mathematical expressions, that are prohibitively large, need not be explicitly obtained. The diversity gain due to multiple relays is shown through plots of the analytical BER, well supported by simulations. 
%
%\end{abstract}
% IEEEtran.cls defaults to using nonbold math in the Abstract.
% This preserves the distinction between vectors and scalars. However,
% if the journal you are submitting to favors bold math in the abstract,
% then you can use LaTeX's standard command \boldmath at the very start
% of the abstract to achieve this. Many IEEE journals frown on math
% in the abstract anyway.

% Note that keywords are not normally used for peerreview papers.
%\begin{IEEEkeywords}
%Cooperative diversity, decode and forward, piecewise linear
%\end{IEEEkeywords}



% For peer review papers, you can put extra information on the cover
% page as needed:
% \ifCLASSOPTIONpeerreview
% \begin{center} \bfseries EDICS Category: 3-BBND \end{center}
% \fi
%
% For peerreview papers, this IEEEtran command inserts a page break and
% creates the second title. It will be ignored for other modes.
%\IEEEpeerreviewmaketitle




\item A team of medical students doing their internship have to assist during surgeries
at a city hospital. The probabilities of surgeries rated as very complex, complex,
routine, simple or very simple are respectively, 0.15, 0.20, 0.31, 0.26, .08. Find
the probabilities that a particular surgery will be rated
\begin{enumerate}
	\item complex or very complex;
	\item neither very complex nor very simple;
	\item routine or complex
	\item routine or simple
\end{enumerate}
\solution
%\begin{table}[H]
	\centering
\begin{tabular}{|c|c|c|}
\hline
Random variable &Value &Definition\\ \hline
\multirow{3}{*}{X} &0 &Slips of Rs 1\\
&1 &Slips of Rs 5\\
&2 &Slips of Rs 13\\ \hline
\multirow{2}{*}{Y} &0 &Box A\\
&1 &Box B\\\hline
\end{tabular}
\caption{}
\label{tab:Distribution}
\end{table}
See \tabref{tab:Distribution}.
\begin{align}
p_{Y}\brak{k}= \begin{cases} 
      \frac{1}{3} & {k=0} \\
      \frac{2}{3 }& {k=1} 
   \end{cases}
   \\
p_{Y|X}\brak{0|0} = \frac{19}{25}\, 
p_{Y|X}\brak{0|1} = \frac{6}{25}\,
p_{Y|X}\brak{1|0} = \frac{45}{50}\,
p_{Y|X}\brak{1|2} = \frac{5}{50}
\end{align}
The desired probability is the probability that a slip drawn at random is marked other than Rs 1,
\begin{align}
&=1-p_X\brak{0}\\
&= p_X(1) + p_X(2)
\end{align}
Using Bayes theorem,
\begin{align}
&= p_Y\brak{0} \times \pr{Y=0 | X=1} + p_Y\brak{1} \times \pr{Y=1|X=2}\\
&=\frac{1}{3} \times \frac{6}{25} + \frac{2}{3} \times \frac{5}{50}\\
&=\frac{11}{75}
\end{align}

\newpage

%\tableofcontents

\bigskip

\renewcommand{\thefigure}{\theenumi}
\renewcommand{\thetable}{\theenumi}
%\renewcommand{\theequation}{\theenumi}

%\begin{abstract}
%%\boldmath
%In this letter, an algorithm for evaluating the exact analytical bit error rate  (BER)  for the piecewise linear (PL) combiner for  multiple relays is presented. Previous results were available only for upto three relays. The algorithm is unique in the sense that  the actual mathematical expressions, that are prohibitively large, need not be explicitly obtained. The diversity gain due to multiple relays is shown through plots of the analytical BER, well supported by simulations. 
%
%\end{abstract}
% IEEEtran.cls defaults to using nonbold math in the Abstract.
% This preserves the distinction between vectors and scalars. However,
% if the journal you are submitting to favors bold math in the abstract,
% then you can use LaTeX's standard command \boldmath at the very start
% of the abstract to achieve this. Many IEEE journals frown on math
% in the abstract anyway.

% Note that keywords are not normally used for peerreview papers.
%\begin{IEEEkeywords}
%Cooperative diversity, decode and forward, piecewise linear
%\end{IEEEkeywords}



% For peer review papers, you can put extra information on the cover
% page as needed:
% \ifCLASSOPTIONpeerreview
% \begin{center} \bfseries EDICS Category: 3-BBND \end{center}
% \fi
%
% For peerreview papers, this IEEEtran command inserts a page break and
% creates the second title. It will be ignored for other modes.
%\IEEEpeerreviewmaketitle




\item A card is selected from a pack of 52 cards.
\begin{enumerate}[label=(\alph*)]
    \item How many points are there in the sample space?
    \item Calculate the probability that the card is an ace of spades.
    \item Calculate the probability that the card is (i) an ace and (ii) black card.
\end{enumerate}
\solution
%Let $X$ be an bernoulli rv defined as in \tabref{tab:exemplar/11/16/3/26}.  Then, 
\begin{equation}
    p =
        \frac{4}{11} 
\end{equation}
\begin{table}[H]
	\centering
	\input{exemplar/11/16/3/26/tables/Table2.tex}
	\caption{}
        \label{tab:exemplar/11/16/3/26}
\end{table}

\item The probability that a non leap year selected at random will contain 53 sundays.
\\
\solution
%\begin{table}[H]
	\centering
\begin{tabular}{|c|c|c|}
\hline
Random variable &Value &Definition\\ \hline
\multirow{3}{*}{X} &0 &Slips of Rs 1\\
&1 &Slips of Rs 5\\
&2 &Slips of Rs 13\\ \hline
\multirow{2}{*}{Y} &0 &Box A\\
&1 &Box B\\\hline
\end{tabular}
\caption{}
\label{tab:Distribution}
\end{table}
See \tabref{tab:Distribution}.
\begin{align}
p_{Y}\brak{k}= \begin{cases} 
      \frac{1}{3} & {k=0} \\
      \frac{2}{3 }& {k=1} 
   \end{cases}
   \\
p_{Y|X}\brak{0|0} = \frac{19}{25}\, 
p_{Y|X}\brak{0|1} = \frac{6}{25}\,
p_{Y|X}\brak{1|0} = \frac{45}{50}\,
p_{Y|X}\brak{1|2} = \frac{5}{50}
\end{align}
The desired probability is the probability that a slip drawn at random is marked other than Rs 1,
\begin{align}
&=1-p_X\brak{0}\\
&= p_X(1) + p_X(2)
\end{align}
Using Bayes theorem,
\begin{align}
&= p_Y\brak{0} \times \pr{Y=0 | X=1} + p_Y\brak{1} \times \pr{Y=1|X=2}\\
&=\frac{1}{3} \times \frac{6}{25} + \frac{2}{3} \times \frac{5}{50}\\
&=\frac{11}{75}
\end{align}

\newpage

%\tableofcontents

\bigskip

\renewcommand{\thefigure}{\theenumi}
\renewcommand{\thetable}{\theenumi}
%\renewcommand{\theequation}{\theenumi}

%\begin{abstract}
%%\boldmath
%In this letter, an algorithm for evaluating the exact analytical bit error rate  (BER)  for the piecewise linear (PL) combiner for  multiple relays is presented. Previous results were available only for upto three relays. The algorithm is unique in the sense that  the actual mathematical expressions, that are prohibitively large, need not be explicitly obtained. The diversity gain due to multiple relays is shown through plots of the analytical BER, well supported by simulations. 
%
%\end{abstract}
% IEEEtran.cls defaults to using nonbold math in the Abstract.
% This preserves the distinction between vectors and scalars. However,
% if the journal you are submitting to favors bold math in the abstract,
% then you can use LaTeX's standard command \boldmath at the very start
% of the abstract to achieve this. Many IEEE journals frown on math
% in the abstract anyway.

% Note that keywords are not normally used for peerreview papers.
%\begin{IEEEkeywords}
%Cooperative diversity, decode and forward, piecewise linear
%\end{IEEEkeywords}



% For peer review papers, you can put extra information on the cover
% page as needed:
% \ifCLASSOPTIONpeerreview
% \begin{center} \bfseries EDICS Category: 3-BBND \end{center}
% \fi
%
% For peerreview papers, this IEEEtran command inserts a page break and
% creates the second title. It will be ignored for other modes.
%\IEEEpeerreviewmaketitle




\item One of the four persons John, Rita, Aslam or Gurpreet will be promoted next
month. Consequently the sample space consists of four elementary outcomes
S = {John promoted, Rita promoted, Aslam promoted, Gurpreet promoted}
You are told that the chances of John’s promotion is same as that of Gurpreet,
Rita’s chances of promotion are twice as likely as Johns. Aslam’s chances are
four times that of John.
\begin{enumerate}
	\item Determine
	\begin{enumerate}
		\item P (John promoted)
		\item P (Rita promoted)
		\item P (Aslam promoted)
		\item P (Gurpreet promoted)
	\end{enumerate}
	\item If A = {John promoted or Gurpreet promoted}, find P (A).
\end{enumerate}
\solution
%\begin{table}[H]
	\centering
\begin{tabular}{|c|c|c|}
\hline
Random variable &Value &Definition\\ \hline
\multirow{3}{*}{X} &0 &Slips of Rs 1\\
&1 &Slips of Rs 5\\
&2 &Slips of Rs 13\\ \hline
\multirow{2}{*}{Y} &0 &Box A\\
&1 &Box B\\\hline
\end{tabular}
\caption{}
\label{tab:Distribution}
\end{table}
See \tabref{tab:Distribution}.
\begin{align}
p_{Y}\brak{k}= \begin{cases} 
      \frac{1}{3} & {k=0} \\
      \frac{2}{3 }& {k=1} 
   \end{cases}
   \\
p_{Y|X}\brak{0|0} = \frac{19}{25}\, 
p_{Y|X}\brak{0|1} = \frac{6}{25}\,
p_{Y|X}\brak{1|0} = \frac{45}{50}\,
p_{Y|X}\brak{1|2} = \frac{5}{50}
\end{align}
The desired probability is the probability that a slip drawn at random is marked other than Rs 1,
\begin{align}
&=1-p_X\brak{0}\\
&= p_X(1) + p_X(2)
\end{align}
Using Bayes theorem,
\begin{align}
&= p_Y\brak{0} \times \pr{Y=0 | X=1} + p_Y\brak{1} \times \pr{Y=1|X=2}\\
&=\frac{1}{3} \times \frac{6}{25} + \frac{2}{3} \times \frac{5}{50}\\
&=\frac{11}{75}
\end{align}

\newpage

%\tableofcontents

\bigskip

\renewcommand{\thefigure}{\theenumi}
\renewcommand{\thetable}{\theenumi}
%\renewcommand{\theequation}{\theenumi}

%\begin{abstract}
%%\boldmath
%In this letter, an algorithm for evaluating the exact analytical bit error rate  (BER)  for the piecewise linear (PL) combiner for  multiple relays is presented. Previous results were available only for upto three relays. The algorithm is unique in the sense that  the actual mathematical expressions, that are prohibitively large, need not be explicitly obtained. The diversity gain due to multiple relays is shown through plots of the analytical BER, well supported by simulations. 
%
%\end{abstract}
% IEEEtran.cls defaults to using nonbold math in the Abstract.
% This preserves the distinction between vectors and scalars. However,
% if the journal you are submitting to favors bold math in the abstract,
% then you can use LaTeX's standard command \boldmath at the very start
% of the abstract to achieve this. Many IEEE journals frown on math
% in the abstract anyway.

% Note that keywords are not normally used for peerreview papers.
%\begin{IEEEkeywords}
%Cooperative diversity, decode and forward, piecewise linear
%\end{IEEEkeywords}



% For peer review papers, you can put extra information on the cover
% page as needed:
% \ifCLASSOPTIONpeerreview
% \begin{center} \bfseries EDICS Category: 3-BBND \end{center}
% \fi
%
% For peerreview papers, this IEEEtran command inserts a page break and
% creates the second title. It will be ignored for other modes.
%\IEEEpeerreviewmaketitle




\item A card is drawn from a deck of 52 cards. Find the probability of getting a king or a heart or a red card.\\
\solution
%\begin{table}[H]
	\centering
\begin{tabular}{|c|c|c|}
\hline
Random variable &Value &Definition\\ \hline
\multirow{3}{*}{X} &0 &Slips of Rs 1\\
&1 &Slips of Rs 5\\
&2 &Slips of Rs 13\\ \hline
\multirow{2}{*}{Y} &0 &Box A\\
&1 &Box B\\\hline
\end{tabular}
\caption{}
\label{tab:Distribution}
\end{table}
See \tabref{tab:Distribution}.
\begin{align}
p_{Y}\brak{k}= \begin{cases} 
      \frac{1}{3} & {k=0} \\
      \frac{2}{3 }& {k=1} 
   \end{cases}
   \\
p_{Y|X}\brak{0|0} = \frac{19}{25}\, 
p_{Y|X}\brak{0|1} = \frac{6}{25}\,
p_{Y|X}\brak{1|0} = \frac{45}{50}\,
p_{Y|X}\brak{1|2} = \frac{5}{50}
\end{align}
The desired probability is the probability that a slip drawn at random is marked other than Rs 1,
\begin{align}
&=1-p_X\brak{0}\\
&= p_X(1) + p_X(2)
\end{align}
Using Bayes theorem,
\begin{align}
&= p_Y\brak{0} \times \pr{Y=0 | X=1} + p_Y\brak{1} \times \pr{Y=1|X=2}\\
&=\frac{1}{3} \times \frac{6}{25} + \frac{2}{3} \times \frac{5}{50}\\
&=\frac{11}{75}
\end{align}

\newpage

%\tableofcontents

\bigskip

\renewcommand{\thefigure}{\theenumi}
\renewcommand{\thetable}{\theenumi}
%\renewcommand{\theequation}{\theenumi}

%\begin{abstract}
%%\boldmath
%In this letter, an algorithm for evaluating the exact analytical bit error rate  (BER)  for the piecewise linear (PL) combiner for  multiple relays is presented. Previous results were available only for upto three relays. The algorithm is unique in the sense that  the actual mathematical expressions, that are prohibitively large, need not be explicitly obtained. The diversity gain due to multiple relays is shown through plots of the analytical BER, well supported by simulations. 
%
%\end{abstract}
% IEEEtran.cls defaults to using nonbold math in the Abstract.
% This preserves the distinction between vectors and scalars. However,
% if the journal you are submitting to favors bold math in the abstract,
% then you can use LaTeX's standard command \boldmath at the very start
% of the abstract to achieve this. Many IEEE journals frown on math
% in the abstract anyway.

% Note that keywords are not normally used for peerreview papers.
%\begin{IEEEkeywords}
%Cooperative diversity, decode and forward, piecewise linear
%\end{IEEEkeywords}



% For peer review papers, you can put extra information on the cover
% page as needed:
% \ifCLASSOPTIONpeerreview
% \begin{center} \bfseries EDICS Category: 3-BBND \end{center}
% \fi
%
% For peerreview papers, this IEEEtran command inserts a page break and
% creates the second title. It will be ignored for other modes.
%\IEEEpeerreviewmaketitle




\item The probability that a student will pass his examination is 0.73, the probability of
the student getting a compartment is 0.13, and the probability that the student will
either pass or get compartment is 0.96. State True or False.\\
\solution
%\begin{table}[H]
	\centering
\begin{tabular}{|c|c|c|}
\hline
Random variable &Value &Definition\\ \hline
\multirow{3}{*}{X} &0 &Slips of Rs 1\\
&1 &Slips of Rs 5\\
&2 &Slips of Rs 13\\ \hline
\multirow{2}{*}{Y} &0 &Box A\\
&1 &Box B\\\hline
\end{tabular}
\caption{}
\label{tab:Distribution}
\end{table}
See \tabref{tab:Distribution}.
\begin{align}
p_{Y}\brak{k}= \begin{cases} 
      \frac{1}{3} & {k=0} \\
      \frac{2}{3 }& {k=1} 
   \end{cases}
   \\
p_{Y|X}\brak{0|0} = \frac{19}{25}\, 
p_{Y|X}\brak{0|1} = \frac{6}{25}\,
p_{Y|X}\brak{1|0} = \frac{45}{50}\,
p_{Y|X}\brak{1|2} = \frac{5}{50}
\end{align}
The desired probability is the probability that a slip drawn at random is marked other than Rs 1,
\begin{align}
&=1-p_X\brak{0}\\
&= p_X(1) + p_X(2)
\end{align}
Using Bayes theorem,
\begin{align}
&= p_Y\brak{0} \times \pr{Y=0 | X=1} + p_Y\brak{1} \times \pr{Y=1|X=2}\\
&=\frac{1}{3} \times \frac{6}{25} + \frac{2}{3} \times \frac{5}{50}\\
&=\frac{11}{75}
\end{align}

\newpage

%\tableofcontents

\bigskip

\renewcommand{\thefigure}{\theenumi}
\renewcommand{\thetable}{\theenumi}
%\renewcommand{\theequation}{\theenumi}

%\begin{abstract}
%%\boldmath
%In this letter, an algorithm for evaluating the exact analytical bit error rate  (BER)  for the piecewise linear (PL) combiner for  multiple relays is presented. Previous results were available only for upto three relays. The algorithm is unique in the sense that  the actual mathematical expressions, that are prohibitively large, need not be explicitly obtained. The diversity gain due to multiple relays is shown through plots of the analytical BER, well supported by simulations. 
%
%\end{abstract}
% IEEEtran.cls defaults to using nonbold math in the Abstract.
% This preserves the distinction between vectors and scalars. However,
% if the journal you are submitting to favors bold math in the abstract,
% then you can use LaTeX's standard command \boldmath at the very start
% of the abstract to achieve this. Many IEEE journals frown on math
% in the abstract anyway.

% Note that keywords are not normally used for peerreview papers.
%\begin{IEEEkeywords}
%Cooperative diversity, decode and forward, piecewise linear
%\end{IEEEkeywords}



% For peer review papers, you can put extra information on the cover
% page as needed:
% \ifCLASSOPTIONpeerreview
% \begin{center} \bfseries EDICS Category: 3-BBND \end{center}
% \fi
%
% For peerreview papers, this IEEEtran command inserts a page break and
% creates the second title. It will be ignored for other modes.
%\IEEEpeerreviewmaketitle




\item A card is selected from a pack of 52 cards\\
\begin{enumerate}[label=(\alph*)]
\item How many points are there in the sample space?
\item Calculate the probability that the cards is an ace of spades.
\item Calculate the probability that the card is (i) an ace (ii)black card.\\
\end{enumerate}
%\input{ncert/11/16/3/4_1/Prob_4.tex}
\item In a non-leap year, the probability of having 53 tuesdays or 53 wednesdays is\\
\solution
%A non-leap year has a total of 365 days, and a week has 7 days.\\
So it can be expressed as 
\begin{align}
365\text{days} &=52\times 7+1 \text{day}
\end{align}
$\implies$ 52 tuesdays or wednesdays\\
Random variable X denotes the days of a week
\begin{align}
p_X\brak{k}&=\frac{1}{7}; \quad \brak{1<k<7}
\end{align}
So the probability of extra day being tuesday or wednesday is
\begin{align}
p_X\brak{3}+p_X\brak{4}&=\frac{1}{7}+\frac{1}{7}=\frac{2}{7}
\end{align}



\item There are 1000 sealed envelopes in a box, 10 of them contain a cash prize of
Rs 100 each, 100 of them contain a cash prize of Rs 50 each and 200 of them
contain a cash prize of Rs 10 each and rest do not contain any cash prize. If they
are well shuffled and an envelope is picked up out, what is the probability that it
contains no cash prize?\\
\solution
%\begin{table}[H]
	\centering
\begin{tabular}{|c|c|c|}
\hline
Random variable &Value &Definition\\ \hline
\multirow{3}{*}{X} &0 &Slips of Rs 1\\
&1 &Slips of Rs 5\\
&2 &Slips of Rs 13\\ \hline
\multirow{2}{*}{Y} &0 &Box A\\
&1 &Box B\\\hline
\end{tabular}
\caption{}
\label{tab:Distribution}
\end{table}
See \tabref{tab:Distribution}.
\begin{align}
p_{Y}\brak{k}= \begin{cases} 
      \frac{1}{3} & {k=0} \\
      \frac{2}{3 }& {k=1} 
   \end{cases}
   \\
p_{Y|X}\brak{0|0} = \frac{19}{25}\, 
p_{Y|X}\brak{0|1} = \frac{6}{25}\,
p_{Y|X}\brak{1|0} = \frac{45}{50}\,
p_{Y|X}\brak{1|2} = \frac{5}{50}
\end{align}
The desired probability is the probability that a slip drawn at random is marked other than Rs 1,
\begin{align}
&=1-p_X\brak{0}\\
&= p_X(1) + p_X(2)
\end{align}
Using Bayes theorem,
\begin{align}
&= p_Y\brak{0} \times \pr{Y=0 | X=1} + p_Y\brak{1} \times \pr{Y=1|X=2}\\
&=\frac{1}{3} \times \frac{6}{25} + \frac{2}{3} \times \frac{5}{50}\\
&=\frac{11}{75}
\end{align}

\newpage

%\tableofcontents

\bigskip

\renewcommand{\thefigure}{\theenumi}
\renewcommand{\thetable}{\theenumi}
%\renewcommand{\theequation}{\theenumi}

%\begin{abstract}
%%\boldmath
%In this letter, an algorithm for evaluating the exact analytical bit error rate  (BER)  for the piecewise linear (PL) combiner for  multiple relays is presented. Previous results were available only for upto three relays. The algorithm is unique in the sense that  the actual mathematical expressions, that are prohibitively large, need not be explicitly obtained. The diversity gain due to multiple relays is shown through plots of the analytical BER, well supported by simulations. 
%
%\end{abstract}
% IEEEtran.cls defaults to using nonbold math in the Abstract.
% This preserves the distinction between vectors and scalars. However,
% if the journal you are submitting to favors bold math in the abstract,
% then you can use LaTeX's standard command \boldmath at the very start
% of the abstract to achieve this. Many IEEE journals frown on math
% in the abstract anyway.

% Note that keywords are not normally used for peerreview papers.
%\begin{IEEEkeywords}
%Cooperative diversity, decode and forward, piecewise linear
%\end{IEEEkeywords}



% For peer review papers, you can put extra information on the cover
% page as needed:
% \ifCLASSOPTIONpeerreview
% \begin{center} \bfseries EDICS Category: 3-BBND \end{center}
% \fi
%
% For peerreview papers, this IEEEtran command inserts a page break and
% creates the second title. It will be ignored for other modes.
%\IEEEpeerreviewmaketitle




\item 
A die is thrown and a card is selected at random from a deck of 52 playing cards. The probability of getting an even number on the die and a spade card.\\
\solution
%\begin{table}[H]
	\centering
\begin{tabular}{|c|c|c|}
\hline
Random variable &Value &Definition\\ \hline
\multirow{3}{*}{X} &0 &Slips of Rs 1\\
&1 &Slips of Rs 5\\
&2 &Slips of Rs 13\\ \hline
\multirow{2}{*}{Y} &0 &Box A\\
&1 &Box B\\\hline
\end{tabular}
\caption{}
\label{tab:Distribution}
\end{table}
See \tabref{tab:Distribution}.
\begin{align}
p_{Y}\brak{k}= \begin{cases} 
      \frac{1}{3} & {k=0} \\
      \frac{2}{3 }& {k=1} 
   \end{cases}
   \\
p_{Y|X}\brak{0|0} = \frac{19}{25}\, 
p_{Y|X}\brak{0|1} = \frac{6}{25}\,
p_{Y|X}\brak{1|0} = \frac{45}{50}\,
p_{Y|X}\brak{1|2} = \frac{5}{50}
\end{align}
The desired probability is the probability that a slip drawn at random is marked other than Rs 1,
\begin{align}
&=1-p_X\brak{0}\\
&= p_X(1) + p_X(2)
\end{align}
Using Bayes theorem,
\begin{align}
&= p_Y\brak{0} \times \pr{Y=0 | X=1} + p_Y\brak{1} \times \pr{Y=1|X=2}\\
&=\frac{1}{3} \times \frac{6}{25} + \frac{2}{3} \times \frac{5}{50}\\
&=\frac{11}{75}
\end{align}

\newpage

%\tableofcontents

\bigskip

\renewcommand{\thefigure}{\theenumi}
\renewcommand{\thetable}{\theenumi}
%\renewcommand{\theequation}{\theenumi}

%\begin{abstract}
%%\boldmath
%In this letter, an algorithm for evaluating the exact analytical bit error rate  (BER)  for the piecewise linear (PL) combiner for  multiple relays is presented. Previous results were available only for upto three relays. The algorithm is unique in the sense that  the actual mathematical expressions, that are prohibitively large, need not be explicitly obtained. The diversity gain due to multiple relays is shown through plots of the analytical BER, well supported by simulations. 
%
%\end{abstract}
% IEEEtran.cls defaults to using nonbold math in the Abstract.
% This preserves the distinction between vectors and scalars. However,
% if the journal you are submitting to favors bold math in the abstract,
% then you can use LaTeX's standard command \boldmath at the very start
% of the abstract to achieve this. Many IEEE journals frown on math
% in the abstract anyway.

% Note that keywords are not normally used for peerreview papers.
%\begin{IEEEkeywords}
%Cooperative diversity, decode and forward, piecewise linear
%\end{IEEEkeywords}



% For peer review papers, you can put extra information on the cover
% page as needed:
% \ifCLASSOPTIONpeerreview
% \begin{center} \bfseries EDICS Category: 3-BBND \end{center}
% \fi
%
% For peerreview papers, this IEEEtran command inserts a page break and
% creates the second title. It will be ignored for other modes.
%\IEEEpeerreviewmaketitle




\item
If 4-digit numbers greater than 5,000 are randomly formed from the digits 0, 1, 3, 5, and 7, what is the probability of forming a number divisible by 5 when:
\begin{enumerate}
    \item The digits are repeated?
    \item The repetition of digits is not allowed?
\end{enumerate}
\solution
%\begin{table}[H]
	\centering
\begin{tabular}{|c|c|c|}
\hline
Random variable &Value &Definition\\ \hline
\multirow{3}{*}{X} &0 &Slips of Rs 1\\
&1 &Slips of Rs 5\\
&2 &Slips of Rs 13\\ \hline
\multirow{2}{*}{Y} &0 &Box A\\
&1 &Box B\\\hline
\end{tabular}
\caption{}
\label{tab:Distribution}
\end{table}
See \tabref{tab:Distribution}.
\begin{align}
p_{Y}\brak{k}= \begin{cases} 
      \frac{1}{3} & {k=0} \\
      \frac{2}{3 }& {k=1} 
   \end{cases}
   \\
p_{Y|X}\brak{0|0} = \frac{19}{25}\, 
p_{Y|X}\brak{0|1} = \frac{6}{25}\,
p_{Y|X}\brak{1|0} = \frac{45}{50}\,
p_{Y|X}\brak{1|2} = \frac{5}{50}
\end{align}
The desired probability is the probability that a slip drawn at random is marked other than Rs 1,
\begin{align}
&=1-p_X\brak{0}\\
&= p_X(1) + p_X(2)
\end{align}
Using Bayes theorem,
\begin{align}
&= p_Y\brak{0} \times \pr{Y=0 | X=1} + p_Y\brak{1} \times \pr{Y=1|X=2}\\
&=\frac{1}{3} \times \frac{6}{25} + \frac{2}{3} \times \frac{5}{50}\\
&=\frac{11}{75}
\end{align}

\newpage

%\tableofcontents

\bigskip

\renewcommand{\thefigure}{\theenumi}
\renewcommand{\thetable}{\theenumi}
%\renewcommand{\theequation}{\theenumi}

%\begin{abstract}
%%\boldmath
%In this letter, an algorithm for evaluating the exact analytical bit error rate  (BER)  for the piecewise linear (PL) combiner for  multiple relays is presented. Previous results were available only for upto three relays. The algorithm is unique in the sense that  the actual mathematical expressions, that are prohibitively large, need not be explicitly obtained. The diversity gain due to multiple relays is shown through plots of the analytical BER, well supported by simulations. 
%
%\end{abstract}
% IEEEtran.cls defaults to using nonbold math in the Abstract.
% This preserves the distinction between vectors and scalars. However,
% if the journal you are submitting to favors bold math in the abstract,
% then you can use LaTeX's standard command \boldmath at the very start
% of the abstract to achieve this. Many IEEE journals frown on math
% in the abstract anyway.

% Note that keywords are not normally used for peerreview papers.
%\begin{IEEEkeywords}
%Cooperative diversity, decode and forward, piecewise linear
%\end{IEEEkeywords}



% For peer review papers, you can put extra information on the cover
% page as needed:
% \ifCLASSOPTIONpeerreview
% \begin{center} \bfseries EDICS Category: 3-BBND \end{center}
% \fi
%
% For peerreview papers, this IEEEtran command inserts a page break and
% creates the second title. It will be ignored for other modes.
%\IEEEpeerreviewmaketitle




\item Consider the probability space $\brak{\Omega, \mathcal{G}, P}$ where $\Omega = [0,2]$ and $\mathcal{G} = \cbrak{\phi, \Omega, [0,1], (1,2]}$. Let $X$ and $Y$ be two functions on $\Omega$ defined as
\begin{align*}
    X(\omega) = 
    \begin{cases}
        1 & \text{if }\omega \in [0, 1]\\
        2 & \text{if }\omega \in (1, 2]
    \end{cases}
\end{align*}
and
\begin{align*}
    Y(\omega) = 
    \begin{cases}
        2 & \text{if }\omega \in [0, 1.5]\\
        3 & \text{if }\omega \in (1.5, 2].
    \end{cases}
\end{align*}
Then which one of the following statements is true?
\begin{enumerate}
    \item [(A)] $X$ is a random variable with respect to $\mathcal{G}$, but $Y$ is not a random variable with respect to $\mathcal{G}$.
    \item [(B)] $Y$ is a random variable with respect to $\mathcal{G}$, but $X$ is not a random variable with respect to $\mathcal{G}$.
    \item [(C)] Neither $X$ nor $Y$ is a random variable with respect to $\mathcal{G}$.
    \item [(D)] Both $X$ and $Y$ are random variables with respect to $\mathcal{G}$.
\end{enumerate} \hfill (GATE ST 2023)\\
\solution
%\begin{table}[H]
	\centering
\begin{tabular}{|c|c|c|}
\hline
Random variable &Value &Definition\\ \hline
\multirow{3}{*}{X} &0 &Slips of Rs 1\\
&1 &Slips of Rs 5\\
&2 &Slips of Rs 13\\ \hline
\multirow{2}{*}{Y} &0 &Box A\\
&1 &Box B\\\hline
\end{tabular}
\caption{}
\label{tab:Distribution}
\end{table}
See \tabref{tab:Distribution}.
\begin{align}
p_{Y}\brak{k}= \begin{cases} 
      \frac{1}{3} & {k=0} \\
      \frac{2}{3 }& {k=1} 
   \end{cases}
   \\
p_{Y|X}\brak{0|0} = \frac{19}{25}\, 
p_{Y|X}\brak{0|1} = \frac{6}{25}\,
p_{Y|X}\brak{1|0} = \frac{45}{50}\,
p_{Y|X}\brak{1|2} = \frac{5}{50}
\end{align}
The desired probability is the probability that a slip drawn at random is marked other than Rs 1,
\begin{align}
&=1-p_X\brak{0}\\
&= p_X(1) + p_X(2)
\end{align}
Using Bayes theorem,
\begin{align}
&= p_Y\brak{0} \times \pr{Y=0 | X=1} + p_Y\brak{1} \times \pr{Y=1|X=2}\\
&=\frac{1}{3} \times \frac{6}{25} + \frac{2}{3} \times \frac{5}{50}\\
&=\frac{11}{75}
\end{align}

\newpage

%\tableofcontents

\bigskip

\renewcommand{\thefigure}{\theenumi}
\renewcommand{\thetable}{\theenumi}
%\renewcommand{\theequation}{\theenumi}

%\begin{abstract}
%%\boldmath
%In this letter, an algorithm for evaluating the exact analytical bit error rate  (BER)  for the piecewise linear (PL) combiner for  multiple relays is presented. Previous results were available only for upto three relays. The algorithm is unique in the sense that  the actual mathematical expressions, that are prohibitively large, need not be explicitly obtained. The diversity gain due to multiple relays is shown through plots of the analytical BER, well supported by simulations. 
%
%\end{abstract}
% IEEEtran.cls defaults to using nonbold math in the Abstract.
% This preserves the distinction between vectors and scalars. However,
% if the journal you are submitting to favors bold math in the abstract,
% then you can use LaTeX's standard command \boldmath at the very start
% of the abstract to achieve this. Many IEEE journals frown on math
% in the abstract anyway.

% Note that keywords are not normally used for peerreview papers.
%\begin{IEEEkeywords}
%Cooperative diversity, decode and forward, piecewise linear
%\end{IEEEkeywords}



% For peer review papers, you can put extra information on the cover
% page as needed:
% \ifCLASSOPTIONpeerreview
% \begin{center} \bfseries EDICS Category: 3-BBND \end{center}
% \fi
%
% For peerreview papers, this IEEEtran command inserts a page break and
% creates the second title. It will be ignored for other modes.
%\IEEEpeerreviewmaketitle




	\item  A die is loaded in such a way that each odd number is twice as likely to occur as
each even number. Find $P(G)$, where $G$ is the event that a number greater than
3 occurs on a single roll of the die.
\\
\solution
		%\begin{table}[H]
	\centering
\begin{tabular}{|c|c|c|}
\hline
Random variable &Value &Definition\\ \hline
\multirow{3}{*}{X} &0 &Slips of Rs 1\\
&1 &Slips of Rs 5\\
&2 &Slips of Rs 13\\ \hline
\multirow{2}{*}{Y} &0 &Box A\\
&1 &Box B\\\hline
\end{tabular}
\caption{}
\label{tab:Distribution}
\end{table}
See \tabref{tab:Distribution}.
\begin{align}
p_{Y}\brak{k}= \begin{cases} 
      \frac{1}{3} & {k=0} \\
      \frac{2}{3 }& {k=1} 
   \end{cases}
   \\
p_{Y|X}\brak{0|0} = \frac{19}{25}\, 
p_{Y|X}\brak{0|1} = \frac{6}{25}\,
p_{Y|X}\brak{1|0} = \frac{45}{50}\,
p_{Y|X}\brak{1|2} = \frac{5}{50}
\end{align}
The desired probability is the probability that a slip drawn at random is marked other than Rs 1,
\begin{align}
&=1-p_X\brak{0}\\
&= p_X(1) + p_X(2)
\end{align}
Using Bayes theorem,
\begin{align}
&= p_Y\brak{0} \times \pr{Y=0 | X=1} + p_Y\brak{1} \times \pr{Y=1|X=2}\\
&=\frac{1}{3} \times \frac{6}{25} + \frac{2}{3} \times \frac{5}{50}\\
&=\frac{11}{75}
\end{align}

\newpage

%\tableofcontents

\bigskip

\renewcommand{\thefigure}{\theenumi}
\renewcommand{\thetable}{\theenumi}
%\renewcommand{\theequation}{\theenumi}

%\begin{abstract}
%%\boldmath
%In this letter, an algorithm for evaluating the exact analytical bit error rate  (BER)  for the piecewise linear (PL) combiner for  multiple relays is presented. Previous results were available only for upto three relays. The algorithm is unique in the sense that  the actual mathematical expressions, that are prohibitively large, need not be explicitly obtained. The diversity gain due to multiple relays is shown through plots of the analytical BER, well supported by simulations. 
%
%\end{abstract}
% IEEEtran.cls defaults to using nonbold math in the Abstract.
% This preserves the distinction between vectors and scalars. However,
% if the journal you are submitting to favors bold math in the abstract,
% then you can use LaTeX's standard command \boldmath at the very start
% of the abstract to achieve this. Many IEEE journals frown on math
% in the abstract anyway.

% Note that keywords are not normally used for peerreview papers.
%\begin{IEEEkeywords}
%Cooperative diversity, decode and forward, piecewise linear
%\end{IEEEkeywords}



% For peer review papers, you can put extra information on the cover
% page as needed:
% \ifCLASSOPTIONpeerreview
% \begin{center} \bfseries EDICS Category: 3-BBND \end{center}
% \fi
%
% For peerreview papers, this IEEEtran command inserts a page break and
% creates the second title. It will be ignored for other modes.
%\IEEEpeerreviewmaketitle




	\item All the jacks, queens and kings are removed from a deck of 52 playing cards. The remaining cards are well shuffled and then one card is drawn at random. Giving ace a value 1 similar value for other cards, find the probability that the card has a value 
		\begin{enumerate}
			\item 7
			\item greater than 7
			\item less than 7
		\end{enumerate}
		%Number of cards left after removing all jacks, queens and kings 
\begin{align}
N	= 52 - 4\times 3
	= 40
\end{align}
%\begin{table}[H]
%\def\arraystretch{1.2}
%\begin{tabular}{|c|c|c|}
%\hline
%	\textbf{Parameter} &\textbf{Value} &\textbf{Description}\\ \hline
%	$X$ &1-10 &Represents the value of the card picked \\ \hline
%\end{tabular}
%\end{table}
Let $1 \le X \le 10$ be the value of the card picked.  Then,
\begin{align}
	p_X(k) &= \Pr(X=k)\ \forall\ 1 \leq k \leq 10\\
	&= \frac{4\times 1}{40}\\
	&= \frac{1}{10}\\
	\therefore p_X(k) &= 
	\begin{cases}
		\frac{1}{10} & 1 \leq k \leq 10\\
		0 & \text{otherwise}
	\end{cases}
\end{align}
and
\begin{align}
	F_{X}(k) &= \sum_{m=0}^{k}p_{X}(m) \quad 1 \leq k \leq 10\\
	&= \frac{k}{10}\\
	\therefore F_{X}(k) &= 
	\begin{cases}
		0 & k \leq 0\\
		\frac{k}{10} & 1\leq k \leq 10\\
		1 & k > 10 
	\end{cases}
\end{align}
\begin{enumerate}
	\item Probability that card has value equal to 7 is
		\begin{align}
			 p_{X}(7)
			= \frac{1}{10}
		\end{align}
	\item Probability that card has value greater than 7 is
		\begin{align}
			1 - F_X(7)
			&= 1 - \frac{7}{10}
			\\
			&= \frac{3}{10}
		\end{align}
	\item Probability that card has value less than 7 is
		\begin{align}
			 F_{X}(6)
			=\frac{6}{10}
		\end{align}
\end{enumerate}

  \item A Lot consists of 48 mobile phones of which 42 are good, 3 have only minor defects and 3 have major defects.Varnika will buy a phone if it is good but the trader will only buy a mobile if it has no major defects. One phone is selected at random from the lot. What is the probability that it is
\begin{enumerate}
	\item acceptable to Varnika?
            \item acceptable to the trader?
\end{enumerate}
\solution
	%\begin{table}[H]
	\centering
\begin{tabular}{|c|c|c|}
\hline
Random variable &Value &Definition\\ \hline
\multirow{3}{*}{X} &0 &Slips of Rs 1\\
&1 &Slips of Rs 5\\
&2 &Slips of Rs 13\\ \hline
\multirow{2}{*}{Y} &0 &Box A\\
&1 &Box B\\\hline
\end{tabular}
\caption{}
\label{tab:Distribution}
\end{table}
See \tabref{tab:Distribution}.
\begin{align}
p_{Y}\brak{k}= \begin{cases} 
      \frac{1}{3} & {k=0} \\
      \frac{2}{3 }& {k=1} 
   \end{cases}
   \\
p_{Y|X}\brak{0|0} = \frac{19}{25}\, 
p_{Y|X}\brak{0|1} = \frac{6}{25}\,
p_{Y|X}\brak{1|0} = \frac{45}{50}\,
p_{Y|X}\brak{1|2} = \frac{5}{50}
\end{align}
The desired probability is the probability that a slip drawn at random is marked other than Rs 1,
\begin{align}
&=1-p_X\brak{0}\\
&= p_X(1) + p_X(2)
\end{align}
Using Bayes theorem,
\begin{align}
&= p_Y\brak{0} \times \pr{Y=0 | X=1} + p_Y\brak{1} \times \pr{Y=1|X=2}\\
&=\frac{1}{3} \times \frac{6}{25} + \frac{2}{3} \times \frac{5}{50}\\
&=\frac{11}{75}
\end{align}

\newpage

%\tableofcontents

\bigskip

\renewcommand{\thefigure}{\theenumi}
\renewcommand{\thetable}{\theenumi}
%\renewcommand{\theequation}{\theenumi}

%\begin{abstract}
%%\boldmath
%In this letter, an algorithm for evaluating the exact analytical bit error rate  (BER)  for the piecewise linear (PL) combiner for  multiple relays is presented. Previous results were available only for upto three relays. The algorithm is unique in the sense that  the actual mathematical expressions, that are prohibitively large, need not be explicitly obtained. The diversity gain due to multiple relays is shown through plots of the analytical BER, well supported by simulations. 
%
%\end{abstract}
% IEEEtran.cls defaults to using nonbold math in the Abstract.
% This preserves the distinction between vectors and scalars. However,
% if the journal you are submitting to favors bold math in the abstract,
% then you can use LaTeX's standard command \boldmath at the very start
% of the abstract to achieve this. Many IEEE journals frown on math
% in the abstract anyway.

% Note that keywords are not normally used for peerreview papers.
%\begin{IEEEkeywords}
%Cooperative diversity, decode and forward, piecewise linear
%\end{IEEEkeywords}



% For peer review papers, you can put extra information on the cover
% page as needed:
% \ifCLASSOPTIONpeerreview
% \begin{center} \bfseries EDICS Category: 3-BBND \end{center}
% \fi
%
% For peerreview papers, this IEEEtran command inserts a page break and
% creates the second title. It will be ignored for other modes.
%\IEEEpeerreviewmaketitle




 \item A student says that if you throw a die, it will show up 1 or not 1. Therefore, the probability of getting 1 and the probability of getting 'not 1' each is equal to $\frac{1}{2}$. Is this correct? Give reasons.\\
 \solution
        %\begin{table}[H]
	\centering
\begin{tabular}{|c|c|c|}
\hline
Random variable &Value &Definition\\ \hline
\multirow{3}{*}{X} &0 &Slips of Rs 1\\
&1 &Slips of Rs 5\\
&2 &Slips of Rs 13\\ \hline
\multirow{2}{*}{Y} &0 &Box A\\
&1 &Box B\\\hline
\end{tabular}
\caption{}
\label{tab:Distribution}
\end{table}
See \tabref{tab:Distribution}.
\begin{align}
p_{Y}\brak{k}= \begin{cases} 
      \frac{1}{3} & {k=0} \\
      \frac{2}{3 }& {k=1} 
   \end{cases}
   \\
p_{Y|X}\brak{0|0} = \frac{19}{25}\, 
p_{Y|X}\brak{0|1} = \frac{6}{25}\,
p_{Y|X}\brak{1|0} = \frac{45}{50}\,
p_{Y|X}\brak{1|2} = \frac{5}{50}
\end{align}
The desired probability is the probability that a slip drawn at random is marked other than Rs 1,
\begin{align}
&=1-p_X\brak{0}\\
&= p_X(1) + p_X(2)
\end{align}
Using Bayes theorem,
\begin{align}
&= p_Y\brak{0} \times \pr{Y=0 | X=1} + p_Y\brak{1} \times \pr{Y=1|X=2}\\
&=\frac{1}{3} \times \frac{6}{25} + \frac{2}{3} \times \frac{5}{50}\\
&=\frac{11}{75}
\end{align}

\newpage

%\tableofcontents

\bigskip

\renewcommand{\thefigure}{\theenumi}
\renewcommand{\thetable}{\theenumi}
%\renewcommand{\theequation}{\theenumi}

%\begin{abstract}
%%\boldmath
%In this letter, an algorithm for evaluating the exact analytical bit error rate  (BER)  for the piecewise linear (PL) combiner for  multiple relays is presented. Previous results were available only for upto three relays. The algorithm is unique in the sense that  the actual mathematical expressions, that are prohibitively large, need not be explicitly obtained. The diversity gain due to multiple relays is shown through plots of the analytical BER, well supported by simulations. 
%
%\end{abstract}
% IEEEtran.cls defaults to using nonbold math in the Abstract.
% This preserves the distinction between vectors and scalars. However,
% if the journal you are submitting to favors bold math in the abstract,
% then you can use LaTeX's standard command \boldmath at the very start
% of the abstract to achieve this. Many IEEE journals frown on math
% in the abstract anyway.

% Note that keywords are not normally used for peerreview papers.
%\begin{IEEEkeywords}
%Cooperative diversity, decode and forward, piecewise linear
%\end{IEEEkeywords}



% For peer review papers, you can put extra information on the cover
% page as needed:
% \ifCLASSOPTIONpeerreview
% \begin{center} \bfseries EDICS Category: 3-BBND \end{center}
% \fi
%
% For peerreview papers, this IEEEtran command inserts a page break and
% creates the second title. It will be ignored for other modes.
%\IEEEpeerreviewmaketitle




   \item Four candidates A, B, C, D have ap-
plied for the assignment to coach a school cricket
team. If A is twice as likely to be selected as B, and
B and C are given about the same chance of being
selected, while C is twice as likely to be selected
as D, what are the probabilities that
\begin{enumerate}
\item C will be selected?
\item A will not be selected?
\end{enumerate}
	%\begin{table}[H]
	\centering
\begin{tabular}{|c|c|c|}
\hline
Random variable &Value &Definition\\ \hline
\multirow{3}{*}{X} &0 &Slips of Rs 1\\
&1 &Slips of Rs 5\\
&2 &Slips of Rs 13\\ \hline
\multirow{2}{*}{Y} &0 &Box A\\
&1 &Box B\\\hline
\end{tabular}
\caption{}
\label{tab:Distribution}
\end{table}
See \tabref{tab:Distribution}.
\begin{align}
p_{Y}\brak{k}= \begin{cases} 
      \frac{1}{3} & {k=0} \\
      \frac{2}{3 }& {k=1} 
   \end{cases}
   \\
p_{Y|X}\brak{0|0} = \frac{19}{25}\, 
p_{Y|X}\brak{0|1} = \frac{6}{25}\,
p_{Y|X}\brak{1|0} = \frac{45}{50}\,
p_{Y|X}\brak{1|2} = \frac{5}{50}
\end{align}
The desired probability is the probability that a slip drawn at random is marked other than Rs 1,
\begin{align}
&=1-p_X\brak{0}\\
&= p_X(1) + p_X(2)
\end{align}
Using Bayes theorem,
\begin{align}
&= p_Y\brak{0} \times \pr{Y=0 | X=1} + p_Y\brak{1} \times \pr{Y=1|X=2}\\
&=\frac{1}{3} \times \frac{6}{25} + \frac{2}{3} \times \frac{5}{50}\\
&=\frac{11}{75}
\end{align}

\newpage

%\tableofcontents

\bigskip

\renewcommand{\thefigure}{\theenumi}
\renewcommand{\thetable}{\theenumi}
%\renewcommand{\theequation}{\theenumi}

%\begin{abstract}
%%\boldmath
%In this letter, an algorithm for evaluating the exact analytical bit error rate  (BER)  for the piecewise linear (PL) combiner for  multiple relays is presented. Previous results were available only for upto three relays. The algorithm is unique in the sense that  the actual mathematical expressions, that are prohibitively large, need not be explicitly obtained. The diversity gain due to multiple relays is shown through plots of the analytical BER, well supported by simulations. 
%
%\end{abstract}
% IEEEtran.cls defaults to using nonbold math in the Abstract.
% This preserves the distinction between vectors and scalars. However,
% if the journal you are submitting to favors bold math in the abstract,
% then you can use LaTeX's standard command \boldmath at the very start
% of the abstract to achieve this. Many IEEE journals frown on math
% in the abstract anyway.

% Note that keywords are not normally used for peerreview papers.
%\begin{IEEEkeywords}
%Cooperative diversity, decode and forward, piecewise linear
%\end{IEEEkeywords}



% For peer review papers, you can put extra information on the cover
% page as needed:
% \ifCLASSOPTIONpeerreview
% \begin{center} \bfseries EDICS Category: 3-BBND \end{center}
% \fi
%
% For peerreview papers, this IEEEtran command inserts a page break and
% creates the second title. It will be ignored for other modes.
%\IEEEpeerreviewmaketitle




 \item A bag contain 24 balls of which $x$ balls are red, $2x$ are white and $3x$ are blue. A ball is selected at random, What is the probability that it is
\begin{enumerate}[label=\alph*)]
\item not red ?
\item white ?
\end{enumerate}
%\begin{table}[H]
	\centering
\begin{tabular}{|c|c|c|}
\hline
Random variable &Value &Definition\\ \hline
\multirow{3}{*}{X} &0 &Slips of Rs 1\\
&1 &Slips of Rs 5\\
&2 &Slips of Rs 13\\ \hline
\multirow{2}{*}{Y} &0 &Box A\\
&1 &Box B\\\hline
\end{tabular}
\caption{}
\label{tab:Distribution}
\end{table}
See \tabref{tab:Distribution}.
\begin{align}
p_{Y}\brak{k}= \begin{cases} 
      \frac{1}{3} & {k=0} \\
      \frac{2}{3 }& {k=1} 
   \end{cases}
   \\
p_{Y|X}\brak{0|0} = \frac{19}{25}\, 
p_{Y|X}\brak{0|1} = \frac{6}{25}\,
p_{Y|X}\brak{1|0} = \frac{45}{50}\,
p_{Y|X}\brak{1|2} = \frac{5}{50}
\end{align}
The desired probability is the probability that a slip drawn at random is marked other than Rs 1,
\begin{align}
&=1-p_X\brak{0}\\
&= p_X(1) + p_X(2)
\end{align}
Using Bayes theorem,
\begin{align}
&= p_Y\brak{0} \times \pr{Y=0 | X=1} + p_Y\brak{1} \times \pr{Y=1|X=2}\\
&=\frac{1}{3} \times \frac{6}{25} + \frac{2}{3} \times \frac{5}{50}\\
&=\frac{11}{75}
\end{align}

\newpage

%\tableofcontents

\bigskip

\renewcommand{\thefigure}{\theenumi}
\renewcommand{\thetable}{\theenumi}
%\renewcommand{\theequation}{\theenumi}

%\begin{abstract}
%%\boldmath
%In this letter, an algorithm for evaluating the exact analytical bit error rate  (BER)  for the piecewise linear (PL) combiner for  multiple relays is presented. Previous results were available only for upto three relays. The algorithm is unique in the sense that  the actual mathematical expressions, that are prohibitively large, need not be explicitly obtained. The diversity gain due to multiple relays is shown through plots of the analytical BER, well supported by simulations. 
%
%\end{abstract}
% IEEEtran.cls defaults to using nonbold math in the Abstract.
% This preserves the distinction between vectors and scalars. However,
% if the journal you are submitting to favors bold math in the abstract,
% then you can use LaTeX's standard command \boldmath at the very start
% of the abstract to achieve this. Many IEEE journals frown on math
% in the abstract anyway.

% Note that keywords are not normally used for peerreview papers.
%\begin{IEEEkeywords}
%Cooperative diversity, decode and forward, piecewise linear
%\end{IEEEkeywords}



% For peer review papers, you can put extra information on the cover
% page as needed:
% \ifCLASSOPTIONpeerreview
% \begin{center} \bfseries EDICS Category: 3-BBND \end{center}
% \fi
%
% For peerreview papers, this IEEEtran command inserts a page break and
% creates the second title. It will be ignored for other modes.
%\IEEEpeerreviewmaketitle




If the letters of the word ASSASSINATION are arranged at random. Find the Probability that
\begin{enumerate}[label=(\alph*)]
\item Four $S's$ come consecutively in the word
\item Two  $I's$ and two $N's$ come together
\item All $A's$ are not coming together
\item No two $A's$ are coming together
\end{enumerate}
%\begin{table}[H]
	\centering
\begin{tabular}{|c|c|c|}
\hline
Random variable &Value &Definition\\ \hline
\multirow{3}{*}{X} &0 &Slips of Rs 1\\
&1 &Slips of Rs 5\\
&2 &Slips of Rs 13\\ \hline
\multirow{2}{*}{Y} &0 &Box A\\
&1 &Box B\\\hline
\end{tabular}
\caption{}
\label{tab:Distribution}
\end{table}
See \tabref{tab:Distribution}.
\begin{align}
p_{Y}\brak{k}= \begin{cases} 
      \frac{1}{3} & {k=0} \\
      \frac{2}{3 }& {k=1} 
   \end{cases}
   \\
p_{Y|X}\brak{0|0} = \frac{19}{25}\, 
p_{Y|X}\brak{0|1} = \frac{6}{25}\,
p_{Y|X}\brak{1|0} = \frac{45}{50}\,
p_{Y|X}\brak{1|2} = \frac{5}{50}
\end{align}
The desired probability is the probability that a slip drawn at random is marked other than Rs 1,
\begin{align}
&=1-p_X\brak{0}\\
&= p_X(1) + p_X(2)
\end{align}
Using Bayes theorem,
\begin{align}
&= p_Y\brak{0} \times \pr{Y=0 | X=1} + p_Y\brak{1} \times \pr{Y=1|X=2}\\
&=\frac{1}{3} \times \frac{6}{25} + \frac{2}{3} \times \frac{5}{50}\\
&=\frac{11}{75}
\end{align}

\newpage

%\tableofcontents

\bigskip

\renewcommand{\thefigure}{\theenumi}
\renewcommand{\thetable}{\theenumi}
%\renewcommand{\theequation}{\theenumi}

%\begin{abstract}
%%\boldmath
%In this letter, an algorithm for evaluating the exact analytical bit error rate  (BER)  for the piecewise linear (PL) combiner for  multiple relays is presented. Previous results were available only for upto three relays. The algorithm is unique in the sense that  the actual mathematical expressions, that are prohibitively large, need not be explicitly obtained. The diversity gain due to multiple relays is shown through plots of the analytical BER, well supported by simulations. 
%
%\end{abstract}
% IEEEtran.cls defaults to using nonbold math in the Abstract.
% This preserves the distinction between vectors and scalars. However,
% if the journal you are submitting to favors bold math in the abstract,
% then you can use LaTeX's standard command \boldmath at the very start
% of the abstract to achieve this. Many IEEE journals frown on math
% in the abstract anyway.

% Note that keywords are not normally used for peerreview papers.
%\begin{IEEEkeywords}
%Cooperative diversity, decode and forward, piecewise linear
%\end{IEEEkeywords}



% For peer review papers, you can put extra information on the cover
% page as needed:
% \ifCLASSOPTIONpeerreview
% \begin{center} \bfseries EDICS Category: 3-BBND \end{center}
% \fi
%
% For peerreview papers, this IEEEtran command inserts a page break and
% creates the second title. It will be ignored for other modes.
%\IEEEpeerreviewmaketitle




	\item One urn contains two black balls (labelled B1 and B2) and one white ball. A
	second urn contains one black ball and two white balls (labelled W1 and W2).
	Suppose the following experiment is performed. One of the two urns is chosen
	at random. Next a ball is randomly chosen from the urn. Then a second ball is
	chosen at random from the same urn without replacing the first ball.
	
	\begin{enumerate}
	\item What is the probability that two black balls are chosen?
	
	\item What is the probability that two balls of opposite colour are chosen?
	\end{enumerate}
	\solution
	%\begin{align}
    \label{eq:12.13.6.18.1}
	\because	\pr{A|B} &> \pr{A},\
\frac{\pr{AB}}{\pr{B}} > \pr{A}
\\
    \label{eq:12.13.6.18.2}
	\implies \pr{AB} &> \pr{A}\pr{B}
	\\
	\text{or, } \frac{\pr{AB}}{\pr{A}} &=\pr{B|A} > \pr{A}
\end{align}

\end{enumerate}

		\item A box of oranges is inspected by examining three randomly selected oranges drawn without replacement. If all the three oranges are good, the box is approved for sale, otherwise, it is rejected. Find the probability that a box containing 15 oranges out of which 12 are good and 3 are bad ones will be approved for sale.
		\label{ncert/12/13/2/3/defs.tex}
		\item Two balls are drawn at random with replacement from a box containing 10 black and 8 red balls. Find the probability that
		\label{ncert/12/13/2/12}
\begin{enumerate}
\item both balls are red.
\item first ball is black and second is red.
\item one of them is black and other is red.
\end{enumerate}

\item In a hostel, 60\% of the students read Hindi newspaper, 40\% read English newspaper and 20\% read both Hindi and English newspapers. A student is selected at random.
		\label{ncert/12/13/2/15}
\begin{enumerate}
\item Find the probability that she reads neither Hindi nor English newspapers.
\item If she reads Hindi newspaper, find the probability that she reads English newspaper.
\item If she reads English newspaper, find the probability that she reads Hindi newspaper.\\
\end{enumerate}
\item The probability of obtaining an even prime number on each die, when a pair of dice is rolled is 
\begin{enumerate}
    \item $0$ 
    
    \item $\frac{1}{3}$ 
    
    \item $\frac{1}{12}$ 
    
    \item $\frac{1}{36}$ 
\end{enumerate}
\solution
		%\begin{enumerate}[label=\thesection.\arabic*,ref=\thesection.\theenumi]
	\item One card is drawn from a well-shuffled deck of 52 cards. Find the probability of getting
\begin{enumerate}
\item A king of red colour 
\item A face card 
\item A red face card
\item The jack of hearts
\item A spade
\item The queen of diamonds

\end{enumerate}
\solution
		%\begin{table}[H]
	\centering
\begin{tabular}{|c|c|c|}
\hline
Random variable &Value &Definition\\ \hline
\multirow{3}{*}{X} &0 &Slips of Rs 1\\
&1 &Slips of Rs 5\\
&2 &Slips of Rs 13\\ \hline
\multirow{2}{*}{Y} &0 &Box A\\
&1 &Box B\\\hline
\end{tabular}
\caption{}
\label{tab:Distribution}
\end{table}
See \tabref{tab:Distribution}.
\begin{align}
p_{Y}\brak{k}= \begin{cases} 
      \frac{1}{3} & {k=0} \\
      \frac{2}{3 }& {k=1} 
   \end{cases}
   \\
p_{Y|X}\brak{0|0} = \frac{19}{25}\, 
p_{Y|X}\brak{0|1} = \frac{6}{25}\,
p_{Y|X}\brak{1|0} = \frac{45}{50}\,
p_{Y|X}\brak{1|2} = \frac{5}{50}
\end{align}
The desired probability is the probability that a slip drawn at random is marked other than Rs 1,
\begin{align}
&=1-p_X\brak{0}\\
&= p_X(1) + p_X(2)
\end{align}
Using Bayes theorem,
\begin{align}
&= p_Y\brak{0} \times \pr{Y=0 | X=1} + p_Y\brak{1} \times \pr{Y=1|X=2}\\
&=\frac{1}{3} \times \frac{6}{25} + \frac{2}{3} \times \frac{5}{50}\\
&=\frac{11}{75}
\end{align}

\newpage

%\tableofcontents

\bigskip

\renewcommand{\thefigure}{\theenumi}
\renewcommand{\thetable}{\theenumi}
%\renewcommand{\theequation}{\theenumi}

%\begin{abstract}
%%\boldmath
%In this letter, an algorithm for evaluating the exact analytical bit error rate  (BER)  for the piecewise linear (PL) combiner for  multiple relays is presented. Previous results were available only for upto three relays. The algorithm is unique in the sense that  the actual mathematical expressions, that are prohibitively large, need not be explicitly obtained. The diversity gain due to multiple relays is shown through plots of the analytical BER, well supported by simulations. 
%
%\end{abstract}
% IEEEtran.cls defaults to using nonbold math in the Abstract.
% This preserves the distinction between vectors and scalars. However,
% if the journal you are submitting to favors bold math in the abstract,
% then you can use LaTeX's standard command \boldmath at the very start
% of the abstract to achieve this. Many IEEE journals frown on math
% in the abstract anyway.

% Note that keywords are not normally used for peerreview papers.
%\begin{IEEEkeywords}
%Cooperative diversity, decode and forward, piecewise linear
%\end{IEEEkeywords}



% For peer review papers, you can put extra information on the cover
% page as needed:
% \ifCLASSOPTIONpeerreview
% \begin{center} \bfseries EDICS Category: 3-BBND \end{center}
% \fi
%
% For peerreview papers, this IEEEtran command inserts a page break and
% creates the second title. It will be ignored for other modes.
%\IEEEpeerreviewmaketitle




	\item Five cards—the ten, jack, queen, king and ace of diamonds, are well-shuffled with their face downwards. One card is then picked up at random.
\begin{enumerate}
\item
What is the probability that the card is the queen? 
\item
If the queen is drawn and put aside, what is the probability that the second card picked up is (a) an ace? (b) a queen?\\
\end{enumerate}
\solution
		%\begin{enumerate}[label=\thesection.\arabic*,ref=\thesection.\theenumi]
	\item One card is drawn from a well-shuffled deck of 52 cards. Find the probability of getting
\begin{enumerate}
\item A king of red colour 
\item A face card 
\item A red face card
\item The jack of hearts
\item A spade
\item The queen of diamonds

\end{enumerate}
\solution
		%\input{ncert/10/15/1/14/main.tex}
	\item Five cards—the ten, jack, queen, king and ace of diamonds, are well-shuffled with their face downwards. One card is then picked up at random.
\begin{enumerate}
\item
What is the probability that the card is the queen? 
\item
If the queen is drawn and put aside, what is the probability that the second card picked up is (a) an ace? (b) a queen?\\
\end{enumerate}
\solution
		%\input{ncert/10/15/1/15/defs.tex}
	\item A bag contains $5$ red balls and some blue balls. If the probability of drawing a blue ball is double that if a red ball, determine the number of blue balls in the bag. 
		\\
\solution
		%\input{ncert/10/15/2/3/defs.tex}
	\item A card is selected from a pack of 52 cards.
 \begin{enumerate}[label=(\alph*)] 
                 \item How many points are there in the sample space?
                 \item Calculate the probability that the card is an ace of spades.
                 \item Calculate the probability that the card is (i) an ace and (ii) black card.
 \end{enumerate}
\solution
		%\input{ncert/11/16/3/4/main.tex}
\item Four cards are drawn from a well-shuffled deck of 52 cards. What is the probability of obtaining 3 diamonds and one spade.
\\
\solution
		%\input{ncert/11/16/4/2/defs.tex}
\item In a certain lottery 10,000 tickets are sold and ten equal prizes are awarded. What is the probability of not getting a prize if you buy (a) one ticket (b) two tickets (c) 10 tickets ?	
\\
\solution
		%\input{ncert/11/16/4/4/defs.tex}
		%
\item 
Out of 100 students, two sections of 40 and 60 are formed. If you and your friend are among the 100 students, what is the probability that
\begin{enumerate}
\item you both enter the same section?
\item you both enter the different sections?
\end{enumerate}
\solution
		%\input{ncert/11/16/4/5/defs.tex}
	\item 
The number lock of a suitcase has 4 wheels each labelled with ten digits i.e. from 0 to 9.The lock opens with a sequence of four digits with no repeats.What is the probability of a person getting the right sequence to open the suitcase.
\\
\solution
		%\input{ncert/11/16/4/10/defs.tex}
		%
\item 
Two cards are drawn at random and without replacement from a pack of 52 playing cards. Find the probability that both the cards are black.
\\
\solution
		%\input{ncert/12/13/2/2/defs.tex}
		\item A box of oranges is inspected by examining three randomly selected oranges drawn without replacement. If all the three oranges are good, the box is approved for sale, otherwise, it is rejected. Find the probability that a box containing 15 oranges out of which 12 are good and 3 are bad ones will be approved for sale.
		\label{ncert/12/13/2/3/defs.tex}
		\item Two balls are drawn at random with replacement from a box containing 10 black and 8 red balls. Find the probability that
		\label{ncert/12/13/2/12}
\begin{enumerate}
\item both balls are red.
\item first ball is black and second is red.
\item one of them is black and other is red.
\end{enumerate}

\item In a hostel, 60\% of the students read Hindi newspaper, 40\% read English newspaper and 20\% read both Hindi and English newspapers. A student is selected at random.
		\label{ncert/12/13/2/15}
\begin{enumerate}
\item Find the probability that she reads neither Hindi nor English newspapers.
\item If she reads Hindi newspaper, find the probability that she reads English newspaper.
\item If she reads English newspaper, find the probability that she reads Hindi newspaper.\\
\end{enumerate}
\item The probability of obtaining an even prime number on each die, when a pair of dice is rolled is 
\begin{enumerate}
    \item $0$ 
    
    \item $\frac{1}{3}$ 
    
    \item $\frac{1}{12}$ 
    
    \item $\frac{1}{36}$ 
\end{enumerate}
\solution
		%\input{ncert/12/13/2/17/defs.tex}
	\item A bag contains 4 red and 4 black balls, another bag contains 2 red and 6 black balls. One of the two bags is selected at random and a ball is drawn from the bag which is found to be red. Find the probability that the ball is drawn from the first bag.
\\
\solution
		%\input{ncert/12/13/3/2/main.tex}
  \item
  Cards with numbers 2 to 101 are placed in a box. A card is selected at random.Find the probability that the card has
\begin{enumerate}[label=(\roman*)]
	\item an even number 
	\item a square number
\end{enumerate}
\solution
%\input{exemplar/10/13/3/32/main.tex}
\item
The king, queen and jack of clubs are removed from a deck of 52 playing cards and then well shuffled. Now one card is drawn at random from the remaining cards.  Determine the probability that the card is
\begin{enumerate}[label=(\roman*)]
\item a club
\item 10 of hearts
\end{enumerate}
\solution
%\input{exemplar/10/13/3/29/main.tex}
\item A team of medical students doing their internship have to assist during surgeries
at a city hospital. The probabilities of surgeries rated as very complex, complex,
routine, simple or very simple are respectively, 0.15, 0.20, 0.31, 0.26, .08. Find
the probabilities that a particular surgery will be rated
\begin{enumerate}
	\item complex or very complex;
	\item neither very complex nor very simple;
	\item routine or complex
	\item routine or simple
\end{enumerate}
\solution
%\input{exemplar/11/16/3/8(1)/main.tex}
\item A card is selected from a pack of 52 cards.
\begin{enumerate}[label=(\alph*)]
    \item How many points are there in the sample space?
    \item Calculate the probability that the card is an ace of spades.
    \item Calculate the probability that the card is (i) an ace and (ii) black card.
\end{enumerate}
\solution
%\input{exemplar/11/16/3/4/main2.tex}
\item The probability that a non leap year selected at random will contain 53 sundays.
\\
\solution
%\input{exemplar/10/13/1/19/main.tex}
\item One of the four persons John, Rita, Aslam or Gurpreet will be promoted next
month. Consequently the sample space consists of four elementary outcomes
S = {John promoted, Rita promoted, Aslam promoted, Gurpreet promoted}
You are told that the chances of John’s promotion is same as that of Gurpreet,
Rita’s chances of promotion are twice as likely as Johns. Aslam’s chances are
four times that of John.
\begin{enumerate}
	\item Determine
	\begin{enumerate}
		\item P (John promoted)
		\item P (Rita promoted)
		\item P (Aslam promoted)
		\item P (Gurpreet promoted)
	\end{enumerate}
	\item If A = {John promoted or Gurpreet promoted}, find P (A).
\end{enumerate}
\solution
%\input{exemplar/11/16/3/10/main.tex}
\item A card is drawn from a deck of 52 cards. Find the probability of getting a king or a heart or a red card.\\
\solution
%\input{exemplar/11/16/3/15/main.tex}
\item The probability that a student will pass his examination is 0.73, the probability of
the student getting a compartment is 0.13, and the probability that the student will
either pass or get compartment is 0.96. State True or False.\\
\solution
%\input{exemplar/11/16/3/31/main.tex}
\item A card is selected from a pack of 52 cards\\
\begin{enumerate}[label=(\alph*)]
\item How many points are there in the sample space?
\item Calculate the probability that the cards is an ace of spades.
\item Calculate the probability that the card is (i) an ace (ii)black card.\\
\end{enumerate}
%\input{ncert/11/16/3/4_1/Prob_4.tex}
\item In a non-leap year, the probability of having 53 tuesdays or 53 wednesdays is\\
\solution
%\input{exemplar/11/16/3/18/main.tex}
\item There are 1000 sealed envelopes in a box, 10 of them contain a cash prize of
Rs 100 each, 100 of them contain a cash prize of Rs 50 each and 200 of them
contain a cash prize of Rs 10 each and rest do not contain any cash prize. If they
are well shuffled and an envelope is picked up out, what is the probability that it
contains no cash prize?\\
\solution
%\input{exemplar/10/13/3/34/main.tex}
\item 
A die is thrown and a card is selected at random from a deck of 52 playing cards. The probability of getting an even number on the die and a spade card.\\
\solution
%\input{exemplar/12/13/3/78/main.tex}
\item
If 4-digit numbers greater than 5,000 are randomly formed from the digits 0, 1, 3, 5, and 7, what is the probability of forming a number divisible by 5 when:
\begin{enumerate}
    \item The digits are repeated?
    \item The repetition of digits is not allowed?
\end{enumerate}
\solution
%\input{ncert/11/16/4/9/main.tex}
\item Consider the probability space $\brak{\Omega, \mathcal{G}, P}$ where $\Omega = [0,2]$ and $\mathcal{G} = \cbrak{\phi, \Omega, [0,1], (1,2]}$. Let $X$ and $Y$ be two functions on $\Omega$ defined as
\begin{align*}
    X(\omega) = 
    \begin{cases}
        1 & \text{if }\omega \in [0, 1]\\
        2 & \text{if }\omega \in (1, 2]
    \end{cases}
\end{align*}
and
\begin{align*}
    Y(\omega) = 
    \begin{cases}
        2 & \text{if }\omega \in [0, 1.5]\\
        3 & \text{if }\omega \in (1.5, 2].
    \end{cases}
\end{align*}
Then which one of the following statements is true?
\begin{enumerate}
    \item [(A)] $X$ is a random variable with respect to $\mathcal{G}$, but $Y$ is not a random variable with respect to $\mathcal{G}$.
    \item [(B)] $Y$ is a random variable with respect to $\mathcal{G}$, but $X$ is not a random variable with respect to $\mathcal{G}$.
    \item [(C)] Neither $X$ nor $Y$ is a random variable with respect to $\mathcal{G}$.
    \item [(D)] Both $X$ and $Y$ are random variables with respect to $\mathcal{G}$.
\end{enumerate} \hfill (GATE ST 2023)\\
\solution
%\input{gate/ST/2023/14/main.tex}
	\item  A die is loaded in such a way that each odd number is twice as likely to occur as
each even number. Find $P(G)$, where $G$ is the event that a number greater than
3 occurs on a single roll of the die.
\\
\solution
		%\input{exemplar/11/16/3/5/main.tex}
	\item All the jacks, queens and kings are removed from a deck of 52 playing cards. The remaining cards are well shuffled and then one card is drawn at random. Giving ace a value 1 similar value for other cards, find the probability that the card has a value 
		\begin{enumerate}
			\item 7
			\item greater than 7
			\item less than 7
		\end{enumerate}
		%\input{exemplar/10/13/3/30/main.tex}
  \item A Lot consists of 48 mobile phones of which 42 are good, 3 have only minor defects and 3 have major defects.Varnika will buy a phone if it is good but the trader will only buy a mobile if it has no major defects. One phone is selected at random from the lot. What is the probability that it is
\begin{enumerate}
	\item acceptable to Varnika?
            \item acceptable to the trader?
\end{enumerate}
\solution
	%\input{exemplar/10/13/3/40/main.tex}
 \item A student says that if you throw a die, it will show up 1 or not 1. Therefore, the probability of getting 1 and the probability of getting 'not 1' each is equal to $\frac{1}{2}$. Is this correct? Give reasons.\\
 \solution
        %\input{exemplar/10/13/2/9/main.tex}
   \item Four candidates A, B, C, D have ap-
plied for the assignment to coach a school cricket
team. If A is twice as likely to be selected as B, and
B and C are given about the same chance of being
selected, while C is twice as likely to be selected
as D, what are the probabilities that
\begin{enumerate}
\item C will be selected?
\item A will not be selected?
\end{enumerate}
	%\input{exemplar/11/16/3/9/main.tex}
 \item A bag contain 24 balls of which $x$ balls are red, $2x$ are white and $3x$ are blue. A ball is selected at random, What is the probability that it is
\begin{enumerate}[label=\alph*)]
\item not red ?
\item white ?
\end{enumerate}
%\input{exemplar/10/13/3/41/main.tex}
If the letters of the word ASSASSINATION are arranged at random. Find the Probability that
\begin{enumerate}[label=(\alph*)]
\item Four $S's$ come consecutively in the word
\item Two  $I's$ and two $N's$ come together
\item All $A's$ are not coming together
\item No two $A's$ are coming together
\end{enumerate}
%\input{exemplar/11/16/3/14/main.tex}
	\item One urn contains two black balls (labelled B1 and B2) and one white ball. A
	second urn contains one black ball and two white balls (labelled W1 and W2).
	Suppose the following experiment is performed. One of the two urns is chosen
	at random. Next a ball is randomly chosen from the urn. Then a second ball is
	chosen at random from the same urn without replacing the first ball.
	
	\begin{enumerate}
	\item What is the probability that two black balls are chosen?
	
	\item What is the probability that two balls of opposite colour are chosen?
	\end{enumerate}
	\solution
	%\input{exemplar/11/16/3/12/main1.tex}
\end{enumerate}

	\item A bag contains $5$ red balls and some blue balls. If the probability of drawing a blue ball is double that if a red ball, determine the number of blue balls in the bag. 
		\\
\solution
		%\begin{enumerate}[label=\thesection.\arabic*,ref=\thesection.\theenumi]
	\item One card is drawn from a well-shuffled deck of 52 cards. Find the probability of getting
\begin{enumerate}
\item A king of red colour 
\item A face card 
\item A red face card
\item The jack of hearts
\item A spade
\item The queen of diamonds

\end{enumerate}
\solution
		%\input{ncert/10/15/1/14/main.tex}
	\item Five cards—the ten, jack, queen, king and ace of diamonds, are well-shuffled with their face downwards. One card is then picked up at random.
\begin{enumerate}
\item
What is the probability that the card is the queen? 
\item
If the queen is drawn and put aside, what is the probability that the second card picked up is (a) an ace? (b) a queen?\\
\end{enumerate}
\solution
		%\input{ncert/10/15/1/15/defs.tex}
	\item A bag contains $5$ red balls and some blue balls. If the probability of drawing a blue ball is double that if a red ball, determine the number of blue balls in the bag. 
		\\
\solution
		%\input{ncert/10/15/2/3/defs.tex}
	\item A card is selected from a pack of 52 cards.
 \begin{enumerate}[label=(\alph*)] 
                 \item How many points are there in the sample space?
                 \item Calculate the probability that the card is an ace of spades.
                 \item Calculate the probability that the card is (i) an ace and (ii) black card.
 \end{enumerate}
\solution
		%\input{ncert/11/16/3/4/main.tex}
\item Four cards are drawn from a well-shuffled deck of 52 cards. What is the probability of obtaining 3 diamonds and one spade.
\\
\solution
		%\input{ncert/11/16/4/2/defs.tex}
\item In a certain lottery 10,000 tickets are sold and ten equal prizes are awarded. What is the probability of not getting a prize if you buy (a) one ticket (b) two tickets (c) 10 tickets ?	
\\
\solution
		%\input{ncert/11/16/4/4/defs.tex}
		%
\item 
Out of 100 students, two sections of 40 and 60 are formed. If you and your friend are among the 100 students, what is the probability that
\begin{enumerate}
\item you both enter the same section?
\item you both enter the different sections?
\end{enumerate}
\solution
		%\input{ncert/11/16/4/5/defs.tex}
	\item 
The number lock of a suitcase has 4 wheels each labelled with ten digits i.e. from 0 to 9.The lock opens with a sequence of four digits with no repeats.What is the probability of a person getting the right sequence to open the suitcase.
\\
\solution
		%\input{ncert/11/16/4/10/defs.tex}
		%
\item 
Two cards are drawn at random and without replacement from a pack of 52 playing cards. Find the probability that both the cards are black.
\\
\solution
		%\input{ncert/12/13/2/2/defs.tex}
		\item A box of oranges is inspected by examining three randomly selected oranges drawn without replacement. If all the three oranges are good, the box is approved for sale, otherwise, it is rejected. Find the probability that a box containing 15 oranges out of which 12 are good and 3 are bad ones will be approved for sale.
		\label{ncert/12/13/2/3/defs.tex}
		\item Two balls are drawn at random with replacement from a box containing 10 black and 8 red balls. Find the probability that
		\label{ncert/12/13/2/12}
\begin{enumerate}
\item both balls are red.
\item first ball is black and second is red.
\item one of them is black and other is red.
\end{enumerate}

\item In a hostel, 60\% of the students read Hindi newspaper, 40\% read English newspaper and 20\% read both Hindi and English newspapers. A student is selected at random.
		\label{ncert/12/13/2/15}
\begin{enumerate}
\item Find the probability that she reads neither Hindi nor English newspapers.
\item If she reads Hindi newspaper, find the probability that she reads English newspaper.
\item If she reads English newspaper, find the probability that she reads Hindi newspaper.\\
\end{enumerate}
\item The probability of obtaining an even prime number on each die, when a pair of dice is rolled is 
\begin{enumerate}
    \item $0$ 
    
    \item $\frac{1}{3}$ 
    
    \item $\frac{1}{12}$ 
    
    \item $\frac{1}{36}$ 
\end{enumerate}
\solution
		%\input{ncert/12/13/2/17/defs.tex}
	\item A bag contains 4 red and 4 black balls, another bag contains 2 red and 6 black balls. One of the two bags is selected at random and a ball is drawn from the bag which is found to be red. Find the probability that the ball is drawn from the first bag.
\\
\solution
		%\input{ncert/12/13/3/2/main.tex}
  \item
  Cards with numbers 2 to 101 are placed in a box. A card is selected at random.Find the probability that the card has
\begin{enumerate}[label=(\roman*)]
	\item an even number 
	\item a square number
\end{enumerate}
\solution
%\input{exemplar/10/13/3/32/main.tex}
\item
The king, queen and jack of clubs are removed from a deck of 52 playing cards and then well shuffled. Now one card is drawn at random from the remaining cards.  Determine the probability that the card is
\begin{enumerate}[label=(\roman*)]
\item a club
\item 10 of hearts
\end{enumerate}
\solution
%\input{exemplar/10/13/3/29/main.tex}
\item A team of medical students doing their internship have to assist during surgeries
at a city hospital. The probabilities of surgeries rated as very complex, complex,
routine, simple or very simple are respectively, 0.15, 0.20, 0.31, 0.26, .08. Find
the probabilities that a particular surgery will be rated
\begin{enumerate}
	\item complex or very complex;
	\item neither very complex nor very simple;
	\item routine or complex
	\item routine or simple
\end{enumerate}
\solution
%\input{exemplar/11/16/3/8(1)/main.tex}
\item A card is selected from a pack of 52 cards.
\begin{enumerate}[label=(\alph*)]
    \item How many points are there in the sample space?
    \item Calculate the probability that the card is an ace of spades.
    \item Calculate the probability that the card is (i) an ace and (ii) black card.
\end{enumerate}
\solution
%\input{exemplar/11/16/3/4/main2.tex}
\item The probability that a non leap year selected at random will contain 53 sundays.
\\
\solution
%\input{exemplar/10/13/1/19/main.tex}
\item One of the four persons John, Rita, Aslam or Gurpreet will be promoted next
month. Consequently the sample space consists of four elementary outcomes
S = {John promoted, Rita promoted, Aslam promoted, Gurpreet promoted}
You are told that the chances of John’s promotion is same as that of Gurpreet,
Rita’s chances of promotion are twice as likely as Johns. Aslam’s chances are
four times that of John.
\begin{enumerate}
	\item Determine
	\begin{enumerate}
		\item P (John promoted)
		\item P (Rita promoted)
		\item P (Aslam promoted)
		\item P (Gurpreet promoted)
	\end{enumerate}
	\item If A = {John promoted or Gurpreet promoted}, find P (A).
\end{enumerate}
\solution
%\input{exemplar/11/16/3/10/main.tex}
\item A card is drawn from a deck of 52 cards. Find the probability of getting a king or a heart or a red card.\\
\solution
%\input{exemplar/11/16/3/15/main.tex}
\item The probability that a student will pass his examination is 0.73, the probability of
the student getting a compartment is 0.13, and the probability that the student will
either pass or get compartment is 0.96. State True or False.\\
\solution
%\input{exemplar/11/16/3/31/main.tex}
\item A card is selected from a pack of 52 cards\\
\begin{enumerate}[label=(\alph*)]
\item How many points are there in the sample space?
\item Calculate the probability that the cards is an ace of spades.
\item Calculate the probability that the card is (i) an ace (ii)black card.\\
\end{enumerate}
%\input{ncert/11/16/3/4_1/Prob_4.tex}
\item In a non-leap year, the probability of having 53 tuesdays or 53 wednesdays is\\
\solution
%\input{exemplar/11/16/3/18/main.tex}
\item There are 1000 sealed envelopes in a box, 10 of them contain a cash prize of
Rs 100 each, 100 of them contain a cash prize of Rs 50 each and 200 of them
contain a cash prize of Rs 10 each and rest do not contain any cash prize. If they
are well shuffled and an envelope is picked up out, what is the probability that it
contains no cash prize?\\
\solution
%\input{exemplar/10/13/3/34/main.tex}
\item 
A die is thrown and a card is selected at random from a deck of 52 playing cards. The probability of getting an even number on the die and a spade card.\\
\solution
%\input{exemplar/12/13/3/78/main.tex}
\item
If 4-digit numbers greater than 5,000 are randomly formed from the digits 0, 1, 3, 5, and 7, what is the probability of forming a number divisible by 5 when:
\begin{enumerate}
    \item The digits are repeated?
    \item The repetition of digits is not allowed?
\end{enumerate}
\solution
%\input{ncert/11/16/4/9/main.tex}
\item Consider the probability space $\brak{\Omega, \mathcal{G}, P}$ where $\Omega = [0,2]$ and $\mathcal{G} = \cbrak{\phi, \Omega, [0,1], (1,2]}$. Let $X$ and $Y$ be two functions on $\Omega$ defined as
\begin{align*}
    X(\omega) = 
    \begin{cases}
        1 & \text{if }\omega \in [0, 1]\\
        2 & \text{if }\omega \in (1, 2]
    \end{cases}
\end{align*}
and
\begin{align*}
    Y(\omega) = 
    \begin{cases}
        2 & \text{if }\omega \in [0, 1.5]\\
        3 & \text{if }\omega \in (1.5, 2].
    \end{cases}
\end{align*}
Then which one of the following statements is true?
\begin{enumerate}
    \item [(A)] $X$ is a random variable with respect to $\mathcal{G}$, but $Y$ is not a random variable with respect to $\mathcal{G}$.
    \item [(B)] $Y$ is a random variable with respect to $\mathcal{G}$, but $X$ is not a random variable with respect to $\mathcal{G}$.
    \item [(C)] Neither $X$ nor $Y$ is a random variable with respect to $\mathcal{G}$.
    \item [(D)] Both $X$ and $Y$ are random variables with respect to $\mathcal{G}$.
\end{enumerate} \hfill (GATE ST 2023)\\
\solution
%\input{gate/ST/2023/14/main.tex}
	\item  A die is loaded in such a way that each odd number is twice as likely to occur as
each even number. Find $P(G)$, where $G$ is the event that a number greater than
3 occurs on a single roll of the die.
\\
\solution
		%\input{exemplar/11/16/3/5/main.tex}
	\item All the jacks, queens and kings are removed from a deck of 52 playing cards. The remaining cards are well shuffled and then one card is drawn at random. Giving ace a value 1 similar value for other cards, find the probability that the card has a value 
		\begin{enumerate}
			\item 7
			\item greater than 7
			\item less than 7
		\end{enumerate}
		%\input{exemplar/10/13/3/30/main.tex}
  \item A Lot consists of 48 mobile phones of which 42 are good, 3 have only minor defects and 3 have major defects.Varnika will buy a phone if it is good but the trader will only buy a mobile if it has no major defects. One phone is selected at random from the lot. What is the probability that it is
\begin{enumerate}
	\item acceptable to Varnika?
            \item acceptable to the trader?
\end{enumerate}
\solution
	%\input{exemplar/10/13/3/40/main.tex}
 \item A student says that if you throw a die, it will show up 1 or not 1. Therefore, the probability of getting 1 and the probability of getting 'not 1' each is equal to $\frac{1}{2}$. Is this correct? Give reasons.\\
 \solution
        %\input{exemplar/10/13/2/9/main.tex}
   \item Four candidates A, B, C, D have ap-
plied for the assignment to coach a school cricket
team. If A is twice as likely to be selected as B, and
B and C are given about the same chance of being
selected, while C is twice as likely to be selected
as D, what are the probabilities that
\begin{enumerate}
\item C will be selected?
\item A will not be selected?
\end{enumerate}
	%\input{exemplar/11/16/3/9/main.tex}
 \item A bag contain 24 balls of which $x$ balls are red, $2x$ are white and $3x$ are blue. A ball is selected at random, What is the probability that it is
\begin{enumerate}[label=\alph*)]
\item not red ?
\item white ?
\end{enumerate}
%\input{exemplar/10/13/3/41/main.tex}
If the letters of the word ASSASSINATION are arranged at random. Find the Probability that
\begin{enumerate}[label=(\alph*)]
\item Four $S's$ come consecutively in the word
\item Two  $I's$ and two $N's$ come together
\item All $A's$ are not coming together
\item No two $A's$ are coming together
\end{enumerate}
%\input{exemplar/11/16/3/14/main.tex}
	\item One urn contains two black balls (labelled B1 and B2) and one white ball. A
	second urn contains one black ball and two white balls (labelled W1 and W2).
	Suppose the following experiment is performed. One of the two urns is chosen
	at random. Next a ball is randomly chosen from the urn. Then a second ball is
	chosen at random from the same urn without replacing the first ball.
	
	\begin{enumerate}
	\item What is the probability that two black balls are chosen?
	
	\item What is the probability that two balls of opposite colour are chosen?
	\end{enumerate}
	\solution
	%\input{exemplar/11/16/3/12/main1.tex}
\end{enumerate}

	\item A card is selected from a pack of 52 cards.
 \begin{enumerate}[label=(\alph*)] 
                 \item How many points are there in the sample space?
                 \item Calculate the probability that the card is an ace of spades.
                 \item Calculate the probability that the card is (i) an ace and (ii) black card.
 \end{enumerate}
\solution
		%\begin{table}[H]
	\centering
\begin{tabular}{|c|c|c|}
\hline
Random variable &Value &Definition\\ \hline
\multirow{3}{*}{X} &0 &Slips of Rs 1\\
&1 &Slips of Rs 5\\
&2 &Slips of Rs 13\\ \hline
\multirow{2}{*}{Y} &0 &Box A\\
&1 &Box B\\\hline
\end{tabular}
\caption{}
\label{tab:Distribution}
\end{table}
See \tabref{tab:Distribution}.
\begin{align}
p_{Y}\brak{k}= \begin{cases} 
      \frac{1}{3} & {k=0} \\
      \frac{2}{3 }& {k=1} 
   \end{cases}
   \\
p_{Y|X}\brak{0|0} = \frac{19}{25}\, 
p_{Y|X}\brak{0|1} = \frac{6}{25}\,
p_{Y|X}\brak{1|0} = \frac{45}{50}\,
p_{Y|X}\brak{1|2} = \frac{5}{50}
\end{align}
The desired probability is the probability that a slip drawn at random is marked other than Rs 1,
\begin{align}
&=1-p_X\brak{0}\\
&= p_X(1) + p_X(2)
\end{align}
Using Bayes theorem,
\begin{align}
&= p_Y\brak{0} \times \pr{Y=0 | X=1} + p_Y\brak{1} \times \pr{Y=1|X=2}\\
&=\frac{1}{3} \times \frac{6}{25} + \frac{2}{3} \times \frac{5}{50}\\
&=\frac{11}{75}
\end{align}

\newpage

%\tableofcontents

\bigskip

\renewcommand{\thefigure}{\theenumi}
\renewcommand{\thetable}{\theenumi}
%\renewcommand{\theequation}{\theenumi}

%\begin{abstract}
%%\boldmath
%In this letter, an algorithm for evaluating the exact analytical bit error rate  (BER)  for the piecewise linear (PL) combiner for  multiple relays is presented. Previous results were available only for upto three relays. The algorithm is unique in the sense that  the actual mathematical expressions, that are prohibitively large, need not be explicitly obtained. The diversity gain due to multiple relays is shown through plots of the analytical BER, well supported by simulations. 
%
%\end{abstract}
% IEEEtran.cls defaults to using nonbold math in the Abstract.
% This preserves the distinction between vectors and scalars. However,
% if the journal you are submitting to favors bold math in the abstract,
% then you can use LaTeX's standard command \boldmath at the very start
% of the abstract to achieve this. Many IEEE journals frown on math
% in the abstract anyway.

% Note that keywords are not normally used for peerreview papers.
%\begin{IEEEkeywords}
%Cooperative diversity, decode and forward, piecewise linear
%\end{IEEEkeywords}



% For peer review papers, you can put extra information on the cover
% page as needed:
% \ifCLASSOPTIONpeerreview
% \begin{center} \bfseries EDICS Category: 3-BBND \end{center}
% \fi
%
% For peerreview papers, this IEEEtran command inserts a page break and
% creates the second title. It will be ignored for other modes.
%\IEEEpeerreviewmaketitle




\item Four cards are drawn from a well-shuffled deck of 52 cards. What is the probability of obtaining 3 diamonds and one spade.
\\
\solution
		%\begin{enumerate}[label=\thesection.\arabic*,ref=\thesection.\theenumi]
	\item One card is drawn from a well-shuffled deck of 52 cards. Find the probability of getting
\begin{enumerate}
\item A king of red colour 
\item A face card 
\item A red face card
\item The jack of hearts
\item A spade
\item The queen of diamonds

\end{enumerate}
\solution
		%\input{ncert/10/15/1/14/main.tex}
	\item Five cards—the ten, jack, queen, king and ace of diamonds, are well-shuffled with their face downwards. One card is then picked up at random.
\begin{enumerate}
\item
What is the probability that the card is the queen? 
\item
If the queen is drawn and put aside, what is the probability that the second card picked up is (a) an ace? (b) a queen?\\
\end{enumerate}
\solution
		%\input{ncert/10/15/1/15/defs.tex}
	\item A bag contains $5$ red balls and some blue balls. If the probability of drawing a blue ball is double that if a red ball, determine the number of blue balls in the bag. 
		\\
\solution
		%\input{ncert/10/15/2/3/defs.tex}
	\item A card is selected from a pack of 52 cards.
 \begin{enumerate}[label=(\alph*)] 
                 \item How many points are there in the sample space?
                 \item Calculate the probability that the card is an ace of spades.
                 \item Calculate the probability that the card is (i) an ace and (ii) black card.
 \end{enumerate}
\solution
		%\input{ncert/11/16/3/4/main.tex}
\item Four cards are drawn from a well-shuffled deck of 52 cards. What is the probability of obtaining 3 diamonds and one spade.
\\
\solution
		%\input{ncert/11/16/4/2/defs.tex}
\item In a certain lottery 10,000 tickets are sold and ten equal prizes are awarded. What is the probability of not getting a prize if you buy (a) one ticket (b) two tickets (c) 10 tickets ?	
\\
\solution
		%\input{ncert/11/16/4/4/defs.tex}
		%
\item 
Out of 100 students, two sections of 40 and 60 are formed. If you and your friend are among the 100 students, what is the probability that
\begin{enumerate}
\item you both enter the same section?
\item you both enter the different sections?
\end{enumerate}
\solution
		%\input{ncert/11/16/4/5/defs.tex}
	\item 
The number lock of a suitcase has 4 wheels each labelled with ten digits i.e. from 0 to 9.The lock opens with a sequence of four digits with no repeats.What is the probability of a person getting the right sequence to open the suitcase.
\\
\solution
		%\input{ncert/11/16/4/10/defs.tex}
		%
\item 
Two cards are drawn at random and without replacement from a pack of 52 playing cards. Find the probability that both the cards are black.
\\
\solution
		%\input{ncert/12/13/2/2/defs.tex}
		\item A box of oranges is inspected by examining three randomly selected oranges drawn without replacement. If all the three oranges are good, the box is approved for sale, otherwise, it is rejected. Find the probability that a box containing 15 oranges out of which 12 are good and 3 are bad ones will be approved for sale.
		\label{ncert/12/13/2/3/defs.tex}
		\item Two balls are drawn at random with replacement from a box containing 10 black and 8 red balls. Find the probability that
		\label{ncert/12/13/2/12}
\begin{enumerate}
\item both balls are red.
\item first ball is black and second is red.
\item one of them is black and other is red.
\end{enumerate}

\item In a hostel, 60\% of the students read Hindi newspaper, 40\% read English newspaper and 20\% read both Hindi and English newspapers. A student is selected at random.
		\label{ncert/12/13/2/15}
\begin{enumerate}
\item Find the probability that she reads neither Hindi nor English newspapers.
\item If she reads Hindi newspaper, find the probability that she reads English newspaper.
\item If she reads English newspaper, find the probability that she reads Hindi newspaper.\\
\end{enumerate}
\item The probability of obtaining an even prime number on each die, when a pair of dice is rolled is 
\begin{enumerate}
    \item $0$ 
    
    \item $\frac{1}{3}$ 
    
    \item $\frac{1}{12}$ 
    
    \item $\frac{1}{36}$ 
\end{enumerate}
\solution
		%\input{ncert/12/13/2/17/defs.tex}
	\item A bag contains 4 red and 4 black balls, another bag contains 2 red and 6 black balls. One of the two bags is selected at random and a ball is drawn from the bag which is found to be red. Find the probability that the ball is drawn from the first bag.
\\
\solution
		%\input{ncert/12/13/3/2/main.tex}
  \item
  Cards with numbers 2 to 101 are placed in a box. A card is selected at random.Find the probability that the card has
\begin{enumerate}[label=(\roman*)]
	\item an even number 
	\item a square number
\end{enumerate}
\solution
%\input{exemplar/10/13/3/32/main.tex}
\item
The king, queen and jack of clubs are removed from a deck of 52 playing cards and then well shuffled. Now one card is drawn at random from the remaining cards.  Determine the probability that the card is
\begin{enumerate}[label=(\roman*)]
\item a club
\item 10 of hearts
\end{enumerate}
\solution
%\input{exemplar/10/13/3/29/main.tex}
\item A team of medical students doing their internship have to assist during surgeries
at a city hospital. The probabilities of surgeries rated as very complex, complex,
routine, simple or very simple are respectively, 0.15, 0.20, 0.31, 0.26, .08. Find
the probabilities that a particular surgery will be rated
\begin{enumerate}
	\item complex or very complex;
	\item neither very complex nor very simple;
	\item routine or complex
	\item routine or simple
\end{enumerate}
\solution
%\input{exemplar/11/16/3/8(1)/main.tex}
\item A card is selected from a pack of 52 cards.
\begin{enumerate}[label=(\alph*)]
    \item How many points are there in the sample space?
    \item Calculate the probability that the card is an ace of spades.
    \item Calculate the probability that the card is (i) an ace and (ii) black card.
\end{enumerate}
\solution
%\input{exemplar/11/16/3/4/main2.tex}
\item The probability that a non leap year selected at random will contain 53 sundays.
\\
\solution
%\input{exemplar/10/13/1/19/main.tex}
\item One of the four persons John, Rita, Aslam or Gurpreet will be promoted next
month. Consequently the sample space consists of four elementary outcomes
S = {John promoted, Rita promoted, Aslam promoted, Gurpreet promoted}
You are told that the chances of John’s promotion is same as that of Gurpreet,
Rita’s chances of promotion are twice as likely as Johns. Aslam’s chances are
four times that of John.
\begin{enumerate}
	\item Determine
	\begin{enumerate}
		\item P (John promoted)
		\item P (Rita promoted)
		\item P (Aslam promoted)
		\item P (Gurpreet promoted)
	\end{enumerate}
	\item If A = {John promoted or Gurpreet promoted}, find P (A).
\end{enumerate}
\solution
%\input{exemplar/11/16/3/10/main.tex}
\item A card is drawn from a deck of 52 cards. Find the probability of getting a king or a heart or a red card.\\
\solution
%\input{exemplar/11/16/3/15/main.tex}
\item The probability that a student will pass his examination is 0.73, the probability of
the student getting a compartment is 0.13, and the probability that the student will
either pass or get compartment is 0.96. State True or False.\\
\solution
%\input{exemplar/11/16/3/31/main.tex}
\item A card is selected from a pack of 52 cards\\
\begin{enumerate}[label=(\alph*)]
\item How many points are there in the sample space?
\item Calculate the probability that the cards is an ace of spades.
\item Calculate the probability that the card is (i) an ace (ii)black card.\\
\end{enumerate}
%\input{ncert/11/16/3/4_1/Prob_4.tex}
\item In a non-leap year, the probability of having 53 tuesdays or 53 wednesdays is\\
\solution
%\input{exemplar/11/16/3/18/main.tex}
\item There are 1000 sealed envelopes in a box, 10 of them contain a cash prize of
Rs 100 each, 100 of them contain a cash prize of Rs 50 each and 200 of them
contain a cash prize of Rs 10 each and rest do not contain any cash prize. If they
are well shuffled and an envelope is picked up out, what is the probability that it
contains no cash prize?\\
\solution
%\input{exemplar/10/13/3/34/main.tex}
\item 
A die is thrown and a card is selected at random from a deck of 52 playing cards. The probability of getting an even number on the die and a spade card.\\
\solution
%\input{exemplar/12/13/3/78/main.tex}
\item
If 4-digit numbers greater than 5,000 are randomly formed from the digits 0, 1, 3, 5, and 7, what is the probability of forming a number divisible by 5 when:
\begin{enumerate}
    \item The digits are repeated?
    \item The repetition of digits is not allowed?
\end{enumerate}
\solution
%\input{ncert/11/16/4/9/main.tex}
\item Consider the probability space $\brak{\Omega, \mathcal{G}, P}$ where $\Omega = [0,2]$ and $\mathcal{G} = \cbrak{\phi, \Omega, [0,1], (1,2]}$. Let $X$ and $Y$ be two functions on $\Omega$ defined as
\begin{align*}
    X(\omega) = 
    \begin{cases}
        1 & \text{if }\omega \in [0, 1]\\
        2 & \text{if }\omega \in (1, 2]
    \end{cases}
\end{align*}
and
\begin{align*}
    Y(\omega) = 
    \begin{cases}
        2 & \text{if }\omega \in [0, 1.5]\\
        3 & \text{if }\omega \in (1.5, 2].
    \end{cases}
\end{align*}
Then which one of the following statements is true?
\begin{enumerate}
    \item [(A)] $X$ is a random variable with respect to $\mathcal{G}$, but $Y$ is not a random variable with respect to $\mathcal{G}$.
    \item [(B)] $Y$ is a random variable with respect to $\mathcal{G}$, but $X$ is not a random variable with respect to $\mathcal{G}$.
    \item [(C)] Neither $X$ nor $Y$ is a random variable with respect to $\mathcal{G}$.
    \item [(D)] Both $X$ and $Y$ are random variables with respect to $\mathcal{G}$.
\end{enumerate} \hfill (GATE ST 2023)\\
\solution
%\input{gate/ST/2023/14/main.tex}
	\item  A die is loaded in such a way that each odd number is twice as likely to occur as
each even number. Find $P(G)$, where $G$ is the event that a number greater than
3 occurs on a single roll of the die.
\\
\solution
		%\input{exemplar/11/16/3/5/main.tex}
	\item All the jacks, queens and kings are removed from a deck of 52 playing cards. The remaining cards are well shuffled and then one card is drawn at random. Giving ace a value 1 similar value for other cards, find the probability that the card has a value 
		\begin{enumerate}
			\item 7
			\item greater than 7
			\item less than 7
		\end{enumerate}
		%\input{exemplar/10/13/3/30/main.tex}
  \item A Lot consists of 48 mobile phones of which 42 are good, 3 have only minor defects and 3 have major defects.Varnika will buy a phone if it is good but the trader will only buy a mobile if it has no major defects. One phone is selected at random from the lot. What is the probability that it is
\begin{enumerate}
	\item acceptable to Varnika?
            \item acceptable to the trader?
\end{enumerate}
\solution
	%\input{exemplar/10/13/3/40/main.tex}
 \item A student says that if you throw a die, it will show up 1 or not 1. Therefore, the probability of getting 1 and the probability of getting 'not 1' each is equal to $\frac{1}{2}$. Is this correct? Give reasons.\\
 \solution
        %\input{exemplar/10/13/2/9/main.tex}
   \item Four candidates A, B, C, D have ap-
plied for the assignment to coach a school cricket
team. If A is twice as likely to be selected as B, and
B and C are given about the same chance of being
selected, while C is twice as likely to be selected
as D, what are the probabilities that
\begin{enumerate}
\item C will be selected?
\item A will not be selected?
\end{enumerate}
	%\input{exemplar/11/16/3/9/main.tex}
 \item A bag contain 24 balls of which $x$ balls are red, $2x$ are white and $3x$ are blue. A ball is selected at random, What is the probability that it is
\begin{enumerate}[label=\alph*)]
\item not red ?
\item white ?
\end{enumerate}
%\input{exemplar/10/13/3/41/main.tex}
If the letters of the word ASSASSINATION are arranged at random. Find the Probability that
\begin{enumerate}[label=(\alph*)]
\item Four $S's$ come consecutively in the word
\item Two  $I's$ and two $N's$ come together
\item All $A's$ are not coming together
\item No two $A's$ are coming together
\end{enumerate}
%\input{exemplar/11/16/3/14/main.tex}
	\item One urn contains two black balls (labelled B1 and B2) and one white ball. A
	second urn contains one black ball and two white balls (labelled W1 and W2).
	Suppose the following experiment is performed. One of the two urns is chosen
	at random. Next a ball is randomly chosen from the urn. Then a second ball is
	chosen at random from the same urn without replacing the first ball.
	
	\begin{enumerate}
	\item What is the probability that two black balls are chosen?
	
	\item What is the probability that two balls of opposite colour are chosen?
	\end{enumerate}
	\solution
	%\input{exemplar/11/16/3/12/main1.tex}
\end{enumerate}

\item In a certain lottery 10,000 tickets are sold and ten equal prizes are awarded. What is the probability of not getting a prize if you buy (a) one ticket (b) two tickets (c) 10 tickets ?	
\\
\solution
		%\begin{enumerate}[label=\thesection.\arabic*,ref=\thesection.\theenumi]
	\item One card is drawn from a well-shuffled deck of 52 cards. Find the probability of getting
\begin{enumerate}
\item A king of red colour 
\item A face card 
\item A red face card
\item The jack of hearts
\item A spade
\item The queen of diamonds

\end{enumerate}
\solution
		%\input{ncert/10/15/1/14/main.tex}
	\item Five cards—the ten, jack, queen, king and ace of diamonds, are well-shuffled with their face downwards. One card is then picked up at random.
\begin{enumerate}
\item
What is the probability that the card is the queen? 
\item
If the queen is drawn and put aside, what is the probability that the second card picked up is (a) an ace? (b) a queen?\\
\end{enumerate}
\solution
		%\input{ncert/10/15/1/15/defs.tex}
	\item A bag contains $5$ red balls and some blue balls. If the probability of drawing a blue ball is double that if a red ball, determine the number of blue balls in the bag. 
		\\
\solution
		%\input{ncert/10/15/2/3/defs.tex}
	\item A card is selected from a pack of 52 cards.
 \begin{enumerate}[label=(\alph*)] 
                 \item How many points are there in the sample space?
                 \item Calculate the probability that the card is an ace of spades.
                 \item Calculate the probability that the card is (i) an ace and (ii) black card.
 \end{enumerate}
\solution
		%\input{ncert/11/16/3/4/main.tex}
\item Four cards are drawn from a well-shuffled deck of 52 cards. What is the probability of obtaining 3 diamonds and one spade.
\\
\solution
		%\input{ncert/11/16/4/2/defs.tex}
\item In a certain lottery 10,000 tickets are sold and ten equal prizes are awarded. What is the probability of not getting a prize if you buy (a) one ticket (b) two tickets (c) 10 tickets ?	
\\
\solution
		%\input{ncert/11/16/4/4/defs.tex}
		%
\item 
Out of 100 students, two sections of 40 and 60 are formed. If you and your friend are among the 100 students, what is the probability that
\begin{enumerate}
\item you both enter the same section?
\item you both enter the different sections?
\end{enumerate}
\solution
		%\input{ncert/11/16/4/5/defs.tex}
	\item 
The number lock of a suitcase has 4 wheels each labelled with ten digits i.e. from 0 to 9.The lock opens with a sequence of four digits with no repeats.What is the probability of a person getting the right sequence to open the suitcase.
\\
\solution
		%\input{ncert/11/16/4/10/defs.tex}
		%
\item 
Two cards are drawn at random and without replacement from a pack of 52 playing cards. Find the probability that both the cards are black.
\\
\solution
		%\input{ncert/12/13/2/2/defs.tex}
		\item A box of oranges is inspected by examining three randomly selected oranges drawn without replacement. If all the three oranges are good, the box is approved for sale, otherwise, it is rejected. Find the probability that a box containing 15 oranges out of which 12 are good and 3 are bad ones will be approved for sale.
		\label{ncert/12/13/2/3/defs.tex}
		\item Two balls are drawn at random with replacement from a box containing 10 black and 8 red balls. Find the probability that
		\label{ncert/12/13/2/12}
\begin{enumerate}
\item both balls are red.
\item first ball is black and second is red.
\item one of them is black and other is red.
\end{enumerate}

\item In a hostel, 60\% of the students read Hindi newspaper, 40\% read English newspaper and 20\% read both Hindi and English newspapers. A student is selected at random.
		\label{ncert/12/13/2/15}
\begin{enumerate}
\item Find the probability that she reads neither Hindi nor English newspapers.
\item If she reads Hindi newspaper, find the probability that she reads English newspaper.
\item If she reads English newspaper, find the probability that she reads Hindi newspaper.\\
\end{enumerate}
\item The probability of obtaining an even prime number on each die, when a pair of dice is rolled is 
\begin{enumerate}
    \item $0$ 
    
    \item $\frac{1}{3}$ 
    
    \item $\frac{1}{12}$ 
    
    \item $\frac{1}{36}$ 
\end{enumerate}
\solution
		%\input{ncert/12/13/2/17/defs.tex}
	\item A bag contains 4 red and 4 black balls, another bag contains 2 red and 6 black balls. One of the two bags is selected at random and a ball is drawn from the bag which is found to be red. Find the probability that the ball is drawn from the first bag.
\\
\solution
		%\input{ncert/12/13/3/2/main.tex}
  \item
  Cards with numbers 2 to 101 are placed in a box. A card is selected at random.Find the probability that the card has
\begin{enumerate}[label=(\roman*)]
	\item an even number 
	\item a square number
\end{enumerate}
\solution
%\input{exemplar/10/13/3/32/main.tex}
\item
The king, queen and jack of clubs are removed from a deck of 52 playing cards and then well shuffled. Now one card is drawn at random from the remaining cards.  Determine the probability that the card is
\begin{enumerate}[label=(\roman*)]
\item a club
\item 10 of hearts
\end{enumerate}
\solution
%\input{exemplar/10/13/3/29/main.tex}
\item A team of medical students doing their internship have to assist during surgeries
at a city hospital. The probabilities of surgeries rated as very complex, complex,
routine, simple or very simple are respectively, 0.15, 0.20, 0.31, 0.26, .08. Find
the probabilities that a particular surgery will be rated
\begin{enumerate}
	\item complex or very complex;
	\item neither very complex nor very simple;
	\item routine or complex
	\item routine or simple
\end{enumerate}
\solution
%\input{exemplar/11/16/3/8(1)/main.tex}
\item A card is selected from a pack of 52 cards.
\begin{enumerate}[label=(\alph*)]
    \item How many points are there in the sample space?
    \item Calculate the probability that the card is an ace of spades.
    \item Calculate the probability that the card is (i) an ace and (ii) black card.
\end{enumerate}
\solution
%\input{exemplar/11/16/3/4/main2.tex}
\item The probability that a non leap year selected at random will contain 53 sundays.
\\
\solution
%\input{exemplar/10/13/1/19/main.tex}
\item One of the four persons John, Rita, Aslam or Gurpreet will be promoted next
month. Consequently the sample space consists of four elementary outcomes
S = {John promoted, Rita promoted, Aslam promoted, Gurpreet promoted}
You are told that the chances of John’s promotion is same as that of Gurpreet,
Rita’s chances of promotion are twice as likely as Johns. Aslam’s chances are
four times that of John.
\begin{enumerate}
	\item Determine
	\begin{enumerate}
		\item P (John promoted)
		\item P (Rita promoted)
		\item P (Aslam promoted)
		\item P (Gurpreet promoted)
	\end{enumerate}
	\item If A = {John promoted or Gurpreet promoted}, find P (A).
\end{enumerate}
\solution
%\input{exemplar/11/16/3/10/main.tex}
\item A card is drawn from a deck of 52 cards. Find the probability of getting a king or a heart or a red card.\\
\solution
%\input{exemplar/11/16/3/15/main.tex}
\item The probability that a student will pass his examination is 0.73, the probability of
the student getting a compartment is 0.13, and the probability that the student will
either pass or get compartment is 0.96. State True or False.\\
\solution
%\input{exemplar/11/16/3/31/main.tex}
\item A card is selected from a pack of 52 cards\\
\begin{enumerate}[label=(\alph*)]
\item How many points are there in the sample space?
\item Calculate the probability that the cards is an ace of spades.
\item Calculate the probability that the card is (i) an ace (ii)black card.\\
\end{enumerate}
%\input{ncert/11/16/3/4_1/Prob_4.tex}
\item In a non-leap year, the probability of having 53 tuesdays or 53 wednesdays is\\
\solution
%\input{exemplar/11/16/3/18/main.tex}
\item There are 1000 sealed envelopes in a box, 10 of them contain a cash prize of
Rs 100 each, 100 of them contain a cash prize of Rs 50 each and 200 of them
contain a cash prize of Rs 10 each and rest do not contain any cash prize. If they
are well shuffled and an envelope is picked up out, what is the probability that it
contains no cash prize?\\
\solution
%\input{exemplar/10/13/3/34/main.tex}
\item 
A die is thrown and a card is selected at random from a deck of 52 playing cards. The probability of getting an even number on the die and a spade card.\\
\solution
%\input{exemplar/12/13/3/78/main.tex}
\item
If 4-digit numbers greater than 5,000 are randomly formed from the digits 0, 1, 3, 5, and 7, what is the probability of forming a number divisible by 5 when:
\begin{enumerate}
    \item The digits are repeated?
    \item The repetition of digits is not allowed?
\end{enumerate}
\solution
%\input{ncert/11/16/4/9/main.tex}
\item Consider the probability space $\brak{\Omega, \mathcal{G}, P}$ where $\Omega = [0,2]$ and $\mathcal{G} = \cbrak{\phi, \Omega, [0,1], (1,2]}$. Let $X$ and $Y$ be two functions on $\Omega$ defined as
\begin{align*}
    X(\omega) = 
    \begin{cases}
        1 & \text{if }\omega \in [0, 1]\\
        2 & \text{if }\omega \in (1, 2]
    \end{cases}
\end{align*}
and
\begin{align*}
    Y(\omega) = 
    \begin{cases}
        2 & \text{if }\omega \in [0, 1.5]\\
        3 & \text{if }\omega \in (1.5, 2].
    \end{cases}
\end{align*}
Then which one of the following statements is true?
\begin{enumerate}
    \item [(A)] $X$ is a random variable with respect to $\mathcal{G}$, but $Y$ is not a random variable with respect to $\mathcal{G}$.
    \item [(B)] $Y$ is a random variable with respect to $\mathcal{G}$, but $X$ is not a random variable with respect to $\mathcal{G}$.
    \item [(C)] Neither $X$ nor $Y$ is a random variable with respect to $\mathcal{G}$.
    \item [(D)] Both $X$ and $Y$ are random variables with respect to $\mathcal{G}$.
\end{enumerate} \hfill (GATE ST 2023)\\
\solution
%\input{gate/ST/2023/14/main.tex}
	\item  A die is loaded in such a way that each odd number is twice as likely to occur as
each even number. Find $P(G)$, where $G$ is the event that a number greater than
3 occurs on a single roll of the die.
\\
\solution
		%\input{exemplar/11/16/3/5/main.tex}
	\item All the jacks, queens and kings are removed from a deck of 52 playing cards. The remaining cards are well shuffled and then one card is drawn at random. Giving ace a value 1 similar value for other cards, find the probability that the card has a value 
		\begin{enumerate}
			\item 7
			\item greater than 7
			\item less than 7
		\end{enumerate}
		%\input{exemplar/10/13/3/30/main.tex}
  \item A Lot consists of 48 mobile phones of which 42 are good, 3 have only minor defects and 3 have major defects.Varnika will buy a phone if it is good but the trader will only buy a mobile if it has no major defects. One phone is selected at random from the lot. What is the probability that it is
\begin{enumerate}
	\item acceptable to Varnika?
            \item acceptable to the trader?
\end{enumerate}
\solution
	%\input{exemplar/10/13/3/40/main.tex}
 \item A student says that if you throw a die, it will show up 1 or not 1. Therefore, the probability of getting 1 and the probability of getting 'not 1' each is equal to $\frac{1}{2}$. Is this correct? Give reasons.\\
 \solution
        %\input{exemplar/10/13/2/9/main.tex}
   \item Four candidates A, B, C, D have ap-
plied for the assignment to coach a school cricket
team. If A is twice as likely to be selected as B, and
B and C are given about the same chance of being
selected, while C is twice as likely to be selected
as D, what are the probabilities that
\begin{enumerate}
\item C will be selected?
\item A will not be selected?
\end{enumerate}
	%\input{exemplar/11/16/3/9/main.tex}
 \item A bag contain 24 balls of which $x$ balls are red, $2x$ are white and $3x$ are blue. A ball is selected at random, What is the probability that it is
\begin{enumerate}[label=\alph*)]
\item not red ?
\item white ?
\end{enumerate}
%\input{exemplar/10/13/3/41/main.tex}
If the letters of the word ASSASSINATION are arranged at random. Find the Probability that
\begin{enumerate}[label=(\alph*)]
\item Four $S's$ come consecutively in the word
\item Two  $I's$ and two $N's$ come together
\item All $A's$ are not coming together
\item No two $A's$ are coming together
\end{enumerate}
%\input{exemplar/11/16/3/14/main.tex}
	\item One urn contains two black balls (labelled B1 and B2) and one white ball. A
	second urn contains one black ball and two white balls (labelled W1 and W2).
	Suppose the following experiment is performed. One of the two urns is chosen
	at random. Next a ball is randomly chosen from the urn. Then a second ball is
	chosen at random from the same urn without replacing the first ball.
	
	\begin{enumerate}
	\item What is the probability that two black balls are chosen?
	
	\item What is the probability that two balls of opposite colour are chosen?
	\end{enumerate}
	\solution
	%\input{exemplar/11/16/3/12/main1.tex}
\end{enumerate}

		%
\item 
Out of 100 students, two sections of 40 and 60 are formed. If you and your friend are among the 100 students, what is the probability that
\begin{enumerate}
\item you both enter the same section?
\item you both enter the different sections?
\end{enumerate}
\solution
		%\begin{enumerate}[label=\thesection.\arabic*,ref=\thesection.\theenumi]
	\item One card is drawn from a well-shuffled deck of 52 cards. Find the probability of getting
\begin{enumerate}
\item A king of red colour 
\item A face card 
\item A red face card
\item The jack of hearts
\item A spade
\item The queen of diamonds

\end{enumerate}
\solution
		%\input{ncert/10/15/1/14/main.tex}
	\item Five cards—the ten, jack, queen, king and ace of diamonds, are well-shuffled with their face downwards. One card is then picked up at random.
\begin{enumerate}
\item
What is the probability that the card is the queen? 
\item
If the queen is drawn and put aside, what is the probability that the second card picked up is (a) an ace? (b) a queen?\\
\end{enumerate}
\solution
		%\input{ncert/10/15/1/15/defs.tex}
	\item A bag contains $5$ red balls and some blue balls. If the probability of drawing a blue ball is double that if a red ball, determine the number of blue balls in the bag. 
		\\
\solution
		%\input{ncert/10/15/2/3/defs.tex}
	\item A card is selected from a pack of 52 cards.
 \begin{enumerate}[label=(\alph*)] 
                 \item How many points are there in the sample space?
                 \item Calculate the probability that the card is an ace of spades.
                 \item Calculate the probability that the card is (i) an ace and (ii) black card.
 \end{enumerate}
\solution
		%\input{ncert/11/16/3/4/main.tex}
\item Four cards are drawn from a well-shuffled deck of 52 cards. What is the probability of obtaining 3 diamonds and one spade.
\\
\solution
		%\input{ncert/11/16/4/2/defs.tex}
\item In a certain lottery 10,000 tickets are sold and ten equal prizes are awarded. What is the probability of not getting a prize if you buy (a) one ticket (b) two tickets (c) 10 tickets ?	
\\
\solution
		%\input{ncert/11/16/4/4/defs.tex}
		%
\item 
Out of 100 students, two sections of 40 and 60 are formed. If you and your friend are among the 100 students, what is the probability that
\begin{enumerate}
\item you both enter the same section?
\item you both enter the different sections?
\end{enumerate}
\solution
		%\input{ncert/11/16/4/5/defs.tex}
	\item 
The number lock of a suitcase has 4 wheels each labelled with ten digits i.e. from 0 to 9.The lock opens with a sequence of four digits with no repeats.What is the probability of a person getting the right sequence to open the suitcase.
\\
\solution
		%\input{ncert/11/16/4/10/defs.tex}
		%
\item 
Two cards are drawn at random and without replacement from a pack of 52 playing cards. Find the probability that both the cards are black.
\\
\solution
		%\input{ncert/12/13/2/2/defs.tex}
		\item A box of oranges is inspected by examining three randomly selected oranges drawn without replacement. If all the three oranges are good, the box is approved for sale, otherwise, it is rejected. Find the probability that a box containing 15 oranges out of which 12 are good and 3 are bad ones will be approved for sale.
		\label{ncert/12/13/2/3/defs.tex}
		\item Two balls are drawn at random with replacement from a box containing 10 black and 8 red balls. Find the probability that
		\label{ncert/12/13/2/12}
\begin{enumerate}
\item both balls are red.
\item first ball is black and second is red.
\item one of them is black and other is red.
\end{enumerate}

\item In a hostel, 60\% of the students read Hindi newspaper, 40\% read English newspaper and 20\% read both Hindi and English newspapers. A student is selected at random.
		\label{ncert/12/13/2/15}
\begin{enumerate}
\item Find the probability that she reads neither Hindi nor English newspapers.
\item If she reads Hindi newspaper, find the probability that she reads English newspaper.
\item If she reads English newspaper, find the probability that she reads Hindi newspaper.\\
\end{enumerate}
\item The probability of obtaining an even prime number on each die, when a pair of dice is rolled is 
\begin{enumerate}
    \item $0$ 
    
    \item $\frac{1}{3}$ 
    
    \item $\frac{1}{12}$ 
    
    \item $\frac{1}{36}$ 
\end{enumerate}
\solution
		%\input{ncert/12/13/2/17/defs.tex}
	\item A bag contains 4 red and 4 black balls, another bag contains 2 red and 6 black balls. One of the two bags is selected at random and a ball is drawn from the bag which is found to be red. Find the probability that the ball is drawn from the first bag.
\\
\solution
		%\input{ncert/12/13/3/2/main.tex}
  \item
  Cards with numbers 2 to 101 are placed in a box. A card is selected at random.Find the probability that the card has
\begin{enumerate}[label=(\roman*)]
	\item an even number 
	\item a square number
\end{enumerate}
\solution
%\input{exemplar/10/13/3/32/main.tex}
\item
The king, queen and jack of clubs are removed from a deck of 52 playing cards and then well shuffled. Now one card is drawn at random from the remaining cards.  Determine the probability that the card is
\begin{enumerate}[label=(\roman*)]
\item a club
\item 10 of hearts
\end{enumerate}
\solution
%\input{exemplar/10/13/3/29/main.tex}
\item A team of medical students doing their internship have to assist during surgeries
at a city hospital. The probabilities of surgeries rated as very complex, complex,
routine, simple or very simple are respectively, 0.15, 0.20, 0.31, 0.26, .08. Find
the probabilities that a particular surgery will be rated
\begin{enumerate}
	\item complex or very complex;
	\item neither very complex nor very simple;
	\item routine or complex
	\item routine or simple
\end{enumerate}
\solution
%\input{exemplar/11/16/3/8(1)/main.tex}
\item A card is selected from a pack of 52 cards.
\begin{enumerate}[label=(\alph*)]
    \item How many points are there in the sample space?
    \item Calculate the probability that the card is an ace of spades.
    \item Calculate the probability that the card is (i) an ace and (ii) black card.
\end{enumerate}
\solution
%\input{exemplar/11/16/3/4/main2.tex}
\item The probability that a non leap year selected at random will contain 53 sundays.
\\
\solution
%\input{exemplar/10/13/1/19/main.tex}
\item One of the four persons John, Rita, Aslam or Gurpreet will be promoted next
month. Consequently the sample space consists of four elementary outcomes
S = {John promoted, Rita promoted, Aslam promoted, Gurpreet promoted}
You are told that the chances of John’s promotion is same as that of Gurpreet,
Rita’s chances of promotion are twice as likely as Johns. Aslam’s chances are
four times that of John.
\begin{enumerate}
	\item Determine
	\begin{enumerate}
		\item P (John promoted)
		\item P (Rita promoted)
		\item P (Aslam promoted)
		\item P (Gurpreet promoted)
	\end{enumerate}
	\item If A = {John promoted or Gurpreet promoted}, find P (A).
\end{enumerate}
\solution
%\input{exemplar/11/16/3/10/main.tex}
\item A card is drawn from a deck of 52 cards. Find the probability of getting a king or a heart or a red card.\\
\solution
%\input{exemplar/11/16/3/15/main.tex}
\item The probability that a student will pass his examination is 0.73, the probability of
the student getting a compartment is 0.13, and the probability that the student will
either pass or get compartment is 0.96. State True or False.\\
\solution
%\input{exemplar/11/16/3/31/main.tex}
\item A card is selected from a pack of 52 cards\\
\begin{enumerate}[label=(\alph*)]
\item How many points are there in the sample space?
\item Calculate the probability that the cards is an ace of spades.
\item Calculate the probability that the card is (i) an ace (ii)black card.\\
\end{enumerate}
%\input{ncert/11/16/3/4_1/Prob_4.tex}
\item In a non-leap year, the probability of having 53 tuesdays or 53 wednesdays is\\
\solution
%\input{exemplar/11/16/3/18/main.tex}
\item There are 1000 sealed envelopes in a box, 10 of them contain a cash prize of
Rs 100 each, 100 of them contain a cash prize of Rs 50 each and 200 of them
contain a cash prize of Rs 10 each and rest do not contain any cash prize. If they
are well shuffled and an envelope is picked up out, what is the probability that it
contains no cash prize?\\
\solution
%\input{exemplar/10/13/3/34/main.tex}
\item 
A die is thrown and a card is selected at random from a deck of 52 playing cards. The probability of getting an even number on the die and a spade card.\\
\solution
%\input{exemplar/12/13/3/78/main.tex}
\item
If 4-digit numbers greater than 5,000 are randomly formed from the digits 0, 1, 3, 5, and 7, what is the probability of forming a number divisible by 5 when:
\begin{enumerate}
    \item The digits are repeated?
    \item The repetition of digits is not allowed?
\end{enumerate}
\solution
%\input{ncert/11/16/4/9/main.tex}
\item Consider the probability space $\brak{\Omega, \mathcal{G}, P}$ where $\Omega = [0,2]$ and $\mathcal{G} = \cbrak{\phi, \Omega, [0,1], (1,2]}$. Let $X$ and $Y$ be two functions on $\Omega$ defined as
\begin{align*}
    X(\omega) = 
    \begin{cases}
        1 & \text{if }\omega \in [0, 1]\\
        2 & \text{if }\omega \in (1, 2]
    \end{cases}
\end{align*}
and
\begin{align*}
    Y(\omega) = 
    \begin{cases}
        2 & \text{if }\omega \in [0, 1.5]\\
        3 & \text{if }\omega \in (1.5, 2].
    \end{cases}
\end{align*}
Then which one of the following statements is true?
\begin{enumerate}
    \item [(A)] $X$ is a random variable with respect to $\mathcal{G}$, but $Y$ is not a random variable with respect to $\mathcal{G}$.
    \item [(B)] $Y$ is a random variable with respect to $\mathcal{G}$, but $X$ is not a random variable with respect to $\mathcal{G}$.
    \item [(C)] Neither $X$ nor $Y$ is a random variable with respect to $\mathcal{G}$.
    \item [(D)] Both $X$ and $Y$ are random variables with respect to $\mathcal{G}$.
\end{enumerate} \hfill (GATE ST 2023)\\
\solution
%\input{gate/ST/2023/14/main.tex}
	\item  A die is loaded in such a way that each odd number is twice as likely to occur as
each even number. Find $P(G)$, where $G$ is the event that a number greater than
3 occurs on a single roll of the die.
\\
\solution
		%\input{exemplar/11/16/3/5/main.tex}
	\item All the jacks, queens and kings are removed from a deck of 52 playing cards. The remaining cards are well shuffled and then one card is drawn at random. Giving ace a value 1 similar value for other cards, find the probability that the card has a value 
		\begin{enumerate}
			\item 7
			\item greater than 7
			\item less than 7
		\end{enumerate}
		%\input{exemplar/10/13/3/30/main.tex}
  \item A Lot consists of 48 mobile phones of which 42 are good, 3 have only minor defects and 3 have major defects.Varnika will buy a phone if it is good but the trader will only buy a mobile if it has no major defects. One phone is selected at random from the lot. What is the probability that it is
\begin{enumerate}
	\item acceptable to Varnika?
            \item acceptable to the trader?
\end{enumerate}
\solution
	%\input{exemplar/10/13/3/40/main.tex}
 \item A student says that if you throw a die, it will show up 1 or not 1. Therefore, the probability of getting 1 and the probability of getting 'not 1' each is equal to $\frac{1}{2}$. Is this correct? Give reasons.\\
 \solution
        %\input{exemplar/10/13/2/9/main.tex}
   \item Four candidates A, B, C, D have ap-
plied for the assignment to coach a school cricket
team. If A is twice as likely to be selected as B, and
B and C are given about the same chance of being
selected, while C is twice as likely to be selected
as D, what are the probabilities that
\begin{enumerate}
\item C will be selected?
\item A will not be selected?
\end{enumerate}
	%\input{exemplar/11/16/3/9/main.tex}
 \item A bag contain 24 balls of which $x$ balls are red, $2x$ are white and $3x$ are blue. A ball is selected at random, What is the probability that it is
\begin{enumerate}[label=\alph*)]
\item not red ?
\item white ?
\end{enumerate}
%\input{exemplar/10/13/3/41/main.tex}
If the letters of the word ASSASSINATION are arranged at random. Find the Probability that
\begin{enumerate}[label=(\alph*)]
\item Four $S's$ come consecutively in the word
\item Two  $I's$ and two $N's$ come together
\item All $A's$ are not coming together
\item No two $A's$ are coming together
\end{enumerate}
%\input{exemplar/11/16/3/14/main.tex}
	\item One urn contains two black balls (labelled B1 and B2) and one white ball. A
	second urn contains one black ball and two white balls (labelled W1 and W2).
	Suppose the following experiment is performed. One of the two urns is chosen
	at random. Next a ball is randomly chosen from the urn. Then a second ball is
	chosen at random from the same urn without replacing the first ball.
	
	\begin{enumerate}
	\item What is the probability that two black balls are chosen?
	
	\item What is the probability that two balls of opposite colour are chosen?
	\end{enumerate}
	\solution
	%\input{exemplar/11/16/3/12/main1.tex}
\end{enumerate}

	\item 
The number lock of a suitcase has 4 wheels each labelled with ten digits i.e. from 0 to 9.The lock opens with a sequence of four digits with no repeats.What is the probability of a person getting the right sequence to open the suitcase.
\\
\solution
		%\begin{enumerate}[label=\thesection.\arabic*,ref=\thesection.\theenumi]
	\item One card is drawn from a well-shuffled deck of 52 cards. Find the probability of getting
\begin{enumerate}
\item A king of red colour 
\item A face card 
\item A red face card
\item The jack of hearts
\item A spade
\item The queen of diamonds

\end{enumerate}
\solution
		%\input{ncert/10/15/1/14/main.tex}
	\item Five cards—the ten, jack, queen, king and ace of diamonds, are well-shuffled with their face downwards. One card is then picked up at random.
\begin{enumerate}
\item
What is the probability that the card is the queen? 
\item
If the queen is drawn and put aside, what is the probability that the second card picked up is (a) an ace? (b) a queen?\\
\end{enumerate}
\solution
		%\input{ncert/10/15/1/15/defs.tex}
	\item A bag contains $5$ red balls and some blue balls. If the probability of drawing a blue ball is double that if a red ball, determine the number of blue balls in the bag. 
		\\
\solution
		%\input{ncert/10/15/2/3/defs.tex}
	\item A card is selected from a pack of 52 cards.
 \begin{enumerate}[label=(\alph*)] 
                 \item How many points are there in the sample space?
                 \item Calculate the probability that the card is an ace of spades.
                 \item Calculate the probability that the card is (i) an ace and (ii) black card.
 \end{enumerate}
\solution
		%\input{ncert/11/16/3/4/main.tex}
\item Four cards are drawn from a well-shuffled deck of 52 cards. What is the probability of obtaining 3 diamonds and one spade.
\\
\solution
		%\input{ncert/11/16/4/2/defs.tex}
\item In a certain lottery 10,000 tickets are sold and ten equal prizes are awarded. What is the probability of not getting a prize if you buy (a) one ticket (b) two tickets (c) 10 tickets ?	
\\
\solution
		%\input{ncert/11/16/4/4/defs.tex}
		%
\item 
Out of 100 students, two sections of 40 and 60 are formed. If you and your friend are among the 100 students, what is the probability that
\begin{enumerate}
\item you both enter the same section?
\item you both enter the different sections?
\end{enumerate}
\solution
		%\input{ncert/11/16/4/5/defs.tex}
	\item 
The number lock of a suitcase has 4 wheels each labelled with ten digits i.e. from 0 to 9.The lock opens with a sequence of four digits with no repeats.What is the probability of a person getting the right sequence to open the suitcase.
\\
\solution
		%\input{ncert/11/16/4/10/defs.tex}
		%
\item 
Two cards are drawn at random and without replacement from a pack of 52 playing cards. Find the probability that both the cards are black.
\\
\solution
		%\input{ncert/12/13/2/2/defs.tex}
		\item A box of oranges is inspected by examining three randomly selected oranges drawn without replacement. If all the three oranges are good, the box is approved for sale, otherwise, it is rejected. Find the probability that a box containing 15 oranges out of which 12 are good and 3 are bad ones will be approved for sale.
		\label{ncert/12/13/2/3/defs.tex}
		\item Two balls are drawn at random with replacement from a box containing 10 black and 8 red balls. Find the probability that
		\label{ncert/12/13/2/12}
\begin{enumerate}
\item both balls are red.
\item first ball is black and second is red.
\item one of them is black and other is red.
\end{enumerate}

\item In a hostel, 60\% of the students read Hindi newspaper, 40\% read English newspaper and 20\% read both Hindi and English newspapers. A student is selected at random.
		\label{ncert/12/13/2/15}
\begin{enumerate}
\item Find the probability that she reads neither Hindi nor English newspapers.
\item If she reads Hindi newspaper, find the probability that she reads English newspaper.
\item If she reads English newspaper, find the probability that she reads Hindi newspaper.\\
\end{enumerate}
\item The probability of obtaining an even prime number on each die, when a pair of dice is rolled is 
\begin{enumerate}
    \item $0$ 
    
    \item $\frac{1}{3}$ 
    
    \item $\frac{1}{12}$ 
    
    \item $\frac{1}{36}$ 
\end{enumerate}
\solution
		%\input{ncert/12/13/2/17/defs.tex}
	\item A bag contains 4 red and 4 black balls, another bag contains 2 red and 6 black balls. One of the two bags is selected at random and a ball is drawn from the bag which is found to be red. Find the probability that the ball is drawn from the first bag.
\\
\solution
		%\input{ncert/12/13/3/2/main.tex}
  \item
  Cards with numbers 2 to 101 are placed in a box. A card is selected at random.Find the probability that the card has
\begin{enumerate}[label=(\roman*)]
	\item an even number 
	\item a square number
\end{enumerate}
\solution
%\input{exemplar/10/13/3/32/main.tex}
\item
The king, queen and jack of clubs are removed from a deck of 52 playing cards and then well shuffled. Now one card is drawn at random from the remaining cards.  Determine the probability that the card is
\begin{enumerate}[label=(\roman*)]
\item a club
\item 10 of hearts
\end{enumerate}
\solution
%\input{exemplar/10/13/3/29/main.tex}
\item A team of medical students doing their internship have to assist during surgeries
at a city hospital. The probabilities of surgeries rated as very complex, complex,
routine, simple or very simple are respectively, 0.15, 0.20, 0.31, 0.26, .08. Find
the probabilities that a particular surgery will be rated
\begin{enumerate}
	\item complex or very complex;
	\item neither very complex nor very simple;
	\item routine or complex
	\item routine or simple
\end{enumerate}
\solution
%\input{exemplar/11/16/3/8(1)/main.tex}
\item A card is selected from a pack of 52 cards.
\begin{enumerate}[label=(\alph*)]
    \item How many points are there in the sample space?
    \item Calculate the probability that the card is an ace of spades.
    \item Calculate the probability that the card is (i) an ace and (ii) black card.
\end{enumerate}
\solution
%\input{exemplar/11/16/3/4/main2.tex}
\item The probability that a non leap year selected at random will contain 53 sundays.
\\
\solution
%\input{exemplar/10/13/1/19/main.tex}
\item One of the four persons John, Rita, Aslam or Gurpreet will be promoted next
month. Consequently the sample space consists of four elementary outcomes
S = {John promoted, Rita promoted, Aslam promoted, Gurpreet promoted}
You are told that the chances of John’s promotion is same as that of Gurpreet,
Rita’s chances of promotion are twice as likely as Johns. Aslam’s chances are
four times that of John.
\begin{enumerate}
	\item Determine
	\begin{enumerate}
		\item P (John promoted)
		\item P (Rita promoted)
		\item P (Aslam promoted)
		\item P (Gurpreet promoted)
	\end{enumerate}
	\item If A = {John promoted or Gurpreet promoted}, find P (A).
\end{enumerate}
\solution
%\input{exemplar/11/16/3/10/main.tex}
\item A card is drawn from a deck of 52 cards. Find the probability of getting a king or a heart or a red card.\\
\solution
%\input{exemplar/11/16/3/15/main.tex}
\item The probability that a student will pass his examination is 0.73, the probability of
the student getting a compartment is 0.13, and the probability that the student will
either pass or get compartment is 0.96. State True or False.\\
\solution
%\input{exemplar/11/16/3/31/main.tex}
\item A card is selected from a pack of 52 cards\\
\begin{enumerate}[label=(\alph*)]
\item How many points are there in the sample space?
\item Calculate the probability that the cards is an ace of spades.
\item Calculate the probability that the card is (i) an ace (ii)black card.\\
\end{enumerate}
%\input{ncert/11/16/3/4_1/Prob_4.tex}
\item In a non-leap year, the probability of having 53 tuesdays or 53 wednesdays is\\
\solution
%\input{exemplar/11/16/3/18/main.tex}
\item There are 1000 sealed envelopes in a box, 10 of them contain a cash prize of
Rs 100 each, 100 of them contain a cash prize of Rs 50 each and 200 of them
contain a cash prize of Rs 10 each and rest do not contain any cash prize. If they
are well shuffled and an envelope is picked up out, what is the probability that it
contains no cash prize?\\
\solution
%\input{exemplar/10/13/3/34/main.tex}
\item 
A die is thrown and a card is selected at random from a deck of 52 playing cards. The probability of getting an even number on the die and a spade card.\\
\solution
%\input{exemplar/12/13/3/78/main.tex}
\item
If 4-digit numbers greater than 5,000 are randomly formed from the digits 0, 1, 3, 5, and 7, what is the probability of forming a number divisible by 5 when:
\begin{enumerate}
    \item The digits are repeated?
    \item The repetition of digits is not allowed?
\end{enumerate}
\solution
%\input{ncert/11/16/4/9/main.tex}
\item Consider the probability space $\brak{\Omega, \mathcal{G}, P}$ where $\Omega = [0,2]$ and $\mathcal{G} = \cbrak{\phi, \Omega, [0,1], (1,2]}$. Let $X$ and $Y$ be two functions on $\Omega$ defined as
\begin{align*}
    X(\omega) = 
    \begin{cases}
        1 & \text{if }\omega \in [0, 1]\\
        2 & \text{if }\omega \in (1, 2]
    \end{cases}
\end{align*}
and
\begin{align*}
    Y(\omega) = 
    \begin{cases}
        2 & \text{if }\omega \in [0, 1.5]\\
        3 & \text{if }\omega \in (1.5, 2].
    \end{cases}
\end{align*}
Then which one of the following statements is true?
\begin{enumerate}
    \item [(A)] $X$ is a random variable with respect to $\mathcal{G}$, but $Y$ is not a random variable with respect to $\mathcal{G}$.
    \item [(B)] $Y$ is a random variable with respect to $\mathcal{G}$, but $X$ is not a random variable with respect to $\mathcal{G}$.
    \item [(C)] Neither $X$ nor $Y$ is a random variable with respect to $\mathcal{G}$.
    \item [(D)] Both $X$ and $Y$ are random variables with respect to $\mathcal{G}$.
\end{enumerate} \hfill (GATE ST 2023)\\
\solution
%\input{gate/ST/2023/14/main.tex}
	\item  A die is loaded in such a way that each odd number is twice as likely to occur as
each even number. Find $P(G)$, where $G$ is the event that a number greater than
3 occurs on a single roll of the die.
\\
\solution
		%\input{exemplar/11/16/3/5/main.tex}
	\item All the jacks, queens and kings are removed from a deck of 52 playing cards. The remaining cards are well shuffled and then one card is drawn at random. Giving ace a value 1 similar value for other cards, find the probability that the card has a value 
		\begin{enumerate}
			\item 7
			\item greater than 7
			\item less than 7
		\end{enumerate}
		%\input{exemplar/10/13/3/30/main.tex}
  \item A Lot consists of 48 mobile phones of which 42 are good, 3 have only minor defects and 3 have major defects.Varnika will buy a phone if it is good but the trader will only buy a mobile if it has no major defects. One phone is selected at random from the lot. What is the probability that it is
\begin{enumerate}
	\item acceptable to Varnika?
            \item acceptable to the trader?
\end{enumerate}
\solution
	%\input{exemplar/10/13/3/40/main.tex}
 \item A student says that if you throw a die, it will show up 1 or not 1. Therefore, the probability of getting 1 and the probability of getting 'not 1' each is equal to $\frac{1}{2}$. Is this correct? Give reasons.\\
 \solution
        %\input{exemplar/10/13/2/9/main.tex}
   \item Four candidates A, B, C, D have ap-
plied for the assignment to coach a school cricket
team. If A is twice as likely to be selected as B, and
B and C are given about the same chance of being
selected, while C is twice as likely to be selected
as D, what are the probabilities that
\begin{enumerate}
\item C will be selected?
\item A will not be selected?
\end{enumerate}
	%\input{exemplar/11/16/3/9/main.tex}
 \item A bag contain 24 balls of which $x$ balls are red, $2x$ are white and $3x$ are blue. A ball is selected at random, What is the probability that it is
\begin{enumerate}[label=\alph*)]
\item not red ?
\item white ?
\end{enumerate}
%\input{exemplar/10/13/3/41/main.tex}
If the letters of the word ASSASSINATION are arranged at random. Find the Probability that
\begin{enumerate}[label=(\alph*)]
\item Four $S's$ come consecutively in the word
\item Two  $I's$ and two $N's$ come together
\item All $A's$ are not coming together
\item No two $A's$ are coming together
\end{enumerate}
%\input{exemplar/11/16/3/14/main.tex}
	\item One urn contains two black balls (labelled B1 and B2) and one white ball. A
	second urn contains one black ball and two white balls (labelled W1 and W2).
	Suppose the following experiment is performed. One of the two urns is chosen
	at random. Next a ball is randomly chosen from the urn. Then a second ball is
	chosen at random from the same urn without replacing the first ball.
	
	\begin{enumerate}
	\item What is the probability that two black balls are chosen?
	
	\item What is the probability that two balls of opposite colour are chosen?
	\end{enumerate}
	\solution
	%\input{exemplar/11/16/3/12/main1.tex}
\end{enumerate}

		%
\item 
Two cards are drawn at random and without replacement from a pack of 52 playing cards. Find the probability that both the cards are black.
\\
\solution
		%\begin{enumerate}[label=\thesection.\arabic*,ref=\thesection.\theenumi]
	\item One card is drawn from a well-shuffled deck of 52 cards. Find the probability of getting
\begin{enumerate}
\item A king of red colour 
\item A face card 
\item A red face card
\item The jack of hearts
\item A spade
\item The queen of diamonds

\end{enumerate}
\solution
		%\input{ncert/10/15/1/14/main.tex}
	\item Five cards—the ten, jack, queen, king and ace of diamonds, are well-shuffled with their face downwards. One card is then picked up at random.
\begin{enumerate}
\item
What is the probability that the card is the queen? 
\item
If the queen is drawn and put aside, what is the probability that the second card picked up is (a) an ace? (b) a queen?\\
\end{enumerate}
\solution
		%\input{ncert/10/15/1/15/defs.tex}
	\item A bag contains $5$ red balls and some blue balls. If the probability of drawing a blue ball is double that if a red ball, determine the number of blue balls in the bag. 
		\\
\solution
		%\input{ncert/10/15/2/3/defs.tex}
	\item A card is selected from a pack of 52 cards.
 \begin{enumerate}[label=(\alph*)] 
                 \item How many points are there in the sample space?
                 \item Calculate the probability that the card is an ace of spades.
                 \item Calculate the probability that the card is (i) an ace and (ii) black card.
 \end{enumerate}
\solution
		%\input{ncert/11/16/3/4/main.tex}
\item Four cards are drawn from a well-shuffled deck of 52 cards. What is the probability of obtaining 3 diamonds and one spade.
\\
\solution
		%\input{ncert/11/16/4/2/defs.tex}
\item In a certain lottery 10,000 tickets are sold and ten equal prizes are awarded. What is the probability of not getting a prize if you buy (a) one ticket (b) two tickets (c) 10 tickets ?	
\\
\solution
		%\input{ncert/11/16/4/4/defs.tex}
		%
\item 
Out of 100 students, two sections of 40 and 60 are formed. If you and your friend are among the 100 students, what is the probability that
\begin{enumerate}
\item you both enter the same section?
\item you both enter the different sections?
\end{enumerate}
\solution
		%\input{ncert/11/16/4/5/defs.tex}
	\item 
The number lock of a suitcase has 4 wheels each labelled with ten digits i.e. from 0 to 9.The lock opens with a sequence of four digits with no repeats.What is the probability of a person getting the right sequence to open the suitcase.
\\
\solution
		%\input{ncert/11/16/4/10/defs.tex}
		%
\item 
Two cards are drawn at random and without replacement from a pack of 52 playing cards. Find the probability that both the cards are black.
\\
\solution
		%\input{ncert/12/13/2/2/defs.tex}
		\item A box of oranges is inspected by examining three randomly selected oranges drawn without replacement. If all the three oranges are good, the box is approved for sale, otherwise, it is rejected. Find the probability that a box containing 15 oranges out of which 12 are good and 3 are bad ones will be approved for sale.
		\label{ncert/12/13/2/3/defs.tex}
		\item Two balls are drawn at random with replacement from a box containing 10 black and 8 red balls. Find the probability that
		\label{ncert/12/13/2/12}
\begin{enumerate}
\item both balls are red.
\item first ball is black and second is red.
\item one of them is black and other is red.
\end{enumerate}

\item In a hostel, 60\% of the students read Hindi newspaper, 40\% read English newspaper and 20\% read both Hindi and English newspapers. A student is selected at random.
		\label{ncert/12/13/2/15}
\begin{enumerate}
\item Find the probability that she reads neither Hindi nor English newspapers.
\item If she reads Hindi newspaper, find the probability that she reads English newspaper.
\item If she reads English newspaper, find the probability that she reads Hindi newspaper.\\
\end{enumerate}
\item The probability of obtaining an even prime number on each die, when a pair of dice is rolled is 
\begin{enumerate}
    \item $0$ 
    
    \item $\frac{1}{3}$ 
    
    \item $\frac{1}{12}$ 
    
    \item $\frac{1}{36}$ 
\end{enumerate}
\solution
		%\input{ncert/12/13/2/17/defs.tex}
	\item A bag contains 4 red and 4 black balls, another bag contains 2 red and 6 black balls. One of the two bags is selected at random and a ball is drawn from the bag which is found to be red. Find the probability that the ball is drawn from the first bag.
\\
\solution
		%\input{ncert/12/13/3/2/main.tex}
  \item
  Cards with numbers 2 to 101 are placed in a box. A card is selected at random.Find the probability that the card has
\begin{enumerate}[label=(\roman*)]
	\item an even number 
	\item a square number
\end{enumerate}
\solution
%\input{exemplar/10/13/3/32/main.tex}
\item
The king, queen and jack of clubs are removed from a deck of 52 playing cards and then well shuffled. Now one card is drawn at random from the remaining cards.  Determine the probability that the card is
\begin{enumerate}[label=(\roman*)]
\item a club
\item 10 of hearts
\end{enumerate}
\solution
%\input{exemplar/10/13/3/29/main.tex}
\item A team of medical students doing their internship have to assist during surgeries
at a city hospital. The probabilities of surgeries rated as very complex, complex,
routine, simple or very simple are respectively, 0.15, 0.20, 0.31, 0.26, .08. Find
the probabilities that a particular surgery will be rated
\begin{enumerate}
	\item complex or very complex;
	\item neither very complex nor very simple;
	\item routine or complex
	\item routine or simple
\end{enumerate}
\solution
%\input{exemplar/11/16/3/8(1)/main.tex}
\item A card is selected from a pack of 52 cards.
\begin{enumerate}[label=(\alph*)]
    \item How many points are there in the sample space?
    \item Calculate the probability that the card is an ace of spades.
    \item Calculate the probability that the card is (i) an ace and (ii) black card.
\end{enumerate}
\solution
%\input{exemplar/11/16/3/4/main2.tex}
\item The probability that a non leap year selected at random will contain 53 sundays.
\\
\solution
%\input{exemplar/10/13/1/19/main.tex}
\item One of the four persons John, Rita, Aslam or Gurpreet will be promoted next
month. Consequently the sample space consists of four elementary outcomes
S = {John promoted, Rita promoted, Aslam promoted, Gurpreet promoted}
You are told that the chances of John’s promotion is same as that of Gurpreet,
Rita’s chances of promotion are twice as likely as Johns. Aslam’s chances are
four times that of John.
\begin{enumerate}
	\item Determine
	\begin{enumerate}
		\item P (John promoted)
		\item P (Rita promoted)
		\item P (Aslam promoted)
		\item P (Gurpreet promoted)
	\end{enumerate}
	\item If A = {John promoted or Gurpreet promoted}, find P (A).
\end{enumerate}
\solution
%\input{exemplar/11/16/3/10/main.tex}
\item A card is drawn from a deck of 52 cards. Find the probability of getting a king or a heart or a red card.\\
\solution
%\input{exemplar/11/16/3/15/main.tex}
\item The probability that a student will pass his examination is 0.73, the probability of
the student getting a compartment is 0.13, and the probability that the student will
either pass or get compartment is 0.96. State True or False.\\
\solution
%\input{exemplar/11/16/3/31/main.tex}
\item A card is selected from a pack of 52 cards\\
\begin{enumerate}[label=(\alph*)]
\item How many points are there in the sample space?
\item Calculate the probability that the cards is an ace of spades.
\item Calculate the probability that the card is (i) an ace (ii)black card.\\
\end{enumerate}
%\input{ncert/11/16/3/4_1/Prob_4.tex}
\item In a non-leap year, the probability of having 53 tuesdays or 53 wednesdays is\\
\solution
%\input{exemplar/11/16/3/18/main.tex}
\item There are 1000 sealed envelopes in a box, 10 of them contain a cash prize of
Rs 100 each, 100 of them contain a cash prize of Rs 50 each and 200 of them
contain a cash prize of Rs 10 each and rest do not contain any cash prize. If they
are well shuffled and an envelope is picked up out, what is the probability that it
contains no cash prize?\\
\solution
%\input{exemplar/10/13/3/34/main.tex}
\item 
A die is thrown and a card is selected at random from a deck of 52 playing cards. The probability of getting an even number on the die and a spade card.\\
\solution
%\input{exemplar/12/13/3/78/main.tex}
\item
If 4-digit numbers greater than 5,000 are randomly formed from the digits 0, 1, 3, 5, and 7, what is the probability of forming a number divisible by 5 when:
\begin{enumerate}
    \item The digits are repeated?
    \item The repetition of digits is not allowed?
\end{enumerate}
\solution
%\input{ncert/11/16/4/9/main.tex}
\item Consider the probability space $\brak{\Omega, \mathcal{G}, P}$ where $\Omega = [0,2]$ and $\mathcal{G} = \cbrak{\phi, \Omega, [0,1], (1,2]}$. Let $X$ and $Y$ be two functions on $\Omega$ defined as
\begin{align*}
    X(\omega) = 
    \begin{cases}
        1 & \text{if }\omega \in [0, 1]\\
        2 & \text{if }\omega \in (1, 2]
    \end{cases}
\end{align*}
and
\begin{align*}
    Y(\omega) = 
    \begin{cases}
        2 & \text{if }\omega \in [0, 1.5]\\
        3 & \text{if }\omega \in (1.5, 2].
    \end{cases}
\end{align*}
Then which one of the following statements is true?
\begin{enumerate}
    \item [(A)] $X$ is a random variable with respect to $\mathcal{G}$, but $Y$ is not a random variable with respect to $\mathcal{G}$.
    \item [(B)] $Y$ is a random variable with respect to $\mathcal{G}$, but $X$ is not a random variable with respect to $\mathcal{G}$.
    \item [(C)] Neither $X$ nor $Y$ is a random variable with respect to $\mathcal{G}$.
    \item [(D)] Both $X$ and $Y$ are random variables with respect to $\mathcal{G}$.
\end{enumerate} \hfill (GATE ST 2023)\\
\solution
%\input{gate/ST/2023/14/main.tex}
	\item  A die is loaded in such a way that each odd number is twice as likely to occur as
each even number. Find $P(G)$, where $G$ is the event that a number greater than
3 occurs on a single roll of the die.
\\
\solution
		%\input{exemplar/11/16/3/5/main.tex}
	\item All the jacks, queens and kings are removed from a deck of 52 playing cards. The remaining cards are well shuffled and then one card is drawn at random. Giving ace a value 1 similar value for other cards, find the probability that the card has a value 
		\begin{enumerate}
			\item 7
			\item greater than 7
			\item less than 7
		\end{enumerate}
		%\input{exemplar/10/13/3/30/main.tex}
  \item A Lot consists of 48 mobile phones of which 42 are good, 3 have only minor defects and 3 have major defects.Varnika will buy a phone if it is good but the trader will only buy a mobile if it has no major defects. One phone is selected at random from the lot. What is the probability that it is
\begin{enumerate}
	\item acceptable to Varnika?
            \item acceptable to the trader?
\end{enumerate}
\solution
	%\input{exemplar/10/13/3/40/main.tex}
 \item A student says that if you throw a die, it will show up 1 or not 1. Therefore, the probability of getting 1 and the probability of getting 'not 1' each is equal to $\frac{1}{2}$. Is this correct? Give reasons.\\
 \solution
        %\input{exemplar/10/13/2/9/main.tex}
   \item Four candidates A, B, C, D have ap-
plied for the assignment to coach a school cricket
team. If A is twice as likely to be selected as B, and
B and C are given about the same chance of being
selected, while C is twice as likely to be selected
as D, what are the probabilities that
\begin{enumerate}
\item C will be selected?
\item A will not be selected?
\end{enumerate}
	%\input{exemplar/11/16/3/9/main.tex}
 \item A bag contain 24 balls of which $x$ balls are red, $2x$ are white and $3x$ are blue. A ball is selected at random, What is the probability that it is
\begin{enumerate}[label=\alph*)]
\item not red ?
\item white ?
\end{enumerate}
%\input{exemplar/10/13/3/41/main.tex}
If the letters of the word ASSASSINATION are arranged at random. Find the Probability that
\begin{enumerate}[label=(\alph*)]
\item Four $S's$ come consecutively in the word
\item Two  $I's$ and two $N's$ come together
\item All $A's$ are not coming together
\item No two $A's$ are coming together
\end{enumerate}
%\input{exemplar/11/16/3/14/main.tex}
	\item One urn contains two black balls (labelled B1 and B2) and one white ball. A
	second urn contains one black ball and two white balls (labelled W1 and W2).
	Suppose the following experiment is performed. One of the two urns is chosen
	at random. Next a ball is randomly chosen from the urn. Then a second ball is
	chosen at random from the same urn without replacing the first ball.
	
	\begin{enumerate}
	\item What is the probability that two black balls are chosen?
	
	\item What is the probability that two balls of opposite colour are chosen?
	\end{enumerate}
	\solution
	%\input{exemplar/11/16/3/12/main1.tex}
\end{enumerate}

		\item A box of oranges is inspected by examining three randomly selected oranges drawn without replacement. If all the three oranges are good, the box is approved for sale, otherwise, it is rejected. Find the probability that a box containing 15 oranges out of which 12 are good and 3 are bad ones will be approved for sale.
		\label{ncert/12/13/2/3/defs.tex}
		\item Two balls are drawn at random with replacement from a box containing 10 black and 8 red balls. Find the probability that
		\label{ncert/12/13/2/12}
\begin{enumerate}
\item both balls are red.
\item first ball is black and second is red.
\item one of them is black and other is red.
\end{enumerate}

\item In a hostel, 60\% of the students read Hindi newspaper, 40\% read English newspaper and 20\% read both Hindi and English newspapers. A student is selected at random.
		\label{ncert/12/13/2/15}
\begin{enumerate}
\item Find the probability that she reads neither Hindi nor English newspapers.
\item If she reads Hindi newspaper, find the probability that she reads English newspaper.
\item If she reads English newspaper, find the probability that she reads Hindi newspaper.\\
\end{enumerate}
\item The probability of obtaining an even prime number on each die, when a pair of dice is rolled is 
\begin{enumerate}
    \item $0$ 
    
    \item $\frac{1}{3}$ 
    
    \item $\frac{1}{12}$ 
    
    \item $\frac{1}{36}$ 
\end{enumerate}
\solution
		%\begin{enumerate}[label=\thesection.\arabic*,ref=\thesection.\theenumi]
	\item One card is drawn from a well-shuffled deck of 52 cards. Find the probability of getting
\begin{enumerate}
\item A king of red colour 
\item A face card 
\item A red face card
\item The jack of hearts
\item A spade
\item The queen of diamonds

\end{enumerate}
\solution
		%\input{ncert/10/15/1/14/main.tex}
	\item Five cards—the ten, jack, queen, king and ace of diamonds, are well-shuffled with their face downwards. One card is then picked up at random.
\begin{enumerate}
\item
What is the probability that the card is the queen? 
\item
If the queen is drawn and put aside, what is the probability that the second card picked up is (a) an ace? (b) a queen?\\
\end{enumerate}
\solution
		%\input{ncert/10/15/1/15/defs.tex}
	\item A bag contains $5$ red balls and some blue balls. If the probability of drawing a blue ball is double that if a red ball, determine the number of blue balls in the bag. 
		\\
\solution
		%\input{ncert/10/15/2/3/defs.tex}
	\item A card is selected from a pack of 52 cards.
 \begin{enumerate}[label=(\alph*)] 
                 \item How many points are there in the sample space?
                 \item Calculate the probability that the card is an ace of spades.
                 \item Calculate the probability that the card is (i) an ace and (ii) black card.
 \end{enumerate}
\solution
		%\input{ncert/11/16/3/4/main.tex}
\item Four cards are drawn from a well-shuffled deck of 52 cards. What is the probability of obtaining 3 diamonds and one spade.
\\
\solution
		%\input{ncert/11/16/4/2/defs.tex}
\item In a certain lottery 10,000 tickets are sold and ten equal prizes are awarded. What is the probability of not getting a prize if you buy (a) one ticket (b) two tickets (c) 10 tickets ?	
\\
\solution
		%\input{ncert/11/16/4/4/defs.tex}
		%
\item 
Out of 100 students, two sections of 40 and 60 are formed. If you and your friend are among the 100 students, what is the probability that
\begin{enumerate}
\item you both enter the same section?
\item you both enter the different sections?
\end{enumerate}
\solution
		%\input{ncert/11/16/4/5/defs.tex}
	\item 
The number lock of a suitcase has 4 wheels each labelled with ten digits i.e. from 0 to 9.The lock opens with a sequence of four digits with no repeats.What is the probability of a person getting the right sequence to open the suitcase.
\\
\solution
		%\input{ncert/11/16/4/10/defs.tex}
		%
\item 
Two cards are drawn at random and without replacement from a pack of 52 playing cards. Find the probability that both the cards are black.
\\
\solution
		%\input{ncert/12/13/2/2/defs.tex}
		\item A box of oranges is inspected by examining three randomly selected oranges drawn without replacement. If all the three oranges are good, the box is approved for sale, otherwise, it is rejected. Find the probability that a box containing 15 oranges out of which 12 are good and 3 are bad ones will be approved for sale.
		\label{ncert/12/13/2/3/defs.tex}
		\item Two balls are drawn at random with replacement from a box containing 10 black and 8 red balls. Find the probability that
		\label{ncert/12/13/2/12}
\begin{enumerate}
\item both balls are red.
\item first ball is black and second is red.
\item one of them is black and other is red.
\end{enumerate}

\item In a hostel, 60\% of the students read Hindi newspaper, 40\% read English newspaper and 20\% read both Hindi and English newspapers. A student is selected at random.
		\label{ncert/12/13/2/15}
\begin{enumerate}
\item Find the probability that she reads neither Hindi nor English newspapers.
\item If she reads Hindi newspaper, find the probability that she reads English newspaper.
\item If she reads English newspaper, find the probability that she reads Hindi newspaper.\\
\end{enumerate}
\item The probability of obtaining an even prime number on each die, when a pair of dice is rolled is 
\begin{enumerate}
    \item $0$ 
    
    \item $\frac{1}{3}$ 
    
    \item $\frac{1}{12}$ 
    
    \item $\frac{1}{36}$ 
\end{enumerate}
\solution
		%\input{ncert/12/13/2/17/defs.tex}
	\item A bag contains 4 red and 4 black balls, another bag contains 2 red and 6 black balls. One of the two bags is selected at random and a ball is drawn from the bag which is found to be red. Find the probability that the ball is drawn from the first bag.
\\
\solution
		%\input{ncert/12/13/3/2/main.tex}
  \item
  Cards with numbers 2 to 101 are placed in a box. A card is selected at random.Find the probability that the card has
\begin{enumerate}[label=(\roman*)]
	\item an even number 
	\item a square number
\end{enumerate}
\solution
%\input{exemplar/10/13/3/32/main.tex}
\item
The king, queen and jack of clubs are removed from a deck of 52 playing cards and then well shuffled. Now one card is drawn at random from the remaining cards.  Determine the probability that the card is
\begin{enumerate}[label=(\roman*)]
\item a club
\item 10 of hearts
\end{enumerate}
\solution
%\input{exemplar/10/13/3/29/main.tex}
\item A team of medical students doing their internship have to assist during surgeries
at a city hospital. The probabilities of surgeries rated as very complex, complex,
routine, simple or very simple are respectively, 0.15, 0.20, 0.31, 0.26, .08. Find
the probabilities that a particular surgery will be rated
\begin{enumerate}
	\item complex or very complex;
	\item neither very complex nor very simple;
	\item routine or complex
	\item routine or simple
\end{enumerate}
\solution
%\input{exemplar/11/16/3/8(1)/main.tex}
\item A card is selected from a pack of 52 cards.
\begin{enumerate}[label=(\alph*)]
    \item How many points are there in the sample space?
    \item Calculate the probability that the card is an ace of spades.
    \item Calculate the probability that the card is (i) an ace and (ii) black card.
\end{enumerate}
\solution
%\input{exemplar/11/16/3/4/main2.tex}
\item The probability that a non leap year selected at random will contain 53 sundays.
\\
\solution
%\input{exemplar/10/13/1/19/main.tex}
\item One of the four persons John, Rita, Aslam or Gurpreet will be promoted next
month. Consequently the sample space consists of four elementary outcomes
S = {John promoted, Rita promoted, Aslam promoted, Gurpreet promoted}
You are told that the chances of John’s promotion is same as that of Gurpreet,
Rita’s chances of promotion are twice as likely as Johns. Aslam’s chances are
four times that of John.
\begin{enumerate}
	\item Determine
	\begin{enumerate}
		\item P (John promoted)
		\item P (Rita promoted)
		\item P (Aslam promoted)
		\item P (Gurpreet promoted)
	\end{enumerate}
	\item If A = {John promoted or Gurpreet promoted}, find P (A).
\end{enumerate}
\solution
%\input{exemplar/11/16/3/10/main.tex}
\item A card is drawn from a deck of 52 cards. Find the probability of getting a king or a heart or a red card.\\
\solution
%\input{exemplar/11/16/3/15/main.tex}
\item The probability that a student will pass his examination is 0.73, the probability of
the student getting a compartment is 0.13, and the probability that the student will
either pass or get compartment is 0.96. State True or False.\\
\solution
%\input{exemplar/11/16/3/31/main.tex}
\item A card is selected from a pack of 52 cards\\
\begin{enumerate}[label=(\alph*)]
\item How many points are there in the sample space?
\item Calculate the probability that the cards is an ace of spades.
\item Calculate the probability that the card is (i) an ace (ii)black card.\\
\end{enumerate}
%\input{ncert/11/16/3/4_1/Prob_4.tex}
\item In a non-leap year, the probability of having 53 tuesdays or 53 wednesdays is\\
\solution
%\input{exemplar/11/16/3/18/main.tex}
\item There are 1000 sealed envelopes in a box, 10 of them contain a cash prize of
Rs 100 each, 100 of them contain a cash prize of Rs 50 each and 200 of them
contain a cash prize of Rs 10 each and rest do not contain any cash prize. If they
are well shuffled and an envelope is picked up out, what is the probability that it
contains no cash prize?\\
\solution
%\input{exemplar/10/13/3/34/main.tex}
\item 
A die is thrown and a card is selected at random from a deck of 52 playing cards. The probability of getting an even number on the die and a spade card.\\
\solution
%\input{exemplar/12/13/3/78/main.tex}
\item
If 4-digit numbers greater than 5,000 are randomly formed from the digits 0, 1, 3, 5, and 7, what is the probability of forming a number divisible by 5 when:
\begin{enumerate}
    \item The digits are repeated?
    \item The repetition of digits is not allowed?
\end{enumerate}
\solution
%\input{ncert/11/16/4/9/main.tex}
\item Consider the probability space $\brak{\Omega, \mathcal{G}, P}$ where $\Omega = [0,2]$ and $\mathcal{G} = \cbrak{\phi, \Omega, [0,1], (1,2]}$. Let $X$ and $Y$ be two functions on $\Omega$ defined as
\begin{align*}
    X(\omega) = 
    \begin{cases}
        1 & \text{if }\omega \in [0, 1]\\
        2 & \text{if }\omega \in (1, 2]
    \end{cases}
\end{align*}
and
\begin{align*}
    Y(\omega) = 
    \begin{cases}
        2 & \text{if }\omega \in [0, 1.5]\\
        3 & \text{if }\omega \in (1.5, 2].
    \end{cases}
\end{align*}
Then which one of the following statements is true?
\begin{enumerate}
    \item [(A)] $X$ is a random variable with respect to $\mathcal{G}$, but $Y$ is not a random variable with respect to $\mathcal{G}$.
    \item [(B)] $Y$ is a random variable with respect to $\mathcal{G}$, but $X$ is not a random variable with respect to $\mathcal{G}$.
    \item [(C)] Neither $X$ nor $Y$ is a random variable with respect to $\mathcal{G}$.
    \item [(D)] Both $X$ and $Y$ are random variables with respect to $\mathcal{G}$.
\end{enumerate} \hfill (GATE ST 2023)\\
\solution
%\input{gate/ST/2023/14/main.tex}
	\item  A die is loaded in such a way that each odd number is twice as likely to occur as
each even number. Find $P(G)$, where $G$ is the event that a number greater than
3 occurs on a single roll of the die.
\\
\solution
		%\input{exemplar/11/16/3/5/main.tex}
	\item All the jacks, queens and kings are removed from a deck of 52 playing cards. The remaining cards are well shuffled and then one card is drawn at random. Giving ace a value 1 similar value for other cards, find the probability that the card has a value 
		\begin{enumerate}
			\item 7
			\item greater than 7
			\item less than 7
		\end{enumerate}
		%\input{exemplar/10/13/3/30/main.tex}
  \item A Lot consists of 48 mobile phones of which 42 are good, 3 have only minor defects and 3 have major defects.Varnika will buy a phone if it is good but the trader will only buy a mobile if it has no major defects. One phone is selected at random from the lot. What is the probability that it is
\begin{enumerate}
	\item acceptable to Varnika?
            \item acceptable to the trader?
\end{enumerate}
\solution
	%\input{exemplar/10/13/3/40/main.tex}
 \item A student says that if you throw a die, it will show up 1 or not 1. Therefore, the probability of getting 1 and the probability of getting 'not 1' each is equal to $\frac{1}{2}$. Is this correct? Give reasons.\\
 \solution
        %\input{exemplar/10/13/2/9/main.tex}
   \item Four candidates A, B, C, D have ap-
plied for the assignment to coach a school cricket
team. If A is twice as likely to be selected as B, and
B and C are given about the same chance of being
selected, while C is twice as likely to be selected
as D, what are the probabilities that
\begin{enumerate}
\item C will be selected?
\item A will not be selected?
\end{enumerate}
	%\input{exemplar/11/16/3/9/main.tex}
 \item A bag contain 24 balls of which $x$ balls are red, $2x$ are white and $3x$ are blue. A ball is selected at random, What is the probability that it is
\begin{enumerate}[label=\alph*)]
\item not red ?
\item white ?
\end{enumerate}
%\input{exemplar/10/13/3/41/main.tex}
If the letters of the word ASSASSINATION are arranged at random. Find the Probability that
\begin{enumerate}[label=(\alph*)]
\item Four $S's$ come consecutively in the word
\item Two  $I's$ and two $N's$ come together
\item All $A's$ are not coming together
\item No two $A's$ are coming together
\end{enumerate}
%\input{exemplar/11/16/3/14/main.tex}
	\item One urn contains two black balls (labelled B1 and B2) and one white ball. A
	second urn contains one black ball and two white balls (labelled W1 and W2).
	Suppose the following experiment is performed. One of the two urns is chosen
	at random. Next a ball is randomly chosen from the urn. Then a second ball is
	chosen at random from the same urn without replacing the first ball.
	
	\begin{enumerate}
	\item What is the probability that two black balls are chosen?
	
	\item What is the probability that two balls of opposite colour are chosen?
	\end{enumerate}
	\solution
	%\input{exemplar/11/16/3/12/main1.tex}
\end{enumerate}

	\item A bag contains 4 red and 4 black balls, another bag contains 2 red and 6 black balls. One of the two bags is selected at random and a ball is drawn from the bag which is found to be red. Find the probability that the ball is drawn from the first bag.
\\
\solution
		%\begin{table}[H]
	\centering
\begin{tabular}{|c|c|c|}
\hline
Random variable &Value &Definition\\ \hline
\multirow{3}{*}{X} &0 &Slips of Rs 1\\
&1 &Slips of Rs 5\\
&2 &Slips of Rs 13\\ \hline
\multirow{2}{*}{Y} &0 &Box A\\
&1 &Box B\\\hline
\end{tabular}
\caption{}
\label{tab:Distribution}
\end{table}
See \tabref{tab:Distribution}.
\begin{align}
p_{Y}\brak{k}= \begin{cases} 
      \frac{1}{3} & {k=0} \\
      \frac{2}{3 }& {k=1} 
   \end{cases}
   \\
p_{Y|X}\brak{0|0} = \frac{19}{25}\, 
p_{Y|X}\brak{0|1} = \frac{6}{25}\,
p_{Y|X}\brak{1|0} = \frac{45}{50}\,
p_{Y|X}\brak{1|2} = \frac{5}{50}
\end{align}
The desired probability is the probability that a slip drawn at random is marked other than Rs 1,
\begin{align}
&=1-p_X\brak{0}\\
&= p_X(1) + p_X(2)
\end{align}
Using Bayes theorem,
\begin{align}
&= p_Y\brak{0} \times \pr{Y=0 | X=1} + p_Y\brak{1} \times \pr{Y=1|X=2}\\
&=\frac{1}{3} \times \frac{6}{25} + \frac{2}{3} \times \frac{5}{50}\\
&=\frac{11}{75}
\end{align}

\newpage

%\tableofcontents

\bigskip

\renewcommand{\thefigure}{\theenumi}
\renewcommand{\thetable}{\theenumi}
%\renewcommand{\theequation}{\theenumi}

%\begin{abstract}
%%\boldmath
%In this letter, an algorithm for evaluating the exact analytical bit error rate  (BER)  for the piecewise linear (PL) combiner for  multiple relays is presented. Previous results were available only for upto three relays. The algorithm is unique in the sense that  the actual mathematical expressions, that are prohibitively large, need not be explicitly obtained. The diversity gain due to multiple relays is shown through plots of the analytical BER, well supported by simulations. 
%
%\end{abstract}
% IEEEtran.cls defaults to using nonbold math in the Abstract.
% This preserves the distinction between vectors and scalars. However,
% if the journal you are submitting to favors bold math in the abstract,
% then you can use LaTeX's standard command \boldmath at the very start
% of the abstract to achieve this. Many IEEE journals frown on math
% in the abstract anyway.

% Note that keywords are not normally used for peerreview papers.
%\begin{IEEEkeywords}
%Cooperative diversity, decode and forward, piecewise linear
%\end{IEEEkeywords}



% For peer review papers, you can put extra information on the cover
% page as needed:
% \ifCLASSOPTIONpeerreview
% \begin{center} \bfseries EDICS Category: 3-BBND \end{center}
% \fi
%
% For peerreview papers, this IEEEtran command inserts a page break and
% creates the second title. It will be ignored for other modes.
%\IEEEpeerreviewmaketitle




  \item
  Cards with numbers 2 to 101 are placed in a box. A card is selected at random.Find the probability that the card has
\begin{enumerate}[label=(\roman*)]
	\item an even number 
	\item a square number
\end{enumerate}
\solution
%\begin{table}[H]
	\centering
\begin{tabular}{|c|c|c|}
\hline
Random variable &Value &Definition\\ \hline
\multirow{3}{*}{X} &0 &Slips of Rs 1\\
&1 &Slips of Rs 5\\
&2 &Slips of Rs 13\\ \hline
\multirow{2}{*}{Y} &0 &Box A\\
&1 &Box B\\\hline
\end{tabular}
\caption{}
\label{tab:Distribution}
\end{table}
See \tabref{tab:Distribution}.
\begin{align}
p_{Y}\brak{k}= \begin{cases} 
      \frac{1}{3} & {k=0} \\
      \frac{2}{3 }& {k=1} 
   \end{cases}
   \\
p_{Y|X}\brak{0|0} = \frac{19}{25}\, 
p_{Y|X}\brak{0|1} = \frac{6}{25}\,
p_{Y|X}\brak{1|0} = \frac{45}{50}\,
p_{Y|X}\brak{1|2} = \frac{5}{50}
\end{align}
The desired probability is the probability that a slip drawn at random is marked other than Rs 1,
\begin{align}
&=1-p_X\brak{0}\\
&= p_X(1) + p_X(2)
\end{align}
Using Bayes theorem,
\begin{align}
&= p_Y\brak{0} \times \pr{Y=0 | X=1} + p_Y\brak{1} \times \pr{Y=1|X=2}\\
&=\frac{1}{3} \times \frac{6}{25} + \frac{2}{3} \times \frac{5}{50}\\
&=\frac{11}{75}
\end{align}

\newpage

%\tableofcontents

\bigskip

\renewcommand{\thefigure}{\theenumi}
\renewcommand{\thetable}{\theenumi}
%\renewcommand{\theequation}{\theenumi}

%\begin{abstract}
%%\boldmath
%In this letter, an algorithm for evaluating the exact analytical bit error rate  (BER)  for the piecewise linear (PL) combiner for  multiple relays is presented. Previous results were available only for upto three relays. The algorithm is unique in the sense that  the actual mathematical expressions, that are prohibitively large, need not be explicitly obtained. The diversity gain due to multiple relays is shown through plots of the analytical BER, well supported by simulations. 
%
%\end{abstract}
% IEEEtran.cls defaults to using nonbold math in the Abstract.
% This preserves the distinction between vectors and scalars. However,
% if the journal you are submitting to favors bold math in the abstract,
% then you can use LaTeX's standard command \boldmath at the very start
% of the abstract to achieve this. Many IEEE journals frown on math
% in the abstract anyway.

% Note that keywords are not normally used for peerreview papers.
%\begin{IEEEkeywords}
%Cooperative diversity, decode and forward, piecewise linear
%\end{IEEEkeywords}



% For peer review papers, you can put extra information on the cover
% page as needed:
% \ifCLASSOPTIONpeerreview
% \begin{center} \bfseries EDICS Category: 3-BBND \end{center}
% \fi
%
% For peerreview papers, this IEEEtran command inserts a page break and
% creates the second title. It will be ignored for other modes.
%\IEEEpeerreviewmaketitle




\item
The king, queen and jack of clubs are removed from a deck of 52 playing cards and then well shuffled. Now one card is drawn at random from the remaining cards.  Determine the probability that the card is
\begin{enumerate}[label=(\roman*)]
\item a club
\item 10 of hearts
\end{enumerate}
\solution
%\begin{table}[H]
	\centering
\begin{tabular}{|c|c|c|}
\hline
Random variable &Value &Definition\\ \hline
\multirow{3}{*}{X} &0 &Slips of Rs 1\\
&1 &Slips of Rs 5\\
&2 &Slips of Rs 13\\ \hline
\multirow{2}{*}{Y} &0 &Box A\\
&1 &Box B\\\hline
\end{tabular}
\caption{}
\label{tab:Distribution}
\end{table}
See \tabref{tab:Distribution}.
\begin{align}
p_{Y}\brak{k}= \begin{cases} 
      \frac{1}{3} & {k=0} \\
      \frac{2}{3 }& {k=1} 
   \end{cases}
   \\
p_{Y|X}\brak{0|0} = \frac{19}{25}\, 
p_{Y|X}\brak{0|1} = \frac{6}{25}\,
p_{Y|X}\brak{1|0} = \frac{45}{50}\,
p_{Y|X}\brak{1|2} = \frac{5}{50}
\end{align}
The desired probability is the probability that a slip drawn at random is marked other than Rs 1,
\begin{align}
&=1-p_X\brak{0}\\
&= p_X(1) + p_X(2)
\end{align}
Using Bayes theorem,
\begin{align}
&= p_Y\brak{0} \times \pr{Y=0 | X=1} + p_Y\brak{1} \times \pr{Y=1|X=2}\\
&=\frac{1}{3} \times \frac{6}{25} + \frac{2}{3} \times \frac{5}{50}\\
&=\frac{11}{75}
\end{align}

\newpage

%\tableofcontents

\bigskip

\renewcommand{\thefigure}{\theenumi}
\renewcommand{\thetable}{\theenumi}
%\renewcommand{\theequation}{\theenumi}

%\begin{abstract}
%%\boldmath
%In this letter, an algorithm for evaluating the exact analytical bit error rate  (BER)  for the piecewise linear (PL) combiner for  multiple relays is presented. Previous results were available only for upto three relays. The algorithm is unique in the sense that  the actual mathematical expressions, that are prohibitively large, need not be explicitly obtained. The diversity gain due to multiple relays is shown through plots of the analytical BER, well supported by simulations. 
%
%\end{abstract}
% IEEEtran.cls defaults to using nonbold math in the Abstract.
% This preserves the distinction between vectors and scalars. However,
% if the journal you are submitting to favors bold math in the abstract,
% then you can use LaTeX's standard command \boldmath at the very start
% of the abstract to achieve this. Many IEEE journals frown on math
% in the abstract anyway.

% Note that keywords are not normally used for peerreview papers.
%\begin{IEEEkeywords}
%Cooperative diversity, decode and forward, piecewise linear
%\end{IEEEkeywords}



% For peer review papers, you can put extra information on the cover
% page as needed:
% \ifCLASSOPTIONpeerreview
% \begin{center} \bfseries EDICS Category: 3-BBND \end{center}
% \fi
%
% For peerreview papers, this IEEEtran command inserts a page break and
% creates the second title. It will be ignored for other modes.
%\IEEEpeerreviewmaketitle




\item A team of medical students doing their internship have to assist during surgeries
at a city hospital. The probabilities of surgeries rated as very complex, complex,
routine, simple or very simple are respectively, 0.15, 0.20, 0.31, 0.26, .08. Find
the probabilities that a particular surgery will be rated
\begin{enumerate}
	\item complex or very complex;
	\item neither very complex nor very simple;
	\item routine or complex
	\item routine or simple
\end{enumerate}
\solution
%\begin{table}[H]
	\centering
\begin{tabular}{|c|c|c|}
\hline
Random variable &Value &Definition\\ \hline
\multirow{3}{*}{X} &0 &Slips of Rs 1\\
&1 &Slips of Rs 5\\
&2 &Slips of Rs 13\\ \hline
\multirow{2}{*}{Y} &0 &Box A\\
&1 &Box B\\\hline
\end{tabular}
\caption{}
\label{tab:Distribution}
\end{table}
See \tabref{tab:Distribution}.
\begin{align}
p_{Y}\brak{k}= \begin{cases} 
      \frac{1}{3} & {k=0} \\
      \frac{2}{3 }& {k=1} 
   \end{cases}
   \\
p_{Y|X}\brak{0|0} = \frac{19}{25}\, 
p_{Y|X}\brak{0|1} = \frac{6}{25}\,
p_{Y|X}\brak{1|0} = \frac{45}{50}\,
p_{Y|X}\brak{1|2} = \frac{5}{50}
\end{align}
The desired probability is the probability that a slip drawn at random is marked other than Rs 1,
\begin{align}
&=1-p_X\brak{0}\\
&= p_X(1) + p_X(2)
\end{align}
Using Bayes theorem,
\begin{align}
&= p_Y\brak{0} \times \pr{Y=0 | X=1} + p_Y\brak{1} \times \pr{Y=1|X=2}\\
&=\frac{1}{3} \times \frac{6}{25} + \frac{2}{3} \times \frac{5}{50}\\
&=\frac{11}{75}
\end{align}

\newpage

%\tableofcontents

\bigskip

\renewcommand{\thefigure}{\theenumi}
\renewcommand{\thetable}{\theenumi}
%\renewcommand{\theequation}{\theenumi}

%\begin{abstract}
%%\boldmath
%In this letter, an algorithm for evaluating the exact analytical bit error rate  (BER)  for the piecewise linear (PL) combiner for  multiple relays is presented. Previous results were available only for upto three relays. The algorithm is unique in the sense that  the actual mathematical expressions, that are prohibitively large, need not be explicitly obtained. The diversity gain due to multiple relays is shown through plots of the analytical BER, well supported by simulations. 
%
%\end{abstract}
% IEEEtran.cls defaults to using nonbold math in the Abstract.
% This preserves the distinction between vectors and scalars. However,
% if the journal you are submitting to favors bold math in the abstract,
% then you can use LaTeX's standard command \boldmath at the very start
% of the abstract to achieve this. Many IEEE journals frown on math
% in the abstract anyway.

% Note that keywords are not normally used for peerreview papers.
%\begin{IEEEkeywords}
%Cooperative diversity, decode and forward, piecewise linear
%\end{IEEEkeywords}



% For peer review papers, you can put extra information on the cover
% page as needed:
% \ifCLASSOPTIONpeerreview
% \begin{center} \bfseries EDICS Category: 3-BBND \end{center}
% \fi
%
% For peerreview papers, this IEEEtran command inserts a page break and
% creates the second title. It will be ignored for other modes.
%\IEEEpeerreviewmaketitle




\item A card is selected from a pack of 52 cards.
\begin{enumerate}[label=(\alph*)]
    \item How many points are there in the sample space?
    \item Calculate the probability that the card is an ace of spades.
    \item Calculate the probability that the card is (i) an ace and (ii) black card.
\end{enumerate}
\solution
%Let $X$ be an bernoulli rv defined as in \tabref{tab:exemplar/11/16/3/26}.  Then, 
\begin{equation}
    p =
        \frac{4}{11} 
\end{equation}
\begin{table}[H]
	\centering
	\input{exemplar/11/16/3/26/tables/Table2.tex}
	\caption{}
        \label{tab:exemplar/11/16/3/26}
\end{table}

\item The probability that a non leap year selected at random will contain 53 sundays.
\\
\solution
%\begin{table}[H]
	\centering
\begin{tabular}{|c|c|c|}
\hline
Random variable &Value &Definition\\ \hline
\multirow{3}{*}{X} &0 &Slips of Rs 1\\
&1 &Slips of Rs 5\\
&2 &Slips of Rs 13\\ \hline
\multirow{2}{*}{Y} &0 &Box A\\
&1 &Box B\\\hline
\end{tabular}
\caption{}
\label{tab:Distribution}
\end{table}
See \tabref{tab:Distribution}.
\begin{align}
p_{Y}\brak{k}= \begin{cases} 
      \frac{1}{3} & {k=0} \\
      \frac{2}{3 }& {k=1} 
   \end{cases}
   \\
p_{Y|X}\brak{0|0} = \frac{19}{25}\, 
p_{Y|X}\brak{0|1} = \frac{6}{25}\,
p_{Y|X}\brak{1|0} = \frac{45}{50}\,
p_{Y|X}\brak{1|2} = \frac{5}{50}
\end{align}
The desired probability is the probability that a slip drawn at random is marked other than Rs 1,
\begin{align}
&=1-p_X\brak{0}\\
&= p_X(1) + p_X(2)
\end{align}
Using Bayes theorem,
\begin{align}
&= p_Y\brak{0} \times \pr{Y=0 | X=1} + p_Y\brak{1} \times \pr{Y=1|X=2}\\
&=\frac{1}{3} \times \frac{6}{25} + \frac{2}{3} \times \frac{5}{50}\\
&=\frac{11}{75}
\end{align}

\newpage

%\tableofcontents

\bigskip

\renewcommand{\thefigure}{\theenumi}
\renewcommand{\thetable}{\theenumi}
%\renewcommand{\theequation}{\theenumi}

%\begin{abstract}
%%\boldmath
%In this letter, an algorithm for evaluating the exact analytical bit error rate  (BER)  for the piecewise linear (PL) combiner for  multiple relays is presented. Previous results were available only for upto three relays. The algorithm is unique in the sense that  the actual mathematical expressions, that are prohibitively large, need not be explicitly obtained. The diversity gain due to multiple relays is shown through plots of the analytical BER, well supported by simulations. 
%
%\end{abstract}
% IEEEtran.cls defaults to using nonbold math in the Abstract.
% This preserves the distinction between vectors and scalars. However,
% if the journal you are submitting to favors bold math in the abstract,
% then you can use LaTeX's standard command \boldmath at the very start
% of the abstract to achieve this. Many IEEE journals frown on math
% in the abstract anyway.

% Note that keywords are not normally used for peerreview papers.
%\begin{IEEEkeywords}
%Cooperative diversity, decode and forward, piecewise linear
%\end{IEEEkeywords}



% For peer review papers, you can put extra information on the cover
% page as needed:
% \ifCLASSOPTIONpeerreview
% \begin{center} \bfseries EDICS Category: 3-BBND \end{center}
% \fi
%
% For peerreview papers, this IEEEtran command inserts a page break and
% creates the second title. It will be ignored for other modes.
%\IEEEpeerreviewmaketitle




\item One of the four persons John, Rita, Aslam or Gurpreet will be promoted next
month. Consequently the sample space consists of four elementary outcomes
S = {John promoted, Rita promoted, Aslam promoted, Gurpreet promoted}
You are told that the chances of John’s promotion is same as that of Gurpreet,
Rita’s chances of promotion are twice as likely as Johns. Aslam’s chances are
four times that of John.
\begin{enumerate}
	\item Determine
	\begin{enumerate}
		\item P (John promoted)
		\item P (Rita promoted)
		\item P (Aslam promoted)
		\item P (Gurpreet promoted)
	\end{enumerate}
	\item If A = {John promoted or Gurpreet promoted}, find P (A).
\end{enumerate}
\solution
%\begin{table}[H]
	\centering
\begin{tabular}{|c|c|c|}
\hline
Random variable &Value &Definition\\ \hline
\multirow{3}{*}{X} &0 &Slips of Rs 1\\
&1 &Slips of Rs 5\\
&2 &Slips of Rs 13\\ \hline
\multirow{2}{*}{Y} &0 &Box A\\
&1 &Box B\\\hline
\end{tabular}
\caption{}
\label{tab:Distribution}
\end{table}
See \tabref{tab:Distribution}.
\begin{align}
p_{Y}\brak{k}= \begin{cases} 
      \frac{1}{3} & {k=0} \\
      \frac{2}{3 }& {k=1} 
   \end{cases}
   \\
p_{Y|X}\brak{0|0} = \frac{19}{25}\, 
p_{Y|X}\brak{0|1} = \frac{6}{25}\,
p_{Y|X}\brak{1|0} = \frac{45}{50}\,
p_{Y|X}\brak{1|2} = \frac{5}{50}
\end{align}
The desired probability is the probability that a slip drawn at random is marked other than Rs 1,
\begin{align}
&=1-p_X\brak{0}\\
&= p_X(1) + p_X(2)
\end{align}
Using Bayes theorem,
\begin{align}
&= p_Y\brak{0} \times \pr{Y=0 | X=1} + p_Y\brak{1} \times \pr{Y=1|X=2}\\
&=\frac{1}{3} \times \frac{6}{25} + \frac{2}{3} \times \frac{5}{50}\\
&=\frac{11}{75}
\end{align}

\newpage

%\tableofcontents

\bigskip

\renewcommand{\thefigure}{\theenumi}
\renewcommand{\thetable}{\theenumi}
%\renewcommand{\theequation}{\theenumi}

%\begin{abstract}
%%\boldmath
%In this letter, an algorithm for evaluating the exact analytical bit error rate  (BER)  for the piecewise linear (PL) combiner for  multiple relays is presented. Previous results were available only for upto three relays. The algorithm is unique in the sense that  the actual mathematical expressions, that are prohibitively large, need not be explicitly obtained. The diversity gain due to multiple relays is shown through plots of the analytical BER, well supported by simulations. 
%
%\end{abstract}
% IEEEtran.cls defaults to using nonbold math in the Abstract.
% This preserves the distinction between vectors and scalars. However,
% if the journal you are submitting to favors bold math in the abstract,
% then you can use LaTeX's standard command \boldmath at the very start
% of the abstract to achieve this. Many IEEE journals frown on math
% in the abstract anyway.

% Note that keywords are not normally used for peerreview papers.
%\begin{IEEEkeywords}
%Cooperative diversity, decode and forward, piecewise linear
%\end{IEEEkeywords}



% For peer review papers, you can put extra information on the cover
% page as needed:
% \ifCLASSOPTIONpeerreview
% \begin{center} \bfseries EDICS Category: 3-BBND \end{center}
% \fi
%
% For peerreview papers, this IEEEtran command inserts a page break and
% creates the second title. It will be ignored for other modes.
%\IEEEpeerreviewmaketitle




\item A card is drawn from a deck of 52 cards. Find the probability of getting a king or a heart or a red card.\\
\solution
%\begin{table}[H]
	\centering
\begin{tabular}{|c|c|c|}
\hline
Random variable &Value &Definition\\ \hline
\multirow{3}{*}{X} &0 &Slips of Rs 1\\
&1 &Slips of Rs 5\\
&2 &Slips of Rs 13\\ \hline
\multirow{2}{*}{Y} &0 &Box A\\
&1 &Box B\\\hline
\end{tabular}
\caption{}
\label{tab:Distribution}
\end{table}
See \tabref{tab:Distribution}.
\begin{align}
p_{Y}\brak{k}= \begin{cases} 
      \frac{1}{3} & {k=0} \\
      \frac{2}{3 }& {k=1} 
   \end{cases}
   \\
p_{Y|X}\brak{0|0} = \frac{19}{25}\, 
p_{Y|X}\brak{0|1} = \frac{6}{25}\,
p_{Y|X}\brak{1|0} = \frac{45}{50}\,
p_{Y|X}\brak{1|2} = \frac{5}{50}
\end{align}
The desired probability is the probability that a slip drawn at random is marked other than Rs 1,
\begin{align}
&=1-p_X\brak{0}\\
&= p_X(1) + p_X(2)
\end{align}
Using Bayes theorem,
\begin{align}
&= p_Y\brak{0} \times \pr{Y=0 | X=1} + p_Y\brak{1} \times \pr{Y=1|X=2}\\
&=\frac{1}{3} \times \frac{6}{25} + \frac{2}{3} \times \frac{5}{50}\\
&=\frac{11}{75}
\end{align}

\newpage

%\tableofcontents

\bigskip

\renewcommand{\thefigure}{\theenumi}
\renewcommand{\thetable}{\theenumi}
%\renewcommand{\theequation}{\theenumi}

%\begin{abstract}
%%\boldmath
%In this letter, an algorithm for evaluating the exact analytical bit error rate  (BER)  for the piecewise linear (PL) combiner for  multiple relays is presented. Previous results were available only for upto three relays. The algorithm is unique in the sense that  the actual mathematical expressions, that are prohibitively large, need not be explicitly obtained. The diversity gain due to multiple relays is shown through plots of the analytical BER, well supported by simulations. 
%
%\end{abstract}
% IEEEtran.cls defaults to using nonbold math in the Abstract.
% This preserves the distinction between vectors and scalars. However,
% if the journal you are submitting to favors bold math in the abstract,
% then you can use LaTeX's standard command \boldmath at the very start
% of the abstract to achieve this. Many IEEE journals frown on math
% in the abstract anyway.

% Note that keywords are not normally used for peerreview papers.
%\begin{IEEEkeywords}
%Cooperative diversity, decode and forward, piecewise linear
%\end{IEEEkeywords}



% For peer review papers, you can put extra information on the cover
% page as needed:
% \ifCLASSOPTIONpeerreview
% \begin{center} \bfseries EDICS Category: 3-BBND \end{center}
% \fi
%
% For peerreview papers, this IEEEtran command inserts a page break and
% creates the second title. It will be ignored for other modes.
%\IEEEpeerreviewmaketitle




\item The probability that a student will pass his examination is 0.73, the probability of
the student getting a compartment is 0.13, and the probability that the student will
either pass or get compartment is 0.96. State True or False.\\
\solution
%\begin{table}[H]
	\centering
\begin{tabular}{|c|c|c|}
\hline
Random variable &Value &Definition\\ \hline
\multirow{3}{*}{X} &0 &Slips of Rs 1\\
&1 &Slips of Rs 5\\
&2 &Slips of Rs 13\\ \hline
\multirow{2}{*}{Y} &0 &Box A\\
&1 &Box B\\\hline
\end{tabular}
\caption{}
\label{tab:Distribution}
\end{table}
See \tabref{tab:Distribution}.
\begin{align}
p_{Y}\brak{k}= \begin{cases} 
      \frac{1}{3} & {k=0} \\
      \frac{2}{3 }& {k=1} 
   \end{cases}
   \\
p_{Y|X}\brak{0|0} = \frac{19}{25}\, 
p_{Y|X}\brak{0|1} = \frac{6}{25}\,
p_{Y|X}\brak{1|0} = \frac{45}{50}\,
p_{Y|X}\brak{1|2} = \frac{5}{50}
\end{align}
The desired probability is the probability that a slip drawn at random is marked other than Rs 1,
\begin{align}
&=1-p_X\brak{0}\\
&= p_X(1) + p_X(2)
\end{align}
Using Bayes theorem,
\begin{align}
&= p_Y\brak{0} \times \pr{Y=0 | X=1} + p_Y\brak{1} \times \pr{Y=1|X=2}\\
&=\frac{1}{3} \times \frac{6}{25} + \frac{2}{3} \times \frac{5}{50}\\
&=\frac{11}{75}
\end{align}

\newpage

%\tableofcontents

\bigskip

\renewcommand{\thefigure}{\theenumi}
\renewcommand{\thetable}{\theenumi}
%\renewcommand{\theequation}{\theenumi}

%\begin{abstract}
%%\boldmath
%In this letter, an algorithm for evaluating the exact analytical bit error rate  (BER)  for the piecewise linear (PL) combiner for  multiple relays is presented. Previous results were available only for upto three relays. The algorithm is unique in the sense that  the actual mathematical expressions, that are prohibitively large, need not be explicitly obtained. The diversity gain due to multiple relays is shown through plots of the analytical BER, well supported by simulations. 
%
%\end{abstract}
% IEEEtran.cls defaults to using nonbold math in the Abstract.
% This preserves the distinction between vectors and scalars. However,
% if the journal you are submitting to favors bold math in the abstract,
% then you can use LaTeX's standard command \boldmath at the very start
% of the abstract to achieve this. Many IEEE journals frown on math
% in the abstract anyway.

% Note that keywords are not normally used for peerreview papers.
%\begin{IEEEkeywords}
%Cooperative diversity, decode and forward, piecewise linear
%\end{IEEEkeywords}



% For peer review papers, you can put extra information on the cover
% page as needed:
% \ifCLASSOPTIONpeerreview
% \begin{center} \bfseries EDICS Category: 3-BBND \end{center}
% \fi
%
% For peerreview papers, this IEEEtran command inserts a page break and
% creates the second title. It will be ignored for other modes.
%\IEEEpeerreviewmaketitle




\item A card is selected from a pack of 52 cards\\
\begin{enumerate}[label=(\alph*)]
\item How many points are there in the sample space?
\item Calculate the probability that the cards is an ace of spades.
\item Calculate the probability that the card is (i) an ace (ii)black card.\\
\end{enumerate}
%\input{ncert/11/16/3/4_1/Prob_4.tex}
\item In a non-leap year, the probability of having 53 tuesdays or 53 wednesdays is\\
\solution
%A non-leap year has a total of 365 days, and a week has 7 days.\\
So it can be expressed as 
\begin{align}
365\text{days} &=52\times 7+1 \text{day}
\end{align}
$\implies$ 52 tuesdays or wednesdays\\
Random variable X denotes the days of a week
\begin{align}
p_X\brak{k}&=\frac{1}{7}; \quad \brak{1<k<7}
\end{align}
So the probability of extra day being tuesday or wednesday is
\begin{align}
p_X\brak{3}+p_X\brak{4}&=\frac{1}{7}+\frac{1}{7}=\frac{2}{7}
\end{align}



\item There are 1000 sealed envelopes in a box, 10 of them contain a cash prize of
Rs 100 each, 100 of them contain a cash prize of Rs 50 each and 200 of them
contain a cash prize of Rs 10 each and rest do not contain any cash prize. If they
are well shuffled and an envelope is picked up out, what is the probability that it
contains no cash prize?\\
\solution
%\begin{table}[H]
	\centering
\begin{tabular}{|c|c|c|}
\hline
Random variable &Value &Definition\\ \hline
\multirow{3}{*}{X} &0 &Slips of Rs 1\\
&1 &Slips of Rs 5\\
&2 &Slips of Rs 13\\ \hline
\multirow{2}{*}{Y} &0 &Box A\\
&1 &Box B\\\hline
\end{tabular}
\caption{}
\label{tab:Distribution}
\end{table}
See \tabref{tab:Distribution}.
\begin{align}
p_{Y}\brak{k}= \begin{cases} 
      \frac{1}{3} & {k=0} \\
      \frac{2}{3 }& {k=1} 
   \end{cases}
   \\
p_{Y|X}\brak{0|0} = \frac{19}{25}\, 
p_{Y|X}\brak{0|1} = \frac{6}{25}\,
p_{Y|X}\brak{1|0} = \frac{45}{50}\,
p_{Y|X}\brak{1|2} = \frac{5}{50}
\end{align}
The desired probability is the probability that a slip drawn at random is marked other than Rs 1,
\begin{align}
&=1-p_X\brak{0}\\
&= p_X(1) + p_X(2)
\end{align}
Using Bayes theorem,
\begin{align}
&= p_Y\brak{0} \times \pr{Y=0 | X=1} + p_Y\brak{1} \times \pr{Y=1|X=2}\\
&=\frac{1}{3} \times \frac{6}{25} + \frac{2}{3} \times \frac{5}{50}\\
&=\frac{11}{75}
\end{align}

\newpage

%\tableofcontents

\bigskip

\renewcommand{\thefigure}{\theenumi}
\renewcommand{\thetable}{\theenumi}
%\renewcommand{\theequation}{\theenumi}

%\begin{abstract}
%%\boldmath
%In this letter, an algorithm for evaluating the exact analytical bit error rate  (BER)  for the piecewise linear (PL) combiner for  multiple relays is presented. Previous results were available only for upto three relays. The algorithm is unique in the sense that  the actual mathematical expressions, that are prohibitively large, need not be explicitly obtained. The diversity gain due to multiple relays is shown through plots of the analytical BER, well supported by simulations. 
%
%\end{abstract}
% IEEEtran.cls defaults to using nonbold math in the Abstract.
% This preserves the distinction between vectors and scalars. However,
% if the journal you are submitting to favors bold math in the abstract,
% then you can use LaTeX's standard command \boldmath at the very start
% of the abstract to achieve this. Many IEEE journals frown on math
% in the abstract anyway.

% Note that keywords are not normally used for peerreview papers.
%\begin{IEEEkeywords}
%Cooperative diversity, decode and forward, piecewise linear
%\end{IEEEkeywords}



% For peer review papers, you can put extra information on the cover
% page as needed:
% \ifCLASSOPTIONpeerreview
% \begin{center} \bfseries EDICS Category: 3-BBND \end{center}
% \fi
%
% For peerreview papers, this IEEEtran command inserts a page break and
% creates the second title. It will be ignored for other modes.
%\IEEEpeerreviewmaketitle




\item 
A die is thrown and a card is selected at random from a deck of 52 playing cards. The probability of getting an even number on the die and a spade card.\\
\solution
%\begin{table}[H]
	\centering
\begin{tabular}{|c|c|c|}
\hline
Random variable &Value &Definition\\ \hline
\multirow{3}{*}{X} &0 &Slips of Rs 1\\
&1 &Slips of Rs 5\\
&2 &Slips of Rs 13\\ \hline
\multirow{2}{*}{Y} &0 &Box A\\
&1 &Box B\\\hline
\end{tabular}
\caption{}
\label{tab:Distribution}
\end{table}
See \tabref{tab:Distribution}.
\begin{align}
p_{Y}\brak{k}= \begin{cases} 
      \frac{1}{3} & {k=0} \\
      \frac{2}{3 }& {k=1} 
   \end{cases}
   \\
p_{Y|X}\brak{0|0} = \frac{19}{25}\, 
p_{Y|X}\brak{0|1} = \frac{6}{25}\,
p_{Y|X}\brak{1|0} = \frac{45}{50}\,
p_{Y|X}\brak{1|2} = \frac{5}{50}
\end{align}
The desired probability is the probability that a slip drawn at random is marked other than Rs 1,
\begin{align}
&=1-p_X\brak{0}\\
&= p_X(1) + p_X(2)
\end{align}
Using Bayes theorem,
\begin{align}
&= p_Y\brak{0} \times \pr{Y=0 | X=1} + p_Y\brak{1} \times \pr{Y=1|X=2}\\
&=\frac{1}{3} \times \frac{6}{25} + \frac{2}{3} \times \frac{5}{50}\\
&=\frac{11}{75}
\end{align}

\newpage

%\tableofcontents

\bigskip

\renewcommand{\thefigure}{\theenumi}
\renewcommand{\thetable}{\theenumi}
%\renewcommand{\theequation}{\theenumi}

%\begin{abstract}
%%\boldmath
%In this letter, an algorithm for evaluating the exact analytical bit error rate  (BER)  for the piecewise linear (PL) combiner for  multiple relays is presented. Previous results were available only for upto three relays. The algorithm is unique in the sense that  the actual mathematical expressions, that are prohibitively large, need not be explicitly obtained. The diversity gain due to multiple relays is shown through plots of the analytical BER, well supported by simulations. 
%
%\end{abstract}
% IEEEtran.cls defaults to using nonbold math in the Abstract.
% This preserves the distinction between vectors and scalars. However,
% if the journal you are submitting to favors bold math in the abstract,
% then you can use LaTeX's standard command \boldmath at the very start
% of the abstract to achieve this. Many IEEE journals frown on math
% in the abstract anyway.

% Note that keywords are not normally used for peerreview papers.
%\begin{IEEEkeywords}
%Cooperative diversity, decode and forward, piecewise linear
%\end{IEEEkeywords}



% For peer review papers, you can put extra information on the cover
% page as needed:
% \ifCLASSOPTIONpeerreview
% \begin{center} \bfseries EDICS Category: 3-BBND \end{center}
% \fi
%
% For peerreview papers, this IEEEtran command inserts a page break and
% creates the second title. It will be ignored for other modes.
%\IEEEpeerreviewmaketitle




\item
If 4-digit numbers greater than 5,000 are randomly formed from the digits 0, 1, 3, 5, and 7, what is the probability of forming a number divisible by 5 when:
\begin{enumerate}
    \item The digits are repeated?
    \item The repetition of digits is not allowed?
\end{enumerate}
\solution
%\begin{table}[H]
	\centering
\begin{tabular}{|c|c|c|}
\hline
Random variable &Value &Definition\\ \hline
\multirow{3}{*}{X} &0 &Slips of Rs 1\\
&1 &Slips of Rs 5\\
&2 &Slips of Rs 13\\ \hline
\multirow{2}{*}{Y} &0 &Box A\\
&1 &Box B\\\hline
\end{tabular}
\caption{}
\label{tab:Distribution}
\end{table}
See \tabref{tab:Distribution}.
\begin{align}
p_{Y}\brak{k}= \begin{cases} 
      \frac{1}{3} & {k=0} \\
      \frac{2}{3 }& {k=1} 
   \end{cases}
   \\
p_{Y|X}\brak{0|0} = \frac{19}{25}\, 
p_{Y|X}\brak{0|1} = \frac{6}{25}\,
p_{Y|X}\brak{1|0} = \frac{45}{50}\,
p_{Y|X}\brak{1|2} = \frac{5}{50}
\end{align}
The desired probability is the probability that a slip drawn at random is marked other than Rs 1,
\begin{align}
&=1-p_X\brak{0}\\
&= p_X(1) + p_X(2)
\end{align}
Using Bayes theorem,
\begin{align}
&= p_Y\brak{0} \times \pr{Y=0 | X=1} + p_Y\brak{1} \times \pr{Y=1|X=2}\\
&=\frac{1}{3} \times \frac{6}{25} + \frac{2}{3} \times \frac{5}{50}\\
&=\frac{11}{75}
\end{align}

\newpage

%\tableofcontents

\bigskip

\renewcommand{\thefigure}{\theenumi}
\renewcommand{\thetable}{\theenumi}
%\renewcommand{\theequation}{\theenumi}

%\begin{abstract}
%%\boldmath
%In this letter, an algorithm for evaluating the exact analytical bit error rate  (BER)  for the piecewise linear (PL) combiner for  multiple relays is presented. Previous results were available only for upto three relays. The algorithm is unique in the sense that  the actual mathematical expressions, that are prohibitively large, need not be explicitly obtained. The diversity gain due to multiple relays is shown through plots of the analytical BER, well supported by simulations. 
%
%\end{abstract}
% IEEEtran.cls defaults to using nonbold math in the Abstract.
% This preserves the distinction between vectors and scalars. However,
% if the journal you are submitting to favors bold math in the abstract,
% then you can use LaTeX's standard command \boldmath at the very start
% of the abstract to achieve this. Many IEEE journals frown on math
% in the abstract anyway.

% Note that keywords are not normally used for peerreview papers.
%\begin{IEEEkeywords}
%Cooperative diversity, decode and forward, piecewise linear
%\end{IEEEkeywords}



% For peer review papers, you can put extra information on the cover
% page as needed:
% \ifCLASSOPTIONpeerreview
% \begin{center} \bfseries EDICS Category: 3-BBND \end{center}
% \fi
%
% For peerreview papers, this IEEEtran command inserts a page break and
% creates the second title. It will be ignored for other modes.
%\IEEEpeerreviewmaketitle




\item Consider the probability space $\brak{\Omega, \mathcal{G}, P}$ where $\Omega = [0,2]$ and $\mathcal{G} = \cbrak{\phi, \Omega, [0,1], (1,2]}$. Let $X$ and $Y$ be two functions on $\Omega$ defined as
\begin{align*}
    X(\omega) = 
    \begin{cases}
        1 & \text{if }\omega \in [0, 1]\\
        2 & \text{if }\omega \in (1, 2]
    \end{cases}
\end{align*}
and
\begin{align*}
    Y(\omega) = 
    \begin{cases}
        2 & \text{if }\omega \in [0, 1.5]\\
        3 & \text{if }\omega \in (1.5, 2].
    \end{cases}
\end{align*}
Then which one of the following statements is true?
\begin{enumerate}
    \item [(A)] $X$ is a random variable with respect to $\mathcal{G}$, but $Y$ is not a random variable with respect to $\mathcal{G}$.
    \item [(B)] $Y$ is a random variable with respect to $\mathcal{G}$, but $X$ is not a random variable with respect to $\mathcal{G}$.
    \item [(C)] Neither $X$ nor $Y$ is a random variable with respect to $\mathcal{G}$.
    \item [(D)] Both $X$ and $Y$ are random variables with respect to $\mathcal{G}$.
\end{enumerate} \hfill (GATE ST 2023)\\
\solution
%\begin{table}[H]
	\centering
\begin{tabular}{|c|c|c|}
\hline
Random variable &Value &Definition\\ \hline
\multirow{3}{*}{X} &0 &Slips of Rs 1\\
&1 &Slips of Rs 5\\
&2 &Slips of Rs 13\\ \hline
\multirow{2}{*}{Y} &0 &Box A\\
&1 &Box B\\\hline
\end{tabular}
\caption{}
\label{tab:Distribution}
\end{table}
See \tabref{tab:Distribution}.
\begin{align}
p_{Y}\brak{k}= \begin{cases} 
      \frac{1}{3} & {k=0} \\
      \frac{2}{3 }& {k=1} 
   \end{cases}
   \\
p_{Y|X}\brak{0|0} = \frac{19}{25}\, 
p_{Y|X}\brak{0|1} = \frac{6}{25}\,
p_{Y|X}\brak{1|0} = \frac{45}{50}\,
p_{Y|X}\brak{1|2} = \frac{5}{50}
\end{align}
The desired probability is the probability that a slip drawn at random is marked other than Rs 1,
\begin{align}
&=1-p_X\brak{0}\\
&= p_X(1) + p_X(2)
\end{align}
Using Bayes theorem,
\begin{align}
&= p_Y\brak{0} \times \pr{Y=0 | X=1} + p_Y\brak{1} \times \pr{Y=1|X=2}\\
&=\frac{1}{3} \times \frac{6}{25} + \frac{2}{3} \times \frac{5}{50}\\
&=\frac{11}{75}
\end{align}

\newpage

%\tableofcontents

\bigskip

\renewcommand{\thefigure}{\theenumi}
\renewcommand{\thetable}{\theenumi}
%\renewcommand{\theequation}{\theenumi}

%\begin{abstract}
%%\boldmath
%In this letter, an algorithm for evaluating the exact analytical bit error rate  (BER)  for the piecewise linear (PL) combiner for  multiple relays is presented. Previous results were available only for upto three relays. The algorithm is unique in the sense that  the actual mathematical expressions, that are prohibitively large, need not be explicitly obtained. The diversity gain due to multiple relays is shown through plots of the analytical BER, well supported by simulations. 
%
%\end{abstract}
% IEEEtran.cls defaults to using nonbold math in the Abstract.
% This preserves the distinction between vectors and scalars. However,
% if the journal you are submitting to favors bold math in the abstract,
% then you can use LaTeX's standard command \boldmath at the very start
% of the abstract to achieve this. Many IEEE journals frown on math
% in the abstract anyway.

% Note that keywords are not normally used for peerreview papers.
%\begin{IEEEkeywords}
%Cooperative diversity, decode and forward, piecewise linear
%\end{IEEEkeywords}



% For peer review papers, you can put extra information on the cover
% page as needed:
% \ifCLASSOPTIONpeerreview
% \begin{center} \bfseries EDICS Category: 3-BBND \end{center}
% \fi
%
% For peerreview papers, this IEEEtran command inserts a page break and
% creates the second title. It will be ignored for other modes.
%\IEEEpeerreviewmaketitle




	\item  A die is loaded in such a way that each odd number is twice as likely to occur as
each even number. Find $P(G)$, where $G$ is the event that a number greater than
3 occurs on a single roll of the die.
\\
\solution
		%\begin{table}[H]
	\centering
\begin{tabular}{|c|c|c|}
\hline
Random variable &Value &Definition\\ \hline
\multirow{3}{*}{X} &0 &Slips of Rs 1\\
&1 &Slips of Rs 5\\
&2 &Slips of Rs 13\\ \hline
\multirow{2}{*}{Y} &0 &Box A\\
&1 &Box B\\\hline
\end{tabular}
\caption{}
\label{tab:Distribution}
\end{table}
See \tabref{tab:Distribution}.
\begin{align}
p_{Y}\brak{k}= \begin{cases} 
      \frac{1}{3} & {k=0} \\
      \frac{2}{3 }& {k=1} 
   \end{cases}
   \\
p_{Y|X}\brak{0|0} = \frac{19}{25}\, 
p_{Y|X}\brak{0|1} = \frac{6}{25}\,
p_{Y|X}\brak{1|0} = \frac{45}{50}\,
p_{Y|X}\brak{1|2} = \frac{5}{50}
\end{align}
The desired probability is the probability that a slip drawn at random is marked other than Rs 1,
\begin{align}
&=1-p_X\brak{0}\\
&= p_X(1) + p_X(2)
\end{align}
Using Bayes theorem,
\begin{align}
&= p_Y\brak{0} \times \pr{Y=0 | X=1} + p_Y\brak{1} \times \pr{Y=1|X=2}\\
&=\frac{1}{3} \times \frac{6}{25} + \frac{2}{3} \times \frac{5}{50}\\
&=\frac{11}{75}
\end{align}

\newpage

%\tableofcontents

\bigskip

\renewcommand{\thefigure}{\theenumi}
\renewcommand{\thetable}{\theenumi}
%\renewcommand{\theequation}{\theenumi}

%\begin{abstract}
%%\boldmath
%In this letter, an algorithm for evaluating the exact analytical bit error rate  (BER)  for the piecewise linear (PL) combiner for  multiple relays is presented. Previous results were available only for upto three relays. The algorithm is unique in the sense that  the actual mathematical expressions, that are prohibitively large, need not be explicitly obtained. The diversity gain due to multiple relays is shown through plots of the analytical BER, well supported by simulations. 
%
%\end{abstract}
% IEEEtran.cls defaults to using nonbold math in the Abstract.
% This preserves the distinction between vectors and scalars. However,
% if the journal you are submitting to favors bold math in the abstract,
% then you can use LaTeX's standard command \boldmath at the very start
% of the abstract to achieve this. Many IEEE journals frown on math
% in the abstract anyway.

% Note that keywords are not normally used for peerreview papers.
%\begin{IEEEkeywords}
%Cooperative diversity, decode and forward, piecewise linear
%\end{IEEEkeywords}



% For peer review papers, you can put extra information on the cover
% page as needed:
% \ifCLASSOPTIONpeerreview
% \begin{center} \bfseries EDICS Category: 3-BBND \end{center}
% \fi
%
% For peerreview papers, this IEEEtran command inserts a page break and
% creates the second title. It will be ignored for other modes.
%\IEEEpeerreviewmaketitle




	\item All the jacks, queens and kings are removed from a deck of 52 playing cards. The remaining cards are well shuffled and then one card is drawn at random. Giving ace a value 1 similar value for other cards, find the probability that the card has a value 
		\begin{enumerate}
			\item 7
			\item greater than 7
			\item less than 7
		\end{enumerate}
		%Number of cards left after removing all jacks, queens and kings 
\begin{align}
N	= 52 - 4\times 3
	= 40
\end{align}
%\begin{table}[H]
%\def\arraystretch{1.2}
%\begin{tabular}{|c|c|c|}
%\hline
%	\textbf{Parameter} &\textbf{Value} &\textbf{Description}\\ \hline
%	$X$ &1-10 &Represents the value of the card picked \\ \hline
%\end{tabular}
%\end{table}
Let $1 \le X \le 10$ be the value of the card picked.  Then,
\begin{align}
	p_X(k) &= \Pr(X=k)\ \forall\ 1 \leq k \leq 10\\
	&= \frac{4\times 1}{40}\\
	&= \frac{1}{10}\\
	\therefore p_X(k) &= 
	\begin{cases}
		\frac{1}{10} & 1 \leq k \leq 10\\
		0 & \text{otherwise}
	\end{cases}
\end{align}
and
\begin{align}
	F_{X}(k) &= \sum_{m=0}^{k}p_{X}(m) \quad 1 \leq k \leq 10\\
	&= \frac{k}{10}\\
	\therefore F_{X}(k) &= 
	\begin{cases}
		0 & k \leq 0\\
		\frac{k}{10} & 1\leq k \leq 10\\
		1 & k > 10 
	\end{cases}
\end{align}
\begin{enumerate}
	\item Probability that card has value equal to 7 is
		\begin{align}
			 p_{X}(7)
			= \frac{1}{10}
		\end{align}
	\item Probability that card has value greater than 7 is
		\begin{align}
			1 - F_X(7)
			&= 1 - \frac{7}{10}
			\\
			&= \frac{3}{10}
		\end{align}
	\item Probability that card has value less than 7 is
		\begin{align}
			 F_{X}(6)
			=\frac{6}{10}
		\end{align}
\end{enumerate}

  \item A Lot consists of 48 mobile phones of which 42 are good, 3 have only minor defects and 3 have major defects.Varnika will buy a phone if it is good but the trader will only buy a mobile if it has no major defects. One phone is selected at random from the lot. What is the probability that it is
\begin{enumerate}
	\item acceptable to Varnika?
            \item acceptable to the trader?
\end{enumerate}
\solution
	%\begin{table}[H]
	\centering
\begin{tabular}{|c|c|c|}
\hline
Random variable &Value &Definition\\ \hline
\multirow{3}{*}{X} &0 &Slips of Rs 1\\
&1 &Slips of Rs 5\\
&2 &Slips of Rs 13\\ \hline
\multirow{2}{*}{Y} &0 &Box A\\
&1 &Box B\\\hline
\end{tabular}
\caption{}
\label{tab:Distribution}
\end{table}
See \tabref{tab:Distribution}.
\begin{align}
p_{Y}\brak{k}= \begin{cases} 
      \frac{1}{3} & {k=0} \\
      \frac{2}{3 }& {k=1} 
   \end{cases}
   \\
p_{Y|X}\brak{0|0} = \frac{19}{25}\, 
p_{Y|X}\brak{0|1} = \frac{6}{25}\,
p_{Y|X}\brak{1|0} = \frac{45}{50}\,
p_{Y|X}\brak{1|2} = \frac{5}{50}
\end{align}
The desired probability is the probability that a slip drawn at random is marked other than Rs 1,
\begin{align}
&=1-p_X\brak{0}\\
&= p_X(1) + p_X(2)
\end{align}
Using Bayes theorem,
\begin{align}
&= p_Y\brak{0} \times \pr{Y=0 | X=1} + p_Y\brak{1} \times \pr{Y=1|X=2}\\
&=\frac{1}{3} \times \frac{6}{25} + \frac{2}{3} \times \frac{5}{50}\\
&=\frac{11}{75}
\end{align}

\newpage

%\tableofcontents

\bigskip

\renewcommand{\thefigure}{\theenumi}
\renewcommand{\thetable}{\theenumi}
%\renewcommand{\theequation}{\theenumi}

%\begin{abstract}
%%\boldmath
%In this letter, an algorithm for evaluating the exact analytical bit error rate  (BER)  for the piecewise linear (PL) combiner for  multiple relays is presented. Previous results were available only for upto three relays. The algorithm is unique in the sense that  the actual mathematical expressions, that are prohibitively large, need not be explicitly obtained. The diversity gain due to multiple relays is shown through plots of the analytical BER, well supported by simulations. 
%
%\end{abstract}
% IEEEtran.cls defaults to using nonbold math in the Abstract.
% This preserves the distinction between vectors and scalars. However,
% if the journal you are submitting to favors bold math in the abstract,
% then you can use LaTeX's standard command \boldmath at the very start
% of the abstract to achieve this. Many IEEE journals frown on math
% in the abstract anyway.

% Note that keywords are not normally used for peerreview papers.
%\begin{IEEEkeywords}
%Cooperative diversity, decode and forward, piecewise linear
%\end{IEEEkeywords}



% For peer review papers, you can put extra information on the cover
% page as needed:
% \ifCLASSOPTIONpeerreview
% \begin{center} \bfseries EDICS Category: 3-BBND \end{center}
% \fi
%
% For peerreview papers, this IEEEtran command inserts a page break and
% creates the second title. It will be ignored for other modes.
%\IEEEpeerreviewmaketitle




 \item A student says that if you throw a die, it will show up 1 or not 1. Therefore, the probability of getting 1 and the probability of getting 'not 1' each is equal to $\frac{1}{2}$. Is this correct? Give reasons.\\
 \solution
        %\begin{table}[H]
	\centering
\begin{tabular}{|c|c|c|}
\hline
Random variable &Value &Definition\\ \hline
\multirow{3}{*}{X} &0 &Slips of Rs 1\\
&1 &Slips of Rs 5\\
&2 &Slips of Rs 13\\ \hline
\multirow{2}{*}{Y} &0 &Box A\\
&1 &Box B\\\hline
\end{tabular}
\caption{}
\label{tab:Distribution}
\end{table}
See \tabref{tab:Distribution}.
\begin{align}
p_{Y}\brak{k}= \begin{cases} 
      \frac{1}{3} & {k=0} \\
      \frac{2}{3 }& {k=1} 
   \end{cases}
   \\
p_{Y|X}\brak{0|0} = \frac{19}{25}\, 
p_{Y|X}\brak{0|1} = \frac{6}{25}\,
p_{Y|X}\brak{1|0} = \frac{45}{50}\,
p_{Y|X}\brak{1|2} = \frac{5}{50}
\end{align}
The desired probability is the probability that a slip drawn at random is marked other than Rs 1,
\begin{align}
&=1-p_X\brak{0}\\
&= p_X(1) + p_X(2)
\end{align}
Using Bayes theorem,
\begin{align}
&= p_Y\brak{0} \times \pr{Y=0 | X=1} + p_Y\brak{1} \times \pr{Y=1|X=2}\\
&=\frac{1}{3} \times \frac{6}{25} + \frac{2}{3} \times \frac{5}{50}\\
&=\frac{11}{75}
\end{align}

\newpage

%\tableofcontents

\bigskip

\renewcommand{\thefigure}{\theenumi}
\renewcommand{\thetable}{\theenumi}
%\renewcommand{\theequation}{\theenumi}

%\begin{abstract}
%%\boldmath
%In this letter, an algorithm for evaluating the exact analytical bit error rate  (BER)  for the piecewise linear (PL) combiner for  multiple relays is presented. Previous results were available only for upto three relays. The algorithm is unique in the sense that  the actual mathematical expressions, that are prohibitively large, need not be explicitly obtained. The diversity gain due to multiple relays is shown through plots of the analytical BER, well supported by simulations. 
%
%\end{abstract}
% IEEEtran.cls defaults to using nonbold math in the Abstract.
% This preserves the distinction between vectors and scalars. However,
% if the journal you are submitting to favors bold math in the abstract,
% then you can use LaTeX's standard command \boldmath at the very start
% of the abstract to achieve this. Many IEEE journals frown on math
% in the abstract anyway.

% Note that keywords are not normally used for peerreview papers.
%\begin{IEEEkeywords}
%Cooperative diversity, decode and forward, piecewise linear
%\end{IEEEkeywords}



% For peer review papers, you can put extra information on the cover
% page as needed:
% \ifCLASSOPTIONpeerreview
% \begin{center} \bfseries EDICS Category: 3-BBND \end{center}
% \fi
%
% For peerreview papers, this IEEEtran command inserts a page break and
% creates the second title. It will be ignored for other modes.
%\IEEEpeerreviewmaketitle




   \item Four candidates A, B, C, D have ap-
plied for the assignment to coach a school cricket
team. If A is twice as likely to be selected as B, and
B and C are given about the same chance of being
selected, while C is twice as likely to be selected
as D, what are the probabilities that
\begin{enumerate}
\item C will be selected?
\item A will not be selected?
\end{enumerate}
	%\begin{table}[H]
	\centering
\begin{tabular}{|c|c|c|}
\hline
Random variable &Value &Definition\\ \hline
\multirow{3}{*}{X} &0 &Slips of Rs 1\\
&1 &Slips of Rs 5\\
&2 &Slips of Rs 13\\ \hline
\multirow{2}{*}{Y} &0 &Box A\\
&1 &Box B\\\hline
\end{tabular}
\caption{}
\label{tab:Distribution}
\end{table}
See \tabref{tab:Distribution}.
\begin{align}
p_{Y}\brak{k}= \begin{cases} 
      \frac{1}{3} & {k=0} \\
      \frac{2}{3 }& {k=1} 
   \end{cases}
   \\
p_{Y|X}\brak{0|0} = \frac{19}{25}\, 
p_{Y|X}\brak{0|1} = \frac{6}{25}\,
p_{Y|X}\brak{1|0} = \frac{45}{50}\,
p_{Y|X}\brak{1|2} = \frac{5}{50}
\end{align}
The desired probability is the probability that a slip drawn at random is marked other than Rs 1,
\begin{align}
&=1-p_X\brak{0}\\
&= p_X(1) + p_X(2)
\end{align}
Using Bayes theorem,
\begin{align}
&= p_Y\brak{0} \times \pr{Y=0 | X=1} + p_Y\brak{1} \times \pr{Y=1|X=2}\\
&=\frac{1}{3} \times \frac{6}{25} + \frac{2}{3} \times \frac{5}{50}\\
&=\frac{11}{75}
\end{align}

\newpage

%\tableofcontents

\bigskip

\renewcommand{\thefigure}{\theenumi}
\renewcommand{\thetable}{\theenumi}
%\renewcommand{\theequation}{\theenumi}

%\begin{abstract}
%%\boldmath
%In this letter, an algorithm for evaluating the exact analytical bit error rate  (BER)  for the piecewise linear (PL) combiner for  multiple relays is presented. Previous results were available only for upto three relays. The algorithm is unique in the sense that  the actual mathematical expressions, that are prohibitively large, need not be explicitly obtained. The diversity gain due to multiple relays is shown through plots of the analytical BER, well supported by simulations. 
%
%\end{abstract}
% IEEEtran.cls defaults to using nonbold math in the Abstract.
% This preserves the distinction between vectors and scalars. However,
% if the journal you are submitting to favors bold math in the abstract,
% then you can use LaTeX's standard command \boldmath at the very start
% of the abstract to achieve this. Many IEEE journals frown on math
% in the abstract anyway.

% Note that keywords are not normally used for peerreview papers.
%\begin{IEEEkeywords}
%Cooperative diversity, decode and forward, piecewise linear
%\end{IEEEkeywords}



% For peer review papers, you can put extra information on the cover
% page as needed:
% \ifCLASSOPTIONpeerreview
% \begin{center} \bfseries EDICS Category: 3-BBND \end{center}
% \fi
%
% For peerreview papers, this IEEEtran command inserts a page break and
% creates the second title. It will be ignored for other modes.
%\IEEEpeerreviewmaketitle




 \item A bag contain 24 balls of which $x$ balls are red, $2x$ are white and $3x$ are blue. A ball is selected at random, What is the probability that it is
\begin{enumerate}[label=\alph*)]
\item not red ?
\item white ?
\end{enumerate}
%\begin{table}[H]
	\centering
\begin{tabular}{|c|c|c|}
\hline
Random variable &Value &Definition\\ \hline
\multirow{3}{*}{X} &0 &Slips of Rs 1\\
&1 &Slips of Rs 5\\
&2 &Slips of Rs 13\\ \hline
\multirow{2}{*}{Y} &0 &Box A\\
&1 &Box B\\\hline
\end{tabular}
\caption{}
\label{tab:Distribution}
\end{table}
See \tabref{tab:Distribution}.
\begin{align}
p_{Y}\brak{k}= \begin{cases} 
      \frac{1}{3} & {k=0} \\
      \frac{2}{3 }& {k=1} 
   \end{cases}
   \\
p_{Y|X}\brak{0|0} = \frac{19}{25}\, 
p_{Y|X}\brak{0|1} = \frac{6}{25}\,
p_{Y|X}\brak{1|0} = \frac{45}{50}\,
p_{Y|X}\brak{1|2} = \frac{5}{50}
\end{align}
The desired probability is the probability that a slip drawn at random is marked other than Rs 1,
\begin{align}
&=1-p_X\brak{0}\\
&= p_X(1) + p_X(2)
\end{align}
Using Bayes theorem,
\begin{align}
&= p_Y\brak{0} \times \pr{Y=0 | X=1} + p_Y\brak{1} \times \pr{Y=1|X=2}\\
&=\frac{1}{3} \times \frac{6}{25} + \frac{2}{3} \times \frac{5}{50}\\
&=\frac{11}{75}
\end{align}

\newpage

%\tableofcontents

\bigskip

\renewcommand{\thefigure}{\theenumi}
\renewcommand{\thetable}{\theenumi}
%\renewcommand{\theequation}{\theenumi}

%\begin{abstract}
%%\boldmath
%In this letter, an algorithm for evaluating the exact analytical bit error rate  (BER)  for the piecewise linear (PL) combiner for  multiple relays is presented. Previous results were available only for upto three relays. The algorithm is unique in the sense that  the actual mathematical expressions, that are prohibitively large, need not be explicitly obtained. The diversity gain due to multiple relays is shown through plots of the analytical BER, well supported by simulations. 
%
%\end{abstract}
% IEEEtran.cls defaults to using nonbold math in the Abstract.
% This preserves the distinction between vectors and scalars. However,
% if the journal you are submitting to favors bold math in the abstract,
% then you can use LaTeX's standard command \boldmath at the very start
% of the abstract to achieve this. Many IEEE journals frown on math
% in the abstract anyway.

% Note that keywords are not normally used for peerreview papers.
%\begin{IEEEkeywords}
%Cooperative diversity, decode and forward, piecewise linear
%\end{IEEEkeywords}



% For peer review papers, you can put extra information on the cover
% page as needed:
% \ifCLASSOPTIONpeerreview
% \begin{center} \bfseries EDICS Category: 3-BBND \end{center}
% \fi
%
% For peerreview papers, this IEEEtran command inserts a page break and
% creates the second title. It will be ignored for other modes.
%\IEEEpeerreviewmaketitle




If the letters of the word ASSASSINATION are arranged at random. Find the Probability that
\begin{enumerate}[label=(\alph*)]
\item Four $S's$ come consecutively in the word
\item Two  $I's$ and two $N's$ come together
\item All $A's$ are not coming together
\item No two $A's$ are coming together
\end{enumerate}
%\begin{table}[H]
	\centering
\begin{tabular}{|c|c|c|}
\hline
Random variable &Value &Definition\\ \hline
\multirow{3}{*}{X} &0 &Slips of Rs 1\\
&1 &Slips of Rs 5\\
&2 &Slips of Rs 13\\ \hline
\multirow{2}{*}{Y} &0 &Box A\\
&1 &Box B\\\hline
\end{tabular}
\caption{}
\label{tab:Distribution}
\end{table}
See \tabref{tab:Distribution}.
\begin{align}
p_{Y}\brak{k}= \begin{cases} 
      \frac{1}{3} & {k=0} \\
      \frac{2}{3 }& {k=1} 
   \end{cases}
   \\
p_{Y|X}\brak{0|0} = \frac{19}{25}\, 
p_{Y|X}\brak{0|1} = \frac{6}{25}\,
p_{Y|X}\brak{1|0} = \frac{45}{50}\,
p_{Y|X}\brak{1|2} = \frac{5}{50}
\end{align}
The desired probability is the probability that a slip drawn at random is marked other than Rs 1,
\begin{align}
&=1-p_X\brak{0}\\
&= p_X(1) + p_X(2)
\end{align}
Using Bayes theorem,
\begin{align}
&= p_Y\brak{0} \times \pr{Y=0 | X=1} + p_Y\brak{1} \times \pr{Y=1|X=2}\\
&=\frac{1}{3} \times \frac{6}{25} + \frac{2}{3} \times \frac{5}{50}\\
&=\frac{11}{75}
\end{align}

\newpage

%\tableofcontents

\bigskip

\renewcommand{\thefigure}{\theenumi}
\renewcommand{\thetable}{\theenumi}
%\renewcommand{\theequation}{\theenumi}

%\begin{abstract}
%%\boldmath
%In this letter, an algorithm for evaluating the exact analytical bit error rate  (BER)  for the piecewise linear (PL) combiner for  multiple relays is presented. Previous results were available only for upto three relays. The algorithm is unique in the sense that  the actual mathematical expressions, that are prohibitively large, need not be explicitly obtained. The diversity gain due to multiple relays is shown through plots of the analytical BER, well supported by simulations. 
%
%\end{abstract}
% IEEEtran.cls defaults to using nonbold math in the Abstract.
% This preserves the distinction between vectors and scalars. However,
% if the journal you are submitting to favors bold math in the abstract,
% then you can use LaTeX's standard command \boldmath at the very start
% of the abstract to achieve this. Many IEEE journals frown on math
% in the abstract anyway.

% Note that keywords are not normally used for peerreview papers.
%\begin{IEEEkeywords}
%Cooperative diversity, decode and forward, piecewise linear
%\end{IEEEkeywords}



% For peer review papers, you can put extra information on the cover
% page as needed:
% \ifCLASSOPTIONpeerreview
% \begin{center} \bfseries EDICS Category: 3-BBND \end{center}
% \fi
%
% For peerreview papers, this IEEEtran command inserts a page break and
% creates the second title. It will be ignored for other modes.
%\IEEEpeerreviewmaketitle




	\item One urn contains two black balls (labelled B1 and B2) and one white ball. A
	second urn contains one black ball and two white balls (labelled W1 and W2).
	Suppose the following experiment is performed. One of the two urns is chosen
	at random. Next a ball is randomly chosen from the urn. Then a second ball is
	chosen at random from the same urn without replacing the first ball.
	
	\begin{enumerate}
	\item What is the probability that two black balls are chosen?
	
	\item What is the probability that two balls of opposite colour are chosen?
	\end{enumerate}
	\solution
	%\begin{align}
    \label{eq:12.13.6.18.1}
	\because	\pr{A|B} &> \pr{A},\
\frac{\pr{AB}}{\pr{B}} > \pr{A}
\\
    \label{eq:12.13.6.18.2}
	\implies \pr{AB} &> \pr{A}\pr{B}
	\\
	\text{or, } \frac{\pr{AB}}{\pr{A}} &=\pr{B|A} > \pr{A}
\end{align}

\end{enumerate}

	\item A bag contains 4 red and 4 black balls, another bag contains 2 red and 6 black balls. One of the two bags is selected at random and a ball is drawn from the bag which is found to be red. Find the probability that the ball is drawn from the first bag.
\\
\solution
		%\begin{table}[H]
	\centering
\begin{tabular}{|c|c|c|}
\hline
Random variable &Value &Definition\\ \hline
\multirow{3}{*}{X} &0 &Slips of Rs 1\\
&1 &Slips of Rs 5\\
&2 &Slips of Rs 13\\ \hline
\multirow{2}{*}{Y} &0 &Box A\\
&1 &Box B\\\hline
\end{tabular}
\caption{}
\label{tab:Distribution}
\end{table}
See \tabref{tab:Distribution}.
\begin{align}
p_{Y}\brak{k}= \begin{cases} 
      \frac{1}{3} & {k=0} \\
      \frac{2}{3 }& {k=1} 
   \end{cases}
   \\
p_{Y|X}\brak{0|0} = \frac{19}{25}\, 
p_{Y|X}\brak{0|1} = \frac{6}{25}\,
p_{Y|X}\brak{1|0} = \frac{45}{50}\,
p_{Y|X}\brak{1|2} = \frac{5}{50}
\end{align}
The desired probability is the probability that a slip drawn at random is marked other than Rs 1,
\begin{align}
&=1-p_X\brak{0}\\
&= p_X(1) + p_X(2)
\end{align}
Using Bayes theorem,
\begin{align}
&= p_Y\brak{0} \times \pr{Y=0 | X=1} + p_Y\brak{1} \times \pr{Y=1|X=2}\\
&=\frac{1}{3} \times \frac{6}{25} + \frac{2}{3} \times \frac{5}{50}\\
&=\frac{11}{75}
\end{align}

\newpage

%\tableofcontents

\bigskip

\renewcommand{\thefigure}{\theenumi}
\renewcommand{\thetable}{\theenumi}
%\renewcommand{\theequation}{\theenumi}

%\begin{abstract}
%%\boldmath
%In this letter, an algorithm for evaluating the exact analytical bit error rate  (BER)  for the piecewise linear (PL) combiner for  multiple relays is presented. Previous results were available only for upto three relays. The algorithm is unique in the sense that  the actual mathematical expressions, that are prohibitively large, need not be explicitly obtained. The diversity gain due to multiple relays is shown through plots of the analytical BER, well supported by simulations. 
%
%\end{abstract}
% IEEEtran.cls defaults to using nonbold math in the Abstract.
% This preserves the distinction between vectors and scalars. However,
% if the journal you are submitting to favors bold math in the abstract,
% then you can use LaTeX's standard command \boldmath at the very start
% of the abstract to achieve this. Many IEEE journals frown on math
% in the abstract anyway.

% Note that keywords are not normally used for peerreview papers.
%\begin{IEEEkeywords}
%Cooperative diversity, decode and forward, piecewise linear
%\end{IEEEkeywords}



% For peer review papers, you can put extra information on the cover
% page as needed:
% \ifCLASSOPTIONpeerreview
% \begin{center} \bfseries EDICS Category: 3-BBND \end{center}
% \fi
%
% For peerreview papers, this IEEEtran command inserts a page break and
% creates the second title. It will be ignored for other modes.
%\IEEEpeerreviewmaketitle




  \item
  Cards with numbers 2 to 101 are placed in a box. A card is selected at random.Find the probability that the card has
\begin{enumerate}[label=(\roman*)]
	\item an even number 
	\item a square number
\end{enumerate}
\solution
%\begin{table}[H]
	\centering
\begin{tabular}{|c|c|c|}
\hline
Random variable &Value &Definition\\ \hline
\multirow{3}{*}{X} &0 &Slips of Rs 1\\
&1 &Slips of Rs 5\\
&2 &Slips of Rs 13\\ \hline
\multirow{2}{*}{Y} &0 &Box A\\
&1 &Box B\\\hline
\end{tabular}
\caption{}
\label{tab:Distribution}
\end{table}
See \tabref{tab:Distribution}.
\begin{align}
p_{Y}\brak{k}= \begin{cases} 
      \frac{1}{3} & {k=0} \\
      \frac{2}{3 }& {k=1} 
   \end{cases}
   \\
p_{Y|X}\brak{0|0} = \frac{19}{25}\, 
p_{Y|X}\brak{0|1} = \frac{6}{25}\,
p_{Y|X}\brak{1|0} = \frac{45}{50}\,
p_{Y|X}\brak{1|2} = \frac{5}{50}
\end{align}
The desired probability is the probability that a slip drawn at random is marked other than Rs 1,
\begin{align}
&=1-p_X\brak{0}\\
&= p_X(1) + p_X(2)
\end{align}
Using Bayes theorem,
\begin{align}
&= p_Y\brak{0} \times \pr{Y=0 | X=1} + p_Y\brak{1} \times \pr{Y=1|X=2}\\
&=\frac{1}{3} \times \frac{6}{25} + \frac{2}{3} \times \frac{5}{50}\\
&=\frac{11}{75}
\end{align}

\newpage

%\tableofcontents

\bigskip

\renewcommand{\thefigure}{\theenumi}
\renewcommand{\thetable}{\theenumi}
%\renewcommand{\theequation}{\theenumi}

%\begin{abstract}
%%\boldmath
%In this letter, an algorithm for evaluating the exact analytical bit error rate  (BER)  for the piecewise linear (PL) combiner for  multiple relays is presented. Previous results were available only for upto three relays. The algorithm is unique in the sense that  the actual mathematical expressions, that are prohibitively large, need not be explicitly obtained. The diversity gain due to multiple relays is shown through plots of the analytical BER, well supported by simulations. 
%
%\end{abstract}
% IEEEtran.cls defaults to using nonbold math in the Abstract.
% This preserves the distinction between vectors and scalars. However,
% if the journal you are submitting to favors bold math in the abstract,
% then you can use LaTeX's standard command \boldmath at the very start
% of the abstract to achieve this. Many IEEE journals frown on math
% in the abstract anyway.

% Note that keywords are not normally used for peerreview papers.
%\begin{IEEEkeywords}
%Cooperative diversity, decode and forward, piecewise linear
%\end{IEEEkeywords}



% For peer review papers, you can put extra information on the cover
% page as needed:
% \ifCLASSOPTIONpeerreview
% \begin{center} \bfseries EDICS Category: 3-BBND \end{center}
% \fi
%
% For peerreview papers, this IEEEtran command inserts a page break and
% creates the second title. It will be ignored for other modes.
%\IEEEpeerreviewmaketitle




\item
The king, queen and jack of clubs are removed from a deck of 52 playing cards and then well shuffled. Now one card is drawn at random from the remaining cards.  Determine the probability that the card is
\begin{enumerate}[label=(\roman*)]
\item a club
\item 10 of hearts
\end{enumerate}
\solution
%\begin{table}[H]
	\centering
\begin{tabular}{|c|c|c|}
\hline
Random variable &Value &Definition\\ \hline
\multirow{3}{*}{X} &0 &Slips of Rs 1\\
&1 &Slips of Rs 5\\
&2 &Slips of Rs 13\\ \hline
\multirow{2}{*}{Y} &0 &Box A\\
&1 &Box B\\\hline
\end{tabular}
\caption{}
\label{tab:Distribution}
\end{table}
See \tabref{tab:Distribution}.
\begin{align}
p_{Y}\brak{k}= \begin{cases} 
      \frac{1}{3} & {k=0} \\
      \frac{2}{3 }& {k=1} 
   \end{cases}
   \\
p_{Y|X}\brak{0|0} = \frac{19}{25}\, 
p_{Y|X}\brak{0|1} = \frac{6}{25}\,
p_{Y|X}\brak{1|0} = \frac{45}{50}\,
p_{Y|X}\brak{1|2} = \frac{5}{50}
\end{align}
The desired probability is the probability that a slip drawn at random is marked other than Rs 1,
\begin{align}
&=1-p_X\brak{0}\\
&= p_X(1) + p_X(2)
\end{align}
Using Bayes theorem,
\begin{align}
&= p_Y\brak{0} \times \pr{Y=0 | X=1} + p_Y\brak{1} \times \pr{Y=1|X=2}\\
&=\frac{1}{3} \times \frac{6}{25} + \frac{2}{3} \times \frac{5}{50}\\
&=\frac{11}{75}
\end{align}

\newpage

%\tableofcontents

\bigskip

\renewcommand{\thefigure}{\theenumi}
\renewcommand{\thetable}{\theenumi}
%\renewcommand{\theequation}{\theenumi}

%\begin{abstract}
%%\boldmath
%In this letter, an algorithm for evaluating the exact analytical bit error rate  (BER)  for the piecewise linear (PL) combiner for  multiple relays is presented. Previous results were available only for upto three relays. The algorithm is unique in the sense that  the actual mathematical expressions, that are prohibitively large, need not be explicitly obtained. The diversity gain due to multiple relays is shown through plots of the analytical BER, well supported by simulations. 
%
%\end{abstract}
% IEEEtran.cls defaults to using nonbold math in the Abstract.
% This preserves the distinction between vectors and scalars. However,
% if the journal you are submitting to favors bold math in the abstract,
% then you can use LaTeX's standard command \boldmath at the very start
% of the abstract to achieve this. Many IEEE journals frown on math
% in the abstract anyway.

% Note that keywords are not normally used for peerreview papers.
%\begin{IEEEkeywords}
%Cooperative diversity, decode and forward, piecewise linear
%\end{IEEEkeywords}



% For peer review papers, you can put extra information on the cover
% page as needed:
% \ifCLASSOPTIONpeerreview
% \begin{center} \bfseries EDICS Category: 3-BBND \end{center}
% \fi
%
% For peerreview papers, this IEEEtran command inserts a page break and
% creates the second title. It will be ignored for other modes.
%\IEEEpeerreviewmaketitle




\item A team of medical students doing their internship have to assist during surgeries
at a city hospital. The probabilities of surgeries rated as very complex, complex,
routine, simple or very simple are respectively, 0.15, 0.20, 0.31, 0.26, .08. Find
the probabilities that a particular surgery will be rated
\begin{enumerate}
	\item complex or very complex;
	\item neither very complex nor very simple;
	\item routine or complex
	\item routine or simple
\end{enumerate}
\solution
%\begin{table}[H]
	\centering
\begin{tabular}{|c|c|c|}
\hline
Random variable &Value &Definition\\ \hline
\multirow{3}{*}{X} &0 &Slips of Rs 1\\
&1 &Slips of Rs 5\\
&2 &Slips of Rs 13\\ \hline
\multirow{2}{*}{Y} &0 &Box A\\
&1 &Box B\\\hline
\end{tabular}
\caption{}
\label{tab:Distribution}
\end{table}
See \tabref{tab:Distribution}.
\begin{align}
p_{Y}\brak{k}= \begin{cases} 
      \frac{1}{3} & {k=0} \\
      \frac{2}{3 }& {k=1} 
   \end{cases}
   \\
p_{Y|X}\brak{0|0} = \frac{19}{25}\, 
p_{Y|X}\brak{0|1} = \frac{6}{25}\,
p_{Y|X}\brak{1|0} = \frac{45}{50}\,
p_{Y|X}\brak{1|2} = \frac{5}{50}
\end{align}
The desired probability is the probability that a slip drawn at random is marked other than Rs 1,
\begin{align}
&=1-p_X\brak{0}\\
&= p_X(1) + p_X(2)
\end{align}
Using Bayes theorem,
\begin{align}
&= p_Y\brak{0} \times \pr{Y=0 | X=1} + p_Y\brak{1} \times \pr{Y=1|X=2}\\
&=\frac{1}{3} \times \frac{6}{25} + \frac{2}{3} \times \frac{5}{50}\\
&=\frac{11}{75}
\end{align}

\newpage

%\tableofcontents

\bigskip

\renewcommand{\thefigure}{\theenumi}
\renewcommand{\thetable}{\theenumi}
%\renewcommand{\theequation}{\theenumi}

%\begin{abstract}
%%\boldmath
%In this letter, an algorithm for evaluating the exact analytical bit error rate  (BER)  for the piecewise linear (PL) combiner for  multiple relays is presented. Previous results were available only for upto three relays. The algorithm is unique in the sense that  the actual mathematical expressions, that are prohibitively large, need not be explicitly obtained. The diversity gain due to multiple relays is shown through plots of the analytical BER, well supported by simulations. 
%
%\end{abstract}
% IEEEtran.cls defaults to using nonbold math in the Abstract.
% This preserves the distinction between vectors and scalars. However,
% if the journal you are submitting to favors bold math in the abstract,
% then you can use LaTeX's standard command \boldmath at the very start
% of the abstract to achieve this. Many IEEE journals frown on math
% in the abstract anyway.

% Note that keywords are not normally used for peerreview papers.
%\begin{IEEEkeywords}
%Cooperative diversity, decode and forward, piecewise linear
%\end{IEEEkeywords}



% For peer review papers, you can put extra information on the cover
% page as needed:
% \ifCLASSOPTIONpeerreview
% \begin{center} \bfseries EDICS Category: 3-BBND \end{center}
% \fi
%
% For peerreview papers, this IEEEtran command inserts a page break and
% creates the second title. It will be ignored for other modes.
%\IEEEpeerreviewmaketitle




\item A card is selected from a pack of 52 cards.
\begin{enumerate}[label=(\alph*)]
    \item How many points are there in the sample space?
    \item Calculate the probability that the card is an ace of spades.
    \item Calculate the probability that the card is (i) an ace and (ii) black card.
\end{enumerate}
\solution
%Let $X$ be an bernoulli rv defined as in \tabref{tab:exemplar/11/16/3/26}.  Then, 
\begin{equation}
    p =
        \frac{4}{11} 
\end{equation}
\begin{table}[H]
	\centering
	\input{exemplar/11/16/3/26/tables/Table2.tex}
	\caption{}
        \label{tab:exemplar/11/16/3/26}
\end{table}

\item The probability that a non leap year selected at random will contain 53 sundays.
\\
\solution
%\begin{table}[H]
	\centering
\begin{tabular}{|c|c|c|}
\hline
Random variable &Value &Definition\\ \hline
\multirow{3}{*}{X} &0 &Slips of Rs 1\\
&1 &Slips of Rs 5\\
&2 &Slips of Rs 13\\ \hline
\multirow{2}{*}{Y} &0 &Box A\\
&1 &Box B\\\hline
\end{tabular}
\caption{}
\label{tab:Distribution}
\end{table}
See \tabref{tab:Distribution}.
\begin{align}
p_{Y}\brak{k}= \begin{cases} 
      \frac{1}{3} & {k=0} \\
      \frac{2}{3 }& {k=1} 
   \end{cases}
   \\
p_{Y|X}\brak{0|0} = \frac{19}{25}\, 
p_{Y|X}\brak{0|1} = \frac{6}{25}\,
p_{Y|X}\brak{1|0} = \frac{45}{50}\,
p_{Y|X}\brak{1|2} = \frac{5}{50}
\end{align}
The desired probability is the probability that a slip drawn at random is marked other than Rs 1,
\begin{align}
&=1-p_X\brak{0}\\
&= p_X(1) + p_X(2)
\end{align}
Using Bayes theorem,
\begin{align}
&= p_Y\brak{0} \times \pr{Y=0 | X=1} + p_Y\brak{1} \times \pr{Y=1|X=2}\\
&=\frac{1}{3} \times \frac{6}{25} + \frac{2}{3} \times \frac{5}{50}\\
&=\frac{11}{75}
\end{align}

\newpage

%\tableofcontents

\bigskip

\renewcommand{\thefigure}{\theenumi}
\renewcommand{\thetable}{\theenumi}
%\renewcommand{\theequation}{\theenumi}

%\begin{abstract}
%%\boldmath
%In this letter, an algorithm for evaluating the exact analytical bit error rate  (BER)  for the piecewise linear (PL) combiner for  multiple relays is presented. Previous results were available only for upto three relays. The algorithm is unique in the sense that  the actual mathematical expressions, that are prohibitively large, need not be explicitly obtained. The diversity gain due to multiple relays is shown through plots of the analytical BER, well supported by simulations. 
%
%\end{abstract}
% IEEEtran.cls defaults to using nonbold math in the Abstract.
% This preserves the distinction between vectors and scalars. However,
% if the journal you are submitting to favors bold math in the abstract,
% then you can use LaTeX's standard command \boldmath at the very start
% of the abstract to achieve this. Many IEEE journals frown on math
% in the abstract anyway.

% Note that keywords are not normally used for peerreview papers.
%\begin{IEEEkeywords}
%Cooperative diversity, decode and forward, piecewise linear
%\end{IEEEkeywords}



% For peer review papers, you can put extra information on the cover
% page as needed:
% \ifCLASSOPTIONpeerreview
% \begin{center} \bfseries EDICS Category: 3-BBND \end{center}
% \fi
%
% For peerreview papers, this IEEEtran command inserts a page break and
% creates the second title. It will be ignored for other modes.
%\IEEEpeerreviewmaketitle




\item One of the four persons John, Rita, Aslam or Gurpreet will be promoted next
month. Consequently the sample space consists of four elementary outcomes
S = {John promoted, Rita promoted, Aslam promoted, Gurpreet promoted}
You are told that the chances of John’s promotion is same as that of Gurpreet,
Rita’s chances of promotion are twice as likely as Johns. Aslam’s chances are
four times that of John.
\begin{enumerate}
	\item Determine
	\begin{enumerate}
		\item P (John promoted)
		\item P (Rita promoted)
		\item P (Aslam promoted)
		\item P (Gurpreet promoted)
	\end{enumerate}
	\item If A = {John promoted or Gurpreet promoted}, find P (A).
\end{enumerate}
\solution
%\begin{table}[H]
	\centering
\begin{tabular}{|c|c|c|}
\hline
Random variable &Value &Definition\\ \hline
\multirow{3}{*}{X} &0 &Slips of Rs 1\\
&1 &Slips of Rs 5\\
&2 &Slips of Rs 13\\ \hline
\multirow{2}{*}{Y} &0 &Box A\\
&1 &Box B\\\hline
\end{tabular}
\caption{}
\label{tab:Distribution}
\end{table}
See \tabref{tab:Distribution}.
\begin{align}
p_{Y}\brak{k}= \begin{cases} 
      \frac{1}{3} & {k=0} \\
      \frac{2}{3 }& {k=1} 
   \end{cases}
   \\
p_{Y|X}\brak{0|0} = \frac{19}{25}\, 
p_{Y|X}\brak{0|1} = \frac{6}{25}\,
p_{Y|X}\brak{1|0} = \frac{45}{50}\,
p_{Y|X}\brak{1|2} = \frac{5}{50}
\end{align}
The desired probability is the probability that a slip drawn at random is marked other than Rs 1,
\begin{align}
&=1-p_X\brak{0}\\
&= p_X(1) + p_X(2)
\end{align}
Using Bayes theorem,
\begin{align}
&= p_Y\brak{0} \times \pr{Y=0 | X=1} + p_Y\brak{1} \times \pr{Y=1|X=2}\\
&=\frac{1}{3} \times \frac{6}{25} + \frac{2}{3} \times \frac{5}{50}\\
&=\frac{11}{75}
\end{align}

\newpage

%\tableofcontents

\bigskip

\renewcommand{\thefigure}{\theenumi}
\renewcommand{\thetable}{\theenumi}
%\renewcommand{\theequation}{\theenumi}

%\begin{abstract}
%%\boldmath
%In this letter, an algorithm for evaluating the exact analytical bit error rate  (BER)  for the piecewise linear (PL) combiner for  multiple relays is presented. Previous results were available only for upto three relays. The algorithm is unique in the sense that  the actual mathematical expressions, that are prohibitively large, need not be explicitly obtained. The diversity gain due to multiple relays is shown through plots of the analytical BER, well supported by simulations. 
%
%\end{abstract}
% IEEEtran.cls defaults to using nonbold math in the Abstract.
% This preserves the distinction between vectors and scalars. However,
% if the journal you are submitting to favors bold math in the abstract,
% then you can use LaTeX's standard command \boldmath at the very start
% of the abstract to achieve this. Many IEEE journals frown on math
% in the abstract anyway.

% Note that keywords are not normally used for peerreview papers.
%\begin{IEEEkeywords}
%Cooperative diversity, decode and forward, piecewise linear
%\end{IEEEkeywords}



% For peer review papers, you can put extra information on the cover
% page as needed:
% \ifCLASSOPTIONpeerreview
% \begin{center} \bfseries EDICS Category: 3-BBND \end{center}
% \fi
%
% For peerreview papers, this IEEEtran command inserts a page break and
% creates the second title. It will be ignored for other modes.
%\IEEEpeerreviewmaketitle




\item A card is drawn from a deck of 52 cards. Find the probability of getting a king or a heart or a red card.\\
\solution
%\begin{table}[H]
	\centering
\begin{tabular}{|c|c|c|}
\hline
Random variable &Value &Definition\\ \hline
\multirow{3}{*}{X} &0 &Slips of Rs 1\\
&1 &Slips of Rs 5\\
&2 &Slips of Rs 13\\ \hline
\multirow{2}{*}{Y} &0 &Box A\\
&1 &Box B\\\hline
\end{tabular}
\caption{}
\label{tab:Distribution}
\end{table}
See \tabref{tab:Distribution}.
\begin{align}
p_{Y}\brak{k}= \begin{cases} 
      \frac{1}{3} & {k=0} \\
      \frac{2}{3 }& {k=1} 
   \end{cases}
   \\
p_{Y|X}\brak{0|0} = \frac{19}{25}\, 
p_{Y|X}\brak{0|1} = \frac{6}{25}\,
p_{Y|X}\brak{1|0} = \frac{45}{50}\,
p_{Y|X}\brak{1|2} = \frac{5}{50}
\end{align}
The desired probability is the probability that a slip drawn at random is marked other than Rs 1,
\begin{align}
&=1-p_X\brak{0}\\
&= p_X(1) + p_X(2)
\end{align}
Using Bayes theorem,
\begin{align}
&= p_Y\brak{0} \times \pr{Y=0 | X=1} + p_Y\brak{1} \times \pr{Y=1|X=2}\\
&=\frac{1}{3} \times \frac{6}{25} + \frac{2}{3} \times \frac{5}{50}\\
&=\frac{11}{75}
\end{align}

\newpage

%\tableofcontents

\bigskip

\renewcommand{\thefigure}{\theenumi}
\renewcommand{\thetable}{\theenumi}
%\renewcommand{\theequation}{\theenumi}

%\begin{abstract}
%%\boldmath
%In this letter, an algorithm for evaluating the exact analytical bit error rate  (BER)  for the piecewise linear (PL) combiner for  multiple relays is presented. Previous results were available only for upto three relays. The algorithm is unique in the sense that  the actual mathematical expressions, that are prohibitively large, need not be explicitly obtained. The diversity gain due to multiple relays is shown through plots of the analytical BER, well supported by simulations. 
%
%\end{abstract}
% IEEEtran.cls defaults to using nonbold math in the Abstract.
% This preserves the distinction between vectors and scalars. However,
% if the journal you are submitting to favors bold math in the abstract,
% then you can use LaTeX's standard command \boldmath at the very start
% of the abstract to achieve this. Many IEEE journals frown on math
% in the abstract anyway.

% Note that keywords are not normally used for peerreview papers.
%\begin{IEEEkeywords}
%Cooperative diversity, decode and forward, piecewise linear
%\end{IEEEkeywords}



% For peer review papers, you can put extra information on the cover
% page as needed:
% \ifCLASSOPTIONpeerreview
% \begin{center} \bfseries EDICS Category: 3-BBND \end{center}
% \fi
%
% For peerreview papers, this IEEEtran command inserts a page break and
% creates the second title. It will be ignored for other modes.
%\IEEEpeerreviewmaketitle




\item The probability that a student will pass his examination is 0.73, the probability of
the student getting a compartment is 0.13, and the probability that the student will
either pass or get compartment is 0.96. State True or False.\\
\solution
%\begin{table}[H]
	\centering
\begin{tabular}{|c|c|c|}
\hline
Random variable &Value &Definition\\ \hline
\multirow{3}{*}{X} &0 &Slips of Rs 1\\
&1 &Slips of Rs 5\\
&2 &Slips of Rs 13\\ \hline
\multirow{2}{*}{Y} &0 &Box A\\
&1 &Box B\\\hline
\end{tabular}
\caption{}
\label{tab:Distribution}
\end{table}
See \tabref{tab:Distribution}.
\begin{align}
p_{Y}\brak{k}= \begin{cases} 
      \frac{1}{3} & {k=0} \\
      \frac{2}{3 }& {k=1} 
   \end{cases}
   \\
p_{Y|X}\brak{0|0} = \frac{19}{25}\, 
p_{Y|X}\brak{0|1} = \frac{6}{25}\,
p_{Y|X}\brak{1|0} = \frac{45}{50}\,
p_{Y|X}\brak{1|2} = \frac{5}{50}
\end{align}
The desired probability is the probability that a slip drawn at random is marked other than Rs 1,
\begin{align}
&=1-p_X\brak{0}\\
&= p_X(1) + p_X(2)
\end{align}
Using Bayes theorem,
\begin{align}
&= p_Y\brak{0} \times \pr{Y=0 | X=1} + p_Y\brak{1} \times \pr{Y=1|X=2}\\
&=\frac{1}{3} \times \frac{6}{25} + \frac{2}{3} \times \frac{5}{50}\\
&=\frac{11}{75}
\end{align}

\newpage

%\tableofcontents

\bigskip

\renewcommand{\thefigure}{\theenumi}
\renewcommand{\thetable}{\theenumi}
%\renewcommand{\theequation}{\theenumi}

%\begin{abstract}
%%\boldmath
%In this letter, an algorithm for evaluating the exact analytical bit error rate  (BER)  for the piecewise linear (PL) combiner for  multiple relays is presented. Previous results were available only for upto three relays. The algorithm is unique in the sense that  the actual mathematical expressions, that are prohibitively large, need not be explicitly obtained. The diversity gain due to multiple relays is shown through plots of the analytical BER, well supported by simulations. 
%
%\end{abstract}
% IEEEtran.cls defaults to using nonbold math in the Abstract.
% This preserves the distinction between vectors and scalars. However,
% if the journal you are submitting to favors bold math in the abstract,
% then you can use LaTeX's standard command \boldmath at the very start
% of the abstract to achieve this. Many IEEE journals frown on math
% in the abstract anyway.

% Note that keywords are not normally used for peerreview papers.
%\begin{IEEEkeywords}
%Cooperative diversity, decode and forward, piecewise linear
%\end{IEEEkeywords}



% For peer review papers, you can put extra information on the cover
% page as needed:
% \ifCLASSOPTIONpeerreview
% \begin{center} \bfseries EDICS Category: 3-BBND \end{center}
% \fi
%
% For peerreview papers, this IEEEtran command inserts a page break and
% creates the second title. It will be ignored for other modes.
%\IEEEpeerreviewmaketitle




\item A card is selected from a pack of 52 cards\\
\begin{enumerate}[label=(\alph*)]
\item How many points are there in the sample space?
\item Calculate the probability that the cards is an ace of spades.
\item Calculate the probability that the card is (i) an ace (ii)black card.\\
\end{enumerate}
%\input{ncert/11/16/3/4_1/Prob_4.tex}
\item In a non-leap year, the probability of having 53 tuesdays or 53 wednesdays is\\
\solution
%A non-leap year has a total of 365 days, and a week has 7 days.\\
So it can be expressed as 
\begin{align}
365\text{days} &=52\times 7+1 \text{day}
\end{align}
$\implies$ 52 tuesdays or wednesdays\\
Random variable X denotes the days of a week
\begin{align}
p_X\brak{k}&=\frac{1}{7}; \quad \brak{1<k<7}
\end{align}
So the probability of extra day being tuesday or wednesday is
\begin{align}
p_X\brak{3}+p_X\brak{4}&=\frac{1}{7}+\frac{1}{7}=\frac{2}{7}
\end{align}



\item There are 1000 sealed envelopes in a box, 10 of them contain a cash prize of
Rs 100 each, 100 of them contain a cash prize of Rs 50 each and 200 of them
contain a cash prize of Rs 10 each and rest do not contain any cash prize. If they
are well shuffled and an envelope is picked up out, what is the probability that it
contains no cash prize?\\
\solution
%\begin{table}[H]
	\centering
\begin{tabular}{|c|c|c|}
\hline
Random variable &Value &Definition\\ \hline
\multirow{3}{*}{X} &0 &Slips of Rs 1\\
&1 &Slips of Rs 5\\
&2 &Slips of Rs 13\\ \hline
\multirow{2}{*}{Y} &0 &Box A\\
&1 &Box B\\\hline
\end{tabular}
\caption{}
\label{tab:Distribution}
\end{table}
See \tabref{tab:Distribution}.
\begin{align}
p_{Y}\brak{k}= \begin{cases} 
      \frac{1}{3} & {k=0} \\
      \frac{2}{3 }& {k=1} 
   \end{cases}
   \\
p_{Y|X}\brak{0|0} = \frac{19}{25}\, 
p_{Y|X}\brak{0|1} = \frac{6}{25}\,
p_{Y|X}\brak{1|0} = \frac{45}{50}\,
p_{Y|X}\brak{1|2} = \frac{5}{50}
\end{align}
The desired probability is the probability that a slip drawn at random is marked other than Rs 1,
\begin{align}
&=1-p_X\brak{0}\\
&= p_X(1) + p_X(2)
\end{align}
Using Bayes theorem,
\begin{align}
&= p_Y\brak{0} \times \pr{Y=0 | X=1} + p_Y\brak{1} \times \pr{Y=1|X=2}\\
&=\frac{1}{3} \times \frac{6}{25} + \frac{2}{3} \times \frac{5}{50}\\
&=\frac{11}{75}
\end{align}

\newpage

%\tableofcontents

\bigskip

\renewcommand{\thefigure}{\theenumi}
\renewcommand{\thetable}{\theenumi}
%\renewcommand{\theequation}{\theenumi}

%\begin{abstract}
%%\boldmath
%In this letter, an algorithm for evaluating the exact analytical bit error rate  (BER)  for the piecewise linear (PL) combiner for  multiple relays is presented. Previous results were available only for upto three relays. The algorithm is unique in the sense that  the actual mathematical expressions, that are prohibitively large, need not be explicitly obtained. The diversity gain due to multiple relays is shown through plots of the analytical BER, well supported by simulations. 
%
%\end{abstract}
% IEEEtran.cls defaults to using nonbold math in the Abstract.
% This preserves the distinction between vectors and scalars. However,
% if the journal you are submitting to favors bold math in the abstract,
% then you can use LaTeX's standard command \boldmath at the very start
% of the abstract to achieve this. Many IEEE journals frown on math
% in the abstract anyway.

% Note that keywords are not normally used for peerreview papers.
%\begin{IEEEkeywords}
%Cooperative diversity, decode and forward, piecewise linear
%\end{IEEEkeywords}



% For peer review papers, you can put extra information on the cover
% page as needed:
% \ifCLASSOPTIONpeerreview
% \begin{center} \bfseries EDICS Category: 3-BBND \end{center}
% \fi
%
% For peerreview papers, this IEEEtran command inserts a page break and
% creates the second title. It will be ignored for other modes.
%\IEEEpeerreviewmaketitle




\item 
A die is thrown and a card is selected at random from a deck of 52 playing cards. The probability of getting an even number on the die and a spade card.\\
\solution
%\begin{table}[H]
	\centering
\begin{tabular}{|c|c|c|}
\hline
Random variable &Value &Definition\\ \hline
\multirow{3}{*}{X} &0 &Slips of Rs 1\\
&1 &Slips of Rs 5\\
&2 &Slips of Rs 13\\ \hline
\multirow{2}{*}{Y} &0 &Box A\\
&1 &Box B\\\hline
\end{tabular}
\caption{}
\label{tab:Distribution}
\end{table}
See \tabref{tab:Distribution}.
\begin{align}
p_{Y}\brak{k}= \begin{cases} 
      \frac{1}{3} & {k=0} \\
      \frac{2}{3 }& {k=1} 
   \end{cases}
   \\
p_{Y|X}\brak{0|0} = \frac{19}{25}\, 
p_{Y|X}\brak{0|1} = \frac{6}{25}\,
p_{Y|X}\brak{1|0} = \frac{45}{50}\,
p_{Y|X}\brak{1|2} = \frac{5}{50}
\end{align}
The desired probability is the probability that a slip drawn at random is marked other than Rs 1,
\begin{align}
&=1-p_X\brak{0}\\
&= p_X(1) + p_X(2)
\end{align}
Using Bayes theorem,
\begin{align}
&= p_Y\brak{0} \times \pr{Y=0 | X=1} + p_Y\brak{1} \times \pr{Y=1|X=2}\\
&=\frac{1}{3} \times \frac{6}{25} + \frac{2}{3} \times \frac{5}{50}\\
&=\frac{11}{75}
\end{align}

\newpage

%\tableofcontents

\bigskip

\renewcommand{\thefigure}{\theenumi}
\renewcommand{\thetable}{\theenumi}
%\renewcommand{\theequation}{\theenumi}

%\begin{abstract}
%%\boldmath
%In this letter, an algorithm for evaluating the exact analytical bit error rate  (BER)  for the piecewise linear (PL) combiner for  multiple relays is presented. Previous results were available only for upto three relays. The algorithm is unique in the sense that  the actual mathematical expressions, that are prohibitively large, need not be explicitly obtained. The diversity gain due to multiple relays is shown through plots of the analytical BER, well supported by simulations. 
%
%\end{abstract}
% IEEEtran.cls defaults to using nonbold math in the Abstract.
% This preserves the distinction between vectors and scalars. However,
% if the journal you are submitting to favors bold math in the abstract,
% then you can use LaTeX's standard command \boldmath at the very start
% of the abstract to achieve this. Many IEEE journals frown on math
% in the abstract anyway.

% Note that keywords are not normally used for peerreview papers.
%\begin{IEEEkeywords}
%Cooperative diversity, decode and forward, piecewise linear
%\end{IEEEkeywords}



% For peer review papers, you can put extra information on the cover
% page as needed:
% \ifCLASSOPTIONpeerreview
% \begin{center} \bfseries EDICS Category: 3-BBND \end{center}
% \fi
%
% For peerreview papers, this IEEEtran command inserts a page break and
% creates the second title. It will be ignored for other modes.
%\IEEEpeerreviewmaketitle




\item
If 4-digit numbers greater than 5,000 are randomly formed from the digits 0, 1, 3, 5, and 7, what is the probability of forming a number divisible by 5 when:
\begin{enumerate}
    \item The digits are repeated?
    \item The repetition of digits is not allowed?
\end{enumerate}
\solution
%\begin{table}[H]
	\centering
\begin{tabular}{|c|c|c|}
\hline
Random variable &Value &Definition\\ \hline
\multirow{3}{*}{X} &0 &Slips of Rs 1\\
&1 &Slips of Rs 5\\
&2 &Slips of Rs 13\\ \hline
\multirow{2}{*}{Y} &0 &Box A\\
&1 &Box B\\\hline
\end{tabular}
\caption{}
\label{tab:Distribution}
\end{table}
See \tabref{tab:Distribution}.
\begin{align}
p_{Y}\brak{k}= \begin{cases} 
      \frac{1}{3} & {k=0} \\
      \frac{2}{3 }& {k=1} 
   \end{cases}
   \\
p_{Y|X}\brak{0|0} = \frac{19}{25}\, 
p_{Y|X}\brak{0|1} = \frac{6}{25}\,
p_{Y|X}\brak{1|0} = \frac{45}{50}\,
p_{Y|X}\brak{1|2} = \frac{5}{50}
\end{align}
The desired probability is the probability that a slip drawn at random is marked other than Rs 1,
\begin{align}
&=1-p_X\brak{0}\\
&= p_X(1) + p_X(2)
\end{align}
Using Bayes theorem,
\begin{align}
&= p_Y\brak{0} \times \pr{Y=0 | X=1} + p_Y\brak{1} \times \pr{Y=1|X=2}\\
&=\frac{1}{3} \times \frac{6}{25} + \frac{2}{3} \times \frac{5}{50}\\
&=\frac{11}{75}
\end{align}

\newpage

%\tableofcontents

\bigskip

\renewcommand{\thefigure}{\theenumi}
\renewcommand{\thetable}{\theenumi}
%\renewcommand{\theequation}{\theenumi}

%\begin{abstract}
%%\boldmath
%In this letter, an algorithm for evaluating the exact analytical bit error rate  (BER)  for the piecewise linear (PL) combiner for  multiple relays is presented. Previous results were available only for upto three relays. The algorithm is unique in the sense that  the actual mathematical expressions, that are prohibitively large, need not be explicitly obtained. The diversity gain due to multiple relays is shown through plots of the analytical BER, well supported by simulations. 
%
%\end{abstract}
% IEEEtran.cls defaults to using nonbold math in the Abstract.
% This preserves the distinction between vectors and scalars. However,
% if the journal you are submitting to favors bold math in the abstract,
% then you can use LaTeX's standard command \boldmath at the very start
% of the abstract to achieve this. Many IEEE journals frown on math
% in the abstract anyway.

% Note that keywords are not normally used for peerreview papers.
%\begin{IEEEkeywords}
%Cooperative diversity, decode and forward, piecewise linear
%\end{IEEEkeywords}



% For peer review papers, you can put extra information on the cover
% page as needed:
% \ifCLASSOPTIONpeerreview
% \begin{center} \bfseries EDICS Category: 3-BBND \end{center}
% \fi
%
% For peerreview papers, this IEEEtran command inserts a page break and
% creates the second title. It will be ignored for other modes.
%\IEEEpeerreviewmaketitle




\item Consider the probability space $\brak{\Omega, \mathcal{G}, P}$ where $\Omega = [0,2]$ and $\mathcal{G} = \cbrak{\phi, \Omega, [0,1], (1,2]}$. Let $X$ and $Y$ be two functions on $\Omega$ defined as
\begin{align*}
    X(\omega) = 
    \begin{cases}
        1 & \text{if }\omega \in [0, 1]\\
        2 & \text{if }\omega \in (1, 2]
    \end{cases}
\end{align*}
and
\begin{align*}
    Y(\omega) = 
    \begin{cases}
        2 & \text{if }\omega \in [0, 1.5]\\
        3 & \text{if }\omega \in (1.5, 2].
    \end{cases}
\end{align*}
Then which one of the following statements is true?
\begin{enumerate}
    \item [(A)] $X$ is a random variable with respect to $\mathcal{G}$, but $Y$ is not a random variable with respect to $\mathcal{G}$.
    \item [(B)] $Y$ is a random variable with respect to $\mathcal{G}$, but $X$ is not a random variable with respect to $\mathcal{G}$.
    \item [(C)] Neither $X$ nor $Y$ is a random variable with respect to $\mathcal{G}$.
    \item [(D)] Both $X$ and $Y$ are random variables with respect to $\mathcal{G}$.
\end{enumerate} \hfill (GATE ST 2023)\\
\solution
%\begin{table}[H]
	\centering
\begin{tabular}{|c|c|c|}
\hline
Random variable &Value &Definition\\ \hline
\multirow{3}{*}{X} &0 &Slips of Rs 1\\
&1 &Slips of Rs 5\\
&2 &Slips of Rs 13\\ \hline
\multirow{2}{*}{Y} &0 &Box A\\
&1 &Box B\\\hline
\end{tabular}
\caption{}
\label{tab:Distribution}
\end{table}
See \tabref{tab:Distribution}.
\begin{align}
p_{Y}\brak{k}= \begin{cases} 
      \frac{1}{3} & {k=0} \\
      \frac{2}{3 }& {k=1} 
   \end{cases}
   \\
p_{Y|X}\brak{0|0} = \frac{19}{25}\, 
p_{Y|X}\brak{0|1} = \frac{6}{25}\,
p_{Y|X}\brak{1|0} = \frac{45}{50}\,
p_{Y|X}\brak{1|2} = \frac{5}{50}
\end{align}
The desired probability is the probability that a slip drawn at random is marked other than Rs 1,
\begin{align}
&=1-p_X\brak{0}\\
&= p_X(1) + p_X(2)
\end{align}
Using Bayes theorem,
\begin{align}
&= p_Y\brak{0} \times \pr{Y=0 | X=1} + p_Y\brak{1} \times \pr{Y=1|X=2}\\
&=\frac{1}{3} \times \frac{6}{25} + \frac{2}{3} \times \frac{5}{50}\\
&=\frac{11}{75}
\end{align}

\newpage

%\tableofcontents

\bigskip

\renewcommand{\thefigure}{\theenumi}
\renewcommand{\thetable}{\theenumi}
%\renewcommand{\theequation}{\theenumi}

%\begin{abstract}
%%\boldmath
%In this letter, an algorithm for evaluating the exact analytical bit error rate  (BER)  for the piecewise linear (PL) combiner for  multiple relays is presented. Previous results were available only for upto three relays. The algorithm is unique in the sense that  the actual mathematical expressions, that are prohibitively large, need not be explicitly obtained. The diversity gain due to multiple relays is shown through plots of the analytical BER, well supported by simulations. 
%
%\end{abstract}
% IEEEtran.cls defaults to using nonbold math in the Abstract.
% This preserves the distinction between vectors and scalars. However,
% if the journal you are submitting to favors bold math in the abstract,
% then you can use LaTeX's standard command \boldmath at the very start
% of the abstract to achieve this. Many IEEE journals frown on math
% in the abstract anyway.

% Note that keywords are not normally used for peerreview papers.
%\begin{IEEEkeywords}
%Cooperative diversity, decode and forward, piecewise linear
%\end{IEEEkeywords}



% For peer review papers, you can put extra information on the cover
% page as needed:
% \ifCLASSOPTIONpeerreview
% \begin{center} \bfseries EDICS Category: 3-BBND \end{center}
% \fi
%
% For peerreview papers, this IEEEtran command inserts a page break and
% creates the second title. It will be ignored for other modes.
%\IEEEpeerreviewmaketitle




	\item  A die is loaded in such a way that each odd number is twice as likely to occur as
each even number. Find $P(G)$, where $G$ is the event that a number greater than
3 occurs on a single roll of the die.
\\
\solution
		%\begin{table}[H]
	\centering
\begin{tabular}{|c|c|c|}
\hline
Random variable &Value &Definition\\ \hline
\multirow{3}{*}{X} &0 &Slips of Rs 1\\
&1 &Slips of Rs 5\\
&2 &Slips of Rs 13\\ \hline
\multirow{2}{*}{Y} &0 &Box A\\
&1 &Box B\\\hline
\end{tabular}
\caption{}
\label{tab:Distribution}
\end{table}
See \tabref{tab:Distribution}.
\begin{align}
p_{Y}\brak{k}= \begin{cases} 
      \frac{1}{3} & {k=0} \\
      \frac{2}{3 }& {k=1} 
   \end{cases}
   \\
p_{Y|X}\brak{0|0} = \frac{19}{25}\, 
p_{Y|X}\brak{0|1} = \frac{6}{25}\,
p_{Y|X}\brak{1|0} = \frac{45}{50}\,
p_{Y|X}\brak{1|2} = \frac{5}{50}
\end{align}
The desired probability is the probability that a slip drawn at random is marked other than Rs 1,
\begin{align}
&=1-p_X\brak{0}\\
&= p_X(1) + p_X(2)
\end{align}
Using Bayes theorem,
\begin{align}
&= p_Y\brak{0} \times \pr{Y=0 | X=1} + p_Y\brak{1} \times \pr{Y=1|X=2}\\
&=\frac{1}{3} \times \frac{6}{25} + \frac{2}{3} \times \frac{5}{50}\\
&=\frac{11}{75}
\end{align}

\newpage

%\tableofcontents

\bigskip

\renewcommand{\thefigure}{\theenumi}
\renewcommand{\thetable}{\theenumi}
%\renewcommand{\theequation}{\theenumi}

%\begin{abstract}
%%\boldmath
%In this letter, an algorithm for evaluating the exact analytical bit error rate  (BER)  for the piecewise linear (PL) combiner for  multiple relays is presented. Previous results were available only for upto three relays. The algorithm is unique in the sense that  the actual mathematical expressions, that are prohibitively large, need not be explicitly obtained. The diversity gain due to multiple relays is shown through plots of the analytical BER, well supported by simulations. 
%
%\end{abstract}
% IEEEtran.cls defaults to using nonbold math in the Abstract.
% This preserves the distinction between vectors and scalars. However,
% if the journal you are submitting to favors bold math in the abstract,
% then you can use LaTeX's standard command \boldmath at the very start
% of the abstract to achieve this. Many IEEE journals frown on math
% in the abstract anyway.

% Note that keywords are not normally used for peerreview papers.
%\begin{IEEEkeywords}
%Cooperative diversity, decode and forward, piecewise linear
%\end{IEEEkeywords}



% For peer review papers, you can put extra information on the cover
% page as needed:
% \ifCLASSOPTIONpeerreview
% \begin{center} \bfseries EDICS Category: 3-BBND \end{center}
% \fi
%
% For peerreview papers, this IEEEtran command inserts a page break and
% creates the second title. It will be ignored for other modes.
%\IEEEpeerreviewmaketitle




	\item All the jacks, queens and kings are removed from a deck of 52 playing cards. The remaining cards are well shuffled and then one card is drawn at random. Giving ace a value 1 similar value for other cards, find the probability that the card has a value 
		\begin{enumerate}
			\item 7
			\item greater than 7
			\item less than 7
		\end{enumerate}
		%Number of cards left after removing all jacks, queens and kings 
\begin{align}
N	= 52 - 4\times 3
	= 40
\end{align}
%\begin{table}[H]
%\def\arraystretch{1.2}
%\begin{tabular}{|c|c|c|}
%\hline
%	\textbf{Parameter} &\textbf{Value} &\textbf{Description}\\ \hline
%	$X$ &1-10 &Represents the value of the card picked \\ \hline
%\end{tabular}
%\end{table}
Let $1 \le X \le 10$ be the value of the card picked.  Then,
\begin{align}
	p_X(k) &= \Pr(X=k)\ \forall\ 1 \leq k \leq 10\\
	&= \frac{4\times 1}{40}\\
	&= \frac{1}{10}\\
	\therefore p_X(k) &= 
	\begin{cases}
		\frac{1}{10} & 1 \leq k \leq 10\\
		0 & \text{otherwise}
	\end{cases}
\end{align}
and
\begin{align}
	F_{X}(k) &= \sum_{m=0}^{k}p_{X}(m) \quad 1 \leq k \leq 10\\
	&= \frac{k}{10}\\
	\therefore F_{X}(k) &= 
	\begin{cases}
		0 & k \leq 0\\
		\frac{k}{10} & 1\leq k \leq 10\\
		1 & k > 10 
	\end{cases}
\end{align}
\begin{enumerate}
	\item Probability that card has value equal to 7 is
		\begin{align}
			 p_{X}(7)
			= \frac{1}{10}
		\end{align}
	\item Probability that card has value greater than 7 is
		\begin{align}
			1 - F_X(7)
			&= 1 - \frac{7}{10}
			\\
			&= \frac{3}{10}
		\end{align}
	\item Probability that card has value less than 7 is
		\begin{align}
			 F_{X}(6)
			=\frac{6}{10}
		\end{align}
\end{enumerate}

  \item A Lot consists of 48 mobile phones of which 42 are good, 3 have only minor defects and 3 have major defects.Varnika will buy a phone if it is good but the trader will only buy a mobile if it has no major defects. One phone is selected at random from the lot. What is the probability that it is
\begin{enumerate}
	\item acceptable to Varnika?
            \item acceptable to the trader?
\end{enumerate}
\solution
	%\begin{table}[H]
	\centering
\begin{tabular}{|c|c|c|}
\hline
Random variable &Value &Definition\\ \hline
\multirow{3}{*}{X} &0 &Slips of Rs 1\\
&1 &Slips of Rs 5\\
&2 &Slips of Rs 13\\ \hline
\multirow{2}{*}{Y} &0 &Box A\\
&1 &Box B\\\hline
\end{tabular}
\caption{}
\label{tab:Distribution}
\end{table}
See \tabref{tab:Distribution}.
\begin{align}
p_{Y}\brak{k}= \begin{cases} 
      \frac{1}{3} & {k=0} \\
      \frac{2}{3 }& {k=1} 
   \end{cases}
   \\
p_{Y|X}\brak{0|0} = \frac{19}{25}\, 
p_{Y|X}\brak{0|1} = \frac{6}{25}\,
p_{Y|X}\brak{1|0} = \frac{45}{50}\,
p_{Y|X}\brak{1|2} = \frac{5}{50}
\end{align}
The desired probability is the probability that a slip drawn at random is marked other than Rs 1,
\begin{align}
&=1-p_X\brak{0}\\
&= p_X(1) + p_X(2)
\end{align}
Using Bayes theorem,
\begin{align}
&= p_Y\brak{0} \times \pr{Y=0 | X=1} + p_Y\brak{1} \times \pr{Y=1|X=2}\\
&=\frac{1}{3} \times \frac{6}{25} + \frac{2}{3} \times \frac{5}{50}\\
&=\frac{11}{75}
\end{align}

\newpage

%\tableofcontents

\bigskip

\renewcommand{\thefigure}{\theenumi}
\renewcommand{\thetable}{\theenumi}
%\renewcommand{\theequation}{\theenumi}

%\begin{abstract}
%%\boldmath
%In this letter, an algorithm for evaluating the exact analytical bit error rate  (BER)  for the piecewise linear (PL) combiner for  multiple relays is presented. Previous results were available only for upto three relays. The algorithm is unique in the sense that  the actual mathematical expressions, that are prohibitively large, need not be explicitly obtained. The diversity gain due to multiple relays is shown through plots of the analytical BER, well supported by simulations. 
%
%\end{abstract}
% IEEEtran.cls defaults to using nonbold math in the Abstract.
% This preserves the distinction between vectors and scalars. However,
% if the journal you are submitting to favors bold math in the abstract,
% then you can use LaTeX's standard command \boldmath at the very start
% of the abstract to achieve this. Many IEEE journals frown on math
% in the abstract anyway.

% Note that keywords are not normally used for peerreview papers.
%\begin{IEEEkeywords}
%Cooperative diversity, decode and forward, piecewise linear
%\end{IEEEkeywords}



% For peer review papers, you can put extra information on the cover
% page as needed:
% \ifCLASSOPTIONpeerreview
% \begin{center} \bfseries EDICS Category: 3-BBND \end{center}
% \fi
%
% For peerreview papers, this IEEEtran command inserts a page break and
% creates the second title. It will be ignored for other modes.
%\IEEEpeerreviewmaketitle




 \item A student says that if you throw a die, it will show up 1 or not 1. Therefore, the probability of getting 1 and the probability of getting 'not 1' each is equal to $\frac{1}{2}$. Is this correct? Give reasons.\\
 \solution
        %\begin{table}[H]
	\centering
\begin{tabular}{|c|c|c|}
\hline
Random variable &Value &Definition\\ \hline
\multirow{3}{*}{X} &0 &Slips of Rs 1\\
&1 &Slips of Rs 5\\
&2 &Slips of Rs 13\\ \hline
\multirow{2}{*}{Y} &0 &Box A\\
&1 &Box B\\\hline
\end{tabular}
\caption{}
\label{tab:Distribution}
\end{table}
See \tabref{tab:Distribution}.
\begin{align}
p_{Y}\brak{k}= \begin{cases} 
      \frac{1}{3} & {k=0} \\
      \frac{2}{3 }& {k=1} 
   \end{cases}
   \\
p_{Y|X}\brak{0|0} = \frac{19}{25}\, 
p_{Y|X}\brak{0|1} = \frac{6}{25}\,
p_{Y|X}\brak{1|0} = \frac{45}{50}\,
p_{Y|X}\brak{1|2} = \frac{5}{50}
\end{align}
The desired probability is the probability that a slip drawn at random is marked other than Rs 1,
\begin{align}
&=1-p_X\brak{0}\\
&= p_X(1) + p_X(2)
\end{align}
Using Bayes theorem,
\begin{align}
&= p_Y\brak{0} \times \pr{Y=0 | X=1} + p_Y\brak{1} \times \pr{Y=1|X=2}\\
&=\frac{1}{3} \times \frac{6}{25} + \frac{2}{3} \times \frac{5}{50}\\
&=\frac{11}{75}
\end{align}

\newpage

%\tableofcontents

\bigskip

\renewcommand{\thefigure}{\theenumi}
\renewcommand{\thetable}{\theenumi}
%\renewcommand{\theequation}{\theenumi}

%\begin{abstract}
%%\boldmath
%In this letter, an algorithm for evaluating the exact analytical bit error rate  (BER)  for the piecewise linear (PL) combiner for  multiple relays is presented. Previous results were available only for upto three relays. The algorithm is unique in the sense that  the actual mathematical expressions, that are prohibitively large, need not be explicitly obtained. The diversity gain due to multiple relays is shown through plots of the analytical BER, well supported by simulations. 
%
%\end{abstract}
% IEEEtran.cls defaults to using nonbold math in the Abstract.
% This preserves the distinction between vectors and scalars. However,
% if the journal you are submitting to favors bold math in the abstract,
% then you can use LaTeX's standard command \boldmath at the very start
% of the abstract to achieve this. Many IEEE journals frown on math
% in the abstract anyway.

% Note that keywords are not normally used for peerreview papers.
%\begin{IEEEkeywords}
%Cooperative diversity, decode and forward, piecewise linear
%\end{IEEEkeywords}



% For peer review papers, you can put extra information on the cover
% page as needed:
% \ifCLASSOPTIONpeerreview
% \begin{center} \bfseries EDICS Category: 3-BBND \end{center}
% \fi
%
% For peerreview papers, this IEEEtran command inserts a page break and
% creates the second title. It will be ignored for other modes.
%\IEEEpeerreviewmaketitle




   \item Four candidates A, B, C, D have ap-
plied for the assignment to coach a school cricket
team. If A is twice as likely to be selected as B, and
B and C are given about the same chance of being
selected, while C is twice as likely to be selected
as D, what are the probabilities that
\begin{enumerate}
\item C will be selected?
\item A will not be selected?
\end{enumerate}
	%\begin{table}[H]
	\centering
\begin{tabular}{|c|c|c|}
\hline
Random variable &Value &Definition\\ \hline
\multirow{3}{*}{X} &0 &Slips of Rs 1\\
&1 &Slips of Rs 5\\
&2 &Slips of Rs 13\\ \hline
\multirow{2}{*}{Y} &0 &Box A\\
&1 &Box B\\\hline
\end{tabular}
\caption{}
\label{tab:Distribution}
\end{table}
See \tabref{tab:Distribution}.
\begin{align}
p_{Y}\brak{k}= \begin{cases} 
      \frac{1}{3} & {k=0} \\
      \frac{2}{3 }& {k=1} 
   \end{cases}
   \\
p_{Y|X}\brak{0|0} = \frac{19}{25}\, 
p_{Y|X}\brak{0|1} = \frac{6}{25}\,
p_{Y|X}\brak{1|0} = \frac{45}{50}\,
p_{Y|X}\brak{1|2} = \frac{5}{50}
\end{align}
The desired probability is the probability that a slip drawn at random is marked other than Rs 1,
\begin{align}
&=1-p_X\brak{0}\\
&= p_X(1) + p_X(2)
\end{align}
Using Bayes theorem,
\begin{align}
&= p_Y\brak{0} \times \pr{Y=0 | X=1} + p_Y\brak{1} \times \pr{Y=1|X=2}\\
&=\frac{1}{3} \times \frac{6}{25} + \frac{2}{3} \times \frac{5}{50}\\
&=\frac{11}{75}
\end{align}

\newpage

%\tableofcontents

\bigskip

\renewcommand{\thefigure}{\theenumi}
\renewcommand{\thetable}{\theenumi}
%\renewcommand{\theequation}{\theenumi}

%\begin{abstract}
%%\boldmath
%In this letter, an algorithm for evaluating the exact analytical bit error rate  (BER)  for the piecewise linear (PL) combiner for  multiple relays is presented. Previous results were available only for upto three relays. The algorithm is unique in the sense that  the actual mathematical expressions, that are prohibitively large, need not be explicitly obtained. The diversity gain due to multiple relays is shown through plots of the analytical BER, well supported by simulations. 
%
%\end{abstract}
% IEEEtran.cls defaults to using nonbold math in the Abstract.
% This preserves the distinction between vectors and scalars. However,
% if the journal you are submitting to favors bold math in the abstract,
% then you can use LaTeX's standard command \boldmath at the very start
% of the abstract to achieve this. Many IEEE journals frown on math
% in the abstract anyway.

% Note that keywords are not normally used for peerreview papers.
%\begin{IEEEkeywords}
%Cooperative diversity, decode and forward, piecewise linear
%\end{IEEEkeywords}



% For peer review papers, you can put extra information on the cover
% page as needed:
% \ifCLASSOPTIONpeerreview
% \begin{center} \bfseries EDICS Category: 3-BBND \end{center}
% \fi
%
% For peerreview papers, this IEEEtran command inserts a page break and
% creates the second title. It will be ignored for other modes.
%\IEEEpeerreviewmaketitle




 \item A bag contain 24 balls of which $x$ balls are red, $2x$ are white and $3x$ are blue. A ball is selected at random, What is the probability that it is
\begin{enumerate}[label=\alph*)]
\item not red ?
\item white ?
\end{enumerate}
%\begin{table}[H]
	\centering
\begin{tabular}{|c|c|c|}
\hline
Random variable &Value &Definition\\ \hline
\multirow{3}{*}{X} &0 &Slips of Rs 1\\
&1 &Slips of Rs 5\\
&2 &Slips of Rs 13\\ \hline
\multirow{2}{*}{Y} &0 &Box A\\
&1 &Box B\\\hline
\end{tabular}
\caption{}
\label{tab:Distribution}
\end{table}
See \tabref{tab:Distribution}.
\begin{align}
p_{Y}\brak{k}= \begin{cases} 
      \frac{1}{3} & {k=0} \\
      \frac{2}{3 }& {k=1} 
   \end{cases}
   \\
p_{Y|X}\brak{0|0} = \frac{19}{25}\, 
p_{Y|X}\brak{0|1} = \frac{6}{25}\,
p_{Y|X}\brak{1|0} = \frac{45}{50}\,
p_{Y|X}\brak{1|2} = \frac{5}{50}
\end{align}
The desired probability is the probability that a slip drawn at random is marked other than Rs 1,
\begin{align}
&=1-p_X\brak{0}\\
&= p_X(1) + p_X(2)
\end{align}
Using Bayes theorem,
\begin{align}
&= p_Y\brak{0} \times \pr{Y=0 | X=1} + p_Y\brak{1} \times \pr{Y=1|X=2}\\
&=\frac{1}{3} \times \frac{6}{25} + \frac{2}{3} \times \frac{5}{50}\\
&=\frac{11}{75}
\end{align}

\newpage

%\tableofcontents

\bigskip

\renewcommand{\thefigure}{\theenumi}
\renewcommand{\thetable}{\theenumi}
%\renewcommand{\theequation}{\theenumi}

%\begin{abstract}
%%\boldmath
%In this letter, an algorithm for evaluating the exact analytical bit error rate  (BER)  for the piecewise linear (PL) combiner for  multiple relays is presented. Previous results were available only for upto three relays. The algorithm is unique in the sense that  the actual mathematical expressions, that are prohibitively large, need not be explicitly obtained. The diversity gain due to multiple relays is shown through plots of the analytical BER, well supported by simulations. 
%
%\end{abstract}
% IEEEtran.cls defaults to using nonbold math in the Abstract.
% This preserves the distinction between vectors and scalars. However,
% if the journal you are submitting to favors bold math in the abstract,
% then you can use LaTeX's standard command \boldmath at the very start
% of the abstract to achieve this. Many IEEE journals frown on math
% in the abstract anyway.

% Note that keywords are not normally used for peerreview papers.
%\begin{IEEEkeywords}
%Cooperative diversity, decode and forward, piecewise linear
%\end{IEEEkeywords}



% For peer review papers, you can put extra information on the cover
% page as needed:
% \ifCLASSOPTIONpeerreview
% \begin{center} \bfseries EDICS Category: 3-BBND \end{center}
% \fi
%
% For peerreview papers, this IEEEtran command inserts a page break and
% creates the second title. It will be ignored for other modes.
%\IEEEpeerreviewmaketitle




If the letters of the word ASSASSINATION are arranged at random. Find the Probability that
\begin{enumerate}[label=(\alph*)]
\item Four $S's$ come consecutively in the word
\item Two  $I's$ and two $N's$ come together
\item All $A's$ are not coming together
\item No two $A's$ are coming together
\end{enumerate}
%\begin{table}[H]
	\centering
\begin{tabular}{|c|c|c|}
\hline
Random variable &Value &Definition\\ \hline
\multirow{3}{*}{X} &0 &Slips of Rs 1\\
&1 &Slips of Rs 5\\
&2 &Slips of Rs 13\\ \hline
\multirow{2}{*}{Y} &0 &Box A\\
&1 &Box B\\\hline
\end{tabular}
\caption{}
\label{tab:Distribution}
\end{table}
See \tabref{tab:Distribution}.
\begin{align}
p_{Y}\brak{k}= \begin{cases} 
      \frac{1}{3} & {k=0} \\
      \frac{2}{3 }& {k=1} 
   \end{cases}
   \\
p_{Y|X}\brak{0|0} = \frac{19}{25}\, 
p_{Y|X}\brak{0|1} = \frac{6}{25}\,
p_{Y|X}\brak{1|0} = \frac{45}{50}\,
p_{Y|X}\brak{1|2} = \frac{5}{50}
\end{align}
The desired probability is the probability that a slip drawn at random is marked other than Rs 1,
\begin{align}
&=1-p_X\brak{0}\\
&= p_X(1) + p_X(2)
\end{align}
Using Bayes theorem,
\begin{align}
&= p_Y\brak{0} \times \pr{Y=0 | X=1} + p_Y\brak{1} \times \pr{Y=1|X=2}\\
&=\frac{1}{3} \times \frac{6}{25} + \frac{2}{3} \times \frac{5}{50}\\
&=\frac{11}{75}
\end{align}

\newpage

%\tableofcontents

\bigskip

\renewcommand{\thefigure}{\theenumi}
\renewcommand{\thetable}{\theenumi}
%\renewcommand{\theequation}{\theenumi}

%\begin{abstract}
%%\boldmath
%In this letter, an algorithm for evaluating the exact analytical bit error rate  (BER)  for the piecewise linear (PL) combiner for  multiple relays is presented. Previous results were available only for upto three relays. The algorithm is unique in the sense that  the actual mathematical expressions, that are prohibitively large, need not be explicitly obtained. The diversity gain due to multiple relays is shown through plots of the analytical BER, well supported by simulations. 
%
%\end{abstract}
% IEEEtran.cls defaults to using nonbold math in the Abstract.
% This preserves the distinction between vectors and scalars. However,
% if the journal you are submitting to favors bold math in the abstract,
% then you can use LaTeX's standard command \boldmath at the very start
% of the abstract to achieve this. Many IEEE journals frown on math
% in the abstract anyway.

% Note that keywords are not normally used for peerreview papers.
%\begin{IEEEkeywords}
%Cooperative diversity, decode and forward, piecewise linear
%\end{IEEEkeywords}



% For peer review papers, you can put extra information on the cover
% page as needed:
% \ifCLASSOPTIONpeerreview
% \begin{center} \bfseries EDICS Category: 3-BBND \end{center}
% \fi
%
% For peerreview papers, this IEEEtran command inserts a page break and
% creates the second title. It will be ignored for other modes.
%\IEEEpeerreviewmaketitle




	\item One urn contains two black balls (labelled B1 and B2) and one white ball. A
	second urn contains one black ball and two white balls (labelled W1 and W2).
	Suppose the following experiment is performed. One of the two urns is chosen
	at random. Next a ball is randomly chosen from the urn. Then a second ball is
	chosen at random from the same urn without replacing the first ball.
	
	\begin{enumerate}
	\item What is the probability that two black balls are chosen?
	
	\item What is the probability that two balls of opposite colour are chosen?
	\end{enumerate}
	\solution
	%\begin{align}
    \label{eq:12.13.6.18.1}
	\because	\pr{A|B} &> \pr{A},\
\frac{\pr{AB}}{\pr{B}} > \pr{A}
\\
    \label{eq:12.13.6.18.2}
	\implies \pr{AB} &> \pr{A}\pr{B}
	\\
	\text{or, } \frac{\pr{AB}}{\pr{A}} &=\pr{B|A} > \pr{A}
\end{align}

\end{enumerate}

	\item A bag contains 4 red and 4 black balls, another bag contains 2 red and 6 black balls. One of the two bags is selected at random and a ball is drawn from the bag which is found to be red. Find the probability that the ball is drawn from the first bag.
\\
\solution
		%\begin{table}[H]
	\centering
\begin{tabular}{|c|c|c|}
\hline
Random variable &Value &Definition\\ \hline
\multirow{3}{*}{X} &0 &Slips of Rs 1\\
&1 &Slips of Rs 5\\
&2 &Slips of Rs 13\\ \hline
\multirow{2}{*}{Y} &0 &Box A\\
&1 &Box B\\\hline
\end{tabular}
\caption{}
\label{tab:Distribution}
\end{table}
See \tabref{tab:Distribution}.
\begin{align}
p_{Y}\brak{k}= \begin{cases} 
      \frac{1}{3} & {k=0} \\
      \frac{2}{3 }& {k=1} 
   \end{cases}
   \\
p_{Y|X}\brak{0|0} = \frac{19}{25}\, 
p_{Y|X}\brak{0|1} = \frac{6}{25}\,
p_{Y|X}\brak{1|0} = \frac{45}{50}\,
p_{Y|X}\brak{1|2} = \frac{5}{50}
\end{align}
The desired probability is the probability that a slip drawn at random is marked other than Rs 1,
\begin{align}
&=1-p_X\brak{0}\\
&= p_X(1) + p_X(2)
\end{align}
Using Bayes theorem,
\begin{align}
&= p_Y\brak{0} \times \pr{Y=0 | X=1} + p_Y\brak{1} \times \pr{Y=1|X=2}\\
&=\frac{1}{3} \times \frac{6}{25} + \frac{2}{3} \times \frac{5}{50}\\
&=\frac{11}{75}
\end{align}

\newpage

%\tableofcontents

\bigskip

\renewcommand{\thefigure}{\theenumi}
\renewcommand{\thetable}{\theenumi}
%\renewcommand{\theequation}{\theenumi}

%\begin{abstract}
%%\boldmath
%In this letter, an algorithm for evaluating the exact analytical bit error rate  (BER)  for the piecewise linear (PL) combiner for  multiple relays is presented. Previous results were available only for upto three relays. The algorithm is unique in the sense that  the actual mathematical expressions, that are prohibitively large, need not be explicitly obtained. The diversity gain due to multiple relays is shown through plots of the analytical BER, well supported by simulations. 
%
%\end{abstract}
% IEEEtran.cls defaults to using nonbold math in the Abstract.
% This preserves the distinction between vectors and scalars. However,
% if the journal you are submitting to favors bold math in the abstract,
% then you can use LaTeX's standard command \boldmath at the very start
% of the abstract to achieve this. Many IEEE journals frown on math
% in the abstract anyway.

% Note that keywords are not normally used for peerreview papers.
%\begin{IEEEkeywords}
%Cooperative diversity, decode and forward, piecewise linear
%\end{IEEEkeywords}



% For peer review papers, you can put extra information on the cover
% page as needed:
% \ifCLASSOPTIONpeerreview
% \begin{center} \bfseries EDICS Category: 3-BBND \end{center}
% \fi
%
% For peerreview papers, this IEEEtran command inserts a page break and
% creates the second title. It will be ignored for other modes.
%\IEEEpeerreviewmaketitle




  \item
  Cards with numbers 2 to 101 are placed in a box. A card is selected at random.Find the probability that the card has
\begin{enumerate}[label=(\roman*)]
	\item an even number 
	\item a square number
\end{enumerate}
\solution
%\begin{table}[H]
	\centering
\begin{tabular}{|c|c|c|}
\hline
Random variable &Value &Definition\\ \hline
\multirow{3}{*}{X} &0 &Slips of Rs 1\\
&1 &Slips of Rs 5\\
&2 &Slips of Rs 13\\ \hline
\multirow{2}{*}{Y} &0 &Box A\\
&1 &Box B\\\hline
\end{tabular}
\caption{}
\label{tab:Distribution}
\end{table}
See \tabref{tab:Distribution}.
\begin{align}
p_{Y}\brak{k}= \begin{cases} 
      \frac{1}{3} & {k=0} \\
      \frac{2}{3 }& {k=1} 
   \end{cases}
   \\
p_{Y|X}\brak{0|0} = \frac{19}{25}\, 
p_{Y|X}\brak{0|1} = \frac{6}{25}\,
p_{Y|X}\brak{1|0} = \frac{45}{50}\,
p_{Y|X}\brak{1|2} = \frac{5}{50}
\end{align}
The desired probability is the probability that a slip drawn at random is marked other than Rs 1,
\begin{align}
&=1-p_X\brak{0}\\
&= p_X(1) + p_X(2)
\end{align}
Using Bayes theorem,
\begin{align}
&= p_Y\brak{0} \times \pr{Y=0 | X=1} + p_Y\brak{1} \times \pr{Y=1|X=2}\\
&=\frac{1}{3} \times \frac{6}{25} + \frac{2}{3} \times \frac{5}{50}\\
&=\frac{11}{75}
\end{align}

\newpage

%\tableofcontents

\bigskip

\renewcommand{\thefigure}{\theenumi}
\renewcommand{\thetable}{\theenumi}
%\renewcommand{\theequation}{\theenumi}

%\begin{abstract}
%%\boldmath
%In this letter, an algorithm for evaluating the exact analytical bit error rate  (BER)  for the piecewise linear (PL) combiner for  multiple relays is presented. Previous results were available only for upto three relays. The algorithm is unique in the sense that  the actual mathematical expressions, that are prohibitively large, need not be explicitly obtained. The diversity gain due to multiple relays is shown through plots of the analytical BER, well supported by simulations. 
%
%\end{abstract}
% IEEEtran.cls defaults to using nonbold math in the Abstract.
% This preserves the distinction between vectors and scalars. However,
% if the journal you are submitting to favors bold math in the abstract,
% then you can use LaTeX's standard command \boldmath at the very start
% of the abstract to achieve this. Many IEEE journals frown on math
% in the abstract anyway.

% Note that keywords are not normally used for peerreview papers.
%\begin{IEEEkeywords}
%Cooperative diversity, decode and forward, piecewise linear
%\end{IEEEkeywords}



% For peer review papers, you can put extra information on the cover
% page as needed:
% \ifCLASSOPTIONpeerreview
% \begin{center} \bfseries EDICS Category: 3-BBND \end{center}
% \fi
%
% For peerreview papers, this IEEEtran command inserts a page break and
% creates the second title. It will be ignored for other modes.
%\IEEEpeerreviewmaketitle




\item
The king, queen and jack of clubs are removed from a deck of 52 playing cards and then well shuffled. Now one card is drawn at random from the remaining cards.  Determine the probability that the card is
\begin{enumerate}[label=(\roman*)]
\item a club
\item 10 of hearts
\end{enumerate}
\solution
%\begin{table}[H]
	\centering
\begin{tabular}{|c|c|c|}
\hline
Random variable &Value &Definition\\ \hline
\multirow{3}{*}{X} &0 &Slips of Rs 1\\
&1 &Slips of Rs 5\\
&2 &Slips of Rs 13\\ \hline
\multirow{2}{*}{Y} &0 &Box A\\
&1 &Box B\\\hline
\end{tabular}
\caption{}
\label{tab:Distribution}
\end{table}
See \tabref{tab:Distribution}.
\begin{align}
p_{Y}\brak{k}= \begin{cases} 
      \frac{1}{3} & {k=0} \\
      \frac{2}{3 }& {k=1} 
   \end{cases}
   \\
p_{Y|X}\brak{0|0} = \frac{19}{25}\, 
p_{Y|X}\brak{0|1} = \frac{6}{25}\,
p_{Y|X}\brak{1|0} = \frac{45}{50}\,
p_{Y|X}\brak{1|2} = \frac{5}{50}
\end{align}
The desired probability is the probability that a slip drawn at random is marked other than Rs 1,
\begin{align}
&=1-p_X\brak{0}\\
&= p_X(1) + p_X(2)
\end{align}
Using Bayes theorem,
\begin{align}
&= p_Y\brak{0} \times \pr{Y=0 | X=1} + p_Y\brak{1} \times \pr{Y=1|X=2}\\
&=\frac{1}{3} \times \frac{6}{25} + \frac{2}{3} \times \frac{5}{50}\\
&=\frac{11}{75}
\end{align}

\newpage

%\tableofcontents

\bigskip

\renewcommand{\thefigure}{\theenumi}
\renewcommand{\thetable}{\theenumi}
%\renewcommand{\theequation}{\theenumi}

%\begin{abstract}
%%\boldmath
%In this letter, an algorithm for evaluating the exact analytical bit error rate  (BER)  for the piecewise linear (PL) combiner for  multiple relays is presented. Previous results were available only for upto three relays. The algorithm is unique in the sense that  the actual mathematical expressions, that are prohibitively large, need not be explicitly obtained. The diversity gain due to multiple relays is shown through plots of the analytical BER, well supported by simulations. 
%
%\end{abstract}
% IEEEtran.cls defaults to using nonbold math in the Abstract.
% This preserves the distinction between vectors and scalars. However,
% if the journal you are submitting to favors bold math in the abstract,
% then you can use LaTeX's standard command \boldmath at the very start
% of the abstract to achieve this. Many IEEE journals frown on math
% in the abstract anyway.

% Note that keywords are not normally used for peerreview papers.
%\begin{IEEEkeywords}
%Cooperative diversity, decode and forward, piecewise linear
%\end{IEEEkeywords}



% For peer review papers, you can put extra information on the cover
% page as needed:
% \ifCLASSOPTIONpeerreview
% \begin{center} \bfseries EDICS Category: 3-BBND \end{center}
% \fi
%
% For peerreview papers, this IEEEtran command inserts a page break and
% creates the second title. It will be ignored for other modes.
%\IEEEpeerreviewmaketitle




\item A team of medical students doing their internship have to assist during surgeries
at a city hospital. The probabilities of surgeries rated as very complex, complex,
routine, simple or very simple are respectively, 0.15, 0.20, 0.31, 0.26, .08. Find
the probabilities that a particular surgery will be rated
\begin{enumerate}
	\item complex or very complex;
	\item neither very complex nor very simple;
	\item routine or complex
	\item routine or simple
\end{enumerate}
\solution
%\begin{table}[H]
	\centering
\begin{tabular}{|c|c|c|}
\hline
Random variable &Value &Definition\\ \hline
\multirow{3}{*}{X} &0 &Slips of Rs 1\\
&1 &Slips of Rs 5\\
&2 &Slips of Rs 13\\ \hline
\multirow{2}{*}{Y} &0 &Box A\\
&1 &Box B\\\hline
\end{tabular}
\caption{}
\label{tab:Distribution}
\end{table}
See \tabref{tab:Distribution}.
\begin{align}
p_{Y}\brak{k}= \begin{cases} 
      \frac{1}{3} & {k=0} \\
      \frac{2}{3 }& {k=1} 
   \end{cases}
   \\
p_{Y|X}\brak{0|0} = \frac{19}{25}\, 
p_{Y|X}\brak{0|1} = \frac{6}{25}\,
p_{Y|X}\brak{1|0} = \frac{45}{50}\,
p_{Y|X}\brak{1|2} = \frac{5}{50}
\end{align}
The desired probability is the probability that a slip drawn at random is marked other than Rs 1,
\begin{align}
&=1-p_X\brak{0}\\
&= p_X(1) + p_X(2)
\end{align}
Using Bayes theorem,
\begin{align}
&= p_Y\brak{0} \times \pr{Y=0 | X=1} + p_Y\brak{1} \times \pr{Y=1|X=2}\\
&=\frac{1}{3} \times \frac{6}{25} + \frac{2}{3} \times \frac{5}{50}\\
&=\frac{11}{75}
\end{align}

\newpage

%\tableofcontents

\bigskip

\renewcommand{\thefigure}{\theenumi}
\renewcommand{\thetable}{\theenumi}
%\renewcommand{\theequation}{\theenumi}

%\begin{abstract}
%%\boldmath
%In this letter, an algorithm for evaluating the exact analytical bit error rate  (BER)  for the piecewise linear (PL) combiner for  multiple relays is presented. Previous results were available only for upto three relays. The algorithm is unique in the sense that  the actual mathematical expressions, that are prohibitively large, need not be explicitly obtained. The diversity gain due to multiple relays is shown through plots of the analytical BER, well supported by simulations. 
%
%\end{abstract}
% IEEEtran.cls defaults to using nonbold math in the Abstract.
% This preserves the distinction between vectors and scalars. However,
% if the journal you are submitting to favors bold math in the abstract,
% then you can use LaTeX's standard command \boldmath at the very start
% of the abstract to achieve this. Many IEEE journals frown on math
% in the abstract anyway.

% Note that keywords are not normally used for peerreview papers.
%\begin{IEEEkeywords}
%Cooperative diversity, decode and forward, piecewise linear
%\end{IEEEkeywords}



% For peer review papers, you can put extra information on the cover
% page as needed:
% \ifCLASSOPTIONpeerreview
% \begin{center} \bfseries EDICS Category: 3-BBND \end{center}
% \fi
%
% For peerreview papers, this IEEEtran command inserts a page break and
% creates the second title. It will be ignored for other modes.
%\IEEEpeerreviewmaketitle




\item A card is selected from a pack of 52 cards.
\begin{enumerate}[label=(\alph*)]
    \item How many points are there in the sample space?
    \item Calculate the probability that the card is an ace of spades.
    \item Calculate the probability that the card is (i) an ace and (ii) black card.
\end{enumerate}
\solution
%Let $X$ be an bernoulli rv defined as in \tabref{tab:exemplar/11/16/3/26}.  Then, 
\begin{equation}
    p =
        \frac{4}{11} 
\end{equation}
\begin{table}[H]
	\centering
	\input{exemplar/11/16/3/26/tables/Table2.tex}
	\caption{}
        \label{tab:exemplar/11/16/3/26}
\end{table}

\item The probability that a non leap year selected at random will contain 53 sundays.
\\
\solution
%\begin{table}[H]
	\centering
\begin{tabular}{|c|c|c|}
\hline
Random variable &Value &Definition\\ \hline
\multirow{3}{*}{X} &0 &Slips of Rs 1\\
&1 &Slips of Rs 5\\
&2 &Slips of Rs 13\\ \hline
\multirow{2}{*}{Y} &0 &Box A\\
&1 &Box B\\\hline
\end{tabular}
\caption{}
\label{tab:Distribution}
\end{table}
See \tabref{tab:Distribution}.
\begin{align}
p_{Y}\brak{k}= \begin{cases} 
      \frac{1}{3} & {k=0} \\
      \frac{2}{3 }& {k=1} 
   \end{cases}
   \\
p_{Y|X}\brak{0|0} = \frac{19}{25}\, 
p_{Y|X}\brak{0|1} = \frac{6}{25}\,
p_{Y|X}\brak{1|0} = \frac{45}{50}\,
p_{Y|X}\brak{1|2} = \frac{5}{50}
\end{align}
The desired probability is the probability that a slip drawn at random is marked other than Rs 1,
\begin{align}
&=1-p_X\brak{0}\\
&= p_X(1) + p_X(2)
\end{align}
Using Bayes theorem,
\begin{align}
&= p_Y\brak{0} \times \pr{Y=0 | X=1} + p_Y\brak{1} \times \pr{Y=1|X=2}\\
&=\frac{1}{3} \times \frac{6}{25} + \frac{2}{3} \times \frac{5}{50}\\
&=\frac{11}{75}
\end{align}

\newpage

%\tableofcontents

\bigskip

\renewcommand{\thefigure}{\theenumi}
\renewcommand{\thetable}{\theenumi}
%\renewcommand{\theequation}{\theenumi}

%\begin{abstract}
%%\boldmath
%In this letter, an algorithm for evaluating the exact analytical bit error rate  (BER)  for the piecewise linear (PL) combiner for  multiple relays is presented. Previous results were available only for upto three relays. The algorithm is unique in the sense that  the actual mathematical expressions, that are prohibitively large, need not be explicitly obtained. The diversity gain due to multiple relays is shown through plots of the analytical BER, well supported by simulations. 
%
%\end{abstract}
% IEEEtran.cls defaults to using nonbold math in the Abstract.
% This preserves the distinction between vectors and scalars. However,
% if the journal you are submitting to favors bold math in the abstract,
% then you can use LaTeX's standard command \boldmath at the very start
% of the abstract to achieve this. Many IEEE journals frown on math
% in the abstract anyway.

% Note that keywords are not normally used for peerreview papers.
%\begin{IEEEkeywords}
%Cooperative diversity, decode and forward, piecewise linear
%\end{IEEEkeywords}



% For peer review papers, you can put extra information on the cover
% page as needed:
% \ifCLASSOPTIONpeerreview
% \begin{center} \bfseries EDICS Category: 3-BBND \end{center}
% \fi
%
% For peerreview papers, this IEEEtran command inserts a page break and
% creates the second title. It will be ignored for other modes.
%\IEEEpeerreviewmaketitle




\item One of the four persons John, Rita, Aslam or Gurpreet will be promoted next
month. Consequently the sample space consists of four elementary outcomes
S = {John promoted, Rita promoted, Aslam promoted, Gurpreet promoted}
You are told that the chances of John’s promotion is same as that of Gurpreet,
Rita’s chances of promotion are twice as likely as Johns. Aslam’s chances are
four times that of John.
\begin{enumerate}
	\item Determine
	\begin{enumerate}
		\item P (John promoted)
		\item P (Rita promoted)
		\item P (Aslam promoted)
		\item P (Gurpreet promoted)
	\end{enumerate}
	\item If A = {John promoted or Gurpreet promoted}, find P (A).
\end{enumerate}
\solution
%\begin{table}[H]
	\centering
\begin{tabular}{|c|c|c|}
\hline
Random variable &Value &Definition\\ \hline
\multirow{3}{*}{X} &0 &Slips of Rs 1\\
&1 &Slips of Rs 5\\
&2 &Slips of Rs 13\\ \hline
\multirow{2}{*}{Y} &0 &Box A\\
&1 &Box B\\\hline
\end{tabular}
\caption{}
\label{tab:Distribution}
\end{table}
See \tabref{tab:Distribution}.
\begin{align}
p_{Y}\brak{k}= \begin{cases} 
      \frac{1}{3} & {k=0} \\
      \frac{2}{3 }& {k=1} 
   \end{cases}
   \\
p_{Y|X}\brak{0|0} = \frac{19}{25}\, 
p_{Y|X}\brak{0|1} = \frac{6}{25}\,
p_{Y|X}\brak{1|0} = \frac{45}{50}\,
p_{Y|X}\brak{1|2} = \frac{5}{50}
\end{align}
The desired probability is the probability that a slip drawn at random is marked other than Rs 1,
\begin{align}
&=1-p_X\brak{0}\\
&= p_X(1) + p_X(2)
\end{align}
Using Bayes theorem,
\begin{align}
&= p_Y\brak{0} \times \pr{Y=0 | X=1} + p_Y\brak{1} \times \pr{Y=1|X=2}\\
&=\frac{1}{3} \times \frac{6}{25} + \frac{2}{3} \times \frac{5}{50}\\
&=\frac{11}{75}
\end{align}

\newpage

%\tableofcontents

\bigskip

\renewcommand{\thefigure}{\theenumi}
\renewcommand{\thetable}{\theenumi}
%\renewcommand{\theequation}{\theenumi}

%\begin{abstract}
%%\boldmath
%In this letter, an algorithm for evaluating the exact analytical bit error rate  (BER)  for the piecewise linear (PL) combiner for  multiple relays is presented. Previous results were available only for upto three relays. The algorithm is unique in the sense that  the actual mathematical expressions, that are prohibitively large, need not be explicitly obtained. The diversity gain due to multiple relays is shown through plots of the analytical BER, well supported by simulations. 
%
%\end{abstract}
% IEEEtran.cls defaults to using nonbold math in the Abstract.
% This preserves the distinction between vectors and scalars. However,
% if the journal you are submitting to favors bold math in the abstract,
% then you can use LaTeX's standard command \boldmath at the very start
% of the abstract to achieve this. Many IEEE journals frown on math
% in the abstract anyway.

% Note that keywords are not normally used for peerreview papers.
%\begin{IEEEkeywords}
%Cooperative diversity, decode and forward, piecewise linear
%\end{IEEEkeywords}



% For peer review papers, you can put extra information on the cover
% page as needed:
% \ifCLASSOPTIONpeerreview
% \begin{center} \bfseries EDICS Category: 3-BBND \end{center}
% \fi
%
% For peerreview papers, this IEEEtran command inserts a page break and
% creates the second title. It will be ignored for other modes.
%\IEEEpeerreviewmaketitle




\item A card is drawn from a deck of 52 cards. Find the probability of getting a king or a heart or a red card.\\
\solution
%\begin{table}[H]
	\centering
\begin{tabular}{|c|c|c|}
\hline
Random variable &Value &Definition\\ \hline
\multirow{3}{*}{X} &0 &Slips of Rs 1\\
&1 &Slips of Rs 5\\
&2 &Slips of Rs 13\\ \hline
\multirow{2}{*}{Y} &0 &Box A\\
&1 &Box B\\\hline
\end{tabular}
\caption{}
\label{tab:Distribution}
\end{table}
See \tabref{tab:Distribution}.
\begin{align}
p_{Y}\brak{k}= \begin{cases} 
      \frac{1}{3} & {k=0} \\
      \frac{2}{3 }& {k=1} 
   \end{cases}
   \\
p_{Y|X}\brak{0|0} = \frac{19}{25}\, 
p_{Y|X}\brak{0|1} = \frac{6}{25}\,
p_{Y|X}\brak{1|0} = \frac{45}{50}\,
p_{Y|X}\brak{1|2} = \frac{5}{50}
\end{align}
The desired probability is the probability that a slip drawn at random is marked other than Rs 1,
\begin{align}
&=1-p_X\brak{0}\\
&= p_X(1) + p_X(2)
\end{align}
Using Bayes theorem,
\begin{align}
&= p_Y\brak{0} \times \pr{Y=0 | X=1} + p_Y\brak{1} \times \pr{Y=1|X=2}\\
&=\frac{1}{3} \times \frac{6}{25} + \frac{2}{3} \times \frac{5}{50}\\
&=\frac{11}{75}
\end{align}

\newpage

%\tableofcontents

\bigskip

\renewcommand{\thefigure}{\theenumi}
\renewcommand{\thetable}{\theenumi}
%\renewcommand{\theequation}{\theenumi}

%\begin{abstract}
%%\boldmath
%In this letter, an algorithm for evaluating the exact analytical bit error rate  (BER)  for the piecewise linear (PL) combiner for  multiple relays is presented. Previous results were available only for upto three relays. The algorithm is unique in the sense that  the actual mathematical expressions, that are prohibitively large, need not be explicitly obtained. The diversity gain due to multiple relays is shown through plots of the analytical BER, well supported by simulations. 
%
%\end{abstract}
% IEEEtran.cls defaults to using nonbold math in the Abstract.
% This preserves the distinction between vectors and scalars. However,
% if the journal you are submitting to favors bold math in the abstract,
% then you can use LaTeX's standard command \boldmath at the very start
% of the abstract to achieve this. Many IEEE journals frown on math
% in the abstract anyway.

% Note that keywords are not normally used for peerreview papers.
%\begin{IEEEkeywords}
%Cooperative diversity, decode and forward, piecewise linear
%\end{IEEEkeywords}



% For peer review papers, you can put extra information on the cover
% page as needed:
% \ifCLASSOPTIONpeerreview
% \begin{center} \bfseries EDICS Category: 3-BBND \end{center}
% \fi
%
% For peerreview papers, this IEEEtran command inserts a page break and
% creates the second title. It will be ignored for other modes.
%\IEEEpeerreviewmaketitle




\item The probability that a student will pass his examination is 0.73, the probability of
the student getting a compartment is 0.13, and the probability that the student will
either pass or get compartment is 0.96. State True or False.\\
\solution
%\begin{table}[H]
	\centering
\begin{tabular}{|c|c|c|}
\hline
Random variable &Value &Definition\\ \hline
\multirow{3}{*}{X} &0 &Slips of Rs 1\\
&1 &Slips of Rs 5\\
&2 &Slips of Rs 13\\ \hline
\multirow{2}{*}{Y} &0 &Box A\\
&1 &Box B\\\hline
\end{tabular}
\caption{}
\label{tab:Distribution}
\end{table}
See \tabref{tab:Distribution}.
\begin{align}
p_{Y}\brak{k}= \begin{cases} 
      \frac{1}{3} & {k=0} \\
      \frac{2}{3 }& {k=1} 
   \end{cases}
   \\
p_{Y|X}\brak{0|0} = \frac{19}{25}\, 
p_{Y|X}\brak{0|1} = \frac{6}{25}\,
p_{Y|X}\brak{1|0} = \frac{45}{50}\,
p_{Y|X}\brak{1|2} = \frac{5}{50}
\end{align}
The desired probability is the probability that a slip drawn at random is marked other than Rs 1,
\begin{align}
&=1-p_X\brak{0}\\
&= p_X(1) + p_X(2)
\end{align}
Using Bayes theorem,
\begin{align}
&= p_Y\brak{0} \times \pr{Y=0 | X=1} + p_Y\brak{1} \times \pr{Y=1|X=2}\\
&=\frac{1}{3} \times \frac{6}{25} + \frac{2}{3} \times \frac{5}{50}\\
&=\frac{11}{75}
\end{align}

\newpage

%\tableofcontents

\bigskip

\renewcommand{\thefigure}{\theenumi}
\renewcommand{\thetable}{\theenumi}
%\renewcommand{\theequation}{\theenumi}

%\begin{abstract}
%%\boldmath
%In this letter, an algorithm for evaluating the exact analytical bit error rate  (BER)  for the piecewise linear (PL) combiner for  multiple relays is presented. Previous results were available only for upto three relays. The algorithm is unique in the sense that  the actual mathematical expressions, that are prohibitively large, need not be explicitly obtained. The diversity gain due to multiple relays is shown through plots of the analytical BER, well supported by simulations. 
%
%\end{abstract}
% IEEEtran.cls defaults to using nonbold math in the Abstract.
% This preserves the distinction between vectors and scalars. However,
% if the journal you are submitting to favors bold math in the abstract,
% then you can use LaTeX's standard command \boldmath at the very start
% of the abstract to achieve this. Many IEEE journals frown on math
% in the abstract anyway.

% Note that keywords are not normally used for peerreview papers.
%\begin{IEEEkeywords}
%Cooperative diversity, decode and forward, piecewise linear
%\end{IEEEkeywords}



% For peer review papers, you can put extra information on the cover
% page as needed:
% \ifCLASSOPTIONpeerreview
% \begin{center} \bfseries EDICS Category: 3-BBND \end{center}
% \fi
%
% For peerreview papers, this IEEEtran command inserts a page break and
% creates the second title. It will be ignored for other modes.
%\IEEEpeerreviewmaketitle




\item A card is selected from a pack of 52 cards\\
\begin{enumerate}[label=(\alph*)]
\item How many points are there in the sample space?
\item Calculate the probability that the cards is an ace of spades.
\item Calculate the probability that the card is (i) an ace (ii)black card.\\
\end{enumerate}
%\input{ncert/11/16/3/4_1/Prob_4.tex}
\item In a non-leap year, the probability of having 53 tuesdays or 53 wednesdays is\\
\solution
%A non-leap year has a total of 365 days, and a week has 7 days.\\
So it can be expressed as 
\begin{align}
365\text{days} &=52\times 7+1 \text{day}
\end{align}
$\implies$ 52 tuesdays or wednesdays\\
Random variable X denotes the days of a week
\begin{align}
p_X\brak{k}&=\frac{1}{7}; \quad \brak{1<k<7}
\end{align}
So the probability of extra day being tuesday or wednesday is
\begin{align}
p_X\brak{3}+p_X\brak{4}&=\frac{1}{7}+\frac{1}{7}=\frac{2}{7}
\end{align}



\item There are 1000 sealed envelopes in a box, 10 of them contain a cash prize of
Rs 100 each, 100 of them contain a cash prize of Rs 50 each and 200 of them
contain a cash prize of Rs 10 each and rest do not contain any cash prize. If they
are well shuffled and an envelope is picked up out, what is the probability that it
contains no cash prize?\\
\solution
%\begin{table}[H]
	\centering
\begin{tabular}{|c|c|c|}
\hline
Random variable &Value &Definition\\ \hline
\multirow{3}{*}{X} &0 &Slips of Rs 1\\
&1 &Slips of Rs 5\\
&2 &Slips of Rs 13\\ \hline
\multirow{2}{*}{Y} &0 &Box A\\
&1 &Box B\\\hline
\end{tabular}
\caption{}
\label{tab:Distribution}
\end{table}
See \tabref{tab:Distribution}.
\begin{align}
p_{Y}\brak{k}= \begin{cases} 
      \frac{1}{3} & {k=0} \\
      \frac{2}{3 }& {k=1} 
   \end{cases}
   \\
p_{Y|X}\brak{0|0} = \frac{19}{25}\, 
p_{Y|X}\brak{0|1} = \frac{6}{25}\,
p_{Y|X}\brak{1|0} = \frac{45}{50}\,
p_{Y|X}\brak{1|2} = \frac{5}{50}
\end{align}
The desired probability is the probability that a slip drawn at random is marked other than Rs 1,
\begin{align}
&=1-p_X\brak{0}\\
&= p_X(1) + p_X(2)
\end{align}
Using Bayes theorem,
\begin{align}
&= p_Y\brak{0} \times \pr{Y=0 | X=1} + p_Y\brak{1} \times \pr{Y=1|X=2}\\
&=\frac{1}{3} \times \frac{6}{25} + \frac{2}{3} \times \frac{5}{50}\\
&=\frac{11}{75}
\end{align}

\newpage

%\tableofcontents

\bigskip

\renewcommand{\thefigure}{\theenumi}
\renewcommand{\thetable}{\theenumi}
%\renewcommand{\theequation}{\theenumi}

%\begin{abstract}
%%\boldmath
%In this letter, an algorithm for evaluating the exact analytical bit error rate  (BER)  for the piecewise linear (PL) combiner for  multiple relays is presented. Previous results were available only for upto three relays. The algorithm is unique in the sense that  the actual mathematical expressions, that are prohibitively large, need not be explicitly obtained. The diversity gain due to multiple relays is shown through plots of the analytical BER, well supported by simulations. 
%
%\end{abstract}
% IEEEtran.cls defaults to using nonbold math in the Abstract.
% This preserves the distinction between vectors and scalars. However,
% if the journal you are submitting to favors bold math in the abstract,
% then you can use LaTeX's standard command \boldmath at the very start
% of the abstract to achieve this. Many IEEE journals frown on math
% in the abstract anyway.

% Note that keywords are not normally used for peerreview papers.
%\begin{IEEEkeywords}
%Cooperative diversity, decode and forward, piecewise linear
%\end{IEEEkeywords}



% For peer review papers, you can put extra information on the cover
% page as needed:
% \ifCLASSOPTIONpeerreview
% \begin{center} \bfseries EDICS Category: 3-BBND \end{center}
% \fi
%
% For peerreview papers, this IEEEtran command inserts a page break and
% creates the second title. It will be ignored for other modes.
%\IEEEpeerreviewmaketitle




\item 
A die is thrown and a card is selected at random from a deck of 52 playing cards. The probability of getting an even number on the die and a spade card.\\
\solution
%\begin{table}[H]
	\centering
\begin{tabular}{|c|c|c|}
\hline
Random variable &Value &Definition\\ \hline
\multirow{3}{*}{X} &0 &Slips of Rs 1\\
&1 &Slips of Rs 5\\
&2 &Slips of Rs 13\\ \hline
\multirow{2}{*}{Y} &0 &Box A\\
&1 &Box B\\\hline
\end{tabular}
\caption{}
\label{tab:Distribution}
\end{table}
See \tabref{tab:Distribution}.
\begin{align}
p_{Y}\brak{k}= \begin{cases} 
      \frac{1}{3} & {k=0} \\
      \frac{2}{3 }& {k=1} 
   \end{cases}
   \\
p_{Y|X}\brak{0|0} = \frac{19}{25}\, 
p_{Y|X}\brak{0|1} = \frac{6}{25}\,
p_{Y|X}\brak{1|0} = \frac{45}{50}\,
p_{Y|X}\brak{1|2} = \frac{5}{50}
\end{align}
The desired probability is the probability that a slip drawn at random is marked other than Rs 1,
\begin{align}
&=1-p_X\brak{0}\\
&= p_X(1) + p_X(2)
\end{align}
Using Bayes theorem,
\begin{align}
&= p_Y\brak{0} \times \pr{Y=0 | X=1} + p_Y\brak{1} \times \pr{Y=1|X=2}\\
&=\frac{1}{3} \times \frac{6}{25} + \frac{2}{3} \times \frac{5}{50}\\
&=\frac{11}{75}
\end{align}

\newpage

%\tableofcontents

\bigskip

\renewcommand{\thefigure}{\theenumi}
\renewcommand{\thetable}{\theenumi}
%\renewcommand{\theequation}{\theenumi}

%\begin{abstract}
%%\boldmath
%In this letter, an algorithm for evaluating the exact analytical bit error rate  (BER)  for the piecewise linear (PL) combiner for  multiple relays is presented. Previous results were available only for upto three relays. The algorithm is unique in the sense that  the actual mathematical expressions, that are prohibitively large, need not be explicitly obtained. The diversity gain due to multiple relays is shown through plots of the analytical BER, well supported by simulations. 
%
%\end{abstract}
% IEEEtran.cls defaults to using nonbold math in the Abstract.
% This preserves the distinction between vectors and scalars. However,
% if the journal you are submitting to favors bold math in the abstract,
% then you can use LaTeX's standard command \boldmath at the very start
% of the abstract to achieve this. Many IEEE journals frown on math
% in the abstract anyway.

% Note that keywords are not normally used for peerreview papers.
%\begin{IEEEkeywords}
%Cooperative diversity, decode and forward, piecewise linear
%\end{IEEEkeywords}



% For peer review papers, you can put extra information on the cover
% page as needed:
% \ifCLASSOPTIONpeerreview
% \begin{center} \bfseries EDICS Category: 3-BBND \end{center}
% \fi
%
% For peerreview papers, this IEEEtran command inserts a page break and
% creates the second title. It will be ignored for other modes.
%\IEEEpeerreviewmaketitle




\item
If 4-digit numbers greater than 5,000 are randomly formed from the digits 0, 1, 3, 5, and 7, what is the probability of forming a number divisible by 5 when:
\begin{enumerate}
    \item The digits are repeated?
    \item The repetition of digits is not allowed?
\end{enumerate}
\solution
%\begin{table}[H]
	\centering
\begin{tabular}{|c|c|c|}
\hline
Random variable &Value &Definition\\ \hline
\multirow{3}{*}{X} &0 &Slips of Rs 1\\
&1 &Slips of Rs 5\\
&2 &Slips of Rs 13\\ \hline
\multirow{2}{*}{Y} &0 &Box A\\
&1 &Box B\\\hline
\end{tabular}
\caption{}
\label{tab:Distribution}
\end{table}
See \tabref{tab:Distribution}.
\begin{align}
p_{Y}\brak{k}= \begin{cases} 
      \frac{1}{3} & {k=0} \\
      \frac{2}{3 }& {k=1} 
   \end{cases}
   \\
p_{Y|X}\brak{0|0} = \frac{19}{25}\, 
p_{Y|X}\brak{0|1} = \frac{6}{25}\,
p_{Y|X}\brak{1|0} = \frac{45}{50}\,
p_{Y|X}\brak{1|2} = \frac{5}{50}
\end{align}
The desired probability is the probability that a slip drawn at random is marked other than Rs 1,
\begin{align}
&=1-p_X\brak{0}\\
&= p_X(1) + p_X(2)
\end{align}
Using Bayes theorem,
\begin{align}
&= p_Y\brak{0} \times \pr{Y=0 | X=1} + p_Y\brak{1} \times \pr{Y=1|X=2}\\
&=\frac{1}{3} \times \frac{6}{25} + \frac{2}{3} \times \frac{5}{50}\\
&=\frac{11}{75}
\end{align}

\newpage

%\tableofcontents

\bigskip

\renewcommand{\thefigure}{\theenumi}
\renewcommand{\thetable}{\theenumi}
%\renewcommand{\theequation}{\theenumi}

%\begin{abstract}
%%\boldmath
%In this letter, an algorithm for evaluating the exact analytical bit error rate  (BER)  for the piecewise linear (PL) combiner for  multiple relays is presented. Previous results were available only for upto three relays. The algorithm is unique in the sense that  the actual mathematical expressions, that are prohibitively large, need not be explicitly obtained. The diversity gain due to multiple relays is shown through plots of the analytical BER, well supported by simulations. 
%
%\end{abstract}
% IEEEtran.cls defaults to using nonbold math in the Abstract.
% This preserves the distinction between vectors and scalars. However,
% if the journal you are submitting to favors bold math in the abstract,
% then you can use LaTeX's standard command \boldmath at the very start
% of the abstract to achieve this. Many IEEE journals frown on math
% in the abstract anyway.

% Note that keywords are not normally used for peerreview papers.
%\begin{IEEEkeywords}
%Cooperative diversity, decode and forward, piecewise linear
%\end{IEEEkeywords}



% For peer review papers, you can put extra information on the cover
% page as needed:
% \ifCLASSOPTIONpeerreview
% \begin{center} \bfseries EDICS Category: 3-BBND \end{center}
% \fi
%
% For peerreview papers, this IEEEtran command inserts a page break and
% creates the second title. It will be ignored for other modes.
%\IEEEpeerreviewmaketitle




\item Consider the probability space $\brak{\Omega, \mathcal{G}, P}$ where $\Omega = [0,2]$ and $\mathcal{G} = \cbrak{\phi, \Omega, [0,1], (1,2]}$. Let $X$ and $Y$ be two functions on $\Omega$ defined as
\begin{align*}
    X(\omega) = 
    \begin{cases}
        1 & \text{if }\omega \in [0, 1]\\
        2 & \text{if }\omega \in (1, 2]
    \end{cases}
\end{align*}
and
\begin{align*}
    Y(\omega) = 
    \begin{cases}
        2 & \text{if }\omega \in [0, 1.5]\\
        3 & \text{if }\omega \in (1.5, 2].
    \end{cases}
\end{align*}
Then which one of the following statements is true?
\begin{enumerate}
    \item [(A)] $X$ is a random variable with respect to $\mathcal{G}$, but $Y$ is not a random variable with respect to $\mathcal{G}$.
    \item [(B)] $Y$ is a random variable with respect to $\mathcal{G}$, but $X$ is not a random variable with respect to $\mathcal{G}$.
    \item [(C)] Neither $X$ nor $Y$ is a random variable with respect to $\mathcal{G}$.
    \item [(D)] Both $X$ and $Y$ are random variables with respect to $\mathcal{G}$.
\end{enumerate} \hfill (GATE ST 2023)\\
\solution
%\begin{table}[H]
	\centering
\begin{tabular}{|c|c|c|}
\hline
Random variable &Value &Definition\\ \hline
\multirow{3}{*}{X} &0 &Slips of Rs 1\\
&1 &Slips of Rs 5\\
&2 &Slips of Rs 13\\ \hline
\multirow{2}{*}{Y} &0 &Box A\\
&1 &Box B\\\hline
\end{tabular}
\caption{}
\label{tab:Distribution}
\end{table}
See \tabref{tab:Distribution}.
\begin{align}
p_{Y}\brak{k}= \begin{cases} 
      \frac{1}{3} & {k=0} \\
      \frac{2}{3 }& {k=1} 
   \end{cases}
   \\
p_{Y|X}\brak{0|0} = \frac{19}{25}\, 
p_{Y|X}\brak{0|1} = \frac{6}{25}\,
p_{Y|X}\brak{1|0} = \frac{45}{50}\,
p_{Y|X}\brak{1|2} = \frac{5}{50}
\end{align}
The desired probability is the probability that a slip drawn at random is marked other than Rs 1,
\begin{align}
&=1-p_X\brak{0}\\
&= p_X(1) + p_X(2)
\end{align}
Using Bayes theorem,
\begin{align}
&= p_Y\brak{0} \times \pr{Y=0 | X=1} + p_Y\brak{1} \times \pr{Y=1|X=2}\\
&=\frac{1}{3} \times \frac{6}{25} + \frac{2}{3} \times \frac{5}{50}\\
&=\frac{11}{75}
\end{align}

\newpage

%\tableofcontents

\bigskip

\renewcommand{\thefigure}{\theenumi}
\renewcommand{\thetable}{\theenumi}
%\renewcommand{\theequation}{\theenumi}

%\begin{abstract}
%%\boldmath
%In this letter, an algorithm for evaluating the exact analytical bit error rate  (BER)  for the piecewise linear (PL) combiner for  multiple relays is presented. Previous results were available only for upto three relays. The algorithm is unique in the sense that  the actual mathematical expressions, that are prohibitively large, need not be explicitly obtained. The diversity gain due to multiple relays is shown through plots of the analytical BER, well supported by simulations. 
%
%\end{abstract}
% IEEEtran.cls defaults to using nonbold math in the Abstract.
% This preserves the distinction between vectors and scalars. However,
% if the journal you are submitting to favors bold math in the abstract,
% then you can use LaTeX's standard command \boldmath at the very start
% of the abstract to achieve this. Many IEEE journals frown on math
% in the abstract anyway.

% Note that keywords are not normally used for peerreview papers.
%\begin{IEEEkeywords}
%Cooperative diversity, decode and forward, piecewise linear
%\end{IEEEkeywords}



% For peer review papers, you can put extra information on the cover
% page as needed:
% \ifCLASSOPTIONpeerreview
% \begin{center} \bfseries EDICS Category: 3-BBND \end{center}
% \fi
%
% For peerreview papers, this IEEEtran command inserts a page break and
% creates the second title. It will be ignored for other modes.
%\IEEEpeerreviewmaketitle




	\item  A die is loaded in such a way that each odd number is twice as likely to occur as
each even number. Find $P(G)$, where $G$ is the event that a number greater than
3 occurs on a single roll of the die.
\\
\solution
		%\begin{table}[H]
	\centering
\begin{tabular}{|c|c|c|}
\hline
Random variable &Value &Definition\\ \hline
\multirow{3}{*}{X} &0 &Slips of Rs 1\\
&1 &Slips of Rs 5\\
&2 &Slips of Rs 13\\ \hline
\multirow{2}{*}{Y} &0 &Box A\\
&1 &Box B\\\hline
\end{tabular}
\caption{}
\label{tab:Distribution}
\end{table}
See \tabref{tab:Distribution}.
\begin{align}
p_{Y}\brak{k}= \begin{cases} 
      \frac{1}{3} & {k=0} \\
      \frac{2}{3 }& {k=1} 
   \end{cases}
   \\
p_{Y|X}\brak{0|0} = \frac{19}{25}\, 
p_{Y|X}\brak{0|1} = \frac{6}{25}\,
p_{Y|X}\brak{1|0} = \frac{45}{50}\,
p_{Y|X}\brak{1|2} = \frac{5}{50}
\end{align}
The desired probability is the probability that a slip drawn at random is marked other than Rs 1,
\begin{align}
&=1-p_X\brak{0}\\
&= p_X(1) + p_X(2)
\end{align}
Using Bayes theorem,
\begin{align}
&= p_Y\brak{0} \times \pr{Y=0 | X=1} + p_Y\brak{1} \times \pr{Y=1|X=2}\\
&=\frac{1}{3} \times \frac{6}{25} + \frac{2}{3} \times \frac{5}{50}\\
&=\frac{11}{75}
\end{align}

\newpage

%\tableofcontents

\bigskip

\renewcommand{\thefigure}{\theenumi}
\renewcommand{\thetable}{\theenumi}
%\renewcommand{\theequation}{\theenumi}

%\begin{abstract}
%%\boldmath
%In this letter, an algorithm for evaluating the exact analytical bit error rate  (BER)  for the piecewise linear (PL) combiner for  multiple relays is presented. Previous results were available only for upto three relays. The algorithm is unique in the sense that  the actual mathematical expressions, that are prohibitively large, need not be explicitly obtained. The diversity gain due to multiple relays is shown through plots of the analytical BER, well supported by simulations. 
%
%\end{abstract}
% IEEEtran.cls defaults to using nonbold math in the Abstract.
% This preserves the distinction between vectors and scalars. However,
% if the journal you are submitting to favors bold math in the abstract,
% then you can use LaTeX's standard command \boldmath at the very start
% of the abstract to achieve this. Many IEEE journals frown on math
% in the abstract anyway.

% Note that keywords are not normally used for peerreview papers.
%\begin{IEEEkeywords}
%Cooperative diversity, decode and forward, piecewise linear
%\end{IEEEkeywords}



% For peer review papers, you can put extra information on the cover
% page as needed:
% \ifCLASSOPTIONpeerreview
% \begin{center} \bfseries EDICS Category: 3-BBND \end{center}
% \fi
%
% For peerreview papers, this IEEEtran command inserts a page break and
% creates the second title. It will be ignored for other modes.
%\IEEEpeerreviewmaketitle




	\item All the jacks, queens and kings are removed from a deck of 52 playing cards. The remaining cards are well shuffled and then one card is drawn at random. Giving ace a value 1 similar value for other cards, find the probability that the card has a value 
		\begin{enumerate}
			\item 7
			\item greater than 7
			\item less than 7
		\end{enumerate}
		%Number of cards left after removing all jacks, queens and kings 
\begin{align}
N	= 52 - 4\times 3
	= 40
\end{align}
%\begin{table}[H]
%\def\arraystretch{1.2}
%\begin{tabular}{|c|c|c|}
%\hline
%	\textbf{Parameter} &\textbf{Value} &\textbf{Description}\\ \hline
%	$X$ &1-10 &Represents the value of the card picked \\ \hline
%\end{tabular}
%\end{table}
Let $1 \le X \le 10$ be the value of the card picked.  Then,
\begin{align}
	p_X(k) &= \Pr(X=k)\ \forall\ 1 \leq k \leq 10\\
	&= \frac{4\times 1}{40}\\
	&= \frac{1}{10}\\
	\therefore p_X(k) &= 
	\begin{cases}
		\frac{1}{10} & 1 \leq k \leq 10\\
		0 & \text{otherwise}
	\end{cases}
\end{align}
and
\begin{align}
	F_{X}(k) &= \sum_{m=0}^{k}p_{X}(m) \quad 1 \leq k \leq 10\\
	&= \frac{k}{10}\\
	\therefore F_{X}(k) &= 
	\begin{cases}
		0 & k \leq 0\\
		\frac{k}{10} & 1\leq k \leq 10\\
		1 & k > 10 
	\end{cases}
\end{align}
\begin{enumerate}
	\item Probability that card has value equal to 7 is
		\begin{align}
			 p_{X}(7)
			= \frac{1}{10}
		\end{align}
	\item Probability that card has value greater than 7 is
		\begin{align}
			1 - F_X(7)
			&= 1 - \frac{7}{10}
			\\
			&= \frac{3}{10}
		\end{align}
	\item Probability that card has value less than 7 is
		\begin{align}
			 F_{X}(6)
			=\frac{6}{10}
		\end{align}
\end{enumerate}

  \item A Lot consists of 48 mobile phones of which 42 are good, 3 have only minor defects and 3 have major defects.Varnika will buy a phone if it is good but the trader will only buy a mobile if it has no major defects. One phone is selected at random from the lot. What is the probability that it is
\begin{enumerate}
	\item acceptable to Varnika?
            \item acceptable to the trader?
\end{enumerate}
\solution
	%\begin{table}[H]
	\centering
\begin{tabular}{|c|c|c|}
\hline
Random variable &Value &Definition\\ \hline
\multirow{3}{*}{X} &0 &Slips of Rs 1\\
&1 &Slips of Rs 5\\
&2 &Slips of Rs 13\\ \hline
\multirow{2}{*}{Y} &0 &Box A\\
&1 &Box B\\\hline
\end{tabular}
\caption{}
\label{tab:Distribution}
\end{table}
See \tabref{tab:Distribution}.
\begin{align}
p_{Y}\brak{k}= \begin{cases} 
      \frac{1}{3} & {k=0} \\
      \frac{2}{3 }& {k=1} 
   \end{cases}
   \\
p_{Y|X}\brak{0|0} = \frac{19}{25}\, 
p_{Y|X}\brak{0|1} = \frac{6}{25}\,
p_{Y|X}\brak{1|0} = \frac{45}{50}\,
p_{Y|X}\brak{1|2} = \frac{5}{50}
\end{align}
The desired probability is the probability that a slip drawn at random is marked other than Rs 1,
\begin{align}
&=1-p_X\brak{0}\\
&= p_X(1) + p_X(2)
\end{align}
Using Bayes theorem,
\begin{align}
&= p_Y\brak{0} \times \pr{Y=0 | X=1} + p_Y\brak{1} \times \pr{Y=1|X=2}\\
&=\frac{1}{3} \times \frac{6}{25} + \frac{2}{3} \times \frac{5}{50}\\
&=\frac{11}{75}
\end{align}

\newpage

%\tableofcontents

\bigskip

\renewcommand{\thefigure}{\theenumi}
\renewcommand{\thetable}{\theenumi}
%\renewcommand{\theequation}{\theenumi}

%\begin{abstract}
%%\boldmath
%In this letter, an algorithm for evaluating the exact analytical bit error rate  (BER)  for the piecewise linear (PL) combiner for  multiple relays is presented. Previous results were available only for upto three relays. The algorithm is unique in the sense that  the actual mathematical expressions, that are prohibitively large, need not be explicitly obtained. The diversity gain due to multiple relays is shown through plots of the analytical BER, well supported by simulations. 
%
%\end{abstract}
% IEEEtran.cls defaults to using nonbold math in the Abstract.
% This preserves the distinction between vectors and scalars. However,
% if the journal you are submitting to favors bold math in the abstract,
% then you can use LaTeX's standard command \boldmath at the very start
% of the abstract to achieve this. Many IEEE journals frown on math
% in the abstract anyway.

% Note that keywords are not normally used for peerreview papers.
%\begin{IEEEkeywords}
%Cooperative diversity, decode and forward, piecewise linear
%\end{IEEEkeywords}



% For peer review papers, you can put extra information on the cover
% page as needed:
% \ifCLASSOPTIONpeerreview
% \begin{center} \bfseries EDICS Category: 3-BBND \end{center}
% \fi
%
% For peerreview papers, this IEEEtran command inserts a page break and
% creates the second title. It will be ignored for other modes.
%\IEEEpeerreviewmaketitle




 \item A student says that if you throw a die, it will show up 1 or not 1. Therefore, the probability of getting 1 and the probability of getting 'not 1' each is equal to $\frac{1}{2}$. Is this correct? Give reasons.\\
 \solution
        %\begin{table}[H]
	\centering
\begin{tabular}{|c|c|c|}
\hline
Random variable &Value &Definition\\ \hline
\multirow{3}{*}{X} &0 &Slips of Rs 1\\
&1 &Slips of Rs 5\\
&2 &Slips of Rs 13\\ \hline
\multirow{2}{*}{Y} &0 &Box A\\
&1 &Box B\\\hline
\end{tabular}
\caption{}
\label{tab:Distribution}
\end{table}
See \tabref{tab:Distribution}.
\begin{align}
p_{Y}\brak{k}= \begin{cases} 
      \frac{1}{3} & {k=0} \\
      \frac{2}{3 }& {k=1} 
   \end{cases}
   \\
p_{Y|X}\brak{0|0} = \frac{19}{25}\, 
p_{Y|X}\brak{0|1} = \frac{6}{25}\,
p_{Y|X}\brak{1|0} = \frac{45}{50}\,
p_{Y|X}\brak{1|2} = \frac{5}{50}
\end{align}
The desired probability is the probability that a slip drawn at random is marked other than Rs 1,
\begin{align}
&=1-p_X\brak{0}\\
&= p_X(1) + p_X(2)
\end{align}
Using Bayes theorem,
\begin{align}
&= p_Y\brak{0} \times \pr{Y=0 | X=1} + p_Y\brak{1} \times \pr{Y=1|X=2}\\
&=\frac{1}{3} \times \frac{6}{25} + \frac{2}{3} \times \frac{5}{50}\\
&=\frac{11}{75}
\end{align}

\newpage

%\tableofcontents

\bigskip

\renewcommand{\thefigure}{\theenumi}
\renewcommand{\thetable}{\theenumi}
%\renewcommand{\theequation}{\theenumi}

%\begin{abstract}
%%\boldmath
%In this letter, an algorithm for evaluating the exact analytical bit error rate  (BER)  for the piecewise linear (PL) combiner for  multiple relays is presented. Previous results were available only for upto three relays. The algorithm is unique in the sense that  the actual mathematical expressions, that are prohibitively large, need not be explicitly obtained. The diversity gain due to multiple relays is shown through plots of the analytical BER, well supported by simulations. 
%
%\end{abstract}
% IEEEtran.cls defaults to using nonbold math in the Abstract.
% This preserves the distinction between vectors and scalars. However,
% if the journal you are submitting to favors bold math in the abstract,
% then you can use LaTeX's standard command \boldmath at the very start
% of the abstract to achieve this. Many IEEE journals frown on math
% in the abstract anyway.

% Note that keywords are not normally used for peerreview papers.
%\begin{IEEEkeywords}
%Cooperative diversity, decode and forward, piecewise linear
%\end{IEEEkeywords}



% For peer review papers, you can put extra information on the cover
% page as needed:
% \ifCLASSOPTIONpeerreview
% \begin{center} \bfseries EDICS Category: 3-BBND \end{center}
% \fi
%
% For peerreview papers, this IEEEtran command inserts a page break and
% creates the second title. It will be ignored for other modes.
%\IEEEpeerreviewmaketitle




   \item Four candidates A, B, C, D have ap-
plied for the assignment to coach a school cricket
team. If A is twice as likely to be selected as B, and
B and C are given about the same chance of being
selected, while C is twice as likely to be selected
as D, what are the probabilities that
\begin{enumerate}
\item C will be selected?
\item A will not be selected?
\end{enumerate}
	%\begin{table}[H]
	\centering
\begin{tabular}{|c|c|c|}
\hline
Random variable &Value &Definition\\ \hline
\multirow{3}{*}{X} &0 &Slips of Rs 1\\
&1 &Slips of Rs 5\\
&2 &Slips of Rs 13\\ \hline
\multirow{2}{*}{Y} &0 &Box A\\
&1 &Box B\\\hline
\end{tabular}
\caption{}
\label{tab:Distribution}
\end{table}
See \tabref{tab:Distribution}.
\begin{align}
p_{Y}\brak{k}= \begin{cases} 
      \frac{1}{3} & {k=0} \\
      \frac{2}{3 }& {k=1} 
   \end{cases}
   \\
p_{Y|X}\brak{0|0} = \frac{19}{25}\, 
p_{Y|X}\brak{0|1} = \frac{6}{25}\,
p_{Y|X}\brak{1|0} = \frac{45}{50}\,
p_{Y|X}\brak{1|2} = \frac{5}{50}
\end{align}
The desired probability is the probability that a slip drawn at random is marked other than Rs 1,
\begin{align}
&=1-p_X\brak{0}\\
&= p_X(1) + p_X(2)
\end{align}
Using Bayes theorem,
\begin{align}
&= p_Y\brak{0} \times \pr{Y=0 | X=1} + p_Y\brak{1} \times \pr{Y=1|X=2}\\
&=\frac{1}{3} \times \frac{6}{25} + \frac{2}{3} \times \frac{5}{50}\\
&=\frac{11}{75}
\end{align}

\newpage

%\tableofcontents

\bigskip

\renewcommand{\thefigure}{\theenumi}
\renewcommand{\thetable}{\theenumi}
%\renewcommand{\theequation}{\theenumi}

%\begin{abstract}
%%\boldmath
%In this letter, an algorithm for evaluating the exact analytical bit error rate  (BER)  for the piecewise linear (PL) combiner for  multiple relays is presented. Previous results were available only for upto three relays. The algorithm is unique in the sense that  the actual mathematical expressions, that are prohibitively large, need not be explicitly obtained. The diversity gain due to multiple relays is shown through plots of the analytical BER, well supported by simulations. 
%
%\end{abstract}
% IEEEtran.cls defaults to using nonbold math in the Abstract.
% This preserves the distinction between vectors and scalars. However,
% if the journal you are submitting to favors bold math in the abstract,
% then you can use LaTeX's standard command \boldmath at the very start
% of the abstract to achieve this. Many IEEE journals frown on math
% in the abstract anyway.

% Note that keywords are not normally used for peerreview papers.
%\begin{IEEEkeywords}
%Cooperative diversity, decode and forward, piecewise linear
%\end{IEEEkeywords}



% For peer review papers, you can put extra information on the cover
% page as needed:
% \ifCLASSOPTIONpeerreview
% \begin{center} \bfseries EDICS Category: 3-BBND \end{center}
% \fi
%
% For peerreview papers, this IEEEtran command inserts a page break and
% creates the second title. It will be ignored for other modes.
%\IEEEpeerreviewmaketitle




 \item A bag contain 24 balls of which $x$ balls are red, $2x$ are white and $3x$ are blue. A ball is selected at random, What is the probability that it is
\begin{enumerate}[label=\alph*)]
\item not red ?
\item white ?
\end{enumerate}
%\begin{table}[H]
	\centering
\begin{tabular}{|c|c|c|}
\hline
Random variable &Value &Definition\\ \hline
\multirow{3}{*}{X} &0 &Slips of Rs 1\\
&1 &Slips of Rs 5\\
&2 &Slips of Rs 13\\ \hline
\multirow{2}{*}{Y} &0 &Box A\\
&1 &Box B\\\hline
\end{tabular}
\caption{}
\label{tab:Distribution}
\end{table}
See \tabref{tab:Distribution}.
\begin{align}
p_{Y}\brak{k}= \begin{cases} 
      \frac{1}{3} & {k=0} \\
      \frac{2}{3 }& {k=1} 
   \end{cases}
   \\
p_{Y|X}\brak{0|0} = \frac{19}{25}\, 
p_{Y|X}\brak{0|1} = \frac{6}{25}\,
p_{Y|X}\brak{1|0} = \frac{45}{50}\,
p_{Y|X}\brak{1|2} = \frac{5}{50}
\end{align}
The desired probability is the probability that a slip drawn at random is marked other than Rs 1,
\begin{align}
&=1-p_X\brak{0}\\
&= p_X(1) + p_X(2)
\end{align}
Using Bayes theorem,
\begin{align}
&= p_Y\brak{0} \times \pr{Y=0 | X=1} + p_Y\brak{1} \times \pr{Y=1|X=2}\\
&=\frac{1}{3} \times \frac{6}{25} + \frac{2}{3} \times \frac{5}{50}\\
&=\frac{11}{75}
\end{align}

\newpage

%\tableofcontents

\bigskip

\renewcommand{\thefigure}{\theenumi}
\renewcommand{\thetable}{\theenumi}
%\renewcommand{\theequation}{\theenumi}

%\begin{abstract}
%%\boldmath
%In this letter, an algorithm for evaluating the exact analytical bit error rate  (BER)  for the piecewise linear (PL) combiner for  multiple relays is presented. Previous results were available only for upto three relays. The algorithm is unique in the sense that  the actual mathematical expressions, that are prohibitively large, need not be explicitly obtained. The diversity gain due to multiple relays is shown through plots of the analytical BER, well supported by simulations. 
%
%\end{abstract}
% IEEEtran.cls defaults to using nonbold math in the Abstract.
% This preserves the distinction between vectors and scalars. However,
% if the journal you are submitting to favors bold math in the abstract,
% then you can use LaTeX's standard command \boldmath at the very start
% of the abstract to achieve this. Many IEEE journals frown on math
% in the abstract anyway.

% Note that keywords are not normally used for peerreview papers.
%\begin{IEEEkeywords}
%Cooperative diversity, decode and forward, piecewise linear
%\end{IEEEkeywords}



% For peer review papers, you can put extra information on the cover
% page as needed:
% \ifCLASSOPTIONpeerreview
% \begin{center} \bfseries EDICS Category: 3-BBND \end{center}
% \fi
%
% For peerreview papers, this IEEEtran command inserts a page break and
% creates the second title. It will be ignored for other modes.
%\IEEEpeerreviewmaketitle




If the letters of the word ASSASSINATION are arranged at random. Find the Probability that
\begin{enumerate}[label=(\alph*)]
\item Four $S's$ come consecutively in the word
\item Two  $I's$ and two $N's$ come together
\item All $A's$ are not coming together
\item No two $A's$ are coming together
\end{enumerate}
%\begin{table}[H]
	\centering
\begin{tabular}{|c|c|c|}
\hline
Random variable &Value &Definition\\ \hline
\multirow{3}{*}{X} &0 &Slips of Rs 1\\
&1 &Slips of Rs 5\\
&2 &Slips of Rs 13\\ \hline
\multirow{2}{*}{Y} &0 &Box A\\
&1 &Box B\\\hline
\end{tabular}
\caption{}
\label{tab:Distribution}
\end{table}
See \tabref{tab:Distribution}.
\begin{align}
p_{Y}\brak{k}= \begin{cases} 
      \frac{1}{3} & {k=0} \\
      \frac{2}{3 }& {k=1} 
   \end{cases}
   \\
p_{Y|X}\brak{0|0} = \frac{19}{25}\, 
p_{Y|X}\brak{0|1} = \frac{6}{25}\,
p_{Y|X}\brak{1|0} = \frac{45}{50}\,
p_{Y|X}\brak{1|2} = \frac{5}{50}
\end{align}
The desired probability is the probability that a slip drawn at random is marked other than Rs 1,
\begin{align}
&=1-p_X\brak{0}\\
&= p_X(1) + p_X(2)
\end{align}
Using Bayes theorem,
\begin{align}
&= p_Y\brak{0} \times \pr{Y=0 | X=1} + p_Y\brak{1} \times \pr{Y=1|X=2}\\
&=\frac{1}{3} \times \frac{6}{25} + \frac{2}{3} \times \frac{5}{50}\\
&=\frac{11}{75}
\end{align}

\newpage

%\tableofcontents

\bigskip

\renewcommand{\thefigure}{\theenumi}
\renewcommand{\thetable}{\theenumi}
%\renewcommand{\theequation}{\theenumi}

%\begin{abstract}
%%\boldmath
%In this letter, an algorithm for evaluating the exact analytical bit error rate  (BER)  for the piecewise linear (PL) combiner for  multiple relays is presented. Previous results were available only for upto three relays. The algorithm is unique in the sense that  the actual mathematical expressions, that are prohibitively large, need not be explicitly obtained. The diversity gain due to multiple relays is shown through plots of the analytical BER, well supported by simulations. 
%
%\end{abstract}
% IEEEtran.cls defaults to using nonbold math in the Abstract.
% This preserves the distinction between vectors and scalars. However,
% if the journal you are submitting to favors bold math in the abstract,
% then you can use LaTeX's standard command \boldmath at the very start
% of the abstract to achieve this. Many IEEE journals frown on math
% in the abstract anyway.

% Note that keywords are not normally used for peerreview papers.
%\begin{IEEEkeywords}
%Cooperative diversity, decode and forward, piecewise linear
%\end{IEEEkeywords}



% For peer review papers, you can put extra information on the cover
% page as needed:
% \ifCLASSOPTIONpeerreview
% \begin{center} \bfseries EDICS Category: 3-BBND \end{center}
% \fi
%
% For peerreview papers, this IEEEtran command inserts a page break and
% creates the second title. It will be ignored for other modes.
%\IEEEpeerreviewmaketitle




	\item One urn contains two black balls (labelled B1 and B2) and one white ball. A
	second urn contains one black ball and two white balls (labelled W1 and W2).
	Suppose the following experiment is performed. One of the two urns is chosen
	at random. Next a ball is randomly chosen from the urn. Then a second ball is
	chosen at random from the same urn without replacing the first ball.
	
	\begin{enumerate}
	\item What is the probability that two black balls are chosen?
	
	\item What is the probability that two balls of opposite colour are chosen?
	\end{enumerate}
	\solution
	%\begin{align}
    \label{eq:12.13.6.18.1}
	\because	\pr{A|B} &> \pr{A},\
\frac{\pr{AB}}{\pr{B}} > \pr{A}
\\
    \label{eq:12.13.6.18.2}
	\implies \pr{AB} &> \pr{A}\pr{B}
	\\
	\text{or, } \frac{\pr{AB}}{\pr{A}} &=\pr{B|A} > \pr{A}
\end{align}

\end{enumerate}
